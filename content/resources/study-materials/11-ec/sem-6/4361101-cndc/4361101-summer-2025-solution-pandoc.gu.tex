\documentclass[10pt,a4paper]{article}

% content/resources/templates/preamble.tex
\usepackage[margin=0.6in]{geometry}
\author{Milav Dabgar}
\usepackage{amsmath,amssymb,amsthm}
\usepackage{booktabs}
\usepackage{multirow}
\usepackage{xcolor}
\usepackage{tcolorbox}
\tcbuselibrary{breakable,skins}
\usepackage[colorlinks=true,linkcolor=blue]{hyperref}
\usepackage{titlesec}
\usepackage{enumitem}
\usepackage{tikz}
\usepackage{pgfplots}
\usepackage{circuitikz}
\usepackage[version=4]{mhchem}
\usepackage{longtable}
\usepackage{array}
\usepackage{float}
\usepackage{caption}
\usepackage{listings}

\lstset{
  basicstyle=\small\ttfamily,
  breaklines=true,
  breakatwhitespace=false,
  postbreak=\mbox{\textcolor{red}{$\hookrightarrow$}\space},
  float=false,
  numbers=left,
  numberstyle=\tiny\color{gray},
  numbersep=10pt,
  xleftmargin=2em,
  keywordstyle=\color{blue},
  commentstyle=\color{green!60!black},
  stringstyle=\color{purple},
  backgroundcolor=\color{gray!5},
  showstringspaces=false,
  tabsize=2,
  captionpos=b,
  keepspaces=true,
  columns=flexible
}

\pgfplotsset{compat=1.18}
\usetikzlibrary{shapes,arrows,positioning,calc,patterns,decorations.pathmorphing,decorations.markings,arrows.meta}

% Color scheme
\definecolor{headcolor}{RGB}{0,102,204}
\definecolor{keycolor}{RGB}{220,20,60}
\definecolor{solutioncolor}{RGB}{34,139,34}
\definecolor{mnemoniccolor}{RGB}{148,0,211}
\definecolor{codecolor}{RGB}{0,0,100}

% Spacing
\setlength{\parskip}{3pt}
\setlist[itemize]{nosep}
\setlist[enumerate]{nosep}

% Title formatting
\titleformat{\section}{\Large\bfseries\color{headcolor}}{\thesection}{1em}{}
\titleformat{\subsection}{\large\bfseries\color{headcolor}}{\thesubsection}{1em}{}

% Pandoc tightlist compatibility
\providecommand{\tightlist}{%
  \setlength{\itemsep}{0pt}\setlength{\parskip}{0pt}}

% Pandoc longtable compatibility
\newcounter{none}
\def\thenone{}


% content/resources/templates/gujarati-boxes.tex
\usepackage{fontspec}
\usepackage{polyglossia}

% Set Gujarati as main language (document is primarily in Gujarati)
% Note: gloss-gujarati.ldf doesn't exist in polyglossia, but it will use hyphenation patterns
\setdefaultlanguage{gujarati}
\setotherlanguage{english}

% Configure Gujarati font properly
% Use Language=Default to prevent polyglossia from trying to add language-specific features
% that don't exist for Gujarati, which causes "empty feature" warnings
\newfontfamily\gujaratifont[Script=Gujarati,AutoFakeBold=2.5,AutoFakeSlant=0.3]{Noto Sans Gujarati}
\setmainfont[Script=Gujarati,AutoFakeBold=2.5,AutoFakeSlant=0.3]{Noto Sans Gujarati}
% Use Noto Sans Gujarati for monospace to support Gujarati in text
\setmonofont[Scale=0.9]{Noto Sans Gujarati}

% Configure English to use the same font
\newfontfamily\englishfont[Script=Gujarati,AutoFakeBold=2.5,AutoFakeSlant=0.3]{Noto Sans Gujarati}

% Translations for polyglossia
\gappto\captionsgujarati{
  \renewcommand{\tablename}{કોષ્ટક}
  \renewcommand{\figurename}{આકૃતિ}
}

% Helper for TikZ nodes to ensure Gujarati font
\newcommand{\gu}[1]{{\gujaratifont #1}}

% Custom environments
\newtcolorbox{solutionbox}{
    breakable,
    enhanced,
    colback=solutioncolor!5!white,
    colframe=solutioncolor!75!black,
    fonttitle=\bfseries,
    title=જવાબ
}

\newtcolorbox{solutionboxnobreak}{
 colback=solutioncolor!5!white,
 colframe=solutioncolor!75!black,
 fonttitle=\bfseries,
 title=જવાબ
}

\newtcolorbox{keyformula}{
 breakable,
 enhanced,
 colback=keycolor!5!white,
 colframe=keycolor!75!black,
 fonttitle=\bfseries,
 title=રાસાયણિક સમીકરણ/સૂત્ર
}

\newtcolorbox{mnemonicbox}{
 breakable,
 enhanced,
 colback=mnemoniccolor!5!white,
 colframe=mnemoniccolor!75!black,
 fonttitle=\bfseries,
 title=મેમરી ટ્રીક
}


\begin{document}

\begin{center}
{\Huge\bfseries\color{headcolor} Subject Name (Gujarati)}\\[5pt]
{\LARGE 4361101 -- Summer 2025}\\[3pt]
{\large Semester 1 Study Material}\\[3pt]
{\normalsize\textit{Detailed Solutions and Explanations}}
\end{center}

\vspace{10pt}

\subsection*{પ્રશ્ન 1(અ) [3
ગુણ]}\label{uxaaauxab0uxab6uxaa8-1uxa85-3-uxa97uxaa3}

\textbf{વિવિધ DSL ટેકનોલોજી જણાવો અને ADSL પર ચર્ચા કરો}

\begin{solutionbox}

\textbf{DSL ટેકનોલોજીના પ્રકારો:}

{\def\LTcaptype{none} % do not increment counter
\begin{longtable}[]{@{}lll@{}}
\toprule\noalign{}
DSL પ્રકાર & પૂરું નામ & સ્પીડ \\
\midrule\noalign{}
\endhead
\bottomrule\noalign{}
\endlastfoot
\textbf{ADSL} & Asymmetric DSL & 1-8 Mbps \\
\textbf{SDSL} & Symmetric DSL & 768 Kbps \\
\textbf{VDSL} & Very high DSL & 52 Mbps \\
\textbf{HDSL} & High bit-rate DSL & 1.5 Mbps \\
\end{longtable}
}

\textbf{ADSL ની વિશેષતાઓ:}

\begin{itemize}
\tightlist
\item
  \textbf{અસમપ્રમાણ}: અલગ upload/download સ્પીડ
\item
  \textbf{Frequency Division}: હાલની તાંબાની ટેલિફોન લાઇનનો ઉપયોગ
\item
  \textbf{Download સ્પીડ}: Upload કરતાં વધારે
\end{itemize}

\end{solutionbox}
\begin{mnemonicbox}
``ADSL ડાઉનલોડ ઝડપી''

\end{mnemonicbox}
\begin{center}\rule{0.5\linewidth}{0.5pt}\end{center}

\subsection*{પ્રશ્ન 1(બ) [4
ગુણ]}\label{uxaaauxab0uxab6uxaa8-1uxaac-4-uxa97uxaa3}

\textbf{આર્કિટેક્ચરના આધારે નેટવર્ક વર્ગીકરણનું વર્ણન કરો.}

\begin{solutionbox}

\textbf{નેટવર્ક આર્કિટેક્ચર વર્ગીકરણ:}

{\def\LTcaptype{none} % do not increment counter
\begin{longtable}[]{@{}lll@{}}
\toprule\noalign{}
આર્કિટેક્ચર & વર્ણન & વિશેષતાઓ \\
\midrule\noalign{}
\endhead
\bottomrule\noalign{}
\endlastfoot
\textbf{Peer-to-Peer} & બધા nodes સમાન & કોઈ કેન્દ્રીય સર્વર નથી \\
\textbf{Client-Server} & કેન્દ્રીકૃત મોડેલ & સમર્પિત સર્વર \\
\end{longtable}
}

\textbf{Client-Server ફાયદાઓ:}

\begin{itemize}
\tightlist
\item
  \textbf{કેન્દ્રીય નિયંત્રણ}: સરળ વ્યવસ્થાપન અને સુરક્ષા
\item
  \textbf{સંસાધન શેરિંગ}: સંસાધનોનો કાર્યક્ષમ ઉપયોગ
\item
  \textbf{સ્કેલેબિલિટી}: વધુ વપરાશકર્તાઓને સંભાળી શકે
\item
  \textbf{ડેટા સુરક્ષા}: બેહતર બેકઅપ અને પુનઃપ્રાપ્તિ
\end{itemize}

\textbf{P2P લાક્ષણિકતાઓ:}

\begin{itemize}
\tightlist
\item
  \textbf{વિકેન્દ્રીકૃત}: નિષ્ફળતાનો એક બિંદુ નથી
\item
  \textbf{ખર્ચ અસરકારક}: સમર્પિત સર્વરની જરૂર નથી
\end{itemize}

\end{solutionbox}
\begin{mnemonicbox}
``Client સારી સેવા આપે''

\end{mnemonicbox}
\begin{center}\rule{0.5\linewidth}{0.5pt}\end{center}

\subsection*{પ્રશ્ન 1(ક) [7
ગુણ]}\label{uxaaauxab0uxab6uxaa8-1uxa95-7-uxa97uxaa3}

\textbf{OSI મોડેલની આકૃતિ દોરો અને બધા સ્તરો સાથે વિગતવાર સમજાવો.}

\begin{solutionbox}

\begin{center}
\textbf{Mermaid Diagram (Code)}
\begin{verbatim}
{Shaded}
{Highlighting}[]
graph LR
    A[Application Layer {- 7] {-}{-}{} B[Presentation Layer {-} 6]}
    B {-{-}{} C[Session Layer {-} 5]}
    C {-{-}{} D[Transport Layer {-} 4]}
    D {-{-}{} E[Network Layer {-} 3]}
    E {-{-}{} F[Data Link Layer {-} 2]}
    F {-{-}{} G[Physical Layer {-} 1]}
{Highlighting}
{Shaded}
\end{verbatim}
\end{center}

\textbf{OSI સ્તરોના કાર્યો:}

{\def\LTcaptype{none} % do not increment counter
\begin{longtable}[]{@{}lll@{}}
\toprule\noalign{}
સ્તર & કાર્ય & ઉદાહરણો \\
\midrule\noalign{}
\endhead
\bottomrule\noalign{}
\endlastfoot
\textbf{Application} & વપરાશકર્તા ઇન્ટરફેસ & HTTP, FTP, SMTP \\
\textbf{Presentation} & ડેટા ફોર્મેટિંગ & Encryption, Compression \\
\textbf{Session} & Session વ્યવસ્થાપન & NetBIOS, RPC \\
\textbf{Transport} & End-to-end ડિલિવરી & TCP, UDP \\
\textbf{Network} & Routing & IP, ICMP \\
\textbf{Data Link} & Frame ડિલિવરી & Ethernet, PPP \\
\textbf{Physical} & Bit પ્રસારણ & Cables, Signals \\
\end{longtable}
}

\textbf{મુખ્ય વિશેષતાઓ:}

\begin{itemize}
\tightlist
\item
  \textbf{સ્તરબદ્ધ અભિગમ}: દરેક સ્તર ચોક્કસ કાર્ય કરે છે
\item
  \textbf{માનકીકરણ}: સાર્વત્રિક સંચાર મોડેલ
\item
  \textbf{સમસ્યા નિવારણ}: નેટવર્ક સમસ્યાઓ ઓળખવામાં સરળ
\end{itemize}

\end{solutionbox}
\begin{mnemonicbox}
``બધા લોકો ધંધો કરવા ડેટા પ્રોસેસિંગ કરે''

\end{mnemonicbox}
\begin{center}\rule{0.5\linewidth}{0.5pt}\end{center}

\subsection*{પ્રશ્ન 1(ક OR) [7
ગુણ]}\label{uxaaauxab0uxab6uxaa8-1uxa95-or-7-uxa97uxaa3}

\textbf{TCP/IP protocol suite નો diagram દોરો અને Application Layer,
Transport Layer અને Network Layer ના કાર્યો વિગતવાર સમજાવો.}

\begin{solutionbox}

\begin{center}
\textbf{Mermaid Diagram (Code)}
\begin{verbatim}
{Shaded}
{Highlighting}[]
graph LR
    A[Application Layer] {-{-}{} B[Transport Layer]}
    B {-{-}{} C[Network Layer]}
    C {-{-}{} D[Data Link Layer]}

    A1[HTTP, FTP, SMTP, DNS] {-{-}{} A}
    B1[TCP, UDP] {-{-}{} B}
    C1[IP, ICMP, ARP] {-{-}{} C}
{Highlighting}
{Shaded}
\end{verbatim}
\end{center}

\textbf{સ્તરોના કાર્યો:}

{\def\LTcaptype{none} % do not increment counter
\begin{longtable}[]{@{}lll@{}}
\toprule\noalign{}
સ્તર & મુખ્ય કાર્ય & Protocols \\
\midrule\noalign{}
\endhead
\bottomrule\noalign{}
\endlastfoot
\textbf{Application} & વપરાશકર્તા સેવાઓ & HTTP, FTP, SMTP \\
\textbf{Transport} & End-to-end ડિલિવરી & TCP, UDP \\
\textbf{Network} & Routing packets & IP, ICMP \\
\end{longtable}
}

\textbf{Application Layer કાર્યો:}

\begin{itemize}
\tightlist
\item
  \textbf{Web સેવાઓ}: વેબ બ્રાઉઝિંગ માટે HTTP
\item
  \textbf{File Transfer}: ફાઇલ શેરિંગ માટે FTP
\item
  \textbf{Email}: મેઇલ ડિલિવરી માટે SMTP
\end{itemize}

\textbf{Transport Layer કાર્યો:}

\begin{itemize}
\tightlist
\item
  \textbf{વિશ્વસનીય ડિલિવરી}: TCP ડેટાની અખંડિતતા સુનિશ્ચિત કરે
\item
  \textbf{અવિશ્વસનીય ડિલિવરી}: ઝડપી પ્રસારણ માટે UDP
\item
  \textbf{Port Numbers}: ચોક્કસ applications ઓળખે
\end{itemize}

\textbf{Network Layer કાર્યો:}

\begin{itemize}
\tightlist
\item
  \textbf{Logical Addressing}: ઉપકરણો માટે IP addresses
\item
  \textbf{Routing}: packets માટે શ્રેષ્ઠ માર્ગ પસંદગી
\item
  \textbf{Fragmentation}: મોટા packets તોડવા
\end{itemize}

\end{solutionbox}
\begin{mnemonicbox}
``Applications Transport Networks''

\end{mnemonicbox}
\begin{center}\rule{0.5\linewidth}{0.5pt}\end{center}

\subsection*{પ્રશ્ન 2(અ) [3
ગુણ]}\label{uxaaauxab0uxab6uxaa8-2uxa85-3-uxa97uxaa3}

\textbf{WWW સમજાવો.}

\begin{solutionbox}

\textbf{World Wide Web (WWW):}

{\def\LTcaptype{none} % do not increment counter
\begin{longtable}[]{@{}ll@{}}
\toprule\noalign{}
ઘટક & વર્ણન \\
\midrule\noalign{}
\endhead
\bottomrule\noalign{}
\endlastfoot
\textbf{Web Browser} & Client software \\
\textbf{Web Server} & વેબસાઇટ્સ host કરે \\
\textbf{HTTP} & સંચાર protocol \\
\textbf{URL} & વેબ address \\
\end{longtable}
}

\textbf{WWW વિશેષતાઓ:}

\begin{itemize}
\tightlist
\item
  \textbf{Hypertext}: HTML વાપરીને linked documents
\item
  \textbf{Client-Server Model}: Browser વિનંતી કરે, server જવાબ આપે
\item
  \textbf{સાર્વત્રિક પ્રવેશ}: Platform independent
\end{itemize}

\textbf{ઘટકો:}

\begin{itemize}
\tightlist
\item
  \textbf{HTML}: વેબ પેજ માટે markup language
\item
  \textbf{Browser}: Firefox, Chrome, Safari
\end{itemize}

\end{solutionbox}
\begin{mnemonicbox}
``Web વિશ્વભર કામ કરે''

\end{mnemonicbox}
\begin{center}\rule{0.5\linewidth}{0.5pt}\end{center}

\subsection*{પ્રશ્ન 2(બ) [4
ગુણ]}\label{uxaaauxab0uxab6uxaa8-2uxaac-4-uxa97uxaa3}

\textbf{FDDI અને CDDI સમજાવો.}

\begin{solutionbox}

\textbf{FDDI vs CDDI સરખામણી:}

{\def\LTcaptype{none} % do not increment counter
\begin{longtable}[]{@{}lll@{}}
\toprule\noalign{}
વિશેષતા & FDDI & CDDI \\
\midrule\noalign{}
\endhead
\bottomrule\noalign{}
\endlastfoot
\textbf{Medium} & Fiber optic & Copper wire \\
\textbf{સ્પીડ} & 100 Mbps & 100 Mbps \\
\textbf{અંતર} & 200 km & 100 meters \\
\textbf{ખર્ચ} & વધારે & ઓછો \\
\end{longtable}
}

\textbf{FDDI વિશેષતાઓ:}

\begin{itemize}
\tightlist
\item
  \textbf{Dual Ring Topology}: Primary અને secondary rings
\item
  \textbf{Token Passing}: Access control પદ્ધતિ
\item
  \textbf{Fault Tolerance}: Self-healing ક્ષમતા
\end{itemize}

\textbf{CDDI વિશેષતાઓ:}

\begin{itemize}
\tightlist
\item
  \textbf{Copper આધારિત}: Twisted pair cables વાપરે
\item
  \textbf{ખર્ચ અસરકારક}: Fiber કરતાં સસ્તું
\item
  \textbf{મર્યાદિત અંતર}: ટૂંકી પ્રસારણ રેન્જ
\end{itemize}

\textbf{ઉપયોગ:}

\begin{itemize}
\tightlist
\item
  \textbf{FDDI}: Backbone networks, લાંબા અંતર
\item
  \textbf{CDDI}: Local area networks, ખર્ચ-સંવેદનશીલ વાતાવરણ
\end{itemize}

\end{solutionbox}
\begin{mnemonicbox}
``Fiber ઝડપી, Copper સસ્તું''

\end{mnemonicbox}
\begin{center}\rule{0.5\linewidth}{0.5pt}\end{center}

\subsection*{પ્રશ્ન 2(ક) [7
ગુણ]}\label{uxaaauxab0uxab6uxaa8-2uxa95-7-uxa97uxaa3}

\textbf{OS, CLI, Administrative Functions, Interfaces ના કાર્યો સાથે નેટવર્ક
મેનેજમેન્ટ સિસ્ટમનું વર્ણન કરો.}

\begin{solutionbox}

\begin{center}
\textbf{Mermaid Diagram (Code)}
\begin{verbatim}
{Shaded}
{Highlighting}[]
graph TD
    A[Network Management System] {-{-}{} B[Operating System]}
    A {-{-}{} C[CLI Interface]}
    A {-{-}{} D[Administrative Functions]}
    A {-{-}{} E[GUI Interfaces]}

    B {-{-}{} B1[Resource Management]}
    C {-{-}{} C1[Command Line]}
    D {-{-}{} D1[User Management]}
    E {-{-}{} E1[Graphical Interface]}
{Highlighting}
{Shaded}
\end{verbatim}
\end{center}

\textbf{નેટવર્ક મેનેજમેન્ટ ઘટકો:}

{\def\LTcaptype{none} % do not increment counter
\begin{longtable}[]{@{}lll@{}}
\toprule\noalign{}
ઘટક & કાર્ય & ઉદાહરણો \\
\midrule\noalign{}
\endhead
\bottomrule\noalign{}
\endlastfoot
\textbf{OS કાર્યો} & સંસાધન વ્યવસ્થાપન & Process, memory, file management \\
\textbf{CLI} & Command interface & Terminal, console commands \\
\textbf{Admin કાર્યો} & સિસ્ટમ નિયંત્રણ & User accounts, security \\
\textbf{Interfaces} & વપરાશકર્તા ક્રિયાપ્રતિક્રિયા & GUI, web interface \\
\end{longtable}
}

\textbf{Operating System કાર્યો:}

\begin{itemize}
\tightlist
\item
  \textbf{Process Management}: ચાલતી applications નિયંત્રણ
\item
  \textbf{Memory Management}: સિસ્ટમ સંસાધનો ફાળવવા
\item
  \textbf{File System}: ડેટા ગોઠવવા અને સંગ્રહ
\end{itemize}

\textbf{CLI કાર્યો:}

\begin{itemize}
\tightlist
\item
  \textbf{સીધા Commands}: Text-based નિયંત્રણ
\item
  \textbf{Scripting}: સ્વચાલિત કાર્ય અમલીકરણ
\item
  \textbf{Remote Access}: SSH, Telnet connections
\end{itemize}

\textbf{Administrative કાર્યો:}

\begin{itemize}
\tightlist
\item
  \textbf{User Management}: વપરાશકર્તા accounts બનાવવા, બદલવા
\item
  \textbf{Security Policies}: Access control, permissions
\item
  \textbf{System Monitoring}: કાર્યક્ષમતા ટ્રેકિંગ
\end{itemize}

\textbf{Interfaces:}

\begin{itemize}
\tightlist
\item
  \textbf{GUI}: સરળ નેવિગેશન માટે graphical user interface
\item
  \textbf{Web Interface}: Browser-based management
\item
  \textbf{SNMP}: Simple Network Management Protocol
\end{itemize}

\end{solutionbox}
\begin{mnemonicbox}
``OS CLI Admin Interfaces''

\end{mnemonicbox}
\begin{center}\rule{0.5\linewidth}{0.5pt}\end{center}

\subsection*{પ્રશ્ન 2(અ OR) [3
ગુણ]}\label{uxaaauxab0uxab6uxaa8-2uxa85-or-3-uxa97uxaa3}

\textbf{Connection-oriented protocol અને connectionless protocol ની
સરખામણી કરો.}

\begin{solutionbox}

\textbf{Protocol સરખામણી:}

{\def\LTcaptype{none} % do not increment counter
\begin{longtable}[]{@{}lll@{}}
\toprule\noalign{}
વિશેષતા & Connection-Oriented & Connectionless \\
\midrule\noalign{}
\endhead
\bottomrule\noalign{}
\endlastfoot
\textbf{Setup} & જરૂરી & જરૂરી નથી \\
\textbf{વિશ્વસનીયતા} & વધારે & ઓછી \\
\textbf{સ્પીડ} & ધીમી & ઝડપી \\
\textbf{ઉદાહરણ} & TCP & UDP \\
\end{longtable}
}

\textbf{Connection-Oriented વિશેષતાઓ:}

\begin{itemize}
\tightlist
\item
  \textbf{Three-way Handshake}: ડેટા transfer પહેલાં connection સ્થાપિત કરે
\item
  \textbf{વિશ્વસનીય ડિલિવરી}: Packet delivery અને order ની ખાતરી
\end{itemize}

\textbf{Connectionless વિશેષતાઓ:}

\begin{itemize}
\tightlist
\item
  \textbf{કોઈ Setup નથી}: સીધું ડેટા પ્રસારણ
\item
  \textbf{Best Effort}: ડિલિવરીની કોઈ ખાતરી નથી
\end{itemize}

\end{solutionbox}
\begin{mnemonicbox}
``TCP કનેક્ટ કરે, UDP ડિલિવર કરે''

\end{mnemonicbox}
\begin{center}\rule{0.5\linewidth}{0.5pt}\end{center}

\subsection*{પ્રશ્ન 2(બ OR) [4
ગુણ]}\label{uxaaauxab0uxab6uxaa8-2uxaac-or-4-uxa97uxaa3}

\textbf{નેટવર્ક ડિવાઇસ Repeater સમજાવો.}

\begin{solutionbox}

\textbf{Repeater કાર્યો:}

{\def\LTcaptype{none} % do not increment counter
\begin{longtable}[]{@{}ll@{}}
\toprule\noalign{}
કાર્ય & વર્ણન \\
\midrule\noalign{}
\endhead
\bottomrule\noalign{}
\endlastfoot
\textbf{Signal Amplification} & નબળા signals વધારે \\
\textbf{Range Extension} & નેટવર્ક અંતર વધારે \\
\textbf{Noise Reduction} & Signal ગુણવત્તા સાફ કરે \\
\end{longtable}
}

\begin{verbatim}
Input Signal    Repeater    Output Signal
     |             |             |
weak {-{-}{-}{-}{-}{-}  [AMPLIFY] {-}{-}{-}{-}{-} strong}
noisy              |          clean
\end{verbatim}

\textbf{Repeater લાક્ષણિકતાઓ:}

\begin{itemize}
\tightlist
\item
  \textbf{Physical Layer Device}: Layer 1 પર કામ કરે
\item
  \textbf{Bit-by-Bit}: Digital signals પુનઃ ઉત્પન્ન કરે
\item
  \textbf{કોઈ Intelligence નથી}: ડેટા filter અથવા route કરી શકતું નથી
\end{itemize}

\textbf{ઉપયોગ:}

\begin{itemize}
\tightlist
\item
  \textbf{LAN Extension}: Ethernet segments વિસ્તૃત કરવા
\item
  \textbf{Signal Recovery}: ક્ષતિગ્રસ્ત signals પુનઃસ્થાપિત કરવા
\end{itemize}

\textbf{મર્યાદાઓ:}

\begin{itemize}
\tightlist
\item
  \textbf{Collision Domain}: Collisions segment કરતું નથી
\item
  \textbf{કોઈ Filtering નથી}: બધા signals forward કરે
\end{itemize}

\end{solutionbox}
\begin{mnemonicbox}
``Repeater Signals પુનઃ ઉત્પન્ન કરે''

\end{mnemonicbox}
\begin{center}\rule{0.5\linewidth}{0.5pt}\end{center}

\subsection*{પ્રશ્ન 2(ક OR) [7
ગુણ]}\label{uxaaauxab0uxab6uxaa8-2uxa95-or-7-uxa97uxaa3}

\textbf{Router, Hub અને Switch વચ્ચેનો ભેદ આપો.}

\begin{solutionbox}

\textbf{નેટવર્ક ડિવાઇસ સરખામણી:}

{\def\LTcaptype{none} % do not increment counter
\begin{longtable}[]{@{}llll@{}}
\toprule\noalign{}
વિશેષતા & Hub & Switch & Router \\
\midrule\noalign{}
\endhead
\bottomrule\noalign{}
\endlastfoot
\textbf{OSI Layer} & Physical (1) & Data Link (2) & Network (3) \\
\textbf{Collision Domain} & એક & અનેક & અનેક \\
\textbf{Broadcast Domain} & એક & એક & અનેક \\
\textbf{Intelligence} & કંઈ નથી & MAC શીખવું & IP routing \\
\textbf{Full Duplex} & ના & હા & હા \\
\end{longtable}
}

\begin{center}
\textbf{Mermaid Diagram (Code)}
\begin{verbatim}
{Shaded}
{Highlighting}[]
graph TD
    A[Network Devices] {-{-}{} B[Hub {-} Layer 1]}
    A {-{-}{} C[Switch {-} Layer 2]}
    A {-{-}{} D[Router {-} Layer 3]}

    B {-{-}{} B1[Shared Bandwidth]}
    C {-{-}{} C1[Dedicated Bandwidth]}
    D {-{-}{} D1[Inter{-}network Connection]}
{Highlighting}
{Shaded}
\end{verbatim}
\end{center}

\textbf{Hub લાક્ષણિકતાઓ:}

\begin{itemize}
\tightlist
\item
  \textbf{Shared Medium}: બધા ports bandwidth શેર કરે
\item
  \textbf{Half Duplex}: એક સાથે send અને receive કરી શકતું નથી
\item
  \textbf{Collision Prone}: એક collision domain
\end{itemize}

\textbf{Switch લાક્ષણિકતાઓ:}

\begin{itemize}
\tightlist
\item
  \textbf{MAC Address Table}: ઉપકરણોના સ્થાનો શીખે
\item
  \textbf{Full Duplex}: એક સાથે send/receive
\item
  \textbf{VLAN Support}: Virtual network segmentation
\end{itemize}

\textbf{Router લાક્ષણિકતાઓ:}

\begin{itemize}
\tightlist
\item
  \textbf{IP Routing}: નેટવર્ક વચ્ચે packets forward કરે
\item
  \textbf{Routing Table}: નેટવર્ક topology જાળવે
\item
  \textbf{NAT Support}: Network Address Translation
\end{itemize}

\textbf{ઉપયોગ:}

\begin{itemize}
\tightlist
\item
  \textbf{Hub}: Legacy networks (મોટે ભાગે અપ્રચલિત)
\item
  \textbf{Switch}: LAN connectivity, VLAN implementation
\item
  \textbf{Router}: Internet connectivity, WAN connections
\end{itemize}

\end{solutionbox}
\begin{mnemonicbox}
``Hub શેર કરે, Switch સ્વિચ કરે, Router રૂટ કરે''

\end{mnemonicbox}
\begin{center}\rule{0.5\linewidth}{0.5pt}\end{center}

\subsection*{પ્રશ્ન 3(અ) [3
ગુણ]}\label{uxaaauxab0uxab6uxaa8-3uxa85-3-uxa97uxaa3}

\textbf{UTP, Coaxial અને Fiber optic cable નો સઘડ આકૃતિ દોરો}

\begin{solutionbox}

\begin{verbatim}
UTP Cable:
   +{-{-} Plastic Jacket}
   |   +{-{-} Twisted Pairs}
   |   |
   +{-{-}{-}+===+  +===+}
       |   |  |   |
       +{-{-}{-}+  +{-}{-}{-}+}

Coaxial Cable:
   +{-{-} Outer Jacket}
   |   +{-{-} Shield}
   |   |   +{-{-} Dielectric}
   |   |   |   +{-{-} Center Conductor}
   +{-{-}{-}+{-}{-}{-}+{-}{-}{-}+===+}
       |   |   |
       +{-{-}{-}+{-}{-}{-}+}

Fiber Optic Cable:
   +{-{-} Outer Jacket}
   |   +{-{-} Strength Member}
   |   |   +{-{-} Cladding}
   |   |   |   +{-{-} Core}
   +{-{-}{-}+{-}{-}{-}+{-}{-}{-}+===+}
       |   |   |
       +{-{-}{-}+{-}{-}{-}+}
\end{verbatim}

\textbf{Cable લાક્ષણિકતાઓ:}

{\def\LTcaptype{none} % do not increment counter
\begin{longtable}[]{@{}lll@{}}
\toprule\noalign{}
Cable પ્રકાર & Core સામગ્રી & Bandwidth \\
\midrule\noalign{}
\endhead
\bottomrule\noalign{}
\endlastfoot
\textbf{UTP} & Copper wire & 100 MHz \\
\textbf{Coaxial} & Copper conductor & 1 GHz \\
\textbf{Fiber Optic} & Glass/Plastic & ખૂબ વધારે \\
\end{longtable}
}

\end{solutionbox}
\begin{mnemonicbox}
``વળેલું તાંબું કાચ''

\end{mnemonicbox}
\begin{center}\rule{0.5\linewidth}{0.5pt}\end{center}

\subsection*{પ્રશ્ન 3(બ) [4
ગુણ]}\label{uxaaauxab0uxab6uxaa8-3uxaac-4-uxa97uxaa3}

\textbf{Circuit switching અને packet switching circuit વચ્ચેનો ભેદ આપો.}

\begin{solutionbox}

\textbf{Switching પદ્ધતિઓ સરખામણી:}

{\def\LTcaptype{none} % do not increment counter
\begin{longtable}[]{@{}lll@{}}
\toprule\noalign{}
વિશેષતા & Circuit Switching & Packet Switching \\
\midrule\noalign{}
\endhead
\bottomrule\noalign{}
\endlastfoot
\textbf{Path} & સમર્પિત & સહેજ \\
\textbf{Setup Time} & જરૂરી & જરૂરી નથી \\
\textbf{Bandwidth} & નિશ્ચિત & ચલાયમાન \\
\textbf{ઉદાહરણ} & ટેલિફોન & Internet \\
\end{longtable}
}

\textbf{Circuit Switching વિશેષતાઓ:}

\begin{itemize}
\tightlist
\item
  \textbf{સમર્પિત Path}: સંચાર કરતા પક્ષો વચ્ચે વિશિષ્ટ કનેક્શન
\item
  \textbf{સ્થિર Bandwidth}: સમગ્ર સંચાર દરમિયાન નિશ્ચિત ડેટા રેટ
\item
  \textbf{Setup Phase}: ડેટા transfer પહેલાં connection સ્થાપિત
\end{itemize}

\textbf{Packet Switching વિશેષતાઓ:}

\begin{itemize}
\tightlist
\item
  \textbf{Store and Forward}: મધ્યવર્તી nodes પર packets સંગ્રહ
\item
  \textbf{Dynamic Routing}: વિવિધ packets માટે વિવિધ paths
\item
  \textbf{Resource Sharing}: અનેક વપરાશકર્તાઓ નેટવર્ક સંસાધનો શેર કરે
\end{itemize}

\textbf{ફાયદાઓ:}

\begin{itemize}
\tightlist
\item
  \textbf{Circuit}: ખાતરીકૃત bandwidth, ઓછી latency
\item
  \textbf{Packet}: કાર્યક્ષમ સંસાધન ઉપયોગ, fault tolerance
\end{itemize}

\end{solutionbox}
\begin{mnemonicbox}
``Circuit કનેક્ટ કરે, Packet શેર કરે''

\end{mnemonicbox}
\begin{center}\rule{0.5\linewidth}{0.5pt}\end{center}

\subsection*{પ્રશ્ન 3(ક) [7
ગુણ]}\label{uxaaauxab0uxab6uxaa8-3uxa95-7-uxa97uxaa3}

\textbf{Unguided media અને guided media સમજાવો.}

\begin{solutionbox}

\textbf{પ્રસારણ માધ્યમ વર્ગીકરણ:}

\begin{center}
\textbf{Mermaid Diagram (Code)}
\begin{verbatim}
{Shaded}
{Highlighting}[]
graph TD
    A[Transmission Media] {-{-}{} B[Guided Media]}
    A {-{-}{} C[Unguided Media]}

    B {-{-}{} B1[Twisted Pair]}
    B {-{-}{} B2[Coaxial Cable]}
    B {-{-}{} B3[Fiber Optic]}
    
    C {-{-}{} C1[Radio Waves]}
    C {-{-}{} C2[Microwaves]}
    C {-{-}{} C3[Infrared]}
{Highlighting}
{Shaded}
\end{verbatim}
\end{center}

\textbf{Guided Media લાક્ષણિકતાઓ:}

{\def\LTcaptype{none} % do not increment counter
\begin{longtable}[]{@{}llll@{}}
\toprule\noalign{}
પ્રકાર & સામગ્રી & અંતર & ખર્ચ \\
\midrule\noalign{}
\endhead
\bottomrule\noalign{}
\endlastfoot
\textbf{Twisted Pair} & તાંબું & 100m & ઓછો \\
\textbf{Coaxial} & તાંબું + Shield & 500m & મધ્યમ \\
\textbf{Fiber Optic} & કાચ & 2km+ & વધારે \\
\end{longtable}
}

\textbf{Unguided Media લાક્ષણિકતાઓ:}

{\def\LTcaptype{none} % do not increment counter
\begin{longtable}[]{@{}llll@{}}
\toprule\noalign{}
પ્રકાર & આવર્તન & રેન્જ & ઉપયોગ \\
\midrule\noalign{}
\endhead
\bottomrule\noalign{}
\endlastfoot
\textbf{Radio Waves} & 3KHz-1GHz & લાંબી & AM/FM રેડિયો \\
\textbf{Microwaves} & 1GHz-300GHz & Line of sight & Satellite \\
\textbf{Infrared} & 300GHz-400THz & ટૂંકી & Remote control \\
\end{longtable}
}

\textbf{Guided Media ફાયદાઓ:}

\begin{itemize}
\tightlist
\item
  \textbf{સુરક્ષા}: Interference થી ભૌતિક સુરક્ષા
\item
  \textbf{વિશ્વસનીયતા}: સ્થિર signal પ્રસારણ
\item
  \textbf{ઉચ્ચ Bandwidth}: વધારે ડેટા ક્ષમતા
\end{itemize}

\textbf{Unguided Media ફાયદાઓ:}

\begin{itemize}
\tightlist
\item
  \textbf{ગતિશીલતા}: Wireless connectivity
\item
  \textbf{કવરેજ}: વિશાળ વિસ્તાર પહોંચ
\item
  \textbf{સ્થાપના}: ભૌતિક cabling ની જરૂર નથી
\end{itemize}

\textbf{ઉપયોગ:}

\begin{itemize}
\tightlist
\item
  \textbf{Guided}: LAN, backbone networks, high-speed connections
\item
  \textbf{Unguided}: Mobile networks, satellite communication, WiFi
\end{itemize}

\end{solutionbox}
\begin{mnemonicbox}
``Guided વાયર, Unguided હવા''

\end{mnemonicbox}
\begin{center}\rule{0.5\linewidth}{0.5pt}\end{center}

\subsection*{પ્રશ્ન 3(અ OR) [3
ગુણ]}\label{uxaaauxab0uxab6uxaa8-3uxa85-or-3-uxa97uxaa3}

\textbf{Computer Networks માં ઉપયોગમાં લેવાતા વિવિધ connectors ની ચર્ચા
કરો.}

\begin{solutionbox}

\textbf{નેટવર્ક Connectors:}

{\def\LTcaptype{none} % do not increment counter
\begin{longtable}[]{@{}lll@{}}
\toprule\noalign{}
Connector & Cable પ્રકાર & ઉપયોગ \\
\midrule\noalign{}
\endhead
\bottomrule\noalign{}
\endlastfoot
\textbf{RJ-45} & UTP/STP & Ethernet \\
\textbf{BNC} & Coaxial & Legacy networks \\
\textbf{SC/ST} & Fiber optic & High-speed networks \\
\end{longtable}
}

\textbf{Connector વિશેષતાઓ:}

\begin{itemize}
\tightlist
\item
  \textbf{RJ-45}: Twisted pair માટે 8-pin modular connector
\item
  \textbf{BNC}: Coaxial cables માટે bayonet connector
\item
  \textbf{SC/ST}: Fiber માટે push-pull અને twist-lock connectors
\end{itemize}

\end{solutionbox}
\begin{mnemonicbox}
``RJ BNC Fiber કનેક્ટ''

\end{mnemonicbox}
\begin{center}\rule{0.5\linewidth}{0.5pt}\end{center}

\subsection*{પ્રશ્ન 3(બ OR) [4
ગુણ]}\label{uxaaauxab0uxab6uxaa8-3uxaac-or-4-uxa97uxaa3}

\textbf{ઉદાહરણો સાથે IP addressing scheme સમજાવો.}

\begin{solutionbox}

\textbf{IP Address Classes:}

{\def\LTcaptype{none} % do not increment counter
\begin{longtable}[]{@{}llll@{}}
\toprule\noalign{}
Class & Range & Default Mask & ઉદાહરણ \\
\midrule\noalign{}
\endhead
\bottomrule\noalign{}
\endlastfoot
\textbf{A} & 1-126 & /8 & 10.0.0.1 \\
\textbf{B} & 128-191 & /16 & 172.16.0.1 \\
\textbf{C} & 192-223 & /24 & 192.168.1.1 \\
\end{longtable}
}

\textbf{IP Address બંધારણ:}

\begin{itemize}
\tightlist
\item
  \textbf{Network ભાગ}: નેટવર્ક ઓળખે
\item
  \textbf{Host ભાગ}: ઉપકરણ ઓળખે
\item
  \textbf{Subnet Mask}: નેટવર્ક અને host ભાગો અલગ કરે
\end{itemize}

\textbf{વિશિષ્ટ Addresses:}

\begin{itemize}
\tightlist
\item
  \textbf{Loopback}: 127.0.0.1 (localhost)
\item
  \textbf{Private}: 10.x.x.x, 172.16.x.x, 192.168.x.x
\item
  \textbf{Broadcast}: બધા host bits 1 પર સેટ
\end{itemize}

\textbf{ઉદાહરણ ગણતરી:} IP: 192.168.1.100/24

\begin{itemize}
\tightlist
\item
  Network: 192.168.1.0
\item
  Broadcast: 192.168.1.255
\end{itemize}

\end{solutionbox}
\begin{mnemonicbox}
``એક મોટો Class નેટવર્ક''

\end{mnemonicbox}
\begin{center}\rule{0.5\linewidth}{0.5pt}\end{center}

\subsection*{પ્રશ્ન 3(ક OR) [7
ગુણ]}\label{uxaaauxab0uxab6uxaa8-3uxa95-or-7-uxa97uxaa3}

\textbf{IPv4 અને IPv6 વચ્ચેનો ભેદ આપો.}

\begin{solutionbox}

\textbf{IPv4 vs IPv6 સરખામણી:}

{\def\LTcaptype{none} % do not increment counter
\begin{longtable}[]{@{}lll@{}}
\toprule\noalign{}
વિશેષતા & IPv4 & IPv6 \\
\midrule\noalign{}
\endhead
\bottomrule\noalign{}
\endlastfoot
\textbf{Address લંબાઈ} & 32 bits & 128 bits \\
\textbf{Address ફોર્મેટ} & દશાંશ & હેક્સાડેસિમલ \\
\textbf{Address સ્પેસ} & 4.3 બિલિયન & 340 undecillion \\
\textbf{Header સાઇઝ} & 20-60 bytes & 40 bytes \\
\textbf{Fragmentation} & Router/Host & ફક્ત Host \\
\textbf{સુરક્ષા} & વૈકલ્પિક & બિલ્ટ-ઇન \\
\end{longtable}
}

\textbf{IPv4 લાક્ષણિકતાઓ:}

\begin{itemize}
\tightlist
\item
  \textbf{Address ઉદાહરણ}: 192.168.1.1
\item
  \textbf{Dotted Decimal}: ચાર octets dots વડે અલગ
\item
  \textbf{Classes}: A, B, C, D, E addressing scheme
\item
  \textbf{NAT જરૂરી}: Address exhaustion ને કારણે
\end{itemize}

\textbf{IPv6 લાક્ષણિકતાઓ:}

\begin{itemize}
\tightlist
\item
  \textbf{Address ઉદાહરણ}: 2001:0db8:85a3::8a2e:0370:7334
\item
  \textbf{Colon Notation}: આઠ hexadecimal digits ના જૂથો
\item
  \textbf{કોઈ Classes નથી}: Hierarchical addressing
\item
  \textbf{Auto-configuration}: Stateless address configuration
\end{itemize}

\textbf{IPv6 ફાયદાઓ:}

\begin{itemize}
\tightlist
\item
  \textbf{મોટી Address સ્પેસ}: Address exhaustion દૂર કરે
\item
  \textbf{સરળ Header}: સુધારેલ processing કાર્યક્ષમતા
\item
  \textbf{Built-in સુરક્ષા}: IPSec ફરજિયાત
\item
  \textbf{બહેતર QoS}: Traffic prioritization માટે flow labeling
\end{itemize}

\textbf{Migration વ્યૂહરચનાઓ:}

\begin{itemize}
\tightlist
\item
  \textbf{Dual Stack}: IPv4 અને IPv6 બંને ચલાવો
\item
  \textbf{Tunneling}: IPv4 માં IPv6 encapsulate કરો
\item
  \textbf{Translation}: Protocols વચ્ચે convert કરો
\end{itemize}

\end{solutionbox}
\begin{mnemonicbox}
``IPv6 વધુ Addresses છે''

\end{mnemonicbox}
\begin{center}\rule{0.5\linewidth}{0.5pt}\end{center}

\subsection*{પ્રશ્ન 4(અ) [3
ગુણ]}\label{uxaaauxab0uxab6uxaa8-4uxa85-3-uxa97uxaa3}

\textbf{File Transfer Protocol સમજાવો.}

\begin{solutionbox}

\textbf{FTP લાક્ષણિકતાઓ:}

{\def\LTcaptype{none} % do not increment counter
\begin{longtable}[]{@{}ll@{}}
\toprule\noalign{}
વિશેષતા & વર્ણન \\
\midrule\noalign{}
\endhead
\bottomrule\noalign{}
\endlastfoot
\textbf{Port Numbers} & 20 (data), 21 (control) \\
\textbf{Protocol} & TCP-આધારિત \\
\textbf{Authentication} & Username/password \\
\end{longtable}
}

\textbf{FTP કામગીરી:}

\begin{itemize}
\tightlist
\item
  \textbf{Upload}: Server પર ફાઇલો transfer કરવા PUT command
\item
  \textbf{Download}: Server માંથી ફાઇલો retrieve કરવા GET command
\item
  \textbf{Directory}: ફાઇલ listings બતાવવા LIST command
\end{itemize}

\textbf{FTP Modes:}

\begin{itemize}
\tightlist
\item
  \textbf{Active Mode}: Server ડેટા connection શરૂ કરે
\item
  \textbf{Passive Mode}: Client ડેટા connection શરૂ કરે
\end{itemize}

\end{solutionbox}
\begin{mnemonicbox}
``FTP ફાઇલો Transfer કરે''

\end{mnemonicbox}
\begin{center}\rule{0.5\linewidth}{0.5pt}\end{center}

\subsection*{પ્રશ્ન 4(બ) [4
ગુણ]}\label{uxaaauxab0uxab6uxaa8-4uxaac-4-uxa97uxaa3}

\textbf{DNS પર નોંધ લખો.}

\begin{solutionbox}

\textbf{Domain Name System (DNS):}

{\def\LTcaptype{none} % do not increment counter
\begin{longtable}[]{@{}ll@{}}
\toprule\noalign{}
ઘટક & કાર્ય \\
\midrule\noalign{}
\endhead
\bottomrule\noalign{}
\endlastfoot
\textbf{DNS Server} & Domain names resolve કરે \\
\textbf{DNS Cache} & તાજેતરના lookups સંગ્રહ કરે \\
\textbf{DNS Records} & Names ને addresses સાથે map કરે \\
\end{longtable}
}

\textbf{DNS વંશવેલો:}

\begin{itemize}
\tightlist
\item
  \textbf{Root Domain}: Top-level (.)
\item
  \textbf{Top-Level Domain}: .com, .org, .net
\item
  \textbf{Second-Level Domain}: google.com
\item
  \textbf{Subdomain}: \textless www.google.com\textgreater{}
\end{itemize}

\textbf{DNS Records:}

\begin{itemize}
\tightlist
\item
  \textbf{A Record}: Domain ને IPv4 address સાથે map કરે
\item
  \textbf{AAAA Record}: Domain ને IPv6 address સાથે map કરે
\item
  \textbf{CNAME}: Canonical name alias
\item
  \textbf{MX}: Mail exchange server
\end{itemize}

\textbf{DNS Resolution પ્રક્રિયા:}

\begin{enumerate}
\tightlist
\item
  \textbf{Local Cache}: Browser cache તપાસો
\item
  \textbf{Recursive Query}: DNS resolver સાથે સંપર્ક
\item
  \textbf{Iterative Query}: Authoritative servers query કરો
\end{enumerate}

\end{solutionbox}
\begin{mnemonicbox}
``DNS નામો Servers''

\end{mnemonicbox}
\begin{center}\rule{0.5\linewidth}{0.5pt}\end{center}

\subsection*{પ્રશ્ન 4(ક) [7
ગુણ]}\label{uxaaauxab0uxab6uxaa8-4uxa95-7-uxa97uxaa3}

\textbf{Electronic Mail સમજાવો.}

\begin{solutionbox}

\begin{center}
\textbf{Mermaid Diagram (Code)}
\begin{verbatim}
{Shaded}
{Highlighting}[]
graph LR
    A[Email Client] {-{-}{} B[SMTP Server]}
    B {-{-}{} C[Internet]}
    C {-{-}{} D[Recipient SMTP]}
    D {-{-}{} E[POP3/IMAP Server]}
    E {-{-}{} F[Recipient Client]}
{Highlighting}
{Shaded}
\end{verbatim}
\end{center}

\textbf{Email સિસ્ટમ ઘટકો:}

{\def\LTcaptype{none} % do not increment counter
\begin{longtable}[]{@{}lll@{}}
\toprule\noalign{}
ઘટક & કાર્ય & Protocol \\
\midrule\noalign{}
\endhead
\bottomrule\noalign{}
\endlastfoot
\textbf{User Agent} & Email client & Outlook, Gmail \\
\textbf{Mail Server} & Store/forward & SMTP, POP3, IMAP \\
\textbf{Message Transfer} & Delivery & SMTP \\
\end{longtable}
}

\textbf{Email Protocols:}

{\def\LTcaptype{none} % do not increment counter
\begin{longtable}[]{@{}lll@{}}
\toprule\noalign{}
Protocol & હેતુ & Port \\
\midrule\noalign{}
\endhead
\bottomrule\noalign{}
\endlastfoot
\textbf{SMTP} & Mail મોકલવા & 25 \\
\textbf{POP3} & Mail retrieve કરવા & 110 \\
\textbf{IMAP} & Mail access કરવા & 143 \\
\end{longtable}
}

\textbf{Email Message ફોર્મેટ:}

\begin{itemize}
\tightlist
\item
  \textbf{Header}: To, From, Subject, Date
\item
  \textbf{Body}: Message content
\item
  \textbf{Attachments}: Binary files
\end{itemize}

\textbf{SMTP vs POP3 vs IMAP:}

\begin{itemize}
\tightlist
\item
  \textbf{SMTP}: Outgoing mail protocol
\item
  \textbf{POP3}: Local device પર mail download કરે
\item
  \textbf{IMAP}: Devices પર mail synchronize કરે
\end{itemize}

\textbf{Email પ્રક્રિયા:}

\begin{enumerate}
\tightlist
\item
  \textbf{Compose}: વપરાશકર્તા message બનાવે
\item
  \textbf{Send}: SMTP server પર transfer કરે
\item
  \textbf{Route}: Destination સુધી internet routing
\item
  \textbf{Deliver}: Recipient mailbox માં store કરે
\item
  \textbf{Retrieve}: POP3/IMAP client પર download કરે
\end{enumerate}

\textbf{સુરક્ષા વિશેષતાઓ:}

\begin{itemize}
\tightlist
\item
  \textbf{Encryption}: સુરક્ષિત mail transmission
\item
  \textbf{Authentication}: Sender identity verify કરે
\item
  \textbf{Spam Filtering}: અનિચ્છનીય mail block કરે
\end{itemize}

\end{solutionbox}
\begin{mnemonicbox}
``SMTP મોકલે, POP3 લે, IMAP એકીકૃત કરે''

\end{mnemonicbox}
\begin{center}\rule{0.5\linewidth}{0.5pt}\end{center}

\subsection*{પ્રશ્ન 4(અ OR) [3
ગુણ]}\label{uxaaauxab0uxab6uxaa8-4uxa85-or-3-uxa97uxaa3}

\textbf{Web browser સમજાવો.}

\begin{solutionbox}

\textbf{Web Browser કાર્યો:}

{\def\LTcaptype{none} % do not increment counter
\begin{longtable}[]{@{}ll@{}}
\toprule\noalign{}
કાર્ય & વર્ણન \\
\midrule\noalign{}
\endhead
\bottomrule\noalign{}
\endlastfoot
\textbf{HTTP Client} & Web pages વિનંતી કરે \\
\textbf{HTML Renderer} & Web content પ્રદર્શિત કરે \\
\textbf{JavaScript Engine} & Scripts execute કરે \\
\end{longtable}
}

\textbf{Browser ઘટકો:}

\begin{itemize}
\tightlist
\item
  \textbf{User Interface}: Address bar, bookmarks, navigation
\item
  \textbf{Rendering Engine}: HTML/CSS interpretation
\item
  \textbf{Networking}: HTTP/HTTPS communication
\end{itemize}

\textbf{લોકપ્રિય Browsers:}

\begin{itemize}
\tightlist
\item
  \textbf{Chrome}: Google નું browser
\item
  \textbf{Firefox}: Mozilla નું browser
\item
  \textbf{Safari}: Apple નું browser
\end{itemize}

\end{solutionbox}
\begin{mnemonicbox}
``Browser Web Render કરે''

\end{mnemonicbox}
\begin{center}\rule{0.5\linewidth}{0.5pt}\end{center}

\subsection*{પ્રશ્ન 4(બ OR) [4
ગુણ]}\label{uxaaauxab0uxab6uxaa8-4uxaac-or-4-uxa97uxaa3}

\textbf{Mail Protocols સમજાવો.}

\begin{solutionbox}

\textbf{Email Protocol સરખામણી:}

{\def\LTcaptype{none} % do not increment counter
\begin{longtable}[]{@{}llll@{}}
\toprule\noalign{}
Protocol & પ્રકાર & કાર્ય & Port \\
\midrule\noalign{}
\endhead
\bottomrule\noalign{}
\endlastfoot
\textbf{SMTP} & Outgoing & Mail મોકલવા & 25 \\
\textbf{POP3} & Incoming & Mail download કરવા & 110 \\
\textbf{IMAP} & Incoming & Mail sync કરવા & 143 \\
\end{longtable}
}

\textbf{SMTP વિશેષતાઓ:}

\begin{itemize}
\tightlist
\item
  \textbf{Push Protocol}: Sender transfer શરૂ કરે
\item
  \textbf{Store and Forward}: મધ્યવર્તી mail servers
\item
  \textbf{Text-based}: ASCII command protocol
\end{itemize}

\textbf{POP3 વિશેષતાઓ:}

\begin{itemize}
\tightlist
\item
  \textbf{Download and Delete}: Server માંથી mail દૂર કરે
\item
  \textbf{Offline Access}: Local mail storage
\item
  \textbf{Single Device}: અનેક devices માટે યોગ્ય નથી
\end{itemize}

\textbf{IMAP વિશેષતાઓ:}

\begin{itemize}
\tightlist
\item
  \textbf{Server Storage}: Mail server પર રહે
\item
  \textbf{Multi-device}: અનેક clients માંથી access
\item
  \textbf{Folder Sync}: Server-client synchronization
\end{itemize}

\end{solutionbox}
\begin{mnemonicbox}
``SMTP મોકલે, POP3 ખેંચે, IMAP એકીકૃત કરે''

\end{mnemonicbox}
\begin{center}\rule{0.5\linewidth}{0.5pt}\end{center}

\subsection*{પ્રશ્ન 4(ક OR) [7
ગુણ]}\label{uxaaauxab0uxab6uxaa8-4uxa95-or-7-uxa97uxaa3}

\textbf{TCP અને UDP protocols નું વર્ણન કરો.}

\begin{solutionbox}

\textbf{TCP vs UDP સરખામણી:}

{\def\LTcaptype{none} % do not increment counter
\begin{longtable}[]{@{}lll@{}}
\toprule\noalign{}
વિશેષતા & TCP & UDP \\
\midrule\noalign{}
\endhead
\bottomrule\noalign{}
\endlastfoot
\textbf{Connection} & Connection-oriented & Connectionless \\
\textbf{વિશ્વસનીયતા} & વિશ્વસનીય & અવિશ્વસનીય \\
\textbf{સ્પીડ} & ધીમી & ઝડપી \\
\textbf{Header સાઇઝ} & 20 bytes & 8 bytes \\
\textbf{Flow Control} & હા & ના \\
\textbf{Error Control} & હા & ના \\
\end{longtable}
}

\begin{center}
\textbf{Mermaid Diagram (Code)}
\begin{verbatim}
{Shaded}
{Highlighting}[]
graph LR
    A[Transport Layer] {-{-}{} B[TCP {-} વિશ્વસનીય]}
    A {-{-}{} C[UDP {-} ઝડપી]}

    B {-{-}{} B1[Web, Email, FTP]}
    C {-{-}{} C1[DNS, Streaming, Gaming]}
{Highlighting}
{Shaded}
\end{verbatim}
\end{center}

\textbf{TCP વિશેષતાઓ:}

\begin{itemize}
\tightlist
\item
  \textbf{Three-way Handshake}: SYN, SYN-ACK, ACK
\item
  \textbf{Sequence Numbers}: ક્રમબદ્ધ packet delivery
\item
  \textbf{Acknowledgments}: Packet receipt confirm કરે
\item
  \textbf{Flow Control}: Buffer overflow અટકાવે
\item
  \textbf{Congestion Control}: Network traffic manage કરે
\end{itemize}

\textbf{UDP વિશેષતાઓ:}

\begin{itemize}
\tightlist
\item
  \textbf{Stateless}: કોઈ connection state maintain કરતું નથી
\item
  \textbf{Best Effort}: Delivery ની કોઈ ખાતરી નથી
\item
  \textbf{Low Overhead}: ન્યૂનતમ header માહિતી
\item
  \textbf{Broadcast Support}: One-to-many communication
\end{itemize}

\textbf{TCP ઉપયોગ:}

\begin{itemize}
\tightlist
\item
  \textbf{Web Browsing}: HTTP/HTTPS
\item
  \textbf{Email}: SMTP, POP3, IMAP
\item
  \textbf{File Transfer}: FTP
\end{itemize}

\textbf{UDP ઉપયોગ:}

\begin{itemize}
\tightlist
\item
  \textbf{DNS Queries}: Domain name resolution
\item
  \textbf{Streaming}: Video/audio transmission
\item
  \textbf{Gaming}: Real-time applications
\end{itemize}

\textbf{TCP Header Fields:}

\begin{itemize}
\tightlist
\item
  \textbf{Source/Destination Port}: Application identification
\item
  \textbf{Sequence Number}: Packet ordering
\item
  \textbf{Window Size}: Flow control
\end{itemize}

\textbf{UDP Header Fields:}

\begin{itemize}
\tightlist
\item
  \textbf{Source/Destination Port}: Application identification
\item
  \textbf{Length}: Datagram size
\item
  \textbf{Checksum}: Error detection
\end{itemize}

\end{solutionbox}
\begin{mnemonicbox}
``TCP સાવચેતીથી પ્રયાસ કરે, UDP ડેટા છોડે''

\end{mnemonicbox}
\begin{center}\rule{0.5\linewidth}{0.5pt}\end{center}

\subsection*{પ્રશ્ન 5(અ) [3
ગુણ]}\label{uxaaauxab0uxab6uxaa8-5uxa85-3-uxa97uxaa3}

\textbf{નેટવર્ક ડિવાઇસ Bridge નું વર્ણન કરો.}

\begin{solutionbox}

\textbf{Bridge લાક્ષણિકતાઓ:}

{\def\LTcaptype{none} % do not increment counter
\begin{longtable}[]{@{}ll@{}}
\toprule\noalign{}
વિશેષતા & વર્ણન \\
\midrule\noalign{}
\endhead
\bottomrule\noalign{}
\endlastfoot
\textbf{OSI Layer} & Data Link (Layer 2) \\
\textbf{કાર્ય} & Collision domains segment કરે \\
\textbf{Learning} & MAC address table \\
\end{longtable}
}

\textbf{Bridge કામગીરી:}

\begin{itemize}
\tightlist
\item
  \textbf{Learning}: Frames માંથી MAC addresses record કરે
\item
  \textbf{Filtering}: જરૂર હોય ત્યારે જ frames forward કરે
\item
  \textbf{Forwarding}: યોગ્ય segment પર frames મોકલે
\end{itemize}

\textbf{Bridge પ્રકારો:}

\begin{itemize}
\tightlist
\item
  \textbf{Transparent Bridge}: આપોઆપ learning
\item
  \textbf{Source Routing}: Frame માં path specify કરેલ
\end{itemize}

\end{solutionbox}
\begin{mnemonicbox}
``Bridge Collisions તોડે''

\end{mnemonicbox}
\begin{center}\rule{0.5\linewidth}{0.5pt}\end{center}

\subsection*{પ્રશ્ન 5(બ) [4
ગુણ]}\label{uxaaauxab0uxab6uxaa8-5uxaac-4-uxa97uxaa3}

\textbf{સામાજિક મુદ્દાઓ અને Hacking સમજાવો તેની સાવચેતીઓની પણ ચર્ચા કરો.}

\begin{solutionbox}

\textbf{નેટવર્કમાં સામાજિક મુદ્દાઓ:}

{\def\LTcaptype{none} % do not increment counter
\begin{longtable}[]{@{}ll@{}}
\toprule\noalign{}
મુદ્દો & અસર \\
\midrule\noalign{}
\endhead
\bottomrule\noalign{}
\endlastfoot
\textbf{Digital Divide} & ટેકનોલોજીની અસમાન પહોંચ \\
\textbf{Privacy ચિંતાઓ} & વ્યક્તિગત ડેટાનો દુરુપયોગ \\
\textbf{Cyberbullying} & ઓનલાઇન હેરાનગતિ \\
\end{longtable}
}

\textbf{Hacking પ્રકારો:}

\begin{itemize}
\tightlist
\item
  \textbf{White Hat}: સુરક્ષા પરીક્ષણ માટે નૈતિક hacking
\item
  \textbf{Black Hat}: ગેરકાયદે લાભ માટે દુષ્ટ hacking
\item
  \textbf{Gray Hat}: નૈતિક અને દુષ્ટ વચ્ચે
\end{itemize}

\textbf{સાવચેતીઓ:}

\begin{itemize}
\tightlist
\item
  \textbf{મજબૂત Passwords}: જટિલ, અનોખા passwords વાપરો
\item
  \textbf{Software Updates}: સિસ્ટમ patched રાખો
\item
  \textbf{Firewall}: અનધિકૃત access block કરો
\item
  \textbf{Antivirus}: Malware detect અને remove કરો
\end{itemize}

\textbf{સુરક્ષા પગલાઓ:}

\begin{itemize}
\tightlist
\item
  \textbf{શિક્ષણ}: વપરાશકર્તા જાગૃતિ તાલીમ
\item
  \textbf{Backup}: નિયમિત ડેટા backup
\item
  \textbf{Monitoring}: નેટવર્ક traffic analysis
\end{itemize}

\end{solutionbox}
\begin{mnemonicbox}
``સુરક્ષિત સિસ્ટમ સમાજ બચાવે''

\end{mnemonicbox}
\begin{center}\rule{0.5\linewidth}{0.5pt}\end{center}

\subsection*{પ્રશ્ન 5(ક) [7
ગુણ]}\label{uxaaauxab0uxab6uxaa8-5uxa95-7-uxa97uxaa3}

\textbf{IP સુરક્ષાને વિગતવાર સમજાવો.}

\begin{solutionbox}

\begin{center}
\textbf{Mermaid Diagram (Code)}
\begin{verbatim}
{Shaded}
{Highlighting}[]
graph LR
    A[IP Security {- IPSec] {-}{-}{} B[Authentication Header {-} AH]}
    A {-{-}{} C[Encapsulating Security Payload {-} ESP]}
    A {-{-}{} D[Security Association {-} SA]}

    B {-{-}{} B1[Data Integrity]}
    C {-{-}{} C1[Data Confidentiality]}
    D {-{-}{} D1[Security Parameters]}
{Highlighting}
{Shaded}
\end{verbatim}
\end{center}

\textbf{IPSec ઘટકો:}

{\def\LTcaptype{none} % do not increment counter
\begin{longtable}[]{@{}
  >{\raggedright\arraybackslash}p{(\linewidth - 4\tabcolsep) * \real{0.2821}}
  >{\raggedright\arraybackslash}p{(\linewidth - 4\tabcolsep) * \real{0.2564}}
  >{\raggedright\arraybackslash}p{(\linewidth - 4\tabcolsep) * \real{0.4615}}@{}}
\toprule\noalign{}
\begin{minipage}[b]{\linewidth}\raggedright
ઘટક
\end{minipage} & \begin{minipage}[b]{\linewidth}\raggedright
કાર્ય
\end{minipage} & \begin{minipage}[b]{\linewidth}\raggedright
સુરક્ષા સેવા
\end{minipage} \\
\midrule\noalign{}
\endhead
\bottomrule\noalign{}
\endlastfoot
\textbf{AH} & Authentication Header & Data integrity, authentication \\
\textbf{ESP} & Encapsulating Security Payload & Confidentiality,
integrity \\
\textbf{SA} & Security Association & Security parameters \\
\end{longtable}
}

\textbf{IPSec Modes:}

{\def\LTcaptype{none} % do not increment counter
\begin{longtable}[]{@{}lll@{}}
\toprule\noalign{}
Mode & વર્ણન & ઉપયોગ \\
\midrule\noalign{}
\endhead
\bottomrule\noalign{}
\endlastfoot
\textbf{Transport} & ફક્ત payload સુરક્ષિત કરે & Host-to-host \\
\textbf{Tunnel} & સંપૂર્ણ packet સુરક્ષિત કરે & Network-to-network \\
\end{longtable}
}

\textbf{IPSec Protocols:}

\begin{itemize}
\tightlist
\item
  \textbf{IKE}: Key management માટે Internet Key Exchange
\item
  \textbf{ISAKMP}: Internet Security Association and Key Management
\item
  \textbf{DES/3DES/AES}: Encryption algorithms
\end{itemize}

\textbf{સુરક્ષા સેવાઓ:}

\begin{itemize}
\tightlist
\item
  \textbf{Authentication}: Sender identity verify કરે
\item
  \textbf{Integrity}: ડેટા modified નથી તેની ખાતરી
\item
  \textbf{Confidentiality}: ડેટા content encrypt કરે
\item
  \textbf{Non-repudiation}: મોકલવાનો ઇનકાર અટકાવે
\end{itemize}

\textbf{IPSec પ્રક્રિયા:}

\begin{enumerate}
\tightlist
\item
  \textbf{Policy Definition}: સુરક્ષા આવશ્યકતાઓ define કરો
\item
  \textbf{Key Exchange}: IKE વાપરીને shared keys સ્થાપિત કરો
\item
  \textbf{SA Establishment}: Security association બનાવો
\item
  \textbf{Data Protection}: Packets પર AH/ESP લાગુ કરો
\item
  \textbf{Transmission}: સુરક્ષિત packets મોકલો
\end{enumerate}

\textbf{ઉપયોગ:}

\begin{itemize}
\tightlist
\item
  \textbf{VPN}: Virtual Private Networks
\item
  \textbf{Remote Access}: સુરક્ષિત remote connections
\item
  \textbf{Site-to-Site}: Branch offices કનેક્ટ કરો
\end{itemize}

\textbf{ફાયદાઓ:}

\begin{itemize}
\tightlist
\item
  \textbf{Transparent સુરક્ષા}: Network layer પર કામ કરે
\item
  \textbf{મજબૂત Authentication}: Cryptographic verification
\item
  \textbf{લવચીક Implementation}: અનેક algorithms support કરે
\end{itemize}

\end{solutionbox}
\begin{mnemonicbox}
``IPSec Authenticates, Encrypts, Secures''

\end{mnemonicbox}
\begin{center}\rule{0.5\linewidth}{0.5pt}\end{center}

\subsection*{પ્રશ્ન 5(અ OR) [3
ગુણ]}\label{uxaaauxab0uxab6uxaa8-5uxa85-or-3-uxa97uxaa3}

\textbf{Wireless LAN સમજાવો.}

\begin{solutionbox}

\textbf{Wireless LAN લાક્ષણિકતાઓ:}

{\def\LTcaptype{none} % do not increment counter
\begin{longtable}[]{@{}ll@{}}
\toprule\noalign{}
વિશેષતા & વર્ણન \\
\midrule\noalign{}
\endhead
\bottomrule\noalign{}
\endlastfoot
\textbf{Standard} & IEEE 802.11 \\
\textbf{Frequency} & 2.4 GHz, 5 GHz \\
\textbf{Access Method} & CSMA/CA \\
\end{longtable}
}

\textbf{WLAN ઘટકો:}

\begin{itemize}
\tightlist
\item
  \textbf{Access Point}: કેન્દ્રીય wireless hub
\item
  \textbf{Wireless Clients}: Laptops, phones, tablets
\item
  \textbf{SSID}: નેટવર્ક identifier
\end{itemize}

\textbf{WLAN Standards:}

\begin{itemize}
\tightlist
\item
  \textbf{802.11a}: 54 Mbps, 5 GHz
\item
  \textbf{802.11g}: 54 Mbps, 2.4 GHz
\item
  \textbf{802.11n}: 600 Mbps, MIMO
\end{itemize}

\end{solutionbox}
\begin{mnemonicbox}
``Wireless તરંગો કામ કરે''

\end{mnemonicbox}
\begin{center}\rule{0.5\linewidth}{0.5pt}\end{center}

\subsection*{પ્રશ્ન 5(બ OR) [4
ગુણ]}\label{uxaaauxab0uxab6uxaa8-5uxaac-or-4-uxa97uxaa3}

\textbf{Symmetric અને asymmetric encryption algorithms વચ્ચેનો ભેદ આપો}

\begin{solutionbox}

\textbf{Encryption Algorithm સરખામણી:}

{\def\LTcaptype{none} % do not increment counter
\begin{longtable}[]{@{}lll@{}}
\toprule\noalign{}
વિશેષતા & Symmetric & Asymmetric \\
\midrule\noalign{}
\endhead
\bottomrule\noalign{}
\endlastfoot
\textbf{Keys} & એક shared key & Key pair (public/private) \\
\textbf{સ્પીડ} & ઝડપી & ધીમી \\
\textbf{Key Distribution} & મુશ્કેલ & સરળ \\
\textbf{ઉદાહરણ} & AES, DES & RSA, ECC \\
\end{longtable}
}

\textbf{Symmetric Encryption:}

\begin{itemize}
\tightlist
\item
  \textbf{સમાન Key}: Encryption અને decryption સમાન key વાપરે
\item
  \textbf{ઝડપી Processing}: મોટા ડેટા માટે કાર્યક્ષમ
\item
  \textbf{Key Management}: Key distribution માં પડકાર
\end{itemize}

\textbf{Asymmetric Encryption:}

\begin{itemize}
\tightlist
\item
  \textbf{Key Pair}: Public key encrypt કરે, private key decrypt કરે
\item
  \textbf{Digital Signatures}: Non-repudiation support
\item
  \textbf{સુરક્ષિત Communication}: પહેલાંથી key exchange ની જરૂર નથી
\end{itemize}

\textbf{ઉપયોગ:}

\begin{itemize}
\tightlist
\item
  \textbf{Symmetric}: Bulk data encryption, disk encryption
\item
  \textbf{Asymmetric}: Key exchange, digital certificates
\end{itemize}

\end{solutionbox}
\begin{mnemonicbox}
``Symmetric સમાન, Asymmetric જોડી''

\end{mnemonicbox}
\begin{center}\rule{0.5\linewidth}{0.5pt}\end{center}

\subsection*{પ્રશ્ન 5(ક OR) [7
ગુણ]}\label{uxaaauxab0uxab6uxaa8-5uxa95-or-7-uxa97uxaa3}

\textbf{Information Technology (Amendment) Act, 2008 નું સંક્ષિપ્ત વર્ણન કરો
અને ભારતમાં cyber laws પર તેની અસર સમજાવો.}

\begin{solutionbox}

\textbf{IT Act 2008 મુખ્ય જોગવાઈઓ:}

{\def\LTcaptype{none} % do not increment counter
\begin{longtable}[]{@{}lll@{}}
\toprule\noalign{}
કલમ & અપરાધ & દંડ \\
\midrule\noalign{}
\endhead
\bottomrule\noalign{}
\endlastfoot
\textbf{66} & Computer hacking & 3 વર્ષ કેદ \\
\textbf{66A} & અપમાનજનક સંદેશા & 3 વર્ષ + દંડ \\
\textbf{66B} & ઓળખ ચોરી & 3 વર્ષ + દંડ \\
\textbf{66C} & Password ચોરી & 3 વર્ષ + દંડ \\
\textbf{66D} & Computer વાપરીને છેતરપિંડી & 3 વર્ષ + દંડ \\
\end{longtable}
}

\begin{center}
\textbf{Mermaid Diagram (Code)}
\begin{verbatim}
{Shaded}
{Highlighting}[]
graph TD
    A[IT Act 2008] {-{-}{} B[Cyber Crimes]}
    A {-{-}{} C[Data Protection]}
    A {-{-}{} D[Digital Signatures]}
    A {-{-}{} E[Penalties]}

    B {-{-}{} B1[Hacking, Identity Theft]}
    C {-{-}{} C1[Sensitive Personal Data]}
    D {-{-}{} D1[Legal Validity]}
    E {-{-}{} E1[Imprisonment + Fine]}
{Highlighting}
{Shaded}
\end{verbatim}
\end{center}

\textbf{મુખ્ય સુધારાઓ:}

{\def\LTcaptype{none} % do not increment counter
\begin{longtable}[]{@{}lll@{}}
\toprule\noalign{}
સુધારો & વર્ણન & અસર \\
\midrule\noalign{}
\endhead
\bottomrule\noalign{}
\endlastfoot
\textbf{કલમ 66A} & ઓનલાઇન અપમાનજનક સામગ્રી & Cyber bullying ને ગુનો
બનાવ્યો \\
\textbf{કલમ 69} & સરકારી interception & Monitoring શક્તિઓ \\
\textbf{કલમ 79} & Intermediary જવાબદારી & Platform જવાબદારીઓ \\
\end{longtable}
}

\textbf{મુખ્ય વિશેષતાઓ:}

\begin{itemize}
\tightlist
\item
  \textbf{Extraterritorial Jurisdiction}: ભારતીય computers ને અસર કરતા
  ભારત બહારના અપરાધો પર લાગુ
\item
  \textbf{Cyber Appellate Tribunal}: વિશિષ્ટ adjudication body
\item
  \textbf{વળતર}: ડેટા breach માટે ₹5 કરોડ સુધીનું નુકસાન
\end{itemize}

\textbf{ડેટા સુરક્ષા જોગવાઈઓ:}

\begin{itemize}
\tightlist
\item
  \textbf{Sensitive Personal Data}: નાણાકીય, આરોગ્ય ડેટા માટે વિશેષ સુરક્ષા
\item
  \textbf{Reasonable Security}: સંસ્થાઓએ પર્યાપ્ત પગલાં લાગુ કરવા
\item
  \textbf{Breach Notification}: સુરક્ષા ઘટનાઓની ફરજિયાત જાણ
\end{itemize}

\textbf{Digital Signature ફ્રેમવર્ક:}

\begin{itemize}
\tightlist
\item
  \textbf{કાનૂની માન્યતા}: Electronic signatures કાનૂની રીતે માન્ય
\item
  \textbf{Certification Authority}: લાયસન્સ મેળવેલી સંસ્થાઓ digital
  certificates જારી કરે
\item
  \textbf{Non-repudiation}: Electronic transactions નો ઇનકાર અટકાવે
\end{itemize}

\textbf{Cybercrime વર્ગો:}

\begin{itemize}
\tightlist
\item
  \textbf{Computer સંબંધિત અપરાધો}: અનધિકૃત પ્રવેશ, ડેટા ચોરી
\item
  \textbf{સંચાર અપરાધો}: અશ્લીલ સામગ્રી, cyber stalking
\item
  \textbf{ઓળખ અપરાધો}: Impersonation, છેતરપિંડી
\end{itemize}

\textbf{કાયદા અમલીકરણ શક્તિઓ:}

\begin{itemize}
\tightlist
\item
  \textbf{શોધ અને જપ્તી}: Computer systems તપાસવાની સત્તા
\item
  \textbf{Preservation Orders}: તપાસ માટે ડેટા retention જરૂરી
\item
  \textbf{Blocking Orders}: Internet માંથી અપમાનજનક સામગ્રી દૂર કરવા
\end{itemize}

\textbf{ઉદ્યોગ પર અસર:}

\begin{itemize}
\tightlist
\item
  \textbf{Compliance આવશ્યકતાઓ}: સંસ્થાઓએ સુરક્ષા પગલાં અપનાવવા
\item
  \textbf{જવાબદારી ફ્રેમવર્ક}: Service providers માટે સ્પષ્ટ જવાબદારીઓ
\item
  \textbf{Business Process}: E-commerce, digital transactions માટે કાનૂની
  ફ્રેમવર્ક
\end{itemize}

\textbf{પડકારો:}

\begin{itemize}
\tightlist
\item
  \textbf{ટેકનોલોજી Gap}: કાયદો ટેકનોલોજી સાથે તાલ મેળવવામાં સંઘર્ષ
\item
  \textbf{Jurisdiction મુદ્દાઓ}: Cross-border cybercrime તપાસ
\item
  \textbf{Privacy ચિંતાઓ}: સુરક્ષા અને વ્યક્તિગત અધિકારો વચ્ચે સંતુલન
\end{itemize}

\textbf{તાજેતરના વિકાસ:}

\begin{itemize}
\tightlist
\item
  \textbf{Personal Data Protection Bill}: વ્યાપક privacy કાયદો
\item
  \textbf{Cybersecurity Framework}: રાષ્ટ્રીય cyber security વ્યૂહરચના
\item
  \textbf{Digital India}: સરકારી digitization પહેલ
\end{itemize}

\end{solutionbox}
\begin{mnemonicbox}
``IT Act Digital India બચાવે''

\end{mnemonicbox}

\end{document}
