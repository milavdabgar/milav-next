\documentclass[10pt,a4paper]{article}

% content/resources/templates/preamble.tex
\usepackage[margin=0.6in]{geometry}
\author{Milav Dabgar}
\usepackage{amsmath,amssymb,amsthm}
\usepackage{booktabs}
\usepackage{multirow}
\usepackage{xcolor}
\usepackage{tcolorbox}
\tcbuselibrary{breakable,skins}
\usepackage[colorlinks=true,linkcolor=blue]{hyperref}
\usepackage{titlesec}
\usepackage{enumitem}
\usepackage{tikz}
\usepackage{pgfplots}
\usepackage{circuitikz}
\usepackage[version=4]{mhchem}
\usepackage{longtable}
\usepackage{array}
\usepackage{float}
\usepackage{caption}
\usepackage{listings}

\lstset{
  basicstyle=\small\ttfamily,
  breaklines=true,
  breakatwhitespace=false,
  postbreak=\mbox{\textcolor{red}{$\hookrightarrow$}\space},
  float=false,
  numbers=left,
  numberstyle=\tiny\color{gray},
  numbersep=10pt,
  xleftmargin=2em,
  keywordstyle=\color{blue},
  commentstyle=\color{green!60!black},
  stringstyle=\color{purple},
  backgroundcolor=\color{gray!5},
  showstringspaces=false,
  tabsize=2,
  captionpos=b,
  keepspaces=true,
  columns=flexible
}

\pgfplotsset{compat=1.18}
\usetikzlibrary{shapes,arrows,positioning,calc,patterns,decorations.pathmorphing,decorations.markings,arrows.meta}

% Color scheme
\definecolor{headcolor}{RGB}{0,102,204}
\definecolor{keycolor}{RGB}{220,20,60}
\definecolor{solutioncolor}{RGB}{34,139,34}
\definecolor{mnemoniccolor}{RGB}{148,0,211}
\definecolor{codecolor}{RGB}{0,0,100}

% Spacing
\setlength{\parskip}{3pt}
\setlist[itemize]{nosep}
\setlist[enumerate]{nosep}

% Title formatting
\titleformat{\section}{\Large\bfseries\color{headcolor}}{\thesection}{1em}{}
\titleformat{\subsection}{\large\bfseries\color{headcolor}}{\thesubsection}{1em}{}

% Pandoc tightlist compatibility
\providecommand{\tightlist}{%
  \setlength{\itemsep}{0pt}\setlength{\parskip}{0pt}}

% Pandoc longtable compatibility
\newcounter{none}
\def\thenone{}


% content/resources/templates/english-boxes.tex
% This file is currently empty - it exists to maintain consistency with the import structure.
% Add custom environments here if needed in the future.


\begin{document}

\begin{center}
{\Huge\bfseries\color{headcolor} Subject Name Solutions}\\[5pt]
{\LARGE 4361101 -- Summer 2024}\\[3pt]
{\large Semester 1 Study Material}\\[3pt]
{\normalsize\textit{Detailed Solutions and Explanations}}
\end{center}

\vspace{10pt}

\subsection*{Question 1(a) [3 marks]}\label{q1a}

\textbf{List the different Network Topologies and discuss any one in
detail.}

\begin{solutionbox}

{\def\LTcaptype{none} % do not increment counter
\begin{longtable}[]{@{}ll@{}}
\toprule\noalign{}
Topology & Description \\
\midrule\noalign{}
\endhead
\bottomrule\noalign{}
\endlastfoot
\textbf{Star} & All devices connected to central hub/switch \\
\textbf{Ring} & Devices connected in circular fashion \\
\textbf{Bus} & All devices connected to single cable \\
\textbf{Mesh} & Every device connected to every other device \\
\textbf{Tree} & Hierarchical structure with root node \\
\textbf{Hybrid} & Combination of two or more topologies \\
\end{longtable}
}

\textbf{Star Topology Details:}

\begin{itemize}
\tightlist
\item
  \textbf{Central Hub}: All nodes connect to one central device
\item
  \textbf{Point-to-Point}: Each connection is dedicated between node and
  hub
\item
  \textbf{Easy Management}: Simple to install and troubleshoot
\end{itemize}

\end{solutionbox}
\begin{mnemonicbox}
``STAR = Single Terminal All Reach''

\end{mnemonicbox}
\subsection*{Question 1(b) [4 marks]}\label{q1b}

\textbf{Explain how point-to-point and broadcast transmission
technologies are used in modern communication systems with examples of
real-world applications and discuss their advantages and limitations.}

\begin{solutionbox}

{\def\LTcaptype{none} % do not increment counter
\begin{longtable}[]{@{}
  >{\raggedright\arraybackslash}p{(\linewidth - 4\tabcolsep) * \real{0.3077}}
  >{\raggedright\arraybackslash}p{(\linewidth - 4\tabcolsep) * \real{0.4103}}
  >{\raggedright\arraybackslash}p{(\linewidth - 4\tabcolsep) * \real{0.2821}}@{}}
\toprule\noalign{}
\begin{minipage}[b]{\linewidth}\raggedright
Technology
\end{minipage} & \begin{minipage}[b]{\linewidth}\raggedright
Point-to-Point
\end{minipage} & \begin{minipage}[b]{\linewidth}\raggedright
Broadcast
\end{minipage} \\
\midrule\noalign{}
\endhead
\bottomrule\noalign{}
\endlastfoot
\textbf{Connection} & Direct link between two devices & One-to-many
communication \\
\textbf{Example} & Telephone, VPN tunnels & Radio, TV, WiFi \\
\textbf{Data Flow} & Bidirectional & Unidirectional/Multidirectional \\
\end{longtable}
}

\textbf{Point-to-Point Applications:}

\begin{itemize}
\tightlist
\item
  \textbf{Dedicated Lines}: Leased lines between offices
\item
  \textbf{Satellite Links}: Ground station to satellite communication
\item
  \textbf{Cable Modems}: Home to ISP connection
\end{itemize}

\textbf{Broadcast Applications:}

\begin{itemize}
\tightlist
\item
  \textbf{WiFi Networks}: Router broadcasts to multiple devices
\item
  \textbf{Television}: One transmitter to many receivers
\end{itemize}

\end{solutionbox}
\begin{mnemonicbox}
``P2P = Private Path, Broadcast = Big Audience''

\end{mnemonicbox}
\subsection*{Question 1(c) [7 marks]}\label{q1c}

\textbf{Describe OSI model with function of all layers.}

\begin{solutionbox}

{\def\LTcaptype{none} % do not increment counter
\begin{longtable}[]{@{}lll@{}}
\toprule\noalign{}
Layer & Name & Function \\
\midrule\noalign{}
\endhead
\bottomrule\noalign{}
\endlastfoot
\textbf{7} & Application & User interface, network services \\
\textbf{6} & Presentation & Data encryption, compression, formatting \\
\textbf{5} & Session & Establishes, manages, terminates sessions \\
\textbf{4} & Transport & Reliable data transfer, error correction \\
\textbf{3} & Network & Routing, logical addressing (IP) \\
\textbf{2} & Data Link & Frame formatting, error detection \\
\textbf{1} & Physical & Bit transmission, electrical signals \\
\end{longtable}
}

\begin{center}
\textbf{Mermaid Diagram (Code)}
\begin{verbatim}
{Shaded}
{Highlighting}[]
graph LR
    A[Application Layer 7] {-{-}{} B[Presentation Layer 6]}
    B {-{-}{} C[Session Layer 5]}
    C {-{-}{} D[Transport Layer 4]}
    D {-{-}{} E[Network Layer 3]}
    E {-{-}{} F[Data Link Layer 2]}
    F {-{-}{} G[Physical Layer 1]}
{Highlighting}
{Shaded}
\end{verbatim}
\end{center}

\textbf{Key Functions:}

\begin{itemize}
\tightlist
\item
  \textbf{Upper Layers (5-7)}: Handle application-related services
\item
  \textbf{Lower Layers (1-4)}: Handle data transmission and routing
\item
  \textbf{Encapsulation}: Each layer adds its own header
\end{itemize}

\end{solutionbox}
\begin{mnemonicbox}
``All People Seem To Need Data Processing''

\end{mnemonicbox}
\subsection*{Question 1(c OR) [7
marks]}\label{question-1c-or-7-marks}

\textbf{Write a functional description of all layer of TCP/IP model.}

\begin{solutionbox}

{\def\LTcaptype{none} % do not increment counter
\begin{longtable}[]{@{}
  >{\raggedright\arraybackslash}p{(\linewidth - 6\tabcolsep) * \real{0.2059}}
  >{\raggedright\arraybackslash}p{(\linewidth - 6\tabcolsep) * \real{0.1765}}
  >{\raggedright\arraybackslash}p{(\linewidth - 6\tabcolsep) * \real{0.2941}}
  >{\raggedright\arraybackslash}p{(\linewidth - 6\tabcolsep) * \real{0.3235}}@{}}
\toprule\noalign{}
\begin{minipage}[b]{\linewidth}\raggedright
Layer
\end{minipage} & \begin{minipage}[b]{\linewidth}\raggedright
Name
\end{minipage} & \begin{minipage}[b]{\linewidth}\raggedright
Function
\end{minipage} & \begin{minipage}[b]{\linewidth}\raggedright
Protocols
\end{minipage} \\
\midrule\noalign{}
\endhead
\bottomrule\noalign{}
\endlastfoot
\textbf{4} & Application & User services, applications & HTTP, FTP,
SMTP, DNS \\
\textbf{3} & Transport & End-to-end communication & TCP, UDP \\
\textbf{2} & Internet & Routing, logical addressing & IP, ICMP, ARP \\
\textbf{1} & Network Access & Physical transmission & Ethernet, WiFi \\
\end{longtable}
}

\begin{center}
\textbf{Mermaid Diagram (Code)}
\begin{verbatim}
{Shaded}
{Highlighting}[]
graph LR
    A[Application Layer] {-{-}{} B[Transport Layer]}
    B {-{-}{} C[Internet Layer]}
    C {-{-}{} D[Network Access Layer]}
{Highlighting}
{Shaded}
\end{verbatim}
\end{center}

\textbf{Layer Functions:}

\begin{itemize}
\tightlist
\item
  \textbf{Application}: Provides network services to applications
\item
  \textbf{Transport}: Ensures reliable or unreliable delivery
\item
  \textbf{Internet}: Routes packets across networks using IP addresses
\item
  \textbf{Network Access}: Handles physical transmission media
\end{itemize}

\end{solutionbox}
\begin{mnemonicbox}
``Applications Transport Internet Networks''

\end{mnemonicbox}
\subsection*{Question 2(a) [3 marks]}\label{q2a}

\textbf{Describe Function of firewall in network security.}

\begin{solutionbox}

\textbf{Firewall Functions:}

\begin{itemize}
\tightlist
\item
  \textbf{Packet Filtering}: Controls incoming and outgoing network
  traffic
\item
  \textbf{Access Control}: Blocks unauthorized access attempts
\item
  \textbf{Traffic Monitoring}: Logs and analyzes network activity
\end{itemize}

\textbf{Types:}

\begin{itemize}
\tightlist
\item
  \textbf{Hardware Firewall}: Physical device protecting entire network
\item
  \textbf{Software Firewall}: Program installed on individual computers
\item
  \textbf{Stateful Inspection}: Tracks connection states and contexts
\end{itemize}

\end{solutionbox}
\begin{mnemonicbox}
``Firewall = Filter, Access, Monitor''

\end{mnemonicbox}
\subsection*{Question 2(b) [4 marks]}\label{q2b}

\textbf{Compare FDDI (Fiber Distributed Data Interface) and CDDI (Copper
Distributed Data Interface) in terms of their key characteristics,
advantages, and applications.}

\begin{solutionbox}

{\def\LTcaptype{none} % do not increment counter
\begin{longtable}[]{@{}lll@{}}
\toprule\noalign{}
Feature & FDDI & CDDI \\
\midrule\noalign{}
\endhead
\bottomrule\noalign{}
\endlastfoot
\textbf{Medium} & Optical fiber & Twisted pair copper \\
\textbf{Speed} & 100 Mbps & 100 Mbps \\
\textbf{Distance} & Up to 200 km & Up to 100 meters \\
\textbf{Cost} & Higher & Lower \\
\textbf{Security} & Higher (difficult to tap) & Lower (easier to tap) \\
\textbf{Installation} & Complex & Simple \\
\end{longtable}
}

\textbf{FDDI Advantages:}

\begin{itemize}
\tightlist
\item
  \textbf{Long Distance}: Supports campus-wide networks
\item
  \textbf{High Security}: Immune to electromagnetic interference
\item
  \textbf{Reliability}: Better error detection and recovery
\end{itemize}

\textbf{CDDI Advantages:}

\begin{itemize}
\tightlist
\item
  \textbf{Cost Effective}: Uses existing copper infrastructure
\item
  \textbf{Easy Installation}: Standard twisted pair cabling
\item
  \textbf{Compatibility}: Works with existing network equipment
\end{itemize}

\end{solutionbox}
\begin{mnemonicbox}
``FDDI = Fiber Distance, CDDI = Copper Cost''

\end{mnemonicbox}
\subsection*{Question 2(c) [7 marks]}\label{q2c}

\textbf{Explain and distinguish Ethernet, Fast Ethernet, Gigabit
Ethernet.}

\begin{solutionbox}

{\def\LTcaptype{none} % do not increment counter
\begin{longtable}[]{@{}
  >{\raggedright\arraybackslash}p{(\linewidth - 8\tabcolsep) * \real{0.1333}}
  >{\raggedright\arraybackslash}p{(\linewidth - 8\tabcolsep) * \real{0.1556}}
  >{\raggedright\arraybackslash}p{(\linewidth - 8\tabcolsep) * \real{0.2222}}
  >{\raggedright\arraybackslash}p{(\linewidth - 8\tabcolsep) * \real{0.2667}}
  >{\raggedright\arraybackslash}p{(\linewidth - 8\tabcolsep) * \real{0.2222}}@{}}
\toprule\noalign{}
\begin{minipage}[b]{\linewidth}\raggedright
Type
\end{minipage} & \begin{minipage}[b]{\linewidth}\raggedright
Speed
\end{minipage} & \begin{minipage}[b]{\linewidth}\raggedright
Standard
\end{minipage} & \begin{minipage}[b]{\linewidth}\raggedright
Cable Type
\end{minipage} & \begin{minipage}[b]{\linewidth}\raggedright
Distance
\end{minipage} \\
\midrule\noalign{}
\endhead
\bottomrule\noalign{}
\endlastfoot
\textbf{Ethernet} & 10 Mbps & 802.3 & Coax/UTP & 100m \\
\textbf{Fast Ethernet} & 100 Mbps & 802.3u & UTP Cat5 & 100m \\
\textbf{Gigabit Ethernet} & 1000 Mbps & 802.3z/ab & Cat5e/6, Fiber &
100m/5km \\
\end{longtable}
}

\begin{center}
\textbf{Mermaid Diagram (Code)}
\begin{verbatim}
{Shaded}
{Highlighting}[]
graph LR
    A[Ethernet 10 Mbps] {-{-}{} B[Fast Ethernet 100 Mbps]}
    B {-{-}{} C[Gigabit Ethernet 1000 Mbps]}
{Highlighting}
{Shaded}
\end{verbatim}
\end{center}

\textbf{Key Differences:}

\begin{itemize}
\tightlist
\item
  \textbf{Speed Evolution}: 10x increase at each generation
\item
  \textbf{Media Support}: From coax to twisted pair to fiber
\item
  \textbf{Applications}: LAN backbone, server connections, desktop
\item
  \textbf{Backward Compatibility}: Newer standards support older devices
\end{itemize}

\textbf{Standards:}

\begin{itemize}
\tightlist
\item
  \textbf{10Base-T}: 10 Mbps over twisted pair
\item
  \textbf{100Base-TX}: 100 Mbps over Category 5 UTP
\item
  \textbf{1000Base-T}: 1 Gbps over Category 5e/6 UTP
\end{itemize}

\end{solutionbox}
\begin{mnemonicbox}
``Every Fast Gigabit = 10, 100, 1000''

\end{mnemonicbox}
\subsection*{Question 2(a OR) [3
marks]}\label{question-2a-or-3-marks}

\textbf{Explain its role and function of router within a network
infrastructure.}

\begin{solutionbox}

\textbf{Router Functions:}

\begin{itemize}
\tightlist
\item
  \textbf{Packet Forwarding}: Routes data packets between different
  networks
\item
  \textbf{Path Determination}: Selects best route using routing tables
\item
  \textbf{Network Isolation}: Separates broadcast domains
\end{itemize}

\textbf{Key Roles:}

\begin{itemize}
\tightlist
\item
  \textbf{Inter-network Communication}: Connects LANs to WANs
\item
  \textbf{Traffic Management}: Controls data flow between networks
\item
  \textbf{Protocol Translation}: Converts between different network
  protocols
\end{itemize}

\end{solutionbox}
\begin{mnemonicbox}
``Router = Route, Isolate, Connect''

\end{mnemonicbox}
\subsection*{Question 2(b OR) [4
marks]}\label{question-2b-or-4-marks}

\textbf{Explain the structure of FDDI (Fiber Distributed Data Interface)
and give its advantages.}

\begin{solutionbox}

\textbf{FDDI Structure:}

\begin{verbatim}
    Node A {-{-}{-}{-}{-}{-}{-}{-} Node B}
      |               |
      |               |
    Node D {-{-}{-}{-}{-}{-}{-}{-} Node C}
    
    Primary Ring: Clockwise
    Secondary Ring: Counter{-clockwise}
\end{verbatim}

\textbf{Components:}

\begin{itemize}
\tightlist
\item
  \textbf{Dual Ring}: Primary and secondary rings for redundancy
\item
  \textbf{Token Passing}: Uses token for media access control
\item
  \textbf{Concentrators}: Connect multiple stations to ring
\end{itemize}

\textbf{Advantages:}

\begin{itemize}
\tightlist
\item
  \textbf{High Reliability}: Dual ring provides fault tolerance
\item
  \textbf{Fast Speed}: 100 Mbps data transmission rate
\item
  \textbf{Long Distance}: Supports up to 200 km ring circumference
\item
  \textbf{Self-Healing}: Automatic reconfiguration when link fails
\end{itemize}

\end{solutionbox}
\begin{mnemonicbox}
``FDDI = Fast, Dual, Distance, Immune''

\end{mnemonicbox}
\subsection*{Question 2(c OR) [7
marks]}\label{question-2c-or-7-marks}

\textbf{Explain roll of network Devices. Describe in brief about all the
devices.}

\begin{solutionbox}

{\def\LTcaptype{none} % do not increment counter
\begin{longtable}[]{@{}
  >{\raggedright\arraybackslash}p{(\linewidth - 4\tabcolsep) * \real{0.3200}}
  >{\raggedright\arraybackslash}p{(\linewidth - 4\tabcolsep) * \real{0.2800}}
  >{\raggedright\arraybackslash}p{(\linewidth - 4\tabcolsep) * \real{0.4000}}@{}}
\toprule\noalign{}
\begin{minipage}[b]{\linewidth}\raggedright
Device
\end{minipage} & \begin{minipage}[b]{\linewidth}\raggedright
Layer
\end{minipage} & \begin{minipage}[b]{\linewidth}\raggedright
Function
\end{minipage} \\
\midrule\noalign{}
\endhead
\bottomrule\noalign{}
\endlastfoot
\textbf{Repeater} & Physical & Regenerates signals, extends distance \\
\textbf{Hub} & Physical & Connects multiple devices, shared bandwidth \\
\textbf{Bridge} & Data Link & Connects LANs, reduces collisions \\
\textbf{Switch} & Data Link & Intelligent hub, dedicated bandwidth \\
\textbf{Router} & Network & Connects different networks, routing \\
\textbf{Gateway} & All Layers & Protocol conversion, network
interconnection \\
\end{longtable}
}

\begin{center}
\textbf{Mermaid Diagram (Code)}
\begin{verbatim}
{Shaded}
{Highlighting}[]
graph TD
    A[Physical Layer] {-{-}{} B[Repeater, Hub]}
    C[Data Link Layer] {-{-}{} D[Bridge, Switch]}
    E[Network Layer] {-{-}{} F[Router]}
    G[All Layers] {-{-}{} H[Gateway]}
{Highlighting}
{Shaded}
\end{verbatim}
\end{center}

\textbf{Device Functions:}

\begin{itemize}
\tightlist
\item
  \textbf{Repeater}: Amplifies and regenerates signals
\item
  \textbf{Hub}: Simple connection point for multiple devices
\item
  \textbf{Bridge}: Intelligent forwarding based on MAC addresses
\item
  \textbf{Switch}: High-performance bridge with multiple ports
\item
  \textbf{Router}: Intelligent path selection between networks
\item
  \textbf{Gateway}: Complete protocol stack conversion
\end{itemize}

\end{solutionbox}
\begin{mnemonicbox}
``Repeat, Hub, Bridge, Switch, Route, Gateway''

\end{mnemonicbox}
\subsection*{Question 3(a) [3 marks]}\label{q3a}

\textbf{Name any three data link layer protocol and explain any one in
detail.}

\begin{solutionbox}

\textbf{Data Link Layer Protocols:}

\begin{itemize}
\tightlist
\item
  \textbf{HDLC} (High-Level Data Link Control)
\item
  \textbf{PPP} (Point-to-Point Protocol)
\item
  \textbf{Ethernet} (IEEE 802.3)
\end{itemize}

\textbf{HDLC Protocol Details:}

\begin{itemize}
\tightlist
\item
  \textbf{Frame Structure}: Flag, Address, Control, Data, FCS, Flag
\item
  \textbf{Error Detection}: Frame Check Sequence (FCS)
\item
  \textbf{Flow Control}: Sliding window mechanism
\end{itemize}

\textbf{HDLC Frame Format:}

\begin{verbatim}
+{-{-}{-}{-}{-}{-}+{-}{-}{-}{-}{-}{-}+{-}{-}{-}{-}{-}{-}+{-}{-}{-}{-}{-}{-}+{-}{-}{-}{-}{-}{-}+{-}{-}{-}{-}{-}{-}+}
| Flag |Addr  |Ctrl  | Data | FCS  | Flag |
| 8bit |8bit  |8bit  |      |16bit | 8bit |
+{-{-}{-}{-}{-}{-}+{-}{-}{-}{-}{-}{-}+{-}{-}{-}{-}{-}{-}+{-}{-}{-}{-}{-}{-}+{-}{-}{-}{-}{-}{-}+{-}{-}{-}{-}{-}{-}+}
\end{verbatim}

\end{solutionbox}
\begin{mnemonicbox}
``HDLC = High Data Link Control''

\end{mnemonicbox}
\subsection*{Question 3(b) [4 marks]}\label{q3b}

\textbf{Explain error control and flow control at data link layer}

\begin{solutionbox}

{\def\LTcaptype{none} % do not increment counter
\begin{longtable}[]{@{}
  >{\raggedright\arraybackslash}p{(\linewidth - 4\tabcolsep) * \real{0.4375}}
  >{\raggedright\arraybackslash}p{(\linewidth - 4\tabcolsep) * \real{0.2812}}
  >{\raggedright\arraybackslash}p{(\linewidth - 4\tabcolsep) * \real{0.2812}}@{}}
\toprule\noalign{}
\begin{minipage}[b]{\linewidth}\raggedright
Control Type
\end{minipage} & \begin{minipage}[b]{\linewidth}\raggedright
Purpose
\end{minipage} & \begin{minipage}[b]{\linewidth}\raggedright
Methods
\end{minipage} \\
\midrule\noalign{}
\endhead
\bottomrule\noalign{}
\endlastfoot
\textbf{Error Control} & Detect and correct transmission errors & CRC,
Checksum, Parity \\
\textbf{Flow Control} & Manage data transmission rate & Stop-and-Wait,
Sliding Window \\
\end{longtable}
}

\textbf{Error Control Methods:}

\begin{itemize}
\tightlist
\item
  \textbf{Detection}: CRC, Checksum identify errors
\item
  \textbf{Correction}: ARQ (Automatic Repeat Request)
\item
  \textbf{Prevention}: Forward Error Correction (FEC)
\end{itemize}

\textbf{Flow Control Methods:}

\begin{itemize}
\tightlist
\item
  \textbf{Stop-and-Wait}: Send one frame, wait for ACK
\item
  \textbf{Sliding Window}: Send multiple frames before ACK
\item
  \textbf{Buffer Management}: Prevent receiver overflow
\end{itemize}

\end{solutionbox}
\begin{mnemonicbox}
``Error = Detect, Flow = Control''

\end{mnemonicbox}
\subsection*{Question 3(c) [7 marks]}\label{q3c}

\textbf{Compare IPv6 and IPv4.}

\begin{solutionbox}

{\def\LTcaptype{none} % do not increment counter
\begin{longtable}[]{@{}lll@{}}
\toprule\noalign{}
Feature & IPv4 & IPv6 \\
\midrule\noalign{}
\endhead
\bottomrule\noalign{}
\endlastfoot
\textbf{Address Length} & 32 bits & 128 bits \\
\textbf{Address Space} & 4.3 billion & 340 undecillion \\
\textbf{Header Size} & 20-60 bytes (variable) & 40 bytes (fixed) \\
\textbf{Notation} & Decimal (192.168.1.1) & Hexadecimal (2001:db8::1) \\
\textbf{Fragmentation} & Router and host & Host only \\
\textbf{Security} & Optional (IPSec) & Built-in (IPSec) \\
\textbf{Configuration} & Manual/DHCP & Auto-configuration \\
\end{longtable}
}

\textbf{IPv4 Example:} 192.168.1.100 \textbf{IPv6 Example:}
2001:0db8:85a3:0000:0000:8a2e:0370:7334

\textbf{Key Differences:}

\begin{itemize}
\tightlist
\item
  \textbf{Address Exhaustion}: IPv4 addresses nearly exhausted
\item
  \textbf{Header Efficiency}: IPv6 simplified header structure
\item
  \textbf{Security}: IPv6 has built-in security features
\item
  \textbf{Quality of Service}: Better QoS support in IPv6
\end{itemize}

\end{solutionbox}
\begin{mnemonicbox}
``IPv6 = Infinite, Integrated, Improved''

\end{mnemonicbox}
\subsection*{Question 3(a OR) [3
marks]}\label{question-3a-or-3-marks}

\textbf{Explain the differences between guided and unguided transmission
media used in computer networks}

\begin{solutionbox}

{\def\LTcaptype{none} % do not increment counter
\begin{longtable}[]{@{}
  >{\raggedright\arraybackslash}p{(\linewidth - 4\tabcolsep) * \real{0.4000}}
  >{\raggedright\arraybackslash}p{(\linewidth - 4\tabcolsep) * \real{0.2667}}
  >{\raggedright\arraybackslash}p{(\linewidth - 4\tabcolsep) * \real{0.3333}}@{}}
\toprule\noalign{}
\begin{minipage}[b]{\linewidth}\raggedright
Media Type
\end{minipage} & \begin{minipage}[b]{\linewidth}\raggedright
Guided
\end{minipage} & \begin{minipage}[b]{\linewidth}\raggedright
Unguided
\end{minipage} \\
\midrule\noalign{}
\endhead
\bottomrule\noalign{}
\endlastfoot
\textbf{Definition} & Physical path exists & No physical path \\
\textbf{Examples} & Twisted pair, Coax, Fiber & Radio, Microwave,
Satellite \\
\textbf{Direction} & Point-to-point & Broadcast \\
\end{longtable}
}

\textbf{Guided Media:}

\begin{itemize}
\tightlist
\item
  \textbf{Twisted Pair}: Telephone lines, LANs
\item
  \textbf{Coaxial Cable}: Cable TV, older networks
\item
  \textbf{Fiber Optic}: High-speed, long-distance
\end{itemize}

\textbf{Unguided Media:}

\begin{itemize}
\tightlist
\item
  \textbf{Radio Waves}: WiFi, Bluetooth
\item
  \textbf{Microwaves}: Point-to-point links
\item
  \textbf{Infrared}: Short-range communication
\end{itemize}

\end{solutionbox}
\begin{mnemonicbox}
``Guided = Ground, Unguided = Air''

\end{mnemonicbox}
\subsection*{Question 3(b OR) [4
marks]}\label{question-3b-or-4-marks}

\textbf{Describe circuit switching and packet switching.}

\begin{solutionbox}

{\def\LTcaptype{none} % do not increment counter
\begin{longtable}[]{@{}lll@{}}
\toprule\noalign{}
Feature & Circuit Switching & Packet Switching \\
\midrule\noalign{}
\endhead
\bottomrule\noalign{}
\endlastfoot
\textbf{Connection} & Dedicated path established & No dedicated path \\
\textbf{Resource Allocation} & Fixed bandwidth & Shared resources \\
\textbf{Example} & Traditional telephone & Internet \\
\textbf{Delay} & Constant & Variable \\
\end{longtable}
}

\textbf{Circuit Switching:}

\begin{itemize}
\tightlist
\item
  \textbf{Setup Phase}: Establishes dedicated connection
\item
  \textbf{Data Transfer}: Continuous transmission
\item
  \textbf{Teardown}: Releases connection resources
\end{itemize}

\textbf{Packet Switching:}

\begin{itemize}
\tightlist
\item
  \textbf{Store-and-Forward}: Packets stored at intermediate nodes
\item
  \textbf{Dynamic Routing}: Each packet routed independently
\item
  \textbf{Resource Sharing}: Bandwidth shared among users
\end{itemize}

\end{solutionbox}
\begin{mnemonicbox}
``Circuit = Continuous, Packet = Pieces''

\end{mnemonicbox}
\subsection*{Question 3(c OR) [7
marks]}\label{question-3c-or-7-marks}

\textbf{Explain IPv4 OR IPv6 in detail.}

\begin{solutionbox}
(IPv4):

\textbf{IPv4 Address Structure:}

\begin{itemize}
\tightlist
\item
  \textbf{32-bit Address}: Divided into 4 octets
\item
  \textbf{Dotted Decimal}: 192.168.1.1 format
\item
  \textbf{Network + Host}: Address split into network and host portions
\end{itemize}

{\def\LTcaptype{none} % do not increment counter
\begin{longtable}[]{@{}lllll@{}}
\toprule\noalign{}
Class & Range & Network Bits & Host Bits & Use \\
\midrule\noalign{}
\endhead
\bottomrule\noalign{}
\endlastfoot
\textbf{A} & 1-126 & 8 & 24 & Large networks \\
\textbf{B} & 128-191 & 16 & 16 & Medium networks \\
\textbf{C} & 192-223 & 24 & 8 & Small networks \\
\end{longtable}
}

\textbf{Special Addresses:}

\begin{itemize}
\tightlist
\item
  \textbf{Loopback}: 127.0.0.1 (local host)
\item
  \textbf{Private}: 192.168.x.x, 10.x.x.x, 172.16-31.x.x
\item
  \textbf{Broadcast}: 255.255.255.255
\end{itemize}

\textbf{Subnetting:}

\begin{itemize}
\tightlist
\item
  \textbf{Subnet Mask}: Identifies network portion
\item
  \textbf{CIDR}: Classless Inter-Domain Routing
\item
  \textbf{Variable Length}: Different subnet sizes
\end{itemize}

\textbf{IPv4 Header:}

\begin{verbatim}
0               16              32
+{-{-}{-}{-}{-}{-}{-}{-}{-}{-}{-}{-}{-}{-}{-}+{-}{-}{-}{-}{-}{-}{-}{-}{-}{-}{-}{-}{-}{-}{-}+}
|Version| IHL   |Type of Service|
+{-{-}{-}{-}{-}{-}{-}{-}{-}{-}{-}{-}{-}{-}{-}+{-}{-}{-}{-}{-}{-}{-}{-}{-}{-}{-}{-}{-}{-}{-}+}
|     Total Length              |
+{-{-}{-}{-}{-}{-}{-}{-}{-}{-}{-}{-}{-}{-}{-}+{-}{-}{-}{-}{-}{-}{-}{-}{-}{-}{-}{-}{-}{-}{-}+}
|Identification |Flags|Fragment |
+{-{-}{-}{-}{-}{-}{-}{-}{-}{-}{-}{-}{-}{-}{-}+{-}{-}{-}{-}{-}{-}{-}{-}{-}{-}{-}{-}{-}{-}{-}+}
| TTL  |Protocol|Header Checksum|
+{-{-}{-}{-}{-}{-}{-}{-}{-}{-}{-}{-}{-}{-}{-}+{-}{-}{-}{-}{-}{-}{-}{-}{-}{-}{-}{-}{-}{-}{-}+}
|     Source Address            |
+{-{-}{-}{-}{-}{-}{-}{-}{-}{-}{-}{-}{-}{-}{-}+{-}{-}{-}{-}{-}{-}{-}{-}{-}{-}{-}{-}{-}{-}{-}+}
|   Destination Address         |
+{-{-}{-}{-}{-}{-}{-}{-}{-}{-}{-}{-}{-}{-}{-}+{-}{-}{-}{-}{-}{-}{-}{-}{-}{-}{-}{-}{-}{-}{-}+}
\end{verbatim}

\end{solutionbox}
\begin{mnemonicbox}
``IPv4 = 4 octets, 32 bits, Classes A-C''

\end{mnemonicbox}
\subsection*{Question 4(a) [3 marks]}\label{q4a}

\textbf{Give full name of ARP and RARP and describe them.}

\begin{solutionbox}

\textbf{Full Names:}

\begin{itemize}
\tightlist
\item
  \textbf{ARP}: Address Resolution Protocol
\item
  \textbf{RARP}: Reverse Address Resolution Protocol
\end{itemize}

{\def\LTcaptype{none} % do not increment counter
\begin{longtable}[]{@{}ll@{}}
\toprule\noalign{}
Protocol & Function \\
\midrule\noalign{}
\endhead
\bottomrule\noalign{}
\endlastfoot
\textbf{ARP} & Maps IP address to MAC address \\
\textbf{RARP} & Maps MAC address to IP address \\
\end{longtable}
}

\textbf{ARP Process:}

\begin{itemize}
\tightlist
\item
  \textbf{Request}: ``Who has IP 192.168.1.1?''
\item
  \textbf{Reply}: ``192.168.1.1 is at MAC 00:1A:2B:3C:4D:5E''
\item
  \textbf{Cache}: Stores mappings for future use
\end{itemize}

\textbf{RARP Process:}

\begin{itemize}
\tightlist
\item
  \textbf{Diskless Workstations}: Get IP from server
\item
  \textbf{Broadcast Request}: Sends MAC address
\item
  \textbf{Server Response}: Returns assigned IP address
\end{itemize}

\end{solutionbox}
\begin{mnemonicbox}
``ARP = Address to MAC, RARP = Reverse''

\end{mnemonicbox}
\subsection*{Question 4(b) [4 marks]}\label{q4b}

\textbf{Describe DSL technology with its advantages and limitations.}

\begin{solutionbox}

\textbf{DSL (Digital Subscriber Line):}

{\def\LTcaptype{none} % do not increment counter
\begin{longtable}[]{@{}lll@{}}
\toprule\noalign{}
Type & Speed & Distance \\
\midrule\noalign{}
\endhead
\bottomrule\noalign{}
\endlastfoot
\textbf{ADSL} & Up to 8 Mbps & 5.5 km \\
\textbf{VDSL} & Up to 52 Mbps & 1.5 km \\
\textbf{SDSL} & Up to 2 Mbps & 3 km \\
\end{longtable}
}

\textbf{Advantages:}

\begin{itemize}
\tightlist
\item
  \textbf{Existing Infrastructure}: Uses telephone lines
\item
  \textbf{Always-On}: Continuous internet connection
\item
  \textbf{Voice + Data}: Simultaneous phone and internet
\item
  \textbf{Cost-Effective}: Affordable for home users
\end{itemize}

\textbf{Limitations:}

\begin{itemize}
\tightlist
\item
  \textbf{Distance Dependent}: Speed decreases with distance
\item
  \textbf{Upload Speed}: Lower than download speed (ADSL)
\item
  \textbf{Line Quality}: Affected by copper wire condition
\item
  \textbf{Availability}: Not available in all areas
\end{itemize}

\end{solutionbox}
\begin{mnemonicbox}
``DSL = Digital Subscriber Line''

\end{mnemonicbox}
\subsection*{Question 4(c) [7 marks]}\label{q4c}

\textbf{Role of DNS- Domain Name System.}

\begin{solutionbox}

\textbf{DNS Functions:}

\begin{itemize}
\tightlist
\item
  \textbf{Name Resolution}: Converts domain names to IP addresses
\item
  \textbf{Hierarchical Structure}: Organized in tree-like structure
\item
  \textbf{Distributed Database}: Information stored across multiple
  servers
\end{itemize}

\begin{center}
\textbf{Mermaid Diagram (Code)}
\begin{verbatim}
{Shaded}
{Highlighting}[]
graph LR
    A[Root Servers] {-{-}{} B[Top Level Domain .com]}
    A {-{-}{} C[Top Level Domain .org]}
    B {-{-}{} D[google.com]}
    B {-{-}{} E[yahoo.com]}
    D {-{-}{} F[drive.google.com]}
    D {-{-}{} G[mail.google.com]}
{Highlighting}
{Shaded}
\end{verbatim}
\end{center}

\textbf{DNS Hierarchy:}

\begin{itemize}
\tightlist
\item
  \textbf{Root Domain}: Highest level (.)
\item
  \textbf{Top-Level Domain}: .com, .org, .net, .edu
\item
  \textbf{Second-Level Domain}: google.com, yahoo.com
\item
  \textbf{Subdomain}: www.google.com, mail.google.com
\end{itemize}

\textbf{DNS Resolution Process:}

\begin{enumerate}
\tightlist
\item
  \textbf{Client Query}: User types www.example.com
\item
  \textbf{Local DNS}: Checks local cache
\item
  \textbf{Root Server}: Queries root DNS server
\item
  \textbf{TLD Server}: Queries .com server
\item
  \textbf{Authoritative Server}: Gets IP address
\item
  \textbf{Response}: Returns IP to client
\end{enumerate}

\textbf{DNS Record Types:}

\begin{itemize}
\tightlist
\item
  \textbf{A Record}: Maps domain to IPv4 address
\item
  \textbf{AAAA Record}: Maps domain to IPv6 address
\item
  \textbf{CNAME}: Canonical name (alias)
\item
  \textbf{MX}: Mail exchange server
\item
  \textbf{NS}: Name server records
\end{itemize}

\end{solutionbox}
\begin{mnemonicbox}
``DNS = Domain Name System''

\end{mnemonicbox}
\subsection*{Question 4(a OR) [3
marks]}\label{question-4a-or-3-marks}

\textbf{Give full name of DHCP and BOOTP. and describe them.}

\begin{solutionbox}

\textbf{Full Names:}

\begin{itemize}
\tightlist
\item
  \textbf{DHCP}: Dynamic Host Configuration Protocol
\item
  \textbf{BOOTP}: Bootstrap Protocol
\end{itemize}

{\def\LTcaptype{none} % do not increment counter
\begin{longtable}[]{@{}ll@{}}
\toprule\noalign{}
Protocol & Function \\
\midrule\noalign{}
\endhead
\bottomrule\noalign{}
\endlastfoot
\textbf{DHCP} & Automatically assigns IP addresses \\
\textbf{BOOTP} & Provides IP address to diskless workstations \\
\end{longtable}
}

\textbf{DHCP Process:}

\begin{itemize}
\tightlist
\item
  \textbf{Discover}: Client broadcasts request
\item
  \textbf{Offer}: Server offers IP address
\item
  \textbf{Request}: Client requests specific IP
\item
  \textbf{Acknowledge}: Server confirms assignment
\end{itemize}

\textbf{BOOTP Process:}

\begin{itemize}
\tightlist
\item
  \textbf{Static Configuration}: Pre-configured IP assignments
\item
  \textbf{Diskless Boot}: Workstations boot from network
\item
  \textbf{Server Response}: Provides IP and boot information
\end{itemize}

\end{solutionbox}
\begin{mnemonicbox}
``DHCP = Dynamic, BOOTP = Bootstrap''

\end{mnemonicbox}
\subsection*{Question 4(b OR) [4
marks]}\label{question-4b-or-4-marks}

\textbf{Differences Between Virtual Circuits and Datagram Networks.}

\begin{solutionbox}

{\def\LTcaptype{none} % do not increment counter
\begin{longtable}[]{@{}lll@{}}
\toprule\noalign{}
Feature & Virtual Circuits & Datagram Networks \\
\midrule\noalign{}
\endhead
\bottomrule\noalign{}
\endlastfoot
\textbf{Connection} & Connection-oriented & Connectionless \\
\textbf{Setup} & Requires setup phase & No setup required \\
\textbf{Routing} & Same path for all packets & Independent routing \\
\textbf{Order} & Packets arrive in order & May arrive out of order \\
\textbf{Reliability} & More reliable & Less reliable \\
\textbf{Overhead} & Higher setup overhead & Lower per-packet overhead \\
\end{longtable}
}

\textbf{Virtual Circuits:}

\begin{itemize}
\tightlist
\item
  \textbf{Path Establishment}: Creates virtual connection
\item
  \textbf{State Information}: Maintains connection state
\item
  \textbf{Examples}: ATM, Frame Relay
\end{itemize}

\textbf{Datagram Networks:}

\begin{itemize}
\tightlist
\item
  \textbf{Independent Packets}: Each packet routed separately
\item
  \textbf{Stateless}: No connection state maintained
\item
  \textbf{Examples}: Internet Protocol (IP)
\end{itemize}

\end{solutionbox}
\begin{mnemonicbox}
``Virtual = Connection, Datagram = Independent''

\end{mnemonicbox}
\subsection*{Question 4(c OR) [7
marks]}\label{question-4c-or-7-marks}

\textbf{Explain TCP and UDP protocol in transport layer}

\begin{solutionbox}

{\def\LTcaptype{none} % do not increment counter
\begin{longtable}[]{@{}lll@{}}
\toprule\noalign{}
Feature & TCP & UDP \\
\midrule\noalign{}
\endhead
\bottomrule\noalign{}
\endlastfoot
\textbf{Connection} & Connection-oriented & Connectionless \\
\textbf{Reliability} & Reliable & Unreliable \\
\textbf{Header Size} & 20 bytes & 8 bytes \\
\textbf{Flow Control} & Yes & No \\
\textbf{Error Control} & Yes & Basic \\
\textbf{Speed} & Slower & Faster \\
\end{longtable}
}

\textbf{TCP (Transmission Control Protocol):}

\begin{itemize}
\tightlist
\item
  \textbf{Three-Way Handshake}: SYN, SYN-ACK, ACK
\item
  \textbf{Flow Control}: Sliding window mechanism
\item
  \textbf{Error Recovery}: Retransmission of lost packets
\item
  \textbf{Congestion Control}: Prevents network overload
\end{itemize}

\textbf{TCP Header:}

\begin{verbatim}
0               16              32
+{-{-}{-}{-}{-}{-}{-}{-}{-}{-}{-}{-}{-}{-}{-}+{-}{-}{-}{-}{-}{-}{-}{-}{-}{-}{-}{-}{-}{-}{-}{-}+}
|Source Port    |Destination Port|
+{-{-}{-}{-}{-}{-}{-}{-}{-}{-}{-}{-}{-}{-}{-}+{-}{-}{-}{-}{-}{-}{-}{-}{-}{-}{-}{-}{-}{-}{-}{-}+}
|      Sequence Number           |
+{-{-}{-}{-}{-}{-}{-}{-}{-}{-}{-}{-}{-}{-}{-}+{-}{-}{-}{-}{-}{-}{-}{-}{-}{-}{-}{-}{-}{-}{-}{-}+}
|   Acknowledgment Number        |
+{-{-}{-}{-}{-}{-}{-}{-}{-}{-}{-}{-}{-}{-}{-}+{-}{-}{-}{-}{-}{-}{-}{-}{-}{-}{-}{-}{-}{-}{-}{-}+}
|Hdr|   |U|A|P|R|S|F|    Window  |
|Len|   |R|C|S|S|Y|I|     Size   |
+{-{-}{-}{-}{-}{-}{-}{-}{-}{-}{-}{-}{-}{-}{-}+{-}{-}{-}{-}{-}{-}{-}{-}{-}{-}{-}{-}{-}{-}{-}{-}+}
\end{verbatim}

\textbf{UDP (User Datagram Protocol):}

\begin{itemize}
\tightlist
\item
  \textbf{Simple Protocol}: Minimal overhead
\item
  \textbf{Best Effort}: No guarantee of delivery
\item
  \textbf{Applications}: DNS, DHCP, streaming media
\item
  \textbf{Real-time Communication}: Voice, video applications
\end{itemize}

\textbf{UDP Header:}

\begin{verbatim}
0               16              32
+{-{-}{-}{-}{-}{-}{-}{-}{-}{-}{-}{-}{-}{-}{-}+{-}{-}{-}{-}{-}{-}{-}{-}{-}{-}{-}{-}{-}{-}{-}{-}+}
|Source Port    |Destination Port|
+{-{-}{-}{-}{-}{-}{-}{-}{-}{-}{-}{-}{-}{-}{-}+{-}{-}{-}{-}{-}{-}{-}{-}{-}{-}{-}{-}{-}{-}{-}{-}+}
|    Length     |   Checksum     |
+{-{-}{-}{-}{-}{-}{-}{-}{-}{-}{-}{-}{-}{-}{-}+{-}{-}{-}{-}{-}{-}{-}{-}{-}{-}{-}{-}{-}{-}{-}{-}+}
\end{verbatim}

\textbf{Applications:}

\begin{itemize}
\tightlist
\item
  \textbf{TCP}: Web browsing, email, file transfer
\item
  \textbf{UDP}: Online gaming, video streaming, DNS queries
\end{itemize}

\end{solutionbox}
\begin{mnemonicbox}
``TCP = Reliable, UDP = Fast''

\end{mnemonicbox}
\subsection*{Question 5(a) [3 marks]}\label{q5a}

\textbf{Explain any two of following. (1) WWW (2) FTP (3) SMTP}

\begin{solutionbox}

\textbf{WWW (World Wide Web):}

\begin{itemize}
\tightlist
\item
  \textbf{HTTP Protocol}: HyperText Transfer Protocol
\item
  \textbf{Web Browser}: Client software (Chrome, Firefox)
\item
  \textbf{Web Server}: Serves web pages (Apache, IIS)
\end{itemize}

\textbf{FTP (File Transfer Protocol):}

\begin{itemize}
\tightlist
\item
  \textbf{File Transfer}: Upload and download files
\item
  \textbf{Two Modes}: Active and passive mode
\item
  \textbf{Authentication}: Username and password required
\end{itemize}

{\def\LTcaptype{none} % do not increment counter
\begin{longtable}[]{@{}lll@{}}
\toprule\noalign{}
Service & Port & Function \\
\midrule\noalign{}
\endhead
\bottomrule\noalign{}
\endlastfoot
\textbf{WWW} & 80/443 & Web page delivery \\
\textbf{FTP} & 20/21 & File transfer \\
\end{longtable}
}

\end{solutionbox}
\begin{mnemonicbox}
``WWW = Web, FTP = Files''

\end{mnemonicbox}
\subsection*{Question 5(b) [4 marks]}\label{q5b}

\textbf{Difference between symmetric and asymmetric encryption
algorithms.}

\begin{solutionbox}

{\def\LTcaptype{none} % do not increment counter
\begin{longtable}[]{@{}
  >{\raggedright\arraybackslash}p{(\linewidth - 4\tabcolsep) * \real{0.2812}}
  >{\raggedright\arraybackslash}p{(\linewidth - 4\tabcolsep) * \real{0.3438}}
  >{\raggedright\arraybackslash}p{(\linewidth - 4\tabcolsep) * \real{0.3750}}@{}}
\toprule\noalign{}
\begin{minipage}[b]{\linewidth}\raggedright
Feature
\end{minipage} & \begin{minipage}[b]{\linewidth}\raggedright
Symmetric
\end{minipage} & \begin{minipage}[b]{\linewidth}\raggedright
Asymmetric
\end{minipage} \\
\midrule\noalign{}
\endhead
\bottomrule\noalign{}
\endlastfoot
\textbf{Keys} & Same key for encryption/decryption & Different keys
(public/private) \\
\textbf{Speed} & Fast & Slow \\
\textbf{Key Distribution} & Difficult & Easy \\
\textbf{Examples} & AES, DES & RSA, ECC \\
\end{longtable}
}

\textbf{Symmetric Encryption:}

\begin{itemize}
\tightlist
\item
  \textbf{Single Key}: Same key used by sender and receiver
\item
  \textbf{Key Management}: Secure key distribution required
\item
  \textbf{Performance}: Fast encryption/decryption
\item
  \textbf{Applications}: Bulk data encryption
\end{itemize}

\textbf{Asymmetric Encryption:}

\begin{itemize}
\tightlist
\item
  \textbf{Key Pair}: Public key for encryption, private key for
  decryption
\item
  \textbf{Key Distribution}: Public key can be shared openly
\item
  \textbf{Performance}: Slower than symmetric
\item
  \textbf{Applications}: Digital signatures, key exchange
\end{itemize}

\end{solutionbox}
\begin{mnemonicbox}
``Symmetric = Same, Asymmetric = Different''

\end{mnemonicbox}
\subsection*{Question 5(c) [7 marks]}\label{q5c}

\textbf{Define the terms ``encryption'' and ``decryption'' in the
context of cryptography.}

\begin{solutionbox}

\textbf{Encryption:}

\begin{itemize}
\tightlist
\item
  \textbf{Definition}: Process of converting plaintext into ciphertext
\item
  \textbf{Purpose}: Protect data confidentiality
\item
  \textbf{Input}: Plaintext + Key
\item
  \textbf{Output}: Ciphertext
\end{itemize}

\textbf{Decryption:}

\begin{itemize}
\tightlist
\item
  \textbf{Definition}: Process of converting ciphertext back to
  plaintext
\item
  \textbf{Purpose}: Retrieve original data
\item
  \textbf{Input}: Ciphertext + Key
\item
  \textbf{Output}: Plaintext
\end{itemize}

\begin{center}
\textbf{Mermaid Diagram (Code)}
\begin{verbatim}
{Shaded}
{Highlighting}[]
graph LR
    A[Plaintext] {-{-}{} B[Encryption]}
    B {-{-}{} C[Ciphertext]}
    C {-{-}{} D[Decryption]}
    D {-{-}{} E[Plaintext]}
    F[Key] {-{-}{} B}
    G[Key] {-{-}{} D}
{Highlighting}
{Shaded}
\end{verbatim}
\end{center}

\textbf{Cryptographic Process:}

\begin{enumerate}
\tightlist
\item
  \textbf{Sender}: Encrypts message using key
\item
  \textbf{Transmission}: Sends ciphertext over network
\item
  \textbf{Receiver}: Decrypts ciphertext using key
\item
  \textbf{Recovery}: Gets original plaintext message
\end{enumerate}

\textbf{Types of Encryption:}

\begin{itemize}
\tightlist
\item
  \textbf{Stream Cipher}: Encrypts one bit/byte at a time
\item
  \textbf{Block Cipher}: Encrypts fixed-size blocks
\item
  \textbf{Hash Function}: One-way encryption (no decryption)
\end{itemize}

\textbf{Applications:}

\begin{itemize}
\tightlist
\item
  \textbf{Data Protection}: Secure file storage
\item
  \textbf{Communication}: Secure messaging
\item
  \textbf{Authentication}: Digital signatures
\item
  \textbf{Privacy}: Personal information protection
\end{itemize}

\textbf{Security Requirements:}

\begin{itemize}
\tightlist
\item
  \textbf{Confidentiality}: Only authorized users can read
\item
  \textbf{Integrity}: Data hasn't been tampered with
\item
  \textbf{Authentication}: Verify sender identity
\item
  \textbf{Non-repudiation}: Sender cannot deny sending
\end{itemize}

\end{solutionbox}
\begin{mnemonicbox}
``Encryption = Hide, Decryption = Reveal''

\end{mnemonicbox}
\subsection*{Question 5(a OR) [3
marks]}\label{question-5a-or-3-marks}

\textbf{Difference between IMAP and POP3}

\begin{solutionbox}

{\def\LTcaptype{none} % do not increment counter
\begin{longtable}[]{@{}lll@{}}
\toprule\noalign{}
Feature & IMAP & POP3 \\
\midrule\noalign{}
\endhead
\bottomrule\noalign{}
\endlastfoot
\textbf{Storage} & Server-side & Client-side \\
\textbf{Access} & Multiple devices & Single device \\
\textbf{Offline} & Limited & Full access \\
\end{longtable}
}

\textbf{IMAP (Internet Message Access Protocol):}

\begin{itemize}
\tightlist
\item
  \textbf{Server Storage}: Messages remain on server
\item
  \textbf{Multi-Device}: Access from multiple devices
\item
  \textbf{Synchronization}: Changes sync across devices
\end{itemize}

\textbf{POP3 (Post Office Protocol 3):}

\begin{itemize}
\tightlist
\item
  \textbf{Download}: Messages downloaded to client
\item
  \textbf{Single Device}: Best for one device access
\item
  \textbf{Storage}: Client manages message storage
\end{itemize}

\end{solutionbox}
\begin{mnemonicbox}
``IMAP = Internet Access, POP3 = Post Office''

\end{mnemonicbox}
\subsection*{Question 5(b OR) [4
marks]}\label{question-5b-or-4-marks}

\textbf{Briefly describe the Information Technology (Amendment) Act,
2008, and its impact on cyber laws in India.}

\begin{solutionbox}

\textbf{IT Act 2008 Key Features:}

\begin{itemize}
\tightlist
\item
  \textbf{Cyber Crimes}: Defines various cyber offenses
\item
  \textbf{Data Protection}: Privacy and security requirements
\item
  \textbf{Digital Signatures}: Legal recognition of e-signatures
\item
  \textbf{Penalties}: Fines and imprisonment for violations
\end{itemize}

\textbf{Major Amendments:}

\begin{itemize}
\tightlist
\item
  \textbf{Section 66A}: Criminalized offensive messages (later struck
  down)
\item
  \textbf{Section 69}: Government power to intercept information
\item
  \textbf{Section 72A}: Punishment for disclosure of personal
  information
\item
  \textbf{Section 43A}: Compensation for data breach
\end{itemize}

\textbf{Impact on Cyber Laws:}

\begin{itemize}
\tightlist
\item
  \textbf{Legal Framework}: Comprehensive cyber law structure
\item
  \textbf{Business Compliance}: Data protection requirements
\item
  \textbf{Individual Rights}: Privacy protection mechanisms
\item
  \textbf{Law Enforcement}: Tools for investigating cyber crimes
\end{itemize}

\end{solutionbox}
\begin{mnemonicbox}
``IT Act = Internet Technology Act''

\end{mnemonicbox}
\subsection*{Question 5(c OR) [7
marks]}\label{question-5c-or-7-marks}

\textbf{Difference between symmetric and asymmetric encryption
algorithms.}

\begin{solutionbox}

{\def\LTcaptype{none} % do not increment counter
\begin{longtable}[]{@{}
  >{\raggedright\arraybackslash}p{(\linewidth - 4\tabcolsep) * \real{0.1569}}
  >{\raggedright\arraybackslash}p{(\linewidth - 4\tabcolsep) * \real{0.4118}}
  >{\raggedright\arraybackslash}p{(\linewidth - 4\tabcolsep) * \real{0.4314}}@{}}
\toprule\noalign{}
\begin{minipage}[b]{\linewidth}\raggedright
Aspect
\end{minipage} & \begin{minipage}[b]{\linewidth}\raggedright
Symmetric Encryption
\end{minipage} & \begin{minipage}[b]{\linewidth}\raggedright
Asymmetric Encryption
\end{minipage} \\
\midrule\noalign{}
\endhead
\bottomrule\noalign{}
\endlastfoot
\textbf{Key Usage} & Same key for encrypt/decrypt & Different keys
(public/private) \\
\textbf{Key Management} & Difficult key distribution & Easy key
distribution \\
\textbf{Performance} & Fast processing & Slow processing \\
\textbf{Key Length} & Shorter keys (128-256 bits) & Longer keys
(1024-4096 bits) \\
\textbf{Scalability} & Poor (n^{2} key pairs needed) & Good (n key pairs
needed) \\
\textbf{Examples} & AES, DES, 3DES, Blowfish & RSA, ECC, DSA, ElGamal \\
\end{longtable}
}

\textbf{Symmetric Encryption Details:}

\begin{itemize}
\tightlist
\item
  \textbf{Algorithm Types}: Stream ciphers, Block ciphers
\item
  \textbf{Key Distribution Problem}: Secure channel needed for key
  exchange
\item
  \textbf{Applications}: Bulk data encryption, VPNs, file encryption
\item
  \textbf{Advantages}: Fast, efficient for large amounts of data
\item
  \textbf{Disadvantages}: Key management complexity, no digital
  signatures
\end{itemize}

\textbf{Asymmetric Encryption Details:}

\begin{itemize}
\tightlist
\item
  \textbf{Public Key Infrastructure}: PKI for key management
\item
  \textbf{Digital Signatures}: Authentication and non-repudiation
\item
  \textbf{Applications}: Email security, SSL/TLS, digital certificates
\item
  \textbf{Advantages}: Secure key exchange, digital signatures
\item
  \textbf{Disadvantages}: Computationally intensive, slower processing
\end{itemize}

\textbf{Hybrid Approach:}

\begin{itemize}
\tightlist
\item
  \textbf{Best of Both}: Combines symmetric and asymmetric encryption
\item
  \textbf{Key Exchange}: Asymmetric for key distribution
\item
  \textbf{Data Encryption}: Symmetric for actual data
\item
  \textbf{Example}: SSL/TLS uses both methods
\end{itemize}

\begin{center}
\textbf{Mermaid Diagram (Code)}
\begin{verbatim}
{Shaded}
{Highlighting}[]
graph TD
    A[Encryption Methods] {-{-}{} B[Symmetric]}
    A {-{-}{} C[Asymmetric]}
    B {-{-}{} D[Same Key]}
    B {-{-}{} E[Fast Processing]}
    C {-{-}{} F[Key Pair]}
    C {-{-}{} G[Slow Processing]}
{Highlighting}
{Shaded}
\end{verbatim}
\end{center}

\textbf{Real-world Applications:}

\begin{itemize}
\tightlist
\item
  \textbf{Banking}: ATM transactions use symmetric encryption
\item
  \textbf{E-commerce}: HTTPS uses hybrid encryption\\
\item
  \textbf{Email}: PGP uses asymmetric for key exchange
\item
  \textbf{Mobile}: WhatsApp uses end-to-end encryption
\end{itemize}

\textbf{Security Considerations:}

\begin{itemize}
\tightlist
\item
  \textbf{Key Length}: Longer keys provide better security
\item
  \textbf{Algorithm Strength}: Choose proven algorithms
\item
  \textbf{Implementation}: Proper coding prevents vulnerabilities
\item
  \textbf{Key Storage}: Secure key management essential
\end{itemize}

\textbf{Performance Comparison:}

{\def\LTcaptype{none} % do not increment counter
\begin{longtable}[]{@{}lll@{}}
\toprule\noalign{}
Operation & Symmetric (AES) & Asymmetric (RSA) \\
\midrule\noalign{}
\endhead
\bottomrule\noalign{}
\endlastfoot
\textbf{Encryption} & \textasciitilde1000 MB/s & \textasciitilde1
MB/s \\
\textbf{Key Generation} & Fast & Slow \\
\textbf{Memory Usage} & Low & High \\
\textbf{CPU Usage} & Low & High \\
\end{longtable}
}

\textbf{Future Trends:}

\begin{itemize}
\tightlist
\item
  \textbf{Quantum Computing}: Threat to current asymmetric algorithms
\item
  \textbf{Post-Quantum Cryptography}: New algorithms being developed
\item
  \textbf{Elliptic Curve}: More efficient asymmetric encryption
\item
  \textbf{Lightweight Cryptography}: For IoT devices
\end{itemize}

\end{solutionbox}
\begin{mnemonicbox}
``Symmetric = Same Speed, Asymmetric = Advanced
Security''

\end{mnemonicbox}

\end{document}
