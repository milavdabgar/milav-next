\documentclass[10pt,a4paper]{article}

% content/resources/templates/preamble.tex
\usepackage[margin=0.6in]{geometry}
\author{Milav Dabgar}
\usepackage{amsmath,amssymb,amsthm}
\usepackage{booktabs}
\usepackage{multirow}
\usepackage{xcolor}
\usepackage{tcolorbox}
\tcbuselibrary{breakable,skins}
\usepackage[colorlinks=true,linkcolor=blue]{hyperref}
\usepackage{titlesec}
\usepackage{enumitem}
\usepackage{tikz}
\usepackage{pgfplots}
\usepackage{circuitikz}
\usepackage[version=4]{mhchem}
\usepackage{longtable}
\usepackage{array}
\usepackage{float}
\usepackage{caption}
\usepackage{listings}

\lstset{
  basicstyle=\small\ttfamily,
  breaklines=true,
  breakatwhitespace=false,
  postbreak=\mbox{\textcolor{red}{$\hookrightarrow$}\space},
  float=false,
  numbers=left,
  numberstyle=\tiny\color{gray},
  numbersep=10pt,
  xleftmargin=2em,
  keywordstyle=\color{blue},
  commentstyle=\color{green!60!black},
  stringstyle=\color{purple},
  backgroundcolor=\color{gray!5},
  showstringspaces=false,
  tabsize=2,
  captionpos=b,
  keepspaces=true,
  columns=flexible
}

\pgfplotsset{compat=1.18}
\usetikzlibrary{shapes,arrows,positioning,calc,patterns,decorations.pathmorphing,decorations.markings,arrows.meta}

% Color scheme
\definecolor{headcolor}{RGB}{0,102,204}
\definecolor{keycolor}{RGB}{220,20,60}
\definecolor{solutioncolor}{RGB}{34,139,34}
\definecolor{mnemoniccolor}{RGB}{148,0,211}
\definecolor{codecolor}{RGB}{0,0,100}

% Spacing
\setlength{\parskip}{3pt}
\setlist[itemize]{nosep}
\setlist[enumerate]{nosep}

% Title formatting
\titleformat{\section}{\Large\bfseries\color{headcolor}}{\thesection}{1em}{}
\titleformat{\subsection}{\large\bfseries\color{headcolor}}{\thesubsection}{1em}{}

% Pandoc tightlist compatibility
\providecommand{\tightlist}{%
  \setlength{\itemsep}{0pt}\setlength{\parskip}{0pt}}

% Pandoc longtable compatibility
\newcounter{none}
\def\thenone{}


% content/resources/templates/gujarati-boxes.tex
\usepackage{fontspec}
\usepackage{polyglossia}

% Set Gujarati as main language (document is primarily in Gujarati)
% Note: gloss-gujarati.ldf doesn't exist in polyglossia, but it will use hyphenation patterns
\setdefaultlanguage{gujarati}
\setotherlanguage{english}

% Configure Gujarati font properly
% Use Language=Default to prevent polyglossia from trying to add language-specific features
% that don't exist for Gujarati, which causes "empty feature" warnings
\newfontfamily\gujaratifont[Script=Gujarati,AutoFakeBold=2.5,AutoFakeSlant=0.3]{Noto Sans Gujarati}
\setmainfont[Script=Gujarati,AutoFakeBold=2.5,AutoFakeSlant=0.3]{Noto Sans Gujarati}
% Use Noto Sans Gujarati for monospace to support Gujarati in text
\setmonofont[Scale=0.9]{Noto Sans Gujarati}

% Configure English to use the same font
\newfontfamily\englishfont[Script=Gujarati,AutoFakeBold=2.5,AutoFakeSlant=0.3]{Noto Sans Gujarati}

% Translations for polyglossia
\gappto\captionsgujarati{
  \renewcommand{\tablename}{કોષ્ટક}
  \renewcommand{\figurename}{આકૃતિ}
}

% Helper for TikZ nodes to ensure Gujarati font
\newcommand{\gu}[1]{{\gujaratifont #1}}

% Custom environments
\newtcolorbox{solutionbox}{
    breakable,
    enhanced,
    colback=solutioncolor!5!white,
    colframe=solutioncolor!75!black,
    fonttitle=\bfseries,
    title=જવાબ
}

\newtcolorbox{solutionboxnobreak}{
 colback=solutioncolor!5!white,
 colframe=solutioncolor!75!black,
 fonttitle=\bfseries,
 title=જવાબ
}

\newtcolorbox{keyformula}{
 breakable,
 enhanced,
 colback=keycolor!5!white,
 colframe=keycolor!75!black,
 fonttitle=\bfseries,
 title=રાસાયણિક સમીકરણ/સૂત્ર
}

\newtcolorbox{mnemonicbox}{
 breakable,
 enhanced,
 colback=mnemoniccolor!5!white,
 colframe=mnemoniccolor!75!black,
 fonttitle=\bfseries,
 title=મેમરી ટ્રીક
}


\begin{document}

\begin{center}
{\Huge\bfseries\color{headcolor} Subject Name (Gujarati)}\\[5pt]
{\LARGE 4361101 -- Summer 2024}\\[3pt]
{\large Semester 1 Study Material}\\[3pt]
{\normalsize\textit{Detailed Solutions and Explanations}}
\end{center}

\vspace{10pt}

\subsection*{પ્રશ્ન 1(અ) [3
ગુણ]}\label{uxaaauxab0uxab6uxaa8-1uxa85-3-uxa97uxaa3}

\textbf{વિવિધ નેટવર્ક ટોપોલોજીની યાદી બનાવો અને કોઈપણ એકની વિગતવાર ચર્ચા
કરો.}

\begin{solutionbox}

{\def\LTcaptype{none} % do not increment counter
\begin{longtable}[]{@{}ll@{}}
\toprule\noalign{}
ટોપોલોજી & વર્ણન \\
\midrule\noalign{}
\endhead
\bottomrule\noalign{}
\endlastfoot
\textbf{સ્ટાર} & બધા ઉપકરણો કેન્દ્રીય હબ/સ્વિચ સાથે જોડાયેલા \\
\textbf{રિંગ} & ઉપકરણો ગોળાકાર ફેશનમાં જોડાયેલા \\
\textbf{બસ} & બધા ઉપકરણો એક જ કેબલ સાથે જોડાયેલા \\
\textbf{મેશ} & દરેક ઉપકરણ બીજા દરેક ઉપકરણ સાથે જોડાયેલું \\
\textbf{ટ્રી} & રૂટ નોડ સાથે વંશવેલો માળખું \\
\textbf{હાઇબ્રિડ} & બે અથવા વધુ ટોપોલોજીનું સંયોજન \\
\end{longtable}
}

\textbf{સ્ટાર ટોપોલોજી વિગતો:}

\begin{itemize}
\tightlist
\item
  \textbf{કેન્દ્રીય હબ}: બધા નોડ્સ એક કેન્દ્રીય ઉપકરણ સાથે જોડાય છે
\item
  \textbf{પોઇન્ટ-ટુ-પોઇન્ટ}: દરેક કનેક્શન નોડ અને હબ વચ્ચે સમર્પિત છે
\item
  \textbf{સરળ મેનેજમેન્ટ}: ઇન્સ્ટોલ અને ટ્રબલશૂટ કરવું સરળ
\end{itemize}

\end{solutionbox}
\begin{mnemonicbox}
``STAR = Single Terminal All Reach''

\end{mnemonicbox}
\subsection*{પ્રશ્ન 1(બ) [4
ગુણ]}\label{uxaaauxab0uxab6uxaa8-1uxaac-4-uxa97uxaa3}

\textbf{આધુનિક સંચાર પ્રણાલીઓમાં પોઇન્ટ-ટુ-પોઇન્ટ અને બ્રોડકાસ્ટ ટ્રાન્સમિશન
ટેકનોલોજીનો ઉપયોગ કેવી રીતે થાય છે તે ઉદાહરણો સાથે સમજાવો. અને તેમના ફાયદા અને
મર્યાદાઓની ચર્ચા કરો.}

\begin{solutionbox}

{\def\LTcaptype{none} % do not increment counter
\begin{longtable}[]{@{}lll@{}}
\toprule\noalign{}
ટેકનોલોજી & પોઇન્ટ-ટુ-પોઇન્ટ & બ્રોડકાસ્ટ \\
\midrule\noalign{}
\endhead
\bottomrule\noalign{}
\endlastfoot
\textbf{કનેક્શન} & બે ઉપકરણો વચ્ચે સીધી લિંક & એક-થી-અનેક સંદેશાવ્યવહાર \\
\textbf{ઉદાહરણ} & ટેલિફોન, VPN ટનલ્સ & રેડિયો, TV, WiFi \\
\textbf{ડેટા ફ્લો} & દ્વિદિશાત્મક & એકદિશાત્મક/બહુદિશાત્મક \\
\end{longtable}
}

\textbf{પોઇન્ટ-ટુ-પોઇન્ટ એપ્લિકેશન્સ:}

\begin{itemize}
\tightlist
\item
  \textbf{સમર્પિત લાઇન્સ}: ઓફિસો વચ્ચે લીઝ્ડ લાઇન્સ
\item
  \textbf{સેટેલાઇટ લિંક્સ}: ગ્રાઉન્ડ સ્ટેશનથી સેટેલાઇટ સંદેશાવ્યવહાર
\item
  \textbf{કેબલ મોડેમ્સ}: ઘરથી ISP કનેક્શન
\end{itemize}

\textbf{બ્રોડકાસ્ટ એપ્લિકેશન્સ:}

\begin{itemize}
\tightlist
\item
  \textbf{WiFi નેટવર્કસ}: રાઉટર બહુવિધ ઉપકરણોને બ્રોડકાસ્ટ કરે છે
\item
  \textbf{ટેલિવિઝન}: એક ટ્રાન્સમિટરથી અનેક રિસીવર્સ
\end{itemize}

\end{solutionbox}
\begin{mnemonicbox}
``P2P = Private Path, Broadcast = Big Audience''

\end{mnemonicbox}
\subsection*{પ્રશ્ન 1(ક) [7
ગુણ]}\label{uxaaauxab0uxab6uxaa8-1uxa95-7-uxa97uxaa3}

\textbf{દરેક લેયરના કાર્ય સાથે OSI મોડેલનું વર્ણન કરો.}

\begin{solutionbox}

{\def\LTcaptype{none} % do not increment counter
\begin{longtable}[]{@{}lll@{}}
\toprule\noalign{}
લેયર & નામ & કાર્ય \\
\midrule\noalign{}
\endhead
\bottomrule\noalign{}
\endlastfoot
\textbf{7} & એપ્લિકેશન & યુઝર ઇન્ટરફેસ, નેટવર્ક સેવાઓ \\
\textbf{6} & પ્રેઝન્ટેશન & ડેટા એન્ક્રિપ્શન, કોમ્પ્રેશન, ફોર્મેટિંગ \\
\textbf{5} & સેશન & સેશન સ્થાપિત કરે, મેનેજ કરે, સમાપ્ત કરે \\
\textbf{4} & ટ્રાન્સપોર્ટ & વિશ્વસનીય ડેટા ટ્રાન્સફર, એરર કરેક્શન \\
\textbf{3} & નેટવર્ક & રાઉટિંગ, લોજિકલ એડ્રેસિંગ (IP) \\
\textbf{2} & ડેટા લિંક & ફ્રેમ ફોર્મેટિંગ, એરર ડિટેક્શન \\
\textbf{1} & ફિઝિકલ & બિટ ટ્રાન્સમિશન, ઇલેક્ટ્રિકલ સિગ્નલ્સ \\
\end{longtable}
}

\begin{center}
\textbf{Mermaid Diagram (Code)}
\begin{verbatim}
{Shaded}
{Highlighting}[]
graph LR
    A[Application Layer 7] {-{-}{} B[Presentation Layer 6]}
    B {-{-}{} C[Session Layer 5]}
    C {-{-}{} D[Transport Layer 4]}
    D {-{-}{} E[Network Layer 3]}
    E {-{-}{} F[Data Link Layer 2]}
    F {-{-}{} G[Physical Layer 1]}
{Highlighting}
{Shaded}
\end{verbatim}
\end{center}

\textbf{મુખ્ય કાર્યો:}

\begin{itemize}
\tightlist
\item
  \textbf{ઉપરના લેયર્સ (5-7)}: એપ્લિકેશન-સંબંધિત સેવાઓ સંભાળે છે
\item
  \textbf{નીચેના લેયર્સ (1-4)}: ડેટા ટ્રાન્સમિશન અને રાઉટિંગ સંભાળે છે
\item
  \textbf{એન્કેપ્સુલેશન}: દરેક લેયર પોતાનું હેડર ઉમેરે છે
\end{itemize}

\end{solutionbox}
\begin{mnemonicbox}
``All People Seem To Need Data Processing''

\end{mnemonicbox}
\subsection*{પ્રશ્ન 1(ક અથવા) [7
ગુણ]}\label{uxaaauxab0uxab6uxaa8-1uxa95-uxa85uxaa5uxab5-7-uxa97uxaa3}

\textbf{TCP/IP મોડેલના દરેક લેયરના કાર્ય સાથે વર્ણન લખો.}

\begin{solutionbox}

{\def\LTcaptype{none} % do not increment counter
\begin{longtable}[]{@{}llll@{}}
\toprule\noalign{}
લેયર & નામ & કાર્ય & પ્રોટોકોલ્સ \\
\midrule\noalign{}
\endhead
\bottomrule\noalign{}
\endlastfoot
\textbf{4} & એપ્લિકેશન & યુઝર સેવાઓ, એપ્લિકેશન્સ & HTTP, FTP, SMTP, DNS \\
\textbf{3} & ટ્રાન્સપોર્ટ & એન્ડ-ટુ-એન્ડ કમ્યુનિકેશન & TCP, UDP \\
\textbf{2} & ઇન્ટરનેટ & રાઉટિંગ, લોજિકલ એડ્રેસિંગ & IP, ICMP, ARP \\
\textbf{1} & નેટવર્ક એક્સેસ & ફિઝિકલ ટ્રાન્સમિશન & Ethernet, WiFi \\
\end{longtable}
}

\begin{center}
\textbf{Mermaid Diagram (Code)}
\begin{verbatim}
{Shaded}
{Highlighting}[]
graph LR
    A[Application Layer] {-{-}{} B[Transport Layer]}
    B {-{-}{} C[Internet Layer]}
    C {-{-}{} D[Network Access Layer]}
{Highlighting}
{Shaded}
\end{verbatim}
\end{center}

\textbf{લેયર કાર્યો:}

\begin{itemize}
\tightlist
\item
  \textbf{એપ્લિકેશન}: એપ્લિકેશન્સને નેટવર્ક સેવાઓ પ્રદાન કરે છે
\item
  \textbf{ટ્રાન્સપોર્ટ}: વિશ્વસનીય અથવા અવિશ્વસનીય ડિલિવરી સુનિશ્ચિત કરે છે
\item
  \textbf{ઇન્ટરનેટ}: IP એડ્રેસનો ઉપયોગ કરીને નેટવર્કમાં પેકેટ્સ રાઉટ કરે છે
\item
  \textbf{નેટવર્ક એક્સેસ}: ફિઝિકલ ટ્રાન્સમિશન મીડિયા સંભાળે છે
\end{itemize}

\end{solutionbox}
\begin{mnemonicbox}
``Applications Transport Internet Networks''

\end{mnemonicbox}
\subsection*{પ્રશ્ન 2(અ) [3
ગુણ]}\label{uxaaauxab0uxab6uxaa8-2uxa85-3-uxa97uxaa3}

\textbf{નેટવર્ક સુરક્ષામાં ફાયરવોલના કાર્યનું વર્ણન કરો.}

\begin{solutionbox}

\textbf{ફાયરવોલ કાર્યો:}

\begin{itemize}
\tightlist
\item
  \textbf{પેકેટ ફિલ્ટરિંગ}: આવતા અને જતા નેટવર્ક ટ્રાફિકને નિયંત્રિત કરે છે
\item
  \textbf{એક્સેસ કંટ્રોલ}: અનધિકૃત એક્સેસ પ્રયાસોને અવરોધે છે
\item
  \textbf{ટ્રાફિક મોનિટરિંગ}: નેટવર્ક એક્ટિવિટીને લોગ કરે અને વિશ્લેષિત કરે છે
\end{itemize}

\textbf{પ્રકારો:}

\begin{itemize}
\tightlist
\item
  \textbf{હાર્ડવેર ફાયરવોલ}: સંપૂર્ણ નેટવર્કનું રક્ષણ કરતું ભૌતિક ઉપકરણ
\item
  \textbf{સોફ્ટવેર ફાયરવોલ}: વ્યક્તિગત કમ્પ્યુટર્સ પર ઇન્સ્ટોલ કરાયેલો પ્રોગ્રામ
\item
  \textbf{સ્ટેટફુલ ઇન્સ્પેક્શન}: કનેક્શન સ્ટેટ્સ અને કોન્ટેક્સ્ટ ટ્રેક કરે છે
\end{itemize}

\end{solutionbox}
\begin{mnemonicbox}
``Firewall = Filter, Access, Monitor''

\end{mnemonicbox}
\subsection*{પ્રશ્ન 2(બ) [4
ગુણ]}\label{uxaaauxab0uxab6uxaa8-2uxaac-4-uxa97uxaa3}

\textbf{FDDI અને CDDI તેમની મુખ્ય લાક્ષણિકતાઓ, ફાયદા અને એપ્લિકેશનના સંદર્ભમાં
સરખામણી કરો.}

\begin{solutionbox}

{\def\LTcaptype{none} % do not increment counter
\begin{longtable}[]{@{}lll@{}}
\toprule\noalign{}
લક્ષણ & FDDI & CDDI \\
\midrule\noalign{}
\endhead
\bottomrule\noalign{}
\endlastfoot
\textbf{મીડિયમ} & ઓપ્ટિકલ ફાઇબર & ટ્વિસ્ટેડ પેર કોપર \\
\textbf{સ્પીડ} & 100 Mbps & 100 Mbps \\
\textbf{અંતર} & 200 કિમી સુધી & 100 મીટર સુધી \\
\textbf{કિંમત} & વધુ & ઓછી \\
\textbf{સુરક્ષા} & વધુ (ટેપ કરવું મુશ્કેલ) & ઓછી (ટેપ કરવું સરળ) \\
\textbf{ઇન્સ્ટોલેશન} & જટિલ & સરળ \\
\end{longtable}
}

\textbf{FDDI ફાયદાઓ:}

\begin{itemize}
\tightlist
\item
  \textbf{લાંબું અંતર}: કેમ્પસ-વ્યાપી નેટવર્કને સપોર્ટ કરે છે
\item
  \textbf{ઉચ્ચ સુરક્ષા}: ઇલેક્ટ્રોમેગ્નેટિક ઇન્ટરફેરન્સથી મુક્ત
\item
  \textbf{વિશ્વસનીયતા}: વધુ સારી એરર ડિટેક્શન અને રિકવરી
\end{itemize}

\textbf{CDDI ફાયદાઓ:}

\begin{itemize}
\tightlist
\item
  \textbf{કિફાયતી}: વર્તમાન કોપર ઇન્ફ્રાસ્ટ્રક્ચરનો ઉપયોગ
\item
  \textbf{સરળ ઇન્સ્ટોલેશન}: સ્ટાન્ડર્ડ ટ્વિસ્ટેડ પેર કેબ્લિંગ
\item
  \textbf{સુસંગતતા}: વર્તમાન નેટવર્ક સાધનો સાથે કામ કરે છે
\end{itemize}

\end{solutionbox}
\begin{mnemonicbox}
``FDDI = Fiber Distance, CDDI = Copper Cost''

\end{mnemonicbox}
\subsection*{પ્રશ્ન 2(ક) [7
ગુણ]}\label{uxaaauxab0uxab6uxaa8-2uxa95-7-uxa97uxaa3}

\textbf{ઇથરનેટ, ફાસ્ટ ઇથરનેટ, ગીગાબિટ ઇથરનેટ સમજાવો અને સરખામણી કરો.}

\begin{solutionbox}

{\def\LTcaptype{none} % do not increment counter
\begin{longtable}[]{@{}
  >{\raggedright\arraybackslash}p{(\linewidth - 8\tabcolsep) * \real{0.1429}}
  >{\raggedright\arraybackslash}p{(\linewidth - 8\tabcolsep) * \real{0.1667}}
  >{\raggedright\arraybackslash}p{(\linewidth - 8\tabcolsep) * \real{0.2143}}
  >{\raggedright\arraybackslash}p{(\linewidth - 8\tabcolsep) * \real{0.3095}}
  >{\raggedright\arraybackslash}p{(\linewidth - 8\tabcolsep) * \real{0.1667}}@{}}
\toprule\noalign{}
\begin{minipage}[b]{\linewidth}\raggedright
પ્રકાર
\end{minipage} & \begin{minipage}[b]{\linewidth}\raggedright
સ્પીડ
\end{minipage} & \begin{minipage}[b]{\linewidth}\raggedright
સ્ટાન્ડર્ડ
\end{minipage} & \begin{minipage}[b]{\linewidth}\raggedright
કેબલ પ્રકાર
\end{minipage} & \begin{minipage}[b]{\linewidth}\raggedright
અંતર
\end{minipage} \\
\midrule\noalign{}
\endhead
\bottomrule\noalign{}
\endlastfoot
\textbf{ઇથરનેટ} & 10 Mbps & 802.3 & Coax/UTP & 100m \\
\textbf{ફાસ્ટ ઇથરનેટ} & 100 Mbps & 802.3u & UTP Cat5 & 100m \\
\textbf{ગીગાબિટ ઇથરનેટ} & 1000 Mbps & 802.3z/ab & Cat5e/6, Fiber &
100m/5km \\
\end{longtable}
}

\begin{center}
\textbf{Mermaid Diagram (Code)}
\begin{verbatim}
{Shaded}
{Highlighting}[]
graph LR
    A[Ethernet 10 Mbps] {-{-}{} B[Fast Ethernet 100 Mbps]}
    B {-{-}{} C[Gigabit Ethernet 1000 Mbps]}
{Highlighting}
{Shaded}
\end{verbatim}
\end{center}

\textbf{મુખ્ય તફાવતો:}

\begin{itemize}
\tightlist
\item
  \textbf{સ્પીડ ઇવોલ્યુશન}: દરેક જનરેશનમાં 10x વધારો
\item
  \textbf{મીડિયા સપોર્ટ}: કોક્સથી ટ્વિસ્ટેડ પેરથી ફાઇબર સુધી
\item
  \textbf{એપ્લિકેશન્સ}: LAN બેકબોન, સર્વર કનેક્શન્સ, ડેસ્કટોપ
\item
  \textbf{બેકવર્ડ કોમ્પેટિબિલિટી}: નવા સ્ટાન્ડર્ડ જૂના ઉપકરણોને સપોર્ટ કરે છે
\end{itemize}

\textbf{સ્ટાન્ડર્ડ્સ:}

\begin{itemize}
\tightlist
\item
  \textbf{10Base-T}: ટ્વિસ્ટેડ પેર પર 10 Mbps
\item
  \textbf{100Base-TX}: કેટેગરી 5 UTP પર 100 Mbps
\item
  \textbf{1000Base-T}: કેટેગરી 5e/6 UTP પર 1 Gbps
\end{itemize}

\end{solutionbox}
\begin{mnemonicbox}
``Every Fast Gigabit = 10, 100, 1000''

\end{mnemonicbox}
\subsection*{પ્રશ્ન 2(અ અથવા) [3
ગુણ]}\label{uxaaauxab0uxab6uxaa8-2uxa85-uxa85uxaa5uxab5-3-uxa97uxaa3}

\textbf{નેટવર્ક ઇન્ફ્રાસ્ટ્રક્ચરમાં રાઉટરની તેની ભૂમિકા અને કાર્ય સમજાવો.}

\begin{solutionbox}

\textbf{રાઉટર કાર્યો:}

\begin{itemize}
\tightlist
\item
  \textbf{પેકેટ ફોરવર્ડિંગ}: વિવિધ નેટવર્કો વચ્ચે ડેટા પેકેટ્સ રાઉટ કરે છે
\item
  \textbf{પાથ ડિટર્મિનેશન}: રાઉટિંગ ટેબલનો ઉપયોગ કરીને શ્રેષ્ઠ રૂટ પસંદ કરે છે
\item
  \textbf{નેટવર્ક આઇસોલેશન}: બ્રોડકાસ્ટ ડોમેઇન્સને અલગ કરે છે
\end{itemize}

\textbf{મુખ્ય ભૂમિકાઓ:}

\begin{itemize}
\tightlist
\item
  \textbf{ઇન્ટર-નેટવર્ક કમ્યુનિકેશન}: LANs ને WANs સાથે જોડે છે
\item
  \textbf{ટ્રાફિક મેનેજમેન્ટ}: નેટવર્કો વચ્ચે ડેટા ફ્લોને નિયંત્રિત કરે છે
\item
  \textbf{પ્રોટોકોલ ટ્રાન્સલેશન}: વિવિધ નેટવર્ક પ્રોટોકોલ્સ વચ્ચે કન્વર્ટ કરે છે
\end{itemize}

\end{solutionbox}
\begin{mnemonicbox}
``Router = Route, Isolate, Connect''

\end{mnemonicbox}
\subsection*{પ્રશ્ન 2(બ અથવા) [4
ગુણ]}\label{uxaaauxab0uxab6uxaa8-2uxaac-uxa85uxaa5uxab5-4-uxa97uxaa3}

\textbf{FDDI (ફાઇબર ડિસ્ટ્રિબ્યુટેડ ડેટા ઇન્ટરફેસ) ની રચના સમજાવો અને તેના ફાયદાઓ
જણાવો.}

\begin{solutionbox}

\textbf{FDDI રચના:}

\begin{verbatim}
    Node A {-{-}{-}{-}{-}{-}{-}{-} Node B}
      |               |
      |               |
    Node D {-{-}{-}{-}{-}{-}{-}{-} Node C}
    
    Primary Ring: Clockwise
    Secondary Ring: Counter{-clockwise}
\end{verbatim}

\textbf{ઘટકો:}

\begin{itemize}
\tightlist
\item
  \textbf{ડ્યુઅલ રિંગ}: રિડન્ડન્સી માટે પ્રાઇમરી અને સેકન્ડરી રિંગ્સ
\item
  \textbf{ટોકન પાસિંગ}: મીડિયા એક્સેસ કંટ્રોલ માટે ટોકનનો ઉપયોગ
\item
  \textbf{કન્સન્ટ્રેટર્સ}: બહુવિધ સ્ટેશનોને રિંગ સાથે જોડે છે
\end{itemize}

\textbf{ફાયદાઓ:}

\begin{itemize}
\tightlist
\item
  \textbf{ઉચ્ચ વિશ્વસનીયતા}: ફોલ્ટ ટોલેરન્સ માટે ડ્યુઅલ રિંગ
\item
  \textbf{ઝડપી સ્પીડ}: 100 Mbps ડેટા ટ્રાન્સમિશન રેટ
\item
  \textbf{લાંબું અંતર}: 200 કિમી સુધી રિંગ સર્કમફરન્સને સપોર્ટ કરે છે
\item
  \textbf{સેલ્ફ-હીલિંગ}: લિંક નિષ્ફળ જાય ત્યારે ઓટોમેટિક રીકોન્ફિગરેશન
\end{itemize}

\end{solutionbox}
\begin{mnemonicbox}
``FDDI = Fast, Dual, Distance, Immune''

\end{mnemonicbox}
\subsection*{પ્રશ્ન 2(ક અથવા) [7
ગુણ]}\label{uxaaauxab0uxab6uxaa8-2uxa95-uxa85uxaa5uxab5-7-uxa97uxaa3}

\textbf{નેટવર્ક ઉપકરણોનો રોલ સમજાવો. બધા ઉપકરણો વિશે ટૂંકમાં વર્ણન કરો.}

\begin{solutionbox}

{\def\LTcaptype{none} % do not increment counter
\begin{longtable}[]{@{}lll@{}}
\toprule\noalign{}
ઉપકરણ & લેયર & કાર્ય \\
\midrule\noalign{}
\endhead
\bottomrule\noalign{}
\endlastfoot
\textbf{રીપીટર} & ફિઝિકલ & સિગ્નલ્સ રિજનરેટ કરે છે, અંતર વધારે છે \\
\textbf{હબ} & ફિઝિકલ & બહુવિધ ઉપકરણો જોડે છે, શેર કરેલ બેન્ડવિડ્થ \\
\textbf{બ્રિજ} & ડેટા લિંક & LANs જોડે છે, કોલિઝન્સ ઘટાડે છે \\
\textbf{સ્વિચ} & ડેટા લિંક & ઇન્ટેલિજન્ટ હબ, સમર્પિત બેન્ડવિડ્થ \\
\textbf{રાઉટર} & નેટવર્ક & વિવિધ નેટવર્કો જોડે છે, રાઉટિંગ \\
\textbf{ગેટવે} & બધા લેયર્સ & પ્રોટોકોલ કન્વર્ઝન, નેટવર્ક ઇન્ટરકનેક્શન \\
\end{longtable}
}

\begin{center}
\textbf{Mermaid Diagram (Code)}
\begin{verbatim}
{Shaded}
{Highlighting}[]
graph TD
    A[Physical Layer] {-{-}{} B[Repeater, Hub]}
    C[Data Link Layer] {-{-}{} D[Bridge, Switch]}
    E[Network Layer] {-{-}{} F[Router]}
    G[All Layers] {-{-}{} H[Gateway]}
{Highlighting}
{Shaded}
\end{verbatim}
\end{center}

\textbf{ઉપકરણ કાર્યો:}

\begin{itemize}
\tightlist
\item
  \textbf{રીપીટર}: સિગ્નલ્સ એમ્પ્લિફાઇ અને રિજનરેટ કરે છે
\item
  \textbf{હબ}: બહુવિધ ઉપકરણો માટે સરળ કનેક્શન પોઇન્ટ
\item
  \textbf{બ્રિજ}: MAC એડ્રેસના આધારે ઇન્ટેલિજન્ટ ફોરવર્ડિંગ
\item
  \textbf{સ્વિચ}: બહુવિધ પોર્ટ્સ સાથે ઉચ્ચ-પ્રદર્શન બ્રિજ
\item
  \textbf{રાઉટર}: નેટવર્કો વચ્ચે ઇન્ટેલિજન્ટ પાથ સિલેક્શન
\item
  \textbf{ગેટવે}: સંપૂર્ણ પ્રોટોકોલ સ્ટેક કન્વર્ઝન
\end{itemize}

\end{solutionbox}
\begin{mnemonicbox}
``Repeat, Hub, Bridge, Switch, Route, Gateway''

\end{mnemonicbox}
\subsection*{પ્રશ્ન 3(અ) [3
ગુણ]}\label{uxaaauxab0uxab6uxaa8-3uxa85-3-uxa97uxaa3}

\textbf{કોઈપણ ત્રણ ડેટા લિંક લેયર પ્રોટોકોલને નામ આપો અને કોઈપણ એકને વિગતવાર
સમજાવો.}

\begin{solutionbox}

\textbf{ડેટા લિંક લેયર પ્રોટોકોલ્સ:}

\begin{itemize}
\tightlist
\item
  \textbf{HDLC} (High-Level Data Link Control)
\item
  \textbf{PPP} (Point-to-Point Protocol)
\item
  \textbf{Ethernet} (IEEE 802.3)
\end{itemize}

\textbf{HDLC પ્રોટોકોલ વિગતો:}

\begin{itemize}
\tightlist
\item
  \textbf{ફ્રેમ સ્ટ્રક્ચર}: Flag, Address, Control, Data, FCS, Flag
\item
  \textbf{એરર ડિટેક્શન}: ફ્રેમ ચેક સિક્વન્સ (FCS)
\item
  \textbf{ફ્લો કંટ્રોલ}: સ્લાઇડિંગ વિન્ડો મેકેનિઝમ
\end{itemize}

\textbf{HDLC ફ્રેમ ફોર્મેટ:}

\begin{verbatim}
+{-{-}{-}{-}{-}{-}+{-}{-}{-}{-}{-}{-}+{-}{-}{-}{-}{-}{-}+{-}{-}{-}{-}{-}{-}+{-}{-}{-}{-}{-}{-}+{-}{-}{-}{-}{-}{-}+}
| Flag |Addr  |Ctrl  | Data | FCS  | Flag |
| 8bit |8bit  |8bit  |      |16bit | 8bit |
+{-{-}{-}{-}{-}{-}+{-}{-}{-}{-}{-}{-}+{-}{-}{-}{-}{-}{-}+{-}{-}{-}{-}{-}{-}+{-}{-}{-}{-}{-}{-}+{-}{-}{-}{-}{-}{-}+}
\end{verbatim}

\end{solutionbox}
\begin{mnemonicbox}
``HDLC = High Data Link Control''

\end{mnemonicbox}
\subsection*{પ્રશ્ન 3(બ) [4
ગુણ]}\label{uxaaauxab0uxab6uxaa8-3uxaac-4-uxa97uxaa3}

\textbf{ડેટા લિંક સ્તર પર error control અને flow control સમજાવો}

\begin{solutionbox}

{\def\LTcaptype{none} % do not increment counter
\begin{longtable}[]{@{}
  >{\raggedright\arraybackslash}p{(\linewidth - 4\tabcolsep) * \real{0.5000}}
  >{\raggedright\arraybackslash}p{(\linewidth - 4\tabcolsep) * \real{0.2000}}
  >{\raggedright\arraybackslash}p{(\linewidth - 4\tabcolsep) * \real{0.3000}}@{}}
\toprule\noalign{}
\begin{minipage}[b]{\linewidth}\raggedright
કંટ્રોલ પ્રકાર
\end{minipage} & \begin{minipage}[b]{\linewidth}\raggedright
હેતુ
\end{minipage} & \begin{minipage}[b]{\linewidth}\raggedright
પદ્ધતિઓ
\end{minipage} \\
\midrule\noalign{}
\endhead
\bottomrule\noalign{}
\endlastfoot
\textbf{Error Control} & ટ્રાન્સમિશન એરર્સ ડિટેક્ટ અને કરેક્ટ કરવા & CRC,
Checksum, Parity \\
\textbf{Flow Control} & ડેટા ટ્રાન્સમિશન રેટ મેનેજ કરવા & Stop-and-Wait,
Sliding Window \\
\end{longtable}
}

\textbf{Error Control પદ્ધતિઓ:}

\begin{itemize}
\tightlist
\item
  \textbf{ડિટેક્શન}: CRC, Checksum એરર્સ ઓળખે છે
\item
  \textbf{કરેક્શન}: ARQ (Automatic Repeat Request)
\item
  \textbf{પ્રિવેન્શન}: Forward Error Correction (FEC)
\end{itemize}

\textbf{Flow Control પદ્ધતિઓ:}

\begin{itemize}
\tightlist
\item
  \textbf{Stop-and-Wait}: એક ફ્રેમ મોકલો, ACK ની રાહ જુઓ
\item
  \textbf{Sliding Window}: ACK પહેલાં બહુવિધ ફ્રેમ્સ મોકલો
\item
  \textbf{બફર મેનેજમેન્ટ}: રિસીવર ઓવરફ્લો અટકાવે છે
\end{itemize}

\end{solutionbox}
\begin{mnemonicbox}
``Error = Detect, Flow = Control''

\end{mnemonicbox}
\subsection*{પ્રશ્ન 3(ક) [7
ગુણ]}\label{uxaaauxab0uxab6uxaa8-3uxa95-7-uxa97uxaa3}

\textbf{IPv6 અને IPv4 ની સરખામણી કરો.}

\begin{solutionbox}

{\def\LTcaptype{none} % do not increment counter
\begin{longtable}[]{@{}lll@{}}
\toprule\noalign{}
લક્ષણ & IPv4 & IPv6 \\
\midrule\noalign{}
\endhead
\bottomrule\noalign{}
\endlastfoot
\textbf{એડ્રેસ લેન્થ} & 32 બિટ્સ & 128 બિટ્સ \\
\textbf{એડ્રેસ સ્પેસ} & 4.3 બિલિયન & 340 અન્ડેસિલિયન \\
\textbf{હેડર સાઇઝ} & 20-60 બાઇટ્સ (વેરિએબલ) & 40 બાઇટ્સ (ફિક્સ્ડ) \\
\textbf{નોટેશન} & ડેસિમલ (192.168.1.1) & હેક્સાડેસિમલ (2001:db8::1) \\
\textbf{ફ્રેગમેન્ટેશન} & રાઉટર અને હોસ્ટ & માત્ર હોસ્ટ \\
\textbf{સિક્યુરિટી} & વૈકલ્પિક (IPSec) & બિલ્ટ-ઇન (IPSec) \\
\textbf{કોન્ફિગરેશન} & મેન્યુઅલ/DHCP & ઓટો-કોન્ફિગરેશન \\
\end{longtable}
}

\textbf{IPv4 ઉદાહરણ:} 192.168.1.100 \textbf{IPv6 ઉદાહરણ:}
2001:0db8:85a3:0000:0000:8a2e:0370:7334

\textbf{મુખ્ય તફાવતો:}

\begin{itemize}
\tightlist
\item
  \textbf{એડ્રેસ એક્ઝોસ્ચન}: IPv4 એડ્રેસ લગભગ સમાપ્ત
\item
  \textbf{હેડર એફિશિયન્સી}: IPv6 સિમ્પ્લિફાઇડ હેડર સ્ટ્રક્ચર
\item
  \textbf{સિક્યુરિટી}: IPv6 માં બિલ્ટ-ઇન સિક્યુરિટી ફીચર્સ
\item
  \textbf{ક્વોલિટી ઓફ સર્વિસ}: IPv6 માં વધુ સારો QoS સપોર્ટ
\end{itemize}

\end{solutionbox}
\begin{mnemonicbox}
``IPv6 = Infinite, Integrated, Improved''

\end{mnemonicbox}
\subsection*{પ્રશ્ન 3(અ અથવા) [3
ગુણ]}\label{uxaaauxab0uxab6uxaa8-3uxa85-uxa85uxaa5uxab5-3-uxa97uxaa3}

\textbf{કમ્પ્યુટર નેટવર્કમાં વપરાતા guided અને unguided ટ્રાન્સમિશન મીડિયા વચ્ચેનો
તફાવત સમજાવો}

\begin{solutionbox}

{\def\LTcaptype{none} % do not increment counter
\begin{longtable}[]{@{}lll@{}}
\toprule\noalign{}
મીડિયા પ્રકાર & Guided & Unguided \\
\midrule\noalign{}
\endhead
\bottomrule\noalign{}
\endlastfoot
\textbf{વ્યાખ્યા} & ભૌતિક પાથ અસ્તિત્વમાં છે & કોઈ ભૌતિક પાથ નથી \\
\textbf{ઉદાહરણો} & ટ્વિસ્ટેડ પેર, Coax, ફાઇબર & રેડિયો, માઇક્રોવેવ, સેટેલાઇટ \\
\textbf{દિશા} & પોઇન્ટ-ટુ-પોઇન્ટ & બ્રોડકાસ્ટ \\
\end{longtable}
}

\textbf{Guided મીડિયા:}

\begin{itemize}
\tightlist
\item
  \textbf{ટ્વિસ્ટેડ પેર}: ટેલિફોન લાઇન્સ, LANs
\item
  \textbf{કોએક્સિયલ કેબલ}: કેબલ TV, જૂના નેટવર્કસ
\item
  \textbf{ફાઇબર ઓપ્ટિક}: હાઇ-સ્પીડ, લોંગ-ડિસ્ટન્સ
\end{itemize}

\textbf{Unguided મીડિયા:}

\begin{itemize}
\tightlist
\item
  \textbf{રેડિયો વેવ્સ}: WiFi, Bluetooth
\item
  \textbf{માઇક્રોવેવ્સ}: પોઇન્ટ-ટુ-પોઇન્ટ લિંક્સ
\item
  \textbf{ઇન્ફ્રારેડ}: શોર્ટ-રેન્જ કમ્યુનિકેશન
\end{itemize}

\end{solutionbox}
\begin{mnemonicbox}
``Guided = Ground, Unguided = Air''

\end{mnemonicbox}
\subsection*{પ્રશ્ન 3(બ અથવા) [4
ગુણ]}\label{uxaaauxab0uxab6uxaa8-3uxaac-uxa85uxaa5uxab5-4-uxa97uxaa3}

\textbf{સર્કિટ સ્વિચિંગ અને પેકેટ સ્વિચિંગનું વર્ણન કરો.}

\begin{solutionbox}

{\def\LTcaptype{none} % do not increment counter
\begin{longtable}[]{@{}lll@{}}
\toprule\noalign{}
લક્ષણ & સર્કિટ સ્વિચિંગ & પેકેટ સ્વિચિંગ \\
\midrule\noalign{}
\endhead
\bottomrule\noalign{}
\endlastfoot
\textbf{કનેક્શન} & સમર્પિત પાથ સ્થાપિત & કોઈ સમર્પિત પાથ નથી \\
\textbf{રિસોર્સ એલોકેશન} & ફિક્સ્ડ બેન્ડવિડ્થ & શેર કરેલા રિસોર્સિસ \\
\textbf{ઉદાહરણ} & પરંપરાગત ટેલિફોન & ઇન્ટરનેટ \\
\textbf{ડિલે} & કોન્સ્ટન્ટ & વેરિએબલ \\
\end{longtable}
}

\textbf{સર્કિટ સ્વિચિંગ:}

\begin{itemize}
\tightlist
\item
  \textbf{સેટઅપ ફેઝ}: સમર્પિત કનેક્શન સ્થાપિત કરે છે
\item
  \textbf{ડેટા ટ્રાન્સફર}: કોન્ટિન્યુઅસ ટ્રાન્સમિશન
\item
  \textbf{ટિયરડાઉન}: કનેક્શન રિસોર્સિસ રિલીઝ કરે છે
\end{itemize}

\textbf{પેકેટ સ્વિચિંગ:}

\begin{itemize}
\tightlist
\item
  \textbf{સ્ટોર-એન્ડ-ફોરવર્ડ}: પેકેટ્સ ઇન્ટરમીડિયેટ નોડ્સ પર સ્ટોર થાય છે
\item
  \textbf{ડાયનેમિક રાઉટિંગ}: દરેક પેકેટ સ્વતંત્ર રીતે રાઉટ થાય છે
\item
  \textbf{રિસોર્સ શેરિંગ}: બેન્ડવિડ્થ યુઝર્સ વચ્ચે શેર થાય છે
\end{itemize}

\end{solutionbox}
\begin{mnemonicbox}
``Circuit = Continuous, Packet = Pieces''

\end{mnemonicbox}
\subsection*{પ્રશ્ન 3(ક અથવા) [7
ગુણ]}\label{uxaaauxab0uxab6uxaa8-3uxa95-uxa85uxaa5uxab5-7-uxa97uxaa3}

\textbf{IPv4 અથવા IPv6 ને વિગતવાર સમજાવો.}

\begin{solutionbox}
(IPv4):

\textbf{IPv4 એડ્રેસ સ્ટ્રક્ચર:}

\begin{itemize}
\tightlist
\item
  \textbf{32-બિટ એડ્રેસ}: 4 ઓક્ટેટ્સમાં વિભાજિત
\item
  \textbf{ડોટેડ ડેસિમલ}: 192.168.1.1 ફોર્મેટ
\item
  \textbf{નેટવર્ક + હોસ્ટ}: એડ્રેસ નેટવર્ક અને હોસ્ટ ભાગોમાં વિભાજિત
\end{itemize}

{\def\LTcaptype{none} % do not increment counter
\begin{longtable}[]{@{}lllll@{}}
\toprule\noalign{}
ક્લાસ & રેન્જ & નેટવર્ક બિટ્સ & હોસ્ટ બિટ્સ & ઉપયોગ \\
\midrule\noalign{}
\endhead
\bottomrule\noalign{}
\endlastfoot
\textbf{A} & 1-126 & 8 & 24 & મોટા નેટવર્કસ \\
\textbf{B} & 128-191 & 16 & 16 & મધ્યમ નેટવર્કસ \\
\textbf{C} & 192-223 & 24 & 8 & નાના નેટવર્કસ \\
\end{longtable}
}

\textbf{સ્પેશિયલ એડ્રેસિસ:}

\begin{itemize}
\tightlist
\item
  \textbf{લૂપબેક}: 127.0.0.1 (લોકલ હોસ્ટ)
\item
  \textbf{પ્રાઇવેટ}: 192.168.x.x, 10.x.x.x, 172.16-31.x.x
\item
  \textbf{બ્રોડકાસ્ટ}: 255.255.255.255
\end{itemize}

\textbf{સબનેટિંગ:}

\begin{itemize}
\tightlist
\item
  \textbf{સબનેટ માસ્ક}: નેટવર્ક પોર્શન ઓળખે છે
\item
  \textbf{CIDR}: Classless Inter-Domain Routing
\item
  \textbf{વેરિએબલ લેન્થ}: વિવિધ સબનેટ સાઇઝિસ
\end{itemize}

\textbf{IPv4 હેડર:}

\begin{verbatim}
0               16              32
+{-{-}{-}{-}{-}{-}{-}{-}{-}{-}{-}{-}{-}{-}{-}+{-}{-}{-}{-}{-}{-}{-}{-}{-}{-}{-}{-}{-}{-}{-}+}
|Version| IHL   |Type of Service|
+{-{-}{-}{-}{-}{-}{-}{-}{-}{-}{-}{-}{-}{-}{-}+{-}{-}{-}{-}{-}{-}{-}{-}{-}{-}{-}{-}{-}{-}{-}+}
|     Total Length              |
+{-{-}{-}{-}{-}{-}{-}{-}{-}{-}{-}{-}{-}{-}{-}+{-}{-}{-}{-}{-}{-}{-}{-}{-}{-}{-}{-}{-}{-}{-}+}
|Identification |Flags|Fragment |
+{-{-}{-}{-}{-}{-}{-}{-}{-}{-}{-}{-}{-}{-}{-}+{-}{-}{-}{-}{-}{-}{-}{-}{-}{-}{-}{-}{-}{-}{-}+}
| TTL  |Protocol|Header Checksum|
+{-{-}{-}{-}{-}{-}{-}{-}{-}{-}{-}{-}{-}{-}{-}+{-}{-}{-}{-}{-}{-}{-}{-}{-}{-}{-}{-}{-}{-}{-}+}
|     Source Address            |
+{-{-}{-}{-}{-}{-}{-}{-}{-}{-}{-}{-}{-}{-}{-}+{-}{-}{-}{-}{-}{-}{-}{-}{-}{-}{-}{-}{-}{-}{-}+}
|   Destination Address         |
+{-{-}{-}{-}{-}{-}{-}{-}{-}{-}{-}{-}{-}{-}{-}+{-}{-}{-}{-}{-}{-}{-}{-}{-}{-}{-}{-}{-}{-}{-}+}
\end{verbatim}

\end{solutionbox}
\begin{mnemonicbox}
``IPv4 = 4 octets, 32 bits, Classes A-C''

\end{mnemonicbox}
\subsection*{પ્રશ્ન 4(અ) [3
ગુણ]}\label{uxaaauxab0uxab6uxaa8-4uxa85-3-uxa97uxaa3}

\textbf{ARP અને RARP ના પૂરા નામ આપો અને તેમનું વર્ણન કરો.}

\begin{solutionbox}

\textbf{પૂરા નામો:}

\begin{itemize}
\tightlist
\item
  \textbf{ARP}: Address Resolution Protocol
\item
  \textbf{RARP}: Reverse Address Resolution Protocol
\end{itemize}

{\def\LTcaptype{none} % do not increment counter
\begin{longtable}[]{@{}ll@{}}
\toprule\noalign{}
પ્રોટોકોલ & કાર્ય \\
\midrule\noalign{}
\endhead
\bottomrule\noalign{}
\endlastfoot
\textbf{ARP} & IP એડ્રેસને MAC એડ્રેસ પર મેપ કરે છે \\
\textbf{RARP} & MAC એડ્રેસને IP એડ્રેસ પર મેપ કરે છે \\
\end{longtable}
}

\textbf{ARP પ્રોસેસ:}

\begin{itemize}
\tightlist
\item
  \textbf{રિક્વેસ્ટ}: ``કોની પાસે IP 192.168.1.1 છે?''
\item
  \textbf{રિપ્લાય}: ``192.168.1.1 MAC 00:1A:2B:3C:4D:5E પર છે''
\item
  \textbf{કેશ}: ભવિષ્યના ઉપયોગ માટે મેપિંગ્સ સ્ટોર કરે છે
\end{itemize}

\textbf{RARP પ્રોસેસ:}

\begin{itemize}
\tightlist
\item
  \textbf{ડિસ્કલેસ વર્કસ્ટેશન્સ}: સર્વરથી IP મેળવે છે
\item
  \textbf{બ્રોડકાસ્ટ રિક્વેસ્ટ}: MAC એડ્રેસ મોકલે છે
\item
  \textbf{સર્વર રિસ્પોન્સ}: એસાઇન કરેલ IP રિટર્ન કરે છે
\end{itemize}

\end{solutionbox}
\begin{mnemonicbox}
``ARP = Address to MAC, RARP = Reverse''

\end{mnemonicbox}
\subsection*{પ્રશ્ન 4(બ) [4
ગુણ]}\label{uxaaauxab0uxab6uxaa8-4uxaac-4-uxa97uxaa3}

\textbf{DSL ટેકનોલોજીનું તેના ફાયદા અને મર્યાદાઓ સાથે વર્ણન કરો.}

\begin{solutionbox}

\textbf{DSL (Digital Subscriber Line):}

{\def\LTcaptype{none} % do not increment counter
\begin{longtable}[]{@{}lll@{}}
\toprule\noalign{}
પ્રકાર & સ્પીડ & અંતર \\
\midrule\noalign{}
\endhead
\bottomrule\noalign{}
\endlastfoot
\textbf{ADSL} & 8 Mbps સુધી & 5.5 કિમી \\
\textbf{VDSL} & 52 Mbps સુધી & 1.5 કિમી \\
\textbf{SDSL} & 2 Mbps સુધી & 3 કિમી \\
\end{longtable}
}

\textbf{ફાયદાઓ:}

\begin{itemize}
\tightlist
\item
  \textbf{વર્તમાન ઇન્ફ્રાસ્ટ્રક્ચર}: ટેલિફોન લાઇન્સનો ઉપયોગ કરે છે
\item
  \textbf{હંમેશા ઓન}: કોન્ટિન્યુઅસ ઇન્ટરનેટ કનેક્શન
\item
  \textbf{વોઇસ + ડેટા}: સાથે સાથે ફોન અને ઇન્ટરનેટ
\item
  \textbf{કિફાયતી}: ઘરેલું ઉપયોગકર્તાઓ માટે પોસાય
\end{itemize}

\textbf{મર્યાદાઓ:}

\begin{itemize}
\tightlist
\item
  \textbf{અંતર આધારિત}: અંતર સાથે સ્પીડ ઘટે છે
\item
  \textbf{અપલોડ સ્પીડ}: ડાઉનલોડ સ્પીડ કરતાં ઓછી (ADSL)
\item
  \textbf{લાઇન ક્વોલિટી}: કોપર વાયરની સ્થિતિથી પ્રભાવિત
\item
  \textbf{ઉપલબ્ધતા}: બધા વિસ્તારોમાં ઉપલબ્ધ નથી
\end{itemize}

\end{solutionbox}
\begin{mnemonicbox}
``DSL = Digital Subscriber Line''

\end{mnemonicbox}
\subsection*{પ્રશ્ન 4(ક) [7
ગુણ]}\label{uxaaauxab0uxab6uxaa8-4uxa95-7-uxa97uxaa3}

\textbf{DNS- ડોમેન નેમ સિસ્ટમની ભૂમિકા.}

\begin{solutionbox}

\textbf{DNS કાર્યો:}

\begin{itemize}
\tightlist
\item
  \textbf{નેમ રિઝોલ્યુશન}: ડોમેન નેમ્સને IP એડ્રેસિસમાં કન્વર્ટ કરે છે
\item
  \textbf{હાયરાર્કિકલ સ્ટ્રક્ચર}: ટ્રી-જેવા માળખામાં ગોઠવાયેલું
\item
  \textbf{ડિસ્ટ્રિબ્યુટેડ ડેટાબેસ}: માહિતી બહુવિધ સર્વર્સ પર સ્ટોર થાય છે
\end{itemize}

\begin{center}
\textbf{Mermaid Diagram (Code)}
\begin{verbatim}
{Shaded}
{Highlighting}[]
graph LR
    A[Root Servers] {-{-}{} B[Top Level Domain .com]}
    A {-{-}{} C[Top Level Domain .org]}
    B {-{-}{} D[google.com]}
    B {-{-}{} E[yahoo.com]}
    D {-{-}{} F[drive.google.com]}
    D {-{-}{} G[mail.google.com]}
{Highlighting}
{Shaded}
\end{verbatim}
\end{center}

\textbf{DNS હાયરાર્કી:}

\begin{itemize}
\tightlist
\item
  \textbf{રૂટ ડોમેન}: સર્વોચ્ચ સ્તર (.)
\item
  \textbf{ટોપ-લેવલ ડોમેન}: .com, .org, .net, .edu
\item
  \textbf{સેકન્ડ-લેવલ ડોમેન}: google.com, yahoo.com
\item
  \textbf{સબડોમેન}: www.google.com, mail.google.com
\end{itemize}

\textbf{DNS રિઝોલ્યુશન પ્રોસેસ:}

\begin{enumerate}
\tightlist
\item
  \textbf{ક્લાયન્ટ ક્વેરી}: યુઝર www.example.com ટાઇપ કરે છે
\item
  \textbf{લોકલ DNS}: લોકલ કેશ ચેક કરે છે
\item
  \textbf{રૂટ સર્વર}: રૂટ DNS સર્વર ક્વેરી કરે છે
\item
  \textbf{TLD સર્વર}: .com સર્વર ક્વેરી કરે છે
\item
  \textbf{ઓથોરિટેટિવ સર્વર}: IP એડ્રેસ મેળવે છે
\item
  \textbf{રિસ્પોન્સ}: ક્લાયન્ટને IP રિટર્ન કરે છે
\end{enumerate}

\textbf{DNS રેકોર્ડ પ્રકારો:}

\begin{itemize}
\tightlist
\item
  \textbf{A રેકોર્ડ}: ડોમેનને IPv4 એડ્રેસ પર મેપ કરે છે
\item
  \textbf{AAAA રેકોર્ડ}: ડોમેનને IPv6 એડ્રેસ પર મેપ કરે છે
\item
  \textbf{CNAME}: કેનોનિકલ નેમ (એલિયાસ)
\item
  \textbf{MX}: મેઇલ એક્સચેન્જ સર્વર
\item
  \textbf{NS}: નેમ સર્વર રેકોર્ડ્સ
\end{itemize}

\end{solutionbox}
\begin{mnemonicbox}
``DNS = Domain Name System''

\end{mnemonicbox}
\subsection*{પ્રશ્ન 4(અ અથવા) [3
ગુણ]}\label{uxaaauxab0uxab6uxaa8-4uxa85-uxa85uxaa5uxab5-3-uxa97uxaa3}

\textbf{DHCP અને BOOTP નું પૂરું નામ આપો. અને તેમનું વર્ણન કરો.}

\begin{solutionbox}

\textbf{પૂરા નામો:}

\begin{itemize}
\tightlist
\item
  \textbf{DHCP}: Dynamic Host Configuration Protocol
\item
  \textbf{BOOTP}: Bootstrap Protocol
\end{itemize}

{\def\LTcaptype{none} % do not increment counter
\begin{longtable}[]{@{}ll@{}}
\toprule\noalign{}
પ્રોટોકોલ & કાર્ય \\
\midrule\noalign{}
\endhead
\bottomrule\noalign{}
\endlastfoot
\textbf{DHCP} & ઓટોમેટિકલી IP એડ્રેસિસ એસાઇન કરે છે \\
\textbf{BOOTP} & ડિસ્કલેસ વર્કસ્ટેશન્સને IP એડ્રેસ પ્રદાન કરે છે \\
\end{longtable}
}

\textbf{DHCP પ્રોસેસ:}

\begin{itemize}
\tightlist
\item
  \textbf{ડિસ્કવર}: ક્લાયન્ટ બ્રોડકાસ્ટ રિક્વેસ્ટ
\item
  \textbf{ઓફર}: સર્વર IP એડ્રેસ ઓફર કરે છે
\item
  \textbf{રિક્વેસ્ટ}: ક્લાયન્ટ સ્પેસિફિક IP રિક્વેસ્ટ કરે છે
\item
  \textbf{એકનોલેજ}: સર્વર એસાઇનમેન્ટ કન્ફર્મ કરે છે
\end{itemize}

\textbf{BOOTP પ્રોસેસ:}

\begin{itemize}
\tightlist
\item
  \textbf{સ્ટેટિક કોન્ફિગરેશન}: પ્રી-કોન્ફિગર્ડ IP એસાઇનમેન્ટ્સ
\item
  \textbf{ડિસ્કલેસ બૂટ}: વર્કસ્ટેશન્સ નેટવર્કથી બૂટ થાય છે
\item
  \textbf{સર્વર રિસ્પોન્સ}: IP અને બૂટ માહિતી પ્રદાન કરે છે
\end{itemize}

\end{solutionbox}
\begin{mnemonicbox}
``DHCP = Dynamic, BOOTP = Bootstrap''

\end{mnemonicbox}
\subsection*{પ્રશ્ન 4(બ અથવા) [4
ગુણ]}\label{uxaaauxab0uxab6uxaa8-4uxaac-uxa85uxaa5uxab5-4-uxa97uxaa3}

\textbf{વર્ચુઅલ સર્કિટ્સ અને ડેટાગ્રામ નેટવર્ક્સ વચ્ચેના તફાવતો લખો.}

\begin{solutionbox}

{\def\LTcaptype{none} % do not increment counter
\begin{longtable}[]{@{}lll@{}}
\toprule\noalign{}
લક્ષણ & વર્ચુઅલ સર્કિટ્સ & ડેટાગ્રામ નેટવર્ક્સ \\
\midrule\noalign{}
\endhead
\bottomrule\noalign{}
\endlastfoot
\textbf{કનેક્શન} & કનેક્શન-ઓરિએન્ટેડ & કનેક્શનલેસ \\
\textbf{સેટઅપ} & સેટઅપ ફેઝ જરૂરી & કોઈ સેટઅપ જરૂરી નથી \\
\textbf{રાઉટિંગ} & બધા પેકેટ્સ માટે એક જ પાથ & સ્વતંત્ર રાઉટિંગ \\
\textbf{ઓર્ડર} & પેકેટ્સ ક્રમમાં આવે છે & ક્રમ બહાર આવી શકે છે \\
\textbf{વિશ્વસનીયતા} & વધુ વિશ્વસનીય & ઓછી વિશ્વસનીય \\
\textbf{ઓવરહેડ} & વધુ સેટઅપ ઓવરહેડ & ઓછો પર-પેકેટ ઓવરહેડ \\
\end{longtable}
}

\textbf{વર્ચુઅલ સર્કિટ્સ:}

\begin{itemize}
\tightlist
\item
  \textbf{પાથ એસ્ટેબ્લિશમેન્ટ}: વર્ચુઅલ કનેક્શન બનાવે છે
\item
  \textbf{સ્ટેટ ઇન્ફોર્મેશન}: કનેક્શન સ્ટેટ મેન્ટેઇન કરે છે
\item
  \textbf{ઉદાહરણો}: ATM, Frame Relay
\end{itemize}

\textbf{ડેટાગ્રામ નેટવર્ક્સ:}

\begin{itemize}
\tightlist
\item
  \textbf{સ્વતંત્ર પેકેટ્સ}: દરેક પેકેટ અલગથી રાઉટ થાય છે
\item
  \textbf{સ્ટેટલેસ}: કોઈ કનેક્શન સ્ટેટ મેન્ટેઇન નથી
\item
  \textbf{ઉદાહરણો}: ઇન્ટરનેટ પ્રોટોકોલ (IP)
\end{itemize}

\end{solutionbox}
\begin{mnemonicbox}
``Virtual = Connection, Datagram = Independent''

\end{mnemonicbox}
\subsection*{પ્રશ્ન 4(ક અથવા) [7
ગુણ]}\label{uxaaauxab0uxab6uxaa8-4uxa95-uxa85uxaa5uxab5-7-uxa97uxaa3}

\textbf{ટ્રાન્સપોર્ટ લેયરમાં TCP અને UDP પ્રોટોકોલ સમજાવો}

\begin{solutionbox}

{\def\LTcaptype{none} % do not increment counter
\begin{longtable}[]{@{}lll@{}}
\toprule\noalign{}
લક્ષણ & TCP & UDP \\
\midrule\noalign{}
\endhead
\bottomrule\noalign{}
\endlastfoot
\textbf{કનેક્શન} & કનેક્શન-ઓરિએન્ટેડ & કનેક્શનલેસ \\
\textbf{વિશ્વસનીયતા} & વિશ્વસનીય & અવિશ્વસનીય \\
\textbf{હેડર સાઇઝ} & 20 બાઇટ્સ & 8 બાઇટ્સ \\
\textbf{ફ્લો કંટ્રોલ} & હા & ના \\
\textbf{એરર કંટ્રોલ} & હા & મૂળભૂત \\
\textbf{સ્પીડ} & ધીમી & ઝડપી \\
\end{longtable}
}

\textbf{TCP (Transmission Control Protocol):}

\begin{itemize}
\tightlist
\item
  \textbf{થ્રી-વે હેન્ડશેક}: SYN, SYN-ACK, ACK
\item
  \textbf{ફ્લો કંટ્રોલ}: સ્લાઇડિંગ વિન્ડો મેકેનિઝમ
\item
  \textbf{એરર રિકવરી}: ગુમ થયેલા પેકેટ્સનું રિટ્રાન્સમિશન
\item
  \textbf{કંજેશન કંટ્રોલ}: નેટવર્ક ઓવરલોડ અટકાવે છે
\end{itemize}

\textbf{TCP હેડર:}

\begin{verbatim}
0               16              32
+{-{-}{-}{-}{-}{-}{-}{-}{-}{-}{-}{-}{-}{-}{-}+{-}{-}{-}{-}{-}{-}{-}{-}{-}{-}{-}{-}{-}{-}{-}{-}+}
|Source Port    |Destination Port|
+{-{-}{-}{-}{-}{-}{-}{-}{-}{-}{-}{-}{-}{-}{-}+{-}{-}{-}{-}{-}{-}{-}{-}{-}{-}{-}{-}{-}{-}{-}{-}+}
|      Sequence Number           |
+{-{-}{-}{-}{-}{-}{-}{-}{-}{-}{-}{-}{-}{-}{-}+{-}{-}{-}{-}{-}{-}{-}{-}{-}{-}{-}{-}{-}{-}{-}{-}+}
|   Acknowledgment Number        |
+{-{-}{-}{-}{-}{-}{-}{-}{-}{-}{-}{-}{-}{-}{-}+{-}{-}{-}{-}{-}{-}{-}{-}{-}{-}{-}{-}{-}{-}{-}{-}+}
|Hdr|   |U|A|P|R|S|F|    Window  |
|Len|   |R|C|S|S|Y|I|     Size   |
+{-{-}{-}{-}{-}{-}{-}{-}{-}{-}{-}{-}{-}{-}{-}+{-}{-}{-}{-}{-}{-}{-}{-}{-}{-}{-}{-}{-}{-}{-}{-}+}
\end{verbatim}

\textbf{UDP (User Datagram Protocol):}

\begin{itemize}
\tightlist
\item
  \textbf{સરળ પ્રોટોકોલ}: ન્યૂનતમ ઓવરહેડ
\item
  \textbf{બેસ્ટ એફર્ટ}: ડિલિવરીની કોઈ ગેરંટી નથી
\item
  \textbf{એપ્લિકેશન્સ}: DNS, DHCP, સ્ટ્રીમિંગ મીડિયા
\item
  \textbf{રીઅલ-ટાઇમ કમ્યુનિકેશન}: વોઇસ, વિડિયો એપ્લિકેશન્સ
\end{itemize}

\textbf{UDP હેડર:}

\begin{verbatim}
0               16              32
+{-{-}{-}{-}{-}{-}{-}{-}{-}{-}{-}{-}{-}{-}{-}+{-}{-}{-}{-}{-}{-}{-}{-}{-}{-}{-}{-}{-}{-}{-}{-}+}
|Source Port    |Destination Port|
+{-{-}{-}{-}{-}{-}{-}{-}{-}{-}{-}{-}{-}{-}{-}+{-}{-}{-}{-}{-}{-}{-}{-}{-}{-}{-}{-}{-}{-}{-}{-}+}
|    Length     |   Checksum     |
+{-{-}{-}{-}{-}{-}{-}{-}{-}{-}{-}{-}{-}{-}{-}+{-}{-}{-}{-}{-}{-}{-}{-}{-}{-}{-}{-}{-}{-}{-}{-}+}
\end{verbatim}

\textbf{એપ્લિકેશન્સ:}

\begin{itemize}
\tightlist
\item
  \textbf{TCP}: વેબ બ્રાઉઝિંગ, ઈમેઇલ, ફાઇલ ટ્રાન્સફર
\item
  \textbf{UDP}: ઓનલાઇન ગેમિંગ, વિડિયો સ્ટ્રીમિંગ, DNS ક્વેરીઝ
\end{itemize}

\end{solutionbox}
\begin{mnemonicbox}
``TCP = Reliable, UDP = Fast''

\end{mnemonicbox}
\subsection*{પ્રશ્ન 5(અ) [3
ગુણ]}\label{uxaaauxab0uxab6uxaa8-5uxa85-3-uxa97uxaa3}

\textbf{નીચેનામાંથી કોઈ પણ બે સમજાવો. (1) WWW (2) FTP (3) SMTP}

\begin{solutionbox}

\textbf{WWW (World Wide Web):}

\begin{itemize}
\tightlist
\item
  \textbf{HTTP પ્રોટોકોલ}: HyperText Transfer Protocol
\item
  \textbf{વેબ બ્રાઉઝર}: ક્લાયન્ટ સોફ્ટવેર (Chrome, Firefox)
\item
  \textbf{વેબ સર્વર}: વેબ પેજ સર્વ કરે છે (Apache, IIS)
\end{itemize}

\textbf{FTP (File Transfer Protocol):}

\begin{itemize}
\tightlist
\item
  \textbf{ફાઇલ ટ્રાન્સફર}: ફાઇલો અપલોડ અને ડાઉનલોડ કરવા
\item
  \textbf{બે મોડ્સ}: એક્ટિવ અને પેસિવ મોડ
\item
  \textbf{ઓથેન્ટિકેશન}: યુઝરનેમ અને પાસવર્ડ જરૂરી
\end{itemize}

{\def\LTcaptype{none} % do not increment counter
\begin{longtable}[]{@{}lll@{}}
\toprule\noalign{}
સેવા & પોર્ટ & કાર્ય \\
\midrule\noalign{}
\endhead
\bottomrule\noalign{}
\endlastfoot
\textbf{WWW} & 80/443 & વેબ પેજ ડિલિવરી \\
\textbf{FTP} & 20/21 & ફાઇલ ટ્રાન્સફર \\
\end{longtable}
}

\end{solutionbox}
\begin{mnemonicbox}
``WWW = Web, FTP = Files''

\end{mnemonicbox}
\subsection*{પ્રશ્ન 5(બ) [4
ગુણ]}\label{uxaaauxab0uxab6uxaa8-5uxaac-4-uxa97uxaa3}

\textbf{સિમેટ્રિક અને એસિમેટ્રિક એન્ક્રિપ્શન અલ્ગોરિધમ્સ વચ્ચેનો તફાવત લખો.}

\begin{solutionbox}

{\def\LTcaptype{none} % do not increment counter
\begin{longtable}[]{@{}lll@{}}
\toprule\noalign{}
લક્ષણ & સિમેટ્રિક & એસિમેટ્રિક \\
\midrule\noalign{}
\endhead
\bottomrule\noalign{}
\endlastfoot
\textbf{કીઝ} & એન્ક્રિપ્શન/ડિક્રિપ્શન માટે એક જ કી & વિવિધ કીઝ
(પબ્લિક/પ્રાઇવેટ) \\
\textbf{સ્પીડ} & ઝડપી & ધીમી \\
\textbf{કી ડિસ્ટ્રિબ્યુશન} & મુશ્કેલ & સરળ \\
\textbf{ઉદાહરણો} & AES, DES & RSA, ECC \\
\end{longtable}
}

\textbf{સિમેટ્રિક એન્ક્રિપ્શન:}

\begin{itemize}
\tightlist
\item
  \textbf{સિંગલ કી}: મોકલનાર અને મેળવનાર બંને એક જ કીનો ઉપયોગ
\item
  \textbf{કી મેનેજમેન્ટ}: સુરક્ષિત કી ડિસ્ટ્રિબ્યુશન જરૂરી
\item
  \textbf{પર્ફોર્મન્સ}: ઝડપી એન્ક્રિપ્શન/ડિક્રિપ્શન
\item
  \textbf{એપ્લિકેશન્સ}: બલ્ક ડેટા એન્ક્રિપ્શન
\end{itemize}

\textbf{એસિમેટ્રિક એન્ક્રિપ્શન:}

\begin{itemize}
\tightlist
\item
  \textbf{કી પેર}: એન્ક્રિપ્શન માટે પબ્લિક કી, ડિક્રિપ્શન માટે પ્રાઇવેટ કી
\item
  \textbf{કી ડિસ્ટ્રિબ્યુશન}: પબ્લિક કી ખુલ્લેઆમ શેર કરી શકાય
\item
  \textbf{પર્ફોર્મન્સ}: સિમેટ્રિક કરતાં ધીમું
\item
  \textbf{એપ્લિકેશન્સ}: ડિજિટલ સિગ્નેચર્સ, કી એક્સચેન્જ
\end{itemize}

\end{solutionbox}
\begin{mnemonicbox}
``Symmetric = Same, Asymmetric = Different''

\end{mnemonicbox}
\subsection*{પ્રશ્ન 5(ક) [7
ગુણ]}\label{uxaaauxab0uxab6uxaa8-5uxa95-7-uxa97uxaa3}

\textbf{ક્રિપ્ટોગ્રાફીના સંદર્ભમાં ``એન્ક્રિપ્શન'' અને ``ડિક્રિપ્શન'' શબ્દોને
વ્યાખ્યાયિત કરો.}

\begin{solutionbox}

\textbf{એન્ક્રિપ્શન:}

\begin{itemize}
\tightlist
\item
  \textbf{વ્યાખ્યા}: પ્લેઇનટેક્સ્ટને સાઇફરટેક્સ્ટમાં કન્વર્ટ કરવાની પ્રક્રિયા
\item
  \textbf{હેતુ}: ડેટાની ગોપનીયતાનું રક્ષણ
\item
  \textbf{ઇનપુટ}: પ્લેઇનટેક્સ્ટ + કી
\item
  \textbf{આઉટપુટ}: સાઇફરટેક્સ્ટ
\end{itemize}

\textbf{ડિક્રિપ્શન:}

\begin{itemize}
\tightlist
\item
  \textbf{વ્યાખ્યા}: સાઇફરટેક્સ્ટને પાછા પ્લેઇનટેક્સ્ટમાં કન્વર્ટ કરવાની પ્રક્રિયા
\item
  \textbf{હેતુ}: મૂળ ડેટા પુનઃપ્રાપ્ત કરવા
\item
  \textbf{ઇનપુટ}: સાઇફરટેક્સ્ટ + કી
\item
  \textbf{આઉટપુટ}: પ્લેઇનટેક્સ્ટ
\end{itemize}

\begin{center}
\textbf{Mermaid Diagram (Code)}
\begin{verbatim}
{Shaded}
{Highlighting}[]
graph LR
    A[Plaintext] {-{-}{} B[Encryption]}
    B {-{-}{} C[Ciphertext]}
    C {-{-}{} D[Decryption]}
    D {-{-}{} E[Plaintext]}
    F[Key] {-{-}{} B}
    G[Key] {-{-}{} D}
{Highlighting}
{Shaded}
\end{verbatim}
\end{center}

\textbf{ક્રિપ્ટોગ્રાફિક પ્રક્રિયા:}

\begin{enumerate}
\tightlist
\item
  \textbf{મોકલનાર}: કીનો ઉપયોગ કરીને મેસેજ એન્ક્રિપ્ટ કરે છે
\item
  \textbf{ટ્રાન્સમિશન}: નેટવર્ક પર સાઇફરટેક્સ્ટ મોકલે છે
\item
  \textbf{મેળવનાર}: કીનો ઉપયોગ કરીને સાઇફરટેક્સ્ટ ડિક્રિપ્ટ કરે છે
\item
  \textbf{રિકવરી}: મૂળ પ્લેઇનટેક્સ્ટ મેસેજ મેળવે છે
\end{enumerate}

\textbf{એન્ક્રિપ્શનના પ્રકારો:}

\begin{itemize}
\tightlist
\item
  \textbf{સ્ટ્રીમ સાઇફર}: એક સમયે એક બિટ/બાઇટ એન્ક્રિપ્ટ કરે છે
\item
  \textbf{બ્લોક સાઇફર}: નિર્ધારિત-સાઇઝના બ્લોક્સ એન્ક્રિપ્ટ કરે છે
\item
  \textbf{હેશ ફંક્શન}: વન-વે એન્ક્રિપ્શન (કોઈ ડિક્રિપ્શન નથી)
\end{itemize}

\textbf{એપ્લિકેશન્સ:}

\begin{itemize}
\tightlist
\item
  \textbf{ડેટા પ્રોટેક્શન}: સુરક્ષિત ફાઇલ સ્ટોરેજ
\item
  \textbf{કમ્યુનિકેશન}: સુરક્ષિત મેસેજિંગ
\item
  \textbf{ઓથેન્ટિકેશન}: ડિજિટલ સિગ્નેચર્સ
\item
  \textbf{પ્રાઇવસી}: વ્યક્તિગત માહિતીનું રક્ષણ
\end{itemize}

\textbf{સિક્યુરિટી આવશ્યકતાઓ:}

\begin{itemize}
\tightlist
\item
  \textbf{ગોપનીયતા}: માત્ર અધિકૃત યુઝર્સ જ વાંચી શકે
\item
  \textbf{અખંડિતા}: ડેટા સાથે છેડછાડ થઈ નથી
\item
  \textbf{ઓથેન્ટિકેશન}: મોકલનારની ઓળખ ચકાસવી
\item
  \textbf{નોન-રિપ્યુડિએશન}: મોકલનાર મોકલવાનો ઇનકાર કરી શકતો નથી
\end{itemize}

\end{solutionbox}
\begin{mnemonicbox}
``Encryption = Hide, Decryption = Reveal''

\end{mnemonicbox}
\subsection*{પ્રશ્ન 5(અ અથવા) [3
ગુણ]}\label{uxaaauxab0uxab6uxaa8-5uxa85-uxa85uxaa5uxab5-3-uxa97uxaa3}

\textbf{IMAP અને POP3 વચ્ચેનો તફાવત લખો.}

\begin{solutionbox}

{\def\LTcaptype{none} % do not increment counter
\begin{longtable}[]{@{}lll@{}}
\toprule\noalign{}
લક્ષણ & IMAP & POP3 \\
\midrule\noalign{}
\endhead
\bottomrule\noalign{}
\endlastfoot
\textbf{સ્ટોરેજ} & સર્વર-સાઇડ & ક્લાયન્ટ-સાઇડ \\
\textbf{એક્સેસ} & બહુવિધ ઉપકરણો & એક ઉપકરણ \\
\textbf{ઓફલાઇન} & મર્યાદિત & સંપૂર્ણ એક્સેસ \\
\end{longtable}
}

\textbf{IMAP (Internet Message Access Protocol):}

\begin{itemize}
\tightlist
\item
  \textbf{સર્વર સ્ટોરેજ}: મેસેજ સર્વર પર રહે છે
\item
  \textbf{મલ્ટિ-ડીવાઇસ}: બહુવિધ ઉપકરણોથી એક્સેસ
\item
  \textbf{સિન્ક્રોનાઇઝેશન}: ફેરફારો બધા ઉપકરણોમાં સિન્ક થાય છે
\end{itemize}

\textbf{POP3 (Post Office Protocol 3):}

\begin{itemize}
\tightlist
\item
  \textbf{ડાઉનલોડ}: મેસેજ ક્લાયન્ટ પર ડાઉનલોડ થાય છે
\item
  \textbf{સિંગલ ડીવાઇસ}: એક ઉપકરણ એક્સેસ માટે શ્રેષ્ઠ
\item
  \textbf{સ્ટોરેજ}: ક્લાયન્ટ મેસેજ સ્ટોરેજ મેનેજ કરે છે
\end{itemize}

\end{solutionbox}
\begin{mnemonicbox}
``IMAP = Internet Access, POP3 = Post Office''

\end{mnemonicbox}
\subsection*{પ્રશ્ન 5(બ અથવા) [4
ગુણ]}\label{uxaaauxab0uxab6uxaa8-5uxaac-uxa85uxaa5uxab5-4-uxa97uxaa3}

\textbf{સંક્ષિપ્તમાં Information Technology (સુધારા) અધિનિયમ, 2008 અને ભારતમાં
સાયબર કાયદાઓ પર તેની અસરનું વર્ણન કરો.}

\begin{solutionbox}

\textbf{IT અધિનિયમ 2008 મુખ્ય લક્ષણો:}

\begin{itemize}
\tightlist
\item
  \textbf{સાયબર ક્રાઇમ્સ}: વિવિધ સાયબર અપરાધોની વ્યાખ્યા
\item
  \textbf{ડેટા પ્રોટેક્શન}: પ્રાઇવસી અને સિક્યુરિટી આવશ્યકતાઓ
\item
  \textbf{ડિજિટલ સિગ્નેચર્સ}: ઈ-સિગ્નેચર્સની કાનૂની માન્યતા
\item
  \textbf{પેનલ્ટીઝ}: ઉલ્લંઘન માટે દંડ અને કેદ
\end{itemize}

\textbf{મુખ્ય સુધારાઓ:}

\begin{itemize}
\tightlist
\item
  \textbf{કલમ 66A}: આક્રામક મેસેજને ગુનાહિત બનાવ્યું (પછીથી રદ)
\item
  \textbf{કલમ 69}: માહિતી ઇન્ટરસેપ્ટ કરવાની સરકારી શક્તિ
\item
  \textbf{કલમ 72A}: વ્યક્તિગત માહિતી જાહેર કરવા માટે સજા
\item
  \textbf{કલમ 43A}: ડેટા બ્રીચ માટે વળતર
\end{itemize}

\textbf{સાયબર કાયદાઓ પર અસર:}

\begin{itemize}
\tightlist
\item
  \textbf{કાનૂની ફ્રેમવર્ક}: વ્યાપક સાયબર કાયદાનું માળખું
\item
  \textbf{બિઝનેસ કોમ્પ્લાયન્સ}: ડેટા સુરક્ષા આવશ્યકતાઓ
\item
  \textbf{વ્યક્તિગત અધિકારો}: પ્રાઇવસી પ્રોટેક્શન મેકેનિઝમ
\item
  \textbf{કાયદાનો અમલ}: સાયબર ક્રાઇમ્સની તપાસ માટે સાધનો
\end{itemize}

\end{solutionbox}
\begin{mnemonicbox}
``IT Act = Internet Technology Act''

\end{mnemonicbox}
\subsection*{પ્રશ્ન 5(ક અથવા) [7
ગુણ]}\label{uxaaauxab0uxab6uxaa8-5uxa95-uxa85uxaa5uxab5-7-uxa97uxaa3}

\textbf{સિમેટ્રિક અને એસિમેટ્રિક એન્ક્રિપ્શન અલ્ગોરિધમ્સ વચ્ચેનો તફાવત.}

\begin{solutionbox}

{\def\LTcaptype{none} % do not increment counter
\begin{longtable}[]{@{}
  >{\raggedright\arraybackslash}p{(\linewidth - 4\tabcolsep) * \real{0.1224}}
  >{\raggedright\arraybackslash}p{(\linewidth - 4\tabcolsep) * \real{0.4286}}
  >{\raggedright\arraybackslash}p{(\linewidth - 4\tabcolsep) * \real{0.4490}}@{}}
\toprule\noalign{}
\begin{minipage}[b]{\linewidth}\raggedright
પાસું
\end{minipage} & \begin{minipage}[b]{\linewidth}\raggedright
સિમેટ્રિક એન્ક્રિપ્શન
\end{minipage} & \begin{minipage}[b]{\linewidth}\raggedright
એસિમેટ્રિક એન્ક્રિપ્શન
\end{minipage} \\
\midrule\noalign{}
\endhead
\bottomrule\noalign{}
\endlastfoot
\textbf{કીનો ઉપયોગ} & એન્ક્રિપ્ટ/ડિક્રિપ્ટ માટે એક જ કી & વિવિધ કીઝ
(પબ્લિક/પ્રાઇવેટ) \\
\textbf{કી મેનેજમેન્ટ} & મુશ્કેલ કી ડિસ્ટ્રિબ્યુશન & સરળ કી ડિસ્ટ્રિબ્યુશન \\
\textbf{પર્ફોર્મન્સ} & ઝડપી પ્રોસેસિંગ & ધીમી પ્રોસેસિંગ \\
\textbf{કી લેન્થ} & ટૂંકી કીઝ (128-256 બિટ્સ) & લાંબી કીઝ (1024-4096 બિટ્સ) \\
\textbf{સ્કેલેબિલિટી} & નબળી (n^{2} કી પેર્સ જરૂરી) & સારી (n કી પેર્સ જરૂરી) \\
\textbf{ઉદાહરણો} & AES, DES, 3DES, Blowfish & RSA, ECC, DSA, ElGamal \\
\end{longtable}
}

\textbf{સિમેટ્રિક એન્ક્રિપ્શન વિગતો:}

\begin{itemize}
\tightlist
\item
  \textbf{અલ્ગોરિધમ પ્રકારો}: સ્ટ્રીમ સાઇફર્સ, બ્લોક સાઇફર્સ
\item
  \textbf{કી ડિસ્ટ્રિબ્યુશન પ્રોબ્લેમ}: કી એક્સચેન્જ માટે સુરક્ષિત ચેનલ જરૂરી
\item
  \textbf{એપ્લિકેશન્સ}: બલ્ક ડેટા એન્ક્રિપ્શન, VPNs, ફાઇલ એન્ક્રિપ્શન
\item
  \textbf{ફાયદાઓ}: ઝડપી, મોટા પ્રમાણમાં ડેટા માટે કાર્યક્ષમ
\item
  \textbf{ગેરફાયદાઓ}: કી મેનેજમેન્ટ જટિલતા, ડિજિટલ સિગ્નેચર્સ નથી
\end{itemize}

\textbf{એસિમેટ્રિક એન્ક્રિપ્શન વિગતો:}

\begin{itemize}
\tightlist
\item
  \textbf{પબ્લિક કી ઇન્ફ્રાસ્ટ્રક્ચર}: કી મેનેજમેન્ટ માટે PKI
\item
  \textbf{ડિજિટલ સિગ્નેચર્સ}: ઓથેન્ટિકેશન અને નોન-રિપ્યુડિએશન
\item
  \textbf{એપ્લિકેશન્સ}: ઈમેઇલ સિક્યુરિટી, SSL/TLS, ડિજિટલ સર્ટિફિકેટ્સ
\item
  \textbf{ફાયદાઓ}: સુરક્ષિત કી એક્સચેન્જ, ડિજિટલ સિગ્નેચર્સ
\item
  \textbf{ગેરફાયદાઓ}: કોમ્પ્યુટેશનલી ઇન્ટેન્સિવ, ધીમી પ્રોસેસિંગ
\end{itemize}

\textbf{હાઇબ્રિડ અપ્રોચ:}

\begin{itemize}
\tightlist
\item
  \textbf{બેસ્ટ ઓફ બોથ}: સિમેટ્રિક અને એસિમેટ્રિક એન્ક્રિપ્શનનું સંયોજન
\item
  \textbf{કી એક્સચેન્જ}: કી ડિસ્ટ્રિબ્યુશન માટે એસિમેટ્રિક
\item
  \textbf{ડેટા એન્ક્રિપ્શન}: વાસ્તવિક ડેટા માટે સિમેટ્રિક
\item
  \textbf{ઉદાહરણ}: SSL/TLS બંને પદ્ધતિનો ઉપયોગ કરે છે
\end{itemize}

\begin{center}
\textbf{Mermaid Diagram (Code)}
\begin{verbatim}
{Shaded}
{Highlighting}[]
graph TD
    A[Encryption Methods] {-{-}{} B[Symmetric]}
    A {-{-}{} C[Asymmetric]}
    B {-{-}{} D[Same Key]}
    B {-{-}{} E[Fast Processing]}
    C {-{-}{} F[Key Pair]}
    C {-{-}{} G[Slow Processing]}
{Highlighting}
{Shaded}
\end{verbatim}
\end{center}

\textbf{વાસ્તવિક-દુનિયાના એપ્લિકેશન્સ:}

\begin{itemize}
\tightlist
\item
  \textbf{બેંકિંગ}: ATM ટ્રાન્ઝેક્શન્સ સિમેટ્રિક એન્ક્રિપ્શનનો ઉપયોગ કરે છે
\item
  \textbf{ઈ-કોમર્સ}: HTTPS હાઇબ્રિડ એન્ક્રિપ્શનનો ઉપયોગ કરે છે
\item
  \textbf{ઈમેઇલ}: PGP કી એક્સચેન્જ માટે એસિમેટ્રિકનો ઉપયોગ કરે છે
\item
  \textbf{મોબાઇલ}: WhatsApp એન્ડ-ટુ-એન્ડ એન્ક્રિપ્શનનો ઉપયોગ કરે છે
\end{itemize}

\textbf{સિક્યુરિટી વિચારણાઓ:}

\begin{itemize}
\tightlist
\item
  \textbf{કી લેન્થ}: લાંબી કીઝ વધુ સારી સિક્યુરિટી પ્રદાન કરે છે
\item
  \textbf{અલ્ગોરિધમ મજબૂતતા}: સાબિત અલ્ગોરિધમ્સ પસંદ કરો
\item
  \textbf{ઇમ્પ્લિમેન્ટેશન}: યોગ્ય કોડિંગ વલ્નરેબિલિટીઝ અટકાવે છે
\item
  \textbf{કી સ્ટોરેજ}: સુરક્ષિત કી મેનેજમેન્ટ આવશ્યક
\end{itemize}

\textbf{પર્ફોર્મન્સ સરખામણી:}

{\def\LTcaptype{none} % do not increment counter
\begin{longtable}[]{@{}lll@{}}
\toprule\noalign{}
ઓપરેશન & સિમેટ્રિક (AES) & એસિમેટ્રિક (RSA) \\
\midrule\noalign{}
\endhead
\bottomrule\noalign{}
\endlastfoot
\textbf{એન્ક્રિપ્શન} & \textasciitilde1000 MB/s & \textasciitilde1 MB/s \\
\textbf{કી જનરેશન} & ઝડપી & ધીમી \\
\textbf{મેમરી ઉપયોગ} & ઓછો & વધુ \\
\textbf{CPU ઉપયોગ} & ઓછો & વધુ \\
\end{longtable}
}

\textbf{ભવિષ્યના ટ્રેન્ડ્સ:}

\begin{itemize}
\tightlist
\item
  \textbf{ક્વોન્ટમ કમ્પ્યુટિંગ}: વર્તમાન એસિમેટ્રિક અલ્ગોરિધમ્સ માટે ખતરો
\item
  \textbf{પોસ્ટ-ક્વોન્ટમ ક્રિપ્ટોગ્રાફી}: નવા અલ્ગોરિધમ્સ વિકસાવાઈ રહ્યા છે
\item
  \textbf{એલિપ્ટિક કર્વ}: વધુ કાર્યક્ષમ એસિમેટ્રિક એન્ક્રિપ્શન
\item
  \textbf{લાઇટવેઇટ ક્રિપ્ટોગ્રાફી}: IoT ઉપકરણો માટે
\end{itemize}

\end{solutionbox}
\begin{mnemonicbox}
``Symmetric = Same Speed, Asymmetric = Advanced
Security''

\end{mnemonicbox}

\end{document}
