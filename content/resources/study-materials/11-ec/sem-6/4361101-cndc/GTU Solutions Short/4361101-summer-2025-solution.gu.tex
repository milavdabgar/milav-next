\documentclass{article}

% content/resources/templates/preamble.tex
\usepackage[margin=0.6in]{geometry}
\author{Milav Dabgar}
\usepackage{amsmath,amssymb,amsthm}
\usepackage{booktabs}
\usepackage{multirow}
\usepackage{xcolor}
\usepackage{tcolorbox}
\tcbuselibrary{breakable,skins}
\usepackage[colorlinks=true,linkcolor=blue]{hyperref}
\usepackage{titlesec}
\usepackage{enumitem}
\usepackage{tikz}
\usepackage{pgfplots}
\usepackage{circuitikz}
\usepackage[version=4]{mhchem}
\usepackage{longtable}
\usepackage{array}
\usepackage{float}
\usepackage{caption}
\usepackage{listings}

\lstset{
  basicstyle=\small\ttfamily,
  breaklines=true,
  breakatwhitespace=false,
  postbreak=\mbox{\textcolor{red}{$\hookrightarrow$}\space},
  float=false,
  numbers=left,
  numberstyle=\tiny\color{gray},
  numbersep=10pt,
  xleftmargin=2em,
  keywordstyle=\color{blue},
  commentstyle=\color{green!60!black},
  stringstyle=\color{purple},
  backgroundcolor=\color{gray!5},
  showstringspaces=false,
  tabsize=2,
  captionpos=b,
  keepspaces=true,
  columns=flexible
}

\pgfplotsset{compat=1.18}
\usetikzlibrary{shapes,arrows,positioning,calc,patterns,decorations.pathmorphing,decorations.markings,arrows.meta}

% Color scheme
\definecolor{headcolor}{RGB}{0,102,204}
\definecolor{keycolor}{RGB}{220,20,60}
\definecolor{solutioncolor}{RGB}{34,139,34}
\definecolor{mnemoniccolor}{RGB}{148,0,211}
\definecolor{codecolor}{RGB}{0,0,100}

% Spacing
\setlength{\parskip}{3pt}
\setlist[itemize]{nosep}
\setlist[enumerate]{nosep}

% Title formatting
\titleformat{\section}{\Large\bfseries\color{headcolor}}{\thesection}{1em}{}
\titleformat{\subsection}{\large\bfseries\color{headcolor}}{\thesubsection}{1em}{}

% Pandoc tightlist compatibility
\providecommand{\tightlist}{%
  \setlength{\itemsep}{0pt}\setlength{\parskip}{0pt}}

% Pandoc longtable compatibility
\newcounter{none}
\def\thenone{}


% content/resources/templates/gujarati-boxes.tex
\usepackage{fontspec}
\usepackage{polyglossia}

% Set Gujarati as main language (document is primarily in Gujarati)
% Note: gloss-gujarati.ldf doesn't exist in polyglossia, but it will use hyphenation patterns
\setdefaultlanguage{gujarati}
\setotherlanguage{english}

% Configure Gujarati font properly
% Use Language=Default to prevent polyglossia from trying to add language-specific features
% that don't exist for Gujarati, which causes "empty feature" warnings
\newfontfamily\gujaratifont[Script=Gujarati,AutoFakeBold=2.5,AutoFakeSlant=0.3]{Noto Sans Gujarati}
\setmainfont[Script=Gujarati,AutoFakeBold=2.5,AutoFakeSlant=0.3]{Noto Sans Gujarati}
% Use Noto Sans Gujarati for monospace to support Gujarati in text
\setmonofont[Scale=0.9]{Noto Sans Gujarati}

% Configure English to use the same font
\newfontfamily\englishfont[Script=Gujarati,AutoFakeBold=2.5,AutoFakeSlant=0.3]{Noto Sans Gujarati}

% Translations for polyglossia
\gappto\captionsgujarati{
  \renewcommand{\tablename}{કોષ્ટક}
  \renewcommand{\figurename}{આકૃતિ}
}

% Helper for TikZ nodes to ensure Gujarati font
\newcommand{\gu}[1]{{\gujaratifont #1}}

% Custom environments
\newtcolorbox{solutionbox}{
    breakable,
    enhanced,
    colback=solutioncolor!5!white,
    colframe=solutioncolor!75!black,
    fonttitle=\bfseries,
    title=જવાબ
}

\newtcolorbox{solutionboxnobreak}{
 colback=solutioncolor!5!white,
 colframe=solutioncolor!75!black,
 fonttitle=\bfseries,
 title=જવાબ
}

\newtcolorbox{keyformula}{
 breakable,
 enhanced,
 colback=keycolor!5!white,
 colframe=keycolor!75!black,
 fonttitle=\bfseries,
 title=રાસાયણિક સમીકરણ/સૂત્ર
}

\newtcolorbox{mnemonicbox}{
 breakable,
 enhanced,
 colback=mnemoniccolor!5!white,
 colframe=mnemoniccolor!75!black,
 fonttitle=\bfseries,
 title=મેમરી ટ્રીક
}


% Custom commands for GTU solutions
% This file defines semantic commands for consistent formatting

% Question command with automatic formatting
\newcommand{\question}[2]{%
  \section*{Question #1}%
  \textbf{#2}%
}

% OR question variant
\newcommand{\questionor}[2]{%
  \section*{Question #1 OR}%
  \textbf{#2}%
}

% Proper table environment with caption
\newenvironment{answertable}[1]{%
  \begin{table}[htbp]
  \centering
  \caption{#1}
}{%
  \end{table}
}

% Proper figure environment for diagrams
\newenvironment{answerdiagram}[1]{%
  \begin{figure}[htbp]
  \centering
  \caption{#1}
}{%
  \end{figure}
}

% Semantic markup for key terms
\newcommand{\keyword}[1]{\textbf{#1}}
\newcommand{\code}[1]{\texttt{#1}}
\newcommand{\classname}[1]{\texttt{#1}}
\newcommand{\methodname}[1]{\texttt{#1}}

% Proper quotation marks
\newcommand{\mnemonic}[1]{``#1''}




\title{Computer Networks \& Data Communication (4361101) - Summer 2025 Solution}
\date{May 08, 2025}

\begin{document}
\maketitle

\questionmarks{1(અ)}{3}{વિવિધ DSL ટેકનોલોજી જણાવો અને ADSL પર ચર્ચા કરો}

\begin{solutionbox}

\textbf{DSL ટેકનોલોજીના પ્રકારો:}

\begin{table}[H]
\centering
\begin{tabulary}{\textwidth}{L L L}
\toprule
\textbf{DSL પ્રકાર} & \textbf{પૂરું નામ} & \textbf{સ્પીડ} \\
\midrule
\textbf{ADSL} & Asymmetric DSL & 1-8 Mbps \\
\textbf{SDSL} & Symmetric DSL & 768 Kbps \\
\textbf{VDSL} & Very high DSL & 52 Mbps \\
\textbf{HDSL} & High bit-rate DSL & 1.5 Mbps \\
\bottomrule
\end{tabulary}
\caption{DSL પ્રકારો}
\end{table}

\textbf{ADSL ની વિશેષતાઓ:}
\begin{itemize}
    \item \textbf{અસમપ્રમાણ}: અલગ upload/download સ્પીડ
    \item \textbf{Frequency Division}: હાલની તાંબાની ટેલિફોન લાઇનનો ઉપયોગ
    \item \textbf{Download સ્પીડ}: Upload કરતાં વધારે
\end{itemize}

\end{solutionbox}

\begin{mnemonicbox}
\mnemonic{ADSL ડાઉનલોડ ઝડપી}
\end{mnemonicbox}

\questionmarks{1(બ)}{4}{આર્કિટેક્ચરના આધારે નેટવર્ક વર્ગીકરણનું વર્ણન કરો.}

\begin{solutionbox}

\textbf{નેટવર્ક આર્કિટેક્ચર વર્ગીકરણ:}

\begin{table}[H]
\centering
\begin{tabulary}{\textwidth}{L L L}
\toprule
\textbf{આર્કિટેક્ચર} & \textbf{વર્ણન} & \textbf{વિશેષતાઓ} \\
\midrule
\textbf{Peer-to-Peer} & બધા nodes સમાન & કોઈ કેન્દ્રીય સર્વર નથી \\
\textbf{Client-Server} & કેન્દ્રીકૃત મોડેલ & સમર્પિત સર્વર \\
\bottomrule
\end{tabulary}
\caption{નેટવર્ક આર્કિટેક્ચર}
\end{table}

\textbf{Client-Server ફાયદાઓ:}
\begin{itemize}
    \item \textbf{કેન્દ્રીય નિયંત્રણ}: સરળ વ્યવસ્થાપન અને સુરક્ષા
    \item \textbf{સંસાધન શેરિંગ}: સંસાધનોનો કાર્યક્ષમ ઉપયોગ
    \item \textbf{સ્કેલેબિલિટી}: વધુ વપરાશકર્તાઓને સંભાળી શકે
    \item \textbf{ડેટા સુરક્ષા}: બેહતર બેકઅપ અને પુનઃપ્રાપ્તિ
\end{itemize}

\textbf{P2P લાક્ષણિકતાઓ:}
\begin{itemize}
    \item \textbf{વિકેન્દ્રીકૃત}: નિષ્ફળતાનો એક બિંદુ નથી
    \item \textbf{ખર્ચ અસરકારક}: સમર્પિત સર્વરની જરૂર નથી
\end{itemize}

\end{solutionbox}

\begin{mnemonicbox}
\mnemonic{Client સારી સેવા આપે}
\end{mnemonicbox}

\questionmarks{1(ક)}{7}{OSI મોડેલની આકૃતિ દોરો અને બધા સ્તરો સાથે વિગતવાર સમજાવો.}

\begin{solutionbox}

\begin{figure}[H]
\centering
\begin{tikzpicture}[node distance=0.8cm, auto]
    \node (app) [gtu block, minimum width=5cm] {7. Application Layer};
    \node (pres) [gtu block, below=of app, minimum width=5cm] {6. Presentation Layer};
    \node (sess) [gtu block, below=of pres, minimum width=5cm] {5. Session Layer};
    \node (trans) [gtu block, below=of sess, minimum width=5cm] {4. Transport Layer};
    \node (net) [gtu block, below=of trans, minimum width=5cm] {3. Network Layer};
    \node (dl) [gtu block, below=of net, minimum width=5cm] {2. Data Link Layer};
    \node (phy) [gtu block, below=of dl, minimum width=5cm] {1. Physical Layer};
    
    \draw[gtu arrow] (app) -- (pres);
    \draw[gtu arrow] (pres) -- (sess);
    \draw[gtu arrow] (sess) -- (trans);
    \draw[gtu arrow] (trans) -- (net);
    \draw[gtu arrow] (net) -- (dl);
    \draw[gtu arrow] (dl) -- (phy);
\end{tikzpicture}
\caption{OSI મોડેલ}
\end{figure}

\textbf{OSI સ્તરોના કાર્યો:}

\begin{table}[H]
\centering
\begin{tabulary}{\textwidth}{L L L}
\toprule
\textbf{સ્તર} & \textbf{કાર્ય} & \textbf{ઉદાહરણો} \\
\midrule
\textbf{Application} & વપરાશકર્તા ઇન્ટરફેસ & HTTP, FTP, SMTP \\
\textbf{Presentation} & ડેટા ફોર્મેટિંગ & Encryption, Compression \\
\textbf{Session} & Session વ્યવસ્થાપન & NetBIOS, RPC \\
\textbf{Transport} & End-to-end ડિલિવરી & TCP, UDP \\
\textbf{Network} & Routing & IP, ICMP \\
\textbf{Data Link} & Frame ડિલિવરી & Ethernet, PPP \\
\textbf{Physical} & Bit પ્રસારણ & Cables, Signals \\
\bottomrule
\end{tabulary}
\caption{OSI સ્તરો}
\end{table}

\textbf{મુખ્ય વિશેષતાઓ:}
\begin{itemize}
    \item \textbf{સ્તરબદ્ધ અભિગમ}: દરેક સ્તર ચોક્કસ કાર્ય કરે છે
    \item \textbf{માનકીકરણ}: સાર્વત્રિક સંચાર મોડેલ
    \item \textbf{સમસ્યા નિવારણ}: નેટવર્ક સમસ્યાઓ ઓળખવામાં સરળ
\end{itemize}

\end{solutionbox}

\begin{mnemonicbox}
\mnemonic{બધા લોકો ધંધો કરવા ડેટા પ્રોસેસિંગ કરે}
\end{mnemonicbox}

\questionmarks{1(ક OR)}{7}{TCP/IP protocol suite નો diagram દોરો અને Application Layer, Transport Layer અને Network Layer ના કાર્યો વિગતવાર સમજાવો.}

\begin{solutionbox}

\begin{figure}[H]
\centering
\begin{tikzpicture}[node distance=1cm, auto]
    \node (app) [gtu block, minimum width=6cm] {Application Layer\\(HTTP, FTP, SMTP, DNS)};
    \node (trans) [gtu block, below=of app, minimum width=6cm] {Transport Layer\\(TCP, UDP)};
    \node (net) [gtu block, below=of trans, minimum width=6cm] {Network Layer\\(IP, ICMP, ARP)};
    \node (link) [gtu block, below=of net, minimum width=6cm] {Data Link Layer\\(Ethernet, Wi-Fi)};
    
    \draw[gtu arrow] (app) -- (trans);
    \draw[gtu arrow] (trans) -- (net);
    \draw[gtu arrow] (net) -- (link);
\end{tikzpicture}
\caption{TCP/IP સ્યુટ}
\end{figure}

\textbf{સ્તરોના કાર્યો:}

\begin{table}[H]
\centering
\begin{tabulary}{\textwidth}{L L L}
\toprule
\textbf{સ્તર} & \textbf{મુખ્ય કાર્ય} & \textbf{Protocols} \\
\midrule
\textbf{Application} & વપરાશકર્તા સેવાઓ & HTTP, FTP, SMTP \\
\textbf{Transport} & End-to-end ડિલિવરી & TCP, UDP \\
\textbf{Network} & Routing packets & IP, ICMP \\
\bottomrule
\end{tabulary}
\caption{TCP/IP સ્તરો}
\end{table}

\textbf{Application Layer કાર્યો:}
\begin{itemize}
    \item \textbf{Web સેવાઓ}: વેબ બ્રાઉઝિંગ માટે HTTP
    \item \textbf{File Transfer}: ફાઇલ શેરિંગ માટે FTP
    \item \textbf{Email}: મેઇલ ડિલિવરી માટે SMTP
\end{itemize}

\textbf{Transport Layer કાર્યો:}
\begin{itemize}
    \item \textbf{વિશ્વસનીય ડિલિવરી}: TCP ડેટાની અખંડિતતા સુનિશ્ચિત કરે
    \item \textbf{અવિશ્વસનીય ડિલિવરી}: ઝડપી પ્રસારણ માટે UDP
    \item \textbf{Port Numbers}: ચોક્કસ applications ઓળખે
\end{itemize}

\textbf{Network Layer કાર્યો:}
\begin{itemize}
    \item \textbf{Logical Addressing}: ઉપકરણો માટે IP addresses
    \item \textbf{Routing}: packets માટે શ્રેષ્ઠ માર્ગ પસંદગી
    \item \textbf{Fragmentation}: મોટા packets તોડવા
\end{itemize}

\end{solutionbox}

\begin{mnemonicbox}
\mnemonic{Applications Transport Networks}
\end{mnemonicbox}

\questionmarks{2(અ)}{3}{WWW સમજાવો.}

\begin{solutionbox}

\textbf{World Wide Web (WWW):}

\begin{table}[H]
\centering
\begin{tabulary}{\textwidth}{L L}
\toprule
\textbf{ઘટક} & \textbf{વર્ણન} \\
\midrule
\textbf{Web Browser} & Client software (e.g., Chrome) \\
\textbf{Web Server} & વેબસાઇટ્સ host કરે (e.g., Apache) \\
\textbf{HTTP} & સંચાર protocol \\
\textbf{URL} & વેબ address \\
\bottomrule
\end{tabulary}
\caption{WWW ઘટકો}
\end{table}

\textbf{WWW વિશેષતાઓ:}
\begin{itemize}
    \item \textbf{Hypertext}: HTML વાપરીને linked documents
    \item \textbf{Client-Server Model}: Browser વિનંતી કરે, server જવાબ આપે
    \item \textbf{સાર્વત્રિક પ્રવેશ}: Platform independent
\end{itemize}

\end{solutionbox}

\begin{mnemonicbox}
\mnemonic{Web વિશ્વભર કામ કરે}
\end{mnemonicbox}

\questionmarks{2(બ)}{4}{FDDI અને CDDI સમજાવો.}

\begin{solutionbox}

\textbf{FDDI vs CDDI સરખામણી:}

\begin{table}[H]
\centering
\begin{tabulary}{\textwidth}{L L L}
\toprule
\textbf{વિશેષતા} & \textbf{FDDI} & \textbf{CDDI} \\
\midrule
\textbf{Medium} & Fiber optic & Copper wire \\
\textbf{સ્પીડ} & 100 Mbps & 100 Mbps \\
\textbf{અંતર} & 200 km & 100 meters \\
\textbf{ખર્ચ} & વધારે & ઓછો \\
\bottomrule
\end{tabulary}
\caption{FDDI vs CDDI}
\end{table}

\textbf{FDDI વિશેષતાઓ:}
\begin{itemize}
    \item \textbf{Dual Ring Topology}: Primary અને secondary rings
    \item \textbf{Token Passing}: Access control પદ્ધતિ
    \item \textbf{Fault Tolerance}: Self-healing ક્ષમતા
\end{itemize}

\textbf{CDDI વિશેષતાઓ:}
\begin{itemize}
    \item \textbf{Copper આધારિત}: Twisted pair cables વાપરે
    \item \textbf{ખર્ચ અસરકારક}: Fiber કરતાં સસ્તું
    \item \textbf{મર્યાદિત અંતર}: ટૂંકી પ્રસારણ રેન્જ
\end{itemize}

\end{solutionbox}

\begin{mnemonicbox}
\mnemonic{Fiber ઝડપી, Copper સસ્તું}
\end{mnemonicbox}

\questionmarks{2(ક)}{7}{OS, CLI, Administrative Functions, Interfaces ના કાર્યો સાથે નેટવર્ક મેનેજમેન્ટ સિસ્ટમનું વર્ણન કરો.}

\begin{solutionbox}

\begin{figure}[H]
\centering
\begin{tikzpicture}[node distance=1.5cm, auto]
    \node (nms) [gtu block, minimum width=4cm] {Network Management System};
    \node (os) [gtu block, below left=of nms, xshift=-1cm] {Operating System};
    \node (cli) [gtu block, below right=of nms, xshift=1cm] {CLI Interface};
    \node (admin) [gtu block, below=of os] {Administrative Functions};
    \node (gui) [gtu block, below=of cli] {GUI Interfaces};
    
    \draw[gtu arrow] (nms) -- (os);
    \draw[gtu arrow] (nms) -- (cli);
    \draw[gtu arrow] (nms) -- (admin);
    \draw[gtu arrow] (nms) -- (gui);
    
    \node [below=0.2cm of os] {\small Resource Management};
    \node [below=0.2cm of cli] {\small Command Line};
    \node [below=0.2cm of admin] {\small User Management};
    \node [below=0.2cm of gui] {\small Graphical Interface};
\end{tikzpicture}
\caption{Network Management System}
\end{figure}

\textbf{નેટવર્ક મેનેજમેન્ટ ઘટકો:}

\begin{table}[H]
\centering
\begin{tabulary}{\textwidth}{L L L}
\toprule
\textbf{ઘટક} & \textbf{કાર્ય} & \textbf{ઉદાહરણો} \\
\midrule
\textbf{OS કાર્યો} & સંસાધન વ્યવસ્થાપન & Process, memory, file management \\
\textbf{CLI} & Command interface & Terminal, console commands \\
\textbf{Admin કાર્યો} & સિસ્ટમ નિયંત્રણ & User accounts, security \\
\textbf{Interfaces} & વપરાશકર્તા ક્રિયાપ્રતિક્રિયા & GUI, web interface \\
\bottomrule
\end{tabulary}
\caption{System Functions}
\end{table}

\textbf{Operating System કાર્યો:}
\begin{itemize}
    \item \textbf{Process Management}: ચાલતી applications નિયંત્રણ
    \item \textbf{Memory Management}: સિસ્ટમ સંસાધનો ફાળવવા
    \item \textbf{File System}: ડેટા ગોઠવવા અને સંગ્રહ
\end{itemize}

\textbf{CLI કાર્યો:}
\begin{itemize}
    \item \textbf{સીધા Commands}: Text-based નિયંત્રણ
    \item \textbf{Scripting}: સ્વચાલિત કાર્ય અમલીકરણ
    \item \textbf{Remote Access}: SSH, Telnet connections
\end{itemize}

\textbf{Administrative કાર્યો:}
\begin{itemize}
    \item \textbf{User Management}: વપરાશકર્તા accounts બનાવવા, બદલવા
    \item \textbf{Security Policies}: Access control, permissions
    \item \textbf{System Monitoring}: કાર્યક્ષમતા ટ્રેકિંગ
\end{itemize}

\end{solutionbox}

\begin{mnemonicbox}
\mnemonic{OS CLI Admin Interfaces}
\end{mnemonicbox}

\questionmarks{2(અ OR)}{3}{Connection-oriented protocol અને connectionless protocol ની સરખામણી કરો.}

\begin{solutionbox}

\textbf{Protocol સરખામણી:}

\begin{table}[H]
\centering
\begin{tabulary}{\textwidth}{L L L}
\toprule
\textbf{વિશેષતા} & \textbf{Connection-Oriented} & \textbf{Connectionless} \\
\midrule
\textbf{Setup} & જરૂરી & જરૂરી નથી \\
\textbf{વિશ્વસનીયતા} & વધારે & ઓછી \\
\textbf{સ્પીડ} & ધીમી & ઝડપી \\
\textbf{ઉદાહરણ} & TCP & UDP \\
\bottomrule
\end{tabulary}
\caption{Connection vs Connectionless}
\end{table}

\textbf{Connection-Oriented વિશેષતાઓ:}
\begin{itemize}
    \item \textbf{Three-way Handshake}: ડેટા transfer પહેલાં connection સ્થાપિત કરે
    \item \textbf{વિશ્વસનીય ડિલિવરી}: Packet delivery અને order ની ખાતરી
\end{itemize}

\textbf{Connectionless વિશેષતાઓ:}
\begin{itemize}
    \item \textbf{કોઈ Setup નથી}: સીધું ડેટા પ્રસારણ
    \item \textbf{Best Effort}: ડિલિવરીની કોઈ ખાતરી નથી
\end{itemize}

\end{solutionbox}

\begin{mnemonicbox}
\mnemonic{TCP કનેક્ટ કરે, UDP ડિલિવર કરે}
\end{mnemonicbox}

\questionmarks{2(બ OR)}{4}{નેટવર્ક ડિવાઇસ Repeater સમજાવો.}

\begin{solutionbox}

\textbf{Repeater કાર્યો:}

\begin{table}[H]
\centering
\begin{tabulary}{\textwidth}{L L}
\toprule
\textbf{કાર્ય} & \textbf{વર્ણન} \\
\midrule
\textbf{Signal Amplification} & નબળા signals વધારે \\
\textbf{Range Extension} & નેટવર્ક અંતર વધારે \\
\textbf{Noise Reduction} & Signal ગુણવત્તા સાફ કરે \\
\bottomrule
\end{tabulary}
\caption{Repeater કાર્યો}
\end{table}

\begin{figure}[H]
\centering
\begin{tikzpicture}[node distance=2.5cm, auto]
    \node (weak) [gtu input] {Weak Signal};
    \node (rep) [gtu process, right=of weak] {REPEATER\\(Amplify)};
    \node (strong) [gtu input, right=of rep] {Strong Signal};
    
    \draw[gtu arrow] (weak) -- node[midway, above] {Noisy} (rep);
    \draw[gtu arrow] (rep) -- node[midway, above] {Clean} (strong);
\end{tikzpicture}
\caption{Repeater કામગીરી}
\end{figure}

\textbf{Repeater લાક્ષણિકતાઓ:}
\begin{itemize}
    \item \textbf{Physical Layer Device}: Layer 1 પર કામ કરે
    \item \textbf{Bit-by-Bit}: Digital signals પુનઃ ઉત્પન્ન કરે
    \item \textbf{કોઈ Intelligence નથી}: ડેટા filter અથવા route કરી શકતું નથી
\end{itemize}

\end{solutionbox}

\begin{mnemonicbox}
\mnemonic{Repeater Signals પુનઃ ઉત્પન્ન કરે}
\end{mnemonicbox}

\questionmarks{2(ક OR)}{7}{Router, Hub અને Switch વચ્ચેનો ભેદ આપો.}

\begin{solutionbox}

\textbf{નેટવર્ક ડિવાઇસ સરખામણી:}

\begin{table}[H]
\centering
\begin{tabulary}{\textwidth}{L L L L}
\toprule
\textbf{વિશેષતા} & \textbf{Hub} & \textbf{Switch} & \textbf{Router} \\
\midrule
\textbf{OSI Layer} & Physical (1) & Data Link (2) & Network (3) \\
\textbf{Collision Domain} & એક & અનેક & અનેક \\
\textbf{Broadcast Domain} & એક & એક & અનેક \\
\textbf{Intelligence} & કંઈ નથી & MAC શીખવું & IP routing \\
\textbf{Full Duplex} & ના & હા & હા \\
\bottomrule
\end{tabulary}
\caption{Hub vs Switch vs Router}
\end{table}

\begin{figure}[H]
\centering
\begin{tikzpicture}[node distance=1.5cm, auto]
    \node (dev) [gtu block, minimum width=4cm] {Network Devices};
    \node (hub) [gtu block, below left=of dev, xshift=-2cm] {Hub\\(Layer 1)};
    \node (sw) [gtu block, below=of dev] {Switch\\(Layer 2)};
    \node (rtr) [gtu block, below right=of dev, xshift=2cm] {Router\\(Layer 3)};
    
    \draw[gtu arrow] (dev) -- (hub);
    \draw[gtu arrow] (dev) -- (sw);
    \draw[gtu arrow] (dev) -- (rtr);
    
    \node [below=0.2cm of hub, align=center, font=\footnotesize] {Shared\\Bandwidth};
    \node [below=0.2cm of sw, align=center, font=\footnotesize] {Dedicated\\Bandwidth};
    \node [below=0.2cm of rtr, align=center, font=\footnotesize] {Inter-network\\Connection};
\end{tikzpicture}
\caption{નેટવર્ક ડિવાઇસ વર્ગીકરણ}
\end{figure}

\textbf{Hub લાક્ષણિકતાઓ:}
\begin{itemize}
    \item \textbf{Shared Medium}: બધા ports bandwidth શેર કરે
    \item \textbf{Half Duplex}: એક સાથે send અને receive કરી શકતું નથી
    \item \textbf{Collision Prone}: એક collision domain
\end{itemize}

\textbf{Switch લાક્ષણિકતાઓ:}
\begin{itemize}
    \item \textbf{MAC Address Table}: ઉપકરણોના સ્થાનો શીખે
    \item \textbf{Full Duplex}: એક સાથે send/receive
    \item \textbf{VLAN Support}: Virtual network segmentation
\end{itemize}

\textbf{Router લાક્ષણિકતાઓ:}
\begin{itemize}
    \item \textbf{IP Routing}: નેટવર્ક વચ્ચે packets forward કરે
    \item \textbf{Routing Table}: નેટવર્ક topology જાળવે
    \item \textbf{NAT Support}: Network Address Translation
\end{itemize}

\end{solutionbox}

\begin{mnemonicbox}
\mnemonic{Hub શેર કરે, Switch સ્વિચ કરે, Router રૂટ કરે}
\end{mnemonicbox}

\questionmarks{3(અ)}{3}{UTP, Coaxial અને Fiber optic cable નો સઘડ આકૃતિ દોરો}

\begin{solutionbox}

\begin{figure}[H]
\centering
\begin{center}
\textbf{1. UTP Cable}
\begin{tikzpicture}
    % Outer Jacket
    \draw[fill=gray!20] (0,0) rectangle (4,1);
    \node at (2, 0.5) {Plastic Jacket};
    % Twisted Pairs (Simulated)
    \draw[decorate, decoration={snake, amplitude=1mm, segment length=3mm}] (4, 0.7) -- (6, 0.7);
    \draw[decorate, decoration={snake, amplitude=1mm, segment length=3mm}] (4, 0.3) -- (6, 0.3);
    \node[right] at (6, 0.5) {Twisted Pairs};
\end{tikzpicture}

\vspace{0.5cm}
\textbf{2. Coaxial Cable}
\begin{tikzpicture}
    % Layers
    \draw[fill=black!80] (0,0) rectangle (1,1.5); % Jacket
    \draw[fill=gray!50] (1,0.2) rectangle (2.5,1.3); % Shield
    \draw[fill=white] (2.5,0.4) rectangle (4,1.1); % Dielectric
    \draw[fill=orange] (4,0.65) rectangle (5,0.85); % Conductor
    
    \node[below] at (0.5,0) {Jacket};
    \node[below] at (1.75,0.2) {Shield};
    \node[below] at (3.25,0.4) {Dielectric};
    \node[below] at (4.5,0.65) {Conductor};
\end{tikzpicture}

\vspace{0.5cm}
\textbf{3. Fiber Optic Cable}
\begin{tikzpicture}
    % Layers
    \draw[fill=blue!30] (0,0) rectangle (1,1.5); % Jacket
    \draw[fill=gray!30] (1,0.2) rectangle (2,1.3); % Strength
    \draw[fill=white] (2,0.4) rectangle (3.5,1.1); % Cladding
    \draw[fill=cyan] (3.5,0.7) rectangle (4.5,0.8); % Core
    
    \node[below] at (0.5,0) {Jacket};
    \node[below] at (1.5,0.2) {Strength};
    \node[below] at (2.75,0.4) {Cladding};
    \node[below] at (4,0.7) {Core};
\end{tikzpicture}
\end{center}
\caption{Transmission Media Cables}
\end{figure}

\textbf{Cable લાક્ષણિકતાઓ:}

\begin{table}[H]
\centering
\begin{tabulary}{\textwidth}{L L L}
\toprule
\textbf{Cable પ્રકાર} & \textbf{Core સામગ્રી} & \textbf{Bandwidth} \\
\midrule
\textbf{UTP} & Copper wire & 100 MHz \\
\textbf{Coaxial} & Copper conductor & 1 GHz \\
\textbf{Fiber Optic} & Glass/Plastic & ખૂબ વધારે \\
\bottomrule
\end{tabulary}
\caption{Cable સરખામણી}
\end{table}

\end{solutionbox}

\begin{mnemonicbox}
\mnemonic{વળેલું તાંબું કાચ}
\end{mnemonicbox}

\questionmarks{3(બ)}{4}{Circuit switching અને packet switching circuit વચ્ચેનો ભેદ આપો.}

\begin{solutionbox}

\textbf{Switching પદ્ધતિઓ સરખામણી:}

\begin{table}[H]
\centering
\begin{tabulary}{\textwidth}{L L L}
\toprule
\textbf{વિશેષતા} & \textbf{Circuit Switching} & \textbf{Packet Switching} \\
\midrule
\textbf{Path} & સમર્પિત & સહેજ \\
\textbf{Setup Time} & જરૂરી & જરૂરી નથી \\
\textbf{Bandwidth} & નિશ્ચિત & ચલાયમાન \\
\textbf{ઉદાહરણ} & ટેલિફોન & Internet \\
\bottomrule
\end{tabulary}
\caption{Circuit vs Packet Switching}
\end{table}

\textbf{Circuit Switching વિશેષતાઓ:}
\begin{itemize}
    \item \textbf{સમર્પિત Path}: સંચાર કરતા પક્ષો વચ્ચે વિશિષ્ટ કનેક્શન
    \item \textbf{સ્થિર Bandwidth}: સમગ્ર સંચાર દરમિયાન નિશ્ચિત ડેટા રેટ
    \item \textbf{Setup Phase}: ડેટા transfer પહેલાં connection સ્થાપિત
\end{itemize}

\textbf{Packet Switching વિશેષતાઓ:}
\begin{itemize}
    \item \textbf{Store and Forward}: મધ્યવર્તી nodes પર packets સંગ્રહ
    \item \textbf{Dynamic Routing}: વિવિધ packets માટે વિવિધ paths
    \item \textbf{Resource Sharing}: અનેક વપરાશકર્તાઓ નેટવર્ક સંસાધનો શેર કરે
\end{itemize}

\end{solutionbox}

\begin{mnemonicbox}
\mnemonic{Circuit કનેક્ટ કરે, Packet શેર કરે}
\end{mnemonicbox}

\questionmarks{3(ક)}{7}{Unguided media અને guided media સમજાવો.}

\begin{solutionbox}

\begin{figure}[H]
\centering
\begin{tikzpicture}[node distance=1cm, auto]
    \node (media) [gtu block] {Transmission Media};
    \node (guided) [gtu block, below left=of media, xshift=-2cm] {Guided Media};
    \node (unguided) [gtu block, below right=of media, xshift=2cm] {Unguided Media};
    
    \node (tp) [gtu block, below=of guided, xshift=-1.5cm, font=\footnotesize] {Twisted Pair};
    \node (coax) [gtu block, right=0.2cm of tp, font=\footnotesize] {Coaxial};
    \node (fiber) [gtu block, right=0.2cm of coax, font=\footnotesize] {Fiber Optic};
    
    \node (radio) [gtu block, below=of unguided, xshift=-1.5cm, font=\footnotesize] {Radio};
    \node (micro) [gtu block, right=0.2cm of radio, font=\footnotesize] {Microwave};
    \node (ir) [gtu block, right=0.2cm of micro, font=\footnotesize] {Infrared};
    
    \draw[gtu arrow] (media) -- (guided);
    \draw[gtu arrow] (media) -- (unguided);
    \draw[gtu arrow] (guided) -- (tp);
    \draw[gtu arrow] (guided) -- (coax);
    \draw[gtu arrow] (guided) -- (fiber);
    \draw[gtu arrow] (unguided) -- (radio);
    \draw[gtu arrow] (unguided) -- (micro);
    \draw[gtu arrow] (unguided) -- (ir);
\end{tikzpicture}
\caption{Transmission Media વર્ગીકરણ}
\end{figure}

\textbf{Guided Media લાક્ષણિકતાઓ:}

\begin{table}[H]
\centering
\begin{tabulary}{\textwidth}{L L L L}
\toprule
\textbf{પ્રકાર} & \textbf{સામગ્રી} & \textbf{અંતર} & \textbf{ખર્ચ} \\
\midrule
\textbf{Twisted Pair} & તાંબું & 100m & ઓછો \\
\textbf{Coaxial} & તાંબું + Shield & 500m & મધ્યમ \\
\textbf{Fiber Optic} & કાચ & 2km+ & વધારે \\
\bottomrule
\end{tabulary}
\caption{Guided Media}
\end{table}

\textbf{Unguided Media લાક્ષણિકતાઓ:}

\begin{table}[H]
\centering
\begin{tabulary}{\textwidth}{L L L L}
\toprule
\textbf{પ્રકાર} & \textbf{આવર્તન} & \textbf{રેન્જ} & \textbf{ઉપયોગ} \\
\midrule
\textbf{Radio Waves} & 3KHz-1GHz & લાંબી & AM/FM રેડિયો \\
\textbf{Microwaves} & 1GHz-300GHz & Line of sight & Satellite \\
\textbf{Infrared} & 300GHz-400THz & ટૂંકી & Remote control \\
\bottomrule
\end{tabulary}
\caption{Unguided Media}
\end{table}

\end{solutionbox}

\begin{mnemonicbox}
\mnemonic{Guided વાયર, Unguided હવા}
\end{mnemonicbox}

\questionmarks{3(અ OR)}{3}{Computer Networks માં ઉપયોગમાં લેવાતા વિવિધ connectors ની ચર્ચા કરો.}

\begin{solutionbox}

\textbf{નેટવર્ક Connectors:}

\begin{table}[H]
\centering
\begin{tabulary}{\textwidth}{L L L}
\toprule
\textbf{Connector} & \textbf{Cable પ્રકાર} & \textbf{ઉપયોગ} \\
\midrule
\textbf{RJ-45} & UTP/STP & Ethernet \\
\textbf{BNC} & Coaxial & Legacy networks \\
\textbf{SC/ST} & Fiber optic & High-speed networks \\
\bottomrule
\end{tabulary}
\caption{Connectors}
\end{table}

\end{solutionbox}

\begin{mnemonicbox}
\mnemonic{RJ BNC Fiber કનેક્ટ}
\end{mnemonicbox}

\questionmarks{3(બ OR)}{4}{ઉદાહરણો સાથે IP addressing scheme સમજાવો.}

\begin{solutionbox}

\textbf{IP Address Classes:}

\begin{table}[H]
\centering
\begin{tabulary}{\textwidth}{L L L L}
\toprule
\textbf{Class} & \textbf{Range} & \textbf{Default Mask} & \textbf{ઉદાહરણ} \\
\midrule
\textbf{A} & 1-126 & /8 & 10.0.0.1 \\
\textbf{B} & 128-191 & /16 & 172.16.0.1 \\
\textbf{C} & 192-223 & /24 & 192.168.1.1 \\
\bottomrule
\end{tabulary}
\caption{IP Classes}
\end{table}

\textbf{IP Address બંધારણ:}
\begin{itemize}
    \item \textbf{Network ભાગ}: નેટવર્ક ઓળખે
    \item \textbf{Host ભાગ}: ઉપકરણ ઓળખે
    \item \textbf{Subnet Mask}: નેટવર્ક અને host ભાગો અલગ કરે
\end{itemize}

\end{solutionbox}

\begin{mnemonicbox}
\mnemonic{એક મોટો Class નેટવર્ક}
\end{mnemonicbox}

\questionmarks{3(ક OR)}{7}{IPv4 અને IPv6 વચ્ચેનો ભેદ આપો.}

\begin{solutionbox}

\textbf{IPv4 vs IPv6 સરખામણી:}

\begin{table}[H]
\centering
\begin{tabulary}{\textwidth}{L L L}
\toprule
\textbf{વિશેષતા} & \textbf{IPv4} & \textbf{IPv6} \\
\midrule
\textbf{Address લંબાઈ} & 32 bits & 128 bits \\
\textbf{Address ફોર્મેટ} & દશાંશ & હેક્સાડેસિમલ \\
\textbf{Address સ્પેસ} & 4.3 બિલિયન & 340 undecillion \\
\textbf{Header સાઇઝ} & 20-60 bytes & 40 bytes \\
\textbf{Fragmentation} & Router/Host & ફક્ત Host \\
\textbf{સુરક્ષા} & વૈકલ્પિક & બિલ્ટ-ઇન \\
\bottomrule
\end{tabulary}
\caption{IPv4 vs IPv6}
\end{table}

\textbf{IPv4 લાક્ષણિકતાઓ:}
\begin{itemize}
    \item \textbf{Address ઉદાહરણ}: 192.168.1.1
    \item \textbf{Dotted Decimal}: ચાર octets dots વડે અલગ
    \item \textbf{Classes}: A, B, C, D, E addressing scheme
\end{itemize}

\textbf{IPv6 લાક્ષણિકતાઓ:}
\begin{itemize}
    \item \textbf{Address ઉદાહરણ}: 2001:0db8:85a3::8a2e:0370:7334
    \item \textbf{Colon Notation}: આઠ hexadecimal digits ના જૂથો
    \item \textbf{કોઈ Classes નથી}: Hierarchical addressing
\end{itemize}

\end{solutionbox}

\begin{mnemonicbox}
\mnemonic{IPv6 વધુ Addresses છે}
\end{mnemonicbox}

\questionmarks{4(અ)}{3}{File Transfer Protocol સમજાવો.}

\begin{solutionbox}

\textbf{FTP લાક્ષણિકતાઓ:}

\begin{table}[H]
\centering
\begin{tabulary}{\textwidth}{L L}
\toprule
\textbf{વિશેષતા} & \textbf{વર્ણન} \\
\midrule
\textbf{Port Numbers} & 20 (data), 21 (control) \\
\textbf{Protocol} & TCP-આધારિત \\
\textbf{Authentication} & Username/password \\
\bottomrule
\end{tabulary}
\caption{FTP Basics}
\end{table}

\textbf{FTP કામગીરી:}
\begin{itemize}
    \item \textbf{Upload}: Server પર ફાઇલો transfer કરવા PUT command
    \item \textbf{Download}: Server માંથી ફાઇલો retrieve કરવા GET command
    \item \textbf{Directory}: ફાઇલ listings બતાવવા LIST command
\end{itemize}

\end{solutionbox}

\begin{mnemonicbox}
\mnemonic{FTP ફાઇલો Transfer કરે}
\end{mnemonicbox}

\questionmarks{4(બ)}{4}{DNS પર નોંધ લખો.}

\begin{solutionbox}

\textbf{Domain Name System (DNS):}

\begin{table}[H]
\centering
\begin{tabulary}{\textwidth}{L L}
\toprule
\textbf{ઘટક} & \textbf{કાર્ય} \\
\midrule
\textbf{DNS Server} & Domain names resolve કરે \\
\textbf{DNS Cache} & તાજેતરના lookups સંગ્રહ કરે \\
\textbf{DNS Records} & Names ને addresses સાથે map કરે \\
\bottomrule
\end{tabulary}
\caption{DNS ઘટકો}
\end{table}

\textbf{DNS વંશવેલો:}
\begin{itemize}
    \item \textbf{Root Domain}: Top-level (.)
    \item \textbf{Top-Level Domain}: .com, .org, .net
    \item \textbf{Second-Level Domain}: google.com
\end{itemize}

\end{solutionbox}

\begin{mnemonicbox}
\mnemonic{DNS નામો Servers}
\end{mnemonicbox}

\questionmarks{4(ક)}{7}{Electronic Mail સમજાવો.}

\begin{solutionbox}

\begin{figure}[H]
\centering
\begin{tikzpicture}[node distance=1.5cm, auto]
    \node (sender) [gtu block] {Email Client};
    \node (smtp) [gtu block, right=of sender] {SMTP Server};
    \node (internet) [gtu block, right=of smtp] {Internet};
    \node (rec_smtp) [gtu block, right=of internet] {Recipient SMTP};
    \node (pop) [gtu block, below=of rec_smtp] {POP3/IMAP};
    \node (receiver) [gtu block, left=of pop] {Recipient Client};
    
    \draw[gtu arrow] (sender) -- (smtp);
    \draw[gtu arrow] (smtp) -- (internet);
    \draw[gtu arrow] (internet) -- (rec_smtp);
    \draw[gtu arrow] (rec_smtp) -- (pop);
    \draw[gtu arrow] (pop) -- (receiver);
\end{tikzpicture}
\caption{Email Delivery System}
\end{figure}

\textbf{Email સિસ્ટમ ઘટકો:}

\begin{table}[H]
\centering
\begin{tabulary}{\textwidth}{L L L}
\toprule
\textbf{ઘટક} & \textbf{કાર્ય} & \textbf{Protocol} \\
\midrule
\textbf{User Agent} & Email client & Outlook, Gmail \\
\textbf{Mail Server} & Store/forward & SMTP, POP3, IMAP \\
\textbf{Message Transfer} & Delivery & SMTP \\
\bottomrule
\end{tabulary}
\caption{Email ઘટકો}
\end{table}

\end{solutionbox}

\begin{mnemonicbox}
\mnemonic{SMTP મોકલે, POP3 લે, IMAP એકીકૃત કરે}
\end{mnemonicbox}

\questionmarks{4(અ OR)}{3}{Web browser સમજાવો.}

\begin{solutionbox}

\textbf{Web Browser કાર્યો:}

\begin{table}[H]
\centering
\begin{tabulary}{\textwidth}{L L}
\toprule
\textbf{કાર્ય} & \textbf{વર્ણન} \\
\midrule
\textbf{HTTP Client} & Web pages વિનંતી કરે \\
\textbf{HTML Renderer} & Web content પ્રદર્શિત કરે \\
\textbf{JavaScript Engine} & Scripts execute કરે \\
\bottomrule
\end{tabulary}
\caption{Browser કાર્યો}
\end{table}

\end{solutionbox}

\begin{mnemonicbox}
\mnemonic{Browser Web Render કરે}
\end{mnemonicbox}

\questionmarks{4(બ OR)}{4}{Mail Protocols સમજાવો.}

\begin{solutionbox}

\textbf{Email Protocol સરખામણી:}

\begin{table}[H]
\centering
\begin{tabulary}{\textwidth}{L L L L}
\toprule
\textbf{Protocol} & \textbf{પ્રકાર} & \textbf{કાર્ય} & \textbf{Port} \\
\midrule
\textbf{SMTP} & Outgoing & Mail મોકલવા & 25 \\
\textbf{POP3} & Incoming & Mail download કરવા & 110 \\
\textbf{IMAP} & Incoming & Mail sync કરવા & 143 \\
\bottomrule
\end{tabulary}
\caption{Mail Protocols}
\end{table}

\textbf{SMTP વિશેષતાઓ:}
\begin{itemize}
    \item \textbf{Push Protocol}: Sender transfer શરૂ કરે
    \item \textbf{Store and Forward}: મધ્યવર્તી mail servers
\end{itemize}

\end{solutionbox}

\begin{mnemonicbox}
\mnemonic{SMTP મોકલે, POP3 ખેંચે, IMAP એકીકૃત કરે}
\end{mnemonicbox}

\questionmarks{4(ક OR)}{7}{TCP અને UDP protocols નું વર્ણન કરો.}

\begin{solutionbox}

\textbf{TCP vs UDP સરખામણી:}

\begin{table}[H]
\centering
\begin{tabulary}{\textwidth}{L L L}
\toprule
\textbf{વિશેષતા} & \textbf{TCP} & \textbf{UDP} \\
\midrule
\textbf{Connection} & Connection-oriented & Connectionless \\
\textbf{વિશ્વસનીયતા} & વિશ્વસનીય & અવિશ્વસનીય \\
\textbf{સ્પીડ} & ધીમી & ઝડપી \\
\textbf{Header સાઇઝ} & 20 bytes & 8 bytes \\
\textbf{Flow Control} & હા & ના \\
\textbf{Error Control} & હા & ના \\
\bottomrule
\end{tabulary}
\caption{TCP vs UDP}
\end{table}

\begin{figure}[H]
\centering
\begin{tikzpicture}[node distance=1.5cm, auto]
    \node (trans) [gtu block] {Transport Layer};
    \node (tcp) [gtu block, below left=of trans, xshift=-1cm] {TCP (Reliable)};
    \node (udp) [gtu block, below right=of trans, xshift=1cm] {UDP (Fast)};
    
    \node (tcp_apps) [gtu block, below=of tcp] {Web, Email, FTP};
    \node (udp_apps) [gtu block, below=of udp] {DNS, Streaming};
    
    \draw[gtu arrow] (trans) -- (tcp);
    \draw[gtu arrow] (trans) -- (udp);
    \draw[gtu arrow] (tcp) -- (tcp_apps);
    \draw[gtu arrow] (udp) -- (udp_apps);
\end{tikzpicture}
\caption{Transport Protocols}
\end{figure}

\end{solutionbox}

\begin{mnemonicbox}
\mnemonic{TCP સાવચેતીથી પ્રયાસ કરે, UDP ડેટા છોડે}
\end{mnemonicbox}

\questionmarks{5(અ)}{3}{નેટવર્ક ડિવાઇસ Bridge નું વર્ણન કરો.}

\begin{solutionbox}

\textbf{Bridge લાક્ષણિકતાઓ:}

\begin{table}[H]
\centering
\begin{tabulary}{\textwidth}{L L}
\toprule
\textbf{વિશેષતા} & \textbf{વર્ણન} \\
\midrule
\textbf{OSI Layer} & Data Link (Layer 2) \\
\textbf{કાર્ય} & Collision domains segment કરે \\
\textbf{Learning} & MAC address table \\
\bottomrule
\end{tabulary}
\caption{Bridge Functions}
\end{table}

\textbf{Bridge કામગીરી:}
\begin{itemize}
    \item \textbf{Learning}: Frames માંથી MAC addresses record કરે
    \item \textbf{Filtering}: જરૂર હોય ત્યારે જ frames forward કરે
    \item \textbf{Forwarding}: યોગ્ય segment પર frames મોકલે
\end{itemize}

\end{solutionbox}

\begin{mnemonicbox}
\mnemonic{Bridge Collisions તોડે}
\end{mnemonicbox}

\questionmarks{5(બ)}{4}{સામાજિક મુદ્દાઓ અને Hacking સમજાવો તેની સાવચેતીઓની પણ ચર્ચા કરો.}

\begin{solutionbox}

\textbf{નેટવર્કમાં સામાજિક મુદ્દાઓ:}

\begin{table}[H]
\centering
\begin{tabulary}{\textwidth}{L L}
\toprule
\textbf{મુદ્દો} & \textbf{અસર} \\
\midrule
\textbf{Digital Divide} & ટેકનોલોજીની અસમાન પહોંચ \\
\textbf{Privacy ચિંતાઓ} & વ્યક્તિગત ડેટાનો દુરુપયોગ \\
\textbf{Cyberbullying} & ઓનલાઇન હેરાનગતિ \\
\bottomrule
\end{tabulary}
\caption{સામાજિક મુદ્દાઓ}
\end{table}

\textbf{Hacking પ્રકારો:}
\begin{itemize}
    \item \textbf{White Hat}: સુરક્ષા પરીક્ષણ માટે નૈતિક hacking
    \item \textbf{Black Hat}: ગેરકાયદે લાભ માટે દુષ્ટ hacking
    \item \textbf{Gray Hat}: નૈતિક અને દુષ્ટ વચ્ચે
\end{itemize}

\textbf{સાવચેતીઓ અને પગલાઓ:}
\begin{itemize}
    \item \textbf{મજબૂત Passwords}: જટિલ, અનોખા passwords વાપરો
    \item \textbf{Software Updates}: સિસ્ટમ patched રાખો
    \item \textbf{Firewall}: અનધિકૃત access block કરો
    \item \textbf{શિક્ષણ}: વપરાશકર્તા જાગૃતિ તાલીમ
\end{itemize}

\end{solutionbox}

\begin{mnemonicbox}
\mnemonic{સુરક્ષિત સિસ્ટમ સમાજ બચાવે}
\end{mnemonicbox}

\questionmarks{5(ક)}{7}{IP સુરક્ષાને વિગતવાર સમજાવો.}

\begin{solutionbox}

\begin{figure}[H]
\centering
\begin{tikzpicture}[node distance=1.5cm, auto]
    \node (ipsec) [gtu block, minimum width=3cm] {IP Security (IPSec)};
    \node (ah) [gtu block, below left=of ipsec, xshift=-0.5cm] {AH};
    \node (esp) [gtu block, below=of ipsec] {ESP};
    \node (sa) [gtu block, below right=of ipsec, xshift=0.5cm] {SA};
    
    \node (ah_desc) [gtu block, below=of ah, text width=2.5cm, font=\scriptsize] {Authentication \& Integrity};
    \node (esp_desc) [gtu block, below=of esp, text width=2.5cm, font=\scriptsize] {Confidentiality \& Integrity};
    \node (sa_desc) [gtu block, below=of sa, text width=2.5cm, font=\scriptsize] {Security Parameters};
    
    \draw[gtu arrow] (ipsec) -- (ah);
    \draw[gtu arrow] (ipsec) -- (esp);
    \draw[gtu arrow] (ipsec) -- (sa);
    \draw[gtu arrow] (ah) -- (ah_desc);
    \draw[gtu arrow] (esp) -- (esp_desc);
    \draw[gtu arrow] (sa) -- (sa_desc);
\end{tikzpicture}
\caption{IPSec Architecture}
\end{figure}

\textbf{IPSec ઘટકો:}

\begin{table}[H]
\centering
\begin{tabulary}{\textwidth}{L L L}
\toprule
\textbf{ઘટક} & \textbf{પૂરું નામ} & \textbf{સેવા} \\
\midrule
\textbf{AH} & Authentication Header & Integrity, Auth \\
\textbf{ESP} & Encapsulating Security Payload & Confidentiality, Integrity \\
\textbf{SA} & Security Association & Parameters \\
\bottomrule
\end{tabulary}
\caption{IPSec Components}
\end{table}

\textbf{IPSec Modes:}
\begin{itemize}
    \item \textbf{Transport}: ફક્ત payload સુરક્ષિત કરે (Host-to-Host)
    \item \textbf{Tunnel}: સંપૂર્ણ packet સુરક્ષિત કરે (Network-to-Network)
\end{itemize}

\textbf{સુરક્ષા સેવાઓ:} Authentication, Integrity, Confidentiality, Non-repudiation.

\end{solutionbox}

\begin{mnemonicbox}
\mnemonic{IPSec Authenticates, Encrypts, Secures}
\end{mnemonicbox}

\questionmarks{5(અ OR)}{3}{Wireless LAN સમજાવો.}

\begin{solutionbox}

\textbf{Wireless LAN (WLAN):}

\begin{table}[H]
\centering
\begin{tabulary}{\textwidth}{L L}
\toprule
\textbf{વિશેષતા} & \textbf{વર્ણન} \\
\midrule
\textbf{Standard} & IEEE 802.11 \\
\textbf{Frequency} & 2.4 GHz, 5 GHz \\
\textbf{Access Method} & CSMA/CA \\
\bottomrule
\end{tabulary}
\caption{WLAN Features}
\end{table}

\textbf{Standards:}
\begin{itemize}
    \item \textbf{802.11a/g}: 54 Mbps
    \item \textbf{802.11n}: 600 Mbps (MIMO)
    \item \textbf{ઘટકો}: Access Points, Clients, SSID
\end{itemize}

\end{solutionbox}

\begin{mnemonicbox}
\mnemonic{Wireless તરંગો કામ કરે}
\end{mnemonicbox}

\questionmarks{5(બ OR)}{4}{Symmetric અને asymmetric encryption algorithms વચ્ચેનો ભેદ આપો}

\begin{solutionbox}

\textbf{Encryption સરખામણી:}

\begin{table}[H]
\centering
\begin{tabulary}{\textwidth}{L L L}
\toprule
\textbf{વિશેષતા} & \textbf{Symmetric} & \textbf{Asymmetric} \\
\midrule
\textbf{Keys} & એક shared key & Public/Private pair \\
\textbf{સ્પીડ} & ઝડપી & ધીમી \\
\textbf{Key Dist.} & મુશ્કેલ & સરળ \\
\textbf{ઉદાહરણ} & AES, DES & RSA, ECC \\
\bottomrule
\end{tabulary}
\caption{Symmetric vs Asymmetric}
\end{table}

\textbf{Symmetric Encryption:}
\begin{itemize}
    \item સમાન key encryption/decryption માટે વપરાય
    \item મોટા ડેટા માટે કાર્યક્ષમ
\end{itemize}

\textbf{Asymmetric Encryption:}
\begin{itemize}
    \item Public key encrypt કરે, Private key decrypt કરે
    \item Digital signatures ને સપોર્ટ કરે
\end{itemize}

\end{solutionbox}

\begin{mnemonicbox}
\mnemonic{Symmetric સમાન, Asymmetric જોડી}
\end{mnemonicbox}

\questionmarks{5(ક OR)}{7}{Information Technology (Amendment) Act, 2008 નું સંક્ષિપ્ત વર્ણન કરો અને ભારતમાં cyber laws પર તેની અસર સમજાવો.}

\begin{solutionbox}

\textbf{IT Act 2008 વિહંગાવલોકન:}

\begin{table}[H]
\centering
\begin{tabulary}{\textwidth}{L L L}
\toprule
\textbf{કલમ} & \textbf{અપરાધ} & \textbf{દંડ} \\
\midrule
\textbf{66} & Hacking & 3 વર્ષ \\
\textbf{66A} & અપમાનજનક સંદેશા & 3 વર્ષ + દંડ \\
\textbf{66C} & ઓળખ ચોરી & 3 વર્ષ + દંડ \\
\bottomrule
\end{tabulary}
\caption{IT Act Penalties}
\end{table}

\begin{figure}[H]
\centering
\begin{tikzpicture}[node distance=1.5cm, auto]
    \node (act) [gtu block] {IT Act 2008};
    \node (crimes) [gtu block, below left=of act] {Cyber Crimes};
    \node (protect) [gtu block, below=of act] {Data Protection};
    \node (signs) [gtu block, below right=of act] {Digital Signatures};
    
    \draw[gtu arrow] (act) -- (crimes);
    \draw[gtu arrow] (act) -- (protect);
    \draw[gtu arrow] (act) -- (signs);
    
    \node [below=0.2cm of crimes, font=\scriptsize] {Hacking, Fraud};
    \node [below=0.2cm of protect, font=\footnotesize] {Privacy, Security};
    \node [below=0.2cm of signs, font=\footnotesize] {Legal Validity};
\end{tikzpicture}
\caption{IT Act Framework}
\end{figure}

\textbf{મુખ્ય સુધારાઓ \& અસર:}
\begin{itemize}
    \item \textbf{Cyber Terrorism}: કલમ 66F હેઠળ રજૂ કરાયું
    \item \textbf{Data Protection}: Corporates માટે ફરજિયાત સુરક્ષા પ્રથાઓ
    \item \textbf{Digital Signatures}: કાનૂની માન્યતા વિસ્તારી
    \item \textbf{Certifying Authorities}: Controllers ની નિમણૂક
\end{itemize}

\textbf{ઉદ્યોગ પર અસર:}
\begin{itemize}
    \item કંપનીઓ માટે Compliance આવશ્યકતાઓ
    \item E-commerce માટે કાનૂની ફ્રેમવર્ક
    \item Intermediaries માટે જવાબદારી (કલમ 79)
\end{itemize}

\end{solutionbox}

\begin{mnemonicbox}
\mnemonic{IT Act Digital India બચાવે}
\end{mnemonicbox}

\end{document}
