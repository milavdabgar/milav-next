\documentclass{article}

% content/resources/templates/preamble.tex
\usepackage[margin=0.6in]{geometry}
\author{Milav Dabgar}
\usepackage{amsmath,amssymb,amsthm}
\usepackage{booktabs}
\usepackage{multirow}
\usepackage{xcolor}
\usepackage{tcolorbox}
\tcbuselibrary{breakable,skins}
\usepackage[colorlinks=true,linkcolor=blue]{hyperref}
\usepackage{titlesec}
\usepackage{enumitem}
\usepackage{tikz}
\usepackage{pgfplots}
\usepackage{circuitikz}
\usepackage[version=4]{mhchem}
\usepackage{longtable}
\usepackage{array}
\usepackage{float}
\usepackage{caption}
\usepackage{listings}

\lstset{
  basicstyle=\small\ttfamily,
  breaklines=true,
  breakatwhitespace=false,
  postbreak=\mbox{\textcolor{red}{$\hookrightarrow$}\space},
  float=false,
  numbers=left,
  numberstyle=\tiny\color{gray},
  numbersep=10pt,
  xleftmargin=2em,
  keywordstyle=\color{blue},
  commentstyle=\color{green!60!black},
  stringstyle=\color{purple},
  backgroundcolor=\color{gray!5},
  showstringspaces=false,
  tabsize=2,
  captionpos=b,
  keepspaces=true,
  columns=flexible
}

\pgfplotsset{compat=1.18}
\usetikzlibrary{shapes,arrows,positioning,calc,patterns,decorations.pathmorphing,decorations.markings,arrows.meta}

% Color scheme
\definecolor{headcolor}{RGB}{0,102,204}
\definecolor{keycolor}{RGB}{220,20,60}
\definecolor{solutioncolor}{RGB}{34,139,34}
\definecolor{mnemoniccolor}{RGB}{148,0,211}
\definecolor{codecolor}{RGB}{0,0,100}

% Spacing
\setlength{\parskip}{3pt}
\setlist[itemize]{nosep}
\setlist[enumerate]{nosep}

% Title formatting
\titleformat{\section}{\Large\bfseries\color{headcolor}}{\thesection}{1em}{}
\titleformat{\subsection}{\large\bfseries\color{headcolor}}{\thesubsection}{1em}{}

% Pandoc tightlist compatibility
\providecommand{\tightlist}{%
  \setlength{\itemsep}{0pt}\setlength{\parskip}{0pt}}

% Pandoc longtable compatibility
\newcounter{none}
\def\thenone{}


% content/resources/templates/gujarati-boxes.tex
\usepackage{fontspec}
\usepackage{polyglossia}

% Set Gujarati as main language (document is primarily in Gujarati)
% Note: gloss-gujarati.ldf doesn't exist in polyglossia, but it will use hyphenation patterns
\setdefaultlanguage{gujarati}
\setotherlanguage{english}

% Configure Gujarati font properly
% Use Language=Default to prevent polyglossia from trying to add language-specific features
% that don't exist for Gujarati, which causes "empty feature" warnings
\newfontfamily\gujaratifont[Script=Gujarati,AutoFakeBold=2.5,AutoFakeSlant=0.3]{Noto Sans Gujarati}
\setmainfont[Script=Gujarati,AutoFakeBold=2.5,AutoFakeSlant=0.3]{Noto Sans Gujarati}
% Use Noto Sans Gujarati for monospace to support Gujarati in text
\setmonofont[Scale=0.9]{Noto Sans Gujarati}

% Configure English to use the same font
\newfontfamily\englishfont[Script=Gujarati,AutoFakeBold=2.5,AutoFakeSlant=0.3]{Noto Sans Gujarati}

% Translations for polyglossia
\gappto\captionsgujarati{
  \renewcommand{\tablename}{કોષ્ટક}
  \renewcommand{\figurename}{આકૃતિ}
}

% Helper for TikZ nodes to ensure Gujarati font
\newcommand{\gu}[1]{{\gujaratifont #1}}

% Custom environments
\newtcolorbox{solutionbox}{
    breakable,
    enhanced,
    colback=solutioncolor!5!white,
    colframe=solutioncolor!75!black,
    fonttitle=\bfseries,
    title=જવાબ
}

\newtcolorbox{solutionboxnobreak}{
 colback=solutioncolor!5!white,
 colframe=solutioncolor!75!black,
 fonttitle=\bfseries,
 title=જવાબ
}

\newtcolorbox{keyformula}{
 breakable,
 enhanced,
 colback=keycolor!5!white,
 colframe=keycolor!75!black,
 fonttitle=\bfseries,
 title=રાસાયણિક સમીકરણ/સૂત્ર
}

\newtcolorbox{mnemonicbox}{
 breakable,
 enhanced,
 colback=mnemoniccolor!5!white,
 colframe=mnemoniccolor!75!black,
 fonttitle=\bfseries,
 title=મેમરી ટ્રીક
}


% Custom commands for GTU solutions
% This file defines semantic commands for consistent formatting

% Question command with automatic formatting
\newcommand{\question}[2]{%
  \section*{Question #1}%
  \textbf{#2}%
}

% OR question variant
\newcommand{\questionor}[2]{%
  \section*{Question #1 OR}%
  \textbf{#2}%
}

% Proper table environment with caption
\newenvironment{answertable}[1]{%
  \begin{table}[htbp]
  \centering
  \caption{#1}
}{%
  \end{table}
}

% Proper figure environment for diagrams
\newenvironment{answerdiagram}[1]{%
  \begin{figure}[htbp]
  \centering
  \caption{#1}
}{%
  \end{figure}
}

% Semantic markup for key terms
\newcommand{\keyword}[1]{\textbf{#1}}
\newcommand{\code}[1]{\texttt{#1}}
\newcommand{\classname}[1]{\texttt{#1}}
\newcommand{\methodname}[1]{\texttt{#1}}

% Proper quotation marks
\newcommand{\mnemonic}[1]{``#1''}


\title{કમ્પ્યુટર નેટવર્કસ અને ડેટા કમ્યુનિકેશન (4361101) - ઉનાળો 2024 સોલ્યુશન}
\date{May 14, 2024}

\begin{document}
\maketitle


\questionmarks{1(અ)}{3}{વિવિધ નેટવર્ક ટોપોલોજીની યાદી બનાવો અને કોઈપણ એકની વિગતવાર ચર્ચા કરો.}

\begin{solutionbox}

\begin{table}[H]
\centering
\begin{tabulary}{\textwidth}{CL}
\toprule
\textbf{ટોપોલોજી} & \textbf{વર્ણન} \\
\midrule
\textbf{સ્ટાર} & બધા ઉપકરણો કેન્દ્રીય હબ/સ્વિચ સાથે જોડાયેલા \\
\textbf{રિંગ} & ઉપકરણો ગોળાકાર ફેશનમાં જોડાયેલા \\
\textbf{બસ} & બધા ઉપકરણો એક જ કેબલ સાથે જોડાયેલા \\
\textbf{મેશ} & દરેક ઉપકરણ બીજા દરેક ઉપકરણ સાથે જોડાયેલું \\
\textbf{ટ્રી} & રૂટ નોડ સાથે વંશવેલો માળખું \\
\textbf{હાઇબ્રિડ} & બે અથવા વધુ ટોપોલોજીનું સંયોજન \\
\bottomrule
\end{tabulary}
\caption{નેટવર્ક ટોપોલોજી}
\end{table}

\textbf{સ્ટાર ટોપોલોજી વિગતો:}

\begin{itemize}
    \item \textbf{કેન્દ્રીય હબ}: બધા નોડ્સ એક કેન્દ્રીય ઉપકરણ સાથે જોડાય છે
    \item \textbf{પોઇન્ટ-ટુ-પોઇન્ટ}: દરેક કનેક્શન નોડ અને હબ વચ્ચે સમર્પિત છે
    \item \textbf{સરળ મેનેજમેન્ટ}: ઇન્સ્ટોલ અને ટ્રબલશૂટ કરવું સરળ
\end{itemize}

\end{solutionbox}

\begin{mnemonicbox}
\mnemonic{STAR = Single Terminal All Reach}
\end{mnemonicbox}

\questionmarks{1(બ)}{4}{આધુનિક સંચાર પ્રણાલીઓમાં પોઇન્ટ-ટુ-પોઇન્ટ અને બ્રોડકાસ્ટ ટ્રાન્સમિશન ટેકનોલોજીનો ઉપયોગ કેવી રીતે થાય છે તે ઉદાહરણો સાથે સમજાવો. અને તેમના ફાયદા અને મર્યાદાઓની ચર્ચા કરો.}

\begin{solutionbox}

\begin{table}[H]
\centering
\begin{tabulary}{\textwidth}{L L L}
\toprule
\textbf{ટેકનોલોજી} & \textbf{પોઇન્ટ-ટુ-પોઇન્ટ} & \textbf{બ્રોડકાસ્ટ} \\
\midrule
\textbf{કનેક્શન} & બે ઉપકરણો વચ્ચે સીધી લિંક & એક-થી-અનેક સંદેશાવ્યવહાર \\
\textbf{ઉદાહરણ} & ટેલિફોન, VPN ટનલ્સ & રેડિયો, TV, WiFi \\
\textbf{ડેટા ફ્લો} & દ્વિદિશાત્મક & એકદિશાત્મક/બહુદિશાત્મક \\
\bottomrule
\end{tabulary}
\caption{ટ્રાન્સમિશન ટેકનોલોજી સરખામણી}
\end{table}

\textbf{પોઇન્ટ-ટુ-પોઇન્ટ એપ્લિકેશન્સ:}
\begin{itemize}
    \item \textbf{સમર્પિત લાઇન્સ}: ઓફિસો વચ્ચે લીઝ્ડ લાઇન્સ
    \item \textbf{સેટેલાઇટ લિંક્સ}: ગ્રાઉન્ડ સ્ટેશનથી સેટેલાઇટ સંદેશાવ્યવહાર
    \item \textbf{કેબલ મોડેમ્સ}: ઘરથી ISP કનેક્શન
\end{itemize}

\textbf{બ્રોડકાસ્ટ એપ્લિકેશન્સ:}
\begin{itemize}
    \item \textbf{WiFi નેટવર્કસ}: રાઉટર બહુવિધ ઉપકરણોને બ્રોડકાસ્ટ કરે છે
    \item \textbf{ટેલિવિઝન}: એક ટ્રાન્સમિટરથી અનેક રિસીવર્સ
\end{itemize}

\end{solutionbox}

\begin{mnemonicbox}
\mnemonic{P2P = Private Path, Broadcast = Big Audience}
\end{mnemonicbox}

\questionmarks{1(ક)}{7}{દરેક લેયરના કાર્ય સાથે OSI મોડેલનું વર્ણન કરો.}

\begin{solutionbox}

\begin{table}[H]
\centering
\begin{tabulary}{\textwidth}{C L L}
\toprule
\textbf{લેયર} & \textbf{નામ} & \textbf{કાર્ય} \\
\midrule
\textbf{7} & એપ્લિકેશન & યુઝર ઇન્ટરફેસ, નેટવર્ક સેવાઓ \\
\textbf{6} & પ્રેઝન્ટેશન & ડેટા એન્ક્રિપ્શન, કોમ્પ્રેશન, ફોર્મેટિંગ \\
\textbf{5} & સેશન & સેશન સ્થાપિત કરે, મેનેજ કરે, સમાપ્ત કરે \\
\textbf{4} & ટ્રાન્સપોર્ટ & વિશ્વસનીય ડેટા ટ્રાન્સફર, એરર કરેક્શન \\
\textbf{3} & નેટવર્ક & રાઉટિંગ, લોજિકલ એડ્રેસિંગ (IP) \\
\textbf{2} & ડેટા લિંક & ફ્રેમ ફોર્મેટિંગ, એરર ડિટેક્શન \\
\textbf{1} & ફિઝિકલ & બિટ ટ્રાન્સમિશન, ઇલેક્ટ્રિકલ સિગ્નલ્સ \\
\bottomrule
\end{tabulary}
\caption{OSI મોડેલ લેયર્સ}
\end{table}

\begin{figure}[H]
\centering
\begin{tikzpicture}[node distance=0.8cm]
    \node (app) [gtu block, minimum width=5cm] {7. Application Layer};
    \node (pres) [gtu block, minimum width=5cm, below=of app] {6. Presentation Layer};
    \node (sess) [gtu block, minimum width=5cm, below=of pres] {5. Session Layer};
    \node (trans) [gtu block, minimum width=5cm, below=of sess] {4. Transport Layer};
    \node (net) [gtu block, minimum width=5cm, below=of trans] {3. Network Layer};
    \node (dl) [gtu block, minimum width=5cm, below=of net] {2. Data Link Layer};
    \node (phy) [gtu block, minimum width=5cm, below=of dl] {1. Physical Layer};
    
    \draw[gtu arrow] (app) -- (pres);
    \draw[gtu arrow] (pres) -- (sess);
    \draw[gtu arrow] (sess) -- (trans);
    \draw[gtu arrow] (trans) -- (net);
    \draw[gtu arrow] (net) -- (dl);
    \draw[gtu arrow] (dl) -- (phy);
    
    \node [right=of app, xshift=1cm] (upper) {Upper Layers (Software)};
    \node [right=of phy, xshift=1cm] (lower) {Lower Layers (Hardware)};
    
    \draw[dashed] (app.east) -- (upper.west);
    \draw[dashed] (phy.east) -- (lower.west);
\end{tikzpicture}
\caption{OSI મોડેલ સ્ટેક}
\end{figure}

\textbf{મુખ્ય કાર્યો:}
\begin{itemize}
    \item \textbf{ઉપરના લેયર્સ (5-7)}: એપ્લિકેશન-સંબંધિત સેવાઓ સંભાળે છે
    \item \textbf{નીચેના લેયર્સ (1-4)}: ડેટા ટ્રાન્સમિશન અને રાઉટિંગ સંભાળે છે
    \item \textbf{એન્કેપ્સુલેશન}: દરેક લેયર પોતાનું હેડર ઉમેરે છે
\end{itemize}

\end{solutionbox}

\begin{mnemonicbox}
\mnemonic{All People Seem To Need Data Processing}
\end{mnemonicbox}

\questionmarks{1(ક OR)}{7}{TCP/IP મોડેલના દરેક લેયરના કાર્ય સાથે વર્ણન લખો.}

\begin{solutionbox}

\begin{table}[H]
\centering
\begin{tabulary}{\textwidth}{C L L L}
\toprule
\textbf{લેયર} & \textbf{નામ} & \textbf{કાર્ય} & \textbf{પ્રોટોકોલ્સ} \\
\midrule
\textbf{4} & એપ્લિકેશન & યુઝર સેવાઓ, એપ્લિકેશન્સ & HTTP, FTP, SMTP, DNS \\
\textbf{3} & ટ્રાન્સપોર્ટ & એન્ડ-ટુ-એન્ડ કમ્યુનિકેશન & TCP, UDP \\
\textbf{2} & ઇન્ટરનેટ & રાઉટિંગ, લોજિકલ એડ્રેસિંગ & IP, ICMP, ARP \\
\textbf{1} & નેટવર્ક એક્સેસ & ફિઝિકલ ટ્રાન્સમિશન & Ethernet, WiFi \\
\bottomrule
\end{tabulary}
\caption{TCP/IP મોડેલ લેયર્સ}
\end{table}

\begin{figure}[H]
\centering
\begin{tikzpicture}[node distance=1cm]
    \node (app) [gtu block, fill=purple!10, minimum width=6cm] {Application Layer};
    \node (trans) [gtu block, fill=blue!10, minimum width=6cm, below=of app] {Transport Layer};
    \node (net) [gtu block, fill=green!10, minimum width=6cm, below=of trans] {Internet Layer};
    \node (link) [gtu block, fill=orange!10, minimum width=6cm, below=of net] {Network Access Layer};

    \draw[gtu arrow] (app) -- (trans);
    \draw[gtu arrow] (trans) -- (net);
    \draw[gtu arrow] (net) -- (link);
\end{tikzpicture}
\caption{TCP/IP મોડેલ સ્ટેક}
\end{figure}

\textbf{લેયર કાર્યો:}
\begin{itemize}
    \item \textbf{એપ્લિકેશન}: એપ્લિકેશન્સને નેટવર્ક સેવાઓ પ્રદાન કરે છે
    \item \textbf{ટ્રાન્સપોર્ટ}: વિશ્વસનીય અથવા અવિશ્વસનીય ડિલિવરી સુનિશ્ચિત કરે છે
    \item \textbf{ઇન્ટરનેટ}: IP એડ્રેસનો ઉપયોગ કરીને નેટવર્કમાં પેકેટ્સ રાઉટ કરે છે
    \item \textbf{નેટવર્ક એક્સેસ}: ફિઝિકલ ટ્રાન્સમિશન મીડિયા સંભાળે છે
\end{itemize}

\end{solutionbox}

\begin{mnemonicbox}
\mnemonic{Applications Transport Internet Networks}
\end{mnemonicbox}



\questionmarks{2(અ)}{3}{ફાયરવોલ એટલે શું? તેના કાર્યો સમજાવો.}

\begin{solutionbox}

\textbf{ફાયરવોલ:}
\begin{itemize}
    \item \textbf{વ્યાખ્યા}: નેટવર્ક સુરક્ષા સિસ્ટમ જે ઇનકમિંગ અને આઉટગોઇંગ નેટવર્ક ટ્રાફિકનું નિરીક્ષણ અને નિયંત્રણ કરે છે.
    \item \textbf{ઉદ્દેશ્ય}: અનાધિકૃત એક્સેસ અને સાયબર હુમલાઓ અટકાવવા.
\end{itemize}

\textbf{કાર્યો:}
\begin{itemize}
    \item \textbf{પેકેટ ફિલ્ટરિંગ}: સુરક્ષા નિયમોના આધારે પેકેટ્સ તપાસે છે
    \item \textbf{એક્સેસ કંટ્રોલ}: કોણ નેટવર્કમાં પ્રવેશી શકે તે મંજૂરી આપે/બ્લોક કરે
    \item \textbf{લોગિંગ અને મોનિટરિંગ}: શંકાસ્પદ પ્રવૃત્તિ રેકોર્ડ કરે છે
    \item \textbf{NAT}: આંતરિક IP એડ્રેસ છુપાવે છે
    \item \textbf{સ્ટેટફુલ ઇન્સ્પેક્શન}: કનેક્શન સ્ટેટ્સ અને સંદર્ભો ટ્રેક કરે છે
\end{itemize}

\end{solutionbox}

\begin{mnemonicbox}
\mnemonic{Firewall = Filter, Access, Monitor}
\end{mnemonicbox}

\questionmarks{2(બ)}{4}{FDDI (ફાઇબર ડિસ્ટ્રિબ્યુટેડ ડેટા ઇન્ટરફેસ) અને CDDI (કોપર ડિસ્ટ્રિબ્યુટેડ ડેટા ઇન્ટરફેસ) ની મુખ્ય લાક્ષણિકતાઓ અને ફાયદાઓ સાથે સરખામણી કરો.}

\begin{solutionbox}

\begin{table}[H]
\centering
\begin{tabulary}{\textwidth}{L L L}
\toprule
\textbf{લાક્ષણિકતા} & \textbf{FDDI} & \textbf{CDDI} \\
\midrule
\textbf{મીડિયા} & ફાઈબર ઓપ્ટિક કેબલ & ટ્વિસ્ટેડ પેર કોપર (STP/UTP) \\
\textbf{સ્પીડ} & 100 Mbps & 100 Mbps \\
\textbf{અંતર} & લાંબા અંતર (200 km સુધી) & ટૂંકા અંતર (100 m) \\
\textbf{કિંમત} & મોંઘું & સસ્તું \\
\textbf{EMI} & અસર કરતું નથી & અસર કરી શકે છે \\
\bottomrule
\end{tabulary}
\caption{FDDI vs CDDI}
\end{table}

\textbf{FDDI ફાયદા:}
\begin{itemize}
    \item \textbf{ઉચ્ચ બેન્ડવિડ્થ}: બેકબોન નેટવર્ક્સ માટે યોગ્ય
    \item \textbf{વિશ્વસનીયતા}: ડ્યુઅલ રિંગ આર્કિટેક્ચર
    \item \textbf{સુરક્ષા}: ટેપ કરવું મુશ્કેલ
\end{itemize}

\textbf{CDDI ફાયદા:}
\begin{itemize}
    \item \textbf{ઓછી કિંમત}: કોપર કેબલ સસ્તું છે
    \item \textbf{ઇન્સ્ટોલેશન}: ફાઈબર કરતા સરળ
    \item \textbf{સુસંગતતા}: હાલના નેટવર્ક સાધનો સાથે કામ કરે છે
\end{itemize}

\end{solutionbox}

\begin{mnemonicbox}
\mnemonic{FDDI = Fiber Distance, CDDI = Copper Cost}
\end{mnemonicbox}

\questionmarks{2(ક)}{7}{ઇથરનેટ, ફાસ્ટ ઇથરનેટ, ગીગાબીટ ઇથરનેટ સમજાવો અને અલગ પાડો.}

\begin{solutionbox}

\begin{table}[H]
\centering
\begin{tabulary}{\textwidth}{L L L L}
\toprule
\textbf{વિશેષતા} & \textbf{ઇથરનેટ} & \textbf{ફાસ્ટ ઇથરનેટ} & \textbf{ગીગાબીટ ઇથરનેટ} \\
\midrule
\textbf{સ્ટાન્ડર્ડ} & IEEE 802.3 & IEEE 802.3u & IEEE 802.3z/ab \\
\textbf{સ્પીડ} & 10 Mbps & 100 Mbps & 1000 Mbps (1 Gbps) \\
\textbf{કેબલ} & Coax/Cat3 & Cat5 & Cat5e/Fiber \\
\textbf{ઉપયોગ} & જૂના LANs & સ્ટાન્ડર્ડ LANs & હાઇ-સ્પીડ બેકબોન્સ \\
\bottomrule
\end{tabulary}
\caption{ઇથરનેટ ઇવોલ્યુશન}
\end{table}

\begin{figure}[H]
\centering
\begin{tikzpicture}[node distance=1.5cm]
    \node (eth) [gtu block, minimum width=3cm] {Ethernet (10 Mbps)};
    \node (fast) [gtu block, minimum width=3cm, right=of eth] {Fast Ethernet (100 Mbps)};
    \node (giga) [gtu block, minimum width=3cm, right=of fast] {Gigabit Ethernet (1000 Mbps)};
    
    \draw[gtu arrow] (eth) -- (fast);
    \draw[gtu arrow] (fast) -- (giga);
\end{tikzpicture}
\caption{ઇથરનેટ સ્પીડ ઇવોલ્યુશન}
\end{figure}

\textbf{જોડાણો:}
\begin{itemize}
    \item \textbf{10Base-T}: ટ્વિસ્ટેડ પેર પર 10 Mbps
    \item \textbf{100Base-TX}: Cat5 પર 100 Mbps
    \item \textbf{1000Base-T}: Cat5e/6 પર 1 Gbps
\end{itemize}

\end{solutionbox}

\begin{mnemonicbox}
\mnemonic{Every Fast Gigabit = 10, 100, 1000}
\end{mnemonicbox}

\questionmarks{2(અ OR)}{3}{નેટવર્ક ઇન્ફ્રાસ્ટ્રક્ચરમાં રાઉટરની ભૂમિકા અને કાર્ય સમજાવો.}

\begin{solutionbox}

\textbf{રાઉટરના કાર્યો:}
\begin{itemize}
    \item \textbf{પેકેટ ફોરવર્ડિંગ}: વિવિધ નેટવર્ક્સ વચ્ચે ડેટા પેકેટ્સ મોકલે છે
    \item \textbf{પાથ સિલેક્શન}: ગંતવ્ય સુધીનો શ્રેષ્ઠ રસ્તો નક્કી કરે છે (Routing Table)
    \item \textbf{IP એડ્રેસિંગ}: IP એડ્રેસના આધારે કામ કરે છે (લેયર 3)
    \item \textbf{ટ્રાફિક મેનેજમેન્ટ}: નેટવર્ક ટ્રાફિક ભીડ ઘટાડે છે (Congestion Control)
    \item \textbf{પ્રોટોકોલ ટ્રાન્સલેશન}: વિવિધ નેટવર્ક પ્રોટોકોલ્સ વચ્ચે રૂપાંતર કરે છે
\end{itemize}

\end{solutionbox}

\begin{mnemonicbox}
\mnemonic{Router = Route, Isolate, Connect}
\end{mnemonicbox}

\questionmarks{2(બ OR)}{4}{FDDI (ફાઇબર ડિસ્ટ્રિબ્યુટેડ ડેટા ઇન્ટરફેસ) નું માળખું સમજાવો અને તેના ફાયદાઓ આપો.}

\begin{solutionbox}

\textbf{FDDI માળખું:}
\begin{itemize}
    \item \textbf{ડ્યુઅલ રિંગ}: બે રિંગ્સ (પ્રાઈમરી અને સેકન્ડરી)
    \item \textbf{કાઉન્ટર-રોટેટિંગ}: પ્રાઈમરી ક્લોકવાઇઝ, સેકન્ડરી કાઉન્ટર-ક્લોકવાઇઝ
    \item \textbf{રીડન્ડન્સી}: જો એક રિંગ તૂટી જાય, તો બીજી રિંગ બેકઅપ તરીકે કામ કરે છે
    \item \textbf{ટોકન પાસિંગ}: એક્સેસ કંટ્રોલ માટે ટોકનનો ઉપયોગ કરે છે
\end{itemize}

\begin{figure}[H]
\centering
\begin{tikzpicture}
    \node[gtu state] (A) at (0,2) {Node A};
    \node[gtu state] (B) at (4,2) {Node B};
    \node[gtu state] (C) at (4,-2) {Node C};
    \node[gtu state] (D) at (0,-2) {Node D};
    
    % Primary Ring (Outer)
    \draw[gtu arrow, blue] (A.north east) to[bend left=20] (B.north west);
    \draw[gtu arrow, blue] (B.south east) to[bend left=20] (C.north east);
    \draw[gtu arrow, blue] (C.south west) to[bend left=20] (D.south east);
    \draw[gtu arrow, blue] (D.north west) to[bend left=20] (A.south west);
    
    % Secondary Ring (Inner)
    \draw[gtu arrow, red, dashed] (B.south west) to[bend left=20] (A.south east);

    \draw[gtu arrow, red, dashed] (C.north west) to[bend left=20] (B.south west); 
    \draw[gtu arrow, red, dashed] (D.south east) to[bend left=20] (C.south west);
    \draw[gtu arrow, red, dashed] (A.south east) to[bend left=20] (D.north east);

    % Legend
    \matrix [draw, below right] at (5, 0) {
      \node [draw=blue, ultra thick, label=right:Primary Ring] {}; \\
      \node [draw=red, dashed, label=right:Secondary Ring] {}; \\
    };
\end{tikzpicture}
\caption{FDDI ડ્યુઅલ રિંગ માળખું}
\end{figure}
\textbf{નોંધ:} ઉપરની આકૃતિ FDDI ની ડ્યુઅલ રિંગ ટોપોલોજી દર્શાવે છે.

\textbf{ફાયદા:}
\begin{itemize}
    \item \textbf{હાઈ સ્પીડ}: 100 Mbps ડેટા રેટ
    \item \textbf{દૂરી}: મલ્ટીમોડ ફાઇબર સાથે 200 કિ.મી. સુધી
    \item \textbf{ફોલ્ટ ટોલરન્સ}: ડ્યુઅલ રિંગ નિષ્ફળતા સામે રક્ષણ આપે છે
    \item \textbf{સેલ્ફ-હીલિંગ}: લિંક નિષ્ફળ જાય ત્યારે ઓટોમેટિક પુનઃરચના
\end{itemize}

\end{solutionbox}

\begin{mnemonicbox}
\mnemonic{FDDI = Fast, Dual, Distance, Immune}
\end{mnemonicbox}

\questionmarks{2(ક OR)}{7}{નેટવર્ક ઉપકરણોની ભૂમિકા સમજાવો. બધા ઉપકરણો વિશે સંક્ષિપ્તમાં વર્ણન કરો.}

\begin{solutionbox}

\begin{table}[H]
\centering
\begin{tabulary}{\textwidth}{L L L}
\toprule
\textbf{ઉપકરણ} & \textbf{OSI લેયર} & \textbf{કાર્ય} \\
\midrule
\textbf{રીપીટર} & ફિઝિકલ (1) & સિગ્નલ પુનઃજનિત કરે, રેન્જ વધારે \\
\textbf{હબ} & ફિઝિકલ (1) & મલ્ટીપોર્ટ રીપીટર, બધાને બ્રોડકાસ્ટ કરે \\
\textbf{બ્રિજ} & ડેટા લિંક (2) & નેટવર્ક સેગમેન્ટ્સ જોડે, MAC ફિલ્ટરિંગ \\
\textbf{સ્વિચ} & ડેટા લિંક (2) & બુદ્ધિશાળી બ્રિજ, ચોક્કસ પોર્ટ પર મોકલે \\
\textbf{રાઉટર} & નેટવર્ક (3) & લોજિકલ એડ્રેસિંગ (IP) અને પાથ પસંદગી \\
\textbf{ગેટવે} & બધા લેયર્સ & અલગ પ્રોટોકોલ નેટવર્ક્સ જોડે (અનુવાદક) \\
\bottomrule
\end{tabulary}
\caption{નેટવર્ક ઉપકરણો}
\end{table}

\begin{figure}[H]
\centering
\begin{tikzpicture}[node distance=1.5cm]
    \node (phy) [gtu block, fill=orange!10] {Physical Layer};
    \node (rep) [gtu block, below left=of phy, minimum width=3cm] {Repeater};
    \node (hub) [gtu block, below right=of phy, minimum width=3cm] {Hub};
    
    \node (dl) [gtu block, fill=green!10, right=of phy, xshift=2cm] {Data Link Layer};
    \node (bri) [gtu block, below left=of dl, minimum width=3cm] {Bridge};
    \node (swi) [gtu block, below right=of dl, minimum width=3cm] {Switch};
    
    \node (net) [gtu block, fill=blue!10, right=of dl, xshift=2cm] {Network Layer};
    \node (rou) [gtu block, below=of net, minimum width=3cm] {Router};

    \draw[gtu arrow] (phy) -- (rep);
    \draw[gtu arrow] (phy) -- (hub);
    \draw[gtu arrow] (dl) -- (bri);
    \draw[gtu arrow] (dl) -- (swi);
    \draw[gtu arrow] (net) -- (rou);
\end{tikzpicture}
\caption{લેયર મુજબ નેટવર્ક ઉપકરણો}
\end{figure}

\textbf{વર્ણન:}
\begin{itemize}
    \item \textbf{રીપીટર}: નબળા સિગ્નલને એમ્પ્લીફાય કરે છે
    \item \textbf{હબ}: સ્ટાર ટોપોલોજીમાં કેન્દ્રીય ઉપકરણ
    \item \textbf{બ્રિજ}: ટ્રાફિક ઘટાડવા માટે નેટવર્કનું વિભાજન કરે છે
    \item \textbf{સ્વિચ}: ઝડપી અને કાર્યક્ષમ પેકેટ ડિલિવરી માટે
    \item \textbf{રાઉટર}: ઇન્ટરનેટ અને WAN કનેક્ટિવિટી માટે
    \item \textbf{ગેટવે}: સંપૂર્ણ પ્રોટોકોલ સ્ટેક રૂપાંતર
\end{itemize}

\end{solutionbox}

\begin{mnemonicbox}
\mnemonic{Repeat, Hub, Bridge, Switch, Route, Gateway}
\end{mnemonicbox}


\questionmarks{3(અ)}{3}{Name any three data link layer protocol and explain any one in detail.}

\begin{solutionbox}

\begin{table}[H]
\centering
\begin{tabulary}{\textwidth}{L L L}
\toprule
\textbf{વિશેષતા} & \textbf{IPv4} & \textbf{IPv6} \\
\midrule
\textbf{એડ્રેસ સાઈઝ} & 32-bit & 128-bit \\
\textbf{એડ્રેસ સ્પેસ} & 4.3 અબજ & અમર્યાદિત (લગભગ) \\
\textbf{હેડર લેન્થ} & 20-60 bytes & 40 bytes (fixed) \\
\textbf{નોટેશન} & Dotted Decimal (192.168.1.1) & Hexadecimal (2001:abcd::1) \\
\textbf{સુરક્ષા} & Optional (IPSec) & Built-in (IPSec) \\
\textbf{કોન્ફિગરેશન} & Manual / DHCP & Auto-configuration (SLAAC) \\
\bottomrule
\end{tabulary}
\caption{IPv4 vs IPv6}
\end{table}

\textbf{IPv6 ના ફાયદા:}
\begin{itemize}
    \item \textbf{વિશાળ એડ્રેસ સ્પેસ}: ભવિષ્યના ઉપકરણો માટે પૂરતા એડ્રેસ
    \item \textbf{સરળ હેડર}: ઝડપી રાઉટિંગ માટે
    \item \textbf{બિલ્ટ-ઇન સુરક્ષા}: IPSec ફરજિયાત છે
    \item \textbf{Quality of Service}: IPv6 માં વધુ સારું QoS સપોર્ટ
\end{itemize}

\end{solutionbox}

\begin{mnemonicbox}
\mnemonic{IPv6 = Infinite, Integrated, Improved}
\end{mnemonicbox}

\questionmarks{3(અ OR)}{3}{કમ્પ્યુટર નેટવર્ક્સમાં વપરાતા ગાઇડેડ અને અનગાઇડેડ ટ્રાન્સમિશન મીડિયા વચ્ચેનો તફાવત સમજાવો.}

\begin{solutionbox}

\begin{table}[H]
\centering
\begin{tabulary}{\textwidth}{L L L}
\toprule
\textbf{વિશેષતા} & \textbf{ગાઇડેડ (Wired)} & \textbf{અનગાઇડેડ (Wireless)} \\
\midrule
\textbf{માધ્યમ} & ભૌતિક કેબલ (Copper, Fiber) & હવા / અવકાશ (EM Waves) \\
\textbf{ઉદાહરણ} & Twisted Pair, Coaxial, Fiber & Radio, Microwave, Infrared \\
\textbf{દખલગીરી} & ઓછી & વધારે \\
\textbf{સ્થાપન} & જટિલ (કેબલિંગ જરૂરી) & સરળ / લવચીક \\
\textbf{ગતિશીલતા} & મર્યાદિત & સંપૂર્ણ ગતિશીલતા \\
\bottomrule
\end{tabulary}
\caption{ગાઇડેડ vs અનગાઇડેડ મીડિયા}
\end{table}

\textbf{ઉદાહરણો:}
\begin{itemize}
    \item \textbf{Twisted Pair}: LAN માં વપરાય છે
    \item \textbf{Fiber Optic}: હાઇ-સ્પીડ ડેટા માટે
    \item \textbf{Radio Waves}: WiFi, Bluetooth
    \item \textbf{Infrared}: ટૂંકા અંતરના સંચાર માટે (રીમોટ)
\end{itemize}

\end{solutionbox}

\begin{mnemonicbox}
\mnemonic{Guided = Ground, Unguided = Air}
\end{mnemonicbox}

\questionmarks{3(બ OR)}{4}{સર્કિટ સ્વિચિંગ અને પેકેટ સ્વિચિંગનું વર્ણન કરો.}

\begin{solutionbox}

\begin{table}[H]
\centering
\begin{tabulary}{\textwidth}{L L L}
\toprule
\textbf{વિશેષતા} & \textbf{સર્કિટ સ્વિચિંગ} & \textbf{પેકેટ સ્વિચિંગ} \\
\midrule
\textbf{કનેક્શન} & સમર્પિત પાથ (Dedicated Path) & કોઈ સમર્પિત પાથ નથી \\
\textbf{બેન્ડવિડ્થ} & આરક્ષિત (Reserved) & વહેંચાયેલ (Shared) \\
\textbf{ડીલે} & ઓછો (સેટઅપ પછી) & ચલિત (Variable) \\
\textbf{ઉદાહરણ} & ટેલિફોન નેટવર્ક & ઇન્ટરનેટ (IP) \\
\textbf{કાર્યક્ષમતા} & ઓછી (રિસોર્સ વેડફાય છે) & ઊંચી (રિસોર્સ શેરિંગ) \\
\bottomrule
\end{tabulary}
\caption{સ્વિચિંગ તકનીકો}
\end{table}

\textbf{સર્કિટ સ્વિચિંગ:}
\begin{itemize}
    \item \textbf{સ્થાપના}: ડેટા ટ્રાન્સફર પહેલા કનેક્શન સેટ કરવું પડે
    \item \textbf{સતત પ્રવાહ}: વૉઇસ કોલ માટે શ્રેષ્ઠ
\end{itemize}

\textbf{પેકેટ સ્વિચિંગ:}
\begin{itemize}
    \item \textbf{પેકેટ્સ}: ડેટા નાના ટુકડાઓમાં વહેંચાય છે
    \item \textbf{સ્વતંત્ર રાઉટિંગ}: દરેક પેકેટ અલગ રસ્તે જઈ શકે છે
    \item \textbf{રિસોર્સ શેરિંગ}: બેન્ડવિડ્થ બધા યુઝર્સ વચ્ચે વહેંચાય છે
\end{itemize}

\end{solutionbox}

\begin{mnemonicbox}
\mnemonic{Circuit = Continuous, Packet = Pieces}
\end{mnemonicbox}

\questionmarks{3(ક OR)}{7}{IPv4 ને વિગતવાર સમજાવો.}

\begin{solutionbox}

\textbf{IPv4 (Internet Protocol version 4):}
\begin{itemize}
    \item \textbf{વ્યાખ્યા}: 32-bit એડ્રેસિંગ સ્કીમ
    \item \textbf{ફોર્મેટ}: 4 ઓક્ટેટ્સ (8-bit each), ડોટેડ ડેસિમલ (દા.ત., 192.168.1.1)
    \item \textbf{કુલ એડ્રેસ}: $2^{32}$ (લગભગ 4.3 અબજ)
\end{itemize}

\textbf{IPv4 ક્લાસિસ:}
\begin{table}[H]
\centering
\begin{tabulary}{\textwidth}{C L L}
\toprule
\textbf{ક્લાસ} & \textbf{રેન્જ (પહેલો ઓક્ટેટ)} & \textbf{ઉપયોગ} \\
\midrule
\textbf{A} & 1 - 126 & ખૂબ મોટા નેટવર્ક્સ \\
\textbf{B} & 128 - 191 & મધ્યમ કદના નેટવર્ક્સ \\
\textbf{C} & 192 - 223 & નાના નેટવર્ક્સ (LAN) \\
\textbf{D} & 224 - 239 & મલ્ટીકાસ્ટિંગ \\
\textbf{E} & 240 - 255 & સંશોધન/વૈજ્ઞાનિક હેતુ \\
\bottomrule
\end{tabulary}
\caption{IPv4 એડ્રેસ ક્લાસિસ}
\end{table}

\textbf{IPv4 હેડર:}
\begin{itemize}
    \item \textbf{Version}: IP વર્ઝન (4)
    \item \textbf{Header Length}: હેડરનું કદ
    \item \textbf{TTL (Time to Live)}: પેકેટનું જીવનકાળ (લૂપ અટકાવવા)
    \item \textbf{Protocol}: ટ્રાન્સપોર્ટ પ્રોટોકોલ (TCP=6, UDP=17)
    \item \textbf{Source/Destination IP}: મોકલનાર અને મેળવનારના એડ્રેસ
\end{itemize}

\begin{figure}[H]
\centering
\begin{tikzpicture}[x=0.45cm, y=0.6cm]
  \draw[thick] (0,0) rectangle (32, -6);
  
  \foreach \y in {-1,-2,-3,-4,-5} \draw (0,\y) -- (32,\y);
  \foreach \x in {4,8,16,19,20,31} \draw (\x,0) -- (\x,-1);
  \draw (16,-1) -- (16,-2);
  \draw (16,-2) -- (16,-3); % Flags area
  \draw (3,-2)node[right]{(Identification)};
  
  \draw (8,-3) -- (8,-4);
  \draw (16,-3) -- (16,-4);
  \draw (24,-3) -- (24,-4);
  
  % Reset lines for cleaner look
  \draw[fill=white] (0,0) rectangle (32,-6);
  \foreach \y in {-1,-2,-3,-4,-5} \draw (0,\y) -- (32,\y);
  
  % Row 1
  \draw (4,0) -- (4,-1);
  \draw (8,0) -- (8,-1);
  \draw (16,0) -- (16,-1);
  \node at (2,-0.5) {\tiny Version};
  \node at (6,-0.5) {\tiny IHL};
  \node at (12,-0.5) {\tiny Type of Service};
  \node at (24,-0.5) {\tiny Total Length};
  
  % Row 2
  \draw (16,-1) -- (16,-2);
  \draw (19,-1) -- (19,-2);
  \node at (8,-1.5) {\tiny Identification};
  \node at (17.5,-1.5) {\tiny Flags};
  \node at (25.5,-1.5) {\tiny Fragment Offset};
  
  % Row 3
  \draw (8,-2) -- (8,-3);
  \draw (16,-2) -- (16,-3);
  \node at (4,-2.5) {\tiny TTL};
  \node at (12,-2.5) {\tiny Protocol};
  \node at (24,-2.5) {\tiny Header Checksum};
  
  % Row 4
  \node at (16,-3.5) {\tiny Source IP Address};
  
  % Row 5
  \node at (16,-4.5) {\tiny Destination IP Address};
  
  % Row 6
  \node at (16,-5.5) {\tiny Options + Padding};
  
  % Scales
  \node[above] at (0,0) {0};
  \node[above] at (16,0) {16};
  \node[above] at (31,0) {31};
\end{tikzpicture}
\caption{IPv4 હેડર સ્ટ્રક્ચર}
\end{figure}

\end{solutionbox}

\begin{mnemonicbox}
\mnemonic{IPv4 = 4 octets, 32 bits, Classes A-C}
\end{mnemonicbox}



\questionmarks{4(અ)}{3}{ARP અને RARP નું પૂરું નામ આપો અને તેનું વર્ણન કરો.}

\begin{solutionbox}

\textbf{પૂરા નામ:}
\begin{itemize}
    \item \textbf{ARP}: Address Resolution Protocol (એડ્રેસ રિઝોલ્યુશન પ્રોટોકોલ)
    \item \textbf{RARP}: Reverse Address Resolution Protocol (રિવર્સ એડ્રેસ રિઝોલ્યુશન પ્રોટોકોલ)
\end{itemize}

\begin{table}[H]
\centering
\begin{tabulary}{\textwidth}{L L}
\toprule
\textbf{પ્રોટોકોલ} & \textbf{કાર્ય} \\
\midrule
\textbf{ARP} & IP એડ્રેસને MAC એડ્રેસમાં ફેરવે છે \\
\textbf{RARP} & MAC એડ્રેસને IP એડ્રેસમાં ફેરવે છે \\
\bottomrule
\end{tabulary}
\caption{ARP vs RARP}
\end{table}

\textbf{ARP પ્રક્રિયા:}
\begin{itemize}
    \item \textbf{Request}: "IP 192.168.1.1 કોની પાસે છે?" (Broadcast)
    \item \textbf{Reply}: "192.168.1.1 MAC 00:1A... પર છે" (Unicast)
    \item \textbf{Cache}: ભવિષ્યના ઉપયોગ માટે મેપિંગ સ્ટોર કરે છે
\end{itemize}

\textbf{RARP પ્રક્રિયા:}
\begin{itemize}
    \item \textbf{Diskless Workstations}: જેમની પાસે IP નથી તેઓ MAC મોકલે છે
    \item \textbf{Server Response}: સર્વર તેમને IP એડ્રેસ અસાઇન કરે છે
\end{itemize}

\end{solutionbox}

\begin{mnemonicbox}
\mnemonic{ARP = Address to MAC, RARP = Reverse}
\end{mnemonicbox}

\questionmarks{4(બ)}{4}{DSL ટેકનોલોજી તેના ફાયદા અને મર્યાદાઓ સાથે વર્ણવો.}

\begin{solutionbox}

\textbf{DSL (Digital Subscriber Line):}

\begin{table}[H]
\centering
\begin{tabulary}{\textwidth}{L L L}
\toprule
\textbf{પ્રકાર} & \textbf{સ્પીડ} & \textbf{અંતર} \\
\midrule
\textbf{ADSL} & 8 Mbps સુધી & 5.5 km \\
\textbf{VDSL} & 52 Mbps સુધી & 1.5 km \\
\textbf{SDSL} & 2 Mbps સુધી & 3 km \\
\bottomrule
\end{tabulary}
\caption{DSL પ્રકારો}
\end{table}

\textbf{ફાયદા:}
\begin{itemize}
    \item \textbf{હાલનું ઇન્ફ્રાસ્ટ્રક્ચર}: ટેલિફોન લાઇનનો ઉપયોગ કરે છે
    \item \textbf{Always-On}: સતત ઇન્ટરનેટ કનેક્શન
    \item \textbf{Voice + Data}: ફોન અને ઇન્ટરનેટ એકસાથે ચાલે છે
\end{itemize}

\textbf{મર્યાદાઓ:}
\begin{itemize}
    \item \textbf{અંતર આધારિત}: એક્સચેન્જથી દૂર જતાં સ્પીડ ઘટે છે
    \item \textbf{Upload Speed}: ડાઉનલોડ કરતા ઓછી હોય છે (ADSL)
    \item \textbf{Line Quality}: કોપર વાયરની ગુણવત્તા પર આધારિત
\end{itemize}

\end{solutionbox}

\begin{mnemonicbox}
\mnemonic{DSL = Digital Subscriber Line}
\end{mnemonicbox}

\questionmarks{4(ક)}{7}{DNS (ડોમેન નેમ સિસ્ટમ) ની ભૂમિકા.}

\begin{solutionbox}

\textbf{DNS કાર્યો:}
\begin{itemize}
    \item \textbf{Name Resolution}: ડોમેન નામો (google.com) ને IP એડ્રેસમાં ફેરવે છે
    \item \textbf{Hierarchical Structure}: વૃક્ષ જેવી રચનામાં ગોઠવાયેલું છે
    \item \textbf{Distributed Database}: માહિતી અનેક સર્વરો પર સંગ્રહિત છે
\end{itemize}

\begin{figure}[H]
\centering
\begin{tikzpicture}[level distance=1.5cm,
  level 1/.style={sibling distance=5cm},
  level 2/.style={sibling distance=2.5cm},
  level 3/.style={sibling distance=2.5cm}]
  \node [gtu root] {Root (.)}
    child {node [gtu child] {.com}
      child {node [gtu child] {google.com}
        child {node [gtu child] {mail} edge from parent node[left, font=\tiny] {sub}}
        child {node [gtu child] {drive} edge from parent node[right, font=\tiny] {sub}}
      }
      child {node [gtu child] {yahoo.com}}
    }
    child {node [gtu child] {.org}};
\end{tikzpicture}
\caption{DNS વંશવેલો માળખું}
\end{figure}

\textbf{DNS હાયરાર્કી:}
\begin{itemize}
    \item \textbf{Root Domain}: સર્વોચ્ચ સ્તર (.)
    \item \textbf{Top-Level Domain (TLD)}: .com, .org, .net
    \item \textbf{Second-Level Domain}: google.com, yahoo.com
    \item \textbf{Subdomain}: mail.google.com
\end{itemize}

\textbf{DNS રેકોર્ડ પ્રકારો:}
\begin{itemize}
    \item \textbf{A Record}: IPv4 એડ્રેસ
    \item \textbf{AAAA Record}: IPv6 એડ્રેસ
    \item \textbf{CNAME}: ઉપનામ (Alias)
    \item \textbf{MX}: મેઇલ સર્વર
\end{itemize}

\end{solutionbox}

\begin{mnemonicbox}
\mnemonic{DNS = Domain Name System}
\end{mnemonicbox}

\questionmarks{4(અ OR)}{3}{DHCP અને BOOTP નું પૂરું નામ આપો અને તેમનું વર્ણન કરો.}

\begin{solutionbox}

\textbf{પૂરા નામ:}
\begin{itemize}
    \item \textbf{DHCP}: Dynamic Host Configuration Protocol
    \item \textbf{BOOTP}: Bootstrap Protocol
\end{itemize}

\begin{table}[H]
\centering
\begin{tabulary}{\textwidth}{L L}
\toprule
\textbf{પ્રોટોકોલ} & \textbf{કાર્ય} \\
\midrule
\textbf{DHCP} & ઓટોમેટિક IP એડ્રેસ અસાઇન કરે છે \\
\textbf{BOOTP} & ડિસ્કલેસ વર્કસ્ટેશન્સને IP આપે છે \\
\bottomrule
\end{tabulary}
\caption{DHCP vs BOOTP}
\end{table}

\textbf{DHCP પ્રક્રિયા:}
\begin{itemize}
    \item \textbf{Discovery}: ક્લાયંટ IP માંગે છે
    \item \textbf{Offer}: સર્વર IP ઓફર કરે છે
    \item \textbf{Request}: ક્લાયંટ તે IP સ્વીકારે છે
    \item \textbf{Ack}: સર્વર કન્ફર્મ કરે છે
\end{itemize}

\end{solutionbox}

\begin{mnemonicbox}
\mnemonic{DHCP = Dynamic, BOOTP = Bootstrap}
\end{mnemonicbox}

\questionmarks{4(બ OR)}{4}{વર્ચ્યુઅલ સર્કિટ્સ અને ડેટાગ્રામ નેટવર્ક્સ વચ્ચેનો તફાવત.}

\begin{solutionbox}

\begin{table}[H]
\centering
\begin{tabulary}{\textwidth}{L L L}
\toprule
\textbf{વિશેષતા} & \textbf{વર્ચ્યુઅલ સર્કિટ્સ} & \textbf{ડેટાગ્રામ નેટવર્ક્સ} \\
\midrule
\textbf{કનેક્શન} & Connection-oriented & Connectionless \\
\textbf{સેટઅપ} & જરૂરી છે & જરૂરી નથી \\
\textbf{રાઉટિંગ} & બધા પેકેટ્સ એક જ રસ્તે & દરેક પેકેટ સ્વતંત્ર રસ્તે \\
\textbf{ક્રમ} & ક્રમબદ્ધ આવે છે & આડાઅવળા આવી શકે \\
\textbf{વિશ્વસનીયતા} & વધુ & ઓછી \\
\bottomrule
\end{tabulary}
\caption{વર્ચ્યુઅલ સર્કિટ્સ vs ડેટાગ્રામ}
\end{table}

\textbf{વર્ચ્યુઅલ સર્કિટ્સ:}
\begin{itemize}
    \item \textbf{સમર્પિત પાથ}: કનેક્શન દરમિયાન પાથ ફિક્સ હોય છે
    \item \textbf{ઉદાહરણ}: ATM, Frame Relay
\end{itemize}

\textbf{ડેટાગ્રામ નેટવર્ક્સ:}
\begin{itemize}
    \item \textbf{સ્વતંત્રતા}: દરેક પેકેટ અલગ અલગ રસ્તો લઈ શકે
    \item \textbf{ઉદાહરણ}: ઈન્ટરનેટ (IP)
\end{itemize}

\end{solutionbox}

\begin{mnemonicbox}
\mnemonic{Virtual = Connection, Datagram = Independent}
\end{mnemonicbox}

\questionmarks{4(ક OR)}{7}{ટ્રાન્સપોર્ટ લેયરમાં TCP અને UDP પ્રોટોકોલ સમજાવો.}

\begin{solutionbox}

\begin{table}[H]
\centering
\begin{tabulary}{\textwidth}{L L L}
\toprule
\textbf{વિશેષતા} & \textbf{TCP} & \textbf{UDP} \\
\midrule
\textbf{કનેક્શન} & Connection-oriented & Connectionless \\
\textbf{વિશ્વસનીયતા} & વિશ્વસનીય (Reliable) & અવિશ્વસનીય (Unreliable) \\
\textbf{હેડર સાઈઝ} & 20 bytes & 8 bytes \\
\textbf{સ્પીડ} & ધીમું & ઝડપી \\
\textbf{ઉપયોગ} & Web, Email, File Transfer & DNS, Streaming, Gaming \\
\bottomrule
\end{tabulary}
\caption{TCP vs UDP}
\end{table}

\textbf{TCP (ટ્રાન્સમિશન કંટ્રોલ પ્રોટોકોલ):}
\begin{itemize}
    \item \textbf{Three-Way Handshake}: કનેક્શન સેટઅપ માટે
    \item \textbf{Flow Control}: સ્લાઈડિંગ વિન્ડો દ્વારા
    \item \textbf{Error Recovery}: ગુમ થયેલા પેકેટ્સ ફરી મોકલે છે
\end{itemize}

\textbf{TCP Header:}
\begin{figure}[H]
\centering
\begin{tikzpicture}[x=0.45cm, y=0.6cm]
  \draw[thick] (0,0) rectangle (32, -6);
  
  \foreach \y in {-1,-2,-3,-4,-5} \draw (0,\y) -- (32,\y);
  \draw (16,0) -- (16,-1);
  \draw (32,0) -- (32,-1);

  \draw (4,-3) -- (4,-4);
  \draw (10,-3) -- (10,-4);
  \draw (16,-3) -- (16,-4);
  
  \draw (16,-5) -- (16,-6);
  
  \node at (8,-0.5) {\tiny Source Port};
  \node at (24,-0.5) {\tiny Destination Port};
  \node at (16,-1.5) {\tiny Sequence Number};
  \node at (16,-2.5) {\tiny Acknowledgment Number};
  
  \node at (2,-3.5) {\tiny Hlen};
  \node at (7,-3.5) {\tiny Resv};
  \node at (13,-3.5) {\tiny Flags};
  \node at (24,-3.5) {\tiny Window Size};
  
  \node at (8,-4.5) {\tiny Checksum};
  \node at (24,-4.5) {\tiny Urgent Pointer};
  
  \node at (16,-5.5) {\tiny Options + Padding};
  
   % Scales
  \node[above] at (0,0) {0};
  \node[above] at (16,0) {16};
  \node[above] at (31,0) {31};
\end{tikzpicture}
\caption{TCP હેડર સ્ટ્રક્ચર}
\end{figure}

\textbf{UDP (યુઝર ડેટાગ્રામ પ્રોટોકોલ):}
\begin{itemize}
    \item \textbf{સરળ પ્રોટોકોલ}: ઓછું ઓવરહેડ
    \item \textbf{Best Effort}: ડિલિવરીની કોઈ ખાતરી નથી
    \item \textbf{Real-time}: વોઈસ અને વિડિયો માટે શ્રેષ્ઠ
\end{itemize}

\textbf{UDP Header:}
\begin{figure}[H]
\centering
\begin{tikzpicture}[x=0.45cm, y=0.6cm]
  \draw[thick] (0,0) rectangle (32, -2);
  \foreach \y in {-1} \draw (0,\y) -- (32,\y);
  \draw (16,0) -- (16,-1);
  \draw (16,-1) -- (16,-2);
  
  \node at (8,-0.5) {\tiny Source Port};
  \node at (24,-0.5) {\tiny Destination Port};
  \node at (8,-1.5) {\tiny Length};
  \node at (24,-1.5) {\tiny Checksum};
  
  % Scales
  \node[above] at (0,0) {0};
  \node[above] at (16,0) {16};
  \node[above] at (31,0) {31};
\end{tikzpicture}
\caption{UDP હેડર સ્ટ્રક્ચર}
\end{figure}

\end{solutionbox}

\begin{mnemonicbox}
\mnemonic{TCP = Reliable, UDP = Fast}
\end{mnemonicbox}



\questionmarks{5(અ)}{3}{નીચેનામાંથી કોઈપણ બે સમજાવો: (1) WWW (2) FTP (3) SMTP}

\begin{solutionbox}

\textbf{WWW (World Wide Web):}
\begin{itemize}
    \item \textbf{HTTP પ્રોટોકોલ}: HyperText Transfer Protocol
    \item \textbf{વેબ બ્રાઉઝર}: ક્લાયંટ સોફ્ટવેર (Chrome, Firefox)
    \item \textbf{વેબ સર્વર}: વેબ પેજીસ સર્વ કરે છે (Apache, IIS)
\end{itemize}

\textbf{FTP (એફટીપી):}
\begin{itemize}
    \item \textbf{ફાઇલ ટ્રાન્સફર}: અપલોડ અને ડાઉનલોડ માટે
    \item \textbf{બે મોડ}: એક્ટિવ અને પેસિવ મોડ
    \item \textbf{ઓથેન્ટિકેશન}: યુઝરનેમ અને પાસવર્ડ જરૂરી
\end{itemize}

\begin{table}[H]
\centering
\begin{tabulary}{\textwidth}{L L L}
\toprule
\textbf{સર્વિસ} & \textbf{પોર્ટ} & \textbf{કાર્ય} \\
\midrule
\textbf{WWW} & 80/443 & વેબ પેજ ડિલિવરી \\
\textbf{FTP} & 20/21 & ફાઇલ ટ્રાન્સફર \\
\bottomrule
\end{tabulary}
\caption{WWW vs FTP}
\end{table}

\end{solutionbox}

\begin{mnemonicbox}
\mnemonic{WWW = Web, FTP = Files}
\end{mnemonicbox}

\questionmarks{5(બ)}{4}{સિમેટ્રિક અને અસિમેટ્રિક એન્ક્રિપ્શન અલ્ગોરિધમ્સ વચ્ચેનો તફાવત.}

\begin{solutionbox}

\begin{table}[H]
\centering
\begin{tabulary}{\textwidth}{L L L}
\toprule
\textbf{વિશેષતા} & \textbf{સિમેટ્રિક} & \textbf{અસિમેટ્રિક} \\
\midrule
\textbf{કી (Key)} & એક જ કી (Encryption \& Decryption) & અલગ કી (Public/Private) \\
\textbf{ઝડપ} & ઝડપી & ધીમી \\
\textbf{કી વિતરણ} & મુશ્કેલ & સરળ \\
\textbf{ઉદાહરણ} & AES, DES & RSA, ECC \\
\bottomrule
\end{tabulary}
\caption{સિમેટ્રિક vs અસિમેટ્રિક}
\end{table}

\textbf{સિમેટ્રિક એન્ક્રિપ્શન:}
\begin{itemize}
    \item \textbf{એક કી}: મોકલનાર અને મેળવનાર એક જ કી વાપરે છે
    \item \textbf{પરફોર્મન્સ}: મોટા ડેટા માટે ઝડપી
\end{itemize}

\textbf{અસિમેટ્રિક એન્ક્રિપ્શન:}
\begin{itemize}
    \item \textbf{કી પેર}: પબ્લિક કી એન્ક્રિપ્ટ કરવા, પ્રાઇવેટ કી ડિક્રિપ્ટ કરવા
    \item \textbf{સુરક્ષા}: કી શેરિંગની જરૂર નથી
\end{itemize}

\end{solutionbox}

\begin{mnemonicbox}
\mnemonic{Symmetric = Same, Asymmetric = Different}
\end{mnemonicbox}

\questionmarks{5(ક)}{7}{ક્રિપ્ટોગ્રાફીના સંદર્ભમાં ``એન્ક્રિપ્શન'' અને ``ડિક્રિપ્શન'' શબ્દો સમજાવો.}

\begin{solutionbox}

\textbf{એન્ક્રિપ્શન:}
\begin{itemize}
    \item \textbf{વ્યાખ્યા}: પ્લેઈનટેક્સ્ટને સાયફરટેક્સ્ટમાં ફેરવવાની પ્રક્રિયા
    \item \textbf{હેતુ}: ડેટાની ગુપ્તતા જાળવવા
    \item \textbf{ઇનપુટ}: Plaintext + Key
    \item \textbf{આઉટપુટ}: Ciphertext
\end{itemize}

\textbf{ડિક્રિપ્શન:}
\begin{itemize}
    \item \textbf{વ્યાખ્યા}: સાયફરટેક્સ્ટને ફરીથી પ્લેઈનટેક્સ્ટમાં ફેરવવાની પ્રક્રિયા
    \item \textbf{હેતુ}: મૂળ માહિતી મેળવવા
    \item \textbf{ઇનપુટ}: Ciphertext + Key
    \item \textbf{આઉટપુટ}: Plaintext
\end{itemize}

\begin{figure}[H]
\centering
\begin{tikzpicture}[node distance=1.5cm, auto]
    \node (plain) [gtu input, minimum width=2.5cm] {Plaintext};
    \node (enc) [gtu process, right=of plain] {Encryption};
    \node (cipher) [gtu input, right=of enc, fill=gray!20, minimum width=2.5cm] {Ciphertext};
    \node (dec) [gtu process, right=of cipher] {Decryption};
    \node (plain2) [gtu input, right=of dec, minimum width=2.5cm] {Plaintext};
    
    \node (key1) [gtu start, above=of enc, fill=yellow!20] {Key};
    \node (key2) [gtu start, above=of dec, fill=yellow!20] {Key};

    \draw[gtu arrow] (plain) -- (enc);
    \draw[gtu arrow] (enc) -- (cipher);
    \draw[gtu arrow] (cipher) -- (dec);
    \draw[gtu arrow] (dec) -- (plain2);
    
    \draw[gtu arrow] (key1) -- (enc);
    \draw[gtu arrow] (key2) -- (dec);
\end{tikzpicture}
\caption{ક્રિપ્ટોગ્રાફી પ્રક્રિયા}
\end{figure}

\textbf{પ્રક્રિયા:}
\begin{enumerate}
    \item \textbf{મોકલનાર}: કી વડે મેસેજ એન્ક્રિપ્ટ કરે છે
    \item \textbf{રીસીવર}: કી વડે મેસેજ ડિક્રિપ્ટ કરે છે
\end{enumerate}

\end{solutionbox}

\questionmarks{5(અ OR)}{3}{IMAP અને POP3 વચ્ચેનો તફાવત લખો.}

\begin{solutionbox}

\begin{table}[H]
\centering
\begin{tabulary}{\textwidth}{L L L}
\toprule
\textbf{લક્ષણ} & \textbf{IMAP} & \textbf{POP3} \\
\midrule
\textbf{સ્ટોરેજ} & સર્વર-સાઇડ & ક્લાયન્ટ-સાઇડ \\
\textbf{એક્સેસ} & બહુવિધ ઉપકરણો & એક ઉપકરણ \\
\textbf{ઓફલાઇન} & મર્યાદિત & સંપૂર્ણ એક્સેસ \\
\bottomrule
\end{tabulary}
\caption{IMAP vs POP3}
\end{table}

\textbf{IMAP (Internet Message Access Protocol):}
\begin{itemize}
    \item \textbf{સર્વર સ્ટોરેજ}: મેસેજ સર્વર પર રહે છે
    \item \textbf{મલ્ટિ-ડીવાઇસ}: બહુવિધ ઉપકરણોથી એક્સેસ
    \item \textbf{સિન્ક્રોનાઇઝેશન}: ફેરફારો બધા ઉપકરણોમાં સિન્ક થાય છે
\end{itemize}

\textbf{POP3 (Post Office Protocol 3):}
\begin{itemize}
    \item \textbf{ડાઉનલોડ}: મેસેજ ક્લાયન્ટ પર ડાઉનલોડ થાય છે
    \item \textbf{સિંગલ ડીવાઇસ}: એક ઉપકરણ એક્સેસ માટે શ્રેષ્ઠ
    \item \textbf{સ્ટોરેજ}: ક્લાયન્ટ મેસેજ સ્ટોરેજ મેનેજ કરે છે
\end{itemize}

\end{solutionbox}

\begin{mnemonicbox}
\mnemonic{IMAP = Internet Access, POP3 = Post Office}
\end{mnemonicbox}

\questionmarks{5(બ OR)}{4}{સંક્ષિપ્તમાં Information Technology (સુધારા) અધિનિયમ, 2008 અને ભારતમાં સાયબર કાયદાઓ પર તેની અસરનું વર્ણન કરો.}

\begin{solutionbox}

\textbf{IT અધિનિયમ 2008 મુખ્ય લક્ષણો:}
\begin{itemize}
    \item \textbf{સાયબર ક્રાઇમ્સ}: વિવિધ સાયબર અપરાધોની વ્યાખ્યા
    \item \textbf{ડેટા પ્રોટેક્શન}: પ્રાઇવસી અને સિક્યુરિટી આવશ્યકતાઓ
    \item \textbf{ડિજિટલ સિગ્નેચર્સ}: ઈ-સિગ્નેચર્સની કાનૂની માન્યતા
    \item \textbf{પેનલ્ટીઝ}: ઉલ્લંઘન માટે દંડ અને કેદ
\end{itemize}

\textbf{મુખ્ય સુધારાઓ:}
\begin{itemize}
    \item \textbf{કલમ 66A}: આક્રામક મેસેજને ગુનાહિત બનાવ્યું (પછીથી રદ)
    \item \textbf{કલમ 69}: માહિતી ઇન્ટરસેપ્ટ કરવાની સરકારી શક્તિ
    \item \textbf{કલમ 72A}: વ્યક્તિગત માહિતી જાહેર કરવા માટે સજા
    \item \textbf{કલમ 43A}: ડેટા બ્રીચ માટે વળતર
\end{itemize}

\textbf{સાયબર કાયદાઓ પર અસર:}
\begin{itemize}
    \item \textbf{કાનૂની ફ્રેમવર્ક}: વ્યાપક સાયબર કાયદાનું માળખું
    \item \textbf{બિઝનેસ કોમ્પ્લાયન્સ}: ડેટા સુરક્ષા આવશ્યકતાઓ
    \item \textbf{વ્યક્તિગત અધિકારો}: પ્રાઇવસી પ્રોટેક્શન મેકેનિઝમ
    \item \textbf{કાયદાનો અમલ}: સાયબર ક્રાઇમ્સની તપાસ માટે સાધનો
\end{itemize}

\end{solutionbox}

\begin{mnemonicbox}
\mnemonic{IT Act = Internet Technology Act}
\end{mnemonicbox}

\questionmarks{5(ક OR)}{7}{સિમેટ્રિક અને એસિમેટ્રિક એન્ક્રિપ્શન અલ્ગોરિધમ્સ વચ્ચેનો તફાવત.}

\begin{solutionbox}

\begin{table}[H]
\centering
\begin{tabulary}{\textwidth}{L L L}
\toprule
\textbf{પાસું} & \textbf{સિમેટ્રિક એન્ક્રિપ્શન} & \textbf{એસિમેટ્રિક એન્ક્રિપ્શન} \\
\midrule
\textbf{કીનો ઉપયોગ} & એન્ક્રિપ્ટ/ડિક્રિપ્ટ માટે એક જ કી & વિવિધ કીઝ (પબ્લિક/પ્રાઇવેટ) \\
\textbf{કી મેનેજમેન્ટ} & મુશ્કેલ કી ડિસ્ટ્રિબ્યુશન & સરળ કી ડિસ્ટ્રિબ્યુશન \\
\textbf{પર્ફોર્મન્સ} & ઝડપી પ્રોસેસિંગ & ધીમી પ્રોસેસિંગ \\
\textbf{કી લેન્થ} & ટૂંકી કીઝ (128-256 બિટ્સ) & લાંબી કીઝ (1024-4096 બિટ્સ) \\
\textbf{સ્કેલેબિલિટી} & નબળી ($n^2$ કી પેર્સ જરૂરી) & સારી ($n$ કી પેર્સ જરૂરી) \\
\bottomrule
\end{tabulary}
\caption{સિમેટ્રિક vs એસિમેટ્રિક સરખામણી}
\end{table}

\textbf{સિમેટ્રિક એન્ક્રિપ્શન વિગતો:}
\begin{itemize}
    \item \textbf{અલ્ગોરિધમ પ્રકારો}: સ્ટ્રીમ સાઇફર્સ, બ્લોક સાઇફર્સ
    \item \textbf{કી ડિસ્ટ્રિબ્યુશન}: કી એક્સચેન્જ માટે સુરક્ષિત ચેનલ જરૂરી
    \item \textbf{એપ્લિકેશન્સ}: બલ્ક ડેટા એન્ક્રિપ્શન, VPNs
\end{itemize}

\textbf{એસિમેટ્રિક એન્ક્રિપ્શન વિગતો:}
\begin{itemize}
    \item \textbf{PKI}: કી મેનેજમેન્ટ માટે પબ્લિક કી ઇન્ફ્રાસ્ટ્રક્ચર
    \item \textbf{ડિજિટલ સિગ્નેચર્સ}: ઓથેન્ટિકેશન માટે
    \item \textbf{એપ્લિકેશન્સ}: ઈમેઇલ સિક્યુરિટી, SSL/TLS
\end{itemize}

\begin{figure}[H]
\centering
\begin{tikzpicture}[node distance=1.5cm]
    \node (root) [gtu root] {Encryption Methods};
    \node (sym) [gtu child, below left=of root, xshift=-1cm] {Symmetric};
    \node (asym) [gtu child, below right=of root, xshift=1cm] {Asymmetric};
    
    \node (same) [gtu child, below=of sym] {Same Key};
    \node (fast) [gtu child, below=of same] {Fast};
    
    \node (pair) [gtu child, below=of asym] {Key Pair};
    \node (slow) [gtu child, below=of pair] {Slow};
    
    \draw[gtu arrow] (root) -- (sym);
    \draw[gtu arrow] (root) -- (asym);
    \draw[gtu arrow] (sym) -- (same);
    \draw[gtu arrow] (same) -- (fast);
    \draw[gtu arrow] (asym) -- (pair);
    \draw[gtu arrow] (pair) -- (slow);
\end{tikzpicture}
\caption{એન્ક્રિપ્શન પદ્ધતિઓ}
\end{figure}

\textbf{વાસ્તવિક-દુનિયાના એપ્લિકેશન્સ:}
\begin{itemize}
    \item \textbf{બેંકિંગ}: ATM ટ્રાન્ઝેક્શન્સ (સિમેટ્રિક)
    \item \textbf{ઈ-કોમર્સ}: HTTPS (હાઇબ્રિડ)
    \item \textbf{ઈમેઇલ}: PGP (એસિમેટ્રિક)
    \item \textbf{મોબાઇલ}: WhatsApp (End-to-End)
\end{itemize}

\end{solutionbox}

\begin{mnemonicbox}
\mnemonic{Symmetric = Same Speed, Asymmetric = Advanced Security}
\end{mnemonicbox}

\end{document}
