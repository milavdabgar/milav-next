\documentclass{article}

% content/resources/templates/preamble.tex
\usepackage[margin=0.6in]{geometry}
\author{Milav Dabgar}
\usepackage{amsmath,amssymb,amsthm}
\usepackage{booktabs}
\usepackage{multirow}
\usepackage{xcolor}
\usepackage{tcolorbox}
\tcbuselibrary{breakable,skins}
\usepackage[colorlinks=true,linkcolor=blue]{hyperref}
\usepackage{titlesec}
\usepackage{enumitem}
\usepackage{tikz}
\usepackage{pgfplots}
\usepackage{circuitikz}
\usepackage[version=4]{mhchem}
\usepackage{longtable}
\usepackage{array}
\usepackage{float}
\usepackage{caption}
\usepackage{listings}

\lstset{
  basicstyle=\small\ttfamily,
  breaklines=true,
  breakatwhitespace=false,
  postbreak=\mbox{\textcolor{red}{$\hookrightarrow$}\space},
  float=false,
  numbers=left,
  numberstyle=\tiny\color{gray},
  numbersep=10pt,
  xleftmargin=2em,
  keywordstyle=\color{blue},
  commentstyle=\color{green!60!black},
  stringstyle=\color{purple},
  backgroundcolor=\color{gray!5},
  showstringspaces=false,
  tabsize=2,
  captionpos=b,
  keepspaces=true,
  columns=flexible
}

\pgfplotsset{compat=1.18}
\usetikzlibrary{shapes,arrows,positioning,calc,patterns,decorations.pathmorphing,decorations.markings,arrows.meta}

% Color scheme
\definecolor{headcolor}{RGB}{0,102,204}
\definecolor{keycolor}{RGB}{220,20,60}
\definecolor{solutioncolor}{RGB}{34,139,34}
\definecolor{mnemoniccolor}{RGB}{148,0,211}
\definecolor{codecolor}{RGB}{0,0,100}

% Spacing
\setlength{\parskip}{3pt}
\setlist[itemize]{nosep}
\setlist[enumerate]{nosep}

% Title formatting
\titleformat{\section}{\Large\bfseries\color{headcolor}}{\thesection}{1em}{}
\titleformat{\subsection}{\large\bfseries\color{headcolor}}{\thesubsection}{1em}{}

% Pandoc tightlist compatibility
\providecommand{\tightlist}{%
  \setlength{\itemsep}{0pt}\setlength{\parskip}{0pt}}

% Pandoc longtable compatibility
\newcounter{none}
\def\thenone{}


% content/resources/templates/gujarati-boxes.tex
\usepackage{fontspec}
\usepackage{polyglossia}

% Set Gujarati as main language (document is primarily in Gujarati)
% Note: gloss-gujarati.ldf doesn't exist in polyglossia, but it will use hyphenation patterns
\setdefaultlanguage{gujarati}
\setotherlanguage{english}

% Configure Gujarati font properly
% Use Language=Default to prevent polyglossia from trying to add language-specific features
% that don't exist for Gujarati, which causes "empty feature" warnings
\newfontfamily\gujaratifont[Script=Gujarati,AutoFakeBold=2.5,AutoFakeSlant=0.3]{Noto Sans Gujarati}
\setmainfont[Script=Gujarati,AutoFakeBold=2.5,AutoFakeSlant=0.3]{Noto Sans Gujarati}
% Use Noto Sans Gujarati for monospace to support Gujarati in text
\setmonofont[Scale=0.9]{Noto Sans Gujarati}

% Configure English to use the same font
\newfontfamily\englishfont[Script=Gujarati,AutoFakeBold=2.5,AutoFakeSlant=0.3]{Noto Sans Gujarati}

% Translations for polyglossia
\gappto\captionsgujarati{
  \renewcommand{\tablename}{કોષ્ટક}
  \renewcommand{\figurename}{આકૃતિ}
}

% Helper for TikZ nodes to ensure Gujarati font
\newcommand{\gu}[1]{{\gujaratifont #1}}

% Custom environments
\newtcolorbox{solutionbox}{
    breakable,
    enhanced,
    colback=solutioncolor!5!white,
    colframe=solutioncolor!75!black,
    fonttitle=\bfseries,
    title=જવાબ
}

\newtcolorbox{solutionboxnobreak}{
 colback=solutioncolor!5!white,
 colframe=solutioncolor!75!black,
 fonttitle=\bfseries,
 title=જવાબ
}

\newtcolorbox{keyformula}{
 breakable,
 enhanced,
 colback=keycolor!5!white,
 colframe=keycolor!75!black,
 fonttitle=\bfseries,
 title=રાસાયણિક સમીકરણ/સૂત્ર
}

\newtcolorbox{mnemonicbox}{
 breakable,
 enhanced,
 colback=mnemoniccolor!5!white,
 colframe=mnemoniccolor!75!black,
 fonttitle=\bfseries,
 title=મેમરી ટ્રીક
}


% Custom commands for GTU solutions
% This file defines semantic commands for consistent formatting

% Question command with automatic formatting
\newcommand{\question}[2]{%
  \section*{Question #1}%
  \textbf{#2}%
}

% OR question variant
\newcommand{\questionor}[2]{%
  \section*{Question #1 OR}%
  \textbf{#2}%
}

% Proper table environment with caption
\newenvironment{answertable}[1]{%
  \begin{table}[htbp]
  \centering
  \caption{#1}
}{%
  \end{table}
}

% Proper figure environment for diagrams
\newenvironment{answerdiagram}[1]{%
  \begin{figure}[htbp]
  \centering
  \caption{#1}
}{%
  \end{figure}
}

% Semantic markup for key terms
\newcommand{\keyword}[1]{\textbf{#1}}
\newcommand{\code}[1]{\texttt{#1}}
\newcommand{\classname}[1]{\texttt{#1}}
\newcommand{\methodname}[1]{\texttt{#1}}

% Proper quotation marks
\newcommand{\mnemonic}[1]{``#1''}


\title{Computer Networks \& Data Communication (4361101) - Winter 2024 Solution}
\date{November 19, 2024}

\begin{document}
\maketitle

\questionmarks{1(અ)}{3}{સ્ટાર ટોપોલોજીનું સવિસ્તાર વર્ણન કરો.}

\begin{solutionbox}
સ્ટાર ટોપોલોજીમાં બધા devices એક કેન્દ્રીય hub અથવા switch સાથે જોડાયેલા હોય છે. દરેક device નો કેન્દ્રીય device સાથે અલગ point-to-point connection હોય છે.

\textbf{આકૃતિ:}

\begin{figure}[H]
\centering
\begin{tikzpicture}[node distance=2cm, auto]
    \node (hub) [gtu block, circle, minimum size=1.5cm] {HUB};
    \node (pc1) [gtu block, above=of hub] {Computer A};
    \node (pc2) [gtu block, right=of hub] {Computer B};
    \node (pc3) [gtu block, below=of hub] {Computer C};
    \node (pc4) [gtu block, left=of hub] {Computer D};
    
    \draw[gtu arrow] (hub) -- (pc1);
    \draw[gtu arrow] (hub) -- (pc2);
    \draw[gtu arrow] (hub) -- (pc3);
    \draw[gtu arrow] (hub) -- (pc4);
\end{tikzpicture}
\caption{Star Topology}
\end{figure}

\textbf{મુખ્ય લક્ષણો:}

\begin{itemize}
    \item \textbf{કેન્દ્રીય હબ}: બધા connections કેન્દ્રીય device મારફતે પસાર થાય છે
    \item \textbf{સમર્પિત લિંક્સ}: દરેક node નો અલગ connection હોય છે
    \item \textbf{સરળ મેનેજમેન્ટ}: devices ને add/remove કરવું સરળ હોય છે
\end{itemize}

\end{solutionbox}

\begin{mnemonicbox}
\mnemonic{Star Shines Central - All devices connect to central point}
\end{mnemonicbox}

\questionmarks{1(બ)}{4}{ક્લાયન્ટ-સર્વર નેટવર્કનું વર્ણન કરો.}

\begin{solutionbox}
ક્લાયન્ટ-સર્વર એ network architecture છે જ્યાં clients કેન્દ્રીકૃત servers પાસેથી services માંગે છે. સર્વર અનેક clients ને resources અને services પ્રદાન કરે છે.

\textbf{કોષ્ટક: ક્લાયન્ટ vs સર્વર}

\begin{table}[H]
\centering
\begin{tabulary}{\textwidth}{L L}
\toprule
\textbf{ક્લાયન્ટ} & \textbf{સર્વર} \\
\midrule
Services માંગે છે & Services પ્રદાન કરે છે \\
મર્યાદિત resources & શક્તિશાળી hardware \\
સર્વર પર આધારિત & સ્વતંત્ર operation \\
\bottomrule
\end{tabulary}
\caption{Client vs Server}
\end{table}

\textbf{મુખ્ય ઘટકો:}

\begin{itemize}
    \item \textbf{ક્લાયન્ટ}: સર્વરથી data/services માંગે છે
    \item \textbf{સર્વર}: કેન્દ્રીકૃત resources અને processing પ્રદાન કરે છે
    \item \textbf{નેટવર્ક}: ક્લાયન્ટ-સર્વર વચ્ચે communication નું માધ્યમ
\end{itemize}

\end{solutionbox}

\begin{mnemonicbox}
\mnemonic{Client Calls, Server Serves}
\end{mnemonicbox}

\questionmarks{1(ક)}{7}{TCP/IP મોડેલના દરેક લેયરના કાર્ય સાથે વર્ણન કરો.}

\begin{solutionbox}
TCP/IP મોડેલમાં ચાર layers છે જે networks પર end-to-end communication પ્રદાન કરે છે.

\textbf{કોષ્ટક: TCP/IP મોડેલ લેયર્સ}

\begin{table}[H]
\centering
\begin{tabulary}{\textwidth}{L L L}
\toprule
\textbf{લેયર} & \textbf{કાર્ય} & \textbf{પ્રોટોકોલ્સ} \\
\midrule
Application & યુઝર interface, network services & HTTP, FTP, SMTP \\
Transport & End-to-end delivery, error control & TCP, UDP \\
Internet & Routing, logical addressing & IP, ICMP, ARP \\
Network Access & Physical transmission & Ethernet, WiFi \\
\bottomrule
\end{tabulary}
\caption{TCP/IP Model Layers}
\end{table}

\textbf{લેયર કાર્યો:}

\begin{itemize}
    \item \textbf{Application Layer}: યુઝર applications ને network services પ્રદાન કરે છે
    \item \textbf{Transport Layer}: processes વચ્ચે વિશ્વસનીય data delivery સુનિશ્ચિત કરે છે
    \item \textbf{Internet Layer}: IP વાપરીને multiple networks પર packets route કરે છે
    \item \textbf{Network Access Layer}: data નું physical transmission હેન્ડલ કરે છે
\end{itemize}

\end{solutionbox}

\begin{mnemonicbox}
\mnemonic{All Transport Internet Networks (ATIN)}
\end{mnemonicbox}

\questionmarks{1(ક OR)}{7}{OSI રેફરન્સ મોડેલના ડેટા લિંક લેયર અને નેટવર્ક લેયરની વિશેષતાઓ વર્ણવો.}

\begin{solutionbox}
ડેટા લિંક અને નેટવર્ક લેયર્સ OSI મોડેલમાં વિશ્વસનીય transmission અને routing capabilities પ્રદાન કરે છે.

\textbf{કોષ્ટક: લેયર તુલના}

\begin{table}[H]
\centering
\begin{tabulary}{\textwidth}{L L L}
\toprule
\textbf{લક્ષણ} & \textbf{ડેટા લિંક લેયર} & \textbf{નેટવર્ક લેયર} \\
\midrule
મુખ્ય કાર્ય & Node-to-node delivery & End-to-end delivery \\
Addressing & MAC addresses & IP addresses \\
Error Control & Frame-level & Packet-level \\
\bottomrule
\end{tabulary}
\caption{Layer Comparison}
\end{table}

\textbf{ડેટા લિંક લેયર કાર્યો:}

\begin{itemize}
    \item \textbf{Framing}: bits ને frames માં વ્યવસ્થિત કરે છે
    \item \textbf{Error Control}: transmission errors શોધે અને સુધારે છે
    \item \textbf{Flow Control}: data transmission rate મેનેજ કરે છે
\end{itemize}

\textbf{નેટવર્ક લેયર કાર્યો:}

\begin{itemize}
    \item \textbf{Routing}: packets માટે શ્રેષ્ઠ path નક્કી કરે છે
    \item \textbf{Logical Addressing}: identification માટે IP addresses વાપરે છે
    \item \textbf{Packet Forwarding}: networks વચ્ચે packets route કરે છે
\end{itemize}

\end{solutionbox}

\begin{mnemonicbox}
\mnemonic{Data Links Locally, Network Routes Globally}
\end{mnemonicbox}

\questionmarks{2(અ)}{3}{રિપીટર અને હબની સરખામણી કરો.}

\begin{solutionbox}
બંને devices signals ને amplify કરે છે પરંતુ network architecture માં અલગ રીતે કામ કરે છે.

\textbf{કોષ્ટક: રિપીટર vs હબ}

\begin{table}[H]
\centering
\begin{tabulary}{\textwidth}{L L L}
\toprule
\textbf{લક્ષણ} & \textbf{રિપીટર} & \textbf{હબ} \\
\midrule
પોર્ટ્સ & 2 ports & અનેક ports \\
કાર્ય & Signal amplification & Signal distribution \\
Collision Domain & એક & એક shared \\
\bottomrule
\end{tabulary}
\caption{Repeater vs Hub}
\end{table}

\textbf{મુખ્ય તફાવતો:}

\begin{itemize}
    \item \textbf{પોર્ટ કાઉન્ટ}: રિપીટરમાં 2 ports, હબમાં અનેક હોય છે
    \item \textbf{ઉપયોગ}: રિપીટર distance વધારે છે, હબ અનેક devices જોડે છે
\end{itemize}

\end{solutionbox}

\begin{mnemonicbox}
\mnemonic{Repeater Extends, Hub Connects}
\end{mnemonicbox}

\questionmarks{2(બ)}{4}{વાયરલેસ LAN નું વર્ણન કરો.}

\begin{solutionbox}
વાયરલેસ LAN ભૌતિક cables વિના network communication માટે radio waves વાપરે છે.

\textbf{આકૃતિ:}

\begin{figure}[H]
\centering
\begin{tikzpicture}[node distance=2.5cm, auto]
    \node (ap) [gtu block, circle, minimum size=1.5cm, fill=blue!10] {Access Point};
    \node (laptop) [gtu block, above left=of ap] {Laptop};
    \node (desktop) [gtu block, above right=of ap] {Desktop};
    \node (mobile) [gtu block, below left=of ap] {Mobile};
    \node (printer) [gtu block, below right=of ap] {Printer};
    
    \draw[dashed, gtu arrow, <->] (ap) -- (laptop) node[midway, sloped, above, font=\tiny] {WiFi};
    \draw[dashed, gtu arrow, <->] (ap) -- (desktop) node[midway, sloped, above, font=\tiny] {WiFi};
    \draw[dashed, gtu arrow, <->] (ap) -- (mobile) node[midway, sloped, above, font=\tiny] {WiFi};
    \draw[dashed, gtu arrow, <->] (ap) -- (printer) node[midway, sloped, above, font=\tiny] {WiFi};
\end{tikzpicture}
\caption{Wireless LAN Architecture}
\end{figure}

\textbf{મુખ્ય ઘટકો:}

\begin{itemize}
    \item \textbf{એક્સેસ પોઇન્ટ}: કેન્દ્રીય wireless communication device
    \item \textbf{વાયરલેસ ક્લાયન્ટ્સ}: WiFi capability વાળા devices
    \item \textbf{રેડિયો ફ્રીક્વન્સીઝ}: સામાન્ય રીતે 2.4GHz અને 5GHz bands વપરાય છે
\end{itemize}

\textbf{ફાયદાઓ:}

\begin{itemize}
    \item \textbf{ગતિશીલતા}: coverage area માં યુઝર્સ મુક્તપણે ફરી શકે છે
    \item \textbf{સરળ ઇન્સ્ટોલેશન}: ભૌતિક cable installation ની જરૂર નથી
\end{itemize}

\end{solutionbox}

\begin{mnemonicbox}
\mnemonic{Wireless Waves Connect}
\end{mnemonicbox}

\questionmarks{2(ક)}{7}{FDDI અને CDDI નું વર્ણન કરો.}

\begin{solutionbox}
FDDI અને CDDI ring-based network technologies છે જે high-speed data transmission પ્રદાન કરે છે.

\textbf{કોષ્ટક: FDDI vs CDDI તુલના}

\begin{table}[H]
\centering
\begin{tabulary}{\textwidth}{L L L}
\toprule
\textbf{લક્ષણ} & \textbf{FDDI} & \textbf{CDDI} \\
\midrule
માધ્યમ & Fiber optic & Copper (UTP) \\
ઝડપ & 100 Mbps & 100 Mbps \\
અંતર & 200 km & 100 meters \\
ખર્ચ & વધુ & ઓછો \\
\bottomrule
\end{tabulary}
\caption{FDDI vs CDDI}
\end{table}

\textbf{FDDI લક્ષણો:}

\begin{itemize}
    \item \textbf{દ્વિ રિંગ}: fault tolerance માટે primary અને secondary rings
    \item \textbf{ટોકન પાસિંગ}: deterministic access method
    \item \textbf{સ્વ-નિકાલ}: failures પાસેથી automatic recovery
\end{itemize}

\textbf{CDDI લક્ષણો:}

\begin{itemize}
    \item \textbf{કોપર માધ્યમ}: unshielded twisted pair cables વાપરે છે
    \item \textbf{સમાન પ્રોટોકોલ}: transmission medium સિવાય FDDI જેવું જ
    \item \textbf{કિંમત અસરકારક}: FDDI કરતાં ઓછી implementation cost
\end{itemize}

\textbf{રિંગ સ્ટ્રક્ચર:}

\begin{figure}[H]
\centering
\begin{tikzpicture}[node distance=2.5cm, auto]
    \node (sta) [gtu block] {Station A};
    \node (stb) [gtu block, right=of sta] {Station B};
    \node (stc) [gtu block, below=of stb] {Station C};
    \node (std) [gtu block, left=of stc] {Station D};
    
    % Primary Ring
    \draw[->, thick, blue] (sta) -- (stb);
    \draw[->, thick, blue] (stb) -- (stc);
    \draw[->, thick, blue] (stc) -- (std);
    \draw[->, thick, blue] (std) -- (sta);
    
    % Secondary Ring (Dashed, Counter-rotating)
    \draw[->, dashed, red] (stb) to[bend right=15] (sta);
    \draw[->, dashed, red] (sta) to[bend right=15] (std);
    \draw[->, dashed, red] (std) to[bend right=15] (stc);
    \draw[->, dashed, red] (stc) to[bend right=15] (stb);
    
    \node[align=center, font=\footnotesize] at (3.5, -1.5) {Blue: Primary Ring\\Red: Secondary Ring};
\end{tikzpicture}
\caption{FDDI Dual Ring Topology}
\end{figure}

\end{solutionbox}

\begin{mnemonicbox}
\mnemonic{FDDI Fiber Fast, CDDI Copper Cheap}
\end{mnemonicbox}

\questionmarks{2(અ OR)}{3}{ફાયરવોલ ડેટાને કેવી રીતે સુરક્ષિત કરે છે.}

\begin{solutionbox}
ફાયરવોલ વિશ્વસનીય આંતરિક network અને અવિશ્વસનીય બાહ્ય networks વચ્ચે security barrier તરીકે કામ કરે છે.

\textbf{સુરક્ષા પદ્ધતિઓ:}

\begin{itemize}
    \item \textbf{પેકેટ ફિલ્ટરિંગ}: security rules માટે packet headers તપાસે છે
    \item \textbf{એક્સેસ કંટ્રોલ}: અનધિકૃત access attempts ને block કરે છે
    \item \textbf{ટ્રાફિક મોનિટરિંગ}: બધા incoming અને outgoing traffic ની દેખરેખ કરે છે
\end{itemize}

\end{solutionbox}

\begin{mnemonicbox}
\mnemonic{Firewall Filters Foes}
\end{mnemonicbox}

\questionmarks{2(બ OR)}{4}{FDDI નું structure સમજાવો અને તેના ફાયદાઓ જણાવો.}

\begin{solutionbox}
FDDI high-speed, fault-tolerant networking માટે dual counter-rotating rings વાપરે છે.

\textbf{સ્ટ્રક્ચર ઘટકો:}

\begin{itemize}
    \item \textbf{પ્રાઇમરી રિંગ}: મુખ્ય data transmission path
    \item \textbf{સેકન્ડરી રિંગ}: fault recovery માટે backup path
    \item \textbf{ડ્યુઅલ એટેચમેન્ટ સ્ટેશન્સ}: બંને rings સાથે જોડાય છે
    \item \textbf{સિંગલ એટેચમેન્ટ સ્ટેશન્સ}: એક ring સાથે જ જોડાય છે
\end{itemize}

\textbf{ફાયદાઓ:}

\begin{itemize}
    \item \textbf{હાઇ સ્પીડ}: 100 Mbps transmission rate
    \item \textbf{ફોલ્ટ ટોલરન્સ}: secondary ring વાપરીને automatic recovery
    \item \textbf{લાંબુ અંતર}: 200 km સુધીના networks સાપોર્ટ કરે છે
\end{itemize}

\end{solutionbox}

\begin{mnemonicbox}
\mnemonic{FDDI Dual Rings Deliver Reliability}
\end{mnemonicbox}

\questionmarks{2(ક OR)}{7}{ઇથરનેટ, ફાસ્ટ ઇથરનેટ, ગીગાબીટ ઇથરનેટ સમજાવો અને સરખામણી કરો.}

\begin{solutionbox}
ઇથરનેટ standards નું વિકાસ વધતી bandwidth અને સુધારેલ performance પ્રદાન કરે છે.

\textbf{કોષ્ટક: ઇથરનેટ તુલના}

\begin{table}[H]
\centering
\begin{tabulary}{\textwidth}{L L L L}
\toprule
\textbf{લક્ષણ} & \textbf{ઇથરનેટ} & \textbf{ફાસ્ટ ઇથરનેટ} & \textbf{ગીગાબીટ ઇથરનેટ} \\
\midrule
ઝડપ & 10 Mbps & 100 Mbps & 1000 Mbps \\
સ્ટાન્ડર્ડ & 802.3 & 802.3u & 802.3z/ab \\
કેબલ & Coax/UTP & UTP/Fiber & UTP/Fiber \\
અંતર & 500m (coax) & 100m (UTP) & 100m (UTP) \\
\bottomrule
\end{tabulary}
\caption{Ethernet Comparison}
\end{table}

\textbf{મુખ્ય તફાવતો:}

\begin{itemize}
    \item \textbf{બેન્ડવિડ્થ}: દરેક generation ઝડપને 10 ના ફેક્ટરથી વધારે છે
    \item \textbf{મીડિયા સપોર્ટ}: નવા standards વધુ cable types સાપોર્ટ કરે છે
    \item \textbf{બેકવર્ડ કમ્પેટિબિલિટી}: ઉચ્ચ standards ઓછી ઝડપને સાપોર્ટ કરે છે
\end{itemize}

\textbf{એપ્લિકેશન્સ:}

\begin{itemize}
    \item \textbf{ઇથરનેટ}: legacy systems, basic connectivity
    \item \textbf{ફાસ્ટ ઇથરનેટ}: desktop connections, નાના networks
    \item \textbf{ગીગાબીટ ઇથરનેટ}: server connections, backbone networks
\end{itemize}

\end{solutionbox}

\begin{mnemonicbox}
\mnemonic{Ethernet Evolves: 10-100-1000}
\end{mnemonicbox}

\questionmarks{3(અ)}{3}{DSL ના પ્રકાર સમજાવો.}

\begin{solutionbox}
DSL વિદ્યમાન telephone lines પર અલગ frequency bands વાપરીને high-speed internet પ્રદાન કરે છે.

\textbf{કોષ્ટક: DSL પ્રકારો}

\begin{table}[H]
\centering
\begin{tabulary}{\textwidth}{L L L}
\toprule
\textbf{પ્રકાર} & \textbf{પૂરું નામ} & \textbf{ઝડપ} \\
\midrule
ADSL & Asymmetric DSL & 8 Mbps સુધી down \\
SDSL & Symmetric DSL & બરાબર up/down \\
VDSL & Very-high-bit-rate DSL & 52 Mbps સુધી \\
\bottomrule
\end{tabulary}
\caption{DSL Types}
\end{table}

\textbf{લાક્ષણિકતાઓ:}

\begin{itemize}
    \item \textbf{ADSL}: ઘરેલુ યુઝર્સ માટે અલગ upload/download ઝડપ
    \item \textbf{SDSL}: બિઝનેસ ઉપયોગ માટે બંને દિશામાં સમાન ઝડપ
\end{itemize}

\end{solutionbox}

\begin{mnemonicbox}
\mnemonic{DSL: અસમમિત, સમમિત, અતિ-ઝડપી}
\end{mnemonicbox}

\questionmarks{3(બ)}{4}{ARP અને RARP નું વર્ણન કરો.}

\begin{solutionbox}
ARP અને RARP IP અને MAC addresses વચ્ચે address resolution પ્રદાન કરે છે.

\textbf{કોષ્ટક: ARP vs RARP}

\begin{table}[H]
\centering
\begin{tabulary}{\textwidth}{L L L}
\toprule
\textbf{લક્ષણ} & \textbf{ARP} & \textbf{RARP} \\
\midrule
હેતુ & IP to MAC & MAC to IP \\
વપરાશકર્તા & બધા devices & Diskless workstations \\
દિશા & Logical to Physical & Physical to Logical \\
\bottomrule
\end{tabulary}
\caption{ARP vs RARP}
\end{table}

\textbf{ARP પ્રક્રિયા:}

\begin{itemize}
    \item \textbf{વિનંતી}: Broadcast "IP address X કોની પાસે છે?"
    \item \textbf{જવાબ}: લક્ષ્ય MAC address સાથે જવાબ આપે છે
    \item \textbf{કેશિંગ}: ARP table માં mapping સ્ટોર કરે છે
\end{itemize}

\textbf{RARP પ્રક્રિયા:}

\begin{itemize}
    \item \textbf{વિનંતી}: "મારું IP address શું છે?"
    \item \textbf{સર્વર જવાબ}: RARP સર્વર IP address પ્રદાન કરે છે
\end{itemize}

\end{solutionbox}

\begin{mnemonicbox}
\mnemonic{ARP: એડ્રેસ રિઝોલ્યુશન પ્રોટોકોલ, RARP: વિપરીત ARP}
\end{mnemonicbox}

\questionmarks{3(ક)}{7}{સર્કિટ સ્વિચિંગ અને પેકેટ સ્વિચિંગનું વર્ણન કરો.}

\begin{solutionbox}
નેટવર્ક્સમાં communication paths સ્થાપિત કરવાની બે મૂળભૂત પદ્ધતિઓ.

\textbf{કોષ્ટક: સર્કિટ vs પેકેટ સ્વિચિંગ}

\begin{table}[H]
\centering
\begin{tabulary}{\textwidth}{L L L}
\toprule
\textbf{લક્ષણ} & \textbf{સર્કિટ સ્વિચિંગ} & \textbf{પેકેટ સ્વિચિંગ} \\
\midrule
પાથ સેટઅપ & સમર્પિત path & સમર્પિત path નહીં \\
રિસોર્સ ઉપયોગ & આખા સમય દરમિયાન આરક્ષિત & ગતિશીલ રીતે shared \\
વિલંબ & સતત & પરિવર્તનશીલ \\
ઉદાહરણો & ટેલિફોન & ઇન્ટરનેટ \\
\bottomrule
\end{tabulary}
\caption{Circuit vs Packet Switching}
\end{table}

\textbf{સર્કિટ સ્વિચિંગ:}

\begin{itemize}
    \item \textbf{પાથ સ્થાપના}: communication પહેલાં સમર્પિત circuit બનાવાય છે
    \item \textbf{રિસોર્સ આરક્ષણ}: આખા session માટે bandwidth આરક્ષિત રહે છે
    \item \textbf{ગેરંટીડ સર્વિસ}: આખા connection દરમિયાન સતત performance
\end{itemize}

\textbf{પેકેટ સ્વિચિંગ:}

\begin{itemize}
    \item \textbf{સ્ટોર એન્ડ ફોરવર્ડ}: packets મધ્યવર્તી nodes પર અસ્થાયી રીતે સ્ટોર થાય છે
    \item \textbf{ડાયનેમિક રાઉટિંગ}: દરેક packet અલગ path લઈ શકે છે
    \item \textbf{રિસોર્સ શેરિંગ}: network resources અનેક connections વચ્ચે shared થાય છે
\end{itemize}

\textbf{આકૃતિ: પેકેટ સ્વિચિંગ}

\begin{figure}[H]
\centering
\begin{tikzpicture}[node distance=2.5cm, auto]
    \node (src) [gtu block] {Source};
    \node (r1) [gtu block, circle, minimum size=1cm, right=of src] {R1};
    \node (r2) [gtu block, circle, minimum size=1cm, right=of r1] {R2};
    \node (r3) [gtu block, circle, minimum size=1cm, below=of r1] {R3};
    \node (dest) [gtu block, right=of r2] {Destination};

    \draw[gtu arrow] (src) -- (r1);
    \draw[gtu arrow] (r1) -- (r2);
    \draw[gtu arrow] (r2) -- (dest);
    \draw[gtu arrow] (src) -- (r3);
    \draw[gtu arrow] (r3) -- (r2);
    
    \node [font=\scriptsize] at (3,-1.5) {Packets take different paths};
\end{tikzpicture}
\caption{Packet Switching}
\end{figure}

\end{solutionbox}

\begin{mnemonicbox}
\mnemonic{સર્કિટ પ્રતિબદ્ધ, પેકેટ વિભાજિત}
\end{mnemonicbox}

\questionmarks{3(અ OR)}{3}{DHCP અને BOOTP પ્રોટોકોલનું વર્ણન કરો.}

\begin{solutionbox}
બંને પ્રોટોકોલ્સ network devices ને આપમેળે IP addresses અસાઇન કરે છે.

\textbf{કોષ્ટક: DHCP vs BOOTP}

\begin{table}[H]
\centering
\begin{tabulary}{\textwidth}{L L L}
\toprule
\textbf{લક્ષણ} & \textbf{DHCP} & \textbf{BOOTP} \\
\midrule
એડ્રેસ પ્રકાર & ડાયનેમિક/સ્ટેટિક & માત્ર સ્ટેટિક \\
લીઝ ટાઇમ & અસ્થાયી & કાયમી \\
કોન્ફિગરેશન & આપમેળે & મેન્યુઅલ સેટઅપ \\
\bottomrule
\end{tabulary}
\caption{DHCP vs BOOTP}
\end{table}

\textbf{કાર્યો:}

\begin{itemize}
    \item \textbf{DHCP}: લીઝ મેનેજમેન્ટ સાથે ડાયનેમિક address assignment
    \item \textbf{BOOTP}: diskless workstations માટે bootstrap પ્રોટોકોલ
\end{itemize}

\end{solutionbox}

\begin{mnemonicbox}
\mnemonic{DHCP ડાયનેમિક, BOOTP બૂટસ્ટ્રેપ}
\end{mnemonicbox}

\questionmarks{3(બ OR)}{4}{IPv4 અને IPv6 પ્રોટોકોલનું વર્ણન કરો.}

\begin{solutionbox}
ઇન્ટરનેટ પ્રોટોકોલ versions addressing અને routing capabilities પ્રદાન કરે છે.

\textbf{કોષ્ટક: IPv4 vs IPv6}

\begin{table}[H]
\centering
\begin{tabulary}{\textwidth}{L L L}
\toprule
\textbf{લક્ષણ} & \textbf{IPv4} & \textbf{IPv6} \\
\midrule
એડ્રેસ સાઇઝ & 32 bits & 128 bits \\
એડ્રેસ ફોર્મેટ & ડોટેડ ડેસિમલ & હેક્સાડેસિમલ \\
એડ્રેસ સ્પેસ & 4.3 બિલિયન & 340 અંડેસિલિયન \\
હેડર સાઇઝ & 20-60 બાઇટ્સ & 40 બાઇટ્સ \\
\bottomrule
\end{tabulary}
\caption{IPv4 vs IPv6}
\end{table}

\textbf{IPv4 લક્ષણો:}

\begin{itemize}
    \item \textbf{એડ્રેસ ફોર્મેટ}: 192.168.1.1 (4 octets)
    \item \textbf{ક્લાસીસ}: A, B, C, D, E address classes
    \item \textbf{NAT જરૂરી}: address shortage માટે NAT જરૂરી
\end{itemize}

\textbf{IPv6 લક્ષણો:}

\begin{itemize}
    \item \textbf{એડ્રેસ ફોર્મેટ}: 2001:db8::1 (8 groups of 4 hex digits)
    \item \textbf{NAT ની જરૂર નથી}: પુષ્કળ address space
    \item \textbf{બિલ્ટ-ઇન સિક્યુરિટી}: IPSec સાપોર્ટ ફરજિયાત
\end{itemize}

\end{solutionbox}

\begin{mnemonicbox}
\mnemonic{IPv4 ચાર ઓક્ટેટ્સ, IPv6 સોળ બાઇટ્સ}
\end{mnemonicbox}

\questionmarks{3(ક OR)}{7}{ટ્વિસ્ટેડ જોડી કેબલ, કોએક્સિયલ કેબલ અને ફાઇબર ઓપ્ટિક કેબલની લેબલ સાથે બાંધકામ વિગતો દોરો અને સમજાવો.}

\begin{solutionbox}
guided transmission media ના ત્રણ મુખ્ય પ્રકારો અલગ construction અને characteristics સાથે.

\textbf{ટ્વિસ્ટેડ પેર કેબલ:}

\begin{figure}[H]
\centering
\begin{tikzpicture}[auto]
    \node (jacket) [circle, draw, minimum size=3cm, thick, fill=gray!20] {};
    \node (pair1) [circle, draw, minimum size=0.8cm, fill=white] at (-0.8,0.8) {Pair 1};
    \node (pair2) [circle, draw, minimum size=0.8cm, fill=white] at (0.8,0.8) {Pair 2};
    \node (pair3) [circle, draw, minimum size=0.8cm, fill=white] at (-0.8,-0.8) {Pair 3};
    \node (pair4) [circle, draw, minimum size=0.8cm, fill=white] at (0.8,-0.8) {Pair 4};
    \node [above=1.6cm of jacket] {Outer Jacket};
\end{tikzpicture}
\caption{Twisted Pair Cable Cross-section}
\end{figure}

\textbf{કોએક્સિયલ કેબલ:}

\begin{figure}[H]
\centering
\begin{tikzpicture}[auto]
    \draw[fill=black!10] (0,0) circle (2cm); \node at (0,1.5) {Outer Jacket};
    \draw[fill=gray!40] (0,0) circle (1.5cm); \node at (0,1) {Shield};
    \draw[fill=white] (0,0) circle (1cm); \node at (0,0.5) {Dielectric};
    \draw[fill=orange] (0,0) circle (0.3cm); \node at (0,0) {Core};
\end{tikzpicture}
\caption{Coaxial Cable Cross-section}
\end{figure}

\textbf{ફાઇબર ઓપ્ટિક કેબલ:}

\begin{figure}[H]
\centering
\begin{tikzpicture}[auto]
    \draw[fill=black!10] (0,0) circle (2cm); \node at (0,1.5) {Outer Jacket};
    \draw[fill=blue!10] (0,0) circle (1.2cm); \node at (0,0.8) {Cladding};
    \draw[fill=white] (0,0) circle (0.4cm); \node at (0,0) {Core};
\end{tikzpicture}
\caption{Fiber Optic Cable Cross-section}
\end{figure}

\textbf{બાંધકામ વિગતો:}

\begin{itemize}
    \item \textbf{ટ્વિસ્ટેડ પેર}: interference ઘટાડવા માટે copper wires twisted કરેલા
    \item \textbf{કોએક્સિયલ}: dielectric અને shield થી ઘેરાયેલું કેન્દ્રીય conductor
    \item \textbf{ફાઇબર ઓપ્ટિક}: total internal reflection માટે cladding સાથે glass core
\end{itemize}

\textbf{લાક્ષણિકતાઓ:}

\begin{itemize}
    \item \textbf{ટ્વિસ્ટેડ પેર}: ઓછો ખર્ચ, સરળ installation, મર્યાદિત bandwidth
    \item \textbf{કોએક્સિયલ}: વધુ સારી shielding, twisted pair કરતાં વધુ bandwidth
    \item \textbf{ફાઇબર ઓપ્ટિક}: સૌથી વધુ bandwidth, electromagnetic interference થી રક્ષિત
\end{itemize}

\end{solutionbox}

\begin{mnemonicbox}
\mnemonic{ટ્વિસ્ટેડ કોપર, કોએક્સ શીલ્ડેડ, ફાઇબર પ્રકાશ}
\end{mnemonicbox}

\questionmarks{4(અ)}{3}{કોઈપણ ત્રણ ડેટા લિંક લેયર પ્રોટોકોલને નામ આપો અને કોઈપણ એકને વિગતવાર સમજાવો.}

\begin{solutionbox}
સામાન્ય data link layer પ્રોટોકોલ્સ: HDLC, PPP, Ethernet.

\textbf{HDLC (High-Level Data Link Control):}

\begin{itemize}
    \item \textbf{ફ્રેમ સ્ટ્રક્ચર}: ફ્લેગ, એડ્રેસ, કંટ્રોલ, ડેટા, FCS, ફ્લેગ
    \item \textbf{એરર કંટ્રોલ}: sequence numbers અને acknowledgments વાપરે છે
    \item \textbf{ફ્લો કંટ્રોલ}: કાર્યક્ષમ transmission માટે sliding window પ્રોટોકોલ
\end{itemize}

\textbf{મુખ્ય લક્ષણો:}

\begin{itemize}
    \item \textbf{બિટ-ઓરિએન્ટેડ}: characters કરતાં bit streams સાથે કામ કરે છે
    \item \textbf{ફુલ-ડુપ્લેક્સ}: સાથે બંને દિશામાં communication
\end{itemize}

\end{solutionbox}

\begin{mnemonicbox}
\mnemonic{HDLC ડેટા લિંક કંટ્રોલ હેન્ડલ કરે}
\end{mnemonicbox}

\questionmarks{4(બ)}{4}{TCP અને UDP પ્રોટોકોલનું વર્ણન કરો.}

\begin{solutionbox}
ટ્રાન્સપોર્ટ લેયર પ્રોટોકોલ્સ અલગ સ્તરની સર્વિસ વિશ્વસનીયતા પ્રદાન કરે છે.

\textbf{કોષ્ટક: TCP vs UDP}

\begin{table}[H]
\centering
\begin{tabulary}{\textwidth}{L L L}
\toprule
\textbf{લક્ષણ} & \textbf{TCP} & \textbf{UDP} \\
\midrule
કનેક્શન & Connection-oriented & Connectionless \\
વિશ્વસનીયતા & વિશ્વસનીય & અવિશ્વસનીય \\
ઝડપ & ધીમું & ઝડપી \\
હેડર સાઇઝ & 20+ બાઇટ્સ & 8 બાઇટ્સ \\
\bottomrule
\end{tabulary}
\caption{TCP vs UDP}
\end{table}

\textbf{TCP લક્ષણો:}

\begin{itemize}
    \item \textbf{કનેક્શન સેટઅપ}: થ્રી-વે હેન્ડશેક connection સ્થાપિત કરે છે
    \item \textbf{એરર રિકવરી}: ખોવાયેલા packets આપમેળે ફરીથી મોકલે છે
    \item \textbf{ફ્લો કંટ્રોલ}: receiver ને overwhelm થવાથી બચાવે છે
\end{itemize}

\textbf{UDP લક્ષણો:}

\begin{itemize}
    \item \textbf{કનેક્શન નહીં}: connection સ્થાપિત કર્યા વિના data મોકલે છે
    \item \textbf{બેસ્ટ એફર્ટ}: delivery અથવા order ની કોઈ ગેરંટી નથી
    \item \textbf{લો ઓવરહેડ}: ઝડપી transmission માટે મિનિમલ હેડર
\end{itemize}

\end{solutionbox}

\begin{mnemonicbox}
\mnemonic{TCP વિશ્વસનીય, UDP અવિશ્વસનીય પણ ઝડપી}
\end{mnemonicbox}

\questionmarks{4(ક)}{7}{ઉદાહરણ સાથે VoIP નું વર્ણન કરો.}

\begin{solutionbox}
વૉઇસ ઓવર ઇન્ટરનેટ પ્રોટોકોલ પરંપરાગત ટેલિફોન સિસ્ટમ્સ બદલે IP networks પર voice communications ટ્રાન્સમિટ કરે છે.

\textbf{VoIP ઘટકો:}

\begin{itemize}
    \item \textbf{IP ફોન}: VoIP કૉલ્સ માટે હાર્ડવેર device
    \item \textbf{સોફ્ટફોન}: કમ્પ્યુટર-બેસ્ડ કૉલ્સ માટે સોફ્ટવેર એપ્લિકેશન
    \item \textbf{ગેટવે}: VoIP ને પરંપરાગત phone networks સાથે જોડે છે
    \item \textbf{PBX}: બિઝનેસ phone systems માટે પ્રાઇવેટ બ્રાન્ચ એક્સચેન્જ
\end{itemize}

\textbf{VoIP પ્રક્રિયા:}

\begin{enumerate}
    \item \textbf{વૉઇસ કેપ્ચર}: માઇક્રોફોન voice ને analog signal માં convert કરે છે
    \item \textbf{ડિજિટાઇઝેશન}: ADC analog ને digital samples માં convert કરે છે
    \item \textbf{કમ્પ્રેશન}: કોડેક audio data ને compress કરે છે
    \item \textbf{પેકેટાઇઝેશન}: voice data ને IP packets માં વિભાજિત કરે છે
    \item \textbf{ટ્રાન્સમિશન}: packets IP network પર મોકલવામાં આવે છે
    \item \textbf{પુનર્નિર્માણ}: receiving end audio ને reassemble અને play કરે છે
\end{enumerate}

\textbf{ઉદાહરણ એપ્લિકેશન્સ:}

\begin{itemize}
    \item \textbf{સ્કાઇપ}: વ્યક્તિગત કૉલ્સ માટે કન્ઝ્યુમર VoIP સર્વિસ
    \item \textbf{વોટ્સએપ કૉલિંગ}: મોબાઇલ VoIP એપ્લિકેશન
    \item \textbf{બિઝનેસ PBX}: VoIP વાપરતી કોર્પોરેટ phone systems
\end{itemize}

\textbf{ફાયદાઓ:}

\begin{itemize}
    \item \textbf{કિંમત અસરકારક}: લાંબા અંતરની કૉલ્સની ઓછી કિંમત
    \item \textbf{ફીચર રિચ}: વિડિયો કૉલિંગ, કોન્ફરન્સિંગ, કૉલ ફોરવર્ડિંગ
    \item \textbf{સ્કેલેબિલિટી}: નવા યુઝર્સ ઉમેરવા સરળ
\end{itemize}

\textbf{ગેરફાયદાઓ:}

\begin{itemize}
    \item \textbf{ઇન્ટરનેટ ડિપેન્ડન્સી}: સ્થિર ઇન્ટરનેટ કનેક્શનની જરૂર
    \item \textbf{ક્વોલિટી ઇશ્યુઝ}: network congestion થી સમસ્યા આવી શકે છે
    \item \textbf{પાવર ડિપેન્ડન્સી}: પરંપરાગત ફોન્સ વિપરીત વીજળીની જરૂર
\end{itemize}

\end{solutionbox}

\begin{mnemonicbox}
\mnemonic{VoIP: ઇન્ટરનેટ પ્રોટોકોલ પર વૉઇસ}
\end{mnemonicbox}

\questionmarks{4(અ OR)}{3}{DNS (ડોમેન નેમ સિસ્ટમ) નું વર્ણન કરો.}

\begin{solutionbox}
DNS માનવ-વાંચી શકાય તેવા domain names ને network communication માટે IP addresses માં translate કરે છે.

\textbf{DNS ઘટકો:}

\begin{itemize}
    \item \textbf{ડોમેન નેમ્સ}: હાયરાર્કિકલ નામકરણ સિસ્ટમ (www.example.com)
    \item \textbf{નેમ સર્વર્સ}: DNS records સ્ટોર કરતા કમ્પ્યુટર્સ
    \item \textbf{રિઝોલ્વર્સ}: DNS servers ને query કરતા ક્લાયન્ટ સોફ્ટવેર
\end{itemize}

\textbf{DNS પ્રક્રિયા:}

\begin{enumerate}
    \item યુઝર બ્રાઉઝરમાં domain name દાખલ કરે છે
    \item સ્થાનિક resolver DNS server ને query કરે છે
    \item DNS server અનુરૂપ IP address પરત કરે છે
\end{enumerate}

\end{solutionbox}

\begin{mnemonicbox}
\mnemonic{DNS: ડોમેન નેમ થી IP એડ્રેસ}
\end{mnemonicbox}

\questionmarks{4(બ OR)}{4}{DSL વિષે ટૂંકી નોંધ લખો.}

\begin{solutionbox}
ડિજિટલ સબ્સ્ક્રાઇબર લાઇન વિદ્યમાન ટેલિફોન infrastructure પર high-speed ઇન્ટરનેટ access પ્રદાન કરે છે.

\textbf{DSL ટેકનોલોજી:}

\begin{itemize}
    \item \textbf{ફ્રીક્વન્સી ડિવિઝન}: voice કૉલ્સ કરતાં વધુ ફ્રીક્વન્સીઝ વાપરે છે
    \item \textbf{સાથે ઉપયોગ}: ઇન્ટરનેટ અને ફોન એકસાથે કામ કરી શકે છે
    \item \textbf{અંતર મર્યાદા}: exchange પાસેથી અંતર સાથે performance ઘટે છે
\end{itemize}

\textbf{DSL પ્રકારો:}

\begin{itemize}
    \item \textbf{ADSL}: રહેવાસી યુઝર્સ માટે અસમમિત ઝડપ
    \item \textbf{SDSL}: બિઝનેસ એપ્લિકેશન્સ માટે સમમિત ઝડપ
    \item \textbf{VDSL}: ટૂંકા અંતર પર ખૂબ વધુ ઝડપ
\end{itemize}

\textbf{ફાયદાઓ:}

\begin{itemize}
    \item \textbf{વિદ્યમાન ઇન્ફ્રાસ્ટ્રક્ચર}: વિદ્યમાન ટેલિફોન લાઇન્સ વાપરે છે
    \item \textbf{હંમેશા ચાલુ}: સતત ઇન્ટરનેટ કનેક્શન
    \item \textbf{કિંમત અસરકારક}: સમર્પિત લાઇન્સ કરતાં ઓછો ખર્ચ
\end{itemize}

\end{solutionbox}

\begin{mnemonicbox}
\mnemonic{DSL: ફોન લાઇન્સ પર ડિજિટલ સબ્સ્ક્રાઇબર લાઇન}
\end{mnemonicbox}

\questionmarks{4(ક OR)}{7}{ફોરમ અને બ્લોગ્સ વિષે ટૂંકી નોંધ લખો.}

\begin{solutionbox}
માહિતી શેરિંગ અને સમુદાયિક ક્રિયાપ્રતિક્રિયા માટે ઓનલાઇન પ્લેટફોર્મ્સ.

\textbf{કોષ્ટક: ફોરમ vs બ્લોગ}

\begin{table}[H]
\centering
\begin{tabulary}{\textwidth}{L L L}
\toprule
\textbf{લક્ષણ} & \textbf{ફોરમ} & \textbf{બ્લોગ} \\
\midrule
સ્ટ્રક્ચર & ચર્ચા threads & કાલક્રમિક posts \\
મોડરેશન & સમુદાય દ્વારા મોડરેટ & લેખક દ્વારા નિયંત્રિત \\
હેતુ & સમુદાયિક સાપોર્ટ & માહિતી શેરિંગ \\
\bottomrule
\end{tabulary}
\caption{Forum vs Blog}
\end{table}

\textbf{ફોરમ લાક્ષણિકતાઓ:}

\begin{itemize}
    \item \textbf{ચર્ચા થ્રેડ્સ}: વિષય પ્રમાણે વ્યવસ્થિત ટોપિક્સ
    \item \textbf{યુઝર પાર્ટિસિપેશન}: અનેક યુઝર્સ ચર્ચામાં યોગદાન આપે છે
    \item \textbf{કેટેગરીઝ}: વિવિધ વિભાગોમાં ટોપિક્સ વ્યવસ્થિત
    \item \textbf{મોડરેશન}: સમુદાયિક નિયમો અને મોડરેટર્સ વ્યવસ્થા જાળવે છે
\end{itemize}

\textbf{બ્લોગ લાક્ષણિકતાઓ:}

\begin{itemize}
    \item \textbf{વ્યક્તિગત પબ્લિશિંગ}: વ્યક્તિ અથવા સંસ્થા content પ્રકાશિત કરે છે
    \item \textbf{કાલક્રમિક ક્રમ}: posts તારીખ પ્રમાણે દર્શાવવામાં આવે છે
    \item \textbf{ટિપ્પણીઓ}: વાચકો blog posts ને જવાબ આપી શકે છે
    \item \textbf{RSS ફીડ્સ}: વાચકો અપડેટ્સ માટે સબ્સ્ક્રાઇબ કરી શકે છે
\end{itemize}

\textbf{ઉદાહરણો:}

\begin{itemize}
    \item \textbf{ટેકનિકલ ફોરમ્સ}: પ્રોગ્રામિંગ પ્રશ્નો માટે Stack Overflow
    \item \textbf{કમ્યુનિટી ફોરમ્સ}: વિવિધ વિષયો માટે Reddit
    \item \textbf{વ્યક્તિગત બ્લોગ્સ}: અનુભવો શેર કરતી વ્યક્તિગત વેબસાઇટ્સ
    \item \textbf{કોર્પોરેટ બ્લોગ્સ}: માર્કેટિંગ અને અપડેટ્સ માટે કંપની બ્લોગ્સ
\end{itemize}

\textbf{ફાયદાઓ:}

\begin{itemize}
    \item \textbf{નોલેજ શેરિંગ}: યુઝર્સ નિપુણતા અને અનુભવો શેર કરે છે
    \item \textbf{કમ્યુનિટી બિલ્ડિંગ}: સામાન્ય રુચિઓવાળા લોકોને એકસાથે લાવે છે
    \item \textbf{પ્રોબ્લેમ સોલ્વિંગ}: ફોરમ્સ યુઝર્સને સોલ્યુશન્સ શોધવામાં મદદ કરે છે
    \item \textbf{કન્ટેન્ટ ક્રિએશન}: બ્લોગ્સ પ્રકાશન માટે પ્લેટફોર્મ પ્રદાન કરે છે
\end{itemize}

\end{solutionbox}

\begin{mnemonicbox}
\mnemonic{ફોરમ્સ ચર્ચા પ્રોત્સાહિત કરે, બ્લોગ્સ માહિતી પ્રસારિત કરે}
\end{mnemonicbox}

\questionmarks{5(અ)}{3}{"એન્ક્રિપ્શન" શબ્દોની વ્યાખ્યા કરો.}

\begin{solutionbox}
એન્ક્રિપ્શન અનધિકૃત access પાસેથી માહિતીને સુરક્ષિત કરવા માટે plaintext data ને ciphertext માં convert કરે છે.

\textbf{એન્ક્રિપ્શન પ્રક્રિયા:}

\begin{itemize}
    \item \textbf{પ્લેઇનટેક્સ્ટ}: મૂળ વાંચી શકાય તેવો ડેટા
    \item \textbf{અલ્ગોરિધમ}: transformation માટે ગાણિતિક પ્રક્રિયા
    \item \textbf{કી}: એન્ક્રિપ્શન અલ્ગોરિધમમાં વપરાતો ગુપ્ત પેરામીટર
    \item \textbf{સાઇફરટેક્સ્ટ}: એન્ક્રિપ્ટેડ વાંચી ન શકાય તેવો ડેટા
\end{itemize}

\textbf{હેતુ:}

\begin{itemize}
    \item \textbf{ગોપનીયતા}: અનધિકૃત ડેટા access અટકાવે છે
    \item \textbf{ડેટા પ્રોટેક્શન}: transmission દરમિયાન સંવેદનશીલ માહિતીને સુરક્ષિત કરે છે
\end{itemize}

\end{solutionbox}

\begin{mnemonicbox}
\mnemonic{એન્ક્રિપ્શન: કી સાથે પ્લેઇન થી સાઇફર}
\end{mnemonicbox}

\questionmarks{5(બ)}{4}{નીચેનામાંથી કોઈપણ બે સમજાવો: (1) WWW (2) FTP (3) SMTP}

\begin{solutionbox}

\textbf{WWW (વર્લ્ડ વાઇડ વેબ):}

\begin{itemize}
    \item \textbf{હાઇપરટેક્સ્ટ સિસ્ટમ}: હાઇપરલિંક્સ દ્વારા જોડાયેલા ડોક્યુમેન્ટ્સ
    \item \textbf{HTTP પ્રોટોકોલ}: વેબ કમ્યુનિકેશન માટે હાઇપરટેક્સ્ટ ટ્રાન્સફર પ્રોટોકોલ
    \item \textbf{વેબ બ્રાઉઝર}: વેબ પેજીસ access કરવા માટે ક્લાયન્ટ સોફ્ટવેર
    \item \textbf{વેબ સર્વર}: વેબસાઇટ્સ હોસ્ટ કરે છે અને વેબ પેજીસ સર્વ કરે છે
\end{itemize}

\textbf{FTP (ફાઇલ ટ્રાન્સફર પ્રોટોકોલ):}

\begin{itemize}
    \item \textbf{ફાઇલ ટ્રાન્સફર}: કમ્પ્યુટર્સ વચ્ચે ફાઇલો ટ્રાન્સફર કરવાનો પ્રોટોકોલ
    \item \textbf{ક્લાયન્ટ-સર્વર}: FTP ક્લાયન્ટ FTP સર્વર સાથે જોડાય છે
    \item \textbf{બે મોડ્સ}: ડેટા ટ્રાન્સફર માટે active અને passive મોડ્સ
    \item \textbf{ઓથેન્ટિકેશન}: access control માટે યુઝરનેમ અને પાસવર્ડ
\end{itemize}

\textbf{લક્ષણો:}

\begin{itemize}
    \item \textbf{WWW}: ગ્રાફિકલ ઇન્ટરફેસ, મલ્ટિમીડિયા સાપોર્ટ, હાઇપરલિંક્સ
    \item \textbf{FTP}: મોટી ફાઇલ ટ્રાન્સફર, ડિરેક્ટરી નેવિગેશન, resume capability
\end{itemize}

\end{solutionbox}

\begin{mnemonicbox}
\mnemonic{WWW: વેબ વર્લ્ડ વાઇડ, FTP: ફાઇલ ટ્રાન્સફર પ્રોટોકોલ}
\end{mnemonicbox}

\questionmarks{5(ક)}{7}{સિમેટ્રિક અને એસિમેટ્રિક એન્ક્રિપ્શન અલ્ગોરિધમ્સ વચ્ચેનો તફાવત}

\begin{solutionbox}
અલગ લાક્ષણિકતાઓ સાથે cryptographic key management ની બે મૂળભૂત પદ્ધતિઓ.

\textbf{કોષ્ટક: સિમેટ્રિક vs એસિમેટ્રિક એન્ક્રિપ્શન}

\begin{table}[H]
\centering
\begin{tabulary}{\textwidth}{L L L}
\toprule
\textbf{લક્ષણ} & \textbf{સિમેટ્રિક} & \textbf{એસિમેટ્રિક} \\
\midrule
કીઝ & એક shared કી & કી પેર (public/private) \\
ઝડપ & ઝડપી & ધીમું \\
કી ડિસ્ટ્રિબ્યુશન & મુશ્કેલ & સરળ \\
કી મેનેજમેન્ટ & મોટા ગ્રુપ્સ માટે જટિલ & સરળ \\
ઉદાહરણો & AES, DES & RSA, ECC \\
\bottomrule
\end{tabulary}
\caption{Symmetric vs Asymmetric Encryption}
\end{table}

\textbf{સિમેટ્રિક એન્ક્રિપ્શન:}

\begin{itemize}
    \item \textbf{સિંગલ કી}: એન્ક્રિપ્શન અને ડિક્રિપ્શન માટે સમાન કી વપરાય છે
    \item \textbf{ઝડપ}: સરળ અલ્ગોરિધમ્સને કારણે ઝડપી પ્રોસેસિંગ
    \item \textbf{કી શેરિંગ પ્રોબ્લેમ}: સુરક્ષિત કી વિતરણની પડકાર
    \item \textbf{સેશન કીઝ}: ઘણીવાર bulk data એન્ક્રિપ્શન માટે વપરાય છે
\end{itemize}

\textbf{એસિમેટ્રિક એન્ક્રિપ્શન:}

\begin{itemize}
    \item \textbf{કી પેર}: એન્ક્રિપ્શન માટે public કી, ડિક્રિપ્શન માટે private કી
    \item \textbf{ડિજિટલ સિગ્નેચર્સ}: private કી sign કરે છે, public કી verify કરે છે
    \item \textbf{કી એક્સચેન્જ}: કી વિતરણ સમસ્યાનું સમાધાન કરે છે
    \item \textbf{કમ્પ્યુટેશનલી ઇન્ટેન્સિવ}: સિમેટ્રિક એન્ક્રિપ્શન કરતાં ધીમું
\end{itemize}

\textbf{ઉપયોગ સ્થિતિઓ:}

\begin{itemize}
    \item \textbf{સિમેટ્રિક}: bulk data એન્ક્રિપ્શન, સુરક્ષિત communications
    \item \textbf{એસિમેટ્રિક}: કી એક્સચેન્જ, ડિજિટલ સિગ્નેચર્સ, ઓથેન્ટિકેશન
\end{itemize}

\textbf{હાઇબ્રિડ અભિગમ:}

\begin{itemize}
    \item \textbf{બંનેનું શ્રેષ્ઠ}: કી એક્સચેન્જ માટે એસિમેટ્રિક, ડેટા માટે સિમેટ્રિક
    \item \textbf{SSL/TLS}: સુરક્ષિત વેબ communications માટે બંને પ્રકારો વાપરે છે
\end{itemize}

\end{solutionbox}

\begin{mnemonicbox}
\mnemonic{સિમેટ્રિક સિંગલ કી, એસિમેટ્રિક કી પેર}
\end{mnemonicbox}

\questionmarks{5(અ OR)}{3}{સાયબર સિક્યુરિટી ઉપર ટૂંક નોંધ લખો.}

\begin{solutionbox}
સાયબર સિક્યુરિટી ડિજિટલ attacks અને અનધિકૃત access પાસેથી ડિજિટલ સિસ્ટમ્સ, નેટવર્ક્સ અને ડેટાને સુરક્ષિત કરે છે.

\textbf{મુખ્ય ઘટકો:}

\begin{itemize}
    \item \textbf{નેટવર્ક સિક્યુરિટી}: intrusions પાસેથી નેટવર્ક infrastructure ને સુરક્ષિત કરે છે
    \item \textbf{ડેટા પ્રોટેક્શન}: theft પાસેથી સંવેદનશીલ માહિતીને સુરક્ષિત રાખે છે
    \item \textbf{એપ્લિકેશન સિક્યુરિટી}: vulnerabilities પાસેથી સોફ્ટવેર એપ્લિકેશન્સને સુરક્ષિત કરે છે
\end{itemize}

\textbf{સામાન્ય ધમકીઓ:}

\begin{itemize}
    \item \textbf{મેલવેર}: સિસ્ટમ્સને નુકસાન પહોંચાડતા વાયરસ, worms, trojans
    \item \textbf{ફિશિંગ}: credentials ચોરવાના કપટપૂર્ણ પ્રયાસો
\end{itemize}

\end{solutionbox}

\begin{mnemonicbox}
\mnemonic{સાયબર સિક્યુરિટી: ડિજિટલ અસ્કયામતોને સુરક્ષિત કરો}
\end{mnemonicbox}

\questionmarks{5(બ OR)}{4}{હેકિંગ અને તેની સાવચેતીઓ સમજાવો.}

\begin{solutionbox}
હેકિંગમાં કમ્પ્યુટર સિસ્ટમ્સમાં અનધિકૃત access સામેલ છે, ઘણીવાર દુર્ભાવનાપૂર્ણ હેતુથી.

\textbf{હેકિંગના પ્રકારો:}

\begin{itemize}
    \item \textbf{વ્હાઇટ હેટ}: સિક્યુરિટી ટેસ્ટિંગ માટે નૈતિક હેકિંગ
    \item \textbf{બ્લેક હેટ}: ગેરકાયદેસર હેતુઓ માટે દુર્ભાવનાપૂર્ણ હેકિંગ
    \item \textbf{ગ્રે હેટ}: નૈતિક અને દુર્ભાવનાપૂર્ણ હેકિંગ વચ્ચે
\end{itemize}

\textbf{સામાન્ય હેકિંગ પદ્ધતિઓ:}

\begin{itemize}
    \item \textbf{પાસવર્ડ એટેક્સ}: બ્રુટ ફોર્સ, ડિક્શનરી attacks
    \item \textbf{સોશિયલ એન્જિનિયરિંગ}: માહિતી પ્રગટ કરવા માટે લોકોને ચાલાકીથી પ્રભાવિત કરવું
    \item \textbf{મેલવેર}: વાયરસ, trojans, ransomware
    \item \textbf{નેટવર્ક એટેક્સ}: મેન-ઇન-ધ-મિડલ, પેકેટ સ્નિફિંગ
\end{itemize}

\textbf{સાવચેતીઓ:}

\begin{itemize}
    \item \textbf{મજબૂત પાસવર્ડ્સ}: બધા એકાઉન્ટ્સ માટે જટિલ, અનન્ય પાસવર્ડ્સ
    \item \textbf{નિયમિત અપડેટ્સ}: સોફ્ટવેર અને સિસ્ટમ્સને અપડેટ રાખો
    \item \textbf{ફાયરવોલ}: અનધિકૃત access block કરવા માટે ફાયરવોલ વાપરો
    \item \textbf{એન્ટીવાયરસ}: એન્ટીવાયરસ સોફ્ટવેર નિયમિત ઇન્સ્ટોલ અને અપડેટ કરો
\end{itemize}

\end{solutionbox}

\begin{mnemonicbox}
\mnemonic{હેકિંગ નુકસાન કરે, સાવચેતીઓ સુરક્ષિત કરે}
\end{mnemonicbox}

\questionmarks{5(ક OR)}{7}{સંક્ષિપ્તમાં Information Technology (Amendment) Act 2008, અને ભારતમાં સાયબર કાયદાઓ પર તેની અસરનું વર્ણન કરો.}

\begin{solutionbox}
IT સુધારા કાયદો 2008 એ ભારતના સાયબર કાયદા ફ્રેમવર્કને નોંધપાત્ર રીતે મજબૂત બનાવ્યો અને સાયબર ક્રાઇમ કાયદાકીય વિસ્તારનો વિસ્તાર કર્યો.

\textbf{મુખ્ય સુધારાઓ:}

\begin{itemize}
    \item \textbf{ડેટા પ્રોટેક્શન}: સંવેદનશીલ વ્યક્તિગત ડેટાને સુરક્ષિત કરવા માટે વધારેલી જોગવાઈઓ
    \item \textbf{સાયબર ક્રાઇમ વ્યાખ્યાઓ}: identity theft સહિત સાયબર ક્રાઇમની વિસ્તૃત વ્યાખ્યાઓ
    \item \textbf{દંડ}: વિવિધ સાયબર અપરાધો માટે વધારેલા દંડ
    \item \textbf{સાયબર આતંકવાદ}: સાયબર આતંકવાદ સાથે વ્યવહાર કરવા માટે જોગવાઈઓ દાખલ કરી
\end{itemize}

\textbf{મુખ્ય જોગવાઈઓ:}

\begin{itemize}
    \item \textbf{કલમ 43A}: બેદરકારી માટે ડેટા પ્રોટેક્શન અને વળતર
    \item \textbf{કલમ 66A}: આક્રામક સંદેશાઓ માટે સજા (બાદમાં રદ કરાઈ)
    \item \textbf{કલમ 66C}: identity theft સજા
    \item \textbf{કલમ 66D}: કમ્પ્યુટર રિસોર્સ વાપરીને વ્યક્તિત્વ દ્વારા છેતરપિંડી
\end{itemize}

\textbf{સાયબર કાયદાઓ પર અસર:}

\begin{itemize}
    \item \textbf{કાયદાકીય માળખું}: સાયબર ક્રાઇમ માટે વ્યાપક કાયદાકીય માળખું પ્રદાન કર્યું
    \item \textbf{બિઝનેસ કમ્પ્લાયન્સ}: બિઝનેસ માટે ડેટા પ્રોટેક્શન પગલાં ફરજિયાત બનાવ્યા
    \item \textbf{કાયદા અમલીકરણ}: તપાસ સાધનો સાથે સત્તાવાળાઓને સશક્ત બનાવ્યા
    \item \textbf{આંતરરાષ્ટ્રીય સહયોગ}: સાયબર ક્રાઇમ તપાસમાં સહયોગને સરળ બનાવ્યું
\end{itemize}

\textbf{નિયમનકારી સંસ્થાઓ:}

\begin{itemize}
    \item \textbf{CERT-In}: ઘટના પ્રતિસાદ માટે કમ્પ્યુટર ઇમર્જન્સી રિસ્પોન્સ ટીમ
    \item \textbf{સાયબર સેલ્સ}: સાયબર ક્રાઇમ તપાસ માટે વિશેષ પોલીસ એકમો
    \item \textbf{એડજુડિકેટિંગ ઓફિસર્સ}: વળતર અને દંડ નિર્ધારણ માટે
\end{itemize}

\end{solutionbox}

\begin{mnemonicbox}
\mnemonic{IT એક્ટ 2008: ભારતના સાયબર કાયદાનો પાયો}
\end{mnemonicbox}

\end{document}
