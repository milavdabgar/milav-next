\documentclass[10pt,a4paper]{article}

% content/resources/templates/preamble.tex
\usepackage[margin=0.6in]{geometry}
\author{Milav Dabgar}
\usepackage{amsmath,amssymb,amsthm}
\usepackage{booktabs}
\usepackage{multirow}
\usepackage{xcolor}
\usepackage{tcolorbox}
\tcbuselibrary{breakable,skins}
\usepackage[colorlinks=true,linkcolor=blue]{hyperref}
\usepackage{titlesec}
\usepackage{enumitem}
\usepackage{tikz}
\usepackage{pgfplots}
\usepackage{circuitikz}
\usepackage[version=4]{mhchem}
\usepackage{longtable}
\usepackage{array}
\usepackage{float}
\usepackage{caption}
\usepackage{listings}

\lstset{
  basicstyle=\small\ttfamily,
  breaklines=true,
  breakatwhitespace=false,
  postbreak=\mbox{\textcolor{red}{$\hookrightarrow$}\space},
  float=false,
  numbers=left,
  numberstyle=\tiny\color{gray},
  numbersep=10pt,
  xleftmargin=2em,
  keywordstyle=\color{blue},
  commentstyle=\color{green!60!black},
  stringstyle=\color{purple},
  backgroundcolor=\color{gray!5},
  showstringspaces=false,
  tabsize=2,
  captionpos=b,
  keepspaces=true,
  columns=flexible
}

\pgfplotsset{compat=1.18}
\usetikzlibrary{shapes,arrows,positioning,calc,patterns,decorations.pathmorphing,decorations.markings,arrows.meta}

% Color scheme
\definecolor{headcolor}{RGB}{0,102,204}
\definecolor{keycolor}{RGB}{220,20,60}
\definecolor{solutioncolor}{RGB}{34,139,34}
\definecolor{mnemoniccolor}{RGB}{148,0,211}
\definecolor{codecolor}{RGB}{0,0,100}

% Spacing
\setlength{\parskip}{3pt}
\setlist[itemize]{nosep}
\setlist[enumerate]{nosep}

% Title formatting
\titleformat{\section}{\Large\bfseries\color{headcolor}}{\thesection}{1em}{}
\titleformat{\subsection}{\large\bfseries\color{headcolor}}{\thesubsection}{1em}{}

% Pandoc tightlist compatibility
\providecommand{\tightlist}{%
  \setlength{\itemsep}{0pt}\setlength{\parskip}{0pt}}

% Pandoc longtable compatibility
\newcounter{none}
\def\thenone{}


% content/resources/templates/english-boxes.tex
% This file is currently empty - it exists to maintain consistency with the import structure.
% Add custom environments here if needed in the future.


\begin{document}

\begin{center}
{\Huge\bfseries\color{headcolor} Subject Name Solutions}\\[5pt]
{\LARGE 4361101 -- Winter 2024}\\[3pt]
{\large Semester 1 Study Material}\\[3pt]
{\normalsize\textit{Detailed Solutions and Explanations}}
\end{center}

\vspace{10pt}

\subsection*{Question 1(a) [3 marks]}\label{q1a}

\textbf{Explain star topology in detail.}

\begin{solutionbox}
Star topology connects all devices to a central hub or
switch. Each device has dedicated point-to-point connection with central
device.

\textbf{Diagram:}

\begin{verbatim}
    Computer A
        |
        |
Computer D {-{-}{-}{-} HUB {-}{-}{-}{-} Computer B}
        |
        |
    Computer C
\end{verbatim}

\textbf{Key Features:}

\begin{itemize}
\tightlist
\item
  \textbf{Central Hub}: All connections pass through central device
\item
  \textbf{Dedicated Links}: Each node has separate connection
\item
  \textbf{Easy Management}: Simple to add/remove devices
\end{itemize}

\end{solutionbox}
\begin{mnemonicbox}
``Star Shines Central'' - All devices connect to
central point

\end{mnemonicbox}
\begin{center}\rule{0.5\linewidth}{0.5pt}\end{center}

\subsection*{Question 1(b) [4 marks]}\label{q1b}

\textbf{Explain client-server network.}

\begin{solutionbox}
Client-server is network architecture where clients
request services from centralized servers. Server provides resources and
services to multiple clients.


{\def\LTcaptype{none} % do not increment counter
\vspace{-5pt}
\captionof{table}{Client vs Server}
\vspace{-10pt}
\begin{longtable}[]{@{}ll@{}}
\toprule\noalign{}
Client & Server \\
\midrule\noalign{}
\endhead
\bottomrule\noalign{}
\endlastfoot
Requests services & Provides services \\
Limited resources & Powerful hardware \\
Depends on server & Independent operation \\
\end{longtable}
}

\textbf{Key Components:}

\begin{itemize}
\tightlist
\item
  \textbf{Client}: Requests data/services from server
\item
  \textbf{Server}: Provides centralized resources and processing
\item
  \textbf{Network}: Medium for communication between client-server
\end{itemize}

\end{solutionbox}
\begin{mnemonicbox}
``Client Calls, Server Serves''

\end{mnemonicbox}
\begin{center}\rule{0.5\linewidth}{0.5pt}\end{center}

\subsection*{Question 1(c) [7 marks]}\label{q1c}

\textbf{Write a functional description of all layer of TCP/IP model.}

\begin{solutionbox}
TCP/IP model has four layers providing end-to-end
communication over networks.


{\def\LTcaptype{none} % do not increment counter
\vspace{-5pt}
\captionof{table}{TCP/IP Model Layers}
\vspace{-10pt}
\begin{longtable}[]{@{}lll@{}}
\toprule\noalign{}
Layer & Function & Protocols \\
\midrule\noalign{}
\endhead
\bottomrule\noalign{}
\endlastfoot
Application & User interface, network services & HTTP, FTP, SMTP \\
Transport & End-to-end delivery, error control & TCP, UDP \\
Internet & Routing, logical addressing & IP, ICMP, ARP \\
Network Access & Physical transmission & Ethernet, WiFi \\
\end{longtable}
}

\textbf{Layer Functions:}

\begin{itemize}
\tightlist
\item
  \textbf{Application Layer}: Provides network services to user
  applications
\item
  \textbf{Transport Layer}: Ensures reliable data delivery between
  processes
\item
  \textbf{Internet Layer}: Routes packets across multiple networks using
  IP
\item
  \textbf{Network Access Layer}: Handles physical transmission of data
\end{itemize}

\end{solutionbox}
\begin{mnemonicbox}
``All Transport Internet Networks'' (ATIN)

\end{mnemonicbox}
\begin{center}\rule{0.5\linewidth}{0.5pt}\end{center}

\subsection*{Question 1(c OR) [7
marks]}\label{question-1c-or-7-marks}

\textbf{Explain the functions of Data Link Layer \& Network Layer of OSI
reference model.}

\begin{solutionbox}
Data Link and Network layers provide reliable
transmission and routing capabilities in OSI model.


{\def\LTcaptype{none} % do not increment counter
\vspace{-5pt}
\captionof{table}{Layer Comparison}
\vspace{-10pt}
\begin{longtable}[]{@{}lll@{}}
\toprule\noalign{}
Feature & Data Link Layer & Network Layer \\
\midrule\noalign{}
\endhead
\bottomrule\noalign{}
\endlastfoot
Main Function & Node-to-node delivery & End-to-end delivery \\
Addressing & MAC addresses & IP addresses \\
Error Control & Frame-level & Packet-level \\
\end{longtable}
}

\textbf{Data Link Layer Functions:}

\begin{itemize}
\tightlist
\item
  \textbf{Framing}: Organizes bits into frames
\item
  \textbf{Error Control}: Detects and corrects transmission errors
\item
  \textbf{Flow Control}: Manages data transmission rate
\end{itemize}

\textbf{Network Layer Functions:}

\begin{itemize}
\tightlist
\item
  \textbf{Routing}: Determines best path for packets
\item
  \textbf{Logical Addressing}: Uses IP addresses for identification
\item
  \textbf{Packet Forwarding}: Routes packets between networks
\end{itemize}

\end{solutionbox}
\begin{mnemonicbox}
``Data Links Locally, Network Routes Globally''

\end{mnemonicbox}
\begin{center}\rule{0.5\linewidth}{0.5pt}\end{center}

\subsection*{Question 2(a) [3 marks]}\label{q2a}

\textbf{Compare repeater and hub.}

\begin{solutionbox}
Both devices amplify signals but operate differently in
network architecture.


{\def\LTcaptype{none} % do not increment counter
\vspace{-5pt}
\captionof{table}{Repeater vs Hub}
\vspace{-10pt}
\begin{longtable}[]{@{}lll@{}}
\toprule\noalign{}
Feature & Repeater & Hub \\
\midrule\noalign{}
\endhead
\bottomrule\noalign{}
\endlastfoot
Ports & 2 ports & Multiple ports \\
Function & Signal amplification & Signal distribution \\
Collision Domain & Single & Single shared \\
\end{longtable}
}

\textbf{Key Differences:}

\begin{itemize}
\tightlist
\item
  \textbf{Port Count}: Repeater has 2 ports, hub has multiple
\item
  \textbf{Usage}: Repeater extends distance, hub connects multiple
  devices
\end{itemize}

\end{solutionbox}
\begin{mnemonicbox}
``Repeater Extends, Hub Connects''

\end{mnemonicbox}
\begin{center}\rule{0.5\linewidth}{0.5pt}\end{center}

\subsection*{Question 2(b) [4 marks]}\label{q2b}

\textbf{Explain wireless LAN.}

\begin{solutionbox}
Wireless LAN uses radio waves for network communication
without physical cables.

\textbf{Diagram:}

\begin{verbatim}
  Laptop     Desktop
    |   {   /   |}
    |    { /    |}
    |   Access  |
    |   Point   |
    |     |     |
    Mobile   Printer
\end{verbatim}

\textbf{Key Components:}

\begin{itemize}
\tightlist
\item
  \textbf{Access Point}: Central wireless communication device
\item
  \textbf{Wireless Clients}: Devices with WiFi capability
\item
  \textbf{Radio Frequencies}: 2.4GHz and 5GHz bands commonly used
\end{itemize}

\textbf{Advantages:}

\begin{itemize}
\tightlist
\item
  \textbf{Mobility}: Users can move freely within coverage area
\item
  \textbf{Easy Installation}: No physical cable installation required
\end{itemize}

\end{solutionbox}
\begin{mnemonicbox}
``Wireless Waves Connect''

\end{mnemonicbox}
\begin{center}\rule{0.5\linewidth}{0.5pt}\end{center}

\subsection*{Question 2(c) [7 marks]}\label{q2c}

\textbf{Explain FDDI \& CDDI.}

\begin{solutionbox}
FDDI and CDDI are ring-based network technologies
providing high-speed data transmission.


{\def\LTcaptype{none} % do not increment counter
\vspace{-5pt}
\captionof{table}{FDDI vs CDDI Comparison}
\vspace{-10pt}
\begin{longtable}[]{@{}lll@{}}
\toprule\noalign{}
Feature & FDDI & CDDI \\
\midrule\noalign{}
\endhead
\bottomrule\noalign{}
\endlastfoot
Medium & Fiber optic & Copper (UTP) \\
Speed & 100 Mbps & 100 Mbps \\
Distance & 200 km & 100 meters \\
Cost & High & Lower \\
\end{longtable}
}

\textbf{FDDI Features:}

\begin{itemize}
\tightlist
\item
  \textbf{Dual Ring}: Primary and secondary rings for fault tolerance
\item
  \textbf{Token Passing}: Deterministic access method
\item
  \textbf{Self-Healing}: Automatic recovery from failures
\end{itemize}

\textbf{CDDI Features:}

\begin{itemize}
\tightlist
\item
  \textbf{Copper Medium}: Uses unshielded twisted pair cables
\item
  \textbf{Same Protocol}: Identical to FDDI except transmission medium
\item
  \textbf{Cost Effective}: Lower implementation cost than FDDI
\end{itemize}

\textbf{Ring Structure:}

\begin{verbatim}
    Station A
        |
Station D {-{-}+{-}{-} Station B}
        |
    Station C
\end{verbatim}

\end{solutionbox}
\begin{mnemonicbox}
``FDDI Fiber Fast, CDDI Copper Cheap''

\end{mnemonicbox}
\begin{center}\rule{0.5\linewidth}{0.5pt}\end{center}

\subsection*{Question 2(a OR) [3
marks]}\label{question-2a-or-3-marks}

\textbf{How does a firewall protect data.}

\begin{solutionbox}
Firewall acts as security barrier between trusted
internal network and untrusted external networks.

\textbf{Protection Methods:}

\begin{itemize}
\tightlist
\item
  \textbf{Packet Filtering}: Examines packet headers for security rules
\item
  \textbf{Access Control}: Blocks unauthorized access attempts
\item
  \textbf{Traffic Monitoring}: Monitors all incoming and outgoing
  traffic
\end{itemize}

\end{solutionbox}
\begin{mnemonicbox}
``Firewall Filters Foes''

\end{mnemonicbox}
\begin{center}\rule{0.5\linewidth}{0.5pt}\end{center}

\subsection*{Question 2(b OR) [4
marks]}\label{question-2b-or-4-marks}

\textbf{Explain the structure of FDDI and give its advantages.}

\begin{solutionbox}
FDDI uses dual counter-rotating rings for high-speed,
fault-tolerant networking.

\textbf{Structure Components:}

\begin{itemize}
\tightlist
\item
  \textbf{Primary Ring}: Main data transmission path
\item
  \textbf{Secondary Ring}: Backup path for fault recovery
\item
  \textbf{Dual Attachment Stations}: Connect to both rings
\item
  \textbf{Single Attachment Stations}: Connect to one ring only
\end{itemize}

\textbf{Advantages:}

\begin{itemize}
\tightlist
\item
  \textbf{High Speed}: 100 Mbps transmission rate
\item
  \textbf{Fault Tolerance}: Automatic recovery using secondary ring
\item
  \textbf{Long Distance}: Supports up to 200 km networks
\end{itemize}

\end{solutionbox}
\begin{mnemonicbox}
``FDDI Dual Rings Deliver Reliability''

\end{mnemonicbox}
\begin{center}\rule{0.5\linewidth}{0.5pt}\end{center}

\subsection*{Question 2(c OR) [7
marks]}\label{question-2c-or-7-marks}

\textbf{Explain and distinguish Ethernet, Fast Ethernet, Gigabit
Ethernet.}

\begin{solutionbox}
Evolution of Ethernet standards providing increasing
bandwidth and improved performance.


{\def\LTcaptype{none} % do not increment counter
\vspace{-5pt}
\captionof{table}{Ethernet Comparison}
\vspace{-10pt}
\begin{longtable}[]{@{}llll@{}}
\toprule\noalign{}
Feature & Ethernet & Fast Ethernet & Gigabit Ethernet \\
\midrule\noalign{}
\endhead
\bottomrule\noalign{}
\endlastfoot
Speed & 10 Mbps & 100 Mbps & 1000 Mbps \\
Standard & 802.3 & 802.3u & 802.3z/ab \\
Cable & Coax/UTP & UTP/Fiber & UTP/Fiber \\
Distance & 500m (coax) & 100m (UTP) & 100m (UTP) \\
\end{longtable}
}

\textbf{Key Differences:}

\begin{itemize}
\tightlist
\item
  \textbf{Bandwidth}: Each generation increases speed by factor of 10
\item
  \textbf{Media Support}: Newer standards support more cable types
\item
  \textbf{Backward Compatibility}: Higher standards support lower speeds
\end{itemize}

\textbf{Applications:}

\begin{itemize}
\tightlist
\item
  \textbf{Ethernet}: Legacy systems, basic connectivity
\item
  \textbf{Fast Ethernet}: Desktop connections, small networks
\item
  \textbf{Gigabit Ethernet}: Server connections, backbone networks
\end{itemize}

\end{solutionbox}
\begin{mnemonicbox}
``Ethernet Evolves: 10-100-1000''

\end{mnemonicbox}
\begin{center}\rule{0.5\linewidth}{0.5pt}\end{center}

\subsection*{Question 3(a) [3 marks]}\label{q3a}

\textbf{Explain types of DSL.}

\begin{solutionbox}
DSL provides high-speed internet over existing
telephone lines using different frequency bands.


{\def\LTcaptype{none} % do not increment counter
\vspace{-5pt}
\captionof{table}{DSL Types}
\vspace{-10pt}
\begin{longtable}[]{@{}lll@{}}
\toprule\noalign{}
Type & Full Form & Speed \\
\midrule\noalign{}
\endhead
\bottomrule\noalign{}
\endlastfoot
ADSL & Asymmetric DSL & Up to 8 Mbps down \\
SDSL & Symmetric DSL & Equal up/down \\
VDSL & Very-high-bit-rate DSL & Up to 52 Mbps \\
\end{longtable}
}

\textbf{Characteristics:}

\begin{itemize}
\tightlist
\item
  \textbf{ADSL}: Different upload/download speeds for home users
\item
  \textbf{SDSL}: Same speed both directions for business use
\end{itemize}

\end{solutionbox}
\begin{mnemonicbox}
``DSL: Asymmetric, Symmetric, Very-fast''

\end{mnemonicbox}
\begin{center}\rule{0.5\linewidth}{0.5pt}\end{center}

\subsection*{Question 3(b) [4 marks]}\label{q3b}

\textbf{Explain ARP \& RARP.}

\begin{solutionbox}
ARP and RARP provide address resolution between IP and
MAC addresses.


{\def\LTcaptype{none} % do not increment counter
\vspace{-5pt}
\captionof{table}{ARP vs RARP}
\vspace{-10pt}
\begin{longtable}[]{@{}lll@{}}
\toprule\noalign{}
Feature & ARP & RARP \\
\midrule\noalign{}
\endhead
\bottomrule\noalign{}
\endlastfoot
Purpose & IP to MAC & MAC to IP \\
Used by & All devices & Diskless workstations \\
Direction & Logical to Physical & Physical to Logical \\
\end{longtable}
}

\textbf{ARP Process:}

\begin{itemize}
\tightlist
\item
  \textbf{Request}: Broadcast ``Who has IP address X?''
\item
  \textbf{Reply}: Target responds with MAC address
\item
  \textbf{Caching}: Stores mapping in ARP table
\end{itemize}

\textbf{RARP Process:}

\begin{itemize}
\tightlist
\item
  \textbf{Request}: ``What is my IP address?''
\item
  \textbf{Server Response}: RARP server provides IP address
\end{itemize}

\end{solutionbox}
\begin{mnemonicbox}
``ARP: Address Resolution Protocol, RARP: Reverse
ARP''

\end{mnemonicbox}
\begin{center}\rule{0.5\linewidth}{0.5pt}\end{center}

\subsection*{Question 3(c) [7 marks]}\label{q3c}

\textbf{Describe circuit switching and packet switching.}

\begin{solutionbox}
Two fundamental approaches for establishing
communication paths in networks.


{\def\LTcaptype{none} % do not increment counter
\vspace{-5pt}
\captionof{table}{Circuit vs Packet Switching}
\vspace{-10pt}
\begin{longtable}[]{@{}lll@{}}
\toprule\noalign{}
Feature & Circuit Switching & Packet Switching \\
\midrule\noalign{}
\endhead
\bottomrule\noalign{}
\endlastfoot
Path Setup & Dedicated path & No dedicated path \\
Resource Usage & Reserved throughout & Shared dynamically \\
Delay & Constant & Variable \\
Examples & Telephone & Internet \\
\end{longtable}
}

\textbf{Circuit Switching:}

\begin{itemize}
\tightlist
\item
  \textbf{Path Establishment}: Dedicated circuit created before
  communication
\item
  \textbf{Resource Reservation}: Bandwidth reserved for entire session
\item
  \textbf{Guaranteed Service}: Consistent performance throughout
  connection
\end{itemize}

\textbf{Packet Switching:}

\begin{itemize}
\tightlist
\item
  \textbf{Store and Forward}: Packets stored temporarily at intermediate
  nodes
\item
  \textbf{Dynamic Routing}: Each packet can take different path
\item
  \textbf{Resource Sharing}: Network resources shared among multiple
  connections
\end{itemize}

\textbf{Diagram: Packet Switching}

\begin{verbatim}
Source {-{-}{-} Router1 {-}{-}{-} Router2 {-}{-}{-} Destination}
    {         |         /}
     {        |        /}
      {{-}{-}{-} Router3 {-}{-}{-}/}
\end{verbatim}

\end{solutionbox}
\begin{mnemonicbox}
``Circuit Commits, Packet Partitions''

\end{mnemonicbox}
\begin{center}\rule{0.5\linewidth}{0.5pt}\end{center}

\subsection*{Question 3(a OR) [3
marks]}\label{question-3a-or-3-marks}

\textbf{Describe DHCP \& BOOTP protocol.}

\begin{solutionbox}
Both protocols automatically assign IP addresses to
network devices.


{\def\LTcaptype{none} % do not increment counter
\vspace{-5pt}
\captionof{table}{DHCP vs BOOTP}
\vspace{-10pt}
\begin{longtable}[]{@{}lll@{}}
\toprule\noalign{}
Feature & DHCP & BOOTP \\
\midrule\noalign{}
\endhead
\bottomrule\noalign{}
\endlastfoot
Address Type & Dynamic/Static & Static only \\
Lease Time & Temporary & Permanent \\
Configuration & Automatic & Manual setup \\
\end{longtable}
}

\textbf{Functions:}

\begin{itemize}
\tightlist
\item
  \textbf{DHCP}: Dynamic address assignment with lease management
\item
  \textbf{BOOTP}: Bootstrap protocol for diskless workstations
\end{itemize}

\end{solutionbox}
\begin{mnemonicbox}
``DHCP Dynamic, BOOTP Bootstrap''

\end{mnemonicbox}
\begin{center}\rule{0.5\linewidth}{0.5pt}\end{center}

\subsection*{Question 3(b OR) [4
marks]}\label{question-3b-or-4-marks}

\textbf{Explain IPv4 \& IPv6 in detail.}

\begin{solutionbox}
Internet Protocol versions providing addressing and
routing capabilities.


{\def\LTcaptype{none} % do not increment counter
\vspace{-5pt}
\captionof{table}{IPv4 vs IPv6}
\vspace{-10pt}
\begin{longtable}[]{@{}lll@{}}
\toprule\noalign{}
Feature & IPv4 & IPv6 \\
\midrule\noalign{}
\endhead
\bottomrule\noalign{}
\endlastfoot
Address Size & 32 bits & 128 bits \\
Address Format & Dotted decimal & Hexadecimal \\
Address Space & 4.3 billion & 340 undecillion \\
Header Size & 20-60 bytes & 40 bytes \\
\end{longtable}
}

\textbf{IPv4 Features:}

\begin{itemize}
\tightlist
\item
  \textbf{Address Format}: 192.168.1.1 (4 octets)
\item
  \textbf{Classes}: A, B, C, D, E address classes
\item
  \textbf{NAT Required}: Address shortage requires NAT
\end{itemize}

\textbf{IPv6 Features:}

\begin{itemize}
\tightlist
\item
  \textbf{Address Format}: 2001:db8::1 (8 groups of 4 hex digits)
\item
  \textbf{No NAT Needed}: Abundant address space
\item
  \textbf{Built-in Security}: IPSec support mandatory
\end{itemize}

\end{solutionbox}
\begin{mnemonicbox}
``IPv4 Four Octets, IPv6 Six-teen Bytes''

\end{mnemonicbox}
\begin{center}\rule{0.5\linewidth}{0.5pt}\end{center}

\subsection*{Question 3(c OR) [7
marks]}\label{question-3c-or-7-marks}

\textbf{Draw and explain constructional details of twisted pair cable,
coaxial cable, and fiber optic cable with label.}

\begin{solutionbox}
Three main types of guided transmission media with
different construction and characteristics.

\textbf{Twisted Pair Cable:}

\begin{verbatim}
  Outer Jacket
       |
   +{-{-}{-}+{-}{-}{-}+}
   |  / {  |  Twisted Pairs}
   | /   { |  (4 pairs)}
   +{-{-}{-}{-}{-}{-}{-}+}
       |
   Insulation
\end{verbatim}

\textbf{Coaxial Cable:}

\begin{verbatim}
   Outer Jacket
        |
    +{-{-}{-}+{-}{-}{-}+}
    |   |   |  Outer Conductor (Shield)
    | +{-+{-}+ |  Dielectric Insulator}
    | | | | |  Inner Conductor (Copper)
    +{-+{-}+{-}+{-}+}
\end{verbatim}

\textbf{Fiber Optic Cable:}

\begin{verbatim}
   Outer Jacket
        |
    +{-{-}{-}+{-}{-}{-}+}
    |   |   |  Cladding
    | +{-+{-}+ |  Core (Glass/Plastic)}
    |   |   |  Light travels here
    +{-{-}{-}+{-}{-}{-}+}
\end{verbatim}

\textbf{Construction Details:}

\begin{itemize}
\tightlist
\item
  \textbf{Twisted Pair}: Copper wires twisted to reduce interference
\item
  \textbf{Coaxial}: Central conductor surrounded by dielectric and
  shield
\item
  \textbf{Fiber Optic}: Glass core with cladding for total internal
  reflection
\end{itemize}

\textbf{Characteristics:}

\begin{itemize}
\tightlist
\item
  \textbf{Twisted Pair}: Low cost, easy installation, limited bandwidth
\item
  \textbf{Coaxial}: Better shielding, higher bandwidth than twisted pair
\item
  \textbf{Fiber Optic}: Highest bandwidth, immune to electromagnetic
  interference
\end{itemize}

\end{solutionbox}
\begin{mnemonicbox}
``Twisted Copper, Coax Shielded, Fiber Light''

\end{mnemonicbox}
\begin{center}\rule{0.5\linewidth}{0.5pt}\end{center}

\subsection*{Question 4(a) [3 marks]}\label{q4a}

\textbf{Name any three data link layer protocol and explain any one in
detail.}

\begin{solutionbox}
Common data link layer protocols: HDLC, PPP, Ethernet.

\textbf{HDLC (High-Level Data Link Control):}

\begin{itemize}
\tightlist
\item
  \textbf{Frame Structure}: Flag, Address, Control, Data, FCS, Flag
\item
  \textbf{Error Control}: Uses sequence numbers and acknowledgments
\item
  \textbf{Flow Control}: Sliding window protocol for efficient
  transmission
\end{itemize}

\textbf{Key Features:}

\begin{itemize}
\tightlist
\item
  \textbf{Bit-oriented}: Works with bit streams rather than characters
\item
  \textbf{Full-duplex}: Simultaneous bidirectional communication
\end{itemize}

\end{solutionbox}
\begin{mnemonicbox}
``HDLC Handles Data Link Control''

\end{mnemonicbox}
\begin{center}\rule{0.5\linewidth}{0.5pt}\end{center}

\subsection*{Question 4(b) [4 marks]}\label{q4b}

\textbf{Explain TCP and UDP protocol.}

\begin{solutionbox}
Transport layer protocols providing different levels of
service reliability.


{\def\LTcaptype{none} % do not increment counter
\vspace{-5pt}
\captionof{table}{TCP vs UDP}
\vspace{-10pt}
\begin{longtable}[]{@{}lll@{}}
\toprule\noalign{}
Feature & TCP & UDP \\
\midrule\noalign{}
\endhead
\bottomrule\noalign{}
\endlastfoot
Connection & Connection-oriented & Connectionless \\
Reliability & Reliable & Unreliable \\
Speed & Slower & Faster \\
Header Size & 20+ bytes & 8 bytes \\
\end{longtable}
}

\textbf{TCP Features:}

\begin{itemize}
\tightlist
\item
  \textbf{Connection Setup}: Three-way handshake establishes connection
\item
  \textbf{Error Recovery}: Retransmits lost packets automatically
\item
  \textbf{Flow Control}: Prevents sender from overwhelming receiver
\end{itemize}

\textbf{UDP Features:}

\begin{itemize}
\tightlist
\item
  \textbf{No Connection}: Sends data without establishing connection
\item
  \textbf{Best Effort}: No guarantee of delivery or order
\item
  \textbf{Low Overhead}: Minimal header for fast transmission
\end{itemize}

\end{solutionbox}
\begin{mnemonicbox}
``TCP Trustworthy, UDP Unreliable but Quick''

\end{mnemonicbox}
\begin{center}\rule{0.5\linewidth}{0.5pt}\end{center}

\subsection*{Question 4(c) [7 marks]}\label{q4c}

\textbf{Describe VoIP with example.}

\begin{solutionbox}
Voice over Internet Protocol transmits voice
communications over IP networks instead of traditional telephone
systems.

\textbf{VoIP Components:}

\begin{itemize}
\tightlist
\item
  \textbf{IP Phone}: Hardware device for VoIP calls
\item
  \textbf{Softphone}: Software application for computer-based calls
\item
  \textbf{Gateway}: Connects VoIP to traditional phone networks
\item
  \textbf{PBX}: Private branch exchange for business phone systems
\end{itemize}

\textbf{VoIP Process:}

\begin{enumerate}
\tightlist
\item
  \textbf{Voice Capture}: Microphone converts voice to analog signal
\item
  \textbf{Digitization}: ADC converts analog to digital samples
\item
  \textbf{Compression}: Codec compresses audio data
\item
  \textbf{Packetization}: Voice data divided into IP packets
\item
  \textbf{Transmission}: Packets sent over IP network
\item
  \textbf{Reconstruction}: Receiving end reassembles and plays audio
\end{enumerate}

\textbf{Example Applications:}

\begin{itemize}
\tightlist
\item
  \textbf{Skype}: Consumer VoIP service for personal calls
\item
  \textbf{WhatsApp Calling}: Mobile VoIP application
\item
  \textbf{Business PBX}: Corporate phone systems using VoIP
\end{itemize}

\textbf{Advantages:}

\begin{itemize}
\tightlist
\item
  \textbf{Cost Effective}: Lower long-distance call costs
\item
  \textbf{Feature Rich}: Video calling, conferencing, call forwarding
\item
  \textbf{Scalability}: Easy to add new users
\end{itemize}

\textbf{Disadvantages:}

\begin{itemize}
\tightlist
\item
  \textbf{Internet Dependency}: Requires stable internet connection
\item
  \textbf{Quality Issues}: May suffer from network congestion
\item
  \textbf{Power Dependency}: Requires electricity unlike traditional
  phones
\end{itemize}

\end{solutionbox}
\begin{mnemonicbox}
``VoIP: Voice over Internet Protocol''

\end{mnemonicbox}
\begin{center}\rule{0.5\linewidth}{0.5pt}\end{center}

\subsection*{Question 4(a OR) [3
marks]}\label{question-4a-or-3-marks}

\textbf{Explain DNS (Domain Name System).}

\begin{solutionbox}
DNS translates human-readable domain names into IP
addresses for network communication.

\textbf{DNS Components:}

\begin{itemize}
\tightlist
\item
  \textbf{Domain Names}: Hierarchical naming system (www.example.com)
\item
  \textbf{Name Servers}: Computers that store DNS records
\item
  \textbf{Resolvers}: Client software that queries DNS servers
\end{itemize}

\textbf{DNS Process:}

\begin{enumerate}
\tightlist
\item
  User enters domain name in browser
\item
  Local resolver queries DNS server
\item
  DNS server returns corresponding IP address
\end{enumerate}

\end{solutionbox}
\begin{mnemonicbox}
``DNS: Domain Name to IP Address''

\end{mnemonicbox}
\begin{center}\rule{0.5\linewidth}{0.5pt}\end{center}

\subsection*{Question 4(b OR) [4
marks]}\label{question-4b-or-4-marks}

\textbf{Write a short note on DSL.}

\begin{solutionbox}
Digital Subscriber Line provides high-speed internet
access over existing telephone infrastructure.

\textbf{DSL Technology:}

\begin{itemize}
\tightlist
\item
  \textbf{Frequency Division}: Uses higher frequencies than voice calls
\item
  \textbf{Simultaneous Use}: Internet and phone can work together
\item
  \textbf{Distance Limitation}: Performance decreases with distance from
  exchange
\end{itemize}

\textbf{DSL Types:}

\begin{itemize}
\tightlist
\item
  \textbf{ADSL}: Asymmetric speeds for residential users
\item
  \textbf{SDSL}: Symmetric speeds for business applications
\item
  \textbf{VDSL}: Very high speeds over short distances
\end{itemize}

\textbf{Advantages:}

\begin{itemize}
\tightlist
\item
  \textbf{Existing Infrastructure}: Uses existing telephone lines
\item
  \textbf{Always On}: Continuous internet connection
\item
  \textbf{Cost Effective}: Lower cost than dedicated lines
\end{itemize}

\end{solutionbox}
\begin{mnemonicbox}
``DSL: Digital Subscriber Line over Phone Lines''

\end{mnemonicbox}
\begin{center}\rule{0.5\linewidth}{0.5pt}\end{center}

\subsection*{Question 4(c OR) [7
marks]}\label{question-4c-or-7-marks}

\textbf{Explain forum and blogs with example.}

\begin{solutionbox}
Online platforms for information sharing and community
interaction.


{\def\LTcaptype{none} % do not increment counter
\vspace{-5pt}
\captionof{table}{Forum vs Blog}
\vspace{-10pt}
\begin{longtable}[]{@{}lll@{}}
\toprule\noalign{}
Feature & Forum & Blog \\
\midrule\noalign{}
\endhead
\bottomrule\noalign{}
\endlastfoot
Structure & Discussion threads & Chronological posts \\
Interaction & Multi-user discussions & Comments on posts \\
Moderation & Community moderated & Author controlled \\
Purpose & Community support & Information sharing \\
\end{longtable}
}

\textbf{Forum Characteristics:}

\begin{itemize}
\tightlist
\item
  \textbf{Discussion Threads}: Topics organized by subject
\item
  \textbf{User Participation}: Multiple users contribute to discussions
\item
  \textbf{Categories}: Topics organized into different sections
\item
  \textbf{Moderation}: Community rules and moderators maintain order
\end{itemize}

\textbf{Blog Characteristics:}

\begin{itemize}
\tightlist
\item
  \textbf{Personal Publishing}: Individual or organization publishes
  content
\item
  \textbf{Chronological Order}: Posts displayed by date
\item
  \textbf{Comments}: Readers can respond to blog posts
\item
  \textbf{RSS Feeds}: Readers can subscribe to updates
\end{itemize}

\textbf{Examples:}

\begin{itemize}
\tightlist
\item
  \textbf{Technical Forums}: Stack Overflow for programming questions
\item
  \textbf{Community Forums}: Reddit for diverse topics
\item
  \textbf{Personal Blogs}: Individual websites sharing experiences
\item
  \textbf{Corporate Blogs}: Company blogs for marketing and updates
\end{itemize}

\textbf{Benefits:}

\begin{itemize}
\tightlist
\item
  \textbf{Knowledge Sharing}: Users share expertise and experiences
\item
  \textbf{Community Building}: Brings together people with common
  interests
\item
  \textbf{Problem Solving}: Forums help users find solutions
\item
  \textbf{Content Creation}: Blogs provide platform for publishing
\end{itemize}

\end{solutionbox}
\begin{mnemonicbox}
``Forums Foster Discussion, Blogs Broadcast
Information''

\end{mnemonicbox}
\begin{center}\rule{0.5\linewidth}{0.5pt}\end{center}

\subsection*{Question 5(a) [3 marks]}\label{q5a}

\textbf{Define the terms ``encryption''.}

\begin{solutionbox}
Encryption converts plaintext data into ciphertext to
protect information from unauthorized access.

\textbf{Encryption Process:}

\begin{itemize}
\tightlist
\item
  \textbf{Plaintext}: Original readable data
\item
  \textbf{Algorithm}: Mathematical process for transformation
\item
  \textbf{Key}: Secret parameter used in encryption algorithm
\item
  \textbf{Ciphertext}: Encrypted unreadable data
\end{itemize}

\textbf{Purpose:}

\begin{itemize}
\tightlist
\item
  \textbf{Confidentiality}: Prevents unauthorized data access
\item
  \textbf{Data Protection}: Secures sensitive information during
  transmission
\end{itemize}

\end{solutionbox}
\begin{mnemonicbox}
``Encryption: Plain to Cipher with Key''

\end{mnemonicbox}
\begin{center}\rule{0.5\linewidth}{0.5pt}\end{center}

\subsection*{Question 5(b) [4 marks]}\label{q5b}

\textbf{Explain any two of following: (1) WWW (2) FTP (3) SMTP}

\begin{solutionbox}

\textbf{WWW (World Wide Web):}

\begin{itemize}
\tightlist
\item
  \textbf{Hypertext System}: Documents linked through hyperlinks
\item
  \textbf{HTTP Protocol}: HyperText Transfer Protocol for web
  communication
\item
  \textbf{Web Browser}: Client software for accessing web pages
\item
  \textbf{Web Server}: Hosts websites and serves web pages
\end{itemize}

\textbf{FTP (File Transfer Protocol):}

\begin{itemize}
\tightlist
\item
  \textbf{File Transfer}: Protocol for transferring files between
  computers
\item
  \textbf{Client-Server}: FTP client connects to FTP server
\item
  \textbf{Two Modes}: Active and passive modes for data transfer
\item
  \textbf{Authentication}: Username and password for access control
\end{itemize}

\textbf{Features:}

\begin{itemize}
\tightlist
\item
  \textbf{WWW}: Graphical interface, multimedia support, hyperlinks
\item
  \textbf{FTP}: Large file transfer, directory navigation, resume
  capability
\end{itemize}

\end{solutionbox}
\begin{mnemonicbox}
``WWW: Web World Wide, FTP: File Transfer Protocol''

\end{mnemonicbox}
\begin{center}\rule{0.5\linewidth}{0.5pt}\end{center}

\subsection*{Question 5(c) [7 marks]}\label{q5c}

\textbf{Difference between symmetric and asymmetric encryption
algorithms}

\begin{solutionbox}
Two fundamental approaches to cryptographic key
management with different characteristics.


{\def\LTcaptype{none} % do not increment counter
\vspace{-5pt}
\captionof{table}{Symmetric vs Asymmetric Encryption}
\vspace{-10pt}
\begin{longtable}[]{@{}lll@{}}
\toprule\noalign{}
Feature & Symmetric & Asymmetric \\
\midrule\noalign{}
\endhead
\bottomrule\noalign{}
\endlastfoot
Keys & Single shared key & Key pair (public/private) \\
Speed & Fast & Slower \\
Key Distribution & Difficult & Easier \\
Key Management & Complex for large groups & Simpler \\
Examples & AES, DES & RSA, ECC \\
\end{longtable}
}

\textbf{Symmetric Encryption:}

\begin{itemize}
\tightlist
\item
  \textbf{Single Key}: Same key used for encryption and decryption
\item
  \textbf{Speed}: Fast processing due to simple algorithms
\item
  \textbf{Key Sharing Problem}: Secure key distribution challenge
\item
  \textbf{Session Keys}: Often used for bulk data encryption
\end{itemize}

\textbf{Asymmetric Encryption:}

\begin{itemize}
\tightlist
\item
  \textbf{Key Pair}: Public key for encryption, private key for
  decryption
\item
  \textbf{Digital Signatures}: Private key signs, public key verifies
\item
  \textbf{Key Exchange}: Solves key distribution problem
\item
  \textbf{Computationally Intensive}: Slower than symmetric encryption
\end{itemize}

\textbf{Usage Scenarios:}

\begin{itemize}
\tightlist
\item
  \textbf{Symmetric}: Bulk data encryption, secure communications
\item
  \textbf{Asymmetric}: Key exchange, digital signatures, authentication
\end{itemize}

\textbf{Hybrid Approach:}

\begin{itemize}
\tightlist
\item
  \textbf{Best of Both}: Asymmetric for key exchange, symmetric for data
\item
  \textbf{SSL/TLS}: Uses both types for secure web communications
\end{itemize}

\textbf{Security Considerations:}

\begin{itemize}
\tightlist
\item
  \textbf{Symmetric}: Key compromise affects all communications
\item
  \textbf{Asymmetric}: Private key compromise affects only one party
\end{itemize}

\end{solutionbox}
\begin{mnemonicbox}
``Symmetric Single Key, Asymmetric Key Pair''

\end{mnemonicbox}
\begin{center}\rule{0.5\linewidth}{0.5pt}\end{center}

\subsection*{Question 5(a OR) [3
marks]}\label{question-5a-or-3-marks}

\textbf{Write brief note on Cyber Security.}

\begin{solutionbox}
Cyber security protects digital systems, networks, and
data from digital attacks and unauthorized access.

\textbf{Key Components:}

\begin{itemize}
\tightlist
\item
  \textbf{Network Security}: Protects network infrastructure from
  intrusions
\item
  \textbf{Data Protection}: Safeguards sensitive information from theft
\item
  \textbf{Application Security}: Secures software applications from
  vulnerabilities
\end{itemize}

\textbf{Common Threats:}

\begin{itemize}
\tightlist
\item
  \textbf{Malware}: Viruses, worms, trojans that damage systems
\item
  \textbf{Phishing}: Fraudulent attempts to steal credentials
\end{itemize}

\end{solutionbox}
\begin{mnemonicbox}
``Cyber Security: Protect Digital Assets''

\end{mnemonicbox}
\begin{center}\rule{0.5\linewidth}{0.5pt}\end{center}

\subsection*{Question 5(b OR) [4
marks]}\label{question-5b-or-4-marks}

\textbf{Explain hacking and its precautions.}

\begin{solutionbox}
Hacking involves unauthorized access to computer
systems, often with malicious intent.

\textbf{Types of Hacking:}

\begin{itemize}
\tightlist
\item
  \textbf{White Hat}: Ethical hacking for security testing
\item
  \textbf{Black Hat}: Malicious hacking for illegal purposes
\item
  \textbf{Gray Hat}: Between ethical and malicious hacking
\end{itemize}

\textbf{Common Hacking Methods:}

\begin{itemize}
\tightlist
\item
  \textbf{Password Attacks}: Brute force, dictionary attacks
\item
  \textbf{Social Engineering}: Manipulating people to reveal information
\item
  \textbf{Malware}: Viruses, trojans, ransomware
\item
  \textbf{Network Attacks}: Man-in-the-middle, packet sniffing
\end{itemize}

\textbf{Precautions:}

\begin{itemize}
\tightlist
\item
  \textbf{Strong Passwords}: Complex, unique passwords for all accounts
\item
  \textbf{Regular Updates}: Keep software and systems updated
\item
  \textbf{Firewall}: Use firewall to block unauthorized access
\item
  \textbf{Antivirus}: Install and update antivirus software regularly
\end{itemize}

\end{solutionbox}
\begin{mnemonicbox}
``Hacking Hurts, Precautions Protect''

\end{mnemonicbox}
\begin{center}\rule{0.5\linewidth}{0.5pt}\end{center}

\subsection*{Question 5(c OR) [7
marks]}\label{question-5c-or-7-marks}

\textbf{Briefly describe the Information Technology (Amendment) Act,
2008, and its impact on cyber laws in India.}

\begin{solutionbox}
The IT Amendment Act 2008 significantly strengthened
India's cyber law framework and expanded the scope of cybercrime
legislation.

\textbf{Key Amendments:}

\begin{itemize}
\tightlist
\item
  \textbf{Data Protection}: Enhanced provisions for protecting sensitive
  personal data
\item
  \textbf{Cybercrime Definitions}: Expanded definitions of cybercrime
  including identity theft
\item
  \textbf{Penalties}: Increased penalties for various cyber offenses
\item
  \textbf{Cyber Terrorism}: Introduced provisions to deal with cyber
  terrorism
\end{itemize}

\textbf{Major Provisions:}

\begin{itemize}
\tightlist
\item
  \textbf{Section 43A}: Data protection and compensation for negligence
\item
  \textbf{Section 66A}: Punishment for offensive messages (later struck
  down)
\item
  \textbf{Section 66C}: Identity theft punishment
\item
  \textbf{Section 66D}: Cheating by personation using computer resource
\end{itemize}

\textbf{Impact on Cyber Laws:}

\begin{itemize}
\tightlist
\item
  \textbf{Legal Framework}: Provided comprehensive legal framework for
  cybercrime
\item
  \textbf{Business Compliance}: Mandated data protection measures for
  businesses
\item
  \textbf{Law Enforcement}: Empowered authorities with investigation
  tools
\item
  \textbf{International Cooperation}: Facilitated cooperation in
  cybercrime investigation
\end{itemize}

\textbf{Regulatory Bodies:}

\begin{itemize}
\tightlist
\item
  \textbf{CERT-In}: Computer Emergency Response Team for incident
  response
\item
  \textbf{Cyber Cells}: Specialized police units for cybercrime
  investigation
\item
  \textbf{Adjudicating Officers}: For compensation and penalty
  determination
\end{itemize}

\textbf{Data Protection Requirements:}

\begin{itemize}
\tightlist
\item
  \textbf{Reasonable Security}: Companies must implement reasonable
  security practices
\item
  \textbf{Breach Notification}: Mandatory reporting of data breaches
\item
  \textbf{Compensation}: Victims can claim compensation for data
  breaches
\end{itemize}

\textbf{Challenges and Criticisms:}

\begin{itemize}
\tightlist
\item
  \textbf{Implementation}: Difficulty in implementation across diverse
  digital landscape
\item
  \textbf{Jurisdiction}: Cross-border cybercrime investigation
  challenges
\item
  \textbf{Technology Gap}: Keeping pace with rapidly evolving technology
\end{itemize}

\textbf{Recent Developments:}

\begin{itemize}
\tightlist
\item
  \textbf{Digital India}: Integration with Digital India initiatives
\item
  \textbf{Privacy Laws}: Preparation for comprehensive data protection
  legislation
\item
  \textbf{Emerging Technologies}: Addressing challenges from AI, IoT,
  blockchain
\end{itemize}

\end{solutionbox}
\begin{mnemonicbox}
``IT Act 2008: India's Cyber Law Foundation''

\end{mnemonicbox}

\end{document}
