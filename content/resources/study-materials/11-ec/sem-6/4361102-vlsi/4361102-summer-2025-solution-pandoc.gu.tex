\documentclass[10pt,a4paper]{article}

% content/resources/templates/preamble.tex
\usepackage[margin=0.6in]{geometry}
\author{Milav Dabgar}
\usepackage{amsmath,amssymb,amsthm}
\usepackage{booktabs}
\usepackage{multirow}
\usepackage{xcolor}
\usepackage{tcolorbox}
\tcbuselibrary{breakable,skins}
\usepackage[colorlinks=true,linkcolor=blue]{hyperref}
\usepackage{titlesec}
\usepackage{enumitem}
\usepackage{tikz}
\usepackage{pgfplots}
\usepackage{circuitikz}
\usepackage[version=4]{mhchem}
\usepackage{longtable}
\usepackage{array}
\usepackage{float}
\usepackage{caption}
\usepackage{listings}

\lstset{
  basicstyle=\small\ttfamily,
  breaklines=true,
  breakatwhitespace=false,
  postbreak=\mbox{\textcolor{red}{$\hookrightarrow$}\space},
  float=false,
  numbers=left,
  numberstyle=\tiny\color{gray},
  numbersep=10pt,
  xleftmargin=2em,
  keywordstyle=\color{blue},
  commentstyle=\color{green!60!black},
  stringstyle=\color{purple},
  backgroundcolor=\color{gray!5},
  showstringspaces=false,
  tabsize=2,
  captionpos=b,
  keepspaces=true,
  columns=flexible
}

\pgfplotsset{compat=1.18}
\usetikzlibrary{shapes,arrows,positioning,calc,patterns,decorations.pathmorphing,decorations.markings,arrows.meta}

% Color scheme
\definecolor{headcolor}{RGB}{0,102,204}
\definecolor{keycolor}{RGB}{220,20,60}
\definecolor{solutioncolor}{RGB}{34,139,34}
\definecolor{mnemoniccolor}{RGB}{148,0,211}
\definecolor{codecolor}{RGB}{0,0,100}

% Spacing
\setlength{\parskip}{3pt}
\setlist[itemize]{nosep}
\setlist[enumerate]{nosep}

% Title formatting
\titleformat{\section}{\Large\bfseries\color{headcolor}}{\thesection}{1em}{}
\titleformat{\subsection}{\large\bfseries\color{headcolor}}{\thesubsection}{1em}{}

% Pandoc tightlist compatibility
\providecommand{\tightlist}{%
  \setlength{\itemsep}{0pt}\setlength{\parskip}{0pt}}

% Pandoc longtable compatibility
\newcounter{none}
\def\thenone{}


% content/resources/templates/gujarati-boxes.tex
\usepackage{fontspec}
\usepackage{polyglossia}

% Set Gujarati as main language (document is primarily in Gujarati)
% Note: gloss-gujarati.ldf doesn't exist in polyglossia, but it will use hyphenation patterns
\setdefaultlanguage{gujarati}
\setotherlanguage{english}

% Configure Gujarati font properly
% Use Language=Default to prevent polyglossia from trying to add language-specific features
% that don't exist for Gujarati, which causes "empty feature" warnings
\newfontfamily\gujaratifont[Script=Gujarati,AutoFakeBold=2.5,AutoFakeSlant=0.3]{Noto Sans Gujarati}
\setmainfont[Script=Gujarati,AutoFakeBold=2.5,AutoFakeSlant=0.3]{Noto Sans Gujarati}
% Use Noto Sans Gujarati for monospace to support Gujarati in text
\setmonofont[Scale=0.9]{Noto Sans Gujarati}

% Configure English to use the same font
\newfontfamily\englishfont[Script=Gujarati,AutoFakeBold=2.5,AutoFakeSlant=0.3]{Noto Sans Gujarati}

% Translations for polyglossia
\gappto\captionsgujarati{
  \renewcommand{\tablename}{કોષ્ટક}
  \renewcommand{\figurename}{આકૃતિ}
}

% Helper for TikZ nodes to ensure Gujarati font
\newcommand{\gu}[1]{{\gujaratifont #1}}

% Custom environments
\newtcolorbox{solutionbox}{
    breakable,
    enhanced,
    colback=solutioncolor!5!white,
    colframe=solutioncolor!75!black,
    fonttitle=\bfseries,
    title=જવાબ
}

\newtcolorbox{solutionboxnobreak}{
 colback=solutioncolor!5!white,
 colframe=solutioncolor!75!black,
 fonttitle=\bfseries,
 title=જવાબ
}

\newtcolorbox{keyformula}{
 breakable,
 enhanced,
 colback=keycolor!5!white,
 colframe=keycolor!75!black,
 fonttitle=\bfseries,
 title=રાસાયણિક સમીકરણ/સૂત્ર
}

\newtcolorbox{mnemonicbox}{
 breakable,
 enhanced,
 colback=mnemoniccolor!5!white,
 colframe=mnemoniccolor!75!black,
 fonttitle=\bfseries,
 title=મેમરી ટ્રીક
}


\begin{document}

\begin{center}
{\Huge\bfseries\color{headcolor} Subject Name (Gujarati)}\\[5pt]
{\LARGE 4361102 -- Summer 2025}\\[3pt]
{\large Semester 1 Study Material}\\[3pt]
{\normalsize\textit{Detailed Solutions and Explanations}}
\end{center}

\vspace{10pt}

\subsection*{પ્રશ્ન 1(અ) [3
ગુણ]}\label{uxaaauxab0uxab6uxaa8-1uxa85-3-uxa97uxaa3}

\textbf{સ્કેલિંગનું મહત્વ જણાવો}

\begin{solutionbox}
સ્કેલિંગ semiconductor technology ને આગળ વધારવા અને device
performance સુધારવા માટે અત્યંત મહત્વપૂર્ણ છે.

{\def\LTcaptype{none} % do not increment counter
\begin{longtable}[]{@{}ll@{}}
\toprule\noalign{}
\textbf{સ્કેલિંગ ફાયદા} & \textbf{વર્ણન} \\
\midrule\noalign{}
\endhead
\bottomrule\noalign{}
\endlastfoot
\textbf{Device Size} & ઊંચી density માટે transistor dimensions ઘટાડે છે \\
\textbf{Speed} & ટૂંકી channel length થી ઝડપી switching \\
\textbf{Power} & પ્રતિ operation ઓછો power consumption \\
\textbf{Cost} & વધુ chips per wafer, function દીઠ ઓછો cost \\
\end{longtable}
}

\begin{itemize}
\tightlist
\item
  \textbf{Technology advancement}: Moore's Law ચાલુ રાખવામાં સક્ષમ બનાવે છે
\item
  \textbf{Performance boost}: ઊંચી frequency operation શક્ય બનાવે છે
\item
  \textbf{Market competitiveness}: નાના, ઝડપી, સસ્તા products
\end{itemize}

\textbf{યાદરાખવાની ટિપ:} ``Small Devices Speed Progress Cheaply''

\end{solutionbox}
\begin{center}\rule{0.5\linewidth}{0.5pt}\end{center}

\subsection*{પ્રશ્ન 1(બ) [4
ગુણ]}\label{uxaaauxab0uxab6uxaa8-1uxaac-4-uxa97uxaa3}

\textbf{Planar MOSFET અને FINFET ની તુલના કરો}

\begin{solutionbox}
FinFET technology નાના nodes પર planar MOSFET ની
મર્યાદાઓનો ઉકેલ આપે છે.

{\def\LTcaptype{none} % do not increment counter
\begin{longtable}[]{@{}lll@{}}
\toprule\noalign{}
\textbf{પેરામીટર} & \textbf{Planar MOSFET} & \textbf{FinFET} \\
\midrule\noalign{}
\endhead
\bottomrule\noalign{}
\endlastfoot
\textbf{Structure} & 2D flat channel & 3D fin-shaped channel \\
\textbf{Gate Control} & Single gate & Tri-gate/multi-gate \\
\textbf{Short Channel Effects} & નાના nodes પર ઊંચી & નોંધપાત્ર રીતે ઓછી \\
\textbf{Leakage Current} & ઊંચી subthreshold leakage & ખૂબ ઓછી leakage \\
\end{longtable}
}

\begin{itemize}
\tightlist
\item
  \textbf{Scalability}: FinFET sub-22nm technology nodes શક્ય બનાવે છે
\item
  \textbf{Power efficiency}: FinFET વધુ સારો power-performance ratio આપે છે
\item
  \textbf{Manufacturing}: FinFET વધુ જટિલ fabrication માંગે છે
\end{itemize}

\textbf{યાદરાખવાની ટિપ:} ``Fins Control Current Better Than Flat''

\end{solutionbox}
\begin{center}\rule{0.5\linewidth}{0.5pt}\end{center}

\subsection*{પ્રશ્ન 1(ક) [7
ગુણ]}\label{uxaaauxab0uxab6uxaa8-1uxa95-7-uxa97uxaa3}

\textbf{N channel MOSFET ની VDS-ID અને VGS-ID લાક્ષણિકતાઓ દોરો અને સમજાવો}

\begin{solutionbox}
N-channel MOSFET characteristics અલગ અલગ operating
regions માં device behavior દર્શાવે છે.

\textbf{આકૃતિ:}

\begin{verbatim}
VGS{-ID Characteristics        VDS{-}ID Characteristics}
                                     
    ID ↑                            ID ↑
      |    VGS4                       |     VGS4{VGS3VGS2VGS1}
      |   VGS3 \_\_\_                    |    \_\_\_\_\_\_\_\_\_\_\_\_\_\_\_\_
      |  VGS2 \_\_\_                     |   /
      | VGS1\_\_\_                       |  /  Linear Region
    VT|/\_\_                            | /
      |\_\_\_\_\_\_\_\_\_\_\_\_\_\_\_\_\_\_\_\_\_\_\_\_      |/\_\_\_\_\_\_\_\_\_\_\_\_\_\_\_\_\_\_\_\_\_\_\_\_
         VGS                             VDS
\end{verbatim}

{\def\LTcaptype{none} % do not increment counter
\begin{longtable}[]{@{}lll@{}}
\toprule\noalign{}
\textbf{પ્રદેશ} & \textbf{સ્થિતિ} & \textbf{કરંટ સમીકરણ} \\
\midrule\noalign{}
\endhead
\bottomrule\noalign{}
\endlastfoot
\textbf{Cutoff} & VGS \textless{} VT & ID = 0 \\
\textbf{Linear} & VDS \textless{} (VGS-VT) & ID ∝ VDS \\
\textbf{Saturation} & VDS \geq (VGS-VT) & ID ∝ (VGS-VT)^{2} \\
\end{longtable}
}

\begin{itemize}
\tightlist
\item
  \textbf{થ્રેશોલ્ડ વોલ્ટેજ (VT)}: conduction માટે લઘુત્તમ VGS
\item
  \textbf{Transconductance}: saturation માં VGS-ID curve નો slope
\item
  \textbf{આઉટપુટ પ્રતિકાર}: saturation region માં વિપરીત slope
\end{itemize}

\textbf{યાદરાખવાની ટિપ:} ``Threshold Gates Linear Saturation''

\end{solutionbox}
\begin{center}\rule{0.5\linewidth}{0.5pt}\end{center}

\subsection*{પ્રશ્ન 1(ક OR) [7
ગુણ]}\label{uxaaauxab0uxab6uxaa8-1uxa95-or-7-uxa97uxaa3}

\textbf{એક્સટર્નલ બાયસ હેઠળ MOS પર ઉદ્ભવતી અલગ અલગ અસર સમજાવો}

\begin{solutionbox}
External bias અલગ અલગ charge distributions બનાવે છે જે MOS
capacitor behavior ને અસર કરે છે.

\textbf{આકૃતિ:}

\begin{center}
\textbf{Mermaid Diagram (Code)}
\begin{verbatim}
{Shaded}
{Highlighting}[]
graph TD
    A[MOS Under Bias] {-{-}{} B[Accumulation VG {} 0]}
    A {-{-}{} C[Depletion 0 {} VG {} VT]}
    A {-{-}{} D[Inversion VG {} VT]}
    B {-{-}{} E[સપાટી પર holes એકઠા થાય છે]}
    C {-{-}{} F[સપાટી carriers થી ખાલી]}
    D {-{-}{} G[Electron inversion layer બને છે]}
{Highlighting}
{Shaded}
\end{verbatim}
\end{center}

{\def\LTcaptype{none} % do not increment counter
\begin{longtable}[]{@{}lll@{}}
\toprule\noalign{}
\textbf{બાયસ સ્થિતિ} & \textbf{સપાટીની સ્થિતિ} & \textbf{કેપેસિટન્સ} \\
\midrule\noalign{}
\endhead
\bottomrule\noalign{}
\endlastfoot
\textbf{Accumulation} & સપાટી પર majority carriers & ઊંચી (Cox) \\
\textbf{Depletion} & કોઈ mobile carriers નથી & મધ્યમ \\
\textbf{Inversion} & Minority carriers channel બનાવે છે & ઊંચી (Cox) \\
\end{longtable}
}

\begin{itemize}
\tightlist
\item
  \textbf{ફ્લેટ બેન્ડ વોલ્ટેજ}: કોઈ charge separation અસ્તિત્વમાં નથી
\item
  \textbf{એનર્જી બેન્ડ બેન્ડિંગ}: carrier distribution નક્કી કરે છે
\item
  \textbf{સપાટીનો વિભવ}: inversion layer formation નિયંત્રિત કરે છે
\end{itemize}

\textbf{યાદરાખવાની ટિપ:} ``Accumulate, Deplete, then Invert''

\end{solutionbox}
\begin{center}\rule{0.5\linewidth}{0.5pt}\end{center}

\subsection*{પ્રશ્ન 2(અ) [3
ગુણ]}\label{uxaaauxab0uxab6uxaa8-2uxa85-3-uxa97uxaa3}

\textbf{આદર્શ ઇન્વર્ટરની વોલ્ટેજ ટ્રાન્સફર લાક્ષણિકતા દોરો}

\begin{solutionbox}
આદર્શ ઇન્વર્ટર infinite gain સાથે logic levels વચ્ચે તીક્ષ્ણ
પરિવર્તન આપે છે.

\textbf{આકૃતિ:}

\begin{verbatim}
    VOUT ↑
        |
    VOH |    |
        |    |
        |    |
        |    |\_\_\_\_\_\_\_\_\_\_\_\_\_\_\_
        |                    |
        |                    | VOL
        |\_\_\_\_\_\_\_\_\_\_\_\_\_\_\_\_\_\_\_\_|\_\_\_\_\_\_\_
           VIL   VIH        VIN
\end{verbatim}

\begin{itemize}
\tightlist
\item
  \textbf{તીક્ષ્ણ પરિવર્તન}: switching point પર infinite slope
\item
  \textbf{નોઈઝ માર્જિન}: NMH = VOH - VIH, NML = VIL - VOL
\item
  \textbf{સંપૂર્ણ લોજિક લેવલ}: VOH = VDD, VOL = 0V
\end{itemize}

\textbf{યાદરાખવાની ટિપ:} ``Sharp Switch, Perfect Levels''

\end{solutionbox}
\begin{center}\rule{0.5\linewidth}{0.5pt}\end{center}

\subsection*{પ્રશ્ન 2(બ) [4
ગુણ]}\label{uxaaauxab0uxab6uxaa8-2uxaac-4-uxa97uxaa3}

\textbf{નોઈઝ ઇમ્યુનિટી અને નોઈઝ માર્જિન સમજાવો}

\begin{solutionbox}
નોઈઝ ઇમ્યુનિટી circuit ની અનચાહેલા સિગ્નલ વેરિએશન ને નકારવાની
ક્ષમતા માપે છે.

{\def\LTcaptype{none} % do not increment counter
\begin{longtable}[]{@{}lll@{}}
\toprule\noalign{}
\textbf{પેરામીટર} & \textbf{વ્યાખ્યા} & \textbf{ફોર્મ્યુલા} \\
\midrule\noalign{}
\endhead
\bottomrule\noalign{}
\endlastfoot
\textbf{NMH} & હાઈ-લેવલ નોઈઝ માર્જિન & VOH - VIH \\
\textbf{NML} & લો-લેવલ નોઈઝ માર્જિન & VIL - VOL \\
\textbf{નોઈઝ ઇમ્યુનિટી} & નોઈઝ નકારવાની ક્ષમતા & Min(NMH, NML) \\
\end{longtable}
}

\begin{itemize}
\tightlist
\item
  \textbf{લોજિક થ્રેશોલ્ડ લેવલ}: VIH (input high), VIL (input low)
\item
  \textbf{આઉટપુટ લેવલ}: VOH (output high), VOL (output low)
\item
  \textbf{વધુ સારી ઇમ્યુનિટી}: મોટા નોઈઝ માર્જિન વધુ સારી સુરક્ષા આપે છે
\item
  \textbf{ડિઝાઇન લક્ષ્ય}: મજબૂત operation માટે નોઈઝ માર્જિન વધારવા
\end{itemize}

\textbf{યાદરાખવાની ટિપ:} ``Margins Protect Against Noise''

\end{solutionbox}
\begin{center}\rule{0.5\linewidth}{0.5pt}\end{center}

\subsection*{પ્રશ્ન 2(ક) [7
ગુણ]}\label{uxaaauxab0uxab6uxaa8-2uxa95-7-uxa97uxaa3}

\textbf{Saturated અને linear depletion load nMOS ઇન્વર્ટર સાથે ઇન્વર્ટર સર્કિટનું
વર્ણન કરો}

\begin{solutionbox}
Depletion load nMOS ઇન્વર્ટર active load resistor તરીકે
depletion transistor વાપરે છે.

\textbf{આકૃતિ:}

\begin{verbatim}
       VDD
        |
        |
    |{-{-}{-}+{-}{-}{-}| MD (Depletion Load)}
    |       | VT { 0}
    |       |
    +{-{-}{-}{-}{-}{-}{-}+{-}{-}{-}{-}{-}{-} VOUT}
    |
    |
|{-{-}{-}+{-}{-}{-}| MN (Driver)}
|       | VT { 0}
|       |
|\_\_\_\_\_\_\_|
   VIN    GND
\end{verbatim}

{\def\LTcaptype{none} % do not increment counter
\begin{longtable}[]{@{}lll@{}}
\toprule\noalign{}
\textbf{લોડ પ્રકાર} & \textbf{ગેટ કનેક્શન} & \textbf{ઓપરેશન} \\
\midrule\noalign{}
\endhead
\bottomrule\noalign{}
\endlastfoot
\textbf{Saturated Load} & VG = VD & હંમેશા saturation માં \\
\textbf{Linear Load} & VG = VDD & Linear region માં કામ કરી શકે છે \\
\end{longtable}
}

\begin{itemize}
\tightlist
\item
  \textbf{ડિપ્લીશન ડિવાઇસ}: VGS = 0 સાથે વહન કરે છે, current source તરીકે કામ
  કરે છે
\item
  \textbf{લોડ લાઇન વિશ્લેષણ}: operating point intersection નક્કી કરે છે
\item
  \textbf{પાવર કન્ઝ્યુમ્પશન}: હંમેશા વહન કરે છે, ઊંચો static power
\item
  \textbf{સ્વિચિંગ સ્પીડ}: pull-up કરતાં pull-down ઝડપી
\end{itemize}

\textbf{યાદરાખવાની ટિપ:} ``Depletion Loads Drive Outputs''

\end{solutionbox}
\begin{center}\rule{0.5\linewidth}{0.5pt}\end{center}

\subsection*{પ્રશ્ન 2(અ OR) [3
ગુણ]}\label{uxaaauxab0uxab6uxaa8-2uxa85-or-3-uxa97uxaa3}

\textbf{એન્હાન્સમેન્ટ લોડ ઇન્વર્ટર દોરો અને સમજાવો}

\begin{solutionbox}
Enhancement load ઇન્વર્ટર ખાસ biasing સાથે enhancement
MOSFET ને load તરીકે વાપરે છે.

\textbf{આકૃતિ:}

\begin{verbatim}
       VDD
        |
        |
    |{-{-}{-}+{-}{-}{-}| ME (Enhancement Load)}
    |   |   | VT { 0}
    |   +{-{-}{-}+}
    +{-{-}{-}{-}{-}{-}{-}+{-}{-}{-}{-}{-}{-} VOUT}
    |
    |
|{-{-}{-}+{-}{-}{-}| MN (Driver)}
|       |
|       |
|\_\_\_\_\_\_\_|
   VIN    GND
\end{verbatim}

\begin{itemize}
\tightlist
\item
  \textbf{બૂટસ્ટ્રેપ કનેક્શન}: લોડ માટે gate ને drain સાથે જોડાયેલ
\item
  \textbf{મર્યાદિત આઉટપુટ હાઈ}: VOUT(max) = VDD - VT
\item
  \textbf{થ્રેશોલ્ડ નુકસાન}: Enhancement load વોલ્ટેજ ડ્રોપ કરાવે છે
\end{itemize}

\textbf{યાદરાખવાની ટિપ:} ``Enhancement Loses Threshold''

\end{solutionbox}
\begin{center}\rule{0.5\linewidth}{0.5pt}\end{center}

\subsection*{પ્રશ્ન 2(બ OR) [4
ગુણ]}\label{uxaaauxab0uxab6uxaa8-2uxaac-or-4-uxa97uxaa3}

\textbf{CMOS ઇન્વર્ટરના ફાયદા જણાવો}

\begin{solutionbox}
CMOS technology NMOS ઇન્વર્ટર કરતાં શ્રેષ્ઠ performance આપે છે.

{\def\LTcaptype{none} % do not increment counter
\begin{longtable}[]{@{}ll@{}}
\toprule\noalign{}
\textbf{ફાયદો} & \textbf{લાભ} \\
\midrule\noalign{}
\endhead
\bottomrule\noalign{}
\endlastfoot
\textbf{શૂન્ય સ્ટેટિક પાવર} & steady state માં કોઈ current path નથી \\
\textbf{રેલ-ટુ-રેલ આઉટપુટ} & સંપૂર્ણ VDD અને 0V આઉટપુટ લેવલ \\
\textbf{ઊંચી નોઈઝ ઇમ્યુનિટી} & મોટા નોઈઝ માર્જિન \\
\textbf{સમપ્રમાણ સ્વિચિંગ} & બરાબર rise અને fall times \\
\end{longtable}
}

\begin{itemize}
\tightlist
\item
  \textbf{પાવર એફિશિયન્સી}: માત્ર switching દરમિયાન dynamic power
\item
  \textbf{સ્કેલેબિલિટી}: બધા technology nodes પર સારી રીતે કામ કરે છે
\item
  \textbf{ફેન-આઉટ ક્ષમતા}: અનેક inputs ડ્રાઇવ કરી શકે છે
\item
  \textbf{તાપમાન સ્થિરતા}: performance તાપમાન પર ઓછી સંવેદનશીલ
\end{itemize}

\textbf{યાદરાખવાની ટિપ:} ``CMOS Saves Power Perfectly''

\end{solutionbox}
\begin{center}\rule{0.5\linewidth}{0.5pt}\end{center}

\subsection*{પ્રશ્ન 2(ક OR) [7
ગુણ]}\label{uxaaauxab0uxab6uxaa8-2uxa95-or-7-uxa97uxaa3}

\textbf{CMOS ઇન્વર્ટરના ઓપરેટિંગ મોડ પ્રદેશ દોરો અને સમજાવો}

\begin{solutionbox}
CMOS ઇન્વર્ટર operation input voltage ના આધારે પાંચ અલગ અલગ
regions સમાવે છે.

\textbf{આકૃતિ:}

\begin{center}
\textbf{Mermaid Diagram (Code)}
\begin{verbatim}
{Shaded}
{Highlighting}[]
graph TD
    A[CMOS Inverter Regions] {-{-}{} B[Region 1: PMOS ON, NMOS OFF]}
    A {-{-}{} C[Region 2: બંને saturation માં]}
    A {-{-}{} D[Region 3: Switching point]}
    A {-{-}{} E[Region 4: બંને saturation માં]}
    A {-{-}{} F[Region 5: PMOS OFF, NMOS ON]}
{Highlighting}
{Shaded}
\end{verbatim}
\end{center}

{\def\LTcaptype{none} % do not increment counter
\begin{longtable}[]{@{}llll@{}}
\toprule\noalign{}
\textbf{પ્રદેશ} & \textbf{NMOS સ્થિતિ} & \textbf{PMOS સ્થિતિ} &
\textbf{આઉટપુટ} \\
\midrule\noalign{}
\endhead
\bottomrule\noalign{}
\endlastfoot
\textbf{1} & OFF & Linear & VOH \approx VDD \\
\textbf{2} & Saturation & Saturation & Transition \\
\textbf{3} & Saturation & Saturation & VDD/2 \\
\textbf{4} & Saturation & Saturation & Transition \\
\textbf{5} & Linear & OFF & VOL \approx 0V \\
\end{longtable}
}

\begin{itemize}
\tightlist
\item
  \textbf{સ્વિચિંગ થ્રેશોલ્ડ}: VTC region 3 પર VDD/2 ને પાર કરે છે
\item
  \textbf{કરંટ ફ્લો}: માત્ર transition regions 2,3,4 દરમિયાન
\item
  \textbf{નોઈઝ માર્જિન}: Regions 1 અને 5 ઇમ્યુનિટી આપે છે
\item
  \textbf{ગેઇન}: Region 3 માં મહત્તમ (switching point)
\end{itemize}

\textbf{યાદરાખવાની ટિપ:} ``Five Regions Control CMOS Switching''

\end{solutionbox}
\begin{center}\rule{0.5\linewidth}{0.5pt}\end{center}

\subsection*{પ્રશ્ન 3(અ) [3
ગુણ]}\label{uxaaauxab0uxab6uxaa8-3uxa85-3-uxa97uxaa3}

\textbf{CMOS વાપરીને બે ઇનપુટ NOR ગેટ દોરો}

\begin{solutionbox}
CMOS NOR ગેટ complementary networks વાપરીને De Morgan's law
અમલમાં મૂકે છે.

\textbf{આકૃતિ:}

\begin{verbatim}
       VDD
        |
    |{-{-}{-}+{-}{-}{-}| MP1}
    |A  |   |
    +{-{-}{-}+{-}{-}{-}+}
    |       |
    |   |{-{-}{-}+{-}{-}{-}| MP2}
    |   |B  |   |
    +{-{-}{-}+{-}{-}{-}+{-}{-}{-}+{-}{-}{-}{-}{-}{-} Y = (A+B)}
    |           |
|{-{-}{-}+{-}{-}{-}|   |{-}{-}{-}+{-}{-}{-}| MN2}
|A      |   |B      |
|       |   |       |
|\_\_\_\_\_\_\_|   |\_\_\_\_\_\_\_|
            |
           GND
\end{verbatim}

\begin{itemize}
\tightlist
\item
  \textbf{પુલ-અપ નેટવર્ક}: સીરીઝ PMOS transistors (હાઈ આઉટપુટ માટે A અને B બંને
  લો)
\item
  \textbf{પુલ-ડાઉન નેટવર્ક}: પેરેલલ NMOS transistors (લો આઉટપુટ માટે A અથવા B
  હાઈ)
\item
  \textbf{લોજિક ફંક્શન}: Y = (A+B)' = A' · B'
\end{itemize}

\textbf{યાદરાખવાની ટિપ:} ``Series PMOS, Parallel NMOS''

\end{solutionbox}
\begin{center}\rule{0.5\linewidth}{0.5pt}\end{center}

\subsection*{પ્રશ્ન 3(બ) [4
ગુણ]}\label{uxaaauxab0uxab6uxaa8-3uxaac-4-uxa97uxaa3}

\textbf{CMOS વાપરીને બૂલિયન ફંક્શન Z= [(A+B)C+DE]' અમલ કરો}

\begin{solutionbox}
જટિલ CMOS લોજિક કાર્યક્ષમ અમલીકરણ માટે AOI (AND-OR-Invert)
સ્ટ્રક્ચર વાપરે છે.

\textbf{આકૃતિ:}

\begin{verbatim}
                    VDD
                     |
         +{-{-}{-}{-}{-}{-}{-}{-}{-}{-}{-}+{-}{-}{-}{-}{-}{-}{-}{-}{-}{-}{-}+}
         |                       |
     |{-{-}{-}+{-}{-}{-}|               |{-}{-}{-}+{-}{-}{-}|}
     |A      |               |D      |
     |\_\_\_\_\_\_\_|               |\_\_\_\_\_\_\_|
         |                       |
     |{-{-}{-}+{-}{-}{-}|               |{-}{-}{-}+{-}{-}{-}|}
     |B      |               |E      |
     |\_\_\_\_\_\_\_|               |\_\_\_\_\_\_\_|
         |                       |
     |{-{-}{-}+{-}{-}{-}|               }
     |C      |               
     |\_\_\_\_\_\_\_|               
         |                       |
         +{-{-}{-}{-}{-}{-}{-}{-}{-}{-}{-}+{-}{-}{-}{-}{-}{-}{-}{-}{-}{-}{-}+{-}{-}{-}{-}{-}{-} Z}
                     |
         +{-{-}{-}{-}{-}{-}{-}{-}{-}{-}{-}+{-}{-}{-}{-}{-}{-}{-}{-}{-}{-}{-}+}
         |           |           |
     |{-{-}{-}+{-}{-}{-}|   |{-}{-}{-}+{-}{-}{-}|   |{-}{-}{-}+{-}{-}{-}|}
     |A      |   |B      |   |C      |
     |       |   |       |   |       |
     |\_\_\_\_\_\_\_|   |\_\_\_\_\_\_\_|   |\_\_\_\_\_\_\_|
                     |           |
                 |{-{-}{-}+{-}{-}{-}|   |{-}{-}{-}+{-}{-}{-}|}
                 |D      |   |E      |
                 |       |   |       |
                 |\_\_\_\_\_\_\_|   |\_\_\_\_\_\_\_|
                             |
                            GND
\end{verbatim}

\begin{itemize}
\tightlist
\item
  \textbf{AOI સ્ટ્રક્ચર}: કાર્યક્ષમ single-stage અમલીકરણ
\item
  \textbf{ડ્યુઅલ નેટવર્ક}: કોમ્પ્લિમેન્ટરી pull-up અને pull-down
\item
  \textbf{લોજિક ઓપ્ટિમાઇઝેશન}: અલગ ગેટ કરતાં ઓછા transistors
\end{itemize}

\textbf{યાદરાખવાની ટિપ:} ``AOI Inverts Complex Logic Efficiently''

\end{solutionbox}
\begin{center}\rule{0.5\linewidth}{0.5pt}\end{center}

\subsection*{પ્રશ્ન 3(ક) [7
ગુણ]}\label{uxaaauxab0uxab6uxaa8-3uxa95-7-uxa97uxaa3}

\textbf{પેરાસિટિક ડિવાઇસ કેપેસિટન્સ સાથે CMOS NAND2 ગેટ દોરો અને સમજાવો}

\begin{solutionbox}
CMOS ગેટમાં પેરાસિટિક કેપેસિટન્સ switching speed અને power
consumption ને અસર કરે છે.

\textbf{આકૃતિ:}

\begin{verbatim}
       VDD
        |
    |{-{-}{-}+{-}{-}{-}| MP1  Cgd1}
    |A  |   |      
    +{-{-}{-}+{-}{-}{-}+{-}{-}{-}{-}{-}{-} Y = (AB)}
    |       |       |
    |   |{-{-}{-}+{-}{-}{-}| MP2  Cgd2}
    |   |B  |   |      |
    +{-{-}{-}+{-}{-}{-}+{-}{-}{-}+{-}{-}{-}{-}{-}{-}+}
    |           |      |
|{-{-}{-}+{-}{-}{-}|   |{-}{-}{-}+{-}{-}{-}|  | Cload}
|A      |   |B      |  |
|       |   |       |  |
|\_\_\_\_\_\_\_|   |\_\_\_\_\_\_\_|  |
    |           |      |
   Cgs1        Cgs2    |
    |           |      |
   GND         GND    GND

પેરાસિટિક કેપેસિટન્સ:
Cgs {- Gate to Source}
Cgd {- Gate to Drain  }
Cdb {- Drain to Bulk}
Csb {- Source to Bulk}
\end{verbatim}

{\def\LTcaptype{none} % do not increment counter
\begin{longtable}[]{@{}lll@{}}
\toprule\noalign{}
\textbf{કેપેસિટન્સ} & \textbf{સ્થાન} & \textbf{અસર} \\
\midrule\noalign{}
\endhead
\bottomrule\noalign{}
\endlastfoot
\textbf{Cgs} & Gate-Source & Input capacitance \\
\textbf{Cgd} & Gate-Drain & Miller effect \\
\textbf{Cdb} & Drain-Bulk & Output loading \\
\textbf{Csb} & Source-Bulk & Source loading \\
\end{longtable}
}

\begin{itemize}
\tightlist
\item
  \textbf{સ્વિચિંગ વિલંબ}: પેરાસિટિક કેપેસિટન્સ transitions ધીમા કરે છે
\item
  \textbf{પાવર કન્ઝ્યુમ્પશન}: પેરાસિટિક caps ચાર્જ/ડિસ્ચાર્જ કરવા
\item
  \textbf{મિલર ઇફેક્ટ}: Cgd feedback બનાવે છે, switching ધીમું કરે છે
\item
  \textbf{લેઆઉટ ઓપ્ટિમાઇઝેશન}: પેરાસિટિક કેપેસિટન્સ ઓછા કરવા
\end{itemize}

\textbf{યાદરાખવાની ટિપ:} ``Parasitics Slow Gates Down''

\end{solutionbox}
\begin{center}\rule{0.5\linewidth}{0.5pt}\end{center}

\subsection*{પ્રશ્ન 3(અ OR) [3
ગુણ]}\label{uxaaauxab0uxab6uxaa8-3uxa85-or-3-uxa97uxaa3}

\textbf{CMOS વાપરીને NOR આધારિત Clocked SR latch દોરો અને સમજાવો}

\begin{solutionbox}
Clocked SR latch synchronous operation માટે clock enable
સાથે NOR gates વાપરે છે.

\textbf{આકૃતિ:}

\begin{center}
\textbf{Mermaid Diagram (Code)}
\begin{verbatim}
{Shaded}
{Highlighting}[]
graph LR
    S {-{-}{} A[NOR1]}
    CLK {-{-}{} A}
    A {-{-}{} Q}
    Q {-{-}{} B[NOR2]}
    R {-{-}{} C[NOR3]}
    CLK {-{-}{} C}
    C {-{-}{} D[NOR4]}
    D {-{-}{} QB[Q{}]}
    QB {-{-}{} B}
    B {-{-}{} Q}
{Highlighting}
{Shaded}
\end{verbatim}
\end{center}

\begin{itemize}
\tightlist
\item
  \textbf{ક્લોક કંટ્રોલ}: S અને R માત્ર CLK = 1 હોય ત્યારે જ અસરકારક
\item
  \textbf{ટ્રાન્સપેરન્ટ મોડ}: clock સક્રિય હોય ત્યારે આઉટપુટ input ને અનુસરે છે
\item
  \textbf{હોલ્ડ મોડ}: clock નિષ્ક્રિય હોય ત્યારે આઉટપુટ સ્થિતિ જાળવે છે
\item
  \textbf{મૂળભૂત બિલ્ડિંગ બ્લોક}: flip-flops માટે પાયો
\end{itemize}

\textbf{યાદરાખવાની ટિપ:} ``Clock Controls Transparent Latching''

\end{solutionbox}
\begin{center}\rule{0.5\linewidth}{0.5pt}\end{center}

\subsection*{પ્રશ્ન 3(બ OR) [4
ગુણ]}\label{uxaaauxab0uxab6uxaa8-3uxaac-or-4-uxa97uxaa3}

\textbf{CMOS વાપરીને બૂલિયન ફંક્શન Z=[AB+C(D+E)]' અમલ કરો}

\begin{solutionbox}
આ ફંક્શન AOI લોજિક સ્ટ્રક્ચર વાપરીને inverted sum-of-products
અમલમાં મૂકે છે.

\textbf{લોજિક વિશ્લેષણ:}

\begin{itemize}
\tightlist
\item
  મૂળ: Z = [AB + C(D+E)]'
\item
  વિસ્તૃત: Z = [AB + CD + CE]'
\item
  અમલીકરણ: ત્રણ AND terms NOR ને આપવામાં આવે છે
\end{itemize}

{\def\LTcaptype{none} % do not increment counter
\begin{longtable}[]{@{}lll@{}}
\toprule\noalign{}
\textbf{ટર્મ} & \textbf{ઇનપુટ્સ} & \textbf{ફંક્શન} \\
\midrule\noalign{}
\endhead
\bottomrule\noalign{}
\endlastfoot
\textbf{ટર્મ 1} & A, B & AB \\
\textbf{ટર્મ 2} & C, D & CD \\
\textbf{ટર્મ 3} & C, E & CE \\
\textbf{આઉટપુટ} & બધા terms & (AB + CD + CE)' \\
\end{longtable}
}

\begin{itemize}
\tightlist
\item
  \textbf{AOI અમલીકરણ}: સિંગલ સ્ટેજ, કાર્યક્ષમ ડિઝાઇન
\item
  \textbf{ટ્રાન્ઝિસ્ટર કાઉન્ટ}: અલગ ગેટ અમલીકરણ કરતાં ઓછા
\item
  \textbf{પરફોર્મન્સ}: ઝડપી switching, ઓછો power
\end{itemize}

\textbf{યાદરાખવાની ટિપ:} ``Three AND Terms Feed One NOR''

\end{solutionbox}
\begin{center}\rule{0.5\linewidth}{0.5pt}\end{center}

\subsection*{પ્રશ્ન 3(ક OR) [7
ગુણ]}\label{uxaaauxab0uxab6uxaa8-3uxa95-or-7-uxa97uxaa3}

\textbf{ઉદાહરણ સાથે AOI અને OAI લોજિકને અલગ પાડો}

\begin{solutionbox}
AOI અને OAI કાર્યક્ષમ CMOS અમલીકરણ માટે પૂરક લોજિક પરિવારો છે.

{\def\LTcaptype{none} % do not increment counter
\begin{longtable}[]{@{}lll@{}}
\toprule\noalign{}
\textbf{પેરામીટર} & \textbf{AOI (AND-OR-Invert)} & \textbf{OAI
(OR-AND-Invert)} \\
\midrule\noalign{}
\endhead
\bottomrule\noalign{}
\endlastfoot
\textbf{સ્ટ્રક્ચર} & AND gates \rightarrow OR \rightarrow Invert & OR gates \rightarrow AND \rightarrow Invert \\
\textbf{ફંક્શન} & (AB + CD + \ldots)' & ((A+B)(C+D)\ldots)' \\
\textbf{PMOS નેટવર્ક} & Series-parallel & Parallel-series \\
\textbf{NMOS નેટવર્ક} & Parallel-series & Series-parallel \\
\end{longtable}
}

\textbf{AOI ઉદાહરણ: Y = (AB + CD)'}

\begin{verbatim}
PMOS: Series A{-B parallel with Series C{-}D}
NMOS: Parallel A,B series with Parallel C,D
\end{verbatim}

\textbf{OAI ઉદાહરણ: Y = ((A+B)(C+D))'}

\begin{verbatim}
PMOS: Parallel A,B series with Parallel C,D  
NMOS: Series A{-B parallel with Series C{-}D}
\end{verbatim}

\begin{itemize}
\tightlist
\item
  \textbf{ડિઝાઇન પસંદગી}: બૂલિયન ફંક્શન ફોર્મ આધારે પસંદ કરો
\item
  \textbf{ઓપ્ટિમાઇઝેશન}: ટ્રાન્ઝિસ્ટર કાઉન્ટ અને વિલંબ ઓછો કરે છે
\item
  \textbf{દ્વૈતતા}: AOI અને OAI De Morgan duals છે
\end{itemize}

\textbf{યાદરાખવાની ટિપ:} ``AOI ANDs then ORs, OAI ORs then ANDs''

\end{solutionbox}
\begin{center}\rule{0.5\linewidth}{0.5pt}\end{center}

\subsection*{પ્રશ્ન 4(અ) [3
ગુણ]}\label{uxaaauxab0uxab6uxaa8-4uxa85-3-uxa97uxaa3}

\textbf{વ્યાખ્યા આપો: 1) Regularity 2) Modularity 3) Locality}

\begin{solutionbox}
VLSI જટિલતાને સંચાલિત કરવા અને સફળ અમલીકરણ સુનિશ્ચિત કરવા માટે
ડિઝાઇન હાયરાર્કી સિદ્ધાંતો આવશ્યક છે.

{\def\LTcaptype{none} % do not increment counter
\begin{longtable}[]{@{}lll@{}}
\toprule\noalign{}
\textbf{સિદ્ધાંત} & \textbf{વ્યાખ્યા} & \textbf{ફાયદો} \\
\midrule\noalign{}
\endhead
\bottomrule\noalign{}
\endlastfoot
\textbf{Regularity} & સમાન સ્ટ્રક્ચર્સનો વારંવાર ઉપયોગ & સરળ લેઆઉટ, ટેસ્ટિંગ \\
\textbf{Modularity} & ડિઝાઇનને નાના બ્લોક્સમાં વહેંચવું & સ્વતંત્ર ડિઝાઇન,
પુનઃઉપયોગ \\
\textbf{Locality} & મોટે ભાગે સ્થાનિક interconnections & ઓછી routing
જટિલતા \\
\end{longtable}
}

\begin{itemize}
\tightlist
\item
  \textbf{ડિઝાઇન કાર્યક્ષમતા}: સિદ્ધાંતો ડિઝાઇન સમય અને પ્રયાસ ઘટાડે છે
\item
  \textbf{વેરિફિકેશન}: મોડ્યુલર અભિગમ ટેસ્ટિંગ સરળ બનાવે છે
\item
  \textbf{સ્કેલેબિલિટી}: મોટા, વધુ જટિલ ડિઝાઇન્સ શક્ય બનાવે છે
\end{itemize}

\textbf{યાદરાખવાની ટિપ:} ``Regular Modules Stay Local''

\end{solutionbox}
\begin{center}\rule{0.5\linewidth}{0.5pt}\end{center}

\subsection*{પ્રશ્ન 4(બ) [4
ગુણ]}\label{uxaaauxab0uxab6uxaa8-4uxaac-4-uxa97uxaa3}

\textbf{CMOS ઇન્વર્ટર વાપરીને SR latch (NAND ગેટ) અમલ કરો}

\begin{solutionbox}
NAND ગેટ વાપરીને SR latch active-low inputs સાથે set-reset
functionality આપે છે.

\textbf{આકૃતિ:}

\begin{center}
\textbf{Mermaid Diagram (Code)}
\begin{verbatim}
{Shaded}
{Highlighting}[]
graph LR
    S{ {-}{-}{} A[NAND1]}
    A {-{-}{} Q}
    Q {-{-}{} B[NAND2]}
    R{ {-}{-}{} B}
    B {-{-}{} QB[Q{}]}
    QB {-{-}{} A}
{Highlighting}
{Shaded}
\end{verbatim}
\end{center}

\textbf{સત્ય કોષ્ટક:}

{\def\LTcaptype{none} % do not increment counter
\begin{longtable}[]{@{}lllll@{}}
\toprule\noalign{}
\textbf{S'} & \textbf{R'} & \textbf{Q} & \textbf{Q'} & \textbf{સ્થિતિ} \\
\midrule\noalign{}
\endhead
\bottomrule\noalign{}
\endlastfoot
0 & 1 & 1 & 0 & Set \\
1 & 0 & 0 & 1 & Reset \\
1 & 1 & Q & Q' & Hold \\
0 & 0 & 1 & 1 & અમાન્ય \\
\end{longtable}
}

\begin{itemize}
\tightlist
\item
  \textbf{ક્રોસ-કપ્લ્ડ સ્ટ્રક્ચર}: મેમરી ફંક્શન આપે છે
\item
  \textbf{એક્ટિવ-લો ઇનપુટ્સ}: S' = 0 સેટ કરે છે, R' = 0 રીસેટ કરે છે
\item
  \textbf{પ્રતિબંધિત સ્થિતિ}: બંને ઇનપુટ્સ એકસાથે લો
\end{itemize}

\textbf{યાદરાખવાની ટિપ:} ``Cross-Coupled NANDS Remember State''

\end{solutionbox}
\begin{center}\rule{0.5\linewidth}{0.5pt}\end{center}

\subsection*{પ્રશ્ન 4(ક) [7
ગુણ]}\label{uxaaauxab0uxab6uxaa8-4uxa95-7-uxa97uxaa3}

\textbf{VLSI ડિઝાઇન ફ્લો સમજાવો}

\begin{solutionbox}
VLSI ડિઝાઇન ફ્લો specification થી fabrication સુધીના
વ્યવસ્થિત પગલાંઓ અનુસરે છે.

\begin{center}
\textbf{Mermaid Diagram (Code)}
\begin{verbatim}
{Shaded}
{Highlighting}[]
graph LR
    A[સિસ્ટમ સ્પેસિફિકેશન] {-{-}{} B[આર્કિટેક્ચરલ ડિઝાઇન]}
    B {-{-}{} C[ફંક્શનલ ડિઝાઇન]}
    C {-{-}{} D[લોજિક ડિઝાઇન]}
    D {-{-}{} E[સર્કિટ ડિઝાઇન]}
    E {-{-}{} F[ફિઝિકલ ડિઝાઇન]}
    F {-{-}{} G[ફેબ્રિકેશન]}
    G {-{-}{} H[ટેસ્ટિંગ અને પેકેજિંગ]}
{Highlighting}
{Shaded}
\end{verbatim}
\end{center}

{\def\LTcaptype{none} % do not increment counter
\begin{longtable}[]{@{}lll@{}}
\toprule\noalign{}
\textbf{લેવલ} & \textbf{પ્રવૃત્તિઓ} & \textbf{આઉટપુટ} \\
\midrule\noalign{}
\endhead
\bottomrule\noalign{}
\endlastfoot
\textbf{સિસ્ટમ} & આવશ્યકતા વિશ્લેષણ & સ્પેસિફિકેશન્સ \\
\textbf{આર્કિટેક્ચર} & બ્લોક-લેવલ ડિઝાઇન & સિસ્ટમ આર્કિટેક્ચર \\
\textbf{લોજિક} & બૂલિયન ઓપ્ટિમાઇઝેશન & ગેટ નેટલિસ્ટ \\
\textbf{સર્કિટ} & ટ્રાન્ઝિસ્ટર સાઇઝિંગ & સર્કિટ નેટલિસ્ટ \\
\textbf{ફિઝિકલ} & લેઆઉટ, routing & GDSII ફાઇલ \\
\end{longtable}
}

\begin{itemize}
\tightlist
\item
  \textbf{ડિઝાઇન વેરિફિકેશન}: દરેક લેવલે માન્યતા જરૂરી
\item
  \textbf{પુનરાવર્તન}: ઓપ્ટિમાઇઝેશન માટે ફીડબેક લૂપ્સ
\item
  \textbf{CAD ટૂલ્સ}: જટિલ ડિઝાઇન્સ માટે ઓટોમેશન આવશ્યક
\item
  \textbf{ટાઇમ-ટુ-માર્કેટ}: કાર્યક્ષમ ફ્લો ડિઝાઇન સાયકલ ઘટાડે છે
\end{itemize}

\textbf{યાદરાખવાની ટિપ:} ``System Architects Love Circuit Physical
Fabrication''

\end{solutionbox}
\begin{center}\rule{0.5\linewidth}{0.5pt}\end{center}

\subsection*{પ્રશ્ન 4(અ OR) [3
ગુણ]}\label{uxaaauxab0uxab6uxaa8-4uxa85-or-3-uxa97uxaa3}

\textbf{Y-ચાર્ટ દોરો અને સમજાવો}

\begin{solutionbox}
Y-ચાર્ટ VLSI ડિઝાઇનમાં ત્રણ ડિઝાઇન ડોમેન અને તેમના abstraction
levels દર્શાવે છે.

\textbf{આકૃતિ:}

\begin{center}
\textbf{Mermaid Diagram (Code)}
\begin{verbatim}
{Shaded}
{Highlighting}[]
graph TD
    A[Behavioral Domain] {-{-}{} D[System Level]}
    B[Structural Domain] {-{-}{} D}
    C[Physical Domain] {-{-}{} D}
    A {-{-}{} E[Algorithm Level]}
    B {-{-}{} E}
    C {-{-}{} E}
    A {-{-}{} F[Gate Level]}
    B {-{-}{} F}
    C {-{-}{} F}
{Highlighting}
{Shaded}
\end{verbatim}
\end{center}

\begin{itemize}
\tightlist
\item
  \textbf{ત્રણ ડોમેન}: Behavioral (ફંક્શન), Structural (components),
  Physical (geometry)
\item
  \textbf{એબ્સ્ટ્રેક્શન લેવલ}: System \rightarrow Algorithm \rightarrow Gate \rightarrow Circuit \rightarrow Layout
\item
  \textbf{ડિઝાઇન પદ્ધતિ}: સમાન abstraction level પર ડોમેન વચ્ચે ફરવું
\end{itemize}

\textbf{યાદરાખવાની ટિપ:} ``Behavior, Structure, Physics at All Levels''

\end{solutionbox}
\begin{center}\rule{0.5\linewidth}{0.5pt}\end{center}

\subsection*{પ્રશ્ન 4(બ OR) [4
ગુણ]}\label{uxaaauxab0uxab6uxaa8-4uxaac-or-4-uxa97uxaa3}

\textbf{CMOS ઇન્વર્ટરનો ઉપયોગ કરીને clocked JK latch (NOR ગેટ) અમલ કરો}

\begin{solutionbox}
JK latch toggle ક્ષમતા સાથે SR latch ની પ્રતિબંધિત સ્થિતિ દૂર
કરે છે.

\textbf{આકૃતિ:}

\begin{center}
\textbf{Mermaid Diagram (Code)}
\begin{verbatim}
{Shaded}
{Highlighting}[]
graph LR
    J {-{-}{} A[AND1]}
    CLK {-{-}{} A}
    QB {-{-}{} A}
    A {-{-}{} B[NOR1]}
    B {-{-}{} Q}
    Q {-{-}{} C[NOR2]}
    K {-{-}{} D[AND2]}
    CLK {-{-}{} D}
    Q {-{-}{} D}
    D {-{-}{} C}
    C {-{-}{} QB[Q{}]}
{Highlighting}
{Shaded}
\end{verbatim}
\end{center}

\textbf{સત્ય કોષ્ટક:}

{\def\LTcaptype{none} % do not increment counter
\begin{longtable}[]{@{}llll@{}}
\toprule\noalign{}
\textbf{J} & \textbf{K} & \textbf{Q(next)} & \textbf{ઓપરેશન} \\
\midrule\noalign{}
\endhead
\bottomrule\noalign{}
\endlastfoot
0 & 0 & Q & Hold \\
0 & 1 & 0 & Reset \\
1 & 0 & 1 & Set \\
1 & 1 & Q' & Toggle \\
\end{longtable}
}

\begin{itemize}
\tightlist
\item
  \textbf{ટોગલ મોડ}: J=K=1 આઉટપુટ સ્થિતિ ફ્લિપ કરે છે
\item
  \textbf{ક્લોક એનેબલ}: માત્ર CLK=1 હોય ત્યારે જ સક્રિય
\item
  \textbf{ફીડબેક}: ઇનપુટ્સ સક્ષમ કરવા માટે વર્તમાન આઉટપુટ વાપરે છે
\end{itemize}

\textbf{યાદરાખવાની ટિપ:} ``JK Toggles, No Forbidden State''

\end{solutionbox}
\begin{center}\rule{0.5\linewidth}{0.5pt}\end{center}

\subsection*{પ્રશ્ન 4(ક OR) [7
ગુણ]}\label{uxaaauxab0uxab6uxaa8-4uxa95-or-7-uxa97uxaa3}

\textbf{લિથોગ્રાફી, એચિંગ, જમાવટ, ઓક્સિડેશન, આયન પ્રત્યારોપણ, પ્રસરણ જેવા શબ્દો
સમજાવો}

\begin{solutionbox}
Integrated circuits બનાવવા માટે આવશ્યક semiconductor
fabrication processes.

{\def\LTcaptype{none} % do not increment counter
\begin{longtable}[]{@{}lll@{}}
\toprule\noalign{}
\textbf{પ્રક્રિયા} & \textbf{હેતુ} & \textbf{પદ્ધતિ} \\
\midrule\noalign{}
\endhead
\bottomrule\noalign{}
\endlastfoot
\textbf{લિથોગ્રાફી} & પેટર્ન ટ્રાન્સફર & માસ્ક દ્વારા UV exposure \\
\textbf{એચિંગ} & મેટેરિયલ રિમૂવલ & ભીના/સૂકા રાસાયણિક પ્રક્રિયાઓ \\
\textbf{જમાવટ} & લેયર ઉમેરો & CVD, PVD, sputtering \\
\textbf{ઓક્સિડેશન} & ઇન્સ્યુલેટર વૃદ્ધિ & થર્મલ/પ્લાઝ્મા ઓક્સિડેશન \\
\textbf{આયન પ્રત્યારોપણ} & ડોપિંગ પરિચય & ઉચ્ચ-ઊર્જા આયન બોમ્બાર્ડમેન્ટ \\
\textbf{પ્રસરણ} & ડોપન્ટ વિતરણ & ઉચ્ચ તાપમાને ફેલાવો \\
\end{longtable}
}

\begin{itemize}
\tightlist
\item
  \textbf{પેટર્ન વ્યાખ્યા}: લિથોગ્રાફી ડિવાઇસ લક્ષણો બનાવે છે
\item
  \textbf{પસંદગીયુક્ત રિમૂવલ}: એચિંગ અનચાહેલ મેટેરિયલ દૂર કરે છે
\item
  \textbf{લેયર બિલ્ડિંગ}: જમાવટ જરૂરી મેટેરિયલ ઉમેરે છે
\item
  \textbf{ડોપિંગ કંટ્રોલ}: પ્રત્યારોપણ અને પ્રસરણ junctions બનાવે છે
\item
  \textbf{ગુણવત્તા નિયંત્રણ}: દરેક પગલું અંતિમ ડિવાઇસ પરફોર્મન્સને અસર કરે છે
\end{itemize}

\textbf{યાદરાખવાની ટિપ:} ``Light Etches Deposited Oxides, Ions Diffuse''

\end{solutionbox}
\begin{center}\rule{0.5\linewidth}{0.5pt}\end{center}

\subsection*{પ્રશ્ન 5(અ) [3
ગુણ]}\label{uxaaauxab0uxab6uxaa8-5uxa85-3-uxa97uxaa3}

\textbf{વેરિલોગની મદદથી 2 ઇનપુટ XNOR ગેટને અમલમાં મૂકો}

\begin{solutionbox}
XNOR ગેટ ઇનપુટ્સ સમાન હોય ત્યારે ઊંચો આઉટપુટ આપે છે.

\begin{verbatim}
module xnor\_gate(
    input a, b,
    output y
);
    assign y = {(}a \^{} b);
endmodule
\end{verbatim}

\begin{itemize}
\tightlist
\item
  \textbf{લોજિક ફંક્શન}: Y = (A \oplus B)' = A'B' + AB
\item
  \textbf{સત્ય કોષ્ટક}: ઇનપુટ્સ મેચ થાય ત્યારે આઉટપુટ ઊંચો
\item
  \textbf{ઉપયોગ}: સમાનતા તુલનાકાર, પેરિટી ચેકર
\end{itemize}

\textbf{યાદરાખવાની ટિપ:} ``XNOR Equals Equal Inputs''

\end{solutionbox}
\begin{center}\rule{0.5\linewidth}{0.5pt}\end{center}

\subsection*{પ્રશ્ન 5(બ) [4
ગુણ]}\label{uxaaauxab0uxab6uxaa8-5uxaac-4-uxa97uxaa3}

\textbf{વેરિલોગમાં CASE સ્ટેટમેન્ટનો ઉપયોગ કરીને એનકોડર (8:3) અમલ કરો}

\begin{solutionbox}
પ્રાયોરિટી એનકોડર 8-બિટ ઇનપુટને 3-બિટ બાઇનરી આઉટપુટમાં રૂપાંતરિત
કરે છે.

\begin{verbatim}
module encoder\_8to3(
    input [7:0] in,
    output reg [2:0] out
);
    always @(*) begin
        case(in)
            8{b00000001}: out = 3{b000};
            8{b00000010}: out = 3{b001};
            8{b00000100}: out = 3{b010};
            8{b00001000}: out = 3{b011};
            8{b00010000}: out = 3{b100};
            8{b00100000}: out = 3{b101};
            8{b01000000}: out = 3{b110};
            8{b10000000}: out = 3{b111};
            default: out = 3{b000};
        endcase
    end
endmodule
\end{verbatim}

\begin{itemize}
\tightlist
\item
  \textbf{વન-હોટ એન્કોડિંગ}: માત્ર એક ઇનપુટ બિટ ઊંચો હોવો જોઈએ
\item
  \textbf{પ્રાયોરિટી સ્ટ્રક્ચર}: ઊંચા બિટ્સ પ્રાધાન્ય લે છે
\item
  \textbf{ડિફોલ્ટ કેસ}: અમાન્ય ઇનપુટ સંયોજનો સંભાળે છે
\end{itemize}

\textbf{યાદરાખવાની ટિપ:} ``One Hot Input, Binary Output''

\end{solutionbox}
\begin{center}\rule{0.5\linewidth}{0.5pt}\end{center}

\subsection*{પ્રશ્ન 5(ક) [7
ગુણ]}\label{uxaaauxab0uxab6uxaa8-5uxa95-7-uxa97uxaa3}

\textbf{યોગ્ય ઉદાહરણો સાથે વેરિલોગમાં કેસ સ્ટેટમેન્ટ સમજાવો}

\begin{solutionbox}
CASE સ્ટેટમેન્ટ expression value આધારે મલ્ટિ-વે બ્રાન્ચિંગ આપે છે.

\textbf{સિન્ટેક્સ:}

\begin{verbatim}
case (expression)
    value1: statement1;
    value2: statement2;
    default: default\_statement;
endcase
\end{verbatim}

\textbf{ઉદાહરણ 1 - 4:1 MUX:}

\begin{verbatim}
module mux\_4to1(
    input [1:0] sel,
    input [3:0] in,
    output reg out
);
    always @(*) begin
        case(sel)
            2{b00}: out = in[0];
            2{b01}: out = in[1];
            2{b10}: out = in[2];
            2{b11}: out = in[3];
        endcase
    end
endmodule
\end{verbatim}

\textbf{ઉદાહરણ 2 - 7-સેગમેન્ટ ડિકોડર:}

\begin{verbatim}
case(digit)
    4{h0}: segments = 7{b1111110};
    4{h1}: segments = 7{b0110000};
    4{h2}: segments = 7{b1101101};
    default: segments = 7{b0000000};
endcase
\end{verbatim}

{\def\LTcaptype{none} % do not increment counter
\begin{longtable}[]{@{}lll@{}}
\toprule\noalign{}
\textbf{વેરિઅન્ટ} & \textbf{સિન્ટેક્સ} & \textbf{ઉપયોગ કેસ} \\
\midrule\noalign{}
\endhead
\bottomrule\noalign{}
\endlastfoot
\textbf{case} & case(expr) & સંપૂર્ણ મેચિંગ \\
\textbf{casex} & casex(expr) & ડોન્ટ કેર (X) \\
\textbf{casez} & casez(expr) & હાઈ-Z (Z) \\
\end{longtable}
}

\begin{itemize}
\tightlist
\item
  \textbf{કોમ્બિનેશનલ લોજિક}: always @(*) બ્લોક વાપરો
\item
  \textbf{સિક્વેન્શિયલ લોજિક}: always @(posedge clk) વાપરો
\item
  \textbf{ડિફોલ્ટ કેસ}: સિન્થેસિસમાં લેચેસ અટકાવે છે
\item
  \textbf{પેરેલલ ઇવેલ્યુએશન}: બધા કેસ એકસાથે ચકાસાય છે
\end{itemize}

\textbf{યાદરાખવાની ટિપ:} ``CASE Chooses Actions Systematically
Everywhere''

\end{solutionbox}
\begin{center}\rule{0.5\linewidth}{0.5pt}\end{center}

\subsection*{પ્રશ્ન 5(અ OR) [3
ગુણ]}\label{uxaaauxab0uxab6uxaa8-5uxa85-or-3-uxa97uxaa3}

\textbf{વેરિલોગ કોડનો ઉપયોગ કરીને full subtractor અમલ કરો}

\begin{solutionbox}
Full subtractor borrow input અને output સાથે બાઇનરી બાદબાકી
કરે છે.

\begin{verbatim}
module full\_subtractor(
    input a, b, bin,
    output diff, bout
);
    assign diff = a \^{} b \^{} bin;
    assign bout = ({}a \& b) | ({}a \& bin) | (b \& bin);
endmodule
\end{verbatim}

\textbf{સત્ય કોષ્ટક:}

{\def\LTcaptype{none} % do not increment counter
\begin{longtable}[]{@{}lllll@{}}
\toprule\noalign{}
\textbf{A} & \textbf{B} & \textbf{Bin} & \textbf{Diff} &
\textbf{Bout} \\
\midrule\noalign{}
\endhead
\bottomrule\noalign{}
\endlastfoot
0 & 0 & 0 & 0 & 0 \\
0 & 0 & 1 & 1 & 1 \\
0 & 1 & 0 & 1 & 1 \\
1 & 1 & 1 & 1 & 1 \\
\end{longtable}
}

\begin{itemize}
\tightlist
\item
  \textbf{અંતર}: ત્રણેય ઇનપુટ્સનો XOR
\item
  \textbf{ઉધાર}: A \textless{} (B + Bin) હોય ત્યારે ઉત્પન્ન થાય છે
\end{itemize}

\textbf{યાદરાખવાની ટિપ:} ``Subtract Borrows When Insufficient''

\end{solutionbox}
\begin{center}\rule{0.5\linewidth}{0.5pt}\end{center}

\subsection*{પ્રશ્ન 5(બ OR) [4
ગુણ]}\label{uxaaauxab0uxab6uxaa8-5uxaac-or-4-uxa97uxaa3}

\textbf{વેરિલોગમાં Behavioural મોડેલિંગ શૈલીનો ઉપયોગ કરીને JK flipflop અમલ
કરો}

\begin{solutionbox}
Behavioral modeling વાપરીને toggle ક્ષમતા સાથે JK flip-flop.

\begin{verbatim}
module jk\_flipflop(
    input j, k, clk, reset,
    output reg q, qbar
);
    always @(posedge clk or posedge reset) begin
        if(reset) begin
            q {=} 1{b0};
            qbar {=} 1{b1};
        end
        else begin
            case(\{j,k\)}
                2{b00}: q {=} q;        // Hold
                2{b01}: q {=} 1{b0};     // Reset
                2{b10}: q {=} 1{b1};     // Set
                2{b11}: q {=} {}q;       // Toggle
            endcase
            qbar {=} {}q;
        end
    end
endmodule
\end{verbatim}

\begin{itemize}
\tightlist
\item
  \textbf{બિહેવિયરલ શૈલી}: ફંક્શનનું વર્ણન કરે છે, સ્ટ્રક્ચરનું નહીં
\item
  \textbf{સિંક્રોનસ રીસેટ}: ક્લોક એજ પર રીસેટ
\item
  \textbf{નોન-બ્લોકિંગ અસાઇનમેન્ટ}: clocked always block માં \textless= વાપરો
\end{itemize}

\textbf{યાદરાખવાની ટિપ:} ``JK Behavior: Hold, Reset, Set, Toggle''

\end{solutionbox}
\begin{center}\rule{0.5\linewidth}{0.5pt}\end{center}

\subsection*{પ્રશ્ન 5(ક OR) [7
ગુણ]}\label{uxaaauxab0uxab6uxaa8-5uxa95-or-7-uxa97uxaa3}

\textbf{ઉદાહરણ આપીને વિવિધ વેરિલોગ મોડેલિંગ શૈલી સમજાવો}

\begin{solutionbox}
વેરિલોગ અલગ અલગ abstraction levels માટે ત્રણ મોડેલિંગ શૈલીઓ આપે
છે.

{\def\LTcaptype{none} % do not increment counter
\begin{longtable}[]{@{}llll@{}}
\toprule\noalign{}
\textbf{શૈલી} & \textbf{એબ્સ્ટ્રેક્શન} & \textbf{વર્ણન} & \textbf{કન્સ્ટ્રક્ટ્સ} \\
\midrule\noalign{}
\endhead
\bottomrule\noalign{}
\endlastfoot
\textbf{Behavioral} & ઊંચી & ફંક્શનનું વર્ણન કરે છે & always, if-else, case \\
\textbf{Dataflow} & મધ્યમ & ડેટા હલચલનું વર્ણન કરે છે & assign, operators \\
\textbf{Structural} & નીચી & કનેક્શન્સનું વર્ણન કરે છે & module instantiation \\
\end{longtable}
}

\textbf{1. બિહેવિયરલ મોડેલિંગ:} સર્કિટ શું કરે છે તેનું વર્ણન કરે છે, કેવી રીતે બનાવેલ છે
તે નહીં.

\begin{verbatim}
// 4{-બિટ કાઉન્ટર}
module counter(
    input clk, reset,
    output reg [3:0] count
);
    always @(posedge clk or posedge reset) begin
        if(reset)
            count {=} 4{b0000};
        else
            count {=} count + 1;
    end
endmodule
\end{verbatim}

\textbf{2. ડેટાફ્લો મોડેલિંગ:} કોમ્બિનેશનલ લોજિક માટે સતત અસાઇનમેન્ટ વાપરે છે.

\begin{verbatim}
// 4{-બિટ એડર}
module adder\_4bit(
    input [3:0] a, b,
    input cin,
    output [3:0] sum,
    output cout
);
    assign \{cout, sum\} = a + b + cin;
    assign overflow = (a[3] \& b[3] \& {}sum[3]) | 
                     ({}a[3] \& {}b[3] \& sum[3]);
endmodule
\end{verbatim}

\textbf{3. સ્ટ્રક્ચરલ મોડેલિંગ:} નીચલા-સ્તરના મોડ્યુલો instantiate અને કનેક્ટ કરે છે.

\begin{verbatim}
// હાફ એડર વાપરીને ફુલ એડર
module full\_adder(
    input a, b, cin,
    output sum, cout
);
    wire s1, c1, c2;
    
    half\_adder ha1(.a(a), .b(b), .sum(s1), .carry(c1));
    half\_adder ha2(.a(s1), .b(cin), .sum(sum), .carry(c2));
    
    assign cout = c1 | c2;
endmodule

module half\_adder(
    input a, b,
    output sum, carry
);
    assign sum = a \^{} b;
    assign carry = a \& b;
endmodule
\end{verbatim}

\textbf{તુલના કોષ્ટક:}

{\def\LTcaptype{none} % do not increment counter
\begin{longtable}[]{@{}llll@{}}
\toprule\noalign{}
\textbf{પાસાં} & \textbf{Behavioral} & \textbf{Dataflow} &
\textbf{Structural} \\
\midrule\noalign{}
\endhead
\bottomrule\noalign{}
\endlastfoot
\textbf{જટિલતા} & ઉચ્ચ-સ્તર & મધ્યમ-સ્તર & નીચલા-સ્તર \\
\textbf{વાંચનીયતા} & સૌથી વાંચવા યોગ્ય & મધ્યમ & સૌથી ઓછી વાંચવા યોગ્ય \\
\textbf{સિન્થેસિસ} & ટૂલ આધારિત & સીધી & સ્પષ્ટ \\
\textbf{ડિબગિંગ} & કઠિન & મધ્યમ & સરળ \\
\textbf{પુનઃઉપયોગ} & ઊંચો & મધ્યમ & ઊંચો \\
\end{longtable}
}

\textbf{મિશ્રિત મોડેલિંગ ઉદાહરણ:}

\begin{verbatim}
module cpu\_alu(
    input [7:0] a, b,
    input [2:0] opcode,
    input clk,
    output reg [7:0] result
);
    // બિહેવિયરલ: કંટ્રોલ લોજિક
    always @(posedge clk) begin
        case(opcode)
            3{b000}: result {=} add\_result;
            3{b001}: result {=} sub\_result;
            3{b010}: result {=} and\_result;
            default: result {=} 8{h00};
        endcase
    end
    
    // ડેટાફ્લો: અંકગણિત ઓપરેશન્સ
    wire [7:0] add\_result = a + b;
    wire [7:0] sub\_result = a {-} b;
    wire [7:0] and\_result = a \& b;
    
    // સ્ટ્રક્ચરલ: સમર્પિત અંકગણિત યુનિટ્સ instantiate કરી શકાય
endmodule
\end{verbatim}

\textbf{ડિઝાઇન માર્ગદર્શિકા:}

\begin{itemize}
\tightlist
\item
  \textbf{બિહેવિયરલ}: જટિલ કંટ્રોલ લોજિક, સ્ટેટ મશીન માટે વાપરો
\item
  \textbf{ડેટાફ્લો}: સરળ કોમ્બિનેશનલ લોજિક માટે વાપરો
\item
  \textbf{સ્ટ્રક્ચરલ}: હાયરાર્કિકલ ડિઝાઇન્સ, IP integration માટે વાપરો
\item
  \textbf{મિશ્રિત અભિગમ}: શ્રેષ્ઠ ડિઝાઇન માટે શૈલીઓને જોડો
\end{itemize}

\textbf{સિમ્યુલેશન વિ સિન્થેસિસ:}

\begin{itemize}
\tightlist
\item
  \textbf{બિહેવિયરલ}: અપેક્ષા મુજબ synthesize નહીં થઈ શકે
\item
  \textbf{ડેટાફ્લો}: સીધી હાર્ડવેર મેપિંગ
\item
  \textbf{સ્ટ્રક્ચરલ}: બાંયધરીકૃત સિન્થેસિસ મેચ
\end{itemize}

\textbf{યાદરાખવાની ટિપ:} ``Behavior Describes, Dataflow Assigns,
Structure Connects''

\end{solutionbox}

\end{document}
