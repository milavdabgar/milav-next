\documentclass[10pt,a4paper]{article}

% content/resources/templates/preamble.tex
\usepackage[margin=0.6in]{geometry}
\author{Milav Dabgar}
\usepackage{amsmath,amssymb,amsthm}
\usepackage{booktabs}
\usepackage{multirow}
\usepackage{xcolor}
\usepackage{tcolorbox}
\tcbuselibrary{breakable,skins}
\usepackage[colorlinks=true,linkcolor=blue]{hyperref}
\usepackage{titlesec}
\usepackage{enumitem}
\usepackage{tikz}
\usepackage{pgfplots}
\usepackage{circuitikz}
\usepackage[version=4]{mhchem}
\usepackage{longtable}
\usepackage{array}
\usepackage{float}
\usepackage{caption}
\usepackage{listings}

\lstset{
  basicstyle=\small\ttfamily,
  breaklines=true,
  breakatwhitespace=false,
  postbreak=\mbox{\textcolor{red}{$\hookrightarrow$}\space},
  float=false,
  numbers=left,
  numberstyle=\tiny\color{gray},
  numbersep=10pt,
  xleftmargin=2em,
  keywordstyle=\color{blue},
  commentstyle=\color{green!60!black},
  stringstyle=\color{purple},
  backgroundcolor=\color{gray!5},
  showstringspaces=false,
  tabsize=2,
  captionpos=b,
  keepspaces=true,
  columns=flexible
}

\pgfplotsset{compat=1.18}
\usetikzlibrary{shapes,arrows,positioning,calc,patterns,decorations.pathmorphing,decorations.markings,arrows.meta}

% Color scheme
\definecolor{headcolor}{RGB}{0,102,204}
\definecolor{keycolor}{RGB}{220,20,60}
\definecolor{solutioncolor}{RGB}{34,139,34}
\definecolor{mnemoniccolor}{RGB}{148,0,211}
\definecolor{codecolor}{RGB}{0,0,100}

% Spacing
\setlength{\parskip}{3pt}
\setlist[itemize]{nosep}
\setlist[enumerate]{nosep}

% Title formatting
\titleformat{\section}{\Large\bfseries\color{headcolor}}{\thesection}{1em}{}
\titleformat{\subsection}{\large\bfseries\color{headcolor}}{\thesubsection}{1em}{}

% Pandoc tightlist compatibility
\providecommand{\tightlist}{%
  \setlength{\itemsep}{0pt}\setlength{\parskip}{0pt}}

% Pandoc longtable compatibility
\newcounter{none}
\def\thenone{}


% content/resources/templates/english-boxes.tex
% This file is currently empty - it exists to maintain consistency with the import structure.
% Add custom environments here if needed in the future.


\begin{document}

\begin{center}
{\Huge\bfseries\color{headcolor} Subject Name Solutions}\\[5pt]
{\LARGE 4361102 -- Summer 2024}\\[3pt]
{\large Semester 1 Study Material}\\[3pt]
{\normalsize\textit{Detailed Solutions and Explanations}}
\end{center}

\vspace{10pt}

\subsection*{Question 1(a) [3 marks]}\label{q1a}

\textbf{Draw the structure of FinFET and write its advantages.}

\begin{solutionbox}

\begin{center}
\textbf{Mermaid Diagram (Code)}
\begin{verbatim}
{Shaded}
{Highlighting}[]
graph LR
    A[Source] {-{-}{} B[Gate]}
    B {-{-}{} C[Drain]}
    D[Fin Structure] {-{-}{} E[Multiple Gates]}
    F[Silicon Substrate] {-{-}{} D}
{Highlighting}
{Shaded}
\end{verbatim}
\end{center}


{\def\LTcaptype{none} % do not increment counter
\vspace{-5pt}
\captionof{table}{FinFET Advantages}
\vspace{-10pt}
\begin{longtable}[]{@{}ll@{}}
\toprule\noalign{}
Advantage & Description \\
\midrule\noalign{}
\endhead
\bottomrule\noalign{}
\endlastfoot
\textbf{Better Control} & Multiple gates provide superior channel
control \\
\textbf{Reduced Leakage} & Lower off-state current due to 3D
structure \\
\textbf{Improved Performance} & Higher drive current and faster
switching \\
\end{longtable}
}

\end{solutionbox}
\begin{mnemonicbox}
``BCR - Better Control Reduces leakage''

\end{mnemonicbox}
\begin{center}\rule{0.5\linewidth}{0.5pt}\end{center}

\subsection*{Question 1(b) [4 marks]}\label{q1b}

\textbf{Explain depletion and inversion of MOS structure under external
bias}

\begin{solutionbox}


{\def\LTcaptype{none} % do not increment counter
\vspace{-5pt}
\captionof{table}{MOS Bias Conditions}
\vspace{-10pt}
\begin{longtable}[]{@{}llll@{}}
\toprule\noalign{}
Bias Type & Gate Voltage & Channel State & Charge Carriers \\
\midrule\noalign{}
\endhead
\bottomrule\noalign{}
\endlastfoot
\textbf{Depletion} & Slightly Positive & Depleted & Holes pushed away \\
\textbf{Inversion} & High Positive & Inverted & Electrons attracted \\
\end{longtable}
}

\textbf{Diagram:}

\begin{verbatim}
VG { 0 (Depletion)        VG  0 (Inversion)}
    +                         +
   Gate                      Gate
   {-{-}{-}{-}                      {-}{-}{-}{-}}
    {-                         {-}}
   Depletion                Electron
   Region                   Channel
   {-{-}{-}{-}{-}{-}{-}{-}                 {-}{-}{-}{-}{-}{-}{-}{-}}
   p{-substrate              p{-}substrate}
\end{verbatim}

\begin{itemize}
\tightlist
\item
  \textbf{Depletion}: Positive gate voltage creates electric field
  pushing holes away
\item
  \textbf{Inversion}: Higher voltage attracts electrons forming
  conducting channel
\end{itemize}

\end{solutionbox}
\begin{mnemonicbox}
``DI - Depletion Inverts to conducting channel''

\end{mnemonicbox}
\begin{center}\rule{0.5\linewidth}{0.5pt}\end{center}

\subsection*{Question 1(c) [7 marks]}\label{q1c}

\textbf{Explain n-channel MOSFET with the help of its Current-Voltage
characteristics.}

\begin{solutionbox}


{\def\LTcaptype{none} % do not increment counter
\vspace{-5pt}
\captionof{table}{MOSFET Operating Regions}
\vspace{-10pt}
\begin{longtable}[]{@{}
  >{\raggedright\arraybackslash}p{(\linewidth - 6\tabcolsep) * \real{0.1569}}
  >{\raggedright\arraybackslash}p{(\linewidth - 6\tabcolsep) * \real{0.2157}}
  >{\raggedright\arraybackslash}p{(\linewidth - 6\tabcolsep) * \real{0.2941}}
  >{\raggedright\arraybackslash}p{(\linewidth - 6\tabcolsep) * \real{0.3333}}@{}}
\toprule\noalign{}
\begin{minipage}[b]{\linewidth}\raggedright
Region
\end{minipage} & \begin{minipage}[b]{\linewidth}\raggedright
Condition
\end{minipage} & \begin{minipage}[b]{\linewidth}\raggedright
Drain Current
\end{minipage} & \begin{minipage}[b]{\linewidth}\raggedright
Characteristics
\end{minipage} \\
\midrule\noalign{}
\endhead
\bottomrule\noalign{}
\endlastfoot
\textbf{Cut-off} & VGS \textless{} VTH & ID \approx 0 & No conduction \\
\textbf{Linear} & VDS \textless{} VGS-VTH & ID ∝ VDS & Resistive
behavior \\
\textbf{Saturation} & VDS \geq VGS-VTH & ID ∝ (VGS-VTH)^{2} & Current
independent of VDS \\
\end{longtable}
}

\begin{center}
\textbf{Mermaid Diagram (Code)}
\begin{verbatim}
{Shaded}
{Highlighting}[]
graph LR
    A[Gate] {-{-}{} B[n{-}channel]}
    C[Source] {-{-}{} B}
    B {-{-}{} D[Drain]}
    E[p{-substrate] {-}{-}{} B}
{Highlighting}
{Shaded}
\end{verbatim}
\end{center}

\textbf{Key Equations:}

\begin{itemize}
\item
  Linear: ID = μnCox(W/L)[(VGS-VTH)VDS - VDS^{2}/2]
\item
  Saturation: ID = (μnCox/2)(W/L)(VGS-VTH)^{2}
\item
  \textbf{Structure}: Gate controls channel between source and drain
\item
  \textbf{Operation}: Gate voltage modulates channel conductivity
\item
  \textbf{Applications}: Digital switching and analog amplification
\end{itemize}

\end{solutionbox}
\begin{mnemonicbox}
``CLS - Cut-off, Linear, Saturation regions''

\end{mnemonicbox}
\begin{center}\rule{0.5\linewidth}{0.5pt}\end{center}

\subsection*{Question 1(c OR) [7
marks]}\label{question-1c-or-7-marks}

\textbf{Define scaling. Compare full voltage scaling with constant
voltage scaling. Write the disadvantages of scaling.}

\begin{solutionbox}

\textbf{Definition:} Scaling reduces device dimensions to increase
density and performance.


{\def\LTcaptype{none} % do not increment counter
\vspace{-5pt}
\captionof{table}{Scaling Comparison}
\vspace{-10pt}
\begin{longtable}[]{@{}lll@{}}
\toprule\noalign{}
Parameter & Full Voltage Scaling & Constant Voltage Scaling \\
\midrule\noalign{}
\endhead
\bottomrule\noalign{}
\endlastfoot
\textbf{Voltage} & Reduced by α & Remains constant \\
\textbf{Power Density} & Constant & Increases by α \\
\textbf{Electric Field} & Constant & Increases by α \\
\textbf{Performance} & Better & Moderate improvement \\
\end{longtable}
}

\textbf{Disadvantages:}

\begin{itemize}
\tightlist
\item
  \textbf{Short Channel Effects}: Channel length modulation increases
\item
  \textbf{Hot Carrier Effects}: High electric fields damage devices
\item
  \textbf{Quantum Effects}: Tunneling currents increase significantly
\end{itemize}

\end{solutionbox}
\begin{mnemonicbox}
``SHQ - Short channel, Hot carriers, Quantum
effects''

\end{mnemonicbox}
\begin{center}\rule{0.5\linewidth}{0.5pt}\end{center}

\subsection*{Question 2(a) [3 marks]}\label{q2a}

\textbf{Draw two input NAND gate using CMOS.}

\begin{solutionbox}

\begin{verbatim}
    VDD
     |
   ┌─┴─┐ pMOS
A──┤   ├──┐
   └───┘  │
          │ Y
   ┌─────┐ │
B──┤     ├─┘
   └─┬─┬─┘ pMOS
     │ │
   ┌─┴─┐ nMOS
A──┤   ├──┐
   └───┘  │
          │
   ┌─────┐ │
B──┤     ├─┘
   └─┬─┬─┘ nMOS
     │ │
    GND
\end{verbatim}


{\def\LTcaptype{none} % do not increment counter
\vspace{-5pt}
\captionof{table}{NAND Truth Table}
\vspace{-10pt}
\begin{longtable}[]{@{}lll@{}}
\toprule\noalign{}
A & B & Y \\
\midrule\noalign{}
\endhead
\bottomrule\noalign{}
\endlastfoot
0 & 0 & 1 \\
0 & 1 & 1 \\
1 & 0 & 1 \\
1 & 1 & 0 \\
\end{longtable}
}

\end{solutionbox}
\begin{mnemonicbox}
``PP-SS: Parallel PMOS, Series NMOS''

\end{mnemonicbox}
\begin{center}\rule{0.5\linewidth}{0.5pt}\end{center}

\subsection*{Question 2(b) [4 marks]}\label{q2b}

\textbf{Explain noise immunity and noise margin for nMOS inverter.}

\begin{solutionbox}


{\def\LTcaptype{none} % do not increment counter
\vspace{-5pt}
\captionof{table}{Noise Parameters}
\vspace{-10pt}
\begin{longtable}[]{@{}lll@{}}
\toprule\noalign{}
Parameter & Definition & Formula \\
\midrule\noalign{}
\endhead
\bottomrule\noalign{}
\endlastfoot
\textbf{NMH} & High noise margin & VOH - VIH \\
\textbf{NML} & Low noise margin & VIL - VOL \\
\textbf{Noise Immunity} & Ability to reject noise & Min(NMH, NML) \\
\end{longtable}
}

\begin{center}
\textbf{Mermaid Diagram (Code)}
\begin{verbatim}
{Shaded}
{Highlighting}[]
graph TD
    A[VIL] {-{-}{} B[VIH]}
    C[VOL] {-{-}{} D[VOH]}
    E[NML] {-{-}{} F[NMH]}
{Highlighting}
{Shaded}
\end{verbatim}
\end{center}

\begin{itemize}
\tightlist
\item
  \textbf{VIL}: Maximum low input voltage
\item
  \textbf{VIH}: Minimum high input voltage\\
\item
  \textbf{Good noise immunity}: Large noise margins prevent false
  switching
\end{itemize}

\end{solutionbox}
\begin{mnemonicbox}
``HILOL - High/Low Input/Output Levels''

\end{mnemonicbox}
\begin{center}\rule{0.5\linewidth}{0.5pt}\end{center}

\subsection*{Question 2(c) [7 marks]}\label{q2c}

\textbf{Explain Voltage Transfer Characteristics (VTC) of CMOS
inverter.}

\begin{solutionbox}


{\def\LTcaptype{none} % do not increment counter
\vspace{-5pt}
\captionof{table}{VTC Regions}
\vspace{-10pt}
\begin{longtable}[]{@{}llll@{}}
\toprule\noalign{}
Region & Input Range & Output & Transistor States \\
\midrule\noalign{}
\endhead
\bottomrule\noalign{}
\endlastfoot
\textbf{A} & 0 to VTN & VDD & pMOS ON, nMOS OFF \\
\textbf{B} & VTN to VDD/2 & Transition & Both partially ON \\
\textbf{C} & VDD/2 to VDD- & VTP & \\
\textbf{D} & VDD- & VTP & to VDD \\
\end{longtable}
}

\begin{center}
\textbf{Mermaid Diagram (Code)}
\begin{verbatim}
{Shaded}
{Highlighting}[]
graph TD
    A[VIN = 0] {-{-}{} B[VOUT = VDD]}
    C[VIN = VDD/2] {-{-}{} D[VOUT = VDD/2]}
    E[VIN = VDD] {-{-}{} F[VOUT = 0]}
{Highlighting}
{Shaded}
\end{verbatim}
\end{center}

\textbf{Key Features:}

\begin{itemize}
\tightlist
\item
  \textbf{Sharp transition}: Ideal switching behavior
\item
  \textbf{High gain}: Large slope in transition region
\item
  \textbf{Rail-to-rail}: Output swings full supply range
\end{itemize}

\end{solutionbox}
\begin{mnemonicbox}
``ASH - A-region, Sharp transition, High gain''

\end{mnemonicbox}
\begin{center}\rule{0.5\linewidth}{0.5pt}\end{center}

\subsection*{Question 2(a OR) [3
marks]}\label{question-2a-or-3-marks}

\textbf{Implement NOR2 gate using depletion load nMOS.}

\begin{solutionbox}

\begin{verbatim}
    VDD
     |
   ┌─┴─┐ Depletion Load
   │   │ (VGS = 0)
   └─┬─┘
     │ Y
   ┌─┴─┐ nMOS
A──┤   ├──┐
   └───┘  │
          │
   ┌─────┐ │
B──┤     ├─┘
   └─┬─┬─┘ nMOS
     │ │
    GND
\end{verbatim}


{\def\LTcaptype{none} % do not increment counter
\vspace{-5pt}
\captionof{table}{NOR2 Truth Table}
\vspace{-10pt}
\begin{longtable}[]{@{}lll@{}}
\toprule\noalign{}
A & B & Y \\
\midrule\noalign{}
\endhead
\bottomrule\noalign{}
\endlastfoot
0 & 0 & 1 \\
0 & 1 & 0 \\
1 & 0 & 0 \\
1 & 1 & 0 \\
\end{longtable}
}

\end{solutionbox}
\begin{mnemonicbox}
``DPN - Depletion load, Parallel NMOS''

\end{mnemonicbox}
\begin{center}\rule{0.5\linewidth}{0.5pt}\end{center}

\subsection*{Question 2(b OR) [4
marks]}\label{question-2b-or-4-marks}

\textbf{Differentiate between enhancement load inverter and Depletion
load inverter.}

\begin{solutionbox}


{\def\LTcaptype{none} % do not increment counter
\vspace{-5pt}
\captionof{table}{Load Inverter Comparison}
\vspace{-10pt}
\begin{longtable}[]{@{}lll@{}}
\toprule\noalign{}
Parameter & Enhancement Load & Depletion Load \\
\midrule\noalign{}
\endhead
\bottomrule\noalign{}
\endlastfoot
\textbf{Threshold Voltage} & VT \textgreater{} 0 & VT \textless{} 0 \\
\textbf{Gate Connection} & VGS = VDS & VGS = 0 \\
\textbf{Logic High} & VDD - VT & VDD \\
\textbf{Power Consumption} & Higher & Lower \\
\textbf{Switching Speed} & Slower & Faster \\
\end{longtable}
}

\begin{itemize}
\tightlist
\item
  \textbf{Enhancement}: Requires positive gate voltage for conduction
\item
  \textbf{Depletion}: Conducts with zero gate voltage
\item
  \textbf{Performance}: Depletion load provides better characteristics
\end{itemize}

\end{solutionbox}
\begin{mnemonicbox}
``EPDLH - Enhancement Positive, Depletion Lower
power, Higher speed''

\end{mnemonicbox}
\begin{center}\rule{0.5\linewidth}{0.5pt}\end{center}

\subsection*{Question 2(c OR) [7
marks]}\label{question-2c-or-7-marks}

\textbf{Explain Depletion load nMOS inverter with its VTC.}

\begin{solutionbox}

\textbf{Circuit Operation:}

\begin{itemize}
\tightlist
\item
  \textbf{Load transistor}: Always conducting (VGS = 0, VT \textless{}
  0)
\item
  \textbf{Driver transistor}: Controlled by input voltage
\item
  \textbf{Output}: Determined by voltage divider action
\end{itemize}

\begin{center}
\textbf{Mermaid Diagram (Code)}
\begin{verbatim}
{Shaded}
{Highlighting}[]
graph LR
    A[VIN Low] {-{-}{} B[Driver OFF]}
    B {-{-}{} C[VOUT = VDD]}
    D[VIN High] {-{-}{} E[Driver ON]}
    E {-{-}{} F[VOUT  0V]}
{Highlighting}
{Shaded}
\end{verbatim}
\end{center}


{\def\LTcaptype{none} % do not increment counter
\vspace{-5pt}
\captionof{table}{Operating Points}
\vspace{-10pt}
\begin{longtable}[]{@{}llll@{}}
\toprule\noalign{}
Input State & Driver & Load & Output \\
\midrule\noalign{}
\endhead
\bottomrule\noalign{}
\endlastfoot
\textbf{VIN = 0} & OFF & ON & VDD \\
\textbf{VIN = VDD} & ON & ON & \approx 0V \\
\end{longtable}
}

\textbf{VTC Characteristics:}

\begin{itemize}
\tightlist
\item
  \textbf{VOH}: VDD (better than enhancement load)
\item
  \textbf{VOL}: Lower due to depletion load characteristics
\item
  \textbf{Transition}: Sharp switching between states
\end{itemize}

\end{solutionbox}
\begin{mnemonicbox}
``DLB - Depletion Load gives Better high output''

\end{mnemonicbox}
\begin{center}\rule{0.5\linewidth}{0.5pt}\end{center}

\subsection*{Question 3(a) [3 marks]}\label{q3a}

\textbf{Implement EX-OR using Depletion load nMOS.}

\begin{solutionbox}

\begin{verbatim}
    VDD      VDD
     |        |
   ┌─┴─┐    ┌─┴─┐  Depletion
   │   │    │   │  Loads
   └─┬─┘    └─┬─┘
     │        │
A────┼────────┼────B
     │        │
   ┌─┴─┐    ┌─┴─┐
   │   │    │   │  nMOS
   └─┬─┘    └─┬─┘
     │        │
    A{       B}
     │        │
     └────┬───┘ Y
          │
         GND
\end{verbatim}


{\def\LTcaptype{none} % do not increment counter
\vspace{-5pt}
\captionof{table}{XOR Truth Table}
\vspace{-10pt}
\begin{longtable}[]{@{}lll@{}}
\toprule\noalign{}
A & B & Y \\
\midrule\noalign{}
\endhead
\bottomrule\noalign{}
\endlastfoot
0 & 0 & 0 \\
0 & 1 & 1 \\
1 & 0 & 1 \\
1 & 1 & 0 \\
\end{longtable}
}

\textbf{Implementation}: Y = A\oplusB = A'B + AB'

\end{solutionbox}
\begin{mnemonicbox}
``XOR - eXclusive OR, different inputs give 1''

\end{mnemonicbox}
\begin{center}\rule{0.5\linewidth}{0.5pt}\end{center}

\subsection*{Question 3(b) [4 marks]}\label{q3b}

\textbf{Explain design hierarchy with example.}

\begin{solutionbox}


{\def\LTcaptype{none} % do not increment counter
\vspace{-5pt}
\captionof{table}{Hierarchy Levels}
\vspace{-10pt}
\begin{longtable}[]{@{}lll@{}}
\toprule\noalign{}
Level & Component & Example \\
\midrule\noalign{}
\endhead
\bottomrule\noalign{}
\endlastfoot
\textbf{System} & Complete chip & Microprocessor \\
\textbf{Module} & Functional blocks & ALU, Memory \\
\textbf{Gate} & Logic gates & NAND, NOR \\
\textbf{Transistor} & Individual devices & MOSFET \\
\end{longtable}
}

\begin{center}
\textbf{Mermaid Diagram (Code)}
\begin{verbatim}
{Shaded}
{Highlighting}[]
graph LR
    A[System Level] {-{-}{} B[Module Level]}
    B {-{-}{} C[Gate Level]}
    C {-{-}{} D[Transistor Level]}
    E[CPU] {-{-}{} F[ALU]}
    F {-{-}{} G[Adder]}
    G {-{-}{} H[MOSFET]}
{Highlighting}
{Shaded}
\end{verbatim}
\end{center}

\textbf{Benefits:}

\begin{itemize}
\tightlist
\item
  \textbf{Modularity}: Independent design and testing
\item
  \textbf{Reusability}: Common blocks used multiple times
\item
  \textbf{Maintainability}: Easy debugging and modification
\end{itemize}

\end{solutionbox}
\begin{mnemonicbox}
``SMG-T: System, Module, Gate, Transistor levels''

\end{mnemonicbox}
\begin{center}\rule{0.5\linewidth}{0.5pt}\end{center}

\subsection*{Question 3(c) [7 marks]}\label{q3c}

\textbf{Draw and explain Y chart design flow.}

\begin{solutionbox}

\begin{center}
\textbf{Mermaid Diagram (Code)}
\begin{verbatim}
{Shaded}
{Highlighting}[]
graph LR
    A[Behavioral Domain] {-{-}{} D[System Specification]}
    B[Structural Domain] {-{-}{} E[Architecture]}
    C[Physical Domain] {-{-}{} F[Floor Plan]}
    D {-{-}{} G[Algorithm]}
    E {-{-}{} H[Logic Design]}
    F {-{-}{} I[Layout]}
    G {-{-}{} J[RTL]}
    H {-{-}{} K[Gate Level]}
    I {-{-}{} L[Transistor Level]}
{Highlighting}
{Shaded}
\end{verbatim}
\end{center}


{\def\LTcaptype{none} % do not increment counter
\vspace{-5pt}
\captionof{table}{Y-Chart Domains}
\vspace{-10pt}
\begin{longtable}[]{@{}lll@{}}
\toprule\noalign{}
Domain & Description & Examples \\
\midrule\noalign{}
\endhead
\bottomrule\noalign{}
\endlastfoot
\textbf{Behavioral} & What system does & Algorithms, RTL \\
\textbf{Structural} & How it's organized & Architecture, Gates \\
\textbf{Physical} & Where components placed & Floorplan, Layout \\
\end{longtable}
}

\textbf{Design Flow:}

\begin{itemize}
\tightlist
\item
  \textbf{Top-down}: Behavioral \rightarrow Structural \rightarrow Physical
\item
  \textbf{Bottom-up}: Physical constraints influence upper levels
\item
  \textbf{Iterative}: Multiple passes for optimization
\end{itemize}

\end{solutionbox}
\begin{mnemonicbox}
``BSP - Behavioral, Structural, Physical domains''

\end{mnemonicbox}
\begin{center}\rule{0.5\linewidth}{0.5pt}\end{center}

\subsection*{Question 3(a OR) [3
marks]}\label{question-3a-or-3-marks}

\textbf{Implement NAND2 - SR latch using CMOS}

\begin{solutionbox}

\begin{verbatim}
    S ────┐   ┌──── Q
          │   │
        ┌─┴─┐ │ ┌─┴─┐
        │   ├─┘ │   │ NAND
        └─┬─┘   └─┬─┘
          │       │
          └───┬───┘
              │
        ┌─────┴─────┐
        │           │
      ┌─┴─┐       ┌─┴─┐
      │   ├───────┤   │ NAND
      └─┬─┘       └─┬─┘
        │           │
        R ──────────┘
                    │
                   Q{}
\end{verbatim}


{\def\LTcaptype{none} % do not increment counter
\vspace{-5pt}
\captionof{table}{SR Latch Operation}
\vspace{-10pt}
\begin{longtable}[]{@{}lllll@{}}
\toprule\noalign{}
S & R & Q & Q' & State \\
\midrule\noalign{}
\endhead
\bottomrule\noalign{}
\endlastfoot
0 & 0 & Q & Q' & Hold \\
0 & 1 & 0 & 1 & Reset \\
1 & 0 & 1 & 0 & Set \\
1 & 1 & 1 & 1 & Invalid \\
\end{longtable}
}

\end{solutionbox}
\begin{mnemonicbox}
``SR-HRI: Set, Reset, Hold, Invalid states''

\end{mnemonicbox}
\begin{center}\rule{0.5\linewidth}{0.5pt}\end{center}

\subsection*{Question 3(b OR) [4
marks]}\label{question-3b-or-4-marks}

\textbf{Which method is used to transfer pattern or mask on the silicon
wafer? Explain it with neat diagrams}

\begin{solutionbox}

\textbf{Method}: \textbf{Lithography} - Pattern transfer using light
exposure

\begin{center}
\textbf{Mermaid Diagram (Code)}
\begin{verbatim}
{Shaded}
{Highlighting}[]
graph LR
    A[UV Light Source] {-{-}{} B[Mask with Pattern]}
    B {-{-}{} C[Photoresist on Wafer]}
    C {-{-}{} D[Exposed Pattern]}
    D {-{-}{} E[Developed Pattern]}
{Highlighting}
{Shaded}
\end{verbatim}
\end{center}

\textbf{Process Steps:}

{\def\LTcaptype{none} % do not increment counter
\begin{longtable}[]{@{}lll@{}}
\toprule\noalign{}
Step & Action & Result \\
\midrule\noalign{}
\endhead
\bottomrule\noalign{}
\endlastfoot
\textbf{Coating} & Apply photoresist & Uniform layer \\
\textbf{Exposure} & UV through mask & Chemical change \\
\textbf{Development} & Remove exposed resist & Pattern transfer \\
\end{longtable}
}

\textbf{Applications}: Creating gates, interconnects, contact holes

\end{solutionbox}
\begin{mnemonicbox}
``CED - Coating, Exposure, Development''

\end{mnemonicbox}
\begin{center}\rule{0.5\linewidth}{0.5pt}\end{center}

\subsection*{Question 3(c OR) [7
marks]}\label{question-3c-or-7-marks}

\textbf{Which are the methods used to deposit metal in MOSFET
fabrication? Explain deposition in detail with proper diagram.}

\begin{solutionbox}


{\def\LTcaptype{none} % do not increment counter
\vspace{-5pt}
\captionof{table}{Metal Deposition Methods}
\vspace{-10pt}
\begin{longtable}[]{@{}
  >{\raggedright\arraybackslash}p{(\linewidth - 4\tabcolsep) * \real{0.2500}}
  >{\raggedright\arraybackslash}p{(\linewidth - 4\tabcolsep) * \real{0.3438}}
  >{\raggedright\arraybackslash}p{(\linewidth - 4\tabcolsep) * \real{0.4062}}@{}}
\toprule\noalign{}
\begin{minipage}[b]{\linewidth}\raggedright
Method
\end{minipage} & \begin{minipage}[b]{\linewidth}\raggedright
Technique
\end{minipage} & \begin{minipage}[b]{\linewidth}\raggedright
Application
\end{minipage} \\
\midrule\noalign{}
\endhead
\bottomrule\noalign{}
\endlastfoot
\textbf{Physical Vapor Deposition} & Sputtering, Evaporation & Aluminum,
Copper \\
\textbf{Chemical Vapor Deposition} & CVD, PECVD & Tungsten, Titanium \\
\textbf{Electroplating} & Electrochemical & Copper interconnects \\
\end{longtable}
}

\begin{center}
\textbf{Mermaid Diagram (Code)}
\begin{verbatim}
{Shaded}
{Highlighting}[]
graph LR
    A[Target Material] {-{-}{} B[Ion Bombardment]}
    B {-{-}{} C[Ejected Atoms]}
    C {-{-}{} D[Substrate Coating]}
    E[Wafer] {-{-}{} D}
{Highlighting}
{Shaded}
\end{verbatim}
\end{center}

\textbf{Sputtering Process:}

\begin{itemize}
\tightlist
\item
  \textbf{Ion bombardment}: Argon ions hit target material
\item
  \textbf{Atom ejection}: Target atoms knocked off
\item
  \textbf{Deposition}: Atoms settle on wafer surface
\item
  \textbf{Control}: Pressure and power determine rate
\end{itemize}

\textbf{Advantages:}

\begin{itemize}
\tightlist
\item
  \textbf{Uniform thickness}: Excellent step coverage
\item
  \textbf{Low temperature}: Preserves device integrity
\item
  \textbf{Variety}: Multiple materials possible
\end{itemize}

\end{solutionbox}
\begin{mnemonicbox}
``IBE-DC: Ion Bombardment Ejects atoms for Deposition
Control''

\end{mnemonicbox}
\begin{center}\rule{0.5\linewidth}{0.5pt}\end{center}

\subsection*{Question 4(a) [3 marks]}\label{q4a}

\textbf{Implement Z= ((A+B+C)·(D+E+F). G)' with depletion nMOS load.}

\begin{solutionbox}

\begin{verbatim}
    VDD
     |
   ┌─┴─┐ Depletion Load
   │   │
   └─┬─┘
     │ Z
A────┼────┐
B────┼────┤ Parallel
C────┼────┘ (OR)
     │
D────┼────┐
E────┼────┤ Parallel  
F────┼────┘ (OR)
     │
G────┼────┘ Series
     │      (AND)
    GND
\end{verbatim}

\textbf{Logic Implementation:}

\begin{itemize}
\tightlist
\item
  \textbf{First level}: (A+B+C) and (D+E+F) OR functions
\item
  \textbf{Second level}: AND with G
\item
  \textbf{Output}: Inverted result due to nMOS structure
\end{itemize}

\end{solutionbox}
\begin{mnemonicbox}
``POI - Parallel OR, Inversion at output''

\end{mnemonicbox}
\begin{center}\rule{0.5\linewidth}{0.5pt}\end{center}

\subsection*{Question 4(b) [4 marks]}\label{q4b}

\textbf{List and explain the design styles used in VERILOG.}

\begin{solutionbox}


{\def\LTcaptype{none} % do not increment counter
\vspace{-5pt}
\captionof{table}{Verilog Design Styles}
\vspace{-10pt}
\begin{longtable}[]{@{}
  >{\raggedright\arraybackslash}p{(\linewidth - 6\tabcolsep) * \real{0.1795}}
  >{\raggedright\arraybackslash}p{(\linewidth - 6\tabcolsep) * \real{0.3333}}
  >{\raggedright\arraybackslash}p{(\linewidth - 6\tabcolsep) * \real{0.2564}}
  >{\raggedright\arraybackslash}p{(\linewidth - 6\tabcolsep) * \real{0.2308}}@{}}
\toprule\noalign{}
\begin{minipage}[b]{\linewidth}\raggedright
Style
\end{minipage} & \begin{minipage}[b]{\linewidth}\raggedright
Description
\end{minipage} & \begin{minipage}[b]{\linewidth}\raggedright
Use Case
\end{minipage} & \begin{minipage}[b]{\linewidth}\raggedright
Example
\end{minipage} \\
\midrule\noalign{}
\endhead
\bottomrule\noalign{}
\endlastfoot
\textbf{Behavioral} & Algorithm description & High-level modeling &
always blocks \\
\textbf{Dataflow} & Boolean expressions & Combinational logic & assign
statements \\
\textbf{Structural} & Component instantiation & Hierarchical design &
module connections \\
\textbf{Gate-level} & Primitive gates & Low-level design & and, or, not
gates \\
\end{longtable}
}

\textbf{Characteristics:}

\begin{itemize}
\tightlist
\item
  \textbf{Behavioral}: Describes what circuit does
\item
  \textbf{Structural}: Shows how components connect
\item
  \textbf{Mixed}: Combines multiple styles for complex designs
\end{itemize}

\end{solutionbox}
\begin{mnemonicbox}
``BDSG - Behavioral, Dataflow, Structural,
Gate-level''

\end{mnemonicbox}
\begin{center}\rule{0.5\linewidth}{0.5pt}\end{center}

\subsection*{Question 4(c) [7 marks]}\label{q4c}

\textbf{Implement NAND2 SR latch using CMOS and also implement NOR2 SR
latch using CMOS.}

\begin{solutionbox}

\textbf{NAND2 SR Latch:}

\begin{verbatim}
module nand\_sr\_latch(
    input S, R,
    output Q, Q\_bar
);
    nand(Q, S, Q\_bar);
    nand(Q\_bar, R, Q);
endmodule
\end{verbatim}

\textbf{NOR2 SR Latch:}

\begin{verbatim}
module nor\_sr\_latch(
    input S, R,
    output Q, Q\_bar
);
    nor(Q\_bar, R, Q);
    nor(Q, S, Q\_bar);
endmodule
\end{verbatim}


{\def\LTcaptype{none} % do not increment counter
\vspace{-5pt}
\captionof{table}{Latch Comparison}
\vspace{-10pt}
\begin{longtable}[]{@{}llll@{}}
\toprule\noalign{}
Type & Active Level & Set Operation & Reset Operation \\
\midrule\noalign{}
\endhead
\bottomrule\noalign{}
\endlastfoot
\textbf{NAND} & Low (0) & S=0, R=1 & S=1, R=0 \\
\textbf{NOR} & High (1) & S=1, R=0 & S=0, R=1 \\
\end{longtable}
}

\textbf{Key Differences:}

\begin{itemize}
\tightlist
\item
  \textbf{NAND}: Set/Reset with low inputs
\item
  \textbf{NOR}: Set/Reset with high inputs
\item
  \textbf{Feedback}: Cross-coupled gates maintain state
\end{itemize}

\end{solutionbox}
\begin{mnemonicbox}
``NAND-Low, NOR-High active''

\end{mnemonicbox}
\begin{center}\rule{0.5\linewidth}{0.5pt}\end{center}

\subsection*{Question 4(a OR) [3
marks]}\label{question-4a-or-3-marks}

\textbf{Implement Y= (ABC + DE + F)' with depletion nMOS load.}

\begin{solutionbox}

\begin{verbatim}
    VDD
     |
   ┌─┴─┐ Depletion Load
   │   │
   └─┬─┘
     │ Y
A────┼────┐
B────┼────┤ Series (AND)
C────┼────┘
     │
D────┼────┐
E────┼────┘ Series (AND)
     │
F────┼────┘ Single
     │
    GND
\end{verbatim}

\textbf{Implementation Logic:}

\begin{itemize}
\tightlist
\item
  \textbf{ABC}: Series connection (AND function)
\item
  \textbf{DE}: Series connection (AND function)\\
\item
  \textbf{F}: Single transistor
\item
  \textbf{Result}: Y = (ABC + DE + F)' due to inversion
\end{itemize}

\end{solutionbox}
\begin{mnemonicbox}
``SSS-I: Series-Series-Single with Inversion''

\end{mnemonicbox}
\begin{center}\rule{0.5\linewidth}{0.5pt}\end{center}

\subsection*{Question 4(b OR) [4
marks]}\label{question-4b-or-4-marks}

\textbf{Write Verilog Code to implement full adder.}

\begin{solutionbox}

\begin{verbatim}
module full\_adder(
    input a, b, cin,
    output sum, cout
);
    assign sum = a \^{} b \^{} cin;
    assign cout = (a \& b) | (cin \& (a \^{} b));
endmodule
\end{verbatim}


{\def\LTcaptype{none} % do not increment counter
\vspace{-5pt}
\captionof{table}{Full Adder Truth Table}
\vspace{-10pt}
\begin{longtable}[]{@{}lllll@{}}
\toprule\noalign{}
A & B & Cin & Sum & Cout \\
\midrule\noalign{}
\endhead
\bottomrule\noalign{}
\endlastfoot
0 & 0 & 0 & 0 & 0 \\
0 & 0 & 1 & 1 & 0 \\
0 & 1 & 0 & 1 & 0 \\
0 & 1 & 1 & 0 & 1 \\
1 & 0 & 0 & 1 & 0 \\
1 & 0 & 1 & 0 & 1 \\
1 & 1 & 0 & 0 & 1 \\
1 & 1 & 1 & 1 & 1 \\
\end{longtable}
}

\textbf{Logic Functions:}

\begin{itemize}
\tightlist
\item
  \textbf{Sum}: Triple XOR operation
\item
  \textbf{Carry}: Majority function of inputs
\end{itemize}

\end{solutionbox}
\begin{mnemonicbox}
``XOR-Sum, Majority-Carry''

\end{mnemonicbox}
\begin{center}\rule{0.5\linewidth}{0.5pt}\end{center}

\subsection*{Question 4(c OR) [7
marks]}\label{question-4c-or-7-marks}

\textbf{Implement Y =(S1'S0'I0 + S1'S0 I1 + S1 S0' I2 + S1 S2 I3) using
depletion load}

\begin{solutionbox}

\textbf{Note}: Assuming S2 in last term should be S0.

\begin{verbatim}
// 4:1 Multiplexer implementation
module mux\_4to1(
    input [1:0] sel,  // S1, S0
    input [3:0] data, // I3, I2, I1, I0
    output Y
);
assign

Y = (sel == 2{b00}) ? data[0] :

               (sel == 2{b01}) ? data[1] :
               (sel == 2{b10}) ? data[2] :
                                data[3];
endmodule
\end{verbatim}


{\def\LTcaptype{none} % do not increment counter
\vspace{-5pt}
\captionof{table}{Multiplexer Selection}
\vspace{-10pt}
\begin{longtable}[]{@{}llll@{}}
\toprule\noalign{}
S1 & S0 & Selected Input & Output \\
\midrule\noalign{}
\endhead
\bottomrule\noalign{}
\endlastfoot
0 & 0 & I0 & Y = I0 \\
0 & 1 & I1 & Y = I1 \\
1 & 0 & I2 & Y = I2 \\
1 & 1 & I3 & Y = I3 \\
\end{longtable}
}

\textbf{Circuit Implementation:}

\begin{itemize}
\tightlist
\item
  \textbf{Decoder}: S1, S0 generate select signals
\item
  \textbf{AND gates}: Each input ANDed with corresponding select
\item
  \textbf{OR gate}: Combines all AND outputs
\end{itemize}

\end{solutionbox}
\begin{mnemonicbox}
``DAO - Decoder, AND gates, OR combination''

\end{mnemonicbox}
\begin{center}\rule{0.5\linewidth}{0.5pt}\end{center}

\subsection*{Question 5(a) [3 marks]}\label{q5a}

\textbf{Implement the logic function G = (PQR +U(S+T))' using CMOS}

\begin{solutionbox}

\begin{verbatim}
    VDD
     |
P────┼────┐
Q────┼────┤ Parallel pMOS
R────┼────┘ (NOR)
     │
U────┼────┐
     │    │
S────┼────┤ Series pMOS  
T────┼────┘ (NAND)
     │
     │ G
P────┼────┐
Q────┼────┤ Series nMOS
R────┼────┘ (AND)
     │
U────┼────┐
     │    │
S────┼────┤ Parallel nMOS
T────┼────┘ (OR)
     │
    GND
\end{verbatim}

\textbf{Implementation}:

\begin{itemize}
\tightlist
\item
  \textbf{pMOS}: Parallel for OR, Series for AND (inverted logic)
\item
  \textbf{nMOS}: Series for AND, Parallel for OR (normal logic)
\item
  \textbf{Result}: De Morgan's law applied automatically
\end{itemize}

\end{solutionbox}
\begin{mnemonicbox}
``PSSP - Parallel Series Series Parallel''

\end{mnemonicbox}
\begin{center}\rule{0.5\linewidth}{0.5pt}\end{center}

\subsection*{Question 5(b) [4 marks]}\label{q5b}

\textbf{Implement 8\times1 multiplexer using Verilog}

\begin{solutionbox}

\begin{verbatim}
module mux\_8to1(
    input [2:0] sel,     // 3{-bit select}
    input [7:0] data,    // 8 data inputs
    output reg Y         // Output
);
    always @(*) begin
        case(sel)
            3{b000}: Y = data[0];
            3{b001}: Y = data[1];
            3{b010}: Y = data[2];
            3{b011}: Y = data[3];
            3{b100}: Y = data[4];
            3{b101}: Y = data[5];
            3{b110}: Y = data[6];
            3{b111}: Y = data[7];
        endcase
    end
endmodule
\end{verbatim}


{\def\LTcaptype{none} % do not increment counter
\vspace{-5pt}
\captionof{table}{8:1 MUX Selection}
\vspace{-10pt}
\begin{longtable}[]{@{}llll@{}}
\toprule\noalign{}
S2 & S1 & S0 & Output \\
\midrule\noalign{}
\endhead
\bottomrule\noalign{}
\endlastfoot
0 & 0 & 0 & data[0] \\
0 & 0 & 1 & data[1] \\
0 & 1 & 0 & data[2] \\
0 & 1 & 1 & data[3] \\
1 & 0 & 0 & data[4] \\
1 & 0 & 1 & data[5] \\
1 & 1 & 0 & data[6] \\
1 & 1 & 1 & data[7] \\
\end{longtable}
}

\end{solutionbox}
\begin{mnemonicbox}
``Case-Always: Use case statement in always block''

\end{mnemonicbox}
\begin{center}\rule{0.5\linewidth}{0.5pt}\end{center}

\subsection*{Question 5(c) [7 marks]}\label{q5c}

\textbf{Implement 4 bit full adder using structural modeling style in
Verilog.}

\begin{solutionbox}

\begin{verbatim}
module full\_adder\_4bit(
    input [3:0] a, b,
    input cin,
    output [3:0] sum,
    output cout
);
    wire c1, c2, c3;
    
    full\_adder fa0(.a(a[0]), .b(b[0]), .cin(cin), 
                   .sum(sum[0]), .cout(c1));
    full\_adder fa1(.a(a[1]), .b(b[1]), .cin(c1), 
                   .sum(sum[1]), .cout(c2));
    full\_adder fa2(.a(a[2]), .b(b[2]), .cin(c2), 
                   .sum(sum[2]), .cout(c3));
    full\_adder fa3(.a(a[3]), .b(b[3]), .cin(c3), 
                   .sum(sum[3]), .cout(cout));
endmodule

module full\_adder(
    input a, b, cin,
    output sum, cout
);
    assign sum = a \^{} b \^{} cin;
    assign cout = (a \& b) | (cin \& (a \^{} b));
endmodule
\end{verbatim}

\textbf{Structural Features:}

\begin{itemize}
\tightlist
\item
  \textbf{Module instantiation}: Four 1-bit full adders
\item
  \textbf{Carry chain}: Connects carries between stages
\item
  \textbf{Hierarchical design}: Reuses basic full adder module
\end{itemize}


{\def\LTcaptype{none} % do not increment counter
\vspace{-5pt}
\captionof{table}{Ripple Carry Addition}
\vspace{-10pt}
\begin{longtable}[]{@{}lllll@{}}
\toprule\noalign{}
Stage & Inputs & Carry In & Sum & Carry Out \\
\midrule\noalign{}
\endhead
\bottomrule\noalign{}
\endlastfoot
\textbf{FA0} & A[0], B[0] & Cin & S[0] & C1 \\
\textbf{FA1} & A[1], B[1] & C1 & S[1] & C2 \\
\textbf{FA2} & A[2], B[2] & C2 & S[2] & C3 \\
\textbf{FA3} & A[3], B[3] & C3 & S[3] & Cout \\
\end{longtable}
}

\end{solutionbox}
\begin{mnemonicbox}
``RCC - Ripple Carry Chain connection''

\end{mnemonicbox}
\begin{center}\rule{0.5\linewidth}{0.5pt}\end{center}

\subsection*{Question 5(a OR) [3
marks]}\label{question-5a-or-3-marks}

\textbf{Implement logic function Y = ((AF(D + E) )+ (B+ C))' using
CMOS.}

\begin{solutionbox}

\begin{verbatim}
    VDD
     |
A────┼────┐
F────┼────┤
     │    │ Series pMOS
D────┼────┤ Parallel
E────┼────┘
     │
B────┼────┐
C────┼────┘ Parallel pMOS
     │
     │ Y
A────┼────┐
F────┼────┤
     │    │ Series nMOS
D────┼────┤ Parallel
E────┼────┘
     │
B────┼────┐
C────┼────┘ Parallel nMOS
     │
    GND
\end{verbatim}

\textbf{Logic Breakdown:}

\begin{itemize}
\tightlist
\item
  \textbf{Inner term}: AF(D + E) = A AND F AND (D OR E)
\item
  \textbf{Outer term}: (B + C) = B OR C
\item
  \textbf{Final}: Y = (AF(D + E) + (B + C))'
\end{itemize}

\textbf{CMOS Implementation:}

\begin{itemize}
\tightlist
\item
  \textbf{PMOS network}: Implements complement of function
\item
  \textbf{NMOS network}: Implements original function
\item
  \textbf{Result}: Natural inversion provides Y
\end{itemize}

\end{solutionbox}
\begin{mnemonicbox}
``PNAI - PMOS Network Applies Inversion''

\end{mnemonicbox}
\begin{center}\rule{0.5\linewidth}{0.5pt}\end{center}

\subsection*{Question 5(b OR) [4
marks]}\label{question-5b-or-4-marks}

\textbf{Implement 4 bit up counter using Verilog}

\begin{solutionbox}

\begin{verbatim}
module counter\_4bit\_up(
    input clk, reset,
    output reg [3:0] count
);
    always @(posedge clk or posedge reset) begin
        if (reset)
            count {=} 4{b0000};
        else
            count {=} count + 1;
    end
endmodule
\end{verbatim}


{\def\LTcaptype{none} % do not increment counter
\vspace{-5pt}
\captionof{table}{Counter Sequence}
\vspace{-10pt}
\begin{longtable}[]{@{}llll@{}}
\toprule\noalign{}
Clock & Reset & Count & Next Count \\
\midrule\noalign{}
\endhead
\bottomrule\noalign{}
\endlastfoot
↑ & 1 & X & 0000 \\
↑ & 0 & 0000 & 0001 \\
↑ & 0 & 0001 & 0010 \\
↑ & 0 & \ldots{} & \ldots{} \\
↑ & 0 & 1111 & 0000 \\
\end{longtable}
}

\textbf{Features:}

\begin{itemize}
\tightlist
\item
  \textbf{Synchronous reset}: Reset on clock edge
\item
  \textbf{Auto rollover}: 1111 \rightarrow 0000
\item
  \textbf{4-bit range}: Counts 0 to 15
\end{itemize}

\end{solutionbox}
\begin{mnemonicbox}
``SRA - Synchronous Reset with Auto rollover''

\end{mnemonicbox}
\begin{center}\rule{0.5\linewidth}{0.5pt}\end{center}

\subsection*{Question 5(c OR) [7
marks]}\label{question-5c-or-7-marks}

\textbf{Implement 3:8 decoder using behavioral modeling style in
Verilog.}

\begin{solutionbox}

\begin{verbatim}
module decoder\_3to8(
    input [2:0] select,
    input enable,
    output reg [7:0] out
);
    always @(*) begin
        if (enable) begin
            case(select)
                3{b000}: out = 8{b00000001};
                3{b001}: out = 8{b00000010};
                3{b010}: out = 8{b00000100};
                3{b011}: out = 8{b00001000};
                3{b100}: out = 8{b00010000};
                3{b101}: out = 8{b00100000};
                3{b110}: out = 8{b01000000};
                3{b111}: out = 8{b10000000};
                default: out = 8{b00000000};
            endcase
        end else begin
            out = 8{b00000000};
        end
    end
endmodule
\end{verbatim}


{\def\LTcaptype{none} % do not increment counter
\vspace{-5pt}
\captionof{table}{3:8 Decoder Truth Table}
\vspace{-10pt}
\begin{longtable}[]{@{}llllllllllll@{}}
\toprule\noalign{}
Enable & A2 & A1 & A0 & Y7 & Y6 & Y5 & Y4 & Y3 & Y2 & Y1 & Y0 \\
\midrule\noalign{}
\endhead
\bottomrule\noalign{}
\endlastfoot
0 & X & X & X & 0 & 0 & 0 & 0 & 0 & 0 & 0 & 0 \\
1 & 0 & 0 & 0 & 0 & 0 & 0 & 0 & 0 & 0 & 0 & 1 \\
1 & 0 & 0 & 1 & 0 & 0 & 0 & 0 & 0 & 0 & 1 & 0 \\
1 & 0 & 1 & 0 & 0 & 0 & 0 & 0 & 0 & 1 & 0 & 0 \\
1 & 0 & 1 & 1 & 0 & 0 & 0 & 0 & 1 & 0 & 0 & 0 \\
1 & 1 & 0 & 0 & 0 & 0 & 0 & 1 & 0 & 0 & 0 & 0 \\
1 & 1 & 0 & 1 & 0 & 0 & 1 & 0 & 0 & 0 & 0 & 0 \\
1 & 1 & 1 & 0 & 0 & 1 & 0 & 0 & 0 & 0 & 0 & 0 \\
1 & 1 & 1 & 1 & 1 & 0 & 0 & 0 & 0 & 0 & 0 & 0 \\
\end{longtable}
}

\textbf{Key Features:}

\begin{itemize}
\tightlist
\item
  \textbf{Behavioral modeling}: Uses always block and case statement
\item
  \textbf{Enable control}: All outputs disabled when enable = 0
\item
  \textbf{One-hot output}: Only one output active at a time
\item
  \textbf{3-bit input}: Selects one of 8 outputs
\end{itemize}

\textbf{Applications:}

\begin{itemize}
\tightlist
\item
  \textbf{Memory addressing}: Chip select generation
\item
  \textbf{Data routing}: Channel selection
\item
  \textbf{Control logic}: State machine outputs
\end{itemize}

\end{solutionbox}
\begin{mnemonicbox}
``BEOH - Behavioral Enable One-Hot decoder''

\end{mnemonicbox}

\end{document}
