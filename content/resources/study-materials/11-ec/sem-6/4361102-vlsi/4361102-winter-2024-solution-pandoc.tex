\documentclass[10pt,a4paper]{article}

% content/resources/templates/preamble.tex
\usepackage[margin=0.6in]{geometry}
\author{Milav Dabgar}
\usepackage{amsmath,amssymb,amsthm}
\usepackage{booktabs}
\usepackage{multirow}
\usepackage{xcolor}
\usepackage{tcolorbox}
\tcbuselibrary{breakable,skins}
\usepackage[colorlinks=true,linkcolor=blue]{hyperref}
\usepackage{titlesec}
\usepackage{enumitem}
\usepackage{tikz}
\usepackage{pgfplots}
\usepackage{circuitikz}
\usepackage[version=4]{mhchem}
\usepackage{longtable}
\usepackage{array}
\usepackage{float}
\usepackage{caption}
\usepackage{listings}

\lstset{
  basicstyle=\small\ttfamily,
  breaklines=true,
  breakatwhitespace=false,
  postbreak=\mbox{\textcolor{red}{$\hookrightarrow$}\space},
  float=false,
  numbers=left,
  numberstyle=\tiny\color{gray},
  numbersep=10pt,
  xleftmargin=2em,
  keywordstyle=\color{blue},
  commentstyle=\color{green!60!black},
  stringstyle=\color{purple},
  backgroundcolor=\color{gray!5},
  showstringspaces=false,
  tabsize=2,
  captionpos=b,
  keepspaces=true,
  columns=flexible
}

\pgfplotsset{compat=1.18}
\usetikzlibrary{shapes,arrows,positioning,calc,patterns,decorations.pathmorphing,decorations.markings,arrows.meta}

% Color scheme
\definecolor{headcolor}{RGB}{0,102,204}
\definecolor{keycolor}{RGB}{220,20,60}
\definecolor{solutioncolor}{RGB}{34,139,34}
\definecolor{mnemoniccolor}{RGB}{148,0,211}
\definecolor{codecolor}{RGB}{0,0,100}

% Spacing
\setlength{\parskip}{3pt}
\setlist[itemize]{nosep}
\setlist[enumerate]{nosep}

% Title formatting
\titleformat{\section}{\Large\bfseries\color{headcolor}}{\thesection}{1em}{}
\titleformat{\subsection}{\large\bfseries\color{headcolor}}{\thesubsection}{1em}{}

% Pandoc tightlist compatibility
\providecommand{\tightlist}{%
  \setlength{\itemsep}{0pt}\setlength{\parskip}{0pt}}

% Pandoc longtable compatibility
\newcounter{none}
\def\thenone{}


% content/resources/templates/english-boxes.tex
% This file is currently empty - it exists to maintain consistency with the import structure.
% Add custom environments here if needed in the future.


\begin{document}

\begin{center}
{\Huge\bfseries\color{headcolor} Subject Name Solutions}\\[5pt]
{\LARGE 4361102 -- Winter 2024}\\[3pt]
{\large Semester 1 Study Material}\\[3pt]
{\normalsize\textit{Detailed Solutions and Explanations}}
\end{center}

\vspace{10pt}

\subsection*{Question 1(a) [3 marks]}\label{q1a}

\textbf{Write advantages of High K FINFET.}

\begin{solutionbox}

{\def\LTcaptype{none} % do not increment counter
\begin{longtable}[]{@{}
  >{\raggedright\arraybackslash}p{(\linewidth - 2\tabcolsep) * \real{0.4688}}
  >{\raggedright\arraybackslash}p{(\linewidth - 2\tabcolsep) * \real{0.5312}}@{}}
\toprule\noalign{}
\begin{minipage}[b]{\linewidth}\raggedright
\textbf{Advantage}
\end{minipage} & \begin{minipage}[b]{\linewidth}\raggedright
\textbf{Description}
\end{minipage} \\
\midrule\noalign{}
\endhead
\bottomrule\noalign{}
\endlastfoot
\textbf{Reduced leakage current} & Better gate control reduces power
consumption \\
\textbf{Improved performance} & Higher drive current and faster
switching \\
\textbf{Better scalability} & Enables continued Moore's law scaling \\
\end{longtable}
}

\begin{itemize}
\tightlist
\item
  \textbf{High K dielectric}: Reduces gate leakage significantly
\item
  \textbf{3D structure}: Better electrostatic control over channel
\item
  \textbf{Lower power}: Reduced static and dynamic power consumption
\end{itemize}

\end{solutionbox}
\begin{mnemonicbox}
``High Performance, Low Power, Better Control''

\end{mnemonicbox}
\begin{center}\rule{0.5\linewidth}{0.5pt}\end{center}

\subsection*{Question 1(b) [4 marks]}\label{q1b}

\textbf{Define terms: (1) pinch off point (2) Threshold Voltage.}

\begin{solutionbox}


{\def\LTcaptype{none} % do not increment counter
\vspace{-5pt}
\captionof{table}{Key MOSFET Parameters}
\vspace{-10pt}
\begin{longtable}[]{@{}
  >{\raggedright\arraybackslash}p{(\linewidth - 4\tabcolsep) * \real{0.2273}}
  >{\raggedright\arraybackslash}p{(\linewidth - 4\tabcolsep) * \real{0.3636}}
  >{\raggedright\arraybackslash}p{(\linewidth - 4\tabcolsep) * \real{0.4091}}@{}}
\toprule\noalign{}
\begin{minipage}[b]{\linewidth}\raggedright
\textbf{Term}
\end{minipage} & \begin{minipage}[b]{\linewidth}\raggedright
\textbf{Definition}
\end{minipage} & \begin{minipage}[b]{\linewidth}\raggedright
\textbf{Significance}
\end{minipage} \\
\midrule\noalign{}
\endhead
\bottomrule\noalign{}
\endlastfoot
\textbf{Pinch-off Point} & Point where channel becomes completely
depleted & Marks transition to saturation region \\
\textbf{Threshold Voltage} & Minimum VGS needed to form conducting
channel & Determines ON/OFF switching point \\
\end{longtable}
}

\begin{itemize}
\tightlist
\item
  \textbf{Pinch-off point}: VDS = VGS - VT, channel narrows to zero
  width
\item
  \textbf{Threshold voltage}: Typically 0.7V for enhancement MOSFET
\item
  \textbf{Critical parameters}: Both determine MOSFET operating regions
\end{itemize}

\end{solutionbox}
\begin{mnemonicbox}
``Threshold Turns ON, Pinch-off Points to
Saturation''

\end{mnemonicbox}
\begin{center}\rule{0.5\linewidth}{0.5pt}\end{center}

\subsection*{Question 1(c) [7 marks]}\label{q1c}

\textbf{Draw and explain structure of MOSFET transistor.}

\begin{solutionbox}

\textbf{Diagram:}

\begin{verbatim}
    Gate (G)
      |
   ┌──┴──┐
   │ SiO2│    
┌──┴─────┴──┐
│  n+   n+  │  Source (S) and Drain (D)
│     p     │  P{-substrate  }
└───────────┘
    Body (B)
\end{verbatim}

\textbf{Structure Components Table:}

{\def\LTcaptype{none} % do not increment counter
\begin{longtable}[]{@{}lll@{}}
\toprule\noalign{}
\textbf{Component} & \textbf{Material} & \textbf{Function} \\
\midrule\noalign{}
\endhead
\bottomrule\noalign{}
\endlastfoot
\textbf{Gate} & Polysilicon/Metal & Controls channel formation \\
\textbf{Gate oxide} & SiO2 & Insulates gate from substrate \\
\textbf{Source/Drain} & n+ doped silicon & Current entry/exit points \\
\textbf{Substrate} & p-type silicon & Provides body connection \\
\end{longtable}
}

\begin{itemize}
\tightlist
\item
  \textbf{Channel formation}: Occurs at oxide-semiconductor interface
\item
  \textbf{Enhancement mode}: Channel forms when VGS \textgreater{} VT
\item
  \textbf{Four-terminal device}: Gate, Source, Drain, Body connections
\end{itemize}

\end{solutionbox}
\begin{mnemonicbox}
``Gate Controls, Oxide Isolates, Source-Drain
Conducts''

\end{mnemonicbox}
\begin{center}\rule{0.5\linewidth}{0.5pt}\end{center}

\subsection*{Question 1(c OR) [7
marks]}\label{question-1c-or-7-marks}

\textbf{Compare Full Voltage Scaling and Constant Voltage Scaling.}

\begin{solutionbox}

\textbf{Comparison Table:}

{\def\LTcaptype{none} % do not increment counter
\begin{longtable}[]{@{}
  >{\raggedright\arraybackslash}p{(\linewidth - 4\tabcolsep) * \real{0.2143}}
  >{\raggedright\arraybackslash}p{(\linewidth - 4\tabcolsep) * \real{0.3571}}
  >{\raggedright\arraybackslash}p{(\linewidth - 4\tabcolsep) * \real{0.4286}}@{}}
\toprule\noalign{}
\begin{minipage}[b]{\linewidth}\raggedright
\textbf{Parameter}
\end{minipage} & \begin{minipage}[b]{\linewidth}\raggedright
\textbf{Full Voltage Scaling}
\end{minipage} & \begin{minipage}[b]{\linewidth}\raggedright
\textbf{Constant Voltage Scaling}
\end{minipage} \\
\midrule\noalign{}
\endhead
\bottomrule\noalign{}
\endlastfoot
\textbf{Supply voltage} & Scaled down by α & Remains constant \\
\textbf{Gate oxide thickness} & Scaled down by α & Scaled down by α \\
\textbf{Channel length} & Scaled down by α & Scaled down by α \\
\textbf{Power density} & Remains constant & Increases by α^{2} \\
\textbf{Performance} & Moderate improvement & Better performance \\
\textbf{Reliability} & Better & Degraded due to high fields \\
\end{longtable}
}

\begin{itemize}
\tightlist
\item
  \textbf{Full scaling}: All dimensions and voltages scaled
  proportionally
\item
  \textbf{Constant voltage}: Only physical dimensions scaled, voltage
  unchanged
\item
  \textbf{Trade-off}: Performance vs power vs reliability
\end{itemize}

\end{solutionbox}
\begin{mnemonicbox}
``Full scales All, Constant keeps Voltage''

\end{mnemonicbox}
\begin{center}\rule{0.5\linewidth}{0.5pt}\end{center}

\subsection*{Question 2(a) [3 marks]}\label{q2a}

\textbf{Draw Resistive Load Inverter. Write the input voltage range for
different operating region of operation.}

\begin{solutionbox}

\textbf{Circuit Diagram:}

\begin{verbatim}
VDD ──┬── RL
      │
      ├── Vout
      │
Vin ──┤ M1 (NMOS)
      │
     GND
\end{verbatim}

\textbf{Operating Regions Table:}

{\def\LTcaptype{none} % do not increment counter
\begin{longtable}[]{@{}lll@{}}
\toprule\noalign{}
\textbf{Region} & \textbf{Input Voltage Range} & \textbf{Output
State} \\
\midrule\noalign{}
\endhead
\bottomrule\noalign{}
\endlastfoot
\textbf{Cut-off} & Vin \textless{} VT & Vout = VDD \\
\textbf{Triode} & VT \textless{} Vin \textless{} VDD-VT & Transition \\
\textbf{Saturation} & Vin \textgreater{} VDD-VT & Vout \approx 0V \\
\end{longtable}
}

\end{solutionbox}
\begin{mnemonicbox}
``Cut-off High, Triode Transition, Saturation Low''

\end{mnemonicbox}
\begin{center}\rule{0.5\linewidth}{0.5pt}\end{center}

\subsection*{Question 2(b) [4 marks]}\label{q2b}

\textbf{Draw and Explain VDS-ID and VGS-ID characteristics of N channel
MOSFET.}

\begin{solutionbox}

\textbf{VDS-ID Characteristics:}

\begin{verbatim}
ID ↑
   │    VGS3
   │   ╱ VGS2
   │  ╱  VGS1
   │ ╱   (VGS3{VGS2VGS1VT)}
   │╱
   └──────── VDS
   Triode  Saturation
\end{verbatim}

\textbf{Characteristics Table:}

{\def\LTcaptype{none} % do not increment counter
\begin{longtable}[]{@{}lll@{}}
\toprule\noalign{}
\textbf{Characteristic} & \textbf{Region} & \textbf{Behavior} \\
\midrule\noalign{}
\endhead
\bottomrule\noalign{}
\endlastfoot
\textbf{VDS-ID} & Triode & Linear increase with VDS \\
\textbf{VDS-ID} & Saturation & Constant ID (square law) \\
\textbf{VGS-ID} & Sub-threshold & Exponential increase \\
\textbf{VGS-ID} & Above VT & Square law relationship \\
\end{longtable}
}

\begin{itemize}
\tightlist
\item
  \textbf{Triode region}: ID increases linearly with VDS
\item
  \textbf{Saturation}: ID independent of VDS, depends on VGS
\item
  \textbf{Square law}: ID ∝ (VGS-VT)^{2} in saturation
\end{itemize}

\end{solutionbox}
\begin{mnemonicbox}
``Linear in Triode, Square in Saturation''

\end{mnemonicbox}
\begin{center}\rule{0.5\linewidth}{0.5pt}\end{center}

\subsection*{Question 2(c) [7 marks]}\label{q2c}

\textbf{Draw \& Explain working of Depletion Load NMOS Inverter
circuit.}

\begin{solutionbox}

\textbf{Circuit Diagram:}

\begin{verbatim}
VDD ──┬─── ML (Depletion)
      │    Gate connected to Source
      ├─── Vout
      │
Vin ──┤    M1 (Enhancement)
      │
     GND
\end{verbatim}

\textbf{Operation Table:}

{\def\LTcaptype{none} % do not increment counter
\begin{longtable}[]{@{}llll@{}}
\toprule\noalign{}
\textbf{Input} & \textbf{M1 State} & \textbf{ML State} &
\textbf{Output} \\
\midrule\noalign{}
\endhead
\bottomrule\noalign{}
\endlastfoot
\textbf{Low (0V)} & Cut-off & Active load & High (VDD) \\
\textbf{High (VDD)} & Saturated & Linear & Low \\
\end{longtable}
}

\begin{itemize}
\tightlist
\item
  \textbf{Depletion load}: Always conducting, acts as current source
\item
  \textbf{Better performance}: Higher output voltage swing than
  resistive load
\item
  \textbf{Gate connection}: ML gate tied to source for depletion
  operation
\item
  \textbf{Improved noise margin}: Better VOH compared to enhancement
  load
\end{itemize}

\end{solutionbox}
\begin{mnemonicbox}
``Depletion Always ON, Enhancement Controls Flow''

\end{mnemonicbox}
\begin{center}\rule{0.5\linewidth}{0.5pt}\end{center}

\subsection*{Question 2(a OR) [3
marks]}\label{question-2a-or-3-marks}

\textbf{Describe advantages of CMOS Inverter.}

\begin{solutionbox}

\textbf{Advantages Table:}

{\def\LTcaptype{none} % do not increment counter
\begin{longtable}[]{@{}ll@{}}
\toprule\noalign{}
\textbf{Advantage} & \textbf{Benefit} \\
\midrule\noalign{}
\endhead
\bottomrule\noalign{}
\endlastfoot
\textbf{Zero static power} & No current in steady state \\
\textbf{Full voltage swing} & Output swings from 0V to VDD \\
\textbf{High noise margins} & Better noise immunity \\
\end{longtable}
}

\begin{itemize}
\tightlist
\item
  \textbf{Complementary operation}: One transistor always OFF
\item
  \textbf{High input impedance}: Gate isolation provides high impedance
\item
  \textbf{Fast switching}: Low parasitic capacitances
\end{itemize}

\end{solutionbox}
\begin{mnemonicbox}
``Zero Power, Full Swing, High Immunity''

\end{mnemonicbox}
\begin{center}\rule{0.5\linewidth}{0.5pt}\end{center}

\subsection*{Question 2(b OR) [4
marks]}\label{question-2b-or-4-marks}

\textbf{Draw and Explain Noise Margin in detail.}

\begin{solutionbox}

\textbf{Voltage Transfer Characteristics:}

\begin{verbatim}
Vout ↑
VDD  ┌─────┐
     │     │
     │     │  NMH
VOH  ┤     └─────
     │           ╲
     │            ╲
     │             ╲
VOL  ┤              └─────
     │                   
  0V └───────────────── Vin
    0V  VIL  VIH      VDD
       
\end{verbatim}

\textbf{Noise Margin Parameters:}

{\def\LTcaptype{none} % do not increment counter
\begin{longtable}[]{@{}lll@{}}
\toprule\noalign{}
\textbf{Parameter} & \textbf{Formula} & \textbf{Typical Value} \\
\midrule\noalign{}
\endhead
\bottomrule\noalign{}
\endlastfoot
\textbf{NMH} & VOH - VIH & 40\% of VDD \\
\textbf{NML} & VIL - VOL & 40\% of VDD \\
\end{longtable}
}

\begin{itemize}
\tightlist
\item
  \textbf{High noise margin}: Immunity to positive noise
\item
  \textbf{Low noise margin}: Immunity to negative noise
\item
  \textbf{Better CMOS}: Higher noise margins than other logic families
\end{itemize}

\end{solutionbox}
\begin{mnemonicbox}
``High goes Higher, Low goes Lower''

\end{mnemonicbox}
\begin{center}\rule{0.5\linewidth}{0.5pt}\end{center}

\subsection*{Question 2(c OR) [7
marks]}\label{question-2c-or-7-marks}

\textbf{Draw and Explain VTC of N MOS Inverter.}

\begin{solutionbox}

\textbf{Voltage Transfer Characteristics:}

\begin{verbatim}
Vout ↑
VDD  ┌─┐
     │ │
     │ │  Region I
     │ └─┐
     │   │  Region II  
     │   │
     │   └─┐
     │     │  Region III
     │     └──── Vin
  0V └─────────────
    0V VT        VDD
\end{verbatim}

\textbf{Operating Regions Table:}

{\def\LTcaptype{none} % do not increment counter
\begin{longtable}[]{@{}llll@{}}
\toprule\noalign{}
\textbf{Region} & \textbf{Vin Range} & \textbf{M1 State} &
\textbf{Vout} \\
\midrule\noalign{}
\endhead
\bottomrule\noalign{}
\endlastfoot
\textbf{I} & 0 to VT & Cut-off & VDD \\
\textbf{II} & VT to VT+VTL & Saturation & Decreasing \\
\textbf{III} & VT+VTL to VDD & Triode & Low \\
\end{longtable}
}

\begin{itemize}
\tightlist
\item
  \textbf{Region I}: M1 OFF, no current flow, Vout = VDD
\item
  \textbf{Region II}: M1 in saturation, sharp transition
\item
  \textbf{Region III}: M1 in triode, gradual decrease
\item
  \textbf{Load line}: Determines operating point intersection
\end{itemize}

\end{solutionbox}
\begin{mnemonicbox}
``Cut-off High, Saturation Sharp, Triode Low''

\end{mnemonicbox}
\begin{center}\rule{0.5\linewidth}{0.5pt}\end{center}

\subsection*{Question 3(a) [3 marks]}\label{q3a}

\textbf{Draw and explain generalized multiple input NOR gate structure
with Depletion NMOS load.}

\begin{solutionbox}

\textbf{Circuit Diagram:}

\begin{verbatim}
VDD ──┬─── ML (Depletion Load)
      │
      ├─── Y = (A+B+C){}
      │
A  ───┤ M1
      │
B  ───┤ M2   Parallel Connection
      │
C  ───┤ M3
      │
     GND
\end{verbatim}

\textbf{Truth Table:}

{\def\LTcaptype{none} % do not increment counter
\begin{longtable}[]{@{}lll@{}}
\toprule\noalign{}
\textbf{Inputs} & \textbf{Any Input High?} & \textbf{Output Y} \\
\midrule\noalign{}
\endhead
\bottomrule\noalign{}
\endlastfoot
\textbf{All Low} & No & High (1) \\
\textbf{Any High} & Yes & Low (0) \\
\end{longtable}
}

\begin{itemize}
\tightlist
\item
  \textbf{Parallel NMOS}: Any input HIGH pulls output LOW
\item
  \textbf{NOR operation}: Y = (A+B+C)'
\item
  \textbf{Depletion load}: Provides pull-up current
\end{itemize}

\end{solutionbox}
\begin{mnemonicbox}
``Parallel Pulls Down, Depletion Pulls Up''

\end{mnemonicbox}
\begin{center}\rule{0.5\linewidth}{0.5pt}\end{center}

\subsection*{Question 3(b) [4 marks]}\label{q3b}

\textbf{Differentiate AOI and OAI logic circuits.}

\begin{solutionbox}

\textbf{Comparison Table:}

{\def\LTcaptype{none} % do not increment counter
\begin{longtable}[]{@{}lll@{}}
\toprule\noalign{}
\textbf{Parameter} & \textbf{AOI (AND-OR-Invert)} & \textbf{OAI
(OR-AND-Invert)} \\
\midrule\noalign{}
\endhead
\bottomrule\noalign{}
\endlastfoot
\textbf{Logic function} & Y = (AB + CD)' & Y = ((A+B)(C+D))' \\
\textbf{NMOS structure} & Series-parallel & Parallel-series \\
\textbf{PMOS structure} & Parallel-series & Series-parallel \\
\textbf{Complexity} & Moderate & Moderate \\
\end{longtable}
}

\begin{itemize}
\tightlist
\item
  \textbf{AOI}: AND terms ORed then inverted
\item
  \textbf{OAI}: OR terms ANDed then inverted\\
\item
  \textbf{CMOS implementation}: Dual network structure
\item
  \textbf{Applications}: Complex logic functions in single stage
\end{itemize}

\end{solutionbox}
\begin{mnemonicbox}
``AOI: AND-OR-Invert, OAI: OR-AND-Invert''

\end{mnemonicbox}
\begin{center}\rule{0.5\linewidth}{0.5pt}\end{center}

\subsection*{Question 3(c) [7 marks]}\label{q3c}

\textbf{Implement two input EX-OR gate using CMOS, and logic function Z
= (AB +CD)' using NMOS Load.}

\begin{solutionbox}

\textbf{EX-OR CMOS Implementation:}

\begin{verbatim}
VDD ─┬─ pMOS network
     │  (A{B + AB)}
     ├─ Y = A  
     │
     ├─ nMOS network
     │  (AB + A{B)}
    GND
\end{verbatim}

\textbf{Z = (AB + CD)' NMOS Implementation:}

\begin{verbatim}
VDD ─┬─ Resistive Load
     │
     ├─ Z = (AB + CD){}
     │
A ─┬─┤ M1 ── B ─┤ M2  (Series: AB)
   │ │           │
C ─┤ M3 ── D ─┤ M4     (Series: CD)
   │           │
  GND ────────┴─      (Parallel connection)
\end{verbatim}

\textbf{Logic Implementation Table:}

{\def\LTcaptype{none} % do not increment counter
\begin{longtable}[]{@{}lll@{}}
\toprule\noalign{}
\textbf{Circuit} & \textbf{Function} & \textbf{Implementation} \\
\midrule\noalign{}
\endhead
\bottomrule\noalign{}
\endlastfoot
\textbf{EX-OR} & A\oplusB & Complementary CMOS \\
\textbf{AOI} & (AB+CD)' & Series-parallel NMOS \\
\end{longtable}
}

\begin{itemize}
\tightlist
\item
  \textbf{EX-OR}: Requires transmission gates for efficient
  implementation
\item
  \textbf{AOI function}: Natural NMOS implementation
\item
  \textbf{Power consideration}: CMOS has zero static power
\end{itemize}

\end{solutionbox}
\begin{mnemonicbox}
``EX-OR needs Transmission, AOI uses
Series-Parallel''

\end{mnemonicbox}
\begin{center}\rule{0.5\linewidth}{0.5pt}\end{center}

\subsection*{Question 3(a OR) [3
marks]}\label{question-3a-or-3-marks}

\textbf{Draw and explain generalized multiple input NAND gate structure
with Depletion NMOS load.}

\begin{solutionbox}

\textbf{Circuit Diagram:}

\begin{verbatim}
VDD ──┬─── ML (Depletion Load)
      │
      ├─── Y = (ABC){}
      │
A  ───┤ M1
      │
B  ───┤ M2   Series Connection  
      │
C  ───┤ M3
      │
     GND
\end{verbatim}

\textbf{Operation Table:}

{\def\LTcaptype{none} % do not increment counter
\begin{longtable}[]{@{}lll@{}}
\toprule\noalign{}
\textbf{Condition} & \textbf{Path to Ground} & \textbf{Output Y} \\
\midrule\noalign{}
\endhead
\bottomrule\noalign{}
\endlastfoot
\textbf{All inputs HIGH} & Complete path & Low (0) \\
\textbf{Any input LOW} & Broken path & High (1) \\
\end{longtable}
}

\begin{itemize}
\tightlist
\item
  \textbf{Series NMOS}: All inputs must be HIGH to pull output LOW
\item
  \textbf{NAND operation}: Y = (ABC)'
\item
  \textbf{Depletion load}: Always provides pull-up current
\end{itemize}

\end{solutionbox}
\begin{mnemonicbox}
``Series Needs All, NAND Says Not-AND''

\end{mnemonicbox}
\begin{center}\rule{0.5\linewidth}{0.5pt}\end{center}

\subsection*{Question 3(b OR) [4
marks]}\label{question-3b-or-4-marks}

\textbf{Implement logic function Y = ((P+R)(S+T))' using CMOS logic.}

\begin{solutionbox}

\textbf{CMOS Implementation:}

\begin{verbatim}
VDD ─┬─ pMOS Network
     │  P─┤├─R in series with S─┤├─T in series
     ├─ Y = ((P+R)(S+T)){}
     │
     ├─ nMOS Network  
     │  (P,R parallel) in series with (S,T parallel)
    GND
\end{verbatim}

\textbf{Truth Table Implementation:}

{\def\LTcaptype{none} % do not increment counter
\begin{longtable}[]{@{}lll@{}}
\toprule\noalign{}
\textbf{PMOS Network} & \textbf{NMOS Network} & \textbf{Operation} \\
\midrule\noalign{}
\endhead
\bottomrule\noalign{}
\endlastfoot
\textbf{(P+R)`+(S+T)'} & \textbf{(P+R)(S+T)} & Complementary \\
\textbf{P'R' + S'T'} & \textbf{PS + PT + RS + RT} & De Morgan's law \\
\end{longtable}
}

\begin{itemize}
\tightlist
\item
  \textbf{PMOS}: Parallel within groups, series between groups
\item
  \textbf{NMOS}: Series within groups, parallel between groups
\item
  \textbf{Dual network}: Ensures complementary operation
\end{itemize}

\end{solutionbox}
\begin{mnemonicbox}
``PMOS does Opposite of NMOS''

\end{mnemonicbox}
\begin{center}\rule{0.5\linewidth}{0.5pt}\end{center}

\subsection*{Question 3(c OR) [7
marks]}\label{question-3c-or-7-marks}

\textbf{Describe the working of SR latch circuit.}

\begin{solutionbox}

\textbf{SR Latch Circuit:}

\begin{verbatim}
S ─┤ NOR   ├─┬─ Q
   │  G1   │ │
   └───────┤ │
           │ │
   ┌───────┤ │
   │ NOR   │ │
R ─┤  G2   ├─┴─ Q{}
   └───────┘
\end{verbatim}

\textbf{Truth Table:}

{\def\LTcaptype{none} % do not increment counter
\begin{longtable}[]{@{}lllll@{}}
\toprule\noalign{}
\textbf{S} & \textbf{R} & \textbf{Q(n+1)} & \textbf{Q'(n+1)} &
\textbf{State} \\
\midrule\noalign{}
\endhead
\bottomrule\noalign{}
\endlastfoot
\textbf{0} & \textbf{0} & Q(n) & Q'(n) & Hold \\
\textbf{0} & \textbf{1} & 0 & 1 & Reset \\
\textbf{1} & \textbf{0} & 1 & 0 & Set \\
\textbf{1} & \textbf{1} & 0 & 0 & Invalid \\
\end{longtable}
}

\begin{itemize}
\tightlist
\item
  \textbf{Set operation}: S=1, R=0 makes Q=1
\item
  \textbf{Reset operation}: S=0, R=1 makes Q=0\\
\item
  \textbf{Hold state}: S=0, R=0 maintains previous state
\item
  \textbf{Invalid state}: S=1, R=1 should be avoided
\item
  \textbf{Cross-coupled}: Output of one gate feeds input of other
\end{itemize}

\end{solutionbox}
\begin{mnemonicbox}
``Set Sets, Reset Resets, Both Bad''

\end{mnemonicbox}
\begin{center}\rule{0.5\linewidth}{0.5pt}\end{center}

\subsection*{Question 4(a) [3 marks]}\label{q4a}

\textbf{Compare Etching methods in chip fabrication.}

\begin{solutionbox}

\textbf{Etching Methods Comparison:}

{\def\LTcaptype{none} % do not increment counter
\begin{longtable}[]{@{}
  >{\raggedright\arraybackslash}p{(\linewidth - 6\tabcolsep) * \real{0.2105}}
  >{\raggedright\arraybackslash}p{(\linewidth - 6\tabcolsep) * \real{0.1754}}
  >{\raggedright\arraybackslash}p{(\linewidth - 6\tabcolsep) * \real{0.2807}}
  >{\raggedright\arraybackslash}p{(\linewidth - 6\tabcolsep) * \real{0.3333}}@{}}
\toprule\noalign{}
\begin{minipage}[b]{\linewidth}\raggedright
\textbf{Method}
\end{minipage} & \begin{minipage}[b]{\linewidth}\raggedright
\textbf{Type}
\end{minipage} & \begin{minipage}[b]{\linewidth}\raggedright
\textbf{Advantages}
\end{minipage} & \begin{minipage}[b]{\linewidth}\raggedright
\textbf{Disadvantages}
\end{minipage} \\
\midrule\noalign{}
\endhead
\bottomrule\noalign{}
\endlastfoot
\textbf{Wet Etching} & Chemical & High selectivity, simple & Isotropic,
undercut \\
\textbf{Dry Etching} & Physical/Chemical & Anisotropic, precise &
Complex equipment \\
\textbf{Plasma Etching} & Ion bombardment & Directional control & Damage
to surface \\
\end{longtable}
}

\begin{itemize}
\tightlist
\item
  \textbf{Wet etching}: Uses liquid chemicals, attacks all directions
\item
  \textbf{Dry etching}: Uses gases/plasma, better directional control\\
\item
  \textbf{Selectivity}: Ability to etch one material over another
\end{itemize}

\end{solutionbox}
\begin{mnemonicbox}
``Wet is Wide, Dry is Directional''

\end{mnemonicbox}
\begin{center}\rule{0.5\linewidth}{0.5pt}\end{center}

\subsection*{Question 4(b) [4 marks]}\label{q4b}

\textbf{Write short note on Lithography.}

\begin{solutionbox}

\textbf{Lithography Process Steps:}

{\def\LTcaptype{none} % do not increment counter
\begin{longtable}[]{@{}lll@{}}
\toprule\noalign{}
\textbf{Step} & \textbf{Process} & \textbf{Purpose} \\
\midrule\noalign{}
\endhead
\bottomrule\noalign{}
\endlastfoot
\textbf{Resist coating} & Spin-on photoresist & Light-sensitive layer \\
\textbf{Exposure} & UV light through mask & Pattern transfer \\
\textbf{Development} & Remove exposed resist & Reveal pattern \\
\textbf{Etching} & Remove unprotected material & Create features \\
\end{longtable}
}

\begin{itemize}
\tightlist
\item
  \textbf{Pattern transfer}: From mask to silicon wafer
\item
  \textbf{Resolution}: Determines minimum feature size
\item
  \textbf{Alignment}: Critical for multiple layer processing
\item
  \textbf{UV wavelength}: Shorter wavelength gives better resolution
\end{itemize}

\end{solutionbox}
\begin{mnemonicbox}
``Coat, Expose, Develop, Etch''

\end{mnemonicbox}
\begin{center}\rule{0.5\linewidth}{0.5pt}\end{center}

\subsection*{Question 4(c) [7 marks]}\label{q4c}

\textbf{Explain Regularity, Modularity and Locality.}

\begin{solutionbox}

\textbf{Design Principles Table:}

{\def\LTcaptype{none} % do not increment counter
\begin{longtable}[]{@{}
  >{\raggedright\arraybackslash}p{(\linewidth - 6\tabcolsep) * \real{0.2586}}
  >{\raggedright\arraybackslash}p{(\linewidth - 6\tabcolsep) * \real{0.2759}}
  >{\raggedright\arraybackslash}p{(\linewidth - 6\tabcolsep) * \real{0.2414}}
  >{\raggedright\arraybackslash}p{(\linewidth - 6\tabcolsep) * \real{0.2241}}@{}}
\toprule\noalign{}
\begin{minipage}[b]{\linewidth}\raggedright
\textbf{Principle}
\end{minipage} & \begin{minipage}[b]{\linewidth}\raggedright
\textbf{Definition}
\end{minipage} & \begin{minipage}[b]{\linewidth}\raggedright
\textbf{Benefits}
\end{minipage} & \begin{minipage}[b]{\linewidth}\raggedright
\textbf{Example}
\end{minipage} \\
\midrule\noalign{}
\endhead
\bottomrule\noalign{}
\endlastfoot
\textbf{Regularity} & Repeated identical structures & Easier design,
testing & Memory arrays \\
\textbf{Modularity} & Hierarchical design blocks & Reusability,
maintainability & Standard cells \\
\textbf{Locality} & Related functions grouped & Reduced interconnect &
Functional blocks \\
\end{longtable}
}

\textbf{Implementation Details:}

\begin{itemize}
\tightlist
\item
  \textbf{Regularity}: Same cell repeated multiple times reduces design
  complexity
\item
  \textbf{Modularity}: Top-down design with well-defined interfaces
\item
  \textbf{Locality}: Minimizes wire delays and routing congestion
\item
  \textbf{Design benefits}: Faster design cycle, better testability
\item
  \textbf{Manufacturing}: Improved yield through regular patterns
\end{itemize}

\textbf{Mnemaid Diagram:}

\begin{center}
\textbf{Mermaid Diagram (Code)}
\begin{verbatim}
{Shaded}
{Highlighting}[]
graph LR
    A[System Level] {-{-}{} B[Module Level]}
    B {-{-}{} C[Cell Level]}
    C {-{-}{} D[Device Level]}
    D {-{-}{} E[Regular Structures]}
{Highlighting}
{Shaded}
\end{verbatim}
\end{center}

\end{solutionbox}
\begin{mnemonicbox}
``Regular Modules with Local Connections''

\end{mnemonicbox}
\begin{center}\rule{0.5\linewidth}{0.5pt}\end{center}

\subsection*{Question 4(a OR) [3
marks]}\label{question-4a-or-3-marks}

\textbf{Define Design Hierarchy.}

\begin{solutionbox}

\textbf{Design Hierarchy Levels:}

{\def\LTcaptype{none} % do not increment counter
\begin{longtable}[]{@{}lll@{}}
\toprule\noalign{}
\textbf{Level} & \textbf{Description} & \textbf{Components} \\
\midrule\noalign{}
\endhead
\bottomrule\noalign{}
\endlastfoot
\textbf{System} & Complete chip functionality & Processors, memories \\
\textbf{Module} & Major functional blocks & ALU, cache, I/O \\
\textbf{Cell} & Basic logic elements & Gates, flip-flops \\
\end{longtable}
}

\begin{itemize}
\tightlist
\item
  \textbf{Top-down approach}: System broken into smaller modules
\item
  \textbf{Abstraction levels}: Each level hides lower level details
\item
  \textbf{Interface definition}: Clear boundaries between levels
\end{itemize}

\end{solutionbox}
\begin{mnemonicbox}
``System to Module to Cell''

\end{mnemonicbox}
\begin{center}\rule{0.5\linewidth}{0.5pt}\end{center}

\subsection*{Question 4(b OR) [4
marks]}\label{question-4b-or-4-marks}

\textbf{Draw and Explain VLSI design flow chart.}

\begin{solutionbox}

\textbf{VLSI Design Flow:}

\begin{center}
\textbf{Mermaid Diagram (Code)}
\begin{verbatim}
{Shaded}
{Highlighting}[]
graph LR
    A[System Specification] {-{-}{} B[Architectural Design]}
    B {-{-}{} C[Logic Design]}
    C {-{-}{} D[Circuit Design]}
    D {-{-}{} E[Layout Design]}
    E {-{-}{} F[Fabrication]}
    F {-{-}{} G[Testing]}
{Highlighting}
{Shaded}
\end{verbatim}
\end{center}

\textbf{Design Flow Table:}

{\def\LTcaptype{none} % do not increment counter
\begin{longtable}[]{@{}
  >{\raggedright\arraybackslash}p{(\linewidth - 6\tabcolsep) * \real{0.2444}}
  >{\raggedright\arraybackslash}p{(\linewidth - 6\tabcolsep) * \real{0.2444}}
  >{\raggedright\arraybackslash}p{(\linewidth - 6\tabcolsep) * \real{0.2667}}
  >{\raggedright\arraybackslash}p{(\linewidth - 6\tabcolsep) * \real{0.2444}}@{}}
\toprule\noalign{}
\begin{minipage}[b]{\linewidth}\raggedright
\textbf{Stage}
\end{minipage} & \begin{minipage}[b]{\linewidth}\raggedright
\textbf{Input}
\end{minipage} & \begin{minipage}[b]{\linewidth}\raggedright
\textbf{Output}
\end{minipage} & \begin{minipage}[b]{\linewidth}\raggedright
\textbf{Tools}
\end{minipage} \\
\midrule\noalign{}
\endhead
\bottomrule\noalign{}
\endlastfoot
\textbf{Architecture} & Specifications & Block diagram & High-level
modeling \\
\textbf{Logic} & Architecture & Gate netlist & HDL synthesis \\
\textbf{Circuit} & Netlist & Transistor sizing & SPICE simulation \\
\textbf{Layout} & Circuit & Mask data & Place \& route \\
\end{longtable}
}

\end{solutionbox}
\begin{mnemonicbox}
``Specify, Architect, Logic, Circuit, Layout,
Fabricate, Test''

\end{mnemonicbox}
\begin{center}\rule{0.5\linewidth}{0.5pt}\end{center}

\subsection*{Question 4(c OR) [7
marks]}\label{question-4c-or-7-marks}

\textbf{Write short note on `VLSI Fabrication Process'}

\begin{solutionbox}

\textbf{Major Fabrication Steps:}

{\def\LTcaptype{none} % do not increment counter
\begin{longtable}[]{@{}lll@{}}
\toprule\noalign{}
\textbf{Process} & \textbf{Purpose} & \textbf{Result} \\
\midrule\noalign{}
\endhead
\bottomrule\noalign{}
\endlastfoot
\textbf{Oxidation} & Grow SiO2 layer & Gate oxide formation \\
\textbf{Lithography} & Pattern transfer & Define device areas \\
\textbf{Etching} & Remove unwanted material & Create device
structures \\
\textbf{Ion Implantation} & Add dopants & Create P/N regions \\
\textbf{Deposition} & Add material layers & Metal interconnects \\
\textbf{Diffusion} & Spread dopants & Junction formation \\
\end{longtable}
}

\textbf{Process Flow:}

\begin{itemize}
\tightlist
\item
  \textbf{Wafer preparation}: Clean silicon substrate
\item
  \textbf{Device formation}: Create transistors through multiple steps
\item
  \textbf{Interconnect}: Add metal layers for connections
\item
  \textbf{Passivation}: Protect completed circuit
\item
  \textbf{Testing}: Verify functionality before packaging
\end{itemize}

\textbf{Clean Room Requirements:}

\begin{itemize}
\tightlist
\item
  \textbf{Class 1-10}: Ultra-clean environment needed
\item
  \textbf{Temperature control}: Precise process control
\item
  \textbf{Chemical purity}: High-grade materials required
\end{itemize}

\end{solutionbox}
\begin{mnemonicbox}
``Oxidize, Pattern, Etch, Implant, Deposit, Diffuse''

\end{mnemonicbox}
\begin{center}\rule{0.5\linewidth}{0.5pt}\end{center}

\subsection*{Question 5(a) [3 marks]}\label{q5a}

\textbf{Compare different styles of Verilog programming in VLSI.}

\begin{solutionbox}

\textbf{Verilog Modeling Styles:}

{\def\LTcaptype{none} % do not increment counter
\begin{longtable}[]{@{}lll@{}}
\toprule\noalign{}
\textbf{Style} & \textbf{Description} & \textbf{Application} \\
\midrule\noalign{}
\endhead
\bottomrule\noalign{}
\endlastfoot
\textbf{Behavioral} & Algorithm description & High-level modeling \\
\textbf{Dataflow} & Boolean expressions & Combinational logic \\
\textbf{Structural} & Gate-level description & Hardware
representation \\
\end{longtable}
}

\begin{itemize}
\tightlist
\item
  \textbf{Behavioral}: Uses always blocks, if-else, case statements
\item
  \textbf{Dataflow}: Uses assign statements with Boolean operators
\item
  \textbf{Structural}: Instantiates gates and modules explicitly
\end{itemize}

\end{solutionbox}
\begin{mnemonicbox}
``Behavior Describes, Dataflow Assigns, Structure
Connects''

\end{mnemonicbox}
\begin{center}\rule{0.5\linewidth}{0.5pt}\end{center}

\subsection*{Question 5(b) [4 marks]}\label{q5b}

\textbf{Write Verilog program of NAND gate using behavioral method.}

\begin{solutionbox}

\begin{verbatim}
module nand\_gate\_behavioral(
    input wire a, b,
    output reg y
);

always @(a or b) begin
if (a == 1{b1} \&\&

b == 1{b1})

        y = 1{b0};
    else
        y = 1{b1};
end

endmodule
\end{verbatim}

\textbf{Code Explanation:}

\begin{itemize}
\tightlist
\item
  \textbf{Always block}: Executes when inputs change
\item
  \textbf{Sensitivity list}: Contains all input signals
\item
  \textbf{Conditional statement}: Implements NAND logic
\item
  \textbf{Reg declaration}: Required for procedural assignment
\end{itemize}

\end{solutionbox}
\begin{mnemonicbox}
``Always watch, IF both high THEN low ELSE high''

\end{mnemonicbox}
\begin{center}\rule{0.5\linewidth}{0.5pt}\end{center}

\subsection*{Question 5(c) [7 marks]}\label{q5c}

\textbf{Draw 4X1 multiplexer circuit. Develop Verilog program of the
circuit using case statement.}

\begin{solutionbox}

\textbf{4X1 Multiplexer Circuit:}

\begin{verbatim}
I0 ──┐
I1 ──┼─── MUX ──── Y
I2 ──┤    4X1
I3 ──┘
  S1,S0 (Select)
\end{verbatim}

\textbf{Verilog Code:}

\begin{verbatim}
module mux\_4x1\_case(
    input wire [1:0] sel,
    input wire i0, i1, i2, i3,
    output reg y
);

always @(*) begin
    case (sel)
        2{b00}: y = i0;
        2{b01}: y = i1;
        2{b10}: y = i2;
        2{b11}: y = i3;
        default: y = 1{bx};
    endcase
end

endmodule
\end{verbatim}

\textbf{Truth Table:}

{\def\LTcaptype{none} % do not increment counter
\begin{longtable}[]{@{}lll@{}}
\toprule\noalign{}
\textbf{S1} & \textbf{S0} & \textbf{Output Y} \\
\midrule\noalign{}
\endhead
\bottomrule\noalign{}
\endlastfoot
\textbf{0} & \textbf{0} & I0 \\
\textbf{0} & \textbf{1} & I1 \\
\textbf{1} & \textbf{0} & I2 \\
\textbf{1} & \textbf{1} & I3 \\
\end{longtable}
}

\end{solutionbox}
\begin{mnemonicbox}
``Case Selects, Default Protects''

\end{mnemonicbox}
\begin{center}\rule{0.5\linewidth}{0.5pt}\end{center}

\subsection*{Question 5(a OR) [3
marks]}\label{question-5a-or-3-marks}

\textbf{Define Testbench with example.}

\begin{solutionbox}

\textbf{Testbench Definition:} Testbench is a Verilog module that
provides stimulus to design under test (DUT) and monitors its response.

\textbf{Example Testbench:}

\begin{verbatim}
module test\_and\_gate;
    reg a, b;
    wire y;
    
    and\_gate dut(.a(a), .b(b), .y(y));
    
    initial begin
a = 0;

b = 0; \#10;

a = 0;

b = 1; \#10;

a = 1;

b = 0; \#10;

a = 1;

b = 1; \#10;

    end
endmodule
\end{verbatim}

\begin{itemize}
\tightlist
\item
  \textbf{DUT instantiation}: Creates instance of design under test
\item
  \textbf{Stimulus generation}: Provides input test vectors
\item
  \textbf{No ports}: Testbench is top-level module
\end{itemize}

\end{solutionbox}
\begin{mnemonicbox}
``Test Provides Stimulus, Monitors Response''

\end{mnemonicbox}
\begin{center}\rule{0.5\linewidth}{0.5pt}\end{center}

\subsection*{Question 5(b OR) [4
marks]}\label{question-5b-or-4-marks}

\textbf{Write Verilog program of Half Adder using Dataflow method.}

\begin{solutionbox}

\begin{verbatim}
module half\_adder\_dataflow(
    input wire a, b,
    output wire sum, carry
);

assign sum = a \^{} b;    // XOR for sum
assign carry = a \& b;  // AND for carry

endmodule
\end{verbatim}

\textbf{Logic Implementation:}

\begin{itemize}
\tightlist
\item
  \textbf{Sum}: XOR operation between inputs
\item
  \textbf{Carry}: AND operation between inputs
\item
  \textbf{Assign statement}: Continuous assignment for dataflow
\item
  \textbf{Boolean operators}: \^{} (XOR), \& (AND)
\end{itemize}

\textbf{Truth Table:}

{\def\LTcaptype{none} % do not increment counter
\begin{longtable}[]{@{}llll@{}}
\toprule\noalign{}
\textbf{A} & \textbf{B} & \textbf{Sum} & \textbf{Carry} \\
\midrule\noalign{}
\endhead
\bottomrule\noalign{}
\endlastfoot
\textbf{0} & \textbf{0} & 0 & 0 \\
\textbf{0} & \textbf{1} & 1 & 0 \\
\textbf{1} & \textbf{0} & 1 & 0 \\
\textbf{1} & \textbf{1} & 0 & 1 \\
\end{longtable}
}

\end{solutionbox}
\begin{mnemonicbox}
``XOR Sums, AND Carries''

\end{mnemonicbox}
\begin{center}\rule{0.5\linewidth}{0.5pt}\end{center}

\subsection*{Question 5(c OR) [7
marks]}\label{question-5c-or-7-marks}

\textbf{Write function of Encoder. Develop code of 8X3 Encoder using
if\ldots.else statement.}

\begin{solutionbox}

\textbf{Encoder Function:} Encoder converts 2^{n} input lines to n output
lines. 8X3 encoder converts 8 inputs to 3-bit binary output.

\textbf{Priority Table:}

{\def\LTcaptype{none} % do not increment counter
\begin{longtable}[]{@{}ll@{}}
\toprule\noalign{}
\textbf{Input} & \textbf{Binary Output} \\
\midrule\noalign{}
\endhead
\bottomrule\noalign{}
\endlastfoot
\textbf{I7} & 111 \\
\textbf{I6} & 110 \\
\textbf{I5} & 101 \\
\textbf{I4} & 100 \\
\textbf{I3} & 011 \\
\textbf{I2} & 010 \\
\textbf{I1} & 001 \\
\textbf{I0} & 000 \\
\end{longtable}
}

\textbf{Verilog Code:}

\begin{verbatim}
module encoder\_8x3(
    input wire [7:0] i,
    output reg [2:0] y
);

always @(*) begin
    if (i[7])
        y = 3{b111};
    else if (i[6])
        y = 3{b110};
    else if (i[5])
        y = 3{b101};
    else if (i[4])
        y = 3{b100};
    else if (i[3])
        y = 3{b011};
    else if (i[2])
        y = 3{b010};
    else if (i[1])
        y = 3{b001};
    else if (i[0])
        y = 3{b000};
    else
        y = 3{bxxx};
end

endmodule
\end{verbatim}

\begin{itemize}
\tightlist
\item
  \textbf{Priority encoding}: Higher index inputs have priority
\item
  \textbf{If-else chain}: Implements priority logic
\item
  \textbf{Binary encoding}: Converts active input to binary
  representation
\end{itemize}

\end{solutionbox}
\begin{mnemonicbox}
``Priority from High to Low, Binary Output Flows''

\end{mnemonicbox}

\end{document}
