\documentclass[10pt,a4paper]{article}

% content/resources/templates/preamble.tex
\usepackage[margin=0.6in]{geometry}
\author{Milav Dabgar}
\usepackage{amsmath,amssymb,amsthm}
\usepackage{booktabs}
\usepackage{multirow}
\usepackage{xcolor}
\usepackage{tcolorbox}
\tcbuselibrary{breakable,skins}
\usepackage[colorlinks=true,linkcolor=blue]{hyperref}
\usepackage{titlesec}
\usepackage{enumitem}
\usepackage{tikz}
\usepackage{pgfplots}
\usepackage{circuitikz}
\usepackage[version=4]{mhchem}
\usepackage{longtable}
\usepackage{array}
\usepackage{float}
\usepackage{caption}
\usepackage{listings}

\lstset{
  basicstyle=\small\ttfamily,
  breaklines=true,
  breakatwhitespace=false,
  postbreak=\mbox{\textcolor{red}{$\hookrightarrow$}\space},
  float=false,
  numbers=left,
  numberstyle=\tiny\color{gray},
  numbersep=10pt,
  xleftmargin=2em,
  keywordstyle=\color{blue},
  commentstyle=\color{green!60!black},
  stringstyle=\color{purple},
  backgroundcolor=\color{gray!5},
  showstringspaces=false,
  tabsize=2,
  captionpos=b,
  keepspaces=true,
  columns=flexible
}

\pgfplotsset{compat=1.18}
\usetikzlibrary{shapes,arrows,positioning,calc,patterns,decorations.pathmorphing,decorations.markings,arrows.meta}

% Color scheme
\definecolor{headcolor}{RGB}{0,102,204}
\definecolor{keycolor}{RGB}{220,20,60}
\definecolor{solutioncolor}{RGB}{34,139,34}
\definecolor{mnemoniccolor}{RGB}{148,0,211}
\definecolor{codecolor}{RGB}{0,0,100}

% Spacing
\setlength{\parskip}{3pt}
\setlist[itemize]{nosep}
\setlist[enumerate]{nosep}

% Title formatting
\titleformat{\section}{\Large\bfseries\color{headcolor}}{\thesection}{1em}{}
\titleformat{\subsection}{\large\bfseries\color{headcolor}}{\thesubsection}{1em}{}

% Pandoc tightlist compatibility
\providecommand{\tightlist}{%
  \setlength{\itemsep}{0pt}\setlength{\parskip}{0pt}}

% Pandoc longtable compatibility
\newcounter{none}
\def\thenone{}


% content/resources/templates/gujarati-boxes.tex
\usepackage{fontspec}
\usepackage{polyglossia}

% Set Gujarati as main language (document is primarily in Gujarati)
% Note: gloss-gujarati.ldf doesn't exist in polyglossia, but it will use hyphenation patterns
\setdefaultlanguage{gujarati}
\setotherlanguage{english}

% Configure Gujarati font properly
% Use Language=Default to prevent polyglossia from trying to add language-specific features
% that don't exist for Gujarati, which causes "empty feature" warnings
\newfontfamily\gujaratifont[Script=Gujarati,AutoFakeBold=2.5,AutoFakeSlant=0.3]{Noto Sans Gujarati}
\setmainfont[Script=Gujarati,AutoFakeBold=2.5,AutoFakeSlant=0.3]{Noto Sans Gujarati}
% Use Noto Sans Gujarati for monospace to support Gujarati in text
\setmonofont[Scale=0.9]{Noto Sans Gujarati}

% Configure English to use the same font
\newfontfamily\englishfont[Script=Gujarati,AutoFakeBold=2.5,AutoFakeSlant=0.3]{Noto Sans Gujarati}

% Translations for polyglossia
\gappto\captionsgujarati{
  \renewcommand{\tablename}{કોષ્ટક}
  \renewcommand{\figurename}{આકૃતિ}
}

% Helper for TikZ nodes to ensure Gujarati font
\newcommand{\gu}[1]{{\gujaratifont #1}}

% Custom environments
\newtcolorbox{solutionbox}{
    breakable,
    enhanced,
    colback=solutioncolor!5!white,
    colframe=solutioncolor!75!black,
    fonttitle=\bfseries,
    title=જવાબ
}

\newtcolorbox{solutionboxnobreak}{
 colback=solutioncolor!5!white,
 colframe=solutioncolor!75!black,
 fonttitle=\bfseries,
 title=જવાબ
}

\newtcolorbox{keyformula}{
 breakable,
 enhanced,
 colback=keycolor!5!white,
 colframe=keycolor!75!black,
 fonttitle=\bfseries,
 title=રાસાયણિક સમીકરણ/સૂત્ર
}

\newtcolorbox{mnemonicbox}{
 breakable,
 enhanced,
 colback=mnemoniccolor!5!white,
 colframe=mnemoniccolor!75!black,
 fonttitle=\bfseries,
 title=મેમરી ટ્રીક
}


\begin{document}

\begin{center}
{\Huge\bfseries\color{headcolor} Subject Name (Gujarati)}\\[5pt]
{\LARGE 4361102 -- Winter 2024}\\[3pt]
{\large Semester 1 Study Material}\\[3pt]
{\normalsize\textit{Detailed Solutions and Explanations}}
\end{center}

\vspace{10pt}

\subsection*{પ્રશ્ન 1(અ) [3
ગુણ]}\label{uxaaauxab0uxab6uxaa8-1uxa85-3-uxa97uxaa3}

\textbf{High K FINFET ના ફાયદા લખો.}

\begin{solutionbox}

{\def\LTcaptype{none} % do not increment counter
\begin{longtable}[]{@{}
  >{\raggedright\arraybackslash}p{(\linewidth - 2\tabcolsep) * \real{0.5000}}
  >{\raggedright\arraybackslash}p{(\linewidth - 2\tabcolsep) * \real{0.5000}}@{}}
\toprule\noalign{}
\begin{minipage}[b]{\linewidth}\raggedright
\textbf{ફાયદો}
\end{minipage} & \begin{minipage}[b]{\linewidth}\raggedright
\textbf{વર્ણન}
\end{minipage} \\
\midrule\noalign{}
\endhead
\bottomrule\noalign{}
\endlastfoot
\textbf{ઓછો leakage current} & સારું \textbf{ગેટ કંટ્રોલ} પાવર consumption
ઘટાડે છે \\
\textbf{સુધારેલી performance} & વધુ \textbf{ડ્રાઇવ કરંટ} અને ઝડપી switching \\
\textbf{વધુ સારી scalability} & \textbf{Moore's law scaling} ચાલુ રાખવાની
મંજૂરી આપે છે \\
\end{longtable}
}

\begin{itemize}
\tightlist
\item
  \textbf{High K dielectric}: \textbf{ગેટ leakage} નોંધપાત્ર રીતે ઘટાડે છે
\item
  \textbf{3D structure}: \textbf{ચેનલ પર વધુ સારું electrostatic control}
\item
  \textbf{ઓછી પાવર}: \textbf{static અને dynamic power} બંને ઘટાડે છે
\end{itemize}

\end{solutionbox}
\begin{mnemonicbox}
``High Performance, Low Power, Better Control''

\end{mnemonicbox}
\begin{center}\rule{0.5\linewidth}{0.5pt}\end{center}

\subsection*{પ્રશ્ન 1(બ) [4
ગુણ]}\label{uxaaauxab0uxab6uxaa8-1uxaac-4-uxa97uxaa3}

\textbf{વ્યાખ્યા કરો: (1) pinch off point (2) Threshold Voltage.}

\begin{solutionbox}

\textbf{મુખ્ય MOSFET Parameters:}

{\def\LTcaptype{none} % do not increment counter
\begin{longtable}[]{@{}
  >{\raggedright\arraybackslash}p{(\linewidth - 4\tabcolsep) * \real{0.2903}}
  >{\raggedright\arraybackslash}p{(\linewidth - 4\tabcolsep) * \real{0.3871}}
  >{\raggedright\arraybackslash}p{(\linewidth - 4\tabcolsep) * \real{0.3226}}@{}}
\toprule\noalign{}
\begin{minipage}[b]{\linewidth}\raggedright
\textbf{શબ્દ}
\end{minipage} & \begin{minipage}[b]{\linewidth}\raggedright
\textbf{વ્યાખ્યા}
\end{minipage} & \begin{minipage}[b]{\linewidth}\raggedright
\textbf{મહત્વ}
\end{minipage} \\
\midrule\noalign{}
\endhead
\bottomrule\noalign{}
\endlastfoot
\textbf{Pinch-off Point} & \textbf{ચેનલ સંપૂર્ણ deplete} થતું સ્થાન &
\textbf{Saturation region} માં પ્રવેશ દર્શાવે છે \\
\textbf{Threshold Voltage} & \textbf{Conducting channel} બનાવવા માટે લઘુતમ
VGS & \textbf{ON/OFF switching point} નિર્ધારે છે \\
\end{longtable}
}

\begin{itemize}
\tightlist
\item
  \textbf{Pinch-off point}: VDS = VGS - VT, \textbf{ચેનલ શૂન્ય પહોળાઈ} સુધી
  સંકુચિત થાય છે
\item
  \textbf{Threshold voltage}: \textbf{Enhancement MOSFET} માટે સામાન્ય રીતે
  0.7V
\item
  \textbf{મહત્વપૂર્ણ parameters}: બંને \textbf{MOSFET operating regions}
  નિર્ધારે છે
\end{itemize}

\end{solutionbox}
\begin{mnemonicbox}
``Threshold Turns ON, Pinch-off Points to
Saturation''

\end{mnemonicbox}
\begin{center}\rule{0.5\linewidth}{0.5pt}\end{center}

\subsection*{પ્રશ્ન 1(ક) [7
ગુણ]}\label{uxaaauxab0uxab6uxaa8-1uxa95-7-uxa97uxaa3}

\textbf{MOSFET transistor નું બંધારણ દોરો અને સમજાવો.}

\begin{solutionbox}

\textbf{ડાયાગ્રામ:}

\begin{verbatim}
    Gate (G)
      |
   ┌──┴──┐
   │ SiO2│    
┌──┴─────┴──┐
│  n+   n+  │  Source (S) and Drain (D)
│     p     │  P{-substrate  }
└───────────┘
    Body (B)
\end{verbatim}

\textbf{બંધારણના ઘટકો:}

{\def\LTcaptype{none} % do not increment counter
\begin{longtable}[]{@{}
  >{\raggedright\arraybackslash}p{(\linewidth - 4\tabcolsep) * \real{0.2903}}
  >{\raggedright\arraybackslash}p{(\linewidth - 4\tabcolsep) * \real{0.3871}}
  >{\raggedright\arraybackslash}p{(\linewidth - 4\tabcolsep) * \real{0.3226}}@{}}
\toprule\noalign{}
\begin{minipage}[b]{\linewidth}\raggedright
\textbf{ઘટક}
\end{minipage} & \begin{minipage}[b]{\linewidth}\raggedright
\textbf{સામગ્રી}
\end{minipage} & \begin{minipage}[b]{\linewidth}\raggedright
\textbf{કાર્ય}
\end{minipage} \\
\midrule\noalign{}
\endhead
\bottomrule\noalign{}
\endlastfoot
\textbf{Gate} & \textbf{Polysilicon/Metal} & \textbf{ચેનલ formation}
કંટ્રોલ કરે છે \\
\textbf{Gate oxide} & \textbf{SiO2} & \textbf{Gate ને substrate} થી અલગ
કરે છે \\
\textbf{Source/Drain} & \textbf{n+ doped silicon} & \textbf{Current ના
પ્રવેશ/બહાર નીકળવાના સ્થળો} \\
\textbf{Substrate} & \textbf{p-type silicon} & \textbf{Body connection}
પૂરું પાડે છે \\
\end{longtable}
}

\begin{itemize}
\tightlist
\item
  \textbf{ચેનલ formation}: \textbf{Oxide-semiconductor interface} પર થાય
  છે
\item
  \textbf{Enhancement mode}: VGS \textgreater{} VT હોય ત્યારે \textbf{ચેનલ
  બને છે}
\item
  \textbf{ચાર-terminal device}: \textbf{Gate, Source, Drain, Body
  connections}
\end{itemize}

\end{solutionbox}
\begin{mnemonicbox}
``Gate Controls, Oxide Isolates, Source-Drain
Conducts''

\end{mnemonicbox}
\begin{center}\rule{0.5\linewidth}{0.5pt}\end{center}

\subsection*{પ્રશ્ન 1(ક OR) [7
ગુણ]}\label{uxaaauxab0uxab6uxaa8-1uxa95-or-7-uxa97uxaa3}

\textbf{Full Voltage Scaling અને Constant Voltage Scaling ની સરખામણી
કરો.}

\begin{solutionbox}

\textbf{સરખામણી કોષ્ટક:}

{\def\LTcaptype{none} % do not increment counter
\begin{longtable}[]{@{}
  >{\raggedright\arraybackslash}p{(\linewidth - 4\tabcolsep) * \real{0.2143}}
  >{\raggedright\arraybackslash}p{(\linewidth - 4\tabcolsep) * \real{0.3571}}
  >{\raggedright\arraybackslash}p{(\linewidth - 4\tabcolsep) * \real{0.4286}}@{}}
\toprule\noalign{}
\begin{minipage}[b]{\linewidth}\raggedright
\textbf{Parameter}
\end{minipage} & \begin{minipage}[b]{\linewidth}\raggedright
\textbf{Full Voltage Scaling}
\end{minipage} & \begin{minipage}[b]{\linewidth}\raggedright
\textbf{Constant Voltage Scaling}
\end{minipage} \\
\midrule\noalign{}
\endhead
\bottomrule\noalign{}
\endlastfoot
\textbf{Supply voltage} & α વડે \textbf{scale down} & \textbf{સ્થિર રહે
છે} \\
\textbf{Gate oxide thickness} & α વડે \textbf{scale down} & α વડે
\textbf{scale down} \\
\textbf{Channel length} & α વડે \textbf{scale down} & α વડે \textbf{scale
down} \\
\textbf{Power density} & \textbf{સ્થિર રહે છે} & α^{2} વડે \textbf{વધે છે} \\
\textbf{Performance} & \textbf{મધ્યમ સુધારો} & \textbf{વધુ સારી
performance} \\
\textbf{Reliability} & \textbf{વધુ સારી} & \textbf{High fields} ને કારણે
નબળી \\
\end{longtable}
}

\begin{itemize}
\tightlist
\item
  \textbf{Full scaling}: \textbf{બધા dimensions અને voltages} પ્રમાણસર
  scale કરાય છે
\item
  \textbf{Constant voltage}: \textbf{ફક્ત physical dimensions} scale કરાય
  છે, voltage અપરિવર્તિત
\item
  \textbf{Trade-off}: \textbf{Performance vs power vs reliability}
\end{itemize}

\end{solutionbox}
\begin{mnemonicbox}
``Full scales All, Constant keeps Voltage''

\end{mnemonicbox}
\begin{center}\rule{0.5\linewidth}{0.5pt}\end{center}

\subsection*{પ્રશ્ન 2(અ) [3
ગુણ]}\label{uxaaauxab0uxab6uxaa8-2uxa85-3-uxa97uxaa3}

\textbf{રેસિસ્ટિવ લોડ ઇનવર્ટર દોરો. જુદા જુદા ઓપરેશન રીજન માટે ઇનપુટ વોલ્ટેજની રેન્જ
લખો.}

\begin{solutionbox}

\textbf{સર્કિટ ડાયાગ્રામ:}

\begin{verbatim}
VDD ──┬── RL
      │
      ├── Vout
      │
Vin ──┤ M1 (NMOS)
      │
     GND
\end{verbatim}

\textbf{ઓપરેટિંગ રીજન કોષ્ટક:}

{\def\LTcaptype{none} % do not increment counter
\begin{longtable}[]{@{}lll@{}}
\toprule\noalign{}
\textbf{રીજન} & \textbf{ઇનપુટ વોલ્ટેજ રેન્જ} & \textbf{આઉટપુટ સ્થિતિ} \\
\midrule\noalign{}
\endhead
\bottomrule\noalign{}
\endlastfoot
\textbf{Cut-off} & Vin \textless{} VT & Vout = VDD \\
\textbf{Triode} & VT \textless{} Vin \textless{} VDD-VT &
\textbf{ટ્રાન્ઝિશન} \\
\textbf{Saturation} & Vin \textgreater{} VDD-VT & Vout \approx 0V \\
\end{longtable}
}

\end{solutionbox}
\begin{mnemonicbox}
``Cut-off High, Triode Transition, Saturation Low''

\end{mnemonicbox}
\begin{center}\rule{0.5\linewidth}{0.5pt}\end{center}

\subsection*{પ્રશ્ન 2(બ) [4
ગુણ]}\label{uxaaauxab0uxab6uxaa8-2uxaac-4-uxa97uxaa3}

\textbf{N channel MOSFET ની VDS-ID અને VGS-ID લાક્ષણિકતાઓ દોરો અને સમજાવો.}

\begin{solutionbox}

\textbf{VDS-ID લાક્ષણિકતાઓ:}

\begin{verbatim}
ID ↑
   │    VGS3
   │   ╱ VGS2
   │  ╱  VGS1
   │ ╱   (VGS3{VGS2VGS1VT)}
   │╱
   └──────── VDS
   Triode  Saturation
\end{verbatim}

\textbf{લાક્ષણિકતાઓ કોષ્ટક:}

{\def\LTcaptype{none} % do not increment counter
\begin{longtable}[]{@{}lll@{}}
\toprule\noalign{}
\textbf{લાક્ષણિકતા} & \textbf{રીજન} & \textbf{વર્તન} \\
\midrule\noalign{}
\endhead
\bottomrule\noalign{}
\endlastfoot
\textbf{VDS-ID} & \textbf{Triode} & VDS સાથે \textbf{Linear વૃદ્ધિ} \\
\textbf{VDS-ID} & \textbf{Saturation} & \textbf{સ્થિર ID} (square law) \\
\textbf{VGS-ID} & \textbf{Sub-threshold} & \textbf{Exponential વૃદ્ધિ} \\
\textbf{VGS-ID} & \textbf{VT ઉપર} & \textbf{Square law relationship} \\
\end{longtable}
}

\begin{itemize}
\tightlist
\item
  \textbf{Triode region}: ID વડે VDS સાથે \textbf{linearly વધે છે}
\item
  \textbf{Saturation}: ID \textbf{VDS થી સ્વતંત્ર}, VGS પર આધારિત
\item
  \textbf{Square law}: \textbf{Saturation} માં ID ∝ (VGS-VT)^{2}
\end{itemize}

\end{solutionbox}
\begin{mnemonicbox}
``Linear in Triode, Square in Saturation''

\end{mnemonicbox}
\begin{center}\rule{0.5\linewidth}{0.5pt}\end{center}

\subsection*{પ્રશ્ન 2(ક) [7
ગુણ]}\label{uxaaauxab0uxab6uxaa8-2uxa95-7-uxa97uxaa3}

\textbf{ડિપ્લેશન લોડ NMOS ઇનવર્ટર સર્કિટ દોરો અને તેની કાર્યપદ્ધતિ સમજાવો.}

\begin{solutionbox}

\textbf{સર્કિટ ડાયાગ્રામ:}

\begin{verbatim}
VDD ──┬─── ML (Depletion)
      │    Gate connected to Source
      ├─── Vout
      │
Vin ──┤    M1 (Enhancement)
      │
     GND
\end{verbatim}

\textbf{ઓપરેશન કોષ્ટક:}

{\def\LTcaptype{none} % do not increment counter
\begin{longtable}[]{@{}llll@{}}
\toprule\noalign{}
\textbf{ઇનપુટ} & \textbf{M1 સ્થિતિ} & \textbf{ML સ્થિતિ} &
\textbf{આઉટપુટ} \\
\midrule\noalign{}
\endhead
\bottomrule\noalign{}
\endlastfoot
\textbf{Low (0V)} & \textbf{Cut-off} & \textbf{Active load} &
\textbf{High (VDD)} \\
\textbf{High (VDD)} & \textbf{Saturated} & \textbf{Linear} &
\textbf{Low} \\
\end{longtable}
}

\begin{itemize}
\tightlist
\item
  \textbf{Depletion load}: \textbf{હંમેશા conducting}, \textbf{current
  source} તરીકે કાર્ય કરે છે
\item
  \textbf{વધુ સારી performance}: \textbf{Resistive load} કરતાં
  \textbf{higher output voltage swing}
\item
  \textbf{Gate connection}: \textbf{Depletion operation} માટે ML નું
  \textbf{gate source સાથે જોડાયેલું}
\item
  \textbf{સુધારેલું noise margin}: \textbf{Enhancement load} કરતાં \textbf{વધુ
  સારું VOH}
\end{itemize}

\end{solutionbox}
\begin{mnemonicbox}
``Depletion Always ON, Enhancement Controls Flow''

\end{mnemonicbox}
\begin{center}\rule{0.5\linewidth}{0.5pt}\end{center}

\subsection*{પ્રશ્ન 2(અ OR) [3
ગુણ]}\label{uxaaauxab0uxab6uxaa8-2uxa85-or-3-uxa97uxaa3}

\textbf{CMOS ઇનવર્ટર ના ફાયદા વર્ણવો.}

\begin{solutionbox}

\textbf{ફાયદા કોષ્ટક:}

{\def\LTcaptype{none} % do not increment counter
\begin{longtable}[]{@{}ll@{}}
\toprule\noalign{}
\textbf{ફાયદો} & \textbf{લાભ} \\
\midrule\noalign{}
\endhead
\bottomrule\noalign{}
\endlastfoot
\textbf{શૂન્ય static power} & \textbf{Steady state} માં કોઈ current નહીં \\
\textbf{સંપૂર્ણ voltage swing} & \textbf{આઉટપુટ 0V થી VDD} સુધી swing કરે છે \\
\textbf{ઉચ્ચ noise margins} & \textbf{વધુ સારી noise immunity} \\
\end{longtable}
}

\begin{itemize}
\tightlist
\item
  \textbf{Complementary operation}: \textbf{એક transistor હંમેશા OFF}
\item
  \textbf{ઉચ્ચ input impedance}: \textbf{Gate isolation} ઉચ્ચ impedance પૂરું
  પાડે છે
\item
  \textbf{ઝડપી switching}: \textbf{ઓછા parasitic capacitances}
\end{itemize}

\end{solutionbox}
\begin{mnemonicbox}
``Zero Power, Full Swing, High Immunity''

\end{mnemonicbox}
\begin{center}\rule{0.5\linewidth}{0.5pt}\end{center}

\subsection*{પ્રશ્ન 2(બ OR) [4
ગુણ]}\label{uxaaauxab0uxab6uxaa8-2uxaac-or-4-uxa97uxaa3}

\textbf{નોઇસ માર્જિન વિગતવાર દોરો અને સમજાવો.}

\begin{solutionbox}

\textbf{વોલ્ટેજ ટ્રાન્સફર લાક્ષણિકતાઓ:}

\begin{verbatim}
Vout ↑
VDD  ┌─────┐
     │     │
     │     │  NMH
VOH  ┤     └─────
     │           ╲
     │            ╲
     │             ╲
VOL  ┤              └─────
     │                   
  0V └───────────────── Vin
    0V  VIL  VIH      VDD
       
\end{verbatim}

\textbf{નોઇસ માર્જિન Parameters:}

{\def\LTcaptype{none} % do not increment counter
\begin{longtable}[]{@{}lll@{}}
\toprule\noalign{}
\textbf{Parameter} & \textbf{Formula} & \textbf{સામાન્ય મૂલ્ય} \\
\midrule\noalign{}
\endhead
\bottomrule\noalign{}
\endlastfoot
\textbf{NMH} & VOH - VIH & \textbf{VDD ના 40\%} \\
\textbf{NML} & VIL - VOL & \textbf{VDD ના 40\%} \\
\end{longtable}
}

\begin{itemize}
\tightlist
\item
  \textbf{High noise margin}: \textbf{Positive noise} સામે immunity
\item
  \textbf{Low noise margin}: \textbf{Negative noise} સામે immunity
\item
  \textbf{વધુ સારા CMOS}: \textbf{અન્ય logic families} કરતાં \textbf{higher
  noise margins}
\end{itemize}

\end{solutionbox}
\begin{mnemonicbox}
``High goes Higher, Low goes Lower''

\end{mnemonicbox}
\begin{center}\rule{0.5\linewidth}{0.5pt}\end{center}

\subsection*{પ્રશ્ન 2(ક OR) [7
ગુણ]}\label{uxaaauxab0uxab6uxaa8-2uxa95-or-7-uxa97uxaa3}

\textbf{N MOS ઇનવર્ટર ની VTC દોરો અને સમજાવો.}

\begin{solutionbox}

\textbf{વોલ્ટેજ ટ્રાન્સફર લાક્ષણિકતાઓ:}

\begin{verbatim}
Vout ↑
VDD  ┌─┐
     │ │
     │ │  Region I
     │ └─┐
     │   │  Region II  
     │   │
     │   └─┐
     │     │  Region III
     │     └──── Vin
  0V └─────────────
    0V VT        VDD
\end{verbatim}

\textbf{ઓપરેટિંગ રીજન કોષ્ટક:}

{\def\LTcaptype{none} % do not increment counter
\begin{longtable}[]{@{}llll@{}}
\toprule\noalign{}
\textbf{રીજન} & \textbf{Vin રેન્જ} & \textbf{M1 સ્થિતિ} & \textbf{Vout} \\
\midrule\noalign{}
\endhead
\bottomrule\noalign{}
\endlastfoot
\textbf{I} & \textbf{0 થી VT} & \textbf{Cut-off} & \textbf{VDD} \\
\textbf{II} & \textbf{VT થી VT+VTL} & \textbf{Saturation} &
\textbf{ઘટતું} \\
\textbf{III} & \textbf{VT+VTL થી VDD} & \textbf{Triode} &
\textbf{નીચું} \\
\end{longtable}
}

\begin{itemize}
\tightlist
\item
  \textbf{Region I}: \textbf{M1 OFF}, કોઈ current flow નહીં, Vout = VDD
\item
  \textbf{Region II}: \textbf{M1 saturation} માં, \textbf{તીવ્ર
  transition}
\item
  \textbf{Region III}: \textbf{M1 triode} માં, \textbf{ધીમેથી ઘટાડો}
\item
  \textbf{Load line}: \textbf{Operating point intersection} નિર્ધારે છે
\end{itemize}

\end{solutionbox}
\begin{mnemonicbox}
``Cut-off High, Saturation Sharp, Triode Low''

\end{mnemonicbox}
\begin{center}\rule{0.5\linewidth}{0.5pt}\end{center}

\subsection*{પ્રશ્ન 3(અ) [3
ગુણ]}\label{uxaaauxab0uxab6uxaa8-3uxa85-3-uxa97uxaa3}

\textbf{Generalized મલ્ટીપલ ઇનપુટ NOR gate નું બાંધકામ ડિપ્લેશન NMOS લોડ સાથે
દોરો અને સમજાવો.}

\begin{solutionbox}

\textbf{સર્કિટ ડાયાગ્રામ:}

\begin{verbatim}
VDD ──┬─── ML (Depletion Load)
      │
      ├─── Y = (A+B+C){}
      │
A  ───┤ M1
      │
B  ───┤ M2   Parallel Connection
      │
C  ───┤ M3
      │
     GND
\end{verbatim}

\textbf{સત્ય કોષ્ટક:}

{\def\LTcaptype{none} % do not increment counter
\begin{longtable}[]{@{}lll@{}}
\toprule\noalign{}
\textbf{ઇનપુટ્સ} & \textbf{કોઈ ઇનપુટ High?} & \textbf{આઉટપુટ Y} \\
\midrule\noalign{}
\endhead
\bottomrule\noalign{}
\endlastfoot
\textbf{બધા Low} & \textbf{ના} & \textbf{High (1)} \\
\textbf{કોઈ High} & \textbf{હા} & \textbf{Low (0)} \\
\end{longtable}
}

\begin{itemize}
\tightlist
\item
  \textbf{Parallel NMOS}: \textbf{કોઈપણ input HIGH} હોય તો
  \textbf{output LOW} થાય છે
\item
  \textbf{NOR operation}: Y = (A+B+C)'
\item
  \textbf{Depletion load}: \textbf{Pull-up current} પૂરું પાડે છે
\end{itemize}

\end{solutionbox}
\begin{mnemonicbox}
``Parallel Pulls Down, Depletion Pulls Up''

\end{mnemonicbox}
\begin{center}\rule{0.5\linewidth}{0.5pt}\end{center}

\subsection*{પ્રશ્ન 3(બ) [4
ગુણ]}\label{uxaaauxab0uxab6uxaa8-3uxaac-4-uxa97uxaa3}

\textbf{AOI અને OAI ના તફાવત લખો.}

\begin{solutionbox}

\textbf{સરખામણી કોષ્ટક:}

{\def\LTcaptype{none} % do not increment counter
\begin{longtable}[]{@{}lll@{}}
\toprule\noalign{}
\textbf{Parameter} & \textbf{AOI (AND-OR-Invert)} & \textbf{OAI
(OR-AND-Invert)} \\
\midrule\noalign{}
\endhead
\bottomrule\noalign{}
\endlastfoot
\textbf{Logic function} & Y = (AB + CD)' & Y = ((A+B)(C+D))' \\
\textbf{NMOS structure} & \textbf{Series-parallel} &
\textbf{Parallel-series} \\
\textbf{PMOS structure} & \textbf{Parallel-series} &
\textbf{Series-parallel} \\
\textbf{જટિલતા} & \textbf{મધ્યમ} & \textbf{મધ્યમ} \\
\end{longtable}
}

\begin{itemize}
\tightlist
\item
  \textbf{AOI}: \textbf{AND terms ORed} પછી \textbf{inverted}
\item
  \textbf{OAI}: \textbf{OR terms ANDed} પછી \textbf{inverted}
\item
  \textbf{CMOS implementation}: \textbf{Dual network structure}
\item
  \textbf{Applications}: \textbf{Single stage} માં \textbf{complex logic
  functions}
\end{itemize}

\end{solutionbox}
\begin{mnemonicbox}
``AOI: AND-OR-Invert, OAI: OR-AND-Invert''

\end{mnemonicbox}
\begin{center}\rule{0.5\linewidth}{0.5pt}\end{center}

\subsection*{પ્રશ્ન 3(ક) [7
ગુણ]}\label{uxaaauxab0uxab6uxaa8-3uxa95-7-uxa97uxaa3}

\textbf{EX-OR gate CMOS ની મદદથી અને લોજીક ફંક્શન Z = (AB +CD)' NMOS લોડની
મદદથી અમલમાં મૂકો.}

\begin{solutionbox}

\textbf{EX-OR CMOS Implementation:}

\begin{verbatim}
VDD ─┬─ pMOS network
     │  (A{B + AB)}
     ├─ Y = A  
     │
     ├─ nMOS network
     │  (AB + A{B)}
    GND
\end{verbatim}

\textbf{Z = (AB + CD)' NMOS Implementation:}

\begin{verbatim}
VDD ─┬─ Resistive Load
     │
     ├─ Z = (AB + CD){}
     │
A ─┬─┤ M1 ── B ─┤ M2  (Series: AB)
   │ │           │
C ─┤ M3 ── D ─┤ M4     (Series: CD)
   │           │
  GND ────────┴─      (Parallel connection)
\end{verbatim}

\textbf{લોજીક Implementation કોષ્ટક:}

{\def\LTcaptype{none} % do not increment counter
\begin{longtable}[]{@{}lll@{}}
\toprule\noalign{}
\textbf{સર્કિટ} & \textbf{ફંક્શન} & \textbf{Implementation} \\
\midrule\noalign{}
\endhead
\bottomrule\noalign{}
\endlastfoot
\textbf{EX-OR} & A\oplusB & \textbf{Complementary CMOS} \\
\textbf{AOI} & (AB+CD)' & \textbf{Series-parallel NMOS} \\
\end{longtable}
}

\begin{itemize}
\tightlist
\item
  \textbf{EX-OR}: \textbf{Efficient implementation} માટે
  \textbf{transmission gates} જરૂરી
\item
  \textbf{AOI function}: \textbf{Natural NMOS implementation}
\item
  \textbf{Power consideration}: \textbf{CMOS માં zero static power}
\end{itemize}

\end{solutionbox}
\begin{mnemonicbox}
``EX-OR needs Transmission, AOI uses
Series-Parallel''

\end{mnemonicbox}
\begin{center}\rule{0.5\linewidth}{0.5pt}\end{center}

\subsection*{પ્રશ્ન 3(અ OR) [3
ગુણ]}\label{uxaaauxab0uxab6uxaa8-3uxa85-or-3-uxa97uxaa3}

\textbf{Generalized મલ્ટીપલ ઇનપુટ NAND gate નું બાંધકામ ડિપ્લેશન NMOS લોડ સાથે
દોરો અને સમજાવો.}

\begin{solutionbox}

\textbf{સર્કિટ ડાયાગ્રામ:}

\begin{verbatim}
VDD ──┬─── ML (Depletion Load)
      │
      ├─── Y = (ABC){}
      │
A  ───┤ M1
      │
B  ───┤ M2   Series Connection  
      │
C  ───┤ M3
      │
     GND
\end{verbatim}

\textbf{ઓપરેશન કોષ્ટક:}

{\def\LTcaptype{none} % do not increment counter
\begin{longtable}[]{@{}lll@{}}
\toprule\noalign{}
\textbf{સ્થિતિ} & \textbf{Ground તરફ પાથ} & \textbf{આઉટપુટ Y} \\
\midrule\noalign{}
\endhead
\bottomrule\noalign{}
\endlastfoot
\textbf{બધા inputs HIGH} & \textbf{સંપૂર્ણ પાથ} & \textbf{Low (0)} \\
\textbf{કોઈ input LOW} & \textbf{તૂટેલો પાથ} & \textbf{High (1)} \\
\end{longtable}
}

\begin{itemize}
\tightlist
\item
  \textbf{Series NMOS}: \textbf{બધા inputs HIGH} હોવા જરૂરી
  \textbf{output LOW} કરવા માટે
\item
  \textbf{NAND operation}: Y = (ABC)'
\item
  \textbf{Depletion load}: \textbf{હંમેશા pull-up current} પૂરું પાડે છે
\end{itemize}

\end{solutionbox}
\begin{mnemonicbox}
``Series Needs All, NAND Says Not-AND''

\end{mnemonicbox}
\begin{center}\rule{0.5\linewidth}{0.5pt}\end{center}

\subsection*{પ્રશ્ન 3(બ OR) [4
ગુણ]}\label{uxaaauxab0uxab6uxaa8-3uxaac-or-4-uxa97uxaa3}

\textbf{((P+R)(S+T))' લોજીક ફંક્શન CMOS લોજીકની મદદથી અમલીકરણ કરો.}

\begin{solutionbox}

\textbf{CMOS Implementation:}

\begin{verbatim}
VDD ─┬─ pMOS Network
     │  P─┤├─R in series with S─┤├─T in series
     ├─ Y = ((P+R)(S+T)){}
     │
     ├─ nMOS Network  
     │  (P,R parallel) in series with (S,T parallel)
    GND
\end{verbatim}

\textbf{સત્ય કોષ્ટક Implementation:}

{\def\LTcaptype{none} % do not increment counter
\begin{longtable}[]{@{}lll@{}}
\toprule\noalign{}
\textbf{PMOS Network} & \textbf{NMOS Network} & \textbf{ઓપરેશન} \\
\midrule\noalign{}
\endhead
\bottomrule\noalign{}
\endlastfoot
\textbf{(P+R)`+(S+T)'} & \textbf{(P+R)(S+T)} & \textbf{Complementary} \\
\textbf{P'R' + S'T'} & \textbf{PS + PT + RS + RT} & \textbf{De Morgan's
law} \\
\end{longtable}
}

\begin{itemize}
\tightlist
\item
  \textbf{PMOS}: \textbf{Groups વિથિન parallel}, \textbf{groups વચ્ચે
  series}
\item
  \textbf{NMOS}: \textbf{Groups વિથિન series}, \textbf{groups વચ્ચે
  parallel}
\item
  \textbf{Dual network}: \textbf{Complementary operation} સુનિશ્ચિત કરે છે
\end{itemize}

\end{solutionbox}
\begin{mnemonicbox}
``PMOS does Opposite of NMOS''

\end{mnemonicbox}
\begin{center}\rule{0.5\linewidth}{0.5pt}\end{center}

\subsection*{પ્રશ્ન 3(ક OR) [7
ગુણ]}\label{uxaaauxab0uxab6uxaa8-3uxa95-or-7-uxa97uxaa3}

\textbf{SR latch circuit ની કાર્યપદ્ધતિ વર્ણવો.}

\begin{solutionbox}

\textbf{SR Latch સર્કિટ:}

\begin{verbatim}
S ─┤ NOR   ├─┬─ Q
   │  G1   │ │
   └───────┤ │
           │ │
   ┌───────┤ │
   │ NOR   │ │
R ─┤  G2   ├─┴─ Q{}
   └───────┘
\end{verbatim}

\textbf{સત્ય કોષ્ટક:}

{\def\LTcaptype{none} % do not increment counter
\begin{longtable}[]{@{}lllll@{}}
\toprule\noalign{}
\textbf{S} & \textbf{R} & \textbf{Q(n+1)} & \textbf{Q'(n+1)} &
\textbf{સ્થિતિ} \\
\midrule\noalign{}
\endhead
\bottomrule\noalign{}
\endlastfoot
\textbf{0} & \textbf{0} & Q(n) & Q'(n) & \textbf{Hold} \\
\textbf{0} & \textbf{1} & 0 & 1 & \textbf{Reset} \\
\textbf{1} & \textbf{0} & 1 & 0 & \textbf{Set} \\
\textbf{1} & \textbf{1} & 0 & 0 & \textbf{અમાન્ય} \\
\end{longtable}
}

\begin{itemize}
\tightlist
\item
  \textbf{Set operation}: S=1, R=0 થી Q=1 થાય છે
\item
  \textbf{Reset operation}: S=0, R=1 થી Q=0 થાય છે
\item
  \textbf{Hold state}: S=0, R=0 \textbf{પહેલાની state} જાળવે છે
\item
  \textbf{અમાન્ય state}: S=1, R=1 \textbf{ટાળવી જોઈએ}
\item
  \textbf{Cross-coupled}: \textbf{એક gate નું output બીજાના input} માં જાય
  છે
\end{itemize}

\end{solutionbox}
\begin{mnemonicbox}
``Set Sets, Reset Resets, Both Bad''

\end{mnemonicbox}
\begin{center}\rule{0.5\linewidth}{0.5pt}\end{center}

\subsection*{પ્રશ્ન 4(અ) [3
ગુણ]}\label{uxaaauxab0uxab6uxaa8-4uxa85-3-uxa97uxaa3}

\textbf{ચિપ ફેબ્રિકેશન માં Etching methods ની સરખામણી કરો.}

\begin{solutionbox}

\textbf{Etching Methods સરખામણી:}

{\def\LTcaptype{none} % do not increment counter
\begin{longtable}[]{@{}
  >{\raggedright\arraybackslash}p{(\linewidth - 6\tabcolsep) * \real{0.2326}}
  >{\raggedright\arraybackslash}p{(\linewidth - 6\tabcolsep) * \real{0.2326}}
  >{\raggedright\arraybackslash}p{(\linewidth - 6\tabcolsep) * \real{0.2558}}
  >{\raggedright\arraybackslash}p{(\linewidth - 6\tabcolsep) * \real{0.2791}}@{}}
\toprule\noalign{}
\begin{minipage}[b]{\linewidth}\raggedright
\textbf{પદ્ધતિ}
\end{minipage} & \begin{minipage}[b]{\linewidth}\raggedright
\textbf{પ્રકાર}
\end{minipage} & \begin{minipage}[b]{\linewidth}\raggedright
\textbf{ફાયદા}
\end{minipage} & \begin{minipage}[b]{\linewidth}\raggedright
\textbf{નુકસાન}
\end{minipage} \\
\midrule\noalign{}
\endhead
\bottomrule\noalign{}
\endlastfoot
\textbf{Wet Etching} & \textbf{રાસાયણિક} & \textbf{ઉચ્ચ selectivity}, સરળ
& \textbf{Isotropic}, undercut \\
\textbf{Dry Etching} & \textbf{ભૌતિક/રાસાયણિક} & \textbf{Anisotropic},
ચોક્કસ & \textbf{જટિલ સાધનો} \\
\textbf{Plasma Etching} & \textbf{Ion bombardment} & \textbf{Directional
control} & \textbf{સપાટીને નુકસાન} \\
\end{longtable}
}

\begin{itemize}
\tightlist
\item
  \textbf{Wet etching}: \textbf{પ્રવાહી રસાયણો} વાપરે છે, \textbf{બધી
  દિશાઓમાં હુમલો}
\item
  \textbf{Dry etching}: \textbf{ગેસ/plasma} વાપરે છે, \textbf{વધુ સારું
  directional control}
\item
  \textbf{Selectivity}: \textbf{એક સામગ્રીને બીજા કરતાં etch} કરવાની ક્ષમતા
\end{itemize}

\end{solutionbox}
\begin{mnemonicbox}
``Wet is Wide, Dry is Directional''

\end{mnemonicbox}
\begin{center}\rule{0.5\linewidth}{0.5pt}\end{center}

\subsection*{પ્રશ્ન 4(બ) [4
ગુણ]}\label{uxaaauxab0uxab6uxaa8-4uxaac-4-uxa97uxaa3}

\textbf{ટૂંક નોંધ લખો : Lithography}

\begin{solutionbox}

\textbf{Lithography Process Steps:}

{\def\LTcaptype{none} % do not increment counter
\begin{longtable}[]{@{}
  >{\raggedright\arraybackslash}p{(\linewidth - 4\tabcolsep) * \real{0.3030}}
  >{\raggedright\arraybackslash}p{(\linewidth - 4\tabcolsep) * \real{0.3939}}
  >{\raggedright\arraybackslash}p{(\linewidth - 4\tabcolsep) * \real{0.3030}}@{}}
\toprule\noalign{}
\begin{minipage}[b]{\linewidth}\raggedright
\textbf{સ્ટેપ}
\end{minipage} & \begin{minipage}[b]{\linewidth}\raggedright
\textbf{પ્રક્રિયા}
\end{minipage} & \begin{minipage}[b]{\linewidth}\raggedright
\textbf{હેતુ}
\end{minipage} \\
\midrule\noalign{}
\endhead
\bottomrule\noalign{}
\endlastfoot
\textbf{Resist coating} & \textbf{Photoresist નું spin-on} &
\textbf{પ્રકાશ-સંવેદનશીલ layer} \\
\textbf{Exposure} & \textbf{Mask દ્વારા UV light} & \textbf{Pattern
transfer} \\
\textbf{Development} & \textbf{Exposed resist દૂર કરવું} & \textbf{Pattern
પ્રગટ કરવું} \\
\textbf{Etching} & \textbf{અસુરક્ષિત material દૂર કરવું} & \textbf{Features
બનાવવા} \\
\end{longtable}
}

\begin{itemize}
\tightlist
\item
  \textbf{Pattern transfer}: \textbf{Mask થી silicon wafer} પર
\item
  \textbf{Resolution}: \textbf{Minimum feature size} નિર્ધારે છે
\item
  \textbf{Alignment}: \textbf{Multiple layer processing} માટે મહત્વપૂર્ણ
\item
  \textbf{UV wavelength}: \textbf{ટૂંકી wavelength} વધુ સારું resolution આપે છે
\end{itemize}

\end{solutionbox}
\begin{mnemonicbox}
``Coat, Expose, Develop, Etch''

\end{mnemonicbox}
\begin{center}\rule{0.5\linewidth}{0.5pt}\end{center}

\subsection*{પ્રશ્ન 4(ક) [7
ગુણ]}\label{uxaaauxab0uxab6uxaa8-4uxa95-7-uxa97uxaa3}

\textbf{Regularity, Modularity and Locality સમજાવો.}

\begin{solutionbox}

\textbf{ડિઝાઈન સિદ્ધાંતો કોષ્ટક:}

{\def\LTcaptype{none} % do not increment counter
\begin{longtable}[]{@{}
  >{\raggedright\arraybackslash}p{(\linewidth - 6\tabcolsep) * \real{0.2766}}
  >{\raggedright\arraybackslash}p{(\linewidth - 6\tabcolsep) * \real{0.2553}}
  >{\raggedright\arraybackslash}p{(\linewidth - 6\tabcolsep) * \real{0.2340}}
  >{\raggedright\arraybackslash}p{(\linewidth - 6\tabcolsep) * \real{0.2340}}@{}}
\toprule\noalign{}
\begin{minipage}[b]{\linewidth}\raggedright
\textbf{સિદ્ધાંત}
\end{minipage} & \begin{minipage}[b]{\linewidth}\raggedright
\textbf{વ્યાખ્યા}
\end{minipage} & \begin{minipage}[b]{\linewidth}\raggedright
\textbf{ફાયદા}
\end{minipage} & \begin{minipage}[b]{\linewidth}\raggedright
\textbf{ઉદાહરણ}
\end{minipage} \\
\midrule\noalign{}
\endhead
\bottomrule\noalign{}
\endlastfoot
\textbf{Regularity} & \textbf{સમાન structures નું પુનરાવર્તન} & \textbf{સરળ
design, testing} & \textbf{Memory arrays} \\
\textbf{Modularity} & \textbf{Hierarchical design blocks} &
\textbf{Reusability, maintainability} & \textbf{Standard cells} \\
\textbf{Locality} & \textbf{સંબંધિત functions નું જૂથ} & \textbf{ઓછું
interconnect} & \textbf{Functional blocks} \\
\end{longtable}
}

\textbf{Implementation વિગતો:}

\begin{itemize}
\tightlist
\item
  \textbf{Regularity}: \textbf{સમાન cell બારંબાર} વાપરવાથી \textbf{design
  complexity} ઘટે છે
\item
  \textbf{Modularity}: \textbf{Well-defined interfaces} સાથે
  \textbf{top-down design}
\item
  \textbf{Locality}: \textbf{Wire delays અને routing congestion} ઘટાડે છે
\item
  \textbf{Design benefits}: \textbf{ઝડપી design cycle}, \textbf{વધુ સારી
  testability}
\item
  \textbf{Manufacturing}: \textbf{Regular patterns} દ્વારા \textbf{સુધારેલી
  yield}
\end{itemize}

\textbf{Mermaid Diagram:}

\begin{center}
\textbf{Mermaid Diagram (Code)}
\begin{verbatim}
{Shaded}
{Highlighting}[]
graph LR
    A[System Level] {-{-}{} B[Module Level]}
    B {-{-}{} C[Cell Level]}
    C {-{-}{} D[Device Level]}
    D {-{-}{} E[Regular Structures]}
{Highlighting}
{Shaded}
\end{verbatim}
\end{center}

\end{solutionbox}
\begin{mnemonicbox}
``Regular Modules with Local Connections''

\end{mnemonicbox}
\begin{center}\rule{0.5\linewidth}{0.5pt}\end{center}

\subsection*{પ્રશ્ન 4(અ OR) [3
ગુણ]}\label{uxaaauxab0uxab6uxaa8-4uxa85-or-3-uxa97uxaa3}

\textbf{Design Hierarchy વ્યાખ્યાયિત કરો.}

\begin{solutionbox}

\textbf{Design Hierarchy Levels:}

{\def\LTcaptype{none} % do not increment counter
\begin{longtable}[]{@{}lll@{}}
\toprule\noalign{}
\textbf{સ્તર} & \textbf{વિવરણ} & \textbf{ઘટકો} \\
\midrule\noalign{}
\endhead
\bottomrule\noalign{}
\endlastfoot
\textbf{System} & \textbf{સંપૂર્ણ chip functionality} & \textbf{Processors,
memories} \\
\textbf{Module} & \textbf{મુખ્ય functional blocks} & \textbf{ALU, cache,
I/O} \\
\textbf{Cell} & \textbf{મૂળભૂત logic elements} & \textbf{Gates,
flip-flops} \\
\end{longtable}
}

\begin{itemize}
\tightlist
\item
  \textbf{Top-down approach}: \textbf{System નાના modules} માં વિભાજિત
\item
  \textbf{Abstraction levels}: \textbf{દરેક level નીચેની details છુપાવે} છે
\item
  \textbf{Interface definition}: \textbf{Levels વચ્ચે સ્પષ્ટ boundaries}
\end{itemize}

\end{solutionbox}
\begin{mnemonicbox}
``System to Module to Cell''

\end{mnemonicbox}
\begin{center}\rule{0.5\linewidth}{0.5pt}\end{center}

\subsection*{પ્રશ્ન 4(બ OR) [4
ગુણ]}\label{uxaaauxab0uxab6uxaa8-4uxaac-or-4-uxa97uxaa3}

\textbf{VLSI design flow chart દોરો અને સમજાવો.}

\begin{solutionbox}

\textbf{VLSI Design Flow:}

\begin{center}
\textbf{Mermaid Diagram (Code)}
\begin{verbatim}
{Shaded}
{Highlighting}[]
graph LR
    A[System Specification] {-{-}{} B[Architectural Design]}
    B {-{-}{} C[Logic Design]}
    C {-{-}{} D[Circuit Design]}
    D {-{-}{} E[Layout Design]}
    E {-{-}{} F[Fabrication]}
    F {-{-}{} G[Testing]}
{Highlighting}
{Shaded}
\end{verbatim}
\end{center}

\textbf{Design Flow કોષ્ટક:}

{\def\LTcaptype{none} % do not increment counter
\begin{longtable}[]{@{}
  >{\raggedright\arraybackslash}p{(\linewidth - 6\tabcolsep) * \real{0.2500}}
  >{\raggedright\arraybackslash}p{(\linewidth - 6\tabcolsep) * \real{0.2500}}
  >{\raggedright\arraybackslash}p{(\linewidth - 6\tabcolsep) * \real{0.2727}}
  >{\raggedright\arraybackslash}p{(\linewidth - 6\tabcolsep) * \real{0.2273}}@{}}
\toprule\noalign{}
\begin{minipage}[b]{\linewidth}\raggedright
\textbf{તબક્કો}
\end{minipage} & \begin{minipage}[b]{\linewidth}\raggedright
\textbf{ઇનપુટ}
\end{minipage} & \begin{minipage}[b]{\linewidth}\raggedright
\textbf{આઉટપુટ}
\end{minipage} & \begin{minipage}[b]{\linewidth}\raggedright
\textbf{સાધનો}
\end{minipage} \\
\midrule\noalign{}
\endhead
\bottomrule\noalign{}
\endlastfoot
\textbf{Architecture} & \textbf{Specifications} & \textbf{Block diagram}
& \textbf{High-level modeling} \\
\textbf{Logic} & \textbf{Architecture} & \textbf{Gate netlist} &
\textbf{HDL synthesis} \\
\textbf{Circuit} & \textbf{Netlist} & \textbf{Transistor sizing} &
\textbf{SPICE simulation} \\
\textbf{Layout} & \textbf{Circuit} & \textbf{Mask data} & \textbf{Place
\& route} \\
\end{longtable}
}

\end{solutionbox}
\begin{mnemonicbox}
``Specify, Architect, Logic, Circuit, Layout,
Fabricate, Test''

\end{mnemonicbox}
\begin{center}\rule{0.5\linewidth}{0.5pt}\end{center}

\subsection*{પ્રશ્ન 4(ક OR) [7
ગુણ]}\label{uxaaauxab0uxab6uxaa8-4uxa95-or-7-uxa97uxaa3}

\textbf{ટૂંક નોંધ લખો : `VLSI Fabrication Process'}

\begin{solutionbox}

\textbf{મુખ્ય Fabrication Steps:}

{\def\LTcaptype{none} % do not increment counter
\begin{longtable}[]{@{}
  >{\raggedright\arraybackslash}p{(\linewidth - 4\tabcolsep) * \real{0.3714}}
  >{\raggedright\arraybackslash}p{(\linewidth - 4\tabcolsep) * \real{0.2857}}
  >{\raggedright\arraybackslash}p{(\linewidth - 4\tabcolsep) * \real{0.3429}}@{}}
\toprule\noalign{}
\begin{minipage}[b]{\linewidth}\raggedright
\textbf{પ્રક્રિયા}
\end{minipage} & \begin{minipage}[b]{\linewidth}\raggedright
\textbf{હેતુ}
\end{minipage} & \begin{minipage}[b]{\linewidth}\raggedright
\textbf{પરિણામ}
\end{minipage} \\
\midrule\noalign{}
\endhead
\bottomrule\noalign{}
\endlastfoot
\textbf{Oxidation} & \textbf{SiO2 layer વૃદ્ધિ} & \textbf{Gate oxide
formation} \\
\textbf{Lithography} & \textbf{Pattern transfer} & \textbf{Device areas
વ્યાખ્યા} \\
\textbf{Etching} & \textbf{અનાવશ્યક material દૂર કરવું} & \textbf{Device
structures બનાવવા} \\
\textbf{Ion Implantation} & \textbf{Dopants ઉમેરવા} & \textbf{P/N regions
બનાવવા} \\
\textbf{Deposition} & \textbf{Material layers ઉમેરવા} & \textbf{Metal
interconnects} \\
\textbf{Diffusion} & \textbf{Dopants ફેલાવવા} & \textbf{Junction
formation} \\
\end{longtable}
}

\textbf{Process Flow:}

\begin{itemize}
\tightlist
\item
  \textbf{Wafer preparation}: \textbf{સ્વચ્છ silicon substrate}
\item
  \textbf{Device formation}: \textbf{બિનેક steps દ્વારા transistors}
  બનાવવા
\item
  \textbf{Interconnect}: \textbf{Connections માટે metal layers} ઉમેરવા
\item
  \textbf{Passivation}: \textbf{પૂર્ણ થયેલા circuit ની સુરક્ષા}
\item
  \textbf{Testing}: \textbf{Packaging પહેલાં functionality verify} કરવી
\end{itemize}

\textbf{Clean Room જરૂરિયાતો:}

\begin{itemize}
\tightlist
\item
  \textbf{Class 1-10}: \textbf{અત્યંત સ્વચ્છ વાતાવરણ} જરૂરી
\item
  \textbf{Temperature control}: \textbf{ચોક્કસ process control}
\item
  \textbf{Chemical purity}: \textbf{ઉચ્ચ-ગ્રેડ materials} જરૂરી
\end{itemize}

\end{solutionbox}
\begin{mnemonicbox}
``Oxidize, Pattern, Etch, Implant, Deposit,
Diffuse''

\end{mnemonicbox}
\begin{center}\rule{0.5\linewidth}{0.5pt}\end{center}

\subsection*{પ્રશ્ન 5(અ) [3
ગુણ]}\label{uxaaauxab0uxab6uxaa8-5uxa85-3-uxa97uxaa3}

\textbf{વેરીલોગ પ્રોગ્રામિંગની જુદી જુદી પદ્ધતિ સરખાવો.}

\begin{solutionbox}

\textbf{Verilog Modeling Styles:}

{\def\LTcaptype{none} % do not increment counter
\begin{longtable}[]{@{}
  >{\raggedright\arraybackslash}p{(\linewidth - 4\tabcolsep) * \real{0.3000}}
  >{\raggedright\arraybackslash}p{(\linewidth - 4\tabcolsep) * \real{0.3667}}
  >{\raggedright\arraybackslash}p{(\linewidth - 4\tabcolsep) * \real{0.3333}}@{}}
\toprule\noalign{}
\begin{minipage}[b]{\linewidth}\raggedright
\textbf{શૈલી}
\end{minipage} & \begin{minipage}[b]{\linewidth}\raggedright
\textbf{વિવરણ}
\end{minipage} & \begin{minipage}[b]{\linewidth}\raggedright
\textbf{ઉપયોગ}
\end{minipage} \\
\midrule\noalign{}
\endhead
\bottomrule\noalign{}
\endlastfoot
\textbf{Behavioral} & \textbf{Algorithm description} &
\textbf{High-level modeling} \\
\textbf{Dataflow} & \textbf{Boolean expressions} & \textbf{Combinational
logic} \\
\textbf{Structural} & \textbf{Gate-level description} & \textbf{Hardware
representation} \\
\end{longtable}
}

\begin{itemize}
\tightlist
\item
  \textbf{Behavioral}: \textbf{Always blocks, if-else, case statements}
  વાપરે છે
\item
  \textbf{Dataflow}: \textbf{Boolean operators સાથે assign statements}
  વાપરે છે
\item
  \textbf{Structural}: \textbf{Gates અને modules explicitly instantiate}
  કરે છે
\end{itemize}

\end{solutionbox}
\begin{mnemonicbox}
``Behavior Describes, Dataflow Assigns, Structure
Connects''

\end{mnemonicbox}
\begin{center}\rule{0.5\linewidth}{0.5pt}\end{center}

\subsection*{પ્રશ્ન 5(બ) [4
ગુણ]}\label{uxaaauxab0uxab6uxaa8-5uxaac-4-uxa97uxaa3}

\textbf{બિહેવિયરલ પદ્ધતિ થી NAND gate નો વેરીલોગ પ્રોગ્રામ લખો.}

\begin{solutionbox}

\begin{verbatim}
module nand\_gate\_behavioral(
    input wire a, b,
    output reg y
);

always @(a or b) begin
if (a == 1{b1} \&\&

b == 1{b1})

        y = 1{b0};
    else
        y = 1{b1};
end

endmodule
\end{verbatim}

\textbf{કોડ સમજૂતી:}

\begin{itemize}
\tightlist
\item
  \textbf{Always block}: \textbf{Inputs બદલાય} ત્યારે execute થાય છે
\item
  \textbf{Sensitivity list}: \textbf{બધા input signals} સમાવે છે
\item
  \textbf{Conditional statement}: \textbf{NAND logic implement} કરે છે
\item
  \textbf{Reg declaration}: \textbf{Procedural assignment} માટે જરૂરી
\end{itemize}

\end{solutionbox}
\begin{mnemonicbox}
``Always watch, IF both high THEN low ELSE high''

\end{mnemonicbox}
\begin{center}\rule{0.5\linewidth}{0.5pt}\end{center}

\subsection*{પ્રશ્ન 5(ક) [7
ગુણ]}\label{uxaaauxab0uxab6uxaa8-5uxa95-7-uxa97uxaa3}

\textbf{4X1 multiplexer ની સર્કિટ દોરો. Case સ્ટેટમેંટ થી આ સર્કિટ નો વેરીલોગ
પ્રોગ્રામ બનાવો.}

\begin{solutionbox}

\textbf{4X1 Multiplexer સર્કિટ:}

\begin{verbatim}
I0 ──┐
I1 ──┼─── MUX ──── Y
I2 ──┤    4X1
I3 ──┘
  S1,S0 (Select)
\end{verbatim}

\textbf{Verilog કોડ:}

\begin{verbatim}
module mux\_4x1\_case(
    input wire [1:0] sel,
    input wire i0, i1, i2, i3,
    output reg y
);

always @(*) begin
    case (sel)
        2{b00}: y = i0;
        2{b01}: y = i1;
        2{b10}: y = i2;
        2{b11}: y = i3;
        default: y = 1{bx};
    endcase
end

endmodule
\end{verbatim}

\textbf{સત્ય કોષ્ટક:}

{\def\LTcaptype{none} % do not increment counter
\begin{longtable}[]{@{}lll@{}}
\toprule\noalign{}
\textbf{S1} & \textbf{S0} & \textbf{આઉટપુટ Y} \\
\midrule\noalign{}
\endhead
\bottomrule\noalign{}
\endlastfoot
\textbf{0} & \textbf{0} & \textbf{I0} \\
\textbf{0} & \textbf{1} & \textbf{I1} \\
\textbf{1} & \textbf{0} & \textbf{I2} \\
\textbf{1} & \textbf{1} & \textbf{I3} \\
\end{longtable}
}

\end{solutionbox}
\begin{mnemonicbox}
``Case Selects, Default Protects''

\end{mnemonicbox}
\begin{center}\rule{0.5\linewidth}{0.5pt}\end{center}

\subsection*{પ્રશ્ન 5(અ OR) [3
ગુણ]}\label{uxaaauxab0uxab6uxaa8-5uxa85-or-3-uxa97uxaa3}

\textbf{ઉદાહરણ સાથે Testbench વ્યાખ્યાયિત કરો.}

\begin{solutionbox}

\textbf{Testbench વ્યાખ્યા:} \textbf{Testbench એ Verilog module} છે જે
\textbf{design under test (DUT) ને stimulus} પૂરું પાડે છે અને \textbf{તેના
response ને monitor} કરે છે.

\textbf{ઉદાહરણ Testbench:}

\begin{verbatim}
module test\_and\_gate;
    reg a, b;
    wire y;
    
    and\_gate dut(.a(a), .b(b), .y(y));
    
    initial begin
a = 0;

b = 0; \#10;

a = 0;

b = 1; \#10;

a = 1;

b = 0; \#10;

a = 1;

b = 1; \#10;

    end
endmodule
\end{verbatim}

\begin{itemize}
\tightlist
\item
  \textbf{DUT instantiation}: \textbf{Design under test નું instance} બનાવે
  છે
\item
  \textbf{Stimulus generation}: \textbf{Input test vectors} પૂરા પાડે છે
\item
  \textbf{કોઈ ports નહીં}: \textbf{Testbench top-level module} છે
\end{itemize}

\end{solutionbox}
\begin{mnemonicbox}
``Test Provides Stimulus, Monitors Response''

\end{mnemonicbox}
\begin{center}\rule{0.5\linewidth}{0.5pt}\end{center}

\subsection*{પ્રશ્ન 5(બ OR) [4
ગુણ]}\label{uxaaauxab0uxab6uxaa8-5uxaac-or-4-uxa97uxaa3}

\textbf{ડેટા ફ્લો પદ્ધતિ થી Half Adder નો વેરીલોગ પ્રોગ્રામ લખો.}

\begin{solutionbox}

\begin{verbatim}
module half\_adder\_dataflow(
    input wire a, b,
    output wire sum, carry
);

assign sum = a \^{} b;    // XOR for sum
assign carry = a \& b;  // AND for carry

endmodule
\end{verbatim}

\textbf{લોજીક Implementation:}

\begin{itemize}
\tightlist
\item
  \textbf{Sum}: \textbf{Inputs વચ્ચે XOR operation}
\item
  \textbf{Carry}: \textbf{Inputs વચ્ચે AND operation}
\item
  \textbf{Assign statement}: \textbf{Dataflow માટે continuous assignment}
\item
  \textbf{Boolean operators}: \textbf{\^{} (XOR), \& (AND)}
\end{itemize}

\textbf{સત્ય કોષ્ટક:}

{\def\LTcaptype{none} % do not increment counter
\begin{longtable}[]{@{}llll@{}}
\toprule\noalign{}
\textbf{A} & \textbf{B} & \textbf{Sum} & \textbf{Carry} \\
\midrule\noalign{}
\endhead
\bottomrule\noalign{}
\endlastfoot
\textbf{0} & \textbf{0} & \textbf{0} & \textbf{0} \\
\textbf{0} & \textbf{1} & \textbf{1} & \textbf{0} \\
\textbf{1} & \textbf{0} & \textbf{1} & \textbf{0} \\
\textbf{1} & \textbf{1} & \textbf{0} & \textbf{1} \\
\end{longtable}
}

\end{solutionbox}
\begin{mnemonicbox}
``XOR Sums, AND Carries''

\end{mnemonicbox}
\begin{center}\rule{0.5\linewidth}{0.5pt}\end{center}

\subsection*{પ્રશ્ન 5(ક OR) [7
ગુણ]}\label{uxaaauxab0uxab6uxaa8-5uxa95-or-7-uxa97uxaa3}

\textbf{Encoder નું કાર્ય લખો. if..else વડે 8X3 Encoder નો વેરીલોગ કોડ
બનાવો.}

\begin{solutionbox}

\textbf{Encoder કાર્ય:} \textbf{Encoder 2^{n} input lines ને n output lines}
માં convert કરે છે. \textbf{8X3 encoder 8 inputs ને 3-bit binary output} માં
convert કરે છે.

\textbf{Priority કોષ્ટક:}

{\def\LTcaptype{none} % do not increment counter
\begin{longtable}[]{@{}ll@{}}
\toprule\noalign{}
\textbf{ઇનપુટ} & \textbf{Binary આઉટપુટ} \\
\midrule\noalign{}
\endhead
\bottomrule\noalign{}
\endlastfoot
\textbf{I7} & \textbf{111} \\
\textbf{I6} & \textbf{110} \\
\textbf{I5} & \textbf{101} \\
\textbf{I4} & \textbf{100} \\
\textbf{I3} & \textbf{011} \\
\textbf{I2} & \textbf{010} \\
\textbf{I1} & \textbf{001} \\
\textbf{I0} & \textbf{000} \\
\end{longtable}
}

\textbf{Verilog કોડ:}

\begin{verbatim}
module encoder\_8x3(
    input wire [7:0] i,
    output reg [2:0] y
);

always @(*) begin
    if (i[7])
        y = 3{b111};
    else if (i[6])
        y = 3{b110};
    else if (i[5])
        y = 3{b101};
    else if (i[4])
        y = 3{b100};
    else if (i[3])
        y = 3{b011};
    else if (i[2])
        y = 3{b010};
    else if (i[1])
        y = 3{b001};
    else if (i[0])
        y = 3{b000};
    else
        y = 3{bxxx};
end

endmodule
\end{verbatim}

\begin{itemize}
\tightlist
\item
  \textbf{Priority encoding}: \textbf{ઉચ્ચ index inputs ને priority}
\item
  \textbf{If-else chain}: \textbf{Priority logic implement} કરે છે
\item
  \textbf{Binary encoding}: \textbf{Active input ને binary
  representation} માં convert કરે છે
\end{itemize}

\end{solutionbox}
\begin{mnemonicbox}
``Priority from High to Low, Binary Output Flows''

\end{mnemonicbox}

\end{document}
