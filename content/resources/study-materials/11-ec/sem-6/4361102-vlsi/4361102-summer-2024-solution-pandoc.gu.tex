\documentclass[10pt,a4paper]{article}

% content/resources/templates/preamble.tex
\usepackage[margin=0.6in]{geometry}
\author{Milav Dabgar}
\usepackage{amsmath,amssymb,amsthm}
\usepackage{booktabs}
\usepackage{multirow}
\usepackage{xcolor}
\usepackage{tcolorbox}
\tcbuselibrary{breakable,skins}
\usepackage[colorlinks=true,linkcolor=blue]{hyperref}
\usepackage{titlesec}
\usepackage{enumitem}
\usepackage{tikz}
\usepackage{pgfplots}
\usepackage{circuitikz}
\usepackage[version=4]{mhchem}
\usepackage{longtable}
\usepackage{array}
\usepackage{float}
\usepackage{caption}
\usepackage{listings}

\lstset{
  basicstyle=\small\ttfamily,
  breaklines=true,
  breakatwhitespace=false,
  postbreak=\mbox{\textcolor{red}{$\hookrightarrow$}\space},
  float=false,
  numbers=left,
  numberstyle=\tiny\color{gray},
  numbersep=10pt,
  xleftmargin=2em,
  keywordstyle=\color{blue},
  commentstyle=\color{green!60!black},
  stringstyle=\color{purple},
  backgroundcolor=\color{gray!5},
  showstringspaces=false,
  tabsize=2,
  captionpos=b,
  keepspaces=true,
  columns=flexible
}

\pgfplotsset{compat=1.18}
\usetikzlibrary{shapes,arrows,positioning,calc,patterns,decorations.pathmorphing,decorations.markings,arrows.meta}

% Color scheme
\definecolor{headcolor}{RGB}{0,102,204}
\definecolor{keycolor}{RGB}{220,20,60}
\definecolor{solutioncolor}{RGB}{34,139,34}
\definecolor{mnemoniccolor}{RGB}{148,0,211}
\definecolor{codecolor}{RGB}{0,0,100}

% Spacing
\setlength{\parskip}{3pt}
\setlist[itemize]{nosep}
\setlist[enumerate]{nosep}

% Title formatting
\titleformat{\section}{\Large\bfseries\color{headcolor}}{\thesection}{1em}{}
\titleformat{\subsection}{\large\bfseries\color{headcolor}}{\thesubsection}{1em}{}

% Pandoc tightlist compatibility
\providecommand{\tightlist}{%
  \setlength{\itemsep}{0pt}\setlength{\parskip}{0pt}}

% Pandoc longtable compatibility
\newcounter{none}
\def\thenone{}


% content/resources/templates/gujarati-boxes.tex
\usepackage{fontspec}
\usepackage{polyglossia}

% Set Gujarati as main language (document is primarily in Gujarati)
% Note: gloss-gujarati.ldf doesn't exist in polyglossia, but it will use hyphenation patterns
\setdefaultlanguage{gujarati}
\setotherlanguage{english}

% Configure Gujarati font properly
% Use Language=Default to prevent polyglossia from trying to add language-specific features
% that don't exist for Gujarati, which causes "empty feature" warnings
\newfontfamily\gujaratifont[Script=Gujarati,AutoFakeBold=2.5,AutoFakeSlant=0.3]{Noto Sans Gujarati}
\setmainfont[Script=Gujarati,AutoFakeBold=2.5,AutoFakeSlant=0.3]{Noto Sans Gujarati}
% Use Noto Sans Gujarati for monospace to support Gujarati in text
\setmonofont[Scale=0.9]{Noto Sans Gujarati}

% Configure English to use the same font
\newfontfamily\englishfont[Script=Gujarati,AutoFakeBold=2.5,AutoFakeSlant=0.3]{Noto Sans Gujarati}

% Translations for polyglossia
\gappto\captionsgujarati{
  \renewcommand{\tablename}{કોષ્ટક}
  \renewcommand{\figurename}{આકૃતિ}
}

% Helper for TikZ nodes to ensure Gujarati font
\newcommand{\gu}[1]{{\gujaratifont #1}}

% Custom environments
\newtcolorbox{solutionbox}{
    breakable,
    enhanced,
    colback=solutioncolor!5!white,
    colframe=solutioncolor!75!black,
    fonttitle=\bfseries,
    title=જવાબ
}

\newtcolorbox{solutionboxnobreak}{
 colback=solutioncolor!5!white,
 colframe=solutioncolor!75!black,
 fonttitle=\bfseries,
 title=જવાબ
}

\newtcolorbox{keyformula}{
 breakable,
 enhanced,
 colback=keycolor!5!white,
 colframe=keycolor!75!black,
 fonttitle=\bfseries,
 title=રાસાયણિક સમીકરણ/સૂત્ર
}

\newtcolorbox{mnemonicbox}{
 breakable,
 enhanced,
 colback=mnemoniccolor!5!white,
 colframe=mnemoniccolor!75!black,
 fonttitle=\bfseries,
 title=મેમરી ટ્રીક
}


\begin{document}

\begin{center}
{\Huge\bfseries\color{headcolor} Subject Name (Gujarati)}\\[5pt]
{\LARGE 4361102 -- Summer 2024}\\[3pt]
{\large Semester 1 Study Material}\\[3pt]
{\normalsize\textit{Detailed Solutions and Explanations}}
\end{center}

\vspace{10pt}

\subsection*{પ્રશ્ન 1(અ) [3
ગુણ]}\label{uxaaauxab0uxab6uxaa8-1uxa85-3-uxa97uxaa3}

\textbf{FinFET ની રચના દોરો અને તેના ફાયદા લખો.}

\begin{solutionbox}

\begin{center}
\textbf{Mermaid Diagram (Code)}
\begin{verbatim}
{Shaded}
{Highlighting}[]
graph LR
    A[Source] {-{-}{} B[Gate]}
    B {-{-}{} C[Drain]}
    D[Fin Structure] {-{-}{} E[Multiple Gates]}
    F[Silicon Substrate] {-{-}{} D}
{Highlighting}
{Shaded}
\end{verbatim}
\end{center}


{\def\LTcaptype{none} % do not increment counter
\vspace{-5pt}
\captionof{table}{FinFET ના ફાયદા}
\vspace{-10pt}
\begin{longtable}[]{@{}ll@{}}
\toprule\noalign{}
ફાયદો & વર્ણન \\
\midrule\noalign{}
\endhead
\bottomrule\noalign{}
\endlastfoot
\textbf{બેહતર નિયંત્રણ} & ગુણાકાર gates બહેતર channel control આપે છે \\
\textbf{ઘટાડેલ લીકેજ} & 3D રચનાના કારણે ઓછું off-state current \\
\textbf{સુધારેલ કામગીરી} & વધુ drive current અને ઝડપી switching \\
\end{longtable}
}

\textbf{યાદાશ્ત ટેકનિક:} ``BCR - Better Control Reduces leakage''

\end{solutionbox}
\begin{center}\rule{0.5\linewidth}{0.5pt}\end{center}

\subsection*{પ્રશ્ન 1(બ) [4
ગુણ]}\label{uxaaauxab0uxab6uxaa8-1uxaac-4-uxa97uxaa3}

\textbf{એક્સટર્નલ બાયઝ હેઠળ MOS રચનાનું ડેપ્લીશન અને ઇનવર્શન સમજાવો}

\begin{solutionbox}


{\def\LTcaptype{none} % do not increment counter
\vspace{-5pt}
\captionof{table}{MOS બાયઝ પરિસ્થિતિઓ}
\vspace{-10pt}
\begin{longtable}[]{@{}llll@{}}
\toprule\noalign{}
બાયઝ પ્રકાર & ગેટ વોલ્ટેજ & ચેનલ સ્થિતિ & ચાર્જ કેરિયર્સ \\
\midrule\noalign{}
\endhead
\bottomrule\noalign{}
\endlastfoot
\textbf{ડેપ્લીશન} & થોડું પોઝિટિવ & Depleted & Holes દૂર ધકેલાય છે \\
\textbf{ઇનવર્શન} & વધુ પોઝિટિવ & Inverted & Electrons આકર્ષાય છે \\
\end{longtable}
}

\textbf{ડાયાગ્રામ:}

\begin{verbatim}
VG { 0 (Depletion)        VG  0 (Inversion)}
    +                         +
   Gate                      Gate
   {-{-}{-}{-}                      {-}{-}{-}{-}}
    {-                         {-}}
   Depletion                Electron
   Region                   Channel
   {-{-}{-}{-}{-}{-}{-}{-}                 {-}{-}{-}{-}{-}{-}{-}{-}}
   p{-substrate              p{-}substrate}
\end{verbatim}

\begin{itemize}
\tightlist
\item
  \textbf{ડેપ્લીશન}: પોઝિટિવ ગેટ વોલ્ટેજ electric field બનાવે છે જે holes ને દૂર
  ધકેલે છે
\item
  \textbf{ઇનવર્શન}: વધુ વોલ્ટેજ electrons ને આકર્ષે છે અને conducting channel
  બનાવે છે
\end{itemize}

\textbf{યાદાશ્ત ટેકનિક:} ``DI - Depletion Inverts to conducting channel''

\end{solutionbox}
\begin{center}\rule{0.5\linewidth}{0.5pt}\end{center}

\subsection*{પ્રશ્ન 1(ક) [7
ગુણ]}\label{uxaaauxab0uxab6uxaa8-1uxa95-7-uxa97uxaa3}

\textbf{n-ચેનલ MOSFET ને તેની કરંટ-વોલ્ટેજ લાક્ષણિકતાઓની મદદથી સમજાવો.}

\begin{solutionbox}


{\def\LTcaptype{none} % do not increment counter
\vspace{-5pt}
\captionof{table}{MOSFET ઓપરેટિંગ વિભાગો}
\vspace{-10pt}
\begin{longtable}[]{@{}llll@{}}
\toprule\noalign{}
વિભાગ & શરત & ડ્રેઇન કરંટ & લાક્ષણિકતાઓ \\
\midrule\noalign{}
\endhead
\bottomrule\noalign{}
\endlastfoot
\textbf{કટ-ઓફ} & VGS \textless{} VTH & ID \approx 0 & કોઈ conduction નથી \\
\textbf{લિનિયર} & VDS \textless{} VGS-VTH & ID ∝ VDS & Resistive વર્તન \\
\textbf{સેચ્યુરેશન} & VDS \geq VGS-VTH & ID ∝ (VGS-VTH)^{2} & કરંટ VDS પર આધારિત
નથી \\
\end{longtable}
}

\begin{center}
\textbf{Mermaid Diagram (Code)}
\begin{verbatim}
{Shaded}
{Highlighting}[]
graph LR
    A[Gate] {-{-}{} B[n{-}channel]}
    C[Source] {-{-}{} B}
    B {-{-}{} D[Drain]}
    E[p{-substrate] {-}{-}{} B}
{Highlighting}
{Shaded}
\end{verbatim}
\end{center}

\textbf{મુખ્ય સમીકરણો:}

\begin{itemize}
\item
  Linear: ID = μnCox(W/L)[(VGS-VTH)VDS - VDS^{2}/2]
\item
  Saturation: ID = (μnCox/2)(W/L)(VGS-VTH)^{2}
\item
  \textbf{રચના}: ગેટ source અને drain વચ્ચે channel ને નિયંત્રિત કરે છે
\item
  \textbf{કામગીરી}: ગેટ વોલ્ટેજ channel conductivity ને modulate કરે છે
\item
  \textbf{ઉપયોગો}: ડિજિટલ switching અને analog amplification
\end{itemize}

\textbf{યાદાશ્ત ટેકનિક:} ``CLS - Cut-off, Linear, Saturation regions''

\end{solutionbox}
\begin{center}\rule{0.5\linewidth}{0.5pt}\end{center}

\subsection*{પ્રશ્ન 1(ક OR) [7
ગુણ]}\label{uxaaauxab0uxab6uxaa8-1uxa95-or-7-uxa97uxaa3}

\textbf{સ્કેલિંગ વ્યાખ્યાયિત કરો. full voltage સ્કેલિંગ સાથે constant voltage
સ્કેલિંગની તુલના કરો. સ્કેલિંગના ગેરફાયદા લખો.}

\begin{solutionbox}

\textbf{વ્યાખ્યા:} સ્કેલિંગ એ ડેન્સિટી અને performance વધારવા માટે device
dimensions ઘટાડવાની પ્રક્રિયા છે.


{\def\LTcaptype{none} % do not increment counter
\vspace{-5pt}
\captionof{table}{સ્કેલિંગ તુલના}
\vspace{-10pt}
\begin{longtable}[]{@{}lll@{}}
\toprule\noalign{}
પેરામીટર & Full Voltage Scaling & Constant Voltage Scaling \\
\midrule\noalign{}
\endhead
\bottomrule\noalign{}
\endlastfoot
\textbf{વોલ્ટેજ} & α દ્વારા ઘટાડાય છે & સ્થિર રહે છે \\
\textbf{પાવર ડેન્સિટી} & સ્થિર & α દ્વારા વધે છે \\
\textbf{ઇલેક્ટ્રિક ફિલ્ડ} & સ્થિર & α દ્વારા વધે છે \\
\textbf{પરફોર્મન્સ} & બેહતર & મધ્યમ સુધારો \\
\end{longtable}
}

\textbf{ગેરફાયદા:}

\begin{itemize}
\tightlist
\item
  \textbf{શોર્ટ ચેનલ ઇફેક્ટ્સ}: ચેનલ લેન્થ modulation વધે છે
\item
  \textbf{હોટ કેરિયર ઇફેક્ટ્સ}: વધુ electric fields devices ને નુકસાન કરે છે
\item
  \textbf{ક્વોન્ટમ ઇફેક્ટ્સ}: ટનલિંગ currents નોંધપાત્ર રીતે વધે છે
\end{itemize}

\textbf{યાદાશ્ત ટેકનિક:} ``SHQ - Short channel, Hot carriers, Quantum
effects''

\end{solutionbox}
\begin{center}\rule{0.5\linewidth}{0.5pt}\end{center}

\subsection*{પ્રશ્ન 2(અ) [3
ગુણ]}\label{uxaaauxab0uxab6uxaa8-2uxa85-3-uxa97uxaa3}

\textbf{CMOS ની મદદથી બે ઇનપુટ NAND ગેટ દોરો.}

\begin{solutionbox}

\begin{verbatim}
    VDD
     |
   ┌─┴─┐ pMOS
A──┤   ├──┐
   └───┘  │
          │ Y
   ┌─────┐ │
B──┤     ├─┘
   └─┬─┬─┘ pMOS
     │ │
   ┌─┴─┐ nMOS
A──┤   ├──┐
   └───┘  │
          │
   ┌─────┐ │
B──┤     ├─┘
   └─┬─┬─┘ nMOS
     │ │
    GND
\end{verbatim}


{\def\LTcaptype{none} % do not increment counter
\vspace{-5pt}
\captionof{table}{NAND સત્ય કોષ્ટક}
\vspace{-10pt}
\begin{longtable}[]{@{}lll@{}}
\toprule\noalign{}
A & B & Y \\
\midrule\noalign{}
\endhead
\bottomrule\noalign{}
\endlastfoot
0 & 0 & 1 \\
0 & 1 & 1 \\
1 & 0 & 1 \\
1 & 1 & 0 \\
\end{longtable}
}

\textbf{યાદાશ્ત ટેકનિક:} ``PP-SS: Parallel PMOS, Series NMOS''

\end{solutionbox}
\begin{center}\rule{0.5\linewidth}{0.5pt}\end{center}

\subsection*{પ્રશ્ન 2(બ) [4
ગુણ]}\label{uxaaauxab0uxab6uxaa8-2uxaac-4-uxa97uxaa3}

\textbf{nMOS ઇન્વર્ટર માટે નોઇઝ ઇમ્યુનિટી અને નોઇઝ માર્જિન સમજાવો.}

\begin{solutionbox}


{\def\LTcaptype{none} % do not increment counter
\vspace{-5pt}
\captionof{table}{નોઇઝ પેરામીટર્સ}
\vspace{-10pt}
\begin{longtable}[]{@{}lll@{}}
\toprule\noalign{}
પેરામીટર & વ્યાખ્યા & ફોર્મ્યુલા \\
\midrule\noalign{}
\endhead
\bottomrule\noalign{}
\endlastfoot
\textbf{NMH} & હાઇ નોઇઝ માર્જિન & VOH - VIH \\
\textbf{NML} & લો નોઇઝ માર્જિન & VIL - VOL \\
\textbf{નોઇઝ ઇમ્યુનિટી} & નોઇઝ રિજેક્ટ કરવાની ક્ષમતા & Min(NMH, NML) \\
\end{longtable}
}

\begin{center}
\textbf{Mermaid Diagram (Code)}
\begin{verbatim}
{Shaded}
{Highlighting}[]
graph TD
    A[VIL] {-{-}{} B[VIH]}
    C[VOL] {-{-}{} D[VOH]}
    E[NML] {-{-}{} F[NMH]}
{Highlighting}
{Shaded}
\end{verbatim}
\end{center}

\begin{itemize}
\tightlist
\item
  \textbf{VIL}: મહત્તમ લો ઇનપુટ વોલ્ટેજ
\item
  \textbf{VIH}: લઘુત્તમ હાઇ ઇનપુટ વોલ્ટેજ
\item
  \textbf{સારી નોઇઝ ઇમ્યુનિટી}: મોટા નોઇઝ માર્જિન ખોટી switching ને રોકે છે
\end{itemize}

\textbf{યાદાશ્ત ટેકનિક:} ``HILOL - High/Low Input/Output Levels''

\end{solutionbox}
\begin{center}\rule{0.5\linewidth}{0.5pt}\end{center}

\subsection*{પ્રશ્ન 2(ક) [7
ગુણ]}\label{uxaaauxab0uxab6uxaa8-2uxa95-7-uxa97uxaa3}

\textbf{CMOS ઇન્વર્ટરની વોલ્ટેજ ટ્રાન્સફર લાક્ષણિકતાઓ (VTC) સમજાવો.}

\begin{solutionbox}


{\def\LTcaptype{none} % do not increment counter
\vspace{-5pt}
\captionof{table}{VTC વિભાગો}
\vspace{-10pt}
\begin{longtable}[]{@{}llll@{}}
\toprule\noalign{}
વિભાગ & ઇનપુટ રેન્જ & આઉટપુટ & ટ્રાન્ઝિસ્ટર સ્થિતિઓ \\
\midrule\noalign{}
\endhead
\bottomrule\noalign{}
\endlastfoot
\textbf{A} & 0 to VTN & VDD & pMOS ON, nMOS OFF \\
\textbf{B} & VTN to VDD/2 & ટ્રાન્ઝિશન & બંને આંશિક રીતે ON \\
\textbf{C} & VDD/2 to VDD-\textbar VTP\textbar{} & ટ્રાન્ઝિશન & બંને આંશિક
રીતે ON \\
\textbf{D} & VDD-\textbar VTP\textbar{} to VDD & 0V & pMOS OFF, nMOS
ON \\
\end{longtable}
}

\begin{center}
\textbf{Mermaid Diagram (Code)}
\begin{verbatim}
{Shaded}
{Highlighting}[]
graph TD
    A[VIN = 0] {-{-}{} B[VOUT = VDD]}
    C[VIN = VDD/2] {-{-}{} D[VOUT = VDD/2]}
    E[VIN = VDD] {-{-}{} F[VOUT = 0]}
{Highlighting}
{Shaded}
\end{verbatim}
\end{center}

\textbf{મુખ્ય લક્ષણો:}

\begin{itemize}
\tightlist
\item
  \textbf{તીક્ષ્ણ ટ્રાન્ઝિશન}: આદર્શ switching વર્તન
\item
  \textbf{હાઇ ગેઇન}: ટ્રાન્ઝિશન વિભાગમાં મોટો slope
\item
  \textbf{રેઇલ-ટુ-રેઇલ}: આઉટપુટ સંપૂર્ણ સપ્લાય રેન્જમાં swing કરે છે
\end{itemize}

\textbf{યાદાશ્ત ટેકનિક:} ``ASH - A-region, Sharp transition, High gain''

\end{solutionbox}
\begin{center}\rule{0.5\linewidth}{0.5pt}\end{center}

\subsection*{પ્રશ્ન 2(અ OR) [3
ગુણ]}\label{uxaaauxab0uxab6uxaa8-2uxa85-or-3-uxa97uxaa3}

\textbf{ડિપ્લીશન લોડ nMOS નો ઉપયોગ કરીને NOR2 ગેટનો અમલ કરો.}

\begin{solutionbox}

\begin{verbatim}
    VDD
     |
   ┌─┴─┐ Depletion Load
   │   │ (VGS = 0)
   └─┬─┘
     │ Y
   ┌─┴─┐ nMOS
A──┤   ├──┐
   └───┘  │
          │
   ┌─────┐ │
B──┤     ├─┘
   └─┬─┬─┘ nMOS
     │ │
    GND
\end{verbatim}


{\def\LTcaptype{none} % do not increment counter
\vspace{-5pt}
\captionof{table}{NOR2 સત્ય કોષ્ટક}
\vspace{-10pt}
\begin{longtable}[]{@{}lll@{}}
\toprule\noalign{}
A & B & Y \\
\midrule\noalign{}
\endhead
\bottomrule\noalign{}
\endlastfoot
0 & 0 & 1 \\
0 & 1 & 0 \\
1 & 0 & 0 \\
1 & 1 & 0 \\
\end{longtable}
}

\textbf{યાદાશ્ત ટેકનિક:} ``DPN - Depletion load, Parallel NMOS''

\end{solutionbox}
\begin{center}\rule{0.5\linewidth}{0.5pt}\end{center}

\subsection*{પ્રશ્ન 2(બ OR) [4
ગુણ]}\label{uxaaauxab0uxab6uxaa8-2uxaac-or-4-uxa97uxaa3}

\textbf{એન્હાન્સમેન્ટ લોડ ઇન્વર્ટર અને ડિપ્લીશન લોડ ઇન્વર્ટર વચ્ચે તફાવત શોધો.}

\begin{solutionbox}


{\def\LTcaptype{none} % do not increment counter
\vspace{-5pt}
\captionof{table}{લોડ ઇન્વર્ટર તુલના}
\vspace{-10pt}
\begin{longtable}[]{@{}lll@{}}
\toprule\noalign{}
પેરામીટર & એન્હાન્સમેન્ટ લોડ & ડિપ્લીશન લોડ \\
\midrule\noalign{}
\endhead
\bottomrule\noalign{}
\endlastfoot
\textbf{થ્રેશોલ્ડ વોલ્ટેજ} & VT \textgreater{} 0 & VT \textless{} 0 \\
\textbf{ગેટ કનેક્શન} & VGS = VDS & VGS = 0 \\
\textbf{લોજિક હાઇ} & VDD - VT & VDD \\
\textbf{પાવર કન્ઝમ્પશન} & વધુ & ઓછું \\
\textbf{સ્વિચિંગ સ્પીડ} & ધીમું & ઝડપી \\
\end{longtable}
}

\begin{itemize}
\tightlist
\item
  \textbf{એન્હાન્સમેન્ટ}: conduction માટે પોઝિટિવ ગેટ વોલ્ટેજની જરૂર
\item
  \textbf{ડિપ્લીશન}: ઝીરો ગેટ વોલ્ટેજ સાથે conduct કરે છે
\item
  \textbf{પરફોર્મન્સ}: ડિપ્લીશન લોડ બેહતર લાક્ષણિકતાઓ આપે છે
\end{itemize}

\textbf{યાદાશ્ત ટેકનિક:} ``EPDLH - Enhancement Positive, Depletion Lower
power, Higher speed''

\end{solutionbox}
\begin{center}\rule{0.5\linewidth}{0.5pt}\end{center}

\subsection*{પ્રશ્ન 2(ક OR) [7
ગુણ]}\label{uxaaauxab0uxab6uxaa8-2uxa95-or-7-uxa97uxaa3}

\textbf{ડિપ્લીશન લોડ nMOS ઇન્વર્ટરને તેના VTC સાથે સમજાવો.}

\begin{solutionbox}

\textbf{સર્કિટ ઓપરેશન:}

\begin{itemize}
\tightlist
\item
  \textbf{લોડ ટ્રાન્ઝિસ્ટર}: હંમેશા conducting (VGS = 0, VT \textless{} 0)
\item
  \textbf{ડ્રાઇવર ટ્રાન્ઝિસ્ટર}: ઇનપુટ વોલ્ટેજ દ્વારા નિયંત્રિત
\item
  \textbf{આઉટપુટ}: વોલ્ટેજ ડિવાઇડર એક્શન દ્વારા નક્કી થાય છે
\end{itemize}

\begin{center}
\textbf{Mermaid Diagram (Code)}
\begin{verbatim}
{Shaded}
{Highlighting}[]
graph LR
    A[VIN Low] {-{-}{} B[Driver OFF]}
    B {-{-}{} C[VOUT = VDD]}
    D[VIN High] {-{-}{} E[Driver ON]}
    E {-{-}{} F[VOUT  0V]}
{Highlighting}
{Shaded}
\end{verbatim}
\end{center}


{\def\LTcaptype{none} % do not increment counter
\vspace{-5pt}
\captionof{table}{ઓપરેટિંગ પોઇન્ટ્સ}
\vspace{-10pt}
\begin{longtable}[]{@{}llll@{}}
\toprule\noalign{}
ઇનપુટ સ્થિતિ & ડ્રાઇવર & લોડ & આઉટપુટ \\
\midrule\noalign{}
\endhead
\bottomrule\noalign{}
\endlastfoot
\textbf{VIN = 0} & OFF & ON & VDD \\
\textbf{VIN = VDD} & ON & ON & \approx 0V \\
\end{longtable}
}

\textbf{VTC લાક્ષણિકતાઓ:}

\begin{itemize}
\tightlist
\item
  \textbf{VOH}: VDD (એન્હાન્સમેન્ટ લોડ કરતાં બેહતર)
\item
  \textbf{VOL}: ડિપ્લીશન લોડ લાક્ષણિકતાઓના કારણે ઓછું
\item
  \textbf{ટ્રાન્ઝિશન}: સ્થિતિઓ વચ્ચે તીક્ષ્ણ switching
\end{itemize}

\textbf{યાદાશ્ત ટેકનિક:} ``DLB - Depletion Load gives Better high output''

\end{solutionbox}
\begin{center}\rule{0.5\linewidth}{0.5pt}\end{center}

\subsection*{પ્રશ્ન 3(અ) [3
ગુણ]}\label{uxaaauxab0uxab6uxaa8-3uxa85-3-uxa97uxaa3}

\textbf{ડિપ્લીશન લોડ nMOS નો ઉપયોગ કરીને EX-OR નો અમલ કરો.}

\begin{solutionbox}

\begin{verbatim}
    VDD      VDD
     |        |
   ┌─┴─┐    ┌─┴─┐  Depletion
   │   │    │   │  Loads
   └─┬─┘    └─┬─┘
     │        │
A────┼────────┼────B
     │        │
   ┌─┴─┐    ┌─┴─┐
   │   │    │   │  nMOS
   └─┬─┘    └─┬─┘
     │        │
    A{       B}
     │        │
     └────┬───┘ Y
          │
         GND
\end{verbatim}


{\def\LTcaptype{none} % do not increment counter
\vspace{-5pt}
\captionof{table}{XOR સત્ય કોષ્ટક}
\vspace{-10pt}
\begin{longtable}[]{@{}lll@{}}
\toprule\noalign{}
A & B & Y \\
\midrule\noalign{}
\endhead
\bottomrule\noalign{}
\endlastfoot
0 & 0 & 0 \\
0 & 1 & 1 \\
1 & 0 & 1 \\
1 & 1 & 0 \\
\end{longtable}
}

\textbf{અમલીકરણ}: Y = A\oplusB = A'B + AB'

\textbf{યાદાશ્ત ટેકનિક:} ``XOR - eXclusive OR, અલગ inputs 1 આપે છે''

\end{solutionbox}
\begin{center}\rule{0.5\linewidth}{0.5pt}\end{center}

\subsection*{પ્રશ્ન 3(બ) [4
ગુણ]}\label{uxaaauxab0uxab6uxaa8-3uxaac-4-uxa97uxaa3}

\textbf{ડિઝાઇન હાઇરાર્કીને ઉદાહરણ સાથે સમજાવો.}

\begin{solutionbox}


{\def\LTcaptype{none} % do not increment counter
\vspace{-5pt}
\captionof{table}{હાઇરાર્કી લેવલ્સ}
\vspace{-10pt}
\begin{longtable}[]{@{}lll@{}}
\toprule\noalign{}
લેવલ & કમ્પોનન્ટ & ઉદાહરણ \\
\midrule\noalign{}
\endhead
\bottomrule\noalign{}
\endlastfoot
\textbf{સિસ્ટમ} & સંપૂર્ણ ચિપ & માઇક્રોપ્રોસેસર \\
\textbf{મોડ્યુલ} & ફંક્શનલ બ્લોક્સ & ALU, મેમરી \\
\textbf{ગેટ} & લોજિક ગેટ્સ & NAND, NOR \\
\textbf{ટ્રાન્ઝિસ્ટર} & વ્યક્તિગત ડિવાઇસેસ & MOSFET \\
\end{longtable}
}

\begin{center}
\textbf{Mermaid Diagram (Code)}
\begin{verbatim}
{Shaded}
{Highlighting}[]
graph LR
    A[સિસ્ટમ લેવલ] {-{-}{} B[મોડ્યુલ લેવલ]}
    B {-{-}{} C[ગેટ લેવલ]}
    C {-{-}{} D[ટ્રાન્ઝિસ્ટર લેવલ]}
    E[CPU] {-{-}{} F[ALU]}
    F {-{-}{} G[Adder]}
    G {-{-}{} H[MOSFET]}
{Highlighting}
{Shaded}
\end{verbatim}
\end{center}

\textbf{ફાયદા:}

\begin{itemize}
\tightlist
\item
  \textbf{મોડ્યુલારિટી}: સ્વતંત્ર ડિઝાઇન અને ટેસ્ટિંગ
\item
  \textbf{પુનઃઉપયોગ}: સામાન્ય બ્લોક્સ ઘણી વખત વપરાય છે
\item
  \textbf{જાળવણીયોગ્યતા}: સરળ debugging અને modification
\end{itemize}

\textbf{યાદાશ્ત ટેકનિક:} ``SMG-T: System, Module, Gate, Transistor
levels''

\end{solutionbox}
\begin{center}\rule{0.5\linewidth}{0.5pt}\end{center}

\subsection*{પ્રશ્ન 3(ક) [7
ગુણ]}\label{uxaaauxab0uxab6uxaa8-3uxa95-7-uxa97uxaa3}

\textbf{Y ચાર્ટ ડિઝાઇન ફ્લો દોરો અને સમજાવો.}

\begin{solutionbox}

\begin{center}
\textbf{Mermaid Diagram (Code)}
\begin{verbatim}
{Shaded}
{Highlighting}[]
graph LR
    A[બિહેવિયરલ ડોમેન] {-{-}{} D[સિસ્ટમ સ્પેસિફિકેશન]}
    B[સ્ટ્રક્ચરલ ડોમેન] {-{-}{} E[આર્કિટેક્ચર]}
    C[ફિઝિકલ ડોમેન] {-{-}{} F[ફ્લોર પ્લાન]}
    D {-{-}{} G[એલ્ગોરિધમ]}
    E {-{-}{} H[લોજિક ડિઝાઇન]}
    F {-{-}{} I[લેઆઉટ]}
    G {-{-}{} J[RTL]}
    H {-{-}{} K[ગેટ લેવલ]}
    I {-{-}{} L[ટ્રાન્ઝિસ્ટર લેવલ]}
{Highlighting}
{Shaded}
\end{verbatim}
\end{center}


{\def\LTcaptype{none} % do not increment counter
\vspace{-5pt}
\captionof{table}{Y-ચાર્ટ ડોમેન્સ}
\vspace{-10pt}
\begin{longtable}[]{@{}lll@{}}
\toprule\noalign{}
ડોમેન & વર્ણન & ઉદાહરણો \\
\midrule\noalign{}
\endhead
\bottomrule\noalign{}
\endlastfoot
\textbf{બિહેવિયરલ} & સિસ્ટમ શું કરે છે & એલ્ગોરિધમ્સ, RTL \\
\textbf{સ્ટ્રક્ચરલ} & તે કેવી રીતે ગોઠવાયેલું છે & આર્કિટેક્ચર, ગેટ્સ \\
\textbf{ફિઝિકલ} & કમ્પોનન્ટ્સ ક્યાં મૂકાયેલા છે & ફ્લોરપ્લાન, લેઆઉટ \\
\end{longtable}
}

\textbf{ડિઝાઇન ફ્લો:}

\begin{itemize}
\tightlist
\item
  \textbf{ટોપ-ડાઉન}: બિહેવિયરલ \rightarrow સ્ટ્રક્ચરલ \rightarrow ફિઝિકલ
\item
  \textbf{બોટમ-અપ}: ફિઝિકલ constraints ઉપરના લેવલ્સને પ્રભાવિત કરે છે
\item
  \textbf{પુનરાવર્તી}: ઓપ્ટિમાઇઝેશન માટે બહુવિધ passes
\end{itemize}

\textbf{યાદાશ્ત ટેકનિક:} ``BSP - Behavioral, Structural, Physical
domains''

\end{solutionbox}
\begin{center}\rule{0.5\linewidth}{0.5pt}\end{center}

\subsection*{પ્રશ્ન 3(અ OR) [3
ગુણ]}\label{uxaaauxab0uxab6uxaa8-3uxa85-or-3-uxa97uxaa3}

\textbf{CMOS નો ઉપયોગ કરીને NAND2 - SR લેચનો અમલ કરો}

\begin{solutionbox}

\begin{verbatim}
    S ────┐   ┌──── Q
          │   │
        ┌─┴─┐ │ ┌─┴─┐
        │   ├─┘ │   │ NAND
        └─┬─┘   └─┬─┘
          │       │
          └───┬───┘
              │
        ┌─────┴─────┐
        │           │
      ┌─┴─┐       ┌─┴─┐
      │   ├───────┤   │ NAND
      └─┬─┘       └─┬─┘
        │           │
        R ──────────┘
                    │
                   Q{}
\end{verbatim}


{\def\LTcaptype{none} % do not increment counter
\vspace{-5pt}
\captionof{table}{SR લેચ ઓપરેશન}
\vspace{-10pt}
\begin{longtable}[]{@{}lllll@{}}
\toprule\noalign{}
S & R & Q & Q' & સ્થિતિ \\
\midrule\noalign{}
\endhead
\bottomrule\noalign{}
\endlastfoot
0 & 0 & Q & Q' & હોલ્ડ \\
0 & 1 & 0 & 1 & રીસેટ \\
1 & 0 & 1 & 0 & સેટ \\
1 & 1 & 1 & 1 & અમાન્ય \\
\end{longtable}
}

\textbf{યાદાશ્ત ટેકનિક:} ``SR-HRI: Set, Reset, Hold, Invalid states''

\end{solutionbox}
\begin{center}\rule{0.5\linewidth}{0.5pt}\end{center}

\subsection*{પ્રશ્ન 3(બ OR) [4
ગુણ]}\label{uxaaauxab0uxab6uxaa8-3uxaac-or-4-uxa97uxaa3}

\textbf{સિલિકોન વેફર પર પેટર્ન અથવા માસ્ક ટ્રાન્સફર કરવા માટે કઈ પદ્ધતિનો ઉપયોગ
થાય છે? તેને સ્વચ્છ આકૃતિઓ સાથે સમજાવો.}

\begin{solutionbox}

\textbf{પદ્ધતિ}: \textbf{લિથોગ્રાફી} - પ્રકાશ એક્સપોઝર વાપરીને પેટર્ન ટ્રાન્સફર

\begin{center}
\textbf{Mermaid Diagram (Code)}
\begin{verbatim}
{Shaded}
{Highlighting}[]
graph LR
    A[UV પ્રકાશ સ્ત્રોત] {-{-}{} B[પેટર્ન સાથે માસ્ક]}
    B {-{-}{} C[વેફર પર ફોટોરેસિસ્ટ]}
    C {-{-}{} D[એક્સપોઝ્ડ પેટર્ન]}
    D {-{-}{} E[ડેવલપ્ડ પેટર્ન]}
{Highlighting}
{Shaded}
\end{verbatim}
\end{center}

\textbf{પ્રક્રિયાના પગલાં:}

{\def\LTcaptype{none} % do not increment counter
\begin{longtable}[]{@{}lll@{}}
\toprule\noalign{}
પગલું & ક્રિયા & પરિણામ \\
\midrule\noalign{}
\endhead
\bottomrule\noalign{}
\endlastfoot
\textbf{કોટિંગ} & ફોટોરેસિસ્ટ લગાવો & સમાન સ્તર \\
\textbf{એક્સપોઝર} & માસ્ક દ્વારા UV & રાસાયણિક ફેરફાર \\
\textbf{ડેવલપમેન્ટ} & એક્સપોઝ્ડ રેસિસ્ટ દૂર કરો & પેટર્ન ટ્રાન્સફર \\
\end{longtable}
}

\textbf{ઉપયોગો}: ગેટ્સ, interconnects, contact holes બનાવવા

\textbf{યાદાશ્ત ટેકનિક:} ``CED - Coating, Exposure, Development''

\end{solutionbox}
\begin{center}\rule{0.5\linewidth}{0.5pt}\end{center}

\subsection*{પ્રશ્ન 3(ક OR) [7
ગુણ]}\label{uxaaauxab0uxab6uxaa8-3uxa95-or-7-uxa97uxaa3}

\textbf{MOSFET ફેબ્રિકેશનમાં મેટલ deposit કરવા માટે કઈ પદ્ધતિઓનો ઉપયોગ થાય છે?
યોગ્ય ડાયાગ્રામ સાથે ડિપોઝિશનને વિગતવાર સમજાવો.}

\begin{solutionbox}


{\def\LTcaptype{none} % do not increment counter
\vspace{-5pt}
\captionof{table}{મેટલ ડિપોઝિશન પદ્ધતિઓ}
\vspace{-10pt}
\begin{longtable}[]{@{}lll@{}}
\toprule\noalign{}
પદ્ધતિ & ટેકનિક & ઉપયોગ \\
\midrule\noalign{}
\endhead
\bottomrule\noalign{}
\endlastfoot
\textbf{ફિઝિકલ વેપર ડિપોઝિશન} & Sputtering, Evaporation & એલ્યુમિનિયમ,
કોપર \\
\textbf{કેમિકલ વેપર ડિપોઝિશન} & CVD, PECVD & ટંગસ્ટન, ટાઇટેનિયમ \\
\textbf{ઇલેક્ટ્રોપ્લેટિંગ} & ઇલેક્ટ્રોકેમિકલ & કોપર interconnects \\
\end{longtable}
}

\begin{center}
\textbf{Mermaid Diagram (Code)}
\begin{verbatim}
{Shaded}
{Highlighting}[]
graph LR
    A[ટાર્ગેટ મટેરિયલ] {-{-}{} B[આયન બોમ્બાર્ડમેન્ટ]}
    B {-{-}{} C[બહાર નીકળેલા એટમ્સ]}
    C {-{-}{} D[સબસ્ટ્રેટ કોટિંગ]}
    E[વેફર] {-{-}{} D}
{Highlighting}
{Shaded}
\end{verbatim}
\end{center}

\textbf{સ્પટરિંગ પ્રક્રિયા:}

\begin{itemize}
\tightlist
\item
  \textbf{આયન બોમ્બાર્ડમેન્ટ}: આર્ગોન આયન્સ ટાર્ગેટ મટેરિયલને અથડાવે છે
\item
  \textbf{એટમ ઇજેક્શન}: ટાર્ગેટ એટમ્સ બહાર નીકળે છે
\item
  \textbf{ડિપોઝિશન}: એટમ્સ વેફર સપાટી પર સ્થિર થાય છે
\item
  \textbf{નિયંત્રણ}: દબાણ અને પાવર દર નક્કી કરે છે
\end{itemize}

\textbf{ફાયદા:}

\begin{itemize}
\tightlist
\item
  \textbf{સમાન જાડાઈ}: ઉત્તમ સ્ટેપ કવરેજ
\item
  \textbf{ઓછું તાપમાન}: ડિવાઇસ integrity જાળવે છે
\item
  \textbf{વિવિધતા}: બહુવિધ મટેરિયલ્સ શક્ય
\end{itemize}

\textbf{યાદાશ્ત ટેકનિક:} ``IBE-DC: Ion Bombardment Ejects atoms for
Deposition Control''

\end{solutionbox}
\begin{center}\rule{0.5\linewidth}{0.5pt}\end{center}

\subsection*{પ્રશ્ન 4(અ) [3
ગુણ]}\label{uxaaauxab0uxab6uxaa8-4uxa85-3-uxa97uxaa3}

\textbf{ડિપ્લીશન nMOS લોડ સાથે Z= ((A+B+C)·(D+E+F). G)' અમલમાં મૂકો.}

\begin{solutionbox}

\begin{verbatim}
    VDD
     |
   ┌─┴─┐ Depletion Load
   │   │
   └─┬─┘
     │ Z
A────┼────┐
B────┼────┤ Parallel
C────┼────┘ (OR)
     │
D────┼────┐
E────┼────┤ Parallel  
F────┼────┘ (OR)
     │
G────┼────┘ Series
     │      (AND)
    GND
\end{verbatim}

\textbf{લોજિક અમલીકરણ:}

\begin{itemize}
\tightlist
\item
  \textbf{પ્રથમ સ્તર}: (A+B+C) અને (D+E+F) OR ફંક્શન્સ
\item
  \textbf{બીજું સ્તર}: G સાથે AND
\item
  \textbf{આઉટપુટ}: nMOS રચનાના કારણે ઉલટાવેલું પરિણામ
\end{itemize}

\textbf{યાદાશ્ત ટેકનિક:} ``POI - Parallel OR, Inversion at output''

\end{solutionbox}
\begin{center}\rule{0.5\linewidth}{0.5pt}\end{center}

\subsection*{પ્રશ્ન 4(બ) [4
ગુણ]}\label{uxaaauxab0uxab6uxaa8-4uxaac-4-uxa97uxaa3}

\textbf{VERILOG માં વપરાતી ડિઝાઇન શૈલીઓની સૂચિ બનાવો અને સમજાવો.}

\begin{solutionbox}


{\def\LTcaptype{none} % do not increment counter
\vspace{-5pt}
\captionof{table}{વેરિલોગ ડિઝાઇન શૈલીઓ}
\vspace{-10pt}
\begin{longtable}[]{@{}
  >{\raggedright\arraybackslash}p{(\linewidth - 6\tabcolsep) * \real{0.1471}}
  >{\raggedright\arraybackslash}p{(\linewidth - 6\tabcolsep) * \real{0.2059}}
  >{\raggedright\arraybackslash}p{(\linewidth - 6\tabcolsep) * \real{0.3824}}
  >{\raggedright\arraybackslash}p{(\linewidth - 6\tabcolsep) * \real{0.2647}}@{}}
\toprule\noalign{}
\begin{minipage}[b]{\linewidth}\raggedright
શૈલી
\end{minipage} & \begin{minipage}[b]{\linewidth}\raggedright
વર્ણન
\end{minipage} & \begin{minipage}[b]{\linewidth}\raggedright
ઉપયોગનો કેસ
\end{minipage} & \begin{minipage}[b]{\linewidth}\raggedright
ઉદાહરણ
\end{minipage} \\
\midrule\noalign{}
\endhead
\bottomrule\noalign{}
\endlastfoot
\textbf{બિહેવિયરલ} & એલ્ગોરિધમ વર્ણન & ઉચ્ચ-સ્તરીય મોડેલિંગ & always blocks \\
\textbf{ડેટાફ્લો} & બૂલિયન expressions & કમ્બિનેશનલ લોજિક & assign
statements \\
\textbf{સ્ટ્રક્ચરલ} & કમ્પોનન્ટ instantiation & હાઇરાર્કિકલ ડિઝાઇન & module
connections \\
\textbf{ગેટ-લેવલ} & પ્રિમિટિવ ગેટ્સ & લો-લેવલ ડિઝાઇન & and, or, not gates \\
\end{longtable}
}

\textbf{લાક્ષણિકતાઓ:}

\begin{itemize}
\tightlist
\item
  \textbf{બિહેવિયરલ}: સર્કિટ શું કરે છે તેનું વર્ણન
\item
  \textbf{સ્ટ્રક્ચરલ}: કમ્પોનન્ટ્સ કેવી રીતે જોડાય છે તે બતાવે છે
\item
  \textbf{મિશ્રિત}: જટિલ ડિઝાઇન માટે બહુવિધ શૈલીઓ જોડે છે
\end{itemize}

\textbf{યાદાશ્ત ટેકનિક:} ``BDSG - Behavioral, Dataflow, Structural,
Gate-level''

\end{solutionbox}
\begin{center}\rule{0.5\linewidth}{0.5pt}\end{center}

\subsection*{પ્રશ્ન 4(ક) [7
ગુણ]}\label{uxaaauxab0uxab6uxaa8-4uxa95-7-uxa97uxaa3}

\textbf{CMOS નો ઉપયોગ કરીને NAND2 SR લેચનો અમલ કરો અને CMOS નો ઉપયોગ કરીને
NOR2 SR લેચનો પણ અમલ કરો.}

\begin{solutionbox}

\textbf{NAND2 SR લેચ:}

\begin{verbatim}
module nand\_sr\_latch(
    input S, R,
    output Q, Q\_bar
);
    nand(Q, S, Q\_bar);
    nand(Q\_bar, R, Q);
endmodule
\end{verbatim}

\textbf{NOR2 SR લેચ:}

\begin{verbatim}
module nor\_sr\_latch(
    input S, R,
    output Q, Q\_bar
);
    nor(Q\_bar, R, Q);
    nor(Q, S, Q\_bar);
endmodule
\end{verbatim}


{\def\LTcaptype{none} % do not increment counter
\vspace{-5pt}
\captionof{table}{લેચ તુલના}
\vspace{-10pt}
\begin{longtable}[]{@{}llll@{}}
\toprule\noalign{}
પ્રકાર & સક્રિય સ્તર & સેટ ઓપરેશન & રીસેટ ઓપરેશન \\
\midrule\noalign{}
\endhead
\bottomrule\noalign{}
\endlastfoot
\textbf{NAND} & લો (0) & S=0, R=1 & S=1, R=0 \\
\textbf{NOR} & હાઇ (1) & S=1, R=0 & S=0, R=1 \\
\end{longtable}
}

\textbf{મુખ્ય તફાવતો:}

\begin{itemize}
\tightlist
\item
  \textbf{NAND}: લો ઇનપુટ્સ સાથે Set/Reset
\item
  \textbf{NOR}: હાઇ ઇનપુટ્સ સાથે Set/Reset
\item
  \textbf{ફીડબેક}: ક્રોસ-કપલ્ડ ગેટ્સ સ્થિતિ જાળવે છે
\end{itemize}

\textbf{યાદાશ્ત ટેકનિક:} ``NAND-Low, NOR-High active''

\end{solutionbox}
\begin{center}\rule{0.5\linewidth}{0.5pt}\end{center}

\subsection*{પ્રશ્ન 4(અ OR) [3
ગુણ]}\label{uxaaauxab0uxab6uxaa8-4uxa85-or-3-uxa97uxaa3}

\textbf{Y= (ABC + DE + F)' ને ડિપ્લીશન nMOS લોડ સાથે અમલમાં મૂકો.}

\begin{solutionbox}

\begin{verbatim}
    VDD
     |
   ┌─┴─┐ Depletion Load
   │   │
   └─┬─┘
     │ Y
A────┼────┐
B────┼────┤ Series (AND)
C────┼────┘
     │
D────┼────┐
E────┼────┘ Series (AND)
     │
F────┼────┘ Single
     │
    GND
\end{verbatim}

\textbf{અમલીકરણ લોજિક:}

\begin{itemize}
\tightlist
\item
  \textbf{ABC}: સીરિઝ કનેક્શન (AND ફંક્શન)
\item
  \textbf{DE}: સીરિઝ કનેક્શન (AND ફંક્શન)\\
\item
  \textbf{F}: સિંગલ ટ્રાન્ઝિસ્ટર
\item
  \textbf{પરિણામ}: ઇનવર્શનના કારણે Y = (ABC + DE + F)'
\end{itemize}

\textbf{યાદાશ્ત ટેકનિક:} ``SSS-I: Series-Series-Single with Inversion''

\end{solutionbox}
\begin{center}\rule{0.5\linewidth}{0.5pt}\end{center}

\subsection*{પ્રશ્ન 4(બ OR) [4
ગુણ]}\label{uxaaauxab0uxab6uxaa8-4uxaac-or-4-uxa97uxaa3}

\textbf{ફુલ એડરને અમલમાં મૂકવા માટે વેરિલોગ કોડ લખો.}

\begin{solutionbox}

\begin{verbatim}
module full\_adder(
    input a, b, cin,
    output sum, cout
);
    assign sum = a \^{} b \^{} cin;
    assign cout = (a \& b) | (cin \& (a \^{} b));
endmodule
\end{verbatim}


{\def\LTcaptype{none} % do not increment counter
\vspace{-5pt}
\captionof{table}{ફુલ એડર સત્ય કોષ્ટક}
\vspace{-10pt}
\begin{longtable}[]{@{}lllll@{}}
\toprule\noalign{}
A & B & Cin & Sum & Cout \\
\midrule\noalign{}
\endhead
\bottomrule\noalign{}
\endlastfoot
0 & 0 & 0 & 0 & 0 \\
0 & 0 & 1 & 1 & 0 \\
0 & 1 & 0 & 1 & 0 \\
0 & 1 & 1 & 0 & 1 \\
1 & 0 & 0 & 1 & 0 \\
1 & 0 & 1 & 0 & 1 \\
1 & 1 & 0 & 0 & 1 \\
1 & 1 & 1 & 1 & 1 \\
\end{longtable}
}

\textbf{લોજિક ફંક્શન્સ:}

\begin{itemize}
\tightlist
\item
  \textbf{સમ}: ટ્રિપલ XOR ઓપરેશન
\item
  \textbf{કેરી}: ઇનપુટ્સનું મેજોરિટી ફંક્શન
\end{itemize}

\textbf{યાદાશ્ત ટેકનિક:} ``XOR-Sum, Majority-Carry''

\end{solutionbox}
\begin{center}\rule{0.5\linewidth}{0.5pt}\end{center}

\subsection*{પ્રશ્ન 4(ક OR) [7
ગુણ]}\label{uxaaauxab0uxab6uxaa8-4uxa95-or-7-uxa97uxaa3}

\textbf{ડિપ્લીશન લોડનો ઉપયોગ કરીને Y = (S1'S0'I0 + S1'S0 I1 + S1 S0' I2 +
S1 S2 I3) લાગૂ કરો}

\begin{solutionbox}

\textbf{નોંધ}: છેલ્લા ટર્મમાં S2 એ S0 હોવું જોઈએ.

\begin{verbatim}
// 4:1 મલ્ટિપ્લેક્સર અમલીકરણ
module mux\_4to1(
    input [1:0] sel,  // S1, S0
    input [3:0] data, // I3, I2, I1, I0
    output Y
);
assign

Y = (sel == 2{b00}) ? data[0] :

               (sel == 2{b01}) ? data[1] :
               (sel == 2{b10}) ? data[2] :
                                data[3];
endmodule
\end{verbatim}


{\def\LTcaptype{none} % do not increment counter
\vspace{-5pt}
\captionof{table}{મલ્ટિપ્લેક્સર સિલેક્શન}
\vspace{-10pt}
\begin{longtable}[]{@{}llll@{}}
\toprule\noalign{}
S1 & S0 & પસંદ કરેલ ઇનપુટ & આઉટપુટ \\
\midrule\noalign{}
\endhead
\bottomrule\noalign{}
\endlastfoot
0 & 0 & I0 & Y = I0 \\
0 & 1 & I1 & Y = I1 \\
1 & 0 & I2 & Y = I2 \\
1 & 1 & I3 & Y = I3 \\
\end{longtable}
}

\textbf{સર્કિટ અમલીકરણ:}

\begin{itemize}
\tightlist
\item
  \textbf{ડીકોડર}: S1, S0 select signals જનરેટ કરે છે
\item
  \textbf{AND ગેટ્સ}: દરેક ઇનપુટ સંબંધિત select સાથે ANDed
\item
  \textbf{OR ગેટ}: બધા AND આઉટપુટ્સને જોડે છે
\end{itemize}

\textbf{યાદાશ્ત ટેકનિક:} ``DAO - Decoder, AND gates, OR combination''

\end{solutionbox}
\begin{center}\rule{0.5\linewidth}{0.5pt}\end{center}

\subsection*{પ્રશ્ન 5(અ) [3
ગુણ]}\label{uxaaauxab0uxab6uxaa8-5uxa85-3-uxa97uxaa3}

\textbf{CMOS નો ઉપયોગ કરીને લોજિક ફંક્શન G = (PQR +U(S+T))' નો અમલ કરો}

\begin{solutionbox}

\begin{verbatim}
    VDD
     |
P────┼────┐
Q────┼────┤ Parallel pMOS
R────┼────┘ (NOR)
     │
U────┼────┐
     │    │
S────┼────┤ Series pMOS  
T────┼────┘ (NAND)
     │
     │ G
P────┼────┐
Q────┼────┤ Series nMOS
R────┼────┘ (AND)
     │
U────┼────┐
     │    │
S────┼────┤ Parallel nMOS
T────┼────┘ (OR)
     │
    GND
\end{verbatim}

\textbf{અમલીકરણ}:

\begin{itemize}
\tightlist
\item
  \textbf{pMOS}: OR માટે Parallel, AND માટે Series (ઉલટી લોજિક)
\item
  \textbf{nMOS}: AND માટે Series, OR માટે Parallel (સામાન્ય લોજિક)
\item
  \textbf{પરિણામ}: ડી મોર્ગનનો નિયમ આપોઆપ લાગુ થાય છે
\end{itemize}

\textbf{યાદાશ્ત ટેકનિક:} ``PSSP - Parallel Series Series Parallel''

\end{solutionbox}
\begin{center}\rule{0.5\linewidth}{0.5pt}\end{center}

\subsection*{પ્રશ્ન 5(બ) [4
ગુણ]}\label{uxaaauxab0uxab6uxaa8-5uxaac-4-uxa97uxaa3}

\textbf{વેરિલોગનો ઉપયોગ કરીને 8\times1 મલ્ટિપ્લેક્સર અમલમાં મૂકો.}

\begin{solutionbox}

\begin{verbatim}
module mux\_8to1(
    input [2:0] sel,     // 3{-bit select}
    input [7:0] data,    // 8 data inputs
    output reg Y         // Output
);
    always @(*) begin
        case(sel)
            3{b000}: Y = data[0];
            3{b001}: Y = data[1];
            3{b010}: Y = data[2];
            3{b011}: Y = data[3];
            3{b100}: Y = data[4];
            3{b101}: Y = data[5];
            3{b110}: Y = data[6];
            3{b111}: Y = data[7];
        endcase
    end
endmodule
\end{verbatim}


{\def\LTcaptype{none} % do not increment counter
\vspace{-5pt}
\captionof{table}{8:1 MUX સિલેક્શન}
\vspace{-10pt}
\begin{longtable}[]{@{}llll@{}}
\toprule\noalign{}
S2 & S1 & S0 & આઉટપુટ \\
\midrule\noalign{}
\endhead
\bottomrule\noalign{}
\endlastfoot
0 & 0 & 0 & data[0] \\
0 & 0 & 1 & data[1] \\
0 & 1 & 0 & data[2] \\
0 & 1 & 1 & data[3] \\
1 & 0 & 0 & data[4] \\
1 & 0 & 1 & data[5] \\
1 & 1 & 0 & data[6] \\
1 & 1 & 1 & data[7] \\
\end{longtable}
}

\textbf{યાદાશ્ત ટેકનિક:} ``Case-Always: always block માં case statement
વાપરો''

\end{solutionbox}
\begin{center}\rule{0.5\linewidth}{0.5pt}\end{center}

\subsection*{પ્રશ્ન 5(ક) [7
ગુણ]}\label{uxaaauxab0uxab6uxaa8-5uxa95-7-uxa97uxaa3}

\textbf{વેરિલોગમાં સ્ટ્રક્ચરલ મોડેલિંગ શૈલીનો ઉપયોગ કરીને 4 બીટ ફુલ એડરને લાગૂ કરો.}

\begin{solutionbox}

\begin{verbatim}
module full\_adder\_4bit(
    input [3:0] a, b,
    input cin,
    output [3:0] sum,
    output cout
);
    wire c1, c2, c3;
    
    full\_adder fa0(.a(a[0]), .b(b[0]), .cin(cin), 
                   .sum(sum[0]), .cout(c1));
    full\_adder fa1(.a(a[1]), .b(b[1]), .cin(c1), 
                   .sum(sum[1]), .cout(c2));
    full\_adder fa2(.a(a[2]), .b(b[2]), .cin(c2), 
                   .sum(sum[2]), .cout(c3));
    full\_adder fa3(.a(a[3]), .b(b[3]), .cin(c3), 
                   .sum(sum[3]), .cout(cout));
endmodule

module full\_adder(
    input a, b, cin,
    output sum, cout
);
    assign sum = a \^{} b \^{} cin;
    assign cout = (a \& b) | (cin \& (a \^{} b));
endmodule
\end{verbatim}

\textbf{સ્ટ્રક્ચરલ લક્ષણો:}

\begin{itemize}
\tightlist
\item
  \textbf{મોડ્યુલ instantiation}: ચાર 1-બીટ ફુલ એડર્સ
\item
  \textbf{કેરી ચેઇન}: સ્ટેજો વચ્ચે carries કનેક્ટ કરે છે
\item
  \textbf{હાઇરાર્કિકલ ડિઝાઇન}: બેસિક ફુલ એડર મોડ્યુલનો પુનઃઉપયોગ
\end{itemize}


{\def\LTcaptype{none} % do not increment counter
\vspace{-5pt}
\captionof{table}{રિપલ કેરી એડિશન}
\vspace{-10pt}
\begin{longtable}[]{@{}lllll@{}}
\toprule\noalign{}
સ્ટેજ & ઇનપુટ્સ & કેરી ઇન & સમ & કેરી આઉટ \\
\midrule\noalign{}
\endhead
\bottomrule\noalign{}
\endlastfoot
\textbf{FA0} & A[0], B[0] & Cin & S[0] & C1 \\
\textbf{FA1} & A[1], B[1] & C1 & S[1] & C2 \\
\textbf{FA2} & A[2], B[2] & C2 & S[2] & C3 \\
\textbf{FA3} & A[3], B[3] & C3 & S[3] & Cout \\
\end{longtable}
}

\textbf{યાદાશ્ત ટેકનિક:} ``RCC - Ripple Carry Chain connection''

\end{solutionbox}
\begin{center}\rule{0.5\linewidth}{0.5pt}\end{center}

\subsection*{પ્રશ્ન 5(અ OR) [3
ગુણ]}\label{uxaaauxab0uxab6uxaa8-5uxa85-or-3-uxa97uxaa3}

\textbf{CMOS નો ઉપયોગ કરીને લોજિક ફંક્શન Y = ((AF(D + E) )+ (B+ C))' ને
અમલમાં મૂકો.}

\begin{verbatim}
    VDD
     |
A────┼────┐
F────┼────┤
     │    │ Series pMOS
D────┼────┤ Parallel
E────┼────┘
     │
B────┼────┐
C────┼────┘ Parallel pMOS
     │
     │ Y
A────┼────┐
F────┼────┤
     │    │ Series nMOS
D────┼────┤ Parallel
E────┼────┘
     │
B────┼────┐
C────┼────┘ Parallel nMOS
     │
    GND
\end{verbatim}

\textbf{લોજિક વિભાજન:}

\begin{itemize}
\tightlist
\item
  \textbf{આંતરિક ટર્મ}: AF(D + E) = A AND F AND (D OR E)
\item
  \textbf{બાહ્ય ટર્મ}: (B + C) = B OR C
\item
  \textbf{અંતિમ}: Y = (AF(D + E) + (B + C))'
\end{itemize}

\textbf{CMOS અમલીકરણ:}

\begin{itemize}
\tightlist
\item
  \textbf{PMOS નેટવર્ક}: ફંક્શનનું complement અમલ કરે છે
\item
  \textbf{NMOS નેટવર્ક}: મૂળ ફંક્શન અમલ કરે છે
\item
  \textbf{પરિણામ}: કુદરતી inversion Y આપે છે
\end{itemize}

\textbf{યાદાશ્ત ટેકનિક:} ``PNAI - PMOS Network Applies Inversion''

\begin{center}\rule{0.5\linewidth}{0.5pt}\end{center}

\subsection*{પ્રશ્ન 5(બ OR) [4
ગુણ]}\label{uxaaauxab0uxab6uxaa8-5uxaac-or-4-uxa97uxaa3}

\textbf{વેરિલોગનો ઉપયોગ કરીને 4 બીટ અપ કાઉન્ટર અમલમાં મૂકવું}

\begin{solutionbox}

\begin{verbatim}
module counter\_4bit\_up(
    input clk, reset,
    output reg [3:0] count
);
    always @(posedge clk or posedge reset) begin
        if (reset)
            count {=} 4{b0000};
        else
            count {=} count + 1;
    end
endmodule
\end{verbatim}


{\def\LTcaptype{none} % do not increment counter
\vspace{-5pt}
\captionof{table}{કાઉન્ટર સિક્વન્સ}
\vspace{-10pt}
\begin{longtable}[]{@{}llll@{}}
\toprule\noalign{}
ક્લોક & રીસેટ & કાઉન્ટ & નેક્સ્ટ કાઉન્ટ \\
\midrule\noalign{}
\endhead
\bottomrule\noalign{}
\endlastfoot
↑ & 1 & X & 0000 \\
↑ & 0 & 0000 & 0001 \\
↑ & 0 & 0001 & 0010 \\
↑ & 0 & \ldots{} & \ldots{} \\
↑ & 0 & 1111 & 0000 \\
\end{longtable}
}

\textbf{લક્ષણો:}

\begin{itemize}
\tightlist
\item
  \textbf{સિંક્રોનસ રીસેટ}: ક્લોક એજ પર રીસેટ
\item
  \textbf{ઓટો રોલઓવર}: 1111 \rightarrow 0000
\item
  \textbf{4-બીટ રેન્જ}: 0 થી 15 સુધી ગણે છે
\end{itemize}

\textbf{યાદાશ્ત ટેકનિક:} ``SRA - Synchronous Reset with Auto rollover''

\end{solutionbox}
\begin{center}\rule{0.5\linewidth}{0.5pt}\end{center}

\subsection*{પ્રશ્ન 5(ક OR) [7
ગુણ]}\label{uxaaauxab0uxab6uxaa8-5uxa95-or-7-uxa97uxaa3}

\textbf{વેરિલોગમાં બિહેવિયરલ મોડેલિંગ સ્ટાઈલનો ઉપયોગ કરીને 3:8 ડીકોડરનો અમલ
કરો}

\begin{solutionbox}

\begin{verbatim}
module decoder\_3to8(
    input [2:0] select,
    input enable,
    output reg [7:0] out
);
    always @(*) begin
        if (enable) begin
            case(select)
                3{b000}: out = 8{b00000001};
                3{b001}: out = 8{b00000010};
                3{b010}: out = 8{b00000100};
                3{b011}: out = 8{b00001000};
                3{b100}: out = 8{b00010000};
                3{b101}: out = 8{b00100000};
                3{b110}: out = 8{b01000000};
                3{b111}: out = 8{b10000000};
                default: out = 8{b00000000};
            endcase
        end else begin
            out = 8{b00000000};
        end
    end
endmodule
\end{verbatim}


{\def\LTcaptype{none} % do not increment counter
\vspace{-5pt}
\captionof{table}{3:8 ડીકોડર સત્ય કોષ્ટક}
\vspace{-10pt}
\begin{longtable}[]{@{}llllllllllll@{}}
\toprule\noalign{}
Enable & A2 & A1 & A0 & Y7 & Y6 & Y5 & Y4 & Y3 & Y2 & Y1 & Y0 \\
\midrule\noalign{}
\endhead
\bottomrule\noalign{}
\endlastfoot
0 & X & X & X & 0 & 0 & 0 & 0 & 0 & 0 & 0 & 0 \\
1 & 0 & 0 & 0 & 0 & 0 & 0 & 0 & 0 & 0 & 0 & 1 \\
1 & 0 & 0 & 1 & 0 & 0 & 0 & 0 & 0 & 0 & 1 & 0 \\
1 & 0 & 1 & 0 & 0 & 0 & 0 & 0 & 0 & 1 & 0 & 0 \\
1 & 0 & 1 & 1 & 0 & 0 & 0 & 0 & 1 & 0 & 0 & 0 \\
1 & 1 & 0 & 0 & 0 & 0 & 0 & 1 & 0 & 0 & 0 & 0 \\
1 & 1 & 0 & 1 & 0 & 0 & 1 & 0 & 0 & 0 & 0 & 0 \\
1 & 1 & 1 & 0 & 0 & 1 & 0 & 0 & 0 & 0 & 0 & 0 \\
1 & 1 & 1 & 1 & 1 & 0 & 0 & 0 & 0 & 0 & 0 & 0 \\
\end{longtable}
}

\textbf{મુખ્ય લક્ષણો:}

\begin{itemize}
\tightlist
\item
  \textbf{બિહેવિયરલ મોડેલિંગ}: always બ્લોક અને case statement વાપરે છે
\item
  \textbf{Enable કંટ્રોલ}: enable = 0 હોય ત્યારે બધા આઉટપુટ્સ disabled
\item
  \textbf{વન-હોટ આઉટપુટ}: એક સમયે માત્ર એક આઉટપુટ active
\item
  \textbf{3-બીટ ઇનપુટ}: 8 આઉટપુટ્સમાંથી એક પસંદ કરે છે
\end{itemize}

\textbf{ઉપયોગો:}

\begin{itemize}
\tightlist
\item
  \textbf{મેમરી એડ્રેસિંગ}: ચિપ select જનરેશન
\item
  \textbf{ડેટા રાઉટિંગ}: ચેનલ સિલેક્શન
\item
  \textbf{કંટ્રોલ લોજિક}: સ્ટેટ મશીન આઉટપુટ્સ
\end{itemize}

\textbf{યાદાશ્ત ટેકનિક:} ``BEOH - Behavioral Enable One-Hot decoder''

\end{solutionbox}

\end{document}
