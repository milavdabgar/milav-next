\documentclass[10pt,a4paper]{article}

% content/resources/templates/preamble.tex
\usepackage[margin=0.6in]{geometry}
\author{Milav Dabgar}
\usepackage{amsmath,amssymb,amsthm}
\usepackage{booktabs}
\usepackage{multirow}
\usepackage{xcolor}
\usepackage{tcolorbox}
\tcbuselibrary{breakable,skins}
\usepackage[colorlinks=true,linkcolor=blue]{hyperref}
\usepackage{titlesec}
\usepackage{enumitem}
\usepackage{tikz}
\usepackage{pgfplots}
\usepackage{circuitikz}
\usepackage[version=4]{mhchem}
\usepackage{longtable}
\usepackage{array}
\usepackage{float}
\usepackage{caption}
\usepackage{listings}

\lstset{
  basicstyle=\small\ttfamily,
  breaklines=true,
  breakatwhitespace=false,
  postbreak=\mbox{\textcolor{red}{$\hookrightarrow$}\space},
  float=false,
  numbers=left,
  numberstyle=\tiny\color{gray},
  numbersep=10pt,
  xleftmargin=2em,
  keywordstyle=\color{blue},
  commentstyle=\color{green!60!black},
  stringstyle=\color{purple},
  backgroundcolor=\color{gray!5},
  showstringspaces=false,
  tabsize=2,
  captionpos=b,
  keepspaces=true,
  columns=flexible
}

\pgfplotsset{compat=1.18}
\usetikzlibrary{shapes,arrows,positioning,calc,patterns,decorations.pathmorphing,decorations.markings,arrows.meta}

% Color scheme
\definecolor{headcolor}{RGB}{0,102,204}
\definecolor{keycolor}{RGB}{220,20,60}
\definecolor{solutioncolor}{RGB}{34,139,34}
\definecolor{mnemoniccolor}{RGB}{148,0,211}
\definecolor{codecolor}{RGB}{0,0,100}

% Spacing
\setlength{\parskip}{3pt}
\setlist[itemize]{nosep}
\setlist[enumerate]{nosep}

% Title formatting
\titleformat{\section}{\Large\bfseries\color{headcolor}}{\thesection}{1em}{}
\titleformat{\subsection}{\large\bfseries\color{headcolor}}{\thesubsection}{1em}{}

% Pandoc tightlist compatibility
\providecommand{\tightlist}{%
  \setlength{\itemsep}{0pt}\setlength{\parskip}{0pt}}

% Pandoc longtable compatibility
\newcounter{none}
\def\thenone{}


% content/resources/templates/english-boxes.tex
% This file is currently empty - it exists to maintain consistency with the import structure.
% Add custom environments here if needed in the future.


\begin{document}

\begin{center}
{\Huge\bfseries\color{headcolor} Subject Name Solutions}\\[5pt]
{\LARGE 4361102 -- Summer 2025}\\[3pt]
{\large Semester 1 Study Material}\\[3pt]
{\normalsize\textit{Detailed Solutions and Explanations}}
\end{center}

\vspace{10pt}

\subsection*{Question 1(a) [3 marks]}\label{q1a}

\textbf{State importance of scaling}

\begin{solutionbox}
Scaling is crucial for advancing semiconductor
technology and improving device performance.

{\def\LTcaptype{none} % do not increment counter
\begin{longtable}[]{@{}ll@{}}
\toprule\noalign{}
\textbf{Scaling Benefits} & \textbf{Description} \\
\midrule\noalign{}
\endhead
\bottomrule\noalign{}
\endlastfoot
\textbf{Device Size} & Reduces transistor dimensions for higher
density \\
\textbf{Speed} & Faster switching due to shorter channel length \\
\textbf{Power} & Lower power consumption per operation \\
\textbf{Cost} & More chips per wafer, reducing cost per function \\
\end{longtable}
}

\begin{itemize}
\tightlist
\item
  \textbf{Technology advancement}: Enables Moore's Law continuation
\item
  \textbf{Performance boost}: Higher frequency operation possible
\item
  \textbf{Market competitiveness}: Smaller, faster, cheaper products
\end{itemize}

\end{solutionbox}
\begin{mnemonicbox}
``Small Devices Speed Progress Cheaply''

\end{mnemonicbox}
\begin{center}\rule{0.5\linewidth}{0.5pt}\end{center}

\subsection*{Question 1(b) [4 marks]}\label{q1b}

\textbf{Compare Planar MOSFET and FINFET}

\begin{solutionbox}
FinFET technology addresses limitations of planar
MOSFET at smaller nodes.

{\def\LTcaptype{none} % do not increment counter
\begin{longtable}[]{@{}
  >{\raggedright\arraybackslash}p{(\linewidth - 4\tabcolsep) * \real{0.3261}}
  >{\raggedright\arraybackslash}p{(\linewidth - 4\tabcolsep) * \real{0.4130}}
  >{\raggedright\arraybackslash}p{(\linewidth - 4\tabcolsep) * \real{0.2609}}@{}}
\toprule\noalign{}
\begin{minipage}[b]{\linewidth}\raggedright
\textbf{Parameter}
\end{minipage} & \begin{minipage}[b]{\linewidth}\raggedright
\textbf{Planar MOSFET}
\end{minipage} & \begin{minipage}[b]{\linewidth}\raggedright
\textbf{FinFET}
\end{minipage} \\
\midrule\noalign{}
\endhead
\bottomrule\noalign{}
\endlastfoot
\textbf{Structure} & 2D flat channel & 3D fin-shaped channel \\
\textbf{Gate Control} & Single gate & Tri-gate/multi-gate \\
\textbf{Short Channel Effects} & High at small nodes & Significantly
reduced \\
\textbf{Leakage Current} & Higher subthreshold leakage & Much lower
leakage \\
\end{longtable}
}

\begin{itemize}
\tightlist
\item
  \textbf{Scalability}: FinFET enables sub-22nm technology nodes
\item
  \textbf{Power efficiency}: FinFET offers better power-performance
  ratio
\item
  \textbf{Manufacturing}: FinFET requires more complex fabrication
\end{itemize}

\end{solutionbox}
\begin{mnemonicbox}
``Fins Control Current Better Than Flat''

\end{mnemonicbox}
\begin{center}\rule{0.5\linewidth}{0.5pt}\end{center}

\subsection*{Question 1(c) [7 marks]}\label{q1c}

\textbf{Draw and Explain VDS-ID AND VGS-ID characteristics of N channel
MOSFET}

\begin{solutionbox}
N-channel MOSFET characteristics show device behavior
in different operating regions.

\textbf{Diagram:}

\begin{verbatim}
VGS{-ID Characteristics        VDS{-}ID Characteristics}
                                     
    ID ↑                            ID ↑
      |    VGS4                       |     VGS4{VGS3VGS2VGS1}
      |   VGS3 \_\_\_                    |    \_\_\_\_\_\_\_\_\_\_\_\_\_\_\_\_
      |  VGS2 \_\_\_                     |   /
      | VGS1\_\_\_                       |  /  Linear Region
    VT|/\_\_                            | /
      |\_\_\_\_\_\_\_\_\_\_\_\_\_\_\_\_\_\_\_\_\_\_\_\_      |/\_\_\_\_\_\_\_\_\_\_\_\_\_\_\_\_\_\_\_\_\_\_\_\_
         VGS                             VDS
\end{verbatim}

{\def\LTcaptype{none} % do not increment counter
\begin{longtable}[]{@{}lll@{}}
\toprule\noalign{}
\textbf{Region} & \textbf{Condition} & \textbf{Current Equation} \\
\midrule\noalign{}
\endhead
\bottomrule\noalign{}
\endlastfoot
\textbf{Cutoff} & VGS \textless{} VT & ID = 0 \\
\textbf{Linear} & VDS \textless{} (VGS-VT) & ID ∝ VDS \\
\textbf{Saturation} & VDS \geq (VGS-VT) & ID ∝ (VGS-VT)^{2} \\
\end{longtable}
}

\begin{itemize}
\tightlist
\item
  \textbf{Threshold voltage (VT)}: Minimum VGS for conduction
\item
  \textbf{Transconductance}: Slope of VGS-ID curve in saturation
\item
  \textbf{Output resistance}: Inverse slope in saturation region
\end{itemize}

\end{solutionbox}
\begin{mnemonicbox}
``Threshold Gates Linear Saturation''

\end{mnemonicbox}
\begin{center}\rule{0.5\linewidth}{0.5pt}\end{center}

\subsection*{Question 1(c OR) [7
marks]}\label{question-1c-or-7-marks}

\textbf{Explain different condition of MOS under external bias}

\begin{solutionbox}
External bias creates different charge distributions
affecting MOS capacitor behavior.

\textbf{Diagram:}

\begin{center}
\textbf{Mermaid Diagram (Code)}
\begin{verbatim}
{Shaded}
{Highlighting}[]
graph TD
    A[MOS Under Bias] {-{-}{} B[Accumulation VG {} 0]}
    A {-{-}{} C[Depletion 0 {} VG {} VT]}
    A {-{-}{} D[Inversion VG {} VT]}
    B {-{-}{} E[Holes accumulate at surface]}
    C {-{-}{} F[Surface depleted of carriers]}
    D {-{-}{} G[Electron inversion layer forms]}
{Highlighting}
{Shaded}
\end{verbatim}
\end{center}

{\def\LTcaptype{none} % do not increment counter
\begin{longtable}[]{@{}lll@{}}
\toprule\noalign{}
\textbf{Bias Condition} & \textbf{Surface State} &
\textbf{Capacitance} \\
\midrule\noalign{}
\endhead
\bottomrule\noalign{}
\endlastfoot
\textbf{Accumulation} & Majority carriers at surface & High (Cox) \\
\textbf{Depletion} & No mobile carriers & Medium \\
\textbf{Inversion} & Minority carriers form channel & High (Cox) \\
\end{longtable}
}

\begin{itemize}
\tightlist
\item
  \textbf{Flat band voltage}: No charge separation exists
\item
  \textbf{Energy band bending}: Determines carrier distribution
\item
  \textbf{Surface potential}: Controls inversion layer formation
\end{itemize}

\end{solutionbox}
\begin{mnemonicbox}
``Accumulate, Deplete, then Invert''

\end{mnemonicbox}
\begin{center}\rule{0.5\linewidth}{0.5pt}\end{center}

\subsection*{Question 2(a) [3 marks]}\label{q2a}

\textbf{Draw voltage transfer characteristic of ideal inverter}

\begin{solutionbox}
Ideal inverter provides sharp transition between logic
levels with infinite gain.

\textbf{Diagram:}

\begin{verbatim}
    VOUT ↑
        |
    VOH |    |
        |    |
        |    |
        |    |\_\_\_\_\_\_\_\_\_\_\_\_\_\_\_
        |                    |
        |                    | VOL
        |\_\_\_\_\_\_\_\_\_\_\_\_\_\_\_\_\_\_\_\_|\_\_\_\_\_\_\_
           VIL   VIH        VIN
\end{verbatim}

\begin{itemize}
\tightlist
\item
  \textbf{Sharp transition}: Infinite slope at switching point
\item
  \textbf{Noise margins}: NMH = VOH - VIH, NML = VIL - VOL
\item
  \textbf{Perfect logic levels}: VOH = VDD, VOL = 0V
\end{itemize}

\end{solutionbox}
\begin{mnemonicbox}
``Sharp Switch, Perfect Levels''

\end{mnemonicbox}
\begin{center}\rule{0.5\linewidth}{0.5pt}\end{center}

\subsection*{Question 2(b) [4 marks]}\label{q2b}

\textbf{Explain noise immunity and noise margin}

\begin{solutionbox}
Noise immunity measures circuit's ability to reject
unwanted signal variations.

{\def\LTcaptype{none} % do not increment counter
\begin{longtable}[]{@{}lll@{}}
\toprule\noalign{}
\textbf{Parameter} & \textbf{Definition} & \textbf{Formula} \\
\midrule\noalign{}
\endhead
\bottomrule\noalign{}
\endlastfoot
\textbf{NMH} & High-level noise margin & VOH - VIH \\
\textbf{NML} & Low-level noise margin & VIL - VOL \\
\textbf{Noise Immunity} & Ability to reject noise & Min(NMH, NML) \\
\end{longtable}
}

\begin{itemize}
\tightlist
\item
  \textbf{Logic threshold levels}: VIH (input high), VIL (input low)
\item
  \textbf{Output levels}: VOH (output high), VOL (output low)
\item
  \textbf{Better immunity}: Larger noise margins provide better
  protection
\item
  \textbf{Design goal}: Maximize noise margins for robust operation
\end{itemize}

\end{solutionbox}
\begin{mnemonicbox}
``Margins Protect Against Noise''

\end{mnemonicbox}
\begin{center}\rule{0.5\linewidth}{0.5pt}\end{center}

\subsection*{Question 2(c) [7 marks]}\label{q2c}

\textbf{Describe inverter circuit with saturated and linear depletion
load nMOS inverter}

\begin{solutionbox}
Depletion load nMOS inverters use depletion transistor
as active load resistor.

\textbf{Diagram:}

\begin{verbatim}
       VDD
        |
        |
    |{-{-}{-}+{-}{-}{-}| MD (Depletion Load)}
    |       | VT { 0}
    |       |
    +{-{-}{-}{-}{-}{-}{-}+{-}{-}{-}{-}{-}{-} VOUT}
    |
    |
|{-{-}{-}+{-}{-}{-}| MN (Driver)}
|       | VT { 0}
|       |
|\_\_\_\_\_\_\_|
   VIN    GND
\end{verbatim}

{\def\LTcaptype{none} % do not increment counter
\begin{longtable}[]{@{}lll@{}}
\toprule\noalign{}
\textbf{Load Type} & \textbf{Gate Connection} & \textbf{Operation} \\
\midrule\noalign{}
\endhead
\bottomrule\noalign{}
\endlastfoot
\textbf{Saturated Load} & VG = VD & Always in saturation \\
\textbf{Linear Load} & VG = VDD & Can operate in linear region \\
\end{longtable}
}

\begin{itemize}
\tightlist
\item
  \textbf{Depletion device}: Conducts with VGS = 0, acts as current
  source
\item
  \textbf{Load line analysis}: Determines operating point intersection
\item
  \textbf{Power consumption}: Always conducting, higher static power
\item
  \textbf{Switching speed}: Faster pull-down than pull-up
\end{itemize}

\end{solutionbox}
\begin{mnemonicbox}
``Depletion Loads Drive Outputs''

\end{mnemonicbox}
\begin{center}\rule{0.5\linewidth}{0.5pt}\end{center}

\subsection*{Question 2(a OR) [3
marks]}\label{question-2a-or-3-marks}

\textbf{Draw and explain enhancement load inverter}

\begin{solutionbox}
Enhancement load inverter uses enhancement MOSFET as
load with special biasing.

\textbf{Diagram:}

\begin{verbatim}
       VDD
        |
        |
    |{-{-}{-}+{-}{-}{-}| ME (Enhancement Load)}
    |   |   | VT { 0}
    |   +{-{-}{-}+}
    +{-{-}{-}{-}{-}{-}{-}+{-}{-}{-}{-}{-}{-} VOUT}
    |
    |
|{-{-}{-}+{-}{-}{-}| MN (Driver)}
|       |
|       |
|\_\_\_\_\_\_\_|
   VIN    GND
\end{verbatim}

\begin{itemize}
\tightlist
\item
  \textbf{Bootstrap connection}: Gate connected to drain for load
\item
  \textbf{Limited output high}: VOUT(max) = VDD - VT
\item
  \textbf{Threshold loss}: Enhancement load causes voltage drop
\end{itemize}

\end{solutionbox}
\begin{mnemonicbox}
``Enhancement Loses Threshold''

\end{mnemonicbox}
\begin{center}\rule{0.5\linewidth}{0.5pt}\end{center}

\subsection*{Question 2(b OR) [4
marks]}\label{question-2b-or-4-marks}

\textbf{List the advantages of CMOS inverter}

\begin{solutionbox}
CMOS technology offers superior performance compared to
NMOS inverters.

{\def\LTcaptype{none} % do not increment counter
\begin{longtable}[]{@{}ll@{}}
\toprule\noalign{}
\textbf{Advantage} & \textbf{Benefit} \\
\midrule\noalign{}
\endhead
\bottomrule\noalign{}
\endlastfoot
\textbf{Zero static power} & No current path in steady state \\
\textbf{Rail-to-rail output} & Full VDD and 0V output levels \\
\textbf{High noise immunity} & Large noise margins \\
\textbf{Symmetric switching} & Equal rise and fall times \\
\end{longtable}
}

\begin{itemize}
\tightlist
\item
  \textbf{Power efficiency}: Only dynamic power during switching
\item
  \textbf{Scalability}: Works well at all technology nodes
\item
  \textbf{Fan-out capability}: Can drive multiple inputs
\item
  \textbf{Temperature stability}: Performance less sensitive to
  temperature
\end{itemize}

\end{solutionbox}
\begin{mnemonicbox}
``CMOS Saves Power Perfectly''

\end{mnemonicbox}
\begin{center}\rule{0.5\linewidth}{0.5pt}\end{center}

\subsection*{Question 2(c OR) [7
marks]}\label{question-2c-or-7-marks}

\textbf{Draw and Explain operating mode of region for CMOS Inverter}

\begin{solutionbox}
CMOS inverter operation involves five distinct regions
based on input voltage.

\textbf{Diagram:}

\begin{center}
\textbf{Mermaid Diagram (Code)}
\begin{verbatim}
{Shaded}
{Highlighting}[]
graph TD
    A[CMOS Inverter Regions] {-{-}{} B[Region 1: PMOS ON, NMOS OFF]}
    A {-{-}{} C[Region 2: Both in saturation]}
    A {-{-}{} D[Region 3: Switching point]}
    A {-{-}{} E[Region 4: Both in saturation]}
    A {-{-}{} F[Region 5: PMOS OFF, NMOS ON]}
{Highlighting}
{Shaded}
\end{verbatim}
\end{center}

{\def\LTcaptype{none} % do not increment counter
\begin{longtable}[]{@{}llll@{}}
\toprule\noalign{}
\textbf{Region} & \textbf{NMOS State} & \textbf{PMOS State} &
\textbf{Output} \\
\midrule\noalign{}
\endhead
\bottomrule\noalign{}
\endlastfoot
\textbf{1} & OFF & Linear & VOH \approx VDD \\
\textbf{2} & Saturation & Saturation & Transition \\
\textbf{3} & Saturation & Saturation & VDD/2 \\
\textbf{4} & Saturation & Saturation & Transition \\
\textbf{5} & Linear & OFF & VOL \approx 0V \\
\end{longtable}
}

\begin{itemize}
\tightlist
\item
  \textbf{Switching threshold}: VTC crosses VDD/2 at region 3
\item
  \textbf{Current flow}: Only during transition regions 2,3,4
\item
  \textbf{Noise margins}: Regions 1 and 5 provide immunity
\item
  \textbf{Gain}: Maximum in region 3 (switching point)
\end{itemize}

\end{solutionbox}
\begin{mnemonicbox}
``Five Regions Control CMOS Switching''

\end{mnemonicbox}
\begin{center}\rule{0.5\linewidth}{0.5pt}\end{center}

\subsection*{Question 3(a) [3 marks]}\label{q3a}

\textbf{Draw two input NOR gate using CMOS}

\begin{solutionbox}
CMOS NOR gate implements De Morgan's law using
complementary networks.

\textbf{Diagram:}

\begin{verbatim}
       VDD
        |
    |{-{-}{-}+{-}{-}{-}| MP1}
    |A  |   |
    +{-{-}{-}+{-}{-}{-}+}
    |       |
    |   |{-{-}{-}+{-}{-}{-}| MP2}
    |   |B  |   |
    +{-{-}{-}+{-}{-}{-}+{-}{-}{-}+{-}{-}{-}{-}{-}{-} Y = (A+B)}
    |           |
|{-{-}{-}+{-}{-}{-}|   |{-}{-}{-}+{-}{-}{-}| MN2}
|A      |   |B      |
|       |   |       |
|\_\_\_\_\_\_\_|   |\_\_\_\_\_\_\_|
            |
           GND
\end{verbatim}

\begin{itemize}
\tightlist
\item
  \textbf{Pull-up network}: Series PMOS transistors (A AND B both low
  for high output)
\item
  \textbf{Pull-down network}: Parallel NMOS transistors (A OR B high for
  low output)
\item
  \textbf{Logic function}: Y = (A+B)' = A' · B'
\end{itemize}

\end{solutionbox}
\begin{mnemonicbox}
``Series PMOS, Parallel NMOS''

\end{mnemonicbox}
\begin{center}\rule{0.5\linewidth}{0.5pt}\end{center}

\subsection*{Question 3(b) [4 marks]}\label{q3b}

\textbf{Implement Boolean function Z= [(A+B)C+DE]' using CMOS}

\begin{solutionbox}
Complex CMOS logic uses AOI (AND-OR-Invert) structure
for efficient implementation.

\textbf{Diagram:}

\begin{verbatim}
                    VDD
                     |
         +{-{-}{-}{-}{-}{-}{-}{-}{-}{-}{-}+{-}{-}{-}{-}{-}{-}{-}{-}{-}{-}{-}+}
         |                       |
     |{-{-}{-}+{-}{-}{-}|               |{-}{-}{-}+{-}{-}{-}|}
     |A      |               |D      |
     |\_\_\_\_\_\_\_|               |\_\_\_\_\_\_\_|
         |                       |
     |{-{-}{-}+{-}{-}{-}|               |{-}{-}{-}+{-}{-}{-}|}
     |B      |               |E      |
     |\_\_\_\_\_\_\_|               |\_\_\_\_\_\_\_|
         |                       |
     |{-{-}{-}+{-}{-}{-}|               }
     |C      |               
     |\_\_\_\_\_\_\_|               
         |                       |
         +{-{-}{-}{-}{-}{-}{-}{-}{-}{-}{-}+{-}{-}{-}{-}{-}{-}{-}{-}{-}{-}{-}+{-}{-}{-}{-}{-}{-} Z}
                     |
         +{-{-}{-}{-}{-}{-}{-}{-}{-}{-}{-}+{-}{-}{-}{-}{-}{-}{-}{-}{-}{-}{-}+}
         |           |           |
     |{-{-}{-}+{-}{-}{-}|   |{-}{-}{-}+{-}{-}{-}|   |{-}{-}{-}+{-}{-}{-}|}
     |A      |   |B      |   |C      |
     |       |   |       |   |       |
     |\_\_\_\_\_\_\_|   |\_\_\_\_\_\_\_|   |\_\_\_\_\_\_\_|
                     |           |
                 |{-{-}{-}+{-}{-}{-}|   |{-}{-}{-}+{-}{-}{-}|}
                 |D      |   |E      |
                 |       |   |       |
                 |\_\_\_\_\_\_\_|   |\_\_\_\_\_\_\_|
                             |
                            GND
\end{verbatim}

\begin{itemize}
\tightlist
\item
  \textbf{AOI structure}: Efficient single-stage implementation
\item
  \textbf{Dual networks}: Complementary pull-up and pull-down
\item
  \textbf{Logic optimization}: Fewer transistors than separate gates
\end{itemize}

\end{solutionbox}
\begin{mnemonicbox}
``AOI Inverts Complex Logic Efficiently''

\end{mnemonicbox}
\begin{center}\rule{0.5\linewidth}{0.5pt}\end{center}

\subsection*{Question 3(c) [7 marks]}\label{q3c}

\textbf{Draw and explain CMOS NAND2 gate with the parasitic device
capacitances}

\begin{solutionbox}
Parasitic capacitances in CMOS gates affect switching
speed and power consumption.

\textbf{Diagram:}

\begin{verbatim}
       VDD
        |
    |{-{-}{-}+{-}{-}{-}| MP1  Cgd1}
    |A  |   |      
    +{-{-}{-}+{-}{-}{-}+{-}{-}{-}{-}{-}{-} Y = (AB)}
    |       |       |
    |   |{-{-}{-}+{-}{-}{-}| MP2  Cgd2}
    |   |B  |   |      |
    +{-{-}{-}+{-}{-}{-}+{-}{-}{-}+{-}{-}{-}{-}{-}{-}+}
    |           |      |
|{-{-}{-}+{-}{-}{-}|   |{-}{-}{-}+{-}{-}{-}|  | Cload}
|A      |   |B      |  |
|       |   |       |  |
|\_\_\_\_\_\_\_|   |\_\_\_\_\_\_\_|  |
    |           |      |
   Cgs1        Cgs2    |
    |           |      |
   GND         GND    GND

Parasitic Capacitances:
Cgs {- Gate to Source}
Cgd {- Gate to Drain  }
Cdb {- Drain to Bulk}
Csb {- Source to Bulk}
\end{verbatim}

{\def\LTcaptype{none} % do not increment counter
\begin{longtable}[]{@{}lll@{}}
\toprule\noalign{}
\textbf{Capacitance} & \textbf{Location} & \textbf{Effect} \\
\midrule\noalign{}
\endhead
\bottomrule\noalign{}
\endlastfoot
\textbf{Cgs} & Gate-Source & Input capacitance \\
\textbf{Cgd} & Gate-Drain & Miller effect \\
\textbf{Cdb} & Drain-Bulk & Output loading \\
\textbf{Csb} & Source-Bulk & Source loading \\
\end{longtable}
}

\begin{itemize}
\tightlist
\item
  \textbf{Switching delay}: Parasitic capacitances slow transitions
\item
  \textbf{Power consumption}: Charging/discharging parasitic caps
\item
  \textbf{Miller effect}: Cgd creates feedback, slows switching
\item
  \textbf{Layout optimization}: Minimize parasitic capacitances
\end{itemize}

\end{solutionbox}
\begin{mnemonicbox}
``Parasitics Slow Gates Down''

\end{mnemonicbox}
\begin{center}\rule{0.5\linewidth}{0.5pt}\end{center}

\subsection*{Question 3(a OR) [3
marks]}\label{question-3a-or-3-marks}

\textbf{Draw and explain NOR based Clocked SR latch using CMOS}

\begin{solutionbox}
Clocked SR latch uses NOR gates with clock enable for
synchronous operation.

\textbf{Diagram:}

\begin{center}
\textbf{Mermaid Diagram (Code)}
\begin{verbatim}
{Shaded}
{Highlighting}[]
graph LR
    S {-{-}{} A[NOR1]}
    CLK {-{-}{} A}
    A {-{-}{} Q}
    Q {-{-}{} B[NOR2]}
    R {-{-}{} C[NOR3]}
    CLK {-{-}{} C}
    C {-{-}{} D[NOR4]}
    D {-{-}{} QB[Q{}]}
    QB {-{-}{} B}
    B {-{-}{} Q}
{Highlighting}
{Shaded}
\end{verbatim}
\end{center}

\begin{itemize}
\tightlist
\item
  \textbf{Clock control}: S and R effective only when CLK = 1
\item
  \textbf{Transparent mode}: Output follows input when clock active
\item
  \textbf{Hold mode}: Output maintains state when clock inactive
\item
  \textbf{Basic building block}: Foundation for flip-flops
\end{itemize}

\end{solutionbox}
\begin{mnemonicbox}
``Clock Controls Transparent Latching''

\end{mnemonicbox}
\begin{center}\rule{0.5\linewidth}{0.5pt}\end{center}

\subsection*{Question 3(b OR) [4
marks]}\label{question-3b-or-4-marks}

\textbf{Implement Boolean function Z=[AB+C(D+E)]' using CMOS}

\begin{solutionbox}
This function implements inverted sum-of-products using
AOI logic structure.

\textbf{Logic Analysis:}

\begin{itemize}
\tightlist
\item
  Original: Z = [AB + C(D+E)]'
\item
  Expanded: Z = [AB + CD + CE]'
\item
  Implementation: Three AND terms fed to NOR
\end{itemize}

{\def\LTcaptype{none} % do not increment counter
\begin{longtable}[]{@{}lll@{}}
\toprule\noalign{}
\textbf{Term} & \textbf{Inputs} & \textbf{Function} \\
\midrule\noalign{}
\endhead
\bottomrule\noalign{}
\endlastfoot
\textbf{Term 1} & A, B & AB \\
\textbf{Term 2} & C, D & CD \\
\textbf{Term 3} & C, E & CE \\
\textbf{Output} & All terms & (AB + CD + CE)' \\
\end{longtable}
}

\begin{itemize}
\tightlist
\item
  \textbf{AOI implementation}: Single stage, efficient design
\item
  \textbf{Transistor count}: Fewer than separate gate implementation
\item
  \textbf{Performance}: Fast switching, low power
\end{itemize}

\end{solutionbox}
\begin{mnemonicbox}
``Three AND Terms Feed One NOR''

\end{mnemonicbox}
\begin{center}\rule{0.5\linewidth}{0.5pt}\end{center}

\subsection*{Question 3(c OR) [7
marks]}\label{question-3c-or-7-marks}

\textbf{Differentiate AOI and OAI Logic with example}

\begin{solutionbox}
AOI and OAI are complementary logic families for
efficient CMOS implementation.

{\def\LTcaptype{none} % do not increment counter
\begin{longtable}[]{@{}lll@{}}
\toprule\noalign{}
\textbf{Parameter} & \textbf{AOI (AND-OR-Invert)} & \textbf{OAI
(OR-AND-Invert)} \\
\midrule\noalign{}
\endhead
\bottomrule\noalign{}
\endlastfoot
\textbf{Structure} & AND gates \rightarrow OR \rightarrow Invert & OR gates \rightarrow AND \rightarrow
Invert \\
\textbf{Function} & (AB + CD + \ldots)' & ((A+B)(C+D)\ldots)' \\
\textbf{PMOS Network} & Series-parallel & Parallel-series \\
\textbf{NMOS Network} & Parallel-series & Series-parallel \\
\end{longtable}
}

\textbf{AOI Example: Y = (AB + CD)'}

\begin{verbatim}
PMOS: Series A{-B in parallel with Series C{-}D}
NMOS: Parallel A,B in series with Parallel C,D
\end{verbatim}

\textbf{OAI Example: Y = ((A+B)(C+D))'}

\begin{verbatim}
PMOS: Parallel A,B in series with Parallel C,D  
NMOS: Series A{-B in parallel with Series C{-}D}
\end{verbatim}

\begin{itemize}
\tightlist
\item
  \textbf{Design choice}: Select based on Boolean function form
\item
  \textbf{Optimization}: Minimizes transistor count and delay
\item
  \textbf{Duality}: AOI and OAI are De Morgan duals
\end{itemize}

\end{solutionbox}
\begin{mnemonicbox}
``AOI ANDs then ORs, OAI ORs then ANDs''

\end{mnemonicbox}
\begin{center}\rule{0.5\linewidth}{0.5pt}\end{center}

\subsection*{Question 4(a) [3 marks]}\label{q4a}

\textbf{Define: 1) Regularity 2) Modularity 3) Locality}

\begin{solutionbox}
Design hierarchy principles essential for managing VLSI
complexity and ensuring successful implementation.

{\def\LTcaptype{none} % do not increment counter
\begin{longtable}[]{@{}
  >{\raggedright\arraybackslash}p{(\linewidth - 4\tabcolsep) * \real{0.3409}}
  >{\raggedright\arraybackslash}p{(\linewidth - 4\tabcolsep) * \real{0.3636}}
  >{\raggedright\arraybackslash}p{(\linewidth - 4\tabcolsep) * \real{0.2955}}@{}}
\toprule\noalign{}
\begin{minipage}[b]{\linewidth}\raggedright
\textbf{Principle}
\end{minipage} & \begin{minipage}[b]{\linewidth}\raggedright
\textbf{Definition}
\end{minipage} & \begin{minipage}[b]{\linewidth}\raggedright
\textbf{Benefit}
\end{minipage} \\
\midrule\noalign{}
\endhead
\bottomrule\noalign{}
\endlastfoot
\textbf{Regularity} & Repeated use of similar structures & Easier
layout, testing \\
\textbf{Modularity} & Breaking design into smaller blocks & Independent
design, reuse \\
\textbf{Locality} & Interconnections mostly local & Reduced routing
complexity \\
\end{longtable}
}

\begin{itemize}
\tightlist
\item
  \textbf{Design efficiency}: Principles reduce design time and effort
\item
  \textbf{Verification}: Modular approach simplifies testing
\item
  \textbf{Scalability}: Enables larger, more complex designs
\end{itemize}

\end{solutionbox}
\begin{mnemonicbox}
``Regular Modules Stay Local''

\end{mnemonicbox}
\begin{center}\rule{0.5\linewidth}{0.5pt}\end{center}

\subsection*{Question 4(b) [4 marks]}\label{q4b}

\textbf{Implement SR latch (NAND gate) using CMOS inverter}

\begin{solutionbox}
SR latch using NAND gates provides set-reset
functionality with active-low inputs.

\textbf{Diagram:}

\begin{center}
\textbf{Mermaid Diagram (Code)}
\begin{verbatim}
{Shaded}
{Highlighting}[]
graph LR
    S{ {-}{-}{} A[NAND1]}
    A {-{-}{} Q}
    Q {-{-}{} B[NAND2]}
    R{ {-}{-}{} B}
    B {-{-}{} QB[Q{}]}
    QB {-{-}{} A}
{Highlighting}
{Shaded}
\end{verbatim}
\end{center}

\textbf{Truth Table:}

{\def\LTcaptype{none} % do not increment counter
\begin{longtable}[]{@{}lllll@{}}
\toprule\noalign{}
\textbf{S'} & \textbf{R'} & \textbf{Q} & \textbf{Q'} & \textbf{State} \\
\midrule\noalign{}
\endhead
\bottomrule\noalign{}
\endlastfoot
0 & 1 & 1 & 0 & Set \\
1 & 0 & 0 & 1 & Reset \\
1 & 1 & Q & Q' & Hold \\
0 & 0 & 1 & 1 & Invalid \\
\end{longtable}
}

\begin{itemize}
\tightlist
\item
  \textbf{Cross-coupled structure}: Provides memory function
\item
  \textbf{Active-low inputs}: S' = 0 sets, R' = 0 resets
\item
  \textbf{Forbidden state}: Both inputs low simultaneously
\end{itemize}

\end{solutionbox}
\begin{mnemonicbox}
``Cross-Coupled NANDS Remember State''

\end{mnemonicbox}
\begin{center}\rule{0.5\linewidth}{0.5pt}\end{center}

\subsection*{Question 4(c) [7 marks]}\label{q4c}

\textbf{Explain VLSI design flow}

\begin{solutionbox}
VLSI design flow follows systematic steps from
specification to fabrication.

\begin{center}
\textbf{Mermaid Diagram (Code)}
\begin{verbatim}
{Shaded}
{Highlighting}[]
graph LR
    A[System Specification] {-{-}{} B[Architectural Design]}
    B {-{-}{} C[Functional Design]}
    C {-{-}{} D[Logic Design]}
    D {-{-}{} E[Circuit Design]}
    E {-{-}{} F[Physical Design]}
    F {-{-}{} G[Fabrication]}
    G {-{-}{} H[Testing \& Packaging]}
{Highlighting}
{Shaded}
\end{verbatim}
\end{center}

{\def\LTcaptype{none} % do not increment counter
\begin{longtable}[]{@{}lll@{}}
\toprule\noalign{}
\textbf{Level} & \textbf{Activities} & \textbf{Output} \\
\midrule\noalign{}
\endhead
\bottomrule\noalign{}
\endlastfoot
\textbf{System} & Requirements analysis & Specifications \\
\textbf{Architecture} & Block-level design & System architecture \\
\textbf{Logic} & Boolean optimization & Gate netlist \\
\textbf{Circuit} & Transistor sizing & Circuit netlist \\
\textbf{Physical} & Layout, routing & GDSII file \\
\end{longtable}
}

\begin{itemize}
\tightlist
\item
  \textbf{Design verification}: Each level requires validation
\item
  \textbf{Iteration}: Feedback loops for optimization
\item
  \textbf{CAD tools}: Automation essential for complex designs
\item
  \textbf{Time-to-market}: Efficient flow reduces design cycle
\end{itemize}

\end{solutionbox}
\begin{mnemonicbox}
``System Architects Love Circuit Physical
Fabrication''

\end{mnemonicbox}
\begin{center}\rule{0.5\linewidth}{0.5pt}\end{center}

\subsection*{Question 4(a OR) [3
marks]}\label{question-4a-or-3-marks}

\textbf{Draw and explain Y-chart}

\begin{solutionbox}
Y-chart represents three design domains and their
abstraction levels in VLSI design.

\textbf{Diagram:}

\begin{center}
\textbf{Mermaid Diagram (Code)}
\begin{verbatim}
{Shaded}
{Highlighting}[]
graph TD
    A[Behavioral Domain] {-{-}{} D[System Level]}
    B[Structural Domain] {-{-}{} D}
    C[Physical Domain] {-{-}{} D}
    A {-{-}{} E[Algorithm Level]}
    B {-{-}{} E}
    C {-{-}{} E}
    A {-{-}{} F[Gate Level]}
    B {-{-}{} F}
    C {-{-}{} F}
{Highlighting}
{Shaded}
\end{verbatim}
\end{center}

\begin{itemize}
\tightlist
\item
  \textbf{Three domains}: Behavioral (function), Structural
  (components), Physical (geometry)
\item
  \textbf{Abstraction levels}: System \rightarrow Algorithm \rightarrow Gate \rightarrow Circuit \rightarrow
  Layout
\item
  \textbf{Design methodology}: Move between domains at same abstraction
  level
\end{itemize}

\end{solutionbox}
\begin{mnemonicbox}
``Behavior, Structure, Physics at All Levels''

\end{mnemonicbox}
\begin{center}\rule{0.5\linewidth}{0.5pt}\end{center}

\subsection*{Question 4(b OR) [4
marks]}\label{question-4b-or-4-marks}

\textbf{Implement clocked JK latch (NOR gate) using CMOS inverter}

\begin{solutionbox}
JK latch eliminates forbidden state of SR latch with
toggle capability.

\textbf{Diagram:}

\begin{center}
\textbf{Mermaid Diagram (Code)}
\begin{verbatim}
{Shaded}
{Highlighting}[]
graph LR
    J {-{-}{} A[AND1]}
    CLK {-{-}{} A}
    QB {-{-}{} A}
    A {-{-}{} B[NOR1]}
    B {-{-}{} Q}
    Q {-{-}{} C[NOR2]}
    K {-{-}{} D[AND2]}
    CLK {-{-}{} D}
    Q {-{-}{} D}
    D {-{-}{} C}
    C {-{-}{} QB[Q{}]}
{Highlighting}
{Shaded}
\end{verbatim}
\end{center}

\textbf{Truth Table:}

{\def\LTcaptype{none} % do not increment counter
\begin{longtable}[]{@{}llll@{}}
\toprule\noalign{}
\textbf{J} & \textbf{K} & \textbf{Q(next)} & \textbf{Operation} \\
\midrule\noalign{}
\endhead
\bottomrule\noalign{}
\endlastfoot
0 & 0 & Q & Hold \\
0 & 1 & 0 & Reset \\
1 & 0 & 1 & Set \\
1 & 1 & Q' & Toggle \\
\end{longtable}
}

\begin{itemize}
\tightlist
\item
  \textbf{Toggle mode}: J=K=1 flips output state
\item
  \textbf{Clock enable}: Active only when CLK=1
\item
  \textbf{Feedback}: Uses current output to enable inputs
\end{itemize}

\end{solutionbox}
\begin{mnemonicbox}
``JK Toggles, No Forbidden State''

\end{mnemonicbox}
\begin{center}\rule{0.5\linewidth}{0.5pt}\end{center}

\subsection*{Question 4(c OR) [7
marks]}\label{question-4c-or-7-marks}

\textbf{Explain the terms Lithography, Etching, Deposition, Oxidation,
Ion implantation, Diffusion}

\begin{solutionbox}
Semiconductor fabrication processes essential for
creating integrated circuits.

{\def\LTcaptype{none} % do not increment counter
\begin{longtable}[]{@{}
  >{\raggedright\arraybackslash}p{(\linewidth - 4\tabcolsep) * \real{0.3421}}
  >{\raggedright\arraybackslash}p{(\linewidth - 4\tabcolsep) * \real{0.3421}}
  >{\raggedright\arraybackslash}p{(\linewidth - 4\tabcolsep) * \real{0.3158}}@{}}
\toprule\noalign{}
\begin{minipage}[b]{\linewidth}\raggedright
\textbf{Process}
\end{minipage} & \begin{minipage}[b]{\linewidth}\raggedright
\textbf{Purpose}
\end{minipage} & \begin{minipage}[b]{\linewidth}\raggedright
\textbf{Method}
\end{minipage} \\
\midrule\noalign{}
\endhead
\bottomrule\noalign{}
\endlastfoot
\textbf{Lithography} & Pattern transfer & UV exposure through masks \\
\textbf{Etching} & Material removal & Wet/dry chemical processes \\
\textbf{Deposition} & Layer addition & CVD, PVD, sputtering \\
\textbf{Oxidation} & Insulator growth & Thermal/plasma oxidation \\
\textbf{Ion Implantation} & Doping introduction & High-energy ion
bombardment \\
\textbf{Diffusion} & Dopant distribution & High-temperature spreading \\
\end{longtable}
}

\begin{itemize}
\tightlist
\item
  \textbf{Pattern definition}: Lithography creates device features
\item
  \textbf{Selective removal}: Etching removes unwanted material\\
\item
  \textbf{Layer building}: Deposition adds required materials
\item
  \textbf{Doping control}: Implantation and diffusion create junctions
\item
  \textbf{Quality control}: Each step affects final device performance
\end{itemize}

\end{solutionbox}
\begin{mnemonicbox}
``Light Etches Deposited Oxides, Ions Diffuse''

\end{mnemonicbox}
\begin{center}\rule{0.5\linewidth}{0.5pt}\end{center}

\subsection*{Question 5(a) [3 marks]}\label{q5a}

\textbf{Implement 2 input XNOR gate using Verilog}

\begin{solutionbox}
XNOR gate produces high output when inputs are equal.

\begin{verbatim}
module xnor\_gate(
    input a, b,
    output y
);
    assign y = {(}a \^{} b);
endmodule
\end{verbatim}

\begin{itemize}
\tightlist
\item
  \textbf{Logic function}: Y = (A \oplus B)' = A'B' + AB
\item
  \textbf{Truth table}: Output high when inputs match
\item
  \textbf{Applications}: Equality comparator, parity checker
\end{itemize}

\end{solutionbox}
\begin{mnemonicbox}
``XNOR Equals Equal Inputs''

\end{mnemonicbox}
\begin{center}\rule{0.5\linewidth}{0.5pt}\end{center}

\subsection*{Question 5(b) [4 marks]}\label{q5b}

\textbf{Implement Encoder (8:3) using CASE statement in Verilog}

\begin{solutionbox}
Priority encoder converts 8-bit input to 3-bit binary
output.

\begin{verbatim}
module encoder\_8to3(
    input [7:0] in,
    output reg [2:0] out
);
    always @(*) begin
        case(in)
            8{b00000001}: out = 3{b000};
            8{b00000010}: out = 3{b001};
            8{b00000100}: out = 3{b010};
            8{b00001000}: out = 3{b011};
            8{b00010000}: out = 3{b100};
            8{b00100000}: out = 3{b101};
            8{b01000000}: out = 3{b110};
            8{b10000000}: out = 3{b111};
            default: out = 3{b000};
        endcase
    end
endmodule
\end{verbatim}

\begin{itemize}
\tightlist
\item
  \textbf{One-hot encoding}: Only one input bit should be high
\item
  \textbf{Priority structure}: Higher bits take precedence
\item
  \textbf{Default case}: Handles invalid input combinations
\end{itemize}

\end{solutionbox}
\begin{mnemonicbox}
``One Hot Input, Binary Output''

\end{mnemonicbox}
\begin{center}\rule{0.5\linewidth}{0.5pt}\end{center}

\subsection*{Question 5(c) [7 marks]}\label{q5c}

\textbf{Explain CASE statement in Verilog with suitable examples}

\begin{solutionbox}
CASE statement provides multi-way branching based on
expression value.

\textbf{Syntax:}

\begin{verbatim}
case (expression)
    value1: statement1;
    value2: statement2;
    default: default\_statement;
endcase
\end{verbatim}

\textbf{Example 1 - 4:1 MUX:}

\begin{verbatim}
module mux\_4to1(
    input [1:0] sel,
    input [3:0] in,
    output reg out
);
    always @(*) begin
        case(sel)
            2{b00}: out = in[0];
            2{b01}: out = in[1];
            2{b10}: out = in[2];
            2{b11}: out = in[3];
        endcase
    end
endmodule
\end{verbatim}

\textbf{Example 2 - 7-Segment Decoder:}

\begin{verbatim}
case(digit)
    4{h0}: segments = 7{b1111110};
    4{h1}: segments = 7{b0110000};
    4{h2}: segments = 7{b1101101};
    default: segments = 7{b0000000};
endcase
\end{verbatim}

{\def\LTcaptype{none} % do not increment counter
\begin{longtable}[]{@{}lll@{}}
\toprule\noalign{}
\textbf{Variant} & \textbf{Syntax} & \textbf{Use Case} \\
\midrule\noalign{}
\endhead
\bottomrule\noalign{}
\endlastfoot
\textbf{case} & case(expr) & Full matching \\
\textbf{casex} & casex(expr) & Don't care (X) \\
\textbf{casez} & casez(expr) & High-Z (Z) \\
\end{longtable}
}

\begin{itemize}
\tightlist
\item
  \textbf{Combinational logic}: Use always @(*) block
\item
  \textbf{Sequential logic}: Use always @(posedge clk)
\item
  \textbf{Default case}: Prevents latches in synthesis
\item
  \textbf{Parallel evaluation}: All cases checked simultaneously
\end{itemize}

\end{solutionbox}
\begin{mnemonicbox}
``CASE Chooses Actions Systematically Everywhere''

\end{mnemonicbox}
\begin{center}\rule{0.5\linewidth}{0.5pt}\end{center}

\subsection*{Question 5(a OR) [3
marks]}\label{question-5a-or-3-marks}

\textbf{Implement full subtractor using Verilog code}

\begin{solutionbox}
Full subtractor performs binary subtraction with borrow
input and output.

\begin{verbatim}
module full\_subtractor(
    input a, b, bin,
    output diff, bout
);
    assign diff = a \^{} b \^{} bin;
    assign bout = ({}a \& b) | ({}a \& bin) | (b \& bin);
endmodule
\end{verbatim}

\textbf{Truth Table:}

{\def\LTcaptype{none} % do not increment counter
\begin{longtable}[]{@{}lllll@{}}
\toprule\noalign{}
\textbf{A} & \textbf{B} & \textbf{Bin} & \textbf{Diff} &
\textbf{Bout} \\
\midrule\noalign{}
\endhead
\bottomrule\noalign{}
\endlastfoot
0 & 0 & 0 & 0 & 0 \\
0 & 0 & 1 & 1 & 1 \\
0 & 1 & 0 & 1 & 1 \\
1 & 1 & 1 & 1 & 1 \\
\end{longtable}
}

\begin{itemize}
\tightlist
\item
  \textbf{Difference}: XOR of all three inputs
\item
  \textbf{Borrow}: Generated when A \textless{} (B + Bin)
\end{itemize}

\end{solutionbox}
\begin{mnemonicbox}
``Subtract Borrows When Insufficient''

\end{mnemonicbox}
\begin{center}\rule{0.5\linewidth}{0.5pt}\end{center}

\subsection*{Question 5(b OR) [4
marks]}\label{question-5b-or-4-marks}

\textbf{Implement JK flipflop using Behavioural modeling style in
Verilog}

\begin{solutionbox}
JK flip-flop with toggle capability using behavioral
modeling.

\begin{verbatim}
module jk\_flipflop(
    input j, k, clk, reset,
    output reg q, qbar
);
    always @(posedge clk or posedge reset) begin
        if(reset) begin
            q {=} 1{b0};
            qbar {=} 1{b1};
        end
        else begin
            case(\{j,k\)}
                2{b00}: q {=} q;        // Hold
                2{b01}: q {=} 1{b0};     // Reset
                2{b10}: q {=} 1{b1};     // Set
                2{b11}: q {=} {}q;       // Toggle
            endcase
            qbar {=} {}q;
        end
    end
endmodule
\end{verbatim}

\begin{itemize}
\tightlist
\item
  \textbf{Behavioral style}: Describes function, not structure
\item
  \textbf{Synchronous reset}: Reset on clock edge
\item
  \textbf{Non-blocking assignment}: Use \textless= in clocked always
  block
\end{itemize}

\end{solutionbox}
\begin{mnemonicbox}
``JK Behavior: Hold, Reset, Set, Toggle''

\end{mnemonicbox}
\begin{center}\rule{0.5\linewidth}{0.5pt}\end{center}

\subsection*{Question 5(c OR) [7
marks]}\label{question-5c-or-7-marks}

\textbf{Explain different Verilog modeling style with examples}

\begin{solutionbox}
Verilog provides three modeling styles for different
abstraction levels.

{\def\LTcaptype{none} % do not increment counter
\begin{longtable}[]{@{}llll@{}}
\toprule\noalign{}
\textbf{Style} & \textbf{Abstraction} & \textbf{Description} &
\textbf{Constructs} \\
\midrule\noalign{}
\endhead
\bottomrule\noalign{}
\endlastfoot
\textbf{Behavioral} & High & Describes function & always, if-else,
case \\
\textbf{Dataflow} & Medium & Describes data movement & assign,
operators \\
\textbf{Structural} & Low & Describes connections & module
instantiation \\
\end{longtable}
}

\textbf{1. Behavioral Modeling:} Describes what the circuit does, not
how it's built.

\begin{verbatim}
// 4{-bit counter}
module counter(
    input clk, reset,
    output reg [3:0] count
);
    always @(posedge clk or posedge reset) begin
        if(reset)
            count {=} 4{b0000};
        else
            count {=} count + 1;
    end
endmodule
\end{verbatim}

\textbf{2. Dataflow Modeling:} Uses continuous assignments for
combinational logic.

\begin{verbatim}
// 4{-bit adder}
module adder\_4bit(
    input [3:0] a, b,
    input cin,
    output [3:0] sum,
    output cout
);
    assign \{cout, sum\} = a + b + cin;
    assign overflow = (a[3] \& b[3] \& {}sum[3]) | 
                     ({}a[3] \& {}b[3] \& sum[3]);
endmodule
\end{verbatim}

\textbf{3. Structural Modeling:} Instantiates and connects lower-level
modules.

\begin{verbatim}
// Full adder using half adders
module full\_adder(
    input a, b, cin,
    output sum, cout
);
    wire s1, c1, c2;
    
    half\_adder ha1(.a(a), .b(b), .sum(s1), .carry(c1));
    half\_adder ha2(.a(s1), .b(cin), .sum(sum), .carry(c2));
    
    assign cout = c1 | c2;
endmodule

module half\_adder(
    input a, b,
    output sum, carry
);
    assign sum = a \^{} b;
    assign carry = a \& b;
endmodule
\end{verbatim}

\textbf{Comparison Table:}

{\def\LTcaptype{none} % do not increment counter
\begin{longtable}[]{@{}llll@{}}
\toprule\noalign{}
\textbf{Aspect} & \textbf{Behavioral} & \textbf{Dataflow} &
\textbf{Structural} \\
\midrule\noalign{}
\endhead
\bottomrule\noalign{}
\endlastfoot
\textbf{Complexity} & High-level & Medium-level & Low-level \\
\textbf{Readability} & Most readable & Moderate & Least readable \\
\textbf{Synthesis} & Tool dependent & Direct & Explicit \\
\textbf{Debugging} & Harder & Moderate & Easier \\
\textbf{Reusability} & High & Medium & High \\
\end{longtable}
}

\textbf{Mixed Modeling Example:}

\begin{verbatim}
module cpu\_alu(
    input [7:0] a, b,
    input [2:0] opcode,
    input clk,
    output reg [7:0] result
);
    // Behavioral: Control logic
    always @(posedge clk) begin
        case(opcode)
            3{b000}: result {=} add\_result;
            3{b001}: result {=} sub\_result;
            3{b010}: result {=} and\_result;
            default: result {=} 8{h00};
        endcase
    end
    
    // Dataflow: Arithmetic operations
    wire [7:0] add\_result = a + b;
    wire [7:0] sub\_result = a {-} b;
    wire [7:0] and\_result = a \& b;
    
    // Structural: Could instantiate dedicated arithmetic units
endmodule
\end{verbatim}

\textbf{Design Guidelines:}

\begin{itemize}
\tightlist
\item
  \textbf{Behavioral}: Use for complex control logic, state machines
\item
  \textbf{Dataflow}: Use for simple combinational logic
\item
  \textbf{Structural}: Use for hierarchical designs, IP integration
\item
  \textbf{Mixed approach}: Combine styles for optimal design
\end{itemize}

\textbf{Simulation vs Synthesis:}

\begin{itemize}
\tightlist
\item
  \textbf{Behavioral}: May not synthesize as expected
\item
  \textbf{Dataflow}: Direct hardware mapping
\item
  \textbf{Structural}: Guaranteed synthesis match
\end{itemize}

\end{solutionbox}
\begin{mnemonicbox}
``Behavior Describes, Dataflow Assigns, Structure
Connects''

\end{mnemonicbox}

\end{document}
