\documentclass{article}

% content/resources/templates/preamble.tex
\usepackage[margin=0.6in]{geometry}
\author{Milav Dabgar}
\usepackage{amsmath,amssymb,amsthm}
\usepackage{booktabs}
\usepackage{multirow}
\usepackage{xcolor}
\usepackage{tcolorbox}
\tcbuselibrary{breakable,skins}
\usepackage[colorlinks=true,linkcolor=blue]{hyperref}
\usepackage{titlesec}
\usepackage{enumitem}
\usepackage{tikz}
\usepackage{pgfplots}
\usepackage{circuitikz}
\usepackage[version=4]{mhchem}
\usepackage{longtable}
\usepackage{array}
\usepackage{float}
\usepackage{caption}
\usepackage{listings}

\lstset{
  basicstyle=\small\ttfamily,
  breaklines=true,
  breakatwhitespace=false,
  postbreak=\mbox{\textcolor{red}{$\hookrightarrow$}\space},
  float=false,
  numbers=left,
  numberstyle=\tiny\color{gray},
  numbersep=10pt,
  xleftmargin=2em,
  keywordstyle=\color{blue},
  commentstyle=\color{green!60!black},
  stringstyle=\color{purple},
  backgroundcolor=\color{gray!5},
  showstringspaces=false,
  tabsize=2,
  captionpos=b,
  keepspaces=true,
  columns=flexible
}

\pgfplotsset{compat=1.18}
\usetikzlibrary{shapes,arrows,positioning,calc,patterns,decorations.pathmorphing,decorations.markings,arrows.meta}

% Color scheme
\definecolor{headcolor}{RGB}{0,102,204}
\definecolor{keycolor}{RGB}{220,20,60}
\definecolor{solutioncolor}{RGB}{34,139,34}
\definecolor{mnemoniccolor}{RGB}{148,0,211}
\definecolor{codecolor}{RGB}{0,0,100}

% Spacing
\setlength{\parskip}{3pt}
\setlist[itemize]{nosep}
\setlist[enumerate]{nosep}

% Title formatting
\titleformat{\section}{\Large\bfseries\color{headcolor}}{\thesection}{1em}{}
\titleformat{\subsection}{\large\bfseries\color{headcolor}}{\thesubsection}{1em}{}

% Pandoc tightlist compatibility
\providecommand{\tightlist}{%
  \setlength{\itemsep}{0pt}\setlength{\parskip}{0pt}}

% Pandoc longtable compatibility
\newcounter{none}
\def\thenone{}


% content/resources/templates/gujarati-boxes.tex
\usepackage{fontspec}
\usepackage{polyglossia}

% Set Gujarati as main language (document is primarily in Gujarati)
% Note: gloss-gujarati.ldf doesn't exist in polyglossia, but it will use hyphenation patterns
\setdefaultlanguage{gujarati}
\setotherlanguage{english}

% Configure Gujarati font properly
% Use Language=Default to prevent polyglossia from trying to add language-specific features
% that don't exist for Gujarati, which causes "empty feature" warnings
\newfontfamily\gujaratifont[Script=Gujarati,AutoFakeBold=2.5,AutoFakeSlant=0.3]{Noto Sans Gujarati}
\setmainfont[Script=Gujarati,AutoFakeBold=2.5,AutoFakeSlant=0.3]{Noto Sans Gujarati}
% Use Noto Sans Gujarati for monospace to support Gujarati in text
\setmonofont[Scale=0.9]{Noto Sans Gujarati}

% Configure English to use the same font
\newfontfamily\englishfont[Script=Gujarati,AutoFakeBold=2.5,AutoFakeSlant=0.3]{Noto Sans Gujarati}

% Translations for polyglossia
\gappto\captionsgujarati{
  \renewcommand{\tablename}{કોષ્ટક}
  \renewcommand{\figurename}{આકૃતિ}
}

% Helper for TikZ nodes to ensure Gujarati font
\newcommand{\gu}[1]{{\gujaratifont #1}}

% Custom environments
\newtcolorbox{solutionbox}{
    breakable,
    enhanced,
    colback=solutioncolor!5!white,
    colframe=solutioncolor!75!black,
    fonttitle=\bfseries,
    title=જવાબ
}

\newtcolorbox{solutionboxnobreak}{
 colback=solutioncolor!5!white,
 colframe=solutioncolor!75!black,
 fonttitle=\bfseries,
 title=જવાબ
}

\newtcolorbox{keyformula}{
 breakable,
 enhanced,
 colback=keycolor!5!white,
 colframe=keycolor!75!black,
 fonttitle=\bfseries,
 title=રાસાયણિક સમીકરણ/સૂત્ર
}

\newtcolorbox{mnemonicbox}{
 breakable,
 enhanced,
 colback=mnemoniccolor!5!white,
 colframe=mnemoniccolor!75!black,
 fonttitle=\bfseries,
 title=મેમરી ટ્રીક
}


% Custom commands for GTU solutions
% This file defines semantic commands for consistent formatting

% Question command with automatic formatting
\newcommand{\question}[2]{%
  \section*{Question #1}%
  \textbf{#2}%
}

% OR question variant
\newcommand{\questionor}[2]{%
  \section*{Question #1 OR}%
  \textbf{#2}%
}

% Proper table environment with caption
\newenvironment{answertable}[1]{%
  \begin{table}[htbp]
  \centering
  \caption{#1}
}{%
  \end{table}
}

% Proper figure environment for diagrams
\newenvironment{answerdiagram}[1]{%
  \begin{figure}[htbp]
  \centering
  \caption{#1}
}{%
  \end{figure}
}

% Semantic markup for key terms
\newcommand{\keyword}[1]{\textbf{#1}}
\newcommand{\code}[1]{\texttt{#1}}
\newcommand{\classname}[1]{\texttt{#1}}
\newcommand{\methodname}[1]{\texttt{#1}}

% Proper quotation marks
\newcommand{\mnemonic}[1]{``#1''}


\title{VLSI Technology (4361102) - Summer 2025 Solution}
\date{May 12, 2025}

\begin{document}
\maketitle

% Question 1
\questionmarks{1(a)}{3}{\textenglish{State importance of scaling}}
\begin{solutionbox}
સ્કેલિંગ \textenglish{semiconductor technology} ને આગળ વધારવા અને \textenglish{device performance} સુધારવા માટે અત્યંત મહત્વપૂર્ણ છે.

\begin{answertable}{\textenglish{Scaling Benefits}}
\begin{tabulary}{\textwidth}{|L|L|}
\hline
\textbf{સ્કેલિંગ ફાયદા} & \textbf{વર્ણન} \\
\hline
\textbf{\textenglish{Device Size}} & ઊંચી \textenglish{density} માટે \textenglish{transistor dimensions} ઘટાડે છે \\
\hline
\textbf{\textenglish{Speed}} & ટૂંકી \textenglish{channel length} થી ઝડપી \textenglish{switching} \\
\hline
\textbf{\textenglish{Power}} & પ્રતિ \textenglish{operation} ઓછો \textenglish{power consumption} \\
\hline
\textbf{\textenglish{Cost}} & વધુ \textenglish{chips per wafer}, \textenglish{function} દીઠ ઓછો \textenglish{cost} \\
\hline
\end{tabulary}
\end{answertable}

\begin{itemize}
    \item \keyword{\textenglish{Technology advancement}}: \textenglish{Moore's Law} ચાલુ રાખવામાં સક્ષમ બનાવે છે
    \item \keyword{\textenglish{Performance boost}}: ઊંચી \textenglish{frequency operation} શક્ય બનાવે છે
    \item \keyword{\textenglish{Market competitiveness}}: નાના, ઝડપી, સસ્તા \textenglish{products}
\end{itemize}
\end{solutionbox}

\begin{mnemonicbox}
\mnemonic{Small Devices Speed Progress Cheaply}
\end{mnemonicbox}

\questionmarks{1(b)}{4}{\textenglish{Compare Planar MOSFET and FINFET}}
\begin{solutionbox}
\textenglish{FinFET technology} નાના \textenglish{nodes} પર \textenglish{planar MOSFET} ની મર્યાદાઓનો ઉકેલ આપે છે.

\begin{answertable}{\textenglish{Planar MOSFET vs FinFET}}
\begin{tabulary}{\textwidth}{|L|L|L|}
\hline
\textbf{પેરામીટર} & \textbf{\textenglish{Planar MOSFET}} & \textbf{\textenglish{FinFET}} \\
\hline
\textbf{\textenglish{Structure}} & \textenglish{2D flat channel} & \textenglish{3D fin-shaped channel} \\
\hline
\textbf{\textenglish{Gate Control}} & \textenglish{Single gate} & \textenglish{Tri-gate/multi-gate} \\
\hline
\textbf{\textenglish{Short Channel Effects}} & નાના \textenglish{nodes} પર ઊંચી & નોંધપાત્ર રીતે ઓછી \\
\hline
\textbf{\textenglish{Leakage Current}} & ઊંચી \textenglish{subthreshold leakage} & ખૂબ ઓછી \textenglish{leakage} \\
\hline
\end{tabulary}
\end{answertable}

\begin{itemize}
    \item \keyword{\textenglish{Scalability}}: \textenglish{FinFET sub-22nm technology nodes} શક્ય બનાવે છે
    \item \keyword{\textenglish{Power efficiency}}: \textenglish{FinFET} વધુ સારો \textenglish{power-performance ratio} આપે છે
    \item \keyword{\textenglish{Manufacturing}}: \textenglish{FinFET} વધુ જટિલ \textenglish{fabrication} માંગે છે
\end{itemize}
\end{solutionbox}

\begin{mnemonicbox}
\mnemonic{Fins Control Current Better Than Flat}
\end{mnemonicbox}

\questionmarks{1(c)}{7}{\textenglish{Draw and Explain VDS-ID AND VGS-ID characteristics of N channel MOSFET}}
\begin{solutionbox}
\textenglish{N-channel MOSFET characteristics} અલગ અલગ \textenglish{operating regions} માં \textenglish{device behavior} દર્શાવે છે.

\textbf{\textenglish{Diagram}:}

\begin{answerdiagram}{\textenglish{MOSFET Characteristics}}
\begin{tikzpicture}
\begin{scope}
    \begin{axis}[
        title={\textenglish{VGS-ID Characteristics}},
        xlabel={$V_{GS}$},
        ylabel={$I_D$},
        xmin=0, xmax=5,
        ymin=0, ymax=10,
        axis lines=middle,
        ticks=none
    ]
    \addplot[domain=2:5, blue, thick] {0.5*(x-2)^2};
    \node at (axis cs:2,0) [below] {$V_T$};
    \end{axis}
\end{scope}

\begin{scope}[xshift=7cm]
    \begin{axis}[
        title={\textenglish{VDS-ID Characteristics}},
        xlabel={$V_{DS}$},
        ylabel={$I_D$},
        xmin=0, xmax=5,
        ymin=0, ymax=10,
        axis lines=middle,
        ticks=none
    ]
    \addplot[domain=0:5, blue, thick] {x < 1 ? 2*x - x^2 : 1};
    \addplot[domain=0:5, red, thick] {x < 2 ? 4*x - x^2 : 4};
    \addplot[domain=0:5, green, thick] {x < 3 ? 6*x - x^2 : 9};
    
    \node at (axis cs:4.5, 1.2) {\tiny $V_{GS1}$};
    \node at (axis cs:4.5, 4.2) {\tiny $V_{GS2}$};
    \node at (axis cs:4.5, 9.2) {\tiny $V_{GS3}$};
    
    \draw[dashed] (axis cs:1,1) -- (axis cs:3,9);
    \node at (axis cs:2, 6) [rotate=60] {\tiny \textenglish{Saturation start}};
    \end{axis}
\end{scope}
\end{tikzpicture}
\end{answerdiagram}

\begin{answertable}{\textenglish{MOSFET Operating Regions}}
\begin{tabulary}{\textwidth}{|L|L|L|}
\hline
\textbf{પ્રદેશ} & \textbf{સ્થિતિ} & \textbf{કરંટ સમીકરણ} \\
\hline
\textbf{\textenglish{Cutoff}} & $V_{GS} < V_T$ & $I_D = 0$ \\
\hline
\textbf{\textenglish{Linear}} & $V_{DS} < (V_{GS}-V_T)$ & $I_D \propto V_{DS}$ \\
\hline
\textbf{\textenglish{Saturation}} & $V_{DS} \ge (V_{GS}-V_T)$ & $I_D \propto (V_{GS}-V_T)^2$ \\
\hline
\end{tabulary}
\end{answertable}

\begin{itemize}
    \item \keyword{\textenglish{Cutoff}}: કોઈ પ્રવાહ વહેતો નથી, \textenglish{open switch} તરીકે વર્તે છે.
    \item \keyword{\textenglish{Linear/Triode}}: પ્રવાહ $V_{DS}$ સાથે રેખીય રીતે વધે છે, \textenglish{resistor} તરીકે વર્તે છે.
    \item \keyword{\textenglish{Saturation}}: પ્રવાહ અચળ રહે છે, $V_{DS}$ થી સ્વતંત્ર, \textenglish{current source} તરીકે વર્તે છે.
\end{itemize}
\end{solutionbox}

\begin{mnemonicbox}
\mnemonic{Threshold Gates Linear Saturation}
\end{mnemonicbox}

\begin{center}
\textbf{\large OR}
\end{center}

\questionmarks{1(c)}{7}{\textenglish{Explain different condition of MOS under external bias}}
\begin{solutionbox}
\textenglish{External bias} અલગ અલગ \textenglish{charge distributions} બનાવે છે જે \textenglish{MOS capacitor behavior} ને અસર કરે છે.

\textbf{\textenglish{Diagram}:}

\begin{answerdiagram}{\textenglish{MOS Bias Modes}}
\begin{tikzpicture}[node distance=1.5cm]
    \node (mos) [draw, rectangle] {\textenglish{MOS Under Bias}};
    
    \node (acc) [draw, rectangle, below of=mos, xshift=-4cm] {\textenglish{Accumulation} $V_G < 0$};
    \node (dep) [draw, rectangle, below of=mos] {\textenglish{Depletion} $0 < V_G < V_T$};
    \node (inv) [draw, rectangle, below of=mos, xshift=4cm] {\textenglish{Inversion} $V_G > V_T$};
    
    \node (acc_desc) [below of=acc] {સપાટી પર \textenglish{holes} એકઠા થાય છે};
    \node (dep_desc) [below of=dep] {સપાટી \textenglish{carriers} થી ખાલી};
    \node (inv_desc) [below of=inv] {\textenglish{Electron inversion layer} બને છે};
    
    \draw[->] (mos) -- (acc);
    \draw[->] (mos) -- (dep);
    \draw[->] (mos) -- (inv);
    
    \draw[->] (acc) -- (acc_desc);
    \draw[->] (dep) -- (dep_desc);
    \draw[->] (inv) -- (inv_desc);
\end{tikzpicture}
\end{answerdiagram}

\begin{answertable}{\textenglish{MOS Operating Modes}}
\begin{tabulary}{\textwidth}{|L|L|L|}
\hline
\textbf{બાયસ સ્થિતિ} & \textbf{સપાટીની સ્થિતિ} & \textbf{કેપેસિટન્સ} \\
\hline
\textbf{\textenglish{Accumulation}} & સપાટી પર \textenglish{majority carriers} & ઊંચી ($C_{ox}$) \\
\hline
\textbf{\textenglish{Depletion}} & કોઈ \textenglish{mobile carriers} નથી & મધ્યમ \\
\hline
\textbf{\textenglish{Inversion}} & \textenglish{Minority carriers channel} બનાવે છે & ઊંચી ($C_{ox}$) \\
\hline
\end{tabulary}
\end{answertable}

\begin{itemize}
    \item \keyword{ફ્લેટ બેન્ડ વોલ્ટેજ}: કોઈ \textenglish{charge separation} અસ્તિત્વમાં નથી
    \item \keyword{એનર્જી બેન્ડ બેન્ડિંગ}: \textenglish{carrier distribution} નક્કી કરે છે
    \item \keyword{સપાટીનો વિભવ}: \textenglish{inversion layer formation} નિયંત્રિત કરે છે
\end{itemize}
\end{solutionbox}

\begin{mnemonicbox}
\mnemonic{Accumulate, Deplete, then Invert}
\end{mnemonicbox}% Question 2
% Question 2
\questionmarks{2(a)}{3}{\textenglish{Draw voltage transfer characteristic of ideal inverter}}
\begin{solutionbox}
આદર્શ ઇન્વર્ટર \textenglish{infinite gain} સાથે \textenglish{logic levels} વચ્ચે તીક્ષ્ણ પરિવર્તન આપે છે.

\textbf{\textenglish{Diagram}:}

\begin{answerdiagram}{\textenglish{Ideal Inverter VTC}}
\begin{tikzpicture}
    \begin{axis}[
        title={\textenglish{Ideal Inverter VTC}},
        xlabel={$V_{IN}$},
        ylabel={$V_{OUT}$},
        xmin=0, xmax=5,
        ymin=0, ymax=5,
        axis lines=middle,
        ticks=none,
        width=8cm, height=8cm
    ]
    % Ideal VTC
    \draw[blue, thick] (axis cs:0,5) -- (axis cs:2.5,5) -- (axis cs:2.5,0) -- (axis cs:5,0);
    
    \node at (axis cs:0,5) [left] {$V_{OH}$};
    \node at (axis cs:5,0) [below] {$V_{OL}$};
    \node at (axis cs:2.5,0) [below] {$V_{TH}$};
    
    \draw[dashed] (axis cs:1,0) -- (axis cs:1,5);
    \node at (axis cs:1,0) [below] {$V_{IL}$};
    
    \draw[dashed] (axis cs:4,0) -- (axis cs:4,5);
    \node at (axis cs:4,0) [below] {$V_{IH}$};

    \end{axis}
\end{tikzpicture}
\end{answerdiagram}

\begin{itemize}
    \item \keyword{તીક્ષ્ણ પરિવર્તન}: \textenglish{switching point} પર \textenglish{infinite slope}
    \item \keyword{નોઈઝ માર્જિન}: $NMH = V_{OH} - V_{IH}$, $NML = V_{IL} - V_{OL}$
    \item \keyword{સંપૂર્ણ લોજિક લેવલ}: $V_{OH} = V_{DD}$, $V_{OL} = 0V$
\end{itemize}
\end{solutionbox}

\begin{mnemonicbox}
\mnemonic{Sharp Switch, Perfect Levels}
\end{mnemonicbox}

\questionmarks{2(b)}{4}{\textenglish{Explain noise immunity and noise margin}}
\begin{solutionbox}
નોઈઝ ઇમ્યુનિટી \textenglish{circuit} ની અનચાહેલા \textenglish{signal variations} ને નકારવાની ક્ષમતા માપે છે.

\begin{answertable}{\textenglish{Noise Parameters}}
\begin{tabulary}{\textwidth}{|L|L|L|}
\hline
\textbf{પેરામીટર} & \textbf{વ્યાખ્યા} & \textbf{ફોર્મ્યુલા} \\
\hline
\textbf{\textenglish{NMH}} & હાઈ-લેવલ નોઈઝ માર્જિન & $V_{OH} - V_{IH}$ \\
\hline
\textbf{\textenglish{NML}} & લો-લેવલ નોઈઝ માર્જિન & $V_{IL} - V_{OL}$ \\
\hline
\textbf{નોઈઝ ઇમ્યુનિટી} & નોઈઝ નકારવાની ક્ષમતા & $Min(NMH, NML)$ \\
\hline
\end{tabulary}
\end{answertable}

\begin{itemize}
    \item \keyword{લોજિક થ્રેશોલ્ડ લેવલ}: $V_{IH}$ (\textenglish{input high}), $V_{IL}$ (\textenglish{input low})
    \item \keyword{આઉટપુટ લેવલ}: $V_{OH}$ (\textenglish{output high}), $V_{OL}$ (\textenglish{output low})
    \item \keyword{વધુ સારી ઇમ્યુનિટી}: મોટા નોઈઝ માર્જિન વધુ સારી સુરક્ષા આપે છે
    \item \keyword{ડિઝાઇન લક્ષ્ય}: મજબૂત \textenglish{operation} માટે નોઈઝ માર્જિન વધારવા
\end{itemize}
\end{solutionbox}

\begin{mnemonicbox}
\mnemonic{Margins Protect Against Noise}
\end{mnemonicbox}

\questionmarks{2(c)}{7}{\textenglish{Describe inverter circuit with saturated and linear depletion load nMOS inverter}}
\begin{solutionbox}
\textenglish{Depletion load nMOS} ઇન્વર્ટર \textenglish{active load resistor} તરીકે \textenglish{depletion transistor} વાપરે છે.

\textbf{\textenglish{Diagram}:}

\begin{answerdiagram}{\textenglish{Depletion Load Inverter}}
\begin{circuitikz}[american]
    \draw (0,0) node[ground] {} -- (0,0.5) node[nmos, anchor=source] (driver) {};
    \node[right=0.5cm] at (driver) {\textenglish{MN (Driver)}};
    
    \draw (driver.drain) -- ++(0,0.5) node[nmos, anchor=source] (load) {};
    \node[right=0.5cm] at (load) {\textenglish{MD (Depletion)}};
    
    \draw (load.drain) -- ++(0,0.5) node[vcc] {\textenglish{VDD}};
    
    \draw[thick] ($(load.source)!0.5!(load.drain)$) ++(-0.2,0) -- ++(0,0.6);
    
    \draw (load.gate) -- ++(-0.5,0) coordinate (gd);
    \draw (gd) |- (load.source);
    
    \draw (driver.gate) to[short,-o] ++(-0.5,0) node[left] {$V_{IN}$};
    \draw (driver.drain) to[short,*-o] ++(1.5,0) node[right] {$V_{OUT}$};
\end{circuitikz}
\end{answerdiagram}

\begin{answertable}{\textenglish{Load Operation Modes}}
\begin{tabulary}{\textwidth}{|L|L|L|}
\hline
\textbf{લોડ પ્રકાર} & \textbf{ગેટ કનેક્શન} & \textbf{ઓપરેશન} \\
\hline
\textbf{\textenglish{Saturated Load}} & $V_G = V_D$ & હંમેશા \textenglish{saturation} માં \\
\hline
\textbf{\textenglish{Linear Load}} & $V_G = V_{DD}$ & \textenglish{Linear region} માં કામ કરી શકે છે \\
\hline
\end{tabulary}
\end{answertable}

\begin{itemize}
    \item \keyword{ડિપ્લીશન ડિવાઇસ}: $V_{GS} = 0$ સાથે વહન કરે છે, \textenglish{current source} તરીકે કામ કરે છે
    \item \keyword{લોડ લાઇન વિશ્લેષણ}: \textenglish{operating point intersection} નક્કી કરે છે
    \item \keyword{પાવર કન્ઝ્યુમ્પશન}: હંમેશા વહન કરે છે, ઊંચો \textenglish{static power}
    \item \keyword{સ્વિચિંગ સ્પીડ}: \textenglish{pull-up} કરતાં \textenglish{pull-down} ઝડપી
\end{itemize}
\end{solutionbox}

\begin{mnemonicbox}
\mnemonic{Depletion Loads Drive Outputs}
\end{mnemonicbox}

\begin{center}
\textbf{\large OR}
\end{center}

\questionmarks{2(a)}{3}{\textenglish{Draw and explain enhancement load inverter}}
\begin{solutionbox}
\textenglish{Enhancement load} ઇન્વર્ટર ખાસ \textenglish{biasing} સાથે \textenglish{enhancement MOSFET} ને \textenglish{load} તરીકે વાપરે છે.

\textbf{\textenglish{Diagram}:}

\begin{answerdiagram}{\textenglish{Enhancement Load Inverter}}
\begin{circuitikz}[american]
    \draw (0,0) node[ground] {} -- (0,0.5) node[nmos, anchor=source] (driver) {};
    \draw (driver.drain) -- ++(0,0.5) node[nmos, anchor=source] (load) {};
    \draw (load.drain) -- ++(0,0.5) node[vcc] {\textenglish{VDD}};
    
    \draw (load.gate) -- (load.drain);
    \draw (load.gate) ++(0,0) node[circ] {};
    
    \draw (driver.gate) to[short,-o] ++(-0.5,0) node[left] {$V_{IN}$};
    \draw (driver.drain) to[short,*-o] ++(1,0) node[right] {$V_{OUT}$};
    
    \node[right=0.5cm] at (driver) {\textenglish{Driver}};
    \node[right=0.5cm] at (load) {\textenglish{Enhancement Load}};
\end{circuitikz}
\end{answerdiagram}

\begin{itemize}
    \item \keyword{બૂટસ્ટ્રેપ કનેક્શન}: લોડ માટે \textenglish{gate} ને \textenglish{drain} સાથે જોડાયેલ
    \item \keyword{મર્યાદિત આઉટપુટ હાઈ}: $V_{OUT(max)} = V_{DD} - V_T$
    \item \keyword{થ્રેશોલ્ડ નુકસાન}: \textenglish{Enhancement load} વોલ્ટેજ ડ્રોપ કરાવે છે
\end{itemize}
\end{solutionbox}

\begin{mnemonicbox}
\mnemonic{Enhancement Loses Threshold}
\end{mnemonicbox}

\questionmarks{2(b)}{4}{\textenglish{List the advantages of CMOS inverter}}
\begin{solutionbox}
\textenglish{CMOS technology NMOS} ઇન્વર્ટર કરતાં શ્રેષ્ઠ \textenglish{performance} આપે છે.

\begin{answertable}{\textenglish{CMOS Advantages}}
\begin{tabulary}{\textwidth}{|L|L|}
\hline
\textbf{ફાયદો} & \textbf{લાભ} \\
\hline
\textbf{શૂન્ય સ્ટેટિક પાવર} & \textenglish{steady state} માં કોઈ \textenglish{current path} નથી \\
\hline
\textbf{રેલ-ટુ-રેલ આઉટપુટ} & સંપૂર્ણ $V_{DD}$ અને 0V આઉટપુટ લેવલ \\
\hline
\textbf{ઊંચી નોઈઝ ઇમ્યુનિટી} & મોટા નોઈઝ માર્જિન \\
\hline
\textbf{સમપ્રમાણ સ્વિચિંગ} & બરાબર \textenglish{rise} અને \textenglish{fall times} \\
\hline
\end{tabulary}
\end{answertable}

\begin{itemize}
    \item \keyword{પાવર એફિશિયન્સી}: માત્ર \textenglish{switching} દરમિયાન \textenglish{dynamic power}
    \item \keyword{સ્કેલેબિલિટી}: બધા \textenglish{technology nodes} પર સારી રીતે કામ કરે છે
    \item \keyword{ફેન-આઉટ ક્ષમતા}: અનેક \textenglish{inputs} ડ્રાઇવ કરી શકે છે
    \item \keyword{તાપમાન સ્થિરતા}: \textenglish{performance} તાપમાન પર ઓછી સંવેદનશીલ
\end{itemize}
\end{solutionbox}

\begin{mnemonicbox}
\mnemonic{CMOS Saves Power Perfectly}
\end{mnemonicbox}

\questionmarks{2(c)}{7}{\textenglish{Draw and Explain operating mode of region for CMOS Inverter}}
\begin{solutionbox}
\textenglish{CMOS} ઇન્વર્ટર \textenglish{operation input voltage} ના આધારે પાંચ અલગ અલગ \textenglish{regions} સમાવે છે.

\textbf{\textenglish{Diagram}:}

\begin{answerdiagram}{\textenglish{CMOS Operation Regions}}
\begin{tikzpicture}[node distance=1.5cm]
    \node (cmos) [draw, rectangle] {\textenglish{CMOS Inverter Regions}};
    
    \node (reg3) [draw, rectangle, below of=cmos] {\textenglish{Region 3: Switching}};
    \node (reg2) [draw, rectangle, left of=reg3, xshift=-3cm] {\textenglish{Region 2: Sat/Sat}};
    \node (reg1) [draw, rectangle, left of=reg2, xshift=-1cm] {\textenglish{Region 1: PMOS On}};
    \node (reg4) [draw, rectangle, right of=reg3, xshift=3cm] {\textenglish{Region 4: Sat/Sat}};
    \node (reg5) [draw, rectangle, right of=reg4, xshift=1cm] {\textenglish{Region 5: NMOS On}};
    
    \draw[->] (cmos) -- (reg3);
    \draw[->] (cmos) -- (reg2);
    \draw[->] (cmos) -- (reg1);
    \draw[->] (cmos) -- (reg4);
    \draw[->] (cmos) -- (reg5);
    
\end{tikzpicture}
\end{answerdiagram}

\begin{answertable}{\textenglish{CMOS Operating Regions}}
\begin{tabulary}{\textwidth}{|L|L|L|L|}
\hline
\textbf{પ્રદેશ} & \textbf{\textenglish{NMOS} સ્થિતિ} & \textbf{\textenglish{PMOS} સ્થિતિ} & \textbf{આઉટપુટ} \\
\hline
\textbf{1} & \textenglish{OFF} & \textenglish{Linear} & $V_{OH} \approx V_{DD}$ \\
\hline
\textbf{2} & \textenglish{Saturation} & \textenglish{Saturation} & \textenglish{Transition} \\
\hline
\textbf{3} & \textenglish{Saturation} & \textenglish{Saturation} & $V_{DD}/2$ \\
\hline
\textbf{4} & \textenglish{Saturation} & \textenglish{Saturation} & \textenglish{Transition} \\
\hline
\textbf{5} & \textenglish{Linear} & \textenglish{OFF} & $V_{OL} \approx 0V$ \\
\hline
\end{tabulary}
\end{answertable}

\begin{itemize}
    \item \keyword{સ્વિચિંગ થ્રેશોલ્ડ}: \textenglish{VTC region 3} પર $V_{DD}/2$ ને પાર કરે છે
    \item \keyword{કરંટ ફ્લો}: માત્ર \textenglish{transition regions 2,3,4} દરમિયાન
    \item \keyword{નોઈઝ માર્જિન}: \textenglish{Regions 1} અને \textenglish{5} ઇમ્યુનિટી આપે છે
    \item \keyword{ગેઇન}: \textenglish{Region 3} માં મહત્તમ (\textenglish{switching point})
\end{itemize}
\end{solutionbox}

% Question 3
\questionmarks{3(a)}{3}{\textenglish{Draw two input NOR gate using CMOS}}
\begin{solutionbox}
\textenglish{CMOS NOR} ગેટ \textenglish{complementary networks} વાપરીને \textenglish{De Morgan's law} અમલમાં મૂકે છે.

\textbf{\textenglish{Diagram}:}

\begin{answerdiagram}{\textenglish{CMOS NOR2 Gate}}
\begin{circuitikz}
    % PUN: Series PMOS
    \draw (0,4) node[vcc] {VDD} to[Tpmos, n=p1, l=MP1 (A)] (0,3);
    \draw (0,3) to[Tpmos, n=p2, l=MP2 (B)] (0,2);
    
    % PDN: Parallel NMOS
    \draw (0,2) -- (0,1.5);
    \draw (-1.5,1.5) -- (1.5,1.5);
    \draw (-1.5,1.5) to[Tnmos, n=n1, l=MN1 (A)] (-1.5,0.5);
    \draw (1.5,1.5) to[Tnmos, n=n2, l=MN2 (B)] (1.5,0.5);
    \draw (-1.5,0.5) -- (1.5,0.5);
    \draw (0,0.5) -- (0,0) node[ground] {};
    
    % Output
    \draw (0,2) to[short, *-o] (2.5,2) node[right] {$Y = \overline{A+B}$};
    
    % Input Connections
    \draw (p1.gate) -- ++(-1,0) node[left] {A};
    \draw (n1.gate) -- ++(-0.5,0) node[left] {A};
    \draw (p2.gate) -- ++(-1,0) node[left] {B};
    \draw (n2.gate) -- ++(0.5,0) node[right] {B};
\end{circuitikz}
\end{answerdiagram}

\begin{itemize}
    \item \keyword{પુલ-અપ નેટવર્ક}: સીરીઝ \textenglish{PMOS transistors} (હાઈ આઉટપુટ માટે A અને B બંને લો)
    \item \keyword{પુલ-ડાઉન નેટવર્ક}: પેરેલલ \textenglish{NMOS transistors} (લો આઉટપુટ માટે A અથવા B હાઈ)
    \item \keyword{લોજિક ફંક્શન}: $Y = (A+B)' = A' \cdot B'$
\end{itemize}
\end{solutionbox}

\begin{mnemonicbox}
\mnemonic{Series PMOS, Parallel NMOS}
\end{mnemonicbox}

\questionmarks{3(b)}{4}{\textenglish{Implement Boolean function Z= [(A+B)C+DE]' using CMOS}}
\begin{solutionbox}
જટિલ \textenglish{CMOS} લોજિક કાર્યક્ષમ અમલીકરણ માટે \textenglish{AOI (AND-OR-Invert)} સ્ટ્રક્ચર વાપરે છે.

\textbf{\textenglish{Diagram}:}

\begin{answerdiagram}{\textenglish{CMOS Implementation of $Z = [(A+B)C+DE]'$}}
\begin{circuitikz}
    % PDN
    \draw (0,0) node[ground] {} -- (0,0.5);
    % Branch 1: D series E
    \draw (0,0.5) -- (2,0.5) to[Tnmos, l=E] (2,1.5) to[Tnmos, l=D] (2,2.5) -- (0,2.5);
    
    % Branch 2: (A || B) series C
    \draw (0,0.5) -- (-2,0.5) to[Tnmos, l=C] (-2,1.5);
    % Parallel A, B
    \draw (-2,1.5) -- (-3,1.5) to[Tnmos, l=A] (-3,2.5) -- (-2,2.5);
    \draw (-2,1.5) -- (-1,1.5) to[Tnmos, l=B] (-1,2.5) -- (-2,2.5);
    \draw (-2,2.5) -- (0,2.5);
    
    % Output node
    \draw (0,2.5) to[short, *-o] (3,2.5) node[right] {Z};
    
    % PUN: Dual of PDN
    \draw (0,2.5) -- (0,3);
    
    % Block 2: D || E
    \draw (0,3) -- (-1,3) to[Tpmos, l=D] (-1,4) -- (0,4);
    \draw (0,3) -- (1,3) to[Tpmos, l=E] (1,4) -- (0,4);
    
    % Block 1: (A series B) || C
    \draw (0,4) -- (0,4.5);
    % Path C
    \draw (0,4.5) -- (1.5,4.5) to[Tpmos, l=C] (1.5,6.5) -- (0,6.5);
    % Path A series B
    \draw (0,4.5) -- (-1.5,4.5) to[Tpmos, l=B] (-1.5,5.5) to[Tpmos, l=A] (-1.5,6.5) -- (0,6.5);
    
    \draw (0,6.5) node[vcc] {VDD};
    
\end{circuitikz}
\end{answerdiagram}

\begin{itemize}
    \item \keyword{\textenglish{AOI} સ્ટ્રક્ચર}: કાર્યક્ષમ \textenglish{single-stage} અમલીકરણ
    \item \keyword{ડ્યુઅલ નેટવર્ક}: \textenglish{complementary pull-up} અને \textenglish{pull-down}
    \item \keyword{લોજિક ઓપ્ટિમાઇઝેશન}: અલગ ગેટ કરતાં ઓછા \textenglish{transistors}
\end{itemize}
\end{solutionbox}

\begin{mnemonicbox}
\mnemonic{AOI Inverts Complex Logic Efficiently}
\end{mnemonicbox}

\questionmarks{3(c)}{7}{\textenglish{Draw and explain CMOS NAND2 gate with the parasitic device capacitances}}
\begin{solutionbox}
\textenglish{CMOS} ગેટમાં પેરાસિટિક કેપેસિટન્સ \textenglish{switching speed} અને \textenglish{power consumption} ને અસર કરે છે.

\textbf{\textenglish{Diagram}:}

\begin{answerdiagram}{\textenglish{CMOS NAND2 with Parasitics}}
\begin{circuitikz}
    % CMOS NAND2
    % Parallel PMOS
    \draw (0,5) node[vcc] {VDD} -- (0,4.5);
    \draw (-1.5,4.5) -- (1.5,4.5);
    \draw (-1.5,4.5) to[Tpmos, l=MP1] (-1.5,3);
    \draw (1.5,4.5) to[Tpmos, l=MP2] (1.5,3);
    \draw (-1.5,3) -- (1.5,3);
    \draw (0,3) -- (0,2.5);
    
    % Series NMOS
    \draw (0,2.5) to[Tnmos, l=MN2] (0,1.5);
    \draw (0,1.5) to[Tnmos, l=MN1] (0,0.5) node[ground] {};
    
    % Output
    \draw (0,2.75) to[short, *-o] (3,2.75) node[right] {$Y$};
    
    % Parasitic Capacitors
    \draw (-1.5,4) to[C, l=$C_{gd1}$, color=red] (-0.5,4);
    \draw (1.5,4) to[C, l=$C_{gd2}$, color=red] (0.5,4);
    
    % Load Cap
    \draw (2.5,2.75) to[C, l=$C_{load}$, color=red] (2.5,0.5) node[ground] {};
    
    % Cgs, Cdb etc can be indicated generally or specific nodes
    \node[right, red] at (0.5, 1) {$C_{gd}, C_{db}$};
    
\end{circuitikz}
\end{answerdiagram}

\begin{answertable}{\textenglish{Parasitic Capacitances}}
\begin{tabulary}{\textwidth}{|L|L|L|}
\hline
\textbf{કેપેસિટન્સ} & \textbf{સ્થાન} & \textbf{અસર} \\
\hline
\textbf{Cgs} & \textenglish{Gate-Source} & \textenglish{Input capacitance} \\
\hline
\textbf{Cgd} & \textenglish{Gate-Drain} & \textenglish{Miller effect} \\
\hline
\textbf{Cdb} & \textenglish{Drain-Bulk} & \textenglish{Output loading} \\
\hline
\textbf{Csb} & \textenglish{Source-Bulk} & \textenglish{Source loading} \\
\hline
\end{tabulary}
\end{answertable}

\begin{itemize}
    \item \keyword{સ્વિચિંગ વિલંબ}: પેરાસિટિક કેપેસિટન્સ \textenglish{transitions} ધીમા કરે છે
    \item \keyword{પાવર કન્ઝ્યુમ્પશન}: પેરાસિટિક \textenglish{caps} ચાર્જ/ડિસ્ચાર્જ કરવા
    \item \keyword{મિલર ઇફેક્ટ}: \textenglish{Cgd feedback} બનાવે છે, \textenglish{switching} ધીમું કરે છે
    \item \keyword{લેઆઉટ ઓપ્ટિમાઇઝેશન}: પેરાસિટિક કેપેસિટન્સ ઓછા કરવા
\end{itemize}
\end{solutionbox}

\begin{mnemonicbox}
\mnemonic{Parasitics Slow Gates Down}
\end{mnemonicbox}

\begin{center}
\textbf{\large OR}
\end{center}

\questionmarks{3(a)}{3}{\textenglish{Draw and explain NOR based Clocked SR latch using CMOS}}
\begin{solutionbox}
\textenglish{Clocked SR latch synchronous operation} માટે \textenglish{clock enable} સાથે \textenglish{NOR gates} વાપરે છે.

\textbf{\textenglish{Diagram}:}

\begin{answerdiagram}{\textenglish{Clocked SR Latch}}
\begin{circuitikz}
    \node[nor port] (nor1) at (0,2) {};
    \node[nor port] (nor2) at (0,-2) {};
    
    % Input NORs for Clock gating
    \node[and port] (and1) at (-3, 2.3) {}; % Actually logic is usually AND-NOR or just extra inputs to NOR?
    % The question says "NOR based". Usually means SR latch (cross coupled NORs) with AND gates for clock, OR simply NORs with clock?
    % Standard implementation: S connects to AND with CLK, output to NOR latch.
    % Let's use logic gates for clarity.
    
    \node[and port] (gate1) at (-3, 2) {};
    \node[and port] (gate2) at (-3, -2) {};
    
    % Connections
    \draw (gate1.out) -- (nor1.in 1);
    \draw (gate2.out) -- (nor2.in 2);
    
    % Cross coupling
    \draw (nor1.out) -- ++(0.5,0) coordinate (q) -- ++(0,-1) -- ($(nor2.in 1) + (-0.5, 1.5)$) -- (nor2.in 1);
    \draw (nor2.out) -- ++(0.5,0) coordinate (qb) -- ++(0,1) -- ($(nor1.in 2) + (-0.5, -1.5)$) -- (nor1.in 2);
    
    % Inputs
    \draw (gate1.in 1) -- ++(-0.5,0) node[left] {S};
    \draw (gate2.in 2) -- ++(-0.5,0) node[left] {R};
    
    % Clock
    \draw (gate1.in 2) -- ++(-0.5,0) coordinate (clk1);
    \draw (gate2.in 1) -- ++(-0.5,0) coordinate (clk2);
    \draw (clk1) -- (clk2); 
    \draw ($(clk1)!0.5!(clk2)$) -- ++(-1,0) node[left] {CLK};
    
    % Outputs
    \draw (nor1.out) -- ++(1,0) node[right] {Q};
    \draw (nor2.out) -- ++(1,0) node[right] {Q'};
    
\end{circuitikz}
\caption{\textenglish{Clocked SR Latch}}
\end{answerdiagram}

\begin{itemize}
    \item \keyword{ક્લોક કંટ્રોલ}: S અને R માત્ર CLK = 1 હોય ત્યારે જ અસરકારક
    \item \keyword{ટ્રાન્સપેરન્ટ મોડ}: \textenglish{clock} સક્રિય હોય ત્યારે આઉટપુટ \textenglish{input} ને અનુસરે છે
    \item \keyword{હોલ્ડ મોડ}: \textenglish{clock} નિષ્ક્રિય હોય ત્યારે આઉટપુટ સ્થિતિ જાળવે છે
    \item \keyword{મૂળભૂત બિલ્ડિંગ બ્લોક}: \textenglish{flip-flops} માટે પાયો
\end{itemize}
\end{solutionbox}

\begin{mnemonicbox}
\mnemonic{Clock Controls Transparent Latching}
\end{mnemonicbox}

\questionmarks{3(b)}{4}{\textenglish{Implement Boolean function Z=[AB+C(D+E)]' using CMOS}}
\begin{solutionbox}
આ ફંક્શન \textenglish{AOI} લોજિક સ્ટ્રક્ચર વાપરીને \textenglish{inverted sum-of-products} અમલમાં મૂકે છે.

\textbf{\textenglish{Logic Analysis}:}
$Z = [AB + C(D+E)]' = [AB + CD + CE]'$

\textbf{\textenglish{Diagram}:}

\begin{answerdiagram}{\textenglish{CMOS Implementation of $Z=[AB+C(D+E)]'$}}
\begin{circuitikz}
    % PDN: AB + CD + CE
    % Or better: AB + C(D+E) as per equation
    % (A series B) || (C series (D || E))
    \draw (0,0) node[ground] {} -- (0,0.5);
    
    % Branch 1: A series B
    \draw (0,0.5) -- (-2,0.5) to[Tnmos, l=B] (-2,1.5) to[Tnmos, l=A] (-2,2.5) -- (0,2.5);
    
    % Branch 2: C series (D || E)
    \draw (0,0.5) -- (2,0.5);
    % D || E
    \draw (2,0.5) -- (1,0.5) to[Tnmos, l=D] (1,1.5) -- (2,1.5);
    \draw (2,0.5) -- (3,0.5) to[Tnmos, l=E] (3,1.5) -- (2,1.5);
    % Series C
    \draw (2,1.5) to[Tnmos, l=C] (2,2.5) -- (0,2.5);
    
    % Output
    \draw (0,2.5) to[short, *-o] (3,2.5) node[right] {Z};
    
    % PUN: Dual of PDN
    \draw (0,2.5) -- (0,3);
    
    % Block 1: A || B
    \draw (0,3) -- (-1.5,3) to[Tpmos, l=A] (-1.5,4) -- (0,4);
    \draw (0,3) -- (1.5,3) to[Tpmos, l=B] (1.5,4) -- (0,4);
    
    % Block 2: C || (D series E)
    \draw (0,4) -- (0,4.5);
    % Path C
    \draw (0,4.5) -- (-1.5,4.5) to[Tpmos, l=C] (-1.5,6.5) -- (0,6.5);
    % Path D series E
    \draw (0,4.5) -- (1.5,4.5) to[Tpmos, l=E] (1.5,5.5) to[Tpmos, l=D] (1.5,6.5) -- (0,6.5);
    
    \draw (0,6.5) node[vcc] {VDD};
    
\end{circuitikz}
\end{answerdiagram}

\begin{answertable}{\textenglish{Logic Terms}}
\begin{tabulary}{\textwidth}{|L|L|L|}
\hline
\textbf{ટર્મ} & \textbf{ઇનપુટ્સ} & \textbf{ફંક્શન} \\
\hline
\textbf{ટર્મ 1} & A, B & AB \\
\hline
\textbf{ટર્મ 2} & C, D & CD \\
\hline
\textbf{ટર્મ 3} & C, E & CE \\
\hline
\textbf{આઉટપુટ} & બધા \textenglish{terms} & (AB + CD + CE)' \\
\hline
\end{tabulary}
\end{answertable}

\begin{itemize}
    \item \keyword{\textenglish{AOI} અમલીકરણ}: \textenglish{single stage}, કાર્યક્ષમ ડિઝાઇન
    \item \keyword{ટ્રાન્ઝિસ્ટર કાઉન્ટ}: અલગ ગેટ અમલીકરણ કરતાં ઓછા
    \item \keyword{પરફોર્મન્સ}: ઝડપી \textenglish{switching}, ઓછો \textenglish{power}
\end{itemize}
\end{solutionbox}

\begin{mnemonicbox}
\mnemonic{Three AND Terms Feed One NOR}
\end{mnemonicbox}

\questionmarks{3(c)}{7}{\textenglish{Differentiate AOI and OAI logic with suitable example}}
\begin{solutionbox}
\textenglish{AOI} અને \textenglish{OAI} કાર્યક્ષમ \textenglish{CMOS} અમલીકરણ માટે પૂરક લોજિક પરિવારો છે.

\begin{answertable}{\textenglish{AOI vs OAI}}
\begin{tabulary}{\textwidth}{|L|L|L|}
\hline
\textbf{પેરામીટર} & \textbf{AOI (AND-OR-Invert)} & \textbf{OAI (OR-AND-Invert)} \\
\hline
\textbf{લોજિક ફંક્શન} & \textenglish{Sum of Products (SOP) inverted} & \textenglish{Product of Sums (POS) inverted} \\
\hline
\textbf{સમીકરણ} & $Y = (AB + CD)'$ & $Y = ((A+B)(C+D))'$ \\
\hline
\textbf{\textenglish{PDN} સ્ટ્રક્ચર} & \textenglish{Parallel branches of series NMOS} & \textenglish{Series branches of parallel NMOS} \\
\hline
\textbf{\textenglish{PUN} સ્ટ્રક્ચર} & \textenglish{Series branches of parallel PMOS} & \textenglish{Parallel branches of series PMOS} \\
\hline
\textbf{ઝડપ} & \textenglish{Typically faster for SOP} & \textenglish{Typically faster for POS} \\
\hline
\end{tabulary}
\end{answertable}

\textbf{\textenglish{Example}:}
\begin{itemize}
    \item \textbf{AOI}: $Y = (AB + CD)'$
    \begin{itemize}
        \item \textenglish{NMOS}: \textenglish{Series A-B in parallel with Series C-D}
        \item \textenglish{PMOS}: \textenglish{Parallel A,B in series with Parallel C,D}
    \end{itemize}
    \item \textbf{OAI}: $Y = ((A+B)(C+D))'$
    \begin{itemize}
        \item \textenglish{PMOS}: \textenglish{Parallel A,B in series with Parallel C,D}
        \item \textenglish{NMOS}: \textenglish{Series A-B in parallel with Series C-D}
    \end{itemize}
\end{itemize}

\begin{itemize}
    \item \keyword{ડિઝાઇન પસંદગી}: બૂલિયન ફંક્શન ફોર્મ આધારે પસંદ કરો
    \item \keyword{ઓપ્ટિમાઇઝેશન}: ટ્રાન્ઝિસ્ટર કાઉન્ટ અને વિલંબ ઓછો કરે છે
    \item \keyword{દ્વૈતતા}: \textenglish{AOI} અને \textenglish{OAI} \textenglish{De Morgan duals} છે
\end{itemize}
\end{solutionbox}

\begin{mnemonicbox}
\mnemonic{AOI ANDs then ORs, OAI ORs then ANDs}
\end{mnemonicbox}

% Question 4
\section*{પ્રશ્ન 4}

\questionmarks{4(a)}{3}{\textenglish{Define: 1) Regularity 2) Modularity 3) Locality}}
\begin{solutionbox}
VLSI જટિલતાને સંચાલિત કરવા અને સફળ અમલીકરણ સુનિશ્ચિત કરવા માટે ડિઝાઇન હાયરાર્કી સિદ્ધાંતો આવશ્યક છે.

\begin{answertable}{\textenglish{Design Principles}}
\begin{tabulary}{\textwidth}{|L|L|L|}
\hline
\textbf{સિદ્ધાંત} & \textbf{વ્યાખ્યા} & \textbf{ફાયદો} \\
\hline
\textbf{Regularity} & સમાન સ્ટ્રક્ચર્સનો વારંવાર ઉપયોગ & સરળ લેઆઉટ, ટેસ્ટિંગ \\
\hline
\textbf{Modularity} & ડિઝાઇનને નાના બ્લોક્સમાં વહેંચવું & સ્વતંત્ર ડિઝાઇન, પુનઃઉપયોગ \\
\hline
\textbf{Locality} & મોટે ભાગે સ્થાનિક \textenglish{interconnections} & ઓછી \textenglish{routing} જટિલતા \\
\hline
\end{tabulary}
\end{answertable}

\begin{itemize}
    \item \keyword{ડિઝાઇન કાર્યક્ષમતા}: સિદ્ધાંતો ડિઝાઇન સમય અને પ્રયાસ ઘટાડે છે
    \item \keyword{વેરિફિકેશન}: મોડ્યુલર અભિગમ ટેસ્ટિંગ સરળ બનાવે છે
    \item \keyword{સ્કેલેબિલિટી}: મોટા, વધુ જટિલ ડિઝાઇન્સ શક્ય બનાવે છે
\end{itemize}
\end{solutionbox}

\begin{mnemonicbox}
\mnemonic{Regular Modules Stay Local}
\end{mnemonicbox}

\questionmarks{4(b)}{4}{\textenglish{Implement SR latch (NAND gate) using CMOS inverter}}
\begin{solutionbox}
NAND ગેટ વાપરીને SR latch \textenglish{active-low inputs} સાથે \textenglish{set-reset functionality} આપે છે.

\textbf{\textenglish{Diagram}:}

\begin{answerdiagram}{\textenglish{NAND SR Latch}}
\begin{circuitikz}
    \node[nand port] (nand1) at (0,2) {};
    \node[nand port] (nand2) at (0,-2) {};
    
    % Cross coupling
    \draw (nand1.out) -- ++(0.5,0) coordinate (q) -- ++(0,-1.5) -- ($(nand2.in 1) + (-0.5, 1)$) -- (nand2.in 1);
    \draw (nand2.out) -- ++(0.5,0) coordinate (qb) -- ++(0,1.5) -- ($(nand1.in 2) + (-0.5, -1)$) -- (nand1.in 2);
    
    % Inputs
    \draw (nand1.in 1) -- ++(-1,0) node[left] {S'};
    \draw (nand2.in 2) -- ++(-1,0) node[left] {R'};
    
    % Outputs
    \draw (nand1.out) -- ++(1,0) node[right] {Q};
    \draw (nand2.out) -- ++(1,0) node[right] {Q'};
\end{circuitikz}
\end{answerdiagram}

\begin{answertable}{\textenglish{NAND SR Latch Truth Table}}
\begin{tabulary}{\textwidth}{|C|C|C|C|L|}
\hline
\textbf{S'} & \textbf{R'} & \textbf{Q} & \textbf{Q'} & \textbf{સ્થિતિ} \\
\hline
0 & 1 & 1 & 0 & Set \\
\hline
1 & 0 & 0 & 1 & Reset \\
\hline
1 & 1 & Q & Q' & Hold \\
\hline
0 & 0 & 1 & 1 & અમાન્ય \\
\hline
\end{tabulary}
\end{answertable}

\begin{itemize}
    \item \keyword{ક્રોસ-કપ્લ્ડ સ્ટ્રક્ચર}: મેમરી ફંક્શન આપે છે
    \item \keyword{એક્ટિવ-લો ઇનપુટ્સ}: S' = 0 સેટ કરે છે, R' = 0 રીસેટ કરે છે
    \item \keyword{પ્રતિબંધિત સ્થિતિ}: બંને ઇનપુટ્સ એકસાથે લો
\end{itemize}
\end{solutionbox}

\begin{mnemonicbox}
\mnemonic{Cross-Coupled NANDS Remember State}
\end{mnemonicbox}

\questionmarks{4(c)}{7}{\textenglish{Explain VLSI design flow}}
\begin{solutionbox}
VLSI ડિઝાઇન ફ્લો \textenglish{specification} થી \textenglish{fabrication} સુધીના વ્યવસ્થિત પગલાંઓ અનુસરે છે.

\textbf{\textenglish{Diagram}:}

\begin{answerdiagram}{\textenglish{VLSI Design Flow}}
\begin{tikzpicture}[node distance=1.5cm, auto]
    \node (spec) [draw, rectangle, align=center] {સિસ્ટમ સ્પેસિફિકેશન};
    \node (arch) [draw, rectangle, below of=spec] {આર્કિટેક્ચરલ ડિઝાઇન};
    \node (func) [draw, rectangle, below of=arch] {ફંક્શનલ ડિઝાઇન};
    \node (logic) [draw, rectangle, below of=func] {લોજિક ડિઝાઇન};
    \node (circuit) [draw, rectangle, below of=logic] {સર્કિટ ડિઝાઇન};
    \node (phys) [draw, rectangle, below of=circuit] {ફિઝિકલ ડિઝાઇન};
    \node (fab) [draw, rectangle, below of=phys] {ફેબ્રિકેશન};
    \node (test) [draw, rectangle, below of=fab] {ટેસ્ટિંગ અને પેકેજિંગ};
    
    \draw[->] (spec) -- (arch);
    \draw[->] (arch) -- (func);
    \draw[->] (func) -- (logic);
    \draw[->] (logic) -- (circuit);
    \draw[->] (circuit) -- (phys);
    \draw[->] (phys) -- (fab);
    \draw[->] (fab) -- (test);
\end{tikzpicture}
\end{answerdiagram}

\begin{answertable}{\textenglish{Design Levels}}
\begin{tabulary}{\textwidth}{|L|L|L|}
\hline
\textbf{લેવલ} & \textbf{પ્રવૃત્તિઓ} & \textbf{આઉટપુટ} \\
\hline
\textbf{સિસ્ટમ} & આવશ્યકતા વિશ્લેષણ & સ્પેસિફિકેશન્સ \\
\hline
\textbf{આર્કિટેક્ચર} & બ્લોક-લેવલ ડિઝાઇન & સિસ્ટમ આર્કિટેક્ચર \\
\hline
\textbf{લોજિક} & બૂલિયન ઓપ્ટિમાઇઝેશન & ગેટ નેટલિસ્ટ \\
\hline
\textbf{સર્કિટ} & ટ્રાન્ઝિસ્ટર સાઇઝિંગ & સર્કિટ નેટલિસ્ટ \\
\hline
\textbf{ફિઝિકલ} & લેઆઉટ, \textenglish{routing} & \textenglish{GDSII} ફાઇલ \\
\hline
\end{tabulary}
\end{answertable}

\begin{itemize}
    \item \keyword{ડિઝાઇન વેરિફિકેશન}: દરેક લેવલે માન્યતા જરૂરી
    \item \keyword{પુનરાવર્તન}: ઓપ્ટિમાઇઝેશન માટે ફીડબેક લૂપ્સ
    \item \keyword{CAD ટૂલ્સ}: જટિલ ડિઝાઇન્સ માટે ઓટોમેશન આવશ્યક
    \item \keyword{ટાઇમ-ટુ-માર્કેટ}: કાર્યક્ષમ ફ્લો ડિઝાઇન સાયકલ ઘટાડે છે
\end{itemize}
\end{solutionbox}

\begin{mnemonicbox}
\mnemonic{System Architects Love Circuit Physical Fabrication}
\end{mnemonicbox}

\begin{center}
\textbf{\large OR}
\end{center}

\questionmarks{4(a)}{3}{\textenglish{Draw and Explain Y-chart}}
\begin{solutionbox}
Y-ચાર્ટ VLSI ડિઝાઇનમાં ત્રણ ડિઝાઇન ડોમેન અને તેમના \textenglish{abstraction levels} દર્શાવે છે.

\textbf{\textenglish{Diagram}:}

\begin{answerdiagram}{\textenglish{Gajski-Kuhn Y-Chart}}
\begin{tikzpicture}
    % Y Chart Grid
    \draw[thick] (0,0) -- (0,4) node[above] {\textbf{\textenglish{Behavioral}}};
    \draw[thick] (0,0) -- (3.46,-2) node[right] {\textbf{\textenglish{Structural}}};
    \draw[thick] (0,0) -- (-3.46,-2) node[left] {\textbf{\textenglish{Physical}}};
    
    % Concentric Circles (Abstraction Levels)
    \foreach \r in {1, 2, 3, 3.5}
        \draw[gray, thin, dashed] (0,0) circle (\r);
        
    % Labels (English for technical simplification)
    \node at (0, 0.5) {\tiny \textenglish{Transistors}};
    \node at (0, 1.5) {\tiny \textenglish{Gates}};
    \node at (0, 2.5) {\tiny \textenglish{Register}};
    \node at (0, 3.2) {\tiny \textenglish{System}};
    
    \node at (60:1.5) [rotate=-30] {\tiny \textenglish{Logic eq}};
    \node at (60:2.5) [rotate=-30] {\tiny \textenglish{Algorithm}};
    
    % Add simplified labels
    \node[align=center] at (1.5, 2) {\textenglish{Algorithm}};
    \node[align=center] at (-1.5, 2) {\textenglish{Layout}};
    \node[align=center] at (0, -2.5) {\textenglish{Circuit/Mask}};
    
\end{tikzpicture}
\end{answerdiagram}

\begin{itemize}
    \item \keyword{ત્રણ ડોમેન}: \textenglish{Behavioral} (ફંક્શન), \textenglish{Structural (components)}, \textenglish{Physical (geometry)}
    \item \keyword{એબ્સ્ટ્રેક્શન લેવલ}: \textenglish{System} $\rightarrow$ \textenglish{Algorithm} $\rightarrow$ \textenglish{Gate} $\rightarrow$ \textenglish{Circuit} $\rightarrow$ \textenglish{Layout}
    \item \keyword{ડિઝાઇન પદ્ધતિ}: સમાન \textenglish{abstraction level} પર ડોમેન વચ્ચે ફરવું
\end{itemize}
\end{solutionbox}

\begin{mnemonicbox}
\mnemonic{Behavior, Structure, Physics at All Levels}
\end{mnemonicbox}

\questionmarks{4(b)}{4}{\textenglish{Implement clocked JK latch (NOR gate) using CMOS inverter}}
\begin{solutionbox}
JK latch \textenglish{toggle} ક્ષમતા સાથે SR latch ની પ્રતિબંધિત સ્થિતિ દૂર કરે છે.

\textbf{\textenglish{Diagram}:}

\begin{answerdiagram}{\textenglish{Clocked JK Latch}}
\begin{circuitikz}
    \node[and port, number inputs=3] (and1) at (-3, 2) {};
    \node[and port, number inputs=3] (and2) at (-3, -2) {};
    
    \node[nor port] (nor1) at (0, 1) {};
    \node[nor port] (nor2) at (0, -1) {};
    
    % Inputs
    \draw (and1.in 1) -- ++(-0.5,0) node[left] {J};
    \draw (and2.in 2) -- ++(-0.5,0) node[left] {K};
    \draw (and1.in 2) -- ++(-0.5,0) coordinate (clk1);
    \draw (and2.in 1) -- ++(-0.5,0) coordinate (clk2);
    \draw (clk1) -- (clk2);
    \draw ($(clk1)!0.5!(clk2)$) -- ++(-0.5,0) node[left] {CLK};
    
    % Connections
    \draw (and1.out) -- (nor1.in 1);
    \draw (and2.out) -- (nor2.in 2);
    
    % Latch Feedback
    \draw (nor1.out) -- ++(0.5,0) coordinate (q) -- ++(0,-0.5) -- ($(nor2.in 1) + (-0.5, 0.5)$) -- (nor2.in 1);
    \draw (nor2.out) -- ++(0.5,0) coordinate (qb) -- ++(0,0.5) -- ($(nor1.in 2) + (-0.5, -0.5)$) -- (nor1.in 2);
    
    % JK Feedback
    \draw (qb) -- ++(0.5,0) -- ++(0,3.5) -- ++(-6,0) |- (and1.in 1);
    \draw (q) -- ++(0.5,0) -- ++(0,-3.5) -- ++(-6,0) |- (and2.in 3);
    
    % Outputs
    \draw (q) -- ++(1,0) node[right] {Q};
    \draw (qb) -- ++(1,0) node[right] {Q'};
\end{circuitikz}
\end{answerdiagram}

\begin{answertable}{\textenglish{JK Latch Truth Table}}
\begin{tabulary}{\textwidth}{|C|C|C|L|}
\hline
\textbf{J} & \textbf{K} & \textbf{Q(next)} & \textbf{ઓપરેશન} \\
\hline
0 & 0 & Q & Hold \\
\hline
0 & 1 & 0 & Reset \\
\hline
1 & 0 & 1 & Set \\
\hline
1 & 1 & Q' & Toggle \\
\hline
\end{tabulary}
\end{answertable}

\begin{itemize}
    \item \keyword{ટોગલ મોડ}: J=K=1 આઉટપુટ સ્થિતિ ફ્લિપ કરે છે
    \item \keyword{ક્લોક એનેબલ}: માત્ર CLK=1 હોય ત્યારે જ સક્રિય
    \item \keyword{ફીડબેક}: ઇનપુટ્સ સક્ષમ કરવા માટે વર્તમાન આઉટપુટ વાપરે છે
\end{itemize}
\end{solutionbox}

\begin{mnemonicbox}
\mnemonic{JK Toggles, No Forbidden State}
\end{mnemonicbox}

\questionmarks{4(c)}{7}{\textenglish{Explain: 1) Lithography 2) Etching 3) Deposition 4) Oxidation 5) Ion Implantation 6) Diffusion}}
\begin{solutionbox}
Integrated circuits બનાવવા માટે આવશ્યક \textenglish{semiconductor fabrication processes}.

\begin{answertable}{\textenglish{Fabrication Processes}}
\begin{tabulary}{\textwidth}{|L|L|L|}
\hline
\textbf{પ્રક્રિયા} & \textbf{હેતુ} & \textbf{પદ્ધતિ} \\
\hline
\textbf{લિથોગ્રાફી} & પેટર્ન ટ્રાન્સફર & માસ્ક દ્વારા \textenglish{UV exposure} \\
\hline
\textbf{એચિંગ} & મેટેરિયલ રિમૂવલ & ભીના/સૂકા રાસાયણિક પ્રક્રિયાઓ \\
\hline
\textbf{જમાવટ} & લેયર ઉમેરો & \textenglish{CVD, PVD, sputtering} \\
\hline
\textbf{ઓક્સિડેશન} & ઇન્સ્યુલેટર વૃદ્ધિ & થર્મલ/પ્લાઝ્મા ઓક્સિડેશન \\
\hline
\textbf{આયન પ્રત્યારોપણ} & ડોપિંગ પરિચય & ઉચ્ચ-ઊર્જા આયન \textenglish{bombardment} \\
\hline
\textbf{પ્રસરણ} & ડોપન્ટ વિતરણ & ઉચ્ચ તાપમાને ફેલાવો \\
\hline
\end{tabulary}
\end{answertable}

\begin{itemize}
    \item \keyword{પેટર્ન વ્યાખ્યા}: લિથોગ્રાફી ડિવાઇસ લક્ષણો બનાવે છે
    \item \keyword{પસંદગીયુક્ત રિમૂવલ}: એચિંગ અનચાહેલ મેટેરિયલ દૂર કરે છે
    \item \keyword{લેયર બિલ્ડિંગ}: જમાવટ જરૂરી મેટેરિયલ ઉમેરે છે
    \item \keyword{ડોપિંગ કંટ્રોલ}: પ્રત્યારોપણ અને પ્રસરણ \textenglish{junctions} બનાવે છે
    \item \keyword{ગુણવત્તા નિયંત્રણ}: દરેક પગલું અંતિમ ડિવાઇસ પરફોર્મન્સને અસર કરે છે
\end{itemize}
\end{solutionbox}

\begin{mnemonicbox}
\mnemonic{Light Etches Deposited Oxides, Ions Diffuse}
\end{mnemonicbox}

% Question 5
\questionmarks{5(a)}{3}{\textenglish{Implement 2 input XNOR gate using Verilog}}
\begin{solutionbox}
XNOR ગેટ ઇનપુટ્સ સમાન હોય ત્યારે ઊંચો (High) આઉટપુટ આપે છે.

\begin{lstlisting}[language=Verilog, caption={\textenglish{Verilog Code for XNOR Gate}}]
module xnor_gate(
    input a, b,
    output y
);
    assign y = ~(a ^ b);
endmodule
\end{lstlisting}

\begin{itemize}
    \item \keyword{લોજિક ફંક્શન}: $Y = (A \oplus B)' = A'B' + AB$
    \item \keyword{હાઈ આઉટપુટ}: જ્યારે બંને ઇનપુટ્સ 0 હોય કે બંને 1 હોય
    \item \keyword{ઉપયોગ}: સમાનતા તુલનાકાર ({Equality comparator}), પેરિટી ચેકર
\end{itemize}
\end{solutionbox}

\begin{mnemonicbox}
\mnemonic{XNOR Equals Equal Inputs}
\end{mnemonicbox}

\questionmarks{5(b)}{4}{\textenglish{Implement Encoder (8:3) using CASE statement in Verilog}}
\begin{solutionbox}
પ્રાયોરિટી એનકોડર 8-બિટ ઇનપુટને 3-બિટ બાઇનરી આઉટપુટમાં રૂપાંતરિત કરે છે, જે માટે \textenglish{behavioral modeling} માં \textenglish{case statement} વપરાય છે.

\begin{lstlisting}[language=Verilog, caption={\textenglish{8:3 Priority Encoder using CASE}}]
module encoder_8to3(
    input [7:0] in,
    output reg [2:0] out
);
    always @(*) begin
        case(in)
            8'b00000001: out = 3'b000;
            8'b00000010: out = 3'b001;
            8'b00000100: out = 3'b010;
            8'b00001000: out = 3'b011;
            8'b00010000: out = 3'b100;
            8'b00100000: out = 3'b101;
            8'b01000000: out = 3'b110;
            8'b10000000: out = 3'b111;
            default: out = 3'b000;
        endcase
    end
endmodule
\end{lstlisting}

\begin{itemize}
    \item \keyword{બિહેવિયરલ મોડેલિંગ}: CASE સ્ટેટમેન્ટ ફંક્શનને વર્ણવે છે
    \item \keyword{કોમ્બિનેશનલ લોજિક}: \code{always @(*)} બ્લોક વપરાય છે
    \item \keyword{ડિફોલ્ટ કેસ}: અમાન્ય સ્થિતિઓ (દા.ત. બધા શૂન્ય) સંભાળે છે
\end{itemize}
\end{solutionbox}

\begin{mnemonicbox}
\mnemonic{One Hot Input, Binary Output}
\end{mnemonicbox}

\questionmarks{5(c)}{7}{\textenglish{Explain case statement in Verilog with suitable examples}}
\begin{solutionbox}
CASE સ્ટેટમેન્ટ એક્સપ્રેશન કિંમત પર આધારિત મલ્ટિ-વે બ્રાન્ચિંગ પૂરું પાડે છે, જે સામાન્ય રીતે કોમ્બિનેશનલ લોજિક (ડિકોડર, મલ્ટિપ્લેક્સર) અને સ્ટેટ મશીન્સમાં વપરાય છે.

\textbf{\textenglish{Syntax}:}
\begin{lstlisting}[language=Verilog]
case (expression)
    value1: statement1;
    value2: statement2;
    default: default_statement;
endcase
\end{lstlisting}

\textbf{\textenglish{Example 1: 4:1 Multiplexer}}
\begin{lstlisting}[language=Verilog]
module mux_4to1(
    input [1:0] sel,
    input [3:0] in,
    output reg out
);
    always @(*) begin
        case(sel)
            2'b00: out = in[0];
            2'b01: out = in[1];
            2'b10: out = in[2];
            2'b11: out = in[3];
        endcase
    end
endmodule
\end{lstlisting}

\textbf{\textenglish{Advantages}:}
\begin{itemize}
    \item \keyword{વાંચનક્ષમતા}: અનેક \code{if-else} સ્ટેટમેન્ટ કરતા વધુ સ્પષ્ટ
    \item \keyword{સિન્થેસિસ}: મલ્ટિપ્લેક્સર કે ROM માં કાર્યક્ષમ રીતે મેપ થાય છે
    \item \keyword{પેરેલલ ઇવેલ્યુએશન}: હાર્ડવેર બધી શરતો એકસાથે તપાસે છે
\end{itemize}

\begin{answertable}{\textenglish{CASE Variants}}
\begin{tabulary}{\textwidth}{|L|L|L|}
\hline
\textbf{વેરિઅન્ટ} & \textbf{સિન્ટેક્સ} & \textbf{ઉપયોગ} \\
\hline
\textbf{case} & \code{case(expr)} & ચોક્કસ બીટ મેચિંગ (0, 1, x, z) \\
\hline
\textbf{casex} & \code{casex(expr)} & 'x' અને 'z' ને \textenglish{don't care} ગણે છે \\
\hline
\textbf{casez} & \code{casez(expr)} & 'z' ને \textenglish{don't care} ગણે છે \\
\hline
\end{tabulary}
\end{answertable}
\end{solutionbox}

\begin{mnemonicbox}
\mnemonic{CASE Chooses Actions Systematically Everywhere}
\end{mnemonicbox}

\end{document}




