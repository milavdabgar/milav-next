\documentclass{article}

% content/resources/templates/preamble.tex
\usepackage[margin=0.6in]{geometry}
\author{Milav Dabgar}
\usepackage{amsmath,amssymb,amsthm}
\usepackage{booktabs}
\usepackage{multirow}
\usepackage{xcolor}
\usepackage{tcolorbox}
\tcbuselibrary{breakable,skins}
\usepackage[colorlinks=true,linkcolor=blue]{hyperref}
\usepackage{titlesec}
\usepackage{enumitem}
\usepackage{tikz}
\usepackage{pgfplots}
\usepackage{circuitikz}
\usepackage[version=4]{mhchem}
\usepackage{longtable}
\usepackage{array}
\usepackage{float}
\usepackage{caption}
\usepackage{listings}

\lstset{
  basicstyle=\small\ttfamily,
  breaklines=true,
  breakatwhitespace=false,
  postbreak=\mbox{\textcolor{red}{$\hookrightarrow$}\space},
  float=false,
  numbers=left,
  numberstyle=\tiny\color{gray},
  numbersep=10pt,
  xleftmargin=2em,
  keywordstyle=\color{blue},
  commentstyle=\color{green!60!black},
  stringstyle=\color{purple},
  backgroundcolor=\color{gray!5},
  showstringspaces=false,
  tabsize=2,
  captionpos=b,
  keepspaces=true,
  columns=flexible
}

\pgfplotsset{compat=1.18}
\usetikzlibrary{shapes,arrows,positioning,calc,patterns,decorations.pathmorphing,decorations.markings,arrows.meta}

% Color scheme
\definecolor{headcolor}{RGB}{0,102,204}
\definecolor{keycolor}{RGB}{220,20,60}
\definecolor{solutioncolor}{RGB}{34,139,34}
\definecolor{mnemoniccolor}{RGB}{148,0,211}
\definecolor{codecolor}{RGB}{0,0,100}

% Spacing
\setlength{\parskip}{3pt}
\setlist[itemize]{nosep}
\setlist[enumerate]{nosep}

% Title formatting
\titleformat{\section}{\Large\bfseries\color{headcolor}}{\thesection}{1em}{}
\titleformat{\subsection}{\large\bfseries\color{headcolor}}{\thesubsection}{1em}{}

% Pandoc tightlist compatibility
\providecommand{\tightlist}{%
  \setlength{\itemsep}{0pt}\setlength{\parskip}{0pt}}

% Pandoc longtable compatibility
\newcounter{none}
\def\thenone{}


% content/resources/templates/gujarati-boxes.tex
\usepackage{fontspec}
\usepackage{polyglossia}

% Set Gujarati as main language (document is primarily in Gujarati)
% Note: gloss-gujarati.ldf doesn't exist in polyglossia, but it will use hyphenation patterns
\setdefaultlanguage{gujarati}
\setotherlanguage{english}

% Configure Gujarati font properly
% Use Language=Default to prevent polyglossia from trying to add language-specific features
% that don't exist for Gujarati, which causes "empty feature" warnings
\newfontfamily\gujaratifont[Script=Gujarati,AutoFakeBold=2.5,AutoFakeSlant=0.3]{Noto Sans Gujarati}
\setmainfont[Script=Gujarati,AutoFakeBold=2.5,AutoFakeSlant=0.3]{Noto Sans Gujarati}
% Use Noto Sans Gujarati for monospace to support Gujarati in text
\setmonofont[Scale=0.9]{Noto Sans Gujarati}

% Configure English to use the same font
\newfontfamily\englishfont[Script=Gujarati,AutoFakeBold=2.5,AutoFakeSlant=0.3]{Noto Sans Gujarati}

% Translations for polyglossia
\gappto\captionsgujarati{
  \renewcommand{\tablename}{કોષ્ટક}
  \renewcommand{\figurename}{આકૃતિ}
}

% Helper for TikZ nodes to ensure Gujarati font
\newcommand{\gu}[1]{{\gujaratifont #1}}

% Custom environments
\newtcolorbox{solutionbox}{
    breakable,
    enhanced,
    colback=solutioncolor!5!white,
    colframe=solutioncolor!75!black,
    fonttitle=\bfseries,
    title=જવાબ
}

\newtcolorbox{solutionboxnobreak}{
 colback=solutioncolor!5!white,
 colframe=solutioncolor!75!black,
 fonttitle=\bfseries,
 title=જવાબ
}

\newtcolorbox{keyformula}{
 breakable,
 enhanced,
 colback=keycolor!5!white,
 colframe=keycolor!75!black,
 fonttitle=\bfseries,
 title=રાસાયણિક સમીકરણ/સૂત્ર
}

\newtcolorbox{mnemonicbox}{
 breakable,
 enhanced,
 colback=mnemoniccolor!5!white,
 colframe=mnemoniccolor!75!black,
 fonttitle=\bfseries,
 title=મેમરી ટ્રીક
}


% Custom commands for GTU solutions
% This file defines semantic commands for consistent formatting

% Question command with automatic formatting
\newcommand{\question}[2]{%
  \section*{Question #1}%
  \textbf{#2}%
}

% OR question variant
\newcommand{\questionor}[2]{%
  \section*{Question #1 OR}%
  \textbf{#2}%
}

% Proper table environment with caption
\newenvironment{answertable}[1]{%
  \begin{table}[htbp]
  \centering
  \caption{#1}
}{%
  \end{table}
}

% Proper figure environment for diagrams
\newenvironment{answerdiagram}[1]{%
  \begin{figure}[htbp]
  \centering
  \caption{#1}
}{%
  \end{figure}
}

% Semantic markup for key terms
\newcommand{\keyword}[1]{\textbf{#1}}
\newcommand{\code}[1]{\texttt{#1}}
\newcommand{\classname}[1]{\texttt{#1}}
\newcommand{\methodname}[1]{\texttt{#1}}

% Proper quotation marks
\newcommand{\mnemonic}[1]{``#1''}

\usetikzlibrary{circuits.ee.IEC}

\title{VLSI (4361102) - Winter 2024 Solution - Gujarati}
\date{November 21, 2024}

\begin{document}
\maketitle

\questionmarks{1}{a}{3}
\textbf{High K FINFET ના ફાયદા લખો.}

\begin{solutionbox}
    \begin{center}
    \captionof{table}{High K FINFET ના ફાયદા}
    \begin{tabulary}{\linewidth}{L L}
        \hline
        \textbf{ફાયદો} & \textbf{વર્ણન} \\
        \hline
        \textbf{ઓછો leakage current} & સારું \textbf{ગેટ કંટ્રોલ} પાવર consumption ઘટાડે છે \\
        \textbf{સુધારેલી performance} & વધુ \textbf{ડ્રાઇવ કરંટ} અને ઝડપી switching \\
        \textbf{વધુ સારી scalability} & \textbf{Moore's law scaling} ચાલુ રાખવાની મંજૂરી આપે છે \\
        \hline
    \end{tabulary}
    \end{center}

    \begin{itemize}
        \item \textbf{High K dielectric}: \textbf{ગેટ leakage} નોંધપાત્ર રીતે ઘટાડે છે
        \item \textbf{3D structure}: \textbf{ચેનલ પર વધુ સારું electrostatic control}
        \item \textbf{ઓછી પાવર}: \textbf{static અને dynamic power} બંને ઘટાડે છે
    \end{itemize}

    \begin{mnemonicbox}
    High Performance, Low Power, Better Control
    \end{mnemonicbox}
\end{solutionbox}

\questionmarks{1}{b}{4}
\textbf{વ્યાખ્યા કરો: (1) pinch off point (2) Threshold Voltage.}

\begin{solutionbox}
    \begin{center}
    \captionof{table}{Key MOSFET Parameters}
    \begin{tabulary}{\linewidth}{L L L}
        \hline
        \textbf{શબ્દ} & \textbf{વ્યાખ્યા} & \textbf{મહત્વ} \\
        \hline
        \textbf{Pinch-off Point} & \textbf{ચેનલ સંપૂર્ણ deplete} થતું સ્થાન & \textbf{Saturation region} માં પ્રવેશ દર્શાવે છે \\
        \textbf{Threshold Voltage} & \textbf{Conducting channel} બનાવવા માટે લઘુતમ $V_{GS}$ & \textbf{ON/OFF switching point} નિર્ધારે છે \\
        \hline
    \end{tabulary}
    \end{center}

    \begin{itemize}
        \item \textbf{Pinch-off point}: $V_{DS} = V_{GS} - V_T$, \textbf{ચેનલ શૂન્ય પહોળાઈ} સુધી સંકુચિત થાય છે
        \item \textbf{Threshold voltage}: \textbf{Enhancement MOSFET} માટે સામાન્ય રીતે 0.7V
        \item \textbf{મહત્વપૂર્ણ parameters}: બંને \textbf{MOSFET operating regions} નિર્ધારે છે
    \end{itemize}

    \begin{mnemonicbox}
    Threshold Turns ON, Pinch-off Points to Saturation
    \end{mnemonicbox}
\end{solutionbox}

\questionmarks{1}{c}{7}
\textbf{MOSFET transistor નું બંધારણ દોરો અને સમજાવો.}

\begin{solutionbox}
    \begin{center}
    \captionof{table}{Structure Components}
    \begin{tabulary}{\linewidth}{L L L}
        \hline
        \textbf{ઘટક} & \textbf{સામગ્રી} & \textbf{કાર્ય} \\
        \hline
        \textbf{Gate} & \textbf{Polysilicon/Metal} & \textbf{ચેનલ formation} કંટ્રોલ કરે છે \\
        \textbf{Gate oxide} & \textbf{SiO2} & \textbf{Gate ને substrate} થી અલગ કરે છે \\
        \textbf{Source/Drain} & \textbf{n+ doped silicon} & \textbf{Current ના પ્રવેશ/બહાર નીકળવાના સ્થળો} \\
        \textbf{Substrate} & \textbf{p-type silicon} & \textbf{Body connection} પૂરું પાડે છે \\
        \hline
    \end{tabulary}
    \end{center}

    \begin{center}
    \begin{tikzpicture}[scale=0.8]
        \draw[fill=gray!10] (0,0) rectangle (8,3);
        \node at (4,1) {p-substrate (Body)};
        
        \draw[fill=white] (1,2) rectangle (2.5,3);
        \node at (1.75,2.5) {n+};
        \node[above] at (1.75,3) {Source};
        
        \draw[fill=white] (5.5,2) rectangle (7,3);
        \node at (6.25,2.5) {n+};
        \node[above] at (6.25,3) {Drain};
        
        \draw[fill=gray!30] (2.5,3) rectangle (5.5,3.5);
        \node at (4,3.25) {SiO$_2$};
        
        \draw[fill=black!10] (2.5,3.5) rectangle (5.5,4.5);
        \node at (4,4) {Gate};
        
        \draw (4,4.5) -- (4,5) node[above] {G};
        \draw (1.75,3) -- (1.75,5) node[above] {S};
        \draw (6.25,3) -- (6.25,5) node[above] {D};
        \draw (4,0) -- (4,-0.5) node[below] {B};
    \end{tikzpicture}
    \end{center}

    \begin{itemize}
        \item \textbf{ચેનલ formation}: \textbf{Oxide-semiconductor interface} પર થાય છે
        \item \textbf{Enhancement mode}: $V_{GS} > V_T$ હોય ત્યારે \textbf{ચેનલ બને છે}
        \item \textbf{ચાર-terminal device}: \textbf{Gate, Source, Drain, Body connections}
    \end{itemize}

    \begin{mnemonicbox}
    Gate Controls, Oxide Isolates, Source-Drain Conducts
    \end{mnemonicbox}
\end{solutionbox}

\questionmarks{1}{c}{7}
\textbf{Full Voltage Scaling અને Constant Voltage Scaling ની સરખામણી કરો.}

\begin{solutionbox}
    \begin{center}
    \captionof{table}{Full Voltage Scaling vs Constant Voltage Scaling}
    \begin{tabulary}{\linewidth}{L L L}
        \hline
        \textbf{Parameter} & \textbf{Full Voltage Scaling} & \textbf{Constant Voltage Scaling} \\
        \hline
        \textbf{Supply voltage} & $\alpha$ વડે \textbf{scale down} & \textbf{સ્થિર રહે છે} \\
        \textbf{Gate oxide thickness} & $\alpha$ વડે \textbf{scale down} & $\alpha$ વડે \textbf{scale down} \\
        \textbf{Channel length} & $\alpha$ વડે \textbf{scale down} & $\alpha$ વડે \textbf{scale down} \\
        \textbf{Power density} & \textbf{સ્થિર રહે છે} & $\alpha^2$ વડે \textbf{વધે છે} \\
        \textbf{Performance} & \textbf{મધ્યમ સુધારો} & \textbf{વધુ સારી performance} \\
        \textbf{Reliability} & \textbf{વધુ સારી} & \textbf{High fields} ને કારણે નબળી \\
        \hline
    \end{tabulary}
    \end{center}

    \begin{itemize}
        \item \textbf{Full scaling}: \textbf{બધા dimensions અને voltages} પ્રમાણસર scale કરાય છે
        \item \textbf{Constant voltage}: \textbf{ફક્ત physical dimensions} scale કરાય છે, voltage અપરિવર્તિત
        \item \textbf{Trade-off}: \textbf{Performance vs power vs reliability}
    \end{itemize}

    \begin{mnemonicbox}
    Full scales All, Constant keeps Voltage
    \end{mnemonicbox}
\end{solutionbox}

\questionmarks{2}{a}{3}
\textbf{રેસિસ્ટિવ લોડ ઇનવર્ટર દોરો. જુદા જુદા ઓપરેશન રીજન માટે ઇનપુટ વોલ્ટેજની રેન્જ લખો.}

\begin{solutionbox}
    \begin{center}
    \begin{tikzpicture}[circuit ee IEC, font=\sffamily]
        \node [contact] (vdd) at (0,4) {};
        \node [above] at (vdd) {$V_{DD}$};
        \draw (vdd) to [resistor={info={$R_L$}}] (0,2);
        \draw (0,2) -- (1,2) node[right] {$V_{out}$};
        \draw (0,2) -- (0,1.5);
        
        \node [nmos, xscale=1.5, yscale=1.5] (m1) at (0,0.75) {};
        \node [right] at (m1.east) {M1};
        
        \draw (m1.gate) -- (-1.5, 0.75) node[left] {$V_{in}$};
        \draw (m1.source) -- (0,-0.5) node[ground] {};
        \draw (m1.drain) -- (0,1.5);
    \end{tikzpicture}
    \end{center}

    \begin{center}
    \captionof{table}{Operating Regions}
    \begin{tabulary}{\linewidth}{L L L}
        \hline
        \textbf{રીજન} & \textbf{ઇનપુટ વોલ્ટેજ રેન્જ} & \textbf{આઉટપુટ સ્થિતિ} \\
        \hline
        \textbf{Cut-off} & $V_{in} < V_T$ & $V_{out} = V_{DD}$ \\
        \textbf{Triode} & $V_T < V_{in} < V_{DD}-V_T$ & \textbf{ટ્રાન્ઝિશન} \\
        \textbf{Saturation} & $V_{in} > V_{DD}-V_T$ & $V_{out} \approx 0V$ \\
        \hline
    \end{tabulary}
    \end{center}

    \begin{mnemonicbox}
    Cut-off High, Triode Transition, Saturation Low
    \end{mnemonicbox}
\end{solutionbox}

\questionmarks{2}{b}{4}
\textbf{N channel MOSFET ની VDS-ID અને VGS-ID લાક્ષણિકતાઓ દોરો અને સમજાવો.}

\begin{solutionbox}
    \begin{center}
    \begin{tikzpicture}[scale=0.8]
        \draw[->] (0,0) -- (6,0) node[right] {$V_{DS}$};
        \draw[->] (0,0) -- (0,5) node[above] {$I_D$};
        
        \draw[thick, domain=0:5, samples=100] plot (\x, {4*(1-exp(-\x))}) node[right] {$V_{GS3}$};
        \draw[thick, domain=0:5, samples=100] plot (\x, {3*(1-exp(-\x))}) node[right] {$V_{GS2}$};
        \draw[thick, domain=0:5, samples=100] plot (\x, {2*(1-exp(-\x))}) node[right] {$V_{GS1}$};
        
        \draw[dashed] (1,0) -- (1,5);
        \node[below] at (0.5,0) {Triode};
        \node[below] at (3,0) {Saturation};
    \end{tikzpicture}
    \end{center}

    \begin{center}
    \captionof{table}{NMOS Characteristics}
    \begin{tabulary}{\linewidth}{L L L}
        \hline
        \textbf{લાક્ષણિકતા} & \textbf{રીજન} & \textbf{વર્તન} \\
        \hline
        \textbf{$V_{DS}-I_D$} & \textbf{Triode} & $V_{DS}$ સાથે \textbf{Linear વૃદ્ધિ} \\
        \textbf{$V_{DS}-I_D$} & \textbf{Saturation} & \textbf{સ્થિર $I_D$} (square law) \\
        \textbf{$V_{GS}-I_D$} & \textbf{Sub-threshold} & \textbf{Exponential વૃદ્ધિ} \\
        \textbf{$V_{GS}-I_D$} & \textbf{$V_T$ ઉપર} & \textbf{Square law relationship} \\
        \hline
    \end{tabulary}
    \end{center}

    \begin{itemize}
        \item \textbf{Triode region}: $I_D$ વડે $V_{DS}$ સાથે \textbf{linearly વધે છે}
        \item \textbf{Saturation}: $I_D$ \textbf{$V_{DS}$ થી સ્વતંત્ર}, $V_{GS}$ પર આધારિત
        \item \textbf{Square law}: \textbf{Saturation} માં $I_D \propto (V_{GS}-V_T)^2$
    \end{itemize}

    \begin{mnemonicbox}
    Linear in Triode, Square in Saturation
    \end{mnemonicbox}
\end{solutionbox}

\questionmarks{2}{c}{7}
\textbf{ડિપ્લેશન લોડ NMOS ઇનવર્ટર સર્કિટ દોરો અને તેની કાર્યપદ્ધતિ સમજાવો.}

\begin{solutionbox}
    \begin{center}
    \begin{tikzpicture}[circuit ee IEC, font=\sffamily]
        \node [contact] (vdd) at (0,4) {};
        \node [above] at (vdd) {$V_{DD}$};
        
        \node [nmos, xscale=1.5, yscale=1.5] (ml) at (0,2.5) {};
        \draw (ml.drain) -- (0,4);
        \draw (ml.gate) -- (-1.2, 2.5) -- (-1.2, 1.8) -- (0, 1.8) -- (ml.source);
        \node[right] at (ml.east) {$M_L$ (Depletion)};
        
        \draw (0,1.8) -- (1.5,1.8) node[right] {$V_{out}$};
        
        \node [nmos, xscale=1.5, yscale=1.5] (m1) at (0,0.5) {};
        \draw (m1.drain) -- (ml.source);
        \draw (m1.source) -- (0,-0.5) node[ground] {};
        \draw (m1.gate) -- (-1.5, 0.5) node[left] {$V_{in}$};
        \node[right] at (m1.east) {$M_1$ (Enhancement)};
    \end{tikzpicture}
    \end{center}

    \begin{center}
    \captionof{table}{Depletion Load Inverter Operation}
    \begin{tabulary}{\linewidth}{L L L L}
        \hline
        \textbf{ઇનપુટ} & \textbf{M1 સ્થિતિ} & \textbf{ML સ્થિતિ} & \textbf{આઉટપુટ} \\
        \hline
        \textbf{Low (0V)} & \textbf{Cut-off} & \textbf{Active load} & \textbf{High ($V_{DD}$)} \\
        \textbf{High ($V_{DD}$)} & \textbf{Saturated} & \textbf{Linear} & \textbf{Low} \\
        \hline
    \end{tabulary}
    \end{center}

    \begin{itemize}
        \item \textbf{Depletion load}: \textbf{હંમેશા conducting}, \textbf{current source} તરીકે કાર્ય કરે છે
        \item \textbf{વધુ સારી performance}: \textbf{Resistive load} કરતાં \textbf{higher output voltage swing}
        \item \textbf{Gate connection}: \textbf{Depletion operation} માટે ML નું \textbf{gate source સાથે જોડાયેલું}
        \item \textbf{સુધારેલું noise margin}: \textbf{Enhancement load} કરતાં \textbf{વધુ સારું $V_{OH}$}
    \end{itemize}

    \begin{mnemonicbox}
    Depletion Always ON, Enhancement Controls Flow
    \end{mnemonicbox}
\end{solutionbox}

\questionmarks{2}{a}{3}
\textbf{CMOS ઇનવર્ટર ના ફાયદા વર્ણવો.}

\begin{solutionbox}
    \begin{center}
    \captionof{table}{Advantages of CMOS Inverter}
    \begin{tabulary}{\linewidth}{L L}
        \hline
        \textbf{ફાયદો} & \textbf{લાભ} \\
        \hline
        \textbf{શૂન્ય static power} & \textbf{Steady state} માં કોઈ current નહીં \\
        \textbf{સંપૂર્ણ voltage swing} & \textbf{આઉટપુટ 0V થી $V_{DD}$} સુધી swing કરે છે \\
        \textbf{ઉચ્ચ noise margins} & \textbf{વધુ સારી noise immunity} \\
        \hline
    \end{tabulary}
    \end{center}

    \begin{itemize}
        \item \textbf{Complementary operation}: \textbf{એક transistor હંમેશા OFF}
        \item \textbf{ઉચ્ચ input impedance}: \textbf{Gate isolation} ઉચ્ચ impedance પૂરું પાડે છે
        \item \textbf{ઝડપી switching}: \textbf{ઓછા parasitic capacitances}
    \end{itemize}

    \begin{mnemonicbox}
    Zero Power, Full Swing, High Immunity
    \end{mnemonicbox}
\end{solutionbox}

\questionmarks{2}{b}{4}
\textbf{નોઇસ માર્જિન વિગતવાર દોરો અને સમજાવો.}

\begin{solutionbox}
    \begin{center}
    \begin{tikzpicture}[scale=0.8]
        \draw[->] (0,0) -- (6,0) node[right] {$V_{in}$};
        \draw[->] (0,0) -- (0,6) node[above] {$V_{out}$};
        
        \draw[thick] (0,5) -- (2,5) -- (4,0) -- (6,0);
        
        \draw[dashed] (1.5,0) -- (1.5,5);
        \node[below] at (1.5,0) {$V_{IL}$};
        
        \draw[dashed] (3.5,0) -- (3.5,5);
        \node[below] at (3.5,0) {$V_{IH}$};
        
        \node[left] at (0,5) {$V_{OH} (V_{DD})$};
        \node[left] at (0,0) {$V_{OL} (0V)$};
        
        \draw[<->] (0,5.2) -- (1.5,5.2) node[midway, above] {$NML$};
        \draw[<->] (3.5,5.2) -- (6,5.2) node[midway, above] {$NMH$};
    \end{tikzpicture}
    \end{center}

    \begin{center}
    \captionof{table}{Noise Margin Parameters}
    \begin{tabulary}{\linewidth}{L L L}
        \hline
        \textbf{Parameter} & \textbf{Formula} & \textbf{સામાન્ય મૂલ્ય} \\
        \hline
        \textbf{$N_{MH}$} & $V_{OH} - V_{IH}$ & \textbf{$V_{DD}$ ના 40\%} \\
        \textbf{$N_{ML}$} & $V_{IL} - V_{OL}$ & \textbf{$V_{DD}$ ના 40\%} \\
        \hline
    \end{tabulary}
    \end{center}

    \begin{itemize}
        \item \textbf{High noise margin}: \textbf{Positive noise} સામે immunity
        \item \textbf{Low noise margin}: \textbf{Negative noise} સામે immunity
        \item \textbf{વધુ સારા CMOS}: \textbf{અન્ય logic families} કરતાં \textbf{higher noise margins}
    \end{itemize}

    \begin{mnemonicbox}
    High goes Higher, Low goes Lower
    \end{mnemonicbox}
\end{solutionbox}

\questionmarks{2}{c}{7}
\textbf{N MOS ઇનવર્ટર ની VTC દોરો અને સમજાવો.}

\begin{solutionbox}
    \begin{center}
    \begin{tikzpicture}[scale=0.8]
        \draw[->] (0,0) -- (6,0) node[right] {$V_{in}$};
        \draw[->] (0,0) -- (0,6) node[above] {$V_{out}$};
        
        \draw[thick] (0,5) -- (2,5); % Cutoff
        \draw[thick] (2,5) to[out=-80, in=170] (2.5,1); % Saturation
        \draw[thick] (2.5,1) -- (5.5,0.5); % Triode
        
        \draw[dashed] (2,0) -- (2,5);
        \node[below] at (2,0) {$V_T$};
        
        \node at (1,3) {Region I};
        \node at (2.8,3) {Region II};
        \node at (4,2) {Region III};
    \end{tikzpicture}
    \end{center}

    \begin{center}
    \captionof{table}{NMOS Inverter Operating Regions}
    \begin{tabulary}{\linewidth}{L L L L}
        \hline
        \textbf{રીજન} & \textbf{$V_{in}$ રેન્જ} & \textbf{M1 સ્થિતિ} & \textbf{$V_{out}$} \\
        \hline
        \textbf{I} & \textbf{0 થી $V_T$} & \textbf{Cut-off} & \textbf{$V_{DD}$} \\
        \textbf{II} & \textbf{$V_T$ થી $V_T+V_{TL}$} & \textbf{Saturation} & \textbf{ઘટતું} \\
        \textbf{III} & \textbf{$V_T+V_{TL}$ થી $V_{DD}$} & \textbf{Triode} & \textbf{નીચું} \\
        \hline
    \end{tabulary}
    \end{center}

    \begin{itemize}
        \item \textbf{Region I}: \textbf{M1 OFF}, કોઈ current flow નહીં, $V_{out} = V_{DD}$
        \item \textbf{Region II}: \textbf{M1 saturation} માં, \textbf{તીવ્ર transition}
        \item \textbf{Region III}: \textbf{M1 triode} માં, \textbf{ધીમેથી ઘટાડો}
        \item \textbf{Load line}: \textbf{Operating point intersection} નિર્ધારે છે
    \end{itemize}

    \begin{mnemonicbox}
    Cut-off High, Saturation Sharp, Triode Low
    \end{mnemonicbox}
\end{solutionbox}

\questionmarks{3}{a}{3}
\textbf{Generalized મલ્ટીપલ ઇનપુટ NOR gate નું બાંધકામ ડિપ્લેશન NMOS લોડ સાથે દોરો અને સમજાવો.}

\begin{solutionbox}
    \begin{center}
    \begin{tikzpicture}[circuit ee IEC, font=\sffamily]
        \node [contact] (vdd) at (0,4) {};
        \node [above] at (vdd) {$V_{DD}$};
        
        \node [nmos, xscale=1.5, yscale=1.5] (ml) at (0,3) {};
        \draw (ml.drain) -- (0,4);
        \draw (ml.gate) -- (-1.2, 3) -- (-1.2, 2.3) -- (0, 2.3) -- (ml.source);
        \node[right] at (ml.east) {$M_L$};
        \draw (0,2.3) -- (2,2.3) node[right] {$Y = (A+B+C)'$};
        
        \node [nmos, xscale=1.2, yscale=1.2] (m1) at (-1.5,1) {};
        \node [nmos, xscale=1.2, yscale=1.2] (m2) at (0,1) {};
        \node [nmos, xscale=1.2, yscale=1.2] (m3) at (1.5,1) {};
        
        \draw (0,2.3) -- (0, 1.8) -- (-1.5, 1.8) -- (m1.drain);
        \draw (0,2.3) -- (m2.drain);
        \draw (0,2.3) -- (0, 1.8) -- (1.5, 1.8) -- (m3.drain);
        
        \draw (m1.source) -- (-1.5, 0);
        \draw (m2.source) -- (0, 0);
        \draw (m3.source) -- (1.5, 0);
        \draw (-1.5,0) -- (1.5,0);
        \node [ground] at (0,0) {};
        
        \draw (m1.gate) -- (-2, 1) node[left] {A};
        \draw (m2.gate) -- (-0.5, 1) node[left] {B};
        \draw (m3.gate) -- (1, 1) node[left] {C};
    \end{tikzpicture}
    \end{center}

    \begin{center}
    \captionof{table}{Truth Table}
    \begin{tabulary}{\linewidth}{L L L}
        \hline
        \textbf{ઇનપુટ્સ} & \textbf{કોઈ ઇનપુટ High?} & \textbf{આઉટપુટ Y} \\
        \hline
        \textbf{બધા Low} & \textbf{ના} & \textbf{High (1)} \\
        \textbf{કોઈ High} & \textbf{હા} & \textbf{Low (0)} \\
        \hline
    \end{tabulary}
    \end{center}

    \begin{itemize}
        \item \textbf{Parallel NMOS}: \textbf{કોઈપણ input HIGH} હોય તો \textbf{output LOW} થાય છે
        \item \textbf{NOR operation}: $Y = (A+B+C)'$
        \item \textbf{Depletion load}: \textbf{Pull-up current} પૂરું પાડે છે
    \end{itemize}

    \begin{mnemonicbox}
    Parallel Pulls Down, Depletion Pulls Up
    \end{mnemonicbox}
\end{solutionbox}

\questionmarks{3}{b}{4}
\textbf{AOI અને OAI ના તફાવત લખો.}

\begin{solutionbox}
    \begin{center}
    \captionof{table}{AOI vs OAI Logic}
    \begin{tabulary}{\linewidth}{L L L}
        \hline
        \textbf{Parameter} & \textbf{AOI (AND-OR-Invert)} & \textbf{OAI (OR-AND-Invert)} \\
        \hline
        \textbf{Logic function} & $Y = (AB + CD)'$ & $Y = ((A+B)(C+D))'$ \\
        \textbf{NMOS structure} & \textbf{Series-parallel} & \textbf{Parallel-series} \\
        \textbf{PMOS structure} & \textbf{Parallel-series} & \textbf{Series-parallel} \\
        \textbf{જટિલતા} & \textbf{મધ્યમ} & \textbf{મધ્યમ} \\
        \hline
    \end{tabulary}
    \end{center}

    \begin{itemize}
        \item \textbf{AOI}: \textbf{AND terms ORed} પછી \textbf{inverted}
        \item \textbf{OAI}: \textbf{OR terms ANDed} પછી \textbf{inverted}
        \item \textbf{CMOS implementation}: \textbf{Dual network structure}
        \item \textbf{Applications}: \textbf{Single stage} માં \textbf{complex logic functions}
    \end{itemize}

    \begin{mnemonicbox}
    AOI: AND-OR-Invert, OAI: OR-AND-Invert
    \end{mnemonicbox}
\end{solutionbox}

\questionmarks{3}{c}{7}
\textbf{EX-OR gate CMOS ની મદદથી અને લોજીક ફંક્શન Z = (AB +CD)' NMOS લોડની મદદથી અમલમાં મૂકો.}

\begin{solutionbox}
    \textbf{EX-OR CMOS Implementation:}
    \begin{center}
    \begin{tikzpicture}[circuit ee IEC, font=\sffamily]
        \draw (0,0) rectangle (4,2);
        \node[align=center] at (2,1) {PMOS Network\\$(A'B + AB')$};
        
        \draw (0,-3) rectangle (4,-1);
        \node[align=center] at (2,-2) {NMOS Network\\$(AB + A'B')$};
        
        \draw (2,2) -- (2,2.5) node[above] {$V_{DD}$};
        \draw (2,-3) -- (2,-3.5) node[ground] {};
        \draw (2,0) -- (2,-1);
        \draw (2,-0.5) -- (4.5,-0.5) node[right] {$Y = A \oplus B$};
    \end{tikzpicture}
    \end{center}

    \textbf{Z = (AB + CD)' NMOS Implementation:}
    \begin{center}
    \begin{tikzpicture}[circuit ee IEC, font=\sffamily]
        \node [contact] (vdd) at (2,4) {};
        \node [above] at (vdd) {$V_{DD}$};
        \draw (vdd) to [resistor={info={$R_L$}}] (2,3);
        \draw (2,3) -- (3.5,3) node[right] {$Z = (AB + CD)'$};
        
        \node [nmos, rotate=90] (m1) at (1,2) {}; % A
        \node [nmos, rotate=90] (m2) at (1,1) {}; % B
        \draw (2,3) -- (1,3) -- (m1.drain);
        \draw (m1.source) -- (m2.drain);
        \draw (m2.source) -- (1,0); 
        
        \node [nmos, rotate=90] (m3) at (3,2) {}; % C
        \node [nmos, rotate=90] (m4) at (3,1) {}; % D
        \draw (2,3) -- (3,3) -- (m3.drain);
        \draw (m3.source) -- (m4.drain);
        \draw (m4.source) -- (3,0);
        
        \draw (1,0) -- (3,0);
        \node [ground] at (2,0) {};
        
        \node [left] at (0.8, 2) {A};
        \node [left] at (0.8, 1) {B};
        \node [right] at (3.2, 2) {C};
        \node [right] at (3.2, 1) {D};
    \end{tikzpicture}
    \end{center}

    \begin{center}
    \captionof{table}{Logic Implementation Comparison}
    \begin{tabulary}{\linewidth}{L L L}
        \hline
        \textbf{સર્કિટ} & \textbf{ફંક્શન} & \textbf{Implementation} \\
        \hline
        \textbf{EX-OR} & $A \oplus B$ & \textbf{Complementary CMOS} \\
        \textbf{AOI} & $(AB+CD)'$ & \textbf{Series-parallel NMOS} \\
        \hline
    \end{tabulary}
    \end{center}

    \begin{itemize}
        \item \textbf{EX-OR}: \textbf{Efficient implementation} માટે \textbf{transmission gates} જરૂરી
        \item \textbf{AOI function}: \textbf{Natural NMOS implementation}
        \item \textbf{Power consideration}: \textbf{CMOS માં zero static power}
    \end{itemize}

    \begin{mnemonicbox}
    EX-OR needs Transmission, AOI uses Series-Parallel
    \end{mnemonicbox}
\end{solutionbox}

\questionmarks{3}{a}{3}
\textbf{Generalized મલ્ટીપલ ઇનપુટ NAND gate નું બાંધકામ ડિપ્લેશન NMOS લોડ સાથે દોરો અને સમજાવો.}

\begin{solutionbox}
    \begin{center}
    \begin{tikzpicture}[circuit ee IEC, font=\sffamily]
        \node [contact] (vdd) at (0,4.5) {};
        \node [above] at (vdd) {$V_{DD}$};
        
        \node [nmos, xscale=1.5, yscale=1.5] (ml) at (0,3.5) {};
        \draw (ml.drain) -- (0,4.5);
        \draw (ml.gate) -- (-1.2, 3.5) -- (-1.2, 2.8) -- (0, 2.8) -- (ml.source);
        \node[right] at (ml.east) {$M_L$};
        \draw (0,2.8) -- (1.5,2.8) node[right] {$Y = (ABC)'$};
        
        \node [nmos, xscale=1.2, yscale=1.2] (m1) at (0,2) {};
        \node [nmos, xscale=1.2, yscale=1.2] (m2) at (0,0.8) {};
        \node [nmos, xscale=1.2, yscale=1.2] (m3) at (0,-0.4) {};
        
        \draw (0,2.8) -- (m1.drain);
        \draw (m1.source) -- (m2.drain);
        \draw (m2.source) -- (m3.drain);
        \draw (m3.source) -- (0,-1) node[ground] {};
        
        \draw (m1.gate) -- (-1, 2) node[left] {A};
        \draw (m2.gate) -- (-1, 0.8) node[left] {B};
        \draw (m3.gate) -- (-1, -0.4) node[left] {C};
    \end{tikzpicture}
    \end{center}

    \begin{center}
    \captionof{table}{NAND Gate Operation}
    \begin{tabulary}{\linewidth}{L L L}
        \hline
        \textbf{સ્થિતિ} & \textbf{Ground તરફ પાથ} & \textbf{આઉટપુટ Y} \\
        \hline
        \textbf{બધા inputs HIGH} & \textbf{સંપૂર્ણ પાથ} & \textbf{Low (0)} \\
        \textbf{કોઈ input LOW} & \textbf{તૂટેલો પાથ} & \textbf{High (1)} \\
        \hline
    \end{tabulary}
    \end{center}

    \begin{itemize}
        \item \textbf{Series NMOS}: \textbf{બધા inputs HIGH} હોવા જરૂરી \textbf{output LOW} કરવા માટે
        \item \textbf{NAND operation}: $Y = (ABC)'$
        \item \textbf{Depletion load}: \textbf{હંમેશા pull-up current} પૂરું પાડે છે
    \end{itemize}

    \begin{mnemonicbox}
    Series Needs All, NAND Says Not-AND
    \end{mnemonicbox}
\end{solutionbox}

\questionmarks{3}{b}{4}
\textbf{((P+R)(S+T))' લોજીક ફંક્શન CMOS લોજીકની મદદથી અમલીકરણ કરો.}

\begin{solutionbox}
    \begin{center}
    \begin{tikzpicture}[circuit ee IEC, font=\sffamily, scale=0.8]
        \node (vdd) at (2,6) {$V_{DD}$};
        \draw (0,4) rectangle (4,5.5);
        \node[align=center] at (2,4.75) {PMOS Network\\Parallel P/R Series S/T};
        
        \draw (2,5.5) -- (vdd);
        \draw (2,4) -- (2,3) node[right] {$Y$};
        
        \draw (0,0.5) rectangle (4,2);
        \node[align=center] at (2,1.25) {NMOS Network\\Series P/R Parallel S/T};
        \draw (2,3) -- (2,2);
        \draw (2,0.5) -- (2,0) node[ground] {};
    \end{tikzpicture}
    \end{center}

    \begin{center}
    \captionof{table}{Truth Table Implementation}
    \begin{tabulary}{\linewidth}{L L L}
        \hline
        \textbf{PMOS Network} & \textbf{NMOS Network} & \textbf{ઓપરેશન} \\
        \hline
        \textbf{$(P+R)'+(S+T)'$} & \textbf{$(P+R)(S+T)$} & \textbf{Complementary} \\
        \textbf{$P'R' + S'T'$} & \textbf{$PS + PT + RS + RT$} & \textbf{De Morgan's law} \\
        \hline
    \end{tabulary}
    \end{center}

    \begin{itemize}
        \item \textbf{PMOS}: \textbf{Groups વિથિન parallel}, \textbf{groups વચ્ચે series}
        \item \textbf{NMOS}: \textbf{Groups વિથિન series}, \textbf{groups વચ્ચે parallel}
        \item \textbf{Dual network}: \textbf{Complementary operation} સુનિશ્ચિત કરે છે
    \end{itemize}

    \begin{mnemonicbox}
    PMOS does Opposite of NMOS
    \end{mnemonicbox}
\end{solutionbox}

\questionmarks{3}{c}{7}
\textbf{SR latch circuit ની કાર્યપદ્ધતિ વર્ણવો.}

\begin{solutionbox}
    \begin{center}
    \begin{tikzpicture}
        \node[gtu block] (nor1) at (0,1.5) {NOR 1};
        \node[gtu block] (nor2) at (0,-1.5) {NOR 2};
        
        \draw[->] (-2,1.7) node[left]{S} -- (nor1.160);
        \draw[->] (-2,-1.7) node[left]{R} -- (nor2.200);
        
        \draw[->] (nor1.0) -- (2,1.5) node[right]{Q};
        \draw[->] (nor2.0) -- (2,-1.5) node[right]{Q'};
        
        \draw (nor1.0) -- (0.5,1.5) -- (0.5,0.5) -- (-1,-0.5) -- (-1,-1.3) -- (nor2.160);
        \draw (nor2.0) -- (0.5,-1.5) -- (0.5,-0.5) -- (-1,0.5) -- (-1,1.3) -- (nor1.200);
    \end{tikzpicture}
    \end{center}

    \begin{center}
    \captionof{table}{SR Latch Truth Table}
    \begin{tabulary}{\linewidth}{L L L L L}
        \hline
        \textbf{S} & \textbf{R} & \textbf{Q(n+1)} & \textbf{Q'(n+1)} & \textbf{સ્થિતિ} \\
        \hline
        \textbf{0} & \textbf{0} & Q(n) & Q'(n) & \textbf{Hold} \\
        \textbf{0} & \textbf{1} & 0 & 1 & \textbf{Reset} \\
        \textbf{1} & \textbf{0} & 1 & 0 & \textbf{Set} \\
        \textbf{1} & \textbf{1} & 0 & 0 & \textbf{અમાન્ય} \\
        \hline
    \end{tabulary}
    \end{center}

    \begin{itemize}
        \item \textbf{Set operation}: S=1, R=0 થી Q=1 થાય છે
        \item \textbf{Reset operation}: S=0, R=1 થી Q=0 થાય છે
        \item \textbf{Hold state}: S=0, R=0 \textbf{પહેલાની state} જાળવે છે
        \item \textbf{અમાન્ય state}: S=1, R=1 \textbf{ટાળવી જોઈએ}
        \item \textbf{Cross-coupled}: \textbf{એક gate નું output બીજાના input} માં જાય છે
    \end{itemize}

    \begin{mnemonicbox}
    Set Sets, Reset Resets, Both Bad
    \end{mnemonicbox}
\end{solutionbox}

\questionmarks{4}{a}{3}
\textbf{ચિપ ફેબ્રિકેશન માં Etching methods ની સરખામણી કરો.}

\begin{solutionbox}
    \begin{center}
    \captionof{table}{Etching Methods Comparison}
    \begin{tabulary}{\linewidth}{L L L L}
        \hline
        \textbf{પદ્ધતિ} & \textbf{પ્રકાર} & \textbf{ફાયદા} & \textbf{નુકસાન} \\
        \hline
        \textbf{Wet Etching} & \textbf{રાસાયણિક} & \textbf{ઉચ્ચ selectivity}, સરળ & \textbf{Isotropic}, undercut \\
        \textbf{Dry Etching} & \textbf{ભૌતિક/રાસાયણિક} & \textbf{Anisotropic}, ચોક્કસ & \textbf{જટિલ સાધનો} \\
        \textbf{Plasma Etching} & \textbf{Ion bombardment} & \textbf{Directional control} & \textbf{સપાટીને નુકસાન} \\
        \hline
    \end{tabulary}
    \end{center}

    \begin{itemize}
        \item \textbf{Wet etching}: \textbf{પ્રવાહી રસાયણો} વાપરે છે, \textbf{બધી દિશાઓમાં હુમલો}
        \item \textbf{Dry etching}: \textbf{ગેસ/plasma} વાપરે છે, \textbf{વધુ સારું directional control}
        \item \textbf{Selectivity}: \textbf{એક સામગ્રીને બીજા કરતાં etch} કરવાની ક્ષમતા
    \end{itemize}

    \begin{mnemonicbox}
    Wet is Wide, Dry is Directional
    \end{mnemonicbox}
\end{solutionbox}

\questionmarks{4}{b}{4}
\textbf{ટૂંક નોંધ લખો : Lithography}

\begin{solutionbox}
    \begin{center}
    \captionof{table}{Lithography Process Steps}
    \begin{tabulary}{\linewidth}{L L L}
        \hline
        \textbf{સ્ટેપ} & \textbf{પ્રક્રિયા} & \textbf{હેતુ} \\
        \hline
        \textbf{Resist coating} & \textbf{Photoresist નું spin-on} & \textbf{પ્રકાશ-સંવેદનશીલ layer} \\
        \textbf{Exposure} & \textbf{Mask દ્વારા UV light} & \textbf{Pattern transfer} \\
        \textbf{Development} & \textbf{Exposed resist દૂર કરવું} & \textbf{Pattern પ્રગટ કરવું} \\
        \textbf{Etching} & \textbf{અસુરક્ષિત material દૂર કરવું} & \textbf{Features બનાવવા} \\
        \hline
    \end{tabulary}
    \end{center}

    \begin{itemize}
        \item \textbf{Pattern transfer}: \textbf{Mask થી silicon wafer} પર
        \item \textbf{Resolution}: \textbf{Minimum feature size} નિર્ધારે છે
        \item \textbf{Alignment}: \textbf{Multiple layer processing} માટે મહત્વપૂર્ણ
        \item \textbf{UV wavelength}: \textbf{ટૂંકી wavelength} વધુ સારું resolution આપે છે
    \end{itemize}

    \begin{mnemonicbox}
    Coat, Expose, Develop, Etch
    \end{mnemonicbox}
\end{solutionbox}

\questionmarks{4}{c}{7}
\textbf{Regularity, Modularity and Locality સમજાવો.}

\begin{solutionbox}
    \begin{center}
    \captionof{table}{Design Principles}
    \begin{tabulary}{\linewidth}{L L L L}
        \hline
        \textbf{સિદ્ધાંત} & \textbf{વ્યાખ્યા} & \textbf{ફાયદા} & \textbf{ઉદાહરણ} \\
        \hline
        \textbf{Regularity} & \textbf{સમાન structures નું પુનરાવર્તન} & \textbf{સરળ design, testing} & \textbf{Memory arrays} \\
        \textbf{Modularity} & \textbf{Hierarchical design blocks} & \textbf{Reusability, maintainability} & \textbf{Standard cells} \\
        \textbf{Locality} & \textbf{સંબંધિત functions નું જૂથ} & \textbf{ઓછું interconnect} & \textbf{Functional blocks} \\
        \hline
    \end{tabulary}
    \end{center}

    \textbf{Implementation વિગતો:}
    \begin{itemize}
        \item \textbf{Regularity}: \textbf{સમાન cell બારંબાર} વાપરવાથી \textbf{design complexity} ઘટે છે
        \item \textbf{Modularity}: \textbf{Well-defined interfaces} સાથે \textbf{top-down design}
        \item \textbf{Locality}: \textbf{Wire delays અને routing congestion} ઘટાડે છે
        \item \textbf{Design benefits}: \textbf{ઝડપી design cycle}, \textbf{વધુ સારી testability}
        \item \textbf{Manufacturing}: \textbf{Regular patterns} દ્વારા \textbf{સુધારેલી yield}
    \end{itemize}

    \begin{center}
    \begin{tikzpicture}
        \node[gtu block] (sys) at (0,0) {System Level};
        \node[gtu block, right=of sys] (mod) {Module Level};
        \node[gtu block, right=of mod] (cell) {Cell Level};
        \node[gtu block, right=of cell] (dev) {Device Level};
        
        \path[gtu arrow] (sys) edge (mod);
        \path[gtu arrow] (mod) edge (cell);
        \path[gtu arrow] (cell) edge (dev);
    \end{tikzpicture}
    \end{center}

    \begin{mnemonicbox}
    Regular Modules with Local Connections
    \end{mnemonicbox}
\end{solutionbox}

\questionmarks{4}{a}{3}
\textbf{Design Hierarchy વ્યાખ્યાયિત કરો.}

\begin{solutionbox}
    \begin{center}
    \captionof{table}{Design Hierarchy Levels}
    \begin{tabulary}{\linewidth}{L L L}
        \hline
        \textbf{સ્તર} & \textbf{વિવરણ} & \textbf{ઘટકો} \\
        \hline
        \textbf{System} & \textbf{સંપૂર્ણ chip functionality} & \textbf{Processors, memories} \\
        \textbf{Module} & \textbf{મુખ્ય functional blocks} & \textbf{ALU, cache, I/O} \\
        \textbf{Cell} & \textbf{મૂળભૂત logic elements} & \textbf{Gates, flip-flops} \\
        \hline
    \end{tabulary}
    \end{center}

    \begin{itemize}
        \item \textbf{Top-down approach}: \textbf{System નાના modules} માં વિભાજિત
        \item \textbf{Abstraction levels}: \textbf{દરેક level નીચેની details છુપાવે} છે
        \item \textbf{Interface definition}: \textbf{Levels વચ્ચે સ્પષ્ટ boundaries}
    \end{itemize}

    \begin{mnemonicbox}
    System to Module to Cell
    \end{mnemonicbox}
\end{solutionbox}

\questionmarks{4}{b}{4}
\textbf{VLSI design flow chart દોરો અને સમજાવો.}

\begin{solutionbox}
    \begin{center}
    \begin{tikzpicture}[node distance=1.5cm, auto]
        \node[gtu block] (spec) {System Specification};
        \node[gtu block, below=of spec] (arch) {Architectural Design};
        \node[gtu block, below=of arch] (logic) {Logic Design};
        \node[gtu block, below=of logic] (circ) {Circuit Design};
        \node[gtu block, below=of circ] (lay) {Layout Design};
        \node[gtu block, below=of lay] (fab) {Fabrication};
        \node[gtu block, below=of fab] (test) {Testing};
        
        \path[gtu arrow] (spec) edge (arch);
        \path[gtu arrow] (arch) edge (logic);
        \path[gtu arrow] (logic) edge (circ);
        \path[gtu arrow] (circ) edge (lay);
        \path[gtu arrow] (lay) edge (fab);
        \path[gtu arrow] (fab) edge (test);
    \end{tikzpicture}
    \end{center}

    \begin{center}
    \captionof{table}{Design Flow}
    \begin{tabulary}{\linewidth}{L L L L}
        \hline
        \textbf{તબક્કો} & \textbf{ઇનપુટ} & \textbf{આઉટપુટ} & \textbf{સાધનો} \\
        \hline
        \textbf{Architecture} & \textbf{Specifications} & \textbf{Block diagram} & \textbf{High-level modeling} \\
        \textbf{Logic} & \textbf{Architecture} & \textbf{Gate netlist} & \textbf{HDL synthesis} \\
        \textbf{Circuit} & \textbf{Netlist} & \textbf{Transistor sizing} & \textbf{SPICE simulation} \\
        \textbf{Layout} & \textbf{Circuit} & \textbf{Mask data} & \textbf{Place \& route} \\
        \hline
    \end{tabulary}
    \end{center}

    \begin{mnemonicbox}
    Specify, Architect, Logic, Circuit, Layout, Fabricate, Test
    \end{mnemonicbox}
\end{solutionbox}

\questionmarks{4}{c}{7}
\textbf{ટૂંક નોંધ લખો : 'VLSI Fabrication Process'}

\begin{solutionbox}
    \begin{center}
    \captionof{table}{Major Fabrication Steps}
    \begin{tabulary}{\linewidth}{L L L}
        \hline
        \textbf{પ્રક્રિયા} & \textbf{હેતુ} & \textbf{પરિણામ} \\
        \hline
        \textbf{Oxidation} & \textbf{SiO2 layer વૃદ્ધિ} & \textbf{Gate oxide formation} \\
        \textbf{Lithography} & \textbf{Pattern transfer} & \textbf{Device areas વ્યાખ્યા} \\
        \textbf{Etching} & \textbf{અનાવશ્યક material દૂર કરવું} & \textbf{Device structures બનાવવા} \\
        \textbf{Ion Implantation} & \textbf{Dopants ઉમેરવા} & \textbf{P/N regions બનાવવા} \\
        \textbf{Deposition} & \textbf{Material layers ઉમેરવા} & \textbf{Metal interconnects} \\
        \textbf{Diffusion} & \textbf{Dopants ફેલાવવા} & \textbf{Junction formation} \\
        \hline
    \end{tabulary}
    \end{center}

    \textbf{Process Flow:}
    \begin{itemize}
        \item \textbf{Wafer preparation}: \textbf{સ્વચ્છ silicon substrate}
        \item \textbf{Device formation}: \textbf{બિનેક steps દ્વારા transistors} બનાવવા
        \item \textbf{Interconnect}: \textbf{Connections માટે metal layers} ઉમેરવા
        \item \textbf{Passivation}: \textbf{પૂર્ણ થયેલા circuit ની સુરક્ષા}
        \item \textbf{Testing}: \textbf{Packaging પહેલાં functionality verify} કરવી
    \end{itemize}

    \textbf{Clean Room જરૂરિયાતો:}
    \begin{itemize}
        \item \textbf{Class 1-10}: \textbf{અત્યંત સ્વચ્છ વાતાવરણ} જરૂરી
        \item \textbf{Temperature control}: \textbf{ચોક્કસ process control}
        \item \textbf{Chemical purity}: \textbf{ઉચ્ચ-ગ્રેડ materials} જરૂરી
    \end{itemize}

    \begin{mnemonicbox}
    Oxidize, Pattern, Etch, Implant, Deposit, Diffuse
    \end{mnemonicbox}
\end{solutionbox}

\questionmarks{5}{a}{3}
\textbf{વેરીલોગ પ્રોગ્રામિંગની જુદી જુદી પદ્ધતિ સરખાવો.}

\begin{solutionbox}
    \begin{center}
    \captionof{table}{Verilog Modeling Styles}
    \begin{tabulary}{\linewidth}{L L L}
        \hline
        \textbf{શૈલી} & \textbf{વિવરણ} & \textbf{ઉપયોગ} \\
        \hline
        \textbf{Behavioral} & \textbf{Algorithm description} & \textbf{High-level modeling} \\
        \textbf{Dataflow} & \textbf{Boolean expressions} & \textbf{Combinational logic} \\
        \textbf{Structural} & \textbf{Gate-level description} & \textbf{Hardware representation} \\
        \hline
    \end{tabulary}
    \end{center}

    \begin{itemize}
        \item \textbf{Behavioral}: \textbf{Always blocks, if-else, case statements} વાપરે છે
        \item \textbf{Dataflow}: \textbf{Boolean operators સાથે assign statements} વાપરે છે
        \item \textbf{Structural}: \textbf{Gates અને modules explicitly instantiate} કરે છે
    \end{itemize}

    \begin{mnemonicbox}
    Behavior Describes, Dataflow Assigns, Structure Connects
    \end{mnemonicbox}
\end{solutionbox}

\questionmarks{5}{b}{4}
\textbf{બિહેવિયરલ પદ્ધતિ થી NAND gate નો વેરીલોગ પ્રોગ્રામ લખો.}

\begin{solutionbox}
\begin{lstlisting}[language=Verilog]
module nand_gate_behavioral(
    input wire a, b,
    output reg y
);

always @(a or b) begin
    if (a == 1'b1 && b == 1'b1)
        y = 1'b0;
    else
        y = 1'b1;
end

endmodule
\end{lstlisting}

    \textbf{કોડ સમજૂતી:}
    \begin{itemize}
        \item \textbf{Always block}: \textbf{Inputs બદલાય} ત્યારે execute થાય છે
        \item \textbf{Sensitivity list}: \textbf{બધા input signals} સમાવે છે
        \item \textbf{Conditional statement}: \textbf{NAND logic implement} કરે છે
        \item \textbf{Reg declaration}: \textbf{Procedural assignment} માટે જરૂરી
    \end{itemize}

    \begin{mnemonicbox}
    Always watch, IF both high THEN low ELSE high
    \end{mnemonicbox}
\end{solutionbox}

\questionmarks{5}{c}{7}
\textbf{4X1 multiplexer ની સર્કિટ દોરો. Case સ્ટેટમેંટ થી આ સર્કિટ નો વેરીલોગ પ્રોગ્રામ બનાવો.}

\begin{solutionbox}
    \textbf{4X1 Multiplexer સર્કિટ:}
    \begin{center}
    \begin{tikzpicture}
        \node[draw, minimum width=2cm, minimum height=3cm] (mux) {MUX 4X1};
        
        \draw[<-] (mux.160) -- +(-1,0) node[left] {$I_0$};
        \draw[<-] (mux.170) -- +(-1,0) node[left] {$I_1$};
        \draw[<-] (mux.190) -- +(-1,0) node[left] {$I_2$};
        \draw[<-] (mux.200) -- +(-1,0) node[left] {$I_3$};
        
        \draw[->] (mux.0) -- +(1,0) node[right] {$Y$};
        
        \draw[<-] (mux.270) -- +(0,-1) node[below] {$S_1, S_0$};
    \end{tikzpicture}
    \end{center}

    \textbf{Verilog કોડ:}
\begin{lstlisting}[language=Verilog]
module mux_4x1_case(
    input wire [1:0] sel,
    input wire i0, i1, i2, i3,
    output reg y
);

always @(*) begin
    case (sel)
        2'b00: y = i0;
        2'b01: y = i1;
        2'b10: y = i2;
        2'b11: y = i3;
        default: y = 1'bx;
    endcase
end

endmodule
\end{lstlisting}

    \begin{center}
    \captionof{table}{Truth Table}
    \begin{tabulary}{\linewidth}{L L L}
        \hline
        \textbf{S1} & \textbf{S0} & \textbf{આઉટપુટ Y} \\
        \hline
        \textbf{0} & \textbf{0} & \textbf{I0} \\
        \textbf{0} & \textbf{1} & \textbf{I1} \\
        \textbf{1} & \textbf{0} & \textbf{I2} \\
        \textbf{1} & \textbf{1} & \textbf{I3} \\
        \hline
    \end{tabulary}
    \end{center}

    \begin{mnemonicbox}
    Case Selects, Default Protects
    \end{mnemonicbox}
\end{solutionbox}

\questionmarks{5}{a}{3}
\textbf{ઉદાહરણ સાથે Testbench વ્યાખ્યાયિત કરો.}

\begin{solutionbox}
    \textbf{Testbench વ્યાખ્યા:}
    \textbf{Testbench એ Verilog module} છે જે \textbf{design under test (DUT) ને stimulus} પૂરું પાડે છે અને \textbf{તેના response ને monitor} કરે છે.

    \textbf{ઉદાહરણ Testbench:}
\begin{lstlisting}[language=Verilog]
module test_and_gate;
    reg a, b;
    wire y;
    
    and_gate dut(.a(a), .b(b), .y(y));
    
    initial begin
        a = 0; b = 0; #10;
        a = 0; b = 1; #10;
        a = 1; b = 0; #10;
        a = 1; b = 1; #10;
    end
endmodule
\end{lstlisting}

    \begin{itemize}
        \item \textbf{DUT instantiation}: \textbf{Design under test નું instance} બનાવે છે
        \item \textbf{Stimulus generation}: \textbf{Input test vectors} પૂરા પાડે છે
        \item \textbf{કોઈ ports નહીં}: \textbf{Testbench top-level module} છે
    \end{itemize}

    \begin{mnemonicbox}
    Test Provides Stimulus, Monitors Response
    \end{mnemonicbox}
\end{solutionbox}

\questionmarks{5}{b}{4}
\textbf{ડેટા ફ્લો પદ્ધતિ થી Half Adder નો વેરીલોગ પ્રોગ્રામ લખો.}

\begin{solutionbox}
\begin{lstlisting}[language=Verilog]
module half_adder_dataflow(
    input wire a, b,
    output wire sum, carry
);

assign sum = a ^ b;    // XOR for sum
assign carry = a & b;  // AND for carry

endmodule
\end{lstlisting}

    \textbf{લોજીક Implementation:}
    \begin{itemize}
        \item \textbf{Sum}: \textbf{Inputs વચ્ચે XOR operation}
        \item \textbf{Carry}: \textbf{Inputs વચ્ચે AND operation}
        \item \textbf{Assign statement}: \textbf{Dataflow માટે continuous assignment}
        \item \textbf{Boolean operators}: \textbf{\^{} (XOR), \& (AND)}
    \end{itemize}

    \begin{center}
    \captionof{table}{Truth Table}
    \begin{tabulary}{\linewidth}{L L L L}
        \hline
        \textbf{A} & \textbf{B} & \textbf{Sum} & \textbf{Carry} \\
        \hline
        \textbf{0} & \textbf{0} & \textbf{0} & \textbf{0} \\
        \textbf{0} & \textbf{1} & \textbf{1} & \textbf{0} \\
        \textbf{1} & \textbf{0} & \textbf{1} & \textbf{0} \\
        \textbf{1} & \textbf{1} & \textbf{0} & \textbf{1} \\
        \hline
    \end{tabulary}
    \end{center}

    \begin{mnemonicbox}
    XOR Sums, AND Carries
    \end{mnemonicbox}
\end{solutionbox}

\questionmarks{5}{c}{7}
\textbf{Encoder નું કાર્ય લખો. if..else વડે 8X3 Encoder નો વેરીલોગ કોડ બનાવો.}

\begin{solutionbox}
    \textbf{Encoder કાર્ય:}
    \textbf{Encoder $2^n$ input lines ને $n$ output lines} માં convert કરે છે. \textbf{8X3 encoder 8 inputs ને 3-bit binary output} માં convert કરે છે.

    \begin{center}
    \captionof{table}{Priority Table}
    \begin{tabulary}{\linewidth}{L L}
        \hline
        \textbf{ઇનપુટ} & \textbf{Binary આઉટપુટ} \\
        \hline
        \textbf{I7} & \textbf{111} \\
        \textbf{I6} & \textbf{110} \\
        \textbf{I5} & \textbf{101} \\
        \textbf{I4} & \textbf{100} \\
        \textbf{I3} & \textbf{011} \\
        \textbf{I2} & \textbf{010} \\
        \textbf{I1} & \textbf{001} \\
        \textbf{I0} & \textbf{000} \\
        \hline
    \end{tabulary}
    \end{center}

    \textbf{Verilog કોડ:}
\begin{lstlisting}[language=Verilog]
module encoder_8x3(
    input wire [7:0] i,
    output reg [2:0] y
);

always @(*) begin
    if (i[7])
        y = 3'b111;
    else if (i[6])
        y = 3'b110;
    else if (i[5])
        y = 3'b101;
    else if (i[4])
        y = 3'b100;
    else if (i[3])
        y = 3'b011;
    else if (i[2])
        y = 3'b010;
    else if (i[1])
        y = 3'b001;
    else if (i[0])
        y = 3'b000;
    else
        y = 3'bxxx;
end

endmodule
\end{lstlisting}

    \begin{itemize}
        \item \textbf{Priority encoding}: \textbf{ઉચ્ચ index inputs ને priority}
        \item \textbf{If-else chain}: \textbf{Priority logic implement} કરે છે
        \item \textbf{Binary encoding}: \textbf{Active input ને binary representation} માં convert કરે છે
    \end{itemize}

    \begin{mnemonicbox}
    Priority from High to Low, Binary Output Flows
    \end{mnemonicbox}
\end{solutionbox}

\end{document}
