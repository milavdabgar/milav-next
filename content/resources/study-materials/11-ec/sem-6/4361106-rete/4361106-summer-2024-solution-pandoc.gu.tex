\documentclass[10pt,a4paper]{article}

% content/resources/templates/preamble.tex
\usepackage[margin=0.6in]{geometry}
\author{Milav Dabgar}
\usepackage{amsmath,amssymb,amsthm}
\usepackage{booktabs}
\usepackage{multirow}
\usepackage{xcolor}
\usepackage{tcolorbox}
\tcbuselibrary{breakable,skins}
\usepackage[colorlinks=true,linkcolor=blue]{hyperref}
\usepackage{titlesec}
\usepackage{enumitem}
\usepackage{tikz}
\usepackage{pgfplots}
\usepackage{circuitikz}
\usepackage[version=4]{mhchem}
\usepackage{longtable}
\usepackage{array}
\usepackage{float}
\usepackage{caption}
\usepackage{listings}

\lstset{
  basicstyle=\small\ttfamily,
  breaklines=true,
  breakatwhitespace=false,
  postbreak=\mbox{\textcolor{red}{$\hookrightarrow$}\space},
  float=false,
  numbers=left,
  numberstyle=\tiny\color{gray},
  numbersep=10pt,
  xleftmargin=2em,
  keywordstyle=\color{blue},
  commentstyle=\color{green!60!black},
  stringstyle=\color{purple},
  backgroundcolor=\color{gray!5},
  showstringspaces=false,
  tabsize=2,
  captionpos=b,
  keepspaces=true,
  columns=flexible
}

\pgfplotsset{compat=1.18}
\usetikzlibrary{shapes,arrows,positioning,calc,patterns,decorations.pathmorphing,decorations.markings,arrows.meta}

% Color scheme
\definecolor{headcolor}{RGB}{0,102,204}
\definecolor{keycolor}{RGB}{220,20,60}
\definecolor{solutioncolor}{RGB}{34,139,34}
\definecolor{mnemoniccolor}{RGB}{148,0,211}
\definecolor{codecolor}{RGB}{0,0,100}

% Spacing
\setlength{\parskip}{3pt}
\setlist[itemize]{nosep}
\setlist[enumerate]{nosep}

% Title formatting
\titleformat{\section}{\Large\bfseries\color{headcolor}}{\thesection}{1em}{}
\titleformat{\subsection}{\large\bfseries\color{headcolor}}{\thesubsection}{1em}{}

% Pandoc tightlist compatibility
\providecommand{\tightlist}{%
  \setlength{\itemsep}{0pt}\setlength{\parskip}{0pt}}

% Pandoc longtable compatibility
\newcounter{none}
\def\thenone{}


% content/resources/templates/gujarati-boxes.tex
\usepackage{fontspec}
\usepackage{polyglossia}

% Set Gujarati as main language (document is primarily in Gujarati)
% Note: gloss-gujarati.ldf doesn't exist in polyglossia, but it will use hyphenation patterns
\setdefaultlanguage{gujarati}
\setotherlanguage{english}

% Configure Gujarati font properly
% Use Language=Default to prevent polyglossia from trying to add language-specific features
% that don't exist for Gujarati, which causes "empty feature" warnings
\newfontfamily\gujaratifont[Script=Gujarati,AutoFakeBold=2.5,AutoFakeSlant=0.3]{Noto Sans Gujarati}
\setmainfont[Script=Gujarati,AutoFakeBold=2.5,AutoFakeSlant=0.3]{Noto Sans Gujarati}
% Use Noto Sans Gujarati for monospace to support Gujarati in text
\setmonofont[Scale=0.9]{Noto Sans Gujarati}

% Configure English to use the same font
\newfontfamily\englishfont[Script=Gujarati,AutoFakeBold=2.5,AutoFakeSlant=0.3]{Noto Sans Gujarati}

% Translations for polyglossia
\gappto\captionsgujarati{
  \renewcommand{\tablename}{કોષ્ટક}
  \renewcommand{\figurename}{આકૃતિ}
}

% Helper for TikZ nodes to ensure Gujarati font
\newcommand{\gu}[1]{{\gujaratifont #1}}

% Custom environments
\newtcolorbox{solutionbox}{
    breakable,
    enhanced,
    colback=solutioncolor!5!white,
    colframe=solutioncolor!75!black,
    fonttitle=\bfseries,
    title=જવાબ
}

\newtcolorbox{solutionboxnobreak}{
 colback=solutioncolor!5!white,
 colframe=solutioncolor!75!black,
 fonttitle=\bfseries,
 title=જવાબ
}

\newtcolorbox{keyformula}{
 breakable,
 enhanced,
 colback=keycolor!5!white,
 colframe=keycolor!75!black,
 fonttitle=\bfseries,
 title=રાસાયણિક સમીકરણ/સૂત્ર
}

\newtcolorbox{mnemonicbox}{
 breakable,
 enhanced,
 colback=mnemoniccolor!5!white,
 colframe=mnemoniccolor!75!black,
 fonttitle=\bfseries,
 title=મેમરી ટ્રીક
}


\begin{document}

\begin{center}
{\Huge\bfseries\color{headcolor} Subject Name (Gujarati)}\\[5pt]
{\LARGE 4361106 -- Summer 2024}\\[3pt]
{\large Semester 1 Study Material}\\[3pt]
{\normalsize\textit{Detailed Solutions and Explanations}}
\end{center}

\vspace{10pt}

\subsection*{પ્રશ્ન 1(અ) [3
ગુણ]}\label{uxaaauxab0uxab6uxaa8-1uxa85-3-uxa97uxaa3}

\textbf{રિન્યુએબલ એનર્જી શું છે? તેનું મહત્વ સમજાવો.}

\begin{solutionbox}
રિન્યુએબલ એનર્જી એ કુદરતી સ્ત્રોતોમાંથી મેળવાતી ઊર્જા છે જે સમય સાથે
ફરીથી બનતી રહે છે, જેમ કે સૌર, પવન, જળ, બાયોમાસ અને ભૂગર્ભીય ઊર્જા.

\textbf{ટેબલ: રિન્યુએબલ એનર્જીનું મહત્વ}

{\def\LTcaptype{none} % do not increment counter
\begin{longtable}[]{@{}ll@{}}
\toprule\noalign{}
પાસું & ફાયદો \\
\midrule\noalign{}
\endhead
\bottomrule\noalign{}
\endlastfoot
\textbf{પર્યાવરણીય} & ગ્રીનહાઉસ ગેસ ઉત્સર્જન અને પ્રદૂષણ ઘટાડે છે \\
\textbf{આર્થિક} & નોકરીઓ બનાવે છે અને લાંબા ગાળે ઊર્જા ખર્ચ ઘટાડે છે \\
\textbf{ઊર્જા સુરક્ષા} & અશ્મિભૂત ઇંધણની આયાત પર નિર્ભરતા ઘટાડે છે \\
\textbf{ટકાઉપણું} & ભાવિ પેઢીઓ માટે અખૂટ ઊર્જા સ્ત્રોતો \\
\end{longtable}
}

\textbf{મુખ્ય મુદ્દાઓ:}

\begin{itemize}
\tightlist
\item
  \textbf{સ્વચ્છ ઊર્જા}: કામગીરી દરમિયાન શૂન્ય કાર્બન ઉત્સર્જન
\item
  \textbf{ખર્ચ-અસરકારક}: ઘટતી ટેકનોલોજી કિંમતો તેને આર્થિક બનાવે છે
\item
  \textbf{રોજગાર સર્જન}: વધતો ઉદ્યોગ રોજગારની તકો પૂરી પાડે છે
\end{itemize}

\textbf{યાદ રાખવાની ટેકનીક:} ``EEES'' - Environmental protection,
Economic benefits, Energy security, Sustainability

\end{solutionbox}
\begin{center}\rule{0.5\linewidth}{0.5pt}\end{center}

\subsection*{પ્રશ્ન 1(બ) [4
ગુણ]}\label{uxaaauxab0uxab6uxaa8-1uxaac-4-uxa97uxaa3}

\textbf{ઇલેક્ટ્રિક વાહનોના પ્રકારોની યાદી બનાવો. દરેકને સંક્ષિપ્તમાં સમજાવો.}

\begin{solutionbox}

\textbf{ટેબલ: ઇલેક્ટ્રિક વાહનોના પ્રકારો}

{\def\LTcaptype{none} % do not increment counter
\begin{longtable}[]{@{}
  >{\raggedright\arraybackslash}p{(\linewidth - 4\tabcolsep) * \real{0.2000}}
  >{\raggedright\arraybackslash}p{(\linewidth - 4\tabcolsep) * \real{0.3667}}
  >{\raggedright\arraybackslash}p{(\linewidth - 4\tabcolsep) * \real{0.4333}}@{}}
\toprule\noalign{}
\begin{minipage}[b]{\linewidth}\raggedright
પ્રકાર
\end{minipage} & \begin{minipage}[b]{\linewidth}\raggedright
સંપૂર્ણ નામ
\end{minipage} & \begin{minipage}[b]{\linewidth}\raggedright
વર્ણન
\end{minipage} \\
\midrule\noalign{}
\endhead
\bottomrule\noalign{}
\endlastfoot
\textbf{BEV} & Battery Electric Vehicle & સંપૂર્ણ ઇલેક્ટ્રિક, માત્ર બેટરીથી ચાલે
છે \\
\textbf{HEV} & Hybrid Electric Vehicle & ગેસોલિન એન્જિન અને ઇલેક્ટ્રિક મોટરનું
મિશ્રણ \\
\textbf{PHEV} & Plug-in Hybrid Electric Vehicle & બાહ્ય પાવર સ્ત્રોતથી ચાર્જ
કરી શકાય છે \\
\textbf{FCEV} & Fuel Cell Electric Vehicle & પાવર માટે હાઇડ્રોજન ફ્યૂઅલ સેલનો
ઉપયોગ \\
\end{longtable}
}

\textbf{મુખ્ય લક્ષણો:}

\begin{itemize}
\tightlist
\item
  \textbf{BEV}: શૂન્ય ઉત્સર્જન, ચાર્જિંગ સ્ટેશનની જરૂર
\item
  \textbf{HEV}: બહેતર ઇંધણ દક્ષતા, રિજનરેટિવ બ્રેકિંગ દ્વારા સ્વ-ચાર્જિંગ
\item
  \textbf{PHEV}: બેવડા પાવર વિકલ્પો, વિસ્તૃત રેન્જ
\item
  \textbf{FCEV}: ઝડપી રિફ્યુઅલિંગ, એકમાત્ર ઉત્સર્જન પાણી
\end{itemize}

\textbf{યાદ રાખવાની ટેકનીક:} ``Big Hybrid Plug Fuel'' BEV, HEV, PHEV,
FCEV માટે

\end{solutionbox}
\begin{center}\rule{0.5\linewidth}{0.5pt}\end{center}

\subsection*{પ્રશ્ન 1(ક) [7
ગુણ]}\label{uxaaauxab0uxab6uxaa8-1uxa95-7-uxa97uxaa3}

\textbf{સૌર ઊર્જા અને સૌર થર્મલ ઊર્જા વચ્ચે શું તફાવત છે? હોમ સોલાર રૂફટોપ સિસ્ટમના
બ્લોક ડાયાગ્રામની ચર્ચા કરો.}

\begin{solutionbox}

\textbf{ટેબલ: સૌર ઊર્જા વિ સૌર થર્મલ ઊર્જા}

{\def\LTcaptype{none} % do not increment counter
\begin{longtable}[]{@{}lll@{}}
\toprule\noalign{}
પેરામીટર & સૌર ઊર્જા (PV) & સૌર થર્મલ ઊર્જા \\
\midrule\noalign{}
\endhead
\bottomrule\noalign{}
\endlastfoot
\textbf{રૂપાંતરણ} & સીધો સૂર્યપ્રકાશ વીજળીમાં & સૂર્યપ્રકાશ ગરમી ઊર્જામાં \\
\textbf{ટેકનોલોજી} & ફોટોવોલ્ટેઇક સેલ્સ & સોલાર કલેક્ટર્સ/પેનલ્સ \\
\textbf{આઉટપુટ} & વિદ્યુત ઊર્જા & ઉષ્મા ઊર્જા (ગરમ પાણી/વરાળ) \\
\textbf{ઉપયોગો} & પાવર જનરેશન, લાઇટિંગ & પાણી ગરમ કરવું, સ્પેસ હીટિંગ \\
\textbf{કાર્યક્ષમતા} & 15-22\% & 70-80\% \\
\end{longtable}
}

\textbf{બ્લોક ડાયાગ્રામ: હોમ સોલાર રૂફટોપ સિસ્ટમ}

\begin{verbatim}
flowchart LR
    A[Solar Panels] {-{-} B[DC Power]}
    B {-{-} C[Charge Controller]}
    C {-{-} D[Battery Bank]}
    C {-{-} E[Inverter]}
    E {-{-} F[AC Power]}
    F {-{-} G[Home Load]}
    F {-{-} H[Grid Connection]}
    I[Monitoring System] {-{-} C}
\end{verbatim}

\textbf{મુખ્ય ઘટકો:}

\begin{itemize}
\tightlist
\item
  \textbf{સોલાર પેનલ્સ}: સૂર્યપ્રકાશને DC વીજળીમાં ફેરવે છે
\item
  \textbf{ચાર્જ કંટ્રોલર}: બેટરી ચાર્જિંગ નિયંત્રિત કરે છે
\item
  \textbf{ઇન્વર્ટર}: DC ને AC પાવરમાં ફેરવે છે
\item
  \textbf{બેટરી બેંક}: વધારાની ઊર્જા સ્ટોર કરે છે
\item
  \textbf{ગ્રિડ કનેક્શન}: બે-માર્ગી પાવર ફ્લો
\end{itemize}

\textbf{યાદ રાખવાની ટેકનીક:} ``Solar Converts Battery Inverter Grid'' મુખ્ય
ઘટકો માટે

\end{solutionbox}
\begin{center}\rule{0.5\linewidth}{0.5pt}\end{center}

\subsection*{પ્રશ્ન 1(ક OR) [7
ગુણ]}\label{uxaaauxab0uxab6uxaa8-1uxa95-or-7-uxa97uxaa3}

\textbf{સૌર ફોટોવોલ્ટેઇક અસર શું છે? ફોટોવોલ્ટેઇક રૂપાંતરણનો સિદ્ધાંત સમજાવો.}

\begin{solutionbox}
સૌર ફોટોવોલ્ટેઇક અસર એ સેમિકંડક્ટર સામગ્રી પર પ્રકાશ પડતાં વિદ્યુત
પ્રવાહ ઉત્પન્ન થવાની ઘટના છે.

\textbf{ફોટોવોલ્ટેઇક રૂપાંતરણનો સિદ્ધાંત:}

\begin{verbatim}
flowchart LR
    A[Sunlight Photons] {-{-} B[P{-}N Junction]}
    B {-{-} C[Electron{-}Hole Pairs]}
    C {-{-} D[Electric Field Separation]}
    D {-{-} E[Current Flow]}
    E {-{-} F[External Circuit]}
\end{verbatim}

\textbf{કાર્યપ્રક્રિયા:}

\begin{itemize}
\tightlist
\item
  \textbf{ફોટોન શોષણ}: પ્રકાશ ફોટોન સેમિકંડક્ટર સામગ્રીને અથડાવે છે
\item
  \textbf{ઇલેક્ટ્રોન ઉત્તેજના}: ઇલેક્ટ્રોન્સ ઊર્જા મેળવીને કંડક્શન બેન્ડમાં જાય છે
\item
  \textbf{P-N જંક્શન}: વિદ્યુત ક્ષેત્ર બનાવીને ચાર્જ અલગ કરે છે
\item
  \textbf{કરંટ જનરેશન}: ઇલેક્ટ્રોન્સનો પ્રવાહ વિદ્યુત પ્રવાહ બનાવે છે
\end{itemize}

\textbf{મુખ્ય મુદ્દાઓ:}

\begin{itemize}
\tightlist
\item
  \textbf{ઊર્જા રૂપાંતરણ}: પ્રકાશ ઊર્જા \rightarrow વિદ્યુત ઊર્જા
\item
  \textbf{સેમિકંડક્ટર મટીરિયલ}: સામાન્ય રીતે સિલિકોન આધારિત
\item
  \textbf{સીધું રૂપાંતરણ}: કોઈ હલનચલન ભાગોની જરૂર નથી
\item
  \textbf{ક્વોન્ટમ અસર}: ફોટોઇલેક્ટ્રિક અસર પર આધારિત
\end{itemize}

\textbf{ટેબલ: PV સેલ સામગ્રીઓ}

{\def\LTcaptype{none} % do not increment counter
\begin{longtable}[]{@{}llll@{}}
\toprule\noalign{}
સામગ્રી & કાર્યક્ષમતા & કિંમત & ઉપયોગ \\
\midrule\noalign{}
\endhead
\bottomrule\noalign{}
\endlastfoot
\textbf{મોનોક્રિસ્ટલાઇન સિલિકોન} & 18-22\% & ઊંચી & રેસિડેન્શિયલ \\
\textbf{પોલિક્રિસ્ટલાઇન સિલિકોન} & 15-17\% & મધ્યમ & કોમર્શિયલ \\
\textbf{થિન ફિલ્મ} & 10-12\% & ઓછી & મોટા પાયે \\
\end{longtable}
}

\textbf{યાદ રાખવાની ટેકનીક:} ``Photons Push Electrons Producing Power''

\end{solutionbox}
\begin{center}\rule{0.5\linewidth}{0.5pt}\end{center}

\subsection*{પ્રશ્ન 2(અ) [3
ગુણ]}\label{uxaaauxab0uxab6uxaa8-2uxa85-3-uxa97uxaa3}

\textbf{નેનો ટેકનોલોજી શું છે? નેનો ટેકનોલોજી પર આધારિત કોઈપણ ત્રણ એપ્લિકેશનની
યાદી બનાવો.}

\begin{solutionbox}
નેનો ટેકનોલોજી એ મોલેક્યુલર અને પરમાણુ સ્તરે (1-100 નેનોમીટર)
પદાર્થોની હેરફેર વિજ્ઞાન છે.

\textbf{ટેબલ: નેનો ટેકનોલોજી એપ્લિકેશન્સ}

{\def\LTcaptype{none} % do not increment counter
\begin{longtable}[]{@{}lll@{}}
\toprule\noalign{}
એપ્લિકેશન & વર્ણન & ફાયદો \\
\midrule\noalign{}
\endhead
\bottomrule\noalign{}
\endlastfoot
\textbf{મેડિકલ} & ડ્રગ ડિલિવરી સિસ્ટમ, કેન્સર ટ્રીટમેન્ટ & લક્ષિત ઉપચાર \\
\textbf{ઇલેક્ટ્રોનિક્સ} & નાના, ઝડપી પ્રોસેસર અને મેમોરી & ઉચ્ચ કાર્યક્ષમતા \\
\textbf{ઊર્જા} & સોલાર સેલ્સ, બેટરીઓ, ફ્યૂઅલ સેલ્સ & બહેતર કાર્યક્ષમતા \\
\end{longtable}
}

\textbf{મુખ્ય મુદ્દાઓ:}

\begin{itemize}
\tightlist
\item
  \textbf{સ્કેલ}: નેનોમીટર સ્તરે કામ કરે છે (10^{-}^{9} મીટર)
\item
  \textbf{ચોકસાઈ}: પરમાણુ સ્તરે હેરફેર
\item
  \textbf{ક્રાંતિકારી}: વિવિધ ઉદ્યોગોનું રૂપાંતરણ
\end{itemize}

\textbf{યાદ રાખવાની ટેકનીક:} ``Nano Makes Everything Better'' - Medical,
Electronics, Energy

\end{solutionbox}
\begin{center}\rule{0.5\linewidth}{0.5pt}\end{center}

\subsection*{પ્રશ્ન 2(બ) [4
ગુણ]}\label{uxaaauxab0uxab6uxaa8-2uxaac-4-uxa97uxaa3}

\textbf{મહત્વપૂર્ણ ઉભરતી નવીનીકરણીય ઊર્જા તકનીક તરીકે ભરતી તરંગ ઊર્જા પર ટૂંકી
નોંધ લખો.}

\begin{solutionbox}
ભરતી તરંગ ઊર્જા સમુદ્રી ભરતીઓ અને તરંગોની ગતિશીલ ઊર્જાનો ઉપયોગ
કરીને વીજળી ઉત્પન્ન કરે છે.

\textbf{મુખ્ય લક્ષણો:}

\begin{itemize}
\tightlist
\item
  \textbf{પૂર્વાનુમાન}: ભરતી નિયમિત પેટર્ન અનુસરે છે
\item
  \textbf{ઉચ્ચ ઘનતા}: પાણી હવા કરતાં 800 ગણું ઘન છે
\item
  \textbf{સ્થિર}: દિવસ-રાત ઉપલબ્ધ
\item
  \textbf{સ્વચ્છ}: કોઈ ઉત્સર્જન અથવા બળતણ વપરાશ નથી
\end{itemize}

\textbf{ટેબલ: ભરતી ઊર્જા સિસ્ટમ્સ}

{\def\LTcaptype{none} % do not increment counter
\begin{longtable}[]{@{}lll@{}}
\toprule\noalign{}
પ્રકાર & પદ્ધતિ & ફાયદો \\
\midrule\noalign{}
\endhead
\bottomrule\noalign{}
\endlastfoot
\textbf{ટાઇડલ બેરેજ} & નદીમુખ પર બંધ & ઉચ્ચ પાવર આઉટપુટ \\
\textbf{ટાઇડલ સ્ટ્રીમ} & પાણીની અંદર ટર્બાઇન & ન્યૂનતમ પર્યાવરણીય અસર \\
\textbf{વેવ એનર્જી} & સપાટીના તરંગ ગતિ & વિપુલ સંસાધન \\
\end{longtable}
}

\textbf{ઉપયોગો:}

\begin{itemize}
\tightlist
\item
  \textbf{કોસ્ટલ પાવર જનરેશન}: દૂરના દરિયાકાંઠાના સમુદાયો
\item
  \textbf{ગ્રિડ ઇન્ટિગ્રેશન}: અન્ય નવીનીકરણીય સ્ત્રોતોના પૂરક
\item
  \textbf{આઇલેન્ડ નેશન્સ}: દરિયાઈ દેશો માટે આદર્શ
\end{itemize}

\textbf{યાદ રાખવાની ટેકનીક:} ``Tides Provide Predictable Power''

\end{solutionbox}
\begin{center}\rule{0.5\linewidth}{0.5pt}\end{center}

\subsection*{પ્રશ્ન 2(ક) [7
ગુણ]}\label{uxaaauxab0uxab6uxaa8-2uxa95-7-uxa97uxaa3}

\textbf{સ્માર્ટ વોટર મોનિટરિંગ સિસ્ટમ શું છે? સ્માર્ટ વોટર ક્વોલિટી મોનિટરિંગ
સિસ્ટમનો બ્લોક ડાયાગ્રામ સમજાવો.}

\begin{solutionbox}
સ્માર્ટ વોટર મોનિટરિંગ સિસ્ટમ IoT સેન્સર્સનો ઉપયોગ કરીને પાણીની
ગુણવત્તાના પેરામીટર્સનું સતત નિરીક્ષણ કરે છે અને નિર્ણય લેવા માટે રીઅલ-ટાઇમ ડેટા પ્રદાન
કરે છે.

\textbf{બ્લોક ડાયાગ્રામ: સ્માર્ટ વોટર ક્વોલિટી મોનિટરિંગ સિસ્ટમ}

\begin{verbatim}
flowchart LR
    A[Water Source] {-{-} B[Sensor Array]}
    B {-{-} C[pH Sensor]}
    B {-{-} D[Turbidity Sensor]}
    B {-{-} E[Temperature Sensor]}
    B {-{-} F[Dissolved Oxygen Sensor]}
    C {-{-} G[Microcontroller]}
    D {-{-} G}
    E {-{-} G}
    F {-{-} G}
    G {-{-} H[Data Processing]}
    H {-{-} I[Wireless Communication]}
    I {-{-} J[Cloud Server]}
    J {-{-} K[Mobile App/Web Dashboard]}
    J {-{-} L[Alert System]}
\end{verbatim}

\textbf{મુખ્ય ઘટકો:}

\begin{itemize}
\tightlist
\item
  \textbf{સેન્સર્સ}: pH, ટર્બિડિટી, તાપમાન, ઓગળેલા ઓક્સિજનનું નિરીક્ષણ
\item
  \textbf{માઇક્રોકંટ્રોલર}: ડેટા પ્રોસેસિંગ માટે Arduino/Raspberry Pi
\item
  \textbf{કમ્યુનિકેશન}: ડેટા ટ્રાન્સમિશન માટે WiFi/GSM
\item
  \textbf{ક્લાઉડ પ્લેટફોર્મ}: ડેટા સ્ટોરેજ અને વિશ્લેષણ
\item
  \textbf{યુઝર ઇન્ટરફેસ}: મોનિટરિંગ માટે મોબાઇલ એપ
\end{itemize}

\textbf{ફાયદા:}

\begin{itemize}
\tightlist
\item
  \textbf{રીઅલ-ટાઇમ મોનિટરિંગ}: સતત પાણીની ગુણવત્તા મૂલ્યાંકન
\item
  \textbf{અર્લી વોર્નિંગ}: દૂષણ માટે તાત્કાલિક અલર્ટ
\item
  \textbf{ડેટા એનાલિટિક્સ}: ઐતિહાસિક પ્રવૃત્તિઓ અને અનુમાનો
\item
  \textbf{ખર્ચ અસરકારક}: મેન્યુઅલ પરીક્ષણ ખર્ચ ઘટાડે છે
\end{itemize}

\textbf{ટેબલ: પાણીની ગુણવત્તાના પેરામીટર્સ}

{\def\LTcaptype{none} % do not increment counter
\begin{longtable}[]{@{}lll@{}}
\toprule\noalign{}
પેરામીટર & સામાન્ય રેન્જ & સેન્સર પ્રકાર \\
\midrule\noalign{}
\endhead
\bottomrule\noalign{}
\endlastfoot
\textbf{pH} & 6.5-8.5 & pH ઇલેક્ટ્રોડ \\
\textbf{ટર્બિડિટી} & \textless1 NTU & ઓપ્ટિકલ સેન્સર \\
\textbf{તાપમાન} & 15-25^\circC & થર્મિસ્ટર \\
\textbf{ઓગળેલા ઓક્સિજન} & \textgreater5 mg/L & ઇલેક્ટ્રોકેમિકલ \\
\end{longtable}
}

\textbf{યાદ રાખવાની ટેકનીક:} ``Smart Sensors Send Signals Safely''

\end{solutionbox}
\begin{center}\rule{0.5\linewidth}{0.5pt}\end{center}

\subsection*{પ્રશ્ન 2(અ OR) [3
ગુણ]}\label{uxaaauxab0uxab6uxaa8-2uxa85-or-3-uxa97uxaa3}

\textbf{વેરેબલ ટેકનોલોજી શું છે? વેરેબલ ટેકનોલોજીની ઓછામાં ઓછી બે એપ્લિકેશનના નામ
આપો?}

\begin{solutionbox}
વેરેબલ ટેકનોલોજી એ ઇલેક્ટ્રોનિક ઉપકરણો છે જે કપડાં અથવા એક્સેસરીઝ
તરીકે પહેરી શકાય છે, જેમાં સ્માર્ટ સેન્સર્સ અને કનેક્ટિવિટી સામેલ છે.

\textbf{એપ્લિકેશન્સ:}

\begin{itemize}
\tightlist
\item
  \textbf{આરોગ્ય નિરીક્ષણ}: હાર્ટ રેટ, પગલાં, ઊંઘની પેટર્ન ટ્રેક કરતી સ્માર્ટવોચ
\item
  \textbf{ફિટનેસ ટ્રેકિંગ}: કેલોરી, અંતર, કસરતનું માપ કરતા એક્ટિવિટી મોનિટર્સ
\item
  \textbf{મેડિકલ ડિવાઇસેસ}: સતત ગ્લુકોઝ મોનિટર્સ, બ્લડ પ્રેશર મોનિટર્સ
\item
  \textbf{સ્માર્ટ ગ્લાસીસ}: ઓગમેન્ટેડ રિયાલિટી ડિસ્પ્લે, હેન્ડ્સ-ફ્રી કમ્પ્યુટિંગ
\end{itemize}

\textbf{મુખ્ય લક્ષણો:}

\begin{itemize}
\tightlist
\item
  \textbf{પોર્ટેબલ}: હળવા અને પહેરવા માટે આરામદાયક
\item
  \textbf{કનેક્ટેડ}: સ્માર્ટફોન સાથે Bluetooth/WiFi કનેક્ટિવિટી
\item
  \textbf{સેન્સર-રિચ}: ડેટા એકત્રીકરણ માટે બહુવિધ સેન્સર્સ
\end{itemize}

\textbf{યાદ રાખવાની ટેકનીક:} ``Wearables Watch Wellness Wirelessly''

\end{solutionbox}
\begin{center}\rule{0.5\linewidth}{0.5pt}\end{center}

\subsection*{પ્રશ્ન 2(બ OR) [4
ગુણ]}\label{uxaaauxab0uxab6uxaa8-2uxaac-or-4-uxa97uxaa3}

\textbf{વિવિધ પ્રકારના સોલાર સેલની યાદી બનાવો. ઇલેક્ટ્રિક વાહન માટે વિવિધ ઊર્જા
સ્ત્રોતોની યાદી બનાવો.}

\begin{solutionbox}

\textbf{ટેબલ: સોલાર સેલના પ્રકારો}

{\def\LTcaptype{none} % do not increment counter
\begin{longtable}[]{@{}llll@{}}
\toprule\noalign{}
પ્રકાર & સામગ્રી & કાર્યક્ષમતા & કિંમત \\
\midrule\noalign{}
\endhead
\bottomrule\noalign{}
\endlastfoot
\textbf{મોનોક્રિસ્ટલાઇન} & સિંગલ ક્રિસ્ટલ સિલિકોન & 18-22\% & ઊંચી \\
\textbf{પોલિક્રિસ્ટલાઇન} & મલ્ટિ-ક્રિસ્ટલ સિલિકોન & 15-17\% & મધ્યમ \\
\textbf{થિન ફિલ્મ} & એમોર્ફસ સિલિકોન & 10-12\% & ઓછી \\
\textbf{કેડમિયમ ટેલ્યુરાઇડ} & CdTe કમ્પાઉન્ડ & 16-18\% & મધ્યમ \\
\end{longtable}
}

\textbf{ટેબલ: ઇલેક્ટ્રિક વાહનો માટે ઊર્જા સ્ત્રોતો}

{\def\LTcaptype{none} % do not increment counter
\begin{longtable}[]{@{}lll@{}}
\toprule\noalign{}
સ્ત્રોત & વર્ણન & ફાયદો \\
\midrule\noalign{}
\endhead
\bottomrule\noalign{}
\endlastfoot
\textbf{બેટરી} & લિથિયમ-આયન સેલ્સ & ઉચ્ચ ઊર્જા ઘનતા \\
\textbf{ફ્યૂઅલ સેલ} & હાઇડ્રોજન રૂપાંતરણ & ઝડપી રિફ્યુઅલિંગ \\
\textbf{અલ્ટ્રાકેપેસિટર} & ઝડપી ચાર્જ/ડિસચાર્જ & ફાસ્ટ ચાર્જિંગ \\
\textbf{રિજનરેટિવ બ્રેકિંગ} & ગતિશીલ ઊર્જા પુનઃપ્રાપ્તિ & ઊર્જા કાર્યક્ષમતા \\
\end{longtable}
}

\textbf{યાદ રાખવાની ટેકનીક:} ``Solar: Mono Poly Thin Cadmium'' / ``EV:
Battery Fuel Ultra Regen''

\end{solutionbox}
\begin{center}\rule{0.5\linewidth}{0.5pt}\end{center}

\subsection*{પ્રશ્ન 2(ક OR) [7
ગુણ]}\label{uxaaauxab0uxab6uxaa8-2uxa95-or-7-uxa97uxaa3}

\textbf{ડ્રોનના બ્લોક ડાયાગ્રામ અને તેના મુખ્ય ઘટકોનું વર્ણન કરો.}

\begin{solutionbox}

\textbf{બ્લોક ડાયાગ્રામ: ડ્રોન સિસ્ટમ}

\begin{verbatim}
flowchart TD
    A[Flight Controller] {-{-} B[ESC 1]}
    A {-{-} C[ESC 2]}
    A {-{-} D[ESC 3]}
    A {-{-} E[ESC 4]}
    B {-{-} F[Motor 1]}
    C {-{-} G[Motor 2]}
    D {-{-} H[Motor 3]}
    E {-{-} I[Motor 4]}
    J[GPS Module] {-{-} A}
    K[IMU Sensors] {-{-} A}
    L[Battery] {-{-} A}
    M[Camera/Gimbal] {-{-} A}
    N[Radio Receiver] {-{-} A}
    O[Remote Controller] {-{-} N}
\end{verbatim}

\textbf{મુખ્ય ઘટકો:}

\textbf{ટેબલ: ડ્રોન ઘટકો}

{\def\LTcaptype{none} % do not increment counter
\begin{longtable}[]{@{}lll@{}}
\toprule\noalign{}
ઘટક & કાર્ય & મહત્વ \\
\midrule\noalign{}
\endhead
\bottomrule\noalign{}
\endlastfoot
\textbf{ફ્લાઇટ કંટ્રોલર} & સેન્ટ્રલ પ્રોસેસિંગ યુનિટ & ડ્રોનનું મગજ \\
\textbf{ESC} & મોટર સ્પીડ કંટ્રોલ & ચોક્કસ મોટર કંટ્રોલ \\
\textbf{મોટર્સ અને પ્રોપેલર્સ} & થ્રસ્ટ જનરેટ કરે છે & ફ્લાઇટ ક્ષમતા \\
\textbf{બેટરી} & પાવર સપ્લાય & ફ્લાઇટ અવધિ \\
\textbf{GPS} & પોઝિશન ટ્રેકિંગ & નેવિગેશન \\
\textbf{IMU} & મોશન સેન્સિંગ & સ્ટેબિલિટી કંટ્રોલ \\
\end{longtable}
}

\textbf{મુખ્ય સિસ્ટમ્સ:}

\begin{itemize}
\tightlist
\item
  \textbf{પ્રોપલ્શન સિસ્ટમ}: લિફ્ટ અને કંટ્રોલ માટે 4 મોટર્સ પ્રોપેલર્સ સાથે
\item
  \textbf{કંટ્રોલ સિસ્ટમ}: સ્ટેબિલાઇઝેશન એલ્ગોરિધમ સાથે ફ્લાઇટ કંટ્રોલર
\item
  \textbf{નેવિગેશન સિસ્ટમ}: પોઝિશનિંગ માટે GPS અને કંપાસ
\item
  \textbf{પાવર સિસ્ટમ}: ઇલેક્ટ્રિકલ પાવર માટે LiPo બેટરી
\item
  \textbf{કમ્યુનિકેશન}: ગ્રાઉન્ડ કંટ્રોલર સાથે રેડિયો લિંક
\end{itemize}

\textbf{કાર્યસિદ્ધાંત:}

\begin{itemize}
\tightlist
\item
  \textbf{લિફ્ટ}: રોટર્સ ઉપરની દિશામાં થ્રસ્ટ બનાવે છે
\item
  \textbf{કંટ્રોલ}: વિવિધ રોટર સ્પીડ મૂવમેન્ટ કંટ્રોલ કરે છે
\item
  \textbf{સ્ટેબિલિટી}: સેન્સર્સ બેલેન્સ અને ઓરિએન્ટેશન જાળવે છે
\end{itemize}

\textbf{યાદ રાખવાની ટેકનીક:} ``Drones Fly Using Motors, Electronics,
Sensors, Power''

\end{solutionbox}
\begin{center}\rule{0.5\linewidth}{0.5pt}\end{center}

\subsection*{પ્રશ્ન 3(અ) [3
ગુણ]}\label{uxaaauxab0uxab6uxaa8-3uxa85-3-uxa97uxaa3}

\textbf{IoT શું છે? IoT ના મુખ્ય ઘટકોની યાદી બનાવો.}

\begin{solutionbox}
IoT (Internet of Things) એ ભૌતિક ઉપકરણોનું નેટવર્ક છે જે ઇન્ટરનેટ
દ્વારા ડેટા એકત્રિત અને વિનિમય કરે છે.

\textbf{ટેબલ: IoT ના મુખ્ય ઘટકો}

{\def\LTcaptype{none} % do not increment counter
\begin{longtable}[]{@{}lll@{}}
\toprule\noalign{}
ઘટક & કાર્ય & ઉદાહરણ \\
\midrule\noalign{}
\endhead
\bottomrule\noalign{}
\endlastfoot
\textbf{સેન્સર્સ} & ડેટા એકત્રીકરણ & તાપમાન, ભેજ સેન્સર્સ \\
\textbf{કનેક્ટિવિટી} & ડેટા ટ્રાન્સમિશન & WiFi, Bluetooth, GSM \\
\textbf{ડેટા પ્રોસેસિંગ} & માહિતી વિશ્લેષણ & ક્લાઉડ કમ્પ્યુટિંગ \\
\textbf{યુઝર ઇન્ટરફેસ} & માનવીય ક્રિયાપ્રતિક્રિયા & મોબાઇલ એપ્સ, ડેશબોર્ડ \\
\end{longtable}
}

\textbf{મુખ્ય લક્ષણો:}

\begin{itemize}
\tightlist
\item
  \textbf{આંતરકનેક્ટેડ}: ઉપકરણો એકબીજા સાથે વાતચીત કરે છે
\item
  \textbf{સ્માર્ટ}: સ્વચાલિત નિર્ણય લેવું
\item
  \textbf{ડેટા-ડ્રિવન}: સતત નિરીક્ષણ અને વિશ્લેષણ
\end{itemize}

\textbf{યાદ રાખવાની ટેકનીક:} ``IoT Connects Smart Devices Using
Internet''

\end{solutionbox}
\begin{center}\rule{0.5\linewidth}{0.5pt}\end{center}

\subsection*{પ્રશ્ન 3(બ) [4
ગુણ]}\label{uxaaauxab0uxab6uxaa8-3uxaac-4-uxa97uxaa3}

\textbf{કાર્બનિક અને અકાર્બનિક ઇલેક્ટ્રોનિક્સ વચ્ચે સરખામણી કરો.}

\begin{solutionbox}

\textbf{ટેબલ: કાર્બનિક વિ અકાર્બનિક ઇલેક્ટ્રોનિક્સ}

{\def\LTcaptype{none} % do not increment counter
\begin{longtable}[]{@{}lll@{}}
\toprule\noalign{}
પેરામીટર & કાર્બનિક ઇલેક્ટ્રોનિક્સ & અકાર્બનિક ઇલેક્ટ્રોનિક્સ \\
\midrule\noalign{}
\endhead
\bottomrule\noalign{}
\endlastfoot
\textbf{સામગ્રી} & કાર્બન આધારિત સંયોજનો & સિલિકોન, ધાતુઓ \\
\textbf{ઉત્પાદન} & ઓછું તાપમાન, પ્રિન્ટિંગ & ઊંચું તાપમાન, ક્લીન રૂમ \\
\textbf{લવચીકતા} & લવચીક, વળી શકાય તેવું & કઠોર, બરડ \\
\textbf{કિંમત} & ઓછી ઉત્પાદન કિંમત & ઊંચી ઉત્પાદન કિંમત \\
\textbf{કાર્યક્ષમતા} & ઓછી ઝડપ, કાર્યક્ષમતા & ઊંચી ઝડપ, કાર્યક્ષમતા \\
\textbf{એપ્લિકેશન્સ} & ડિસ્પ્લે, સોલાર સેલ્સ & પ્રોસેસર્સ, મેમોરી \\
\end{longtable}
}

\textbf{મુખ્ય તફાવતો:}

\begin{itemize}
\tightlist
\item
  \textbf{પ્રોસેસિંગ}: કાર્બનિક સોલ્યુશન આધારિત પ્રોસેસિંગ વાપરે છે
\item
  \textbf{સબસ્ટ્રેટ}: કાર્બનિક પ્લાસ્ટિક સબસ્ટ્રેટ વાપરી શકે છે
\item
  \textbf{ટકાઉપણું}: અકાર્બનિક વધુ સ્થિર અને ટકાઉ
\item
  \textbf{નવીનતા}: કાર્બનિક નવા ફોર્મ ફેક્ટર્સ સક્ષમ કરે છે
\end{itemize}

\textbf{યાદ રાખવાની ટેકનીક:} ``Organic: Flexible, Cheap, Printable vs
Inorganic: Fast, Stable, Expensive''

\end{solutionbox}
\begin{center}\rule{0.5\linewidth}{0.5pt}\end{center}

\subsection*{પ્રશ્ન 3(ક) [7
ગુણ]}\label{uxaaauxab0uxab6uxaa8-3uxa95-7-uxa97uxaa3}

\textbf{સ્માર્ટ સ્ટ્રીટ લાઇટ કંટ્રોલ અને મોનિટરિંગ સિસ્ટમનો બ્લોક ડાયાગ્રામ દોરો.
ઉદ્યોગમાં AR/VR ટેકનોલોજીના ફાયદા અને ઉપયોગની ચર્ચા કરો.}

\begin{solutionbox}

\textbf{બ્લોક ડાયાગ્રામ: સ્માર્ટ સ્ટ્રીટ લાઇટ સિસ્ટમ}

\begin{verbatim}
flowchart LR
    A[Light Sensor] {-{-} B[Microcontroller]}
    C[Motion Sensor] {-{-} B}
    D[Remote Control] {-{-} B}
    B {-{-} E[LED Driver]}
    E {-{-} F[LED Street Light]}
    B {-{-} G[Wireless Module]}
    G {-{-} H[Central Control]}
    H {-{-} I[Monitoring Dashboard]}
\end{verbatim}

\textbf{ઉદ્યોગમાં AR/VR ટેકનોલોજી:}

\textbf{ટેબલ: AR/VR એપ્લિકેશન્સ}

{\def\LTcaptype{none} % do not increment counter
\begin{longtable}[]{@{}lll@{}}
\toprule\noalign{}
ઉદ્યોગ & AR એપ્લિકેશન & VR એપ્લિકેશન \\
\midrule\noalign{}
\endhead
\bottomrule\noalign{}
\endlastfoot
\textbf{મેન્યુફેક્ચરિંગ} & એસેમ્બલી સૂચનાઓ & ટ્રેનિંગ સિમ્યુલેશન \\
\textbf{હેલ્થકેર} & સર્જરી સહાયતા & મેડિકલ ટ્રેનિંગ \\
\textbf{શિક્ષણ} & ઇન્ટરેક્ટિવ લર્નિંગ & વર્ચ્યુઅલ ક્લાસરૂમ \\
\textbf{રિટેલ} & પ્રોડક્ટ વિઝ્યુઅલાઇઝેશન & વર્ચ્યુઅલ શોરૂમ \\
\end{longtable}
}

\textbf{ફાયદા:}

\begin{itemize}
\tightlist
\item
  \textbf{વિકસિત પ્રશિક્ષણ}: સુરક્ષિત, પુનરાવર્તિત શીખવાનું વાતાવરણ
\item
  \textbf{રિમોટ કોલેબોરેશન}: વર્ચ્યુઅલ મીટિંગ્સ અને શેર્ડ વર્કસ્પેસ
\item
  \textbf{ડિઝાઇન વિઝ્યુઅલાઇઝેશન}: 3D પ્રોટોટાઇપિંગ અને મોડેલિંગ
\item
  \textbf{મેઇન્ટેનન્સ સપોર્ટ}: રીઅલ-ટાઇમ માર્ગદર્શન અને સમસ્યા નિવારણ
\end{itemize}

\textbf{મુખ્ય ફાયદા:}

\begin{itemize}
\tightlist
\item
  \textbf{કિંમત ઘટાડો}: ઓછા પ્રશિક્ષણ અને પ્રવાસ ખર્ચ
\item
  \textbf{સલામતી}: જોખમ-મુક્ત પ્રશિક્ષણ વાતાવરણ
\item
  \textbf{કાર્યક્ષમતા}: ઝડપી શીખવું અને સમસ્યા-નિવારણ
\item
  \textbf{નવીનતા}: માનવ-કમ્પ્યુટર ક્રિયાપ્રતિક્રિયાની નવી રીતો
\end{itemize}

\textbf{યાદ રાખવાની ટેકનીક:} ``AR/VR: Training, Design, Remote,
Maintenance''

\end{solutionbox}
\begin{center}\rule{0.5\linewidth}{0.5pt}\end{center}

\subsection*{પ્રશ્ન 3(અ OR) [3
ગુણ]}\label{uxaaauxab0uxab6uxaa8-3uxa85-or-3-uxa97uxaa3}

\textbf{સ્માર્ટ સિસ્ટમ શું છે? કોઈપણ ચાર પ્રકારની સ્માર્ટ સિસ્ટમની યાદી બનાવો.}

\begin{solutionbox}
સ્માર્ટ સિસ્ટમ એ બુદ્ધિશાળી સિસ્ટમ છે જે સેન્સર્સ, ડેટા પ્રોસેસિંગ અને
ઓટોમેશનનો ઉપયોગ કરીને નિર્ણયો લે છે અને બદલાતી પરિસ્થિતિઓમાં અનુકૂલન કરે છે.

\textbf{ટેબલ: સ્માર્ટ સિસ્ટમના પ્રકારો}

{\def\LTcaptype{none} % do not increment counter
\begin{longtable}[]{@{}lll@{}}
\toprule\noalign{}
પ્રકાર & વર્ણન & ઉદાહરણ \\
\midrule\noalign{}
\endhead
\bottomrule\noalign{}
\endlastfoot
\textbf{સ્માર્ટ હોમ} & સ્વચાલિત ઘર નિયંત્રણ & લાઇટિંગ, HVAC, સિક્યુરિટી \\
\textbf{સ્માર્ટ સિટી} & શહેરી ઇન્ફ્રાસ્ટ્રક્ચર મેનેજમેન્ટ & ટ્રાફિક, યુટિલિટીઝ,
કચરો \\
\textbf{સ્માર્ટ ગ્રિડ} & બુદ્ધિશાળી પાવર વિતરણ & ઊર્જા મેનેજમેન્ટ \\
\textbf{સ્માર્ટ હેલ્થકેર} & મેડિકલ મોનિટરિંગ સિસ્ટમ & દર્દી મોનિટરિંગ,
ડાયાગ્નોસ્ટિક્સ \\
\end{longtable}
}

\textbf{મુખ્ય લક્ષણો:}

\begin{itemize}
\tightlist
\item
  \textbf{સ્વચાલિત}: સ્વ-સંચાલન ક્ષમતાઓ
\item
  \textbf{કનેક્ટેડ}: ઇન્ટરનેટ કનેક્ટિવિટી
\item
  \textbf{અનુકૂલનશીલ}: સમય સાથે શીખવું અને સુધારવું
\end{itemize}

\textbf{યાદ રાખવાની ટેકનીક:} ``Smart: Home, City, Grid, Health''

\end{solutionbox}
\begin{center}\rule{0.5\linewidth}{0.5pt}\end{center}

\subsection*{પ્રશ્ન 3(બ OR) [4
ગુણ]}\label{uxaaauxab0uxab6uxaa8-3uxaac-or-4-uxa97uxaa3}

\textbf{ઓર્ગેનિક ઇલેક્ટ્રોનિક્સના ફાયદા અને એપ્લિકેશનની યાદી બનાવો.}

\begin{solutionbox}

\textbf{ટેબલ: ઓર્ગેનિક ઇલેક્ટ્રોનિક્સના ફાયદા}

{\def\LTcaptype{none} % do not increment counter
\begin{longtable}[]{@{}lll@{}}
\toprule\noalign{}
ફાયદો & વર્ણન & લાભ \\
\midrule\noalign{}
\endhead
\bottomrule\noalign{}
\endlastfoot
\textbf{લવચીકતા} & વળી શકાય, ખેંચાય તેવું & પહેરી શકાય તેવા ઉપકરણો \\
\textbf{ઓછી કિંમત} & સસ્તું ઉત્પાદન & મોટા પાયે ઉત્પાદન \\
\textbf{મોટો વિસ્તાર} & મોટી સપાટી પર પ્રિન્ટિંગ & મોટા ડિસ્પ્લે \\
\textbf{ઓછું તાપમાન} & રૂમ ટેમ્પરેચર પ્રોસેસિંગ & ઊર્જા કાર્યક્ષમ \\
\end{longtable}
}

\textbf{એપ્લિકેશન્સ:}

\begin{itemize}
\tightlist
\item
  \textbf{OLED ડિસ્પ્લે}: સ્માર્ટફોન, TV, લાઇટિંગ
\item
  \textbf{ઓર્ગેનિક સોલાર સેલ્સ}: લવચીક સોલાર પેનલ્સ
\item
  \textbf{ઓર્ગેનિક ટ્રાન્ઝિસ્ટર}: લવચીક સર્કિટ્સ
\item
  \textbf{ઇલેક્ટ્રોનિક પેપર}: E-રીડર્સ, સ્માર્ટ લેબલ્સ
\end{itemize}

\textbf{મુખ્ય ફાયદા:}

\begin{itemize}
\tightlist
\item
  \textbf{હળવા}: પોર્ટેબલ ઉપકરણો માટે યોગ્ય
\item
  \textbf{પારદર્શક}: સી-થ્રુ ઇલેક્ટ્રોનિક્સ
\item
  \textbf{પર્યાવરણને અનુકૂળ}: બાયોડિગ્રેડેબલ સામગ્રી
\end{itemize}

\textbf{યાદ રાખવાની ટેકનીક:} ``Organic: Flexible, Cheap, Large,
Low-temp''

\end{solutionbox}
\begin{center}\rule{0.5\linewidth}{0.5pt}\end{center}

\subsection*{પ્રશ્ન 3(ક OR) [7
ગુણ]}\label{uxaaauxab0uxab6uxaa8-3uxa95-or-7-uxa97uxaa3}

\textbf{(i) પહેરી શકાય તેવી સ્માર્ટ ઘડિયાળ અને (ii) બાયોમેટ્રિક સિસ્ટમનો મૂળભૂત
બ્લોક ડાયાગ્રામ દોરો.}

\begin{solutionbox}

\textbf{(i) વેરેબલ સ્માર્ટ વોચ બ્લોક ડાયાગ્રામ:}

\begin{verbatim}
flowchart TD
    A[Sensors] {-{-} B[Microprocessor]}
    C[Display] {-{-} B}
    D[Battery] {-{-} B}
    E[Wireless Module] {-{-} B}
    B {-{-} F[Memory]}
    B {-{-} G[Charging Port]}
    H[Heart Rate Sensor] {-{-} A}
    I[Accelerometer] {-{-} A}
    J[GPS] {-{-} A}
\end{verbatim}

\textbf{(ii) બાયોમેટ્રિક સિસ્ટમ બ્લોક ડાયાગ્રામ:}

\begin{verbatim}
flowchart LR
    A[Biometric Sensor] {-{-} B[Signal Processing]}
    B {-{-} C[Feature Extraction]}
    C {-{-} D[Template Matching]}
    E[Database] {-{-} D}
    D {-{-} F[Decision Module]}
    F {-{-} G[Access Control]}
    H[Enrollment Module] {-{-} E}
\end{verbatim}

\textbf{સ્માર્ટ વોચ ઘટકો:}

\begin{itemize}
\tightlist
\item
  \textbf{સેન્સર્સ}: હાર્ટ રેટ, એક્સેલેરોમીટર, જાયરોસ્કોપ
\item
  \textbf{પ્રોસેસર}: ARM આધારિત માઇક્રોકંટ્રોલર
\item
  \textbf{ડિસ્પ્લે}: ટચસ્ક્રીન OLED/LCD
\item
  \textbf{કનેક્ટિવિટી}: Bluetooth, WiFi, સેલ્યુલર
\item
  \textbf{પાવર}: રિચાર્જેબલ લિથિયમ બેટરી
\end{itemize}

\textbf{બાયોમેટ્રિક સિસ્ટમ ઘટકો:}

\begin{itemize}
\tightlist
\item
  \textbf{સેન્સર મોડ્યુલ}: બાયોમેટ્રિક ડેટા કેપ્ચર કરે છે
\item
  \textbf{પ્રોસેસિંગ યુનિટ}: ફીચર્સનું વિશ્લેષણ અને નિષ્કર્ષણ
\item
  \textbf{ડેટાબેસ}: નોંધાયેલા ટેમ્પ્લેટ્સ સ્ટોર કરે છે
\item
  \textbf{મેચિંગ એન્જિન}: સ્ટોર કરેલા ડેટા સાથે સરખામણી
\item
  \textbf{ડિસિઝન લોજિક}: પ્રવેશ મંજૂર અથવા નકારે છે
\end{itemize}

\textbf{મુખ્ય લક્ષણો:}

\begin{itemize}
\tightlist
\item
  \textbf{ઓથેન્ટિકેશન}: સુરક્ષિત યુઝર આઇડેન્ટિફિકેશન
\item
  \textbf{રીઅલ-ટાઇમ}: તાત્કાલિક પ્રોસેસિંગ અને પ્રતિસાદ
\item
  \textbf{ચોકસાઈ}: આઇડેન્ટિફિકેશનમાં ઉચ્ચ ચોકસાઈ
\end{itemize}

\textbf{યાદ રાખવાની ટેકનીક:} ``Smart Watch: Sense, Process, Display,
Connect'' / ``Biometric: Capture, Process, Match, Decide''

\end{solutionbox}
\begin{center}\rule{0.5\linewidth}{0.5pt}\end{center}

\subsection*{પ્રશ્ન 4(અ) [3
ગુણ]}\label{uxaaauxab0uxab6uxaa8-4uxa85-3-uxa97uxaa3}

\textbf{રાસ્પબેરી પાઇમાં NOOBS, GPIO અને LXDE નું સંપૂર્ણ સ્વરૂપ આપો.}

\begin{solutionbox}

\textbf{ટેબલ: રાસ્પબેરી પાઇ સંક્ષેપ}

{\def\LTcaptype{none} % do not increment counter
\begin{longtable}[]{@{}lll@{}}
\toprule\noalign{}
સંક્ષેપ & સંપૂર્ણ સ્વરૂપ & હેતુ \\
\midrule\noalign{}
\endhead
\bottomrule\noalign{}
\endlastfoot
\textbf{NOOBS} & New Out Of Box Software & સરળ OS ઇન્સ્ટોલેશન \\
\textbf{GPIO} & General Purpose Input Output & હાર્ડવેર ઇન્ટરફેસ પિન્સ \\
\textbf{LXDE} & Lightweight X11 Desktop Environment & ડેસ્કટોપ ઇન્ટરફેસ \\
\end{longtable}
}

\textbf{કાર્યો:}

\begin{itemize}
\tightlist
\item
  \textbf{NOOBS}: શરૂઆતીઓ માટે રાસ્પબેરી પાઇ સેટઅપ સરળ બનાવે છે
\item
  \textbf{GPIO}: બાહ્ય હાર્ડવેર માટે 40-પિન કનેક્ટર
\item
  \textbf{LXDE}: યુઝર-ફ્રેન્ડલી ગ્રાફિકલ ઇન્ટરફેસ
\end{itemize}

\textbf{યાદ રાખવાની ટેકનીક:} ``New GPIO, Lightweight Experience''

\end{solutionbox}
\begin{center}\rule{0.5\linewidth}{0.5pt}\end{center}

\subsection*{પ્રશ્ન 4(બ) [4
ગુણ]}\label{uxaaauxab0uxab6uxaa8-4uxaac-4-uxa97uxaa3}

\textbf{OLED પર ટૂંકી નોંધ લખો.}

\begin{solutionbox}
OLED (Organic Light Emitting Diode) એ ડિસ્પ્લે ટેકનોલોજી છે જે
કાર્બનિક સંયોજનોનો ઉપયોગ કરે છે જે વિદ્યુત પ્રવાહ લાગુ કરવામાં આવે ત્યારે પ્રકાશ
ઉત્સર્જન કરે છે.

\textbf{મુખ્ય લક્ષણો:}

\begin{itemize}
\tightlist
\item
  \textbf{સ્વ-પ્રકાશિત}: બેકલાઇટની જરૂર નથી
\item
  \textbf{પાતળું પ્રોફાઇલ}: અત્યંત પાતળા ડિસ્પ્લે
\item
  \textbf{ઉચ્ચ કોન્ટ્રાસ્ટ}: સાચા કાળા પિક્સેલ્સ
\item
  \textbf{વાઇડ વ્યુઇંગ એંગલ}: કોઈ કલર ડિસ્ટોર્શન નથી
\end{itemize}

\textbf{ટેબલ: OLED વિ LCD}

{\def\LTcaptype{none} % do not increment counter
\begin{longtable}[]{@{}lll@{}}
\toprule\noalign{}
પેરામીટર & OLED & LCD \\
\midrule\noalign{}
\endhead
\bottomrule\noalign{}
\endlastfoot
\textbf{બેકલાઇટ} & જરૂરી નથી & જરૂરી \\
\textbf{કોન્ટ્રાસ્ટ} & અનંત & 1000:1 \\
\textbf{જાડાઈ} & અલ્ટ્રા-થિન & જાડું \\
\textbf{પાવર} & ઓછું (ડાર્ક ઇમેજ) & સતત \\
\end{longtable}
}

\textbf{એપ્લિકેશન્સ:}

\begin{itemize}
\tightlist
\item
  \textbf{સ્માર્ટફોન}: Samsung, iPhone ડિસ્પ્લે
\item
  \textbf{TV}: પ્રીમિયમ ટેલિવિઝન સેટ્સ
\item
  \textbf{ઓટોમોટિવ}: ડેશબોર્ડ ડિસ્પ્લે
\item
  \textbf{વેરેબલ્સ}: સ્માર્ટવોચ સ્ક્રીન
\end{itemize}

\textbf{ફાયદા:}

\begin{itemize}
\tightlist
\item
  \textbf{ઊર્જા કાર્યક્ષમ}: ઓછો પાવર વપરાશ
\item
  \textbf{લવચીક}: વળી શકાય તેવું બનાવી શકાય
\item
  \textbf{ફાસ્ટ રિસ્પોન્સ}: કોઈ મોશન બ્લર નથી
\end{itemize}

\textbf{યાદ રાખવાની ટેકનીક:} ``OLED: Organic, Light, Emitting, Display''

\end{solutionbox}
\begin{center}\rule{0.5\linewidth}{0.5pt}\end{center}

\subsection*{પ્રશ્ન 4(ક) [7
ગુણ]}\label{uxaaauxab0uxab6uxaa8-4uxa95-7-uxa97uxaa3}

\textbf{રાસ્પબેરી પાઇનું આર્કિટેક્ચર અને બ્લોક ડાયાગ્રામ સમજાવો.}

\begin{solutionbox}

\textbf{બ્લોક ડાયાગ્રામ: રાસ્પબેરી પાઇ આર્કિટેક્ચર}

\begin{verbatim}
flowchart TD
    A[ARM Cortex CPU] {-{-} B[System Bus]}
    C[GPU] {-{-} B}
    D[RAM] {-{-} B}
    E[Storage] {-{-} F[SD Card Slot]}
    F {-{-} B}
    B {-{-} G[GPIO Pins]}
    B {-{-} H[USB Ports]}
    B {-{-} I[Ethernet]}
    B {-{-} J[HDMI]}
    B {-{-} K[Audio Jack]}
    B {-{-} L[Camera Interface]}
    B {-{-} M[Display Interface]}
\end{verbatim}

\textbf{મુખ્ય ઘટકો:}

\textbf{ટેબલ: રાસ્પબેરી પાઇ ઘટકો}

{\def\LTcaptype{none} % do not increment counter
\begin{longtable}[]{@{}lll@{}}
\toprule\noalign{}
ઘટક & સ્પેસિફિકેશન & કાર્ય \\
\midrule\noalign{}
\endhead
\bottomrule\noalign{}
\endlastfoot
\textbf{CPU} & ARM Cortex-A72 Quad-core & મુખ્ય પ્રોસેસિંગ \\
\textbf{GPU} & VideoCore VI & ગ્રાફિક્સ પ્રોસેસિંગ \\
\textbf{RAM} & 4GB LPDDR4 & સિસ્ટમ મેમોરી \\
\textbf{સ્ટોરેજ} & MicroSD કાર્ડ & ઓપરેટિંગ સિસ્ટમ \\
\textbf{GPIO} & 40-પિન હેડર & હાર્ડવેર ઇન્ટરફેસ \\
\textbf{કનેક્ટિવિટી} & WiFi, Bluetooth, Ethernet & નેટવર્ક એક્સેસ \\
\end{longtable}
}

\textbf{આર્કિટેક્ચર લક્ષણો:}

\begin{itemize}
\tightlist
\item
  \textbf{SoC ડિઝાઇન}: સિસ્ટમ ઓન ચિપ ઇન્ટિગ્રેશન
\item
  \textbf{લો પાવર}: ઊર્જા-કાર્યક્ષમ ARM પ્રોસેસર
\item
  \textbf{એક્સપેન્ડેબલ}: હાર્ડવેર પ્રોજેક્ટ્સ માટે GPIO પિન્સ
\item
  \textbf{મલ્ટિમીડિયા}: વીડિયો માટે હાર્ડવેર એક્સેલેરેશન
\end{itemize}

\textbf{ઇન્ટરફેસ:}

\begin{itemize}
\tightlist
\item
  \textbf{વીડિયો}: 4K સુધી HDMI આઉટપુટ
\item
  \textbf{ઓડિયો}: 3.5mm જેક અને HDMI ઓડિયો
\item
  \textbf{કેમેરા}: CSI કેમેરા કનેક્ટર
\item
  \textbf{ડિસ્પ્લે}: DSI ડિસ્પ્લે કનેક્ટર
\end{itemize}

\textbf{એપ્લિકેશન્સ:}

\begin{itemize}
\tightlist
\item
  \textbf{શિક્ષણ}: પ્રોગ્રામિંગ અને ઇલેક્ટ્રોનિક્સ શીખવું
\item
  \textbf{IoT પ્રોજેક્ટ્સ}: હોમ ઓટોમેશન, સેન્સર્સ
\item
  \textbf{મીડિયા સેન્ટર}: હોમ એન્ટરટેઇનમેન્ટ સિસ્ટમ
\item
  \textbf{રોબોટિક્સ}: રોબોટ્સ માટે કંટ્રોલ સિસ્ટમ્સ
\end{itemize}

\textbf{યાદ રાખવાની ટેકનીક:} ``Pi: Processor, Interfaces, Projects,
Internet''

\end{solutionbox}
\begin{center}\rule{0.5\linewidth}{0.5pt}\end{center}

\subsection*{પ્રશ્ન 4(અ OR) [3
ગુણ]}\label{uxaaauxab0uxab6uxaa8-4uxa85-or-3-uxa97uxaa3}

\textbf{રાસ્પબેરી પાઇ શું છે અને તેના ફાયદા અને ગેરફાયદા શું છે?}

\begin{solutionbox}
રાસ્પબેરી પાઇ એ નાનું, સસ્તું સિંગલ-બોર્ડ કમ્પ્યુટર છે જે શિક્ષણ અને
શોખીન પ્રોજેક્ટ્સ માટે ડિઝાઇન કરવામાં આવ્યું છે.

\textbf{ટેબલ: ફાયદા અને ગેરફાયદા}

{\def\LTcaptype{none} % do not increment counter
\begin{longtable}[]{@{}ll@{}}
\toprule\noalign{}
ફાયદા & ગેરફાયદા \\
\midrule\noalign{}
\endhead
\bottomrule\noalign{}
\endlastfoot
\textbf{ઓછી કિંમત} & \textbf{મર્યાદિત કાર્યક્ષમતા} \\
\textbf{નાનું સાઇઝ} & \textbf{બિલ્ટ-ઇન સ્ટોરેજ નથી} \\
\textbf{GPIO પિન્સ} & \textbf{SD કાર્ડની જરૂર} \\
\textbf{Linux સપોર્ટ} & \textbf{રીઅલ-ટાઇમ OS નથી} \\
\textbf{શૈક્ષણિક} & \textbf{પાવર સપ્લાય સમસ્યાઓ} \\
\textbf{કમ્યુનિટી સપોર્ટ} & \textbf{મર્યાદિત RAM} \\
\end{longtable}
}

\textbf{મુખ્ય લક્ષણો:}

\begin{itemize}
\tightlist
\item
  \textbf{સસ્તું}: ખર્ચ-અસરકારк કમ્પ્યુટિંગ સોલ્યુશન
\item
  \textbf{વર્સેટાઇલ}: બહુવિધ પ્રોગ્રામિંગ ભાષાઓ સપોર્ટેડ
\item
  \textbf{ઓપન સોર્સ}: મફત સોફ્ટવેર અને ડોક્યુમેન્ટેશન
\end{itemize}

\textbf{યાદ રાખવાની ટેકનીક:} ``Pi: Cheap, Small, Educational vs Limited,
External, Power''

\end{solutionbox}
\begin{center}\rule{0.5\linewidth}{0.5pt}\end{center}

\subsection*{પ્રશ્ન 4(બ OR) [4
ગુણ]}\label{uxaaauxab0uxab6uxaa8-4uxaac-or-4-uxa97uxaa3}

\textbf{OFET પર ટૂંકી નોંધ લખો.}

\begin{solutionbox}
OFET (Organic Field Effect Transistor) એ કાર્બનિક
સેમિકંડક્ટિંગ સામગ્રીનો ઉપયોગ કરીને સ્વિચિંગ અને એમ્પ્લિફિકેશન માટેનો ટ્રાન્ઝિસ્ટર છે.

\textbf{મુખ્ય લક્ષણો:}

\begin{itemize}
\tightlist
\item
  \textbf{ઓર્ગેનિક મટીરિયલ્સ}: કાર્બન આધારિત સેમિકંડક્ટર્સ
\item
  \textbf{લો ટેમ્પરેચર}: સોલ્યુશન આધારિત પ્રોસેસિંગ
\item
  \textbf{ફ્લેક્સિબલ}: પ્લાસ્ટિક સબસ્ટ્રેટ પર બનાવી શકાય
\item
  \textbf{લાર્જ એરિયા}: મોટા ડિસ્પ્લે માટે યોગ્ય
\end{itemize}

\textbf{ટેબલ: OFET સ્ટ્રક્ચર}

{\def\LTcaptype{none} % do not increment counter
\begin{longtable}[]{@{}lll@{}}
\toprule\noalign{}
ઘટક & સામગ્રી & કાર્ય \\
\midrule\noalign{}
\endhead
\bottomrule\noalign{}
\endlastfoot
\textbf{ગેટ} & મેટલ ઇલેક્ટ્રોડ & કરંટ ફ્લો કંટ્રોલ કરે છે \\
\textbf{ડાઇઇલેક્ટ્રિક} & ઇન્સ્યુલેટિંગ લેયર & ગેટને ચેનલથી અલગ કરે છે \\
\textbf{સોર્સ/ડ્રેઇન} & મેટલ કોન્ટેક્ટ્સ & કરંટ ઇન્જેક્શન/કલેક્શન \\
\textbf{ચેનલ} & ઓર્ગેનિક સેમિકંડક્ટર & કરંટ કંડક્શન પાથ \\
\end{longtable}
}

\textbf{એપ્લિકેશન્સ:}

\begin{itemize}
\tightlist
\item
  \textbf{ફ્લેક્સિબલ ડિસ્પ્લે}: વળી શકાય તેવી સ્ક્રીન્સ
\item
  \textbf{સ્માર્ટ કાર્ડ્સ}: RFID એપ્લિકેશન્સ
\item
  \textbf{સેન્સર્સ}: કેમિકલ અને બાયોલોજિકલ ડિટેક્શન
\item
  \textbf{લોજિક સર્કિટ્સ}: સિમ્પલ ડિજિટલ સર્કિટ્સ
\end{itemize}

\textbf{ફાયદા:}

\begin{itemize}
\tightlist
\item
  \textbf{મેકેનિકલ ફ્લેક્સિબિલિટી}: વળી શકાય તેવી ઇલેક્ટ્રોનિક્સ
\item
  \textbf{લો કોસ્ટ}: સસ્તું ઉત્પાદન
\item
  \textbf{રૂમ ટેમ્પરેચર}: ઊંચા તાપમાનની પ્રોસેસિંગ નથી
\end{itemize}

\textbf{મર્યાદાઓ:}

\begin{itemize}
\tightlist
\item
  \textbf{લોઅર મોબિલિટી}: સિલિકોન કરતાં ધીમું
\item
  \textbf{સ્ટેબિલિટી ઇશ્યુઝ}: સમય સાથે ક્ષીણતા
\item
  \textbf{મર્યાદિત કાર્યક્ષમતા}: ઓછી સ્વિચિંગ સ્પીડ્સ
\end{itemize}

\textbf{યાદ રાખવાની ટેકનીક:} ``OFET: Organic, Flexible, Easy,
Transistor''

\end{solutionbox}
\begin{center}\rule{0.5\linewidth}{0.5pt}\end{center}

\subsection*{પ્રશ્ન 4(ક OR) [7
ગુણ]}\label{uxaaauxab0uxab6uxaa8-4uxa95-or-7-uxa97uxaa3}

\textbf{રાસ્પબેરી પાઇ પોર્ટ્સના પ્રકારોની સૂચિ બનાવો. રાસ્પબેરી પાઇની વિવિધ
ઓપરેટિંગ સિસ્ટમ્સની ચર્ચા કરો.}

\begin{solutionbox}

\textbf{ટેબલ: રાસ્પબેરી પાઇ પોર્ટ્સ}

{\def\LTcaptype{none} % do not increment counter
\begin{longtable}[]{@{}lll@{}}
\toprule\noalign{}
પોર્ટ પ્રકાર & સંખ્યા & કાર્ય \\
\midrule\noalign{}
\endhead
\bottomrule\noalign{}
\endlastfoot
\textbf{USB} & 4 પોર્ટ્સ & પેરિફેરલ્સ કનેક્ટ કરવા \\
\textbf{HDMI} & 2 માઇક્રો HDMI & વીડિયો આઉટપુટ \\
\textbf{GPIO} & 40 પિન્સ & હાર્ડવેર ઇન્ટરફેસ \\
\textbf{Ethernet} & 1 પોર્ટ & વાયર્ડ નેટવર્ક \\
\textbf{ઓડિયો} & 3.5mm જેક & ઓડિયો આઉટપુટ \\
\textbf{પાવર} & USB-C & પાવર ઇનપુટ \\
\textbf{કેમેરા} & CSI કનેક્ટર & કેમેરા મોડ્યુલ \\
\textbf{ડિસ્પ્લે} & DSI કનેક્ટર & ડિસ્પ્લે પેનલ \\
\end{longtable}
}

\textbf{રાસ્પબેરી પાઇ માટે ઓપરેટિંગ સિસ્ટમ્સ:}

\textbf{ટેબલ: રાસ્પબેરી પાઇ ઓપરેટિંગ સિસ્ટમ્સ}

{\def\LTcaptype{none} % do not increment counter
\begin{longtable}[]{@{}lll@{}}
\toprule\noalign{}
OS & પ્રકાર & શ્રેષ્ઠ માટે \\
\midrule\noalign{}
\endhead
\bottomrule\noalign{}
\endlastfoot
\textbf{Raspberry Pi OS} & Debian આધારિત & સામાન્ય ઉપયોગ, શરૂઆતીઓ \\
\textbf{Ubuntu} & Linux વિતરણ & સર્વર એપ્લિકેશન્સ \\
\textbf{LibreELEC} & મીડિયા સેન્ટર & હોમ એન્ટરટેઇનમેન્ટ \\
\textbf{RetroPie} & ગેમિંગ & રેટ્રો ગેમિંગ કન્સોલ \\
\textbf{Windows 10 IoT} & Microsoft OS & IoT ડેવેલપમેન્ટ \\
\textbf{OSMC} & મીડિયા સેન્ટર & મીડિયા સ્ટ્રીમિંગ \\
\end{longtable}
}

\textbf{Raspberry Pi OS ના મુખ્ય લક્ષણો:}

\begin{itemize}
\tightlist
\item
  \textbf{પ્રી-ઇન્સ્ટોલ્ડ સોફ્ટવેર}: પ્રોગ્રામિંગ ટૂલ્સ, ઓફિસ સ્યુટ
\item
  \textbf{GPIO સપોર્ટ}: હાર્ડવેર ઇન્ટરફેસિંગ લાઇબ્રેરીઓ
\item
  \textbf{શૈક્ષણિક}: Scratch, Python, Minecraft Pi
\item
  \textbf{લાઇટવેઇટ}: ARM પ્રોસેસર્સ માટે ઓપ્ટિમાઇઝ્ડ
\end{itemize}

\textbf{ઇન્સ્ટોલેશન પદ્ધતિઓ:}

\begin{itemize}
\tightlist
\item
  \textbf{NOOBS}: શરૂઆતી-મૈત્રીપૂર્ણ ઇન્સ્ટોલર
\item
  \textbf{Raspberry Pi Imager}: ઓફિશિયલ ઇમેજિંગ ટૂલ
\item
  \textbf{ડાયરેક્ટ ફ્લેશ}: એડવાન્સ્ડ યુઝર્સ
\end{itemize}

\textbf{ફાયદા:}

\begin{itemize}
\tightlist
\item
  \textbf{વેરાઇટી}: વિવિધ હેતુઓ માટે બહુવિધ OS વિકલ્પો
\item
  \textbf{કમ્યુનિટી}: મોટો યુઝર બેઝ અને સપોર્ટ
\item
  \textbf{અપડેટ્સ}: નિયમિત સિક્યુરિટી અને ફીચર અપડેટ્સ
\item
  \textbf{કસ્ટમાઇઝેશન}: ઓપન સોર્સ લવચીકતા
\end{itemize}

\textbf{યાદ રાખવાની ટેકનીક:} ``Pi Ports: USB, HDMI, GPIO, Ethernet'' /
``Pi OS: Official, Ubuntu, Media, Gaming''

\end{solutionbox}
\begin{center}\rule{0.5\linewidth}{0.5pt}\end{center}

\subsection*{પ્રશ્ન 5(અ) [3
ગુણ]}\label{uxaaauxab0uxab6uxaa8-5uxa85-3-uxa97uxaa3}

\textbf{મશીન લર્નિંગ માટે NumPy python library સમજાવો.}

\begin{solutionbox}
NumPy (Numerical Python) એ વૈજ્ઞાનિક કમ્પ્યુટિંગ માટેની મૂળભૂત
લાઇબ્રેરી છે, જે મોટા મલ્ટિ-ડાઇમેન્શનલ એરેઝ અને ગાણિતિક ફંક્શન્સ માટે સપોર્ટ પ્રદાન કરે
છે.

\textbf{મુખ્ય લક્ષણો:}

\begin{itemize}
\tightlist
\item
  \textbf{N-dimensional Arrays}: કાર્યક્ષમ એરે ઓપરેશન્સ
\item
  \textbf{ગાણિતિક ફંક્શન્સ}: લિનિયર અલજેબ્રા, ફોરિયર ટ્રાન્સફોર્મ
\item
  \textbf{બ્રોડકાસ્ટિંગ}: વિવિધ આકારના એરે પર ઓપરેશન્સ
\item
  \textbf{મેમોરી એફિશિયન્ટ}: Python lists કરતાં ઝડપી
\end{itemize}

\textbf{ટેબલ: મશીન લર્નિંગમાં NumPy}

{\def\LTcaptype{none} % do not increment counter
\begin{longtable}[]{@{}lll@{}}
\toprule\noalign{}
ફંક્શન & ઉપયોગ & ઉદાહરણ \\
\midrule\noalign{}
\endhead
\bottomrule\noalign{}
\endlastfoot
\textbf{એરેઝ} & ડેટા સ્ટોરેજ & np.array([1,2,3]) \\
\textbf{લિનિયર અલજેબ્રા} & મેટ્રિક્સ ઓપરેશન્સ & np.dot(a,b) \\
\textbf{સ્ટેટિસ્ટિક્સ} & ડેટા એનાલિસિસ & np.mean(), np.std() \\
\textbf{રેન્ડમ} & ડેટા જનરેશન & np.random.rand() \\
\end{longtable}
}

\textbf{ML માં એપ્લિકેશન્સ:}

\begin{itemize}
\tightlist
\item
  \textbf{ડેટા પ્રીપ્રોસેસિંગ}: એરે મેનિપ્યુલેશન અને ક્લીનિંગ
\item
  \textbf{ફીચર એન્જિનિયરિંગ}: ગાણિતિક રૂપાંતરણો
\item
  \textbf{મોડલ ઇમ્પ્લિમેન્ટેશન}: એલ્ગોરિધમ માટે મેટ્રિક્સ ઓપરેશન્સ
\end{itemize}

\textbf{યાદ રાખવાની ટેકનીક:} ``NumPy: Numbers, Python, Arrays, Math''

\end{solutionbox}
\begin{center}\rule{0.5\linewidth}{0.5pt}\end{center}

\subsection*{પ્રશ્ન 5(બ) [4
ગુણ]}\label{uxaaauxab0uxab6uxaa8-5uxaac-4-uxa97uxaa3}

\textbf{ઓર્ગેનિક ફોટોવોલ્ટેઇક સેલ (OPV) શું છે? તેના કાર્ય સિદ્ધાંતને સમજાવો.}

\begin{solutionbox}
OPV (Organic Photovoltaic) સેલ એ કાર્બનિક સેમિકંડક્ટર્સનો ઉપયોગ
કરીને પ્રકાશને વીજળીમાં રૂપાંતરિત કરતા સોલાર સેલ છે.

\textbf{કાર્યસિદ્ધાંત:}

\begin{verbatim}
flowchart LR
    A[Sunlight] {-{-} B[Organic Active Layer]}
    B {-{-} C[Exciton Generation]}
    C {-{-} D[Charge Separation]}
    D {-{-} E[Electron Transport]}
    E {-{-} F[Current Collection]}
\end{verbatim}

\textbf{મુખ્ય પગલાં:}

\begin{itemize}
\tightlist
\item
  \textbf{પ્રકાશ શોષણ}: કાર્બનિક મોલેક્યુલ્સ ફોટોન્સ શોષે છે
\item
  \textbf{એક્સિટન ફોર્મેશન}: બાઉન્ડ ઇલેક્ટ્રોન-હોલ પેર્સ બને છે
\item
  \textbf{ચાર્જ સેપરેશન}: ડોનર-એક્સેપ્ટર ઇન્ટરફેસ પર એક્સિટન્સ વિભાજિત થાય છે
\item
  \textbf{ચાર્જ ટ્રાન્સપોર્ટ}: ઇલેક્ટ્રોન્સ અને હોલ્સ ઇલેક્ટ્રોડ્સ તરફ જાય છે
\item
  \textbf{કરંટ કલેક્શન}: બાહ્ય સર્કિટ પ્રવાહ પૂર્ણ કરે છે
\end{itemize}

\textbf{ટેબલ: OPV સ્ટ્રક્ચર}

{\def\LTcaptype{none} % do not increment counter
\begin{longtable}[]{@{}lll@{}}
\toprule\noalign{}
લેયર & સામગ્રી & કાર્ય \\
\midrule\noalign{}
\endhead
\bottomrule\noalign{}
\endlastfoot
\textbf{એનોડ} & ITO & પારદર્શક ઇલેક્ટ્રોડ \\
\textbf{એક્ટિવ લેયર} & ઓર્ગેનિક બ્લેન્ડ & પ્રકાશ શોષણ \\
\textbf{કેથોડ} & એલ્યુમિનિયમ & બેક ઇલેક્ટ્રોડ \\
\textbf{બફર લેયર્સ} & PEDOT:PSS & કાર્યક્ષમતા સુધારે છે \\
\end{longtable}
}

\textbf{ફાયદા:}

\begin{itemize}
\tightlist
\item
  \textbf{લવચીક}: પ્લાસ્ટિક પર બનાવી શકાય
\item
  \textbf{હળવા}: પોર્ટેબલ એપ્લિકેશન્સ
\item
  \textbf{ઓછી કિંમત}: સોલ્યુશન પ્રોસેસિંગ
\item
  \textbf{પારદર્શક}: સી-થ્રુ પેનલ્સ
\end{itemize}

\textbf{મર્યાદાઓ:}

\begin{itemize}
\tightlist
\item
  \textbf{ઓછી કાર્યક્ષમતા}: 10-15\% વિ 20\%+ સિલિકોન
\item
  \textbf{સ્ટેબિલિટી}: ડિગ્રેડેશન ઇશ્યુઝ
\item
  \textbf{લાઇફટાઇમ}: અકાર્બનિક સેલ્સ કરતાં ઓછું
\end{itemize}

\textbf{યાદ રાખવાની ટેકનીક:} ``OPV: Organic, Photons, Voltage, Excitons''

\end{solutionbox}
\begin{center}\rule{0.5\linewidth}{0.5pt}\end{center}

\subsection*{પ્રશ્ન 5(ક) [7
ગુણ]}\label{uxaaauxab0uxab6uxaa8-5uxa95-7-uxa97uxaa3}

\textbf{કોઈપણ ચાર મશીન લર્નિંગ ટૂલ્સની યાદી બનાવો. કોઈપણ એકની સંક્ષિપ્તમાં ચર્ચા
કરો.}

\begin{solutionbox}

\textbf{ટેબલ: મશીન લર્નિંગ ટૂલ્સ}

{\def\LTcaptype{none} % do not increment counter
\begin{longtable}[]{@{}lll@{}}
\toprule\noalign{}
ટૂલ & પ્રકાર & શ્રેષ્ઠ માટે \\
\midrule\noalign{}
\endhead
\bottomrule\noalign{}
\endlastfoot
\textbf{TensorFlow} & ડીપ લર્નિંગ ફ્રેમવર્ક & ન્યુરલ નેટવર્ક્સ \\
\textbf{Scikit-learn} & જનરલ ML લાઇબ્રેરી & પરંપરાગત એલ્ગોરિધમ \\
\textbf{PyTorch} & ડીપ લર્નિંગ ફ્રેમવર્ક & સંશોધન અને વિકાસ \\
\textbf{Keras} & હાઇ-લેવલ API & ઝડપી પ્રોટોટાઇપિંગ \\
\end{longtable}
}

\textbf{વિગતવાર ચર્ચા: TensorFlow}

TensorFlow એ Google દ્વારા વિકસિત ML મોડેલ્સ બનાવવા અને તૈનાત કરવા માટેનું
ઓપન-સોર્સ મશીન લર્નિંગ ફ્રેમવર્ક છે.

\textbf{મુખ્ય લક્ષણો:}

\textbf{ટેબલ: TensorFlow ઘટકો}

{\def\LTcaptype{none} % do not increment counter
\begin{longtable}[]{@{}lll@{}}
\toprule\noalign{}
ઘટક & કાર્ય & ફાયદો \\
\midrule\noalign{}
\endhead
\bottomrule\noalign{}
\endlastfoot
\textbf{ટેન્સર્સ} & મલ્ટિ-ડાઇમેન્શનલ એરેઝ & ડેટા રિપ્રેઝન્ટેશન \\
\textbf{ગ્રાફ્સ} & કોમ્પ્યુટેશનલ ફ્લો & મોડલ વિઝ્યુઅલાઇઝેશન \\
\textbf{સેશન્સ} & એક્ઝિક્યુશન એન્વાયરનમેન્ટ & રિસોર્સ મેનેજમેન્ટ \\
\textbf{એસ્ટિમેટર્સ} & હાઇ-લેવલ APIs & સરળ મોડલ બિલ્ડિંગ \\
\end{longtable}
}

\textbf{આર્કિટેક્ચર:}

\begin{itemize}
\tightlist
\item
  \textbf{ફ્રન્ટએન્ડ}: Python, C++, Java APIs
\item
  \textbf{બેકએન્ડ}: CPU, GPU, TPU સપોર્ટ
\item
  \textbf{ડિસ્ટ્રિબ્યુટેડ}: મલ્ટિ-ડિવાઇસ ટ્રેનિંગ
\item
  \textbf{પ્રોડક્શન}: મોડલ સર્વિંગ અને ડિપ્લોયમેન્ટ
\end{itemize}

\textbf{એપ્લિકેશન્સ:}

\begin{itemize}
\tightlist
\item
  \textbf{ઇમેજ રેકગ્નિશન}: કમ્પ્યુટર વિઝન ટાસ્ક
\item
  \textbf{નેચરલ લેંગ્વેજ}: ટેક્સ્ટ પ્રોસેસિંગ અને ટ્રાન્સલેશન
\item
  \textbf{રેકમેન્ડેશન સિસ્ટમ્સ}: વ્યક્તિગત કન્ટેન્ટ
\item
  \textbf{ટાઇમ સિરીઝ}: ફોરકાસ્ટિંગ અને પ્રિડિક્શન
\end{itemize}

\textbf{ફાયદા:}

\begin{itemize}
\tightlist
\item
  \textbf{સ્કેલેબિલિટી}: મોબાઇલથી ડેટા સેન્ટર સુધી
\item
  \textbf{ફ્લેક્સિબિલિટી}: સંશોધનથી પ્રોડક્શન સુધી
\item
  \textbf{કમ્યુનિટી}: મોટું ઇકોસિસ્ટમ અને સપોર્ટ
\item
  \textbf{વિઝ્યુઅલાઇઝેશન}: મોનિટરિંગ માટે TensorBoard
\end{itemize}

\textbf{કોડ ઉદાહરણ:}

\begin{verbatim}
import tensorflow as tf
model = tf.keras.Sequential([
    tf.keras.layers.Dense(128, activation={relu}),
    tf.keras.layers.Dense(10, activation={softmax})
])
\end{verbatim}

\textbf{ઉદ્યોગમાં ઉપયોગ:}

\begin{itemize}
\tightlist
\item
  \textbf{Google}: સર્ચ અને એડ્સ ઓપ્ટિમાઇઝેશન
\item
  \textbf{હેલ્થકેર}: મેડિકલ ઇમેજ એનાલિસિસ
\item
  \textbf{ફાઇનાન્સ}: ફ્રોડ ડિટેક્શન સિસ્ટમ્સ
\item
  \textbf{ઓટોમોટિવ}: ઓટોનોમસ વહિકલ ડેવેલપમેન્ટ
\end{itemize}

\textbf{યાદ રાખવાની ટેકનીક:} ``TensorFlow: Tensors, Graphs, Scale,
Deploy''

\end{solutionbox}
\begin{center}\rule{0.5\linewidth}{0.5pt}\end{center}

\subsection*{પ્રશ્ન 5(અ OR) [3
ગુણ]}\label{uxaaauxab0uxab6uxaa8-5uxa85-or-3-uxa97uxaa3}

\textbf{મશીન લર્નિંગ માટે પાન્ડા python library સમજાવો.}

\begin{solutionbox}
Pandas એ ડેટા મેનિપ્યુલેશન અને એનાલિસિસ માટેની Python લાઇબ્રેરી છે,
જે સ્ટ્રક્ચર્ડ ડેટા હેન્ડલ કરવા માટે ડેટા સ્ટ્રક્ચર્ અને ટૂલ્સ પ્રદાન કરે છે.

\textbf{મુખ્ય લક્ષણો:}

\begin{itemize}
\tightlist
\item
  \textbf{DataFrame}: 2D લેબલ્ડ ડેટા સ્ટ્રક્ચર
\item
  \textbf{Series}: 1D લેબલ્ડ એરે
\item
  \textbf{ડેટા ક્લીનિંગ}: મિસિંગ વેલ્યુઝ, ડુપ્લિકેટ્સ હેન્ડલ કરવું
\item
  \textbf{ફાઇલ I/O}: CSV, Excel, JSON, SQL રીડ/રાઇટ
\end{itemize}

\textbf{ટેબલ: મશીન લર્નિંગમાં Pandas}

{\def\LTcaptype{none} % do not increment counter
\begin{longtable}[]{@{}lll@{}}
\toprule\noalign{}
ફંક્શન & ઉપયોગ & ઉદાહરણ \\
\midrule\noalign{}
\endhead
\bottomrule\noalign{}
\endlastfoot
\textbf{ડેટા લોડિંગ} & ડેટાસેટ્સ ઇમ્પોર્ટ & pd.read\_csv() \\
\textbf{ડેટા ક્લીનિંગ} & મિસિંગ રિમૂવ/ફિલ & df.dropna() \\
\textbf{ડેટા સિલેક્શન} & ડેટા ફિલ્ટર & df[df[`col'] \textgreater{}
5] \\
\textbf{એગ્રીગેશન} & ગ્રુપ અને સમરાઇઝ & df.groupby().mean() \\
\end{longtable}
}

\textbf{ML માં એપ્લિકેશન્સ:}

\begin{itemize}
\tightlist
\item
  \textbf{ડેટા પ્રીપ્રોસેસિંગ}: ડેટાસેટ્સ ક્લીન અને તૈયાર કરવું
\item
  \textbf{ફીચર એન્જિનિયરિંગ}: અસ્તિત્વમાંના ડેટામાંથી નવા ફીચર્સ બનાવવા
\item
  \textbf{એક્સપ્લોરેટરી એનાલિસિસ}: ડેટા પેટર્ન અને સંબંધો સમજવા
\end{itemize}

\textbf{યાદ રાખવાની ટેકનીક:} ``Pandas: Python, Analysis, Data,
Structure''

\end{solutionbox}
\begin{center}\rule{0.5\linewidth}{0.5pt}\end{center}

\subsection*{પ્રશ્ન 5(બ OR) [4
ગુણ]}\label{uxaaauxab0uxab6uxaa8-5uxaac-or-4-uxa97uxaa3}

\textbf{ઓગમેન્ટેડ રિયાલિટી અને વર્ચ્યુઅલ રિયાલિટી વચ્ચેનો તફાવત સમજાવો.}

\begin{solutionbox}

\textbf{ટેબલ: AR વિ VR સરખામણી}

{\def\LTcaptype{none} % do not increment counter
\begin{longtable}[]{@{}lll@{}}
\toprule\noalign{}
પેરામીટર & ઓગમેન્ટેડ રિયાલિટી (AR) & વર્ચ્યુઅલ રિયાલિટી (VR) \\
\midrule\noalign{}
\endhead
\bottomrule\noalign{}
\endlastfoot
\textbf{પર્યાવરણ} & વાસ્તવિક વિશ્વ + ડિજિટલ ઓવરલે & સંપૂર્ણપણે વર્ચ્યુઅલ વિશ્વ \\
\textbf{હાર્ડવેર} & સ્માર્ટફોન, AR ગ્લાસીસ & VR હેડસેટ, કંટ્રોલર્સ \\
\textbf{ઇમર્શન} & આંશિક ઇમર્શન & સંપૂર્ણ ઇમર્શન \\
\textbf{ઇન્ટરેક્શન} & વાસ્તવિક વિશ્વ + ડિજિટલ ઓબ્જેક્ટ્સ & માત્ર વર્ચ્યુઅલ ઓબ્જેક્ટ્સ \\
\textbf{કિંમત} & ઓછી કિંમત & ઊંચી કિંમત \\
\textbf{મોબિલિટી} & મોબાઇલ અને પોર્ટેબલ & સ્ટેશનરી સેટઅપ \\
\end{longtable}
}

\textbf{મુખ્ય તફાવતો:}

\begin{itemize}
\tightlist
\item
  \textbf{રિયાલિટી મિક્સ}: AR વાસ્તવિક અને વર્ચ્યુઅલ મિશ્રણ કરે છે, VR વાસ્તવિકતા
  બદલે છે
\item
  \textbf{યુઝર એક્સપિરિયન્સ}: AR વાસ્તવિકતા વધારે છે, VR નવી વાસ્તવિકતા બનાવે છે
\item
  \textbf{એપ્લિકેશન્સ}: AR નેવિગેશન, શોપિંગ માટે; VR ગેમિંગ, ટ્રેનિંગ માટે
\item
  \textbf{હાર્ડવેર આવશ્યકતાઓ}: AR ઓછા શક્તિશાળી હાર્ડવેરની જરૂર
\end{itemize}

\textbf{ઉદાહરણો:}

\begin{itemize}
\tightlist
\item
  \textbf{AR}: Pokemon Go, Snapchat ફિલ્ટર્સ, Google Maps નેવિગેશન
\item
  \textbf{VR}: Oculus ગેમ્સ, વર્ચ્યુઅલ ટૂર્સ, ફ્લાઇટ સિમ્યુલેટર્સ
\end{itemize}

\textbf{ઉપયોગ કેસેસ:}

\begin{itemize}
\tightlist
\item
  \textbf{AR}: રિટેલ, શિક્ષણ, મેઇન્ટેનન્સ, માર્કેટિંગ
\item
  \textbf{VR}: એન્ટરટેઇનમેન્ટ, ટ્રેનિંગ, થેરાપી, ડિઝાઇન
\end{itemize}

\textbf{યાદ રાખવાની ટેકનીક:} ``AR: Augments Reality vs VR: Virtual
Reality''

\end{solutionbox}
\begin{center}\rule{0.5\linewidth}{0.5pt}\end{center}

\subsection*{પ્રશ્ન 5(ક OR) [7
ગુણ]}\label{uxaaauxab0uxab6uxaa8-5uxa95-or-7-uxa97uxaa3}

\textbf{મશીન લર્નિંગ શું છે? મશીન લર્નિંગના વિવિધ પ્રકારોની ચર્ચા કરો.}

\begin{solutionbox}
મશીન લર્નિંગ એ આર્ટિફિશિયલ ઇન્ટેલિજન્સનો ઉપવિભાગ છે જે કમ્પ્યુટર્સને
સ્પષ્ટ રીતે પ્રોગ્રામ કર્યા વિના ડેટામાંથી શીખવા અને નિર્ણયો લેવા સક્ષમ બનાવે છે.

\textbf{વ્યાખ્યા:} મશીન લર્નિંગ ડેટાનું વિશ્લેષણ કરવા, પેટર્ન ઓળખવા અને શીખેલા પેટર્ન
આધારે અનુમાન અથવા નિર્ણયો લેવા માટે એલ્ગોરિધમનો ઉપયોગ કરે છે.

\textbf{મશીન લર્નિંગના પ્રકારો:}

\textbf{ટેબલ: મશીન લર્નિંગના પ્રકારો}

{\def\LTcaptype{none} % do not increment counter
\begin{longtable}[]{@{}
  >{\raggedright\arraybackslash}p{(\linewidth - 6\tabcolsep) * \real{0.1500}}
  >{\raggedright\arraybackslash}p{(\linewidth - 6\tabcolsep) * \real{0.3250}}
  >{\raggedright\arraybackslash}p{(\linewidth - 6\tabcolsep) * \real{0.2500}}
  >{\raggedright\arraybackslash}p{(\linewidth - 6\tabcolsep) * \real{0.2750}}@{}}
\toprule\noalign{}
\begin{minipage}[b]{\linewidth}\raggedright
પ્રકાર
\end{minipage} & \begin{minipage}[b]{\linewidth}\raggedright
વર્ણન
\end{minipage} & \begin{minipage}[b]{\linewidth}\raggedright
ઉદાહરણો
\end{minipage} & \begin{minipage}[b]{\linewidth}\raggedright
ઉપયોગ કેસેસ
\end{minipage} \\
\midrule\noalign{}
\endhead
\bottomrule\noalign{}
\endlastfoot
\textbf{સુપરવાઇઝ્ડ} & લેબલ્ડ ડેટામાંથી શીખે છે & ક્લાસિફિકેશન, રિગ્રેશન & ઇમેઇલ સ્પામ,
કિંમત પૂર્વાનુમાન \\
\textbf{અનસુપરવાઇઝ્ડ} & અનલેબલ્ડ ડેટામાં પેટર્ન શોધે છે & ક્લસ્ટરિંગ, એસોસિએશન &
કસ્ટમર સેગમેન્ટેશન \\
\textbf{રિઇન્ફોર્સમેન્ટ} & ટ્રાયલ અને એરર દ્વારા શીખે છે & Q-learning, પોલિસી
ગ્રેડિએન્ટ & ગેમ પ્લેઇંગ, રોબોટિક્સ \\
\end{longtable}
}

\textbf{1. સુપરવાઇઝ્ડ લર્નિંગ:}

\begin{verbatim}
flowchart LR
    A[Training Data] {-{-} B[Algorithm]}
    B {-{-} C[Model]}
    D[New Data] {-{-} C}
    C {-{-} E[Prediction]}
\end{verbatim}

\textbf{સુપરવાઇઝ્ડ લર્નિંગના પ્રકારો:}

\begin{itemize}
\tightlist
\item
  \textbf{ક્લાસિફિકેશન}: કેટેગરીઝનું અનુમાન (સ્પામ/નોટ સ્પામ)
\item
  \textbf{રિગ્રેશન}: સતત વેલ્યુઝનું અનુમાન (ઘરની કિંમતો)
\end{itemize}

\textbf{2. અનસુપરવાઇઝ્ડ લર્નિંગ:}

\begin{itemize}
\tightlist
\item
  \textbf{ક્લસ્ટરિંગ}: સમાન ડેટા પોઇન્ટ્સને ગ્રુપ કરે છે
\item
  \textbf{એસોસિએશન}: વેરિએબલ્સ વચ્ચેના સંબંધો શોધે છે
\item
  \textbf{ડાઇમેન્શનાલિટી રિડક્શન}: ડેટા કોમ્પ્લેક્સિટી ઘટાડે છે
\end{itemize}

\textbf{3. રિઇન્ફોર્સમેન્ટ લર્નિંગ:}

\begin{itemize}
\tightlist
\item
  \textbf{એજન્ટ}: લર્નિંગ એન્ટિટી
\item
  \textbf{એન્વાયરનમેન્ટ}: લર્ન થતી સિસ્ટમ
\item
  \textbf{રિવોર્ડ}: ફીડબેક મેકેનિઝમ
\item
  \textbf{પોલિસી}: ક્રિયાઓ માટેની રણનીતિ
\end{itemize}

\textbf{પ્રકાર પ્રમાણે એપ્લિકેશન્સ:}

\textbf{ટેબલ: ML એપ્લિકેશન્સ}

{\def\LTcaptype{none} % do not increment counter
\begin{longtable}[]{@{}lll@{}}
\toprule\noalign{}
પ્રકાર & એપ્લિકેશન & ઉદ્યોગ \\
\midrule\noalign{}
\endhead
\bottomrule\noalign{}
\endlastfoot
\textbf{સુપરવાઇઝ્ડ} & મેડિકલ ડાયાગ્નોસિસ & હેલ્થકેર \\
\textbf{અનસુપરવાઇઝ્ડ} & માર્કેટ બાસ્કેટ એનાલિસિસ & રિટેલ \\
\textbf{રિઇન્ફોર્સમેન્ટ} & ઓટોનોમસ ડ્રાઇવિંગ & ઓટોમોટિવ \\
\end{longtable}
}

\textbf{મુખ્ય એલ્ગોરિધમ:}

\begin{itemize}
\tightlist
\item
  \textbf{સુપરવાઇઝ્ડ}: લિનિયર રિગ્રેશન, ડિસિઝન ટ્રીઝ, SVM, ન્યુરલ નેટવર્ક્સ
\item
  \textbf{અનસુપરવાઇઝ્ડ}: K-Means, DBSCAN, PCA, Apriori
\item
  \textbf{રિઇન્ફોર્સમેન્ટ}: Q-Learning, Actor-Critic, Deep Q-Networks
\end{itemize}

\textbf{મશીન લર્નિંગ પ્રક્રિયા:}

\begin{enumerate}
\tightlist
\item
  \textbf{ડેટા એકત્રીકરણ}: સંબંધિત ડેટાસેટ્સ એકત્રિત કરવા
\item
  \textbf{ડેટા પ્રીપ્રોસેસિંગ}: ડેટા ક્લીન અને તૈયાર કરવા
\item
  \textbf{ફીચર સિલેક્શન}: મહત્વપૂર્ણ વેરિએબલ્સ પસંદ કરવા
\item
  \textbf{મોડલ ટ્રેનિંગ}: ડેટા પર એલ્ગોરિધમ ટ્રેન કરવું
\item
  \textbf{મોડલ ઇવેલ્યુએશન}: કાર્યક્ષમતા ટેસ્ટ કરવી
\item
  \textbf{ડિપ્લોયમેન્ટ}: પ્રોડક્શનમાં અમલીકરણ
\end{enumerate}

\textbf{ફાયદા:}

\begin{itemize}
\tightlist
\item
  \textbf{ઓટોમેશન}: મેન્યુઅલ કામ ઘટાડે છે
\item
  \textbf{ચોકસાઈ}: ઘણા કાર્યોમાં માનવીય કાર્યક્ષમતા કરતાં સારું
\item
  \textbf{સ્કેલેબિલિટી}: મોટા ડેટાસેટ્સ હેન્ડલ કરે છે
\item
  \textbf{અનુકૂલનક્ષમતા}: વધુ ડેટા સાથે સુધારે છે
\end{itemize}

\textbf{પડકારો:}

\begin{itemize}
\tightlist
\item
  \textbf{ડેટા ક્વોલિટી}: સ્વચ્છ, સંબંધિત ડેટાની જરૂર
\item
  \textbf{ઓવરફિટિંગ}: મોડલ ટ્રેનિંગ ડેટા માટે ખૂબ વિશિષ્ટ
\item
  \textbf{ઇન્ટરપ્રિટેબિલિટી}: કેટલાક એલ્ગોરિધમનું બ્લેક બોક્સ સ્વભાવ
\item
  \textbf{કોમ્પ્યુટેશનલ રિસોર્સ}: નોંધપાત્ર પ્રોસેસિંગ પાવરની જરૂર
\end{itemize}

\textbf{વાસ્તવિક દુનિયાના ઉદાહરણો:}

\begin{itemize}
\tightlist
\item
  \textbf{Netflix}: મૂવી રેકમેન્ડેશન્સ (સુપરવાઇઝ્ડ)
\item
  \textbf{Amazon}: કસ્ટમર સેગમેન્ટેશન (અનસુપરવાઇઝ્ડ)
\item
  \textbf{AlphaGo}: ગેમ પ્લેઇંગ (રિઇન્ફોર્સમેન્ટ)
\end{itemize}

\textbf{ભાવિ ટ્રેન્ડ્સ:}

\begin{itemize}
\tightlist
\item
  \textbf{ડીપ લર્નિંગ}: બહુવિધ લેયર્સ સાથે ન્યુરલ નેટવર્ક્સ
\item
  \textbf{AutoML}: ઓટોમેટેડ મશીન લર્નિંગ પાઇપલાઇન્સ
\item
  \textbf{એજ AI}: મોબાઇલ અને IoT ડિવાઇસેસ પર ML
\item
  \textbf{એક્સપ્લેનેબલ AI}: ML નિર્ણયોને ઇન્ટરપ્રિટેબલ બનાવવું
\end{itemize}

\textbf{યાદ રાખવાની ટેકનીક:} ``ML Types: Supervised teaches, Unsupervised
discovers, Reinforcement rewards''

\end{solutionbox}

\end{document}
