\documentclass[10pt,a4paper]{article}

% content/resources/templates/preamble.tex
\usepackage[margin=0.6in]{geometry}
\author{Milav Dabgar}
\usepackage{amsmath,amssymb,amsthm}
\usepackage{booktabs}
\usepackage{multirow}
\usepackage{xcolor}
\usepackage{tcolorbox}
\tcbuselibrary{breakable,skins}
\usepackage[colorlinks=true,linkcolor=blue]{hyperref}
\usepackage{titlesec}
\usepackage{enumitem}
\usepackage{tikz}
\usepackage{pgfplots}
\usepackage{circuitikz}
\usepackage[version=4]{mhchem}
\usepackage{longtable}
\usepackage{array}
\usepackage{float}
\usepackage{caption}
\usepackage{listings}

\lstset{
  basicstyle=\small\ttfamily,
  breaklines=true,
  breakatwhitespace=false,
  postbreak=\mbox{\textcolor{red}{$\hookrightarrow$}\space},
  float=false,
  numbers=left,
  numberstyle=\tiny\color{gray},
  numbersep=10pt,
  xleftmargin=2em,
  keywordstyle=\color{blue},
  commentstyle=\color{green!60!black},
  stringstyle=\color{purple},
  backgroundcolor=\color{gray!5},
  showstringspaces=false,
  tabsize=2,
  captionpos=b,
  keepspaces=true,
  columns=flexible
}

\pgfplotsset{compat=1.18}
\usetikzlibrary{shapes,arrows,positioning,calc,patterns,decorations.pathmorphing,decorations.markings,arrows.meta}

% Color scheme
\definecolor{headcolor}{RGB}{0,102,204}
\definecolor{keycolor}{RGB}{220,20,60}
\definecolor{solutioncolor}{RGB}{34,139,34}
\definecolor{mnemoniccolor}{RGB}{148,0,211}
\definecolor{codecolor}{RGB}{0,0,100}

% Spacing
\setlength{\parskip}{3pt}
\setlist[itemize]{nosep}
\setlist[enumerate]{nosep}

% Title formatting
\titleformat{\section}{\Large\bfseries\color{headcolor}}{\thesection}{1em}{}
\titleformat{\subsection}{\large\bfseries\color{headcolor}}{\thesubsection}{1em}{}

% Pandoc tightlist compatibility
\providecommand{\tightlist}{%
  \setlength{\itemsep}{0pt}\setlength{\parskip}{0pt}}

% Pandoc longtable compatibility
\newcounter{none}
\def\thenone{}


% content/resources/templates/english-boxes.tex
% This file is currently empty - it exists to maintain consistency with the import structure.
% Add custom environments here if needed in the future.


\begin{document}

\begin{center}
{\Huge\bfseries\color{headcolor} Subject Name Solutions}\\[5pt]
{\LARGE 4361106 -- Summer 2025}\\[3pt]
{\large Semester 1 Study Material}\\[3pt]
{\normalsize\textit{Detailed Solutions and Explanations}}
\end{center}

\vspace{10pt}

\subsection*{Question 1(a) [3 marks]}\label{q1a}

\textbf{Define Renewable Energy and explain its importance.}

\begin{solutionbox}

\textbf{Renewable Energy} is energy derived from natural sources that
are continuously replenished, such as solar, wind, hydroelectric,
biomass, and geothermal energy.


{\def\LTcaptype{none} % do not increment counter
\vspace{-5pt}
\captionof{table}{Types of Renewable Energy Sources}
\vspace{-10pt}
\begin{longtable}[]{@{}lll@{}}
\toprule\noalign{}
Type & Source & Advantage \\
\midrule\noalign{}
\endhead
\bottomrule\noalign{}
\endlastfoot
\textbf{Solar} & Sun's radiation & Clean, abundant \\
\textbf{Wind} & Air movement & No emissions \\
\textbf{Hydro} & Water flow & Reliable power \\
\textbf{Biomass} & Organic matter & Carbon neutral \\
\end{longtable}
}

\textbf{Importance:}

\begin{itemize}
\tightlist
\item
  \textbf{Environmental protection}: Reduces pollution and greenhouse
  gases
\item
  \textbf{Energy security}: Reduces dependence on fossil fuels
\item
  \textbf{Economic benefits}: Creates jobs and reduces energy costs
\end{itemize}

\end{solutionbox}
\begin{mnemonicbox}
``SEEB'' - Solar, Environmental, Economic, Biomass

\end{mnemonicbox}
\subsection*{Question 1(b) [4 marks]}\label{q1b}

\textbf{Explain Solar Photovoltaic effect \& Principle of photovoltaic
conversion.}

\begin{solutionbox}

\textbf{Photovoltaic Effect} is the generation of electric current when
light strikes a semiconductor material.

\textbf{Working Principle:}

\begin{itemize}
\tightlist
\item
  \textbf{Photon absorption}: Light photons hit solar cell surface
\item
  \textbf{Electron excitation}: Electrons gain energy and move to
  conduction band
\item
  \textbf{Charge separation}: Built-in electric field separates positive
  and negative charges
\item
  \textbf{Current generation}: Flow of electrons creates DC electricity
\end{itemize}

\textbf{Diagram:}

\begin{verbatim}
    Light Photons
         ↓
    ┌─────────────┐
    │  P{-type     │  Holes (+)}
    │─────────────│  Junction
    │  N{-type     │  Electrons ({-})}
    └─────────────┘
         ↓
    Electric Current
\end{verbatim}

\end{solutionbox}
\begin{mnemonicbox}
``PACE'' - Photons, Absorption, Charge, Electricity

\end{mnemonicbox}
\subsection*{Question 1(c) [7 marks]}\label{q1c}

\textbf{Describe the types of Electric Vehicle (EV) and different Energy
sources for EV.}

\begin{solutionbox}


{\def\LTcaptype{none} % do not increment counter
\vspace{-5pt}
\captionof{table}{Types of Electric Vehicles}
\vspace{-10pt}
\begin{longtable}[]{@{}llll@{}}
\toprule\noalign{}
EV Type & Full Form & Power Source & Range \\
\midrule\noalign{}
\endhead
\bottomrule\noalign{}
\endlastfoot
\textbf{BEV} & Battery Electric Vehicle & Battery only & 150-400 km \\
\textbf{HEV} & Hybrid Electric Vehicle & Battery + Engine & 600+ km \\
\textbf{PHEV} & Plug-in Hybrid & Battery + Engine & 50-100 km
electric \\
\textbf{FCEV} & Fuel Cell Electric & Hydrogen fuel cell & 400-600 km \\
\end{longtable}
}

\textbf{Energy Sources for EVs:}

\begin{itemize}
\tightlist
\item
  \textbf{Battery}: Lithium-ion batteries store electrical energy
\item
  \textbf{Fuel Cell}: Converts hydrogen to electricity
\item
  \textbf{Ultracapacitor}: Quick energy storage and release
\item
  \textbf{Flywheel}: Mechanical energy storage
\item
  \textbf{Regenerative Braking}: Recovers energy during braking
\item
  \textbf{Hybrid Sources}: Combination of multiple energy sources
\end{itemize}

\textbf{Diagram: EV Architecture}

\begin{verbatim}
┌────────────┐    ┌─────────────┐    ┌──────────┐
│   Battery  │────│  Controller │────│  Motor   │
└────────────┘    └─────────────┘    └──────────┘
                         │
                  ┌─────────────┐
                  │ Charging    │
                  │ System      │
                  └─────────────┘
\end{verbatim}

\end{solutionbox}
\begin{mnemonicbox}
``BHPF-BUFR'' - Battery, Hybrid, Plugin, FuelCell -
Battery, Ultracap, Flywheel, Regen

\end{mnemonicbox}
\subsection*{Question 1(c) OR [7
marks]}\label{q1c}

\textbf{Discuss different types of Renewable Energy Sources.}

\begin{solutionbox}


{\def\LTcaptype{none} % do not increment counter
\vspace{-5pt}
\captionof{table}{Renewable Energy Sources Comparison}
\vspace{-10pt}
\begin{longtable}[]{@{}
  >{\raggedright\arraybackslash}p{(\linewidth - 6\tabcolsep) * \real{0.1739}}
  >{\raggedright\arraybackslash}p{(\linewidth - 6\tabcolsep) * \real{0.2826}}
  >{\raggedright\arraybackslash}p{(\linewidth - 6\tabcolsep) * \real{0.2609}}
  >{\raggedright\arraybackslash}p{(\linewidth - 6\tabcolsep) * \real{0.2826}}@{}}
\toprule\noalign{}
\begin{minipage}[b]{\linewidth}\raggedright
Source
\end{minipage} & \begin{minipage}[b]{\linewidth}\raggedright
How it Works
\end{minipage} & \begin{minipage}[b]{\linewidth}\raggedright
Advantages
\end{minipage} & \begin{minipage}[b]{\linewidth}\raggedright
Applications
\end{minipage} \\
\midrule\noalign{}
\endhead
\bottomrule\noalign{}
\endlastfoot
\textbf{Solar} & Converts sunlight to electricity & Clean, abundant &
Rooftop systems, farms \\
\textbf{Wind} & Wind turns turbines & No fuel cost & Wind farms,
offshore \\
\textbf{Hydroelectric} & Water flow generates power & Reliable,
long-lasting & Dams, rivers \\
\textbf{Biomass} & Organic matter combustion & Carbon neutral & Power
plants, heating \\
\textbf{Geothermal} & Earth's heat energy & Constant availability &
Heating, electricity \\
\end{longtable}
}

\textbf{Emerging Trends:}

\begin{itemize}
\tightlist
\item
  \textbf{Tidal Wave}: Ocean wave energy conversion
\item
  \textbf{Solar Thermal}: Concentrated solar power systems
\item
  \textbf{Hydrogen}: Clean fuel from renewable sources
\end{itemize}

\textbf{Benefits:}

\begin{itemize}
\tightlist
\item
  \textbf{Sustainability}: Never depletes
\item
  \textbf{Environmental}: Minimal pollution
\item
  \textbf{Economic}: Reduces energy costs long-term
\end{itemize}

\end{solutionbox}
\begin{mnemonicbox}
``SWHBG-THS'' - Solar, Wind, Hydro, Biomass,
Geothermal - Tidal, Hydrogen, Solar thermal

\end{mnemonicbox}
\subsection*{Question 2(a) [3 marks]}\label{q2a}

\textbf{Define Nanotechnology \& List Applications of Nanotechnology.}

\begin{solutionbox}

\textbf{Nanotechnology} is the science of manipulating matter at atomic
and molecular scale (1-100 nanometers).

\textbf{Applications:}

\begin{itemize}
\tightlist
\item
  \textbf{Electronics}: Smaller, faster processors
\item
  \textbf{Medicine}: Drug delivery systems
\item
  \textbf{Energy}: Solar cells, batteries
\item
  \textbf{Materials}: Stronger, lighter composites
\end{itemize}

\end{solutionbox}
\begin{mnemonicbox}
``NEMS'' - Nano Electronics, Medicine, Solar

\end{mnemonicbox}
\subsection*{Question 2(b) [4 marks]}\label{q2b}

\textbf{Give Full forms of: UAV, IOT, AI, M2M}

\begin{solutionbox}


{\def\LTcaptype{none} % do not increment counter
\vspace{-5pt}
\captionof{table}{Technology Abbreviations}
\vspace{-10pt}
\begin{longtable}[]{@{}lll@{}}
\toprule\noalign{}
Abbreviation & Full Form & Application \\
\midrule\noalign{}
\endhead
\bottomrule\noalign{}
\endlastfoot
\textbf{UAV} & Unmanned Aerial Vehicle & Surveillance, delivery \\
\textbf{IOT} & Internet of Things & Smart homes, cities \\
\textbf{AI} & Artificial Intelligence & Machine learning, automation \\
\textbf{M2M} & Machine to Machine & Industrial automation \\
\end{longtable}
}

\end{solutionbox}
\begin{mnemonicbox}
``UIAM'' - UAV, IOT, AI, M2M

\end{mnemonicbox}
\subsection*{Question 2(c) [7 marks]}\label{q2c}

\textbf{Describe the block diagram of a drone and its major components.}

\begin{solutionbox}

\textbf{Block Diagram:}

\begin{center}
\textbf{Mermaid Diagram (Code)}
\begin{verbatim}
{Shaded}
{Highlighting}[]
graph TD
    A[Flight Controller] {-{-}{} B[Motors \& Propellers]}
    A {-{-}{} C[GPS Module]}
    A {-{-}{} D[IMU Sensors]}
    A {-{-}{} E[Camera]}
    F[Battery] {-{-}{} A}
    G[Remote Controller] {-{-}{} H[Receiver]}
    H {-{-}{} A}
    A {-{-}{} I[Gimbal]}
{Highlighting}
{Shaded}
\end{verbatim}
\end{center}

\textbf{Major Components:}

\begin{itemize}
\tightlist
\item
  \textbf{Flight Controller}: Brain of drone, processes sensor data
\item
  \textbf{Motors \& Propellers}: Provide thrust and control movement
\item
  \textbf{Battery}: Powers all electronic components
\item
  \textbf{GPS Module}: Provides location and navigation data
\item
  \textbf{IMU Sensors}: Measure acceleration, rotation, magnetic field
\item
  \textbf{Camera}: Captures images and videos
\item
  \textbf{Gimbal}: Stabilizes camera for smooth footage
\end{itemize}

\textbf{Working Principle:}

\begin{itemize}
\tightlist
\item
  \textbf{Control}: Remote sends commands to receiver
\item
  \textbf{Processing}: Flight controller interprets commands
\item
  \textbf{Stabilization}: IMU sensors maintain balance
\item
  \textbf{Navigation}: GPS provides position feedback
\end{itemize}

\end{solutionbox}
\begin{mnemonicbox}
``FMBGIC'' - Flight controller, Motors, Battery, GPS,
IMU, Camera

\end{mnemonicbox}
\subsection*{Question 2(a) OR [3
marks]}\label{q2a}

\textbf{Discuss IOT and its importance.}

\begin{solutionbox}

\textbf{Internet of Things (IOT)} connects everyday devices to the
internet for data exchange and remote control.

\textbf{Importance:}

\begin{itemize}
\tightlist
\item
  \textbf{Automation}: Smart homes and cities
\item
  \textbf{Efficiency}: Optimized resource usage
\item
  \textbf{Monitoring}: Real-time data collection
\end{itemize}

\end{solutionbox}
\begin{mnemonicbox}
``AEM'' - Automation, Efficiency, Monitoring

\end{mnemonicbox}
\subsection*{Question 2(b) OR [4
marks]}\label{q2b}

\textbf{Define wearable technology. Name at least three applications of
wearable technology.}

\begin{solutionbox}

\textbf{Wearable Technology} refers to electronic devices worn on the
body to monitor health, fitness, or provide information.

\textbf{Applications:}

\begin{itemize}
\tightlist
\item
  \textbf{Smart Watches}: Fitness tracking, notifications
\item
  \textbf{Smart Glasses}: Augmented reality, navigation
\item
  \textbf{Health Monitors}: Heart rate, blood pressure monitoring
\end{itemize}

\end{solutionbox}
\begin{mnemonicbox}
``WSH'' - Watches, Smart glasses, Health monitors

\end{mnemonicbox}
\subsection*{Question 2(c) OR [7
marks]}\label{q2c}

\textbf{Explain with the help of Block diagram Smart Street light
control and monitoring.}

\begin{solutionbox}

\textbf{Block Diagram:}

\begin{center}
\textbf{Mermaid Diagram (Code)}
\begin{verbatim}
{Shaded}
{Highlighting}[]
graph TD
    A[Light Sensor] {-{-}{} B[Microcontroller]}
    C[Motion Sensor] {-{-}{} B}
    D[Communication Module] {-{-}{} B}
    B {-{-}{} E[LED Street Light]}
    B {-{-}{} F[Dimming Control]}
    G[Central Control System] {-{-}{} D}
    H[Power Supply] {-{-}{} B}
{Highlighting}
{Shaded}
\end{verbatim}
\end{center}

\textbf{Components:}

\begin{itemize}
\tightlist
\item
  \textbf{Light Sensor}: Detects ambient light levels
\item
  \textbf{Motion Sensor}: Detects pedestrian/vehicle movement
\item
  \textbf{Microcontroller}: Processes sensor data and controls lighting
\item
  \textbf{Communication Module}: Wireless connection to control center
\item
  \textbf{LED Street Light}: Energy-efficient lighting
\item
  \textbf{Dimming Control}: Adjusts brightness based on need
\end{itemize}

\textbf{Working:}

\begin{itemize}
\tightlist
\item
  \textbf{Auto ON/OFF}: Lights turn on at dusk, off at dawn
\item
  \textbf{Motion Detection}: Increases brightness when movement detected
\item
  \textbf{Remote Monitoring}: Central system monitors all lights
\item
  \textbf{Energy Saving}: Dims lights when no activity detected
\end{itemize}

\end{solutionbox}
\begin{mnemonicbox}
``LMCL'' - Light sensor, Motion sensor, Controller,
LED

\end{mnemonicbox}
\subsection*{Question 3(a) [3 marks]}\label{q3a}

\textbf{Compare Organic and Inorganic electronics.}

\begin{solutionbox}


{\def\LTcaptype{none} % do not increment counter
\vspace{-5pt}
\captionof{table}{Organic vs Inorganic Electronics}
\vspace{-10pt}
\begin{longtable}[]{@{}lll@{}}
\toprule\noalign{}
Parameter & Organic Electronics & Inorganic Electronics \\
\midrule\noalign{}
\endhead
\bottomrule\noalign{}
\endlastfoot
\textbf{Material} & Carbon-based compounds & Silicon, metals \\
\textbf{Cost} & Lower manufacturing cost & Higher cost \\
\textbf{Flexibility} & Flexible, bendable & Rigid structure \\
\textbf{Processing} & Low temperature & High temperature \\
\end{longtable}
}

\end{solutionbox}
\begin{mnemonicbox}
``MCFP'' - Material, Cost, Flexibility, Processing

\end{mnemonicbox}
\subsection*{Question 3(b) [4 marks]}\label{q3b}

\textbf{Write a short note on OPVD.}

\begin{solutionbox}

\textbf{OPVD (Organic Photovoltaic Devices)} are solar cells made from
organic semiconducting materials.

\textbf{Characteristics:}

\begin{itemize}
\tightlist
\item
  \textbf{Flexible}: Can be made on flexible substrates
\item
  \textbf{Low-cost}: Cheaper manufacturing process
\item
  \textbf{Lightweight}: Suitable for portable applications
\item
  \textbf{Semi-transparent}: Can be integrated into windows
\end{itemize}

\textbf{Applications:}

\begin{itemize}
\tightlist
\item
  \textbf{Building Integration}: Solar windows
\item
  \textbf{Portable Devices}: Flexible solar chargers
\item
  \textbf{Wearable Electronics}: Solar-powered gadgets
\end{itemize}

\end{solutionbox}
\begin{mnemonicbox}
``FLLW'' - Flexible, Low-cost, Lightweight, Windows

\end{mnemonicbox}
\subsection*{Question 3(c) [7 marks]}\label{q3c}

\textbf{Explain Biometric systems and their basic block diagram.}

\begin{solutionbox}

\textbf{Biometric System} identifies individuals based on unique
biological characteristics.

\textbf{Block Diagram:}

\begin{center}
\textbf{Mermaid Diagram (Code)}
\begin{verbatim}
{Shaded}
{Highlighting}[]
graph LR
    A[Biometric Sensor] {-{-}{} B[Signal Processing]}
    B {-{-}{} C[Feature Extraction]}
    C {-{-}{} D[Template Matching]}
    D {-{-}{} E[Decision Module]}
    F[Database] {-{-}{} D}
    E {-{-}{} G[Accept/Reject]}
{Highlighting}
{Shaded}
\end{verbatim}
\end{center}

\textbf{Components:}

\begin{itemize}
\tightlist
\item
  \textbf{Sensor Module}: Captures biometric data (fingerprint, iris,
  face)
\item
  \textbf{Signal Processing}: Enhances and cleans captured signal
\item
  \textbf{Feature Extraction}: Identifies unique characteristics
\item
  \textbf{Database Module}: Stores biometric templates
\item
  \textbf{Matching Module}: Compares captured data with stored templates
\item
  \textbf{Decision Module}: Makes final accept/reject decision
\end{itemize}

\textbf{Types of Biometrics:}

\begin{itemize}
\tightlist
\item
  \textbf{Fingerprint}: Ridge patterns on fingers
\item
  \textbf{Iris}: Eye iris patterns
\item
  \textbf{Face Recognition}: Facial features
\item
  \textbf{Voice}: Voice patterns and characteristics
\end{itemize}

\textbf{Applications:}

\begin{itemize}
\tightlist
\item
  \textbf{Security}: Access control systems
\item
  \textbf{Banking}: ATM authentication
\item
  \textbf{Mobile}: Phone unlocking
\item
  \textbf{Border Control}: Immigration systems
\end{itemize}

\end{solutionbox}
\begin{mnemonicbox}
``SFEMD'' - Sensor, Feature extraction, Matching,
Database, Decision

\end{mnemonicbox}
\subsection*{Question 3(a) OR [3
marks]}\label{q3a}

\textbf{List the advantages and applications of organic electronics.}

\begin{solutionbox}

\textbf{Advantages:}

\begin{itemize}
\tightlist
\item
  \textbf{Flexible}: Bendable electronic devices
\item
  \textbf{Low-cost}: Cheaper manufacturing
\item
  \textbf{Large-area}: Can cover large surfaces
\end{itemize}

\textbf{Applications:}

\begin{itemize}
\tightlist
\item
  \textbf{OLED Displays}: Flexible screens
\item
  \textbf{Solar Cells}: Lightweight panels
\item
  \textbf{RFID Tags}: Flexible identification
\end{itemize}

\end{solutionbox}
\begin{mnemonicbox}
``FLL-OSR'' - Flexible, Low-cost, Large-area - OLED,
Solar, RFID

\end{mnemonicbox}
\subsection*{Question 3(b) OR [4
marks]}\label{q3b}

\textbf{Write a short note on OLED.}

\begin{solutionbox}

\textbf{OLED (Organic Light Emitting Diode)} is a display technology
using organic compounds that emit light when electric current is
applied.

\textbf{Advantages:}

\begin{itemize}
\tightlist
\item
  \textbf{Self-illuminating}: No backlight needed
\item
  \textbf{High contrast}: True black colors
\item
  \textbf{Flexible}: Can be bent and curved
\item
  \textbf{Energy efficient}: Lower power consumption
\end{itemize}

\textbf{Applications:}

\begin{itemize}
\tightlist
\item
  \textbf{Smartphones}: OLED screens
\item
  \textbf{TVs}: Ultra-thin displays
\item
  \textbf{Wearables}: Smartwatch displays
\end{itemize}

\end{solutionbox}
\begin{mnemonicbox}
``SHFE'' - Self-illuminating, High contrast,
Flexible, Efficient

\end{mnemonicbox}
\subsection*{Question 3(c) OR [7
marks]}\label{q3c}

\textbf{Explain AR/VR core technology and discuss its applications.}

\begin{solutionbox}

\textbf{AR (Augmented Reality)} overlays digital information on real
world, while \textbf{VR (Virtual Reality)} creates completely immersive
digital environment.

\textbf{Core Technologies:}

\begin{itemize}
\tightlist
\item
  \textbf{Display Systems}: Head-mounted displays, screens
\item
  \textbf{Tracking Systems}: Motion sensors, cameras
\item
  \textbf{Processing Units}: GPU, specialized chips
\item
  \textbf{Input Methods}: Controllers, gesture recognition
\end{itemize}

\textbf{AR Applications:}

\begin{itemize}
\tightlist
\item
  \textbf{Gaming}: Pokemon Go, mobile AR games
\item
  \textbf{Education}: Interactive learning experiences
\item
  \textbf{Navigation}: GPS overlays on real roads
\item
  \textbf{Shopping}: Virtual try-on experiences
\end{itemize}

\textbf{VR Applications:}

\begin{itemize}
\tightlist
\item
  \textbf{Entertainment}: Immersive gaming, movies
\item
  \textbf{Training}: Flight simulators, medical training
\item
  \textbf{Architecture}: Virtual building walkthroughs
\item
  \textbf{Therapy}: Treatment of phobias, PTSD
\end{itemize}


{\def\LTcaptype{none} % do not increment counter
\vspace{-5pt}
\captionof{table}{AR vs VR Comparison}
\vspace{-10pt}
\begin{longtable}[]{@{}lll@{}}
\toprule\noalign{}
Aspect & AR & VR \\
\midrule\noalign{}
\endhead
\bottomrule\noalign{}
\endlastfoot
\textbf{Reality} & Mixed with real world & Completely virtual \\
\textbf{Equipment} & Smartphone, AR glasses & VR headset, controllers \\
\textbf{Immersion} & Partial & Complete \\
\textbf{Mobility} & Mobile friendly & Stationary setup \\
\end{longtable}
}

\end{solutionbox}
\begin{mnemonicbox}
``DTPI-GENT'' - Display, Tracking, Processing, Input
- Gaming, Education, Navigation, Training

\end{mnemonicbox}
\subsection*{Question 4(a) [3 marks]}\label{q4a}

\textbf{Draw Block Diagram of a Home Solar rooftop system.}

\begin{solutionbox}

\textbf{Block Diagram:}

\begin{verbatim}
┌─────────────┐    ┌─────────────┐    ┌─────────────┐
│Solar Panels │────│   Inverter  │────│AC Load Panel│
└─────────────┘    └─────────────┘    └─────────────┘
                          │                   │
                   ┌─────────────┐    ┌─────────────┐
                   │   Battery   │    │Utility Grid │
                   │   Storage   │    │ Connection  │
                   └─────────────┘    └─────────────┘
\end{verbatim}

\textbf{Components:}

\begin{itemize}
\tightlist
\item
  \textbf{Solar Panels}: Convert sunlight to DC electricity
\item
  \textbf{Inverter}: Converts DC to AC power
\item
  \textbf{Battery Storage}: Stores excess energy
\end{itemize}

\end{solutionbox}
\begin{mnemonicbox}
``SIB'' - Solar panels, Inverter, Battery

\end{mnemonicbox}
\subsection*{Question 4(b) [4 marks]}\label{q4b}

\textbf{Explain working principle of OFET.}

\begin{solutionbox}

\textbf{OFET (Organic Field Effect Transistor)} uses organic
semiconductors to control current flow.

\textbf{Working Principle:}

\begin{itemize}
\tightlist
\item
  \textbf{Gate Voltage}: Applied voltage creates electric field
\item
  \textbf{Channel Formation}: Electric field modulates conductivity
\item
  \textbf{Current Control}: Source-drain current controlled by gate
\item
  \textbf{Switching}: ON/OFF states for digital applications
\end{itemize}

\textbf{Structure:}

\begin{itemize}
\tightlist
\item
  \textbf{Source/Drain}: Current injection points
\item
  \textbf{Gate}: Control electrode
\item
  \textbf{Organic Layer}: Active semiconductor material
\end{itemize}

\end{solutionbox}
\begin{mnemonicbox}
``GCCS'' - Gate voltage, Channel, Current, Switching

\end{mnemonicbox}
\subsection*{Question 4(c) [7 marks]}\label{q4c}

\textbf{List various Machine learning tools. Discuss any two in brief.}

\begin{solutionbox}

\textbf{Machine Learning Tools:}

\begin{itemize}
\tightlist
\item
  \textbf{TensorFlow}: Google's ML framework
\item
  \textbf{PyTorch}: Facebook's deep learning library
\item
  \textbf{Scikit-learn}: Python ML library
\item
  \textbf{Keras}: High-level neural network API
\item
  \textbf{Machine Learning for Kids}: Educational platform
\item
  \textbf{Scratch}: Visual programming for ML
\end{itemize}

\textbf{TensorFlow:}

\begin{itemize}
\tightlist
\item
  \textbf{Purpose}: Deep learning and neural networks
\item
  \textbf{Features}: Large-scale ML, production deployment
\item
  \textbf{Applications}: Image recognition, NLP, recommendation systems
\item
  \textbf{Advantages}: Scalable, extensive documentation
\end{itemize}

\textbf{Scikit-learn:}

\begin{itemize}
\tightlist
\item
  \textbf{Purpose}: General machine learning algorithms
\item
  \textbf{Features}: Classification, regression, clustering
\item
  \textbf{Applications}: Data analysis, predictive modeling
\item
  \textbf{Advantages}: Easy to use, well-documented
\end{itemize}


{\def\LTcaptype{none} % do not increment counter
\vspace{-5pt}
\captionof{table}{ML Tools Comparison}
\vspace{-10pt}
\begin{longtable}[]{@{}llll@{}}
\toprule\noalign{}
Tool & Type & Best For & Difficulty \\
\midrule\noalign{}
\endhead
\bottomrule\noalign{}
\endlastfoot
\textbf{TensorFlow} & Deep Learning & Complex models & Advanced \\
\textbf{Scikit-learn} & General ML & Beginners & Easy \\
\end{longtable}
}

\end{solutionbox}
\begin{mnemonicbox}
``TPSKMS-TF.SL'' - TensorFlow, PyTorch, Scikit,
Keras, ML4Kids, Scratch - TensorFlow, Scikit-learn

\end{mnemonicbox}
\subsection*{Question 4(a) OR [3
marks]}\label{q4a}

\textbf{Briefly explain Emerging Trends in Renewable Energy.}

\begin{solutionbox}

\textbf{Emerging Trends:}

\begin{itemize}
\tightlist
\item
  \textbf{Floating Solar}: Solar panels on water bodies
\item
  \textbf{Perovskite Cells}: Next-generation solar technology
\item
  \textbf{Green Hydrogen}: Clean fuel from renewable sources
\end{itemize}

\textbf{Benefits:}

\begin{itemize}
\tightlist
\item
  \textbf{Higher efficiency}: Better energy conversion
\item
  \textbf{Cost reduction}: Cheaper renewable energy
\end{itemize}

\end{solutionbox}
\begin{mnemonicbox}
``FPG'' - Floating solar, Perovskite, Green hydrogen

\end{mnemonicbox}
\subsection*{Question 4(b) OR [4
marks]}\label{q4b}

\textbf{Give Full forms of: AR, OLED, OPVD, OFET}

\begin{solutionbox}


{\def\LTcaptype{none} % do not increment counter
\vspace{-5pt}
\captionof{table}{Technology Full Forms}
\vspace{-10pt}
\begin{longtable}[]{@{}lll@{}}
\toprule\noalign{}
Abbreviation & Full Form & Technology Area \\
\midrule\noalign{}
\endhead
\bottomrule\noalign{}
\endlastfoot
\textbf{AR} & Augmented Reality & Mixed reality \\
\textbf{OLED} & Organic Light Emitting Diode & Display technology \\
\textbf{OPVD} & Organic Photovoltaic Device & Solar cells \\
\textbf{OFET} & Organic Field Effect Transistor & Electronics \\
\end{longtable}
}

\end{solutionbox}
\begin{mnemonicbox}
``AOOO'' - AR, OLED, OPVD, OFET

\end{mnemonicbox}
\subsection*{Question 4(c) OR [7
marks]}\label{q4c}

\textbf{Explain Block diagram of Raspberry Pi.}

\begin{solutionbox}

\textbf{Block Diagram:}

\begin{center}
\textbf{Mermaid Diagram (Code)}
\begin{verbatim}
{Shaded}
{Highlighting}[]
graph TD
    A[ARM Processor] {-{-}{} B[RAM Memory]}
    A {-{-}{} C[GPIO Pins]}
    A {-{-}{} D[USB Ports]}
    A {-{-}{} E[HDMI Output]}
    A {-{-}{} F[Ethernet Port]}
    G[MicroSD Card] {-{-}{} A}
    H[Power Supply] {-{-}{} A}
    A {-{-}{} I[Audio/Video]}
{Highlighting}
{Shaded}
\end{verbatim}
\end{center}

\textbf{Components:}

\begin{itemize}
\tightlist
\item
  \textbf{ARM Processor}: Central processing unit (Quad-core)
\item
  \textbf{RAM Memory}: System memory (1GB-8GB)
\item
  \textbf{GPIO Pins}: 40 pins for interfacing sensors/devices
\item
  \textbf{USB Ports}: Connect peripherals
\item
  \textbf{HDMI Output}: Video display connection
\item
  \textbf{Ethernet Port}: Network connectivity
\item
  \textbf{MicroSD Card}: Storage for OS and data
\item
  \textbf{Power Supply}: 5V micro-USB or USB-C
\end{itemize}

\textbf{Features:}

\begin{itemize}
\tightlist
\item
  \textbf{Operating System}: Raspberry Pi OS (Linux-based)
\item
  \textbf{Programming}: Python, C++, Scratch support
\item
  \textbf{Connectivity}: Wi-Fi, Bluetooth built-in
\item
  \textbf{Expandability}: Camera, display connectors
\end{itemize}

\textbf{Applications:}

\begin{itemize}
\tightlist
\item
  \textbf{IoT Projects}: Home automation
\item
  \textbf{Education}: Learning programming
\item
  \textbf{Robotics}: Robot control systems
\item
  \textbf{Media Center}: Home entertainment
\end{itemize}

\end{solutionbox}
\begin{mnemonicbox}
``ARGC-EPMS'' - ARM, RAM, GPIO, Connectivity -
Ethernet, Power, MicroSD, Storage

\end{mnemonicbox}
\subsection*{Question 5(a) [3 marks]}\label{q5a}

\textbf{Interface LED with Raspberry Pi.}

\begin{solutionbox}

\textbf{Circuit Connection:}

\begin{verbatim}
Raspberry Pi          LED Circuit
GPIO Pin 18 ────── 220Ω ────── LED ────── GND
                   Resistor    Anode     Cathode
\end{verbatim}

\textbf{Python Code:}

\begin{verbatim}
import RPi.GPIO as GPIO
import time

GPIO.setmode(GPIO.BCM)
GPIO.setup(18, GPIO.OUT)

while True:
    GPIO.output(18, GPIO.HIGH)  \# LED ON
    time.sleep(1)
    GPIO.output(18, GPIO.LOW)   \# LED OFF
    time.sleep(1)
\end{verbatim}

\end{solutionbox}
\begin{mnemonicbox}
``GPIO-RC'' - GPIO pin, Resistor, Code

\end{mnemonicbox}
\subsection*{Question 5(b) [4 marks]}\label{q5b}

\textbf{Explain Pandas python library For Machine Learning.}

\begin{solutionbox}

\textbf{Pandas} is a Python library for data manipulation and analysis,
essential for ML data preprocessing.

\textbf{Key Features:}

\begin{itemize}
\tightlist
\item
  \textbf{DataFrame}: Tabular data structure
\item
  \textbf{Data Cleaning}: Handle missing values, duplicates
\item
  \textbf{Data Import}: Read CSV, Excel, JSON files
\item
  \textbf{Data Analysis}: Statistical operations, grouping
\end{itemize}

\textbf{ML Applications:}

\begin{itemize}
\tightlist
\item
  \textbf{Data Preprocessing}: Clean and prepare datasets
\item
  \textbf{Feature Engineering}: Create new features from data
\item
  \textbf{Data Exploration}: Understand data patterns
\item
  \textbf{Data Transformation}: Normalize, scale data
\end{itemize}

\textbf{Common Functions:}

\begin{verbatim}
import pandas as pd
df = pd.read\_csv({data.csv})    \# Load data
df.info()                       \# Data info
df.describe()                   \# Statistics
\end{verbatim}

\end{solutionbox}
\begin{mnemonicbox}
``DCIF'' - DataFrame, Cleaning, Import, Functions

\end{mnemonicbox}
\subsection*{Question 5(c) [7 marks]}\label{q5c}

\textbf{Explain types of machine learning techniques: supervised,
unsupervised and reinforcement learning.}

\begin{solutionbox}


{\def\LTcaptype{none} % do not increment counter
\vspace{-5pt}
\captionof{table}{Machine Learning Types}
\vspace{-10pt}
\begin{longtable}[]{@{}
  >{\raggedright\arraybackslash}p{(\linewidth - 6\tabcolsep) * \real{0.1667}}
  >{\raggedright\arraybackslash}p{(\linewidth - 6\tabcolsep) * \real{0.3889}}
  >{\raggedright\arraybackslash}p{(\linewidth - 6\tabcolsep) * \real{0.1667}}
  >{\raggedright\arraybackslash}p{(\linewidth - 6\tabcolsep) * \real{0.2778}}@{}}
\toprule\noalign{}
\begin{minipage}[b]{\linewidth}\raggedright
Type
\end{minipage} & \begin{minipage}[b]{\linewidth}\raggedright
Data Required
\end{minipage} & \begin{minipage}[b]{\linewidth}\raggedright
Goal
\end{minipage} & \begin{minipage}[b]{\linewidth}\raggedright
Examples
\end{minipage} \\
\midrule\noalign{}
\endhead
\bottomrule\noalign{}
\endlastfoot
\textbf{Supervised} & Labeled data & Predict outcomes & Classification,
Regression \\
\textbf{Unsupervised} & Unlabeled data & Find patterns & Clustering,
Dimensionality reduction \\
\textbf{Reinforcement} & Reward signals & Learn optimal actions & Game
playing, Robotics \\
\end{longtable}
}

\textbf{Supervised Learning:}

\begin{itemize}
\tightlist
\item
  \textbf{Definition}: Learns from input-output pairs
\item
  \textbf{Process}: Training with known answers
\item
  \textbf{Applications}: Email spam detection, image recognition
\item
  \textbf{Algorithms}: Linear regression, decision trees, neural
  networks
\end{itemize}

\textbf{Unsupervised Learning:}

\begin{itemize}
\tightlist
\item
  \textbf{Definition}: Finds hidden patterns in data
\item
  \textbf{Process}: No target variable provided
\item
  \textbf{Applications}: Customer segmentation, anomaly detection
\item
  \textbf{Algorithms}: K-means clustering, PCA, hierarchical clustering
\end{itemize}

\textbf{Reinforcement Learning:}

\begin{itemize}
\tightlist
\item
  \textbf{Definition}: Learns through trial and error
\item
  \textbf{Process}: Agent interacts with environment
\item
  \textbf{Applications}: Game AI, autonomous vehicles, robotics
\item
  \textbf{Components}: Agent, environment, rewards, actions
\end{itemize}

\textbf{Diagram: ML Learning Process}

\begin{center}
\textbf{Mermaid Diagram (Code)}
\begin{verbatim}
{Shaded}
{Highlighting}[]
graph LR
    A[Data] {-{-}{} B\{Learning Type\}}
    B {-{-}{} C[Supervised]}
    B {-{-}{} D[Unsupervised]}
    B {-{-}{} E[Reinforcement]}
    C {-{-}{} F[Prediction Model]}
    D {-{-}{} G[Pattern Discovery]}
    E {-{-}{} H[Decision Policy]}
{Highlighting}
{Shaded}
\end{verbatim}
\end{center}

\end{solutionbox}
\begin{mnemonicbox}
``SUR-PLR-CPD'' - Supervised, Unsupervised,
Reinforcement - Prediction, Learning, Rewards - Classification,
Patterns, Decisions

\end{mnemonicbox}
\subsection*{Question 5(a) OR [3
marks]}\label{q5a}

\textbf{Explain NumPy python library For Machine Learning.}

\begin{solutionbox}

\textbf{NumPy} is fundamental library for numerical computing in Python,
essential for ML operations.

\textbf{Key Features:}

\begin{itemize}
\tightlist
\item
  \textbf{Arrays}: Multi-dimensional array objects
\item
  \textbf{Mathematical Functions}: Linear algebra operations
\item
  \textbf{Broadcasting}: Operations on different sized arrays
\end{itemize}

\textbf{ML Applications:}

\begin{itemize}
\tightlist
\item
  \textbf{Data Storage}: Efficient numerical data storage
\item
  \textbf{Matrix Operations}: Neural network computations
\item
  \textbf{Mathematical Computations}: Statistical operations
\end{itemize}

\end{solutionbox}
\begin{mnemonicbox}
``AMB'' - Arrays, Mathematical functions,
Broadcasting

\end{mnemonicbox}
\subsection*{Question 5(b) OR [4
marks]}\label{q5b}

\textbf{Write Installation steps of Raspberry Pi OS on SD card using
Raspberry Pi Imager.}

\begin{solutionbox}

\textbf{Installation Steps:}

\begin{enumerate}
\tightlist
\item
  \textbf{Download}: Install Raspberry Pi Imager from official website
\item
  \textbf{Insert SD Card}: Connect SD card (16GB+) to computer
\item
  \textbf{Select OS}: Choose Raspberry Pi OS from list
\item
  \textbf{Select Storage}: Choose SD card as target
\item
  \textbf{Write}: Click ``Write'' to flash OS to SD card
\item
  \textbf{Eject}: Safely remove SD card after completion
\end{enumerate}

\textbf{Pre-configuration Options:}

\begin{itemize}
\tightlist
\item
  \textbf{Enable SSH}: For remote access
\item
  \textbf{Set Username/Password}: Security credentials
\item
  \textbf{Configure Wi-Fi}: Network settings
\end{itemize}

\end{solutionbox}
\begin{mnemonicbox}
``DISWS-ESP'' - Download, Insert, Select OS, Write,
Storage - Enable SSH, Set credentials, Pre-configure

\end{mnemonicbox}
\subsection*{Question 5(c) OR [7
marks]}\label{q5c}

\textbf{Interface Temperature and humidity sensors with Raspberry Pi and
write Python Program for it.}

\begin{solutionbox}

\textbf{Circuit Connection:}

\begin{verbatim}
DHT22 Sensor          Raspberry Pi
VCC  ────────────────  3.3V (Pin 1)
DATA ────────────────  GPIO 4 (Pin 7)
GND  ────────────────  GND (Pin 6)
\end{verbatim}

\textbf{Python Program:}

\begin{verbatim}
import Adafruit\_DHT
import time

\# Sensor type and GPIO pin
sensor = Adafruit\_DHT.DHT22
pin = 4

while True:
    try:
        \# Read sensor data
        humidity, temperature = Adafruit\_DHT.read\_retry(sensor, pin)
        
        if humidity is not None and temperature is not None:
            print(f{Temperature: }\{temperature:.1f\}^{})
            print(f{Humidity: }\{humidity:.1f\}\%{})
        else:
            print({Failed to read sensor data})
            
        time.sleep(2)  \# Wait 2 seconds
        
    except KeyboardInterrupt:
        print("{n}Program stopped")
        break
\end{verbatim}

\textbf{Required Library:}

\begin{verbatim}
pip install Adafruit\_DHT
\end{verbatim}

\textbf{Components Used:}

\begin{itemize}
\tightlist
\item
  \textbf{DHT22}: Temperature and humidity sensor
\item
  \textbf{Raspberry Pi}: Processing unit
\item
  \textbf{Python}: Programming language
\item
  \textbf{Adafruit Library}: Sensor interface library
\end{itemize}

\textbf{Features:}

\begin{itemize}
\tightlist
\item
  \textbf{Real-time Reading}: Continuous monitoring
\item
  \textbf{Error Handling}: Handles sensor read failures
\item
  \textbf{Data Display}: Shows temperature and humidity values
\item
  \textbf{User Control}: Keyboard interrupt to stop program
\end{itemize}

\textbf{Applications:}

\begin{itemize}
\tightlist
\item
  \textbf{Weather Station}: Local weather monitoring
\item
  \textbf{Home Automation}: Climate control systems
\item
  \textbf{Agriculture}: Greenhouse monitoring
\item
  \textbf{Industrial}: Environmental monitoring
\end{itemize}

\end{solutionbox}
\begin{mnemonicbox}
``DHT-RPL'' - DHT sensor, Raspberry Pi, Python,
Library

\end{mnemonicbox}

\end{document}
