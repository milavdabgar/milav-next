\documentclass{article}

% content/resources/templates/preamble.tex
\usepackage[margin=0.6in]{geometry}
\author{Milav Dabgar}
\usepackage{amsmath,amssymb,amsthm}
\usepackage{booktabs}
\usepackage{multirow}
\usepackage{xcolor}
\usepackage{tcolorbox}
\tcbuselibrary{breakable,skins}
\usepackage[colorlinks=true,linkcolor=blue]{hyperref}
\usepackage{titlesec}
\usepackage{enumitem}
\usepackage{tikz}
\usepackage{pgfplots}
\usepackage{circuitikz}
\usepackage[version=4]{mhchem}
\usepackage{longtable}
\usepackage{array}
\usepackage{float}
\usepackage{caption}
\usepackage{listings}

\lstset{
  basicstyle=\small\ttfamily,
  breaklines=true,
  breakatwhitespace=false,
  postbreak=\mbox{\textcolor{red}{$\hookrightarrow$}\space},
  float=false,
  numbers=left,
  numberstyle=\tiny\color{gray},
  numbersep=10pt,
  xleftmargin=2em,
  keywordstyle=\color{blue},
  commentstyle=\color{green!60!black},
  stringstyle=\color{purple},
  backgroundcolor=\color{gray!5},
  showstringspaces=false,
  tabsize=2,
  captionpos=b,
  keepspaces=true,
  columns=flexible
}

\pgfplotsset{compat=1.18}
\usetikzlibrary{shapes,arrows,positioning,calc,patterns,decorations.pathmorphing,decorations.markings,arrows.meta}

% Color scheme
\definecolor{headcolor}{RGB}{0,102,204}
\definecolor{keycolor}{RGB}{220,20,60}
\definecolor{solutioncolor}{RGB}{34,139,34}
\definecolor{mnemoniccolor}{RGB}{148,0,211}
\definecolor{codecolor}{RGB}{0,0,100}

% Spacing
\setlength{\parskip}{3pt}
\setlist[itemize]{nosep}
\setlist[enumerate]{nosep}

% Title formatting
\titleformat{\section}{\Large\bfseries\color{headcolor}}{\thesection}{1em}{}
\titleformat{\subsection}{\large\bfseries\color{headcolor}}{\thesubsection}{1em}{}

% Pandoc tightlist compatibility
\providecommand{\tightlist}{%
  \setlength{\itemsep}{0pt}\setlength{\parskip}{0pt}}

% Pandoc longtable compatibility
\newcounter{none}
\def\thenone{}


% content/resources/templates/gujarati-boxes.tex
\usepackage{fontspec}
\usepackage{polyglossia}

% Set Gujarati as main language (document is primarily in Gujarati)
% Note: gloss-gujarati.ldf doesn't exist in polyglossia, but it will use hyphenation patterns
\setdefaultlanguage{gujarati}
\setotherlanguage{english}

% Configure Gujarati font properly
% Use Language=Default to prevent polyglossia from trying to add language-specific features
% that don't exist for Gujarati, which causes "empty feature" warnings
\newfontfamily\gujaratifont[Script=Gujarati,AutoFakeBold=2.5,AutoFakeSlant=0.3]{Noto Sans Gujarati}
\setmainfont[Script=Gujarati,AutoFakeBold=2.5,AutoFakeSlant=0.3]{Noto Sans Gujarati}
% Use Noto Sans Gujarati for monospace to support Gujarati in text
\setmonofont[Scale=0.9]{Noto Sans Gujarati}

% Configure English to use the same font
\newfontfamily\englishfont[Script=Gujarati,AutoFakeBold=2.5,AutoFakeSlant=0.3]{Noto Sans Gujarati}

% Translations for polyglossia
\gappto\captionsgujarati{
  \renewcommand{\tablename}{કોષ્ટક}
  \renewcommand{\figurename}{આકૃતિ}
}

% Helper for TikZ nodes to ensure Gujarati font
\newcommand{\gu}[1]{{\gujaratifont #1}}

% Custom environments
\newtcolorbox{solutionbox}{
    breakable,
    enhanced,
    colback=solutioncolor!5!white,
    colframe=solutioncolor!75!black,
    fonttitle=\bfseries,
    title=જવાબ
}

\newtcolorbox{solutionboxnobreak}{
 colback=solutioncolor!5!white,
 colframe=solutioncolor!75!black,
 fonttitle=\bfseries,
 title=જવાબ
}

\newtcolorbox{keyformula}{
 breakable,
 enhanced,
 colback=keycolor!5!white,
 colframe=keycolor!75!black,
 fonttitle=\bfseries,
 title=રાસાયણિક સમીકરણ/સૂત્ર
}

\newtcolorbox{mnemonicbox}{
 breakable,
 enhanced,
 colback=mnemoniccolor!5!white,
 colframe=mnemoniccolor!75!black,
 fonttitle=\bfseries,
 title=મેમરી ટ્રીક
}


% Custom commands for GTU solutions
% This file defines semantic commands for consistent formatting

% Question command with automatic formatting
\newcommand{\question}[2]{%
  \section*{Question #1}%
  \textbf{#2}%
}

% OR question variant
\newcommand{\questionor}[2]{%
  \section*{Question #1 OR}%
  \textbf{#2}%
}

% Proper table environment with caption
\newenvironment{answertable}[1]{%
  \begin{table}[htbp]
  \centering
  \caption{#1}
}{%
  \end{table}
}

% Proper figure environment for diagrams
\newenvironment{answerdiagram}[1]{%
  \begin{figure}[htbp]
  \centering
  \caption{#1}
}{%
  \end{figure}
}

% Semantic markup for key terms
\newcommand{\keyword}[1]{\textbf{#1}}
\newcommand{\code}[1]{\texttt{#1}}
\newcommand{\classname}[1]{\texttt{#1}}
\newcommand{\methodname}[1]{\texttt{#1}}

% Proper quotation marks
\newcommand{\mnemonic}[1]{``#1''}


\title{Renewable Energy \& Emerging Trends in Electronics (4361106) - Summer 2024 Solution (Gujarati)}
\date{May 18, 2024}

\begin{document}
\maketitle

\questionmarks{1(a)}{3}{રિન્યુએબલ એનર્જી શું છે? તેનું મહત્વ સમજાવો.}

\begin{solutionbox}
\textbf{જવાબ}:
રિન્યુએબલ એનર્જી એ કુદરતી સ્ત્રોતોમાંથી મેળવાતી ઊર્જા છે જે સમય સાથે ફરીથી બનતી રહે છે, જેમ કે સૌર, પવન, જળ, બાયોમાસ અને ભૂગર્ભીય ઊર્જા.

\begin{center}
\captionof{table}{રિન્યુએબલ એનર્જીનું મહત્વ}
\begin{tabulary}{\linewidth}{|L|L|}
\hline
\textbf{પાસું} & \textbf{ફાયદો} \\ \hline
\textbf{પર્યાવરણીય} & ગ્રીનહાઉસ ગેસ ઉત્સર્જન અને પ્રદૂષણ ઘટાડે છે \\ \hline
\textbf{આર્થિક} & નોકરીઓ બનાવે છે અને લાંબા ગાળે ઊર્જા ખર્ચ ઘટાડે છે \\ \hline
\textbf{ઊર્જા સુરક્ષા} & અશ્મિભૂત ઇંધણની આયાત પર નિર્ભરતા ઘટાડે છે \\ \hline
\textbf{ટકાઉપણું} & ભાવિ પેઢીઓ માટે અખૂટ ઊર્જા સ્ત્રોતો \\ \hline
\end{tabulary}
\end{center}

\textbf{મુખ્ય મુદ્દાઓ:}
\begin{itemize}
\item \textbf{સ્વચ્છ ઊર્જા}: કામગીરી દરમિયાન શૂન્ય કાર્બન ઉત્સર્જન
\item \textbf{ખર્ચ-અસરકારક}: ઘટતી ટેકનોલોજી કિંમતો તેને આર્થિક બનાવે છે
\item \textbf{રોજગાર સર્જન}: વધતો ઉદ્યોગ રોજગારની તકો પૂરી પાડે છે
\end{itemize}
\end{solutionbox}

\begin{mnemonicbox}
\mnemonic{"EEES" - Environmental protection, Economic benefits, Energy security, Sustainability}
\end{mnemonicbox}

\questionmarks{1(b)}{4}{ઇલેક્ટ્રિક વાહનોના પ્રકારોની યાદી બનાવો. દરેકને સંક્ષિપ્તમાં સમજાવો.}

\begin{solutionbox}
\textbf{જવાબ}:

\begin{center}
\captionof{table}{ઇલેક્ટ્રિક વાહનોના પ્રકારો}
\begin{tabulary}{\linewidth}{|L|L|L|}
\hline
\textbf{પ્રકાર} & \textbf{સંપૂર્ણ નામ} & \textbf{વર્ણન} \\ \hline
\textbf{BEV} & Battery Electric Vehicle & સંપૂર્ણ ઇલેક્ટ્રિક, માત્ર બેટરીથી ચાલે છે \\ \hline
\textbf{HEV} & Hybrid Electric Vehicle & ગેસોલિન એન્જિન અને ઇલેક્ટ્રિક મોટરનું મિશ્રણ \\ \hline
\textbf{PHEV} & Plug-in Hybrid Electric Vehicle & બાહ્ય પાવર સ્ત્રોતથી ચાર્જ કરી શકાય છે \\ \hline
\textbf{FCEV} & Fuel Cell Electric Vehicle & પાવર માટે હાઇડ્રોજન ફ્યૂઅલ સેલનો ઉપયોગ \\ \hline
\end{tabulary}
\end{center}

\textbf{મુખ્ય લક્ષણો:}
\begin{itemize}
\item \textbf{BEV}: શૂન્ય ઉત્સર્જન, ચાર્જિંગ સ્ટેશનની જરૂર
\item \textbf{HEV}: બહેતર ઇંધણ દક્ષતા, રિજનરેટિવ બ્રેકિંગ દ્વારા સ્વ-ચાર્જિંગ
\item \textbf{PHEV}: બેવડા પાવર વિકલ્પો, વિસ્તૃત રેન્જ
\item \textbf{FCEV}: ઝડપી રિફ્યુઅલિંગ, એકમાત્ર ઉત્સર્જન પાણી
\end{itemize}
\end{solutionbox}

\begin{mnemonicbox}
\mnemonic{"Big Hybrid Plug Fuel" BEV, HEV, PHEV, FCEV માટે}
\end{mnemonicbox}

\questionmarks{1(c)}{7}{સૌર ઊર્જા અને સૌર થર્મલ ઊર્જા વચ્ચે શું તફાવત છે? હોમ સોલાર રૂફટોપ સિસ્ટમના બ્લોક ડાયાગ્રામની ચર્ચા કરો.}

\begin{solutionbox}
\textbf{જવાબ}:

\begin{center}
\captionof{table}{સૌર ઊર્જા વિ સૌર થર્મલ ઊર્જા}
\begin{tabulary}{\linewidth}{|L|L|L|}
\hline
\textbf{પેરામીટર} & \textbf{સૌર ઊર્જા (PV)} & \textbf{સૌર થર્મલ ઊર્જા} \\ \hline
\textbf{રૂપાંતરણ} & સીધો સૂર્યપ્રકાશ વીજળીમાં & સૂર્યપ્રકાશ ગરમી ઊર્જામાં \\ \hline
\textbf{ટેકનોલોજી} & ફોટોવોલ્ટેઇક સેલ્સ & સોલાર કલેક્ટર્સ/પેનલ્સ \\ \hline
\textbf{આઉટપુટ} & વિદ્યુત ઊર્જા & ઉષ્મા ઊર્જા (ગરમ પાણી/વરાળ) \\ \hline
\textbf{ઉપયોગો} & પાવર જનરેશન, લાઇટિંગ & પાણી ગરમ કરવું, સ્પેસ હીટિંગ \\ \hline
\textbf{કાર્યક્ષમતા} & 15-22\% & 70-80\% \\ \hline
\end{tabulary}
\end{center}

\textbf{બ્લોક ડાયાગ્રામ: હોમ સોલાર રૂફટોપ સિસ્ટમ}

\begin{center}
\begin{tikzpicture}[node distance=1.5cm]
    \node[gtu block] (solar) {સોલાર પેનલ્સ};
    \node[gtu block, right=of solar] (cc) {ચાર્જ કંટ્રોલર};
    \node[gtu block, right=of cc] (inv) {ઇન્વર્ટર};
    \node[gtu block, below=of cc] (batt) {બેટરી બેંક};
    \node[gtu block, right=of inv] (load) {AC લોડ (ઘર)};
    \node[gtu block, above=of inv] (grid) {ગ્રીડ કનેક્શન};

    \draw[gtu arrow] (solar) -- (cc);
    \draw[gtu arrow] (cc) -- (inv);
    \draw[gtu arrow] (cc) -- (batt);
    \draw[gtu arrow] (batt) -- (cc);
    \draw[gtu arrow] (inv) -- (load);
    \draw[gtu arrow] (inv) -- (grid);
    \draw[gtu arrow] (grid) -- (inv);
\end{tikzpicture}
\captionof{figure}{હોમ સોલાર રૂફટોપ સિસ્ટમ}
\end{center}

\textbf{મુખ્ય ઘટકો:}
\begin{itemize}
\item \textbf{સોલાર પેનલ્સ}: સૂર્યપ્રકાશને DC વીજળીમાં ફેરવે છે
\item \textbf{ચાર્જ કંટ્રોલર}: બેટરી ચાર્જિંગ નિયંત્રિત કરે છે
\item \textbf{ઇન્વર્ટર}: DC ને AC પાવરમાં ફેરવે છે
\item \textbf{બેટરી બેંક}: વધારાની ઊર્જા સ્ટોર કરે છે
\item \textbf{ગ્રિડ કનેક્શન}: બે-માર્ગી પાવર ફ્લો
\end{itemize}
\end{solutionbox}

\begin{mnemonicbox}
\mnemonic{"Solar Converts Battery Inverter Grid" મુખ્ય ઘટકો માટે}
\end{mnemonicbox}

\questionmarks{1(c) OR}{7}{સૌર ફોટોવોલ્ટેઇક અસર શું છે? ફોટોવોલ્ટેઇક રૂપાંતરણનો સિદ્ધાંત સમજાવો.}

\begin{solutionbox}
\textbf{જવાબ}:
સૌર ફોટોવોલ્ટેઇક અસર એ સેમિકંડક્ટર સામગ્રી પર પ્રકાશ પડતાં વિદ્યુત પ્રવાહ ઉત્પન્ન થવાની ઘટના છે.

\textbf{ફોટોવોલ્ટેઇક રૂપાંતરણનો સિદ્ધાંત:}

\begin{center}
\begin{tikzpicture}[node distance=1.5cm]
    \node[gtu block] (sun) {સૂર્યપ્રકાશ (Photons)};
    \node[gtu block, right=of sun] (pn) {P-N જંક્શન};
    \node[gtu block, right=of pn] (eh) {ઇલેક્ટ્રોન-હોલ પેર્સ};
    \node[gtu block, below=of eh] (ef) {ઇલેક્ટ્રિક ફિલ્ડ};
    \node[gtu block, left=of ef] (curr) {કરંટ ફ્લો};
    \node[gtu block, left=of curr] (load) {બાહ્ય સર્કિટ};

    \draw[gtu arrow] (sun) -- (pn);
    \draw[gtu arrow] (pn) -- (eh);
    \draw[gtu arrow] (eh) -- (ef);
    \draw[gtu arrow] (ef) -- (curr);
    \draw[gtu arrow] (curr) -- (load);
\end{tikzpicture}
\captionof{figure}{PV રૂપાંતરણ પ્રક્રિયા}
\end{center}

\textbf{કાર્યપ્રક્રિયા:}
\begin{itemize}
\item \textbf{ફોટોન શોષણ}: પ્રકાશ ફોટોન સેમિકંડક્ટર સામગ્રીને અથડાવે છે
\item \textbf{ઇલેક્ટ્રોન ઉત્તેજના}: ઇલેક્ટ્રોન્સ ઊર્જા મેળવીને કંડક્શન બેન્ડમાં જાય છે
\item \textbf{P-N જંક્શન}: વિદ્યુત ક્ષેત્ર બનાવીને ચાર્જ અલગ કરે છે
\item \textbf{કરંટ જનરેશન}: ઇલેક્ટ્રોન્સનો પ્રવાહ વિદ્યુત પ્રવાહ બનાવે છે
\end{itemize}
\end{solutionbox}

\begin{mnemonicbox}
\mnemonic{"Photons Push Electrons Producing Power"}
\end{mnemonicbox}

\questionmarks{2(a)}{3}{નેનો ટેકનોલોજી શું છે? નેનો ટેકનોલોજી પર આધારિત કોઈપણ ત્રણ એપ્લિકેશનની યાદી બનાવો.}

\begin{solutionbox}
\textbf{જવાબ}:
નેનો ટેકનોલોજી એ મોલેક્યુલર અને પરમાણુ સ્તરે (1-100 નેનોમીટર) પદાર્થોની હેરફેર વિજ્ઞાન છે.

\begin{center}
\captionof{table}{નેનો ટેકનોલોજી એપ્લિકેશન્સ}
\begin{tabulary}{\linewidth}{|L|L|L|}
\hline
\textbf{એપ્લિકેશન} & \textbf{વર્ણન} & \textbf{ફાયદો} \\ \hline
\textbf{મેડિકલ} & ડ્રગ ડિલિવરી સિસ્ટમ, કેન્સર ટ્રીટમેન્ટ & લક્ષિત ઉપચાર \\ \hline
\textbf{ઇલેક્ટ્રોનિક્સ} & નાના, ઝડપી પ્રોસેસર અને મેમોરી & ઉચ્ચ કાર્યક્ષમતા \\ \hline
\textbf{ઊર્જા} & સોલાર સેલ્સ, બેટરીઓ, ફ્યૂઅલ સેલ્સ & બહેતર કાર્યક્ષમતા \\ \hline
\end{tabulary}
\end{center}
\end{solutionbox}

\begin{mnemonicbox}
\mnemonic{"Nano Makes Everything Better" - Medical, Electronics, Energy}
\end{mnemonicbox}

\questionmarks{2(b)}{4}{મહત્વપૂર્ણ ઉભરતી નવીનીકરણીય ઊર્જા તકનીક તરીકે ભરતી તરંગ ઊર્જા પર ટૂંકી નોંધ લખો.}

\begin{solutionbox}
\textbf{જવાબ}:
ભરતી તરંગ ઊર્જા સમુદ્રી ભરતીઓ અને તરંગોની ગતિશીલ ઊર્જાનો ઉપયોગ કરીને વીજળી ઉત્પન્ન કરે છે.

\begin{center}
\captionof{table}{ભરતી ઊર્જા સિસ્ટમ્સ}
\begin{tabulary}{\linewidth}{|L|L|L|}
\hline
\textbf{પ્રકાર} & \textbf{પદ્ધતિ} & \textbf{ફાયદો} \\ \hline
\textbf{ટાઇડલ બેરેજ} & નદીમુખ પર બંધ & ઉચ્ચ પાવર આઉટપુટ \\ \hline
\textbf{ટાઇડલ સ્ટ્રીમ} & પાણીની અંદર ટર્બાઇન & ન્યૂનતમ પર્યાવરણીય અસર \\ \hline
\textbf{વેવ એનર્જી} & સપાટીના તરંગ ગતિ & વિપુલ સંસાધન \\ \hline
\end{tabulary}
\end{center}

\textbf{મુખ્ય લક્ષણો:}
\begin{itemize}
\item \textbf{પૂર્વાનુમાન}: ભરતી નિયમિત પેટર્ન અનુસરે છે
\item \textbf{ઉચ્ચ ઘનતા}: પાણી હવા કરતાં 800 ગણું ઘન છે, જે વધુ ઊર્જા આપે છે
\end{itemize}
\end{solutionbox}

\begin{mnemonicbox}
\mnemonic{"Tides Provide Predictable Power"}
\end{mnemonicbox}

\questionmarks{2(c)}{7}{સ્માર્ટ વોટર મોનિટરિંગ સિસ્ટમ શું છે? સ્માર્ટ વોટર ક્વોલિટી મોનિટરિંગ સિસ્ટમનો બ્લોક ડાયાગ્રામ સમજાવો.}

\begin{solutionbox}
\textbf{જવાબ}:
સ્માર્ટ વોટર મોનિટરિંગ સિસ્ટમ IoT સેન્સર્સનો ઉપયોગ કરીને પાણીની ગુણવત્તાના પેરામીટર્સનું સતત નિરીક્ષણ કરે છે.

\textbf{બ્લોક ડાયાગ્રામ: સ્માર્ટ વોટર ક્વોલિટી મોનિટરિંગ સિસ્ટમ}

\begin{center}
\begin{tikzpicture}[node distance=1.5cm]
    \node[gtu block] (micro) {માઇક્રોકંટ્રોલર (Controller)};
    
    \node[gtu block, left=of micro] (ph) {pH સેન્સર};
    \node[gtu block, above left=of micro] (temp) {તાપમાન સેન્સર};
    \node[gtu block, below left=of micro] (turb) {ટર્બિડિટી સેન્સર};
    
    \node[gtu block, right=of micro] (comm) {વાયરલેસ મોડ્યુલ (WiFi/GSM)};
    \node[gtu block, right=of comm] (cloud) {ક્લાઉડ સર્વર};
    
    \node[gtu state, above=of cloud] (app) {મોબાઇલ એપ};
    \node[gtu state, below=of cloud] (dash) {વેબ ડેશબોર્ડ};
    
    \draw[gtu arrow] (ph) -- (micro);
    \draw[gtu arrow] (temp) -- (micro);
    \draw[gtu arrow] (turb) -- (micro);
    \draw[gtu arrow] (micro) -- (comm);
    \draw[gtu arrow] (comm) -- (cloud);
    \draw[gtu arrow] (cloud) -- (app);
    \draw[gtu arrow] (cloud) -- (dash);
\end{tikzpicture}
\captionof{figure}{સ્માર્ટ વોટર મોનિટરિંગ સિસ્ટમ}
\end{center}

\textbf{મુખ્ય ઘટકો:}
\begin{itemize}
\item \textbf{સેન્સર્સ}: pH, ટર્બિડિટી, તાપમાન, ઓગળેલા ઓક્સિજનનું નિરીક્ષણ
\item \textbf{માઇક્રોકંટ્રોલર}: ડેટા પ્રોસેસિંગ માટે Arduino/Raspberry Pi
\item \textbf{કમ્યુનિકેશન}: ડેટા ટ્રાન્સમિશન માટે WiFi/GSM
\item \textbf{ક્લાઉડ પ્લેટફોર્મ}: ડેટા સ્ટોરેજ અને વિશ્લેષણ
\item \textbf{યુઝર ઇન્ટરફેસ}: મોનિટરિંગ માટે મોબાઇલ એપ
\end{itemize}
\end{solutionbox}

\begin{mnemonicbox}
\mnemonic{"Smart Sensors Send Signals Safely"}
\end{mnemonicbox}

\questionmarks{2(a) OR}{3}{વેરેબલ ટેકનોલોજી શું છે? વેરેબલ ટેકનોલોજીની ઓછામાં ઓછી બે એપ્લિકેશનના નામ આપો?}

\begin{solutionbox}
\textbf{જવાબ}:
વેરેબલ ટેકનોલોજી એ ઇલેક્ટ્રોનિક ઉપકરણો છે જે કપડાં અથવા એક્સેસરીઝ તરીકે પહેરી શકાય છે, જેમાં સ્માર્ટ સેન્સર્સ અને કનેક્ટિવિટી સામેલ છે.

\textbf{એપ્લિકેશન્સ:}
\begin{itemize}
\item \textbf{આરોગ્ય નિરીક્ષણ}: હાર્ટ રેટ, પગલાં, ઊંઘની પેટર્ન ટ્રેક કરતી સ્માર્ટવોચ
\item \textbf{ફિટનેસ ટ્રેકિંગ}: કેલોરી, અંતર, કસરતનું માપ કરતા એક્ટિવિટી મોનિટર્સ
\item \textbf{મેડિકલ ડિવાઇસેસ}: સતત ગ્લુકોઝ મોનિટર્સ
\item \textbf{સ્માર્ટ ગ્લાસીસ}: ઓગમેન્ટેડ રિયાલિટી ડિસ્પ્લે
\end{itemize}
\end{solutionbox}

\begin{mnemonicbox}
\mnemonic{"Wearables Watch Wellness Wirelessly"}
\end{mnemonicbox}

\questionmarks{2(b) OR}{4}{વિવિધ પ્રકારના સોલાર સેલની યાદી બનાવો. ઇલેક્ટ્રિક વાહન માટે વિવિધ ઊર્જા સ્ત્રોતોની યાદી બનાવો.}

\begin{solutionbox}
\textbf{જવાબ}:

\begin{center}
\captionof{table}{સોલાર સેલના પ્રકારો}
\begin{tabulary}{\linewidth}{|L|L|L|}
\hline
\textbf{પ્રકાર} & \textbf{સામગ્રી} & \textbf{કાર્યક્ષમતા} \\ \hline
\textbf{મોનોક્રિસ્ટલાઇન} & સિંગલ ક્રિસ્ટલ સિલિકોન & 18-22\% \\ \hline
\textbf{પોલિક્રિસ્ટલાઇન} & મલ્ટિ-ક્રિસ્ટલ સિલિકોન & 15-17\% \\ \hline
\textbf{થિન ફિલ્મ} & એમોર્ફસ સિલિકોન & 10-12\% \\ \hline
\end{tabulary}
\end{center}

\begin{center}
\captionof{table}{ઇલેક્ટ્રિક વાહનો માટે ઊર્જા સ્ત્રોતો}
\begin{tabulary}{\linewidth}{|L|L|L|}
\hline
\textbf{સ્ત્રોત} & \textbf{વર્ણન} & \textbf{ફાયદો} \\ \hline
\textbf{બેટરી} & લિથિયમ-આયન સેલ્સ & ઉચ્ચ ઊર્જા ઘનતા \\ \hline
\textbf{ફ્યૂઅલ સેલ} & હાઇડ્રોજન રૂપાંતરણ & ઝડપી રિફ્યુઅલિંગ \\ \hline
\textbf{અલ્ટ્રાકેપેસિટર} & ઝડપી ચાર્જ/ડિસચાર્જ & ફાસ્ટ ચાર્જિંગ \\ \hline
\end{tabulary}
\end{center}
\end{solutionbox}

\begin{mnemonicbox}
\mnemonic{"Solar: Mono Poly Thin Cadmium" / "EV: Battery Fuel Ultra Regen"}
\end{mnemonicbox}

\questionmarks{2(c) OR}{7}{ડ્રોનના બ્લોક ડાયાગ્રામ અને તેના મુખ્ય ઘટકોનું વર્ણન કરો.}

\begin{solutionbox}
\textbf{જવાબ}:

\textbf{બ્લોક ડાયાગ્રામ: ડ્રોન સિસ્ટમ}

\begin{center}
\begin{tikzpicture}[node distance=1.5cm]
    \node[gtu block] (fc) {ફ્લાઇટ કંટ્રોલર};
    
    \node[gtu block, above left=of fc] (esc1) {ESC 1};
    \node[gtu block, above right=of fc] (esc2) {ESC 2};
    \node[gtu block, below left=of fc] (esc3) {ESC 3};
    \node[gtu block, below right=of fc] (esc4) {ESC 4};
    
    \node[gtu state, above=of esc1] (m1) {M1};
    \node[gtu state, above=of esc2] (m2) {M2};
    \node[gtu state, below=of esc3] (m3) {M3};
    \node[gtu state, below=of esc4] (m4) {M4};
    
    \node[gtu block, left=of fc] (rc) {રેડિયો રિસીવર};
    \node[gtu block, right=of fc] (bat) {બેટરી};
    \node[gtu block, below=of fc] (sensor) {IMU/GPS};
    
    \draw[gtu arrow] (fc) -- (esc1);
    \draw[gtu arrow] (fc) -- (esc2);
    \draw[gtu arrow] (fc) -- (esc3);
    \draw[gtu arrow] (fc) -- (esc4);
    
    \draw[gtu arrow] (esc1) -- (m1);
    \draw[gtu arrow] (esc2) -- (m2);
    \draw[gtu arrow] (esc3) -- (m3);
    \draw[gtu arrow] (esc4) -- (m4);
    
    \draw[gtu arrow] (rc) -- (fc);
    \draw[gtu arrow] (bat) -- (fc);
    \draw[gtu arrow] (sensor) -- (fc);
\end{tikzpicture}
\captionof{figure}{ડ્રોન સિસ્ટમ આર્કિટેક્ચર}
\end{center}

\textbf{મુખ્ય ઘટકો:}
\begin{itemize}
\item \textbf{ફ્લાઇટ કંટ્રોલર}: ડ્રોનનું મગજ (CPU)
\item \textbf{ESC} (Electronic Speed Controller): મોટરની ઝડપ નિયંત્રિત કરે છે
\item \textbf{મોટર્સ}: બ્રશલેસ DC મોટર્સ લિફ્ટ માટે
\item \textbf{બેટરી}: LiPo બેટરી પાવર માટે
\item \textbf{સેન્સર્સ}: IMU (જાયરો, એક્સેલેરોમીટર), GPS સ્ટેબિલિટી માટે
\end{itemize}
\end{solutionbox}

\begin{mnemonicbox}
\mnemonic{"Drones Fly Using Motors, Electronics, Sensors, Power"}
\end{mnemonicbox}

\questionmarks{3(a)}{3}{IoT શું છે? IoT ના મુખ્ય ઘટકોની યાદી બનાવો.}

\begin{solutionbox}
\textbf{જવાબ}:
IoT (Internet of Things) એ ભૌતિક ઉપકરણોનું નેટવર્ક છે જે ઇન્ટરનેટ દ્વારા ડેટા એકત્રિત અને વિનિમય કરે છે.

\begin{center}
\captionof{table}{IoT ના મુખ્ય ઘટકો}
\begin{tabulary}{\linewidth}{|L|L|L|}
\hline
\textbf{ઘટક} & \textbf{કાર્ય} & \textbf{ઉદાહરણ} \\ \hline
\textbf{સેન્સર્સ} & ડેટા એકત્રીકરણ & તાપમાન, ભેજ સેન્સર્સ \\ \hline
\textbf{કનેક્ટિવિટી} & ડેટા ટ્રાન્સમિશન & WiFi, Bluetooth, GSM \\ \hline
\textbf{ડેટા પ્રોસેસિંગ} & માહિતી વિશ્લેષણ & ક્લાઉડ કમ્પ્યુટિંગ \\ \hline
\textbf{યુઝર ઇન્ટરફેસ} & માનવીય ક્રિયાપ્રતિક્રિયા & મોબાઇલ એપ્સ, ડેશબોર્ડ \\ \hline
\end{tabulary}
\end{center}
\end{solutionbox}

\begin{mnemonicbox}
\mnemonic{"IoT Connects Smart Devices Using Internet"}
\end{mnemonicbox}

\questionmarks{3(b)}{4}{કાર્બનિક અને અકાર્બનિક ઇલેક્ટ્રોનિક્સ વચ્ચે સરખામણી કરો.}

\begin{solutionbox}
\textbf{જવાબ}:

\begin{center}
\captionof{table}{કાર્બનિક વિ અકાર્બનિક ઇલેક્ટ્રોનિક્સ}
\begin{tabulary}{\linewidth}{|L|L|L|}
\hline
\textbf{પેરામીટર} & \textbf{કાર્બનિક ઇલેક્ટ્રોનિક્સ} & \textbf{અકાર્બનિક ઇલેક્ટ્રોનિક્સ} \\ \hline
\textbf{સામગ્રી} & કાર્બન આધારિત સંયોજનો & સિલિકોન, ધાતુઓ \\ \hline
\textbf{ઉત્પાદન} & ઓછું તાપમાન, પ્રિન્ટિંગ & ઊંચું તાપમાન, ક્લીન રૂમ \\ \hline
\textbf{લવચીકતા} & લવચીક, વળી શકાય તેવું & કઠોર, બરડ \\ \hline
\textbf{કિંમત} & ઓછી ઉત્પાદન કિંમત & ઊંચી ઉત્પાદન કિંમત \\ \hline
\textbf{કાર્યક્ષમતા} & ઓછી ઝડપ, કાર્યક્ષમતા & ઊંચી ઝડપ, કાર્યક્ષમતા \\ \hline
\end{tabulary}
\end{center}
\end{solutionbox}

\begin{mnemonicbox}
\mnemonic{"Organic: Flexible, Cheap, Printable vs Inorganic: Fast, Stable, Expensive"}
\end{mnemonicbox}

\questionmarks{3(c)}{7}{સ્માર્ટ સ્ટ્રીટ લાઇટ કંટ્રોલ અને મોનિટરિંગ સિસ્ટમનો બ્લોક ડાયાગ્રામ દોરો. ઉદ્યોગમાં AR/VR ટેકનોલોજીના ફાયદા અને ઉપયોગની ચર્ચા કરો.}

\begin{solutionbox}
\textbf{જવાબ}:

\textbf{બ્લોક ડાયાગ્રામ: સ્માર્ટ સ્ટ્રીટ લાઇટ સિસ્ટમ}

\begin{center}
\begin{tikzpicture}[node distance=1.5cm]
    \node[gtu block] (micro) {માઇક્રોકંટ્રોલર};
    
    \node[gtu block, left=of micro] (light) {લાઇટ સેન્સર};
    \node[gtu block, above=of micro] (motion) {મોશન સેન્સર};
    \node[gtu block, below=of micro] (remote) {રિમોટ કંટ્રોલ};
    
    \node[gtu block, right=of micro] (driver) {LED ડ્રાઈવર};
    \node[gtu state, right=of driver] (led) {LED સ્ટ્રીટ લાઇટ};
    
    \node[gtu block, below right=of micro] (wireless) {વાયરલેસ મોડ્યુલ};
    \node[gtu block, right=of wireless] (central) {સેન્ટ્રલ કંટ્રોલ};
    \node[gtu state, right=of central] (dash) {ડેશબોર્ડ};
    
    \draw[gtu arrow] (light) -- (micro);
    \draw[gtu arrow] (motion) -- (micro);
    \draw[gtu arrow] (remote) -- (micro);
    \draw[gtu arrow] (micro) -- (driver);
    \draw[gtu arrow] (driver) -- (led);
    \draw[gtu arrow] (micro) -- (wireless);
    \draw[gtu arrow] (wireless) -- (central);
    \draw[gtu arrow] (central) -- (dash);
\end{tikzpicture}
\captionof{figure}{સ્માર્ટ સ્ટ્રીટ લાઇટ સિસ્ટમ}
\end{center}

\textbf{ઉદ્યોગમાં AR/VR ટેકનોલોજી:}

\begin{center}
\captionof{table}{AR/VR એપ્લિકેશન્સ}
\begin{tabulary}{\linewidth}{|L|L|L|}
\hline
\textbf{ઉદ્યોગ} & \textbf{AR એપ્લિકેશન} & \textbf{VR એપ્લિકેશન} \\ \hline
\textbf{મેન્યુફેક્ચરિંગ} & એસેમ્બલી સૂચનાઓ & ટ્રેનિંગ સિમ્યુલેશન \\ \hline
\textbf{હેલ્થકેર} & સર્જરી સહાયતા & મેડિકલ ટ્રેનિંગ \\ \hline
\textbf{શિક્ષણ} & ઇન્ટરેક્ટિવ લર્નિંગ & વર્ચ્યુઅલ ક્લાસરૂમ \\ \hline
\end{tabulary}
\end{center}
\end{solutionbox}

\begin{mnemonicbox}
\mnemonic{"AR/VR: Training, Design, Remote, Maintenance"}
\end{mnemonicbox}

\questionmarks{3(a) OR}{3}{સ્માર્ટ સિસ્ટમ શું છે? કોઈપણ ચાર પ્રકારની સ્માર્ટ સિસ્ટમની યાદી બનાવો.}

\begin{solutionbox}
\textbf{જવાબ}:
સ્માર્ટ સિસ્ટમ એ બુદ્ધિશાળી સિસ્ટમ છે જે સેન્સર્સ, ડેટા પ્રોસેસિંગ અને ઓટોમેશનનો ઉપયોગ કરીને નિર્ણયો લે છે.

\begin{center}
\captionof{table}{સ્માર્ટ સિસ્ટમના પ્રકારો}
\begin{tabulary}{\linewidth}{|L|L|L|}
\hline
\textbf{પ્રકાર} & \textbf{વર્ણન} & \textbf{ઉદાહરણ} \\ \hline
\textbf{સ્માર્ટ હોમ} & સ્વચાલિત ઘર નિયંત્રણ & લાઇટિંગ, HVAC, સિક્યુરિટી \\ \hline
\textbf{સ્માર્ટ સિટી} & શહેરી ઇન્ફ્રાસ્ટ્રક્ચર મેનેજમેન્ટ & ટ્રાફિક, યુટિલિટીઝ, કચરો \\ \hline
\textbf{સ્માર્ટ ગ્રિડ} & બુદ્ધિશાળી પાવર વિતરણ & ઊર્જા મેનેજમેન્ટ \\ \hline
\textbf{સ્માર્ટ હેલ્થકેર} & મેડિકલ મોનિટરિંગ & દર્દી મોનિટરિંગ \\ \hline
\end{tabulary}
\end{center}
\end{solutionbox}

\begin{mnemonicbox}
\mnemonic{"Smart: Home, City, Grid, Health"}
\end{mnemonicbox}

\questionmarks{3(b) OR}{4}{ઓર્ગેનિક ઇલેક્ટ્રોનિક્સના ફાયદા અને એપ્લિકેશનની યાદી બનાવો.}

\begin{solutionbox}
\textbf{જવાબ}:

\begin{center}
\captionof{table}{ઓર્ગેનિક ઇલેક્ટ્રોનિક્સના ફાયદા}
\begin{tabulary}{\linewidth}{|L|L|L|}
\hline
\textbf{ફાયદો} & \textbf{વર્ણન} & \textbf{લાભ} \\ \hline
\textbf{લવચીકતા} & વળી શકાય, ખેંચાય તેવું & પહેરી શકાય તેવા ઉપકરણો \\ \hline
\textbf{ઓછી કિંમત} & સસ્તું ઉત્પાદન & મોટા પાયે ઉત્પાદન \\ \hline
\textbf{મોટો વિસ્તાર} & મોટી સપાટી પર પ્રિન્ટિંગ & મોટા ડિસ્પ્લે \\ \hline
\end{tabulary}
\end{center}

\textbf{એપ્લિકેશન્સ:}
\begin{itemize}
\item \textbf{OLED ડિસ્પ્લે}: સ્માર્ટફોન, TV
\item \textbf{ઓર્ગેનિક સોલાર સેલ્સ}: લવચીક સોલાર પેનલ્સ
\end{itemize}
\end{solutionbox}

\begin{mnemonicbox}
\mnemonic{"Organic: Flexible, Cheap, Large, Low-temp"}
\end{mnemonicbox}

\questionmarks{3(c) OR}{7}{(i) પહેરી શકાય તેવી સ્માર્ટ ઘડિયાળ અને (ii) બાયોમેટ્રિક સિસ્ટમનો મૂળભૂત બ્લોક ડાયાગ્રામ દોરો.}

\begin{solutionbox}
\textbf{જવાબ}:

\textbf{(i) વેરેબલ સ્માર્ટ વોચ બ્લોક ડાયાગ્રામ:}

\begin{center}
\begin{tikzpicture}[node distance=1.5cm]
    \node[gtu block] (proc) {માઇક્રોપ્રોસેસર};
    
    \node[gtu block, left=of proc] (sens) {સેન્સર્સ};
    \node[gtu block, below left=of sens] (hr) {Heart Rate};
    \node[gtu block, left=of sens] (acc) {Accelerometer};
    \node[gtu block, above left=of sens] (gps) {GPS};
    
    \node[gtu block, above=of proc] (disp) {ડિસ્પ્લે};
    \node[gtu block, right=of proc] (mem) {મેમોરી};
    \node[gtu block, below=of proc] (batt) {બેટરી};
    \node[gtu block, below right=of proc] (charge) {ચાર્જિંગ};
    \node[gtu block, above right=of proc] (wireless) {વાયરલેસ};
    
    \draw[gtu arrow] (hr) -- (sens);
    \draw[gtu arrow] (acc) -- (sens);
    \draw[gtu arrow] (gps) -- (sens);
    \draw[gtu arrow] (sens) -- (proc);
    \draw[gtu arrow] (disp) -- (proc);
    \draw[gtu arrow] (batt) -- (proc);
    \draw[gtu arrow] (wireless) -- (proc);
    \draw[gtu arrow] (proc) -- (mem);
    \draw[gtu arrow] (proc) -- (charge);
\end{tikzpicture}
\captionof{figure}{સ્માર્ટ વોચ આર્કિટેક્ચર}
\end{center}

\textbf{(ii) બાયોમેટ્રિક સિસ્ટમ બ્લોક ડાયાગ્રામ:}

\begin{center}
\begin{tikzpicture}[node distance=1.2cm]
    \node[gtu block] (sensor) {બાયોમેટ્રિક સેન્સર};
    \node[gtu block, right=of sensor] (signal) {સિગ્નલ પ્રોસેસિંગ};
    \node[gtu block, right=of signal] (feat) {ફીચર એક્સટ્રેક્શન};
    \node[gtu block, below=of feat] (match) {ટેમ્પ્લેટ મેચિંગ};
    \node[gtu block, left=of match] (db) {ડેટાબેસ};
    \node[gtu block, left=of db] (enroll) {એનરોલમેન્ટ};
    \node[gtu decision, right=of match] (decide) {ડિસિઝન};
    \node[gtu state, below=of decide] (access) {એક્સેસ કંટ્રોલ};

    \draw[gtu arrow] (sensor) -- (signal);
    \draw[gtu arrow] (signal) -- (feat);
    \draw[gtu arrow] (feat) -- (match);
    \draw[gtu arrow] (enroll) -- (db);
    \draw[gtu arrow] (db) -- (match);
    \draw[gtu arrow] (match) -- (decide);
    \draw[gtu arrow] (decide) -- (access);
\end{tikzpicture}
\captionof{figure}{બાયોમેટ્રિક સિસ્ટમ}
\end{center}
\end{solutionbox}

\begin{mnemonicbox}
\mnemonic{"Smart Watch: Sense, Process, Display, Connect" / "Biometric: Capture, Process, Match, Decide"}
\end{mnemonicbox}

\questionmarks{4(a)}{3}{રાસ્પબેરી પાઇમાં NOOBS, GPIO અને LXDE નું સંપૂર્ણ સ્વરૂપ આપો.}

\begin{solutionbox}
\textbf{જવાબ}:

\begin{center}
\captionof{table}{રાસ્પબેરી પાઇ સંક્ષેપ}
\begin{tabulary}{\linewidth}{|L|L|L|}
\hline
\textbf{સંક્ષેપ} & \textbf{સંપૂર્ણ સ્વરૂપ} & \textbf{હેતુ} \\ \hline
\textbf{NOOBS} & New Out Of Box Software & સરળ OS ઇન્સ્ટોલેશન \\ \hline
\textbf{GPIO} & General Purpose Input Output & હાર્ડવેર ઇન્ટરફેસ પિન્સ \\ \hline
\textbf{LXDE} & Lightweight X11 Desktop Environment & ડેસ્કટોપ ઇન્ટરફેસ \\ \hline
\end{tabulary}
\end{center}
\end{solutionbox}

\begin{mnemonicbox}
\mnemonic{"New GPIO, Lightweight Experience"}
\end{mnemonicbox}

\questionmarks{4(b)}{4}{OLED પર ટૂંકી નોંધ લખો.}

\begin{solutionbox}
\textbf{જવાબ}:
OLED (Organic Light Emitting Diode) એ ડિસ્પ્લે ટેકનોલોજી છે જે કાર્બનિક સંયોજનોનો ઉપયોગ કરે છે જે વિદ્યુત પ્રવાહ લાગુ કરવામાં આવે ત્યારે પ્રકાશ ઉત્સર્જન કરે છે.

\begin{center}
\captionof{table}{OLED વિ LCD}
\begin{tabulary}{\linewidth}{|L|L|L|}
\hline
\textbf{પેરામીટર} & \textbf{OLED} & \textbf{LCD} \\ \hline
\textbf{બેકલાઇટ} & જરૂરી નથી & જરૂરી \\ \hline
\textbf{કોન્ટ્રાસ્ટ} & અનંત & 1000:1 \\ \hline
\textbf{જાડાઈ} & અલ્ટ્રા-થિન & જાડું \\ \hline
\end{tabulary}
\end{center}
\end{solutionbox}

\begin{mnemonicbox}
\mnemonic{"OLED: Organic, Light, Emitting, Display"}
\end{mnemonicbox}

\questionmarks{4(c)}{7}{રાસ્પબેરી પાઇનું આર્કિટેક્ચર અને બ્લોક ડાયાગ્રામ સમજાવો.}

\begin{solutionbox}
\textbf{જવાબ}:

\textbf{બ્લોક ડાયાગ્રામ: રાસ્પબેરી પાઇ આર્કિટેક્ચર}

\begin{center}
\begin{tikzpicture}[node distance=1.5cm]
    \node[gtu block] (cpu) {ARM Cortex CPU};
    \node[gtu block, below=of cpu] (bus) {સિસ્ટમ બસ};
    \node[gtu block, left=of bus] (gpu) {GPU};
    \node[gtu block, right=of bus] (ram) {RAM};
    
    \node[gtu block, below left=of bus] (sd) {SD Card Slot};
    \node[gtu block, below right=of bus] (gpio) {GPIO Pins};
    
    \node[gtu block, below=of sd] (usb) {USB Ports};
    \node[gtu block, below=of gpio] (eth) {Ethernet};
    
    \node[gtu block, right=of ram] (hdmi) {HDMI};
    \node[gtu block, left=of gpu] (cam) {કેમેરા ઇન્ટરફેસ};
    
    \draw[gtu arrow] (cpu) -- (bus);
    \draw[gtu arrow] (gpu) -- (bus);
    \draw[gtu arrow] (ram) -- (bus);
    \draw[gtu arrow] (sd) -- (bus);
    \draw[gtu arrow] (bus) -- (gpio);
    \draw[gtu arrow] (bus) -- (usb);
    \draw[gtu arrow] (bus) -- (eth);
    \draw[gtu arrow] (bus) -- (hdmi);
    \draw[gtu arrow] (bus) -- (cam);
\end{tikzpicture}
\captionof{figure}{રાસ્પબેરી પાઇ આર્કિટેક્ચર}
\end{center}

\textbf{મુખ્ય ઘટકો:}
\begin{itemize}
\item \textbf{CPU}: ARM Cortex-A72 Quad-core (મુખ્ય પ્રોસેસિંગ)
\item \textbf{GPU}: VideoCore VI (ગ્રાફિક્સ પ્રોસેસિંગ)
\item \textbf{RAM}: 4GB LPDDR4 (સિસ્ટમ મેમોરી)
\item \textbf{સ્ટોરેજ}: MicroSD કાર્ડ (ઓપરેટિંગ સિસ્ટમ)
\end{itemize}
\end{solutionbox}

\begin{mnemonicbox}
\mnemonic{"Pi: Processor, Interfaces, Projects, Internet"}
\end{mnemonicbox}

\questionmarks{4(a) OR}{3}{રાસ્પબેરી પાઇ શું છે અને તેના ફાયદા અને ગેરફાયદા શું છે?}

\begin{solutionbox}
\textbf{જવાબ}:
રાસ્પબેરી પાઇ એ નાનું, સસ્તું સિંગલ-બોર્ડ કમ્પ્યુટર છે જે શિક્ષણ અને શોખીન પ્રોજેક્ટ્સ માટે ડિઝાઇન કરવામાં આવ્યું છે.

\begin{center}
\captionof{table}{ફાયદા અને ગેરફાયદા}
\begin{tabulary}{\linewidth}{|L|L|}
\hline
\textbf{ફાયદા} & \textbf{ગેરફાયદા} \\ \hline
ઓછી કિંમત & મર્યાદિત કાર્યક્ષમતા \\ \hline
નાનું સાઇઝ & બિલ્ટ-ઇન સ્ટોરેજ નથી \\ \hline
GPIO પિન્સ & SD કાર્ડની જરૂર \\ \hline
Linux સપોર્ટ & રીઅલ-ટાઇમ OS નથી \\ \hline
\end{tabulary}
\end{center}
\end{solutionbox}

\begin{mnemonicbox}
\mnemonic{"Pi: Cheap, Small, Educational vs Limited, External, Power"}
\end{mnemonicbox}

\questionmarks{4(b) OR}{4}{OFET પર ટૂંકી નોંધ લખો.}

\begin{solutionbox}
\textbf{જવાબ}:
OFET (Organic Field Effect Transistor) એ કાર્બનિક સેમિકંડક્ટિંગ સામગ્રીનો ઉપયોગ કરીને સ્વિચિંગ અને એમ્પ્લિફિકેશન માટેનો ટ્રાન્ઝિસ્ટર છે.

\begin{center}
\captionof{table}{OFET સ્ટ્રક્ચર}
\begin{tabulary}{\linewidth}{|L|L|L|}
\hline
\textbf{ઘટક} & \textbf{સામગ્રી} & \textbf{કાર્ય} \\ \hline
\textbf{ગેટ} & મેટલ ઇલેક્ટ્રોડ & કરંટ ફ્લો કંટ્રોલ કરે છે \\ \hline
\textbf{ડાઇઇલેક્ટ્રિક} & ઇન્સ્યુલેટિંગ લેયર & ગેટને ચેનલથી અલગ કરે છે \\ \hline
\textbf{સોર્સ/ડ્રેઇન} & મેટલ કોન્ટેક્ટ્સ & કરંટ ઇન્જેક્શન/કલેક્શન \\ \hline
\textbf{ચેનલ} & ઓર્ગેનિક સેમિકંડક્ટર & કરંટ કંડક્શન પાથ \\ \hline
\end{tabulary}
\end{center}
\end{solutionbox}

\begin{mnemonicbox}
\mnemonic{"OFET: Organic, Flexible, Easy, Transistor"}
\end{mnemonicbox}

\questionmarks{4(c) OR}{7}{રાસ્પબેરી પાઇ પોર્ટ્સના પ્રકારોની સૂચિ બનાવો. રાસ્પબેરી પાઇની વિવિધ ઓપરેટિંગ સિસ્ટમ્સની ચર્ચા કરો.}

\begin{solutionbox}
\textbf{જવાબ}:

\begin{center}
\captionof{table}{રાસ્પબેરી પાઇ પોર્ટ્સ}
\begin{tabulary}{\linewidth}{|L|L|L|}
\hline
\textbf{પોર્ટ પ્રકાર} & \textbf{સંખ્યા} & \textbf{કાર્ય} \\ \hline
\textbf{USB} & 4 પોર્ટ્સ & પેરિફેરલ્સ કનેક્ટ કરવા \\ \hline
\textbf{HDMI} & 2 માઇક્રો HDMI & વીડિયો આઉટપુટ \\ \hline
\textbf{GPIO} & 40 પિન્સ & હાર્ડવેર ઇન્ટરફેસ \\ \hline
\textbf{Ethernet} & 1 પોર્ટ & વાયર્ડ નેટવર્ક \\ \hline
\end{tabulary}
\end{center}

\textbf{ઓપરેટિંગ સિસ્ટમ્સ:}

\begin{center}
\captionof{table}{રાસ્પબેરી પાઇ ઓપરેટિંગ સિસ્ટમ્સ}
\begin{tabulary}{\linewidth}{|L|L|L|}
\hline
\textbf{OS} & \textbf{પ્રકાર} & \textbf{શ્રેષ્ઠ માટે} \\ \hline
\textbf{Raspberry Pi OS} & Debian આધારિત & સામાન્ય ઉપયોગ \\ \hline
\textbf{Ubuntu} & Linux વિતરણ & સર્વર એપ્લિકેશન્સ \\ \hline
\textbf{LibreELEC} & મીડિયા સેન્ટર & હોમ એન્ટરટેઇનમેન્ટ \\ \hline
\textbf{RetroPie} & ગેમિંગ & રેટ્રો ગેમિંગ \\ \hline
\end{tabulary}
\end{center}
\end{solutionbox}

\begin{mnemonicbox}
\mnemonic{"Pi Ports: USB, HDMI, GPIO, Ethernet" / "Pi OS: Official, Ubuntu, Media, Gaming"}
\end{mnemonicbox}

\questionmarks{5(a)}{3}{મશીન લર્નિંગ માટે NumPy python library સમજાવો.}

\begin{solutionbox}
\textbf{જવાબ}:
NumPy (Numerical Python) એ વૈજ્ઞાનિક કમ્પ્યુટિંગ માટેની મૂળભૂત લાઇબ્રેરી છે.

\begin{center}
\captionof{table}{મશીન લર્નિંગમાં NumPy}
\begin{tabulary}{\linewidth}{|L|L|L|}
\hline
\textbf{ફંક્શન} & \textbf{ઉપયોગ} & \textbf{ઉદાહરણ} \\ \hline
\textbf{એરેઝ} & ડેટા સ્ટોરેજ & \code{np.array([1,2,3])} \\ \hline
\textbf{લિનિયર અલજેબ્રા} & મેટ્રિક્સ ઓપરેશન્સ & \code{np.dot(a,b)} \\ \hline
\textbf{સ્ટેટિસ્ટિક્સ} & ડેટા એનાલિસિસ & \code{np.mean()}, \code{np.std()} \\ \hline
\end{tabulary}
\end{center}
\end{solutionbox}

\begin{mnemonicbox}
\mnemonic{"NumPy: Numbers, Python, Arrays, Math"}
\end{mnemonicbox}

\questionmarks{5(b)}{4}{ઓર્ગેનિક ફોટોવોલ્ટેઇક સેલ (OPV) શું છે? તેના કાર્ય સિદ્ધાંતને સમજાવો.}

\begin{solutionbox}
\textbf{જવાબ}:
OPV (Organic Photovoltaic) સેલ એ કાર્બનિક સેમિકંડક્ટર્સનો ઉપયોગ કરીને પ્રકાશને વીજળીમાં રૂપાંતરિત કરતા સોલાર સેલ છે.

\begin{center}
\captionof{figure}{OPV કાર્યસિદ્ધાંત}
\begin{tikzpicture}[node distance=1.5cm]
    \node[gtu block] (sun) {સૂર્યપ્રકાશ};
    \node[gtu block, right=of sun] (absorb) {ઓર્ગેનિક એક્ટિવ લેયર};
    \node[gtu block, right=of absorb] (exciton) {એક્સિટન જનરેશન};
    \node[gtu block, below=of exciton] (sep) {ચાર્જ સેપરેશન};
    \node[gtu block, left=of sep] (trans) {ઇલેક્ટ્રોન ટ્રાન્સપોર્ટ};
    \node[gtu block, left=of trans] (curr) {કરંટ કલેક્શન};
    
    \draw[gtu arrow] (sun) -- (absorb);
    \draw[gtu arrow] (absorb) -- (exciton);
    \draw[gtu arrow] (exciton) -- (sep);
    \draw[gtu arrow] (sep) -- (trans);
    \draw[gtu arrow] (trans) -- (curr);
\end{tikzpicture}
\end{center}
\end{solutionbox}

\begin{mnemonicbox}
\mnemonic{"OPV: Organic, Photons, Voltage, Excitons"}
\end{mnemonicbox}

\questionmarks{5(c)}{7}{કોઈપણ ચાર મશીન લર્નિંગ ટૂલ્સની યાદી બનાવો. કોઈપણ એકની સંક્ષિપ્તમાં ચર્ચા કરો.}

\begin{solutionbox}
\textbf{જવાબ}:

\begin{center}
\captionof{table}{મશીન લર્નિંગ ટૂલ્સ}
\begin{tabulary}{\linewidth}{|L|L|L|}
\hline
\textbf{ટૂલ} & \textbf{પ્રકાર} & \textbf{શ્રેષ્ઠ માટે} \\ \hline
\textbf{TensorFlow} & ડીપ લર્નિંગ ફ્રેમવર્ક & ન્યુરલ નેટવર્ક્સ \\ \hline
\textbf{Scikit-learn} & જનરલ ML લાઇબ્રેરી & પરંપરાગત એલ્ગોરિધમ \\ \hline
\textbf{PyTorch} & ડીપ લર્નિંગ ફ્રેમવર્ક & સંશોધન અને વિકાસ \\ \hline
\textbf{Keras} & હાઇ-લેવલ API & ઝડપી પ્રોટોટાઇપિંગ \\ \hline
\end{tabulary}
\end{center}

\textbf{વિગતવાર ચર્ચા: TensorFlow}
TensorFlow એ Google દ્વારા વિકસિત ML મોડેલ્સ બનાવવા અને તૈનાત કરવા માટેનું ઓપન-સોર્સ મશીન લર્નિંગ ફ્રેમવર્ક છે.

\textbf{કોડ ઉદાહરણ:}
\begin{lstlisting}[language=Python]
import tensorflow as tf
model = tf.keras.Sequential([
    tf.keras.layers.Dense(128, activation='relu'),
    tf.keras.layers.Dense(10, activation='softmax')
])
\end{lstlisting}
\end{solutionbox}

\begin{mnemonicbox}
\mnemonic{"TensorFlow: Tensors, Graphs, Scale, Deploy"}
\end{mnemonicbox}

\questionmarks{5(a) OR}{3}{મશીન લર્નિંગ માટે પાન્ડા python library સમજાવો.}

\begin{solutionbox}
\textbf{જવાબ}:
Pandas એ ડેટા મેનિપ્યુલેશન અને એનાલિસિસ માટેની Python લાઇબ્રેરી છે.

\begin{center}
\captionof{table}{Pandas ફંક્શન્સ}
\begin{tabulary}{\linewidth}{|L|L|L|}
\hline
\textbf{ફંક્શન} & \textbf{ઉપયોગ} & \textbf{ઉદાહરણ} \\ \hline
\textbf{ડેટા લોડિંગ} & ડેટાસેટ્સ ઇમ્પોર્ટ & \code{pd.read\_csv()} \\ \hline
\textbf{ડેટા ક્લીનિંગ} & મિસિંગ રિમૂવ/ફિલ & \code{df.dropna()} \\ \hline
\textbf{ડેટા સિલેક્શન} & ડેટા ફિલ્ટર & \code{df[df['col'] > 5]} \\ \hline
\textbf{એગ્રીગેશન} & ગ્રુપ અને સમરાઇઝ & \code{df.groupby().mean()} \\ \hline
\end{tabulary}
\end{center}
\end{solutionbox}

\begin{mnemonicbox}
\mnemonic{"Pandas: Python, Analysis, Data, Structure"}
\end{mnemonicbox}

\questionmarks{5(b) OR}{4}{ઓગમેન્ટેડ રિયાલિટી અને વર્ચ્યુઅલ રિયાલિટી વચ્ચેનો તફાવત સમજાવો.}

\begin{solutionbox}
\textbf{જવાબ}:

\begin{center}
\captionof{table}{AR વિ VR સરખામણી}
\begin{tabulary}{\linewidth}{|L|L|L|}
\hline
\textbf{પેરામીટર} & \textbf{ઓગમેન્ટેડ રિયાલિટી (AR)} & \textbf{વર્ચ્યુઅલ રિયાલિટી (VR)} \\ \hline
\textbf{પર્યાવરણ} & વાસ્તવિક વિશ્વ + ડિજિટલ ઓવરલે & સંપૂર્ણપણે વર્ચ્યુઅલ વિશ્વ \\ \hline
\textbf{હાર્ડવેર} & સ્માર્ટફોન, AR ગ્લાસીસ & VR હેડસેટ, કંટ્રોલર્સ \\ \hline
\textbf{ઇમર્શન} & આંશિક ઇમર્શન & સંપૂર્ણ ઇમર્શન \\ \hline
\textbf{ઇન્ટરેક્શન} & વાસ્તવિક વિશ્વ + ડિજિટલ ઓબ્જેક્ટ્સ & માત્ર વર્ચ્યુઅલ ઓબ્જેક્ટ્સ \\ \hline
\end{tabulary}
\end{center}
\end{solutionbox}

\begin{mnemonicbox}
\mnemonic{"AR: Augments Reality vs VR: Virtual Reality"}
\end{mnemonicbox}

\questionmarks{5(c) OR}{7}{મશીન લર્નિંગ શું છે? મશીન લર્નિંગના વિવિધ પ્રકારોની ચર્ચા કરો.}

\begin{solutionbox}
\textbf{જવાબ}:
મશીન લર્નિંગ એ આર્ટિફિશિયલ ઇન્ટેલિજન્સનો ઉપવિભાગ છે જે કમ્પ્યુટર્સને સ્પષ્ટ રીતે પ્રોગ્રામ કર્યા વિના ડેટામાંથી શીખવા અને નિર્ણયો લેવા સક્ષમ બનાવે છે.

\textbf{સુપરવાઇઝ્ડ લર્નિંગ પ્રોસેસ:}

\begin{center}
\begin{tikzpicture}[node distance=1.5cm]
    \node[gtu block] (data) {ટ્રેનિંગ ડેટા};
    \node[gtu block, right=of data] (algo) {એલ્ગોરિધમ};
    \node[gtu block, right=of algo] (model) {મોડલ};
    
    \node[gtu block, below=of model] (new) {નવો ડેટા};
    \node[gtu block, right=of model] (pred) {પ્રિડિક્શન};
    
    \draw[gtu arrow] (data) -- (algo);
    \draw[gtu arrow] (algo) -- (model);
    \draw[gtu arrow] (new) -- (model);
    \draw[gtu arrow] (model) -- (pred);
\end{tikzpicture}
\captionof{figure}{સુપરવાઇઝ્ડ લર્નિંગ ફ્લો}
\end{center}

\textbf{મશીન લર્નિંગના પ્રકારો:}

\begin{center}
\captionof{table}{ML પ્રકારો}
\begin{tabulary}{\linewidth}{|L|L|L|}
\hline
\textbf{પ્રકાર} & \textbf{વર્ણન} & \textbf{ઉપયોગ કેસેસ} \\ \hline
\textbf{સુપરવાઇઝ્ડ} & લેબલ્ડ ડેટામાંથી શીખે છે & સ્પામ, કિંમત પૂર્વાનુમાન \\ \hline
\textbf{અનસુપરવાઇઝ્ડ} & અનલેબલ્ડ ડેટામાં પેટર્ન શોધે છે & કસ્ટમર સેગમેન્ટેશન \\ \hline
\textbf{રિઇન્ફોર્સમેન્ટ} & ટ્રાયલ અને એરર દ્વારા શીખે છે & ગેમ પ્લેઇંગ, રોબોટિક્સ \\ \hline
\end{tabulary}
\end{center}
\end{solutionbox}

\begin{mnemonicbox}
\mnemonic{"ML Types: Supervised teaches, Unsupervised discovers, Reinforcement rewards"}
\end{mnemonicbox}
\end{document}
