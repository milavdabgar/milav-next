\documentclass{article}

% content/resources/templates/preamble.tex
\usepackage[margin=0.6in]{geometry}
\author{Milav Dabgar}
\usepackage{amsmath,amssymb,amsthm}
\usepackage{booktabs}
\usepackage{multirow}
\usepackage{xcolor}
\usepackage{tcolorbox}
\tcbuselibrary{breakable,skins}
\usepackage[colorlinks=true,linkcolor=blue]{hyperref}
\usepackage{titlesec}
\usepackage{enumitem}
\usepackage{tikz}
\usepackage{pgfplots}
\usepackage{circuitikz}
\usepackage[version=4]{mhchem}
\usepackage{longtable}
\usepackage{array}
\usepackage{float}
\usepackage{caption}
\usepackage{listings}

\lstset{
  basicstyle=\small\ttfamily,
  breaklines=true,
  breakatwhitespace=false,
  postbreak=\mbox{\textcolor{red}{$\hookrightarrow$}\space},
  float=false,
  numbers=left,
  numberstyle=\tiny\color{gray},
  numbersep=10pt,
  xleftmargin=2em,
  keywordstyle=\color{blue},
  commentstyle=\color{green!60!black},
  stringstyle=\color{purple},
  backgroundcolor=\color{gray!5},
  showstringspaces=false,
  tabsize=2,
  captionpos=b,
  keepspaces=true,
  columns=flexible
}

\pgfplotsset{compat=1.18}
\usetikzlibrary{shapes,arrows,positioning,calc,patterns,decorations.pathmorphing,decorations.markings,arrows.meta}

% Color scheme
\definecolor{headcolor}{RGB}{0,102,204}
\definecolor{keycolor}{RGB}{220,20,60}
\definecolor{solutioncolor}{RGB}{34,139,34}
\definecolor{mnemoniccolor}{RGB}{148,0,211}
\definecolor{codecolor}{RGB}{0,0,100}

% Spacing
\setlength{\parskip}{3pt}
\setlist[itemize]{nosep}
\setlist[enumerate]{nosep}

% Title formatting
\titleformat{\section}{\Large\bfseries\color{headcolor}}{\thesection}{1em}{}
\titleformat{\subsection}{\large\bfseries\color{headcolor}}{\thesubsection}{1em}{}

% Pandoc tightlist compatibility
\providecommand{\tightlist}{%
  \setlength{\itemsep}{0pt}\setlength{\parskip}{0pt}}

% Pandoc longtable compatibility
\newcounter{none}
\def\thenone{}


% content/resources/templates/gujarati-boxes.tex
\usepackage{fontspec}
\usepackage{polyglossia}

% Set Gujarati as main language (document is primarily in Gujarati)
% Note: gloss-gujarati.ldf doesn't exist in polyglossia, but it will use hyphenation patterns
\setdefaultlanguage{gujarati}
\setotherlanguage{english}

% Configure Gujarati font properly
% Use Language=Default to prevent polyglossia from trying to add language-specific features
% that don't exist for Gujarati, which causes "empty feature" warnings
\newfontfamily\gujaratifont[Script=Gujarati,AutoFakeBold=2.5,AutoFakeSlant=0.3]{Noto Sans Gujarati}
\setmainfont[Script=Gujarati,AutoFakeBold=2.5,AutoFakeSlant=0.3]{Noto Sans Gujarati}
% Use Noto Sans Gujarati for monospace to support Gujarati in text
\setmonofont[Scale=0.9]{Noto Sans Gujarati}

% Configure English to use the same font
\newfontfamily\englishfont[Script=Gujarati,AutoFakeBold=2.5,AutoFakeSlant=0.3]{Noto Sans Gujarati}

% Translations for polyglossia
\gappto\captionsgujarati{
  \renewcommand{\tablename}{કોષ્ટક}
  \renewcommand{\figurename}{આકૃતિ}
}

% Helper for TikZ nodes to ensure Gujarati font
\newcommand{\gu}[1]{{\gujaratifont #1}}

% Custom environments
\newtcolorbox{solutionbox}{
    breakable,
    enhanced,
    colback=solutioncolor!5!white,
    colframe=solutioncolor!75!black,
    fonttitle=\bfseries,
    title=જવાબ
}

\newtcolorbox{solutionboxnobreak}{
 colback=solutioncolor!5!white,
 colframe=solutioncolor!75!black,
 fonttitle=\bfseries,
 title=જવાબ
}

\newtcolorbox{keyformula}{
 breakable,
 enhanced,
 colback=keycolor!5!white,
 colframe=keycolor!75!black,
 fonttitle=\bfseries,
 title=રાસાયણિક સમીકરણ/સૂત્ર
}

\newtcolorbox{mnemonicbox}{
 breakable,
 enhanced,
 colback=mnemoniccolor!5!white,
 colframe=mnemoniccolor!75!black,
 fonttitle=\bfseries,
 title=મેમરી ટ્રીક
}


% Custom commands for GTU solutions
% This file defines semantic commands for consistent formatting

% Question command with automatic formatting
\newcommand{\question}[2]{%
  \section*{Question #1}%
  \textbf{#2}%
}

% OR question variant
\newcommand{\questionor}[2]{%
  \section*{Question #1 OR}%
  \textbf{#2}%
}

% Proper table environment with caption
\newenvironment{answertable}[1]{%
  \begin{table}[htbp]
  \centering
  \caption{#1}
}{%
  \end{table}
}

% Proper figure environment for diagrams
\newenvironment{answerdiagram}[1]{%
  \begin{figure}[htbp]
  \centering
  \caption{#1}
}{%
  \end{figure}
}

% Semantic markup for key terms
\newcommand{\keyword}[1]{\textbf{#1}}
\newcommand{\code}[1]{\texttt{#1}}
\newcommand{\classname}[1]{\texttt{#1}}
\newcommand{\methodname}[1]{\texttt{#1}}

% Proper quotation marks
\newcommand{\mnemonic}[1]{``#1''}


\title{રિન્યુએબલ એનર્જી અને ઇલેક્ટ્રોનિક્સમાં ઉભરતા વલણ (4361106) - ઉનાળો 2025 ઉકેલ}
\date{May 14, 2025}

\begin{document}
\maketitle

\questionmarks{1(a)}{3}{રિન્યુએબલ એનર્જીની વ્યાખ્યા આપો અને તેનું મહત્વ સમજાવો.}

\begin{solutionbox}
\textbf{જવાબ}:
\textbf{રિન્યુએબલ એનર્જી} એ કુદરતી સ્ત્રોતોમાંથી મેળવવામાં આવતી ઊર્જા છે જે સતત ભરપાઈ થતી રહે છે, જેમ કે સૌર, પવન, પાણી, બાયોમાસ અને ભૂગર્ભીય ઊર્જા.

\begin{center}
\captionof{table}{રિન્યુએબલ એનર્જી સ્ત્રોતોના પ્રકારો}
\begin{tabulary}{\linewidth}{|L|L|L|}
\hline
\textbf{પ્રકાર} & \textbf{સ્ત્રોત} & \textbf{ફાયદો} \\ \hline
\textbf{સોલર} & સૂર્યનું કિરણોત્સર્ગ & સ્વચ્છ, પુષ્કળ \\ \hline
\textbf{વિન્ડ} & હવાની હલનચલન & કોઈ ઉત્સર્જન નહીં \\ \hline
\textbf{હાઇડ્રો} & પાણીનો પ્રવાહ & વિશ્વસનીય પાવર \\ \hline
\textbf{બાયોમાસ} & કાર્બનિક પદાર્થ & કાર્બન તટસ્થ \\ \hline
\end{tabulary}
\end{center}

\textbf{મહત્વ:}
\begin{itemize}
\item \textbf{પર્યાવરણ સુરક્ષા}: પ્રદૂષણ અને ગ્રીનહાઉસ ગેસો ઘટાડે છે
\item \textbf{ઊર્જા સુરક્ષા}: અશ્મિભૂત ઇંધન પર નિર્ભરતા ઘટાડે છે
\item \textbf{આર્થિક ફાયદા}: રોજગાર સર્જન અને ઊર્જા ખર્ચ ઘટાડે છે
\end{itemize}
\end{solutionbox}

\begin{mnemonicbox}
\mnemonic{"SEEB" - Solar, Environmental, Economic, Biomass}
\end{mnemonicbox}

\questionmarks{1(b)}{4}{સૌર ફોટોવોલ્ટેઇક અસર અને ફોટોવોલ્ટેઇક રૂપાંતરનો સિદ્ધાંત સમજાવો.}

\begin{solutionbox}
\textbf{જવાબ}:
\textbf{ફોટોવોલ્ટેઇક અસર} એ સેમિકંડક્ટર પદાર્થ પર પ્રકાશ પડવાથી વિદ્યુત વિવાહની ઉત્પત્તિ છે.

\textbf{કાર્યસિદ્ધાંત:}
\begin{enumerate}
\item \textbf{ફોટોન શોષણ}: પ્રકાશ ફોટોન્સ સોલર સેલની સપાટી પર અથડાય છે
\item \textbf{ઇલેક્ટ્રોન ઉત્તેજના}: ઇલેક્ટ્રોન્સ ઊર્જા મેળવે છે અને કંડક્શન બેન્ડમાં જાય છે
\item \textbf{ચાર્જ વિભાજન}: બિલ્ટ-ઇન ઇલેક્ટ્રિક ફીલ્ડ પોઝિટિવ અને નેગેટિવ ચાર્જ અલગ કરે છે
\item \textbf{કરંટ ઉત્પાદન}: ઇલેક્ટ્રોન્સનો પ્રવાહ DC વીજળી બનાવે છે
\end{enumerate}

\textbf{આકૃતિ:}

\begin{center}
\begin{tikzpicture}[node distance=1.5cm]
    % Layers
    \node[gtu block, minimum width=4cm, minimum height=1cm, fill=blue!10] (n) {N-type Semiconductor};
    \node[gtu block, minimum width=4cm, minimum height=1cm, fill=red!10, below=0cm of n] (p) {P-type Semiconductor};
    
    % Labels
    \node[right=0.2cm of n] {Electrons (-)};
    \node[right=0.2cm of p] {Holes (+)};
    \node[left=0.2cm of n.south west, anchor=east] {Junction};
    
    % Light
    \foreach \x in {-1.5,-0.5,0.5,1.5}
        \draw[->, decorate, decoration={snake, amplitude=2pt, segment length=5pt}, orange, thick] (\x, 1.5) -- (\x, 0.6) node[midway, right, font=\tiny] {};
    \node[above=1.5cm of n, orange] {Light Photons};

    % Circuit
    \draw[thick] (p.south) -- ++(0,-0.5) -- ++(-2.5,0) -- ++(0,2.5) to[R, l=Load] ++(0,1) -- ++(2.5,0) -- (n.north);
    \node[left=2.6cm of n] {Electric Current};
\end{tikzpicture}
\captionof{figure}{ફોટોવોલ્ટેઇક રૂપાંતર સિદ્ધાંત}
\end{center}
\end{solutionbox}

\begin{mnemonicbox}
\mnemonic{"PACE" - Photons, Absorption, Charge, Electricity}
\end{mnemonicbox}

\questionmarks{1(c)}{7}{ઇલેક્ટ્રિક વ્હીકલ (EV) ના પ્રકારો અને EV માટે વિવિધ ઊર્જા સ્ત્રોતોનું વર્ણન કરો.}

\begin{solutionbox}
\textbf{જવાબ}:

\begin{center}
\captionof{table}{ઇલેક્ટ્રિક વ્હીકલના પ્રકારો}
\begin{tabulary}{\linewidth}{|L|L|L|L|}
\hline
\textbf{EV પ્રકાર} & \textbf{સંપૂર્ણ સ્વરૂપ} & \textbf{પાવર સ્ત્રોત} & \textbf{રેંજ} \\ \hline
\textbf{BEV} & બેટરી ઇલેક્ટ્રિક વ્હીકલ & માત્ર બેટરી | 150-400 કિમી \\ \hline
\textbf{HEV} & હાયબ્રિડ ઇલેક્ટ્રિક વ્હીકલ & બેટરી + એન્જિન | 600+ કિમી \\ \hline
\textbf{PHEV} & પ્લગ-ઇન હાયબ્રિડ & બેટરી + એન્જિન | 50-100 કિમી ઇલેક્ટ્રિક \\ \hline
\textbf{FCEV} & ફ્યુઅલ સેલ ઇલેક્ટ્રિક & હાઇડ્રોજન ફ્યુઅલ સેલ | 400-600 કિમી \\ \hline
\end{tabulary}
\end{center}

\textbf{EV માટે ઊર્જા સ્ત્રોતો:}
\begin{itemize}
\item \textbf{બેટરી}: લિથિયમ-આયન બેટરીઓ વિદ્યુત ઊર્જા સંગ્રહ કરે છે
\item \textbf{ફ્યુઅલ સેલ}: હાઇડ્રોજનને વીજળીમાં રૂપાંતરિત કરે છે
\item \textbf{અલ્ટ્રાકેપેસિટર}: ઝડપી ઊર્જા સંગ્રહ અને છોડવાની પ્રક્રિયા
\item \textbf{ફ્લાયવ્હીલ}: યાંત્રિક ઊર્જા સંગ્રહ
\item \textbf{રિજનરેટિવ બ્રેકિંગ}: બ્રેકિંગ દરમિયાન ઊર્જા પુનઃપ્રાપ્ત કરે છે
\item \textbf{હાયબ્રિડ સ્ત્રોતો}: બહુવિધ ઊર્જા સ્ત્રોતોનું સંયોજન
\end{itemize}

\textbf{આકૃતિ: EV આર્કિટેક્ચર}

\begin{center}
\begin{tikzpicture}[node distance=1.5cm]
    \node[gtu block] (cont) {Controller};
    \node[gtu block, left=of cont] (bat) {Battery};
    \node[gtu block, right=of cont] (motor) {Motor};
    \node[gtu block, below=of cont] (charge) {Charging System};
    
    \draw[gtu arrow] (bat) -- (cont);
    \draw[gtu arrow] (cont) -- (motor);
    \draw[gtu arrow] (charge) -- (cont);
    \draw[gtu arrow] (charge) -- (bat);
\end{tikzpicture}
\captionof{figure}{EV આર્કિટેક્ચર}
\end{center}
\end{solutionbox}

\begin{mnemonicbox}
\mnemonic{"BHPF-BUFR" - Battery, Hybrid, Plugin, FuelCell - Battery, Ultracap, Flywheel, Regen}
\end{mnemonicbox}

\questionmarks{1(c) OR}{7}{વિવિધ પ્રકારના રિન્યુએબલ ઊર્જા સ્ત્રોતોની ચર્ચા કરો.}

\begin{solutionbox}
\textbf{જવાબ}:

\begin{center}
\captionof{table}{રિન્યુએબલ ઊર્જા સ્ત્રોતોની સરખામણી}
\begin{tabulary}{\linewidth}{|L|L|L|L|}
\hline
\textbf{સ્ત્રોત} & \textbf{કેવી રીતે કામ કરે છે} & \textbf{ફાયદા} & \textbf{ઉપયોગ} \\ \hline
\textbf{સૌર} & સૂર્યપ્રકાશને વીજળીમાં રૂપાંતરિત કરે છે & સ્વચ્છ, પુષ્કળ & રૂફટોપ સિસ્ટમ, ફાર્મ \\ \hline
\textbf{પવન} & પવન ટર્બાઇન ફેરવે છે & કોઈ ઇંધન ખર્ચ નથી & વિન્ડ ફાર્મ, ઓફશોર \\ \hline
\textbf{હાઇડ્રોઇલેક્ટ્રિક} & પાણીનો પ્રવાહ પાવર જનરેટ કરે છે & વિશ્વસનીય, લાંબા સમય સુધી ચાલે છે & ડેમ, નદીઓ \\ \hline
\textbf{બાયોમાસ} & કાર્બનિક પદાર્થોનું દહન & કાર્બન તટસ્થ & પાવર પ્લાન્ટ, હીટિંગ \\ \hline
\textbf{જીઓથર્મલ} & પૃથ્વીની ગરમ ઊર્જા & સતત ઉપલબ્ધતા & હીટિંગ, વીજળી \\ \hline
\end{tabulary}
\end{center}

\textbf{ઉભરતા વલણો:}
\begin{itemize}
\item \textbf{ટાઇડલ વેવ}: મહાસાગરની તરંગ ઊર્જા રૂપાંતરણ
\item \textbf{સૌર થર્મલ}: કેન્દ્રિત સૌર ઊર્જા સિસ્ટમ
\item \textbf{હાઇડ્રોજન}: રિન્યુએબલ સ્ત્રોતોમાંથી સ્વચ્છ ઇંધન
\end{itemize}

\textbf{ફાયદા:}
\begin{itemize}
\item \textbf{ટકાઉપણું}: ક્યારેય ખતમ થતું નથી
\item \textbf{પર્યાવરણીય}: ન્યુનતમ પ્રદૂષણ
\item \textbf{આર્થિક}: લાંબા ગાળે ઊર્જા ખર્ચ ઘટાડે છે
\end{itemize}
\end{solutionbox}

\begin{mnemonicbox}
\mnemonic{"SWHBG-THS" - Solar, Wind, Hydro, Biomass, Geothermal - Tidal, Hydrogen, Solar thermal}
\end{mnemonicbox}

\questionmarks{2(a)}{3}{નેનોટેકનોલોજી વ્યાખ્યાયિત કરો અને નેનોટેકનોલોજીની એપ્લિકેશનોની સૂચિ બનાવો.}

\begin{solutionbox}
\textbf{જવાબ}:
\textbf{નેનોટેકનોલોજી} એ અણુ અને આણવિક સ્તરે (1-100 નેનોમીટર) પદાર્થનું હેરફેર કરવાનું વિજ્ઞાન છે.

\textbf{એપ્લિકેશનો:}
\begin{itemize}
\item \textbf{ઇલેક્ટ્રોનિક્સ}: નાના, ઝડપી પ્રોસેસર
\item \textbf{મેડિસિન}: દવા પહોંચાડવાની સિસ્ટમ
\item \textbf{ઊર્જા}: સૌર સેલ, બેટરીઓ
\item \textbf{સામગ્રી}: મજબૂત, હળવા કમ્પોઝિટ
\end{itemize}
\end{solutionbox}

\begin{mnemonicbox}
\mnemonic{"NEMS" - Nano Electronics, Medicine, Solar}
\end{mnemonicbox}

\questionmarks{2(b)}{4}{સંપૂર્ણ સ્વરૂપો આપો: UAV, IOT, AI, M2M}

\begin{solutionbox}
\textbf{જવાબ}:

\begin{center}
\captionof{table}{ટેકનોલોજી સંક્ષેપો}
\begin{tabulary}{\linewidth}{|L|L|L|}
\hline
\textbf{સંક્ષેપ} & \textbf{સંપૂર્ણ સ્વરૂપ} & \textbf{એપ્લિકેશન} \\ \hline
\textbf{UAV} & Unmanned Aerial Vehicle & સર્વેલન્સ, ડિલિવરી \\ \hline
\textbf{IOT} & Internet of Things & સ્માર્ટ હોમ, શહેરો \\ \hline
\textbf{AI} & Artificial Intelligence & મશીન લર્નિંગ, ઓટોમેશન \\ \hline
\textbf{M2M} & Machine to Machine & ઇન્ડસ્ટ્રિયલ ઓટોમેશન \\ \hline
\end{tabulary}
\end{center}
\end{solutionbox}

\begin{mnemonicbox}
\mnemonic{"UIAM" - UAV, IOT, AI, M2M}
\end{mnemonicbox}

\questionmarks{2(c)}{7}{ડ્રોનના બ્લોક ડાયાગ્રામ અને તેના મુખ્ય ઘટકોનું વર્ણન કરો.}

\begin{solutionbox}
\textbf{જવાબ}:

\textbf{બ્લોક ડાયાગ્રામ:}

\begin{center}
\begin{tikzpicture}[node distance=1.5cm]
    \node[gtu block] (fc) {Flight Controller};
    
    \node[gtu block, below=of fc] (bat) {Battery};
    \node[gtu block, above=of fc] (gps) {GPS Module};
    \node[gtu block, left=of fc] (imu) {IMU Sensors};
    \node[gtu block, right=of fc] (cam) {Camera/Gimbal};
    
    \node[gtu state, above left=of fc] (m1) {Motor 1};
    \node[gtu state, above right=of fc] (m2) {Motor 2};
    \node[gtu state, below left=of fc] (m3) {Motor 3};
    \node[gtu state, below right=of fc] (m4) {Motor 4};
    
    \node[gtu block, left=of imu] (rx) {Receiver};
    \node[gtu block, left=of rx] (tx) {Remote Controller};
    
    \draw[gtu arrow] (fc) -- (m1);
    \draw[gtu arrow] (fc) -- (m2);
    \draw[gtu arrow] (fc) -- (m3);
    \draw[gtu arrow] (fc) -- (m4);
    
    \draw[gtu arrow] (bat) -- (fc);
    \draw[gtu arrow] (gps) -- (fc);
    \draw[gtu arrow] (imu) -- (fc);
    \draw[gtu arrow] (fc) -- (cam);
    
    \draw[gtu arrow] (rx) -- (fc);
    \draw[gtu arrow] (tx) -- (rx);
\end{tikzpicture}
\captionof{figure}{ડ્રોન બ્લોક ડાયાગ્રામ}
\end{center}

\textbf{મુખ્ય ઘટકો:}
\begin{itemize}
\item \textbf{ફ્લાઇટ કંટ્રોલર}: ડ્રોનનું મગજ, સેન્સર ડેટાને પ્રોસેસ કરે છે
\item \textbf{મોટર્સ અને પ્રોપેલર્સ}: થ્રસ્ટ અને કંટ્રોલ મૂવમેન્ટ પ્રદાન કરે છે
\item \textbf{બેટરી}: બધા ઇલેક્ટ્રોનિક ઘટકોને પાવર આપે છે
\item \textbf{GPS મોડ્યુલ}: સ્થાન અને નેવિગેશન ડેટા પ્રદાન કરે છે
\item \textbf{IMU સેન્સર્સ}: પ્રવેગ, પરિભ્રમણ, ચુંબકીય ક્ષેત્ર માપે છે
\item \textbf{કેમેરા}: છબીઓ અને વીડિયો કેપ્ચર કરે છે
\item \textbf{ગિમ્બલ}: સરળ ફૂટેજ માટે કેમેરાને સ્થિર કરે છે
\end{itemize}

\textbf{કાર્યસિદ્ધાંત:}
\begin{itemize}
\item \textbf{કંટ્રોલ}: રિમોટ રિસીવરને કમાન્ડ મોકલે છે
\item \textbf{પ્રોસેસિંગ}: ફ્લાઇટ કંટ્રોલર કમાન્ડનું અર્થઘટન કરે છે
\item \textbf{સ્થિરીકરણ}: IMU સેન્સર સંતુલન જાળવે છે
\item \textbf{નેવિગેશન}: GPS પોઝિશન ફીડબેક પ્રદાન કરે છે
\end{itemize}
\end{solutionbox}

\begin{mnemonicbox}
\mnemonic{"FMBGIC" - Flight controller, Motors, Battery, GPS, IMU, Camera}
\end{mnemonicbox}

\questionmarks{2(a) OR}{3}{IOT અને તેના મહત્વની ચર્ચા કરો.}

\begin{solutionbox}
\textbf{જવાબ}:
\textbf{ઇન્ટરનેટ ઓફ થિંગ્સ (IOT)} રોજિંદા ઉપકરણોને ડેટા એક્સચેન્જ અને રિમોટ કંટ્રોલ માટે ઇન્ટરનેટ સાથે જોડે છે.

\textbf{મહત્વ:}
\begin{itemize}
\item \textbf{ઓટોમેશન}: સ્માર્ટ હોમ અને શહેરો
\item \textbf{કાર્યક્ષમતા}: સંસાધનોનો ઓપ્ટિમાઇઝ્ડ ઉપયોગ
\item \textbf{મોનિટરિંગ}: રીઅલ-ટાઇમ ડેટા કલેક્શન
\end{itemize}
\end{solutionbox}

\begin{mnemonicbox}
\mnemonic{"AEM" - Automation, Efficiency, Monitoring}
\end{mnemonicbox}

\questionmarks{2(b) OR}{4}{વેરેબલ ટેકનોલોજી વ્યાખ્યાયિત કરો. વેરેબલ ટેકનોલોજીની ઓછામાં ઓછી ત્રણ એપ્લિકેશનના નામ આપો.}

\begin{solutionbox}
\textbf{જવાબ}:
\textbf{વેરેબલ ટેકનોલોજી} એ શરીર પર પહેરવામાં આવતા ઇલેક્ટ્રોનિક ઉપકરણોનો સંદર્ભ આપે છે જે આરોગ્ય, ફિટનેસ અથવા માહિતી પ્રદાન કરવા માટે મોનિટર કરે છે.

\textbf{એપ્લિકેશનો:}
\begin{itemize}
\item \textbf{સ્માર્ટ વોચ}: ફિટનેસ ટ્રેકિંગ, નોટિફિકેશન
\item \textbf{સ્માર્ટ ગ્લાસ}: ઓગમેન્ટેડ રિયાલિટી, નેવિગેશન
\item \textbf{હેલ્થ મોનિટર્સ}: હાર્ટ રેટ, બ્લડ પ્રેશર મોનિટરિંગ
\end{itemize}
\end{solutionbox}

\begin{mnemonicbox}
\mnemonic{"WSH" - Watches, Smart glasses, Health monitors}
\end{mnemonicbox}

\questionmarks{2(c) OR}{7}{બ્લોક ડાયાગ્રામની મદદથી સ્માર્ટ સ્ટ્રીટ લાઇટ કંટ્રોલ અને મોનિટરિંગ સમજાવો.}

\begin{solutionbox}
\textbf{જવાબ}:

\textbf{બ્લોક ડાયાગ્રામ:}

\begin{center}
\begin{tikzpicture}[node distance=1.5cm]
    \node[gtu block] (micro) {Microcontroller};
    
    \node[gtu block, left=of micro] (light) {Light Sensor};
    \node[gtu block, above=of micro] (motion) {Motion Sensor};
    \node[gtu block, right=of micro] (comm) {Communication Module};
    
    \node[gtu block, below=of micro] (led) {LED Street Light};
    \node[gtu block, right=of led] (dim) {Dimming Control};
    
    \node[gtu block, right=of comm] (central) {Central Control System};
    \node[gtu block, left=of led] (power) {Power Supply};
    
    \draw[gtu arrow] (light) -- (micro);
    \draw[gtu arrow] (motion) -- (micro);
    \draw[gtu arrow] (micro) -- (comm);
    \draw[gtu arrow] (comm) -- (central);
    \draw[gtu arrow] (micro) -- (led);
    \draw[gtu arrow] (micro) -- (dim);
    \draw[gtu arrow] (power) -- (micro);
    \draw[gtu arrow] (central) -- (comm);
\end{tikzpicture}
\captionof{figure}{સ્માર્ટ સ્ટ્રીટ લાઇટ કંટ્રોલ સિસ્ટમ}
\end{center}

\textbf{ઘટકો:}
\begin{itemize}
\item \textbf{લાઇટ સેન્સર}: આસપાસના પ્રકાશના સ્તરને શોધે છે
\item \textbf{મોશન સેન્સર}: પદયાત્રી/વાહનની હલનચલન શોધે છે
\item \textbf{માઇક્રોકંટ્રોલર}: સેન્સર ડેટાને પ્રોસેસ કરે છે અને લાઇટિંગ કંટ્રોલ કરે છે
\item \textbf{કમ્યુનિકેશન મોડ્યુલ}: કંટ્રોલ સેન્ટર સાથે વાયરલેસ કનેક્શન
\item \textbf{LED સ્ટ્રીટ લાઇટ}: ઊર્જા-કાર્યક્ષમ લાઇટિંગ
\item \textbf{ડિમિંગ કંટ્રોલ}: જરૂરિયાત આધારિત તેજ ગોઠવે છે
\end{itemize}

\textbf{કાર્યપ્રણાલી:}
\begin{itemize}
\item \textbf{ઓટો ON/OFF}: સાંજે લાઇટ ચાલુ, સવારે બંધ
\item \textbf{મોશન ડિટેક્શન}: હલનચલન શોધાતાં તેજ વધારે છે
\item \textbf{રિમોટ મોનિટરિંગ}: સેન્ટ્રલ સિસ્ટમ બધી લાઇટ મોનિટર કરે છે
\item \textbf{ઊર્જા બચત}: કોઈ પ્રવૃત્તિ ન હોય ત્યારે લાઇટ ડિમ કરે છે
\end{itemize}
\end{solutionbox}

\begin{mnemonicbox}
\mnemonic{"LMCL" - Light sensor, Motion sensor, Controller, LED}
\end{mnemonicbox}

\questionmarks{3(a)}{3}{ઓર્ગેનિક અને ઇનઓર્ગેનિક ઇલેક્ટ્રોનિક્સની સરખામણી કરો.}

\begin{solutionbox}
\textbf{જવાબ}:

\begin{center}
\captionof{table}{ઓર્ગેનિક vs ઇનઓર્ગેનિક ઇલેક્ટ્રોનિક્સ}
\begin{tabulary}{\linewidth}{|L|L|L|}
\hline
\textbf{પરિમાણ} & \textbf{ઓર્ગેનિક ઇલેક્ટ્રોનિક્સ} & \textbf{ઇનઓર્ગેનિક ઇલેક્ટ્રોનિક્સ} \\ \hline
\textbf{સામગ્રી} & કાર્બન-આધારિત સંયોજનો & સિલિકોન, ધાતુઓ \\ \hline
\textbf{કિંમત} & ઓછી ઉત્પાદન કિંમત & વધારે કિંમત \\ \hline
\textbf{લવચીકતા} & લવચીક, વાંકી શકાય તેવું & કઠોર માળખું \\ \hline
\textbf{પ્રોસેસિંગ} & ઓછું તાપમાન & વધારે તાપમાન \\ \hline
\end{tabulary}
\end{center}
\end{solutionbox}

\begin{mnemonicbox}
\mnemonic{"MCFP" - Material, Cost, Flexibility, Processing}
\end{mnemonicbox}

\questionmarks{3(b)}{4}{OPVD પર ટૂંકનોંધ લખો.}

\begin{solutionbox}
\textbf{જવાબ}:
\textbf{OPVD (ઓર્ગેનિક ફોટોવોલ્ટેઇક ડિવાઇસ)} એ ઓર્ગેનિક સેમિકંડક્ટીંગ સામગ્રીમાંથી બનાવેલા સોલર સેલ છે.

\textbf{લાક્ષણિકતાઓ:}
\begin{itemize}
\item \textbf{લવચીક}: લવચીક સબસ્ટ્રેટ પર બનાવી શકાય છે
\item \textbf{ઓછી કિંમત}: સસ્તી ઉત્પાદન પ્રક્રિયા
\item \textbf{હળવાવજન}: પોર્ટેબલ એપ્લિકેશન માટે યોગ્ય
\item \textbf{અર્ધ-પારદર્શક}: વિન્ડોમાં એકીકૃત કરી શકાય છે
\end{itemize}

\textbf{એપ્લિકેશનો:}
\begin{itemize}
\item \textbf{બિલ્ડિંગ એકીકરણ}: સોલર વિન્ડો
\item \textbf{પોર્ટેબલ ડિવાઇસ}: લવચીક સોલર ચાર્જર
\item \textbf{વેરેબલ ઇલેક્ટ્રોનિક્સ}: સોલર-પાવર્ડ ગેજેટ
\end{itemize}
\end{solutionbox}

\begin{mnemonicbox}
\mnemonic{"FLLW" - Flexible, Low-cost, Lightweight, Windows}
\end{mnemonicbox}

\questionmarks{3(c)}{7}{બાયોમેટ્રિક સિસ્ટમ અને તેમના મૂળભૂત બ્લોક ડાયાગ્રામ સમજાવો.}

\begin{solutionbox}
\textbf{જવાબ}:
\textbf{બાયોમેટ્રિક સિસ્ટમ} અનન્ય જૈવિક લાક્ષણિકતાઓના આધારે વ્યક્તિઓને ઓળખે છે.

\textbf{બ્લોક ડાયાગ્રામ:}

\begin{center}
\begin{tikzpicture}[node distance=1.5cm]
    \node[gtu block] (sensor) {Biometric Sensor};
    \node[gtu block, right=of sensor] (signal) {Signal Processing};
    \node[gtu block, right=of signal] (feat) {Feature Extraction};
    \node[gtu block, below=of feat] (match) {Template Matching};
    \node[gtu block, left=of match] (db) {Database};
    \node[gtu decision, right=of match] (decide) {Decision};
    \node[gtu state, below=of decide] (access) {Access Control};
    
    \draw[gtu arrow] (sensor) -- (signal);
    \draw[gtu arrow] (signal) -- (feat);
    \draw[gtu arrow] (feat) -- (match);
    \draw[gtu arrow] (db) -- (match);
    \draw[gtu arrow] (match) -- (decide);
    \draw[gtu arrow] (decide) -- (access);
\end{tikzpicture}
\captionof{figure}{બાયોમેટ્રિક સિસ્ટમ ઘટકો}
\end{center}

\textbf{ઘટકો:}
\begin{itemize}
\item \textbf{સેન્સર મોડ્યુલ}: બાયોમેટ્રિક ડેટા કેપ્ચર કરે છે (ફિંગરપ્રિન્ટ, આઇરિસ, ચહેરો)
\item \textbf{સિગ્નલ પ્રોસેસિંગ}: કેપ્ચર્ડ સિગ્નલને વધારે છે અને સાફ કરે છે
\item \textbf{ફીચર એક્સટ્રેક્શન}: અનન્ય લાક્ષણિકતાઓને ઓળખે છે
\item \textbf{ડેટાબેઝ મોડ્યુલ}: બાયોમેટ્રિક ટેમ્પલેટ સ્ટોર કરે છે
\item \textbf{મેચિંગ મોડ્યુલ}: કેપ્ચર્ડ ડેટાને સ્ટોર્ડ ટેમ્પલેટ સાથે સરખાવે છે
\item \textbf{ડિસિઝન મોડ્યુલ}: અંતિમ સ્વીકાર/નકાર નિર્ણય લે છે
\end{itemize}

\textbf{બાયોમેટ્રિક્સના પ્રકારો:}
\begin{itemize}
\item \textbf{ફિંગરપ્રિન્ટ}: આંગળીઓ પર રિજ પેટર્ન
\item \textbf{આઇરિસ}: આંખના આઇરિસ પેટર્ન
\item \textbf{ચહેરાની ઓળખ}: ચહેરાની વિશેષતાઓ
\item \textbf{અવાજ}: અવાજની પેટર્ન અને લાક્ષણિકતાઓ
\end{itemize}
\end{solutionbox}

\begin{mnemonicbox}
\mnemonic{"SFEMD" - Sensor, Feature extraction, Matching, Database, Decision}
\end{mnemonicbox}

\questionmarks{3(a) OR}{3}{ઓર્ગેનિક ઇલેક્ટ્રોનિક્સના ફાયદા અને એપ્લિકેશનની યાદી બનાવો.}

\begin{solutionbox}
\textbf{જવાબ}:

\textbf{ફાયદા:}
\begin{itemize}
\item \textbf{લવચીક}: વાંકી શકાય તેવા ઇલેક્ટ્રોનિક ઉપકરણો
\item \textbf{ઓછી કિંમત}: સસ્તી ઉત્પાદન
\item \textbf{મોટા વિસ્તાર}: મોટી સપાટીઓને ઢાંકી શકે છે
\end{itemize}

\textbf{એપ્લિકેશન:}
\begin{itemize}
\item \textbf{OLED ડિસ્પ્લે}: લવચીક સ્ક્રીન
\item \textbf{સોલર સેલ}: હળવાવજન પેનલ
\item \textbf{RFID ટેગ}: લવચીક ઓળખ
\end{itemize}
\end{solutionbox}

\begin{mnemonicbox}
\mnemonic{"FLL-OSR" - Flexible, Low-cost, Large-area - OLED, Solar, RFID}
\end{mnemonicbox}

\questionmarks{3(b) OR}{4}{OLED પર ટૂંકનોંધ લખો.}

\begin{solutionbox}
\textbf{જવાબ}:
\textbf{OLED (ઓર્ગેનિક લાઇટ એમિટિંગ ડાયોડ)} એ ડિસ્પ્લે ટેકનોલોજી છે જે ઓર્ગેનિક સંયોજનોનો ઉપયોગ કરે છે જે ઇલેક્ટ્રિક કરંટ લાગુ કરવામાં આવે ત્યારે પ્રકાશ ઉત્સર્જન કરે છે.

\textbf{ફાયદા:}
\begin{itemize}
\item \textbf{સ્વ-પ્રકાશિત}: બેકલાઇટની જરૂર નથી
\item \textbf{હાઇ કોન્ટ્રાસ્ટ}: સાચા કાળા રંગો
\item \textbf{લવચીક}: વાંકી અને વળાંકવાળું બનાવી શકાય છે
\item \textbf{ઊર્જા કાર્યક્ષમ}: ઓછો પાવર વપરાશ
\end{itemize}

\textbf{એપ્લિકેશન:}
\begin{itemize}
\item \textbf{સ્માર્ટફોન}: OLED સ્ક્રીન
\item \textbf{ટીવી}: અલ્ટ્રા-થિન ડિસ્પ્લે
\item \textbf{વેરેબલ}: સ્માર્ટવોચ ડિસ્પ્લે
\end{itemize}
\end{solutionbox}

\begin{mnemonicbox}
\mnemonic{"SHFE" - Self-illuminating, High contrast, Flexible, Efficient}
\end{mnemonicbox}

\questionmarks{3(c) OR}{7}{AR/VR કોર ટેકનોલોજી સમજાવો અને તેની એપ્લિકેશનોની ચર્ચા કરો.}

\begin{solutionbox}
\textbf{જવાબ}:
\textbf{AR (ઓગમેન્ટેડ રિયાલિટી)} વાસ્તવિક વિશ્વ પર ડિજિટલ માહિતીને ઓવરલે કરે છે, જ્યારે \textbf{VR (વર્ચ્યુઅલ રિયાલિટી)} સંપૂર્ણપણે ઇમર્સિવ ડિજિટલ વાતાવરણ બનાવે છે.

\textbf{કોર ટેકનોલોજી:}
\begin{itemize}
\item \textbf{ડિસ્પ્લે સિસ્ટમ}: હેડ-માઉન્ટેડ ડિસ્પ્લે, સ્ક્રીન
\item \textbf{ટ્રેકિંગ સિસ્ટમ}: મોશન સેન્સર, કેમેરા
\item \textbf{પ્રોસેસિંગ યુનિટ}: GPU, સ્પેશિયલાઇઝ્ડ ચિપ્સ
\item \textbf{ઇનપુટ મેથડ}: કંટ્રોલર, જેસ્ચર રેકગ્નિશન
\end{itemize}

\textbf{કોષ્ટક: AR vs VR સરખામણી}

\begin{center}
\captionof{table}{AR vs VR સરખામણી}
\begin{tabulary}{\linewidth}{|L|L|L|}
\hline
\textbf{પાસું} & \textbf{AR} & \textbf{VR} \\ \hline
\textbf{વાસ્તવિકતા} & વાસ્તવિક વિશ્વ સાથે મિશ્રિત & સંપૂર્ણપણે વર્ચ્યુઅલ \\ \hline
\textbf{સાધનો} & સ્માર્ટફોન, AR ચશ્મા & VR હેડસેટ, કંટ્રોલર \\ \hline
\textbf{ઇમર્શન} & આંશિક & સંપૂર્ણ \\ \hline
\textbf{ગતિશીલતા} & મોબાઇલ ફ્રેન્ડલી & સ્થિર સેટઅપ \\ \hline
\end{tabulary}
\end{center}

\textbf{એપ્લિકેશન:}
\begin{itemize}
\item \textbf{AR}: ગેમિંગ (Pokemon Go), શિક્ષણ, નેવિગેશન, શોપિંગ
\item \textbf{VR}: મનોરંજન, ટ્રેનિંગ, આર્કિટેક્ચર, થેરાપી
\end{itemize}
\end{solutionbox}

\begin{mnemonicbox}
\mnemonic{"DTPI-GENT" - Display, Tracking, Processing, Input - Gaming, Education, Navigation, Training}
\end{mnemonicbox}

\questionmarks{4(a)}{3}{હોમ સોલર રૂફટોપ સિસ્ટમનો બ્લોક ડાયાગ્રામ દોરો.}

\begin{solutionbox}
\textbf{જવાબ}:

\textbf{બ્લોક ડાયાગ્રામ:}

\begin{center}
\begin{tikzpicture}[node distance=1.5cm]
    \node[gtu block] (solar) {Solar Panels};
    \node[gtu block, right=of solar] (inv) {Inverter};
    \node[gtu block, right=of inv] (load) {AC Load Panel};
    
    \node[gtu block, below=of inv] (batt) {Battery Storage};
    \node[gtu block, right=of batt] (grid) {Utility Grid Connection};
    
    \draw[gtu arrow] (solar) -- (inv);
    \draw[gtu arrow] (inv) -- (load);
    \draw[gtu arrow] (inv) -- (batt);
    \draw[gtu arrow] (batt) -- (inv);
    \draw[gtu arrow] (inv) -- (grid);
    \draw[gtu arrow] (grid) -- (inv);
\end{tikzpicture}
\captionof{figure}{હોમ સોલર રૂફટોપ સિસ્ટમ}
\end{center}

\textbf{ઘટકો:}
\begin{itemize}
\item \textbf{સોલર પેનલ્સ}: સૂર્યપ્રકાશને DC વીજળીમાં રૂપાંતરિત કરે છે
\item \textbf{ઇન્વર્ટર}: DC ને AC પાવરમાં રૂપાંતરિત કરે છે
\item \textbf{બેટરી સ્ટોરેજ}: વધારાની ઊર્જા સંગ્રહ કરે છે
\end{itemize}
\end{solutionbox}

\begin{mnemonicbox}
\mnemonic{"SIB" - Solar panels, Inverter, Battery}
\end{mnemonicbox}

\questionmarks{4(b)}{4}{OFET નો કાર્યસિદ્ધાંત સમજાવો.}

\begin{solutionbox}
\textbf{જવાબ}:
\textbf{OFET (ઓર્ગેનિક ફીલ્ડ ઇફેક્ટ ટ્રાન્ઝિસ્ટર)} કરંટ ફ્લોને કંટ્રોલ કરવા માટે ઓર્ગેનિક સેમિકંડક્ટરનો ઉપયોગ કરે છે.

\textbf{કાર્યસિદ્ધાંત:}
\begin{enumerate}
\item \textbf{ગેટ વોલ્ટેજ}: લાગુ વોલ્ટેજ ઇલેક્ટ્રિક ફીલ્ડ બનાવે છે
\item \textbf{ચેનલ ફોર્મેશન}: ઇલેક્ટ્રિક ફીલ્ડ કંડક્ટિવિટી મોડ્યુલેટ કરે છે
\item \textbf{કરંટ કંટ્રોલ}: સોર્સ-ડ્રેન કરંટ ગેટ દ્વારા કંટ્રોલ થાય છે
\item \textbf{સ્વિચિંગ}: ડિજિટલ એપ્લિકેશન માટે ON/OFF સ્ટેટ
\end{enumerate}

\textbf{માળખું:}
\begin{itemize}
\item \textbf{સોર્સ/ડ્રેન}: કરંટ ઇન્જેક્શન પોઇન્ટ
\item \textbf{ગેટ}: કંટ્રોલ ઇલેક્ટ્રોડ
\item \textbf{ઓર્ગેનિક લેયર}: એક્ટિવ સેમિકંડક્ટર મટેરિયલ
\end{itemize}
\end{solutionbox}

\begin{mnemonicbox}
\mnemonic{"GCCS" - Gate voltage, Channel, Current, Switching}
\end{mnemonicbox}

\questionmarks{4(c)}{7}{વિવિધ મશીન લર્નિંગ ટૂલ્સની યાદી બનાવો. કોઈપણ બેની ટૂંકમાં ચર્ચા કરો.}

\begin{solutionbox}
\textbf{જવાબ}:
\textbf{મશીન લર્નિંગ ટૂલ્સ:}
\begin{itemize}
\item \textbf{TensorFlow}: ગૂગલનું ML ફ્રેમવર્ક
\item \textbf{PyTorch}: ફેસબુકની ડીપ લર્નિંગ લાઇબ્રેરી
\item \textbf{Scikit-learn}: પાયથોન ML લાઇબ્રેરી
\item \textbf{Keras}: હાઇ-લેવલ ન્યુરલ નેટવર્ક API
\item \textbf{Machine Learning for Kids}: શૈક્ષણિક પ્લેટફોર્મ
\item \textbf{Scratch}: ML માટે વિઝ્યુઅલ પ્રોગ્રામિંગ
\end{itemize}

\textbf{કોષ્ટક: ML ટૂલ્સ સરખામણી}

\begin{center}
\captionof{table}{ML ટૂલ્સ સરખામણી}
\begin{tabulary}{\linewidth}{|L|L|L|L|}
\hline
\textbf{ટૂલ} & \textbf{પ્રકાર} & \textbf{સર્વોત્તમ} & \textbf{મુશ્કેલી} \\ \hline
\textbf{TensorFlow} & Deep Learning & જટિલ મોડેલ & એડવાન્સ \\ \hline
\textbf{Scikit-learn} & General ML & બિગિનર્સ & સરળ \\ \hline
\end{tabulary}
\end{center}

\textbf{વિગતવાર ચર્ચા:}
\begin{itemize}
\item \textbf{TensorFlow}: ડીપ લર્નિંગ અને ન્યુરલ નેટવર્ક. મોટા પાયે ML અને પ્રોડક્શન માટે સારું છે.
\item \textbf{Scikit-learn}: સામાન્ય અલગોરિધમ જેમ કે વર્ગીકરણ, રીગ્રેસન. ઉપયોગમાં સરળ અને સારી રીતે ડોક્યુમેન્ટેડ.
\end{itemize}
\end{solutionbox}

\begin{mnemonicbox}
\mnemonic{"TPSKMS" - TensorFlow, PyTorch, Scikit, Keras, ML4Kids, Scratch}
\end{mnemonicbox}

\questionmarks{4(a) OR}{3}{રિન્યુએબલ એનર્જીમાં ઇમર્જિંગ ટ્રેન્ડ્સને સંક્ષિપ્તમાં સમજાવો.}

\begin{solutionbox}
\textbf{જવાબ}:

\textbf{ઉભરતા વલણો:}
\begin{itemize}
\item \textbf{ફ્લોટિંગ સોલર}: પાણીના શરીર પર સોલર પેનલ
\item \textbf{પેરોવ્સકાઇટ સેલ}: આગામી પેઢીની સોલર ટેકનોલોજી
\item \textbf{ગ્રીન હાઇડ્રોજન}: રિન્યુએબલ સ્ત્રોતોમાંથી સ્વચ્છ ઇંધન
\end{itemize}

\textbf{ફાયદા:}
\begin{itemize}
\item \textbf{વધારે કાર્યક્ષમતા}: બહેતર ઊર્જા રૂપાંતરણ
\item \textbf{કિંમત ઘટાડો}: સસ્તી રિન્યુએબલ એનર્જી
\end{itemize}
\end{solutionbox}

\begin{mnemonicbox}
\mnemonic{"FPG" - Floating solar, Perovskite, Green hydrogen}
\end{mnemonicbox}

\questionmarks{4(b) OR}{4}{સંપૂર્ણ સ્વરૂપો આપો: AR, OLED, OPVD, OFET}

\begin{solutionbox}
\textbf{જવાબ}:

\begin{center}
\captionof{table}{ટેકનોલોજી સંપૂર્ણ સ્વરૂપો}
\begin{tabulary}{\linewidth}{|L|L|L|}
\hline
\textbf{સંક્ષેપ} & \textbf{સંપૂર્ણ સ્વરૂપ} & \textbf{ટેકનોલોજી વિસ્તાર} \\ \hline
\textbf{AR} & Augmented Reality & મિક્સ્ડ રિયાલિટી \\ \hline
\textbf{OLED} & Organic Light Emitting Diode & ડિસ્પ્લે ટેકનોલોજી \\ \hline
\textbf{OPVD} & Organic Photovoltaic Device & સોલર સેલ \\ \hline
\textbf{OFET} & Organic Field Effect Transistor & ઇલેક્ટ્રોનિક્સ \\ \hline
\end{tabulary}
\end{center}
\end{solutionbox}

\begin{mnemonicbox}
\mnemonic{"AOOO" - AR, OLED, OPVD, OFET}
\end{mnemonicbox}

\questionmarks{4(c) OR}{7}{રાસ્પબેરી પાઈનો બ્લોક ડાયાગ્રામ સમજાવો.}

\begin{solutionbox}
\textbf{જવાબ}:

\textbf{બ્લોક ડાયાગ્રામ:}

\begin{center}
\begin{tikzpicture}[node distance=1.5cm]
    \node[gtu block, minimum width=3cm] (arm) {ARM Processor};
    \node[gtu block, right=of arm] (ram) {RAM Memory};
    \node[gtu block, left=of arm] (power) {Power Supply};
    
    \node[gtu block, above=of arm] (gpio) {GPIO Pins};
    \node[gtu block, below=of arm] (sd) {MicroSD Card};
    
    \node[gtu block, above right=of arm] (usb) {USB Ports};
    \node[gtu block, below right=of arm] (eth) {Ethernet};
    \node[gtu block, above left=of arm] (hdmi) {HDMI Output};
    
    \draw[gtu arrow] (arm) -- (ram);
    \draw[gtu arrow] (power) -- (arm);
    \draw[gtu arrow] (arm) -- (gpio);
    \draw[gtu arrow] (sd) -- (arm);
    \draw[gtu arrow] (arm) -- (usb);
    \draw[gtu arrow] (arm) -- (eth);
    \draw[gtu arrow] (arm) -- (hdmi);
\end{tikzpicture}
\captionof{figure}{રાસ્પબેરી પાઈ બ્લોક ડાયાગ્રામ}
\end{center}

\textbf{ઘટકો:}
\begin{itemize}
\item \textbf{ARM પ્રોસેસર}: સેન્ટ્રલ પ્રોસેસિંગ યુનિટ (ક્વાડ-કોર)
\item \textbf{RAM મેમરી}: સિસ્ટમ મેમરી (1GB-8GB)
\item \textbf{GPIO પિન્સ}: સેન્સર/ઉપકરણોને ઇન્ટરફેસ કરવા માટે 40 પિન્સ
\item \textbf{USB પોર્ટ્સ}: પેરિફેરલ્સ કનેક્ટ કરે છે
\item \textbf{HDMI આઉટપુટ}: વીડિયો ડિસ્પ્લે કનેક્શન
\item \textbf{ઇથરનેટ પોર્ટ}: નેટવર્ક કનેક્ટિવિટી
\item \textbf{માઇક્રો SD કાર્ડ}: OS અને ડેટા માટે સ્ટોરેજ
\end{itemize}
\end{solutionbox}

\begin{mnemonicbox}
\mnemonic{"ARGC-EPMS" - ARM, RAM, GPIO, Connectivity - Ethernet, Power, MicroSD, Storage}
\end{mnemonicbox}

\questionmarks{5(a)}{3}{રાસ્પબેરી પાઈ સાથે LED ઇન્ટરફેસ કરો.}

\begin{solutionbox}
\textbf{જવાબ}:

\textbf{સર્કિટ કનેક્શન:}

\begin{center}
\begin{tikzpicture}[node distance=1.5cm]
    \node[gtu block] (pi) {Raspberry Pi (GPIO 18)};
    \node[coordinate, right=2cm of pi] (c1) {};
    \node[coordinate, right=1cm of c1] (c2) {};
    
    \draw (pi.east) -- (c1) to[R, l=220$\Omega$] (c2) to[led, l=LED] ++(2,0) node[ground] {};
\end{tikzpicture}
\captionof{figure}{LED ઇન્ટરફેસિંગ}
\end{center}

\textbf{Python Code:}
\begin{lstlisting}[language=Python]
import RPi.GPIO as GPIO
import time

GPIO.setmode(GPIO.BCM)
GPIO.setup(18, GPIO.OUT)

while True:
    GPIO.output(18, GPIO.HIGH)  # LED ON
    time.sleep(1)
    GPIO.output(18, GPIO.LOW)   # LED OFF
    time.sleep(1)
\end{lstlisting}
\end{solutionbox}

\begin{mnemonicbox}
\mnemonic{"GPIO-RC" - GPIO pin, Resistor, Code}
\end{mnemonicbox}

\questionmarks{5(b)}{4}{મશીન લર્નિંગ માટે Pandas પાયથોન લાઇબ્રેરી સમજાવો.}

\begin{solutionbox}
\textbf{જવાબ}:
\textbf{Pandas} એ ડેટા મેનિપ્યુલેશન અને એનાલિસિસ માટેની પાયથોન લાઇબ્રેરી છે, જે ML ડેટા પ્રીપ્રોસેસિંગ માટે આવશ્યક છે.

\textbf{મુખ્ય વિશેષતાઓ:}
\begin{itemize}
\item \textbf{DataFrame}: ટેબ્યુલર ડેટા સ્ટ્રક્ચર
\item \textbf{ડેટા ક્લીનિંગ}: ગુમ થયેલ વેલ્યુ, ડુપ્લિકેટ હેન્ડલ કરે છે
\item \textbf{ડેટા ઇમ્પોર્ટ}: CSV, Excel, JSON ફાઇલો વાંચે છે
\item \textbf{ડેટા એનાલિસિસ}: આંકડાકીય ઓપરેશન્સ, ગ્રુપિંગ
\end{itemize}

\textbf{ML એપ્લિકેશન:}
\begin{itemize}
\item \textbf{ડેટા પ્રીપ્રોસેસિંગ}: ડેટાસેટ સાફ અને તૈયાર કરે છે
\item \textbf{ફીચર એન્જિનિયરિંગ}: ડેટામાંથી નવી વિશેષતાઓ બનાવે છે
\item \textbf{ડેટા એક્સપ્લોરેશન}: ડેટા પેટર્ન સમજે છે
\end{itemize}

\textbf{સામાન્ય ફંક્શન્સ:}
\begin{lstlisting}[language=Python]
import pandas as pd
df = pd.read_csv('data.csv')    # ડેટા લોડ કરો
df.info()                       # ડેટા માહિતી
df.describe()                   # આંકડાકીય માહિતી
\end{lstlisting}
\end{solutionbox}

\begin{mnemonicbox}
\mnemonic{"DCIF" - DataFrame, Cleaning, Import, Functions}
\end{mnemonicbox}

\questionmarks{5(c)}{7}{મશીન લર્નિંગ તકનીકોના પ્રકારો સમજાવો: સુપરવાઇઝ્ડ, અનસુપરવાઇઝ્ડ અને રિઇન્ફોર્સમેન્ટ લર્નિંગ.}

\begin{solutionbox}
\textbf{જવાબ}:

\begin{center}
\captionof{table}{મશીન લર્નિંગ પ્રકારો}
\begin{tabulary}{\linewidth}{|L|L|L|L|}
\hline
\textbf{પ્રકાર} & \textbf{જરૂરી ડેટા} & \textbf{ધ્યેય} & \textbf{ઉદાહરણો} \\ \hline
\textbf{સુપરવાઇઝ્ડ} & લેબલ્ડ ડેટા & પરિણામોની આગાહી & ક્લાસિફિકેશન, રિગ્રેશન \\ \hline
\textbf{અનસુપરવાઇઝ્ડ} & અનલેબલ્ડ ડેટા & પેટર્ન શોધવું & ક્લસ્ટરિંગ, ડાઇમેન્શનલિટી રિડક્શન \\ \hline
\textbf{રિઇન્ફોર્સમેન્ટ} & રિવાર્ડ સિગ્નલ્સ & શ્રેષ્ઠ ક્રિયાઓ શીખવી & ગેમ પ્લેઇંગ, રોબોટિક્સ \\ \hline
\end{tabulary}
\end{center}

\textbf{આકૃતિ: ML લર્નિંગ પ્રક્રિયા}

\begin{center}
\begin{tikzpicture}[node distance=1.5cm]
    \node[gtu block] (data) {Data};
    \node[gtu decision, right=of data] (type) {Learning Type};
    
    \node[gtu block, right=of type] (unsup) {Unsupervised};
    \node[gtu block, above=of unsup] (sup) {Supervised};
    \node[gtu block, below=of unsup] (rl) {Reinforcement};
    
    \node[gtu block, right=of sup] (pred) {Prediction};
    \node[gtu block, right=of unsup] (pat) {Patterns};
    \node[gtu block, right=of rl] (pol) {Policy};
    
    \draw[gtu arrow] (data) -- (type);
    \draw[gtu arrow] (type) -- (sup);
    \draw[gtu arrow] (type) -- (unsup);
    \draw[gtu arrow] (type) -- (rl);
    \draw[gtu arrow] (sup) -- (pred);
    \draw[gtu arrow] (unsup) -- (pat);
    \draw[gtu arrow] (rl) -- (pol);
\end{tikzpicture}
\captionof{figure}{ML લર્નિંગ પ્રક્રિયા}
\end{center}

\textbf{વર્ણન:}
\begin{itemize}
\item \textbf{સુપરવાઇઝ્ડ લર્નિંગ}: ઇનપુટ-આઉટપુટ જોડીઓમાંથી શીખે છે. પ્રક્રિયા જાણીતા જવાબો સાથે ટ્રેનિંગ કરે છે. એપ્લિકેશન: ઇમેઇલ સ્પામ ડિટેક્શન.
\item \textbf{અનસુપરવાઇઝ્ડ લર્નિંગ}: ડેટામાં છુપાયેલા પેટર્ન શોધે છે. કોઈ ટાર્ગેટ વેરિએબલ નથી. એપ્લિકેશન: ગ્રાહક સેગમેન્ટેશન.
\item \textbf{રિઇન્ફોર્સમેન્ટ લર્નિંગ}: ટ્રાયલ અને એરર દ્વારા શીખે છે. પર્યાવરણ સાથે ઇન્ટરેક્ટ કરે છે. એપ્લિકેશન: ગેમ AI.
\end{itemize}
\end{solutionbox}

\begin{mnemonicbox}
\mnemonic{"SUR-PLR-CPD" - Supervised, Unsupervised, Reinforcement - Prediction, Learning, Rewards}
\end{mnemonicbox}

\questionmarks{5(a) OR}{3}{મશીન લર્નિંગ માટે NumPy પાયથોન લાઇબ્રેરી સમજાવો.}

\begin{solutionbox}
\textbf{જવાબ}:
\textbf{NumPy} એ પાયથોનમાં ન્યુમેરિકલ કમ્પ્યુટિંગ માટેની મૂળભૂત લાઇબ્રેરી છે, જે ML ઓપરેશન્સ માટે આવશ્યક છે.

\textbf{મુખ્ય વિશેષતાઓ:}
\begin{itemize}
\item \textbf{એરે}: મલ્ટિ-ડાઇમેન્શનલ એરે ઓબ્જેક્ટ
\item \textbf{મેથેમેટિકલ ફંક્શન્સ}: લિનિયર આલ્જેબ્રા ઓપરેશન્સ
\item \textbf{બ્રોડકાસ્ટિંગ}: અલગ સાઇઝના એરે પર ઓપરેશન્સ
\end{itemize}

\textbf{ML એપ્લિકેશન:}
\begin{itemize}
\item \textbf{ડેટા સ્ટોરેજ}: કાર્યક્ષમ ન્યુમેરિકલ ડેટા સ્ટોરેજ
\item \textbf{મેટ્રિક્સ ઓપરેશન્સ}: ન્યુરલ નેટવર્ક કમ્પ્યુટેશન્સ
\end{itemize}
\end{solutionbox}

\begin{mnemonicbox}
\mnemonic{"AMB" - Arrays, Mathematical functions, Broadcasting}
\end{mnemonicbox}

\questionmarks{5(b) OR}{4}{Raspberry Pi Imager નો ઉપયોગ કરીને SD કાર્ડ પર Raspberry Pi OS ઇન્સ્ટોલેશનના સ્ટેપ્સ લખો.}

\begin{solutionbox}
\textbf{જવાબ}:

\textbf{ઇન્સ્ટોલેશન સ્ટેપ્સ:}
\begin{enumerate}
\item \textbf{ડાઉનલોડ}: ઓફિશિયલ વેબસાઇટથી Raspberry Pi Imager ઇન્સ્ટોલ કરો
\item \textbf{SD કાર્ડ ઇન્સર્ટ}: કમ્પ્યુટરમાં SD કાર્ડ (16GB+) કનેક્ટ કરો
\item \textbf{OS સિલેક્ટ}: યાદીમાંથી Raspberry Pi OS પસંદ કરો
\item \textbf{સ્ટોરેજ સિલેક્ટ}: ટાર્ગેટ તરીકે SD કાર્ડ પસંદ કરો
\item \textbf{રાઇટ}: OS ને SD કાર્ડમાં ફ્લેશ કરવા માટે "Write" ક્લિક કરો
\item \textbf{ઇજેક્ટ}: પૂર્ણ થયા પછી SD કાર્ડને સુરક્ષિત રીતે કાઢો
\end{enumerate}

\textbf{પૂર્વ-ગોઠવણી વિકલ્પો:}
\begin{itemize}
\item \textbf{SSH એનેબલ}: રિમોટ એક્સેસ માટે
\item \textbf{યુઝરનેમ/પાસવર્ડ સેટ}: સુરક્ષા ક્રેડેન્શિયલ્સ
\item \textbf{Wi-Fi કોન્ફિગર}: નેટવર્ક સેટિંગ્સ
\end{itemize}
\end{solutionbox}

\begin{mnemonicbox}
\mnemonic{"DISWS-ESP" - Download, Insert, Select OS, Write, Storage - Enable SSH, Set credentials, Pre-configure}
\end{mnemonicbox}

\questionmarks{5(c) OR}{7}{Raspberry Pi સાથે Temperature અને humidity સેન્સર ઇન્ટરફેસ કરો અને તેના માટે Python પ્રોગ્રામ લખો.}

\begin{solutionbox}
\textbf{જવાબ}:

\textbf{સર્કિટ કનેક્શન:}

\begin{center}
\begin{tikzpicture}[node distance=1.5cm]
    \node[gtu block] (dht) {DHT22 Sensor};
    \node[gtu block, right=3cm of dht] (pi) {Raspberry Pi};
    
    \draw[thick] (dht.east) -- (pi.west) node[midway, above] {Data (GPIO 4)};
    \node[above=0.5cm of dht] (vcc) {VCC (3.3V)};
    \node[below=0.5cm of dht] (gnd) {GND};
    
    \draw (dht.north) -- (vcc);
    \draw (dht.south) -- (gnd);
\end{tikzpicture}
\captionof{figure}{DHT22 સેન્સર ઇન્ટરફેસિંગ}
\end{center}

\textbf{Python પ્રોગ્રામ:}
\begin{lstlisting}[language=Python]
import Adafruit_DHT
import time

# Sensor type and GPIO pin
sensor = Adafruit_DHT.DHT22
pin = 4

while True:
    humidity, temperature = Adafruit_DHT.read_retry(sensor, pin)
    if humidity is not None and temperature is not None:
        print(f'Temp={temperature:0.1f}*C  Humidity={humidity:0.1f}%')
    else:
        print('Failed to get reading. Try again!')
    time.sleep(2)
\end{lstlisting}
\end{solutionbox}

\begin{mnemonicbox}
\mnemonic{"DHT-Code" - Sensor, Pin, Read loop}
\end{mnemonicbox}

\end{document}
