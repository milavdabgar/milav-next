\documentclass{article}
% Adjust the relative path to point to the latex-templates directory

% content/resources/templates/preamble.tex
\usepackage[margin=0.6in]{geometry}
\author{Milav Dabgar}
\usepackage{amsmath,amssymb,amsthm}
\usepackage{booktabs}
\usepackage{multirow}
\usepackage{xcolor}
\usepackage{tcolorbox}
\tcbuselibrary{breakable,skins}
\usepackage[colorlinks=true,linkcolor=blue]{hyperref}
\usepackage{titlesec}
\usepackage{enumitem}
\usepackage{tikz}
\usepackage{pgfplots}
\usepackage{circuitikz}
\usepackage[version=4]{mhchem}
\usepackage{longtable}
\usepackage{array}
\usepackage{float}
\usepackage{caption}
\usepackage{listings}

\lstset{
  basicstyle=\small\ttfamily,
  breaklines=true,
  breakatwhitespace=false,
  postbreak=\mbox{\textcolor{red}{$\hookrightarrow$}\space},
  float=false,
  numbers=left,
  numberstyle=\tiny\color{gray},
  numbersep=10pt,
  xleftmargin=2em,
  keywordstyle=\color{blue},
  commentstyle=\color{green!60!black},
  stringstyle=\color{purple},
  backgroundcolor=\color{gray!5},
  showstringspaces=false,
  tabsize=2,
  captionpos=b,
  keepspaces=true,
  columns=flexible
}

\pgfplotsset{compat=1.18}
\usetikzlibrary{shapes,arrows,positioning,calc,patterns,decorations.pathmorphing,decorations.markings,arrows.meta}

% Color scheme
\definecolor{headcolor}{RGB}{0,102,204}
\definecolor{keycolor}{RGB}{220,20,60}
\definecolor{solutioncolor}{RGB}{34,139,34}
\definecolor{mnemoniccolor}{RGB}{148,0,211}
\definecolor{codecolor}{RGB}{0,0,100}

% Spacing
\setlength{\parskip}{3pt}
\setlist[itemize]{nosep}
\setlist[enumerate]{nosep}

% Title formatting
\titleformat{\section}{\Large\bfseries\color{headcolor}}{\thesection}{1em}{}
\titleformat{\subsection}{\large\bfseries\color{headcolor}}{\thesubsection}{1em}{}

% Pandoc tightlist compatibility
\providecommand{\tightlist}{%
  \setlength{\itemsep}{0pt}\setlength{\parskip}{0pt}}

% Pandoc longtable compatibility
\newcounter{none}
\def\thenone{}


% content/resources/templates/gujarati-boxes.tex
\usepackage{fontspec}
\usepackage{polyglossia}

% Set Gujarati as main language (document is primarily in Gujarati)
% Note: gloss-gujarati.ldf doesn't exist in polyglossia, but it will use hyphenation patterns
\setdefaultlanguage{gujarati}
\setotherlanguage{english}

% Configure Gujarati font properly
% Use Language=Default to prevent polyglossia from trying to add language-specific features
% that don't exist for Gujarati, which causes "empty feature" warnings
\newfontfamily\gujaratifont[Script=Gujarati,AutoFakeBold=2.5,AutoFakeSlant=0.3]{Noto Sans Gujarati}
\setmainfont[Script=Gujarati,AutoFakeBold=2.5,AutoFakeSlant=0.3]{Noto Sans Gujarati}
% Use Noto Sans Gujarati for monospace to support Gujarati in text
\setmonofont[Scale=0.9]{Noto Sans Gujarati}

% Configure English to use the same font
\newfontfamily\englishfont[Script=Gujarati,AutoFakeBold=2.5,AutoFakeSlant=0.3]{Noto Sans Gujarati}

% Translations for polyglossia
\gappto\captionsgujarati{
  \renewcommand{\tablename}{કોષ્ટક}
  \renewcommand{\figurename}{આકૃતિ}
}

% Helper for TikZ nodes to ensure Gujarati font
\newcommand{\gu}[1]{{\gujaratifont #1}}

% Custom environments
\newtcolorbox{solutionbox}{
    breakable,
    enhanced,
    colback=solutioncolor!5!white,
    colframe=solutioncolor!75!black,
    fonttitle=\bfseries,
    title=જવાબ
}

\newtcolorbox{solutionboxnobreak}{
 colback=solutioncolor!5!white,
 colframe=solutioncolor!75!black,
 fonttitle=\bfseries,
 title=જવાબ
}

\newtcolorbox{keyformula}{
 breakable,
 enhanced,
 colback=keycolor!5!white,
 colframe=keycolor!75!black,
 fonttitle=\bfseries,
 title=રાસાયણિક સમીકરણ/સૂત્ર
}

\newtcolorbox{mnemonicbox}{
 breakable,
 enhanced,
 colback=mnemoniccolor!5!white,
 colframe=mnemoniccolor!75!black,
 fonttitle=\bfseries,
 title=મેમરી ટ્રીક
}


% Custom commands for GTU solutions
% This file defines semantic commands for consistent formatting

% Question command with automatic formatting
\newcommand{\question}[2]{%
  \section*{Question #1}%
  \textbf{#2}%
}

% OR question variant
\newcommand{\questionor}[2]{%
  \section*{Question #1 OR}%
  \textbf{#2}%
}

% Proper table environment with caption
\newenvironment{answertable}[1]{%
  \begin{table}[htbp]
  \centering
  \caption{#1}
}{%
  \end{table}
}

% Proper figure environment for diagrams
\newenvironment{answerdiagram}[1]{%
  \begin{figure}[htbp]
  \centering
  \caption{#1}
}{%
  \end{figure}
}

% Semantic markup for key terms
\newcommand{\keyword}[1]{\textbf{#1}}
\newcommand{\code}[1]{\texttt{#1}}
\newcommand{\classname}[1]{\texttt{#1}}
\newcommand{\methodname}[1]{\texttt{#1}}

% Proper quotation marks
\newcommand{\mnemonic}[1]{``#1''}


\title{Renewable Energy \& Emerging Trends in Electronics (4361106) - Winter 2024 Solution (Gujarati)}
\date{November 25, 2024}

\begin{document}
\maketitle

\questionmarks{1(અ)}{3}{વિવિધ પ્રકારના નવીનીકરણીય ઉર્જા સ્રોતોની યાદી બનાવો અને કોઈપણ એકને વિગતવાર સમજાવો.}

\begin{solutionbox}
\textbf{કોષ્ટક: નવીનીકરણીય ઉર્જા સ્રોતોના પ્રકારો}
\begin{center}
\captionof{table}{નવીનીકરણીય ઉર્જા સ્રોતોના પ્રકારો}
\begin{tabulary}{\linewidth}{|L|L|L|}
\hline
\textbf{પ્રકાર} & \textbf{સ્રોત} & \textbf{ઉપયોગ} \\ \hline
સૌર & સૂર્યનું કિરણોત્સર્ગ & સોલાર પેનલ, હીટિંગ \\ \hline
પવન & હવાની હલનચલન & વિન્ડ ટર્બાઇન \\ \hline
જલવિદ્યુત & વહેતું પાણી & ડેમ, ટર્બાઇન \\ \hline
બાયોમાસ & કાર્બનિક પદાર્થ & બાયોફ્યુઅલ, હીટિંગ \\ \hline
ભૂઉષ્મીય & પૃથ્વીની ગરમી & પાવર પ્લાન્ટ, હીટિંગ \\ \hline
\end{tabulary}
\end{center}

\textbf{સૌર ઉર્જા સમજૂતી}:
\begin{itemize}
    \item \keyword{ફોટોવોલ્ટેઇક અસર}: સિલિકોન સેલ વાપરીને સૂર્યપ્રકાશને સીધો વીજળીમાં ફેરવે છે.
    \item \keyword{ફાયદાઓ}: સ્વચ્છ, વિપુલ, નવીનીકરણીય.
    \item \keyword{ઉપયોગો}: છત પરની સિસ્ટમ, સોલાર ફાર્મ.
\end{itemize}
\end{solutionbox}

\begin{mnemonicbox}
\mnemonic{SWHBG - સૂર્ય વિજય હાંસલ કરે ભલાઈથી જઈને}
\end{mnemonicbox}

\questionmarks{1(બ)}{4}{વિવિધ પ્રકારના સોલાર સેલની યાદી બનાવો અને કોઈપણ બેને સમજાવો.}

\begin{solutionbox}
\textbf{કોષ્ટક: સોલાર સેલના પ્રકારો}
\begin{center}
\captionof{table}{સોલાર સેલના પ્રકારો}
\begin{tabulary}{\linewidth}{|L|L|L|L|}
\hline
\textbf{પ્રકાર} & \textbf{કાર્યક્ષમતા} & \textbf{કિંમત} & \textbf{ઉપયોગ} \\ \hline
સિલિકોન & 15-20\% & મધ્યમ & રહેણાંક \\ \hline
મોનોક્રિસ્ટેલાઇન & 18-22\% & ઊંચી & પ્રીમિયમ સિસ્ટમ \\ \hline
પોલીક્રિસ્ટેલાઇન & 15-17\% & ઓછી & બજેટ સિસ્ટમ \\ \hline
થિન ફિલ્મ & 10-12\% & ખૂબ ઓછી & મોટા ઇન્સ્ટોલેશન \\ \hline
એમોર્ફસ સિલિકોન & 6-8\% & ઓછી & નાના ઉપકરણો \\ \hline
\end{tabulary}
\end{center}

\textbf{મોનોક્રિસ્ટેલાઇન સિલિકોન}:
\begin{itemize}
    \item \keyword{બંધારણ}: એકસાર ક્રિસ્ટલ બંધારણ સાથે એકસમાન દેખાવ.
    \item \keyword{કાર્યક્ષમતા}: સિલિકોન સેલમાં સૌથી વધુ (18-22\%).
\end{itemize}

\textbf{પોલીક્રિસ્ટેલાઇન સિલિકોન}:
\begin{itemize}
    \item \keyword{બંધારણ}: નીલા ડાઘવાળા દેખાવ સાથે બહુવિધ ક્રિસ્ટલ.
    \item \keyword{કિંમત}: મોનોક્રિસ્ટેલાઇન કરતાં ઓછી ઉત્પાદન કિંમત.
\end{itemize}
\end{solutionbox}

\begin{mnemonicbox}
\mnemonic{મારા પોલી થિન એમ્પ - મોસ્ટ પોપ્યુલર ટાઇપ્સ અવેઇલેબલ}
\end{mnemonicbox}

\questionmarks{1(ક)}{7}{હોમ સોલાર રૂફટોપ સિસ્ટમનો બ્લોક ડાયાગ્રામ દોરો અને સમજાવો.}

\begin{solutionbox}
\begin{center}
\begin{tikzpicture}[node distance=2cm, auto]
    \node [gtu block] (solar) {સોલાર પેનલ\\(PV Array)};
    \node [gtu block, below of=solar] (inverter) {ઇન્વર્ટર\\(DC to AC)};
    \node [gtu block, below of=inverter] (meter) {મીટર\\(Bidirectional)};
    
    \node [gtu block, below left=2cm and 1cm of meter] (load) {ઘરનો લોડ};
    \node [gtu block, below right=2cm and 1cm of meter] (grid) {ગ્રિડ\\કનેક્શન};

    \path [gtu arrow] (solar) -- node[right] {DC Power} (inverter);
    \path [gtu arrow] (inverter) -- node[right] {AC Power} (meter);
    \path [gtu arrow] (meter) -| (load);
    \path [gtu arrow] (meter) -| (grid);
    \path [gtu arrow, dashed] (grid) |- (meter);
\end{tikzpicture}
\captionof{figure}{હોમ સોલાર રૂફટોપ સિસ્ટમ}
\end{center}

\textbf{ઘટકોની સમજૂતી}:
\begin{itemize}
    \item \keyword{સોલાર પેનલ}: ફોટોવોલ્ટેઇક અસર વાપરીને સૂર્યપ્રકાશને DC વીજળીમાં ફેરવે છે.
    \item \keyword{ઇન્વર્ટર}: ઘરના ઉપયોગ માટે DC પાવરને AC પાવરમાં ફેરવે છે.
    \item \keyword{દ્વિદિશીય મીટર}: પાવર વપરાશ અને ગ્રિડમાં ફીડ થતી વધારાની પાવર માપે છે.
    \item \keyword{ઘરનો લોડ}: વિદ્યુત ઉપકરણો અને ડિવાઇસ.
    \item \keyword{ગ્રિડ કનેક્શન}: બેકઅપ અને વધારાની પાવર વેચવા માટે યુટિલિટી ગ્રિડ સાથે જોડાય છે.
\end{itemize}
\end{solutionbox}

\begin{mnemonicbox}
\mnemonic{સોલાર ઇન્વર્ટર મીટર હોમ ગ્રિડ - સિમ્પલ ઇન્સ્ટોલેશન મેક્સ હેપ્પી જનરેશન}
\end{mnemonicbox}

\questionmarks{1(ક અથવા)}{7}{સૌર ફોટોવોલ્ટેઇક અસર અને ફોટોવોલ્ટેઇક રૂપાંતરનો સિદ્ધાંત આકૃતિ સાથે સમજાવો.}

\begin{solutionbox}
\begin{center}
\begin{tikzpicture}[node distance=1.5cm, auto]
    % Draw layers
    \fill[blue!10] (0, 2) rectangle (5, 3) node[pos=0.5] {N-Type Layer (Phosphorus)};
    \fill[red!10] (0, 0) rectangle (5, 2) node[pos=0.5] {P-Type Layer (Boron)};
    \draw (0,0) rectangle (5,3);
    \draw [dashed] (0,2) -- (5,2) node[right] {P-N Junction};
    
    % Sunlight
    \foreach \x in {1, 2.5, 4}
        \draw [->, decorate, decoration={snake}, thick, orange] (\x, 4.5) -- (\x, 3.2);
    \node at (2.5, 4.8) {સૂર્યપ્રકાશ (Photons)};
    
    % Circuit
    \draw [thick] (0, 2.5) -- (-1, 2.5) -- (-1, -1) -- (2.5, -1);
    \draw [thick] (0, 0.5) -- (-0.5, 0.5) -- (-0.5, -0.5) -- (2.5, -0.5);
    
    \node [gtu block, minimum width=2cm] at (2.5, -0.75) (load) {બાહ્ય\\સર્કિટ};
    \draw [thick] (load) -| (6, 0.5) -- (5, 0.5);
    \draw [thick] (load) -| (6, 2.5) -- (5, 2.5);
    
    \node at (5.5, 2.8) {(-)};
    \node at (5.5, 0.2) {(+)};

\end{tikzpicture}
\captionof{figure}{ફોટોવોલ્ટેઇક અસર}
\end{center}

\textbf{ફોટોવોલ્ટેઇક અસર પ્રક્રિયા}:
\begin{itemize}
    \item \keyword{ફોટોન શોષણ}: સૌર ફોટોન સિલિકોન અણુઓ સાથે ટકરાય છે.
    \item \keyword{ઇલેક્ટ્રોન ઉત્તેજના}: ઇલેક્ટ્રોન ઊર્જા મેળવે છે અને કન્ડક્શન બેન્ડમાં જાય છે.
    \item \keyword{ચાર્જ વિભાજન}: P-N જંકશન વિદ્યુત ક્ષેત્ર બનાવે છે.
    \item \keyword{કરંટ પ્રવાહ}: ઇલેક્ટ્રોન બાહ્ય સર્કિટ દ્વારા વહે છે.
\end{itemize}
\end{solutionbox}

\begin{mnemonicbox}
\mnemonic{ફોટોન્સ પુશ ઇલેક્ટ્રોન્સ પાસ્ટ જંકશન - પાવર પ્રોડક્શન પરફેક્ટલી પ્લાન્ડ}
\end{mnemonicbox}

\questionmarks{2(અ)}{3}{નેનો ટેકનોલોજી શું છે? તેની એપ્લિકેશનોની સૂચિ બનાવો.}

\begin{solutionbox}
\textbf{વ્યાખ્યા}: નેનો ટેકનોલોજી એ પરમાણુ અને આણવિક સ્તરે (1-100 નેનોમીટર) પદાર્થની હેરફેર છે.

\textbf{કોષ્ટક: નેનો ટેકનોલોજીના ઉપયોગો}
\begin{center}
\captionof{table}{નેનો ટેકનોલોજીના ઉપયોગો}
\begin{tabulary}{\linewidth}{|L|L|L|}
\hline
\textbf{ક્ષેત્ર} & \textbf{ઉપયોગ} & \textbf{ફાયદો} \\ \hline
ઇલેક્ટ્રોનિક્સ & ટ્રાન્ઝિસ્ટર, મેમોરી & લઘુકરણ \\ \hline
દવા & ડ્રગ ડિલિવરી, ઇમેજિંગ & લક્ષિત સારવાર \\ \hline
ઊર્જા & સોલાર સેલ, બેટરી & ઉચ્ચ કાર્યક્ષમતા \\ \hline
સામગ્રી & કોમ્પોઝિટ, કોટિંગ & વધારેલા ગુણધર્મો \\ \hline
પર્યાવરણ & પાણીની શુદ્ધિકરણ & સ્વચ્છ તકનીક \\ \hline
\end{tabulary}
\end{center}
\end{solutionbox}

\begin{mnemonicbox}
\mnemonic{નેનો મેક્સ એવરીથિંગ મોર એફિશિયન્ટ}
\end{mnemonicbox}

\questionmarks{2(બ)}{4}{વિવિધ પ્રકારની EV ટેકનોલોજીની યાદી બનાવો અને કોઈપણ બેને સમજાવો.}

\begin{solutionbox}
\textbf{કોષ્ટક: EV ટેકનોલોજીના પ્રકારો}
\begin{center}
\captionof{table}{EV ટેકનોલોજીના પ્રકારો}
\begin{tabulary}{\linewidth}{|L|L|L|L|}
\hline
\textbf{પ્રકાર} & \textbf{પૂરું નામ} & \textbf{પાવર સ્રોત} & \textbf{રેન્જ} \\ \hline
BEV & બેટરી ઇલેક્ટ્રિક વ્હિકલ & માત્ર બેટરી & 150-400 કિમી \\ \hline
HEV & હાઇબ્રિડ ઇલેક્ટ્રિક વ્હિકલ & એન્જિન + બેટરી & 600+ કિમી \\ \hline
PHEV & પ્લગ-ઇન હાઇબ્રિડ ઇલેક્ટ્રિક & એન્જિન + બેટરી & 50-80 કિમી ઇલેક્ટ્રિક \\ \hline
FCEV & ફ્યુઅલ સેલ ઇલેક્ટ્રિક વ્હિકલ & હાઇડ્રોજન ફ્યુઅલ સેલ & 400-600 કિમી \\ \hline
\end{tabulary}
\end{center}

\textbf{બેટરી ઇલેક્ટ્રિક વ્હિકલ (BEV)}: શૂન્ય ઉત્સર્જન સાથે સંપૂર્ણ ઇલેક્ટ્રિક ડ્રાઇવ.
\\
\textbf{હાઇબ્રિડ ઇલેક્ટ્રિક વ્હિકલ (HEV)}: ઇન્ટરનલ કમ્બશન એન્જિન + ઇલેક્ટ્રિક મોટર; રિજનરેટિવ બ્રેકિંગ.
\end{solutionbox}

\begin{mnemonicbox}
\mnemonic{બિગ હાઇબ્રિડ પ્લગ ફ્યુઅલ - બેટર ટ્રાન્સપોર્ટેશન ઓપ્શન્સ}
\end{mnemonicbox}

\questionmarks{2(ક)}{7}{ડ્રોન અને તેના મુખ્ય ઘટકોના બ્લોક ડાયાગ્રામનું વર્ણન કરો.}

\begin{solutionbox}
\begin{center}
\begin{tikzpicture}[node distance=2cm, auto]
    \node [gtu block] (fc) {ફ્લાઇટ કંટ્રોલર\\(Microprocessor)};
    
    \node [gtu block, above left=1.5cm of fc] (cam) {કેમેરા};
    \node [gtu block, above right=1.5cm of fc] (gps) {GPS મોડ્યુલ};
    
    \node [gtu block, below left=1.5cm of fc] (motors) {મોટર અને\\પ્રોપેલર};
    \node [gtu block, below right=1.5cm of fc] (sensors) {સેન્સર્સ\\(Gyro)};
    
    \node [gtu block, below=1.5cm of motors] (batt) {બેટરી પેક};
    \node [gtu block, below=1.5cm of sensors] (rx) {ટ્રાન્સમિટર/\\રિસીવર};

    \path [gtu arrow] (cam) -- (fc);
    \path [gtu arrow] (gps) -- (fc);
    \path [gtu arrow] (fc) -- (motors);
    \path [gtu arrow] (fc) -- (sensors);
    \path [gtu arrow] (sensors) -- (fc);
    \path [gtu arrow] (batt) -- (motors);
    \path [gtu arrow] (rx) -- (fc);
    \path [gtu arrow, dashed] (fc) -- (rx);
\end{tikzpicture}
\captionof{figure}{ડ્રોન બ્લોક ડાયાગ્રામ}
\end{center}

\textbf{મુખ્ય ઘટકો}: \keyword{ફ્લાઇટ કંટ્રોલર}, \keyword{મોટર અને પ્રોપેલર}, \keyword{સેન્સર પેકેજ} (જાયરોસ્કોપ, એક્સેલેરોમીટર), \keyword{પાવર સિસ્ટમ} (LiPo બેટરી), \keyword{કમ્યુનિકેશન} (ટ્રાન્સમિટર/ રિસીવર).
\end{solutionbox}

\begin{mnemonicbox}
\mnemonic{ફ્લાઇંગ કંટ્રોલર્સ મોટર સેન્સર્સ પાવર કમ્યુનિકેશન - ડ્રોન્સ ફ્લાઇ પરફેક્ટલી}
\end{mnemonicbox}

\questionmarks{2(અ અથવા)}{3}{UAV શું છે? તેની એપ્લિકેશનોની યાદી બનાવો.}

\begin{solutionbox}
\textbf{વ્યાખ્યા}: UAV (અનમેન્ડ એરિયલ વ્હિકલ) એ એવું વિમાન છે જે બોર્ડ પર માનવ પાઇલટ વિના ચલાવવામાં આવે છે.
\textbf{ઉપયોગો}: કૃષિ, સુરક્ષા, ડિલિવરી, ફોટોગ્રાફી, નિરીક્ષણ.
\end{solutionbox}

\begin{mnemonicbox}
\mnemonic{અનમેન્ડ એરક્રાફ્ટ વર્સેટાઇલ - એપ્લિકેશન્સ આર વાસ્ટ}
\end{mnemonicbox}

\questionmarks{2(બ અથવા)}{4}{વિવિધ પ્રકારના EV ઊર્જા સ્રોતોની યાદી બનાવો અને કોઈપણ બેને સમજાવો.}

\begin{solutionbox}
\textbf{કોષ્ટક: EV ઊર્જા સ્રોતો}
\begin{center}
\captionof{table}{EV ઊર્જા સ્રોતો}
\begin{tabulary}{\linewidth}{|L|L|L|L|}
\hline
\textbf{પ્રકાર} & \textbf{ટેકનોલોજી} & \textbf{સંગ્રહ} & \textbf{કાર્યક્ષમતા} \\ \hline
બેટરી & લિથિયમ-આયન & રાસાયણિક & 90-95\% \\ \hline
ફ્યુઅલ સેલ & હાઇડ્રોજન & રાસાયણિક & 50-60\% \\ \hline
અલ્ટ્રાકેપેસિટર & ઇલેક્ટ્રિક ફિલ્ડ & વિદ્યુત & 95\%+ \\ \hline
ફ્લાયવ્હીલ & ગતિ ઊર્જા & યાંત્રિક & 85-90\% \\ \hline
\end{tabulary}
\end{center}
\end{solutionbox}

\begin{mnemonicbox}
\mnemonic{બેટરી ફ્યુઅલ અલ્ટ્રા ફ્લાઇ રિજન - એનર્જી સોર્સીસ ઇનેબલ વ્હિકલ્સ}
\end{mnemonicbox}

\questionmarks{2(ક અથવા)}{7}{વિવિધ પ્રકારની સ્માર્ટ સિસ્ટમ્સની યાદી બનાવો. કોઈપણ 2 સ્માર્ટ સિસ્ટમોને આકૃતિ સાથે સમજાવો.}

\begin{solutionbox}
\textbf{પ્રકારો}: સ્માર્ટ હોમ્સ, સ્માર્ટ કાર્સ, સ્માર્ટ સિટી, સ્માર્ટ ગ્રિડ, સ્માર્ટ હેલ્થ.

\begin{center}
\begin{tikzpicture}[node distance=2cm, auto]
    % Smart Street Light
    \node [gtu block] (micro) {માઇક્રોકંટ્રોલર};
    \node [gtu block, above left=1.5cm of micro] (motion) {મોશન સેન્સર};
    \node [gtu block, above right=1.5cm of micro] (light) {લાઇટ સેન્સર};
    \node [gtu block, below left=1.5cm of micro] (led) {LED સ્ટ્રીટ\\લાઇટ};
    \node [gtu block, below right=1.5cm of micro] (wireless) {વાયરલેસ\\કમ્યુનિકેશન};
    
    \path [gtu arrow] (motion) -- (micro);
    \path [gtu arrow] (light) -- (micro);
    \path [gtu arrow] (micro) -- (led);
    \path [gtu arrow] (micro) -- (wireless);
\end{tikzpicture}
\captionof{figure}{સ્માર્ટ સ્ટ્રીટ લાઇટ સિસ્ટમ}
\end{center}

\begin{center}
\begin{tikzpicture}[node distance=2cm, auto]
    % Smart Water Pollution
    \node [gtu block] (logger) {ડેટા લોગર\\(માઇક્રોકંટ્રોલર)};
    \node [gtu block, above left=1.5cm of logger] (ph) {pH સેન્સર};
    \node [gtu block, above right=1.5cm of logger] (temp) {Temp સેન્સર};
    \node [gtu block, below left=1.5cm of logger] (comms) {GSM/WiFi};
    \node [gtu block, below right=1.5cm of logger] (cloud) {Cloud DB};
    
    \path [gtu arrow] (ph) -- (logger);
    \path [gtu arrow] (temp) -- (logger);
    \path [gtu arrow] (logger) -- (comms);
    \path [gtu arrow] (comms) -- (cloud);
\end{tikzpicture}
\captionof{figure}{સ્માર્ટ વોટર પોલ્યુશન મોનિટરિંગ}
\end{center}
\end{solutionbox}

\questionmarks{3(અ)}{3}{સ્માર્ટ સ્ટ્રીટ લાઇટ કંટ્રોલ અને મોનિટરિંગ સિસ્ટમનો બ્લોક ડાયાગ્રામ દોરો.}

\begin{solutionbox}
\begin{center}
\begin{tikzpicture}[node distance=1.5cm, auto]
    \node [gtu block] (sensors) {સેન્સર્સ\\(PIR, LDR)};
    \node [gtu block, below=1cm of sensors] (micro) {માઇક્રોકંટ્રોલર\\(Arduino)};
    \node [gtu block, below left=1.5cm of micro] (driver) {LED ડ્રાઇવર};
    \node [gtu block, below right=1.5cm of micro] (wifi) {WiFi/GSM\\મોડ્યુલ};
    
    \node [gtu block, below=1cm of driver] (led) {LED સ્ટ્રીટ\\લાઇટ};
    \node [gtu block, below=1cm of wifi] (cloud) {Cloud સર્વર};
    
    \path [gtu arrow] (sensors) -- (micro);
    \path [gtu arrow] (micro) -- (driver);
    \path [gtu arrow] (micro) -- (wifi);
    \path [gtu arrow] (driver) -- (led);
    \path [gtu arrow] (wifi) -- (cloud);
\end{tikzpicture}
\captionof{figure}{સ્માર્ટ સ્ટ્રીટ લાઇટ કંટ્રોલ}
\end{center}
\end{solutionbox}

\questionmarks{3(બ)}{4}{પહેરી શકાય તેવી આરોગ્ય નિરીક્ષણ સિસ્ટમનો બ્લોક ડાયાગ્રામ દોરો અને સમજાવો.}

\begin{solutionbox}
\begin{center}
\begin{tikzpicture}[node distance=1.8cm, auto]
    \node [gtu block] (micro) {માઇક્રોપ્રોસેસર\\(પ્રોસેસિંગ)};
    \node [gtu block, above left=1.5cm of micro] (hr) {હાર્ટ રેટ\\સેન્સર};
    \node [gtu block, above right=1.5cm of micro] (temp) {Temperature\\સેન્સર};
    
    \node [gtu block, below left=1.5cm of micro] (disp) {ડિસ્પ્લે\\(OLED)};
    \node [gtu block, below right=1.5cm of micro] (bt) {બ્લૂટૂથ};
    \node [gtu block, below=1cm of bt] (phone) {સ્માર્ટફોન\\એપ};
    
    \path [gtu arrow] (hr) -- (micro);
    \path [gtu arrow] (temp) -- (micro);
    \path [gtu arrow] (micro) -- (disp);
    \path [gtu arrow] (micro) -- (bt);
    \path [gtu arrow] (bt) -- (phone);
\end{tikzpicture}
\captionof{figure}{વેરેબલ હેલ્થ વોચ}
\end{center}
\end{solutionbox}

\questionmarks{3(ક)}{7}{બાયોમેટ્રિક સિસ્ટમ્સ અને તેમના મૂળભૂત બ્લોક ડાયાગ્રામને સમજાવો.}

\begin{solutionbox}
\begin{center}
\begin{tikzpicture}[node distance=1.5cm, auto]
    \node [gtu block] (sensor) {સેન્સર\\(Scanner)};
    \node [gtu block, right=1.1cm of sensor] (pre) {પ્રી-પ્રોસેસ\\મોડ્યુલ};
    \node [gtu block, right=1.1cm of pre] (feature) {ફીચર\\એક્સટ્રેક્શન};
    
    \node [gtu block, below=1cm of feature] (match) {મેચિંગ\\મોડ્યુલ};
    \node [gtu block, right=1.5cm of match, shape=cylinder, shape border rotate=90, aspect=0.25] (db) {ડેટાબેઝ};
    \node [gtu block, below=1cm of match] (decision) {નિર્ણય\\મોડ્યુલ};
    
    \path [gtu arrow] (sensor) -- (pre);
    \path [gtu arrow] (pre) -- (feature);
    \path [gtu arrow] (feature) -- (match);
    \path [gtu arrow] (db) -- (match);
    \path [gtu arrow] (match) -- (decision);
\end{tikzpicture}
\captionof{figure}{બાયોમેટ્રિક સિસ્ટમ}
\end{center}
\end{solutionbox}

\questionmarks{3(અ અથવા)}{3}{જળ પ્રદૂષણ મોનિટરિંગ સિસ્ટમનો બ્લોક ડાયાગ્રામ દોરો.}

\begin{solutionbox}
\begin{center}
\begin{tikzpicture}[node distance=1.5cm, auto]
    \node [gtu block] (sensors) {Water Quality\\Sensors\\(pH, DO, Temp)};
    \node [gtu block, below=1cm of sensors] (micro) {Microcontroller\\(Data Logger)};
    
    \node [gtu block, below left=1.5cm of micro] (lcd) {Local LCD\\Display};
    \node [gtu block, below right=1.5cm of micro] (gsm) {GSM/WiFi\\Module};
    \node [gtu block, below=1cm of gsm] (cloud) {Cloud\\Database};
    
    \path [gtu arrow] (sensors) -- (micro);
    \path [gtu arrow] (micro) -- (lcd);
    \path [gtu arrow] (micro) -- (gsm);
    \path [gtu arrow] (gsm) -- (cloud);
\end{tikzpicture}
\captionof{figure}{વોટર પોલ્યુશન મોનિટરિંગ}
\end{center}
\end{solutionbox}

\questionmarks{3(બ અથવા)}{4}{સ્માર્ટ વૉચનો બ્લોક ડાયાગ્રામ દોરો અને સમજાવો.}

\begin{solutionbox}
\begin{center}
\begin{tikzpicture}[node distance=2cm, auto]
    \node [gtu block] (soc) {SoC\\(ARM પ્રોસેસર)};
    
    \node [gtu block, above left=1.5cm of soc] (touch) {ટચસ્ક્રીન\\ડિસ્પ્લે};
    \node [gtu block, above right=1.5cm of soc] (sensors) {સેન્સર્સ\\(Accel, Gyro)};
    
    \node [gtu block, below left=1.5cm of soc] (batt) {બેટરી\\પેક};
    \node [gtu block, below right=1.5cm of soc] (comms) {બ્લૂટૂથ/WiFi\\મોડ્યુલ};
    
    \path [gtu arrow] (touch) -- (soc);
    \path [gtu arrow] (sensors) -- (soc);
    \path [gtu arrow] (soc) -- (touch);
    \path [gtu arrow] (batt) -- (soc);
    \path [gtu arrow] (soc) -- (comms);
    \path [gtu arrow] (comms) -- (soc);
\end{tikzpicture}
\captionof{figure}{સ્માર્ટ વૉચ}
\end{center}
\end{solutionbox}

\questionmarks{3(ક અથવા)}{7}{AR/VR કોર ટેકનોલોજીને સમજાવો અને તેની એપ્લિકેશનોની ચર્ચા કરો.}

\begin{solutionbox}
\textbf{કોર ઘટકો}: ડિસ્પ્લે ટેકનોલોજી, ટ્રેકિંગ સિસ્ટમ્સ (Motion, Eye), પ્રોસેસિંગ પાવર (GPU, AI/ML).
\textbf{ઉપયોગો}: શિક્ષણ (Interactive textbooks), આરોગ્યસંભાળ (Surgery assistance), મનોરંજન (Gaming), ઉદ્યોગ (Training), રિટેઇલ (Virtual try-on).
\end{solutionbox}

\questionmarks{4(અ)}{3}{ઇનઓર્ગેનિક અને ઓર્ગેનિક ઇલેક્ટ્રોનિક્સ વચ્ચે તફાવત કરો.}

\begin{solutionbox}
\textbf{કોષ્ટક: ઇનઓર્ગેનિક વિરુદ્ધ ઓર્ગેનિક ઇલેક્ટ્રોનિક્સ}
\begin{center}
\captionof{table}{ઇનઓર્ગેનિક વિરુદ્ધ ઓર્ગેનિક ઇલેક્ટ્રોનિક્સ}
\begin{tabulary}{\linewidth}{|L|L|L|}
\hline
\textbf{પેરામીટર} & \textbf{ઇનઓર્ગેનિક ઇલેક્ટ્રોનિક્સ} & \textbf{ઓર્ગેનિક ઇલેક્ટ્રોનિક્સ} \\ \hline
સામગ્રી & સિલિકોન, જર્મેનિયમ & કાર્બન-આધારિત સંયોજનો \\ \hline
પ્રોસેસિંગ & ઉચ્ચ તાપમાન & નીચા તાપમાન \\ \hline
લવચીકતા & સખત & લવચીક \\ \hline
કિંમત & ઊંચી & ઓછી \\ \hline
પ્રદર્શન & હાઇ સ્પીડ, સ્થિર & લોઅર સ્પીડ, સુધારાતું \\ \hline
\end{tabulary}
\end{center}
\end{solutionbox}

\questionmarks{4(બ)}{4}{વિવિધ પ્રકારના ઓર્ગેનિક ઘટકોની યાદી બનાવો અને કોઈપણ બેને સમજાવો.}

\begin{solutionbox}
\textbf{કોષ્ટક: ઓર્ગેનિક ઘટકોના પ્રકારો}
\begin{center}
\captionof{table}{ઓર્ગેનિક ઘટકોના પ્રકારો}
\begin{tabulary}{\linewidth}{|L|L|L|}
\hline
\textbf{ઘટક} & \textbf{પૂરું નામ} & \textbf{ઉપયોગ} \\ \hline
OLED & ઓર્ગેનિક લાઇટ એમિટિંગ ડાયોડ & ડિસ્પ્લે \\ \hline
OFET & ઓર્ગેનિક ફિલ્ડ ઇફેક્ટ ટ્રાન્ઝિસ્ટર & સ્વિચિંગ \\ \hline
OPVD & ઓર્ગેનિક ફોટોવોલ્ટેઇક ડિવાઇસ & સોલાર સેલ \\ \hline
OECT & ઓર્ગેનિક ઇલેક્ટ્રોકેમિકલ ટ્રાન્ઝિસ્ટર & બાયોસેન્સર્સ \\ \hline
\end{tabulary}
\end{center}
\textbf{OLED}: ઇલેક્ટ્રોડ્સ વચ્ચે ઓર્ગેનિક લેયર્સ; જતે પ્રકાશિત કરે છે.
\\
\textbf{OFET}: ઓર્ગેનિક સેમિકન્ડક્ટર ચેનલ; ગેટ વોલ્ટેજ દ્વારા નિયંત્રિત.
\end{solutionbox}

\questionmarks{4(ક)}{7}{ઇલેક્ટ્રિક વ્હિકલનો બ્લોક ડાયાગ્રામ દોરો અને સમજાવો.}

\begin{solutionbox}
\begin{center}
\begin{tikzpicture}[node distance=2cm, auto]
    \node [gtu block] (batt) {બેટરી પેક};
    \node [gtu block, right=3cm of batt] (charger) {ચાર્જર\\(AC/DC)};
    
    \node [gtu block, below=1.5cm of batt] (pe) {પાવર ઇલેક્ટ્રોનિક્સ\\(ઇન્વર્ટર/કન્વર્ટર)};
    
    \node [gtu block, below=1.5cm of pe] (motor) {ઇલેક્ટ્રિક\\મોટર};
    \node [gtu block, right=3cm of motor] (vcu) {વ્હિકલ\\કંટ્રોલર};
    
    \node [gtu block, below=1.5cm of motor] (trans) {ટ્રાન્સમિશન\\સિસ્ટમ};
    \node [gtu block, right=3cm of trans] (regen) {રિજનરેટિવ\\બ્રેકિંગ};
    
    \node [gtu block, below=1cm of trans] (wheels) {Wheels};
    
    \path [gtu arrow] (charger) -- (batt);
    \path [gtu arrow] (batt) -- (pe);
    \path [gtu arrow] (pe) -- (motor);
    \path [gtu arrow] (motor) -- (trans);
    \path [gtu arrow] (trans) -- (wheels);
    \path [gtu arrow] (vcu) -| (pe);
    \path [gtu arrow] (vcu) -- (motor);
    \path [gtu arrow, dashed] (regen) -- (batt);
\end{tikzpicture}
\captionof{figure}{ઇલેક્ટ્રિક વ્હિકલ બ્લોક ડાયાગ્રામ}
\end{center}
\end{solutionbox}

\questionmarks{4(અ અથવા)}{3}{ઓર્ગેનિક ઇલેક્ટ્રોનિક્સના ફાયદા લખો.}

\begin{solutionbox}
\textbf{મુખ્ય ફાયદા}:
\begin{itemize}
    \item \keyword{લવચીકતા}: વાંકી શકાય, વાળી શકાય.
    \item \keyword{ઓછી કિંમત}: સસ્તી સામગ્રી, પ્રિન્ટિંગ.
    \item \keyword{મોટો વિસ્તાર}: સરળ સ્કેલિંગ.
    \item \keyword{હલકું વજન}: પાતળું, હલકું.
    \item \keyword{પારદર્શકતા}: પારદર્શી ડિવાઇસ.
\end{itemize}
\end{solutionbox}

\questionmarks{4(બ અથવા)}{4}{AR/VR ઉદ્યોગના પરિપ્રેક્ષ્યો અને તકો વિશે લખો.}

\begin{solutionbox}
\textbf{તકો}: 5G નેટવર્ક્સ, AI ઇન્ટિગ્રેશન, હાર્ડવેર મિનિયેચરાઇઝેશન.
\textbf{પડકારો}: મોશન સિકનેસ, બેટરી લાઇફ, કન્ટેન્ટ ક્રિએશન.
\end{solutionbox}

\questionmarks{4(ક અથવા)}{7}{EV આર્કિટેક્ચર દોરો અને સમજાવો.}

\begin{solutionbox}
\begin{center}
\begin{tikzpicture}[node distance=2cm, auto]
    \node [gtu block] (hv) {High Voltage\\Battery Pack};
    
    \node [gtu block, below=1.5cm of hv] (inv) {Traction\\Inverter};
    \node [gtu block, left=1cm of inv] (dcdc) {DC-DC\\Converter};
    \node [gtu block, right=1cm of inv] (obc) {Onboard\\Charger};
    
    \node [gtu block, below=1.5cm of dcdc] (lv) {12V Battery};
    \node [gtu block, below=1.5cm of inv] (motor) {AC Motor};
    \node [gtu block, below=1.5cm of obc] (port) {Charging Port};
    
    \node [gtu block, below=1cm of motor] (trans) {Transmission\\\& Wheels};
    
    \path [gtu arrow] (hv) -- node[right] {HV DC} (inv);
    \path [gtu arrow] (hv) -| (dcdc);
    \path [gtu arrow] (dcdc) -- (lv);
    \path [gtu arrow] (obc) |- (hv);
    \path [gtu arrow] (port) -- (obc);
    \path [gtu arrow] (inv) -- (motor);
    \path [gtu arrow] (motor) -- (trans);
\end{tikzpicture}
\captionof{figure}{EV આર્કિટેક્ચર}
\end{center}
\end{solutionbox}

\questionmarks{5(અ)}{3}{મોનોક્રિસ્ટેલાઇન સિલિકોન સોલાર સેલ વિશે ટૂંકમાં લખો.}

\begin{solutionbox}
\textbf{લક્ષણો}:
\begin{itemize}
    \item \keyword{કાર્યક્ષમતા}: 18-22\% (સર્વોચ્ચ).
    \item \keyword{બંધારણ}: સિંગલ ક્રિસ્ટલ, એકસમાન દેખાવ.
    \item \keyword{આયુષ્ય}: 25+ વર્ષ.
\end{itemize}
\end{solutionbox}

\questionmarks{5(બ)}{4}{ડ્રોનના કાર્યસિદ્ધાંતનું વર્ણન કરો.}

\begin{solutionbox}
\textbf{મૂળભૂત ભૌતિકશાસ્ત્ર}: લિફ્ટ જનરેશન, થ્રસ્ટ કંટ્રોલ, સ્ટેબિલિટી (જાયરોસ્કોપ).
\textbf{કંટ્રોલ}:
\begin{itemize}
    \item \keyword{ઉપર/નીચે}: બધી મોટર સ્પીડ વધારવી/ઘટાડવી.
    \item \keyword{દિશા}: મોટર્સની સ્પીડમાં ફેરફાર.
\end{itemize}
\end{solutionbox}

\questionmarks{5(ક)}{7}{Raspberry Pi નો બ્લોક ડાયાગ્રામ સમજાવો.}

\begin{solutionbox}
\begin{center}
\begin{tikzpicture}[node distance=2cm, auto]
    \node [gtu block] (bus) {System Bus};
    
    \node [gtu block, above=1cm of bus] (cpu) {ARM પ્રોસેસર};
    \node [gtu block, right=1cm of cpu] (ram) {Memory\\(RAM)};
    
    \node [gtu block, below left=1.5cm of bus] (gpio) {GPIO\\(40 Pins)};
    \node [gtu block, below=1.5cm of bus] (usb) {USB/HDMI\\Ports};
    \node [gtu block, below right=1.5cm of bus] (eth) {Ethernet};
    
    \node [gtu block, below=1cm of gpio] (sd) {Storage\\(microSD)};
    \node [gtu block, below=1cm of eth] (pwr) {Power\\Mgmt};
    
    \path [gtu arrow] (cpu) -- (bus);
    \path [gtu arrow] (ram) -- (cpu);
    \path [gtu arrow] (bus) -- (gpio);
    \path [gtu arrow] (bus) -| (usb);
    \path [gtu arrow] (bus) -| (eth);
    \path [gtu arrow] (gpio) -- (sd);
\end{tikzpicture}
\captionof{figure}{Raspberry Pi બ્લોક ડાયાગ્રામ}
\end{center}
\end{solutionbox}

\questionmarks{5(અ અથવા)}{3}{પોલીક્રિસ્ટેલાઇન સિલિકોન સોલાર સેલ વિશે ટૂંકમાં લખો.}

\begin{solutionbox}
\textbf{લક્ષણો}:
\begin{itemize}
    \item \keyword{કાર્યક્ષમતા}: 15-17\%.
    \item \keyword{બંધારણ}: બહુવિધ ક્રિસ્ટલ, બ્લુ સ્પેકલ્ડ દેખાવ.
    \item \keyword{કિંમત}: મધ્યમ (મોનો કરતા ઓછી).
\end{itemize}
\end{solutionbox}

\questionmarks{5(બ અથવા)}{4}{મશીન લર્નિંગ ટેકનિકના પ્રકારોની સરખામણી કરો: સુપરવાઇઝ્ડ અને અનસુપરવાઇઝ્ડ.}

\begin{solutionbox}
\textbf{કોષ્ટક: સુપરવાઇઝ્ડ વિરુદ્ધ અનસુપરવાઇઝ્ડ લર્નિંગ}
\begin{center}
\captionof{table}{સુપરવાઇઝ્ડ વિરુદ્ધ અનસુપરવાઇઝ્ડ લર્નિંગ}
\begin{tabulary}{\linewidth}{|L|L|L|}
\hline
\textbf{પાસું} & \textbf{સુપરવાઇઝ્ડ લર્નિંગ} & \textbf{અનસુપરવાઇઝ્ડ લર્નિંગ} \\ \hline
ડેટા ટાઇપ & લેબલ્ડ ડેટા & અનલેબલ્ડ ડેટા \\ \hline
લક્ષ્ય & પ્રિડિક્શન & પેટર્ન ડિસ્કવરી \\ \hline
ઉદાહરણો & ક્લાસિફિકેશન, રિગ્રેશન & ક્લસ્ટરિંગ, એસોસિએશન \\ \hline
અલ્ગોરિધમ & SVM, ડિસિઝન ટ્રીઝ & K-means, PCA \\ \hline
\end{tabulary}
\end{center}
\end{solutionbox}

\questionmarks{5(ક અથવા)}{7}{સ્માર્ટ હોમનો બ્લોક ડાયાગ્રામ દોરો અને સમજાવો.}

\begin{solutionbox}
\begin{center}
\begin{tikzpicture}[node distance=2cm, auto]
    \node [gtu block] (hub) {Smart Controller\\(Hub)};
    
    \node [gtu block, below left=2cm of hub] (light) {Lighting\\Control};
    \node [gtu block, below=2cm of hub] (hvac) {HVAC\\Control};
    \node [gtu block, below right=2cm of hub] (sec) {Security\\System};
    
    \node [gtu block, below=1cm of light] (dev1) {Smart Bulbs};
    \node [gtu block, below=1cm of hvac] (dev2) {Thermostat};
    \node [gtu block, below=1cm of sec] (dev3) {Locks/Cams};
    
    \node [gtu block, above=1.5cm of hub] (inet) {Internet\\Gateway};
    \node [gtu block, above=1cm of inet] (app) {Smartphone\\App};
    
    \path [gtu arrow] (app) -- (inet);
    \path [gtu arrow] (inet) -- (hub);
    \path [gtu arrow] (hub) -- (light);
    \path [gtu arrow] (hub) -- (hvac);
    \path [gtu arrow] (hub) -- (sec);
    \path [gtu arrow] (light) -- (dev1);
    \path [gtu arrow] (hvac) -- (dev2);
    \path [gtu arrow] (sec) -- (dev3);
    
\end{tikzpicture}
\captionof{figure}{સ્માર્ટ હોમ સિસ્ટમ}
\end{center}
\end{solutionbox}

\end{document}
