\documentclass{article}
% Adjust the relative path to point to the latex-templates directory

% content/resources/templates/preamble.tex
\usepackage[margin=0.6in]{geometry}
\author{Milav Dabgar}
\usepackage{amsmath,amssymb,amsthm}
\usepackage{booktabs}
\usepackage{multirow}
\usepackage{xcolor}
\usepackage{tcolorbox}
\tcbuselibrary{breakable,skins}
\usepackage[colorlinks=true,linkcolor=blue]{hyperref}
\usepackage{titlesec}
\usepackage{enumitem}
\usepackage{tikz}
\usepackage{pgfplots}
\usepackage{circuitikz}
\usepackage[version=4]{mhchem}
\usepackage{longtable}
\usepackage{array}
\usepackage{float}
\usepackage{caption}
\usepackage{listings}

\lstset{
  basicstyle=\small\ttfamily,
  breaklines=true,
  breakatwhitespace=false,
  postbreak=\mbox{\textcolor{red}{$\hookrightarrow$}\space},
  float=false,
  numbers=left,
  numberstyle=\tiny\color{gray},
  numbersep=10pt,
  xleftmargin=2em,
  keywordstyle=\color{blue},
  commentstyle=\color{green!60!black},
  stringstyle=\color{purple},
  backgroundcolor=\color{gray!5},
  showstringspaces=false,
  tabsize=2,
  captionpos=b,
  keepspaces=true,
  columns=flexible
}

\pgfplotsset{compat=1.18}
\usetikzlibrary{shapes,arrows,positioning,calc,patterns,decorations.pathmorphing,decorations.markings,arrows.meta}

% Color scheme
\definecolor{headcolor}{RGB}{0,102,204}
\definecolor{keycolor}{RGB}{220,20,60}
\definecolor{solutioncolor}{RGB}{34,139,34}
\definecolor{mnemoniccolor}{RGB}{148,0,211}
\definecolor{codecolor}{RGB}{0,0,100}

% Spacing
\setlength{\parskip}{3pt}
\setlist[itemize]{nosep}
\setlist[enumerate]{nosep}

% Title formatting
\titleformat{\section}{\Large\bfseries\color{headcolor}}{\thesection}{1em}{}
\titleformat{\subsection}{\large\bfseries\color{headcolor}}{\thesubsection}{1em}{}

% Pandoc tightlist compatibility
\providecommand{\tightlist}{%
  \setlength{\itemsep}{0pt}\setlength{\parskip}{0pt}}

% Pandoc longtable compatibility
\newcounter{none}
\def\thenone{}


% content/resources/templates/english-boxes.tex
% This file is currently empty - it exists to maintain consistency with the import structure.
% Add custom environments here if needed in the future.


% Custom commands for GTU solutions
% This file defines semantic commands for consistent formatting

% Question command with automatic formatting
\newcommand{\question}[2]{%
  \section*{Question #1}%
  \textbf{#2}%
}

% OR question variant
\newcommand{\questionor}[2]{%
  \section*{Question #1 OR}%
  \textbf{#2}%
}

% Proper table environment with caption
\newenvironment{answertable}[1]{%
  \begin{table}[htbp]
  \centering
  \caption{#1}
}{%
  \end{table}
}

% Proper figure environment for diagrams
\newenvironment{answerdiagram}[1]{%
  \begin{figure}[htbp]
  \centering
  \caption{#1}
}{%
  \end{figure}
}

% Semantic markup for key terms
\newcommand{\keyword}[1]{\textbf{#1}}
\newcommand{\code}[1]{\texttt{#1}}
\newcommand{\classname}[1]{\texttt{#1}}
\newcommand{\methodname}[1]{\texttt{#1}}

% Proper quotation marks
\newcommand{\mnemonic}[1]{``#1''}


\title{Renewable Energy \& Emerging Trends in Electronics (4361106) - Winter 2024 Solution}
\date{November 25, 2024}

\begin{document}
\maketitle

\questionmarks{1(a)}{3}{List different types of Renewable Energy Sources and explain any one in detail.}

\begin{solutionbox}
\textbf{Table: Types of Renewable Energy Sources}
\begin{center}
\captionof{table}{Types of Renewable Energy Sources}
\begin{tabulary}{\linewidth}{|L|L|L|}
\hline
\textbf{Type} & \textbf{Source} & \textbf{Application} \\ \hline
Solar & Sun's radiation & Solar panels, heating \\ \hline
Wind & Moving air & Wind turbines \\ \hline
Hydroelectric & Flowing water & Dams, turbines \\ \hline
Biomass & Organic matter & Biofuels, heating \\ \hline
Geothermal & Earth's heat & Power plants, heating \\ \hline
\end{tabulary}
\end{center}

\textbf{Solar Energy Explanation}:
\begin{itemize}
    \item \keyword{Photovoltaic Effect}: Converts sunlight directly into electricity using silicon cells.
    \item \keyword{Advantages}: Clean, abundant, renewable.
    \item \keyword{Applications}: Rooftop systems, solar farms.
\end{itemize}
\end{solutionbox}

\begin{mnemonicbox}
\mnemonic{SWHBG - Sun Wins Hearts By Going}
\end{mnemonicbox}

\questionmarks{1(b)}{4}{List the different types of Solar Cells and explain any two.}

\begin{solutionbox}
\textbf{Table: Types of Solar Cells}
\begin{center}
\captionof{table}{Types of Solar Cells}
\begin{tabulary}{\linewidth}{|L|L|L|L|}
\hline
\textbf{Type} & \textbf{Efficiency} & \textbf{Cost} & \textbf{Application} \\ \hline
Silicon & 15-20\% & Medium & Residential \\ \hline
Monocrystalline & 18-22\% & High & Premium systems \\ \hline
Polycrystalline & 15-17\% & Low & Budget systems \\ \hline
Thin Film & 10-12\% & Very Low & Large installations \\ \hline
Amorphous Silicon & 6-8\% & Low & Small devices \\ \hline
\end{tabulary}
\end{center}

\textbf{Monocrystalline Silicon}:
\begin{itemize}
    \item \keyword{Structure}: Single crystal structure with uniform appearance.
    \item \keyword{Efficiency}: Highest among silicon cells (18-22\%).
\end{itemize}

\textbf{Polycrystalline Silicon}:
\begin{itemize}
    \item \keyword{Structure}: Multiple crystals with blue speckled appearance.
    \item \keyword{Cost}: Lower manufacturing cost than monocrystalline.
\end{itemize}
\end{solutionbox}

\begin{mnemonicbox}
\mnemonic{My Poly Thin Amp - Most Popular Types Available}
\end{mnemonicbox}

\questionmarks{1(c)}{7}{Draw and explain Block Diagram of a Home Solar rooftop system.}

\begin{solutionbox}
\begin{center}
\begin{tikzpicture}[node distance=2cm, auto]
    \node [gtu block] (solar) {Solar Panels\\(PV Array)};
    \node [gtu block, below of=solar] (inverter) {Inverter\\(DC to AC)};
    \node [gtu block, below of=inverter] (meter) {Meter\\(Bidirectional)};
    
    \node [gtu block, below left=2cm and 1cm of meter] (load) {Home Load};
    \node [gtu block, below right=2cm and 1cm of meter] (grid) {Grid\\Connection};

    \path [gtu arrow] (solar) -- node[right] {DC Power} (inverter);
    \path [gtu arrow] (inverter) -- node[right] {AC Power} (meter);
    \path [gtu arrow] (meter) -| (load);
    \path [gtu arrow] (meter) -| (grid);
    \path [gtu arrow, dashed] (grid) |- (meter); % Bidirectional flow
\end{tikzpicture}
\captionof{figure}{Home Solar Rooftop System}
\end{center}

\textbf{Components Explanation}:
\begin{itemize}
    \item \keyword{Solar Panels}: Convert sunlight to DC electricity using photovoltaic effect.
    \item \keyword{Inverter}: Converts DC power to AC power for home use.
    \item \keyword{Bidirectional Meter}: Measures power consumption and excess power fed to grid.
    \item \keyword{Home Load}: Electrical appliances and devices.
    \item \keyword{Grid Connection}: Connects to utility grid for backup and selling excess power.
\end{itemize}

\textbf{Working Principle}:
\begin{itemize}
    \item \keyword{Day Operation}: Solar panels generate electricity, inverter converts to AC.
    \item \keyword{Excess Power}: Fed back to grid through net metering.
    \item \keyword{Night Operation}: Power drawn from grid when solar not available.
\end{itemize}
\end{solutionbox}

\begin{mnemonicbox}
\mnemonic{Solar Inverter Meter Home Grid - Simple Installation Makes Happy Generation}
\end{mnemonicbox}

\questionmarks{1(c OR)}{7}{Explain with diagram Solar Photovoltaic effect \& Principle of photovoltaic conversion.}

\begin{solutionbox}
\begin{center}
\begin{tikzpicture}[node distance=1.5cm, auto]
    % Draw layers
    \fill[blue!10] (0, 2) rectangle (5, 3) node[pos=0.5] {N-Type Layer (Phosphorus)};
    \fill[red!10] (0, 0) rectangle (5, 2) node[pos=0.5] {P-Type Layer (Boron)};
    \draw (0,0) rectangle (5,3);
    \draw [dashed] (0,2) -- (5,2) node[right] {P-N Junction};
    
    % Sunlight
    \foreach \x in {1, 2.5, 4}
        \draw [->, decorate, decoration={snake}, thick, orange] (\x, 4.5) -- (\x, 3.2);
    \node at (2.5, 4.8) {Sunlight (Photons)};
    
    % Circuit
    \draw [thick] (0, 2.5) -- (-1, 2.5) -- (-1, -1) -- (2.5, -1);
    \draw [thick] (0, 0.5) -- (-0.5, 0.5) -- (-0.5, -0.5) -- (2.5, -0.5);
    
    \node [gtu block, minimum width=2cm] at (2.5, -0.75) (load) {External\\Circuit};
    \draw [thick] (load) -| (6, 0.5) -- (5, 0.5);
    \draw [thick] (load) -| (6, 2.5) -- (5, 2.5);
    
    \node at (5.5, 2.8) {(-)};
    \node at (5.5, 0.2) {(+)};

\end{tikzpicture}
\captionof{figure}{Solar Photovoltaic Effect}
\end{center}

\textbf{Photovoltaic Effect Process}:
\begin{itemize}
    \item \keyword{Photon Absorption}: Solar photons hit silicon atoms.
    \item \keyword{Electron Excitation}: Electrons gain energy and move to conduction band.
    \item \keyword{Charge Separation}: P-N junction creates electric field.
    \item \keyword{Current Flow}: Electrons flow through external circuit.
\end{itemize}

\textbf{Key Parameters}:
\begin{itemize}
    \item \keyword{Band Gap}: Energy difference between valence and conduction bands.
    \item \keyword{Open Circuit Voltage}: Maximum voltage when no current flows.
    \item \keyword{Short Circuit Current}: Maximum current when terminals are shorted.
\end{itemize}

\textbf{Conversion Efficiency}:
\begin{itemize}
    \item \keyword{Theoretical Maximum}: ~33\% for single junction cells.
    \item \keyword{Practical Efficiency}: 15-22\% for commercial cells.
\end{itemize}
\end{solutionbox}

\begin{mnemonicbox}
\mnemonic{Photons Push Electrons Past Junction - Power Production Perfectly Planned}
\end{mnemonicbox}

\questionmarks{2(a)}{3}{What is Nanotechnology? List its applications.}

\begin{solutionbox}
\textbf{Definition}: Nanotechnology is the manipulation of matter at atomic and molecular scale (1-100 nanometers).

\textbf{Table: Applications of Nanotechnology}
\begin{center}
\captionof{table}{Applications of Nanotechnology}
\begin{tabulary}{\linewidth}{|L|L|L|}
\hline
\textbf{Field} & \textbf{Application} & \textbf{Benefit} \\ \hline
Electronics & Transistors, Memory & Miniaturization \\ \hline
Medicine & Drug delivery, Imaging & Targeted treatment \\ \hline
Energy & Solar cells, Batteries & Higher efficiency \\ \hline
Materials & Composites, Coatings & Enhanced properties \\ \hline
Environment & Water purification & Clean technology \\ \hline
\end{tabulary}
\end{center}

\textbf{Key Features}:
\begin{itemize}
    \item \keyword{Scale}: 1 nanometer = $10^{-9}$ meters.
    \item \keyword{Properties}: Different properties at nanoscale.
    \item \keyword{Applications}: Cross-disciplinary technology.
\end{itemize}
\end{solutionbox}

\begin{mnemonicbox}
\mnemonic{Nano Makes Everything More Efficient}
\end{mnemonicbox}

\questionmarks{2(b)}{4}{List the different types of EV technologies and explain any two.}

\begin{solutionbox}
\textbf{Table: Types of EV Technologies}
\begin{center}
\captionof{table}{Types of EV Technologies}
\begin{tabulary}{\linewidth}{|L|L|L|L|}
\hline
\textbf{Type} & \textbf{Full Form} & \textbf{Power Source} & \textbf{Range} \\ \hline
BEV & Battery Electric Vehicle & Battery only & 150-400 km \\ \hline
HEV & Hybrid Electric Vehicle & Engine + Battery & 600+ km \\ \hline
PHEV & Plug-in Hybrid Electric & Engine + Battery & 50-80 km electric \\ \hline
FCEV & Fuel Cell Electric Vehicle & Hydrogen fuel cell & 400-600 km \\ \hline
\end{tabulary}
\end{center}

\textbf{Battery Electric Vehicle (BEV)}:
\begin{itemize}
    \item \keyword{Power Source}: Rechargeable battery pack only.
    \item \keyword{Operation}: Pure electric drive with zero emissions.
    \item \keyword{Charging}: External charging from grid required.
\end{itemize}

\textbf{Hybrid Electric Vehicle (HEV)}:
\begin{itemize}
    \item \keyword{Power Source}: Internal combustion engine + electric motor.
    \item \keyword{Operation}: Automatic switching between power sources.
    \item \keyword{Efficiency}: Regenerative braking recovers energy.
\end{itemize}
\end{solutionbox}

\begin{mnemonicbox}
\mnemonic{Big Hybrid Plug Fuel - Better Transportation Options}
\end{mnemonicbox}

\questionmarks{2(c)}{7}{Describe the Block diagram of a drone and its major components.}

\begin{solutionbox}
\begin{center}
\begin{tikzpicture}[node distance=2cm, auto]
    \node [gtu block] (fc) {Flight Controller\\(Microprocessor)};
    
    \node [gtu block, above left=1.5cm of fc] (cam) {Camera};
    \node [gtu block, above right=1.5cm of fc] (gps) {GPS Module};
    
    \node [gtu block, below left=1.5cm of fc] (motors) {Motors \&\\Propellers};
    \node [gtu block, below right=1.5cm of fc] (sensors) {Sensors\\(Gyro, Accel)};
    
    \node [gtu block, below=1.5cm of motors] (batt) {Battery Pack};
    \node [gtu block, below=1.5cm of sensors] (rx) {Transmitter\\\& Receiver};

    \path [gtu arrow] (cam) -- (fc);
    \path [gtu arrow] (gps) -- (fc);
    \path [gtu arrow] (fc) -- (motors);
    \path [gtu arrow] (fc) -- (sensors);
    \path [gtu arrow] (sensors) -- (fc); % Feedback
    \path [gtu arrow] (batt) -- (motors);
    \path [gtu arrow] (rx) -- (fc);
    \path [gtu arrow, dashed] (fc) -- (rx); % Telemetry
\end{tikzpicture}
\captionof{figure}{Drone Block Diagram}
\end{center}

\textbf{Major Components}:
\begin{itemize}
    \item \keyword{Flight Controller}: Central processing unit controlling all operations; provides stabilization, navigation, autopilot functions.
    \item \keyword{Motors and Propellers}: Brushless motors for high efficiency; propellers generate thrust.
    \item \keyword{Sensors Package}: Gyroscope (angular velocity), Accelerometer (acceleration/tilt), Barometer (altitude).
    \item \keyword{Power System}: LiPo Battery for high power density; ESC (Electronic Speed Controllers).
    \item \keyword{Communication}: Transmitter/Receiver for remote control; GPS for position tracking.
\end{itemize}
\end{solutionbox}

\begin{mnemonicbox}
\mnemonic{Flying Controllers Motor Sensors Power Communication - Drones Fly Perfectly}
\end{mnemonicbox}

\questionmarks{2(a OR)}{3}{What is UAV? List its applications.}

\begin{solutionbox}
\textbf{Definition}: UAV (Unmanned Aerial Vehicle) is an aircraft operated without human pilot onboard.

\textbf{Table: UAV Applications}
\begin{center}
\captionof{table}{UAV Applications}
\begin{tabulary}{\linewidth}{|L|L|L|}
\hline
\textbf{Sector} & \textbf{Application} & \textbf{Benefit} \\ \hline
Agriculture & Crop monitoring, Spraying & Precision farming \\ \hline
Security & Surveillance, Border patrol & Enhanced monitoring \\ \hline
Delivery & Package delivery & Fast transportation \\ \hline
Photography & Aerial photography & New perspectives \\ \hline
Inspection & Infrastructure inspection & Safe access \\ \hline
\end{tabulary}
\end{center}
\end{solutionbox}

\begin{mnemonicbox}
\mnemonic{Unmanned Aircraft Versatile - Applications Are Vast}
\end{mnemonicbox}

\questionmarks{2(b OR)}{4}{List the different types of EV energy sources and explain any two.}

\begin{solutionbox}
\textbf{Table: EV Energy Sources}
\begin{center}
\captionof{table}{EV Energy Sources}
\begin{tabulary}{\linewidth}{|L|L|L|L|}
\hline
\textbf{Type} & \textbf{Technology} & \textbf{Storage} & \textbf{Efficiency} \\ \hline
Battery & Lithium-ion & Chemical & 90-95\% \\ \hline
Fuel Cell & Hydrogen & Chemical & 50-60\% \\ \hline
Ultracapacitor & Electric field & Electrical & 95\%+ \\ \hline
Flywheel & Kinetic energy & Mechanical & 85-90\% \\ \hline
Regenerative Braking & Motor generator & Kinetic to electrical & 70-80\% \\ \hline
\end{tabulary}
\end{center}

\textbf{Battery System}: Lithium-ion cells with high energy density; mature technology.
\\
\textbf{Fuel Cell System}: Hydrogen combines with oxygen to produce electricity; quick refueling, long range.
\end{solutionbox}

\begin{mnemonicbox}
\mnemonic{Battery Fuel Ultra Fly Regen - Energy Sources Enable Vehicles}
\end{mnemonicbox}

\questionmarks{2(c OR)}{7}{List the different types of Smart Systems. Explain with a diagram any 2 smart systems.}

\begin{solutionbox}
\textbf{Types of Smart Systems}: Smart Homes, Smart Cars, Smart City, Smart Grid, Smart Health.

\begin{center}
\begin{tikzpicture}[node distance=2cm, auto]
    % Smart Street Light
    \node [gtu block] (micro) {Microcontroller\\(Control Logic)};
    \node [gtu block, above left=1.5cm of micro] (motion) {Motion Sensor};
    \node [gtu block, above right=1.5cm of micro] (light) {Light Sensor};
    \node [gtu block, below left=1.5cm of micro] (led) {LED Street\\Light};
    \node [gtu block, below right=1.5cm of micro] (wireless) {Wireless\\Communication};
    
    \path [gtu arrow] (motion) -- (micro);
    \path [gtu arrow] (light) -- (micro);
    \path [gtu arrow] (micro) -- (led);
    \path [gtu arrow] (micro) -- (wireless);
\end{tikzpicture}
\captionof{figure}{Smart Street Light System}
\end{center}

\begin{center}
\begin{tikzpicture}[node distance=2cm, auto]
    % Smart Water Pollution
    \node [gtu block] (logger) {Data Logger\\(Microcontroller)};
    \node [gtu block, above left=1.5cm of logger] (ph) {pH Sensor};
    \node [gtu block, above right=1.5cm of logger] (temp) {Temp Sensor};
    \node [gtu block, below left=1.5cm of logger] (comms) {GSM/WiFi};
    \node [gtu block, below right=1.5cm of logger] (cloud) {Cloud DB};
    
    \path [gtu arrow] (ph) -- (logger);
    \path [gtu arrow] (temp) -- (logger);
    \path [gtu arrow] (logger) -- (comms);
    \path [gtu arrow] (comms) -- (cloud);
\end{tikzpicture}
\captionof{figure}{Smart Water Pollution Monitoring}
\end{center}

\textbf{Features}: Automation, energy efficiency, remote monitoring.
\end{solutionbox}

\questionmarks{3(a)}{3}{Draw the Block diagram of a Smart Street light control and monitoring system.}

\begin{solutionbox}
\begin{center}
\begin{tikzpicture}[node distance=1.5cm, auto]
    \node [gtu block] (sensors) {Sensors\\(PIR, LDR)};
    \node [gtu block, below=1cm of sensors] (micro) {Microcontroller\\(Arduino)};
    \node [gtu block, below left=1.5cm of micro] (driver) {LED Driver};
    \node [gtu block, below right=1.5cm of micro] (wifi) {WiFi/GSM\\Module};
    
    \node [gtu block, below=1cm of driver] (led) {LED Street\\Light};
    \node [gtu block, below=1cm of wifi] (cloud) {Cloud Server};
    
    \path [gtu arrow] (sensors) -- (micro);
    \path [gtu arrow] (micro) -- (driver);
    \path [gtu arrow] (micro) -- (wifi);
    \path [gtu arrow] (driver) -- (led);
    \path [gtu arrow] (wifi) -- (cloud);
\end{tikzpicture}
\captionof{figure}{Smart Street Light Control}
\end{center}
\end{solutionbox}

\questionmarks{3(b)}{4}{Draw and explain the block diagram of a wearable health monitoring system.}

\begin{solutionbox}
\begin{center}
\begin{tikzpicture}[node distance=1.8cm, auto]
    \node [gtu block] (micro) {Microprocessor\\(Data Processing)};
    \node [gtu block, above left=1.5cm of micro] (hr) {Heart Rate\\Sensor};
    \node [gtu block, above right=1.5cm of micro] (temp) {Temperature\\Sensor};
    
    \node [gtu block, below left=1.5cm of micro] (disp) {Display\\(OLED)};
    \node [gtu block, below right=1.5cm of micro] (bt) {Bluetooth};
    \node [gtu block, below=1cm of bt] (phone) {Smartphone\\App};
    
    \path [gtu arrow] (hr) -- (micro);
    \path [gtu arrow] (temp) -- (micro);
    \path [gtu arrow] (micro) -- (disp);
    \path [gtu arrow] (micro) -- (bt);
    \path [gtu arrow] (bt) -- (phone);
\end{tikzpicture}
\captionof{figure}{Wearable Health Monitoring System}
\end{center}
\textbf{Explanation}: Sensors monitor vital signs; processor analyzes data; Bluetooth sends data to smartphone; triggers alerts if needed.
\end{solutionbox}

\questionmarks{3(c)}{7}{Explain Biometric systems and their basic block diagram.}

\begin{solutionbox}
\begin{center}
\begin{tikzpicture}[node distance=1.5cm, auto]
    \node [gtu block] (sensor) {Sensor\\(Scanner)};
    \node [gtu block, right=1cm of sensor] (pre) {Pre-process\\Module};
    \node [gtu block, right=1cm of pre] (feature) {Feature\\Extraction};
    
    \node [gtu block, below=1cm of feature] (match) {Matching\\Module};
    \node [gtu block, right=1.5cm of match, shape=cylinder, shape border rotate=90, aspect=0.25] (db) {Database};
    \node [gtu block, below=1cm of match] (decision) {Decision\\Module};
    
    \path [gtu arrow] (sensor) -- (pre);
    \path [gtu arrow] (pre) -- (feature);
    \path [gtu arrow] (feature) -- (match);
    \path [gtu arrow] (db) -- (match);
    \path [gtu arrow] (match) -- (decision);
\end{tikzpicture}
\captionof{figure}{Biometric System Block Diagram}
\end{center}

\textbf{Components}:
\begin{itemize}
    \item \keyword{Sensor Module}: Captures raw biometric data.
    \item \keyword{Pre-processing}: Noise removal and enhancement.
    \item \keyword{Feature Extraction}: Extracts unique characteristics (template).
    \item \keyword{Matching Module}: Compares template with database.
    \item \keyword{Database}: Stores enrolled templates securely.
    \item \keyword{Decision Module}: Accepts/Rejects based on score.
\end{itemize}
\end{solutionbox}

\questionmarks{3(a OR)}{3}{Draw the Block diagram of a Water pollution monitoring system.}

\begin{solutionbox}
\begin{center}
\begin{tikzpicture}[node distance=1.5cm, auto]
    \node [gtu block] (sensors) {Water Quality\\Sensors\\(pH, DO, Temp)};
    \node [gtu block, below=1cm of sensors] (micro) {Microcontroller\\(Data Logger)};
    
    \node [gtu block, below left=1.5cm of micro] (lcd) {Local LCD\\Display};
    \node [gtu block, below right=1.5cm of micro] (gsm) {GSM/WiFi\\Module};
    \node [gtu block, below=1cm of gsm] (cloud) {Cloud\\Database};
    
    \path [gtu arrow] (sensors) -- (micro);
    \path [gtu arrow] (micro) -- (lcd);
    \path [gtu arrow] (micro) -- (gsm);
    \path [gtu arrow] (gsm) -- (cloud);
\end{tikzpicture}
\captionof{figure}{Water Pollution Monitoring System}
\end{center}
\end{solutionbox}

\questionmarks{3(b OR)}{4}{Draw and explain the block diagram of a Smart Watch.}

\begin{solutionbox}
\begin{center}
\begin{tikzpicture}[node distance=2cm, auto]
    \node [gtu block] (soc) {System on Chip\\(ARM Processor)};
    
    \node [gtu block, above left=1.5cm of soc] (touch) {Touchscreen\\Display};
    \node [gtu block, above right=1.5cm of soc] (sensors) {Sensors\\(Accel, Gyro)};
    
    \node [gtu block, below left=1.5cm of soc] (batt) {Battery\\Pack};
    \node [gtu block, below right=1.5cm of soc] (comms) {Bluetooth/WiFi\\Module};
    
    \path [gtu arrow] (touch) -- (soc);
    \path [gtu arrow] (sensors) -- (soc);
    \path [gtu arrow] (soc) -- (touch); % Feedback
    \path [gtu arrow] (batt) -- (soc);
    \path [gtu arrow] (soc) -- (comms);
    \path [gtu arrow] (comms) -- (soc);
\end{tikzpicture}
\captionof{figure}{Smart Watch Block Diagram}
\end{center}
\end{solutionbox}

\questionmarks{3(c OR)}{7}{Explain AR/VR core technology and discuss its applications.}

\begin{solutionbox}
\textbf{Core Components}: Display Technology (See-through vs OLED), Tracking Systems (Motion, Eye, Hand), Processing Power (GPU, CV, AI/ML).

\textbf{Applications}:
\begin{itemize}
    \item \keyword{Education}: Interactive textbooks, virtual classrooms.
    \item \keyword{Healthcare}: Surgery assistance, therapy.
    \item \keyword{Entertainment}: Gaming, virtual concerts.
    \item \keyword{Industry}: Maintenance, training.
\end{itemize}
\end{solutionbox}

\questionmarks{4(a)}{3}{Differentiate between Inorganic and Organic electronics.}

\begin{solutionbox}
\textbf{Table: Inorganic vs Organic Electronics}
\begin{center}
\captionof{table}{Inorganic vs Organic Electronics}
\begin{tabulary}{\linewidth}{|L|L|L|}
\hline
\textbf{Parameter} & \textbf{Inorganic Electronics} & \textbf{Organic Electronics} \\ \hline
Materials & Silicon, Germanium & Carbon-based compounds \\ \hline
Processing & High temperature & Low temperature \\ \hline
Flexibility & Rigid & Flexible \\ \hline
Cost & High & Low \\ \hline
Performance & High speed, stable & Lower speed, improving \\ \hline
\end{tabulary}
\end{center}
\end{solutionbox}

\questionmarks{4(b)}{4}{List different types of organic components and explain any two.}

\begin{solutionbox}
\textbf{Table: Types of Organic Components}
\begin{center}
\captionof{table}{Types of Organic Components}
\begin{tabulary}{\linewidth}{|L|L|L|}
\hline
\textbf{Component} & \textbf{Full Form} & \textbf{Application} \\ \hline
OLED & Organic Light Emitting Diode & Displays \\ \hline
OFET & Organic Field Effect Transistor & Switching \\ \hline
OPVD & Organic Photovoltaic Device & Solar cells \\ \hline
OECT & Organic Electrochemical Transistor & Biosensors \\ \hline
\end{tabulary}
\end{center}
\textbf{OLED}: Self-illuminating, flexible, wide viewing angle.
\textbf{OFET}: Organic semiconductor channel, current controlled by gate.
\end{solutionbox}

\questionmarks{4(c)}{7}{Draw and explain the block diagram of an electric vehicle.}

\begin{solutionbox}
\begin{center}
\begin{tikzpicture}[node distance=2cm, auto]
    \node [gtu block] (batt) {Battery Pack};
    \node [gtu block, right=3cm of batt] (charger) {Charger\\(AC/DC)};
    
    \node [gtu block, below=1.5cm of batt] (pe) {Power Electronics\\(Inverter/Converter)};
    
    \node [gtu block, below=1.5cm of pe] (motor) {Electric\\Motor};
    \node [gtu block, right=3cm of motor] (vcu) {Vehicle\\Controller};
    
    \node [gtu block, below=1.5cm of motor] (trans) {Transmission\\System};
    \node [gtu block, right=3cm of trans] (regen) {Regenerative\\Braking};
    
    \node [gtu block, below=1cm of trans] (wheels) {Wheels};
    
    \path [gtu arrow] (charger) -- (batt);
    \path [gtu arrow] (batt) -- (pe);
    \path [gtu arrow] (pe) -- (motor);
    \path [gtu arrow] (motor) -- (trans);
    \path [gtu arrow] (trans) -- (wheels);
    \path [gtu arrow] (vcu) -| (pe);
    \path [gtu arrow] (vcu) -- (motor);
    \path [gtu arrow, dashed] (regen) -- (batt); % Regen flow
\end{tikzpicture}
\captionof{figure}{Electric Vehicle Block Diagram}
\end{center}
\end{solutionbox}

\questionmarks{4(a OR)}{3}{Write the Advantages of Organic Electronics.}

\begin{solutionbox}
\begin{itemize}
    \item \keyword{Flexibility}: Bendable, rollable.
    \item \keyword{Low Cost}: Cheap materials, printing.
    \item \keyword{Large Area}: Easy scaling.
    \item \keyword{Light Weight}: Thin, lightweight.
    \item \keyword{Transparency}: See-through devices.
\end{itemize}
\end{solutionbox}

\questionmarks{4(b OR)}{4}{Write about AR/VR Industry perspectives and opportunities.}

\begin{solutionbox}
\textbf{Market Segments}: Gaming, Enterprise, Healthcare, Education.
\textbf{Opportunities}: 5G Networks, AI Integration, Hardware Miniaturization.
\textbf{Challenges}: Motion Sickness, Battery Life, Content Creation.
\end{solutionbox}

\questionmarks{4(c OR)}{7}{Draw and explain the EV architecture.}

\begin{solutionbox}
\begin{center}
\begin{tikzpicture}[node distance=2cm, auto]
    \node [gtu block] (hv) {High Voltage\\Battery Pack};
    
    \node [gtu block, below=1.5cm of hv] (inv) {Traction\\Inverter};
    \node [gtu block, left=1cm of inv] (dcdc) {DC-DC\\Converter};
    \node [gtu block, right=1cm of inv] (obc) {Onboard\\Charger};
    
    \node [gtu block, below=1.5cm of dcdc] (lv) {12V Battery};
    \node [gtu block, below=1.5cm of inv] (motor) {AC Motor};
    \node [gtu block, below=1.5cm of obc] (port) {Charging Port};
    
    \node [gtu block, below=1cm of motor] (trans) {Transmission\\\& Wheels};
    
    \path [gtu arrow] (hv) -- node[right] {HV DC} (inv);
    \path [gtu arrow] (hv) -| (dcdc);
    \path [gtu arrow] (dcdc) -- (lv);
    \path [gtu arrow] (obc) |- (hv);
    \path [gtu arrow] (port) -- (obc);
    \path [gtu arrow] (inv) -- (motor);
    \path [gtu arrow] (motor) -- (trans);
\end{tikzpicture}
\captionof{figure}{EV Architecture}
\end{center}
\end{solutionbox}

\questionmarks{5(a)}{3}{Write briefly about Monocrystalline Silicon solar cells.}

\begin{solutionbox}
\textbf{Characteristics}:
\begin{itemize}
    \item \keyword{Efficiency}: 18-22\% (Highest).
    \item \keyword{Structure}: Single crystal, uniform dark blue/black color.
    \item \keyword{Lifespan}: 25+ years.
\end{itemize}
\textbf{Manufacturing}: Czochralski Method.
\end{solutionbox}

\questionmarks{5(b)}{4}{Describe the working principle of a drone.}

\begin{solutionbox}
\textbf{Basic Physics}: Lift generation via propellers, thrust control by speed variation, stability via gyroscope.
\textbf{Control}:
\begin{itemize}
    \item \keyword{Ascend/Descend}: Increase/Decrease speed of all motors.
    \item \keyword{Forward/Backward/Left/Right}: Tilt by varying speed of specific motors.
    \item \keyword{Rotation}: Torque differential.
\end{itemize}
\end{solutionbox}

\questionmarks{5(c)}{7}{Explain the Block diagram of Raspberry Pi.}

\begin{solutionbox}
\begin{center}
\begin{tikzpicture}[node distance=2cm, auto]
    \node [gtu block] (bus) {System Bus};
    
    \node [gtu block, above=1cm of bus] (cpu) {ARM Processor};
    \node [gtu block, right=1cm of cpu] (ram) {Memory\\(RAM)};
    
    \node [gtu block, below left=1.5cm of bus] (gpio) {GPIO\\(40 Pins)};
    \node [gtu block, below=1.5cm of bus] (usb) {USB/HDMI\\Ports};
    \node [gtu block, below right=1.5cm of bus] (eth) {Ethernet};
    
    \node [gtu block, below=1cm of gpio] (sd) {Storage\\(microSD)};
    \node [gtu block, below=1cm of eth] (pwr) {Power\\Mgmt};
    
    \path [gtu arrow] (cpu) -- (bus);
    \path [gtu arrow] (ram) -- (cpu);
    \path [gtu arrow] (bus) -- (gpio);
    \path [gtu arrow] (bus) -| (usb);
    \path [gtu arrow] (bus) -| (eth);
    \path [gtu arrow] (gpio) -- (sd);
\end{tikzpicture}
\captionof{figure}{Raspberry Pi Block Diagram}
\end{center}
\textbf{Core Components}: ARM Processor (SoC), RAM, GPIO (General Purpose Input/Output), Connectivity (USB, HDMI, Ethernet, WiFi), Storage (microSD).
\end{solutionbox}

\questionmarks{5(a OR)}{3}{Write briefly about Polycrystalline Silicon solar cells.}

\begin{solutionbox}
\textbf{Characteristics}:
\begin{itemize}
    \item \keyword{Efficiency}: 15-17\%.
    \item \keyword{Structure}: Multiple crystals, blue speckled appearance.
    \item \keyword{Cost}: Medium (Lower than Mono).
\end{itemize}
\textbf{Manufacturing}: Casting Method.
\end{solutionbox}

\questionmarks{5(b OR)}{4}{Compare Types of machine learning techniques: supervised and unsupervised.}

\begin{solutionbox}
\textbf{Table: Supervised vs Unsupervised Learning}
\begin{center}
\captionof{table}{Supervised vs Unsupervised Learning}
\begin{tabulary}{\linewidth}{|L|L|L|}
\hline
\textbf{Aspect} & \textbf{Supervised Learning} & \textbf{Unsupervised Learning} \\ \hline
Data Type & Labeled data & Unlabeled data \\ \hline
Goal & Prediction & Pattern discovery \\ \hline
Examples & Classification, Regression & Clustering, Association \\ \hline
Algorithms & SVM, Decision Trees & K-means, PCA \\ \hline
\end{tabulary}
\end{center}
\end{solutionbox}

\questionmarks{5(c OR)}{7}{Draw and explain the block diagram of a Smart Home.}

\begin{solutionbox}
\begin{center}
\begin{tikzpicture}[node distance=2cm, auto]
    \node [gtu block] (hub) {Smart Controller\\(Hub)};
    
    \node [gtu block, below left=2cm of hub] (light) {Lighting\\Control};
    \node [gtu block, below=2cm of hub] (hvac) {HVAC\\Control};
    \node [gtu block, below right=2cm of hub] (sec) {Security\\System};
    
    \node [gtu block, below=1cm of light] (dev1) {Smart Bulbs};
    \node [gtu block, below=1cm of hvac] (dev2) {Thermostat};
    \node [gtu block, below=1cm of sec] (dev3) {Locks/Cams};
    
    \node [gtu block, above=1.5cm of hub] (inet) {Internet\\Gateway};
    \node [gtu block, above=1cm of inet] (app) {Smartphone\\App};
    
    \path [gtu arrow] (app) -- (inet);
    \path [gtu arrow] (inet) -- (hub);
    \path [gtu arrow] (hub) -- (light);
    \path [gtu arrow] (hub) -- (hvac);
    \path [gtu arrow] (hub) -- (sec);
    \path [gtu arrow] (light) -- (dev1);
    \path [gtu arrow] (hvac) -- (dev2);
    \path [gtu arrow] (sec) -- (dev3);
    
\end{tikzpicture}
\captionof{figure}{Smart Home System}
\end{center}
\textbf{Components}: Smart Controller (ZigBee/Z-Wave), Lighting (Smart Bulbs), HVAC (Thermostat), Security (Locks, Cameras), Internet Gateway, Smartphone App.
\end{solutionbox}

\end{document}
