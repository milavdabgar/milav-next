\documentclass{article}

% content/resources/templates/preamble.tex
\usepackage[margin=0.6in]{geometry}
\author{Milav Dabgar}
\usepackage{amsmath,amssymb,amsthm}
\usepackage{booktabs}
\usepackage{multirow}
\usepackage{xcolor}
\usepackage{tcolorbox}
\tcbuselibrary{breakable,skins}
\usepackage[colorlinks=true,linkcolor=blue]{hyperref}
\usepackage{titlesec}
\usepackage{enumitem}
\usepackage{tikz}
\usepackage{pgfplots}
\usepackage{circuitikz}
\usepackage[version=4]{mhchem}
\usepackage{longtable}
\usepackage{array}
\usepackage{float}
\usepackage{caption}
\usepackage{listings}

\lstset{
  basicstyle=\small\ttfamily,
  breaklines=true,
  breakatwhitespace=false,
  postbreak=\mbox{\textcolor{red}{$\hookrightarrow$}\space},
  float=false,
  numbers=left,
  numberstyle=\tiny\color{gray},
  numbersep=10pt,
  xleftmargin=2em,
  keywordstyle=\color{blue},
  commentstyle=\color{green!60!black},
  stringstyle=\color{purple},
  backgroundcolor=\color{gray!5},
  showstringspaces=false,
  tabsize=2,
  captionpos=b,
  keepspaces=true,
  columns=flexible
}

\pgfplotsset{compat=1.18}
\usetikzlibrary{shapes,arrows,positioning,calc,patterns,decorations.pathmorphing,decorations.markings,arrows.meta}

% Color scheme
\definecolor{headcolor}{RGB}{0,102,204}
\definecolor{keycolor}{RGB}{220,20,60}
\definecolor{solutioncolor}{RGB}{34,139,34}
\definecolor{mnemoniccolor}{RGB}{148,0,211}
\definecolor{codecolor}{RGB}{0,0,100}

% Spacing
\setlength{\parskip}{3pt}
\setlist[itemize]{nosep}
\setlist[enumerate]{nosep}

% Title formatting
\titleformat{\section}{\Large\bfseries\color{headcolor}}{\thesection}{1em}{}
\titleformat{\subsection}{\large\bfseries\color{headcolor}}{\thesubsection}{1em}{}

% Pandoc tightlist compatibility
\providecommand{\tightlist}{%
  \setlength{\itemsep}{0pt}\setlength{\parskip}{0pt}}

% Pandoc longtable compatibility
\newcounter{none}
\def\thenone{}


% content/resources/templates/english-boxes.tex

% Custom environments
\newtcolorbox{solutionbox}{
 breakable,
 enhanced,
 colback=solutioncolor!5!white,
 colframe=solutioncolor!75!black,
 fonttitle=\bfseries,
 title=Solution
}

\newtcolorbox{solutionboxnobreak}{
 colback=solutioncolor!5!white,
 colframe=solutioncolor!75!black,
 fonttitle=\bfseries,
 title=Solution
}

\newtcolorbox{keyformula}{
 breakable,
 enhanced,
 colback=keycolor!5!white,
 colframe=keycolor!75!black,
 fonttitle=\bfseries,
 title=Key Formula
}

\newtcolorbox{mnemonicboxenv}{
 breakable,
 enhanced,
 colback=mnemoniccolor!5!white,
 colframe=mnemoniccolor!75!black,
 fonttitle=\bfseries,
 title=Mnemonic
}

\newcommand{\mnemonicbox}[1]{%
  \begin{mnemonicboxenv}
    #1
  \end{mnemonicboxenv}
}


% Custom commands for GTU solutions
% This file defines semantic commands for consistent formatting

% Question command with automatic formatting
\newcommand{\question}[2]{%
  \section*{Question #1}%
  \textbf{#2}%
}

% OR question variant
\newcommand{\questionor}[2]{%
  \section*{Question #1 OR}%
  \textbf{#2}%
}

% Proper table environment with caption
\newenvironment{answertable}[1]{%
  \begin{table}[htbp]
  \centering
  \caption{#1}
}{%
  \end{table}
}

% Proper figure environment for diagrams
\newenvironment{answerdiagram}[1]{%
  \begin{figure}[htbp]
  \centering
  \caption{#1}
}{%
  \end{figure}
}

% Semantic markup for key terms
\newcommand{\keyword}[1]{\textbf{#1}}
\newcommand{\code}[1]{\texttt{#1}}
\newcommand{\classname}[1]{\texttt{#1}}
\newcommand{\methodname}[1]{\texttt{#1}}

% Proper quotation marks
\newcommand{\mnemonic}[1]{``#1''}


\title{Renewable Energy \& Emerging Trends in Electronics (4361106) - Summer 2025 Solution}
\date{May 14, 2025}

\begin{document}
\maketitle

\questionmarks{1(a)}{3}{Define Renewable Energy and explain its importance.}

\begin{solutionbox}
\textbf{Answer}:
\textbf{Renewable Energy} is energy derived from natural sources that are continuously replenished, such as solar, wind, hydroelectric, biomass, and geothermal energy.

\begin{center}
\captionof{table}{Types of Renewable Energy Sources}
\begin{tabulary}{\linewidth}{|L|L|L|}
\hline
\textbf{Type} & \textbf{Source} & \textbf{Advantage} \\ \hline
\textbf{Solar} & Sun's radiation & Clean, abundant \\ \hline
\textbf{Wind} & Air movement & No emissions \\ \hline
\textbf{Hydro} & Water flow & Reliable power \\ \hline
\textbf{Biomass} & Organic matter & Carbon neutral \\ \hline
\end{tabulary}
\end{center}

\textbf{Importance:}
\begin{itemize}
\item \textbf{Environmental protection}: Reduces pollution and greenhouse gases
\item \textbf{Energy security}: Reduces dependence on fossil fuels
\item \textbf{Economic benefits}: Creates jobs and reduces energy costs
\end{itemize}
\end{solutionbox}

\begin{mnemonicbox}
\mnemonic{"SEEB" - Solar, Environmental, Economic, Biomass}
\end{mnemonicbox}

\questionmarks{1(b)}{4}{Explain Solar Photovoltaic effect \& Principle of photovoltaic conversion.}

\begin{solutionbox}
\textbf{Answer}:
\textbf{Photovoltaic Effect} is the generation of electric current when light strikes a semiconductor material.

\textbf{Working Principle:}
\begin{enumerate}
\item \textbf{Photon absorption}: Light photons hit solar cell surface
\item \textbf{Electron excitation}: Electrons gain energy and move to conduction band
\item \textbf{Charge separation}: Built-in electric field separates positive and negative charges
\item \textbf{Current generation}: Flow of electrons creates DC electricity
\end{enumerate}

\textbf{Diagram:}

\begin{center}
\begin{tikzpicture}[node distance=1.5cm]
    % Layers
    \node[gtu block, minimum width=4cm, minimum height=1cm, fill=blue!10] (n) {N-type Semiconductor};
    \node[gtu block, minimum width=4cm, minimum height=1cm, fill=red!10, below=0cm of n] (p) {P-type Semiconductor};
    
    % Labels
    \node[right=0.2cm of n] {Electrons (-)};
    \node[right=0.2cm of p] {Holes (+)};
    \node[left=0.2cm of n.south west, anchor=east] {Junction};
    
    % Light
    \foreach \x in {-1.5,-0.5,0.5,1.5}
        \draw[->, decorate, decoration={snake, amplitude=2pt, segment length=5pt}, orange, thick] (\x, 1.5) -- (\x, 0.6) node[midway, right, font=\tiny] {};
    \node[above=1.5cm of n, orange] {Light Photons};

    % Circuit
    \draw[thick] (p.south) -- ++(0,-0.5) -- ++(-2.5,0) -- ++(0,2.5) to[R, l=Load] ++(0,1) -- ++(2.5,0) -- (n.north);
    \node[left=2.6cm of n] {Electric Current};
\end{tikzpicture}
\captionof{figure}{Photovoltaic Conversion Principle}
\end{center}
\end{solutionbox}

\begin{mnemonicbox}
\mnemonic{"PACE" - Photons, Absorption, Charge, Electricity}
\end{mnemonicbox}

\questionmarks{1(c)}{7}{Describe the types of Electric Vehicle (EV) and different Energy sources for EV.}

\begin{solutionbox}
\textbf{Answer}:

\begin{center}
\captionof{table}{Types of Electric Vehicles}
\begin{tabulary}{\linewidth}{|L|L|L|L|}
\hline
\textbf{EV Type} & \textbf{Full Form} & \textbf{Power Source} & \textbf{Range} \\ \hline
\textbf{BEV} & Battery Electric Vehicle & Battery only & 150-400 km \\ \hline
\textbf{HEV} & Hybrid Electric Vehicle & Battery + Engine & 600+ km \\ \hline
\textbf{PHEV} & Plug-in Hybrid & Battery + Engine & 50-100 km electric \\ \hline
\textbf{FCEV} & Fuel Cell Electric & Hydrogen fuel cell & 400-600 km \\ \hline
\end{tabulary}
\end{center}

\textbf{Energy Sources for EVs:}
\begin{itemize}
\item \textbf{Battery}: Lithium-ion batteries store electrical energy
\item \textbf{Fuel Cell}: Converts hydrogen to electricity
\item \textbf{Ultracapacitor}: Quick energy storage and release
\item \textbf{Flywheel}: Mechanical energy storage
\item \textbf{Regenerative Braking}: Recovers energy during braking
\item \textbf{Hybrid Sources}: Combination of multiple energy sources
\end{itemize}

\textbf{Diagram: EV Architecture}

\begin{center}
\begin{tikzpicture}[node distance=1.5cm]
    \node[gtu block] (cont) {Controller};
    \node[gtu block, left=of cont] (bat) {Battery};
    \node[gtu block, right=of cont] (motor) {Motor};
    \node[gtu block, below=of cont] (charge) {Charging System};
    
    \draw[gtu arrow] (bat) -- (cont);
    \draw[gtu arrow] (cont) -- (motor);
    \draw[gtu arrow] (charge) -- (cont);
    \draw[gtu arrow] (charge) -- (bat);
\end{tikzpicture}
\captionof{figure}{Basic EV Architecture}
\end{center}
\end{solutionbox}

\begin{mnemonicbox}
\mnemonic{"BHPF-BUFR" - Battery, Hybrid, Plugin, FuelCell - Battery, Ultracap, Flywheel, Regen}
\end{mnemonicbox}

\questionmarks{1(c) OR}{7}{Discuss different types of Renewable Energy Sources.}

\begin{solutionbox}
\textbf{Answer}:

\begin{center}
\captionof{table}{Renewable Energy Sources Comparison}
\begin{tabulary}{\linewidth}{|L|L|L|L|}
\hline
\textbf{Source} & \textbf{How it Works} & \textbf{Advantages} & \textbf{Applications} \\ \hline
\textbf{Solar} & Converts sunlight to electricity & Clean, abundant & Rooftop systems, farms \\ \hline
\textbf{Wind} & Wind turns turbines & No fuel cost & Wind farms, offshore \\ \hline
\textbf{Hydroelectric} & Water flow generates power & Reliable, long-lasting & Dams, rivers \\ \hline
\textbf{Biomass} & Organic matter combustion & Carbon neutral & Power plants, heating \\ \hline
\textbf{Geothermal} & Earth's heat energy & Constant availability & Heating, electricity \\ \hline
\end{tabulary}
\end{center}

\textbf{Emerging Trends:}
\begin{itemize}
\item \textbf{Tidal Wave}: Ocean wave energy conversion
\item \textbf{Solar Thermal}: Concentrated solar power systems
\item \textbf{Hydrogen}: Clean fuel from renewable sources
\end{itemize}

\textbf{Benefits:}
\begin{itemize}
\item \textbf{Sustainability}: Never depletes
\item \textbf{Environmental}: Minimal pollution
\item \textbf{Economic}: Reduces energy costs long-term
\end{itemize}
\end{solutionbox}

\begin{mnemonicbox}
\mnemonic{"SWHBG-THS" - Solar, Wind, Hydro, Biomass, Geothermal - Tidal, Hydrogen, Solar thermal}
\end{mnemonicbox}

\questionmarks{2(a)}{3}{Define Nanotechnology \& List Applications of Nanotechnology.}

\begin{solutionbox}
\textbf{Answer}:
\textbf{Nanotechnology} is the science of manipulating matter at atomic and molecular scale (1-100 nanometers).

\textbf{Applications:}
\begin{itemize}
\item \textbf{Electronics}: Smaller, faster processors
\item \textbf{Medicine}: Drug delivery systems
\item \textbf{Energy}: Solar cells, batteries
\item \textbf{Materials}: Stronger, lighter composites
\end{itemize}
\end{solutionbox}

\begin{mnemonicbox}
\mnemonic{"NEMS" - Nano Electronics, Medicine, Solar}
\end{mnemonicbox}

\questionmarks{2(b)}{4}{Give Full forms of: UAV, IOT, AI, M2M}

\begin{solutionbox}
\textbf{Answer}:

\begin{center}
\captionof{table}{Technology Abbreviations}
\begin{tabulary}{\linewidth}{|L|L|L|}
\hline
\textbf{Abbreviation} & \textbf{Full Form} & \textbf{Application} \\ \hline
\textbf{UAV} & Unmanned Aerial Vehicle & Surveillance, delivery \\ \hline
\textbf{IOT} & Internet of Things & Smart homes, cities \\ \hline
\textbf{AI} & Artificial Intelligence & Machine learning, automation \\ \hline
\textbf{M2M} & Machine to Machine & Industrial automation \\ \hline
\end{tabulary}
\end{center}
\end{solutionbox}

\begin{mnemonicbox}
\mnemonic{"UIAM" - UAV, IOT, AI, M2M}
\end{mnemonicbox}

\questionmarks{2(c)}{7}{Describe the block diagram of a drone and its major components.}

\begin{solutionbox}
\textbf{Answer}:

\textbf{Block Diagram:}

\begin{center}
\begin{tikzpicture}[node distance=1.5cm]
    \node[gtu block] (fc) {Flight Controller};
    
    \node[gtu block, below=of fc] (bat) {Battery};
    \node[gtu block, above=of fc] (gps) {GPS Module};
    \node[gtu block, left=of fc] (imu) {IMU Sensors};
    \node[gtu block, right=of fc] (cam) {Camera/Gimbal};
    
    \node[gtu state, above left=of fc] (m1) {Motor 1};
    \node[gtu state, above right=of fc] (m2) {Motor 2};
    \node[gtu state, below left=of fc] (m3) {Motor 3};
    \node[gtu state, below right=of fc] (m4) {Motor 4};
    
    \node[gtu block, left=of imu] (rx) {Receiver};
    \node[gtu block, left=of rx] (tx) {Remote Controller};
    
    \draw[gtu arrow] (fc) -- (m1);
    \draw[gtu arrow] (fc) -- (m2);
    \draw[gtu arrow] (fc) -- (m3);
    \draw[gtu arrow] (fc) -- (m4);
    
    \draw[gtu arrow] (bat) -- (fc);
    \draw[gtu arrow] (gps) -- (fc);
    \draw[gtu arrow] (imu) -- (fc);
    \draw[gtu arrow] (fc) -- (cam);
    
    \draw[gtu arrow] (rx) -- (fc);
    \draw[gtu arrow] (tx) -- (rx);
\end{tikzpicture}
\captionof{figure}{Drone Block Diagram}
\end{center}

\textbf{Major Components:}
\begin{itemize}
\item \textbf{Flight Controller}: Brain of drone, processes sensor data
\item \textbf{Motors \& Propellers}: Provide thrust and control movement
\item \textbf{Battery}: Powers all electronic components
\item \textbf{GPS Module}: Provides location and navigation data
\item \textbf{IMU Sensors}: Measure acceleration, rotation, magnetic field
\item \textbf{Camera}: Captures images and videos
\item \textbf{Gimbal}: Stabilizes camera for smooth footage
\end{itemize}

\textbf{Working Principle:}
\begin{itemize}
\item \textbf{Control}: Remote sends commands to receiver
\item \textbf{Processing}: Flight controller interprets commands
\item \textbf{Stabilization}: IMU sensors maintain balance
\item \textbf{Navigation}: GPS provides position feedback
\end{itemize}
\end{solutionbox}

\begin{mnemonicbox}
\mnemonic{"FMBGIC" - Flight controller, Motors, Battery, GPS, IMU, Camera}
\end{mnemonicbox}

\questionmarks{2(a) OR}{3}{Discuss IOT and its importance.}

\begin{solutionbox}
\textbf{Answer}:
\textbf{Internet of Things (IOT)} connects everyday devices to the internet for data exchange and remote control.

\textbf{Importance:}
\begin{itemize}
\item \textbf{Automation}: Smart homes and cities
\item \textbf{Efficiency}: Optimized resource usage
\item \textbf{Monitoring}: Real-time data collection
\end{itemize}
\end{solutionbox}

\begin{mnemonicbox}
\mnemonic{"AEM" - Automation, Efficiency, Monitoring}
\end{mnemonicbox}

\questionmarks{2(b) OR}{4}{Define wearable technology. Name at least three applications of wearable technology.}

\begin{solutionbox}
\textbf{Answer}:
\textbf{Wearable Technology} refers to electronic devices worn on the body to monitor health, fitness, or provide information.

\textbf{Applications:}
\begin{itemize}
\item \textbf{Smart Watches}: Fitness tracking, notifications
\item \textbf{Smart Glasses}: Augmented reality, navigation
\item \textbf{Health Monitors}: Heart rate, blood pressure monitoring
\end{itemize}
\end{solutionbox}

\begin{mnemonicbox}
\mnemonic{"WSH" - Watches, Smart glasses, Health monitors}
\end{mnemonicbox}

\questionmarks{2(c) OR}{7}{Explain with the help of Block diagram Smart Street light control and monitoring.}

\begin{solutionbox}
\textbf{Answer}:

\textbf{Block Diagram:}

\begin{center}
\begin{tikzpicture}[node distance=1.5cm]
    \node[gtu block] (micro) {Microcontroller};
    
    \node[gtu block, left=of micro] (light) {Light Sensor};
    \node[gtu block, above=of micro] (motion) {Motion Sensor};
    \node[gtu block, right=of micro] (comm) {Communication Module};
    
    \node[gtu block, below=of micro] (led) {LED Street Light};
    \node[gtu block, right=of led] (dim) {Dimming Control};
    
    \node[gtu block, right=of comm] (central) {Central Control System};
    \node[gtu block, left=of led] (power) {Power Supply};
    
    \draw[gtu arrow] (light) -- (micro);
    \draw[gtu arrow] (motion) -- (micro);
    \draw[gtu arrow] (micro) -- (comm);
    \draw[gtu arrow] (comm) -- (central);
    \draw[gtu arrow] (micro) -- (led);
    \draw[gtu arrow] (micro) -- (dim);
    \draw[gtu arrow] (power) -- (micro);
    \draw[gtu arrow] (central) -- (comm);
\end{tikzpicture}
\captionof{figure}{Smart Street Light Control System}
\end{center}

\textbf{Components:}
\begin{itemize}
\item \textbf{Light Sensor}: Detects ambient light levels
\item \textbf{Motion Sensor}: Detects pedestrian/vehicle movement
\item \textbf{Microcontroller}: Processes sensor data and controls lighting
\item \textbf{Communication Module}: Wireless connection to control center
\item \textbf{LED Street Light}: Energy-efficient lighting
\item \textbf{Dimming Control}: Adjusts brightness based on need
\end{itemize}

\textbf{Working:}
\begin{itemize}
\item \textbf{Auto ON/OFF}: Lights turn on at dusk, off at dawn
\item \textbf{Motion Detection}: Increases brightness when movement detected
\item \textbf{Remote Monitoring}: Central system monitors all lights
\item \textbf{Energy Saving}: Dims lights when no activity detected
\end{itemize}
\end{solutionbox}

\begin{mnemonicbox}
\mnemonic{"LMCL" - Light sensor, Motion sensor, Controller, LED}
\end{mnemonicbox}

\questionmarks{3(a)}{3}{Compare Organic and Inorganic electronics.}

\begin{solutionbox}
\textbf{Answer}:

\begin{center}
\captionof{table}{Organic vs Inorganic Electronics}
\begin{tabulary}{\linewidth}{|L|L|L|}
\hline
\textbf{Parameter} & \textbf{Organic Electronics} & \textbf{Inorganic Electronics} \\ \hline
\textbf{Material} & Carbon-based compounds & Silicon, metals \\ \hline
\textbf{Cost} & Lower manufacturing cost & Higher cost \\ \hline
\textbf{Flexibility} & Flexible, bendable & Rigid structure \\ \hline
\textbf{Processing} & Low temperature & High temperature \\ \hline
\end{tabulary}
\end{center}
\end{solutionbox}

\begin{mnemonicbox}
\mnemonic{"MCFP" - Material, Cost, Flexibility, Processing}
\end{mnemonicbox}

\questionmarks{3(b)}{4}{Write a short note on OPVD.}

\begin{solutionbox}
\textbf{Answer}:
\textbf{OPVD (Organic Photovoltaic Devices)} are solar cells made from organic semiconducting materials.

\textbf{Characteristics:}
\begin{itemize}
\item \textbf{Flexible}: Can be made on flexible substrates
\item \textbf{Low-cost}: Cheaper manufacturing process
\item \textbf{Lightweight}: Suitable for portable applications
\item \textbf{Semi-transparent}: Can be integrated into windows
\end{itemize}

\textbf{Applications:}
\begin{itemize}
\item \textbf{Building Integration}: Solar windows
\item \textbf{Portable Devices}: Flexible solar chargers
\item \textbf{Wearable Electronics}: Solar-powered gadgets
\end{itemize}
\end{solutionbox}

\begin{mnemonicbox}
\mnemonic{"FLLW" - Flexible, Low-cost, Lightweight, Windows}
\end{mnemonicbox}

\questionmarks{3(c)}{7}{Explain Biometric systems and their basic block diagram.}

\begin{solutionbox}
\textbf{Answer}:
\textbf{Biometric System} identifies individuals based on unique biological characteristics.

\textbf{Block Diagram:}

\begin{center}
\begin{tikzpicture}[node distance=1.5cm]
    \node[gtu block] (sensor) {Biometric Sensor};
    \node[gtu block, right=of sensor] (signal) {Signal Processing};
    \node[gtu block, right=of signal] (feat) {Feature Extraction};
    \node[gtu block, below=of feat] (match) {Template Matching};
    \node[gtu block, left=of match] (db) {Database};
    \node[gtu decision, right=of match] (decide) {Decision};
    \node[gtu state, below=of decide] (access) {Access Control};
    
    \draw[gtu arrow] (sensor) -- (signal);
    \draw[gtu arrow] (signal) -- (feat);
    \draw[gtu arrow] (feat) -- (match);
    \draw[gtu arrow] (db) -- (match);
    \draw[gtu arrow] (match) -- (decide);
    \draw[gtu arrow] (decide) -- (access);
\end{tikzpicture}
\captionof{figure}{Biometric System Components}
\end{center}

\textbf{Components:}
\begin{itemize}
\item \textbf{Sensor Module}: Captures biometric data (fingerprint, iris, face)
\item \textbf{Signal Processing}: Enhances and cleans captured signal
\item \textbf{Feature Extraction}: Identifies unique characteristics
\item \textbf{Database Module}: Stores biometric templates
\item \textbf{Matching Module}: Compares captured data with stored templates
\item \textbf{Decision Module}: Makes final accept/reject decision
\end{itemize}

\textbf{Types of Biometrics:}
\begin{itemize}
\item \textbf{Fingerprint}: Ridge patterns on fingers
\item \textbf{Iris}: Eye iris patterns
\item \textbf{Face Recognition}: Facial features
\item \textbf{Voice}: Voice patterns and characteristics
\end{itemize}
\end{solutionbox}

\begin{mnemonicbox}
\mnemonic{"SFEMD" - Sensor, Feature extraction, Matching, Database, Decision}
\end{mnemonicbox}

\questionmarks{3(a) OR}{3}{List the advantages and applications of organic electronics.}

\begin{solutionbox}
\textbf{Answer}:

\textbf{Advantages:}
\begin{itemize}
\item \textbf{Flexible}: Bendable electronic devices
\item \textbf{Low-cost}: Cheaper manufacturing
\item \textbf{Large-area}: Can cover large surfaces
\end{itemize}

\textbf{Applications:}
\begin{itemize}
\item \textbf{OLED Displays}: Flexible screens
\item \textbf{Solar Cells}: Lightweight panels
\item \textbf{RFID Tags}: Flexible identification
\end{itemize}
\end{solutionbox}

\begin{mnemonicbox}
\mnemonic{"FLL-OSR" - Flexible, Low-cost, Large-area - OLED, Solar, RFID}
\end{mnemonicbox}

\questionmarks{3(b) OR}{4}{Write a short note on OLED.}

\begin{solutionbox}
\textbf{Answer}:
\textbf{OLED (Organic Light Emitting Diode)} is a display technology using organic compounds that emit light when electric current is applied.

\textbf{Advantages:}
\begin{itemize}
\item \textbf{Self-illuminating}: No backlight needed
\item \textbf{High contrast}: True black colors
\item \textbf{Flexible}: Can be bent and curved
\item \textbf{Energy efficient}: Lower power consumption
\end{itemize}

\textbf{Applications:}
\begin{itemize}
\item \textbf{Smartphones}: OLED screens
\item \textbf{TVs}: Ultra-thin displays
\item \textbf{Wearables}: Smartwatch displays
\end{itemize}
\end{solutionbox}

\begin{mnemonicbox}
\mnemonic{"SHFE" - Self-illuminating, High contrast, Flexible, Efficient}
\end{mnemonicbox}

\questionmarks{3(c) OR}{7}{Explain AR/VR core technology and discuss its applications.}

\begin{solutionbox}
\textbf{Answer}:
\textbf{AR (Augmented Reality)} overlays digital information on real world, while \textbf{VR (Virtual Reality)} creates completely immersive digital environment.

\textbf{Core Technologies:}
\begin{itemize}
\item \textbf{Display Systems}: Head-mounted displays, screens
\item \textbf{Tracking Systems}: Motion sensors, cameras
\item \textbf{Processing Units}: GPU, specialized chips
\item \textbf{Input Methods}: Controllers, gesture recognition
\end{itemize}

\textbf{Table: AR vs VR Comparison}

\begin{center}
\captionof{table}{AR vs VR Comparison}
\begin{tabulary}{\linewidth}{|L|L|L|}
\hline
\textbf{Aspect} & \textbf{AR} & \textbf{VR} \\ \hline
\textbf{Reality} & Mixed with real world & Completely virtual \\ \hline
\textbf{Equipment} & Smartphone, AR glasses & VR headset, controllers \\ \hline
\textbf{Immersion} & Partial & Complete \\ \hline
\textbf{Mobility} & Mobile friendly & Stationary setup \\ \hline
\end{tabulary}
\end{center}

\textbf{Applications:}
\begin{itemize}
\item \textbf{AR}: Gaming (Pokemon Go), Education, Navigation, Shopping
\item \textbf{VR}: Entertainment, Training, Architecture, Therapy
\end{itemize}
\end{solutionbox}

\begin{mnemonicbox}
\mnemonic{"DTPI-GENT" - Display, Tracking, Processing, Input - Gaming, Education, Navigation, Training}
\end{mnemonicbox}

\questionmarks{4(a)}{3}{Draw Block Diagram of a Home Solar rooftop system.}

\begin{solutionbox}
\textbf{Answer}:

\textbf{Block Diagram:}

\begin{center}
\begin{tikzpicture}[node distance=1.5cm]
    \node[gtu block] (solar) {Solar Panels};
    \node[gtu block, right=of solar] (inv) {Inverter};
    \node[gtu block, right=of inv] (load) {AC Load Panel};
    
    \node[gtu block, below=of inv] (batt) {Battery Storage};
    \node[gtu block, right=of batt] (grid) {Utility Grid Connection};
    
    \draw[gtu arrow] (solar) -- (inv);
    \draw[gtu arrow] (inv) -- (load);
    \draw[gtu arrow] (inv) -- (batt);
    \draw[gtu arrow] (batt) -- (inv);
    \draw[gtu arrow] (inv) -- (grid);
    \draw[gtu arrow] (grid) -- (inv);
\end{tikzpicture}
\captionof{figure}{Home Solar Rooftop System}
\end{center}

\textbf{Components:}
\begin{itemize}
\item \textbf{Solar Panels}: Convert sunlight to DC electricity
\item \textbf{Inverter}: Converts DC to AC power
\item \textbf{Battery Storage}: Stores excess energy
\end{itemize}
\end{solutionbox}

\begin{mnemonicbox}
\mnemonic{"SIB" - Solar panels, Inverter, Battery}
\end{mnemonicbox}

\questionmarks{4(b)}{4}{Explain working principle of OFET.}

\begin{solutionbox}
\textbf{Answer}:
\textbf{OFET (Organic Field Effect Transistor)} uses organic semiconductors to control current flow.

\textbf{Working Principle:}
\begin{enumerate}
\item \textbf{Gate Voltage}: Applied voltage creates electric field
\item \textbf{Channel Formation}: Electric field modulates conductivity
\item \textbf{Current Control}: Source-drain current controlled by gate
\item \textbf{Switching}: ON/OFF states for digital applications
\end{enumerate}

\textbf{Structure:}
\begin{itemize}
\item \textbf{Source/Drain}: Current injection points
\item \textbf{Gate}: Control electrode
\item \textbf{Organic Layer}: Active semiconductor material
\end{itemize}
\end{solutionbox}

\begin{mnemonicbox}
\mnemonic{"GCCS" - Gate voltage, Channel, Current, Switching}
\end{mnemonicbox}

\questionmarks{4(c)}{7}{List various Machine learning tools. Discuss any two in brief.}

\begin{solutionbox}
\textbf{Answer}:
\textbf{Machine Learning Tools:}
\begin{itemize}
\item \textbf{TensorFlow}: Google's ML framework
\item \textbf{PyTorch}: Facebook's deep learning library
\item \textbf{Scikit-learn}: Python ML library
\item \textbf{Keras}: High-level neural network API
\item \textbf{Machine Learning for Kids}: Educational platform
\item \textbf{Scratch}: Visual programming for ML
\end{itemize}

\textbf{Table: ML Tools Comparison}

\begin{center}
\captionof{table}{ML Tools Comparison}
\begin{tabulary}{\linewidth}{|L|L|L|L|}
\hline
\textbf{Tool} & \textbf{Type} & \textbf{Best For} & \textbf{Difficulty} \\ \hline
\textbf{TensorFlow} & Deep Learning & Complex models & Advanced \\ \hline
\textbf{Scikit-learn} & General ML & Beginners & Easy \\ \hline
\end{tabulary}
\end{center}

\textbf{Detailed Discussion:}
\begin{itemize}
\item \textbf{TensorFlow}: Deep learning and neural networks. Good for large-scale ML and production.
\item \textbf{Scikit-learn}: General algorithms like classification, regression. Easy to use and well-documented.
\end{itemize}
\end{solutionbox}

\begin{mnemonicbox}
\mnemonic{"TPSKMS" - TensorFlow, PyTorch, Scikit, Keras, ML4Kids, Scratch}
\end{mnemonicbox}

\questionmarks{4(a) OR}{3}{Briefly explain Emerging Trends in Renewable Energy.}

\begin{solutionbox}
\textbf{Answer}:

\textbf{Emerging Trends:}
\begin{itemize}
\item \textbf{Floating Solar}: Solar panels on water bodies
\item \textbf{Perovskite Cells}: Next-generation solar technology
\item \textbf{Green Hydrogen}: Clean fuel from renewable sources
\end{itemize}

\textbf{Benefits:}
\begin{itemize}
\item \textbf{Higher efficiency}: Better energy conversion
\item \textbf{Cost reduction}: Cheaper renewable energy
\end{itemize}
\end{solutionbox}

\begin{mnemonicbox}
\mnemonic{"FPG" - Floating solar, Perovskite, Green hydrogen}
\end{mnemonicbox}

\questionmarks{4(b) OR}{4}{Give Full forms of: AR, OLED, OPVD, OFET}

\begin{solutionbox}
\textbf{Answer}:

\begin{center}
\captionof{table}{Technology Full Forms}
\begin{tabulary}{\linewidth}{|L|L|L|}
\hline
\textbf{Abbreviation} & \textbf{Full Form} & \textbf{Technology Area} \\ \hline
\textbf{AR} & Augmented Reality & Mixed reality \\ \hline
\textbf{OLED} & Organic Light Emitting Diode & Display technology \\ \hline
\textbf{OPVD} & Organic Photovoltaic Device & Solar cells \\ \hline
\textbf{OFET} & Organic Field Effect Transistor & Electronics \\ \hline
\end{tabulary}
\end{center}
\end{solutionbox}

\begin{mnemonicbox}
\mnemonic{"AOOO" - AR, OLED, OPVD, OFET}
\end{mnemonicbox}

\questionmarks{4(c) OR}{7}{Explain Block diagram of Raspberry Pi.}

\begin{solutionbox}
\textbf{Answer}:

\textbf{Block Diagram:}

\begin{center}
\begin{tikzpicture}[node distance=1.5cm]
    \node[gtu block, minimum width=3cm] (arm) {ARM Processor};
    \node[gtu block, right=of arm] (ram) {RAM Memory};
    \node[gtu block, left=of arm] (power) {Power Supply};
    
    \node[gtu block, above=of arm] (gpio) {GPIO Pins};
    \node[gtu block, below=of arm] (sd) {MicroSD Card};
    
    \node[gtu block, above right=of arm] (usb) {USB Ports};
    \node[gtu block, below right=of arm] (eth) {Ethernet};
    \node[gtu block, above left=of arm] (hdmi) {HDMI Output};
    
    \draw[gtu arrow] (arm) -- (ram);
    \draw[gtu arrow] (power) -- (arm);
    \draw[gtu arrow] (arm) -- (gpio);
    \draw[gtu arrow] (sd) -- (arm);
    \draw[gtu arrow] (arm) -- (usb);
    \draw[gtu arrow] (arm) -- (eth);
    \draw[gtu arrow] (arm) -- (hdmi);
\end{tikzpicture}
\captionof{figure}{Raspberry Pi Block Diagram}
\end{center}

\textbf{Components:}
\begin{itemize}
\item \textbf{ARM Processor}: Central processing unit (Quad-core)
\item \textbf{RAM Memory}: System memory (1GB-8GB)
\item \textbf{GPIO Pins}: 40 pins for interfacing sensors/devices
\item \textbf{USB Ports}: Connect peripherals
\item \textbf{HDMI Output}: Video display connection
\item \textbf{Ethernet Port}: Network connectivity
\item \textbf{MicroSD Card}: Storage for OS and data
\end{itemize}
\end{solutionbox}

\begin{mnemonicbox}
\mnemonic{"ARGC-EPMS" - ARM, RAM, GPIO, Connectivity - Ethernet, Power, MicroSD, Storage}
\end{mnemonicbox}

\questionmarks{5(a)}{3}{Interface LED with Raspberry Pi.}

\begin{solutionbox}
\textbf{Answer}:

\textbf{Circuit Connection:}

\begin{center}
\begin{tikzpicture}[node distance=1.5cm]
    \node[gtu block] (pi) {Raspberry Pi (GPIO 18)};
    \node[coordinate, right=2cm of pi] (c1) {};
    \node[coordinate, right=1cm of c1] (c2) {};
    
    \draw (pi.east) -- (c1) to[R, l=220$\Omega$] (c2) to[led, l=LED] ++(2,0) node[ground] {};
\end{tikzpicture}
\captionof{figure}{LED Interfacing with Raspberry Pi}
\end{center}

\textbf{Python Code:}
\begin{lstlisting}[language=Python]
import RPi.GPIO as GPIO
import time

GPIO.setmode(GPIO.BCM)
GPIO.setup(18, GPIO.OUT)

while True:
    GPIO.output(18, GPIO.HIGH)  # LED ON
    time.sleep(1)
    GPIO.output(18, GPIO.LOW)   # LED OFF
    time.sleep(1)
\end{lstlisting}
\end{solutionbox}

\begin{mnemonicbox}
\mnemonic{"GPIO-RC" - GPIO pin, Resistor, Code}
\end{mnemonicbox}

\questionmarks{5(b)}{4}{Explain Pandas python library For Machine Learning.}

\begin{solutionbox}
\textbf{Answer}:
\textbf{Pandas} is a Python library for data manipulation and analysis, essential for ML data preprocessing.

\textbf{Key Features:}
\begin{itemize}
\item \textbf{DataFrame}: Tabular data structure
\item \textbf{Data Cleaning}: Handle missing values, duplicates
\item \textbf{Data Import}: Read CSV, Excel, JSON files
\item \textbf{Data Analysis}: Statistical operations, grouping
\end{itemize}

\textbf{ML Applications:}
\begin{itemize}
\item \textbf{Data Preprocessing}: Clean and prepare datasets
\item \textbf{Feature Engineering}: Create new features from data
\item \textbf{Data Exploration}: Understand data patterns
\end{itemize}

\textbf{Common Functions:}
\begin{lstlisting}[language=Python]
import pandas as pd
df = pd.read_csv('data.csv')    # Load data
df.info()                       # Data info
df.describe()                   # Statistics
\end{lstlisting}
\end{solutionbox}

\begin{mnemonicbox}
\mnemonic{"DCIF" - DataFrame, Cleaning, Import, Functions}
\end{mnemonicbox}

\questionmarks{5(c)}{7}{Explain types of machine learning techniques: supervised, unsupervised and reinforcement learning.}

\begin{solutionbox}
\textbf{Answer}:

\begin{center}
\captionof{table}{Machine Learning Types}
\begin{tabulary}{\linewidth}{|L|L|L|L|}
\hline
\textbf{Type} & \textbf{Data Required} & \textbf{Goal} & \textbf{Examples} \\ \hline
\textbf{Supervised} & Labeled data & Predict outcomes & Classification, Regression \\ \hline
\textbf{Unsupervised} & Unlabeled data & Find patterns & Clustering, Dimensionality reduction \\ \hline
\textbf{Reinforcement} & Reward signals & Learn optimal actions & Game playing, Robotics \\ \hline
\end{tabulary}
\end{center}

\textbf{Diagram: ML Learning Process}

\begin{center}
\begin{tikzpicture}[node distance=1.5cm]
    \node[gtu block] (data) {Data};
    \node[gtu decision, right=of data] (type) {Learning Type};
    
    \node[gtu block, right=of type] (unsup) {Unsupervised};
    \node[gtu block, above=of unsup] (sup) {Supervised};
    \node[gtu block, below=of unsup] (rl) {Reinforcement};
    
    \node[gtu block, right=of sup] (pred) {Prediction};
    \node[gtu block, right=of unsup] (pat) {Patterns};
    \node[gtu block, right=of rl] (pol) {Policy};
    
    \draw[gtu arrow] (data) -- (type);
    \draw[gtu arrow] (type) -- (sup);
    \draw[gtu arrow] (type) -- (unsup);
    \draw[gtu arrow] (type) -- (rl);
    \draw[gtu arrow] (sup) -- (pred);
    \draw[gtu arrow] (unsup) -- (pat);
    \draw[gtu arrow] (rl) -- (pol);
\end{tikzpicture}
\captionof{figure}{Machine Learning Types Flow}
\end{center}

\textbf{Descriptions:}
\begin{itemize}
\item \textbf{Supervised Learning}: Learns from input-output pairs. Process involves training with known answers. Applications: Email spam detection.
\item \textbf{Unsupervised Learning}: Finds hidden patterns in data without target variables. Applications: Customer segmentation.
\item \textbf{Reinforcement Learning}: Learns through trial and error interacting with an environment. Applications: Game AI.
\end{itemize}
\end{solutionbox}

\begin{mnemonicbox}
\mnemonic{"SUR-PLR-CPD" - Supervised, Unsupervised, Reinforcement - Prediction, Learning, Rewards}
\end{mnemonicbox}

\questionmarks{5(a) OR}{3}{Explain NumPy python library For Machine Learning.}

\begin{solutionbox}
\textbf{Answer}:
\textbf{NumPy} is fundamental library for numerical computing in Python, essential for ML operations.

\textbf{Key Features:}
\begin{itemize}
\item \textbf{Arrays}: Multi-dimensional array objects
\item \textbf{Mathematical Functions}: Linear algebra operations
\item \textbf{Broadcasting}: Operations on different sized arrays
\end{itemize}

\textbf{ML Applications:}
\begin{itemize}
\item \textbf{Data Storage}: Efficient numerical data storage
\item \textbf{Matrix Operations}: Neural network computations
\end{itemize}
\end{solutionbox}

\begin{mnemonicbox}
\mnemonic{"AMB" - Arrays, Mathematical functions, Broadcasting}
\end{mnemonicbox}

\questionmarks{5(b) OR}{4}{Write Installation steps of Raspberry Pi OS on SD card using Raspberry Pi Imager.}

\begin{solutionbox}
\textbf{Answer}:

\textbf{Installation Steps:}
\begin{enumerate}
\item \textbf{Download}: Install Raspberry Pi Imager from official website
\item \textbf{Insert SD Card}: Connect SD card (16GB+) to computer
\item \textbf{Select OS}: Choose Raspberry Pi OS from list
\item \textbf{Select Storage}: Choose SD card as target
\item \textbf{Write}: Click "Write" to flash OS to SD card
\item \textbf{Eject}: Safely remove SD card after completion
\end{enumerate}

\textbf{Pre-configuration Options:}
\begin{itemize}
\item \textbf{Enable SSH}: For remote access
\item \textbf{Set Username/Password}: Security credentials
\item \textbf{Configure Wi-Fi}: Network settings
\end{itemize}
\end{solutionbox}

\begin{mnemonicbox}
\mnemonic{"DISWS-ESP" - Download, Insert, Select OS, Write, Storage - Enable SSH, Set credentials, Pre-configure}
\end{mnemonicbox}

\questionmarks{5(c) OR}{7}{Interface Temperature and humidity sensors with Raspberry Pi and write Python Program for it.}

\begin{solutionbox}
\textbf{Answer}:

\textbf{Circuit Connection:}

\begin{center}
\begin{tikzpicture}[node distance=1.5cm]
    \node[gtu block] (dht) {DHT22 Sensor};
    \node[gtu block, right=3cm of dht] (pi) {Raspberry Pi};
    
    \draw[thick] (dht.east) -- (pi.west) node[midway, above] {Data (GPIO 4)};
    \node[above=0.5cm of dht] (vcc) {VCC (3.3V)};
    \node[below=0.5cm of dht] (gnd) {GND};
    
    \draw (dht.north) -- (vcc);
    \draw (dht.south) -- (gnd);
\end{tikzpicture}
\captionof{figure}{DHT22 Sensor Interfacing}
\end{center}

\textbf{Python Program:}
\begin{lstlisting}[language=Python]
import Adafruit_DHT
import time

# Sensor type and GPIO pin
sensor = Adafruit_DHT.DHT22
pin = 4

while True:
    humidity, temperature = Adafruit_DHT.read_retry(sensor, pin)
    if humidity is not None and temperature is not None:
        print(f'Temp={temperature:0.1f}*C  Humidity={humidity:0.1f}%')
    else:
        print('Failed to get reading. Try again!')
    time.sleep(2)
\end{lstlisting}
\end{solutionbox}

\begin{mnemonicbox}
\mnemonic{"DHT-Code" - Sensor, Pin, Read loop}
\end{mnemonicbox}

\end{document}
