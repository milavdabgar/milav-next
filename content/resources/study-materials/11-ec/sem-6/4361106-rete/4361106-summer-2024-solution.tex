\documentclass{article}

% content/resources/templates/preamble.tex
\usepackage[margin=0.6in]{geometry}
\author{Milav Dabgar}
\usepackage{amsmath,amssymb,amsthm}
\usepackage{booktabs}
\usepackage{multirow}
\usepackage{xcolor}
\usepackage{tcolorbox}
\tcbuselibrary{breakable,skins}
\usepackage[colorlinks=true,linkcolor=blue]{hyperref}
\usepackage{titlesec}
\usepackage{enumitem}
\usepackage{tikz}
\usepackage{pgfplots}
\usepackage{circuitikz}
\usepackage[version=4]{mhchem}
\usepackage{longtable}
\usepackage{array}
\usepackage{float}
\usepackage{caption}
\usepackage{listings}

\lstset{
  basicstyle=\small\ttfamily,
  breaklines=true,
  breakatwhitespace=false,
  postbreak=\mbox{\textcolor{red}{$\hookrightarrow$}\space},
  float=false,
  numbers=left,
  numberstyle=\tiny\color{gray},
  numbersep=10pt,
  xleftmargin=2em,
  keywordstyle=\color{blue},
  commentstyle=\color{green!60!black},
  stringstyle=\color{purple},
  backgroundcolor=\color{gray!5},
  showstringspaces=false,
  tabsize=2,
  captionpos=b,
  keepspaces=true,
  columns=flexible
}

\pgfplotsset{compat=1.18}
\usetikzlibrary{shapes,arrows,positioning,calc,patterns,decorations.pathmorphing,decorations.markings,arrows.meta}

% Color scheme
\definecolor{headcolor}{RGB}{0,102,204}
\definecolor{keycolor}{RGB}{220,20,60}
\definecolor{solutioncolor}{RGB}{34,139,34}
\definecolor{mnemoniccolor}{RGB}{148,0,211}
\definecolor{codecolor}{RGB}{0,0,100}

% Spacing
\setlength{\parskip}{3pt}
\setlist[itemize]{nosep}
\setlist[enumerate]{nosep}

% Title formatting
\titleformat{\section}{\Large\bfseries\color{headcolor}}{\thesection}{1em}{}
\titleformat{\subsection}{\large\bfseries\color{headcolor}}{\thesubsection}{1em}{}

% Pandoc tightlist compatibility
\providecommand{\tightlist}{%
  \setlength{\itemsep}{0pt}\setlength{\parskip}{0pt}}

% Pandoc longtable compatibility
\newcounter{none}
\def\thenone{}


% content/resources/templates/english-boxes.tex

% Custom environments
\newtcolorbox{solutionbox}{
 breakable,
 enhanced,
 colback=solutioncolor!5!white,
 colframe=solutioncolor!75!black,
 fonttitle=\bfseries,
 title=Solution
}

\newtcolorbox{solutionboxnobreak}{
 colback=solutioncolor!5!white,
 colframe=solutioncolor!75!black,
 fonttitle=\bfseries,
 title=Solution
}

\newtcolorbox{keyformula}{
 breakable,
 enhanced,
 colback=keycolor!5!white,
 colframe=keycolor!75!black,
 fonttitle=\bfseries,
 title=Key Formula
}

\newtcolorbox{mnemonicboxenv}{
 breakable,
 enhanced,
 colback=mnemoniccolor!5!white,
 colframe=mnemoniccolor!75!black,
 fonttitle=\bfseries,
 title=Mnemonic
}

\newcommand{\mnemonicbox}[1]{%
  \begin{mnemonicboxenv}
    #1
  \end{mnemonicboxenv}
}


% Custom commands for GTU solutions
% This file defines semantic commands for consistent formatting

% Question command with automatic formatting
\newcommand{\question}[2]{%
  \section*{Question #1}%
  \textbf{#2}%
}

% OR question variant
\newcommand{\questionor}[2]{%
  \section*{Question #1 OR}%
  \textbf{#2}%
}

% Proper table environment with caption
\newenvironment{answertable}[1]{%
  \begin{table}[htbp]
  \centering
  \caption{#1}
}{%
  \end{table}
}

% Proper figure environment for diagrams
\newenvironment{answerdiagram}[1]{%
  \begin{figure}[htbp]
  \centering
  \caption{#1}
}{%
  \end{figure}
}

% Semantic markup for key terms
\newcommand{\keyword}[1]{\textbf{#1}}
\newcommand{\code}[1]{\texttt{#1}}
\newcommand{\classname}[1]{\texttt{#1}}
\newcommand{\methodname}[1]{\texttt{#1}}

% Proper quotation marks
\newcommand{\mnemonic}[1]{``#1''}


\title{Renewable Energy & Emerging Trends in Electronics (4361106) - Summer 2024 Solution}
\date{May 18, 2024}

\begin{document}
\maketitle

\questionmarks{1(a)}{3}{What is Renewable energy? Explain its importance.}

\begin{solutionbox}
\textbf{Answer}:
Renewable energy is energy derived from natural sources that replenish themselves over time, such as solar, wind, hydro, biomass, and geothermal.

\begin{center}
\captionof{table}{Importance of Renewable Energy}
\begin{tabulary}{\linewidth}{|L|L|}
\hline
\textbf{Aspect} & \textbf{Benefit} \\ \hline
\textbf{Environmental} & Reduces greenhouse gas emissions and pollution \\ \hline
\textbf{Economic} & Creates jobs and reduces energy costs long-term \\ \hline
\textbf{Energy Security} & Reduces dependence on fossil fuel imports \\ \hline
\textbf{Sustainability} & Inexhaustible energy sources for future generations \\ \hline
\end{tabulary}
\end{center}

\textbf{Key Points:}
\begin{itemize}
\item \textbf{Clean Energy}: Zero carbon emissions during operation
\item \textbf{Cost-effective}: Decreasing technology costs make it economical
\item \textbf{Job Creation}: Growing industry providing employment opportunities
\end{itemize}
\end{solutionbox}

\begin{mnemonicbox}
\mnemonic{"EEES" - Environmental protection, Economic benefits, Energy security, Sustainability}
\end{mnemonicbox}

\questionmarks{1(b)}{4}{List the types of Electric Vehicles. Explain each in brief.}

\begin{solutionbox}
\textbf{Answer}:

\begin{center}
\captionof{table}{Types of Electric Vehicles}
\begin{tabulary}{\linewidth}{|L|L|L|}
\hline
\textbf{Type} & \textbf{Full Form} & \textbf{Description} \\ \hline
\textbf{BEV} & Battery Electric Vehicle & Fully electric, powered only by battery \\ \hline
\textbf{HEV} & Hybrid Electric Vehicle & Combines gasoline engine with electric motor \\ \hline
\textbf{PHEV} & Plug-in Hybrid Electric Vehicle & Can be charged from external power source \\ \hline
\textbf{FCEV} & Fuel Cell Electric Vehicle & Uses hydrogen fuel cells for power \\ \hline
\end{tabulary}
\end{center}

\textbf{Key Features:}
\begin{itemize}
\item \textbf{BEV}: Zero emissions, requires charging stations
\item \textbf{HEV}: Better fuel efficiency, self-charging through regenerative braking
\item \textbf{PHEV}: Dual power options, extended range
\item \textbf{FCEV}: Quick refueling, water as only emission
\end{itemize}
\end{solutionbox}

\begin{mnemonicbox}
\mnemonic{"Big Hybrid Plug Fuel" for BEV, HEV, PHEV, FCEV}
\end{mnemonicbox}

\questionmarks{1(c)}{7}{What is the difference between solar energy and solar thermal energy? Discuss the block diagram of home solar rooftop system.}

\begin{solutionbox}
\textbf{Answer}:

\begin{center}
\captionof{table}{Solar Energy vs Solar Thermal Energy}
\begin{tabulary}{\linewidth}{|L|L|L|}
\hline
\textbf{Parameter} & \textbf{Solar Energy (PV)} & \textbf{Solar Thermal Energy} \\ \hline
\textbf{Conversion} & Direct sunlight to electricity & Sunlight to heat energy \\ \hline
\textbf{Technology} & Photovoltaic cells & Solar collectors/panels \\ \hline
\textbf{Output} & Electrical energy & Thermal energy (hot water/steam) \\ \hline
\textbf{Applications} & Power generation, lighting & Water heating, space heating \\ \hline
\textbf{Efficiency} & 15-22\% & 70-80\% \\ \hline
\end{tabulary}
\end{center}

\textbf{Block Diagram: Home Solar Rooftop System}

\begin{center}
\begin{tikzpicture}[node distance=1.5cm]
    \node[gtu block] (panel) {Solar Panels};
    \node[gtu block, right=of panel] (dc) {DC Power};
    \node[gtu block, right=of dc] (cc) {Charge Controller};
    \node[gtu block, below=of cc] (batt) {Battery Bank};
    \node[gtu block, right=of cc] (inv) {Inverter};
    \node[gtu block, right=of inv] (ac) {AC Power};
    \node[gtu state, above right=of ac] (load) {Home Load};
    \node[gtu state, below right=of ac] (grid) {Grid Connection};
    \node[gtu block, above=of cc] (mon) {Monitoring System};

    \draw[gtu arrow] (panel) -- (dc);
    \draw[gtu arrow] (dc) -- (cc);
    \draw[gtu arrow] (cc) -- (batt);
    \draw[gtu arrow] (cc) -- (inv);
    \draw[gtu arrow] (inv) -- (ac);
    \draw[gtu arrow] (ac) -- (load);
    \draw[gtu arrow] (ac) -- (grid);
    \draw[gtu arrow] (mon) -- (cc);
\end{tikzpicture}
\captionof{figure}{Home Solar Rooftop System}
\end{center}

\textbf{Key Components:}
\begin{itemize}
\item \textbf{Solar Panels}: Convert sunlight to DC electricity
\item \textbf{Charge Controller}: Regulates battery charging
\item \textbf{Inverter}: Converts DC to AC power
\item \textbf{Battery Bank}: Stores excess energy
\item \textbf{Grid Connection}: Two-way power flow
\end{itemize}
\end{solutionbox}

\begin{mnemonicbox}
\mnemonic{"Solar Converts Battery Inverter Grid" for main components}
\end{mnemonicbox}

\questionmarks{1(c) OR}{7}{What is solar photovoltaic effect? Explain principle of photovoltaic conversion.}

\begin{solutionbox}
\textbf{Answer}:
Solar photovoltaic effect is the generation of electric current when light falls on semiconductor materials.

\textbf{Principle of Photovoltaic Conversion:}

\begin{center}
\begin{tikzpicture}[node distance=1.5cm]
    \node[gtu block] (sun) {Sunlight Photons};
    \node[gtu block, right=of sun] (pn) {P-N Junction};
    \node[gtu block, right=of pn] (pair) {Electron-Hole Pairs};
    \node[gtu block, below=of pair] (field) {Electric Field Separation};
    \node[gtu block, left=of field] (flow) {Current Flow};
    \node[gtu state, left=of flow] (circ) {External Circuit};

    \draw[gtu arrow] (sun) -- (pn);
    \draw[gtu arrow] (pn) -- (pair);
    \draw[gtu arrow] (pair) -- (field);
    \draw[gtu arrow] (field) -- (flow);
    \draw[gtu arrow] (flow) -- (circ);
\end{tikzpicture}
\captionof{figure}{Photovoltaic Conversion Process}
\end{center}

\textbf{Working Process:}
\begin{itemize}
\item \textbf{Photon Absorption}: Light photons hit semiconductor material
\item \textbf{Electron Excitation}: Electrons gain energy and move to conduction band
\item \textbf{P-N Junction}: Creates electric field separating charges
\item \textbf{Current Generation}: Flow of electrons creates electrical current
\end{itemize}

\textbf{Key Points:}
\begin{itemize}
\item \textbf{Energy Conversion}: Light energy $\rightarrow$ Electrical energy
\item \textbf{Semiconductor Material}: Usually silicon-based
\item \textbf{Direct Conversion}: No moving parts required
\item \textbf{Quantum Effect}: Based on photoelectric effect principle
\end{itemize}

\begin{center}
\captionof{table}{PV Cell Materials}
\begin{tabulary}{\linewidth}{|L|L|L|L|}
\hline
\textbf{Material} & \textbf{Efficiency} & \textbf{Cost} & \textbf{Application} \\ \hline
\textbf{Monocrystalline Silicon} & 18-22\% & High & Residential \\ \hline
\textbf{Polycrystalline Silicon} & 15-17\% & Medium & Commercial \\ \hline
\textbf{Thin Film} & 10-12\% & Low & Large scale \\ \hline
\end{tabulary}
\end{center}
\end{solutionbox}

\begin{mnemonicbox}
\mnemonic{"Photons Push Electrons Producing Power"}
\end{mnemonicbox}

\questionmarks{2(a)}{3}{What is nanotechnology? List any three applications based on nanotechnology.}

\begin{solutionbox}
\textbf{Answer}:
Nanotechnology is the science of manipulating matter at the molecular and atomic scale (1-100 nanometers).

\begin{center}
\captionof{table}{Nanotechnology Applications}
\begin{tabulary}{\linewidth}{|L|L|L|}
\hline
\textbf{Application} & \textbf{Description} & \textbf{Benefit} \\ \hline
\textbf{Medical} & Drug delivery systems, cancer treatment & Targeted therapy \\ \hline
\textbf{Electronics} & Smaller, faster processors and memory & Higher performance \\ \hline
\textbf{Energy} & Solar cells, batteries, fuel cells & Better efficiency \\ \hline
\end{tabulary}
\end{center}

\textbf{Key Points:}
\begin{itemize}
\item \textbf{Scale}: Works at nanometer level ($10^{-9}$ meters)
\item \textbf{Precision}: Atomic-level manipulation
\item \textbf{Revolutionary}: Transforms multiple industries
\end{itemize}
\end{solutionbox}

\begin{mnemonicbox}
\mnemonic{"Nano Makes Everything Better" - Medical, Electronics, Energy}
\end{mnemonicbox}

\questionmarks{2(b)}{4}{Write short note on Tidal wave energy as important emerging renewable energy technology.}

\begin{solutionbox}
\textbf{Answer}:
Tidal wave energy harnesses the kinetic energy of ocean tides and waves to generate electricity.

\textbf{Key Features:}
\begin{itemize}
\item \textbf{Predictable}: Tides follow regular patterns
\item \textbf{High Density}: Water is 800 times denser than air
\item \textbf{Consistent}: Available day and night
\item \textbf{Clean}: No emissions or fuel consumption
\end{itemize}

\begin{center}
\captionof{table}{Tidal Energy Systems}
\begin{tabulary}{\linewidth}{|L|L|L|}
\hline
\textbf{Type} & \textbf{Method} & \textbf{Advantage} \\ \hline
\textbf{Tidal Barrage} & Dam across estuary & High power output \\ \hline
\textbf{Tidal Stream} & Underwater turbines & Minimal environmental impact \\ \hline
\textbf{Wave Energy} & Surface wave motion & Abundant resource \\ \hline
\end{tabulary}
\end{center}

\textbf{Applications:}
\begin{itemize}
\item \textbf{Coastal Power Generation}: Remote coastal communities
\item \textbf{Grid Integration}: Supplement to other renewable sources
\item \textbf{Island Nations}: Ideal for maritime countries
\end{itemize}
\end{solutionbox}

\begin{mnemonicbox}
\mnemonic{"Tides Provide Predictable Power"}
\end{mnemonicbox}

\questionmarks{2(c)}{7}{What is smart water monitoring system? Explain the block diagram of Smart water Quality monitoring system.}

\begin{solutionbox}
\textbf{Answer}:
Smart water monitoring system uses IoT sensors to continuously monitor water quality parameters and provide real-time data for decision making.

\textbf{Block Diagram: Smart Water Quality Monitoring System}

\begin{center}
\begin{tikzpicture}[node distance=1.5cm]
    \node[gtu block] (source) {Water Source};
    \node[gtu block, right=of source] (sensor) {Sensor Array};
    \node[gtu block, above right=of sensor] (ph) {pH Sensor};
    \node[gtu block, above=of sensor] (turb) {Turbidity Sensor};
    \node[gtu block, below=of sensor] (temp) {Temperature Sensor};
    \node[gtu block, below right=of sensor] (do) {Dissolved Oxygen};
    
    \node[gtu block, right=of sensor, xshift=3cm] (micro) {Microcontroller};
    
    \draw[gtu arrow] (source) -- (sensor);
    \draw[gtu arrow] (sensor) -- (ph);
    \draw[gtu arrow] (sensor) -- (turb);
    \draw[gtu arrow] (sensor) -- (temp);
    \draw[gtu arrow] (sensor) -- (do);
    
    \draw[gtu arrow] (ph) -- (micro);
    \draw[gtu arrow] (turb) -- (micro);
    \draw[gtu arrow] (temp) -- (micro);
    \draw[gtu arrow] (do) -- (micro);
    
    \node[gtu block, right=of micro] (proc) {Data Processing};
    \node[gtu block, right=of proc] (wireless) {Wireless Comm};
    \node[gtu block, below=of wireless] (cloud) {Cloud Server};
    \node[gtu state, left=of cloud] (app) {Mobile App};
    \node[gtu state, right=of cloud] (alert) {Alert System};
    
    \draw[gtu arrow] (micro) -- (proc);
    \draw[gtu arrow] (proc) -- (wireless);
    \draw[gtu arrow] (wireless) -- (cloud);
    \draw[gtu arrow] (cloud) -- (app);
    \draw[gtu arrow] (cloud) -- (alert);
\end{tikzpicture}
\captionof{figure}{Smart Water Quality Monitoring System}
\end{center}

\textbf{Key Components:}
\begin{itemize}
\item \textbf{Sensors}: Monitor pH, turbidity, temperature, dissolved oxygen
\item \textbf{Microcontroller}: Arduino/Raspberry Pi for data processing
\item \textbf{Communication}: WiFi/GSM for data transmission
\item \textbf{Cloud Platform}: Data storage and analysis
\item \textbf{User Interface}: Mobile app for monitoring
\end{itemize}

\textbf{Benefits:}
\begin{itemize}
\item \textbf{Real-time Monitoring}: Continuous water quality assessment
\item \textbf{Early Warning}: Immediate alerts for contamination
\item \textbf{Data Analytics}: Historical trends and predictions
\item \textbf{Cost Effective}: Reduces manual testing costs
\end{itemize}

\begin{center}
\captionof{table}{Water Quality Parameters}
\begin{tabulary}{\linewidth}{|L|L|L|}
\hline
\textbf{Parameter} & \textbf{Normal Range} & \textbf{Sensor Type} \\ \hline
\textbf{pH} & 6.5-8.5 & pH electrode \\ \hline
\textbf{Turbidity} & <1 NTU & Optical sensor \\ \hline
\textbf{Temperature} & 15-25$^{\circ}$C & Thermistor \\ \hline
\textbf{Dissolved Oxygen} & >5 mg/L & Electrochemical \\ \hline
\end{tabulary}
\end{center}
\end{solutionbox}

\begin{mnemonicbox}
\mnemonic{"Smart Sensors Send Signals Safely"}
\end{mnemonicbox}

\questionmarks{2(a) OR}{3}{What is wearable technology? Name atleast two applications of wearable technology?}

\begin{solutionbox}
\textbf{Answer}:
Wearable technology refers to electronic devices that can be worn as clothing or accessories, incorporating smart sensors and connectivity.

\textbf{Applications:}
\begin{itemize}
\item \textbf{Health Monitoring}: Smartwatches tracking heart rate, steps, sleep patterns
\item \textbf{Fitness Tracking}: Activity monitors measuring calories, distance, exercise
\item \textbf{Medical Devices}: Continuous glucose monitors, blood pressure monitors
\item \textbf{Smart Glasses}: Augmented reality displays, hands-free computing
\end{itemize}

\textbf{Key Features:}
\begin{itemize}
\item \textbf{Portable}: Lightweight and comfortable to wear
\item \textbf{Connected}: Bluetooth/WiFi connectivity to smartphones
\item \textbf{Sensor-rich}: Multiple sensors for data collection
\end{itemize}
\end{solutionbox}

\begin{mnemonicbox}
\mnemonic{"Wearables Watch Wellness Wirelessly"}
\end{mnemonicbox}

\questionmarks{2(b) OR}{4}{List the different types of solar cell. List different energy sources for Electric vehicle.}

\begin{solutionbox}
\textbf{Answer}:

\begin{center}
\captionof{table}{Types of Solar Cells}
\begin{tabulary}{\linewidth}{|L|L|L|L|}
\hline
\textbf{Type} & \textbf{Material} & \textbf{Efficiency} & \textbf{Cost} \\ \hline
\textbf{Monocrystalline} & Single crystal silicon & 18-22\% & High \\ \hline
\textbf{Polycrystalline} & Multi-crystal silicon & 15-17\% & Medium \\ \hline
\textbf{Thin Film} & Amorphous silicon & 10-12\% & Low \\ \hline
\textbf{Cadmium Telluride} & CdTe compound & 16-18\% & Medium \\ \hline
\end{tabulary}
\end{center}

\begin{center}
\captionof{table}{Energy Sources for Electric Vehicles}
\begin{tabulary}{\linewidth}{|L|L|L|}
\hline
\textbf{Source} & \textbf{Description} & \textbf{Advantage} \\ \hline
\textbf{Battery} & Lithium-ion cells & High energy density \\ \hline
\textbf{Fuel Cell} & Hydrogen conversion & Quick refueling \\ \hline
\textbf{Ultracapacitor} & Rapid charge/discharge & Fast charging \\ \hline
\textbf{Regenerative Braking} & Kinetic energy recovery & Energy efficiency \\ \hline
\end{tabulary}
\end{center}
\end{solutionbox}

\begin{mnemonicbox}
\mnemonic{"Solar: Mono Poly Thin Cadmium" / "EV: Battery Fuel Ultra Regen"}
\end{mnemonicbox}

\questionmarks{2(c) OR}{7}{Describe the block diagram of a drone and its major components.}

\begin{solutionbox}
\textbf{Answer}:

\textbf{Block Diagram: Drone System}

\begin{center}
\begin{tikzpicture}[node distance=1.5cm]
    \node[gtu block] (fc) {Flight Controller};
    
    \node[gtu block, above left=of fc] (esc1) {ESC 1};
    \node[gtu block, above right=of fc] (esc2) {ESC 2};
    \node[gtu block, below right=of fc] (esc3) {ESC 3};
    \node[gtu block, below left=of fc] (esc4) {ESC 4};
    
    \node[gtu state, above=of esc1] (m1) {Motor 1};
    \node[gtu state, above=of esc2] (m2) {Motor 2};
    \node[gtu state, below=of esc3] (m3) {Motor 3};
    \node[gtu state, below=of esc4] (m4) {Motor 4};
    
    \draw[gtu arrow] (fc) -- (esc1); \draw[gtu arrow] (esc1) -- (m1);
    \draw[gtu arrow] (fc) -- (esc2); \draw[gtu arrow] (esc2) -- (m2);
    \draw[gtu arrow] (fc) -- (esc3); \draw[gtu arrow] (esc3) -- (m3);
    \draw[gtu arrow] (fc) -- (esc4); \draw[gtu arrow] (esc4) -- (m4);
    
    \node[gtu block, left=of fc] (gps) {GPS Module};
    \node[gtu block, below=of gps] (imu) {IMU Sensors};
    \node[gtu block, right=of fc] (batt) {Battery};
    \node[gtu block, above=of batt] (cam) {Camera/Gimbal};
    \node[gtu block, below=of batt] (rx) {Radio Receiver};
    \node[gtu state, right=of rx] (tx) {Remote Controller};
    
    \draw[gtu arrow] (gps) -- (fc);
    \draw[gtu arrow] (imu) -- (fc);
    \draw[gtu arrow] (batt) -- (fc);
    \draw[gtu arrow] (cam) -- (fc);
    \draw[gtu arrow] (rx) -- (fc);
    \draw[gtu arrow] (tx) -- (rx);
\end{tikzpicture}
\captionof{figure}{Drone System Architecture}
\end{center}

\textbf{Major Components:}

\begin{center}
\captionof{table}{Drone Components}
\begin{tabulary}{\linewidth}{|L|L|L|}
\hline
\textbf{Component} & \textbf{Function} & \textbf{Importance} \\ \hline
\textbf{Flight Controller} & Central processing unit & Brain of drone \\ \hline
\textbf{ESC} & Motor speed control & Precise motor control \\ \hline
\textbf{Motors \& Propellers} & Generate thrust & Flight capability \\ \hline
\textbf{Battery} & Power supply & Flight duration \\ \hline
\textbf{GPS} & Position tracking & Navigation \\ \hline
\textbf{IMU} & Motion sensing & Stability control \\ \hline
\end{tabulary}
\end{center}

\textbf{Key Systems:}
\begin{itemize}
\item \textbf{Propulsion System}: 4 motors with propellers for lift and control
\item \textbf{Control System}: Flight controller with stabilization algorithms
\item \textbf{Navigation System}: GPS and compass for positioning
\item \textbf{Power System}: LiPo battery for electrical power
\item \textbf{Communication}: Radio link with ground controller
\end{itemize}

\textbf{Working Principle:}
\begin{itemize}
\item \textbf{Lift}: Rotors create upward thrust
\item \textbf{Control}: Varying rotor speeds controls movement
\item \textbf{Stability}: Sensors maintain balance and orientation
\end{itemize}
\end{solutionbox}

\begin{mnemonicbox}
\mnemonic{"Drones Fly Using Motors, Electronics, Sensors, Power"}
\end{mnemonicbox}

\questionmarks{3(a)}{3}{What is IoT? List Key Components of IoT.}

\begin{solutionbox}
\textbf{Answer}:
IoT (Internet of Things) is a network of interconnected physical devices that collect and exchange data through the internet.

\begin{center}
\captionof{table}{Key Components of IoT}
\begin{tabulary}{\linewidth}{|L|L|L|}
\hline
\textbf{Component} & \textbf{Function} & \textbf{Example} \\ \hline
\textbf{Sensors} & Data collection & Temperature, humidity sensors \\ \hline
\textbf{Connectivity} & Data transmission & WiFi, Bluetooth, GSM \\ \hline
\textbf{Data Processing} & Information analysis & Cloud computing \\ \hline
\textbf{User Interface} & Human interaction & Mobile apps, dashboards \\ \hline
\end{tabulary}
\end{center}

\textbf{Key Features:}
\begin{itemize}
\item \textbf{Interconnected}: Devices communicate with each other
\item \textbf{Smart}: Automated decision making
\item \textbf{Data-driven}: Continuous monitoring and analysis
\end{itemize}
\end{solutionbox}

\begin{mnemonicbox}
\mnemonic{"IoT Connects Smart Devices Using Internet"}
\end{mnemonicbox}

\questionmarks{3(b)}{4}{Compare between organic and inorganic electronics.}

\begin{solutionbox}
\textbf{Answer}:

\begin{center}
\captionof{table}{Organic vs Inorganic Electronics}
\begin{tabulary}{\linewidth}{|L|L|L|}
\hline
\textbf{Parameter} & \textbf{Organic Electronics} & \textbf{Inorganic Electronics} \\ \hline
\textbf{Material} & Carbon-based compounds & Silicon, metals \\ \hline
\textbf{Manufacturing} & Low temperature, printing & High temperature, clean room \\ \hline
\textbf{Flexibility} & Flexible, bendable & Rigid, brittle \\ \hline
\textbf{Cost} & Lower production cost & Higher production cost \\ \hline
\textbf{Performance} & Lower speed, efficiency & Higher speed, efficiency \\ \hline
\textbf{Applications} & Displays, solar cells & Processors, memory \\ \hline
\end{tabulary}
\end{center}

\textbf{Key Differences:}
\begin{itemize}
\item \textbf{Processing}: Organic uses solution-based processing
\item \textbf{Substrate}: Organic can use plastic substrates
\item \textbf{Durability}: Inorganic more stable and durable
\item \textbf{Innovation}: Organic enables new form factors
\end{itemize}
\end{solutionbox}

\begin{mnemonicbox}
\mnemonic{"Organic: Flexible, Cheap, Printable vs Inorganic: Fast, Stable, Expensive"}
\end{mnemonicbox}

\questionmarks{3(c)}{7}{Draw block diagram of smart street light control and monitoring system. Discuss advantages and applications of AR/VR technology in industry.}

\begin{solutionbox}
\textbf{Answer}:

\textbf{Block Diagram: Smart Street Light System}

\begin{center}
\begin{tikzpicture}[node distance=1.5cm]
    \node[gtu block] (micro) {Microcontroller};
    
    \node[gtu block, left=of micro] (light) {Light Sensor};
    \node[gtu block, above=of micro] (motion) {Motion Sensor};
    \node[gtu block, below=of micro] (remote) {Remote Control};
    
    \node[gtu block, right=of micro] (driver) {LED Driver};
    \node[gtu state, right=of driver] (led) {LED Street Light};
    
    \node[gtu block, below right=of micro] (wireless) {Wireless Module};
    \node[gtu block, right=of wireless] (central) {Central Control};
    \node[gtu state, right=of central] (dash) {Dashboard};
    
    \draw[gtu arrow] (light) -- (micro);
    \draw[gtu arrow] (motion) -- (micro);
    \draw[gtu arrow] (remote) -- (micro);
    \draw[gtu arrow] (micro) -- (driver);
    \draw[gtu arrow] (driver) -- (led);
    \draw[gtu arrow] (micro) -- (wireless);
    \draw[gtu arrow] (wireless) -- (central);
    \draw[gtu arrow] (central) -- (dash);
\end{tikzpicture}
\captionof{figure}{Smart Street Light Control System}
\end{center}

\textbf{AR/VR Technology in Industry:}

\begin{center}
\captionof{table}{AR/VR Applications}
\begin{tabulary}{\linewidth}{|L|L|L|}
\hline
\textbf{Industry} & \textbf{AR Application} & \textbf{VR Application} \\ \hline
\textbf{Manufacturing} & Assembly instructions & Training simulations \\ \hline
\textbf{Healthcare} & Surgery assistance & Medical training \\ \hline
\textbf{Education} & Interactive learning & Virtual classrooms \\ \hline
\textbf{Retail} & Product visualization & Virtual showrooms \\ \hline
\end{tabulary}
\end{center}

\textbf{Advantages:}
\begin{itemize}
\item \textbf{Enhanced Training}: Safe, repeatable learning environments
\item \textbf{Remote Collaboration}: Virtual meetings and shared workspaces
\item \textbf{Design Visualization}: 3D prototyping and modeling
\item \textbf{Maintenance Support}: Real-time guidance and troubleshooting
\end{itemize}
\end{solutionbox}

\begin{mnemonicbox}
\mnemonic{"AR/VR: Training, Design, Remote, Maintenance"}
\end{mnemonicbox}

\questionmarks{3(a) OR}{3}{What is Smart System? List any four types of smart system.}

\begin{solutionbox}
\textbf{Answer}:
Smart System is an intelligent system that uses sensors, data processing, and automation to make decisions and adapt to changing conditions.

\begin{center}
\captionof{table}{Types of Smart Systems}
\begin{tabulary}{\linewidth}{|L|L|L|}
\hline
\textbf{Type} & \textbf{Description} & \textbf{Example} \\ \hline
\textbf{Smart Home} & Automated home control & Lighting, HVAC, security \\ \hline
\textbf{Smart City} & Urban infrastructure management & Traffic, utilities, waste \\ \hline
\textbf{Smart Grid} & Intelligent power distribution & Energy management \\ \hline
\textbf{Smart Healthcare} & Medical monitoring systems & Patient monitoring, diagnostics \\ \hline
\end{tabulary}
\end{center}
\end{solutionbox}

\begin{mnemonicbox}
\mnemonic{"Smart: Home, City, Grid, Health"}
\end{mnemonicbox}

\questionmarks{3(b) OR}{4}{List the advantages and applications of organic electronics.}

\begin{solutionbox}
\textbf{Answer}:

\begin{center}
\captionof{table}{Advantages of Organic Electronics}
\begin{tabulary}{\linewidth}{|L|L|L|}
\hline
\textbf{Advantage} & \textbf{Description} & \textbf{Benefit} \\ \hline
\textbf{Flexibility} & Bendable, stretchable & Wearable devices \\ \hline
\textbf{Low Cost} & Cheap manufacturing & Mass production \\ \hline
\textbf{Large Area} & Printing on large surfaces & Big displays \\ \hline
\textbf{Low Temperature} & Room temperature processing & Energy efficient \\ \hline
\end{tabulary}
\end{center}

\textbf{Applications:}
\begin{itemize}
\item \textbf{OLED Displays}: Smartphones, TVs, lighting
\item \textbf{Organic Solar Cells}: Flexible solar panels
\item \textbf{Organic Transistors}: Flexible circuits
\item \textbf{Electronic Paper}: E-readers, smart labels
\end{itemize}
\end{solutionbox}

\begin{mnemonicbox}
\mnemonic{"Organic: Flexible, Cheap, Large, Low-temp"}
\end{mnemonicbox}

\questionmarks{3(c) OR}{7}{Draw basic block diagram of (i) wearable smart watch and (ii) biometric system.}

\begin{solutionbox}
\textbf{Answer}:

\textbf{(i) Wearable Smart Watch Block Diagram:}

\begin{center}
\begin{tikzpicture}[node distance=1.5cm]
    \node[gtu block] (proc) {Microprocessor};
    
    \node[gtu block, left=of proc] (sens) {Sensors};
    \node[gtu block, below left=of sens] (hr) {Heart Rate};
    \node[gtu block, left=of sens] (acc) {Accelerometer};
    \node[gtu block, above left=of sens] (gps) {GPS};
    
    \node[gtu block, above=of proc] (disp) {Display};
    \node[gtu block, right=of proc] (mem) {Memory};
    \node[gtu block, below=of proc] (batt) {Battery};
    \node[gtu block, below right=of proc] (charge) {Charging Port};
    \node[gtu block, above right=of proc] (wireless) {Wireless Module};
    
    \draw[gtu arrow] (hr) -- (sens);
    \draw[gtu arrow] (acc) -- (sens);
    \draw[gtu arrow] (gps) -- (sens);
    \draw[gtu arrow] (sens) -- (proc);
    \draw[gtu arrow] (disp) -- (proc);
    \draw[gtu arrow] (batt) -- (proc);
    \draw[gtu arrow] (wireless) -- (proc);
    \draw[gtu arrow] (proc) -- (mem);
    \draw[gtu arrow] (proc) -- (charge);
\end{tikzpicture}
\captionof{figure}{Smart Watch Architecture}
\end{center}

\textbf{(ii) Biometric System Block Diagram:}

\begin{center}
\begin{tikzpicture}[node distance=1.2cm]
    \node[gtu block] (sensor) {Biometric Sensor};
    \node[gtu block, right=of sensor] (signal) {Signal Processing};
    \node[gtu block, right=of signal] (feat) {Feature Extraction};
    \node[gtu block, below=of feat] (match) {Template Matching};
    \node[gtu block, left=of match] (db) {Database};
    \node[gtu block, left=of db] (enroll) {Enrollment Module};
    \node[gtu decision, right=of match] (decide) {Decision Module};
    \node[gtu state, below=of decide] (access) {Access Control};

    \draw[gtu arrow] (sensor) -- (signal);
    \draw[gtu arrow] (signal) -- (feat);
    \draw[gtu arrow] (feat) -- (match);
    \draw[gtu arrow] (enroll) -- (db);
    \draw[gtu arrow] (db) -- (match);
    \draw[gtu arrow] (match) -- (decide);
    \draw[gtu arrow] (decide) -- (access);
\end{tikzpicture}
\captionof{figure}{Biometric System}
\end{center}

\textbf{Components:}
\begin{itemize}
\item \textbf{Smart Watch}: Sensors (HR, Accel, GPS), Processor (ARM), Display (OLED), Connectivity, Power.
\item \textbf{Biometric}: Sensor, Processing Unit, Database, Matching Engine, Decision Logic.
\end{itemize}
\end{solutionbox}

\begin{mnemonicbox}
\mnemonic{"Smart Watch: Sense, Process, Display, Connect" / "Biometric: Capture, Process, Match, Decide"}
\end{mnemonicbox}

\questionmarks{4(a)}{3}{Give full form of NOOBS, GPIO & LXDE in raspberry pi.}

\begin{solutionbox}
\textbf{Answer}:

\begin{center}
\captionof{table}{Raspberry Pi Acronyms}
\begin{tabulary}{\linewidth}{|L|L|L|}
\hline
\textbf{Acronym} & \textbf{Full Form} & \textbf{Purpose} \\ \hline
\textbf{NOOBS} & New Out Of Box Software & Easy OS installation \\ \hline
\textbf{GPIO} & General Purpose Input Output & Hardware interface pins \\ \hline
\textbf{LXDE} & Lightweight X11 Desktop Environment & Desktop interface \\ \hline
\end{tabulary}
\end{center}
\end{solutionbox}

\begin{mnemonicbox}
\mnemonic{"New GPIO, Lightweight Experience"}
\end{mnemonicbox}

\questionmarks{4(b)}{4}{Write a short note on OLED.}

\begin{solutionbox}
\textbf{Answer}:
OLED (Organic Light Emitting Diode) is a display technology using organic compounds that emit light when electric current is applied.

\begin{center}
\captionof{table}{OLED vs LCD}
\begin{tabulary}{\linewidth}{|L|L|L|}
\hline
\textbf{Parameter} & \textbf{OLED} & \textbf{LCD} \\ \hline
\textbf{Backlight} & Not required & Required \\ \hline
\textbf{Contrast} & Infinite & 1000:1 \\ \hline
\textbf{Thickness} & Ultra-thin & Thicker \\ \hline
\textbf{Power} & Lower (dark images) & Constant \\ \hline
\end{tabulary}
\end{center}

\textbf{Applications:}
\begin{itemize}
\item \textbf{Smartphones}: Samsung, iPhone displays
\item \textbf{TVs}: Premium television sets
\item \textbf{Automotive}: Dashboard displays
\item \textbf{Wearables}: Smartwatch screens
\end{itemize}
\end{solutionbox}

\begin{mnemonicbox}
\mnemonic{"OLED: Organic, Light, Emitting, Display"}
\end{mnemonicbox}

\questionmarks{4(c)}{7}{Explain the architecture and block diagram of Raspberry Pi.}

\begin{solutionbox}
\textbf{Answer}:

\textbf{Block Diagram: Raspberry Pi Architecture}

\begin{center}
\begin{tikzpicture}[node distance=1.5cm]
    \node[gtu block] (cpu) {ARM Cortex CPU};
    \node[gtu block, below=of cpu] (bus) {System Bus};
    \node[gtu block, left=of bus] (gpu) {GPU};
    \node[gtu block, right=of bus] (ram) {RAM};
    
    \node[gtu block, below left=of bus] (sd) {SD Card Slot};
    \node[gtu block, below right=of bus] (gpio) {GPIO Pins};
    
    \node[gtu block, below=of sd] (usb) {USB Ports};
    \node[gtu block, below=of gpio] (eth) {Ethernet};
    
    \node[gtu block, right=of ram] (hdmi) {HDMI};
    \node[gtu block, left=of gpu] (cam) {Camera Interface};
    
    \draw[gtu arrow] (cpu) -- (bus);
    \draw[gtu arrow] (gpu) -- (bus);
    \draw[gtu arrow] (ram) -- (bus);
    \draw[gtu arrow] (sd) -- (bus);
    \draw[gtu arrow] (bus) -- (gpio);
    \draw[gtu arrow] (bus) -- (usb);
    \draw[gtu arrow] (bus) -- (eth);
    \draw[gtu arrow] (bus) -- (hdmi);
    \draw[gtu arrow] (bus) -- (cam);
\end{tikzpicture}
\captionof{figure}{Raspberry Pi Architecture}
\end{center}

\textbf{Key Components:}
\begin{itemize}
\item \textbf{CPU}: ARM Cortex-A72 Quad-core (Main processing)
\item \textbf{GPU}: VideoCore VI (Graphics processing)
\item \textbf{RAM}: 4GB LPDDR4 (System memory)
\item \textbf{Storage}: MicroSD card (Operating system)
\item \textbf{GPIO}: 40-pin header (Hardware interface)
\item \textbf{Connectivity}: WiFi, Bluetooth, Ethernet (Network access)
\end{itemize}
\end{solutionbox}

\begin{mnemonicbox}
\mnemonic{"Pi: Processor, Interfaces, Projects, Internet"}
\end{mnemonicbox}

\questionmarks{4(a) OR}{3}{What is Raspberry Pi and its advantages and disadvantages?}

\begin{solutionbox}
\textbf{Answer}:
Raspberry Pi is a small, affordable single-board computer designed for education and hobbyist projects.

\begin{center}
\captionof{table}{Advantages and Disadvantages}
\begin{tabulary}{\linewidth}{|L|L|}
\hline
\textbf{Advantages} & \textbf{Disadvantages} \\ \hline
Low Cost & Limited Performance \\ \hline
Small Size & No Built-in Storage \\ \hline
GPIO Pins & Requires SD Card \\ \hline
Linux Support & No Real-time OS \\ \hline
Educational & Power Supply Issues \\ \hline
Community Support & Limited RAM \\ \hline
\end{tabulary}
\end{center}
\end{solutionbox}

\begin{mnemonicbox}
\mnemonic{"Pi: Cheap, Small, Educational vs Limited, External, Power"}
\end{mnemonicbox}

\questionmarks{4(b) OR}{4}{Write a short note on OFET.}

\begin{solutionbox}
\textbf{Answer}:
OFET (Organic Field Effect Transistor) is a transistor using organic semiconducting materials for switching and amplification.

\begin{center}
\captionof{table}{OFET Structure}
\begin{tabulary}{\linewidth}{|L|L|L|}
\hline
\textbf{Component} & \textbf{Material} & \textbf{Function} \\ \hline
\textbf{Gate} & Metal electrode & Controls current flow \\ \hline
\textbf{Dielectric} & Insulating layer & Isolates gate from channel \\ \hline
\textbf{Source/Drain} & Metal contacts & Current injection/collection \\ \hline
\textbf{Channel} & Organic semiconductor & Current conduction path \\ \hline
\end{tabulary}
\end{center}

\textbf{Applications:}
\begin{itemize}
\item \textbf{Flexible Displays}: Bendable screens
\item \textbf{Smart Cards}: RFID applications
\item \textbf{Sensors}: Chemical and biological detection
\end{itemize}
\end{solutionbox}

\begin{mnemonicbox}
\mnemonic{"OFET: Organic, Flexible, Easy, Transistor"}
\end{mnemonicbox}

\questionmarks{4(c) OR}{7}{List the types of Ports in Raspberry Pi. Discuss various operating systems of raspberry Pi.}

\begin{solutionbox}
\textbf{Answer}:

\begin{center}
\captionof{table}{Raspberry Pi Ports}
\begin{tabulary}{\linewidth}{|L|L|L|}
\hline
\textbf{Port Type} & \textbf{Quantity} & \textbf{Function} \\ \hline
\textbf{USB} & 4 ports & Connect peripherals \\ \hline
\textbf{HDMI} & 2 micro HDMI & Video output \\ \hline
\textbf{GPIO} & 40 pins & Hardware interface \\ \hline
\textbf{Ethernet} & 1 port & Wired network \\ \hline
\textbf{Audio} & 3.5mm jack & Audio output \\ \hline
\textbf{Camera/Display} & CSI/DSI & Module interfaces \\ \hline
\end{tabulary}
\end{center}

\textbf{Operating Systems:}

\begin{center}
\captionof{table}{Raspberry Pi Operating Systems}
\begin{tabulary}{\linewidth}{|L|L|L|}
\hline
\textbf{OS} & \textbf{Type} & \textbf{Best For} \\ \hline
\textbf{Raspberry Pi OS} & Debian-based & General use, beginners \\ \hline
\textbf{Ubuntu} & Linux distribution & Server applications \\ \hline
\textbf{LibreELEC} & Media center & Home entertainment \\ \hline
\textbf{RetroPie} & Gaming & Retro gaming console \\ \hline
\textbf{Windows 10 IoT} & Microsoft OS & IoT development \\ \hline
\end{tabulary}
\end{center}
\end{solutionbox}

\begin{mnemonicbox}
\mnemonic{"Pi Ports: USB, HDMI, GPIO, Ethernet" / "Pi OS: Official, Ubuntu, Media, Gaming"}
\end{mnemonicbox}

\questionmarks{5(a)}{3}{Explain NumPy python library For Machine Learning.}

\begin{solutionbox}
\textbf{Answer}:
NumPy (Numerical Python) is a fundamental library for scientific computing, providing support for large multi-dimensional arrays and mathematical functions.

\begin{center}
\captionof{table}{NumPy in Machine Learning}
\begin{tabulary}{\linewidth}{|L|L|L|}
\hline
\textbf{Function} & \textbf{Usage} & \textbf{Example} \\ \hline
\textbf{Arrays} & Data storage & \code{np.array([1,2,3])} \\ \hline
\textbf{Linear Algebra} & Matrix operations & \code{np.dot(a,b)} \\ \hline
\textbf{Statistics} & Data analysis & \code{np.mean()}, \code{np.std()} \\ \hline
\textbf{Random} & Data generation & \code{np.random.rand()} \\ \hline
\end{tabulary}
\end{center}

\textbf{Key Features:}
\begin{itemize}
\item \textbf{N-dimensional Arrays}: Efficient array operations
\item \textbf{Mathematical Functions}: Linear algebra, Fourier transforms
\item \textbf{Memory Efficient}: Faster than Python lists
\end{itemize}
\end{solutionbox}

\begin{mnemonicbox}
\mnemonic{"NumPy: Numbers, Python, Arrays, Math"}
\end{mnemonicbox}

\questionmarks{5(b)}{4}{What is organic photovoltaic cell (OPV)? Explain its working principle.}

\begin{solutionbox}
\textbf{Answer}:
OPV (Organic Photovoltaic) cell is a solar cell using organic semiconductors to convert light into electricity.

\begin{center}
\captionof{figure}{OPV Working Principle}
\begin{tikzpicture}[node distance=1.5cm]
    \node[gtu block] (sun) {Sunlight};
    \node[gtu block, right=of sun] (absorb) {Organic Active Layer};
    \node[gtu block, right=of absorb] (exciton) {Exciton Generation};
    \node[gtu block, below=of exciton] (sep) {Charge Separation};
    \node[gtu block, left=of sep] (trans) {Electron Transport};
    \node[gtu block, left=of trans] (curr) {Current Collection};
    
    \draw[gtu arrow] (sun) -- (absorb);
    \draw[gtu arrow] (absorb) -- (exciton);
    \draw[gtu arrow] (exciton) -- (sep);
    \draw[gtu arrow] (sep) -- (trans);
    \draw[gtu arrow] (trans) -- (curr);
\end{tikzpicture}
\end{center}

\textbf{Key Steps:}
\begin{itemize}
\item \textbf{Absorption}: Organic molecules absorb photons
\item \textbf{Exciton}: Bound electron-hole pairs created
\item \textbf{Separation}: Excitons split at donor-acceptor interface
\item \textbf{Transport}: Electrons and holes move to electrodes
\item \textbf{Collection}: External circuit completes the flow
\end{itemize}
\end{solutionbox}

\begin{mnemonicbox}
\mnemonic{"OPV: Organic, Photons, Voltage, Excitons"}
\end{mnemonicbox}

\questionmarks{5(c)}{7}{List any four Machine learning tools. Discuss any one in brief.}

\begin{solutionbox}
\textbf{Answer}:

\begin{center}
\captionof{table}{Machine Learning Tools}
\begin{tabulary}{\linewidth}{|L|L|L|}
\hline
\textbf{Tool} & \textbf{Type} & \textbf{Best For} \\ \hline
\textbf{TensorFlow} & Deep learning framework & Neural networks \\ \hline
\textbf{Scikit-learn} & General ML library & Traditional algorithms \\ \hline
\textbf{PyTorch} & Deep learning framework & Research and development \\ \hline
\textbf{Keras} & High-level API & Rapid prototyping \\ \hline
\end{tabulary}
\end{center}

\textbf{Detailed Discussion: TensorFlow}
TensorFlow is an open-source machine learning framework developed by Google for building and deploying ML models.

\textbf{Features:}
\begin{itemize}
\item \textbf{Tensors}: Multi-dimensional arrays for data representation
\item \textbf{Graphs}: Computational flow for model visualization
\item \textbf{Flexibility}: Research to production versatility
\end{itemize}

\textbf{Code Example:}
\begin{lstlisting}[language=Python]
import tensorflow as tf
model = tf.keras.Sequential([
    tf.keras.layers.Dense(128, activation='relu'),
    tf.keras.layers.Dense(10, activation='softmax')
])
\end{lstlisting}
\end{solutionbox}

\begin{mnemonicbox}
\mnemonic{"TensorFlow: Tensors, Graphs, Scale, Deploy"}
\end{mnemonicbox}

\questionmarks{5(a) OR}{3}{Explain Pandas python library For Machine Learning.}

\begin{solutionbox}
\textbf{Answer}:
Pandas is a Python library for data manipulation and analysis, providing data structures and tools for handling structured data.

\begin{center}
\captionof{table}{Pandas Functions}
\begin{tabulary}{\linewidth}{|L|L|L|}
\hline
\textbf{Function} & \textbf{Usage} & \textbf{Example} \\ \hline
\textbf{Data Loading} & Import datasets & \code{pd.read\_csv()} \\ \hline
\textbf{Data Cleaning} & Remove/fill missing & \code{df.dropna()} \\ \hline
\textbf{Data Selection} & Filter data & \code{df[df['col'] > 5]} \\ \hline
\textbf{Aggregation} & Group and summarize & \code{df.groupby().mean()} \\ \hline
\end{tabulary}
\end{center}
\end{solutionbox}

\begin{mnemonicbox}
\mnemonic{"Pandas: Python, Analysis, Data, Structure"}
\end{mnemonicbox}

\questionmarks{5(b) OR}{4}{Explain the Differences between augmented reality and virtual reality.}

\begin{solutionbox}
\textbf{Answer}:

\begin{center}
\captionof{table}{AR vs VR Comparison}
\begin{tabulary}{\linewidth}{|L|L|L|}
\hline
\textbf{Parameter} & \textbf{Augmented Reality (AR)} & \textbf{Virtual Reality (VR)} \\ \hline
\textbf{Environment} & Real world + digital overlay & Completely virtual world \\ \hline
\textbf{Device} & Smartphone, AR glasses & VR headset, controllers \\ \hline
\textbf{Immersion} & Partial immersion & Full immersion \\ \hline
\textbf{Interaction} & Real world + digital objects & Virtual objects only \\ \hline
\textbf{Example} & Pokemon Go, Google Maps AR & Oculus Quest, Flight Sims \\ \hline
\end{tabulary}
\end{center}
\end{solutionbox}

\begin{mnemonicbox}
\mnemonic{"AR: Augments Reality vs VR: Virtual Reality"}
\end{mnemonicbox}

\questionmarks{5(c) OR}{7}{What is Machine learning? Discuss various types of Machine learning.}

\begin{solutionbox}
\textbf{Answer}:
Machine Learning is a subset of artificial intelligence that enables computers to learn and make decisions from data without being explicitly programmed.

\textbf{Supervised Learning Process:}

\begin{center}
\begin{tikzpicture}[node distance=1.5cm]
    \node[gtu block] (data) {Training Data};
    \node[gtu block, right=of data] (algo) {Algorithm};
    \node[gtu block, right=of algo] (model) {Model};
    
    \node[gtu block, below=of model] (new) {New Data};
    \node[gtu block, right=of model] (pred) {Prediction};
    
    \draw[gtu arrow] (data) -- (algo);
    \draw[gtu arrow] (algo) -- (model);
    \draw[gtu arrow] (new) -- (model);
    \draw[gtu arrow] (model) -- (pred);
\end{tikzpicture}
\captionof{figure}{Supervised Learning Flow}
\end{center}

\textbf{Types of Machine Learning:}

\begin{center}
\captionof{table}{ML Types}
\begin{tabulary}{\linewidth}{|L|L|L|}
\hline
\textbf{Type} & \textbf{Description} & \textbf{Use Cases} \\ \hline
\textbf{Supervised} & Learns from labeled data & Spam, Price prediction \\ \hline
\textbf{Unsupervised} & Finds patterns in unlabeled data & Customer segmentation \\ \hline
\textbf{Reinforcement} & Learns through trial and error & Robotics, Game playing \\ \hline
\end{tabulary}
\end{center}
\end{solutionbox}

\begin{mnemonicbox}
\mnemonic{"ML Types: Supervised teaches, Unsupervised discovers, Reinforcement rewards"}
\end{mnemonicbox}
\end{document}
