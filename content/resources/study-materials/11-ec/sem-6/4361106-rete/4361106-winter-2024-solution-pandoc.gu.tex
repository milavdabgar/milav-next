\documentclass[10pt,a4paper]{article}

% content/resources/templates/preamble.tex
\usepackage[margin=0.6in]{geometry}
\author{Milav Dabgar}
\usepackage{amsmath,amssymb,amsthm}
\usepackage{booktabs}
\usepackage{multirow}
\usepackage{xcolor}
\usepackage{tcolorbox}
\tcbuselibrary{breakable,skins}
\usepackage[colorlinks=true,linkcolor=blue]{hyperref}
\usepackage{titlesec}
\usepackage{enumitem}
\usepackage{tikz}
\usepackage{pgfplots}
\usepackage{circuitikz}
\usepackage[version=4]{mhchem}
\usepackage{longtable}
\usepackage{array}
\usepackage{float}
\usepackage{caption}
\usepackage{listings}

\lstset{
  basicstyle=\small\ttfamily,
  breaklines=true,
  breakatwhitespace=false,
  postbreak=\mbox{\textcolor{red}{$\hookrightarrow$}\space},
  float=false,
  numbers=left,
  numberstyle=\tiny\color{gray},
  numbersep=10pt,
  xleftmargin=2em,
  keywordstyle=\color{blue},
  commentstyle=\color{green!60!black},
  stringstyle=\color{purple},
  backgroundcolor=\color{gray!5},
  showstringspaces=false,
  tabsize=2,
  captionpos=b,
  keepspaces=true,
  columns=flexible
}

\pgfplotsset{compat=1.18}
\usetikzlibrary{shapes,arrows,positioning,calc,patterns,decorations.pathmorphing,decorations.markings,arrows.meta}

% Color scheme
\definecolor{headcolor}{RGB}{0,102,204}
\definecolor{keycolor}{RGB}{220,20,60}
\definecolor{solutioncolor}{RGB}{34,139,34}
\definecolor{mnemoniccolor}{RGB}{148,0,211}
\definecolor{codecolor}{RGB}{0,0,100}

% Spacing
\setlength{\parskip}{3pt}
\setlist[itemize]{nosep}
\setlist[enumerate]{nosep}

% Title formatting
\titleformat{\section}{\Large\bfseries\color{headcolor}}{\thesection}{1em}{}
\titleformat{\subsection}{\large\bfseries\color{headcolor}}{\thesubsection}{1em}{}

% Pandoc tightlist compatibility
\providecommand{\tightlist}{%
  \setlength{\itemsep}{0pt}\setlength{\parskip}{0pt}}

% Pandoc longtable compatibility
\newcounter{none}
\def\thenone{}


% content/resources/templates/gujarati-boxes.tex
\usepackage{fontspec}
\usepackage{polyglossia}

% Set Gujarati as main language (document is primarily in Gujarati)
% Note: gloss-gujarati.ldf doesn't exist in polyglossia, but it will use hyphenation patterns
\setdefaultlanguage{gujarati}
\setotherlanguage{english}

% Configure Gujarati font properly
% Use Language=Default to prevent polyglossia from trying to add language-specific features
% that don't exist for Gujarati, which causes "empty feature" warnings
\newfontfamily\gujaratifont[Script=Gujarati,AutoFakeBold=2.5,AutoFakeSlant=0.3]{Noto Sans Gujarati}
\setmainfont[Script=Gujarati,AutoFakeBold=2.5,AutoFakeSlant=0.3]{Noto Sans Gujarati}
% Use Noto Sans Gujarati for monospace to support Gujarati in text
\setmonofont[Scale=0.9]{Noto Sans Gujarati}

% Configure English to use the same font
\newfontfamily\englishfont[Script=Gujarati,AutoFakeBold=2.5,AutoFakeSlant=0.3]{Noto Sans Gujarati}

% Translations for polyglossia
\gappto\captionsgujarati{
  \renewcommand{\tablename}{કોષ્ટક}
  \renewcommand{\figurename}{આકૃતિ}
}

% Helper for TikZ nodes to ensure Gujarati font
\newcommand{\gu}[1]{{\gujaratifont #1}}

% Custom environments
\newtcolorbox{solutionbox}{
    breakable,
    enhanced,
    colback=solutioncolor!5!white,
    colframe=solutioncolor!75!black,
    fonttitle=\bfseries,
    title=જવાબ
}

\newtcolorbox{solutionboxnobreak}{
 colback=solutioncolor!5!white,
 colframe=solutioncolor!75!black,
 fonttitle=\bfseries,
 title=જવાબ
}

\newtcolorbox{keyformula}{
 breakable,
 enhanced,
 colback=keycolor!5!white,
 colframe=keycolor!75!black,
 fonttitle=\bfseries,
 title=રાસાયણિક સમીકરણ/સૂત્ર
}

\newtcolorbox{mnemonicbox}{
 breakable,
 enhanced,
 colback=mnemoniccolor!5!white,
 colframe=mnemoniccolor!75!black,
 fonttitle=\bfseries,
 title=મેમરી ટ્રીક
}


\begin{document}

\begin{center}
{\Huge\bfseries\color{headcolor} Subject Name (Gujarati)}\\[5pt]
{\LARGE 4361106 -- Winter 2024}\\[3pt]
{\large Semester 1 Study Material}\\[3pt]
{\normalsize\textit{Detailed Solutions and Explanations}}
\end{center}

\vspace{10pt}

\subsection*{પ્રશ્ન 1(અ) [3
ગુણ]}\label{uxaaauxab0uxab6uxaa8-1uxa85-3-uxa97uxaa3}

\textbf{વિવિધ પ્રકારના નવીનીકરણીય ઉર્જા સ્રોતોની યાદી બનાવો અને કોઈપણ એકને
વિગતવાર સમજાવો.}

\begin{solutionbox}


{\def\LTcaptype{none} % do not increment counter
\vspace{-5pt}
\captionof{table}{નવીનીકરણીય ઉર્જા સ્રોતોના પ્રકારો}
\vspace{-10pt}
\begin{longtable}[]{@{}lll@{}}
\toprule\noalign{}
પ્રકાર & સ્રોત & ઉપયોગ \\
\midrule\noalign{}
\endhead
\bottomrule\noalign{}
\endlastfoot
સૌર & સૂર્યનું કિરણોત્સર્ગ & સોલાર પેનલ, હીટિંગ \\
પવન & હવાની હલનચલન & વિન્ડ ટર્બાઇન \\
જલવિદ્યુત & વહેતું પાણી & ડેમ, ટર્બાઇન \\
બાયોમાસ & કાર્બનિક પદાર્થ & બાયોફ્યુઅલ, હીટિંગ \\
ભૂઉષ્મીય & પૃથ્વીની ગરમી & પાવર પ્લાન્ટ, હીટિંગ \\
\end{longtable}
}

\textbf{સૌર ઉર્જા સમજૂતી}:

\begin{itemize}
\tightlist
\item
  \textbf{ફોટોવોલ્ટેઇક અસર}: સિલિકોન સેલ વાપરીને સૂર્યપ્રકાશને સીધો વીજળીમાં ફેરવે
  છે
\item
  \textbf{ફાયદાઓ}: સ્વચ્છ, વિપુલ, નવીનીકરણીય
\item
  \textbf{ઉપયોગો}: છત પરની સિસ્ટમ, સોલાર ફાર્મ
\end{itemize}

\end{solutionbox}
\begin{mnemonicbox}
``SWHBG - સૂર્ય વિજય હાંસલ કરે ભલાઈથી જઈને''

\end{mnemonicbox}
\begin{center}\rule{0.5\linewidth}{0.5pt}\end{center}

\subsection*{પ્રશ્ન 1(બ) [4
ગુણ]}\label{uxaaauxab0uxab6uxaa8-1uxaac-4-uxa97uxaa3}

\textbf{વિવિધ પ્રકારના સોલાર સેલની યાદી બનાવો અને કોઈપણ બેને સમજાવો.}

\begin{solutionbox}


{\def\LTcaptype{none} % do not increment counter
\vspace{-5pt}
\captionof{table}{સોલાર સેલના પ્રકારો}
\vspace{-10pt}
\begin{longtable}[]{@{}llll@{}}
\toprule\noalign{}
પ્રકાર & કાર્યક્ષમતા & કિંમત & ઉપયોગ \\
\midrule\noalign{}
\endhead
\bottomrule\noalign{}
\endlastfoot
સિલિકોન & 15-20\% & મધ્યમ & રહેણાંક \\
મોનોક્રિસ્ટેલાઇન & 18-22\% & ઊંચી & પ્રીમિયમ સિસ્ટમ \\
પોલીક્રિસ્ટેલાઇન & 15-17\% & ઓછી & બજેટ સિસ્ટમ \\
થિન ફિલ્મ & 10-12\% & ખૂબ ઓછી & મોટા ઇન્સ્ટોલેશન \\
એમોર્ફસ સિલિકોન & 6-8\% & ઓછી & નાના ઉપકરણો \\
\end{longtable}
}

\textbf{મોનોક્રિસ્ટેલાઇન સિલિકોન}:

\begin{itemize}
\tightlist
\item
  \textbf{બંધારણ}: એકસાર ક્રિસ્ટલ બંધારણ સાથે એકસમાન દેખાવ
\item
  \textbf{કાર્યક્ષમતા}: સિલિકોન સેલમાં સૌથી વધુ (18-22\%)
\end{itemize}

\textbf{પોલીક્રિસ્ટેલાઇન સિલિકોન}:

\begin{itemize}
\tightlist
\item
  \textbf{બંધારણ}: નીલા ડાઘવાળા દેખાવ સાથે બહુવિધ ક્રિસ્ટલ
\item
  \textbf{કિંમત}: મોનોક્રિસ્ટેલાઇન કરતાં ઓછી ઉત્પાદન કિંમત
\end{itemize}

\end{solutionbox}
\begin{mnemonicbox}
``મારા પોલી થિન એમ્પ - મોસ્ટ પોપ્યુલર ટાઇપ્સ અવેઇલેબલ''

\end{mnemonicbox}
\begin{center}\rule{0.5\linewidth}{0.5pt}\end{center}

\subsection*{પ્રશ્ન 1(ક) [7
ગુણ]}\label{uxaaauxab0uxab6uxaa8-1uxa95-7-uxa97uxaa3}

\textbf{હોમ સોલાર રૂફટોપ સિસ્ટમનો બ્લોક ડાયાગ્રામ દોરો અને સમજાવો.}

\begin{solutionbox}

\begin{verbatim}
                    ┌─────────────┐
                    │ Solar Panels│
                    │  (PV Array) │
                    └──────┬──────┘
                           │ DC Power
                           ▼
                    ┌─────────────┐
                    │   Inverter  │
                    │  (DC to AC) │
                    └──────┬──────┘
                           │ AC Power
                           ▼
                    ┌───────────────┐
                    │    Meter      │
                    │(Bidirectional)│
                    └──────┬────────┘
                           │
                  ┌────────┼────────┐
                  ▼                 ▼
            ┌──────────┐      ┌──────────┐
            │Home Load │      │   Grid   │
            │          │      │Connection│
            └──────────┘      └──────────┘
\end{verbatim}

\textbf{ઘટકોની સમજૂતી}:

\begin{itemize}
\tightlist
\item
  \textbf{સોલાર પેનલ}: ફોટોવોલ્ટેઇક અસર વાપરીને સૂર્યપ્રકાશને DC વીજળીમાં ફેરવે છે
\item
  \textbf{ઇન્વર્ટર}: ઘરના ઉપયોગ માટે DC પાવરને AC પાવરમાં ફેરવે છે
\item
  \textbf{દ્વિદિશીય મીટર}: પાવર વપરાશ અને ગ્રિડમાં ફીડ થતી વધારાની પાવર માપે
  છે
\item
  \textbf{ઘરનો લોડ}: વિદ્યુત ઉપકરણો અને ડિવાઇસ
\item
  \textbf{ગ્રિડ કનેક્શન}: બેકઅપ અને વધારાની પાવર વેચવા માટે યુટિલિટી ગ્રિડ સાથે
  જોડાય છે
\end{itemize}

\textbf{કાર્ય સિદ્ધાંત}:

\begin{itemize}
\tightlist
\item
  \textbf{દિવસનું ચાલન}: સોલાર પેનલ વીજળી ઉત્પન્ન કરે છે, ઇન્વર્ટર AC માં ફેરવે છે
\item
  \textbf{વધારાની પાવર}: નેટ મીટરિંગ દ્વારા ગ્રિડમાં પાછી ફીડ કરવામાં આવે છે
\item
  \textbf{રાત્રિનું ચાલન}: જ્યારે સોલાર ઉપલબ્ધ ન હોય ત્યારે ગ્રિડમાંથી પાવર લેવામાં
  આવે છે
\end{itemize}

\end{solutionbox}
\begin{mnemonicbox}
``સોલાર ઇન્વર્ટર મીટર હોમ ગ્રિડ - સિમ્પલ ઇન્સ્ટોલેશન મેક્સ
હેપ્પી જનરેશન''

\end{mnemonicbox}
\begin{center}\rule{0.5\linewidth}{0.5pt}\end{center}

\subsection*{પ્રશ્ન 1(ક) અથવા [7
ગુણ]}\label{uxaaauxab0uxab6uxaa8-1uxa95-uxa85uxaa5uxab5-7-uxa97uxaa3}

\textbf{સૌર ફોટોવોલ્ટેઇક અસર અને ફોટોવોલ્ટેઇક રૂપાંતરનો સિદ્ધાંત આકૃતિ સાથે
સમજાવો.}

\begin{solutionbox}

\begin{verbatim}
      Sunlight (Photons)
            ↓
    ┌─────────────────┐
    │   N{-Type Layer  │ (Negative)}
    │  (Phosphorus)   │
    ├─────────────────┤  P{-N Junction}
    │   P{-Type Layer  │ (Positive)}
    │    (Boron)      │
    └─────────────────┘
            │
    ┌───────┴───────┐
    │   External    │
    │    Circuit    │
    └───────────────┘
\end{verbatim}

\textbf{ફોટોવોલ્ટેઇક અસર પ્રક્રિયા}:

\begin{itemize}
\tightlist
\item
  \textbf{ફોટોન શોષણ}: સૌર ફોટોન સિલિકોન અણુઓ સાથે ટકરાય છે
\item
  \textbf{ઇલેક્ટ્રોન ઉત્તેજના}: ઇલેક્ટ્રોન ઊર્જા મેળવે છે અને કન્ડક્શન બેન્ડમાં જાય છે
\item
  \textbf{ચાર્જ વિભાજન}: P-N જંકશન વિદ્યુત ક્ષેત્ર બનાવે છે
\item
  \textbf{કરંટ પ્રવાહ}: ઇલેક્ટ્રોન બાહ્ય સર્કિટ દ્વારા વહે છે
\end{itemize}

\textbf{મુખ્ય પેરામીટર}:

\begin{itemize}
\tightlist
\item
  \textbf{બેન્ડ ગેપ}: વેલેન્સ અને કન્ડક્શન બેન્ડ વચ્ચેનો ઊર્જા તફાવત
\item
  \textbf{ઓપન સર્કિટ વોલ્ટેજ}: જ્યારે કોઈ કરંટ વહેતો ન હોય ત્યારે મહત્તમ વોલ્ટેજ
\item
  \textbf{શોર્ટ સર્કિટ કરંટ}: જ્યારે ટર્મિનલ શોર્ટ હોય ત્યારે મહત્તમ કરંટ
\end{itemize}

\textbf{રૂપાંતર કાર્યક્ષમતા}:

\begin{itemize}
\tightlist
\item
  \textbf{સૈદ્ધાંતિક મહત્તમ}: સિંગલ જંકશન સેલ માટે \textasciitilde33\%
\item
  \textbf{વ્યવહારિક કાર્યક્ષમતા}: વાણિજ્યિક સેલ માટે 15-22\%
\end{itemize}

\end{solutionbox}
\begin{mnemonicbox}
``ફોટોન્સ પુશ ઇલેક્ટ્રોન્સ પાસ્ટ જંકશન - પાવર પ્રોડક્શન
પરફેક્ટલી પ્લાન્ડ''

\end{mnemonicbox}
\begin{center}\rule{0.5\linewidth}{0.5pt}\end{center}

\subsection*{પ્રશ્ન 2(અ) [3
ગુણ]}\label{uxaaauxab0uxab6uxaa8-2uxa85-3-uxa97uxaa3}

\textbf{નેનો ટેકનોલોજી શું છે? તેની એપ્લિકેશનોની સૂચિ બનાવો.}

\begin{solutionbox}

\textbf{વ્યાખ્યા}: નેનો ટેકનોલોજી એ પરમાણુ અને આણવિક સ્તરે (1-100 નેનોમીટર)
પદાર્થની હેરફેર છે.


{\def\LTcaptype{none} % do not increment counter
\vspace{-5pt}
\captionof{table}{નેનો ટેકનોલોજીના ઉપયોગો}
\vspace{-10pt}
\begin{longtable}[]{@{}lll@{}}
\toprule\noalign{}
ક્ષેત્ર & ઉપયોગ & ફાયદો \\
\midrule\noalign{}
\endhead
\bottomrule\noalign{}
\endlastfoot
ઇલેક્ટ્રોનિક્સ & ટ્રાન્ઝિસ્ટર, મેમોરી & લઘુકરણ \\
દવા & ડ્રગ ડિલિવરી, ઇમેજિંગ & લક્ષિત સારવાર \\
ઊર્જા & સોલાર સેલ, બેટરી & ઉચ્ચ કાર્યક્ષમતા \\
સામગ્રી & કોમ્પોઝિટ, કોટિંગ & વધારેલા ગુણધર્મો \\
પર્યાવરણ & પાણીની શુદ્ધિકરણ & સ્વચ્છ તકનીક \\
\end{longtable}
}

\textbf{મુખ્ય લક્ષણો}:

\begin{itemize}
\tightlist
\item
  \textbf{સ્કેલ}: 1 નેનોમીટર = 10^{-}^{9} મીટર
\item
  \textbf{ગુણધર્મો}: નેનોસ્કેલ પર અલગ ગુણધર્મો
\item
  \textbf{ઉપયોગો}: આંતરશાખીય તકનીક
\end{itemize}

\end{solutionbox}
\begin{mnemonicbox}
``નેનો મેક્સ એવરીથિંગ મોર એફિશિયન્ટ''

\end{mnemonicbox}
\begin{center}\rule{0.5\linewidth}{0.5pt}\end{center}

\subsection*{પ્રશ્ન 2(બ) [4
ગુણ]}\label{uxaaauxab0uxab6uxaa8-2uxaac-4-uxa97uxaa3}

\textbf{વિવિધ પ્રકારની EV ટેકનોલોજીની યાદી બનાવો અને કોઈપણ બેને સમજાવો.}

\begin{solutionbox}


{\def\LTcaptype{none} % do not increment counter
\vspace{-5pt}
\captionof{table}{EV ટેકનોલોજીના પ્રકારો}
\vspace{-10pt}
\begin{longtable}[]{@{}llll@{}}
\toprule\noalign{}
પ્રકાર & પૂરું નામ & પાવર સ્રોત & રેન્જ \\
\midrule\noalign{}
\endhead
\bottomrule\noalign{}
\endlastfoot
BEV & બેટરી ઇલેક્ટ્રિક વ્હિકલ & માત્ર બેટરી & 150-400 કિમી \\
HEV & હાઇબ્રિડ ઇલેક્ટ્રિક વ્હિકલ & એન્જિન + બેટરી & 600+ કિમી \\
PHEV & પ્લગ-ઇન હાઇબ્રિડ ઇલેક્ટ્રિક & એન્જિન + બેટરી & 50-80 કિમી ઇલેક્ટ્રિક \\
FCEV & ફ્યુઅલ સેલ ઇલેક્ટ્રિક વ્હિકલ & હાઇડ્રોજન ફ્યુઅલ સેલ & 400-600 કિમી \\
\end{longtable}
}

\textbf{બેટરી ઇલેક્ટ્રિક વ્હિકલ (BEV)}:

\begin{itemize}
\tightlist
\item
  \textbf{પાવર સ્રોત}: માત્ર રિચાર્જેબલ બેટરી પેક
\item
  \textbf{ચાલન}: શૂન્ય ઉત્સર્જન સાથે સંપૂર્ણ ઇલેક્ટ્રિક ડ્રાઇવ
\item
  \textbf{ચાર્જિંગ}: ગ્રિડમાંથી બાહ્ય ચાર્જિંગ જરૂરી
\end{itemize}

\textbf{હાઇબ્રિડ ઇલેક્ટ્રિક વ્હિકલ (HEV)}:

\begin{itemize}
\tightlist
\item
  \textbf{પાવર સ્રોત}: આંતરિક કમ્બશન એન્જિન + ઇલેક્ટ્રિક મોટર
\item
  \textbf{ચાલન}: પાવર સ્રોતો વચ્ચે ઓટોમેટિક સ્વિચિંગ
\item
  \textbf{કાર્યક્ષમતા}: રિજનરેટિવ બ્રેકિંગ ઊર્જા પુનઃપ્રાપ્ત કરે છે
\end{itemize}

\end{solutionbox}
\begin{mnemonicbox}
``બિગ હાઇબ્રિડ પ્લગ ફ્યુઅલ - બેટર ટ્રાન્સપોર્ટેશન ઓપ્શન્સ''

\end{mnemonicbox}
\begin{center}\rule{0.5\linewidth}{0.5pt}\end{center}

\subsection*{પ્રશ્ન 2(ક) [7
ગુણ]}\label{uxaaauxab0uxab6uxaa8-2uxa95-7-uxa97uxaa3}

\textbf{ડ્રોન અને તેના મુખ્ય ઘટકોના બ્લોક ડાયાગ્રામનું વર્ણન કરો.}

\begin{solutionbox}

\begin{verbatim}
    ┌──────────────┐         ┌──────────────┐
    │    Camera    │         │     GPS      │
    │              │         │   Module     │
    └──────┬───────┘         └──────┬───────┘
           │                        │
           ▼                        ▼
    ┌─────────────────────────────────────┐
    │         Flight Controller           │
    │      (Microprocessor Unit)          │
    └──────┬─────────────────────┬────────┘
           │                     │
           ▼                     ▼
    ┌─────────────┐       ┌─────────────┐
    │   Motors    │       │   Sensors   │
    │ \& Propellers│       │(Gyro, Accel)│
    └─────────────┘       └─────────────┘
           │                     │
           ▼                     ▼
    ┌─────────────┐       ┌─────────────┐
    │   Battery   │       │ Transmitter │
    │    Pack     │       │ \& Receiver  │
    └─────────────┘       └─────────────┘
\end{verbatim}

\textbf{મુખ્ય ઘટકો}:

\textbf{ફ્લાઇટ કંટ્રોલર}:

\begin{itemize}
\tightlist
\item
  \textbf{કાર્ય}: તમામ ઓપરેશન્સ નિયંત્રિત કરતું કેન્દ્રીય પ્રોસેસિંગ યુનિટ
\item
  \textbf{લક્ષણો}: સ્થિરતા, નેવિગેશન, ઓટોપાઇલટ ફંક્શન્સ
\end{itemize}

\textbf{મોટર અને પ્રોપેલર}:

\begin{itemize}
\tightlist
\item
  \textbf{બ્રશલેસ મોટર}: ઉચ્ચ કાર્યક્ષમતા, ચોક્કસ સ્પીડ કંટ્રોલ
\item
  \textbf{પ્રોપેલર}: લિફ્ટ અને મૂવમેન્ટ માટે થ્રસ્ટ જનરેટ કરે છે
\end{itemize}

\textbf{સેન્સર પેકેજ}:

\begin{itemize}
\tightlist
\item
  \textbf{જાયરોસ્કોપ}: સ્થિરતા માટે કોણીય વેગ માપે છે
\item
  \textbf{એક્સેલેરોમીટર}: પ્રવેગ અને ઝુકાવ શોધે છે
\item
  \textbf{બેરોમીટર}: ઊંચાઈ માપણ
\end{itemize}

\textbf{પાવર સિસ્ટમ}:

\begin{itemize}
\tightlist
\item
  \textbf{બેટરી}: ઉચ્ચ પાવર ડેન્સિટી માટે લિથિયમ પોલિમર (LiPo)
\item
  \textbf{ESC}: મોટર કંટ્રોલ માટે ઇલેક્ટ્રોનિક સ્પીડ કંટ્રોલર
\end{itemize}

\textbf{કમ્યુનિકેશન}:

\begin{itemize}
\tightlist
\item
  \textbf{ટ્રાન્સમિટર/રિસીવર}: રિમોટ કંટ્રોલર સાથે રેડિયો કમ્યુનિકેશન
\item
  \textbf{GPS}: પોઝિશન ટ્રેકિંગ અને નેવિગેશન
\end{itemize}

\end{solutionbox}
\begin{mnemonicbox}
``ફ્લાઇંગ કંટ્રોલર્સ મોટર સેન્સર્સ પાવર કમ્યુનિકેશન - ડ્રોન્સ
ફ્લાઇ પરફેક્ટલી''

\end{mnemonicbox}
\begin{center}\rule{0.5\linewidth}{0.5pt}\end{center}

\subsection*{પ્રશ્ન 2(અ) અથવા [3
ગુણ]}\label{uxaaauxab0uxab6uxaa8-2uxa85-uxa85uxaa5uxab5-3-uxa97uxaa3}

\textbf{UAV શું છે? તેની એપ્લિકેશનોની યાદી બનાવો.}

\begin{solutionbox}

\textbf{વ્યાખ્યા}: UAV (અનમેન્ડ એરિયલ વ્હિકલ) એ એવું વિમાન છે જે બોર્ડ પર માનવ
પાઇલટ વિના ચલાવવામાં આવે છે.


{\def\LTcaptype{none} % do not increment counter
\vspace{-5pt}
\captionof{table}{UAV ઉપયોગો}
\vspace{-10pt}
\begin{longtable}[]{@{}lll@{}}
\toprule\noalign{}
ક્ષેત્ર & ઉપયોગ & ફાયદો \\
\midrule\noalign{}
\endhead
\bottomrule\noalign{}
\endlastfoot
કૃષિ & પાક મોનિટરિંગ, છંટકાવ & ચોક્કસ ખેતી \\
સુરક્ષા & દેખરેખ, બોર્ડર પેટ્રોલ & વધારેલી નિરીક્ષણ \\
ડિલિવરી & પેકેજ ડિલિવરી & ઝડપી પરિવહન \\
ફોટોગ્રાફી & હવાઈ ફોટોગ્રાફી & નવા દ્રષ્ટિકોણ \\
નિરીક્ષણ & ઇન્ફ્રાસ્ટ્રક્ચર નિરીક્ષણ & સલામત પહોંચ \\
\end{longtable}
}

\textbf{મુખ્ય લક્ષણો}:

\begin{itemize}
\tightlist
\item
  \textbf{સ્વચાલિત}: સ્વ-નિયંત્રિત ફ્લાઇટ ક્ષમતાઓ
\item
  \textbf{રિમોટ કંટ્રોલ}: ગ્રાઉન્ડ સ્ટેશનમાંથી સંચાલિત
\item
  \textbf{બહુમુખી}: બહુવિધ પેલોડ વિકલ્પો
\end{itemize}

\end{solutionbox}
\begin{mnemonicbox}
``અનમેન્ડ એરક્રાફ્ટ વર્સેટાઇલ - એપ્લિકેશન્સ આર વાસ્ટ''

\end{mnemonicbox}
\begin{center}\rule{0.5\linewidth}{0.5pt}\end{center}

\subsection*{પ્રશ્ન 2(બ) અથવા [4
ગુણ]}\label{uxaaauxab0uxab6uxaa8-2uxaac-uxa85uxaa5uxab5-4-uxa97uxaa3}

\textbf{વિવિધ પ્રકારના EV ઊર્જા સ્રોતોની યાદી બનાવો અને કોઈપણ બેને સમજાવો.}

\begin{solutionbox}


{\def\LTcaptype{none} % do not increment counter
\vspace{-5pt}
\captionof{table}{EV ઊર્જા સ્રોતો}
\vspace{-10pt}
\begin{longtable}[]{@{}llll@{}}
\toprule\noalign{}
પ્રકાર & ટેકનોલોજી & સંગ્રહ & કાર્યક્ષમતા \\
\midrule\noalign{}
\endhead
\bottomrule\noalign{}
\endlastfoot
બેટરી & લિથિયમ-આયન & રાસાયણિક & 90-95\% \\
ફ્યુઅલ સેલ & હાઇડ્રોજન & રાસાયણિક & 50-60\% \\
અલ્ટ્રાકેપેસિટર & ઇલેક્ટ્રિક ફિલ્ડ & વિદ્યુત & 95\%+ \\
ફ્લાયવ્હીલ & ગતિ ઊર્જા & યાંત્રિક & 85-90\% \\
રિજનરેટિવ બ્રેકિંગ & મોટર જનરેટર & ગતિશીલથી વિદ્યુત & 70-80\% \\
\end{longtable}
}

\textbf{બેટરી સિસ્ટમ}:

\begin{itemize}
\tightlist
\item
  \textbf{ટેકનોલોજી}: ઉચ્ચ ઊર્જા ઘનતા સાથે લિથિયમ-આયન સેલ
\item
  \textbf{ફાયદાઓ}: પરિપક્વ તકનીક, સારો ઊર્જા સંગ્રહ
\item
  \textbf{ચાર્જિંગ}: બાહ્ય ચાર્જિંગ ઇન્ફ્રાસ્ટ્રક્ચર જરૂરી
\end{itemize}

\textbf{ફ્યુઅલ સેલ સિસ્ટમ}:

\begin{itemize}
\tightlist
\item
  \textbf{ટેકનોલોજી}: હાઇડ્રોજન ઓક્સિજન સાથે જોડાઈને વીજળી ઉત્પન્ન કરે છે
\item
  \textbf{ફાયદાઓ}: ઝડપી રિફ્યુઅલિંગ, લાંબી રેન્જ
\item
  \textbf{પડકારો}: હાઇડ્રોજન ઇન્ફ્રાસ્ટ્રક્ચર મર્યાદિત
\end{itemize}

\end{solutionbox}
\begin{mnemonicbox}
``બેટરી ફ્યુઅલ અલ્ટ્રા ફ્લાઇ રિજન - એનર્જી સોર્સીસ ઇનેબલ
વ્હિકલ્સ''

\end{mnemonicbox}
\begin{center}\rule{0.5\linewidth}{0.5pt}\end{center}

\subsection*{પ્રશ્ન 2(ક) અથવા [7
ગુણ]}\label{uxaaauxab0uxab6uxaa8-2uxa95-uxa85uxaa5uxab5-7-uxa97uxaa3}

\textbf{વિવિધ પ્રકારની સ્માર્ટ સિસ્ટમ્સની યાદી બનાવો. કોઈપણ 2 સ્માર્ટ સિસ્ટમોને
આકૃતિ સાથે સમજાવો.}

\begin{solutionbox}


{\def\LTcaptype{none} % do not increment counter
\vspace{-5pt}
\captionof{table}{સ્માર્ટ સિસ્ટમના પ્રકારો}
\vspace{-10pt}
\begin{longtable}[]{@{}lll@{}}
\toprule\noalign{}
સિસ્ટમ & કાર્ય & ટેકનોલોજી \\
\midrule\noalign{}
\endhead
\bottomrule\noalign{}
\endlastfoot
સ્માર્ટ હોમ્સ & હોમ ઓટોમેશન & IoT, સેન્સર્સ \\
સ્માર્ટ કાર્સ & સેલ્ફ-ડ્રાઇવિંગ & AI, સેન્સર્સ \\
સ્માર્ટ સિટી & શહેરી વ્યવસ્થાપન & IoT, બિગ ડેટા \\
સ્માર્ટ ગ્રિડ & પાવર મેનેજમેન્ટ & કમ્યુનિકેશન \\
સ્માર્ટ હેલ્થ & આરોગ્ય નિરીક્ષણ & વેરેબલ્સ, AI \\
\end{longtable}
}

\textbf{સ્માર્ટ સ્ટ્રીટ લાઇટ સિસ્ટમ}:

\begin{verbatim}
    ┌─────────────┐    ┌─────────────┐
    │   Motion    │    │   Light     │
    │   Sensor    │    │   Sensor    │
    └──────┬──────┘    └──────┬──────┘
           │                  │
           ▼                  ▼
    ┌────────────────────────────────┐
    │      Microcontroller           │
    │    (Control Logic)             │
    └──────┬─────────────────────────┘
           │
           ▼
    ┌─────────────┐    ┌─────────────┐
    │ LED Street  │    │ Wireless    │
    │    Light    │    │Communication│
    └─────────────┘    └─────────────┘
\end{verbatim}

\textbf{સ્માર્ટ વોટર પોલ્યુશન મોનિટરિંગ}:

\begin{verbatim}
    ┌─────────────┐    ┌─────────────┐
    │    pH       │    │Temperature  │
    │   Sensor    │    │   Sensor    │
    └──────┬──────┘    └──────┬──────┘
           │                  │
           ▼                  ▼
    ┌────────────────────────────────┐
    │      Data Logger               │
    │   (Microcontroller)            │
    └──────┬─────────────────────────┘
           │
           ▼
    ┌─────────────┐    ┌─────────────┐
    │   GSM/WiFi  │    │   Cloud     │
    │Communication│    │  Database   │
    └─────────────┘    └─────────────┘
\end{verbatim}

\textbf{લક્ષણો}:

\begin{itemize}
\tightlist
\item
  \textbf{ઓટોમેશન}: પર્યાવરણીય પરિસ્થિતિઓ માટે બુદ્ધિશાળી પ્રતિભાવ
\item
  \textbf{ઊર્જા કાર્યક્ષમતા}: ઑપ્ટિમાઇઝ્ડ પાવર વપરાશ
\item
  \textbf{રિમોટ મોનિટરિંગ}: રિયલ-ટાઇમ ડેટા સંગ્રહ અને વિશ્લેષણ
\end{itemize}

\end{solutionbox}
\begin{mnemonicbox}
``સ્માર્ટ સિસ્ટમ્સ સેવ એનર્જી એફિશિયન્ટલી''

\end{mnemonicbox}
\begin{center}\rule{0.5\linewidth}{0.5pt}\end{center}

\subsection*{પ્રશ્ન 3(અ) [3
ગુણ]}\label{uxaaauxab0uxab6uxaa8-3uxa85-3-uxa97uxaa3}

\textbf{સ્માર્ટ સ્ટ્રીટ લાઇટ કંટ્રોલ અને મોનિટરિંગ સિસ્ટમનો બ્લોક ડાયાગ્રામ દોરો.}

\begin{solutionbox}

\begin{verbatim}
                    ┌─────────────┐
                    │   Sensors   │
                    │ (PIR, LDR)  │
                    └──────┬──────┘
                           │
                           ▼
                    ┌───────────────┐
                    │Microcontroller│
                    │ (Arduino)     │
                    └──────┬────────┘
                           │
                  ┌────────┼────────┐
                  ▼                 ▼
            ┌──────────┐      ┌──────────┐
            │LED Driver│      │ WiFi/GSM │
            │          │      │ Module   │
            └────┬─────┘      └─────┬────┘
                 ▼                  ▼
            ┌──────────┐      ┌──────────┐
            │LED Street│      │  Cloud   │
            │  Light   │      │ Server   │
            └──────────┘      └──────────┘
\end{verbatim}

\textbf{ઘટકો}:

\begin{itemize}
\tightlist
\item
  \textbf{PIR સેન્સર}: ઑટોમેટિક સ્વિચિંગ માટે ગતિ શોધ
\item
  \textbf{LDR સેન્સર}: પ્રકાશની તીવ્રતા માપણ
\item
  \textbf{માઇક્રોકંટ્રોલર}: કંટ્રોલ લોજિક અને નિર્ણય લેવા
\end{itemize}

\end{solutionbox}
\begin{mnemonicbox}
``સ્માર્ટ સ્ટ્રીટ્સ સેવ પાવર પરફેક્ટલી''

\end{mnemonicbox}
\begin{center}\rule{0.5\linewidth}{0.5pt}\end{center}

\subsection*{પ્રશ્ન 3(બ) [4
ગુણ]}\label{uxaaauxab0uxab6uxaa8-3uxaac-4-uxa97uxaa3}

\textbf{પહેરી શકાય તેવી આરોગ્ય નિરીક્ષણ સિસ્ટમનો બ્લોક ડાયાગ્રામ દોરો અને
સમજાવો.}

\begin{solutionbox}

\begin{verbatim}
    ┌─────────────┐    ┌─────────────┐
    │ Heart Rate  │    │Temperature  │
    │   Sensor    │    │   Sensor    │
    └──────┬──────┘    └──────┬──────┘
           │                  │
           ▼                  ▼
    ┌────────────────────────────────┐
    │      Microprocessor            │
    │     (Data Processing)          │
    └──────┬─────────────────────────┘
           │
           ▼
    ┌─────────────┐    ┌─────────────┐
    │   Display   │    │  Bluetooth  │
    │   (OLED)    │    │Communication│
    └─────────────┘    └──────┬──────┘
                              │
                              ▼
                       ┌─────────────┐
                       │ Smartphone  │
                       │    App      │
                       └─────────────┘
\end{verbatim}

\textbf{સમજૂતી}:

\begin{itemize}
\tightlist
\item
  \textbf{સેન્સર્સ}: જરૂરી સંકેતોનું સતત નિરીક્ષણ કરે છે
\item
  \textbf{પ્રોસેસિંગ}: ડેટાનું વિશ્લેષણ કરે છે અને અસાધારણતા શોધે છે
\item
  \textbf{કમ્યુનિકેશન}: બ્લૂટૂથ દ્વારા સ્માર્ટફોનમાં ડેટા મોકલે છે
\item
  \textbf{એલર્ટ}: જરૂર પડ્યે વપરાશકર્તા અને ઇમર્જન્સી કોન્ટેક્ટને સૂચના આપે છે
\end{itemize}

\textbf{ઉપયોગો}:

\begin{itemize}
\tightlist
\item
  \textbf{ફિટનેસ ટ્રેકિંગ}: પગલાંની ગણતરી, કેલરી બર્ન
\item
  \textbf{આરોગ્ય નિરીક્ષણ}: હાર્ટ રેટ, બ્લડ પ્રેશર
\item
  \textbf{ઇમર્જન્સી એલર્ટ}: ગંભીર સ્થિતિમાં ઑટોમેટિક SOS
\end{itemize}

\end{solutionbox}
\begin{mnemonicbox}
``વેરેબલ હેલ્થ વોચીસ મોનિટર કન્ટિન્યુઅસલી''

\end{mnemonicbox}
\begin{center}\rule{0.5\linewidth}{0.5pt}\end{center}

\subsection*{પ્રશ્ન 3(ક) [7
ગુણ]}\label{uxaaauxab0uxab6uxaa8-3uxa95-7-uxa97uxaa3}

\textbf{બાયોમેટ્રિક સિસ્ટમ્સ અને તેમના મૂળભૂત બ્લોક ડાયાગ્રામને સમજાવો.}

\begin{solutionbox}

\begin{verbatim}
    ┌─────────────┐
    │   Sensor    │
    │  (Scanner)  │
    └──────┬──────┘
           │ Raw Data
           ▼
    ┌─────────────┐
    │ Pre{-process │}
    │   Module    │
    └──────┬──────┘
           │ Processed Data
           ▼
    ┌─────────────┐
    │  Feature    │
    │ Extraction  │
    └──────┬──────┘
           │ Template
           ▼
    ┌─────────────┐    ┌─────────────┐
    │   Matching  │──│  Database   │
    │   Module    │    │ (Templates) │
    └──────┬──────┘    └─────────────┘
           │ Match Score
           ▼
    ┌─────────────┐
    │  Decision   │
    │   Module    │
    └─────────────┘
\end{verbatim}

\textbf{ઘટકોની સમજૂતી}:

\textbf{સેન્સર મોડ્યુલ}:

\begin{itemize}
\tightlist
\item
  \textbf{કાર્ય}: બાયોમેટ્રિક ડેટા કેપ્ચર કરે છે (ફિંગરપ્રિન્ટ, ચહેરો, આઈરિસ)
\item
  \textbf{ટેકનોલોજી}: ઓપ્ટિકલ, કેપેસિટિવ, અથવા થર્મલ સેન્સર્સ
\end{itemize}

\textbf{પ્રી-પ્રોસેસિંગ}:

\begin{itemize}
\tightlist
\item
  \textbf{કાર્ય}: નોઈઝ દૂર કરવું અને ઇમેજ સુધારો
\item
  \textbf{ઓપરેશન્સ}: ફિલ્ટરિંગ, નોર્મલાઈઝેશન, ગુણવત્તા મૂલ્યાંકન
\end{itemize}

\textbf{ફીચર એક્સટ્રેક્શન}:

\begin{itemize}
\tightlist
\item
  \textbf{કાર્ય}: અનોખી લાક્ષણિકતાઓ કાઢે છે
\item
  \textbf{આઉટપુટ}: બાયોમેટ્રિકનું પ્રતિનિધિત્વ કરતું ગાણિતિક ટેમ્પ્લેટ
\end{itemize}

\textbf{મેચિંગ મોડ્યુલ}:

\begin{itemize}
\tightlist
\item
  \textbf{કાર્ય}: કેપ્ચર કરેલા ટેમ્પ્લેટને ડેટાબેઝ સાથે સરખાવે છે
\item
  \textbf{અલ્ગોરિધમ}: પેટર્ન મેચિંગ અલ્ગોરિધમ્સ
\end{itemize}

\textbf{ડેટાબેઝ}:

\begin{itemize}
\tightlist
\item
  \textbf{કાર્ય}: નોંધાયેલા બાયોમેટ્રિક ટેમ્પ્લેટ્સ સંગ્રહિત કરે છે
\item
  \textbf{સુરક્ષા}: ગોપનીયતા માટે એન્ક્રિપ્ટેડ સંગ્રહ
\end{itemize}

\textbf{નિર્ણય મોડ્યુલ}:

\begin{itemize}
\tightlist
\item
  \textbf{કાર્ય}: થ્રેશોલ્ડના આધારે સ્વીકાર અથવા નકાર
\item
  \textbf{પેરામીટર્સ}: False Accept Rate (FAR), False Reject Rate (FRR)
\end{itemize}

\textbf{બાયોમેટ્રિક્સના પ્રકારો}:

\begin{itemize}
\tightlist
\item
  \textbf{શારીરિક}: ફિંગરપ્રિન્ટ, ચહેરો, આઈરિસ, રેટિના
\item
  \textbf{વર્તણૂકલક્ષી}: અવાજ, હસ્તાક્ષર, ચાલ
\end{itemize}

\textbf{ઉપયોગો}:

\begin{itemize}
\tightlist
\item
  \textbf{એક્સેસ કંટ્રોલ}: બિલ્ડિંગ સુરક્ષા, ડિવાઇસ અનલોકિંગ
\item
  \textbf{ઓળખ}: બોર્ડર કંટ્રોલ, ફોરેન્સિક્સ
\item
  \textbf{પ્રમાણીકરણ}: બેન્કિંગ, હાજરી સિસ્ટમ્સ
\end{itemize}

\end{solutionbox}
\begin{mnemonicbox}
``સેન્સર્સ પ્રોસેસ ફીચર્સ મેચ ડેટાબેઝ ડિસાઈડ - બાયોમેટ્રિક
સિક્યુરિટી બેટર ડન''

\end{mnemonicbox}
\begin{center}\rule{0.5\linewidth}{0.5pt}\end{center}

\subsection*{પ્રશ્ન 3(અ) અથવા [3
ગુણ]}\label{uxaaauxab0uxab6uxaa8-3uxa85-uxa85uxaa5uxab5-3-uxa97uxaa3}

\textbf{જળ પ્રદૂષણ મોનિટરિંગ સિસ્ટમનો બ્લોક ડાયાગ્રામ દોરો.}

\begin{solutionbox}

\begin{verbatim}
                    ┌─────────────┐
                    │Water Quality│
                    │  Sensors    │
                    │(pH,DO,Temp) │
                    └──────┬──────┘
                           │
                           ▼
                    ┌───────────────┐
                    │Microcontroller│
                    │ (Data Logger) │
                    └──────┬────────┘
                           │
                  ┌────────┼────────┐
                  ▼                 ▼
            ┌──────────┐      ┌──────────┐
            │Local LCD │      │ GSM/WiFi │
            │ Display  │      │ Module   │
            └──────────┘      └─────┬────┘
                                    ▼
                              ┌──────────┐
                              │  Cloud   │
                              │ Database │
                              └──────────┘
\end{verbatim}

\textbf{સેન્સર્સ}:

\begin{itemize}
\tightlist
\item
  \textbf{pH સેન્સર}: પાણીની અમ્લતા/ક્ષારતા માપે છે
\item
  \textbf{DO સેન્સર}: ઓગળેલા ઓક્સિજનનું માપણ
\item
  \textbf{તાપમાન}: પાણીના તાપમાનનું નિરીક્ષણ
\end{itemize}

\end{solutionbox}
\begin{mnemonicbox}
``વોટર ક્વોલિટી મોનિટરિંગ પ્રિવેન્ટ્સ પોલ્યુશન''

\end{mnemonicbox}
\begin{center}\rule{0.5\linewidth}{0.5pt}\end{center}

\subsection*{પ્રશ્ન 3(બ) અથવા [4
ગુણ]}\label{uxaaauxab0uxab6uxaa8-3uxaac-uxa85uxaa5uxab5-4-uxa97uxaa3}

\textbf{સ્માર્ટ વૉચનો બ્લોક ડાયાગ્રામ દોરો અને સમજાવો.}

\begin{solutionbox}

\begin{verbatim}
    ┌─────────────┐    ┌─────────────┐
    │ Touchscreen │    │   Sensors   │
    │   Display   │    │(Accel,Gyro) │
    └──────┬──────┘    └──────┬──────┘
           │                  │
           ▼                  ▼
    ┌────────────────────────────────┐
    │      System on Chip            │
    │    (ARM Processor)             │
    └──────┬─────────────────────────┘
           │
           ▼
    ┌─────────────┐    ┌─────────────┐
    │   Battery   │    │  Bluetooth  │
    │    Pack     │    │/WiFi Module │
    └─────────────┘    └─────────────┘
\end{verbatim}

\textbf{સમજૂતી}:

\begin{itemize}
\tightlist
\item
  \textbf{ડિસ્પ્લે}: યુઝર ઇન્ટરફેસ માટે OLED ટચસ્ક્રીન
\item
  \textbf{સેન્સર્સ}: મોશન ટ્રેકિંગ અને આરોગ્ય નિરીક્ષણ
\item
  \textbf{પ્રોસેસર}: લો-પાવર ARM-આધારિત SoC
\item
  \textbf{કનેક્ટિવિટી}: સ્માર્ટફોન પેરિંગ માટે બ્લૂટૂથ
\end{itemize}

\textbf{લક્ષણો}:

\begin{itemize}
\tightlist
\item
  \textbf{આરોગ્ય ટ્રેકિંગ}: હાર્ટ રેટ, પગલાં, ઊંઘ
\item
  \textbf{નોટિફિકેશન્સ}: કૉલ્સ, મેસેજ, એપ્સ
\item
  \textbf{એપ્સ}: હવામાન, સંગીત, પેમેન્ટ્સ
\end{itemize}

\end{solutionbox}
\begin{mnemonicbox}
``સ્માર્ટ વૉચીસ શો હેલ્થ ઇન્ફર્મેશન''

\end{mnemonicbox}
\begin{center}\rule{0.5\linewidth}{0.5pt}\end{center}

\subsection*{પ્રશ્ન 3(ક) અથવા [7
ગુણ]}\label{uxaaauxab0uxab6uxaa8-3uxa95-uxa85uxaa5uxab5-7-uxa97uxaa3}

\textbf{AR/VR કોર ટેકનોલોજીને સમજાવો અને તેની એપ્લિકેશનોની ચર્ચા કરો.}

\begin{solutionbox}

\textbf{AR/VR કોર ટેકનોલોજીઓ}:


{\def\LTcaptype{none} % do not increment counter
\vspace{-5pt}
\captionof{table}{AR વિરુદ્ધ VR ટેકનોલોજી}
\vspace{-10pt}
\begin{longtable}[]{@{}lll@{}}
\toprule\noalign{}
પાસું & Augmented Reality (AR) & Virtual Reality (VR) \\
\midrule\noalign{}
\endhead
\bottomrule\noalign{}
\endlastfoot
વાતાવરણ & વાસ્તવિક + ડિજિટલ ઓવરલે & સંપૂર્ણ વર્ચ્યુઅલ \\
હાર્ડવેર & સ્માર્ટફોન, AR ચશ્મા & VR હેડસેટ, કંટ્રોલર્સ \\
નિમજ્જન & આંશિક & સંપૂર્ણ \\
ઇન્ટરેક્શન & ટચ, જેસ્ચર & કંટ્રોલર્સ, હેન્ડ ટ્રેકિંગ \\
\end{longtable}
}

\textbf{કોર ઘટકો}:

\textbf{ડિસ્પ્લે ટેકનોલોજી}:

\begin{itemize}
\tightlist
\item
  \textbf{AR}: સી-થ્રુ ડિસ્પ્લે, પ્રોજેક્શન
\item
  \textbf{VR}: હાઇ-રિઝોલ્યુશન OLED/LCD સ્ક્રીન્સ
\end{itemize}

\textbf{ટ્રેકિંગ સિસ્ટમ્સ}:

\begin{itemize}
\tightlist
\item
  \textbf{મોશન ટ્રેકિંગ}: 6-DOF (ડિગ્રીઝ ઓફ ફ્રીડમ) ટ્રેકિંગ
\item
  \textbf{આઈ ટ્રેકિંગ}: ઇન્ટરેક્શન માટે નજર શોધ
\item
  \textbf{હેન્ડ ટ્રેકિંગ}: જેસ્ચર રેકગ્નિશન
\end{itemize}

\textbf{પ્રોસેસિંગ પાવર}:

\begin{itemize}
\tightlist
\item
  \textbf{ગ્રાફિક્સ પ્રોસેસિંગ}: રિયલ-ટાઇમ 3D રેન્ડરિંગ
\item
  \textbf{કમ્પ્યુટર વિઝન}: ઓબ્જેક્ટ રેકગ્નિશન અને ટ્રેકિંગ
\item
  \textbf{AI/ML}: સીન અંડરસ્ટેન્ડિંગ અને ઓપ્ટિમાઈઝેશન
\end{itemize}

\textbf{ઉપયોગો}:

\textbf{શિક્ષણ}:

\begin{itemize}
\tightlist
\item
  \textbf{AR}: ઇન્ટરેક્ટિવ પાઠ્યપુસ્તકો, 3D મોડલ ઓવરલે
\item
  \textbf{VR}: વર્ચ્યુઅલ ક્લાસરૂમ, ઐતિહાસિક સિમ્યુલેશન
\end{itemize}

\textbf{આરોગ્યસંભાળ}:

\begin{itemize}
\tightlist
\item
  \textbf{AR}: સર્જરી સહાયતા, તબીબી તાલીમ
\item
  \textbf{VR}: થેરાપી, પીડા વ્યવસ્થાપન, તાલીમ
\end{itemize}

\textbf{મનોરંજન}:

\begin{itemize}
\tightlist
\item
  \textbf{AR}: પોકેમોન ગો, સ્નેપચેટ ફિલ્ટર્સ
\item
  \textbf{VR}: ગેમિંગ, વર્ચ્યુઅલ કોન્સર્ટ, મૂવીઝ
\end{itemize}

\textbf{ઉદ્યોગ}:

\begin{itemize}
\tightlist
\item
  \textbf{AR}: મેઇન્ટેનન્સ સૂચનાઓ, ગુણવત્તા નિરીક્ષણ
\item
  \textbf{VR}: તાલીમ સિમ્યુલેશન, ડિઝાઇન રિવ્યુ
\end{itemize}

\textbf{રિટેઇલ}:

\begin{itemize}
\tightlist
\item
  \textbf{AR}: વર્ચ્યુઅલ ટ્રાઇ-ઓન, પ્રોડક્ટ વિઝ્યુઅલાઇઝેશન
\item
  \textbf{VR}: વર્ચ્યુઅલ શોરૂમ, નિમજ્જનકારી શોપિંગ
\end{itemize}

\textbf{ભવિષ્યના ટ્રેન્ડ્સ}:

\begin{itemize}
\tightlist
\item
  \textbf{મિક્સ્ડ રિયાલિટી}: AR અને VR નું સંયોજન
\item
  \textbf{હેપ્ટિક ફીડબેક}: સ્પર્શ સંવેદના
\item
  \textbf{ક્લાઉડ રેન્ડરિંગ}: રિમોટ પ્રોસેસિંગ પાવર
\end{itemize}

\end{solutionbox}
\begin{mnemonicbox}
``AR VR ડિસ્પ્લે ટ્રેક પ્રોસેસ એપ્લાઇ - ટેકનોલોજી ટ્રાન્સફોર્મ્સ
રિયાલિટી''

\end{mnemonicbox}
\begin{center}\rule{0.5\linewidth}{0.5pt}\end{center}

\subsection*{પ્રશ્ન 4(અ) [3
ગુણ]}\label{uxaaauxab0uxab6uxaa8-4uxa85-3-uxa97uxaa3}

\textbf{ઇનઓર્ગેનિક અને ઓર્ગેનિક ઇલેક્ટ્રોનિક્સ વચ્ચે તફાવત કરો.}

\begin{solutionbox}


{\def\LTcaptype{none} % do not increment counter
\vspace{-5pt}
\captionof{table}{ઇનઓર્ગેનિક વિરુદ્ધ ઓર્ગેનિક ઇલેક્ટ્રોનિક્સ}
\vspace{-10pt}
\begin{longtable}[]{@{}lll@{}}
\toprule\noalign{}
પેરામીટર & ઇનઓર્ગેનિક ઇલેક્ટ્રોનિક્સ & ઓર્ગેનિક ઇલેક્ટ્રોનિક્સ \\
\midrule\noalign{}
\endhead
\bottomrule\noalign{}
\endlastfoot
સામગ્રી & સિલિકોન, જર્મેનિયમ & કાર્બન-આધારિત સંયોજનો \\
પ્રોસેસિંગ & ઉચ્ચ તાપમાન & નીચા તાપમાન \\
લવચીકતા & સખત & લવચીક \\
કિંમત & ઊંચી & ઓછી \\
પ્રદર્શન & હાઇ સ્પીડ, સ્થિર & લોઅર સ્પીડ, સુધારાતું \\
\end{longtable}
}

\textbf{મુખ્ય તફાવતો}:

\begin{itemize}
\tightlist
\item
  \textbf{બંધારણ}: ઇનઓર્ગેનિક ક્રિસ્ટલાઇન મટીરિયલ વાપરે છે, ઓર્ગેનિક પોલિમર ચેઇન
  વાપરે છે
\item
  \textbf{ઉત્પાદન}: ઇનઓર્ગેનિકને ક્લીન રૂમ જોઈએ છે, ઓર્ગેનિક પ્રિન્ટિંગ મેથડ વાપરે છે
\item
  \textbf{ઉપયોગો}: ઇનઓર્ગેનિક હાઇ-પરફોર્મન્સ માટે, ઓર્ગેનિક લાર્જ-એરિયા ડિવાઇસ
  માટે
\end{itemize}

\end{solutionbox}
\begin{mnemonicbox}
``ઇનઓર્ગેનિક ઇઝ રિજિડ, ઓર્ગેનિક ઓફર્સ ફ્લેક્સિબિલિટી''

\end{mnemonicbox}
\begin{center}\rule{0.5\linewidth}{0.5pt}\end{center}

\subsection*{પ્રશ્ન 4(બ) [4
ગુણ]}\label{uxaaauxab0uxab6uxaa8-4uxaac-4-uxa97uxaa3}

\textbf{વિવિધ પ્રકારના ઓર્ગેનિક ઘટકોની યાદી બનાવો અને કોઈપણ બેને સમજાવો.}

\begin{solutionbox}


{\def\LTcaptype{none} % do not increment counter
\vspace{-5pt}
\captionof{table}{ઓર્ગેનિક ઘટકોના પ્રકારો}
\vspace{-10pt}
\begin{longtable}[]{@{}lll@{}}
\toprule\noalign{}
ઘટક & પૂરું નામ & ઉપયોગ \\
\midrule\noalign{}
\endhead
\bottomrule\noalign{}
\endlastfoot
OLED & ઓર્ગેનિક લાઇટ એમિટિંગ ડાયોડ & ડિસ્પ્લે \\
OFET & ઓર્ગેનિક ફિલ્ડ ઇફેક્ટ ટ્રાન્ઝિસ્ટર & સ્વિચિંગ \\
OPVD & ઓર્ગેનિક ફોટોવોલ્ટેઇક ડિવાઇસ & સોલાર સેલ \\
OECT & ઓર્ગેનિક ઇલેક્ટ્રોકેમિકલ ટ્રાન્ઝિસ્ટર & બાયોસેન્સર્સ \\
\end{longtable}
}

\textbf{ઓર્ગેનિક LED (OLED)}:

\begin{itemize}
\tightlist
\item
  \textbf{બંધારણ}: ઇલેક્ટ્રોડ્સ વચ્ચે ઓર્ગેનિક લેયર્સ
\item
  \textbf{કાર્ય}: જ્યારે કરંટ વહે ત્યારે ઇલેક્ટ્રોલ્યુમિનેસન્સ
\item
  \textbf{ફાયદાઓ}: સેલ્ફ-ઇલ્યુમિનેટિંગ, લવચીક, વાઇડ વ્યુઇંગ એન્ગલ
\end{itemize}

\textbf{ઓર્ગેનિક FET (OFET)}:

\begin{itemize}
\tightlist
\item
  \textbf{બંધારણ}: ઓર્ગેનિક સેમિકન્ડક્ટર ચેનલ
\item
  \textbf{કાર્ય}: ગેટ વોલ્ટેજ દ્વારા કરંટ નિયંત્રિત
\item
  \textbf{ઉપયોગો}: લવચીક સર્કિટ, સેન્સર્સ
\end{itemize}

\end{solutionbox}
\begin{mnemonicbox}
``ઓર્ગેનિક ઓન્લી ઓફર્સ આઉટસ્ટેન્ડિંગ ઓપ્શન્સ''

\end{mnemonicbox}
\begin{center}\rule{0.5\linewidth}{0.5pt}\end{center}

\subsection*{પ્રશ્ન 4(ક) [7
ગુણ]}\label{uxaaauxab0uxab6uxaa8-4uxa95-7-uxa97uxaa3}

\textbf{ઇલેક્ટ્રિક વ્હિકલનો બ્લોક ડાયાગ્રામ દોરો અને સમજાવો.}

\begin{solutionbox}

\begin{verbatim}
    ┌─────────────┐    ┌─────────────┐
    │   Battery   │    │   Charger   │
    │    Pack     │    │   (AC/DC)   │
    └──────┬──────┘    └──────┬──────┘
           │                  │
           ▼                  ▼
    ┌────────────────────────────────┐
    │      Power Electronics         │
    │    (Inverter/Converter)        │
    └──────┬─────────────────────────┘
           │
           ▼
    ┌─────────────┐    ┌─────────────┐
    │ Electric    │    │ Vehicle     │
    │  Motor      │    │ Controller  │
    └──────┬──────┘    └──────┬──────┘
           │                  │
           ▼                  ▼
    ┌─────────────┐    ┌─────────────┐
    │Transmission │    │ Regenerative│
    │   System    │    │  Braking    │
    └──────┬──────┘    └─────────────┘
           │
           ▼
    ┌─────────────┐
    │   Wheels    │
    │             │
    └─────────────┘
\end{verbatim}

\textbf{ઘટકોની સમજૂતી}:

\textbf{બેટરી પેક}:

\begin{itemize}
\tightlist
\item
  \textbf{ટેકનોલોજી}: સિરીઝ/પેરેલલમાં લિથિયમ-આયન સેલ
\item
  \textbf{કાર્ય}: વ્હિકલ પ્રોપલ્શન માટે ઊર્જા સંગ્રહ
\item
  \textbf{વ્યવસ્થાપન}: સુરક્ષા માટે બેટરી મેનેજમેન્ટ સિસ્ટમ (BMS)
\end{itemize}

\textbf{પાવર ઇલેક્ટ્રોનિક્સ}:

\begin{itemize}
\tightlist
\item
  \textbf{ઇન્વર્ટર}: મોટર ડ્રાઇવ માટે DC ને AC માં ફેરવે છે
\item
  \textbf{કન્વર્ટર}: સહાયક સિસ્ટમ્સ માટે DC-DC કન્વર્ઝન
\item
  \textbf{કંટ્રોલ}: ચોક્કસ મોટર સ્પીડ અને ટોર્ક કંટ્રોલ
\end{itemize}

\textbf{ઇલેક્ટ્રિક મોટર}:

\begin{itemize}
\tightlist
\item
  \textbf{પ્રકાર}: પર્મેનન્ટ મેગ્નેટ સિંક્રોનસ અથવા ઇન્ડક્શન મોટર
\item
  \textbf{ફાયદાઓ}: ઉચ્ચ કાર્યક્ષમતા (90-95\%), તાત્કાલિક ટોર્ક
\item
  \textbf{કંટ્રોલ}: સ્પીડ કંટ્રોલ માટે વેરિએબલ ફ્રીક્વન્સી ડ્રાઇવ
\end{itemize}

\textbf{વ્હિકલ કંટ્રોલર}:

\begin{itemize}
\tightlist
\item
  \textbf{કાર્ય}: તમામ સિસ્ટમ્સનું વ્યવસ્થાપન કરતું કેન્દ્રીય કંટ્રોલ યુનિટ
\item
  \textbf{લક્ષણો}: એક્સેલેરેટર ઇનપુટ, મોટર કંટ્રોલ, સુરક્ષા નિરીક્ષણ
\item
  \textbf{કમ્યુનિકેશન}: સિસ્ટમ ઇન્ટિગ્રેશન માટે CAN બસ
\end{itemize}

\textbf{ચાર્જિંગ સિસ્ટમ}:

\begin{itemize}
\tightlist
\item
  \textbf{AC ચાર્જિંગ}: લેવલ 1 (120V) અને લેવલ 2 (240V)
\item
  \textbf{DC ફાસ્ટ ચાર્જિંગ}: ઝડપી ટોપ-અપ માટે હાઇ-પાવર ચાર્જિંગ
\item
  \textbf{ઓનબોર્ડ ચાર્જર}: AC ગ્રિડ પાવરને DC માં ફેરવે છે
\end{itemize}

\textbf{રિજનરેટિવ બ્રેકિંગ}:

\begin{itemize}
\tightlist
\item
  \textbf{કાર્ય}: ગતિશીલ ઊર્જાને પાછી વિદ્યુત ઊર્જામાં ફેરવે છે
\item
  \textbf{કાર્યક્ષમતા}: બ્રેકિંગ દરમિયાન 15-25\% ઊર્જા પુનઃપ્રાપ્ત કરે છે
\item
  \textbf{ઇન્ટિગ્રેશન}: યાંત્રિક બ્રેક્સ સાથે કામ કરે છે
\end{itemize}

\textbf{ફાયદાઓ}:

\begin{itemize}
\tightlist
\item
  \textbf{કાર્યક્ષમતા}: ICE વ્હિકલ્સ કરતાં 3-4 ગણી વધુ કાર્યક્ષમ
\item
  \textbf{ઉત્સર્જન}: શૂન્ય સ્થાનિક ઉત્સર્જન
\item
  \textbf{જાળવણી}: ઓછા હલનચલન ભાગો, ઓછી જાળવણી
\end{itemize}

\end{solutionbox}
\begin{mnemonicbox}
``બેટરી પાવર્સ મોટર થ્રુ કંટ્રોલર - ઇલેક્ટ્રિક વ્હિકલ્સ વેરી
એફિશિયન્ટ''

\end{mnemonicbox}
\begin{center}\rule{0.5\linewidth}{0.5pt}\end{center}

\subsection*{પ્રશ્ન 4(અ) અથવા [3
ગુણ]}\label{uxaaauxab0uxab6uxaa8-4uxa85-uxa85uxaa5uxab5-3-uxa97uxaa3}

\textbf{ઓર્ગેનિક ઇલેક્ટ્રોનિક્સના ફાયદા લખો.}

\begin{solutionbox}


{\def\LTcaptype{none} % do not increment counter
\vspace{-5pt}
\captionof{table}{ઓર્ગેનિક ઇલેક્ટ્રોનિક્સના ફાયદા}
\vspace{-10pt}
\begin{longtable}[]{@{}lll@{}}
\toprule\noalign{}
ફાયદો & વર્ણન & ઉપયોગ \\
\midrule\noalign{}
\endhead
\bottomrule\noalign{}
\endlastfoot
લવચીકતા & વાંકી શકાય, વાળી શકાય & લવચીક ડિસ્પ્લે \\
ઓછી કિંમત & સસ્તી સામગ્રી, પ્રિન્ટિંગ & કન્ઝ્યુમર ઇલેક્ટ્રોનિક્સ \\
મોટો વિસ્તાર & સરળ સ્કેલિંગ & મોટા ડિસ્પ્લે \\
હલકું વજન & પાતળું, હલકું & વેરેબલ્સ \\
પારદર્શકતા & પારદર્શી ડિવાઇસ & સ્માર્ટ વિન્ડો \\
\end{longtable}
}

\textbf{મુખ્ય ફાયદા}:

\begin{itemize}
\tightlist
\item
  \textbf{પ્રોસેસિંગ}: લો-ટેમ્પરેચર મેન્યુફેક્ચરિંગ
\item
  \textbf{ઊર્જા}: લો-પાવર ઓપરેશન
\item
  \textbf{કસ્ટમાઇઝેશન}: ટ્યુનેબલ પ્રોપર્ટીઝ
\item
  \textbf{ઇન્ટિગ્રેશન}: પ્લાસ્ટિક સાથે કોમ્પેટિબલ
\end{itemize}

\end{solutionbox}
\begin{mnemonicbox}
``ઓર્ગેનિક એડવાન્ટેજીસ આર ઓબવિયસલી આઉટસ્ટેન્ડિંગ''

\end{mnemonicbox}
\begin{center}\rule{0.5\linewidth}{0.5pt}\end{center}

\subsection*{પ્રશ્ન 4(બ) અથવા [4
ગુણ]}\label{uxaaauxab0uxab6uxaa8-4uxaac-uxa85uxaa5uxab5-4-uxa97uxaa3}

\textbf{AR/VR ઉદ્યોગના પરિપ્રેક્ષ્યો અને તકો વિશે લખો.}

\begin{solutionbox}

\textbf{બજારના પરિપ્રેક્ષ્યો}:


{\def\LTcaptype{none} % do not increment counter
\vspace{-5pt}
\captionof{table}{AR/VR બજાર સેગમેન્ટ્સ}
\vspace{-10pt}
\begin{longtable}[]{@{}llll@{}}
\toprule\noalign{}
સેગમેન્ટ & બજારનું કદ & વૃદ્ધિ દર & મુખ્ય ખેલાડીઓ \\
\midrule\noalign{}
\endhead
\bottomrule\noalign{}
\endlastfoot
ગેમિંગ & \$12B & 25\% & Meta, Sony \\
એન્ટરપ્રાઇઝ & \$8B & 35\% & Microsoft, Magic Leap \\
આરોગ્યસંભાળ & \$3B & 40\% & વિવિધ સ્ટાર્ટઅપ \\
શિક્ષણ & \$2B & 30\% & Google, Apple \\
\end{longtable}
}

\textbf{તકો}:

\begin{itemize}
\tightlist
\item
  \textbf{5G નેટવર્ક્સ}: ક્લાઉડ-આધારિત VR/AR ને સક્ષમ બનાવે છે
\item
  \textbf{AI ઇન્ટિગ્રેશન}: બુદ્ધિશાળી કન્ટેન્ટ એડેપ્ટેશન
\item
  \textbf{હાર્ડવેર મિનિયેચરાઇઝેશન}: હલકા, વધુ આરામદાયક ડિવાઇસ
\end{itemize}

\textbf{પડકારો}:

\begin{itemize}
\tightlist
\item
  \textbf{મોશન સિકનેસ}: VR કમ્ફર્ટ ઇશ્યુઝ
\item
  \textbf{બેટરી લાઇફ}: પાવર કન્ઝમ્પશન ઓપ્ટિમાઇઝેશન
\item
  \textbf{કન્ટેન્ટ ક્રિએશન}: ક્વોલિટી ઇમર્સિવ કન્ટેન્ટની જરૂર
\end{itemize}

\textbf{ભવિષ્યનો દૃષ્ટિકોણ}:

\begin{itemize}
\tightlist
\item
  \textbf{મેટાવર્સ}: વર્ચ્યુઅલ વર્લ્ડ અને સોશિયલ ઇન્ટરેક્શન
\item
  \textbf{રિમોટ વર્ક}: વર્ચ્યુઅલ કોલેબોરેશન પ્લેટફોર્મ
\item
  \textbf{ડિજિટલ ટ્વિન્સ}: ઇન્ડસ્ટ્રિયલ એપ્લિકેશન્સ
\end{itemize}

\end{solutionbox}
\begin{mnemonicbox}
``AR VR માર્કેટ ગ્રોઇંગ રેપિડલી''

\end{mnemonicbox}
\begin{center}\rule{0.5\linewidth}{0.5pt}\end{center}

\subsection*{પ્રશ્ન 4(ક) અથવા [7
ગુણ]}\label{uxaaauxab0uxab6uxaa8-4uxa95-uxa85uxaa5uxab5-7-uxa97uxaa3}

\textbf{EV આર્કિટેક્ચર દોરો અને સમજાવો.}

\begin{solutionbox}

\begin{verbatim}
                    ┌─────────────┐
                    │ High Voltage│
                    │ Battery Pack│
                    └──────┬──────┘
                           │ HV DC Bus
                           ▼
    ┌─────────────┐ ┌─────────────┐ ┌─────────────┐
    │DC{-DC        │ │   Traction  │ │   Onboard   │}
    │Converter    │ │   Inverter  │ │   Charger   │
    └──────┬──────┘ └──────┬──────┘ └──────┬──────┘
           │               │               │
           ▼               ▼               ▼
    ┌─────────────┐ ┌─────────────┐ ┌─────────────┐
    │12V Battery  │ │ AC Motor    │ │  Charging   │
    │\& Auxiliaries│ │ (Traction)  │ │   Port      │
    └─────────────┘ └──────┬──────┘ └─────────────┘
                           │
                           ▼
                    ┌─────────────┐
                    │Transmission │
                    │  \& Wheels   │
                    └─────────────┘
\end{verbatim}

\textbf{EV આર્કિટેક્ચર ઘટકો}:

\textbf{હાઇ વોલ્ટેજ બેટરી પેક}:

\begin{itemize}
\tightlist
\item
  \textbf{વોલ્ટેજ}: આધુનિક EVs માટે 300-800V
\item
  \textbf{કેપેસિટી}: 40-100+ kWh ઊર્જા સંગ્રહ
\item
  \textbf{વ્યવસ્થાપન}: સુરક્ષા અને ઓપ્ટિમાઇઝેશન માટે બેટરી મેનેજમેન્ટ સિસ્ટમ (BMS)
\end{itemize}

\textbf{ટ્રેક્શન ઇન્વર્ટર}:

\begin{itemize}
\tightlist
\item
  \textbf{કાર્ય}: મોટર માટે DC બેટરી પાવરને 3-ફેઝ AC માં ફેરવે છે
\item
  \textbf{કંટ્રોલ}: વેરિએબલ ફ્રીક્વન્સી અને વોલ્ટેજ કંટ્રોલ
\item
  \textbf{કાર્યક્ષમતા}: 95-98\% પાવર કન્વર્ઝન એફિશિયન્સી
\end{itemize}

\textbf{AC ટ્રેક્શન મોટર}:

\begin{itemize}
\tightlist
\item
  \textbf{પ્રકાર}: પર્મેનન્ટ મેગ્નેટ સિંક્રોનસ મોટર (PMSM) અથવા ઇન્ડક્શન મોટર
\item
  \textbf{પાવર}: વ્હિકલ ક્લાસ પર આધાર રાખીને 100-400+ kW
\item
  \textbf{ટોર્ક}: ઝીરો RPM થી તાત્કાલિક ટોર્ક ડિલિવરી
\end{itemize}

\textbf{DC-DC કન્વર્ટર}:

\begin{itemize}
\tightlist
\item
  \textbf{કાર્ય}: ઓક્ઝિલરી માટે HV બેટરી વોલ્ટેજને 12V માં સ્ટેપ ડાઉન કરે છે
\item
  \textbf{પાવર}: 2-5 kW ટિપિકલ કેપેસિટી
\item
  \textbf{આઇસોલેશન}: HV અને LV સિસ્ટમ્સ વચ્ચે ગેલ્વેનિક આઇસોલેશન
\end{itemize}

\textbf{ઓનબોર્ડ ચાર્જર}:

\begin{itemize}
\tightlist
\item
  \textbf{કાર્ય}: બેટરી ચાર્જિંગ માટે AC ગ્રિડ પાવરને DC માં ફેરવે છે
\item
  \textbf{પાવર}: AC ચાર્જિંગ માટે 3-22 kW
\item
  \textbf{સ્ટાન્ડર્ડ}: SAE J1772, CCS, CHAdeMO કોમ્પેટિબિલિટી
\end{itemize}

\textbf{12V ઓક્ઝિલરી બેટરી}:

\begin{itemize}
\tightlist
\item
  \textbf{કાર્ય}: વ્હિકલ બંધ હોય ત્યારે લાઇટ્સ, ઇન્ફોટેનમેન્ટ, HVAC પાવર કરે છે
\item
  \textbf{પ્રકાર}: લીડ-એસિડ અથવા Li-ion ઓક્ઝિલરી બેટરી
\item
  \textbf{બેકઅપ}: સુરક્ષા સિસ્ટમ્સ માટે ઇમર્જન્સી પાવર
\end{itemize}

\textbf{વ્હિકલ કંટ્રોલ યુનિટ}:

\begin{itemize}
\tightlist
\item
  \textbf{કાર્ય}: તમામ સિસ્ટમ્સનો સમન્વય કરનારું કેન્દ્રીય કંટ્રોલર
\item
  \textbf{કમ્યુનિકેશન}: CAN બસ નેટવર્ક ઇન્ટિગ્રેશન
\item
  \textbf{સુરક્ષા}: ફંક્શનલ સેફ્ટી (ISO 26262) કોમ્પ્લાયન્સ
\end{itemize}

\textbf{થર્મલ મેનેજમેન્ટ}:

\begin{itemize}
\tightlist
\item
  \textbf{બેટરી કૂલિંગ}: તાપમાન નિયંત્રણ માટે લિક્વિડ કૂલિંગ
\item
  \textbf{મોટર કૂલિંગ}: હાઇ પાવર ઓપરેશન દરમિયાન ઓવરહીટિંગ અટકાવે છે
\item
  \textbf{ઇન્ટિગ્રેશન}: કેબિન હીટિંગ માટે હીટ પંપ સિસ્ટમ્સ
\end{itemize}

\textbf{સુરક્ષા સિસ્ટમ્સ}:

\begin{itemize}
\tightlist
\item
  \textbf{HV આઇસોલેશન}: ઇન્સ્યુલેશન મોનિટરિંગ અને કોન્ટેક્ટર કંટ્રોલ
\item
  \textbf{ક્રેશ સેફ્ટી}: અકસ્માતમાં ઓટોમેટિક HV ડિસ્કનેક્ટ
\item
  \textbf{ગ્રાઉન્ડ ફોલ્ટ}: ડિટેક્શન અને પ્રોટેક્શન સિસ્ટમ્સ
\end{itemize}

\end{solutionbox}
\begin{mnemonicbox}
``હાઇ વોલ્ટેજ બેટરી પાવર્સ ટ્રેક્શન થ્રુ કંટ્રોલ - EV આર્કિટેક્ચર
એફિશિયન્ટલી એરેન્જ્ડ''

\end{mnemonicbox}
\begin{center}\rule{0.5\linewidth}{0.5pt}\end{center}

\subsection*{પ્રશ્ન 5(અ) [3
ગુણ]}\label{uxaaauxab0uxab6uxaa8-5uxa85-3-uxa97uxaa3}

\textbf{મોનોક્રિસ્ટેલાઇન સિલિકોન સોલાર સેલ વિશે ટૂંકમાં લખો.}

\begin{solutionbox}

\textbf{મોનોક્રિસ્ટેલાઇન સિલિકોન સોલાર સેલ}:


{\def\LTcaptype{none} % do not increment counter
\vspace{-5pt}
\captionof{table}{મોનોક્રિસ્ટેલાઇન સિલિકોન લક્ષણો}
\vspace{-10pt}
\begin{longtable}[]{@{}lll@{}}
\toprule\noalign{}
પેરામીટર & મૂલ્ય & વર્ણન \\
\midrule\noalign{}
\endhead
\bottomrule\noalign{}
\endlastfoot
કાર્યક્ષમતા & 18-22\% & સિલિકોન સેલ્સમાં સર્વોચ્ચ \\
બંધારણ & સિંગલ ક્રિસ્ટલ & એકસમાન ક્રિસ્ટલ લેટિસ \\
રંગ & ડાર્ક બ્લુ/બ્લેક & એકસમાન દેખાવ \\
આયુષ્ય & 25+ વર્ષ & લાંબગાળાની વિશ્વસનીયતા \\
કિંમત & ઊંચી & પ્રીમિયમ પ્રાઇસિંગ \\
\end{longtable}
}

\textbf{ઉત્પાદન પ્રક્રિયા}:

\begin{itemize}
\tightlist
\item
  \textbf{ઝોક્રાલસ્કી મેથડ}: પીગળેલા સિલિકોનમાંથી સિંગલ ક્રિસ્ટલ વૃદ્ધિ
\item
  \textbf{વેફર કટિંગ}: ક્રિસ્ટલ ઇન્ગોટમાંથી પાતળા સ્લાઇસ કાપવા
\item
  \textbf{ડોપિંગ}: P-type અને N-type પ્રદેશો બનાવવા
\end{itemize}

\textbf{ફાયદાઓ}:

\begin{itemize}
\tightlist
\item
  \textbf{ઉચ્ચ કાર્યક્ષમતા}: વિસ્તાર દીઠ શ્રેષ્ઠ પાવર આઉટપુટ
\item
  \textbf{સ્પેસ એફિશિયન્ટ}: સમાન પાવર માટે ઓછા વિસ્તારની જરૂર
\item
  \textbf{ટકાઉપણું}: લાંબું ઓપરેશનલ જીવન
\end{itemize}

\textbf{ઉપયોગો}:

\begin{itemize}
\tightlist
\item
  \textbf{રહેણાંક સિસ્ટમ્સ}: પ્રીમિયમ રૂફટોપ ઇન્સ્ટોલેશન
\item
  \textbf{કોમર્શિયલ}: ઉચ્ચ કાર્યક્ષમતાની જરૂરિયાતો
\item
  \textbf{સ્પેસ એપ્લિકેશન્સ}: જ્યાં કાર્યક્ષમતા મહત્વપૂર્ણ છે
\end{itemize}

\end{solutionbox}
\begin{mnemonicbox}
``મોનો મીન્સ સિંગલ ક્રિસ્ટલ - મેક્સિમમ એફિશિયન્સી''

\end{mnemonicbox}
\begin{center}\rule{0.5\linewidth}{0.5pt}\end{center}

\subsection*{પ્રશ્ન 5(બ) [4
ગુણ]}\label{uxaaauxab0uxab6uxaa8-5uxaac-4-uxa97uxaa3}

\textbf{ડ્રોનના કાર્યસિદ્ધાંતનું વર્ણન કરો.}

\begin{solutionbox}

\textbf{ડ્રોન કાર્યસિદ્ધાંત}:

\textbf{મૂળભૂત ભૌતિકશાસ્ત્ર}:

\begin{itemize}
\tightlist
\item
  \textbf{લિફ્ટ જનરેશન}: પ્રોપેલર્સ ડાઉનવર્ડ એરફ્લો બનાવે છે (ન્યૂટનનો ત્રીજો કાયદો)
\item
  \textbf{થ્રસ્ટ કંટ્રોલ}: વેરિએબલ પ્રોપેલર સ્પીડ વર્ટિકલ મૂવમેન્ટ કંટ્રોલ કરે છે
\item
  \textbf{સ્ટેબિલિટી}: જાયરોસ્કોપિક ઇફેક્ટ અને એક્ટિવ કંટ્રોલ બેલેન્સ જાળવે છે
\end{itemize}

\textbf{ફ્લાઇટ કંટ્રોલ મેકેનિઝમ}:


{\def\LTcaptype{none} % do not increment counter
\vspace{-5pt}
\captionof{table}{ડ્રોન મૂવમેન્ટ કંટ્રોલ}
\vspace{-10pt}
\begin{longtable}[]{@{}lll@{}}
\toprule\noalign{}
હલનચલન & કંટ્રોલ મેથડ & મોટર એક્શન \\
\midrule\noalign{}
\endhead
\bottomrule\noalign{}
\endlastfoot
ઉપર જવું & બધી મોટર સ્પીડ વધારવી & બધા પ્રોપ્સ ઝડપી \\
નીચે આવવું & બધી મોટર સ્પીડ ઓછી કરવી & બધા પ્રોપ્સ ધીમા \\
આગળ & આગળ ઝુકાવવું & પાછળની મોટર્સ ઝડપી \\
પાછળ & પાછળ ઝુકાવવું & આગળની મોટર્સ ઝડપી \\
ડાબે/જમણે & ડાબે/જમણે બેંક કરવું & વિરુદ્ધ બાજુ ઝડપી \\
ફેરવવું & ટોર્ક ડિફરન્શિયલ & ડાયાગોનલ પેર્સ \\
\end{longtable}
}

\textbf{કંટ્રોલ સિસ્ટમ્સ}:

\begin{itemize}
\tightlist
\item
  \textbf{જાયરોસ્કોપ}: સ્ટેબિલિટી માટે કોણીય વેગ માપે છે
\item
  \textbf{એક્સેલેરોમીટર}: પ્રવેગ અને ટિલ્ટ એન્ગલ શોધે છે
\item
  \textbf{મેગ્નેટોમીટર}: કમ્પાસ હેડિંગ રેફરન્સ
\item
  \textbf{બેરોમીટર}: એલ્ટિટ્યુડ મેઝરમેન્ટ અને હોલ્ડ
\end{itemize}

\textbf{ફ્લાઇટ મોડ્સ}:

\begin{itemize}
\tightlist
\item
  \textbf{મેન્યુઅલ}: ડાયરેક્ટ પાઇલટ કંટ્રોલ
\item
  \textbf{સ્ટેબિલાઇઝ્ડ}: ઓટો-લેવલિંગ સહાયતા
\item
  \textbf{GPS હોલ્ડ}: GPS વાપરીને પોઝિશન હોલ્ડિંગ
\item
  \textbf{ઓટોનોમસ}: પ્રી-પ્રોગ્રામ્ડ ફ્લાઇટ પાથ
\end{itemize}

\end{solutionbox}
\begin{mnemonicbox}
``પ્રોપેલર્સ પુશ એર ડાઉન - ડ્રોન ફ્લાઇઝ અપ''

\end{mnemonicbox}
\begin{center}\rule{0.5\linewidth}{0.5pt}\end{center}

\subsection*{પ્રશ્ન 5(ક) [7
ગુણ]}\label{uxaaauxab0uxab6uxaa8-5uxa95-7-uxa97uxaa3}

\textbf{Raspberry Pi નો બ્લોક ડાયાગ્રામ સમજાવો.}

\begin{solutionbox}

\begin{verbatim}
    ┌─────────────┐    ┌─────────────┐
    │    ARM      │    │   Memory    │
    │ Processor   │◄──►│    (RAM)    │
    │  (Cortex)   │    │   1{-8 GB    │}
    └──────┬──────┘    └─────────────┘
           │
           ▼
    ┌────────────────────────────────┐
    │        System Bus              │
    └──┬──────┬──────┬──────┬────────┘
       │      │      │      │
       ▼      ▼      ▼      ▼
┌──────────┐ ┌────┐ ┌────┐ ┌─────────┐
│   GPIO   │ │USB │ │HDMI│ │Ethernet │
│40 Pins   │ │Port│ │Port│ │  Port   │
└──────────┘ └────┘ └────┘ └─────────┘
       │
       ▼
┌─────────────┐    ┌─────────────┐
│   Storage   │    │   Power     │
│  (microSD)  │    │ Management  │
└─────────────┘    └─────────────┘
\end{verbatim}

\textbf{કોર ઘટકો}:

\textbf{ARM પ્રોસેસર}:

\begin{itemize}
\tightlist
\item
  \textbf{પ્રકાર}: બ્રોડકોમ SoC (સિસ્ટમ ઓન ચિપ)
\item
  \textbf{આર્કિટેક્ચર}: ARM Cortex-A સિરીઝ (32/64-બિટ)
\item
  \textbf{સ્પીડ}: મોડલ પર આધાર રાખીને 1.2-1.8 GHz
\item
  \textbf{લક્ષણો}: ગ્રાફિક્સ પ્રોસેસિંગ માટે બિલ્ટ-ઇન GPU
\end{itemize}

\textbf{મેમોરી (RAM)}:

\begin{itemize}
\tightlist
\item
  \textbf{પ્રકાર}: LPDDR4 SDRAM
\item
  \textbf{કેપેસિટી}: Pi મોડલ પર આધાર રાખીને 1GB થી 8GB
\item
  \textbf{શેર્ડ}: GPU સિસ્ટમ મેમોરી શેર કરે છે
\item
  \textbf{પરફોર્મન્સ}: હાઇ-સ્પીડ મેમોરી ઇન્ટરફેસ
\end{itemize}

\textbf{GPIO (જનરલ પર્પઝ ઇનપુટ/આઉટપુટ)}:

\begin{itemize}
\tightlist
\item
  \textbf{પિન્સ}: બાહ્ય ડિવાઇસ માટે 40-પિન કનેક્ટર
\item
  \textbf{ફંક્શન્સ}: ડિજિટલ I/O, PWM, SPI, I2C, UART
\item
  \textbf{વોલ્ટેજ}: 3.3V લોજિક લેવલ્સ
\item
  \textbf{કરંટ}: સુરક્ષા માટે પિન દીઠ મર્યાદિત કરંટ
\end{itemize}

\textbf{કનેક્ટિવિટી વિકલ્પો}:

\begin{itemize}
\tightlist
\item
  \textbf{USB પોર્ટ્સ}: પેરિફેરલ્સ માટે 2-4 USB 2.0/3.0 પોર્ટ્સ
\item
  \textbf{HDMI}: ડિજિટલ વીડિયો અને ઓડિયો આઉટપુટ
\item
  \textbf{ઇથરનેટ}: વાયર્ડ નેટવર્ક કનેક્ટિવિટી (નવા મોડલ્સ પર ગીગાબિટ)
\item
  \textbf{WiFi/બ્લૂટૂથ}: નવા મોડલ્સ પર બિલ્ટ-ઇન વાયરલેસ
\end{itemize}

\textbf{સ્ટોરેજ}:

\begin{itemize}
\tightlist
\item
  \textbf{microSD}: OS અને ડેટા માટે પ્રાથમિક સ્ટોરેજ
\item
  \textbf{બૂટ}: microSD કાર્ડથી બૂટ કરે છે
\item
  \textbf{કેપેસિટી}: 8GB મિનિમમ, 32GB+ રેકમેન્ડેડ
\end{itemize}

\textbf{પાવર મેનેજમેન્ટ}:

\begin{itemize}
\tightlist
\item
  \textbf{સપ્લાય}: USB-C અથવા micro-USB દ્વારા 5V DC
\item
  \textbf{કરંટ}: 2.5-3A ટિપિકલ રિક્વાયરમેન્ટ
\item
  \textbf{રેગ્યુલેશન}: 3.3V અને 1.8V રેઇલ્સ માટે ઓન-બોર્ડ વોલ્ટેજ રેગ્યુલેટર્સ
\end{itemize}

\textbf{વધારાના લક્ષણો}:

\begin{itemize}
\tightlist
\item
  \textbf{કેમેરા ઇન્ટરફેસ}: Pi કેમેરા માટે CSI કનેક્ટર
\item
  \textbf{ડિસ્પ્લે ઇન્ટરફેસ}: ઓફિશિયલ ટચસ્ક્રીન માટે DSI કનેક્ટર
\item
  \textbf{ઓડિયો}: 3.5mm એનાલોગ ઓડિયો આઉટપુટ
\item
  \textbf{રિયલ-ટાઇમ ક્લોક}: ટાઇમકીપિંગ માટે વૈકલ્પિક RTC
\end{itemize}

\textbf{સોફ્ટવેર સપોર્ટ}:

\begin{itemize}
\tightlist
\item
  \textbf{ઓપરેટિંગ સિસ્ટમ}: Raspberry Pi OS (Debian-આધારિત)
\item
  \textbf{પ્રોગ્રામિંગ}: Python, C++, Scratch, Java સપોર્ટ
\item
  \textbf{GPIO કંટ્રોલ}: હાર્ડવેર ઇન્ટરફેસિંગ માટે લાઇબ્રેરીઓ
\end{itemize}

\textbf{ઉપયોગો}:

\begin{itemize}
\tightlist
\item
  \textbf{શિક્ષણ}: પ્રોગ્રામિંગ અને ઇલેક્ટ્રોનિક્સ શીખવું
\item
  \textbf{IoT પ્રોજેક્ટ્સ}: સેન્સર મોનિટરિંગ, હોમ ઓટોમેશન
\item
  \textbf{મીડિયા સેન્ટર}: વીડિયો સ્ટ્રીમિંગ અને પ્લેબેક
\item
  \textbf{ઇન્ડસ્ટ્રિયલ}: પ્રોટોટાઇપિંગ અને સ્મોલ-સ્કેલ ઓટોમેશન
\end{itemize}

\textbf{ફાયદાઓ}:

\begin{itemize}
\tightlist
\item
  \textbf{કોસ્ટ-ઇફેક્ટિવ}: લો-કોસ્ટ કમ્પ્યુટિંગ પ્લેટફોર્મ
\item
  \textbf{કમ્યુનિટી}: મોટો કમ્યુનિટી સપોર્ટ અને રિસોર્સીસ
\item
  \textbf{ફ્લેક્સિબિલિટી}: I/O ક્ષમતાઓ સાથે જનરલ-પર્પઝ કમ્પ્યુટિંગ
\item
  \textbf{એજ્યુકેશન}: શીખવા અને પ્રયોગ માટે ડિઝાઇન કરવામાં આવ્યું
\end{itemize}

\end{solutionbox}
\begin{mnemonicbox}
``Raspberry Pi પ્રોસેસીસ એવરીથિંગ થ્રુ GPIO - પરફેક્ટ
પ્લેટફોર્મ ફોર પ્રોજેક્ટ્સ''

\end{mnemonicbox}
\begin{center}\rule{0.5\linewidth}{0.5pt}\end{center}

\subsection*{પ્રશ્ન 5(અ) અથવા [3
ગુણ]}\label{uxaaauxab0uxab6uxaa8-5uxa85-uxa85uxaa5uxab5-3-uxa97uxaa3}

\textbf{પોલીક્રિસ્ટેલાઇન સિલિકોન સોલાર સેલ વિશે ટૂંકમાં લખો.}

\begin{solutionbox}

\textbf{પોલીક્રિસ્ટેલાઇન સિલિકોન સોલાર સેલ}:


{\def\LTcaptype{none} % do not increment counter
\vspace{-5pt}
\captionof{table}{પોલીક્રિસ્ટેલાઇન સિલિકોન લક્ષણો}
\vspace{-10pt}
\begin{longtable}[]{@{}lll@{}}
\toprule\noalign{}
પેરામીટર & મૂલ્ય & વર્ણન \\
\midrule\noalign{}
\endhead
\bottomrule\noalign{}
\endlastfoot
કાર્યક્ષમતા & 15-17\% & સારી કાર્યક્ષમતા, મોનો કરતાં ઓછી \\
બંધારણ & બહુવિધ ક્રિસ્ટલ & ગ્રેઇન બાઉન્ડરીઝ દેખાય છે \\
રંગ & બ્લુ સ્પેકલ્ડ & બિન-એકસમાન દેખાવ \\
આયુષ્ય & 25+ વર્ષ & વિશ્વસનીય પરફોર્મન્સ \\
કિંમત & મધ્યમ & કોસ્ટ-ઇફેક્ટિવ વિકલ્પ \\
\end{longtable}
}

\textbf{ઉત્પાદન પ્રક્રિયા}:

\begin{itemize}
\tightlist
\item
  \textbf{કાસ્ટિંગ મેથડ}: પીગળેલા સિલિકોનને ચોરસ મોલ્ડમાં ઠંડું કરવામાં આવે છે
\item
  \textbf{મલ્ટિપલ ક્રિસ્ટલ્સ}: રેન્ડમ ક્રિસ્ટલ ઓરિએન્ટેશન ગ્રેઇન્સ બનાવે છે
\item
  \textbf{વેફર પ્રોડક્શન}: ઓછા વેસ્ટ સાથે ચોરસ વેફર્સ
\end{itemize}

\textbf{ફાયદાઓ}:

\begin{itemize}
\tightlist
\item
  \textbf{કોસ્ટ-ઇફેક્ટિવ}: મોનોક્રિસ્ટેલાઇન કરતાં ઓછી ઉત્પાદન કિંમત
\item
  \textbf{ઓછો વેસ્ટ}: ચોરસ આકાર સામગ્રીનો વેસ્ટ ઘટાડે છે
\item
  \textbf{સારું પરફોર્મન્સ}: મોટાભાગના ઉપયોગો માટે વાજબી કાર્યક્ષમતા
\end{itemize}

\textbf{ઉપયોગો}:

\begin{itemize}
\tightlist
\item
  \textbf{રહેણાંક}: બજેટ-ફ્રેન્ડલી સોલાર ઇન્સ્ટોલેશન્સ
\item
  \textbf{યુટિલિટી સ્કેલ}: મોટા સોલાર ફાર્મ જ્યાં કિંમત મહત્વની છે
\item
  \textbf{કોમર્શિયલ}: મધ્યમ-સ્કેલ ઇન્સ્ટોલેશન્સ
\end{itemize}

\end{solutionbox}
\begin{mnemonicbox}
``પોલી મીન્સ મેની ક્રિસ્ટલ્સ - મોર એફોર્ડેબલ ચોઇસ''

\end{mnemonicbox}
\begin{center}\rule{0.5\linewidth}{0.5pt}\end{center}

\subsection*{પ્રશ્ન 5(બ) અથવા [4
ગુણ]}\label{uxaaauxab0uxab6uxaa8-5uxaac-uxa85uxaa5uxab5-4-uxa97uxaa3}

\textbf{મશીન લર્નિંગ ટેકનિકના પ્રકારોની સરખામણી કરો: સુપરવાઇઝ્ડ અને
અનસુપરવાઇઝ્ડ.}

\begin{solutionbox}


{\def\LTcaptype{none} % do not increment counter
\vspace{-5pt}
\captionof{table}{સુપરવાઇઝ્ડ વિરુદ્ધ અનસુપરવાઇઝ્ડ લર્નિંગ}
\vspace{-10pt}
\begin{longtable}[]{@{}lll@{}}
\toprule\noalign{}
પાસું & સુપરવાઇઝ્ડ લર્નિંગ & અનસુપરવાઇઝ્ડ લર્નિંગ \\
\midrule\noalign{}
\endhead
\bottomrule\noalign{}
\endlastfoot
ડેટા ટાઇપ & લેબલ્ડ ડેટા & અનલેબલ્ડ ડેટા \\
લક્ષ્ય & પ્રિડિક્શન & પેટર્ન ડિસ્કવરી \\
ઉદાહરણો & ક્લાસિફિકેશન, રિગ્રેશન & ક્લસ્ટરિંગ, એસોસિએશન \\
અલ્ગોરિધમ & SVM, ડિસિઝન ટ્રીઝ & K-means, PCA \\
મૂલ્યાંકન & એક્યુરેસી, પ્રિસિઝન & સિલ્હૌએટ સ્કોર \\
\end{longtable}
}

\textbf{સુપરવાઇઝ્ડ લર્નિંગ}:

\begin{itemize}
\tightlist
\item
  \textbf{ટ્રેનિંગ}: શીખવા માટે ઇનપુટ-આઉટપુટ પેર્સ વાપરે છે
\item
  \textbf{પ્રકારો}: ક્લાસિફિકેશન (કેટેગરીઝ) અને રિગ્રેશન (કન્ટિન્યુઅસ વેલ્યુઝ)
\item
  \textbf{ઉપયોગો}: ઇમેઇલ સ્પામ ડિટેક્શન, પ્રાઇસ પ્રિડિક્શન
\end{itemize}

\textbf{અનસુપરવાઇઝ્ડ લર્નિંગ}:

\begin{itemize}
\tightlist
\item
  \textbf{ટ્રેનિંગ}: લેબલ્સ વગર ડેટામાં છુપાયેલા પેટર્ન્સ શોધે છે
\item
  \textbf{પ્રકારો}: ક્લસ્ટરિંગ (ગ્રુપિંગ) અને ડાઇમેન્શનેલિટી રિડક્શન
\item
  \textbf{ઉપયોગો}: કસ્ટમર સેગમેન્ટેશન, એનોમલી ડિટેક્શન
\end{itemize}

\textbf{મુખ્ય તફાવતો}:

\begin{itemize}
\tightlist
\item
  \textbf{ગાઇડન્સ}: સુપરવાઇઝ્ડને શિક્ષક છે, અનસુપરવાઇઝ્ડ સ્વતંત્ર રીતે શીખે છે
\item
  \textbf{કોમ્પ્લેક્સિટી}: સુપરવાઇઝ્ડ વધુ સીધુ, અનસુપરવાઇઝ્ડ વધુ એક્સપ્લોરેટરી
\item
  \textbf{વેલિડેશન}: સુપરવાઇઝ્ડ વેલિડેટ કરવું સરળ, અનસુપરવાઇઝ્ડને ડોમેઇન એક્સપર્ટાઇઝ
  જોઈએ
\end{itemize}

\end{solutionbox}
\begin{mnemonicbox}
``સુપરવાઇઝ્ડ સીઝ સોલ્યુશન્સ, અનસુપરવાઇઝ્ડ અનકવર્સ સીક્રેટ્સ''

\end{mnemonicbox}
\begin{center}\rule{0.5\linewidth}{0.5pt}\end{center}

\subsection*{પ્રશ્ન 5(ક) અથવા [7
ગુણ]}\label{uxaaauxab0uxab6uxaa8-5uxa95-uxa85uxaa5uxab5-7-uxa97uxaa3}

\textbf{સ્માર્ટ હોમનો બ્લોક ડાયાગ્રામ દોરો અને સમજાવો.}

\begin{solutionbox}

\begin{verbatim}
                    ┌─────────────┐
                    │   Smart     │
                    │ Controller  │
                    │  (Hub)      │
                    └──────┬──────┘
                           │
            ┌──────────────┼──────────────┐
            │              │              │
            ▼              ▼              ▼
    ┌─────────────┐ ┌─────────────┐ ┌─────────────┐
    │   Lighting  │ │    HVAC     │ │  Security   │
    │   Control   │ │  Control    │ │   System    │
    └──────┬──────┘ └──────┬──────┘ └──────┬──────┘
           │               │               │
           ▼               ▼               ▼
    ┌─────────────┐ ┌─────────────┐ ┌─────────────┐
    │Smart Bulbs  │ │Thermostat   │ │Door Locks   │
    │\& Switches   │ │\& Sensors    │ │\& Cameras    │
    └─────────────┘ └─────────────┘ └─────────────┘
                           │
                           ▼
                    ┌─────────────┐
                    │ Internet    │
                    │ Gateway     │
                    │ (WiFi/LTE)  │
                    └──────┬──────┘
                           │
                           ▼
                    ┌─────────────┐
                    │ Smartphone  │
                    │    App      │
                    └─────────────┘
\end{verbatim}

\textbf{સ્માર્ટ હોમ સિસ્ટમ ઘટકો}:

\textbf{સ્માર્ટ કંટ્રોલર (હબ)}:

\begin{itemize}
\tightlist
\item
  \textbf{કાર્ય}: તમામ ડિવાઇસનું સમન્વય કરતું કેન્દ્રીય કંટ્રોલ યુનિટ
\item
  \textbf{પ્રોટોકોલ્સ}: ZigBee, Z-Wave, WiFi, બ્લૂટૂથ કમ્યુનિકેશન
\item
  \textbf{પ્રોસેસિંગ}: લોકલ ઓટોમેશન રૂલ્સ અને રિમોટ કનેક્ટિવિટી
\item
  \textbf{ઇન્ટિગ્રેશન}: વોઇસ આસિસ્ટન્ટ (Alexa, Google) સાથે કામ કરે છે
\end{itemize}

\textbf{લાઇટિંગ કંટ્રોલ સિસ્ટમ}:

\begin{itemize}
\tightlist
\item
  \textbf{સ્માર્ટ બલ્બ્સ}: વાયરલેસ કનેક્ટિવિટી સાથે LED બલ્બ્સ
\item
  \textbf{સ્માર્ટ સ્વિચીસ}: હાલની લાઇટિંગને સ્માર્ટ કંટ્રોલ સાથે રેટ્રોફિટ
\item
  \textbf{લક્ષણો}: ડિમિંગ, કલર ચેન્જિંગ, શેડ્યુલિંગ, મોશન સેન્સિંગ
\item
  \textbf{એનર્જી સેવિંગ}: ઓક્યુપન્સી આધારિત ઓટોમેટિક ઓન/ઓફ
\end{itemize}

\textbf{HVAC કંટ્રોલ સિસ્ટમ}:

\begin{itemize}
\tightlist
\item
  \textbf{સ્માર્ટ થર્મોસ્ટેટ}: પ્રોગ્રામેબલ ટેમ્પરેચર કંટ્રોલ
\item
  \textbf{સેન્સર્સ}: તાપમાન, ભેજ, ઓક્યુપન્સી ડિટેક્શન
\item
  \textbf{લર્નિંગ}: ઉપયોગ પેટર્ન આધારિત એડેપ્ટિવ શેડ્યુલિંગ
\item
  \textbf{એફિશિયન્સી}: એનર્જી ઓપ્ટિમાઇઝેશન અને રિમોટ કંટ્રોલ
\end{itemize}

\textbf{સિક્યુરિટી સિસ્ટમ}:

\begin{itemize}
\tightlist
\item
  \textbf{સ્માર્ટ લોક્સ}: સ્માર્ટફોન કંટ્રોલ સાથે કીલેસ એન્ટ્રી
\item
  \textbf{કેમેરાઝ}: રેકોર્ડિંગ સાથે ઇન્ડોર/આઉટડોર સર્વેલન્સ
\item
  \textbf{સેન્સર્સ}: ડોર/વિન્ડો, મોશન, ગ્લાસ બ્રેક ડિટેક્શન
\item
  \textbf{એલર્ટ્સ}: સ્માર્ટફોનને રિયલ-ટાઇમ નોટિફિકેશન્સ
\end{itemize}

\textbf{ઇન્ટરનેટ ગેટવે}:

\begin{itemize}
\tightlist
\item
  \textbf{કનેક્ટિવિટી}: ક્લાઉડ સર્વિસીસ માટે હાઇ-સ્પીડ ઇન્ટરનેટ
\item
  \textbf{રાઉટર}: ડિવાઇસ કનેક્ટિવિટી માટે WiFi નેટવર્ક
\item
  \textbf{સિક્યુરિટી}: નેટવર્ક ફાયરવોલ અને ડિવાઇસ ઓથેન્ટિકેશન
\item
  \textbf{બેકઅપ}: ક્રિટિકલ ફંક્શન્સ માટે સેલ્યુલર બેકઅપ
\end{itemize}

\textbf{સ્માર્ટફોન ઇન્ટિગ્રેશન}:

\begin{itemize}
\tightlist
\item
  \textbf{મોબાઇલ એપ}: રિમોટ કંટ્રોલ અને મોનિટરિંગ ઇન્ટરફેસ
\item
  \textbf{વોઇસ કંટ્રોલ}: વોઇસ આસિસ્ટન્ટ્સ સાથે ઇન્ટિગ્રેશન
\item
  \textbf{ઓટોમેશન}: સીન ક્રિએશન અને શેડ્યુલિંગ
\item
  \textbf{નોટિફિકેશન્સ}: સિક્યુરિટી એલર્ટ્સ અને સિસ્ટમ સ્ટેટસ
\end{itemize}

\textbf{સ્માર્ટ હોમ લક્ષણો}:

\textbf{ઓટોમેશન સિનેરિયોઝ}:

\begin{itemize}
\tightlist
\item
  \textbf{ગુડ મોર્નિંગ}: લાઇટ્સ ઓન, કોફી મેકર સ્ટાર્ટ, થર્મોસ્ટેટ એડજસ્ટ
\item
  \textbf{અવે મોડ}: બધી લાઇટ્સ ઓફ, સિક્યુરિટી આર્મ્ડ, થર્મોસ્ટેટ સેટબેક
\item
  \textbf{ગુડ નાઇટ}: ડોર્સ લોક, લાઇટ્સ ડિમ, સિક્યુરિટી સેન્સર્સ એક્ટિવ
\item
  \textbf{મૂવી મોડ}: લાઇટ્સ ડિમ, બ્લાઇન્ડ્સ ક્લોઝ, એન્ટરટેનમેન્ટ સિસ્ટમ ઓન
\end{itemize}

\textbf{એનર્જી મેનેજમેન્ટ}:

\begin{itemize}
\tightlist
\item
  \textbf{લોડ મોનિટરિંગ}: ડિવાઇસ દ્વારા એનર્જી ઉપયોગ ટ્રેક કરવું
\item
  \textbf{પીક શેવિંગ}: ઊંચા ઇલેક્ટ્રિસિટી રેટ પીરિયડ્સ ટાળવા
\item
  \textbf{સોલાર ઇન્ટિગ્રેશન}: સોલાર પેનલ્સ અને બેટરીઝ સાથે સમન્વય
\item
  \textbf{સ્માર્ટ એપ્લાયન્સીસ}: ડિશવોશર, વોશર લો-કોસ્ટ અવર્સ દરમિયાન ચલાવવા
\end{itemize}

\textbf{સિક્યુરિટી લક્ષણો}:

\begin{itemize}
\tightlist
\item
  \textbf{પેરિમીટર પ્રોટેક્શન}: ડોર/વિન્ડો સેન્સર્સ, કેમેરાઝ
\item
  \textbf{ઇન્ટીરિયર પ્રોટેક્શન}: મોશન સેન્સર્સ, ગ્લાસ બ્રેક ડિટેક્ટર્સ
\item
  \textbf{એક્સેસ કંટ્રોલ}: સ્માર્ટ લોક્સ, કીપેડ એન્ટ્રી, વિઝિટર મેનેજમેન્ટ
\item
  \textbf{ઇમર્જન્સી રિસ્પોન્સ}: સિક્યુરિટી કંપનીને ઓટોમેટિક એલર્ટ્સ
\end{itemize}

\textbf{ફાયદા}:

\begin{itemize}
\tightlist
\item
  \textbf{કન્વીનિયન્સ}: રિમોટ કંટ્રોલ અને ઓટોમેશન
\item
  \textbf{એનર્જી એફિશિયન્સી}: ઓપ્ટિમાઇઝ્ડ ઉપયોગ પેટર્ન્સ
\item
  \textbf{સિક્યુરિટી}: વધારેલા ઘર સુરક્ષા
\item
  \textbf{કમ્ફર્ટ}: પર્સનલાઇઝ્ડ એનવાયરનમેન્ટ કંટ્રોલ
\item
  \textbf{પ્રોપર્ટી વેલ્યુ}: વધારેલું ઘર મૂલ્ય
\end{itemize}

\textbf{કમ્યુનિકેશન પ્રોટોકોલ્સ}:

\begin{itemize}
\tightlist
\item
  \textbf{WiFi}: કેમેરાઝ અને સ્ટ્રીમિંગ માટે હાઇ બેન્ડવિડ્થ
\item
  \textbf{ZigBee}: સેન્સર્સ માટે લો પાવર મેશ નેટવર્ક
\item
  \textbf{Z-Wave}: ક્રિટિકલ ડિવાઇસ માટે વિશ્વસનીય મેશ
\item
  \textbf{બ્લૂટૂથ}: શોર્ટ-રેન્જ ડાયરેક્ટ ડિવાઇસ કનેક્શન
\end{itemize}

\textbf{ભવિષ્યના ટ્રેન્ડ્સ}:

\begin{itemize}
\tightlist
\item
  \textbf{AI ઇન્ટિગ્રેશન}: બેટર ઓટોમેશન માટે મશીન લર્નિંગ
\item
  \textbf{એજ કમ્પ્યુટિંગ}: ઝડપી રિસ્પોન્સ માટે લોકલ પ્રોસેસિંગ
\item
  \textbf{એનર્જી સ્ટોરેજ}: બેટરી બેકઅપ અને ગ્રિડ સર્વિસીસ
\item
  \textbf{હેલ્થ મોનિટરિંગ}: એર ક્વોલિટી, સ્લીપ ટ્રેકિંગ ઇન્ટિગ્રેશન
\end{itemize}

\end{solutionbox}
\begin{mnemonicbox}
``સ્માર્ટ હોમ્સ કંટ્રોલ એવરીથિંગ થ્રુ ઇન્ટરનેટ - કન્વીનિયન્સ
કમ્ફર્ટ સિક્યુરિટી એફિશિયન્સી''

\end{mnemonicbox}

\end{document}
