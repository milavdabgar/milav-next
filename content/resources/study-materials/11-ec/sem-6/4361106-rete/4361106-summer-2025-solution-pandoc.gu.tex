\documentclass[10pt,a4paper]{article}

% content/resources/templates/preamble.tex
\usepackage[margin=0.6in]{geometry}
\author{Milav Dabgar}
\usepackage{amsmath,amssymb,amsthm}
\usepackage{booktabs}
\usepackage{multirow}
\usepackage{xcolor}
\usepackage{tcolorbox}
\tcbuselibrary{breakable,skins}
\usepackage[colorlinks=true,linkcolor=blue]{hyperref}
\usepackage{titlesec}
\usepackage{enumitem}
\usepackage{tikz}
\usepackage{pgfplots}
\usepackage{circuitikz}
\usepackage[version=4]{mhchem}
\usepackage{longtable}
\usepackage{array}
\usepackage{float}
\usepackage{caption}
\usepackage{listings}

\lstset{
  basicstyle=\small\ttfamily,
  breaklines=true,
  breakatwhitespace=false,
  postbreak=\mbox{\textcolor{red}{$\hookrightarrow$}\space},
  float=false,
  numbers=left,
  numberstyle=\tiny\color{gray},
  numbersep=10pt,
  xleftmargin=2em,
  keywordstyle=\color{blue},
  commentstyle=\color{green!60!black},
  stringstyle=\color{purple},
  backgroundcolor=\color{gray!5},
  showstringspaces=false,
  tabsize=2,
  captionpos=b,
  keepspaces=true,
  columns=flexible
}

\pgfplotsset{compat=1.18}
\usetikzlibrary{shapes,arrows,positioning,calc,patterns,decorations.pathmorphing,decorations.markings,arrows.meta}

% Color scheme
\definecolor{headcolor}{RGB}{0,102,204}
\definecolor{keycolor}{RGB}{220,20,60}
\definecolor{solutioncolor}{RGB}{34,139,34}
\definecolor{mnemoniccolor}{RGB}{148,0,211}
\definecolor{codecolor}{RGB}{0,0,100}

% Spacing
\setlength{\parskip}{3pt}
\setlist[itemize]{nosep}
\setlist[enumerate]{nosep}

% Title formatting
\titleformat{\section}{\Large\bfseries\color{headcolor}}{\thesection}{1em}{}
\titleformat{\subsection}{\large\bfseries\color{headcolor}}{\thesubsection}{1em}{}

% Pandoc tightlist compatibility
\providecommand{\tightlist}{%
  \setlength{\itemsep}{0pt}\setlength{\parskip}{0pt}}

% Pandoc longtable compatibility
\newcounter{none}
\def\thenone{}


% content/resources/templates/gujarati-boxes.tex
\usepackage{fontspec}
\usepackage{polyglossia}

% Set Gujarati as main language (document is primarily in Gujarati)
% Note: gloss-gujarati.ldf doesn't exist in polyglossia, but it will use hyphenation patterns
\setdefaultlanguage{gujarati}
\setotherlanguage{english}

% Configure Gujarati font properly
% Use Language=Default to prevent polyglossia from trying to add language-specific features
% that don't exist for Gujarati, which causes "empty feature" warnings
\newfontfamily\gujaratifont[Script=Gujarati,AutoFakeBold=2.5,AutoFakeSlant=0.3]{Noto Sans Gujarati}
\setmainfont[Script=Gujarati,AutoFakeBold=2.5,AutoFakeSlant=0.3]{Noto Sans Gujarati}
% Use Noto Sans Gujarati for monospace to support Gujarati in text
\setmonofont[Scale=0.9]{Noto Sans Gujarati}

% Configure English to use the same font
\newfontfamily\englishfont[Script=Gujarati,AutoFakeBold=2.5,AutoFakeSlant=0.3]{Noto Sans Gujarati}

% Translations for polyglossia
\gappto\captionsgujarati{
  \renewcommand{\tablename}{કોષ્ટક}
  \renewcommand{\figurename}{આકૃતિ}
}

% Helper for TikZ nodes to ensure Gujarati font
\newcommand{\gu}[1]{{\gujaratifont #1}}

% Custom environments
\newtcolorbox{solutionbox}{
    breakable,
    enhanced,
    colback=solutioncolor!5!white,
    colframe=solutioncolor!75!black,
    fonttitle=\bfseries,
    title=જવાબ
}

\newtcolorbox{solutionboxnobreak}{
 colback=solutioncolor!5!white,
 colframe=solutioncolor!75!black,
 fonttitle=\bfseries,
 title=જવાબ
}

\newtcolorbox{keyformula}{
 breakable,
 enhanced,
 colback=keycolor!5!white,
 colframe=keycolor!75!black,
 fonttitle=\bfseries,
 title=રાસાયણિક સમીકરણ/સૂત્ર
}

\newtcolorbox{mnemonicbox}{
 breakable,
 enhanced,
 colback=mnemoniccolor!5!white,
 colframe=mnemoniccolor!75!black,
 fonttitle=\bfseries,
 title=મેમરી ટ્રીક
}


\begin{document}

\begin{center}
{\Huge\bfseries\color{headcolor} Subject Name (Gujarati)}\\[5pt]
{\LARGE 4361106 -- Summer 2025}\\[3pt]
{\large Semester 1 Study Material}\\[3pt]
{\normalsize\textit{Detailed Solutions and Explanations}}
\end{center}

\vspace{10pt}

\subsection*{પ્રશ્ન 1(અ) [3
ગુણ]}\label{uxaaauxab0uxab6uxaa8-1uxa85-3-uxa97uxaa3}

\textbf{રિન્યુએબલ એનર્જીની વ્યાખ્યા આપો અને તેનું મહત્વ સમજાવો.}

\begin{solutionbox}

\textbf{રિન્યુએબલ એનર્જી} એ કુદરતી સ્ત્રોતોમાંથી મેળવવામાં આવતી ઊર્જા છે જે સતત
ભરપાઈ થતી રહે છે, જેમ કે સૌર, પવન, પાણી, બાયોમાસ અને ભૂગર્ભીય ઊર્જા.


{\def\LTcaptype{none} % do not increment counter
\vspace{-5pt}
\captionof{table}{રિન્યુએબલ એનર્જી સ્ત્રોતોના પ્રકારો}
\vspace{-10pt}
\begin{longtable}[]{@{}lll@{}}
\toprule\noalign{}
પ્રકાર & સ્ત્રોત & ફાયદો \\
\midrule\noalign{}
\endhead
\bottomrule\noalign{}
\endlastfoot
\textbf{સોલર} & સૂર્યનું કિરણોત્સર્ગ & સ્વચ્છ, પુષ્કળ \\
\textbf{વિન્ડ} & હવાની હલનચલન & કોઈ ઉત્સર્જન નહીં \\
\textbf{હાઇડ્રો} & પાણીનો પ્રવાહ & વિશ્વસનીય પાવર \\
\textbf{બાયોમાસ} & કાર્બનિક પદાર્થ & કાર્બન તટસ્થ \\
\end{longtable}
}

\textbf{મહત્વ:}

\begin{itemize}
\tightlist
\item
  \textbf{પર્યાવરણ સુરક્ષા}: પ્રદૂષણ અને ગ્રીનહાઉસ ગેસો ઘટાડે છે
\item
  \textbf{ઊર્જા સુરક્ષા}: અશ્મિભૂત ઇંધન પર નિર્ભરતા ઘટાડે છે
\item
  \textbf{આર્થિક ફાયદા}: રોજગાર સર્જન અને ઊર્જા ખર્ચ ઘટાડે છે
\end{itemize}

\end{solutionbox}
\begin{mnemonicbox}
``SEEB'' - સોલર, એન્વાયર્નમેન્ટલ, ઇકોનોમિક, બાયોમાસ

\end{mnemonicbox}
\subsection*{પ્રશ્ન 1(બ) [4
ગુણ]}\label{uxaaauxab0uxab6uxaa8-1uxaac-4-uxa97uxaa3}

\textbf{સૌર ફોટોવોલ્ટેઇક અસર અને ફોટોવોલ્ટેઇક રૂપાંતરનો સિદ્ધાંત સમજાવો.}

\begin{solutionbox}

\textbf{ફોટોવોલ્ટેઇક અસર} એ સેમિકંડક્ટર પદાર્થ પર પ્રકાશ પડવાથી વિદ્યુત વિવાહની
ઉત્પત્તિ છે.

\textbf{કાર્યસિદ્ધાંત:}

\begin{itemize}
\tightlist
\item
  \textbf{ફોટોન શોષણ}: પ્રકાશ ફોટોન્સ સોલર સેલની સપાટી પર અથડાય છે
\item
  \textbf{ઇલેક્ટ્રોન ઉત્તેજના}: ઇલેક્ટ્રોન્સ ઊર્જા મેળવે છે અને કંડક્શન બેન્ડમાં જાય છે
\item
  \textbf{ચાર્જ વિભાજન}: બિલ્ટ-ઇન ઇલેક્ટ્રિક ફીલ્ડ પોઝિટિવ અને નેગેટિવ ચાર્જ અલગ
  કરે છે
\item
  \textbf{કરંટ ઉત્પાદન}: ઇલેક્ટ્રોન્સનો પ્રવાહ DC વીજળી બનાવે છે
\end{itemize}

\textbf{આકૃતિ:}

\begin{verbatim}
    Light Photons
         ↓
    ┌─────────────┐
    │  P{-type     │  Holes (+)}
    │─────────────│  Junction
    │  N{-type     │  Electrons ({-})}
    └─────────────┘
         ↓
    Electric Current
\end{verbatim}

\end{solutionbox}
\begin{mnemonicbox}
``PACE'' - ફોટોન્સ, શોષણ, ચાર્જ, ઇલેક્ટ્રિસિટી

\end{mnemonicbox}
\subsection*{પ્રશ્ન 1(ક) [7
ગુણ]}\label{uxaaauxab0uxab6uxaa8-1uxa95-7-uxa97uxaa3}

\textbf{ઇલેક્ટ્રિક વ્હીકલ (EV) ના પ્રકારો અને EV માટે વિવિધ ઊર્જા સ્ત્રોતોનું વર્ણન
કરો.}

\begin{solutionbox}


{\def\LTcaptype{none} % do not increment counter
\vspace{-5pt}
\captionof{table}{ઇલેક્ટ્રિક વ્હીકલના પ્રકારો}
\vspace{-10pt}
\begin{longtable}[]{@{}llll@{}}
\toprule\noalign{}
EV પ્રકાર & સંપૂર્ણ સ્વરૂપ & પાવર સ્ત્રોત & રેંજ \\
\midrule\noalign{}
\endhead
\bottomrule\noalign{}
\endlastfoot
\textbf{BEV} & બેટરી ઇલેક્ટ્રિક વ્હીકલ & માત્ર બેટરી & 150-400 કિમી \\
\textbf{HEV} & હાયબ્રિડ ઇલેક્ટ્રિક વ્હીકલ & બેટરી + એન્જિન & 600+ કિમી \\
\textbf{PHEV} & પ્લગ-ઇન હાયબ્રિડ & બેટરી + એન્જિન & 50-100 કિમી ઇલેક્ટ્રિક \\
\textbf{FCEV} & ફ્યુઅલ સેલ ઇલેક્ટ્રિક & હાઇડ્રોજન ફ્યુઅલ સેલ & 400-600 કિમી \\
\end{longtable}
}

\textbf{EV માટે ઊર્જા સ્ત્રોતો:}

\begin{itemize}
\tightlist
\item
  \textbf{બેટરી}: લિથિયમ-આયન બેટરીઓ વિદ્યુત ઊર્જા સંગ્રહ કરે છે
\item
  \textbf{ફ્યુઅલ સેલ}: હાઇડ્રોજનને વીજળીમાં રૂપાંતરિત કરે છે
\item
  \textbf{અલ્ટ્રાકેપેસિટર}: ઝડપી ઊર્જા સંગ્રહ અને છોડવાની પ્રક્રિયા
\item
  \textbf{ફ્લાયવ્હીલ}: યાંત્રિક ઊર્જા સંગ્રહ
\item
  \textbf{રિજનરેટિવ બ્રેકિંગ}: બ્રેકિંગ દરમિયાન ઊર્જા પુનઃપ્રાપ્ત કરે છે
\item
  \textbf{હાયબ્રિડ સ્ત્રોતો}: બહુવિધ ઊર્જા સ્ત્રોતોનું સંયોજન
\end{itemize}

\textbf{આકૃતિ: EV આર્કિટેક્ચર}

\begin{verbatim}
┌────────────┐    ┌─────────────┐    ┌──────────┐
│   Battery  │────│  Controller │────│  Motor   │
└────────────┘    └─────────────┘    └──────────┘
                         │
                  ┌─────────────┐
                  │ Charging    │
                  │ System      │
                  └─────────────┘
\end{verbatim}

\end{solutionbox}
\begin{mnemonicbox}
``BHPF-BUFR'' - બેટરી, હાયબ્રિડ, પ્લગઇન, ફ્યુઅલસેલ - બેટરી,
અલ્ટ્રાકેપ, ફ્લાયવ્હીલ, રિજન

\end{mnemonicbox}
\subsection*{પ્રશ્ન 1(ક) અથવા [7
ગુણ]}\label{uxaaauxab0uxab6uxaa8-1uxa95-uxa85uxaa5uxab5-7-uxa97uxaa3}

\textbf{વિવિધ પ્રકારના રિન્યુએબલ ઊર્જા સ્ત્રોતોની ચર્ચા કરો.}

\begin{solutionbox}


{\def\LTcaptype{none} % do not increment counter
\vspace{-5pt}
\captionof{table}{રિન્યુએબલ ઊર્જા સ્ત્રોતોની સરખામણી}
\vspace{-10pt}
\begin{longtable}[]{@{}
  >{\raggedright\arraybackslash}p{(\linewidth - 6\tabcolsep) * \real{0.1795}}
  >{\raggedright\arraybackslash}p{(\linewidth - 6\tabcolsep) * \real{0.4615}}
  >{\raggedright\arraybackslash}p{(\linewidth - 6\tabcolsep) * \real{0.1795}}
  >{\raggedright\arraybackslash}p{(\linewidth - 6\tabcolsep) * \real{0.1795}}@{}}
\toprule\noalign{}
\begin{minipage}[b]{\linewidth}\raggedright
સ્ત્રોત
\end{minipage} & \begin{minipage}[b]{\linewidth}\raggedright
કેવી રીતે કામ કરે છે
\end{minipage} & \begin{minipage}[b]{\linewidth}\raggedright
ફાયદા
\end{minipage} & \begin{minipage}[b]{\linewidth}\raggedright
ઉપયોગ
\end{minipage} \\
\midrule\noalign{}
\endhead
\bottomrule\noalign{}
\endlastfoot
\textbf{સૌર} & સૂર્યપ્રકાશને વીજળીમાં રૂપાંતરિત કરે છે & સ્વચ્છ, પુષ્કળ & રૂફટોપ
સિસ્ટમ, ફાર્મ \\
\textbf{પવન} & પવન ટર્બાઇન ફેરવે છે & કોઈ ઇંધન ખર્ચ નથી & વિન્ડ ફાર્મ, ઓફશોર \\
\textbf{હાઇડ્રોઇલેક્ટ્રિક} & પાણીનો પ્રવાહ પાવર જનરેટ કરે છે & વિશ્વસનીય, લાંબા
સમય સુધી ચાલે છે & ડેમ, નદીઓ \\
\textbf{બાયોમાસ} & કાર્બનિક પદાર્થોનું દહન & કાર્બન તટસ્થ & પાવર પ્લાન્ટ,
હીટિંગ \\
\textbf{જીઓથર્મલ} & પૃથ્વીની ગરમ ઊર્જા & સતત ઉપલબ્ધતા & હીટિંગ, વીજળી \\
\end{longtable}
}

\textbf{ઉભરતા વલણો:}

\begin{itemize}
\tightlist
\item
  \textbf{ટાઇડલ વેવ}: મહાસાગરની તરંગ ઊર્જા રૂપાંતરણ
\item
  \textbf{સૌર થર્મલ}: કેન્દ્રિત સૌર ઊર્જા સિસ્ટમ
\item
  \textbf{હાઇડ્રોજન}: રિન્યુએબલ સ્ત્રોતોમાંથી સ્વચ્છ ઇંધન
\end{itemize}

\textbf{ફાયદા:}

\begin{itemize}
\tightlist
\item
  \textbf{ટકાઉપણું}: ક્યારેય ખતમ થતું નથી
\item
  \textbf{પર્યાવરણીય}: ન્યુનતમ પ્રદૂષણ
\item
  \textbf{આર્થિક}: લાંબા ગાળે ઊર્જા ખર્ચ ઘટાડે છે
\end{itemize}

\end{solutionbox}
\begin{mnemonicbox}
``SWHBG-THS'' - સૌર, વિન્ડ, હાઇડ્રો, બાયોમાસ, જીઓથર્મલ
- ટાઇડલ, હાઇડ્રોજન, સૌર થર્મલ

\end{mnemonicbox}
\subsection*{પ્રશ્ન 2(અ) [3
ગુણ]}\label{uxaaauxab0uxab6uxaa8-2uxa85-3-uxa97uxaa3}

\textbf{નેનોટેકનોલોજી વ્યાખ્યાયિત કરો અને નેનોટેકનોલોજીની એપ્લિકેશનોની સૂચિ
બનાવો.}

\begin{solutionbox}

\textbf{નેનોટેકનોલોજી} એ અણુ અને આણવિક સ્તરે (1-100 નેનોમીટર) પદાર્થનું હેરફેર કરવાનું
વિજ્ઞાન છે.

\textbf{એપ્લિકેશનો:}

\begin{itemize}
\tightlist
\item
  \textbf{ઇલેક્ટ્રોનિક્સ}: નાના, ઝડપી પ્રોસેસર
\item
  \textbf{મેડિસિન}: દવા પહોંચાડવાની સિસ્ટમ
\item
  \textbf{ઊર્જા}: સૌર સેલ, બેટરીઓ
\item
  \textbf{સામગ્રી}: મજબૂત, હળવા કમ્પોઝિટ
\end{itemize}

\end{solutionbox}
\begin{mnemonicbox}
``NEMS'' - નેનો ઇલેક્ટ્રોનિક્સ, મેડિસિન, સૌર

\end{mnemonicbox}
\subsection*{પ્રશ્ન 2(બ) [4
ગુણ]}\label{uxaaauxab0uxab6uxaa8-2uxaac-4-uxa97uxaa3}

\textbf{સંપૂર્ણ સ્વરૂપો આપો: UAV, IOT, AI, M2M}

\begin{solutionbox}


{\def\LTcaptype{none} % do not increment counter
\vspace{-5pt}
\captionof{table}{ટેકનોલોજી સંક્ષેપો}
\vspace{-10pt}
\begin{longtable}[]{@{}lll@{}}
\toprule\noalign{}
સંક્ષેપ & સંપૂર્ણ સ્વરૂપ & એપ્લિકેશન \\
\midrule\noalign{}
\endhead
\bottomrule\noalign{}
\endlastfoot
\textbf{UAV} & અનમેન્ડ એરિયલ વ્હીકલ & સર્વેલન્સ, ડિલિવરી \\
\textbf{IOT} & ઇન્ટરનેટ ઓફ થિંગ્સ & સ્માર્ટ હોમ, શહેરો \\
\textbf{AI} & આર્ટિફિશિયલ ઇન્ટેલિજન્સ & મશીન લર્નિંગ, ઓટોમેશન \\
\textbf{M2M} & મશીન ટુ મશીન & ઇન્ડસ્ટ્રિયલ ઓટોમેશન \\
\end{longtable}
}

\end{solutionbox}
\begin{mnemonicbox}
``UIAM'' - UAV, IOT, AI, M2M

\end{mnemonicbox}
\subsection*{પ્રશ્ન 2(ક) [7
ગુણ]}\label{uxaaauxab0uxab6uxaa8-2uxa95-7-uxa97uxaa3}

\textbf{ડ્રોનના બ્લોક ડાયાગ્રામ અને તેના મુખ્ય ઘટકોનું વર્ણન કરો.}

\begin{solutionbox}

\textbf{બ્લોક ડાયાગ્રામ:}

\begin{center}
\textbf{Mermaid Diagram (Code)}
\begin{verbatim}
{Shaded}
{Highlighting}[]
graph TD
    A[ફ્લાઇટ કંટ્રોલર] {-{-}{} B[મોટર્સ અને પ્રોપેલર્સ]}
    A {-{-}{} C[GPS મોડ્યુલ]}
    A {-{-}{} D[IMU સેન્સર્સ]}
    A {-{-}{} E[કેમેરા]}
    F[બેટરી] {-{-}{} A}
    G[રિમોટ કંટ્રોલર] {-{-}{} H[રિસીવર]}
    H {-{-}{} A}
    A {-{-}{} I[ગિમ્બલ]}
{Highlighting}
{Shaded}
\end{verbatim}
\end{center}

\textbf{મુખ્ય ઘટકો:}

\begin{itemize}
\tightlist
\item
  \textbf{ફ્લાઇટ કંટ્રોલર}: ડ્રોનનું મગજ, સેન્સર ડેટાને પ્રોસેસ કરે છે
\item
  \textbf{મોટર્સ અને પ્રોપેલર્સ}: થ્રસ્ટ અને કંટ્રોલ મૂવમેન્ટ પ્રદાન કરે છે
\item
  \textbf{બેટરી}: બધા ઇલેક્ટ્રોનિક ઘટકોને પાવર આપે છે
\item
  \textbf{GPS મોડ્યુલ}: સ્થાન અને નેવિગેશન ડેટા પ્રદાન કરે છે
\item
  \textbf{IMU સેન્સર્સ}: પ્રવેગ, પરિભ્રમણ, ચુંબકીય ક્ષેત્ર માપે છે
\item
  \textbf{કેમેરા}: છબીઓ અને વીડિયો કેપ્ચર કરે છે
\item
  \textbf{ગિમ્બલ}: સરળ ફૂટેજ માટે કેમેરાને સ્થિર કરે છે
\end{itemize}

\textbf{કાર્યસિદ્ધાંત:}

\begin{itemize}
\tightlist
\item
  \textbf{કંટ્રોલ}: રિમોટ રિસીવરને કમાન્ડ મોકલે છે
\item
  \textbf{પ્રોસેસિંગ}: ફ્લાઇટ કંટ્રોલર કમાન્ડનું અર્થઘટન કરે છે
\item
  \textbf{સ્થિરીકરણ}: IMU સેન્સર સંતુલન જાળવે છે
\item
  \textbf{નેવિગેશન}: GPS પોઝિશન ફીડબેક પ્રદાન કરે છે
\end{itemize}

\end{solutionbox}
\begin{mnemonicbox}
``FMBGIC'' - ફ્લાઇટ કંટ્રોલર, મોટર્સ, બેટરી, GPS, IMU,
કેમેરા

\end{mnemonicbox}
\subsection*{પ્રશ્ન 2(અ) અથવા [3
ગુણ]}\label{uxaaauxab0uxab6uxaa8-2uxa85-uxa85uxaa5uxab5-3-uxa97uxaa3}

\textbf{IOT અને તેના મહત્વની ચર્ચા કરો.}

\begin{solutionbox}

\textbf{ઇન્ટરનેટ ઓફ થિંગ્સ (IOT)} રોજિંદા ઉપકરણોને ડેટા એક્સચેન્જ અને રિમોટ કંટ્રોલ
માટે ઇન્ટરનેટ સાથે જોડે છે.

\textbf{મહત્વ:}

\begin{itemize}
\tightlist
\item
  \textbf{ઓટોમેશન}: સ્માર્ટ હોમ અને શહેરો
\item
  \textbf{કાર્યક્ષમતા}: સંસાધનોનો ઓપ્ટિમાઇઝ્ડ ઉપયોગ
\item
  \textbf{મોનિટરિંગ}: રીઅલ-ટાઇમ ડેટા કલેક્શન
\end{itemize}

\end{solutionbox}
\begin{mnemonicbox}
``AEM'' - ઓટોમેશન, કાર્યક્ષમતા, મોનિટરિંગ

\end{mnemonicbox}
\subsection*{પ્રશ્ન 2(બ) અથવા [4
ગુણ]}\label{uxaaauxab0uxab6uxaa8-2uxaac-uxa85uxaa5uxab5-4-uxa97uxaa3}

\textbf{વેરેબલ ટેકનોલોજી વ્યાખ્યાયિત કરો. વેરેબલ ટેકનોલોજીની ઓછામાં ઓછી ત્રણ
એપ્લિકેશનના નામ આપો.}

\begin{solutionbox}

\textbf{વેરેબલ ટેકનોલોજી} એ શરીર પર પહેરવામાં આવતા ઇલેક્ટ્રોનિક ઉપકરણોનો સંદર્ભ આપે
છે જે આરોગ્ય, ફિટનેસ અથવા માહિતી પ્રદાન કરવા માટે મોનિટર કરે છે.

\textbf{એપ્લિકેશનો:}

\begin{itemize}
\tightlist
\item
  \textbf{સ્માર્ટ વોચ}: ફિટનેસ ટ્રેકિંગ, નોટિફિકેશન
\item
  \textbf{સ્માર્ટ ગ્લાસ}: ઓગમેન્ટેડ રિયાલિટી, નેવિગેશન
\item
  \textbf{હેલ્થ મોનિટર્સ}: હાર્ટ રેટ, બ્લડ પ્રેશર મોનિટરિંગ
\end{itemize}

\end{solutionbox}
\begin{mnemonicbox}
``WSH'' - વોચ, સ્માર્ટ ગ્લાસ, હેલ્થ મોનિટર્સ

\end{mnemonicbox}
\subsection*{પ્રશ્ન 2(ક) અથવા [7
ગુણ]}\label{uxaaauxab0uxab6uxaa8-2uxa95-uxa85uxaa5uxab5-7-uxa97uxaa3}

\textbf{બ્લોક ડાયાગ્રામની મદદથી સ્માર્ટ સ્ટ્રીટ લાઇટ કંટ્રોલ અને મોનિટરિંગ
સમજાવો.}

\begin{solutionbox}

\textbf{બ્લોક ડાયાગ્રામ:}

\begin{center}
\textbf{Mermaid Diagram (Code)}
\begin{verbatim}
{Shaded}
{Highlighting}[]
graph TD
    A[લાઇટ સેન્સર] {-{-}{} B[માઇક્રોકંટ્રોલર]}
    C[મોશન સેન્સર] {-{-}{} B}
    D[કમ્યુનિકેશન મોડ્યુલ] {-{-}{} B}
    B {-{-}{} E[LED સ્ટ્રીટ લાઇટ]}
    B {-{-}{} F[ડિમિંગ કંટ્રોલ]}
    G[સેન્ટ્રલ કંટ્રોલ સિસ્ટમ] {-{-}{} D}
    H[પાવર સપ્લાય] {-{-}{} B}
{Highlighting}
{Shaded}
\end{verbatim}
\end{center}

\textbf{ઘટકો:}

\begin{itemize}
\tightlist
\item
  \textbf{લાઇટ સેન્સર}: આસપાસના પ્રકાશના સ્તરને શોધે છે
\item
  \textbf{મોશન સેન્સર}: પદયાત્રી/વાહનની હલનચલન શોધે છે
\item
  \textbf{માઇક્રોકંટ્રોલર}: સેન્સર ડેટાને પ્રોસેસ કરે છે અને લાઇટિંગ કંટ્રોલ કરે છે
\item
  \textbf{કમ્યુનિકેશન મોડ્યુલ}: કંટ્રોલ સેન્ટર સાથે વાયરલેસ કનેક્શન
\item
  \textbf{LED સ્ટ્રીટ લાઇટ}: ઊર્જા-કાર્યક્ષમ લાઇટિંગ
\item
  \textbf{ડિમિંગ કંટ્રોલ}: જરૂરિયાત આધારિત તેજ ગોઠવે છે
\end{itemize}

\textbf{કાર્યપ્રણાલી:}

\begin{itemize}
\tightlist
\item
  \textbf{ઓટો ON/OFF}: સાંજે લાઇટ ચાલુ, સવારે બંધ
\item
  \textbf{મોશન ડિટેક્શન}: હલનચલન શોધાતાં તેજ વધારે છે
\item
  \textbf{રિમોટ મોનિટરિંગ}: સેન્ટ્રલ સિસ્ટમ બધી લાઇટ મોનિટર કરે છે
\item
  \textbf{ઊર્જા બચત}: કોઈ પ્રવૃત્તિ ન હોય ત્યારે લાઇટ ડિમ કરે છે
\end{itemize}

\end{solutionbox}
\begin{mnemonicbox}
``LMCL'' - લાઇટ સેન્સર, મોશન સેન્સર, કંટ્રોલર, LED

\end{mnemonicbox}
\subsection*{પ્રશ્ન 3(અ) [3
ગુણ]}\label{uxaaauxab0uxab6uxaa8-3uxa85-3-uxa97uxaa3}

\textbf{ઓર્ગેનિક અને ઇનઓર્ગેનિક ઇલેક્ટ્રોનિક્સની સરખામણી કરો.}

\begin{solutionbox}


{\def\LTcaptype{none} % do not increment counter
\vspace{-5pt}
\captionof{table}{ઓર્ગેનિક vs ઇનઓર્ગેનિક ઇલેક્ટ્રોનિક્સ}
\vspace{-10pt}
\begin{longtable}[]{@{}lll@{}}
\toprule\noalign{}
પરિમાણ & ઓર્ગેનિક ઇલેક્ટ્રોનિક્સ & ઇનઓર્ગેનિક ઇલેક્ટ્રોનિક્સ \\
\midrule\noalign{}
\endhead
\bottomrule\noalign{}
\endlastfoot
\textbf{સામગ્રી} & કાર્બન-આધારિત સંયોજનો & સિલિકોન, ધાતુઓ \\
\textbf{કિંમત} & ઓછી ઉત્પાદન કિંમત & વધારે કિંમત \\
\textbf{લવચીકતા} & લવચીક, વાંકી શકાય તેવું & કઠોર માળખું \\
\textbf{પ્રોસેસિંગ} & ઓછું તાપમાન & વધારે તાપમાન \\
\end{longtable}
}

\end{solutionbox}
\begin{mnemonicbox}
``MCFP'' - મટેરિયલ, કોસ્ટ, ફ્લેક્સિબિલિટી, પ્રોસેસિંગ

\end{mnemonicbox}
\subsection*{પ્રશ્ન 3(બ) [4
ગુણ]}\label{uxaaauxab0uxab6uxaa8-3uxaac-4-uxa97uxaa3}

\textbf{OPVD પર ટૂંકનોંધ લખો.}

\begin{solutionbox}

\textbf{OPVD (ઓર્ગેનિક ફોટોવોલ્ટેઇક ડિવાઇસ)} એ ઓર્ગેનિક સેમિકંડક્ટીંગ સામગ્રીમાંથી
બનાવેલા સોલર સેલ છે.

\textbf{લાક્ષણિકતાઓ:}

\begin{itemize}
\tightlist
\item
  \textbf{લવચીક}: લવચીક સબસ્ટ્રેટ પર બનાવી શકાય છે
\item
  \textbf{ઓછી કિંમત}: સસ્તી ઉત્પાદન પ્રક્રિયા
\item
  \textbf{હળવાવજન}: પોર્ટેબલ એપ્લિકેશન માટે યોગ્ય
\item
  \textbf{અર્ધ-પારદર્શક}: વિન્ડોમાં એકીકૃત કરી શકાય છે
\end{itemize}

\textbf{એપ્લિકેશનો:}

\begin{itemize}
\tightlist
\item
  \textbf{બિલ્ડિંગ એકીકરણ}: સોલર વિન્ડો
\item
  \textbf{પોર્ટેબલ ડિવાઇસ}: લવચીક સોલર ચાર્જર
\item
  \textbf{વેરેબલ ઇલેક્ટ્રોનિક્સ}: સોલર-પાવર્ડ ગેજેટ
\end{itemize}

\end{solutionbox}
\begin{mnemonicbox}
``FLLW'' - ફ્લેક્સિબલ, લો-કોસ્ટ, લાઇટવેઇટ, વિન્ડોઝ

\end{mnemonicbox}
\subsection*{પ્રશ્ન 3(ક) [7
ગુણ]}\label{uxaaauxab0uxab6uxaa8-3uxa95-7-uxa97uxaa3}

\textbf{બાયોમેટ્રિક સિસ્ટમ અને તેમના મૂળભૂત બ્લોક ડાયાગ્રામ સમજાવો.}

\begin{solutionbox}

\textbf{બાયોમેટ્રિક સિસ્ટમ} અનન્ય જૈવિક લાક્ષણિકતાઓના આધારે વ્યક્તિઓને ઓળખે છે.

\textbf{બ્લોક ડાયાગ્રામ:}

\begin{center}
\textbf{Mermaid Diagram (Code)}
\begin{verbatim}
{Shaded}
{Highlighting}[]
graph LR
    A[બાયોમેટ્રિક સેન્સર] {-{-}{} B[સિગ્નલ પ્રોસેસિંગ]}
    B {-{-}{} C[ફીચર એક્સટ્રેક્શન]}
    C {-{-}{} D[ટેમ્પલેટ મેચિંગ]}
    D {-{-}{} E[ડિસિઝન મોડ્યુલ]}
    F[ડેટાબેઝ] {-{-}{} D}
    E {-{-}{} G[સ્વીકાર/નકાર]}
{Highlighting}
{Shaded}
\end{verbatim}
\end{center}

\textbf{ઘટકો:}

\begin{itemize}
\tightlist
\item
  \textbf{સેન્સર મોડ્યુલ}: બાયોમેટ્રિક ડેટા કેપ્ચર કરે છે (ફિંગરપ્રિન્ટ, આઇરિસ, ચહેરો)
\item
  \textbf{સિગ્નલ પ્રોસેસિંગ}: કેપ્ચર્ડ સિગ્નલને વધારે છે અને સાફ કરે છે
\item
  \textbf{ફીચર એક્સટ્રેક્શન}: અનન્ય લાક્ષણિકતાઓને ઓળખે છે
\item
  \textbf{ડેટાબેઝ મોડ્યુલ}: બાયોમેટ્રિક ટેમ્પલેટ સ્ટોર કરે છે
\item
  \textbf{મેચિંગ મોડ્યુલ}: કેપ્ચર્ડ ડેટાને સ્ટોર્ડ ટેમ્પલેટ સાથે સરખાવે છે
\item
  \textbf{ડિસિઝન મોડ્યુલ}: અંતિમ સ્વીકાર/નકાર નિર્ણય લે છે
\end{itemize}

\textbf{બાયોમેટ્રિક્સના પ્રકારો:}

\begin{itemize}
\tightlist
\item
  \textbf{ફિંગરપ્રિન્ટ}: આંગળીઓ પર રિજ પેટર્ન
\item
  \textbf{આઇરિસ}: આંખના આઇરિસ પેટર્ન
\item
  \textbf{ચહેરાની ઓળખ}: ચહેરાની વિશેષતાઓ
\item
  \textbf{અવાજ}: અવાજની પેટર્ન અને લાક્ષણિકતાઓ
\end{itemize}

\textbf{એપ્લિકેશન:}

\begin{itemize}
\tightlist
\item
  \textbf{સુરક્ષા}: એક્સેસ કંટ્રોલ સિસ્ટમ
\item
  \textbf{બેંકિંગ}: ATM ઓથેન્ટિકેશન
\item
  \textbf{મોબાઇલ}: ફોન અનલોકિંગ
\item
  \textbf{બોર્ડર કંટ્રોલ}: ઇમિગ્રેશન સિસ્ટમ
\end{itemize}

\end{solutionbox}
\begin{mnemonicbox}
``SFEMD'' - સેન્સર, ફીચર એક્સટ્રેક્શન, મેચિંગ, ડેટાબેઝ, ડિસિઝન

\end{mnemonicbox}
\subsection*{પ્રશ્ન 3(અ) અથવા [3
ગુણ]}\label{uxaaauxab0uxab6uxaa8-3uxa85-uxa85uxaa5uxab5-3-uxa97uxaa3}

\textbf{ઓર્ગેનિક ઇલેક્ટ્રોનિક્સના ફાયદા અને એપ્લિકેશનની યાદી બનાવો.}

\begin{solutionbox}

\textbf{ફાયદા:}

\begin{itemize}
\tightlist
\item
  \textbf{લવચીક}: વાંકી શકાય તેવા ઇલેક્ટ્રોનિક ઉપકરણો
\item
  \textbf{ઓછી કિંમત}: સસ્તી ઉત્પાદન
\item
  \textbf{મોટા વિસ્તાર}: મોટી સપાટીઓને ઢાંકી શકે છે
\end{itemize}

\textbf{એપ્લિકેશન:}

\begin{itemize}
\tightlist
\item
  \textbf{OLED ડિસ્પ્લે}: લવચીક સ્ક્રીન
\item
  \textbf{સોલર સેલ}: હળવાવજન પેનલ
\item
  \textbf{RFID ટેગ}: લવચીક ઓળખ
\end{itemize}

\end{solutionbox}
\begin{mnemonicbox}
``FLL-OSR'' - ફ્લેક્સિબલ, લો-કોસ્ટ, લાર્જ-એરિયા - OLED,
સોલર, RFID

\end{mnemonicbox}
\subsection*{પ્રશ્ન 3(બ) અથવા [4
ગુણ]}\label{uxaaauxab0uxab6uxaa8-3uxaac-uxa85uxaa5uxab5-4-uxa97uxaa3}

\textbf{OLED પર ટૂંકનોંધ લખો.}

\begin{solutionbox}

\textbf{OLED (ઓર્ગેનિક લાઇટ એમિટિંગ ડાયોડ)} એ ડિસ્પ્લે ટેકનોલોજી છે જે ઓર્ગેનિક
સંયોજનોનો ઉપયોગ કરે છે જે ઇલેક્ટ્રિક કરંટ લાગુ કરવામાં આવે ત્યારે પ્રકાશ ઉત્સર્જન કરે છે.

\textbf{ફાયદા:}

\begin{itemize}
\tightlist
\item
  \textbf{સ્વ-પ્રકાશિત}: બેકલાઇટની જરૂર નથી
\item
  \textbf{હાઇ કોન્ટ્રાસ્ટ}: સાચા કાળા રંગો
\item
  \textbf{લવચીક}: વાંકી અને વળાંકવાળું બનાવી શકાય છે
\item
  \textbf{ઊર્જા કાર્યક્ષમ}: ઓછો પાવર વપરાશ
\end{itemize}

\textbf{એપ્લિકેશન:}

\begin{itemize}
\tightlist
\item
  \textbf{સ્માર્ટફોન}: OLED સ્ક્રીન
\item
  \textbf{ટીવી}: અલ્ટ્રા-થિન ડિસ્પ્લે
\item
  \textbf{વેરેબલ}: સ્માર્ટવોચ ડિસ્પ્લે
\end{itemize}

\end{solutionbox}
\begin{mnemonicbox}
``SHFE'' - સ્વ-પ્રકાશિત, હાઇ કોન્ટ્રાસ્ટ, ફ્લેક્સિબલ,
કાર્યક્ષમ

\end{mnemonicbox}
\subsection*{પ્રશ્ન 3(ક) અથવા [7
ગુણ]}\label{uxaaauxab0uxab6uxaa8-3uxa95-uxa85uxaa5uxab5-7-uxa97uxaa3}

\textbf{AR/VR કોર ટેકનોલોજી સમજાવો અને તેની એપ્લિકેશનોની ચર્ચા કરો.}

\begin{solutionbox}

\textbf{AR (ઓગમેન્ટેડ રિયાલિટી)} વાસ્તવિક વિશ્વ પર ડિજિટલ માહિતીને ઓવરલે કરે છે,
જ્યારે \textbf{VR (વર્ચ્યુઅલ રિયાલિટી)} સંપૂર્ણપણે ઇમર્સિવ ડિજિટલ વાતાવરણ બનાવે છે.

\textbf{કોર ટેકનોલોજી:}

\begin{itemize}
\tightlist
\item
  \textbf{ડિસ્પ્લે સિસ્ટમ}: હેડ-માઉન્ટેડ ડિસ્પ્લે, સ્ક્રીન
\item
  \textbf{ટ્રેકિંગ સિસ્ટમ}: મોશન સેન્સર, કેમેરા
\item
  \textbf{પ્રોસેસિંગ યુનિટ}: GPU, સ્પેશિયલાઇઝ્ડ ચિપ્સ
\item
  \textbf{ઇનપુટ મેથડ}: કંટ્રોલર, જેસ્ચર રેકગ્નિશન
\end{itemize}

\textbf{AR એપ્લિકેશન:}

\begin{itemize}
\tightlist
\item
  \textbf{ગેમિંગ}: પોકેમોન ગો, મોબાઇલ AR ગેમ્સ
\item
  \textbf{શિક્ષણ}: ઇન્ટરેક્ટિવ લર્નિંગ અનુભવો
\item
  \textbf{નેવિગેશન}: વાસ્તવિક રસ્તાઓ પર GPS ઓવરલે
\item
  \textbf{શોપિંગ}: વર્ચ્યુઅલ ટ્રાય-ઓન અનુભવો
\end{itemize}

\textbf{VR એપ્લિકેશન:}

\begin{itemize}
\tightlist
\item
  \textbf{મનોરંજન}: ઇમર્સિવ ગેમિંગ, મૂવીઝ
\item
  \textbf{ટ્રેનિંગ}: ફ્લાઇટ સિમ્યુલેટર, મેડિકલ ટ્રેનિંગ
\item
  \textbf{આર્કિટેક્ચર}: વર્ચ્યુઅલ બિલ્ડિંગ વોકથ્રુ
\item
  \textbf{થેરાપી}: ફોબિયા, PTSD ની સારવાર
\end{itemize}


{\def\LTcaptype{none} % do not increment counter
\vspace{-5pt}
\captionof{table}{AR vs VR સરખામણી}
\vspace{-10pt}
\begin{longtable}[]{@{}lll@{}}
\toprule\noalign{}
પાસું & AR & VR \\
\midrule\noalign{}
\endhead
\bottomrule\noalign{}
\endlastfoot
\textbf{વાસ્તવિકતા} & વાસ્તવિક વિશ્વ સાથે મિશ્રિત & સંપૂર્ણપણે વર્ચ્યુઅલ \\
\textbf{સાધનો} & સ્માર્ટફોન, AR ચશ્મા & VR હેડસેટ, કંટ્રોલર \\
\textbf{ઇમર્શન} & આંશિક & સંપૂર્ણ \\
\textbf{ગતિશીલતા} & મોબાઇલ ફ્રેન્ડલી & સ્થિર સેટઅપ \\
\end{longtable}
}

\end{solutionbox}
\begin{mnemonicbox}
``DTPI-GENT'' - ડિસ્પ્લે, ટ્રેકિંગ, પ્રોસેસિંગ, ઇનપુટ - ગેમિંગ,
એજ્યુકેશન, નેવિગેશન, ટ્રેનિંગ

\end{mnemonicbox}
\subsection*{પ્રશ્ન 4(અ) [3
ગુણ]}\label{uxaaauxab0uxab6uxaa8-4uxa85-3-uxa97uxaa3}

\textbf{હોમ સોલર રૂફટોપ સિસ્ટમનો બ્લોક ડાયાગ્રામ દોરો.}

\begin{solutionbox}

\textbf{બ્લોક ડાયાગ્રામ:}

\begin{verbatim}
┌─────────────┐    ┌─────────────┐    ┌─────────────┐
│Solar Panels │────│   Inverter  │────│AC Load Panel│
└─────────────┘    └─────────────┘    └─────────────┘
                          │                   │
                   ┌─────────────┐    ┌─────────────┐
                   │   Battery   │    │Utility Grid │
                   │   Storage   │    │ Connection  │
                   └─────────────┘    └─────────────┘
\end{verbatim}

\textbf{ઘટકો:}

\begin{itemize}
\tightlist
\item
  \textbf{સોલર પેનલ્સ}: સૂર્યપ્રકાશને DC વીજળીમાં રૂપાંતરિત કરે છે
\item
  \textbf{ઇન્વર્ટર}: DC ને AC પાવરમાં રૂપાંતરિત કરે છે
\item
  \textbf{બેટરી સ્ટોરેજ}: વધારાની ઊર્જા સંગ્રહ કરે છે
\end{itemize}

\end{solutionbox}
\begin{mnemonicbox}
``SIB'' - સોલર પેનલ્સ, ઇન્વર્ટર, બેટરી

\end{mnemonicbox}
\subsection*{પ્રશ્ન 4(બ) [4
ગુણ]}\label{uxaaauxab0uxab6uxaa8-4uxaac-4-uxa97uxaa3}

\textbf{OFET નો કાર્યસિદ્ધાંત સમજાવો.}

\begin{solutionbox}

\textbf{OFET (ઓર્ગેનિક ફીલ્ડ ઇફેક્ટ ટ્રાન્ઝિસ્ટર)} કરંટ ફ્લોને કંટ્રોલ કરવા માટે
ઓર્ગેનિક સેમિકંડક્ટરનો ઉપયોગ કરે છે.

\textbf{કાર્યસિદ્ધાંત:}

\begin{itemize}
\tightlist
\item
  \textbf{ગેટ વોલ્ટેજ}: લાગુ વોલ્ટેજ ઇલેક્ટ્રિક ફીલ્ડ બનાવે છે
\item
  \textbf{ચેનલ ફોર્મેશન}: ઇલેક્ટ્રિક ફીલ્ડ કંડક્ટિવિટી મોડ્યુલેટ કરે છે
\item
  \textbf{કરંટ કંટ્રોલ}: સોર્સ-ડ્રેન કરંટ ગેટ દ્વારા કંટ્રોલ થાય છે
\item
  \textbf{સ્વિચિંગ}: ડિજિટલ એપ્લિકેશન માટે ON/OFF સ્ટેટ
\end{itemize}

\textbf{માળખું:}

\begin{itemize}
\tightlist
\item
  \textbf{સોર્સ/ડ્રેન}: કરંટ ઇન્જેક્શન પોઇન્ટ
\item
  \textbf{ગેટ}: કંટ્રોલ ઇલેક્ટ્રોડ
\item
  \textbf{ઓર્ગેનિક લેયર}: એક્ટિવ સેમિકंડક્ટર મટેરિયલ
\end{itemize}

\end{solutionbox}
\begin{mnemonicbox}
``GCCS'' - ગેટ વોલ્ટેજ, ચેનલ, કરંટ, સ્વિચિંગ

\end{mnemonicbox}
\subsection*{પ્રશ્ન 4(ક) [7
ગુણ]}\label{uxaaauxab0uxab6uxaa8-4uxa95-7-uxa97uxaa3}

\textbf{વિવિધ મશીન લર્નિંગ ટૂલ્સની યાદી બનાવો. કોઈપણ બેની ટૂંકમાં ચર્ચા કરો.}

\begin{solutionbox}

\textbf{મશીન લર્નિંગ ટૂલ્સ:}

\begin{itemize}
\tightlist
\item
  \textbf{TensorFlow}: ગૂગલનું ML ફ્રેમવર્ક
\item
  \textbf{PyTorch}: ફેસબુકની ડીપ લર્નિંગ લાઇબ્રેરી
\item
  \textbf{Scikit-learn}: પાયથોન ML લાઇબ્રેરી
\item
  \textbf{Keras}: હાઇ-લેવલ ન્યુરલ નેટવર્ક API
\item
  \textbf{Machine Learning for Kids}: શૈક્ષણિક પ્લેટફોર્મ
\item
  \textbf{Scratch}: ML માટે વિઝ્યુઅલ પ્રોગ્રામિંગ
\end{itemize}

\textbf{TensorFlow:}

\begin{itemize}
\tightlist
\item
  \textbf{હેતુ}: ડીપ લર્નિંગ અને ન્યુરલ નેટવર્ક
\item
  \textbf{વિશેષતાઓ}: મોટા પાયે ML, પ્રોડક્શન ડિપ્લોયમેન્ટ
\item
  \textbf{એપ્લિકેશન}: ઇમેજ રેકગ્નિશન, NLP, રેકમેન્ડેશન સિસ્ટમ
\item
  \textbf{ફાયદા}: સ્કેલેબલ, વ્યાપક ડોક્યુમેન્ટેશન
\end{itemize}

\textbf{Scikit-learn:}

\begin{itemize}
\tightlist
\item
  \textbf{હેતુ}: સામાન્ય મશીન લર્નિંગ અલગોરિધમ
\item
  \textbf{વિશેષતાઓ}: ક્લાસિફિકેશન, રિગ્રેશન, ક્લસ્ટરિંગ
\item
  \textbf{એપ્લિકેશન}: ડેટા એનાલિસિસ, પ્રિડિક્ટિવ મોડેલિંગ
\item
  \textbf{ફાયદા}: ઉપયોગમાં સરળ, સારી રીતે ડોક્યુમેન્ટેડ
\end{itemize}


{\def\LTcaptype{none} % do not increment counter
\vspace{-5pt}
\captionof{table}{ML ટૂલ્સ સરખામણી}
\vspace{-10pt}
\begin{longtable}[]{@{}llll@{}}
\toprule\noalign{}
ટૂલ & પ્રકાર & સર્વોત્તમ & મુશ્કેલી \\
\midrule\noalign{}
\endhead
\bottomrule\noalign{}
\endlastfoot
\textbf{TensorFlow} & ડીપ લર્નિંગ & જટિલ મોડેલ & એડવાન્સ \\
\textbf{Scikit-learn} & જનરલ ML & બિગિનર્સ & સરળ \\
\end{longtable}
}

\end{solutionbox}
\begin{mnemonicbox}
``TPSKMS-TF.SL'' - TensorFlow, PyTorch, Scikit,
Keras, ML4Kids, Scratch - TensorFlow, Scikit-learn

\end{mnemonicbox}
\subsection*{પ્રશ્ન 4(અ) અથવા [3
ગુણ]}\label{uxaaauxab0uxab6uxaa8-4uxa85-uxa85uxaa5uxab5-3-uxa97uxaa3}

\textbf{રિન્યુએબલ એનર્જીમાં ઇમર્જિંગ ટ્રેન્ડ્સને સંક્ષિપ્તમાં સમજાવો.}

\begin{solutionbox}

\textbf{ઉભરતા વલણો:}

\begin{itemize}
\tightlist
\item
  \textbf{ફ્લોટિંગ સોલર}: પાણીના શરીર પર સોલર પેનલ
\item
  \textbf{પેરોવ્સકાઇટ સેલ}: આગામી પેઢીની સોલર ટેકનોલોજી
\item
  \textbf{ગ્રીન હાઇડ્રોજન}: રિન્યુએબલ સ્ત્રોતોમાંથી સ્વચ્છ ઇંધન
\end{itemize}

\textbf{ફાયદા:}

\begin{itemize}
\tightlist
\item
  \textbf{વધારે કાર્યક્ષમતા}: બહેતર ઊર્જા રૂપાંતરણ
\item
  \textbf{કિંમત ઘટાડો}: સસ્તી રિન્યુએબલ એનર્જી
\end{itemize}

\end{solutionbox}
\begin{mnemonicbox}
``FPG'' - ફ્લોટિંગ સોલર, પેરોવ્સકાઇટ, ગ્રીન હાઇડ્રોજન

\end{mnemonicbox}
\subsection*{પ્રશ્ન 4(બ) અથવા [4
ગુણ]}\label{uxaaauxab0uxab6uxaa8-4uxaac-uxa85uxaa5uxab5-4-uxa97uxaa3}

\textbf{સંપૂર્ણ સ્વરૂપો આપો: AR, OLED, OPVD, OFET}

\begin{solutionbox}


{\def\LTcaptype{none} % do not increment counter
\vspace{-5pt}
\captionof{table}{ટેકનોલોજી સંપૂર્ણ સ્વરૂપો}
\vspace{-10pt}
\begin{longtable}[]{@{}lll@{}}
\toprule\noalign{}
સંક્ષેપ & સંપૂર્ણ સ્વરૂપ & ટેકનોલોજી વિસ્તાર \\
\midrule\noalign{}
\endhead
\bottomrule\noalign{}
\endlastfoot
\textbf{AR} & ઓગમેન્ટેડ રિયાલિટી & મિક્સ્ડ રિયાલિટી \\
\textbf{OLED} & ઓર્ગેનિક લાઇટ એમિટિંગ ડાયોડ & ડિસ્પ્લે ટેકનોલોજી \\
\textbf{OPVD} & ઓર્ગેનિક ફોટોવોલ્ટેઇક ડિવાઇસ & સોલર સેલ \\
\textbf{OFET} & ઓર્ગેનિક ફીલ્ડ ઇફેક્ટ ટ્રાન્ઝિસ્ટર & ઇલેક્ટ્રોનિક્સ \\
\end{longtable}
}

\end{solutionbox}
\begin{mnemonicbox}
``AOOO'' - AR, OLED, OPVD, OFET

\end{mnemonicbox}
\subsection*{પ્રશ્ન 4(ક) અથવા [7
ગુણ]}\label{uxaaauxab0uxab6uxaa8-4uxa95-uxa85uxaa5uxab5-7-uxa97uxaa3}

\textbf{રાસ્પબેરી પાઈનો બ્લોક ડાયાગ્રામ સમજાવો.}

\begin{solutionbox}

\textbf{બ્લોક ડાયાગ્રામ:}

\begin{center}
\textbf{Mermaid Diagram (Code)}
\begin{verbatim}
{Shaded}
{Highlighting}[]
graph TD
    A[ARM પ્રોસેસર] {-{-}{} B[RAM મેમરી]}
    A {-{-}{} C[GPIO પિન્સ]}
    A {-{-}{} D[USB પોર્ટ્સ]}
    A {-{-}{} E[HDMI આઉટપુટ]}
    A {-{-}{} F[ઇથરનેટ પોર્ટ]}
    G[માઇક્રો SD કાર્ડ] {-{-}{} A}
    H[પાવર સપ્લાય] {-{-}{} A}
    A {-{-}{} I[ઓડિયો/વીડિયો]}
{Highlighting}
{Shaded}
\end{verbatim}
\end{center}

\textbf{ઘટકો:}

\begin{itemize}
\tightlist
\item
  \textbf{ARM પ્રોસેસર}: સેન્ટ્રલ પ્રોસેસિંગ યુનિટ (ક્વાડ-કોર)
\item
  \textbf{RAM મેમરી}: સિસ્ટમ મેમરી (1GB-8GB)
\item
  \textbf{GPIO પિન્સ}: સેન્સર/ઉપકરણોને ઇન્ટરફેસ કરવા માટે 40 પિન્સ
\item
  \textbf{USB પોર્ટ્સ}: પેરિફેરલ્સ કનેક્ટ કરે છે
\item
  \textbf{HDMI આઉટપુટ}: વીડિયો ડિસ્પ્લે કનેક્શન
\item
  \textbf{ઇથરનેટ પોર્ટ}: નેટવર્ક કનેક્ટિવિટી
\item
  \textbf{માઇક્રો SD કાર્ડ}: OS અને ડેટા માટે સ્ટોરેજ
\item
  \textbf{પાવર સપ્લાય}: 5V માઇક્રો-USB અથવા USB-C
\end{itemize}

\textbf{વિશેષતાઓ:}

\begin{itemize}
\tightlist
\item
  \textbf{ઓપરેટિંગ સિસ્ટમ}: રાસ્પબેરી પાઈ OS (લિનક્સ-આધારિત)
\item
  \textbf{પ્રોગ્રામિંગ}: પાયથોન, C++, Scratch સપોર્ટ
\item
  \textbf{કનેક્ટિવિટી}: બિલ્ટ-ઇન Wi-Fi, બ્લુટૂથ
\item
  \textbf{વિસ્તરણક્ષમતા}: કેમેરા, ડિસ્પ્લે કનેક્ટર
\end{itemize}

\textbf{એપ્લિકેશન:}

\begin{itemize}
\tightlist
\item
  \textbf{IoT પ્રોજેક્ટ્સ}: હોમ ઓટોમેશન
\item
  \textbf{શિક્ષણ}: પ્રોગ્રામિંગ શીખવું
\item
  \textbf{રોબોટિક્સ}: રોબોટ કંટ્રોલ સિસ્ટમ
\item
  \textbf{મીડિયા સેન્ટર}: હોમ એન્ટરટેઇનમેન્ટ
\end{itemize}

\end{solutionbox}
\begin{mnemonicbox}
``ARGC-EPMS'' - ARM, RAM, GPIO, કનેક્ટિવિટી - ઇથરનેટ,
પાવર, માઇક્રોSD, સ્ટોરેજ

\end{mnemonicbox}
\subsection*{પ્રશ્ન 5(અ) [3
ગુણ]}\label{uxaaauxab0uxab6uxaa8-5uxa85-3-uxa97uxaa3}

\textbf{રાસ્પબેરી પાઈ સાથે LED ઇન્ટરફેસ કરો.}

\begin{solutionbox}

\textbf{સર્કિટ કનેક્શન:}

\begin{verbatim}
રાસ્પબેરી પાઈ          LED સર્કિટ
GPIO Pin 18 ────── 220Ω ────── LED ────── GND
                   રેઝિસ્ટર   એનોડ     કેથોડ
\end{verbatim}

\textbf{પાયથોન કોડ:}

\begin{verbatim}
import RPi.GPIO as GPIO
import time

GPIO.setmode(GPIO.BCM)
GPIO.setup(18, GPIO.OUT)

while True:
    GPIO.output(18, GPIO.HIGH)  \# LED ON
    time.sleep(1)
    GPIO.output(18, GPIO.LOW)   \# LED OFF
    time.sleep(1)
\end{verbatim}

\end{solutionbox}
\begin{mnemonicbox}
``GPIO-RC'' - GPIO પિન, રેઝિસ્ટર, કોડ

\end{mnemonicbox}
\subsection*{પ્રશ્ન 5(બ) [4
ગુણ]}\label{uxaaauxab0uxab6uxaa8-5uxaac-4-uxa97uxaa3}

\textbf{મશીન લર્નિંગ માટે Pandas પાયથોન લાઇબ્રેરી સમજાવો.}

\begin{solutionbox}

\textbf{Pandas} એ ડેટા મેનિપ્યુલેશન અને એનાલિસિસ માટેની પાયથોન લાઇબ્રેરી છે, જે ML
ડેટા પ્રીપ્રોસેસિંગ માટે આવશ્યક છે.

\textbf{મુખ્ય વિશેષતાઓ:}

\begin{itemize}
\tightlist
\item
  \textbf{DataFrame}: ટેબ્યુલર ડેટા સ્ટ્રક્ચર
\item
  \textbf{ડેટા ક્લીનિંગ}: ગુમ થયેલ વેલ્યુ, ડુપ્લિકેટ હેન્ડલ કરે છે
\item
  \textbf{ડેટા ઇમ્પોર્ટ}: CSV, Excel, JSON ફાઇલો વાંચે છે
\item
  \textbf{ડેટા એનાલિસિસ}: આંકડાકીય ઓપરેશન્સ, ગ્રુપિંગ
\end{itemize}

\textbf{ML એપ્લિકેશન:}

\begin{itemize}
\tightlist
\item
  \textbf{ડેટા પ્રીપ્રોસેસિંગ}: ડેટાસેટ સાફ અને તૈયાર કરે છે
\item
  \textbf{ફીચર એન્જિનિયરિંગ}: ડેટામાંથી નવી વિશેષતાઓ બનાવે છે
\item
  \textbf{ડેટા એક્સપ્લોરેશન}: ડેટા પેટર્ન સમજે છે
\item
  \textbf{ડેટા ટ્રાન્સફોર્મેશન}: ડેટાને નોર્મલાઇઝ, સ્કેલ કરે છે
\end{itemize}

\textbf{સામાન્ય ફંક્શન્સ:}

\begin{verbatim}
import pandas as pd
df = pd.read\_csv({data.csv})    \# ડેટા લોડ કરો
df.info()                       \# ડેટા માહિતી
df.describe()                   \# આંકડાકીય માહિતી
\end{verbatim}

\end{solutionbox}
\begin{mnemonicbox}
``DCIF'' - DataFrame, ક્લીનિંગ, ઇમ્પોર્ટ, ફંક્શન્સ

\end{mnemonicbox}
\subsection*{પ્રશ્ન 5(ક) [7
ગુણ]}\label{uxaaauxab0uxab6uxaa8-5uxa95-7-uxa97uxaa3}

\textbf{મશીન લર્નિંગ તકનીકોના પ્રકારો સમજાવો: સુપરવાઇઝ્ડ, અનસુપરવાઇઝ્ડ અને
રિઇન્ફોર્સમેન્ટ લર્નિંગ.}

\begin{solutionbox}


{\def\LTcaptype{none} % do not increment counter
\vspace{-5pt}
\captionof{table}{મશીન લર્નિંગ પ્રકારો}
\vspace{-10pt}
\begin{longtable}[]{@{}
  >{\raggedright\arraybackslash}p{(\linewidth - 6\tabcolsep) * \real{0.1875}}
  >{\raggedright\arraybackslash}p{(\linewidth - 6\tabcolsep) * \real{0.3438}}
  >{\raggedright\arraybackslash}p{(\linewidth - 6\tabcolsep) * \real{0.1875}}
  >{\raggedright\arraybackslash}p{(\linewidth - 6\tabcolsep) * \real{0.2812}}@{}}
\toprule\noalign{}
\begin{minipage}[b]{\linewidth}\raggedright
પ્રકાર
\end{minipage} & \begin{minipage}[b]{\linewidth}\raggedright
જરૂરી ડેટા
\end{minipage} & \begin{minipage}[b]{\linewidth}\raggedright
ધ્યેય
\end{minipage} & \begin{minipage}[b]{\linewidth}\raggedright
ઉદાહરણો
\end{minipage} \\
\midrule\noalign{}
\endhead
\bottomrule\noalign{}
\endlastfoot
\textbf{સુપરવાઇઝ્ડ} & લેબલ્ડ ડેટા & પરિણામોની આગાહી & ક્લાસિફિકેશન, રિગ્રેશન \\
\textbf{અનસુપરવાઇઝ્ડ} & અનલેબલ્ડ ડેટા & પેટર્ન શોધવું & ક્લસ્ટરિંગ, ડાઇમેન્શનલિટી
રિડક્શન \\
\textbf{રિઇન્ફોર્સમેન્ટ} & રિવાર્ડ સિગ્નલ્સ & શ્રેષ્ઠ ક્રિયાઓ શીખવી & ગેમ પ્લેઇંગ,
રોબોટિક્સ \\
\end{longtable}
}

\textbf{સુપરવાઇઝ્ડ લર્નિંગ:}

\begin{itemize}
\tightlist
\item
  \textbf{વ્યાખ્યા}: ઇનપુટ-આઉટપુટ જોડીઓમાંથી શીખે છે
\item
  \textbf{પ્રક્રિયા}: જાણીતા જવાબો સાથે ટ્રેનિંગ
\item
  \textbf{એપ્લિકેશન}: ઇમેઇલ સ્પામ ડિટેક્શન, ઇમેજ રેકગ્નિશન
\item
  \textbf{અલગોરિધમ}: લિનિયર રિગ્રેશન, ડિસિઝન ટ્રી, ન્યુરલ નેટવર્ક
\end{itemize}

\textbf{અનસુપરવાઇઝ્ડ લર્નિંગ:}

\begin{itemize}
\tightlist
\item
  \textbf{વ્યાખ્યા}: ડેટામાં છુપાયેલા પેટર્ન શોધે છે
\item
  \textbf{પ્રક્રિયા}: કોઈ ટાર્ગેટ વેરિએબલ પ્રદાન કરવામાં આવતું નથી
\item
  \textbf{એપ્લિકેશન}: કસ્ટમર સેગમેન્ટેશન, એનોમલી ડિટેક્શન
\item
  \textbf{અલગોરિધમ}: K-means ક્લસ્ટરિંગ, PCA, હાઇરાર્કિકલ ક્લસ્ટરિંગ
\end{itemize}

\textbf{રિઇન્ફોર્સમેન્ટ લર્નિંગ:}

\begin{itemize}
\tightlist
\item
  \textbf{વ્યાખ્યા}: ટ્રાયલ અને એરર દ્વારા શીખે છે
\item
  \textbf{પ્રક્રિયા}: એજન્ટ વાતાવરણ સાથે ઇન્ટરેક્ટ કરે છે
\item
  \textbf{એપ્લિકેશન}: ગેમ AI, ઓટોનોમસ વ્હીકલ, રોબોટિક્સ
\item
  \textbf{ઘટકો}: એજન્ટ, વાતાવરણ, રિવાર્ડ, ક્રિયાઓ
\end{itemize}

\textbf{આકૃતિ: ML લર્નિંગ પ્રક્રિયા}

\begin{center}
\textbf{Mermaid Diagram (Code)}
\begin{verbatim}
{Shaded}
{Highlighting}[]
graph LR
    A[ડેટા] {-{-}{} B\{લર્નિંગ પ્રકાર\}}
    B {-{-}{} C[સુપરવાઇઝ્ડ]}
    B {-{-}{} D[અનસુપરવાઇઝ્ડ]}
    B {-{-}{} E[રિઇન્ફોર્સમેન્ટ]}
    C {-{-}{} F[પ્રિડિક્શન મોડેલ]}
    D {-{-}{} G[પેટર્ન ડિસ્કવરી]}
    E {-{-}{} H[ડિસિઝન પોલિસી]}
{Highlighting}
{Shaded}
\end{verbatim}
\end{center}

\end{solutionbox}
\begin{mnemonicbox}
``SUR-PLR-CPD'' - સુપરવાઇઝ્ડ, અનસુપરવાઇઝ્ડ, રિઇન્ફોર્સમેન્ટ
- પ્રિડિક્શન, લર્નિંગ, રિવાર્ડ - ક્લાસિફિકેશન, પેટર્ન, ડિસિઝન

\end{mnemonicbox}
\subsection*{પ્રશ્ન 5(અ) અથવા [3
ગુણ]}\label{uxaaauxab0uxab6uxaa8-5uxa85-uxa85uxaa5uxab5-3-uxa97uxaa3}

\textbf{મશીન લર્નિંગ માટે NumPy પાયથોન લાઇબ્રેરી સમજાવો.}

\begin{solutionbox}

\textbf{NumPy} એ પાયથોનમાં ન્યુમેરિકલ કમ્પ્યુટિંગ માટેની મૂળભૂત લાઇબ્રેરી છે, જે ML
ઓપરેશન્સ માટે આવશ્યક છે.

\textbf{મુખ્ય વિશેષતાઓ:}

\begin{itemize}
\tightlist
\item
  \textbf{એરે}: મલ્ટિ-ડાઇમેન્શનલ એરે ઓબ્જેક્ટ
\item
  \textbf{મેથેમેટિકલ ફંક્શન્સ}: લિનિયર આલ્જેબ્રા ઓપરેશન્સ
\item
  \textbf{બ્રોડકાસ્ટિંગ}: અલગ સાઇઝના એરે પર ઓપરેશન્સ
\end{itemize}

\textbf{ML એપ્લિકેશન:}

\begin{itemize}
\tightlist
\item
  \textbf{ડેટા સ્ટોરેજ}: કાર્યક્ષમ ન્યુમેરિકલ ડેટા સ્ટોરેજ
\item
  \textbf{મેટ્રિક્સ ઓપરેશન્સ}: ન્યુરલ નેટવર્ક કમ્પ્યુટેશન્સ
\item
  \textbf{મેથેમેટિકલ કમ્પ્યુટેશન્સ}: આંકડાકીય ઓપરેશન્સ
\end{itemize}

\end{solutionbox}
\begin{mnemonicbox}
``AMB'' - એરે, મેથેમેટિકલ ફંક્શન્સ, બ્રોડકાસ્ટિંગ

\end{mnemonicbox}
\subsection*{પ્રશ્ન 5(બ) અથવા [4
ગુણ]}\label{uxaaauxab0uxab6uxaa8-5uxaac-uxa85uxaa5uxab5-4-uxa97uxaa3}

\textbf{Raspberry Pi Imager નો ઉપયોગ કરીને SD કાર્ડ પર Raspberry Pi OS
ઇન્સ્ટોલેશનના સ્ટેપ્સ લખો.}

\begin{solutionbox}

\textbf{ઇન્સ્ટોલેશન સ્ટેપ્સ:}

\begin{enumerate}
\tightlist
\item
  \textbf{ડાઉનલોડ}: ઓફિશિયલ વેબસાઇટથી Raspberry Pi Imager ઇન્સ્ટોલ કરો
\item
  \textbf{SD કાર્ડ ઇન્સર્ટ}: કમ્પ્યુટરમાં SD કાર્ડ (16GB+) કનેક્ટ કરો
\item
  \textbf{OS સિલેક્ટ}: યાદીમાંથી Raspberry Pi OS પસંદ કરો
\item
  \textbf{સ્ટોરેજ સિલેક્ટ}: ટાર્ગેટ તરીકે SD કાર્ડ પસંદ કરો
\item
  \textbf{રાઇટ}: OS ને SD કાર્ડમાં ફ્લેશ કરવા માટે ``Write'' ક્લિક કરો
\item
  \textbf{ઇજેક્ટ}: પૂર્ણ થયા પછી SD કાર્ડને સુરક્ષિત રીતે કાઢો
\end{enumerate}

\textbf{પૂર્વ-ગોઠવણી વિકલ્પો:}

\begin{itemize}
\tightlist
\item
  \textbf{SSH એનેબલ}: રિમોટ એક્સેસ માટે
\item
  \textbf{યુઝરનેમ/પાસવર્ડ સેટ}: સુરક્ષા ક્રેડેન્શિયલ્સ
\item
  \textbf{Wi-Fi કોન્ફિગર}: નેટવર્ક સેટિંગ્સ
\end{itemize}

\end{solutionbox}
\begin{mnemonicbox}
``DISWS-ESP'' - ડાઉનલોડ, ઇન્સર્ટ, સિલેક્ટ OS, રાઇટ,
સ્ટોરેજ - SSH એનેબલ, ક્રેડેન્શિયલ્સ સેટ, પૂર્વ-કોન્ફિગર

\end{mnemonicbox}
\subsection*{પ્રશ્ન 5(ક) અથવા [7
ગુણ]}\label{uxaaauxab0uxab6uxaa8-5uxa95-uxa85uxaa5uxab5-7-uxa97uxaa3}

\textbf{Raspberry Pi સાથે Temperature અને humidity સેન્સર ઇન્ટરફેસ કરો અને તેના
માટે Python પ્રોગ્રામ લખો.}

\begin{solutionbox}

\textbf{સર્કિટ કનેક્શન:}

\begin{verbatim}
DHT22 સેન્સર          રાસ્પબેરી પાઈ
VCC  ────────────────  3.3V (Pin 1)
DATA ────────────────  GPIO 4 (Pin 7)
GND  ────────────────  GND (Pin 6)
\end{verbatim}

\textbf{પાયથોન પ્રોગ્રામ:}

\begin{verbatim}
import Adafruit\_DHT
import time

\# સેન્સર પ્રકાર અને GPIO પિન
sensor = Adafruit\_DHT.DHT22
pin = 4

while True:
    try:
        \# સેન્સર ડેટા વાંચો
        humidity, temperature = Adafruit\_DHT.read\_retry(sensor, pin)
        
        if humidity is not None and temperature is not None:
            print(f{તાપમાન: }\{temperature:.1f\}^{})
            print(f{ભેજ: }\{humidity:.1f\}\%{})
        else:
            print({સેન્સર ડેટા વાંચવામાં નિષ્ફળ})
            
        time.sleep(2)  \# 2 સેકંડ રાહ જુઓ
        
    except KeyboardInterrupt:
        print("{n}પ્રોગ્રામ બંધ")
        break
\end{verbatim}

\textbf{જરૂરી લાઇબ્રેરી:}

\begin{verbatim}
pip install Adafruit\_DHT
\end{verbatim}

\textbf{ઉપયોગમાં લેવાયેલા ઘટકો:}

\begin{itemize}
\tightlist
\item
  \textbf{DHT22}: તાપમાન અને ભેજ સેન્સર
\item
  \textbf{રાસ્પબેરી પાઈ}: પ્રોસેસિંગ યુનિટ
\item
  \textbf{પાયથોન}: પ્રોગ્રામિંગ લેંગ્વેજ
\item
  \textbf{Adafruit લાઇબ્રેરી}: સેન્સર ઇન્ટરફેસ લાઇબ્રેરી
\end{itemize}

\textbf{વિશેષતાઓ:}

\begin{itemize}
\tightlist
\item
  \textbf{રીઅલ-ટાઇમ રીડિંગ}: સતત મોનિટરિંગ
\item
  \textbf{એરર હેન્ડલિંગ}: સેન્સર રીડ ફેઇલ્યુર હેન્ડલ કરે છે
\item
  \textbf{ડેટા ડિસ્પ્લે}: તાપમાન અને ભેજના મૂલ્યો બતાવે છે
\item
  \textbf{યુઝર કંટ્રોલ}: પ્રોગ્રામ બંધ કરવા માટે કીબોર્ડ ઇન્ટરપ્ટ
\end{itemize}

\textbf{એપ્લિકેશન:}

\begin{itemize}
\tightlist
\item
  \textbf{વેધર સ્ટેશન}: સ્થાનિક હવામાન મોનિટરિંગ
\item
  \textbf{હોમ ઓટોમેશન}: ક્લાઇમેટ કંટ્રોલ સિસ્ટમ
\item
  \textbf{કૃષિ}: ગ્રીનહાઉસ મોનિટરિંગ
\item
  \textbf{ઇન્ડસ્ટ્રિયલ}: પર્યાવરણીય મોનિટરિંગ
\end{itemize}

\end{solutionbox}
\begin{mnemonicbox}
``DHT-RPL'' - DHT સેન્સર, રાસ્પબેરી પાઈ, પાયથોન, લાઇબ્રેરી

\end{mnemonicbox}

\end{document}
