\documentclass[10pt,a4paper]{article}

% content/resources/templates/preamble.tex
\usepackage[margin=0.6in]{geometry}
\author{Milav Dabgar}
\usepackage{amsmath,amssymb,amsthm}
\usepackage{booktabs}
\usepackage{multirow}
\usepackage{xcolor}
\usepackage{tcolorbox}
\tcbuselibrary{breakable,skins}
\usepackage[colorlinks=true,linkcolor=blue]{hyperref}
\usepackage{titlesec}
\usepackage{enumitem}
\usepackage{tikz}
\usepackage{pgfplots}
\usepackage{circuitikz}
\usepackage[version=4]{mhchem}
\usepackage{longtable}
\usepackage{array}
\usepackage{float}
\usepackage{caption}
\usepackage{listings}

\lstset{
  basicstyle=\small\ttfamily,
  breaklines=true,
  breakatwhitespace=false,
  postbreak=\mbox{\textcolor{red}{$\hookrightarrow$}\space},
  float=false,
  numbers=left,
  numberstyle=\tiny\color{gray},
  numbersep=10pt,
  xleftmargin=2em,
  keywordstyle=\color{blue},
  commentstyle=\color{green!60!black},
  stringstyle=\color{purple},
  backgroundcolor=\color{gray!5},
  showstringspaces=false,
  tabsize=2,
  captionpos=b,
  keepspaces=true,
  columns=flexible
}

\pgfplotsset{compat=1.18}
\usetikzlibrary{shapes,arrows,positioning,calc,patterns,decorations.pathmorphing,decorations.markings,arrows.meta}

% Color scheme
\definecolor{headcolor}{RGB}{0,102,204}
\definecolor{keycolor}{RGB}{220,20,60}
\definecolor{solutioncolor}{RGB}{34,139,34}
\definecolor{mnemoniccolor}{RGB}{148,0,211}
\definecolor{codecolor}{RGB}{0,0,100}

% Spacing
\setlength{\parskip}{3pt}
\setlist[itemize]{nosep}
\setlist[enumerate]{nosep}

% Title formatting
\titleformat{\section}{\Large\bfseries\color{headcolor}}{\thesection}{1em}{}
\titleformat{\subsection}{\large\bfseries\color{headcolor}}{\thesubsection}{1em}{}

% Pandoc tightlist compatibility
\providecommand{\tightlist}{%
  \setlength{\itemsep}{0pt}\setlength{\parskip}{0pt}}

% Pandoc longtable compatibility
\newcounter{none}
\def\thenone{}


% content/resources/templates/english-boxes.tex
% This file is currently empty - it exists to maintain consistency with the import structure.
% Add custom environments here if needed in the future.


\begin{document}

\begin{center}
{\Huge\bfseries\color{headcolor} Subject Name Solutions}\\[5pt]
{\LARGE 4361106 -- Summer 2024}\\[3pt]
{\large Semester 1 Study Material}\\[3pt]
{\normalsize\textit{Detailed Solutions and Explanations}}
\end{center}

\vspace{10pt}

\subsection*{Question 1(a) [3 marks]}\label{q1a}

\textbf{What is Renewable energy? Explain its importance.}

\begin{solutionbox}
Renewable energy is energy derived from natural sources
that replenish themselves over time, such as solar, wind, hydro,
biomass, and geothermal.


{\def\LTcaptype{none} % do not increment counter
\vspace{-5pt}
\captionof{table}{Importance of Renewable Energy}
\vspace{-10pt}
\begin{longtable}[]{@{}
  >{\raggedright\arraybackslash}p{(\linewidth - 2\tabcolsep) * \real{0.4706}}
  >{\raggedright\arraybackslash}p{(\linewidth - 2\tabcolsep) * \real{0.5294}}@{}}
\toprule\noalign{}
\begin{minipage}[b]{\linewidth}\raggedright
Aspect
\end{minipage} & \begin{minipage}[b]{\linewidth}\raggedright
Benefit
\end{minipage} \\
\midrule\noalign{}
\endhead
\bottomrule\noalign{}
\endlastfoot
\textbf{Environmental} & Reduces greenhouse gas emissions and
pollution \\
\textbf{Economic} & Creates jobs and reduces energy costs long-term \\
\textbf{Energy Security} & Reduces dependence on fossil fuel imports \\
\textbf{Sustainability} & Inexhaustible energy sources for future
generations \\
\end{longtable}
}

\textbf{Key Points:}

\begin{itemize}
\tightlist
\item
  \textbf{Clean Energy}: Zero carbon emissions during operation
\item
  \textbf{Cost-effective}: Decreasing technology costs make it
  economical
\item
  \textbf{Job Creation}: Growing industry providing employment
  opportunities
\end{itemize}

\end{solutionbox}
\begin{mnemonicbox}
``EEES'' - Environmental protection, Economic
benefits, Energy security, Sustainability

\end{mnemonicbox}
\begin{center}\rule{0.5\linewidth}{0.5pt}\end{center}

\subsection*{Question 1(b) [4 marks]}\label{q1b}

\textbf{List the types of Electric Vehicles. Explain each in brief.}

\begin{solutionbox}


{\def\LTcaptype{none} % do not increment counter
\vspace{-5pt}
\captionof{table}{Types of Electric Vehicles}
\vspace{-10pt}
\begin{longtable}[]{@{}
  >{\raggedright\arraybackslash}p{(\linewidth - 4\tabcolsep) * \real{0.2000}}
  >{\raggedright\arraybackslash}p{(\linewidth - 4\tabcolsep) * \real{0.3667}}
  >{\raggedright\arraybackslash}p{(\linewidth - 4\tabcolsep) * \real{0.4333}}@{}}
\toprule\noalign{}
\begin{minipage}[b]{\linewidth}\raggedright
Type
\end{minipage} & \begin{minipage}[b]{\linewidth}\raggedright
Full Form
\end{minipage} & \begin{minipage}[b]{\linewidth}\raggedright
Description
\end{minipage} \\
\midrule\noalign{}
\endhead
\bottomrule\noalign{}
\endlastfoot
\textbf{BEV} & Battery Electric Vehicle & Fully electric, powered only
by battery \\
\textbf{HEV} & Hybrid Electric Vehicle & Combines gasoline engine with
electric motor \\
\textbf{PHEV} & Plug-in Hybrid Electric Vehicle & Can be charged from
external power source \\
\textbf{FCEV} & Fuel Cell Electric Vehicle & Uses hydrogen fuel cells
for power \\
\end{longtable}
}

\textbf{Key Features:}

\begin{itemize}
\tightlist
\item
  \textbf{BEV}: Zero emissions, requires charging stations
\item
  \textbf{HEV}: Better fuel efficiency, self-charging through
  regenerative braking
\item
  \textbf{PHEV}: Dual power options, extended range
\item
  \textbf{FCEV}: Quick refueling, water as only emission
\end{itemize}

\end{solutionbox}
\begin{mnemonicbox}
``Big Hybrid Plug Fuel'' for BEV, HEV, PHEV, FCEV

\end{mnemonicbox}
\begin{center}\rule{0.5\linewidth}{0.5pt}\end{center}

\subsection*{Question 1(c) [7 marks]}\label{q1c}

\textbf{What is the difference between solar energy and solar thermal
energy? Discuss the block diagram of home solar rooftop system.}

\begin{solutionbox}


{\def\LTcaptype{none} % do not increment counter
\vspace{-5pt}
\captionof{table}{Solar Energy vs Solar Thermal Energy}
\vspace{-10pt}
\begin{longtable}[]{@{}
  >{\raggedright\arraybackslash}p{(\linewidth - 4\tabcolsep) * \real{0.2115}}
  >{\raggedright\arraybackslash}p{(\linewidth - 4\tabcolsep) * \real{0.3654}}
  >{\raggedright\arraybackslash}p{(\linewidth - 4\tabcolsep) * \real{0.4231}}@{}}
\toprule\noalign{}
\begin{minipage}[b]{\linewidth}\raggedright
Parameter
\end{minipage} & \begin{minipage}[b]{\linewidth}\raggedright
Solar Energy (PV)
\end{minipage} & \begin{minipage}[b]{\linewidth}\raggedright
Solar Thermal Energy
\end{minipage} \\
\midrule\noalign{}
\endhead
\bottomrule\noalign{}
\endlastfoot
\textbf{Conversion} & Direct sunlight to electricity & Sunlight to heat
energy \\
\textbf{Technology} & Photovoltaic cells & Solar collectors/panels \\
\textbf{Output} & Electrical energy & Thermal energy (hot
water/steam) \\
\textbf{Applications} & Power generation, lighting & Water heating,
space heating \\
\textbf{Efficiency} & 15-22\% & 70-80\% \\
\end{longtable}
}

\textbf{Block Diagram: Home Solar Rooftop System}

\begin{verbatim}
flowchart LR
    A[Solar Panels] {-{-} B[DC Power]}
    B {-{-} C[Charge Controller]}
    C {-{-} D[Battery Bank]}
    C {-{-} E[Inverter]}
    E {-{-} F[AC Power]}
    F {-{-} G[Home Load]}
    F {-{-} H[Grid Connection]}
    I[Monitoring System] {-{-} C}
\end{verbatim}

\textbf{Key Components:}

\begin{itemize}
\tightlist
\item
  \textbf{Solar Panels}: Convert sunlight to DC electricity
\item
  \textbf{Charge Controller}: Regulates battery charging
\item
  \textbf{Inverter}: Converts DC to AC power
\item
  \textbf{Battery Bank}: Stores excess energy
\item
  \textbf{Grid Connection}: Two-way power flow
\end{itemize}

\end{solutionbox}
\begin{mnemonicbox}
``Solar Converts Battery Inverter Grid'' for main
components

\end{mnemonicbox}
\begin{center}\rule{0.5\linewidth}{0.5pt}\end{center}

\subsection*{Question 1(c OR) [7
marks]}\label{question-1c-or-7-marks}

\textbf{What is solar photovoltaic effect? Explain principle of
photovoltaic conversion.}

\begin{solutionbox}
Solar photovoltaic effect is the generation of electric
current when light falls on semiconductor materials.

\textbf{Principle of Photovoltaic Conversion:}

\begin{verbatim}
flowchart LR
    A[Sunlight Photons] {-{-} B[P{-}N Junction]}
    B {-{-} C[Electron{-}Hole Pairs]}
    C {-{-} D[Electric Field Separation]}
    D {-{-} E[Current Flow]}
    E {-{-} F[External Circuit]}
\end{verbatim}

\textbf{Working Process:}

\begin{itemize}
\tightlist
\item
  \textbf{Photon Absorption}: Light photons hit semiconductor material
\item
  \textbf{Electron Excitation}: Electrons gain energy and move to
  conduction band
\item
  \textbf{P-N Junction}: Creates electric field separating charges
\item
  \textbf{Current Generation}: Flow of electrons creates electrical
  current
\end{itemize}

\textbf{Key Points:}

\begin{itemize}
\tightlist
\item
  \textbf{Energy Conversion}: Light energy \rightarrow Electrical energy
\item
  \textbf{Semiconductor Material}: Usually silicon-based
\item
  \textbf{Direct Conversion}: No moving parts required
\item
  \textbf{Quantum Effect}: Based on photoelectric effect principle
\end{itemize}


{\def\LTcaptype{none} % do not increment counter
\vspace{-5pt}
\captionof{table}{PV Cell Materials}
\vspace{-10pt}
\begin{longtable}[]{@{}llll@{}}
\toprule\noalign{}
Material & Efficiency & Cost & Application \\
\midrule\noalign{}
\endhead
\bottomrule\noalign{}
\endlastfoot
\textbf{Monocrystalline Silicon} & 18-22\% & High & Residential \\
\textbf{Polycrystalline Silicon} & 15-17\% & Medium & Commercial \\
\textbf{Thin Film} & 10-12\% & Low & Large scale \\
\end{longtable}
}

\end{solutionbox}
\begin{mnemonicbox}
``Photons Push Electrons Producing Power''

\end{mnemonicbox}
\begin{center}\rule{0.5\linewidth}{0.5pt}\end{center}

\subsection*{Question 2(a) [3 marks]}\label{q2a}

\textbf{What is nanotechnology? List any three applications based on
nanotechnology.}

\begin{solutionbox}
Nanotechnology is the science of manipulating matter at
the molecular and atomic scale (1-100 nanometers).


{\def\LTcaptype{none} % do not increment counter
\vspace{-5pt}
\captionof{table}{Nanotechnology Applications}
\vspace{-10pt}
\begin{longtable}[]{@{}
  >{\raggedright\arraybackslash}p{(\linewidth - 4\tabcolsep) * \real{0.3714}}
  >{\raggedright\arraybackslash}p{(\linewidth - 4\tabcolsep) * \real{0.3714}}
  >{\raggedright\arraybackslash}p{(\linewidth - 4\tabcolsep) * \real{0.2571}}@{}}
\toprule\noalign{}
\begin{minipage}[b]{\linewidth}\raggedright
Application
\end{minipage} & \begin{minipage}[b]{\linewidth}\raggedright
Description
\end{minipage} & \begin{minipage}[b]{\linewidth}\raggedright
Benefit
\end{minipage} \\
\midrule\noalign{}
\endhead
\bottomrule\noalign{}
\endlastfoot
\textbf{Medical} & Drug delivery systems, cancer treatment & Targeted
therapy \\
\textbf{Electronics} & Smaller, faster processors and memory & Higher
performance \\
\textbf{Energy} & Solar cells, batteries, fuel cells & Better
efficiency \\
\end{longtable}
}

\textbf{Key Points:}

\begin{itemize}
\tightlist
\item
  \textbf{Scale}: Works at nanometer level (10^{-}^{9} meters)
\item
  \textbf{Precision}: Atomic-level manipulation
\item
  \textbf{Revolutionary}: Transforms multiple industries
\end{itemize}

\end{solutionbox}
\begin{mnemonicbox}
``Nano Makes Everything Better'' - Medical,
Electronics, Energy

\end{mnemonicbox}
\begin{center}\rule{0.5\linewidth}{0.5pt}\end{center}

\subsection*{Question 2(b) [4 marks]}\label{q2b}

\textbf{Write short note on Tidal wave energy as important emerging
renewable energy technology.}

\begin{solutionbox}
Tidal wave energy harnesses the kinetic energy of ocean
tides and waves to generate electricity.

\textbf{Key Features:}

\begin{itemize}
\tightlist
\item
  \textbf{Predictable}: Tides follow regular patterns
\item
  \textbf{High Density}: Water is 800 times denser than air
\item
  \textbf{Consistent}: Available day and night
\item
  \textbf{Clean}: No emissions or fuel consumption
\end{itemize}


{\def\LTcaptype{none} % do not increment counter
\vspace{-5pt}
\captionof{table}{Tidal Energy Systems}
\vspace{-10pt}
\begin{longtable}[]{@{}
  >{\raggedright\arraybackslash}p{(\linewidth - 4\tabcolsep) * \real{0.2400}}
  >{\raggedright\arraybackslash}p{(\linewidth - 4\tabcolsep) * \real{0.3200}}
  >{\raggedright\arraybackslash}p{(\linewidth - 4\tabcolsep) * \real{0.4400}}@{}}
\toprule\noalign{}
\begin{minipage}[b]{\linewidth}\raggedright
Type
\end{minipage} & \begin{minipage}[b]{\linewidth}\raggedright
Method
\end{minipage} & \begin{minipage}[b]{\linewidth}\raggedright
Advantage
\end{minipage} \\
\midrule\noalign{}
\endhead
\bottomrule\noalign{}
\endlastfoot
\textbf{Tidal Barrage} & Dam across estuary & High power output \\
\textbf{Tidal Stream} & Underwater turbines & Minimal environmental
impact \\
\textbf{Wave Energy} & Surface wave motion & Abundant resource \\
\end{longtable}
}

\textbf{Applications:}

\begin{itemize}
\tightlist
\item
  \textbf{Coastal Power Generation}: Remote coastal communities
\item
  \textbf{Grid Integration}: Supplement to other renewable sources
\item
  \textbf{Island Nations}: Ideal for maritime countries
\end{itemize}

\end{solutionbox}
\begin{mnemonicbox}
``Tides Provide Predictable Power''

\end{mnemonicbox}
\begin{center}\rule{0.5\linewidth}{0.5pt}\end{center}

\subsection*{Question 2(c) [7 marks]}\label{q2c}

\textbf{What is smart water monitoring system? Explain the block diagram
of Smart water Quality monitoring system.}

\begin{solutionbox}
Smart water monitoring system uses IoT sensors to
continuously monitor water quality parameters and provide real-time data
for decision making.

\textbf{Block Diagram: Smart Water Quality Monitoring System}

\begin{verbatim}
flowchart LR
    A[Water Source] {-{-} B[Sensor Array]}
    B {-{-} C[pH Sensor]}
    B {-{-} D[Turbidity Sensor]}
    B {-{-} E[Temperature Sensor]}
    B {-{-} F[Dissolved Oxygen Sensor]}
    C {-{-} G[Microcontroller]}
    D {-{-} G}
    E {-{-} G}
    F {-{-} G}
    G {-{-} H[Data Processing]}
    H {-{-} I[Wireless Communication]}
    I {-{-} J[Cloud Server]}
    J {-{-} K[Mobile App/Web Dashboard]}
    J {-{-} L[Alert System]}
\end{verbatim}

\textbf{Key Components:}

\begin{itemize}
\tightlist
\item
  \textbf{Sensors}: Monitor pH, turbidity, temperature, dissolved oxygen
\item
  \textbf{Microcontroller}: Arduino/Raspberry Pi for data processing
\item
  \textbf{Communication}: WiFi/GSM for data transmission
\item
  \textbf{Cloud Platform}: Data storage and analysis
\item
  \textbf{User Interface}: Mobile app for monitoring
\end{itemize}

\textbf{Benefits:}

\begin{itemize}
\tightlist
\item
  \textbf{Real-time Monitoring}: Continuous water quality assessment
\item
  \textbf{Early Warning}: Immediate alerts for contamination
\item
  \textbf{Data Analytics}: Historical trends and predictions
\item
  \textbf{Cost Effective}: Reduces manual testing costs
\end{itemize}


{\def\LTcaptype{none} % do not increment counter
\vspace{-5pt}
\captionof{table}{Water Quality Parameters}
\vspace{-10pt}
\begin{longtable}[]{@{}lll@{}}
\toprule\noalign{}
Parameter & Normal Range & Sensor Type \\
\midrule\noalign{}
\endhead
\bottomrule\noalign{}
\endlastfoot
\textbf{pH} & 6.5-8.5 & pH electrode \\
\textbf{Turbidity} & \textless1 NTU & Optical sensor \\
\textbf{Temperature} & 15-25^\circC & Thermistor \\
\textbf{Dissolved Oxygen} & \textgreater5 mg/L & Electrochemical \\
\end{longtable}
}

\end{solutionbox}
\begin{mnemonicbox}
``Smart Sensors Send Signals Safely''

\end{mnemonicbox}
\begin{center}\rule{0.5\linewidth}{0.5pt}\end{center}

\subsection*{Question 2(a OR) [3
marks]}\label{question-2a-or-3-marks}

\textbf{What is wearable technology? Name atleast two applications of
wearable technology?}

\begin{solutionbox}
Wearable technology refers to electronic devices that
can be worn as clothing or accessories, incorporating smart sensors and
connectivity.

\textbf{Applications:}

\begin{itemize}
\tightlist
\item
  \textbf{Health Monitoring}: Smartwatches tracking heart rate, steps,
  sleep patterns
\item
  \textbf{Fitness Tracking}: Activity monitors measuring calories,
  distance, exercise
\item
  \textbf{Medical Devices}: Continuous glucose monitors, blood pressure
  monitors
\item
  \textbf{Smart Glasses}: Augmented reality displays, hands-free
  computing
\end{itemize}

\textbf{Key Features:}

\begin{itemize}
\tightlist
\item
  \textbf{Portable}: Lightweight and comfortable to wear
\item
  \textbf{Connected}: Bluetooth/WiFi connectivity to smartphones
\item
  \textbf{Sensor-rich}: Multiple sensors for data collection
\end{itemize}

\end{solutionbox}
\begin{mnemonicbox}
``Wearables Watch Wellness Wirelessly''

\end{mnemonicbox}
\begin{center}\rule{0.5\linewidth}{0.5pt}\end{center}

\subsection*{Question 2(b OR) [4
marks]}\label{question-2b-or-4-marks}

\textbf{List the different types of solar cell. List different energy
sources for Electric vehicle.}

\begin{solutionbox}


{\def\LTcaptype{none} % do not increment counter
\vspace{-5pt}
\captionof{table}{Types of Solar Cells}
\vspace{-10pt}
\begin{longtable}[]{@{}llll@{}}
\toprule\noalign{}
Type & Material & Efficiency & Cost \\
\midrule\noalign{}
\endhead
\bottomrule\noalign{}
\endlastfoot
\textbf{Monocrystalline} & Single crystal silicon & 18-22\% & High \\
\textbf{Polycrystalline} & Multi-crystal silicon & 15-17\% & Medium \\
\textbf{Thin Film} & Amorphous silicon & 10-12\% & Low \\
\textbf{Cadmium Telluride} & CdTe compound & 16-18\% & Medium \\
\end{longtable}
}


{\def\LTcaptype{none} % do not increment counter
\vspace{-5pt}
\captionof{table}{Energy Sources for Electric Vehicles}
\vspace{-10pt}
\begin{longtable}[]{@{}
  >{\raggedright\arraybackslash}p{(\linewidth - 4\tabcolsep) * \real{0.2500}}
  >{\raggedright\arraybackslash}p{(\linewidth - 4\tabcolsep) * \real{0.4062}}
  >{\raggedright\arraybackslash}p{(\linewidth - 4\tabcolsep) * \real{0.3438}}@{}}
\toprule\noalign{}
\begin{minipage}[b]{\linewidth}\raggedright
Source
\end{minipage} & \begin{minipage}[b]{\linewidth}\raggedright
Description
\end{minipage} & \begin{minipage}[b]{\linewidth}\raggedright
Advantage
\end{minipage} \\
\midrule\noalign{}
\endhead
\bottomrule\noalign{}
\endlastfoot
\textbf{Battery} & Lithium-ion cells & High energy density \\
\textbf{Fuel Cell} & Hydrogen conversion & Quick refueling \\
\textbf{Ultracapacitor} & Rapid charge/discharge & Fast charging \\
\textbf{Regenerative Braking} & Kinetic energy recovery & Energy
efficiency \\
\end{longtable}
}

\end{solutionbox}
\begin{mnemonicbox}
``Solar: Mono Poly Thin Cadmium'' / ``EV: Battery
Fuel Ultra Regen''

\end{mnemonicbox}
\begin{center}\rule{0.5\linewidth}{0.5pt}\end{center}

\subsection*{Question 2(c OR) [7
marks]}\label{question-2c-or-7-marks}

\textbf{Describe the block diagram of a drone and its major components.}

\begin{solutionbox}

\textbf{Block Diagram: Drone System}

\begin{verbatim}
flowchart TD
    A[Flight Controller] {-{-} B[ESC 1]}
    A {-{-} C[ESC 2]}
    A {-{-} D[ESC 3]}
    A {-{-} E[ESC 4]}
    B {-{-} F[Motor 1]}
    C {-{-} G[Motor 2]}
    D {-{-} H[Motor 3]}
    E {-{-} I[Motor 4]}
    J[GPS Module] {-{-} A}
    K[IMU Sensors] {-{-} A}
    L[Battery] {-{-} A}
    M[Camera/Gimbal] {-{-} A}
    N[Radio Receiver] {-{-} A}
    O[Remote Controller] {-{-} N}
\end{verbatim}

\textbf{Major Components:}


{\def\LTcaptype{none} % do not increment counter
\vspace{-5pt}
\captionof{table}{Drone Components}
\vspace{-10pt}
\begin{longtable}[]{@{}lll@{}}
\toprule\noalign{}
Component & Function & Importance \\
\midrule\noalign{}
\endhead
\bottomrule\noalign{}
\endlastfoot
\textbf{Flight Controller} & Central processing unit & Brain of drone \\
\textbf{ESC} & Motor speed control & Precise motor control \\
\textbf{Motors \& Propellers} & Generate thrust & Flight capability \\
\textbf{Battery} & Power supply & Flight duration \\
\textbf{GPS} & Position tracking & Navigation \\
\textbf{IMU} & Motion sensing & Stability control \\
\end{longtable}
}

\textbf{Key Systems:}

\begin{itemize}
\tightlist
\item
  \textbf{Propulsion System}: 4 motors with propellers for lift and
  control
\item
  \textbf{Control System}: Flight controller with stabilization
  algorithms
\item
  \textbf{Navigation System}: GPS and compass for positioning
\item
  \textbf{Power System}: LiPo battery for electrical power
\item
  \textbf{Communication}: Radio link with ground controller
\end{itemize}

\textbf{Working Principle:}

\begin{itemize}
\tightlist
\item
  \textbf{Lift}: Rotors create upward thrust
\item
  \textbf{Control}: Varying rotor speeds controls movement
\item
  \textbf{Stability}: Sensors maintain balance and orientation
\end{itemize}

\end{solutionbox}
\begin{mnemonicbox}
``Drones Fly Using Motors, Electronics, Sensors,
Power''

\end{mnemonicbox}
\begin{center}\rule{0.5\linewidth}{0.5pt}\end{center}

\subsection*{Question 3(a) [3 marks]}\label{q3a}

\textbf{What is IoT? List Key Components of IoT.}

\begin{solutionbox}
IoT (Internet of Things) is a network of interconnected
physical devices that collect and exchange data through the internet.


{\def\LTcaptype{none} % do not increment counter
\vspace{-5pt}
\captionof{table}{Key Components of IoT}
\vspace{-10pt}
\begin{longtable}[]{@{}lll@{}}
\toprule\noalign{}
Component & Function & Example \\
\midrule\noalign{}
\endhead
\bottomrule\noalign{}
\endlastfoot
\textbf{Sensors} & Data collection & Temperature, humidity sensors \\
\textbf{Connectivity} & Data transmission & WiFi, Bluetooth, GSM \\
\textbf{Data Processing} & Information analysis & Cloud computing \\
\textbf{User Interface} & Human interaction & Mobile apps, dashboards \\
\end{longtable}
}

\textbf{Key Features:}

\begin{itemize}
\tightlist
\item
  \textbf{Interconnected}: Devices communicate with each other
\item
  \textbf{Smart}: Automated decision making
\item
  \textbf{Data-driven}: Continuous monitoring and analysis
\end{itemize}

\end{solutionbox}
\begin{mnemonicbox}
``IoT Connects Smart Devices Using Internet''

\end{mnemonicbox}
\begin{center}\rule{0.5\linewidth}{0.5pt}\end{center}

\subsection*{Question 3(b) [4 marks]}\label{q3b}

\textbf{Compare between organic and inorganic electronics.}

\begin{solutionbox}


{\def\LTcaptype{none} % do not increment counter
\vspace{-5pt}
\captionof{table}{Organic vs Inorganic Electronics}
\vspace{-10pt}
\begin{longtable}[]{@{}
  >{\raggedright\arraybackslash}p{(\linewidth - 4\tabcolsep) * \real{0.2037}}
  >{\raggedright\arraybackslash}p{(\linewidth - 4\tabcolsep) * \real{0.3704}}
  >{\raggedright\arraybackslash}p{(\linewidth - 4\tabcolsep) * \real{0.4259}}@{}}
\toprule\noalign{}
\begin{minipage}[b]{\linewidth}\raggedright
Parameter
\end{minipage} & \begin{minipage}[b]{\linewidth}\raggedright
Organic Electronics
\end{minipage} & \begin{minipage}[b]{\linewidth}\raggedright
Inorganic Electronics
\end{minipage} \\
\midrule\noalign{}
\endhead
\bottomrule\noalign{}
\endlastfoot
\textbf{Material} & Carbon-based compounds & Silicon, metals \\
\textbf{Manufacturing} & Low temperature, printing & High temperature,
clean room \\
\textbf{Flexibility} & Flexible, bendable & Rigid, brittle \\
\textbf{Cost} & Lower production cost & Higher production cost \\
\textbf{Performance} & Lower speed, efficiency & Higher speed,
efficiency \\
\textbf{Applications} & Displays, solar cells & Processors, memory \\
\end{longtable}
}

\textbf{Key Differences:}

\begin{itemize}
\tightlist
\item
  \textbf{Processing}: Organic uses solution-based processing
\item
  \textbf{Substrate}: Organic can use plastic substrates
\item
  \textbf{Durability}: Inorganic more stable and durable
\item
  \textbf{Innovation}: Organic enables new form factors
\end{itemize}

\end{solutionbox}
\begin{mnemonicbox}
``Organic: Flexible, Cheap, Printable vs Inorganic:
Fast, Stable, Expensive''

\end{mnemonicbox}
\begin{center}\rule{0.5\linewidth}{0.5pt}\end{center}

\subsection*{Question 3(c) [7 marks]}\label{q3c}

\textbf{Draw block diagram of smart street light control and monitoring
system. Discuss advantages and applications of AR/VR technology in
industry.}

\begin{solutionbox}

\textbf{Block Diagram: Smart Street Light System}

\begin{verbatim}
flowchart LR
    A[Light Sensor] {-{-} B[Microcontroller]}
    C[Motion Sensor] {-{-} B}
    D[Remote Control] {-{-} B}
    B {-{-} E[LED Driver]}
    E {-{-} F[LED Street Light]}
    B {-{-} G[Wireless Module]}
    G {-{-} H[Central Control]}
    H {-{-} I[Monitoring Dashboard]}
\end{verbatim}

\textbf{AR/VR Technology in Industry:}


{\def\LTcaptype{none} % do not increment counter
\vspace{-5pt}
\captionof{table}{AR/VR Applications}
\vspace{-10pt}
\begin{longtable}[]{@{}lll@{}}
\toprule\noalign{}
Industry & AR Application & VR Application \\
\midrule\noalign{}
\endhead
\bottomrule\noalign{}
\endlastfoot
\textbf{Manufacturing} & Assembly instructions & Training simulations \\
\textbf{Healthcare} & Surgery assistance & Medical training \\
\textbf{Education} & Interactive learning & Virtual classrooms \\
\textbf{Retail} & Product visualization & Virtual showrooms \\
\end{longtable}
}

\textbf{Advantages:}

\begin{itemize}
\tightlist
\item
  \textbf{Enhanced Training}: Safe, repeatable learning environments
\item
  \textbf{Remote Collaboration}: Virtual meetings and shared workspaces
\item
  \textbf{Design Visualization}: 3D prototyping and modeling
\item
  \textbf{Maintenance Support}: Real-time guidance and troubleshooting
\end{itemize}

\textbf{Key Benefits:}

\begin{itemize}
\tightlist
\item
  \textbf{Cost Reduction}: Lower training and travel costs
\item
  \textbf{Safety}: Risk-free training environments
\item
  \textbf{Efficiency}: Faster learning and problem-solving
\item
  \textbf{Innovation}: New ways of human-computer interaction
\end{itemize}

\end{solutionbox}
\begin{mnemonicbox}
``AR/VR: Training, Design, Remote, Maintenance''

\end{mnemonicbox}
\begin{center}\rule{0.5\linewidth}{0.5pt}\end{center}

\subsection*{Question 3(a OR) [3
marks]}\label{question-3a-or-3-marks}

\textbf{What is Smart System? List any four types of smart system.}

\begin{solutionbox}
Smart System is an intelligent system that uses
sensors, data processing, and automation to make decisions and adapt to
changing conditions.


{\def\LTcaptype{none} % do not increment counter
\vspace{-5pt}
\captionof{table}{Types of Smart Systems}
\vspace{-10pt}
\begin{longtable}[]{@{}
  >{\raggedright\arraybackslash}p{(\linewidth - 4\tabcolsep) * \real{0.2143}}
  >{\raggedright\arraybackslash}p{(\linewidth - 4\tabcolsep) * \real{0.4643}}
  >{\raggedright\arraybackslash}p{(\linewidth - 4\tabcolsep) * \real{0.3214}}@{}}
\toprule\noalign{}
\begin{minipage}[b]{\linewidth}\raggedright
Type
\end{minipage} & \begin{minipage}[b]{\linewidth}\raggedright
Description
\end{minipage} & \begin{minipage}[b]{\linewidth}\raggedright
Example
\end{minipage} \\
\midrule\noalign{}
\endhead
\bottomrule\noalign{}
\endlastfoot
\textbf{Smart Home} & Automated home control & Lighting, HVAC,
security \\
\textbf{Smart City} & Urban infrastructure management & Traffic,
utilities, waste \\
\textbf{Smart Grid} & Intelligent power distribution & Energy
management \\
\textbf{Smart Healthcare} & Medical monitoring systems & Patient
monitoring, diagnostics \\
\end{longtable}
}

\textbf{Key Features:}

\begin{itemize}
\tightlist
\item
  \textbf{Automated}: Self-operating capabilities
\item
  \textbf{Connected}: Internet connectivity
\item
  \textbf{Adaptive}: Learning and improving over time
\end{itemize}

\end{solutionbox}
\begin{mnemonicbox}
``Smart: Home, City, Grid, Health''

\end{mnemonicbox}
\begin{center}\rule{0.5\linewidth}{0.5pt}\end{center}

\subsection*{Question 3(b OR) [4
marks]}\label{question-3b-or-4-marks}

\textbf{List the advantages and applications of organic electronics.}

\begin{solutionbox}


{\def\LTcaptype{none} % do not increment counter
\vspace{-5pt}
\captionof{table}{Advantages of Organic Electronics}
\vspace{-10pt}
\begin{longtable}[]{@{}lll@{}}
\toprule\noalign{}
Advantage & Description & Benefit \\
\midrule\noalign{}
\endhead
\bottomrule\noalign{}
\endlastfoot
\textbf{Flexibility} & Bendable, stretchable & Wearable devices \\
\textbf{Low Cost} & Cheap manufacturing & Mass production \\
\textbf{Large Area} & Printing on large surfaces & Big displays \\
\textbf{Low Temperature} & Room temperature processing & Energy
efficient \\
\end{longtable}
}

\textbf{Applications:}

\begin{itemize}
\tightlist
\item
  \textbf{OLED Displays}: Smartphones, TVs, lighting
\item
  \textbf{Organic Solar Cells}: Flexible solar panels
\item
  \textbf{Organic Transistors}: Flexible circuits
\item
  \textbf{Electronic Paper}: E-readers, smart labels
\end{itemize}

\textbf{Key Benefits:}

\begin{itemize}
\tightlist
\item
  \textbf{Lightweight}: Suitable for portable devices
\item
  \textbf{Transparent}: See-through electronics
\item
  \textbf{Environmentally Friendly}: Biodegradable materials
\end{itemize}

\end{solutionbox}
\begin{mnemonicbox}
``Organic: Flexible, Cheap, Large, Low-temp''

\end{mnemonicbox}
\begin{center}\rule{0.5\linewidth}{0.5pt}\end{center}

\subsection*{Question 3(c OR) [7
marks]}\label{question-3c-or-7-marks}

\textbf{Draw basic block diagram of (i) wearable smart watch and (ii)
biometric system.}

\begin{solutionbox}

\textbf{(i) Wearable Smart Watch Block Diagram:}

\begin{verbatim}
flowchart TD
    A[Sensors] {-{-} B[Microprocessor]}
    C[Display] {-{-} B}
    D[Battery] {-{-} B}
    E[Wireless Module] {-{-} B}
    B {-{-} F[Memory]}
    B {-{-} G[Charging Port]}
    H[Heart Rate Sensor] {-{-} A}
    I[Accelerometer] {-{-} A}
    J[GPS] {-{-} A}
\end{verbatim}

\textbf{(ii) Biometric System Block Diagram:}

\begin{verbatim}
flowchart LR
    A[Biometric Sensor] {-{-} B[Signal Processing]}
    B {-{-} C[Feature Extraction]}
    C {-{-} D[Template Matching]}
    E[Database] {-{-} D}
    D {-{-} F[Decision Module]}
    F {-{-} G[Access Control]}
    H[Enrollment Module] {-{-} E}
\end{verbatim}

\textbf{Smart Watch Components:}

\begin{itemize}
\tightlist
\item
  \textbf{Sensors}: Heart rate, accelerometer, gyroscope
\item
  \textbf{Processor}: ARM-based microcontroller
\item
  \textbf{Display}: Touchscreen OLED/LCD
\item
  \textbf{Connectivity}: Bluetooth, WiFi, cellular
\item
  \textbf{Power}: Rechargeable lithium battery
\end{itemize}

\textbf{Biometric System Components:}

\begin{itemize}
\tightlist
\item
  \textbf{Sensor Module}: Captures biometric data
\item
  \textbf{Processing Unit}: Analyzes and extracts features
\item
  \textbf{Database}: Stores enrolled templates
\item
  \textbf{Matching Engine}: Compares with stored data
\item
  \textbf{Decision Logic}: Grants or denies access
\end{itemize}

\textbf{Key Features:}

\begin{itemize}
\tightlist
\item
  \textbf{Authentication}: Secure user identification
\item
  \textbf{Real-time}: Instant processing and response
\item
  \textbf{Accuracy}: High precision in identification
\end{itemize}

\end{solutionbox}
\begin{mnemonicbox}
``Smart Watch: Sense, Process, Display, Connect'' /
``Biometric: Capture, Process, Match, Decide''

\end{mnemonicbox}
\begin{center}\rule{0.5\linewidth}{0.5pt}\end{center}

\subsection*{Question 4(a) [3 marks]}\label{q4a}

\textbf{Give full form of NOOBS, GPIO \& LXDE in raspberry pi.}

\begin{solutionbox}


{\def\LTcaptype{none} % do not increment counter
\vspace{-5pt}
\captionof{table}{Raspberry Pi Acronyms}
\vspace{-10pt}
\begin{longtable}[]{@{}lll@{}}
\toprule\noalign{}
Acronym & Full Form & Purpose \\
\midrule\noalign{}
\endhead
\bottomrule\noalign{}
\endlastfoot
\textbf{NOOBS} & New Out Of Box Software & Easy OS installation \\
\textbf{GPIO} & General Purpose Input Output & Hardware interface
pins \\
\textbf{LXDE} & Lightweight X11 Desktop Environment & Desktop
interface \\
\end{longtable}
}

\textbf{Functions:}

\begin{itemize}
\tightlist
\item
  \textbf{NOOBS}: Simplifies Raspberry Pi setup for beginners
\item
  \textbf{GPIO}: 40-pin connector for external hardware
\item
  \textbf{LXDE}: User-friendly graphical interface
\end{itemize}

\end{solutionbox}
\begin{mnemonicbox}
``New GPIO, Lightweight Experience''

\end{mnemonicbox}
\begin{center}\rule{0.5\linewidth}{0.5pt}\end{center}

\subsection*{Question 4(b) [4 marks]}\label{q4b}

\textbf{Write a short note on OLED.}

\begin{solutionbox}
OLED (Organic Light Emitting Diode) is a display
technology using organic compounds that emit light when electric current
is applied.

\textbf{Key Features:}

\begin{itemize}
\tightlist
\item
  \textbf{Self-illuminating}: No backlight required
\item
  \textbf{Thin Profile}: Extremely thin displays
\item
  \textbf{High Contrast}: True black pixels
\item
  \textbf{Wide Viewing Angle}: No color distortion
\end{itemize}


{\def\LTcaptype{none} % do not increment counter
\vspace{-5pt}
\captionof{table}{OLED vs LCD}
\vspace{-10pt}
\begin{longtable}[]{@{}lll@{}}
\toprule\noalign{}
Parameter & OLED & LCD \\
\midrule\noalign{}
\endhead
\bottomrule\noalign{}
\endlastfoot
\textbf{Backlight} & Not required & Required \\
\textbf{Contrast} & Infinite & 1000:1 \\
\textbf{Thickness} & Ultra-thin & Thicker \\
\textbf{Power} & Lower (dark images) & Constant \\
\end{longtable}
}

\textbf{Applications:}

\begin{itemize}
\tightlist
\item
  \textbf{Smartphones}: Samsung, iPhone displays
\item
  \textbf{TVs}: Premium television sets
\item
  \textbf{Automotive}: Dashboard displays
\item
  \textbf{Wearables}: Smartwatch screens
\end{itemize}

\textbf{Advantages:}

\begin{itemize}
\tightlist
\item
  \textbf{Energy Efficient}: Lower power consumption
\item
  \textbf{Flexible}: Can be made bendable
\item
  \textbf{Fast Response}: No motion blur
\end{itemize}

\end{solutionbox}
\begin{mnemonicbox}
``OLED: Organic, Light, Emitting, Display''

\end{mnemonicbox}
\begin{center}\rule{0.5\linewidth}{0.5pt}\end{center}

\subsection*{Question 4(c) [7 marks]}\label{q4c}

\textbf{Explain the architecture and block diagram of Raspberry Pi.}

\begin{solutionbox}

\textbf{Block Diagram: Raspberry Pi Architecture}

\begin{verbatim}
flowchart TD
    A[ARM Cortex CPU] {-{-} B[System Bus]}
    C[GPU] {-{-} B}
    D[RAM] {-{-} B}
    E[Storage] {-{-} F[SD Card Slot]}
    F {-{-} B}
    B {-{-} G[GPIO Pins]}
    B {-{-} H[USB Ports]}
    B {-{-} I[Ethernet]}
    B {-{-} J[HDMI]}
    B {-{-} K[Audio Jack]}
    B {-{-} L[Camera Interface]}
    B {-{-} M[Display Interface]}
\end{verbatim}

\textbf{Key Components:}


{\def\LTcaptype{none} % do not increment counter
\vspace{-5pt}
\captionof{table}{Raspberry Pi Components}
\vspace{-10pt}
\begin{longtable}[]{@{}lll@{}}
\toprule\noalign{}
Component & Specification & Function \\
\midrule\noalign{}
\endhead
\bottomrule\noalign{}
\endlastfoot
\textbf{CPU} & ARM Cortex-A72 Quad-core & Main processing \\
\textbf{GPU} & VideoCore VI & Graphics processing \\
\textbf{RAM} & 4GB LPDDR4 & System memory \\
\textbf{Storage} & MicroSD card & Operating system \\
\textbf{GPIO} & 40-pin header & Hardware interface \\
\textbf{Connectivity} & WiFi, Bluetooth, Ethernet & Network access \\
\end{longtable}
}

\textbf{Architecture Features:}

\begin{itemize}
\tightlist
\item
  \textbf{SoC Design}: System on Chip integration
\item
  \textbf{Low Power}: Energy-efficient ARM processor
\item
  \textbf{Expandable}: GPIO pins for hardware projects
\item
  \textbf{Multimedia}: Hardware acceleration for video
\end{itemize}

\textbf{Interfaces:}

\begin{itemize}
\tightlist
\item
  \textbf{Video}: HDMI output up to 4K
\item
  \textbf{Audio}: 3.5mm jack and HDMI audio
\item
  \textbf{Camera}: CSI camera connector
\item
  \textbf{Display}: DSI display connector
\end{itemize}

\textbf{Applications:}

\begin{itemize}
\tightlist
\item
  \textbf{Education}: Learning programming and electronics
\item
  \textbf{IoT Projects}: Home automation, sensors
\item
  \textbf{Media Center}: Home entertainment system
\item
  \textbf{Robotics}: Control systems for robots
\end{itemize}

\end{solutionbox}
\begin{mnemonicbox}
``Pi: Processor, Interfaces, Projects, Internet''

\end{mnemonicbox}
\begin{center}\rule{0.5\linewidth}{0.5pt}\end{center}

\subsection*{Question 4(a OR) [3
marks]}\label{question-4a-or-3-marks}

\textbf{What is Raspberry Pi and its advantages and disadvantages?}

\begin{solutionbox}
Raspberry Pi is a small, affordable single-board
computer designed for education and hobbyist projects.


{\def\LTcaptype{none} % do not increment counter
\vspace{-5pt}
\captionof{table}{Advantages and Disadvantages}
\vspace{-10pt}
\begin{longtable}[]{@{}ll@{}}
\toprule\noalign{}
Advantages & Disadvantages \\
\midrule\noalign{}
\endhead
\bottomrule\noalign{}
\endlastfoot
\textbf{Low Cost} & \textbf{Limited Performance} \\
\textbf{Small Size} & \textbf{No Built-in Storage} \\
\textbf{GPIO Pins} & \textbf{Requires SD Card} \\
\textbf{Linux Support} & \textbf{No Real-time OS} \\
\textbf{Educational} & \textbf{Power Supply Issues} \\
\textbf{Community Support} & \textbf{Limited RAM} \\
\end{longtable}
}

\textbf{Key Features:}

\begin{itemize}
\tightlist
\item
  \textbf{Affordable}: Cost-effective computing solution
\item
  \textbf{Versatile}: Multiple programming languages supported
\item
  \textbf{Open Source}: Free software and documentation
\end{itemize}

\end{solutionbox}
\begin{mnemonicbox}
``Pi: Cheap, Small, Educational vs Limited, External,
Power''

\end{mnemonicbox}
\begin{center}\rule{0.5\linewidth}{0.5pt}\end{center}

\subsection*{Question 4(b OR) [4
marks]}\label{question-4b-or-4-marks}

\textbf{Write a short note on OFET.}

\begin{solutionbox}
OFET (Organic Field Effect Transistor) is a transistor
using organic semiconducting materials for switching and amplification.

\textbf{Key Features:}

\begin{itemize}
\tightlist
\item
  \textbf{Organic Materials}: Carbon-based semiconductors
\item
  \textbf{Low Temperature}: Solution-based processing
\item
  \textbf{Flexible}: Can be made on plastic substrates
\item
  \textbf{Large Area}: Suitable for big displays
\end{itemize}


{\def\LTcaptype{none} % do not increment counter
\vspace{-5pt}
\captionof{table}{OFET Structure}
\vspace{-10pt}
\begin{longtable}[]{@{}lll@{}}
\toprule\noalign{}
Component & Material & Function \\
\midrule\noalign{}
\endhead
\bottomrule\noalign{}
\endlastfoot
\textbf{Gate} & Metal electrode & Controls current flow \\
\textbf{Dielectric} & Insulating layer & Isolates gate from channel \\
\textbf{Source/Drain} & Metal contacts & Current injection/collection \\
\textbf{Channel} & Organic semiconductor & Current conduction path \\
\end{longtable}
}

\textbf{Applications:}

\begin{itemize}
\tightlist
\item
  \textbf{Flexible Displays}: Bendable screens
\item
  \textbf{Smart Cards}: RFID applications
\item
  \textbf{Sensors}: Chemical and biological detection
\item
  \textbf{Logic Circuits}: Simple digital circuits
\end{itemize}

\textbf{Advantages:}

\begin{itemize}
\tightlist
\item
  \textbf{Mechanical Flexibility}: Bendable electronics
\item
  \textbf{Low Cost}: Cheap manufacturing
\item
  \textbf{Room Temperature}: No high-temperature processing
\end{itemize}

\textbf{Limitations:}

\begin{itemize}
\tightlist
\item
  \textbf{Lower Mobility}: Slower than silicon
\item
  \textbf{Stability Issues}: Degradation over time
\item
  \textbf{Limited Performance}: Lower switching speeds
\end{itemize}

\end{solutionbox}
\begin{mnemonicbox}
``OFET: Organic, Flexible, Easy, Transistor''

\end{mnemonicbox}
\begin{center}\rule{0.5\linewidth}{0.5pt}\end{center}

\subsection*{Question 4(c OR) [7
marks]}\label{question-4c-or-7-marks}

\textbf{List the types of Ports in Raspberry Pi. Discuss various
operating systems of raspberry Pi.}

\begin{solutionbox}


{\def\LTcaptype{none} % do not increment counter
\vspace{-5pt}
\captionof{table}{Raspberry Pi Ports}
\vspace{-10pt}
\begin{longtable}[]{@{}lll@{}}
\toprule\noalign{}
Port Type & Quantity & Function \\
\midrule\noalign{}
\endhead
\bottomrule\noalign{}
\endlastfoot
\textbf{USB} & 4 ports & Connect peripherals \\
\textbf{HDMI} & 2 micro HDMI & Video output \\
\textbf{GPIO} & 40 pins & Hardware interface \\
\textbf{Ethernet} & 1 port & Wired network \\
\textbf{Audio} & 3.5mm jack & Audio output \\
\textbf{Power} & USB-C & Power input \\
\textbf{Camera} & CSI connector & Camera module \\
\textbf{Display} & DSI connector & Display panel \\
\end{longtable}
}

\textbf{Operating Systems for Raspberry Pi:}


{\def\LTcaptype{none} % do not increment counter
\vspace{-5pt}
\captionof{table}{Raspberry Pi Operating Systems}
\vspace{-10pt}
\begin{longtable}[]{@{}lll@{}}
\toprule\noalign{}
OS & Type & Best For \\
\midrule\noalign{}
\endhead
\bottomrule\noalign{}
\endlastfoot
\textbf{Raspberry Pi OS} & Debian-based & General use, beginners \\
\textbf{Ubuntu} & Linux distribution & Server applications \\
\textbf{LibreELEC} & Media center & Home entertainment \\
\textbf{RetroPie} & Gaming & Retro gaming console \\
\textbf{Windows 10 IoT} & Microsoft OS & IoT development \\
\textbf{OSMC} & Media center & Media streaming \\
\end{longtable}
}

\textbf{Key Features of Raspberry Pi OS:}

\begin{itemize}
\tightlist
\item
  \textbf{Pre-installed Software}: Programming tools, office suite
\item
  \textbf{GPIO Support}: Hardware interfacing libraries
\item
  \textbf{Educational}: Scratch, Python, Minecraft Pi
\item
  \textbf{Lightweight}: Optimized for ARM processors
\end{itemize}

\textbf{Installation Methods:}

\begin{itemize}
\tightlist
\item
  \textbf{NOOBS}: Beginner-friendly installer
\item
  \textbf{Raspberry Pi Imager}: Official imaging tool
\item
  \textbf{Direct Flash}: Advanced users
\end{itemize}

\textbf{Benefits:}

\begin{itemize}
\tightlist
\item
  \textbf{Variety}: Multiple OS options for different purposes
\item
  \textbf{Community}: Large user base and support
\item
  \textbf{Updates}: Regular security and feature updates
\item
  \textbf{Customization}: Open source flexibility
\end{itemize}

\end{solutionbox}
\begin{mnemonicbox}
``Pi Ports: USB, HDMI, GPIO, Ethernet'' / ``Pi OS:
Official, Ubuntu, Media, Gaming''

\end{mnemonicbox}
\begin{center}\rule{0.5\linewidth}{0.5pt}\end{center}

\subsection*{Question 5(a) [3 marks]}\label{q5a}

\textbf{Explain NumPy python library For Machine Learning.}

\begin{solutionbox}
NumPy (Numerical Python) is a fundamental library for
scientific computing, providing support for large multi-dimensional
arrays and mathematical functions.

\textbf{Key Features:}

\begin{itemize}
\tightlist
\item
  \textbf{N-dimensional Arrays}: Efficient array operations
\item
  \textbf{Mathematical Functions}: Linear algebra, Fourier transforms
\item
  \textbf{Broadcasting}: Operations on arrays of different shapes
\item
  \textbf{Memory Efficient}: Faster than Python lists
\end{itemize}


{\def\LTcaptype{none} % do not increment counter
\vspace{-5pt}
\captionof{table}{NumPy in Machine Learning}
\vspace{-10pt}
\begin{longtable}[]{@{}lll@{}}
\toprule\noalign{}
Function & Usage & Example \\
\midrule\noalign{}
\endhead
\bottomrule\noalign{}
\endlastfoot
\textbf{Arrays} & Data storage & np.array([1,2,3]) \\
\textbf{Linear Algebra} & Matrix operations & np.dot(a,b) \\
\textbf{Statistics} & Data analysis & np.mean(), np.std() \\
\textbf{Random} & Data generation & np.random.rand() \\
\end{longtable}
}

\textbf{Applications in ML:}

\begin{itemize}
\tightlist
\item
  \textbf{Data Preprocessing}: Array manipulation and cleaning
\item
  \textbf{Feature Engineering}: Mathematical transformations
\item
  \textbf{Model Implementation}: Matrix operations for algorithms
\end{itemize}

\end{solutionbox}
\begin{mnemonicbox}
``NumPy: Numbers, Python, Arrays, Math''

\end{mnemonicbox}
\begin{center}\rule{0.5\linewidth}{0.5pt}\end{center}

\subsection*{Question 5(b) [4 marks]}\label{q5b}

\textbf{What is organic photovoltaic cell (OPV)? Explain its working
principle.}

\begin{solutionbox}
OPV (Organic Photovoltaic) cell is a solar cell using
organic semiconductors to convert light into electricity.

\textbf{Working Principle:}

\begin{verbatim}
flowchart LR
    A[Sunlight] {-{-} B[Organic Active Layer]}
    B {-{-} C[Exciton Generation]}
    C {-{-} D[Charge Separation]}
    D {-{-} E[Electron Transport]}
    E {-{-} F[Current Collection]}
\end{verbatim}

\textbf{Key Steps:}

\begin{itemize}
\tightlist
\item
  \textbf{Light Absorption}: Organic molecules absorb photons
\item
  \textbf{Exciton Formation}: Bound electron-hole pairs created
\item
  \textbf{Charge Separation}: Excitons split at donor-acceptor interface
\item
  \textbf{Charge Transport}: Electrons and holes move to electrodes
\item
  \textbf{Current Collection}: External circuit completes the flow
\end{itemize}


{\def\LTcaptype{none} % do not increment counter
\vspace{-5pt}
\captionof{table}{OPV Structure}
\vspace{-10pt}
\begin{longtable}[]{@{}lll@{}}
\toprule\noalign{}
Layer & Material & Function \\
\midrule\noalign{}
\endhead
\bottomrule\noalign{}
\endlastfoot
\textbf{Anode} & ITO & Transparent electrode \\
\textbf{Active Layer} & Organic blend & Light absorption \\
\textbf{Cathode} & Aluminum & Back electrode \\
\textbf{Buffer Layers} & PEDOT:PSS & Improve efficiency \\
\end{longtable}
}

\textbf{Advantages:}

\begin{itemize}
\tightlist
\item
  \textbf{Flexible}: Can be made on plastic
\item
  \textbf{Lightweight}: Portable applications
\item
  \textbf{Low Cost}: Solution processing
\item
  \textbf{Transparent}: See-through panels
\end{itemize}

\textbf{Limitations:}

\begin{itemize}
\tightlist
\item
  \textbf{Lower Efficiency}: 10-15\% vs 20\%+ silicon
\item
  \textbf{Stability}: Degradation issues
\item
  \textbf{Lifetime}: Shorter than inorganic cells
\end{itemize}

\end{solutionbox}
\begin{mnemonicbox}
``OPV: Organic, Photons, Voltage, Excitons''

\end{mnemonicbox}
\begin{center}\rule{0.5\linewidth}{0.5pt}\end{center}

\subsection*{Question 5(c) [7 marks]}\label{q5c}

\textbf{List any four Machine learning tools. Discuss any one in brief.}

\begin{solutionbox}


{\def\LTcaptype{none} % do not increment counter
\vspace{-5pt}
\captionof{table}{Machine Learning Tools}
\vspace{-10pt}
\begin{longtable}[]{@{}lll@{}}
\toprule\noalign{}
Tool & Type & Best For \\
\midrule\noalign{}
\endhead
\bottomrule\noalign{}
\endlastfoot
\textbf{TensorFlow} & Deep learning framework & Neural networks \\
\textbf{Scikit-learn} & General ML library & Traditional algorithms \\
\textbf{PyTorch} & Deep learning framework & Research and development \\
\textbf{Keras} & High-level API & Rapid prototyping \\
\end{longtable}
}

\textbf{Detailed Discussion: TensorFlow}

TensorFlow is an open-source machine learning framework developed by
Google for building and deploying ML models.

\textbf{TensorFlow Features:}


{\def\LTcaptype{none} % do not increment counter
\vspace{-5pt}
\captionof{table}{TensorFlow Components}
\vspace{-10pt}
\begin{longtable}[]{@{}lll@{}}
\toprule\noalign{}
Component & Function & Benefit \\
\midrule\noalign{}
\endhead
\bottomrule\noalign{}
\endlastfoot
\textbf{Tensors} & Multi-dimensional arrays & Data representation \\
\textbf{Graphs} & Computational flow & Model visualization \\
\textbf{Sessions} & Execution environment & Resource management \\
\textbf{Estimators} & High-level APIs & Easy model building \\
\end{longtable}
}

\textbf{Architecture:}

\begin{itemize}
\tightlist
\item
  \textbf{Frontend}: Python, C++, Java APIs
\item
  \textbf{Backend}: CPU, GPU, TPU support\\
\item
  \textbf{Distributed}: Multi-device training
\item
  \textbf{Production}: Model serving and deployment
\end{itemize}

\textbf{Applications:}

\begin{itemize}
\tightlist
\item
  \textbf{Image Recognition}: Computer vision tasks
\item
  \textbf{Natural Language}: Text processing and translation
\item
  \textbf{Recommendation Systems}: Personalized content
\item
  \textbf{Time Series}: Forecasting and prediction
\end{itemize}

\textbf{Advantages:}

\begin{itemize}
\tightlist
\item
  \textbf{Scalability}: From mobile to data center
\item
  \textbf{Flexibility}: Research to production
\item
  \textbf{Community}: Large ecosystem and support
\item
  \textbf{Visualization}: TensorBoard for monitoring
\end{itemize}

\textbf{Code Example:}

\begin{verbatim}
import tensorflow as tf
model = tf.keras.Sequential([
    tf.keras.layers.Dense(128, activation={relu}),
    tf.keras.layers.Dense(10, activation={softmax})
])
\end{verbatim}

\textbf{Use Cases in Industry:}

\begin{itemize}
\tightlist
\item
  \textbf{Google}: Search and ads optimization
\item
  \textbf{Healthcare}: Medical image analysis
\item
  \textbf{Finance}: Fraud detection systems
\item
  \textbf{Automotive}: Autonomous vehicle development
\end{itemize}

\end{solutionbox}
\begin{mnemonicbox}
``TensorFlow: Tensors, Graphs, Scale, Deploy''

\end{mnemonicbox}
\begin{center}\rule{0.5\linewidth}{0.5pt}\end{center}

\subsection*{Question 5(a OR) [3
marks]}\label{question-5a-or-3-marks}

\textbf{Explain Pandas python library For Machine Learning.}

\begin{solutionbox}
Pandas is a Python library for data manipulation and
analysis, providing data structures and tools for handling structured
data.

\textbf{Key Features:}

\begin{itemize}
\tightlist
\item
  \textbf{DataFrame}: 2D labeled data structure
\item
  \textbf{Series}: 1D labeled array
\item
  \textbf{Data Cleaning}: Handle missing values, duplicates
\item
  \textbf{File I/O}: Read/write CSV, Excel, JSON, SQL
\end{itemize}


{\def\LTcaptype{none} % do not increment counter
\vspace{-5pt}
\captionof{table}{Pandas in Machine Learning}
\vspace{-10pt}
\begin{longtable}[]{@{}lll@{}}
\toprule\noalign{}
Function & Usage & Example \\
\midrule\noalign{}
\endhead
\bottomrule\noalign{}
\endlastfoot
\textbf{Data Loading} & Import datasets & pd.read\_csv() \\
\textbf{Data Cleaning} & Remove/fill missing & df.dropna() \\
\textbf{Data Selection} & Filter data & df[df[`col']
\textgreater{} 5] \\
\textbf{Aggregation} & Group and summarize & df.groupby().mean() \\
\end{longtable}
}

\textbf{Applications in ML:}

\begin{itemize}
\tightlist
\item
  \textbf{Data Preprocessing}: Clean and prepare datasets
\item
  \textbf{Feature Engineering}: Create new features from existing data
\item
  \textbf{Exploratory Analysis}: Understand data patterns and
  relationships
\end{itemize}

\end{solutionbox}
\begin{mnemonicbox}
``Pandas: Python, Analysis, Data, Structure''

\end{mnemonicbox}
\begin{center}\rule{0.5\linewidth}{0.5pt}\end{center}

\subsection*{Question 5(b OR) [4
marks]}\label{question-5b-or-4-marks}

\textbf{Explain the Differences between augmented reality and virtual
reality.}

\begin{solutionbox}


{\def\LTcaptype{none} % do not increment counter
\vspace{-5pt}
\captionof{table}{AR vs VR Comparison}
\vspace{-10pt}
\begin{longtable}[]{@{}
  >{\raggedright\arraybackslash}p{(\linewidth - 4\tabcolsep) * \real{0.1930}}
  >{\raggedright\arraybackslash}p{(\linewidth - 4\tabcolsep) * \real{0.4211}}
  >{\raggedright\arraybackslash}p{(\linewidth - 4\tabcolsep) * \real{0.3860}}@{}}
\toprule\noalign{}
\begin{minipage}[b]{\linewidth}\raggedright
Parameter
\end{minipage} & \begin{minipage}[b]{\linewidth}\raggedright
Augmented Reality (AR)
\end{minipage} & \begin{minipage}[b]{\linewidth}\raggedright
Virtual Reality (VR)
\end{minipage} \\
\midrule\noalign{}
\endhead
\bottomrule\noalign{}
\endlastfoot
\textbf{Environment} & Real world + digital overlay & Completely virtual
world \\
\textbf{Hardware} & Smartphone, AR glasses & VR headset, controllers \\
\textbf{Immersion} & Partial immersion & Full immersion \\
\textbf{Interaction} & Real world + digital objects & Virtual objects
only \\
\textbf{Cost} & Lower cost & Higher cost \\
\textbf{Mobility} & Mobile and portable & Stationary setup \\
\end{longtable}
}

\textbf{Key Differences:}

\begin{itemize}
\tightlist
\item
  \textbf{Reality Mix}: AR blends real and virtual, VR replaces reality
\item
  \textbf{User Experience}: AR enhances reality, VR creates new reality
\item
  \textbf{Applications}: AR for navigation, shopping; VR for gaming,
  training
\item
  \textbf{Hardware Requirements}: AR needs less powerful hardware
\end{itemize}

\textbf{Examples:}

\begin{itemize}
\tightlist
\item
  \textbf{AR}: Pokemon Go, Snapchat filters, Google Maps navigation
\item
  \textbf{VR}: Oculus games, virtual tours, flight simulators
\end{itemize}

\textbf{Use Cases:}

\begin{itemize}
\tightlist
\item
  \textbf{AR}: Retail, education, maintenance, marketing
\item
  \textbf{VR}: Entertainment, training, therapy, design
\end{itemize}

\end{solutionbox}
\begin{mnemonicbox}
``AR: Augments Reality vs VR: Virtual Reality''

\end{mnemonicbox}
\begin{center}\rule{0.5\linewidth}{0.5pt}\end{center}

\subsection*{Question 5(c OR) [7
marks]}\label{question-5c-or-7-marks}

\textbf{What is Machine learning? Discuss various types of Machine
learning.}

\begin{solutionbox}
Machine Learning is a subset of artificial intelligence
that enables computers to learn and make decisions from data without
being explicitly programmed.

\textbf{Definition:} Machine learning uses algorithms to analyze data,
identify patterns, and make predictions or decisions based on the
learned patterns.

\textbf{Types of Machine Learning:}


{\def\LTcaptype{none} % do not increment counter
\vspace{-5pt}
\captionof{table}{Types of Machine Learning}
\vspace{-10pt}
\begin{longtable}[]{@{}
  >{\raggedright\arraybackslash}p{(\linewidth - 6\tabcolsep) * \real{0.1500}}
  >{\raggedright\arraybackslash}p{(\linewidth - 6\tabcolsep) * \real{0.3250}}
  >{\raggedright\arraybackslash}p{(\linewidth - 6\tabcolsep) * \real{0.2500}}
  >{\raggedright\arraybackslash}p{(\linewidth - 6\tabcolsep) * \real{0.2750}}@{}}
\toprule\noalign{}
\begin{minipage}[b]{\linewidth}\raggedright
Type
\end{minipage} & \begin{minipage}[b]{\linewidth}\raggedright
Description
\end{minipage} & \begin{minipage}[b]{\linewidth}\raggedright
Examples
\end{minipage} & \begin{minipage}[b]{\linewidth}\raggedright
Use Cases
\end{minipage} \\
\midrule\noalign{}
\endhead
\bottomrule\noalign{}
\endlastfoot
\textbf{Supervised} & Learns from labeled data & Classification,
Regression & Email spam, Price prediction \\
\textbf{Unsupervised} & Finds patterns in unlabeled data & Clustering,
Association & Customer segmentation \\
\textbf{Reinforcement} & Learns through trial and error & Q-learning,
Policy gradient & Game playing, Robotics \\
\end{longtable}
}

\textbf{1. Supervised Learning:}

\begin{verbatim}
flowchart LR
    A[Training Data] {-{-} B[Algorithm]}
    B {-{-} C[Model]}
    D[New Data] {-{-} C}
    C {-{-} E[Prediction]}
\end{verbatim}

\textbf{Supervised Learning Types:}

\begin{itemize}
\tightlist
\item
  \textbf{Classification}: Predicts categories (spam/not spam)
\item
  \textbf{Regression}: Predicts continuous values (house prices)
\end{itemize}

\textbf{2. Unsupervised Learning:}

\begin{itemize}
\tightlist
\item
  \textbf{Clustering}: Groups similar data points
\item
  \textbf{Association}: Finds relationships between variables
\item
  \textbf{Dimensionality Reduction}: Reduces data complexity
\end{itemize}

\textbf{3. Reinforcement Learning:}

\begin{itemize}
\tightlist
\item
  \textbf{Agent}: Learning entity
\item
  \textbf{Environment}: System being learned
\item
  \textbf{Reward}: Feedback mechanism
\item
  \textbf{Policy}: Strategy for actions
\end{itemize}

\textbf{Applications by Type:}


{\def\LTcaptype{none} % do not increment counter
\vspace{-5pt}
\captionof{table}{ML Applications}
\vspace{-10pt}
\begin{longtable}[]{@{}lll@{}}
\toprule\noalign{}
Type & Application & Industry \\
\midrule\noalign{}
\endhead
\bottomrule\noalign{}
\endlastfoot
\textbf{Supervised} & Medical diagnosis & Healthcare \\
\textbf{Unsupervised} & Market basket analysis & Retail \\
\textbf{Reinforcement} & Autonomous driving & Automotive \\
\end{longtable}
}

\textbf{Key Algorithms:}

\begin{itemize}
\tightlist
\item
  \textbf{Supervised}: Linear Regression, Decision Trees, SVM, Neural
  Networks
\item
  \textbf{Unsupervised}: K-Means, DBSCAN, PCA, Apriori
\item
  \textbf{Reinforcement}: Q-Learning, Actor-Critic, Deep Q-Networks
\end{itemize}

\textbf{Machine Learning Process:}

\begin{enumerate}
\tightlist
\item
  \textbf{Data Collection}: Gather relevant datasets
\item
  \textbf{Data Preprocessing}: Clean and prepare data
\item
  \textbf{Feature Selection}: Choose important variables
\item
  \textbf{Model Training}: Train algorithm on data
\item
  \textbf{Model Evaluation}: Test performance
\item
  \textbf{Deployment}: Implement in production
\end{enumerate}

\textbf{Benefits:}

\begin{itemize}
\tightlist
\item
  \textbf{Automation}: Reduces manual work
\item
  \textbf{Accuracy}: Better than human performance in many tasks
\item
  \textbf{Scalability}: Handles large datasets
\item
  \textbf{Adaptability}: Improves with more data
\end{itemize}

\textbf{Challenges:}

\begin{itemize}
\tightlist
\item
  \textbf{Data Quality}: Requires clean, relevant data
\item
  \textbf{Overfitting}: Model too specific to training data
\item
  \textbf{Interpretability}: Black box nature of some algorithms
\item
  \textbf{Computational Resources}: Requires significant processing
  power
\end{itemize}

\textbf{Real-world Examples:}

\begin{itemize}
\tightlist
\item
  \textbf{Netflix}: Movie recommendations (supervised)
\item
  \textbf{Amazon}: Customer segmentation (unsupervised)
\item
  \textbf{AlphaGo}: Game playing (reinforcement)
\end{itemize}

\textbf{Future Trends:}

\begin{itemize}
\tightlist
\item
  \textbf{Deep Learning}: Neural networks with multiple layers
\item
  \textbf{AutoML}: Automated machine learning pipelines
\item
  \textbf{Edge AI}: ML on mobile and IoT devices
\item
  \textbf{Explainable AI}: Making ML decisions interpretable
\end{itemize}

\end{solutionbox}
\begin{mnemonicbox}
``ML Types: Supervised teaches, Unsupervised
discovers, Reinforcement rewards''

\end{mnemonicbox}

\end{document}
