\documentclass[10pt,a4paper]{article}

% content/resources/templates/preamble.tex
\usepackage[margin=0.6in]{geometry}
\author{Milav Dabgar}
\usepackage{amsmath,amssymb,amsthm}
\usepackage{booktabs}
\usepackage{multirow}
\usepackage{xcolor}
\usepackage{tcolorbox}
\tcbuselibrary{breakable,skins}
\usepackage[colorlinks=true,linkcolor=blue]{hyperref}
\usepackage{titlesec}
\usepackage{enumitem}
\usepackage{tikz}
\usepackage{pgfplots}
\usepackage{circuitikz}
\usepackage[version=4]{mhchem}
\usepackage{longtable}
\usepackage{array}
\usepackage{float}
\usepackage{caption}
\usepackage{listings}

\lstset{
  basicstyle=\small\ttfamily,
  breaklines=true,
  breakatwhitespace=false,
  postbreak=\mbox{\textcolor{red}{$\hookrightarrow$}\space},
  float=false,
  numbers=left,
  numberstyle=\tiny\color{gray},
  numbersep=10pt,
  xleftmargin=2em,
  keywordstyle=\color{blue},
  commentstyle=\color{green!60!black},
  stringstyle=\color{purple},
  backgroundcolor=\color{gray!5},
  showstringspaces=false,
  tabsize=2,
  captionpos=b,
  keepspaces=true,
  columns=flexible
}

\pgfplotsset{compat=1.18}
\usetikzlibrary{shapes,arrows,positioning,calc,patterns,decorations.pathmorphing,decorations.markings,arrows.meta}

% Color scheme
\definecolor{headcolor}{RGB}{0,102,204}
\definecolor{keycolor}{RGB}{220,20,60}
\definecolor{solutioncolor}{RGB}{34,139,34}
\definecolor{mnemoniccolor}{RGB}{148,0,211}
\definecolor{codecolor}{RGB}{0,0,100}

% Spacing
\setlength{\parskip}{3pt}
\setlist[itemize]{nosep}
\setlist[enumerate]{nosep}

% Title formatting
\titleformat{\section}{\Large\bfseries\color{headcolor}}{\thesection}{1em}{}
\titleformat{\subsection}{\large\bfseries\color{headcolor}}{\thesubsection}{1em}{}

% Pandoc tightlist compatibility
\providecommand{\tightlist}{%
  \setlength{\itemsep}{0pt}\setlength{\parskip}{0pt}}

% Pandoc longtable compatibility
\newcounter{none}
\def\thenone{}


% content/resources/templates/english-boxes.tex
% This file is currently empty - it exists to maintain consistency with the import structure.
% Add custom environments here if needed in the future.


\begin{document}

\begin{center}
{\Huge\bfseries\color{headcolor} Subject Name Solutions}\\[5pt]
{\LARGE 4341102 -- Winter 2023}\\[3pt]
{\large Semester 1 Study Material}\\[3pt]
{\normalsize\textit{Detailed Solutions and Explanations}}
\end{center}

\vspace{10pt}

\subsection*{Question 1(a) [3 marks]}\label{q1a}

\textbf{Discuss the various communication channels characteristics.}

\begin{solutionbox}

{\def\LTcaptype{none} % do not increment counter
\begin{longtable}[]{@{}
  >{\raggedright\arraybackslash}p{(\linewidth - 2\tabcolsep) * \real{0.6486}}
  >{\raggedright\arraybackslash}p{(\linewidth - 2\tabcolsep) * \real{0.3514}}@{}}
\toprule\noalign{}
\begin{minipage}[b]{\linewidth}\raggedright
Channel Characteristic
\end{minipage} & \begin{minipage}[b]{\linewidth}\raggedright
Description
\end{minipage} \\
\midrule\noalign{}
\endhead
\bottomrule\noalign{}
\endlastfoot
\textbf{Bit rate} & Maximum number of bits transmitted per second \\
\textbf{Baud rate} & Number of signal units/symbols transmitted per
second \\
\textbf{Bandwidth} & Range of frequencies required for transmission \\
\textbf{Repeater distance} & Maximum distance between repeaters to
maintain signal quality \\
\textbf{Noise immunity} & Ability to resist interference from external
sources \\
\end{longtable}
}

\end{solutionbox}
\begin{mnemonicbox}
``BBRN'' - ``Better Bandwidth Requires Nice
planning''

\end{mnemonicbox}
\subsection*{Question 1(b) [4 marks]}\label{q1b}

\textbf{Give the difference between even and odd signal.}

\begin{solutionbox}

{\def\LTcaptype{none} % do not increment counter
\begin{longtable}[]{@{}
  >{\raggedright\arraybackslash}p{(\linewidth - 2\tabcolsep) * \real{0.5200}}
  >{\raggedright\arraybackslash}p{(\linewidth - 2\tabcolsep) * \real{0.4800}}@{}}
\toprule\noalign{}
\begin{minipage}[b]{\linewidth}\raggedright
Even Signal
\end{minipage} & \begin{minipage}[b]{\linewidth}\raggedright
Odd Signal
\end{minipage} \\
\midrule\noalign{}
\endhead
\bottomrule\noalign{}
\endlastfoot
\textbf{Mathematical representation}: x(−t) = x(t) &
\textbf{Mathematical representation}: x(−t) = −x(t) \\
\textbf{Symmetry}: Mirror symmetry around y-axis & \textbf{Symmetry}:
Origin symmetry (rotational) \\
\textbf{Fourier series}: Contains only cosine terms & \textbf{Fourier
series}: Contains only sine terms \\
\textbf{Examples}: cos(t), t^{2} & \textbf{Examples}: sin(t), t^{3} \\
\end{longtable}
}

\begin{center}
\textbf{Mermaid Diagram (Code)}
\begin{verbatim}
{Shaded}
{Highlighting}[]
graph LR
    A["Signal x(t)"] {-{-}{} B\{Test symmetry\}}
    B {-{-}{}|"x({-}t) = x(t)"| C[Even Signal]}
    B {-{-}{}|"x({-}t) = {-}x(t)"| D[Odd Signal]}
    C {-{-}{} E[Mirror symmetry]}
    D {-{-}{} F[Origin symmetry]}
{Highlighting}
{Shaded}
\end{verbatim}
\end{center}

\end{solutionbox}
\begin{mnemonicbox}
``EVEN signals are Equal when flipped, ODD signals
are Opposite when flipped''

\end{mnemonicbox}
\subsection*{Question 1(c) [7 marks]}\label{q1c}

\textbf{Define repeater. Explain how repeater works with help of
necessary circuit and waveforms.}

\begin{solutionbox}

\textbf{Repeater}: A device that receives, amplifies, and retransmits a
signal to extend the transmission distance without degradation.

\textbf{Working Principle}: Repeaters regenerate digital signals to
overcome attenuation and noise accumulation in transmission lines.

\textbf{Circuit Diagram}:

\begin{verbatim}
            +{-{-}{-}{-}{-}{-}{-}{-}{-}{-}{-}{-}{-}+                  +{-}{-}{-}{-}{-}{-}{-}{-}{-}{-}{-}{-}{-}+}
 Input      |             |    Regenerated   |             |    Output
Signal {-{-}{-}{-}|  Receiver   |{-}{-}{-}{-}{-}{-}Signal{-}{-}{-}{-}{-}|  Transmitter|{-}{-}{-}{-}Signal{-}{-}}
            |             |                  |             |
            +{-{-}{-}{-}{-}{-}{-}{-}{-}{-}{-}{-}{-}+                  +{-}{-}{-}{-}{-}{-}{-}{-}{-}{-}{-}{-}{-}+}
                    |                               |
                    |       +{-{-}{-}{-}{-}{-}{-}{-}{-}{-}{-}{-}+          |}
                    +{-{-}{-}{-}{-}{-}|   Clock    |{-}{-}{-}{-}{-}{-}{-}{-}{-}+}
                            | Recovery   |
                            +{-{-}{-}{-}{-}{-}{-}{-}{-}{-}{-}{-}+}
\end{verbatim}

\textbf{Waveform}:

\begin{verbatim}
  Input Signal            Repeater              Output Signal
     \_\_\_\_\_                                         \_\_\_\_\_
    |     |                                       |     |
\_\_\_\_|     |\_\_\_\_     {-{-}     \_\_\_         {-}{-}   \_\_\_\_|     |\_\_\_\_}
    |     |                                       |     |
    |     |                                       |     |
      Degraded                                    Regenerated
\end{verbatim}

\begin{itemize}
\tightlist
\item
  \textbf{Signal reception}: Detects incoming weak/distorted signals
\item
  \textbf{Amplification}: Strengthens the signal power
\item
  \textbf{Regeneration}: Reconstructs original digital waveform
\item
  \textbf{Retransmission}: Sends restored signal to next segment
\end{itemize}

\end{solutionbox}
\begin{mnemonicbox}
``RARE'' - ``Receive, Amplify, Regenerate, Emit''

\end{mnemonicbox}
\subsection*{Question 1(c) OR [7
marks]}\label{q1c}

\textbf{Draw block diagram of digital communication system and explain
in detail.}

\begin{solutionbox}

\begin{verbatim}
flowchart LR
    A[Information Source] {-{-} B[Source Encoder]}
    B {-{-} C[Channel Encoder]}
    C {-{-} D[Digital Modulator]}
    D {-{-} E[Channel]}
    E {-{-} F[Digital Demodulator]}
    F {-{-} G[Channel Decoder]}
    G {-{-} H[Source Decoder]}
    H {-{-} I[Information Sink]}
\end{verbatim}

{\def\LTcaptype{none} % do not increment counter
\begin{longtable}[]{@{}
  >{\raggedright\arraybackslash}p{(\linewidth - 2\tabcolsep) * \real{0.4118}}
  >{\raggedright\arraybackslash}p{(\linewidth - 2\tabcolsep) * \real{0.5882}}@{}}
\toprule\noalign{}
\begin{minipage}[b]{\linewidth}\raggedright
Block
\end{minipage} & \begin{minipage}[b]{\linewidth}\raggedright
Function
\end{minipage} \\
\midrule\noalign{}
\endhead
\bottomrule\noalign{}
\endlastfoot
\textbf{Information Source} & Generates message to be transmitted
(voice, video, data) \\
\textbf{Source Encoder} & Converts source data to digital form and
removes redundancy \\
\textbf{Channel Encoder} & Adds controlled redundancy for error
detection/correction \\
\textbf{Digital Modulator} & Converts digital data to signals suitable
for transmission \\
\textbf{Channel} & Physical medium through which signals travel \\
\textbf{Digital Demodulator} & Extracts digital data from received
signals \\
\textbf{Channel Decoder} & Detects/corrects errors using added
redundancy \\
\textbf{Source Decoder} & Reconstructs original source information \\
\end{longtable}
}

\end{solutionbox}
\begin{mnemonicbox}
``Send Clear Data Messages, Carefully Decode Secure
Information''

\end{mnemonicbox}
\subsection*{Question 2(a) [3 marks]}\label{q2a}

\textbf{Define Unit step function, Unit impulse function, Unit ramp
function.}

\begin{solutionbox}

{\def\LTcaptype{none} % do not increment counter
\begin{longtable}[]{@{}
  >{\raggedright\arraybackslash}p{(\linewidth - 4\tabcolsep) * \real{0.2439}}
  >{\raggedright\arraybackslash}p{(\linewidth - 4\tabcolsep) * \real{0.2927}}
  >{\raggedright\arraybackslash}p{(\linewidth - 4\tabcolsep) * \real{0.4634}}@{}}
\toprule\noalign{}
\begin{minipage}[b]{\linewidth}\raggedright
Function
\end{minipage} & \begin{minipage}[b]{\linewidth}\raggedright
Definition
\end{minipage} & \begin{minipage}[b]{\linewidth}\raggedright
Mathematical Form
\end{minipage} \\
\midrule\noalign{}
\endhead
\bottomrule\noalign{}
\endlastfoot
\textbf{Unit Step Function} & Takes value 0 for negative time and 1 for
positive time & u(t) = \{0, t \textless{} 0; 1, t \geq 0\} \\
\textbf{Unit Impulse Function} & Infinitely high, zero width pulse with
area 1 & δ(t) = \{\infty,

t = 0; 0, t \neq 0\} \\

\textbf{Unit Ramp Function} & Increases linearly with time for positive
time values & r(t) = \{0, t \textless{} 0; t, t \geq 0\} \\
\end{longtable}
}

\end{solutionbox}
\begin{mnemonicbox}
``SIR'' - ``Step Instantly, Impulse Rapidly, Ramp
Gradually''

\end{mnemonicbox}
\subsection*{Question 2(b) [4 marks]}\label{q2b}

\textbf{Define Continues time and discrete time signals and explain with
example.}

\begin{solutionbox}

{\def\LTcaptype{none} % do not increment counter
\begin{longtable}[]{@{}
  >{\raggedright\arraybackslash}p{(\linewidth - 6\tabcolsep) * \real{0.2600}}
  >{\raggedright\arraybackslash}p{(\linewidth - 6\tabcolsep) * \real{0.2400}}
  >{\raggedright\arraybackslash}p{(\linewidth - 6\tabcolsep) * \real{0.1800}}
  >{\raggedright\arraybackslash}p{(\linewidth - 6\tabcolsep) * \real{0.3200}}@{}}
\toprule\noalign{}
\begin{minipage}[b]{\linewidth}\raggedright
Signal Type
\end{minipage} & \begin{minipage}[b]{\linewidth}\raggedright
Definition
\end{minipage} & \begin{minipage}[b]{\linewidth}\raggedright
Example
\end{minipage} & \begin{minipage}[b]{\linewidth}\raggedright
Representation
\end{minipage} \\
\midrule\noalign{}
\endhead
\bottomrule\noalign{}
\endlastfoot
\textbf{Continuous-time Signal} & Defined for all values of time within
its duration & Sinusoidal wave x(t) = sin(t) & Smooth, unbroken curve \\
\textbf{Discrete-time Signal} & Defined only at specific time instants &
Digital samples x[n] = sin(nTs) & Sequence of distinct values \\
\end{longtable}
}

\textbf{Diagram}:

\begin{verbatim}
Continuous{-time:   }
      /{      /      /      /      /  }
     /  {    /      /      /      /   }
{-{-}{-}{-}/    {-}{-}/    {-}{-}/    {-}{-}/    {-}{-}/    {-}{-}{-}{-} t}
    
Discrete{-time:}
      o       o       o       o       o  
     /       /       /       /       /    
{-{-}{-}{-}/{-}{-}{-}{-}{-}{-}{-}/{-}{-}{-}{-}{-}{-}{-}/{-}{-}{-}{-}{-}{-}{-}/{-}{-}{-}{-}{-}{-}{-}/{-}{-}{-}{-} n}
    o       o       o       o       o
\end{verbatim}

\begin{itemize}
\tightlist
\item
  \textbf{Continuous-time}: Defined for all time t \in R (infinite values)
\item
  \textbf{Discrete-time}: Defined only at specific instants n \in Z
  (countable values)
\end{itemize}

\end{solutionbox}
\begin{mnemonicbox}
``CADD'' - ``Continuous Always, Discrete Dots''

\end{mnemonicbox}
\subsection*{Question 2(c) [7 marks]}\label{q2c}

\textbf{Explain the block diagram of ASK modulator and de-modulator with
waveform.}

\begin{solutionbox}

\textbf{ASK (Amplitude Shift Keying)}: A digital modulation technique
where binary data is represented by varying the amplitude of a carrier
wave.

\textbf{ASK Modulator}:

\begin{verbatim}
flowchart LR
    A[Digital Input] {-{-} B[Product Modulator]}
    C[Carrier Generator] {-{-} B}
    B {-{-} D[Bandpass Filter]}
    D {-{-} E[ASK Output]}
\end{verbatim}

\textbf{ASK Demodulator}:

\begin{verbatim}
flowchart LR
    A[ASK Input] {-{-} B[Envelope Detector]}
    B {-{-} C[Low Pass Filter]}
    C {-{-} D[Comparator]}
    D {-{-} E[Digital Output]}
\end{verbatim}

\textbf{Waveforms}:

\begin{verbatim}
Digital Input:
    \_\_\_     \_\_\_         \_\_\_     \_\_\_
\_\_\_|   |\_\_\_|   |\_\_\_\_\_\_\_|   |\_\_\_|   |\_\_\_
    1   0   1     0     1   0   1

Carrier Wave:
/{//////////////////}

ASK Output:
    /{/    //        //    //}
\_\_\_/    {\_\_/    \_\_\_\_\_\_/    \_\_/    \_\_\_}
High  Low  High   Low   High Low  High
\end{verbatim}

\begin{itemize}
\tightlist
\item
  \textbf{Modulator}: Varies carrier amplitude based on digital input
\item
  \textbf{Demodulator}: Extracts envelope and compares to threshold
\end{itemize}

\end{solutionbox}
\begin{mnemonicbox}
``APE'' - ``Amplify when Positive, Eliminate when
zero''

\end{mnemonicbox}
\subsection*{Question 2(a) OR [3
marks]}\label{q2a}

\textbf{Explain Singularity function.}

\begin{solutionbox}

\textbf{Singularity Function}: Mathematical functions that have
discontinuities or undefined values at specific points.

{\def\LTcaptype{none} % do not increment counter
\begin{longtable}[]{@{}
  >{\raggedright\arraybackslash}p{(\linewidth - 2\tabcolsep) * \real{0.7143}}
  >{\raggedright\arraybackslash}p{(\linewidth - 2\tabcolsep) * \real{0.2857}}@{}}
\toprule\noalign{}
\begin{minipage}[b]{\linewidth}\raggedright
Common Singularity Functions
\end{minipage} & \begin{minipage}[b]{\linewidth}\raggedright
Properties
\end{minipage} \\
\midrule\noalign{}
\endhead
\bottomrule\noalign{}
\endlastfoot
\textbf{Unit Step Function u(t)} & Jumps from 0 to 1 at t=0 \\
\textbf{Unit Impulse Function δ(t)} & Infinite at t=0, zero elsewhere,
with area=1 \\
\textbf{Unit Ramp Function r(t)} & Derivative of unit step is impulse \\
\end{longtable}
}

\textbf{Relationships}:

\begin{itemize}
\tightlist
\item
  δ(t) = d/dt[u(t)]
\item
  u(t) = \intδ(t)dt
\item
  r(t) = \intu(t)dt
\end{itemize}

\end{solutionbox}
\begin{mnemonicbox}
``SIR'' - ``Singularities Include Rapid changes''

\end{mnemonicbox}
\subsection*{Question 2(b) OR [4
marks]}\label{q2b}

\textbf{Give the difference between bit rate and baud rate.}

\begin{solutionbox}

{\def\LTcaptype{none} % do not increment counter
\begin{longtable}[]{@{}
  >{\raggedright\arraybackslash}p{(\linewidth - 4\tabcolsep) * \real{0.3438}}
  >{\raggedright\arraybackslash}p{(\linewidth - 4\tabcolsep) * \real{0.3125}}
  >{\raggedright\arraybackslash}p{(\linewidth - 4\tabcolsep) * \real{0.3438}}@{}}
\toprule\noalign{}
\begin{minipage}[b]{\linewidth}\raggedright
Parameter
\end{minipage} & \begin{minipage}[b]{\linewidth}\raggedright
Bit Rate
\end{minipage} & \begin{minipage}[b]{\linewidth}\raggedright
Baud Rate
\end{minipage} \\
\midrule\noalign{}
\endhead
\bottomrule\noalign{}
\endlastfoot
\textbf{Definition} & Number of bits transmitted per second & Number of
symbols transmitted per second \\
\textbf{Unit} & bits per second (bps) & symbols per second (Baud) \\
\textbf{Relation} & Bit rate = Baud rate \times Number of bits per symbol &
Baud rate = Bit rate \div Number of bits per symbol \\
\textbf{Example} & In QPSK, if Baud rate = 1200, Bit rate = 2400 bps &
In 16-QAM, if Bit rate = 9600 bps, Baud rate = 2400 \\
\end{longtable}
}

\begin{center}
\textbf{Mermaid Diagram (Code)}
\begin{verbatim}
{Shaded}
{Highlighting}[]
graph TD
    A[Transmission Rate] {-{-}{} B[Bit Rate]}
    A {-{-}{} C[Baud Rate]}
    B {-{-}{}|"bits/second"| D[Information Transfer Rate]}
    C {-{-}{}|"symbols/second"| E[Modulation Rate]}
    F[Modulation Technique] {-{-}{} G[Bits per Symbol]}
    G {-{-}{} H["Bit Rate = Baud Rate  Bits per Symbol"]}
{Highlighting}
{Shaded}
\end{verbatim}
\end{center}

\end{solutionbox}
\begin{mnemonicbox}
``BBSR'' - ``Bits for Binary Speed, Bauds for Symbol
Rate''

\end{mnemonicbox}
\subsection*{Question 2(c) OR [7
marks]}\label{q2c}

\textbf{Explain the Principle of 8-PSK signal. Also draw constellation
diagram and waveforms of its.}

\begin{solutionbox}

\textbf{8-PSK (Phase Shift Keying)}: A digital modulation technique
where data is encoded by shifting the phase of a carrier signal to 8
different positions.

\textbf{Principle}:

\begin{itemize}
\tightlist
\item
  Each symbol represents 3 bits (log_{2}8 = 3)
\item
  Phase shifts in multiples of 45^\circ (360^\circ\div8)
\item
  Maintains constant amplitude
\end{itemize}

\textbf{Constellation Diagram}:

\begin{verbatim}
                  000(0^)
                     o
                     |
            001(45^) o         o 111(315^)
                   /             {}
                  /               {}
        010(90^) o                 o 110(270^)
                  {               /}
                   {             /}
           011(135^) o         o 101(225^)
                     |
                     o
                  100(180^)
\end{verbatim}

\textbf{Waveform}:

\begin{verbatim}
Data:   000    001    010    011    100    101    110    111
        \_\_\_    \_\_\_    \_\_\_    \_\_\_    \_\_\_    \_\_\_    \_\_\_    \_\_\_
       |   |  |   |  |   |  |   |  |   |  |   |  |   |  |   |
       |   |  |   |  |   |  |   |  |   |  |   |  |   |  |   |
Phase:  0^    45^    90^   135^   180^   225^   270^   315^
        /{     /     /     /     /     /     /     /}
       /  {   /     /     /     /     /     /     /  }
Signal:/    { /     /     /     /     /     /     /    }
\end{verbatim}

\begin{itemize}
\tightlist
\item
  \textbf{Bandwidth efficiency}: 3 bits per symbol
\item
  \textbf{Constant amplitude}: Better power efficiency
\item
  \textbf{Error probability}: Higher than BPSK/QPSK but lower than
  16-PSK
\end{itemize}

\end{solutionbox}
\begin{mnemonicbox}
``8 Points Shifted in K-circle'' (8-PSK)

\end{mnemonicbox}
\subsection*{Question 3(a) [3 marks]}\label{q3a}

\textbf{Explain the block diagram of FSK modulator.}

\begin{solutionbox}

\textbf{FSK (Frequency Shift Keying)}: A digital modulation technique
where binary data is represented by varying the frequency of a carrier
wave.

\begin{verbatim}
flowchart LR
    A[Binary Input] {-{-} B\{Switch\}}
    C[Oscillator f1] {-{-} B}
    D[Oscillator f2] {-{-} B}
    B {-{-} E[Bandpass Filter]}
    E {-{-} F[FSK Output]}
\end{verbatim}

{\def\LTcaptype{none} % do not increment counter
\begin{longtable}[]{@{}ll@{}}
\toprule\noalign{}
Component & Function \\
\midrule\noalign{}
\endhead
\bottomrule\noalign{}
\endlastfoot
\textbf{Binary Input} & Digital data (0s and 1s) to be transmitted \\
\textbf{Oscillator 1} & Generates carrier at frequency f_{1} for bit `1' \\
\textbf{Oscillator 2} & Generates carrier at frequency f_{2} for bit `0' \\
\textbf{Switch} & Selects appropriate frequency based on input bit \\
\textbf{Bandpass Filter} & Smooths transitions between frequencies \\
\end{longtable}
}

\end{solutionbox}
\begin{mnemonicbox}
``FISO'' - ``Frequency Input Selects Oscillator''

\end{mnemonicbox}
\subsection*{Question 3(b) [4 marks]}\label{q3b}

\textbf{Draw the ASK and FSK modulation waveform for the sequence of
1010110011.}

\begin{solutionbox}

\begin{verbatim}
Binary Input: 1    0    1    0    1    1    0    0    1    1
             \_   \_   \_   \_   \_   \_   \_   \_   \_   \_   \_   \_   \_
             |   |   |   |   |   |   |   |   |   |   |   |   |
             |   |   |   |   |   |   |   |   |   |   |   |   |
            \_|\_\_\_|\_\_\_|\_\_\_|\_\_\_|\_\_\_|\_\_\_|\_\_\_|\_\_\_|\_\_\_|\_\_\_|\_\_\_|\_\_\_|\_

ASK Output:  |||       |||       |||||||||       ||||||||| 
             |||       |||       |||||||||       |||||||||
            \_|||\_\_\_\_\_|||\_\_\_\_\_|||||||\_\_\_\_\_|||||||\_\_\_\_\_\_\_

FSK Output:  |||       |||       |||       |||       |||
             |||       |||       |||       |||       |||
            \_|||\_{\_|||\_\_|||||||\_\_|||||||\_ }
                 {              }
High freq   (1)  Low(0)  (1)  Low(0)  (1)(1)  Low(0)  Low(0)  (1)(1)
\end{verbatim}

\textbf{Explanation}:

\begin{itemize}
\tightlist
\item
  \textbf{ASK}: High amplitude for bit `1', low amplitude for bit `0'
\item
  \textbf{FSK}: Higher frequency f_{1} for bit `1', lower frequency f_{2} for
  bit `0'
\end{itemize}

\end{solutionbox}
\begin{mnemonicbox}
``ASK changes Amplitude, FSK changes Frequency''

\end{mnemonicbox}
\subsection*{Question 3(c) [7 marks]}\label{q3c}

\textbf{Explain PSK signal generation and detection with help of its
functional diagram.}

\begin{solutionbox}

\textbf{PSK (Phase Shift Keying)}: A digital modulation technique where
data is encoded by changing the phase of a carrier signal.

\textbf{PSK Modulator}:

\begin{verbatim}
flowchart LR
    A[Binary Input] {-{-} B[Bipolar Converter]}
    B {-{-} C[Product Modulator]}
    D[Carrier Generator] {-{-} C}
    C {-{-} E[PSK Output]}
\end{verbatim}

\textbf{PSK Demodulator}:

\begin{verbatim}
flowchart LR
    A[PSK Input] {-{-} B[Product Demodulator]}
    C[Carrier Recovery] {-{-} B}
    B {-{-} D[Low Pass Filter]}
    D {-{-} E[Decision Device]}
    E {-{-} F[Binary Output]}
\end{verbatim}

\textbf{Waveforms}:

\begin{verbatim}
Binary Input:  1     0     1     1     0
              \_    \_    \_    \_    \_    \_    \_
              |    |    |    |    |    |    |
              |    |    |    |    |    |    |
             \_|\_\_\_\_|\_\_\_\_|\_\_\_\_|\_\_\_\_|\_\_\_\_|\_\_\_\_|\_\_\_

Bipolar:      +A   {-A    +A   +A   {-}A}
              \_    \_    \_    \_    \_    \_    \_
              |    |    |    |    |    |    |
              |    |    |    |    |    |    |
             \_|\_\_\_\_|\_\_\_\_|\_\_\_\_|\_\_\_\_|\_\_\_\_|\_\_\_\_|\_\_\_

Carrier:      /{///////////////}

PSK Output:   /{/  //  //  //  //}
              phase phase phase phase phase
               0^   180^   0^    0^   180^
\end{verbatim}

\begin{itemize}
\tightlist
\item
  \textbf{Generation}: Binary 1 \rightarrow 0^\circ phase, Binary 0 \rightarrow 180^\circ phase
\item
  \textbf{Detection}: Coherent demodulation with carrier recovery
\item
  \textbf{Advantages}: Better noise immunity than ASK
\end{itemize}

\end{solutionbox}
\begin{mnemonicbox}
``PSK Phases Shift with Knowledge of carrier''

\end{mnemonicbox}
\subsection*{Question 3(a) OR [3
marks]}\label{q3a}

\textbf{Compare Bits PER Symbol for digital modulation techniques-ASK,
FSK, PSK, QPSK, 8-PSK and 16-QAM.}

\begin{solutionbox}

{\def\LTcaptype{none} % do not increment counter
\begin{longtable}[]{@{}
  >{\raggedright\arraybackslash}p{(\linewidth - 6\tabcolsep) * \real{0.3235}}
  >{\raggedright\arraybackslash}p{(\linewidth - 6\tabcolsep) * \real{0.2500}}
  >{\raggedright\arraybackslash}p{(\linewidth - 6\tabcolsep) * \real{0.1176}}
  >{\raggedright\arraybackslash}p{(\linewidth - 6\tabcolsep) * \real{0.3088}}@{}}
\toprule\noalign{}
\begin{minipage}[b]{\linewidth}\raggedright
Modulation Technique
\end{minipage} & \begin{minipage}[b]{\linewidth}\raggedright
Bits per Symbol
\end{minipage} & \begin{minipage}[b]{\linewidth}\raggedright
States
\end{minipage} & \begin{minipage}[b]{\linewidth}\raggedright
Bandwidth Efficiency
\end{minipage} \\
\midrule\noalign{}
\endhead
\bottomrule\noalign{}
\endlastfoot
\textbf{ASK} & 1 & 2 & 1 bit/Hz \\
\textbf{FSK} & 1 & 2 & 0.5 bit/Hz \\
\textbf{PSK (BPSK)} & 1 & 2 & 1 bit/Hz \\
\textbf{QPSK} & 2 & 4 & 2 bits/Hz \\
\textbf{8-PSK} & 3 & 8 & 3 bits/Hz \\
\textbf{16-QAM} & 4 & 16 & 4 bits/Hz \\
\end{longtable}
}

\begin{center}
\textbf{Mermaid Diagram (Code)}
\begin{verbatim}
{Shaded}
{Highlighting}[]
graph TD
    A[Modulation Techniques]
    A {-{-}{} B[ASK/FSK/BPSK{}br /{}1 bit/symbol]}
    A {-{-}{} C[QPSK{}br /{}2 bits/symbol]}
    A {-{-}{} D[8{-}PSK{}br /{}3 bits/symbol]}
    A {-{-}{} E[16{-}QAM{}br /{}4 bits/symbol]}
{Highlighting}
{Shaded}
\end{verbatim}
\end{center}

\end{solutionbox}
\begin{mnemonicbox}
``As Frequency/Phase States Quadruple, Bandwidth
Efficiency Doubles''

\end{mnemonicbox}
\subsection*{Question 3(b) OR [4
marks]}\label{q3b}

\textbf{Draw and explain the constellation diagram of 16-QAM.}

\begin{solutionbox}

\textbf{16-QAM (Quadrature Amplitude Modulation)}: A modulation
technique that combines amplitude and phase modulation, where each
symbol represents 4 bits.

\textbf{Constellation Diagram}:

\begin{verbatim}
      Q
      \^{}
      |   o   o   o   o
      |
      |   o   o   o   o
      |
{-{-}{-}{-}{-}{-}|{-}{-}{-}{-}{-}{-}{-}{-}{-}{-}{-}{-}{-}{-}{-} I}
      |   o   o   o   o
      |
      |   o   o   o   o
      |
\end{verbatim}

\textbf{Explanation}:

\begin{itemize}
\tightlist
\item
  \textbf{16 distinct states}: Each point represents a unique 4-bit
  combination
\item
  \textbf{Carries 4 bits per symbol}: log_{2}16 = 4
\item
  \textbf{Modulation parameters}: Both amplitude and phase are varied
\item
  \textbf{Symbol mapping}: Gray coding used to minimize bit errors
\end{itemize}

\end{solutionbox}
\begin{mnemonicbox}
``16 Quadrants Arranged in Matrix''

\end{mnemonicbox}
\subsection*{Question 3(c) OR [7
marks]}\label{q3c}

\textbf{Explain the Principle of MSK signal. Also draw constellation
diagram and waveforms of its.}

\begin{solutionbox}

\textbf{MSK (Minimum Shift Keying)}: A continuous phase FSK modulation
with a modulation index of 0.5, ensuring smooth phase transitions.

\textbf{Principle}:

\begin{itemize}
\tightlist
\item
  Special case of CPFSK (Continuous Phase FSK)
\item
  Frequency separation exactly equals half the bit rate
\item
  Maintains continuous phase, avoiding abrupt transitions
\item
  Modulation index h = 0.5
\end{itemize}

\textbf{Constellation Diagram}:

\begin{verbatim}
      Q
      \^{}
      |       o
      |     /   {}
      |    /     { }
      |   o       o
{-{-}{-}{-}{-}{-}|{-}{-}{-}{-}{-}{-}{-}{-}{-}{-}{-}{-}{-}{-}{-} I}
      |   o       o
      |    {     /}
      |     {   /}
      |       o
\end{verbatim}

\textbf{Waveforms}:

\begin{verbatim}
Data:      1      0      1      1      0
           \_      \_      \_      \_      \_      \_
           |      |      |      |      |      |
          \_|\_\_\_\_\_\_|\_\_\_\_\_\_|\_\_\_\_\_\_|\_\_\_\_\_\_|\_\_\_\_\_\_|\_\_\_

MSK:      /        {      /      /        }
         /          {    /      /          }
        /            {  /      /            }
       /              {/      /              }
      /                                        {}
\end{verbatim}

Key Features:

\begin{itemize}
\tightlist
\item
  \textbf{Constant envelope}: Better power efficiency
\item
  \textbf{Spectral efficiency}: Narrower bandwidth than BFSK
\item
  \textbf{Continuous phase}: Smoother transitions, reduced spectral
  spreading
\item
  \textbf{OQPSK relation}: Can be viewed as offset QPSK with sinusoidal
  pulse shaping
\end{itemize}

\end{solutionbox}
\begin{mnemonicbox}
``MSK Makes Smooth K-transitions''

\end{mnemonicbox}
\subsection*{Question 4(a) [3 marks]}\label{q4a}

\textbf{Describe the procedure to troubleshoot the FDD multiplexing
circuit.}

\begin{solutionbox}

{\def\LTcaptype{none} % do not increment counter
\begin{longtable}[]{@{}
  >{\raggedright\arraybackslash}p{(\linewidth - 2\tabcolsep) * \real{0.1818}}
  >{\raggedright\arraybackslash}p{(\linewidth - 2\tabcolsep) * \real{0.8182}}@{}}
\toprule\noalign{}
\begin{minipage}[b]{\linewidth}\raggedright
Step
\end{minipage} & \begin{minipage}[b]{\linewidth}\raggedright
Troubleshooting Procedure
\end{minipage} \\
\midrule\noalign{}
\endhead
\bottomrule\noalign{}
\endlastfoot
\textbf{1. Signal Verification} & Check input signals at each frequency
band \\
\textbf{2. Filter Analysis} & Verify bandpass filters for each
channel \\
\textbf{3. Modulator Testing} & Test frequency translation in each
channel \\
\textbf{4. Power Levels} & Measure signal strength at input/output \\
\textbf{5. Isolation Check} & Test for cross-talk between channels \\
\end{longtable}
}

\begin{verbatim}
flowchart LR
    A[Start] {-{-} B[Check Input Signals]}
    B {-{-} C\{Signals OK?\}}
    C {-{-}|Yes| D[Test Filters]}
    C {-{-}|No| E[Fix Input Source]}
    D {-{-} F\{Filters OK?\}}
    F {-{-}|Yes| G[Test Modulators]}
    F {-{-}|No| H[Replace/Adjust Filters]}
\end{verbatim}

\end{solutionbox}
\begin{mnemonicbox}
``SFMPI'' - ``Signal, Filter, Modulator, Power,
Isolation''

\end{mnemonicbox}
\subsection*{Question 4(b) [4 marks]}\label{q4b}

\textbf{Compare E1 carrier with T1 carrier.}

\begin{solutionbox}

{\def\LTcaptype{none} % do not increment counter
\begin{longtable}[]{@{}
  >{\raggedright\arraybackslash}p{(\linewidth - 4\tabcolsep) * \real{0.3143}}
  >{\raggedright\arraybackslash}p{(\linewidth - 4\tabcolsep) * \real{0.3429}}
  >{\raggedright\arraybackslash}p{(\linewidth - 4\tabcolsep) * \real{0.3429}}@{}}
\toprule\noalign{}
\begin{minipage}[b]{\linewidth}\raggedright
Parameter
\end{minipage} & \begin{minipage}[b]{\linewidth}\raggedright
E1 Carrier
\end{minipage} & \begin{minipage}[b]{\linewidth}\raggedright
T1 Carrier
\end{minipage} \\
\midrule\noalign{}
\endhead
\bottomrule\noalign{}
\endlastfoot
\textbf{Standard} & European standard & North American standard \\
\textbf{Data Rate} & 2.048 Mbps & 1.544 Mbps \\
\textbf{Voice Channels} & 30 channels & 24 channels \\
\textbf{Time Slots} & 32 time slots (TS0, TS1-TS15, TS16, TS17-TS31) &
24 time slots + framing bit \\
\textbf{Signaling} & Channel 16 used for signaling & Robbed bit
signaling \\
\textbf{Frame Size} & 256 bits & 193 bits \\
\textbf{Bit Rate per Channel} & 64 kbps & 64 kbps \\
\end{longtable}
}

\end{solutionbox}
\begin{mnemonicbox}
``ET-DR'' - ``European Thirty, Double Rate''

\end{mnemonicbox}
\subsection*{Question 4(c) [7 marks]}\label{q4c}

\textbf{Explain CDMA technique in detail.}

\begin{solutionbox}

\textbf{CDMA (Code Division Multiple Access)}: A multiple access
technique where multiple users share the same frequency band
simultaneously by using unique spreading codes.

\begin{verbatim}
flowchart LR
    A[User Data] {-{-} B[Spreading]}
    C[Unique Code] {-{-} B}
    B {-{-} D[Transmission]}
    D {-{-} E[Despreading]}
    F[Same Code] {-{-} E}
    E {-{-} G[User Data Recovery]}
\end{verbatim}

{\def\LTcaptype{none} % do not increment counter
\begin{longtable}[]{@{}
  >{\raggedright\arraybackslash}p{(\linewidth - 2\tabcolsep) * \real{0.5000}}
  >{\raggedright\arraybackslash}p{(\linewidth - 2\tabcolsep) * \real{0.5000}}@{}}
\toprule\noalign{}
\begin{minipage}[b]{\linewidth}\raggedright
Key Feature
\end{minipage} & \begin{minipage}[b]{\linewidth}\raggedright
Description
\end{minipage} \\
\midrule\noalign{}
\endhead
\bottomrule\noalign{}
\endlastfoot
\textbf{Spreading Codes} & Unique orthogonal or pseudo-random codes
assigned to each user \\
\textbf{Process Gain} & Ratio of spread bandwidth to original
bandwidth \\
\textbf{Interference Rejection} & Users with different codes appear as
noise to each other \\
\textbf{Soft Handoff} & Mobile can communicate with multiple base
stations simultaneously \\
\textbf{Power Control} & Critical to solve near-far problem \\
\textbf{Capacity} & Not strictly limited by frequency, but by acceptable
noise level \\
\end{longtable}
}

\textbf{Working Principle}:

\begin{itemize}
\tightlist
\item
  Each bit is multiplied by a high-rate spreading code (chips)
\item
  Resulting signal occupies much wider bandwidth
\item
  Receiver uses same code to recover original data
\item
  Other signals appear as random noise, rejected by correlation
\end{itemize}

\end{solutionbox}
\begin{mnemonicbox}
``CUPS'' - ``Codes Uniquely Provide Separation''

\end{mnemonicbox}
\subsection*{Question 4(a) OR [3
marks]}\label{q4a}

\textbf{Write a short not on classification of multiplexing techniques.}

\begin{solutionbox}

\textbf{Multiplexing Techniques}: Methods to combine multiple signals
for transmission over a single medium.

{\def\LTcaptype{none} % do not increment counter
\begin{longtable}[]{@{}
  >{\raggedright\arraybackslash}p{(\linewidth - 4\tabcolsep) * \real{0.2308}}
  >{\raggedright\arraybackslash}p{(\linewidth - 4\tabcolsep) * \real{0.3846}}
  >{\raggedright\arraybackslash}p{(\linewidth - 4\tabcolsep) * \real{0.3846}}@{}}
\toprule\noalign{}
\begin{minipage}[b]{\linewidth}\raggedright
Type
\end{minipage} & \begin{minipage}[b]{\linewidth}\raggedright
Based On
\end{minipage} & \begin{minipage}[b]{\linewidth}\raggedright
Examples
\end{minipage} \\
\midrule\noalign{}
\endhead
\bottomrule\noalign{}
\endlastfoot
\textbf{Frequency Division Multiplexing (FDM)} & Frequency domain &
Radio broadcasting, cable TV \\
\textbf{Time Division Multiplexing (TDM)} & Time domain & Digital
telephone systems, GSM \\
\textbf{Code Division Multiplexing (CDM)} & Code domain & CDMA cellular
systems \\
\textbf{Wavelength Division Multiplexing (WDM)} & Wavelength domain &
Fiber optic communications \\
\textbf{Space Division Multiplexing (SDM)} & Spatial domain & MIMO
wireless systems \\
\end{longtable}
}

\begin{center}
\textbf{Mermaid Diagram (Code)}
\begin{verbatim}
{Shaded}
{Highlighting}[]
graph TD
    A[Multiplexing Techniques] {-{-}{} B[Frequency Division]}
    A {-{-}{} C[Time Division]}
    A {-{-}{} D[Code Division]}
    A {-{-}{} E[Wavelength Division]}
    A {-{-}{} F[Space Division]}
{Highlighting}
{Shaded}
\end{verbatim}
\end{center}

\end{solutionbox}
\begin{mnemonicbox}
``FTCWS'' - ``Five Techniques Create Wide Systems''

\end{mnemonicbox}
\subsection*{Question 4(b) OR [4
marks]}\label{q4b}

\textbf{Draw and explain block diagram of Time Division Multiplexing
technique (TDM).}

\begin{solutionbox}

\textbf{Time Division Multiplexing (TDM)}: A technique where multiple
signals share the same channel by allocating different time slots to
each signal.

\begin{verbatim}
flowchart LR
    A1[Input 1] {-{-} B1[Sampler 1]}
    A2[Input 2] {-{-} B2[Sampler 2]}
    A3[Input 3] {-{-} B3[Sampler 3]}
    A4[Input 4] {-{-} B4[Sampler 4]}
    B1 {-{-} C[Commutator]}
    B2 {-{-} C}
    B3 {-{-} C}
    B4 {-{-} C}
    C {-{-} D[TDM Channel]}
    D {-{-} E[Decommutator]}
    E {-{-} F1[Filter 1] {-}{-} G1[Output 1]}
    E {-{-} F2[Filter 2] {-}{-} G2[Output 2]}
    E {-{-} F3[Filter 3] {-}{-} G3[Output 3]}
    E {-{-} F4[Filter 4] {-}{-} G4[Output 4]}
\end{verbatim}

{\def\LTcaptype{none} % do not increment counter
\begin{longtable}[]{@{}
  >{\raggedright\arraybackslash}p{(\linewidth - 2\tabcolsep) * \real{0.5238}}
  >{\raggedright\arraybackslash}p{(\linewidth - 2\tabcolsep) * \real{0.4762}}@{}}
\toprule\noalign{}
\begin{minipage}[b]{\linewidth}\raggedright
Component
\end{minipage} & \begin{minipage}[b]{\linewidth}\raggedright
Function
\end{minipage} \\
\midrule\noalign{}
\endhead
\bottomrule\noalign{}
\endlastfoot
\textbf{Samplers} & Sample each input signal at rate \geq 2 \times highest
frequency \\
\textbf{Commutator} & Sequentially selects samples from each input
channel \\
\textbf{TDM Channel} & Carries the combined signal \\
\textbf{Decommutator} & Distributes received samples to appropriate
channels \\
\textbf{Filters} & Reconstruct original signals from samples \\
\end{longtable}
}

\end{solutionbox}
\begin{mnemonicbox}
``SCTDF'' - ``Sample, Combine, Transmit, Distribute,
Filter''

\end{mnemonicbox}
\subsection*{Question 4(c) OR [7
marks]}\label{q4c}

\textbf{Explain TDMA technique in detail.}

\begin{solutionbox}

\textbf{TDMA (Time Division Multiple Access)}: A channel access method
where multiple users share the same frequency channel by dividing it
into different time slots.

\begin{verbatim}
flowchart TD
    A[TDMA Frame] {-{-} B[Slot 1br /User 1]}
    A {-{-} C[Slot 2br /User 2]}
    A {-{-} D[Slot 3br /User 3]}
    A {-{-} E[Slot 4br /User 4]}
    A {-{-} F[Slot 5br /User 5]}
    A {-{-} G[Slot 6br /User 6]}
\end{verbatim}

{\def\LTcaptype{none} % do not increment counter
\begin{longtable}[]{@{}
  >{\raggedright\arraybackslash}p{(\linewidth - 2\tabcolsep) * \real{0.5000}}
  >{\raggedright\arraybackslash}p{(\linewidth - 2\tabcolsep) * \real{0.5000}}@{}}
\toprule\noalign{}
\begin{minipage}[b]{\linewidth}\raggedright
Key Feature
\end{minipage} & \begin{minipage}[b]{\linewidth}\raggedright
Description
\end{minipage} \\
\midrule\noalign{}
\endhead
\bottomrule\noalign{}
\endlastfoot
\textbf{Frame Structure} & Fixed-length frames divided into time
slots \\
\textbf{Guard Time} & Small time gaps between slots to prevent
overlap \\
\textbf{Synchronization} & Requires precise timing coordination \\
\textbf{Channel Utilization} & Each user gets entire bandwidth for short
duration \\
\textbf{Power Efficiency} & Transmitters operate intermittently, saving
power \\
\textbf{Capacity} & Limited by available time slots in frame \\
\end{longtable}
}

\textbf{Implementation Details}:

\begin{itemize}
\tightlist
\item
  Each user transmits in rapid bursts within assigned slot
\item
  Non-continuous transmission allows handsets to measure signal
  strengths of nearby cells
\item
  Used in GSM (8 slots per frame), DECT, satellite systems
\item
  Easily adapts to varying data rates by assigning multiple slots
\end{itemize}

\end{solutionbox}
\begin{mnemonicbox}
``TDMA Takes Distinct Moments for Access''

\end{mnemonicbox}
\subsection*{Question 5(a) [3 marks]}\label{q5a}

\textbf{Define probability and write it Significance of in
communication.}

\begin{solutionbox}

\textbf{Probability}: A measure of the likelihood of an event occurring,
expressed as a number between 0 and 1.

{\def\LTcaptype{none} % do not increment counter
\begin{longtable}[]{@{}
  >{\raggedright\arraybackslash}p{(\linewidth - 2\tabcolsep) * \real{0.7045}}
  >{\raggedright\arraybackslash}p{(\linewidth - 2\tabcolsep) * \real{0.2955}}@{}}
\toprule\noalign{}
\begin{minipage}[b]{\linewidth}\raggedright
Significance in Communication
\end{minipage} & \begin{minipage}[b]{\linewidth}\raggedright
Explanation
\end{minipage} \\
\midrule\noalign{}
\endhead
\bottomrule\noalign{}
\endlastfoot
\textbf{Reliability Analysis} & Calculating error probabilities and
system reliability \\
\textbf{Noise Performance} & Evaluating system performance in presence
of random noise \\
\textbf{Information Theory} & Foundation for Shannon's channel capacity
theorem \\
\textbf{Signal Detection} & Determining optimal detection thresholds \\
\end{longtable}
}

\end{solutionbox}
\begin{mnemonicbox}
``PRONIS'' - ``PRObability Numerically Indicates
Signal quality''

\end{mnemonicbox}
\subsection*{Question 5(b) [4 marks]}\label{q5b}

\textbf{Explain Huffman code with suitable example.}

\begin{solutionbox}

\textbf{Huffman Code}: A variable-length prefix coding algorithm that
assigns shorter codes to more frequent symbols.

\textbf{Example}: Consider symbols A, B, C, D with probabilities 0.4,
0.3, 0.2, 0.1

\textbf{Huffman Coding Process}:

\begin{center}
\textbf{Mermaid Diagram (Code)}
\begin{verbatim}
{Shaded}
{Highlighting}[]
graph LR
    A[A:0.4, B:0.3, C:0.2, D:0.1] {-{-}{} B[A:0.4, B:0.3, CD:0.3]}
    B {-{-}{} C[A:0.4, BCD:0.6]}
    C {-{-}{} D[ABCD:1.0]}
    D {-{-}{} E["A(0) | BCD(1)"]}
    E {-{-}{} F["A(0) | B(10) | CD(11)"]}
    F {-{-}{} G["A(0) | B(10) | C(110) | D(111)"]}
{Highlighting}
{Shaded}
\end{verbatim}
\end{center}

{\def\LTcaptype{none} % do not increment counter
\begin{longtable}[]{@{}lll@{}}
\toprule\noalign{}
Symbol & Probability & Huffman Code \\
\midrule\noalign{}
\endhead
\bottomrule\noalign{}
\endlastfoot
A & 0.4 & 0 \\
B & 0.3 & 10 \\
C & 0.2 & 110 \\
D & 0.1 & 111 \\
\end{longtable}
}

\textbf{Average Code Length} = 0.4\times1 + 0.3\times2 + 0.2\times3 + 0.1\times3 = 1.9
bits/symbol

\end{solutionbox}
\begin{mnemonicbox}
``HEMP'' - ``Huffman Encodes More Probable symbols
with shorter codes''

\end{mnemonicbox}
\subsection*{Question 5(c) [7 marks]}\label{q5c}

\textbf{Explain concept and key features of Internet of Things (IoT).}

\begin{solutionbox}

\textbf{Internet of Things (IoT)}: A network of physical objects
embedded with sensors, software, and connectivity that enables them to
collect and exchange data.

\begin{center}
\textbf{Mermaid Diagram (Code)}
\begin{verbatim}
{Shaded}
{Highlighting}[]
graph TD
    A[IoT Ecosystem] {-{-}{} B[Smart Devices]}
    A {-{-}{} C[Connectivity]}
    A {-{-}{} D[Data Analytics]}
    A {-{-}{} E[User Interface]}
    A {-{-}{} F[Security]}
    B {-{-}{} G[Sensors \& Actuators]}
    C {-{-}{} H[Protocols \& Standards]}
    D {-{-}{} I[Cloud Computing]}
    E {-{-}{} J[Apps \& Services]}
    F {-{-}{} K[Authentication \& Encryption]}
{Highlighting}
{Shaded}
\end{verbatim}
\end{center}

{\def\LTcaptype{none} % do not increment counter
\begin{longtable}[]{@{}
  >{\raggedright\arraybackslash}p{(\linewidth - 2\tabcolsep) * \real{0.5000}}
  >{\raggedright\arraybackslash}p{(\linewidth - 2\tabcolsep) * \real{0.5000}}@{}}
\toprule\noalign{}
\begin{minipage}[b]{\linewidth}\raggedright
Key Feature
\end{minipage} & \begin{minipage}[b]{\linewidth}\raggedright
Description
\end{minipage} \\
\midrule\noalign{}
\endhead
\bottomrule\noalign{}
\endlastfoot
\textbf{Connectivity} & Devices connected to internet/each other via
various protocols (Wi-Fi, Bluetooth, LPWAN, 5G) \\
\textbf{Sensing Capability} & Ability to detect physical parameters
through sensors \\
\textbf{Intelligence} & Data processing at device (edge) or cloud
level \\
\textbf{Interoperability} & Ability to work across different platforms
and systems \\
\textbf{Automation} & Autonomous functioning without human
intervention \\
\textbf{Scalability} & Ability to handle growth in number of connected
devices \\
\end{longtable}
}

\textbf{Applications}:

\begin{itemize}
\tightlist
\item
  Smart homes (thermostats, security systems)
\item
  Healthcare (wearable devices, remote monitoring)
\item
  Industrial automation (predictive maintenance)
\item
  Smart cities (traffic management, waste management)
\item
  Agriculture (precision farming, livestock monitoring)
\end{itemize}

\end{solutionbox}
\begin{mnemonicbox}
``CSIA'' - ``Connect, Sense, Interpret, Automate''

\end{mnemonicbox}
\subsection*{Question 5(a) OR [3
marks]}\label{q5a}

\textbf{Define Channel Capacity in terms of SNR and its importance in
communication.}

\begin{solutionbox}

\textbf{Channel Capacity}: Maximum rate at which information can be
transmitted over a communication channel with arbitrarily small error
probability.

\textbf{Shannon's Channel Capacity Formula}: C = B \times log_{2}(1 + SNR)

Where:

\begin{itemize}
\tightlist
\item
  C = Channel capacity in bits per second
\item
  B = Bandwidth in Hertz
\item
  SNR = Signal-to-Noise Ratio
\end{itemize}

{\def\LTcaptype{none} % do not increment counter
\begin{longtable}[]{@{}
  >{\raggedright\arraybackslash}p{(\linewidth - 2\tabcolsep) * \real{0.6829}}
  >{\raggedright\arraybackslash}p{(\linewidth - 2\tabcolsep) * \real{0.3171}}@{}}
\toprule\noalign{}
\begin{minipage}[b]{\linewidth}\raggedright
Importance in Communication
\end{minipage} & \begin{minipage}[b]{\linewidth}\raggedright
Explanation
\end{minipage} \\
\midrule\noalign{}
\endhead
\bottomrule\noalign{}
\endlastfoot
\textbf{Performance Limit} & Sets theoretical maximum data rate for
error-free transmission \\
\textbf{System Design} & Guides selection of modulation, coding
schemes \\
\textbf{Bandwidth Efficiency} & Shows tradeoff between bandwidth and
SNR \\
\textbf{Link Budget Analysis} & Helps determine required transmit
power \\
\end{longtable}
}

\end{solutionbox}
\begin{mnemonicbox}
``CBLSN'' - ``Capacity equals Bandwidth times Log of
Signal-to-Noise ratio''

\end{mnemonicbox}
\subsection*{Question 5(b) OR [4
marks]}\label{q5b}

\textbf{Explain Shanon Fano code with suitable example.}

\begin{solutionbox}

\textbf{Shannon-Fano Coding}: A technique that assigns variable-length
codes to symbols based on their probabilities by recursively dividing
the set of symbols into two subsets with approximately equal
probabilities.

\textbf{Example}: Consider symbols A, B, C, D with probabilities 0.4,
0.3, 0.2, 0.1

\textbf{Shannon-Fano Procedure}:

\begin{enumerate}
\tightlist
\item
  Sort symbols by probability: A(0.4), B(0.3), C(0.2), D(0.1)
\item
  Divide into groups with nearly equal probability:

  \begin{itemize}
  \tightlist
  \item
    Group 1: A(0.4) - assigned `0'
  \item
    Group 2: B(0.3), C(0.2), D(0.1) = 0.6 - assigned `1'
  \end{itemize}
\item
  Recursively divide Group 2:

  \begin{itemize}
  \tightlist
  \item
    Group 2.1: B(0.3) - assigned `10'
  \item
    Group 2.2: C(0.2), D(0.1) = 0.3 - assigned `11'
  \end{itemize}
\item
  Divide Group 2.2:

  \begin{itemize}
  \tightlist
  \item
    C(0.2) - assigned `110'
  \item
    D(0.1) - assigned `111'
  \end{itemize}
\end{enumerate}

{\def\LTcaptype{none} % do not increment counter
\begin{longtable}[]{@{}lll@{}}
\toprule\noalign{}
Symbol & Probability & Shannon-Fano Code \\
\midrule\noalign{}
\endhead
\bottomrule\noalign{}
\endlastfoot
A & 0.4 & 0 \\
B & 0.3 & 10 \\
C & 0.2 & 110 \\
D & 0.1 & 111 \\
\end{longtable}
}

\textbf{Average Code Length} = 0.4\times1 + 0.3\times2 + 0.2\times3 + 0.1\times3 = 1.9
bits/symbol

\end{solutionbox}
\begin{mnemonicbox}
``SFDS'' - ``Shannon Fano Divides Symbolsets''

\end{mnemonicbox}
\subsection*{Question 5(c) OR [7
marks]}\label{q5c}

\textbf{Draw and explain block diagram of Digital telephone exchange.}

\begin{solutionbox}

\textbf{Digital Telephone Exchange}: A system that connects telephone
calls by converting analog voice signals to digital form and switching
them through digital circuits.

\begin{verbatim}
flowchart LR
    A[Subscribers] {-{-} B["Digital Line Unitsbr /(DLU)"]}
    B {-{-} C["Line/Trunk Groupbr /(LTG)"]}
    C {-{-} D["Switching Networkbr /(SN)"]}
    D {-{-} E["Central Processorbr /(CP)"]}
    E {-{-} D}
    D {-{-} C}
    C {-{-} B}
    B {-{-} A}
    F[Operation \& Maintenance{br /Center] {-}{-} E}
\end{verbatim}

{\def\LTcaptype{none} % do not increment counter
\begin{longtable}[]{@{}
  >{\raggedright\arraybackslash}p{(\linewidth - 2\tabcolsep) * \real{0.4118}}
  >{\raggedright\arraybackslash}p{(\linewidth - 2\tabcolsep) * \real{0.5882}}@{}}
\toprule\noalign{}
\begin{minipage}[b]{\linewidth}\raggedright
Block
\end{minipage} & \begin{minipage}[b]{\linewidth}\raggedright
Function
\end{minipage} \\
\midrule\noalign{}
\endhead
\bottomrule\noalign{}
\endlastfoot
\textbf{Digital Line Units (DLU)} & Interface between subscriber lines
and exchange, perform A/D conversion, line coding \\
\textbf{Line/Trunk Group (LTG)} & Manages signaling,
multiplexes/demultiplexes subscriber channels \\
\textbf{Switching Network (SN)} & Core switching fabric, establishes
connection paths between channels \\
\textbf{Central Processor (CP)} & Controls all exchange operations, call
processing, routing decisions \\
\textbf{Operation \& Maintenance Center} & Monitors system performance,
fault detection, traffic analysis \\
\end{longtable}
}

\textbf{Key Features}:

\begin{itemize}
\tightlist
\item
  \textbf{Time Division Switching}: Connects different time slots
\item
  \textbf{Space Division Switching}: Connects different physical paths
\item
  \textbf{Stored Program Control}: Software-based call processing
\item
  \textbf{Common Channel Signaling}: Separate signaling channel (SS7)
\item
  \textbf{Non-blocking Architecture}: All calls can be connected
  simultaneously
\end{itemize}

\end{solutionbox}
\begin{mnemonicbox}
``DLSCO'' - ``Digital Lines Switch Calls Orderly''

\end{mnemonicbox}

\end{document}
