\documentclass[10pt,a4paper]{article}

% content/resources/templates/preamble.tex
\usepackage[margin=0.6in]{geometry}
\author{Milav Dabgar}
\usepackage{amsmath,amssymb,amsthm}
\usepackage{booktabs}
\usepackage{multirow}
\usepackage{xcolor}
\usepackage{tcolorbox}
\tcbuselibrary{breakable,skins}
\usepackage[colorlinks=true,linkcolor=blue]{hyperref}
\usepackage{titlesec}
\usepackage{enumitem}
\usepackage{tikz}
\usepackage{pgfplots}
\usepackage{circuitikz}
\usepackage[version=4]{mhchem}
\usepackage{longtable}
\usepackage{array}
\usepackage{float}
\usepackage{caption}
\usepackage{listings}

\lstset{
  basicstyle=\small\ttfamily,
  breaklines=true,
  breakatwhitespace=false,
  postbreak=\mbox{\textcolor{red}{$\hookrightarrow$}\space},
  float=false,
  numbers=left,
  numberstyle=\tiny\color{gray},
  numbersep=10pt,
  xleftmargin=2em,
  keywordstyle=\color{blue},
  commentstyle=\color{green!60!black},
  stringstyle=\color{purple},
  backgroundcolor=\color{gray!5},
  showstringspaces=false,
  tabsize=2,
  captionpos=b,
  keepspaces=true,
  columns=flexible
}

\pgfplotsset{compat=1.18}
\usetikzlibrary{shapes,arrows,positioning,calc,patterns,decorations.pathmorphing,decorations.markings,arrows.meta}

% Color scheme
\definecolor{headcolor}{RGB}{0,102,204}
\definecolor{keycolor}{RGB}{220,20,60}
\definecolor{solutioncolor}{RGB}{34,139,34}
\definecolor{mnemoniccolor}{RGB}{148,0,211}
\definecolor{codecolor}{RGB}{0,0,100}

% Spacing
\setlength{\parskip}{3pt}
\setlist[itemize]{nosep}
\setlist[enumerate]{nosep}

% Title formatting
\titleformat{\section}{\Large\bfseries\color{headcolor}}{\thesection}{1em}{}
\titleformat{\subsection}{\large\bfseries\color{headcolor}}{\thesubsection}{1em}{}

% Pandoc tightlist compatibility
\providecommand{\tightlist}{%
  \setlength{\itemsep}{0pt}\setlength{\parskip}{0pt}}

% Pandoc longtable compatibility
\newcounter{none}
\def\thenone{}


% content/resources/templates/gujarati-boxes.tex
\usepackage{fontspec}
\usepackage{polyglossia}

% Set Gujarati as main language (document is primarily in Gujarati)
% Note: gloss-gujarati.ldf doesn't exist in polyglossia, but it will use hyphenation patterns
\setdefaultlanguage{gujarati}
\setotherlanguage{english}

% Configure Gujarati font properly
% Use Language=Default to prevent polyglossia from trying to add language-specific features
% that don't exist for Gujarati, which causes "empty feature" warnings
\newfontfamily\gujaratifont[Script=Gujarati,AutoFakeBold=2.5,AutoFakeSlant=0.3]{Noto Sans Gujarati}
\setmainfont[Script=Gujarati,AutoFakeBold=2.5,AutoFakeSlant=0.3]{Noto Sans Gujarati}
% Use Noto Sans Gujarati for monospace to support Gujarati in text
\setmonofont[Scale=0.9]{Noto Sans Gujarati}

% Configure English to use the same font
\newfontfamily\englishfont[Script=Gujarati,AutoFakeBold=2.5,AutoFakeSlant=0.3]{Noto Sans Gujarati}

% Translations for polyglossia
\gappto\captionsgujarati{
  \renewcommand{\tablename}{કોષ્ટક}
  \renewcommand{\figurename}{આકૃતિ}
}

% Helper for TikZ nodes to ensure Gujarati font
\newcommand{\gu}[1]{{\gujaratifont #1}}

% Custom environments
\newtcolorbox{solutionbox}{
    breakable,
    enhanced,
    colback=solutioncolor!5!white,
    colframe=solutioncolor!75!black,
    fonttitle=\bfseries,
    title=જવાબ
}

\newtcolorbox{solutionboxnobreak}{
 colback=solutioncolor!5!white,
 colframe=solutioncolor!75!black,
 fonttitle=\bfseries,
 title=જવાબ
}

\newtcolorbox{keyformula}{
 breakable,
 enhanced,
 colback=keycolor!5!white,
 colframe=keycolor!75!black,
 fonttitle=\bfseries,
 title=રાસાયણિક સમીકરણ/સૂત્ર
}

\newtcolorbox{mnemonicbox}{
 breakable,
 enhanced,
 colback=mnemoniccolor!5!white,
 colframe=mnemoniccolor!75!black,
 fonttitle=\bfseries,
 title=મેમરી ટ્રીક
}


\begin{document}

\begin{center}
{\Huge\bfseries\color{headcolor} Subject Name (Gujarati)}\\[5pt]
{\LARGE 4341102 -- Summer 2025}\\[3pt]
{\large Semester 1 Study Material}\\[3pt]
{\normalsize\textit{Detailed Solutions and Explanations}}
\end{center}

\vspace{10pt}

\subsection*{પ્રશ્ન 1(અ) [3
ગુણ]}\label{uxaaauxab0uxab6uxaa8-1uxa85-3-uxa97uxaa3}

\textbf{ડિજિટલ કોમ્યુનિકેશન સિસ્ટમનો બ્લોક ડાયાગ્રામ દોરો.}

\begin{solutionbox}

\begin{center}
\textbf{Mermaid Diagram (Code)}
\begin{verbatim}
{Shaded}
{Highlighting}[]
graph LR
    A[માહિતી સ્રોત] {-{-}{} B[સ્રોત એન્કોડર]}
    B {-{-}{} C[ચેનલ એન્કોડર]}
    C {-{-}{} D[ડિજિટલ મોડ્યુલેટર]}
    D {-{-}{} E[ચેનલ]}
    E {-{-}{} F[ડિજિટલ ડિ{-}મોડ્યુલેટર]}
    F {-{-}{} G[ચેનલ ડિકોડર]}
    G {-{-}{} H[સ્રોત ડિકોડર]}
    H {-{-}{} I[માહિતી સિંક]}
    
    J[નોઇઝ સ્રોત] {-{-}{} E}
{Highlighting}
{Shaded}
\end{verbatim}
\end{center}

\textbf{મુખ્ય ઘટકો:}

\begin{itemize}
\tightlist
\item
  \textbf{માહિતી સ્રોત}: સંદેશ સિગ્નલ જનરેટ કરે છે
\item
  \textbf{સ્રોત એન્કોડર}: એનાલોગને ડિજિટલમાં કન્વર્ટ કરે છે
\item
  \textbf{ચેનલ એન્કોડર}: એરર કરેક્શન કોડ ઉમેરે છે
\item
  \textbf{ડિજિટલ મોડ્યુલેટર}: ડિજિટલ બિટ્સને એનાલોગ સિગ્નલમાં કન્વર્ટ કરે છે
\end{itemize}

\textbf{યાદગાર વાક્ય:} ``સ્રોત ચેનલ મોડ્યુલેટર ચેનલમાંથી ડિ-મોડ્યુલેટર ચેનલ સિંક સુધી
જાય છે''

\end{solutionbox}
\subsection*{પ્રશ્ન 1(બ) [4
ગુણ]}\label{uxaaauxab0uxab6uxaa8-1uxaac-4-uxa97uxaa3}

\textbf{ડિજિટલ કોમ્યુનિકેશન સિસ્ટમના ટ્રાન્સમીટર અને રીસીવરના કાર્યો લખો.}

\begin{solutionbox}

{\def\LTcaptype{none} % do not increment counter
\begin{longtable}[]{@{}ll@{}}
\toprule\noalign{}
ઘટક & કાર્ય \\
\midrule\noalign{}
\endhead
\bottomrule\noalign{}
\endlastfoot
\textbf{ટ્રાન્સમીટર} & માહિતી સિગ્નલને ટ્રાન્સમિશન માટે યોગ્ય સ્વરૂપમાં કન્વર્ટ કરે
છે \\
\textbf{સ્રોત એન્કોડર} & એનાલોગ ટુ ડિજિટલ કન્વર્ઝન, સેમ્પલિંગ, ક્વાન્ટાઇઝેશન \\
\textbf{ચેનલ એન્કોડર} & એરર ડિટેક્શન અને કરેક્શન કોડિંગ \\
\textbf{ડિજિટલ મોડ્યુલેટર} & ડિજિટલ બિટ્સને એનાલોગ વેવફોર્મમાં કન્વર્ટ કરે છે \\
\end{longtable}
}

{\def\LTcaptype{none} % do not increment counter
\begin{longtable}[]{@{}ll@{}}
\toprule\noalign{}
ઘટક & કાર્ય \\
\midrule\noalign{}
\endhead
\bottomrule\noalign{}
\endlastfoot
\textbf{રીસીવર} & પ્રાપ્ત સિગ્નલમાંથી મૂળ માહિતી પુનઃપ્રાપ્ત કરે છે \\
\textbf{ડિજિટલ ડિ-મોડ્યુલેટર} & પ્રાપ્ત એનાલોગ સિગ્નલને ડિજિટલ બિટ્સમાં કન્વર્ટ કરે
છે \\
\textbf{ચેનલ ડિકોડર} & એરર ડિટેક્શન અને કરેક્શન \\
\textbf{સ્રોત ડિકોડર} & ડિજિટલ ટુ એનાલોગ કન્વર્ઝન \\
\end{longtable}
}

\textbf{મુખ્ય કાર્યો:}

\begin{itemize}
\tightlist
\item
  \textbf{સિગ્નલ પ્રોસેસિંગ}: એન્કોડિંગ, મોડ્યુલેશન, ફિલ્ટરિંગ
\item
  \textbf{એરર કન્ટ્રોલ}: ટ્રાન્સમિશન એરર્સનું ડિટેક્શન અને કરેક્શન
\item
  \textbf{સિગ્નલ રિકવરી}: રીસીવર પર ડિ-મોડ્યુલેશન અને ડિકોડિંગ
\end{itemize}

\textbf{યાદગાર વાક્ય:} ``ટ્રાન્સમીટર એન્કોડ કરી મોડ્યુલેટ કરે, રીસીવર ડિ-મોડ્યુલેટ
કરી ડિકોડ કરે''

\end{solutionbox}
\subsection*{પ્રશ્ન 1(ક) [7
ગુણ]}\label{uxaaauxab0uxab6uxaa8-1uxa95-7-uxa97uxaa3}

\textbf{વ્યાખ્યા કરો અને ઉદાહરણ સાથે સમજાવો: કન્ટિન્યુઅસ ટાઇમ અને ડિસક્રીટ ટાઇમ
સિગ્નલ્સ, રીઅલ અને કોમ્પ્લેક્સ સિગ્નલ્સ તથા ઇવન અને ઓડ સિગ્નલ્સ.}

\begin{solutionbox}

{\def\LTcaptype{none} % do not increment counter
\begin{longtable}[]{@{}
  >{\raggedright\arraybackslash}p{(\linewidth - 4\tabcolsep) * \real{0.4848}}
  >{\raggedright\arraybackslash}p{(\linewidth - 4\tabcolsep) * \real{0.2424}}
  >{\raggedright\arraybackslash}p{(\linewidth - 4\tabcolsep) * \real{0.2727}}@{}}
\toprule\noalign{}
\begin{minipage}[b]{\linewidth}\raggedright
સિગ્નલનો પ્રકાર
\end{minipage} & \begin{minipage}[b]{\linewidth}\raggedright
વ્યાખ્યા
\end{minipage} & \begin{minipage}[b]{\linewidth}\raggedright
ઉદાહરણ
\end{minipage} \\
\midrule\noalign{}
\endhead
\bottomrule\noalign{}
\endlastfoot
\textbf{કન્ટિન્યુઅસ ટાઇમ} & તમામ સમય વેલ્યુઝ માટે વ્યાખ્યાયિત સિગ્નલ & x(t) =
sin(2πt) \\
\textbf{ડિસક્રીટ ટાઇમ} & ફક્ત ચોક્કસ સમય ક્ષણોએ જ વ્યાખ્યાયિત સિગ્નલ & x[n]
= sin(2πn/8) \\
\textbf{રીઅલ સિગ્નલ} & ફક્ત વાસ્તવિક વેલ્યુઝ ધરાવતું સિગ્નલ & x(t) = 5cos(t) \\
\textbf{કોમ્પ્લેક્સ સિગ્નલ} & વાસ્તવિક અને કાલ્પનિક ભાગો ધરાવતું સિગ્નલ & x(t) = 3
+ j4sin(t) \\
\end{longtable}
}

\textbf{ઇવન અને ઓડ સિગ્નલ્સ:}

\begin{center}
\textbf{Mermaid Diagram (Code)}
\begin{verbatim}
{Shaded}
{Highlighting}[]
graph LR
    A["સિગ્નલ x[n]"] {-{-}{} B\{"x[{-}n] ચકાસો"\}}
    B {-{-}{} C["x[n] = x[{-}n]{}br/{}ઇવન સિગ્નલ"]}
    B {-{-}{} D["x[n] = {-}x[{-}n]{}br/{}ઓડ સિગ્નલ"]}
    
    C {-{-}{} E["ઉદાહરણ: cos[n]"]}
    D {-{-}{} F["ઉદાહરણ: sin[n]"]}
{Highlighting}
{Shaded}
\end{verbatim}
\end{center}

\textbf{ગુણધર્મો:}

\begin{itemize}
\tightlist
\item
  \textbf{ઇવન સિગ્નલ}: y-અક્ષની આસપાસ સપ્રમાણ, x(t) = x(-t)
\item
  \textbf{ઓડ સિગ્નલ}: મૂળબિંદુની આસપાસ વિરોધી-સપ્રમાણ, x(t) = -x(-t)
\item
  \textbf{કોમ્પ્લેક્સ સિગ્નલ}: z(t) = x(t) + jy(t)
\item
  \textbf{ડિસક્રીટ સિગ્નલ}: કન્ટિન્યુઅસ સિગ્નલનું સેમ્પલ કરેલું સ્વરૂપ
\end{itemize}

\textbf{યાદગાર વાક્ય:} ``કન્ટિન્યુઅસ સર્વત્ર, ડિસક્રીટ ચોક્કસ, રીઅલ સાદું, કોમ્પ્લેક્સ
મિશ્રિત''

\end{solutionbox}
\subsection*{પ્રશ્ન 1(ક અથવા) [7
ગુણ]}\label{uxaaauxab0uxab6uxaa8-1uxa95-uxa85uxaa5uxab5-7-uxa97uxaa3}

\textbf{વ્યાખ્યા કરો અને ઉદાહરણ સાથે સમજાવો: યુનિટ સ્ટેપ ફંકશન, યુનિટ ઇમ્પલ્સ ફંકશન
અને યુનિટ રેમ્પ ફંકશન}

\begin{solutionbox}

{\def\LTcaptype{none} % do not increment counter
\begin{longtable}[]{@{}lll@{}}
\toprule\noalign{}
ફંકશન & વ્યાખ્યા & ગાણિતિક સ્વરૂપ \\
\midrule\noalign{}
\endhead
\bottomrule\noalign{}
\endlastfoot
\textbf{યુનિટ સ્ટેપ} & u(t) = t\geq0 માટે 1, t\textless0 માટે 0 & u(t) = t\geq0 માટે
1 \\
\textbf{યુનિટ ઇમ્પલ્સ} & δ(t) = t=0 માટે \infty, અન્યત્ર 0 & \intδ(t)dt = 1 \\
\textbf{યુનિટ રેમ્પ} & r(t) = t\geq0 માટે t, t\textless0 માટે 0 & r(t) =
t·u(t) \\
\end{longtable}
}

\begin{verbatim}
Unit Step Function:        Unit Impulse Function:      Unit Ramp Function:
                          
    1 |{-{-}{-}{-}                    |                           |  /}
      |                         | |                         | /
    0 |\_\_\_\_                   0 |\_|\_\_\_\_                   0 |/\_\_\_\_
      0    t                     0    t                     0    t
\end{verbatim}

\textbf{ઉપયોગો:}

\begin{itemize}
\tightlist
\item
  \textbf{યુનિટ સ્ટેપ}: સ્વિચ ઓપરેશન્સ, સિસ્ટમ રિસ્પોન્સ વિશ્લેષણ
\item
  \textbf{યુનિટ ઇમ્પલ્સ}: સિસ્ટમ ઇમ્પલ્સ રિસ્પોન્સ, કોન્વોલ્યુશન
\item
  \textbf{યુનિટ રેમ્પ}: સિસ્ટમ રેમ્પ રિસ્પોન્સ, ઇન્ટિગ્રેશન
\end{itemize}

\textbf{ગુણધર્મો:}

\begin{itemize}
\tightlist
\item
  \textbf{સ્ટેપ}: રેમ્પનો વ્યુત્પન્ન, ઇમ્પલ્સનો સંકલન
\item
  \textbf{ઇમ્પલ્સ}: સ્ટેપ ફંકશનનો વ્યુત્પન્ન
\item
  \textbf{રેમ્પ}: સ્ટેપ ફંકશનનો સંકલન
\end{itemize}

\textbf{યાદગાર વાક્ય:} ``સ્ટેપ અચાનક, ઇમ્પલ્સ તાત્કાલિક, રેમ્પ વધતું''

\end{solutionbox}
\subsection*{પ્રશ્ન 2(અ) [3
ગુણ]}\label{uxaaauxab0uxab6uxaa8-2uxa85-3-uxa97uxaa3}

\textbf{વ્યાખ્યાયિત કરો: બિટ રેટ, બોડ રેટ અને બેન્ડવિડ્થ.}

\begin{solutionbox}

{\def\LTcaptype{none} % do not increment counter
\begin{longtable}[]{@{}lll@{}}
\toprule\noalign{}
પેરામીટર & વ્યાખ્યા & એકમ \\
\midrule\noalign{}
\endhead
\bottomrule\noalign{}
\endlastfoot
\textbf{બિટ રેટ} & પ્રતિ સેકન્ડે ટ્રાન્સમિટ થતી બિટ્સની સંખ્યા & bps (બિટ્સ પર
સેકન્ડ) \\
\textbf{બોડ રેટ} & પ્રતિ સેકન્ડે સિગ્નલ ફેરફારોની સંખ્યા & Baud (સિમ્બોલ્સ પર
સેકન્ડ) \\
\textbf{બેન્ડવિડ્થ} & સિગ્નલમાં ફ્રીક્વેન્સીઝની રેન્જ & Hz (હર્ટ્ઝ) \\
\end{longtable}
}

\textbf{સંબંધ:}

\begin{itemize}
\tightlist
\item
  બિટ રેટ = બોડ રેટ \times log_{2}(M)
\item
  M = સિગ્નલ લેવલ્સની સંખ્યા
\item
  બેન્ડવિડ્થ ∝ બોડ રેટ
\end{itemize}

\textbf{મુખ્ય મુદ્દાઓ:}

\begin{itemize}
\tightlist
\item
  \textbf{ઊંચો બિટ રેટ}: વધુ ડેટા ટ્રાન્સમિશન
\item
  \textbf{બોડ રેટ}: સિમ્બોલ ટ્રાન્સમિશન રેટ
\item
  \textbf{બેન્ડવિડ્થ}: કબજામાં લેવાયેલું ફ્રીક્વેન્સી સ્પેક્ટ્રમ
\end{itemize}

\textbf{યાદગાર વાક્ય:} ``બિટ્સ બોડ બેન્ડવિડ્થ - ડેટા સિમ્બોલ ફ્રીક્વેન્સી''

\end{solutionbox}
\subsection*{પ્રશ્ન 2(બ) [4
ગુણ]}\label{uxaaauxab0uxab6uxaa8-2uxaac-4-uxa97uxaa3}

\textbf{એનર્જી અને પાવર સિગ્નલ સમજાવો.}

\begin{solutionbox}

{\def\LTcaptype{none} % do not increment counter
\begin{longtable}[]{@{}lll@{}}
\toprule\noalign{}
સિગ્નલનો પ્રકાર & વ્યાખ્યા & ગાણિતિક સ્વરૂપ \\
\midrule\noalign{}
\endhead
\bottomrule\noalign{}
\endlastfoot
\textbf{એનર્જી સિગ્નલ} & મર્યાદિત એનર્જી, ઝીરો એવરેજ પાવર & E = \int \\
\textbf{પાવર સિગ્નલ} & મર્યાદિત એવરેજ પાવર, અનંત એનર્જી & P = lim(T\rightarrow\infty) 1/T
\int \\
\end{longtable}
}

\textbf{વર્ગીકરણ:}

\begin{center}
\textbf{Mermaid Diagram (Code)}
\begin{verbatim}
{Shaded}
{Highlighting}[]
graph LR
    A[સિગ્નલ] {-{-}{} B\{એનર્જી મર્યાદિત છે?\}}
    B {-{-}{}|હા| C[એનર્જી સિગ્નલ{}br/{}P = 0]}
    B {-{-}{}|ના| D\{પાવર મર્યાદિત છે?\}}
    D {-{-}{}|હા| E[પાવર સિગ્નલ{}br/{}E = ]}
    D {-{-}{}|ના| F[ન તો એનર્જી{}br/{}ન પાવર]}
{Highlighting}
{Shaded}
\end{verbatim}
\end{center}

\textbf{ઉદાહરણો:}

\begin{itemize}
\tightlist
\item
  \textbf{એનર્જી સિગ્નલ}: ઘટતું exponential સિગ્નલ e\^{}(-t)u(t)
\item
  \textbf{પાવર સિગ્નલ}: Sinusoidal સિગ્નલ sin(ωt)
\item
  \textbf{બંનેમાંથી કોઈ નહીં}: રેમ્પ સિગ્નલ t·u(t)
\end{itemize}

\textbf{ગુણધર્મો:}

\begin{itemize}
\tightlist
\item
  એનર્જી અને પાવર સિગ્નલ્સ એકબીજાને બાકાત રાખે છે
\item
  આવર્તિ સિગ્નલ્સ સામાન્ય રીતે પાવર સિગ્નલ્સ હોય છે
\item
  બિન-આવર્તિ મર્યાદિત અવધિના સિગ્નલ્સ એનર્જી સિગ્નલ્સ હોય છે
\end{itemize}

\textbf{યાદગાર વાક્ય:} ``એનર્જી સમાપ્ત, પાવર ચાલુ''

\end{solutionbox}
\subsection*{પ્રશ્ન 2(ક) [7
ગુણ]}\label{uxaaauxab0uxab6uxaa8-2uxa95-7-uxa97uxaa3}

\textbf{ASK, FSK અને PSK મોડ્યુલેશન ટેકનિકો વચ્ચે સરખામણી કરો અને તેના વેવફોર્મ્સ
દોરો.}

\begin{solutionbox}

{\def\LTcaptype{none} % do not increment counter
\begin{longtable}[]{@{}
  >{\raggedright\arraybackslash}p{(\linewidth - 6\tabcolsep) * \real{0.3750}}
  >{\raggedright\arraybackslash}p{(\linewidth - 6\tabcolsep) * \real{0.2083}}
  >{\raggedright\arraybackslash}p{(\linewidth - 6\tabcolsep) * \real{0.2083}}
  >{\raggedright\arraybackslash}p{(\linewidth - 6\tabcolsep) * \real{0.2083}}@{}}
\toprule\noalign{}
\begin{minipage}[b]{\linewidth}\raggedright
પેરામીટર
\end{minipage} & \begin{minipage}[b]{\linewidth}\raggedright
ASK
\end{minipage} & \begin{minipage}[b]{\linewidth}\raggedright
FSK
\end{minipage} & \begin{minipage}[b]{\linewidth}\raggedright
PSK
\end{minipage} \\
\midrule\noalign{}
\endhead
\bottomrule\noalign{}
\endlastfoot
\textbf{પૂરું નામ} & Amplitude Shift Keying & Frequency Shift Keying &
Phase Shift Keying \\
\textbf{બદલાતો પેરામીટર} & એમ્પ્લિટ્યુડ & ફ્રીક્વેન્સી & ફેઝ \\
\textbf{બેન્ડવિડ્થ} & સાંકડી & પહોળી & સાંકડી \\
\textbf{નોઇઝ ઇમ્યુનિટી} & નબળી & સારી & શ્રેષ્ઠ \\
\textbf{પાવર એફિશિયન્સી} & નબળી & સારી & શ્રેષ્ઠ \\
\textbf{અમલીકરણ} & સરળ & મધ્યમ & જટિલ \\
\end{longtable}
}

\begin{verbatim}
ASK Waveform:
Data:    1    0    1    1    0
        \_\_\_       \_\_\_  \_\_\_      
       |   |     |   ||   |     
    \_\_\_|   |\_\_\_\_\_|   ||   |\_\_\_\_\_
    
FSK Waveform:
       {              }
      {                   }
    {                   }
    
PSK Waveform:
       \_\_\_       \_\_\_ \_\_\_      
      |   |     |   |   |     
    \_\_|   |\_\_\_\_\_|   |   |\_\_\_\_\_
      phase shift at data change
\end{verbatim}

\textbf{ઉપયોગો:}

\begin{itemize}
\tightlist
\item
  \textbf{ASK}: ઓપ્ટિકલ કોમ્યુનિકેશન, સરળ રેડિયો સિસ્ટમ્સ
\item
  \textbf{FSK}: ટેલિફોન મોડેમ્સ, રેડિયો સિસ્ટમ્સ
\item
  \textbf{PSK}: સેટેલાઇટ કોમ્યુનિકેશન, વાયરલેસ સિસ્ટમ્સ
\end{itemize}

\textbf{ફાયદાઓ:}

\begin{itemize}
\tightlist
\item
  \textbf{ASK}: સરળ અમલીકરણ, ઓછી કિંમત
\item
  \textbf{FSK}: સારી નોઇઝ પર્ફોર્મન્સ, કોન્સ્ટન્ટ એન્વેલોપ
\item
  \textbf{PSK}: શ્રેષ્ઠ નોઇઝ પર્ફોર્મન્સ, બેન્ડવિડ્થ એફિશિયન્ટ
\end{itemize}

\textbf{યાદગાર વાક્ય:} ``ASK એમ્પ્લિટ્યુડ, FSK ફ્રીક્વેન્સી, PSK ફેઝ''

\end{solutionbox}
\subsection*{પ્રશ્ન 2(અ અથવા) [3
ગુણ]}\label{uxaaauxab0uxab6uxaa8-2uxa85-uxa85uxaa5uxab5-3-uxa97uxaa3}

\textbf{8-બિટ જનરેટરમાંથી સિગ્નલ જનરેટરનો બિટ દર 1600 bps છે. સિગ્નલનો બોડ રેટ
ની ગणતરી કરો.}

\begin{solutionbox}

\textbf{આપેલ:}

\begin{itemize}
\tightlist
\item
  બિટ રેટ = 1600 bps
\item
  પ્રતિ સિમ્બોલ બિટ્સની સંખ્યા = 8 બિટ્સ
\end{itemize}

\textbf{સૂત્ર:} બોડ રેટ = બિટ રેટ / પ્રતિ સિમ્બોલ બિટ્સની સંખ્યા

\textbf{ગણતરી:} બોડ રેટ = 1600 bps / 8 બિટ્સ બોડ રેટ = 200 Baud

\textbf{પરિણામ:} સિગ્નલનો બોડ રેટ \textbf{200 Baud} છે.

\textbf{સમજૂતી:}

\begin{itemize}
\tightlist
\item
  દરેક સિમ્બોલ 8 બિટ્સની માહિતી ધરાવે છે
\item
  પ્રતિ સેકન્ડે 1600 બિટ્સ \div પ્રતિ સિમ્બોલ 8 બિટ્સ = પ્રતિ સેકન્ડે 200 સિમ્બોલ્સ
\item
  તેથી, બોડ રેટ = 200 Baud
\end{itemize}

\textbf{યાદગાર વાક્ય:} ``બિટ રેટને બિટ્સ પર સિમ્બોલથી ભાગવાથી બોડ મળે''

\end{solutionbox}
\subsection*{પ્રશ્ન 2(બ અથવા) [4
ગુણ]}\label{uxaaauxab0uxab6uxaa8-2uxaac-uxa85uxaa5uxab5-4-uxa97uxaa3}

\textbf{શોધો કે સિગ્નલ્સ ઇવન અથવા ઓડ છે કે નહીં:} \textbf{1. x(t) =
e\^{}(-5t)} \textbf{2. x(t) = sin 2t} \textbf{3. x(t) = cos 5t}

\begin{solutionbox}

{\def\LTcaptype{none} % do not increment counter
\begin{longtable}[]{@{}
  >{\raggedright\arraybackslash}p{(\linewidth - 6\tabcolsep) * \real{0.2121}}
  >{\raggedright\arraybackslash}p{(\linewidth - 6\tabcolsep) * \real{0.3636}}
  >{\raggedright\arraybackslash}p{(\linewidth - 6\tabcolsep) * \real{0.2424}}
  >{\raggedright\arraybackslash}p{(\linewidth - 6\tabcolsep) * \real{0.1818}}@{}}
\toprule\noalign{}
\begin{minipage}[b]{\linewidth}\raggedright
સિગ્નલ
\end{minipage} & \begin{minipage}[b]{\linewidth}\raggedright
x(-t) ટેસ્ટ
\end{minipage} & \begin{minipage}[b]{\linewidth}\raggedright
પરિણામ
\end{minipage} & \begin{minipage}[b]{\linewidth}\raggedright
પ્રકાર
\end{minipage} \\
\midrule\noalign{}
\endhead
\bottomrule\noalign{}
\endlastfoot
x(t) = e\^{}(-5t) & x(-t) = e\^{}(5t) \neq x(t) \neq -x(t) & બંનેમાંથી કોઈ નહીં &
ન તો ઇવન ન ઓડ \\
x(t) = sin 2t & x(-t) = sin(-2t) = -sin 2t = -x(t) & -x(t) & \textbf{ઓડ
સિગ્નલ} \\
x(t) = cos 5t & x(-t) = cos(-5t) = cos 5t = x(t) & x(t) & \textbf{ઇવન
સિગ્નલ} \\
\end{longtable}
}

\textbf{ટેસ્ટ પ્રક્રિયા:}

\begin{enumerate}
\tightlist
\item
  \textbf{ઇવન સિગ્નલ ટેસ્ટ}: તપાસો કે x(t) = x(-t)
\item
  \textbf{ઓડ સિગ્નલ ટેસ્ટ}: તપાસો કે x(t) = -x(-t)
\end{enumerate}

\textbf{વપરાયેલ ગુણધર્મો:}

\begin{itemize}
\tightlist
\item
  \textbf{Exponential}: e\^{}(-at) ન તો ઇવન ન ઓડ છે (a \textgreater{} 0)
\item
  \textbf{Sine ફંકશન}: sin(-x) = -sin(x) \rightarrow ઓડ ફંકશન
\item
  \textbf{Cosine ફંકશન}: cos(-x) = cos(x) \rightarrow ઇવન ફંકશન
\end{itemize}

\textbf{પરિણામો:}

\begin{itemize}
\tightlist
\item
  \textbf{સિગ્નલ 1}: ન તો ઇવન ન ઓડ
\item
  \textbf{સિગ્નલ 2}: ઓડ સિગ્નલ
\item
  \textbf{સિગ્નલ 3}: ઇવન સિગ્નલ
\end{itemize}

\textbf{યાદગાર વાક્ય:} ``Cosine ઇવન, Sine ઓડ, Exponential બંનેમાંથી કોઈ
નહીં''

\end{solutionbox}
\subsection*{પ્રશ્ન 2(ક અથવા) [7
ગુણ]}\label{uxaaauxab0uxab6uxaa8-2uxa95-uxa85uxaa5uxab5-7-uxa97uxaa3}

\textbf{QPSK સિગ્નલનો સિદ્ધાંત સમજાવો. તેના મોડ્યુલેટર અને ડિ-મોડ્યુલેટરના બ્લોક
ડાયાગ્રામ દોરો. તેમજ તેના કોન્સ્ટેલેશન ડાયાગ્રામ અને વેવફોર્મ્સ દોરો.}

\begin{solutionbox}

\textbf{QPSK સિદ્ધાંત:} QPSK (Quadrature Phase Shift Keying) 2 બિટ્સ પર
સિમ્બોલ દર્શાવવા માટે ચાર અલગ ફેઝ સ્ટેટ્સનો ઉપયોગ કરે છે.

{\def\LTcaptype{none} % do not increment counter
\begin{longtable}[]{@{}llll@{}}
\toprule\noalign{}
બિટ્સ & ફેઝ & I & Q \\
\midrule\noalign{}
\endhead
\bottomrule\noalign{}
\endlastfoot
00 & 45^\circ & +1 & +1 \\
01 & 135^\circ & -1 & +1 \\
10 & -45^\circ & +1 & -1 \\
11 & -135^\circ & -1 & -1 \\
\end{longtable}
}

\textbf{QPSK મોડ્યુલેટર:}

\begin{center}
\textbf{Mermaid Diagram (Code)}
\begin{verbatim}
{Shaded}
{Highlighting}[]
graph LR
    A[ડેટા સ્ટ્રીમ] {-{-}{} B[સીરિયલ ટુ પેરેલલ]}
    B {-{-}{} C[I ચેનલ]}
    B {-{-}{} D[Q ચેનલ]}
    C {-{-}{} E[મિક્સર 1]}
    D {-{-}{} F[મિક્સર 2]}
    G["કેરિયર cos(ωt)"] {-{-}{} E}
    H["કેરિયર sin(ωt)"] {-{-}{} F}
    E {-{-}{} I[એડર]}
    F {-{-}{} I}
    I {-{-}{} J[QPSK આઉટપુટ]}
{Highlighting}
{Shaded}
\end{verbatim}
\end{center}

\textbf{કોન્સ્ટેલેશન ડાયાગ્રામ:}

\begin{verbatim}
        Q
        |
   01   |   00
  ({-1,1)| (1,1)}
        |
  {-{-}{-}{-}{-}{-}+{-}{-}{-}{-}{-}{-} I}
        |
  ({-1,{-}1)|(1,{-}1)}
   11   |   10
        |
\end{verbatim}

\textbf{QPSK ડિ-મોડ્યુલેટર:}

\begin{center}
\textbf{Mermaid Diagram (Code)}
\begin{verbatim}
{Shaded}
{Highlighting}[]
graph LR
    A[QPSK ઇનપુટ] {-{-}{} B[મિક્સર 1]}
    A {-{-}{} C[મિક્સર 2]}
    D["cos(ωt)"] {-{-}{} B}
    E["sin(ωt)"] {-{-}{} C}
    B {-{-}{} F[LPF]}
    C {-{-}{} G[LPF]}
    F {-{-}{} H[ડિસિઝન ડિવાઇસ]}
    G {-{-}{} I[ડિસિઝન ડિવાઇસ]}
    H {-{-}{} J[પેરેલલ ટુ સીરિયલ]}
    I {-{-}{} J}
    J {-{-}{} K[ડેટા આઉટપુટ]}
{Highlighting}
{Shaded}
\end{verbatim}
\end{center}

\textbf{ફાયદાઓ:}

\begin{itemize}
\tightlist
\item
  \textbf{બેન્ડવિડ્થ એફિશિયન્ટ}: પ્રતિ સિમ્બોલ 2 બિટ્સ
\item
  \textbf{સારી નોઇઝ પર્ફોર્મન્સ}: કોન્સ્ટન્ટ એન્વેલોપ
\item
  \textbf{વ્યાપક ઉપયોગ}: ડિજિટલ કોમ્યુનિકેશનમાં સ્ટાન્ડર્ડ
\end{itemize}

\textbf{ઉપયોગો:}

\begin{itemize}
\tightlist
\item
  સેટેલાઇટ કોમ્યુનિકેશન
\item
  ડિજિટલ TV બ્રોડકાસ્ટિંગ
\item
  વાયરલેસ કોમ્યુનિકેશન સિસ્ટમ્સ
\end{itemize}

\textbf{યાદગાર વાક્ય:} ``QPSK - ક્વાડ્રેચર ફેઝ, 2 બિટ્સ, 4 ફેઝ''

\end{solutionbox}
\subsection*{પ્રશ્ન 3(અ) [3
ગુણ]}\label{uxaaauxab0uxab6uxaa8-3uxa85-3-uxa97uxaa3}

\textbf{FSK મોડ્યુલેટરનો બ્લોક ડાયાગ્રામ દોરો}

\begin{solutionbox}

\begin{center}
\textbf{Mermaid Diagram (Code)}
\begin{verbatim}
{Shaded}
{Highlighting}[]
graph LR
    A[ડિજિટલ ડેટા] {-{-}{} B[સ્વિચ]}
    C[ઓસિલેટર 1{br/{}f1] {-}{-}{} B}
    D[ઓસિલેટર 2{br/{}f2] {-}{-}{} B}
    B {-{-}{} E[FSK આઉટપુટ]}
    
    F[ડેટા = 1] {-.{-}{} C}
    G[ડેટા = 0] {-.{-}{} D}
{Highlighting}
{Shaded}
\end{verbatim}
\end{center}

\textbf{ઘટકો:}

\begin{itemize}
\tightlist
\item
  \textbf{ડિજિટલ ડેટા ઇનપુટ}: બાઇનરી ડેટા સ્ટ્રીમ (0s અને 1s)
\item
  \textbf{બે ઓસિલેટર્સ}: બિટ `1' માટે f_{1}, બિટ `0' માટે f_{2}
\item
  \textbf{ઇલેક્ટ્રોનિક સ્વિચ}: ઇનપુટ બિટના આધારે ફ્રીક્વેન્સી પસંદ કરે છે
\item
  \textbf{FSK આઉટપુટ}: ફ્રીક્વેન્સી મોડ્યુલેટેડ સિગ્નલ
\end{itemize}

\textbf{કામગીરી:}

\begin{itemize}
\tightlist
\item
  \textbf{બિટ `1'}: સ્વિચ ઓસિલેટર 1 (ઊંચી ફ્રીક્વેન્સી) સાથે જોડાય છે
\item
  \textbf{બિટ `0'}: સ્વિચ ઓસિલેટર 2 (નીચી ફ્રીક્વેન્સી) સાથે જોડાય છે
\item
  \textbf{આઉટપુટ}: ડેટાના આધારે સતત ફ્રીક્વેન્સી બદલાતી રહે છે
\end{itemize}

\textbf{યાદગાર વાક્ય:} ``FSK - ડેટા કીઝના આધારે ફ્રીક્વેન્સી સ્વિચ''

\end{solutionbox}
\subsection*{પ્રશ્ન 3(બ) [4
ગુણ]}\label{uxaaauxab0uxab6uxaa8-3uxaac-4-uxa97uxaa3}

\textbf{PSK મોડ્યુલેટરનો બ્લોક ડાયાગ્રામ દોરો અને સમજાવો.}

\begin{solutionbox}

\begin{center}
\textbf{Mermaid Diagram (Code)}
\begin{verbatim}
{Shaded}
{Highlighting}[]
graph LR
    A[ડિજિટલ ડેટા] {-{-}{} B[બેલેન્સ્ડ મોડ્યુલેટર]}
    C["કેરિયર ઓસિલેટર{br/{}cos(ωt)"] {-}{-}{} B}
    B {-{-}{} D[PSK આઉટપુટ]}
    
    E[ડેટા = 1] {-.{-}{} F[0^ ફેઝ]}
    G[ડેટા = 0] {-.{-}{} H[180^ ફેઝ]}
{Highlighting}
{Shaded}
\end{verbatim}
\end{center}

\textbf{ઘટકો અને કાર્ય:}

{\def\LTcaptype{none} % do not increment counter
\begin{longtable}[]{@{}ll@{}}
\toprule\noalign{}
ઘટક & કાર્ય \\
\midrule\noalign{}
\endhead
\bottomrule\noalign{}
\endlastfoot
\textbf{ડિજિટલ ડેટા} & બાઇનરી ઇનપુટ સ્ટ્રીમ (0s અને 1s) \\
\textbf{કેરિયર ઓસિલેટર} & રેફરન્સ કેરિયર સિગ્નલ બનાવે છે \\
\textbf{બેલેન્સ્ડ મોડ્યુલેટર} & ડેટાને કેરિયર સાથે ગુણાકાર કરે છે \\
\textbf{PSK આઉટપુટ} & ફેઝ મોડ્યુલેટેડ સિગ્નલ \\
\end{longtable}
}

\textbf{કામગીરી:}

\begin{itemize}
\tightlist
\item
  \textbf{ડેટા `1'}: આઉટપુટ = +cos(ωt) (0^\circ ફેઝ)
\item
  \textbf{ડેટા `0'}: આઉટપુટ = -cos(ωt) (180^\circ ફેઝ)
\item
  \textbf{ફેઝ શિફ્ટ}: `1' અને `0' વચ્ચે 180^\circ તફાવત
\end{itemize}

\textbf{ગાણિતિક અભિવ્યક્તિ:}

\begin{itemize}
\tightlist
\item
  PSK સિગ્નલ: s(t) = A·d(t)·cos(ωt)
\item
  જ્યાં d(t) = `1' માટે +1, `0' માટે -1
\end{itemize}

\textbf{ફાયદાઓ:}

\begin{itemize}
\tightlist
\item
  \textbf{કોન્સ્ટન્ટ એન્વેલોપ}: બહેતર નોઇઝ ઇમ્યુનિટી
\item
  \textbf{બેન્ડવિડ્થ એફિશિયન્ટ}: ASK જેટલું જ બેન્ડવિડ્થ લે છે
\item
  \textbf{સરળ ડિટેક્શન}: કોહેરન્ટ ડિટેક્શન જરૂરી
\end{itemize}

\textbf{યાદગાર વાક્ય:} ``PSK - બેલેન્સ્ડ મોડ્યુલેટર કીનો ઉપયોગ કરીને ફેઝ શિફ્ટ''

\end{solutionbox}
\subsection*{પ્રશ્ન 3(ક) [7
ગુણ]}\label{uxaaauxab0uxab6uxaa8-3uxa95-7-uxa97uxaa3}

\textbf{ASK મોડ્યુલેટર અને ડિ-મોડ્યુલેટરના બ્લોક ડાયાગ્રામને વેવફોર્મ સાથે સમજાવો.}

\begin{solutionbox}

\textbf{ASK મોડ્યુલેટર:}

\begin{center}
\textbf{Mermaid Diagram (Code)}
\begin{verbatim}
{Shaded}
{Highlighting}[]
graph LR
    A[ડિજિટલ ડેટા] {-{-}{} B[ગુણાકાર]}
    C["કેરિયર cos(ωt)"] {-{-}{} B}
    B {-{-}{} D[ASK આઉટપુટ]}
{Highlighting}
{Shaded}
\end{verbatim}
\end{center}

\textbf{ASK ડિ-મોડ્યુલેટર:}

\begin{center}
\textbf{Mermaid Diagram (Code)}
\begin{verbatim}
{Shaded}
{Highlighting}[]
graph LR
    A[ASK ઇનપુટ] {-{-}{} B[ગુણાકાર] }
    C[લોકલ કેરિયર] {-{-}{} B}
    B {-{-}{} D[લો પાસ ફિલ્ટર]}
    D {-{-}{} E[ડિસિઝન ડિવાઇસ]}
    E {-{-}{} F[ડિજિટલ આઉટપુટ]}
    G[થ્રેશહોલ્ડ] {-{-}{} E}
{Highlighting}
{Shaded}
\end{verbatim}
\end{center}

\textbf{વેવફોર્મ્સ:}

\begin{verbatim}
Digital Data:
    1    0    1    1    0
   \_\_\_       \_\_\_  \_\_\_      
  |   |     |   ||   |     
\_\_|   |\_\_\_\_\_|   ||   |\_\_\_\_\_

Carrier Signal:
                 
            
                  

ASK Output:
            
                
                
\end{verbatim}

\textbf{મોડ્યુલેશન પ્રક્રિયા:}

{\def\LTcaptype{none} % do not increment counter
\begin{longtable}[]{@{}lll@{}}
\toprule\noalign{}
ડેટા બિટ & કેરિયર & ASK આઉટપુટ \\
\midrule\noalign{}
\endhead
\bottomrule\noalign{}
\endlastfoot
\textbf{`1'} & A·cos(ωt) & A·cos(ωt) \\
\textbf{`0'} & A·cos(ωt) & 0 \\
\end{longtable}
}

\textbf{ડિ-મોડ્યુલેશન પ્રક્રિયા:}

\begin{enumerate}
\tightlist
\item
  \textbf{ગુણાકાર}: ASK સિગ્નલ \times લોકલ કેરિયર
\item
  \textbf{લો પાસ ફિલ્ટરિંગ}: ઊંચી ફ્રીક્વેન્સી ઘટકો દૂર કરો
\item
  \textbf{ડિસિઝન}: થ્રેશહોલ્ડ સાથે સરખાવીને ડેટા પુનઃપ્રાપ્ત કરો
\end{enumerate}

\textbf{ઉપયોગો:}

\begin{itemize}
\tightlist
\item
  \textbf{ઓપ્ટિકલ કોમ્યુનિકેશન}: LED/લેઝર ઓન-ઓફ કીઇંગ
\item
  \textbf{સરળ રેડિયો સિસ્ટમ્સ}: AM રેડિયો મોડિફિકેશન
\item
  \textbf{શોર્ટ રેન્જ કોમ્યુનિકેશન}: IR રિમોટ કન્ટ્રોલ્સ
\end{itemize}

\textbf{ફાયદાઓ/નુકસાનો:}

{\def\LTcaptype{none} % do not increment counter
\begin{longtable}[]{@{}ll@{}}
\toprule\noalign{}
ફાયદાઓ & નુકસાનો \\
\midrule\noalign{}
\endhead
\bottomrule\noalign{}
\endlastfoot
સરળ અમલીકરણ & નબળી નોઇઝ પર્ફોર્મન્સ \\
ઓછી કિંમત & બેન્ડવિડ્થ અકુશળ \\
સરળ ડિટેક્શન & ફેડિંગ માટે સંવેદનશીલ \\
\end{longtable}
}

\textbf{યાદગાર વાક્ય:} ``ASK - એમ્પ્લિટ્યુડ સ્વિચ, ગુણાકાર અને ફિલ્ટર કી''

\end{solutionbox}
\subsection*{પ્રશ્ન 3(અ અથવા) [3
ગુણ]}\label{uxaaauxab0uxab6uxaa8-3uxa85-uxa85uxaa5uxab5-3-uxa97uxaa3}

\textbf{MSK નો સિદ્ધાંત લખો અને કોન્સ્ટેલેશન ડાયાગ્રામ દોરો.}

\begin{solutionbox}

\textbf{MSK સિદ્ધાંત:} MSK (Minimum Shift Keying) એ સતત-ફેઝ FSK નું એક સ્વરૂપ છે
જ્યાં ફ્રીક્વેન્સી ડેવિએશન બરાબર બિટ રેટનો અડધો છે.

\textbf{મુખ્ય ગુણધર્મો:}

\begin{itemize}
\tightlist
\item
  \textbf{સતત ફેઝ}: કોઈ ફેઝ અસાતત્યતા નથી
\item
  \textbf{ન્યૂનતમ ફ્રીક્વેન્સી વિભાજન}: Δf = Rb/2
\item
  \textbf{કોન્સ્ટન્ટ એન્વેલોપ}: નોનલીનિયર એમ્પ્લિફાયર્સ માટે સારું
\end{itemize}

\textbf{કોન્સ્ટેલેશન ડાયાગ્રામ:}

\begin{verbatim}
        Q
        |
●  (I=0,

Q=1)

        |
   ●{-{-}{-}{-}+{-}{-}{-}{-}●  I}
        |
●  (I=0,

Q={-1)}

        |
        
Points rotate continuously
between 1 on I and Q axes
\end{verbatim}

\textbf{ગાણિતિક રજૂઆત:}

\begin{itemize}
\tightlist
\item
  \textbf{બિટ `1'}: f_{1} = fc + Rb/4
\item
  \textbf{બિટ `0'}: f_{2} = fc - Rb/4
\item
  \textbf{ફ્રીક્વેન્સી ડેવિએશન}: Δf = Rb/2
\end{itemize}

\textbf{લાક્ષણિકતાઓ:}

\begin{itemize}
\tightlist
\item
  \textbf{સ્પેક્ટ્રલ એફિશિયન્સી}: પરંપરાગત FSK કરતાં બહેતર
\item
  \textbf{સતત ફેઝ}: આઉટ-ઓફ-બેન્ડ રેડિએશન ઘટાડે છે
\item
  \textbf{ઓર્થોગોનલ ડિટેક્શન}: OQPSK તરીકે ડિટેક્ટ કરી શકાય છે
\end{itemize}

\textbf{યાદગાર વાક્ય:} ``MSK - મિનિમમ શિફ્ટ, સતત ફેઝ કી''

\end{solutionbox}
\subsection*{પ્રશ્ન 3(બ અથવા) [4
ગુણ]}\label{uxaaauxab0uxab6uxaa8-3uxaac-uxa85uxaa5uxab5-4-uxa97uxaa3}

\textbf{16-QAM નો કોન્સ્ટેલેશન ડાયાગ્રામ દોરો અને સમજાવો}

\begin{solutionbox}

\textbf{16-QAM કોન્સ્ટેલેશન:}

\begin{verbatim}
           Q
           |
     ●  ●  |  ●  ●
           |
     ●  ● {-3{-}1 1 3 ● I}
           |
     ●  ●  |  ●  ●
           |
     ●  ●  |  ●  ●
           |
\end{verbatim}

\textbf{16-QAM મેપિંગ ટેબલ:}

{\def\LTcaptype{none} % do not increment counter
\begin{longtable}[]{@{}lllll@{}}
\toprule\noalign{}
બિટ્સ & I & Q & એમ્પ્લિટ્યુડ & ફેઝ \\
\midrule\noalign{}
\endhead
\bottomrule\noalign{}
\endlastfoot
0000 & -3 & -3 & \sqrt18 & 225^\circ \\
0001 & -3 & -1 & \sqrt10 & 198.4^\circ \\
0010 & -3 & +1 & \sqrt10 & 161.6^\circ \\
0011 & -3 & +3 & \sqrt18 & 135^\circ \\
0100 & -1 & -3 & \sqrt10 & 251.6^\circ \\
0101 & -1 & -1 & \sqrt2 & 225^\circ \\
\ldots{} & \ldots{} & \ldots{} & \ldots{} & \ldots{} \\
\end{longtable}
}

\textbf{મુખ્ય લાક્ષણિકતાઓ:}

\begin{itemize}
\tightlist
\item
  \textbf{16 સિમ્બોલ પોઇન્ટ્સ}: પ્રતિ સિમ્બોલ 4 બિટ્સ
\item
  \textbf{ગ્રે કોડિંગ}: નજીકના સિમ્બોલ્સ 1 બિટથી અલગ પડે છે
\item
  \textbf{વેરિયેબલ એમ્પ્લિટ્યુડ}: અલગ પાવર લેવલ્સ
\item
  \textbf{ઊંચો ડેટા રેટ}: QPSK કરતાં 4 ગણો ડેટા રેટ
\end{itemize}

\textbf{સિગ્નલ રજૂઆત:} s(t) = I(t)·cos(ωt) - Q(t)·sin(ωt)

\textbf{ઉપયોગો:}

\begin{itemize}
\tightlist
\item
  \textbf{ડિજિટલ કેબલ TV}: ઊંચો ડેટા રેટ ટ્રાન્સમિશન
\item
  \textbf{માઇક્રોવેવ લિંક્સ}: પોઇન્ટ-ટુ-પોઇન્ટ કોમ્યુનિકેશન
\item
  \textbf{WiFi સિસ્ટમ્સ}: 802.11 સ્ટાન્ડર્ડ્સ
\end{itemize}

\textbf{ફાયદાઓ:}

\begin{itemize}
\tightlist
\item
  \textbf{ઊંચી સ્પેક્ટ્રલ એફિશિયન્સી}: પ્રતિ સિમ્બોલ 4 બિટ્સ
\item
  \textbf{સારી BER પર્ફોર્મન્સ}: યોગ્ય કોડિંગ સાથે
\item
  \textbf{લવચીક અમલીકરણ}: સોફ્ટવેર ડિફાઇન્ડ રેડિયો
\end{itemize}

\textbf{ટ્રેડ-ઓફ્સ:}

\begin{itemize}
\tightlist
\item
  \textbf{ઊંચી જટિલતા}: QPSK કરતાં વધુ જટિલ
\item
  \textbf{પાવર વેરીએશન}: લીનિયર એમ્પ્લિફાયર્સ જરૂરી
\item
  \textbf{નોઇઝ સેન્સિટિવિટી}: કોન્સ્ટન્ટ એન્વેલોપ સ્કીમ્સ કરતાં ઊંચી
\end{itemize}

\textbf{યાદગાર વાક્ય:} ``16-QAM - 16 પોઇન્ટ્સ, 4 બિટ્સ, ક્વાડ્રેચર એમ્પ્લિટ્યુડ
મોડ્યુલેશન''

\end{solutionbox}
\subsection*{પ્રશ્ન 3(ક અથવા) [7
ગુણ]}\label{uxaaauxab0uxab6uxaa8-3uxa95-uxa85uxaa5uxab5-7-uxa97uxaa3}

\textbf{ડિજિટલ મોડ્યુલેશન ટેકનિક્સ-ASK, FSK, PSK, QPSK,8-PSK, MSK અને 16-QAM
માટે બિટ્સ પર સિમ્બોલની સરખામણી કરો}

\begin{solutionbox}

\textbf{બિટ્સ પર સિમ્બોલ સરખામણી:}

{\def\LTcaptype{none} % do not increment counter
\begin{longtable}[]{@{}llll@{}}
\toprule\noalign{}
મોડ્યુલેશન & બિટ્સ પર સિમ્બોલ & સિમ્બોલ રેટ & ડેટા રેટ સંબંધ \\
\midrule\noalign{}
\endhead
\bottomrule\noalign{}
\endlastfoot
\textbf{ASK} & 1 & Rs = Rb & Rb = Rs \times 1 \\
\textbf{FSK} & 1 & Rs = Rb & Rb = Rs \times 1 \\
\textbf{PSK (BPSK)} & 1 & Rs = Rb & Rb = Rs \times 1 \\
\textbf{QPSK} & 2 & Rs = Rb/2 & Rb = Rs \times 2 \\
\textbf{8-PSK} & 3 & Rs = Rb/3 & Rb = Rs \times 3 \\
\textbf{MSK} & 1 & Rs = Rb & Rb = Rs \times 1 \\
\textbf{16-QAM} & 4 & Rs = Rb/4 & Rb = Rs \times 4 \\
\end{longtable}
}

\textbf{વિગતવાર વિશ્લેષણ:}

\begin{center}
\textbf{Mermaid Diagram (Code)}
\begin{verbatim}
{Shaded}
{Highlighting}[]
graph LR
    A[ડિજિટલ મોડ્યુલેશન] {-{-}{} B[M{-}ary મોડ્યુલેશન]}
    B {-{-}{} C["બિટ્સ પર સિમ્બોલ = log_{2}(M)"]}
    C {-{-}{} D[ઊંચો M = વધુ બિટ્સ પર સિમ્બોલ]}
    D {-{-}{} E[ઊંચો ડેટા રેટ]}
    E {-{-}{} F[પરંતુ ઊંચી જટિલતા]}
{Highlighting}
{Shaded}
\end{verbatim}
\end{center}

\textbf{બેન્ડવિડ્થ એફિશિયન્સી:}

{\def\LTcaptype{none} % do not increment counter
\begin{longtable}[]{@{}llll@{}}
\toprule\noalign{}
મોડ્યુલેશન & M & બિટ્સ/સિમ્બોલ & બેન્ડવિડ્થ એફિશિયન્સી \\
\midrule\noalign{}
\endhead
\bottomrule\noalign{}
\endlastfoot
ASK, FSK, PSK & 2 & 1 & 1 bit/s/Hz \\
QPSK & 4 & 2 & 2 bits/s/Hz \\
8-PSK & 8 & 3 & 3 bits/s/Hz \\
16-QAM & 16 & 4 & 4 bits/s/Hz \\
\end{longtable}
}

\textbf{પાવર આવશ્યકતાઓ:}

{\def\LTcaptype{none} % do not increment counter
\begin{longtable}[]{@{}lll@{}}
\toprule\noalign{}
મોડ્યુલેશન & સંબંધિત પાવર & BER પર્ફોર્મન્સ \\
\midrule\noalign{}
\endhead
\bottomrule\noalign{}
\endlastfoot
\textbf{PSK} & રેફરન્સ & શ્રેષ્ઠ \\
\textbf{ASK} & +3dB પેનલ્ટી & નબળી \\
\textbf{FSK} & PSK જેટલી & સારી \\
\textbf{QPSK} & PSK જેટલી & PSK જેટલી \\
\textbf{8-PSK} & +2.5dB પેનલ્ટી & મધ્યમ \\
\textbf{16-QAM} & +4dB પેનલ્ટી & કોડિંગ સાથે સારી \\
\end{longtable}
}

\textbf{ટ્રેડ-ઓફ્સ:}

\begin{itemize}
\tightlist
\item
  \textbf{ઊંચો M}: વધુ બિટ્સ પર સિમ્બોલ પરંતુ ઊંચી જટિલતા
\item
  \textbf{બેન્ડવિડ્થ વિ પાવર}: સ્પેક્ટ્રલ અને પાવર એફિશિયન્સી વચ્ચે ટ્રેડ-ઓફ
\item
  \textbf{અમલીકરણ}: ઊંચા ઓર્ડરના મોડ્યુલેશનને બહેતર હાર્ડવેર જોઈએ છે
\end{itemize}

\textbf{ઉપયોગો:}

\begin{itemize}
\tightlist
\item
  \textbf{નીચો રેટ}: સરળ સિસ્ટમ્સ માટે ASK, FSK, PSK
\item
  \textbf{મધ્યમ રેટ}: સંતુલિત પર્ફોર્મન્સ માટે QPSK
\item
  \textbf{ઊંચો રેટ}: હાઇ-સ્પીડ સિસ્ટમ્સ માટે 8-PSK, 16-QAM
\end{itemize}

\textbf{સૂત્ર:} બિટ્સ પર સિમ્બોલ = log_{2}(M), જ્યાં M = સિમ્બોલ્સની સંખ્યા

\textbf{યાદગાર વાક્ય:} ``વધુ સિમ્બોલ્સ, વધુ બિટ્સ, વધુ જટિલતા''

\end{solutionbox}
\subsection*{પ્રશ્ન 4(અ) [3
ગુણ]}\label{uxaaauxab0uxab6uxaa8-4uxa85-3-uxa97uxaa3}

\textbf{સંભાવનાની વ્યાખ્યા કરો અને કોમ્યુનિકેશનમાં તેનું મહત્વ લખો}

\begin{solutionbox}

\textbf{સંભાવનાની વ્યાખ્યા:} સંભાવના એ કોઈ ઘટના બનવાની શક્યતાનું માપ છે, જે 0 અને
1 વચ્ચેની સંખ્યા તરીકે દર્શાવવામાં આવે છે.

P(ઘટના) = અનુકૂળ પરિણામોની સંખ્યા / કુલ શક્ય પરિણામોની સંખ્યા

\textbf{કોમ્યુનિકેશનમાં મહત્વ:}

{\def\LTcaptype{none} % do not increment counter
\begin{longtable}[]{@{}ll@{}}
\toprule\noalign{}
ઉપયોગ & મહત્વ \\
\midrule\noalign{}
\endhead
\bottomrule\noalign{}
\endlastfoot
\textbf{એરર વિશ્લેષણ} & બિટ એરર રેટ (BER) ની ગણતરી \\
\textbf{ચેનલ મોડેલિંગ} & નોઇઝ અને ફેડિંગ આંકડાશાસ્ત્ર \\
\textbf{કોડિંગ થિયરી} & એરર કરેક્શન સંભાવના \\
\textbf{સિગ્નલ ડિટેક્શન} & ડિટેક્શન અને ફોલ્સ એલાર્મ રેટ્સ \\
\end{longtable}
}

\textbf{મુખ્ય ઉપયોગો:}

\begin{itemize}
\tightlist
\item
  \textbf{BER ગણતરી}: P(error) = Q(\sqrt(2Eb/N0))
\item
  \textbf{ચેનલ કેપેસિટી}: શેનોનનું થિયરમ સંભાવનાનો ઉપયોગ કરે છે
\item
  \textbf{ઇન્ફોર્મેશન થિયરી}: એન્ટ્રહોપી સંભાવના પર આધારિત છે
\item
  \textbf{સિસ્ટમ ડિઝાઇન}: પર્ફોર્મન્સ પૂર્વાનુમાન
\end{itemize}

\textbf{ગાણિતિક સાધનો:}

\begin{itemize}
\tightlist
\item
  \textbf{ગૌસિયન ડિસ્ટ્રિબ્યુશન}: નોઇઝ વિશ્લેષણ માટે
\item
  \textbf{રેલે ડિસ્ટ્રિબ્યુશન}: ફેડિંગ ચેનલ્સ માટે
\item
  \textbf{પોઇસન ડિસ્ટ્રિબ્યુશન}: આગમન પ્રક્રિયાઓ માટે
\end{itemize}

\textbf{યાદગાર વાક્ય:} ``સંભાવના કોમ્યુનિકેશન સિસ્ટમ્સમાં પર્ફોર્મન્સની આગાહી કરે
છે''

\end{solutionbox}
\subsection*{પ્રશ્ન 4(બ) [4
ગુણ]}\label{uxaaauxab0uxab6uxaa8-4uxaac-4-uxa97uxaa3}

\textbf{હફમેન કોડ યોગ્ય દાખલા સાથે સમજાવો}

\begin{solutionbox}

\textbf{હફમેન કોડિંગ સિદ્ધાંત:} વેરિએબલ લેન્થ કોડિંગ જ્યાં વારંવાર આવતા સિમ્બોલ્સને
ટૂંકા કોડ મળે છે.

\textbf{એલ્ગોરિધમ:}

\begin{enumerate}
\tightlist
\item
  સંભાવનાઓ સાથે સિમ્બોલ્સની યાદી બનાવો
\item
  બે સૌથી ઓછી સંભાવના વાળા સિમ્બોલ્સને જોડો
\item
  જ્યાં સુધી એક સિમ્બોલ બાકી ન રહે ત્યાં સુધી પુનરાવર્તન કરો
\item
  કોડ આપો: ડાબે = 0, જમણે = 1
\end{enumerate}

\textbf{ઉદાહરણ:}

{\def\LTcaptype{none} % do not increment counter
\begin{longtable}[]{@{}lll@{}}
\toprule\noalign{}
સિમ્બોલ & સંભાવના & હફમેન કોડ \\
\midrule\noalign{}
\endhead
\bottomrule\noalign{}
\endlastfoot
A & 0.4 & 0 \\
B & 0.3 & 10 \\
C & 0.2 & 110 \\
D & 0.1 & 111 \\
\end{longtable}
}

\textbf{હફમેન ટ્રી નિર્માણ:}

\begin{center}
\textbf{Mermaid Diagram (Code)}
\begin{verbatim}
{Shaded}
{Highlighting}[]
graph LR
    A1[1.0] {-{-}{} B1[A: 0.4]}
    A1 {-{-}{} C1[0.6]}
    C1 {-{-}{} D1[B: 0.3] }
    C1 {-{-}{} E1[0.3]}
    E1 {-{-}{} F1[C: 0.2]}
    E1 {-{-}{} G1[D: 0.1]}
{Highlighting}
{Shaded}
\end{verbatim}
\end{center}

\textbf{કોડ એસાઇનમેન્ટ:}

\begin{itemize}
\tightlist
\item
  \textbf{A}: 0 (1 બિટ)
\item
  \textbf{B}: 10 (2 બિટ)\\
\item
  \textbf{C}: 110 (3 બિટ)
\item
  \textbf{D}: 111 (3 બિટ)
\end{itemize}

\textbf{એવરેજ કોડ લેન્થ:} L = 0.4\times1 + 0.3\times2 + 0.2\times3 + 0.1\times3 = 1.9
બિટ્સ/સિમ્બોલ

\textbf{ફાયદાઓ:}

\begin{itemize}
\tightlist
\item
  \textbf{ઓપ્ટિમલ}: ન્યૂનતમ એવરેજ કોડ લેન્થ
\item
  \textbf{પ્રીફિક્સ પ્રોપર્ટી}: કોઈ કોડ બીજાનો પ્રીફિક્સ નથી
\item
  \textbf{એફિશિયન્ટ}: ટ્રાન્સમિશન બેન્ડવિડ્થ ઘટાડે છે
\end{itemize}

\textbf{યાદગાર વાક્ય:} ``હફમેન - વારંવાર આવતા સિમ્બોલ્સને ટૂંકા કોડ''

\end{solutionbox}
\subsection*{પ્રશ્ન 4(ક) [7
ગુણ]}\label{uxaaauxab0uxab6uxaa8-4uxa95-7-uxa97uxaa3}

\textbf{ઇન્ટરનેટ ઓફ થિંગ્સ (IoT) નો ખ્યાલ અને મુખ્ય લક્ષણો સમજાવો.}

\begin{solutionbox}

\textbf{IoT ખ્યાલ:} ઇન્ટરનેટ ઓફ થિંગ્સ એ સેન્સર્સ, સોફ્ટવેર અને કનેક્ટિવિટી સાથે એમ્બેડેડ
ભૌતિક ઉપકરણોનું નેટવર્ક છે જે ડેટા એકત્રિત કરવા અને વિનિમય કરવા માટે છે.

\textbf{IoT આર્કિટેક્ચર:}

\begin{center}
\textbf{Mermaid Diagram (Code)}
\begin{verbatim}
{Shaded}
{Highlighting}[]
graph LR
    A[ભૌતિક ઉપકરણો] {-{-}{} B[કનેક્ટિવિટી લેયર]}
    B {-{-}{} C[ડેટા પ્રોસેસિંગ]}
    C {-{-}{} D[એપ્લિકેશન લેયર]}
    D {-{-}{} E[બિઝનેસ લેયર]}
    
    F[સેન્સર્સ] {-{-}{} A}
    G[એક્ચ્યુએટર્સ] {-{-}{} A}
    H[WiFi/Bluetooth] {-{-}{} B}
    I[Cellular/LoRa] {-{-}{} B}
    J[ક્લાઉડ કમ્પ્યુટિંગ] {-{-}{} C}
    K[એજ કમ્પ્યુટિંગ] {-{-}{} C}
{Highlighting}
{Shaded}
\end{verbatim}
\end{center}

\textbf{મુખ્ય લક્ષણો:}

{\def\LTcaptype{none} % do not increment counter
\begin{longtable}[]{@{}lll@{}}
\toprule\noalign{}
લક્ષણ & વર્ણન & ઉદાહરણ \\
\midrule\noalign{}
\endhead
\bottomrule\noalign{}
\endlastfoot
\textbf{કનેક્ટિવિટી} & ઉપકરણો ઇન્ટરનેટ સાથે જોડાયેલા & WiFi, 4G, 5G \\
\textbf{બુદ્ધિમત્તા} & સ્માર્ટ નિર્ણય લેવા & AI અલ્ગોરિધમ્સ \\
\textbf{સેન્સિંગ} & પર્યાવરણમાંથી ડેટા એકત્રીકરણ & તાપમાન, ભેજ \\
\textbf{એક્ચ્યુએશન} & ભૌતિક પ્રક્રિયાઓનું નિયંત્રણ & મોટર્સ, વાલ્વ્સ \\
\textbf{ઇન્ટરઓપરેબિલિટી} & ઉપકરણો સાથે મળીને કાર્ય & સ્ટાન્ડર્ડ પ્રોટોકોલ્સ \\
\end{longtable}
}

\textbf{IoT પ્રોટોકોલ સ્ટેક:}

{\def\LTcaptype{none} % do not increment counter
\begin{longtable}[]{@{}lll@{}}
\toprule\noalign{}
લેયર & પ્રોટોકોલ્સ & કાર્ય \\
\midrule\noalign{}
\endhead
\bottomrule\noalign{}
\endlastfoot
\textbf{એપ્લિકેશન} & HTTP, CoAP, MQTT & ડેટા વિનિમય \\
\textbf{ટ્રાન્સપોર્ટ} & TCP, UDP & વિશ્વસનીય ટ્રાન્સમિશન \\
\textbf{નેટવર્ક} & IPv6, 6LoWPAN & રાઉટિંગ \\
\textbf{ભૌતિક} & WiFi, ZigBee, LoRa & કનેક્ટિવિટી \\
\end{longtable}
}

\textbf{ઉપયોગો:}

\begin{itemize}
\tightlist
\item
  \textbf{સ્માર્ટ હોમ}: સ્વચાલિત લાઇટિંગ, સિક્યોરિટી
\item
  \textbf{ઇન્ડસ્ટ્રિયલ IoT}: મેન્યુફેક્ચરિંગ ઓટોમેશન
\item
  \textbf{આરોગ્યસેવા}: દૂરસ્થ પેશન્ટ મોનિટરિંગ
\item
  \textbf{સ્માર્ટ સિટીઝ}: ટ્રાફિક મેનેજમેન્ટ, યુટિલિટીઝ
\end{itemize}

\textbf{પડકારો:}

\begin{itemize}
\tightlist
\item
  \textbf{સિક્યોરિટી}: ઉપકરણની નબળાઈઓ, ડેટા પ્રાઇવેસી
\item
  \textbf{સ્કેલેબિલિટી}: અબજો ઉપકરણો
\item
  \textbf{ઇન્ટરઓપરેબિલિટી}: અલગ અલગ સ્ટાન્ડર્ડ્સ
\item
  \textbf{પાવર કન્ઝમ્પશન}: બેટરી ચાલિત ઉપકરણો
\end{itemize}

\textbf{ફાયદાઓ:}

\begin{itemize}
\tightlist
\item
  \textbf{ઓટોમેશન}: માનવ હસ્તક્ષેપ ઘટાડો
\item
  \textbf{એફિશિયન્સી}: સંસાધનોનો શ્રેષ્ઠ ઉપયોગ
\item
  \textbf{રીઅલ-ટાઇમ મોનિટરિંગ}: તાત્કાલિક ડેટા ઍક્સેસ
\item
  \textbf{કોસ્ટ રિડક્શન}: પ્રિડિક્ટિવ મેઇન્ટેનન્સ
\end{itemize}

\textbf{ટેકનોલોજીઓ:}

\begin{itemize}
\tightlist
\item
  \textbf{કોમ્યુનિકેશન}: WiFi, Bluetooth, Cellular, LoRa
\item
  \textbf{પ્રોસેસિંગ}: એજ કમ્પ્યુટિંગ, ક્લાઉડ કમ્પ્યુટિંગ
\item
  \textbf{એનાલિટિક્સ}: બિગ ડેટા, મશીન લર્નિંગ
\item
  \textbf{સિક્યોરિટી}: એન્ક્રિપ્શન, ઓથેન્ટિકેશન
\end{itemize}

\textbf{યાદગાર વાક્ય:} ``IoT - ઇન્ટરનેટ ઓફ થિંગ્સ, સ્માર્ટ કનેક્ટેડ ઉપકરણો સર્વત્ર''

\end{solutionbox}
\subsection*{પ્રશ્ન 4(અ અથવા) [3
ગુણ]}\label{uxaaauxab0uxab6uxaa8-4uxa85-uxa85uxaa5uxab5-3-uxa97uxaa3}

\textbf{એરર કરેક્શન કોડની વ્યાખ્યા કરો અને સામાન્ય એરર કરેક્ટિંગ કોડની યાદી આપો.}

\begin{solutionbox}

\textbf{એરર કરેક્શન કોડ વ્યાખ્યા:} એરર કરેક્શન કોડ એ એવી તકનીકો છે જે ટ્રાન્સમિશન
એરર્સને સ્વચાલિત રીતે શોધવા અને સુધારવા માટે ડેટામાં રિડન્ડન્ટ બિટ્સ ઉમેરે છે.

\textbf{સામાન્ય એરર કરેક્ટિંગ કોડઝ:}

{\def\LTcaptype{none} % do not increment counter
\begin{longtable}[]{@{}lll@{}}
\toprule\noalign{}
કોડનો પ્રકાર & વર્ણન & ક્ષમતા \\
\midrule\noalign{}
\endhead
\bottomrule\noalign{}
\endlastfoot
\textbf{હેમિંગ કોડ} & સિંગલ એરર કરેક્શન & 1-બિટ એરર સુધારે છે \\
\textbf{રીડ-સોલોમન} & બર્સ્ટ એરર્સ માટે બ્લોક કોડ & મલ્ટિપલ એરર્સ સુધારે છે \\
\textbf{BCH કોડ} & બાઇનરી સાયક્લિક કોડ & t એરર્સ સુધારે છે \\
\textbf{કોન્વોલ્યુશનલ કોડ} & સતત એન્કોડિંગ & નોઇઝી ચેનલ્સ માટે સારું \\
\textbf{ટર્બો કોડ} & ઇટરેટિવ ડિકોડિંગ & શેનોન લિમિટની નજીક \\
\textbf{LDPC કોડ} & લો ડેન્સિટી પેરિટી ચેક & શ્રેષ્ઠ પર્ફોર્મન્સ \\
\end{longtable}
}

\textbf{ઉપયોગો:}

\begin{itemize}
\tightlist
\item
  \textbf{મેમરી સિસ્ટમ્સ}: ECC RAM
\item
  \textbf{સ્ટોરેજ ડિવાઇસેસ}: હાર્ડ ડ્રાઇવ્સ, CDs
\item
  \textbf{કોમ્યુનિકેશન}: સેટેલાઇટ, સેલ્યુલર
\item
  \textbf{બ્રોડકાસ્ટિંગ}: ડિજિટલ TV, રેડિયો
\end{itemize}

\textbf{યાદગાર વાક્ય:} ``એરર કરેક્શન કોડઝ - હેમિંગ રીડ BCH કોન્વોલ્યુશનલ ટર્બો
LDPC''

\end{solutionbox}
\subsection*{પ્રશ્ન 4(બ અથવા) [4
ગુણ]}\label{uxaaauxab0uxab6uxaa8-4uxaac-uxa85uxaa5uxab5-4-uxa97uxaa3}

\textbf{શેનોન ફેનો કોડ યોગ્ય દાખલા સાથે સમજાવો}

\begin{solutionbox}

\textbf{શેનોન-ફેનો કોડિંગ એલ્ગોરિધમ:} ટોપ-ડાઉન અપ્રોચ જે સિમ્બોલ્સને લગભગ સમાન
સંભાવનાઓ ધરાવતા બે જૂથોમાં વિભાજિત કરે છે.

\textbf{એલ્ગોરિધમ સ્ટેપ્સ:}

\begin{enumerate}
\tightlist
\item
  સિમ્બોલ્સને ઘટતા સંભાવના ક્રમમાં ગોઠવો
\item
  લગભગ સમાન કુલ સંભાવના સાથે બે જૂથોમાં વિભાજિત કરો
\item
  પહેલા જૂથને `0', બીજા જૂથને `1' આપો
\item
  દરેક સબગ્રુપ માટે પુનરાવર્તન કરો
\end{enumerate}

\textbf{ઉદાહરણ:}

{\def\LTcaptype{none} % do not increment counter
\begin{longtable}[]{@{}lll@{}}
\toprule\noalign{}
સિમ્બોલ & સંભાવના & શેનોન-ફેનો કોડ \\
\midrule\noalign{}
\endhead
\bottomrule\noalign{}
\endlastfoot
A & 0.4 & 00 \\
B & 0.3 & 01 \\
C & 0.2 & 10 \\
D & 0.1 & 11 \\
\end{longtable}
}

\textbf{કન્સ્ટ્રક્શન ટ્રી:}

\begin{center}
\textbf{Mermaid Diagram (Code)}
\begin{verbatim}
{Shaded}
{Highlighting}[]
graph TD
    A1[A,B,C,D: 1.0] {-{-}{} B1[A,B: 0.7]}
    A1 {-{-}{} C1[C,D: 0.3]}
    B1 {-{-}{} D1[A: 0.4]}
    B1 {-{-}{} E1[B: 0.3]}
    C1 {-{-}{} F1[C: 0.2]}
    C1 {-{-}{} G1[D: 0.1]}
{Highlighting}
{Shaded}
\end{verbatim}
\end{center}

\textbf{કોડ એસાઇનમેન્ટ:}

\begin{itemize}
\tightlist
\item
  જૂથ 1 (A,B): કોડ `0' થી શરૂ થાય છે
\item
  જૂથ 2 (C,D): કોડ `1' થી શરૂ થાય છે
\item
  A: 00, B: 01, C: 10, D: 11
\end{itemize}

\textbf{હફમેન સાથે સરખામણી:}

\begin{itemize}
\tightlist
\item
  \textbf{શેનોન-ફેનો}: ટોપ-ડાઉન અપ્રોચ
\item
  \textbf{હફમેન}: બોટમ-અપ અપ્રોચ
\item
  \textbf{હફમેન}: હંમેશા ઓપ્ટિમલ
\item
  \textbf{શેનોન-ફેનો}: ઓપ્ટિમલ ન પણ હોય
\end{itemize}

\textbf{એવરેજ કોડ લેન્થ:} L = 0.4\times2 + 0.3\times2 + 0.2\times2 + 0.1\times2 = 2.0
બિટ્સ/સિમ્બોલ

\textbf{યાદગાર વાક્ય:} ``શેનોન-ફેનો - જૂથો વિભાજિત કરો, ટોપ-ડાઉન કોડ આપો''

\end{solutionbox}
\subsection*{પ્રશ્ન 4(ક અથવા) [7
ગુણ]}\label{uxaaauxab0uxab6uxaa8-4uxa95-uxa85uxaa5uxab5-7-uxa97uxaa3}

\textbf{ઓડિયો સિગ્નલના વિવિધ પ્રમાણભૂત ફોર્મેટ્સ સમજાવો.}

\begin{solutionbox}

\textbf{ઓડિયો સિગ્નલ ફોર્મેટ્સ:}

{\def\LTcaptype{none} % do not increment counter
\begin{longtable}[]{@{}lllll@{}}
\toprule\noalign{}
ફોર્મેટ & પૂરું નામ & કોમ્પ્રેશન & ગુણવત્તા & ફાઇલ સાઇઝ \\
\midrule\noalign{}
\endhead
\bottomrule\noalign{}
\endlastfoot
\textbf{WAV} & Waveform Audio File & અનકોમ્પ્રેસ્ડ & સૌથી ઊંચી & સૌથી મોટી \\
\textbf{MP3} & MPEG Layer 3 & લોસી & સારી & નાની \\
\textbf{AAC} & Advanced Audio Coding & લોસી & MP3 કરતાં બહેતર & નાની \\
\textbf{FLAC} & Free Lossless Audio Codec & લોસલેસ & મૂળ & મધ્યમ \\
\textbf{OGG} & Ogg Vorbis & લોસી & સારી & નાની \\
\end{longtable}
}

\textbf{ઓડિયો પેરામીટર્સ:}

\begin{center}
\textbf{Mermaid Diagram (Code)}
\begin{verbatim}
{Shaded}
{Highlighting}[]
graph TD
    A[ઓડિયો સિગ્નલ] {-{-}{} B[સેમ્પલિંગ રેટ]}
    A {-{-}{} C[બિટ ડેપ્થ]}
    A {-{-}{} D[ચેનલ્સ]}
    A {-{-}{} E[કોમ્પ્રેશન]}
    
    B {-{-}{} F[44.1 kHz CD ગુણવત્તા]}
    C {-{-}{} G[16{-}બિટ સ્ટાન્ડર્ડ]}
    D {-{-}{} H[મોનો/સ્ટેરિયો]}
    E {-{-}{} I[લોસી/લોસલેસ]}
{Highlighting}
{Shaded}
\end{verbatim}
\end{center}

\textbf{સેમ્પલિંગ સ્ટાન્ડર્ડ્સ:}

{\def\LTcaptype{none} % do not increment counter
\begin{longtable}[]{@{}llll@{}}
\toprule\noalign{}
સ્ટાન્ડર્ડ & સેમ્પલિંગ રેટ & બિટ ડેપ્થ & ઉપયોગ \\
\midrule\noalign{}
\endhead
\bottomrule\noalign{}
\endlastfoot
\textbf{CD ગુણવત્તા} & 44.1 kHz & 16-બિટ & કન્ઝ્યુમર ઓડિયો \\
\textbf{સ્ટુડિયો ગુણવત્તા} & 48 kHz & 24-બિટ & પ્રોફેશનલ રેકોર્ડિંગ \\
\textbf{હાઇ રેઝોલ્યુશન} & 96 kHz & 24-બિટ & ઓડિયોફાઇલ \\
\textbf{ટેલિફોન} & 8 kHz & 8-બિટ & વૉઇસ કોમ્યુનિકેશન \\
\end{longtable}
}

\textbf{કોમ્પ્રેશનના પ્રકારો:}

\begin{itemize}
\tightlist
\item
  \textbf{લોસલેસ}: મૂળ ગુણવત્તા સાચવાય છે (FLAC, ALAC)
\item
  \textbf{લોસી}: નાની સાઇઝ માટે કંઈક ગુણવત્તા ગુમાવાય છે (MP3, AAC)
\item
  \textbf{અનકોમ્પ્રેસ્ડ}: કોઈ કોમ્પ્રેશન નથી (WAV, AIFF)
\end{itemize}

\textbf{ઉપયોગો:}

\begin{itemize}
\tightlist
\item
  \textbf{બ્રોડકાસ્ટિંગ}: ડિજિટલ રેડિયો માટે AAC
\item
  \textbf{સ્ટ્રીમિંગ}: ઇન્ટરનેટ માટે MP3, AAC
\item
  \textbf{પ્રોફેશનલ}: સ્ટુડિયો માટે WAV, FLAC
\item
  \textbf{મોબાઇલ}: સ્માર્ટફોન માટે AAC
\end{itemize}

\textbf{ફાઇલ સાઇઝ સરખામણી (3-મિનિટ ગીત):}

\begin{itemize}
\tightlist
\item
  \textbf{WAV}: 30 MB
\item
  \textbf{FLAC}: 20 MB
\item
  \textbf{MP3}: 3 MB
\item
  \textbf{AAC}: 2.5 MB
\end{itemize}

\textbf{ગુણવત્તા વિ સાઇઝ ટ્રેડ-ઓફ:}

\begin{itemize}
\tightlist
\item
  \textbf{સૌથી ઊંચી ગુણવત્તા}: WAV (અનકોમ્પ્રેસ્ડ)
\item
  \textbf{શ્રેષ્ઠ સંતુલન}: AAC (લોસી કોમ્પ્રેસ્ડ)
\item
  \textbf{સૌથી નાની સાઇઝ}: લો બિટરેટ MP3
\item
  \textbf{લોસલેસ કોમ્પ્રેસ્ડ}: FLAC
\end{itemize}

\textbf{યાદગાર વાક્ય:} ``WAV MP3 AAC FLAC - વેવ, લેયર3, એડવાન્સ્ડ, ફ્રી
લોસલેસ''

\end{solutionbox}
\subsection*{પ્રશ્ન 5(અ) [3
ગુણ]}\label{uxaaauxab0uxab6uxaa8-5uxa85-3-uxa97uxaa3}

\textbf{E1 કેરિયર મલ્ટિપ્લેક્સિંગ હાયરાર્કી સમજાવો.}

\begin{solutionbox}

\textbf{E1 કેરિયર સિસ્ટમ:} વૉઇસ ચેનલ્સને મલ્ટિપ્લેક્સ કરવા માટેનું યુરોપીયન ડિજિટલ
ટ્રાન્સમિશન સ્ટાન્ડર્ડ.

\textbf{E1 હાયરાર્કી:}

{\def\LTcaptype{none} % do not increment counter
\begin{longtable}[]{@{}lllll@{}}
\toprule\noalign{}
લેવલ & નામ & બિટ રેટ & વૉઇસ ચેનલ્સ & મલ્ટિપ્લેક્સિંગ \\
\midrule\noalign{}
\endhead
\bottomrule\noalign{}
\endlastfoot
\textbf{E0} & બેઝિક ચેનલ & 64 kbps & 1 & - \\
\textbf{E1} & પ્રાઇમરી રેટ & 2.048 Mbps & 30 & 30 \times E0 + 2 \\
\textbf{E2} & સેકન્ડરી રેટ & 8.448 Mbps & 120 & 4 \times E1 \\
\textbf{E3} & ટર્શિયરી રેટ & 34.368 Mbps & 480 & 4 \times E2 \\
\textbf{E4} & ક્વેટર્નરી રેટ & 139.264 Mbps & 1920 & 4 \times E3 \\
\end{longtable}
}

\textbf{E1 ફ્રેમ સ્ટ્રક્ચર:}

\begin{verbatim}
Frame (125 μs, 256 bits)
|TS0|TS1|TS2|...|TS15|TS16|TS17|...|TS31|
 32 time slots  8 bits = 256 bits

TS0: Synchronization + Alarm
TS16: Signaling (voice channels)
TS1{-15, 17{-}31: 30 voice channels}
\end{verbatim}

\textbf{મલ્ટિપ્લેક્સિંગ પ્રક્રિયા:}

\begin{itemize}
\tightlist
\item
  \textbf{લેવલ 1}: 30 વૉઇસ ચેનલ્સ + 2 કન્ટ્રોલ \rightarrow E1
\item
  \textbf{લેવલ 2}: 4 E1 સ્ટ્રીમ્સ \rightarrow E2
\item
  \textbf{લેવલ 3}: 4 E2 સ્ટ્રીમ્સ \rightarrow E3
\item
  \textbf{લેવલ 4}: 4 E3 સ્ટ્રીમ્સ \rightarrow E4
\end{itemize}

\textbf{ઉપયોગો:}

\begin{itemize}
\tightlist
\item
  \textbf{ISDN}: પ્રાઇમરી રેટ ઇન્ટરફેસ
\item
  \textbf{સેલ્યુલર}: બેઝ સ્ટેશન કનેક્ટિવિટી
\item
  \textbf{એન્ટરપ્રાઇઝ}: પ્રાઇવેટ બ્રાન્ચ એક્સચેન્જ (PBX)
\item
  \textbf{ઇન્ટરનેટ}: ડિજિટલ સબ્સ્ક્રાઇબર લાઇન (DSL)
\end{itemize}

\textbf{યાદગાર વાક્ય:} ``E1 - 30 અવાજો, 2.048 Mbps, યુરોપીયન સ્ટાન્ડર્ડ''

\end{solutionbox}
\subsection*{પ્રશ્ન 5(બ) [4
ગુણ]}\label{uxaaauxab0uxab6uxaa8-5uxaac-4-uxa97uxaa3}

\textbf{TDMA સાથે FDMA ની સરખામણી કરો.}

\begin{solutionbox}

\textbf{FDMA વિ TDMA સરખામણી:}

{\def\LTcaptype{none} % do not increment counter
\begin{longtable}[]{@{}
  >{\raggedright\arraybackslash}p{(\linewidth - 4\tabcolsep) * \real{0.4286}}
  >{\raggedright\arraybackslash}p{(\linewidth - 4\tabcolsep) * \real{0.2857}}
  >{\raggedright\arraybackslash}p{(\linewidth - 4\tabcolsep) * \real{0.2857}}@{}}
\toprule\noalign{}
\begin{minipage}[b]{\linewidth}\raggedright
પેરામીટર
\end{minipage} & \begin{minipage}[b]{\linewidth}\raggedright
FDMA
\end{minipage} & \begin{minipage}[b]{\linewidth}\raggedright
TDMA
\end{minipage} \\
\midrule\noalign{}
\endhead
\bottomrule\noalign{}
\endlastfoot
\textbf{પૂરું નામ} & Frequency Division Multiple Access & Time Division
Multiple Access \\
\textbf{ડોમેન} & ફ્રીક્વેન્સી & સમય \\
\textbf{ચેનલ એલોકેશન} & દરેક યુઝરને અલગ ફ્રીક્વેન્સી મળે છે & દરેક યુઝરને અલગ ટાઇમ
સ્લોટ મળે છે \\
\textbf{યુઝર દીઠ બેન્ડવિડ્થ} & સતત સાંકડી બેન્ડવિડ્થ & ટૂંકા સમય માટે સંપૂર્ણ
બેન્ડવિડ્થ \\
\textbf{ગાર્ડ બેન્ડ્સ} & ફ્રીક્વેન્સીઝ વચ્ચે જરૂરી & જરૂરી નથી \\
\textbf{સિંક્રોનાઇઝેશન} & મહત્વપૂર્ણ નથી & મહત્વપૂર્ણ છે \\
\textbf{લવચીકતા} & ઓછી લવચીક & વધુ લવચીક \\
\textbf{હેન્ડઓફ} & સરળ & જટિલ \\
\textbf{નીયર-ફાર ઇફેક્ટ} & ઓછી સમસ્યાજનક & વધુ સમસ્યાજનક \\
\end{longtable}
}

\textbf{FDMA સિસ્ટમ:}

\begin{center}
\textbf{Mermaid Diagram (Code)}
\begin{verbatim}
{Shaded}
{Highlighting}[]
graph TD
    A[કુલ બેન્ડવિડ્થ] {-{-}{} B[યુઝર 1: f1]}
    A {-{-}{} C[યુઝર 2: f2]}
    A {-{-}{} D[યુઝર 3: f3]}
    A {-{-}{} E[યુઝર N: fn]}
    
    F[ગાર્ડ બેન્ડ] {-{-}{} B}
    F {-{-}{} C}
    F {-{-}{} D}
{Highlighting}
{Shaded}
\end{verbatim}
\end{center}

\textbf{TDMA સિસ્ટમ:}

\begin{verbatim}
gantt
    title TDMA ટાઇમ સ્લોટ્સ
    dateFormat X
    axisFormat \%s
    
    section ફ્રેમ
    યુઝર 1 :done, u1, 0, 1
    યુઝર 2 :done, u2, 1, 2
    યુઝર 3 :done, u3, 2, 3
    યુઝર 4 :done, u4, 3, 4
\end{verbatim}

\textbf{ફાયદાઓ/નુકસાનો:}

{\def\LTcaptype{none} % do not increment counter
\begin{longtable}[]{@{}ll@{}}
\toprule\noalign{}
FDMA ફાયદાઓ & FDMA નુકસાનો \\
\midrule\noalign{}
\endhead
\bottomrule\noalign{}
\endlastfoot
સરળ અમલીકરણ & ગાર્ડ બેન્ડ્સને કારણે બેન્ડવિડ્થનો વેસ્ટેજ \\
સિંક્રોનાઇઝેશનની જરૂર નથી & ઓછી લવચીક \\
સતત ટ્રાન્સમિશન & વિવિધ રેટ્સ સામાવવાનું મુશ્કેલ \\
\end{longtable}
}

{\def\LTcaptype{none} % do not increment counter
\begin{longtable}[]{@{}ll@{}}
\toprule\noalign{}
TDMA ફાયદાઓ & TDMA નુકસાનો \\
\midrule\noalign{}
\endhead
\bottomrule\noalign{}
\endlastfoot
બેન્ડવિડ્થનો કુશળ ઉપયોગ & જટિલ સિંક્રોનાઇઝેશન \\
લવચીક ડેટા રેટ્સ & બેટરી લાઇફ સમસ્યાઓ (બર્સ્ટ ટ્રાન્સમિશન) \\
યુઝર્સ ઉમેરવા/કાઢવા સરળ & નીયર-ફાર પ્રોબ્લેમ \\
\end{longtable}
}

\textbf{ઉપયોગો:}

\begin{itemize}
\tightlist
\item
  \textbf{FDMA}: AMPS (1G), સેટેલાઇટ કોમ્યુનિકેશન
\item
  \textbf{TDMA}: GSM (2G), સેટેલાઇટ કોમ્યુનિકેશન
\end{itemize}

\textbf{યાદગાર વાક્ય:} ``FDMA ફ્રીક્વેન્સી, TDMA ટાઇમ - મલ્ટિપલ ઍક્સેસ માટે અલગ
ડોમેન્સ''

\end{solutionbox}
\subsection*{પ્રશ્ન 5(ક) [7
ગુણ]}\label{uxaaauxab0uxab6uxaa8-5uxa95-7-uxa97uxaa3}

\textbf{CDMA ટેકનિકને વિગતવાર સમજાવો.}

\textbf{CDMA સિદ્ધાંત:} કોડ ડિવિઝન મલ્ટિપલ ઍક્સેસ મલ્ટિપલ યુઝર્સને યુનિક સ્પ્રેડિંગ
કોડ્સનો ઉપયોગ કરીને સમાન ફ્રીક્વેન્સી અને સમય શેર કરવાની મંજૂરી આપે છે.

\textbf{CDMA સિસ્ટમ આર્કિટેક્ચર:}

\begin{center}
\textbf{Mermaid Diagram (Code)}
\begin{verbatim}
{Shaded}
{Highlighting}[]
graph LR
    A[યુઝર ડેટા] {-{-}{} B[સ્પ્રેડિંગ કોડ]}
    B {-{-}{} C[મોડ્યુલેટર]}
    C {-{-}{} D[ચેનલ]}
    D {-{-}{} E[કોરિલેટર]}
    E {-{-}{} F[ડિસ્પ્રેડિંગ]}
    F {-{-}{} G[ડેટા રિકવરી]}
    
    H[સ્યુડો{-રેન્ડમ કોડ] {-}{-}{} B}
    I[સેમ PN કોડ] {-{-}{} F}
{Highlighting}
{Shaded}
\end{verbatim}
\end{center}

\textbf{સ્પ્રેડિંગ પ્રક્રિયા:}

{\def\LTcaptype{none} % do not increment counter
\begin{longtable}[]{@{}lll@{}}
\toprule\noalign{}
પેરામીટર & સ્પ્રેડિંગ પહેલાં & સ્પ્રેડિંગ પછી \\
\midrule\noalign{}
\endhead
\bottomrule\noalign{}
\endlastfoot
\textbf{ડેટા રેટ} & Rb & Rb \\
\textbf{ચિપ રેટ} & - & Rc (= N \times Rb) \\
\textbf{બેન્ડવિડ્થ} & Rb & Rc \\
\textbf{પ્રોસેસિંગ ગેઇન} & 1 & N = Rc/Rb \\
\end{longtable}
}

\textbf{CDMA ગુણધર્મો:}

\begin{verbatim}
Original Data:    1  0  1
PN Code:         101 010 101
XOR Result:      101 010 101
(Spread Signal)

At Receiver:
Received:        101 010 101
Same PN Code:    101 010 101  
XOR Result:       1   0   1
(Original Data)
\end{verbatim}

\textbf{મુખ્ય લક્ષણો:}

{\def\LTcaptype{none} % do not increment counter
\begin{longtable}[]{@{}lll@{}}
\toprule\noalign{}
લક્ષણ & વર્ણન & ફાયદો \\
\midrule\noalign{}
\endhead
\bottomrule\noalign{}
\endlastfoot
\textbf{સ્પ્રેડિંગ} & PN કોડ સાથે ડેટા XOR & બેન્ડવિડ્થ વિસ્તરણ \\
\textbf{પ્રોસેસિંગ ગેઇન} & Rc/Rb રેશિયો & ઇન્ટરફેરન્સ રિજેક્શન \\
\textbf{સોફ્ટ હેન્ડઓફ} & એક સાથે બહુવિધ કનેક્શન્સ & બહેતર ગુણવત્તા \\
\textbf{પાવર કન્ટ્રોલ} & ડાયનામિક પાવર એડજસ્ટમેન્ટ & નીયર-ફાર સોલ્યુશન \\
\end{longtable}
}

\textbf{CDMA ફાયદાઓ:}

\begin{itemize}
\tightlist
\item
  \textbf{કેપેસિટી}: FDMA/TDMA કરતાં ઊંચી યુઝર કેપેસિટી
\item
  \textbf{સિક્યોરિટી}: સ્પ્રેડિંગ કોડથી એન્ક્રિપ્ટેડ
\item
  \textbf{સોફ્ટ હેન્ડઓફ}: હેન્ડઓફ દરમિયાન કોલ ડ્રોપિંગ નથી
\item
  \textbf{એન્ટી-જેમિંગ}: સ્પ્રેડ સ્પેક્ટ્રમ ઇમ્યુનિટી
\item
  \textbf{ફ્રીક્વેન્સી પ્લાનિંગ નથી}: સમાન ફ્રીક્વેન્સી રીયુઝ
\end{itemize}

\textbf{CDMA નુકસાનો:}

\begin{itemize}
\tightlist
\item
  \textbf{નીયર-ફાર પ્રોબ્લેમ}: પાવર કન્ટ્રોલ જરૂરી
\item
  \textbf{જટિલતા}: FDMA/TDMA કરતાં વધુ જટિલ
\item
  \textbf{સેલ્ફ ઇન્ટરફેરન્સ}: યુઝર્સ એકબીજા સાથે ઇન્ટરફેર કરે છે
\item
  \textbf{બ્રીધિંગ ઇફેક્ટ}: લોડિંગ સાથે કવરેજ બદલાય છે
\end{itemize}

\textbf{ગાણિતિક વિશ્લેષણ:}

\begin{itemize}
\tightlist
\item
  \textbf{પ્રોસેસિંગ ગેઇન}: G = Rc/Rb = 10log_{1}_{0}(Rc/Rb) dB
\item
  \textbf{કેપેસિટી}: M \approx 1 + G/(Eb/I_{0})
\item
  \textbf{BER}: સક્રિય યુઝર્સની સંખ્યા પર આધારિત
\end{itemize}

\textbf{પાવર કન્ટ્રોલ:}

\begin{itemize}
\tightlist
\item
  \textbf{ઓપન લૂપ}: પ્રાપ્ત સિગ્નલ સ્ટ્રેન્થના આધારે
\item
  \textbf{ક્લોઝ્ડ લૂપ}: બેઝ સ્ટેશન મોબાઇલને કમાન્ડ કરે છે
\item
  \textbf{આવશ્યકતા}: \pm1 dB ચોકસાઈ જરૂરી
\end{itemize}

\textbf{ઉપયોગો:}

\begin{itemize}
\tightlist
\item
  \textbf{IS-95 (cdmaOne)}: 2G CDMA સ્ટાન્ડર્ડ
\item
  \textbf{WCDMA}: 3G UMTS સિસ્ટમ
\item
  \textbf{GPS}: સેટેલાઇટ નેવિગેશન
\item
  \textbf{WiFi}: સ્પ્રેડ સ્પેક્ટ્રમ વિકલ્પ
\end{itemize}

\textbf{PN કોડ ગુણધર્મો:}

\begin{itemize}
\tightlist
\item
  \textbf{ઓટોકોરિલેશન}: સિંક્રોનાઇઝ્ડ માટે ઊંચું, અનસિંક્રોનાઇઝ્ડ માટે નીચું
\item
  \textbf{ક્રોસ-કોરિલેશન}: અલગ કોડ્સ વચ્ચે નીચું
\item
  \textbf{બેલેન્સ}: 1s અને 0s ની સમાન સંખ્યા
\item
  \textbf{રન લેન્થ}: સતત બિટ્સનું વિતરણ
\end{itemize}

\textbf{યાદગાર વાક્ય:} ``CDMA - કોડ ડિવિઝન, સમાન ફ્રીક્વેન્સી/સમય, મલ્ટિપલ
ઍક્સેસ માટે યુનિક કોડ્સ''

\subsection*{પ્રશ્ન 5(અ અથવા) [3
ગુણ]}\label{uxaaauxab0uxab6uxaa8-5uxa85-uxa85uxaa5uxab5-3-uxa97uxaa3}

\textbf{ટાઇમ ડિવિઝન મલ્ટિપ્લેક્સિંગ ટેકનિક (TDM) નો બ્લોક ડાયાગ્રામ દોરો.}

\begin{solutionbox}

\textbf{TDM બ્લોક ડાયાગ્રામ:}

\begin{center}
\textbf{Mermaid Diagram (Code)}
\begin{verbatim}
{Shaded}
{Highlighting}[]
graph LR
    A[ઇનપુટ 1] {-{-}{} E[મલ્ટિપ્લેક્સર]}
    B[ઇનપુટ 2] {-{-}{} E}
    C[ઇનપુટ 3] {-{-}{} E}
    D[ઇનપુટ N] {-{-}{} E}
    
    E {-{-}{} F[TDM સિગ્નલ]}
    F {-{-}{} G[ચેનલ]}
    G {-{-}{} H[ડિમલ્ટિપ્લેક્સર]}
    
    H {-{-}{} I[આઉટપુટ 1]}
    H {-{-}{} J[આઉટપુટ 2] }
    H {-{-}{} K[આઉટપુટ 3]}
    H {-{-}{} L[આઉટપુટ N]}
    
    M[ક્લોક/સિંક] {-{-}{} E}
    N[ક્લોક/સિંક] {-{-}{} H}
{Highlighting}
{Shaded}
\end{verbatim}
\end{center}

\textbf{TDM ફ્રેમ સ્ટ્રક્ચર:}

\begin{verbatim}
|{{-}{-}{-}{-} Frame Period T {-}{-}{-}{-}|}
|Ch1|Ch2|Ch3|...|ChN|Sync|
 TS1 TS2 TS3     TSN
 
Each time slot = T/N
Frame Rate = 1/T
\end{verbatim}

\textbf{ઘટકો:}

\begin{itemize}
\tightlist
\item
  \textbf{મલ્ટિપ્લેક્સર}: ઇનપુટ્સને અનુક્રમે સેમ્પલ કરે છે
\item
  \textbf{ક્લોક/સિંક્રોનાઇઝેશન}: સ્વિચિંગ ટાઇમિંગ કન્ટ્રોલ કરે છે
\item
  \textbf{ચેનલ}: ટ્રાન્સમિશન માધ્યમ
\item
  \textbf{ડિમલ્ટિપ્લેક્સર}: મલ્ટિપ્લેક્સ્ડ સિગ્નલને અલગ કરે છે
\end{itemize}

\textbf{કામગીરી:}

\begin{itemize}
\tightlist
\item
  દરેક ઇનપુટ ચેનલને ડેડિકેટેડ ટાઇમ સ્લોટ મળે છે
\item
  સેમ્પલિંગ રેટ નાયક્વિસ્ટ માપદંડ સંતોષવો જોઈએ
\item
  રીસીવર પર ફ્રેમ સિંક્રોનાઇઝેશન જરૂરી
\end{itemize}

\textbf{યાદગાર વાક્ય:} ``TDM - ટાઇમ ડિવિઝન, સિક્વેન્શિયલ સેમ્પલિંગ,
મલ્ટિપ્લેક્સિંગ''

\end{solutionbox}
\subsection*{પ્રશ્ન 5(બ અથવા) [4
ગુણ]}\label{uxaaauxab0uxab6uxaa8-5uxaac-uxa85uxaa5uxab5-4-uxa97uxaa3}

\textbf{મલ્ટિપ્લેક્સિંગ ટેકનિકોના વર્ગીકરણ પર ટૂંકી નોંધ લખો.}

\begin{solutionbox}

\textbf{મલ્ટિપ્લેક્સિંગ ટેકનિકોનું વર્ગીકરણ:}

\begin{center}
\textbf{Mermaid Diagram (Code)}
\begin{verbatim}
{Shaded}
{Highlighting}[]
graph TD
    A[મલ્ટિપ્લેક્સિંગ] {-{-}{} B[એનાલોગ મલ્ટિપ્લેક્સિંગ]}
    A {-{-}{} C[ડિજિટલ મલ્ટિપ્લેક્સિંગ]}
    
    B {-{-}{} D[FDM {-} ફ્રીક્વેન્સી ડિવિઝન]}
    B {-{-}{} E[WDM {-} વેવલેન્થ ડિવિઝન]}
    
    C {-{-}{} F[TDM {-} ટાઇમ ડિવિઝન]}
    C {-{-}{} G[CDM {-} કોડ ડિવિઝન]}
    C {-{-}{} H[SDM {-} સ્પેસ ડિવિઝન]}
    
    F {-{-}{} I[સિંક્રોનસ TDM]}
    F {-{-}{} J[એસિંક્રોનસ TDM]}
{Highlighting}
{Shaded}
\end{verbatim}
\end{center}

\textbf{વિગતવાર વર્ગીકરણ:}

{\def\LTcaptype{none} % do not increment counter
\begin{longtable}[]{@{}llll@{}}
\toprule\noalign{}
પ્રકાર & પદ્ધતિ & ડોમેન & ઉપયોગ \\
\midrule\noalign{}
\endhead
\bottomrule\noalign{}
\endlastfoot
\textbf{FDM} & ફ્રીક્વેન્સી વિભાજન & ફ્રીક્વેન્સી & રેડિયો, TV બ્રોડકાસ્ટિંગ \\
\textbf{TDM} & ટાઇમ સ્લોટ એલોકેશન & સમય & ડિજિટલ ટેલિફોની \\
\textbf{CDM} & કોડ વિભાજન & કોડ & CDMA સેલ્યુલર \\
\textbf{WDM} & વેવલેન્થ વિભાજન & વેવલેન્થ & ઓપ્ટિકલ ફાઇબર \\
\textbf{SDM} & સ્પેસ વિભાજન & સ્પેસ & MIMO સિસ્ટમ્સ \\
\end{longtable}
}

\textbf{સિંક્રોનસ વિ એસિંક્રોનસ TDM:}

{\def\LTcaptype{none} % do not increment counter
\begin{longtable}[]{@{}lll@{}}
\toprule\noalign{}
પેરામીટર & સિંક્રોનસ TDM & એસિંક્રોનસ TDM \\
\midrule\noalign{}
\endhead
\bottomrule\noalign{}
\endlastfoot
\textbf{ટાઇમ સ્લોટ્સ} & ફિક્સ્ડ એલોકેશન & ડાયનામિક એલોકેશન \\
\textbf{એફિશિયન્સી} & નીચી & ઊંચી \\
\textbf{જટિલતા} & સરળ & જટિલ \\
\textbf{બેન્ડવિડ્થ વેસ્ટ} & થઈ શકે છે & મિનિમલ \\
\end{longtable}
}

\textbf{પસંદગીના માપદંડો:}

\begin{itemize}
\tightlist
\item
  \textbf{સિગ્નલનો પ્રકાર}: એનાલોગ \rightarrow FDM, ડિજિટલ \rightarrow TDM
\item
  \textbf{બેન્ડવિડ્થ}: મર્યાદિત \rightarrow TDM, પુષ્કળ \rightarrow FDM
\item
  \textbf{સિંક્રોનાઇઝેશન}: મહત્વપૂર્ણ \rightarrow સિંક્રોનસ, લવચીક \rightarrow એસિંક્રોનસ
\item
  \textbf{ઉપયોગ}: અવાજ \rightarrow TDM, ડેટા \rightarrow સ્ટેટિસ્ટિકલ TDM
\end{itemize}

\textbf{આધુનિક ટેકનિકો:}

\begin{itemize}
\tightlist
\item
  \textbf{OFDM}: ઓર્થોગોનલ ફ્રીક્વેન્સી ડિવિઝન મલ્ટિપ્લેક્સિંગ
\item
  \textbf{MIMO}: મલ્ટિપલ ઇનપુટ મલ્ટિપલ આઉટપુટ
\item
  \textbf{કેરિયર એગ્રીગેશન}: મલ્ટિપલ ફ્રીક્વેન્સી બેન્ડ્સ
\end{itemize}

\textbf{યાદગાર વાક્ય:} ``FDM TDM CDM WDM SDM - ફ્રીક્વેન્સી ટાઇમ કોડ વેવ સ્પેસ
ડિવિઝન મલ્ટિપ્લેક્સિંગ''

\end{solutionbox}
\subsection*{પ્રશ્ન 5(ક અથવા) [7
ગુણ]}\label{uxaaauxab0uxab6uxaa8-5uxa95-uxa85uxaa5uxab5-7-uxa97uxaa3}

\textbf{કોડ ડિવિઝન મલ્ટિપ્લેક્સિંગ સર્કિટમાં મુશ્કેલી નિવારણ માટેની પ્રક્રિયાનું વર્ણન
કરો}

\begin{solutionbox}

\textbf{CDMA ટ્રબલશૂટિંગ પ્રક્રિયા:}

\textbf{1. સિસ્ટમ ઓવરવ્યુ ચેક:}

\begin{center}
\textbf{Mermaid Diagram (Code)}
\begin{verbatim}
{Shaded}
{Highlighting}[]
graph TD
    A[CDMA સિસ્ટમ] {-{-}{} B[ટ્રાન્સમીટર સેક્શન]}
    A {-{-}{} C[ચેનલ સેક્શન]}
    A {-{-}{} D[રીસીવર સેક્શન]}
    
    B {-{-}{} E[ડેટા ઇનપુટ બરાબર?]}
    B {-{-}{} F[PN કોડ જનરેશન બરાબર?]}
    B {-{-}{} G[સ્પ્રેડિંગ ફંકશન બરાબર?]}
    
    C {-{-}{} H[પાથ લોસ મેઝરમેન્ટ]}
    C {-{-}{} I[ઇન્ટરફેરન્સ ચેક]}
    
    D {-{-}{} J[કોરિલેશન બરાબર?]}
    D {-{-}{} K[ડિસ્પ્રેડિંગ બરાબર?]}
    D {-{-}{} L[ડેટા રિકવરી બરાબર?]}
{Highlighting}
{Shaded}
\end{verbatim}
\end{center}

\textbf{2. સ્ટેપ-બાય-સ્ટેપ ટ્રબલશૂટિંગ:}

{\def\LTcaptype{none} % do not increment counter
\begin{longtable}[]{@{}llll@{}}
\toprule\noalign{}
સ્ટેપ & પેરામીટર & ટેસ્ટ મેથડ & અપેક્ષિત પરિણામ \\
\midrule\noalign{}
\endhead
\bottomrule\noalign{}
\endlastfoot
\textbf{1} & ઇનપુટ ડેટા & ડેટા સ્ટ્રીમ વેરિફાઇ કરો & સ્વચ્છ ડિજિટલ સિગ્નલ \\
\textbf{2} & PN કોડ & કોડ જનરેશન ચેક કરો & યોગ્ય સિક્વેન્સ \\
\textbf{3} & સ્પ્રેડિંગ & XOR આઉટપુટ મોનિટર કરો & સ્પ્રેડ સ્પેક્ટ્રમ સિગ્નલ \\
\textbf{4} & ટ્રાન્સમિશન** & પાવર લેવલ માપો & પર્યાપ્ત સિગ્નલ સ્ટ્રેન્થ \\
\textbf{5} & રિસેપ્શન & પ્રાપ્ત સિગ્નલ ચેક કરો & નોઇઝ ફ્લોર ઉપર \\
\textbf{6} & કોરિલેશન & કોરિલેટર આઉટપુટ વેરિફાઇ કરો & યોગ્ય ટાઇમિંગ પર પીક \\
\textbf{7} & ડિસ્પ્રેડિંગ & લોકલ PN સાથે XOR ચેક કરો & ડિસ્પ્રેડ સિગ્નલ \\
\textbf{8} & ડેટા રિકવરી** & આઉટપુટ ડેટા વેરિફાઇ કરો & મૂળ ડેટા પુનઃપ્રાપ્ત \\
\end{longtable}
}

\textbf{3. સામાન્ય સમસ્યાઓ અને ઉકેલો:}

{\def\LTcaptype{none} % do not increment counter
\begin{longtable}[]{@{}
  >{\raggedright\arraybackslash}p{(\linewidth - 6\tabcolsep) * \real{0.2000}}
  >{\raggedright\arraybackslash}p{(\linewidth - 6\tabcolsep) * \real{0.2286}}
  >{\raggedright\arraybackslash}p{(\linewidth - 6\tabcolsep) * \real{0.3714}}
  >{\raggedright\arraybackslash}p{(\linewidth - 6\tabcolsep) * \real{0.2000}}@{}}
\toprule\noalign{}
\begin{minipage}[b]{\linewidth}\raggedright
સમસ્યા
\end{minipage} & \begin{minipage}[b]{\linewidth}\raggedright
લક્ષણો
\end{minipage} & \begin{minipage}[b]{\linewidth}\raggedright
સંભવિત કારણો
\end{minipage} & \begin{minipage}[b]{\linewidth}\raggedright
ઉકેલો
\end{minipage} \\
\midrule\noalign{}
\endhead
\bottomrule\noalign{}
\endlastfoot
\textbf{સિગ્નલ નથી} & ઝીરો આઉટપુટ & પાવર સપ્લાય નિષ્ફળતા & પાવર કનેક્શન્સ ચેક
કરો \\
\textbf{ઊંચો BER} & ઘણી બિટ એરર્સ & નબળો કોરિલેશન & ટાઇમિંગ/પાવર એડજસ્ટ
કરો \\
\textbf{ઇન્ટરફેરન્સ} & ડિગ્રેડેડ પર્ફોર્મન્સ & અન્ય યુઝર્સ/નોઇઝ & પાવર કન્ટ્રોલ
એડજસ્ટમેન્ટ \\
\textbf{સિંક લોસ} & અન્તરવાળો સિગ્નલ & PN કોડ મિસમેચ & કોડ સિક્વેન્સિસ વેરિફાઇ
કરો \\
\end{longtable}
}

\textbf{4. જરૂરી ટેસ્ટ ઇક્વિપમેન્ટ:}

{\def\LTcaptype{none} % do not increment counter
\begin{longtable}[]{@{}lll@{}}
\toprule\noalign{}
ઇક્વિપમેન્ટ & હેતુ & માપ \\
\midrule\noalign{}
\endhead
\bottomrule\noalign{}
\endlastfoot
\textbf{સ્પેક્ટ્રમ એનાલાઇઝર} & સિગ્નલ એનાલિસિસ & પાવર સ્પેક્ટ્રલ ડેન્સિટી \\
\textbf{BER ટેસ્ટર} & એરર મેઝરમેન્ટ & બિટ એરર રેટ \\
\textbf{પાવર મીટર} & પાવર મેઝરમેન્ટ & ટ્રાન્સમિટેડ/રીસીવ્ડ પાવર \\
\textbf{ઓસિલોસ્કોપ} & વેવફોર્મ એનાલિસિસ & ટાઇમ ડોમેન સિગ્નલ્સ \\
\textbf{વેક્ટર એનાલાઇઝર} & મોડ્યુલેશન ક્વાલિટી & EVM, કોન્સ્ટેલેશન \\
\end{longtable}
}

\textbf{5. મેઝરમેન્ટ પ્રક્રિયાઓ:}

\textbf{પ્રોસેસિંગ ગેઇન વેરિફિકેશન:}

\begin{verbatim}
Gp = 10 log_{1}_{0}(Rc/Rb) dB
જ્યાં: Rc = ચિપ રેટ, Rb = બિટ રેટ
\end{verbatim}

\textbf{BER વિ Eb/N0 મેઝરમેન્ટ:}

\begin{verbatim}
BER = Q(\sqrt(2Eb/N0))
વિવિધ પાવર લેવલ્સ પર માપો
\end{verbatim}

\textbf{નીયર-ફાર ઇફેક્ટ ચેક:}

\begin{itemize}
\tightlist
\item
  વિવિધ યુઝર્સના પાવર લેવલ્સ માપો
\item
  પાવર કન્ટ્રોલ ઓપરેશન વેરિફાઇ કરો
\item
  ડાયનામિક રેન્જ આવશ્યકતાઓ ચેક કરો
\end{itemize}

\textbf{6. પર્ફોર્મન્સ ઓપ્ટિમાઇઝેશન:}

{\def\LTcaptype{none} % do not increment counter
\begin{longtable}[]{@{}lll@{}}
\toprule\noalign{}
પેરામીટર & ઓપ્ટિમાઇઝેશન મેથડ & ટાર્ગેટ વેલ્યુ \\
\midrule\noalign{}
\endhead
\bottomrule\noalign{}
\endlastfoot
\textbf{પાવર કન્ટ્રોલ} & લૂપ ગેઇન એડજસ્ટ કરો & \pm1 dB ચોકસાઈ \\
\textbf{કોડ સિલેક્શન} & ઓર્થોગોનલ કોડ્સ પસંદ કરો & નીચો ક્રોસ-કોરિલેશન \\
\textbf{ટાઇમિંગ} & PN જનરેટર્સ સિંક્રોનાઇઝ કરો & \textless0.5 ચિપ ચોકસાઈ \\
\textbf{ફિલ્ટરિંગ} & સિગ્નલ્સ બેન્ડલિમિટ કરો & ISI મિનિમાઇઝ કરો \\
\end{longtable}
}

\textbf{7. ડોક્યુમેન્ટેશન:}

\begin{itemize}
\tightlist
\item
  બધા મેઝરમેન્ટ્સ રેકોર્ડ કરો
\item
  સમસ્યાના લક્ષણો ડોક્યુમેન્ટ કરો
\item
  લાગુ કરેલા ઉકેલો નોંધો
\item
  ટ્રબલશૂટિંગ લોગ બનાવો
\end{itemize}

\textbf{સિસ્ટેમેટિક એપ્રોચ:}

\begin{enumerate}
\tightlist
\item
  \textbf{આઇસોલેટ}: ખામીયુક્ત સેક્શન ઓળખો
\item
  \textbf{માપો}: યોગ્ય ટેસ્ટ ઇક્વિપમેન્ટનો ઉપયોગ કરો
\item
  \textbf{એનાલાઇઝ}: સ્પેસિફિકેશન્સ સાથે સરખાવો
\item
  \textbf{સુધારો}: યોગ્ય ઉકેલ લાગુ કરો
\item
  \textbf{વેરિફાઇ}: સમસ્યા ઉકેલાઈ હોવાની પુષ્ટિ કરો
\end{enumerate}

\textbf{સેફ્ટી કન્સિડરેશન્સ:}

\begin{itemize}
\tightlist
\item
  પાવર લેવલ્સ સુરક્ષિત મર્યાદામાં
\item
  યોગ્ય ગ્રાઉન્ડિંગ પ્રક્રિયાઓ
\item
  RF એક્સપોઝર ગાઇડલાઇન્સ
\item
  ઇક્વિપમેન્ટ કેલિબ્રેશન સ્ટેટસ
\end{itemize}

\textbf{યાદગાર વાક્ય:} ``CDMA ટ્રબલશૂટ - ડેટા, PN કોડ, સ્પ્રેડિંગ, ચેનલ, કોરિલેશન,
રિકવરી ચેક કરો''

\end{solutionbox}

\end{document}
