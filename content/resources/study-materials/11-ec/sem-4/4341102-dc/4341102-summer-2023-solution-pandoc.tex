\documentclass[10pt,a4paper]{article}

% content/resources/templates/preamble.tex
\usepackage[margin=0.6in]{geometry}
\author{Milav Dabgar}
\usepackage{amsmath,amssymb,amsthm}
\usepackage{booktabs}
\usepackage{multirow}
\usepackage{xcolor}
\usepackage{tcolorbox}
\tcbuselibrary{breakable,skins}
\usepackage[colorlinks=true,linkcolor=blue]{hyperref}
\usepackage{titlesec}
\usepackage{enumitem}
\usepackage{tikz}
\usepackage{pgfplots}
\usepackage{circuitikz}
\usepackage[version=4]{mhchem}
\usepackage{longtable}
\usepackage{array}
\usepackage{float}
\usepackage{caption}
\usepackage{listings}

\lstset{
  basicstyle=\small\ttfamily,
  breaklines=true,
  breakatwhitespace=false,
  postbreak=\mbox{\textcolor{red}{$\hookrightarrow$}\space},
  float=false,
  numbers=left,
  numberstyle=\tiny\color{gray},
  numbersep=10pt,
  xleftmargin=2em,
  keywordstyle=\color{blue},
  commentstyle=\color{green!60!black},
  stringstyle=\color{purple},
  backgroundcolor=\color{gray!5},
  showstringspaces=false,
  tabsize=2,
  captionpos=b,
  keepspaces=true,
  columns=flexible
}

\pgfplotsset{compat=1.18}
\usetikzlibrary{shapes,arrows,positioning,calc,patterns,decorations.pathmorphing,decorations.markings,arrows.meta}

% Color scheme
\definecolor{headcolor}{RGB}{0,102,204}
\definecolor{keycolor}{RGB}{220,20,60}
\definecolor{solutioncolor}{RGB}{34,139,34}
\definecolor{mnemoniccolor}{RGB}{148,0,211}
\definecolor{codecolor}{RGB}{0,0,100}

% Spacing
\setlength{\parskip}{3pt}
\setlist[itemize]{nosep}
\setlist[enumerate]{nosep}

% Title formatting
\titleformat{\section}{\Large\bfseries\color{headcolor}}{\thesection}{1em}{}
\titleformat{\subsection}{\large\bfseries\color{headcolor}}{\thesubsection}{1em}{}

% Pandoc tightlist compatibility
\providecommand{\tightlist}{%
  \setlength{\itemsep}{0pt}\setlength{\parskip}{0pt}}

% Pandoc longtable compatibility
\newcounter{none}
\def\thenone{}


% content/resources/templates/english-boxes.tex
% This file is currently empty - it exists to maintain consistency with the import structure.
% Add custom environments here if needed in the future.


\begin{document}

\begin{center}
{\Huge\bfseries\color{headcolor} Subject Name Solutions}\\[5pt]
{\LARGE 4341102 -- Summer 2023}\\[3pt]
{\large Semester 1 Study Material}\\[3pt]
{\normalsize\textit{Detailed Solutions and Explanations}}
\end{center}

\vspace{10pt}

\subsection*{Question 1(a) [3 marks]}\label{q1a}

\textbf{Define signal and give its classification.}

\begin{solutionbox}
A signal is a physical quantity that varies with time,
space, or any other independent variable and contains information.

\textbf{Classification of Signals:}

{\def\LTcaptype{none} % do not increment counter
\begin{longtable}[]{@{}ll@{}}
\toprule\noalign{}
Classification Criteria & Types of Signals \\
\midrule\noalign{}
\endhead
\bottomrule\noalign{}
\endlastfoot
\textbf{Time Domain} & Continuous-time signals, Discrete-time signals \\
\textbf{Amplitude} & Analog signals, Digital signals \\
\textbf{Nature} & Deterministic signals, Random signals \\
\textbf{Symmetry} & Even signals, Odd signals \\
\textbf{Energy/Power} & Energy signals, Power signals \\
\end{longtable}
}

\end{solutionbox}
\begin{mnemonicbox}
``CADEN'' (Continuous/Discrete, Analog/Digital,
Deterministic/Random, Even/Odd, Energy/Power)

\end{mnemonicbox}
\subsection*{Question 1(b) [4 marks]}\label{q1b}

\textbf{Explain continuous and discrete time signals.}

\begin{solutionbox}

{\def\LTcaptype{none} % do not increment counter
\begin{longtable}[]{@{}
  >{\raggedright\arraybackslash}p{(\linewidth - 2\tabcolsep) * \real{0.5102}}
  >{\raggedright\arraybackslash}p{(\linewidth - 2\tabcolsep) * \real{0.4898}}@{}}
\toprule\noalign{}
\begin{minipage}[b]{\linewidth}\raggedright
Continuous-time Signals
\end{minipage} & \begin{minipage}[b]{\linewidth}\raggedright
Discrete-time Signals
\end{minipage} \\
\midrule\noalign{}
\endhead
\bottomrule\noalign{}
\endlastfoot
Defined for all values of time & Defined only at specific time
instants \\
Represented as x(t) & Represented as x[n] or x(nT) \\
Example: Analog signals like sinusoidal wave & Example: Digital signals
like sampled speech \\
Continuous curve on graph & Series of points on graph \\
Processing requires analog circuits & Processing can be done with
digital processors \\
\end{longtable}
}

\textbf{Diagram:}

\begin{center}
\textbf{Mermaid Diagram (Code)}
\begin{verbatim}
{Shaded}
{Highlighting}[]
graph LR
    A[Signals] {-{-}{} B[Continuous{-}time]}
    A {-{-}{} C[Discrete{-}time]}
    B {-{-}{} D[Defined for all t]}
    C {-{-}{} E[Defined at specific instants nT]}
    D {-{-}{} F["Example: sin(t)"]}
    E {-{-}{} G["Example: sin(nT)"]}
{Highlighting}
{Shaded}
\end{verbatim}
\end{center}

\end{solutionbox}
\begin{mnemonicbox}
``CAD'' - Continuous signals are Analog and Defined
for all time; Discrete signals are digital and defined at specific
points.

\end{mnemonicbox}
\subsection*{Question 1(c) [7 marks]}\label{q1c}

\textbf{Explain Unit Impulse and Unit Step function.}

\begin{solutionbox}

{\def\LTcaptype{none} % do not increment counter
\begin{longtable}[]{@{}
  >{\raggedright\arraybackslash}p{(\linewidth - 2\tabcolsep) * \real{0.5263}}
  >{\raggedright\arraybackslash}p{(\linewidth - 2\tabcolsep) * \real{0.4737}}@{}}
\toprule\noalign{}
\begin{minipage}[b]{\linewidth}\raggedright
Unit Impulse Function (δ(t))
\end{minipage} & \begin{minipage}[b]{\linewidth}\raggedright
Unit Step Function (u(t))
\end{minipage} \\
\midrule\noalign{}
\endhead
\bottomrule\noalign{}
\endlastfoot
Infinitely high at t=0, zero elsewhere & Value is 1 for t\geq0, 0 for
t\textless0 \\
Area under curve = 1 & Integral gives ramp function \\
Used to represent instantaneous events & Used to represent sudden
transitions \\
Mathematical basis for LTI system analysis & Used for system response
analysis \\
Laplace transform = 1 & Laplace transform = 1/s \\
\end{longtable}
}

\textbf{Diagram:}

\begin{verbatim}
            |     \^{}
            |     |
            |     |
Unit        |     |               Unit
Impulse     | δ(t)| Area = 1      Step        u(t)
            |     |               Function    {-{-}{-}{-}{-}{-}{-}{-}{-}{-}}
            |     |                           |
            +{-{-}{-}{-}{-}+{-}{-}{-}{-}{-}{-}{-}                   +{-}{-}{-}{-}{-}{-}{-}{-}{-}{-}{-}{-}}
           {-1  0  1  t                       {-}1  0  1  t    }
\end{verbatim}

\textbf{Properties:}

\begin{itemize}
\tightlist
\item
  \textbf{Sampling property}: \intf(t)δ(t-t_{0})dt = f(t_{0})
\item
  \textbf{Unit step is integral of impulse}: u(t) = \intδ(τ)dτ from -\infty to t
\item
  \textbf{Impulse is derivative of unit step}: δ(t) = du(t)/dt
\end{itemize}

\end{solutionbox}
\begin{mnemonicbox}
``SHARP-FLAT'' - Impulse is Sharp and momentary; Step
is Flat and persistent.

\end{mnemonicbox}
\subsection*{Question 1(c) OR [7
marks]}\label{q1c}

\textbf{Explain block diagram of digital communication system.}

\begin{solutionbox}

\textbf{Block Diagram of Digital Communication System:}

\begin{verbatim}
flowchart LR
    A[Source] {-{-} B[Source Encoder]}
    B {-{-} C[Channel Encoder]}
    C {-{-} D[Digital Modulator]}
    D {-{-} E[Channel]}
    E {-{-} F[Digital Demodulator]}
    F {-{-} G[Channel Decoder]}
    G {-{-} H[Source Decoder]}
    H {-{-} I[Destination]}
\end{verbatim}

\textbf{Explanation:}

{\def\LTcaptype{none} % do not increment counter
\begin{longtable}[]{@{}
  >{\raggedright\arraybackslash}p{(\linewidth - 2\tabcolsep) * \real{0.4118}}
  >{\raggedright\arraybackslash}p{(\linewidth - 2\tabcolsep) * \real{0.5882}}@{}}
\toprule\noalign{}
\begin{minipage}[b]{\linewidth}\raggedright
Block
\end{minipage} & \begin{minipage}[b]{\linewidth}\raggedright
Function
\end{minipage} \\
\midrule\noalign{}
\endhead
\bottomrule\noalign{}
\endlastfoot
\textbf{Source} & Generates the message to be transmitted \\
\textbf{Source Encoder} & Converts message to digital form, removes
redundancy \\
\textbf{Channel Encoder} & Adds controlled redundancy for error
detection/correction \\
\textbf{Digital Modulator} & Maps digital bits to signals suitable for
transmission \\
\textbf{Channel} & Physical medium through which signal travels \\
\textbf{Digital Demodulator} & Recovers digital data from received
signal \\
\textbf{Channel Decoder} & Detects/corrects errors using added
redundancy \\
\textbf{Source Decoder} & Reconstructs original message from received
bits \\
\textbf{Destination} & Receives the transmitted message \\
\end{longtable}
}

\end{solutionbox}
\begin{mnemonicbox}
``SECDCSD'' - ``Seven Engineers Can Design
Communication Systems Diligently''

\end{mnemonicbox}
\subsection*{Question 2(a) [3 marks]}\label{q2a}

\textbf{A signal has a bit rate of 8000 bit/second and a baud rate of
1000 baud. How many data elements are carried by each signal element?}

\begin{solutionbox}

Number of data elements (bits) per signal element: = Bit rate \div Baud
rate = 8000 bits/second \div 1000 baud = 8 bits/signal element


{\def\LTcaptype{none} % do not increment counter
\begin{longtable}[]{@{}lll@{}}
\toprule\noalign{}
Parameter & Value & Relation \\
\midrule\noalign{}
\endhead
\bottomrule\noalign{}
\endlastfoot
Bit rate & 8000 bits/sec & Given \\
Baud rate & 1000 baud & Given \\
Bits/signal & 8 bits & Bit rate \div Baud rate \\
\end{longtable}
}

\end{solutionbox}
\begin{mnemonicbox}
``Bits Divided By Bauds'' (BDBB)

\end{mnemonicbox}
\subsection*{Question 2(b) [4 marks]}\label{q2b}

\textbf{Explain Energy and power signals.}

\begin{solutionbox}

{\def\LTcaptype{none} % do not increment counter
\begin{longtable}[]{@{}
  >{\raggedright\arraybackslash}p{(\linewidth - 2\tabcolsep) * \real{0.5161}}
  >{\raggedright\arraybackslash}p{(\linewidth - 2\tabcolsep) * \real{0.4839}}@{}}
\toprule\noalign{}
\begin{minipage}[b]{\linewidth}\raggedright
Energy Signals
\end{minipage} & \begin{minipage}[b]{\linewidth}\raggedright
Power Signals
\end{minipage} \\
\midrule\noalign{}
\endhead
\bottomrule\noalign{}
\endlastfoot
Finite total energy & Infinite total energy but finite average power \\
Zero average power & Non-zero average power \\
E = \int\textbar x(t)\textbar^{2}dt (finite) &

P = lim(T\rightarrow\infty) 1/2T

\int\textbar x(t)\textbar^{2}dt (finite) \\
Examples: Pulse, Decaying exponential & Examples: Sine wave, Square
wave \\
Localized in time & Exist for all time \\
\end{longtable}
}

\textbf{Diagram:}

\begin{center}
\textbf{Mermaid Diagram (Code)}
\begin{verbatim}
{Shaded}
{Highlighting}[]
graph TD
    A[Signals] {-{-}{} B[Energy Signals]}
    A {-{-}{} C[Power Signals]}
    B {-{-}{} D[Finite Energy]}
    B {-{-}{} E[Zero Average Power]}
    C {-{-}{} F[Infinite Energy]}
    C {-{-}{} G[Finite Average Power]}
    D {-{-}{} H[Example: Pulse]}
    G {-{-}{} I[Example: Sine Wave]}
{Highlighting}
{Shaded}
\end{verbatim}
\end{center}

\end{solutionbox}
\begin{mnemonicbox}
``FEZIL'' - Finite Energy is Zero in Long-term; Power
signals are Infinite in Length

\end{mnemonicbox}
\subsection*{Question 2(c) [7 marks]}\label{q2c}

\textbf{Explain the block diagram of FSK modulator and de-modulator with
waveform.}

\begin{solutionbox}

\textbf{FSK Modulator and Demodulator:}

\begin{verbatim}
flowchart TD
    subgraph Modulator
        A[Digital Input] {-{-} B[Voltage Controlled Oscillator]}
        B {-{-} C[FSK Output]}
    end
    subgraph Demodulator
        D[FSK Input] {-{-} E[Bandpass Filter 1nFrequency f1]}
        D {-{-} F[Bandpass Filter 2nFrequency f2]}
        E {-{-} G[Envelope Detector 1]}
        F {-{-} H[Envelope Detector 2]}
        G {-{-} I[Comparator]}
        H {-{-} I}
        I {-{-} J[Digital Output]}
    end
\end{verbatim}

\textbf{Waveforms:}

\begin{verbatim}
Digital Input: \_\_\_\_\_‾‾‾‾‾\_\_\_\_\_‾‾‾‾‾\_\_\_\_\_
                 0     1     0     1     0

FSK Output:  /{//MMMMMM///MMMMMM///}
             f1    f2    f1    f2    f1
             
Received at BPF1: /{//\_\_\_\_\_///\_\_\_\_\_///}
                   f1          f1          f1
                   
Received at BPF2: \_\_\_\_\_MMMMMM\_\_\_\_\_MMMMMM\_\_\_\_\_
                        f2          f2
                        
Digital Output: \_\_\_\_\_‾‾‾‾‾\_\_\_\_\_‾‾‾‾‾\_\_\_\_\_
                 0     1     0     1     0
\end{verbatim}

\textbf{Key Principles:}

\begin{itemize}
\tightlist
\item
  \textbf{Bit 0}: Transmitted as frequency f_{1}
\item
  \textbf{Bit 1}: Transmitted as frequency f_{2}
\item
  \textbf{Demodulation}: Uses bandpass filters to separate frequencies
\item
  \textbf{Detection}: Envelope detectors recover the digital signal
\end{itemize}

\end{solutionbox}
\begin{mnemonicbox}
``FIST'' - Frequency Is Shifted for Transmission

\end{mnemonicbox}
\subsection*{Question 2(a) OR [3
marks]}\label{q2a}

\textbf{A signal carries 4 bit/signal elements. If 1000 signal elements
sent per second. Find the bit rate.}

\begin{solutionbox}

Bit rate = Number of bits per signal element \times Signal elements per
second Bit rate = 4 bits/signal element \times 1000 signal elements/second
Bit rate = 4000 bits/second


{\def\LTcaptype{none} % do not increment counter
\begin{longtable}[]{@{}lll@{}}
\toprule\noalign{}
Parameter & Value & Relation \\
\midrule\noalign{}
\endhead
\bottomrule\noalign{}
\endlastfoot
Bits per symbol & 4 & Given \\
Symbol rate & 1000 symbols/sec & Given \\
Bit rate & 4000 bits/sec & Bits/symbol \times Symbol rate \\
\end{longtable}
}

\end{solutionbox}
\begin{mnemonicbox}
``BBS'' - Bit rate equals Bits per symbol times
Symbol rate

\end{mnemonicbox}
\subsection*{Question 2(b) OR [4
marks]}\label{q2b}

\textbf{Explain Even and Odd signals.}

\begin{solutionbox}

{\def\LTcaptype{none} % do not increment counter
\begin{longtable}[]{@{}ll@{}}
\toprule\noalign{}
Even Signals & Odd Signals \\
\midrule\noalign{}
\endhead
\bottomrule\noalign{}
\endlastfoot
Symmetric around y-axis & Anti-symmetric around y-axis \\
x(-t) = x(t) & x(-t) = -x(t) \\
Example: cos(t) & Example: sin(t) \\
Fourier transform is real & Fourier transform is imaginary \\
Sum of even signals is even & Sum of odd signals is odd \\
\end{longtable}
}

\textbf{Diagram:}

\begin{verbatim}
Even Signal x(t)             Odd Signal x(t)
    |                            |
    |     *     *                |     *
    |   *         *              |   *   {}
    | *             *            | *      {}
{-{-}{-}{-}+{-}{-}{-}{-}{-}{-}{-}{-}{-}{-}{-}{-}{-}{-}{-}+{-}{-}{-}{-}   {-}{-}{-}{-}+{-}{-}{-}{-}{-}{-}{-}{-}+{-}{-}{-}{-}+{-}{-}{-}{-}}
    |                            |        /     *
    |                            |      /   *
    |                            |     *
\end{verbatim}

\textbf{Properties:}

\begin{itemize}
\tightlist
\item
  Any signal can be expressed as sum of even and odd components
\item
  Even component: x_{1}(t) = [x(t) + x(-t)]/2
\item
  Odd component: x_{2}(t) = [x(t) - x(-t)]/2
\end{itemize}

\end{solutionbox}
\begin{mnemonicbox}
``SAME-FLIP'' - Even signals are the SAME when
flipped; Odd signals FLIP their sign.

\end{mnemonicbox}
\subsection*{Question 2(c) OR [7
marks]}\label{q2c}

\textbf{Explain the block diagram of QPSK modulator and de-modulator
with constellation diagram.}

\begin{solutionbox}

\textbf{QPSK Modulator and Demodulator:}

\begin{verbatim}
flowchart TD
    subgraph Modulator
        A[Binary Input] {-{-} B[Serial to ParallelnConverter]}
        B {-{-} C[Even Bits]}
        B {-{-} D[Odd Bits]}
        C {-{-} E[Multiplier]}
        D {-{-} F[Multiplier]}
        G["cos(2πft)"] {-{-} E}
        H["sin(2πft)"] {-{-} F}
        E {-{-} I[Summer]}
        F {-{-} I}
        I {-{-} J[QPSK Output]}
    end

    subgraph Demodulator
        K[QPSK Input] {-{-} L[Multiplier 1]}
        K {-{-} M[Multiplier 2]}
        N["cos(2πft)"] {-{-} L}
        O["sin(2πft)"] {-{-} M}
        L {-{-} P[Integrator 1]}
        M {-{-} Q[Integrator 2]}
        P {-{-} R[Decision Device 1]}
        Q {-{-} S[Decision Device 2]}
        R {-{-} T[Parallel to SerialnConverter]}
        S {-{-} T}
        T {-{-} U[Binary Output]}
    end
\end{verbatim}

\textbf{Constellation Diagram:}

\begin{verbatim}
             Q
             |
      01     |     00
      •      |      •
             |
{-{-}{-}{-}{-}{-}{-}{-}{-}{-}{-}{-}{-}+{-}{-}{-}{-}{-}{-}{-}{-}{-}{-}{-}{-}{-}}
             |
      11     |     10
      •      |      •
             |             I
\end{verbatim}

\textbf{Key Characteristics:}

\begin{itemize}
\tightlist
\item
  \textbf{Input}: 2 bits determine each symbol
\item
  \textbf{Phases}: 4 phases (0^\circ, 90^\circ, 180^\circ, 270^\circ)
\item
  \textbf{Bits to phases}:

  \begin{itemize}
  \tightlist
  \item
    00: 45^\circ
  \item
    01: 135^\circ
  \item
    11: 225^\circ
  \item
    10: 315^\circ
  \end{itemize}
\item
  \textbf{Bandwidth efficiency}: 2 bits per symbol
\end{itemize}

\end{solutionbox}
\begin{mnemonicbox}
``QUADrature'' - 4 phases for 4 possible 2-bit
combinations

\end{mnemonicbox}
\subsection*{Question 3(a) [3 marks]}\label{q3a}

\textbf{Explain the working of ASK modulator with block diagram and
output waveforms.}

\begin{solutionbox}

\textbf{ASK Modulator Block Diagram:}

\begin{verbatim}
flowchart LR
    A[Digital Input] {-{-} B[Multiplier]}
    C["Carrier Generator{nsin(2πft)"] {-}{-} B}
    B {-{-} D[ASK Output]}
\end{verbatim}

\textbf{Waveforms:}

\begin{verbatim}
Digital Input: \_\_\_\_\_‾‾‾‾‾\_\_\_\_\_‾‾‾‾‾\_\_\_\_\_
                 0     1     0     1     0

Carrier:      /{/////////////}

ASK Output:   \_\_\_\_\_/{//\_\_\_\_\_///\_\_\_\_\_}
                 0     1     0     1     0
\end{verbatim}

\textbf{Working Principle:}

\begin{itemize}
\tightlist
\item
  Digital 1: Carrier signal is transmitted
\item
  Digital 0: No signal (or low amplitude) is transmitted
\item
  Output amplitude varies with input digital signal
\end{itemize}

\end{solutionbox}
\begin{mnemonicbox}
``ASKY'' - Amplitude Switches the Carrier? Yes!

\end{mnemonicbox}
\subsection*{Question 3(b) [4 marks]}\label{q3b}

\textbf{Draw the constellation diagram of 8-PSK and 16-QAM.}

\begin{solutionbox}

\textbf{8-PSK Constellation Diagram:}

\begin{verbatim}
                  Q
                  |
        011  •    |    • 000
              {   |   /}
         110 •  { | /  • 001}
                 {|/}
        {-{-}{-}{-}{-}{-}{-}{-}{-}+{-}{-}{-}{-}{-}{-}{-}{-}{-} I}
                /|{}
         101 •  / | {  • 010}
              /   |   {}
        100  •    |    • 011
                  |
\end{verbatim}

\textbf{16-QAM Constellation Diagram:}

\begin{verbatim}
                Q
        •   •   |   •   •
                |
        •   •   |   •   •
                |
        {-{-}{-}{-}{-}{-}{-}{-}+{-}{-}{-}{-}{-}{-}{-}{-}}
                |
        •   •   |   •   •
                |
        •   •   |   •   •
                |         I
\end{verbatim}

\textbf{Key Differences:}

\begin{itemize}
\tightlist
\item
  \textbf{8-PSK}: 8 symbols, equal amplitude, phases at 45^\circ intervals
\item
  \textbf{16-QAM}: 16 symbols, varying amplitudes and phases
\end{itemize}

\end{solutionbox}
\begin{mnemonicbox}
``P-Phase Q-Quantity'' - PSK varies Phase only; QAM
varies both amplitude (Quantity) and phase

\end{mnemonicbox}
\subsection*{Question 3(c) [7 marks]}\label{q3c}

\textbf{Draw the ASK and FSK modulation waveform for the sequence of
1100101101.}

\begin{solutionbox}

\textbf{Modulation Waveforms:}

\begin{verbatim}
Binary Input:  ‾‾‾‾‾‾‾‾‾‾\_\_\_\_\_‾‾‾‾‾\_\_\_\_\_‾‾‾‾‾‾‾‾‾‾
               1  1  0  0  1  0  1  1  0  1
               
Carrier:       /{/////////////////}

ASK Output:    /{/////\_\_\_\_\_///\_\_\_\_\_/////}
               1  1  0  0  1  0  1  1  0  1
               
FSK Output:    MMMMMMMMMM/{///MMMMM////MMMMMMMMMM}
               1  1  0  0  1  0  1  1  0  1
               f2 f2 f1 f1 f2 f1 f2 f2 f1 f2
\end{verbatim}

\textbf{Key Characteristics:}

\begin{itemize}
\tightlist
\item
  \textbf{ASK}: Carrier present for bit 1, absent for bit 0
\item
  \textbf{FSK}: Higher frequency (f_{2}) for bit 1, lower frequency (f_{1})
  for bit 0
\end{itemize}

\textbf{Table of Modulation Methods:}

{\def\LTcaptype{none} % do not increment counter
\begin{longtable}[]{@{}llll@{}}
\toprule\noalign{}
Modulation & Bit 0 & Bit 1 & Parameter Varied \\
\midrule\noalign{}
\endhead
\bottomrule\noalign{}
\endlastfoot
ASK & Zero or low amplitude & High amplitude & Amplitude \\
FSK & Frequency f_{1} & Frequency f_{2} & Frequency \\
\end{longtable}
}

\end{solutionbox}
\begin{mnemonicbox}
``AFRO'' - Amplitude For 1, Remove for 0 (ASK);
Frequency Rises for 1, Off-peak for 0 (FSK)

\end{mnemonicbox}
\subsection*{Question 3(a) OR [3
marks]}\label{q3a}

\textbf{Explain the working of PSK modulator with block diagram and
output waveforms.}

\begin{solutionbox}

\textbf{PSK Modulator Block Diagram:}

\begin{verbatim}
flowchart LR
    A[Digital Input] {-{-} B[Polar Convertern0-1, 1+1]}
    B {-{-} C[Multiplier]}
    D["Carrier Generator{nsin(2πft)"] {-}{-} C}
    C {-{-} E[PSK Output]}
\end{verbatim}

\textbf{Waveforms:}

\begin{verbatim}
Digital Input: \_\_\_\_\_‾‾‾‾‾\_\_\_\_\_‾‾‾‾‾\_\_\_\_\_
                 0     1     0     1     0

Carrier:      /{/////////////}

PSK Output:   {//////////////}
               0     1     0     1     0
               180^  0^   180^  0^   180^
\end{verbatim}

\textbf{Working Principle:}

\begin{itemize}
\tightlist
\item
  Digital 1: Carrier signal with 0^\circ phase
\item
  Digital 0: Carrier signal with 180^\circ phase (inverted)
\item
  Amplitude remains constant, only phase changes
\end{itemize}

\end{solutionbox}
\begin{mnemonicbox}
``PSKIT'' - Phase Shift Keeps Information True

\end{mnemonicbox}
\subsection*{Question 3(b) OR [4
marks]}\label{q3b}

\textbf{Draw the MSK modulation waveform for the sequence of
1101001101.}

\begin{solutionbox}

\textbf{MSK Modulation Waveform:}

\begin{verbatim}
Binary Input:  ‾‾‾‾‾\_\_\_\_\_‾‾‾‾‾\_\_\_\_\_‾‾‾‾‾‾‾‾‾‾
               1  1  0  1  0  0  1  1  0  1
               
MSK Output:    {//MMMMM//MMMMM////MMMMM}
               1  1  0  1  0  0  1  1  0  1
\end{verbatim}

\textbf{Characteristics of MSK:}

\begin{itemize}
\tightlist
\item
  Continuous phase transitions (no phase jumps)
\item
  Frequency shifts between f_{1} and f_{2}
\item
  Minimum frequency separation: Δf = 1/(2T)
\item
  Smoother transitions than FSK
\end{itemize}


{\def\LTcaptype{none} % do not increment counter
\begin{longtable}[]{@{}ll@{}}
\toprule\noalign{}
Feature & MSK Characteristic \\
\midrule\noalign{}
\endhead
\bottomrule\noalign{}
\endlastfoot
Phase continuity & Continuous, no abrupt changes \\
Frequency deviation & Minimum possible (1/2T) \\
Spectral efficiency & Better than conventional FSK \\
Bandwidth & 1.5 times bit rate \\
\end{longtable}
}

\end{solutionbox}
\begin{mnemonicbox}
``MINIMUM SMOOTH'' - MSK uses Minimum frequency
separation with Smooth transitions

\end{mnemonicbox}
\subsection*{Question 3(c) OR [7
marks]}\label{q3c}

\textbf{Draw BPSK and QPSK modulation waveform for 1100101011.}

\begin{solutionbox}

\textbf{BPSK and QPSK Modulation Waveforms:}

\begin{verbatim}
Binary Input:     ‾‾‾‾‾‾‾‾‾‾\_\_\_\_\_‾‾‾‾‾\_\_\_\_\_‾‾‾‾‾‾‾‾‾‾
                  1  1  0  0  1  0  1  0  1  1
                  
BPSK Output:      /{/////////////////}
                  0^  0^ 180^180^ 0^ 180^ 0^ 180^ 0^  0^
                  
QPSK (I channel): /{//\_\_\_\_\_///\_\_\_\_\_///}
                  11   00    10    01    11
                  
QPSK (Q channel): /{/////\_\_\_\_\_///\_\_\_\_\_}
                  11   00    10    01    11
                  
QPSK (combined):  {///MMMMM///MMMMM///}
                  11   00    10    01    11
\end{verbatim}

\textbf{Key Differences:}

\begin{itemize}
\tightlist
\item
  \textbf{BPSK}: 1 bit per symbol, 2 phases (0^\circ and 180^\circ)
\item
  \textbf{QPSK}: 2 bits per symbol, 4 phases (45^\circ, 135^\circ, 225^\circ, 315^\circ)
\item
  \textbf{QPSK Pairs}: 00, 01, 10, 11 map to different phases
\end{itemize}


{\def\LTcaptype{none} % do not increment counter
\begin{longtable}[]{@{}llll@{}}
\toprule\noalign{}
Modulation & Bits/Symbol & Number of Phases & Bandwidth Efficiency \\
\midrule\noalign{}
\endhead
\bottomrule\noalign{}
\endlastfoot
BPSK & 1 & 2 & 1 bit/Hz \\
QPSK & 2 & 4 & 2 bits/Hz \\
\end{longtable}
}

\end{solutionbox}
\begin{mnemonicbox}
``ONE-TWO'' - ONE bit for BPSK, TWO bits for QPSK

\end{mnemonicbox}
\subsection*{Question 4(a) [3 marks]}\label{q4a}

\textbf{Encode the data using Huffman code for below probability
sequence. P = \{ 0.4, 0.2, 0.2, 0.1, 0.1\}}

\begin{solutionbox}

\textbf{Huffman Coding Process:}

{\def\LTcaptype{none} % do not increment counter
\begin{longtable}[]{@{}lll@{}}
\toprule\noalign{}
Symbol & Probability & Huffman Code \\
\midrule\noalign{}
\endhead
\bottomrule\noalign{}
\endlastfoot
A & 0.4 & 0 \\
B & 0.2 & 10 \\
C & 0.2 & 11 \\
D & 0.1 & 110 \\
E & 0.1 & 111 \\
\end{longtable}
}

\textbf{Huffman Tree:}

\begin{verbatim}
                [1.0]
               /     {}
              /       {}
           [0.6]      [0.4] A:0
          /     {}
         /       {}
      [0.4]     [0.2] B:10
     /     {}
    /       {}
 [0.2] C:11 [0.2]
           /     {}
          /       {}
     [0.1] D:110 [0.1] E:111
\end{verbatim}

\end{solutionbox}
\begin{mnemonicbox}
``Higher Probability Means Shorter Code''

\end{mnemonicbox}
\subsection*{Question 4(b) [4 marks]}\label{q4b}

\textbf{Define Probability and Entropy.}

\begin{solutionbox}

{\def\LTcaptype{none} % do not increment counter
\begin{longtable}[]{@{}
  >{\raggedright\arraybackslash}p{(\linewidth - 6\tabcolsep) * \real{0.2093}}
  >{\raggedright\arraybackslash}p{(\linewidth - 6\tabcolsep) * \real{0.2791}}
  >{\raggedright\arraybackslash}p{(\linewidth - 6\tabcolsep) * \real{0.2093}}
  >{\raggedright\arraybackslash}p{(\linewidth - 6\tabcolsep) * \real{0.3023}}@{}}
\toprule\noalign{}
\begin{minipage}[b]{\linewidth}\raggedright
Concept
\end{minipage} & \begin{minipage}[b]{\linewidth}\raggedright
Definition
\end{minipage} & \begin{minipage}[b]{\linewidth}\raggedright
Formula
\end{minipage} & \begin{minipage}[b]{\linewidth}\raggedright
Significance
\end{minipage} \\
\midrule\noalign{}
\endhead
\bottomrule\noalign{}
\endlastfoot
\textbf{Probability} & Measure of likelihood of an event occurring &
P(A) = Number of favorable outcomes / Total number of possible outcomes
& Used to model uncertainty in communication \\
\textbf{Entropy} & Measure of uncertainty or randomness in a system &
H(X) = -\sum P(xi) log_{2} P(xi) & Indicates average information content \\
\end{longtable}
}

\textbf{Key Characteristics:}

\begin{itemize}
\tightlist
\item
  \textbf{Probability Range}: 0 \leq P(A) \leq 1
\item
  \textbf{Entropy Units}: Bits (using log_{2})
\item
  \textbf{Maximum Entropy}: When all events are equally likely
\item
  \textbf{Minimum Entropy}: When outcome is certain (probability = 1)
\end{itemize}

\end{solutionbox}
\begin{mnemonicbox}
``PURE'' - Probability Underpins Randomness
Estimation

\end{mnemonicbox}
\subsection*{Question 4(c) [7 marks]}\label{q4c}

\textbf{Explain CDMA technique in detail.}

\begin{solutionbox}

\textbf{CDMA (Code Division Multiple Access):}

\begin{verbatim}
flowchart LR
    A[User Data] {-{-} B[Spreadingnwith Unique Code]}
    B {-{-} C[Modulation]}
    C {-{-} D[Transmission]}
    D {-{-} E[Reception]}
    E {-{-} F[Demodulation]}
    F {-{-} G[Despreading withnMatching Code]}
    G {-{-} H[Original User Data]}
\end{verbatim}

\textbf{Table of CDMA Characteristics:}

{\def\LTcaptype{none} % do not increment counter
\begin{longtable}[]{@{}
  >{\raggedright\arraybackslash}p{(\linewidth - 2\tabcolsep) * \real{0.4091}}
  >{\raggedright\arraybackslash}p{(\linewidth - 2\tabcolsep) * \real{0.5909}}@{}}
\toprule\noalign{}
\begin{minipage}[b]{\linewidth}\raggedright
Feature
\end{minipage} & \begin{minipage}[b]{\linewidth}\raggedright
Description
\end{minipage} \\
\midrule\noalign{}
\endhead
\bottomrule\noalign{}
\endlastfoot
\textbf{Access Method} & Multiple users share same frequency and time \\
\textbf{Separation} & Users distinguished by unique spreading codes \\
\textbf{Spreading Codes} & Orthogonal or pseudo-orthogonal sequences \\
\textbf{Processing Gain} & Ratio of spread bandwidth to original
bandwidth \\
\textbf{Multiple Access} & Uses code space rather than frequency or time
division \\
\textbf{Interference Rejection} & Inherent ability to reject narrowband
interference \\
\end{longtable}
}

\textbf{Key Advantages:}

\begin{itemize}
\tightlist
\item
  \textbf{Capacity}: Higher than FDMA/TDMA in many scenarios
\item
  \textbf{Security}: Inherent encryption through spreading codes
\item
  \textbf{Multipath Rejection}: Rake receivers can combine multipath
  components
\item
  \textbf{Soft Handoff}: Mobile can communicate with multiple base
  stations
\end{itemize}

\end{solutionbox}
\begin{mnemonicbox}
``CODES'' - Capacity Optimized with Direct-sequence
Encoding Schemes

\end{mnemonicbox}
\subsection*{Question 4(a) OR [3
marks]}\label{q4a}

\textbf{Encode the data using Shanon Fano code for below probability
sequence. P = \{ 0.5, 0.25, 0.125, 0.125\}}

\begin{solutionbox}

\textbf{Shannon-Fano Coding Process:}

{\def\LTcaptype{none} % do not increment counter
\begin{longtable}[]{@{}lll@{}}
\toprule\noalign{}
Symbol & Probability & Shannon-Fano Code \\
\midrule\noalign{}
\endhead
\bottomrule\noalign{}
\endlastfoot
A & 0.5 & 0 \\
B & 0.25 & 10 \\
C & 0.125 & 110 \\
D & 0.125 & 111 \\
\end{longtable}
}

\textbf{Shannon-Fano Tree:}

\begin{verbatim}
               [1.0]
              /     {}
             /       {}
        [0.5] A      [0.5]
                    /     {}
                   /       {}
              [0.25] B     [0.25]
                          /     {}
                         /       {}
                  [0.125] C     [0.125] D
                   Code:110     Code:111
\end{verbatim}

\end{solutionbox}
\begin{mnemonicbox}
``Split For Optimum'' - Shannon-Fano splits groups
for optimum coding

\end{mnemonicbox}
\subsection*{Question 4(b) OR [4
marks]}\label{q4b}

\textbf{Define Information and Channel Capacity.}

\begin{solutionbox}

{\def\LTcaptype{none} % do not increment counter
\begin{longtable}[]{@{}
  >{\raggedright\arraybackslash}p{(\linewidth - 6\tabcolsep) * \real{0.2093}}
  >{\raggedright\arraybackslash}p{(\linewidth - 6\tabcolsep) * \real{0.2791}}
  >{\raggedright\arraybackslash}p{(\linewidth - 6\tabcolsep) * \real{0.2093}}
  >{\raggedright\arraybackslash}p{(\linewidth - 6\tabcolsep) * \real{0.3023}}@{}}
\toprule\noalign{}
\begin{minipage}[b]{\linewidth}\raggedright
Concept
\end{minipage} & \begin{minipage}[b]{\linewidth}\raggedright
Definition
\end{minipage} & \begin{minipage}[b]{\linewidth}\raggedright
Formula
\end{minipage} & \begin{minipage}[b]{\linewidth}\raggedright
Significance
\end{minipage} \\
\midrule\noalign{}
\endhead
\bottomrule\noalign{}
\endlastfoot
\textbf{Information} & Measure of reduction in uncertainty & I(x) =
-log_{2} P(x) & Less probable events carry more information \\
\textbf{Channel Capacity} & Maximum rate at which information can be
transmitted with arbitrarily small error & C = B log_{2}(1 + S/N) &
Fundamental limit of reliable communication \\
\end{longtable}
}

\textbf{Key Points:}

\begin{itemize}
\tightlist
\item
  \textbf{Information Units}: Bits (using log_{2})
\item
  \textbf{Channel Capacity Units}: Bits per second
\item
  \textbf{Factors Affecting Capacity}:

  \begin{itemize}
  \tightlist
  \item
    Bandwidth (B)
  \item
    Signal-to-Noise Ratio (S/N)
  \end{itemize}
\end{itemize}

\end{solutionbox}
\begin{mnemonicbox}
``INCHES'' - Information Numerically Calculated,
Hopping through Efficient Shannon limit

\end{mnemonicbox}
\subsection*{Question 4(c) OR [7
marks]}\label{q4c}

\textbf{Explain TDMA technique in detail.}

\begin{solutionbox}

\textbf{TDMA (Time Division Multiple Access):}

\begin{verbatim}
flowchart LR
    A[User 1] {-{-} B[Time Slot 1]}
    C[User 2] {-{-} D[Time Slot 2]}
    E[User 3] {-{-} F[Time Slot 3]}
    G[User 4] {-{-} H[Time Slot 4]}
    B {-{-} I[Multiplexer]}
    D {-{-} I}
    F {-{-} I}
    H {-{-} I}
    I {-{-} J[Transmission Channel]}
    J {-{-} K[Demultiplexer]}
    K {-{-} L[Time Slot 1]}
    K {-{-} M[Time Slot 2]}
    K {-{-} N[Time Slot 3]}
    K {-{-} O[Time Slot 4]}
    L {-{-} P[User 1]}
    M {-{-} Q[User 2]}
    N {-{-} R[User 3]}
    O {-{-} S[User 4]}
\end{verbatim}

\textbf{Table of TDMA Characteristics:}

{\def\LTcaptype{none} % do not increment counter
\begin{longtable}[]{@{}
  >{\raggedright\arraybackslash}p{(\linewidth - 2\tabcolsep) * \real{0.4091}}
  >{\raggedright\arraybackslash}p{(\linewidth - 2\tabcolsep) * \real{0.5909}}@{}}
\toprule\noalign{}
\begin{minipage}[b]{\linewidth}\raggedright
Feature
\end{minipage} & \begin{minipage}[b]{\linewidth}\raggedright
Description
\end{minipage} \\
\midrule\noalign{}
\endhead
\bottomrule\noalign{}
\endlastfoot
\textbf{Access Method} & Multiple users share same frequency at
different time slots \\
\textbf{Frame Structure} & Time divided into frames, frames into
slots \\
\textbf{Guard Time} & Short periods between slots to prevent overlap \\
\textbf{Synchronization} & Precise timing required between transmitter
and receiver \\
\textbf{Efficiency} & High spectrum utilization \\
\textbf{Power Consumption} & Transmitter on only during assigned
slots \\
\end{longtable}
}

\textbf{TDMA Frame Structure:}

\begin{verbatim}
|{{-}{-}{-}{-}{-}{-}{-}{-}{-}{-}{-}{-}{-}{-}{-}{-}{-}{-}{-} TDMA Frame {-}{-}{-}{-}{-}{-}{-}{-}{-}{-}{-}{-}{-}{-}{-}{-}{-}{-}{-}|}
| TS1 | TS2 | TS3 | TS4 | TS1 | TS2 | TS3 | TS4 | ...
|User1|User2|User3|User4|User1|User2|User3|User4| ...
\end{verbatim}

\end{solutionbox}
\begin{mnemonicbox}
``TIME'' - Transmission In Measured Epochs

\end{mnemonicbox}
\subsection*{Question 5(a) [3 marks]}\label{q5a}

\textbf{Explain T1 carrier system.}

\begin{solutionbox}

\textbf{T1 Carrier System:}

{\def\LTcaptype{none} % do not increment counter
\begin{longtable}[]{@{}ll@{}}
\toprule\noalign{}
Characteristic & Specification \\
\midrule\noalign{}
\endhead
\bottomrule\noalign{}
\endlastfoot
\textbf{Data Rate} & 1.544 Mbps \\
\textbf{Channels} & 24 voice channels \\
\textbf{Voice Sampling} & 8000 samples/second \\
\textbf{Sample Size} & 8 bits per sample \\
\textbf{Frame Size} & 193 bits (24\times8 + 1) \\
\textbf{Frame Rate} & 8000 frames/second \\
\end{longtable}
}

\textbf{T1 Frame Structure:}

\begin{verbatim}
|{{-}{-}{-}{-}{-}{-}{-}{-}{-}{-}{-}{-}{-}{-}{-} T1 Frame (193 bits) {-}{-}{-}{-}{-}{-}{-}{-}{-}{-}{-}{-}{-}{-}{-}{-}{-}{-}|}
| F | Ch1 | Ch2 | Ch3 | ... | Ch24 | F | Ch1 | Ch2 | ... |
| 1 |  8  |  8  |  8  | ... |  8   | 1 |  8  |  8  | ... |
\end{verbatim}

\end{solutionbox}
\begin{mnemonicbox}
``T1-24-8-8'' - T1 has 24 channels, 8 bits, 8kHz

\end{mnemonicbox}
\subsection*{Question 5(b) [4 marks]}\label{q5b}

\textbf{Explain Time Division Multiplexing technique (TDM) in detail.}

\begin{solutionbox}

\textbf{Time Division Multiplexing (TDM):}

\begin{verbatim}
flowchart LR
    A[Signal 1] {-{-} E[Multiplexer]}
    B[Signal 2] {-{-} E}
    C[Signal 3] {-{-} E}
    D[Signal 4] {-{-} E}
    E {-{-} F[Transmission Channel]}
    F {-{-} G[Demultiplexer]}
    G {-{-} H[Signal 1]}
    G {-{-} I[Signal 2]}
    G {-{-} J[Signal 3]}
    G {-{-} K[Signal 4]}
\end{verbatim}

\textbf{Table of TDM Characteristics:}

{\def\LTcaptype{none} % do not increment counter
\begin{longtable}[]{@{}
  >{\raggedright\arraybackslash}p{(\linewidth - 2\tabcolsep) * \real{0.4091}}
  >{\raggedright\arraybackslash}p{(\linewidth - 2\tabcolsep) * \real{0.5909}}@{}}
\toprule\noalign{}
\begin{minipage}[b]{\linewidth}\raggedright
Feature
\end{minipage} & \begin{minipage}[b]{\linewidth}\raggedright
Description
\end{minipage} \\
\midrule\noalign{}
\endhead
\bottomrule\noalign{}
\endlastfoot
\textbf{Principle} & Multiple signals share a single channel by taking
turns \\
\textbf{Time Allocation} & Each signal assigned a fixed time slot \\
\textbf{Synchronization} & Precise timing required between multiplexer
and demultiplexer \\
\textbf{Interleaving} & Samples from different sources interleaved in
time \\
\textbf{Types} & Synchronous TDM and Asynchronous (Statistical) TDM \\
\end{longtable}
}

\textbf{TDM Frame Structure:}

\begin{verbatim}
|{{-}{-}{-}{-}{-}{-}{-}{-}{-}{-}{-}{-}{-}{-}{-}{-} TDM Frame {-}{-}{-}{-}{-}{-}{-}{-}{-}{-}{-}{-}{-}{-}{-}{-}|}
| S1 | S2 | S3 | S4 | S1 | S2 | S3 | S4 | ... |
\end{verbatim}

\end{solutionbox}
\begin{mnemonicbox}
``TWIST'' - Time Windows Interleaving Signals
Together

\end{mnemonicbox}
\subsection*{Question 5(c) [7 marks]}\label{q5c}

\textbf{Explain security components of information security in detail.}

\begin{solutionbox}

\textbf{Information Security Components:}

\begin{center}
\textbf{Mermaid Diagram (Code)}
\begin{verbatim}
{Shaded}
{Highlighting}[]
graph TD
    A[Information Security] {-{-}{} B[Confidentiality]}
    A {-{-}{} C[Integrity]}
    A {-{-}{} D[Availability]}
    B {-{-}{} E[Encryption]}
    B {-{-}{} F[Access Control]}
    C {-{-}{} G[Digital Signatures]}
    C {-{-}{} H[Hashing]}
    D {-{-}{} I[Redundancy]}
    D {-{-}{} J[Backup Systems]}
{Highlighting}
{Shaded}
\end{verbatim}
\end{center}

\textbf{Table of Security Components:}

{\def\LTcaptype{none} % do not increment counter
\begin{longtable}[]{@{}
  >{\raggedright\arraybackslash}p{(\linewidth - 4\tabcolsep) * \real{0.2292}}
  >{\raggedright\arraybackslash}p{(\linewidth - 4\tabcolsep) * \real{0.2708}}
  >{\raggedright\arraybackslash}p{(\linewidth - 4\tabcolsep) * \real{0.5000}}@{}}
\toprule\noalign{}
\begin{minipage}[b]{\linewidth}\raggedright
Component
\end{minipage} & \begin{minipage}[b]{\linewidth}\raggedright
Description
\end{minipage} & \begin{minipage}[b]{\linewidth}\raggedright
Implementation Methods
\end{minipage} \\
\midrule\noalign{}
\endhead
\bottomrule\noalign{}
\endlastfoot
\textbf{Confidentiality} & Ensuring information is accessible only to
authorized users & Encryption, Access control, Authentication \\
\textbf{Integrity} & Maintaining accuracy and consistency of data &
Digital signatures, Hashing, Checksums \\
\textbf{Availability} & Ensuring information is accessible when needed &
Redundancy, Backup systems, Disaster recovery \\
\textbf{Authentication} & Verifying identity of users & Passwords,
Biometrics, Digital certificates \\
\textbf{Non-repudiation} & Preventing denial of sending/receiving
information & Digital signatures, Audit trails \\
\end{longtable}
}

\textbf{Common Security Threats:}

\begin{itemize}
\tightlist
\item
  \textbf{Malware}: Viruses, worms, trojans, ransomware
\item
  \textbf{Social Engineering}: Phishing, pretexting
\item
  \textbf{Man-in-the-Middle Attacks}: Intercepting communications
\item
  \textbf{Denial-of-Service}: Preventing legitimate access
\end{itemize}

\end{solutionbox}
\begin{mnemonicbox}
``CIA'' - Confidentiality, Integrity, Availability

\end{mnemonicbox}
\subsection*{Question 5(a) OR [3
marks]}\label{q5a}

\textbf{Explain E1 carrier system.}

\begin{solutionbox}

\textbf{E1 Carrier System:}

{\def\LTcaptype{none} % do not increment counter
\begin{longtable}[]{@{}ll@{}}
\toprule\noalign{}
Characteristic & Specification \\
\midrule\noalign{}
\endhead
\bottomrule\noalign{}
\endlastfoot
\textbf{Data Rate} & 2.048 Mbps \\
\textbf{Channels} & 32 time slots (30 voice + 2 signaling) \\
\textbf{Voice Sampling} & 8000 samples/second \\
\textbf{Sample Size} & 8 bits per sample \\
\textbf{Frame Size} & 256 bits (32\times8) \\
\textbf{Frame Rate} & 8000 frames/second \\
\end{longtable}
}

\textbf{E1 Frame Structure:}

\begin{verbatim}
|{{-}{-}{-}{-}{-}{-}{-}{-}{-}{-}{-}{-}{-}{-}{-}{-}{-} E1 Frame (256 bits) {-}{-}{-}{-}{-}{-}{-}{-}{-}{-}{-}{-}{-}{-}{-}{-}{-}|}
| TS0 | TS1 | TS2 | ... | TS15 | TS16 | TS17 | ... | TS31 |
|  8  |  8  |  8  | ... |  8   |  8   |  8   | ... |  8   |
\end{verbatim}

\textbf{Special Time Slots:}

\begin{itemize}
\tightlist
\item
  \textbf{TS0}: Frame alignment signal
\item
  \textbf{TS16}: Signaling channel
\end{itemize}

\end{solutionbox}
\begin{mnemonicbox}
``E1-32-8-8'' - E1 has 32 channels, 8 bits, 8kHz

\end{mnemonicbox}
\subsection*{Question 5(b) OR [4
marks]}\label{q5b}

\textbf{Explain Frequency Division Multiplexing technique (FDM) in
detail.}

\begin{solutionbox}

\textbf{Frequency Division Multiplexing (FDM):}

\begin{verbatim}
flowchart LR
    A[Signal 1] {-{-} B[Modulator 1nf1]}
    C[Signal 2] {-{-} D[Modulator 2nf2]}
    E[Signal 3] {-{-} F[Modulator 3nf3]}
    G[Signal 4] {-{-} H[Modulator 4nf4]}
    B {-{-} I[Combiner/Mixer]}
    D {-{-} I}
    F {-{-} I}
    H {-{-} I}
    I {-{-} J[Transmission Channel]}
    J {-{-} K[Filters/Separators]}
    K {-{-} L[Demodulator 1nf1]}
    K {-{-} M[Demodulator 2nf2]}
    K {-{-} N[Demodulator 3nf3]}
    K {-{-} O[Demodulator 4nf4]}
    L {-{-} P[Signal 1]}
    M {-{-} Q[Signal 2]}
    N {-{-} R[Signal 3]}
    O {-{-} S[Signal 4]}
\end{verbatim}

\textbf{Table of FDM Characteristics:}

{\def\LTcaptype{none} % do not increment counter
\begin{longtable}[]{@{}
  >{\raggedright\arraybackslash}p{(\linewidth - 2\tabcolsep) * \real{0.4091}}
  >{\raggedright\arraybackslash}p{(\linewidth - 2\tabcolsep) * \real{0.5909}}@{}}
\toprule\noalign{}
\begin{minipage}[b]{\linewidth}\raggedright
Feature
\end{minipage} & \begin{minipage}[b]{\linewidth}\raggedright
Description
\end{minipage} \\
\midrule\noalign{}
\endhead
\bottomrule\noalign{}
\endlastfoot
\textbf{Principle} & Multiple signals share a single channel by using
different frequency bands \\
\textbf{Guard Bands} & Unused frequency bands between channels to
prevent interference \\
\textbf{Channel Bandwidth} & Each signal allocated a specific frequency
range \\
\textbf{Implementation} & Uses modulators to shift signals to different
frequency bands \\
\textbf{Applications} & Radio broadcasting, television, cable systems \\
\end{longtable}
}

\textbf{FDM Spectrum:}

\begin{verbatim}
  Power
    \^{}
    |    \_\_\_      \_\_\_      \_\_\_      \_\_\_
    |   /   {    /       /       /   }
    |  /     {  /       /       /     }
    | /       {/       /       /       }
    +{-{-}{-}{-}{-}{-}{-}{-}{-}{-}{-}{-}{-}{-}{-}{-}{-}{-}{-}{-}{-}{-}{-}{-}{-}{-}{-}{-}{-}{-}{-}{-}{-}{-}{-}{-}{-}{-}{-}{-}{-} Frequency}
        Ch1      Ch2      Ch3      Ch4
      |{{-}{-}{-}|{-}|{-}{-}{-}|{-}|{-}{-}{-}|{-}|{-}{-}{-}|}
             GB      GB      GB
\end{verbatim}

\end{solutionbox}
\begin{mnemonicbox}
``FROG'' - FRequencies Organized with Gaps

\end{mnemonicbox}
\subsection*{Question 5(c) OR [7
marks]}\label{q5c}

\textbf{Explain concept and key features of Internet of Things (IoT).}

\begin{solutionbox}

\textbf{Internet of Things (IoT) Concept:}

\begin{center}
\textbf{Mermaid Diagram (Code)}
\begin{verbatim}
{Shaded}
{Highlighting}[]
graph TD
    A[Internet of Things] {-{-}{} B[Connected Devices]}
    A {-{-}{} C[Data Collection]}
    A {-{-}{} D[Data Analytics]}
    A {-{-}{} E[Automation]}
    B {-{-}{} F[Sensors]}
    B {-{-}{} G[Actuators]}
    C {-{-}{} H[Cloud Storage]}
    D {-{-}{} I[AI/Machine Learning]}
    E {-{-}{} J[Smart Applications]}
{Highlighting}
{Shaded}
\end{verbatim}
\end{center}

\textbf{Table of IoT Key Features:}

{\def\LTcaptype{none} % do not increment counter
\begin{longtable}[]{@{}
  >{\raggedright\arraybackslash}p{(\linewidth - 2\tabcolsep) * \real{0.4091}}
  >{\raggedright\arraybackslash}p{(\linewidth - 2\tabcolsep) * \real{0.5909}}@{}}
\toprule\noalign{}
\begin{minipage}[b]{\linewidth}\raggedright
Feature
\end{minipage} & \begin{minipage}[b]{\linewidth}\raggedright
Description
\end{minipage} \\
\midrule\noalign{}
\endhead
\bottomrule\noalign{}
\endlastfoot
\textbf{Connectivity} & Devices connected to internet and each other \\
\textbf{Intelligence} & Smart processing, decision-making
capabilities \\
\textbf{Sensing} & Gathering data from environment through sensors \\
\textbf{Expressing} & Taking actions through actuators \\
\textbf{Energy Efficiency} & Low power consumption for battery-operated
devices \\
\textbf{Security} & Protection against unauthorized access and
attacks \\
\textbf{Scalability} & Ability to add more devices to the network \\
\end{longtable}
}

\textbf{IoT Architecture Layers:}

\begin{verbatim}
               +{-{-}{-}{-}{-}{-}{-}{-}{-}{-}{-}{-}{-}{-}{-}{-}{-}{-}+}
               |    Application   |
               +{-{-}{-}{-}{-}{-}{-}{-}{-}{-}{-}{-}{-}{-}{-}{-}{-}{-}+}
               |  Data Analytics  |
               +{-{-}{-}{-}{-}{-}{-}{-}{-}{-}{-}{-}{-}{-}{-}{-}{-}{-}+}
               |  Data Processing |
               +{-{-}{-}{-}{-}{-}{-}{-}{-}{-}{-}{-}{-}{-}{-}{-}{-}{-}+}
               |  Data Transport  |
               +{-{-}{-}{-}{-}{-}{-}{-}{-}{-}{-}{-}{-}{-}{-}{-}{-}{-}+}
               |    Perception    |
               +{-{-}{-}{-}{-}{-}{-}{-}{-}{-}{-}{-}{-}{-}{-}{-}{-}{-}+}
\end{verbatim}

\textbf{IoT Applications:}

\begin{itemize}
\tightlist
\item
  Smart homes and buildings
\item
  Healthcare monitoring
\item
  Industrial automation
\item
  Smart cities
\item
  Agriculture monitoring
\item
  Supply chain management
\end{itemize}

\end{solutionbox}
\begin{mnemonicbox}
s ``CASED'' - Connected, Automated, Sensing,
Expressing, Data-driven

\end{mnemonicbox}

\end{document}
