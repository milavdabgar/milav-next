\documentclass{article}

% content/resources/templates/preamble.tex
\usepackage[margin=0.6in]{geometry}
\author{Milav Dabgar}
\usepackage{amsmath,amssymb,amsthm}
\usepackage{booktabs}
\usepackage{multirow}
\usepackage{xcolor}
\usepackage{tcolorbox}
\tcbuselibrary{breakable,skins}
\usepackage[colorlinks=true,linkcolor=blue]{hyperref}
\usepackage{titlesec}
\usepackage{enumitem}
\usepackage{tikz}
\usepackage{pgfplots}
\usepackage{circuitikz}
\usepackage[version=4]{mhchem}
\usepackage{longtable}
\usepackage{array}
\usepackage{float}
\usepackage{caption}
\usepackage{listings}

\lstset{
  basicstyle=\small\ttfamily,
  breaklines=true,
  breakatwhitespace=false,
  postbreak=\mbox{\textcolor{red}{$\hookrightarrow$}\space},
  float=false,
  numbers=left,
  numberstyle=\tiny\color{gray},
  numbersep=10pt,
  xleftmargin=2em,
  keywordstyle=\color{blue},
  commentstyle=\color{green!60!black},
  stringstyle=\color{purple},
  backgroundcolor=\color{gray!5},
  showstringspaces=false,
  tabsize=2,
  captionpos=b,
  keepspaces=true,
  columns=flexible
}

\pgfplotsset{compat=1.18}
\usetikzlibrary{shapes,arrows,positioning,calc,patterns,decorations.pathmorphing,decorations.markings,arrows.meta}

% Color scheme
\definecolor{headcolor}{RGB}{0,102,204}
\definecolor{keycolor}{RGB}{220,20,60}
\definecolor{solutioncolor}{RGB}{34,139,34}
\definecolor{mnemoniccolor}{RGB}{148,0,211}
\definecolor{codecolor}{RGB}{0,0,100}

% Spacing
\setlength{\parskip}{3pt}
\setlist[itemize]{nosep}
\setlist[enumerate]{nosep}

% Title formatting
\titleformat{\section}{\Large\bfseries\color{headcolor}}{\thesection}{1em}{}
\titleformat{\subsection}{\large\bfseries\color{headcolor}}{\thesubsection}{1em}{}

% Pandoc tightlist compatibility
\providecommand{\tightlist}{%
  \setlength{\itemsep}{0pt}\setlength{\parskip}{0pt}}

% Pandoc longtable compatibility
\newcounter{none}
\def\thenone{}


% content/resources/templates/gujarati-boxes.tex
\usepackage{fontspec}
\usepackage{polyglossia}

% Set Gujarati as main language (document is primarily in Gujarati)
% Note: gloss-gujarati.ldf doesn't exist in polyglossia, but it will use hyphenation patterns
\setdefaultlanguage{gujarati}
\setotherlanguage{english}

% Configure Gujarati font properly
% Use Language=Default to prevent polyglossia from trying to add language-specific features
% that don't exist for Gujarati, which causes "empty feature" warnings
\newfontfamily\gujaratifont[Script=Gujarati,AutoFakeBold=2.5,AutoFakeSlant=0.3]{Noto Sans Gujarati}
\setmainfont[Script=Gujarati,AutoFakeBold=2.5,AutoFakeSlant=0.3]{Noto Sans Gujarati}
% Use Noto Sans Gujarati for monospace to support Gujarati in text
\setmonofont[Scale=0.9]{Noto Sans Gujarati}

% Configure English to use the same font
\newfontfamily\englishfont[Script=Gujarati,AutoFakeBold=2.5,AutoFakeSlant=0.3]{Noto Sans Gujarati}

% Translations for polyglossia
\gappto\captionsgujarati{
  \renewcommand{\tablename}{કોષ્ટક}
  \renewcommand{\figurename}{આકૃતિ}
}

% Helper for TikZ nodes to ensure Gujarati font
\newcommand{\gu}[1]{{\gujaratifont #1}}

% Custom environments
\newtcolorbox{solutionbox}{
    breakable,
    enhanced,
    colback=solutioncolor!5!white,
    colframe=solutioncolor!75!black,
    fonttitle=\bfseries,
    title=જવાબ
}

\newtcolorbox{solutionboxnobreak}{
 colback=solutioncolor!5!white,
 colframe=solutioncolor!75!black,
 fonttitle=\bfseries,
 title=જવાબ
}

\newtcolorbox{keyformula}{
 breakable,
 enhanced,
 colback=keycolor!5!white,
 colframe=keycolor!75!black,
 fonttitle=\bfseries,
 title=રાસાયણિક સમીકરણ/સૂત્ર
}

\newtcolorbox{mnemonicbox}{
 breakable,
 enhanced,
 colback=mnemoniccolor!5!white,
 colframe=mnemoniccolor!75!black,
 fonttitle=\bfseries,
 title=મેમરી ટ્રીક
}


% Custom commands for GTU solutions
% This file defines semantic commands for consistent formatting

% Question command with automatic formatting
\newcommand{\question}[2]{%
  \section*{Question #1}%
  \textbf{#2}%
}

% OR question variant
\newcommand{\questionor}[2]{%
  \section*{Question #1 OR}%
  \textbf{#2}%
}

% Proper table environment with caption
\newenvironment{answertable}[1]{%
  \begin{table}[htbp]
  \centering
  \caption{#1}
}{%
  \end{table}
}

% Proper figure environment for diagrams
\newenvironment{answerdiagram}[1]{%
  \begin{figure}[htbp]
  \centering
  \caption{#1}
}{%
  \end{figure}
}

% Semantic markup for key terms
\newcommand{\keyword}[1]{\textbf{#1}}
\newcommand{\code}[1]{\texttt{#1}}
\newcommand{\classname}[1]{\texttt{#1}}
\newcommand{\methodname}[1]{\texttt{#1}}

% Proper quotation marks
\newcommand{\mnemonic}[1]{``#1''}


\title{ડિજિટલ કોમ્યુનિકેશન (4341102) - સમર 2023 સોલ્યુશન}
\date{જુલાઈ 15, 2023}

\begin{document}
\maketitle

\questionmarks{1}{a}{3}
\textbf{સિગ્નલને વ્યાખ્યાયિત કરો અને તેનું વર્ગીકરણ આપો.}

\begin{solutionbox}
સિગ્નલ એ એક \keyword{ભૌતિક માત્રા} છે જે સમય, સ્થળ અથવા અન્ય સ્વતંત્ર ચલ સાથે બદલાય છે અને તેમાં માહિતી સમાયેલી હોય છે.

\textbf{સિગ્નલનું વર્ગીકરણ:}

\begin{tabulary}{\linewidth}{L L}
\hline
\textbf{વર્ગીકરણ માપદંડ} & \textbf{સિગ્નલના પ્રકાર} \\
\hline
\textbf{સમય ડોમેન} & કંટીન્યુઅસ-ટાઈમ સિગ્નલ, ડિસ્ક્રીટ-ટાઈમ સિગ્નલ \\
\textbf{એમ્પ્લિટ્યુડ} & એનાલોગ સિગ્નલ, ડિજિટલ સિગ્નલ \\
\textbf{પ્રકૃતિ} & ડીટર્મિનિસ્ટિક સિગ્નલ, રેન્ડમ સિગ્નલ \\
\textbf{સિમેટ્રી} & ઈવન સિગ્નલ, ઓડ સિગ્નલ \\
\textbf{એનર્જી/પાવર} & એનર્જી સિગ્નલ, પાવર સિગ્નલ \\
\hline
\end{tabulary}

\begin{mnemonicbox}
\textbf{મેમરી ટ્રીક:} "CADEN" (Continuous/Discrete, Analog/Digital, Deterministic/Random, Even/Odd, Energy/Power)
\end{mnemonicbox}
\end{solutionbox}

\questionmarks{1}{b}{4}
\textbf{કંટીન્યુઅસ અને ડિસ્ક્રીટ ટાઈમ સિગ્નલ સમજાવો.}

\begin{solutionbox}
\begin{tabulary}{\linewidth}{L L}
\hline
\textbf{કંટીન્યુઅસ-ટાઈમ સિગ્નલ} & \textbf{ડિસ્ક્રીટ-ટાઈમ સિગ્નલ} \\
\hline
સમયના તમામ મૂલ્યો માટે વ્યાખ્યાયિત & માત્ર ચોક્કસ સમય અંતરાલ પર વ્યાખ્યાયિત \\
$x(t)$ તરીકે રજુ થાય છે & $x[n]$ અથવા $x(nT)$ તરીકે રજુ થાય છે \\
ઉદાહરણ: સાઇન વેવ જેવા એનાલોગ સિગ્નલ & ઉદાહરણ: સેમ્પલ કરેલા સ્પીચ જેવા ડિજિટલ સિગ્નલ \\
ગ્રાફ પર સળંગ વક્ર & ગ્રાફ પર બિંદુઓની શ્રેણી \\
પ્રોસેસિંગ માટે એનાલોગ સર્કિટની જરૂર પડે & પ્રોસેસિંગ ડિજિટલ પ્રોસેસર દ્વારા કરી શકાય \\
\hline
\end{tabulary}

\textbf{આકૃતિ:}

\begin{center}
\begin{tikzpicture}[node distance=2.5cm, auto]
    \node [gtu block] (sig) {\gu{સિગ્નલ}};
    \node [gtu block, below left of=sig, xshift=-1cm] (cont) {\gu{કંટીન્યુઅસ-ટાઈમ}};
    \node [gtu block, below right of=sig, xshift=1cm] (disc) {\gu{ડિસ્ક્રીટ-ટાઈમ}};
    \node [gtu block, below of=cont] (contex) {\gu{બધા $t$ માટે વ્યાખ્યાયિત}\\\gu{ઉદાહરણ:} $\sin(t)$};
    \node [gtu block, below of=disc] (discex) {\gu{ચોક્કસ સમય $nT$ પર વ્યાખ્યાયિત}\\\gu{ઉદાહરણ:} $\sin(nT)$};

    \path [gtu arrow] (sig) -- (cont);
    \path [gtu arrow] (sig) -- (disc);
    \path [gtu arrow] (cont) -- (contex);
    \path [gtu arrow] (disc) -- (discex);
\end{tikzpicture}
\end{center}

\begin{mnemonicbox}
\textbf{મેમરી ટ્રીક:} "CAD" - Continuous signals are Analog and Defined for all time; Discrete signals are digital and defined at specific points.
\end{mnemonicbox}
\end{solutionbox}

\questionmarks{1}{c}{7}
\textbf{યુનિટ ઇમ્પલ્સ અને યુનિટ સ્ટેપ ફંક્શન સમજાવો.}

\begin{solutionbox}
\begin{tabulary}{\linewidth}{L L}
\hline
\textbf{યુનિટ ઇમ્પલ્સ ફંક્શન ($\delta(t)$)} & \textbf{યુનિટ સ્ટેપ ફંક્શન ($u(t)$)} \\
\hline
$t=0$ પર અનંત ઊંચાઈ, બાકી જગ્યાએ શૂન્ય & $t \ge 0$ માટે મૂલ્ય 1, $t<0$ માટે 0 \\
વક્ર નીચેનું ક્ષેત્રફળ = 1 & ઇન્ટિગ્રલ રેમ્પ ફંક્શન આપે છે \\
તાત્કાલિક ઘટનાઓને રજૂ કરવા માટે & અચાનક બદલાવને રજૂ કરવા માટે \\
LTI સિસ્ટમ એનાલિસિસનો ગાણિતિક આધાર & સિસ્ટમ રિસ્પોન્સ એનાલિસિસ માટે ઉપયોગી \\
લાપ્લાસ ટ્રાન્સફોર્મ = 1 & લાપ્લાસ ટ્રાન્સફોર્મ = $1/s$ \\
\hline
\end{tabulary}

\textbf{આકૃતિ:}

\begin{center}
\begin{tikzpicture}
    % Unit Impulse
    \begin{scope}[xshift=-4cm]
        \draw[->] (-2,0) -- (2,0) node[right] {$t$};
        \draw[->] (0,-0.5) -- (0,2.5);
        \draw[thick, ->] (0,0) -- (0,2) node[above] {$\delta(t)$};
        \node at (0.5, 1) {\gu{ક્ષેત્રફળ = 1}};
        \node[below] at (0,-0.5) {\gu{યુનિટ ઇમ્પલ્સ}};
    \end{scope}

    % Unit Step
    \begin{scope}[xshift=4cm]
        \draw[->] (-2,0) -- (2,0) node[right] {$t$};
        \draw[->] (0,-0.5) -- (0,2.5) node[above] {$u(t)$};
        \draw[thick] (-2,0) -- (0,0);
        \draw[thick] (0,0) -- (0,1.5) -- (2,1.5);
        \node[left] at (0,1.5) {1};
        \node[below] at (0,-0.5) {\gu{યુનિટ સ્ટેપ}};
    \end{scope}
\end{tikzpicture}
\end{center}

\textbf{ગુણધર્મો:}
\begin{itemize}
    \item \textbf{સેમ્પલિંગ પ્રોપર્ટી}: $\int f(t)\delta(t-t_0)dt = f(t_0)$
    \item \textbf{યુનિટ સ્ટેપ ઇમ્પલ્સનું ઇન્ટિગ્રલ છે}: $u(t) = \int_{-\infty}^{t} \delta(\tau)d\tau$
    \item \textbf{ઇમ્પલ્સ યુનિટ સ્ટેપનો ડેરિવેટિવ છે}: $\delta(t) = \frac{du(t)}{dt}$
\end{itemize}

\begin{mnemonicbox}
\textbf{મેમરી ટ્રીક:} "SHARP-FLAT" - Impulse is Sharp and momentary; Step is Flat and persistent.
\end{mnemonicbox}
\end{solutionbox}

\questionmarks{1}{c}{7}
\textbf{ડિજિટલ કોમ્યુનિકેશન સિસ્ટમનો બ્લોક ડાયાગ્રામ સમજાવો.}

\begin{solutionbox}
\textbf{ડિજિટલ કોમ્યુનિકેશન સિસ્ટમનો બ્લોક ડાયાગ્રામ:}

\begin{center}
\begin{tikzpicture}[node distance=1.8cm, auto]
    \node [gtu block] (source) {\gu{સોર્સ}};
    \node [gtu block, right of=source, node distance=2.5cm] (sec) {\gu{સોર્સ}\\\gu{એન્કોડર}};
    \node [gtu block, right of=sec, node distance=2.5cm] (cec) {\gu{ચેનલ}\\\gu{એન્કોડર}};
    \node [gtu block, right of=cec, node distance=2.5cm] (mod) {\gu{ડિજિટલ}\\\gu{મોડ્યુલેટર}};
    
    \node [gtu block, below of=mod, node distance=2cm] (chan) {\gu{ચેનલ}};
    
    \node [gtu block, below of=chan, node distance=2cm] (demod) {\gu{ડિજિટલ}\\\gu{ડિમોડ્યુલેટર}};
    \node [gtu block, left of=demod, node distance=2.5cm] (cdec) {\gu{ચેનલ}\\\gu{ડિકોડર}};
    \node [gtu block, left of=cdec, node distance=2.5cm] (sdec) {\gu{સોર્સ}\\\gu{ડિકોડર}};
    \node [gtu block, left of=sdec, node distance=2.5cm] (dest) {\gu{ડેસ્ટિનેશન}};

    \path [gtu arrow] (source) -- (sec);
    \path [gtu arrow] (sec) -- (cec);
    \path [gtu arrow] (cec) -- (mod);
    \path [gtu arrow] (mod) -- (chan);
    \path [gtu arrow] (chan) -- (demod);
    \path [gtu arrow] (demod) -- (cdec);
    \path [gtu arrow] (cdec) -- (sdec);
    \path [gtu arrow] (sdec) -- (dest);
\end{tikzpicture}
\end{center}

\textbf{સમજૂતી:}

\begin{tabulary}{\linewidth}{L L}
\hline
\textbf{બ્લોક} & \textbf{કાર્ય} \\
\hline
\textbf{સોર્સ} & ટ્રાન્સમિટ કરવાનો મેસેજ ઉત્પન્ન કરે છે \\
\textbf{સોર્સ એન્કોડર} & મેસેજને ડિજિટલ ફોર્મમાં રૂપાંતરિત કરે છે, રિડન્ડન્સી દૂર કરે છે \\
\textbf{ચેનલ એન્કોડર} & એરર ડિટેક્શન/કરેક્શન માટે નિયંત્રિત રિડન્ડન્સી ઉમેરે છે \\
\textbf{ડિજિટલ મોડ્યુલેટર} & ડિજિટલ બિટ્સને ટ્રાન્સમિશન માટે યોગ્ય સિગ્નલમાં રૂપાંતરિત કરે છે \\
\textbf{ચેનલ} & ભૌતિક માધ્યમ જેના દ્વારા સિગ્નલ પ્રવાસ કરે છે \\
\textbf{ડિજિટલ ડિમોડ્યુલેટર} & પ્રાપ્ત સિગ્નલમાંથી ડિજિટલ ડેટા પુનઃપ્રાપ્ત કરે છે \\
\textbf{ચેનલ ડિકોડર} & ઉમેરેલી રિડન્ડન્સીનો ઉપયોગ કરીને એરર શોધે/સુધારે છે \\
\textbf{સોર્સ ડિકોડર} & પ્રાપ્ત બિટ્સમાંથી મૂળ સંદેશ પુનઃનિર્માણ કરે છે \\
\textbf{ડેસ્ટિનેશન} & પ્રેષિત સંદેશ પ્રાપ્ત કરે છે \\
\hline
\end{tabulary}

\begin{mnemonicbox}
\textbf{મેમરી ટ્રીક:} "SECDCSD" - "Seven Engineers Can Design Communication Systems Diligently"
\end{mnemonicbox}
\end{solutionbox}

\questionmarks{2}{a}{3}
\textbf{સિગ્નલમાં 8000 બીટ/સેકન્ડનો બીટ રેટ અને 1000 બોડનો બોડ દર હોય છે. દરેક સિગ્નલ એલીમેંટ દ્વારા કેટલા ડેટા એલીમેંટ વહન કરવામાં આવે છે?}

\begin{solutionbox}
દરેક સિગ્નલ એલિમેન્ટ દ્વારા વહન કરાતા ડેટા એલિમેન્ટ (બિટ્સ)ની સંખ્યા:
\[ = \frac{\text{બીટ રેટ}}{\text{બોડ રેટ}} \]
\[ = \frac{8000 \text{ બિટ્સ/સેકન્ડ}}{1000 \text{ બોડ}} \]
\[ = 8 \text{ બિટ્સ/સિગ્નલ એલિમેન્ટ} \]

\textbf{કોષ્ટક:}

\begin{tabulary}{\linewidth}{L L L}
\hline
\textbf{પેરામીટર} & \textbf{મૂલ્ય} & \textbf{સંબંધ} \\
\hline
બીટ રેટ & 8000 બિટ્સ/સેક & આપેલ \\
બોડ રેટ & 1000 બોડ & આપેલ \\
બિટ્સ/સિગ્નલ & 8 બિટ્સ & બીટ રેટ $\div$ બોડ રેટ \\
\hline
\end{tabulary}

\begin{mnemonicbox}
\textbf{મેમરી ટ્રીક:} "Bits Divided By Bauds" (BDBB)
\end{mnemonicbox}
\end{solutionbox}

\questionmarks{2}{b}{4}
\textbf{એનર્જી અને પાવર સિગ્નલ સમજાવો.}

\begin{solutionbox}
\begin{tabulary}{\linewidth}{L L}
\hline
\textbf{એનર્જી સિગ્નલ} & \textbf{પાવર સિગ્નલ} \\
\hline
અંતિમ કુલ એનર્જી & અનંત કુલ એનર્જી પરંતુ અંતિમ સરેરાશ પાવર \\
શૂન્ય સરેરાશ પાવર & બિન-શૂન્ય સરેરાશ પાવર \\
$E = \int |x(t)|^2 dt$ (અંતિમ) & $P = \lim_{T\to\infty} \frac{1}{2T} \int |x(t)|^2 dt$ (અંતિમ) \\
ઉદાહરણ: પલ્સ, ક્ષયિત એક્સપોનેન્શિયલ & ઉદાહરણ: સાઇન વેવ, સ્ક્વેર વેવ \\
સમયમાં સીમિત & બધા સમય માટે અસ્તિત્વમાં \\
\hline
\end{tabulary}

\textbf{આકૃતિ:}

\begin{center}
\begin{tikzpicture}[node distance=2cm, auto]
    \node [gtu block] (sig) {\gu{સિગ્નલ}};
    \node [gtu block, below left of=sig, xshift=-1.5cm] (energy) {\gu{એનર્જી સિગ્નલ}};
    \node [gtu block, below right of=sig, xshift=1.5cm] (power) {\gu{પાવર સિગ્નલ}};
    \node [gtu block, below of=energy, node distance=2.5cm] (eprop) {\gu{અંતિમ એનર્જી}\\\gu{શૂન્ય સરેરાશ પાવર}\\\gu{ઉદાહરણ: પલ્સ}};
    \node [gtu block, below of=power, node distance=2.5cm] (pprop) {\gu{અનંત એનર્જી}\\\gu{અંતિમ સરેરાશ પાવર}\\\gu{ઉદાહરણ: સાઇન વેવ}};

    \path [gtu arrow] (sig) -- (energy);
    \path [gtu arrow] (sig) -- (power);
    \path [gtu arrow] (energy) -- (eprop);
    \path [gtu arrow] (power) -- (pprop);
\end{tikzpicture}
\end{center}

\begin{mnemonicbox}
\textbf{મેમરી ટ્રીક:} "FEZIL" - Finite Energy is Zero in Long-term; Power signals are Infinite in Length
\end{mnemonicbox}
\end{solutionbox}

\questionmarks{2}{c}{7}
\textbf{FSK મોડ્યુલેટર અને ડી-મોડ્યુલેટરના બ્લોક ડાયાગ્રામને વેવફોર્મ સાથે સમજાવો.}

\begin{solutionbox}
\textbf{FSK મોડ્યુલેટર અને ડિમોડ્યુલેટર:}

\begin{center}
\begin{tikzpicture}[node distance=2cm, auto]
    % Modulator
    \node [gtu block, minimum width=2.5cm] (input) {\gu{ડિજિટલ ઇનપુટ}};
    \node [gtu block, right of=input, node distance=4cm] (vco) {Voltage Controlled\\Oscillator (VCO)};
    \node [gtu block, right of=vco, node distance=4cm] (output) {FSK \gu{આઉટપુટ}};
    
    \path [gtu arrow] (input) -- (vco);
    \path [gtu arrow] (vco) -- (output);
    
    \node [below of=vco, node distance=1.5cm, font=\bfseries] {\gu{મોડ્યુલેટર}};

    % Demodulator
    \node [gtu block, minimum width=2cm, below of=input, node distance=3.5cm] (fskinput) {FSK \gu{ઇનપુટ}};
    
    \node [gtu block, right of=fskinput, node distance=3cm, yshift=1cm] (bpf1) {BPF 1\\($f_1$)};
    \node [gtu block, right of=fskinput, node distance=3cm, yshift=-1cm] (bpf2) {BPF 2\\($f_2$)};
    
    \node [gtu block, right of=bpf1, node distance=2.5cm] (env1) {\gu{એન્વેલપ}\\\gu{ડિટેક્ટર 1}};
    \node [gtu block, right of=bpf2, node distance=2.5cm] (env2) {\gu{એન્વેલપ}\\\gu{ડિટેક્ટર 2}};
    
    \node [gtu block, right of=env1, node distance=2.5cm, yshift=-1cm] (comp) {\gu{કમ્પેરેટર}};
    \node [gtu block, right of=comp, node distance=2.5cm] (dout) {\gu{ડિજિટલ આઉટપુટ}};

    \path [gtu arrow] (fskinput) -- ++(1,0) |- (bpf1);
    \path [gtu arrow] (fskinput) -- ++(1,0) |- (bpf2);
    \path [gtu arrow] (bpf1) -- (env1);
    \path [gtu arrow] (bpf2) -- (env2);
    \path [gtu arrow] (env1) -| (comp);
    \path [gtu arrow] (env2) -| (comp);
    \path [gtu arrow] (comp) -- (dout);
    
    \node [below of=comp, node distance=2cm, font=\bfseries] {\gu{ડિમોડ્યુલેટર}};
\end{tikzpicture}
\end{center}

\textbf{વેવફોર્મ:}

\begin{center}
\begin{tikzpicture}
    % Digital Input
    \draw (0,4) node[left] {\gu{ડિજિટલ ઇનપુટ}};
    \draw (0,4) -- (1,4) -- (1,4.5) -- (2,4.5) -- (2,4) -- (3,4) -- (3,4.5) -- (4,4.5) -- (4,4) -- (5,4);
    \foreach \x/\val in {0.5/0, 1.5/1, 2.5/0, 3.5/1, 4.5/0}
        \node at (\x, 3.7) {\val};

    % FSK Output
    \draw (0,2.5) node[left] {FSK \gu{આઉટપુટ}};
    % f1: low freq, f2: high freq
    \draw[samples=100, domain=0:1] plot (\x, {2.5 + 0.3*sin(1800*(\x - floor(\x)))});
    \draw[samples=100, domain=1:2] plot (\x, {2.5 + 0.3*sin(3600*(\x - floor(\x)))});
    \draw[samples=100, domain=2:3] plot (\x, {2.5 + 0.3*sin(1800*(\x - floor(\x)))});
    \draw[samples=100, domain=3:4] plot (\x, {2.5 + 0.3*sin(3600*(\x - floor(\x)))});
    \draw[samples=100, domain=4:5] plot (\x, {2.5 + 0.3*sin(1800*(\x - floor(\x)))});
    
    \foreach \x/\val in {0.5/$f_1$, 1.5/$f_2$, 2.5/$f_1$, 3.5/$f_2$, 4.5/$f_1$}
        \node at (\x, 2) {\val};
\end{tikzpicture}
\end{center}

\textbf{મુખ્ય સિદ્ધાંતો:}
\begin{itemize}
    \item \textbf{બિટ 0}: ફ્રીક્વન્સી $f_1$ તરીકે ટ્રાન્સમિટ થાય છે
    \item \textbf{બિટ 1}: ફ્રીક્વન્સી $f_2$ તરીકે ટ્રાન્સમિટ થાય છે
    \item \textbf{ડિમોડ્યુલેશન}: ફ્રીક્વન્સીઓને અલગ કરવા માટે બેન્ડપાસ ફિલ્ટર્સનો ઉપયોગ કરે છે
    \item \textbf{ડિટેક્શન}: એન્વેલપ ડિટેક્ટર્સ ડિજિટલ સિગ્નલને પુનઃપ્રાપ્ત કરે છે
\end{itemize}

\begin{mnemonicbox}
\textbf{મેમરી ટ્રીક:} "FIST" - Frequency Is Shifted for Transmission
\end{mnemonicbox}
\end{solutionbox}

\questionmarks{2}{a}{3}
\textbf{સિગ્નલ 4 બીટ/સિગ્નલ એલીમેંટ ધરાવે છે. જો 1000 સિગ્નલ એલીમેંટ પ્રતિ સેકન્ડ મોકલવામાં આવે છે. તો બીટ રેટ શોધો.}

\begin{solutionbox}
બીટ રેટ = સિગ્નલ એલિમેન્ટ દીઠ બિટ્સની સંખ્યા $\times$ પ્રતિ સેકન્ડ સિગ્નલ એલિમેન્ટ \\
બીટ રેટ = 4 બિટ્સ/સિગ્નલ એલિમેન્ટ $\times$ 1000 સિગ્નલ એલિમેન્ટ/સેકન્ડ \\
બીટ રેટ = 4000 બિટ્સ/સેકન્ડ

\textbf{કોષ્ટક:}

\begin{tabulary}{\linewidth}{L L L}
\hline
\textbf{પેરામીટર} & \textbf{મૂલ્ય} & \textbf{સંબંધ} \\
\hline
સિમ્બોલ દીઠ બિટ્સ & 4 & આપેલ \\
સિમ્બોલ રેટ & 1000 સિમ્બોલ/સેક & આપેલ \\
બીટ રેટ & 4000 બિટ્સ/સેક & બિટ્સ/સિમ્બોલ $\times$ સિમ્બોલ રેટ \\
\hline
\end{tabulary}

\begin{mnemonicbox}
\textbf{મેમરી ટ્રીક:} "BBS" - Bit rate equals Bits per symbol times Symbol rate
\end{mnemonicbox}
\end{solutionbox}

\questionmarks{2}{b}{4}
\textbf{ઈવન અને ઓડ સિગ્નલ સમજાવો.}

\begin{solutionbox}
\begin{tabulary}{\linewidth}{L L}
\hline
\textbf{ઈવન સિગ્નલ} & \textbf{ઓડ સિગ્નલ} \\
\hline
y-અક્ષની આસપાસ સિમેટ્રિક & y-અક્ષની આસપાસ એન્ટી-સિમેટ્રિક \\
$x(-t) = x(t)$ & $x(-t) = -x(t)$ \\
ઉદાહરણ: $\cos(t)$ & ઉદાહરણ: $\sin(t)$ \\
ફૂરિયર ટ્રાન્સફોર્મ વાસ્તવિક છે & ફૂરિયર ટ્રાન્સફોર્મ કાલ્પનિક છે \\
ઈવન સિગ્નલનો સરવાળો ઈવન છે & ઓડ સિગ્નલનો સરવાળો ઓડ છે \\
\hline
\end{tabulary}

\textbf{આકૃતિ:}

\begin{center}
\begin{tikzpicture}
    % Even Signal
    \begin{scope}[xshift=-4cm]
        \draw[->] (-2.5,0) -- (2.5,0) node[right] {$t$};
        \draw[->] (0,-1.5) -- (0,2) node[above] {$x(t)$ (ઈવન)};
        \draw[thick, domain=-2:2, samples=50] plot (\x, {cos(90*\x)});
        \node[above] at (0,2) {};
    \end{scope}

    % Odd Signal
    \begin{scope}[xshift=4cm]
        \draw[->] (-2.5,0) -- (2.5,0) node[right] {$t$};
        \draw[->] (0,-1.5) -- (0,2) node[above] {$x(t)$ (ઓડ)};
        \draw[thick, domain=-2:2, samples=50] plot (\x, {sin(90*\x)});
    \end{scope}
\end{tikzpicture}
\end{center}

\textbf{ગુણધર્મો:}
\begin{itemize}
    \item કોઈપણ સિગ્નલને ઈવન અને ઓડ ઘટકોના સરવાળા તરીકે વ્યક્ત કરી શકાય છે
    \item ઈવન ઘટક: $x_e(t) = [x(t) + x(-t)]/2$
    \item ઓડ ઘટક: $x_o(t) = [x(t) - x(-t)]/2$
\end{itemize}

\begin{mnemonicbox}
\textbf{મેમરી ટ્રીક:} "SAME-FLIP" - Even signals are the SAME when flipped; Odd signals FLIP their sign.
\end{mnemonicbox}
\end{solutionbox}

\questionmarks{2}{c}{7}
\textbf{QPSK મોડ્યુલેટર અને ડી-મોડ્યુલેટરના બ્લોક ડાયાગ્રામને કોન્સોલેશન ડાયાગ્રામ સાથે સમજાવો.}

\begin{solutionbox}
\textbf{QPSK મોડ્યુલેટર અને ડિમોડ્યુલેટર:}

\begin{center}
\begin{tikzpicture}[node distance=2cm, auto]
    % Modulator
    \node [gtu block] (input) {\gu{બાઇનરી ઇનપુટ}};
    \node [gtu block, right of=input, node distance=2.5cm] (sp) {S/P\\\gu{કન્વર્ટર}};
    
    \node [gtu block, right of=sp, node distance=2.5cm, yshift=1cm] (mult1) {\gu{મલ્ટિપ્લાયર}};
    \node [gtu block, right of=sp, node distance=2.5cm, yshift=-1cm] (mult2) {\gu{મલ્ટિપ્લાયર}};
    
    \node [left of=mult1, node distance=1.2cm, above] {\gu{ઈવન બિટ્સ}};
    \node [left of=mult2, node distance=1.2cm, below] {\gu{ઓડ બિટ્સ}};
    
    \node [above of=mult1, node distance=1cm] (cos) {$\cos(2\pi ft)$};
    \node [below of=mult2, node distance=1cm] (sin) {$\sin(2\pi ft)$};
    
    \node [gtu block, right of=mult1, node distance=2cm, yshift=-1cm] (sum) {\gu{સમર}};
    \node [gtu block, right of=sum, node distance=2cm] (out) {QPSK \gu{આઉટપુટ}};

    \path [gtu arrow] (input) -- (sp);
    \path [gtu arrow] (sp) |- (mult1);
    \path [gtu arrow] (sp) |- (mult2);
    \path [gtu arrow] (cos) -- (mult1);
    \path [gtu arrow] (sin) -- (mult2);
    \path [gtu arrow] (mult1) -| (sum);
    \path [gtu arrow] (mult2) -| (sum);
    \path [gtu arrow] (sum) -- (out);
\end{tikzpicture}
\end{center}

\textbf{કોન્સ્ટેલેશન ડાયાગ્રામ:}

\begin{center}
\begin{tikzpicture}
    \draw[->] (-2,0) -- (2,0) node[right] {$I$};
    \draw[->] (0,-2) -- (0,2) node[above] {$Q$};
    
    \node[circle, fill, inner sep=1.5pt, label={45:00}] at (1,1) {};
    \node[circle, fill, inner sep=1.5pt, label={135:01}] at (-1,1) {};
    \node[circle, fill, inner sep=1.5pt, label={225:11}] at (-1,-1) {};
    \node[circle, fill, inner sep=1.5pt, label={315:10}] at (1,-1) {};
    
    \draw[dashed] (1,0) -- (1,1) -- (0,1);
    \draw[dashed] (-1,0) -- (-1,1) -- (0,1);
    \draw[dashed] (-1,0) -- (-1,-1) -- (0,-1);
    \draw[dashed] (1,0) -- (1,-1) -- (0,-1);
\end{tikzpicture}
\end{center}

\textbf{મુખ્ય લક્ષણો:}
\begin{itemize}
    \item \textbf{ઇનપુટ}: દરેક સિમ્બોલ 2 બિટ્સ દ્વારા નક્કી થાય છે
    \item \textbf{ફેઝ}: 4 ફેઝ ($0^\circ, 90^\circ, 180^\circ, 270^\circ$)
    \item \textbf{બિટ્સથી ફેઝ}: 00: $45^\circ$, 01: $135^\circ$, 11: $225^\circ$, 10: $315^\circ$
    \item \textbf{બેન્ડવિડ્થ એફિશિયન્સી}: 2 બિટ્સ પ્રતિ સિમ્બોલ
\end{itemize}

\begin{mnemonicbox}
\textbf{મેમરી ટ્રીક:} "QUADrature" - 4 phases for 4 possible 2-bit combinations
\end{mnemonicbox}
\end{solutionbox}

\questionmarks{3}{a}{3}
\textbf{ASK મોડ્યુલેટરનું કાર્ય બ્લોક ડાયાગ્રામ અને વેવફોર્મ સાથે સમજાવો.}

\begin{solutionbox}
\textbf{ASK મોડ્યુલેટર બ્લોક ડાયાગ્રામ:}

\begin{center}
\begin{tikzpicture}[node distance=2.5cm, auto]
    \node [gtu block] (input) {\gu{ડિજિટલ ઇનપુટ}};
    \node [gtu block, right of=input] (mult) {\gu{મલ્ટિપ્લાયર}};
    \node [gtu block, below of=mult] (gen) {\gu{કેરિયર જનરેટર}\\$\sin(2\pi ft)$};
    \node [gtu block, right of=mult] (out) {ASK \gu{આઉટપુટ}};

    \path [gtu arrow] (input) -- (mult);
    \path [gtu arrow] (gen) -- (mult);
    \path [gtu arrow] (mult) -- (out);
\end{tikzpicture}
\end{center}

\textbf{વેવફોર્મ:}

\begin{center}
\begin{tikzpicture}
    % Digital Input
    \draw (0,4) node[left] {\gu{ડિજિટલ ઇનપુટ}};
    \draw (0,4) -- (1,4) -- (1,4.5) -- (2,4.5) -- (2,4) -- (3,4) -- (3,4.5) -- (4,4.5) -- (4,4) -- (5,4);
    \foreach \x/\val in {0.5/0, 1.5/1, 2.5/0, 3.5/1, 4.5/0}
        \node at (\x, 3.7) {\val};

    % Carrier
    \draw (0,2.5) node[left] {\gu{કેરિયર}};
    \draw[samples=100, domain=0:5] plot (\x, {2.5 + 0.3*sin(3600*(\x - floor(\x)))});

    % ASK Output
    \draw (0,1) node[left] {ASK \gu{આઉટપુટ}};
    % 0: flat, 1: carrier
    \draw[samples=100, domain=0:1] plot (\x, {1});
    \draw[samples=100, domain=1:2] plot (\x, {1 + 0.3*sin(3600*(\x - floor(\x)))});
    \draw[samples=100, domain=2:3] plot (\x, {1});
    \draw[samples=100, domain=3:4] plot (\x, {1 + 0.3*sin(3600*(\x - floor(\x)))});
    \draw[samples=100, domain=4:5] plot (\x, {1});
\end{tikzpicture}
\end{center}

\textbf{કાર્ય સિદ્ધાંત:}
\begin{itemize}
    \item ડિજિટલ 1: કેરિયર સિગ્નલ ટ્રાન્સમિટ થાય છે
    \item ડિજિટલ 0: કોઈ સિગ્નલ નહીં (અથવા ઓછી એમ્પ્લિટ્યુડ) ટ્રાન્સમિટ થાય છે
    \item આઉટપુટ એમ્પ્લિટ્યુડ ઇનપુટ ડિજિટલ સિગ્નલ સાથે બદલાય છે
\end{itemize}

\begin{mnemonicbox}
\textbf{મેમરી ટ્રીક:} "ASKY" - Amplitude Switches the Carrier? Yes!
\end{mnemonicbox}
\end{solutionbox}

\questionmarks{3}{b}{4}
\textbf{8-PSK અને 16-QAM ના કોન્સોલેશન ડાયાગ્રામ દોરો.}

\begin{solutionbox}
\textbf{8-PSK કોન્સ્ટેલેશન ડાયાગ્રામ:}

\begin{center}
\begin{tikzpicture}
    \draw[->] (-2.5,0) -- (2.5,0) node[right] {$I$};
    \draw[->] (0,-2.5) -- (0,2.5) node[above] {$Q$};
    
    \foreach \ang/\lbl in {0/000, 45/001, 90/011, 135/010, 180/110, 225/111, 270/101, 315/100} {
        \node[circle, fill, inner sep=1.5pt] at (\ang:2) {};
        \node at (\ang:2.4) {\lbl};
        \draw[dashed] (0,0) -- (\ang:2);
    }
\end{tikzpicture}
\end{center}

\textbf{16-QAM કોન્સ્ટેલેશન ડાયાગ્રામ:}

\begin{center}
\begin{tikzpicture}
    \draw[->] (-3,0) -- (3,0) node[right] {$I$};
    \draw[->] (0,-3) -- (0,3) node[above] {$Q$};
    
    \foreach \x in {-1.5, -0.5, 0.5, 1.5}
        \foreach \y in {-1.5, -0.5, 0.5, 1.5}
            \node[circle, fill, inner sep=1.5pt] at (\x,\y) {};
            
    \node at (0, -3.5) {\gu{16 બિંદુઓ બદલાતી એમ્પ્લિટ્યુડ અને ફેઝ સાથે}};
\end{tikzpicture}
\end{center}

\begin{mnemonicbox}
\textbf{મેમરી ટ્રીક:} "P-Phase Q-Quantity" - PSK varies Phase only; QAM varies both amplitude (Quantity) and phase
\end{mnemonicbox}
\end{solutionbox}

\questionmarks{3}{c}{7}
\textbf{1100101101 ના ક્રમ માટે ASK અને FSK મોડ્યુલેશન વેવફોર્મ દોરો.}

\begin{solutionbox}
\textbf{મોડ્યુલેશન વેવફોર્મ:}

\begin{center}
\begin{tikzpicture}
    % Sequence: 1 1 0 0 1 0 1 1 0 1
    % Binary Input
    \draw (0,4) node[left] {\gu{ઇનપુટ}};
    \draw (0,4.5) -- (2,4.5) -- (2,4) -- (4,4) -- (4,4.5) -- (5,4.5) -- (5,4) -- (6,4) -- (6,4.5) -- (8,4.5) -- (8,4) -- (9,4) -- (9,4.5) -- (10,4.5);
    \foreach \x/\val in {0.5/1, 1.5/1, 2.5/0, 3.5/0, 4.5/1, 5.5/0, 6.5/1, 7.5/1, 8.5/0, 9.5/1}
        \node at (\x, 3.7) {\val};

    % ASK Output
    \draw (0,2.5) node[left] {ASK};
    \foreach \x/\val in {0/1, 1/1, 2/0, 3/0, 4/1, 5/0, 6/1, 7/1, 8/0, 9/1} {
        \ifnum\val=1
            \draw[samples=20, domain=\x:\x+1, variable=\u] plot[variable=\u] (\u, {2.5 + 0.3*sin(3600*(\u - floor(\u)))});
        \else
            \draw (\x,2.5) -- (\x+1,2.5);
        \fi
    }

    % FSK Output
    \draw (0,1) node[left] {FSK};
    \foreach \x/\val in {0/1, 1/1, 2/0, 3/0, 4/1, 5/0, 6/1, 7/1, 8/0, 9/1} {
        \ifnum\val=1
            % f2 (high freq) for 1
            \draw[samples=40, domain=\x:\x+1, variable=\u] plot[variable=\u] (\u, {1 + 0.3*sin(7200*(\u - floor(\u)))});
        \else
            % f1 (low freq) for 0
            \draw[samples=20, domain=\x:\x+1, variable=\u] plot[variable=\u] (\u, {1 + 0.3*sin(3600*(\u - floor(\u)))});
        \fi
    }
\end{tikzpicture}
\end{center}

\begin{tabulary}{\linewidth}{L L L L}
\hline
\textbf{મોડ્યુલેશન} & \textbf{બિટ 0} & \textbf{બિટ 1} & \textbf{બદલાતો પેરામીટર} \\
\hline
ASK & શૂન્ય અથવા ઓછી એમ્પ્લિટ્યુડ & ઉચ્ચ એમ્પ્લિટ્યુડ & એમ્પ્લિટ્યુડ \\
FSK & ફ્રીક્વન્સી $f_1$ & ફ્રીક્વન્સી $f_2$ & ફ્રીક્વન્સી \\
\hline
\end{tabulary}

\begin{mnemonicbox}
\textbf{મેમરી ટ્રીક:} "AFRO" - Amplitude For 1, Remove for 0 (ASK); Frequency Rises for 1, Off-peak for 0 (FSK)
\end{mnemonicbox}
\end{solutionbox}

\questionmarks{3}{a}{3}
\textbf{PSK મોડ્યુલેટરનું કાર્ય બ્લોક ડાયાગ્રામ અને વેવફોર્મ સાથે સમજાવો.}

\begin{solutionbox}
\textbf{PSK મોડ્યુલેટર બ્લોક ડાયાગ્રામ:}

\begin{center}
\begin{tikzpicture}[node distance=2.5cm, auto]
    \node [gtu block] (input) {\gu{ડિજિટલ ઇનપુટ}};
    \node [gtu block, right of=input] (polar) {\gu{પોલર કન્વર્ટર}\\$0\to-1, 1\to+1$};
    \node [gtu block, right of=polar] (mult) {\gu{મલ્ટિપ્લાયર}};
    \node [gtu block, below of=mult] (gen) {\gu{કેરિયર જનરેટર}\\$\sin(2\pi ft)$};
    \node [gtu block, right of=mult] (out) {PSK \gu{આઉટપુટ}};

    \path [gtu arrow] (input) -- (polar);
    \path [gtu arrow] (polar) -- (mult);
    \path [gtu arrow] (gen) -- (mult);
    \path [gtu arrow] (mult) -- (out);
\end{tikzpicture}
\end{center}

\textbf{વેવફોર્મ:}

\begin{center}
\begin{tikzpicture}
    % Digital Input
    \draw (0,4) node[left] {\gu{ડિજિટલ ઇનપુટ}};
    \draw (0,4) -- (1,4) -- (1,4.5) -- (2,4.5) -- (2,4) -- (3,4) -- (3,4.5) -- (4,4.5) -- (4,4) -- (5,4);
    \foreach \x/\val in {0.5/0, 1.5/1, 2.5/0, 3.5/1, 4.5/0}
        \node at (\x, 3.7) {\val};

    % PSK Output
    \draw (0,2.5) node[left] {PSK \gu{આઉટપુટ}};
    % 0 -> 180 deg (inverted sine), 1 -> 0 deg (sine)
    \draw[samples=100, domain=0:1] plot (\x, {2.5 - 0.3*sin(3600*(\x - floor(\x)))});
    \draw[samples=100, domain=1:2] plot (\x, {2.5 + 0.3*sin(3600*(\x - floor(\x)))});
    \draw[samples=100, domain=2:3] plot (\x, {2.5 - 0.3*sin(3600*(\x - floor(\x)))});
    \draw[samples=100, domain=3:4] plot (\x, {2.5 + 0.3*sin(3600*(\x - floor(\x)))});
    \draw[samples=100, domain=4:5] plot (\x, {2.5 - 0.3*sin(3600*(\x - floor(\x)))});
    
    \foreach \x/\val in {0.5/$180^\circ$, 1.5/$0^\circ$, 2.5/$180^\circ$, 3.5/$0^\circ$, 4.5/$180^\circ$}
        \node at (\x, 1.8) {\val};
\end{tikzpicture}
\end{center}

\textbf{કાર્ય સિદ્ધાંત:}
\begin{itemize}
    \item ડિજિટલ 1: $0^\circ$ ફેઝ સાથે કેરિયર સિગ્નલ
    \item ડિજિટલ 0: $180^\circ$ ફેઝ સાથે કેરિયર સિગ્નલ (ઉલટું)
    \item એમ્પ્લિટ્યુડ સ્થિર રહે છે, માત્ર ફેઝ બદલાય છે
\end{itemize}

\begin{mnemonicbox}
\textbf{મેમરી ટ્રીક:} "PSKIT" - Phase Shift Keeps Information True
\end{mnemonicbox}
\end{solutionbox}

\questionmarks{3}{b}{4}
\textbf{1101001101 ના ક્રમ માટે MSK મોડ્યુલેશન વેવફોર્મ દોરો.}

\begin{solutionbox}
\textbf{MSK મોડ્યુલેશન વેવફોર્મ:}

\begin{center}
\begin{tikzpicture}
    % Input: 1 1 0 1 0 0 1 1 0 1
    \draw (0,2.5) node[left] {\gu{બાઇનરી ઇનપુટ}};
    \draw (0,2.5) -- (2,2.5) -- (2,2) -- (3,2) -- (3,2.5) -- (4,2.5) -- (4,2) -- (6,2) -- (6,2.5) -- (8,2.5) -- (8,2) -- (9,2) -- (9,2.5) -- (10,2.5);
    \foreach \x/\val in {0.5/1, 1.5/1, 2.5/0, 3.5/1, 4.5/0, 5.5/0, 6.5/1, 7.5/1, 8.5/0, 9.5/1}
        \node at (\x, 1.8) {\val};

    \draw (0,1) node[left] {MSK \gu{આઉટપુટ}};
    % Simplified representation of MSK (continuous phase, freq shift)
    % Not mathematically exact but visually representative of smoothness
    \draw[samples=200, domain=0:10] plot (\x, {1 + 0.3*sin((3600*(\x - floor(\x)) + 5*sin(360*\x)))}); 
    
    \node at (5, 0.2) {\gu{સતત ફેઝ ટ્રાન્ઝિશન}};
\end{tikzpicture}
\end{center}

\textbf{MSKના લક્ષણો:}
\begin{itemize}
    \item સતત ફેઝ ટ્રાન્ઝિશન (કોઈ ફેઝ જમ્પ નહીં)
    \item $f_1$ અને $f_2$ વચ્ચે ફ્રીક્વન્સી શિફ્ટ
    \item ન્યૂનતમ ફ્રીક્વન્સી સેપરેશન: $\Delta f = 1/(2T)$
    \item FSK કરતાં વધુ સ્મૂધ ટ્રાન્ઝિશન
\end{itemize}

\begin{tabulary}{\linewidth}{L L}
\hline
\textbf{લક્ષણ} & \textbf{MSK લક્ષણ} \\
\hline
ફેઝ કન્ટિન્યુઇટી & સતત, કોઈ અચાનક બદલાવ નહીં \\
ફ્રીક્વન્સી ડેવિએશન & ન્યૂનતમ શક્ય (1/2T) \\
સ્પેક્ટ્રલ એફિશિયન્સી & પરંપરાગત FSK કરતાં વધુ સારી \\
બેન્ડવિડ્થ & બીટ રેટનો 1.5 ગણો \\
\hline
\end{tabulary}

\begin{mnemonicbox}
\textbf{મેમરી ટ્રીક:} "MINIMUM SMOOTH" - MSK uses Minimum frequency separation with Smooth transitions
\end{mnemonicbox}
\end{solutionbox}

\questionmarks{3}{c}{7}
\textbf{1100101011 માટે BPSK અને QPSK મોડ્યુલેશન વેવફોર્મ દોરો.}

\begin{solutionbox}
\textbf{BPSK અને QPSK મોડ્યુલેશન વેવફોર્મ:}

\begin{center}
\begin{tikzpicture}
    % Input: 1 1 0 0 1 0 1 0 1 1
    \draw (0,5.5) node[left] {\gu{બાઇનરી ઇનપુટ}};
    \draw (0,5.5) -- (2,5.5) -- (2,5) -- (4,5) -- (4,5.5) -- (5,5.5) -- (5,5) -- (6,5) -- (6,5.5) -- (7,5.5) -- (7,5) -- (8,5) -- (8,5.5) -- (10,5.5);
    \foreach \x/\val in {0.5/1, 1.5/1, 2.5/0, 3.5/0, 4.5/1, 5.5/0, 6.5/1, 7.5/0, 8.5/1, 9.5/1}
        \node at (\x, 5.2) {\val};

    % BPSK Output
    \draw (0,4) node[left] {BPSK \gu{આઉટપુટ}};
    \foreach \x/\val in {0/1, 1/1, 2/0, 3/0, 4/1, 5/0, 6/1, 7/0, 8/1, 9/1} {
        \ifnum\val=1
            \draw[samples=20, domain=\x:\x+1] plot[variable=\u] (\u, {4 + 0.3*sin(3600*(\u - floor(\u)))});
        \else
            \draw[samples=20, domain=\x:\x+1] plot[variable=\u] (\u, {4 - 0.3*sin(3600*(\u - floor(\u)))});
        \fi
    }

    % QPSK - Simplified
    % I channel (odd bits: 1 1 1 1 1) -> 1 0 1 1 1 (based on index 1,3,5...) -> 1, 0, 1, 1, 1
    % Q channel (even bits: 1 0 0 0 1) -> 1, 0, 0, 0, 1
    
    \draw (0,2.5) node[left] {QPSK \gu{કમ્બાઈન્ડ}};
    \draw[samples=200, domain=0:10] plot (\x, {2.5 + 0.3*sin((3600*(\x - floor(\x)) + 90*(int(\x/2))))});
    
    \node at (5, 1.8) {\gu{દર 2 બિટ્સમાં ફેઝ શિફ્ટ}};

\end{tikzpicture}
\end{center}

\textbf{મુખ્ય તફાવતો:}
\begin{itemize}
    \item \textbf{BPSK}: 1 બીટ પ્રતિ સિમ્બોલ, 2 ફેઝ ($0^\circ$ અને $180^\circ$)
    \item \textbf{QPSK}: 2 બીટ પ્રતિ સિમ્બોલ, 4 ફેઝ ($45^\circ, 135^\circ, 225^\circ, 315^\circ$)
    \item \textbf{QPSK જોડી}: 00, 01, 10, 11 અલગ-અલગ ફેઝને મેપ કરે છે
\end{itemize}

\begin{tabulary}{\linewidth}{L L L L}
\hline
\textbf{મોડ્યુલેશન} & \textbf{બીટ્સ/સિમ્બોલ} & \textbf{ફેઝની સંખ્યા} & \textbf{બેન્ડવિડ્થ એફિશિયન્સી} \\
\hline
BPSK & 1 & 2 & 1 બીટ/Hz \\
QPSK & 2 & 4 & 2 બીટ/Hz \\
\hline
\end{tabulary}

\begin{mnemonicbox}
\textbf{મેમરી ટ્રીક:} "ONE-TWO" - ONE bit for BPSK, TWO bits for QPSK
\end{mnemonicbox}
\end{solutionbox}

\questionmarks{4}{a}{3}
\textbf{નીચેની પ્રોબેબીલીટી ક્રમ માટે હફમેન કોડનો ઉપયોગ કરીને ડેટાને એન્કોડ કરો. P = \{ 0.4, 0.2, 0.2, 0.1, 0.1\}}

\begin{solutionbox}
\textbf{હફમેન કોડિંગ પ્રક્રિયા:}

\begin{tabulary}{\linewidth}{L L L}
\hline
\textbf{સિમ્બોલ} & \textbf{પ્રોબેબિલિટી} & \textbf{હફમેન કોડ} \\
\hline
A & 0.4 & 0 \\
B & 0.2 & 10 \\
C & 0.2 & 11 \\
D & 0.1 & 110 \\
E & 0.1 & 111 \\
\hline
\end{tabulary}

\textbf{હફમેન ટ્રી:}

\begin{center}
\begin{tikzpicture}[level distance=1.5cm,
  level 1/.style={sibling distance=3cm},
  level 2/.style={sibling distance=1.5cm},
  level 3/.style={sibling distance=1.5cm}]
  \node {1.0}
    child {node {0.6}
      child {node {0.4 (A)} edge from parent node[left] {0}}
      child {node {0.2 (B)} edge from parent node[right] {1}}
    edge from parent node[left] {0}}
    child {node {0.4}
      child {node {0.2 (C)} edge from parent node[left] {0}}
      child {node {0.2}
        child {node {0.1 (D)} edge from parent node[left] {0}}
        child {node {0.1 (E)} edge from parent node[right] {1}}
      edge from parent node[right] {1}}
    edge from parent node[right] {1}};
\end{tikzpicture}
\end{center}

\begin{mnemonicbox}
\textbf{મેમરી ટ્રીક:} "Higher Probability Means Shorter Code"
\end{mnemonicbox}
\end{solutionbox}

\questionmarks{4}{b}{4}
\textbf{સંભાવના અને એન્ટ્રોપી વ્યાખ્યાયિત કરો.}

\begin{solutionbox}
\begin{tabulary}{\linewidth}{L L L L}
\hline
\textbf{સંકલ્પના} & \textbf{વ્યાખ્યા} & \textbf{સૂત્ર} & \textbf{મહત્વ} \\
\hline
\textbf{સંભાવના} & ઘટના ઘટવાની સંભાવનાનું માપ & $P(A) = \frac{\text{અનુકૂળ પરિણામો}}{\text{કુલ પરિણામો}}$ & કોમ્યુનિકેશનમાં અનિશ્ચિતતા મોડેલ કરવા માટે ઉપયોગી \\
\textbf{એન્ટ્રોપી} & સિસ્ટમમાં અનિશ્ચિતતા અથવા રેન્ડમનેસનું માપ & $H(X) = -\sum P(x_i) \log_2 P(x_i)$ & સરેરાશ માહિતી સામગ્રી દર્શાવે છે \\
\hline
\end{tabulary}

\textbf{મુખ્ય લક્ષણો:}
\begin{itemize}
    \item \textbf{સંભાવના રેન્જ}: $0 \le P(A) \le 1$
    \item \textbf{એન્ટ્રોપી એકમો}: બિટ્સ ($\log_2$ નો ઉપયોગ કરીને)
    \item \textbf{મહત્તમ એન્ટ્રોપી}: જ્યારે બધી ઘટનાઓ સમાન સંભાવના ધરાવે છે
    \item \textbf{ન્યૂનતમ એન્ટ્રોપી}: જ્યારે પરિણામ નિશ્ચિત હોય (સંભાવના = 1)
\end{itemize}

\begin{mnemonicbox}
\textbf{મેમરી ટ્રીક:} "PURE" - Probability Underpins Randomness Estimation
\end{mnemonicbox}
\end{solutionbox}

\questionmarks{4}{c}{7}
\textbf{CDMA ટેકનિકને વિગતવાર સમજાવો.}

\begin{solutionbox}
\textbf{CDMA (કોડ ડિવિઝન મલ્ટિપલ એક્સેસ):}

\begin{center}
\begin{tikzpicture}[node distance=2.5cm, auto]
    \node [gtu block] (data) {\gu{યુઝર ડેટા}};
    \node [gtu block, right of=data] (spread) {\gu{સ્પ્રેડિંગ}\\\gu{(યુનિક કોડ)}};
    \node [gtu block, right of=spread] (mod) {\gu{મોડ્યુલેશન}};
    \node [gtu block, right of=mod] (trans) {\gu{ટ્રાન્સમિશન}};
    \node [gtu block, right of=trans] (rec) {\gu{રિસેપ્શન}};
    
    \node [gtu block, below of=rec] (demod) {\gu{ડિમોડ્યુલેશન}};
    \node [gtu block, left of=demod] (despread) {\gu{ડિસ્પ્રેડિંગ}\\\gu{(મેચિંગ કોડ)}};
    \node [gtu block, left of=despread] (out) {\gu{મૂળ ડેટા}};

    \path [gtu arrow] (data) -- (spread);
    \path [gtu arrow] (spread) -- (mod);
    \path [gtu arrow] (mod) -- (trans);
    \path [gtu arrow] (trans) -- (rec);
    \path [gtu arrow] (rec) -- (demod);
    \path [gtu arrow] (demod) -- (despread);
    \path [gtu arrow] (despread) -- (out);
\end{tikzpicture}
\end{center}

\textbf{CDMA લક્ષણોનું કોષ્ટક:}

\begin{tabulary}{\linewidth}{L L}
\hline
\textbf{લક્ષણ} & \textbf{વર્ણન} \\
\hline
\textbf{એક્સેસ મેથડ} & બહુવિધ વપરાશકર્તાઓ એક જ ફ્રીક્વન્સી અને સમય શેર કરે છે \\
\textbf{વિભાજન} & વપરાશકર્તાઓને અનન્ય સ્પ્રેડિંગ કોડ દ્વારા અલગ પાડવામાં આવે છે \\
\textbf{સ્પ્રેડિંગ કોડ} & ઓર્થોગોનલ અથવા પ્સ્યુડો-ઓર્થોગોનલ સિક્વન્સ \\
\textbf{પ્રોસેસિંગ ગેઇન} & સ્પ્રેડ બેન્ડવિડ્થનો મૂળ બેન્ડવિડ્થ સાથેનો ગુણોત્તર \\
\textbf{મલ્ટિપલ એક્સેસ} & ફ્રીક્વન્સી અથવા સમય વિભાજનને બદલે કોડ સ્પેસનો ઉપયોગ કરે છે \\
\textbf{ઇન્ટરફેરન્સ રિજેક્શન} & નેરોબેન્ડ ઇન્ટરફેરન્સને નકારવાની અંતર્ગત ક્ષમતા \\
\hline
\end{tabulary}

\textbf{મુખ્ય ફાયદાઓ:}
\begin{itemize}
    \item \textbf{ક્ષમતા}: ઘણા કિસ્સાઓમાં FDMA/TDMA કરતાં વધારે
    \item \textbf{સુરક્ષા}: સ્પ્રેડિંગ કોડ દ્વારા અંતર્ગત એન્ક્રિપ્શન
    \item \textbf{મલ્ટિપાથ રિજેક્શન}: રેક રિસીવર મલ્ટિપાથ ઘટકોને જોડી શકે છે
    \item \textbf{સોફ્ટ હેન્ડઓફ}: મોબાઇલ એક સાથે બહુવિધ બેઝ સ્ટેશનો સાથે વાતચીત કરી શકે છે
\end{itemize}

\begin{mnemonicbox}
\textbf{મેમરી ટ્રીક:} "CODES" - Capacity Optimized with Direct-sequence Encoding Schemes
\end{mnemonicbox}
\end{solutionbox}

\questionmarks{4}{a}{3}
\textbf{નીચેના પ્રોબેબીલીટી ક્રમ માટે શેનોન ફેનો કોડનો ઉપયોગ કરીને ડેટાને એન્કોડ કરો. P = \{ 0.5, 0.25, 0.125, 0.125\}}

\begin{solutionbox}
\textbf{શેનોન-ફેનો કોડિંગ પ્રક્રિયા:}

\begin{tabulary}{\linewidth}{L L L}
\hline
\textbf{સિમ્બોલ} & \textbf{પ્રોબેબિલિટી} & \textbf{શેનોન-ફેનો કોડ} \\
\hline
A & 0.5 & 0 \\
B & 0.25 & 10 \\
C & 0.125 & 110 \\
D & 0.125 & 111 \\
\hline
\end{tabulary}

\textbf{શેનોન-ફેનો ટ્રી:}

\begin{center}
\begin{tikzpicture}[level distance=1.5cm,
  level 1/.style={sibling distance=3cm},
  level 2/.style={sibling distance=1.5cm},
  level 3/.style={sibling distance=1.5cm}]
  \node {1.0}
    child {node {0.5 (A)} edge from parent node[left] {0}}
    child {node {0.5}
      child {node {0.25 (B)} edge from parent node[left] {0}}
      child {node {0.25}
        child {node {0.125 (C)} edge from parent node[left] {0}}
        child {node {0.125 (D)} edge from parent node[right] {1}}
      edge from parent node[right] {1}}
    edge from parent node[right] {1}};
\end{tikzpicture}
\end{center}

\begin{mnemonicbox}
\textbf{મેમરી ટ્રીક:} "Split For Optimum" - Shannon-Fano splits groups for optimum coding
\end{mnemonicbox}
\end{solutionbox}

\questionmarks{4}{b}{4}
\textbf{ઈન્ફોર્મેશન અને ચેનલ કેપેસિટી વ્યાખ્યાયિત કરો.}

\begin{solutionbox}
\begin{tabulary}{\linewidth}{L L L L}
\hline
\textbf{સંકલ્પના} & \textbf{વ્યાખ્યા} & \textbf{સૂત્ર} & \textbf{મહત્વ} \\
\hline
\textbf{ઈન્ફોર્મેશન} & અનિશ્ચિતતામાં ઘટાડાનું માપ & $I(x) = -\log_2 P(x)$ & ઓછી સંભાવના ધરાવતી ઘટનાઓ વધુ માહિતી ધરાવે છે \\
\textbf{ચેનલ કેપેસિટી} & મહત્તમ દર જે પર નિર્ધારિત ત્રુટિ સાથે માહિતી પ્રસારિત કરી શકાય & $C = B \log_2(1 + S/N)$ & વિશ્વસનીય કોમ્યુનિકેશનની મૂળભૂત મર્યાદા \\
\hline
\end{tabulary}

\textbf{મુખ્ય મુદ્દાઓ:}
\begin{itemize}
    \item \textbf{ઈન્ફોર્મેશન એકમો}: બિટ્સ ($\log_2$ નો ઉપયોગ કરીને)
    \item \textbf{ચેનલ કેપેસિટી એકમો}: બિટ્સ પ્રતિ સેકન્ડ
    \item \textbf{કેપેસિટીને અસર કરતા પરિબળો}:
    \begin{itemize}
        \item બેન્ડવિડ્થ (B)
        \item સિગ્નલ-ટુ-નોઇઝ રેશિયો (S/N)
    \end{itemize}
\end{itemize}

\begin{mnemonicbox}
\textbf{મેમરી ટ્રીક:} "INCHES" - Information Numerically Calculated, Hopping through Efficient Shannon limit
\end{mnemonicbox}
\end{solutionbox}

\questionmarks{4}{c}{7}
\textbf{TDMA ટેકનિકને વિગતવાર સમજાવો.}

\begin{solutionbox}
\textbf{TDMA (ટાઇમ ડિવિઝન મલ્ટિપલ એક્સેસ):}

\begin{center}
\begin{tikzpicture}[node distance=2cm, auto]
    \node [gtu block] (mux) {\gu{મલ્ટિપ્લેક્સર}};
    \node [gtu block, left of=mux, yshift=1.5cm, xshift=-2cm] (u1) {\gu{યુઝર 1 (TS1)}};
    \node [gtu block, left of=mux, yshift=0.5cm, xshift=-2cm] (u2) {\gu{યુઝર 2 (TS2)}};
    \node [gtu block, left of=mux, yshift=-0.5cm, xshift=-2cm] (u3) {\gu{યુઝર 3 (TS3)}};
    \node [gtu block, left of=mux, yshift=-1.5cm, xshift=-2cm] (u4) {\gu{યુઝર 4 (TS4)}};

    \node [gtu block, right of=mux, node distance=3cm] (chan) {\gu{ચેનલ}};
    \node [gtu block, right of=chan, node distance=3cm] (demux) {\gu{ડિમલ્ટિપ્લેક્સર}};

    \node [gtu block, right of=demux, yshift=1.5cm, xshift=2cm] (r1) {\gu{યુઝર 1}};
    \node [gtu block, right of=demux, yshift=0.5cm, xshift=2cm] (r2) {\gu{યુઝર 2}};
    \node [gtu block, right of=demux, yshift=-0.5cm, xshift=2cm] (r3) {\gu{યુઝર 3}};
    \node [gtu block, right of=demux, yshift=-1.5cm, xshift=2cm] (r4) {\gu{યુઝર 4}};

    \path [gtu arrow] (u1) -- (mux);
    \path [gtu arrow] (u2) -- (mux);
    \path [gtu arrow] (u3) -- (mux);
    \path [gtu arrow] (u4) -- (mux);
    \path [gtu arrow] (mux) -- (chan);
    \path [gtu arrow] (chan) -- (demux);
    \path [gtu arrow] (demux) -- (r1);
    \path [gtu arrow] (demux) -- (r2);
    \path [gtu arrow] (demux) -- (r3);
    \path [gtu arrow] (demux) -- (r4);
\end{tikzpicture}
\end{center}

\textbf{TDMA લક્ષણોનું કોષ્ટક:}

\begin{tabulary}{\linewidth}{L L}
\hline
\textbf{લક્ષણ} & \textbf{વર્ણન} \\
\hline
\textbf{એક્સેસ મેથડ} & બહુવિધ વપરાશકર્તાઓ એક જ ફ્રીક્વન્સી અલગ-અલગ ટાઇમ સ્લોટમાં શેર કરે છે \\
\textbf{ફ્રેમ સ્ટ્રક્ચર} & સમય ફ્રેમમાં વિભાજિત, ફ્રેમ સ્લોટમાં વિભાજિત \\
\textbf{ગાર્ડ ટાઇમ} & ઓવરલેપ ટાળવા માટે સ્લોટ વચ્ચે ટૂંકા સમયગાળા \\
\textbf{સિન્ક્રોનાઇઝેશન} & ટ્રાન્સમિટર અને રિસીવર વચ્ચે ચોક્કસ ટાઇમિંગની જરૂર \\
\textbf{કાર્યક્ષમતા} & ઉચ્ચ સ્પેક્ટ્રમ ઉપયોગ \\
\textbf{પાવર કન્ઝમ્પશન} & ટ્રાન્સમિટર માત્ર સોંપાયેલા સ્લોટ દરમિયાન ચાલુ \\
\hline
\end{tabulary}

\textbf{TDMA ફ્રેમ સ્ટ્રક્ચર:}

\begin{center}
\begin{tikzpicture}
    \draw (0,0) rectangle (2,1) node[pos=0.5] {TS1};
    \draw (2,0) rectangle (4,1) node[pos=0.5] {TS2};
    \draw (4,0) rectangle (6,1) node[pos=0.5] {TS3};
    \draw (6,0) rectangle (8,1) node[pos=0.5] {TS4};
    \draw (8,0) rectangle (10,1) node[pos=0.5] {TS1...};
    
    \draw[<->] (0,1.2) -- (8,1.2) node[midway, above] {TDMA \gu{ફ્રેમ}};
\end{tikzpicture}
\end{center}

\begin{mnemonicbox}
\textbf{મેમરી ટ્રીક:} "TIME" - Transmission In Measured Epochs
\end{mnemonicbox}
\end{solutionbox}

\questionmarks{5}{a}{3}
\textbf{T1 કેરિયર સિસ્ટમ સમજાવો.}

\begin{solutionbox}
\textbf{T1 કેરિયર સિસ્ટમ:}

\begin{tabulary}{\linewidth}{L L}
\hline
\textbf{લક્ષણ} & \textbf{સ્પેસિફિકેશન} \\
\hline
\textbf{ડેટા રેટ} & 1.544 Mbps \\
\textbf{ચેનલ} & 24 વોઇસ ચેનલ \\
\textbf{વોઇસ સેમ્પલિંગ} & 8000 સેમ્પલ/સેકન્ડ \\
\textbf{સેમ્પલ સાઇઝ} & 8 બિટ્સ પ્રતિ સેમ્પલ \\
\textbf{ફ્રેમ સાઇઝ} & 193 બિટ્સ ($24\times8 + 1$) \\
\textbf{ફ્રેમ રેટ} & 8000 ફ્રેમ/સેકન્ડ \\
\hline
\end{tabulary}

\textbf{T1 ફ્રેમ સ્ટ્રક્ચર:}

\begin{center}
\begin{tikzpicture}
    \draw (0,0) rectangle (0.5,1) node[pos=0.5] {F};
    \draw (0.5,0) rectangle (2,1) node[pos=0.5] {Ch1};
    \draw (2,0) rectangle (3.5,1) node[pos=0.5] {Ch2};
    \draw (3.5,0) rectangle (5,1) node[pos=0.5] {...};
    \draw (5,0) rectangle (6.5,1) node[pos=0.5] {Ch24};
    
    \node[below] at (0.25,0) {1};
    \node[below] at (1.25,0) {8 \gu{બિટ્સ}};
    \node[below] at (2.75,0) {8 \gu{બિટ્સ}};
    \node[below] at (5.75,0) {8 \gu{બિટ્સ}};
    
    \draw[<->] (0,1.2) -- (6.5,1.2) node[midway, above] {T1 \gu{ફ્રેમ} (193 \gu{બિટ્સ})};
\end{tikzpicture}
\end{center}

\begin{mnemonicbox}
\textbf{મેમરી ટ્રીક:} "T1-24-8-8" - T1 has 24 channels, 8 bits, 8kHz
\end{mnemonicbox}
\end{solutionbox}

\questionmarks{5}{b}{4}
\textbf{ટાઈમ ડિવિઝન મલ્ટિપ્લેક્સિંગ ટેકનિક (TDM) ને વિગતવાર સમજાવો.}

\begin{solutionbox}
\textbf{ટાઇમ ડિવિઝન મલ્ટિપ્લેક્સિંગ (TDM):}

\begin{center}
\begin{tikzpicture}[node distance=2cm, auto]
    \node [gtu block] (mux) {\gu{મલ્ટિપ્લેક્સર}};
    \node [gtu block, left of=mux, yshift=1.5cm, xshift=-2cm] (s1) {\gu{સિગ્નલ 1}};
    \node [gtu block, left of=mux, yshift=0.5cm, xshift=-2cm] (s2) {\gu{સિગ્નલ 2}};
    \node [gtu block, left of=mux, yshift=-0.5cm, xshift=-2cm] (s3) {\gu{સિગ્નલ 3}};
    \node [gtu block, left of=mux, yshift=-1.5cm, xshift=-2cm] (s4) {\gu{સિગ્નલ 4}};

    \node [gtu block, right of=mux, node distance=3cm] (chan) {\gu{ચેનલ}};
    \node [gtu block, right of=chan, node distance=3cm] (demux) {\gu{ડિમલ્ટિપ્લેક્સર}};

    \node [gtu block, right of=demux, yshift=1.5cm, xshift=2cm] (o1) {\gu{સિગ્નલ 1}};
    \node [gtu block, right of=demux, yshift=0.5cm, xshift=2cm] (o2) {\gu{સિગ્નલ 2}};
    \node [gtu block, right of=demux, yshift=-0.5cm, xshift=2cm] (o3) {\gu{સિગ્નલ 3}};
    \node [gtu block, right of=demux, yshift=-1.5cm, xshift=2cm] (o4) {\gu{સિગ્નલ 4}};

    \path [gtu arrow] (s1) -- (mux);
    \path [gtu arrow] (s2) -- (mux);
    \path [gtu arrow] (s3) -- (mux);
    \path [gtu arrow] (s4) -- (mux);
    \path [gtu arrow] (mux) -- (chan);
    \path [gtu arrow] (chan) -- (demux);
    \path [gtu arrow] (demux) -- (o1);
    \path [gtu arrow] (demux) -- (o2);
    \path [gtu arrow] (demux) -- (o3);
    \path [gtu arrow] (demux) -- (o4);
\end{tikzpicture}
\end{center}

\textbf{TDM લક્ષણોનું કોષ્ટક:}

\begin{tabulary}{\linewidth}{L L}
\hline
\textbf{લક્ષણ} & \textbf{વર્ણન} \\
\hline
\textbf{સિદ્ધાંત} & બહુવિધ સિગ્નલ વારાફરતી લઈને એક ચેનલ શેર કરે છે \\
\textbf{સમય ફાળવણી} & દરેક સિગ્નલને નિશ્ચિત સમય સ્લોટ ફાળવવામાં આવે છે \\
\textbf{સિન્ક્રોનાઇઝેશન} & મલ્ટિપ્લેક્સર અને ડિમલ્ટિપ્લેક્સર વચ્ચે ચોક્કસ ટાઇમિંગની જરૂર \\
\textbf{ઇન્ટરલીવિંગ} & વિવિધ સ્ત્રોતોના સેમ્પલ સમયમાં ઇન્ટરલીવ્ડ \\
\textbf{પ્રકારો} & સિન્ક્રોનસ TDM અને એસિન્ક્રોનસ (સ્ટેટિસ્ટિકલ) TDM \\
\hline
\end{tabulary}

\textbf{TDM ફ્રેમ સ્ટ્રક્ચર:}

\begin{center}
\begin{tikzpicture}
    \draw (0,0) rectangle (1.5,1) node[pos=0.5] {Si1};
    \draw (1.5,0) rectangle (3,1) node[pos=0.5] {Si2};
    \draw (3,0) rectangle (4.5,1) node[pos=0.5] {Si3};
    \draw (4.5,0) rectangle (6,1) node[pos=0.5] {Si4};
    \draw (6,0) rectangle (7.5,1) node[pos=0.5] {Si1...};
    
    \draw[<->] (0,1.2) -- (6,1.2) node[midway, above] {TDM \gu{ફ્રેમ}};
\end{tikzpicture}
\end{center}

\begin{mnemonicbox}
\textbf{મેમરી ટ્રીક:} "TWIST" - Time Windows Interleaving Signals Together
\end{mnemonicbox}
\end{solutionbox}

\questionmarks{5}{c}{7}
\textbf{ઇન્ફોમેશન સિક્યોરિટીમાં આવતા સિક્યોરિટી ઘટકોને વિગતવાર સમજાવો.}

\begin{solutionbox}
\textbf{ઇન્ફોર્મેશન સિક્યોરિટી ઘટકો:}

\begin{center}
\begin{tikzpicture}[node distance=2cm, auto]
    \node [gtu block] (sec) {\gu{ઇન્ફોર્મેશન}\\\gu{સિક્યોરિટી}};
    \node [gtu block, below left of=sec, node distance=3cm] (conf) {\gu{કોન્ફિડેન્શિયાલિટી}};
    \node [gtu block, below of=sec, node distance=2.5cm] (int) {\gu{ઇન્ટેગ્રિટી}};
    \node [gtu block, below right of=sec, node distance=3cm] (avail) {\gu{એવેલેબિલિટી}};
    
    \node [gtu block, below of=conf, node distance=2cm] (crypt) {\gu{એન્ક્રિપ્શન}\\\gu{એક્સેસ કંટ્રોલ}};
    \node [gtu block, below of=int, node distance=2cm] (hash) {\gu{ડિજિટલ સિગ્નેચર}\\\gu{હેશિંગ}};
    \node [gtu block, below of=avail, node distance=2cm] (backup) {\gu{રેડન્ડન્સી}\\\gu{બેકઅપ}};

    \path [gtu arrow] (sec) -- (conf);
    \path [gtu arrow] (sec) -- (int);
    \path [gtu arrow] (sec) -- (avail);
    \path [gtu arrow] (conf) -- (crypt);
    \path [gtu arrow] (int) -- (hash);
    \path [gtu arrow] (avail) -- (backup);
\end{tikzpicture}
\end{center}

\textbf{સિક્યોરિટી ઘટકોનું કોષ્ટક:}

\begin{tabulary}{\linewidth}{L L L}
\hline
\textbf{ઘટક} & \textbf{વર્ણન} & \textbf{અમલીકરણ પદ્ધતિઓ} \\
\hline
\textbf{કોન્ફિડેન્શિયાલિટી} & માહિતી માત્ર અધિકૃત વપરાશકર્તાઓને જ ઉપલબ્ધ થાય તેની ખાતરી & એન્ક્રિપ્શન, એક્સેસ કંટ્રોલ, ઓથેન્ટિકેશન \\
\textbf{ઇન્ટેગ્રિટી} & ડેટાની સચોટતા અને સુસંગતતા જાળવવી & ડિજિટલ સિગ્નેચર, હેશિંગ, ચેકસમ \\
\textbf{એવેલેબિલિટી} & જ્યારે જરૂર હોય ત્યારે માહિતી ઉપલબ્ધ થાય તેની ખાતરી & રેડન્ડન્સી, બેકઅપ સિસ્ટમ, ડિઝાસ્ટર રિકવરી \\
\textbf{ઓથેન્ટિકેશન} & વપરાશકર્તાઓની ઓળખની ચકાસણી & પાસવર્ડ, બાયોમેટ્રિક્સ, ડિજિટલ સર્ટિફિકેટ \\
\textbf{નોન-રેપ્યુડિએશન} & માહિતી મોકલવા/પ્રાપ્ત કરવાના ઇન્કારને રોકવું & ડિજિટલ સિગ્નેચર, ઓડિટ ટ્રેઇલ્સ \\
\hline
\end{tabulary}

\begin{mnemonicbox}
\textbf{મેમરી ટ્રીક:} "CIA" - Confidentiality, Integrity, Availability
\end{mnemonicbox}
\end{solutionbox}

\questionmarks{5}{a}{3}
\textbf{E1 કેરિયર સિસ્ટમ સમજાવો.}

\begin{solutionbox}
\textbf{E1 કેરિયર સિસ્ટમ:}

\begin{tabulary}{\linewidth}{L L}
\hline
\textbf{લક્ષણ} & \textbf{સ્પેસિફિકેશન} \\
\hline
\textbf{ડેટા રેટ} & 2.048 Mbps \\
\textbf{ચેનલ} & 32 ટાઇમ સ્લોટ (30 વોઇસ + 2 સિગ્નલિંગ) \\
\textbf{વોઇસ સેમ્પલિંગ} & 8000 સેમ્પલ/સેકન્ડ \\
\textbf{સેમ્પલ સાઇઝ} & 8 બિટ્સ પ્રતિ સેમ્પલ \\
\textbf{ફ્રેમ સાઇઝ} & 256 બિટ્સ ($32\times8$) \\
\textbf{ફ્રેમ રેટ} & 8000 ફ્રેમ/સેકન્ડ \\
\hline
\end{tabulary}

\textbf{E1 ફ્રેમ સ્ટ્રક્ચર:}

\begin{center}
\begin{tikzpicture}
    \draw (0,0) rectangle (1.5,1) node[pos=0.5] {TS0};
    \draw (1.5,0) rectangle (3,1) node[pos=0.5] {TS1};
    \draw (3,0) rectangle (4.5,1) node[pos=0.5] {...};
    \draw (4.5,0) rectangle (6,1) node[pos=0.5] {TS16};
    \draw (6,0) rectangle (7.5,1) node[pos=0.5] {...};
    \draw (7.5,0) rectangle (9,1) node[pos=0.5] {TS31};
    
    \node[below] at (0.75,0) {\gu{એલાઇન}};
    \node[below] at (5.25,0) {\gu{સિગ્નલ}};
    
    \draw[<->] (0,1.2) -- (9,1.2) node[midway, above] {E1 \gu{ફ્રેમ} (256 \gu{બિટ્સ})};
\end{tikzpicture}
\end{center}

\begin{mnemonicbox}
\textbf{મેમરી ટ્રીક:} "E1-32-8-8" - E1 has 32 channels, 8 bits, 8kHz
\end{mnemonicbox}
\end{solutionbox}

\questionmarks{5}{b}{4}
\textbf{ફ્રીક્વન્સી ડિવિઝન મલ્ટિપ્લેક્સિંગ ટેકનિક (FDM) ને વિગતવાર સમજાવો.}

\begin{solutionbox}
\textbf{ફ્રીક્વન્સી ડિવિઝન મલ્ટિપ્લેક્સિંગ (FDM):}

\begin{center}
\begin{tikzpicture}[node distance=2.5cm, auto]
    \node [gtu block] (comb) {\gu{કમ્બાઇનર}};
    \node [gtu block, left of=comb, yshift=1.5cm, xshift=-2cm] (m1) {\gu{મોડ્યુલેટર} $f_1$};
    \node [gtu block, left of=comb, yshift=0.5cm, xshift=-2cm] (m2) {\gu{મોડ્યુલેટર} $f_2$};
    \node [gtu block, left of=comb, yshift=-0.5cm, xshift=-2cm] (m3) {\gu{મોડ્યુલેટર} $f_3$};
    
    \node [left of=m1, node distance=2cm] {\gu{સિગ્નલ 1}};
    \node [left of=m2, node distance=2cm] {\gu{સિગ્નલ 2}};
    \node [left of=m3, node distance=2cm] {\gu{સિગ્નલ 3}};

    \node [gtu block, right of=comb, node distance=3cm] (chan) {\gu{ચેનલ}};
    \node [gtu block, right of=chan, node distance=3cm] (filt) {\gu{ફિલ્ટર્સ}};

    \node [gtu block, right of=filt, yshift=1.5cm, xshift=2cm] (d1) {\gu{ડિમોડ્યુલેટર} $f_1$};
    \node [gtu block, right of=filt, yshift=0.5cm, xshift=2cm] (d2) {\gu{ડિમોડ્યુલેટર} $f_2$};
    \node [gtu block, right of=filt, yshift=-0.5cm, xshift=2cm] (d3) {\gu{ડિમોડ્યુલેટર} $f_3$};

    \path [gtu arrow] (m1) -- (comb);
    \path [gtu arrow] (m2) -- (comb);
    \path [gtu arrow] (m3) -- (comb);
    \path [gtu arrow] (comb) -- (chan);
    \path [gtu arrow] (chan) -- (filt);
    \path [gtu arrow] (filt) -- (d1);
    \path [gtu arrow] (filt) -- (d2);
    \path [gtu arrow] (filt) -- (d3);
\end{tikzpicture}
\end{center}

\textbf{FDM સ્પેક્ટ્રમ:}

\begin{center}
\begin{tikzpicture}
    \draw[->] (0,0) -- (10,0) node[right] {\gu{ફ્રીક્વન્સી}};
    \draw[->] (0,0) -- (0,2) node[above] {\gu{પાવર}};
    
    \draw (1,0) to[out=90,in=180] (1.5,1.5) to[out=0,in=90] (2,0);
    \draw (3,0) to[out=90,in=180] (3.5,1.5) to[out=0,in=90] (4,0);
    \draw (5,0) to[out=90,in=180] (5.5,1.5) to[out=0,in=90] (6,0);
    \draw (7,0) to[out=90,in=180] (7.5,1.5) to[out=0,in=90] (8,0);
    
    \node at (1.5, -0.5) {Ch1};
    \node at (3.5, -0.5) {Ch2};
    \node at (5.5, -0.5) {Ch3};
    \node at (7.5, -0.5) {Ch4};
    
    \draw[<->] (2,1) -- (3,1) node[midway, above] {\gu{ગાર્ડ}};
    \draw[<->] (4,1) -- (5,1) node[midway, above] {\gu{બેન્ડ}};
\end{tikzpicture}
\end{center}

\begin{mnemonicbox}
\textbf{મેમરી ટ્રીક:} "FROG" - FRequencies Organized with Gaps
\end{mnemonicbox}
\end{solutionbox}

\questionmarks{5}{c}{7}
\textbf{ઈન્ટરનેટ ઓફ થિંગ્સ (IoT) ના ખ્યાલ અને મુખ્ય લક્ષણો સમજાવો.}

\begin{solutionbox}
\textbf{ઈન્ટરનેટ ઓફ થિંગ્સ (IoT) ખ્યાલ:}

\begin{center}
\begin{tikzpicture}[node distance=2.5cm, auto]
    \node [gtu block, circle, text width=2cm] (iot) {\gu{ઈન્ટરનેટ ઓફ}\\\gu{થિંગ્સ}};
    
    \node [gtu block, above of=iot] (data) {\gu{ડેટા કલેક્શન}\\\gu{(સેન્સર્સ)}};
    \node [gtu block, right of=iot] (proc) {\gu{એનાલિટિક્સ}\\\gu{(AI/ML)}};
    \node [gtu block, below of=iot] (act) {\gu{એક્શન}\\\gu{(એક્ચ્યુએટર્સ)}};
    \node [gtu block, left of=iot] (conn) {\gu{કનેક્ટિવિટી}\\\gu{(ક્લાઉડ)}};
    
    \path [gtu arrow] (iot) -- (data);
    \path [gtu arrow] (iot) -- (proc);
    \path [gtu arrow] (iot) -- (act);
    \path [gtu arrow] (iot) -- (conn);
\end{tikzpicture}
\end{center}

\textbf{IoT આર્કિટેક્ચર લેયર્સ:}

\begin{center}
\begin{tikzpicture}
    \draw[fill=blue!10] (0,4) rectangle (6,5) node[midway] {\gu{એપ્લિકેશન લેયર}};
    \draw[fill=green!10] (0,3) rectangle (6,4) node[midway] {\gu{ડેટા પ્રોસેસિંગ લેયર}};
    \draw[fill=yellow!10] (0,2) rectangle (6,3) node[midway] {\gu{નેટવર્ક લેયર}};
    \draw[fill=red!10] (0,1) rectangle (6,2) node[midway] {\gu{પરસેપ્શન લેયર (સેન્સર્સ)}};
\end{tikzpicture}
\end{center}

\textbf{IoTના મુખ્ય લક્ષણોનું કોષ્ટક:}

\begin{tabulary}{\linewidth}{L L}
\hline
\textbf{લક્ષણ} & \textbf{વર્ણન} \\
\hline
\textbf{કનેક્ટિવિટી} & ડિવાઇસીસ ઇન્ટરનેટ અને એકબીજા સાથે જોડાયેલી \\
\textbf{ઇન્ટેલિજન્સ} & સ્માર્ટ પ્રોસેસિંગ, નિર્ણય લેવાની ક્ષમતાઓ \\
\textbf{સેન્સિંગ} & સેન્સર્સ દ્વારા પર્યાવરણમાંથી ડેટા એકત્રિત કરવો \\
\textbf{એક્સપ્રેસિંગ} & એક્ચ્યુએટર્સ દ્વારા કાર્યવાહી કરવી \\
\textbf{એનર્જી એફિશિયન્સી} & બેટરી-સંચાલિત ડિવાઇસીસ માટે ઓછી પાવર વપરાશ \\
\textbf{સિક્યોરિટી} & અનધિકૃત એક્સેસ અને હુમલાઓથી સુરક્ષા \\
\textbf{સ્કેલેબિલિટી} & નેટવર્કમાં વધુ ડિવાઇસીસ ઉમેરવાની ક્ષમતા \\
\hline
\end{tabulary}

\begin{mnemonicbox}
\textbf{મેમરી ટ્રીક:} "CASED" - Connected, Automated, Sensing, Expressing, Data-driven
\end{mnemonicbox}
\end{solutionbox}

\end{document}
