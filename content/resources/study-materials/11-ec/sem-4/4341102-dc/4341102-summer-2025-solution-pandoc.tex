\documentclass[10pt,a4paper]{article}

% content/resources/templates/preamble.tex
\usepackage[margin=0.6in]{geometry}
\author{Milav Dabgar}
\usepackage{amsmath,amssymb,amsthm}
\usepackage{booktabs}
\usepackage{multirow}
\usepackage{xcolor}
\usepackage{tcolorbox}
\tcbuselibrary{breakable,skins}
\usepackage[colorlinks=true,linkcolor=blue]{hyperref}
\usepackage{titlesec}
\usepackage{enumitem}
\usepackage{tikz}
\usepackage{pgfplots}
\usepackage{circuitikz}
\usepackage[version=4]{mhchem}
\usepackage{longtable}
\usepackage{array}
\usepackage{float}
\usepackage{caption}
\usepackage{listings}

\lstset{
  basicstyle=\small\ttfamily,
  breaklines=true,
  breakatwhitespace=false,
  postbreak=\mbox{\textcolor{red}{$\hookrightarrow$}\space},
  float=false,
  numbers=left,
  numberstyle=\tiny\color{gray},
  numbersep=10pt,
  xleftmargin=2em,
  keywordstyle=\color{blue},
  commentstyle=\color{green!60!black},
  stringstyle=\color{purple},
  backgroundcolor=\color{gray!5},
  showstringspaces=false,
  tabsize=2,
  captionpos=b,
  keepspaces=true,
  columns=flexible
}

\pgfplotsset{compat=1.18}
\usetikzlibrary{shapes,arrows,positioning,calc,patterns,decorations.pathmorphing,decorations.markings,arrows.meta}

% Color scheme
\definecolor{headcolor}{RGB}{0,102,204}
\definecolor{keycolor}{RGB}{220,20,60}
\definecolor{solutioncolor}{RGB}{34,139,34}
\definecolor{mnemoniccolor}{RGB}{148,0,211}
\definecolor{codecolor}{RGB}{0,0,100}

% Spacing
\setlength{\parskip}{3pt}
\setlist[itemize]{nosep}
\setlist[enumerate]{nosep}

% Title formatting
\titleformat{\section}{\Large\bfseries\color{headcolor}}{\thesection}{1em}{}
\titleformat{\subsection}{\large\bfseries\color{headcolor}}{\thesubsection}{1em}{}

% Pandoc tightlist compatibility
\providecommand{\tightlist}{%
  \setlength{\itemsep}{0pt}\setlength{\parskip}{0pt}}

% Pandoc longtable compatibility
\newcounter{none}
\def\thenone{}


% content/resources/templates/english-boxes.tex
% This file is currently empty - it exists to maintain consistency with the import structure.
% Add custom environments here if needed in the future.


\begin{document}

\begin{center}
{\Huge\bfseries\color{headcolor} Subject Name Solutions}\\[5pt]
{\LARGE 4341102 -- Summer 2025}\\[3pt]
{\large Semester 1 Study Material}\\[3pt]
{\normalsize\textit{Detailed Solutions and Explanations}}
\end{center}

\vspace{10pt}

\subsection*{Question 1(a) [3 marks]}\label{q1a}

\textbf{Draw block diagram of digital communication system.}

\begin{solutionbox}

\begin{center}
\textbf{Mermaid Diagram (Code)}
\begin{verbatim}
{Shaded}
{Highlighting}[]
graph LR
    A[Information Source] {-{-}{} B[Source Encoder]}
    B {-{-}{} C[Channel Encoder]}
    C {-{-}{} D[Digital Modulator]}
    D {-{-}{} E[Channel]}
    E {-{-}{} F[Digital Demodulator]}
    F {-{-}{} G[Channel Decoder]}
    G {-{-}{} H[Source Decoder]}
    H {-{-}{} I[Information Sink]}
    
    J[Noise Source] {-{-}{} E}
{Highlighting}
{Shaded}
\end{verbatim}
\end{center}

\textbf{Key Components:}

\begin{itemize}
\tightlist
\item
  \textbf{Information Source}: Generates message signal
\item
  \textbf{Source Encoder}: Converts analog to digital
\item
  \textbf{Channel Encoder}: Adds error correction codes
\item
  \textbf{Digital Modulator}: Converts digital bits to analog signal
\end{itemize}

\end{solutionbox}
\begin{mnemonicbox}
``Source Channel Modulator travels through Channel to
Demodulator Channel Sink''

\end{mnemonicbox}
\subsection*{Question 1(b) [4 marks]}\label{q1b}

\textbf{Write the function of transmitter and receiver of digital
communication system.}

\begin{solutionbox}

{\def\LTcaptype{none} % do not increment counter
\begin{longtable}[]{@{}
  >{\raggedright\arraybackslash}p{(\linewidth - 2\tabcolsep) * \real{0.5238}}
  >{\raggedright\arraybackslash}p{(\linewidth - 2\tabcolsep) * \real{0.4762}}@{}}
\toprule\noalign{}
\begin{minipage}[b]{\linewidth}\raggedright
Component
\end{minipage} & \begin{minipage}[b]{\linewidth}\raggedright
Function
\end{minipage} \\
\midrule\noalign{}
\endhead
\bottomrule\noalign{}
\endlastfoot
\textbf{Transmitter} & Converts information signal into suitable form
for transmission \\
\textbf{Source Encoder} & Analog to digital conversion, sampling,
quantization \\
\textbf{Channel Encoder} & Error detection and correction coding \\
\textbf{Digital Modulator} & Converts digital bits to analog waveform \\
\end{longtable}
}

{\def\LTcaptype{none} % do not increment counter
\begin{longtable}[]{@{}
  >{\raggedright\arraybackslash}p{(\linewidth - 2\tabcolsep) * \real{0.5238}}
  >{\raggedright\arraybackslash}p{(\linewidth - 2\tabcolsep) * \real{0.4762}}@{}}
\toprule\noalign{}
\begin{minipage}[b]{\linewidth}\raggedright
Component
\end{minipage} & \begin{minipage}[b]{\linewidth}\raggedright
Function
\end{minipage} \\
\midrule\noalign{}
\endhead
\bottomrule\noalign{}
\endlastfoot
\textbf{Receiver} & Recovers original information from received
signal \\
\textbf{Digital Demodulator} & Converts received analog signal to
digital bits \\
\textbf{Channel Decoder} & Error detection and correction \\
\textbf{Source Decoder} & Digital to analog conversion \\
\end{longtable}
}

\textbf{Key Functions:}

\begin{itemize}
\tightlist
\item
  \textbf{Signal Processing}: Encoding, modulation, filtering
\item
  \textbf{Error Control}: Detection and correction of transmission
  errors
\item
  \textbf{Signal Recovery}: Demodulation and decoding at receiver
\end{itemize}

\end{solutionbox}
\begin{mnemonicbox}
``Transmitter Encodes Modulates, Receiver Demodulates
Decodes''

\end{mnemonicbox}
\subsection*{Question 1(c) [7 marks]}\label{q1c}

\textbf{Define and explain with example: Continues time and discrete
time signals, Real and complex signals and even and odd signals.}

\begin{solutionbox}

{\def\LTcaptype{none} % do not increment counter
\begin{longtable}[]{@{}
  >{\raggedright\arraybackslash}p{(\linewidth - 4\tabcolsep) * \real{0.3824}}
  >{\raggedright\arraybackslash}p{(\linewidth - 4\tabcolsep) * \real{0.3529}}
  >{\raggedright\arraybackslash}p{(\linewidth - 4\tabcolsep) * \real{0.2647}}@{}}
\toprule\noalign{}
\begin{minipage}[b]{\linewidth}\raggedright
Signal Type
\end{minipage} & \begin{minipage}[b]{\linewidth}\raggedright
Definition
\end{minipage} & \begin{minipage}[b]{\linewidth}\raggedright
Example
\end{minipage} \\
\midrule\noalign{}
\endhead
\bottomrule\noalign{}
\endlastfoot
\textbf{Continuous Time} & Signal defined for all time values & x(t) =
sin(2πt) \\
\textbf{Discrete Time} & Signal defined only at specific time instants &
x[n] = sin(2πn/8) \\
\textbf{Real Signal} & Signal with real values only & x(t) = 5cos(t) \\
\textbf{Complex Signal} & Signal with real and imaginary parts & x(t) =
3 + j4sin(t) \\
\end{longtable}
}

\textbf{Even and Odd Signals:}

\begin{center}
\textbf{Mermaid Diagram (Code)}
\begin{verbatim}
{Shaded}
{Highlighting}[]
graph LR
    A["Signal x[n]"] {-{-}{} B\{"Check x[{-}n]"\}}
    B {-{-}{} C["x[n] = x[{-}n]{}br/{}EVEN SIGNAL"]}
    B {-{-}{} D["x[n] = {-}x[{-}n]{}br/{}ODD SIGNAL"]}
    
    C {-{-}{} E["Example: cos[n]"]}
    D {-{-}{} F["Example: sin[n]"]}
{Highlighting}
{Shaded}
\end{verbatim}
\end{center}

\textbf{Properties:}

\begin{itemize}
\tightlist
\item
  \textbf{Even Signal}: Symmetric about y-axis, x(t) = x(-t)
\item
  \textbf{Odd Signal}: Anti-symmetric about origin, x(t) = -x(-t)
\item
  \textbf{Complex Signal}: z(t) = x(t) + jy(t)
\item
  \textbf{Discrete Signal}: Sampled version of continuous signal
\end{itemize}

\end{solutionbox}
\begin{mnemonicbox}
``Continuous Everywhere, Discrete Specific, Real
Simple, Complex Combined''

\end{mnemonicbox}
\subsection*{Question 1(c OR) [7
marks]}\label{question-1c-or-7-marks}

\textbf{Define and explain with example: Unit step function, Unit
impulse function, Unit ramp function}

\begin{solutionbox}

{\def\LTcaptype{none} % do not increment counter
\begin{longtable}[]{@{}lll@{}}
\toprule\noalign{}
Function & Definition & Mathematical Form \\
\midrule\noalign{}
\endhead
\bottomrule\noalign{}
\endlastfoot
\textbf{Unit Step} & u(t) = 1 for t\geq0, 0 for t\textless0 & u(t) = 1 for
t\geq0 \\
\textbf{Unit Impulse} & δ(t) = \infty for t=0, 0 elsewhere & \intδ(t)dt = 1 \\
\textbf{Unit Ramp} & r(t) = t for t\geq0, 0 for t\textless0 & r(t) =
t·u(t) \\
\end{longtable}
}

\begin{verbatim}
Unit Step Function:        Unit Impulse Function:      Unit Ramp Function:
                          
    1 |{-{-}{-}{-}                    |                           |  /}
      |                         | |                         | /
    0 |\_\_\_\_                   0 |\_|\_\_\_\_                   0 |/\_\_\_\_
      0    t                     0    t                     0    t
\end{verbatim}

\textbf{Applications:}

\begin{itemize}
\tightlist
\item
  \textbf{Unit Step}: Switch operations, system response analysis
\item
  \textbf{Unit Impulse}: System impulse response, convolution
\item
  \textbf{Unit Ramp}: System ramp response, integration
\end{itemize}

\textbf{Properties:}

\begin{itemize}
\tightlist
\item
  \textbf{Step}: Derivative of ramp, integral of impulse
\item
  \textbf{Impulse}: Derivative of step function
\item
  \textbf{Ramp}: Integral of step function
\end{itemize}

\end{solutionbox}
\begin{mnemonicbox}
``Step Sudden, Impulse Instant, Ramp Rising''

\end{mnemonicbox}
\subsection*{Question 2(a) [3 marks]}\label{q2a}

\textbf{Define: bit rate, baud rate and bandwidth.}

\begin{solutionbox}

{\def\LTcaptype{none} % do not increment counter
\begin{longtable}[]{@{}
  >{\raggedright\arraybackslash}p{(\linewidth - 4\tabcolsep) * \real{0.3793}}
  >{\raggedright\arraybackslash}p{(\linewidth - 4\tabcolsep) * \real{0.4138}}
  >{\raggedright\arraybackslash}p{(\linewidth - 4\tabcolsep) * \real{0.2069}}@{}}
\toprule\noalign{}
\begin{minipage}[b]{\linewidth}\raggedright
Parameter
\end{minipage} & \begin{minipage}[b]{\linewidth}\raggedright
Definition
\end{minipage} & \begin{minipage}[b]{\linewidth}\raggedright
Unit
\end{minipage} \\
\midrule\noalign{}
\endhead
\bottomrule\noalign{}
\endlastfoot
\textbf{Bit Rate} & Number of bits transmitted per second & bps (bits
per second) \\
\textbf{Baud Rate} & Number of signal changes per second & Baud (symbols
per second) \\
\textbf{Bandwidth} & Range of frequencies in signal & Hz (Hertz) \\
\end{longtable}
}

\textbf{Relationship:}

\begin{itemize}
\tightlist
\item
  Bit Rate = Baud Rate \times log_{2}(M)
\item
  M = number of signal levels
\item
  Bandwidth ∝ Baud Rate
\end{itemize}

\textbf{Key Points:}

\begin{itemize}
\tightlist
\item
  \textbf{Higher bit rate}: More data transmission
\item
  \textbf{Baud rate}: Symbol transmission rate
\item
  \textbf{Bandwidth}: Frequency spectrum occupied
\end{itemize}

\end{solutionbox}
\begin{mnemonicbox}
``Bits Baud Bandwidth - Data Symbol Frequency''

\end{mnemonicbox}
\subsection*{Question 2(b) [4 marks]}\label{q2b}

\textbf{Explain Energy and power signal.}

\begin{solutionbox}

{\def\LTcaptype{none} % do not increment counter
\begin{longtable}[]{@{}
  >{\raggedright\arraybackslash}p{(\linewidth - 4\tabcolsep) * \real{0.3023}}
  >{\raggedright\arraybackslash}p{(\linewidth - 4\tabcolsep) * \real{0.2791}}
  >{\raggedright\arraybackslash}p{(\linewidth - 4\tabcolsep) * \real{0.4186}}@{}}
\toprule\noalign{}
\begin{minipage}[b]{\linewidth}\raggedright
Signal Type
\end{minipage} & \begin{minipage}[b]{\linewidth}\raggedright
Definition
\end{minipage} & \begin{minipage}[b]{\linewidth}\raggedright
Mathematical Form
\end{minipage} \\
\midrule\noalign{}
\endhead
\bottomrule\noalign{}
\endlastfoot
\textbf{Energy Signal} & Finite energy, zero average power & E = \int \\
\textbf{Power Signal} & Finite average power, infinite energy & P =
lim(T\rightarrow\infty) 1/T \int \\
\end{longtable}
}

\textbf{Classification:}

\begin{center}
\textbf{Mermaid Diagram (Code)}
\begin{verbatim}
{Shaded}
{Highlighting}[]
graph LR
    A[Signal] {-{-}{} B\{Energy Finite?\}}
    B {-{-}{}|Yes| C[Energy Signal{}br/{}P = 0]}
    B {-{-}{}|No| D\{Power Finite?\}}
    D {-{-}{}|Yes| E[Power Signal{}br/{}E = ]}
    D {-{-}{}|No| F[Neither Energy{}br/{}nor Power]}
{Highlighting}
{Shaded}
\end{verbatim}
\end{center}

\textbf{Examples:}

\begin{itemize}
\tightlist
\item
  \textbf{Energy Signal}: Exponentially decaying signal e\^{}(-t)u(t)
\item
  \textbf{Power Signal}: Sinusoidal signal sin(ωt)
\item
  \textbf{Neither}: Ramp signal t·u(t)
\end{itemize}

\textbf{Properties:}

\begin{itemize}
\tightlist
\item
  Energy and power signals are mutually exclusive
\item
  Periodic signals are generally power signals
\item
  Non-periodic finite duration signals are energy signals
\end{itemize}

\end{solutionbox}
\begin{mnemonicbox}
``Energy Ends, Power Persists''

\end{mnemonicbox}
\subsection*{Question 2(c) [7 marks]}\label{q2c}

\textbf{Give the comparison between ASK, FSK and PSK modulation
techniques and draw their waveforms.}

\begin{solutionbox}

{\def\LTcaptype{none} % do not increment counter
\begin{longtable}[]{@{}
  >{\raggedright\arraybackslash}p{(\linewidth - 6\tabcolsep) * \real{0.4231}}
  >{\raggedright\arraybackslash}p{(\linewidth - 6\tabcolsep) * \real{0.1923}}
  >{\raggedright\arraybackslash}p{(\linewidth - 6\tabcolsep) * \real{0.1923}}
  >{\raggedright\arraybackslash}p{(\linewidth - 6\tabcolsep) * \real{0.1923}}@{}}
\toprule\noalign{}
\begin{minipage}[b]{\linewidth}\raggedright
Parameter
\end{minipage} & \begin{minipage}[b]{\linewidth}\raggedright
ASK
\end{minipage} & \begin{minipage}[b]{\linewidth}\raggedright
FSK
\end{minipage} & \begin{minipage}[b]{\linewidth}\raggedright
PSK
\end{minipage} \\
\midrule\noalign{}
\endhead
\bottomrule\noalign{}
\endlastfoot
\textbf{Full Form} & Amplitude Shift Keying & Frequency Shift Keying &
Phase Shift Keying \\
\textbf{Varied Parameter} & Amplitude & Frequency & Phase \\
\textbf{Bandwidth} & Narrow & Wide & Narrow \\
\textbf{Noise Immunity} & Poor & Good & Excellent \\
\textbf{Power Efficiency} & Poor & Good & Excellent \\
\textbf{Implementation} & Simple & Moderate & Complex \\
\end{longtable}
}

\begin{verbatim}
ASK Waveform:
Data:    1    0    1    1    0
        \_\_\_       \_\_\_  \_\_\_      
       |   |     |   ||   |     
    \_\_\_|   |\_\_\_\_\_|   ||   |\_\_\_\_\_
    
FSK Waveform:
       {              }
      {                   }
    {                   }
    
PSK Waveform:
       \_\_\_       \_\_\_ \_\_\_      
      |   |     |   |   |     
    \_\_|   |\_\_\_\_\_|   |   |\_\_\_\_\_
      phase shift at data change
\end{verbatim}

\textbf{Applications:}

\begin{itemize}
\tightlist
\item
  \textbf{ASK}: Optical communication, simple radio systems
\item
  \textbf{FSK}: Telephone modems, radio systems
\item
  \textbf{PSK}: Satellite communication, wireless systems
\end{itemize}

\textbf{Advantages:}

\begin{itemize}
\tightlist
\item
  \textbf{ASK}: Simple implementation, low cost
\item
  \textbf{FSK}: Good noise performance, constant envelope
\item
  \textbf{PSK}: Best noise performance, bandwidth efficient
\end{itemize}

\end{solutionbox}
\begin{mnemonicbox}
``ASK Amplitude, FSK Frequency, PSK Phase''

\end{mnemonicbox}
\subsection*{Question 2(a OR) [3
marks]}\label{question-2a-or-3-marks}

\textbf{A bit rate of signal generator from 8-bit generator is 1600 bps.
Calculate the baud rate of signal.}

\begin{solutionbox}

\textbf{Given:}

\begin{itemize}
\tightlist
\item
  Bit rate = 1600 bps
\item
  Number of bits per symbol = 8 bits
\end{itemize}

\textbf{Formula:} Baud Rate = Bit Rate / Number of bits per symbol

\textbf{Calculation:} Baud Rate = 1600 bps / 8 bits Baud Rate = 200 Baud

\textbf{Result:} The baud rate of the signal is \textbf{200 Baud}.

\textbf{Explanation:}

\begin{itemize}
\tightlist
\item
  Each symbol carries 8 bits of information
\item
  1600 bits per second \div 8 bits per symbol = 200 symbols per second
\item
  Therefore, baud rate = 200 Baud
\end{itemize}

\end{solutionbox}
\begin{mnemonicbox}
``Bit Rate divided by Bits per Symbol gives Baud''

\end{mnemonicbox}
\subsection*{Question 2(b OR) [4
marks]}\label{question-2b-or-4-marks}

\textbf{Find whether the signals are even or odd:} \textbf{1. x(t) =
e\^{}(-5t)} \textbf{2. x(t) = sin 2t} \textbf{3. x(t) = cos 5t}

\begin{solutionbox}

{\def\LTcaptype{none} % do not increment counter
\begin{longtable}[]{@{}
  >{\raggedright\arraybackslash}p{(\linewidth - 6\tabcolsep) * \real{0.2353}}
  >{\raggedright\arraybackslash}p{(\linewidth - 6\tabcolsep) * \real{0.3529}}
  >{\raggedright\arraybackslash}p{(\linewidth - 6\tabcolsep) * \real{0.2353}}
  >{\raggedright\arraybackslash}p{(\linewidth - 6\tabcolsep) * \real{0.1765}}@{}}
\toprule\noalign{}
\begin{minipage}[b]{\linewidth}\raggedright
Signal
\end{minipage} & \begin{minipage}[b]{\linewidth}\raggedright
Test x(-t)
\end{minipage} & \begin{minipage}[b]{\linewidth}\raggedright
Result
\end{minipage} & \begin{minipage}[b]{\linewidth}\raggedright
Type
\end{minipage} \\
\midrule\noalign{}
\endhead
\bottomrule\noalign{}
\endlastfoot
x(t) = e\^{}(-5t) & x(-t) = e\^{}(5t) \neq x(t) \neq -x(t) & Neither & Neither
Even nor Odd \\
x(t) = sin 2t & x(-t) = sin(-2t) = -sin 2t = -x(t) & -x(t) & \textbf{Odd
Signal} \\
x(t) = cos 5t & x(-t) = cos(-5t) = cos 5t = x(t) & x(t) & \textbf{Even
Signal} \\
\end{longtable}
}

\textbf{Test Procedure:}

\begin{enumerate}
\tightlist
\item
  \textbf{Even Signal Test}: Check if x(t) = x(-t)
\item
  \textbf{Odd Signal Test}: Check if x(t) = -x(-t)
\end{enumerate}

\textbf{Properties Used:}

\begin{itemize}
\tightlist
\item
  \textbf{Exponential}: e\^{}(-at) is neither even nor odd (a
  \textgreater{} 0)
\item
  \textbf{Sine Function}: sin(-x) = -sin(x) \rightarrow Odd function
\item
  \textbf{Cosine Function}: cos(-x) = cos(x) \rightarrow Even function
\end{itemize}

\textbf{Results:}

\begin{itemize}
\tightlist
\item
  \textbf{Signal 1}: Neither even nor odd
\item
  \textbf{Signal 2}: Odd signal
\item
  \textbf{Signal 3}: Even signal
\end{itemize}

\end{solutionbox}
\begin{mnemonicbox}
``Cosine Even, Sine Odd, Exponential Neither''

\end{mnemonicbox}
\subsection*{Question 2(c OR) [7
marks]}\label{question-2c-or-7-marks}

\textbf{Explain the Principle of QPSK signal. Draw its modulator and
demodulator diagram. Also draw constellation diagram and waveforms of
its.}

\begin{solutionbox}

\textbf{QPSK Principle:} QPSK (Quadrature Phase Shift Keying) uses four
different phase states to represent 2 bits per symbol.

{\def\LTcaptype{none} % do not increment counter
\begin{longtable}[]{@{}llll@{}}
\toprule\noalign{}
Bits & Phase & I & Q \\
\midrule\noalign{}
\endhead
\bottomrule\noalign{}
\endlastfoot
00 & 45^\circ & +1 & +1 \\
01 & 135^\circ & -1 & +1 \\
10 & -45^\circ & +1 & -1 \\
11 & -135^\circ & -1 & -1 \\
\end{longtable}
}

\textbf{QPSK Modulator:}

\begin{center}
\textbf{Mermaid Diagram (Code)}
\begin{verbatim}
{Shaded}
{Highlighting}[]
graph LR
    A[Data Stream] {-{-}{} B[Serial to Parallel]}
    B {-{-}{} C[I Channel]}
    B {-{-}{} D[Q Channel]}
    C {-{-}{} E[Mixer 1]}
    D {-{-}{} F[Mixer 2]}
    G["Carrier cos(ωt)"] {-{-}{} E}
    H["Carrier sin(ωt)"] {-{-}{} F}
    E {-{-}{} I[Adder]}
    F {-{-}{} I}
    I {-{-}{} J[QPSK Output]}
{Highlighting}
{Shaded}
\end{verbatim}
\end{center}

\textbf{Constellation Diagram:}

\begin{verbatim}
        Q
        |
   01   |   00
  ({-1,1)| (1,1)}
        |
  {-{-}{-}{-}{-}{-}+{-}{-}{-}{-}{-}{-} I}
        |
  ({-1,{-}1)|(1,{-}1)}
   11   |   10
        |
\end{verbatim}

\textbf{QPSK Demodulator:}

\begin{center}
\textbf{Mermaid Diagram (Code)}
\begin{verbatim}
{Shaded}
{Highlighting}[]
graph LR
    A[QPSK Input] {-{-}{} B[Mixer 1]}
    A {-{-}{} C[Mixer 2]}
    D["cos(ωt)"] {-{-}{} B}
    E["sin(ωt)"] {-{-}{} C}
    B {-{-}{} F[LPF]}
    C {-{-}{} G[LPF]}
    F {-{-}{} H[Decision Device]}
    G {-{-}{} I[Decision Device]}
    H {-{-}{} J[Parallel to Serial]}
    I {-{-}{} J}
    J {-{-}{} K[Data Output]}
{Highlighting}
{Shaded}
\end{verbatim}
\end{center}

\textbf{Advantages:}

\begin{itemize}
\tightlist
\item
  \textbf{Bandwidth Efficient}: 2 bits per symbol
\item
  \textbf{Good Noise Performance}: Constant envelope
\item
  \textbf{Widely Used}: Standard in digital communication
\end{itemize}

\textbf{Applications:}

\begin{itemize}
\tightlist
\item
  Satellite communication
\item
  Digital TV broadcasting
\item
  Wireless communication systems
\end{itemize}

\end{solutionbox}
\begin{mnemonicbox}
``QPSK - Quadrature Phase, 2 bits, 4 phases''

\end{mnemonicbox}
\subsection*{Question 3(a) [3 marks]}\label{q3a}

\textbf{Draw the block diagram of FSK modulator}

\begin{solutionbox}

\begin{center}
\textbf{Mermaid Diagram (Code)}
\begin{verbatim}
{Shaded}
{Highlighting}[]
graph LR
    A[Digital Data] {-{-}{} B[Switch]}
    C[Oscillator 1{br/{}f1] {-}{-}{} B}
    D[Oscillator 2{br/{}f2] {-}{-}{} B}
    B {-{-}{} E[FSK Output]}
    
    F[Data = 1] {-.{-}{} C}
    G[Data = 0] {-.{-}{} D}
{Highlighting}
{Shaded}
\end{verbatim}
\end{center}

\textbf{Components:}

\begin{itemize}
\tightlist
\item
  \textbf{Digital Data Input}: Binary data stream (0s and 1s)
\item
  \textbf{Two Oscillators}: f_{1} for bit `1', f_{2} for bit `0'
\item
  \textbf{Electronic Switch}: Selects frequency based on input bit
\item
  \textbf{FSK Output}: Frequency modulated signal
\end{itemize}

\textbf{Operation:}

\begin{itemize}
\tightlist
\item
  \textbf{Bit `1'}: Switch connects oscillator 1 (higher frequency)
\item
  \textbf{Bit `0'}: Switch connects oscillator 2 (lower frequency)
\item
  \textbf{Output}: Continuous frequency shifting based on data
\end{itemize}

\end{solutionbox}
\begin{mnemonicbox}
``FSK - Frequency Switch based on data Keys''

\end{mnemonicbox}
\subsection*{Question 3(b) [4 marks]}\label{q3b}

\textbf{Draw and explain block diagram of PSK modulator.}

\begin{solutionbox}

\begin{center}
\textbf{Mermaid Diagram (Code)}
\begin{verbatim}
{Shaded}
{Highlighting}[]
graph LR
    A[Digital Data] {-{-}{} B[Balanced Modulator]}
    C["Carrier Oscillator{br/{}cos(ωt)"] {-}{-}{} B}
    B {-{-}{} D[PSK Output]}
    
    E[Data = 1] {-.{-}{} F[0^ phase]}
    G[Data = 0] {-.{-}{} H[180^ phase]}
{Highlighting}
{Shaded}
\end{verbatim}
\end{center}

\textbf{Components and Function:}

{\def\LTcaptype{none} % do not increment counter
\begin{longtable}[]{@{}ll@{}}
\toprule\noalign{}
Component & Function \\
\midrule\noalign{}
\endhead
\bottomrule\noalign{}
\endlastfoot
\textbf{Digital Data} & Binary input stream (0s and 1s) \\
\textbf{Carrier Oscillator} & Generates reference carrier signal \\
\textbf{Balanced Modulator} & Multiplies data with carrier \\
\textbf{PSK Output} & Phase modulated signal \\
\end{longtable}
}

\textbf{Operation:}

\begin{itemize}
\tightlist
\item
  \textbf{Data `1'}: Output = +cos(ωt) (0^\circ phase)
\item
  \textbf{Data `0'}: Output = -cos(ωt) (180^\circ phase)
\item
  \textbf{Phase Shift}: 180^\circ difference between `1' and `0'
\end{itemize}

\textbf{Mathematical Expression:}

\begin{itemize}
\tightlist
\item
  PSK Signal: s(t) = A·d(t)·cos(ωt)
\item
  Where d(t) = +1 for `1', -1 for `0'
\end{itemize}

\textbf{Advantages:}

\begin{itemize}
\tightlist
\item
  \textbf{Constant Envelope}: Better noise immunity
\item
  \textbf{Bandwidth Efficient}: Occupies same bandwidth as ASK
\item
  \textbf{Simple Detection}: Coherent detection required
\end{itemize}

\end{solutionbox}
\begin{mnemonicbox}
``PSK - Phase Shift using balanced modulator Key''

\end{mnemonicbox}
\subsection*{Question 3(c) [7 marks]}\label{q3c}

\textbf{Explain the block diagram of ASK modulator and de-modulator with
waveform.}

\begin{solutionbox}

\textbf{ASK Modulator:}

\begin{center}
\textbf{Mermaid Diagram (Code)}
\begin{verbatim}
{Shaded}
{Highlighting}[]
graph LR
    A[Digital Data] {-{-}{} B[Multiplier]}
    C["Carrier cos(ωt)"] {-{-}{} B}
    B {-{-}{} D[ASK Output]}
{Highlighting}
{Shaded}
\end{verbatim}
\end{center}

\textbf{ASK Demodulator:}

\begin{center}
\textbf{Mermaid Diagram (Code)}
\begin{verbatim}
{Shaded}
{Highlighting}[]
graph LR
    A[ASK Input] {-{-}{} B[Multiplier] }
    C[Local Carrier] {-{-}{} B}
    B {-{-}{} D[Low Pass Filter]}
    D {-{-}{} E[Decision Device]}
    E {-{-}{} F[Digital Output]}
    G[Threshold] {-{-}{} E}
{Highlighting}
{Shaded}
\end{verbatim}
\end{center}

\textbf{Waveforms:}

\begin{verbatim}
Digital Data:
    1    0    1    1    0
   \_\_\_       \_\_\_  \_\_\_      
  |   |     |   ||   |     
\_\_|   |\_\_\_\_\_|   ||   |\_\_\_\_\_

Carrier Signal:
                 
            
                  

ASK Output:
            
                
                
\end{verbatim}

\textbf{Modulation Process:}

{\def\LTcaptype{none} % do not increment counter
\begin{longtable}[]{@{}lll@{}}
\toprule\noalign{}
Data Bit & Carrier & ASK Output \\
\midrule\noalign{}
\endhead
\bottomrule\noalign{}
\endlastfoot
\textbf{`1'} & A·cos(ωt) & A·cos(ωt) \\
\textbf{`0'} & A·cos(ωt) & 0 \\
\end{longtable}
}

\textbf{Demodulation Process:}

\begin{enumerate}
\tightlist
\item
  \textbf{Multiplication}: ASK signal \times Local carrier
\item
  \textbf{Low Pass Filtering}: Remove high frequency components
\item
  \textbf{Decision}: Compare with threshold to recover data
\end{enumerate}

\textbf{Applications:}

\begin{itemize}
\tightlist
\item
  \textbf{Optical Communication}: LED/Laser on-off keying
\item
  \textbf{Simple Radio Systems}: AM radio modification
\item
  \textbf{Short Range Communication}: IR remote controls
\end{itemize}

\textbf{Advantages/Disadvantages:}

{\def\LTcaptype{none} % do not increment counter
\begin{longtable}[]{@{}ll@{}}
\toprule\noalign{}
Advantages & Disadvantages \\
\midrule\noalign{}
\endhead
\bottomrule\noalign{}
\endlastfoot
Simple implementation & Poor noise performance \\
Low cost & Bandwidth inefficient \\
Easy detection & Susceptible to fading \\
\end{longtable}
}

\end{solutionbox}
\begin{mnemonicbox}
``ASK - Amplitude Switch, multiply and filter Key''

\end{mnemonicbox}
\subsection*{Question 3(a OR) [3
marks]}\label{question-3a-or-3-marks}

\textbf{Write Principle and draw the constellation diagram of MSK.}

\begin{solutionbox}

\textbf{MSK Principle:} MSK (Minimum Shift Keying) is a form of
continuous-phase FSK where the frequency deviation is exactly half the
bit rate.

\textbf{Key Properties:}

\begin{itemize}
\tightlist
\item
  \textbf{Continuous Phase}: No phase discontinuities
\item
  \textbf{Minimum Frequency Separation}: Δf = Rb/2
\item
  \textbf{Constant Envelope}: Good for nonlinear amplifiers
\end{itemize}

\textbf{Constellation Diagram:}

\begin{verbatim}
        Q
        |
●  (I=0,

Q=1)

        |
   ●{-{-}{-}{-}+{-}{-}{-}{-}●  I}
        |
●  (I=0,

Q={-1)}

        |
        
Points rotate continuously
between 1 on I and Q axes
\end{verbatim}

\textbf{Mathematical Representation:}

\begin{itemize}
\tightlist
\item
  \textbf{Bit `1'}: f_{1} = fc + Rb/4
\item
  \textbf{Bit `0'}: f_{2} = fc - Rb/4
\item
  \textbf{Frequency Deviation}: Δf = Rb/2
\end{itemize}

\textbf{Characteristics:}

\begin{itemize}
\tightlist
\item
  \textbf{Spectral Efficiency}: Better than conventional FSK
\item
  \textbf{Continuous Phase}: Reduces out-of-band radiation
\item
  \textbf{Orthogonal Detection}: Can be detected as OQPSK
\end{itemize}

\end{solutionbox}
\begin{mnemonicbox}
``MSK - Minimum Shift, Continuous phase Key''

\end{mnemonicbox}
\subsection*{Question 3(b OR) [4
marks]}\label{question-3b-or-4-marks}

\textbf{Draw and explain the constellation diagram of 16-QAM}

\begin{solutionbox}

\textbf{16-QAM Constellation:}

\begin{verbatim}
           Q
           |
     ●  ●  |  ●  ●
           |
     ●  ● {-3{-}1 1 3 ● I}
           |
     ●  ●  |  ●  ●
           |
     ●  ●  |  ●  ●
           |
\end{verbatim}

\textbf{16-QAM Mapping Table:}

{\def\LTcaptype{none} % do not increment counter
\begin{longtable}[]{@{}lllll@{}}
\toprule\noalign{}
Bits & I & Q & Amplitude & Phase \\
\midrule\noalign{}
\endhead
\bottomrule\noalign{}
\endlastfoot
0000 & -3 & -3 & \sqrt18 & 225^\circ \\
0001 & -3 & -1 & \sqrt10 & 198.4^\circ \\
0010 & -3 & +1 & \sqrt10 & 161.6^\circ \\
0011 & -3 & +3 & \sqrt18 & 135^\circ \\
0100 & -1 & -3 & \sqrt10 & 251.6^\circ \\
0101 & -1 & -1 & \sqrt2 & 225^\circ \\
\ldots{} & \ldots{} & \ldots{} & \ldots{} & \ldots{} \\
\end{longtable}
}

\textbf{Key Features:}

\begin{itemize}
\tightlist
\item
  \textbf{16 Symbol Points}: 4 bits per symbol
\item
  \textbf{Gray Coding}: Adjacent symbols differ by 1 bit
\item
  \textbf{Variable Amplitude}: Different power levels
\item
  \textbf{High Data Rate}: 4 times QPSK data rate
\end{itemize}

\textbf{Signal Representation:} s(t) = I(t)·cos(ωt) - Q(t)·sin(ωt)

\textbf{Applications:}

\begin{itemize}
\tightlist
\item
  \textbf{Digital Cable TV}: High data rate transmission
\item
  \textbf{Microwave Links}: Point-to-point communication
\item
  \textbf{WiFi Systems}: 802.11 standards
\end{itemize}

\textbf{Advantages:}

\begin{itemize}
\tightlist
\item
  \textbf{High Spectral Efficiency}: 4 bits per symbol
\item
  \textbf{Good BER Performance}: With proper coding
\item
  \textbf{Flexible Implementation}: Software defined radio
\end{itemize}

\textbf{Trade-offs:}

\begin{itemize}
\tightlist
\item
  \textbf{Higher Complexity}: More complex than QPSK
\item
  \textbf{Power Variation}: Requires linear amplifiers
\item
  \textbf{Noise Sensitivity}: Higher than constant envelope schemes
\end{itemize}

\end{solutionbox}
\begin{mnemonicbox}
``16-QAM - 16 points, 4 bits, Quadrature Amplitude
Modulation''

\end{mnemonicbox}
\subsection*{Question 3(c OR) [7
marks]}\label{question-3c-or-7-marks}

\textbf{Compare Bits PER Symbol for digital modulation techniques-ASK,
FSK, PSK, QPSK,8-PSK, MSK and 16-QAM}

\begin{solutionbox}

\textbf{Bits per Symbol Comparison:}

{\def\LTcaptype{none} % do not increment counter
\begin{longtable}[]{@{}llll@{}}
\toprule\noalign{}
Modulation & Bits per Symbol & Symbol Rate & Data Rate Relationship \\
\midrule\noalign{}
\endhead
\bottomrule\noalign{}
\endlastfoot
\textbf{ASK} & 1 & Rs = Rb & Rb = Rs \times 1 \\
\textbf{FSK} & 1 & Rs = Rb & Rb = Rs \times 1 \\
\textbf{PSK (BPSK)} & 1 & Rs = Rb & Rb = Rs \times 1 \\
\textbf{QPSK} & 2 & Rs = Rb/2 & Rb = Rs \times 2 \\
\textbf{8-PSK} & 3 & Rs = Rb/3 & Rb = Rs \times 3 \\
\textbf{MSK} & 1 & Rs = Rb & Rb = Rs \times 1 \\
\textbf{16-QAM} & 4 & Rs = Rb/4 & Rb = Rs \times 4 \\
\end{longtable}
}

\textbf{Detailed Analysis:}

\begin{center}
\textbf{Mermaid Diagram (Code)}
\begin{verbatim}
{Shaded}
{Highlighting}[]
graph LR
    A[Digital Modulation] {-{-}{} B[M{-}ary Modulation]}
    B {-{-}{} C["Bits per Symbol = log_{2}(M)"]}
    C {-{-}{} D[Higher M = More bits per symbol]}
    D {-{-}{} E[Higher Data Rate]}
    E {-{-}{} F[But Higher Complexity]}
{Highlighting}
{Shaded}
\end{verbatim}
\end{center}

\textbf{Bandwidth Efficiency:}

{\def\LTcaptype{none} % do not increment counter
\begin{longtable}[]{@{}llll@{}}
\toprule\noalign{}
Modulation & M & Bits/Symbol & Bandwidth Efficiency \\
\midrule\noalign{}
\endhead
\bottomrule\noalign{}
\endlastfoot
ASK, FSK, PSK & 2 & 1 & 1 bit/s/Hz \\
QPSK & 4 & 2 & 2 bits/s/Hz \\
8-PSK & 8 & 3 & 3 bits/s/Hz \\
16-QAM & 16 & 4 & 4 bits/s/Hz \\
\end{longtable}
}

\textbf{Power Requirements:}

{\def\LTcaptype{none} % do not increment counter
\begin{longtable}[]{@{}lll@{}}
\toprule\noalign{}
Modulation & Relative Power & BER Performance \\
\midrule\noalign{}
\endhead
\bottomrule\noalign{}
\endlastfoot
\textbf{PSK} & Reference & Best \\
\textbf{ASK} & +3dB penalty & Poor \\
\textbf{FSK} & Same as PSK & Good \\
\textbf{QPSK} & Same as PSK & Same as PSK \\
\textbf{8-PSK} & +2.5dB penalty & Moderate \\
\textbf{16-QAM} & +4dB penalty & Good with coding \\
\end{longtable}
}

\textbf{Trade-offs:}

\begin{itemize}
\tightlist
\item
  \textbf{Higher M}: More bits per symbol but higher complexity
\item
  \textbf{Bandwidth vs Power}: Trade-off between spectral and power
  efficiency
\item
  \textbf{Implementation}: Higher order modulation needs better hardware
\end{itemize}

\textbf{Applications:}

\begin{itemize}
\tightlist
\item
  \textbf{Low Rate}: ASK, FSK, PSK for simple systems
\item
  \textbf{Medium Rate}: QPSK for balanced performance
\item
  \textbf{High Rate}: 8-PSK, 16-QAM for high-speed systems
\end{itemize}

\textbf{Formula:} Bits per Symbol = log_{2}(M), where M = number of symbols

\end{solutionbox}
\begin{mnemonicbox}
``More symbols, More bits, More complexity''

\end{mnemonicbox}
\subsection*{Question 4(a) [3 marks]}\label{q4a}

\textbf{Define probability and write it Significance of in
communication}

\begin{solutionbox}

\textbf{Definition of Probability:} Probability is the measure of
likelihood that an event will occur, expressed as a number between 0 and
1.

P(Event) = Number of favorable outcomes / Total number of possible
outcomes

\textbf{Significance in Communication:}

{\def\LTcaptype{none} % do not increment counter
\begin{longtable}[]{@{}ll@{}}
\toprule\noalign{}
Application & Significance \\
\midrule\noalign{}
\endhead
\bottomrule\noalign{}
\endlastfoot
\textbf{Error Analysis} & Calculate bit error rate (BER) \\
\textbf{Channel Modeling} & Noise and fading statistics \\
\textbf{Coding Theory} & Error correction probability \\
\textbf{Signal Detection} & Detection and false alarm rates \\
\end{longtable}
}

\textbf{Key Applications:}

\begin{itemize}
\tightlist
\item
  \textbf{BER Calculation}: P(error) = Q(\sqrt(2Eb/N0))
\item
  \textbf{Channel Capacity}: Shannon's theorem uses probability
\item
  \textbf{Information Theory}: Entropy based on probability
\item
  \textbf{System Design}: Performance prediction
\end{itemize}

\textbf{Mathematical Tools:}

\begin{itemize}
\tightlist
\item
  \textbf{Gaussian Distribution}: For noise analysis
\item
  \textbf{Rayleigh Distribution}: For fading channels
\item
  \textbf{Poisson Distribution}: For arrival processes
\end{itemize}

\end{solutionbox}
\begin{mnemonicbox}
``Probability Predicts Performance in communication
systems''

\end{mnemonicbox}
\subsection*{Question 4(b) [4 marks]}\label{q4b}

\textbf{Explain Huffman code with suitable example}

\begin{solutionbox}

\textbf{Huffman Coding Principle:} Variable length coding where
frequently occurring symbols get shorter codes.

\textbf{Algorithm:}

\begin{enumerate}
\tightlist
\item
  List symbols with probabilities
\item
  Combine two lowest probability symbols
\item
  Repeat until single symbol remains
\item
  Assign codes: left = 0, right = 1
\end{enumerate}

\textbf{Example:}

{\def\LTcaptype{none} % do not increment counter
\begin{longtable}[]{@{}lll@{}}
\toprule\noalign{}
Symbol & Probability & Huffman Code \\
\midrule\noalign{}
\endhead
\bottomrule\noalign{}
\endlastfoot
A & 0.4 & 0 \\
B & 0.3 & 10 \\
C & 0.2 & 110 \\
D & 0.1 & 111 \\
\end{longtable}
}

\textbf{Huffman Tree Construction:}

\begin{center}
\textbf{Mermaid Diagram (Code)}
\begin{verbatim}
{Shaded}
{Highlighting}[]
graph LR
    A1[1.0] {-{-}{} B1[A: 0.4]}
    A1 {-{-}{} C1[0.6]}
    C1 {-{-}{} D1[B: 0.3] }
    C1 {-{-}{} E1[0.3]}
    E1 {-{-}{} F1[C: 0.2]}
    E1 {-{-}{} G1[D: 0.1]}
{Highlighting}
{Shaded}
\end{verbatim}
\end{center}

\textbf{Code Assignment:}

\begin{itemize}
\tightlist
\item
  \textbf{A}: 0 (1 bit)
\item
  \textbf{B}: 10 (2 bits)\\
\item
  \textbf{C}: 110 (3 bits)
\item
  \textbf{D}: 111 (3 bits)
\end{itemize}

\textbf{Average Code Length:} L = 0.4\times1 + 0.3\times2 + 0.2\times3 + 0.1\times3 = 1.9
bits/symbol

\textbf{Advantages:}

\begin{itemize}
\tightlist
\item
  \textbf{Optimal}: Minimum average code length
\item
  \textbf{Prefix Property}: No code is prefix of another
\item
  \textbf{Efficient}: Reduces transmission bandwidth
\end{itemize}

\end{solutionbox}
\begin{mnemonicbox}
``Huffman - Frequent symbols get Shorter codes''

\end{mnemonicbox}
\subsection*{Question 4(c) [7 marks]}\label{q4c}

\textbf{Explain concept and key features of Internet of Things (IoT).}

\begin{solutionbox}

\textbf{IoT Concept:} Internet of Things is the network of physical
devices embedded with sensors, software, and connectivity to collect and
exchange data.

\textbf{IoT Architecture:}

\begin{center}
\textbf{Mermaid Diagram (Code)}
\begin{verbatim}
{Shaded}
{Highlighting}[]
graph LR
    A[Physical Devices] {-{-}{} B[Connectivity Layer]}
    B {-{-}{} C[Data Processing]}
    C {-{-}{} D[Application Layer]}
    D {-{-}{} E[Business Layer]}
    
    F[Sensors] {-{-}{} A}
    G[Actuators] {-{-}{} A}
    H[WiFi/Bluetooth] {-{-}{} B}
    I[Cellular/LoRa] {-{-}{} B}
    J[Cloud Computing] {-{-}{} C}
    K[Edge Computing] {-{-}{} C}
{Highlighting}
{Shaded}
\end{verbatim}
\end{center}

\textbf{Key Features:}

{\def\LTcaptype{none} % do not increment counter
\begin{longtable}[]{@{}
  >{\raggedright\arraybackslash}p{(\linewidth - 4\tabcolsep) * \real{0.2903}}
  >{\raggedright\arraybackslash}p{(\linewidth - 4\tabcolsep) * \real{0.4194}}
  >{\raggedright\arraybackslash}p{(\linewidth - 4\tabcolsep) * \real{0.2903}}@{}}
\toprule\noalign{}
\begin{minipage}[b]{\linewidth}\raggedright
Feature
\end{minipage} & \begin{minipage}[b]{\linewidth}\raggedright
Description
\end{minipage} & \begin{minipage}[b]{\linewidth}\raggedright
Example
\end{minipage} \\
\midrule\noalign{}
\endhead
\bottomrule\noalign{}
\endlastfoot
\textbf{Connectivity} & Devices connected to internet & WiFi, 4G, 5G \\
\textbf{Intelligence} & Smart decision making & AI algorithms \\
\textbf{Sensing} & Data collection from environment & Temperature,
humidity \\
\textbf{Actuation} & Control physical processes & Motors, valves \\
\textbf{Interoperability} & Devices work together & Standard
protocols \\
\end{longtable}
}

\textbf{IoT Protocol Stack:}

{\def\LTcaptype{none} % do not increment counter
\begin{longtable}[]{@{}lll@{}}
\toprule\noalign{}
Layer & Protocols & Function \\
\midrule\noalign{}
\endhead
\bottomrule\noalign{}
\endlastfoot
\textbf{Application} & HTTP, CoAP, MQTT & Data exchange \\
\textbf{Transport} & TCP, UDP & Reliable transmission \\
\textbf{Network} & IPv6, 6LoWPAN & Routing \\
\textbf{Physical} & WiFi, ZigBee, LoRa & Connectivity \\
\end{longtable}
}

\textbf{Applications:}

\begin{itemize}
\tightlist
\item
  \textbf{Smart Home}: Automated lighting, security
\item
  \textbf{Industrial IoT}: Manufacturing automation
\item
  \textbf{Healthcare}: Remote patient monitoring
\item
  \textbf{Smart Cities}: Traffic management, utilities
\end{itemize}

\textbf{Challenges:}

\begin{itemize}
\tightlist
\item
  \textbf{Security}: Device vulnerabilities, data privacy
\item
  \textbf{Scalability}: Billions of devices
\item
  \textbf{Interoperability}: Different standards
\item
  \textbf{Power Consumption}: Battery-operated devices
\end{itemize}

\textbf{Benefits:}

\begin{itemize}
\tightlist
\item
  \textbf{Automation}: Reduced human intervention
\item
  \textbf{Efficiency}: Optimized resource usage\\
\item
  \textbf{Real-time Monitoring}: Instant data access
\item
  \textbf{Cost Reduction}: Predictive maintenance
\end{itemize}

\textbf{Technologies:}

\begin{itemize}
\tightlist
\item
  \textbf{Communication}: WiFi, Bluetooth, Cellular, LoRa
\item
  \textbf{Processing}: Edge computing, cloud computing
\item
  \textbf{Analytics}: Big data, machine learning
\item
  \textbf{Security}: Encryption, authentication
\end{itemize}

\end{solutionbox}
\begin{mnemonicbox}
``IoT - Internet of Things, Smart Connected Devices
everywhere''

\end{mnemonicbox}
\subsection*{Question 4(a OR) [3
marks]}\label{question-4a-or-3-marks}

\textbf{Define error correction code and list common error correcting
code.}

\begin{solutionbox}

\textbf{Error Correction Code Definition:} Error correction codes are
techniques that add redundant bits to data to detect and correct
transmission errors automatically.

\textbf{Common Error Correcting Codes:}

{\def\LTcaptype{none} % do not increment counter
\begin{longtable}[]{@{}
  >{\raggedright\arraybackslash}p{(\linewidth - 4\tabcolsep) * \real{0.3056}}
  >{\raggedright\arraybackslash}p{(\linewidth - 4\tabcolsep) * \real{0.3611}}
  >{\raggedright\arraybackslash}p{(\linewidth - 4\tabcolsep) * \real{0.3333}}@{}}
\toprule\noalign{}
\begin{minipage}[b]{\linewidth}\raggedright
Code Type
\end{minipage} & \begin{minipage}[b]{\linewidth}\raggedright
Description
\end{minipage} & \begin{minipage}[b]{\linewidth}\raggedright
Capability
\end{minipage} \\
\midrule\noalign{}
\endhead
\bottomrule\noalign{}
\endlastfoot
\textbf{Hamming Code} & Single error correction & Correct 1-bit error \\
\textbf{Reed-Solomon} & Block code for burst errors & Correct multiple
errors \\
\textbf{BCH Code} & Binary cyclic code & Correct t errors \\
\textbf{Convolutional Code} & Continuous encoding & Good for noisy
channels \\
\textbf{Turbo Code} & Iterative decoding & Near Shannon limit \\
\textbf{LDPC Code} & Low density parity check & Excellent performance \\
\end{longtable}
}

\textbf{Applications:}

\begin{itemize}
\tightlist
\item
  \textbf{Memory Systems}: ECC RAM
\item
  \textbf{Storage Devices}: Hard drives, CDs
\item
  \textbf{Communication}: Satellite, cellular
\item
  \textbf{Broadcasting}: Digital TV, radio
\end{itemize}

\end{solutionbox}
\begin{mnemonicbox}
``Error Correction Codes - Hamming Reed BCH
Convolutional Turbo LDPC''

\end{mnemonicbox}
\subsection*{Question 4(b OR) [4
marks]}\label{question-4b-or-4-marks}

\textbf{Explain Shanon Fano code with suitable example}

\begin{solutionbox}

\textbf{Shannon-Fano Coding Algorithm:} Top-down approach that divides
symbols into two groups with approximately equal probabilities.

\textbf{Algorithm Steps:}

\begin{enumerate}
\tightlist
\item
  Arrange symbols in decreasing probability order
\item
  Divide into two groups with nearly equal total probability
\item
  Assign `0' to first group, `1' to second group
\item
  Repeat for each subgroup
\end{enumerate}

\textbf{Example:}

{\def\LTcaptype{none} % do not increment counter
\begin{longtable}[]{@{}lll@{}}
\toprule\noalign{}
Symbol & Probability & Shannon-Fano Code \\
\midrule\noalign{}
\endhead
\bottomrule\noalign{}
\endlastfoot
A & 0.4 & 00 \\
B & 0.3 & 01 \\
C & 0.2 & 10 \\
D & 0.1 & 11 \\
\end{longtable}
}

\textbf{Construction Tree:}

\begin{center}
\textbf{Mermaid Diagram (Code)}
\begin{verbatim}
{Shaded}
{Highlighting}[]
graph TD
    A1[A,B,C,D: 1.0] {-{-}{} B1[A,B: 0.7]}
    A1 {-{-}{} C1[C,D: 0.3]}
    B1 {-{-}{} D1[A: 0.4]}
    B1 {-{-}{} E1[B: 0.3]}
    C1 {-{-}{} F1[C: 0.2]}
    C1 {-{-}{} G1[D: 0.1]}
{Highlighting}
{Shaded}
\end{verbatim}
\end{center}

\textbf{Code Assignment:}

\begin{itemize}
\tightlist
\item
  Group 1 (A,B): Code starts with `0'
\item
  Group 2 (C,D): Code starts with `1'
\item
  A: 00, B: 01, C: 10, D: 11
\end{itemize}

\textbf{Comparison with Huffman:}

\begin{itemize}
\tightlist
\item
  \textbf{Shannon-Fano}: Top-down approach
\item
  \textbf{Huffman}: Bottom-up approach
\item
  \textbf{Huffman}: Always optimal
\item
  \textbf{Shannon-Fano}: May not be optimal
\end{itemize}

\textbf{Average Code Length:} L = 0.4\times2 + 0.3\times2 + 0.2\times2 + 0.1\times2 = 2.0
bits/symbol

\end{solutionbox}
\begin{mnemonicbox}
``Shannon-Fano - Split groups, assign codes
Top-down''

\end{mnemonicbox}
\subsection*{Question 4(c OR) [7
marks]}\label{question-4c-or-7-marks}

\textbf{Explain different standard formats of audio signal.}

\begin{solutionbox}

\textbf{Audio Signal Formats:}

{\def\LTcaptype{none} % do not increment counter
\begin{longtable}[]{@{}lllll@{}}
\toprule\noalign{}
Format & Full Form & Compression & Quality & File Size \\
\midrule\noalign{}
\endhead
\bottomrule\noalign{}
\endlastfoot
\textbf{WAV} & Waveform Audio File & Uncompressed & Highest & Largest \\
\textbf{MP3} & MPEG Layer 3 & Lossy & Good & Small \\
\textbf{AAC} & Advanced Audio Coding & Lossy & Better than MP3 &
Small \\
\textbf{FLAC} & Free Lossless Audio Codec & Lossless & Original &
Medium \\
\textbf{OGG} & Ogg Vorbis & Lossy & Good & Small \\
\end{longtable}
}

\textbf{Audio Parameters:}

\begin{center}
\textbf{Mermaid Diagram (Code)}
\begin{verbatim}
{Shaded}
{Highlighting}[]
graph TD
    A[Audio Signal] {-{-}{} B[Sampling Rate]}
    A {-{-}{} C[Bit Depth]}
    A {-{-}{} D[Channels]}
    A {-{-}{} E[Compression]}
    
    B {-{-}{} F[44.1 kHz CD Quality]}
    C {-{-}{} G[16{-}bit Standard]}
    D {-{-}{} H[Mono/Stereo]}
    E {-{-}{} I[Lossy/Lossless]}
{Highlighting}
{Shaded}
\end{verbatim}
\end{center}

\textbf{Sampling Standards:}

{\def\LTcaptype{none} % do not increment counter
\begin{longtable}[]{@{}llll@{}}
\toprule\noalign{}
Standard & Sampling Rate & Bit Depth & Application \\
\midrule\noalign{}
\endhead
\bottomrule\noalign{}
\endlastfoot
\textbf{CD Quality} & 44.1 kHz & 16-bit & Consumer audio \\
\textbf{Studio Quality} & 48 kHz & 24-bit & Professional recording \\
\textbf{High Resolution} & 96 kHz & 24-bit & Audiophile \\
\textbf{Telephone} & 8 kHz & 8-bit & Voice communication \\
\end{longtable}
}

\textbf{Compression Types:}

\begin{itemize}
\tightlist
\item
  \textbf{Lossless}: Original quality preserved (FLAC, ALAC)
\item
  \textbf{Lossy}: Some quality lost for smaller size (MP3, AAC)
\item
  \textbf{Uncompressed}: No compression (WAV, AIFF)
\end{itemize}

\textbf{Applications:}

\begin{itemize}
\tightlist
\item
  \textbf{Broadcasting}: AAC for digital radio
\item
  \textbf{Streaming}: MP3, AAC for internet
\item
  \textbf{Professional}: WAV, FLAC for studios
\item
  \textbf{Mobile}: AAC for smartphones
\end{itemize}

\textbf{File Size Comparison (3-minute song):}

\begin{itemize}
\tightlist
\item
  \textbf{WAV}: 30 MB
\item
  \textbf{FLAC}: 20 MB
\item
  \textbf{MP3}: 3 MB
\item
  \textbf{AAC}: 2.5 MB
\end{itemize}

\textbf{Quality vs Size Trade-off:}

\begin{itemize}
\tightlist
\item
  \textbf{Highest Quality}: WAV (uncompressed)
\item
  \textbf{Best Balance}: AAC (lossy compressed)
\item
  \textbf{Smallest Size}: Low bitrate MP3
\item
  \textbf{Lossless Compressed}: FLAC
\end{itemize}

\end{solutionbox}
\begin{mnemonicbox}
``WAV MP3 AAC FLAC - Wave, Layer3, Advanced, Free
Lossless''

\end{mnemonicbox}
\subsection*{Question 5(a) [3 marks]}\label{q5a}

\textbf{Explain E1 carrier multiplexing hierarchy.}

\begin{solutionbox}

\textbf{E1 Carrier System:} European digital transmission standard for
multiplexing voice channels.

\textbf{E1 Hierarchy:}

{\def\LTcaptype{none} % do not increment counter
\begin{longtable}[]{@{}lllll@{}}
\toprule\noalign{}
Level & Name & Bit Rate & Voice Channels & Multiplexing \\
\midrule\noalign{}
\endhead
\bottomrule\noalign{}
\endlastfoot
\textbf{E0} & Basic Channel & 64 kbps & 1 & - \\
\textbf{E1} & Primary Rate & 2.048 Mbps & 30 & 30 \times E0 + 2 \\
\textbf{E2} & Secondary Rate & 8.448 Mbps & 120 & 4 \times E1 \\
\textbf{E3} & Tertiary Rate & 34.368 Mbps & 480 & 4 \times E2 \\
\textbf{E4} & Quaternary Rate & 139.264 Mbps & 1920 & 4 \times E3 \\
\end{longtable}
}

\textbf{E1 Frame Structure:}

\begin{verbatim}
Frame (125 μs, 256 bits)
|TS0|TS1|TS2|...|TS15|TS16|TS17|...|TS31|
 32 time slots  8 bits = 256 bits

TS0: Synchronization + Alarm
TS16: Signaling (voice channels)
TS1{-15, 17{-}31: 30 voice channels}
\end{verbatim}

\textbf{Multiplexing Process:}

\begin{itemize}
\tightlist
\item
  \textbf{Level 1}: 30 voice channels + 2 control \rightarrow E1
\item
  \textbf{Level 2}: 4 E1 streams \rightarrow E2
\item
  \textbf{Level 3}: 4 E2 streams \rightarrow E3
\item
  \textbf{Level 4}: 4 E3 streams \rightarrow E4
\end{itemize}

\textbf{Applications:}

\begin{itemize}
\tightlist
\item
  \textbf{ISDN}: Primary rate interface
\item
  \textbf{Cellular}: Base station connectivity
\item
  \textbf{Enterprise}: Private branch exchange (PBX)
\item
  \textbf{Internet}: Digital subscriber line (DSL)
\end{itemize}

\end{solutionbox}
\begin{mnemonicbox}
``E1 - 30 voices, 2.048 Mbps, European standard''

\end{mnemonicbox}
\subsection*{Question 5(b) [4 marks]}\label{q5b}

\textbf{Compare FDMA with TDMA.}

\begin{solutionbox}

\textbf{FDMA vs TDMA Comparison:}

{\def\LTcaptype{none} % do not increment counter
\begin{longtable}[]{@{}
  >{\raggedright\arraybackslash}p{(\linewidth - 4\tabcolsep) * \real{0.4783}}
  >{\raggedright\arraybackslash}p{(\linewidth - 4\tabcolsep) * \real{0.2609}}
  >{\raggedright\arraybackslash}p{(\linewidth - 4\tabcolsep) * \real{0.2609}}@{}}
\toprule\noalign{}
\begin{minipage}[b]{\linewidth}\raggedright
Parameter
\end{minipage} & \begin{minipage}[b]{\linewidth}\raggedright
FDMA
\end{minipage} & \begin{minipage}[b]{\linewidth}\raggedright
TDMA
\end{minipage} \\
\midrule\noalign{}
\endhead
\bottomrule\noalign{}
\endlastfoot
\textbf{Full Form} & Frequency Division Multiple Access & Time Division
Multiple Access \\
\textbf{Domain} & Frequency & Time \\
\textbf{Channel Allocation} & Each user gets different frequency & Each
user gets different time slot \\
\textbf{Bandwidth per User} & Narrow bandwidth continuously & Full
bandwidth for short duration \\
\textbf{Guard Bands} & Required between frequencies & Not required \\
\textbf{Synchronization} & Not critical & Critical \\
\textbf{Flexibility} & Less flexible & More flexible \\
\textbf{Handoff} & Simple & Complex \\
\textbf{Near-Far Effect} & Less problematic & More problematic \\
\end{longtable}
}

\textbf{FDMA System:}

\begin{center}
\textbf{Mermaid Diagram (Code)}
\begin{verbatim}
{Shaded}
{Highlighting}[]
graph TD
    A[Total Bandwidth] {-{-}{} B[User 1: f1]}
    A {-{-}{} C[User 2: f2]}
    A {-{-}{} D[User 3: f3]}
    A {-{-}{} E[User N: fn]}
    
    F[Guard Band] {-{-}{} B}
    F {-{-}{} C}
    F {-{-}{} D}
{Highlighting}
{Shaded}
\end{verbatim}
\end{center}

\textbf{TDMA System:}

\begin{verbatim}
gantt
    title TDMA Time Slots
    dateFormat X
    axisFormat \%s
    
    section Frame
    User 1 :done, u1, 0, 1
    User 2 :done, u2, 1, 2
    User 3 :done, u3, 2, 3
    User 4 :done, u4, 3, 4
\end{verbatim}

\textbf{Advantages/Disadvantages:}

{\def\LTcaptype{none} % do not increment counter
\begin{longtable}[]{@{}ll@{}}
\toprule\noalign{}
FDMA Advantages & FDMA Disadvantages \\
\midrule\noalign{}
\endhead
\bottomrule\noalign{}
\endlastfoot
Simple implementation & Waste of bandwidth due to guard bands \\
No synchronization needed & Less flexible \\
Continuous transmission & Difficult to accommodate varying rates \\
\end{longtable}
}

{\def\LTcaptype{none} % do not increment counter
\begin{longtable}[]{@{}ll@{}}
\toprule\noalign{}
TDMA Advantages & TDMA Disadvantages \\
\midrule\noalign{}
\endhead
\bottomrule\noalign{}
\endlastfoot
Efficient bandwidth usage & Complex synchronization \\
Flexible data rates & Battery life issues (burst transmission) \\
Easy to add/remove users & Near-far problem \\
\end{longtable}
}

\textbf{Applications:}

\begin{itemize}
\tightlist
\item
  \textbf{FDMA}: AMPS (1G), satellite communication
\item
  \textbf{TDMA}: GSM (2G), satellite communication
\end{itemize}

\end{solutionbox}
\begin{mnemonicbox}
``FDMA Frequency, TDMA Time - different domains for
multiple access''

\end{mnemonicbox}
\subsection*{Question 5(c) [7 marks]}\label{q5c}

\textbf{Explain CDMA technique in detail.}

\begin{solutionbox}

\textbf{CDMA Principle:} Code Division Multiple Access allows multiple
users to share the same frequency and time by using unique spreading
codes.

\textbf{CDMA System Architecture:}

\begin{center}
\textbf{Mermaid Diagram (Code)}
\begin{verbatim}
{Shaded}
{Highlighting}[]
graph LR
    A[User Data] {-{-}{} B[Spreading Code]}
    B {-{-}{} C[Modulator]}
    C {-{-}{} D[Channel]}
    D {-{-}{} E[Correlator]}
    E {-{-}{} F[Despreading]}
    F {-{-}{} G[Data Recovery]}
    
    H[Pseudo{-random Code] {-}{-}{} B}
    I[Same PN Code] {-{-}{} F}
{Highlighting}
{Shaded}
\end{verbatim}
\end{center}

\textbf{Spreading Process:}

{\def\LTcaptype{none} % do not increment counter
\begin{longtable}[]{@{}lll@{}}
\toprule\noalign{}
Parameter & Before Spreading & After Spreading \\
\midrule\noalign{}
\endhead
\bottomrule\noalign{}
\endlastfoot
\textbf{Data Rate} & Rb & Rb \\
\textbf{Chip Rate} & - & Rc (= N \times Rb) \\
\textbf{Bandwidth} & Rb & Rc \\
\textbf{Processing Gain} & 1 & N = Rc/Rb \\
\end{longtable}
}

\textbf{CDMA Properties:}

\begin{verbatim}
Original Data:    1  0  1
PN Code:         101 010 101
XOR Result:      101 010 101
(Spread Signal)

At Receiver:
Received:        101 010 101
Same PN Code:    101 010 101  
XOR Result:       1   0   1
(Original Data)
\end{verbatim}

\textbf{Key Features:}

{\def\LTcaptype{none} % do not increment counter
\begin{longtable}[]{@{}lll@{}}
\toprule\noalign{}
Feature & Description & Benefit \\
\midrule\noalign{}
\endhead
\bottomrule\noalign{}
\endlastfoot
\textbf{Spreading} & Data XOR with PN code & Bandwidth expansion \\
\textbf{Processing Gain} & Rc/Rb ratio & Interference rejection \\
\textbf{Soft Handoff} & Simultaneous connections & Better quality \\
\textbf{Power Control} & Dynamic power adjustment & Near-far solution \\
\end{longtable}
}

\textbf{CDMA Advantages:}

\begin{itemize}
\tightlist
\item
  \textbf{Capacity}: Higher user capacity than FDMA/TDMA
\item
  \textbf{Security}: Encrypted by spreading code
\item
  \textbf{Soft Handoff}: No call dropping during handoff
\item
  \textbf{Anti-jamming}: Spread spectrum immunity
\item
  \textbf{No Frequency Planning}: Same frequency reuse
\end{itemize}

\textbf{CDMA Disadvantages:}

\begin{itemize}
\tightlist
\item
  \textbf{Near-Far Problem}: Requires power control
\item
  \textbf{Complexity}: More complex than FDMA/TDMA
\item
  \textbf{Self Interference}: Users interfere with each other
\item
  \textbf{Breathing Effect}: Coverage varies with loading
\end{itemize}

\textbf{Mathematical Analysis:}

\begin{itemize}
\tightlist
\item
  \textbf{Processing Gain}: G = Rc/Rb = 10log_{1}_{0}(Rc/Rb) dB
\item
  \textbf{Capacity}: M \approx 1 + G/(Eb/I_{0})
\item
  \textbf{BER}: Depends on number of active users
\end{itemize}

\textbf{Power Control:}

\begin{itemize}
\tightlist
\item
  \textbf{Open Loop}: Based on received signal strength
\item
  \textbf{Closed Loop}: Base station commands mobile
\item
  \textbf{Requirement}: \pm1 dB accuracy needed
\end{itemize}

\textbf{Applications:}

\begin{itemize}
\tightlist
\item
  \textbf{IS-95 (cdmaOne)}: 2G CDMA standard
\item
  \textbf{WCDMA}: 3G UMTS system
\item
  \textbf{GPS}: Satellite navigation
\item
  \textbf{WiFi}: Spread spectrum option
\end{itemize}

\textbf{PN Code Properties:}

\begin{itemize}
\tightlist
\item
  \textbf{Autocorrelation}: High for synchronized, low for
  unsynchronized
\item
  \textbf{Cross-correlation}: Low between different codes
\item
  \textbf{Balance}: Equal number of 1s and 0s
\item
  \textbf{Run Length}: Distribution of consecutive bits
\end{itemize}

\end{solutionbox}
\begin{mnemonicbox}
``CDMA - Code Division, same frequency/time, unique
codes for Multiple Access''

\end{mnemonicbox}
\subsection*{Question 5(a OR) [3
marks]}\label{question-5a-or-3-marks}

\textbf{Draw block diagram of Time Division Multiplexing technique
(TDM).}

\begin{solutionbox}

\textbf{TDM Block Diagram:}

\begin{center}
\textbf{Mermaid Diagram (Code)}
\begin{verbatim}
{Shaded}
{Highlighting}[]
graph LR
    A[Input 1] {-{-}{} E[Multiplexer]}
    B[Input 2] {-{-}{} E}
    C[Input 3] {-{-}{} E}
    D[Input N] {-{-}{} E}
    
    E {-{-}{} F[TDM Signal]}
    F {-{-}{} G[Channel]}
    G {-{-}{} H[Demultiplexer]}
    
    H {-{-}{} I[Output 1]}
    H {-{-}{} J[Output 2] }
    H {-{-}{} K[Output 3]}
    H {-{-}{} L[Output N]}
    
    M[Clock/Sync] {-{-}{} E}
    N[Clock/Sync] {-{-}{} H}
{Highlighting}
{Shaded}
\end{verbatim}
\end{center}

\textbf{TDM Frame Structure:}

\begin{verbatim}
|{{-}{-}{-}{-} Frame Period T {-}{-}{-}{-}|}
|Ch1|Ch2|Ch3|...|ChN|Sync|
 TS1 TS2 TS3     TSN
 
Each time slot = T/N
Frame Rate = 1/T
\end{verbatim}

\textbf{Components:}

\begin{itemize}
\tightlist
\item
  \textbf{Multiplexer}: Samples inputs sequentially
\item
  \textbf{Clock/Synchronization}: Controls switching timing
\item
  \textbf{Channel}: Transmission medium
\item
  \textbf{Demultiplexer}: Separates multiplexed signal
\end{itemize}

\textbf{Operation:}

\begin{itemize}
\tightlist
\item
  Each input channel gets dedicated time slot
\item
  Sampling rate must satisfy Nyquist criterion
\item
  Frame synchronization required at receiver
\end{itemize}

\end{solutionbox}
\begin{mnemonicbox}
``TDM - Time Division, sequential sampling,
Multiplexing''

\end{mnemonicbox}
\subsection*{Question 5(b OR) [4
marks]}\label{question-5b-or-4-marks}

\textbf{Write a short not on classification of multiplexing techniques.}

\begin{solutionbox}

\textbf{Classification of Multiplexing Techniques:}

\begin{center}
\textbf{Mermaid Diagram (Code)}
\begin{verbatim}
{Shaded}
{Highlighting}[]
graph TD
    A[Multiplexing] {-{-}{} B[Analog Multiplexing]}
    A {-{-}{} C[Digital Multiplexing]}
    
    B {-{-}{} D[FDM {-} Frequency Division]}
    B {-{-}{} E[WDM {-} Wavelength Division]}
    
    C {-{-}{} F[TDM {-} Time Division]}
    C {-{-}{} G[CDM {-} Code Division]}
    C {-{-}{} H[SDM {-} Space Division]}
    
    F {-{-}{} I[Synchronous TDM]}
    F {-{-}{} J[Asynchronous TDM]}
{Highlighting}
{Shaded}
\end{verbatim}
\end{center}

\textbf{Detailed Classification:}

{\def\LTcaptype{none} % do not increment counter
\begin{longtable}[]{@{}llll@{}}
\toprule\noalign{}
Type & Method & Domain & Application \\
\midrule\noalign{}
\endhead
\bottomrule\noalign{}
\endlastfoot
\textbf{FDM} & Frequency separation & Frequency & Radio, TV
broadcasting \\
\textbf{TDM} & Time slot allocation & Time & Digital telephony \\
\textbf{CDM} & Code separation & Code & CDMA cellular \\
\textbf{WDM} & Wavelength separation & Wavelength & Optical fiber \\
\textbf{SDM} & Space separation & Space & MIMO systems \\
\end{longtable}
}

\textbf{Synchronous vs Asynchronous TDM:}

{\def\LTcaptype{none} % do not increment counter
\begin{longtable}[]{@{}lll@{}}
\toprule\noalign{}
Parameter & Synchronous TDM & Asynchronous TDM \\
\midrule\noalign{}
\endhead
\bottomrule\noalign{}
\endlastfoot
\textbf{Time Slots} & Fixed allocation & Dynamic allocation \\
\textbf{Efficiency} & Lower & Higher \\
\textbf{Complexity} & Simple & Complex \\
\textbf{Bandwidth Waste} & May occur & Minimal \\
\end{longtable}
}

\textbf{Selection Criteria:}

\begin{itemize}
\tightlist
\item
  \textbf{Nature of Signal}: Analog \rightarrow FDM, Digital \rightarrow TDM
\item
  \textbf{Bandwidth}: Limited \rightarrow TDM, Abundant \rightarrow FDM
\item
  \textbf{Synchronization}: Critical \rightarrow Synchronous, Flexible \rightarrow
  Asynchronous
\item
  \textbf{Application}: Voice \rightarrow TDM, Data \rightarrow Statistical TDM
\end{itemize}

\textbf{Modern Techniques:}

\begin{itemize}
\tightlist
\item
  \textbf{OFDM}: Orthogonal Frequency Division Multiplexing
\item
  \textbf{MIMO}: Multiple Input Multiple Output
\item
  \textbf{Carrier Aggregation}: Multiple frequency bands
\end{itemize}

\end{solutionbox}
\begin{mnemonicbox}
``FDM TDM CDM WDM SDM - Frequency Time Code Wave
Space Division Multiplexing''

\end{mnemonicbox}
\subsection*{Question 5(c OR) [7
marks]}\label{question-5c-or-7-marks}

\textbf{Describe the procedure to troubleshoot the code division
multiplexing circuit}

\begin{solutionbox}

\textbf{CDMA Troubleshooting Procedure:}

\textbf{1. System Overview Check:}

\begin{center}
\textbf{Mermaid Diagram (Code)}
\begin{verbatim}
{Shaded}
{Highlighting}[]
graph TD
    A[CDMA System] {-{-}{} B[Transmitter Section]}
    A {-{-}{} C[Channel Section]}
    A {-{-}{} D[Receiver Section]}
    
    B {-{-}{} E[Data Input OK?]}
    B {-{-}{} F[PN Code Generation OK?]}
    B {-{-}{} G[Spreading Function OK?]}
    
    C {-{-}{} H[Path Loss Measurement]}
    C {-{-}{} I[Interference Check]}
    
    D {-{-}{} J[Correlation OK?]}
    D {-{-}{} K[Despreading OK?]}
    D {-{-}{} L[Data Recovery OK?]}
{Highlighting}
{Shaded}
\end{verbatim}
\end{center}

\textbf{2. Step-by-Step Troubleshooting:}

{\def\LTcaptype{none} % do not increment counter
\begin{longtable}[]{@{}
  >{\raggedright\arraybackslash}p{(\linewidth - 6\tabcolsep) * \real{0.1277}}
  >{\raggedright\arraybackslash}p{(\linewidth - 6\tabcolsep) * \real{0.2340}}
  >{\raggedright\arraybackslash}p{(\linewidth - 6\tabcolsep) * \real{0.2766}}
  >{\raggedright\arraybackslash}p{(\linewidth - 6\tabcolsep) * \real{0.3617}}@{}}
\toprule\noalign{}
\begin{minipage}[b]{\linewidth}\raggedright
Step
\end{minipage} & \begin{minipage}[b]{\linewidth}\raggedright
Parameter
\end{minipage} & \begin{minipage}[b]{\linewidth}\raggedright
Test Method
\end{minipage} & \begin{minipage}[b]{\linewidth}\raggedright
Expected Result
\end{minipage} \\
\midrule\noalign{}
\endhead
\bottomrule\noalign{}
\endlastfoot
\textbf{1} & Input Data & Verify data stream & Clean digital signal \\
\textbf{2} & PN Code & Check code generation & Proper sequence \\
\textbf{3} & Spreading & Monitor XOR output & Spread spectrum signal \\
\textbf{4} & Transmission** & Measure power level & Adequate signal
strength \\
\textbf{5} & Reception & Check received signal & Above noise floor \\
\textbf{6} & Correlation & Verify correlator output & Peak at correct
timing \\
\textbf{7} & Despreading & Check XOR with local PN & Despread signal \\
\textbf{8} & Data Recovery** & Verify output data & Original data
recovered \\
\end{longtable}
}

\textbf{3. Common Problems and Solutions:}

{\def\LTcaptype{none} % do not increment counter
\begin{longtable}[]{@{}
  >{\raggedright\arraybackslash}p{(\linewidth - 6\tabcolsep) * \real{0.1915}}
  >{\raggedright\arraybackslash}p{(\linewidth - 6\tabcolsep) * \real{0.2128}}
  >{\raggedright\arraybackslash}p{(\linewidth - 6\tabcolsep) * \real{0.3617}}
  >{\raggedright\arraybackslash}p{(\linewidth - 6\tabcolsep) * \real{0.2340}}@{}}
\toprule\noalign{}
\begin{minipage}[b]{\linewidth}\raggedright
Problem
\end{minipage} & \begin{minipage}[b]{\linewidth}\raggedright
Symptoms
\end{minipage} & \begin{minipage}[b]{\linewidth}\raggedright
Possible Causes
\end{minipage} & \begin{minipage}[b]{\linewidth}\raggedright
Solutions
\end{minipage} \\
\midrule\noalign{}
\endhead
\bottomrule\noalign{}
\endlastfoot
\textbf{No Signal} & Zero output & Power supply failure & Check power
connections \\
\textbf{High BER} & Many bit errors & Poor correlation & Adjust
timing/power \\
\textbf{Interference} & Degraded performance & Other users/noise & Power
control adjustment \\
\textbf{Sync Loss} & Intermittent signal & PN code mismatch & Verify
code sequences \\
\end{longtable}
}

\textbf{4. Test Equipment Required:}

{\def\LTcaptype{none} % do not increment counter
\begin{longtable}[]{@{}lll@{}}
\toprule\noalign{}
Equipment & Purpose & Measurement \\
\midrule\noalign{}
\endhead
\bottomrule\noalign{}
\endlastfoot
\textbf{Spectrum Analyzer} & Signal analysis & Power spectral density \\
\textbf{BER Tester} & Error measurement & Bit error rate \\
\textbf{Power Meter} & Power measurement & Transmitted/received power \\
\textbf{Oscilloscope} & Waveform analysis & Time domain signals \\
\textbf{Vector Analyzer} & Modulation quality & EVM, constellation \\
\end{longtable}
}

\textbf{5. Measurement Procedures:}

\textbf{Processing Gain Verification:}

\begin{verbatim}
Gp = 10 log_{1}_{0}(Rc/Rb) dB
Where: Rc = chip rate, Rb = bit rate
\end{verbatim}

\textbf{BER vs Eb/N0 Measurement:}

\begin{verbatim}
BER = Q(\sqrt(2Eb/N0))
Measure at various power levels
\end{verbatim}

\textbf{Near-Far Effect Check:}

\begin{itemize}
\tightlist
\item
  Measure power levels of different users
\item
  Verify power control operation
\item
  Check dynamic range requirements
\end{itemize}

\textbf{6. Performance Optimization:}

{\def\LTcaptype{none} % do not increment counter
\begin{longtable}[]{@{}lll@{}}
\toprule\noalign{}
Parameter & Optimization Method & Target Value \\
\midrule\noalign{}
\endhead
\bottomrule\noalign{}
\endlastfoot
\textbf{Power Control} & Adjust loop gain & \pm1 dB accuracy \\
\textbf{Code Selection} & Choose orthogonal codes & Low
cross-correlation \\
\textbf{Timing} & Synchronize PN generators & \textless0.5 chip
accuracy \\
\textbf{Filtering} & Bandlimit signals & Minimize ISI \\
\end{longtable}
}

\textbf{7. Documentation:}

\begin{itemize}
\tightlist
\item
  Record all measurements
\item
  Document problem symptoms
\item
  Note solutions applied
\item
  Create troubleshooting log
\end{itemize}

\textbf{Systematic Approach:}

\begin{enumerate}
\tightlist
\item
  \textbf{Isolate}: Identify faulty section
\item
  \textbf{Measure}: Use appropriate test equipment
\item
  \textbf{Analyze}: Compare with specifications
\item
  \textbf{Correct}: Apply appropriate solution
\item
  \textbf{Verify}: Confirm problem resolution
\end{enumerate}

\textbf{Safety Considerations:}

\begin{itemize}
\tightlist
\item
  Power levels within safe limits
\item
  Proper grounding procedures
\item
  RF exposure guidelines
\item
  Equipment calibration status
\end{itemize}

\end{solutionbox}
\begin{mnemonicbox}
``CDMA Troubleshoot - Check Data, PN code, Spreading,
Channel, Correlation, Recovery''

\end{mnemonicbox}

\end{document}
