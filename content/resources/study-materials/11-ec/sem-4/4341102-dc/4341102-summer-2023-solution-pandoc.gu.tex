\documentclass[10pt,a4paper]{article}

% content/resources/templates/preamble.tex
\usepackage[margin=0.6in]{geometry}
\author{Milav Dabgar}
\usepackage{amsmath,amssymb,amsthm}
\usepackage{booktabs}
\usepackage{multirow}
\usepackage{xcolor}
\usepackage{tcolorbox}
\tcbuselibrary{breakable,skins}
\usepackage[colorlinks=true,linkcolor=blue]{hyperref}
\usepackage{titlesec}
\usepackage{enumitem}
\usepackage{tikz}
\usepackage{pgfplots}
\usepackage{circuitikz}
\usepackage[version=4]{mhchem}
\usepackage{longtable}
\usepackage{array}
\usepackage{float}
\usepackage{caption}
\usepackage{listings}

\lstset{
  basicstyle=\small\ttfamily,
  breaklines=true,
  breakatwhitespace=false,
  postbreak=\mbox{\textcolor{red}{$\hookrightarrow$}\space},
  float=false,
  numbers=left,
  numberstyle=\tiny\color{gray},
  numbersep=10pt,
  xleftmargin=2em,
  keywordstyle=\color{blue},
  commentstyle=\color{green!60!black},
  stringstyle=\color{purple},
  backgroundcolor=\color{gray!5},
  showstringspaces=false,
  tabsize=2,
  captionpos=b,
  keepspaces=true,
  columns=flexible
}

\pgfplotsset{compat=1.18}
\usetikzlibrary{shapes,arrows,positioning,calc,patterns,decorations.pathmorphing,decorations.markings,arrows.meta}

% Color scheme
\definecolor{headcolor}{RGB}{0,102,204}
\definecolor{keycolor}{RGB}{220,20,60}
\definecolor{solutioncolor}{RGB}{34,139,34}
\definecolor{mnemoniccolor}{RGB}{148,0,211}
\definecolor{codecolor}{RGB}{0,0,100}

% Spacing
\setlength{\parskip}{3pt}
\setlist[itemize]{nosep}
\setlist[enumerate]{nosep}

% Title formatting
\titleformat{\section}{\Large\bfseries\color{headcolor}}{\thesection}{1em}{}
\titleformat{\subsection}{\large\bfseries\color{headcolor}}{\thesubsection}{1em}{}

% Pandoc tightlist compatibility
\providecommand{\tightlist}{%
  \setlength{\itemsep}{0pt}\setlength{\parskip}{0pt}}

% Pandoc longtable compatibility
\newcounter{none}
\def\thenone{}


% content/resources/templates/gujarati-boxes.tex
\usepackage{fontspec}
\usepackage{polyglossia}

% Set Gujarati as main language (document is primarily in Gujarati)
% Note: gloss-gujarati.ldf doesn't exist in polyglossia, but it will use hyphenation patterns
\setdefaultlanguage{gujarati}
\setotherlanguage{english}

% Configure Gujarati font properly
% Use Language=Default to prevent polyglossia from trying to add language-specific features
% that don't exist for Gujarati, which causes "empty feature" warnings
\newfontfamily\gujaratifont[Script=Gujarati,AutoFakeBold=2.5,AutoFakeSlant=0.3]{Noto Sans Gujarati}
\setmainfont[Script=Gujarati,AutoFakeBold=2.5,AutoFakeSlant=0.3]{Noto Sans Gujarati}
% Use Noto Sans Gujarati for monospace to support Gujarati in text
\setmonofont[Scale=0.9]{Noto Sans Gujarati}

% Configure English to use the same font
\newfontfamily\englishfont[Script=Gujarati,AutoFakeBold=2.5,AutoFakeSlant=0.3]{Noto Sans Gujarati}

% Translations for polyglossia
\gappto\captionsgujarati{
  \renewcommand{\tablename}{કોષ્ટક}
  \renewcommand{\figurename}{આકૃતિ}
}

% Helper for TikZ nodes to ensure Gujarati font
\newcommand{\gu}[1]{{\gujaratifont #1}}

% Custom environments
\newtcolorbox{solutionbox}{
    breakable,
    enhanced,
    colback=solutioncolor!5!white,
    colframe=solutioncolor!75!black,
    fonttitle=\bfseries,
    title=જવાબ
}

\newtcolorbox{solutionboxnobreak}{
 colback=solutioncolor!5!white,
 colframe=solutioncolor!75!black,
 fonttitle=\bfseries,
 title=જવાબ
}

\newtcolorbox{keyformula}{
 breakable,
 enhanced,
 colback=keycolor!5!white,
 colframe=keycolor!75!black,
 fonttitle=\bfseries,
 title=રાસાયણિક સમીકરણ/સૂત્ર
}

\newtcolorbox{mnemonicbox}{
 breakable,
 enhanced,
 colback=mnemoniccolor!5!white,
 colframe=mnemoniccolor!75!black,
 fonttitle=\bfseries,
 title=મેમરી ટ્રીક
}


\begin{document}

\begin{center}
{\Huge\bfseries\color{headcolor} Subject Name (Gujarati)}\\[5pt]
{\LARGE 4341102 -- Summer 2023}\\[3pt]
{\large Semester 1 Study Material}\\[3pt]
{\normalsize\textit{Detailed Solutions and Explanations}}
\end{center}

\vspace{10pt}

\subsection*{પ્રશ્ન 1(અ) [3
ગુણ]}\label{uxaaauxab0uxab6uxaa8-1uxa85-3-uxa97uxaa3}

\textbf{સિગ્નલને વ્યાખ્યાયિત કરો અને તેનું વર્ગીકરણ આપો.}

\begin{solutionbox}
સિગ્નલ એ એક ભૌતિક માત્રા છે જે સમય, સ્થળ અથવા અન્ય સ્વતંત્ર ચલ સાથે
બદલાય છે અને તેમાં માહિતી સમાયેલી હોય છે.

\textbf{સિગ્નલનું વર્ગીકરણ:}

{\def\LTcaptype{none} % do not increment counter
\begin{longtable}[]{@{}ll@{}}
\toprule\noalign{}
વર્ગીકરણ માપદંડ & સિગ્નલના પ્રકાર \\
\midrule\noalign{}
\endhead
\bottomrule\noalign{}
\endlastfoot
\textbf{સમય ડોમેન} & કંટીન્યુઅસ-ટાઈમ સિગ્નલ, ડિસ્ક્રીટ-ટાઈમ સિગ્નલ \\
\textbf{એમ્પ્લિટ્યુડ} & એનાલોગ સિગ્નલ, ડિજિટલ સિગ્નલ \\
\textbf{પ્રકૃતિ} & ડીટર્મિનિસ્ટિક સિગ્નલ, રેન્ડમ સિગ્નલ \\
\textbf{સિમેટ્રી} & ઈવન સિગ્નલ, ઓડ સિગ્નલ \\
\textbf{એનર્જી/પાવર} & એનર્જી સિગ્નલ, પાવર સિગ્નલ \\
\end{longtable}
}

\end{solutionbox}
\begin{mnemonicbox}
``CADEN'' (Continuous/Discrete, Analog/Digital,
Deterministic/Random, Even/Odd, Energy/Power)

\end{mnemonicbox}
\subsection*{પ્રશ્ન 1(બ) [4
ગુણ]}\label{uxaaauxab0uxab6uxaa8-1uxaac-4-uxa97uxaa3}

\textbf{કંટીન્યુઅસ અને ડિસ્ક્રીટ ટાઈમ સિગ્નલ સમજાવો.}

\begin{solutionbox}

{\def\LTcaptype{none} % do not increment counter
\begin{longtable}[]{@{}
  >{\raggedright\arraybackslash}p{(\linewidth - 2\tabcolsep) * \real{0.5000}}
  >{\raggedright\arraybackslash}p{(\linewidth - 2\tabcolsep) * \real{0.5000}}@{}}
\toprule\noalign{}
\begin{minipage}[b]{\linewidth}\raggedright
કંટીન્યુઅસ-ટાઈમ સિગ્નલ
\end{minipage} & \begin{minipage}[b]{\linewidth}\raggedright
ડિસ્ક્રીટ-ટાઈમ સિગ્નલ
\end{minipage} \\
\midrule\noalign{}
\endhead
\bottomrule\noalign{}
\endlastfoot
સમયના તમામ મૂલ્યો માટે વ્યાખ્યાયિત & માત્ર ચોક્કસ સમય અંતરાલ પર વ્યાખ્યાયિત \\
x(t) તરીકે રજુ થાય છે & x[n] અથવા x(nT) તરીકે રજુ થાય છે \\
ઉદાહરણ: સાઇન વેવ જેવા એનાલોગ સિગ્નલ & ઉદાહરણ: સેમ્પલ કરેલા સ્પીચ જેવા ડિજિટલ
સિગ્નલ \\
ગ્રાફ પર સળંગ વક્ર & ગ્રાફ પર બિંદુઓની શ્રેણી \\
પ્રોસેસિંગ માટે એનાલોગ સર્કિટની જરૂર પડે & પ્રોસેસિંગ ડિજિટલ પ્રોસેસર દ્વારા કરી
શકાય \\
\end{longtable}
}

\textbf{આકૃતિ:}

\begin{center}
\textbf{Mermaid Diagram (Code)}
\begin{verbatim}
{Shaded}
{Highlighting}[]
graph LR
    A[સિગ્નલ] {-{-}{} B[કંટીન્યુઅસ{-}ટાઈમ]}
    A {-{-}{} C[ડિસ્ક્રીટ{-}ટાઈમ]}
    B {-{-}{} D[બધા t માટે વ્યાખ્યાયિત]}
    C {-{-}{} E[ચોક્કસ સમય nT પર વ્યાખ્યાયિત]}
    D {-{-}{} F["ઉદાહરણ: sin(t)"]}
    E {-{-}{} G["ઉદાહરણ: sin(nT)"]}
{Highlighting}
{Shaded}
\end{verbatim}
\end{center}

\end{solutionbox}
\begin{mnemonicbox}
``CAD'' - Continuous signals are Analog and Defined
for all time; Discrete signals are digital and defined at specific
points.

\end{mnemonicbox}
\subsection*{પ્રશ્ન 1(ક) [7
ગુણ]}\label{uxaaauxab0uxab6uxaa8-1uxa95-7-uxa97uxaa3}

\textbf{યુનિટ ઇમ્પલ્સ અને યુનિટ સ્ટેપ ફંક્શન સમજાવો.}

\begin{solutionbox}

{\def\LTcaptype{none} % do not increment counter
\begin{longtable}[]{@{}
  >{\raggedright\arraybackslash}p{(\linewidth - 2\tabcolsep) * \real{0.5091}}
  >{\raggedright\arraybackslash}p{(\linewidth - 2\tabcolsep) * \real{0.4909}}@{}}
\toprule\noalign{}
\begin{minipage}[b]{\linewidth}\raggedright
યુનિટ ઇમ્પલ્સ ફંક્શન (δ(t))
\end{minipage} & \begin{minipage}[b]{\linewidth}\raggedright
યુનિટ સ્ટેપ ફંક્શન (u(t))
\end{minipage} \\
\midrule\noalign{}
\endhead
\bottomrule\noalign{}
\endlastfoot
t=0 પર અનંત ઊંચાઈ, બાકી જગ્યાએ શૂન્ય & t\geq0 માટે મૂલ્ય 1, t\textless0 માટે 0 \\
વક્ર નીચેનું ક્ષેત્રફળ = 1 & ઇન્ટિગ્રલ રેમ્પ ફંક્શન આપે છે \\
તાત્કાલિક ઘટનાઓને રજૂ કરવા માટે & અચાનક બદલાવને રજૂ કરવા માટે \\
LTI સિસ્ટમ એનાલિસિસનો ગાણિતિક આધાર & સિસ્ટમ રિસ્પોન્સ એનાલિસિસ માટે ઉપયોગી \\
લાપ્લાસ ટ્રાન્સફોર્મ = 1 & લાપ્લાસ ટ્રાન્સફોર્મ = 1/s \\
\end{longtable}
}

\textbf{આકૃતિ:}

\begin{verbatim}
            |     \^{}
            |     |
            |     |
Unit        |     |               Unit
Impulse     | δ(t)| Area = 1      Step        u(t)
            |     |               Function    {-{-}{-}{-}{-}{-}{-}{-}{-}{-}}
            |     |                           |
            +{-{-}{-}{-}{-}+{-}{-}{-}{-}{-}{-}{-}                   +{-}{-}{-}{-}{-}{-}{-}{-}{-}{-}{-}{-}}
           {-1  0  1  t                       {-}1  0  1  t    }
\end{verbatim}

\textbf{ગુણધર્મો:}

\begin{itemize}
\tightlist
\item
  \textbf{સેમ્પલિંગ પ્રોપર્ટી}: \intf(t)δ(t-t_{0})dt = f(t_{0})
\item
  \textbf{યુનિટ સ્ટેપ ઇમ્પલ્સનું ઇન્ટિગ્રલ છે}: u(t) = \intδ(τ)dτ from -\infty to t
\item
  \textbf{ઇમ્પલ્સ યુનિટ સ્ટેપનો ડેરિવેટિવ છે}: δ(t) = du(t)/dt
\end{itemize}

\end{solutionbox}
\begin{mnemonicbox}
``SHARP-FLAT'' - Impulse is Sharp and momentary;
Step is Flat and persistent.

\end{mnemonicbox}
\subsection*{પ્રશ્ન 1(ક) OR [7
ગુણ]}\label{uxaaauxab0uxab6uxaa8-1uxa95-or-7-uxa97uxaa3}

\textbf{ડિજિટલ કોમ્યુનિકેશન સિસ્ટમનો બ્લોક ડાયાગ્રામ સમજાવો.}

\begin{solutionbox}

\textbf{ડિજિટલ કોમ્યુનિકેશન સિસ્ટમનો બ્લોક ડાયાગ્રામ:}

\begin{verbatim}
flowchart LR
    A[સોર્સ] {-{-} B[સોર્સ એન્કોડર]}
    B {-{-} C[ચેનલ એન્કોડર]}
    C {-{-} D[ડિજિટલ મોડ્યુલેટર]}
    D {-{-} E[ચેનલ]}
    E {-{-} F[ડિજિટલ ડિમોડ્યુલેટર]}
    F {-{-} G[ચેનલ ડિકોડર]}
    G {-{-} H[સોર્સ ડિકોડર]}
    H {-{-} I[ડેસ્ટિનેશન]}
\end{verbatim}

\textbf{સમજૂતી:}

{\def\LTcaptype{none} % do not increment counter
\begin{longtable}[]{@{}
  >{\raggedright\arraybackslash}p{(\linewidth - 2\tabcolsep) * \real{0.5385}}
  >{\raggedright\arraybackslash}p{(\linewidth - 2\tabcolsep) * \real{0.4615}}@{}}
\toprule\noalign{}
\begin{minipage}[b]{\linewidth}\raggedright
બ્લોક
\end{minipage} & \begin{minipage}[b]{\linewidth}\raggedright
કાર્ય
\end{minipage} \\
\midrule\noalign{}
\endhead
\bottomrule\noalign{}
\endlastfoot
\textbf{સોર્સ} & ટ્રાન્સમિટ કરવાનો મેસેજ ઉત્પન્ન કરે છે \\
\textbf{સોર્સ એન્કોડર} & મેસેજને ડિજિટલ ફોર્મમાં રૂપાંતરિત કરે છે, રિડન્ડન્સી દૂર કરે
છે \\
\textbf{ચેનલ એન્કોડર} & એરર ડિટેક્શન/કરેક્શન માટે નિયંત્રિત રિડન્ડન્સી ઉમેરે છે \\
\textbf{ડિજિટલ મોડ્યુલેટર} & ડિજિટલ બિટ્સને ટ્રાન્સમિશન માટે યોગ્ય સિગ્નલમાં
રૂપાંતરિત કરે છે \\
\textbf{ચેનલ} & ભૌતિક માધ્યમ જેના દ્વારા સિગ્નલ પ્રવાસ કરે છે \\
\textbf{ડિજિટલ ડિમોડ્યુલેટર} & પ્રાપ્ત સિગ્નલમાંથી ડિજિટલ ડેટા પુનઃપ્રાપ્ત કરે છે \\
\textbf{ચેનલ ડિકોડર} & ઉમેરેલી રિડન્ડન્સીનો ઉપયોગ કરીને એરર શોધે/સુધારે છે \\
\textbf{સોર્સ ડિકોડર} & પ્રાપ્ત બિટ્સમાંથી મૂળ સંદેશ પુનઃનિર્માણ કરે છે \\
\textbf{ડેસ્ટિનેશન} & પ્રેષિત સંદેશ પ્રાપ્ત કરે છે \\
\end{longtable}
}

\end{solutionbox}
\begin{mnemonicbox}
``SECDCSD'' - ``Seven Engineers Can Design
Communication Systems Diligently''

\end{mnemonicbox}
\subsection*{પ્રશ્ન 2(અ) [3
ગુણ]}\label{uxaaauxab0uxab6uxaa8-2uxa85-3-uxa97uxaa3}

\textbf{સિગ્નલમાં 8000 બીટ/સેકન્ડનો બીટ રેટ અને 1000 બોડનો બોડ દર હોય છે. દરેક
સિગ્નલ એલીમેંટ દ્વારા કેટલા ડેટા એલીમેંટ વહન કરવામાં આવે છે?}

\begin{solutionbox}

દરેક સિગ્નલ એલિમેન્ટ દ્વારા વહન કરાતા ડેટા એલિમેન્ટ (બિટ્સ)ની સંખ્યા: = બીટ રેટ \div
બોડ રેટ = 8000 બિટ્સ/સેકન્ડ \div 1000 બોડ = 8 બિટ્સ/સિગ્નલ એલિમેન્ટ


{\def\LTcaptype{none} % do not increment counter
\begin{longtable}[]{@{}lll@{}}
\toprule\noalign{}
પેરામીટર & મૂલ્ય & સંબંધ \\
\midrule\noalign{}
\endhead
\bottomrule\noalign{}
\endlastfoot
બીટ રેટ & 8000 બિટ્સ/સેક & આપેલ \\
બોડ રેટ & 1000 બોડ & આપેલ \\
બિટ્સ/સિગ્નલ & 8 બિટ્સ & બીટ રેટ \div બોડ રેટ \\
\end{longtable}
}

\end{solutionbox}
\begin{mnemonicbox}
``Bits Divided By Bauds'' (BDBB)

\end{mnemonicbox}
\subsection*{પ્રશ્ન 2(બ) [4
ગુણ]}\label{uxaaauxab0uxab6uxaa8-2uxaac-4-uxa97uxaa3}

\textbf{એનર્જી અને પાવર સિગ્નલ સમજાવો.}

\begin{solutionbox}

{\def\LTcaptype{none} % do not increment counter
\begin{longtable}[]{@{}ll@{}}
\toprule\noalign{}
એનર્જી સિગ્નલ & પાવર સિગ્નલ \\
\midrule\noalign{}
\endhead
\bottomrule\noalign{}
\endlastfoot
અંતિમ કુલ એનર્જી & અનંત કુલ એનર્જી પરંતુ અંતિમ સરેરાશ પાવર \\
શૂન્ય સરેરાશ પાવર & બિન-શૂન્ય સરેરાશ પાવર \\
E = \int\textbar x(t)\textbar^{2}dt (અંતિમ) &

P = lim(T\rightarrow\infty) 1/2T

\int\textbar x(t)\textbar^{2}dt (અંતિમ) \\
ઉદાહરણ: પલ્સ, ક્ષયિત એક્સપોનેન્શિયલ & ઉદાહરણ: સાઇન વેવ, સ્ક્વેર વેવ \\
સમયમાં સીમિત & બધા સમય માટે અસ્તિત્વમાં \\
\end{longtable}
}

\textbf{આકૃતિ:}

\begin{center}
\textbf{Mermaid Diagram (Code)}
\begin{verbatim}
{Shaded}
{Highlighting}[]
graph TD
    A[સિગ્નલ] {-{-}{} B[એનર્જી સિગ્નલ]}
    A {-{-}{} C[પાવર સિગ્નલ]}
    B {-{-}{} D[અંતિમ એનર્જી]}
    B {-{-}{} E[શૂન્ય સરેરાશ પાવર]}
    C {-{-}{} F[અનંત એનર્જી]}
    C {-{-}{} G[અંતિમ સરેરાશ પાવર]}
    D {-{-}{} H[ઉદાહરણ: પલ્સ]}
    G {-{-}{} I[ઉદાહરણ: સાઇન વેવ]}
{Highlighting}
{Shaded}
\end{verbatim}
\end{center}

\end{solutionbox}
\begin{mnemonicbox}
``FEZIL'' - Finite Energy is Zero in Long-term;
Power signals are Infinite in Length

\end{mnemonicbox}
\subsection*{પ્રશ્ન 2(ક) [7
ગુણ]}\label{uxaaauxab0uxab6uxaa8-2uxa95-7-uxa97uxaa3}

\textbf{FSK મોડ્યુલેટર અને ડી-મોડ્યુલેટરના બ્લોક ડાયાગ્રામને વેવફોર્મ સાથે સમજાવો.}

\begin{solutionbox}

\textbf{FSK મોડ્યુલેટર અને ડિમોડ્યુલેટર:}

\begin{verbatim}
flowchart TD
    subgraph Modulator
        A[ડિજિટલ ઇનપુટ] {-{-} B[વોલ્ટેજ કંટ્રોલ્ડ ઓસિલેટર]}
        B {-{-} C[FSK આઉટપુટ]}
    end
    subgraph Demodulator
        D[FSK ઇનપુટ] {-{-} E[બેન્ડપાસ ફિલ્ટર 1nફ્રીક્વન્સી f1]}
        D {-{-} F[બેન્ડપાસ ફિલ્ટર 2nફ્રીક્વન્સી f2]}
        E {-{-} G[એન્વેલપ ડિટેક્ટર 1]}
        F {-{-} H[એન્વેલપ ડિટેક્ટર 2]}
        G {-{-} I[કમ્પેરેટર]}
        H {-{-} I}
        I {-{-} J[ડિજિટલ આઉટપુટ]}
    end
\end{verbatim}

\textbf{વેવફોર્મ:}

\begin{verbatim}
Digital Input: \_\_\_\_\_‾‾‾‾‾\_\_\_\_\_‾‾‾‾‾\_\_\_\_\_
                 0     1     0     1     0

FSK Output:  /{//MMMMMM///MMMMMM///}
             f1    f2    f1    f2    f1
             
Received at BPF1: /{//\_\_\_\_\_///\_\_\_\_\_///}
                   f1          f1          f1
                   
Received at BPF2: \_\_\_\_\_MMMMMM\_\_\_\_\_MMMMMM\_\_\_\_\_
                        f2          f2
                        
Digital Output: \_\_\_\_\_‾‾‾‾‾\_\_\_\_\_‾‾‾‾‾\_\_\_\_\_
                 0     1     0     1     0
\end{verbatim}

\textbf{મુખ્ય સિદ્ધાંતો:}

\begin{itemize}
\tightlist
\item
  \textbf{બિટ 0}: ફ્રીક્વન્સી f_{1} તરીકે ટ્રાન્સમિટ થાય છે
\item
  \textbf{બિટ 1}: ફ્રીક્વન્સી f_{2} તરીકે ટ્રાન્સમિટ થાય છે
\item
  \textbf{ડિમોડ્યુલેશન}: ફ્રીક્વન્સીઓને અલગ કરવા માટે બેન્ડપાસ ફિલ્ટર્સનો ઉપયોગ કરે છે
\item
  \textbf{ડિટેક્શન}: એન્વેલપ ડિટેક્ટર્સ ડિજિટલ સિગ્નલને પુનઃપ્રાપ્ત કરે છે
\end{itemize}

\end{solutionbox}
\begin{mnemonicbox}
``FIST'' - Frequency Is Shifted for Transmission

\end{mnemonicbox}
\subsection*{પ્રશ્ન 2(અ) OR [3
ગુણ]}\label{uxaaauxab0uxab6uxaa8-2uxa85-or-3-uxa97uxaa3}

\textbf{સિગ્નલ 4 બીટ/સિગ્નલ એલીમેંટ ધરાવે છે. જો 1000 સિગ્નલ એલીમેંટ પ્રતિ સેકન્ડ
મોકલવામાં આવે છે. તો બીટ રેટ શોધો.}

\begin{solutionbox}

બીટ રેટ = સિગ્નલ એલિમેન્ટ દીઠ બિટ્સની સંખ્યા \times પ્રતિ સેકન્ડ સિગ્નલ એલિમેન્ટ બીટ રેટ =
4 બિટ્સ/સિગ્નલ એલિમેન્ટ \times 1000 સિગ્નલ એલિમેન્ટ/સેકન્ડ બીટ રેટ = 4000 બિટ્સ/સેકન્ડ


{\def\LTcaptype{none} % do not increment counter
\begin{longtable}[]{@{}lll@{}}
\toprule\noalign{}
પેરામીટર & મૂલ્ય & સંબંધ \\
\midrule\noalign{}
\endhead
\bottomrule\noalign{}
\endlastfoot
સિમ્બોલ દીઠ બિટ્સ & 4 & આપેલ \\
સિમ્બોલ રેટ & 1000 સિમ્બોલ/સેક & આપેલ \\
બીટ રેટ & 4000 બિટ્સ/સેક & બિટ્સ/સિમ્બોલ \times સિમ્બોલ રેટ \\
\end{longtable}
}

\end{solutionbox}
\begin{mnemonicbox}
``BBS'' - Bit rate equals Bits per symbol times
Symbol rate

\end{mnemonicbox}
\subsection*{પ્રશ્ન 2(બ) OR [4
ગુણ]}\label{uxaaauxab0uxab6uxaa8-2uxaac-or-4-uxa97uxaa3}

\textbf{ઈવન અને ઓડ સિગ્નલ સમજાવો.}

\begin{solutionbox}

{\def\LTcaptype{none} % do not increment counter
\begin{longtable}[]{@{}ll@{}}
\toprule\noalign{}
ઈવન સિગ્નલ & ઓડ સિગ્નલ \\
\midrule\noalign{}
\endhead
\bottomrule\noalign{}
\endlastfoot
y-અક્ષની આસપાસ સિમેટ્રિક & y-અક્ષની આસપાસ એન્ટી-સિમેટ્રિક \\
x(-t) = x(t) & x(-t) = -x(t) \\
ઉદાહરણ: cos(t) & ઉદાહરણ: sin(t) \\
ફૂરિયર ટ્રાન્સફોર્મ વાસ્તવિક છે & ફૂરિયર ટ્રાન્સફોર્મ કાલ્પનિક છે \\
ઈવન સિગ્નલનો સરવાળો ઈવન છે & ઓડ સિગ્નલનો સરવાળો ઓડ છે \\
\end{longtable}
}

\textbf{આકૃતિ:}

\begin{verbatim}
Even Signal x(t)             Odd Signal x(t)
    |                            |
    |     *     *                |     *
    |   *         *              |   *   {}
    | *             *            | *      {}
{-{-}{-}{-}+{-}{-}{-}{-}{-}{-}{-}{-}{-}{-}{-}{-}{-}{-}{-}+{-}{-}{-}{-}   {-}{-}{-}{-}+{-}{-}{-}{-}{-}{-}{-}{-}+{-}{-}{-}{-}+{-}{-}{-}{-}}
    |                            |        /     *
    |                            |      /   *
    |                            |     *
\end{verbatim}

\textbf{ગુણધર્મો:}

\begin{itemize}
\tightlist
\item
  કોઈપણ સિગ્નલને ઈવન અને ઓડ ઘટકોના સરવાળા તરીકે વ્યક્ત કરી શકાય છે
\item
  ઈવન ઘટક: x_{1}(t) = [x(t) + x(-t)]/2
\item
  ઓડ ઘટક: x_{2}(t) = [x(t) - x(-t)]/2
\end{itemize}

\end{solutionbox}
\begin{mnemonicbox}
``SAME-FLIP'' - Even signals are the SAME when
flipped; Odd signals FLIP their sign.

\end{mnemonicbox}
\subsection*{પ્રશ્ન 2(ક) OR [7
ગુણ]}\label{uxaaauxab0uxab6uxaa8-2uxa95-or-7-uxa97uxaa3}

\textbf{QPSK મોડ્યુલેટર અને ડી-મોડ્યુલેટરના બ્લોક ડાયાગ્રામને કોન્સોલેશન ડાયાગ્રામ
સાથે સમજાવો.}

\begin{solutionbox}

\textbf{QPSK મોડ્યુલેટર અને ડિમોડ્યુલેટર:}

\begin{verbatim}
flowchart TD
    subgraph Modulator
        A[બાઇનરી ઇનપુટ] {-{-} B[સીરિયલ ટુ પેરેલલnકન્વર્ટર]}
        B {-{-} C[ઈવન બિટ્સ]}
        B {-{-} D[ઓડ બિટ્સ]}
        C {-{-} E[મલ્ટિપ્લાયર]}
        D {-{-} F[મલ્ટિપ્લાયર]}
        G["cos(2πft)"] {-{-} E}
        H["sin(2πft)"] {-{-} F}
        E {-{-} I[સમર]}
        F {-{-} I}
        I {-{-} J[QPSK આઉટપુટ]}
    end

    subgraph Demodulator
        K[QPSK ઇનપુટ] {-{-} L[મલ્ટિપ્લાયર 1]}
        K {-{-} M[મલ્ટિપ્લાયર 2]}
        N["cos(2πft)"] {-{-} L}
        O["sin(2πft)"] {-{-} M}
        L {-{-} P[ઇન્ટિગ્રેટર 1]}
        M {-{-} Q[ઇન્ટિગ્રેટર 2]}
        P {-{-} R[ડિસિઝન ડિવાઇસ 1]}
        Q {-{-} S[ડિસિઝન ડિવાઇસ 2]}
        R {-{-} T[પેરેલલ ટુ સીરિયલnકન્વર્ટર]}
        S {-{-} T}
        T {-{-} U[બાઇનરી આઉટપુટ]}
    end
\end{verbatim}

\textbf{કોન્સ્ટેલેશન ડાયાગ્રામ:}

\begin{verbatim}
             Q
             |
      01     |     00
      •      |      •
             |
{-{-}{-}{-}{-}{-}{-}{-}{-}{-}{-}{-}{-}+{-}{-}{-}{-}{-}{-}{-}{-}{-}{-}{-}{-}{-}}
             |
      11     |     10
      •      |      •
             |             I
\end{verbatim}

\textbf{મુખ્ય લક્ષણો:}

\begin{itemize}
\tightlist
\item
  \textbf{ઇનપુટ}: દરેક સિમ્બોલ 2 બિટ્સ દ્વારા નક્કી થાય છે
\item
  \textbf{ફેઝ}: 4 ફેઝ (0^\circ, 90^\circ, 180^\circ, 270^\circ)
\item
  \textbf{બિટ્સથી ફેઝ}:

  \begin{itemize}
  \tightlist
  \item
    00: 45^\circ
  \item
    01: 135^\circ
  \item
    11: 225^\circ
  \item
    10: 315^\circ
  \end{itemize}
\item
  \textbf{બેન્ડવિડ્થ એફિશિયન્સી}: 2 બિટ્સ પ્રતિ સિમ્બોલ
\end{itemize}

\end{solutionbox}
\begin{mnemonicbox}
``QUADrature'' - 4 phases for 4 possible 2-bit
combinations

\end{mnemonicbox}
\subsection*{પ્રશ્ન 3(અ) [3
ગુણ]}\label{uxaaauxab0uxab6uxaa8-3uxa85-3-uxa97uxaa3}

\textbf{ASK મોડ્યુલેટરનું કાર્ય બ્લોક ડાયાગ્રામ અને વેવફોર્મ સાથે સમજાવો.}

\begin{solutionbox}

\textbf{ASK મોડ્યુલેટર બ્લોક ડાયાગ્રામ:}

\begin{verbatim}
flowchart LR
    A[ડિજિટલ ઇનપુટ] {-{-} B[મલ્ટિપ્લાયર]}
    C["કેરિયર જનરેટર{nsin(2πft)"] {-}{-} B}
    B {-{-} D[ASK આઉટપુટ]}
\end{verbatim}

\textbf{વેવફોર્મ:}

\begin{verbatim}
ડિજિટલ ઇનપુટ: \_\_\_\_\_‾‾‾‾‾\_\_\_\_\_‾‾‾‾‾\_\_\_\_\_
                 0     1     0     1     0

કેરિયર:      /{/////////////}

ASK આઉટપુટ:   \_\_\_\_\_/{//\_\_\_\_\_///\_\_\_\_\_}
                 0     1     0     1     0
\end{verbatim}

\textbf{કાર્ય સિદ્ધાંત:}

\begin{itemize}
\tightlist
\item
  ડિજિટલ 1: કેરિયર સિગ્નલ ટ્રાન્સમિટ થાય છે
\item
  ડિજિટલ 0: કોઈ સિગ્નલ નહીં (અથવા ઓછી એમ્પ્લિટ્યુડ) ટ્રાન્સમિટ થાય છે
\item
  આઉટપુટ એમ્પ્લિટ્યુડ ઇનપુટ ડિજિટલ સિગ્નલ સાથે બદલાય છે
\end{itemize}

\end{solutionbox}
\begin{mnemonicbox}
``ASKY'' - Amplitude Switches the Carrier? Yes!

\end{mnemonicbox}
\subsection*{પ્રશ્ન 3(બ) [4
ગુણ]}\label{uxaaauxab0uxab6uxaa8-3uxaac-4-uxa97uxaa3}

\textbf{8-PSK અને 16-QAM ના કોન્સોલેશન ડાયાગ્રામ દોરો.}

\begin{solutionbox}

\textbf{8-PSK કોન્સ્ટેલેશન ડાયાગ્રામ:}

\begin{verbatim}
                  Q
                  |
        011  •    |    • 000
              {   |   /}
         110 •  { | /  • 001}
                 {|/}
        {-{-}{-}{-}{-}{-}{-}{-}{-}+{-}{-}{-}{-}{-}{-}{-}{-}{-} I}
                /|{}
         101 •  / | {  • 010}
              /   |   {}
        100  •    |    • 011
                  |
\end{verbatim}

\textbf{16-QAM કોન્સ્ટેલેશન ડાયાગ્રામ:}

\begin{verbatim}
                Q
        •   •   |   •   •
                |
        •   •   |   •   •
                |
        {-{-}{-}{-}{-}{-}{-}{-}+{-}{-}{-}{-}{-}{-}{-}{-}}
                |
        •   •   |   •   •
                |
        •   •   |   •   •
                |         I
\end{verbatim}

\textbf{મુખ્ય તફાવતો:}

\begin{itemize}
\tightlist
\item
  \textbf{8-PSK}: 8 સિમ્બોલ, સમાન એમ્પ્લિટ્યુડ, 45^\circ અંતરે ફેઝ
\item
  \textbf{16-QAM}: 16 સિમ્બોલ, બદલાતી એમ્પ્લિટ્યુડ અને ફેઝ
\end{itemize}

\end{solutionbox}
\begin{mnemonicbox}
``P-Phase Q-Quantity'' - PSK varies Phase only; QAM
varies both amplitude (Quantity) and phase

\end{mnemonicbox}
\subsection*{પ્રશ્ન 3(ક) [7
ગુણ]}\label{uxaaauxab0uxab6uxaa8-3uxa95-7-uxa97uxaa3}

\textbf{1100101101 ના ક્રમ માટે ASK અને FSK મોડ્યુલેશન વેવફોર્મ દોરો.}

\begin{solutionbox}

\textbf{મોડ્યુલેશન વેવફોર્મ:}

\begin{verbatim}
Binary Input:  ‾‾‾‾‾‾‾‾‾‾\_\_\_\_\_‾‾‾‾‾\_\_\_\_\_‾‾‾‾‾‾‾‾‾‾
               1  1  0  0  1  0  1  1  0  1
               
Carrier:       /{/////////////////}

ASK Output:    /{/////\_\_\_\_\_///\_\_\_\_\_/////}
               1  1  0  0  1  0  1  1  0  1
               
FSK Output:    MMMMMMMMMM/{///MMMMM////MMMMMMMMMM}
               1  1  0  0  1  0  1  1  0  1
               f2 f2 f1 f1 f2 f1 f2 f2 f1 f2
\end{verbatim}

\textbf{મુખ્ય લક્ષણો:}

\begin{itemize}
\tightlist
\item
  \textbf{ASK}: બિટ 1 માટે કેરિયર હાજર, બિટ 0 માટે ગેરહાજર
\item
  \textbf{FSK}: બિટ 1 માટે ઉચ્ચ ફ્રીક્વન્સી (f_{2}), બિટ 0 માટે નીચી ફ્રીક્વન્સી
  (f_{1})
\end{itemize}

\textbf{મોડ્યુલેશન પદ્ધતિનું કોષ્ટક:}

{\def\LTcaptype{none} % do not increment counter
\begin{longtable}[]{@{}llll@{}}
\toprule\noalign{}
મોડ્યુલેશન & બિટ 0 & બિટ 1 & બદલાતો પેરામીટર \\
\midrule\noalign{}
\endhead
\bottomrule\noalign{}
\endlastfoot
ASK & શૂન્ય અથવા ઓછી એમ્પ્લિટ્યુડ & ઉચ્ચ એમ્પ્લિટ્યુડ & એમ્પ્લિટ્યુડ \\
FSK & ફ્રીક્વન્સી f_{1} & ફ્રીક્વન્સી f_{2} & ફ્રીક્વન્સી \\
\end{longtable}
}

\end{solutionbox}
\begin{mnemonicbox}
``AFRO'' - Amplitude For 1, Remove for 0 (ASK);
Frequency Rises for 1, Off-peak for 0 (FSK)

\end{mnemonicbox}
\subsection*{પ્રશ્ન 3(અ) OR [3
ગુણ]}\label{uxaaauxab0uxab6uxaa8-3uxa85-or-3-uxa97uxaa3}

\textbf{PSK મોડ્યુલેટરનું કાર્ય બ્લોક ડાયાગ્રામ અને વેવફોર્મ સાથે સમજાવો.}

\begin{solutionbox}

\textbf{PSK મોડ્યુલેટર બ્લોક ડાયાગ્રામ:}

\begin{verbatim}
flowchart LR
    A[ડિજિટલ ઇનપુટ] {-{-} B[પોલર કન્વર્ટરn0-1, 1+1]}
    B {-{-} C[મલ્ટિપ્લાયર]}
    D["કેરિયર જનરેટર{nsin(2πft)"] {-}{-} C}
    C {-{-} E[PSK આઉટપુટ]}
\end{verbatim}

\textbf{વેવફોર્મ:}

\begin{verbatim}
Digital Input: \_\_\_\_\_‾‾‾‾‾\_\_\_\_\_‾‾‾‾‾\_\_\_\_\_
                 0     1     0     1     0

Carrier:      /{/////////////}

PSK Output:   {//////////////}
               0     1     0     1     0
               180^  0^   180^  0^   180^
\end{verbatim}

\textbf{કાર્ય સિદ્ધાંત:}

\begin{itemize}
\tightlist
\item
  ડિજિટલ 1: 0^\circ ફેઝ સાથે કેરિયર સિગ્નલ
\item
  ડિજિટલ 0: 180^\circ ફેઝ સાથે કેરિયર સિગ્નલ (ઉલટું)
\item
  એમ્પ્લિટ્યુડ સ્થિર રહે છે, માત્ર ફેઝ બદલાય છે
\end{itemize}

\end{solutionbox}
\begin{mnemonicbox}
``PSKIT'' - Phase Shift Keeps Information True

\end{mnemonicbox}
\subsection*{પ્રશ્ન 3(બ) OR [4
ગુણ]}\label{uxaaauxab0uxab6uxaa8-3uxaac-or-4-uxa97uxaa3}

\textbf{1101001101 ના ક્રમ માટે MSK મોડ્યુલેશન વેવફોર્મ દોરો.}

\begin{solutionbox}

\textbf{MSK મોડ્યુલેશન વેવફોર્મ:}

\begin{verbatim}
Binary Input:  ‾‾‾‾‾\_\_\_\_\_‾‾‾‾‾\_\_\_\_\_‾‾‾‾‾‾‾‾‾‾
               1  1  0  1  0  0  1  1  0  1
               
MSK Output:    {//MMMMM//MMMMM////MMMMM}
               1  1  0  1  0  0  1  1  0  1
\end{verbatim}

\textbf{MSKના લક્ષણો:}

\begin{itemize}
\tightlist
\item
  સતત ફેઝ ટ્રાન્ઝિશન (કોઈ ફેઝ જમ્પ નહીં)
\item
  f_{1} અને f_{2} વચ્ચે ફ્રીક્વન્સી શિફ્ટ
\item
  ન્યૂનતમ ફ્રીક્વન્સી સેપરેશન: Δf = 1/(2T)
\item
  FSK કરતાં વધુ સ્મૂધ ટ્રાન્ઝિશન
\end{itemize}


{\def\LTcaptype{none} % do not increment counter
\begin{longtable}[]{@{}ll@{}}
\toprule\noalign{}
લક્ષણ & MSK લક્ષણ \\
\midrule\noalign{}
\endhead
\bottomrule\noalign{}
\endlastfoot
ફેઝ કન્ટિન્યુઇટી & સતત, કોઈ અચાનક બદલાવ નહીં \\
ફ્રીક્વન્સી ડેવિએશન & ન્યૂનતમ શક્ય (1/2T) \\
સ્પેક્ટ્રલ એફિશિયન્સી & પરંપરાગત FSK કરતાં વધુ સારી \\
બેન્ડવિડ્થ & બીટ રેટનો 1.5 ગણો \\
\end{longtable}
}

\end{solutionbox}
\begin{mnemonicbox}
``MINIMUM SMOOTH'' - MSK uses Minimum frequency
separation with Smooth transitions

\end{mnemonicbox}
\subsection*{પ્રશ્ન 3(ક) OR [7
ગુણ]}\label{uxaaauxab0uxab6uxaa8-3uxa95-or-7-uxa97uxaa3}

\textbf{1100101011 માટે BPSK અને QPSK મોડ્યુલેશન વેવફોર્મ દોરો.}

\begin{solutionbox}

\textbf{BPSK અને QPSK મોડ્યુલેશન વેવફોર્મ:}

\begin{verbatim}
Binary Input:     ‾‾‾‾‾‾‾‾‾‾\_\_\_\_\_‾‾‾‾‾\_\_\_\_\_‾‾‾‾‾‾‾‾‾‾
                  1  1  0  0  1  0  1  0  1  1
                  
BPSK Output:      /{/////////////////}
                  0^  0^ 180^180^ 0^ 180^ 0^ 180^ 0^  0^
                  
QPSK (I channel): /{//\_\_\_\_\_///\_\_\_\_\_///}
                  11   00    10    01    11
                  
QPSK (Q channel): /{/////\_\_\_\_\_///\_\_\_\_\_}
                  11   00    10    01    11
                  
QPSK (combined):  {///MMMMM///MMMMM///}
                  11   00    10    01    11
\end{verbatim}

\textbf{મુખ્ય તફાવતો:}

\begin{itemize}
\tightlist
\item
  \textbf{BPSK}: 1 બીટ પ્રતિ સિમ્બોલ, 2 ફેઝ (0^\circ અને 180^\circ)
\item
  \textbf{QPSK}: 2 બીટ પ્રતિ સિમ્બોલ, 4 ફેઝ (45^\circ, 135^\circ, 225^\circ, 315^\circ)
\item
  \textbf{QPSK જોડી}: 00, 01, 10, 11 અલગ-અલગ ફેઝને મેપ કરે છે
\end{itemize}


{\def\LTcaptype{none} % do not increment counter
\begin{longtable}[]{@{}llll@{}}
\toprule\noalign{}
મોડ્યુલેશન & બીટ્સ/સિમ્બોલ & ફેઝની સંખ્યા & બેન્ડવિડ્થ એફિશિયન્સી \\
\midrule\noalign{}
\endhead
\bottomrule\noalign{}
\endlastfoot
BPSK & 1 & 2 & 1 બીટ/Hz \\
QPSK & 2 & 4 & 2 બીટ/Hz \\
\end{longtable}
}

\end{solutionbox}
\begin{mnemonicbox}
``ONE-TWO'' - ONE bit for BPSK, TWO bits for QPSK

\end{mnemonicbox}
\subsection*{પ્રશ્ન 4(અ) [3
ગુણ]}\label{uxaaauxab0uxab6uxaa8-4uxa85-3-uxa97uxaa3}

\textbf{નીચેની પ્રોબેબીલીટી ક્રમ માટે હફમેન કોડનો ઉપયોગ કરીને ડેટાને એન્કોડ કરો. P
= \{ 0.4, 0.2, 0.2, 0.1, 0.1\}}

\begin{solutionbox}

\textbf{હફમેન કોડિંગ પ્રક્રિયા:}

{\def\LTcaptype{none} % do not increment counter
\begin{longtable}[]{@{}lll@{}}
\toprule\noalign{}
સિમ્બોલ & પ્રોબેબિલિટી & હફમેન કોડ \\
\midrule\noalign{}
\endhead
\bottomrule\noalign{}
\endlastfoot
A & 0.4 & 0 \\
B & 0.2 & 10 \\
C & 0.2 & 11 \\
D & 0.1 & 110 \\
E & 0.1 & 111 \\
\end{longtable}
}

\textbf{હફમેન ટ્રી:}

\begin{verbatim}
                [1.0]
               /     {}
              /       {}
           [0.6]      [0.4] A:0
          /     {}
         /       {}
      [0.4]     [0.2] B:10
     /     {}
    /       {}
 [0.2] C:11 [0.2]
           /     {}
          /       {}
     [0.1] D:110 [0.1] E:111
\end{verbatim}

\end{solutionbox}
\begin{mnemonicbox}
``Higher Probability Means Shorter Code''

\end{mnemonicbox}
\subsection*{પ્રશ્ન 4(બ) [4
ગુણ]}\label{uxaaauxab0uxab6uxaa8-4uxaac-4-uxa97uxaa3}

\textbf{સંભાવના અને એન્ટ્રોપી વ્યાખ્યાયિત કરો.}

\begin{solutionbox}

{\def\LTcaptype{none} % do not increment counter
\begin{longtable}[]{@{}
  >{\raggedright\arraybackslash}p{(\linewidth - 6\tabcolsep) * \real{0.3103}}
  >{\raggedright\arraybackslash}p{(\linewidth - 6\tabcolsep) * \real{0.3103}}
  >{\raggedright\arraybackslash}p{(\linewidth - 6\tabcolsep) * \real{0.1724}}
  >{\raggedright\arraybackslash}p{(\linewidth - 6\tabcolsep) * \real{0.2069}}@{}}
\toprule\noalign{}
\begin{minipage}[b]{\linewidth}\raggedright
સંકલ્પના
\end{minipage} & \begin{minipage}[b]{\linewidth}\raggedright
વ્યાખ્યા
\end{minipage} & \begin{minipage}[b]{\linewidth}\raggedright
સૂત્ર
\end{minipage} & \begin{minipage}[b]{\linewidth}\raggedright
મહત્વ
\end{minipage} \\
\midrule\noalign{}
\endhead
\bottomrule\noalign{}
\endlastfoot
\textbf{સંભાવના} & ઘટના ઘટવાની સંભાવનાનું માપ & P(A) = અનુકૂળ પરિણામોની સંખ્યા /
કુલ શક્ય પરિણામોની સંખ્યા & કોમ્યુનિકેશનમાં અનિશ્ચિતતા મોડેલ કરવા માટે ઉપયોગી \\
\textbf{એન્ટ્રોપી} & સિસ્ટમમાં અનિશ્ચિતતા અથવા રેન્ડમનેસનું માપ & H(X) = -\sum P(xi)
log_{2} P(xi) & સરેરાશ માહિતી સામગ્રી દર્શાવે છે \\
\end{longtable}
}

\textbf{મુખ્ય લક્ષણો:}

\begin{itemize}
\tightlist
\item
  \textbf{સંભાવના રેન્જ}: 0 \leq P(A) \leq 1
\item
  \textbf{એન્ટ્રોપી એકમો}: બિટ્સ (log_{2}નો ઉપયોગ કરીને)
\item
  \textbf{મહત્તમ એન્ટ્રોપી}: જ્યારે બધી ઘટનાઓ સમાન સંભાવના ધરાવે છે
\item
  \textbf{ન્યૂનતમ એન્ટ્રોપી}: જ્યારે પરિણામ નિશ્ચિત હોય (સંભાવના = 1)
\end{itemize}

\end{solutionbox}
\begin{mnemonicbox}
``PURE'' - Probability Underpins Randomness
Estimation

\end{mnemonicbox}
\subsection*{પ્રશ્ન 4(ક) [7
ગુણ]}\label{uxaaauxab0uxab6uxaa8-4uxa95-7-uxa97uxaa3}

\textbf{CDMA ટેકનિકને વિગતવાર સમજાવો.}

\begin{solutionbox}

\textbf{CDMA (કોડ ડિવિઝન મલ્ટિપલ એક્સેસ):}

\begin{verbatim}
flowchart LR
    A[યુઝર ડેટા] {-{-} B[યુનિક કોડ સાથેnસ્પ્રેડિંગ]}
    B {-{-} C[મોડ્યુલેશન]}
    C {-{-} D[ટ્રાન્સમિશન]}
    D {-{-} E[રિસેપ્શન]}
    E {-{-} F[ડિમોડ્યુલેશન]}
    F {-{-} G[મેચિંગ કોડ સાથેnડિસ્પ્રેડિંગ]}
    G {-{-} H[મૂળ યુઝર ડેટા]}
\end{verbatim}

\textbf{CDMA લક્ષણોનું કોષ્ટક:}

{\def\LTcaptype{none} % do not increment counter
\begin{longtable}[]{@{}ll@{}}
\toprule\noalign{}
લક્ષણ & વર્ણન \\
\midrule\noalign{}
\endhead
\bottomrule\noalign{}
\endlastfoot
\textbf{એક્સેસ મેથડ} & બહુવિધ વપરાશકર્તાઓ એક જ ફ્રીક્વન્સી અને સમય શેર કરે છે \\
\textbf{વિભાજન} & વપરાશકર્તાઓને અનન્ય સ્પ્રેડિંગ કોડ દ્વારા અલગ પાડવામાં આવે છે \\
\textbf{સ્પ્રેડિંગ કોડ} & ઓર્થોગોનલ અથવા પ્સ્યુડો-ઓર્થોગોનલ સિક્વન્સ \\
\textbf{પ્રોસેસિંગ ગેઇન} & સ્પ્રેડ બેન્ડવિડ્થનો મૂળ બેન્ડવિડ્થ સાથેનો ગુણોત્તર \\
\textbf{મલ્ટિપલ એક્સેસ} & ફ્રીક્વન્સી અથવા સમય વિભાજનને બદલે કોડ સ્પેસનો ઉપયોગ કરે
છે \\
\textbf{ઇન્ટરફેરન્સ રિજેક્શન} & નેરોબેન્ડ ઇન્ટરફેરન્સને નકારવાની અંતર્ગત ક્ષમતા \\
\end{longtable}
}

\textbf{મુખ્ય ફાયદાઓ:}

\begin{itemize}
\tightlist
\item
  \textbf{ક્ષમતા}: ઘણા કિસ્સાઓમાં FDMA/TDMA કરતાં વધારે
\item
  \textbf{સુરક્ષા}: સ્પ્રેડિંગ કોડ દ્વારા અંતર્ગત એન્ક્રિપ્શન
\item
  \textbf{મલ્ટિપાથ રિજેક્શન}: રેક રિસીવર મલ્ટિપાથ ઘટકોને જોડી શકે છે
\item
  \textbf{સોફ્ટ હેન્ડઓફ}: મોબાઇલ એક સાથે બહુવિધ બેઝ સ્ટેશનો સાથે વાતચીત કરી શકે છે
\end{itemize}

\end{solutionbox}
\begin{mnemonicbox}
``CODES'' - Capacity Optimized with Direct-sequence
Encoding Schemes

\end{mnemonicbox}
\subsection*{પ્રશ્ન 4(અ) OR [3
ગુણ]}\label{uxaaauxab0uxab6uxaa8-4uxa85-or-3-uxa97uxaa3}

\textbf{નીચેના પ્રોબેબીલીટી ક્રમ માટે શેનોન ફેનો કોડનો ઉપયોગ કરીને ડેટાને એન્કોડ
કરો. P = \{ 0.5, 0.25, 0.125, 0.125\}}

\begin{solutionbox}

\textbf{શેનોન-ફેનો કોડિંગ પ્રક્રિયા:}

{\def\LTcaptype{none} % do not increment counter
\begin{longtable}[]{@{}lll@{}}
\toprule\noalign{}
સિમ્બોલ & પ્રોબેબિલિટી & શેનોન-ફેનો કોડ \\
\midrule\noalign{}
\endhead
\bottomrule\noalign{}
\endlastfoot
A & 0.5 & 0 \\
B & 0.25 & 10 \\
C & 0.125 & 110 \\
D & 0.125 & 111 \\
\end{longtable}
}

\textbf{શેનોન-ફેનો ટ્રી:}

\begin{verbatim}
               [1.0]
              /     {}
             /       {}
        [0.5] A      [0.5]
                    /     {}
                   /       {}
              [0.25] B     [0.25]
                          /     {}
                         /       {}
                  [0.125] C     [0.125] D
                   કોડ:110     કોડ:111
\end{verbatim}

\end{solutionbox}
\begin{mnemonicbox}
``Split For Optimum'' - Shannon-Fano splits groups
for optimum coding

\end{mnemonicbox}
\subsection*{પ્રશ્ન 4(બ) OR [4
ગુણ]}\label{uxaaauxab0uxab6uxaa8-4uxaac-or-4-uxa97uxaa3}

\textbf{ઈન્ફોર્મેશન અને ચેનલ કેપેસિટી વ્યાખ્યાયિત કરો.}

\begin{solutionbox}

{\def\LTcaptype{none} % do not increment counter
\begin{longtable}[]{@{}
  >{\raggedright\arraybackslash}p{(\linewidth - 6\tabcolsep) * \real{0.3103}}
  >{\raggedright\arraybackslash}p{(\linewidth - 6\tabcolsep) * \real{0.3103}}
  >{\raggedright\arraybackslash}p{(\linewidth - 6\tabcolsep) * \real{0.1724}}
  >{\raggedright\arraybackslash}p{(\linewidth - 6\tabcolsep) * \real{0.2069}}@{}}
\toprule\noalign{}
\begin{minipage}[b]{\linewidth}\raggedright
સંકલ્પના
\end{minipage} & \begin{minipage}[b]{\linewidth}\raggedright
વ્યાખ્યા
\end{minipage} & \begin{minipage}[b]{\linewidth}\raggedright
સૂત્ર
\end{minipage} & \begin{minipage}[b]{\linewidth}\raggedright
મહત્વ
\end{minipage} \\
\midrule\noalign{}
\endhead
\bottomrule\noalign{}
\endlastfoot
\textbf{ઈન્ફોર્મેશન} & અનિશ્ચિતતામાં ઘટાડાનું માપ & I(x) = -log_{2} P(x) & ઓછી
સંભાવના ધરાવતી ઘટનાઓ વધુ માહિતી ધરાવે છે \\
\textbf{ચેનલ કેપેસિટી} & મહત્તમ દર જે પર નિર્ધારિત ત્રુટિ સાથે માહિતી પ્રસારિત કરી
શકાય & C = B log_{2}(1 + S/N) & વિશ્વસનીય કોમ્યુનિકેશનની મૂળભૂત મર્યાદા \\
\end{longtable}
}

\textbf{મુખ્ય મુદ્દાઓ:}

\begin{itemize}
\tightlist
\item
  \textbf{ઈન્ફોર્મેશન એકમો}: બિટ્સ (log_{2}નો ઉપયોગ કરીને)
\item
  \textbf{ચેનલ કેપેસિટી એકમો}: બિટ્સ પ્રતિ સેકન્ડ
\item
  \textbf{કેપેસિટીને અસર કરતા પરિબળો}:

  \begin{itemize}
  \tightlist
  \item
    બેન્ડવિડ્થ (B)
  \item
    સિગ્નલ-ટુ-નોઇઝ રેશિયો (S/N)
  \end{itemize}
\end{itemize}

\end{solutionbox}
\begin{mnemonicbox}
``INCHES'' - Information Numerically Calculated,
Hopping through Efficient Shannon limit

\end{mnemonicbox}
\subsection*{પ્રશ્ન 4(ક) OR [7
ગુણ]}\label{uxaaauxab0uxab6uxaa8-4uxa95-or-7-uxa97uxaa3}

\textbf{TDMA ટેકનિકને વિગતવાર સમજાવો.}

\begin{solutionbox}

\textbf{TDMA (ટાઇમ ડિવિઝન મલ્ટિપલ એક્સેસ):}

\begin{verbatim}
flowchart LR
    A[યુઝર 1] {-{-} B[ટાઇમ સ્લોટ 1]}
    C[યુઝર 2] {-{-} D[ટાઇમ સ્લોટ 2]}
    E[યુઝર 3] {-{-} F[ટાઇમ સ્લોટ 3]}
    G[યુઝર 4] {-{-} H[ટાઇમ સ્લોટ 4]}
    B {-{-} I[મલ્ટિપ્લેક્સર]}
    D {-{-} I}
    F {-{-} I}
    H {-{-} I}
    I {-{-} J[ટ્રાન્સમિશન ચેનલ]}
    J {-{-} K[ડિમલ્ટિપ્લેક્સર]}
    K {-{-} L[ટાઇમ સ્લોટ 1]}
    K {-{-} M[ટાઇમ સ્લોટ 2]}
    K {-{-} N[ટાઇમ સ્લોટ 3]}
    K {-{-} O[ટાઇમ સ્લોટ 4]}
    L {-{-} P[યુઝર 1]}
    M {-{-} Q[યુઝર 2]}
    N {-{-} R[યુઝર 3]}
    O {-{-} S[યુઝર 4]}
\end{verbatim}

\textbf{TDMA લક્ષણોનું કોષ્ટક:}

{\def\LTcaptype{none} % do not increment counter
\begin{longtable}[]{@{}
  >{\raggedright\arraybackslash}p{(\linewidth - 2\tabcolsep) * \real{0.5000}}
  >{\raggedright\arraybackslash}p{(\linewidth - 2\tabcolsep) * \real{0.5000}}@{}}
\toprule\noalign{}
\begin{minipage}[b]{\linewidth}\raggedright
લક્ષણ
\end{minipage} & \begin{minipage}[b]{\linewidth}\raggedright
વર્ણન
\end{minipage} \\
\midrule\noalign{}
\endhead
\bottomrule\noalign{}
\endlastfoot
\textbf{એક્સેસ મેથડ} & બહુવિધ વપરાશકર્તાઓ એક જ ફ્રીક્વન્સી અલગ-અલગ ટાઇમ સ્લોટમાં
શેર કરે છે \\
\textbf{ફ્રેમ સ્ટ્રક્ચર} & સમય ફ્રેમમાં વિભાજિત, ફ્રેમ સ્લોટમાં વિભાજિત \\
\textbf{ગાર્ડ ટાઇમ} & ઓવરલેપ ટાળવા માટે સ્લોટ વચ્ચે ટૂંકા સમયગાળા \\
\textbf{સિન્ક્રોનાઇઝેશન} & ટ્રાન્સમિટર અને રિસીવર વચ્ચે ચોક્કસ ટાઇમિંગની જરૂર \\
\textbf{કાર્યક્ષમતા} & ઉચ્ચ સ્પેક્ટ્રમ ઉપયોગ \\
\textbf{પાવર કન્ઝમ્પશન} & ટ્રાન્સમિટર માત્ર સોંપાયેલા સ્લોટ દરમિયાન ચાલુ \\
\end{longtable}
}

\textbf{TDMA ફ્રેમ સ્ટ્રક્ચર:}

\begin{verbatim}
|{{-}{-}{-}{-}{-}{-}{-}{-}{-}{-}{-}{-}{-}{-}{-}{-}{-}{-}{-} TDMA Frame {-}{-}{-}{-}{-}{-}{-}{-}{-}{-}{-}{-}{-}{-}{-}{-}{-}{-}{-}|}
| TS1 | TS2 | TS3 | TS4 | TS1 | TS2 | TS3 | TS4 | ...
|User1|User2|User3|User4|User1|User2|User3|User4| ...
\end{verbatim}

\end{solutionbox}
\begin{mnemonicbox}
``TIME'' - Transmission In Measured Epochs

\end{mnemonicbox}
\subsection*{પ્રશ્ન 5(અ) [3
ગુણ]}\label{uxaaauxab0uxab6uxaa8-5uxa85-3-uxa97uxaa3}

\textbf{T1 કેરિયર સિસ્ટમ સમજાવો.}

\begin{solutionbox}

\textbf{T1 કેરિયર સિસ્ટમ:}

{\def\LTcaptype{none} % do not increment counter
\begin{longtable}[]{@{}ll@{}}
\toprule\noalign{}
લક્ષણ & સ્પેસિફિકેશન \\
\midrule\noalign{}
\endhead
\bottomrule\noalign{}
\endlastfoot
\textbf{ડેટા રેટ} & 1.544 Mbps \\
\textbf{ચેનલ} & 24 વોઇસ ચેનલ \\
\textbf{વોઇસ સેમ્પલિંગ} & 8000 સેમ્પલ/સેકન્ડ \\
\textbf{સેમ્પલ સાઇઝ} & 8 બિટ્સ પ્રતિ સેમ્પલ \\
\textbf{ફ્રેમ સાઇઝ} & 193 બિટ્સ (24\times8 + 1) \\
\textbf{ફ્રેમ રેટ} & 8000 ફ્રેમ/સેકન્ડ \\
\end{longtable}
}

\textbf{T1 ફ્રેમ સ્ટ્રક્ચર:}

\begin{verbatim}
|{{-}{-}{-}{-}{-}{-}{-}{-}{-}{-}{-}{-}{-}{-}{-} T1 Frame (193 bits) {-}{-}{-}{-}{-}{-}{-}{-}{-}{-}{-}{-}{-}{-}{-}{-}{-}{-}|}
| F | Ch1 | Ch2 | Ch3 | ... | Ch24 | F | Ch1 | Ch2 | ... |
| 1 |  8  |  8  |  8  | ... |  8   | 1 |  8  |  8  | ... |
\end{verbatim}

\end{solutionbox}
\begin{mnemonicbox}
``T1-24-8-8'' - T1 has 24 channels, 8 bits, 8kHz

\end{mnemonicbox}
\subsection*{પ્રશ્ન 5(બ) [4
ગુણ]}\label{uxaaauxab0uxab6uxaa8-5uxaac-4-uxa97uxaa3}

\textbf{ટાઈમ ડિવિઝન મલ્ટિપ્લેક્સિંગ ટેકનિક (TDM) ને વિગતવાર સમજાવો.}

\begin{solutionbox}

\textbf{ટાઇમ ડિવિઝન મલ્ટિપ્લેક્સિંગ (TDM):}

\begin{verbatim}
flowchart LR
    A[સિગ્નલ 1] {-{-} E[મલ્ટિપ્લેક્સર]}
    B[સિગ્નલ 2] {-{-} E}
    C[સિગ્નલ 3] {-{-} E}
    D[સિગ્નલ 4] {-{-} E}
    E {-{-} F[ટ્રાન્સમિશન ચેનલ]}
    F {-{-} G[ડિમલ્ટિપ્લેક્સર]}
    G {-{-} H[સિગ્નલ 1]}
    G {-{-} I[સિગ્નલ 2]}
    G {-{-} J[સિગ્નલ 3]}
    G {-{-} K[સિગ્નલ 4]}
\end{verbatim}

\textbf{TDM લક્ષણોનું કોષ્ટક:}

{\def\LTcaptype{none} % do not increment counter
\begin{longtable}[]{@{}ll@{}}
\toprule\noalign{}
લક્ષણ & વર્ણન \\
\midrule\noalign{}
\endhead
\bottomrule\noalign{}
\endlastfoot
\textbf{સિદ્ધાંત} & બહુવિધ સિગ્નલ વારાફરતી લઈને એક ચેનલ શેર કરે છે \\
\textbf{સમય ફાળવણી} & દરેક સિગ્નલને નિશ્ચિત સમય સ્લોટ ફાળવવામાં આવે છે \\
\textbf{સિન્ક્રોનાઇઝેશન} & મલ્ટિપ્લેક્સર અને ડિમલ્ટિપ્લેક્સર વચ્ચે ચોક્કસ ટાઇમિંગની
જરૂર \\
\textbf{ઇન્ટરલીવિંગ} & વિવિધ સ્ત્રોતોના સેમ્પલ સમયમાં ઇન્ટરલીવ્ડ \\
\textbf{પ્રકારો} & સિન્ક્રોનસ TDM અને એસિન્ક્રોનસ (સ્ટેટિસ્ટિકલ) TDM \\
\end{longtable}
}

\textbf{TDM ફ્રેમ સ્ટ્રક્ચર:}

\begin{verbatim}
|{{-}{-}{-}{-}{-}{-}{-}{-}{-}{-}{-}{-}{-}{-}{-}{-} TDM Frame {-}{-}{-}{-}{-}{-}{-}{-}{-}{-}{-}{-}{-}{-}{-}{-}|}
| S1 | S2 | S3 | S4 | S1 | S2 | S3 | S4 | ... |
\end{verbatim}

\end{solutionbox}
\begin{mnemonicbox}
``TWIST'' - Time Windows Interleaving Signals
Together

\end{mnemonicbox}
\subsection*{પ્રશ્ન 5(ક) [7
ગુણ]}\label{uxaaauxab0uxab6uxaa8-5uxa95-7-uxa97uxaa3}

\textbf{ઇન્ફોમેશન સિક્યોરિટીમાં આવતા સિક્યોરિટી ઘટકોને વિગતવાર સમજાવો.}

\begin{solutionbox}

\textbf{ઇન્ફોર્મેશન સિક્યોરિટી ઘટકો:}

\begin{center}
\textbf{Mermaid Diagram (Code)}
\begin{verbatim}
{Shaded}
{Highlighting}[]
graph TD
    A[ઇન્ફોર્મેશન સિક્યોરિટી] {-{-}{} B[કોન્ફિડેન્શિયાલિટી]}
    A {-{-}{} C[ઇન્ટેગ્રિટી]}
    A {-{-}{} D[એવેલેબિલિટી]}
    B {-{-}{} E[એન્ક્રિપ્શન]}
    B {-{-}{} F[એક્સેસ કંટ્રોલ]}
    C {-{-}{} G[ડિજિટલ સિગ્નેચર]}
    C {-{-}{} H[હેશિંગ]}
    D {-{-}{} I[રેડન્ડન્સી]}
    D {-{-}{} J[બેકઅપ સિસ્ટમ]}
{Highlighting}
{Shaded}
\end{verbatim}
\end{center}

\textbf{સિક્યોરિટી ઘટકોનું કોષ્ટક:}

{\def\LTcaptype{none} % do not increment counter
\begin{longtable}[]{@{}
  >{\raggedright\arraybackslash}p{(\linewidth - 4\tabcolsep) * \real{0.2292}}
  >{\raggedright\arraybackslash}p{(\linewidth - 4\tabcolsep) * \real{0.2708}}
  >{\raggedright\arraybackslash}p{(\linewidth - 4\tabcolsep) * \real{0.5000}}@{}}
\toprule\noalign{}
\begin{minipage}[b]{\linewidth}\raggedright
ઘટક
\end{minipage} & \begin{minipage}[b]{\linewidth}\raggedright
વર્ણન
\end{minipage} & \begin{minipage}[b]{\linewidth}\raggedright
અમલીકરણ પદ્ધતિઓ
\end{minipage} \\
\midrule\noalign{}
\endhead
\bottomrule\noalign{}
\endlastfoot
\textbf{કોન્ફિડેન્શિયાલિટી} & માહિતી માત્ર અધિકૃત વપરાશકર્તાઓને જ ઉપલબ્ધ થાય તેની
ખાતરી & એન્ક્રિપ્શન, એક્સેસ કંટ્રોલ, ઓથેન્ટિકેશન \\
\textbf{ઇન્ટેગ્રિટી} & ડેટાની સચોટતા અને સુસંગતતા જાળવવી & ડિજિટલ સિગ્નેચર, હેશિંગ,
ચેકસમ \\
\textbf{એવેલેબિલિટી} & જ્યારે જરૂર હોય ત્યારે માહિતી ઉપલબ્ધ થાય તેની ખાતરી &
રેડન્ડન્સી, બેકઅપ સિસ્ટમ, ડિઝાસ્ટર રિકવરી \\
\textbf{ઓથેન્ટિકેશન} & વપરાશકર્તાઓની ઓળખની ચકાસણી & પાસવર્ડ, બાયોમેટ્રિક્સ,
ડિજિટલ સર્ટિફિકેટ \\
\textbf{નોન-રેપ્યુડિએશન} & માહિતી મોકલવા/પ્રાપ્ત કરવાના ઇન્કારને રોકવું & ડિજિટલ
સિગ્નેચર, ઓડિટ ટ્રેઇલ્સ \\
\end{longtable}
}

\textbf{સામાન્ય સુરક્ષા ધમકીઓ:}

\begin{itemize}
\tightlist
\item
  \textbf{માલવેર}: વાયરસ, વોર્મ્સ, ટ્રોજન, રેન્સમવેર
\item
  \textbf{સોશિયલ એન્જિનિયરિંગ}: ફિશિંગ, પ્રીટેક્સ્ટિંગ
\item
  \textbf{મેન-ઇન-ધ-મિડલ એટેક}: વાતચીતને અવરોધવી
\item
  \textbf{ડિનાયલ-ઓફ-સર્વિસ}: કાયદેસર એક્સેસને રોકવી
\end{itemize}

\end{solutionbox}
\begin{mnemonicbox}
``CIA'' - Confidentiality, Integrity, Availability

\end{mnemonicbox}
\subsection*{પ્રશ્ન 5(અ) OR [3
ગુણ]}\label{uxaaauxab0uxab6uxaa8-5uxa85-or-3-uxa97uxaa3}

\textbf{E1 કેરિયર સિસ્ટમ સમજાવો.}

\begin{solutionbox}

\textbf{E1 કેરિયર સિસ્ટમ:}

{\def\LTcaptype{none} % do not increment counter
\begin{longtable}[]{@{}ll@{}}
\toprule\noalign{}
લક્ષણ & સ્પેસિફિકેશન \\
\midrule\noalign{}
\endhead
\bottomrule\noalign{}
\endlastfoot
\textbf{ડેટા રેટ} & 2.048 Mbps \\
\textbf{ચેનલ} & 32 ટાઇમ સ્લોટ (30 વોઇસ + 2 સિગ્નલિંગ) \\
\textbf{વોઇસ સેમ્પલિંગ} & 8000 સેમ્પલ/સેકન્ડ \\
\textbf{સેમ્પલ સાઇઝ} & 8 બિટ્સ પ્રતિ સેમ્પલ \\
\textbf{ફ્રેમ સાઇઝ} & 256 બિટ્સ (32\times8) \\
\textbf{ફ્રેમ રેટ} & 8000 ફ્રેમ/સેકન્ડ \\
\end{longtable}
}

\textbf{E1 ફ્રેમ સ્ટ્રક્ચર:}

\begin{verbatim}
|{{-}{-}{-}{-}{-}{-}{-}{-}{-}{-}{-}{-}{-}{-}{-}{-}{-} E1 Frame (256 bits) {-}{-}{-}{-}{-}{-}{-}{-}{-}{-}{-}{-}{-}{-}{-}{-}{-}|}
| TS0 | TS1 | TS2 | ... | TS15 | TS16 | TS17 | ... | TS31 |
|  8  |  8  |  8  | ... |  8   |  8   |  8   | ... |  8   |
\end{verbatim}

\textbf{સ્પેશિયલ ટાઇમ સ્લોટ:}

\begin{itemize}
\tightlist
\item
  \textbf{TS0}: ફ્રેમ એલાઇનમેન્ટ સિગ્નલ
\item
  \textbf{TS16}: સિગ્નલિંગ ચેનલ
\end{itemize}

\end{solutionbox}
\begin{mnemonicbox}
``E1-32-8-8'' - E1 has 32 channels, 8 bits, 8kHz

\end{mnemonicbox}
\subsection*{પ્રશ્ન 5(બ) OR [4
ગુણ]}\label{uxaaauxab0uxab6uxaa8-5uxaac-or-4-uxa97uxaa3}

\textbf{ફ્રીક્વન્સી ડિવિઝન મલ્ટિપ્લેક્સિંગ ટેકનિક (FDM) ને વિગતવાર સમજાવો.}

\begin{solutionbox}

\textbf{ફ્રીક્વન્સી ડિવિઝન મલ્ટિપ્લેક્સિંગ (FDM):}

\begin{verbatim}
flowchart LR
    A[સિગ્નલ 1] {-{-} B[મોડ્યુલેટર 1nf1]}
    C[સિગ્નલ 2] {-{-} D[મોડ્યુલેટર 2nf2]}
    E[સિગ્નલ 3] {-{-} F[મોડ્યુલેટર 3nf3]}
    G[સિગ્નલ 4] {-{-} H[મોડ્યુલેટર 4nf4]}
    B {-{-} I[કમ્બાઇનર/મિક્સર]}
    D {-{-} I}
    F {-{-} I}
    H {-{-} I}
    I {-{-} J[ટ્રાન્સમિશન ચેનલ]}
    J {-{-} K[ફિલ્ટર્સ/સેપરેટર્સ]}
    K {-{-} L[ડિમોડ્યુલેટર 1nf1]}
    K {-{-} M[ડિમોડ્યુલેટર 2nf2]}
    K {-{-} N[ડિમોડ્યુલેટર 3nf3]}
    K {-{-} O[ડિમોડ્યુલેટર 4nf4]}
    L {-{-} P[સિગ્નલ 1]}
    M {-{-} Q[સિગ્નલ 2]}
    N {-{-} R[સિગ્નલ 3]}
    O {-{-} S[સિગ્નલ 4]}
\end{verbatim}

\textbf{FDM લક્ષણોનું કોષ્ટક:}

{\def\LTcaptype{none} % do not increment counter
\begin{longtable}[]{@{}
  >{\raggedright\arraybackslash}p{(\linewidth - 2\tabcolsep) * \real{0.5000}}
  >{\raggedright\arraybackslash}p{(\linewidth - 2\tabcolsep) * \real{0.5000}}@{}}
\toprule\noalign{}
\begin{minipage}[b]{\linewidth}\raggedright
લક્ષણ
\end{minipage} & \begin{minipage}[b]{\linewidth}\raggedright
વર્ણન
\end{minipage} \\
\midrule\noalign{}
\endhead
\bottomrule\noalign{}
\endlastfoot
\textbf{સિદ્ધાંત} & બહુવિધ સિગ્નલ અલગ-અલગ ફ્રીક્વન્સી બેન્ડનો ઉપયોગ કરીને એક ચેનલ
શેર કરે છે \\
\textbf{ગાર્ડ બેન્ડ} & ઇન્ટરફેરન્સને રોકવા માટે ચેનલો વચ્ચે વપરાય ન હોય તેવા
ફ્રીક્વન્સી બેન્ડ \\
\textbf{ચેનલ બેન્ડવિડ્થ} & દરેક સિગ્નલને ચોક્કસ ફ્રીક્વન્સી રેન્જ ફાળવેલી હોય છે \\
\textbf{અમલીકરણ} & સિગ્નલને અલગ-અલગ ફ્રીક્વન્સી બેન્ડમાં શિફ્ટ કરવા માટે
મોડ્યુલેટર્સનો ઉપયોગ \\
\textbf{ઉપયોગો} & રેડિયો બ્રોડકાસ્ટિંગ, ટેલિવિઝન, કેબલ સિસ્ટમ \\
\end{longtable}
}

\textbf{FDM સ્પેક્ટ્રમ:}

\begin{verbatim}
  Power
    \^{}
    |    \_\_\_      \_\_\_      \_\_\_      \_\_\_
    |   /   {    /       /       /   }
    |  /     {  /       /       /     }
    | /       {/       /       /       }
    +{-{-}{-}{-}{-}{-}{-}{-}{-}{-}{-}{-}{-}{-}{-}{-}{-}{-}{-}{-}{-}{-}{-}{-}{-}{-}{-}{-}{-}{-}{-}{-}{-}{-}{-}{-}{-}{-}{-}{-}{-} Frequency}
        Ch1      Ch2      Ch3      Ch4
      |{{-}{-}{-}|{-}|{-}{-}{-}|{-}|{-}{-}{-}|{-}|{-}{-}{-}|}
             GB      GB      GB
\end{verbatim}

\end{solutionbox}
\begin{mnemonicbox}
``FROG'' - FRequencies Organized with Gaps

\end{mnemonicbox}
\subsection*{પ્રશ્ન 5(ક) OR [7
ગુણ]}\label{uxaaauxab0uxab6uxaa8-5uxa95-or-7-uxa97uxaa3}

\textbf{ઈન્ટરનેટ ઓફ થિંગ્સ (IoT) ના ખ્યાલ અને મુખ્ય લક્ષણો સમજાવો.}

\begin{solutionbox}

\textbf{ઈન્ટરનેટ ઓફ થિંગ્સ (IoT) ખ્યાલ:}

\begin{center}
\textbf{Mermaid Diagram (Code)}
\begin{verbatim}
{Shaded}
{Highlighting}[]
graph TD
    A[ઈન્ટરનેટ ઓફ થિંગ્સ] {-{-}{} B[કનેક્ટેડ ડિવાઇસીસ]}
    A {-{-}{} C[ડેટા કલેક્શન]}
    A {-{-}{} D[ડેટા એનાલિટિક્સ]}
    A {-{-}{} E[ઓટોમેશન]}
    B {-{-}{} F[સેન્સર્સ]}
    B {-{-}{} G[એક્ચ્યુએટર્સ]}
    C {-{-}{} H[ક્લાઉડ સ્ટોરેજ]}
    D {-{-}{} I[AI/મશીન લર્નિંગ]}
    E {-{-}{} J[સ્માર્ટ એપ્લિકેશન્સ]}
{Highlighting}
{Shaded}
\end{verbatim}
\end{center}

\textbf{IoTના મુખ્ય લક્ષણોનું કોષ્ટક:}

{\def\LTcaptype{none} % do not increment counter
\begin{longtable}[]{@{}ll@{}}
\toprule\noalign{}
લક્ષણ & વર્ણન \\
\midrule\noalign{}
\endhead
\bottomrule\noalign{}
\endlastfoot
\textbf{કનેક્ટિવિટી} & ડિવાઇસીસ ઇન્ટરનેટ અને એકબીજા સાથે જોડાયેલી \\
\textbf{ઇન્ટેલિજન્સ} & સ્માર્ટ પ્રોસેસિંગ, નિર્ણય લેવાની ક્ષમતાઓ \\
\textbf{સેન્સિંગ} & સેન્સર્સ દ્વારા પર્યાવરણમાંથી ડેટા એકત્રિત કરવો \\
\textbf{એક્સપ્રેસિંગ} & એક્ચ્યુએટર્સ દ્વારા કાર્યવાહી કરવી \\
\textbf{એનર્જી એફિશિયન્સી} & બેટરી-સંચાલિત ડિવાઇસીસ માટે ઓછી પાવર વપરાશ \\
\textbf{સિક્યોરિટી} & અનધિકૃત એક્સેસ અને હુમલાઓથી સુરક્ષા \\
\textbf{સ્કેલેબિલિટી} & નેટવર્કમાં વધુ ડિવાઇસીસ ઉમેરવાની ક્ષમતા \\
\end{longtable}
}

\textbf{IoT આર્કિટેક્ચર લેયર્સ:}

\begin{verbatim}
               +{-{-}{-}{-}{-}{-}{-}{-}{-}{-}{-}{-}{-}{-}{-}{-}{-}{-}+}
               |    Application   |
               +{-{-}{-}{-}{-}{-}{-}{-}{-}{-}{-}{-}{-}{-}{-}{-}{-}{-}+}
               |  Data Analytics  |
               +{-{-}{-}{-}{-}{-}{-}{-}{-}{-}{-}{-}{-}{-}{-}{-}{-}{-}+}
               |  Data Processing |
               +{-{-}{-}{-}{-}{-}{-}{-}{-}{-}{-}{-}{-}{-}{-}{-}{-}{-}+}
               |  Data Transport  |
               +{-{-}{-}{-}{-}{-}{-}{-}{-}{-}{-}{-}{-}{-}{-}{-}{-}{-}+}
               |    Perception    |
               +{-{-}{-}{-}{-}{-}{-}{-}{-}{-}{-}{-}{-}{-}{-}{-}{-}{-}+}
\end{verbatim}

\textbf{IoT એપ્લિકેશન્સ:}

\begin{itemize}
\tightlist
\item
  સ્માર્ટ હોમ અને બિલ્ડિંગ
\item
  હેલ્થકેર મોનિટરિંગ
\item
  ઇન્ડસ્ટ્રિયલ ઓટોમેશન
\item
  સ્માર્ટ સિટીઝ
\item
  એગ્રીકલ્ચર મોનિટરિંગ
\item
  સપ્લાય ચેઇન મેનેજમેન્ટ
\end{itemize}

\end{solutionbox}
\begin{mnemonicbox}
``CASED'' - Connected, Automated, Sensing,
Expressing, Data-driven

\end{mnemonicbox}

\end{document}
