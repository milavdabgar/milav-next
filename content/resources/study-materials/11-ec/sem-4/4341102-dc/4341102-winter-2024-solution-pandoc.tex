\documentclass[10pt,a4paper]{article}

% content/resources/templates/preamble.tex
\usepackage[margin=0.6in]{geometry}
\author{Milav Dabgar}
\usepackage{amsmath,amssymb,amsthm}
\usepackage{booktabs}
\usepackage{multirow}
\usepackage{xcolor}
\usepackage{tcolorbox}
\tcbuselibrary{breakable,skins}
\usepackage[colorlinks=true,linkcolor=blue]{hyperref}
\usepackage{titlesec}
\usepackage{enumitem}
\usepackage{tikz}
\usepackage{pgfplots}
\usepackage{circuitikz}
\usepackage[version=4]{mhchem}
\usepackage{longtable}
\usepackage{array}
\usepackage{float}
\usepackage{caption}
\usepackage{listings}

\lstset{
  basicstyle=\small\ttfamily,
  breaklines=true,
  breakatwhitespace=false,
  postbreak=\mbox{\textcolor{red}{$\hookrightarrow$}\space},
  float=false,
  numbers=left,
  numberstyle=\tiny\color{gray},
  numbersep=10pt,
  xleftmargin=2em,
  keywordstyle=\color{blue},
  commentstyle=\color{green!60!black},
  stringstyle=\color{purple},
  backgroundcolor=\color{gray!5},
  showstringspaces=false,
  tabsize=2,
  captionpos=b,
  keepspaces=true,
  columns=flexible
}

\pgfplotsset{compat=1.18}
\usetikzlibrary{shapes,arrows,positioning,calc,patterns,decorations.pathmorphing,decorations.markings,arrows.meta}

% Color scheme
\definecolor{headcolor}{RGB}{0,102,204}
\definecolor{keycolor}{RGB}{220,20,60}
\definecolor{solutioncolor}{RGB}{34,139,34}
\definecolor{mnemoniccolor}{RGB}{148,0,211}
\definecolor{codecolor}{RGB}{0,0,100}

% Spacing
\setlength{\parskip}{3pt}
\setlist[itemize]{nosep}
\setlist[enumerate]{nosep}

% Title formatting
\titleformat{\section}{\Large\bfseries\color{headcolor}}{\thesection}{1em}{}
\titleformat{\subsection}{\large\bfseries\color{headcolor}}{\thesubsection}{1em}{}

% Pandoc tightlist compatibility
\providecommand{\tightlist}{%
  \setlength{\itemsep}{0pt}\setlength{\parskip}{0pt}}

% Pandoc longtable compatibility
\newcounter{none}
\def\thenone{}


% content/resources/templates/english-boxes.tex
% This file is currently empty - it exists to maintain consistency with the import structure.
% Add custom environments here if needed in the future.


\begin{document}

\begin{center}
{\Huge\bfseries\color{headcolor} Subject Name Solutions}\\[5pt]
{\LARGE 4341102 -- Winter 2024}\\[3pt]
{\large Semester 1 Study Material}\\[3pt]
{\normalsize\textit{Detailed Solutions and Explanations}}
\end{center}

\vspace{10pt}

\subsection*{Question 1(a) [3 marks]}\label{q1a}

\textbf{Define Continuous time Signal and Discrete time Signal with Wave
form.}

\begin{solutionbox}


{\def\LTcaptype{none} % do not increment counter
\vspace{-5pt}
\captionof{table}{Comparison of Signal Types}
\vspace{-10pt}
\begin{longtable}[]{@{}
  >{\raggedright\arraybackslash}p{(\linewidth - 4\tabcolsep) * \real{0.3023}}
  >{\raggedright\arraybackslash}p{(\linewidth - 4\tabcolsep) * \real{0.2791}}
  >{\raggedright\arraybackslash}p{(\linewidth - 4\tabcolsep) * \real{0.4186}}@{}}
\toprule\noalign{}
\begin{minipage}[b]{\linewidth}\raggedright
Signal Type
\end{minipage} & \begin{minipage}[b]{\linewidth}\raggedright
Definition
\end{minipage} & \begin{minipage}[b]{\linewidth}\raggedright
Waveform Example
\end{minipage} \\
\midrule\noalign{}
\endhead
\bottomrule\noalign{}
\endlastfoot
\textbf{Continuous time Signal} & Signal defined for all time instants
with continuous values & Smooth, unbroken curve \\
\textbf{Discrete time Signal} & Signal defined only at specific time
instants with samples & Series of distinct points \\
\end{longtable}
}

\textbf{Diagram:}

\begin{center}
\textbf{Mermaid Diagram (Code)}
\begin{verbatim}
{Shaded}
{Highlighting}[]
graph TD
    subgraph Continuous
        A[Continuous Time Signal] {-{-}{} B["x(t)"]}
        B {-{-}{} C[Defined for all t]}
    end
    subgraph Discrete
        D[Discrete Time Signal] {-{-}{} E["x(n)"]}
        E {-{-}{} F[Defined for integer n]}
    end
{Highlighting}
{Shaded}
\end{verbatim}
\end{center}

\begin{itemize}
\tightlist
\item
  \textbf{Amplitude continuity}: In continuous signals, amplitude can
  take any value, while discrete signals have specific amplitude values
\item
  \textbf{Mathematical notation}: Continuous signals use x(t), discrete
  signals use x[n] or x(n)
\end{itemize}

\end{solutionbox}
\begin{mnemonicbox}
``CoSiDi'' - \textbf{Co}ntinuous \textbf{Si}gnals
flow like rivers, \textbf{Di}screte signals are like stepping stones

\end{mnemonicbox}
\subsection*{Question 1(b) [4 marks]}\label{q1b}

\textbf{Explain periodic and aperiodic signal.}

\begin{solutionbox}


{\def\LTcaptype{none} % do not increment counter
\vspace{-5pt}
\captionof{table}{Periodic vs.~Aperiodic Signals}
\vspace{-10pt}
\begin{longtable}[]{@{}
  >{\raggedright\arraybackslash}p{(\linewidth - 4\tabcolsep) * \real{0.2273}}
  >{\raggedright\arraybackslash}p{(\linewidth - 4\tabcolsep) * \real{0.3636}}
  >{\raggedright\arraybackslash}p{(\linewidth - 4\tabcolsep) * \real{0.4091}}@{}}
\toprule\noalign{}
\begin{minipage}[b]{\linewidth}\raggedright
Property
\end{minipage} & \begin{minipage}[b]{\linewidth}\raggedright
Periodic Signal
\end{minipage} & \begin{minipage}[b]{\linewidth}\raggedright
Aperiodic Signal
\end{minipage} \\
\midrule\noalign{}
\endhead
\bottomrule\noalign{}
\endlastfoot
\textbf{Definition} & Repeats exactly after fixed time interval & Does
not repeat or has infinite period \\
\textbf{Mathematical Expression} & x(t) = x(t + nT) for all t & x(t) \neq
x(t + T) for any T \\
\textbf{Energy/Power} & Infinite energy, finite power & Finite energy,
zero average power \\
\textbf{Examples} & Sine waves, square waves & Single pulse, damped
sinusoid \\
\end{longtable}
}

\textbf{Diagram:}

\begin{center}
\textbf{Mermaid Diagram (Code)}
\begin{verbatim}
{Shaded}
{Highlighting}[]
graph TD
    subgraph Periodic
        A["x(t) = x(t+T)"] {-{-}{} B[Repeats exactly]}
        B {-{-}{} C[Fundamental period T]}
    end
    subgraph Aperiodic
        D["x(t)  x(t+T)"] {-{-}{} E[Never repeats exactly]}
        E {-{-}{} F[No fundamental period]}
    end
{Highlighting}
{Shaded}
\end{verbatim}
\end{center}

\begin{itemize}
\tightlist
\item
  \textbf{Spectral property}: Periodic signals have discrete frequency
  components, aperiodic have continuous spectrum
\item
  \textbf{Fourier analysis}: Periodic signals use Fourier series,
  aperiodic use Fourier transform
\end{itemize}

\end{solutionbox}
\begin{mnemonicbox}
``PART'' - \textbf{P}eriodic signals \textbf{A}lways
\textbf{R}epeat in \textbf{T}ime

\end{mnemonicbox}
\subsection*{Question 1(c) [7 marks]}\label{q1c}

\textbf{Explain block diagram of digital communication system.}

\begin{solutionbox}

\textbf{Diagram: Digital Communication System}

\begin{verbatim}
flowchart LR
    A[Source] {-{-} B[Source Encoder]}
    B {-{-} C[Channel Encoder]}
    C {-{-} D[Digital Modulator]}
    D {-{-} E[Channel]}
    E {-{-} F[Digital Demodulator]}
    F {-{-} G[Channel Decoder]}
    G {-{-} H[Source Decoder]}
    H {-{-} I[Destination]}
\end{verbatim}


{\def\LTcaptype{none} % do not increment counter
\vspace{-5pt}
\captionof{table}{Functions of Digital Communication System Blocks}
\vspace{-10pt}
\begin{longtable}[]{@{}
  >{\raggedright\arraybackslash}p{(\linewidth - 4\tabcolsep) * \real{0.2692}}
  >{\raggedright\arraybackslash}p{(\linewidth - 4\tabcolsep) * \real{0.3846}}
  >{\raggedright\arraybackslash}p{(\linewidth - 4\tabcolsep) * \real{0.3462}}@{}}
\toprule\noalign{}
\begin{minipage}[b]{\linewidth}\raggedright
Block
\end{minipage} & \begin{minipage}[b]{\linewidth}\raggedright
Function
\end{minipage} & \begin{minipage}[b]{\linewidth}\raggedright
Example
\end{minipage} \\
\midrule\noalign{}
\endhead
\bottomrule\noalign{}
\endlastfoot
\textbf{Source} & Generates message to be transmitted & Microphone,
Keyboard \\
\textbf{Source Encoder} & Removes redundancy, compresses data & Huffman
coding, JPEG \\
\textbf{Channel Encoder} & Adds controlled redundancy for error
detection/correction & Hamming codes, CRC \\
\textbf{Digital Modulator} & Converts digital data to analog signals &
ASK, FSK, PSK \\
\textbf{Channel} & Medium that carries the signal & Wired, Wireless,
Optical fiber \\
\textbf{Digital Demodulator} & Converts received signal back to digital
& ASK, FSK, PSK demodulators \\
\textbf{Channel Decoder} & Detects/corrects errors using added
redundancy & Error correction circuits \\
\textbf{Source Decoder} & Reconstructs original message & Data
decompression \\
\end{longtable}
}

\begin{itemize}
\tightlist
\item
  \textbf{Advantage}: Noise immunity, secure transmission, multiplexing
  capability, integration with digital systems
\item
  \textbf{Key processes}: Sampling, quantization, coding,
  modulation/demodulation
\end{itemize}

\end{solutionbox}
\begin{mnemonicbox}
``SECMCDS'' - \textbf{S}ource \textbf{E}ncodes,
\textbf{C}hannel codes, \textbf{M}odulates, \textbf{C}hannel,
\textbf{D}emodulates, \textbf{S}ink receives

\end{mnemonicbox}
\subsection*{Question 1(c) OR [7
marks]}\label{q1c}

\textbf{Explain singularity functions.}

\begin{solutionbox}


{\def\LTcaptype{none} % do not increment counter
\vspace{-5pt}
\captionof{table}{Common Singularity Functions}
\vspace{-10pt}
\begin{longtable}[]{@{}
  >{\raggedright\arraybackslash}p{(\linewidth - 6\tabcolsep) * \real{0.1639}}
  >{\raggedright\arraybackslash}p{(\linewidth - 6\tabcolsep) * \real{0.4098}}
  >{\raggedright\arraybackslash}p{(\linewidth - 6\tabcolsep) * \real{0.1967}}
  >{\raggedright\arraybackslash}p{(\linewidth - 6\tabcolsep) * \real{0.2295}}@{}}
\toprule\noalign{}
\begin{minipage}[b]{\linewidth}\raggedright
Function
\end{minipage} & \begin{minipage}[b]{\linewidth}\raggedright
Mathematical Definition
\end{minipage} & \begin{minipage}[b]{\linewidth}\raggedright
Properties
\end{minipage} & \begin{minipage}[b]{\linewidth}\raggedright
Applications
\end{minipage} \\
\midrule\noalign{}
\endhead
\bottomrule\noalign{}
\endlastfoot
\textbf{Unit Step} & u(t) = 1 for t \geq 0, 0 for t \textless{} 0 &
Discontinuous at t=0 & Switch-on signals, Heaviside function \\
\textbf{Unit Impulse} & δ(t) = \infty for t = 0, 0 elsewhere, \intδ(t)dt = 1 &
Infinitely tall, zero-width & Impulse response, sampling \\
\textbf{Unit Ramp} & r(t) = t·u(t) & Continuous but not differentiable
at t=0 & Linear time functions \\
\textbf{Unit Parabola} & p(t) = (t^{2}/2)·u(t) & Second integral of unit
impulse & Acceleration to position \\
\end{longtable}
}

\textbf{Diagram:}

\begin{verbatim}
   \^{}
   |                    ┌────────────────
   |                    │ Unit Step
   |────────────────────┘
   |
   +{-{-}{-}{-}{-}{-}{-}{-}{-}{-}{-}{-}{-}{-}{-}{-}{-}{-}{-}{-}{-}{-}{-}{-}{-} t}
   |
   \^{                     /}
   |                    /
   |                   / Unit Ramp
   |─────────────────/
   |                /
   +{-{-}{-}{-}{-}{-}{-}{-}{-}{-}{-}{-}{-}{-}{-}/{-}{-}{-}{-}{-}{-}{-}{-}{-}{-}{-}{-} t}
   |              /
   \^{}
   |             .
   |             │ Unit Impulse
   |─────────────┼──────────────{ t}
   |             {}
\end{verbatim}

\begin{itemize}
\tightlist
\item
  \textbf{Integration relationship}: Each function is the integral of
  the previous one
\item
  \textbf{Mathematical toolkit}: Used to analyze complex systems by
  breaking into simpler components
\end{itemize}

\end{solutionbox}
\begin{mnemonicbox}
``SIPR'' - \textbf{S}tep \textbf{I}mpulse
\textbf{P}arable \textbf{R}amp - functions ordered by increasing order
of integration

\end{mnemonicbox}
\subsection*{Question 2(a) [3 marks]}\label{q2a}

\textbf{A signal carries 10 bit/signal elements. If 100 signal elements
sent per second. Find the bit rate.}

\begin{solutionbox}

\textbf{Solution:}

\begin{verbatim}
Bit Rate = Number of bits per signal element \times Number of signal elements per second
Bit Rate = 10 bits/signal element \times 100 signal elements/second
Bit Rate = 1000 bits/second = 1 kbps
\end{verbatim}

\textbf{Diagram:}

\begin{center}
\textbf{Mermaid Diagram (Code)}
\begin{verbatim}
{Shaded}
{Highlighting}[]
graph LR
    A[Signal Elements: 100/s] {-{-}{} B[Each Element: 10 bits]}
    B {-{-}{} C[Bit Rate = 1000 bits/s]}
{Highlighting}
{Shaded}
\end{verbatim}
\end{center}

\begin{itemize}
\tightlist
\item
  \textbf{Bit rate}: Number of bits transmitted per second (bps)
\item
  \textbf{Signal element}: Physical manifestation of one or more bits
\end{itemize}

\end{solutionbox}
\begin{mnemonicbox}
``BEE'' - \textbf{B}it rate equals \textbf{E}lements
times bits per \textbf{E}lement

\end{mnemonicbox}
\subsection*{Question 2(b) [4 marks]}\label{q2b}

\textbf{Explain Even and Odd signal.}

\begin{solutionbox}


{\def\LTcaptype{none} % do not increment counter
\vspace{-5pt}
\captionof{table}{Even vs.~Odd Signals}
\vspace{-10pt}
\begin{longtable}[]{@{}
  >{\raggedright\arraybackslash}p{(\linewidth - 4\tabcolsep) * \real{0.2857}}
  >{\raggedright\arraybackslash}p{(\linewidth - 4\tabcolsep) * \real{0.3714}}
  >{\raggedright\arraybackslash}p{(\linewidth - 4\tabcolsep) * \real{0.3429}}@{}}
\toprule\noalign{}
\begin{minipage}[b]{\linewidth}\raggedright
Property
\end{minipage} & \begin{minipage}[b]{\linewidth}\raggedright
Even Signal
\end{minipage} & \begin{minipage}[b]{\linewidth}\raggedright
Odd Signal
\end{minipage} \\
\midrule\noalign{}
\endhead
\bottomrule\noalign{}
\endlastfoot
\textbf{Definition} & f(-t) = f(t) & f(-t) = -f(t) \\
\textbf{Symmetry} & Mirror symmetry about y-axis & Origin symmetry
(rotational) \\
\textbf{Fourier Series} & Contains only cosine terms & Contains only
sine terms \\
\textbf{Examples} & Cosine, & t \\
\end{longtable}
}

\textbf{Diagram:}

\begin{verbatim}
  Even Signal             Odd Signal
     \^{                       \^{}}
     |                       |
     |     .{-.               |      /}
     |    /   {              |     /}
     |{-{-}{-}/{-}{-}{-}{-}{-}{-}{-}{-}{-}{-}       |{-}{-}{-}{-}/{-}{-}{-}{-}{-}{-}{-}{-}{-}}
     |  /       {            |   /      }
     | {                    |  /        }
     |                       | /          {}
\end{verbatim}

\begin{itemize}
\tightlist
\item
  \textbf{Decomposition}: Any signal can be decomposed as sum of even
  and odd components
\item
  \textbf{Even part}: f\_e(t) = [f(t) + f(-t)]/2
\item
  \textbf{Odd part}: f\_o(t) = [f(t) - f(-t)]/2
\end{itemize}

\end{solutionbox}
\begin{mnemonicbox}
``ESOM'' - \textbf{E}ven \textbf{S}ignals have mirror
symmetry, \textbf{O}dd signals flip when \textbf{M}irrored

\end{mnemonicbox}
\subsection*{Question 2(c) [7 marks]}\label{q2c}

\textbf{Explain the block diagram of ASK modulator and de-modulator with
waveform.}

\begin{solutionbox}

\textbf{ASK Modulator Diagram:}

\begin{verbatim}
flowchart LR
    A[Digital Input] {-{-} B[Product Modulator]}
    C[Carrier Generator fc] {-{-} B}
    B {-{-} D[ASK Output]}
\end{verbatim}

\textbf{ASK Demodulator Diagram:}

\begin{verbatim}
flowchart LR
    A[ASK Input] {-{-} B[Band{-}Pass Filter]}
    B {-{-} C[Envelope Detector]}
    C {-{-} D[Low{-}Pass Filter]}
    D {-{-} E[Comparator]}
    E {-{-} F[Digital Output]}
\end{verbatim}

\textbf{Waveform:}

\begin{verbatim}
Digital Input
   \_    \_\_     \_
  | |  |  |   | |
\_\_|\_|\_\_|  |\_\_\_|\_|\_\_\_

Carrier Signal
 /{/////////}

ASK Output
     /{/    ///}
    /    {  /      }
\_\_\_/      {/        \_\_\_}
\end{verbatim}


{\def\LTcaptype{none} % do not increment counter
\vspace{-5pt}
\captionof{table}{ASK Modulation and Demodulation Process}
\vspace{-10pt}
\begin{longtable}[]{@{}
  >{\raggedright\arraybackslash}p{(\linewidth - 4\tabcolsep) * \real{0.1915}}
  >{\raggedright\arraybackslash}p{(\linewidth - 4\tabcolsep) * \real{0.2128}}
  >{\raggedright\arraybackslash}p{(\linewidth - 4\tabcolsep) * \real{0.5957}}@{}}
\toprule\noalign{}
\begin{minipage}[b]{\linewidth}\raggedright
Process
\end{minipage} & \begin{minipage}[b]{\linewidth}\raggedright
Function
\end{minipage} & \begin{minipage}[b]{\linewidth}\raggedright
Mathematical Representation
\end{minipage} \\
\midrule\noalign{}
\endhead
\bottomrule\noalign{}
\endlastfoot
\textbf{Modulation} & Varies amplitude of carrier & s(t) =
A·m(t)·cos(2πf\_c·t) \\
\textbf{Filtering} & Removes noise outside band & Bandpass filter
centered at f\_c \\
\textbf{Detection} & Recovers envelope & Using diode and capacitor \\
\textbf{Decision} & Converts to digital & Threshold comparison \\
\end{longtable}
}

\begin{itemize}
\tightlist
\item
  \textbf{Binary ASK}: Carrier present for `1', absent for `0'
\item
  \textbf{Bandwidth}: Minimum BW = bit rate, typically twice bit rate
  used
\end{itemize}

\end{solutionbox}
\begin{mnemonicbox}
``AMPS'' - \textbf{A}SK \textbf{M}odulates carrier
\textbf{P}ower (amplitude) with digital \textbf{S}ignal

\end{mnemonicbox}
\subsection*{Question 2(a) OR [3
marks]}\label{q2a}

\textbf{A signal has a bit rate of 4000 bit/second and a baud rate of
1000 baud. How many data elements are carried by each signal element?}

\begin{solutionbox}

\textbf{Solution:}

\begin{verbatim}
Number of bits per signal element = Bit rate / Baud rate
Number of bits per signal element = 4000 bits/second / 1000 signal elements/second
Number of bits per signal element = 4 bits/signal element
\end{verbatim}

\textbf{Diagram:}

\begin{center}
\textbf{Mermaid Diagram (Code)}
\begin{verbatim}
{Shaded}
{Highlighting}[]
graph LR
    A[Bit Rate: 4000 bps] {-{-}{} C[Divide]}
    B[Baud Rate: 1000 baud] {-{-}{} C}
    C {-{-}{} D[4 bits/signal element]}
{Highlighting}
{Shaded}
\end{verbatim}
\end{center}

\begin{itemize}
\tightlist
\item
  \textbf{Bit rate}: Data transmission speed in bits per second
\item
  \textbf{Baud rate}: Number of signal elements (symbols) per second
\end{itemize}

\end{solutionbox}
\begin{mnemonicbox}
``BBR'' - \textbf{B}its per symbol equals
\textbf{B}it rate divided by \textbf{B}aud \textbf{R}ate

\end{mnemonicbox}
\subsection*{Question 2(b) OR [4
marks]}\label{q2b}

\textbf{Discuss the various communication channels characteristics.}

\begin{solutionbox}


{\def\LTcaptype{none} % do not increment counter
\vspace{-5pt}
\captionof{table}{Communication Channel Characteristics}
\vspace{-10pt}
\begin{longtable}[]{@{}
  >{\raggedright\arraybackslash}p{(\linewidth - 4\tabcolsep) * \real{0.3902}}
  >{\raggedright\arraybackslash}p{(\linewidth - 4\tabcolsep) * \real{0.3171}}
  >{\raggedright\arraybackslash}p{(\linewidth - 4\tabcolsep) * \real{0.2927}}@{}}
\toprule\noalign{}
\begin{minipage}[b]{\linewidth}\raggedright
Characteristic
\end{minipage} & \begin{minipage}[b]{\linewidth}\raggedright
Description
\end{minipage} & \begin{minipage}[b]{\linewidth}\raggedright
Importance
\end{minipage} \\
\midrule\noalign{}
\endhead
\bottomrule\noalign{}
\endlastfoot
\textbf{Bandwidth} & Range of frequencies channel can transmit &
Determines maximum data rate \\
\textbf{Noise} & Unwanted signals that corrupt transmission & Affects
signal quality and error rate \\
\textbf{Attenuation} & Loss of signal strength during transmission &
Limits transmission distance \\
\textbf{Distortion} & Change in signal shape/timing & Causes intersymbol
interference \\
\textbf{Channel capacity} & Maximum data rate with arbitrary small error
& Given by Shannon's theorem \\
\end{longtable}
}

\textbf{Diagram:}

\begin{center}
\textbf{Mermaid Diagram (Code)}
\begin{verbatim}
{Shaded}
{Highlighting}[]
graph TD
    A[Channel Characteristics] {-{-}{} B[Bandwidth]}
    A {-{-}{} C[Noise]}
    A {-{-}{} D[Attenuation]}
    A {-{-}{} E[Distortion]}
    A {-{-}{} F[Channel Capacity]}
    C {-{-}{} G[SNR]}
    B {-{-}{} H[Data Rate]}
    F {-{-}{} H}
{Highlighting}
{Shaded}
\end{verbatim}
\end{center}

\begin{itemize}
\tightlist
\item
  \textbf{SNR (Signal-to-Noise Ratio)}: Ratio of signal power to noise
  power
\item
  \textbf{Channel capacity}: C = B·log_{2}(1+SNR), where B is bandwidth
\end{itemize}

\end{solutionbox}
\begin{mnemonicbox}
``BAND-C'' - \textbf{B}andwidth,
\textbf{A}ttenuation, \textbf{N}oise, \textbf{D}istortion define
\textbf{C}apacity

\end{mnemonicbox}
\subsection*{Question 2(c) OR [7
marks]}\label{q2c}

\textbf{Compare ASK, FSK and PSK.}

\begin{solutionbox}


{\def\LTcaptype{none} % do not increment counter
\vspace{-5pt}
\captionof{table}{Comparison of Digital Modulation Techniques}
\vspace{-10pt}
\begin{longtable}[]{@{}
  >{\raggedright\arraybackslash}p{(\linewidth - 6\tabcolsep) * \real{0.4231}}
  >{\raggedright\arraybackslash}p{(\linewidth - 6\tabcolsep) * \real{0.1923}}
  >{\raggedright\arraybackslash}p{(\linewidth - 6\tabcolsep) * \real{0.1923}}
  >{\raggedright\arraybackslash}p{(\linewidth - 6\tabcolsep) * \real{0.1923}}@{}}
\toprule\noalign{}
\begin{minipage}[b]{\linewidth}\raggedright
Parameter
\end{minipage} & \begin{minipage}[b]{\linewidth}\raggedright
ASK
\end{minipage} & \begin{minipage}[b]{\linewidth}\raggedright
FSK
\end{minipage} & \begin{minipage}[b]{\linewidth}\raggedright
PSK
\end{minipage} \\
\midrule\noalign{}
\endhead
\bottomrule\noalign{}
\endlastfoot
\textbf{Principle} & Varies amplitude & Varies frequency & Varies
phase \\
\textbf{Mathematical Expression} & s(t) = A·m(t)·cos(2πf\_c·t) & s(t) =
A·cos(2π[f\_c+m(t)Δf]t) & s(t) = A·cos(2πf\_c·t+m(t)·π) \\
\textbf{Bandwidth} & r\_b (minimum) & 2(Δf+r\_b/2) & 2r\_b \\
\textbf{Power Efficiency} & Poor & Moderate & Good \\
\textbf{Noise Immunity} & Poor & Better & Best \\
\textbf{Implementation Complexity} & Simple & Moderate & Complex \\
\textbf{Applications} & Low-cost systems & Noise-prone environments &
High-performance systems \\
\end{longtable}
}

\textbf{Diagram:}

\begin{verbatim}
Digital Input:
   \_    \_\_     \_
  | |  |  |   | |
\_\_|\_|\_\_|  |\_\_\_|\_|\_\_\_

ASK:
     /{/    ///}
    /    {  /      }
\_\_\_/      {/        \_\_\_}

FSK:
 /{//      ///}
/      {    /      }
        {//        //}

PSK:
 /{//////////}
/  {  /    /    /  }
    {/    /    /    }
\end{verbatim}

\begin{itemize}
\tightlist
\item
  \textbf{Bit error rate}: PSK \textless{} FSK \textless{} ASK (PSK is
  best)
\item
  \textbf{Complexity order}: ASK \textless{} FSK \textless{} PSK (ASK is
  simplest)
\end{itemize}

\end{solutionbox}
\begin{mnemonicbox}
``AFP'' - \textbf{A}mplitude, \textbf{F}requency,
\textbf{P}hase are modified in ASK, FSK, PSK respectively

\end{mnemonicbox}
\subsection*{Question 3(a) [3 marks]}\label{q3a}

\textbf{Explain the working of FSK modulator with block diagram and
output Waveform.}

\begin{solutionbox}

\textbf{FSK Modulator Block Diagram:}

\begin{verbatim}
flowchart LR
    A[Digital Input] {-{-} B[Switch Controller]}
    B {-{-} C[Oscillator 1 {-} f1]}
    B {-{-} D[Oscillator 2 {-} f2]}
    C {-{-} E[Output]}
    D {-{-} E}
\end{verbatim}

\textbf{Waveform:}

\begin{verbatim}
Digital Input:
   \_    \_      
  | |  | |     
\_\_|\_|\_\_|\_|\_\_\_\_\_

FSK Output:
 /{//    ///}
/      {  /      }
        {/        }
        /{        /}
       /  {      /  }
\end{verbatim}


{\def\LTcaptype{none} % do not increment counter
\vspace{-5pt}
\captionof{table}{FSK Modulation Process}
\vspace{-10pt}
\begin{longtable}[]{@{}
  >{\raggedright\arraybackslash}p{(\linewidth - 2\tabcolsep) * \real{0.3158}}
  >{\raggedright\arraybackslash}p{(\linewidth - 2\tabcolsep) * \real{0.6842}}@{}}
\toprule\noalign{}
\begin{minipage}[b]{\linewidth}\raggedright
Step
\end{minipage} & \begin{minipage}[b]{\linewidth}\raggedright
Description
\end{minipage} \\
\midrule\noalign{}
\endhead
\bottomrule\noalign{}
\endlastfoot
\textbf{Digital Input} & Binary data (0s and 1s) \\
\textbf{Frequency Selection} & f_{1} for bit `1', f_{2} for bit `0' \\
\textbf{Waveform Generation} & s(t) = A·cos(2πf_{1}t) for bit `1', s(t) =
A·cos(2πf_{2}t) for bit `0' \\
\textbf{Output} & Continuous phase frequency-shifted signal \\
\end{longtable}
}

\begin{itemize}
\tightlist
\item
  \textbf{Binary FSK}: Uses two frequencies f_{1} and f_{2} separated by
  frequency deviation
\item
  \textbf{Advantage}: Better noise immunity than ASK
\end{itemize}

\end{solutionbox}
\begin{mnemonicbox}
``FAST'' - \textbf{F}requency \textbf{A}lternates
between \textbf{S}eparate \textbf{T}ones

\end{mnemonicbox}
\subsection*{Question 3(b) [4 marks]}\label{q3b}

\textbf{Draw the PSK modulation waveform for the sequence of
1010110110.}

\begin{solutionbox}

\textbf{BPSK Modulation for 1010110110:}

\begin{verbatim}
Digital Input:
   \_    \_   \_\_\_   \_ \_
  | |  | | |   | | | |
\_\_| |\_\_| |\_|   |\_| | |\_\_

Carrier Signal:
 /{//////////}

BPSK Output:
 /{// /// /// /}
      {/      /      /}
      /{/// //// /}
                      {/}
\end{verbatim}


{\def\LTcaptype{none} % do not increment counter
\vspace{-5pt}
\captionof{table}{BPSK Mapping}
\vspace{-10pt}
\begin{longtable}[]{@{}lll@{}}
\toprule\noalign{}
Bit & Phase & Interpretation \\
\midrule\noalign{}
\endhead
\bottomrule\noalign{}
\endlastfoot
\textbf{1} & 0^\circ & In-phase with carrier (positive) \\
\textbf{0} & 180^\circ & Out-of-phase with carrier (negative) \\
\end{longtable}
}

\textbf{Diagram:}

\begin{center}
\textbf{Mermaid Diagram (Code)}
\begin{verbatim}
{Shaded}
{Highlighting}[]
graph LR
    A[Bit Stream 1010110110] {-{-}{} B[Phase Mapping]}
    B {-{-}{} C[1=0^ Phase]}
    B {-{-}{} D[0=180^ Phase]}
    C {-{-}{} E[Modulated Signal]}
    D {-{-}{} E}
{Highlighting}
{Shaded}
\end{verbatim}
\end{center}

\begin{itemize}
\tightlist
\item
  \textbf{Phase shift}: 180^\circ transition at each bit change
\item
  \textbf{Constant amplitude}: Unlike ASK, amplitude remains constant
\end{itemize}

\end{solutionbox}
\begin{mnemonicbox}
``POPI'' - \textbf{P}hase \textbf{O}pposites for bit
\textbf{P}airs represent \textbf{I}nformation

\end{mnemonicbox}
\subsection*{Question 3(c) [7 marks]}\label{q3c}

\textbf{Draw the ASK and FSK modulation waveform for the sequence of
1100110101.}

\begin{solutionbox}

\textbf{Input Bit Sequence: 1100110101}

\textbf{ASK Modulation:}

\begin{verbatim}
Digital Input:
   \_\_    \_\_    \_  \_
  |  |  |  |  | || |
\_\_|  |\_\_|  |\_\_| ||\_|\_\_

ASK Output:
 /{///    ////  // /}
         {  /         /     }
         /\_\_\_\_\_\_\_\_\_\_\_\_\_\_      {\_}
\end{verbatim}

\textbf{FSK Modulation:}

\begin{verbatim}
Digital Input:
   \_\_    \_\_    \_  \_
  |  |  |  |  | || |
\_\_|  |\_\_|  |\_\_| ||\_|\_\_

FSK Output (f1=high, f0=low):
 /{///        ////      //    }
        {      /            /      /}
         {    /            /      /  }
          {//            /            }
           Higher freq     Higher freq   Higher freq
           for 1s          for 1s        for 1s
         
             Lower freq      Lower freq    Lower freq
             for 0s          for 0s        for 0s
\end{verbatim}


{\def\LTcaptype{none} % do not increment counter
\vspace{-5pt}
\captionof{table}{Comparison for the Sequence 1100110101}
\vspace{-10pt}
\begin{longtable}[]{@{}
  >{\raggedright\arraybackslash}p{(\linewidth - 6\tabcolsep) * \real{0.2154}}
  >{\raggedright\arraybackslash}p{(\linewidth - 6\tabcolsep) * \real{0.1692}}
  >{\raggedright\arraybackslash}p{(\linewidth - 6\tabcolsep) * \real{0.3077}}
  >{\raggedright\arraybackslash}p{(\linewidth - 6\tabcolsep) * \real{0.3077}}@{}}
\toprule\noalign{}
\begin{minipage}[b]{\linewidth}\raggedright
Bit Position
\end{minipage} & \begin{minipage}[b]{\linewidth}\raggedright
Bit Value
\end{minipage} & \begin{minipage}[b]{\linewidth}\raggedright
ASK Representation
\end{minipage} & \begin{minipage}[b]{\linewidth}\raggedright
FSK Representation
\end{minipage} \\
\midrule\noalign{}
\endhead
\bottomrule\noalign{}
\endlastfoot
\textbf{1-2} & 11 & Carrier present & Higher frequency \\
\textbf{3-4} & 00 & Carrier absent & Lower frequency \\
\textbf{5-7} & 110 & Carrier present/absent & Higher/lower frequency \\
\textbf{8-10} & 101 & Carrier present/absent/present &
Higher/lower/higher frequency \\
\end{longtable}
}

\begin{itemize}
\tightlist
\item
  \textbf{ASK modulation}: Simple on-off keying where carrier is present
  for `1' and absent for `0'
\item
  \textbf{FSK modulation}: Frequency shifts between two distinct values
  based on bit value
\end{itemize}

\end{solutionbox}
\begin{mnemonicbox}
``AFOP'' - \textbf{A}SK switches carrier
\textbf{O}n-\textbf{O}ff while \textbf{F}SK shifts between
\textbf{P}airs of frequencies

\end{mnemonicbox}
\subsection*{Question 3(a) OR [3
marks]}\label{q3a}

\textbf{Explain the working of MSK modulator with block diagram and
output Waveform.}

\begin{solutionbox}

\textbf{MSK Modulator Block Diagram:}

\begin{verbatim}
flowchart LR
    A[Digital Input] {-{-} B[Serial to Parallel]}
    B {-{-} C[Even Bits]}
    B {-{-} D[Odd Bits]}
    C {-{-} E[Cos Modulator]}
    D {-{-} F[Sin Modulator]}
    G[90^ Phase Shifter] {-{-} F}
    H[Carrier Generator] {-{-} E}
    H {-{-} G}
    E {-{-} I[Combiner]}
    F {-{-} I}
    I {-{-} J[MSK Output]}
\end{verbatim}

\textbf{Waveform:}

\begin{verbatim}
Digital Input:
   \_    \_      
  | |  | |     
\_\_|\_|\_\_|\_|\_\_\_\_\_

MSK Output:
  \_{-\_     \_{-}\_  }
 /   {   /    }
/     {\_/     \_}
       \_{-\_     \_{-}}
      /   {   /  }
     /     {\_/    }
\end{verbatim}


{\def\LTcaptype{none} % do not increment counter
\vspace{-5pt}
\captionof{table}{MSK Modulation Process}
\vspace{-10pt}
\begin{longtable}[]{@{}
  >{\raggedright\arraybackslash}p{(\linewidth - 2\tabcolsep) * \real{0.5517}}
  >{\raggedright\arraybackslash}p{(\linewidth - 2\tabcolsep) * \real{0.4483}}@{}}
\toprule\noalign{}
\begin{minipage}[b]{\linewidth}\raggedright
Characteristic
\end{minipage} & \begin{minipage}[b]{\linewidth}\raggedright
Description
\end{minipage} \\
\midrule\noalign{}
\endhead
\bottomrule\noalign{}
\endlastfoot
\textbf{Principle} & Special case of OQPSK with sinusoidal pulse
shaping \\
\textbf{Phase Continuity} & Ensures smooth phase transitions (no abrupt
phase changes) \\
\textbf{Frequency Deviation} & \pm0.25 bit rate from carrier frequency \\
\textbf{Bandwidth Efficiency} & Better than conventional FSK \\
\end{longtable}
}

\begin{itemize}
\tightlist
\item
  \textbf{Phase continuity}: Key advantage - reduces bandwidth compared
  to FSK
\item
  \textbf{Constant envelope}: Resistant to non-linear amplification
\end{itemize}

\end{solutionbox}
\begin{mnemonicbox}
``MCPS'' - \textbf{M}SK ensures \textbf{C}ontinuous
\textbf{P}hase \textbf{S}hifts

\end{mnemonicbox}
\subsection*{Question 3(b) OR [4
marks]}\label{q3b}

\textbf{Draw the constellation diagram of 8-PSK and 16-QAM.}

\begin{solutionbox}

\textbf{8-PSK Constellation Diagram:}

\begin{verbatim}
       001 * * 000
           {|/}
    010 *{-{-}{-}+{-}{-}{-}* 111}
           /|{}
       011 * * 101
           100
\end{verbatim}

\textbf{16-QAM Constellation Diagram:}

\begin{verbatim}
    *   *   *   *
    
    *   *   *   *
    
    *   *   *   *
    
    *   *   *   *
\end{verbatim}


{\def\LTcaptype{none} % do not increment counter
\vspace{-5pt}
\captionof{table}{Comparison of Constellation Diagrams}
\vspace{-10pt}
\begin{longtable}[]{@{}lll@{}}
\toprule\noalign{}
Parameter & 8-PSK & 16-QAM \\
\midrule\noalign{}
\endhead
\bottomrule\noalign{}
\endlastfoot
\textbf{Bits per Symbol} & 3 bits & 4 bits \\
\textbf{Symbol Positions} & 8 points on circle & 16 points in grid \\
\textbf{Amplitude Levels} & 1 (constant) & 3 (variable) \\
\textbf{Phase Angles} & 8 angles (45^\circ apart) & 12 angles \\
\textbf{Error Sensitivity} & Moderate & Higher than 8-PSK \\
\textbf{Spectral Efficiency} & 3 bits/Hz & 4 bits/Hz \\
\end{longtable}
}

\begin{itemize}
\tightlist
\item
  \textbf{8-PSK}: Points equally spaced around circle with constant
  amplitude
\item
  \textbf{16-QAM}: Points arranged in square grid with different
  amplitudes and phases
\end{itemize}

\end{solutionbox}
\begin{mnemonicbox}
``CEPA'' - \textbf{C}onstellation points in PSK have
\textbf{E}qual amplitudes but different \textbf{P}hases, QAM has both
\textbf{A}mplitude and phase variations

\end{mnemonicbox}
\subsection*{Question 3(c) OR [7
marks]}\label{q3c}

\textbf{Draw BPSK and QPSK modulation waveform for 1010101011.}

\begin{solutionbox}

\textbf{Input Bit Sequence: 1010101011}

\textbf{BPSK Modulation:}

\begin{verbatim}
Digital Input:
   \_ \_ \_ \_ \_ \_ \_ \_
  | | | | | | | | |
\_\_| |\_| |\_| |\_| |\_| |\_\_

BPSK Output:
 /{// /// /// /// /}
      {/      /      /      /}
      /{/// //// ////}
\end{verbatim}

\textbf{QPSK Modulation (Grouping bits: 10,10,10,10,11):}

\begin{verbatim}
Grouped Bits:
   10    10    10    10    11
   
I{-channel (odd bits):}
   \_     \_     \_     \_     \_
  | |   | |   | |   | |   | |
\_\_| |\_\_\_| |\_\_\_| |\_\_\_| |\_\_\_| |\_\_

Q{-channel (even bits):}
    \_     \_     \_     \_      
   | |   | |   | |   | |    |
\_\_\_| |\_\_\_| |\_\_\_| |\_\_\_| |\_\_\_\_|

QPSK Output:
 /{  /  /  /  /}
/  {/  /  /  /  }
    Phase    Phase   Different 
    00       00      phase for 11
\end{verbatim}


{\def\LTcaptype{none} % do not increment counter
\vspace{-5pt}
\captionof{table}{BPSK vs.~QPSK for 1010101011}
\vspace{-10pt}
\begin{longtable}[]{@{}lll@{}}
\toprule\noalign{}
Characteristic & BPSK & QPSK \\
\midrule\noalign{}
\endhead
\bottomrule\noalign{}
\endlastfoot
\textbf{Bits per symbol} & 1 & 2 \\
\textbf{Number of symbols} & 10 & 5 \\
\textbf{Symbol rate} & Same as bit rate & Half of bit rate \\
\textbf{Bandwidth efficiency} & 1 bit/Hz & 2 bits/Hz \\
\textbf{Phase states} & 2 (0^\circ, 180^\circ) & 4 (45^\circ, 135^\circ, 225^\circ, 315^\circ) \\
\end{longtable}
}

\begin{itemize}
\tightlist
\item
  \textbf{BPSK}: Each bit causes a potential 180^\circ phase shift
\item
  \textbf{QPSK}: Processes two bits at once, uses four phase states
\end{itemize}

\end{solutionbox}
\begin{mnemonicbox}
``BQSE'' - \textbf{B}PSK takes \textbf{1} bit while
\textbf{Q}PSK takes \textbf{2} bits, doubling \textbf{S}pectral
\textbf{E}fficiency

\end{mnemonicbox}
\subsection*{Question 4(a) [3 marks]}\label{q4a}

\textbf{Encode the data using Shanon Fano code for below probability
sequence. P = \{ 0.30, 0.25, 0.20, 0.12, 0.08, 0.05\}}

\begin{solutionbox}


{\def\LTcaptype{none} % do not increment counter
\vspace{-5pt}
\captionof{table}{Shannon-Fano Coding Process}
\vspace{-10pt}
\begin{longtable}[]{@{}llll@{}}
\toprule\noalign{}
Symbol & Probability & Division Steps & Shannon-Fano Code \\
\midrule\noalign{}
\endhead
\bottomrule\noalign{}
\endlastfoot
\textbf{A} & 0.30 & Top Group & 0 \\
\textbf{B} & 0.25 & Top Group & 10 \\
\textbf{C} & 0.20 & Bottom Group & 110 \\
\textbf{D} & 0.12 & Bottom Group & 1110 \\
\textbf{E} & 0.08 & Bottom Group & 1111 0 \\
\textbf{F} & 0.05 & Bottom Group & 1111 1 \\
\end{longtable}
}

\textbf{Diagram:}

\begin{center}
\textbf{Mermaid Diagram (Code)}
\begin{verbatim}
{Shaded}
{Highlighting}[]
graph TD
    A[Symbols] {-{-}{} B[A:0.30, B:0.25, C:0.20, D:0.12, E:0.08, F:0.05]}
    B {-{-}{} C[A:0.30, B:0.25]}
    B {-{-}{} D[C:0.20, D:0.12, E:0.08, F:0.05]}
    C {-{-}{} E[A:0.30]}
    C {-{-}{} F[B:0.25]}
    D {-{-}{} G[C:0.20, D:0.12]}
    D {-{-}{} H[E:0.08, F:0.05]}
    G {-{-}{} I[C:0.20]}
    G {-{-}{} J[D:0.12]}
    H {-{-}{} K[E:0.08]}
    H {-{-}{} L[F:0.05]}
    E {-{-}{} M[Code: 0]}
    F {-{-}{} N[Code: 10]}
    I {-{-}{} O[Code: 110]}
    J {-{-}{} P[Code: 1110]}
    K {-{-}{} Q[Code: 11110]}
    L {-{-}{} R[Code: 11111]}
{Highlighting}
{Shaded}
\end{verbatim}
\end{center}

\begin{itemize}
\tightlist
\item
  \textbf{Shannon-Fano algorithm}: Recursively divide symbols into two
  groups with nearly equal probabilities
\item
  \textbf{Code efficiency}: Not always optimal but generally good
  compression
\end{itemize}

\end{solutionbox}
\begin{mnemonicbox}
``SPDF'' - \textbf{S}plit \textbf{P}robabilities and
assign \textbf{D}igits based on \textbf{F}requency

\end{mnemonicbox}
\subsection*{Question 4(b) [4 marks]}\label{q4b}

\textbf{Explain Hamming code.}

\begin{solutionbox}


{\def\LTcaptype{none} % do not increment counter
\vspace{-5pt}
\captionof{table}{Hamming Code Properties}
\vspace{-10pt}
\begin{longtable}[]{@{}
  >{\raggedright\arraybackslash}p{(\linewidth - 2\tabcolsep) * \real{0.4348}}
  >{\raggedright\arraybackslash}p{(\linewidth - 2\tabcolsep) * \real{0.5652}}@{}}
\toprule\noalign{}
\begin{minipage}[b]{\linewidth}\raggedright
Property
\end{minipage} & \begin{minipage}[b]{\linewidth}\raggedright
Description
\end{minipage} \\
\midrule\noalign{}
\endhead
\bottomrule\noalign{}
\endlastfoot
\textbf{Type} & Linear error-correcting code \\
\textbf{Error Detection} & Can detect up to 2 bit errors \\
\textbf{Error Correction} & Can correct single bit errors \\
\textbf{Parity Bits (r)} & For n data bits: 2\^{}r \geq n + r + 1 \\
\textbf{Code Structure} & Systematic: message bits + parity bits \\
\textbf{Positions of Parity Bits} & Powers of 2: positions 1, 2, 4, 8,
16\ldots{} \\
\end{longtable}
}

\textbf{Diagram:}

\begin{center}
\textbf{Mermaid Diagram (Code)}
\begin{verbatim}
{Shaded}
{Highlighting}[]
graph LR
    A[Hamming Code] {-{-}{} B[Parity Bits]}
    A {-{-}{} C[Data Bits]}
    B {-{-}{} D[Position 1]}
    B {-{-}{} E[Position 2]}
    B {-{-}{} F[Position 4]}
    B {-{-}{} G[Position 8]}
    A {-{-}{} H["Example: Hamming(7,4)"]}
    H {-{-}{} I[4 data bits + 3 parity bits]}
{Highlighting}
{Shaded}
\end{verbatim}
\end{center}

\begin{itemize}
\tightlist
\item
  \textbf{Encoding}: Calculate parity bits to ensure specific bit
  positions have even/odd parity
\item
  \textbf{Decoding}: Calculate syndrome to determine error position
\end{itemize}

\end{solutionbox}
\begin{mnemonicbox}
``PSEC'' - \textbf{P}arity bits in \textbf{P}ower of
2 positions \textbf{S}ystematically \textbf{E}nable error
\textbf{C}orrection

\end{mnemonicbox}
\subsection*{Question 4(c) [7 marks]}\label{q4c}

\textbf{Compare TDMA and FDMA.}

\begin{solutionbox}


{\def\LTcaptype{none} % do not increment counter
\vspace{-5pt}
\captionof{table}{Comparison of TDMA and FDMA}
\vspace{-10pt}
\begin{longtable}[]{@{}
  >{\raggedright\arraybackslash}p{(\linewidth - 4\tabcolsep) * \real{0.4783}}
  >{\raggedright\arraybackslash}p{(\linewidth - 4\tabcolsep) * \real{0.2609}}
  >{\raggedright\arraybackslash}p{(\linewidth - 4\tabcolsep) * \real{0.2609}}@{}}
\toprule\noalign{}
\begin{minipage}[b]{\linewidth}\raggedright
Parameter
\end{minipage} & \begin{minipage}[b]{\linewidth}\raggedright
TDMA
\end{minipage} & \begin{minipage}[b]{\linewidth}\raggedright
FDMA
\end{minipage} \\
\midrule\noalign{}
\endhead
\bottomrule\noalign{}
\endlastfoot
\textbf{Basic Principle} & Divides time into slots & Divides frequency
into channels \\
\textbf{Resource Allocation} & Each user gets full bandwidth for short
time & Each user gets narrow bandwidth for entire time \\
\textbf{Guard Time/Band} & Requires guard time between slots & Requires
guard bands between channels \\
\textbf{Synchronization} & Critical (timing-dependent) & Not required
(frequency separation) \\
\textbf{Efficiency} & Better for bursty data & Better for continuous
data \\
\textbf{Interference} & Less susceptible to interference & More
susceptible to adjacent channel interference \\
\textbf{Hardware Complexity} & Complex (needs buffering,
synchronization) & Simpler (fixed filters) \\
\textbf{Power Consumption} & Lower (transmitter on only during time
slot) & Higher (continuous transmission) \\
\textbf{Capacity} & Easily expanded by adding time slots & Limited by
available spectrum \\
\textbf{Applications} & GSM, DECT cordless phones & Analog cellular,
satellite systems \\
\end{longtable}
}

\textbf{Diagram:}

\begin{center}
\textbf{Mermaid Diagram (Code)}
\begin{verbatim}
{Shaded}
{Highlighting}[]
graph TD
    subgraph TDMA
        A[Time Slots] {-{-}{} A1[User 1]}
        A {-{-}{} A2[User 2]}
        A {-{-}{} A3[User 3]}
        A {-{-}{} A4[Guard Time]}
    end
    subgraph FDMA
        B[Frequency Bands] {-{-}{} B1[User 1]}
        B {-{-}{} B2[User 2]}
        B {-{-}{} B3[User 3]}
        B {-{-}{} B4[Guard Bands]}
    end
{Highlighting}
{Shaded}
\end{verbatim}
\end{center}

\begin{itemize}
\tightlist
\item
  \textbf{System flexibility}: TDMA can dynamically allocate slots, FDMA
  is fixed allocation
\item
  \textbf{Implementation}: TDMA requires digital technology, FDMA works
  with analog/digital
\end{itemize}

\end{solutionbox}
\begin{mnemonicbox}
``TIME-FREQ'' - \textbf{T}DMA splits
\textbf{I}ntervals of ti\textbf{ME}, \textbf{F}DMA splits
\textbf{R}anges of fr\textbf{EQ}uency

\end{mnemonicbox}
\subsection*{Question 4(a) OR [3
marks]}\label{q4a}

\textbf{Encode the data using Huffman code for below probability
sequence. P = \{ 0.4, 0.19, 0.16, 0.15, 0.1\}}

\begin{solutionbox}


{\def\LTcaptype{none} % do not increment counter
\vspace{-5pt}
\captionof{table}{Huffman Coding Process}
\vspace{-10pt}
\begin{longtable}[]{@{}lll@{}}
\toprule\noalign{}
Symbol & Probability & Huffman Code \\
\midrule\noalign{}
\endhead
\bottomrule\noalign{}
\endlastfoot
\textbf{A} & 0.40 & 0 \\
\textbf{B} & 0.19 & 10 \\
\textbf{C} & 0.16 & 110 \\
\textbf{D} & 0.15 & 111 \\
\textbf{E} & 0.10 & 110 \\
\end{longtable}
}

\textbf{Diagram:}

\begin{center}
\textbf{Mermaid Diagram (Code)}
\begin{verbatim}
{Shaded}
{Highlighting}[]
graph LR
    Z[Root: 1.0] {-{-}{} A[A: 0.4]}
    Z {-{-}{} Y[0.6]}
    Y {-{-}{} B[B: 0.19]}
    Y {-{-}{} X[0.41]}
    X {-{-}{} C[C: 0.16]}
    X {-{-}{} W[0.25]}
    W {-{-}{} D[D: 0.15]}
    W {-{-}{} E[E: 0.1]}
    A {-{-} 0 {-}{-}{} AA[Code: 0]}
    B {-{-} 1 {-}{-}{} BB[Code: 10]}
    C {-{-} 0 {-}{-}{} CC[Code: 110]}
    D {-{-} 0 {-}{-}{} DD[Code: 1110]}
    E {-{-} 1 {-}{-}{} EE[Code: 1111]}
{Highlighting}
{Shaded}
\end{verbatim}
\end{center}

\begin{itemize}
\tightlist
\item
  \textbf{Huffman algorithm}: Build a binary tree from bottom up,
  starting with least probable symbols
\item
  \textbf{Optimality}: Produces minimal average code length
\end{itemize}

\end{solutionbox}
\begin{mnemonicbox}
``HUMP'' - \textbf{H}uffman creates shorter codes for
\textbf{H}igher \textbf{P}robabilities

\end{mnemonicbox}
\subsection*{Question 4(b) OR [4
marks]}\label{q4b}

\textbf{Define Channel Capacity in terms of SNR and its importance in
communication.}

\begin{solutionbox}

\textbf{Shannon's Channel Capacity Formula:}

\begin{verbatim}
C = B \times log_{2}(1 + SNR)
\end{verbatim}

Where: -

C = Channel capacity in bits per second -

B = Bandwidth in Hz -

SNR = Signal-to-Noise Ratio


{\def\LTcaptype{none} % do not increment counter
\vspace{-5pt}
\captionof{table}{Channel Capacity Characteristics}
\vspace{-10pt}
\begin{longtable}[]{@{}
  >{\raggedright\arraybackslash}p{(\linewidth - 4\tabcolsep) * \real{0.2424}}
  >{\raggedright\arraybackslash}p{(\linewidth - 4\tabcolsep) * \real{0.3939}}
  >{\raggedright\arraybackslash}p{(\linewidth - 4\tabcolsep) * \real{0.3636}}@{}}
\toprule\noalign{}
\begin{minipage}[b]{\linewidth}\raggedright
Aspect
\end{minipage} & \begin{minipage}[b]{\linewidth}\raggedright
Description
\end{minipage} & \begin{minipage}[b]{\linewidth}\raggedright
Importance
\end{minipage} \\
\midrule\noalign{}
\endhead
\bottomrule\noalign{}
\endlastfoot
\textbf{Definition} & Maximum error-free data rate possible & Sets
fundamental limits \\
\textbf{SNR Dependence} & Logarithmically increases with SNR & Shows
diminishing returns of power \\
\textbf{Bandwidth Dependence} & Linearly increases with bandwidth &
Shows value of spectrum \\
\textbf{Theoretical Bound} & Can't exceed Shannon limit with any coding
& Guides system design \\
\end{longtable}
}

\textbf{Diagram:}

\begin{center}
\textbf{Mermaid Diagram (Code)}
\begin{verbatim}
{Shaded}
{Highlighting}[]
graph LR
    A[Channel Capacity] {-{-}{} B[Bandwidth B]}
    A {-{-}{} C[Signal{-}to{-}Noise Ratio]}
    B {-{-}{} D["C = B  log_{2}(1 + SNR)"]}
    C {-{-}{} D}
    D {-{-}{} E[Theoretical Maximum]}
    E {-{-}{} F[Error{-}free Communication]}
{Highlighting}
{Shaded}
\end{verbatim}
\end{center}

\begin{itemize}
\tightlist
\item
  \textbf{Shannon-Hartley theorem}: Establishes theoretical maximum data
  transfer rate
\item
  \textbf{Error probability}: Can be made arbitrarily small if data rate
  \textless{} channel capacity
\end{itemize}

\end{solutionbox}
\begin{mnemonicbox}
``SNRB'' - \textbf{S}hannon capacity depends on
\textbf{N}oise ratio and \textbf{B}andwidth

\end{mnemonicbox}
\subsection*{Question 4(c) OR [7
marks]}\label{q4c}

\textbf{Explain FDMA Technique in detail.}

\begin{solutionbox}

\textbf{FDMA (Frequency Division Multiple Access)}


{\def\LTcaptype{none} % do not increment counter
\vspace{-5pt}
\captionof{table}{FDMA System Characteristics}
\vspace{-10pt}
\begin{longtable}[]{@{}
  >{\raggedright\arraybackslash}p{(\linewidth - 4\tabcolsep) * \real{0.2286}}
  >{\raggedright\arraybackslash}p{(\linewidth - 4\tabcolsep) * \real{0.3714}}
  >{\raggedright\arraybackslash}p{(\linewidth - 4\tabcolsep) * \real{0.4000}}@{}}
\toprule\noalign{}
\begin{minipage}[b]{\linewidth}\raggedright
Aspect
\end{minipage} & \begin{minipage}[b]{\linewidth}\raggedright
Description
\end{minipage} & \begin{minipage}[b]{\linewidth}\raggedright
Significance
\end{minipage} \\
\midrule\noalign{}
\endhead
\bottomrule\noalign{}
\endlastfoot
\textbf{Basic Principle} & Divides available spectrum into channels &
Enables multiple simultaneous users \\
\textbf{Channel Allocation} & Fixed frequency bands per user &
Simplifies hardware design \\
\textbf{Guard Bands} & Frequency separation between channels & Prevents
adjacent channel interference \\
\textbf{Duplexing} & Often paired with FDD (separate Tx/Rx bands) &
Enables simultaneous two-way communication \\
\textbf{Bandwidth Utilization} & Each channel has fixed bandwidth &
Potentially inefficient for bursty data \\
\textbf{Intermodulation} & Products of multiple carriers & Requires
careful power amplifier design \\
\end{longtable}
}

\textbf{Diagram:}

\begin{center}
\textbf{Mermaid Diagram (Code)}
\begin{verbatim}
{Shaded}
{Highlighting}[]
graph TD
    A[Available Spectrum] {-{-}{} B[Guard Band]}
    A {-{-}{} C[User 1 Channel]}
    A {-{-}{} D[Guard Band]}
    A {-{-}{} E[User 2 Channel]}
    A {-{-}{} F[Guard Band]}
    A {-{-}{} G[User 3 Channel]}
    A {-{-}{} H[Guard Band]}
    A {-{-}{} I[User 4 Channel]}
{Highlighting}
{Shaded}
\end{verbatim}
\end{center}

\textbf{FDMA Implementation:}

\begin{verbatim}
   \^{}
   |
 F |  +{-{-}{-}{-}{-}+  +{-}{-}{-}{-}{-}+  +{-}{-}{-}{-}{-}+  +{-}{-}{-}{-}{-}+}
 r |  |User1|  |User2|  |User3|  |User4|
 e |  |     |  |     |  |     |  |     |
 q |  |     |  |     |  |     |  |     |
   |  +{-{-}{-}{-}{-}+  +{-}{-}{-}{-}{-}+  +{-}{-}{-}{-}{-}+  +{-}{-}{-}{-}{-}+}
   |
   |   Guard    Guard    Guard    
   |   Band     Band     Band     
   +{-{-}{-}{-}{-}{-}{-}{-}{-}{-}{-}{-}{-}{-}{-}{-}{-}{-}{-}{-}{-}{-}{-}{-}{-}{-}{-}{-}{-}{-}{-}{-}{-}{-}{-}{-}{-}{-}{-}{-}}
                     Time
\end{verbatim}

\begin{itemize}
\tightlist
\item
  \textbf{Implementation}: Relatively simple using bandpass filters
\item
  \textbf{Advantages}: No synchronization required, continuous
  transmission
\item
  \textbf{Disadvantages}: Spectrum inefficiency, limited flexibility
\end{itemize}

\end{solutionbox}
\begin{mnemonicbox}
``FDMA-CIGS'' - \textbf{F}requency \textbf{D}ivision
creates \textbf{M}ultiple \textbf{A}ccess through \textbf{C}hannels with
\textbf{I}ndividual \textbf{G}uard band \textbf{S}eparation

\end{mnemonicbox}
\subsection*{Question 5(a) [3 marks]}\label{q5a}

\textbf{Explain TDMA Access technique.}

\begin{solutionbox}

\textbf{TDMA (Time Division Multiple Access)}


{\def\LTcaptype{none} % do not increment counter
\vspace{-5pt}
\captionof{table}{TDMA Key Characteristics}
\vspace{-10pt}
\begin{longtable}[]{@{}ll@{}}
\toprule\noalign{}
Characteristic & Description \\
\midrule\noalign{}
\endhead
\bottomrule\noalign{}
\endlastfoot
\textbf{Basic Principle} & Divides time into frames and slots \\
\textbf{Resource Sharing} & Each user assigned specific time slot \\
\textbf{Guard Time} & Small time separation between slots \\
\textbf{Frame Structure} & Multiple slots form a complete frame \\
\textbf{Synchronization} & Timing reference required for all users \\
\end{longtable}
}

\textbf{Diagram:}

\begin{center}
\textbf{Mermaid Diagram (Code)}
\begin{verbatim}
{Shaded}
{Highlighting}[]
graph TD
    A[TDMA Frame] {-{-}{} B[Slot 1 {-} User 1]}
    A {-{-}{} C[Slot 2 {-} User 2]}
    A {-{-}{} D[Slot 3 {-} User 3]}
    A {-{-}{} E[Slot 4 {-} User 4]}
    A {-{-}{} F[Slot 5 {-} User 5]}
    A {-{-}{} G[Slot 6 {-} User 6]}
{Highlighting}
{Shaded}
\end{verbatim}
\end{center}

\begin{itemize}
\tightlist
\item
  \textbf{Digital implementation}: Requires ADC/DAC for analog signals
\item
  \textbf{Burst transmission}: Users transmit only during assigned slots
\end{itemize}

\end{solutionbox}
\begin{mnemonicbox}
``TIME'' - \textbf{T}ime slots \textbf{I}ndividually
\textbf{M}anaged for \textbf{E}ach user

\end{mnemonicbox}
\subsection*{Question 5(b) [4 marks]}\label{q5b}

\textbf{Explain E1 Career system.}

\begin{solutionbox}

\textbf{E1 Carrier System}


{\def\LTcaptype{none} % do not increment counter
\vspace{-5pt}
\captionof{table}{E1 Carrier System Specifications}
\vspace{-10pt}
\begin{longtable}[]{@{}lll@{}}
\toprule\noalign{}
Parameter & Specification & Details \\
\midrule\noalign{}
\endhead
\bottomrule\noalign{}
\endlastfoot
\textbf{Total Bit Rate} & 2.048 Mbps & European standard \\
\textbf{Number of Channels} & 32 time slots (0-31) & 30 voice + 2
control \\
\textbf{Voice Channels} & Time slots 1-15, 17-31 & Each 64 kbps \\
\textbf{Signaling Channel} & Time slot 16 & For channel signaling \\
\textbf{Frame Alignment} & Time slot 0 & Synchronization \\
\textbf{Frame Duration} & 125 μs & 8000 frames per second \\
\textbf{Sampling Rate} & 8 kHz & Follows Nyquist theorem \\
\end{longtable}
}

\textbf{Diagram:}

\begin{center}
\textbf{Mermaid Diagram (Code)}
\begin{verbatim}
{Shaded}
{Highlighting}[]
graph TD
    A[E1 Frame {- 2.048 Mbps] {-}{-}{} B[TS0: Framing]}
    A {-{-}{} C[TS1{-}15: Voice Channels]}
    A {-{-}{} D[TS16: Signaling]}
    A {-{-}{} E[TS17{-}31: Voice Channels]}
    B {-{-}{} F[Frame Alignment Signal]}
    D {-{-}{} G[Channel Associated Signaling]}
{Highlighting}
{Shaded}
\end{verbatim}
\end{center}

\begin{itemize}
\tightlist
\item
  \textbf{Multiplexing technique}: TDM (Time Division Multiplexing)
\item
  \textbf{PCM encoding}: 8-bit samples at 8 kHz sampling rate
\end{itemize}

\end{solutionbox}
\begin{mnemonicbox}
``E132'' - \textbf{E1} has \textbf{32} time slots
with \textbf{2}.048 Mbps

\end{mnemonicbox}
\subsection*{Question 5(c) [7 marks]}\label{q5c}

\textbf{Explain block diagram of Digital telephone exchange, elements of
hardware sub systems.}

\begin{solutionbox}

\textbf{Digital Telephone Exchange Block Diagram}

\begin{verbatim}
flowchart TD
    A[Digital Telephone Exchange] {-{-} B[DLU: Digital Line Unit]}
    A {-{-} C[LTG: Line/Trunk Group]}
    A {-{-} D[SN: Switching Network]}
    A {-{-} E[CP: Central Processor]}
    B {-{-} F[Interface to Subscribers]}
    C {-{-} G[Interface to Trunks]}
    D {-{-} H[Digital Switching]}
    E {-{-} I[System Control]}
\end{verbatim}


{\def\LTcaptype{none} % do not increment counter
\vspace{-5pt}
\captionof{table}{Hardware Subsystems of Digital Telephone Exchange}
\vspace{-10pt}
\begin{longtable}[]{@{}
  >{\raggedright\arraybackslash}p{(\linewidth - 4\tabcolsep) * \real{0.2973}}
  >{\raggedright\arraybackslash}p{(\linewidth - 4\tabcolsep) * \real{0.2703}}
  >{\raggedright\arraybackslash}p{(\linewidth - 4\tabcolsep) * \real{0.4324}}@{}}
\toprule\noalign{}
\begin{minipage}[b]{\linewidth}\raggedright
Subsystem
\end{minipage} & \begin{minipage}[b]{\linewidth}\raggedright
Function
\end{minipage} & \begin{minipage}[b]{\linewidth}\raggedright
Key Components
\end{minipage} \\
\midrule\noalign{}
\endhead
\bottomrule\noalign{}
\endlastfoot
\textbf{DLU (Digital Line Unit)} & Interface between subscriber lines
and exchange & Line cards, CODEC, SLIC, PCM conversion \\
\textbf{LTG (Line/Trunk Group)} & Handles trunk lines, interfaces with
other exchanges & Trunk cards, signaling units, echo cancellers \\
\textbf{SN (Switching Network)} & Routes calls between ports, provides
connectivity & Time/space switches, connection memory, control logic \\
\textbf{CP (Central Processor)} & Controls overall system operation &
Main processor, memory, operating system, databases \\
\textbf{Peripherals} & Supporting functions & Power supply, alarm
systems, maintenance terminals \\
\end{longtable}
}

\textbf{Hardware Elements Details:}

\begin{itemize}
\tightlist
\item
  \textbf{DLU}: Converts analog voice to 64 kbps PCM, handles line
  signaling
\item
  \textbf{LTG}: Manages E1/T1 trunks, implements protocols like SS7
\item
  \textbf{SN}: Typically time-division switching fabric, non-blocking
  architecture
\item
  \textbf{CP}: Call processing, billing, maintenance, administrative
  functions
\end{itemize}

\end{solutionbox}
\begin{mnemonicbox}
``DLSC'' - \textbf{D}LU connects subscribers,
\textbf{L}TG connects trunks, \textbf{S}N switches calls, \textbf{C}P
controls everything

\end{mnemonicbox}
\subsection*{Question 5(a) OR [3
marks]}\label{q5a}

\textbf{Compare TDM and FDM.}

\begin{solutionbox}


{\def\LTcaptype{none} % do not increment counter
\vspace{-5pt}
\captionof{table}{Comparison of TDM and FDM}
\vspace{-10pt}
\begin{longtable}[]{@{}
  >{\raggedright\arraybackslash}p{(\linewidth - 4\tabcolsep) * \real{0.5238}}
  >{\raggedright\arraybackslash}p{(\linewidth - 4\tabcolsep) * \real{0.2381}}
  >{\raggedright\arraybackslash}p{(\linewidth - 4\tabcolsep) * \real{0.2381}}@{}}
\toprule\noalign{}
\begin{minipage}[b]{\linewidth}\raggedright
Parameter
\end{minipage} & \begin{minipage}[b]{\linewidth}\raggedright
TDM
\end{minipage} & \begin{minipage}[b]{\linewidth}\raggedright
FDM
\end{minipage} \\
\midrule\noalign{}
\endhead
\bottomrule\noalign{}
\endlastfoot
\textbf{Domain Division} & Time & Frequency \\
\textbf{Channel Separation} & Guard time & Guard bands \\
\textbf{Multiplexing Process} & Sequential time slots & Parallel
frequency bands \\
\textbf{Implementation} & Digital (primarily) & Analog or digital \\
\textbf{Crosstalk} & Generally less & More susceptible \\
\textbf{Synchronization} & Critical & Not required \\
\end{longtable}
}

\textbf{Diagram:}

\begin{verbatim}
TDM:
  Time {-{-}}
  +{-{-}{-}{-}{-}{-}{-}{-}{-}{-}{-}+{-}{-}{-}{-}{-}{-}+{-}{-}{-}{-}{-}{-}+{-}{-}{-}{-}{-}{-}+}
  | Channel 1 | Ch 2 | Ch 3 | Ch 1 |...
  +{-{-}{-}{-}{-}{-}{-}{-}{-}{-}{-}+{-}{-}{-}{-}{-}{-}+{-}{-}{-}{-}{-}{-}+{-}{-}{-}{-}{-}{-}+}
  
FDM:
  \^{}
  |   +{-{-}{-}{-}{-}+}
F |   | Ch3 |
r |   +{-{-}{-}{-}{-}+}
e |   | Ch2 |
q |   +{-{-}{-}{-}{-}+}
  |   | Ch1 |
  |   +{-{-}{-}{-}{-}+}
  +{-{-}{-}{-}{-}{-}{-}{-}{-}{-}{-}{-}{-}{-}{-}}
        Time
\end{verbatim}

\begin{itemize}
\tightlist
\item
  \textbf{Bandwidth utilization}: TDM more efficient for digital, FDM
  better for analog
\item
  \textbf{System complexity}: TDM requires precise timing, FDM needs
  precise filters
\end{itemize}

\end{solutionbox}
\begin{mnemonicbox}
``TFDS'' - \textbf{T}ime and \textbf{F}requency
\textbf{D}ivision \textbf{S}ystems divide different domains

\end{mnemonicbox}
\subsection*{Question 5(b) OR [4
marks]}\label{q5b}

\textbf{Discuss T1 Multiplexing hierarchy.}

\begin{solutionbox}


{\def\LTcaptype{none} % do not increment counter
\vspace{-5pt}
\captionof{table}{T1 Multiplexing Hierarchy}
\vspace{-10pt}
\begin{longtable}[]{@{}
  >{\raggedright\arraybackslash}p{(\linewidth - 8\tabcolsep) * \real{0.0986}}
  >{\raggedright\arraybackslash}p{(\linewidth - 8\tabcolsep) * \real{0.1831}}
  >{\raggedright\arraybackslash}p{(\linewidth - 8\tabcolsep) * \real{0.1549}}
  >{\raggedright\arraybackslash}p{(\linewidth - 8\tabcolsep) * \real{0.3662}}
  >{\raggedright\arraybackslash}p{(\linewidth - 8\tabcolsep) * \real{0.1972}}@{}}
\toprule\noalign{}
\begin{minipage}[b]{\linewidth}\raggedright
Level
\end{minipage} & \begin{minipage}[b]{\linewidth}\raggedright
Designation
\end{minipage} & \begin{minipage}[b]{\linewidth}\raggedright
Data Rate
\end{minipage} & \begin{minipage}[b]{\linewidth}\raggedright
Number of Voice Channels
\end{minipage} & \begin{minipage}[b]{\linewidth}\raggedright
Multiplexing
\end{minipage} \\
\midrule\noalign{}
\endhead
\bottomrule\noalign{}
\endlastfoot
\textbf{T1} & DS1 & 1.544 Mbps & 24 & 24 DS0 (64 kbps) \\
\textbf{T2} & DS2 & 6.312 Mbps & 96 & 4 DS1 \\
\textbf{T3} & DS3 & 44.736 Mbps & 672 & 7 DS2 \\
\textbf{T4} & DS4 & 274.176 Mbps & 4032 & 6 DS3 \\
\end{longtable}
}

\textbf{Diagram:}

\begin{center}
\textbf{Mermaid Diagram (Code)}
\begin{verbatim}
{Shaded}
{Highlighting}[]
graph LR
    A[Individual Voice Channels {- DS0 64 kbps] {-}{-}{} B[T1/DS1 {-} 1.544 Mbps]}
    B {-{-}{} C[T2/DS2 {-} 6.312 Mbps]}
    C {-{-}{} D[T3/DS3 {-} 44.736 Mbps]}
    D {-{-}{} E[T4/DS4 {-} 274.176 Mbps]}
{Highlighting}
{Shaded}
\end{verbatim}
\end{center}

\textbf{T1 Frame Structure:}

\begin{verbatim}
T1 Frame (193 bits):
  F  Ch1  Ch2  ...  Ch24  F  Ch1  ...
  |  |    |         |     |
  |  8    8         8     |
  |  bits bits      bits  |
  |                       |
  Framing bit (1 bit)     Next frame
\end{verbatim}

\begin{itemize}
\tightlist
\item
  \textbf{T1 frame format}: 193 bits (24 channels \times 8 bits + 1 framing
  bit)
\item
  \textbf{Frame duration}: 125 μs (8000 frames per second)
\end{itemize}

\end{solutionbox}
\begin{mnemonicbox}
``T-QUAD'' - \textbf{T}1, T2, T3, T4 form a
\textbf{QUAD}ruple hierarchy of multiplexing levels

\end{mnemonicbox}
\subsection*{Question 5(c) OR [7
marks]}\label{q5c}

\textbf{List Features, Characteristics, Advantages and Disadvantages of
IoT.}

\begin{solutionbox}


{\def\LTcaptype{none} % do not increment counter
\vspace{-5pt}
\captionof{table}{Internet of Things (IoT) Overview}
\vspace{-10pt}
\begin{longtable}[]{@{}
  >{\raggedright\arraybackslash}p{(\linewidth - 2\tabcolsep) * \real{0.4545}}
  >{\raggedright\arraybackslash}p{(\linewidth - 2\tabcolsep) * \real{0.5455}}@{}}
\toprule\noalign{}
\begin{minipage}[b]{\linewidth}\raggedright
Category
\end{minipage} & \begin{minipage}[b]{\linewidth}\raggedright
Key Points
\end{minipage} \\
\midrule\noalign{}
\endhead
\bottomrule\noalign{}
\endlastfoot
\textbf{Features} & Device connectivity, Sensor integration, Automated
control, Data analytics, Remote monitoring \\
\textbf{Characteristics} & Low power consumption, Small form factor,
Wireless communication, Real-time data processing, Scalability \\
\textbf{Advantages} & Improved efficiency, Data-driven decisions, Remote
management, Predictive maintenance, Resource optimization \\
\textbf{Disadvantages} & Security vulnerabilities, Privacy concerns,
Interoperability issues, Implementation complexity, Power constraints \\
\end{longtable}
}

\textbf{Features of IoT:}

\begin{center}
\textbf{Mermaid Diagram (Code)}
\begin{verbatim}
{Shaded}
{Highlighting}[]
graph TD
    A[IoT Features] {-{-}{} B[Connectivity]}
    A {-{-}{} C[Intelligence]}
    A {-{-}{} D[Sensing]}
    A {-{-}{} E[Automation]}
    A {-{-}{} F[Cloud Integration]}
    A {-{-}{} G[Data Analytics]}
{Highlighting}
{Shaded}
\end{verbatim}
\end{center}

\textbf{Advantages \& Disadvantages:}

\begin{verbatim}
Advantages                    Disadvantages
+{-{-}{-}{-}{-}{-}{-}{-}{-}{-}{-}{-}{-}{-}{-}{-}{-}{-}{-}{-}{-}+      +{-}{-}{-}{-}{-}{-}{-}{-}{-}{-}{-}{-}{-}{-}{-}{-}{-}{-}{-}{-}{-}{-}+}
| ✓ Automation        |      | ✗ Security risks     |
| ✓ Enhanced data     |      | ✗ Privacy concerns   |
| ✓ Remote control    |      | ✗ Complex setup      |
| ✓ Cost reduction    |      | ✗ High initial cost  |
| ✓ Quality of life   |      | ✗ Battery life       |
| ✓ Resource savings  |      | ✗ Compatibility      |
+{-{-}{-}{-}{-}{-}{-}{-}{-}{-}{-}{-}{-}{-}{-}{-}{-}{-}{-}{-}{-}+      +{-}{-}{-}{-}{-}{-}{-}{-}{-}{-}{-}{-}{-}{-}{-}{-}{-}{-}{-}{-}{-}{-}+}
\end{verbatim}

\textbf{Characteristics Details:}

\begin{itemize}
\tightlist
\item
  \textbf{Interconnectivity}: Anything can be connected to global
  information \& communication infrastructure
\item
  \textbf{Things-related services}: IoT provides thing-related services
  like privacy protection
\item
  \textbf{Heterogeneity}: Devices based on different hardware/software
  platforms
\item
  \textbf{Dynamic changes}: Device state changes dynamically
  (connecting/disconnecting, sleeping/waking)
\item
  \textbf{Enormous scale}: Number of devices requiring management
  exceeds traditional internet connected devices
\end{itemize}

\end{solutionbox}
\begin{mnemonicbox}
``CASED'' - \textbf{C}onnectivity,
\textbf{A}utomation, \textbf{S}ensing, \textbf{E}fficiency,
\textbf{D}ata analytics - key IoT features

\end{mnemonicbox}

\end{document}
