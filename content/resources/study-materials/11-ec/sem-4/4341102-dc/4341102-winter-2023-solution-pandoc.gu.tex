\documentclass[10pt,a4paper]{article}

% content/resources/templates/preamble.tex
\usepackage[margin=0.6in]{geometry}
\author{Milav Dabgar}
\usepackage{amsmath,amssymb,amsthm}
\usepackage{booktabs}
\usepackage{multirow}
\usepackage{xcolor}
\usepackage{tcolorbox}
\tcbuselibrary{breakable,skins}
\usepackage[colorlinks=true,linkcolor=blue]{hyperref}
\usepackage{titlesec}
\usepackage{enumitem}
\usepackage{tikz}
\usepackage{pgfplots}
\usepackage{circuitikz}
\usepackage[version=4]{mhchem}
\usepackage{longtable}
\usepackage{array}
\usepackage{float}
\usepackage{caption}
\usepackage{listings}

\lstset{
  basicstyle=\small\ttfamily,
  breaklines=true,
  breakatwhitespace=false,
  postbreak=\mbox{\textcolor{red}{$\hookrightarrow$}\space},
  float=false,
  numbers=left,
  numberstyle=\tiny\color{gray},
  numbersep=10pt,
  xleftmargin=2em,
  keywordstyle=\color{blue},
  commentstyle=\color{green!60!black},
  stringstyle=\color{purple},
  backgroundcolor=\color{gray!5},
  showstringspaces=false,
  tabsize=2,
  captionpos=b,
  keepspaces=true,
  columns=flexible
}

\pgfplotsset{compat=1.18}
\usetikzlibrary{shapes,arrows,positioning,calc,patterns,decorations.pathmorphing,decorations.markings,arrows.meta}

% Color scheme
\definecolor{headcolor}{RGB}{0,102,204}
\definecolor{keycolor}{RGB}{220,20,60}
\definecolor{solutioncolor}{RGB}{34,139,34}
\definecolor{mnemoniccolor}{RGB}{148,0,211}
\definecolor{codecolor}{RGB}{0,0,100}

% Spacing
\setlength{\parskip}{3pt}
\setlist[itemize]{nosep}
\setlist[enumerate]{nosep}

% Title formatting
\titleformat{\section}{\Large\bfseries\color{headcolor}}{\thesection}{1em}{}
\titleformat{\subsection}{\large\bfseries\color{headcolor}}{\thesubsection}{1em}{}

% Pandoc tightlist compatibility
\providecommand{\tightlist}{%
  \setlength{\itemsep}{0pt}\setlength{\parskip}{0pt}}

% Pandoc longtable compatibility
\newcounter{none}
\def\thenone{}


% content/resources/templates/gujarati-boxes.tex
\usepackage{fontspec}
\usepackage{polyglossia}

% Set Gujarati as main language (document is primarily in Gujarati)
% Note: gloss-gujarati.ldf doesn't exist in polyglossia, but it will use hyphenation patterns
\setdefaultlanguage{gujarati}
\setotherlanguage{english}

% Configure Gujarati font properly
% Use Language=Default to prevent polyglossia from trying to add language-specific features
% that don't exist for Gujarati, which causes "empty feature" warnings
\newfontfamily\gujaratifont[Script=Gujarati,AutoFakeBold=2.5,AutoFakeSlant=0.3]{Noto Sans Gujarati}
\setmainfont[Script=Gujarati,AutoFakeBold=2.5,AutoFakeSlant=0.3]{Noto Sans Gujarati}
% Use Noto Sans Gujarati for monospace to support Gujarati in text
\setmonofont[Scale=0.9]{Noto Sans Gujarati}

% Configure English to use the same font
\newfontfamily\englishfont[Script=Gujarati,AutoFakeBold=2.5,AutoFakeSlant=0.3]{Noto Sans Gujarati}

% Translations for polyglossia
\gappto\captionsgujarati{
  \renewcommand{\tablename}{કોષ્ટક}
  \renewcommand{\figurename}{આકૃતિ}
}

% Helper for TikZ nodes to ensure Gujarati font
\newcommand{\gu}[1]{{\gujaratifont #1}}

% Custom environments
\newtcolorbox{solutionbox}{
    breakable,
    enhanced,
    colback=solutioncolor!5!white,
    colframe=solutioncolor!75!black,
    fonttitle=\bfseries,
    title=જવાબ
}

\newtcolorbox{solutionboxnobreak}{
 colback=solutioncolor!5!white,
 colframe=solutioncolor!75!black,
 fonttitle=\bfseries,
 title=જવાબ
}

\newtcolorbox{keyformula}{
 breakable,
 enhanced,
 colback=keycolor!5!white,
 colframe=keycolor!75!black,
 fonttitle=\bfseries,
 title=રાસાયણિક સમીકરણ/સૂત્ર
}

\newtcolorbox{mnemonicbox}{
 breakable,
 enhanced,
 colback=mnemoniccolor!5!white,
 colframe=mnemoniccolor!75!black,
 fonttitle=\bfseries,
 title=મેમરી ટ્રીક
}


\begin{document}

\begin{center}
{\Huge\bfseries\color{headcolor} Subject Name (Gujarati)}\\[5pt]
{\LARGE 4341102 -- Winter 2023}\\[3pt]
{\large Semester 1 Study Material}\\[3pt]
{\normalsize\textit{Detailed Solutions and Explanations}}
\end{center}

\vspace{10pt}

\subsection*{પ્રશ્ન 1(અ) [3
ગુણ]}\label{uxaaauxab0uxab6uxaa8-1uxa85-3-uxa97uxaa3}

\textbf{કોમ્યુનિકેશન ની વિવિધ ચેનલોની લાક્ષણિકતાઓ ચર્ચો.}

\begin{solutionbox}

{\def\LTcaptype{none} % do not increment counter
\begin{longtable}[]{@{}ll@{}}
\toprule\noalign{}
ચેનલ લાક્ષણિકતા & વર્ણન \\
\midrule\noalign{}
\endhead
\bottomrule\noalign{}
\endlastfoot
\textbf{બિટ રેટ} & પ્રતિ સેકન્ડ મહત્તમ પ્રસારિત બિટ્સની સંખ્યા \\
\textbf{બોડ રેટ} & પ્રતિ સેકન્ડ પ્રસારિત સિગ્નલ એકમો/પ્રતીકોની સંખ્યા \\
\textbf{બેન્ડવિડ્થ} & પ્રસારણ માટે જરૂરી આવૃત્તિઓની શ્રેણી \\
\textbf{રિપીટર અંતર} & સિગ્નલ ગુણવત્તા જાળવવા માટે રિપીટર્સ વચ્ચેનું મહત્તમ અંતર \\
\textbf{નોઈઝ ઇમ્યુનિટી} & બાહ્ય સ્ત્રોતોથી દખલ સામે પ્રતિકાર કરવાની ક્ષમતા \\
\end{longtable}
}

\end{solutionbox}
\begin{mnemonicbox}
``BBRN'' - ``બેટર બેન્ડવિડ્થ રિક્વાયર્સ નાઇસ પ્લાનિંગ''

\end{mnemonicbox}
\subsection*{પ્રશ્ન 1(બ) [4
ગુણ]}\label{uxaaauxab0uxab6uxaa8-1uxaac-4-uxa97uxaa3}

\textbf{ઈવન અને ઓડ સિગ્નલ વચ્ચે તફાવત આપો.}

\begin{solutionbox}

{\def\LTcaptype{none} % do not increment counter
\begin{longtable}[]{@{}
  >{\raggedright\arraybackslash}p{(\linewidth - 2\tabcolsep) * \real{0.5217}}
  >{\raggedright\arraybackslash}p{(\linewidth - 2\tabcolsep) * \real{0.4783}}@{}}
\toprule\noalign{}
\begin{minipage}[b]{\linewidth}\raggedright
ઈવન સિગ્નલ
\end{minipage} & \begin{minipage}[b]{\linewidth}\raggedright
ઓડ સિગ્નલ
\end{minipage} \\
\midrule\noalign{}
\endhead
\bottomrule\noalign{}
\endlastfoot
\textbf{ગાણિતિક રજૂઆત}: x(−t) = x(t) & \textbf{ગાણિતિક રજૂઆત}: x(−t) =
−x(t) \\
\textbf{સિમેટ્રી}: y-અક્ષની આસપાસ મિરર સિમેટ્રી & \textbf{સિમેટ્રી}: ઓરિજિન
સિમેટ્રી (રોટેશનલ) \\
\textbf{ફૂરિયર સીરીઝ}: ફક્ત કોસાઈન ટર્મ્સ ધરાવે છે & \textbf{ફૂરિયર સીરીઝ}: ફક્ત
સાઈન ટર્મ્સ ધરાવે છે \\
\textbf{ઉદાહરણો}: cos(t), t^{2} & \textbf{ઉદાહરણો}: sin(t), t^{3} \\
\end{longtable}
}

\begin{center}
\textbf{Mermaid Diagram (Code)}
\begin{verbatim}
{Shaded}
{Highlighting}[]
graph LR
    A["Signal x(t)"] {-{-}{} B\{Test symmetry\}}
    B {-{-}{}|"x({-}t) = x(t)"| C[Even Signal]}
    B {-{-}{}|"x({-}t) = {-}x(t)"| D[Odd Signal]}
    C {-{-}{} E[Mirror symmetry]}
    D {-{-}{} F[Origin symmetry]}
{Highlighting}
{Shaded}
\end{verbatim}
\end{center}

\end{solutionbox}
\begin{mnemonicbox}
``ઈવન સિગ્નલ્સ ફ્લિપ થતાં સમાન રહે છે, ઓડ સિગ્નલ્સ ફ્લિપ થતાં
વિપરીત થાય છે''

\end{mnemonicbox}
\subsection*{પ્રશ્ન 1(ક) [7
ગુણ]}\label{uxaaauxab0uxab6uxaa8-1uxa95-7-uxa97uxaa3}

\textbf{રિપીટર ને વ્યાખ્યાયિત કરો. રિપીટર કેવી રીતે કામ કરે છે તે જરૂરી સર્કિટ અને
વેવફોર્મ્સ સાથે સમજાવો.}

\begin{solutionbox}

\textbf{રિપીટર}: એક ઉપકરણ જે સિગ્નલને પ્રાપ્ત કરે છે, એમ્પ્લિફાય કરે છે, અને
પુનઃપ્રસારિત કરે છે જેથી પ્રસારણ અંતરને ડિગ્રેડેશન વિના વધારી શકાય.

\textbf{કાર્ય સિદ્ધાંત}: રિપીટર્સ ડિજિટલ સિગ્નલ્સને પુનર્જનન કરે છે જેથી ટ્રાન્સમિશન
લાઈન્સમાં ક્ષીણન અને નોઈઝ એકત્રીકરણને દૂર કરી શકાય.

\textbf{સર્કિટ ડાયાગ્રામ}:

\begin{verbatim}
            +{-{-}{-}{-}{-}{-}{-}{-}{-}{-}{-}{-}{-}+                  +{-}{-}{-}{-}{-}{-}{-}{-}{-}{-}{-}{-}{-}+}
 Input      |             |    Regenerated   |             |    Output
Signal {-{-}{-}{-}|  Receiver   |{-}{-}{-}{-}{-}{-}Signal{-}{-}{-}{-}{-}|  Transmitter|{-}{-}{-}{-}Signal{-}{-}}
            |             |                  |             |
            +{-{-}{-}{-}{-}{-}{-}{-}{-}{-}{-}{-}{-}+                  +{-}{-}{-}{-}{-}{-}{-}{-}{-}{-}{-}{-}{-}+}
                    |                               |
                    |       +{-{-}{-}{-}{-}{-}{-}{-}{-}{-}{-}{-}+          |}
                    +{-{-}{-}{-}{-}{-}|   Clock    |{-}{-}{-}{-}{-}{-}{-}{-}{-}+}
                            | Recovery   |
                            +{-{-}{-}{-}{-}{-}{-}{-}{-}{-}{-}{-}+}
\end{verbatim}

\textbf{વેવફોર્મ}:

\begin{verbatim}
  Input Signal            Repeater              Output Signal
     \_\_\_\_\_                                         \_\_\_\_\_
    |     |                                       |     |
\_\_\_\_|     |\_\_\_\_     {-{-}     \_\_\_         {-}{-}   \_\_\_\_|     |\_\_\_\_}
    |     |                                       |     |
    |     |                                       |     |
      Degraded                                    Regenerated
\end{verbatim}

\begin{itemize}
\tightlist
\item
  \textbf{સિગ્નલ રિસેપ્શન}: આવતા નબળા/વિકૃત સિગ્નલ્સને શોધે છે
\item
  \textbf{એમ્પ્લિફિકેશન}: સિગ્નલ પાવરને મજબૂત કરે છે
\item
  \textbf{રિજનરેશન}: મૂળ ડિજિટલ વેવફોર્મને પુનઃનિર્માણ કરે છે
\item
  \textbf{રિટ્રાન્સમિશન}: પુનઃસ્થાપિત સિગ્નલને આગલા સેગમેન્ટમાં મોકલે છે
\end{itemize}

\end{solutionbox}
\begin{mnemonicbox}
``RARE'' - ``રિસીવ, એમ્પ્લિફાય, રિજનરેટ, એમિટ''

\end{mnemonicbox}
\subsection*{પ્રશ્ન 1(ક) અથવા [7
ગુણ]}\label{uxaaauxab0uxab6uxaa8-1uxa95-uxa85uxaa5uxab5-7-uxa97uxaa3}

\textbf{ડિજિટલ કોમ્યુનિકેશન સિસ્ટમ નો બ્લોક ડાયાગ્રામ દોરો અને ઊંડાણથી સમજાવો.}

\begin{solutionbox}

\begin{verbatim}
flowchart LR
    A[Information Source] {-{-} B[Source Encoder]}
    B {-{-} C[Channel Encoder]}
    C {-{-} D[Digital Modulator]}
    D {-{-} E[Channel]}
    E {-{-} F[Digital Demodulator]}
    F {-{-} G[Channel Decoder]}
    G {-{-} H[Source Decoder]}
    H {-{-} I[Information Sink]}
\end{verbatim}

{\def\LTcaptype{none} % do not increment counter
\begin{longtable}[]{@{}
  >{\raggedright\arraybackslash}p{(\linewidth - 2\tabcolsep) * \real{0.5000}}
  >{\raggedright\arraybackslash}p{(\linewidth - 2\tabcolsep) * \real{0.5000}}@{}}
\toprule\noalign{}
\begin{minipage}[b]{\linewidth}\raggedright
બ્લોક
\end{minipage} & \begin{minipage}[b]{\linewidth}\raggedright
કાર્ય
\end{minipage} \\
\midrule\noalign{}
\endhead
\bottomrule\noalign{}
\endlastfoot
\textbf{ઇન્ફોર્મેશન સોર્સ} & પ્રસારિત કરવા માટેનો સંદેશ તૈયાર કરે છે (વૉઇસ, વિડિઓ,
ડેટા) \\
\textbf{સોર્સ એન્કોડર} & સોર્સ ડેટાને ડિજિટલ ફોર્મમાં રૂપાંતરિત કરે છે અને રિડન્ડન્સી
દૂર કરે છે \\
\textbf{ચેનલ એન્કોડર} & ભૂલ શોધ/સુધારણા માટે નિયંત્રિત રિડન્ડન્સી ઉમેરે છે \\
\textbf{ડિજિટલ મોડ્યુલેટર} & ડિજિટલ ડેટાને પ્રસારણ માટે યોગ્ય સિગ્નલ્સમાં રૂપાંતરિત
કરે છે \\
\textbf{ચેનલ} & ભૌતિક માધ્યમ જેના દ્વારા સિગ્નલ્સ પ્રવાસ કરે છે \\
\textbf{ડિજિટલ ડિમોડ્યુલેટર} & પ્રાપ્ત સિગ્નલ્સમાંથી ડિજિટલ ડેટા કાઢે છે \\
\textbf{ચેનલ ડિકોડર} & ઉમેરાયેલ રિડન્ડન્સીનો ઉપયોગ કરીને ભૂલો શોધે/સુધારે છે \\
\textbf{સોર્સ ડિકોડર} & મૂળ સોર્સ માહિતીનું પુનઃનિર્માણ કરે છે \\
\end{longtable}
}

\end{solutionbox}
\begin{mnemonicbox}
``સ્પષ્ટ ડેટા સંદેશો મોકલો, કાળજીપૂર્વક સુરક્ષિત માહિતી ડિકોડ
કરો''

\end{mnemonicbox}
\subsection*{પ્રશ્ન 2(અ) [3
ગુણ]}\label{uxaaauxab0uxab6uxaa8-2uxa85-3-uxa97uxaa3}

\textbf{યુનિટ સ્ટેપ ફંકશન, યુનિટ ઇમ્પલ્સ ફંકશન અને યુનિટ રેમ્પ ફંકશન ને વ્યાખ્યાયિત
કરો.}

\begin{solutionbox}

{\def\LTcaptype{none} % do not increment counter
\begin{longtable}[]{@{}
  >{\raggedright\arraybackslash}p{(\linewidth - 4\tabcolsep) * \real{0.2667}}
  >{\raggedright\arraybackslash}p{(\linewidth - 4\tabcolsep) * \real{0.3000}}
  >{\raggedright\arraybackslash}p{(\linewidth - 4\tabcolsep) * \real{0.4333}}@{}}
\toprule\noalign{}
\begin{minipage}[b]{\linewidth}\raggedright
ફંક્શન
\end{minipage} & \begin{minipage}[b]{\linewidth}\raggedright
વ્યાખ્યા
\end{minipage} & \begin{minipage}[b]{\linewidth}\raggedright
ગાણિતિક રૂપ
\end{minipage} \\
\midrule\noalign{}
\endhead
\bottomrule\noalign{}
\endlastfoot
\textbf{યુનિટ સ્ટેપ ફંક્શન} & નકારાત્મક સમય માટે 0 અને હકારાત્મક સમય માટે 1 મૂલ્ય લે
છે & u(t) = \{0, t \textless{} 0; 1, t \geq 0\} \\
\textbf{યુનિટ ઇમ્પલ્સ ફંક્શન} & અનંત ઊંચો, શૂન્ય પહોળાઈનો પલ્સ જેનું ક્ષેત્રફળ 1 છે &
δ(t) = \{\infty,

t = 0; 0, t \neq 0\} \\

\textbf{યુનિટ રેમ્પ ફંક્શન} & હકારાત્મક સમય માટે સમય સાથે રેખીય રીતે વધે છે & r(t) =
\{0, t \textless{} 0; t, t \geq 0\} \\
\end{longtable}
}

\end{solutionbox}
\begin{mnemonicbox}
``SIR'' - ``સ્ટેપ ઇન્સટન્ટલી, ઇમ્પલ્સ રેપિડલી, રેમ્પ
ગ્રેજ્યુઅલી''

\end{mnemonicbox}
\subsection*{પ્રશ્ન 2(બ) [4
ગુણ]}\label{uxaaauxab0uxab6uxaa8-2uxaac-4-uxa97uxaa3}

\textbf{કંટીન્યુયસ ટાઇમ અને ડિસક્રીટ ટાઇમ સિગ્નલ્સ ને વ્યાખ્યાયિત કરો અને ઉદાહરણ
સાથે સમજાવો.}

\begin{solutionbox}

{\def\LTcaptype{none} % do not increment counter
\begin{longtable}[]{@{}
  >{\raggedright\arraybackslash}p{(\linewidth - 6\tabcolsep) * \real{0.3590}}
  >{\raggedright\arraybackslash}p{(\linewidth - 6\tabcolsep) * \real{0.2308}}
  >{\raggedright\arraybackslash}p{(\linewidth - 6\tabcolsep) * \real{0.2051}}
  >{\raggedright\arraybackslash}p{(\linewidth - 6\tabcolsep) * \real{0.2051}}@{}}
\toprule\noalign{}
\begin{minipage}[b]{\linewidth}\raggedright
સિગ્નલ પ્રકાર
\end{minipage} & \begin{minipage}[b]{\linewidth}\raggedright
વ્યાખ્યા
\end{minipage} & \begin{minipage}[b]{\linewidth}\raggedright
ઉદાહરણ
\end{minipage} & \begin{minipage}[b]{\linewidth}\raggedright
રજૂઆત
\end{minipage} \\
\midrule\noalign{}
\endhead
\bottomrule\noalign{}
\endlastfoot
\textbf{કન્ટિન્યુઅસ-ટાઈમ સિગ્નલ} & તેના સમયગાળા દરમિયાન બધા સમય મૂલ્યો માટે
વ્યાખ્યાયિત & સાઈન વેવ x(t) = sin(t) & સ્મૂથ, અવિરત કર્વ \\
\textbf{ડિસ્ક્રીટ-ટાઈમ સિગ્નલ} & ફક્ત ચોક્કસ સમય ક્ષણો પર વ્યાખ્યાયિત & ડિજિટલ
સેમ્પલ્સ x[n] = sin(nTs) & અલગ મૂલ્યોની શ્રેણી \\
\end{longtable}
}

\textbf{ડાયાગ્રામ}:

\begin{verbatim}
Continuous{-time:   }
      /{      /      /      /      /  }
     /  {    /      /      /      /   }
{-{-}{-}{-}/    {-}{-}/    {-}{-}/    {-}{-}/    {-}{-}/    {-}{-}{-}{-} t}
    
Discrete{-time:}
      o       o       o       o       o  
     /       /       /       /       /    
{-{-}{-}{-}/{-}{-}{-}{-}{-}{-}{-}/{-}{-}{-}{-}{-}{-}{-}/{-}{-}{-}{-}{-}{-}{-}/{-}{-}{-}{-}{-}{-}{-}/{-}{-}{-}{-} n}
    o       o       o       o       o
\end{verbatim}

\begin{itemize}
\tightlist
\item
  \textbf{કન્ટિન્યુઅસ-ટાઈમ}: બધા સમય t \in R માટે વ્યાખ્યાયિત (અનંત મૂલ્યો)
\item
  \textbf{ડિસ્ક્રીટ-ટાઈમ}: ફક્ત ચોક્કસ ક્ષણો n \in Z પર વ્યાખ્યાયિત (ગણી શકાય તેવા
  મૂલ્યો)
\end{itemize}

\end{solutionbox}
\begin{mnemonicbox}
``CADD'' - ``કન્ટિન્યુઅસ ઓલવેઝ, ડિસ્ક્રીટ ડોટ્સ''

\end{mnemonicbox}
\subsection*{પ્રશ્ન 2(ક) [7
ગુણ]}\label{uxaaauxab0uxab6uxaa8-2uxa95-7-uxa97uxaa3}

\textbf{ASK મોડયુલેટર અને ડી-મોડ્યુલેટરના બ્લોક ડાયાગ્રામને વેવફોર્મ સાથે સમજાવો.}

\begin{solutionbox}

\textbf{ASK (એમ્પ્લિટ્યુડ શિફ્ટ કીઇંગ)}: એક ડિજિટલ મોડ્યુલેશન ટેકનિક જ્યાં બાઇનરી
ડેટા કેરિયર વેવની એમ્પ્લિટ્યુડ બદલીને રજૂ કરવામાં આવે છે.

\textbf{ASK મોડ્યુલેટર}:

\begin{verbatim}
flowchart LR
    A[Digital Input] {-{-} B[Product Modulator]}
    C[Carrier Generator] {-{-} B}
    B {-{-} D[Bandpass Filter]}
    D {-{-} E[ASK Output]}
\end{verbatim}

\textbf{ASK ડિમોડ્યુલેટર}:

\begin{verbatim}
flowchart LR
    A[ASK Input] {-{-} B[Envelope Detector]}
    B {-{-} C[Low Pass Filter]}
    C {-{-} D[Comparator]}
    D {-{-} E[Digital Output]}
\end{verbatim}

\textbf{વેવફોર્મ્સ}:

\begin{verbatim}
Digital Input:
    \_\_\_     \_\_\_         \_\_\_     \_\_\_
\_\_\_|   |\_\_\_|   |\_\_\_\_\_\_\_|   |\_\_\_|   |\_\_\_
    1   0   1     0     1   0   1

Carrier Wave:
/{//////////////////}

ASK Output:
    /{/    //        //    //}
\_\_\_/    {\_\_/    \_\_\_\_\_\_/    \_\_/    \_\_\_}
High  Low  High   Low   High Low  High
\end{verbatim}

\begin{itemize}
\tightlist
\item
  \textbf{મોડ્યુલેટર}: ડિજિટલ ઇનપુટના આધારે કેરિયર એમ્પ્લિટ્યુડ બદલે છે
\item
  \textbf{ડિમોડ્યુલેટર}: એન્વેલોપ એક્સટ્રેક્ટ કરે છે અને થ્રેશોલ્ડ સાથે સરખાવે છે
\end{itemize}

\end{solutionbox}
\begin{mnemonicbox}
``APE'' - ``પોઝિટિવ હોય ત્યારે એમ્પ્લિફાય કરો, ઝીરો હોય
ત્યારે એલિમિનેટ કરો''

\end{mnemonicbox}
\subsection*{પ્રશ્ન 2(અ) અથવા [3
ગુણ]}\label{uxaaauxab0uxab6uxaa8-2uxa85-uxa85uxaa5uxab5-3-uxa97uxaa3}

\textbf{સિંગ્યુલરિટી ફંક્શન સમજાવો.}

\begin{solutionbox}

\textbf{સિંગ્યુલરિટી ફંક્શન}: ગાણિતિક ફંક્શન્સ જેમાં ચોક્કસ બિંદુઓ પર અવિરતતા અથવા
અવ્યાખ્યાયિત મૂલ્યો હોય છે.

{\def\LTcaptype{none} % do not increment counter
\begin{longtable}[]{@{}ll@{}}
\toprule\noalign{}
સામાન્ય સિંગ્યુલરિટી ફંક્શન્સ & ગુણધર્મો \\
\midrule\noalign{}
\endhead
\bottomrule\noalign{}
\endlastfoot
\textbf{યુનિટ સ્ટેપ ફંક્શન u(t)} & t=0 પર 0 થી 1 પર કૂદકો મારે છે \\
\textbf{યુનિટ ઇમ્પલ્સ ફંક્શન δ(t)} & t=0 પર અનંત, બીજે ક્યાંય શૂન્ય, ક્ષેત્રફળ=1 \\
\textbf{યુનિટ રેમ્પ ફંક્શન r(t)} & યુનિટ સ્ટેપનું ડેરિવેટિવ ઇમ્પલ્સ છે \\
\end{longtable}
}

\textbf{સંબંધો}:

\begin{itemize}
\tightlist
\item
  δ(t) = d/dt[u(t)]
\item
  u(t) = \intδ(t)dt
\item
  r(t) = \intu(t)dt
\end{itemize}

\end{solutionbox}
\begin{mnemonicbox}
``SIR'' - ``સિંગ્યુલરિટીઝ ઇન્ક્લુડ રેપિડ ચેન્જીસ''

\end{mnemonicbox}
\subsection*{પ્રશ્ન 2(બ) અથવા [4
ગુણ]}\label{uxaaauxab0uxab6uxaa8-2uxaac-uxa85uxaa5uxab5-4-uxa97uxaa3}

\textbf{બીટ રેટ અને બોડ રેટ વચ્ચેનો તફાવત આપો.}

\begin{solutionbox}

{\def\LTcaptype{none} % do not increment counter
\begin{longtable}[]{@{}
  >{\raggedright\arraybackslash}p{(\linewidth - 4\tabcolsep) * \real{0.3571}}
  >{\raggedright\arraybackslash}p{(\linewidth - 4\tabcolsep) * \real{0.3214}}
  >{\raggedright\arraybackslash}p{(\linewidth - 4\tabcolsep) * \real{0.3214}}@{}}
\toprule\noalign{}
\begin{minipage}[b]{\linewidth}\raggedright
પેરામીટર
\end{minipage} & \begin{minipage}[b]{\linewidth}\raggedright
બિટ રેટ
\end{minipage} & \begin{minipage}[b]{\linewidth}\raggedright
બોડ રેટ
\end{minipage} \\
\midrule\noalign{}
\endhead
\bottomrule\noalign{}
\endlastfoot
\textbf{વ્યાખ્યા} & પ્રતિ સેકન્ડ પ્રસારિત બિટ્સની સંખ્યા & પ્રતિ સેકન્ડ પ્રસારિત
સિમ્બોલ્સની સંખ્યા \\
\textbf{એકમ} & બિટ્સ પ્રતિ સેકન્ડ (bps) & સિમ્બોલ્સ પ્રતિ સેકન્ડ (બોડ) \\
\textbf{સંબંધ} & બિટ રેટ = બોડ રેટ \times પ્રતિ સિમ્બોલ બિટ્સની સંખ્યા & બોડ રેટ = બિટ
રેટ \div પ્રતિ સિમ્બોલ બિટ્સની સંખ્યા \\
\textbf{ઉદાહરણ} & QPSK માં, જો બોડ રેટ = 1200, બિટ રેટ = 2400 bps & 16-QAM
માં, જો બિટ રેટ = 9600 bps, બોડ રેટ = 2400 \\
\end{longtable}
}

\begin{center}
\textbf{Mermaid Diagram (Code)}
\begin{verbatim}
{Shaded}
{Highlighting}[]
graph TD
    A[Transmission Rate] {-{-}{} B[Bit Rate]}
    A {-{-}{} C[Baud Rate]}
    B {-{-}{}|"bits/second"| D[Information Transfer Rate]}
    C {-{-}{}|"symbols/second"| E[Modulation Rate]}
    F[Modulation Technique] {-{-}{} G[Bits per Symbol]}
    G {-{-}{} H["Bit Rate = Baud Rate  Bits per Symbol"]}
{Highlighting}
{Shaded}
\end{verbatim}
\end{center}

\end{solutionbox}
\begin{mnemonicbox}
``BBSR'' - ``બિટ્સ ફોર બાઇનરી સ્પીડ, બોડ્સ ફોર સિમ્બોલ
રેટ''

\end{mnemonicbox}
\subsection*{પ્રશ્ન 2(ક) અથવા [7
ગુણ]}\label{uxaaauxab0uxab6uxaa8-2uxa95-uxa85uxaa5uxab5-7-uxa97uxaa3}

\textbf{8-PSK સિગ્નલ નો સિધ્ધાંત સમજાવો. તેમજ તેના કોન્સ્ટેલેશન ડાયાગ્રામ અને
વેવફોર્મ્સ દોરો.}

\begin{solutionbox}

\textbf{8-PSK (ફેઝ શિફ્ટ કીઇંગ)}: એક ડિજિટલ મોડ્યુલેશન ટેકનિક જ્યાં ડેટા કેરિયર
સિગ્નલના ફેઝને 8 અલગ અલગ પોઝિશન પર શિફ્ટ કરીને એન્કોડ કરવામાં આવે છે.

\textbf{સિદ્ધાંત}:

\begin{itemize}
\tightlist
\item
  દરેક સિમ્બોલ 3 બિટ્સ રજૂ કરે છે (log_{2}8 = 3)
\item
  45^\circ ના ગુણાંકોમાં ફેઝ શિફ્ટ (360^\circ\div8)
\item
  સ્થિર એમ્પ્લિટ્યુડ જાળવે છે
\end{itemize}

\textbf{કોન્સ્ટેલેશન ડાયાગ્રામ}:

\begin{verbatim}
                  000(0^)
                     o
                     |
            001(45^) o         o 111(315^)
                   /             {}
                  /               {}
        010(90^) o                 o 110(270^)
                  {               /}
                   {             /}
           011(135^) o         o 101(225^)
                     |
                     o
                  100(180^)
\end{verbatim}

\textbf{વેવફોર્મ}:

\begin{verbatim}
Data:   000    001    010    011    100    101    110    111
        \_\_\_    \_\_\_    \_\_\_    \_\_\_    \_\_\_    \_\_\_    \_\_\_    \_\_\_
       |   |  |   |  |   |  |   |  |   |  |   |  |   |  |   |
       |   |  |   |  |   |  |   |  |   |  |   |  |   |  |   |
Phase:  0^    45^    90^   135^   180^   225^   270^   315^
        /{     /     /     /     /     /     /     /}
       /  {   /     /     /     /     /     /     /  }
Signal:/    { /     /     /     /     /     /     /    }
\end{verbatim}

\begin{itemize}
\tightlist
\item
  \textbf{બેન્ડવિડ્થ કાર્યક્ષમતા}: 3 બિટ્સ પ્રતિ સિમ્બોલ
\item
  \textbf{સ્થિર એમ્પ્લિટ્યુડ}: વધુ સારી પાવર કાર્યક્ષમતા
\item
  \textbf{ભૂલની સંભાવના}: BPSK/QPSK કરતાં વધારે પરંતુ 16-PSK કરતાં ઓછી
\end{itemize}

\end{solutionbox}
\begin{mnemonicbox}
``8 પોઇન્ટ્સ શિફ્ટેડ ઇન K-સર્કલ'' (8-PSK)

\end{mnemonicbox}
\subsection*{પ્રશ્ન 3(અ) [3
ગુણ]}\label{uxaaauxab0uxab6uxaa8-3uxa85-3-uxa97uxaa3}

\textbf{FSK મોડયુલેટરનો બ્લોક ડાયાગ્રામ સમજાવો.}

\begin{solutionbox}

\textbf{FSK (ફ્રિક્વન્સી શિફ્ટ કીઇંગ)}: એક ડિજિટલ મોડ્યુલેશન ટેકનિક જ્યાં બાઇનરી
ડેટા કેરિયર વેવની ફ્રિક્વન્સી બદલીને રજૂ કરવામાં આવે છે.

\begin{verbatim}
flowchart LR
    A[Binary Input] {-{-} B\{Switch\}}
    C[Oscillator f1] {-{-} B}
    D[Oscillator f2] {-{-} B}
    B {-{-} E[Bandpass Filter]}
    E {-{-} F[FSK Output]}
\end{verbatim}

{\def\LTcaptype{none} % do not increment counter
\begin{longtable}[]{@{}ll@{}}
\toprule\noalign{}
કોમ્પોનન્ટ & કાર્ય \\
\midrule\noalign{}
\endhead
\bottomrule\noalign{}
\endlastfoot
\textbf{બાઇનરી ઇનપુટ} & પ્રસારિત કરવાનો ડિજિટલ ડેટા (0s અને 1s) \\
\textbf{ઓસીલેટર 1} & બિટ `1' માટે ફ્રિક્વન્સી f_{1} પર કેરિયર જનરેટ કરે છે \\
\textbf{ઓસીલેટર 2} & બિટ `0' માટે ફ્રિક્વન્સી f_{2} પર કેરિયર જનરેટ કરે છે \\
\textbf{સ્વિચ} & ઇનપુટ બિટના આધારે યોગ્ય ફ્રિક્વન્સી પસંદ કરે છે \\
\textbf{બેન્ડપાસ ફિલ્ટર} & ફ્રિક્વન્સીઓ વચ્ચેના ટ્રાન્ઝિશન્સને સ્મૂધ કરે છે \\
\end{longtable}
}

\end{solutionbox}
\begin{mnemonicbox}
``FISO'' - ``ફ્રિક્વન્સી ઇનપુટ સિલેક્ટ્સ ઓસિલેટર''

\end{mnemonicbox}
\subsection*{પ્રશ્ન 3(બ) [4
ગુણ]}\label{uxaaauxab0uxab6uxaa8-3uxaac-4-uxa97uxaa3}

\textbf{1010110011 શ્રેણી માટે ASK અને FSK ના મોડયુલેશન વેવફોર્મ્સ દોરો.}

\begin{solutionbox}

\begin{verbatim}
Binary Input: 1    0    1    0    1    1    0    0    1    1
             \_   \_   \_   \_   \_   \_   \_   \_   \_   \_   \_   \_   \_
             |   |   |   |   |   |   |   |   |   |   |   |   |
             |   |   |   |   |   |   |   |   |   |   |   |   |
            \_|\_\_\_|\_\_\_|\_\_\_|\_\_\_|\_\_\_|\_\_\_|\_\_\_|\_\_\_|\_\_\_|\_\_\_|\_\_\_|\_\_\_|\_

ASK Output:  |||       |||       |||||||||       ||||||||| 
             |||       |||       |||||||||       |||||||||
            \_|||\_\_\_\_\_|||\_\_\_\_\_|||||||\_\_\_\_\_|||||||\_\_\_\_\_\_\_

FSK Output:  |||       |||       |||       |||       |||
             |||       |||       |||       |||       |||
            \_|||\_{\_|||\_\_|||||||\_\_|||||||\_ }
                 {              }
High freq   (1)  Low(0)  (1)  Low(0)  (1)(1)  Low(0)  Low(0)  (1)(1)
\end{verbatim}

\textbf{સમજૂતી}:

\begin{itemize}
\tightlist
\item
  \textbf{ASK}: બિટ `1' માટે ઉચ્ચ એમ્પ્લિટ્યુડ, બિટ `0' માટે નીચી એમ્પ્લિટ્યુડ
\item
  \textbf{FSK}: બિટ `1' માટે ઉચ્ચતર ફ્રિક્વન્સી f_{1}, બિટ `0' માટે નીચી ફ્રિક્વન્સી
  f_{2}
\end{itemize}

\end{solutionbox}
\begin{mnemonicbox}
``ASK એમ્પ્લિટ્યુડ બદલે છે, FSK ફ્રિક્વન્સી બદલે છે''

\end{mnemonicbox}
\subsection*{પ્રશ્ન 3(ક) [7
ગુણ]}\label{uxaaauxab0uxab6uxaa8-3uxa95-7-uxa97uxaa3}

\textbf{PSK સિગ્નલ નું નિર્માણ અને શોધ તેના કાર્યરત ડાયાગ્રામ ની મદદ સાથે
સમજાવો.}

\begin{solutionbox}

\textbf{PSK (ફેઝ શિફ્ટ કીઇંગ)}: એક ડિજિટલ મોડ્યુલેશન ટેકનિક જ્યાં ડેટાને કેરિયર
સિગ્નલના ફેઝ બદલીને એન્કોડ કરવામાં આવે છે.

\textbf{PSK મોડ્યુલેટર}:

\begin{verbatim}
flowchart LR
    A[Binary Input] {-{-} B[Bipolar Converter]}
    B {-{-} C[Product Modulator]}
    D[Carrier Generator] {-{-} C}
    C {-{-} E[PSK Output]}
\end{verbatim}

\textbf{PSK ડિમોડ્યુલેટર}:

\begin{verbatim}
flowchart LR
    A[PSK Input] {-{-} B[Product Demodulator]}
    C[Carrier Recovery] {-{-} B}
    B {-{-} D[Low Pass Filter]}
    D {-{-} E[Decision Device]}
    E {-{-} F[Binary Output]}
\end{verbatim}

\textbf{વેવફોર્મ્સ}:

\begin{verbatim}
Binary Input:  1     0     1     1     0
              \_    \_    \_    \_    \_    \_    \_
              |    |    |    |    |    |    |
              |    |    |    |    |    |    |
             \_|\_\_\_\_|\_\_\_\_|\_\_\_\_|\_\_\_\_|\_\_\_\_|\_\_\_\_|\_\_\_

Bipolar:      +A   {-A    +A   +A   {-}A}
              \_    \_    \_    \_    \_    \_    \_
              |    |    |    |    |    |    |
              |    |    |    |    |    |    |
             \_|\_\_\_\_|\_\_\_\_|\_\_\_\_|\_\_\_\_|\_\_\_\_|\_\_\_\_|\_\_\_

Carrier:      /{///////////////}

PSK Output:   /{/  //  //  //  //}
              phase phase phase phase phase
               0^   180^   0^    0^   180^
\end{verbatim}

\begin{itemize}
\tightlist
\item
  \textbf{ઉત્પાદન}: બાઇનરી 1 \rightarrow 0^\circ ફેઝ, બાઇનરી 0 \rightarrow 180^\circ ફેઝ
\item
  \textbf{શોધ}: કેરિયર રિકવરી સાથે કોહેરન્ટ ડિમોડ્યુલેશન
\item
  \textbf{ફાયદા}: ASK કરતાં વધુ સારી નોઈઝ ઇમ્યુનિટી
\end{itemize}

\end{solutionbox}
\begin{mnemonicbox}
``PSK ફેઝીસ શિફ્ટ વિથ નોલેજ ઓફ કેરિયર''

\end{mnemonicbox}
\subsection*{પ્રશ્ન 3(અ) અથવા [3
ગુણ]}\label{uxaaauxab0uxab6uxaa8-3uxa85-uxa85uxaa5uxab5-3-uxa97uxaa3}

\textbf{ASK,FSK,PSK,QPSK,8-PSK અને 16-QAM ડિજિટલ મોડ્યુલેશન ટેકનિક્સ માટે બિટ્સ
પર સિમ્બોલ સરખાવો.}

\begin{solutionbox}

{\def\LTcaptype{none} % do not increment counter
\begin{longtable}[]{@{}llll@{}}
\toprule\noalign{}
મોડ્યુલેશન ટેકનિક & પ્રતિ સિમ્બોલ બિટ્સ & સ્ટેટ્સ & બેન્ડવિડ્થ કાર્યક્ષમતા \\
\midrule\noalign{}
\endhead
\bottomrule\noalign{}
\endlastfoot
\textbf{ASK} & 1 & 2 & 1 bit/Hz \\
\textbf{FSK} & 1 & 2 & 0.5 bit/Hz \\
\textbf{PSK (BPSK)} & 1 & 2 & 1 bit/Hz \\
\textbf{QPSK} & 2 & 4 & 2 bits/Hz \\
\textbf{8-PSK} & 3 & 8 & 3 bits/Hz \\
\textbf{16-QAM} & 4 & 16 & 4 bits/Hz \\
\end{longtable}
}

\begin{center}
\textbf{Mermaid Diagram (Code)}
\begin{verbatim}
{Shaded}
{Highlighting}[]
graph TD
    A[Modulation Techniques]
    A {-{-}{} B[ASK/FSK/BPSK{}br /{}1 bit/symbol]}
    A {-{-}{} C[QPSK{}br /{}2 bits/symbol]}
    A {-{-}{} D[8{-}PSK{}br /{}3 bits/symbol]}
    A {-{-}{} E[16{-}QAM{}br /{}4 bits/symbol]}
{Highlighting}
{Shaded}
\end{verbatim}
\end{center}

\end{solutionbox}
\begin{mnemonicbox}
``જેમ ફ્રિક્વન્સી/ફેઝ સ્ટેટ્સ ચોગણા થાય, બેન્ડવિડ્થ કાર્યક્ષમતા
બમણી થાય''

\end{mnemonicbox}
\subsection*{પ્રશ્ન 3(બ) અથવા [4
ગુણ]}\label{uxaaauxab0uxab6uxaa8-3uxaac-uxa85uxaa5uxab5-4-uxa97uxaa3}

\textbf{16 QAM નો કોન્સ્ટેલેશન ડાયાગ્રામ દોરો અને સમજાવો.}

\begin{solutionbox}

\textbf{16-QAM (ક્વોડ્રેચર એમ્પ્લિટ્યુડ મોડ્યુલેશન)}: એક મોડ્યુલેશન ટેકનિક જે એમ્પ્લિટ્યુડ
અને ફેઝ મોડ્યુલેશનને સંયોજિત કરે છે, જ્યાં દરેક સિમ્બોલ 4 બિટ્સ રજૂ કરે છે.

\textbf{કોન્સ્ટેલેશન ડાયાગ્રામ}:

\begin{verbatim}
      Q
      \^{}
      |   o   o   o   o
      |
      |   o   o   o   o
      |
{-{-}{-}{-}{-}{-}|{-}{-}{-}{-}{-}{-}{-}{-}{-}{-}{-}{-}{-}{-}{-} I}
      |   o   o   o   o
      |
      |   o   o   o   o
      |
\end{verbatim}

\textbf{સમજૂતી}:

\begin{itemize}
\tightlist
\item
  \textbf{16 અલગ અલગ સ્ટેટ્સ}: દરેક પોઇન્ટ એક અનન્ય 4-બિટ સંયોજન રજૂ કરે છે
\item
  \textbf{પ્રતિ સિમ્બોલ 4 બિટ્સ}: log_{2}16 = 4
\item
  \textbf{મોડ્યુલેશન પેરામીટર્સ}: એમ્પ્લિટ્યુડ અને ફેઝ બંને બદલાય છે
\item
  \textbf{સિમ્બોલ મેપિંગ}: બિટ ભૂલોને ઓછી કરવા માટે ગ્રે કોડિંગનો ઉપયોગ થાય છે
\end{itemize}

\end{solutionbox}
\begin{mnemonicbox}
``16 ક્વોડ્રન્ટ્સ એરેન્જ્ડ ઇન મેટ્રિક્સ''

\end{mnemonicbox}
\subsection*{પ્રશ્ન 3(ક) અથવા [7
ગુણ]}\label{uxaaauxab0uxab6uxaa8-3uxa95-uxa85uxaa5uxab5-7-uxa97uxaa3}

\textbf{MSK સિગ્નલ નો સિધ્ધાંત સમજાવો. તેમજ તેના કોન્સ્ટેલેશન ડાયાગ્રામ અને વેવફોર્મ્સ
દોરો.}

\begin{solutionbox}

\textbf{MSK (મિનિમમ શિફ્ટ કીઇંગ)}: 0.5 ના મોડ્યુલેશન ઇન્ડેક્સ સાથે એક સતત ફેઝ FSK
મોડ્યુલેશન, જે સરળ ફેઝ પરિવર્તનો સુનિશ્ચિત કરે છે.

\textbf{સિદ્ધાંત}:

\begin{itemize}
\tightlist
\item
  CPFSK (કન્ટિન્યુઅસ ફેઝ FSK) નો વિશેષ કેસ
\item
  ફ્રિક્વન્સી સેપરેશન બિટ રેટના અડધા જેટલું જ હોય છે
\item
  અચાનક પરિવર્તનો ટાળીને સતત ફેઝ જાળવે છે
\item
  મોડ્યુલેશન ઇન્ડેક્સ h = 0.5
\end{itemize}

\textbf{કોન્સ્ટેલેશન ડાયાગ્રામ}:

\begin{verbatim}
      Q
      \^{}
      |       o
      |     /   {}
      |    /     { }
      |   o       o
{-{-}{-}{-}{-}{-}|{-}{-}{-}{-}{-}{-}{-}{-}{-}{-}{-}{-}{-}{-}{-} I}
      |   o       o
      |    {     /}
      |     {   /}
      |       o
\end{verbatim}

\textbf{વેવફોર્મ્સ}:

\begin{verbatim}
Data:      1      0      1      1      0
           \_      \_      \_      \_      \_      \_
           |      |      |      |      |      |
          \_|\_\_\_\_\_\_|\_\_\_\_\_\_|\_\_\_\_\_\_|\_\_\_\_\_\_|\_\_\_\_\_\_|\_\_\_

MSK:      /        {      /      /        }
         /          {    /      /          }
        /            {  /      /            }
       /              {/      /              }
      /                                        {}
\end{verbatim}

મુખ્ય લક્ષણો:

\begin{itemize}
\tightlist
\item
  \textbf{સ્થિર એન્વેલોપ}: વધુ સારી પાવર કાર્યક્ષમતા
\item
  \textbf{સ્પેક્ટ્રલ કાર્યક્ષમતા}: BFSK કરતાં સાંકડી બેન્ડવિડ્થ
\item
  \textbf{સતત ફેઝ}: સરળ ટ્રાન્ઝિશન્સ, ઘટાડેલ સ્પેક્ટ્રલ ફેલાવો
\item
  \textbf{OQPSK સંબંધ}: સાઇનસોઇડલ પલ્સ શેપિંગ સાથે ઓફસેટ QPSK તરીકે જોઈ શકાય છે
\end{itemize}

\end{solutionbox}
\begin{mnemonicbox}
``MSK મેક્સ સ્મૂથ K-ટ્રાન્ઝિશન્સ''

\end{mnemonicbox}
\subsection*{પ્રશ્ન 4(અ) [3
ગુણ]}\label{uxaaauxab0uxab6uxaa8-4uxa85-3-uxa97uxaa3}

\textbf{FDD મલ્ટિપ્લેક્સિંગ સર્કિટ માં ખામી નિવારણ ની પ્રક્રિયા વર્ણવો.}

\begin{solutionbox}

{\def\LTcaptype{none} % do not increment counter
\begin{longtable}[]{@{}ll@{}}
\toprule\noalign{}
સ્ટેપ & ખામી નિવારણ પ્રક્રિયા \\
\midrule\noalign{}
\endhead
\bottomrule\noalign{}
\endlastfoot
\textbf{1. સિગ્નલ વેરિફિકેશન} & દરેક ફ્રિક્વન્સી બેન્ડ પર ઇનપુટ સિગ્નલ્સ ચેક કરો \\
\textbf{2. ફિલ્ટર એનાલિસિસ} & દરેક ચેનલ માટે બેન્ડપાસ ફિલ્ટર્સ ચકાસો \\
\textbf{3. મોડ્યુલેટર ટેસ્ટિંગ} & દરેક ચેનલમાં ફ્રિક્વન્સી ટ્રાન્સલેશન ટેસ્ટ કરો \\
\textbf{4. પાવર લેવલ્સ} & ઇનપુટ/આઉટપુટ પર સિગ્નલ સ્ટ્રેન્થ માપો \\
\textbf{5. આઇસોલેશન ચેક} & ચેનલો વચ્ચે ક્રોસ-ટોક માટે ટેસ્ટ કરો \\
\end{longtable}
}

\begin{verbatim}
flowchart LR
    A[Start] {-{-} B[Check Input Signals]}
    B {-{-} C\{Signals OK?\}}
    C {-{-}|Yes| D[Test Filters]}
    C {-{-}|No| E[Fix Input Source]}
    D {-{-} F\{Filters OK?\}}
    F {-{-}|Yes| G[Test Modulators]}
    F {-{-}|No| H[Replace/Adjust Filters]}
\end{verbatim}

\end{solutionbox}
\begin{mnemonicbox}
``SFMPI'' - ``સિગ્નલ, ફિલ્ટર, મોડ્યુલેટર, પાવર, આઇસોલેશન''

\end{mnemonicbox}
\subsection*{પ્રશ્ન 4(બ) [4
ગુણ]}\label{uxaaauxab0uxab6uxaa8-4uxaac-4-uxa97uxaa3}

\textbf{E1 કેરિયર ને T1 કેરિયર સાથે સરખાવો.}

\begin{solutionbox}

{\def\LTcaptype{none} % do not increment counter
\begin{longtable}[]{@{}
  >{\raggedright\arraybackslash}p{(\linewidth - 4\tabcolsep) * \real{0.3125}}
  >{\raggedright\arraybackslash}p{(\linewidth - 4\tabcolsep) * \real{0.3438}}
  >{\raggedright\arraybackslash}p{(\linewidth - 4\tabcolsep) * \real{0.3438}}@{}}
\toprule\noalign{}
\begin{minipage}[b]{\linewidth}\raggedright
પેરામીટર
\end{minipage} & \begin{minipage}[b]{\linewidth}\raggedright
E1 કેરિયર
\end{minipage} & \begin{minipage}[b]{\linewidth}\raggedright
T1 કેરિયર
\end{minipage} \\
\midrule\noalign{}
\endhead
\bottomrule\noalign{}
\endlastfoot
\textbf{સ્ટાન્ડર્ડ} & યુરોપિયન સ્ટાન્ડર્ડ & નોર્થ અમેરિકન સ્ટાન્ડર્ડ \\
\textbf{ડેટા રેટ} & 2.048 Mbps & 1.544 Mbps \\
\textbf{વૉઇસ ચેનલ્સ} & 30 ચેનલ્સ & 24 ચેનલ્સ \\
\textbf{ટાઇમ સ્લોટ્સ} & 32 ટાઇમ સ્લોટ્સ (TS0, TS1-TS15, TS16, TS17-TS31) &
24 ટાઇમ સ્લોટ્સ + ફ્રેમિંગ બિટ \\
\textbf{સિગ્નલિંગ} & ચેનલ 16 સિગ્નલિંગ માટે વપરાય છે & રોબ્ડ બિટ સિગ્નલિંગ \\
\textbf{ફ્રેમ સાઈઝ} & 256 બિટ્સ & 193 બિટ્સ \\
\textbf{બિટ રેટ પર ચેનલ} & 64 kbps & 64 kbps \\
\end{longtable}
}

\end{solutionbox}
\begin{mnemonicbox}
``ET-DR'' - ``યુરોપિયન થર્ટી, ડબલ રેટ

\end{mnemonicbox}
\subsection*{પ્રશ્ન 4(ક) [7
ગુણ]}\label{uxaaauxab0uxab6uxaa8-4uxa95-7-uxa97uxaa3}

\textbf{CDMA ટેકનિકને વિગતવાર સમજાવો.}

\begin{solutionbox}

\textbf{CDMA (કોડ ડિવિઝન મલ્ટિપલ એક્સેસ)}: એક મલ્ટિપલ એક્સેસ ટેકનિક જ્યાં એક જ
ફ્રિક્વન્સી બેન્ડને એક સાથે અનેક યુઝર્સ દ્વારા અનન્ય સ્પ્રેડિંગ કોડ્સનો ઉપયોગ કરીને શેર
કરવામાં આવે છે.

\begin{verbatim}
flowchart LR
    A[User Data] {-{-} B[Spreading]}
    C[Unique Code] {-{-} B}
    B {-{-} D[Transmission]}
    D {-{-} E[Despreading]}
    F[Same Code] {-{-} E}
    E {-{-} G[User Data Recovery]}
\end{verbatim}

{\def\LTcaptype{none} % do not increment counter
\begin{longtable}[]{@{}
  >{\raggedright\arraybackslash}p{(\linewidth - 2\tabcolsep) * \real{0.6471}}
  >{\raggedright\arraybackslash}p{(\linewidth - 2\tabcolsep) * \real{0.3529}}@{}}
\toprule\noalign{}
\begin{minipage}[b]{\linewidth}\raggedright
મુખ્ય લક્ષણ
\end{minipage} & \begin{minipage}[b]{\linewidth}\raggedright
વર્ણન
\end{minipage} \\
\midrule\noalign{}
\endhead
\bottomrule\noalign{}
\endlastfoot
\textbf{સ્પ્રેડિંગ કોડ્સ} & દરેક યુઝરને અનન્ય ઓર્થોગોનલ અથવા સ્યુડો-રેન્ડમ કોડ્સ આપવામાં
આવે છે \\
\textbf{પ્રોસેસ ગેઇન} & સ્પ્રેડ બેન્ડવિડ્થનો મૂળ બેન્ડવિડ્થ સાથેનો ગુણોત્તર \\
\textbf{ઇન્ટરફેરન્સ રિજેક્શન} & અલગ કોડ્સ ધરાવતા યુઝર્સ એકબીજા માટે નોઇઝ તરીકે
દેખાય છે \\
\textbf{સોફ્ટ હેન્ડઓફ} & મોબાઇલ એક સાથે બહુવિધ બેઝ સ્ટેશનો સાથે કોમ્યુનિકેટ કરી શકે
છે \\
\textbf{પાવર કંટ્રોલ} & નજીક-દૂર સમસ્યા હલ કરવા માટે મહત્વપૂર્ણ \\
\textbf{કેપેસિટી} & ફ્રિક્વન્સી દ્વારા સખત રીતે મર્યાદિત નથી, પરંતુ સ્વીકાર્ય નોઇઝ
લેવલ દ્વારા \\
\end{longtable}
}

\textbf{કામકાજનો સિદ્ધાંત}:

\begin{itemize}
\tightlist
\item
  દરેક બિટને હાઇ-રેટ સ્પ્રેડિંગ કોડ (ચિપ્સ) સાથે ગુણાકાર કરવામાં આવે છે
\item
  પરિણામી સિગ્નલ ઘણી વધારે પહોળી બેન્ડવિડ્થ રોકે છે
\item
  રિસીવર મૂળ ડેટા પુનર્પ્રાપ્ત કરવા માટે સમાન કોડનો ઉપયોગ કરે છે
\item
  અન્ય સિગ્નલ્સ રેન્ડમ નોઇઝ તરીકે દેખાય છે, કોરિલેશન દ્વારા નકારવામાં આવે છે
\end{itemize}

\end{solutionbox}
\begin{mnemonicbox}
``CUPS'' - ``કોડ્સ યુનિકલી પ્રોવાઇડ સેપરેશન''

\end{mnemonicbox}
\subsection*{પ્રશ્ન 4(અ) અથવા [3
ગુણ]}\label{uxaaauxab0uxab6uxaa8-4uxa85-uxa85uxaa5uxab5-3-uxa97uxaa3}

\textbf{મલ્ટિપ્લેક્સિંગ ટેકનિક્સ ના વર્ગીકરણ પર ટંકનોંધ લખો.}

\begin{solutionbox}

\textbf{મલ્ટિપ્લેક્સિંગ ટેકનિક્સ}: એક જ માધ્યમ પર પ્રસારણ માટે બહુવિધ સિગ્નલ્સને
સંયોજિત કરવાની પદ્ધતિઓ.

{\def\LTcaptype{none} % do not increment counter
\begin{longtable}[]{@{}
  >{\raggedright\arraybackslash}p{(\linewidth - 4\tabcolsep) * \real{0.2800}}
  >{\raggedright\arraybackslash}p{(\linewidth - 4\tabcolsep) * \real{0.3200}}
  >{\raggedright\arraybackslash}p{(\linewidth - 4\tabcolsep) * \real{0.4000}}@{}}
\toprule\noalign{}
\begin{minipage}[b]{\linewidth}\raggedright
પ્રકાર
\end{minipage} & \begin{minipage}[b]{\linewidth}\raggedright
આધારિત
\end{minipage} & \begin{minipage}[b]{\linewidth}\raggedright
ઉદાહરણો
\end{minipage} \\
\midrule\noalign{}
\endhead
\bottomrule\noalign{}
\endlastfoot
\textbf{ફ્રિક્વન્સી ડિવિઝન મલ્ટિપ્લેક્સિંગ (FDM)} & ફ્રિક્વન્સી ડોમેન & રેડિયો
બ્રોડકાસ્ટિંગ, કેબલ TV \\
\textbf{ટાઇમ ડિવિઝન મલ્ટિપ્લેક્સિંગ (TDM)} & ટાઇમ ડોમેન & ડિજિટલ ટેલિફોન સિસ્ટમ,
GSM \\
\textbf{કોડ ડિવિઝન મલ્ટિપ્લેક્સિંગ (CDM)} & કોડ ડોમેન & CDMA સેલ્યુલર સિસ્ટમ \\
\textbf{વેવલેન્થ ડિવિઝન મલ્ટિપ્લેક્સિંગ (WDM)} & વેવલેન્થ ડોમેન & ફાઇબર ઓપ્ટિક
કોમ્યુનિકેશન \\
\textbf{સ્પેસ ડિવિઝન મલ્ટિપ્લેક્સિંગ (SDM)} & સ્પેશિયલ ડોમેન & MIMO વાયરલેસ
સિસ્ટમ \\
\end{longtable}
}

\begin{center}
\textbf{Mermaid Diagram (Code)}
\begin{verbatim}
{Shaded}
{Highlighting}[]
graph TD
    A[Multiplexing Techniques] {-{-}{} B[Frequency Division]}
    A {-{-}{} C[Time Division]}
    A {-{-}{} D[Code Division]}
    A {-{-}{} E[Wavelength Division]}
    A {-{-}{} F[Space Division]}
{Highlighting}
{Shaded}
\end{verbatim}
\end{center}

\end{solutionbox}
\begin{mnemonicbox}
``FTCWS'' - ``ફાઇવ ટેકનિક્સ ક્રિએટ વાઇડ સિસ્ટમ્સ''

\end{mnemonicbox}
\subsection*{પ્રશ્ન 4(બ) અથવા [4
ગુણ]}\label{uxaaauxab0uxab6uxaa8-4uxaac-uxa85uxaa5uxab5-4-uxa97uxaa3}

\textbf{ટાઈમ ડિવિજન મલ્ટિપ્લેક્સિંગ ટેકનિક (TDM)નો બ્લોક ડાયાગ્રામ દોરો અને
સમજાવો.}

\begin{solutionbox}

\textbf{ટાઇમ ડિવિઝન મલ્ટિપ્લેક્સિંગ (TDM)}: એક ટેકનિક જ્યાં બહુવિધ સિગ્નલ્સ એક જ
ચેનલને શેર કરે છે, દરેક સિગ્નલને અલગ અલગ ટાઇમ સ્લોટ્સ ફાળવીને.

\begin{verbatim}
flowchart LR
    A1[Input 1] {-{-} B1[Sampler 1]}
    A2[Input 2] {-{-} B2[Sampler 2]}
    A3[Input 3] {-{-} B3[Sampler 3]}
    A4[Input 4] {-{-} B4[Sampler 4]}
    B1 {-{-} C[Commutator]}
    B2 {-{-} C}
    B3 {-{-} C}
    B4 {-{-} C}
    C {-{-} D[TDM Channel]}
    D {-{-} E[Decommutator]}
    E {-{-} F1[Filter 1] {-}{-} G1[Output 1]}
    E {-{-} F2[Filter 2] {-}{-} G2[Output 2]}
    E {-{-} F3[Filter 3] {-}{-} G3[Output 3]}
    E {-{-} F4[Filter 4] {-}{-} G4[Output 4]}
\end{verbatim}

{\def\LTcaptype{none} % do not increment counter
\begin{longtable}[]{@{}ll@{}}
\toprule\noalign{}
કોમ્પોનન્ટ & કાર્ય \\
\midrule\noalign{}
\endhead
\bottomrule\noalign{}
\endlastfoot
\textbf{સેમ્પલર્સ} & દરેક ઇનપુટ સિગ્નલને \geq 2 \times ઉચ્ચતમ ફ્રિક્વન્સી રેટે સેમ્પલ કરે છે \\
\textbf{કોમ્યુટેટર} & ક્રમશઃ દરેક ઇનપુટ ચેનલમાંથી સેમ્પલ્સ પસંદ કરે છે \\
\textbf{TDM ચેનલ} & સંયોજિત સિગ્નલ વહન કરે છે \\
\textbf{ડિકોમ્યુટેટર} & પ્રાપ્ત સેમ્પલ્સને યોગ્ય ચેનલ્સમાં વિતરિત કરે છે \\
\textbf{ફિલ્ટર્સ} & સેમ્પલ્સમાંથી મૂળ સિગ્નલ્સનું પુનઃનિર્માણ કરે છે \\
\end{longtable}
}

\end{solutionbox}
\begin{mnemonicbox}
``SCTDF'' - ``સેમ્પલ, કમ્બાઇન, ટ્રાન્સમિટ, ડિસ્ટ્રિબ્યુટ,
ફિલ્ટર''

\end{mnemonicbox}
\subsection*{પ્રશ્ન 4(ક) અથવા [7
ગુણ]}\label{uxaaauxab0uxab6uxaa8-4uxa95-uxa85uxaa5uxab5-7-uxa97uxaa3}

\textbf{TDMA ટેકનિકને વિગતવાર સમજાવો.}

\begin{solutionbox}

\textbf{TDMA (ટાઇમ ડિવિઝન મલ્ટિપલ એક્સેસ)}: એક ચેનલ એક્સેસ મેથડ જ્યાં બહુવિધ યુઝર્સ
એક જ ફ્રિક્વન્સી ચેનલને અલગ અલગ ટાઇમ સ્લોટ્સમાં વિભાજિત કરીને શેર કરે છે.

\begin{verbatim}
flowchart TD
    A[TDMA Frame] {-{-} B[Slot 1br /User 1]}
    A {-{-} C[Slot 2br /User 2]}
    A {-{-} D[Slot 3br /User 3]}
    A {-{-} E[Slot 4br /User 4]}
    A {-{-} F[Slot 5br /User 5]}
    A {-{-} G[Slot 6br /User 6]}
\end{verbatim}

{\def\LTcaptype{none} % do not increment counter
\begin{longtable}[]{@{}ll@{}}
\toprule\noalign{}
મુખ્ય લક્ષણ & વર્ણન \\
\midrule\noalign{}
\endhead
\bottomrule\noalign{}
\endlastfoot
\textbf{ફ્રેમ સ્ટ્રક્ચર} & ટાઇમ સ્લોટ્સમાં વિભાજિત નિશ્ચિત લંબાઈના ફ્રેમ્સ \\
\textbf{ગાર્ડ ટાઇમ} & ઓવરલેપ રોકવા માટે સ્લોટ્સ વચ્ચે નાના સમય અંતરાલ \\
\textbf{સિન્ક્રોનાઇઝેશન} & ચોક્કસ ટાઇમિંગ કોઓર્ડિનેશનની જરૂર \\
\textbf{ચેનલ યુટિલાઇઝેશન} & દરેક યુઝરને ટૂંકા સમયગાળા માટે સંપૂર્ણ બેન્ડવિડ્થ મળે છે \\
\textbf{પાવર કાર્યક્ષમતા} & ટ્રાન્સમીટર્સ વિરામયુક્ત કામ કરે છે, પાવર બચાવે છે \\
\textbf{કેપેસિટી} & ફ્રેમમાં ઉપલબ્ધ ટાઇમ સ્લોટ્સ દ્વારા મર્યાદિત \\
\end{longtable}
}

\textbf{અમલીકરણની વિગતો}:

\begin{itemize}
\tightlist
\item
  દરેક યુઝર ફાળવેલ સ્લોટમાં ઝડપી બર્સ્ટમાં ટ્રાન્સમિટ કરે છે
\item
  અવિરત ટ્રાન્સમિશન ન હોવાથી હેન્ડસેટ્સ નજીકના સેલ્સની સિગ્નલ સ્ટ્રેન્થ માપી શકે છે
\item
  GSM (પ્રતિ ફ્રેમ 8 સ્લોટ્સ), DECT, સેટેલાઇટ સિસ્ટમ્સમાં વપરાય છે
\item
  અનેક સ્લોટ્સ ફાળવીને અલગ અલગ ડેટા રેટ્સ સાથે સરળતાથી અનુકૂલ થઈ શકે છે
\end{itemize}

\end{solutionbox}
\begin{mnemonicbox}
``TDMA ટેક્સ ડિસ્ટિંક્ટ મોમેન્ટ્સ ફોર એક્સેસ''

\end{mnemonicbox}
\subsection*{પ્રશ્ન 5(અ) [3
ગુણ]}\label{uxaaauxab0uxab6uxaa8-5uxa85-3-uxa97uxaa3}

\textbf{સંભાવના વ્યાખ્યાયિત કરો અને કોમ્યુનિકેશન માં તેનું મહત્વ લખો.}

\begin{solutionbox}

\textbf{સંભાવના}: કોઈ ઘટના ઘટવાની સંભાવનાનું માપ, 0 અને 1 વચ્ચેના નંબર તરીકે વ્યક્ત
થાય છે.

{\def\LTcaptype{none} % do not increment counter
\begin{longtable}[]{@{}ll@{}}
\toprule\noalign{}
કોમ્યુનિકેશનમાં મહત્વ & સમજૂતી \\
\midrule\noalign{}
\endhead
\bottomrule\noalign{}
\endlastfoot
\textbf{વિશ્વસનીયતા વિશ્લેષણ} & ભૂલ સંભાવના અને સિસ્ટમ વિશ્વસનીયતા ગણતરી \\
\textbf{નોઇઝ પર્ફોર્મન્સ} & રેન્ડમ નોઇઝની હાજરીમાં સિસ્ટમ પર્ફોર્મન્સની મૂલ્યાંકન \\
\textbf{ઇન્ફોર્મેશન થિયરી} & શેનનના ચેનલ કેપેસિટી સિદ્ધાંત માટે આધાર \\
\textbf{સિગ્નલ ડિટેક્શન} & ઓપ્ટિમલ ડિટેક્શન થ્રેશોલ્ડ નક્કી કરવું \\
\end{longtable}
}

\end{solutionbox}
\begin{mnemonicbox}
``PRONIS'' - ``પ્રોબેબિલિટી ન્યુમેરિકલી ઇન્ડિકેટ્સ સિગ્નલ
ક્વોલિટી''

\end{mnemonicbox}
\subsection*{પ્રશ્ન 5(બ) [4
ગુણ]}\label{uxaaauxab0uxab6uxaa8-5uxaac-4-uxa97uxaa3}

\textbf{હાફમેન કોડ યોગ્ય દાખલા સાથે સમજાવો.}

\begin{solutionbox}

\textbf{હફમેન કોડ}: વેરિએબલ-લેન્થ પ્રીફિક્સ કોડિંગ અલ્ગોરિધમ જે વધુ વારંવાર આવતા
સિમ્બોલ્સને ટૂંકા કોડ આપે છે.

\textbf{ઉદાહરણ}: સિમ્બોલ્સ A, B, C, D ની સંભાવના 0.4, 0.3, 0.2, 0.1 અનુક્રમે
વિચારો.

\textbf{હફમેન કોડિંગ પ્રક્રિયા}:

\begin{center}
\textbf{Mermaid Diagram (Code)}
\begin{verbatim}
{Shaded}
{Highlighting}[]
graph LR
    A[A:0.4, B:0.3, C:0.2, D:0.1] {-{-}{} B[A:0.4, B:0.3, CD:0.3]}
    B {-{-}{} C[A:0.4, BCD:0.6]}
    C {-{-}{} D[ABCD:1.0]}
    D {-{-}{} E["A(0) | BCD(1)"]}
    E {-{-}{} F["A(0) | B(10) | CD(11)"]}
    F {-{-}{} G["A(0) | B(10) | C(110) | D(111)"]}
{Highlighting}
{Shaded}
\end{verbatim}
\end{center}

{\def\LTcaptype{none} % do not increment counter
\begin{longtable}[]{@{}lll@{}}
\toprule\noalign{}
સિમ્બોલ & સંભાવના & હફમેન કોડ \\
\midrule\noalign{}
\endhead
\bottomrule\noalign{}
\endlastfoot
A & 0.4 & 0 \\
B & 0.3 & 10 \\
C & 0.2 & 110 \\
D & 0.1 & 111 \\
\end{longtable}
}

\textbf{સરેરાશ કોડ લંબાઈ} = 0.4\times1 + 0.3\times2 + 0.2\times3 + 0.1\times3 = 1.9
બિટ્સ/સિમ્બોલ

\end{solutionbox}
\begin{mnemonicbox}
``HEMP'' - ``હફમેન એન્કોડ્સ મોર પ્રોબેબલ સિમ્બોલ્સ વિથ
શોર્ટર કોડ્સ''

\end{mnemonicbox}
\subsection*{પ્રશ્ન 5(ક) [7
ગુણ]}\label{uxaaauxab0uxab6uxaa8-5uxa95-7-uxa97uxaa3}

\textbf{ઈન્ટરનેટ ઓફ થિંગ્સ(IoT) ના ખ્યાલ અને મુખ્ય લક્ષણો સમજાવો.}

\begin{solutionbox}

\textbf{ઇન્ટરનેટ ઓફ થિંગ્સ (IoT)}: સેન્સર્સ, સોફ્ટવેર અને કનેક્ટિવિટી સાથે એમ્બેડેડ
ભૌતિક વસ્તુઓનું નેટવર્ક જે તેમને ડેટા એકત્રિત કરવા અને આદાન-પ્રદાન કરવા સક્ષમ બનાવે છે.

\begin{center}
\textbf{Mermaid Diagram (Code)}
\begin{verbatim}
{Shaded}
{Highlighting}[]
graph TD
    A[IoT Ecosystem] {-{-}{} B[Smart Devices]}
    A {-{-}{} C[Connectivity]}
    A {-{-}{} D[Data Analytics]}
    A {-{-}{} E[User Interface]}
    A {-{-}{} F[Security]}
    B {-{-}{} G[Sensors \& Actuators]}
    C {-{-}{} H[Protocols \& Standards]}
    D {-{-}{} I[Cloud Computing]}
    E {-{-}{} J[Apps \& Services]}
    F {-{-}{} K[Authentication \& Encryption]}
{Highlighting}
{Shaded}
\end{verbatim}
\end{center}

{\def\LTcaptype{none} % do not increment counter
\begin{longtable}[]{@{}
  >{\raggedright\arraybackslash}p{(\linewidth - 2\tabcolsep) * \real{0.6471}}
  >{\raggedright\arraybackslash}p{(\linewidth - 2\tabcolsep) * \real{0.3529}}@{}}
\toprule\noalign{}
\begin{minipage}[b]{\linewidth}\raggedright
મુખ્ય લક્ષણ
\end{minipage} & \begin{minipage}[b]{\linewidth}\raggedright
વર્ણન
\end{minipage} \\
\midrule\noalign{}
\endhead
\bottomrule\noalign{}
\endlastfoot
\textbf{કનેક્ટિવિટી} & ડિવાઇસીસ વિવિધ પ્રોટોકોલ્સ (Wi-Fi, Bluetooth, LPWAN,
5G) દ્વારા ઇન્ટરનેટ/એકબીજા સાથે જોડાયેલ \\
\textbf{સેન્સિંગ કેપેબિલિટી} & સેન્સર્સ દ્વારા ભૌતિક પેરામીટર્સને ડિટેક્ટ કરવાની
ક્ષમતા \\
\textbf{ઇન્ટેલિજન્સ} & ડિવાઇસ (એજ) અથવા ક્લાઉડ લેવલ પર ડેટા પ્રોસેસિંગ \\
\textbf{ઇન્ટરઓપરેબિલિટી} & વિવિધ પ્લેટફોર્મ્સ અને સિસ્ટમ્સ પર કામ કરવાની ક્ષમતા \\
\textbf{ઓટોમેશન} & માનવ હસ્તક્ષેપ વિના સ્વાયત્ત કાર્ય \\
\textbf{સ્કેલેબિલિટી} & કનેક્ટેડ ડિવાઇસીસની સંખ્યામાં વૃદ્ધિને સંભાળવાની ક્ષમતા \\
\end{longtable}
}

\textbf{એપ્લિકેશન્સ}:

\begin{itemize}
\tightlist
\item
  સ્માર્ટ હોમ્સ (થર્મોસ્ટેટ, સિક્યુરિટી સિસ્ટમ)
\item
  હેલ્થકેર (વેરેબલ ડિવાઇસીસ, રિમોટ મોનિટરિંગ)
\item
  ઔદ્યોગિક ઓટોમેશન (પ્રિડિક્ટિવ મેન્ટેનન્સ)
\item
  સ્માર્ટ સિટીઝ (ટ્રાફિક મેનેજમેન્ટ, વેસ્ટ મેનેજમેન્ટ)
\item
  એગ્રીકલ્ચર (પ્રિસિઝન ફાર્મિંગ, લાઇવસ્ટોક મોનિટરિંગ)
\end{itemize}

\end{solutionbox}
\begin{mnemonicbox}
``CSIA'' - ``કનેક્ટ, સેન્સ, ઇન્ટરપ્રેટ, ઓટોમેટ''

\end{mnemonicbox}
\subsection*{પ્રશ્ન 5(અ) અથવા [3
ગુણ]}\label{uxaaauxab0uxab6uxaa8-5uxa85-uxa85uxaa5uxab5-3-uxa97uxaa3}

\textbf{ચેનલ કેપસીટી ને SNR ના સંદર્ભમાં વ્યાખ્યાયિત કરો અને કોમ્યુનિકેશન માં તેનું મહત્વ
લખો.}

\begin{solutionbox}

\textbf{ચેનલ કેપેસિટી}: કોમ્યુનિકેશન ચેનલ પર લગભગ નગણ્ય ભૂલ સંભાવના સાથે માહિતી
પ્રસારિત કરી શકાય તે મહત્તમ દર.

\textbf{શેનનની ચેનલ કેપેસિટી ફોર્મ્યુલા}: C = B \times log_{2}(1 + SNR)

જ્યાં:

\begin{itemize}
\tightlist
\item
  C = ચેનલ કેપેસિટી (બિટ્સ પર સેકન્ડ)
\item
  B = બેન્ડવિડ્થ (હર્ટ્ઝ)
\item
  SNR = સિગ્નલ-ટુ-નોઇઝ રેશિયો
\end{itemize}

{\def\LTcaptype{none} % do not increment counter
\begin{longtable}[]{@{}
  >{\raggedright\arraybackslash}p{(\linewidth - 2\tabcolsep) * \real{0.7037}}
  >{\raggedright\arraybackslash}p{(\linewidth - 2\tabcolsep) * \real{0.2963}}@{}}
\toprule\noalign{}
\begin{minipage}[b]{\linewidth}\raggedright
કોમ્યુનિકેશનમાં મહત્વ
\end{minipage} & \begin{minipage}[b]{\linewidth}\raggedright
સમજૂતી
\end{minipage} \\
\midrule\noalign{}
\endhead
\bottomrule\noalign{}
\endlastfoot
\textbf{પર્ફોર્મન્સ લિમિટ} & ભૂલ-મુક્ત ટ્રાન્સમિશન માટે સૈદ્ધાંતિક મહત્તમ ડેટા રેટ સેટ
કરે છે \\
\textbf{સિસ્ટમ ડિઝાઇન} & મોડ્યુલેશન, કોડિંગ સ્કીમ્સની પસંદગીને માર્ગદર્શન આપે છે \\
\textbf{બેન્ડવિડ્થ કાર્યક્ષમતા} & બેન્ડવિડ્થ અને SNR વચ્ચેના ટ્રેડઓફ બતાવે છે \\
\textbf{લિંક બજેટ એનાલિસિસ} & જરૂરી ટ્રાન્સમિટ પાવર નક્કી કરવામાં મદદ કરે છે \\
\end{longtable}
}

\end{solutionbox}
\begin{mnemonicbox}
``CBLSN'' - ``કેપેસિટી ઇક્વલ્સ બેન્ડવિડ્થ ટાઇમ્સ લોગ ઓફ
સિગ્નલ-ટુ-નોઇઝ રેશિયો''

\end{mnemonicbox}
\subsection*{પ્રશ્ન 5(બ) અથવા [4
ગુણ]}\label{uxaaauxab0uxab6uxaa8-5uxaac-uxa85uxaa5uxab5-4-uxa97uxaa3}

\textbf{શેનો ફેનો કોડ યોગ્ય દાખલા સાથે સમજાવો.}

\begin{solutionbox}

\textbf{શેનન-ફેનો કોડિંગ}: સિમ્બોલ્સના સેટને લગભગ સમાન સંભાવના સાથે બે સબસેટ્સમાં
પુનરાવર્તી રીતે વિભાજિત કરીને તેમની સંભાવનાના આધારે સિમ્બોલ્સને વેરિએબલ-લેન્થ કોડ
આપવાની ટેકનિક.

\textbf{ઉદાહરણ}: સિમ્બોલ્સ A, B, C, D ની સંભાવના 0.4, 0.3, 0.2, 0.1 અનુક્રમે
વિચારો.

\textbf{શેનન-ફેનો પ્રક્રિયા}:

\begin{enumerate}
\tightlist
\item
  સિમ્બોલ્સને સંભાવના અનુસાર ક્રમબદ્ધ કરો: A(0.4), B(0.3), C(0.2), D(0.1)
\item
  લગભગ સમાન સંભાવના સાથે ગ્રૂપમાં વિભાજિત કરો:

  \begin{itemize}
  \tightlist
  \item
    ગ્રૂપ 1: A(0.4) - `0' આપવામાં આવે છે
  \item
    ગ્રૂપ 2: B(0.3), C(0.2), D(0.1) = 0.6 - `1' આપવામાં આવે છે
  \end{itemize}
\item
  ગ્રૂપ 2 ને પુનરાવર્તી રીતે વિભાજિત કરો:

  \begin{itemize}
  \tightlist
  \item
    ગ્રૂપ 2.1: B(0.3) - `10' આપવામાં આવે છે
  \item
    ગ્રૂપ 2.2: C(0.2), D(0.1) = 0.3 - `11' આપવામાં આવે છે
  \end{itemize}
\item
  ગ્રૂપ 2.2 વિભાજિત કરો:

  \begin{itemize}
  \tightlist
  \item
    C(0.2) - `110' આપવામાં આવે છે
  \item
    D(0.1) - `111' આપવામાં આવે છે
  \end{itemize}
\end{enumerate}

{\def\LTcaptype{none} % do not increment counter
\begin{longtable}[]{@{}lll@{}}
\toprule\noalign{}
સિમ્બોલ & સંભાવના & શેનન-ફેનો કોડ \\
\midrule\noalign{}
\endhead
\bottomrule\noalign{}
\endlastfoot
A & 0.4 & 0 \\
B & 0.3 & 10 \\
C & 0.2 & 110 \\
D & 0.1 & 111 \\
\end{longtable}
}

\textbf{સરેરાશ કોડ લંબાઈ} = 0.4\times1 + 0.3\times2 + 0.2\times3 + 0.1\times3 = 1.9
બિટ્સ/સિમ્બોલ

\end{solutionbox}
\begin{mnemonicbox}
``SFDS'' - ``શેનન ફેનો ડિવાઇડ્સ સિમ્બોલસેટ્સ''

\end{mnemonicbox}
\subsection*{પ્રશ્ન 5(ક) અથવા [7
ગુણ]}\label{uxaaauxab0uxab6uxaa8-5uxa95-uxa85uxaa5uxab5-7-uxa97uxaa3}

\textbf{ડિજિટલ ટેલિફોન એક્સચેંજ નો બ્લોક ડાયાગ્રામ દોરો અને સમજાવો.}

\begin{solutionbox}

\textbf{ડિજિટલ ટેલિફોન એક્સચેંજ}: એક સિસ્ટમ જે એનાલોગ વૉઇસ સિગ્નલ્સને ડિજિટલ
ફોર્મમાં રૂપાંતરિત કરીને અને ડિજિટલ સર્કિટ્સ દ્વારા સ્વિચિંગ કરીને ટેલિફોન કૉલ્સ જોડે
છે.

\begin{verbatim}
flowchart LR
    A[Subscribers] {-{-} B["Digital Line Unitsbr /(DLU)"]}
    B {-{-} C["Line/Trunk Groupbr /(LTG)"]}
    C {-{-} D["Switching Networkbr /(SN)"]}
    D {-{-} E["Central Processorbr /(CP)"]}
    E {-{-} D}
    D {-{-} C}
    C {-{-} B}
    B {-{-} A}
    F[Operation \& Maintenance{br /Center] {-}{-} E}
\end{verbatim}

{\def\LTcaptype{none} % do not increment counter
\begin{longtable}[]{@{}
  >{\raggedright\arraybackslash}p{(\linewidth - 2\tabcolsep) * \real{0.5000}}
  >{\raggedright\arraybackslash}p{(\linewidth - 2\tabcolsep) * \real{0.5000}}@{}}
\toprule\noalign{}
\begin{minipage}[b]{\linewidth}\raggedright
બ્લોક
\end{minipage} & \begin{minipage}[b]{\linewidth}\raggedright
કાર્ય
\end{minipage} \\
\midrule\noalign{}
\endhead
\bottomrule\noalign{}
\endlastfoot
\textbf{ડિજિટલ લાઇન યુનિટ્સ (DLU)} & સબ્સ્ક્રાઇબર લાઇન્સ અને એક્સચેંજ વચ્ચે ઇન્ટરફેસ,
A/D રૂપાંતરણ, લાઇન કોડિંગ કરે છે \\
\textbf{લાઇન/ટ્રંક ગ્રુપ (LTG)} & સિગ્નલિંગ મેનેજ કરે છે, સબ્સ્ક્રાઇબર ચેનલ્સને
મલ્ટિપ્લેક્સ/ડિમલ્ટિપ્લેક્સ કરે છે \\
\textbf{સ્વિચિંગ નેટવર્ક (SN)} & કોર સ્વિચિંગ ફેબ્રિક, ચેનલ્સ વચ્ચે કનેક્શન પાથ સ્થાપિત
કરે છે \\
\textbf{સેન્ટ્રલ પ્રોસેસર (CP)} & બધી એક્સચેંજ ઓપરેશન્સ, કૉલ પ્રોસેસિંગ, રાઉટિંગ
નિર્ણયો નિયંત્રિત કરે છે \\
\textbf{ઓપરેશન \& મેન્ટેનન્સ સેન્ટર} & સિસ્ટમ પર્ફોર્મન્સ મોનિટર કરે છે, ફોલ્ટ ડિટેક્શન,
ટ્રાફિક એનાલિસિસ \\
\end{longtable}
}

\textbf{મુખ્ય લક્ષણો}:

\begin{itemize}
\tightlist
\item
  \textbf{ટાઇમ ડિવિઝન સ્વિચિંગ}: અલગ અલગ ટાઇમ સ્લોટ્સ જોડે છે
\item
  \textbf{સ્પેસ ડિવિઝન સ્વિચિંગ}: અલગ અલગ ભૌતિક પાથ જોડે છે
\item
  \textbf{સ્ટોર્ડ પ્રોગ્રામ કંટ્રોલ}: સોફ્ટવેર-આધારિત કૉલ પ્રોસેસિંગ
\item
  \textbf{કોમન ચેનલ સિગ્નલિંગ}: અલગ સિગ્નલિંગ ચેનલ (SS7)
\item
  \textbf{નોન-બ્લોકિંગ આર્કિટેક્ચર}: બધા કૉલ્સ એક સાથે જોડી શકાય છે
\end{itemize}

\end{solutionbox}
\begin{mnemonicbox}
``DLSCO'' - ``ડિજિટલ લાઇન્સ સ્વિચ કૉલ્સ ઓર્ડરલી''

\end{mnemonicbox}

\end{document}
