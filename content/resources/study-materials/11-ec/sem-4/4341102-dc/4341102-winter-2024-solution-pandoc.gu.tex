\documentclass[10pt,a4paper]{article}

% content/resources/templates/preamble.tex
\usepackage[margin=0.6in]{geometry}
\author{Milav Dabgar}
\usepackage{amsmath,amssymb,amsthm}
\usepackage{booktabs}
\usepackage{multirow}
\usepackage{xcolor}
\usepackage{tcolorbox}
\tcbuselibrary{breakable,skins}
\usepackage[colorlinks=true,linkcolor=blue]{hyperref}
\usepackage{titlesec}
\usepackage{enumitem}
\usepackage{tikz}
\usepackage{pgfplots}
\usepackage{circuitikz}
\usepackage[version=4]{mhchem}
\usepackage{longtable}
\usepackage{array}
\usepackage{float}
\usepackage{caption}
\usepackage{listings}

\lstset{
  basicstyle=\small\ttfamily,
  breaklines=true,
  breakatwhitespace=false,
  postbreak=\mbox{\textcolor{red}{$\hookrightarrow$}\space},
  float=false,
  numbers=left,
  numberstyle=\tiny\color{gray},
  numbersep=10pt,
  xleftmargin=2em,
  keywordstyle=\color{blue},
  commentstyle=\color{green!60!black},
  stringstyle=\color{purple},
  backgroundcolor=\color{gray!5},
  showstringspaces=false,
  tabsize=2,
  captionpos=b,
  keepspaces=true,
  columns=flexible
}

\pgfplotsset{compat=1.18}
\usetikzlibrary{shapes,arrows,positioning,calc,patterns,decorations.pathmorphing,decorations.markings,arrows.meta}

% Color scheme
\definecolor{headcolor}{RGB}{0,102,204}
\definecolor{keycolor}{RGB}{220,20,60}
\definecolor{solutioncolor}{RGB}{34,139,34}
\definecolor{mnemoniccolor}{RGB}{148,0,211}
\definecolor{codecolor}{RGB}{0,0,100}

% Spacing
\setlength{\parskip}{3pt}
\setlist[itemize]{nosep}
\setlist[enumerate]{nosep}

% Title formatting
\titleformat{\section}{\Large\bfseries\color{headcolor}}{\thesection}{1em}{}
\titleformat{\subsection}{\large\bfseries\color{headcolor}}{\thesubsection}{1em}{}

% Pandoc tightlist compatibility
\providecommand{\tightlist}{%
  \setlength{\itemsep}{0pt}\setlength{\parskip}{0pt}}

% Pandoc longtable compatibility
\newcounter{none}
\def\thenone{}


% content/resources/templates/gujarati-boxes.tex
\usepackage{fontspec}
\usepackage{polyglossia}

% Set Gujarati as main language (document is primarily in Gujarati)
% Note: gloss-gujarati.ldf doesn't exist in polyglossia, but it will use hyphenation patterns
\setdefaultlanguage{gujarati}
\setotherlanguage{english}

% Configure Gujarati font properly
% Use Language=Default to prevent polyglossia from trying to add language-specific features
% that don't exist for Gujarati, which causes "empty feature" warnings
\newfontfamily\gujaratifont[Script=Gujarati,AutoFakeBold=2.5,AutoFakeSlant=0.3]{Noto Sans Gujarati}
\setmainfont[Script=Gujarati,AutoFakeBold=2.5,AutoFakeSlant=0.3]{Noto Sans Gujarati}
% Use Noto Sans Gujarati for monospace to support Gujarati in text
\setmonofont[Scale=0.9]{Noto Sans Gujarati}

% Configure English to use the same font
\newfontfamily\englishfont[Script=Gujarati,AutoFakeBold=2.5,AutoFakeSlant=0.3]{Noto Sans Gujarati}

% Translations for polyglossia
\gappto\captionsgujarati{
  \renewcommand{\tablename}{કોષ્ટક}
  \renewcommand{\figurename}{આકૃતિ}
}

% Helper for TikZ nodes to ensure Gujarati font
\newcommand{\gu}[1]{{\gujaratifont #1}}

% Custom environments
\newtcolorbox{solutionbox}{
    breakable,
    enhanced,
    colback=solutioncolor!5!white,
    colframe=solutioncolor!75!black,
    fonttitle=\bfseries,
    title=જવાબ
}

\newtcolorbox{solutionboxnobreak}{
 colback=solutioncolor!5!white,
 colframe=solutioncolor!75!black,
 fonttitle=\bfseries,
 title=જવાબ
}

\newtcolorbox{keyformula}{
 breakable,
 enhanced,
 colback=keycolor!5!white,
 colframe=keycolor!75!black,
 fonttitle=\bfseries,
 title=રાસાયણિક સમીકરણ/સૂત્ર
}

\newtcolorbox{mnemonicbox}{
 breakable,
 enhanced,
 colback=mnemoniccolor!5!white,
 colframe=mnemoniccolor!75!black,
 fonttitle=\bfseries,
 title=મેમરી ટ્રીક
}


\begin{document}

\begin{center}
{\Huge\bfseries\color{headcolor} Subject Name (Gujarati)}\\[5pt]
{\LARGE 4341102 -- Winter 2024}\\[3pt]
{\large Semester 1 Study Material}\\[3pt]
{\normalsize\textit{Detailed Solutions and Explanations}}
\end{center}

\vspace{10pt}

\subsection*{પ્રશ્ન 1(અ) [3
ગુણ]}\label{uxaaauxab0uxab6uxaa8-1uxa85-3-uxa97uxaa3}

\textbf{વેવ ફોર્મ સાથે કંટીન્યુઅસ ટાઇમ સિગ્નલ અને ડિસ્ક્રીટ ટાઇમ સિગ્નલ વ્યાખ્યાયિત
કરો.}

\begin{solutionbox}


{\def\LTcaptype{none} % do not increment counter
\vspace{-5pt}
\captionof{table}{સિગ્નલ પ્રકારોની તુલના}
\vspace{-10pt}
\begin{longtable}[]{@{}
  >{\raggedright\arraybackslash}p{(\linewidth - 4\tabcolsep) * \real{0.3023}}
  >{\raggedright\arraybackslash}p{(\linewidth - 4\tabcolsep) * \real{0.2791}}
  >{\raggedright\arraybackslash}p{(\linewidth - 4\tabcolsep) * \real{0.4186}}@{}}
\toprule\noalign{}
\begin{minipage}[b]{\linewidth}\raggedright
સિગ્નલ પ્રકાર
\end{minipage} & \begin{minipage}[b]{\linewidth}\raggedright
વ્યાખ્યા
\end{minipage} & \begin{minipage}[b]{\linewidth}\raggedright
વેવફોર્મ ઉદાહરણ
\end{minipage} \\
\midrule\noalign{}
\endhead
\bottomrule\noalign{}
\endlastfoot
\textbf{કંટીન્યુઅસ ટાઇમ સિગ્નલ} & દરેક સમય બિંદુ પર સતત મૂલ્યો સાથે વ્યાખ્યાયિત થયેલું
સિગ્નલ & સ્મૂધ, અવિચ્છિન્ન વક્ર \\
\textbf{ડિસ્ક્રીટ ટાઇમ સિગ્નલ} & ફક્ત ચોક્કસ સમય બિંદુઓ પર સેમ્પલ્સ સાથે વ્યાખ્યાયિત
થયેલું સિગ્નલ & અલગ-અલગ બિંદુઓની શ્રેણી \\
\end{longtable}
}

\textbf{આકૃતિ:}

\begin{center}
\textbf{Mermaid Diagram (Code)}
\begin{verbatim}
{Shaded}
{Highlighting}[]
graph TD
    subgraph Continuous
        A[કંટીન્યુઅસ ટાઇમ સિગ્નલ] {-{-}{} B["x(t)"]}
        B {-{-}{} C[દરેક t માટે વ્યાખ્યાયિત]}
    end
    subgraph Discrete
        D[ડિસ્ક્રીટ ટાઇમ સિગ્નલ] {-{-}{} E["x(n)"]}
        E {-{-}{} F[ફક્ત પૂર્ણાંક n માટે વ્યાખ્યાયિત]}
    end
{Highlighting}
{Shaded}
\end{verbatim}
\end{center}

\begin{itemize}
\tightlist
\item
  \textbf{એમ્પ્લિટ્યુડ સાતત્ય}: કંટીન્યુઅસ સિગ્નલમાં, એમ્પ્લિટ્યુડ કોઈપણ મૂલ્ય લઈ શકે છે,
  જ્યારે ડિસ્ક્રીટ સિગ્નલમાં ચોક્કસ એમ્પ્લિટ્યુડ મૂલ્યો હોય છે
\item
  \textbf{ગાણિતિક નોંધ}: કંટીન્યુઅસ સિગ્નલ માટે x(t), ડિસ્ક્રીટ સિગ્નલ માટે
  x[n] અથવા x(n) વપરાય છે
\end{itemize}

\end{solutionbox}
\begin{mnemonicbox}
``કોસીડી'' - \textbf{કો}ન્ટિન્યુઅસ \textbf{સી}ગ્નલ નદીની
જેમ વહે છે, \textbf{ડી}સ્ક્રીટ સિગ્નલ પગલાં જેવા હોય છે

\end{mnemonicbox}
\subsection*{પ્રશ્ન 1(બ) [4
ગુણ]}\label{uxaaauxab0uxab6uxaa8-1uxaac-4-uxa97uxaa3}

\textbf{પિરિયોડિક અને એપિરિયોડિક સિગ્નલ સમજાવો.}

\begin{solutionbox}


{\def\LTcaptype{none} % do not increment counter
\vspace{-5pt}
\captionof{table}{પિરિયોડિક અને એપિરિયોડિક સિગ્નલની તુલના}
\vspace{-10pt}
\begin{longtable}[]{@{}
  >{\raggedright\arraybackslash}p{(\linewidth - 4\tabcolsep) * \real{0.2273}}
  >{\raggedright\arraybackslash}p{(\linewidth - 4\tabcolsep) * \real{0.3636}}
  >{\raggedright\arraybackslash}p{(\linewidth - 4\tabcolsep) * \real{0.4091}}@{}}
\toprule\noalign{}
\begin{minipage}[b]{\linewidth}\raggedright
ગુણધર્મ
\end{minipage} & \begin{minipage}[b]{\linewidth}\raggedright
પિરિયોડિક સિગ્નલ
\end{minipage} & \begin{minipage}[b]{\linewidth}\raggedright
એપિરિયોડિક સિગ્નલ
\end{minipage} \\
\midrule\noalign{}
\endhead
\bottomrule\noalign{}
\endlastfoot
\textbf{વ્યાખ્યા} & નિશ્ચિત સમય અંતરાલ પછી એકદમ પુનરાવર્તિત થાય છે & પુનરાવર્તિત
થતું નથી અથવા અનંત પીરિયડ ધરાવે છે \\
\textbf{ગાણિતિક અભિવ્યક્તિ} & x(t) = x(t + nT) દરેક t માટે & x(t) \neq x(t + T)
કોઈપણ T માટે \\
\textbf{ઊર્જા/પાવર} & અનંત ઊર્જા, મર્યાદિત પાવર & મર્યાદિત ઊર્જા, શૂન્ય સરેરાશ
પાવર \\
\textbf{ઉદાહરણો} & સાઇન વેવ્સ, સ્ક્વેર વેવ્સ & સિંગલ પલ્સ, ડેમ્પ્ડ સાઇન્યુસોઇડ \\
\end{longtable}
}

\textbf{આકૃતિ:}

\begin{center}
\textbf{Mermaid Diagram (Code)}
\begin{verbatim}
{Shaded}
{Highlighting}[]
graph TD
    subgraph Periodic
        A["x(t) = x(t+T)"] {-{-}{} B[એકદમ પુનરાવર્તિત થાય છે]}
        B {-{-}{} C[મૂળભૂત પીરિયડ T]}
    end
    subgraph Aperiodic
        D["x(t)  x(t+T)"] {-{-}{} E[ક્યારેય એકદમ પુનરાવર્તિત થતું નથી]}
        E {-{-}{} F[કોઈ મૂળભૂત પીરિયડ નથી]}
    end
{Highlighting}
{Shaded}
\end{verbatim}
\end{center}

\begin{itemize}
\tightlist
\item
  \textbf{સ્પેક્ટ્રલ પ્રોપર્ટી}: પિરિયોડિક સિગ્નલમાં ડિસ્ક્રીટ ફ્રિક્વન્સી કોમ્પોનન્ટ્સ
  હોય છે, એપિરિયોડિકમાં સતત સ્પેક્ટ્રમ હોય છે
\item
  \textbf{ફૂરિયર એનાલિસિસ}: પિરિયોડિક સિગ્નલ માટે ફૂરિયર સીરીઝ, એપિરિયોડિક
  માટે ફૂરિયર ટ્રાન્સફોર્મ વપરાય છે
\end{itemize}

\end{solutionbox}
\begin{mnemonicbox}
``પાઅસ'' - \textbf{પિ}રિયોડિક સિગ્નલ્સ હંમેશા
\textbf{સ}મયમાં \textbf{આ}વર્તિત થાય છે

\end{mnemonicbox}
\subsection*{પ્રશ્ન 1(ક) [7
ગુણ]}\label{uxaaauxab0uxab6uxaa8-1uxa95-7-uxa97uxaa3}

\textbf{ડિજિટલ કોમ્યુનિકેશન સિસ્ટમનો બ્લોક ડાયાગ્રામ સમજાવો.}

\begin{solutionbox}

\textbf{આકૃતિ: ડિજિટલ કોમ્યુનિકેશન સિસ્ટમ}

\begin{verbatim}
flowchart LR
    A[સોર્સ] {-{-} B[સોર્સ એનકોડર]}
    B {-{-} C[ચેનલ એનકોડર]}
    C {-{-} D[ડિજિટલ મોડ્યુલેટર]}
    D {-{-} E[ચેનલ]}
    E {-{-} F[ડિજિટલ ડિમોડ્યુલેટર]}
    F {-{-} G[ચેનલ ડિકોડર]}
    G {-{-} H[સોર્સ ડિકોડર]}
    H {-{-} I[ડેસ્ટિનેશન]}
\end{verbatim}


{\def\LTcaptype{none} % do not increment counter
\vspace{-5pt}
\captionof{table}{ડિજિટલ કોમ્યુનિકેશન સિસ્ટમના બ્લોક્સના કાર્યો}
\vspace{-10pt}
\begin{longtable}[]{@{}
  >{\raggedright\arraybackslash}p{(\linewidth - 4\tabcolsep) * \real{0.2692}}
  >{\raggedright\arraybackslash}p{(\linewidth - 4\tabcolsep) * \real{0.3846}}
  >{\raggedright\arraybackslash}p{(\linewidth - 4\tabcolsep) * \real{0.3462}}@{}}
\toprule\noalign{}
\begin{minipage}[b]{\linewidth}\raggedright
બ્લોક
\end{minipage} & \begin{minipage}[b]{\linewidth}\raggedright
કાર્ય
\end{minipage} & \begin{minipage}[b]{\linewidth}\raggedright
ઉદાહરણ
\end{minipage} \\
\midrule\noalign{}
\endhead
\bottomrule\noalign{}
\endlastfoot
\textbf{સોર્સ} & ટ્રાન્સમિટ કરવાના સંદેશાનું જનરેશન & માઇક્રોફોન, કીબોર્ડ \\
\textbf{સોર્સ એનકોડર} & રિડન્ડન્સી દૂર કરે છે, ડેટા કોમ્પ્રેસ કરે છે & હફમેન કોડિંગ,
JPEG \\
\textbf{ચેનલ એનકોડર} & ભૂલ શોધવા/સુધારવા માટે નિયંત્રિત રિડન્ડન્સી ઉમેરે છે & હેમિંગ
કોડ્સ, CRC \\
\textbf{ડિજિટલ મોડ્યુલેટર} & ડિજિટલ ડેટાને એનાલોગ સિગ્નલમાં રૂપાંતરિત કરે છે & ASK,
FSK, PSK \\
\textbf{ચેનલ} & સિગ્નલ વહન કરતું માધ્યમ & વાયર્ડ, વાયરલેસ, ઓપ્ટિકલ ફાઇબર \\
\textbf{ડિજિટલ ડિમોડ્યુલેટર} & પ્રાપ્ત સિગ્નલને પાછું ડિજિટલમાં રૂપાંતરિત કરે છે &
ASK, FSK, PSK ડિમોડ્યુલેટર્સ \\
\textbf{ચેનલ ડિકોડર} & ઉમેરાયેલી રિડન્ડન્સીનો ઉપયોગ કરી ભૂલો શોધે/સુધારે છે & ભૂલ
સુધારણા સર્કિટ્સ \\
\textbf{સોર્સ ડિકોડર} & મૂળ સંદેશાનું પુનઃનિર્માણ કરે છે & ડેટા ડિકોમ્પ્રેશન \\
\end{longtable}
}

\begin{itemize}
\tightlist
\item
  \textbf{ફાયદો}: નોઇઝ ઇમ્યુનિટી, સુરક્ષિત ટ્રાન્સમિશન, મલ્ટિપ્લેક્સિંગ ક્ષમતા,
  ડિજિટલ સિસ્ટમ્સ સાથે એકીકરણ
\item
  \textbf{મુખ્ય પ્રક્રિયાઓ}: સેમ્પલિંગ, ક્વોન્ટાઇઝેશન, કોડિંગ, મોડ્યુલેશન/ડિમોડ્યુલેશન
\end{itemize}

\end{solutionbox}
\begin{mnemonicbox}
``સેચમદેસિ'' - \textbf{સો}ર્સ \textbf{એ}ન્કોડ કરે,
\textbf{ચે}નલ કોડ, \textbf{મો}ડ્યુલેટ, \textbf{ચે}નલ, \textbf{ડિ}મોડ્યુલેટ,
\textbf{સિ}ંક પ્રાપ્ત કરે

\end{mnemonicbox}
\subsection*{પ્રશ્ન 1(ક) OR [7
ગુણ]}\label{uxaaauxab0uxab6uxaa8-1uxa95-or-7-uxa97uxaa3}

\textbf{સિંગ્યુલારીટી ફંકશન સમજાવો.}

\begin{solutionbox}


{\def\LTcaptype{none} % do not increment counter
\vspace{-5pt}
\captionof{table}{સામાન્ય સિંગ્યુલારીટી ફંકશન્સ}
\vspace{-10pt}
\begin{longtable}[]{@{}
  >{\raggedright\arraybackslash}p{(\linewidth - 6\tabcolsep) * \real{0.1639}}
  >{\raggedright\arraybackslash}p{(\linewidth - 6\tabcolsep) * \real{0.4098}}
  >{\raggedright\arraybackslash}p{(\linewidth - 6\tabcolsep) * \real{0.1967}}
  >{\raggedright\arraybackslash}p{(\linewidth - 6\tabcolsep) * \real{0.2295}}@{}}
\toprule\noalign{}
\begin{minipage}[b]{\linewidth}\raggedright
ફંકશન
\end{minipage} & \begin{minipage}[b]{\linewidth}\raggedright
ગાણિતિક વ્યાખ્યા
\end{minipage} & \begin{minipage}[b]{\linewidth}\raggedright
ગુણધર્મો
\end{minipage} & \begin{minipage}[b]{\linewidth}\raggedright
ઉપયોગો
\end{minipage} \\
\midrule\noalign{}
\endhead
\bottomrule\noalign{}
\endlastfoot
\textbf{યુનિટ સ્ટેપ} & u(t) = 1 જ્યારે t \geq 0, 0 જ્યારે t \textless{} 0 & t=0 પર
અસાતત્ય & સ્વિચ-ઓન સિગ્નલ્સ, હેવિસાઇડ ફંકશન \\
\textbf{યુનિટ ઇમ્પલ્સ} & δ(t) = \infty જ્યારે t = 0, અન્યત્ર 0, \intδ(t)dt = 1 & અનંત
ઊંચાઈવાળું, શૂન્ય પહોળાઈવાળું & ઇમ્પલ્સ રિસ્પોન્સ, સેમ્પલિંગ \\
\textbf{યુનિટ રેમ્પ} & r(t) = t·u(t) & સાતત્ય પરંતુ t=0 પર ડિફરેન્શિયેબલ નથી &
લિનિયર ટાઇમ ફંકશન્સ \\
\textbf{યુનિટ પેરાબોલા} & p(t) = (t^{2}/2)·u(t) & યુનિટ ઇમ્પલ્સનું બીજું ઇન્ટિગ્રલ &
એક્સેલરેશનથી પોઝિશન \\
\end{longtable}
}

\textbf{આકૃતિ:}

\begin{verbatim}
   \^{}
   |                    ┌────────────────
   |                    │ Unit Step
   |────────────────────┘
   |
   +{-{-}{-}{-}{-}{-}{-}{-}{-}{-}{-}{-}{-}{-}{-}{-}{-}{-}{-}{-}{-}{-}{-}{-}{-} t}
   |
   \^{                     /}
   |                    /
   |                   / Unit Ramp
   |─────────────────/
   |                /
   +{-{-}{-}{-}{-}{-}{-}{-}{-}{-}{-}{-}{-}{-}{-}/{-}{-}{-}{-}{-}{-}{-}{-}{-}{-}{-}{-} t}
   |              /
   \^{}
   |             .
   |             │ Unit Impulse
   |─────────────┼──────────────{ t}
   |             {}
\end{verbatim}

\begin{itemize}
\tightlist
\item
  \textbf{ઇન્ટિગ્રેશન સંબંધ}: દરેક ફંકશન એ અગાઉના ફંકશનનું ઇન્ટિગ્રલ છે
\item
  \textbf{ગાણિતિક ટૂલકિટ}: જટિલ સિસ્ટમ્સને સરળ ઘટકોમાં વિભાજિત કરીને વિશ્લેષણ
  કરવા માટે ઉપયોગી
\end{itemize}

\end{solutionbox}
\begin{mnemonicbox}
``સ્ટેઇંપેરે'' - \textbf{સ્ટે}પ \textbf{ઇં}પલ્સ
\textbf{પે}રાબોલા \textbf{રે}મ્પ - ઇન્ટિગ્રેશનના વધતા ક્રમમાં ફંકશન્સ

\end{mnemonicbox}
\subsection*{પ્રશ્ન 2(અ) [3
ગુણ]}\label{uxaaauxab0uxab6uxaa8-2uxa85-3-uxa97uxaa3}

\textbf{સિગ્નલ 10 બીટ/સિગ્નલ એલીમેન્ટ ધરાવે છે. જો સેકન્ડ દીઠ 100 સિગ્નલ એલીમેન્ટ
મોકલવામાં આવે છે. બીટ રેટ શોધો.}

\begin{solutionbox}

\textbf{ઉકેલ:}

\begin{verbatim}
બીટ રેટ = સિગ્નલ એલિમેન્ટ દીઠ બિટ્સની સંખ્યા \times પ્રતિ સેકન્ડ સિગ્નલ એલિમેન્ટની સંખ્યા
બીટ રેટ = 10 બિટ્સ/સિગ્નલ એલિમેન્ટ \times 100 સિગ્નલ એલિમેન્ટ/સેકન્ડ
બીટ રેટ = 1000 બિટ્સ/સેકન્ડ = 1 kbps
\end{verbatim}

\textbf{આકૃતિ:}

\begin{center}
\textbf{Mermaid Diagram (Code)}
\begin{verbatim}
{Shaded}
{Highlighting}[]
graph LR
    A[સિગ્નલ એલિમેન્ટ્સ: 100/s] {-{-}{} B[દરેક એલિમેન્ટ: 10 બિટ્સ]}
    B {-{-}{} C[બીટ રેટ = 1000 બિટ્સ/s]}
{Highlighting}
{Shaded}
\end{verbatim}
\end{center}

\begin{itemize}
\tightlist
\item
  \textbf{બીટ રેટ}: પ્રતિ સેકંડ ટ્રાન્સમિટ થતા બિટ્સની સંખ્યા (bps)
\item
  \textbf{સિગ્નલ એલિમેન્ટ}: એક કે વધુ બિટ્સનું ભૌતિક પ્રગટીકરણ
\end{itemize}

\end{solutionbox}
\begin{mnemonicbox}
``બીએઈ'' - \textbf{બી}ટ રેટ એ \textbf{એ}લિમેન્ટ્સ ગુણ્યા
દરેક \textbf{ઈ}લેમેન્ટ દીઠ બિટ્સ

\end{mnemonicbox}
\subsection*{પ્રશ્ન 2(બ) [4
ગુણ]}\label{uxaaauxab0uxab6uxaa8-2uxaac-4-uxa97uxaa3}

\textbf{ઈવન અને ઓડ સિગ્નલ સમજાવો.}

\begin{solutionbox}


{\def\LTcaptype{none} % do not increment counter
\vspace{-5pt}
\captionof{table}{ઈવન અને ઓડ સિગ્નલની તુલના}
\vspace{-10pt}
\begin{longtable}[]{@{}lll@{}}
\toprule\noalign{}
ગુણધર્મ & ઈવન સિગ્નલ & ઓડ સિગ્નલ \\
\midrule\noalign{}
\endhead
\bottomrule\noalign{}
\endlastfoot
\textbf{વ્યાખ્યા} & f(-t) = f(t) & f(-t) = -f(t) \\
\textbf{સિમેટ્રી} & y-અક્ષની આસપાસ મિરર સિમેટ્રી & ઓરિજિન સિમેટ્રી (રોટેશનલ) \\
\textbf{ફૂરિયર સીરીઝ} & માત્ર કોસાઇન ટર્મ્સ ધરાવે છે & માત્ર સાઇન ટર્મ્સ ધરાવે
છે \\
\textbf{ઉદાહરણો} & કોસાઇન, & t \\
\end{longtable}
}

\textbf{આકૃતિ:}

\begin{verbatim}
  Even Signal             Odd Signal
     \^{                       \^{}}
     |                       |
     |     .{-.               |      /}
     |    /   {              |     /}
     |{-{-}{-}/{-}{-}{-}{-}{-}{-}{-}{-}{-}{-}       |{-}{-}{-}{-}/{-}{-}{-}{-}{-}{-}{-}{-}{-}}
     |  /       {            |   /      }
     | {                    |  /        }
     |                       | /          {}
\end{verbatim}

\begin{itemize}
\tightlist
\item
  \textbf{ડિકમ્પોઝિશન}: કોઈપણ સિગ્નલને ઈવન અને ઓડ ઘટકોના સરવાળા તરીકે વિભાજિત
  કરી શકાય છે
\item
  \textbf{ઈવન પાર્ટ}: f\_e(t) = [f(t) + f(-t)]/2
\item
  \textbf{ઓડ પાર્ટ}: f\_o(t) = [f(t) - f(-t)]/2
\end{itemize}

\end{solutionbox}
\begin{mnemonicbox}
``ઈસઓપ'' - \textbf{ઈ}વન \textbf{સિ}ગ્નલ્સ મિરર સિમેટ્રી
ધરાવે છે, \textbf{ઓ}ડ સિગ્નલ્સ મિરર થતાં ઊલટા થઈ જાય છે - \textbf{પ}રાવર્તન

\end{mnemonicbox}
\subsection*{પ્રશ્ન 2(ક) [7
ગુણ]}\label{uxaaauxab0uxab6uxaa8-2uxa95-7-uxa97uxaa3}

\textbf{ASK મોડ્યુલેટર અને ડી-મોડ્યુલેટરના બ્લોક ડાયાગ્રામને વેવફોર્મ સાથે સમજાવો.}

\begin{solutionbox}

\textbf{ASK મોડ્યુલેટર ડાયાગ્રામ:}

\begin{verbatim}
flowchart LR
    A[ડિજિટલ ઇનપુટ] {-{-} B[પ્રોડક્ટ મોડ્યુલેટર]}
    C[કેરિયર જનરેટર fc] {-{-} B}
    B {-{-} D[ASK આઉટપુટ]}
\end{verbatim}

\textbf{ASK ડિમોડ્યુલેટર ડાયાગ્રામ:}

\begin{verbatim}
flowchart LR
    A[ASK ઇનપુટ] {-{-} B[બેન્ડ{-}પાસ ફિલ્ટર]}
    B {-{-} C[એન્વેલપ ડિટેક્ટર]}
    C {-{-} D[લો{-}પાસ ફિલ્ટર]}
    D {-{-} E[કમ્પેરેટર]}
    E {-{-} F[ડિજિટલ આઉટપુટ]}
\end{verbatim}

\textbf{વેવફોર્મ:}

\begin{verbatim}
Digital Input
   \_    \_\_     \_
  | |  |  |   | |
\_\_|\_|\_\_|  |\_\_\_|\_|\_\_\_

Carrier Signal
 /{/////////}

ASK Output
     /{/    ///}
    /    {  /      }
\_\_\_/      {/        \_\_\_}
\end{verbatim}


{\def\LTcaptype{none} % do not increment counter
\vspace{-5pt}
\captionof{table}{ASK મોડ્યુલેશન અને ડિમોડ્યુલેશન પ્રક્રિયા}
\vspace{-10pt}
\begin{longtable}[]{@{}lll@{}}
\toprule\noalign{}
પ્રક્રિયા & કાર્ય & ગાણિતિક રજૂઆત \\
\midrule\noalign{}
\endhead
\bottomrule\noalign{}
\endlastfoot
\textbf{મોડ્યુલેશન} & કેરિયરની એમ્પ્લિટ્યુડમાં ફેરફાર & s(t) =
A·m(t)·cos(2πf\_c·t) \\
\textbf{ફિલ્ટરિંગ} & બેન્ડની બહારનો નોઇઝ દૂર કરે છે & f\_c પર કેન્દ્રિત બેન્ડપાસ
ફિલ્ટર \\
\textbf{ડિટેક્શન} & એન્વેલપ પુનઃપ્રાપ્ત કરે છે & ડાયોડ અને કેપેસિટરનો ઉપયોગ \\
\textbf{નિર્ણય} & ડિજિટલમાં રૂપાંતરિત કરે છે & થ્રેશોલ્ડ કમ્પેરિઝન \\
\end{longtable}
}

\begin{itemize}
\tightlist
\item
  \textbf{બાઇનરી ASK}: `1' માટે કેરિયર હાજર, `0' માટે ગેરહાજર
\item
  \textbf{બેન્ડવિડ્થ}: ન્યૂનતમ BW = બિટ રેટ, સામાન્ય રીતે બેવડો બિટ રેટ વપરાય છે
\end{itemize}

\end{solutionbox}
\begin{mnemonicbox}
``એએમપીએસ'' - \textbf{એ}એસકે કેરિયર \textbf{એ}મ્પ્લિટ્યુડને
ડિજિટલ સિગ્નલ સાથે \textbf{મો}ડ્યુલેટ કરે છે

\end{mnemonicbox}
\subsection*{પ્રશ્ન 2(અ) OR [3
ગુણ]}\label{uxaaauxab0uxab6uxaa8-2uxa85-or-3-uxa97uxaa3}

\textbf{સિગ્નલમાં 4000 બીટ/સેકન્ડનો બીટ રેટ અને 1000 બોડનો બોડ રેટ હોય છે. દરેક
સિગ્નલ એલીમેન્ટ દ્વારા કેટલા ડેટા એલીમેન્ટ વહન કરવામાં આવે છે?}

\begin{solutionbox}

\textbf{ઉકેલ:}

\begin{verbatim}
સિગ્નલ એલિમેન્ટ દીઠ બિટ્સની સંખ્યા = બિટ રેટ / બોડ રેટ
સિગ્નલ એલિમેન્ટ દીઠ બિટ્સની સંખ્યા = 4000 બિટ્સ/સેકન્ડ / 1000 સિગ્નલ એલિમેન્ટ/સેકન્ડ
સિગ્નલ એલિમેન્ટ દીઠ બિટ્સની સંખ્યા = 4 બિટ્સ/સિગ્નલ એલિમેન્ટ
\end{verbatim}

\textbf{આકૃતિ:}

\begin{center}
\textbf{Mermaid Diagram (Code)}
\begin{verbatim}
{Shaded}
{Highlighting}[]
graph LR
    A[બિટ રેટ: 4000 bps] {-{-}{} C[ભાગ કરો]}
    B[બોડ રેટ: 1000 બોડ] {-{-}{} C}
    C {-{-}{} D[4 બિટ્સ/સિગ્નલ એલિમેન્ટ]}
{Highlighting}
{Shaded}
\end{verbatim}
\end{center}

\begin{itemize}
\tightlist
\item
  \textbf{બિટ રેટ}: બિટ્સ પ્રતિ સેકંડમાં ડેટા ટ્રાન્સમિશન સ્પીડ
\item
  \textbf{બોડ રેટ}: સિગ્નલ એલિમેન્ટ્સ (સિમ્બોલ્સ) પ્રતિ સેકંડની સંખ્યા
\end{itemize}

\end{solutionbox}
\begin{mnemonicbox}
``બીબીઆર'' - સિમ્બોલ દીઠ \textbf{બી}ટ્સ બરાબર
\textbf{બી}ટ રેટ ભાગ્યા બોડ \textbf{ર}ેટ

\end{mnemonicbox}
\subsection*{પ્રશ્ન 2(બ) OR [4
ગુણ]}\label{uxaaauxab0uxab6uxaa8-2uxaac-or-4-uxa97uxaa3}

\textbf{વિવિધ સંચાર ચેનલોની લાક્ષણિકતાઓની ચર્ચા કરો.}

\begin{solutionbox}


{\def\LTcaptype{none} % do not increment counter
\vspace{-5pt}
\captionof{table}{સંચાર ચેનલની લાક્ષણિકતાઓ}
\vspace{-10pt}
\begin{longtable}[]{@{}
  >{\raggedright\arraybackslash}p{(\linewidth - 4\tabcolsep) * \real{0.3902}}
  >{\raggedright\arraybackslash}p{(\linewidth - 4\tabcolsep) * \real{0.3171}}
  >{\raggedright\arraybackslash}p{(\linewidth - 4\tabcolsep) * \real{0.2927}}@{}}
\toprule\noalign{}
\begin{minipage}[b]{\linewidth}\raggedright
લાક્ષણિકતા
\end{minipage} & \begin{minipage}[b]{\linewidth}\raggedright
વર્ણન
\end{minipage} & \begin{minipage}[b]{\linewidth}\raggedright
મહત્વ
\end{minipage} \\
\midrule\noalign{}
\endhead
\bottomrule\noalign{}
\endlastfoot
\textbf{બેન્ડવિડ્થ} & ચેનલ ટ્રાન્સમિટ કરી શકે તેવી ફ્રિક્વન્સીઓની રેન્જ & મહત્તમ ડેટા
રેટ નક્કી કરે છે \\
\textbf{નોઇઝ} & અનચાહ્યા સિગ્નલ્સ જે ટ્રાન્સમિશનને બગાડે છે & સિગ્નલ ક્વોલિટી અને ભૂલ
દરને અસર કરે છે \\
\textbf{એટેન્યુએશન} & ટ્રાન્સમિશન દરમિયાન સિગ્નલ સ્ટ્રેન્થની ઘટાડો & ટ્રાન્સમિશન
અંતરને મર્યાદિત કરે છે \\
\textbf{ડિસ્ટોર્શન} & સિગ્નલના આકાર/ટાઈમિંગમાં ફેરફાર & ઇન્ટરસિમ્બોલ ઇન્ટરફેરન્સ
કારણે બને છે \\
\textbf{ચેનલ કેપેસિટી} & મનસ્વી નાના એરર સાથે મહત્તમ ડેટા રેટ & શેનનના થિયરમ
દ્વારા આપવામાં આવે છે \\
\end{longtable}
}

\textbf{આકૃતિ:}

\begin{center}
\textbf{Mermaid Diagram (Code)}
\begin{verbatim}
{Shaded}
{Highlighting}[]
graph TD
    A[ચેનલ લાક્ષણિકતાઓ] {-{-}{} B[બેન્ડવિડ્થ]}
    A {-{-}{} C[નોઇઝ]}
    A {-{-}{} D[એટેન્યુએશન]}
    A {-{-}{} E[ડિસ્ટોર્શન]}
    A {-{-}{} F[ચેનલ કેપેસિટી]}
    C {-{-}{} G[SNR]}
    B {-{-}{} H[ડેટા રેટ]}
    F {-{-}{} H}
{Highlighting}
{Shaded}
\end{verbatim}
\end{center}

\begin{itemize}
\tightlist
\item
  \textbf{SNR (સિગ્નલ-ટુ-નોઇઝ રેશિયો)}: સિગ્નલ પાવર અને નોઇઝ પાવરનો ગુણોત્તર
\item
  \textbf{ચેનલ કેપેસિટી}: C = B·log_{2}(1+SNR), જ્યાં B એ બેન્ડવિડ્થ છે
\end{itemize}

\end{solutionbox}
\begin{mnemonicbox}
``બએનડક'' - \textbf{બે}ન્ડવિડ્થ, \textbf{એ}ટેન્યુએશન,
\textbf{ન}ોઇઝ, \textbf{ડિ}સ્ટોર્શન \textbf{ક}ેપેસિટી નક્કી કરે છે

\end{mnemonicbox}
\subsection*{પ્રશ્ન 2(ક) OR [7
ગુણ]}\label{uxaaauxab0uxab6uxaa8-2uxa95-or-7-uxa97uxaa3}

\textbf{ASK, FSK અને PSK ની સરખામણી કરો.}

\begin{solutionbox}


{\def\LTcaptype{none} % do not increment counter
\vspace{-5pt}
\captionof{table}{ડિજિટલ મોડ્યુલેશન ટેકનિક્સની સરખામણી}
\vspace{-10pt}
\begin{longtable}[]{@{}
  >{\raggedright\arraybackslash}p{(\linewidth - 6\tabcolsep) * \real{0.4231}}
  >{\raggedright\arraybackslash}p{(\linewidth - 6\tabcolsep) * \real{0.1923}}
  >{\raggedright\arraybackslash}p{(\linewidth - 6\tabcolsep) * \real{0.1923}}
  >{\raggedright\arraybackslash}p{(\linewidth - 6\tabcolsep) * \real{0.1923}}@{}}
\toprule\noalign{}
\begin{minipage}[b]{\linewidth}\raggedright
પેરામીટર
\end{minipage} & \begin{minipage}[b]{\linewidth}\raggedright
ASK
\end{minipage} & \begin{minipage}[b]{\linewidth}\raggedright
FSK
\end{minipage} & \begin{minipage}[b]{\linewidth}\raggedright
PSK
\end{minipage} \\
\midrule\noalign{}
\endhead
\bottomrule\noalign{}
\endlastfoot
\textbf{સિદ્ધાંત} & એમ્પ્લિટ્યુડમાં ફેરફાર & ફ્રિક્વન્સીમાં ફેરફાર & ફેઝમાં ફેરફાર \\
\textbf{ગાણિતિક અભિવ્યક્તિ} & s(t) = A·m(t)·cos(2πf\_c·t) & s(t) =
A·cos(2π[f\_c+m(t)Δf]t) & s(t) = A·cos(2πf\_c·t+m(t)·π) \\
\textbf{બેન્ડવિડ્થ} & r\_b (ન્યૂનતમ) & 2(Δf+r\_b/2) & 2r\_b \\
\textbf{પાવર એફિશિયન્સી} & નબળી & મધ્યમ & સારી \\
\textbf{નોઇઝ ઇમ્યુનિટી} & નબળી & વધુ સારી & શ્રેષ્ઠ \\
\textbf{અમલીકરણ જટિલતા} & સરળ & મધ્યમ & જટિલ \\
\textbf{ઉપયોગો} & ઓછી કિંમતની સિસ્ટમ્સ & નોઇઝવાળા વાતાવરણ & ઉચ્ચ
કાર્યક્ષમતાવાળી સિસ્ટમ્સ \\
\end{longtable}
}

\textbf{આકૃતિ:}

\begin{verbatim}
Digital Input:
   \_    \_\_     \_
  | |  |  |   | |
\_\_|\_|\_\_|  |\_\_\_|\_|\_\_\_

ASK:
     /{/    ///}
    /    {  /      }
\_\_\_/      {/        \_\_\_}

FSK:
 /{//      ///}
/      {    /      }
        {//        //}

PSK:
 /{//////////}
/  {  /    /    /  }
    {/    /    /    }
\end{verbatim}

\begin{itemize}
\tightlist
\item
  \textbf{બિટ એરર રેટ}: PSK \textless{} FSK \textless{} ASK (PSK શ્રેષ્ઠ છે)
\item
  \textbf{જટિલતા ક્રમ}: ASK \textless{} FSK \textless{} PSK (ASK સૌથી સરળ
  છે)
\end{itemize}

\end{solutionbox}
\begin{mnemonicbox}
``એફપી'' - \textbf{એ}મ્પ્લિટ્યુડ, \textbf{ફ્રી}ક્વન્સી,
\textbf{ફે}ઝ - ASK, FSK, PSK માં સંશોધિત થાય છે

\end{mnemonicbox}
\subsection*{પ્રશ્ન 3(અ) [3
ગુણ]}\label{uxaaauxab0uxab6uxaa8-3uxa85-3-uxa97uxaa3}

\textbf{બ્લોક ડાયાગ્રામ અને આઉટપુટ વેવફોર્મ સાથે FSK મોડ્યુલેટરનું કાર્ય સમજાવો.}

\begin{solutionbox}

\textbf{FSK મોડ્યુલેટર બ્લોક ડાયાગ્રામ:}

\begin{verbatim}
flowchart LR
    A[ડિજિટલ ઇનપુટ] {-{-} B[સ્વિચ કંટ્રોલર]}
    B {-{-} C[ઓસીલેટર 1 {-} f1]}
    B {-{-} D[ઓસીલેટર 2 {-} f2]}
    C {-{-} E[આઉટપુટ]}
    D {-{-} E}
\end{verbatim}

\textbf{વેવફોર્મ:}

\begin{verbatim}
Digital Input:
   \_    \_      
  | |  | |     
\_\_|\_|\_\_|\_|\_\_\_\_\_

FSK Output:
 /{//    ///}
/      {  /      }
        {/        }
        /{        /}
       /  {      /  }
\end{verbatim}


{\def\LTcaptype{none} % do not increment counter
\vspace{-5pt}
\captionof{table}{FSK મોડ્યુલેશન પ્રક્રિયા}
\vspace{-10pt}
\begin{longtable}[]{@{}
  >{\raggedright\arraybackslash}p{(\linewidth - 2\tabcolsep) * \real{0.3158}}
  >{\raggedright\arraybackslash}p{(\linewidth - 2\tabcolsep) * \real{0.6842}}@{}}
\toprule\noalign{}
\begin{minipage}[b]{\linewidth}\raggedright
સ્ટેપ
\end{minipage} & \begin{minipage}[b]{\linewidth}\raggedright
વર્ણન
\end{minipage} \\
\midrule\noalign{}
\endhead
\bottomrule\noalign{}
\endlastfoot
\textbf{ડિજિટલ ઇનપુટ} & બાઇનરી ડેટા (0s અને 1s) \\
\textbf{ફ્રિક્વન્સી પસંદગી} & f_{1} બિટ `1' માટે, f_{2} બિટ `0' માટે \\
\textbf{વેવફોર્મ જનરેશન} & s(t) = A·cos(2πf_{1}t) બિટ `1' માટે, s(t) =
A·cos(2πf_{2}t) બિટ `0' માટે \\
\textbf{આઉટપુટ} & સતત ફેઝ ફ્રિક્વન્સી-શિફ્ટેડ સિગ્નલ \\
\end{longtable}
}

\begin{itemize}
\tightlist
\item
  \textbf{બાઇનરી FSK}: બે ફ્રિક્વન્સી f_{1} અને f_{2} વપરાય છે જે ફ્રિક્વન્સી ડેવિએશન
  દ્વારા અલગ પડે છે
\item
  \textbf{ફાયદો}: ASK કરતાં વધુ સારી નોઇઝ ઇમ્યુનિટી
\end{itemize}

\end{solutionbox}
\begin{mnemonicbox}
``ફઆફાસ્ટ'' - \textbf{ફ્રી}ક્વન્સી \textbf{આ}વર્તન
\textbf{ફ}રક \textbf{સ્વ}ર વચ્ચે બદલાય છે

\end{mnemonicbox}
\subsection*{પ્રશ્ન 3(બ) [4
ગુણ]}\label{uxaaauxab0uxab6uxaa8-3uxaac-4-uxa97uxaa3}

\textbf{1010110110 ના ક્રમ માટે PSK મોડ્યુલેશન વેવફોર્મ દોરો.}

\begin{solutionbox}

\textbf{1010110110 માટે BPSK મોડ્યુલેશન:}

\begin{verbatim}
Digital Input:
   \_    \_   \_\_\_   \_ \_
  | |  | | |   | | | |
\_\_| |\_\_| |\_|   |\_| | |\_\_

Carrier Signal:
 /{//////////}

BPSK Output:
 /{// /// /// /}
      {/      /      /}
      /{/// //// /}
                      {/}
\end{verbatim}


{\def\LTcaptype{none} % do not increment counter
\vspace{-5pt}
\captionof{table}{BPSK મેપિંગ}
\vspace{-10pt}
\begin{longtable}[]{@{}lll@{}}
\toprule\noalign{}
બિટ & ફેઝ & અર્થઘટન \\
\midrule\noalign{}
\endhead
\bottomrule\noalign{}
\endlastfoot
\textbf{1} & 0^\circ & કેરિયર સાથે ઇન-ફેઝ (પોઝિટીવ) \\
\textbf{0} & 180^\circ & કેરિયરથી આઉટ-ઓફ-ફેઝ (નેગેટિવ) \\
\end{longtable}
}

\textbf{આકૃતિ:}

\begin{center}
\textbf{Mermaid Diagram (Code)}
\begin{verbatim}
{Shaded}
{Highlighting}[]
graph LR
    A[બિટ સ્ટ્રીમ 1010110110] {-{-}{} B[ફેઝ મેપિંગ]}
    B {-{-}{} C[1=0^ ફેઝ]}
    B {-{-}{} D[0=180^ ફેઝ]}
    C {-{-}{} E[મોડ્યુલેટેડ સિગ્નલ]}
    D {-{-}{} E}
{Highlighting}
{Shaded}
\end{verbatim}
\end{center}

\begin{itemize}
\tightlist
\item
  \textbf{ફેઝ શિફ્ટ}: દરેક બિટ બદલાવ પર 180^\circ ફેરફાર
\item
  \textbf{સ્થિર એમ્પ્લિટ્યુડ}: ASKથી વિપરીત, એમ્પ્લિટ્યુડ સ્થિર રહે છે
\end{itemize}

\end{solutionbox}
\begin{mnemonicbox}
``ફોફા'' - \textbf{ફે}ઝ વિરુદ્ધાર્થી બિટ \textbf{જો}ડી
માટે \textbf{ફ}ીચર \textbf{આ}પે છે

\end{mnemonicbox}
\subsection*{પ્રશ્ન 3(ક) [7
ગુણ]}\label{uxaaauxab0uxab6uxaa8-3uxa95-7-uxa97uxaa3}

\textbf{1100110101 ના ક્રમ માટે ASK અને FSK મોડ્યુલેશન વેવફોર્મ દોરો.}

\begin{solutionbox}

\textbf{ઇનપુટ બિટ સિક્વન્સ: 1100110101}

\textbf{ASK મોડ્યુલેશન:}

\begin{verbatim}
Digital Input:
   \_\_    \_\_    \_  \_
  |  |  |  |  | || |
\_\_|  |\_\_|  |\_\_| ||\_|\_\_

ASK Output:
 /{///    ////  // /}
         {  /         /     }
         /\_\_\_\_\_\_\_\_\_\_\_\_\_\_      {\_}
\end{verbatim}

\textbf{FSK મોડ્યુલેશન:}

\begin{verbatim}
Digital Input:
   \_\_    \_\_    \_  \_
  |  |  |  |  | || |
\_\_|  |\_\_|  |\_\_| ||\_|\_\_

FSK Output (f1=high, f0=low):
 /{///        ////      //    }
        {      /            /      /}
         {    /            /      /  }
          {//            /            }
           Higher freq     Higher freq   Higher freq
           for 1s          for 1s        for 1s
         
             Lower freq      Lower freq    Lower freq
             for 0s          for 0s        for 0s
\end{verbatim}


{\def\LTcaptype{none} % do not increment counter
\vspace{-5pt}
\captionof{table}{1100110101 સિક્વન્સ માટે તુલના}
\vspace{-10pt}
\begin{longtable}[]{@{}llll@{}}
\toprule\noalign{}
બિટ પોઝિશન & બિટ વેલ્યુ & ASK રજૂઆત & FSK રજૂઆત \\
\midrule\noalign{}
\endhead
\bottomrule\noalign{}
\endlastfoot
\textbf{1-2} & 11 & કેરિયર હાજર & ઉચ્ચ ફ્રિક્વન્સી \\
\textbf{3-4} & 00 & કેરિયર ગેરહાજર & નીચી ફ્રિક્વન્સી \\
\textbf{5-7} & 110 & કેરિયર હાજર/ગેરહાજર & ઉચ્ચ/નીચી ફ્રિક્વન્સી \\
\textbf{8-10} & 101 & કેરિયર હાજર/ગેરહાજર/હાજર & ઉચ્ચ/નીચી/ઉચ્ચ ફ્રિક્વન્સી \\
\end{longtable}
}

\begin{itemize}
\tightlist
\item
  \textbf{ASK મોડ્યુલેશન}: સરળ ઓન-ઓફ કીઇંગ જ્યાં `1' માટે કેરિયર હાજર અને `0' માટે
  ગેરહાજર હોય છે
\item
  \textbf{FSK મોડ્યુલેશન}: બિટ વેલ્યુના આધારે બે અલગ-અલગ મૂલ્યો વચ્ચે ફ્રિક્વન્સી શિફ્ટ
  થાય છે
\end{itemize}

\end{solutionbox}
\begin{mnemonicbox}
``એબફ્ફ'' - \textbf{એ}એસકે કેરિયર
\textbf{બં}ધ-\textbf{ચા}લુ કરે છે, જ્યારે \textbf{ફ્રી}ક્વન્સી \textbf{જો}ડી વચ્ચે
FSK શિફ્ટ કરે છે

\end{mnemonicbox}
\subsection*{પ્રશ્ન 3(અ) OR [3
ગુણ]}\label{uxaaauxab0uxab6uxaa8-3uxa85-or-3-uxa97uxaa3}

\textbf{બ્લોક ડાયાગ્રામ અને આઉટપુટ વેવફોર્મ સાથે MSK મોડ્યુલેટરનું કાર્ય સમજાવો.}

\begin{solutionbox}

\textbf{MSK મોડ્યુલેટર બ્લોક ડાયાગ્રામ:}

\begin{verbatim}
flowchart LR
    A[ડિજિટલ ઇનપુટ] {-{-} B[સીરિયલ ટુ પેરેલલ]}
    B {-{-} C[ઈવન બિટ્સ]}
    B {-{-} D[ઓડ બિટ્સ]}
    C {-{-} E[કોસ મોડ્યુલેટર]}
    D {-{-} F[સાઇન મોડ્યુલેટર]}
    G[90^ ફેઝ શિફ્ટર] {-{-} F}
    H[કેરિયર જનરેટર] {-{-} E}
    H {-{-} G}
    E {-{-} I[કોમ્બાઇનર]}
    F {-{-} I}
    I {-{-} J[MSK આઉટપુટ]}
\end{verbatim}

\textbf{વેવફોર્મ:}

\begin{verbatim}
Digital Input:
   \_    \_      
  | |  | |     
\_\_|\_|\_\_|\_|\_\_\_\_\_

MSK Output:
  \_{-\_     \_{-}\_  }
 /   {   /    }
/     {\_/     \_}
       \_{-\_     \_{-}}
      /   {   /  }
     /     {\_/    }
\end{verbatim}


{\def\LTcaptype{none} % do not increment counter
\vspace{-5pt}
\captionof{table}{MSK મોડ્યુલેશન પ્રક્રિયા}
\vspace{-10pt}
\begin{longtable}[]{@{}ll@{}}
\toprule\noalign{}
લાક્ષણિકતા & વર્ણન \\
\midrule\noalign{}
\endhead
\bottomrule\noalign{}
\endlastfoot
\textbf{સિદ્ધાંત} & સાઇન્યુસોઇડલ પલ્સ શેપિંગ સાથે OQPSKનો ખાસ કેસ \\
\textbf{ફેઝ સાતત્ય} & સરળ ફેઝ ટ્રાન્ઝિશન સુનિશ્ચિત કરે છે (અચાનક ફેઝ ફેરફાર નહીં) \\
\textbf{ફ્રિક્વન્સી ડેવિએશન} & કેરિયર ફ્રિક્વન્સીથી \pm0.25 બિટ રેટ \\
\textbf{બેન્ડવિડ્થ એફિશિયન્સી} & પરંપરાગત FSK કરતાં વધારે સારી \\
\end{longtable}
}

\begin{itemize}
\tightlist
\item
  \textbf{ફેઝ સાતત્ય}: મુખ્ય ફાયદો - FSKની તુલનામાં બેન્ડવિડ્થ ઘટાડે છે
\item
  \textbf{સ્થિર એન્વેલપ}: નોન-લિનિયર એમ્પ્લિફિકેશન પ્રત્યે પ્રતિરોધક
\end{itemize}

\end{solutionbox}
\begin{mnemonicbox}
``એમસફ'' - \textbf{એમ}એસકે \textbf{સ}તત \textbf{ફે}ઝ
શિફ્ટ્સ સુનિશ્ચિત કરે છે

\end{mnemonicbox}
\subsection*{પ્રશ્ન 3(બ) OR [4
ગુણ]}\label{uxaaauxab0uxab6uxaa8-3uxaac-or-4-uxa97uxaa3}

\textbf{8-PSK અને 16-QAM ના નક્ષત્ર રેખાકૃતિ દોરો.}

\begin{solutionbox}

\textbf{8-PSK કોન્સ્ટેલેશન ડાયાગ્રામ:}

\begin{verbatim}
       001 * * 000
           {|/}
    010 *{-{-}{-}+{-}{-}{-}* 111}
           /|{}
       011 * * 101
           100
\end{verbatim}

\textbf{16-QAM કોન્સ્ટેલેશન ડાયાગ્રામ:}

\begin{verbatim}
    *   *   *   *
    
    *   *   *   *
    
    *   *   *   *
    
    *   *   *   *
\end{verbatim}


{\def\LTcaptype{none} % do not increment counter
\vspace{-5pt}
\captionof{table}{કોન્સ્ટેલેશન ડાયાગ્રામ્સની તુલના}
\vspace{-10pt}
\begin{longtable}[]{@{}lll@{}}
\toprule\noalign{}
પેરામીટર & 8-PSK & 16-QAM \\
\midrule\noalign{}
\endhead
\bottomrule\noalign{}
\endlastfoot
\textbf{સિમ્બોલ દીઠ બિટ્સ} & 3 બિટ્સ & 4 બિટ્સ \\
\textbf{સિમ્બોલ પોઝિશન્સ} & વર્તુળ પર 8 પોઇન્ટ્સ & ગ્રિડમાં 16 પોઇન્ટ્સ \\
\textbf{એમ્પ્લિટ્યુડ લેવલ્સ} & 1 (સ્થિર) & 3 (વેરિએબલ) \\
\textbf{ફેઝ એંગલ્સ} & 8 ખૂણા (45^\circ તફાવત) & 12 ખૂણા \\
\textbf{એરર સેન્સિટિવિટી} & મધ્યમ & 8-PSK કરતાં વધારે \\
\textbf{સ્પેક્ટ્રલ એફિશિયન્સી} & 3 બિટ્સ/Hz & 4 બિટ્સ/Hz \\
\end{longtable}
}

\begin{itemize}
\tightlist
\item
  \textbf{8-PSK}: સમાન એમ્પ્લિટ્યુડ સાથે વર્તુળની આસપાસ સમાન અંતરે પોઇન્ટ્સ
\item
  \textbf{16-QAM}: અલગ-અલગ એમ્પ્લિટ્યુડ અને ફેઝ સાથે ચોરસ ગ્રિડમાં પોઇન્ટ્સ
  ગોઠવાયેલા હોય છે
\end{itemize}

\end{solutionbox}
\begin{mnemonicbox}
``સીપા'' - \textbf{કો}ન્સ્ટેલેશન પોઇન્ટ્સ PSKમાં સમાન
\textbf{એ}મ્પ્લિટ્યુડ પરંતુ અલગ \textbf{ફે}ઝ ધરાવે છે, QAMમાં \textbf{એ}મ્પ્લિટ્યુડ અને
ફેઝ બંને ફેરફાર ધરાવે છે

\end{mnemonicbox}
\subsection*{પ્રશ્ન 3(ક) OR [7
ગુણ]}\label{uxaaauxab0uxab6uxaa8-3uxa95-or-7-uxa97uxaa3}

\textbf{1010101011 માટે BPSK અને QPSK મોડ્યુલેશન વેવફોર્મ દોરો.}

\begin{solutionbox}

\textbf{ઇનપુટ બિટ સિક્વન્સ: 1010101011}

\textbf{BPSK મોડ્યુલેશન:}

\begin{verbatim}
Digital Input:
   \_ \_ \_ \_ \_ \_ \_ \_
  | | | | | | | | |
\_\_| |\_| |\_| |\_| |\_| |\_\_

BPSK Output:
 /{// /// /// /// /}
      {/      /      /      /}
      /{/// //// ////}
\end{verbatim}

\textbf{QPSK મોડ્યુલેશન (બિટ્સ ગ્રુપિંગ: 10,10,10,10,11):}

\begin{verbatim}
Grouped Bits:
   10    10    10    10    11
   
I{-channel (odd bits):}
   \_     \_     \_     \_     \_
  | |   | |   | |   | |   | |
\_\_| |\_\_\_| |\_\_\_| |\_\_\_| |\_\_\_| |\_\_

Q{-channel (even bits):}
    \_     \_     \_     \_      
   | |   | |   | |   | |    |
\_\_\_| |\_\_\_| |\_\_\_| |\_\_\_| |\_\_\_\_|

QPSK Output:
 /{  /  /  /  /}
/  {/  /  /  /  }
    Phase    Phase   Different 
    00       00      phase for 11
\end{verbatim}


{\def\LTcaptype{none} % do not increment counter
\vspace{-5pt}
\captionof{table}{1010101011 માટે BPSK અને QPSK ની તુલના}
\vspace{-10pt}
\begin{longtable}[]{@{}lll@{}}
\toprule\noalign{}
લાક્ષણિકતા & BPSK & QPSK \\
\midrule\noalign{}
\endhead
\bottomrule\noalign{}
\endlastfoot
\textbf{સિમ્બોલ દીઠ બિટ્સ} & 1 & 2 \\
\textbf{સિમ્બોલની સંખ્યા} & 10 & 5 \\
\textbf{સિમ્બોલ રેટ} & બિટ રેટ જેટલો જ & બિટ રેટનો અર્ધો \\
\textbf{બેન્ડવિડ્થ એફિશિયન્સી} & 1 બિટ/Hz & 2 બિટ્સ/Hz \\
\textbf{ફેઝ સ્ટેટ્સ} & 2 (0^\circ, 180^\circ) & 4 (45^\circ, 135^\circ, 225^\circ, 315^\circ) \\
\end{longtable}
}

\begin{itemize}
\tightlist
\item
  \textbf{BPSK}: દરેક બિટ 180^\circ ફેઝ શિફ્ટ લાવી શકે છે
\item
  \textbf{QPSK}: એક સાથે બે બિટ પ્રોસેસ કરે છે, ચાર ફેઝ સ્ટેટ્સ વાપરે છે
\end{itemize}

\end{solutionbox}
\begin{mnemonicbox}
``બીક્ય્સસ'' - \textbf{બી}પીએસકે \textbf{1} બિટ લે છે
જ્યારે \textbf{ક્યુ}પીએસકે \textbf{2} બિટ લે છે, જેનાથી \textbf{સ્પે}ક્ટ્રલ
\textbf{સ}ક્ષમતા બમણી થાય છે

\end{mnemonicbox}
\subsection*{પ્રશ્ન 4(અ) [3
ગુણ]}\label{uxaaauxab0uxab6uxaa8-4uxa85-3-uxa97uxaa3}

\textbf{નીચેના સંભવિત ક્રમ માટે શેનોન ફેનો કોડનો ઉપયોગ કરીને ડેટાને એન્કોડ કરો. P
= \{ 0.30, 0.25, 0.20, 0.12, 0.08, 0.05\}}

\begin{solutionbox}


{\def\LTcaptype{none} % do not increment counter
\vspace{-5pt}
\captionof{table}{શેનન-ફેનો કોડિંગ પ્રક્રિયા}
\vspace{-10pt}
\begin{longtable}[]{@{}llll@{}}
\toprule\noalign{}
સિમ્બોલ & પ્રોબેબિલિટી & ડિવિઝન સ્ટેપ્સ & શેનન-ફેનો કોડ \\
\midrule\noalign{}
\endhead
\bottomrule\noalign{}
\endlastfoot
\textbf{A} & 0.30 & ટોપ ગ્રુપ & 0 \\
\textbf{B} & 0.25 & ટોપ ગ્રુપ & 10 \\
\textbf{C} & 0.20 & બોટમ ગ્રુપ & 110 \\
\textbf{D} & 0.12 & બોટમ ગ્રુપ & 1110 \\
\textbf{E} & 0.08 & બોટમ ગ્રુપ & 1111 0 \\
\textbf{F} & 0.05 & બોટમ ગ્રુપ & 1111 1 \\
\end{longtable}
}

\textbf{આકૃતિ:}

\begin{center}
\textbf{Mermaid Diagram (Code)}
\begin{verbatim}
{Shaded}
{Highlighting}[]
graph TD
    A[સિમ્બોલ્સ] {-{-}{} B[A:0.30, B:0.25, C:0.20, D:0.12, E:0.08, F:0.05]}
    B {-{-}{} C[A:0.30, B:0.25]}
    B {-{-}{} D[C:0.20, D:0.12, E:0.08, F:0.05]}
    C {-{-}{} E[A:0.30]}
    C {-{-}{} F[B:0.25]}
    D {-{-}{} G[C:0.20, D:0.12]}
    D {-{-}{} H[E:0.08, F:0.05]}
    G {-{-}{} I[C:0.20]}
    G {-{-}{} J[D:0.12]}
    H {-{-}{} K[E:0.08]}
    H {-{-}{} L[F:0.05]}
    E {-{-}{} M[કોડ: 0]}
    F {-{-}{} N[કોડ: 10]}
    I {-{-}{} O[કોડ: 110]}
    J {-{-}{} P[કોડ: 1110]}
    K {-{-}{} Q[કોડ: 11110]}
    L {-{-}{} R[કોડ: 11111]}
{Highlighting}
{Shaded}
\end{verbatim}
\end{center}

\begin{itemize}
\tightlist
\item
  \textbf{શેનન-ફેનો એલ્ગોરિધમ}: લગભગ સમાન પ્રોબેબિલિટી ધરાવતા બે જૂથોમાં
  રિકર્સિવલી સિમ્બોલ્સને વિભાજિત કરે છે
\item
  \textbf{કોડ એફિશિયન્સી}: હંમેશા શ્રેષ્ઠ ન હોય શકે પરંતુ સામાન્ય રીતે સારું કોમ્પ્રેશન
\end{itemize}

\end{solutionbox}
\begin{mnemonicbox}
``સપઆઅ'' - \textbf{સં}ભાવના \textbf{પ્ર}માણે
\textbf{અં}કો \textbf{આ}વૃત્તિ આધારિત ફાળવાય છે

\end{mnemonicbox}
\subsection*{પ્રશ્ન 4(બ) [4
ગુણ]}\label{uxaaauxab0uxab6uxaa8-4uxaac-4-uxa97uxaa3}

\textbf{હેમિંગ કોડ સમજાવો.}

\begin{solutionbox}


{\def\LTcaptype{none} % do not increment counter
\vspace{-5pt}
\captionof{table}{હેમિંગ કોડના ગુણધર્મો}
\vspace{-10pt}
\begin{longtable}[]{@{}ll@{}}
\toprule\noalign{}
ગુણધર્મ & વર્ણન \\
\midrule\noalign{}
\endhead
\bottomrule\noalign{}
\endlastfoot
\textbf{પ્રકાર} & લિનિયર એરર-કરેક્ટિંગ કોડ \\
\textbf{એરર ડિટેક્શન} & 2 બિટ સુધીની ભૂલ શોધી શકે છે \\
\textbf{એરર કરેક્શન} & સિંગલ બિટ ભૂલોને સુધારી શકે છે \\
\textbf{પેરિટી બિટ્સ (r)} & n ડેટા બિટ્સ માટે: 2\^{}r \geq n + r + 1 \\
\textbf{કોડ સ્ટ્રક્ચર} & સિસ્ટેમેટિક: મેસેજ બિટ્સ + પેરિટી બિટ્સ \\
\textbf{પેરિટી બિટ્સની પોઝિશન} & 2ની ઘાત: પોઝિશન 1, 2, 4, 8, 16\ldots{} \\
\end{longtable}
}

\textbf{આકૃતિ:}

\begin{center}
\textbf{Mermaid Diagram (Code)}
\begin{verbatim}
{Shaded}
{Highlighting}[]
graph LR
    A[હેમિંગ કોડ] {-{-}{} B[પેરિટી બિટ્સ]}
    A {-{-}{} C[ડેટા બિટ્સ]}
    B {-{-}{} D[પોઝિશન 1]}
    B {-{-}{} E[પોઝિશન 2]}
    B {-{-}{} F[પોઝિશન 4]}
    B {-{-}{} G[પોઝિશન 8]}
    A {-{-}{} H["ઉદાહરણ: હેમિંગ(7,4)"]}
    H {-{-}{} I[4 ડેટા બિટ્સ + 3 પેરિટી બિટ્સ]}
{Highlighting}
{Shaded}
\end{verbatim}
\end{center}

\begin{itemize}
\tightlist
\item
  \textbf{એનકોડિંગ}: ચોક્કસ બિટ પોઝિશન્સ પર ઇવન/ઓડ પેરિટી સુનિશ્ચિત કરવા માટે
  પેરિટી બિટ્સની ગણતરી
\item
  \textbf{ડિકોડિંગ}: ભૂલની પોઝિશન નક્કી કરવા માટે સિન્ડ્રોમની ગણતરી
\end{itemize}

\end{solutionbox}
\begin{mnemonicbox}
``સાપો'' - \textbf{સ}ત્તાની ઘાત પોઝિશનમાં
\textbf{પે}રિટી બિટ \textbf{સ}િસ્ટેમેટિક રીતે ભૂલ \textbf{સુ}ધાર \textbf{ઓ}ળખે

\end{mnemonicbox}
\subsection*{પ્રશ્ન 4(ક) [7
ગુણ]}\label{uxaaauxab0uxab6uxaa8-4uxa95-7-uxa97uxaa3}

\textbf{TDMA અને FDMA ની સરખામણી કરો.}

\begin{solutionbox}


{\def\LTcaptype{none} % do not increment counter
\vspace{-5pt}
\captionof{table}{TDMA અને FDMA ની તુલના}
\vspace{-10pt}
\begin{longtable}[]{@{}
  >{\raggedright\arraybackslash}p{(\linewidth - 4\tabcolsep) * \real{0.4783}}
  >{\raggedright\arraybackslash}p{(\linewidth - 4\tabcolsep) * \real{0.2609}}
  >{\raggedright\arraybackslash}p{(\linewidth - 4\tabcolsep) * \real{0.2609}}@{}}
\toprule\noalign{}
\begin{minipage}[b]{\linewidth}\raggedright
પેરામીટર
\end{minipage} & \begin{minipage}[b]{\linewidth}\raggedright
TDMA
\end{minipage} & \begin{minipage}[b]{\linewidth}\raggedright
FDMA
\end{minipage} \\
\midrule\noalign{}
\endhead
\bottomrule\noalign{}
\endlastfoot
\textbf{મૂળ સિદ્ધાંત} & સમયને સ્લોટ્સમાં વિભાજિત કરે છે & ફ્રિક્વન્સીને ચેનલ્સમાં
વિભાજિત કરે છે \\
\textbf{રિસોર્સ ફાળવણી} & દરેક વપરાશકર્તાને ટૂંકા સમય માટે સંપૂર્ણ બેન્ડવિડ્થ મળે છે &
દરેક વપરાશકર્તાને સંપૂર્ણ સમય માટે સાંકડી બેન્ડવિડ્થ મળે છે \\
\textbf{ગાર્ડ ટાઇમ/બેન્ડ} & સ્લોટ્સ વચ્ચે ગાર્ડ ટાઇમની જરૂર પડે છે & ચેનલ્સ વચ્ચે ગાર્ડ
બેન્ડની જરૂર પડે છે \\
\textbf{સિન્ક્રોનાઇઝેશન} & અત્યંત મહત્વપૂર્ણ (ટાઇમિંગ-આધારિત) & જરૂરી નથી
(ફ્રિક્વન્સી સેપરેશન) \\
\textbf{એફિશિયન્સી} & બર્સ્ટી ડેટા માટે વધુ સારી & સતત ડેટા માટે વધુ સારી \\
\textbf{ઇન્ટરફેરન્સ} & ઇન્ટરફેરન્સને ઓછો અસરગ્રસ્ત & એડજેસન્ટ ચેનલ ઇન્ટરફેરન્સથી વધુ
અસરગ્રસ્ત \\
\textbf{હાર્ડવેર જટિલતા} & જટિલ (બફરિંગ, સિન્ક્રોનાઇઝેશનની જરૂર) & સરળ (ફિક્સ્ડ
ફિલ્ટર્સ) \\
\textbf{પાવર કન્ઝમ્પશન} & ઓછો (ટ્રાન્સમિટર ફક્ત ટાઇમ સ્લોટ દરમિયાન ચાલુ) & વધારે
(સતત ટ્રાન્સમિશન) \\
\textbf{ક્ષમતા} & ટાઇમ સ્લોટ્સ ઉમેરીને સરળતાથી વિસ્તૃત કરી શકાય & ઉપલબ્ધ
સ્પેક્ટ્રમથી મર્યાદિત \\
\textbf{ઉપયોગો} & GSM, DECT કોર્ડલેસ ફોન & એનાલોગ સેલ્યુલર, સેટેલાઇટ સિસ્ટમ્સ \\
\end{longtable}
}

\textbf{આકૃતિ:}

\begin{center}
\textbf{Mermaid Diagram (Code)}
\begin{verbatim}
{Shaded}
{Highlighting}[]
graph TD
    subgraph TDMA
        A[ટાઇમ સ્લોટ્સ] {-{-}{} A1[યુઝર 1]}
        A {-{-}{} A2[યુઝર 2]}
        A {-{-}{} A3[યુઝર 3]}
        A {-{-}{} A4[ગાર્ડ ટાઇમ]}
    end
    subgraph FDMA
        B[ફ્રિક્વન્સી બેન્ડ્સ] {-{-}{} B1[યુઝર 1]}
        B {-{-}{} B2[યુઝર 2]}
        B {-{-}{} B3[યુઝર 3]}
        B {-{-}{} B4[ગાર્ડ બેન્ડ્સ]}
    end
{Highlighting}
{Shaded}
\end{verbatim}
\end{center}

\begin{itemize}
\tightlist
\item
  \textbf{સિસ્ટમ ફ્લેક્સિબિલિટી}: TDMA ગતિશીલ રીતે સ્લોટ્સ ફાળવી શકે છે, FDMA
  ફિક્સ્ડ એલોકેશન છે
\item
  \textbf{અમલીકરણ}: TDMA માટે ડિજિટલ ટેકનોલોજીની જરૂર પડે છે, FDMA
  એનાલોગ/ડિજિટલ સાથે કામ કરે છે
\end{itemize}

\end{solutionbox}
\begin{mnemonicbox}
``સમયઆ'' - \textbf{સ}મયના \textbf{અં}તરાલોને
\textbf{ટી}ડીએમએ વિભાજિત કરે છે, \textbf{ફ્રિ}ક્વન્સીના \textbf{રે}ન્જને
\textbf{એફ}ડીએમએ વિભાજિત કરે છે

\end{mnemonicbox}
\subsection*{પ્રશ્ન 4(અ) OR [3
ગુણ]}\label{uxaaauxab0uxab6uxaa8-4uxa85-or-3-uxa97uxaa3}

\textbf{નીચેના સંભવિત ક્રમ માટે હફમેન કોડનો ઉપયોગ કરીને ડેટાને એન્કોડ કરો. P = \{
0.4, 0.19, 0.16, 0.15, 0.1\}}

\begin{solutionbox}


{\def\LTcaptype{none} % do not increment counter
\vspace{-5pt}
\captionof{table}{હફમેન કોડિંગ પ્રક્રિયા}
\vspace{-10pt}
\begin{longtable}[]{@{}lll@{}}
\toprule\noalign{}
સિમ્બોલ & પ્રોબેબિલિટી & હફમેન કોડ \\
\midrule\noalign{}
\endhead
\bottomrule\noalign{}
\endlastfoot
\textbf{A} & 0.40 & 0 \\
\textbf{B} & 0.19 & 10 \\
\textbf{C} & 0.16 & 110 \\
\textbf{D} & 0.15 & 111 \\
\textbf{E} & 0.10 & 110 \\
\end{longtable}
}

\textbf{આકૃતિ:}

\begin{center}
\textbf{Mermaid Diagram (Code)}
\begin{verbatim}
{Shaded}
{Highlighting}[]
graph LR
    Z[રૂટ: 1.0] {-{-}{} A[A: 0.4]}
    Z {-{-}{} Y[0.6]}
    Y {-{-}{} B[B: 0.19]}
    Y {-{-}{} X[0.41]}
    X {-{-}{} C[C: 0.16]}
    X {-{-}{} W[0.25]}
    W {-{-}{} D[D: 0.15]}
    W {-{-}{} E[E: 0.1]}
    A {-{-} 0 {-}{-}{} AA[કોડ: 0]}
    B {-{-} 1 {-}{-}{} BB[કોડ: 10]}
    C {-{-} 0 {-}{-}{} CC[કોડ: 110]}
    D {-{-} 0 {-}{-}{} DD[કોડ: 1110]}
    E {-{-} 1 {-}{-}{} EE[કોડ: 1111]}
{Highlighting}
{Shaded}
\end{verbatim}
\end{center}

\begin{itemize}
\tightlist
\item
  \textbf{હફમેન એલ્ગોરિધમ}: ઓછામાં ઓછી સંભાવના ધરાવતા સિમ્બોલ્સથી શરૂઆત કરીને,
  નીચેથી ઉપર બાઇનરી ટ્રી બનાવે છે
\item
  \textbf{ઓપ્ટિમાલિટી}: મિનિમલ એવરેજ કોડ લેન્થ આપે છે
\end{itemize}

\end{solutionbox}
\begin{mnemonicbox}
``હઆસ'' - \textbf{હ}ફમેન ઉચ્ચ \textbf{આ}વૃત્તિના
\textbf{સં}કેતો માટે ટૂંકા કોડ બનાવે છે

\end{mnemonicbox}
\subsection*{પ્રશ્ન 4(બ) OR [4
ગુણ]}\label{uxaaauxab0uxab6uxaa8-4uxaac-or-4-uxa97uxaa3}

\textbf{SNR અને સંચારમાં તેના મહત્વના સંદર્ભમાં ચેનલ ક્ષમતાને વ્યાખ્યાયિત કરો.}

\begin{solutionbox}

\textbf{શેનનનું ચેનલ ક્ષમતા ફોર્મ્યુલા:}

\begin{verbatim}
C = B \times log_{2}(1 + SNR)
\end{verbatim}

જ્યાં:

\begin{itemize}
\tightlist
\item
  C = ચેનલ ક્ષમતા બિટ્સ પ્રતિ સેકન્ડમાં
\item
  B = બેન્ડવિડ્થ Hz માં
\item
  SNR = સિગ્નલ-ટુ-નોઇઝ રેશિયો
\end{itemize}


{\def\LTcaptype{none} % do not increment counter
\vspace{-5pt}
\captionof{table}{ચેનલ ક્ષમતાની લાક્ષણિકતાઓ}
\vspace{-10pt}
\begin{longtable}[]{@{}
  >{\raggedright\arraybackslash}p{(\linewidth - 4\tabcolsep) * \real{0.2424}}
  >{\raggedright\arraybackslash}p{(\linewidth - 4\tabcolsep) * \real{0.3939}}
  >{\raggedright\arraybackslash}p{(\linewidth - 4\tabcolsep) * \real{0.3636}}@{}}
\toprule\noalign{}
\begin{minipage}[b]{\linewidth}\raggedright
પાસું
\end{minipage} & \begin{minipage}[b]{\linewidth}\raggedright
વર્ણન
\end{minipage} & \begin{minipage}[b]{\linewidth}\raggedright
મહત્વ
\end{minipage} \\
\midrule\noalign{}
\endhead
\bottomrule\noalign{}
\endlastfoot
\textbf{વ્યાખ્યા} & શક્ય એરર-ફ્રી ડેટા રેટનું મહત્તમ મૂલ્ય & મૂળભૂત સીમાઓ નક્કી કરે
છે \\
\textbf{SNR પર આધાર} & SNR સાથે લોગેરિધમિક રીતે વધે છે & પાવરના ઘટતા વળતરો
દર્શાવે છે \\
\textbf{બેન્ડવિડ્થ પર આધાર} & બેન્ડવિડ્થ સાથે લિનિયર રીતે વધે છે & સ્પેક્ટ્રમનું મૂલ્ય
દર્શાવે છે \\
\textbf{થિયોરેટિકલ બાઉન્ડ} & કોઈપણ કોડિંગ સાથે શેનન લિમિટને વટાવી શકાતી નથી &
સિસ્ટમ ડિઝાઇનને માર્ગદર્શન આપે છે \\
\end{longtable}
}

\textbf{આકૃતિ:}

\begin{center}
\textbf{Mermaid Diagram (Code)}
\begin{verbatim}
{Shaded}
{Highlighting}[]
graph LR
    A[ચેનલ ક્ષમતા] {-{-}{} B[બેન્ડવિડ્થ B]}
    A {-{-}{} C[સિગ્નલ{-}ટુ{-}નોઇઝ રેશિયો]}
    B {-{-}{} D["C = B  log_{2}(1 + SNR)"]}
    C {-{-}{} D}
    D {-{-}{} E[થિયોરેટિકલ માક્સિમમ]}
    E {-{-}{} F[એરર{-}ફ્રી કોમ્યુનિકેશન]}
{Highlighting}
{Shaded}
\end{verbatim}
\end{center}

\begin{itemize}
\tightlist
\item
  \textbf{શેનન-હાર્ટલી થિયરમ}: ડેટા ટ્રાન્સફર રેટની થિયોરેટિકલ મહત્તમ મર્યાદા
  સ્થાપિત કરે છે
\item
  \textbf{એરર પ્રોબેબિલિટી}: જો ડેટા રેટ \textless{} ચેનલ ક્ષમતા હોય તો મનસ્વી
  રીતે નાની બનાવી શકાય છે
\end{itemize}

\end{solutionbox}
\begin{mnemonicbox}
``શનબ'' - \textbf{શે}નન ક્ષમતા \textbf{ન}ોઇઝ રેશિયો અને
\textbf{બે}ન્ડવિડ્થ પર આધાર રાખે છે

\end{mnemonicbox}
\subsection*{પ્રશ્ન 4(ક) OR [7
ગુણ]}\label{uxaaauxab0uxab6uxaa8-4uxa95-or-7-uxa97uxaa3}

\textbf{FDMA ટેકનિકને વિગતવાર સમજાવો.}

\begin{solutionbox}

\textbf{FDMA (ફ્રિક્વન્સી ડિવિઝન મલ્ટિપલ એક્સેસ)}


{\def\LTcaptype{none} % do not increment counter
\vspace{-5pt}
\captionof{table}{FDMA સિસ્ટમની લાક્ષણિકતાઓ}
\vspace{-10pt}
\begin{longtable}[]{@{}
  >{\raggedright\arraybackslash}p{(\linewidth - 4\tabcolsep) * \real{0.2286}}
  >{\raggedright\arraybackslash}p{(\linewidth - 4\tabcolsep) * \real{0.3714}}
  >{\raggedright\arraybackslash}p{(\linewidth - 4\tabcolsep) * \real{0.4000}}@{}}
\toprule\noalign{}
\begin{minipage}[b]{\linewidth}\raggedright
પાસું
\end{minipage} & \begin{minipage}[b]{\linewidth}\raggedright
વર્ણન
\end{minipage} & \begin{minipage}[b]{\linewidth}\raggedright
મહત્વ
\end{minipage} \\
\midrule\noalign{}
\endhead
\bottomrule\noalign{}
\endlastfoot
\textbf{મૂળ સિદ્ધાંત} & ઉપલબ્ધ સ્પેક્ટ્રમને ચેનલોમાં વિભાજિત કરે છે & અનેક સમકાલીન
વપરાશકર્તાઓને સક્ષમ બનાવે છે \\
\textbf{ચેનલ ફાળવણી} & દરેક વપરાશકર્તા માટે ફિક્સ્ડ ફ્રિક્વન્સી બેન્ડ & હાર્ડવેર
ડિઝાઇનને સરળ બનાવે છે \\
\textbf{ગાર્ડ બેન્ડ્સ} & ચેનલો વચ્ચે ફ્રિક્વન્સી સેપરેશન & એડજેસન્ટ ચેનલ ઇન્ટરફેરન્સને
અટકાવે છે \\
\textbf{ડુપ્લેક્સિંગ} & ઘણીવાર FDD (સેપરેટ Tx/Rx બેન્ડ્સ) સાથે જોડાયેલું & સમકાલીન
બે-માર્ગી સંચારને સક્ષમ બનાવે છે \\
\textbf{બેન્ડવિડ્થ ઉપયોગ} & દરેક ચેનલ ફિક્સ્ડ બેન્ડવિડ્થ ધરાવે છે & બર્સ્ટી ડેટા માટે
સંભવિત રીતે અકાર્યક્ષમ \\
\textbf{ઇન્ટરમોડ્યુલેશન} & મલ્ટિપલ કેરિયર્સના પ્રોડક્ટ્સ & કાળજીપૂર્વક પાવર
એમ્પ્લિફાયર ડિઝાઇનની જરૂર \\
\end{longtable}
}

\textbf{આકૃતિ:}

\begin{center}
\textbf{Mermaid Diagram (Code)}
\begin{verbatim}
{Shaded}
{Highlighting}[]
graph TD
    A[ઉપલબ્ધ સ્પેક્ટ્રમ] {-{-}{} B[ગાર્ડ બેન્ડ]}
    A {-{-}{} C[યુઝર 1 ચેનલ]}
    A {-{-}{} D[ગાર્ડ બેન્ડ]}
    A {-{-}{} E[યુઝર 2 ચેનલ]}
    A {-{-}{} F[ગાર્ડ બેન્ડ]}
    A {-{-}{} G[યુઝર 3 ચેનલ]}
    A {-{-}{} H[ગાર્ડ બેન્ડ]}
    A {-{-}{} I[યુઝર 4 ચેનલ]}
{Highlighting}
{Shaded}
\end{verbatim}
\end{center}

\textbf{FDMA અમલીકરણ:}

\begin{verbatim}
   \^{}
   |
 F |  +{-{-}{-}{-}{-}+  +{-}{-}{-}{-}{-}+  +{-}{-}{-}{-}{-}+  +{-}{-}{-}{-}{-}+}
 r |  |User1|  |User2|  |User3|  |User4|
 e |  |     |  |     |  |     |  |     |
 q |  |     |  |     |  |     |  |     |
   |  +{-{-}{-}{-}{-}+  +{-}{-}{-}{-}{-}+  +{-}{-}{-}{-}{-}+  +{-}{-}{-}{-}{-}+}
   |
   |   Guard    Guard    Guard    
   |   Band     Band     Band     
   +{-{-}{-}{-}{-}{-}{-}{-}{-}{-}{-}{-}{-}{-}{-}{-}{-}{-}{-}{-}{-}{-}{-}{-}{-}{-}{-}{-}{-}{-}{-}{-}{-}{-}{-}{-}{-}{-}{-}{-}}
                     Time
\end{verbatim}

\begin{itemize}
\tightlist
\item
  \textbf{અમલીકરણ}: બેન્ડપાસ ફિલ્ટર્સનો ઉપયોગ કરીને તુલનાત્મક રીતે સરળ
\item
  \textbf{ફાયદા}: સિન્ક્રોનાઇઝેશનની જરૂર નથી, સતત ટ્રાન્સમિશન
\item
  \textbf{ગેરફાયદા}: સ્પેક્ટ્રમ અકાર્યક્ષમતા, મર્યાદિત ફ્લેક્સિબિલિટી
\end{itemize}

\end{solutionbox}
\begin{mnemonicbox}
``ફગવચ'' - \textbf{ફ્રિ}ક્વન્સી ડિવિઝન \textbf{ગા}ર્ડ
બેન્ડ સાથે \textbf{વિ}ભિન્ન \textbf{ચે}નલો બનાવે છે

\end{mnemonicbox}
\subsection*{પ્રશ્ન 5(અ) [3
ગુણ]}\label{uxaaauxab0uxab6uxaa8-5uxa85-3-uxa97uxaa3}

\textbf{TDMA એક્સેસ ટેકનિક સમજાવો.}

\begin{solutionbox}

\textbf{TDMA (ટાઇમ ડિવિઝન મલ્ટિપલ એક્સેસ)}


{\def\LTcaptype{none} % do not increment counter
\vspace{-5pt}
\captionof{table}{TDMA મુખ્ય લાક્ષણિકતાઓ}
\vspace{-10pt}
\begin{longtable}[]{@{}ll@{}}
\toprule\noalign{}
લાક્ષણિકતા & વર્ણન \\
\midrule\noalign{}
\endhead
\bottomrule\noalign{}
\endlastfoot
\textbf{મૂળ સિદ્ધાંત} & સમયને ફ્રેમ્સ અને સ્લોટ્સમાં વિભાજિત કરે છે \\
\textbf{રિસોર્સ શેરિંગ} & દરેક યુઝરને ચોક્કસ ટાઇમ સ્લોટ ફાળવવામાં આવે છે \\
\textbf{ગાર્ડ ટાઇમ} & સ્લોટ્સ વચ્ચે નાનું સમય અંતર \\
\textbf{ફ્રેમ સ્ટ્રક્ચર} & અનેક સ્લોટ્સ મળીને સંપૂર્ણ ફ્રેમ બનાવે છે \\
\textbf{સિન્ક્રોનાઇઝેશન} & બધા વપરાશકર્તાઓ માટે ટાઇમિંગ રેફરન્સની જરૂર \\
\end{longtable}
}

\textbf{આકૃતિ:}

\begin{center}
\textbf{Mermaid Diagram (Code)}
\begin{verbatim}
{Shaded}
{Highlighting}[]
graph TD
    A[TDMA ફ્રેમ] {-{-}{} B[સ્લોટ 1 {-} યુઝર 1]}
    A {-{-}{} C[સ્લોટ 2 {-} યુઝર 2]}
    A {-{-}{} D[સ્લોટ 3 {-} યુઝર 3]}
    A {-{-}{} E[સ્લોટ 4 {-} યુઝર 4]}
    A {-{-}{} F[સ્લોટ 5 {-} યુઝર 5]}
    A {-{-}{} G[સ્લોટ 6 {-} યુઝર 6]}
{Highlighting}
{Shaded}
\end{verbatim}
\end{center}

\begin{itemize}
\tightlist
\item
  \textbf{ડિજિટલ અમલીકરણ}: એનાલોગ સિગ્નલ્સ માટે ADC/DAC ની જરૂર
\item
  \textbf{બર્સ્ટ ટ્રાન્સમિશન}: વપરાશકર્તાઓ ફક્ત ફાળવેલા સ્લોટ્સમાં જ ટ્રાન્સમિટ કરે છે
\end{itemize}

\end{solutionbox}
\begin{mnemonicbox}
``ટેદવ'' - \textbf{ટા}ઇમ સ્લોટ્સ \textbf{દ}રેક
\textbf{વ}પરાશકર્તા માટે અલગથી વ્યવસ્થિત

\end{mnemonicbox}
\subsection*{પ્રશ્ન 5(બ) [4
ગુણ]}\label{uxaaauxab0uxab6uxaa8-5uxaac-4-uxa97uxaa3}

\textbf{E1 કેરીયર સિસ્ટમ સમજાવો.}

\begin{solutionbox}

\textbf{E1 કેરીયર સિસ્ટમ}


{\def\LTcaptype{none} % do not increment counter
\vspace{-5pt}
\captionof{table}{E1 કેરીયર સિસ્ટમ સ્પેસિફિકેશન્સ}
\vspace{-10pt}
\begin{longtable}[]{@{}lll@{}}
\toprule\noalign{}
પેરામીટર & સ્પેસિફિકેશન & વિગતો \\
\midrule\noalign{}
\endhead
\bottomrule\noalign{}
\endlastfoot
\textbf{કુલ બિટ રેટ} & 2.048 Mbps & યુરોપિયન સ્ટાન્ડર્ડ \\
\textbf{ચેનલોની સંખ્યા} & 32 ટાઇમ સ્લોટ્સ (0-31) & 30 વોઇસ + 2 કંટ્રોલ \\
\textbf{વોઇસ ચેનલ્સ} & ટાઇમ સ્લોટ્સ 1-15, 17-31 & દરેક 64 kbps \\
\textbf{સિગ્નલિંગ ચેનલ} & ટાઇમ સ્લોટ 16 & ચેનલ સિગ્નલિંગ માટે \\
\textbf{ફ્રેમ એલાઇનમેન્ટ} & ટાઇમ સ્લોટ 0 & સિન્ક્રોનાઇઝેશન \\
\textbf{ફ્રેમ અવધિ} & 125 μs & 8000 ફ્રેમ્સ પ્રતિ સેકન્ડ \\
\textbf{સેમ્પલિંગ રેટ} & 8 kHz & નાયક્વિસ્ટ થિયરમને અનુસરે છે \\
\end{longtable}
}

\textbf{આકૃતિ:}

\begin{center}
\textbf{Mermaid Diagram (Code)}
\begin{verbatim}
{Shaded}
{Highlighting}[]
graph TD
    A[E1 ફ્રેમ {- 2.048 Mbps] {-}{-}{} B[TS0: ફ્રેમિંગ]}
    A {-{-}{} C[TS1{-}15: વોઇસ ચેનલ્સ]}
    A {-{-}{} D[TS16: સિગ્નલિંગ]}
    A {-{-}{} E[TS17{-}31: વોઇસ ચેનલ્સ]}
    B {-{-}{} F[ફ્રેમ એલાઇનમેન્ટ સિગ્નલ]}
    D {-{-}{} G[ચેનલ એસોસિએટેડ સિગ્નલિંગ]}
{Highlighting}
{Shaded}
\end{verbatim}
\end{center}

\begin{itemize}
\tightlist
\item
  \textbf{મલ્ટિપ્લેક્સિંગ ટેકનિક}: TDM (ટાઇમ ડિવિઝન મલ્ટિપ્લેક્સિંગ)
\item
  \textbf{PCM એનકોડિંગ}: 8 kHz સેમ્પલિંગ રેટ પર 8-બિટ સેમ્પલ્સ
\end{itemize}

\end{solutionbox}
\begin{mnemonicbox}
``ઈ132'' - \textbf{E1} માં \textbf{32} ટાઇમ સ્લોટ્સ
\textbf{2}.048 Mbps સાથે

\end{mnemonicbox}
\subsection*{પ્રશ્ન 5(ક) [7
ગુણ]}\label{uxaaauxab0uxab6uxaa8-5uxa95-7-uxa97uxaa3}

\textbf{ડિજિટલ ટેલિફોન એક્સચેન્જના બ્લોક ડાયાગ્રામ, હાર્ડવેર સબ સિસ્ટમના એલીમેન્ટ
સમજાવો.}

\begin{solutionbox}

\textbf{ડિજિટલ ટેલિફોન એક્સચેન્જ બ્લોક ડાયાગ્રામ}

\begin{verbatim}
flowchart TD
    A[ડિજિટલ ટેલિફોન એક્સચેન્જ] {-{-} B[DLU: ડિજિટલ લાઇન યુનિટ]}
    A {-{-} C[LTG: લાઇન/ટ્રંક ગ્રુપ]}
    A {-{-} D[SN: સ્વિચિંગ નેટવર્ક]}
    A {-{-} E[CP: સેન્ટ્રલ પ્રોસેસર]}
    B {-{-} F[સબસ્ક્રાઇબર્સ માટે ઇન્ટરફેસ]}
    C {-{-} G[ટ્રંક્સ માટે ઇન્ટરફેસ]}
    D {-{-} H[ડિજિટલ સ્વિચિંગ]}
    E {-{-} I[સિસ્ટમ કંટ્રોલ]}
\end{verbatim}


{\def\LTcaptype{none} % do not increment counter
\vspace{-5pt}
\captionof{table}{ડિજિટલ ટેલિફોન એક્સચેન્જના હાર્ડવેર સબસિસ્ટમ્સ}
\vspace{-10pt}
\begin{longtable}[]{@{}
  >{\raggedright\arraybackslash}p{(\linewidth - 4\tabcolsep) * \real{0.2973}}
  >{\raggedright\arraybackslash}p{(\linewidth - 4\tabcolsep) * \real{0.2703}}
  >{\raggedright\arraybackslash}p{(\linewidth - 4\tabcolsep) * \real{0.4324}}@{}}
\toprule\noalign{}
\begin{minipage}[b]{\linewidth}\raggedright
સબસિસ્ટમ
\end{minipage} & \begin{minipage}[b]{\linewidth}\raggedright
કાર્ય
\end{minipage} & \begin{minipage}[b]{\linewidth}\raggedright
મુખ્ય ઘટકો
\end{minipage} \\
\midrule\noalign{}
\endhead
\bottomrule\noalign{}
\endlastfoot
\textbf{DLU (ડિજિટલ લાઇન યુનિટ)} & સબસ્ક્રાઇબર લાઇન્સ અને એક્સચેન્જ વચ્ચે ઇન્ટરફેસ &
લાઇન કાર્ડ્સ, CODEC, SLIC, PCM કન્વર્ઝન \\
\textbf{LTG (લાઇન/ટ્રંક ગ્રુપ)} & ટ્રંક લાઇન્સ સંભાળે છે, અન્ય એક્સચેન્જ સાથે ઇન્ટરફેસ &
ટ્રંક કાર્ડ્સ, સિગ્નલિંગ યુનિટ્સ, ઇકો કેન્સેલર્સ \\
\textbf{SN (સ્વિચિંગ નેટવર્ક)} & પોર્ટ્સ વચ્ચે કોલ્સ રૂટ કરે છે, કનેક્ટિવિટી પ્રદાન કરે
છે & ટાઇમ/સ્પેસ સ્વિચ, કનેક્શન મેમોરી, કંટ્રોલ લોજિક \\
\textbf{CP (સેન્ટ્રલ પ્રોસેસર)} & સમગ્ર સિસ્ટમ ઓપરેશન નિયંત્રિત કરે છે & મુખ્ય પ્રોસેસર,
મેમોરી, ઓપરેટિંગ સિસ્ટમ, ડેટાબેઝ \\
\textbf{પેરિફેરલ્સ} & સપોર્ટિંગ ફંક્શન્સ & પાવર સપ્લાય, અલાર્મ સિસ્ટમ્સ, મેઇન્ટેનન્સ
ટર્મિનલ્સ \\
\end{longtable}
}

\textbf{હાર્ડવેર એલિમેન્ટ્સ વિગતો:}

\begin{itemize}
\tightlist
\item
  \textbf{DLU}: એનાલોગ વોઇસને 64 kbps PCM માં કન્વર્ટ કરે છે, લાઇન સિગ્નલિંગ
  સંભાળે છે
\item
  \textbf{LTG}: E1/T1 ટ્રંક્સ મેનેજ કરે છે, SS7 જેવા પ્રોટોકોલ્સ અમલમાં મૂકે છે
\item
  \textbf{SN}: સામાન્ય રીતે ટાઇમ-ડિવિઝન સ્વિચિંગ ફેબ્રિક, નોન-બ્લોકિંગ આર્કિટેક્ચર
\item
  \textbf{CP}: કોલ પ્રોસેસિંગ, બિલિંગ, મેઇન્ટેનન્સ, એડમિનિસ્ટ્રેટિવ ફંક્શન્સ
\end{itemize}

\end{solutionbox}
\begin{mnemonicbox}
``ડલસપ્ર'' - \textbf{ડી}એલયુ સબસ્ક્રાઇબર્સ જોડે છે,
\textbf{લા}ઇન ટ્રંક ગ્રુપ ટ્રંક્સ જોડે છે, \textbf{સ્વિ}ચિંગ નેટવર્ક કોલ્સ સ્વિચ કરે છે,
\textbf{પ્ર}ોસેસર બધું નિયંત્રિત કરે છે

\end{mnemonicbox}
\subsection*{પ્રશ્ન 5(અ) OR [3
ગુણ]}\label{uxaaauxab0uxab6uxaa8-5uxa85-or-3-uxa97uxaa3}

\textbf{TDM અને FDM ની સરખામણી કરો.}

\begin{solutionbox}


{\def\LTcaptype{none} % do not increment counter
\vspace{-5pt}
\captionof{table}{TDM અને FDM ની તુલના}
\vspace{-10pt}
\begin{longtable}[]{@{}lll@{}}
\toprule\noalign{}
પેરામીટર & TDM & FDM \\
\midrule\noalign{}
\endhead
\bottomrule\noalign{}
\endlastfoot
\textbf{ડોમેન ડિવિઝન} & સમય & ફ્રિક્વન્સી \\
\textbf{ચેનલ સેપરેશન} & ગાર્ડ ટાઇમ & ગાર્ડ બેન્ડ્સ \\
\textbf{મલ્ટિપ્લેક્સિંગ પ્રક્રિયા} & ક્રમિક ટાઇમ સ્લોટ્સ & સમાંતર ફ્રિક્વન્સી બેન્ડ્સ \\
\textbf{અમલીકરણ} & ડિજિટલ (મુખ્યત્વે) & એનાલોગ અથવા ડિજિટલ \\
\textbf{ક્રોસટોક} & સામાન્ય રીતે ઓછું & વધુ સંવેદનશીલ \\
\textbf{સિન્ક્રોનાઇઝેશન} & અત્યંત મહત્વપૂર્ણ & જરૂરી નથી \\
\end{longtable}
}

\textbf{આકૃતિ:}

\begin{verbatim}
TDM:
  Time {-{-}}
  +{-{-}{-}{-}{-}{-}{-}{-}{-}{-}{-}+{-}{-}{-}{-}{-}{-}+{-}{-}{-}{-}{-}{-}+{-}{-}{-}{-}{-}{-}+}
  | Channel 1 | Ch 2 | Ch 3 | Ch 1 |...
  +{-{-}{-}{-}{-}{-}{-}{-}{-}{-}{-}+{-}{-}{-}{-}{-}{-}+{-}{-}{-}{-}{-}{-}+{-}{-}{-}{-}{-}{-}+}
  
FDM:
  \^{}
  |   +{-{-}{-}{-}{-}+}
F |   | Ch3 |
r |   +{-{-}{-}{-}{-}+}
e |   | Ch2 |
q |   +{-{-}{-}{-}{-}+}
  |   | Ch1 |
  |   +{-{-}{-}{-}{-}+}
  +{-{-}{-}{-}{-}{-}{-}{-}{-}{-}{-}{-}{-}{-}{-}}
        Time
\end{verbatim}

\begin{itemize}
\tightlist
\item
  \textbf{બેન્ડવિડ્થ ઉપયોગ}: ડિજિટલ માટે TDM વધુ કાર્યક્ષમ, એનાલોગ માટે FDM વધુ
  સારું
\item
  \textbf{સિસ્ટમ જટિલતા}: TDM ને ચોક્કસ ટાઇમિંગની જરૂર પડે છે, FDM ને ચોક્કસ
  ફિલ્ટર્સની જરૂર પડે છે
\end{itemize}

\end{solutionbox}
\begin{mnemonicbox}
``ટફવિ'' - \textbf{ટા}ઇમ અને \textbf{ફ્રિ}ક્વન્સી
\textbf{વિ}ભાજન સિસ્ટમ્સ અલગ-અલગ ડોમેન વિભાજિત કરે છે

\end{mnemonicbox}
\subsection*{પ્રશ્ન 5(બ) OR [4
ગુણ]}\label{uxaaauxab0uxab6uxaa8-5uxaac-or-4-uxa97uxaa3}

\textbf{T1 મલ્ટિપ્લેક્સિંગ હાયરાર્કી દોરો અને સમજાવો.}

\begin{solutionbox}


{\def\LTcaptype{none} % do not increment counter
\vspace{-5pt}
\captionof{table}{T1 મલ્ટિપ્લેક્સિંગ હાયરાર્કી}
\vspace{-10pt}
\begin{longtable}[]{@{}
  >{\raggedright\arraybackslash}p{(\linewidth - 8\tabcolsep) * \real{0.0986}}
  >{\raggedright\arraybackslash}p{(\linewidth - 8\tabcolsep) * \real{0.1831}}
  >{\raggedright\arraybackslash}p{(\linewidth - 8\tabcolsep) * \real{0.1549}}
  >{\raggedright\arraybackslash}p{(\linewidth - 8\tabcolsep) * \real{0.3662}}
  >{\raggedright\arraybackslash}p{(\linewidth - 8\tabcolsep) * \real{0.1972}}@{}}
\toprule\noalign{}
\begin{minipage}[b]{\linewidth}\raggedright
લેવલ
\end{minipage} & \begin{minipage}[b]{\linewidth}\raggedright
ડેઝિગ્નેશન
\end{minipage} & \begin{minipage}[b]{\linewidth}\raggedright
ડેટા રેટ
\end{minipage} & \begin{minipage}[b]{\linewidth}\raggedright
વોઇસ ચેનલોની સંખ્યા
\end{minipage} & \begin{minipage}[b]{\linewidth}\raggedright
મલ્ટિપ્લેક્સિંગ
\end{minipage} \\
\midrule\noalign{}
\endhead
\bottomrule\noalign{}
\endlastfoot
\textbf{T1} & DS1 & 1.544 Mbps & 24 & 24 DS0 (64 kbps) \\
\textbf{T2} & DS2 & 6.312 Mbps & 96 & 4 DS1 \\
\textbf{T3} & DS3 & 44.736 Mbps & 672 & 7 DS2 \\
\textbf{T4} & DS4 & 274.176 Mbps & 4032 & 6 DS3 \\
\end{longtable}
}

\textbf{આકૃતિ:}

\begin{center}
\textbf{Mermaid Diagram (Code)}
\begin{verbatim}
{Shaded}
{Highlighting}[]
graph LR
    A[વ્યક્તિગત વોઇસ ચેનલ્સ {- DS0 64 kbps] {-}{-}{} B[T1/DS1 {-} 1.544 Mbps]}
    B {-{-}{} C[T2/DS2 {-} 6.312 Mbps]}
    C {-{-}{} D[T3/DS3 {-} 44.736 Mbps]}
    D {-{-}{} E[T4/DS4 {-} 274.176 Mbps]}
{Highlighting}
{Shaded}
\end{verbatim}
\end{center}

\textbf{T1 ફ્રેમ સ્ટ્રક્ચર:}

\begin{verbatim}
T1 Frame (193 bits):
  F  Ch1  Ch2  ...  Ch24  F  Ch1  ...
  |  |    |         |     |
  |  8    8         8     |
  |  bits bits      bits  |
  |                       |
  Framing bit (1 bit)     Next frame
\end{verbatim}

\begin{itemize}
\tightlist
\item
  \textbf{T1 ફ્રેમ ફોર્મેટ}: 193 બિટ્સ (24 ચેનલ્સ \times 8 બિટ્સ + 1 ફ્રેમિંગ બિટ)
\item
  \textbf{ફ્રેમ અવધિ}: 125 μs (8000 ફ્રેમ્સ પ્રતિ સેકન્ડ)
\end{itemize}

\end{solutionbox}
\begin{mnemonicbox}
``ટીચાર'' - \textbf{ટી}1, ટી2, ટી3, ટી4 મલ્ટિપ્લેક્સિંગના
\textbf{ચા}ર સ્તરોની હાયરાર્કી બનાવે છે

\end{mnemonicbox}
\subsection*{પ્રશ્ન 5(ક) OR [7
ગુણ]}\label{uxaaauxab0uxab6uxaa8-5uxa95-or-7-uxa97uxaa3}

\textbf{IoT ના લક્ષણો, લાક્ષણિકતાઓ, ફાયદા અને ગેરફાયદાની સૂચિ બનાવો.}

\begin{solutionbox}


{\def\LTcaptype{none} % do not increment counter
\vspace{-5pt}
\captionof{table}{ઇન્ટરનેટ ઓફ થિંગ્સ (IoT) ઓવરવ્યુ}
\vspace{-10pt}
\begin{longtable}[]{@{}
  >{\raggedright\arraybackslash}p{(\linewidth - 2\tabcolsep) * \real{0.4545}}
  >{\raggedright\arraybackslash}p{(\linewidth - 2\tabcolsep) * \real{0.5455}}@{}}
\toprule\noalign{}
\begin{minipage}[b]{\linewidth}\raggedright
શ્રેણી
\end{minipage} & \begin{minipage}[b]{\linewidth}\raggedright
મુખ્ય મુદ્દાઓ
\end{minipage} \\
\midrule\noalign{}
\endhead
\bottomrule\noalign{}
\endlastfoot
\textbf{લક્ષણો} & ડિવાઇસ કનેક્ટિવિટી, સેન્સર ઇન્ટિગ્રેશન, ઓટોમેટેડ કંટ્રોલ, ડેટા
એનાલિટિક્સ, રિમોટ મોનિટરિંગ \\
\textbf{લાક્ષણિકતાઓ} & લો પાવર કન્ઝમ્પશન, સ્મોલ ફોર્મ ફેક્ટર, વાયરલેસ કોમ્યુનિકેશન,
રિયલ-ટાઇમ ડેટા પ્રોસેસિંગ, સ્કેલેબિલિટી \\
\textbf{ફાયદા} & બહેતર કાર્યક્ષમતા, ડેટા-ડ્રિવન નિર્ણયો, રિમોટ મેનેજમેન્ટ,
પ્રિડિક્ટિવ મેઇન્ટેનન્સ, રિસોર્સ ઓપ્ટિમાઇઝેશન \\
\textbf{ગેરફાયદા} & સિક્યોરિટી વલ્નરેબિલિટીઝ, પ્રાઇવસી સંબંધિત ચિંતાઓ,
ઇન્ટરઓપરેબિલિટી સમસ્યાઓ, અમલીકરણ જટિલતા, પાવર બંધનો \\
\end{longtable}
}

\textbf{IoT ના લક્ષણો:}

\begin{center}
\textbf{Mermaid Diagram (Code)}
\begin{verbatim}
{Shaded}
{Highlighting}[]
graph TD
    A[IoT લક્ષણો] {-{-}{} B[કનેક્ટિવિટી]}
    A {-{-}{} C[ઇન્ટેલિજન્સ]}
    A {-{-}{} D[સેન્સિંગ]}
    A {-{-}{} E[ઓટોમેશન]}
    A {-{-}{} F[ક્લાઉડ ઇન્ટિગ્રેશન]}
    A {-{-}{} G[ડેટા એનાલિટિક્સ]}
{Highlighting}
{Shaded}
\end{verbatim}
\end{center}

\textbf{ફાયદા અને ગેરફાયદા:}

\begin{verbatim}
Advantages                    Disadvantages
+{-{-}{-}{-}{-}{-}{-}{-}{-}{-}{-}{-}{-}{-}{-}{-}{-}{-}{-}{-}{-}+      +{-}{-}{-}{-}{-}{-}{-}{-}{-}{-}{-}{-}{-}{-}{-}{-}{-}{-}{-}{-}{-}{-}+}
| ✓ Automation        |      | ✗ Security risks     |
| ✓ Enhanced data     |      | ✗ Privacy concerns   |
| ✓ Remote control    |      | ✗ Complex setup      |
| ✓ Cost reduction    |      | ✗ High initial cost  |
| ✓ Quality of life   |      | ✗ Battery life       |
| ✓ Resource savings  |      | ✗ Compatibility      |
+{-{-}{-}{-}{-}{-}{-}{-}{-}{-}{-}{-}{-}{-}{-}{-}{-}{-}{-}{-}{-}+      +{-}{-}{-}{-}{-}{-}{-}{-}{-}{-}{-}{-}{-}{-}{-}{-}{-}{-}{-}{-}{-}{-}+}
\end{verbatim}

\textbf{લાક્ષણિકતા વિગતો:}

\begin{itemize}
\tightlist
\item
  \textbf{ઇન્ટરકનેક્ટિવિટી}: કોઈપણ વસ્તુને વૈશ્વિક માહિતી અને સંચાર ઇન્ફ્રાસ્ટ્રક્ચર
  સાથે જોડી શકાય છે
\item
  \textbf{થિંગ-સંબંધિત સેવાઓ}: IoT પ્રાઇવસી પ્રોટેક્શન જેવી થિંગ-સંબંધિત સેવાઓ પ્રદાન
  કરે છે
\item
  \textbf{હેટરોજેનિટી}: ડિવાઇસિસ અલગ-અલગ હાર્ડવેર/સોફ્ટવેર પ્લેટફોર્મ પર આધારિત
\item
  \textbf{ડાયનેમિક ચેન્જીસ}: ડિવાઇસ સ્ટેટ્સ ડાયનેમિકલી બદલાય છે
  (કનેક્ટિંગ/ડિસકનેક્ટિંગ, સ્લીપિંગ/વેકિંગ)
\item
  \textbf{વિશાળ સ્કેલ}: મેનેજમેન્ટની જરૂર પડતા ડિવાઇસની સંખ્યા પરંપરાગત ઇન્ટરનેટ
  કનેક્ટેડ ડિવાઇસોથી વધુ છે
\end{itemize}

\end{solutionbox}
\begin{mnemonicbox}
``કઓસેડ'' - \textbf{ક}નેક્ટિવિટી, \textbf{ઓ}ટોમેશન,
\textbf{સે}ન્સિંગ, \textbf{કા}ર્યક્ષમતા, \textbf{ડે}ટા એનાલિટિક્સ - IoTના મુખ્ય
લક્ષણો

\end{mnemonicbox}

\end{document}
