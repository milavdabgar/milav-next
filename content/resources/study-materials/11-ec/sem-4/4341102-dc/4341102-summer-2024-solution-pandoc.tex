\documentclass[10pt,a4paper]{article}

% content/resources/templates/preamble.tex
\usepackage[margin=0.6in]{geometry}
\author{Milav Dabgar}
\usepackage{amsmath,amssymb,amsthm}
\usepackage{booktabs}
\usepackage{multirow}
\usepackage{xcolor}
\usepackage{tcolorbox}
\tcbuselibrary{breakable,skins}
\usepackage[colorlinks=true,linkcolor=blue]{hyperref}
\usepackage{titlesec}
\usepackage{enumitem}
\usepackage{tikz}
\usepackage{pgfplots}
\usepackage{circuitikz}
\usepackage[version=4]{mhchem}
\usepackage{longtable}
\usepackage{array}
\usepackage{float}
\usepackage{caption}
\usepackage{listings}

\lstset{
  basicstyle=\small\ttfamily,
  breaklines=true,
  breakatwhitespace=false,
  postbreak=\mbox{\textcolor{red}{$\hookrightarrow$}\space},
  float=false,
  numbers=left,
  numberstyle=\tiny\color{gray},
  numbersep=10pt,
  xleftmargin=2em,
  keywordstyle=\color{blue},
  commentstyle=\color{green!60!black},
  stringstyle=\color{purple},
  backgroundcolor=\color{gray!5},
  showstringspaces=false,
  tabsize=2,
  captionpos=b,
  keepspaces=true,
  columns=flexible
}

\pgfplotsset{compat=1.18}
\usetikzlibrary{shapes,arrows,positioning,calc,patterns,decorations.pathmorphing,decorations.markings,arrows.meta}

% Color scheme
\definecolor{headcolor}{RGB}{0,102,204}
\definecolor{keycolor}{RGB}{220,20,60}
\definecolor{solutioncolor}{RGB}{34,139,34}
\definecolor{mnemoniccolor}{RGB}{148,0,211}
\definecolor{codecolor}{RGB}{0,0,100}

% Spacing
\setlength{\parskip}{3pt}
\setlist[itemize]{nosep}
\setlist[enumerate]{nosep}

% Title formatting
\titleformat{\section}{\Large\bfseries\color{headcolor}}{\thesection}{1em}{}
\titleformat{\subsection}{\large\bfseries\color{headcolor}}{\thesubsection}{1em}{}

% Pandoc tightlist compatibility
\providecommand{\tightlist}{%
  \setlength{\itemsep}{0pt}\setlength{\parskip}{0pt}}

% Pandoc longtable compatibility
\newcounter{none}
\def\thenone{}


% content/resources/templates/english-boxes.tex
% This file is currently empty - it exists to maintain consistency with the import structure.
% Add custom environments here if needed in the future.


\begin{document}

\begin{center}
{\Huge\bfseries\color{headcolor} Subject Name Solutions}\\[5pt]
{\LARGE 4341102 -- Summer 2024}\\[3pt]
{\large Semester 1 Study Material}\\[3pt]
{\normalsize\textit{Detailed Solutions and Explanations}}
\end{center}

\vspace{10pt}

\subsection*{Question 1(a) [3 marks]}\label{q1a}

\textbf{Define Continuous time Signal and Discrete time Signal with Wave
form.}

\begin{solutionbox}

{\def\LTcaptype{none} % do not increment counter
\begin{longtable}[]{@{}
  >{\raggedright\arraybackslash}p{(\linewidth - 4\tabcolsep) * \real{0.3714}}
  >{\raggedright\arraybackslash}p{(\linewidth - 4\tabcolsep) * \real{0.3429}}
  >{\raggedright\arraybackslash}p{(\linewidth - 4\tabcolsep) * \real{0.2857}}@{}}
\toprule\noalign{}
\begin{minipage}[b]{\linewidth}\raggedright
Signal Type
\end{minipage} & \begin{minipage}[b]{\linewidth}\raggedright
Definition
\end{minipage} & \begin{minipage}[b]{\linewidth}\raggedright
Waveform
\end{minipage} \\
\midrule\noalign{}
\endhead
\bottomrule\noalign{}
\endlastfoot
\textbf{Continuous Time Signal} & Signal defined for all values of time
with no breaks &
\texttt{mermaid\ graph\ LR;\ A[t]\ -\/-\textgreater{}\ B[x(t)];\ style\ B\ fill:\#fff,stroke:\#333,stroke-width:2px} \\
\textbf{Discrete Time Signal} & Signal defined only at discrete time
intervals &
\texttt{mermaid\ graph\ LR;\ A[n]\ -\/-\textgreater{}\ B[x[n]];\ style\ B\ fill:\#fff,stroke:\#333,stroke-width:2px} \\
\end{longtable}
}

\textbf{Diagram}:

\begin{verbatim}
                      Continuous                          Discrete
 Signal                                                       o
 Amplitude    /{      /                                      |}
             /  {    /                                       o     o}
            /    {  /                                        |     |}
           /      {/                                  o      |     |      o}
          /                {                           |      o     |      |}
 {-{-}{-}{-}{-}{-}{-}{-}/{-}{-}{-}{-}{-}{-}{-}{-}{-}{-}{-}{-}{-}{-}{-}{-}{-}{-}{-}{-}{-}{-}{-}{-}{-}{-}{-} time   {-}{-}{-}{-}{-}o{-}{-}{-}|{-}{-}{-}{-}{-}{-}|{-}{-}{-}{-}{-}|{-}{-}{-}{-}{-}{-}o{-}{-}{-}{-}{-} time}
                                                   |   |      |     o      |
                                                   o   o      |            |
                                                                           o
\end{verbatim}

\end{solutionbox}
\begin{mnemonicbox}
``Continuous Curves, Discrete Dots''

\end{mnemonicbox}
\subsection*{Question 1(b) [4 marks]}\label{q1b}

\textbf{Explain Energy and power signal.}

\begin{solutionbox}

{\def\LTcaptype{none} % do not increment counter
\begin{longtable}[]{@{}
  >{\raggedright\arraybackslash}p{(\linewidth - 4\tabcolsep) * \real{0.2750}}
  >{\raggedright\arraybackslash}p{(\linewidth - 4\tabcolsep) * \real{0.3750}}
  >{\raggedright\arraybackslash}p{(\linewidth - 4\tabcolsep) * \real{0.3500}}@{}}
\toprule\noalign{}
\begin{minipage}[b]{\linewidth}\raggedright
Parameter
\end{minipage} & \begin{minipage}[b]{\linewidth}\raggedright
Energy Signal
\end{minipage} & \begin{minipage}[b]{\linewidth}\raggedright
Power Signal
\end{minipage} \\
\midrule\noalign{}
\endhead
\bottomrule\noalign{}
\endlastfoot
\textbf{Definition} & Has finite energy but zero average power & Has
finite average power but infinite energy \\
\textbf{Mathematical Expression} & \int\textbar x(t)\textbar^{2}dt \textless{}
\infty & lim(T\rightarrow\infty) (1/2T)\int\textbar x(t)\textbar^{2}dt \textless{} \infty \\
\textbf{Examples} & Pulse, Decaying exponential & Sine wave, Square
wave \\
\textbf{Nature} & Finite duration or decreasing amplitude & Periodic or
infinite duration \\
\end{longtable}
}

\textbf{Diagram}:

\begin{verbatim}
     Energy Signal                      Power Signal
        /{                              /    /    /}
       /  {                            /    /    /  }
      /    {                          /    /    /    }
 {-{-}{-}{-}/{-}{-}{-}{-}{-}{-}{-}{-}{-}{-}{-}{-}{-} time   {-}{-}{-}{-}{-}{-}{-}{-}{-}/{-}{-}{-}{-}{-}{-}{-}{-}{-}{-}{-}{-}{-}{-}{-}{-}{-}{-}{-}{-} time}
    /        {                      /}
   /          {                    /}
                                 Never ends...
\end{verbatim}

\end{solutionbox}
\begin{mnemonicbox}
``Energy Expires, Power Persists''

\end{mnemonicbox}
\subsection*{Question 1(c) [7 marks]}\label{q1c}

\textbf{Explain block diagram of digital communication system.}

\begin{solutionbox}

\begin{center}
\textbf{Mermaid Diagram (Code)}
\begin{verbatim}
{Shaded}
{Highlighting}[]
graph LR
    A[Source] {-{-}{} B[Source Encoder]}
    B {-{-}{} C[Channel Encoder]}
    C {-{-}{} D[Digital Modulator]}
    D {-{-}{} E[Channel]}
    E {-{-}{} F[Digital Demodulator]}
    F {-{-}{} G[Channel Decoder]}
    G {-{-}{} H[Source Decoder]}
    H {-{-}{} I[Destination]}
{Highlighting}
{Shaded}
\end{verbatim}
\end{center}

{\def\LTcaptype{none} % do not increment counter
\begin{longtable}[]{@{}
  >{\raggedright\arraybackslash}p{(\linewidth - 2\tabcolsep) * \real{0.4118}}
  >{\raggedright\arraybackslash}p{(\linewidth - 2\tabcolsep) * \real{0.5882}}@{}}
\toprule\noalign{}
\begin{minipage}[b]{\linewidth}\raggedright
Block
\end{minipage} & \begin{minipage}[b]{\linewidth}\raggedright
Function
\end{minipage} \\
\midrule\noalign{}
\endhead
\bottomrule\noalign{}
\endlastfoot
\textbf{Source} & Generates message to be transmitted \\
\textbf{Source Encoder} & Converts message to digital sequence, removes
redundancy \\
\textbf{Channel Encoder} & Adds controlled redundancy for error
detection/correction \\
\textbf{Digital Modulator} & Converts digital symbols to waveforms
suitable for channel \\
\textbf{Channel} & Transmission medium, adds noise and distortion \\
\textbf{Digital Demodulator} & Converts received waveforms back to
digital symbols \\
\textbf{Channel Decoder} & Detects/corrects errors using added
redundancy \\
\textbf{Source Decoder} & Reconstructs original message from digital
sequence \\
\end{longtable}
}

\end{solutionbox}
\begin{mnemonicbox}
``Send Signals Carefully, Digital Messages
Communicate Data Safely''

\end{mnemonicbox}
\subsection*{Question 1(c) OR [7
marks]}\label{q1c}

\textbf{Explain Unit Step function and Unit impulse function.}

\begin{solutionbox}

{\def\LTcaptype{none} % do not increment counter
\begin{longtable}[]{@{}
  >{\raggedright\arraybackslash}p{(\linewidth - 6\tabcolsep) * \real{0.1639}}
  >{\raggedright\arraybackslash}p{(\linewidth - 6\tabcolsep) * \real{0.4098}}
  >{\raggedright\arraybackslash}p{(\linewidth - 6\tabcolsep) * \real{0.1967}}
  >{\raggedright\arraybackslash}p{(\linewidth - 6\tabcolsep) * \real{0.2295}}@{}}
\toprule\noalign{}
\begin{minipage}[b]{\linewidth}\raggedright
Function
\end{minipage} & \begin{minipage}[b]{\linewidth}\raggedright
Mathematical Definition
\end{minipage} & \begin{minipage}[b]{\linewidth}\raggedright
Properties
\end{minipage} & \begin{minipage}[b]{\linewidth}\raggedright
Applications
\end{minipage} \\
\midrule\noalign{}
\endhead
\bottomrule\noalign{}
\endlastfoot
\textbf{Unit Step Function (u(t))} & u(t) = 0 for t \textless{} 0u(t) =
1 for t \geq 0 & - Represents sudden transition- Integral of impulse
function & System response analysis \\
\textbf{Unit Impulse Function (δ(t))} & δ(t) = 0 for t \neq 0\intδ(t)dt = 1 &
- Infinitesimally narrow pulse- Sampling property- Derivative of step
function & Sampling, system analysis \\
\end{longtable}
}

\textbf{Diagrams}:

\begin{verbatim}
          Unit Step Function                 Unit Impulse Function
                    \_\_\_\_\_\_                            \^{}
                   |                                  |
                   |                                  |
                   |                               (infinite)
        \_\_\_\_\_\_\_\_\_\_\_|                                  |
                                                      |
        {-{-}{-}{-}{-}{-}{-}{-}{-}{-}0{-}{-}{-}{-}{-}{-}{-}{-}{-}{-}{-} t       {-}{-}{-}{-}{-}{-}0{-}{-}{-}{-}{-}{-}{-}{-}{-}{-}{-}{-}{-}{-}{-}{-}{-}{-}  t}
\end{verbatim}

\end{solutionbox}
\begin{mnemonicbox}
``Step Stays steady after zero, Impulse Instantly
appears then vanishes''

\end{mnemonicbox}
\subsection*{Question 2(a) [3 marks]}\label{q2a}

\textbf{A signal carries 8 bit/signal elements. If 1000 signal elements
sent per second. Find the bit rate.}

\begin{solutionbox}

{\def\LTcaptype{none} % do not increment counter
\begin{longtable}[]{@{}
  >{\raggedright\arraybackslash}p{(\linewidth - 2\tabcolsep) * \real{0.6111}}
  >{\raggedright\arraybackslash}p{(\linewidth - 2\tabcolsep) * \real{0.3889}}@{}}
\toprule\noalign{}
\begin{minipage}[b]{\linewidth}\raggedright
Parameter
\end{minipage} & \begin{minipage}[b]{\linewidth}\raggedright
Value
\end{minipage} \\
\midrule\noalign{}
\endhead
\bottomrule\noalign{}
\endlastfoot
Bits per signal element & 8 bits \\
Signal elements per second & 1000 \\
\textbf{Calculation} & Bit rate = (Bits per signal element) \times (Signal
elements per second) \\
\textbf{Bit rate} & = 8 \times 1000 = 8000 bits/second or 8 kbps \\
\end{longtable}
}

\end{solutionbox}
\begin{mnemonicbox}
``Bits per signal \times Signals per second = Bits per
second''

\end{mnemonicbox}
\subsection*{Question 2(b) [4 marks]}\label{q2b}

\textbf{Explain Even and Odd signal.}

\begin{solutionbox}

{\def\LTcaptype{none} % do not increment counter
\begin{longtable}[]{@{}
  >{\raggedright\arraybackslash}p{(\linewidth - 6\tabcolsep) * \real{0.2167}}
  >{\raggedright\arraybackslash}p{(\linewidth - 6\tabcolsep) * \real{0.4167}}
  >{\raggedright\arraybackslash}p{(\linewidth - 6\tabcolsep) * \real{0.2000}}
  >{\raggedright\arraybackslash}p{(\linewidth - 6\tabcolsep) * \real{0.1667}}@{}}
\toprule\noalign{}
\begin{minipage}[b]{\linewidth}\raggedright
Signal Type
\end{minipage} & \begin{minipage}[b]{\linewidth}\raggedright
Mathematical Definition
\end{minipage} & \begin{minipage}[b]{\linewidth}\raggedright
Properties
\end{minipage} & \begin{minipage}[b]{\linewidth}\raggedright
Examples
\end{minipage} \\
\midrule\noalign{}
\endhead
\bottomrule\noalign{}
\endlastfoot
\textbf{Even Signal} & x(-t) = x(t) & - Symmetric about y-axis- Cosine
is even & Cosine function, \textbar t\textbar{} \\
\textbf{Odd Signal} & x(-t) = -x(t) & - Anti-symmetric about y-axis-
Sine is odd & Sine function, t \\
\end{longtable}
}

\textbf{Diagram}:

\begin{verbatim}
        Even Signal                        Odd Signal
            /{                                 /}
           /  {                               /}
          /    {                             /}
         /      {                           /}
 {-{-}{-}{-}{-}{-}{-}0{-}{-}{-}{-}{-}{-}{-}{-}{-}                  {-}{-}{-}{-}{-}{-}{-}0{-}{-}{-}{-}{-}{-}{-}}
         {      /                         /}
          {    /                         /}
           {  /                         /}
            {/                         v}
\end{verbatim}

\end{solutionbox}
\begin{mnemonicbox}
``Even reflects Exactly, Odd reflects Oppositely''

\end{mnemonicbox}
\subsection*{Question 2(c) [7 marks]}\label{q2c}

\textbf{Explain the block diagram of ASK modulator and de-modulator with
waveform.}

\begin{solutionbox}

\textbf{ASK Modulator:}

\begin{center}
\textbf{Mermaid Diagram (Code)}
\begin{verbatim}
{Shaded}
{Highlighting}[]
graph LR
    A[Digital Input] {-{-}{} B[Product Modulator]}
    C[Carrier Generator] {-{-}{} B}
    B {-{-}{} D[ASK Output]}
{Highlighting}
{Shaded}
\end{verbatim}
\end{center}

\textbf{ASK Demodulator:}

\begin{center}
\textbf{Mermaid Diagram (Code)}
\begin{verbatim}
{Shaded}
{Highlighting}[]
graph LR
    A[ASK Signal] {-{-}{} B[Envelope Detector]}
    B {-{-}{} C[Comparator]}
    C {-{-}{} D[Digital Output]}
{Highlighting}
{Shaded}
\end{verbatim}
\end{center}

\textbf{Waveforms:}

\begin{verbatim}
Digital Input:   \_\_\_\_\_       \_\_\_\_\_
                |     |     |     |
         \_\_\_\_\_\_\_|     |\_\_\_\_\_|     |\_\_\_\_\_

Carrier:  /{//////////////}

ASK Output:      /{//       ///}
                |      |     |      |
         \_\_\_\_\_\_\_|      |\_\_\_\_\_|      |\_\_\_\_\_
\end{verbatim}

{\def\LTcaptype{none} % do not increment counter
\begin{longtable}[]{@{}
  >{\raggedright\arraybackslash}p{(\linewidth - 2\tabcolsep) * \real{0.4091}}
  >{\raggedright\arraybackslash}p{(\linewidth - 2\tabcolsep) * \real{0.5909}}@{}}
\toprule\noalign{}
\begin{minipage}[b]{\linewidth}\raggedright
Concept
\end{minipage} & \begin{minipage}[b]{\linewidth}\raggedright
Description
\end{minipage} \\
\midrule\noalign{}
\endhead
\bottomrule\noalign{}
\endlastfoot
\textbf{ASK Modulation} & Amplitude varies according to digital data (0
or 1) \\
\textbf{Modulator Components} & Product modulator multiplies carrier
with digital signal \\
\textbf{Demodulator Components} & Envelope detector extracts amplitude,
comparator regenerates digital signal \\
\end{longtable}
}

\end{solutionbox}
\begin{mnemonicbox}
``ASK Adjusts Signal's Knockout amplitude''

\end{mnemonicbox}
\subsection*{Question 2(a) OR [3
marks]}\label{q2a}

\textbf{A signal has a bit rate of 4000 bit/second and a baud rate of
1000 baud. How many data elements are carried by each signal element?}

\begin{solutionbox}

{\def\LTcaptype{none} % do not increment counter
\begin{longtable}[]{@{}ll@{}}
\toprule\noalign{}
Parameter & Value \\
\midrule\noalign{}
\endhead
\bottomrule\noalign{}
\endlastfoot
Bit rate & 4000 bits/second \\
Baud rate & 1000 baud (signal elements/second) \\
\textbf{Formula} & Number of data elements = Bit rate \div Baud rate \\
\textbf{Data elements per signal} & = 4000 \div 1000 = 4 bits/signal
element \\
\end{longtable}
}

\end{solutionbox}
\begin{mnemonicbox}
``Bits divided by Bauds equals Bits per signal''

\end{mnemonicbox}
\subsection*{Question 2(b) OR [4
marks]}\label{q2b}

\textbf{Explain Periodic and aperiodic signal.}

\begin{solutionbox}

{\def\LTcaptype{none} % do not increment counter
\begin{longtable}[]{@{}
  >{\raggedright\arraybackslash}p{(\linewidth - 6\tabcolsep) * \real{0.2203}}
  >{\raggedright\arraybackslash}p{(\linewidth - 6\tabcolsep) * \real{0.2034}}
  >{\raggedright\arraybackslash}p{(\linewidth - 6\tabcolsep) * \real{0.4068}}
  >{\raggedright\arraybackslash}p{(\linewidth - 6\tabcolsep) * \real{0.1695}}@{}}
\toprule\noalign{}
\begin{minipage}[b]{\linewidth}\raggedright
Signal Type
\end{minipage} & \begin{minipage}[b]{\linewidth}\raggedright
Definition
\end{minipage} & \begin{minipage}[b]{\linewidth}\raggedright
Mathematical Condition
\end{minipage} & \begin{minipage}[b]{\linewidth}\raggedright
Examples
\end{minipage} \\
\midrule\noalign{}
\endhead
\bottomrule\noalign{}
\endlastfoot
\textbf{Periodic Signal} & Repeats after fixed time interval & x(t) =
x(t+T) for all t & Sine wave, Square wave \\
\textbf{Aperiodic Signal} & Does not repeat after any time interval &
x(t) \neq x(t+T) for any T & Pulse, Noise \\
\end{longtable}
}

\textbf{Diagram}:

\begin{verbatim}
    Periodic Signal                Aperiodic Signal
    /{    /    /                       /}
   /  {  /    /                       /  }
  /    {/    /                       /    \_\_\_\_\_\_\_\_\_\_\_}
                                      /
 One period (T) {-{-}|                 /}
\end{verbatim}

\end{solutionbox}
\begin{mnemonicbox}
``Periodic Perfectly repeats, Aperiodic Alters
always''

\end{mnemonicbox}
\subsection*{Question 2(c) OR [7
marks]}\label{q2c}

\textbf{Explain the block diagram of PSK modulator and de-modulator with
waveform.}

\begin{solutionbox}

\textbf{PSK Modulator:}

\begin{center}
\textbf{Mermaid Diagram (Code)}
\begin{verbatim}
{Shaded}
{Highlighting}[]
graph LR
    A[Digital Input] {-{-}{} B[Phase Shifter]}
    C[Carrier Generator] {-{-}{} B}
    B {-{-}{} D[PSK Output]}
{Highlighting}
{Shaded}
\end{verbatim}
\end{center}

\textbf{PSK Demodulator:}

\begin{center}
\textbf{Mermaid Diagram (Code)}
\begin{verbatim}
{Shaded}
{Highlighting}[]
graph LR
    A[PSK Signal] {-{-}{} B[Product Detector]}
    C[Carrier Recovery] {-{-}{} B}
    B {-{-}{} D[Low Pass Filter]}
    D {-{-}{} E[Decision Device]}
    E {-{-}{} F[Digital Output]}
{Highlighting}
{Shaded}
\end{verbatim}
\end{center}

\textbf{Waveforms:}

\begin{verbatim}
Digital Input:   \_\_\_\_\_       \_\_\_\_\_
                |     |     |     |
         \_\_\_\_\_\_\_|     |\_\_\_\_\_|     |\_\_\_\_\_

Carrier:  /{//////////////}

PSK Output: /{////////////}
           (0^)   (180^) (0^)  (180^)
           Phase shifts at bit transitions
\end{verbatim}

{\def\LTcaptype{none} % do not increment counter
\begin{longtable}[]{@{}ll@{}}
\toprule\noalign{}
Parameter & Description \\
\midrule\noalign{}
\endhead
\bottomrule\noalign{}
\endlastfoot
\textbf{PSK Modulation} & Phase shifts according to digital data (0 or
1) \\
\textbf{Phase States} & 0^\circ for bit `1', 180^\circ for bit `0' \\
\textbf{Advantages} & Better noise immunity than ASK \\
\end{longtable}
}

\end{solutionbox}
\begin{mnemonicbox}
``PSK Phases Shift with Knowledge''

\end{mnemonicbox}
\subsection*{Question 3(a) [3 marks]}\label{q3a}

\textbf{Explain the working of FSK modulator with block diagram and
output Waveform.}

\begin{solutionbox}

\textbf{FSK Modulator Block Diagram:}

\begin{center}
\textbf{Mermaid Diagram (Code)}
\begin{verbatim}
{Shaded}
{Highlighting}[]
graph LR
    A[Digital Input] {-{-}{} B[Voltage Controlled Oscillator]}
    B {-{-}{} C[FSK Output]}
{Highlighting}
{Shaded}
\end{verbatim}
\end{center}

\textbf{FSK Waveforms:}

\begin{verbatim}
Digital Input:   \_\_\_\_\_       \_\_\_\_\_
                |     |     |     |
         \_\_\_\_\_\_\_|     |\_\_\_\_\_|     |\_\_\_\_\_

FSK Output: /{//  /////  ///}
           (f1)    (f2)       (f1)
\end{verbatim}

\begin{itemize}
\tightlist
\item
  \textbf{Principle}: Digital bit `1' sends carrier with frequency f1,
  bit `0' sends carrier with frequency f2
\item
  \textbf{Working}: Voltage controlled oscillator changes frequency
  based on input bit value
\end{itemize}

\end{solutionbox}
\begin{mnemonicbox}
``Frequency Shifts for Knowledge transmission''

\end{mnemonicbox}
\subsection*{Question 3(b) [4 marks]}\label{q3b}

\textbf{Draw the PSK modulation waveform for the sequence of
1010110110.}

\begin{solutionbox}

\begin{verbatim}
Digital Input:  \_\_\_     \_\_\_     \_\_\_\_\_\_\_     \_\_\_\_\_\_\_   
               |   |   |   |   |       |   |       |  
          \_\_\_\_\_|   |\_\_\_|   |\_\_\_|       |\_\_\_|       |\_\_\_
          
          1     0     1     0     1     1     0     1     1     0
          
PSK Output:    
          /{/ // // // // // // // // //}
          0^   180^ 0^   180^ 0^   0^   180^ 0^   0^   180^
          
Phase:    0^   180^ 0^   180^ 0^   0^   180^ 0^   0^   180^
\end{verbatim}

\textbf{Table for PSK modulation:}

{\def\LTcaptype{none} % do not increment counter
\begin{longtable}[]{@{}ll@{}}
\toprule\noalign{}
Bit & Phase \\
\midrule\noalign{}
\endhead
\bottomrule\noalign{}
\endlastfoot
1 & 0^\circ \\
0 & 180^\circ \\
\end{longtable}
}

\end{solutionbox}
\begin{mnemonicbox}
``One-Zero, Phase-Shifts, Keep-Signal Modulated''

\end{mnemonicbox}
\subsection*{Question 3(c) [7 marks]}\label{q3c}

\textbf{Draw the ASK and FSK modulation waveform for the sequence of
1100110101.}

\begin{solutionbox}

\textbf{Digital Input Sequence: 1100110101}

\begin{verbatim}
Digital Input:  \_\_\_\_\_\_\_         \_\_\_\_\_\_\_     \_\_\_     \_\_\_
               |       |       |       |   |   |   |   |
          \_\_\_\_\_|       |\_\_\_\_\_\_\_|       |\_\_\_|   |\_\_\_|   |\_\_\_
          
          1     1     0     0     1     1     0     1     0     1
          
ASK Output:     
          /{/  //         //  //        //        //}
          
          On    On    Off   Off   On    On    Off   On    Off   On
          
FSK Output:     
          /{/  //  /// /// //  //  /// //  /// //}
          
          f1    f1    f2    f2    f1    f1    f2    f1    f2    f1
\end{verbatim}

\textbf{Table for comparison:}

{\def\LTcaptype{none} % do not increment counter
\begin{longtable}[]{@{}lll@{}}
\toprule\noalign{}
Bit & ASK & FSK \\
\midrule\noalign{}
\endhead
\bottomrule\noalign{}
\endlastfoot
1 & Carrier ON (high amplitude) & Higher frequency (f1) \\
0 & Carrier OFF (zero/low amplitude) & Lower frequency (f2) \\
\end{longtable}
}

\end{solutionbox}
\begin{mnemonicbox}
``Amplitude Shows Knowledge, Frequency Shifts
Knowledge''

\end{mnemonicbox}
\subsection*{Question 3(a) OR [3
marks]}\label{q3a}

\textbf{Explain the working of MSK modulator with block diagram and
output Waveform.}

\begin{solutionbox}

\textbf{MSK Modulator Block Diagram:}

\begin{center}
\textbf{Mermaid Diagram (Code)}
\begin{verbatim}
{Shaded}
{Highlighting}[]
graph LR
    A[Digital Input] {-{-}{} B[Serial to Parallel]}
    B {-{-}{}|I{-}Channel| C[I{-}Channel Modulator]}
    B {-{-}{}|Q{-}Channel| D[Q{-}Channel Modulator]}
    E[Carrier Generator] {-{-}{} C}
    E {-{-}{}|90^ Phase Shift| D}
    C {-{-}{} F[Adder]}
    D {-{-}{} F}
    F {-{-}{} G[MSK Output]}
{Highlighting}
{Shaded}
\end{verbatim}
\end{center}

\textbf{MSK Features:}

\begin{itemize}
\tightlist
\item
  Continuous phase FSK with frequency deviation exactly half bit rate
\item
  Phase changes occur smoothly (no abrupt phase changes)
\item
  Better spectral efficiency than FSK
\end{itemize}

\end{solutionbox}
\begin{mnemonicbox}
``Minimum Shift Keeps spectrum narrow''

\end{mnemonicbox}
\subsection*{Question 3(b) [4 marks]}\label{q3b}

\textbf{Draw the constellation diagram of 8-PSK and 16-QAM.}

\begin{solutionbox}

\textbf{8-PSK Constellation:}

\begin{verbatim}
           001  *    *  000
               /|{  /|}
                |    |
          010 * |    | * 111
              { |    | /}
               {|    |/}
          011  *     *  110
               /|{  /|}
               / {   /}
          100 *   { /  * 101}
\end{verbatim}

\textbf{16-QAM Constellation:}

\begin{verbatim}
     *     *     *     *
    0000  0001  0100  0101
     
     *     *     *     *
    0010  0011  0110  0111
    
     *     *     *     *
    1000  1001  1100  1101
     
     *     *     *     *
    1010  1011  1110  1111
\end{verbatim}

{\def\LTcaptype{none} % do not increment counter
\begin{longtable}[]{@{}
  >{\raggedright\arraybackslash}p{(\linewidth - 2\tabcolsep) * \real{0.4800}}
  >{\raggedright\arraybackslash}p{(\linewidth - 2\tabcolsep) * \real{0.5200}}@{}}
\toprule\noalign{}
\begin{minipage}[b]{\linewidth}\raggedright
Modulation
\end{minipage} & \begin{minipage}[b]{\linewidth}\raggedright
Description
\end{minipage} \\
\midrule\noalign{}
\endhead
\bottomrule\noalign{}
\endlastfoot
\textbf{8-PSK} & 8 points equally spaced around circle, 3 bits per
symbol \\
\textbf{16-QAM} & 16 points in square grid, varying amplitude and phase,
4 bits per symbol \\
\end{longtable}
}

\end{solutionbox}
\begin{mnemonicbox}
``PSK Points on Single circle, QAM Quadrature
Amplitude Matrix''

\end{mnemonicbox}
\subsection*{Question 3(c) OR [7
marks]}\label{q3c}

\textbf{Draw BPSK and QPSK modulation waveform for 1010101011.}

\begin{solutionbox}

\textbf{BPSK Modulation:}

\begin{verbatim}
Digital Input:  \_\_\_     \_\_\_     \_\_\_     \_\_\_       
               |   |   |   |   |   |   |   |    
          \_\_\_\_\_|   |\_\_\_|   |\_\_\_|   |\_\_\_|   |\_\_\_\_\_
          
          1     0     1     0     1     0     1     0     1     1
          
BPSK Output:    
          /{/  //  //  //  //  //  //  //  //  //}
          0^    180^  0^    180^  0^    180^  0^    180^  0^    0^
\end{verbatim}

\textbf{QPSK Modulation (grouping bits in pairs):}

\begin{verbatim}
                    10           10           10           11
Input Pairs:    |{-{-}{-}{-}{-}{-}{-}{-}|   |{-}{-}{-}{-}{-}{-}{-}{-}|   |{-}{-}{-}{-}{-}{-}{-}{-}|   |{-}{-}{-}{-}{-}{-}{-}{-}|}
                
I{-Channel:      \_\_\_      \_\_\_      \_\_\_      \_\_\_      \_\_\_}
               |   |    |   |    |   |    |   |    |   |
          \_\_\_\_\_|   |\_\_\_\_|   |\_\_\_\_|   |\_\_\_\_|   |\_\_\_\_|   |\_\_\_\_
                1      0      1      0      1      0      1      1
                
Q{-Channel:     \_\_\_      \_\_\_      \_\_\_           \_\_\_}
              |   |    |   |    |   |         |   |
          \_\_\_\_|   |\_\_\_\_|   |\_\_\_\_|   |\_\_\_\_\_\_\_\_\_|   |\_\_\_\_
                0      1      0      1      0      1      1
                
QPSK Phase:     90^     270^    90^     270^    90^     270^    90^     45^
\end{verbatim}

{\def\LTcaptype{none} % do not increment counter
\begin{longtable}[]{@{}ll@{}}
\toprule\noalign{}
Bit Pair & QPSK Phase \\
\midrule\noalign{}
\endhead
\bottomrule\noalign{}
\endlastfoot
10 & 90^\circ \\
00 & 180^\circ \\
01 & 270^\circ \\
11 & 0^\circ \\
\end{longtable}
}

\end{solutionbox}
\begin{mnemonicbox}
``Binary Phase Shifts Keys, Quadrature Phase Shifts
Keys''

\end{mnemonicbox}
\subsection*{Question 4(a) [3 marks]}\label{q4a}

\textbf{Encode the data using Shanon Fano code for below probability
sequence. P = \{0.30, 0.25, 0.20, 0.12, 0.08, 0.05\}}

\begin{solutionbox}

{\def\LTcaptype{none} % do not increment counter
\begin{longtable}[]{@{}lll@{}}
\toprule\noalign{}
Symbol & Probability & Shannon-Fano Code \\
\midrule\noalign{}
\endhead
\bottomrule\noalign{}
\endlastfoot
S1 & 0.30 & 00 \\
S2 & 0.25 & 01 \\
S3 & 0.20 & 10 \\
S4 & 0.12 & 110 \\
S5 & 0.08 & 1110 \\
S6 & 0.05 & 1111 \\
\end{longtable}
}

\textbf{Process:}

\begin{enumerate}
\tightlist
\item
  Sort symbols by decreasing probability
\item
  Split into two groups with nearly equal probability (0.30+0.25=0.55,
  0.20+0.12+0.08+0.05=0.45)
\item
  Assign 0 to first group, 1 to second group
\item
  Recursively continue for each subgroup
\end{enumerate}

\end{solutionbox}
\begin{mnemonicbox}
``Separate, Fano divides, Code efficiently''

\end{mnemonicbox}
\subsection*{Question 4(b) [4 marks]}\label{q4b}

\textbf{Explain Hamming code.}

\begin{solutionbox}

{\def\LTcaptype{none} % do not increment counter
\begin{longtable}[]{@{}
  >{\raggedright\arraybackslash}p{(\linewidth - 2\tabcolsep) * \real{0.3810}}
  >{\raggedright\arraybackslash}p{(\linewidth - 2\tabcolsep) * \real{0.6190}}@{}}
\toprule\noalign{}
\begin{minipage}[b]{\linewidth}\raggedright
Aspect
\end{minipage} & \begin{minipage}[b]{\linewidth}\raggedright
Description
\end{minipage} \\
\midrule\noalign{}
\endhead
\bottomrule\noalign{}
\endlastfoot
\textbf{Definition} & Linear error-correcting code that detects double
errors and corrects single errors \\
\textbf{Parity bits} & For m data bits, need k parity bits where 2\^{}k
\geq m+k+1 \\
\textbf{Position} & Parity bits placed at positions 1, 2, 4, 8,
16\ldots{} (powers of 2) \\
\textbf{Error detection} & Calculate syndrome to locate error
position \\
\end{longtable}
}

\textbf{Example Hamming(7,4):}

\begin{verbatim}
Positions:  1   2   3   4   5   6   7
            P1  P2  D1  P4  D2  D3  D4
            
Parity check equations:
P1 checks: P1, D1, D2, D4
P2 checks: P2, D1, D3, D4
P4 checks: P4, D2, D3, D4
\end{verbatim}

\end{solutionbox}
\begin{mnemonicbox}
``Hamming Helps Handle Bit Errors''

\end{mnemonicbox}
\subsection*{Question 4(c) [7 marks]}\label{q4c}

\textbf{Compare TDMA and FDMA.}

\begin{solutionbox}

{\def\LTcaptype{none} % do not increment counter
\begin{longtable}[]{@{}
  >{\raggedright\arraybackslash}p{(\linewidth - 4\tabcolsep) * \real{0.1209}}
  >{\raggedright\arraybackslash}p{(\linewidth - 4\tabcolsep) * \real{0.4176}}
  >{\raggedright\arraybackslash}p{(\linewidth - 4\tabcolsep) * \real{0.4615}}@{}}
\toprule\noalign{}
\begin{minipage}[b]{\linewidth}\raggedright
Parameter
\end{minipage} & \begin{minipage}[b]{\linewidth}\raggedright
TDMA (Time Division Multiple Access)
\end{minipage} & \begin{minipage}[b]{\linewidth}\raggedright
FDMA (Frequency Division Multiple Access)
\end{minipage} \\
\midrule\noalign{}
\endhead
\bottomrule\noalign{}
\endlastfoot
\textbf{Basic Principle} & Divides channel by time slots & Divides
channel by frequency bands \\
\textbf{Resource Allocation} & Each user gets entire bandwidth for short
time & Each user gets portion of bandwidth all the time \\
\textbf{Guard Period} & Time guard bands between slots & Frequency guard
bands between channels \\
\textbf{Synchronization} & Strict timing synchronization required & No
timing synchronization needed \\
\textbf{Efficiency} & Higher, due to burst transmission & Lower, due to
fixed assignment \\
\textbf{Complexity} & More complex & Relatively simple \\
\textbf{Examples} & GSM, DECT & FM radio, Traditional satellite
systems \\
\end{longtable}
}

\textbf{Diagram:}

\begin{verbatim}
TDMA:                          FDMA:
        User 1  User 2  User 3         \^{}
Time    |{-{-}{-}{-}{-}|{-}{-}{-}{-}{-}|{-}{-}{-}{-}{-}|{-}{-}{-}        | User 3}
slots   |{-{-}{-}{-}{-}|{-}{-}{-}{-}{-}|{-}{-}{-}{-}{-}|{-}{-}{-}        |{-}{-}{-}{-}{-}{-}{-}}
        |{-{-}{-}{-}{-}|{-}{-}{-}{-}{-}|{-}{-}{-}{-}{-}|{-}{-}{-}  Freq. | User 2}
                                       |{-{-}{-}{-}{-}{-}{-}}
                                       | User 1
                                       |{-{-}{-}{-}{-}{-}{-}{-}{-}}
                                          Time
\end{verbatim}

\end{solutionbox}
\begin{mnemonicbox}
``Time Divides Multiple Access, Frequency Divides
Multiple Access''

\end{mnemonicbox}
\subsection*{Question 4(a) OR [3
marks]}\label{q4a}

\textbf{Encode the data using Huffman code for below probability
sequence. P = \{0.4, 0.2, 0.2, 0.1, 0.1\}}

\begin{solutionbox}

{\def\LTcaptype{none} % do not increment counter
\begin{longtable}[]{@{}lll@{}}
\toprule\noalign{}
Symbol & Probability & Huffman Code \\
\midrule\noalign{}
\endhead
\bottomrule\noalign{}
\endlastfoot
S1 & 0.4 & 0 \\
S2 & 0.2 & 10 \\
S3 & 0.2 & 11 \\
S4 & 0.1 & 100 \\
S5 & 0.1 & 101 \\
\end{longtable}
}

\textbf{Process:}

\begin{enumerate}
\tightlist
\item
  Start with sorted probabilities
\item
  Combine lowest two probabilities (0.1+0.1=0.2)
\item
  Rearrange and repeat until only two nodes remain
\item
  Assign bits by traversing tree
\end{enumerate}

\textbf{Tree Construction:}

\begin{verbatim}
                  1.0
                 /   {}
                /     {}
             0.6       0.4(S1)
            /   {}
           /     {}
        0.4      0.2(S2,S3)
       /   {     / }
    0.2    0.2  0  1
   /   {}
0.1    0.1
\end{verbatim}

\end{solutionbox}
\begin{mnemonicbox}
``Huffman Helps encode High-frequency data''

\end{mnemonicbox}
\subsection*{Question 4(b) OR [4
marks]}\label{q4b}

\textbf{Explain parity code.}

\begin{solutionbox}

{\def\LTcaptype{none} % do not increment counter
\begin{longtable}[]{@{}
  >{\raggedright\arraybackslash}p{(\linewidth - 2\tabcolsep) * \real{0.3810}}
  >{\raggedright\arraybackslash}p{(\linewidth - 2\tabcolsep) * \real{0.6190}}@{}}
\toprule\noalign{}
\begin{minipage}[b]{\linewidth}\raggedright
Aspect
\end{minipage} & \begin{minipage}[b]{\linewidth}\raggedright
Description
\end{minipage} \\
\midrule\noalign{}
\endhead
\bottomrule\noalign{}
\endlastfoot
\textbf{Definition} & Simple error detection scheme that adds parity
bit \\
\textbf{Types} & Even parity: total 1s is evenOdd parity: total 1s is
odd \\
\textbf{Calculation} & XOR all data bits to generate parity bit \\
\textbf{Capability} & Detects odd number of errors, cannot correct
errors \\
\end{longtable}
}

\textbf{Examples:}

\begin{verbatim}
Even Parity:
Data: 1011  Parity: 0  Coded: 10110 (Even number of 1s: 4)

Odd Parity:
Data: 1011  Parity: 1  Coded: 10111 (Odd number of 1s: 5)
\end{verbatim}

\end{solutionbox}
\begin{mnemonicbox}
``Parity Provides Primitive Error detection''

\end{mnemonicbox}
\subsection*{Question 4(c) OR [7
marks]}\label{q4c}

\textbf{Explain FDMA Technique in detail.}

\begin{solutionbox}

\textbf{FDMA (Frequency Division Multiple Access):}

\begin{center}
\textbf{Mermaid Diagram (Code)}
\begin{verbatim}
{Shaded}
{Highlighting}[]
graph TD
    A[Available Bandwidth] {-{-}{} B[Frequency Division]}
    B {-{-}{} C[User 1 Channel]}
    B {-{-}{} D[User 2 Channel]}
    B {-{-}{} E[User 3 Channel]}
    B {-{-}{} F[User N Channel]}
{Highlighting}
{Shaded}
\end{verbatim}
\end{center}

{\def\LTcaptype{none} % do not increment counter
\begin{longtable}[]{@{}
  >{\raggedright\arraybackslash}p{(\linewidth - 2\tabcolsep) * \real{0.4583}}
  >{\raggedright\arraybackslash}p{(\linewidth - 2\tabcolsep) * \real{0.5417}}@{}}
\toprule\noalign{}
\begin{minipage}[b]{\linewidth}\raggedright
Parameter
\end{minipage} & \begin{minipage}[b]{\linewidth}\raggedright
Description
\end{minipage} \\
\midrule\noalign{}
\endhead
\bottomrule\noalign{}
\endlastfoot
\textbf{Basic Principle} & Total bandwidth divided into non-overlapping
frequency bands \\
\textbf{Channel Assignment} & Each user assigned dedicated frequency
band \\
\textbf{Guard Bands} & Small frequency gaps between channels to prevent
interference \\
\textbf{Duplexing} & Usually implemented with FDD (Frequency Division
Duplexing) \\
\textbf{Advantages} & Simple implementation, no synchronization
required \\
\textbf{Disadvantages} & Inefficient for bursty traffic, fixed
allocation wastes bandwidth \\
\textbf{Applications} & AM/FM radio, Traditional cable TV,
First-generation mobile systems \\
\end{longtable}
}

\textbf{Frequency Allocation:}

\begin{verbatim}
Frequency
    \^{}
    |   Guard Bands
    |    ↓  ↓  ↓  ↓  ↓
    |   |{-{-}|{-}{-}|{-}{-}|{-}{-}|{-}{-}}
    |   |  |  |  |  |
    |   |  |  |  |  |{-{-} User N}
    |   |  |  |  |
    |   |  |  |  |{-{-} User 3}
    |   |  |  |
    |   |  |  |{-{-} User 2}
    |   |  |
    |   |  |{-{-} User 1}
    |   |
    |{-{-}{-}|{-}{-}{-}{-}{-}{-}{-}{-}{-}{-}{-}{-}{-}{-}{-}{-}{-} Time}
\end{verbatim}

\end{solutionbox}
\begin{mnemonicbox}
``Fixed Division for Multiple Access''

\end{mnemonicbox}
\subsection*{Question 5(a) [3 marks]}\label{q5a}

\textbf{Explain E1 Career system.}

\begin{solutionbox}

{\def\LTcaptype{none} % do not increment counter
\begin{longtable}[]{@{}ll@{}}
\toprule\noalign{}
Parameter & Description \\
\midrule\noalign{}
\endhead
\bottomrule\noalign{}
\endlastfoot
\textbf{Description} & European standard digital transmission format \\
\textbf{Capacity} & 2.048 Mbps \\
\textbf{Channel Structure} & 32 time slots (numbered 0-31) \\
\textbf{Voice Channels} & 30 voice channels (64 kbps each) \\
\textbf{Signaling} & Time slot 16 for signaling \\
\textbf{Frame Alignment} & Time slot 0 for synchronization \\
\end{longtable}
}

\textbf{Diagram:}

\begin{verbatim}
One E1 Frame (32 time slots)
 \_\_\_\_\_\_\_\_\_\_\_\_\_\_\_\_\_\_\_\_\_\_\_\_\_\_\_\_\_\_\_\_\_\_\_\_\_\_\_\_\_\_\_\_\_\_\_\_\_\_\_\_\_\_\_
|   |   |   |   |   |   |   |   |   |   |   |   |   |   |
| 0 | 1 | 2 |...| 15| 16| 17|...| 30| 31| 0 | 1 | 2 |...|
|\_\_\_|\_\_\_|\_\_\_|\_\_\_|\_\_\_|\_\_\_|\_\_\_|\_\_\_|\_\_\_|\_\_\_|\_\_\_|\_\_\_|\_\_\_|\_\_\_|

TS0: Frame alignment
TS16: Signaling
TS1{-15, TS17{-}31: Voice/data channels (30 channels)}
\end{verbatim}

\end{solutionbox}
\begin{mnemonicbox}
``E1 Enables 30 + 2 time slots''

\end{mnemonicbox}
\subsection*{Question 5(b) [4 marks]}\label{q5b}

\textbf{Explain TDMA Access technique.}

\begin{solutionbox}

{\def\LTcaptype{none} % do not increment counter
\begin{longtable}[]{@{}
  >{\raggedright\arraybackslash}p{(\linewidth - 2\tabcolsep) * \real{0.4583}}
  >{\raggedright\arraybackslash}p{(\linewidth - 2\tabcolsep) * \real{0.5417}}@{}}
\toprule\noalign{}
\begin{minipage}[b]{\linewidth}\raggedright
Parameter
\end{minipage} & \begin{minipage}[b]{\linewidth}\raggedright
Description
\end{minipage} \\
\midrule\noalign{}
\endhead
\bottomrule\noalign{}
\endlastfoot
\textbf{Definition} & Multiple access technique that divides time into
slots for different users \\
\textbf{Working Principle} & Each user gets entire bandwidth for a short
time period \\
\textbf{Frame Structure} & Time divided into frames, frames divided into
slots \\
\textbf{Guard Time} & Small time gap between slots to prevent overlap \\
\textbf{Synchronization} & Requires precise timing synchronization \\
\end{longtable}
}

\textbf{TDMA Frame Structure:}

\begin{verbatim}
             One TDMA Frame
 \_\_\_\_\_\_\_\_\_\_\_\_\_\_\_\_\_\_\_\_\_\_\_\_\_\_\_\_\_\_\_\_\_\_\_\_\_\_\_\_
|      |      |      |      |      |     |
| TS 1 | TS 2 | TS 3 | TS 4 | TS 5 | ... |
|\_\_\_\_\_\_|\_\_\_\_\_\_|\_\_\_\_\_\_|\_\_\_\_\_\_|\_\_\_\_\_\_|\_\_\_\_\_|
   |      |      |
   |      |      |{-{-}{-} User 3}
   |      |
   |      |{-{-}{-} User 2}
   |
   |{-{-}{-} User 1}

Each time slot (TS) contains:
{- User data}
{- Guard time}
{- Synchronization bits}
{- Control bits}
\end{verbatim}

\end{solutionbox}
\begin{mnemonicbox}
``Time Divides Multiple Access''

\end{mnemonicbox}
\subsection*{Question 5(c) [7 marks]}\label{q5c}

\textbf{Explain IoT − Concept, Features, Advantages and Disadvantages.}

\begin{solutionbox}

\textbf{IoT Concept:}

\begin{center}
\textbf{Mermaid Diagram (Code)}
\begin{verbatim}
{Shaded}
{Highlighting}[]
graph LR
    A[Physical Objects] {-{-}{}|Sensors| B[Internet Connectivity]}
    B {-{-}{} C[Data Collection]}
    C {-{-}{} D[Data Analysis]}
    D {-{-}{} E[Automated Actions]}
    E {-{-}{} A}
{Highlighting}
{Shaded}
\end{verbatim}
\end{center}

{\def\LTcaptype{none} % do not increment counter
\begin{longtable}[]{@{}
  >{\raggedright\arraybackslash}p{(\linewidth - 2\tabcolsep) * \real{0.3810}}
  >{\raggedright\arraybackslash}p{(\linewidth - 2\tabcolsep) * \real{0.6190}}@{}}
\toprule\noalign{}
\begin{minipage}[b]{\linewidth}\raggedright
Aspect
\end{minipage} & \begin{minipage}[b]{\linewidth}\raggedright
Description
\end{minipage} \\
\midrule\noalign{}
\endhead
\bottomrule\noalign{}
\endlastfoot
\textbf{Concept} & Network of physical objects embedded with sensors,
software, and connectivity \\
\textbf{Features} & - Connectivity (devices connected to internet)-
Intelligence (smart decision making)- Sensing (collecting data from
environment)- Automation (minimal human intervention)- Scalability
(handles many devices) \\
\textbf{Advantages} & - Improved efficiency and productivity- Better
resource management- Enhanced decision making- Convenience and
time-saving- New business opportunities \\
\textbf{Disadvantages} & - Security vulnerabilities- Privacy concerns-
Complexity in implementation- Compatibility issues- Dependence on
internet \\
\end{longtable}
}

\textbf{Application Areas:}

\begin{itemize}
\tightlist
\item
  Smart homes, cities
\item
  Healthcare monitoring
\item
  Industrial automation
\item
  Agriculture
\item
  Transportation
\end{itemize}

\end{solutionbox}
\begin{mnemonicbox}
``Internet of Things: Connected, Automated, Smarter
Decisions''

\end{mnemonicbox}
\subsection*{Question 5(a) OR [4
marks]}\label{q5a}

\textbf{Explain T1 Career TDM system.}

\begin{solutionbox}

{\def\LTcaptype{none} % do not increment counter
\begin{longtable}[]{@{}
  >{\raggedright\arraybackslash}p{(\linewidth - 2\tabcolsep) * \real{0.4583}}
  >{\raggedright\arraybackslash}p{(\linewidth - 2\tabcolsep) * \real{0.5417}}@{}}
\toprule\noalign{}
\begin{minipage}[b]{\linewidth}\raggedright
Parameter
\end{minipage} & \begin{minipage}[b]{\linewidth}\raggedright
Description
\end{minipage} \\
\midrule\noalign{}
\endhead
\bottomrule\noalign{}
\endlastfoot
\textbf{Description} & North American standard digital transmission
format \\
\textbf{Capacity} & 1.544 Mbps \\
\textbf{Channel Structure} & 24 time slots (channels) + 1 framing bit \\
\textbf{Voice Channels} & 24 voice channels (64 kbps each) \\
\textbf{Frame Structure} & 193 bits per frame (24 \times 8 + 1) \\
\textbf{Signaling} & Robbed bit signaling (least significant bit) \\
\end{longtable}
}

\textbf{Diagram:}

\begin{verbatim}
One T1 Frame (193 bits)
 \_\_\_\_\_\_\_\_\_\_\_\_\_\_\_\_\_\_\_\_\_\_\_\_\_\_\_\_\_\_\_\_\_\_\_\_\_\_\_\_\_\_\_\_\_\_\_\_\_\_\_\_\_\_\_\_\_\_\_
|   |       |       |       |       |       |       |       |
| F | Ch 1  | Ch 2  | Ch 3  |  ...  | Ch 22 | Ch 23 | Ch 24 |
|\_\_\_|\_\_\_\_\_\_\_|\_\_\_\_\_\_\_|\_\_\_\_\_\_\_|\_\_\_\_\_\_\_|\_\_\_\_\_\_\_|\_\_\_\_\_\_\_|\_\_\_\_\_\_\_|

F: Framing bit
Each channel: 8 bits (1 byte)
\end{verbatim}

\end{solutionbox}
\begin{mnemonicbox}
``T1 Transmits 24 channels''

\end{mnemonicbox}
\subsection*{Question 5(b) OR [3
marks]}\label{q5b}

\textbf{Compare TDM and FDM.}

\begin{solutionbox}

{\def\LTcaptype{none} % do not increment counter
\begin{longtable}[]{@{}
  >{\raggedright\arraybackslash}p{(\linewidth - 4\tabcolsep) * \real{0.1325}}
  >{\raggedright\arraybackslash}p{(\linewidth - 4\tabcolsep) * \real{0.4096}}
  >{\raggedright\arraybackslash}p{(\linewidth - 4\tabcolsep) * \real{0.4578}}@{}}
\toprule\noalign{}
\begin{minipage}[b]{\linewidth}\raggedright
Parameter
\end{minipage} & \begin{minipage}[b]{\linewidth}\raggedright
TDM (Time Division Multiplexing)
\end{minipage} & \begin{minipage}[b]{\linewidth}\raggedright
FDM (Frequency Division Multiplexing)
\end{minipage} \\
\midrule\noalign{}
\endhead
\bottomrule\noalign{}
\endlastfoot
\textbf{Basic Principle} & Divides channel by time & Divides channel by
frequency \\
\textbf{Signal Separation} & In time domain & In frequency domain \\
\textbf{Guard Bands} & Time guard bands & Frequency guard bands \\
\textbf{Implementation} & Digital technique & Analog technique
(originally) \\
\textbf{Crosstalk} & Less susceptible & More susceptible \\
\textbf{Synchronization} & Required & Not required \\
\end{longtable}
}

\textbf{Diagram:}

\begin{verbatim}
TDM:                         FDM:
     Ch1  Ch2  Ch3  Ch1            \^{}
Time  |{-{-}|{-}{-}|{-}{-}|{-}{-}|{-}{-}             | Ch3}
      |{-{-}|{-}{-}|{-}{-}|{-}{-}|{-}{-}  Frequency  |{-}{-}{-}{-}{-}}
      |{-{-}|{-}{-}|{-}{-}|{-}{-}|{-}{-}             | Ch2}
                                   |{-{-}{-}{-}{-}}
                                   | Ch1
                                   |{-{-}{-}{-}{-}{-}{-}}
                                      Time
\end{verbatim}

\end{solutionbox}
\begin{mnemonicbox}
``Time Divides Multiplexing, Frequency Divides
Multiplexing''

\end{mnemonicbox}
\subsection*{Question 5(c) OR [7
marks]}\label{q5c}

\textbf{Explain security components of information security.}

\begin{solutionbox}

\textbf{The CIA Triad of Information Security:}

\begin{center}
\textbf{Mermaid Diagram (Code)}
\begin{verbatim}
{Shaded}
{Highlighting}[]
graph TD
    A[Information Security] {-{-}{} B[Confidentiality]}
    A {-{-}{} C[Integrity]}
    A {-{-}{} D[Availability]}
    B {-{-}{} E[Encryption, Access Controls]}
    C {-{-}{} F[Hashing, Digital Signatures]}
    D {-{-}{} G[Redundancy, Fault{-}tolerance]}
{Highlighting}
{Shaded}
\end{verbatim}
\end{center}

{\def\LTcaptype{none} % do not increment counter
\begin{longtable}[]{@{}
  >{\raggedright\arraybackslash}p{(\linewidth - 4\tabcolsep) * \real{0.2292}}
  >{\raggedright\arraybackslash}p{(\linewidth - 4\tabcolsep) * \real{0.2708}}
  >{\raggedright\arraybackslash}p{(\linewidth - 4\tabcolsep) * \real{0.5000}}@{}}
\toprule\noalign{}
\begin{minipage}[b]{\linewidth}\raggedright
Component
\end{minipage} & \begin{minipage}[b]{\linewidth}\raggedright
Description
\end{minipage} & \begin{minipage}[b]{\linewidth}\raggedright
Implementation Methods
\end{minipage} \\
\midrule\noalign{}
\endhead
\bottomrule\noalign{}
\endlastfoot
\textbf{Confidentiality} & Protection against unauthorized access & -
Encryption- Access controls- Authentication- Steganography \\
\textbf{Integrity} & Ensuring data is accurate and unaltered & -
Hashing- Digital signatures- Version control- Checksums \\
\textbf{Availability} & Ensuring systems are accessible when needed & -
Redundancy- Backups- Disaster recovery- Fault tolerance \\
\textbf{Authentication} & Verifying identity & - Passwords- Biometrics-
Smart cards- Multi-factor \\
\textbf{Non-repudiation} & Preventing denial of actions & - Digital
signatures- Audit logs- Timestamps \\
\end{longtable}
}

\textbf{Security Threats:}

\begin{itemize}
\tightlist
\item
  Malware (viruses, worms, trojans)
\item
  Social engineering
\item
  Denial of Service (DoS)
\item
  Man-in-the-middle attacks
\item
  Insider threats
\end{itemize}

\end{solutionbox}
\begin{mnemonicbox}
``CIA Protects All Network Data''

\end{mnemonicbox}

\end{document}
