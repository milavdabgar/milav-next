\documentclass[10pt,a4paper]{article}

% content/resources/templates/preamble.tex
\usepackage[margin=0.6in]{geometry}
\author{Milav Dabgar}
\usepackage{amsmath,amssymb,amsthm}
\usepackage{booktabs}
\usepackage{multirow}
\usepackage{xcolor}
\usepackage{tcolorbox}
\tcbuselibrary{breakable,skins}
\usepackage[colorlinks=true,linkcolor=blue]{hyperref}
\usepackage{titlesec}
\usepackage{enumitem}
\usepackage{tikz}
\usepackage{pgfplots}
\usepackage{circuitikz}
\usepackage[version=4]{mhchem}
\usepackage{longtable}
\usepackage{array}
\usepackage{float}
\usepackage{caption}
\usepackage{listings}

\lstset{
  basicstyle=\small\ttfamily,
  breaklines=true,
  breakatwhitespace=false,
  postbreak=\mbox{\textcolor{red}{$\hookrightarrow$}\space},
  float=false,
  numbers=left,
  numberstyle=\tiny\color{gray},
  numbersep=10pt,
  xleftmargin=2em,
  keywordstyle=\color{blue},
  commentstyle=\color{green!60!black},
  stringstyle=\color{purple},
  backgroundcolor=\color{gray!5},
  showstringspaces=false,
  tabsize=2,
  captionpos=b,
  keepspaces=true,
  columns=flexible
}

\pgfplotsset{compat=1.18}
\usetikzlibrary{shapes,arrows,positioning,calc,patterns,decorations.pathmorphing,decorations.markings,arrows.meta}

% Color scheme
\definecolor{headcolor}{RGB}{0,102,204}
\definecolor{keycolor}{RGB}{220,20,60}
\definecolor{solutioncolor}{RGB}{34,139,34}
\definecolor{mnemoniccolor}{RGB}{148,0,211}
\definecolor{codecolor}{RGB}{0,0,100}

% Spacing
\setlength{\parskip}{3pt}
\setlist[itemize]{nosep}
\setlist[enumerate]{nosep}

% Title formatting
\titleformat{\section}{\Large\bfseries\color{headcolor}}{\thesection}{1em}{}
\titleformat{\subsection}{\large\bfseries\color{headcolor}}{\thesubsection}{1em}{}

% Pandoc tightlist compatibility
\providecommand{\tightlist}{%
  \setlength{\itemsep}{0pt}\setlength{\parskip}{0pt}}

% Pandoc longtable compatibility
\newcounter{none}
\def\thenone{}


% content/resources/templates/gujarati-boxes.tex
\usepackage{fontspec}
\usepackage{polyglossia}

% Set Gujarati as main language (document is primarily in Gujarati)
% Note: gloss-gujarati.ldf doesn't exist in polyglossia, but it will use hyphenation patterns
\setdefaultlanguage{gujarati}
\setotherlanguage{english}

% Configure Gujarati font properly
% Use Language=Default to prevent polyglossia from trying to add language-specific features
% that don't exist for Gujarati, which causes "empty feature" warnings
\newfontfamily\gujaratifont[Script=Gujarati,AutoFakeBold=2.5,AutoFakeSlant=0.3]{Noto Sans Gujarati}
\setmainfont[Script=Gujarati,AutoFakeBold=2.5,AutoFakeSlant=0.3]{Noto Sans Gujarati}
% Use Noto Sans Gujarati for monospace to support Gujarati in text
\setmonofont[Scale=0.9]{Noto Sans Gujarati}

% Configure English to use the same font
\newfontfamily\englishfont[Script=Gujarati,AutoFakeBold=2.5,AutoFakeSlant=0.3]{Noto Sans Gujarati}

% Translations for polyglossia
\gappto\captionsgujarati{
  \renewcommand{\tablename}{કોષ્ટક}
  \renewcommand{\figurename}{આકૃતિ}
}

% Helper for TikZ nodes to ensure Gujarati font
\newcommand{\gu}[1]{{\gujaratifont #1}}

% Custom environments
\newtcolorbox{solutionbox}{
    breakable,
    enhanced,
    colback=solutioncolor!5!white,
    colframe=solutioncolor!75!black,
    fonttitle=\bfseries,
    title=જવાબ
}

\newtcolorbox{solutionboxnobreak}{
 colback=solutioncolor!5!white,
 colframe=solutioncolor!75!black,
 fonttitle=\bfseries,
 title=જવાબ
}

\newtcolorbox{keyformula}{
 breakable,
 enhanced,
 colback=keycolor!5!white,
 colframe=keycolor!75!black,
 fonttitle=\bfseries,
 title=રાસાયણિક સમીકરણ/સૂત્ર
}

\newtcolorbox{mnemonicbox}{
 breakable,
 enhanced,
 colback=mnemoniccolor!5!white,
 colframe=mnemoniccolor!75!black,
 fonttitle=\bfseries,
 title=મેમરી ટ્રીક
}


\begin{document}

\begin{center}
{\Huge\bfseries\color{headcolor} Subject Name (Gujarati)}\\[5pt]
{\LARGE 4341102 -- Summer 2024}\\[3pt]
{\large Semester 1 Study Material}\\[3pt]
{\normalsize\textit{Detailed Solutions and Explanations}}
\end{center}

\vspace{10pt}

\subsection*{પ્રશ્ન 1(અ) [3
ગુણ]}\label{uxaaauxab0uxab6uxaa8-1uxa85-3-uxa97uxaa3}

\textbf{વેવ ફોર્મ સાથે કંટીન્યુઅસ ટાઇમ સિગ્નલ અને ડિસ્ક્રીટ ટાઇમ સિગ્નલ વ્યાખ્યાયિત
કરો.}

\begin{solutionbox}

{\def\LTcaptype{none} % do not increment counter
\begin{longtable}[]{@{}
  >{\raggedright\arraybackslash}p{(\linewidth - 4\tabcolsep) * \real{0.4194}}
  >{\raggedright\arraybackslash}p{(\linewidth - 4\tabcolsep) * \real{0.2903}}
  >{\raggedright\arraybackslash}p{(\linewidth - 4\tabcolsep) * \real{0.2903}}@{}}
\toprule\noalign{}
\begin{minipage}[b]{\linewidth}\raggedright
સિગ્નલ પ્રકાર
\end{minipage} & \begin{minipage}[b]{\linewidth}\raggedright
વ્યાખ્યા
\end{minipage} & \begin{minipage}[b]{\linewidth}\raggedright
વેવફોર્મ
\end{minipage} \\
\midrule\noalign{}
\endhead
\bottomrule\noalign{}
\endlastfoot
\textbf{કંટીન્યુઅસ ટાઇમ સિગ્નલ} & સમયની તમામ કિંમતો માટે વ્યાખ્યાયિત સિગ્નલ જેમાં
કોઈ વિરામ નથી &
\texttt{mermaid\ graph\ LR;\ A[t]\ -\/-\textgreater{}\ B[x(t)];\ style\ B\ fill:\#fff,stroke:\#333,stroke-width:2px} \\
\textbf{ડિસ્ક્રીટ ટાઇમ સિગ્નલ} & માત્ર અલગ-અલગ સમય અંતરાલો પર વ્યાખ્યાયિત સિગ્નલ
&
\texttt{mermaid\ graph\ LR;\ A[n]\ -\/-\textgreater{}\ B[x[n]];\ style\ B\ fill:\#fff,stroke:\#333,stroke-width:2px} \\
\end{longtable}
}

\textbf{આકૃતિ}:

\begin{verbatim}
                      Continuous                          Discrete
 Signal                                                       o
 Amplitude    /{      /                                      |}
             /  {    /                                       o     o}
            /    {  /                                        |     |}
           /      {/                                  o      |     |      o}
          /                {                           |      o     |      |}
 {-{-}{-}{-}{-}{-}{-}{-}/{-}{-}{-}{-}{-}{-}{-}{-}{-}{-}{-}{-}{-}{-}{-}{-}{-}{-}{-}{-}{-}{-}{-}{-}{-}{-}{-} time   {-}{-}{-}{-}{-}o{-}{-}{-}|{-}{-}{-}{-}{-}{-}|{-}{-}{-}{-}{-}|{-}{-}{-}{-}{-}{-}o{-}{-}{-}{-}{-} time}
                                                   |   |      |     o      |
                                                   o   o      |            |
                                                                           o
\end{verbatim}

\end{solutionbox}
\begin{mnemonicbox}
``કંટીન્યુઅસમાં કર્વ, ડિસ્ક્રીટમાં ડોટ્સ''

\end{mnemonicbox}
\subsection*{પ્રશ્ન 1(બ) [4
ગુણ]}\label{uxaaauxab0uxab6uxaa8-1uxaac-4-uxa97uxaa3}

\textbf{એનર્જી અને પાવર સિગ્નલ સમજાવો.}

\begin{solutionbox}

{\def\LTcaptype{none} % do not increment counter
\begin{longtable}[]{@{}
  >{\raggedright\arraybackslash}p{(\linewidth - 4\tabcolsep) * \real{0.2821}}
  >{\raggedright\arraybackslash}p{(\linewidth - 4\tabcolsep) * \real{0.3846}}
  >{\raggedright\arraybackslash}p{(\linewidth - 4\tabcolsep) * \real{0.3333}}@{}}
\toprule\noalign{}
\begin{minipage}[b]{\linewidth}\raggedright
પેરામીટર
\end{minipage} & \begin{minipage}[b]{\linewidth}\raggedright
એનર્જી સિગ્નલ
\end{minipage} & \begin{minipage}[b]{\linewidth}\raggedright
પાવર સિગ્નલ
\end{minipage} \\
\midrule\noalign{}
\endhead
\bottomrule\noalign{}
\endlastfoot
\textbf{વ્યાખ્યા} & મર્યાદિત એનર્જી પરંતુ શૂન્ય સરેરાશ પાવર ધરાવે છે & મર્યાદિત
સરેરાશ પાવર પરંતુ અનંત એનર્જી ધરાવે છે \\
\textbf{ગાણિતિક સૂત્ર} & \int\textbar x(t)\textbar^{2}dt \textless{} \infty &
lim(T\rightarrow\infty) (1/2T)\int\textbar x(t)\textbar^{2}dt \textless{} \infty \\
\textbf{ઉદાહરણો} & પલ્સ, ડિકેઇંગ એક્સપોનેન્શિયલ & સાઇન વેવ, સ્ક્વેર વેવ \\
\textbf{પ્રકૃતિ} & મર્યાદિત સમયગાળો અથવા ઘટતી એમ્પ્લિટ્યૂડ & પિરિયોડિક અથવા અનંત
સમયગાળો \\
\end{longtable}
}

\textbf{આકૃતિ}:

\begin{verbatim}
     Energy Signal                      Power Signal
        /{                              /    /    /}
       /  {                            /    /    /  }
      /    {                          /    /    /    }
 {-{-}{-}{-}/{-}{-}{-}{-}{-}{-}{-}{-}{-}{-}{-}{-}{-} time   {-}{-}{-}{-}{-}{-}{-}{-}{-}/{-}{-}{-}{-}{-}{-}{-}{-}{-}{-}{-}{-}{-}{-}{-}{-}{-}{-}{-}{-} time}
    /        {                      /}
   /          {                    /}
                                 Never ends...
\end{verbatim}

\end{solutionbox}
\begin{mnemonicbox}
``એનર્જી ખતમ થાય, પાવર કાયમ રહે''

\end{mnemonicbox}
\subsection*{પ્રશ્ન 1(ક) [7
ગુણ]}\label{uxaaauxab0uxab6uxaa8-1uxa95-7-uxa97uxaa3}

\textbf{ડિજિટલ કોમ્યુનિકેશન સિસ્ટમનો બ્લોક ડાયાગ્રામ સમજાવો.}

\begin{solutionbox}

\begin{center}
\textbf{Mermaid Diagram (Code)}
\begin{verbatim}
{Shaded}
{Highlighting}[]
graph LR
    A[Source] {-{-}{} B[Source Encoder]}
    B {-{-}{} C[Channel Encoder]}
    C {-{-}{} D[Digital Modulator]}
    D {-{-}{} E[Channel]}
    E {-{-}{} F[Digital Demodulator]}
    F {-{-}{} G[Channel Decoder]}
    G {-{-}{} H[Source Decoder]}
    H {-{-}{} I[Destination]}
{Highlighting}
{Shaded}
\end{verbatim}
\end{center}

{\def\LTcaptype{none} % do not increment counter
\begin{longtable}[]{@{}
  >{\raggedright\arraybackslash}p{(\linewidth - 2\tabcolsep) * \real{0.5385}}
  >{\raggedright\arraybackslash}p{(\linewidth - 2\tabcolsep) * \real{0.4615}}@{}}
\toprule\noalign{}
\begin{minipage}[b]{\linewidth}\raggedright
બ્લોક
\end{minipage} & \begin{minipage}[b]{\linewidth}\raggedright
કાર્ય
\end{minipage} \\
\midrule\noalign{}
\endhead
\bottomrule\noalign{}
\endlastfoot
\textbf{Source} & પ્રસારિત કરવા માટે સંદેશ ઉત્પન્ન કરે છે \\
\textbf{Source Encoder} & સંદેશને ડિજિટલ ક્રમમાં રૂપાંતરિત કરે છે, વધારાનું
રિડન્ડન્સી દૂર કરે છે \\
\textbf{Channel Encoder} & ભૂલ શોધવા/સુધારવા માટે નિયંત્રિત રિડન્ડન્સી ઉમેરે છે \\
\textbf{Digital Modulator} & ડિજિટલ સિમ્બોલ્સને ચેનલ માટે યોગ્ય વેવફોર્મમાં
રૂપાંતરિત કરે છે \\
\textbf{Channel} & પ્રસારણ માધ્યમ, નોઈઝ અને ડિસ્ટોર્શન ઉમેરે છે \\
\textbf{Digital Demodulator} & પ્રાપ્ત વેવફોર્મને પાછા ડિજિટલ સિમ્બોલ્સમાં
રૂપાંતરિત કરે છે \\
\textbf{Channel Decoder} & ઉમેરેલા રિડન્ડન્સીનો ઉપયોગ કરીને ભૂલોને શોધે/સુધારે
છે \\
\textbf{Source Decoder} & ડિજિટલ ક્રમમાંથી મૂળ સંદેશ પુનઃનિર્માણ કરે છે \\
\end{longtable}
}

\end{solutionbox}
\begin{mnemonicbox}
``સંદેશને સૂચના સંરક્ષિત, ડિજિટલ માધ્યમથી ચોક્કસ ડેટા
સંચારિત''

\end{mnemonicbox}
\subsection*{પ્રશ્ન 1(ક) અથવા [7
ગુણ]}\label{uxaaauxab0uxab6uxaa8-1uxa95-uxa85uxaa5uxab5-7-uxa97uxaa3}

\textbf{યુનિટ સ્ટેપ ફંક્શન અને યુનિટ ઈમ્પલ્સ ફંક્શન સમજાવો.}

\begin{solutionbox}

{\def\LTcaptype{none} % do not increment counter
\begin{longtable}[]{@{}
  >{\raggedright\arraybackslash}p{(\linewidth - 6\tabcolsep) * \real{0.1905}}
  >{\raggedright\arraybackslash}p{(\linewidth - 6\tabcolsep) * \real{0.4048}}
  >{\raggedright\arraybackslash}p{(\linewidth - 6\tabcolsep) * \real{0.2143}}
  >{\raggedright\arraybackslash}p{(\linewidth - 6\tabcolsep) * \real{0.1905}}@{}}
\toprule\noalign{}
\begin{minipage}[b]{\linewidth}\raggedright
ફંક્શન
\end{minipage} & \begin{minipage}[b]{\linewidth}\raggedright
ગાણિતિક વ્યાખ્યા
\end{minipage} & \begin{minipage}[b]{\linewidth}\raggedright
ગુણધર્મો
\end{minipage} & \begin{minipage}[b]{\linewidth}\raggedright
ઉપયોગો
\end{minipage} \\
\midrule\noalign{}
\endhead
\bottomrule\noalign{}
\endlastfoot
\textbf{યુનિટ સ્ટેપ ફંક્શન (u(t))} & u(t) = 0 જ્યારે t \textless{} 0u(t) = 1
જ્યારે t \geq 0 & - અચાનક પરિવર્તન દર્શાવે છે- ઈમ્પલ્સ ફંક્શનનું ઈન્ટિગ્રલ & સિસ્ટમ
રિસ્પોન્સ એનાલિસિસ \\
\textbf{યુનિટ ઈમ્પલ્સ ફંક્શન (δ(t))} & δ(t) = 0 જ્યારે t \neq 0\intδ(t)dt = 1 & -
અત્યંત સાંકડો પલ્સ- સેમ્પલિંગ પ્રોપર્ટી- સ્ટેપ ફંક્શનનું ડેરિવેટિવ & સેમ્પલિંગ, સિસ્ટમ
એનાલિસિસ \\
\end{longtable}
}

\textbf{આકૃતિઓ}:

\begin{verbatim}
          Unit Step Function                 Unit Impulse Function
                    \_\_\_\_\_\_                            \^{}
                   |                                  |
                   |                                  |
                   |                               (infinite)
        \_\_\_\_\_\_\_\_\_\_\_|                                  |
                                                      |
        {-{-}{-}{-}{-}{-}{-}{-}{-}{-}0{-}{-}{-}{-}{-}{-}{-}{-}{-}{-}{-} t       {-}{-}{-}{-}{-}{-}0{-}{-}{-}{-}{-}{-}{-}{-}{-}{-}{-}{-}{-}{-}{-}{-}{-}{-}  t}
\end{verbatim}

\end{solutionbox}
\begin{mnemonicbox}
``સ્ટેપ શૂન્ય પછી સ્થિર રહે, ઈમ્પલ્સ ક્ષણિક ઉદ્ભવે અને અદૃશ્ય
થાય''

\end{mnemonicbox}
\subsection*{પ્રશ્ન 2(અ) [3
ગુણ]}\label{uxaaauxab0uxab6uxaa8-2uxa85-3-uxa97uxaa3}

\textbf{સિગ્નલ 8 બીટ/સિગ્નલ એલીમેન્ટ ધરાવે છે. જો સેકન્ડ દીઠ 1000 સિગ્નલ એલીમેન્ટ
મોકલવામાં આવે છે. બીટ રેટ શોધો.}

\begin{solutionbox}

{\def\LTcaptype{none} % do not increment counter
\begin{longtable}[]{@{}
  >{\raggedright\arraybackslash}p{(\linewidth - 2\tabcolsep) * \real{0.6111}}
  >{\raggedright\arraybackslash}p{(\linewidth - 2\tabcolsep) * \real{0.3889}}@{}}
\toprule\noalign{}
\begin{minipage}[b]{\linewidth}\raggedright
પેરામીટર
\end{minipage} & \begin{minipage}[b]{\linewidth}\raggedright
કિંમત
\end{minipage} \\
\midrule\noalign{}
\endhead
\bottomrule\noalign{}
\endlastfoot
સિગ્નલ એલીમેન્ટ દીઠ બિટ્સ & 8 બિટ્સ \\
સેકન્ડ દીઠ સિગ્નલ એલીમેન્ટ્સ & 1000 \\
\textbf{ગણતરી} & બિટ રેટ = (સિગ્નલ એલીમેન્ટ દીઠ બિટ્સ) \times (સેકન્ડ દીઠ સિગ્નલ
એલીમેન્ટ્સ) \\
\textbf{બિટ રેટ} & = 8 \times 1000 = 8000 બિટ્સ/સેકન્ડ અથવા 8 kbps \\
\end{longtable}
}

\end{solutionbox}
\begin{mnemonicbox}
``સિગ્નલ દીઠ બિટ્સ \times સેકન્ડ દીઠ સિગ્નલ = સેકન્ડ દીઠ બિટ્સ''

\end{mnemonicbox}
\subsection*{પ્રશ્ન 2(બ) [4
ગુણ]}\label{uxaaauxab0uxab6uxaa8-2uxaac-4-uxa97uxaa3}

\textbf{ઈવન અને ઓડ સિગ્નલ સમજાવો.}

\begin{solutionbox}

{\def\LTcaptype{none} % do not increment counter
\begin{longtable}[]{@{}
  >{\raggedright\arraybackslash}p{(\linewidth - 6\tabcolsep) * \real{0.2857}}
  >{\raggedright\arraybackslash}p{(\linewidth - 6\tabcolsep) * \real{0.3469}}
  >{\raggedright\arraybackslash}p{(\linewidth - 6\tabcolsep) * \real{0.1837}}
  >{\raggedright\arraybackslash}p{(\linewidth - 6\tabcolsep) * \real{0.1837}}@{}}
\toprule\noalign{}
\begin{minipage}[b]{\linewidth}\raggedright
સિગ્નલ પ્રકાર
\end{minipage} & \begin{minipage}[b]{\linewidth}\raggedright
ગાણિતિક વ્યાખ્યા
\end{minipage} & \begin{minipage}[b]{\linewidth}\raggedright
ગુણધર્મો
\end{minipage} & \begin{minipage}[b]{\linewidth}\raggedright
ઉદાહરણો
\end{minipage} \\
\midrule\noalign{}
\endhead
\bottomrule\noalign{}
\endlastfoot
\textbf{ઈવન સિગ્નલ} & x(-t) = x(t) & - y-અક્ષ પર સમમિત- કોસાઇન એક ઈવન ફંક્શન
છે & કોસાઇન ફંક્શન, \textbar t\textbar{} \\
\textbf{ઓડ સિગ્નલ} & x(-t) = -x(t) & - y-અક્ષ પર એન્ટી-સમમિત- સાઇન એક ઓડ
ફંક્શન છે & સાઇન ફંક્શન, t \\
\end{longtable}
}

\textbf{આકૃતિ}:

\begin{verbatim}
        Even Signal                        Odd Signal
            /{                                 /}
           /  {                               /}
          /    {                             /}
         /      {                           /}
 {-{-}{-}{-}{-}{-}{-}0{-}{-}{-}{-}{-}{-}{-}{-}{-}                  {-}{-}{-}{-}{-}{-}{-}0{-}{-}{-}{-}{-}{-}{-}}
         {      /                         /}
          {    /                         /}
           {  /                         /}
            {/                         v}
\end{verbatim}

\end{solutionbox}
\begin{mnemonicbox}
``ઈવન એકસરખું પ્રતિબિંબિત થાય, ઓડ વિપરીત પ્રતિબિંબિત થાય''

\end{mnemonicbox}
\subsection*{પ્રશ્ન 2(ક) [7
ગુણ]}\label{uxaaauxab0uxab6uxaa8-2uxa95-7-uxa97uxaa3}

\textbf{ASK મોડ્યુલેટર અને ડી-મોડ્યુલેટરના બ્લોક ડાયાગ્રામને વેવફોર્મ સાથે સમજાવો.}

\begin{solutionbox}

\textbf{ASK મોડ્યુલેટર:}

\begin{center}
\textbf{Mermaid Diagram (Code)}
\begin{verbatim}
{Shaded}
{Highlighting}[]
graph LR
    A[Digital Input] {-{-}{} B[Product Modulator]}
    C[Carrier Generator] {-{-}{} B}
    B {-{-}{} D[ASK Output]}
{Highlighting}
{Shaded}
\end{verbatim}
\end{center}

\textbf{ASK ડિમોડ્યુલેટર:}

\begin{center}
\textbf{Mermaid Diagram (Code)}
\begin{verbatim}
{Shaded}
{Highlighting}[]
graph LR
    A[ASK Signal] {-{-}{} B[Envelope Detector]}
    B {-{-}{} C[Comparator]}
    C {-{-}{} D[Digital Output]}
{Highlighting}
{Shaded}
\end{verbatim}
\end{center}

\textbf{વેવફોર્મ્સ:}

\begin{verbatim}
Digital Input:   \_\_\_\_\_       \_\_\_\_\_
                |     |     |     |
         \_\_\_\_\_\_\_|     |\_\_\_\_\_|     |\_\_\_\_\_

Carrier:  /{//////////////}

ASK Output:      /{//       ///}
                |      |     |      |
         \_\_\_\_\_\_\_|      |\_\_\_\_\_|      |\_\_\_\_\_
\end{verbatim}

{\def\LTcaptype{none} % do not increment counter
\begin{longtable}[]{@{}
  >{\raggedright\arraybackslash}p{(\linewidth - 2\tabcolsep) * \real{0.4615}}
  >{\raggedright\arraybackslash}p{(\linewidth - 2\tabcolsep) * \real{0.5385}}@{}}
\toprule\noalign{}
\begin{minipage}[b]{\linewidth}\raggedright
વિષય
\end{minipage} & \begin{minipage}[b]{\linewidth}\raggedright
વર્ણન
\end{minipage} \\
\midrule\noalign{}
\endhead
\bottomrule\noalign{}
\endlastfoot
\textbf{ASK મોડ્યુલેશન} & ડિજિટલ ડેટા (0 અથવા 1) અનુસાર એમ્પ્લિટ્યૂડ બદલાય છે \\
\textbf{મોડ્યુલેટર ઘટકો} & પ્રોડક્ટ મોડ્યુલેટર કેરિયરને ડિજિટલ સિગ્નલ સાથે ગુણાકાર કરે
છે \\
\textbf{ડિમોડ્યુલેટર ઘટકો} & એન્વેલોપ ડિટેક્ટર એમ્પ્લિટ્યૂડ શોધે છે, કમ્પેરેટર ડિજિટલ
સિગ્નલ પુનઃનિર્માણ કરે છે \\
\end{longtable}
}

\end{solutionbox}
\begin{mnemonicbox}
``ASK એમ્પ્લિટ્યૂડ સિગ્નલ કાંટાકૂટ''

\end{mnemonicbox}
\subsection*{પ્રશ્ન 2(અ) અથવા [3
ગુણ]}\label{uxaaauxab0uxab6uxaa8-2uxa85-uxa85uxaa5uxab5-3-uxa97uxaa3}

\textbf{સિગ્નલમાં 4000 બીટ/સેકન્ડનો બીટ રેટ અને 1000 બોદનો બોદ દર હોય છે. દરેક
સિગ્નલ એલીમેન્ટ દ્વારા કેટલા ડેટા એલીમેન્ટ વહન કરવામાં આવે છે?}

\begin{solutionbox}

{\def\LTcaptype{none} % do not increment counter
\begin{longtable}[]{@{}ll@{}}
\toprule\noalign{}
પેરામીટર & કિંમત \\
\midrule\noalign{}
\endhead
\bottomrule\noalign{}
\endlastfoot
બિટ રેટ & 4000 બિટ્સ/સેકન્ડ \\
બોદ રેટ & 1000 બોદ (સિગ્નલ એલીમેન્ટ્સ/સેકન્ડ) \\
\textbf{સૂત્ર} & ડેટા એલીમેન્ટ્સની સંખ્યા = બિટ રેટ \div બોદ રેટ \\
\textbf{સિગ્નલ દીઠ ડેટા એલીમેન્ટ્સ} & = 4000 \div 1000 = 4 બિટ્સ/સિગ્નલ એલીમેન્ટ \\
\end{longtable}
}

\end{solutionbox}
\begin{mnemonicbox}
``બિટ્સને બોદથી ભાગતા સિગ્નલ દીઠ બિટ્સ મળે''

\end{mnemonicbox}
\subsection*{પ્રશ્ન 2(બ) અથવા [4
ગુણ]}\label{uxaaauxab0uxab6uxaa8-2uxaac-uxa85uxaa5uxab5-4-uxa97uxaa3}

\textbf{પિરિઓડિક અને એપિરિઓડિક સિગ્નલ સમજાવો.}

\begin{solutionbox}

{\def\LTcaptype{none} % do not increment counter
\begin{longtable}[]{@{}
  >{\raggedright\arraybackslash}p{(\linewidth - 6\tabcolsep) * \real{0.2979}}
  >{\raggedright\arraybackslash}p{(\linewidth - 6\tabcolsep) * \real{0.1915}}
  >{\raggedright\arraybackslash}p{(\linewidth - 6\tabcolsep) * \real{0.3191}}
  >{\raggedright\arraybackslash}p{(\linewidth - 6\tabcolsep) * \real{0.1915}}@{}}
\toprule\noalign{}
\begin{minipage}[b]{\linewidth}\raggedright
સિગ્નલ પ્રકાર
\end{minipage} & \begin{minipage}[b]{\linewidth}\raggedright
વ્યાખ્યા
\end{minipage} & \begin{minipage}[b]{\linewidth}\raggedright
ગાણિતિક શરત
\end{minipage} & \begin{minipage}[b]{\linewidth}\raggedright
ઉદાહરણો
\end{minipage} \\
\midrule\noalign{}
\endhead
\bottomrule\noalign{}
\endlastfoot
\textbf{પિરિઓડિક સિગ્નલ} & ચોક્કસ સમય પછી પુનરાવર્તન થાય છે & x(t) = x(t+T)
દરેક t માટે & સાઇન વેવ, સ્ક્વેર વેવ \\
\textbf{એપિરિઓડિક સિગ્નલ} & કોઈપણ સમય પછી પુનરાવર્તન થતું નથી & x(t) \neq x(t+T)
કોઈપણ T માટે & પલ્સ, નોઈઝ \\
\end{longtable}
}

\textbf{આકૃતિ}:

\begin{verbatim}
    Periodic Signal                Aperiodic Signal
    /{    /    /                       /}
   /  {  /    /                       /  }
  /    {/    /                       /    \_\_\_\_\_\_\_\_\_\_\_}
                                      /
 One period (T) {-{-}|                 /}
\end{verbatim}

\end{solutionbox}
\begin{mnemonicbox}
``પિરિઓડિક પરફેક્ટ રીતે પુનરાવર્તિત થાય, એપિરિઓડિક હંમેશા
બદલાતું રહે''

\end{mnemonicbox}
\subsection*{પ્રશ્ન 2(ક) અથવા [7
ગુણ]}\label{uxaaauxab0uxab6uxaa8-2uxa95-uxa85uxaa5uxab5-7-uxa97uxaa3}

\textbf{PSK મોડ્યુલેટર અને ડી-મોડ્યુલેટરના બ્લોક ડાયાગ્રામને વેવફોર્મ સાથે સમજાવો.}

\begin{solutionbox}

\textbf{PSK મોડ્યુલેટર:}

\begin{center}
\textbf{Mermaid Diagram (Code)}
\begin{verbatim}
{Shaded}
{Highlighting}[]
graph LR
    A[Digital Input] {-{-}{} B[Phase Shifter]}
    C[Carrier Generator] {-{-}{} B}
    B {-{-}{} D[PSK Output]}
{Highlighting}
{Shaded}
\end{verbatim}
\end{center}

\textbf{PSK ડિમોડ્યુલેટર:}

\begin{center}
\textbf{Mermaid Diagram (Code)}
\begin{verbatim}
{Shaded}
{Highlighting}[]
graph LR
    A[PSK Signal] {-{-}{} B[Product Detector]}
    C[Carrier Recovery] {-{-}{} B}
    B {-{-}{} D[Low Pass Filter]}
    D {-{-}{} E[Decision Device]}
    E {-{-}{} F[Digital Output]}
{Highlighting}
{Shaded}
\end{verbatim}
\end{center}

\textbf{વેવફોર્મ્સ:}

\begin{verbatim}
Digital Input:   \_\_\_\_\_       \_\_\_\_\_
                |     |     |     |
         \_\_\_\_\_\_\_|     |\_\_\_\_\_|     |\_\_\_\_\_

Carrier:  /{//////////////}

PSK Output: /{////////////}
           (0^)   (180^) (0^)  (180^)
           Phase shifts at bit transitions
\end{verbatim}

{\def\LTcaptype{none} % do not increment counter
\begin{longtable}[]{@{}ll@{}}
\toprule\noalign{}
પેરામીટર & વર્ણન \\
\midrule\noalign{}
\endhead
\bottomrule\noalign{}
\endlastfoot
\textbf{PSK મોડ્યુલેશન} & ડિજિટલ ડેટા (0 અથવા 1) અનુસાર ફેઝ બદલાય છે \\
\textbf{ફેઝ સ્ટેટ્સ} & બિટ `1' માટે 0^\circ, બિટ `0' માટે 180^\circ \\
\textbf{ફાયદા} & ASK કરતાં નોઈઝ સામે વધુ પ્રતિરક્ષા \\
\end{longtable}
}

\end{solutionbox}
\begin{mnemonicbox}
``PSK ફેઝ શિફ્ટ કરે જાણકારીથી''

\end{mnemonicbox}
\subsection*{પ્રશ્ન 3(અ) [3
ગુણ]}\label{uxaaauxab0uxab6uxaa8-3uxa85-3-uxa97uxaa3}

\textbf{બ્લોક ડાયાગ્રામ અને આઉટપુટ વેવફોર્મ સાથે FSK મોડ્યુલેટરનું કાર્ય સમજાવો.}

\begin{solutionbox}

\textbf{FSK મોડ્યુલેટર બ્લોક ડાયાગ્રામ:}

\begin{center}
\textbf{Mermaid Diagram (Code)}
\begin{verbatim}
{Shaded}
{Highlighting}[]
graph LR
    A[Digital Input] {-{-}{} B[Voltage Controlled Oscillator]}
    B {-{-}{} C[FSK Output]}
{Highlighting}
{Shaded}
\end{verbatim}
\end{center}

\textbf{FSK વેવફોર્મ્સ:}

\begin{verbatim}
Digital Input:   \_\_\_\_\_       \_\_\_\_\_
                |     |     |     |
         \_\_\_\_\_\_\_|     |\_\_\_\_\_|     |\_\_\_\_\_

FSK Output: /{//  /////  ///}
           (f1)    (f2)       (f1)
\end{verbatim}

\begin{itemize}
\tightlist
\item
  \textbf{સિદ્ધાંત}: ડિજિટલ બિટ `1' ફ્રિક્વન્સી f1 સાથે કેરિયર મોકલે છે, બિટ `0'
  ફ્રિક્વન્સી f2 સાથે કેરિયર મોકલે છે
\item
  \textbf{કાર્યપ્રણાલી}: વોલ્ટેજ કંટ્રોલ્ડ ઓસિલેટર ઇનપુટ બિટ મૂલ્ય આધારે ફ્રિક્વન્સી
  બદલે છે
\end{itemize}

\end{solutionbox}
\begin{mnemonicbox}
``ફ્રિક્વન્સી શિફ્ટ કરે જાણકારી સંચાર''

\end{mnemonicbox}
\subsection*{પ્રશ્ન 3(બ) [4
ગુણ]}\label{uxaaauxab0uxab6uxaa8-3uxaac-4-uxa97uxaa3}

\textbf{1010110110 ના ક્રમ માટે PSK મોડ્યુલેશન વેવફોર્મ દોરો.}

\begin{solutionbox}

\begin{verbatim}
Digital Input:  \_\_\_     \_\_\_     \_\_\_\_\_\_\_     \_\_\_\_\_\_\_   
               |   |   |   |   |       |   |       |  
          \_\_\_\_\_|   |\_\_\_|   |\_\_\_|       |\_\_\_|       |\_\_\_
          
          1     0     1     0     1     1     0     1     1     0
          
PSK Output:    
          /{/ // // // // // // // // //}
          0^   180^ 0^   180^ 0^   0^   180^ 0^   0^   180^
          
Phase:    0^   180^ 0^   180^ 0^   0^   180^ 0^   0^   180^
\end{verbatim}

\textbf{PSK મોડ્યુલેશન માટે ટેબલ:}

{\def\LTcaptype{none} % do not increment counter
\begin{longtable}[]{@{}ll@{}}
\toprule\noalign{}
બિટ & ફેઝ \\
\midrule\noalign{}
\endhead
\bottomrule\noalign{}
\endlastfoot
1 & 0^\circ \\
0 & 180^\circ \\
\end{longtable}
}

\end{solutionbox}
\begin{mnemonicbox}
``એક-શૂન્ય, ફેઝ-શિફ્ટ, સિગ્નલ મોડ્યુલેટેડ''

\end{mnemonicbox}
\subsection*{પ્રશ્ન 3(ક) [7
ગુણ]}\label{uxaaauxab0uxab6uxaa8-3uxa95-7-uxa97uxaa3}

\textbf{1100110101 ના ક્રમ માટે ASK અને FSK મોડ્યુલેશન વેવફોર્મ દોરો.}

\begin{solutionbox}

\textbf{ડિજિટલ ઇનપુટ સિક્વન્સ: 1100110101}

\begin{verbatim}
Digital Input:  \_\_\_\_\_\_\_         \_\_\_\_\_\_\_     \_\_\_     \_\_\_
               |       |       |       |   |   |   |   |
          \_\_\_\_\_|       |\_\_\_\_\_\_\_|       |\_\_\_|   |\_\_\_|   |\_\_\_
          
          1     1     0     0     1     1     0     1     0     1
          
ASK Output:     
          /{/  //         //  //        //        //}
          
          On    On    Off   Off   On    On    Off   On    Off   On
          
FSK Output:     
          /{/  //  /// /// //  //  /// //  /// //}
          
          f1    f1    f2    f2    f1    f1    f2    f1    f2    f1
\end{verbatim}

\textbf{તુલના માટે ટેબલ:}

{\def\LTcaptype{none} % do not increment counter
\begin{longtable}[]{@{}lll@{}}
\toprule\noalign{}
બિટ & ASK & FSK \\
\midrule\noalign{}
\endhead
\bottomrule\noalign{}
\endlastfoot
1 & કેરિયર ON (ઉચ્ચ એમ્પ્લિટ્યૂડ) & ઉચ્ચ ફ્રિક્વન્સી (f1) \\
0 & કેરિયર OFF (શૂન્ય/નીચી એમ્પ્લિટ્યૂડ) & નીચી ફ્રિક્વન્સી (f2) \\
\end{longtable}
}

\end{solutionbox}
\begin{mnemonicbox}
``એમ્પ્લિટ્યૂડ જાણકારી દર્શાવે, ફ્રિક્વન્સી જાણકારી બદલાવે''

\end{mnemonicbox}
\subsection*{પ્રશ્ન 3(અ) અથવા [3
ગુણ]}\label{uxaaauxab0uxab6uxaa8-3uxa85-uxa85uxaa5uxab5-3-uxa97uxaa3}

\textbf{બ્લોક ડાયાગ્રામ અને આઉટપુટ વેવફોર્મ સાથે MSK મોડ્યુલેટરનું કાર્ય સમજાવો.}

\begin{solutionbox}

\textbf{MSK મોડ્યુલેટર બ્લોક ડાયાગ્રામ:}

\begin{center}
\textbf{Mermaid Diagram (Code)}
\begin{verbatim}
{Shaded}
{Highlighting}[]
graph LR
    A[Digital Input] {-{-}{} B[Serial to Parallel]}
    B {-{-}{}|I{-}Channel| C[I{-}Channel Modulator]}
    B {-{-}{}|Q{-}Channel| D[Q{-}Channel Modulator]}
    E[Carrier Generator] {-{-}{} C}
    E {-{-}{}|90^ Phase Shift| D}
    C {-{-}{} F[Adder]}
    D {-{-}{} F}
    F {-{-}{} G[MSK Output]}
{Highlighting}
{Shaded}
\end{verbatim}
\end{center}

\textbf{MSK વિશેષતાઓ:}

\begin{itemize}
\tightlist
\item
  કન્ટિન્યુઅસ ફેઝ FSK જેમાં ફ્રિક્વન્સી ડેવિએશન એક્ઝેક્ટલી બિટ રેટના અર્ધા જેટલું હોય છે
\item
  ફેઝમાં ફેરફાર સરળતાથી થાય છે (અચાનક ફેઝ પરિવર્તન નથી)
\item
  FSK કરતાં વધુ સારી સ્પેક્ટ્રલ કાર્યક્ષમતા
\end{itemize}

\end{solutionbox}
\begin{mnemonicbox}
``મિનિમમ શિફ્ટ સ્પેક્ટ્રમને સાંકડું રાખે''

\end{mnemonicbox}
\subsection*{પ્રશ્ન 3(બ) અથવા [4
ગુણ]}\label{uxaaauxab0uxab6uxaa8-3uxaac-uxa85uxaa5uxab5-4-uxa97uxaa3}

\textbf{8-PSK અને 16-QAM ના નક્ષત્ર રેખાંકિત દોરો.}

\begin{solutionbox}

\textbf{8-PSK નક્ષત્ર રેખાંકિત:}

\begin{verbatim}
           001  *    *  000
               /|{  /|}
                |    |
          010 * |    | * 111
              { |    | /}
               {|    |/}
          011  *     *  110
               /|{  /|}
               / {   /}
          100 *   { /  * 101}
\end{verbatim}

\textbf{16-QAM નક્ષત્ર રેખાંકિત:}

\begin{verbatim}
     *     *     *     *
    0000  0001  0100  0101
     
     *     *     *     *
    0010  0011  0110  0111
    
     *     *     *     *
    1000  1001  1100  1101
     
     *     *     *     *
    1010  1011  1110  1111
\end{verbatim}

{\def\LTcaptype{none} % do not increment counter
\begin{longtable}[]{@{}
  >{\raggedright\arraybackslash}p{(\linewidth - 2\tabcolsep) * \real{0.6316}}
  >{\raggedright\arraybackslash}p{(\linewidth - 2\tabcolsep) * \real{0.3684}}@{}}
\toprule\noalign{}
\begin{minipage}[b]{\linewidth}\raggedright
મોડ્યુલેશન
\end{minipage} & \begin{minipage}[b]{\linewidth}\raggedright
વર્ણન
\end{minipage} \\
\midrule\noalign{}
\endhead
\bottomrule\noalign{}
\endlastfoot
\textbf{8-PSK} & 8 પોઇન્ટ્સ વર્તુળ પર સરખા અંતરે, 3 બિટ્સ પ્રતિ સિમ્બોલ \\
\textbf{16-QAM} & 16 પોઇન્ટ્સ ચોરસ ગ્રીડમાં, બદલાતા એમ્પ્લિટ્યૂડ અને ફેઝ, 4 બિટ્સ
પ્રતિ સિમ્બોલ \\
\end{longtable}
}

\end{solutionbox}
\begin{mnemonicbox}
``PSK પોઇન્ટ્સ એક વર્તુળ પર, QAM ચોરસ એમ્પ્લિટ્યૂડ મેટ્રિક્સ''

\end{mnemonicbox}
\subsection*{પ્રશ્ન 3(ક) અથવા [7
ગુણ]}\label{uxaaauxab0uxab6uxaa8-3uxa95-uxa85uxaa5uxab5-7-uxa97uxaa3}

\textbf{1010101011 માટે BPSK અને QPSK મોડ્યુલેશન વેવફોર્મ દોરો.}

\begin{solutionbox}

\textbf{BPSK મોડ્યુલેશન:}

\begin{verbatim}
Digital Input:  \_\_\_     \_\_\_     \_\_\_     \_\_\_       
               |   |   |   |   |   |   |   |    
          \_\_\_\_\_|   |\_\_\_|   |\_\_\_|   |\_\_\_|   |\_\_\_\_\_
          
          1     0     1     0     1     0     1     0     1     1
          
BPSK Output:    
          /{/  //  //  //  //  //  //  //  //  //}
          0^    180^  0^    180^  0^    180^  0^    180^  0^    0^
\end{verbatim}

\textbf{QPSK મોડ્યુલેશન (બિટ્સને જોડીમાં વર્ગીકૃત કરીને):}

\begin{verbatim}
                    10           10           10           11
Input Pairs:    |{-{-}{-}{-}{-}{-}{-}{-}|   |{-}{-}{-}{-}{-}{-}{-}{-}|   |{-}{-}{-}{-}{-}{-}{-}{-}|   |{-}{-}{-}{-}{-}{-}{-}{-}|}
                
I{-Channel:      \_\_\_      \_\_\_      \_\_\_      \_\_\_      \_\_\_}
               |   |    |   |    |   |    |   |    |   |
          \_\_\_\_\_|   |\_\_\_\_|   |\_\_\_\_|   |\_\_\_\_|   |\_\_\_\_|   |\_\_\_\_
                1      0      1      0      1      0      1      1
                
Q{-Channel:     \_\_\_      \_\_\_      \_\_\_           \_\_\_}
              |   |    |   |    |   |         |   |
          \_\_\_\_|   |\_\_\_\_|   |\_\_\_\_|   |\_\_\_\_\_\_\_\_\_|   |\_\_\_\_
                0      1      0      1      0      1      1
                
QPSK Phase:     90^     270^    90^     270^    90^     270^    90^     45^
\end{verbatim}

{\def\LTcaptype{none} % do not increment counter
\begin{longtable}[]{@{}ll@{}}
\toprule\noalign{}
બિટ જોડી & QPSK ફેઝ \\
\midrule\noalign{}
\endhead
\bottomrule\noalign{}
\endlastfoot
10 & 90^\circ \\
00 & 180^\circ \\
01 & 270^\circ \\
11 & 0^\circ \\
\end{longtable}
}

\end{solutionbox}
\begin{mnemonicbox}
``બાઇનરી ફેઝ શિફ્ટ કી, ક્વોડ્રેચર ફેઝ શિફ્ટ કી''

\end{mnemonicbox}
\subsection*{પ્રશ્ન 4(અ) [3
ગુણ]}\label{uxaaauxab0uxab6uxaa8-4uxa85-3-uxa97uxaa3}

\textbf{નીચેના સંભવિત ક્રમ માટે શેનોન ફેનો કોડનો ઉપયોગ કરીને ડેટાને એન્કોડ કરો. P
= \{0.30, 0.25, 0.20, 0.12, 0.08, 0.05\}}

\begin{solutionbox}

{\def\LTcaptype{none} % do not increment counter
\begin{longtable}[]{@{}lll@{}}
\toprule\noalign{}
સિમ્બોલ & સંભાવના & શેનોન-ફેનો કોડ \\
\midrule\noalign{}
\endhead
\bottomrule\noalign{}
\endlastfoot
S1 & 0.30 & 00 \\
S2 & 0.25 & 01 \\
S3 & 0.20 & 10 \\
S4 & 0.12 & 110 \\
S5 & 0.08 & 1110 \\
S6 & 0.05 & 1111 \\
\end{longtable}
}

\textbf{પ્રક્રિયા:}

\begin{enumerate}
\tightlist
\item
  સિમ્બોલ્સને ઘટતી સંભાવના અનુસાર ગોઠવો
\item
  લગભગ સમાન સંભાવના સાથે બે જૂથોમાં વિભાજિત કરો (0.30+0.25=0.55,
  0.20+0.12+0.08+0.05=0.45)
\item
  પ્રથમ જૂથને 0, બીજા જૂથને 1 આપો
\item
  દરેક પેટા જૂથ માટે આ પ્રક્રિયા પુનરાવર્તિત કરો
\end{enumerate}

\end{solutionbox}
\begin{mnemonicbox}
``વિભાજન, ફેનો વહેંચે, કોડ કાર્યક્ષમ''

\end{mnemonicbox}
\subsection*{પ્રશ્ન 4(બ) [4
ગુણ]}\label{uxaaauxab0uxab6uxaa8-4uxaac-4-uxa97uxaa3}

\textbf{હેમિંગ કોડ સમજાવો.}

\begin{solutionbox}

{\def\LTcaptype{none} % do not increment counter
\begin{longtable}[]{@{}
  >{\raggedright\arraybackslash}p{(\linewidth - 2\tabcolsep) * \real{0.4615}}
  >{\raggedright\arraybackslash}p{(\linewidth - 2\tabcolsep) * \real{0.5385}}@{}}
\toprule\noalign{}
\begin{minipage}[b]{\linewidth}\raggedright
પાસું
\end{minipage} & \begin{minipage}[b]{\linewidth}\raggedright
વર્ણન
\end{minipage} \\
\midrule\noalign{}
\endhead
\bottomrule\noalign{}
\endlastfoot
\textbf{વ્યાખ્યા} & લિનિયર ઇરર-કરેક્ટિંગ કોડ જે ડબલ ભૂલોને શોધે છે અને સિંગલ ભૂલોને
સુધારે છે \\
\textbf{પેરિટી બિટ્સ} & m ડેટા બિટ્સ માટે, k પેરિટી બિટ્સ જોઈએ જ્યાં 2\^{}k \geq
m+k+1 \\
\textbf{પોઝિશન} & પેરિટી બિટ્સ 1, 2, 4, 8, 16\ldots{} (2ની પાવર) સ્થાનો પર
મુકાય છે \\
\textbf{ભૂલ શોધ} & ભૂલની સ્થિતિ શોધવા માટે સિન્ડ્રોમ ગણતરી \\
\end{longtable}
}

\textbf{ઉદાહરણ હેમિંગ(7,4):}

\begin{verbatim}
Positions:  1   2   3   4   5   6   7
            P1  P2  D1  P4  D2  D3  D4
            
Parity check equations:
P1 checks: P1, D1, D2, D4
P2 checks: P2, D1, D3, D4
P4 checks: P4, D2, D3, D4
\end{verbatim}

\end{solutionbox}
\begin{mnemonicbox}
``હેમિંગ હેન્ડલ બિટ ભૂલો''

\end{mnemonicbox}
\subsection*{પ્રશ્ન 4(ક) [7
ગુણ]}\label{uxaaauxab0uxab6uxaa8-4uxa95-7-uxa97uxaa3}

\textbf{TDMA અને FDMA ની સરખામણી કરો.}

\begin{solutionbox}

{\def\LTcaptype{none} % do not increment counter
\begin{longtable}[]{@{}
  >{\raggedright\arraybackslash}p{(\linewidth - 4\tabcolsep) * \real{0.1209}}
  >{\raggedright\arraybackslash}p{(\linewidth - 4\tabcolsep) * \real{0.4176}}
  >{\raggedright\arraybackslash}p{(\linewidth - 4\tabcolsep) * \real{0.4615}}@{}}
\toprule\noalign{}
\begin{minipage}[b]{\linewidth}\raggedright
પેરામીટર
\end{minipage} & \begin{minipage}[b]{\linewidth}\raggedright
TDMA (ટાઇમ ડિવિઝન મલ્ટિપલ એક્સેસ)
\end{minipage} & \begin{minipage}[b]{\linewidth}\raggedright
FDMA (ફ્રિક્વન્સી ડિવિઝન મલ્ટિપલ એક્સેસ)
\end{minipage} \\
\midrule\noalign{}
\endhead
\bottomrule\noalign{}
\endlastfoot
\textbf{મૂળભૂત સિદ્ધાંત} & ચેનલને સમય સ્લોટ્સ દ્વારા વિભાજિત કરે છે & ચેનલને ફ્રિક્વન્સી
બેન્ડ્સ દ્વારા વિભાજિત કરે છે \\
\textbf{રિસોર્સ એલોકેશન} & દરેક યુઝરને ટૂંકા સમય માટે સંપૂર્ણ બેન્ડવિડ્થ મળે & દરેક યુઝરને
બેન્ડવિડ્થનો ભાગ હંમેશા મળે \\
\textbf{ગાર્ડ પીરિયડ} & સ્લોટ્સ વચ્ચે ટાઈમ ગાર્ડ બેન્ડ્સ & ચેનલો વચ્ચે ફ્રિક્વન્સી ગાર્ડ
બેન્ડ્સ \\
\textbf{સિન્ક્રોનાઈઝેશન} & ચુસ્ત સમય સિન્ક્રોનાઈઝેશન જરૂરી & સમય સિન્ક્રોનાઈઝેશનની
જરૂર નથી \\
\textbf{કાર્યક્ષમતા} & બર્સ્ટ ટ્રાન્સમિશનને કારણે ઉચ્ચ & ફિક્સ્ડ એસાઇન્મેન્ટને કારણે
નીચી \\
\textbf{જટિલતા} & વધુ જટિલ & તુલનાત્મક રીતે સરળ \\
\textbf{ઉદાહરણો} & GSM, DECT & FM રેડિયો, પરંપરાગત સેટેલાઇટ સિસ્ટમ્સ \\
\end{longtable}
}

\textbf{આકૃતિ:}

\begin{verbatim}
TDMA:                          FDMA:
        User 1  User 2  User 3         \^{}
Time    |{-{-}{-}{-}{-}|{-}{-}{-}{-}{-}|{-}{-}{-}{-}{-}|{-}{-}{-}        | User 3}
slots   |{-{-}{-}{-}{-}|{-}{-}{-}{-}{-}|{-}{-}{-}{-}{-}|{-}{-}{-}        |{-}{-}{-}{-}{-}{-}{-}}
        |{-{-}{-}{-}{-}|{-}{-}{-}{-}{-}|{-}{-}{-}{-}{-}|{-}{-}{-}  Freq. | User 2}
                                       |{-{-}{-}{-}{-}{-}{-}}
                                       | User 1
                                       |{-{-}{-}{-}{-}{-}{-}{-}{-}}
                                          Time
\end{verbatim}

\end{solutionbox}
\begin{mnemonicbox}
``સમય વિભાજિત મલ્ટિપલ એક્સેસ, ફ્રિક્વન્સી વિભાજિત મલ્ટિપલ
એક્સેસ''

\end{mnemonicbox}
\subsection*{પ્રશ્ન 4(અ) અથવા [3
ગુણ]}\label{uxaaauxab0uxab6uxaa8-4uxa85-uxa85uxaa5uxab5-3-uxa97uxaa3}

\textbf{નીચેના સંભવિત ક્રમ માટે હફમેન કોડનો ઉપયોગ કરીને ડેટાને એન્કોડ કરો. P =
\{0.4, 0.2, 0.2, 0.1, 0.1\}}

\begin{solutionbox}

{\def\LTcaptype{none} % do not increment counter
\begin{longtable}[]{@{}lll@{}}
\toprule\noalign{}
સિમ્બોલ & સંભાવના & હફમેન કોડ \\
\midrule\noalign{}
\endhead
\bottomrule\noalign{}
\endlastfoot
S1 & 0.4 & 0 \\
S2 & 0.2 & 10 \\
S3 & 0.2 & 11 \\
S4 & 0.1 & 100 \\
S5 & 0.1 & 101 \\
\end{longtable}
}

\textbf{પ્રક્રિયા:}

\begin{enumerate}
\tightlist
\item
  ક્રમાંકિત સંભાવના સાથે શરૂ કરો
\item
  સૌથી નીચી બે સંભાવનાઓને જોડો (0.1+0.1=0.2)
\item
  ફરીથી ગોઠવો અને માત્ર બે નોડ્સ રહે ત્યાં સુધી પુનરાવર્તન કરો
\item
  ટ્રી પર ફરીને બિટ્સ આપો
\end{enumerate}

\textbf{ટ્રી કન્સ્ટ્રક્શન:}

\begin{verbatim}
                  1.0
                 /   {}
                /     {}
             0.6       0.4(S1)
            /   {}
           /     {}
        0.4      0.2(S2,S3)
       /   {     / }
    0.2    0.2  0  1
   /   {}
0.1    0.1
\end{verbatim}

\end{solutionbox}
\begin{mnemonicbox}
``હફમેન હાઈ-ફ્રિક્વન્સી ડેટા એન્કોડ કરે''

\end{mnemonicbox}
\subsection*{પ્રશ્ન 4(બ) અથવા [4
ગુણ]}\label{uxaaauxab0uxab6uxaa8-4uxaac-uxa85uxaa5uxab5-4-uxa97uxaa3}

\textbf{પેરિટી કોડ સમજાવો.}

\begin{solutionbox}

{\def\LTcaptype{none} % do not increment counter
\begin{longtable}[]{@{}
  >{\raggedright\arraybackslash}p{(\linewidth - 2\tabcolsep) * \real{0.4615}}
  >{\raggedright\arraybackslash}p{(\linewidth - 2\tabcolsep) * \real{0.5385}}@{}}
\toprule\noalign{}
\begin{minipage}[b]{\linewidth}\raggedright
પાસું
\end{minipage} & \begin{minipage}[b]{\linewidth}\raggedright
વર્ણન
\end{minipage} \\
\midrule\noalign{}
\endhead
\bottomrule\noalign{}
\endlastfoot
\textbf{વ્યાખ્યા} & સરળ ભૂલ શોધ સ્કીમ જે પેરિટી બિટ ઉમેરે છે \\
\textbf{પ્રકારો} & ઈવન પેરિટી: કુલ 1ની સંખ્યા ઈવનઓડ પેરિટી: કુલ 1ની સંખ્યા ઓડ \\
\textbf{ગણતરી} & પેરિટી બિટ ઉત્પન્ન કરવા માટે બધા ડેટા બિટ્સને XOR કરો \\
\textbf{ક્ષમતા} & ઓડ સંખ્યાની ભૂલોને શોધે, ભૂલોને સુધારી શકતું નથી \\
\end{longtable}
}

\textbf{ઉદાહરણો:}

\begin{verbatim}
Even Parity:
Data: 1011  Parity: 0  Coded: 10110 (Even number of 1s: 4)

Odd Parity:
Data: 1011  Parity: 1  Coded: 10111 (Odd number of 1s: 5)
\end{verbatim}

\end{solutionbox}
\begin{mnemonicbox}
``પેરિટી પ્રાથમિક ભૂલ શોધ પૂરી પાડે''

\end{mnemonicbox}
\subsection*{પ્રશ્ન 4(ક) અથવા [7
ગુણ]}\label{uxaaauxab0uxab6uxaa8-4uxa95-uxa85uxaa5uxab5-7-uxa97uxaa3}

\textbf{FDMA ટેકનિકને વિગતવાર સમજાવો.}

\begin{solutionbox}

\textbf{FDMA (ફ્રિક્વન્સી ડિવિઝન મલ્ટિપલ એક્સેસ):}

\begin{center}
\textbf{Mermaid Diagram (Code)}
\begin{verbatim}
{Shaded}
{Highlighting}[]
graph TD
    A[Available Bandwidth] {-{-}{} B[Frequency Division]}
    B {-{-}{} C[User 1 Channel]}
    B {-{-}{} D[User 2 Channel]}
    B {-{-}{} E[User 3 Channel]}
    B {-{-}{} F[User N Channel]}
{Highlighting}
{Shaded}
\end{verbatim}
\end{center}

{\def\LTcaptype{none} % do not increment counter
\begin{longtable}[]{@{}
  >{\raggedright\arraybackslash}p{(\linewidth - 2\tabcolsep) * \real{0.6111}}
  >{\raggedright\arraybackslash}p{(\linewidth - 2\tabcolsep) * \real{0.3889}}@{}}
\toprule\noalign{}
\begin{minipage}[b]{\linewidth}\raggedright
પેરામીટર
\end{minipage} & \begin{minipage}[b]{\linewidth}\raggedright
વર્ણન
\end{minipage} \\
\midrule\noalign{}
\endhead
\bottomrule\noalign{}
\endlastfoot
\textbf{મૂળભૂત સિદ્ધાંત} & કુલ બેન્ડવિડ્થને નોન-ઓવરલેપિંગ ફ્રિક્વન્સી બેન્ડ્સમાં વિભાજિત
કરવામાં આવે છે \\
\textbf{ચેનલ એસાઇનમેન્ટ} & દરેક યુઝરને સમર્પિત ફ્રિક્વન્સી બેન્ડ સોંપવામાં આવે છે \\
\textbf{ગાર્ડ બેન્ડ્સ} & દખલને રોકવા માટે ચેનલો વચ્ચે નાના ફ્રિક્વન્સી અંતરો \\
\textbf{ડુપ્લેક્સિંગ} & સામાન્ય રીતે FDD (ફ્રિક્વન્સી ડિવિઝન ડુપ્લેક્સિંગ) સાથે અમલમાં
મુકાય છે \\
\textbf{ફાયદા} & સરળ અમલીકરણ, સિન્ક્રોનાઈઝેશનની જરૂર નથી \\
\textbf{ગેરફાયદા} & બર્સ્ટી ટ્રાફિક માટે અકાર્યક્ષમ, ફિક્સ્ડ એલોકેશન બેન્ડવિડ્થ બગાડે
છે \\
\textbf{એપ્લિકેશન્સ} & AM/FM રેડિયો, પરંપરાગત કેબલ ટીવી, પ્રથમ પેઢીના મોબાઇલ
સિસ્ટમ્સ \\
\end{longtable}
}

\textbf{ફ્રિક્વન્સી એલોકેશન:}

\begin{verbatim}
Frequency
    \^{}
    |   Guard Bands
    |    ↓  ↓  ↓  ↓  ↓
    |   |{-{-}|{-}{-}|{-}{-}|{-}{-}|{-}{-}}
    |   |  |  |  |  |
    |   |  |  |  |  |{-{-} User N}
    |   |  |  |  |
    |   |  |  |  |{-{-} User 3}
    |   |  |  |
    |   |  |  |{-{-} User 2}
    |   |  |
    |   |  |{-{-} User 1}
    |   |
    |{-{-}{-}|{-}{-}{-}{-}{-}{-}{-}{-}{-}{-}{-}{-}{-}{-}{-}{-}{-} Time}
\end{verbatim}

\end{solutionbox}
\begin{mnemonicbox}
``ફિક્સ્ડ ડિવિઝન મલ્ટિપલ એક્સેસ''

\end{mnemonicbox}
\subsection*{પ્રશ્ન 5(અ) [3
ગુણ]}\label{uxaaauxab0uxab6uxaa8-5uxa85-3-uxa97uxaa3}

\textbf{E1 કેરીયર સિસ્ટમ સમજાવો.}

\begin{solutionbox}

{\def\LTcaptype{none} % do not increment counter
\begin{longtable}[]{@{}ll@{}}
\toprule\noalign{}
પેરામીટર & વર્ણન \\
\midrule\noalign{}
\endhead
\bottomrule\noalign{}
\endlastfoot
\textbf{વર્ણન} & યુરોપિયન સ્ટાન્ડર્ડ ડિજિટલ ટ્રાન્સમિશન ફોર્મેટ \\
\textbf{ક્ષમતા} & 2.048 Mbps \\
\textbf{ચેનલ સ્ટ્રક્ચર} & 32 ટાઇમ સ્લોટ્સ (0-31 સુધી ક્રમાંકિત) \\
\textbf{વોઇસ ચેનલ્સ} & 30 વોઇસ ચેનલ્સ (દરેક 64 kbps) \\
\textbf{સિગ્નલિંગ} & સિગ્નલિંગ માટે ટાઇમ સ્લોટ 16 \\
\textbf{ફ્રેમ એલાઇન્મેન્ટ} & સિન્ક્રોનાઈઝેશન માટે ટાઇમ સ્લોટ 0 \\
\end{longtable}
}

\textbf{આકૃતિ:}

\begin{verbatim}
One E1 Frame (32 time slots)
 \_\_\_\_\_\_\_\_\_\_\_\_\_\_\_\_\_\_\_\_\_\_\_\_\_\_\_\_\_\_\_\_\_\_\_\_\_\_\_\_\_\_\_\_\_\_\_\_\_\_\_\_\_\_\_
|   |   |   |   |   |   |   |   |   |   |   |   |   |   |
| 0 | 1 | 2 |...| 15| 16| 17|...| 30| 31| 0 | 1 | 2 |...|
|\_\_\_|\_\_\_|\_\_\_|\_\_\_|\_\_\_|\_\_\_|\_\_\_|\_\_\_|\_\_\_|\_\_\_|\_\_\_|\_\_\_|\_\_\_|\_\_\_|

TS0: Frame alignment
TS16: Signaling
TS1{-15, TS17{-}31: Voice/data channels (30 channels)}
\end{verbatim}

\end{solutionbox}
\begin{mnemonicbox}
``E1 30 + 2 ટાઇમ સ્લોટ્સ''

\end{mnemonicbox}
\subsection*{પ્રશ્ન 5(બ) [4
ગુણ]}\label{uxaaauxab0uxab6uxaa8-5uxaac-4-uxa97uxaa3}

\textbf{TDMA એક્સેસ ટેકનિક સમજાવો.}

\begin{solutionbox}

{\def\LTcaptype{none} % do not increment counter
\begin{longtable}[]{@{}
  >{\raggedright\arraybackslash}p{(\linewidth - 2\tabcolsep) * \real{0.6111}}
  >{\raggedright\arraybackslash}p{(\linewidth - 2\tabcolsep) * \real{0.3889}}@{}}
\toprule\noalign{}
\begin{minipage}[b]{\linewidth}\raggedright
પેરામીટર
\end{minipage} & \begin{minipage}[b]{\linewidth}\raggedright
વર્ણન
\end{minipage} \\
\midrule\noalign{}
\endhead
\bottomrule\noalign{}
\endlastfoot
\textbf{વ્યાખ્યા} & મલ્ટિપલ એક્સેસ ટેકનિક જે સમયને વિભિન્ન યુઝર્સ માટે સ્લોટ્સમાં
વિભાજિત કરે છે \\
\textbf{કાર્ય સિદ્ધાંત} & દરેક યુઝરને ટૂંકા સમય માટે સંપૂર્ણ બેન્ડવિડ્થ મળે છે \\
\textbf{ફ્રેમ સ્ટ્રક્ચર} & સમય ફ્રેમ્સમાં વિભાજિત, ફ્રેમ્સ સ્લોટ્સમાં વિભાજિત \\
\textbf{ગાર્ડ ટાઇમ} & ઓવરલેપ અટકાવવા માટે સ્લોટ્સ વચ્ચે નાનો સમય અંતરાલ \\
\textbf{સિન્ક્રોનાઈઝેશન} & ચોક્કસ સમય સિન્ક્રોનાઈઝેશનની જરૂર પડે છે \\
\end{longtable}
}

\textbf{TDMA ફ્રેમ સ્ટ્રક્ચર:}

\begin{verbatim}
             One TDMA Frame
 \_\_\_\_\_\_\_\_\_\_\_\_\_\_\_\_\_\_\_\_\_\_\_\_\_\_\_\_\_\_\_\_\_\_\_\_\_\_\_\_
|      |      |      |      |      |     |
| TS 1 | TS 2 | TS 3 | TS 4 | TS 5 | ... |
|\_\_\_\_\_\_|\_\_\_\_\_\_|\_\_\_\_\_\_|\_\_\_\_\_\_|\_\_\_\_\_\_|\_\_\_\_\_|
   |      |      |
   |      |      |{-{-}{-} User 3}
   |      |
   |      |{-{-}{-} User 2}
   |
   |{-{-}{-} User 1}

Each time slot (TS) contains:
{- User data}
{- Guard time}
{- Synchronization bits}
{- Control bits}
\end{verbatim}

\end{solutionbox}
\begin{mnemonicbox}
``સમય વિભાજિત મલ્ટિપલ એક્સેસ''

\end{mnemonicbox}
\subsection*{પ્રશ્ન 5(ક) [7
ગુણ]}\label{uxaaauxab0uxab6uxaa8-5uxa95-7-uxa97uxaa3}

\textbf{IoT − ખ્યાલ, લક્ષણો, ફાયદા અને ગેરફાયદા સમજાવો.}

\begin{solutionbox}

\textbf{IoT ખ્યાલ:}

\begin{center}
\textbf{Mermaid Diagram (Code)}
\begin{verbatim}
{Shaded}
{Highlighting}[]
graph LR
    A[ભૌતિક વસ્તુઓ] {-{-}{}|સેન્સર્સ| B[ઈન્ટરનેટ કનેક્ટિવિટી]}
    B {-{-}{} C[ડેટા સંગ્રહ]}
    C {-{-}{} D[ડેટા એનાલિસિસ]}
    D {-{-}{} E[ઓટોમેટેડ એક્શન્સ]}
    E {-{-}{} A}
{Highlighting}
{Shaded}
\end{verbatim}
\end{center}

{\def\LTcaptype{none} % do not increment counter
\begin{longtable}[]{@{}
  >{\raggedright\arraybackslash}p{(\linewidth - 2\tabcolsep) * \real{0.4615}}
  >{\raggedright\arraybackslash}p{(\linewidth - 2\tabcolsep) * \real{0.5385}}@{}}
\toprule\noalign{}
\begin{minipage}[b]{\linewidth}\raggedright
પાસું
\end{minipage} & \begin{minipage}[b]{\linewidth}\raggedright
વર્ણન
\end{minipage} \\
\midrule\noalign{}
\endhead
\bottomrule\noalign{}
\endlastfoot
\textbf{ખ્યાલ} & ભૌતિક વસ્તુઓનું નેટવર્ક જેમાં સેન્સર્સ, સોફ્ટવેર, અને કનેક્ટિવિટી એમ્બેડ
કરેલા હોય \\
\textbf{લક્ષણો} & - કનેક્ટિવિટી (ઇન્ટરનેટ સાથે જોડાયેલા ડિવાઇસિસ)- ઇન્ટેલિજન્સ
(સ્માર્ટ નિર્ણય લેવાની ક્ષમતા)- સેન્સિંગ (પર્યાવરણમાંથી ડેટા એકત્રિત કરવું)- ઓટોમેશન
(ન્યૂનતમ માનવ હસ્તક્ષેપ)- સ્કેલેબિલિટી (ઘણા ડિવાઇસિસ સંભાળે) \\
\textbf{ફાયદા} & - સુધારેલ કાર્યક્ષમતા અને ઉત્પાદકતા- બેહતર સંસાધન વ્યવસ્થાપન- વધુ
સારા નિર્ણયો લેવાની ક્ષમતા- સુવિધા અને સમય બચાવ- નવા વ્યાવસાયિક અવસરો \\
\textbf{ગેરફાયદા} & - સુરક્ષા કમજોરીઓ- ગોપનીયતા સંબંધી ચિંતાઓ- અમલીકરણમાં
જટિલતા- સુસંગતતા સમસ્યાઓ- ઇન્ટરનેટ પર નિર્ભરતા \\
\end{longtable}
}

\textbf{એપ્લિકેશન ક્ષેત્રો:}

\begin{itemize}
\tightlist
\item
  સ્માર્ટ હોમ્સ, શહેરો
\item
  હેલ્થકેર મોનિટરિંગ
\item
  ઔદ્યોગિક ઓટોમેશન
\item
  કૃષિ
\item
  પરિવહન
\end{itemize}

\end{solutionbox}
\begin{mnemonicbox}
``ઇન્ટરનેટ ઓફ થિંગ્સ: કનેક્ટેડ, ઓટોમેટેડ, સ્માર્ટર નિર્ણયો''

\end{mnemonicbox}
\subsection*{પ્રશ્ન 5(અ) અથવા [4
ગુણ]}\label{uxaaauxab0uxab6uxaa8-5uxa85-uxa85uxaa5uxab5-4-uxa97uxaa3}

\textbf{T1 કેરીયર TDM સિસ્ટમ સમજાવો.}

\begin{solutionbox}

{\def\LTcaptype{none} % do not increment counter
\begin{longtable}[]{@{}ll@{}}
\toprule\noalign{}
પેરામીટર & વર્ણન \\
\midrule\noalign{}
\endhead
\bottomrule\noalign{}
\endlastfoot
\textbf{વર્ણન} & નોર્થ અમેરિકન સ્ટાન્ડર્ડ ડિજિટલ ટ્રાન્સમિશન ફોર્મેટ \\
\textbf{ક્ષમતા} & 1.544 Mbps \\
\textbf{ચેનલ સ્ટ્રક્ચર} & 24 ટાઇમ સ્લોટ્સ (ચેનલ્સ) + 1 ફ્રેમિંગ બિટ \\
\textbf{વોઇસ ચેનલ્સ} & 24 વોઇસ ચેનલ્સ (દરેક 64 kbps) \\
\textbf{ફ્રેમ સ્ટ્રક્ચર} & 193 બિટ્સ પ્રતિ ફ્રેમ (24 \times 8 + 1) \\
\textbf{સિગ્નલિંગ} & રોબ્ડ બિટ સિગ્નલિંગ (લીસ્ટ સિગ્નિફિકન્ટ બિટ) \\
\end{longtable}
}

\textbf{આકૃતિ:}

\begin{verbatim}
One T1 Frame (193 bits)
 \_\_\_\_\_\_\_\_\_\_\_\_\_\_\_\_\_\_\_\_\_\_\_\_\_\_\_\_\_\_\_\_\_\_\_\_\_\_\_\_\_\_\_\_\_\_\_\_\_\_\_\_\_\_\_\_\_\_\_
|   |       |       |       |       |       |       |       |
| F | Ch 1  | Ch 2  | Ch 3  |  ...  | Ch 22 | Ch 23 | Ch 24 |
|\_\_\_|\_\_\_\_\_\_\_|\_\_\_\_\_\_\_|\_\_\_\_\_\_\_|\_\_\_\_\_\_\_|\_\_\_\_\_\_\_|\_\_\_\_\_\_\_|\_\_\_\_\_\_\_|

F: Framing bit
Each channel: 8 bits (1 byte)
\end{verbatim}

\end{solutionbox}
\begin{mnemonicbox}
``T1 24 ચેનલ્સ ટ્રાન્સમિટ કરે''

\end{mnemonicbox}
\subsection*{પ્રશ્ન 5(બ) અથવા [3
ગુણ]}\label{uxaaauxab0uxab6uxaa8-5uxaac-uxa85uxaa5uxab5-3-uxa97uxaa3}

\textbf{TDM અને FDM ની સરખામણી કરો.}

\begin{solutionbox}

{\def\LTcaptype{none} % do not increment counter
\begin{longtable}[]{@{}
  >{\raggedright\arraybackslash}p{(\linewidth - 4\tabcolsep) * \real{0.1325}}
  >{\raggedright\arraybackslash}p{(\linewidth - 4\tabcolsep) * \real{0.4096}}
  >{\raggedright\arraybackslash}p{(\linewidth - 4\tabcolsep) * \real{0.4578}}@{}}
\toprule\noalign{}
\begin{minipage}[b]{\linewidth}\raggedright
પેરામીટર
\end{minipage} & \begin{minipage}[b]{\linewidth}\raggedright
TDM (ટાઇમ ડિવિઝન મલ્ટિપ્લેક્સિંગ)
\end{minipage} & \begin{minipage}[b]{\linewidth}\raggedright
FDM (ફ્રિક્વન્સી ડિવિઝન મલ્ટિપ્લેક્સિંગ)
\end{minipage} \\
\midrule\noalign{}
\endhead
\bottomrule\noalign{}
\endlastfoot
\textbf{મૂળભૂત સિદ્ધાંત} & ચેનલને સમય દ્વારા વિભાજિત કરે & ચેનલને ફ્રિક્વન્સી દ્વારા
વિભાજિત કરે \\
\textbf{સિગ્નલ સેપરેશન} & ટાઇમ ડોમેઇનમાં & ફ્રિક્વન્સી ડોમેઇનમાં \\
\textbf{ગાર્ડ બેન્ડ્સ} & ટાઇમ ગાર્ડ બેન્ડ્સ & ફ્રિક્વન્સી ગાર્ડ બેન્ડ્સ \\
\textbf{અમલીકરણ} & ડિજિટલ ટેકનિક & એનાલોગ ટેકનિક (મૂળ રીતે) \\
\textbf{ક્રોસટોક} & ઓછી સંવેદનશીલ & વધુ સંવેદનશીલ \\
\textbf{સિન્ક્રોનાઈઝેશન} & જરૂરી & જરૂરી નથી \\
\end{longtable}
}

\textbf{આકૃતિ:}

\begin{verbatim}
TDM:                         FDM:
     Ch1  Ch2  Ch3  Ch1            \^{}
Time  |{-{-}|{-}{-}|{-}{-}|{-}{-}|{-}{-}             | Ch3}
      |{-{-}|{-}{-}|{-}{-}|{-}{-}|{-}{-}  Frequency  |{-}{-}{-}{-}{-}}
      |{-{-}|{-}{-}|{-}{-}|{-}{-}|{-}{-}             | Ch2}
                                   |{-{-}{-}{-}{-}}
                                   | Ch1
                                   |{-{-}{-}{-}{-}{-}{-}}
                                      Time
\end{verbatim}

\end{solutionbox}
\begin{mnemonicbox}
``સમય વિભાજિત મલ્ટિપ્લેક્સિંગ, ફ્રિક્વન્સી વિભાજિત
મલ્ટિપ્લેક્સિંગ''

\end{mnemonicbox}
\subsection*{પ્રશ્ન 5(ક) અથવા [7
ગુણ]}\label{uxaaauxab0uxab6uxaa8-5uxa95-uxa85uxaa5uxab5-7-uxa97uxaa3}

\textbf{માહિતી સુરક્ષાના સુરક્ષા ઘટકો સમજાવો.}

\begin{solutionbox}

\textbf{માહિતી સુરક્ષાનો CIA ત્રિકોણ:}

\begin{center}
\textbf{Mermaid Diagram (Code)}
\begin{verbatim}
{Shaded}
{Highlighting}[]
graph TD
    A[માહિતી સુરક્ષા] {-{-}{} B[Confidentiality {-} ગોપનીયતા]}
    A {-{-}{} C[Integrity {-} અખંડિતતા]}
    A {-{-}{} D[Availability {-} ઉપલબ્ધતા]}
    B {-{-}{} E[એન્ક્રિપ્શન, એક્સેસ કંટ્રોલ]}
    C {-{-}{} F[હેશિંગ, ડિજિટલ સિગ્નેચર]}
    D {-{-}{} G[રિડન્ડન્સી, ફોલ્ટ{-}ટોલરન્સ]}
{Highlighting}
{Shaded}
\end{verbatim}
\end{center}

{\def\LTcaptype{none} % do not increment counter
\begin{longtable}[]{@{}
  >{\raggedright\arraybackslash}p{(\linewidth - 4\tabcolsep) * \real{0.1667}}
  >{\raggedright\arraybackslash}p{(\linewidth - 4\tabcolsep) * \real{0.2333}}
  >{\raggedright\arraybackslash}p{(\linewidth - 4\tabcolsep) * \real{0.6000}}@{}}
\toprule\noalign{}
\begin{minipage}[b]{\linewidth}\raggedright
ઘટક
\end{minipage} & \begin{minipage}[b]{\linewidth}\raggedright
વર્ણન
\end{minipage} & \begin{minipage}[b]{\linewidth}\raggedright
અમલીકરણ પદ્ધતિઓ
\end{minipage} \\
\midrule\noalign{}
\endhead
\bottomrule\noalign{}
\endlastfoot
\textbf{ગોપનીયતા (Confidentiality)} & અનધિકૃત એક્સેસથી સુરક્ષા & - એન્ક્રિપ્શન-
એક્સેસ કંટ્રોલ- ઓથેન્ટિકેશન- સ્ટેગનોગ્રાફી \\
\textbf{અખંડિતતા (Integrity)} & ડેટા સચોટ અને અપરિવર્તિત છે તેની ખાતરી & -
હેશિંગ- ડિજિટલ સિગ્નેચર- વર્ઝન કંટ્રોલ- ચેકસમ \\
\textbf{ઉપલબ્ધતા (Availability)} & જરૂર પડે ત્યારે સિસ્ટમ્સ એક્સેસિબલ હોવાની
ખાતરી & - રિડન્ડન્સી- બેકઅપ- ડિઝાસ્ટર રિકવરી- ફોલ્ટ ટોલરન્સ \\
\textbf{ઓથેન્ટિકેશન (Authentication)} & ઓળખની ચકાસણી & - પાસવર્ડ-
બાયોમેટ્રિક્સ- સ્માર્ટ કાર્ડ્સ- મલ્ટિ-ફેક્ટર \\
\textbf{નોન-રીપ્યુડિએશન (Non-repudiation)} & ક્રિયાઓના ઇનકાર અટકાવવા & -
ડિજિટલ સિગ્નેચર- ઓડિટ લોગ- ટાઇમસ્ટેમ્પ \\
\end{longtable}
}

\textbf{સુરક્ષા ખતરાઓ:}

\begin{itemize}
\tightlist
\item
  માલવેર (વાયરસ, વોર્મ્સ, ટ્રોજન)
\item
  સોશિયલ એન્જિનિયરિંગ
\item
  ડિનાયલ ઓફ સર્વિસ (DoS)
\item
  મેન-ઇન-ધ-મિડલ એટેક્સ
\item
  ઇન્સાઇડર થ્રેટ્સ
\end{itemize}

\end{solutionbox}
\begin{mnemonicbox}
``CIA સર્વ નેટવર્ક ડેટા સુરક્ષિત રાખે''

\end{mnemonicbox}

\end{document}
