\documentclass[10pt,a4paper]{article}

% content/resources/templates/preamble.tex
\usepackage[margin=0.6in]{geometry}
\author{Milav Dabgar}
\usepackage{amsmath,amssymb,amsthm}
\usepackage{booktabs}
\usepackage{multirow}
\usepackage{xcolor}
\usepackage{tcolorbox}
\tcbuselibrary{breakable,skins}
\usepackage[colorlinks=true,linkcolor=blue]{hyperref}
\usepackage{titlesec}
\usepackage{enumitem}
\usepackage{tikz}
\usepackage{pgfplots}
\usepackage{circuitikz}
\usepackage[version=4]{mhchem}
\usepackage{longtable}
\usepackage{array}
\usepackage{float}
\usepackage{caption}
\usepackage{listings}

\lstset{
  basicstyle=\small\ttfamily,
  breaklines=true,
  breakatwhitespace=false,
  postbreak=\mbox{\textcolor{red}{$\hookrightarrow$}\space},
  float=false,
  numbers=left,
  numberstyle=\tiny\color{gray},
  numbersep=10pt,
  xleftmargin=2em,
  keywordstyle=\color{blue},
  commentstyle=\color{green!60!black},
  stringstyle=\color{purple},
  backgroundcolor=\color{gray!5},
  showstringspaces=false,
  tabsize=2,
  captionpos=b,
  keepspaces=true,
  columns=flexible
}

\pgfplotsset{compat=1.18}
\usetikzlibrary{shapes,arrows,positioning,calc,patterns,decorations.pathmorphing,decorations.markings,arrows.meta}

% Color scheme
\definecolor{headcolor}{RGB}{0,102,204}
\definecolor{keycolor}{RGB}{220,20,60}
\definecolor{solutioncolor}{RGB}{34,139,34}
\definecolor{mnemoniccolor}{RGB}{148,0,211}
\definecolor{codecolor}{RGB}{0,0,100}

% Spacing
\setlength{\parskip}{3pt}
\setlist[itemize]{nosep}
\setlist[enumerate]{nosep}

% Title formatting
\titleformat{\section}{\Large\bfseries\color{headcolor}}{\thesection}{1em}{}
\titleformat{\subsection}{\large\bfseries\color{headcolor}}{\thesubsection}{1em}{}

% Pandoc tightlist compatibility
\providecommand{\tightlist}{%
  \setlength{\itemsep}{0pt}\setlength{\parskip}{0pt}}

% Pandoc longtable compatibility
\newcounter{none}
\def\thenone{}


% content/resources/templates/gujarati-boxes.tex
\usepackage{fontspec}
\usepackage{polyglossia}

% Set Gujarati as main language (document is primarily in Gujarati)
% Note: gloss-gujarati.ldf doesn't exist in polyglossia, but it will use hyphenation patterns
\setdefaultlanguage{gujarati}
\setotherlanguage{english}

% Configure Gujarati font properly
% Use Language=Default to prevent polyglossia from trying to add language-specific features
% that don't exist for Gujarati, which causes "empty feature" warnings
\newfontfamily\gujaratifont[Script=Gujarati,AutoFakeBold=2.5,AutoFakeSlant=0.3]{Noto Sans Gujarati}
\setmainfont[Script=Gujarati,AutoFakeBold=2.5,AutoFakeSlant=0.3]{Noto Sans Gujarati}
% Use Noto Sans Gujarati for monospace to support Gujarati in text
\setmonofont[Scale=0.9]{Noto Sans Gujarati}

% Configure English to use the same font
\newfontfamily\englishfont[Script=Gujarati,AutoFakeBold=2.5,AutoFakeSlant=0.3]{Noto Sans Gujarati}

% Translations for polyglossia
\gappto\captionsgujarati{
  \renewcommand{\tablename}{કોષ્ટક}
  \renewcommand{\figurename}{આકૃતિ}
}

% Helper for TikZ nodes to ensure Gujarati font
\newcommand{\gu}[1]{{\gujaratifont #1}}

% Custom environments
\newtcolorbox{solutionbox}{
    breakable,
    enhanced,
    colback=solutioncolor!5!white,
    colframe=solutioncolor!75!black,
    fonttitle=\bfseries,
    title=જવાબ
}

\newtcolorbox{solutionboxnobreak}{
 colback=solutioncolor!5!white,
 colframe=solutioncolor!75!black,
 fonttitle=\bfseries,
 title=જવાબ
}

\newtcolorbox{keyformula}{
 breakable,
 enhanced,
 colback=keycolor!5!white,
 colframe=keycolor!75!black,
 fonttitle=\bfseries,
 title=રાસાયણિક સમીકરણ/સૂત્ર
}

\newtcolorbox{mnemonicbox}{
 breakable,
 enhanced,
 colback=mnemoniccolor!5!white,
 colframe=mnemoniccolor!75!black,
 fonttitle=\bfseries,
 title=મેમરી ટ્રીક
}


\begin{document}

\begin{center}
{\Huge\bfseries\color{headcolor} Subject Name (Gujarati)}\\[5pt]
{\LARGE 4341106 -- Summer 2023}\\[3pt]
{\large Semester 1 Study Material}\\[3pt]
{\normalsize\textit{Detailed Solutions and Explanations}}
\end{center}

\vspace{10pt}

\subsection*{પ્રશ્ન 1(અ) [3
ગુણ]}\label{uxaaauxab0uxab6uxaa8-1uxa85-3-uxa97uxaa3}

\textbf{ઇલેક્ટ્રોમેગ્નેટિક તરંગોના કોઈપણ ત્રણ ગુણધર્મો લખો}

\begin{solutionbox}

{\def\LTcaptype{none} % do not increment counter
\begin{longtable}[]{@{}l@{}}
\toprule\noalign{}
ઇલેક્ટ્રોમેગ્નેટિક તરંગોના ગુણધર્મો \\
\midrule\noalign{}
\endhead
\bottomrule\noalign{}
\endlastfoot
1. EM તરંગો નિર્વાત અથવા પદાર્થ માધ્યમમાં પ્રવાસ કરી શકે છે \\
2. EM તરંગો ફ્રી સ્પેસમાં પ્રકાશની ગતિએ પ્રવાસ કરે છે (3\times10^{8} m/s) \\
3. EM તરંગો દોલનશીલ વીજળી અને ચુંબકીય ક્ષેત્રો સાથે આડી લાક્ષણિકતાઓ દર્શાવે છે \\
\end{longtable}
}

\end{solutionbox}
\begin{mnemonicbox}
``VTS'' - Vacuum travel, Transverse nature, Speed of
light

\end{mnemonicbox}
\subsection*{પ્રશ્ન 1(બ) [4
ગુણ]}\label{uxaaauxab0uxab6uxaa8-1uxaac-4-uxa97uxaa3}

\textbf{વ્યાખ્યા લખો: (1) રેડિયેશન રેઝિસ્ટન્સ (2) ડાયરેક્ટિવિટી (3) ગેઈન}

\begin{solutionbox}

{\def\LTcaptype{none} % do not increment counter
\begin{longtable}[]{@{}
  >{\raggedright\arraybackslash}p{(\linewidth - 2\tabcolsep) * \real{0.3333}}
  >{\raggedright\arraybackslash}p{(\linewidth - 2\tabcolsep) * \real{0.6667}}@{}}
\toprule\noalign{}
\begin{minipage}[b]{\linewidth}\raggedright
શબ્દ
\end{minipage} & \begin{minipage}[b]{\linewidth}\raggedright
વ્યાખ્યા
\end{minipage} \\
\midrule\noalign{}
\endhead
\bottomrule\noalign{}
\endlastfoot
\textbf{રેડિયેશન રેઝિસ્ટન્સ} & તે સમકક્ષ અવરોધ છે જે એન્ટેના ઇનપુટ કરંટની બરાબર હોય
ત્યારે એન્ટેના દ્વારા વિકિરણ કરવામાં આવતી ઊર્જા જેટલી જ ઊર્જા વેડફે છે \\
\textbf{ડાયરેક્ટિવિટી} & ચોક્કસ દિશામાં મહત્તમ વિકિરણ તીવ્રતા અને બધી દિશાઓમાં
સરેરાશ વિકિરણ તીવ્રતાનો ગુણોત્તર \\
\textbf{ગેઈન} & નિર્દિષ્ટ દિશામાં રેડિયો તરંગોમાં ઇનપુટ પાવરને કેટલી કાર્યક્ષમતાથી
રૂપાંતરિત કરે છે તે માપતા ડાયરેક્ટિવિટી અને રેડિયેશન એફિશિયન્સીનો ગુણાકાર \\
\end{longtable}
}

\end{solutionbox}
\begin{mnemonicbox}
``RDG'' - Resistance dissipates power, Direction
concentration, Gain includes efficiency

\end{mnemonicbox}
\subsection*{પ્રશ્ન 1(ક) [7
ગુણ]}\label{uxaaauxab0uxab6uxaa8-1uxa95-7-uxa97uxaa3}

\textbf{ઇલેક્ટ્રોમેગ્નેટિક તરંગોના નિર્માણની ભૌતિક ખ્યાલ સુઘડ રેખાકૃતિ સાથે સમજાવો}

\begin{solutionbox}

ઇલેક્ટ્રોમેગ્નેટિક તરંગો ત્યારે ઉત્પન્ન થાય છે જ્યારે ઇલેક્ટ્રિક ચાર્જ પ્રવેગ કરે છે અથવા દોલન
કરે છે, જે અવકાશમાં પ્રસરિત થતા યુગ્મિત દોલનશીલ ઇલેક્ટ્રિક અને ચુંબકીય ક્ષેત્રો બનાવે છે.

\begin{center}
\textbf{Mermaid Diagram (Code)}
\begin{verbatim}
{Shaded}
{Highlighting}[]
graph LR
    A[Electric Current Flow] {-{-}{}|Oscillation| B[Oscillating Electric Field]}
    B {-{-}{}|Induces| C[Oscillating Magnetic Field]}
    C {-{-}{}|Induces| D[Oscillating Electric Field]}
    D {-{-}{} E[Self{-}sustaining wave propagation]}
{Highlighting}
{Shaded}
\end{verbatim}
\end{center}

\textbf{ડાયગ્રામ: ડાયપોલ એન્ટેના EM તરંગ ઉત્પાદન}

\begin{verbatim}
                 +
                 |
                 |
    Oscillator   |      Electric field lines
    +{-{-}{-}{-}{-}{-}{-}{-}{-}+  |      ∽∽∽∽∽∽∽∽∽∽∽∽∽∽∽}
    |         |  |
    |    {    |{-}{-}+      Magnetic field lines}
    |         |  |      ⊙⊙⊙⊙⊙⊙⊙⊙⊙⊙⊙⊙⊙⊙⊙
    +{-{-}{-}{-}{-}{-}{-}{-}{-}+  |}
                 |
                 |
                 {-}
\end{verbatim}

\begin{itemize}
\tightlist
\item
  \textbf{મૂળભૂત ખ્યાલ}: જ્યારે AC કરંટ એન્ટેનામાં વહે છે, ત્યારે ઇલેક્ટ્રોન ઉપર અને નીચે
  પ્રવેગ કરે છે
\item
  \textbf{ઇલેક્ટ્રિક ફિલ્ડ}: એન્ટેનામાં ચાર્જ વિભાજનથી બને છે
\item
  \textbf{મેગ્નેટિક ફિલ્ડ}: કરંટ પ્રવાહથી ઉત્પન્ન થાય છે, ઇલેક્ટ્રિક ફિલ્ડને લંબરૂપે
\item
  \textbf{પ્રસરણ}: ફિલ્ડ એન્ટેનાથી અલગ થઈને પ્રકાશની ગતિએ બહારની તરફ પ્રસરે છે
\item
  \textbf{સ્વ-ટકાઉ}: તરંગ પ્રવાસ કરતાં દરેક ફિલ્ડ ઘટક અન્ય ઘટકને પુનર્જીવિત કરે છે
\end{itemize}

\end{solutionbox}
\begin{mnemonicbox}
``COMAP'' - Current Oscillations Make Alternating
Propagations

\end{mnemonicbox}
\subsection*{પ્રશ્ન 1(ક) OR [7
ગુણ]}\label{uxaaauxab0uxab6uxaa8-1uxa95-or-7-uxa97uxaa3}

\textbf{435 MHZ આવૃત્તિ માટે 4 એલિમેન્ટ વાળુ યાગી ઉદા એન્ટેના ની ડિઝાઇન બનાવો.}

\begin{solutionbox}

435 MHz માટે 4-એલિમેન્ટ યાગી-ઉદા એન્ટેના માટે:

{\def\LTcaptype{none} % do not increment counter
\begin{longtable}[]{@{}llll@{}}
\toprule\noalign{}
એલિમેન્ટ & લંબાઈ ફોર્મ્યુલા & અંતર ફોર્મ્યુલા & ગણતરી કરેલ મૂલ્ય \\
\midrule\noalign{}
\endhead
\bottomrule\noalign{}
\endlastfoot
\textbf{રિફ્લેક્ટર} & 0.5λ \times 1.05 & - & 36.2 cm \\
\textbf{ડ્રાઇવન એલિમેન્ટ} & 0.5λ & - & 34.5 cm \\
\textbf{ડાયરેક્ટર 1} & 0.45λ & ડ્રાઇવનથી 0.2λ & 31.0 cm, 13.8 cm અંતર \\
\textbf{ડાયરેક્ટર 2} & 0.43λ & ડાયરેક્ટર 1થી 0.25λ & 29.6 cm, 17.2 cm અંતર \\
\end{longtable}
}

\textbf{વપરાયેલા સૂત્રો}:

\begin{itemize}
\tightlist
\item
તરંગલંબાઈ:

λ = c/f = 3\times10^{8}/435\times10^{6} = 0.69 મીટર

\item
હાફ-વેવ ડાયપોલ:

L = 0.5λ = 34.5 cm

\item
  એલિમેન્ટ અંતર: S = 0.15λ થી 0.25λ
\end{itemize}

\begin{center}
\textbf{Mermaid Diagram (Code)}
\begin{verbatim}
{Shaded}
{Highlighting}[]
graph LR
    A[Reflector: 36.2cm] {-{-}{-} B[Driven Element: 34.5cm]}
    B {-{-}{-} C[Director 1: 31.0cm]}
    C {-{-}{-} D[Director 2: 29.6cm]}

    style A fill:\#f9f,stroke:\#333,stroke{-width:2px}
    style B fill:\#bbf,stroke:\#333,stroke{-width:2px}
    style C fill:\#fbb,stroke:\#333,stroke{-width:2px}
    style D fill:\#fbb,stroke:\#333,stroke{-width:2px}
{Highlighting}
{Shaded}
\end{verbatim}
\end{center}

\end{solutionbox}
\begin{mnemonicbox}
``RDDS'' - Reflector Driven Directors Shrink

\end{mnemonicbox}
\subsection*{પ્રશ્ન 2(અ) [3
ગુણ]}\label{uxaaauxab0uxab6uxaa8-2uxa85-3-uxa97uxaa3}

\textbf{લુપ એન્ટેના આકૃતિની મદદથી સમજાવો}

\begin{solutionbox}

લુપ એન્ટેના એક વાહક ને લુપ આકારમાં બનાવીને વિકિરણ ઘટક બનાવવામાં આવે છે.

\begin{verbatim}
    ┌───────────┐
    │           │
    │           │
    │           │
    │           │
    │     ↺     │ Current flow
    │           │
    │           │
    │           │
    └─────┬─────┘
          │
          │ Feed point
       ───┴───
\end{verbatim}

\begin{itemize}
\tightlist
\item
  \textbf{નાના લુપ}: પરિઘ \textless{} λ/10, રેડિએશન પેટર્ન મેગ્નેટિક ડાયપોલ જેવા
\item
  \textbf{મોટા લુપ}: પરિઘ \approx તરંગલંબાઈ, દ્વિદિશાત્મક રેડિએશન પેટર્ન
\item
  \textbf{ઉપયોગો}: દિશા શોધવી, AM રેડિયો રિસેપ્શન, RFID ટેગ્સ
\end{itemize}

\end{solutionbox}
\begin{mnemonicbox}
``SLC'' - Size affects Loop Characteristics

\end{mnemonicbox}
\subsection*{પ્રશ્ન 2(બ) [4
ગુણ]}\label{uxaaauxab0uxab6uxaa8-2uxaac-4-uxa97uxaa3}

\textbf{નોન રેઝોનેંટ વાયર એન્ટેના સમજાવો}

\begin{solutionbox}

{\def\LTcaptype{none} % do not increment counter
\begin{longtable}[]{@{}
  >{\raggedright\arraybackslash}p{(\linewidth - 2\tabcolsep) * \real{0.5200}}
  >{\raggedright\arraybackslash}p{(\linewidth - 2\tabcolsep) * \real{0.4800}}@{}}
\toprule\noalign{}
\begin{minipage}[b]{\linewidth}\raggedright
લક્ષણ
\end{minipage} & \begin{minipage}[b]{\linewidth}\raggedright
વર્ણન
\end{minipage} \\
\midrule\noalign{}
\endhead
\bottomrule\noalign{}
\endlastfoot
\textbf{વ્યાખ્યા} & એવા આવૃત્તિઓ પર કાર્ય કરતા એન્ટેના જ્યાં તેની ભૌતિક લંબાઈ
અર્ધ-તરંગલંબાઈના ગુણાંક નથી \\
\textbf{ઇમ્પીડન્સ} & જટિલ, રેઝિસ્ટિવ અને રિએક્ટિવ બંને ઘટકો સાથે \\
\textbf{સ્ટેન્ડિંગ વેવ્સ} & એન્ટેનાની લંબાઈ પર હાજર \\
\textbf{ઉદાહરણ} & રોમ્બિક એન્ટેના, અંતમાં અવરોધથી ટર્મિનેટ કરેલ \\
\textbf{ફાયદો} & વાઇડબેન્ડ ઓપરેશન, મલ્ટીપલ ફ્રીક્વન્સી માટે યોગ્ય \\
\end{longtable}
}

\end{solutionbox}
\begin{mnemonicbox}
``NITRO'' - Non-resonance Incurs Termination for
Resistance and Operation

\end{mnemonicbox}
\subsection*{પ્રશ્ન 2(ક) [7
ગુણ]}\label{uxaaauxab0uxab6uxaa8-2uxa95-7-uxa97uxaa3}

\textbf{હાફ વેવ ડાયપોલ એન્ટેના નું રેડિયેશન રેઝીસ્ટંસ કેટલું હોય છે? λ/2, λ અને λ/4
લમ્બાઇ ના એન્ટેના રેડિયેશન ની પેટર્ન દોરો}

\begin{solutionbox}

હાફ-વેવ ડાયપોલનું રેડિયેશન રેઝીસ્ટંસ આશરે 73 ઓહ્મ હોય છે.

\textbf{રેડિયેશન પેટર્ન:}

\begin{verbatim}
   λ/2 Dipole             λ Dipole              λ/4 Dipole
   
      0^                     0^                     0^
      |                      |                      |
270^{-{-}+{-}{-}90^   vs.     270^-{-}{-}+{-}{-}{-}90^   vs.    270^-{-}{-}+{-}{-}{-}90^}
      |                      |                      |
     180^                   180^                   180^
    (Figure{-8)         (Multiple lobes)         (Broad pattern)}
\end{verbatim}

{\def\LTcaptype{none} % do not increment counter
\begin{longtable}[]{@{}
  >{\raggedright\arraybackslash}p{(\linewidth - 2\tabcolsep) * \real{0.3750}}
  >{\raggedright\arraybackslash}p{(\linewidth - 2\tabcolsep) * \real{0.6250}}@{}}
\toprule\noalign{}
\begin{minipage}[b]{\linewidth}\raggedright
ડાયપોલ લંબાઈ
\end{minipage} & \begin{minipage}[b]{\linewidth}\raggedright
પેટર્ન લક્ષણો
\end{minipage} \\
\midrule\noalign{}
\endhead
\bottomrule\noalign{}
\endlastfoot
\textbf{λ/2 ડાયપોલ} & ફિગર-8 પેટર્ન; એન્ટેના અક્ષને લંબરૂપે મહત્તમ વિકિરણ; HPBW =
78^\circ \\
\textbf{λ ડાયપોલ} & મલ્ટી-લોબ્ડ પેટર્ન; એન્ટેના અક્ષ પર કોણે ચાર મુખ્ય લોબ \\
\textbf{λ/4 ડાયપોલ} & λ/2 કરતાં વધુ વિશાળ પેટર્ન; સમતુલ્ય ડાયપોલ પૂર્ણ કરવા માટે
ગ્રાઉન્ડ પ્લેનની જરૂર \\
\end{longtable}
}

\end{solutionbox}
\begin{mnemonicbox}
``SHORT'' - Smaller Half-dipole Offers
Rounded-Transmissions

\end{mnemonicbox}
\subsection*{પ્રશ્ન 2(અ) OR [3
ગુણ]}\label{uxaaauxab0uxab6uxaa8-2uxa85-or-3-uxa97uxaa3}

\textbf{ફોલ્ડેડ ડાઇપોલ એન્ટેના આકૃતિની મદદથી સમજાવો}

\begin{solutionbox}

ફોલ્ડેડ ડાયપોલ એ હાફ-વેવ ડાયપોલનો એક પ્રકાર છે જેમાં છેડાઓને પાછા વાળીને લૂપ બનાવવા
માટે જોડવામાં આવે છે.

\begin{verbatim}
    ┌───────────────────────────┐
    │                           │
    │                           │
    │                           │
    └───────────┬───────────────┘
                │
                │ Feed point
             ───┴───
\end{verbatim}

\begin{itemize}
\tightlist
\item
  \textbf{ઇનપુટ ઇમ્પીડન્સ}: આશરે 300 ઓહ્મ (સામાન્ય ડાયપોલના 4 ગણા)
\item
  \textbf{બેન્ડવિડ્થ}: સામાન્ય ડાયપોલ કરતાં વધારે
\item
  \textbf{ઉપયોગો}: TV રિસેપ્શન, FM રેડિયો, બેલેન્સ્ડ ટ્રાન્સમિશન લાઇન્સ
\end{itemize}

\end{solutionbox}
\begin{mnemonicbox}
``FIB'' - Folded Increases Bandwidth

\end{mnemonicbox}
\subsection*{પ્રશ્ન 2(બ) OR [4
ગુણ]}\label{uxaaauxab0uxab6uxaa8-2uxaac-or-4-uxa97uxaa3}

\textbf{રોમ્બિક એન્ટેના આકૃતિની મદદથી સમજાવો}

\begin{solutionbox}

રોમ્બિક એન્ટેના એક રોમ્બસ અથવા હીરા આકારમાં ગોઠવાયેલા ચાર તારોનો બનેલો હોય છે.

\begin{verbatim}
              Direction of 
                radiation
                   ↓
            A ◄─────────► B
            /             {}
           /               {}
          /                 {}
Feed ────┐                   ┌──── Termination
          {                 /}
           {               /}
            {             /}
            D ◄─────────► C
\end{verbatim}

{\def\LTcaptype{none} % do not increment counter
\begin{longtable}[]{@{}ll@{}}
\toprule\noalign{}
લક્ષણ & વર્ણન \\
\midrule\noalign{}
\endhead
\bottomrule\noalign{}
\endlastfoot
\textbf{આકાર} & ડાયમંડ/રોમ્બસ, દૂરના છેડે ટર્મિનેટિંગ રેઝિસ્ટર સાથે \\
\textbf{ઓપરેશન} & નોન-રેઝોનન્ટ ટ્રાવેલિંગ-વેવ એન્ટેના \\
\textbf{ડાયરેક્ટિવિટી} & ઉચ્ચ ગેઇન, યુનિડાયરેક્શનલ પેટર્ન \\
\textbf{બેન્ડવિડ્થ} & ખૂબ વિશાળ આવૃત્તિ શ્રેણી \\
\textbf{ઉપયોગો} & HF કમ્યુનિકેશન્સ, પોઇન્ટ-ટુ-પોઇન્ટ લિંક્સ \\
\end{longtable}
}

\end{solutionbox}
\begin{mnemonicbox}
``TREND'' - Terminated Rhombic Enables Numerous
Directions

\end{mnemonicbox}
\subsection*{પ્રશ્ન 2(ક) OR [7
ગુણ]}\label{uxaaauxab0uxab6uxaa8-2uxa95-or-7-uxa97uxaa3}

\textbf{આકૃતિની મદદથી એન્ડ ફાયર અને બ્રોડ સાઇડ એન્ટેના નો તફાવત સમજાવો}

\begin{solutionbox}

{\def\LTcaptype{none} % do not increment counter
\begin{longtable}[]{@{}lll@{}}
\toprule\noalign{}
પેરામીટર & બ્રોડસાઇડ એરે & એન્ડ ફાયર એરે \\
\midrule\noalign{}
\endhead
\bottomrule\noalign{}
\endlastfoot
\textbf{મહત્તમ વિકિરણની દિશા} & એરે અક્ષને લંબરૂપે & એરે અક્ષ સાથે \\
\textbf{એલિમેન્ટ ફેઝિંગ} & સમાન ફેઝ (0^\circ) & પ્રગતિશીલ ફેઝ શિફ્ટ \\
\textbf{એલિમેન્ટ અંતર} & સામાન્ય રીતે λ/2 & સામાન્ય રીતે λ/4 \\
\textbf{રેડિયેશન પેટર્ન} & ફેન-આકારનો બીમ & પેન્સિલ-આકારનો બીમ \\
\textbf{ઉપયોગો} & બ્રોડકાસ્ટિંગ, બેઝ સ્ટેશન્સ & પોઇન્ટ-ટુ-પોઇન્ટ લિંક્સ \\
\end{longtable}
}

\textbf{ડાયાગ્રામ સરખામણી:}

\begin{verbatim}
   Broadside Array                     End fire Array
   
   ─o─────o─────o─────o─         ─o─────o─────o─────o─
        Array Axis                     Array Axis
         
         ↑ ↑ ↑ ↑                 
      Main radiation                Main radiation
        direction                     direction
\end{verbatim}

\end{solutionbox}
\begin{mnemonicbox}
``PAPER'' - Perpendicular And Parallel Emission
Respectively

\end{mnemonicbox}
\subsection*{પ્રશ્ન 3(અ) [3
ગુણ]}\label{uxaaauxab0uxab6uxaa8-3uxa85-3-uxa97uxaa3}

\textbf{આકૃતિની મદદથી ઇન્વર્ટેડ વી એન્ટેના સમજાવો}

\begin{solutionbox}

ઇન્વર્ટેડ V એન્ટેના એ ડાયપોલ છે જેની બાહુઓ નીચેની તરફ વળેલી હોય છે, ઉલટા ``V'' જેવી
દેખાય છે.

\begin{verbatim}
                   ▲
                   │ Support
                   │
                   │
                  /│{}
                 / │ {}
                /  │  {}
               /   │   {}
              /    │    {}
             /     │     {}
            ◄─────┐│┌─────►
                  ││
                Feed point
\end{verbatim}

\begin{itemize}
\tightlist
\item
  \textbf{ખૂણો}: બાહુઓ સામાન્ય રીતે 90^\circ-120^\circ ખૂણો બનાવે છે
\item
  \textbf{ઇમ્પીડન્સ}: આશરે 50 ઓહ્મ, આડા ડાયપોલ કરતાં ઓછું
\item
  \textbf{પેટર્ન}: સર્વવ્યાપી, આડા ડાયપોલ કરતાં થોડું વધુ વિશાળ
\item
  \textbf{ઉપયોગો}: એમેચ્યોર રેડિયો, શોર્ટવેવ કમ્યુનિકેશન્સ
\end{itemize}

\end{solutionbox}
\begin{mnemonicbox}
``AVS'' - Angle Varies Signal

\end{mnemonicbox}
\subsection*{પ્રશ્ન 3(બ) [4
ગુણ]}\label{uxaaauxab0uxab6uxaa8-3uxaac-4-uxa97uxaa3}

\textbf{આકૃતિની મદદથી પેરાબોલિક રીફ્લેક્ટર એન્ટેના સમજાવો}

\begin{solutionbox}

\begin{verbatim}
           │        ╱│╲        ┌───────►
           │      ╱  │  ╲      │
           │    ╱    │    ╲    │
           │  ╱      │      ╲  │
    ───────┼╱        ↓        ╲┼─┘
           │                   │
     Feed ─┤                   │
           │                   │
    ───────┼╲                 ╱┼───────►
           │  ╲             ╱  │
           │    ╲         ╱    │
           │      ╲     ╱      │
           │        ╲│╱        │
                   Focus
\end{verbatim}

{\def\LTcaptype{none} % do not increment counter
\begin{longtable}[]{@{}
  >{\raggedright\arraybackslash}p{(\linewidth - 2\tabcolsep) * \real{0.5238}}
  >{\raggedright\arraybackslash}p{(\linewidth - 2\tabcolsep) * \real{0.4762}}@{}}
\toprule\noalign{}
\begin{minipage}[b]{\linewidth}\raggedright
ઘટક
\end{minipage} & \begin{minipage}[b]{\linewidth}\raggedright
કાર્ય
\end{minipage} \\
\midrule\noalign{}
\endhead
\bottomrule\noalign{}
\endlastfoot
\textbf{પેરાબોલિક રિફ્લેક્ટર} & આવતા સિગ્નલ્સને એકત્રિત કરે છે અને કેન્દ્રિત કરે છે અથવા
ટ્રાન્સમિટ થયેલા સિગ્નલોને નિર્દેશિત કરે છે \\
\textbf{ફીડ એલિમેન્ટ} & પેરાબોલાના ફોકલ પોઇન્ટ પર સ્થિત, સિગ્નલ્સને
એકત્રિત/પ્રસારિત કરે છે \\
\textbf{ફોકલ લેન્થ} & વર્ટેક્સથી ફોકસ સુધીનું અંતર, બીમની લાક્ષણિકતાઓ નક્કી કરે છે \\
\textbf{ઉપયોગો} & સેટેલાઇટ કમ્યુનિકેશન, રડાર, રેડિયો એસ્ટ્રોનોમી, માઇક્રોવેવ
લિંક્સ \\
\end{longtable}
}

\end{solutionbox}
\begin{mnemonicbox}
``FOLD'' - Focus Of Large Dish

\end{mnemonicbox}
\subsection*{પ્રશ્ન 3(ક) [7
ગુણ]}\label{uxaaauxab0uxab6uxaa8-3uxa95-7-uxa97uxaa3}

\textbf{HF, VHF અને UHF માટેની આવૃત્તિની રેન્જ લખો. માઇક્રોસ્ટ્રીપ એન્ટેના વિશે ટૂંક
નોંધ લખો.}

\begin{solutionbox}

{\def\LTcaptype{none} % do not increment counter
\begin{longtable}[]{@{}ll@{}}
\toprule\noalign{}
ફ્રિક્વન્સી બેન્ડ & રેન્જ \\
\midrule\noalign{}
\endhead
\bottomrule\noalign{}
\endlastfoot
\textbf{HF (હાઇ ફ્રિક્વન્સી)} & 3 MHz - 30 MHz \\
\textbf{VHF (વેરી હાઇ ફ્રિક્વન્સી)} & 30 MHz - 300 MHz \\
\textbf{UHF (અલ્ટ્રા હાઇ ફ્રિક્વન્સી)} & 300 MHz - 3 GHz \\
\end{longtable}
}

\textbf{માઇક્રોસ્ટ્રીપ એન્ટેના:}

\begin{verbatim}
         ┌─────────────────────┐
         │  Radiating Patch    │
         └─────────────────────┘
    ┌───────────────────────────────┐ ─┐
    │      Dielectric Substrate     │  │h
    └───────────────────────────────┘ ─┘
    ┌───────────────────────────────┐
    │       Ground Plane            │
    └───────────────────────────────┘
\end{verbatim}

\begin{itemize}
\tightlist
\item
  \textbf{રચના}: ડાયલેક્ટ્રિક સબસ્ટ્રેટ પર ગ્રાઉન્ડ પ્લેન સાથે કન્ડક્ટિવ પેચ
\item
  \textbf{ફીડિંગ મેથડ્સ}: માઇક્રોસ્ટ્રીપ લાઇન, કોએક્સિયલ પ્રોબ, એપર્ચર-કપલ્ડ
\item
  \textbf{ફાયદા}: લો પ્રોફાઇલ, હળવા વજનના, સરળ ફેબ્રિકેશન, PCB સાથે સુસંગત
\item
  \textbf{મર્યાદાઓ}: સાંકડી બેન્ડવિડ્થ, ઓછો ગેઇન, ઓછી પાવર હેન્ડલિંગ
\item
  \textbf{ઉપયોગો}: મોબાઇલ ડિવાઇસ, RFID, GPS, સેટેલાઇટ કમ્યુનિકેશન્સ
\end{itemize}

\end{solutionbox}
\begin{mnemonicbox}
``PATCH'' - Planar Antenna That's Cheaply Handled

\end{mnemonicbox}
\subsection*{પ્રશ્ન 3(અ) OR [3
ગુણ]}\label{uxaaauxab0uxab6uxaa8-3uxa85-or-3-uxa97uxaa3}

\textbf{``LINE OF SIGHT'' શબ્દ માટે મોર્સ કોડ લખો}

\begin{solutionbox}

{\def\LTcaptype{none} % do not increment counter
\begin{longtable}[]{@{}ll@{}}
\toprule\noalign{}
અક્ષર & મોર્સ કોડ \\
\midrule\noalign{}
\endhead
\bottomrule\noalign{}
\endlastfoot
L & .-.. \\
I & .. \\
N & -. \\
E & . \\
(સ્પેસ) & / \\
O & --- \\
F & ..-. \\
(સ્પેસ) & / \\
S & \ldots{} \\
I & .. \\
G & --. \\
H & \ldots. \\
T & - \\
\end{longtable}
}

``LINE OF SIGHT'' મોર્સ કોડમાં: .-.. .. -. . / --- ..-. / \ldots{} .. --.
\ldots. -

\end{solutionbox}
\begin{mnemonicbox}
``Listen In Now, Every Other Frequency Supports
Immediate Global Heightened Transmission''

\end{mnemonicbox}
\subsection*{પ્રશ્ન 3(બ) OR [4
ગુણ]}\label{uxaaauxab0uxab6uxaa8-3uxaac-or-4-uxa97uxaa3}

\textbf{આકૃતિની મદદથી ટર્નસ્ટાઇલ અને સુપર ટર્નસ્ટાઇલ એન્ટેના સમજાવો}

\begin{solutionbox}

\textbf{ટર્નસ્ટાઇલ એન્ટેના:}

\begin{verbatim}
         ───┬───
            │
            │
    ────────┼────────
            │
            │
         ───┴───
\end{verbatim}

\textbf{સુપર ટર્નસ્ટાઇલ એન્ટેના:}

\begin{verbatim}
       ┌───┐       ┌───┐
       │   │       │   │ 
       │   │       │   │
       └───┘       └───┘
       
       ┌───┐       ┌───┐
       │   │       │   │
       │   │       │   │
       └───┘       └───┘
\end{verbatim}

{\def\LTcaptype{none} % do not increment counter
\begin{longtable}[]{@{}ll@{}}
\toprule\noalign{}
પ્રકાર & લક્ષણો \\
\midrule\noalign{}
\endhead
\bottomrule\noalign{}
\endlastfoot
\textbf{ટર્નસ્ટાઇલ} & કાટખૂણે બે આડા ડાયપોલ, 90^\circ ફેઝ શિફ્ટ સાથે ફીડ કરેલ \\
\textbf{સુપર ટર્નસ્ટાઇલ} & લંબચોરસ લૂપ્સ બનાવતા મલ્ટીપલ એલિમેન્ટ્સ સાથે સુધારો \\
\textbf{પેટર્ન} & આડા પ્લેનમાં સર્વવ્યાપી, ઊભા પ્લેનમાં ફિગર-8 \\
\textbf{પોલરાઇઝેશન} & આડું અથવા સર્ક્યુલર પોલરાઇઝેશન \\
\textbf{ઉપયોગો} & TV બ્રોડકાસ્ટિંગ, FM બ્રોડકાસ્ટિંગ, સેટેલાઇટ કમ્યુનિકેશન્સ \\
\end{longtable}
}

\end{solutionbox}
\begin{mnemonicbox}
``TOPS'' - Turnstile Offers Perpendicular Symmetry

\end{mnemonicbox}
\subsection*{પ્રશ્ન 3(ક) OR [7
ગુણ]}\label{uxaaauxab0uxab6uxaa8-3uxa95-or-7-uxa97uxaa3}

\textbf{પોલરાઇઝેશન શું છે? આકૃતિની મદદથી હેલીકલ એન્ટેના સમજાવો}

\begin{solutionbox}

\textbf{પોલરાઇઝેશન} એ અવકાશમાં પ્રસરણ કરતી વખતે ઇલેક્ટ્રોમેગ્નેટિક તરંગના ઇલેક્ટ્રિક
ફિલ્ડ વેક્ટરનું અભિગમન છે.

\textbf{હેલીકલ એન્ટેના:}

\begin{verbatim}
          ┌─┐     ┌─┐
         /   {   /   }
        │     { /     │}
        │      X      │
        │     / {     │}
         {   /      /}
          └─┘     └─┘
               │
               │
           ────┴────
          Ground plane
\end{verbatim}

{\def\LTcaptype{none} % do not increment counter
\begin{longtable}[]{@{}ll@{}}
\toprule\noalign{}
પેરામીટર & વર્ણન \\
\midrule\noalign{}
\endhead
\bottomrule\noalign{}
\endlastfoot
\textbf{રચના} & ગ્રાઉન્ડ પ્લેન પર હેલિકલ આકારમાં વાયર વીંટાળેલો \\
\textbf{વ્યાસ} & સામાન્ય રીતે λ/π \\
\textbf{પિચ} & વીંટાળા વચ્ચેનું અંતર, સામાન્ય રીતે λ/4 \\
\textbf{વીંટાળા} & ગેઇન જરૂરિયાતો આધારિત 3-10 વીંટાળા \\
\textbf{મોડ્સ} & નોર્મલ મોડ (બ્રોડસાઇડ) અથવા એક્સિયલ મોડ (એન્ડ-ફાયર) \\
\textbf{પોલરાઇઝેશન} & એક્સિયલ મોડમાં સર્ક્યુલર પોલરાઇઝેશન \\
\textbf{ઉપયોગો} & સેટેલાઇટ કમ્યુનિકેશન્સ, સ્પેસ ટેલિમેટ્રી, ટ્રેકિંગ \\
\end{longtable}
}

\end{solutionbox}
\begin{mnemonicbox}
``HASP'' - Helical Antenna Supports Polarization

\end{mnemonicbox}
\subsection*{પ્રશ્ન 4(અ) [3
ગુણ]}\label{uxaaauxab0uxab6uxaa8-4uxa85-3-uxa97uxaa3}

\textbf{ટ્રોપોસ્ફેરિક સ્કેટર્ડ પ્રોપોગેશન સમજાવો}

\begin{solutionbox}

{\def\LTcaptype{none} % do not increment counter
\begin{longtable}[]{@{}
  >{\raggedright\arraybackslash}p{(\linewidth - 2\tabcolsep) * \real{0.3810}}
  >{\raggedright\arraybackslash}p{(\linewidth - 2\tabcolsep) * \real{0.6190}}@{}}
\toprule\noalign{}
\begin{minipage}[b]{\linewidth}\raggedright
પાસું
\end{minipage} & \begin{minipage}[b]{\linewidth}\raggedright
વર્ણન
\end{minipage} \\
\midrule\noalign{}
\endhead
\bottomrule\noalign{}
\endlastfoot
\textbf{મિકેનિઝમ} & રેડિયો સિગ્નલ્સ ટ્રોપોસ્ફિયરિક અનિયમિતતાઓ અને રિફ્રેક્ટિવ
ઇન્ડેક્સ વેરિએશન્સથી વિખેરાય છે \\
\textbf{ફ્રિક્વન્સી} & સામાન્ય રીતે VHF, UHF (100 MHz - 10 GHz) \\
\textbf{રેન્જ} & 100-800 km, લાઇન-ઓફ-સાઇટથી આગળ \\
\textbf{વિશ્વસનીયતા} & લાઇન-ઓફ-સાઇટ કરતાં હવામાનથી ઓછી અસરગ્રસ્ત; આયનોસ્ફેરિક
કરતાં વધુ વિશ્વસનીય \\
\textbf{ઉપયોગો} & મિલિટરી કમ્યુનિકેશન્સ, દૂરસ્થ વિસ્તારો જ્યાં અન્ય સિસ્ટમ્સ
વ્યવહારિક નથી \\
\end{longtable}
}

\end{solutionbox}
\begin{mnemonicbox}
``STRIP'' - Scatter Through Refractive Index
Patterns

\end{mnemonicbox}
\subsection*{પ્રશ્ન 4(બ) [4
ગુણ]}\label{uxaaauxab0uxab6uxaa8-4uxaac-4-uxa97uxaa3}

\textbf{વ્યાખ્યા લખો: (1) વર્ચ્યુઅલ હાઇટ (2) મેક્સિમમ યુઝેબલ ફ્રિક્વન્સી - MUF (3)
ક્રિટિકલ ફ્રિક્વન્સી}

\begin{solutionbox}

{\def\LTcaptype{none} % do not increment counter
\begin{longtable}[]{@{}
  >{\raggedright\arraybackslash}p{(\linewidth - 2\tabcolsep) * \real{0.3333}}
  >{\raggedright\arraybackslash}p{(\linewidth - 2\tabcolsep) * \real{0.6667}}@{}}
\toprule\noalign{}
\begin{minipage}[b]{\linewidth}\raggedright
શબ્દ
\end{minipage} & \begin{minipage}[b]{\linewidth}\raggedright
વ્યાખ્યા
\end{minipage} \\
\midrule\noalign{}
\endhead
\bottomrule\noalign{}
\endlastfoot
\textbf{વર્ચ્યુઅલ હાઇટ} & આયનોસ્ફિયરનું આભાસી ઊંચાઈ જે પૃથ્વી પર પાછા પરાવર્તિત
થયેલા રેડિયો સિગ્નલના સમય વિલંબથી ગણવામાં આવે છે, જાણે કે પરાવર્તન એક જ બિંદુએ થયું
હોય \\
\textbf{મેક્સિમમ યુઝેબલ ફ્રિક્વન્સી (MUF)} & નિર્દિષ્ટ પાથ અને સમય માટે આયનોસ્ફિયરિક
પરાવર્તન દ્વારા વિશ્વસનીય કમ્યુનિકેશન માટે ઉપયોગ કરી શકાય તેવી ઉચ્ચતમ ફ્રિક્વન્સી \\
\textbf{ક્રિટિકલ ફ્રિક્વન્સી} & ઊભી દિશામાં આયનોસ્ફિયર તરફ પ્રસારિત થયા પછી
પાછી પરાવર્તિત થઈ શકે તેવી ઉચ્ચતમ ફ્રિક્વન્સી (જ્યારે આપાત કોણ 90^\circ હોય) \\
\end{longtable}
}

\end{solutionbox}
\begin{mnemonicbox}
``VMC'' - Virtual height Measures Critical
reflection

\end{mnemonicbox}
\subsection*{પ્રશ્ન 4(ક) [7
ગુણ]}\label{uxaaauxab0uxab6uxaa8-4uxa95-7-uxa97uxaa3}

\textbf{ઇલેક્ટ્રો મેગ્નેટીક વેવ પર ગ્રાઉંડની અસરો સમજાવો}

\begin{solutionbox}

\begin{verbatim}
              /|{ Direct wave}
               |
   Transmitter | Receiver
      o        |        o
       {       |       /}
        {      |      /}
         {     |     /}
          {    |    /}
           {   |   /}
            {  |  /}
             { | /}
              {|/}
     ──────────────────────
           Ground
     .............|...........
                  |
                  | Ground reflected wave
                 {|/}
\end{verbatim}

{\def\LTcaptype{none} % do not increment counter
\begin{longtable}[]{@{}
  >{\raggedright\arraybackslash}p{(\linewidth - 2\tabcolsep) * \real{0.3810}}
  >{\raggedright\arraybackslash}p{(\linewidth - 2\tabcolsep) * \real{0.6190}}@{}}
\toprule\noalign{}
\begin{minipage}[b]{\linewidth}\raggedright
અસર
\end{minipage} & \begin{minipage}[b]{\linewidth}\raggedright
વર્ણન
\end{minipage} \\
\midrule\noalign{}
\endhead
\bottomrule\noalign{}
\endlastfoot
\textbf{ગ્રાઉન્ડ રિફ્લેક્શન} & સિગ્નલ ગ્રાઉન્ડ પરથી પરાવર્તિત થાય છે, જેનાથી
મલ્ટીપાથ રિસેપ્શન થાય છે \\
\textbf{ગ્રાઉન્ડ એબ્સોર્પશન} & સિગ્નલ ઊર્જાનો એક ભાગ ભૂમિ દ્વારા શોષાય છે, જેથી
સિગ્નલ શક્તિ ઘટે છે \\
\textbf{ગ્રાઉન્ડ ડિફ્રેક્શન} & તરંગો અવરોધોની આસપાસ વળે છે, લાઇન-ઓફ-સાઇટથી આગળ
કવરેજ વધારે છે \\
\textbf{પૃથ્વીની વક્રતા} & એન્ટેનાની ઊંચાઈના આધારે લાઇન-ઓફ-સાઇટ અંતરને મર્યાદિત કરે
છે \\
\textbf{ગ્રાઉન્ડ કન્ડક્ટિવિટી} & ઉચ્ચ કન્ડક્ટિવિટી (પાણી, ભીની માટી) નબળા
કન્ડક્ટર્સ (સૂકા, ખડકાળ ભૂમિ) કરતાં વધુ સારો પ્રસરણ મંજૂરી આપે છે \\
\end{longtable}
}

\textbf{તરંગ વર્તન સમીકરણ:}

\begin{itemize}
\tightlist
\item
  રેન્જ (km) \approx 4.12(\sqrth_{1} + \sqrth_{2}) જ્યાં h_{1}, h_{2} એન્ટેનાની મીટરમાં ઊંચાઈ છે
\end{itemize}

\end{solutionbox}
\begin{mnemonicbox}
``RADAR'' - Reflection Absorption Diffraction Affect
Range

\end{mnemonicbox}
\subsection*{પ્રશ્ન 4(અ) OR [3
ગુણ]}\label{uxaaauxab0uxab6uxaa8-4uxa85-or-3-uxa97uxaa3}

\textbf{ડક્ટ પ્રોપોગેશન સમજાવો}

\begin{solutionbox}

ડક્ટ પ્રોપોગેશન ત્યારે થાય છે જ્યારે રેડિયો તરંગો વિશેષ રિફ્રેક્ટિવ ગુણધર્મો સાથેના
વાતાવરણીય સ્તરોમાં ફસાઈ જાય છે.

\begin{verbatim}
   ──────────────────────────────────
   Normal atmosphere
   ──────────────────────────────────
   Temperature inversion layer
   ∿∿∿∿∿∿∿∿∿∿∿∿∿∿∿∿∿∿∿∿∿∿∿∿∿∿∿∿∿∿∿∿
   o TX                           o RX
   ──────────────────────────────────
   Normal atmosphere
   ──────────────────────────────────
\end{verbatim}

\begin{itemize}
\tightlist
\item
  \textbf{ફોર્મેશન}: તાપમાન વિપરીતતા અથવા ભેજ ગ્રેડિયન્ટ વાતાવરણીય ડક્ટ બનાવે છે
\item
  \textbf{અસર}: સિગ્નલ્સ ડક્ટની અંદર ફસાય છે, સામાન્ય રેન્જથી ઘણી દૂર સુધી પ્રસરણની
  મંજૂરી આપે છે
\item
  \textbf{ફ્રિક્વન્સી}: UHF અને માઇક્રોવેવ બેન્ડમાં સૌથી સામાન્ય
\item
  \textbf{ઉપયોગો}: વિસ્તારિત ઓવર-વોટર કમ્યુનિકેશન્સ, રડાર એનોમલીઝ
\end{itemize}

\end{solutionbox}
\begin{mnemonicbox}
``TIDE'' - Trapped In Ducting Environment

\end{mnemonicbox}
\subsection*{પ્રશ્ન 4(બ) OR [4
ગુણ]}\label{uxaaauxab0uxab6uxaa8-4uxaac-or-4-uxa97uxaa3}

\textbf{આઇનોસ્ફીયર ના જુદા જુદા સ્તરો સમજાવો}

\begin{solutionbox}

{\def\LTcaptype{none} % do not increment counter
\begin{longtable}[]{@{}
  >{\raggedright\arraybackslash}p{(\linewidth - 4\tabcolsep) * \real{0.2059}}
  >{\raggedright\arraybackslash}p{(\linewidth - 4\tabcolsep) * \real{0.2941}}
  >{\raggedright\arraybackslash}p{(\linewidth - 4\tabcolsep) * \real{0.5000}}@{}}
\toprule\noalign{}
\begin{minipage}[b]{\linewidth}\raggedright
સ્તર
\end{minipage} & \begin{minipage}[b]{\linewidth}\raggedright
ઊંચાઈ
\end{minipage} & \begin{minipage}[b]{\linewidth}\raggedright
લક્ષણો
\end{minipage} \\
\midrule\noalign{}
\endhead
\bottomrule\noalign{}
\endlastfoot
\textbf{D સ્તર} & 60-90 km & દિવસના સમયે HF તરંગોને શોષે છે, રાત્રે ગાયબ થઈ જાય
છે \\
\textbf{E સ્તર} & 90-150 km & 10 MHz સુધીની આવૃત્તિઓને પરાવર્તિત કરે છે, સ્પોરેડિક
E ઘટના \\
\textbf{F1 સ્તર} & 150-210 km & દિવસ દરમિયાન હાજર, રાત્રે F2 સાથે ભળી જાય
છે \\
\textbf{F2 સ્તર} & 210-400+ km & મુખ્ય પરાવર્તન સ્તર, ઉચ્ચતમ ઇલેક્ટ્રોન ઘનતા,
દિવસ અને રાત હાજર \\
\end{longtable}
}

\end{solutionbox}
\begin{mnemonicbox}
``DEAF'' - D absorbs, E reflects, All merge, F2
persists

\end{mnemonicbox}
\subsection*{પ્રશ્ન 4(ક) OR [7
ગુણ]}\label{uxaaauxab0uxab6uxaa8-4uxa95-or-7-uxa97uxaa3}

\textbf{ગ્રાઉંડ વેવ અને સ્કાય વેવ પ્રોપોગેશન સમજાવો}

\begin{solutionbox}

\textbf{ગ્રાઉન્ડ વેવ પ્રોપોગેશન:}

\begin{verbatim}
    TX                                RX
     o─────────────────────────────────o
      {                               /}
       {                             /}
        {                           /}
         {-{-}{-}{-}{-}{-}{-}{-}{-}{-}{-}{-}{-}{-}{-}{-}{-}{-}{-}{-}{-}{-}{-}{-}{-}{-}{-}{-}{-}}
              Earth{s surface}
\end{verbatim}

\begin{itemize}
\tightlist
\item
  \textbf{ફ્રિક્વન્સી રેન્જ}: LF, MF (30 kHz - 3 MHz)
\item
  \textbf{ઘટકો}: ડાયરેક્ટ, ગ્રાઉન્ડ-રિફ્લેક્ટેડ, સરફેસ વેવ્સ
\item
  \textbf{રેન્જ}: આવૃત્તિ, ગ્રાઉન્ડ કન્ડક્ટિવિટી, ટ્રાન્સમીટર પાવર પર નિર્ભર
\item
  \textbf{ઉપયોગો}: AM બ્રોડકાસ્ટિંગ, નેવિગેશન સિસ્ટમ્સ, મેરીટાઇમ કમ્યુનિકેશન્સ
\end{itemize}

\textbf{સ્કાય વેવ પ્રોપોગેશન:}

\begin{verbatim}
                     /|{}
                      |  Ionosphere
    ─────────────────────────────────────
                /     |     {}
               /      |      {}
     TX o─────/       |       {────o RX}
              {                /}
               {              /}
                {            /}
                 {-{-}{-}{-}{-}{-}{-}{-}{-}{-}{-}{-}}
                 Earth{s surface}
\end{verbatim}

\begin{itemize}
\tightlist
\item
  \textbf{મિકેનિઝમ}: આયનોસ્ફિયર દ્વારા તરંગો પૃથ્વી પર પાછા વળે છે
\item
  \textbf{ફ્રિક્વન્સી}: મુખ્યત્વે HF (3-30 MHz)
\item
  \textbf{રેન્જ}: 100-10,000+ km, મલ્ટીપલ હોપ્સ શક્ય
\item
  \textbf{વેરિએબિલિટી}: દિવસનો સમય, ઋતુ, સૌર પ્રવૃત્તિ, આવૃત્તિ
\item
  \textbf{ઉપયોગો}: આંતરરાષ્ટ્રીય પ્રસારણ, એમેચ્યોર રેડિયો, લશ્કરી
\end{itemize}

\end{solutionbox}
\begin{mnemonicbox}
``GIST'' - Ground-Interface Surface Transmission vs
Ionospheric Sky Transmission

\end{mnemonicbox}
\subsection*{પ્રશ્ન 5(અ) [3
ગુણ]}\label{uxaaauxab0uxab6uxaa8-5uxa85-3-uxa97uxaa3}

\textbf{ત્રણ જુદી જુદી જાતના ઉપગ્રહો સમજાવો}

\begin{solutionbox}

{\def\LTcaptype{none} % do not increment counter
\begin{longtable}[]{@{}
  >{\raggedright\arraybackslash}p{(\linewidth - 2\tabcolsep) * \real{0.4848}}
  >{\raggedright\arraybackslash}p{(\linewidth - 2\tabcolsep) * \real{0.5152}}@{}}
\toprule\noalign{}
\begin{minipage}[b]{\linewidth}\raggedright
ઉપગ્રહ પ્રકાર
\end{minipage} & \begin{minipage}[b]{\linewidth}\raggedright
લક્ષણો
\end{minipage} \\
\midrule\noalign{}
\endhead
\bottomrule\noalign{}
\endlastfoot
\textbf{LEO (લો અર્થ ઓર્બિટ)} & ઊંચાઈ: 160-2,000 km, અવધિ: 90 મિનિટ,
ઉપયોગો: પૃથ્વી નિરીક્ષણ, કમ્યુનિકેશન્સ \\
\textbf{MEO (મીડિયમ અર્થ ઓર્બિટ)} & ઊંચાઈ: 2,000-35,786 km, અવધિ: 2-24
કલાક, ઉપયોગો: નેવિગેશન (GPS) \\
\textbf{GEO (જિઓસ્ટેશનરી ઓર્બિટ)} & ઊંચાઈ: 35,786 km, અવધિ: 24 કલાક, ઉપયોગો:
TV બ્રોડકાસ્ટિંગ, હવામાન નિરીક્ષણ \\
\end{longtable}
}

\end{solutionbox}
\begin{mnemonicbox}
``LMG'' - Low Medium Geostationary

\end{mnemonicbox}
\subsection*{પ્રશ્ન 5(બ) [4
ગુણ]}\label{uxaaauxab0uxab6uxaa8-5uxaac-4-uxa97uxaa3}

\textbf{સ્માર્ટ એન્ટેના શું છે? તેના બે ઉપયોગો જણાવો}

\begin{solutionbox}

સ્માર્ટ એન્ટેના એવી એન્ટેના સિસ્ટમ છે જે સ્પેશિયલ સિગ્નેચર્સને ઓળખવા અને ડાયનેમિકલી
રેડિએશન પેટર્ન એડજસ્ટ કરવા માટે ડિજિટલ સિગ્નલ પ્રોસેસિંગ એલ્ગોરિધમનો ઉપયોગ કરે છે.

{\def\LTcaptype{none} % do not increment counter
\begin{longtable}[]{@{}
  >{\raggedright\arraybackslash}p{(\linewidth - 2\tabcolsep) * \real{0.4091}}
  >{\raggedright\arraybackslash}p{(\linewidth - 2\tabcolsep) * \real{0.5909}}@{}}
\toprule\noalign{}
\begin{minipage}[b]{\linewidth}\raggedright
ફીચર
\end{minipage} & \begin{minipage}[b]{\linewidth}\raggedright
વર્ણન
\end{minipage} \\
\midrule\noalign{}
\endhead
\bottomrule\noalign{}
\endlastfoot
\textbf{પ્રકારો} & સ્વિચ્ડ બીમ સિસ્ટમ્સ, એડેપ્ટિવ એરે સિસ્ટમ્સ \\
\textbf{ઓપરેશન} & બદલાતી પરિસ્થિતિઓને અનુકૂળ થવા માટે મલ્ટીપલ એન્ટેના એલિમેન્ટ્સ અને
સિગ્નલ પ્રોસેસિંગનો ઉપયોગ કરે છે \\
\textbf{લાભો} & ક્ષમતા વધારી, કવરેજમાં સુધારો, દખલમાં ઘટાડો \\
\end{longtable}
}

\textbf{ઉપયોગો:}

\begin{enumerate}
\tightlist
\item
  મોબાઇલ સેલ્યુલર નેટવર્ક્સ (4G, 5G) ક્ષમતા અને કવરેજ વધારવા માટે
\item
  સુધારેલા થ્રૂપુટ અને ઘટાડેલા દખલગીરી માટે વાયરલેસ LAN
\end{enumerate}

\end{solutionbox}
\begin{mnemonicbox}
``SMART'' - Signal Manipulation And Response
Technology

\end{mnemonicbox}
\subsection*{પ્રશ્ન 5(ક) [7
ગુણ]}\label{uxaaauxab0uxab6uxaa8-5uxa95-7-uxa97uxaa3}

\textbf{ઉપગ્રહ આધારિત સંદેશા વ્યવહાર શું છે? ડેટા કમ્યુનિકેશન વિશે સમજાવો.}

\begin{solutionbox}

\textbf{સેટેલાઇટ કમ્યુનિકેશન} એ પૃથ્વી પરના વિવિધ બિંદુઓ વચ્ચે કમ્યુનિકેશન લિંક્સ પ્રદાન
કરવા માટે કૃત્રિમ ઉપગ્રહોનો ઉપયોગ છે.

\begin{verbatim}
               ┌───────┐
               │       │
               │  SAT  │
               │       │
               └───────┘
                /     {}
               /       {}
       Uplink /         { Downlink}
             /           {}
            /             {}
     ┌─────┐               ┌─────┐
     │     │               │     │
     │ TX  │               │ RX  │
     │     │               │     │
     └─────┘               └─────┘
\end{verbatim}

\textbf{ઉપગ્રહ દ્વારા ડેટા કમ્યુનિકેશન:}

{\def\LTcaptype{none} % do not increment counter
\begin{longtable}[]{@{}
  >{\raggedright\arraybackslash}p{(\linewidth - 2\tabcolsep) * \real{0.5238}}
  >{\raggedright\arraybackslash}p{(\linewidth - 2\tabcolsep) * \real{0.4762}}@{}}
\toprule\noalign{}
\begin{minipage}[b]{\linewidth}\raggedright
ઘટક
\end{minipage} & \begin{minipage}[b]{\linewidth}\raggedright
કાર્ય
\end{minipage} \\
\midrule\noalign{}
\endhead
\bottomrule\noalign{}
\endlastfoot
\textbf{અર્થ સ્ટેશન} & ઉપગ્રહોને/થી સિગ્નલ્સ ટ્રાન્સમિટ/રિસીવ કરે છે \\
\textbf{ટ્રાન્સપોન્ડર} & અલગ-અલગ આવૃત્તિઓ પર સિગ્નલ્સ પ્રાપ્ત કરે છે, એમ્પલિફાય કરે છે
અને ફરીથી પ્રસારિત કરે છે \\
\textbf{એક્સેસ મેથડ્સ} & FDMA, TDMA, CDMA મલ્ટિપલ યુઝર્સને ઉપગ્રહ ક્ષમતા શેર કરવાની
મંજૂરી આપે છે \\
\textbf{પ્રોટોકોલ્સ} & સેટેલાઇટ લેટેન્સી, સ્પેશિયલાઇઝ્ડ પ્રોટોકોલ્સ માટે TCP/IP
એડેપ્ટેશન \\
\textbf{ઉપયોગો} & ઇન્ટરનેટ બેકહોલ, VSAT નેટવર્ક્સ, IoT, કોર્પોરેટ નેટવર્ક્સ \\
\textbf{ફાયદા} & વિશાળ કવરેજ વિસ્તાર, ટેરેસ્ટ્રિયલ ઇન્ફ્રાસ્ટ્રક્ચરથી સ્વતંત્રતા \\
\textbf{પડકારો} & સિગ્નલ ડિલે (લેટેન્સી), પાવર મર્યાદાઓ, હવામાન અસરો \\
\end{longtable}
}

\end{solutionbox}
\begin{mnemonicbox}
``UPDATA'' - Uplink Provides Data Access To All

\end{mnemonicbox}
\subsection*{પ્રશ્ન 5(અ) OR [3
ગુણ]}\label{uxaaauxab0uxab6uxaa8-5uxa85-or-3-uxa97uxaa3}

\textbf{કેપલરના ઉપગ્રહ વિશેના નિયમો લખો}

\begin{solutionbox}

{\def\LTcaptype{none} % do not increment counter
\begin{longtable}[]{@{}
  >{\raggedright\arraybackslash}p{(\linewidth - 2\tabcolsep) * \real{0.5357}}
  >{\raggedright\arraybackslash}p{(\linewidth - 2\tabcolsep) * \real{0.4643}}@{}}
\toprule\noalign{}
\begin{minipage}[b]{\linewidth}\raggedright
કેપલરના નિયમો
\end{minipage} & \begin{minipage}[b]{\linewidth}\raggedright
વર્ણન
\end{minipage} \\
\midrule\noalign{}
\endhead
\bottomrule\noalign{}
\endlastfoot
\textbf{પ્રથમ નિયમ} & ઉપગ્રહો ઇલિપ્ટિકલ પાથમાં ભ્રમણ કરે છે જેમાં પૃથ્વી એલિપ્સના એક
ફોકસ પર હોય છે \\
\textbf{બીજો નિયમ} & ઉપગ્રહ અને પૃથ્વીને જોડતી રેખા સમાન સમયમાં સમાન ક્ષેત્રફળ
પસાર કરે છે (એન્ગ્યુલર મોમેન્ટમ સંરક્ષણ) \\
\textbf{ત્રીજો નિયમ} & કક્ષીય અવધિનો વર્ગ કક્ષાના અર્ધ-મેજર અક્ષના ઘનફળના
સમપ્રમાણમાં હોય છે \\
\end{longtable}
}

\end{solutionbox}
\begin{mnemonicbox}
``ESP'' - Elliptical orbits, Sweep equal areas,
Period-distance relation

\end{mnemonicbox}
\subsection*{પ્રશ્ન 5(બ) OR [4
ગુણ]}\label{uxaaauxab0uxab6uxaa8-5uxaac-or-4-uxa97uxaa3}

\textbf{બેઝ સ્ટેશન અને મોબાઇલ સ્ટેશન એન્ટેના વિશે સમજાવો}

\begin{solutionbox}

\textbf{બેઝ સ્ટેશન એન્ટેના:}

\begin{verbatim}
       ┌─┐
       │ │
       │ │
       │ │
       │ │
       │ │ Vertical collinear
       │ │
       │ │
       │ │
       └─┘
\end{verbatim}

\begin{itemize}
\tightlist
\item
  \textbf{પ્રકારો}: ઓમ્નિડાયરેક્શનલ, સેક્ટર, પેનલ એન્ટેના
\item
  \textbf{ગેઇન}: સામાન્ય રીતે 10-18 dBi
\item
  \textbf{માઉન્ટિંગ}: ટાવર અથવા છત પર ઇન્સ્ટોલેશન
\item
  \textbf{ફીચર્સ}: ડાઉનટિલ્ટ ક્ષમતા, મલ્ટીપલ ફ્રિક્વન્સી બેન્ડ
\end{itemize}

\textbf{મોબાઇલ સ્ટેશન એન્ટેના:}

\begin{verbatim}
       ┌───────┐
       │       │
       │  ─┬─  │ Internal antenna
       │       │
       └───────┘  Smartphone
\end{verbatim}

\begin{itemize}
\tightlist
\item
  \textbf{પ્રકારો}: ઇન્ટરનલ PIFA, પેચ, મોનોપોલ એન્ટેના
\item
  \textbf{ગેઇન}: લો ગેઇન (0-3 dBi)
\item
  \textbf{સાઇઝ}: કોમ્પેક્ટ, ઘણી વખત ડિવાઇસની અંદર એકીકૃત
\item
  \textbf{લક્ષણો}: ઓમ્નિડાયરેક્શનલ પેટર્ન, મલ્ટીપલ બેન્ડ
\end{itemize}

\end{solutionbox}
\begin{mnemonicbox}
``BIMS'' - Base stations Install Multiple Sectors,
Mobile stations Stay small

\end{mnemonicbox}
\subsection*{પ્રશ્ન 5(ક) OR [7
ગુણ]}\label{uxaaauxab0uxab6uxaa8-5uxa95-or-7-uxa97uxaa3}

\textbf{DTH રીસીવર સિસ્ટમ વિસ્તારથી સમજાવો}

\begin{solutionbox}

DTH (ડાયરેક્ટ-ટુ-હોમ) રિસીવર સિસ્ટમ ઉપગ્રહ દ્વારા સીધા વપરાશકર્તાઓને ટેલિવિઝન
સિગ્નલ્સ પહોંચાડે છે.

\begin{verbatim}
                    ┌───────┐
                    │ { │ Satellite}
                    └───────┘
                        │
                        │
                        V
                    ┌───────┐
                    │ ///// │ Dish antenna
                    └───┬───┘
                        │
                        │
         ┌──────────────┴──────────┐
         │                         │
    ┌────┴─────┐             ┌─────┴─────┐
    │  LNB     │             │ Set{-top   │}
    │(Outdoor) │─────Cable───│   Box     │──────► TV
    └──────────┘             │ (Indoor)  │
                             └───────────┘
\end{verbatim}

{\def\LTcaptype{none} % do not increment counter
\begin{longtable}[]{@{}
  >{\raggedright\arraybackslash}p{(\linewidth - 2\tabcolsep) * \real{0.5238}}
  >{\raggedright\arraybackslash}p{(\linewidth - 2\tabcolsep) * \real{0.4762}}@{}}
\toprule\noalign{}
\begin{minipage}[b]{\linewidth}\raggedright
ઘટક
\end{minipage} & \begin{minipage}[b]{\linewidth}\raggedright
કાર્ય
\end{minipage} \\
\midrule\noalign{}
\endhead
\bottomrule\noalign{}
\endlastfoot
\textbf{ડિશ એન્ટેના} & ઉપગ્રહ સિગ્નલ્સ એકત્રિત કરવા માટે પેરાબોલિક રિફ્લેક્ટર
(45-90 cm સામાન્ય વ્યાસ) \\
\textbf{LNB (લો નોઇઝ બ્લોક)} & કોએક્સિયલ કેબલ દ્વારા ટ્રાન્સમિશન માટે
ઉચ્ચ-આવૃત્તિના ઉપગ્રહ સિગ્નલ્સને નીચી આવૃત્તિઓમાં રૂપાંતરિત કરે છે \\
\textbf{કોએક્સિયલ કેબલ} & LNBથી સેટ-ટોપ બોક્સ સુધી સિગ્નલ્સ લઈ જાય છે \\
\textbf{સેટ-ટોપ બોક્સ} & સિગ્નલ્સને ડીકોડ/ડીમોડ્યુલેટ કરે છે, યુઝર ઇન્ટરફેસ, કન્ડિશનલ
એક્સેસ પ્રદાન કરે છે \\
\textbf{કન્ડિશનલ એક્સેસ મોડ્યુલ} & સુરક્ષા અને સબ્સ્ક્રિપ્શન મેનેજમેન્ટ પ્રદાન કરે છે \\
\textbf{ફીચર્સ} & ઇલેક્ટ્રોનિક પ્રોગ્રામ ગાઇડ, રેકોર્ડિંગ, ઇન્ટરેક્ટિવ સર્વિસીસ \\
\end{longtable}
}

\end{solutionbox}
\begin{mnemonicbox}
``DISCS'' - Dish Intercepts Signals, Converter Sends
to Set-top box

\end{mnemonicbox}

\end{document}
