\documentclass[10pt,a4paper]{article}

% content/resources/templates/preamble.tex
\usepackage[margin=0.6in]{geometry}
\author{Milav Dabgar}
\usepackage{amsmath,amssymb,amsthm}
\usepackage{booktabs}
\usepackage{multirow}
\usepackage{xcolor}
\usepackage{tcolorbox}
\tcbuselibrary{breakable,skins}
\usepackage[colorlinks=true,linkcolor=blue]{hyperref}
\usepackage{titlesec}
\usepackage{enumitem}
\usepackage{tikz}
\usepackage{pgfplots}
\usepackage{circuitikz}
\usepackage[version=4]{mhchem}
\usepackage{longtable}
\usepackage{array}
\usepackage{float}
\usepackage{caption}
\usepackage{listings}

\lstset{
  basicstyle=\small\ttfamily,
  breaklines=true,
  breakatwhitespace=false,
  postbreak=\mbox{\textcolor{red}{$\hookrightarrow$}\space},
  float=false,
  numbers=left,
  numberstyle=\tiny\color{gray},
  numbersep=10pt,
  xleftmargin=2em,
  keywordstyle=\color{blue},
  commentstyle=\color{green!60!black},
  stringstyle=\color{purple},
  backgroundcolor=\color{gray!5},
  showstringspaces=false,
  tabsize=2,
  captionpos=b,
  keepspaces=true,
  columns=flexible
}

\pgfplotsset{compat=1.18}
\usetikzlibrary{shapes,arrows,positioning,calc,patterns,decorations.pathmorphing,decorations.markings,arrows.meta}

% Color scheme
\definecolor{headcolor}{RGB}{0,102,204}
\definecolor{keycolor}{RGB}{220,20,60}
\definecolor{solutioncolor}{RGB}{34,139,34}
\definecolor{mnemoniccolor}{RGB}{148,0,211}
\definecolor{codecolor}{RGB}{0,0,100}

% Spacing
\setlength{\parskip}{3pt}
\setlist[itemize]{nosep}
\setlist[enumerate]{nosep}

% Title formatting
\titleformat{\section}{\Large\bfseries\color{headcolor}}{\thesection}{1em}{}
\titleformat{\subsection}{\large\bfseries\color{headcolor}}{\thesubsection}{1em}{}

% Pandoc tightlist compatibility
\providecommand{\tightlist}{%
  \setlength{\itemsep}{0pt}\setlength{\parskip}{0pt}}

% Pandoc longtable compatibility
\newcounter{none}
\def\thenone{}


% content/resources/templates/english-boxes.tex
% This file is currently empty - it exists to maintain consistency with the import structure.
% Add custom environments here if needed in the future.


\begin{document}

\begin{center}
{\Huge\bfseries\color{headcolor} Subject Name Solutions}\\[5pt]
{\LARGE 4341106 -- Winter 2024}\\[3pt]
{\large Semester 1 Study Material}\\[3pt]
{\normalsize\textit{Detailed Solutions and Explanations}}
\end{center}

\vspace{10pt}

\subsection*{Question 1(a) [3 marks]}\label{q1a}

\textbf{Define: (1) Directivity, (2) Gain, and (3) HPBW}

\begin{solutionbox}


{\def\LTcaptype{none} % do not increment counter
\vspace{-5pt}
\captionof{table}{Antenna Parameters Definitions}
\vspace{-10pt}
\begin{longtable}[]{@{}
  >{\raggedright\arraybackslash}p{(\linewidth - 2\tabcolsep) * \real{0.4783}}
  >{\raggedright\arraybackslash}p{(\linewidth - 2\tabcolsep) * \real{0.5217}}@{}}
\toprule\noalign{}
\begin{minipage}[b]{\linewidth}\raggedright
Parameter
\end{minipage} & \begin{minipage}[b]{\linewidth}\raggedright
Definition
\end{minipage} \\
\midrule\noalign{}
\endhead
\bottomrule\noalign{}
\endlastfoot
\textbf{Directivity} & The ratio of radiation intensity in a given
direction to the average radiation intensity in all directions, \\
\textbf{Gain} & The ratio of power radiated in a specific direction to
the power that would be radiated by an isotropic antenna with the same
input power \\
\textbf{HPBW (Half Power Beam Width)} & The angular width of the main
lobe where the power falls to half (-3dB) of its maximum value \\
\end{longtable}
}

\end{solutionbox}
\begin{mnemonicbox}
``DGH: Direction Gets Higher power with narrow beam''

\end{mnemonicbox}
\subsection*{Question 1(b) [4 marks]}\label{q1b}

\textbf{List the properties of electromagnetic waves}

\begin{solutionbox}


{\def\LTcaptype{none} % do not increment counter
\vspace{-5pt}
\captionof{table}{Properties of Electromagnetic Waves}
\vspace{-10pt}
\begin{longtable}[]{@{}
  >{\raggedright\arraybackslash}p{(\linewidth - 2\tabcolsep) * \real{0.4348}}
  >{\raggedright\arraybackslash}p{(\linewidth - 2\tabcolsep) * \real{0.5652}}@{}}
\toprule\noalign{}
\begin{minipage}[b]{\linewidth}\raggedright
Property
\end{minipage} & \begin{minipage}[b]{\linewidth}\raggedright
Description
\end{minipage} \\
\midrule\noalign{}
\endhead
\bottomrule\noalign{}
\endlastfoot
\textbf{Transverse nature} & Electric and magnetic fields are
perpendicular to each other and to direction of propagation \\
\textbf{Velocity} & Travel at speed of light (3\times10^{8} m/s) in free
space \\
\textbf{Frequency range} & Vary from few Hz to several THz \\
\textbf{Energy transport} & Carry energy from one point to another
without need of medium \\
\textbf{Reflection} & Can be reflected from conducting surfaces \\
\textbf{Refraction} & Change direction when passing between different
media \\
\textbf{Diffraction} & Can bend around obstacles \\
\textbf{Polarization} & The orientation of electric field vector \\
\end{longtable}
}

\end{solutionbox}
\begin{mnemonicbox}
``TVFERRDP: Travel Very Fast, Energy Reflects
Refracts Diffracts Polarizes''

\end{mnemonicbox}
\subsection*{Question 1(c) [7 marks]}\label{q1c}

\textbf{Explain physical concept of generation of Electromagnetic wave}

\begin{solutionbox}

\textbf{Diagram: Generation of Electromagnetic Wave}

\begin{center}
\textbf{Mermaid Diagram (Code)}
\begin{verbatim}
{Shaded}
{Highlighting}[]
graph LR
    A[Oscillating Electric Charge] {-{-}{} B[Time{-}varying Electric Field]}
    B {-{-}{} C[Time{-}varying Magnetic Field]}
    C {-{-}{} D[Time{-}varying Electric Field]}
    D {-{-}{} E[Self{-}sustaining EM Wave]}
    style A fill:\#f9f,stroke:\#333
    style E fill:\#bbf,stroke:\#333
{Highlighting}
{Shaded}
\end{verbatim}
\end{center}

\textbf{Process of EM Wave Generation:}

\begin{itemize}
\tightlist
\item
  \textbf{Accelerating charge}: When electric charge accelerates, it
  produces time-varying electric field
\item
  \textbf{Changing electric field}: This creates a time-varying magnetic
  field
\item
  \textbf{Changing magnetic field}: In turn creates a time-varying
  electric field
\item
  \textbf{Self-propagation}: This mutual creation of fields results in
  self-propagating wave
\item
  \textbf{Energy transfer}: EM waves transfer energy from transmitter to
  receiver
\end{itemize}

\textbf{Maxwell's Equations}: These four equations mathematically
describe the generation and propagation of EM waves:

\begin{enumerate}
\tightlist
\item
  Electric field from charges (Gauss's law)
\item
  No magnetic monopoles exist
\item
  Electric fields from changing magnetic fields (Faraday's law)
\item
  Magnetic fields from currents and changing electric fields (Ampere's
  law)
\end{enumerate}

\end{solutionbox}
\begin{mnemonicbox}
``CASES: Charges Accelerate, Self-sustaining
Electric-Magnetic fields''

\end{mnemonicbox}
\subsection*{Question 1(c) OR [7
marks]}\label{q1c}

\textbf{Explain how electromagnetic field radiated from a center fed
dipole}

\begin{solutionbox}

\textbf{Diagram: Radiation from Center-Fed Dipole}

\begin{center}
\textbf{Mermaid Diagram (Code)}
\begin{verbatim}
{Shaded}
{Highlighting}[]
graph LR
    A[RF Generator] {-{-}{} B[Center{-}Fed Dipole]}
    B {-{-}{} C\{Current Flow\}}
    C {-{-}{} D[Electric Field]}
    C {-{-}{} E[Magnetic Field]}
    D {-{-}{} F[Radiation Pattern]}
    E {-{-}{} F}
    style A fill:\#f9f,stroke:\#333
    style F fill:\#bbf,stroke:\#333
{Highlighting}
{Shaded}
\end{verbatim}
\end{center}

\textbf{Radiation Process:}

{\def\LTcaptype{none} % do not increment counter
\begin{longtable}[]{@{}
  >{\raggedright\arraybackslash}p{(\linewidth - 2\tabcolsep) * \real{0.4375}}
  >{\raggedright\arraybackslash}p{(\linewidth - 2\tabcolsep) * \real{0.5625}}@{}}
\toprule\noalign{}
\begin{minipage}[b]{\linewidth}\raggedright
Stage
\end{minipage} & \begin{minipage}[b]{\linewidth}\raggedright
Process
\end{minipage} \\
\midrule\noalign{}
\endhead
\bottomrule\noalign{}
\endlastfoot
\textbf{1. Current excitation} & RF signal applied at center of dipole
creates alternating current \\
\textbf{2. Current distribution} & Sinusoidal current distribution forms
along dipole, maximum at center, zero at ends \\
\textbf{3. Electric field} & Oscillating charges create time-varying
electric field perpendicular to dipole \\
\textbf{4. Magnetic field} & Current flow creates magnetic field
perpendicular to both dipole and electric field \\
\textbf{5. Near field} & Complex field pattern forms close to antenna
(\textless{} λ/2π) \\
\textbf{6. Far field} & At distances \textgreater{} 2λ, radiation
stabilizes to form distinctive pattern with main and side lobes \\
\end{longtable}
}

\textbf{Characteristics:}

\begin{itemize}
\tightlist
\item
  \textbf{Maximum radiation}: Perpendicular to dipole axis
\item
  \textbf{Null radiation}: Along dipole axis
\item
  \textbf{Omnidirectional}: In azimuth plane (perpendicular to dipole)
\item
  \textbf{Polarization}: Same as orientation of dipole
\end{itemize}

\end{solutionbox}
\begin{mnemonicbox}
``COME-FR: Current Oscillates, Making
Electric-magnetic Fields that Radiate''

\end{mnemonicbox}
\subsection*{Question 2(a) [3 marks]}\label{q2a}

\textbf{Differentiate the resonant and non-resonant antennas}

\begin{solutionbox}


{\def\LTcaptype{none} % do not increment counter
\vspace{-5pt}
\captionof{table}{Resonant vs Non-Resonant Antennas}
\vspace{-10pt}
\begin{longtable}[]{@{}
  >{\raggedright\arraybackslash}p{(\linewidth - 4\tabcolsep) * \real{0.2037}}
  >{\raggedright\arraybackslash}p{(\linewidth - 4\tabcolsep) * \real{0.3519}}
  >{\raggedright\arraybackslash}p{(\linewidth - 4\tabcolsep) * \real{0.4444}}@{}}
\toprule\noalign{}
\begin{minipage}[b]{\linewidth}\raggedright
Parameter
\end{minipage} & \begin{minipage}[b]{\linewidth}\raggedright
Resonant Antennas
\end{minipage} & \begin{minipage}[b]{\linewidth}\raggedright
Non-Resonant Antennas
\end{minipage} \\
\midrule\noalign{}
\endhead
\bottomrule\noalign{}
\endlastfoot
\textbf{Physical length} & Multiple of λ/2 (usually λ/2 or λ) & Not
related to wavelength (typically \textgreater{} λ) \\
\textbf{Standing waves} & Strong standing waves present & Minimal
standing waves \\
\textbf{Current distribution} & Sinusoidal with maximum at center &
Traveling wave with uniform amplitude \\
\textbf{Input impedance} & Resistive (at resonant frequency) & Complex
(resistive + reactive) \\
\textbf{Bandwidth} & Narrow bandwidth & Wide bandwidth \\
\textbf{Examples} & Half-wave dipole, folded dipole & Rhombic antenna,
traveling wave antenna \\
\end{longtable}
}

\end{solutionbox}
\begin{mnemonicbox}
``SIN-CIB: Size, Impedance, Narrow vs Complex,
Impedance, Broad''

\end{mnemonicbox}
\subsection*{Question 2(b) [4 marks]}\label{q2b}

\textbf{Explain Yagi antenna and discuss its radiation characteristics}

\begin{solutionbox}

\textbf{Diagram: Yagi-Uda Antenna}

\begin{verbatim}
      Feed point
         |
         v
   R     D     D1    D2    D3  
   |     |     |     |     |
   |     |     |     |     |
  [=]{-{-}{-}[=]{-}{-}{-}[=]{-}{-}{-}[=]{-}{-}{-}[=]}
   |     |     |     |     |
   |     |     |     |     |
 Reflector Driven  Directors
         Element
\end{verbatim}

\textbf{Yagi Antenna Components:}

\begin{itemize}
\tightlist
\item
  \textbf{Driven element}: Half-wave dipole connected to transmission
  line
\item
  \textbf{Reflector}: Slightly longer than driven element, placed behind
  it
\item
  \textbf{Directors}: Multiple elements shorter than driven element,
  placed in front
\end{itemize}

\textbf{Radiation Characteristics:}

\begin{itemize}
\tightlist
\item
  \textbf{Directivity}: High (7-12 dBi) with more directors
\item
  \textbf{Radiation pattern}: Unidirectional, narrow beam along director
  axis
\item
  \textbf{Front-to-back ratio}: 15-20 dB (good rejection of signals from
  rear)
\item
  \textbf{Bandwidth}: Moderate (around 5\% of center frequency)
\item
  \textbf{Gain}: Increases with number of directors (typically 3-20 dBi)
\end{itemize}

\end{solutionbox}
\begin{mnemonicbox}
``DRDU: Directors Radiate, Driven powers,
Unidirectional beam''

\end{mnemonicbox}
\subsection*{Question 2(c) [7 marks]}\label{q2c}

\textbf{Describe radiation characteristics of resonant wire antennas and
draw the current distribution of λ/2, 3λ/2 and 5λ/2 antenna}

\begin{solutionbox}

\textbf{Diagram: Current Distribution on Resonant Wire Antennas}

\begin{verbatim}
λ/2:     |{{-}{-}{-}{-}{-}{-}{-} λ/2 {-}{-}{-}{-}{-}{-}{-}|}
         +{-{-}{-}{-}{-}{-}{-}{-}{-}{-}+{-}{-}{-}{-}{-}{-}{-}{-}{-}{-}+}
         |          |          |
         v          \^{          v}
         |          |          |
         |          |          |
Current: *          *          *
        min        max        min

3λ/2:    |{{-}{-}{-}{-}{-}{-}{-}{-}{-}{-}{-}{-}{-} 3λ/2 {-}{-}{-}{-}{-}{-}{-}{-}{-}{-}{-}{-}{-}|}
         +{-{-}{-}{-}{-}+{-}{-}{-}{-}{-}+{-}{-}{-}{-}{-}+{-}{-}{-}{-}{-}+{-}{-}{-}{-}{-}+{-}{-}{-}{-}{-}+}
         |     |     |     |     |     |     |
         v     \^{     v     \^{}     v     \^{}     v}
         |     |     |     |     |     |     |
         |     |     |     |     |     |     |
Current: *     *     *     *     *     *     *
        min   max   min   max   min   max   min

5λ/2:    |{{-}{-}{-}{-}{-}{-}{-}{-}{-}{-}{-}{-}{-}{-}{-}{-}{-}{-} 5λ/2 {-}{-}{-}{-}{-}{-}{-}{-}{-}{-}{-}{-}{-}{-}{-}{-}{-}{-}|}
         +{-{-}{-}{-}+{-}{-}{-}{-}+{-}{-}{-}{-}+{-}{-}{-}{-}+{-}{-}{-}{-}+{-}{-}{-}{-}+{-}{-}{-}{-}+{-}{-}{-}{-}+{-}{-}{-}{-}+}
         |    |    |    |    |    |    |    |    |    |
         v    \^{    v    \^{}    v    \^{}    v    \^{}    v    \^{}}
         |    |    |    |    |    |    |    |    |    |
         |    |    |    |    |    |    |    |    |    |
Current: *    *    *    *    *    *    *    *    *    *
        min  max  min  max  min  max  min  max  min  max
\end{verbatim}

\textbf{Radiation Characteristics of Resonant Wire Antennas:}

{\def\LTcaptype{none} % do not increment counter
\begin{longtable}[]{@{}
  >{\raggedright\arraybackslash}p{(\linewidth - 2\tabcolsep) * \real{0.5517}}
  >{\raggedright\arraybackslash}p{(\linewidth - 2\tabcolsep) * \real{0.4483}}@{}}
\toprule\noalign{}
\begin{minipage}[b]{\linewidth}\raggedright
Characteristic
\end{minipage} & \begin{minipage}[b]{\linewidth}\raggedright
Description
\end{minipage} \\
\midrule\noalign{}
\endhead
\bottomrule\noalign{}
\endlastfoot
\textbf{Current distribution} & Sinusoidal, with maximum at center for
λ/2, additional maxima for longer antennas \\
\textbf{Input impedance} & Approximately 73Ω for λ/2, varies for longer
antennas \\
\textbf{Radiation pattern} & Figure-8 pattern (λ/2), more complex lobes
for longer antennas \\
\textbf{Directivity} & 2.15 dBi for λ/2, increases with length but with
multiple lobes \\
\textbf{Polarization} & Linear, parallel to wire orientation \\
\textbf{Efficiency} & High for properly constructed antennas \\
\end{longtable}
}

\textbf{Key Points:}

\begin{itemize}
\tightlist
\item
  λ/2 antenna has single current maximum at center
\item
  3λ/2 antenna has three half-cycles of current distribution
\item
  5λ/2 antenna has five half-cycles of current distribution
\item
  More half-wavelengths create more radiation lobes
\item
  Feed point is typically at current maximum for best impedance match
\end{itemize}

\end{solutionbox}
\begin{mnemonicbox}
``SIMPLE: Sinusoidal In Middle Produces Lobes
Efficiently''

\end{mnemonicbox}
\subsection*{Question 2(a) OR [3
marks]}\label{q2a}

\textbf{Differentiate the broad side and end fire array antennas}

\begin{solutionbox}


{\def\LTcaptype{none} % do not increment counter
\vspace{-5pt}
\captionof{table}{Broadside vs End Fire Array Antennas}
\vspace{-10pt}
\begin{longtable}[]{@{}
  >{\raggedright\arraybackslash}p{(\linewidth - 4\tabcolsep) * \real{0.2500}}
  >{\raggedright\arraybackslash}p{(\linewidth - 4\tabcolsep) * \real{0.3864}}
  >{\raggedright\arraybackslash}p{(\linewidth - 4\tabcolsep) * \real{0.3636}}@{}}
\toprule\noalign{}
\begin{minipage}[b]{\linewidth}\raggedright
Parameter
\end{minipage} & \begin{minipage}[b]{\linewidth}\raggedright
Broadside Array
\end{minipage} & \begin{minipage}[b]{\linewidth}\raggedright
End Fire Array
\end{minipage} \\
\midrule\noalign{}
\endhead
\bottomrule\noalign{}
\endlastfoot
\textbf{Direction of maximum radiation} & Perpendicular to the array
axis & Along the array axis \\
\textbf{Phase difference} & 0^\circ (in-phase) & 180^\circ or progressive phase \\
\textbf{Element spacing} & Typically λ/2 & Typically λ/4 to λ/2 \\
\textbf{Radiation pattern} & Narrow in plane containing array axis &
Narrow in plane perpendicular to array elements \\
\textbf{Directivity} & High, increases with number of elements & High,
increases with number of elements \\
\textbf{Applications} & Fixed point-to-point links & Direction finding,
radar \\
\end{longtable}
}

\end{solutionbox}
\begin{mnemonicbox}
``BEPODS: Broadside-End, Perpendicular-Or-Direction,
Spacing''

\end{mnemonicbox}
\subsection*{Question 2(b) OR [4
marks]}\label{q2b}

\textbf{Explain loop antenna and discuss its radiation characteristics}

\begin{solutionbox}

\textbf{Diagram: Loop Antenna Types}

\begin{center}
\textbf{Mermaid Diagram (Code)}
\begin{verbatim}
{Shaded}
{Highlighting}[]
graph TD
    A[Loop Antenna] {-{-}{} B[Small Loop{}br /{}Circumference {} λ/10]}
    A {-{-}{} C[Large Loop{}br /{}Circumference  λ]}
    style A fill:\#f9f,stroke:\#333
    style B fill:\#bbf,stroke:\#333
    style C fill:\#bbf,stroke:\#333
{Highlighting}
{Shaded}
\end{verbatim}
\end{center}

\textbf{Loop Antenna Characteristics:}

{\def\LTcaptype{none} % do not increment counter
\begin{longtable}[]{@{}
  >{\raggedright\arraybackslash}p{(\linewidth - 4\tabcolsep) * \real{0.3143}}
  >{\raggedright\arraybackslash}p{(\linewidth - 4\tabcolsep) * \real{0.3429}}
  >{\raggedright\arraybackslash}p{(\linewidth - 4\tabcolsep) * \real{0.3429}}@{}}
\toprule\noalign{}
\begin{minipage}[b]{\linewidth}\raggedright
Parameter
\end{minipage} & \begin{minipage}[b]{\linewidth}\raggedright
Small Loop
\end{minipage} & \begin{minipage}[b]{\linewidth}\raggedright
Large Loop
\end{minipage} \\
\midrule\noalign{}
\endhead
\bottomrule\noalign{}
\endlastfoot
\textbf{Current distribution} & Uniform around loop & Varies around
circumference \\
\textbf{Radiation pattern} & Figure-8 (perpendicular to loop plane) &
More complex with multiple lobes \\
\textbf{Directivity} & Low (1.5 dBi) & Higher (3-4 dBi) \\
\textbf{Polarization} & Magnetic field perpendicular to loop & Electric
field in plane of loop \\
\textbf{Input impedance} & Very low (\textless{} 10Ω) & Higher
(50-200Ω) \\
\textbf{Applications} & Direction finding, AM receivers & HF
communications, RFID \\
\end{longtable}
}

\end{solutionbox}
\begin{mnemonicbox}
``SCALED: Size Changes Antenna's Lobes, Efficiency,
and Direction''

\end{mnemonicbox}
\subsection*{Question 2(c) OR [7
marks]}\label{q2c}

\textbf{Describe radiation characteristics of non resonant wire antennas
and draw the radiation pattern of λ/2, 3λ/2 and 5λ/2 antenna}

\begin{solutionbox}

\textbf{Diagram: Radiation Patterns of Wire Antennas}

\begin{verbatim}
λ/2 Dipole:
                  * *
               *       *
              *         *
             *           *
            *             *
           *      {-{-}{-}      *}
           *     |   |     *
           *     |   |     *
           *      {-{-}{-}      *}
            *             *
             *           *
              *         *
               *       *
                  * *

3λ/2 Dipole:
                 *     *
              *           *
             *       *     *
            *      / {      *}
           *      /   {      *}
          *      |     |      *
          *      |     |      *
          *      |     |      *
          *      {     /      *}
           *      {   /      *}
            *      { /      *}
             *       *     *
              *           *
                 *     *

5λ/2 Dipole:
                *   *   *
             *               *
            *    *       *    *
           *   /   {   /      *}
          *   /     { /        *}
         *   |       |       |   *
         *   |       |       |   *
         *   |       |       |   *
         *   {       |       /   *}
          *   {     /      /   *}
           *   {   /      /   *}
            *    *       *    *
             *               *
                *   *   *
\end{verbatim}

\textbf{Non-Resonant Wire Antenna Characteristics:}

{\def\LTcaptype{none} % do not increment counter
\begin{longtable}[]{@{}
  >{\raggedright\arraybackslash}p{(\linewidth - 2\tabcolsep) * \real{0.5517}}
  >{\raggedright\arraybackslash}p{(\linewidth - 2\tabcolsep) * \real{0.4483}}@{}}
\toprule\noalign{}
\begin{minipage}[b]{\linewidth}\raggedright
Characteristic
\end{minipage} & \begin{minipage}[b]{\linewidth}\raggedright
Description
\end{minipage} \\
\midrule\noalign{}
\endhead
\bottomrule\noalign{}
\endlastfoot
\textbf{Current distribution} & Traveling waves with minimal standing
waves \\
\textbf{Termination} & Usually terminated with resistive load to prevent
reflections \\
\textbf{Bandwidth} & Wide bandwidth operation \\
\textbf{Input impedance} & More constant across frequency range \\
\textbf{Radiation pattern} & λ/2: Single main lobe on each side3λ/2:
Three main lobes on each side5λ/2: Five main lobes on each side \\
\textbf{Directivity} & Increases with length but divided among multiple
lobes \\
\textbf{Efficiency} & Lower than resonant antennas due to resistive
termination \\
\end{longtable}
}

\textbf{Key Points:}

\begin{itemize}
\tightlist
\item
  Non-resonant antennas use traveling waves instead of standing waves
\item
  Rhombic antenna is a common non-resonant antenna
\item
  λ/2 pattern has 2 main lobes (figure-8 pattern)
\item
  3λ/2 pattern has 6 main lobes (3 on each side)
\item
  5λ/2 pattern has 10 main lobes (5 on each side)
\item
  More lobes appear as length increases
\item
  Main beam angle changes with frequency
\end{itemize}

\end{solutionbox}
\begin{mnemonicbox}
``TRIBE-WL: Traveling Resistance Improves Bandwidth,
Efficiency Worse, Lobes multiply''

\end{mnemonicbox}
\subsection*{Question 3(a) [3 marks]}\label{q3a}

\textbf{Write short note on micro strip (patch) antenna}

\begin{solutionbox}

\textbf{Diagram: Microstrip Patch Antenna}

\begin{verbatim}
       Top View                 Side View
    +{-{-}{-}{-}{-}{-}{-}{-}{-}{-}{-}{-}+           +{-}{-}{-}{-}{-}{-}{-}{-}{-}{-}{-}{-}+}
    |            |           |////////////| {{-} Patch}
    |            |           +{-{-}{-}{-}{-}{-}{-}{-}{-}{-}{-}{-}+}
    |    Patch   |           |            | {{-} Dielectric}
    |            |           +{-{-}{-}{-}{-}{-}{-}{-}{-}{-}{-}{-}+}
    |            |           |\_\_\_\_\_\_\_\_\_\_\_\_| {{-} Ground plane}
    +{-{-}{-}{-}{-}{-}{-}{-}{-}{-}{-}{-}+}
    |   Feed     |
    +{-{-}{-}+{-}{-}{-}{-}+{-}{-}{-}+}
        |
\end{verbatim}

\textbf{Microstrip Patch Antenna:}

\begin{itemize}
\tightlist
\item
  \textbf{Structure}: Metal patch on dielectric substrate with ground
  plane
\item
  \textbf{Size}: Typically λ/2 \times λ/2 or λ/2 \times λ/4
\item
  \textbf{Feed methods}: Microstrip line, coaxial probe, aperture
  coupling
\item
  \textbf{Radiation}: From fringing fields at patch edges
\item
  \textbf{Polarization}: Linear or circular depending on patch shape
\item
  \textbf{Bandwidth}: Narrow (3-5\% of center frequency)
\item
  \textbf{Applications}: Mobile devices, satellites, aircraft, RFID
\end{itemize}

\end{solutionbox}
\begin{mnemonicbox}
``SLIM-PCB: Small, Lightweight, Integrable Microwave
Printed Circuit Board''

\end{mnemonicbox}
\subsection*{Question 3(b) [4 marks]}\label{q3b}

\textbf{Explain helical antenna and discuss its radiation
characteristics}

\begin{solutionbox}

\textbf{Diagram: Helical Antenna}

\begin{verbatim}
                  \^{}
                 /|{}
                / | {  Direction of maximum radiation}
               /  |  {}
              /   |   {}
             /    |    {}
            /     |     {}
      coil /      |      {}
          /       |       {}
  +{-{-}{-}{-}{-}{-}+{-}{-}{-}{-}{-}{-}{-}{-}+{-}{-}{-}{-}{-}{-}{-}{-}{-}+}
  |      |        |         |
  |      +{-{-}{-}{-}{-}{-}{-}{-}+         |}
  |      |                  |
  |      |                  |
  |      |                  |
  |      |                  |
  | \_\_\_\_\_+                  |
  |/     |                  |
  +{-{-}{-}{-}{-}{-}+{-}{-}{-}{-}{-}{-}{-}{-}{-}{-}{-}{-}{-}{-}{-}{-}{-}{-}+}
         |
      Ground plane
\end{verbatim}

\textbf{Helical Antenna Characteristics:}

{\def\LTcaptype{none} % do not increment counter
\begin{longtable}[]{@{}
  >{\raggedright\arraybackslash}p{(\linewidth - 4\tabcolsep) * \real{0.3056}}
  >{\raggedright\arraybackslash}p{(\linewidth - 4\tabcolsep) * \real{0.3611}}
  >{\raggedright\arraybackslash}p{(\linewidth - 4\tabcolsep) * \real{0.3333}}@{}}
\toprule\noalign{}
\begin{minipage}[b]{\linewidth}\raggedright
Parameter
\end{minipage} & \begin{minipage}[b]{\linewidth}\raggedright
Normal Mode
\end{minipage} & \begin{minipage}[b]{\linewidth}\raggedright
Axial Mode
\end{minipage} \\
\midrule\noalign{}
\endhead
\bottomrule\noalign{}
\endlastfoot
\textbf{Helix circumference} & Small (\textless{} λ/π) & About λ \\
\textbf{Radiation pattern} & Omnidirectional (like dipole) & Directional
(end-fire) \\
\textbf{Polarization} & Linear, perpendicular to helix axis & Circular
(RHCP or LHCP) \\
\textbf{Input impedance} & High (120-200Ω) & 100-200Ω \\
\textbf{Bandwidth} & Narrow & Wide (up to 70\%) \\
\textbf{Applications} & Mobile phones, FM radio & Satellite comms, space
telemetry \\
\end{longtable}
}

\textbf{Key Parameters:}

\begin{itemize}
\tightlist
\item
  Diameter (D)
\item
  Spacing between turns (S)
\item
  Number of turns (N)
\item
  Pitch angle (α)
\end{itemize}

\end{solutionbox}
\begin{mnemonicbox}
``NASA-CP: Normal Axial Spacing Affects Circular
Polarization''

\end{mnemonicbox}
\subsection*{Question 3(c) [7 marks]}\label{q3c}

\textbf{Explain horn antenna and discuss its radiation characteristics}

\begin{solutionbox}

\textbf{Diagram: Types of Horn Antennas}

\begin{center}
\textbf{Mermaid Diagram (Code)}
\begin{verbatim}
{Shaded}
{Highlighting}[]
graph TD
    A[Horn Antenna] {-{-}{} B[E{-}plane Horn]}
    A {-{-}{} C[H{-}plane Horn]}
    A {-{-}{} D[Pyramidal Horn]}
    A {-{-}{} E[Conical Horn]}
    style A fill:\#f9f,stroke:\#333
    style B fill:\#bbf,stroke:\#333
    style C fill:\#bbf,stroke:\#333
    style D fill:\#bbf,stroke:\#333
    style E fill:\#bbf,stroke:\#333
{Highlighting}
{Shaded}
\end{verbatim}
\end{center}

\textbf{Diagram: Horn Antenna Structure}

\begin{verbatim}
      Waveguide                Horn
    +{-{-}{-}{-}{-}{-}{-}{-}{-}{-}{-}+{-}{-}{-}{-}{-}{-}{-}{-}{-}{-}{-}{-}{-}+}
    |           |            /|
    |           |           / |
    |           |          /  |
    |    RF     |         /   |
    |   Feed    |        /    |
    |           |       /     |
    |           |      /      |
    |           |     /       |
    +{-{-}{-}{-}{-}{-}{-}{-}{-}{-}{-}+{-}{-}{-}{-}+{-}{-}{-}{-}{-}{-}{-}{-}+}
\end{verbatim}

\textbf{Horn Antenna Characteristics:}

{\def\LTcaptype{none} % do not increment counter
\begin{longtable}[]{@{}
  >{\raggedright\arraybackslash}p{(\linewidth - 2\tabcolsep) * \real{0.5517}}
  >{\raggedright\arraybackslash}p{(\linewidth - 2\tabcolsep) * \real{0.4483}}@{}}
\toprule\noalign{}
\begin{minipage}[b]{\linewidth}\raggedright
Characteristic
\end{minipage} & \begin{minipage}[b]{\linewidth}\raggedright
Description
\end{minipage} \\
\midrule\noalign{}
\endhead
\bottomrule\noalign{}
\endlastfoot
\textbf{Operating principle} & Gradual transition from waveguide to free
space \\
\textbf{Frequency range} & Microwave and mm-wave (1-300 GHz) \\
\textbf{Directivity} & Medium to high (10-20 dBi) \\
\textbf{Radiation pattern} & Directional with main lobe in forward
direction \\
\textbf{Beamwidth} & E-plane: 40-50^\circ, H-plane: 40-50^\circ, Pyramidal:
depends on dimensions \\
\textbf{Polarization} & Linear (matches waveguide) \\
\textbf{Bandwidth} & Very wide (\textgreater100\%) \\
\textbf{Efficiency} & Very high (\textgreater90\%) \\
\textbf{Applications} & Radar, satellite communications, EMC testing,
radio astronomy \\
\end{longtable}
}

\textbf{Types of Horn Antennas:}

\begin{itemize}
\tightlist
\item
  \textbf{E-plane horn}: Flared in electric field direction
\item
  \textbf{H-plane horn}: Flared in magnetic field direction
\item
  \textbf{Pyramidal horn}: Flared in both planes
\item
  \textbf{Conical horn}: Circular waveguide with conical flare
\end{itemize}

\end{solutionbox}
\begin{mnemonicbox}
``POWER-HF: Pyramidal Or Waveguide Extended, Radiates
High Frequencies''

\end{mnemonicbox}
\subsection*{Question 3(a) OR [3
marks]}\label{q3a}

\textbf{Write short note on slot antenna}

\begin{solutionbox}

\textbf{Diagram: Slot Antenna}

\begin{verbatim}
            +{-{-}{-}{-}{-}{-}{-}{-}{-}{-}{-}{-}{-}{-}{-}{-}{-}{-}{-}{-}{-}{-}{-}{-}{-}{-}{-}{-}{-}{-}+}
            |                              |
            |                              |
            |                              |
            |         +{-{-}{-}{-}{-}{-}{-}{-}{-}+          |}
            |         |         |          |
            |         |  Slot   |          |
            |         |         |          |
            |         +{-{-}{-}{-}{-}{-}{-}{-}{-}+          |}
            |                              |
            |                              |
            |                              |
            +{-{-}{-}{-}{-}{-}{-}{-}{-}{-}{-}{-}{-}{-}{-}{-}{-}{-}{-}{-}{-}{-}{-}{-}{-}{-}{-}{-}{-}{-}+}
                    Conductive Sheet
\end{verbatim}

\textbf{Slot Antenna:}

\begin{itemize}
\tightlist
\item
  \textbf{Structure}: Narrow slot cut in conductive sheet/plane
\item
  \textbf{Size}: Typically λ/2 long for resonance
\item
  \textbf{Feed method}: Across the slot at center or offset
\item
  \textbf{Radiation pattern}: Similar to dipole but rotated 90^\circ
  (Babinet's principle)
\item
  \textbf{Polarization}: Linear, perpendicular to slot length
\item
  \textbf{Impedance}: High (several hundred ohms)
\item
  \textbf{Applications}: Aircraft, satellites, base stations
\end{itemize}

\textbf{Key Points:}

\begin{itemize}
\tightlist
\item
  Complementary to dipole (Babinet's principle)
\item
  Radiates equally from both sides of plane
\item
  Can be flush-mounted (advantage for aerodynamics)
\item
  Can be covered with dielectric without affecting performance
\end{itemize}

\end{solutionbox}
\begin{mnemonicbox}
``SCRAP: Slot Cut Radiates Alternating Polarization''

\end{mnemonicbox}
\subsection*{Question 3(b) OR [4
marks]}\label{q3b}

\textbf{Explain parabolic reflector antenna and discuss its radiation
characteristics}

\begin{solutionbox}

\textbf{Diagram: Parabolic Reflector Antenna}

\begin{verbatim}
                      +
                     /|{}
                    / | {}
                   /  |  {}
        Incoming  /   |   {  Reflected}
          Waves  /    |    {   Waves}
                /     |     {}
               /      |      {}
              /       |       {}
             /        |        {}
            /         |         {}
           /          |          {}
          /           |           {}
     +{-{-}{-}+{-}{-}{-}{-}{-}{-}{-}{-}{-}{-}{-}{-}+{-}{-}{-}{-}{-}{-}{-}{-}{-}{-}{-}{-}+{-}{-}{-}+}
         {            |            /}
          {           |           /}
           {          |          /}
            {         |         /}
             {        |        /}
              {       |       /}
               {      |      /}
                {     |     /}
                 {    |    /}
                  {   |   /}
                   {  |  /}
                    { | /}
                     {|/}
                      +
                     Feed
                     Point
\end{verbatim}

\textbf{Parabolic Reflector Antenna Characteristics:}

{\def\LTcaptype{none} % do not increment counter
\begin{longtable}[]{@{}
  >{\raggedright\arraybackslash}p{(\linewidth - 2\tabcolsep) * \real{0.5517}}
  >{\raggedright\arraybackslash}p{(\linewidth - 2\tabcolsep) * \real{0.4483}}@{}}
\toprule\noalign{}
\begin{minipage}[b]{\linewidth}\raggedright
Characteristic
\end{minipage} & \begin{minipage}[b]{\linewidth}\raggedright
Description
\end{minipage} \\
\midrule\noalign{}
\endhead
\bottomrule\noalign{}
\endlastfoot
\textbf{Operating principle} & Focuses parallel incoming waves to focal
point (receiving) or collimates waves from focal point (transmitting) \\
\textbf{Frequency range} & From UHF to millimeter waves (300 MHz - 300
GHz) \\
\textbf{Directivity} & Very high (30-40 dBi for large dishes) \\
\textbf{Radiation pattern} & Highly directional, narrow main beam \\
\textbf{Beamwidth} & Inversely proportional to diameter (θ \approx 70λ/D
degrees) \\
\textbf{Feed types} & Prime focus, Cassegrain, Gregorian, offset \\
\textbf{Efficiency} & 50-70\% depending on feed design and blockage \\
\textbf{Applications} & Satellite communications, radio astronomy,
radar, microwave links \\
\end{longtable}
}

\textbf{Key Parameters:}

\begin{itemize}
\tightlist
\item
  Diameter (D)
\item
  Focal length (f)
\item
  f/D ratio (typically 0.3-0.6)
\end{itemize}

\end{solutionbox}
\begin{mnemonicbox}
``FIND-SHF: Focused, Intense Narrow Directivity for
Super High Frequencies''

\end{mnemonicbox}
\subsection*{Question 3(c) OR [7
marks]}\label{q3c}

\textbf{Describe V and inverted V antenna}

\begin{solutionbox}

\textbf{Diagram: V and Inverted V Antennas}

\begin{verbatim}
V Antenna:

            Feed
            Point
              +
             / {}
            /   {}
           /     {}
          /       {}
         /         {}
        /           {}
       /             {}
      /               {}
     /                 {}
    +                   +
   Ground              Ground


Inverted V Antenna:

              +
              |
              | Support
              |
              |
       +{-{-}{-}{-}{-}{-}+{-}{-}{-}{-}{-}{-}+}
      /               {}
     /                 {}
    /                   {}
   /                     {}
  /                       {}
 /                         {}
+                           +
|                           |
Feed Point
\end{verbatim}

\textbf{V Antenna Characteristics:}

{\def\LTcaptype{none} % do not increment counter
\begin{longtable}[]{@{}
  >{\raggedright\arraybackslash}p{(\linewidth - 2\tabcolsep) * \real{0.5517}}
  >{\raggedright\arraybackslash}p{(\linewidth - 2\tabcolsep) * \real{0.4483}}@{}}
\toprule\noalign{}
\begin{minipage}[b]{\linewidth}\raggedright
Characteristic
\end{minipage} & \begin{minipage}[b]{\linewidth}\raggedright
Description
\end{minipage} \\
\midrule\noalign{}
\endhead
\bottomrule\noalign{}
\endlastfoot
\textbf{Construction} & Two equal length wires arranged in V-shape \\
\textbf{Angle between arms} & 10-90^\circ (affects directivity) \\
\textbf{Length of each arm} & Typically multiple wavelengths (1-6λ) \\
\textbf{Radiation pattern} & Bidirectional for larger angles,
unidirectional for smaller angles \\
\textbf{Directivity} & 3-15 dBi (increases with arm length and decreases
with angle) \\
\textbf{Input impedance} & 300-900Ω (depends on included angle) \\
\textbf{Applications} & HF long-distance communications, shortwave
broadcasting \\
\end{longtable}
}

\textbf{Inverted V Antenna Characteristics:}

{\def\LTcaptype{none} % do not increment counter
\begin{longtable}[]{@{}
  >{\raggedright\arraybackslash}p{(\linewidth - 2\tabcolsep) * \real{0.5517}}
  >{\raggedright\arraybackslash}p{(\linewidth - 2\tabcolsep) * \real{0.4483}}@{}}
\toprule\noalign{}
\begin{minipage}[b]{\linewidth}\raggedright
Characteristic
\end{minipage} & \begin{minipage}[b]{\linewidth}\raggedright
Description
\end{minipage} \\
\midrule\noalign{}
\endhead
\bottomrule\noalign{}
\endlastfoot
\textbf{Construction} & Similar to dipole but bent down in V-shape \\
\textbf{Angle between arms} & 90-120^\circ typically \\
\textbf{Length of each arm} & λ/4 each (total λ/2) \\
\textbf{Radiation pattern} & Omnidirectional (slightly more overhead
than dipole) \\
\textbf{Input impedance} & Lower than dipole (typically 50Ω) \\
\textbf{Height requirement} & Only center needs to be high \\
\textbf{Applications} & Amateur radio, general HF communications \\
\end{longtable}
}

\textbf{Key Differences:}

\begin{itemize}
\tightlist
\item
  V antenna is horizontally oriented, Inverted V is vertically oriented
  with center up
\item
  V antenna usually has longer arms for directivity
\item
  Inverted V requires only one support point (center)
\item
  V antenna has higher directivity, Inverted V is more omnidirectional
\end{itemize}

\end{solutionbox}
\begin{mnemonicbox}
``VOVO: V Outward (radiation), V One-support
(inverted)''

\end{mnemonicbox}
\subsection*{Question 4(a) [3 marks]}\label{q4a}

\textbf{Define: (1) Reflection, (2) Refraction and (3) Diffraction}

\begin{solutionbox}


{\def\LTcaptype{none} % do not increment counter
\vspace{-5pt}
\captionof{table}{Wave Phenomena Definitions}
\vspace{-10pt}
\begin{longtable}[]{@{}
  >{\raggedright\arraybackslash}p{(\linewidth - 2\tabcolsep) * \real{0.5000}}
  >{\raggedright\arraybackslash}p{(\linewidth - 2\tabcolsep) * \real{0.5000}}@{}}
\toprule\noalign{}
\begin{minipage}[b]{\linewidth}\raggedright
Phenomenon
\end{minipage} & \begin{minipage}[b]{\linewidth}\raggedright
Definition
\end{minipage} \\
\midrule\noalign{}
\endhead
\bottomrule\noalign{}
\endlastfoot
\textbf{Reflection} & The bouncing back of electromagnetic waves when
they strike a boundary between two different media without penetrating
the second medium \\
\textbf{Refraction} & The bending of electromagnetic waves when they
pass from one medium to another due to change in wave velocity \\
\textbf{Diffraction} & The bending of electromagnetic waves around
obstacles or through openings, allowing waves to propagate into shadowed
regions \\
\end{longtable}
}

\end{solutionbox}
\begin{mnemonicbox}
``RRD: Rays Rebound, Redirect, Disperse''

\end{mnemonicbox}
\subsection*{Question 4(b) [4 marks]}\label{q4b}

\textbf{List HAM radio application for communication}

\begin{solutionbox}


{\def\LTcaptype{none} % do not increment counter
\vspace{-5pt}
\captionof{table}{HAM Radio Applications for Communication}
\vspace{-10pt}
\begin{longtable}[]{@{}
  >{\raggedright\arraybackslash}p{(\linewidth - 2\tabcolsep) * \real{0.4783}}
  >{\raggedright\arraybackslash}p{(\linewidth - 2\tabcolsep) * \real{0.5217}}@{}}
\toprule\noalign{}
\begin{minipage}[b]{\linewidth}\raggedright
Application Category
\end{minipage} & \begin{minipage}[b]{\linewidth}\raggedright
Specific Applications
\end{minipage} \\
\midrule\noalign{}
\endhead
\bottomrule\noalign{}
\endlastfoot
\textbf{Emergency communications} & Disaster relief, emergency response,
weather reporting \\
\textbf{Public service} & Community events, search and rescue, traffic
monitoring \\
\textbf{Technical experimentation} & Antenna design, propagation
studies, digital modes testing \\
\textbf{International goodwill} & DX communication, contesting,
international friendship \\
\textbf{Personal recreation} & Casual conversations, hobby groups, radio
clubs \\
\textbf{Educational outreach} & School programs, STEM activities,
training new operators \\
\textbf{Space communication} & Satellite operation, ISS contact, EME
(moon bounce) \\
\textbf{Digital communication} & APRS, packet radio, FT8, RTTY, PSK31 \\
\end{longtable}
}

\end{solutionbox}
\begin{mnemonicbox}
``EPTIPS-D: Emergency, Public, Technical,
International, Personal, Space, Digital''

\end{mnemonicbox}
\subsection*{Question 4(c) [7 marks]}\label{q4c}

\textbf{Explain ionosphere's layers and sky wave propagation}

\begin{solutionbox}

\textbf{Diagram: Ionospheric Layers and Sky Wave Propagation}

\begin{center}
\textbf{Mermaid Diagram (Code)}
\begin{verbatim}
{Shaded}
{Highlighting}[]
graph TD
    A[Transmitter] {-{-}{} B[Ionosphere]}
    B {-{-}{} C[F2 Layer{}br /{}250{-}450 km]}
    B {-{-}{} D[F1 Layer{}br /{}170{-}220 km]}
    B {-{-}{} E[E Layer{}br /{}90{-}120 km]}
    B {-{-}{} F[D Layer{}br /{}60{-}90 km]}
    C {-{-}{} G[Receiver]}
    style A fill:\#f9f,stroke:\#333
    style G fill:\#bbf,stroke:\#333
{Highlighting}
{Shaded}
\end{verbatim}
\end{center}

\textbf{Ionospheric Layers:}

{\def\LTcaptype{none} % do not increment counter
\begin{longtable}[]{@{}
  >{\raggedright\arraybackslash}p{(\linewidth - 6\tabcolsep) * \real{0.1207}}
  >{\raggedright\arraybackslash}p{(\linewidth - 6\tabcolsep) * \real{0.1724}}
  >{\raggedright\arraybackslash}p{(\linewidth - 6\tabcolsep) * \real{0.2931}}
  >{\raggedright\arraybackslash}p{(\linewidth - 6\tabcolsep) * \real{0.4138}}@{}}
\toprule\noalign{}
\begin{minipage}[b]{\linewidth}\raggedright
Layer
\end{minipage} & \begin{minipage}[b]{\linewidth}\raggedright
Altitude
\end{minipage} & \begin{minipage}[b]{\linewidth}\raggedright
Characteristics
\end{minipage} & \begin{minipage}[b]{\linewidth}\raggedright
Effect on Radio Waves
\end{minipage} \\
\midrule\noalign{}
\endhead
\bottomrule\noalign{}
\endlastfoot
\textbf{D Layer} & 60-90 km & Low ionization, exists only during
daylight & Absorbs LF/MF signals, minimal refraction \\
\textbf{E Layer} & 90-120 km & Medium ionization, stronger during day &
Refracts HF waves up to 5 MHz \\
\textbf{F1 Layer} & 170-220 km & Present only during day, merges with F2
at night & Refracts higher HF frequencies \\
\textbf{F2 Layer} & 250-450 km & Highest ionization, present day and
night & Main layer for long-distance HF communication \\
\end{longtable}
}

\textbf{Sky Wave Propagation Parameters:}

{\def\LTcaptype{none} % do not increment counter
\begin{longtable}[]{@{}
  >{\raggedright\arraybackslash}p{(\linewidth - 2\tabcolsep) * \real{0.4783}}
  >{\raggedright\arraybackslash}p{(\linewidth - 2\tabcolsep) * \real{0.5217}}@{}}
\toprule\noalign{}
\begin{minipage}[b]{\linewidth}\raggedright
Parameter
\end{minipage} & \begin{minipage}[b]{\linewidth}\raggedright
Definition
\end{minipage} \\
\midrule\noalign{}
\endhead
\bottomrule\noalign{}
\endlastfoot
\textbf{Virtual Height} & Apparent height where reflection seems to
occur (higher than actual due to gradual refraction) \\
\textbf{Critical Frequency} & Maximum frequency that can be reflected
when transmitted vertically \\
\textbf{Maximum Usable Frequency (MUF)} & Highest frequency that can be
used for communication between two points \\
\textbf{Skip Distance} & Minimum distance from transmitter where sky
waves return to Earth \\
\textbf{Lowest Usable Frequency (LUF)} & Minimum frequency that provides
reliable communication (below which D-layer absorption is too high) \\
\textbf{Optimum Working Frequency (OWF)} & Typically 85\% of MUF,
provides most reliable communication \\
\end{longtable}
}

\end{solutionbox}
\begin{mnemonicbox}
``DEFMSL: During day, Every Frequency Makes Somewhat
Longer paths''

\end{mnemonicbox}
\subsection*{Question 4(a) OR [3
marks]}\label{q4a}

\textbf{Define: (1) MUF, (2) LUF and (3) Skip distance}

\begin{solutionbox}


{\def\LTcaptype{none} % do not increment counter
\vspace{-5pt}
\captionof{table}{Sky Wave Propagation Terms}
\vspace{-10pt}
\begin{longtable}[]{@{}
  >{\raggedright\arraybackslash}p{(\linewidth - 2\tabcolsep) * \real{0.3333}}
  >{\raggedright\arraybackslash}p{(\linewidth - 2\tabcolsep) * \real{0.6667}}@{}}
\toprule\noalign{}
\begin{minipage}[b]{\linewidth}\raggedright
Term
\end{minipage} & \begin{minipage}[b]{\linewidth}\raggedright
Definition
\end{minipage} \\
\midrule\noalign{}
\endhead
\bottomrule\noalign{}
\endlastfoot
\textbf{MUF (Maximum Usable Frequency)} & The highest frequency that can
be used for reliable communication between two specific points via
ionospheric reflection \\
\textbf{LUF (Lowest Usable Frequency)} & The minimum frequency that
provides adequate signal strength for reliable communication despite
D-layer absorption \\
\textbf{Skip Distance} & The minimum distance from a transmitter at
which a sky wave of a specific frequency returns to Earth \\
\end{longtable}
}

\end{solutionbox}
\begin{mnemonicbox}
``MLS: Maximum frequency Leaps, Lowest frequency
Seeps, Skip distance Spans''

\end{mnemonicbox}
\subsection*{Question 4(b) OR [4
marks]}\label{q4b}

\textbf{List HAM radio digital modes of communication}

\begin{solutionbox}


{\def\LTcaptype{none} % do not increment counter
\vspace{-5pt}
\captionof{table}{HAM Radio Digital Modes}
\vspace{-10pt}
\begin{longtable}[]{@{}
  >{\raggedright\arraybackslash}p{(\linewidth - 4\tabcolsep) * \real{0.2692}}
  >{\raggedright\arraybackslash}p{(\linewidth - 4\tabcolsep) * \real{0.2500}}
  >{\raggedright\arraybackslash}p{(\linewidth - 4\tabcolsep) * \real{0.4808}}@{}}
\toprule\noalign{}
\begin{minipage}[b]{\linewidth}\raggedright
Digital Mode
\end{minipage} & \begin{minipage}[b]{\linewidth}\raggedright
Description
\end{minipage} & \begin{minipage}[b]{\linewidth}\raggedright
Typical Frequency Bands
\end{minipage} \\
\midrule\noalign{}
\endhead
\bottomrule\noalign{}
\endlastfoot
\textbf{FT8} & Low power, narrow bandwidth, automated exchange & HF
bands (especially 20m, 40m, 80m) \\
\textbf{PSK31} & Phase Shift Keying, keyboard-to-keyboard & HF bands
(especially 20m, 40m) \\
\textbf{RTTY} & Radio Teletype, oldest digital mode & HF bands \\
\textbf{APRS} & Automatic Packet Reporting System, position reporting &
VHF (typically 144.39 MHz in US) \\
\textbf{SSTV} & Slow Scan Television, image transmission & HF bands
(especially 20m) \\
\textbf{JT65/JT9} & Weak signal modes for EME and DX & HF and VHF
bands \\
\textbf{WINLINK} & Email over radio & HF and VHF bands \\
\textbf{DMR} & Digital Mobile Radio, voice digital mode & VHF and UHF
bands \\
\end{longtable}
}

\end{solutionbox}
\begin{mnemonicbox}
``PRAW-JDW: PSK, RTTY, APRS, WINLINK, JT65, DMR''

\end{mnemonicbox}
\subsection*{Question 4(c) OR [7
marks]}\label{q4c}

\textbf{Explain space wave propagation}

\begin{solutionbox}

\textbf{Diagram: Space Wave Propagation}

\begin{verbatim}
                             /{/////////  Troposphere}
      Tx                    /                    {              Rx}
  +{-{-}{-}+{-}{-}{-}+                /                               +{-}{-}{-}{-}+{-}{-}{-}{-}+}
  |       |  Direct Wave  /                        {        |         |}
  |       |{-{-}{-}{-}{-}{-}{-}{-}{-}{-}{-}{-}{-}{-}|{-}{-}{-}{-}{-}{-}{-}{-}{-}{-}{-}{-}{-}{-}{-}{-}{-}{-}{-}{-}{-}{-}{-}{-}{-}{-}|{-}{-}{-}{-}{-}{-}|         |}
  |       |              |                          |       |         |
  +{-{-}{-}{-}{-}{-}{-}+              |                          |       +{-}{-}{-}{-}{-}{-}{-}{-}{-}+}
      |                  |                          |             \^{}
      |                  |                          |             |
      |                  |                          |             |
      |                  |                          |             |
      |                  v                          v             |
      |              +{-{-}{-}{-}{-}{-}+                    +{-}{-}{-}{-}{-}{-}+         |}
      |              |      |                    |      |         |
      |              |      |                    |      |         |
      |              |      |                    |      |         |
      |              |      |                    |      |         |
      |              +{-{-}{-}{-}{-}{-}+                    +{-}{-}{-}{-}{-}{-}+         |}
      |                 |                           |             |
      |                 |                           |             |
      v                 |        Earth              |             |
  Reflected Wave {-{-}{-}{-}{-}{-}{-}|{-}{-}{-}{-}{-}{-}{-}{-}{-}{-}{-}{-}{-}{-}{-}{-}{-}{-}{-}{-}{-}{-}{-}{-}{-}{-}{-}|{-}{-}{-}{-}{-}{-}{-}{-}{-}{-}{-}{-}|}
\end{verbatim}

\textbf{Space Wave Propagation:}

Space wave propagation refers to radio waves that travel through the
troposphere (lower atmosphere) rather than via ionospheric reflection.
It includes:

{\def\LTcaptype{none} % do not increment counter
\begin{longtable}[]{@{}
  >{\raggedright\arraybackslash}p{(\linewidth - 2\tabcolsep) * \real{0.4583}}
  >{\raggedright\arraybackslash}p{(\linewidth - 2\tabcolsep) * \real{0.5417}}@{}}
\toprule\noalign{}
\begin{minipage}[b]{\linewidth}\raggedright
Component
\end{minipage} & \begin{minipage}[b]{\linewidth}\raggedright
Description
\end{minipage} \\
\midrule\noalign{}
\endhead
\bottomrule\noalign{}
\endlastfoot
\textbf{Direct wave} & Travels in straight line from transmitter to
receiver (line-of-sight) \\
\textbf{Ground-reflected wave} & Reflects off Earth's surface before
reaching receiver \\
\textbf{Surface wave} & Follows Earth's curvature due to diffraction \\
\end{longtable}
}

\textbf{Types of Space Wave Propagation:}

\begin{enumerate}
\tightlist
\item
  \textbf{Tropospheric Scatter Propagation:}

  \begin{itemize}
  \tightlist
  \item
    \textbf{Mechanism}: Signal scattering by irregularities in
    troposphere
  \item
    \textbf{Frequency range}: VHF, UHF, SHF (100 MHz - 10 GHz)
  \item
    \textbf{Distance}: 100-800 km (beyond horizon)
  \item
    \textbf{Characteristics}: High power required, fading common,
    reliable
  \item
    \textbf{Applications}: Military communications, backup links
  \end{itemize}
\item
  \textbf{Duct Propagation:}

  \begin{itemize}
  \tightlist
  \item
    \textbf{Mechanism}: Trapping of waves in atmospheric ducts (layers
    with abnormal refractive index)
  \item
    \textbf{Frequency range}: VHF, UHF, microwave
  \item
    \textbf{Distance}: Up to 2000 km (far beyond horizon)
  \item
    \textbf{Characteristics}: Seasonal/weather dependent, mainly over
    water
  \item
    \textbf{Applications}: Maritime communications, coastal radar
  \end{itemize}
\end{enumerate}

\textbf{Factors Affecting Space Wave Propagation:}

\begin{itemize}
\tightlist
\item
  \textbf{Height of antennas}: Higher antennas increase range
\item
  \textbf{Frequency}: Higher frequencies experience less diffraction
\item
  \textbf{Terrain}: Obstacles block signals (Fresnel zone clearance
  needed)
\item
  \textbf{Weather}: Temperature inversions, humidity affect ducting
\item
  \textbf{Earth's curvature}: Limits line-of-sight distance
\end{itemize}

\end{solutionbox}
\begin{mnemonicbox}
``DRIFT-SD: Direct Routes, Irregular Formations of
Troposphere, Scatter and Ducts''

\end{mnemonicbox}
\subsection*{Question 5(a) [3 marks]}\label{q5a}

\textbf{Define: (1) Beam area (2) Beam efficiency, and (3) Effective
aperture}

\begin{solutionbox}


{\def\LTcaptype{none} % do not increment counter
\vspace{-5pt}
\captionof{table}{Antenna Beam Parameters}
\vspace{-10pt}
\begin{longtable}[]{@{}
  >{\raggedright\arraybackslash}p{(\linewidth - 2\tabcolsep) * \real{0.4783}}
  >{\raggedright\arraybackslash}p{(\linewidth - 2\tabcolsep) * \real{0.5217}}@{}}
\toprule\noalign{}
\begin{minipage}[b]{\linewidth}\raggedright
Parameter
\end{minipage} & \begin{minipage}[b]{\linewidth}\raggedright
Definition
\end{minipage} \\
\midrule\noalign{}
\endhead
\bottomrule\noalign{}
\endlastfoot
\textbf{Beam Area} & The solid angle through which all of the power
radiated by the antenna would pass if the radiation intensity was
constant at its maximum value \\
\textbf{Beam Efficiency} & The ratio of power radiated in the main beam
to the total power radiated by the antenna \\
\textbf{Effective Aperture} & The ratio of power received by the antenna
to the power density of the incident wave \\
\end{longtable}
}

\end{solutionbox}
\begin{mnemonicbox}
``BEA: Beam area Encloses, efficiency Excludes
sidelobes, Aperture Extracts power''

\end{mnemonicbox}
\subsection*{Question 5(b) [4 marks]}\label{q5b}

\textbf{Describe need of smart antenna}

\begin{solutionbox}

\textbf{Diagram: Smart Antenna System}

\begin{center}
\textbf{Mermaid Diagram (Code)}
\begin{verbatim}
{Shaded}
{Highlighting}[]
graph LR
    A[Antenna Array] {-{-}{} B[Signal Processing]}
    B {-{-}{} C[Adaptive Algorithm]}
    C {-{-}{} D[Beamforming]}
    D {-{-}{} E[Interference Reduction]}
    D {-{-}{} F[Coverage Enhancement]}
    D {-{-}{} G[Capacity Increase]}
    style A fill:\#f9f,stroke:\#333
    style G fill:\#bbf,stroke:\#333
{Highlighting}
{Shaded}
\end{verbatim}
\end{center}

\textbf{Need for Smart Antennas:}

{\def\LTcaptype{none} % do not increment counter
\begin{longtable}[]{@{}
  >{\raggedright\arraybackslash}p{(\linewidth - 2\tabcolsep) * \real{0.3158}}
  >{\raggedright\arraybackslash}p{(\linewidth - 2\tabcolsep) * \real{0.6842}}@{}}
\toprule\noalign{}
\begin{minipage}[b]{\linewidth}\raggedright
Need
\end{minipage} & \begin{minipage}[b]{\linewidth}\raggedright
Description
\end{minipage} \\
\midrule\noalign{}
\endhead
\bottomrule\noalign{}
\endlastfoot
\textbf{Spectrum efficiency} & Reuse frequencies more effectively in
same geographic area \\
\textbf{Capacity enhancement} & Support more users in same bandwidth
through spatial separation \\
\textbf{Coverage extension} & Increase range by focusing energy in
desired directions \\
\textbf{Interference reduction} & Minimize effects of co-channel
interference and jammers \\
\textbf{Energy efficiency} & Reduce transmitted power by focusing energy
only where needed \\
\textbf{Multipath mitigation} & Reduce fading by selecting optimal
signal paths \\
\textbf{Location services} & Enable direction finding and positioning
applications \\
\textbf{Signal quality} & Improve SNR through spatial filtering \\
\end{longtable}
}

\end{solutionbox}
\begin{mnemonicbox}
``SLIM-ACES: Spectrum efficiency, Location services,
Interference reduction, Multipath mitigation, Adaptive beams, Capacity,
Energy, Signal quality''

\end{mnemonicbox}
\subsection*{Question 5(c) [7 marks]}\label{q5c}

\textbf{Draw the DTH Receiver indoor and outdoor black diagram and
discuss its functions}

\begin{solutionbox}

\textbf{Diagram: DTH Receiver System Block Diagram}

\begin{verbatim}
      OUTDOOR UNIT                          INDOOR UNIT
+{-{-}{-}{-}{-}{-}{-}{-}{-}{-}{-}{-}{-}{-}{-}{-}{-}{-}{-}{-}{-}+             +{-}{-}{-}{-}{-}{-}{-}{-}{-}{-}{-}{-}{-}{-}{-}{-}{-}{-}{-}{-}{-}{-}{-}{-}{-}+}
|                     |             |                         |
|  +{-{-}{-}{-}{-}{-}{-}{-}{-}{-}{-}{-}{-}+    |             |   +{-}{-}{-}{-}{-}{-}{-}{-}{-}{-}{-}{-}{-}{-}+      |}
|  |             |    |             |   |              |      |
|  |  Satellite  |    |  Coaxial    |   |    Tuner/    |      |
|  |   Dish      |{-{-}{-}{-}+{-}{-}{-}Cable{-}{-}{-}{-}{-}+{-}{-}| Demodulator  |      |}
|  |             |    |             |   |              |      |
|  +{-{-}{-}{-}{-}{-}{-}{-}{-}{-}{-}{-}{-}+    |             |   +{-}{-}{-}{-}{-}{-}{-}{-}{-}{-}{-}{-}{-}{-}+      |}
|        |            |             |          |              |
|  +{-{-}{-}{-}{-}{-}{-}{-}{-}{-}{-}{-}{-}+    |             |   +{-}{-}{-}{-}{-}{-}{-}{-}{-}{-}{-}{-}{-}{-}+      |}
|  |    LNB      |    |             |   |   MPEG{-2/4   |      |}
|  | (Low Noise  |    |             |   |   Decoder    |      |   +{-{-}{-}{-}{-}{-}{-}+}
|  |  Block)     |    |             |   |              |{-{-}{-}{-}{-}{-}|{-}{-}|  TV   |}
|  +{-{-}{-}{-}{-}{-}{-}{-}{-}{-}{-}{-}{-}+    |             |   +{-}{-}{-}{-}{-}{-}{-}{-}{-}{-}{-}{-}{-}{-}+      |   |       |}
|                     |             |          |              |   +{-{-}{-}{-}{-}{-}{-}+}
+{-{-}{-}{-}{-}{-}{-}{-}{-}{-}{-}{-}{-}{-}{-}{-}{-}{-}{-}{-}{-}+             |   +{-}{-}{-}{-}{-}{-}{-}{-}{-}{-}{-}{-}{-}{-}+      |}
                                    |   | Conditional  |      |
                                    |   |   Access     |      |
                                    |   |   Module     |      |
                                    |   +{-{-}{-}{-}{-}{-}{-}{-}{-}{-}{-}{-}{-}{-}+      |}
                                    |          |              |
                                    |   +{-{-}{-}{-}{-}{-}{-}{-}{-}{-}{-}{-}{-}{-}+      |}
                                    |   |   System     |      |
                                    |   | Controller/  |      |
                                    |   |     CPU      |      |
                                    |   +{-{-}{-}{-}{-}{-}{-}{-}{-}{-}{-}{-}{-}{-}+      |}
                                    |          |              |
                                    |   +{-{-}{-}{-}{-}{-}{-}{-}{-}{-}{-}{-}{-}{-}+      |}
                                    |   |    User      |      |
                                    |   |  Interface   |      |
                                    |   +{-{-}{-}{-}{-}{-}{-}{-}{-}{-}{-}{-}{-}{-}+      |}
                                    |                         |
                                    +{-{-}{-}{-}{-}{-}{-}{-}{-}{-}{-}{-}{-}{-}{-}{-}{-}{-}{-}{-}{-}{-}{-}{-}{-}+}
\end{verbatim}

\textbf{DTH Receiver System Components and Functions:}

\textbf{Outdoor Unit Components:}

{\def\LTcaptype{none} % do not increment counter
\begin{longtable}[]{@{}
  >{\raggedright\arraybackslash}p{(\linewidth - 2\tabcolsep) * \real{0.5238}}
  >{\raggedright\arraybackslash}p{(\linewidth - 2\tabcolsep) * \real{0.4762}}@{}}
\toprule\noalign{}
\begin{minipage}[b]{\linewidth}\raggedright
Component
\end{minipage} & \begin{minipage}[b]{\linewidth}\raggedright
Function
\end{minipage} \\
\midrule\noalign{}
\endhead
\bottomrule\noalign{}
\endlastfoot
\textbf{Satellite Dish} & Collects and reflects weak satellite signals
to focal point \\
\textbf{LNB (Low Noise Block)} & Receives signals from dish, amplifies
them with minimal noise addition, and converts high frequency (10-12
GHz) to lower IF frequency (950-2150 MHz) \\
\end{longtable}
}

\textbf{Indoor Unit Components:}

{\def\LTcaptype{none} % do not increment counter
\begin{longtable}[]{@{}
  >{\raggedright\arraybackslash}p{(\linewidth - 2\tabcolsep) * \real{0.5238}}
  >{\raggedright\arraybackslash}p{(\linewidth - 2\tabcolsep) * \real{0.4762}}@{}}
\toprule\noalign{}
\begin{minipage}[b]{\linewidth}\raggedright
Component
\end{minipage} & \begin{minipage}[b]{\linewidth}\raggedright
Function
\end{minipage} \\
\midrule\noalign{}
\endhead
\bottomrule\noalign{}
\endlastfoot
\textbf{Tuner/Demodulator} & Selects desired channel frequency,
demodulates signal to extract digital data stream \\
\textbf{MPEG-2/4 Decoder} & Decodes compressed video/audio signals into
viewable/audible content \\
\textbf{Conditional Access Module} & Provides security and decryption
for subscribed channels \\
\textbf{System Controller/CPU} & Manages overall operation, processes
user commands, updates software \\
\textbf{User Interface} & Provides on-screen display, receives remote
control inputs \\
\end{longtable}
}

\textbf{Signal Flow Process:}

\begin{enumerate}
\tightlist
\item
  Satellite dish collects signals and focuses them to LNB
\item
  LNB amplifies, filters and converts signals to lower frequency
\item
  Coaxial cable carries IF signals to indoor unit
\item
  Tuner selects channel and demodulates signal
\item
  Conditional access module decrypts authorized content
\item
  MPEG decoder converts digital stream to audio/video
\item
  Output sent to television for viewing
\end{enumerate}

\end{solutionbox}
\begin{mnemonicbox}
``SALT-DCU: Satellite dish And LNB Transmit,
Demodulator Converts and Unscrambles''

\end{mnemonicbox}
\subsection*{Question 5(a) OR [3
marks]}\label{q5a}

\textbf{Define: (1) Antenna, (2) Folded dipole, and (3) Antenna array}

\begin{solutionbox}


{\def\LTcaptype{none} % do not increment counter
\vspace{-5pt}
\captionof{table}{Antenna Definitions}
\vspace{-10pt}
\begin{longtable}[]{@{}
  >{\raggedright\arraybackslash}p{(\linewidth - 2\tabcolsep) * \real{0.3333}}
  >{\raggedright\arraybackslash}p{(\linewidth - 2\tabcolsep) * \real{0.6667}}@{}}
\toprule\noalign{}
\begin{minipage}[b]{\linewidth}\raggedright
Term
\end{minipage} & \begin{minipage}[b]{\linewidth}\raggedright
Definition
\end{minipage} \\
\midrule\noalign{}
\endhead
\bottomrule\noalign{}
\endlastfoot
\textbf{Antenna} & A device that converts electrical signals into
electromagnetic waves for transmission or electromagnetic waves into
electrical signals for reception \\
\textbf{Folded Dipole} & A dipole antenna modified by adding a second
conductor connected at both ends to the first, forming a narrow loop
with feed point at the bottom center \\
\textbf{Antenna Array} & A system of multiple antenna elements arranged
in a specific geometric pattern to achieve desired radiation
characteristics \\
\end{longtable}
}

\end{solutionbox}
\begin{mnemonicbox}
``AFD: Antenna Feeds, Folded Doubles impedance,
Directivity increases with Arrays''

\end{mnemonicbox}
\subsection*{Question 5(b) OR [4
marks]}\label{q5b}

\textbf{Describe application of smart antenna}

\begin{solutionbox}


{\def\LTcaptype{none} % do not increment counter
\vspace{-5pt}
\captionof{table}{Smart Antenna Applications}
\vspace{-10pt}
\begin{longtable}[]{@{}
  >{\raggedright\arraybackslash}p{(\linewidth - 2\tabcolsep) * \real{0.4286}}
  >{\raggedright\arraybackslash}p{(\linewidth - 2\tabcolsep) * \real{0.5714}}@{}}
\toprule\noalign{}
\begin{minipage}[b]{\linewidth}\raggedright
Application Area
\end{minipage} & \begin{minipage}[b]{\linewidth}\raggedright
Specific Applications
\end{minipage} \\
\midrule\noalign{}
\endhead
\bottomrule\noalign{}
\endlastfoot
\textbf{Mobile Communications} & Base stations for 4G/5G networks,
capacity enhancement, coverage improvement \\
\textbf{Wi-Fi Systems} & MIMO routers, extended range access points,
interference mitigation in dense deployments \\
\textbf{Radar Systems} & Phased array radars, target tracking,
electronic warfare, weather radars \\
\textbf{Satellite Communications} & Adaptive beamforming, tracking earth
stations, interference rejection \\
\textbf{Military/Defense} & Jammers, secure communications,
reconnaissance, surveillance \\
\textbf{IoT Networks} & Low-power wide-area networks, directional
coverage for sensors \\
\textbf{Vehicle Communications} & V2X communications, autonomous
vehicles, collision avoidance \\
\textbf{Indoor Positioning} & Location-based services, asset tracking,
emergency services \\
\end{longtable}
}

\textbf{Key Smart Antenna Technologies:}

\begin{itemize}
\tightlist
\item
  \textbf{Switched Beam}: Predetermined fixed beam patterns
\item
  \textbf{Adaptive Array}: Dynamic beam adjustment based on signal
  environment
\item
  \textbf{MIMO (Multiple Input Multiple Output)}: Multiple antennas for
  spatial multiplexing
\end{itemize}

\end{solutionbox}
\begin{mnemonicbox}
``SWIM-MIV: Satellite, Wireless, IoT, Military,
Mobile, Indoor positioning, Vehicles''

\end{mnemonicbox}
\subsection*{Question 5(c) OR [7
marks]}\label{q5c}

\textbf{Explain Terrestrial mobile communication antennas and also
discuss about base station and mobile station antennas}

\begin{solutionbox}

\textbf{Diagram: Terrestrial Mobile Communication System}

\begin{center}
\textbf{Mermaid Diagram (Code)}
\begin{verbatim}
{Shaded}
{Highlighting}[]
graph TD
    A[Base Station] {-{-}{-} B[Mobile Station]}
    A {-{-}{-} C[Mobile Station]}
    A {-{-}{-} D[Mobile Station]}
    E[Base Station Antennas] {-{-}{-} F[High Gain{}br /{}Sectorized]}
    E {-{-}{-} G[Omnidirectional]}
    E {-{-}{-} H[Smart Antennas]}
    I[Mobile Antennas] {-{-}{-} J[Whip/Monopole]}
    I {-{-}{-} K[Helical]}
    I {-{-}{-} L[PIFA/Patch]}
    style A fill:\#f9f,stroke:\#333
    style I fill:\#bbf,stroke:\#333
{Highlighting}
{Shaded}
\end{verbatim}
\end{center}

\textbf{Base Station Antennas:}

{\def\LTcaptype{none} % do not increment counter
\begin{longtable}[]{@{}
  >{\raggedright\arraybackslash}p{(\linewidth - 4\tabcolsep) * \real{0.3111}}
  >{\raggedright\arraybackslash}p{(\linewidth - 4\tabcolsep) * \real{0.3778}}
  >{\raggedright\arraybackslash}p{(\linewidth - 4\tabcolsep) * \real{0.3111}}@{}}
\toprule\noalign{}
\begin{minipage}[b]{\linewidth}\raggedright
Antenna Type
\end{minipage} & \begin{minipage}[b]{\linewidth}\raggedright
Characteristics
\end{minipage} & \begin{minipage}[b]{\linewidth}\raggedright
Applications
\end{minipage} \\
\midrule\noalign{}
\endhead
\bottomrule\noalign{}
\endlastfoot
\textbf{Omnidirectional} & - 360^\circ horizontal coverage- 6-12 dBi gain-
Vertical polarization- Collinear arrays & - Rural areas- Low traffic
density- Small cells \\
\textbf{Sectorized} & - 65-120^\circ sector coverage- 12-20 dBi gain-
Vertical/slant polarization- Panel design & - Urban/suburban areas-
Frequency reuse- High capacity networks \\
\textbf{Diversity Antennas} & - Multiple elements- Space/polarization
diversity- Reduced fading & - Multipath environments- High reliability
links \\
\textbf{Smart Antennas} & - Adaptive beamforming- Multiple elements-
15-25 dBi gain & - High capacity areas- Interference reduction- 4G/5G
systems \\
\end{longtable}
}

\textbf{Mobile Station Antennas:}

{\def\LTcaptype{none} % do not increment counter
\begin{longtable}[]{@{}
  >{\raggedright\arraybackslash}p{(\linewidth - 4\tabcolsep) * \real{0.3111}}
  >{\raggedright\arraybackslash}p{(\linewidth - 4\tabcolsep) * \real{0.3778}}
  >{\raggedright\arraybackslash}p{(\linewidth - 4\tabcolsep) * \real{0.3111}}@{}}
\toprule\noalign{}
\begin{minipage}[b]{\linewidth}\raggedright
Antenna Type
\end{minipage} & \begin{minipage}[b]{\linewidth}\raggedright
Characteristics
\end{minipage} & \begin{minipage}[b]{\linewidth}\raggedright
Applications
\end{minipage} \\
\midrule\noalign{}
\endhead
\bottomrule\noalign{}
\endlastfoot
\textbf{Whip/Monopole} & - External antenna- λ/4 length-
Omnidirectional- 2-3 dBi gain & - Vehicle-mounted phones- Older
handsets- Rural area devices \\
\textbf{Helical} & - Compact size- Good bandwidth- Flexible design- 0-2
dBi gain & - Portable radios- Early mobile phones \\
\textbf{PIFA (Planar Inverted-F)} & - Internal antenna- Compact size-
Multiband operation- 0-2 dBi gain & - Modern smartphones- Tablets- IoT
devices \\
\textbf{Patch/Microstrip} & - Low profile- Directional pattern- Dual
polarization- 5-8 dBi gain & - Data cards- Fixed wireless terminals-
High-speed data devices \\
\end{longtable}
}

\textbf{Key Considerations for Mobile Communication Antennas:}

\begin{enumerate}
\tightlist
\item
  \textbf{Base Station Requirements:}

  \begin{itemize}
  \tightlist
  \item
    High gain for coverage
  \item
    Focused beams for capacity
  \item
    Downtilt to control interference
  \item
    Diversity for multipath mitigation
  \item
    Weather resistance
  \end{itemize}
\item
  \textbf{Mobile Station Requirements:}

  \begin{itemize}
  \tightlist
  \item
    Small size and low profile
  \item
    Multiband operation
  \item
    Omnidirectional pattern
  \item
    SAR (Specific Absorption Rate) compliance
  \item
    Integration with device design
  \end{itemize}
\end{enumerate}

\end{solutionbox}
\begin{mnemonicbox}
``BOMBS-WHIP: Base Omni/Multi-Beam/Smart,
Whip/Helical/Inverted-F/Patch''

\end{mnemonicbox}

\end{document}
