\documentclass[10pt,a4paper]{article}

% content/resources/templates/preamble.tex
\usepackage[margin=0.6in]{geometry}
\author{Milav Dabgar}
\usepackage{amsmath,amssymb,amsthm}
\usepackage{booktabs}
\usepackage{multirow}
\usepackage{xcolor}
\usepackage{tcolorbox}
\tcbuselibrary{breakable,skins}
\usepackage[colorlinks=true,linkcolor=blue]{hyperref}
\usepackage{titlesec}
\usepackage{enumitem}
\usepackage{tikz}
\usepackage{pgfplots}
\usepackage{circuitikz}
\usepackage[version=4]{mhchem}
\usepackage{longtable}
\usepackage{array}
\usepackage{float}
\usepackage{caption}
\usepackage{listings}

\lstset{
  basicstyle=\small\ttfamily,
  breaklines=true,
  breakatwhitespace=false,
  postbreak=\mbox{\textcolor{red}{$\hookrightarrow$}\space},
  float=false,
  numbers=left,
  numberstyle=\tiny\color{gray},
  numbersep=10pt,
  xleftmargin=2em,
  keywordstyle=\color{blue},
  commentstyle=\color{green!60!black},
  stringstyle=\color{purple},
  backgroundcolor=\color{gray!5},
  showstringspaces=false,
  tabsize=2,
  captionpos=b,
  keepspaces=true,
  columns=flexible
}

\pgfplotsset{compat=1.18}
\usetikzlibrary{shapes,arrows,positioning,calc,patterns,decorations.pathmorphing,decorations.markings,arrows.meta}

% Color scheme
\definecolor{headcolor}{RGB}{0,102,204}
\definecolor{keycolor}{RGB}{220,20,60}
\definecolor{solutioncolor}{RGB}{34,139,34}
\definecolor{mnemoniccolor}{RGB}{148,0,211}
\definecolor{codecolor}{RGB}{0,0,100}

% Spacing
\setlength{\parskip}{3pt}
\setlist[itemize]{nosep}
\setlist[enumerate]{nosep}

% Title formatting
\titleformat{\section}{\Large\bfseries\color{headcolor}}{\thesection}{1em}{}
\titleformat{\subsection}{\large\bfseries\color{headcolor}}{\thesubsection}{1em}{}

% Pandoc tightlist compatibility
\providecommand{\tightlist}{%
  \setlength{\itemsep}{0pt}\setlength{\parskip}{0pt}}

% Pandoc longtable compatibility
\newcounter{none}
\def\thenone{}


% content/resources/templates/gujarati-boxes.tex
\usepackage{fontspec}
\usepackage{polyglossia}

% Set Gujarati as main language (document is primarily in Gujarati)
% Note: gloss-gujarati.ldf doesn't exist in polyglossia, but it will use hyphenation patterns
\setdefaultlanguage{gujarati}
\setotherlanguage{english}

% Configure Gujarati font properly
% Use Language=Default to prevent polyglossia from trying to add language-specific features
% that don't exist for Gujarati, which causes "empty feature" warnings
\newfontfamily\gujaratifont[Script=Gujarati,AutoFakeBold=2.5,AutoFakeSlant=0.3]{Noto Sans Gujarati}
\setmainfont[Script=Gujarati,AutoFakeBold=2.5,AutoFakeSlant=0.3]{Noto Sans Gujarati}
% Use Noto Sans Gujarati for monospace to support Gujarati in text
\setmonofont[Scale=0.9]{Noto Sans Gujarati}

% Configure English to use the same font
\newfontfamily\englishfont[Script=Gujarati,AutoFakeBold=2.5,AutoFakeSlant=0.3]{Noto Sans Gujarati}

% Translations for polyglossia
\gappto\captionsgujarati{
  \renewcommand{\tablename}{કોષ્ટક}
  \renewcommand{\figurename}{આકૃતિ}
}

% Helper for TikZ nodes to ensure Gujarati font
\newcommand{\gu}[1]{{\gujaratifont #1}}

% Custom environments
\newtcolorbox{solutionbox}{
    breakable,
    enhanced,
    colback=solutioncolor!5!white,
    colframe=solutioncolor!75!black,
    fonttitle=\bfseries,
    title=જવાબ
}

\newtcolorbox{solutionboxnobreak}{
 colback=solutioncolor!5!white,
 colframe=solutioncolor!75!black,
 fonttitle=\bfseries,
 title=જવાબ
}

\newtcolorbox{keyformula}{
 breakable,
 enhanced,
 colback=keycolor!5!white,
 colframe=keycolor!75!black,
 fonttitle=\bfseries,
 title=રાસાયણિક સમીકરણ/સૂત્ર
}

\newtcolorbox{mnemonicbox}{
 breakable,
 enhanced,
 colback=mnemoniccolor!5!white,
 colframe=mnemoniccolor!75!black,
 fonttitle=\bfseries,
 title=મેમરી ટ્રીક
}


\begin{document}

\begin{center}
{\Huge\bfseries\color{headcolor} Subject Name (Gujarati)}\\[5pt]
{\LARGE 4341106 -- Summer 2024}\\[3pt]
{\large Semester 1 Study Material}\\[3pt]
{\normalsize\textit{Detailed Solutions and Explanations}}
\end{center}

\vspace{10pt}

\subsection*{પ્રશ્ન 1(અ) [3
ગુણ]}\label{uxaaauxab0uxab6uxaa8-1uxa85-3-uxa97uxaa3}

\textbf{બીમ વિસ્તાર અને બીમની કાર્યક્ષમતા વ્યાખ્યાયિત કરો.}

\begin{solutionbox}

\textbf{બીમ વિસ્તાર}: એ ઘન કોણ છે જેના માધ્યમથી એન્ટેના દ્વારા વિકિરણિત તમામ
પાવર પસાર થશે જો રેડિએશન ઇન્ટેન્સિટી આ કોણ પર સમાન હોય અને મહત્તમ મૂલ્યની બરાબર
હોય.

\textbf{બીમ કાર્યક્ષમતા}: મુખ્ય બીમમાં રહેલી શક્તિનો એન્ટેના દ્વારા વિકિરણિત કુલ
શક્તિ સાથેનો ગુણોત્તર.

\textbf{આકૃતિ}:

\begin{center}
\textbf{Mermaid Diagram (Code)}
\begin{verbatim}
{Shaded}
{Highlighting}[]
graph LR
    A[બીમ વિસ્તાર] {-{-}{} B[વિકિરણિત પાવરનો{}br /{}મોટાભાગનો ઘન કોણ]}
    C[બીમ કાર્યક્ષમતા] {-{-}{} D[મુખ્ય બીમ પાવર/કુલ પાવર]}
    D {-{-}{} E[ઉચ્ચ કાર્યક્ષમતા = બેહતર એન્ટેના]}
{Highlighting}
{Shaded}
\end{verbatim}
\end{center}

\end{solutionbox}
\begin{mnemonicbox}
``BEAM: બેહતર કાર્યક્ષમતા આદર્શ મહત્તમ કામગીરી''

\end{mnemonicbox}
\subsection*{પ્રશ્ન 1(બ) [4
ગુણ]}\label{uxaaauxab0uxab6uxaa8-1uxaac-4-uxa97uxaa3}

\textbf{EM ક્ષેત્ર શું છે? સેન્ટર ફેડ ડાયપોલ માંથી તેના કિરણોત્સર્જનને સમજાવો.}

\begin{solutionbox}

EM ક્ષેત્ર એક ભૌતિક ક્ષેત્ર છે જે વિદ્યુત ચાર્જ વાળી વસ્તુઓ દ્વારા ઉત્પન્ન થાય છે અને ચાર્જ
કણો પર બળ સાથે અસર કરે છે.

\textbf{આકૃતિ}:

\begin{verbatim}
                        |
                        |
     E{-field            |            E{-}field}
    (vertical)          |           (vertical)
                        |
           ↑            |           ↑
           |   current  |           |
           |      ↓     |           |
     {-{-}{-}{-}{-}{-}+{-}{-}{-}{-}{-}{-}{-}{-}{-}{-}{-}{-}+{-}{-}{-}{-}{-}{-}{-}{-}{-}{-} dipole antenna}
           |      ↑     |           |
           |   current  |           |
           ↓            |           ↓
                        |
     H{-field            |            H{-}field}
    (circular)          |           (circular)
                        |
\end{verbatim}

\begin{itemize}
\tightlist
\item
  \textbf{ઇલેક્ટ્રિક ફિલ્ડ}: એન્ટેના અક્ષને લંબરૂપે, એન્ટેનાના છેડા પર મહત્તમ
\item
  \textbf{ચુંબકીય ક્ષેત્ર}: એન્ટેના અક્ષની આસપાસ વર્તુળાકાર
\item
  \textbf{રેડિએશન પદ્ધતિ}: અલ્ટરનેટિંગ કરંટ સમય-ભિન્ન ક્ષેત્રો બનાવે છે
\item
  \textbf{ફિલ્ડ વર્તન}: નિયર ફિલ્ડ (રિએક્ટિવ) \rightarrow ઇન્ટરમીડિયેટ \rightarrow ફાર ફિલ્ડ
  (રેડિએટિંગ)
\end{itemize}

\end{solutionbox}
\begin{mnemonicbox}
``CERD: કરંટ એક્સાઇટ્સ રેડિએટિંગ ડાયપોલ''

\end{mnemonicbox}
\subsection*{પ્રશ્ન 1(ક) [7
ગુણ]}\label{uxaaauxab0uxab6uxaa8-1uxa95-7-uxa97uxaa3}

\textbf{પોઈન્ટિંગ વેક્ટરનો ઉપયોગ કરીને પ્રાથમિક ડાયપોલ દ્વારા વિકિરણ થતી શક્તિ
સમજાવો.}

\begin{solutionbox}

પ્રાથમિક ડાયપોલ દ્વારા વિકિરણિત શક્તિની ગણતરી પોઈન્ટિંગ વેક્ટર દ્વારા થઈ શકે છે, જે
પાવર ફ્લો ઘનતાનું પ્રતિનિધિત્વ કરે છે.


{\def\LTcaptype{none} % do not increment counter
\vspace{-5pt}
\captionof{table}{પોઈન્ટિંગ વેક્ટર વિશ્લેષણના મુખ્ય પગલાં}
\vspace{-10pt}
\begin{longtable}[]{@{}ll@{}}
\toprule\noalign{}
પગલું & વર્ણન \\
\midrule\noalign{}
\endhead
\bottomrule\noalign{}
\endlastfoot
1 & E-ફિલ્ડ ઘટકોની ગણતરી કરો (Eθ, Eφ) \\
2 & H-ફિલ્ડ ઘટકોની ગણતરી કરો (Hθ, Hφ) \\
3 & પોઈન્ટિંગ વેક્ટર નક્કી કરો: P = E \times H \\
4 & ગોળાકાર સપાટી પર ઇન્ટિગ્રેટ કરો \\
\end{longtable}
}

\textbf{આકૃતિ}:

\begin{center}
\textbf{Mermaid Diagram (Code)}
\begin{verbatim}
{Shaded}
{Highlighting}[]
graph LR
    A[પોઈન્ટિંગ વેક્ટર{br /{}P = E  H] {-}{-}{} B[સમય{-}સરેરાશ{}br /{}પાવર ઘનતા]}
    B {-{-}{} C[સ્ફિયર પર ઇન્ટિગ્રેટ{}br /{}P = ·ds]}
    C {-{-}{} D[વિકિરણિત પાવર{}br /{}P = 80π^{2}I^{2}l^{2}/λ^{2}]}
{Highlighting}
{Shaded}
\end{verbatim}
\end{center}

\begin{itemize}
\tightlist
\item
  \textbf{ઇલેક્ટ્રિક ફિલ્ડ}: E = (jη I_{0}dl/2λr) sin θ e^{-}ʲᵏʳ
\item
  \textbf{ચુંબકીય ક્ષેત્ર}: H = (j I_{0}dl/2λr) sin θ e^{-}ʲᵏʳ
\item
  \textbf{પોઈન્ટિંગ વેક્ટર}: P = E \times H* =
  (η\textbar I_{0}\textbar^{2}\textbar dl\textbar^{2}/8π^{2}r^{2}) sin^{2} θ
\item
  \textbf{કુલ પાવર}: P = (η\textbar I_{0}\textbar^{2}\textbar dl\textbar^{2}/12π)
  = 80π^{2}I^{2}l^{2}/λ^{2}
\end{itemize}

\end{solutionbox}
\begin{mnemonicbox}
``PEHP: પોઈન્ટિંગ એક્સપ્લેન્સ હાઉ પાવર પ્રોપેગેટ્સ''

\end{mnemonicbox}
\subsection*{પ્રશ્ન 1(ક) અથવા [7
ગુણ]}\label{uxaaauxab0uxab6uxaa8-1uxa95-uxa85uxaa5uxab5-7-uxa97uxaa3}

\textbf{એન્ટેના, રેડિયેશન પેટર્ન, ડાયરેક્ટિવિટી, ગેઇન, FBR, આઇસોટ્રોપિક રેડિએટર અને
ઇફેક્ટિવ એપર્ચર વ્યાખ્યાયિત કરો.}

\begin{solutionbox}


{\def\LTcaptype{none} % do not increment counter
\vspace{-5pt}
\captionof{table}{મુખ્ય એન્ટેના પેરામીટર્સ}
\vspace{-10pt}
\begin{longtable}[]{@{}
  >{\raggedright\arraybackslash}p{(\linewidth - 2\tabcolsep) * \real{0.4783}}
  >{\raggedright\arraybackslash}p{(\linewidth - 2\tabcolsep) * \real{0.5217}}@{}}
\toprule\noalign{}
\begin{minipage}[b]{\linewidth}\raggedright
પેરામીટર
\end{minipage} & \begin{minipage}[b]{\linewidth}\raggedright
વ્યાખ્યા
\end{minipage} \\
\midrule\noalign{}
\endhead
\bottomrule\noalign{}
\endlastfoot
એન્ટેના & એક ઉપકરણ જે ગાઇડેડ ઇલેક્ટ્રોમેગ્નેટિક વેવ્સને ફ્રી-સ્પેસ વેવ્સમાં અને વિપરીત રૂપાંતર
કરે છે \\
રેડિએશન પેટર્ન & સ્પેસ કોઓર્ડિનેટ્સના ફંક્શન તરીકે રેડિએશન પ્રોપર્ટીની ગ્રાફિકલ રજૂઆત \\
ડાયરેક્ટિવિટી & અપાયેલી દિશામાં રેડિએશન ઇન્ટેન્સિટીનો સરેરાશ રેડિએશન ઇન્ટેન્સિટી
સાથેનો ગુણોત્તર \\
ગેઇન & રેડિએશન ઇન્ટેન્સિટીનો સમાન ઇનપુટ પાવર સાથે આઇસોટ્રોપિક સ્રોતના ઇન્ટેન્સિટી
સાથેનો ગુણોત્તર \\
FBR (ફ્રન્ટ-ટુ-બેક રેશિયો) & ફોરવર્ડ દિશામાં વિકિરણિત શક્તિનો બેકવર્ડ દિશામાં
વિકિરણિત શક્તિ સાથેનો ગુણોત્તર \\
આઇસોટ્રોપિક રેડિએટર & સૈદ્ધાંતિક એન્ટેના જે બધી દિશામાં સમાન રીતે વિકિરણ કરે છે \\
ઇફેક્ટિવ એપર્ચર & એન્ટેના દ્વારા પ્રાપ્ત શક્તિનો આવતી પાવર ઘનતા સાથેનો ગુણોત્તર \\
\end{longtable}
}

\textbf{આકૃતિ}:

\begin{verbatim}
pie
    title "એન્ટેના પરફોર્મન્સ ફેક્ટર્સ"
    "ડાયરેક્ટિવિટી" : 25
    "ગેઇન" : 25
    "ઇફેક્ટિવ એપર્ચર" : 20
    "રેડિએશન પેટર્ન" : 15
    "FBR" : 15
\end{verbatim}

\end{solutionbox}
\begin{mnemonicbox}
``DIAGRAM: ડાયરેક્ટિવિટી ઇમ્પ્રુવ્સ એન્ટેના ગેઇન, રેડિએશન એન્ડ
મોર''

\end{mnemonicbox}
\subsection*{પ્રશ્ન 2(અ) [3
ગુણ]}\label{uxaaauxab0uxab6uxaa8-2uxa85-3-uxa97uxaa3}

\textbf{પેટર્ન ગુણાકારનો સિદ્ધાંત સમજાવો.}

\begin{solutionbox}

પેટર્ન ગુણાકાર સિદ્ધાંત જણાવે છે કે એરેનું રેડિએશન પેટર્ન એલિમેન્ટ પેટર્ન અને એરે ફેક્ટરનું
ગુણનફળ હોય છે.

\textbf{આકૃતિ}:

\begin{center}
\textbf{Mermaid Diagram (Code)}
\begin{verbatim}
{Shaded}
{Highlighting}[]
graph LR
    A[એરે પેટર્ન] {-{-}{} B["એલિમેન્ટ પેટર્ન  એરે ફેક્ટર"]}
    B {-{-}{} C[કુલ ફિલ્ડ પેટર્ન]}
    C {-{-}{} D[ડાયરેક્ટિવિટી એન્હેંસમેન્ટ]}
{Highlighting}
{Shaded}
\end{verbatim}
\end{center}

\begin{itemize}
\tightlist
\item
  \textbf{એલિમેન્ટ પેટર્ન}: સિંગલ એલિમેન્ટનું રેડિએશન પેટર્ન
\item
  \textbf{એરે ફેક્ટર}: એલિમેન્ટ્સની ગોઠવણીને કારણે આવતું પેટર્ન
\item
  \textbf{પરિણામ}: વધુ તીક્ષ્ણ બીમ, વધુ ડાયરેક્ટિવિટી
\end{itemize}

\end{solutionbox}
\begin{mnemonicbox}
``PEAM: પેટર્ન ઈક્વલ્સ એરે ટાઇમ્સ એલિમેન્ટ મેથડ''

\end{mnemonicbox}
\subsection*{પ્રશ્ન 2(બ) [4
ગુણ]}\label{uxaaauxab0uxab6uxaa8-2uxaac-4-uxa97uxaa3}

\textbf{લૂપ એન્ટેના દોરો અને સમજાવો.}

\begin{solutionbox}

લૂપ એન્ટેના એક બંધ સર્કિટ એન્ટેના છે જેમાં તારના એક અથવા વધુ પૂર્ણ આંટા હોય છે.

\textbf{આકૃતિ}:

\begin{verbatim}
      ┌───────────┐
      │           │
      │           │
      │           │
  feed│           │
   ┌──┴──┐        │
   │     │        │
   └─────┘        │
      │           │
      │           │
      └───────────┘
\end{verbatim}

\begin{itemize}
\tightlist
\item
  \textbf{નાનો લૂપ}: પરિઘ \textless{} λ/10, ફિગર-8 પેટર્ન
\item
  \textbf{મોટો લૂપ}: પરિઘ \approx λ, સપાટીને લંબરૂપે મહત્તમ રેડિએશન
\item
  \textbf{ઉપયોગો}: દિશા શોધવી, AM રેડિયો રિસેપ્શન
\item
  \textbf{રેડિએશન રેઝિસ્ટન્સ}: નાના લૂપ માટે (પરિઘ/λ)^{4} ના પ્રમાણમાં
\end{itemize}

\end{solutionbox}
\begin{mnemonicbox}
``LOOP: લો આઉટપુટ, ઓરિએન્ટેશન પ્રિસાઇઝ''

\end{mnemonicbox}
\subsection*{પ્રશ્ન 2(ક) [7
ગુણ]}\label{uxaaauxab0uxab6uxaa8-2uxa95-7-uxa97uxaa3}

\textbf{યાગી-ઉડા એન્ટેના ડિઝાઇન કરો અને તેને સમજાવો.}

\begin{solutionbox}

યાગી-ઉડા એ એક દિશાત્મક એન્ટેના છે જેમાં ડ્રાઇવન એલિમેન્ટ, રિફ્લેક્ટર અને ડાયરેક્ટર્સ હોય
છે.


{\def\LTcaptype{none} % do not increment counter
\vspace{-5pt}
\captionof{table}{યાગી-ઉડા એન્ટેના ડિઝાઇન ગાઇડલાઇન્સ}
\vspace{-10pt}
\begin{longtable}[]{@{}lll@{}}
\toprule\noalign{}
એલિમેન્ટ & લંબાઈ & ડ્રાઇવન એલિમેન્ટથી અંતર \\
\midrule\noalign{}
\endhead
\bottomrule\noalign{}
\endlastfoot
રિફ્લેક્ટર & 0.5λ \times 1.05 & 0.15λ - 0.25λ \\
ડ્રાઇવન એલિમેન્ટ & 0.5λ & સંદર્ભ બિંદુ \\
ડાયરેક્ટર 1 & 0.5λ \times 0.95 & 0.1λ - 0.15λ \\
ડાયરેક્ટર 2 & 0.5λ \times 0.92 & 0.2λ - 0.3λ \\
વધારાના ડાયરેક્ટર્સ & ઘટતા & 0.3λ - 0.4λ \\
\end{longtable}
}

\textbf{આકૃતિ}:

\begin{verbatim}
          Director 2     Director 1     Driven      Reflector
             ┌┐             ┌┐          Element        ┌┐
             ││             ││            ┌┐           ││
             ││             ││            ││           ││
             ││             ││            ││           ││
     ────────┴┴─────────────┴┴────────────┴┴───────────┴┴────────►
             │{{-}{-}0.15λ{-}{-}│{-}{-}0.15λ{-}{-}│{-}{-}0.25λ{-}{-}│      Radiation}
             │{{-}{-}{-}{-}{-}{-}{-}{-}{-}{-}{-} Boom Length {-}{-}{-}{-}{-}{-}{-}{-}{-}{-}│      Direction}
\end{verbatim}

\begin{itemize}
\tightlist
\item
  \textbf{કાર્ય}: રિફ્લેક્ટર સિગ્નલને પરાવર્તિત કરે છે, ડાયરેક્ટર્સ તેને આગળ માર્ગદર્શન
  આપે છે
\item
  \textbf{ગેઇન}: ડાયરેક્ટર્સની સંખ્યા સાથે વધે છે (ઘટતા વળતર સાથે)
\item
  \textbf{ઇમ્પિડન્સ}: 20-30 ઓહ્મ (સામાન્ય રીતે બેલન સાથે મેચ કરાયેલ)
\item
  \textbf{ઉપયોગો}: TV રિસેપ્શન, પોઇન્ટ-ટુ-પોઇન્ટ કોમ્યુનિકેશન
\end{itemize}

\end{solutionbox}
\begin{mnemonicbox}
``YARD: યાગી એચિવ્સ રેડિકલ ડાયરેક્ટિવિટી''

\end{mnemonicbox}
\subsection*{પ્રશ્ન 2(અ) અથવા [3
ગુણ]}\label{uxaaauxab0uxab6uxaa8-2uxa85-uxa85uxaa5uxab5-3-uxa97uxaa3}

\textbf{બ્રોડ ફાયર અને એન્ડ ફાયર એરે એન્ટેનાની સરખામણી કરો.}

\begin{solutionbox}


{\def\LTcaptype{none} % do not increment counter
\vspace{-5pt}
\captionof{table}{બ્રોડ સાઇડ અને એન્ડ ફાયર એરેની સરખામણી}
\vspace{-10pt}
\begin{longtable}[]{@{}lll@{}}
\toprule\noalign{}
પેરામીટર & બ્રોડ સાઇડ એરે & એન્ડ ફાયર એરે \\
\midrule\noalign{}
\endhead
\bottomrule\noalign{}
\endlastfoot
મહત્તમ રેડિએશનની દિશા & એરે એક્સિસને લંબરૂપે & એરે એક્સિસ સાથે \\
ફેઝ તફાવત & 0^\circ & 180^\circ \pm βd \\
બીમ પહોળાઈ & સાંકડી & પહોળી \\
ડાયરેક્ટિવિટી & ઉચ્ચ & નીચી \\
ઉપયોગો & બ્રોડકાસ્ટિંગ & પોઇન્ટ-ટુ-પોઇન્ટ લિંક્સ \\
\end{longtable}
}

\textbf{આકૃતિ}:

\begin{center}
\textbf{Mermaid Diagram (Code)}
\begin{verbatim}
{Shaded}
{Highlighting}[]
graph LR
    A[એરે એન્ટેના] {-{-}{} B[બ્રોડ સાઇડ]}
    A {-{-}{} C[એન્ડ ફાયર]}
    B {-{-}{} D[મહત્તમ રેડિએશન એરે એક્સિસને{}br /{}લંબરૂપે]}
    C {-{-}{} E[મહત્તમ રેડિએશન એરે એક્સિસ{}br /{}સાથે]}
{Highlighting}
{Shaded}
\end{verbatim}
\end{center}

\end{solutionbox}
\begin{mnemonicbox}
``BEPS: બ્રોડસાઇડ એમિટ્સ પર્પેન્ડિક્યુલરલી, સાઇડવેઝ''

\end{mnemonicbox}
\subsection*{પ્રશ્ન 2(બ) અથવા [4
ગુણ]}\label{uxaaauxab0uxab6uxaa8-2uxaac-uxa85uxaa5uxab5-4-uxa97uxaa3}

\textbf{ફોલ્ડેડ ડિપોલ એન્ટેના દોરો અને સમજાવો.}

\begin{solutionbox}

ફોલ્ડેડ ડિપોલમાં અર્ધ-તરંગ લંબાઈનો ડિપોલ હોય છે જેના છેડા પાછા વાળીને જોડાયેલા હોય
છે, જે એક સાંકડો લૂપ બનાવે છે.

\textbf{આકૃતિ}:

\begin{verbatim}
       λ/2
    ┌───────┐
    │       │
    │       │
    │       │
feed│       │
 ┌──┴──┐    │
 │     │    │
 └─────┘    │
    │       │
    │       │
    └───────┘
\end{verbatim}

\begin{itemize}
\tightlist
\item
  \textbf{ઇમ્પિડન્સ}: સ્ટાન્ડર્ડ ડિપોલ કરતાં 4 ગણો વધારે (\approx300Ω)
\item
  \textbf{બેન્ડવિડ્થ}: સરળ ડિપોલ કરતાં વધુ પહોળી
\item
  \textbf{ઉપયોગો}: TV એન્ટેના, FM રિસીવિંગ એન્ટેના
\item
  \textbf{ફાયદો}: ઓછી નોઇઝ સંવેદનશીલતા
\end{itemize}

\end{solutionbox}
\begin{mnemonicbox}
``FIBER: ફોલ્ડેડ ઇમ્પિડન્સ બૂસ્ટર એન્હેંસિસ રિસેપ્શન''

\end{mnemonicbox}
\subsection*{પ્રશ્ન 2(ક) અથવા [7
ગુણ]}\label{uxaaauxab0uxab6uxaa8-2uxa95-uxa85uxaa5uxab5-7-uxa97uxaa3}

\textbf{બિન-રેઝોનન્ટ એન્ટેનાના નામ આપો અને કોઈપણ એકને તેની રેડિએશન પેટર્ન સાથે
વિગતવાર સમજાવો.}

\begin{solutionbox}

બિન-રેઝોનન્ટ એન્ટેનામાં રોમ્બિક, V એન્ટેના, ટર્મિનેટેડ ફોલ્ડેડ ડિપોલ, બેવરેજ અને
લોંગ-વાયર એન્ટેનાનો સમાવેશ થાય છે.

\textbf{રોમ્બિક એન્ટેના વિગતવાર:}

\textbf{આકૃતિ}:

\begin{verbatim}
                    ┌───────┐
                   /         {}
                  /           {}
                 /             {}
                /               {}
               /                 {}
              /                   {}
             /                     {}
            /                       {}
           /                         {}
    ┌─────┘                           └─────┐
    │                                       │
 ───┴───                                 ───┴───
 Feeder                               Terminating
                                       Resistor
\end{verbatim}


{\def\LTcaptype{none} % do not increment counter
\vspace{-5pt}
\captionof{table}{રોમ્બિક એન્ટેનાની ખાસિયતો}
\vspace{-10pt}
\begin{longtable}[]{@{}ll@{}}
\toprule\noalign{}
પેરામીટર & વર્ણન \\
\midrule\noalign{}
\endhead
\bottomrule\noalign{}
\endlastfoot
સ્ટ્રક્ચર & ચાર લાંબા તાર રોમ્બસ આકારમાં ગોઠવેલા \\
ટર્મિનેશન & દૂરના છેડે રેઝિસ્ટિવ લોડ (બિન-રેઝોનન્ટ) \\
ડાયરેક્ટિવિટી & ઉચ્ચ (8-15 dB) \\
ફ્રિક્વન્સી રેન્જ & વિશાળ બેન્ડવિડ્થ (મલ્ટી-ઓક્ટેવ) \\
રેડિએશન પેટર્ન & એકદિશીય, શંકુ આકારનું \\
ઉપયોગો & HF પોઇન્ટ-ટુ-પોઇન્ટ કોમ્યુનિકેશન \\
\end{longtable}
}

\begin{itemize}
\tightlist
\item
  \textbf{ફાયદા}: ઉચ્ચ ગેઈન, વિશાળ બેન્ડવિડ્થ, સરળ બનાવટ
\item
  \textbf{નુકસાન}: મોટા ભૌતિક કદ, ટર્મિનેટિંગ રેઝિસ્ટરમાં પાવર નુકસાન
\item
  \textbf{પેટર્ન}: મુખ્ય લોબ રોમ્બસની મુખ્ય અક્ષ સાથે
\end{itemize}

\end{solutionbox}
\begin{mnemonicbox}
``RHOMBIC: વિશ્વસનીય ઉચ્ચ-આઉટપુટ મલ્ટી-બેન્ડ અદ્ભુત
કોમ્યુનિકેશન''

\end{mnemonicbox}
\subsection*{પ્રશ્ન 3(અ) [3
ગુણ]}\label{uxaaauxab0uxab6uxaa8-3uxa85-3-uxa97uxaa3}

\textbf{વિવિધ રેઝોનન્ટ વાયર એન્ટેનાની રેડિએશન પેટર્નની તુલના કરો.}

\begin{solutionbox}


{\def\LTcaptype{none} % do not increment counter
\vspace{-5pt}
\captionof{table}{રેઝોનન્ટ વાયર એન્ટેનાની રેડિએશન પેટર્ન}
\vspace{-10pt}
\begin{longtable}[]{@{}llll@{}}
\toprule\noalign{}
એન્ટેના પ્રકાર & પેટર્ન આકાર & ડાયરેક્ટિવિટી & પોલરાઈઝેશન \\
\midrule\noalign{}
\endhead
\bottomrule\noalign{}
\endlastfoot
હાફ-વેવ ડિપોલ & ફિગર-8 (ડોનટ) & 2.15 dBi & લિનિયર \\
ફુલ-વેવ ડિપોલ & ચાર-લોબ્ડ & 3.8 dBi & લિનિયર \\
3λ/2 ડિપોલ & છ-લોબ્ડ & 4.2 dBi & લિનિયર \\
2λ ડિપોલ & આઠ-લોબ્ડ & 4.5 dBi & લિનિયર \\
\end{longtable}
}

\textbf{આકૃતિ}:

\begin{center}
\textbf{Mermaid Diagram (Code)}
\begin{verbatim}
{Shaded}
{Highlighting}[]
graph TD
    A[રેઝોનન્ટ વાયર એન્ટેના] {-{-}{} B[હાફ{-}વેવ ડિપોલ{}br /{}ફિગર{-}8 પેટર્ન]}
    A {-{-}{} C[ફુલ{-}વેવ ડિપોલ{}br /{}ચાર{-}લોબ્ડ પેટર્ન]}
    A {-{-}{} D[મલ્ટી{-}વેવલેન્થ ડિપોલ{}br /{}મલ્ટી{-}લોબ્ડ પેટર્ન]}
{Highlighting}
{Shaded}
\end{verbatim}
\end{center}

\end{solutionbox}
\begin{mnemonicbox}
``MOLD: વધુ તરંગલંબાઈથી ઘણા ડાયરેક્ટિવિટી લોબ્સ બને છે''

\end{mnemonicbox}
\subsection*{પ્રશ્ન 3(બ) [4
ગુણ]}\label{uxaaauxab0uxab6uxaa8-3uxaac-4-uxa97uxaa3}

\textbf{V અને ઇન્વર્ટેડ V એન્ટેના રેડીએશન પેટર્ન સાથે દોરો.}

\begin{solutionbox}

\textbf{આકૃતિ: V-એન્ટેના}

\begin{verbatim}
        
        
      /{}
     /  {}
    /    {}
   /      {}
  /        {}
 /          {}
Feed        Feed
Point       Point
    
Radiation Pattern: Bidirectional along axis
\end{verbatim}

\textbf{આકૃતિ: ઇન્વર્ટેડ V-એન્ટેના}

\begin{verbatim}
         Feed
         Point
           |
           V
          / {}
         /   {}
        /     {}
       /       {}
      /         {}
     Ground     Ground
     
Radiation Pattern: Omnidirectional with slight elevation
\end{verbatim}

\begin{itemize}
\tightlist
\item
  \textbf{V-એન્ટેના}: V-આકારમાં બે તાર, દ્વિ-દિશાત્મક પેટર્ન
\item
  \textbf{ઇન્વર્ટેડ V}: હાફ-વેવ ડિપોલ જેના આર્મ્સ નીચેની તરફ ઢળતા, ઓમ્નીડાયરેક્શનલ
\item
  \textbf{ઉપયોગો}: એમેચ્યોર રેડિયો, FM રિસેપ્શન
\item
  \textbf{ફાયદા}: સરળ, લવચીક ઇન્સ્ટોલેશન વિકલ્પો
\end{itemize}

\end{solutionbox}
\begin{mnemonicbox}
``VIPS: V-આકાર પેટર્ન પસંદગીમાં સુધારો કરે છે''

\end{mnemonicbox}
\subsection*{પ્રશ્ન 3(ક) [7
ગુણ]}\label{uxaaauxab0uxab6uxaa8-3uxa95-7-uxa97uxaa3}

\textbf{મોર્સ કોડ અને પ્રેક્ટિસ ઓસિલેટર સમજાવો.}

\begin{solutionbox}

મોર્સ કોડ એ ડોટ્સ અને ડેશનાં પ્રમાણિત ક્રમનો ઉપયોગ કરીને ટેક્સ્ટ ટ્રાન્સમિટ કરવાની એક
પદ્ધતિ છે.


{\def\LTcaptype{none} % do not increment counter
\vspace{-5pt}
\captionof{table}{મૂળભૂત મોર્સ કોડ તત્વો}
\vspace{-10pt}
\begin{longtable}[]{@{}lll@{}}
\toprule\noalign{}
તત્વ & સમય & ધ્વનિ \\
\midrule\noalign{}
\endhead
\bottomrule\noalign{}
\endlastfoot
ડોટ (.) & 1 યુનિટ & ટૂંકો બીપ \\
ડેશ (-) & 3 યુનિટ & લાંબો બીપ \\
તત્વો વચ્ચે અંતર & 1 યુનિટ & ટૂંકી શાંતિ \\
અક્ષરો વચ્ચે અંતર & 3 યુનિટ & મધ્યમ શાંતિ \\
શબ્દો વચ્ચે અંતર & 7 યુનિટ & લાંબી શાંતિ \\
\end{longtable}
}

\textbf{આકૃતિ: સરળ મોર્સ કોડ પ્રેક્ટિસ ઓસિલેટર}

\begin{verbatim}
      +9V
       |
       R1
       |
      ┌┴┐
 C1   │ │                    Speaker
┌──┬──┤8├──┬────┬─────────────┬──┐
│  │  │5│  │    │             │  │
│  │  │5│  │    R2            C2 │
│  │  │5│  │    │             │  │
│  │  └┬┘  │    │             │  │
│  │   │   │    │             │  │
└──┴───┴───┴────┴─────────────┴──┘
  Key        Ground
\end{verbatim}

\begin{itemize}
\tightlist
\item
  \textbf{ઘટકો}: 555 ટાઇમર, રેઝિસ્ટર્સ, કેપેસિટર્સ, કી, સ્પીકર
\item
  \textbf{કાર્ય}: કી બંધ થવાથી સર્કિટ પૂર્ણ થાય છે, ઓસિલેશન બને છે
\item
  \textbf{ફ્રિક્વન્સી}: સામાન્ય રીતે 600-800 Hz (R2 સાથે એડજસ્ટેબલ)
\item
  \textbf{ઉપયોગો}: હેમ રેડિયો ટ્રેનિંગ, ઇમરજન્સી કોમ્યુનિકેશન
\end{itemize}

\end{solutionbox}
\begin{mnemonicbox}
``TEMPO: ટાઇમિંગ એલિમેન્ટ્સ મેક પરફેક્ટ ઓસિલેશન''

\end{mnemonicbox}
\subsection*{પ્રશ્ન 3(અ) અથવા [3
ગુણ]}\label{uxaaauxab0uxab6uxaa8-3uxa85-uxa85uxaa5uxab5-3-uxa97uxaa3}

\textbf{માઈક્રોસ્ટ્રિપ પેચ એન્ટેના દોરો અને સમજાવો.}

\begin{solutionbox}

માઈક્રોસ્ટ્રિપ પેચ એન્ટેનામાં ગ્રાઉન્ડેડ સબસ્ટ્રેટ પર ધાતુનો પેચ હોય છે.

\textbf{આકૃતિ}:

\begin{verbatim}
    ┌───────────────┐  {-{-} Patch (metal)}
    │               │
    │               │  Thickness
    │               │  ↕  
════════════════════════ {-{-} Substrate}
    |               |
    |               |  {-{-} Ground plane}
    └───────────────┘
    
    ↑               ↑
    Feed            Radiation
    point
\end{verbatim}

\begin{itemize}
\tightlist
\item
  \textbf{સ્ટ્રક્ચર}: ડાયલેક્ટ્રિક સબસ્ટ્રેટ પર ગ્રાઉન્ડ પ્લેન સાથે ધાતુનો પેચ
\item
  \textbf{ફાયદા}: ઓછી પ્રોફાઇલ, હળવું વજન, સરળ ઉત્પાદન, અનુરૂપ
\item
  \textbf{નુકસાન}: સાંકડી બેન્ડવિડ્થ, ઓછી કાર્યક્ષમતા, ઓછી પાવર હેન્ડલિંગ
\item
  \textbf{ઉપયોગો}: મોબાઇલ ડિવાઇસિસ, RFID, સેટેલાઇટ કોમ્યુનિકેશન
\end{itemize}

\end{solutionbox}
\begin{mnemonicbox}
``MAPS: માઈક્રોસ્ટ્રિપ એન્ટેના પેચિસ આર સિમ્પલ''

\end{mnemonicbox}
\subsection*{પ્રશ્ન 3(બ) અથવા [4
ગુણ]}\label{uxaaauxab0uxab6uxaa8-3uxaac-uxa85uxaa5uxab5-4-uxa97uxaa3}

\textbf{હોર્ન એન્ટેના દોરો અને સમજાવો.}

\begin{solutionbox}

હોર્ન એન્ટેના એ ફ્લેર્ડ ઓપન એન્ડ સાથેનો વેવગાઇડ છે જે રેડિયો વેવ્સને એક બીમમાં નિર્દેશિત
કરે છે.

\textbf{આકૃતિ}:

\begin{verbatim}
            ┌───────────┐
            │           │
            │           │
       ┌────┤           │
       │    │           │
       │    │           │
Feed   │    │           │
point  │    │           │
       │    │           │
       └────┤           │
            │           │
            │           │
            └───────────┘
         Waveguide       Horn
\end{verbatim}

\begin{itemize}
\tightlist
\item
  \textbf{પ્રકારો}: E-પ્લેન, H-પ્લેન, પિરામિડલ, કોનિકલ
\item
  \textbf{ફ્રિક્વન્સી રેન્જ}: માઇક્રોવેવ (1-20 GHz)
\item
  \textbf{ફાયદા}: ઉચ્ચ ગેઇન, વિશાળ બેન્ડવિડ્થ, ઓછો VSWR
\item
  \textbf{ઉપયોગો}: સેટેલાઇટ કોમ્યુનિકેશન, રડાર, રેડિયો એસ્ટ્રોનોમી
\end{itemize}

\end{solutionbox}
\begin{mnemonicbox}
``HEWB: હોર્ન્સ એન્હેન્સ વેવગાઇડ બીમવિડ્થ''

\end{mnemonicbox}
\subsection*{પ્રશ્ન 3(ક) અથવા [7
ગુણ]}\label{uxaaauxab0uxab6uxaa8-3uxa95-uxa85uxaa5uxab5-7-uxa97uxaa3}

\textbf{પેરાબોલિક રિફ્લેક્ટર એન્ટેના માટે વિવિધ ફીડ સિસ્ટમની યાદી બનાવો અને કોઈપણ
એકને સમજાવો.}

\begin{solutionbox}


{\def\LTcaptype{none} % do not increment counter
\vspace{-5pt}
\captionof{table}{પેરાબોલિક રિફ્લેક્ટર ફીડ સિસ્ટમ્સ}
\vspace{-10pt}
\begin{longtable}[]{@{}lll@{}}
\toprule\noalign{}
ફીડ સિસ્ટમ & પોઝિશન & ખાસિયતો \\
\midrule\noalign{}
\endhead
\bottomrule\noalign{}
\endlastfoot
ફ્રન્ટ ફીડ & ફોકસ પર, ડિશની સામે & સરળ, થોડું બ્લોકેજ \\
કેસેગ્રેન & સેકન્ડરી રિફ્લેક્ટર સાથે ડિશના કેન્દ્રમાં ફીડ & ઘટાડેલ નોઇઝ, કોમ્પેક્ટ \\
ગ્રેગોરિયન & સેકન્ડરી કોન્કેવ રિફ્લેક્ટર & બેહતર ગેઇન, મોટું કદ \\
ઓફસેટ ફીડ & મુખ્ય અક્ષથી ઓફસેટ ફીડ & કોઈ બ્લોકેજ નહીં, એસિમેટ્રિક \\
વેવગાઇડ ફીડ & ફોકસ પર ડાયરેક્ટ વેવગાઇડ & સરળ, સીમિત લવચીકતા \\
\end{longtable}
}

\textbf{ફ્રન્ટ ફીડ સિસ્ટમ (વિગતવાર):}

\textbf{આકૃતિ}:

\begin{center}
\textbf{Mermaid Diagram (Code)}
\begin{verbatim}
{Shaded}
{Highlighting}[]
graph LR
    A[પેરાબોલિક રિફ્લેક્ટર] {-{-}{-} B[ફોકલ પોઇન્ટ]}
    B {-{-}{-} C[ફીડ હોર્ન]}
    C {-{-}{-} D[વેવગાઇડ/કોએક્સ]}
    D {-{-}{-} E[રિસીવર/ટ્રાન્સમિટર]}
{Highlighting}
{Shaded}
\end{verbatim}
\end{center}

\begin{itemize}
\tightlist
\item
  \textbf{કાર્ય}: ફોકલ પોઇન્ટ પર ફીડ મુકાય છે, રિફ્લેક્ટરને પ્રકાશિત કરે છે
\item
  \textbf{ફાયદા}: સરળ ડિઝાઇન, સરળ એલાઇનમેન્ટ, મહત્તમ કાર્યક્ષમતા
\item
  \textbf{નુકસાન}: ફીડ અને સપોર્ટ સ્ટ્રક્ચર એપર્ચરનો ભાગ અવરોધે છે
\item
  \textbf{ઉપયોગો}: સેટેલાઇટ ડિશ, રેડિયો ટેલિસ્કોપ, રડાર
\end{itemize}

\end{solutionbox}
\begin{mnemonicbox}
``FACTS: ફોકસ્ડ એપર્ચર કેપ્ચર્સ ટ્રાન્સમિટેડ સિગ્નલ્સ''

\end{mnemonicbox}
\subsection*{પ્રશ્ન 4(અ) [3
ગુણ]}\label{uxaaauxab0uxab6uxaa8-4uxa85-3-uxa97uxaa3}

\textbf{HAM રેડિયોના કાર્યકારી સિદ્ધાંતને સમજાવો.}

\begin{solutionbox}

HAM રેડિયો (એમેચ્યોર રેડિયો) બિન-વ્યાવસાયિક સંચાર માટે નિયુક્ત ફ્રિક્વન્સી બેન્ડ પર
કાર્ય કરે છે.

\textbf{આકૃતિ}:

\begin{center}
\textbf{Mermaid Diagram (Code)}
\begin{verbatim}
{Shaded}
{Highlighting}[]
graph LR
    A[ટ્રાન્સમીટર] {-{-}{} B[એન્ટેના]}
    B {-{-}{} C[પ્રોપેગેશન માધ્યમ]}
    C {-{-}{} D[રિસીવર એન્ટેના]}
    D {-{-}{} E[રિસીવર]}
{Highlighting}
{Shaded}
\end{verbatim}
\end{center}

\begin{itemize}
\tightlist
\item
  \textbf{કાર્ય}: ટ્રાન્સમીટર RF સિગ્નલ જનરેટ કરે છે, એન્ટેના સિગ્નલ વિકિરણિત કરે છે
\item
  \textbf{ફ્રિક્વન્સી બેન્ડ}: HF (3-30 MHz), VHF (30-300 MHz), UHF (300-3000
  MHz)
\item
  \textbf{મોડ્સ}: AM, FM, SSB, CW (મોર્સ), ડિજિટલ મોડ્સ
\item
  \textbf{લાઇસન્સ}: કાયદેસર સંચાલન માટે જરૂરી (કૌશલ્ય આધારિત સ્તર)
\end{itemize}

\end{solutionbox}
\begin{mnemonicbox}
``TEAM: ટ્રાન્સમિશન એનેબલ્સ એમેચ્યોર મેસેજીસ''

\end{mnemonicbox}
\subsection*{પ્રશ્ન 4(બ) [4
ગુણ]}\label{uxaaauxab0uxab6uxaa8-4uxaac-4-uxa97uxaa3}

\textbf{ડક્ટ પ્રોપેગેશન સમજાવો.}

\begin{solutionbox}

ડક્ટ પ્રોપેગેશન ત્યારે થાય છે જ્યારે રેડિયો તરંગો વિવિધ રિફ્રેક્ટિવ ઇન્ડેક્સ ધરાવતા
વાતાવરણીય સ્તરોમાં ફસાય છે.

\textbf{આકૃતિ}:

\begin{verbatim}
        {-{-}{-}{-}{-}{-}{-}{-}{-}{-}{-}{-}{-}{-}{-}{-}{-}{-}  -{-} Upper atmosphere}
       
         ===============  {-{-} Duct layer (temperature inversion)}
        /     Trapped    {}
       /       waves      {}
      /                    {}
     /                      {}
    /                        {}
   /                          {}
  Transmitter                Receiver
  
      ==================== Ground/Sea
\end{verbatim}

\begin{itemize}
\tightlist
\item
  \textbf{રચના}: તાપમાન ઇન્વર્ઝન રિફ્રેક્ટિવ ઇન્ડેક્સ ગ્રેડિયન્ટ બનાવે છે
\item
  \textbf{ફ્રિક્વન્સી રેન્જ}: VHF, UHF, માઇક્રોવેવ ફ્રિક્વન્સી
\item
  \textbf{ફાયદા}: વિસ્તૃત કોમ્યુનિકેશન રેન્જ (ક્ષિતિજથી આગળ)
\item
  \textbf{ઘટના}: સમુદ્રો પર સામાન્ય, હવામાન સાથે બદલાય છે
\end{itemize}

\end{solutionbox}
\begin{mnemonicbox}
``TRIP: ટ્રેપ્ડ રેઝ ઇન એટમોસ્ફિરિક પાથ્સ''

\end{mnemonicbox}
\subsection*{પ્રશ્ન 4(ક) [7
ગુણ]}\label{uxaaauxab0uxab6uxaa8-4uxa95-7-uxa97uxaa3}

\textbf{ટ્રોપોસ્ફેરિક સ્કેટર્ડ પ્રોપેગેશન વિગતવાર સમજાવો.}

\begin{solutionbox}

ટ્રોપોસ્ફેરિક સ્કેટર ક્ષિતિજથી આગળના કોમ્યુનિકેશન માટે ટ્રોપોસ્ફિયરની સ્કેટરિંગ
પ્રોપર્ટીનો ઉપયોગ કરે છે.


{\def\LTcaptype{none} % do not increment counter
\vspace{-5pt}
\captionof{table}{ટ્રોપોસ્ફેરિક સ્કેટર ખાસિયતો}
\vspace{-10pt}
\begin{longtable}[]{@{}ll@{}}
\toprule\noalign{}
પેરામીટર & વર્ણન \\
\midrule\noalign{}
\endhead
\bottomrule\noalign{}
\endlastfoot
મેકેનિઝમ & ટ્રોપોસ્ફેરિક અનિયમિતતાઓ દ્વારા રેડિયો તરંગોનું ફોરવર્ડ સ્કેટરિંગ \\
ફ્રિક્વન્સી રેન્જ & 300 MHz થી 10 GHz (UHF/SHF) \\
રેન્જ & 100-800 km \\
પાથ લોસ & ઉચ્ચ (ઉચ્ચ-પાવર ટ્રાન્સમિટર્સની જરૂર પડે છે) \\
વિશ્વસનીયતા & હવામાન પરિસ્થિતિઓથી અસરગ્રસ્ત \\
\end{longtable}
}

\textbf{આકૃતિ}:

\begin{center}
\textbf{Mermaid Diagram (Code)}
\begin{verbatim}
{Shaded}
{Highlighting}[]
graph LR
    A[ટ્રાન્સમીટર] {-{-}{} B[હાઈ ગેઇન એન્ટેના]}
    B {-{-}{} C[ટ્રોપોસ્ફિયરમાં{}br /{}સ્કેટરિંગ વોલ્યુમ]}
    C {-{-}{} D[રિસીવિંગ એન્ટેના]}
    D {-{-}{} E[રિસીવર]}
    F[ફેક્ટર્સ] {-{-}{} G[હવામાન]}
    F {-{-}{} H[ફ્રિક્વન્સી]}
    F {-{-}{} I[એન્ટેના સાઈઝ]}
{Highlighting}
{Shaded}
\end{verbatim}
\end{center}

\begin{itemize}
\tightlist
\item
  \textbf{મેકેનિઝમ}: રિફ્રેક્ટિવ ઇન્ડેક્સ અનિયમિતતાઓ દ્વારા સિગ્નલ સ્કેટર થાય છે
\item
  \textbf{ઇક્વિપમેન્ટ}: હાઇ-પાવર ટ્રાન્સમીટર્સ, મોટા એન્ટેના, સંવેદનશીલ રિસીવર્સ
\item
  \textbf{ઉપયોગો}: મિલિટરી, બેકઅપ કોમ્યુનિકેશન, દૂરસ્થ વિસ્તારો
\item
  \textbf{ફાયદા}: લાઇન-ઓફ-સાઇટથી આગળ, પ્રમાણમાં સ્થિર
\end{itemize}

\end{solutionbox}
\begin{mnemonicbox}
``STARS: સ્કેટર ટ્રોપોસ્ફેરિક અલાઉઝ રેન્જ બિયોન્ડ સાઇટ''

\end{mnemonicbox}
\subsection*{પ્રશ્ન 4(અ) અથવા [3
ગુણ]}\label{uxaaauxab0uxab6uxaa8-4uxa85-uxa85uxaa5uxab5-3-uxa97uxaa3}

\textbf{ટર્નસ્ટાઇલ અને સુપર ટર્નસ્ટાઇલ એન્ટેના દોરો.}

\begin{solutionbox}

\textbf{આકૃતિ: ટર્નસ્ટાઇલ એન્ટેના}

\begin{verbatim}
          │   │
     ─────┼───┼─────
          │   │
          │   │
     ─────┼───┼─────
          │   │

  Two dipoles at 90^ fed with 90^ phase difference
\end{verbatim}

\textbf{આકૃતિ: સુપર ટર્નસ્ટાઇલ (બેટવિંગ) એન્ટેના}

\begin{verbatim}
      ┌───┬───┐
      │   │   │
      │   │   │
      │   │   │
    ──┼───┼───┼──
      │   │   │
      │   │   │
      │   │   │
      └───┴───┘

    Multiple elements for broadband operation
\end{verbatim}

\begin{itemize}
\tightlist
\item
  \textbf{ટર્નસ્ટાઇલ}: જમણા ખૂણે બે ડિપોલ, સર્ક્યુલર પોલરાઇઝેશન
\item
  \textbf{સુપર ટર્નસ્ટાઇલ}: વધારેલી બેન્ડવિડ્થ માટે મલ્ટિપલ એલિમેન્ટ્સ
\item
  \textbf{ઉપયોગો}: TV બ્રોડકાસ્ટિંગ, FM બ્રોડકાસ્ટિંગ, સેટેલાઇટ કોમ્યુનિકેશન
\item
  \textbf{ફાયદો}: ઓમ્નિડાયરેક્શનલ હોરિઝોન્ટલ પેટર્ન
\end{itemize}

\end{solutionbox}
\begin{mnemonicbox}
``TACO: ટર્નસ્ટાઇલ એન્ટેના ક્રિએટ ઓમ્નિડાયરેક્શનલ પેટર્ન''

\end{mnemonicbox}
\subsection*{પ્રશ્ન 4(બ) અથવા [4
ગુણ]}\label{uxaaauxab0uxab6uxaa8-4uxaac-uxa85uxaa5uxab5-4-uxa97uxaa3}

\textbf{MUF, LUF અને OUF નું સંપૂર્ણ સ્વરૂપ આપો.}

\begin{solutionbox}


{\def\LTcaptype{none} % do not increment counter
\vspace{-5pt}
\captionof{table}{આયનોસ્ફેરિક પ્રોપેગેશન પેરામીટર્સ}
\vspace{-10pt}
\begin{longtable}[]{@{}
  >{\raggedright\arraybackslash}p{(\linewidth - 4\tabcolsep) * \real{0.3684}}
  >{\raggedright\arraybackslash}p{(\linewidth - 4\tabcolsep) * \real{0.2895}}
  >{\raggedright\arraybackslash}p{(\linewidth - 4\tabcolsep) * \real{0.3421}}@{}}
\toprule\noalign{}
\begin{minipage}[b]{\linewidth}\raggedright
સંક્ષિપ્ત નામ
\end{minipage} & \begin{minipage}[b]{\linewidth}\raggedright
સંપૂર્ણ નામ
\end{minipage} & \begin{minipage}[b]{\linewidth}\raggedright
વર્ણન
\end{minipage} \\
\midrule\noalign{}
\endhead
\bottomrule\noalign{}
\endlastfoot
MUF & Maximum Usable Frequency & ઉચ્ચતમ ફ્રિક્વન્સી જે આયનોસ્ફિયર દ્વારા
પરાવર્તિત થઈ શકે છે \\
LUF & Lowest Usable Frequency & ન્યૂનતમ ફ્રિક્વન્સી જે પૂરતો સિગ્નલ-ટુ-નોઇઝ રેશિયો
પ્રદાન કરે છે \\
OUF & Optimum Usable Frequency & શ્રેષ્ઠ કાર્યકારી ફ્રિક્વન્સી (MUF નો 85\%) \\
\end{longtable}
}

\textbf{આકૃતિ}:

\begin{center}
\textbf{Mermaid Diagram (Code)}
\begin{verbatim}
{Shaded}
{Highlighting}[]
graph TD
    A[આયનોસ્ફેરિક ફ્રિક્વન્સી] {-{-}{} B[MUF]}
    A {-{-}{} C[LUF]}
    A {-{-}{} D[OUF]}
    B {-{-}{} E[ઉચ્ચતમ ફ્રિક્વન્સી{}br /{}જે પૃથ્વી પર પાછી આવે છે]}
    C {-{-}{} F[ન્યૂનતમ ફ્રિક્વન્સી{}br /{}પૂરતા SNR સાથે]}
    D {-{-}{} G[શ્રેષ્ઠ કાર્યકારી ફ્રિક્વન્સી{}br /{}MUF નો 85\%]}
{Highlighting}
{Shaded}
\end{verbatim}
\end{center}

\end{solutionbox}
\begin{mnemonicbox}
``MLO: મેક્સિમમ અને લોવેસ્ટ ઓપ્ટિમમ નક્કી કરે છે''

\end{mnemonicbox}
\subsection*{પ્રશ્ન 4(ક) અથવા [7
ગુણ]}\label{uxaaauxab0uxab6uxaa8-4uxa95-uxa85uxaa5uxab5-7-uxa97uxaa3}

\textbf{વર્ચ્યુઅલ ઊંચાઈ, ક્રિટિકલ ફ્રિક્વન્સી અને સ્કીપ ડિસ્ટન્સ વિગતવાર સમજાવો.}

\begin{solutionbox}


{\def\LTcaptype{none} % do not increment counter
\vspace{-5pt}
\captionof{table}{મુખ્ય આયનોસ્ફેરિક પ્રોપેગેશન પેરામીટર્સ}
\vspace{-10pt}
\begin{longtable}[]{@{}
  >{\raggedright\arraybackslash}p{(\linewidth - 4\tabcolsep) * \real{0.2973}}
  >{\raggedright\arraybackslash}p{(\linewidth - 4\tabcolsep) * \real{0.3243}}
  >{\raggedright\arraybackslash}p{(\linewidth - 4\tabcolsep) * \real{0.3784}}@{}}
\toprule\noalign{}
\begin{minipage}[b]{\linewidth}\raggedright
પેરામીટર
\end{minipage} & \begin{minipage}[b]{\linewidth}\raggedright
વ્યાખ્યા
\end{minipage} & \begin{minipage}[b]{\linewidth}\raggedright
મહત્વ
\end{minipage} \\
\midrule\noalign{}
\endhead
\bottomrule\noalign{}
\endlastfoot
વર્ચ્યુઅલ ઊંચાઈ & સીધી-લાઇન પ્રસારણ ધારીને દેખાતી પરાવર્તન ઊંચાઈ & મહત્તમ સંચાર રેન્જ
નક્કી કરે છે \\
ક્રિટિકલ ફ્રિક્વન્સી & ઊભા આપાત પર પરાવર્તિત મહત્તમ ફ્રિક્વન્સી & આયનાઇઝેશન ઘનતા
દર્શાવે છે \\
સ્કીપ ડિસ્ટન્સ & ન્યૂનતમ અંતર જ્યાં આયનોસ્ફેરિક સિગ્નલ્સ પ્રાપ્ત થઈ શકે છે & ``સ્કીપ
ઝોન'' બનાવે છે જેમાં કોઈ રિસેપ્શન નથી \\
\end{longtable}
}

\textbf{આકૃતિ}:

\begin{verbatim}
                  /|{}
                 / | {}
                /  |  {       Critical freq: Maximum}
               /   |   {      frequency at 90^ incidence}
              /    |    {}
             /     |     {}
            /      |      {}
Transmitter/       |       {Receiver}
           /       |        {}
          /        |         {}
         /         |          {}
        /          |           {}
       /           |            {}
      /            |             {}
     /             |              {}
    /              |               {}
   /               |                {}
  /                |                 {}
 /                 |                  {}
Earth              |                  Earth
       |{{-}{-}{-}{-}{-}{-} Skip Distance {-}{-}{-}{-}{-}{-}{-}{-}|}

Virtual height: Apparent reflection height
\end{verbatim}

\begin{itemize}
\tightlist
\item
  \textbf{વર્ચ્યુઅલ ઊંચાઈ}: સામાન્ય રીતે F લેયર માટે 300-400 km, સમય/સિઝન સાથે
  બદલાય છે
\item
  \textbf{ક્રિટિકલ ફ્રિક્વન્સી}: સામાન્ય રીતે F2 લેયર માટે 5-10 MHz, સૌર પ્રવૃત્તિ
  પર આધાર રાખે છે
\item
  \textbf{સ્કીપ ડિસ્ટન્સ}: D = 2h tan θ દ્વારા આપવામાં આવે છે, જ્યાં h એ વર્ચ્યુઅલ
  ઊંચાઈ અને θ આપાત કોણ છે
\end{itemize}

\end{solutionbox}
\begin{mnemonicbox}
``VCS: વર્ચ્યુઅલ ઊંચાઈ સ્કીપ ડિસ્ટન્સ નિયંત્રિત કરે છે''

\end{mnemonicbox}
\subsection*{પ્રશ્ન 5(અ) [3
ગુણ]}\label{uxaaauxab0uxab6uxaa8-5uxa85-3-uxa97uxaa3}

\textbf{સુઘડ આકૃતિ સાથે વિવિધ આયોનોસ્ફીયર સ્તરો દર્શાવો.}

\begin{solutionbox}

\textbf{આકૃતિ: આયનોસ્ફેરિક લેયર્સ}

\begin{verbatim}
Height (km)
   \^{}
   |
400|                   F2 Layer
   |           {-{-}{-}{-}{-}{-}{-}{-}{-}{-}{-}{-}{-}{-}{-}{-}{-}{-}{-}{-}{-}{-}}
   |
300|                   F1 Layer (daytime)
   |           {-{-}{-}{-}{-}{-}{-}{-}{-}{-}{-}{-}{-}{-}{-}{-}{-}{-}{-}{-}{-}{-}}
   |
200|                   E Layer
   |           {-{-}{-}{-}{-}{-}{-}{-}{-}{-}{-}{-}{-}{-}{-}{-}{-}{-}{-}{-}{-}{-}}
   |
100|                   D Layer
   |           {-{-}{-}{-}{-}{-}{-}{-}{-}{-}{-}{-}{-}{-}{-}{-}{-}{-}{-}{-}{-}{-}}
   |
   +{-{-}{-}{-}{-}{-}{-}{-}{-}{-}{-}{-}{-}{-}{-}{-}{-}{-}{-}{-}{-}{-}{-}{-}{-}{-}{-}{-}{-}{-}{-}{-}{-}{-}{-}{-}{-}}
                 Electron Density
\end{verbatim}

\begin{itemize}
\tightlist
\item
  \textbf{D લેયર}: 60-90 km, HF તરંગોને શોષે છે, રાત્રે અદૃશ્ય થાય છે
\item
  \textbf{E લેયર}: 90-150 km, MF/નીચા HF પરાવર્તિત કરે છે, રાત્રે નબળી પડે છે
\item
  \textbf{F1 લેયર}: 150-220 km, માત્ર દિવસ સમયે હાજર
\item
  \textbf{F2 લેયર}: 220-400 km, મુખ્ય પરાવર્તન સ્તર, દિવસ/રાત હાજર
\end{itemize}

\end{solutionbox}
\begin{mnemonicbox}
``DEAF: નીચેથી ઉપર - D, E, And F લેયર્સ''

\end{mnemonicbox}
\subsection*{પ્રશ્ન 5(બ) [4
ગુણ]}\label{uxaaauxab0uxab6uxaa8-5uxaac-4-uxa97uxaa3}

\textbf{વિવિધ પ્રકારની સેટેલાઇટ કોમ્યુનિકેશન સિસ્ટમના નામ આપો અને તેની સરખામણી
કરો.}

\begin{solutionbox}


{\def\LTcaptype{none} % do not increment counter
\vspace{-5pt}
\captionof{table}{સેટેલાઇટ કોમ્યુનિકેશન સિસ્ટમ્સ}
\vspace{-10pt}
\begin{longtable}[]{@{}
  >{\raggedright\arraybackslash}p{(\linewidth - 6\tabcolsep) * \real{0.2167}}
  >{\raggedright\arraybackslash}p{(\linewidth - 6\tabcolsep) * \real{0.2667}}
  >{\raggedright\arraybackslash}p{(\linewidth - 6\tabcolsep) * \real{0.2333}}
  >{\raggedright\arraybackslash}p{(\linewidth - 6\tabcolsep) * \real{0.2833}}@{}}
\toprule\noalign{}
\begin{minipage}[b]{\linewidth}\raggedright
સિસ્ટમ પ્રકાર
\end{minipage} & \begin{minipage}[b]{\linewidth}\raggedright
ફ્રિક્વન્સી બેન્ડ
\end{minipage} & \begin{minipage}[b]{\linewidth}\raggedright
ઉપયોગો
\end{minipage} & \begin{minipage}[b]{\linewidth}\raggedright
ખાસિયતો
\end{minipage} \\
\midrule\noalign{}
\endhead
\bottomrule\noalign{}
\endlastfoot
ટેલિકોમ્યુનિકેશન & C, Ku, Ka બેન્ડ & ફોન, ડેટા, ઇન્ટરનેટ & ગ્લોબલ કવરેજ, ઉચ્ચ
ક્ષમતા \\
બ્રોડકાસ્ટિંગ & Ku, C બેન્ડ & TV, રેડિયો ટ્રાન્સમિશન & હાઇ પાવર, વિશાળ કવરેજ \\
ડેટા કોમ્યુનિકેશન & L, S, Ka બેન્ડ & IoT, VSAT, M2M & ઓછી થી મધ્યમ ડેટા દર \\
મિલિટરી & X, EHF બેન્ડ & સિક્યોર કોમ્યુનિકેશન & એનક્રિપ્ટેડ, જામ-રેસિસ્ટન્ટ \\
નેવિગેશન & L બેન્ડ & GPS, GLONASS, ગેલિલિયો & ચોક્કસ ટાઇમિંગ, પોઝિશનિંગ \\
\end{longtable}
}

\textbf{આકૃતિ}:

\begin{verbatim}
pie
    title "સેટેલાઇટ કોમ્યુનિકેશન સિસ્ટમ્સ"
    "ટેલિકોમ્યુનિકેશન" : 30
    "બ્રોડકાસ્ટિંગ" : 25
    "ડેટા કોમ્યુનિકેશન" : 20
    "મિલિટરી" : 15
    "નેવિગેશન" : 10
\end{verbatim}

\end{solutionbox}
\begin{mnemonicbox}
``TBDMN: ટેલિકોમ, બ્રોડકાસ્ટિંગ, ડેટા, મિલિટરી, નેવિગેશન''

\end{mnemonicbox}
\subsection*{પ્રશ્ન 5(ક) [7
ગુણ]}\label{uxaaauxab0uxab6uxaa8-5uxa95-7-uxa97uxaa3}

\textbf{DTH રીસીવર સિસ્ટમ દોરો અને સમજાવો.}

\begin{solutionbox}

DTH (ડાયરેક્ટ-ટુ-હોમ) સિસ્ટમ સેટેલાઇટ મારફતે સીધા દર્શકોને ટેલિવિઝન પ્રોગ્રામિંગ
ડિલિવર કરે છે.

\textbf{આકૃતિ}:

\begin{verbatim}
                     TV
                     |
                     V
                 Set{-top Box}
                     |
                     V
                  LNB/LNBF {{-}{-}{-}{-} Satellite signals}
                     |
                     V
                 Dish Antenna
                  (0.6{-1.2m)}
\end{verbatim}


{\def\LTcaptype{none} % do not increment counter
\vspace{-5pt}
\captionof{table}{DTH સિસ્ટમ કોમ્પોનન્ટ્સ}
\vspace{-10pt}
\begin{longtable}[]{@{}
  >{\raggedright\arraybackslash}p{(\linewidth - 4\tabcolsep) * \real{0.3056}}
  >{\raggedright\arraybackslash}p{(\linewidth - 4\tabcolsep) * \real{0.2778}}
  >{\raggedright\arraybackslash}p{(\linewidth - 4\tabcolsep) * \real{0.4167}}@{}}
\toprule\noalign{}
\begin{minipage}[b]{\linewidth}\raggedright
કોમ્પોનન્ટ
\end{minipage} & \begin{minipage}[b]{\linewidth}\raggedright
કાર્ય
\end{minipage} & \begin{minipage}[b]{\linewidth}\raggedright
સ્પેસિફિકેશન
\end{minipage} \\
\midrule\noalign{}
\endhead
\bottomrule\noalign{}
\endlastfoot
ડિશ એન્ટેના & સેટેલાઇટ સિગ્નલ્સ એકત્રિત કરે છે & 45-120 cm વ્યાસ \\
LNB (લો નોઇઝ બ્લોક) & ઉચ્ચ ફ્રિક્વન્સીને નીચા IF માં રૂપાંતરિત કરે છે & નોઇઝ ફિગર:
0.3-1.0 dB \\
કોએક્સિયલ કેબલ & IF સિગ્નલને રિસીવર સુધી લઈ જાય છે & RG-6 પ્રકાર, 75 ઓહ્મ \\
સેટ-ટોપ બોક્સ & સિગ્નલ્સ ડિમોડ્યુલેટ/ડિકોડ કરે છે & MPEG-2/4 ડિકોડર \\
TV સેટ & પ્રોગ્રામિંગ દર્શાવે છે & HDMI/કોમ્પોનન્ટ ઇનપુટ \\
\end{longtable}
}

\begin{itemize}
\tightlist
\item
  \textbf{ફ્રિક્વન્સી}: Ku-બેન્ડ (10.7-12.75 GHz) અથવા C-બેન્ડ (3.7-4.2 GHz)
\item
  \textbf{મોડ્યુલેશન}: QPSK અથવા 8PSK ડિજિટલ મોડ્યુલેશન
\item
  \textbf{સિગ્નલ પ્રોસેસિંગ}: ડિજિટલ કમ્પ્રેશન (MPEG-2/4)
\item
  \textbf{ફીચર્સ}: EPG (ઇલેક્ટ્રોનિક પ્રોગ્રામ ગાઇડ), PVR (રેકોર્ડિંગ)
\end{itemize}

\end{solutionbox}
\begin{mnemonicbox}
``DOCS: ડિશ ઓબ્ટેઇન્સ, કન્વર્ટ્સ અને શોઝ સિગ્નલ્સ''

\end{mnemonicbox}
\subsection*{પ્રશ્ન 5(અ) અથવા [3
ગુણ]}\label{uxaaauxab0uxab6uxaa8-5uxa85-uxa85uxaa5uxab5-3-uxa97uxaa3}

\textbf{સ્માર્ટ એન્ટેનાની જરૂર શું છે? તેના ઉપયોગો લખો.}

\begin{solutionbox}

સ્માર્ટ એન્ટેના એડેપ્ટિવ સિગ્નલ પ્રોસેસિંગનો ઉપયોગ કરીને ડાયનામિકલી રેડિએશન પેટર્ન
ઓપ્ટિમાઇઝ કરે છે.

\textbf{જરૂરિયાતો}:

\begin{itemize}
\tightlist
\item
  ભીડભાડવાળા નેટવર્કમાં વધારેલી ક્ષમતા
\item
  સુધારેલ સિગ્નલ ક્વોલિટી અને કવરેજ
\item
  ઘટાડેલો ઇન્ટરફેરન્સ અને મલ્ટિપાથ ફેડિંગ
\item
  વધારેલી સ્પેક્ટ્રલ એફિશિયન્સી
\end{itemize}

\textbf{આકૃતિ}:

\begin{center}
\textbf{Mermaid Diagram (Code)}
\begin{verbatim}
{Shaded}
{Highlighting}[]
graph TD
    A[સ્માર્ટ એન્ટેના] {-{-}{} B[એડેપ્ટિવ{}br /{}બીમફોર્મિંગ]}
    A {-{-}{} C[સ્પેશિયલ{}br /{}મલ્ટિપ્લેક્સિંગ]}
    A {-{-}{} D[ઇન્ટરફેરન્સ{}br /{}સપ્રેશન]}
{Highlighting}
{Shaded}
\end{verbatim}
\end{center}

\textbf{ઉપયોગો}:

\begin{itemize}
\tightlist
\item
  મોબાઇલ કોમ્યુનિકેશન નેટવર્ક (4G/5G)
\item
  ઉચ્ચ ડેટા દર માટે MIMO સિસ્ટમ્સ
\item
  વધુ સારી ટાર્ગેટ ડિટેક્શન સાથે રડાર સિસ્ટમ્સ
\item
  સુધારેલા કવરેજ સાથે વાયરલેસ LAN
\end{itemize}

\end{solutionbox}
\begin{mnemonicbox}
``SAFE: સ્માર્ટ એન્ટેના ફોર એફિશિયન્સી''

\end{mnemonicbox}
\subsection*{પ્રશ્ન 5(બ) અથવા [4
ગુણ]}\label{uxaaauxab0uxab6uxaa8-5uxaac-uxa85uxaa5uxab5-4-uxa97uxaa3}

\textbf{કેપ્લરનો ત્રીજો નિયમ સમજાવો.}

\begin{solutionbox}

કેપ્લરનો ત્રીજો નિયમ સેટેલાઇટની ભ્રમણ કાળનો તેના સેમી-મેજર એક્સિસ સાથેનો સંબંધ દર્શાવે
છે.

\textbf{ફોર્મ્યુલા}: T^{2} = (4π^{2}/GM) \times a^{3}

જ્યાં:

\begin{itemize}
\tightlist
\item
  T = ભ્રમણ કાળ
\item
  a = સેમી-મેજર એક્સિસ
\item
  G = ગુરુત્વાકર્ષણ અચળાંક
\item
  M = કેન્દ્રીય પિંડનો દ્રવ્યમાન
\end{itemize}

\textbf{આકૃતિ}:

\begin{center}
\textbf{Mermaid Diagram (Code)}
\begin{verbatim}
{Shaded}
{Highlighting}[]
graph LR
    A[કેપ્લરનો ત્રીજો નિયમ] {-{-}{} B["T^{2} ∝ a^{3}"]}
    B {-{-}{} C[T = ભ્રમણ કાળ]}
    B {-{-}{} D[a = સેમી{-}મેજર એક્સિસ]}
    E[ઉપયોગો] {-{-}{} F[સેટેલાઇટ ઓર્બિટ નિર્ધારણ]}
    E {-{-}{} G[સ્પેસક્રાફ્ટ મિશન પ્લાનિંગ]}
{Highlighting}
{Shaded}
\end{verbatim}
\end{center}

\begin{itemize}
\tightlist
\item
  \textbf{અર્થ}: મોટા ઓર્બિટને લાંબો ભ્રમણ કાળ હોય છે
\item
  \textbf{ઉપયોગ}: સેટેલાઇટ ઓર્બિટની ખાસિયતો નક્કી કરે છે
\item
  \textbf{જિયોસ્ટેશનરી ઓર્બિટ}: ભ્રમણ કાળ = 24 કલાક, ઊંચાઈ \approx 35,786 km
\end{itemize}

\end{solutionbox}
\begin{mnemonicbox}
``CAP: ક્યુબ ઓફ એક્સિસ ઈક્વલ્સ પીરિયડ સ્ક્વેર્ડ''

\end{mnemonicbox}
\subsection*{પ્રશ્ન 5(ક) અથવા [7
ગુણ]}\label{uxaaauxab0uxab6uxaa8-5uxa95-uxa85uxaa5uxab5-7-uxa97uxaa3}

\textbf{ટેરેસ્ટ્રીયલ મોબાઈલ કોમ્યુનિકેશન માટે એન્ટેનાના વિવિધ પ્રકારો ઓળખો અને
વિગતવાર સમજાવો.}

\begin{solutionbox}


{\def\LTcaptype{none} % do not increment counter
\vspace{-5pt}
\captionof{table}{ટેરેસ્ટ્રીયલ મોબાઇલ કોમ્યુનિકેશન એન્ટેના}
\vspace{-10pt}
\begin{longtable}[]{@{}
  >{\raggedright\arraybackslash}p{(\linewidth - 6\tabcolsep) * \real{0.2500}}
  >{\raggedright\arraybackslash}p{(\linewidth - 6\tabcolsep) * \real{0.2500}}
  >{\raggedright\arraybackslash}p{(\linewidth - 6\tabcolsep) * \real{0.2500}}
  >{\raggedright\arraybackslash}p{(\linewidth - 6\tabcolsep) * \real{0.2500}}@{}}
\toprule\noalign{}
\begin{minipage}[b]{\linewidth}\raggedright
એન્ટેના પ્રકાર
\end{minipage} & \begin{minipage}[b]{\linewidth}\raggedright
ટિપિકલ ગેઇન
\end{minipage} & \begin{minipage}[b]{\linewidth}\raggedright
પોલરાઇઝેશન
\end{minipage} & \begin{minipage}[b]{\linewidth}\raggedright
ઉપયોગો
\end{minipage} \\
\midrule\noalign{}
\endhead
\bottomrule\noalign{}
\endlastfoot
બેઝ સ્ટેશન એન્ટેના & 10-18 dBi & વર્ટિકલ/ડ્યુઅલ & સેલ ટાવર્સ, ફિક્સ્ડ ઇન્ફ્રાસ્ટ્રક્ચર \\
મોબાઇલ સ્ટેશન એન્ટેના & 0-3 dBi & વર્ટિકલ & સ્માર્ટફોન, વાહનો, પોર્ટેબલ
ડિવાઇસિસ \\
રિપીટર એન્ટેના & 5-10 dBi & સર્ક્યુલર/ડ્યુઅલ & સિગ્નલ બૂસ્ટિંગ, કવરેજ એક્સટેન્શન \\
ડાયવર્સિટી એન્ટેના & વેરિએબલ & મલ્ટિપલ & મલ્ટિપાથ મિટિગેશન, MIMO સિસ્ટમ્સ \\
\end{longtable}
}

\textbf{બેઝ સ્ટેશન એન્ટેના (વિગતવાર)}:

\textbf{આકૃતિ}:

\begin{verbatim}
        ┌────┐
        │    │
        │    │
        │    │      Array of
        │    │     radiating
        │    │     elements
        │    │
        │    │
        │    │
        │    │
        └────┘
          |
        Sector coverage
\end{verbatim}

\begin{itemize}
\tightlist
\item
  \textbf{પ્રકારો}: પેનલ એરે, કોલિનિયર એરે, સેક્ટર એન્ટેના
\item
  \textbf{ખાસિયતો}:

  \begin{itemize}
  \tightlist
  \item
    ઉચ્ચ ગેઇન (10-18 dBi)
  \item
    દિશાત્મક રેડિએશન પેટર્ન (60^\circ-120^\circ સેક્ટર)
  \item
    ડાઉનટિલ્ટ ક્ષમતા (ઇલેક્ટ્રિકલ/મિકેનિકલ)
  \item
    મલ્ટિપલ-બેન્ડ ઓપરેશન
  \end{itemize}
\item
  \textbf{અદ્યતન ફીચર્સ}:

  \begin{itemize}
  \tightlist
  \item
    મલ્ટિપલ-ઇનપુટ મલ્ટિપલ-આઉટપુટ (MIMO)
  \item
    રિમોટ ઇલેક્ટ્રિકલ ટિલ્ટ (RET)
  \item
    ઇન્ટિગ્રેટેડ ડિપ્લેક્સર/ટ્રિપ્લેક્સર
  \end{itemize}
\end{itemize}

\textbf{મોબાઇલ સ્ટેશન એન્ટેના}:

\begin{itemize}
\tightlist
\item
  કોમ્પેક્ટ સાઇઝ (આંતરિક/બાહ્ય)
\item
  ઓમ્નિડાયરેક્શનલ પેટર્ન
\item
  મલ્ટિપલ બેન્ડ સપોર્ટ (700-2600 MHz)
\item
  ઇમ્પ્લિમેન્ટેશન: PIFA, હેલિકલ, મોનોપોલ ડિઝાઇન
\end{itemize}

\end{solutionbox}
\begin{mnemonicbox}
``BEST: બેઝ-સ્ટેશન્સ એમ્પ્લોય સેક્ટર ટેકનોલોજી''

\end{mnemonicbox}

\end{document}
