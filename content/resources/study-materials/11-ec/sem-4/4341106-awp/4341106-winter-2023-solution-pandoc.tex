\documentclass[10pt,a4paper]{article}

% content/resources/templates/preamble.tex
\usepackage[margin=0.6in]{geometry}
\author{Milav Dabgar}
\usepackage{amsmath,amssymb,amsthm}
\usepackage{booktabs}
\usepackage{multirow}
\usepackage{xcolor}
\usepackage{tcolorbox}
\tcbuselibrary{breakable,skins}
\usepackage[colorlinks=true,linkcolor=blue]{hyperref}
\usepackage{titlesec}
\usepackage{enumitem}
\usepackage{tikz}
\usepackage{pgfplots}
\usepackage{circuitikz}
\usepackage[version=4]{mhchem}
\usepackage{longtable}
\usepackage{array}
\usepackage{float}
\usepackage{caption}
\usepackage{listings}

\lstset{
  basicstyle=\small\ttfamily,
  breaklines=true,
  breakatwhitespace=false,
  postbreak=\mbox{\textcolor{red}{$\hookrightarrow$}\space},
  float=false,
  numbers=left,
  numberstyle=\tiny\color{gray},
  numbersep=10pt,
  xleftmargin=2em,
  keywordstyle=\color{blue},
  commentstyle=\color{green!60!black},
  stringstyle=\color{purple},
  backgroundcolor=\color{gray!5},
  showstringspaces=false,
  tabsize=2,
  captionpos=b,
  keepspaces=true,
  columns=flexible
}

\pgfplotsset{compat=1.18}
\usetikzlibrary{shapes,arrows,positioning,calc,patterns,decorations.pathmorphing,decorations.markings,arrows.meta}

% Color scheme
\definecolor{headcolor}{RGB}{0,102,204}
\definecolor{keycolor}{RGB}{220,20,60}
\definecolor{solutioncolor}{RGB}{34,139,34}
\definecolor{mnemoniccolor}{RGB}{148,0,211}
\definecolor{codecolor}{RGB}{0,0,100}

% Spacing
\setlength{\parskip}{3pt}
\setlist[itemize]{nosep}
\setlist[enumerate]{nosep}

% Title formatting
\titleformat{\section}{\Large\bfseries\color{headcolor}}{\thesection}{1em}{}
\titleformat{\subsection}{\large\bfseries\color{headcolor}}{\thesubsection}{1em}{}

% Pandoc tightlist compatibility
\providecommand{\tightlist}{%
  \setlength{\itemsep}{0pt}\setlength{\parskip}{0pt}}

% Pandoc longtable compatibility
\newcounter{none}
\def\thenone{}


% content/resources/templates/english-boxes.tex
% This file is currently empty - it exists to maintain consistency with the import structure.
% Add custom environments here if needed in the future.


\begin{document}

\begin{center}
{\Huge\bfseries\color{headcolor} Subject Name Solutions}\\[5pt]
{\LARGE 4341106 -- Winter 2023}\\[3pt]
{\large Semester 1 Study Material}\\[3pt]
{\normalsize\textit{Detailed Solutions and Explanations}}
\end{center}

\vspace{10pt}

\subsection*{Question 1(a) [3 marks]}\label{q1a}

\textbf{Define: (1) Directivity, (2) Gain, and (3) HPBW}

\begin{solutionbox}


{\def\LTcaptype{none} % do not increment counter
\vspace{-5pt}
\captionof{table}{Key Antenna Parameters}
\vspace{-10pt}
\begin{longtable}[]{@{}
  >{\raggedright\arraybackslash}p{(\linewidth - 2\tabcolsep) * \real{0.4783}}
  >{\raggedright\arraybackslash}p{(\linewidth - 2\tabcolsep) * \real{0.5217}}@{}}
\toprule\noalign{}
\begin{minipage}[b]{\linewidth}\raggedright
Parameter
\end{minipage} & \begin{minipage}[b]{\linewidth}\raggedright
Definition
\end{minipage} \\
\midrule\noalign{}
\endhead
\bottomrule\noalign{}
\endlastfoot
\textbf{Directivity} & Ratio of maximum radiation intensity to average
radiation intensity of an antenna \\
\textbf{Gain} & Ratio of power radiated in a particular direction to the
power that would be radiated by an isotropic antenna \\
\textbf{HPBW (Half Power Beam Width)} & Angular width where radiation
intensity is half (3dB less) of the maximum value \\
\end{longtable}
}

\end{solutionbox}
\begin{mnemonicbox}
``DGH: Direction Gives Half-power''

\end{mnemonicbox}
\subsection*{Question 1(b) [4 marks]}\label{q1b}

\textbf{List the properties of electromagnetic waves}

\begin{solutionbox}


{\def\LTcaptype{none} % do not increment counter
\vspace{-5pt}
\captionof{table}{Properties of Electromagnetic Waves}
\vspace{-10pt}
\begin{longtable}[]{@{}
  >{\raggedright\arraybackslash}p{(\linewidth - 2\tabcolsep) * \real{0.4348}}
  >{\raggedright\arraybackslash}p{(\linewidth - 2\tabcolsep) * \real{0.5652}}@{}}
\toprule\noalign{}
\begin{minipage}[b]{\linewidth}\raggedright
Property
\end{minipage} & \begin{minipage}[b]{\linewidth}\raggedright
Description
\end{minipage} \\
\midrule\noalign{}
\endhead
\bottomrule\noalign{}
\endlastfoot
\textbf{Transverse Waves} & Electric and magnetic fields perpendicular
to direction of propagation \\
\textbf{Velocity} & Speed of light (3\times10\^{}8 m/s) in vacuum \\
\textbf{No Medium Required} & Can travel through vacuum, unlike
mechanical waves \\
\textbf{Polarization} & Direction of electric field vector \\
\textbf{Energy Transport} & Carries energy through space \\
\textbf{Reflection/Refraction} & Can be reflected and refracted at
boundaries \\
\textbf{Interference/Diffraction} & Show wave-like properties \\
\end{longtable}
}

\end{solutionbox}
\begin{mnemonicbox}
``TVNPER: Transverse Velocity No-medium Polarized
Energy Reflection''

\end{mnemonicbox}
\subsection*{Question 1(c) [7 marks]}\label{q1c}

\textbf{Explain physical concept of generation of Electromagnetic wave}

\begin{solutionbox}

\textbf{Diagram: Generation of Electromagnetic Wave}

\begin{center}
\textbf{Mermaid Diagram (Code)}
\begin{verbatim}
{Shaded}
{Highlighting}[]
graph LR
    A[Accelerating Charge] {-{-}{}|Produces| B[Time{-}varying Electric Field]}
    B {-{-}{}|Produces| C[Time{-}varying Magnetic Field]}
    C {-{-}{}|Produces| D[Time{-}varying Electric Field]}
    D {-{-}{} C}
    C {-{-}{} E[Self{-}sustaining EM Wave]}
{Highlighting}
{Shaded}
\end{verbatim}
\end{center}

\begin{itemize}
\tightlist
\item
  \textbf{Charge Acceleration}: When electric charges accelerate, they
  generate changing electric fields
\item
  \textbf{Field Coupling}: A changing electric field produces a changing
  magnetic field and vice versa
\item
  \textbf{Self-Propagation}: This cyclic generation of fields allows
  waves to travel without a medium
\item
  \textbf{Field Orientation}: Electric and magnetic fields are
  perpendicular to each other and the direction of propagation
\item
  \textbf{Energy Transport}: Energy alternates between electric and
  magnetic fields as wave propagates
\end{itemize}

\end{solutionbox}
\begin{mnemonicbox}
``CASES: Charge Acceleration Self-propagates
Electro-magnetic Signals''

\end{mnemonicbox}
\subsection*{Question 1(c) OR [7
marks]}\label{q1c}

\textbf{Explain how electromagnetic field radiated from a center fed
dipole}

\begin{solutionbox}

\textbf{Diagram: Field Radiation from Center-Fed Dipole}

\begin{center}
\textbf{Mermaid Diagram (Code)}
\begin{verbatim}
{Shaded}
{Highlighting}[]
graph LR
    A[Alternating Current Input] {-{-}{}|Creates| B[Oscillating Charges]}
    B {-{-}{}|Generates| C[Time{-}varying Electric Field]}
    C {-{-}{}|Generates| D[Time{-}varying Magnetic Field]}
    C {-{-}{} E[EM Wave Radiation]}
    D {-{-}{} E}
{Highlighting}
{Shaded}
\end{verbatim}
\end{center}

\begin{itemize}
\tightlist
\item
  \textbf{Center Feeding}: AC signal applied at center of dipole creates
  oscillating current
\item
  \textbf{Charge Distribution}: Current creates opposite charges at
  dipole ends that change with AC frequency
\item
  \textbf{Field Generation}: Oscillating charges create time-varying
  electric field
\item
  \textbf{Magnetic Coupling}: Time-varying electric field generates
  perpendicular magnetic field
\item
  \textbf{Near/Far Fields}: Near dipole, fields are complex; far from
  dipole, fields form uniform radiation pattern
\item
  \textbf{Radiation Pattern}: Maximum radiation perpendicular to dipole
  axis, zero radiation along axis
\end{itemize}

\end{solutionbox}
\begin{mnemonicbox}
``CORONA: Current Oscillates, Radiation Occurs,
Near-far Areas''

\end{mnemonicbox}
\subsection*{Question 2(a) [3 marks]}\label{q2a}

\textbf{Differentiate the resonant and non-resonant antennas}

\begin{solutionbox}


{\def\LTcaptype{none} % do not increment counter
\vspace{-5pt}
\captionof{table}{Resonant vs Non-Resonant Antennas}
\vspace{-10pt}
\begin{longtable}[]{@{}lll@{}}
\toprule\noalign{}
Feature & Resonant Antennas & Non-Resonant Antennas \\
\midrule\noalign{}
\endhead
\bottomrule\noalign{}
\endlastfoot
\textbf{Length} & Integer multiple of λ/2 & Not related to wavelength \\
\textbf{Standing Waves} & Present & Not present \\
\textbf{Impedance} & Resistive (real) & Complex (real + imaginary) \\
\textbf{Bandwidth} & Narrow & Wide \\
\textbf{Example} & Half-wave dipole & Rhombic antenna \\
\end{longtable}
}

\end{solutionbox}
\begin{mnemonicbox}
``RESI: Resonant Exhibits Standing-waves
Impedance-real''

\end{mnemonicbox}
\subsection*{Question 2(b) [4 marks]}\label{q2b}

\textbf{Explain Yagi antenna and discuss its radiation characteristics}

\begin{solutionbox}

\textbf{Diagram: Yagi-Uda Antenna Structure}

\begin{verbatim}
   Reflector    Driven    Directors
     (R)       Element      (D)
      |          (DE)     |  |  |
      |           |       |  |  |
 {-{-}{-}{-}{-}|{-}{-}{-}{-}{-}{-}{-}{-}{-}{-}{-}|{-}{-}{-}{-}{-}{-}{-}|{-}{-}|{-}{-}|{-}{-}{-}{-}{-}{-}{-} Direction of}
      |           |       |  |  |         Maximum Radiation
      |           |       |  |  |
      
   Longest       λ/2     Shortest
\end{verbatim}

\begin{itemize}
\tightlist
\item
  \textbf{Structure}: Contains one reflector, one driven element, and
  multiple directors
\item
  \textbf{Directivity}: High directivity in direction of directors
  (8-12dB)
\item
  \textbf{Gain}: Higher gain with more directors (up to 15dB)
\item
  \textbf{Bandwidth}: 2-5\% of center frequency
\item
  \textbf{Applications}: TV reception, point-to-point communication,
  amateur radio
\end{itemize}

\end{solutionbox}
\begin{mnemonicbox}
``DRAGONS: Directional Reflector And Gain-improving
Directors Offer Narrow Signals''

\end{mnemonicbox}
\subsection*{Question 2(c) [7 marks]}\label{q2c}

\textbf{Describe radiation characteristics of resonant wire antennas and
draw the current distribution of λ/2, 3λ/2 and 5λ/2 antenna}

\begin{solutionbox}

\textbf{Diagram: Current Distribution in Resonant Wire Antennas}

\begin{verbatim}
λ/2 Antenna:
     +{-{-}{-}{-}{-}{-}{-}{-}+}
     |        |
     v        v
 {-{-}{-}{-}+{-}{-}{-}{-}{-}{-}{-}{-}+{-}{-}{-}{-}}
     \^{        \^{}}
     |        |
     +{-{-}{-}{-}{-}{-}{-}{-}+}
     I\_max at center
     Zero at ends

3λ/2 Antenna:
     +{-{-}{-}+{-}{-}{-}+{-}{-}{-}+}
     |   |   |   |
     v   \^{   v   \^{}}
 {-{-}{-}{-}+{-}{-}{-}+{-}{-}{-}+{-}{-}{-}+{-}{-}{-}{-}}
     \^{   v   \^{}   v}
     |   |   |   |
     +{-{-}{-}+{-}{-}{-}+{-}{-}{-}+}
     3 current nodes
     
5λ/2 Antenna:
     +{-{-}{-}+{-}{-}{-}+{-}{-}{-}+{-}{-}{-}+{-}{-}{-}+}
     |   |   |   |   |   |
     v   \^{   v   \^{}   v   \^{}}
 {-{-}{-}{-}+{-}{-}{-}+{-}{-}{-}+{-}{-}{-}+{-}{-}{-}+{-}{-}{-}+{-}{-}{-}{-}}
     \^{   v   \^{}   v   \^{}   v}
     |   |   |   |   |   |
     +{-{-}{-}+{-}{-}{-}+{-}{-}{-}+{-}{-}{-}+{-}{-}{-}+}
     5 current nodes
\end{verbatim}

\begin{itemize}
\tightlist
\item
  \textbf{Half-Wave (λ/2)}: Current maximum at center, zero at ends;
  radiation pattern is figure-eight shaped
\item
  \textbf{Three Half-Wave (3λ/2)}: Three current maxima, phase reversal
  at λ/2 points; multiple lobes in radiation pattern
\item
  \textbf{Five Half-Wave (5λ/2)}: Five current maxima, more complex
  radiation pattern with multiple lobes
\item
  \textbf{Standing Waves}: All resonant antennas exhibit standing wave
  current distribution
\item
  \textbf{Feed Point}: Usually at current maximum for optimum impedance
  matching
\end{itemize}

\end{solutionbox}
\begin{mnemonicbox}
``NODE: Number Of Distributions Equals
wavelength-multiple''

\end{mnemonicbox}
\subsection*{Question 2(a) OR [3
marks]}\label{q2a}

\textbf{Differentiate the broad side and end fire array antennas}

\begin{solutionbox}


{\def\LTcaptype{none} % do not increment counter
\vspace{-5pt}
\captionof{table}{Broadside vs End Fire Array Antennas}
\vspace{-10pt}
\begin{longtable}[]{@{}
  >{\raggedright\arraybackslash}p{(\linewidth - 4\tabcolsep) * \real{0.2250}}
  >{\raggedright\arraybackslash}p{(\linewidth - 4\tabcolsep) * \real{0.4000}}
  >{\raggedright\arraybackslash}p{(\linewidth - 4\tabcolsep) * \real{0.3750}}@{}}
\toprule\noalign{}
\begin{minipage}[b]{\linewidth}\raggedright
Feature
\end{minipage} & \begin{minipage}[b]{\linewidth}\raggedright
Broadside Array
\end{minipage} & \begin{minipage}[b]{\linewidth}\raggedright
End Fire Array
\end{minipage} \\
\midrule\noalign{}
\endhead
\bottomrule\noalign{}
\endlastfoot
\textbf{Maximum Radiation} & Perpendicular to array axis & Along array
axis \\
\textbf{Element Spacing} & Typically λ/2 & Typically λ/4 to λ/2 \\
\textbf{Phase Difference} & 0^\circ (in-phase) & 180^\circ (opposite phase) \\
\textbf{Directivity} & High & High \\
\textbf{Pattern} & Bidirectional & Unidirectional \\
\end{longtable}
}

\end{solutionbox}
\begin{mnemonicbox}
``PEPS: Perpendicular Elements Produce Sideways
radiation''

\end{mnemonicbox}
\subsection*{Question 2(b) OR [4
marks]}\label{q2b}

\textbf{Explain loop antenna and discuss its radiation characteristics}

\begin{solutionbox}

\textbf{Diagram: Loop Antenna}

\begin{verbatim}
    +{-{-}{-}{-}{-}+}
    |     |
    |     |
+{-{-}{-}+     +{-}{-}{-}+}
|             |
+{-{-}{-}{-}{-}{-}+{-}{-}{-}{-}{-}{-}+}
       |
     Feed
     Point
\end{verbatim}

\begin{itemize}
\tightlist
\item
  \textbf{Structure}: Closed-loop conductor with circumference of one
  wavelength or less
\item
  \textbf{Types}: Small loops (circumference \textless{} λ/10) and large
  loops (circumference \approx λ)
\item
  \textbf{Polarization}: Electric field polarized in plane of loop
\item
  \textbf{Radiation Pattern}: Figure-eight pattern for small loops, more
  directional for large loops
\item
  \textbf{Applications}: Direction finding, AM reception, RFID tags
\item
  \textbf{Impedance}: High impedance for small loops, resonant for large
  loops
\end{itemize}

\end{solutionbox}
\begin{mnemonicbox}
``SPIRAL: Small Patterns In Receiving And Locating
signals''

\end{mnemonicbox}
\subsection*{Question 2(c) OR [7
marks]}\label{q2c}

\textbf{Describe radiation characteristics of non resonant wire antennas
and draw the radiation pattern of λ/2, 3λ/2 and 5λ/2 antenna}

\begin{solutionbox}

\textbf{Diagram: Radiation Patterns of Wire Antennas}

\begin{verbatim}
λ/2 Antenna Pattern:

    \^{}
    |     .{-.}
    |    /   {}
    |   |     |
{-{-}{-}{-}+{-}{-}{-}+{-}{-}{-}{-}{-}+{-}{-}{-}{-}{-}}
    |   |     |
    |    {   /}
    |     {{-}}
    v
    
3λ/2 Antenna Pattern:

    \^{}
    |    .{-. .{-}.}
    |   /   X   {}
    |  |  / {    |}
{-{-}{-}{-}+{-}{-}+{-}+{-}{-}{-}+{-}+{-}{-}{-}{-}}
    |  |  { /    |}
    |   {   X   /}
    |    {{-} {-}}
    v
    
5λ/2 Antenna Pattern:

    \^{}
    |  .{-. .{-}. .{-}.}
    | /   X   X   {}
    ||  / { /     |}
{-{-}{-}{-}++{-}+{-}{-}{-}+{-}{-}{-}+{-}+{-}{-}}
    ||  { /  /    |}
    | {   X   X   /}
    |  {{-} {-} {-}}
    v
\end{verbatim}

\begin{itemize}
\tightlist
\item
  \textbf{Non-Resonant Properties}: Traveling waves rather than standing
  waves
\item
  \textbf{λ/2 Antenna}: Simple bidirectional pattern, maximum radiation
  perpendicular to wire
\item
  \textbf{3λ/2 Antenna}: Multiple lobes, more complex pattern with side
  lobes
\item
  \textbf{5λ/2 Antenna}: Even more complex pattern with multiple main
  and side lobes
\item
  \textbf{Feed Point Impedance}: Non-resonant, typically requires
  impedance matching
\item
  \textbf{Bandwidth}: Wider than resonant antennas
\end{itemize}

\end{solutionbox}
\begin{mnemonicbox}
``TWIST: Traveling Waves Increase Side-lobe
Transmission''

\end{mnemonicbox}
\subsection*{Question 3(a) [3 marks]}\label{q3a}

\textbf{Write short note on micro strip (patch) antenna}

\begin{solutionbox}

\textbf{Diagram: Microstrip Patch Antenna Structure}

\begin{verbatim}
   +{-{-}{-}{-}{-}{-}{-}+}
   |       |
   | Patch |
   |       |
   +{-{-}{-}{-}{-}{-}{-}+}
   | Substrate
   |
   +{-{-}{-}{-}{-}{-}{-}{-}{-}{-}{-}{-}+}
   |Ground Plane|
   +{-{-}{-}{-}{-}{-}{-}{-}{-}{-}{-}{-}+}
\end{verbatim}

\begin{itemize}
\tightlist
\item
  \textbf{Structure}: Metal patch on dielectric substrate with ground
  plane below
\item
  \textbf{Size}: Typically half-wavelength in size
\item
  \textbf{Profile}: Low-profile, lightweight, easy to fabricate
\item
  \textbf{Radiation}: Radiates from patch edges, omnidirectional or
  directional patterns
\item
  \textbf{Applications}: Mobile devices, satellites, GPS receivers
\end{itemize}

\end{solutionbox}
\begin{mnemonicbox}
``PSALM: Patch Substrate Above Layer of Metal''

\end{mnemonicbox}
\subsection*{Question 3(b) [4 marks]}\label{q3b}

\textbf{Explain helical antenna and discuss its radiation
characteristics}

\begin{solutionbox}

\textbf{Diagram: Helical Antenna}

\begin{verbatim}
      \^{}
      |
    +{-{-}{-}+}
   /     {}
  +       +
 /|       |{}
+ |       | +
| |       | |  {-{-}}
+ |       | +
 {|       |/}
  +       +
   {     /}
    +{-{-}{-}+}
    
  Ground Plane
\end{verbatim}

\begin{itemize}
\tightlist
\item
  \textbf{Structure}: Conducting wire wound in helix shape above ground
  plane
\item
  \textbf{Modes}: Axial mode (end-fire) and normal mode (broadside)
\item
  \textbf{Axial Mode}: When circumference \approx λ, radiation along helix
  axis
\item
  \textbf{Normal Mode}: When circumference \textless\textless{} λ,
  radiation perpendicular to axis
\item
  \textbf{Polarization}: Circular polarization in axial mode
\item
  \textbf{Applications}: Satellite communication, space telemetry, radio
  astronomy
\end{itemize}

\end{solutionbox}
\begin{mnemonicbox}
``MOCHA: Mode Of Circular Helix Antennas''

\end{mnemonicbox}
\subsection*{Question 3(c) [7 marks]}\label{q3c}

\textbf{Explain horn antenna and discuss its radiation characteristics}

\begin{solutionbox}

\textbf{Diagram: Horn Antenna Types}

\begin{verbatim}
Pyramidal Horn:
    +{-{-}{-}{-}{-}{-}{-}{-}+}
    |        |
    |        |
+{-{-}{-}+        +{-}{-}{-}+}
|                |
+{-+{-}{-}{-}{-}{-}{-}{-}{-}{-}{-}{-}{-}+{-}+}
  |            |
  +{-{-}{-}{-}{-}{-}{-}{-}{-}{-}{-}{-}+}
  
Sectoral Horn:
    +{-{-}{-}{-}{-}{-}{-}{-}+}
    |        |
    |        |
+{-{-}{-}+        +{-}{-}{-}+}
|                |
+{-{-}{-}{-}{-}{-}{-}{-}{-}{-}{-}{-}{-}{-}{-}{-}+}

Conical Horn:
      +{-{-}{-}{-}+}
     /      {}
    /        {}
   /          {}
  +            +
  |            |
  +{-{-}{-}{-}{-}{-}{-}{-}{-}{-}{-}{-}+}
\end{verbatim}

\begin{itemize}
\tightlist
\item
  \textbf{Structure}: Waveguide with flared end to match impedance with
  free space
\item
  \textbf{Types}: Pyramidal (rectangular), sectoral (E-plane or
  H-plane), and conical (circular)
\item
  \textbf{Directivity}: 10-20 dB, higher than waveguide alone
\item
  \textbf{Bandwidth}: Very wide bandwidth
\item
  \textbf{Radiation Pattern}: Main lobe with small side lobes
\item
  \textbf{Applications}: Microwave communications, radar, satellite
  tracking, EMC testing
\item
  \textbf{Advantages}: High gain, simple construction, low VSWR
\end{itemize}

\end{solutionbox}
\begin{mnemonicbox}
``POWERS: Pyramidal Or Widening End Radiates
Strongly''

\end{mnemonicbox}
\subsection*{Question 3(a) OR [3
marks]}\label{q3a}

\textbf{Write short note on slot antenna}

\begin{solutionbox}

\textbf{Diagram: Slot Antenna}

\begin{verbatim}
+{-{-}{-}{-}{-}{-}{-}{-}{-}{-}{-}{-}{-}{-}{-}{-}{-}{-}{-}+}
|                   |
|    +{-{-}{-}{-}{-}{-}{-}+      |}
|    |       |      |
|    |  Slot |      |
|    |       |      |
|    +{-{-}{-}{-}{-}{-}{-}+      |}
|                   |
+{-{-}{-}{-}{-}{-}{-}{-}{-}{-}{-}{-}{-}{-}{-}{-}{-}{-}{-}+}
 Conducting Surface
\end{verbatim}

\begin{itemize}
\tightlist
\item
  \textbf{Structure}: Rectangular/circular slot cut in conducting
  surface
\item
  \textbf{Babinet's Principle}: Complementary to dipole antenna
\item
  \textbf{Radiation Pattern}: Similar to dipole but with E and H fields
  interchanged
\item
  \textbf{Polarization}: Electric field perpendicular to slot length
\item
  \textbf{Impedance}: High impedance compared to dipole
\item
  \textbf{Applications}: Aircraft, spacecraft, base stations, flush
  mounting
\end{itemize}

\end{solutionbox}
\begin{mnemonicbox}
``CROPS: Complementary Radiation Opening
Perpendicular to Surface''

\end{mnemonicbox}
\subsection*{Question 3(b) OR [4
marks]}\label{q3b}

\textbf{Explain parabolic reflector antenna and discuss its radiation
characteristics}

\begin{solutionbox}

\textbf{Diagram: Parabolic Reflector Antenna}

\begin{verbatim}
            \^{}
           / {}
          /   {}
         /     {}
        /       {}
       /         {}
      /           {}
     /             {}
    +{-{-}{-}{-}{-}{-}{-}{-}{-}{-}{-}{-}{-}{-}{-}+}
          |  |
          |  |
          +{-{-}+}
          Feed
\end{verbatim}

\begin{itemize}
\tightlist
\item
  \textbf{Structure}: Parabolic reflector with feed at focal point
\item
  \textbf{Working Principle}: Parallel rays from reflector converge at
  focal point
\item
  \textbf{Directivity}: Very high (30-40 dB)
\item
  \textbf{Beamwidth}: Very narrow, inversely proportional to diameter
\item
  \textbf{Efficiency}: 50-70\% depending on feed design
\item
  \textbf{Applications}: Satellite communications, radio astronomy,
  radar systems
\item
  \textbf{Types}: Prime focus, Cassegrain, offset feed
\end{itemize}

\end{solutionbox}
\begin{mnemonicbox}
``DISH: Directing Incoming Signals to Hub''

\end{mnemonicbox}
\subsection*{Question 3(c) OR [7
marks]}\label{q3c}

\textbf{Describe V and inverted V antenna}

\begin{solutionbox}

\textbf{Diagram: V and Inverted V Antennas}

\begin{verbatim}
V Antenna:
       /{}
      /  {}
     /    {}
    /      {}
   /        {}
  +          +
  |          |
  +{-{-}{-}{-}{-}{-}{-}{-}{-}{-}+}
    Feed Point

Inverted V Antenna:
  +{-{-}{-}{-}{-}{-}{-}{-}{-}{-}+}
  |          |
  +          +
   {        /}
    {      /}
     {    /}
      {  /}
       {/}
    Feed Point
\end{verbatim}


{\def\LTcaptype{none} % do not increment counter
\vspace{-5pt}
\captionof{table}{Comparison of V and Inverted V Antennas}
\vspace{-10pt}
\begin{longtable}[]{@{}
  >{\raggedright\arraybackslash}p{(\linewidth - 4\tabcolsep) * \real{0.2308}}
  >{\raggedright\arraybackslash}p{(\linewidth - 4\tabcolsep) * \real{0.2821}}
  >{\raggedright\arraybackslash}p{(\linewidth - 4\tabcolsep) * \real{0.4872}}@{}}
\toprule\noalign{}
\begin{minipage}[b]{\linewidth}\raggedright
Feature
\end{minipage} & \begin{minipage}[b]{\linewidth}\raggedright
V Antenna
\end{minipage} & \begin{minipage}[b]{\linewidth}\raggedright
Inverted V Antenna
\end{minipage} \\
\midrule\noalign{}
\endhead
\bottomrule\noalign{}
\endlastfoot
\textbf{Shape} & Arms extend upward from feed & Arms extend downward
from apex \\
\textbf{Angle} & Typically 90^\circ between arms & Typically 90-120^\circ between
arms \\
\textbf{Height} & Requires two tall supports & Requires one tall
support \\
\textbf{Impedance} & 40-50 ohms & 20-30 ohms \\
\textbf{Radiation Pattern} & Bidirectional & More omnidirectional \\
\textbf{Applications} & Directional HF communications & HF amateur
radio, limited space \\
\end{longtable}
}

\end{solutionbox}
\begin{mnemonicbox}
``VIVA: V Is Vertical Arrangement, Inverted V Aims
downward''

\end{mnemonicbox}
\subsection*{Question 4(a) [3 marks]}\label{q4a}

\textbf{Define: (1) Reflection, (2) Refraction and (3) Diffraction}

\begin{solutionbox}


{\def\LTcaptype{none} % do not increment counter
\vspace{-5pt}
\captionof{table}{Wave Phenomenon Definitions}
\vspace{-10pt}
\begin{longtable}[]{@{}
  >{\raggedright\arraybackslash}p{(\linewidth - 2\tabcolsep) * \real{0.5000}}
  >{\raggedright\arraybackslash}p{(\linewidth - 2\tabcolsep) * \real{0.5000}}@{}}
\toprule\noalign{}
\begin{minipage}[b]{\linewidth}\raggedright
Phenomenon
\end{minipage} & \begin{minipage}[b]{\linewidth}\raggedright
Definition
\end{minipage} \\
\midrule\noalign{}
\endhead
\bottomrule\noalign{}
\endlastfoot
\textbf{Reflection} & Bouncing back of waves when they strike the
boundary between two media \\
\textbf{Refraction} & Bending of waves when they pass from one medium to
another with different propagation velocity \\
\textbf{Diffraction} & Bending of waves around obstacles or through
openings \\
\end{longtable}
}

\end{solutionbox}
\begin{mnemonicbox}
``RRD: Rebounding, Redirecting, Detour''

\end{mnemonicbox}
\subsection*{Question 4(b) [4 marks]}\label{q4b}

\textbf{List HAM radio application for communication}

\begin{solutionbox}


{\def\LTcaptype{none} % do not increment counter
\vspace{-5pt}
\captionof{table}{HAM Radio Applications}
\vspace{-10pt}
\begin{longtable}[]{@{}
  >{\raggedright\arraybackslash}p{(\linewidth - 2\tabcolsep) * \real{0.5000}}
  >{\raggedright\arraybackslash}p{(\linewidth - 2\tabcolsep) * \real{0.5000}}@{}}
\toprule\noalign{}
\begin{minipage}[b]{\linewidth}\raggedright
Application
\end{minipage} & \begin{minipage}[b]{\linewidth}\raggedright
Description
\end{minipage} \\
\midrule\noalign{}
\endhead
\bottomrule\noalign{}
\endlastfoot
\textbf{Emergency Communication} & Disaster relief when normal
infrastructure fails \\
\textbf{DX Communication} & Long-distance international
communications \\
\textbf{Satellite Communication} & Using amateur radio satellites for
extended range \\
\textbf{Digital Modes} & Text/data transmission (RTTY, PSK31, FT8) \\
\textbf{Morse Code} & Traditional CW communication \\
\textbf{Voice Communication} & Using SSB, FM, AM modulation \\
\textbf{Public Service} & Supporting events like marathons, parades \\
\end{longtable}
}

\end{solutionbox}
\begin{mnemonicbox}
``EDSDMVP: Emergency DX Satellite Digital Morse Voice
Public-service''

\end{mnemonicbox}
\subsection*{Question 4(c) [7 marks]}\label{q4c}

\textbf{Explain ionosphere's layers and sky wave propagation}

\begin{solutionbox}

\textbf{Diagram: Ionospheric Layers and Sky Wave Propagation}

\begin{center}
\textbf{Mermaid Diagram (Code)}
\begin{verbatim}
{Shaded}
{Highlighting}[]
graph TD
    A[Transmitter] {-{-}{}|Sky Wave| B[F2 Layer: 250{-}400 km]}
    A {-{-}{}|Sky Wave| C[F1 Layer: 150{-}250 km]}
    A {-{-}{}|Sky Wave| D[E Layer: 90{-}150 km]}
    A {-{-}{}|Sky Wave| E[D Layer: 60{-}90 km]}
    B {-{-}{}|Reflection| F[Receiver at long distance]}
    C {-{-}{}|Reflection| F}
    D {-{-}{}|Reflection/Absorption| F}
    E {-{-}{}|Absorption| G[Signal Loss]}
{Highlighting}
{Shaded}
\end{verbatim}
\end{center}

\begin{itemize}
\tightlist
\item
  \textbf{D Layer (60-90 km)}: Exists during daylight, absorbs HF
  signals below 10 MHz
\item
  \textbf{E Layer (90-150 km)}: Reflects signals 3-5 MHz, stronger
  during day, sporadic-E in summer
\item
  \textbf{F1 Layer (150-250 km)}: Daytime only, merges with F2 at night
\item
  \textbf{F2 Layer (250-400 km)}: Main reflecting layer, enables
  long-distance HF communication
\item
  \textbf{Propagation Factors}:

  \begin{itemize}
  \tightlist
  \item
    \textbf{Virtual Height}: Apparent height of reflection
  \item
    \textbf{Critical Frequency}: Maximum frequency reflected vertically
  \item
    \textbf{MUF}: Maximum Usable Frequency for a given distance
  \item
    \textbf{Skip Distance}: Minimum distance for sky wave reception
  \end{itemize}
\end{itemize}

\end{solutionbox}
\begin{mnemonicbox}
``DEFV: D-absorbs, E-reflects, F-provides
Very-long-distance''

\end{mnemonicbox}
\subsection*{Question 4(a) OR [3
marks]}\label{q4a}

\textbf{Define: (1) MUF, (2) LUF and (3) Skip distance}

\begin{solutionbox}


{\def\LTcaptype{none} % do not increment counter
\vspace{-5pt}
\captionof{table}{Ionospheric Propagation Terms}
\vspace{-10pt}
\begin{longtable}[]{@{}
  >{\raggedright\arraybackslash}p{(\linewidth - 2\tabcolsep) * \real{0.3333}}
  >{\raggedright\arraybackslash}p{(\linewidth - 2\tabcolsep) * \real{0.6667}}@{}}
\toprule\noalign{}
\begin{minipage}[b]{\linewidth}\raggedright
Term
\end{minipage} & \begin{minipage}[b]{\linewidth}\raggedright
Definition
\end{minipage} \\
\midrule\noalign{}
\endhead
\bottomrule\noalign{}
\endlastfoot
\textbf{MUF (Maximum Usable Frequency)} & Highest frequency that can be
reflected by ionosphere for a given distance and time \\
\textbf{LUF (Lowest Usable Frequency)} & Lowest frequency that provides
adequate signal strength for communication \\
\textbf{Skip Distance} & Minimum distance from transmitter where sky
wave returns to Earth \\
\end{longtable}
}

\end{solutionbox}
\begin{mnemonicbox}
``MLS: Maximum-highest, Lowest-minimum,
Skip-nearest''

\end{mnemonicbox}
\subsection*{Question 4(b) OR [4
marks]}\label{q4b}

\textbf{List HAM radio digital modes of communication}

\begin{solutionbox}


{\def\LTcaptype{none} % do not increment counter
\vspace{-5pt}
\captionof{table}{HAM Radio Digital Modes}
\vspace{-10pt}
\begin{longtable}[]{@{}
  >{\raggedright\arraybackslash}p{(\linewidth - 2\tabcolsep) * \real{0.4667}}
  >{\raggedright\arraybackslash}p{(\linewidth - 2\tabcolsep) * \real{0.5333}}@{}}
\toprule\noalign{}
\begin{minipage}[b]{\linewidth}\raggedright
Digital Mode
\end{minipage} & \begin{minipage}[b]{\linewidth}\raggedright
Characteristics
\end{minipage} \\
\midrule\noalign{}
\endhead
\bottomrule\noalign{}
\endlastfoot
\textbf{FT8} & Weak signal, narrow bandwidth, automated exchanges \\
\textbf{PSK31} & Keyboard-to-keyboard text communication, narrow
bandwidth \\
\textbf{RTTY} & Radio teletype, robust older digital mode \\
\textbf{SSTV} & Slow Scan Television for image transmission \\
\textbf{JT65/JT9} & Very weak signal modes for extreme distance \\
\textbf{Packet Radio} & Computer-based data transmission with error
correction \\
\textbf{APRS} & Automatic Position Reporting System with GPS \\
\textbf{Digital Voice} & DMR, D-STAR, Fusion, P25 digital voice
protocols \\
\end{longtable}
}

\end{solutionbox}
\begin{mnemonicbox}
``FIRST PAD: FT8 Is RTTY SSTV Then Packet APRS
Digital-voice''

\end{mnemonicbox}
\subsection*{Question 4(c) OR [7
marks]}\label{q4c}

\textbf{Explain space wave propagation}

\begin{solutionbox}

\textbf{Diagram: Space Wave Propagation}

\begin{center}
\textbf{Mermaid Diagram (Code)}
\begin{verbatim}
{Shaded}
{Highlighting}[]
graph LR
    A[Transmitter] {-{-}{}|Direct Wave| B[Receiver]}
    A {-{-}{}|Ground Reflected Wave| B}
    A {-{-}{}|Tropospheric Scatter| C[Extended Range Receiver]}
    A {-{-}{}|Ducting| D[Very Extended Range]}

    subgraph Troposphere
    A
    B
    C
    D
    E[Temperature Inversion Layer]
    end
    
    A {-{-}{}|Follows| E {-}{-}{}|Waveguide Effect| D}
{Highlighting}
{Shaded}
\end{verbatim}
\end{center}

\begin{itemize}
\tightlist
\item
  \textbf{Components}: Direct wave, ground-reflected wave, tropospheric
  waves
\item
  \textbf{Line of Sight}: Primary mechanism limited by Earth's curvature
\item
  \textbf{Frequency Range}: VHF, UHF, and microwave frequencies
\item
  \textbf{Tropospheric Scattering}: Forward scattering extends range
  beyond horizon
\item
  \textbf{Duct Propagation}:

  \begin{itemize}
  \tightlist
  \item
    Occurs in temperature inversion layers
  \item
    Creates waveguide effect trapping signals
  \item
    Enables very long distance VHF/UHF propagation
  \end{itemize}
\item
  \textbf{Factors Affecting}: Antenna height, terrain, atmospheric
  conditions
\item
  \textbf{Applications}: TV broadcasting, microwave links, mobile
  communications
\end{itemize}

\end{solutionbox}
\begin{mnemonicbox}
``DRIFT: Direct Reflection Inversion Forward
Tropospheric''

\end{mnemonicbox}
\subsection*{Question 5(a) [3 marks]}\label{q5a}

\textbf{Define: (1) Beam area (2) Beam efficiency, and (3) Effective
aperture}

\begin{solutionbox}


{\def\LTcaptype{none} % do not increment counter
\vspace{-5pt}
\captionof{table}{Antenna Beam Parameters}
\vspace{-10pt}
\begin{longtable}[]{@{}
  >{\raggedright\arraybackslash}p{(\linewidth - 2\tabcolsep) * \real{0.4783}}
  >{\raggedright\arraybackslash}p{(\linewidth - 2\tabcolsep) * \real{0.5217}}@{}}
\toprule\noalign{}
\begin{minipage}[b]{\linewidth}\raggedright
Parameter
\end{minipage} & \begin{minipage}[b]{\linewidth}\raggedright
Definition
\end{minipage} \\
\midrule\noalign{}
\endhead
\bottomrule\noalign{}
\endlastfoot
\textbf{Beam Area} & Solid angle through which all power radiated by
antenna would flow if radiation intensity was constant \\
\textbf{Beam Efficiency} & Ratio of power in main beam to total radiated
power \\
\textbf{Effective Aperture} & Area over which antenna captures RF
energy, related to gain \\
\end{longtable}
}

\end{solutionbox}
\begin{mnemonicbox}
``BEA: Beam Efficiency Aperture''

\end{mnemonicbox}
\subsection*{Question 5(b) [4 marks]}\label{q5b}

\textbf{Describe need of smart antenna}

\begin{solutionbox}

\textbf{Diagram: Smart Antenna Benefits}

\begin{center}
\textbf{Mermaid Diagram (Code)}
\begin{verbatim}
{Shaded}
{Highlighting}[]
graph TD
    A[Smart Antenna] {-{-}{}|Provides| B[Increased Capacity]}
    A {-{-}{}|Provides| C[Enhanced Coverage]}
    A {-{-}{}|Reduces| D[Interference]}
    A {-{-}{}|Improves| E[Signal Quality]}
    A {-{-}{}|Saves| F[Battery Power]}
    A {-{-}{}|Enables| G[Spatial Multiplexing]}
{Highlighting}
{Shaded}
\end{verbatim}
\end{center}

\begin{itemize}
\tightlist
\item
  \textbf{Capacity Improvement}: Serves more users in same bandwidth
\item
  \textbf{Coverage Enhancement}: Extends range by focusing energy
\item
  \textbf{Interference Reduction}: Nulls out unwanted signals
\item
  \textbf{Signal Quality}: Better SNR through beam focusing
\item
  \textbf{Energy Efficiency}: Lower transmit power requirements
\item
  \textbf{Spatial Multiplexing}: Multiple data streams in same frequency
\item
  \textbf{Adaptive Operation}: Dynamically adapts to changing
  environment
\end{itemize}

\end{solutionbox}
\begin{mnemonicbox}
``PRECISE: Power Reduction, Enhanced Coverage,
Interference Suppression, Enhanced Signal''

\end{mnemonicbox}
\subsection*{Question 5(c) [7 marks]}\label{q5c}

\textbf{Draw the DTH Receiver indoor and outdoor black diagram and
discuss its functions}

\begin{solutionbox}

\textbf{Diagram: DTH System Block Diagram}

\begin{center}
\textbf{Mermaid Diagram (Code)}
\begin{verbatim}
{Shaded}
{Highlighting}[]
graph LR
    subgraph Outdoor Unit
    A[Dish Antenna] {-{-}{}|Collects| B[LNB {-} Low Noise Block]}
    end

    subgraph Indoor Unit
    C[Tuner] {-{-}{} D[Demodulator]}
    D {-{-}{} E[Decoder]}
    E {-{-}{} F[MPEG Processor]}
    F {-{-}{} G[Video/Audio Output]}
    H[Smart Card] {-{-}{} E}
    I[User Interface] {-{-}{} E}
    end
    
    B {-{-}{}|Coaxial Cable| C}
{Highlighting}
{Shaded}
\end{verbatim}
\end{center}

\textbf{Outdoor Unit Components and Functions:}

\begin{itemize}
\tightlist
\item
  \textbf{Dish Antenna}: Collects satellite signals, typically 45-90 cm
  diameter
\item
  \textbf{LNB (Low Noise Block)}:

  \begin{itemize}
  \tightlist
  \item
    Converts high frequency satellite signals (10-12 GHz) to lower IF
    frequencies (950-2150 MHz)
  \item
    Amplifies weak signals with minimal noise
  \item
    Contains local oscillator and polarization selection
  \end{itemize}
\end{itemize}

\textbf{Indoor Unit Components and Functions:}

\begin{itemize}
\tightlist
\item
  \textbf{Tuner}: Selects desired transponder frequency
\item
  \textbf{Demodulator}: Extracts digital signal from modulated carrier
\item
  \textbf{Decoder}: Decrypts encrypted channels using smart card
  authorization
\item
  \textbf{MPEG Processor}: Decompresses video/audio data streams
\item
  \textbf{User Interface}: On-screen menus, program guide, channel
  selection
\item
  \textbf{Smart Card}: Contains subscription details and decryption keys
\end{itemize}

\end{solutionbox}
\begin{mnemonicbox}
``COLD-TDUMS: Collection, Oscillator, Low-noise,
Downconversion - Tuner Demodulator Unscrambler MPEG Smart-card''

\end{mnemonicbox}
\subsection*{Question 5(a) OR [3
marks]}\label{q5a}

\textbf{Define: (1) Antenna, (2) Folded dipole, and (3) Antenna array}

\begin{solutionbox}


{\def\LTcaptype{none} % do not increment counter
\vspace{-5pt}
\captionof{table}{Antenna Definitions}
\vspace{-10pt}
\begin{longtable}[]{@{}
  >{\raggedright\arraybackslash}p{(\linewidth - 2\tabcolsep) * \real{0.3333}}
  >{\raggedright\arraybackslash}p{(\linewidth - 2\tabcolsep) * \real{0.6667}}@{}}
\toprule\noalign{}
\begin{minipage}[b]{\linewidth}\raggedright
Term
\end{minipage} & \begin{minipage}[b]{\linewidth}\raggedright
Definition
\end{minipage} \\
\midrule\noalign{}
\endhead
\bottomrule\noalign{}
\endlastfoot
\textbf{Antenna} & Device that converts electrical energy to radio waves
and vice versa \\
\textbf{Folded Dipole} & Dipole with ends folded back and connected,
forming a loop with higher impedance \\
\textbf{Antenna Array} & Multiple antennas arranged in specific pattern
for improved directivity/gain \\
\end{longtable}
}

\end{solutionbox}
\begin{mnemonicbox}
``AFA: Antenna Folded Array''

\end{mnemonicbox}
\subsection*{Question 5(b) OR [4
marks]}\label{q5b}

\textbf{Describe application of smart antenna}

\begin{solutionbox}


{\def\LTcaptype{none} % do not increment counter
\vspace{-5pt}
\captionof{table}{Smart Antenna Applications}
\vspace{-10pt}
\begin{longtable}[]{@{}
  >{\raggedright\arraybackslash}p{(\linewidth - 2\tabcolsep) * \real{0.5000}}
  >{\raggedright\arraybackslash}p{(\linewidth - 2\tabcolsep) * \real{0.5000}}@{}}
\toprule\noalign{}
\begin{minipage}[b]{\linewidth}\raggedright
Application
\end{minipage} & \begin{minipage}[b]{\linewidth}\raggedright
Description
\end{minipage} \\
\midrule\noalign{}
\endhead
\bottomrule\noalign{}
\endlastfoot
\textbf{Mobile Communications} & Increases capacity, reduces
interference in cellular networks \\
\textbf{Base Stations} & Sector-specific coverage, adaptive
beamforming \\
\textbf{MIMO Systems} & Multiple-input-multiple-output for spatial
multiplexing \\
\textbf{Radar Systems} & Improved target detection and tracking \\
\textbf{Satellite Communications} & Spot beam generation, interference
mitigation \\
\textbf{Wi-Fi Networks} & Enhanced range and throughput for wireless
LANs \\
\textbf{IoT Networks} & Low-power, long-range connectivity for IoT
devices \\
\end{longtable}
}

\end{solutionbox}
\begin{mnemonicbox}
``MBMRSWI: Mobile Base MIMO Radar Satellite Wi-Fi
IoT''

\end{mnemonicbox}
\subsection*{Question 5(c) OR [7
marks]}\label{q5c}

\textbf{Explain Terrestrial mobile communication antennas and also
discuss about base station and mobile station antennas}

\begin{solutionbox}

\textbf{Diagram: Mobile Communication Antenna Types}

\begin{center}
\textbf{Mermaid Diagram (Code)}
\begin{verbatim}
{Shaded}
{Highlighting}[]
graph TD
    A[Terrestrial Mobile Antennas] {-{-}{} B[Base Station Antennas]}
    A {-{-}{} C[Mobile Station Antennas]}

    B {-{-}{} D[Panel Antennas]}
    B {-{-}{} E[Sector Antennas]}
    B {-{-}{} F[Omnidirectional Antennas]}
    B {-{-}{} G[Smart Antennas]}
    
    C {-{-}{} H[Whip Antennas]}
    C {-{-}{} I[Helical Antennas]}
    C {-{-}{} J[Planar Inverted{-}F Antennas]}
    C {-{-}{} K[Internal PCB Antennas]}
{Highlighting}
{Shaded}
\end{verbatim}
\end{center}

\textbf{Base Station Antennas:}

\begin{itemize}
\tightlist
\item
  \textbf{Panel/Sector Antennas}: 65^\circ-120^\circ coverage per sector,
  typically three sectors per site
\item
  \textbf{Characteristics}:

  \begin{itemize}
  \tightlist
  \item
    High gain (10-18 dBi)
  \item
    Vertical polarization
  \item
    Downtilt capability (mechanical or electrical)
  \item
    Multi-band operation
  \end{itemize}
\item
  \textbf{Height}: Mounted on towers 15-50m high for maximum coverage
\item
  \textbf{Pattern Control}: Minimizes interference to adjacent cells
\end{itemize}

\textbf{Mobile Station Antennas:}

\begin{itemize}
\tightlist
\item
  \textbf{External Antennas}: Less common today, mainly for vehicles or
  rural areas

  \begin{itemize}
  \tightlist
  \item
    Whip antennas (¼λ monopoles)
  \item
    Helical designs for flexibility
  \end{itemize}
\item
  \textbf{Internal Antennas}: Now dominant in handsets

  \begin{itemize}
  \tightlist
  \item
    PIFA (Planar Inverted-F Antenna)s
  \item
    PCB trace antennas
  \item
    Characteristics:

    \begin{itemize}
    \tightlist
    \item
      Small size
    \item
      Multi-band operation
    \item
      Omnidirectional pattern
    \item
      Lower efficiency (typically -3 to -6 dBi)
    \end{itemize}
  \end{itemize}
\end{itemize}

\end{solutionbox}
\begin{mnemonicbox}
``BEST-POMME: Base-station External Sector Tower -
Portable Omnidirectional Multi-band Mobile Embedded''

\end{mnemonicbox}

\end{document}
