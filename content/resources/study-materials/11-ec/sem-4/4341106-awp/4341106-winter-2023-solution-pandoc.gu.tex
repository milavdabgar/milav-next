\documentclass[10pt,a4paper]{article}

% content/resources/templates/preamble.tex
\usepackage[margin=0.6in]{geometry}
\author{Milav Dabgar}
\usepackage{amsmath,amssymb,amsthm}
\usepackage{booktabs}
\usepackage{multirow}
\usepackage{xcolor}
\usepackage{tcolorbox}
\tcbuselibrary{breakable,skins}
\usepackage[colorlinks=true,linkcolor=blue]{hyperref}
\usepackage{titlesec}
\usepackage{enumitem}
\usepackage{tikz}
\usepackage{pgfplots}
\usepackage{circuitikz}
\usepackage[version=4]{mhchem}
\usepackage{longtable}
\usepackage{array}
\usepackage{float}
\usepackage{caption}
\usepackage{listings}

\lstset{
  basicstyle=\small\ttfamily,
  breaklines=true,
  breakatwhitespace=false,
  postbreak=\mbox{\textcolor{red}{$\hookrightarrow$}\space},
  float=false,
  numbers=left,
  numberstyle=\tiny\color{gray},
  numbersep=10pt,
  xleftmargin=2em,
  keywordstyle=\color{blue},
  commentstyle=\color{green!60!black},
  stringstyle=\color{purple},
  backgroundcolor=\color{gray!5},
  showstringspaces=false,
  tabsize=2,
  captionpos=b,
  keepspaces=true,
  columns=flexible
}

\pgfplotsset{compat=1.18}
\usetikzlibrary{shapes,arrows,positioning,calc,patterns,decorations.pathmorphing,decorations.markings,arrows.meta}

% Color scheme
\definecolor{headcolor}{RGB}{0,102,204}
\definecolor{keycolor}{RGB}{220,20,60}
\definecolor{solutioncolor}{RGB}{34,139,34}
\definecolor{mnemoniccolor}{RGB}{148,0,211}
\definecolor{codecolor}{RGB}{0,0,100}

% Spacing
\setlength{\parskip}{3pt}
\setlist[itemize]{nosep}
\setlist[enumerate]{nosep}

% Title formatting
\titleformat{\section}{\Large\bfseries\color{headcolor}}{\thesection}{1em}{}
\titleformat{\subsection}{\large\bfseries\color{headcolor}}{\thesubsection}{1em}{}

% Pandoc tightlist compatibility
\providecommand{\tightlist}{%
  \setlength{\itemsep}{0pt}\setlength{\parskip}{0pt}}

% Pandoc longtable compatibility
\newcounter{none}
\def\thenone{}


% content/resources/templates/gujarati-boxes.tex
\usepackage{fontspec}
\usepackage{polyglossia}

% Set Gujarati as main language (document is primarily in Gujarati)
% Note: gloss-gujarati.ldf doesn't exist in polyglossia, but it will use hyphenation patterns
\setdefaultlanguage{gujarati}
\setotherlanguage{english}

% Configure Gujarati font properly
% Use Language=Default to prevent polyglossia from trying to add language-specific features
% that don't exist for Gujarati, which causes "empty feature" warnings
\newfontfamily\gujaratifont[Script=Gujarati,AutoFakeBold=2.5,AutoFakeSlant=0.3]{Noto Sans Gujarati}
\setmainfont[Script=Gujarati,AutoFakeBold=2.5,AutoFakeSlant=0.3]{Noto Sans Gujarati}
% Use Noto Sans Gujarati for monospace to support Gujarati in text
\setmonofont[Scale=0.9]{Noto Sans Gujarati}

% Configure English to use the same font
\newfontfamily\englishfont[Script=Gujarati,AutoFakeBold=2.5,AutoFakeSlant=0.3]{Noto Sans Gujarati}

% Translations for polyglossia
\gappto\captionsgujarati{
  \renewcommand{\tablename}{કોષ્ટક}
  \renewcommand{\figurename}{આકૃતિ}
}

% Helper for TikZ nodes to ensure Gujarati font
\newcommand{\gu}[1]{{\gujaratifont #1}}

% Custom environments
\newtcolorbox{solutionbox}{
    breakable,
    enhanced,
    colback=solutioncolor!5!white,
    colframe=solutioncolor!75!black,
    fonttitle=\bfseries,
    title=જવાબ
}

\newtcolorbox{solutionboxnobreak}{
 colback=solutioncolor!5!white,
 colframe=solutioncolor!75!black,
 fonttitle=\bfseries,
 title=જવાબ
}

\newtcolorbox{keyformula}{
 breakable,
 enhanced,
 colback=keycolor!5!white,
 colframe=keycolor!75!black,
 fonttitle=\bfseries,
 title=રાસાયણિક સમીકરણ/સૂત્ર
}

\newtcolorbox{mnemonicbox}{
 breakable,
 enhanced,
 colback=mnemoniccolor!5!white,
 colframe=mnemoniccolor!75!black,
 fonttitle=\bfseries,
 title=મેમરી ટ્રીક
}


\begin{document}

\begin{center}
{\Huge\bfseries\color{headcolor} Subject Name (Gujarati)}\\[5pt]
{\LARGE 4341106 -- Winter 2023}\\[3pt]
{\large Semester 1 Study Material}\\[3pt]
{\normalsize\textit{Detailed Solutions and Explanations}}
\end{center}

\vspace{10pt}

\subsection*{પ્રશ્ન 1(અ) [3
ગુણ]}\label{uxaaauxab0uxab6uxaa8-1uxa85-3-uxa97uxaa3}

\textbf{વ્યાખ્યાયિત કરો: (1) ડાયરેક્ટિવિટી, (2) ગેઇન, અને (3) HPBW}

\begin{solutionbox}


{\def\LTcaptype{none} % do not increment counter
\vspace{-5pt}
\captionof{table}{મહત્વના અન્ટેના પરિમાણો}
\vspace{-10pt}
\begin{longtable}[]{@{}
  >{\raggedright\arraybackslash}p{(\linewidth - 2\tabcolsep) * \real{0.4783}}
  >{\raggedright\arraybackslash}p{(\linewidth - 2\tabcolsep) * \real{0.5217}}@{}}
\toprule\noalign{}
\begin{minipage}[b]{\linewidth}\raggedright
પરિમાણ
\end{minipage} & \begin{minipage}[b]{\linewidth}\raggedright
વ્યાખ્યા
\end{minipage} \\
\midrule\noalign{}
\endhead
\bottomrule\noalign{}
\endlastfoot
\textbf{ડાયરેક્ટિવિટી} & અન્ટેનાની મહત્તમ વિકિરણ તીવ્રતા અને સરેરાશ વિકિરણ
તીવ્રતાનો ગુણોત્તર \\
\textbf{ગેઇન} & ચોક્કસ દિશામાં વિકિરિત થતી શક્તિ અને આઇસોટ્રોપિક અન્ટેના દ્વારા
વિકિરિત થતી શક્તિનો ગુણોત્તર \\
\textbf{HPBW (હાફ પાવર બીમ વિડ્થ)} & કોણીય પહોળાઈ જ્યાં વિકિરણની તીવ્રતા
મહત્તમ મૂલ્યના અડધી (3dB ઓછી) હોય છે \\
\end{longtable}
}

\end{solutionbox}
\begin{mnemonicbox}
``DGH: દિશા ગેઇન હાફ-પાવર''

\end{mnemonicbox}
\subsection*{પ્રશ્ન 1(બ) [4
ગુણ]}\label{uxaaauxab0uxab6uxaa8-1uxaac-4-uxa97uxaa3}

\textbf{ઇલેક્ટ્રોમેગ્નેટિક તરંગોના ગુણધર્મોની સૂચિ બનાવો}

\begin{solutionbox}


{\def\LTcaptype{none} % do not increment counter
\vspace{-5pt}
\captionof{table}{ઇલેક્ટ્રોમેગ્નેટિક તરંગોના ગુણધર્મો}
\vspace{-10pt}
\begin{longtable}[]{@{}
  >{\raggedright\arraybackslash}p{(\linewidth - 2\tabcolsep) * \real{0.4348}}
  >{\raggedright\arraybackslash}p{(\linewidth - 2\tabcolsep) * \real{0.5652}}@{}}
\toprule\noalign{}
\begin{minipage}[b]{\linewidth}\raggedright
ગુણધર્મ
\end{minipage} & \begin{minipage}[b]{\linewidth}\raggedright
વર્ણન
\end{minipage} \\
\midrule\noalign{}
\endhead
\bottomrule\noalign{}
\endlastfoot
\textbf{ટ્રાન્સવર્સ તરંગો} & ઇલેક્ટ્રિક અને મેગ્નેટિક ક્ષેત્રો પ્રસરણની દિશાને લંબરૂપે હોય
છે \\
\textbf{વેગ} & નિર્વાતમાં પ્રકાશનો વેગ (3\times10\^{}8 m/s) \\
\textbf{માધ્યમની જરૂર નથી} & યાંત્રિક તરંગોથી વિપરીત, નિર્વાતમાં પણ પ્રવાસ કરી
શકે છે \\
\textbf{ધ્રુવીકરણ} & ઇલેક્ટ્રિક ક્ષેત્ર વેક્ટરની દિશા \\
\textbf{ઊર્જા વહન} & અવકાશમાં ઊર્જા વહન કરે છે \\
\textbf{પરાવર્તન/વક્રીભવન} & સીમાઓ પર પરાવર્તિત અને વક્રીભૂત થઈ શકે છે \\
\textbf{વ્યતિકરણ/વિવર્તન} & તરંગ જેવા ગુણધર્મો દર્શાવે છે \\
\end{longtable}
}

\end{solutionbox}
\begin{mnemonicbox}
``TVNPER: ટ્રાન્સવર્સ વેગ નો-માધ્યમ પોલરાઇઝ્ડ એનર્જી
રિફ્લેક્શન''

\end{mnemonicbox}
\subsection*{પ્રશ્ન 1(ક) [7
ગુણ]}\label{uxaaauxab0uxab6uxaa8-1uxa95-7-uxa97uxaa3}

\textbf{ઈલેક્ટ્રોમેગ્નેટિક તરંગોના નિર્માણનો ભૌતિક ખ્યાલ સમજાવો}

\begin{solutionbox}

\textbf{આકૃતિ: ઇલેક્ટ્રોમેગ્નેટિક તરંગનું નિર્માણ}

\begin{center}
\textbf{Mermaid Diagram (Code)}
\begin{verbatim}
{Shaded}
{Highlighting}[]
graph LR
    A[ત્વરિત આવેશ] {-{-}{}|ઉત્પન્ન કરે છે| B[સમય{-}ભિન્ન ઇલેક્ટ્રિક ક્ષેત્ર]}
    B {-{-}{}|ઉત્પન્ન કરે છે| C[સમય{-}ભિન્ન ચુંબકીય ક્ષેત્ર]}
    C {-{-}{}|ઉત્પન્ન કરે છે| D[સમય{-}ભિન્ન ઇલેક્ટ્રિક ક્ષેત્ર]}
    D {-{-}{} C}
    C {-{-}{} E[સ્વ{-}પ્રસરણશીલ EM તરંગ]}
{Highlighting}
{Shaded}
\end{verbatim}
\end{center}

\begin{itemize}
\tightlist
\item
  \textbf{આવેશનું ત્વરણ}: જ્યારે ઇલેક્ટ્રિક આવેશો ત્વરિત થાય છે, ત્યારે તેઓ બદલાતા
  ઇલેક્ટ્રિક ક્ષેત્રો ઉત્પન્ન કરે છે
\item
  \textbf{ક્ષેત્ર જોડાણ}: બદલાતું ઇલેક્ટ્રિક ક્ષેત્ર બદલાતું ચુંબકીય ક્ષેત્ર ઉત્પન્ન કરે છે અને
  તેનાથી ઉલટું પણ થાય છે
\item
  \textbf{સ્વ-પ્રસરણ}: ક્ષેત્રોના ચક્રીય નિર્માણથી તરંગો કોઈ માધ્યમ વિના પ્રવાસ
  કરી શકે છે
\item
  \textbf{ક્ષેત્ર અભિમુખતા}: ઇલેક્ટ્રિક અને ચુંબકીય ક્ષેત્રો એકબીજાને અને પ્રસરણની દિશાને
  લંબરૂપ હોય છે
\item
  \textbf{ઊર્જા વહન}: તરંગ પ્રસરણ સાથે ઊર્જા ઇલેક્ટ્રિક અને ચુંબકીય ક્ષેત્રો વચ્ચે
  વારાફરતી આવે છે
\end{itemize}

\end{solutionbox}
\begin{mnemonicbox}
``CASES: ચાર્જ એક્સેલરેશન સેલ્ફ-પ્રોપેગેટ્સ ઇલેક્ટ્રો-મેગ્નેટિક
સિગ્નલ્સ''

\end{mnemonicbox}
\subsection*{પ્રશ્ન 1(ક) અથવા [7
ગુણ]}\label{uxaaauxab0uxab6uxaa8-1uxa95-uxa85uxaa5uxab5-7-uxa97uxaa3}

\textbf{સેન્ટર ફેડ ડાયપોલ માંથી ઇલેક્ટ્રોમેગ્નેટિક ક્ષેત્ર કેવી રીતે વિકિરણ થાય છે તે
સમજાવો}

\begin{solutionbox}

\textbf{આકૃતિ: સેન્ટર-ફેડ ડાયપોલમાંથી ક્ષેત્ર વિકિરણ}

\begin{center}
\textbf{Mermaid Diagram (Code)}
\begin{verbatim}
{Shaded}
{Highlighting}[]
graph LR
    A[આવર્તક પ્રવાહ ઇનપુટ] {-{-}{}|ઉત્પન્ન કરે છે| B[આવર્તક આવેશો]}
    B {-{-}{}|ઉત્પન્ન કરે છે| C[સમય{-}ભિન્ન ઇલેક્ટ્રિક ક્ષેત્ર]}
    C {-{-}{}|ઉત્પન્ન કરે છે| D[સમય{-}ભિન્ન ચુંબકીય ક્ષેત્ર]}
    C {-{-}{} E[EM તરંગ વિકિરણ]}
    D {-{-}{} E}
{Highlighting}
{Shaded}
\end{verbatim}
\end{center}

\begin{itemize}
\tightlist
\item
  \textbf{સેન્ટર ફીડિંગ}: ડાયપોલના કેન્દ્રમાં AC સિગ્નલ આપવાથી આવર્તક પ્રવાહ ઉત્પન્ન
  થાય છે
\item
  \textbf{આવેશ વિતરણ}: પ્રવાહ ડાયપોલના છેડા પર વિરુદ્ધ આવેશો ઉત્પન્ન કરે છે જે AC
  આવૃત્તિ સાથે બદલાય છે
\item
  \textbf{ક્ષેત્ર નિર્માણ}: આવર્તક આવેશો સમય-ભિન્ન ઇલેક્ટ્રિક ક્ષેત્ર ઉત્પન્ન કરે છે
\item
  \textbf{ચુંબકીય જોડાણ}: સમય-ભિન્ન ઇલેક્ટ્રિક ક્ષેત્ર લંબરૂપ ચુંબકીય ક્ષેત્ર ઉત્પન્ન કરે છે
\item
  \textbf{નજીક/દૂરના ક્ષેત્રો}: ડાયપોલની નજીક, ક્ષેત્રો જટિલ હોય છે; ડાયપોલથી
  દૂર, ક્ષેત્રો એકસમાન વિકિરણ પેટર્ન બનાવે છે
\item
  \textbf{વિકિરણ પેટર્ન}: ડાયપોલ અક્ષને લંબરૂપે મહત્તમ વિકિરણ, અક્ષ સાથે શૂન્ય
  વિકિરણ
\end{itemize}

\end{solutionbox}
\begin{mnemonicbox}
``CORONA: કરંટ ઓસિલેટ્સ, રેડિએશન ઓકર્સ, નીયર-ફાર એરિયાઝ''

\end{mnemonicbox}
\subsection*{પ્રશ્ન 2(અ) [3
ગુણ]}\label{uxaaauxab0uxab6uxaa8-2uxa85-3-uxa97uxaa3}

\textbf{રેઝોનન્ટ અને નોન-રેઝોનન્ટ એન્ટેનામાં તફાવત કરો}

\begin{solutionbox}


{\def\LTcaptype{none} % do not increment counter
\vspace{-5pt}
\captionof{table}{રેઝોનન્ટ બનામ નોન-રેઝોનન્ટ અન્ટેના}
\vspace{-10pt}
\begin{longtable}[]{@{}lll@{}}
\toprule\noalign{}
લક્ષણ & રેઝોનન્ટ અન્ટેના & નોન-રેઝોનન્ટ અન્ટેના \\
\midrule\noalign{}
\endhead
\bottomrule\noalign{}
\endlastfoot
\textbf{લંબાઈ} & λ/2 નો પૂર્ણાંક ગુણાંક & તરંગલંબાઈ સાથે સંબંધિત નથી \\
\textbf{સ્થાયી તરંગો} & હાજર & હાજર નથી \\
\textbf{પ્રતિબાધા} & અવરોધક (વાસ્તવિક) & જટિલ (વાસ્તવિક + કાલ્પનિક) \\
\textbf{બેન્ડવિડ્થ} & સાંકડી & વિશાળ \\
\textbf{ઉદાહરણ} & અર્ધ-તરંગ ડાયપોલ & રોમ્બિક અન્ટેના \\
\end{longtable}
}

\end{solutionbox}
\begin{mnemonicbox}
``RESI: રેઝોનન્ટ એક્ઝિબિટ્સ સ્ટેન્ડિંગ-વેવ્સ ઇમ્પિડન્સ-રિયલ''

\end{mnemonicbox}
\subsection*{પ્રશ્ન 2(બ) [4
ગુણ]}\label{uxaaauxab0uxab6uxaa8-2uxaac-4-uxa97uxaa3}

\textbf{યાગી એન્ટેના સમજાવો અને તેની રેડિયેશન લાક્ષણિકતાઓની ચર્ચા કરો}

\begin{solutionbox}

\textbf{આકૃતિ: યાગી-ઉદા અન્ટેનાની સંરચના}

\begin{verbatim}
   Reflector    Driven    Directors
     (R)       Element      (D)
      |          (DE)     |  |  |
      |           |       |  |  |
 {-{-}{-}{-}{-}|{-}{-}{-}{-}{-}{-}{-}{-}{-}{-}{-}|{-}{-}{-}{-}{-}{-}{-}|{-}{-}|{-}{-}|{-}{-}{-}{-}{-}{-}{-} Direction of}
      |           |       |  |  |         Maximum Radiation
      |           |       |  |  |
      
   Longest       λ/2     Shortest
\end{verbatim}

\begin{itemize}
\tightlist
\item
  \textbf{સંરચના}: એક રિફ્લેક્ટર, એક ડ્રાઇવન એલિમેન્ટ અને અનેક ડાયરેક્ટર્સ ધરાવે છે
\item
  \textbf{દિશાત્મકતા}: ડાયરેક્ટર્સની દિશામાં ઉચ્ચ દિશાત્મકતા (8-12dB)
\item
  \textbf{ગેઇન}: વધુ ડાયરેક્ટર્સ સાથે ઉચ્ચ ગેઇન (15dB સુધી)
\item
  \textbf{બેન્ડવિડ્થ}: કેન્દ્ર આવૃત્તિનો 2-5\%
\item
  \textbf{એપ્લિકેશન્સ}: ટીવી રિસેપ્શન, પોઇન્ટ-ટુ-પોઇન્ટ કોમ્યુનિકેશન, એમેચ્યોર રેડિયો
\end{itemize}

\end{solutionbox}
\begin{mnemonicbox}
``DRAGONS: ડાયરેક્શનલ રિફ્લેક્ટર એન્ડ ગેઇન-ઇમ્પ્રુવિંગ ડાયરેક્ટર્સ
ઓફર નેરો સિગ્નલ્સ''

\end{mnemonicbox}
\subsection*{પ્રશ્ન 2(ક) [7
ગુણ]}\label{uxaaauxab0uxab6uxaa8-2uxa95-7-uxa97uxaa3}

\textbf{રેઝોનન્ટ વાયર એન્ટેનાની રેડિયેશન લાક્ષણિકતાઓનું વર્ણન કરો અને λ/2, 3λ/2 અને
5λ/2 એન્ટેનાનું કરંટ વિતરણ દોરો}

\begin{solutionbox}

\textbf{આકૃતિ: રેઝોનન્ટ વાયર એન્ટેનામાં કરંટ વિતરણ}

\begin{verbatim}
λ/2 Antenna:
     +{-{-}{-}{-}{-}{-}{-}{-}+}
     |        |
     v        v
 {-{-}{-}{-}+{-}{-}{-}{-}{-}{-}{-}{-}+{-}{-}{-}{-}}
     \^{        \^{}}
     |        |
     +{-{-}{-}{-}{-}{-}{-}{-}+}
     I\_max at center
     Zero at ends

3λ/2 Antenna:
     +{-{-}{-}+{-}{-}{-}+{-}{-}{-}+}
     |   |   |   |
     v   \^{   v   \^{}}
 {-{-}{-}{-}+{-}{-}{-}+{-}{-}{-}+{-}{-}{-}+{-}{-}{-}{-}}
     \^{   v   \^{}   v}
     |   |   |   |
     +{-{-}{-}+{-}{-}{-}+{-}{-}{-}+}
     3 current nodes
     
5λ/2 Antenna:
     +{-{-}{-}+{-}{-}{-}+{-}{-}{-}+{-}{-}{-}+{-}{-}{-}+}
     |   |   |   |   |   |
     v   \^{   v   \^{}   v   \^{}}
 {-{-}{-}{-}+{-}{-}{-}+{-}{-}{-}+{-}{-}{-}+{-}{-}{-}+{-}{-}{-}+{-}{-}{-}{-}}
     \^{   v   \^{}   v   \^{}   v}
     |   |   |   |   |   |
     +{-{-}{-}+{-}{-}{-}+{-}{-}{-}+{-}{-}{-}+{-}{-}{-}+}
     5 current nodes
\end{verbatim}

\begin{itemize}
\tightlist
\item
  \textbf{અર્ધ-તરંગ (λ/2)}: કેન્દ્રમાં પ્રવાહ મહત્તમ, છેડા પર શૂન્ય; વિકિરણ પેટર્ન
  આંકડા-આઠ આકારની હોય છે
\item
  \textbf{ત્રણ અર્ધ-તરંગ (3λ/2)}: ત્રણ પ્રવાહ મહત્તમ, λ/2 બિંદુઓ પર ફેઝ રિવર્સલ;
  વિકિરણ પેટર્નમાં અનેક લોબ્સ
\item
  \textbf{પાંચ અર્ધ-તરંગ (5λ/2)}: પાંચ પ્રવાહ મહત્તમ, વધુ જટિલ વિકિરણ પેટર્ન અનેક
  લોબ્સ સાથે
\item
  \textbf{સ્થાયી તરંગો}: બધા રેઝોનન્ટ અન્ટેનામાં સ્થાયી તરંગ પ્રવાહ વિતરણ જોવા મળે
  છે
\item
  \textbf{ફીડ પોઇન્ટ}: ઉત્તમ પ્રતિબાધા મેચિંગ માટે સામાન્ય રીતે પ્રવાહ મહત્તમ પર
  હોય છે
\end{itemize}

\end{solutionbox}
\begin{mnemonicbox}
``NODE: નંબર ઓફ ડિસ્ટ્રિબ્યુશન્સ ઇક્વલ્સ વેવલેન્થ-મલ્ટિપલ''

\end{mnemonicbox}
\subsection*{પ્રશ્ન 2(અ) અથવા [3
ગુણ]}\label{uxaaauxab0uxab6uxaa8-2uxa85-uxa85uxaa5uxab5-3-uxa97uxaa3}

\textbf{બ્રોડ સાઇડ અને એન્ડ ફાયર એરે એન્ટેનામાં તફાવત કરો}

\begin{solutionbox}


{\def\LTcaptype{none} % do not increment counter
\vspace{-5pt}
\captionof{table}{બ્રોડસાઇડ બનામ એન્ડ ફાયર એરે અન્ટેના}
\vspace{-10pt}
\begin{longtable}[]{@{}lll@{}}
\toprule\noalign{}
લક્ષણ & બ્રોડસાઇડ એરે & એન્ડ ફાયર એરે \\
\midrule\noalign{}
\endhead
\bottomrule\noalign{}
\endlastfoot
\textbf{મહત્તમ વિકિરણ} & એરે અક્ષને લંબરૂપે & એરે અક્ષની સાથે \\
\textbf{એલિમેન્ટ અંતર} & સામાન્ય રીતે λ/2 & સામાન્ય રીતે λ/4 થી λ/2 \\
\textbf{ફેઝ તફાવત} & 0^\circ (સમાન-ફેઝ) & 180^\circ (વિરુદ્ધ ફેઝ) \\
\textbf{દિશાત્મકતા} & ઉચ્ચ & ઉચ્ચ \\
\textbf{પેટર્ન} & દ્વિદિશાત્મક & એકદિશાત્મક \\
\end{longtable}
}

\end{solutionbox}
\begin{mnemonicbox}
``PEPS: પરપેન્ડિક્યુલર એલિમેન્ટ્સ પ્રોડ્યુસ સાઇડવેઝ રેડિએશન''

\end{mnemonicbox}
\subsection*{પ્રશ્ન 2(બ) અથવા [4
ગુણ]}\label{uxaaauxab0uxab6uxaa8-2uxaac-uxa85uxaa5uxab5-4-uxa97uxaa3}

\textbf{લુપ એન્ટેના સમજાવો અને તેની રેડીયેસન લાક્ષણિકતાઓની ચર્ચા કરો}

\begin{solutionbox}

\textbf{આકૃતિ: લુપ અન્ટેના}

\begin{verbatim}
    +{-{-}{-}{-}{-}+}
    |     |
    |     |
+{-{-}{-}+     +{-}{-}{-}+}
|             |
+{-{-}{-}{-}{-}{-}+{-}{-}{-}{-}{-}{-}+}
       |
     Feed
     Point
\end{verbatim}

\begin{itemize}
\tightlist
\item
  \textbf{સંરચના}: એક તરંગલંબાઈ અથવા ઓછી પરિધિવાળા બંધ-લૂપ વાહક
\item
  \textbf{પ્રકારો}: નાની લૂપ્સ (પરિધિ \textless{} λ/10) અને મોટી લૂપ્સ (પરિધિ
  \approx λ)
\item
  \textbf{ધ્રુવીકરણ}: ઇલેક્ટ્રિક ફીલ્ડ લૂપના સમતલમાં ધ્રુવીકૃત
\item
  \textbf{વિકિરણ પેટર્ન}: નાની લૂપ્સ માટે આંકડા-આઠ પેટર્ન, મોટી લૂપ્સ માટે વધુ
  દિશાત્મક
\item
  \textbf{એપ્લિકેશન્સ}: દિશા શોધ, AM રિસેપ્શન, RFID ટૅગ્સ
\item
  \textbf{પ્રતિબાધા}: નાની લૂપ્સ માટે ઉચ્ચ પ્રતિબાધા, મોટી લૂપ્સ માટે રેઝોનન્ટ
\end{itemize}

\end{solutionbox}
\begin{mnemonicbox}
``SPIRAL: સ્મોલ પેટર્ન્સ ઇન રિસીવિંગ એન્ડ લોકેટિંગ સિગ્નલ્સ''

\end{mnemonicbox}
\subsection*{પ્રશ્ન 2(ક) અથવા [7
ગુણ]}\label{uxaaauxab0uxab6uxaa8-2uxa95-uxa85uxaa5uxab5-7-uxa97uxaa3}

\textbf{નોન રેઝોનન્ટ વાયર એન્ટેનાની રેડિયેશન લાક્ષણિકતાઓનું વર્ણન કરો અને λ/2, 3λ/2
અને 5λ/2 એન્ટેનાની રેડિયેશન પેટર્ન દોરો}

\begin{solutionbox}

\textbf{આકૃતિ: વાયર અન્ટેનાની વિકિરણ પેટર્ન}

\begin{verbatim}
λ/2 Antenna Pattern:

    \^{}
    |     .{-.}
    |    /   {}
    |   |     |
{-{-}{-}{-}+{-}{-}{-}+{-}{-}{-}{-}{-}+{-}{-}{-}{-}{-}}
    |   |     |
    |    {   /}
    |     {{-}}
    v
    
3λ/2 Antenna Pattern:

    \^{}
    |    .{-. .{-}.}
    |   /   X   {}
    |  |  / {    |}
{-{-}{-}{-}+{-}{-}+{-}+{-}{-}{-}+{-}+{-}{-}{-}{-}}
    |  |  { /    |}
    |   {   X   /}
    |    {{-} {-}}
    v
    
5λ/2 Antenna Pattern:

    \^{}
    |  .{-. .{-}. .{-}.}
    | /   X   X   {}
    ||  / { /     |}
{-{-}{-}{-}++{-}+{-}{-}{-}+{-}{-}{-}+{-}+{-}{-}}
    ||  { /  /    |}
    | {   X   X   /}
    |  {{-} {-} {-}}
    v
\end{verbatim}

\begin{itemize}
\tightlist
\item
  \textbf{નોન-રેઝોનન્ટ ગુણધર્મો}: સ્થાયી તરંગોને બદલે પ્રવાસી તરંગો
\item
  \textbf{λ/2 અન્ટેના}: સરળ દ્વિદિશાત્મક પેટર્ન, વાયરને લંબરૂપે મહત્તમ વિકિરણ
\item
  \textbf{3λ/2 અન્ટેના}: અનેક લોબ્સ, સાઇડ લોબ્સ સાથે વધુ જટિલ પેટર્ન
\item
  \textbf{5λ/2 અન્ટેના}: અનેક મુખ્ય અને સાઇડ લોબ્સ સાથે વધુ જટિલ પેટર્ન
\item
  \textbf{ફીડ પોઇન્ટ પ્રતિબાધા}: નોન-રેઝોનન્ટ, સામાન્ય રીતે પ્રતિબાધા મેચિંગની
  જરૂર પડે છે
\item
  \textbf{બેન્ડવિડ્થ}: રેઝોનન્ટ અન્ટેના કરતાં વધારે
\end{itemize}

\end{solutionbox}
\begin{mnemonicbox}
``TWIST: ટ્રાવેલિંગ વેવ્સ ઇન્ક્રીઝ સાઇડ-લોબ ટ્રાન્સમિશન''

\end{mnemonicbox}
\subsection*{પ્રશ્ન 3(અ) [3
ગુણ]}\label{uxaaauxab0uxab6uxaa8-3uxa85-3-uxa97uxaa3}

\textbf{માઇક્રો સ્ટ્રીપ (પેચ) એન્ટેના પર ટૂંકી નોંધ લખો}

\begin{solutionbox}

\textbf{આકૃતિ: માઇક્રોસ્ટ્રિપ પેચ અન્ટેનાની સંરચના}

\begin{verbatim}
   +{-{-}{-}{-}{-}{-}{-}+}
   |       |
   | Patch |
   |       |
   +{-{-}{-}{-}{-}{-}{-}+}
   | Substrate
   |
   +{-{-}{-}{-}{-}{-}{-}{-}{-}{-}{-}{-}+}
   |Ground Plane|
   +{-{-}{-}{-}{-}{-}{-}{-}{-}{-}{-}{-}+}
\end{verbatim}

\begin{itemize}
\tightlist
\item
  \textbf{સંરચના}: ડાઇઇલેક્ટ્રિક સબસ્ટ્રેટ પર ધાતુનો પેચ અને નીચે ગ્રાઉન્ડ પ્લેન
\item
  \textbf{કદ}: સામાન્ય રીતે અર્ધ-તરંગલંબાઈનું કદ
\item
  \textbf{પ્રોફાઇલ}: નીચી-પ્રોફાઇલ, હલકા વજન, સરળતાથી બનાવી શકાય
\item
  \textbf{વિકિરણ}: પેચના કિનારાઓથી વિકિરણ, ઓમ્નિદિશાત્મક અથવા દિશાત્મક પેટર્ન
\item
  \textbf{એપ્લિકેશન્સ}: મોબાઇલ ઉપકરણો, ઉપગ્રહો, GPS રિસીવર્સ
\end{itemize}

\end{solutionbox}
\begin{mnemonicbox}
``PSALM: પેચ સબસ્ટ્રેટ અબવ લેયર ઓફ મેટલ''

\end{mnemonicbox}
\subsection*{પ્રશ્ન 3(બ) [4
ગુણ]}\label{uxaaauxab0uxab6uxaa8-3uxaac-4-uxa97uxaa3}

\textbf{હેલિકલ એન્ટેના સમજાવો અને તેની રેડિયેશન લાક્ષણિકતાઓની ચર્ચા કરો}

\begin{solutionbox}

\textbf{આકૃતિ: હેલિકલ એન્ટેના}

\begin{verbatim}
      \^{}
      |
    +{-{-}{-}+}
   /     {}
  +       +
 /|       |{}
+ |       | +
| |       | |  {-{-}}
+ |       | +
 {|       |/}
  +       +
   {     /}
    +{-{-}{-}+}
    
  Ground Plane
\end{verbatim}

\begin{itemize}
\tightlist
\item
  \textbf{સંરચના}: ગ્રાઉન્ડ પ્લેન ઉપર હેલિક્સ આકારમાં વેલાયેલા વાહક તાર
\item
  \textbf{મોડ્સ}: એક્સિયલ મોડ (એન્ડ-ફાયર) અને નોર્મલ મોડ (બ્રોડસાઇડ)
\item
  \textbf{એક્સિયલ મોડ}: જ્યારે પરિધિ \approx λ હોય, ત્યારે હેલિક્સ અક્ષ સાથે વિકિરણ
\item
  \textbf{નોર્મલ મોડ}: જ્યારે પરિધિ \textless\textless{} λ હોય, ત્યારે અક્ષને
  લંબરૂપે વિકિરણ
\item
  \textbf{ધ્રુવીકરણ}: એક્સિયલ મોડમાં વર્તુળાકાર ધ્રુવીકરણ
\item
  \textbf{એપ્લિકેશન્સ}: ઉપગ્રહ સંચાર, અવકાશ ટેલિમેટ્રી, રેડિયો ખગોળશાસ્ત્ર
\end{itemize}

\end{solutionbox}
\begin{mnemonicbox}
``MOCHA: મોડ ઓફ સર્ક્યુલર હેલિક્સ એન્ટેનાઝ''

\end{mnemonicbox}
\subsection*{પ્રશ્ન 3(ક) [7
ગુણ]}\label{uxaaauxab0uxab6uxaa8-3uxa95-7-uxa97uxaa3}

\textbf{હોર્ન એન્ટેના સમજાવો અને તેની રેડિયેશન લાક્ષણિકતાઓની ચર્ચા કરો}

\begin{solutionbox}

\textbf{આકૃતિ: હોર્ન એન્ટેનાના પ્રકારો}

\begin{verbatim}
Pyramidal Horn:
    +{-{-}{-}{-}{-}{-}{-}{-}+}
    |        |
    |        |
+{-{-}{-}+        +{-}{-}{-}+}
|                |
+{-+{-}{-}{-}{-}{-}{-}{-}{-}{-}{-}{-}{-}+{-}+}
  |            |
  +{-{-}{-}{-}{-}{-}{-}{-}{-}{-}{-}{-}+}
  
Sectoral Horn:
    +{-{-}{-}{-}{-}{-}{-}{-}+}
    |        |
    |        |
+{-{-}{-}+        +{-}{-}{-}+}
|                |
+{-{-}{-}{-}{-}{-}{-}{-}{-}{-}{-}{-}{-}{-}{-}{-}+}

Conical Horn:
      +{-{-}{-}{-}+}
     /      {}
    /        {}
   /          {}
  +            +
  |            |
  +{-{-}{-}{-}{-}{-}{-}{-}{-}{-}{-}{-}+}
\end{verbatim}

\begin{itemize}
\tightlist
\item
  \textbf{સંરચના}: વેવગાઇડ સાથે ફ્લેર્ડ એન્ડ જે મુક્ત અવકાશ સાથે પ્રતિબાધા મેળ કરે છે
\item
  \textbf{પ્રકારો}: પિરામિડલ (લંબચોરસ), સેક્ટોરલ (E-પ્લેન અથવા H-પ્લેન), અને
  કોનિકલ (વર્તુળાકાર)
\item
  \textbf{દિશાત્મકતા}: 10-20 dB, માત્ર વેવગાઇડ કરતાં વધારે
\item
  \textbf{બેન્ડવિડ્થ}: ખૂબ પહોળી બેન્ડવિડ્થ
\item
  \textbf{વિકિરણ પેટર્ન}: નાના સાઇડ લોબ્સ સાથે મુખ્ય લોબ
\item
  \textbf{એપ્લિકેશન્સ}: માઇક્રોવેવ સંચાર, રડાર, ઉપગ્રહ ટ્રેકિંગ, EMC પરીક્ષણ
\item
  \textbf{ફાયદાઓ}: ઉચ્ચ ગેઇન, સરળ નિર્માણ, નીચો VSWR
\end{itemize}

\end{solutionbox}
\begin{mnemonicbox}
``POWERS: પિરામિડલ ઓર વાઇડનિંગ એન્ડ રેડિએટ્સ સ્ટ્રોંગલી''

\end{mnemonicbox}
\subsection*{પ્રશ્ન 3(અ) અથવા [3
ગુણ]}\label{uxaaauxab0uxab6uxaa8-3uxa85-uxa85uxaa5uxab5-3-uxa97uxaa3}

\textbf{સ્લોટ એન્ટેના પર ટૂંકી નોંધ લખો}

\begin{solutionbox}

\textbf{આકૃતિ: સ્લોટ એન્ટેના}

\begin{verbatim}
+{-{-}{-}{-}{-}{-}{-}{-}{-}{-}{-}{-}{-}{-}{-}{-}{-}{-}{-}+}
|                   |
|    +{-{-}{-}{-}{-}{-}{-}+      |}
|    |       |      |
|    |  Slot |      |
|    |       |      |
|    +{-{-}{-}{-}{-}{-}{-}+      |}
|                   |
+{-{-}{-}{-}{-}{-}{-}{-}{-}{-}{-}{-}{-}{-}{-}{-}{-}{-}{-}+}
 Conducting Surface
\end{verbatim}

\begin{itemize}
\tightlist
\item
  \textbf{સંરચના}: વાહક સપાટી પર કાપેલો લંબચોરસ/વર્તુળાકાર સ્લોટ
\item
  \textbf{બાબિનેટનો સિદ્ધાંત}: ડાયપોલ એન્ટેનાનો પૂરક
\item
  \textbf{વિકિરણ પેટર્ન}: ડાયપોલ જેવું પરંતુ E અને H ક્ષેત્રો આંતરિત થયેલા
\item
  \textbf{ધ્રુવીકરણ}: ઇલેક્ટ્રિક ફીલ્ડ સ્લોટની લંબાઈને લંબરૂપ
\item
  \textbf{પ્રતિબાધા}: ડાયપોલની તુલનામાં ઉચ્ચ પ્રતિબાધા
\item
  \textbf{એપ્લિકેશન્સ}: વિમાન, અવકાશયાન, બેઝ સ્ટેશન, ફ્લશ માઉન્ટિંગ
\end{itemize}

\end{solutionbox}
\begin{mnemonicbox}
``CROPS: કોમ્પ્લિમેન્ટરી રેડિએશન ઓપનિંગ પર્પેન્ડિક્યુલર ટુ
સર્ફેસ''

\end{mnemonicbox}
\subsection*{પ્રશ્ન 3(બ) અથવા [4
ગુણ]}\label{uxaaauxab0uxab6uxaa8-3uxaac-uxa85uxaa5uxab5-4-uxa97uxaa3}

\textbf{પેરાબોલિક રિફ્લેક્ટર એન્ટેના સમજાવો અને તેની રેડિયેશન લાક્ષણિકતાઓની ચર્ચા
કરો}

\begin{solutionbox}

\textbf{આકૃતિ: પેરાબોલિક રિફ્લેક્ટર એન્ટેના}

\begin{verbatim}
            \^{}
           / {}
          /   {}
         /     {}
        /       {}
       /         {}
      /           {}
     /             {}
    +{-{-}{-}{-}{-}{-}{-}{-}{-}{-}{-}{-}{-}{-}+{-}}
          |  |
          |  |
          +{-{-}+}
          Feed
\end{verbatim}

\begin{itemize}
\tightlist
\item
  \textbf{સંરચના}: ફોકલ પોઇન્ટ પર ફીડ સાથે પેરાબોલિક રિફ્લેક્ટર
\item
  \textbf{કાર્ય સિદ્ધાંત}: રિફ્લેક્ટરથી સમાંતર કિરણો ફોકલ પોઇન્ટ પર એકત્રિત થાય છે
\item
  \textbf{દિશાત્મકતા}: ખૂબ જ ઉચ્ચ (30-40 dB)
\item
  \textbf{બીમવિડ્થ}: ખૂબ જ સાંકડી, વ્યાસના વ્યસ્ત પ્રમાણમાં
\item
  \textbf{કાર્યક્ષમતા}: ફીડ ડિઝાઇન પર આધારિત 50-70\%
\item
  \textbf{એપ્લિકેશન્સ}: ઉપગ્રહ સંચાર, રેડિયો ખગોળશાસ્ત્ર, રડાર સિસ્ટમ્સ
\item
  \textbf{પ્રકારો}: પ્રાઇમ ફોકસ, કેસેગ્રેન, ઓફસેટ ફીડ
\end{itemize}

\end{solutionbox}
\begin{mnemonicbox}
``DISH: ડાયરેક્ટિંગ ઇનકમિંગ સિગ્નલ્સ ટુ હબ''

\end{mnemonicbox}
\subsection*{પ્રશ્ન 3(ક) અથવા [7
ગુણ]}\label{uxaaauxab0uxab6uxaa8-3uxa95-uxa85uxaa5uxab5-7-uxa97uxaa3}

\textbf{V અને ઊંધી V એન્ટેનાનું વર્ણન કરો}

\begin{solutionbox}

\textbf{આકૃતિ: V અને ઊંધી V એન્ટેના}

\begin{verbatim}
V Antenna:
       /{}
      /  {}
     /    {}
    /      {}
   /        {}
  +          +
  |          |
  +{-{-}{-}{-}{-}{-}{-}{-}{-}{-}+}
    Feed Point

Inverted V Antenna:
  +{-{-}{-}{-}{-}{-}{-}{-}{-}{-}+}
  |          |
  +          +
   {        /}
    {      /}
     {    /}
      {  /}
       {/}
    Feed Point
\end{verbatim}


{\def\LTcaptype{none} % do not increment counter
\vspace{-5pt}
\captionof{table}{V અને ઊંધી V એન્ટેનાની તુલના}
\vspace{-10pt}
\begin{longtable}[]{@{}lll@{}}
\toprule\noalign{}
લક્ષણ & V એન્ટેના & ઊંધી V એન્ટેના \\
\midrule\noalign{}
\endhead
\bottomrule\noalign{}
\endlastfoot
\textbf{આકાર} & ભુજાઓ ફીડથી ઉપર તરફ વિસ્તરે છે & ભુજાઓ શિખરથી નીચે તરફ વિસ્તરે
છે \\
\textbf{ખૂણો} & ભુજાઓ વચ્ચે સામાન્ય રીતે 90^\circ & ભુજાઓ વચ્ચે સામાન્ય રીતે 90-120^\circ \\
\textbf{ઊંચાઈ} & બે ઊંચા સપોર્ટની જરૂર & એક ઊંચા સપોર્ટની જરૂર \\
\textbf{પ્રતિબાધા} & 40-50 ઓહ્મ & 20-30 ઓહ્મ \\
\textbf{વિકિરણ પેટર્ન} & દ્વિદિશાત્મક & વધુ સર્વદિશાત્મક \\
\textbf{એપ્લિકેશન્સ} & દિશાત્મક HF સંચાર & HF એમેચર રેડિયો, મર્યાદિત જગ્યા \\
\end{longtable}
}

\end{solutionbox}
\begin{mnemonicbox}
``VIVA: V ઇઝ વર્ટિકલ અરેન્જમેન્ટ, ઇન્વર્ટેડ V એઇમ્સ ડાઉનવર્ડ''

\end{mnemonicbox}
\subsection*{પ્રશ્ન 4(અ) [3
ગુણ]}\label{uxaaauxab0uxab6uxaa8-4uxa85-3-uxa97uxaa3}

\textbf{વ્યાખ્યાયિત કરો: (1) રીફ્લેક્શન, (2) રીફ્રેક્શન અને (3) ડીફ્રેક્શન}

\begin{solutionbox}


{\def\LTcaptype{none} % do not increment counter
\vspace{-5pt}
\captionof{table}{તરંગ ઘટનાની વ્યાખ્યાઓ}
\vspace{-10pt}
\begin{longtable}[]{@{}
  >{\raggedright\arraybackslash}p{(\linewidth - 2\tabcolsep) * \real{0.5000}}
  >{\raggedright\arraybackslash}p{(\linewidth - 2\tabcolsep) * \real{0.5000}}@{}}
\toprule\noalign{}
\begin{minipage}[b]{\linewidth}\raggedright
ઘટના
\end{minipage} & \begin{minipage}[b]{\linewidth}\raggedright
વ્યાખ્યા
\end{minipage} \\
\midrule\noalign{}
\endhead
\bottomrule\noalign{}
\endlastfoot
\textbf{રીફ્લેક્શન (પરાવર્તન)} & જ્યારે તરંગો બે માધ્યમની સરહદ પર અથડાય ત્યારે તેનું
પાછું વળવું \\
\textbf{રીફ્રેક્શન (વક્રીભવન)} & જ્યારે તરંગો એક માધ્યમથી બીજા માધ્યમમાં જાય ત્યારે
તેમની પ્રસરણ ગતિમાં ફેરફારને કારણે તેમનું વાંકા વળવું \\
\textbf{ડીફ્રેક્શન (વિવર્તન)} & અવરોધો આસપાસ અથવા ઓપનિંગ્સમાંથી તરંગોનું વળવું \\
\end{longtable}
}

\end{solutionbox}
\begin{mnemonicbox}
``RRD: રિબાઉન્ડિંગ, રિડાયરેક્ટિંગ, ડિટૂર''

\end{mnemonicbox}
\subsection*{પ્રશ્ન 4(બ) [4
ગુણ]}\label{uxaaauxab0uxab6uxaa8-4uxaac-4-uxa97uxaa3}

\textbf{સંચાર માટે HAM રેડિયો એપ્લિકેશનની સૂચિ બનાવો}

\begin{solutionbox}


{\def\LTcaptype{none} % do not increment counter
\vspace{-5pt}
\captionof{table}{HAM રેડિયો એપ્લિકેશન્સ}
\vspace{-10pt}
\begin{longtable}[]{@{}ll@{}}
\toprule\noalign{}
એપ્લિકેશન & વર્ણન \\
\midrule\noalign{}
\endhead
\bottomrule\noalign{}
\endlastfoot
\textbf{આપાતકાલીન સંચાર} & સામાન્ય માળખું નિષ્ફળ જાય ત્યારે આપત્તિ રાહત \\
\textbf{DX સંચાર} & લાંબા અંતરનો આંતરરાષ્ટ્રીય સંચાર \\
\textbf{ઉપગ્રહ સંચાર} & વિસ્તારિત રેન્જ માટે એમેચર રેડિયો ઉપગ્રહોનો ઉપયોગ \\
\textbf{ડિજિટલ મોડ્સ} & ટેક્સ્ટ/ડેટા ટ્રાન્સમિશન (RTTY, PSK31, FT8) \\
\textbf{મોર્સ કોડ} & પરંપરાગત CW સંચાર \\
\textbf{વોઇસ કોમ્યુનિકેશન} & SSB, FM, AM મોડ્યુલેશનનો ઉપયોગ \\
\textbf{જાહેર સેવા} & મેરેથોન, પરેડ જેવા કાર્યક્રમોને સમર્થન \\
\end{longtable}
}

\end{solutionbox}
\begin{mnemonicbox}
``EDSDMVP: ઇમરજન્સી DX સેટેલાઇટ ડિજિટલ મોર્સ વોઇસ
પબ્લિક-સર્વિસ''

\end{mnemonicbox}
\subsection*{પ્રશ્ન 4(ક) [7
ગુણ]}\label{uxaaauxab0uxab6uxaa8-4uxa95-7-uxa97uxaa3}

\textbf{આયનોસ્ફિયરના સ્તરો અને આકાશી તરંગોના પ્રસારને સમજાવો}

\begin{solutionbox}

\textbf{આકૃતિ: આયનોસ્ફેરિક સ્તરો અને સ્કાય વેવ પ્રોપેગેશન}

\begin{center}
\textbf{Mermaid Diagram (Code)}
\begin{verbatim}
{Shaded}
{Highlighting}[]
graph TD
    A[ટ્રાન્સમીટર] {-{-}{}|સ્કાય વેવ| B[F2 સ્તર: 250{-}400 km]}
    A {-{-}{}|સ્કાય વેવ| C[F1 સ્તર: 150{-}250 km]}
    A {-{-}{}|સ્કાય વેવ| D[E સ્તર: 90{-}150 km]}
    A {-{-}{}|સ્કાય વેવ| E[D સ્તર: 60{-}90 km]}
    B {-{-}{}|પરાવર્તન| F[લાંબા અંતરે રિસીવર]}
    C {-{-}{}|પરાવર્તન| F}
    D {-{-}{}|પરાવર્તન/અવશોષણ| F}
    E {-{-}{}|અવશોષણ| G[સિગ્નલ નુકસાન]}
{Highlighting}
{Shaded}
\end{verbatim}
\end{center}

\begin{itemize}
\tightlist
\item
  \textbf{D સ્તર (60-90 km)}: દિવસના પ્રકાશમાં અસ્તિત્વમાં રહે છે, 10 MHz નીચેના
  HF સિગ્નલોને શોષે છે
\item
  \textbf{E સ્તર (90-150 km)}: 3-5 MHz સિગ્નલોને પરાવર્તિત કરે છે, દિવસ
  દરમિયાન વધુ મજબૂત, ઉનાળામાં સ્પોરાડિક-E
\item
  \textbf{F1 સ્તર (150-250 km)}: માત્ર દિવસ દરમિયાન, રાત્રે F2 સાથે ભળી જાય છે
\item
  \textbf{F2 સ્તર (250-400 km)}: મુખ્ય પરાવર્તક સ્તર, લાંબા અંતરના HF સંચારને
  સક્ષમ બનાવે છે
\item
  \textbf{પ્રસરણ પરિબળો}:

  \begin{itemize}
  \tightlist
  \item
    \textbf{વર્ચ્યુઅલ હાઇટ}: પરાવર્તનની દેખીતી ઊંચાઈ
  \item
    \textbf{ક્રિટિકલ ફ્રિકવન્સી}: ઊંચી તરફ પરાવર્તિત મહત્તમ આવૃત્તિ
  \item
    \textbf{MUF}: આપેલા અંતર માટે મહત્તમ ઉપયોગી આવૃત્તિ
  \item
    \textbf{સ્કિપ ડિસ્ટન્સ}: સ્કાય વેવ રિસેપ્શન માટે ન્યૂનતમ અંતર
  \end{itemize}
\end{itemize}

\end{solutionbox}
\begin{mnemonicbox}
``DEFV: D-એબ્ઝોર્બ્સ, E-રિફ્લેક્ટ્સ, F-પ્રોવાઇડ્સ
વેરી-લોંગ-ડિસ્ટન્સ''

\end{mnemonicbox}
\subsection*{પ્રશ્ન 4(અ) અથવા [3
ગુણ]}\label{uxaaauxab0uxab6uxaa8-4uxa85-uxa85uxaa5uxab5-3-uxa97uxaa3}

\textbf{વ્યાખ્યાયિત કરો: (1) MUF, (2) LUF અને (3) સ્કીપ અંતર}

\begin{solutionbox}


{\def\LTcaptype{none} % do not increment counter
\vspace{-5pt}
\captionof{table}{આયનોસ્ફેરિક પ્રોપેગેશન શબ્દો}
\vspace{-10pt}
\begin{longtable}[]{@{}
  >{\raggedright\arraybackslash}p{(\linewidth - 2\tabcolsep) * \real{0.3333}}
  >{\raggedright\arraybackslash}p{(\linewidth - 2\tabcolsep) * \real{0.6667}}@{}}
\toprule\noalign{}
\begin{minipage}[b]{\linewidth}\raggedright
શબ્દ
\end{minipage} & \begin{minipage}[b]{\linewidth}\raggedright
વ્યાખ્યા
\end{minipage} \\
\midrule\noalign{}
\endhead
\bottomrule\noalign{}
\endlastfoot
\textbf{MUF (મહત્તમ ઉપયોગી આવૃત્તિ)} & આપેલા અંતર અને સમય માટે આયનોસ્ફિયર દ્વારા
પરાવર્તિત થઈ શકે તેવી ઉચ્ચતમ આવૃત્તિ \\
\textbf{LUF (ન્યૂનતમ ઉપયોગી આવૃત્તિ)} & સંચાર માટે પર્યાપ્ત સિગ્નલ શક્તિ પ્રદાન
કરતી ન્યૂનતમ આવૃત્તિ \\
\textbf{સ્કિપ અંતર} & ટ્રાન્સમીટરથી ન્યૂનતમ અંતર જ્યાં સ્કાય વેવ પૃથ્વી પર પાછો આવે
છે \\
\end{longtable}
}

\end{solutionbox}
\begin{mnemonicbox}
``MLS: મેક્સિમમ-હાયેસ્ટ, લોવેસ્ટ-મિનિમમ, સ્કિપ-નિયરેસ્ટ''

\end{mnemonicbox}
\subsection*{પ્રશ્ન 4(બ) અથવા [4
ગુણ]}\label{uxaaauxab0uxab6uxaa8-4uxaac-uxa85uxaa5uxab5-4-uxa97uxaa3}

\textbf{સંચારના HAM રેડિયો ડિજિટલ મોડ્સની સૂચિ બનાવો}

\begin{solutionbox}


{\def\LTcaptype{none} % do not increment counter
\vspace{-5pt}
\captionof{table}{HAM રેડિયો ડિજિટલ મોડ્સ}
\vspace{-10pt}
\begin{longtable}[]{@{}ll@{}}
\toprule\noalign{}
ડિજિટલ મોડ & લાક્ષણિકતાઓ \\
\midrule\noalign{}
\endhead
\bottomrule\noalign{}
\endlastfoot
\textbf{FT8} & નબળા સિગ્નલ, સાંકડી બેન્ડવિડ્થ, ઓટોમેટેડ એક્સચેન્જ \\
\textbf{PSK31} & કીબોર્ડ-ટુ-કીબોર્ડ ટેક્સ્ટ કોમ્યુનિકેશન, સાંકડી બેન્ડવિડ્થ \\
\textbf{RTTY} & રેડિયો ટેલિટાઇપ, મજબૂત જૂનો ડિજિટલ મોડ \\
\textbf{SSTV} & સ્લો સ્કેન ટેલિવિઝન ઇમેજ ટ્રાન્સમિશન માટે \\
\textbf{JT65/JT9} & અત્યંત નબળા સિગ્નલ મોડ્સ અત્યંત અંતર માટે \\
\textbf{પેકેટ રેડિયો} & ભૂલ સુધારણા સાથે કમ્પ્યુટર-આધારિત ડેટા ટ્રાન્સમિશન \\
\textbf{APRS} & GPS સાથે ઓટોમેટિક પોઝિશન રિપોર્ટિંગ સિસ્ટમ \\
\textbf{ડિજિટલ વોઇસ} & DMR, D-STAR, ફ્યુઝન, P25 ડિજિટલ વોઇસ પ્રોટોકોલ \\
\end{longtable}
}

\end{solutionbox}
\begin{mnemonicbox}
``FIRST PAD: FT8 ઇઝ RTTY SSTV ધેન પેકેટ APRS
ડિજિટલ-વોઇસ''

\end{mnemonicbox}
\subsection*{પ્રશ્ન 4(ક) અથવા [7
ગુણ]}\label{uxaaauxab0uxab6uxaa8-4uxa95-uxa85uxaa5uxab5-7-uxa97uxaa3}

\textbf{અવકાશ તરંગોના પ્રસારને સમજાવો}

\begin{solutionbox}

\textbf{આકૃતિ: સ્પેસ વેવ પ્રોપેગેશન}

\begin{center}
\textbf{Mermaid Diagram (Code)}
\begin{verbatim}
{Shaded}
{Highlighting}[]
graph LR
    A[ટ્રાન્સમીટર] {-{-}{}|ડાયરેક્ટ વેવ| B[રિસીવર]}
    A {-{-}{}|ગ્રાઉન્ડ રિફ્લેક્ટેડ વેવ| B}
    A {-{-}{}|ટ્રોપોસ્ફેરિક સ્કેટર| C[એક્સ્ટેન્ડેડ રેન્જ રિસીવર]}
    A {-{-}{}|ડક્ટિંગ| D[વધુ એક્સ્ટેન્ડેડ રેન્જ]}

    subgraph "ટ્રોપોસ્ફિયર"
    A
    B
    C
    D
    E[તાપમાન ઇન્વર્ઝન સ્તર]
    end
    
    A {-{-}{}|ફોલોઝ| E {-}{-}{}|વેવગાઇડ ઇફેક્ટ| D}
{Highlighting}
{Shaded}
\end{verbatim}
\end{center}

\begin{itemize}
\tightlist
\item
  \textbf{ઘટકો}: ડાયરેક્ટ વેવ, ગ્રાઉન્ડ-રિફ્લેક્ટેડ વેવ, ટ્રોપોસ્ફેરિક વેવ્સ
\item
  \textbf{સીધી દૃષ્ટિ}: પૃથ્વીની વક્રતાથી મર્યાદિત પ્રાથમિક પદ્ધતિ
\item
  \textbf{આવૃત્તિ રેન્જ}: VHF, UHF, અને માઇક્રોવેવ આવૃત્તિઓ
\item
  \textbf{ટ્રોપોસ્ફેરિક સ્કેટરિંગ}: ફોરવર્ડ સ્કેટરિંગ ક્ષિતિજથી આગળની રેન્જ વિસ્તારે છે
\item
  \textbf{ડક્ટ પ્રોપેગેશન}:

  \begin{itemize}
  \tightlist
  \item
    તાપમાન ઇન્વર્ઝન સ્તરોમાં થાય છે
  \item
    સિગ્નલોને ટ્રેપ કરતી વેવગાઇડ અસર બનાવે છે
  \item
    ખૂબ લાંબા અંતર VHF/UHF પ્રસારણને શક્ય બનાવે છે
  \end{itemize}
\item
  \textbf{અસર કરતા પરિબળો}: એન્ટેનાની ઊંચાઈ, ભૂમિ, વાતાવરણીય પરિસ્થિતિઓ
\item
  \textbf{એપ્લિકેશન્સ}: ટીવી પ્રસારણ, માઇક્રોવેવ લિંક્સ, મોબાઇલ સંચાર
\end{itemize}

\end{solutionbox}
\begin{mnemonicbox}
``DRIFT: ડાયરેક્ટ રિફ્લેક્શન ઇન્વર્ઝન ફોરવર્ડ ટ્રોપોસ્ફેરિક''

\end{mnemonicbox}
\subsection*{પ્રશ્ન 5(અ) [3
ગુણ]}\label{uxaaauxab0uxab6uxaa8-5uxa85-3-uxa97uxaa3}

\textbf{વ્યાખ્યા કરો: (1) બીમ એરિયા (2) બીમ કાર્યક્ષમતા, અને (3) અસરકારક
અપર્ચર}

\begin{solutionbox}


{\def\LTcaptype{none} % do not increment counter
\vspace{-5pt}
\captionof{table}{અન્ટેના બીમ પરિમાણો}
\vspace{-10pt}
\begin{longtable}[]{@{}
  >{\raggedright\arraybackslash}p{(\linewidth - 2\tabcolsep) * \real{0.4783}}
  >{\raggedright\arraybackslash}p{(\linewidth - 2\tabcolsep) * \real{0.5217}}@{}}
\toprule\noalign{}
\begin{minipage}[b]{\linewidth}\raggedright
પરિમાણ
\end{minipage} & \begin{minipage}[b]{\linewidth}\raggedright
વ્યાખ્યા
\end{minipage} \\
\midrule\noalign{}
\endhead
\bottomrule\noalign{}
\endlastfoot
\textbf{બીમ એરિયા} & જો રેડિએશન ઇન્ટેન્સિટી સ્થિર હોય તો અન્ટેના દ્વારા વિકિરિત
બધી શક્તિ જે ઘન ખૂણામાંથી પસાર થશે તે \\
\textbf{બીમ કાર્યક્ષમતા} & મુખ્ય બીમમાં શક્તિનો કુલ વિકિરિત શક્તિ સાથેનો
ગુણોત્તર \\
\textbf{અસરકારક અપર્ચર} & અન્ટેના જેના પર RF ઊર્જા કેપ્ચર કરે છે તે ક્ષેત્ર, ગેઇન સાથે
સંબંધિત \\
\end{longtable}
}

\end{solutionbox}
\begin{mnemonicbox}
``BEA: બીમ એફિશિયન્સી એપર્ચર''

\end{mnemonicbox}
\subsection*{પ્રશ્ન 5(બ) [4
ગુણ]}\label{uxaaauxab0uxab6uxaa8-5uxaac-4-uxa97uxaa3}

\textbf{સ્માર્ટ એન્ટેનાની જરૂરિયાતનું વર્ણન કરો}

\begin{solutionbox}

\textbf{આકૃતિ: સ્માર્ટ એન્ટેનાના ફાયદા}

\begin{center}
\textbf{Mermaid Diagram (Code)}
\begin{verbatim}
{Shaded}
{Highlighting}[]
graph TD
    A[સ્માર્ટ એન્ટેના] {-{-}{}|પ્રદાન કરે છે| B[વધારેલી ક્ષમતા]}
    A {-{-}{}|પ્રદાન કરે છે| C[વધારેલો કવરેજ]}
    A {-{-}{}|ઘટાડે છે| D[હસ્તક્ષેપ]}
    A {-{-}{}|સુધારે છે| E[સિગ્નલની ગુણવત્તા]}
    A {-{-}{}|બચાવે છે| F[બેટરી પાવર]}
    A {-{-}{}|સક્ષમ કરે છે| G[સ્પેશિયલ મલ્ટિપ્લેક્સિંગ]}
{Highlighting}
{Shaded}
\end{verbatim}
\end{center}

\begin{itemize}
\tightlist
\item
  \textbf{ક્ષમતા સુધારણા}: એક જ બેન્ડવિડ્થમાં વધુ વપરાશકર્તાઓને સેવા આપે છે
\item
  \textbf{કવરેજ વધારો}: ઊર્જા કેન્દ્રિત કરીને રેન્જ વિસ્તારે છે
\item
  \textbf{હસ્તક્ષેપ ઘટાડો}: અનિચ્છનીય સિગ્નલ્સને શૂન્ય કરે છે
\item
  \textbf{સિગ્નલ ગુણવત્તા}: બીમ કેન્દ્રિત કરવા દ્વારા વધુ સારો SNR
\item
  \textbf{ઊર્જા કાર્યક્ષમતા}: ઓછી ટ્રાન્સમિટ પાવર જરૂરિયાતો
\item
  \textbf{સ્પેશિયલ મલ્ટિપ્લેક્સિંગ}: એક જ આવૃત્તિમાં અનેક ડેટા સ્ટ્રીમ્સ
\item
  \textbf{એડેપ્ટિવ ઓપરેશન}: બદલાતા વાતાવરણ સાથે ગતિશીલ રીતે અનુકૂલન
\end{itemize}

\end{solutionbox}
\begin{mnemonicbox}
``PRECISE: પાવર રિડક્શન, એન્હાન્સ્ડ કવરેજ, ઇન્ટરફેરન્સ
સપ્રેશન, એન્હાન્સ્ડ સિગ્નલ''

\end{mnemonicbox}
\subsection*{પ્રશ્ન 5(ક) [7
ગુણ]}\label{uxaaauxab0uxab6uxaa8-5uxa95-7-uxa97uxaa3}

\textbf{DTH રીસીવર ઇન્ડોર અને આઉટડોર બ્લેક ડાયાગ્રામ દોરો અને તેના કાર્યોની ચર્ચા
કરો}

\begin{solutionbox}

\textbf{આકૃતિ: DTH સિસ્ટમ બ્લોક ડાયાગ્રામ}

\begin{center}
\textbf{Mermaid Diagram (Code)}
\begin{verbatim}
{Shaded}
{Highlighting}[]
graph LR
    subgraph "આઉટડોર યુનિટ"
    A[ડિશ એન્ટેના] {-{-}{}|એકત્રિત કરે છે| B[LNB {-} લો નોઇઝ બ્લોક]}
    end

    subgraph "ઇનડોર યુનિટ"
    C[ટ્યુનર] {-{-}{} D[ડિમોડ્યુલેટર]}
    D {-{-}{} E[ડિકોડર]}
    E {-{-}{} F[MPEG પ્રોસેસર]}
    F {-{-}{} G[વિડિયો/ઓડિયો આઉટપુટ]}
    H[સ્માર્ટ કાર્ડ] {-{-}{} E}
    I[યુઝર ઇન્ટરફેસ] {-{-}{} E}
    end
    
    B {-{-}{}|કોએક્સિયલ કેબલ| C}
{Highlighting}
{Shaded}
\end{verbatim}
\end{center}

\textbf{આઉટડોર યુનિટ ઘટકો અને કાર્યો:}

\begin{itemize}
\tightlist
\item
  \textbf{ડિશ એન્ટેના}: ઉપગ્રહ સિગ્નલ એકત્રિત કરે છે, સામાન્ય રીતે 45-90 સેમી વ્યાસ
\item
  \textbf{LNB (લો નોઇઝ બ્લોક)}:

  \begin{itemize}
  \tightlist
  \item
    ઉચ્ચ આવૃત્તિના ઉપગ્રહ સિગ્નલ (10-12 GHz) ને નીચી IF આવૃત્તિઓ (950-2150 MHz)
    માં રૂપાંતરિત કરે છે
  \item
    લઘુતમ ઘોંઘાટ સાથે નબળા સિગ્નલ્સને મજબૂત કરે છે
  \item
    સ્થાનિક ઓસિલેટર અને ધ્રુવીકરણ પસંદગી ધરાવે છે
  \end{itemize}
\end{itemize}

\textbf{ઇનડોર યુનિટ ઘટકો અને કાર્યો:}

\begin{itemize}
\tightlist
\item
  \textbf{ટ્યુનર}: ઇચ્છિત ટ્રાન્સપોન્ડર આવૃત્તિ પસંદ કરે છે
\item
  \textbf{ડિમોડ્યુલેટર}: મોડ્યુલેટેડ કેરિયરમાંથી ડિજિટલ સિગ્નલ અલગ કરે છે
\item
  \textbf{ડિકોડર}: સ્માર્ટ કાર્ડ અધિકૃતતા વાપરીને એન્ક્રિપ્ટેડ ચેનલોને ડિક્રિપ્ટ કરે છે
\item
  \textbf{MPEG પ્રોસેસર}: વિડિયો/ઓડિયો ડેટા સ્ટ્રીમ્સને ડિકમ્પ્રેસ કરે છે
\item
  \textbf{યુઝર ઇન્ટરફેસ}: ઓન-સ્ક્રીન મેનુ, પ્રોગ્રામ ગાઇડ, ચેનલ પસંદગી
\item
  \textbf{સ્માર્ટ કાર્ડ}: સબ્સ્ક્રિપ્શન વિગતો અને ડિક્રિપ્શન કી ધરાવે છે
\end{itemize}

\end{solutionbox}
\begin{mnemonicbox}
``COLD-TDUMS: કલેક્શન, ઓસિલેટર, લો-નોઇઝ, ડાઉનકન્વર્ઝન -
ટ્યુનર ડિમોડ્યુલેટર અનસ્ક્રેમ્બલર MPEG સ્માર્ટ-કાર્ડ''

\end{mnemonicbox}
\subsection*{પ્રશ્ન 5(અ) અથવા [3
ગુણ]}\label{uxaaauxab0uxab6uxaa8-5uxa85-uxa85uxaa5uxab5-3-uxa97uxaa3}

\textbf{વ્યાખ્યાયિત કરો: (1) એન્ટેના, (2) ફોલ્ડેડ ડાયપોલ, અને (3) એન્ટેના એરે}

\begin{solutionbox}


{\def\LTcaptype{none} % do not increment counter
\vspace{-5pt}
\captionof{table}{એન્ટેના વ્યાખ્યાઓ}
\vspace{-10pt}
\begin{longtable}[]{@{}
  >{\raggedright\arraybackslash}p{(\linewidth - 2\tabcolsep) * \real{0.3333}}
  >{\raggedright\arraybackslash}p{(\linewidth - 2\tabcolsep) * \real{0.6667}}@{}}
\toprule\noalign{}
\begin{minipage}[b]{\linewidth}\raggedright
શબ્દ
\end{minipage} & \begin{minipage}[b]{\linewidth}\raggedright
વ્યાખ્યા
\end{minipage} \\
\midrule\noalign{}
\endhead
\bottomrule\noalign{}
\endlastfoot
\textbf{એન્ટેના} & ઉપકરણ જે ઇલેક્ટ્રિકલ ઊર્જાને રેડિયો તરંગોમાં અને તેનાથી ઉલટું
રૂપાંતરિત કરે છે \\
\textbf{ફોલ્ડેડ ડાયપોલ} & ડાયપોલ જેના છેડા પાછા વાળીને જોડાયેલા છે, ઉચ્ચ
પ્રતિબાધા સાથે લૂપ બનાવે છે \\
\textbf{એન્ટેના એરે} & સુધારેલી દિશાત્મકતા/ગેઇન માટે ચોક્કસ પેટર્નમાં ગોઠવાયેલ અનેક
એન્ટેના \\
\end{longtable}
}

\end{solutionbox}
\begin{mnemonicbox}
``AFA: એન્ટેના ફોલ્ડેડ એરે''

\end{mnemonicbox}
\subsection*{પ્રશ્ન 5(બ) અથવા [4
ગુણ]}\label{uxaaauxab0uxab6uxaa8-5uxaac-uxa85uxaa5uxab5-4-uxa97uxaa3}

\textbf{સ્માર્ટ એન્ટેનાના ઉપયોગનું વર્ણન કરો}

\begin{solutionbox}


{\def\LTcaptype{none} % do not increment counter
\vspace{-5pt}
\captionof{table}{સ્માર્ટ એન્ટેના એપ્લિકેશન્સ}
\vspace{-10pt}
\begin{longtable}[]{@{}ll@{}}
\toprule\noalign{}
એપ્લિકેશન & વર્ણન \\
\midrule\noalign{}
\endhead
\bottomrule\noalign{}
\endlastfoot
\textbf{મોબાઇલ કોમ્યુનિકેશન્સ} & સેલ્યુલર નેટવર્ક્સમાં ક્ષમતા વધારે છે, હસ્તક્ષેપ ઘટાડે
છે \\
\textbf{બેઝ સ્ટેશન્સ} & સેક્ટર-વિશિષ્ટ કવરેજ, એડેપ્ટિવ બીમફોર્મિંગ \\
\textbf{MIMO સિસ્ટમ્સ} & સ્પેશિયલ મલ્ટિપ્લેક્સિંગ માટે મલ્ટિપલ-ઇનપુટ-મલ્ટિપલ-આઉટપુટ \\
\textbf{રડાર સિસ્ટમ્સ} & સુધારેલી લક્ષ્ય શોધ અને ટ્રેકિંગ \\
\textbf{ઉપગ્રહ સંચાર} & સ્પોટ બીમ જનરેશન, હસ્તક્ષેપ નિવારણ \\
\textbf{Wi-Fi નેટવર્ક્સ} & વાયરલેસ LAN માટે વર્ધિત રેન્જ અને થ્રૂપુટ \\
\textbf{IoT નેટવર્ક્સ} & IoT ઉપકરણો માટે ઓછી-પાવર, લાંબા-અંતરની કનેક્ટિવિટી \\
\end{longtable}
}

\end{solutionbox}
\begin{mnemonicbox}
``MBMRSWI: મોબાઇલ બેઝ MIMO રડાર સેટેલાઇટ Wi-Fi IoT''

\end{mnemonicbox}
\subsection*{પ્રશ્ન 5(ક) અથવા [7
ગુણ]}\label{uxaaauxab0uxab6uxaa8-5uxa95-uxa85uxaa5uxab5-7-uxa97uxaa3}

\textbf{ટેરેસ્ટ્રીયલ મોબાઈલ કોમ્યુનિકેશન એન્ટેના સમજાવો અને બેઝ સ્ટેશન અને મોબાઈલ સ્ટેશન
એન્ટેના વિશે પણ ચર્ચા કરો}

\begin{solutionbox}

\textbf{આકૃતિ: મોબાઇલ કોમ્યુનિકેશન એન્ટેનાના પ્રકારો}

\begin{center}
\textbf{Mermaid Diagram (Code)}
\begin{verbatim}
{Shaded}
{Highlighting}[]
graph TD
    A[ટેરેસ્ટ્રીયલ મોબાઇલ એન્ટેના] {-{-}{} B[બેઝ સ્ટેશન એન્ટેના]}
    A {-{-}{} C[મોબાઇલ સ્ટેશન એન્ટેના]}

    B {-{-}{} D[પેનલ એન્ટેના]}
    B {-{-}{} E[સેક્ટર એન્ટેના]}
    B {-{-}{} F[ઓમ્નિડાયરેક્શનલ એન્ટેના]}
    B {-{-}{} G[સ્માર્ટ એન્ટેના]}
    
    C {-{-}{} H[વ્હિપ એન્ટેના]}
    C {-{-}{} I[હેલિકલ એન્ટેના]}
    C {-{-}{} J[પ્લેનર ઇન્વર્ટેડ{-}F એન્ટેના]}
    C {-{-}{} K[ઇન્ટરનલ PCB એન્ટેના]}
{Highlighting}
{Shaded}
\end{verbatim}
\end{center}

\textbf{બેઝ સ્ટેશન એન્ટેના:}

\begin{itemize}
\tightlist
\item
  \textbf{પેનલ/સેક્ટર એન્ટેના}: પ્રતિ સેક્ટર 65^\circ-120^\circ કવરેજ, સામાન્ય રીતે સાઇટ દીઠ
  ત્રણ સેક્ટર
\item
  \textbf{લાક્ષણિકતાઓ}:

  \begin{itemize}
  \tightlist
  \item
    ઉચ્ચ ગેઇન (10-18 dBi)
  \item
    ઊભું ધ્રુવીકરણ
  \item
    ડાઉનટિલ્ટ ક્ષમતા (યાંત્રિક અથવા ઇલેક્ટ્રિકલ)
  \item
    મલ્ટી-બેન્ડ ઓપરેશન
  \end{itemize}
\item
  \textbf{ઊંચાઈ}: મહત્તમ કવરેજ માટે 15-50m ઊંચા ટાવર પર લગાવેલ
\item
  \textbf{પેટર્ન કંટ્રોલ}: અડજસન્ટ સેલમાં હસ્તક્ષેપને ન્યૂનતમ કરે છે
\end{itemize}

\textbf{મોબાઇલ સ્ટેશન એન્ટેના:}

\begin{itemize}
\tightlist
\item
  \textbf{બાહ્ય એન્ટેના}: આજે ઓછા સામાન્ય, મુખ્યત્વે વાહનો અથવા ગ્રામીણ વિસ્તારો
  માટે

  \begin{itemize}
  \tightlist
  \item
    વ્હિપ એન્ટેના (¼λ મોનોપોલ)
  \item
    નમનીયતા માટે હેલિકલ ડિઝાઇન
  \end{itemize}
\item
  \textbf{આંતરિક એન્ટેના}: હવે હેન્ડસેટમાં પ્રબળ

  \begin{itemize}
  \tightlist
  \item
    PIFA (પ્લેનર ઇન્વર્ટેડ-F એન્ટેના)
  \item
    PCB ટ્રેસ એન્ટેના
  \item
    લાક્ષણિકતાઓ:

    \begin{itemize}
    \tightlist
    \item
      નાનું કદ
    \item
      મલ્ટી-બેન્ડ ઓપરેશન
    \item
      ઓમ્નિડાયરેક્શનલ પેટર્ન
    \item
      ઓછી કાર્યક્ષમતા (સામાન્ય રીતે -3 થી -6 dBi)
    \end{itemize}
  \end{itemize}
\end{itemize}

\end{solutionbox}
\begin{mnemonicbox}
``BEST-POMME: બેઝ-સ્ટેશન એક્સટર્નલ સેક્ટર ટાવર - પોર્ટેબલ
ઓમ્નિડાયરેક્શનલ મલ્ટી-બેન્ડ મોબાઇલ એમ્બેડેડ''

\end{mnemonicbox}

\end{document}
