\documentclass[10pt,a4paper]{article}

% content/resources/templates/preamble.tex
\usepackage[margin=0.6in]{geometry}
\author{Milav Dabgar}
\usepackage{amsmath,amssymb,amsthm}
\usepackage{booktabs}
\usepackage{multirow}
\usepackage{xcolor}
\usepackage{tcolorbox}
\tcbuselibrary{breakable,skins}
\usepackage[colorlinks=true,linkcolor=blue]{hyperref}
\usepackage{titlesec}
\usepackage{enumitem}
\usepackage{tikz}
\usepackage{pgfplots}
\usepackage{circuitikz}
\usepackage[version=4]{mhchem}
\usepackage{longtable}
\usepackage{array}
\usepackage{float}
\usepackage{caption}
\usepackage{listings}

\lstset{
  basicstyle=\small\ttfamily,
  breaklines=true,
  breakatwhitespace=false,
  postbreak=\mbox{\textcolor{red}{$\hookrightarrow$}\space},
  float=false,
  numbers=left,
  numberstyle=\tiny\color{gray},
  numbersep=10pt,
  xleftmargin=2em,
  keywordstyle=\color{blue},
  commentstyle=\color{green!60!black},
  stringstyle=\color{purple},
  backgroundcolor=\color{gray!5},
  showstringspaces=false,
  tabsize=2,
  captionpos=b,
  keepspaces=true,
  columns=flexible
}

\pgfplotsset{compat=1.18}
\usetikzlibrary{shapes,arrows,positioning,calc,patterns,decorations.pathmorphing,decorations.markings,arrows.meta}

% Color scheme
\definecolor{headcolor}{RGB}{0,102,204}
\definecolor{keycolor}{RGB}{220,20,60}
\definecolor{solutioncolor}{RGB}{34,139,34}
\definecolor{mnemoniccolor}{RGB}{148,0,211}
\definecolor{codecolor}{RGB}{0,0,100}

% Spacing
\setlength{\parskip}{3pt}
\setlist[itemize]{nosep}
\setlist[enumerate]{nosep}

% Title formatting
\titleformat{\section}{\Large\bfseries\color{headcolor}}{\thesection}{1em}{}
\titleformat{\subsection}{\large\bfseries\color{headcolor}}{\thesubsection}{1em}{}

% Pandoc tightlist compatibility
\providecommand{\tightlist}{%
  \setlength{\itemsep}{0pt}\setlength{\parskip}{0pt}}

% Pandoc longtable compatibility
\newcounter{none}
\def\thenone{}


% content/resources/templates/english-boxes.tex
% This file is currently empty - it exists to maintain consistency with the import structure.
% Add custom environments here if needed in the future.


\begin{document}

\begin{center}
{\Huge\bfseries\color{headcolor} Subject Name Solutions}\\[5pt]
{\LARGE 4341106 -- Summer 2023}\\[3pt]
{\large Semester 1 Study Material}\\[3pt]
{\normalsize\textit{Detailed Solutions and Explanations}}
\end{center}

\vspace{10pt}

\subsection*{Question 1(a) [3 marks]}\label{q1a}

\textbf{Write any three properties of Electromagnetic waves}

\begin{solutionbox}

{\def\LTcaptype{none} % do not increment counter
\begin{longtable}[]{@{}
  >{\raggedright\arraybackslash}p{(\linewidth - 0\tabcolsep) * \real{1.0000}}@{}}
\toprule\noalign{}
\begin{minipage}[b]{\linewidth}\raggedright
Properties of Electromagnetic Waves
\end{minipage} \\
\midrule\noalign{}
\endhead
\bottomrule\noalign{}
\endlastfoot
1. EM waves can travel through vacuum or material media \\
2. EM waves travel at the speed of light in free space (3\times10^{8} m/s) \\
3. EM waves exhibit transverse wave characteristics with oscillating
electric and magnetic fields \\
\end{longtable}
}

\end{solutionbox}
\begin{mnemonicbox}
``VTS'' - Vacuum travel, Transverse nature, Speed of
light

\end{mnemonicbox}
\subsection*{Question 1(b) [4 marks]}\label{q1b}

\textbf{Define: (1) Radiation resistance (2) Directivity (3) Gain}

\begin{solutionbox}

{\def\LTcaptype{none} % do not increment counter
\begin{longtable}[]{@{}
  >{\raggedright\arraybackslash}p{(\linewidth - 2\tabcolsep) * \real{0.3333}}
  >{\raggedright\arraybackslash}p{(\linewidth - 2\tabcolsep) * \real{0.6667}}@{}}
\toprule\noalign{}
\begin{minipage}[b]{\linewidth}\raggedright
Term
\end{minipage} & \begin{minipage}[b]{\linewidth}\raggedright
Definition
\end{minipage} \\
\midrule\noalign{}
\endhead
\bottomrule\noalign{}
\endlastfoot
\textbf{Radiation resistance} & The equivalent resistance that would
dissipate the same amount of power as radiated by an antenna when the
current at the feed point is equal to the antenna input current \\
\textbf{Directivity} & The ratio of maximum radiation intensity in a
specific direction to the average radiation intensity in all
directions \\
\textbf{Gain} & The product of directivity and radiation efficiency,
measuring how efficiently an antenna converts input power into radio
waves in a specific direction \\
\end{longtable}
}

\end{solutionbox}
\begin{mnemonicbox}
``RDG'' - Resistance dissipates power, Direction
concentration, Gain includes efficiency

\end{mnemonicbox}
\subsection*{Question 1(c) [7 marks]}\label{q1c}

\textbf{Explain physical concept of generation of Electromagnetic waves
with neat diagram}

\begin{solutionbox}

Electromagnetic waves are generated when electric charges accelerate or
oscillate, creating coupled oscillating electric and magnetic fields
that propagate through space.

\begin{center}
\textbf{Mermaid Diagram (Code)}
\begin{verbatim}
{Shaded}
{Highlighting}[]
graph LR
    A[Electric Current Flow] {-{-}{}|Oscillation| B[Oscillating Electric Field]}
    B {-{-}{}|Induces| C[Oscillating Magnetic Field]}
    C {-{-}{}|Induces| D[Oscillating Electric Field]}
    D {-{-}{} E[Self{-}sustaining wave propagation]}
{Highlighting}
{Shaded}
\end{verbatim}
\end{center}

\textbf{Diagram: Dipole Antenna EM Wave Generation}

\begin{verbatim}
                 +
                 |
                 |
    Oscillator   |      Electric field lines
    +{-{-}{-}{-}{-}{-}{-}{-}{-}+  |      ∽∽∽∽∽∽∽∽∽∽∽∽∽∽∽}
    |         |  |
    |    {    |{-}{-}+      Magnetic field lines}
    |         |  |      ⊙⊙⊙⊙⊙⊙⊙⊙⊙⊙⊙⊙⊙⊙⊙
    +{-{-}{-}{-}{-}{-}{-}{-}{-}+  |}
                 |
                 |
                 {-}
\end{verbatim}

\begin{itemize}
\tightlist
\item
  \textbf{Basic concept}: When AC current flows in the antenna,
  electrons accelerate up and down
\item
  \textbf{Electric field}: Created by charge separation in the antenna
\item
  \textbf{Magnetic field}: Produced by the current flow, perpendicular
  to electric field
\item
  \textbf{Propagation}: Fields detach from antenna and propagate outward
  at the speed of light
\item
  \textbf{Self-sustaining}: Each field component regenerates the other
  as wave travels
\end{itemize}

\end{solutionbox}
\begin{mnemonicbox}
``COMAP'' - Current Oscillations Make Alternating
Propagations

\end{mnemonicbox}
\subsection*{Question 1(c) OR [7
marks]}\label{q1c}

\textbf{Design 4 Element Yagi Uda antenna for frequency of 435 MHz with
necessary equations}

\begin{solutionbox}

For a 4-element Yagi-Uda antenna at 435 MHz:

{\def\LTcaptype{none} % do not increment counter
\begin{longtable}[]{@{}
  >{\raggedright\arraybackslash}p{(\linewidth - 6\tabcolsep) * \real{0.1500}}
  >{\raggedright\arraybackslash}p{(\linewidth - 6\tabcolsep) * \real{0.2667}}
  >{\raggedright\arraybackslash}p{(\linewidth - 6\tabcolsep) * \real{0.2833}}
  >{\raggedright\arraybackslash}p{(\linewidth - 6\tabcolsep) * \real{0.3000}}@{}}
\toprule\noalign{}
\begin{minipage}[b]{\linewidth}\raggedright
Element
\end{minipage} & \begin{minipage}[b]{\linewidth}\raggedright
Length Formula
\end{minipage} & \begin{minipage}[b]{\linewidth}\raggedright
Spacing Formula
\end{minipage} & \begin{minipage}[b]{\linewidth}\raggedright
Calculated Value
\end{minipage} \\
\midrule\noalign{}
\endhead
\bottomrule\noalign{}
\endlastfoot
\textbf{Reflector} & 0.5λ \times 1.05 & - & 36.2 cm \\
\textbf{Driven element} & 0.5λ & - & 34.5 cm \\
\textbf{Director 1} & 0.45λ & 0.2λ from driven & 31.0 cm at 13.8 cm
spacing \\
\textbf{Director 2} & 0.43λ & 0.25λ from Director 1 & 29.6 cm at 17.2 cm
spacing \\
\end{longtable}
}

\textbf{Equations used}:

\begin{itemize}
\tightlist
\item
Wavelength:

λ = c/f = 3\times10^{8}/435\times10^{6} = 0.69 meters

\item
Half-wave dipole:

L = 0.5λ = 34.5 cm

\item
  Element spacing: S = 0.15λ to 0.25λ
\end{itemize}

\begin{center}
\textbf{Mermaid Diagram (Code)}
\begin{verbatim}
{Shaded}
{Highlighting}[]
graph LR
    A[Reflector: 36.2cm] {-{-}{-} B[Driven Element: 34.5cm]}
    B {-{-}{-} C[Director 1: 31.0cm]}
    C {-{-}{-} D[Director 2: 29.6cm]}

    style A fill:\#f9f,stroke:\#333,stroke{-width:2px}
    style B fill:\#bbf,stroke:\#333,stroke{-width:2px}
    style C fill:\#fbb,stroke:\#333,stroke{-width:2px}
    style D fill:\#fbb,stroke:\#333,stroke{-width:2px}
{Highlighting}
{Shaded}
\end{verbatim}
\end{center}

\end{solutionbox}
\begin{mnemonicbox}
``RDDS'' - Reflector Driven Directors Shrink

\end{mnemonicbox}
\subsection*{Question 2(a) [3 marks]}\label{q2a}

\textbf{Explain Loop antenna with diagram}

\begin{solutionbox}

Loop antenna is a radiating element formed by shaping a conductor into a
loop.

\begin{verbatim}
    ┌───────────┐
    │           │
    │           │
    │           │
    │           │
    │     ↺     │ Current flow
    │           │
    │           │
    │           │
    └─────┬─────┘
          │
          │ Feed point
       ───┴───
\end{verbatim}

\begin{itemize}
\tightlist
\item
  \textbf{Small loops}: Circumference \textless{} λ/10, radiation
  pattern similar to magnetic dipole
\item
  \textbf{Large loops}: Circumference \approx wavelength, bidirectional
  radiation pattern
\item
  \textbf{Applications}: Direction finding, AM radio reception, RFID
  tags
\end{itemize}

\end{solutionbox}
\begin{mnemonicbox}
``SLC'' - Size affects Loop Characteristics

\end{mnemonicbox}
\subsection*{Question 2(b) [4 marks]}\label{q2b}

\textbf{Explain Non Resonant wire antenna}

\begin{solutionbox}

{\def\LTcaptype{none} % do not increment counter
\begin{longtable}[]{@{}
  >{\raggedright\arraybackslash}p{(\linewidth - 2\tabcolsep) * \real{0.5517}}
  >{\raggedright\arraybackslash}p{(\linewidth - 2\tabcolsep) * \real{0.4483}}@{}}
\toprule\noalign{}
\begin{minipage}[b]{\linewidth}\raggedright
Characteristic
\end{minipage} & \begin{minipage}[b]{\linewidth}\raggedright
Description
\end{minipage} \\
\midrule\noalign{}
\endhead
\bottomrule\noalign{}
\endlastfoot
\textbf{Definition} & Antenna operating at frequencies where its
physical length is not a multiple of half-wavelength \\
\textbf{Impedance} & Complex with both resistive and reactive
components \\
\textbf{Standing waves} & Present along the antenna length \\
\textbf{Example} & Rhombic antenna, terminated with resistance at the
end \\
\textbf{Advantage} & Wideband operation, suitable for multiple
frequencies \\
\end{longtable}
}

\end{solutionbox}
\begin{mnemonicbox}
``NITRO'' - Non-resonance Incurs Termination for
Resistance and Operation

\end{mnemonicbox}
\subsection*{Question 2(c) [7 marks]}\label{q2c}

\textbf{What is Radiation resistance of half wave dipole? Draw radiation
patterns of Dipoles of length λ/2, λ and λ/4 antenna}

\begin{solutionbox}

The radiation resistance of a half-wave dipole is approximately 73 ohms.

\textbf{Radiation patterns:}

\begin{verbatim}
   λ/2 Dipole             λ Dipole              λ/4 Dipole
   
      0^                     0^                     0^
      |                      |                      |
270^{-{-}+{-}{-}90^   vs.     270^-{-}{-}+{-}{-}{-}90^   vs.    270^-{-}{-}+{-}{-}{-}90^}
      |                      |                      |
     180^                   180^                   180^
    (Figure{-8)         (Multiple lobes)         (Broad pattern)}
\end{verbatim}

{\def\LTcaptype{none} % do not increment counter
\begin{longtable}[]{@{}
  >{\raggedright\arraybackslash}p{(\linewidth - 2\tabcolsep) * \real{0.3750}}
  >{\raggedright\arraybackslash}p{(\linewidth - 2\tabcolsep) * \real{0.6250}}@{}}
\toprule\noalign{}
\begin{minipage}[b]{\linewidth}\raggedright
Dipole Length
\end{minipage} & \begin{minipage}[b]{\linewidth}\raggedright
Pattern Characteristics
\end{minipage} \\
\midrule\noalign{}
\endhead
\bottomrule\noalign{}
\endlastfoot
\textbf{λ/2 dipole} & Figure-8 pattern; maximum radiation perpendicular
to antenna axis; HPBW = 78^\circ \\
\textbf{λ dipole} & Multi-lobed pattern; four main lobes at angles to
antenna axis \\
\textbf{λ/4 dipole} & Broader pattern than λ/2; requires ground plane to
complete the equivalent dipole \\
\end{longtable}
}

\end{solutionbox}
\begin{mnemonicbox}
``SHORT'' - Smaller Half-dipole Offers
Rounded-Transmissions

\end{mnemonicbox}
\subsection*{Question 2(a) OR [3
marks]}\label{q2a}

\textbf{Explain Folded dipole antenna with figure}

\begin{solutionbox}

Folded dipole is a variation of the half-wave dipole with ends folded
back and connected to form a loop.

\begin{verbatim}
    ┌───────────────────────────┐
    │                           │
    │                           │
    │                           │
    └───────────┬───────────────┘
                │
                │ Feed point
             ───┴───
\end{verbatim}

\begin{itemize}
\tightlist
\item
  \textbf{Input impedance}: Approximately 300 ohms (4 times that of
  simple dipole)
\item
  \textbf{Bandwidth}: Wider than simple dipole
\item
  \textbf{Applications}: TV reception, FM radio, balanced transmission
  lines
\end{itemize}

\end{solutionbox}
\begin{mnemonicbox}
``FIB'' - Folded Increases Bandwidth

\end{mnemonicbox}
\subsection*{Question 2(b) OR [4
marks]}\label{q2b}

\textbf{Explain Rhombic antenna with figure}

\begin{solutionbox}

Rhombic antenna consists of four wires arranged in a rhombus or diamond
shape.

\begin{verbatim}
              Direction of 
                radiation
                   ↓
            A ◄─────────► B
            /             {}
           /               {}
          /                 {}
Feed ────┐                   ┌──── Termination
          {                 /}
           {               /}
            {             /}
            D ◄─────────► C
\end{verbatim}

{\def\LTcaptype{none} % do not increment counter
\begin{longtable}[]{@{}ll@{}}
\toprule\noalign{}
Characteristic & Description \\
\midrule\noalign{}
\endhead
\bottomrule\noalign{}
\endlastfoot
\textbf{Shape} & Diamond/rhombus with terminating resistor at far end \\
\textbf{Operation} & Non-resonant traveling-wave antenna \\
\textbf{Directivity} & High gain, unidirectional pattern \\
\textbf{Bandwidth} & Very wide frequency range \\
\textbf{Applications} & HF communications, point-to-point links \\
\end{longtable}
}

\end{solutionbox}
\begin{mnemonicbox}
``TREND'' - Terminated Rhombic Enables Numerous
Directions

\end{mnemonicbox}
\subsection*{Question 2(c) OR [7
marks]}\label{q2c}

\textbf{Differentiate between Broadside array and End fire array with
suitable diagram}

\begin{solutionbox}

{\def\LTcaptype{none} % do not increment counter
\begin{longtable}[]{@{}
  >{\raggedright\arraybackslash}p{(\linewidth - 4\tabcolsep) * \real{0.2500}}
  >{\raggedright\arraybackslash}p{(\linewidth - 4\tabcolsep) * \real{0.3864}}
  >{\raggedright\arraybackslash}p{(\linewidth - 4\tabcolsep) * \real{0.3636}}@{}}
\toprule\noalign{}
\begin{minipage}[b]{\linewidth}\raggedright
Parameter
\end{minipage} & \begin{minipage}[b]{\linewidth}\raggedright
Broadside Array
\end{minipage} & \begin{minipage}[b]{\linewidth}\raggedright
End fire Array
\end{minipage} \\
\midrule\noalign{}
\endhead
\bottomrule\noalign{}
\endlastfoot
\textbf{Direction of maximum radiation} & Perpendicular to array axis &
Along array axis \\
\textbf{Element phasing} & Same phase (0^\circ) & Progressive phase shift \\
\textbf{Element spacing} & λ/2 typically & λ/4 typically \\
\textbf{Radiation pattern} & Fan-shaped beam & Pencil-shaped beam \\
\textbf{Applications} & Broadcasting, base stations & Point-to-point
links \\
\end{longtable}
}

\textbf{Diagram comparison:}

\begin{verbatim}
   Broadside Array                     End fire Array
   
   ─o─────o─────o─────o─         ─o─────o─────o─────o─
        Array Axis                     Array Axis
         
         ↑ ↑ ↑ ↑                 
      Main radiation                Main radiation
        direction                     direction
\end{verbatim}

\end{solutionbox}
\begin{mnemonicbox}
``PAPER'' - Perpendicular And Parallel Emission
Respectively

\end{mnemonicbox}
\subsection*{Question 3(a) [3 marks]}\label{q3a}

\textbf{Draw and Explain Inverted V antenna}

\begin{solutionbox}

Inverted V antenna is a dipole with arms angled downward, resembling an
inverted ``V''.

\begin{verbatim}
                   ▲
                   │ Support
                   │
                   │
                  /│{}
                 / │ {}
                /  │  {}
               /   │   {}
              /    │    {}
             /     │     {}
            ◄─────┐│┌─────►
                  ││
                Feed point
\end{verbatim}

\begin{itemize}
\tightlist
\item
  \textbf{Angle}: Arms typically form 90^\circ-120^\circ angle
\item
  \textbf{Impedance}: Close to 50 ohms, lower than horizontal dipole
\item
  \textbf{Pattern}: Omnidirectional, slightly broader than horizontal
  dipole
\item
  \textbf{Applications}: Amateur radio, shortwave communications
\end{itemize}

\end{solutionbox}
\begin{mnemonicbox}
``AVS'' - Angle Varies Signal

\end{mnemonicbox}
\subsection*{Question 3(b) [4 marks]}\label{q3b}

\textbf{Draw and explain parabolic reflector antenna}

\begin{solutionbox}

\begin{verbatim}
           │        ╱│╲        ┌───────►
           │      ╱  │  ╲      │
           │    ╱    │    ╲    │
           │  ╱      │      ╲  │
    ───────┼╱        ↓        ╲┼─┘
           │                   │
     Feed ─┤                   │
           │                   │
    ───────┼╲                 ╱┼───────►
           │  ╲             ╱  │
           │    ╲         ╱    │
           │      ╲     ╱      │
           │        ╲│╱        │
                   Focus
\end{verbatim}

{\def\LTcaptype{none} % do not increment counter
\begin{longtable}[]{@{}
  >{\raggedright\arraybackslash}p{(\linewidth - 2\tabcolsep) * \real{0.5238}}
  >{\raggedright\arraybackslash}p{(\linewidth - 2\tabcolsep) * \real{0.4762}}@{}}
\toprule\noalign{}
\begin{minipage}[b]{\linewidth}\raggedright
Component
\end{minipage} & \begin{minipage}[b]{\linewidth}\raggedright
Function
\end{minipage} \\
\midrule\noalign{}
\endhead
\bottomrule\noalign{}
\endlastfoot
\textbf{Parabolic reflector} & Collects and focuses incoming signals or
directs transmitted signals \\
\textbf{Feed element} & Located at focal point of parabola to
collect/emit signals \\
\textbf{Focal length} & Distance from vertex to focus, determines beam
characteristics \\
\textbf{Applications} & Satellite communications, radar, radio
astronomy, microwave links \\
\end{longtable}
}

\end{solutionbox}
\begin{mnemonicbox}
``FOLD'' - Focus Of Large Dish

\end{mnemonicbox}
\subsection*{Question 3(c) [7 marks]}\label{q3c}

\textbf{Write down range of frequencies for HF, VHF and UHF. Write short
note on Microstrip antenna.}

\begin{solutionbox}

{\def\LTcaptype{none} % do not increment counter
\begin{longtable}[]{@{}ll@{}}
\toprule\noalign{}
Frequency Band & Range \\
\midrule\noalign{}
\endhead
\bottomrule\noalign{}
\endlastfoot
\textbf{HF (High Frequency)} & 3 MHz - 30 MHz \\
\textbf{VHF (Very High Frequency)} & 30 MHz - 300 MHz \\
\textbf{UHF (Ultra High Frequency)} & 300 MHz - 3 GHz \\
\end{longtable}
}

\textbf{Microstrip Antenna:}

\begin{verbatim}
         ┌─────────────────────┐
         │  Radiating Patch    │
         └─────────────────────┘
    ┌───────────────────────────────┐ ─┐
    │      Dielectric Substrate     │  │h
    └───────────────────────────────┘ ─┘
    ┌───────────────────────────────┐
    │       Ground Plane            │
    └───────────────────────────────┘
\end{verbatim}

\begin{itemize}
\tightlist
\item
  \textbf{Structure}: Conductive patch on dielectric substrate with
  ground plane
\item
  \textbf{Feeding methods}: Microstrip line, coaxial probe,
  aperture-coupled
\item
  \textbf{Advantages}: Low profile, lightweight, easy fabrication,
  compatible with PCB
\item
  \textbf{Limitations}: Narrow bandwidth, low gain, low power handling
\item
  \textbf{Applications}: Mobile devices, RFID, GPS, satellite
  communications
\end{itemize}

\end{solutionbox}
\begin{mnemonicbox}
``PATCH'' - Planar Antenna That's Cheaply Handled

\end{mnemonicbox}
\subsection*{Question 3(a) OR [3
marks]}\label{q3a}

\textbf{Write Morse code for word: ``LINE OF SIGHT''}

\begin{solutionbox}

{\def\LTcaptype{none} % do not increment counter
\begin{longtable}[]{@{}ll@{}}
\toprule\noalign{}
Letter & Morse Code \\
\midrule\noalign{}
\endhead
\bottomrule\noalign{}
\endlastfoot
L & .-.. \\
I & .. \\
N & -. \\
E & . \\
(space) & / \\
O & --- \\
F & ..-. \\
(space) & / \\
S & \ldots{} \\
I & .. \\
G & --. \\
H & \ldots. \\
T & - \\
\end{longtable}
}

``LINE OF SIGHT'' in Morse code: .-.. .. -. . / --- ..-. / \ldots{} ..
--. \ldots. -

\end{solutionbox}
\begin{mnemonicbox}
``Listen In Now, Every Other Frequency Supports
Immediate Global Heightened Transmission''

\end{mnemonicbox}
\subsection*{Question 3(b) OR [4
marks]}\label{q3b}

\textbf{Draw and explain Turnstile \& Super turnstile antenna}

\begin{solutionbox}

\textbf{Turnstile Antenna:}

\begin{verbatim}
         ───┬───
            │
            │
    ────────┼────────
            │
            │
         ───┴───
\end{verbatim}

\textbf{Super Turnstile Antenna:}

\begin{verbatim}
       ┌───┐       ┌───┐
       │   │       │   │ 
       │   │       │   │
       └───┘       └───┘
       
       ┌───┐       ┌───┐
       │   │       │   │
       │   │       │   │
       └───┘       └───┘
\end{verbatim}

{\def\LTcaptype{none} % do not increment counter
\begin{longtable}[]{@{}
  >{\raggedright\arraybackslash}p{(\linewidth - 2\tabcolsep) * \real{0.2609}}
  >{\raggedright\arraybackslash}p{(\linewidth - 2\tabcolsep) * \real{0.7391}}@{}}
\toprule\noalign{}
\begin{minipage}[b]{\linewidth}\raggedright
Type
\end{minipage} & \begin{minipage}[b]{\linewidth}\raggedright
Characteristics
\end{minipage} \\
\midrule\noalign{}
\endhead
\bottomrule\noalign{}
\endlastfoot
\textbf{Turnstile} & Two horizontal dipoles at right angles, fed 90^\circ out
of phase \\
\textbf{Super Turnstile} & Modification with multiple elements forming
rectangular loops \\
\textbf{Pattern} & Omnidirectional in horizontal plane, figure-8 in
vertical \\
\textbf{Polarization} & Horizontal or circular polarization \\
\textbf{Applications} & TV broadcasting, FM broadcasting, satellite
communications \\
\end{longtable}
}

\end{solutionbox}
\begin{mnemonicbox}
``TOPS'' - Turnstile Offers Perpendicular Symmetry

\end{mnemonicbox}
\subsection*{Question 3(c) OR [7
marks]}\label{q3c}

\textbf{What is Polarization? Explain Helical antenna in detail with
diagram}

\begin{solutionbox}

\textbf{Polarization} is the orientation of the electric field vector of
an electromagnetic wave as it propagates through space.

\textbf{Helical Antenna:}

\begin{verbatim}
          ┌─┐     ┌─┐
         /   {   /   }
        │     { /     │}
        │      X      │
        │     / {     │}
         {   /      /}
          └─┘     └─┘
               │
               │
           ────┴────
          Ground plane
\end{verbatim}

{\def\LTcaptype{none} % do not increment counter
\begin{longtable}[]{@{}
  >{\raggedright\arraybackslash}p{(\linewidth - 2\tabcolsep) * \real{0.4583}}
  >{\raggedright\arraybackslash}p{(\linewidth - 2\tabcolsep) * \real{0.5417}}@{}}
\toprule\noalign{}
\begin{minipage}[b]{\linewidth}\raggedright
Parameter
\end{minipage} & \begin{minipage}[b]{\linewidth}\raggedright
Description
\end{minipage} \\
\midrule\noalign{}
\endhead
\bottomrule\noalign{}
\endlastfoot
\textbf{Structure} & Conductor wound in helical shape above ground
plane \\
\textbf{Diameter} & Typically λ/π \\
\textbf{Pitch} & Spacing between turns, usually λ/4 \\
\textbf{Turns} & 3-10 turns depending on gain requirements \\
\textbf{Modes} & Normal mode (broadside) or Axial mode (end-fire) \\
\textbf{Polarization} & Circular polarization in axial mode \\
\textbf{Applications} & Satellite communications, space telemetry,
tracking \\
\end{longtable}
}

\end{solutionbox}
\begin{mnemonicbox}
``HASP'' - Helical Antenna Supports Polarization

\end{mnemonicbox}
\subsection*{Question 4(a) [3 marks]}\label{q4a}

\textbf{Explain Tropospheric scattered propagation}

\begin{solutionbox}

{\def\LTcaptype{none} % do not increment counter
\begin{longtable}[]{@{}
  >{\raggedright\arraybackslash}p{(\linewidth - 2\tabcolsep) * \real{0.3810}}
  >{\raggedright\arraybackslash}p{(\linewidth - 2\tabcolsep) * \real{0.6190}}@{}}
\toprule\noalign{}
\begin{minipage}[b]{\linewidth}\raggedright
Aspect
\end{minipage} & \begin{minipage}[b]{\linewidth}\raggedright
Description
\end{minipage} \\
\midrule\noalign{}
\endhead
\bottomrule\noalign{}
\endlastfoot
\textbf{Mechanism} & Radio signals scatter from tropospheric
irregularities and refractive index variations \\
\textbf{Frequency} & Typically VHF, UHF (100 MHz - 10 GHz) \\
\textbf{Range} & 100-800 km, beyond line-of-sight \\
\textbf{Reliability} & Less affected by weather than line-of-sight; more
reliable than ionospheric \\
\textbf{Applications} & Military communications, remote areas where
other systems aren't practical \\
\end{longtable}
}

\end{solutionbox}
\begin{mnemonicbox}
``STRIP'' - Scatter Through Refractive Index Patterns

\end{mnemonicbox}
\subsection*{Question 4(b) [4 marks]}\label{q4b}

\textbf{Define: (1) Virtual Height (2) Maximum Usable Frequency - MUF
(3) Critical Frequency}

\begin{solutionbox}

{\def\LTcaptype{none} % do not increment counter
\begin{longtable}[]{@{}
  >{\raggedright\arraybackslash}p{(\linewidth - 2\tabcolsep) * \real{0.3333}}
  >{\raggedright\arraybackslash}p{(\linewidth - 2\tabcolsep) * \real{0.6667}}@{}}
\toprule\noalign{}
\begin{minipage}[b]{\linewidth}\raggedright
Term
\end{minipage} & \begin{minipage}[b]{\linewidth}\raggedright
Definition
\end{minipage} \\
\midrule\noalign{}
\endhead
\bottomrule\noalign{}
\endlastfoot
\textbf{Virtual Height} & The apparent height of the ionosphere
calculated from the time delay of a radio signal reflected back to
Earth, as if reflection occurred at a single point \\
\textbf{Maximum Usable Frequency (MUF)} & The highest frequency that can
be used for reliable communication via ionospheric reflection for a
specified path and time \\
\textbf{Critical Frequency} & The highest frequency that can be
reflected back when transmitted vertically to the ionosphere (when angle
of incidence is 90^\circ) \\
\end{longtable}
}

\end{solutionbox}
\begin{mnemonicbox}
``VMC'' - Virtual height Measures Critical reflection

\end{mnemonicbox}
\subsection*{Question 4(c) [7 marks]}\label{q4c}

\textbf{Explain effect of ground on electromagnetic wave propagation}

\begin{solutionbox}

\begin{verbatim}
              /|{ Direct wave}
               |
   Transmitter | Receiver
      o        |        o
       {       |       /}
        {      |      /}
         {     |     /}
          {    |    /}
           {   |   /}
            {  |  /}
             { | /}
              {|/}
     ──────────────────────
           Ground
     .............|...........
                  |
                  | Ground reflected wave
                 {|/}
\end{verbatim}

{\def\LTcaptype{none} % do not increment counter
\begin{longtable}[]{@{}
  >{\raggedright\arraybackslash}p{(\linewidth - 2\tabcolsep) * \real{0.3810}}
  >{\raggedright\arraybackslash}p{(\linewidth - 2\tabcolsep) * \real{0.6190}}@{}}
\toprule\noalign{}
\begin{minipage}[b]{\linewidth}\raggedright
Effect
\end{minipage} & \begin{minipage}[b]{\linewidth}\raggedright
Description
\end{minipage} \\
\midrule\noalign{}
\endhead
\bottomrule\noalign{}
\endlastfoot
\textbf{Ground reflection} & Signal reflects off ground, causing
multipath reception \\
\textbf{Ground absorption} & Part of signal energy absorbed by ground,
reducing signal strength \\
\textbf{Ground diffraction} & Waves bend around obstacles, extending
coverage beyond line-of-sight \\
\textbf{Earth curvature} & Limits line-of-sight distance based on
antenna height \\
\textbf{Ground conductivity} & Higher conductivity (water, wet soil)
allows better propagation than poor conductors (dry, rocky terrain) \\
\end{longtable}
}

\textbf{Wave behavior equation:}

\begin{itemize}
\tightlist
\item
  Range (km) \approx 4.12(\sqrth_{1} + \sqrth_{2}) where h_{1}, h_{2} are antenna heights in
  meters
\end{itemize}

\end{solutionbox}
\begin{mnemonicbox}
``RADAR'' - Reflection Absorption Diffraction Affect
Range

\end{mnemonicbox}
\subsection*{Question 4(a) OR [3
marks]}\label{q4a}

\textbf{Explain Duct Propagation}

\begin{solutionbox}

Duct propagation occurs when radio waves become trapped in atmospheric
layers with special refractive properties.

\begin{verbatim}
   ──────────────────────────────────
   Normal atmosphere
   ──────────────────────────────────
   Temperature inversion layer
   ∿∿∿∿∿∿∿∿∿∿∿∿∿∿∿∿∿∿∿∿∿∿∿∿∿∿∿∿∿∿∿∿
   o TX                           o RX
   ──────────────────────────────────
   Normal atmosphere
   ──────────────────────────────────
\end{verbatim}

\begin{itemize}
\tightlist
\item
  \textbf{Formation}: Temperature inversions or moisture gradients
  create atmospheric ducts
\item
  \textbf{Effect}: Signals trapped within duct, allowing propagation far
  beyond normal range
\item
  \textbf{Frequencies}: Most common in UHF and microwave bands
\item
  \textbf{Applications}: Extended over-water communications, radar
  anomalies
\end{itemize}

\end{solutionbox}
\begin{mnemonicbox}
``TIDE'' - Trapped In Ducting Environment

\end{mnemonicbox}
\subsection*{Question 4(b) OR [4
marks]}\label{q4b}

\textbf{Explain different layers of Ionosphere}

\begin{solutionbox}

{\def\LTcaptype{none} % do not increment counter
\begin{longtable}[]{@{}
  >{\raggedright\arraybackslash}p{(\linewidth - 4\tabcolsep) * \real{0.2059}}
  >{\raggedright\arraybackslash}p{(\linewidth - 4\tabcolsep) * \real{0.2941}}
  >{\raggedright\arraybackslash}p{(\linewidth - 4\tabcolsep) * \real{0.5000}}@{}}
\toprule\noalign{}
\begin{minipage}[b]{\linewidth}\raggedright
Layer
\end{minipage} & \begin{minipage}[b]{\linewidth}\raggedright
Altitude
\end{minipage} & \begin{minipage}[b]{\linewidth}\raggedright
Characteristics
\end{minipage} \\
\midrule\noalign{}
\endhead
\bottomrule\noalign{}
\endlastfoot
\textbf{D Layer} & 60-90 km & Absorbs HF waves during daytime,
disappears at night \\
\textbf{E Layer} & 90-150 km & Reflects frequencies up to 10 MHz,
sporadic E phenomenon \\
\textbf{F1 Layer} & 150-210 km & Present during day, merges with F2 at
night \\
\textbf{F2 Layer} & 210-400+ km & Main reflecting layer, highest
electron density, present day and night \\
\end{longtable}
}

\end{solutionbox}
\begin{mnemonicbox}
``DEAF'' - D absorbs, E reflects, All merge, F2
persists

\end{mnemonicbox}
\subsection*{Question 4(c) OR [7
marks]}\label{q4c}

\textbf{Explain Ground wave and Sky wave propagation}

\begin{solutionbox}

\textbf{Ground Wave Propagation:}

\begin{verbatim}
    TX                                RX
     o─────────────────────────────────o
      {                               /}
       {                             /}
        {                           /}
         {-{-}{-}{-}{-}{-}{-}{-}{-}{-}{-}{-}{-}{-}{-}{-}{-}{-}{-}{-}{-}{-}{-}{-}{-}{-}{-}{-}{-}}
              Earth{s surface}
\end{verbatim}

\begin{itemize}
\tightlist
\item
  \textbf{Frequency range}: LF, MF (30 kHz - 3 MHz)
\item
  \textbf{Components}: Direct, ground-reflected, surface waves
\item
  \textbf{Range}: Depends on frequency, ground conductivity, transmitter
  power
\item
  \textbf{Applications}: AM broadcasting, navigation systems, maritime
  communications
\end{itemize}

\textbf{Sky Wave Propagation:}

\begin{verbatim}
                     /|{}
                      |  Ionosphere
    ─────────────────────────────────────
                /     |     {}
               /      |      {}
     TX o─────/       |       {────o RX}
              {                /}
               {              /}
                {            /}
                 {-{-}{-}{-}{-}{-}{-}{-}{-}{-}{-}{-}}
                 Earth{s surface}
\end{verbatim}

\begin{itemize}
\tightlist
\item
  \textbf{Mechanism}: Waves refracted by ionosphere back to Earth
\item
  \textbf{Frequency}: Mainly HF (3-30 MHz)
\item
  \textbf{Range}: 100-10,000+ km, multiple hops possible
\item
  \textbf{Variability}: Time of day, season, solar activity, frequency
\item
  \textbf{Applications}: International broadcasting, amateur radio,
  military
\end{itemize}

\end{solutionbox}
\begin{mnemonicbox}
``GIST'' - Ground-Interface Surface Transmission vs
Ionospheric Sky Transmission

\end{mnemonicbox}
\subsection*{Question 5(a) [3 marks]}\label{q5a}

\textbf{Explain three different types of Satellites}

\begin{solutionbox}

{\def\LTcaptype{none} % do not increment counter
\begin{longtable}[]{@{}
  >{\raggedright\arraybackslash}p{(\linewidth - 2\tabcolsep) * \real{0.4848}}
  >{\raggedright\arraybackslash}p{(\linewidth - 2\tabcolsep) * \real{0.5152}}@{}}
\toprule\noalign{}
\begin{minipage}[b]{\linewidth}\raggedright
Satellite Type
\end{minipage} & \begin{minipage}[b]{\linewidth}\raggedright
Characteristics
\end{minipage} \\
\midrule\noalign{}
\endhead
\bottomrule\noalign{}
\endlastfoot
\textbf{LEO (Low Earth Orbit)} & Altitude: 160-2,000 km, Period: 90 min,
Applications: Earth observation, communications \\
\textbf{MEO (Medium Earth Orbit)} & Altitude: 2,000-35,786 km, Period:
2-24 hours, Applications: Navigation (GPS) \\
\textbf{GEO (Geostationary Orbit)} & Altitude: 35,786 km, Period: 24
hours, Applications: TV broadcasting, weather monitoring \\
\end{longtable}
}

\end{solutionbox}
\begin{mnemonicbox}
``LMG'' - Low Medium Geostationary

\end{mnemonicbox}
\subsection*{Question 5(b) [4 marks]}\label{q5b}

\textbf{What are smart antennas? Write two applications of it}

\begin{solutionbox}

Smart antennas are antenna systems that use digital signal processing
algorithms to identify spatial signatures and dynamically adjust
radiation patterns.

{\def\LTcaptype{none} % do not increment counter
\begin{longtable}[]{@{}
  >{\raggedright\arraybackslash}p{(\linewidth - 2\tabcolsep) * \real{0.4091}}
  >{\raggedright\arraybackslash}p{(\linewidth - 2\tabcolsep) * \real{0.5909}}@{}}
\toprule\noalign{}
\begin{minipage}[b]{\linewidth}\raggedright
Feature
\end{minipage} & \begin{minipage}[b]{\linewidth}\raggedright
Description
\end{minipage} \\
\midrule\noalign{}
\endhead
\bottomrule\noalign{}
\endlastfoot
\textbf{Types} & Switched beam systems, Adaptive array systems \\
\textbf{Operation} & Uses multiple antenna elements and signal
processing to adapt to changing conditions \\
\textbf{Benefits} & Increased capacity, improved coverage, reduced
interference \\
\end{longtable}
}

\textbf{Applications:}

\begin{enumerate}
\tightlist
\item
  Mobile cellular networks (4G, 5G) for increased capacity and coverage
\item
  Wireless LANs for improved throughput and reduced interference
\end{enumerate}

\end{solutionbox}
\begin{mnemonicbox}
``SMART'' - Signal Manipulation And Response
Technology

\end{mnemonicbox}
\subsection*{Question 5(c) [7 marks]}\label{q5c}

\textbf{What is Satellite communication? Explain Data Communication}

\begin{solutionbox}

\textbf{Satellite Communication} is the use of artificial satellites to
provide communication links between various points on Earth.

\begin{verbatim}
               ┌───────┐
               │       │
               │  SAT  │
               │       │
               └───────┘
                /     {}
               /       {}
       Uplink /         { Downlink}
             /           {}
            /             {}
     ┌─────┐               ┌─────┐
     │     │               │     │
     │ TX  │               │ RX  │
     │     │               │     │
     └─────┘               └─────┘
\end{verbatim}

\textbf{Data Communication via Satellite:}

{\def\LTcaptype{none} % do not increment counter
\begin{longtable}[]{@{}
  >{\raggedright\arraybackslash}p{(\linewidth - 2\tabcolsep) * \real{0.5238}}
  >{\raggedright\arraybackslash}p{(\linewidth - 2\tabcolsep) * \real{0.4762}}@{}}
\toprule\noalign{}
\begin{minipage}[b]{\linewidth}\raggedright
Component
\end{minipage} & \begin{minipage}[b]{\linewidth}\raggedright
Function
\end{minipage} \\
\midrule\noalign{}
\endhead
\bottomrule\noalign{}
\endlastfoot
\textbf{Earth Station} & Transmits/receives signals to/from
satellites \\
\textbf{Transponder} & Receives, amplifies and retransmits signals at
different frequencies \\
\textbf{Access methods} & FDMA, TDMA, CDMA to allow multiple users to
share satellite capacity \\
\textbf{Protocols} & TCP/IP adaptation for satellite latency,
specialized protocols \\
\textbf{Applications} & Internet backhaul, VSAT networks, IoT, corporate
networks \\
\textbf{Advantages} & Wide coverage area, independence from terrestrial
infrastructure \\
\textbf{Challenges} & Signal delay (latency), power limitations, weather
effects \\
\end{longtable}
}

\end{solutionbox}
\begin{mnemonicbox}
``UPDATA'' - Uplink Provides Data Access To All

\end{mnemonicbox}
\subsection*{Question 5(a) OR [3
marks]}\label{q5a}

\textbf{Write laws of Kepler for satellite}

\begin{solutionbox}

{\def\LTcaptype{none} % do not increment counter
\begin{longtable}[]{@{}
  >{\raggedright\arraybackslash}p{(\linewidth - 2\tabcolsep) * \real{0.5357}}
  >{\raggedright\arraybackslash}p{(\linewidth - 2\tabcolsep) * \real{0.4643}}@{}}
\toprule\noalign{}
\begin{minipage}[b]{\linewidth}\raggedright
Kepler's Laws
\end{minipage} & \begin{minipage}[b]{\linewidth}\raggedright
Description
\end{minipage} \\
\midrule\noalign{}
\endhead
\bottomrule\noalign{}
\endlastfoot
\textbf{First Law} & Satellites orbit in elliptical paths with the Earth
at one focus of the ellipse \\
\textbf{Second Law} & A line joining the satellite and Earth sweeps out
equal areas in equal times (conservation of angular momentum) \\
\textbf{Third Law} & The square of the orbital period is proportional to
the cube of the semi-major axis of the orbit \\
\end{longtable}
}

\end{solutionbox}
\begin{mnemonicbox}
``ESP'' - Elliptical orbits, Sweep equal areas,
Period-distance relation

\end{mnemonicbox}
\subsection*{Question 5(b) OR [4
marks]}\label{q5b}

\textbf{Explain Base station and Mobile station antennas}

\begin{solutionbox}

\textbf{Base Station Antennas:}

\begin{verbatim}
       ┌─┐
       │ │
       │ │
       │ │
       │ │
       │ │ Vertical collinear
       │ │
       │ │
       │ │
       └─┘
\end{verbatim}

\begin{itemize}
\tightlist
\item
  \textbf{Types}: Omnidirectional, sector, panel antennas
\item
  \textbf{Gain}: Typically 10-18 dBi
\item
  \textbf{Mounting}: Tower or rooftop installation
\item
  \textbf{Features}: Downtilt capability, multiple frequency bands
\end{itemize}

\textbf{Mobile Station Antennas:}

\begin{verbatim}
       ┌───────┐
       │       │
       │  ─┬─  │ Internal antenna
       │       │
       └───────┘  Smartphone
\end{verbatim}

\begin{itemize}
\tightlist
\item
  \textbf{Types}: Internal PIFA, patch, monopole antennas
\item
  \textbf{Gain}: Low gain (0-3 dBi)
\item
  \textbf{Size}: Compact, often integrated inside device
\item
  \textbf{Characteristics}: Omnidirectional pattern, multiple bands
\end{itemize}

\end{solutionbox}
\begin{mnemonicbox}
``BIMS'' - Base stations Install Multiple Sectors,
Mobile stations Stay small

\end{mnemonicbox}
\subsection*{Question 5(c) OR [7
marks]}\label{q5c}

\textbf{Explain DTH receiver system in detail}

\begin{solutionbox}

DTH (Direct-to-Home) receiver system delivers television signals
directly to users via satellite.

\begin{verbatim}
                    ┌───────┐
                    │ { │ Satellite}
                    └───────┘
                        │
                        │
                        V
                    ┌───────┐
                    │ ///// │ Dish antenna
                    └───┬───┘
                        │
                        │
         ┌──────────────┴──────────┐
         │                         │
    ┌────┴─────┐             ┌─────┴─────┐
    │  LNB     │             │ Set{-top   │}
    │(Outdoor) │─────Cable───│   Box     │──────► TV
    └──────────┘             │ (Indoor)  │
                             └───────────┘
\end{verbatim}

{\def\LTcaptype{none} % do not increment counter
\begin{longtable}[]{@{}
  >{\raggedright\arraybackslash}p{(\linewidth - 2\tabcolsep) * \real{0.5238}}
  >{\raggedright\arraybackslash}p{(\linewidth - 2\tabcolsep) * \real{0.4762}}@{}}
\toprule\noalign{}
\begin{minipage}[b]{\linewidth}\raggedright
Component
\end{minipage} & \begin{minipage}[b]{\linewidth}\raggedright
Function
\end{minipage} \\
\midrule\noalign{}
\endhead
\bottomrule\noalign{}
\endlastfoot
\textbf{Dish Antenna} & Parabolic reflector to collect satellite signals
(45-90 cm typical diameter) \\
\textbf{LNB (Low Noise Block)} & Converts high-frequency satellite
signals to lower frequencies for transmission through coaxial cable \\
\textbf{Coaxial Cable} & Carries signals from LNB to set-top box \\
\textbf{Set-top Box} & Decodes/demodulates signals, provides user
interface, conditional access \\
\textbf{Conditional Access Module} & Provides security and subscription
management \\
\textbf{Features} & Electronic Program Guide, recording, interactive
services \\
\end{longtable}
}

\end{solutionbox}
\begin{mnemonicbox}
``DISCS'' - Dish Intercepts Signals, Converter Sends
to Set-top box

\end{mnemonicbox}

\end{document}
