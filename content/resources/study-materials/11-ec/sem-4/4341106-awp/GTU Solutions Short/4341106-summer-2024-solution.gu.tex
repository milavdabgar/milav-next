\documentclass{article}

% content/resources/templates/preamble.tex
\usepackage[margin=0.6in]{geometry}
\author{Milav Dabgar}
\usepackage{amsmath,amssymb,amsthm}
\usepackage{booktabs}
\usepackage{multirow}
\usepackage{xcolor}
\usepackage{tcolorbox}
\tcbuselibrary{breakable,skins}
\usepackage[colorlinks=true,linkcolor=blue]{hyperref}
\usepackage{titlesec}
\usepackage{enumitem}
\usepackage{tikz}
\usepackage{pgfplots}
\usepackage{circuitikz}
\usepackage[version=4]{mhchem}
\usepackage{longtable}
\usepackage{array}
\usepackage{float}
\usepackage{caption}
\usepackage{listings}

\lstset{
  basicstyle=\small\ttfamily,
  breaklines=true,
  breakatwhitespace=false,
  postbreak=\mbox{\textcolor{red}{$\hookrightarrow$}\space},
  float=false,
  numbers=left,
  numberstyle=\tiny\color{gray},
  numbersep=10pt,
  xleftmargin=2em,
  keywordstyle=\color{blue},
  commentstyle=\color{green!60!black},
  stringstyle=\color{purple},
  backgroundcolor=\color{gray!5},
  showstringspaces=false,
  tabsize=2,
  captionpos=b,
  keepspaces=true,
  columns=flexible
}

\pgfplotsset{compat=1.18}
\usetikzlibrary{shapes,arrows,positioning,calc,patterns,decorations.pathmorphing,decorations.markings,arrows.meta}

% Color scheme
\definecolor{headcolor}{RGB}{0,102,204}
\definecolor{keycolor}{RGB}{220,20,60}
\definecolor{solutioncolor}{RGB}{34,139,34}
\definecolor{mnemoniccolor}{RGB}{148,0,211}
\definecolor{codecolor}{RGB}{0,0,100}

% Spacing
\setlength{\parskip}{3pt}
\setlist[itemize]{nosep}
\setlist[enumerate]{nosep}

% Title formatting
\titleformat{\section}{\Large\bfseries\color{headcolor}}{\thesection}{1em}{}
\titleformat{\subsection}{\large\bfseries\color{headcolor}}{\thesubsection}{1em}{}

% Pandoc tightlist compatibility
\providecommand{\tightlist}{%
  \setlength{\itemsep}{0pt}\setlength{\parskip}{0pt}}

% Pandoc longtable compatibility
\newcounter{none}
\def\thenone{}


% content/resources/templates/gujarati-boxes.tex
\usepackage{fontspec}
\usepackage{polyglossia}

% Set Gujarati as main language (document is primarily in Gujarati)
% Note: gloss-gujarati.ldf doesn't exist in polyglossia, but it will use hyphenation patterns
\setdefaultlanguage{gujarati}
\setotherlanguage{english}

% Configure Gujarati font properly
% Use Language=Default to prevent polyglossia from trying to add language-specific features
% that don't exist for Gujarati, which causes "empty feature" warnings
\newfontfamily\gujaratifont[Script=Gujarati,AutoFakeBold=2.5,AutoFakeSlant=0.3]{Noto Sans Gujarati}
\setmainfont[Script=Gujarati,AutoFakeBold=2.5,AutoFakeSlant=0.3]{Noto Sans Gujarati}
% Use Noto Sans Gujarati for monospace to support Gujarati in text
\setmonofont[Scale=0.9]{Noto Sans Gujarati}

% Configure English to use the same font
\newfontfamily\englishfont[Script=Gujarati,AutoFakeBold=2.5,AutoFakeSlant=0.3]{Noto Sans Gujarati}

% Translations for polyglossia
\gappto\captionsgujarati{
  \renewcommand{\tablename}{કોષ્ટક}
  \renewcommand{\figurename}{આકૃતિ}
}

% Helper for TikZ nodes to ensure Gujarati font
\newcommand{\gu}[1]{{\gujaratifont #1}}

% Custom environments
\newtcolorbox{solutionbox}{
    breakable,
    enhanced,
    colback=solutioncolor!5!white,
    colframe=solutioncolor!75!black,
    fonttitle=\bfseries,
    title=જવાબ
}

\newtcolorbox{solutionboxnobreak}{
 colback=solutioncolor!5!white,
 colframe=solutioncolor!75!black,
 fonttitle=\bfseries,
 title=જવાબ
}

\newtcolorbox{keyformula}{
 breakable,
 enhanced,
 colback=keycolor!5!white,
 colframe=keycolor!75!black,
 fonttitle=\bfseries,
 title=રાસાયણિક સમીકરણ/સૂત્ર
}

\newtcolorbox{mnemonicbox}{
 breakable,
 enhanced,
 colback=mnemoniccolor!5!white,
 colframe=mnemoniccolor!75!black,
 fonttitle=\bfseries,
 title=મેમરી ટ્રીક
}


% Custom commands for GTU solutions
% This file defines semantic commands for consistent formatting

% Question command with automatic formatting
\newcommand{\question}[2]{%
  \section*{Question #1}%
  \textbf{#2}%
}

% OR question variant
\newcommand{\questionor}[2]{%
  \section*{Question #1 OR}%
  \textbf{#2}%
}

% Proper table environment with caption
\newenvironment{answertable}[1]{%
  \begin{table}[htbp]
  \centering
  \caption{#1}
}{%
  \end{table}
}

% Proper figure environment for diagrams
\newenvironment{answerdiagram}[1]{%
  \begin{figure}[htbp]
  \centering
  \caption{#1}
}{%
  \end{figure}
}

% Semantic markup for key terms
\newcommand{\keyword}[1]{\textbf{#1}}
\newcommand{\code}[1]{\texttt{#1}}
\newcommand{\classname}[1]{\texttt{#1}}
\newcommand{\methodname}[1]{\texttt{#1}}

% Proper quotation marks
\newcommand{\mnemonic}[1]{``#1''}


\title{એન્ટેના અને વેવ પ્રોપેગેશન (4341106) - સમર 2024 સોલ્યુશન}
\date{19 જૂન, 2024}

\begin{document}
\maketitle

\questionmarks{1(અ)}{3}{બીમ વિસ્તાર અને બીમની કાર્યક્ષમતા વ્યાખ્યાયિત કરો.}

\begin{solutionbox}
\textbf{બીમ વિસ્તાર (Beam Area)}: એ એક કાલ્પનિક ઘન કોણ (\textit{solid angle}) છે જેના માધ્યમથી એન્ટેના દ્વારા વિકિરણિત (\textit{radiated}) તમામ પાવર પસાર થશે જો રેડિએશન ઇન્ટેન્સિટી આ કોણ પર સમાન (\textit{constant}) હોય અને મહત્તમ મૂલ્યની બરાબર હોય. તેને $\Omega_A$ વડે દર્શાવાય છે.

\textbf{બીમ કાર્યક્ષમતા (Beam Efficiency)}: મુખ્ય બીમ (\textit{main beam}) માં રહેલી શક્તિનો એન્ટેના દ્વારા વિકિરણિત કુલ શક્તિ સાથેનો ગુણોત્તર.

\begin{answerdiagram}{બીમ કાર્યક્ષમતાનો ખ્યાલ}
\begin{tikzpicture}
    % Main Beam
    \fill [blue!20] (0,0) -- (15:3) arc (15:-15:3) -- cycle;
    \draw [thick, blue] (0,0) -- (15:3);
    \draw [thick, blue] (0,0) -- (-15:3);
    \node [blue] at (2, 0) {મુખ્ય બીમ પાવર};
    
    % Side lobes (minor)
    \foreach \angle in {45, 90, 135, 180, 225, 270, 315} {
        \draw [thick, red, fill=red!10] (0,0) -- (\angle-10:1.5) arc (\angle-10:\angle+10:1.5) -- cycle;
    }
    \node [red, align=center] at (0, 2) {સાઇડ લોબ્સ\\(વેડફાતો પાવર)};
    
    \node [align=left, anchor=west] at (4, 1) {$\text{બીમ કાર્યક્ષમતા} = \frac{\text{મુખ્ય બીમ પાવર}}{\text{કુલ રેડિએટેડ પાવર}}$};
    \node [align=left, anchor=west] at (4, -1) {ઉચ્ચ કાર્યક્ષમતા = બેહતર એન્ટેના};
\end{tikzpicture}
\end{answerdiagram}
\end{solutionbox}

\begin{mnemonicbox}
\mnemonic{"BEAM: બેહતર કાર્યક્ષમતા આદર્શ મહત્તમ કામગીરી"}
\end{mnemonicbox}

\questionmarks{1(બ)}{4}{EM ક્ષેત્ર શું છે? સેન્ટર ફેડ ડાયપોલ માંથી તેના કિરણોત્સર્જનને સમજાવો.}

\begin{solutionbox}
\textbf{EM ક્ષેત્ર (Electromagnetic Field)}: EM ક્ષેત્ર એક ભૌતિક ક્ષેત્ર છે જે વિદ્યુત ચાર્જ વાળી વસ્તુઓ દ્વારા ઉત્પન્ન થાય છે અને તે ક્ષેત્રની નજીકમાં રહેલા અન્ય ચાર્જ કણો પર બળ લગાડે છે. તે ઇલેક્ટ્રિક ($E$) અને મેગ્નેટિક ($H$) ફિલ્ડનું સંયોજન છે.

\textbf{સેન્ટર ફેડ ડાયપોલમાંથી રેડિએશન}:
જ્યારે એન્ટેનામાંથી અલ્ટરનેટિંગ કરંટ ($AC$) પ્રસાર થાય છે, ત્યારે તે સમય સાથે બદલાતા ઇલેક્ટ્રિક અને મેગ્નેટિક ફિલ્ડ બનાવે છે જે એન્ટેનાથી છૂટા પડીને બહારની તરફ ગતિ કરે છે.

\begin{answerdiagram}{ડાયપોલની આસપાસના ફિલ્ડ્સ}
\begin{tikzpicture}[scale=0.8]
    % Dipole
    \draw [ultra thick] (0, 2) -- (0, 0.2);
    \draw [ultra thick] (0, -2) -- (0, -0.2);
    \draw [thick] (-0.5, 0.2) -- (0, 0.2);
    \draw [thick] (-0.5, -0.2) -- (0, -0.2);
    \node [left] at (-0.5, 0) {સ્રોત $\sim$};
    \node [above] at (0, 2) {ડાયપોલ એન્ટેના};

    % E-Fields (Loops)
    \foreach \x in {1, 2, 3} {
        \draw [blue, thick, ->] (0.5*\x, 1) arc (0:180:0.5*\x) node [midway, above] {};
        \draw [blue, thick, ->] (-0.5*\x, -1) arc (180:360:0.5*\x);
    }
    \node [blue] at (2, 2.5) {ઇલેક્ટ્રિક ફિલ્ડ ($E$)};

    % H-Fields (Circles around wire)
    \foreach \y in {-1.5, -0.5, 0.5, 1.5} {
        \draw [red, thick] (0, \y) ellipse (0.3 and 0.1);
    }
    \node [red] at (2, 0.5) {ચુંબકીય ક્ષેત્ર ($H$)};
    
    \draw [->, thick] (4, 0) -- (5, 0) node [right] {પ્રસારણ દિશા};
\end{tikzpicture}
\end{answerdiagram}

\begin{itemize}
    \item \textbf{ઇલેક્ટ્રિક ફિલ્ડ (E)}: એન્ટેના અક્ષને લંબરૂપે હોય છે અને એન્ટેનાના છેડા પર મહત્તમ હોય છે.
    \item \textbf{ચુંબકીય ક્ષેત્ર (H)}: એન્ટેના અક્ષ (તાર) ની આસપાસ વર્તુળાકાર હોય છે.
    \item \textbf{રેડિએશન મિકેનિઝમ}: વિદ્યુત પ્રવાહની દિશા બદલાતી હોવાથી, ફિલ્ડ લાઈન્સ તારથી છૂટી પડીને બંધ લૂપ બનાવે છે અને અવકાશમાં આગળ વધે છે.
\end{itemize}
\end{solutionbox}

\begin{mnemonicbox}
\mnemonic{"CERD: કરંટ એક્સાઇટ્સ રેડિએટિંગ ડાયપોલ"}
\end{mnemonicbox}

\questionmarks{1(ક)}{7}{પોઈન્ટિંગ વેક્ટરનો ઉપયોગ કરીને પ્રાથમિક ડાયપોલ દ્વારા વિકિરણ થતી શક્તિ સમજાવો.}

\begin{solutionbox}
પ્રાથમિક ડાયપોલ (\textit{Hertzian Dipole}) દ્વારા વિકિરણિત શક્તિની ગણતરી પોઈન્ટિંગ વેક્ટર દ્વારા થઈ શકે છે, જે પાવર ફ્લો ઘનતા ($Power Density$) દર્શાવે છે.

\begin{center}
\begin{tikzpicture}[node distance=1.5cm, auto]
    \node [gtu block] (step1) {1. $E$-ફિલ્ડ ઘટકો ગણો ($E_\theta, E_\phi$)};
    \node [gtu block, below=of step1] (step2) {2. $H$-ફિલ્ડ ઘટકો ગણો ($H_\theta, H_\phi$)};
    \node [gtu block, below=of step2] (step3) {3. પોઈન્ટિંગ વેક્ટર $P = E \times H^*$};
    \node [gtu block, below=of step3] (step4) {4. સ્ફિયર પર ઇન્ટિગ્રેશન};
    \node [gtu block, below=of step4] (result) {પરિણામ: $P_{rad} = 80\pi^2 I^2 (l/\lambda)^2$};

    \draw [gtu arrow] (step1) -- (step2);
    \draw [gtu arrow] (step2) -- (step3);
    \draw [gtu arrow] (step3) -- (step4);
    \draw [gtu arrow] (step4) -- (result);
\end{tikzpicture}
\end{center}

\textbf{ગણતરીના મુખ્ય પગલાં:}
\begin{enumerate}
    \item \textbf{ફિલ્ડ સમીકરણો}:
    $$E_\theta = j\frac{\eta I_0 dl}{2\lambda r} \sin\theta e^{-j\beta r}$$
    $$H_\phi = j\frac{I_0 dl}{2\lambda r} \sin\theta e^{-j\beta r}$$
    
    \item \textbf{સરેરાશ પાવર ઘનતા} ($P_{avg}$):
    $$P_{avg} = \frac{1}{2} Re(E \times H^*) = \frac{1}{2} \frac{\eta |I_0|^2 (dl)^2}{4\lambda^2 r^2} \sin^2\theta$$
    
    \item \textbf{કુલ રેડિએટેડ પાવર} ($P_{rad}$):
    બંધ ગોળાકાર સપાટી ($sphere$) પર સંકલન ($integration$) લેતા:
    $$P_{rad} = \int_{0}^{2\pi} \int_{0}^{\pi} P_{avg} r^2 \sin\theta d\theta d\phi$$
    $$P_{rad} = \eta \frac{\pi}{3} \left(\frac{I_0 dl}{\lambda}\right)^2$$
    જ્યાં $\eta \approx 120\pi \approx 377 \Omega$ મુકતા:
    $$P_{rad} = 80\pi^2 I_{rms}^2 \left(\frac{dl}{\lambda}\right)^2 \text{ Watts}$$
\end{enumerate}
\end{solutionbox}

\begin{mnemonicbox}
\mnemonic{"PEHP: પોઈન્ટિંગ એક્સપ્લેન્સ હાઉ પાવર પ્રોપેગેટ્સ"}
\end{mnemonicbox}

\questionmarks{1(ક) અથવા}{7}{એન્ટેના, રેડિયેશન પેટર્ન, ડાયરેક્ટિવિટી, ગેઇન, FBR, આઇસોટ્રોપિક રેડિએટર અને ઇફેક્ટિવ એપર્ચર વ્યાખ્યાયિત કરો.}

\begin{solutionbox}
\begin{tabulary}{\linewidth}{|L|L|}
\hline
\textbf{પેરામીટર} & \textbf{વ્યાખ્યા} \\ \hline
\keyword{એન્ટેના} & એક ઉપકરણ જે ગાઇડેડ ઇલેક્ટ્રોમેગ્નેટિક વેવ્સને ફ્રી-સ્પેસ વેવ્સમાં અને વિપરીત રૂપાંતર કરે છે. તે એક ટ્રાન્સડ્યુસર છે. \\ \hline
\keyword{રેડિએશન પેટર્ન} & સ્પેસ કોઓર્ડિનેટ્સના ફંક્શન તરીકે એન્ટેનાની રેડિએશન પ્રોપર્ટી (જેમ કે ફિલ્ડ સ્ટ્રેન્થ) ની ગ્રાફિકલ રજૂઆત. \\ \hline
\keyword{ડાયરેક્ટિવિટી ($D$)} & કોઈ ચોક્કસ દિશામાં રેડિએશન ઇન્ટેન્સિટીનો અને સરેરાશ રેડિએશન ઇન્ટેન્સિટીનો ગુણોત્તર. \\ \hline
\keyword{ગેઇન ($G$)} & ડાયરેક્ટિવિટી જેવું જ, પરંતુ તેમાં એન્ટેનાની કાર્યક્ષમતા ($\eta$) ધ્યાનમાં લેવાય છે. $G = \eta D$. \\ \hline
\keyword{FBR (ફ્રન્ટ-ટુ-બેક રેશિયો)} & ફોરવર્ડ દિશામાં (મુખ્ય લોબ) વિકિરણિત શક્તિનો અને બરાબર તેની વિરુદ્ધ (પાછળની) દિશામાં વિકિરણિત શક્તિનો ગુણોત્તર. \\ \hline
\keyword{આઇસોટ્રોપિક રેડિએટર} & એક સૈદ્ધાંતિક (\textit{theoretical}) એન્ટેના જે બધી દિશામાં સમાન રીતે ઉર્જા વિકિરણ કરે છે. તે વાસ્તવમાં અસ્તિત્વમાં નથી. \\ \hline
\keyword{ઇફેક્ટિવ એપર્ચર ($A_e$)} & એન્ટેના દ્વારા લોડને મળતી શક્તિનો અને એન્ટેના પર આપાત થતી પાવર ઘનતાનો ગુણોત્તર. તે એન્ટેનાની પાવર કેપ્ચર કરવાની ક્ષમતા દર્શાવે છે. \\ \hline
\end{tabulary}

\begin{answerdiagram}{એન્ટેના પેરામીટર્સ}
\begin{tikzpicture}
    \pie[text=legend, radius=2, font=\footnotesize]{
     25/ડાયરેક્ટિવિટી,
     25/ગેઇન,
     20/ઇફેક્ટિવ એપર્ચર,
     15/રેડિએશન પેટર્ન,
     15/FBR
    }
\end{tikzpicture}
\end{answerdiagram}
\end{solutionbox}

\begin{mnemonicbox}
\mnemonic{"DIAGRAM: ડાયરેક્ટિવિટી ઇમ્પ્રુવ્સ એન્ટેના ગેઇન, રેડિએશન એન્ડ મોર"}
\end{mnemonicbox}

\questionmarks{2(અ)}{3}{પેટર્ન ગુણાકારનો સિદ્ધાંત સમજાવો.}

\begin{solutionbox}
\textbf{પેટર્ન ગુણાકાર સિદ્ધાંત (Pattern Multiplication)}:
કોઈપણ એન્ટેના એરેની કુલ રેડિએશન પેટર્ન એ બે પેટર્નનો ગુણાકાર છે:
\begin{enumerate}
    \item સિંગલ એન્ટેના એલિમેન્ટની પેટર્ન (\textit{Element Pattern}).
    \item એરેના આઇસોટ્રોપિક સ્ત્રોતોની પેટર્ન (\textit{Array Factor}).
\end{enumerate}

$$ \text{કુલ પેટર્ન} = \text{એલિમેન્ટ પેટર્ન} \times \text{એરે ફેક્ટર} $$

\begin{answerdiagram}{પેટર્ન ગુણાકાર}
\begin{tikzpicture}
    % Element Pattern
    \draw [thick, blue] (0,0) ellipse (0.5 and 1.5);
    \node [below] at (0, -2) {એલિમેન્ટ પેટર્ન};
    \node at (1.5, 0) {$\times$};
    
    % Array Factor
    \draw [thick, red] (3,0) -- (4, 1) -- (5,0) -- (4, -1) -- cycle; 
    \node [below] at (4, -2) {એરે ફેક્ટર};
    \node at (5.5, 0) {$=$};
    
    % Result
    \draw [thick, purple] (7,0) ellipse (0.5 and 1.5); 
    \draw [thick, purple] (8,0.5) circle (0.2); 
    \node [below] at (7.5, -2) {કુલ પેટર્ન};
\end{tikzpicture}
\end{answerdiagram}

આ સિદ્ધાંત જટિલ એન્ટેના એરેની ડિઝાઇનમાં મદદરૂપ થાય છે.
\end{solutionbox}

\begin{mnemonicbox}
\mnemonic{"PEAM: પેટર્ન ઈક્વલ્સ એરે ટાઇમ્સ એલિમેન્ટ મેથડ"}
\end{mnemonicbox}

\questionmarks{2(બ)}{4}{લૂપ એન્ટેના દોરો અને સમજાવો.}

\begin{solutionbox}
લૂપ એન્ટેના એક બંધ સર્કિટ એન્ટેના છે જેમાં તારના એક અથવા વધુ પૂર્ણ આંટા હોય છે.

\begin{answerdiagram}{લૂપ એન્ટેનાનું માળખું}
\begin{tikzpicture}
    \draw [ultra thick] (0,0) circle (1.5); % Loop
    \draw [fill=white] (-0.2, -1.5) rectangle (0.2, -1.3); % Feed gap
    \draw [thick] (-0.1, -1.5) -- (-0.1, -2);
    \draw [thick] (0.1, -1.5) -- (0.1, -2);
    \node [below] at (0, -2) {ફીડ લાઈન્સ};
    \node at (0, 0) {લૂપ કંડક્ટર};
    
    % Dimensions
    \draw [dashed] (0,0) -- (1.5, 0);
    \node [above] at (0.75, 0) {ત્રિજ્યા $a$};
\end{tikzpicture}
\end{answerdiagram}

\begin{itemize}
    \item \textbf{નાનો લૂપ} (પરિઘ $< \lambda/10$): મેગ્નેટિક ડાયપોલ જેવું વર્તન કરે છે. તેની પેટર્ન ડાયપોલ જેવી (ફિગર-8) હોય છે.
    \item \textbf{મોટો લૂપ} (પરિઘ $\approx \lambda$): રેઝોનન્ટ લૂપ. લૂપના પ્લેનને લંબરૂપે મહત્તમ રેડિએશન આપે છે.
    \item \textbf{ઉપયોગો}: દિશા શોધવા (Direction Finding), AM રેડિયો રિસેપ્શનમાં વપરાય છે.
\end{itemize}
\end{solutionbox}

\begin{mnemonicbox}
\mnemonic{"LOOP: લો આઉટપુટ, ઓરિએન્ટેશન પ્રિસાઇઝ"}
\end{mnemonicbox}

\questionmarks{2(ક)}{7}{યાગી-ઉડા એન્ટેના ડિઝાઇન કરો અને તેને સમજાવો.}

\begin{solutionbox}
યાગી-ઉડા એ એક હાઈ-ગેઇન દિશાત્મક (\textit{Directional}) એન્ટેના છે. તેમાં એક મુખ્ય (Driven) એલિમેન્ટ અને અન્ય પેરાસિટિક એલિમેન્ટ્સ (Reflector અને Directors) હોય છે.

\textbf{ડિઝાઇન ડેટા:}
\begin{tabulary}{\linewidth}{|L|L|L|}
\hline
\textbf{એલિમેન્ટ} & \textbf{લંબાઈ} & \textbf{અંતર (Spacing)} \\ \hline
રિફ્લેક્ટર (Reflector) & $0.55\lambda$ (સૌથી લાંબો) & ડ્રાઇવનથી $0.25\lambda$ પાછળ \\ \hline
ડ્રાઇવન એલિમેન્ટ (Driven) & $0.5\lambda$ (રેઝોનન્ટ) & $\dots$ \\ \hline
ડાયરેક્ટર 1 & $0.45\lambda$ (ટૂંકા થતા જાય) & ડ્રાઇવનથી $0.1\lambda$ આગળ \\ \hline
\end{tabulary}

\begin{answerdiagram}{યાગી-ઉડા એન્ટેના}
\begin{tikzpicture}[scale=0.8]
    % Boom
    \draw [thick] (-2, 0) -- (6, 0);
    \node [right] at (6, 0) {બૂમ (Boom)};
    
    % Reflector
    \draw [ultra thick, blue] (-1, -2) -- (-1, 2);
    \node [above] at (-1, 2) {રિફ્લેક્ટર};
    \node [below] at (-1, -2) {$>0.5\lambda$};
    
    % Driven Element
    \draw [ultra thick, red] (1, -1.8) -- (1, 1.8);
    \node [above] at (1, 1.8) {ડ્રાઇવન};
    \node [below] at (1, -1.8) {$\approx 0.5\lambda$};
    \fill [white] (0.9, -0.1) rectangle (1.1, 0.1); % Feed point
    \draw (1, -0.1) -- (1, -0.5) node [below] {ફીડ};
    
    % Directors
    \draw [ultra thick, green!70!black] (3, -1.6) -- (3, 1.6);
    \node [above] at (3, 1.6) {ડાયરેક્ટર 1};
    
    \draw [ultra thick, green!70!black] (5, -1.5) -- (5, 1.5);
    \node [above] at (5, 1.5) {ડાયરેક્ટર 2};
    \node [below] at (4, -2) {ડાયરેક્ટર્સ ($<0.5\lambda$)};
    
    % Radiation
    \draw [->, ultra thick, purple] (7, 0) -- (9, 0) node [right] {રેડિએશન દિશા};
\end{tikzpicture}
\end{answerdiagram}

\begin{itemize}
    \item \textbf{રિફ્લેક્ટર}: ડ્રાઇવન કરતા મોટો હોય છે, સિગ્નલને આગળ ધકેલે છે.
    \item \textbf{ડાયરેક્ટર્સ}: ડ્રાઇવન કરતા નાના હોય છે, સિગ્નલને દિશા આપે છે.
    \item \textbf{ગેઇન}: ખૂબ જ ઊંચો હોય છે.
    \item \textbf{ઉપયોગો}: TV રિસેપ્શન, પોઇન્ટ-ટુ-પોઇન્ટ કોમ્યુનિકેશન.
\end{itemize}
\end{solutionbox}

\begin{mnemonicbox}
\mnemonic{"YARD: યાગી એચિવ્સ રેડિકલ ડાયરેક્ટિવિટી"}
\end{mnemonicbox}

\orquestionmarks{2(અ)}{3}{બ્રોડ ફાયર અને એન્ડ ફાયર એરે એન્ટેનાની સરખામણી કરો.}

\begin{solutionbox}
\begin{tabulary}{\linewidth}{|L|L|L|}
\hline
\textbf{પેરામીટર} & \textbf{બ્રોડ સાઇડ એરે} & \textbf{એન્ડ ફાયર એરે} \\ \hline
\keyword{રેડિએશન દિશા} & એરેની અક્ષને લંબરૂપે ($90^\circ$). & એરેની અક્ષની સમાંતર ($0^\circ$ કે $180^\circ$). \\ \hline
\keyword{ફેઝ તફાવત} & બધા એલિમેન્ટ સમાન ફેઝમાં ($0^\circ$). & પ્રોગ્રેસિવ ફેઝ શિફ્ટ ($180^\circ \pm \beta d$). \\ \hline
\keyword{બીમ પહોળાઈ} & સાંકડી (Narrow). & પહોળી (Wider). \\ \hline
\keyword{ડાયરેક્ટિવિટી} & પ્રમાણમાં વધારે. & થોડી ઓછી. \\ \hline
\end{tabulary}

\begin{answerdiagram}{એરે સરખામણી}
\begin{tikzpicture}
    % Broadside
    \node [align=center] at (2, 2.5) {Broadside};
    \foreach \x in {0, 1, 2, 3} \fill (\x, 0) circle (0.1);
    \draw (-0.5, 0) -- (3.5, 0);
    \foreach \x in {0, 1, 2, 3} \draw [->, thick, blue] (\x, 0.2) -- (\x, 1.5);
    \node [right, blue] at (3, 1) {રેડિએશન $\perp$ અક્ષ};

    % End Fire
    \node [align=center] at (8, 2.5) {End Fire};
    \foreach \x in {6, 7, 8, 9} \fill (\x, 0) circle (0.1);
    \draw (5.5, 0) -- (9.5, 0);
    \draw [->, thick, red] (9.5, 0) -- (11, 0) node [right] {રેડિએશન $\parallel$ અક્ષ};
\end{tikzpicture}
\end{answerdiagram}
\end{solutionbox}

\begin{mnemonicbox}
\mnemonic{"BEPS: બ્રોડસાઇડ એમિટ્સ પર્પેન્ડિક્યુલરલી, સાઇડવેઝ"}
\end{mnemonicbox}

\orquestionmarks{2(બ)}{4}{ફોલ્ડેડ ડિપોલ એન્ટેના દોરો અને સમજાવો.}

\begin{solutionbox}
ફોલ્ડેડ ડિપોલમાં અર્ધ-તરંગ લંબાઈ ($\lambda/2$) નો ડિપોલ હોય છે જેના બંને છેડા પાછા વાળીને જોડાયેલા હોય છે, જે એક સાંકડો લૂપ બનાવે છે.

\begin{answerdiagram}{ફોલ્ડેડ ડિપોલ}
\begin{tikzpicture}
    \draw [ultra thick, rounded corners] (-3, 0.5) rectangle (3, -0.5);
    \fill [white] (-0.2, -0.6) rectangle (0.2, -0.4); % Break
    \draw [thick] (-0.2, -0.5) -- (-0.2, -1.2);
    \draw [thick] (0.2, -0.5) -- (0.2, -1.2);
    \node [below] at (0, -1.2) {ફીડ પોઈન્ટ ($300\Omega$)};
    
    \draw [<->] (-3, 0.8) -- (3, 0.8) node [midway, above] {લંબાઈ $\approx \lambda/2$};
\end{tikzpicture}
\end{answerdiagram}

\begin{itemize}
    \item \textbf{ઇમ્પિડન્સ}: સામાન્ય ડિપોલ કરતા 4 ગણો વધારે હોય છે ($73 \times 4 \approx 300\Omega$).
    \item \textbf{બેન્ડવિડ્થ}: સાદા ડિપોલ કરતા વધુ સારી બેન્ડવિડ્થ મળે છે.
    \item \textbf{ઉપયોગો}: TV એન્ટેનામાં મુખ્ય એલિમેન્ટ તરીકે વપરાય છે.
\end{itemize}
\end{solutionbox}

\begin{mnemonicbox}
\mnemonic{"FIBER: ફોલ્ડેડ ઇમ્પિડન્સ બૂસ્ટર એન્હેંસિસ રિસેપ્શન"}
\end{mnemonicbox}

\orquestionmarks{2(ક)}{7}{બિન-રેઝોનન્ટ (Non-resonant) એન્ટેનાના નામ આપો અને કોઈપણ એકને તેની રેડિએશન પેટર્ન સાથે વિગતવાર સમજાવો.}

\begin{solutionbox}
\textbf{બિન-રેઝોનન્ટ એન્ટેનાના નામ}:
1. રોમ્બિક એન્ટેના (Rhombic Antenna)
2. V-એન્ટેના
3. બેવરેજ એન્ટેના (Wave Antenna)

\textbf{રોમ્બિક એન્ટેના}:
રોમ્બિક એન્ટેનામાં ચાર લાંબા તાર સમબાજુ ચતુષ્કોણ (Rhombus) આકારમાં ગોઠવેલા હોય છે. તેના છેડે ટર્મિનેટિંગ રેઝિસ્ટર જોડવાથી તે ટ્રાવેલિંગ વેવ એન્ટેના બને છે.

\begin{answerdiagram}{રોમ્બિક એન્ટેના}
\begin{tikzpicture}[scale=0.8]
    \coordinate (Feed) at (0,0);
    \coordinate (Top) at (4, 2);
    \coordinate (Bottom) at (4, -2);
    \coordinate (Load) at (8, 0);

    \draw [thick] (Feed) -- (Top) -- (Load);
    \draw [thick] (Feed) -- (Bottom) -- (Load);
    
    \draw [->] (-1, 0) -- (0, 0) node [left] {ફીડ};
    \draw [thick, decorate, decoration={zigzag, amplitude=3pt, segment length=6pt}] (Load) -- (9, 0);
    \node [right] at (9, 0) {રેઝિસ્ટર $R$};
    
    \draw [->, ultra thick, dashed, blue] (0, 0) -- (10, 0) node [right] {મુખ્ય રેડિએશન};
    
    % Side lobes
    \draw [blue, opacity=0.5] (4,0) ellipse (3 and 1); 
\end{tikzpicture}
\end{answerdiagram}

\textbf{ખાસિયતો}:
\begin{itemize}
    \item \textbf{સ્ટ્રક્ચર}: રોમ્બસ આકાર. છેડે રેઝિસ્ટર લોડ હોય છે.
    \item \textbf{ડાયરેક્ટિવિટી}: ખૂબ ઊંચી (8-15 dB).
    \item \textbf{બેન્ડવિડ્થ}: ખૂબ જ વિશાળ (Wideband).
    \item \textbf{ઉપયોગો}: HF બેન્ડમાં લાંબા અંતરના (Point-to-Point) કોમ્યુનિકેશન માટે.
\end{itemize}
\end{solutionbox}

\begin{mnemonicbox}
\mnemonic{"RHOMBIC: વિશ્વસનીય ઉચ્ચ-આઉટપુટ મલ્ટી-બેન્ડ અદ્ભુત કોમ્યુનિકેશન"}
\end{mnemonicbox}

\questionmarks{3(અ)}{3}{વિવિધ રેઝોનન્ટ વાયર એન્ટેનાની રેડિએશન પેટર્નની તુલના કરો.}

\begin{solutionbox}
\begin{tabulary}{\linewidth}{|L|L|L|L|}
\hline
\textbf{એન્ટેના પ્રકાર} & \textbf{પેટર્ન આકાર} & \textbf{ડાયરેક્ટિવિટી} & \textbf{પોલરાઈઝેશન} \\ \hline
હાફ-વેવ ડિપોલ ($\lambda/2$) & ફિગર-8 (ડોનટ આકાર), 2 લોબ & 2.15 dBi & લિનિયર \\ \hline
ફુલ-વેવ ડિપોલ ($\lambda$) & ચાર-લોબ (ક્લોવરલીફ) & 3.8 dBi & લિનિયર \\ \hline
$3\lambda/2$ ડિપોલ & છ-લોબ & 4.2 dBi & લિનિયર \\ \hline
$2\lambda$ ડિપોલ & આઠ-લોબ & 4.5 dBi & લિનિયર \\ \hline
\end{tabulary}

\begin{answerdiagram}{રેઝોનન્ટ એન્ટેના પેટર્ન}
\begin{tikzpicture}
    % Half Wave
    \node at (0, 2) {$\lambda/2$};
    \draw [thick, blue] (0, 0) ellipse (0.3 and 0.8);
    \draw [ultra thick] (0, 0.9) -- (0, -0.9); % Antenna
    
    % Full Wave
    \node at (3, 2) {$\lambda$};
    \draw [thick, blue] (3, 0) .. controls (3.5, 0.5) .. (3.8, 0.8) .. controls (3.5, 0) .. (3,0);
    \draw [thick, blue] (3, 0) .. controls (2.5, 0.5) .. (2.2, 0.8) .. controls (2.5, 0) .. (3,0);
    \draw [thick, blue] (3, 0) .. controls (3.5, -0.5) .. (3.8, -0.8) .. controls (3.5, 0) .. (3,0);
    \draw [thick, blue] (3, 0) .. controls (2.5, -0.5) .. (2.2, -0.8) .. controls (2.5, 0) .. (3,0);
    \draw [ultra thick] (3, 1) -- (3, -1);

    % 3/2 Lambda (Simplified)
    \node at (6, 2) {$3\lambda/2$};
    \draw [thick, blue] (6,0) circle (0.8); \node at (6,0) {6 લોબ્સ};
    \draw [ultra thick] (6, 1.2) -- (6, -1.2);
\end{tikzpicture}
\end{answerdiagram}
\end{solutionbox}

\begin{mnemonicbox}
\mnemonic{"MOLD: વધુ તરંગલંબાઈથી ઘણા ડાયરેક્ટિવિટી લોબ્સ બને છે"}
\end{mnemonicbox}

\questionmarks{3(બ)}{4}{V અને ઇન્વર્ટેડ V એન્ટેના રેડીએશન પેટર્ન સાથે દોરો.}

\begin{solutionbox}
\begin{answerdiagram}{V અને ઇન્વર્ટેડ-V એન્ટેના}
\begin{tikzpicture}
    % V Antenna
    \node [font=\bfseries] at (2, 3) {V-એન્ટેના};
    \draw [ultra thick] (0, 0) -- (4, 1.5);
    \draw [ultra thick] (0, 0) -- (4, -1.5);
    \draw [thick, ->] (-1, 0) -- (0, 0) node [midway, above] {ફીડ};
    \draw [dashed, red] (4.5, 0) ellipse (1.5 and 0.8);
    \node [red, right] at (6, 0) {દ્વિ-દિશાત્મક};
    \node at (2, 0) {$\theta$};

    % Inverted V
    \node [font=\bfseries] at (8, 3) {ઇન્વર્ટેડ V-એન્ટેના};
    \draw [ultra thick] (8, 2) -- (6, -0.5);
    \draw [ultra thick] (8, 2) -- (10, -0.5);
    \draw [thick] (8, 2) -- (8, 2.5) node [above] {ફીડ};
    \draw [thick] (5, -1) -- (11, -1); \node [right] at (11, -1) {ગ્રાઉન્ડ};
    \foreach \x in {6, 8, 10} \draw (\x, -1) -- (\x-0.2, -1.2);
    
    \draw [dashed, red] (8, 0.5) ellipse (2 and 0.5);
    \node [red, below] at (8, 0) {ઓમ્નીડાયરેક્શનલ};
\end{tikzpicture}
\end{answerdiagram}

\begin{itemize}
    \item \textbf{V-એન્ટેના}: બે તાર V આકારમાં ગોઠવાયેલા હોય છે. મુખ્ય રેડિએશન V ની અક્ષની દિશામાં હોય છે.
    \item \textbf{ઇન્વર્ટેડ V}: ઊંધા V આકારનો ડાયપોલ. તેની પેટર્ન લગભગ ઓમ્નીડાયરેક્શનલ હોય છે.
\end{itemize}
\end{solutionbox}

\begin{mnemonicbox}
\mnemonic{"VIPS: V-આકાર પેટર્ન પસંદગીમાં સુધારો કરે છે"}
\end{mnemonicbox}

\questionmarks{3(ક)}{7}{મોર્સ કોડ અને પ્રેક્ટિસ ઓસિલેટર સમજાવો.}

\begin{solutionbox}
\textbf{મોર્સ કોડ}: ટેલિકોમ્યુનિકેશનમાં વપરાતી એક એનકોડિંગ પદ્ધતિ છે જે ડોટ્સ (.) અને ડેશ (-) ના ક્રમનો ઉપયોગ કરીને ટેક્સ્ટ ટ્રાન્સમિટ કરે છે.

\begin{tabulary}{\linewidth}{|L|L|L|}
\hline
\textbf{તત્વ} & \textbf{સમય} & \textbf{ધ્વનિ} \\ \hline
ડોટ (.) & 1 યુનિટ & ટૂંકો બીપ \\ \hline
ડેશ (-) & 3 યુનિટ & લાંબો બીપ \\ \hline
તત્વો વચ્ચે જગ્યા & 1 યુનિટ & શાંતિ \\ \hline
અક્ષરો વચ્ચે જગ્યા & 3 યુનિટ & શાંતિ \\ \hline
શબ્દો વચ્ચે જગ્યા & 7 યુનિટ & શાંતિ \\ \hline
\end{tabulary}

\begin{answerdiagram}{મોર્સ કોડ પ્રેક્ટિસ ઓસિલેટર (555 Timer)}
\begin{tikzpicture}[circuit ee IEC, set resistor graphic=var resistor IEC graphic]
    \node [draw, rectangle, minimum width=2cm, minimum height=2.5cm, align=center] (IC) at (0,0) {555\\Timer};
    
    % Power
    \draw (IC.north) -- ++(0, 1) node [above] {+9V};
    
    % Speaker Output (Pin 3)
    \draw (IC.east) -- ++(1, 0) to [loudspeaker] ++(1, 0) -- ++(0, -2) coordinate (GND);
    
    % Key and Ground (Pin 1)
    \draw (IC.south) -- ++(0, -0.5) to [make contact={info={Key}}] ++(0, -1) -- (0, -2) node [ground] {};
    
    % Timing Components
    \draw (-2, 1) -- (0, 1); % Vcc rail
    \draw (-2, 1) to [resistor={info={$R_1$}}] (-2, 0) -- (IC.west);
    \draw (-2, 0) to [resistor={info={$R_2$}}] (-2, -1.5) to [capacitor={info={$C_1$}}] (-2, -2) -- (0, -2);
\end{tikzpicture}
\end{answerdiagram}

\textbf{કાર્યપદ્ધતિ}:
555 ટાઈમર અસ્ટેબલ મલ્ટિવાઈબ્રેટર તરીકે કામ કરે છે. જ્યારે કી (Key) દબાવવામાં આવે છે, ત્યારે સર્કિટ પૂર્ણ થાય છે અને સ્પીકરમાંથી 600-800 Hz નો અવાજ ઉત્પન્ન થાય છે. આનો ઉપયોગ હેમ રેડિયો શીખવા માટે થાય છે.
\end{solutionbox}

\begin{mnemonicbox}
\mnemonic{"TEMPO: ટાઇમિંગ એલિમેન્ટ્સ મેક પરફેક્ટ ઓસિલેશન"}
\end{mnemonicbox}

\orquestionmarks{3(અ)}{3}{માઈક્રોસ્ટ્રિપ પેચ એન્ટેના દોરો અને સમજાવો.}

\begin{solutionbox}
માઈક્રોસ્ટ્રિપ પેચ એન્ટેનામાં એક ડાઇલેક્ટ્રિક સબસ્ટ્રેટની એક બાજુ રેડિએટિંગ પેચ (ધાતુનો ટુકડો) અને બીજી બાજુ ગ્રાઉન્ડ પ્લેન હોય છે.

\begin{answerdiagram}{માઈક્રોસ્ટ્રિપ પેચ એન્ટેના}
\begin{tikzpicture}
    % Ground Plane
    \fill [gray!20] (0,0) rectangle (6,4);
    \draw [thick] (0,0) rectangle (6,4);
    \node [below] at (3,0) {ગ્રાઉન્ડ પ્લેન (નીચે)};

    % Substrate (implied thickness)
    \draw [dashed] (0.5, 0.5) rectangle (5.5, 3.5);
    \node at (5, 3) {સબસ્ટ્રેટ $\epsilon_r$};

    % Patch
    \fill [orange!40] (1.5, 1) rectangle (4.5, 3);
    \draw [thick] (1.5, 1) rectangle (4.5, 3);
    \node at (3, 2) {રેડિએટિંગ પેચ};

    % Feed Line
    \draw [thick, fill=orange!40] (3, 1) -- (3, 0) -- (3.2, 0) -- (3.2, 1);
    \node [below right] at (3.2, 0.5) {ફીડ લાઇન};
    
    % Radiation
    \foreach \x in {1.5, 4.5} \draw [->, red, decorate, decoration={coil, aspect=0, segment length=5pt, amplitude=5pt}] (\x, 3.2) -- (\x, 4.2);
    \node [red, above] at (3, 4.2) {રેડિએશન};
\end{tikzpicture}
\end{answerdiagram}

\begin{itemize}
    \item \textbf{ફાયદા}: લો પ્રોફાઇલ, વજનમાં હલકા, સસ્તા, કોઈપણ સપાટી પર લગાવી શકાય.
    \item \textbf{નુકસાન}: ઓછી બેન્ડવિડ્થ, ઓછી કાર્યક્ષમતા.
    \item \textbf{ઉપયોગો}: મોબાઈલ ફોન, GPS, સેટેલાઇટ ફોન.
\end{itemize}
\end{solutionbox}

\begin{mnemonicbox}
\mnemonic{"MAPS: માઈક્રોસ્ટ્રિપ એન્ટેના પેચિસ આર સિમ્પલ"}
\end{mnemonicbox}

\orquestionmarks{3(બ)}{4}{હોર્ન એન્ટેના દોરો અને સમજાવો.}

\begin{solutionbox}
હોર્ન એન્ટેના એ વેવગાઈડ છે જેનો છેડો ખુલ્લો અને પહોળો (flared) હોય છે.

\begin{answerdiagram}{હોર્ન એન્ટેના}
\begin{tikzpicture}
    % Waveguide section
    \draw [thick] (0, 0.5) rectangle (2, 1.5);
    \node at (1, 1) {વેવગાઈડ};
    
    % Flared Horn
    \draw [thick] (2, 1.5) -- (5, 2.5);
    \draw [thick] (2, 0.5) -- (5, -0.5);
    \draw [thick] (5, 2.5) -- (5, -0.5); % Aperture
    \draw [dashed] (5, 2.5) ellipse (0.2 and 1.5);
    
    % Labels
    \node [right] at (5, 1) {એપર્ચર (મુખ)};
    \draw [->, red] (3, 1) -- (6, 1) node [right] {બીમ};
\end{tikzpicture}
\end{answerdiagram}

\begin{itemize}
    \item \textbf{કાર્ય}: તે વેવગાઈડ અને મુક્ત અવકાશ (Free Space) વચ્ચે ઈમ્પિડન્સ મેચિંગ કરે છે.
    \item \textbf{પ્રકારો}: E-પ્લેન, H-પ્લેન, પિરામિડલ, કોનિકલ.
    \item \textbf{ઉપયોગો}: પેરાબોલિક ડિશના ફીડ તરીકે, રડાર, સેટેલાઇટ કોમ્યુનિકેશન.
\end{itemize}
\end{solutionbox}

\begin{mnemonicbox}
\mnemonic{"HEWB: હોર્ન્સ એન્હેન્સ વેવગાઇડ બીમવિડ્થ"}
\end{mnemonicbox}

\orquestionmarks{3(ક)}{7}{પેરાબોલિક રિફ્લેક્ટર એન્ટેના માટે વિવિધ ફીડ સિસ્ટમની યાદી બનાવો અને કોઈપણ એકને સમજાવો.}

\begin{solutionbox}
\textbf{ફીડ સિસ્ટમ્સ (Feed Systems)}:
1. ફ્રન્ટ ફીડ (Front Feed)
2. કેસેગ્રેન ફીડ (Cassegrain Feed)
3. ગ્રેગોરિયન ફીડ (Gregorian Feed)
4. ઓફસેટ ફીડ (Offset Feed)

\textbf{ફ્રન્ટ ફીડ સિસ્ટમ}:
આમાં પ્રાથમિક રેડિએટર (ફીડ હોર્ન) પેરાબોલિક રિફ્લેક્ટરના ફોકસ બિંદુ પર મૂકવામાં આવે છે.

\begin{answerdiagram}{ફ્રન્ટ ફીડ પેરાબોલિક રિફ્લેક્ટર}
\begin{tikzpicture}
    % Parabola
    \draw [ultra thick] (0, -2) parabola bend (1, 0) (0, 2);
    \node [below] at (0, -2) {પેરાબોલિક ડિશ};

    % Focus and Feed
    \fill (3, 0) circle (0.1);
    \node [above] at (3, 0.2) {ફોકસ};
    \draw [thick, fill=gray] (2.8, -0.2) rectangle (3.2, 0.2);
    \node [right] at (3.2, 0) {ફીડ હોર્ન};
    
    % Rays
    \draw [->, red] (-1, 1.5) -- (0.5, 1.5) -- (3, 0);
    \draw [->, red] (-1, -1.5) -- (0.5, -1.5) -- (3, 0);
    \node [red, left] at (-1, 0) {આવતા તરંગો};
    
    % Support
    \draw [thin] (1, 0) -- (3, 0);
\end{tikzpicture}
\end{answerdiagram}

\begin{itemize}
    \item \textbf{કાર્ય}: ડિશ સમાંતર કિરણોને પરાવર્તિત કરીને ફોકસ પર કેન્દ્રિત કરે છે જ્યાં ફીડ તેમને એકત્રિત કરે છે.
    \item \textbf{ફાયદા}: રચના સરળ છે.
    \item \textbf{નુકસાન}: ફીડ અને તેના સપોર્ટ સ્ટ્રક્ચરને કારણે થોડો પાવર બ્લોક થાય છે (Aperture Blockage), જેથી કાર્યક્ષમતા ઘટે છે.
\end{itemize}
\end{solutionbox}

\begin{mnemonicbox}
\mnemonic{"FACTS: ફોકસ્ડ એપર્ચર કેપ્ચર્સ ટ્રાન્સમિટેડ સિગ્નલ્સ"}
\end{mnemonicbox}

\questionmarks{4(અ)}{3}{HAM રેડિયોના કાર્યકારી સિદ્ધાંતને સમજાવો.}

\begin{solutionbox}
હેમ રેડિયો (એમેચ્યોર રેડિયો) એ એક શોખ અને સેવા છે જે લોકોને ઈલેક્ટ્રોનિક્સ અને કોમ્યુનિકેશન સાથે લાવે છે.

\begin{answerdiagram}{HAM રેડિયો કોમ્યુનિકેશન સિસ્ટમ}
\begin{tikzpicture}[node distance=2cm, auto]
    \node [gtu block] (tx) {ટ્રાન્સમીટર};
    \node [gtu block, right=of tx] (ant1) {એન્ટેના};
    \node [cloud, draw, aspect=2, red, right=of ant1] (med) {આયનોસ્ફિયર};
    \node [gtu block, right=of med] (ant2) {એન્ટેના};
    \node [gtu block, right=of ant2] (rx) {રિસીવર};

    \draw [gtu arrow] (tx) -- (ant1);
    \draw [gtu arrow, dashed] (ant1) -- node[above] {રેડિયો મોજા} (med);
    \draw [gtu arrow, dashed] (med) -- (ant2);
    \draw [gtu arrow] (ant2) -- (rx);
\end{tikzpicture}
\end{answerdiagram}

\begin{itemize}
    \item \textbf{સિદ્ધાંત}: યુઝર્સ નિયુક્ત ફ્રિક્વન્સી બેન્ડ પર સંદેશાવ્યવહાર કરે છે (જેમ કે ભારતમાં WPC દ્વારા ફાળવેલ).
    \item \textbf{મોડ્સ}: વોઇસ (AM/FM/SSB), ટેક્સ્ટ (મોર્સ કોડ), ડિજિટલ (પેકેટ રેડિયો).
    \item \textbf{મુખ્ય હેતુ}: બિન-વ્યાવસાયિક સંદેશાઓ, વાયરલેસ પ્રયોગો, સ્વ-તાલીમ અને કટોકટીમાં મદદ.
\end{itemize}
\end{solutionbox}

\begin{mnemonicbox}
\mnemonic{"TEAM: ટ્રાન્સમિશન એનેબલ્સ એમેચ્યોર મેસેજીસ"}
\end{mnemonicbox}

\questionmarks{4(બ)}{4}{ડક્ટ પ્રોપેગેશન સમજાવો.}

\begin{solutionbox}
ડક્ટ પ્રોપેગેશન એ એવી ઘટના છે જેમાં રેડિયો સિગ્નલ વાતાવરણના બે સ્તરો વચ્ચે અથવા સ્તર અને જમીન વચ્ચે "ફસાઈ" જાય છે અને સામાન્ય દૃષ્ટિ રેખા (Line-of-Sight) કરતાં ઘણું દૂર સુધી જાય છે.

\begin{answerdiagram}{એટમોસ્ફેરિક ડક્ટિંગ}
\begin{tikzpicture}
    % Ground
    \fill [brown!30] (0,0) rectangle (8, -0.5);
    \draw [thick] (0,0) -- (8,0);
    \node [below] at (4, -0.5) {પૃથ્વી સપાટી};

    % Atmosphere Layers
    \fill [cyan!10] (0,0) rectangle (8, 3);
    \draw [dashed, blue] (0, 2) -- (8, 2);
    \draw [dashed, blue] (0, 1) -- (8, 1);
    \node [right, blue] at (8, 1.5) {ડક્ટ લેયર (ઇન્વર્ઝન)};

    % Wave
    \draw [thick, red] (1, 0.5) sin (2, 1.8) cos (3, 1.2) sin (4, 1.8) cos (5, 1.2) sin (6, 1.8) cos (7, 0.5);
    \node [above, red] at (4, 1.8) {ફસાયેલ સિગ્નલ};
    
    % TX/RX
    \draw [thick] (1, 0) -- (1, 0.5); \node [below] at (1,0) {TX};
    \draw [thick] (7, 0) -- (7, 0.5); \node [below] at (7,0) {RX};
\end{tikzpicture}
\end{answerdiagram}

\begin{itemize}
    \item \textbf{કારણ}: ટેમ્પરેચર ઇન્વર્ઝન (ઠંડી હવા પર ગરમ હવા) અથવા ઊંચાઈ સાથે ભેજમાં ઘટાડો.
    \item \textbf{અસર}: રિફ્રેક્ટિવ ઇન્ડેક્સ ઝડપથી બદલાય છે, જેથી તરંગ પૃથ્વી તરફ વળે છે (Super-refraction).
    \item \textbf{રેન્જ}: VHF/UHF સિગ્નલો સેંકડો કિલોમીટર સુધી પહોંચી શકે છે.
\end{itemize}
\end{solutionbox}

\begin{mnemonicbox}
\mnemonic{"TRIP: ટ્રેપ્ડ રેઝ ઇન એટમોસ્ફિરિક પાથ્સ"}
\end{mnemonicbox}

\questionmarks{4(ક)}{7}{ટ્રોપોસ્ફેરિક સ્કેટર્ડ પ્રોપેગેશન વિગતવાર સમજાવો.}

\begin{solutionbox}
ટ્રોપોસ્ફેરિક સ્કેટર (Tropo-scatter) એ માઇક્રોવેવ રેડિયો સિગ્નલોને ક્ષિતિજ (horizon) થી પણ ઘણા દૂર (300 km સુધી) મોકલવાની પદ્ધતિ છે. તે માટે ટ્રોપોસ્ફિયર (પૃથ્વીનું સૌથી નીચેનું વાતાવરણ) ના સ્કેટરિંગ ગુણધર્મોનો ઉપયોગ થાય છે.

\begin{answerdiagram}{ટ્રોપોસ્ફેરિક સ્કેટર}
\begin{tikzpicture}
    % Earth Curve
    \draw [brown, thick] (-4, -1) .. controls (0, 0.5) .. (4, -1);
    \node [below] at (0, 0) {પૃથ્વી (ક્ષિતિજ)};

    % Troposphere
    \fill [cyan!5] (-4, 2) rectangle (4, 4);
    \node [blue] at (0, 3.5) {ટ્રોપોસ્ફિયર સ્કેટરિંગ વોલ્યુમ};

    % TX
    \draw [thick] (-3, -0.5) -- (-3, 0.5);
    \node [black] at (-3, -0.8) {TX};
    \draw [->, red, thick] (-3, 0.5) -- (0, 2.5);
    
    % RX
    \draw [thick] (3, -0.5) -- (3, 0.5);
    \node [black] at (3, -0.8) {RX};
    \draw [<-, red, thick] (3, 0.5) -- (0, 2.5);
    
    % Scattering
    \draw [orange, fill=orange!20] (0, 2.5) circle (0.5);
    \node [above right, orange] at (0.5, 2.8) {સ્કેટરિંગ વિસ્તાર};
\end{tikzpicture}
\end{answerdiagram}

\textbf{મિકેનિઝમ અને ખાસિયતો}:
\begin{tabulary}{\linewidth}{|L|L|}
\hline
\textbf{લક્ષણ} & \textbf{વર્ણન} \\ \hline
\keyword{સિદ્ધાંત} & વાતાવરણમાં રહેલી અશાંતિ (turbulence) અને અનિયમિતતાઓને કારણે રેડિયો તરંગોનું આગળની તરફ વિખેરણ (Forward Scattering). \\ \hline
\keyword{ફ્રિક્વન્સી} & સામાન્ય રીતે UHF અને SHF (300 MHz - 10 GHz). \\ \hline
\keyword{વિસ્તાર} & કોમન સ્કેટરિંગ વોલ્યુમ: જ્યાં ટ્રાન્સમીટર અને રિસીવર એન્ટેનાના બીમ એકબીજાને છેદે છે. \\ \hline
\keyword{જરૂરિયાતો} & હાઈ ગેઈન એન્ટેના અને હાઈ પાવર ટ્રાન્સમીટર જરૂરી છે કારણ કે પાથ લોસ (Path Loss) ખૂબ વધારે હોય છે. \\ \hline
\keyword{વિશ્વસનીયતા} & આયનોસ્ફેરિક વિક્ષેપોથી મુક્ત છે, તેથી પ્રમાણમાં વિશ્વસનીય છે. \\ \hline
\end{tabulary}
\end{solutionbox}

\begin{mnemonicbox}
\mnemonic{"STARS: સ્કેટર ટ્રોપોસ્ફેરિક અલાઉઝ રેન્જ બિયોન્ડ સાઇટ"}
\end{mnemonicbox}

\orquestionmarks{4(અ)}{3}{ટર્નસ્ટાઇલ અને સુપર ટર્નસ્ટાઇલ એન્ટેના દોરો.}

\begin{solutionbox}
\begin{answerdiagram}{ટર્નસ્ટાઇલ અને સુપર ટર્નસ્ટાઇલ (બેટવિંગ)}
\begin{tikzpicture}
    % Turnstile
    \node [font=\bfseries] at (2, 3) {ટર્નસ્ટાઇલ};
    \draw [thick] (0, 1.5) -- (4, 1.5); % Horizontal 1
    \draw [thick] (2, 0) -- (2, 3);   % Horizontal 2 (Cross) - visual perspective
    \fill (2, 1.5) circle (0.1);
    \node at (2, -0.5) {બે ક્રોસ ડાયપોલ ($90^\circ$ ફેઝ)};
    
    % Super Turnstile (Batwing)
    \node [font=\bfseries] at (8, 3) {સુપર ટર્નસ્ટાઇલ (બેટવિંગ)};
    \draw [thick] (7, 0) -- (7, 2) -- (6, 2) -- (6, 1) -- cycle;
    \draw [thick] (9, 0) -- (9, 2) -- (10, 2) -- (10, 1) -- cycle;
    \draw [ultra thick] (8, -1) -- (8, 2.5) node [above] {માસ્ટ};
    \draw [thick] (7, 1) -- (9, 1); % Connection
    \node at (8, -0.5) {બ્રોડબેન્ડ માટે શીટ એલિમેન્ટ્સ};
\end{tikzpicture}
\end{answerdiagram}

\begin{itemize}
    \item \textbf{ટર્નસ્ટાઇલ}: હોરિઝોન્ટલ ઓમ્નીડાયરેક્શનલ પેટર્ન આપે છે. VHF/UHF માં વપરાય છે.
    \item \textbf{સુપર ટર્નસ્ટાઇલ}: ટર્નસ્ટાઇલનું સુધારેલું રૂપ જે વધુ બેન્ડવિડ્થ આપે છે. TV બ્રોડકાસ્ટિંગમાં બહોળા પ્રમાણમાં ઉપયોગ થાય છે.
\end{itemize}
\end{solutionbox}

\begin{mnemonicbox}
\mnemonic{"TACO: ટર્નસ્ટાઇલ એન્ટેના ક્રિએટ ઓમ્નિડાયરેક્શનલ પેટર્ન"}
\end{mnemonicbox}

\orquestionmarks{4(બ)}{4}{MUF, LUF અને OUF નું સંપૂર્ણ સ્વરૂપ આપો.}

\begin{solutionbox}
\begin{tabulary}{\linewidth}{|L|L|L|}
\hline
\textbf{ટૂંકું નામ} & \textbf{સંપૂર્ણ નામ} & \textbf{વર્ણન} \\ \hline
\keyword{MUF} & Maximum Usable Frequency & બે સ્થળો વચ્ચે સ્કાયવેવ કોમ્યુનિકેશન માટે ઉપયોગમાં લઈ શકાતી મહત્તમ આવર્તન. $f_{MUF} = f_c \sec \theta$. \\ \hline
\keyword{LUF} & Lowest Usable Frequency & સૌથી ઓછી આવર્તન જે સંતોષકારક સિગ્નલ-ટુ-નોઇઝ રેશિયો આપે છે. આનાથી નીચે શોષણ (absorption) વધી જાય છે. \\ \hline
\keyword{OUF/OWF} & Optimum Usable Frequency & વિશ્વસનીય સંચાર માટે પસંદ કરેલી ફ્રિક્વન્સી, જે સામાન્ય રીતે MUF ના 85\% હોય છે. \\ \hline
\end{tabulary}

\begin{answerdiagram}{ફ્રિક્વન્સી પસંદગી}
\begin{tikzpicture}
    \draw [->] (0,0) -- (0, 4) node [above] {ફ્રિક્વન્સી};
    \draw [dashed] (0, 3) -- (4, 3) node [right] {MUF (પરાવર્તન સીમા)};
    \draw [dashed] (0, 2.5) -- (4, 2.5) node [right] {OUF (0.85 $\times$ MUF)};
    \draw [dashed] (0, 1) -- (4, 1) node [right] {LUF (શોષણ સીમા)};
    
    \fill [green!20] (0.2, 1) rectangle (3.8, 3);
    \node at (2, 2) {ઉપયોગી રેન્જ};
\end{tikzpicture}
\end{answerdiagram}
\end{solutionbox}

\begin{mnemonicbox}
\mnemonic{"MLO: મેક્સિમમ અને લોવેસ્ટ ઓપ્ટિમમ નક્કી કરે છે"}
\end{mnemonicbox}

\orquestionmarks{4(ક)}{7}{વર્ચ્યુઅલ ઊંચાઈ, ક્રિટિકલ ફ્રિક્વન્સી અને સ્કીપ ડિસ્ટન્સ વિગતવાર સમજાવો.}

\begin{solutionbox}
આ પેરામીટર્સ આયનોસ્ફેરિક પ્રોપેગેશન (Skywave) માટે ખૂબ મહત્વના છે.

\begin{answerdiagram}{આયનોસ્ફેરિક પરાવર્તન}
\begin{tikzpicture}
    % Ionosphere
    \fill [cyan!10] (-4, 3) rectangle (4, 4);
    \draw [dashed, blue] (-4, 3) -- (4, 3);
    \node [right, blue] at (4, 3.5) {આયનોસ્ફિયર};
    
    % Virtual Height
    \draw [dashed] (0, 0) -- (0, 3.5);
    \node [right] at (0, 2) {વાસ્તવિક માર્ગ (વક્ર)};
    \node [left] at (0, 2) {વર્ચ્યુઅલ ઊંચાઈ $h'$};
    \draw [red, thick] (-3, 0) -- (0, 3.5) -- (3, 0); % Straight lines
    \node [red, above] at (0, 3.5) {આભાસી પરાવર્તન બિંદુ};

    % Ground
    \draw [brown, thick] (-4, 0) -- (4, 0);
    \node [below] at (-3, 0) {TX};
    \node [below] at (3, 0) {RX};

    % Skip Distance
    \draw [<->] (-3, -0.5) -- (3, -0.5) node [midway, below] {સ્કીપ ડિસ્ટન્સ $D_{skip}$};
\end{tikzpicture}
\end{answerdiagram}

\begin{enumerate}
    \item \textbf{વર્ચ્યુઅલ ઊંચાઈ ($h'$)}: જો રેડિયો વેવ પ્રકાશની ગતિએ સીધી રેખામાં ગતિ કરે અને પરાવર્તિત થાય તો જે ઊંચાઈએથી પરાવર્તન થયું હોત તે ઊંચાઈ. તે વાસ્તવિક ઊંચાઈ કરતા હંમેશા વધારે હોય છે.
    
    \item \textbf{ક્રિટિકલ ફ્રિક્વન્સી ($f_c$)}: મહત્તમ ફ્રિક્વન્સી જે પૃથ્વી પર પાછી આવે છે જ્યારે સિગ્નલને લંબરૂપે (vertically) આકાશમાં મોકલવામાં આવે છે.
    $$ f_c = 9\sqrt{N_{max}} $$
    જ્યાં $N_{max}$ મહત્તમ ઇલેક્ટ્રોન ઘનતા છે.
    
    \item \textbf{સ્કીપ ડિસ્ટન્સ ($D_{skip}$)}: ટ્રાન્સમીટરથી તે ન્યૂનતમ અંતર જ્યાં આપેલ ફ્રિક્વન્સી (જે $f_c$ કરતા વધારે છે) નું સ્કાયવેવ પરાવર્તિત થઈને પાછું પૃથ્વી પર આવે છે.
    $$ D_{skip} = 2h \sqrt{\left(\frac{f}{f_c}\right)^2 - 1} $$
\end{enumerate}
\end{solutionbox}

\begin{mnemonicbox}
\mnemonic{"VCS: વર્ચ્યુઅલ ઊંચાઈ સ્કીપ ડિસ્ટન્સ નિયંત્રિત કરે છે"}
\end{mnemonicbox}

\questionmarks{5(અ)}{3}{સુઘડ આકૃતિ સાથે વિવિધ આયોનોસ્ફીયર સ્તરો દર્શાવો.}

\begin{solutionbox}
\begin{answerdiagram}{આયનોસ્ફેરિક સ્તરો}
\begin{tikzpicture}
    % Earth
    \draw [ultra thick, brown] (-3, -1) .. controls (0, 0) .. (3, -1);
    \node at (0, -0.5) {પૃથ્વી સપાટી};

    % Layers
    \draw [dotted] (-3, 0.5) -- (3, 0.5);
    \node [right] at (3, 0.5) {D લેયર (60-90 km)};
    \draw [dotted] (-3, 1.2) -- (3, 1.2);
    \node [right] at (3, 1.2) {E લેયર (90-140 km)};
    \draw [dotted] (-3, 2.0) -- (3, 2.0);
    \node [right] at (3, 2.0) {F1 લેયર (140-250 km)};
    \draw [dotted] (-3, 3.0) -- (3, 3.0);
    \node [right] at (3, 3.0) {F2 લેયર (250-400 km)};
    
    % Sun
    \fill [orange] (-4, 3) circle (0.5);
    \foreach \angle in {0, 45, ..., 360} \draw [thick, orange] (-4,3) ++(\angle:0.6) -- ++(\angle:0.3);
    \node [above] at (-4, 3.6) {સૂર્ય (દિવસ)};
\end{tikzpicture}
\end{answerdiagram}

\begin{itemize}
    \item \textbf{D લેયર}: HF વેવ્સનું શોષણ કરે છે, રાત્રે અદૃશ્ય થાય છે.
    \item \textbf{E લેયર}: કેટલાક HF વેવ્સ પરાવર્તિત કરે છે.
    \item \textbf{F1/F2 લેયર}: લાંબા અંતરના સ્કાયવેવ કોમ્યુનિકેશન માટે મુખ્ય પરાવર્તક સ્તરો છે. રાત્રે F1 અને F2 જોડાઈ જાય છે.
\end{itemize}
\end{solutionbox}

\begin{mnemonicbox}
\mnemonic{"DEAF: નીચેથી ઉપર - D, E, And F લેયર્સ"}
\end{mnemonicbox}

\questionmarks{5(બ)}{4}{વિવિધ પ્રકારની સેટેલાઇટ કોમ્યુનિકેશન સિસ્ટમના નામ આપો અને તેની સરખામણી કરો.}

\begin{solutionbox}
\begin{tabulary}{\linewidth}{|L|L|L|}
\hline
\textbf{સિસ્ટમ} & \textbf{ઓર્બિટ} & \textbf{ખાસિયતો} \\ \hline
\keyword{GEO (Geostationary)} & 35,786 km (વિષુવવૃત્તીય) & પૃથ્વીની સાપેક્ષે સ્થિર. હાઈ લેટન્સી ($\sim$240ms). 3 સેટેલાઇટથી ગ્લોબલ કવરેજ. TV, હવામાન માટે વપરાય. \\ \hline
\keyword{MEO (Medium Earth)} & 2,000 - 35,000 km & GEO કરતા ઓછી લેટન્સી. GPS, GLONASS નેવિગેશન માટે વપરાય. \\ \hline
\keyword{LEO (Low Earth)} & 160 - 2,000 km & ખૂબ ઓછી લેટન્સી ($\sim$20ms). પૃથ્વીની સાપેક્ષે ઝડપથી ફરે છે. કવરેજ માટે ઘણા સેટેલાઇટની જરૂર પડે. Starlink. \\ \hline
\end{tabulary}

\begin{answerdiagram}{ઓર્બિટ સરખામણી}
\begin{tikzpicture}
    \draw [blue] (0,0) circle (1); \node at (0,0) {પૃથ્વી};
    \draw [red] (0,0) circle (1.5); \node [red, right] at (1.5, 0) {LEO};
    \draw [orange] (0,0) circle (2.5); \node [orange, right] at (2.5, 0) {MEO};
    \draw [black] (0,0) circle (4); \node [black, right] at (4, 0) {GEO};
\end{tikzpicture}
\end{answerdiagram}
\end{solutionbox}

\begin{mnemonicbox}
\mnemonic{"TBDMN: ટેલિકોમ, બ્રોડકાસ્ટિંગ, ડેટા, મિલિટરી, નેવિગેશન"}
\end{mnemonicbox}

\questionmarks{5(ક)}{7}{DTH રીસીવર સિસ્ટમ દોરો અને સમજાવો.}

\begin{solutionbox}
\textbf{DTH (Direct-To-Home)}: સેટેલાઇટ ટેલિવિઝન બ્રોડકાસ્ટિંગ સિસ્ટમ છે જેમાં ટીવી પ્રોગ્રામ્સ સેટેલાઇટ મારફતે સીધા ગ્રાહકના ઘર સુધી Ku-બેન્ડમાં પહોંચાડાય છે.

\begin{answerdiagram}{DTH સિસ્ટમ બ્લોક ડાયાગ્રામ}
\begin{tikzpicture}[node distance=1.5cm, auto]
    % Satellite
    \node (sat) at (0, 3) [cloud, draw, aspect=2, blue] {સેટેલાઇટ (Ku-બેન્ડ)};
    
    % Dish
    \node (dish) at (0, 1) [draw, semicircle, rotate=90, minimum width=1cm] {}; 
    \draw [thick] (dish.south) -- ++(0.5, 0); 
    \node [left] at (-0.5, 1) {ડિશ + LNB};
    
    % Blocks
    \node [gtu block, below=of dish] (stb) {સેટ-ટોપ બોક્સ (STB)};
    \node [gtu block, below=of stb] (tv) {ટીવી સેટ};

    \draw [gtu arrow, dashed] (sat) -- node [right] {સિગ્નલ (10-12 GHz)} (dish);
    \draw [gtu arrow] (dish) -- node [right] {Coax (IF 950-2150 MHz)} (stb);
    \draw [gtu arrow] (stb) -- node [right] {વિડિઓ/ઓડિયો} (tv);
\end{tikzpicture}
\end{answerdiagram}

\textbf{ઘટકો}:
\begin{enumerate}
    \item \textbf{પેરાબોલિક ડિશ}: 60-90 cm ઓફસેટ ડિશ જે નબળા સેટેલાઇટ સિગ્નલ એકત્રિત કરે છે.
    \item \textbf{LNBF (Low Noise Block)}: Ku-બેન્ડ સિગ્નલને એમ્પ્લીફાય કરે છે અને તેને નીચી IF (Intermediate Frequency) માં ફેરવે છે.
    \item \textbf{કોએક્સિયલ કેબલ}: RG-6 કેબલ IF સિગ્નલને STB સુધી લઈ જાય છે.
    \item \textbf{સેટ-ટોપ બોક્સ (STB)}: તેમાં ટ્યુનર, ડિમોડ્યુલેટર (QPSK) અને ડિકોડર (MPEG-4) હોય છે જે ડિજિટલ સિગ્નલને ટીવી માટે ઓડિયો/વિડિયોમાં ફેરવે છે.
    \item \textbf{સ્માર્ટ કાર્ડ}: પેઇડ ચેનલોને ડિક્રિપ્ટ કરવા માટે.
\end{enumerate}
\end{solutionbox}

\begin{mnemonicbox}
\mnemonic{"DOCS: ડિશ ઓબ્ટેઇન્સ, કન્વર્ટ્સ અને શોઝ સિગ્નલ્સ"}
\end{mnemonicbox}

\orquestionmarks{5(અ)}{3}{સ્માર્ટ એન્ટેનાની જરૂર શું છે? તેના ઉપયોગો લખો.}

\begin{solutionbox}
\textbf{સ્માર્ટ એન્ટેના}: એન્ટેના એરે અને સ્માર્ટ સિગ્નલ પ્રોસેસિંગનો સમન્વય છે, જે ઇચ્છિત યુઝર તરફ બીમ ફોકસ કરી શકે છે.

\textbf{જરૂરિયાત (Need)}:
\begin{itemize}
    \item \textbf{ક્ષમતા વધારવા}: ભીડભાડવાળા નેટવર્કમાં વધુ યુઝર્સને સેવા આપી શકાય (SDMA).
    \item \textbf{રેન્જ વધારવા}: હાઈ ગેઇનને કારણે કવરેજ વધે છે.
    \item \textbf{ઈન્ટરફીયરન્સ ઘટાડવા}: અનિચ્છનીય સિગ્નલો તરફ 'Null' ઉત્પન્ન કરી શકાય.
    \item \textbf{પાવર બચત}: માત્ર જરૂરી દિશામાં જ ઉર્જા મોકલે છે.
\end{itemize}

\textbf{ઉપયોગો}:
\begin{itemize}
    \item 4G/5G સેલ્યુલર નેટવર્ક (MIMO).
    \item રડાર સિસ્ટમ્સ.
    \item આધુનિક Wi-Fi રાઉટર્સ.
    \item સેટેલાઇટ કોમ્યુનિકેશન.
\end{itemize}
\end{solutionbox}

\begin{mnemonicbox}
\mnemonic{"SAFE: સ્માર્ટ એન્ટેના ફોર એફિશિયન્સી"}
\end{mnemonicbox}

\orquestionmarks{5(બ)}{4}{કેપ્લરનો ત્રીજો નિયમ સમજાવો.}

\begin{solutionbox}
\textbf{કેપ્લરનો ત્રીજો નિયમ (આવર્તકાળનો નિયમ)}:
કોઈપણ ઉપગ્રહના કક્ષીય આવર્તકાળ ($T$) નો વર્ગ તેની ભ્રમણકક્ષાની અર્ધ-દીર્ઘ અક્ષ ($a$) ના ઘનના સમપ્રમાણમાં હોય છે.

$$ T^2 \propto a^3 $$
$$ T^2 = \left( \frac{4\pi^2}{GM} \right) a^3 $$

જ્યાં:
\begin{itemize}
    \item $T$: આવર્તકાળ (Orbital period) સેકન્ડમાં.
    \item $a$: સેમી-મેજર એક્સિસ (મીટર).
    \item $G$: ગુરુત્વાકર્ષણ અચળાંક.
    \item $M$: પૃથ્વીનું દળ.
\end{itemize}

\begin{answerdiagram}{કેપ્લરનો ત્રીજો નિયમ}
\begin{tikzpicture}
    \draw [thick, dashed] (0,0) ellipse (3 and 2);
    \draw [fill=blue] (1,0) circle (0.5); \node [below] at (1, -0.6) {પૃથ્વી};
    \draw [fill=gray] (-3, 0) circle (0.2); \node [left] at (-3, 0) {ઉપગ્રહ};
    
    \draw [->] (1,0) -- (-2, 1.73); \node [midway, above] {$a$ (ત્રિજ્યા)};
    \node at (0, 0) {ઓર્બિટ};
    \node at (0, -2.5) {મોટી ત્રિજ્યા $a \rightarrow$ લાંબો સમય $T$};
\end{tikzpicture}
\end{answerdiagram}
\end{solutionbox}

\begin{mnemonicbox}
\mnemonic{"CAP: ક્યુબ ઓફ એક્સિસ ઈક્વલ્સ પીરિયડ સ્ક્વેર્ડ"}
\end{mnemonicbox}

\orquestionmarks{5(ક)}{7}{ટેરેસ્ટ્રીયલ મોબાઈલ કોમ્યુનિકેશન માટે એન્ટેનાના વિવિધ પ્રકારો ઓળખો અને વિગતવાર સમજાવો.}

\begin{solutionbox}
મુખ્યત્વે બે પ્રકારના એન્ટેના હોય છે: બેઝ સ્ટેશન એન્ટેના અને મોબાઈલ સ્ટેશન એન્ટેના.

\begin{tabulary}{\linewidth}{|L|L|L|}
\hline
\textbf{પ્રકાર} & \textbf{ઉદાહરણ} & \textbf{ખાસિયત} \\ \hline
\keyword{બેઝ સ્ટેશન} & સેક્ટરલ પેનલ & ડાયરેક્શનલ ($120^\circ$ કવરેજ), હાઈ ગેઇન. \\ \hline
\keyword{મોબાઈલ સ્ટેશન} & PIFA, મોનોપોલ & ઓમ્નીડાયરેક્શનલ, કોમ્પેક્ટ સાઈઝ. \\ \hline
\end{tabulary}

\begin{answerdiagram}{બેઝ સ્ટેશન સેક્ટર એન્ટેના}
\begin{tikzpicture}
    % Tower
    \draw [thick] (0,0) -- (0, 4);
    % Panel
    \draw [fill=gray!20] (0.1, 3) rectangle (0.5, 3.8);
    % Radiation
    \draw [->, red, thick] (0.5, 3.4) -- (2, 4);
    \draw [->, red, thick] (0.5, 3.4) -- (2.5, 3.4);
    \draw [->, red, thick] (0.5, 3.4) -- (2, 2.8);
    \node [right, red] at (2.5, 3.4) {સેક્ટર બીમ};
\end{tikzpicture}
\end{answerdiagram}

\textbf{સમજૂતી}:
\begin{enumerate}
    \item \textbf{બેઝ સ્ટેશન એન્ટેના}:
    \begin{itemize}
        \item સામાન્ય રીતે કોલિનિયર એરે પેનલ એન્ટેના વપરાય છે.
        \item તે મોબાઈલ ટાવર પર લાગે છે અને ૧૨૦ ડિગ્રી સેક્ટર કવર કરે છે.
        \item હોરિઝોન્ટલ દિશામાં પહોળો બીમ અને વર્ટિકલ દિશામાં સાંકડો બીમ આપે છે.
        \item ઇલેક્ટ્રિકલ ટિલ્ટની સુવિધા હોય છે.
    \end{itemize}
    \item \textbf{મોબાઈલ એન્ટેના}:
    \begin{itemize}
        \item \textbf{PIFA}: સ્માર્ટફોનની અંદર PCB પર હોય છે. નાનું અને સપાટ હોય છે.
        \item \textbf{મોનોપોલ/વ્હીપ}: વાહનો પર લાગે છે.
        \item તે બધી દિશામાંથી સિગ્નલ પકડી શકે તે માટે ઓમ્નીડાયરેક્શનલ પેટર્ન ધરાવે છે.
    \end{itemize}
\end{enumerate}
\end{solutionbox}

\begin{mnemonicbox}
\mnemonic{"BEST: બેઝ-સ્ટેશન્સ એમ્પ્લોય સેક્ટર ટેકનોલોજી"}
\end{mnemonicbox}

\end{document}
