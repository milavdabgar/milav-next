\documentclass{article}

% content/resources/templates/preamble.tex
\usepackage[margin=0.6in]{geometry}
\author{Milav Dabgar}
\usepackage{amsmath,amssymb,amsthm}
\usepackage{booktabs}
\usepackage{multirow}
\usepackage{xcolor}
\usepackage{tcolorbox}
\tcbuselibrary{breakable,skins}
\usepackage[colorlinks=true,linkcolor=blue]{hyperref}
\usepackage{titlesec}
\usepackage{enumitem}
\usepackage{tikz}
\usepackage{pgfplots}
\usepackage{circuitikz}
\usepackage[version=4]{mhchem}
\usepackage{longtable}
\usepackage{array}
\usepackage{float}
\usepackage{caption}
\usepackage{listings}

\lstset{
  basicstyle=\small\ttfamily,
  breaklines=true,
  breakatwhitespace=false,
  postbreak=\mbox{\textcolor{red}{$\hookrightarrow$}\space},
  float=false,
  numbers=left,
  numberstyle=\tiny\color{gray},
  numbersep=10pt,
  xleftmargin=2em,
  keywordstyle=\color{blue},
  commentstyle=\color{green!60!black},
  stringstyle=\color{purple},
  backgroundcolor=\color{gray!5},
  showstringspaces=false,
  tabsize=2,
  captionpos=b,
  keepspaces=true,
  columns=flexible
}

\pgfplotsset{compat=1.18}
\usetikzlibrary{shapes,arrows,positioning,calc,patterns,decorations.pathmorphing,decorations.markings,arrows.meta}

% Color scheme
\definecolor{headcolor}{RGB}{0,102,204}
\definecolor{keycolor}{RGB}{220,20,60}
\definecolor{solutioncolor}{RGB}{34,139,34}
\definecolor{mnemoniccolor}{RGB}{148,0,211}
\definecolor{codecolor}{RGB}{0,0,100}

% Spacing
\setlength{\parskip}{3pt}
\setlist[itemize]{nosep}
\setlist[enumerate]{nosep}

% Title formatting
\titleformat{\section}{\Large\bfseries\color{headcolor}}{\thesection}{1em}{}
\titleformat{\subsection}{\large\bfseries\color{headcolor}}{\thesubsection}{1em}{}

% Pandoc tightlist compatibility
\providecommand{\tightlist}{%
  \setlength{\itemsep}{0pt}\setlength{\parskip}{0pt}}

% Pandoc longtable compatibility
\newcounter{none}
\def\thenone{}


% content/resources/templates/gujarati-boxes.tex
\usepackage{fontspec}
\usepackage{polyglossia}

% Set Gujarati as main language (document is primarily in Gujarati)
% Note: gloss-gujarati.ldf doesn't exist in polyglossia, but it will use hyphenation patterns
\setdefaultlanguage{gujarati}
\setotherlanguage{english}

% Configure Gujarati font properly
% Use Language=Default to prevent polyglossia from trying to add language-specific features
% that don't exist for Gujarati, which causes "empty feature" warnings
\newfontfamily\gujaratifont[Script=Gujarati,AutoFakeBold=2.5,AutoFakeSlant=0.3]{Noto Sans Gujarati}
\setmainfont[Script=Gujarati,AutoFakeBold=2.5,AutoFakeSlant=0.3]{Noto Sans Gujarati}
% Use Noto Sans Gujarati for monospace to support Gujarati in text
\setmonofont[Scale=0.9]{Noto Sans Gujarati}

% Configure English to use the same font
\newfontfamily\englishfont[Script=Gujarati,AutoFakeBold=2.5,AutoFakeSlant=0.3]{Noto Sans Gujarati}

% Translations for polyglossia
\gappto\captionsgujarati{
  \renewcommand{\tablename}{કોષ્ટક}
  \renewcommand{\figurename}{આકૃતિ}
}

% Helper for TikZ nodes to ensure Gujarati font
\newcommand{\gu}[1]{{\gujaratifont #1}}

% Custom environments
\newtcolorbox{solutionbox}{
    breakable,
    enhanced,
    colback=solutioncolor!5!white,
    colframe=solutioncolor!75!black,
    fonttitle=\bfseries,
    title=જવાબ
}

\newtcolorbox{solutionboxnobreak}{
 colback=solutioncolor!5!white,
 colframe=solutioncolor!75!black,
 fonttitle=\bfseries,
 title=જવાબ
}

\newtcolorbox{keyformula}{
 breakable,
 enhanced,
 colback=keycolor!5!white,
 colframe=keycolor!75!black,
 fonttitle=\bfseries,
 title=રાસાયણિક સમીકરણ/સૂત્ર
}

\newtcolorbox{mnemonicbox}{
 breakable,
 enhanced,
 colback=mnemoniccolor!5!white,
 colframe=mnemoniccolor!75!black,
 fonttitle=\bfseries,
 title=મેમરી ટ્રીક
}


% Custom commands for GTU solutions
% This file defines semantic commands for consistent formatting

% Question command with automatic formatting
\newcommand{\question}[2]{%
  \section*{Question #1}%
  \textbf{#2}%
}

% OR question variant
\newcommand{\questionor}[2]{%
  \section*{Question #1 OR}%
  \textbf{#2}%
}

% Proper table environment with caption
\newenvironment{answertable}[1]{%
  \begin{table}[htbp]
  \centering
  \caption{#1}
}{%
  \end{table}
}

% Proper figure environment for diagrams
\newenvironment{answerdiagram}[1]{%
  \begin{figure}[htbp]
  \centering
  \caption{#1}
}{%
  \end{figure}
}

% Semantic markup for key terms
\newcommand{\keyword}[1]{\textbf{#1}}
\newcommand{\code}[1]{\texttt{#1}}
\newcommand{\classname}[1]{\texttt{#1}}
\newcommand{\methodname}[1]{\texttt{#1}}

% Proper quotation marks
\newcommand{\mnemonic}[1]{``#1''}


\title{Antenna and Wave Propagation (4341106) - Summer 2023 Solution}
\date{July 20, 2023}

\begin{document}
\maketitle

\questionmarks{1(a)}{3}{ઇલેક્ટ્રોમેગ્નેટિક તરંગોના કોઈપણ ત્રણ ગુણધર્મો લખો}

\begin{solutionbox}
\textbf{ઇલેક્ટ્રોમેગ્નેટિક તરંગોના ગુણધર્મો:}

\begin{tabulary}{\linewidth}{|L|}
\hline
1. EM તરંગો નિર્વાત અથવા પદાર્થ માધ્યમમાં પ્રવાસ કરી શકે છે \\ \hline
2. EM તરંગો ફ્રી સ્પેસમાં પ્રકાશની ગતિએ પ્રવાસ કરે છે ($3 \times 10^8$ m/s) \\ \hline
3. EM તરંગો દોલનશીલ વીજળી અને ચુંબકીય ક્ષેત્રો સાથે આડી લાક્ષણિકતાઓ દર્શાવે છે \\ \hline
\end{tabulary}
\end{solutionbox}

\begin{mnemonicbox}
\mnemonic{"VTS" - Vacuum travel, Transverse nature, Speed of light}
\end{mnemonicbox}

\questionmarks{1(b)}{4}{વ્યાખ્યા લખો: (1) રેડિયેશન રેઝિસ્ટન્સ (2) ડાયરેક્ટિવિટી (3) ગેઈન}

\begin{solutionbox}
\textbf{વ્યાખ્યા:}

\begin{tabulary}{\linewidth}{|L|L|}
\hline
\textbf{શબ્દ} & \textbf{વ્યાખ્યા} \\ \hline
\keyword{રેડિયેશન રેઝિસ્ટન્સ} & તે સમકક્ષ અવરોધ છે જે એન્ટેના ઇનપુટ કરંટની બરાબર હોય ત્યારે એન્ટેના દ્વારા વિકિરણ કરવામાં આવતી ઊર્જા જેટલી જ ઊર્જા વેડફે છે \\ \hline
\keyword{ડાયરેક્ટિવિટી} & ચોક્કસ દિશામાં મહત્તમ વિકિરણ તીવ્રતા અને બધી દિશાઓમાં સરેરાશ વિકિરણ તીવ્રતાનો ગુણોત્તર \\ \hline
\keyword{ગેઈન} & નિર્દિષ્ટ દિશામાં રેડિયો તરંગોમાં ઇનપુટ પાવરને કેટલી કાર્યક્ષમતાથી રૂપાંતરિત કરે છે તે માપતા ડાયરેક્ટિવિટી અને રેડિયેશન એફિશિયન્સીનો ગુણાકાર \\ \hline
\end{tabulary}
\end{solutionbox}

\begin{mnemonicbox}
\mnemonic{"RDG" - Resistance dissipates power, Direction concentration, Gain includes efficiency}
\end{mnemonicbox}

\questionmarks{1(c)}{7}{ઇલેક્ટ્રોમેગ્નેટિક તરંગોના નિર્માણની ભૌતિક ખ્યાલ સુઘડ રેખાકૃતિ સાથે સમજાવો}

\begin{solutionbox}
ઇલેક્ટ્રોમેગ્નેટિક તરંગો ત્યારે ઉત્પન્ન થાય છે જ્યારે ઇલેક્ટ્રિક ચાર્જ પ્રવેગ કરે છે અથવા દોલન કરે છે, જે અવકાશમાં પ્રસરિત થતા યુગ્મિત દોલનશીલ ઇલેક્ટ્રિક અને ચુંબકીય ક્ષેત્રો બનાવે છે.

\begin{center}
\begin{tikzpicture}[node distance=2cm, auto]
    \node [gtu block] (curr) {Electric Current Flow};
    \node [gtu block, right=of curr] (elec) {Oscillating E-Field};
    \node [gtu block, below=of elec] (mag) {Oscillating H-Field};
    \node [gtu block, right=of elec] (prop) {Wave Propagation};

    \draw [gtu arrow] (curr) -- node {Oscillation} (elec);
    \draw [gtu arrow] (elec) -- node {Induces} (mag);
    \draw [gtu arrow] (mag) -- node [left] {Induces} (elec);
    \draw [gtu arrow] (elec) -- (prop);
\end{tikzpicture}
\end{center}

\begin{answerdiagram}{Dipole Antenna EM Wave Generation}
\begin{tikzpicture}[scale=1]
    % Dipole
    \draw [thick] (0, 1) -- (0, 0.2);
    \draw [thick] (0, -1) -- (0, -0.2);
    \node at (0, 1.2) {+};
    \node at (0, -1.2) {-};
    
    % Source
    \draw (-0.5, 0.2) -- (0, 0.2);
    \draw (-0.5, -0.2) -- (0, -0.2);
    \draw (-0.5, -0.5) rectangle (-1.5, 0.5);
    \node at (-1, 0) {$\sim$};
    \node [above] at (-1, 0.5) {Oscillator};

    % E-field
    \foreach \x in {1, 2, 3} {
        \draw [blue, ->] (0.5*\x, 1.5) sin (0.5*\x+0.25, 1) cos (0.5*\x+0.5, 1.5);
        \draw [blue, ->] (0.5*\x, -1.5) sin (0.5*\x+0.25, -1) cos (0.5*\x+0.5, -1.5);
    }
    \node [blue, right] at (3, 1.5) {E-Field};

    % H-field (circles)
    \foreach \x in {1, 2, 3} {
        \draw [red] (0.5*\x + 0.25, 0) circle (0.1);
        \fill [red] (0.5*\x + 0.25, 0) circle (0.03);
    }
    \node [red, right] at (3, 0) {H-Field};
\end{tikzpicture}
\end{answerdiagram}

\begin{itemize}
    \item \textbf{મૂળભૂત ખ્યાલ}: જ્યારે AC કરંટ એન્ટેનામાં વહે છે, ત્યારે ઇલેક્ટ્રોન ઉપર અને નીચે પ્રવેગ કરે છે.
    \item \textbf{ઇલેક્ટ્રિક ફિલ્ડ}: એન્ટેનામાં ચાર્જ વિભાજનથી બને છે.
    \item \textbf{મેગ્નેટિક ફિલ્ડ}: કરંટ પ્રવાહથી ઉત્પન્ન થાય છે, ઇલેક્ટ્રિક ફિલ્ડને લંબરૂપે.
    \item \textbf{પ્રસરણ}: ફિલ્ડ એન્ટેનાથી અલગ થઈને પ્રકાશની ગતિએ બહારની તરફ પ્રસરે છે.
    \item \textbf{સ્વ-ટકાઉ}: તરંગ પ્રવાસ કરતાં દરેક ફિલ્ડ ઘટક અન્ય ઘટકને પુનર્જીવિત કરે છે.
\end{itemize}
\end{solutionbox}

\begin{mnemonicbox}
\mnemonic{"COMAP" - Current Oscillations Make Alternating Propagations}
\end{mnemonicbox}

\orquestionmarks{1(c)}{7}{435 MHZ આવૃત્તિ માટે 4 એલિમેન્ટ વાળુ યાગી ઉદા એન્ટેના ની ડિઝાઇન બનાવો.}

\begin{solutionbox}
\textbf{435 MHz માટે 4-એલિમેન્ટ યાગી-ઉદા એન્ટેના માટે:}

\textbf{વપરાયેલા સૂત્રો}:
\begin{itemize}
    \item તરંગલંબાઈ: $\lambda = c/f = 3\times10^8 / 435\times10^6 = 0.69$ મીટર
    \item હાફ-વેવ ડાયપોલ: $L = 0.5\lambda = 34.5$ cm
    \item એલિમેન્ટ અંતર: $S = 0.15\lambda$ થી $0.25\lambda$
\end{itemize}

\textbf{ગણતરી કરેલ મૂલ્ય:}
\begin{tabulary}{\linewidth}{|L|L|L|L|}
\hline
\textbf{એલિમેન્ટ} & \textbf{લંબાઈ ફોર્મ્યુલા} & \textbf{અંતર ફોર્મ્યુલા} & \textbf{મૂલ્ય} \\ \hline
\textbf{રિફ્લેક્ટર} & $0.5\lambda \times 1.05$ & - & 36.2 cm \\ \hline
\textbf{ડ્રાઇવન એલિમેન્ટ} & $0.5\lambda$ & - & 34.5 cm \\ \hline
\textbf{ડાયરેક્ટર 1} & $0.45\lambda$ & ડ્રાઇવનથી $0.2\lambda$ & 31.0 cm (Sp: ~13.8 cm) \\ \hline
\textbf{ડાયરેક્ટર 2} & $0.43\lambda$ & ડાયરેક્ટર 1થી $0.25\lambda$ & 29.6 cm (Sp: ~17.2 cm) \\ \hline
\end{tabulary}

\begin{answerdiagram}{4-Element Yagi-Uda Antenna Layout}
\begin{tikzpicture}[scale=0.15]
    % Boom
    \draw [thick] (-5, 0) -- (40, 0);
    
    % Elements (vertical lines centered on boom)
    % Reflector
    \draw [ultra thick] (0, -18.1) -- (0, 18.1) node [above] {Reflector (36.2cm)};
    
    % Driven
    \draw [ultra thick] (10, -17.25) -- (10, 17.25) node [above] {Driven (34.5cm)};
    \node [below] at (5, 0) {Spacing ~13.8cm}; % Approx spacing visual
    
    % Director 1
    \draw [ultra thick] (23.8, -15.5) -- (23.8, 15.5) node [above] {D1 (31.0cm)};
    
    % Director 2
    \draw [ultra thick] (41, -14.8) -- (41, 14.8) node [above] {D2 (29.6cm)};
    
    \draw [<->] (0, -20) -- (41, -20) node [midway, below] {Direction of Radiation $\longrightarrow$};
\end{tikzpicture}
\end{answerdiagram}
\end{solutionbox}

\begin{mnemonicbox}
\mnemonic{"RDDS" - Reflector Driven Directors Shrink}
\end{mnemonicbox}

\questionmarks{2(a)}{3}{લુપ એન્ટેના આકૃતિની મદદથી સમજાવો}

\begin{solutionbox}
લુપ એન્ટેના એક વાહક ને લુપ આકારમાં બનાવીને વિકિરણ ઘટક બનાવવામાં આવે છે.

\begin{answerdiagram}{Loop Antenna}
\begin{tikzpicture}
    \draw [thick] (0,0) circle (1.5cm);
    \draw [->] (0.8, 0.8) arc (45:135:1.1) node [midway, above] {Current flow};
    \draw [thick] (-0.2, -1.5) -- (-0.2, -2.5);
    \draw [thick] (0.2, -1.5) -- (0.2, -2.5);
    \node [below] at (0, -2.5) {Feed point};
    \node at (0,0) {Loop};
\end{tikzpicture}
\end{answerdiagram}

\begin{itemize}
    \item \textbf{નાના લુપ}: પરિઘ $< \lambda/10$, રેડિએશન પેટર્ન મેગ્નેટિક ડાયપોલ જેવા.
    \item \textbf{મોટા લુપ}: પરિઘ $\approx$ તરંગલંબાઈ, દ્વિદિશાત્મક રેડિએશન પેટર્ન (Bidirectional).
    \item \textbf{ઉપયોગો}: દિશા શોધવી (Direction finding), AM રેડિયો રિસેપ્શન, RFID ટેગ્સ.
\end{itemize}
\end{solutionbox}

\begin{mnemonicbox}
\mnemonic{"SLC" - Size affects Loop Characteristics}
\end{mnemonicbox}

\questionmarks{2(b)}{4}{નોન રેઝોનેંટ વાયર એન્ટેના સમજાવો}

\begin{solutionbox}
\textbf{નોન રેઝોનેંટ વાયર એન્ટેના:}

\begin{tabulary}{\linewidth}{|L|L|}
\hline
\textbf{લક્ષણ} & \textbf{વર્ણન} \\ \hline
\keyword{વ્યાખ્યા} & એવા આવૃત્તિઓ પર કાર્ય કરતા એન્ટેના જ્યાં તેની ભૌતિક લંબાઈ અર્ધ-તરંગલંબાઈના ગુણાંક નથી \\ \hline
\keyword{ઇમ્પીડન્સ} & જટિલ, રેઝિસ્ટિવ અને રિએક્ટિવ બંને ઘટકો સાથે \\ \hline
\keyword{સ્ટેન્ડિંગ વેવ્સ} & એન્ટેનાની લંબાઈ પર હાજર \\ \hline
\keyword{ઉદાહરણ} & રોમ્બિક એન્ટેના, અંતમાં અવરોધથી ટર્મિનેટ કરેલ \\ \hline
\keyword{ફાયદો} & વાઇડબેન્ડ ઓપરેશન, મલ્ટીપલ ફ્રીક્વન્સી માટે યોગ્ય \\ \hline
\end{tabulary}
\end{solutionbox}

\begin{mnemonicbox}
\mnemonic{"NITRO" - Non-resonance Incurs Termination for Resistance and Operation}
\end{mnemonicbox}

\questionmarks{2(c)}{7}{હાફ વેવ ડાયપોલ એન્ટેના નું રેડિયેશન રેઝીસ્ટંસ કેટલું હોય છે? $\lambda/2$, $\lambda$ અને $\lambda/4$ લમ્બાઇ ના એન્ટેના રેડિયેશન ની પેટર્ન દોરો}

\begin{solutionbox}
હાફ-વેવ ડાયપોલનું રેડિયેશન રેઝીસ્ટંસ આશરે \textbf{73 ઓહ્મ} હોય છે.

\textbf{રેડિયેશન પેટર્ન:}

\begin{answerdiagram}{Dipole Radiation Patterns}
\begin{tikzpicture}[scale=0.8]
    % Lambda/2
    \begin{scope}[xshift=0cm]
        \draw [help lines] (0,0) circle (1.5);
        \draw [thick, blue, rotate=90] (0,0) ellipse (0.5 and 1.5);
        \draw [thick, blue, rotate=90] (0,0) ellipse (0.5 and 1.5); % Figure 8 visual
        \fill [blue, opacity=0.2, rotate=90] (0,0) ellipse (0.5 and 1.5);
        \node [below] at (0, -2) {$\lambda/2$ Dipole};
        \node [below] at (0, -2.5) {(Figure-8)};
        \draw [thick] (0, -0.2) -- (0, 0.2); % Antenna
    \end{scope}

    % Lambda
    \begin{scope}[xshift=5cm]
        \draw [help lines] (0,0) circle (1.5);
        % Multi-lobed
        \draw [thick, red] plot [domain=0:360, samples=100] ({1.5*cos(2*\x)*cos(\x)}, {1.5*cos(2*\x)*sin(\x)});
        \node [below] at (0, -2) {$\lambda$ Dipole};
        \node [below] at (0, -2.5) {(Multi-lobed)};
        \draw [thick] (0, -0.4) -- (0, 0.4); % Antenna
    \end{scope}

    % Lambda/4
    \begin{scope}[xshift=10cm]
        \draw [help lines] (0,0) circle (1.5);
        \draw [thick, orange] (0,0) circle (1.2); % Omnidirectional-ish / Broad
        \node [below] at (0, -2) {$\lambda/4$ Dipole};
        \node [below] at (0, -2.5) {(Broad pattern)};
        \draw [thick] (0, 0) -- (0, 0.5); % Monopole
        \draw (-0.5, 0) -- (0.5, 0); % Ground
    \end{scope}
\end{tikzpicture}
\end{answerdiagram}

\begin{tabulary}{\linewidth}{|L|L|}
\hline
\textbf{ડાયપોલ લંબાઈ} & \textbf{પેટર્ન લક્ષણો} \\ \hline
\textbf{$\lambda/2$ ડાયપોલ} & ફિગર-8 પેટર્ન; એન્ટેના અક્ષને લંબરૂપે મહત્તમ વિકિરણ; HPBW = $78^\circ$ \\ \hline
\textbf{$\lambda$ ડાયપોલ} & મલ્ટી-લોબ્ડ પેટર્ન; એન્ટેના અક્ષ પર કોણે ચાર મુખ્ય લોબ \\ \hline
\textbf{$\lambda/4$ ડાયપોલ} & $\lambda/2$ કરતાં વધુ વિશાળ પેટર્ન; સમતુલ્ય ડાયપોલ પૂર્ણ કરવા માટે ગ્રાઉન્ડ પ્લેનની જરૂર \\ \hline
\end{tabulary}
\end{solutionbox}

\begin{mnemonicbox}
\mnemonic{"SHORT" - Smaller Half-dipole Offers Rounded-Transmissions}
\end{mnemonicbox}

\orquestionmarks{2(a)}{3}{ફોલ્ડેડ ડાઇપોલ એન્ટેના આકૃતિની મદદથી સમજાવો}

\begin{solutionbox}
ફોલ્ડેડ ડાયપોલ એ હાફ-વેવ ડાયપોલનો એક પ્રકાર છે જેમાં છેડાઓને પાછા વાળીને લૂપ બનાવવા માટે જોડવામાં આવે છે.

\begin{answerdiagram}{Folded Dipole}
\begin{tikzpicture}
    \draw [thick, rounded corners=5pt] (-2, 0.5) rectangle (2, -0.5);
    \fill [white] (-0.2, -0.6) rectangle (0.2, -0.4); % Break bottom
    \draw [thick] (-0.2, -0.5) -- (-0.2, -1);
    \draw [thick] (0.2, -0.5) -- (0.2, -1);
    \node [below] at (0, -1) {Feed point};
    \node at (0, 0) {Folded Loop};
\end{tikzpicture}
\end{answerdiagram}

\begin{itemize}
    \item \textbf{ઇનપુટ ઇમ્પીડન્સ}: આશરે 300 ઓહ્મ (સામાન્ય ડાયપોલના 4 ગણા).
    \item \textbf{બેન્ડવિડ્થ}: સામાન્ય ડાયપોલ કરતાં વધારે.
    \item \textbf{ઉપયોગો}: TV રિસેપ્શન, FM રેડિયો, બેલેન્સ્ડ ટ્રાન્સમિશન લાઇન્સ.
\end{itemize}
\end{solutionbox}

\begin{mnemonicbox}
\mnemonic{"FIB" - Folded Increases Bandwidth}
\end{mnemonicbox}

\orquestionmarks{2(b)}{4}{રોમ્બિક એન્ટેના આકૃતિની મદદથી સમજાવો}

\begin{solutionbox}
રોમ્બિક એન્ટેના એક રોમ્બસ અથવા હીરા આકારમાં ગોઠવાયેલા ચાર તારોનો બનેલો હોય છે.

\begin{answerdiagram}{Rhombic Antenna}
\begin{tikzpicture}[scale=0.8]
    \coordinate (A) at (0,0);
    \coordinate (B) at (3,1.5);
    \coordinate (C) at (6,0);
    \coordinate (D) at (3,-1.5);
    
    \draw [thick] (A) -- (B) -- (C) -- (D) -- cycle;
    
    \draw [->] (-1, 0) -- (0, 0) node [left] {Feed};
    \draw [thick] (A) -- (-0.5, 0.5);
    \draw [thick] (A) -- (-0.5, -0.5);
    
    \node [right] at (C) {Load $R$};
    \draw [thick] (C) -- (6.5, 0.5);
    \draw [thick] (C) -- (6.5, -0.5);
    \draw [decorate, decoration={zigzag, amplitude=2pt, segment length=5pt}] (6.5, 0.5) -- (6.5, -0.5);
    
    \draw [->, thick] (2, 0) -- (4, 0) node [midway, above] {Direction of Radiation};
\end{tikzpicture}
\end{answerdiagram}

\begin{tabulary}{\linewidth}{|L|L|}
\hline
\textbf{લક્ષણ} & \textbf{વર્ણન} \\ \hline
\keyword{આકાર} & ડાયમંડ/રોમ્બસ, દૂરના છેડે ટર્મિનેટિંગ રેઝિસ્ટર સાથે \\ \hline
\keyword{ઓપરેશન} & નોન-રેઝોનન્ટ ટ્રાવેલિંગ-વેવ એન્ટેના \\ \hline
\keyword{ડાયરેક્ટિવિટી} & ઉચ્ચ ગેઇન, યુનિડાયરેક્શનલ પેટર્ન \\ \hline
\keyword{બેન્ડવિડ્થ} & ખૂબ વિશાળ આવૃત્તિ શ્રેણી \\ \hline
\keyword{ઉપયોગો} & HF કમ્યુનિકેશન્સ, પોઇન્ટ-ટુ-પોઇન્ટ લિંક્સ \\ \hline
\end{tabulary}
\end{solutionbox}

\begin{mnemonicbox}
\mnemonic{"TREND" - Terminated Rhombic Enables Numerous Directions}
\end{mnemonicbox}

\orquestionmarks{2(c)}{7}{આકૃતિની મદદથી એન્ડ ફાયર અને બ્રોડ સાઇડ એન્ટેના નો તફાવત સમજાવો}

\begin{solutionbox}
\textbf{તફાવત:}

\begin{tabulary}{\linewidth}{|L|L|L|}
\hline
\textbf{પેરામીટર} & \textbf{બ્રોડસાઇડ એરે} & \textbf{એન્ડ ફાયર એરે} \\ \hline
\keyword{મહત્તમ વિકિરણની દિશા} & એરે અક્ષને લંબરૂપે & એરે અક્ષ સાથે \\ \hline
\keyword{એલિમેન્ટ ફેઝિંગ} & સમાન ફેઝ ($0^\circ$) & પ્રગતિશીલ ફેઝ શિફ્ટ \\ \hline
\keyword{એલિમેન્ટ અંતર} & સામાન્ય રીતે $\lambda/2$ & સામાન્ય રીતે $\lambda/4$ \\ \hline
\keyword{રેડિયેશન પેટર્ન} & ફેન-આકારનો બીમ & પેન્સિલ-આકારનો બીમ \\ \hline
\keyword{ઉપયોગો} & બ્રોડકાસ્ટિંગ, બેઝ સ્ટેશન્સ & પોઇન્ટ-ટુ-પોઇન્ટ લિંક્સ \\ \hline
\end{tabulary}

\begin{answerdiagram}{Array Comparison}
\begin{tikzpicture}
    % Broadside
    \begin{scope}[xshift=0cm]
        \foreach \x in {0,1,2,3} \fill (\x, 0) circle (0.1);
        \draw (-0.5, 0) -- (3.5, 0) node [right] {Axis};
        \foreach \x in {0,1,2,3} \draw [->, thick, blue] (\x, 0.2) -- (\x, 1);
        \node [above] at (1.5, 1) {Broadside Radiation};
    \end{scope}
    
    % End fire
    \begin{scope}[xshift=6cm]
        \foreach \x in {0,1,2,3} \fill (\x, 0) circle (0.1);
        \draw (-0.5, 0) -- (3.5, 0) node [right] {Axis};
        \draw [->, thick, red] (3.5, 0) -- (5, 0) node [right] {End-fire Radiation};
    \end{scope}
\end{tikzpicture}
\end{answerdiagram}
\end{solutionbox}

\begin{mnemonicbox}
\mnemonic{"PAPER" - Perpendicular And Parallel Emission Respectively}
\end{mnemonicbox}

\questionmarks{3(a)}{3}{આકૃતિની મદદથી ઇન્વર્ટેડ વી એન્ટેના સમજાવો}

\begin{solutionbox}
ઇન્વર્ટેડ V એન્ટેના એ ડાયપોલ છે જેની બાહુઓ નીચેની તરફ વળેલી હોય છે, ઉલટા "V" જેવી દેખાય છે.

\begin{answerdiagram}{Inverted V Antenna}
\begin{tikzpicture}
    \draw [thick] (0,0) -- (0, 2); % Support
    \draw [thick] (0, 2) -- (-1.5, 1); % Left arm
    \draw [thick] (0, 2) -- (1.5, 1); % Right arm
    \draw [thick] (0, 2) -- (0, 1.8); % Feed line
    \node [right] at (0, 1.5) {Support};
    \node [below] at (0, 0) {Ground};
    \draw (-2, 0) -- (2, 0);
    \draw [->] (-0.5, 1.5) arc (135:45:0.7) node [midway, above] {$90^\circ - 120^\circ$};
\end{tikzpicture}
\end{answerdiagram}

\begin{itemize}
    \item \textbf{ખૂણો}: બાહુઓ સામાન્ય રીતે $90^\circ-120^\circ$ ખૂણો બનાવે છે.
    \item \textbf{ઇમ્પીડન્સ}: આશરે 50 ઓહ્મ, આડા ડાયપોલ કરતાં ઓછું.
    \item \textbf{પેટર્ન}: સર્વવ્યાપી, આડા ડાયપોલ કરતાં થોડું વધુ વિશાળ.
    \item \textbf{ઉપયોગો}: એમેચ્યોર રેડિયો, શોર્ટવેવ કમ્યુનિકેશન્સ.
\end{itemize}
\end{solutionbox}

\begin{mnemonicbox}
\mnemonic{"AVS" - Angle Varies Signal}
\end{mnemonicbox}

\questionmarks{3(b)}{4}{આકૃતિની મદદથી પેરાબોલિક રીફ્લેક્ટર એન્ટેના સમજાવો}

\begin{solutionbox}
\begin{answerdiagram}{Parabolic Reflector Antenna}
\begin{tikzpicture}
    \draw [thick] (2, 2) parabola bend (0, 0) (2, -2);
    \draw [thick] (-0.5, 0) -- (0.5, 0); % Feed
    \node [left] at (-0.5, 0) {Feed};
    \fill (0.5, 0) circle (0.1) node [above] {Focus};
    \draw [->, dashed] (5, 1.5) -- (1.5, 1.5) -- (0.5, 0); % Incoming ray 1
    \draw [->, dashed] (5, -1.5) -- (1.5, -1.5) -- (0.5, 0); % Incoming ray 2
    \node [right] at (2.5, 0) {Parallel Rays};
\end{tikzpicture}
\end{answerdiagram}

\begin{tabulary}{\linewidth}{|L|L|}
\hline
\textbf{ઘટક} & \textbf{કાર્ય} \\ \hline
\keyword{પેરાબોલિક રિફ્લેક્ટર} & આવતા સિગ્નલ્સને એકત્રિત કરે છે અને કેન્દ્રિત કરે છે અથવા ટ્રાન્સમિટ થયેલા સિગ્નલોને નિર્દેશિત કરે છે \\ \hline
\keyword{ફીડ એલિમેન્ટ} & પેરાબોલાના ફોકલ પોઇન્ટ પર સ્થિત, સિગ્નલ્સને એકત્રિત/પ્રસારિત કરે છે \\ \hline
\keyword{ફોકલ લેન્થ} & વર્ટેક્સથી ફોકસ સુધીનું અંતર, બીમની લાક્ષણિકતાઓ નક્કી કરે છે \\ \hline
\keyword{ઉપયોગો} & સેટેલાઇટ કમ્યુનિકેશન, રડાર, રેડિયો એસ્ટ્રોનોમી, માઇક્રોવેવ લિંક્સ \\ \hline
\end{tabulary}
\end{solutionbox}

\begin{mnemonicbox}
\mnemonic{"FOLD" - Focus Of Large Dish}
\end{mnemonicbox}

\questionmarks{3(c)}{7}{HF, VHF અને UHF માટેની આવૃત્તિની રેન્જ લખો. માઇક્રોસ્ટ્રીપ એન્ટેના વિશે ટૂંક નોંધ લખો.}

\begin{solutionbox}
\textbf{આવૃત્તિની રેન્જ:}

\begin{tabulary}{\linewidth}{|L|L|}
\hline
\textbf{ફ્રિક્વન્સી બેન્ડ} & \textbf{રેન્જ} \\ \hline
\textbf{HF (હાઇ ફ્રિક્વન્સી)} & 3 MHz - 30 MHz \\ \hline
\textbf{VHF (વેરી હાઇ ફ્રિક્વન્સી)} & 30 MHz - 300 MHz \\ \hline
\textbf{UHF (અલ્ટ્રા હાઇ ફ્રિક્વન્સી)} & 300 MHz - 3 GHz \\ \hline
\end{tabulary}

\textbf{માઇક્રોસ્ટ્રીપ એન્ટેના:}

\begin{answerdiagram}{Microstrip Antenna Structure}
\begin{tikzpicture}
    \draw [fill=gray!20] (0,0) rectangle (4, 0.5); % Substrate
    \draw [fill=gray!50] (1, 0.5) rectangle (3, 0.6); % Patch
    \draw [thick] (0, 0) -- (4, 0); % Ground plane (bottom line)
    \node [below] at (2, 0) {Ground Plane};
    \node at (2, 0.25) {Dielectric Substrate};
    \node [above] at (2, 0.6) {Radiating Patch};
    \draw [->] (1.5, -0.5) -- (1.5, 0.5) node [midway, right] {Feed};
\end{tikzpicture}
\end{answerdiagram}

\begin{itemize}
    \item \textbf{રચના}: ડાયલેક્ટ્રિક સબસ્ટ્રેટ પર ગ્રાઉન્ડ પ્લેન સાથે કન્ડક્ટિવ પેચ.
    \item \textbf{ફીડિંગ મેથડ્સ}: માઇક્રોસ્ટ્રીપ લાઇન, કોએક્સિયલ પ્રોબ, એપર્ચર-કપલ્ડ.
    \item \textbf{ફાયદા}: લો પ્રોફાઇલ, હળવા વજનના, સરળ ફેબ્રિકેશન, PCB સાથે સુસંગત.
    \item \textbf{મર્યાદાઓ}: સાંકડી બેન્ડવિડ્થ, ઓછો ગેઇન, ઓછી પાવર હેન્ડલિંગ.
    \item \textbf{ઉપયોગો}: મોબાઇલ ડિવાઇસ, RFID, GPS, સેટેલાઇટ કમ્યુનિકેશન્સ.
\end{itemize}
\end{solutionbox}

\begin{mnemonicbox}
\mnemonic{"PATCH" - Planar Antenna That's Cheaply Handled}
\end{mnemonicbox}

\orquestionmarks{3(a)}{3}{"LINE OF SIGHT" શબ્દ માટે મોર્સ કોડ લખો}

\begin{solutionbox}
\textbf{મોર્સ કોડ:}

\begin{tabulary}{\linewidth}{|L|L|L|L|}
\hline
\textbf{અક્ષર} & \textbf{મોર્સ કોડ} & \textbf{અક્ષર} & \textbf{મોર્સ કોડ} \\ \hline
L & .\,-.. & F & ..-. \\ \hline
I & .. & S & ... \\ \hline
N & -. & G & --. \\ \hline
E & . & H & .... \\ \hline
O & --- & T & - \\ \hline
\end{tabulary}

\textbf{"LINE OF SIGHT"}:
\code{.-.. .. -. . / --- ..-. / ... .. --. .... -}
\end{solutionbox}

\begin{mnemonicbox}
\mnemonic{"Listen In Now, Every Other Frequency Supports Immediate Global Heightened Transmission"}
\end{mnemonicbox}

\orquestionmarks{3(b)}{4}{આકૃતિની મદદથી ટર્નસ્ટાઇલ અને સુપર ટર્નસ્ટાઇલ એન્ટેના સમજાવો}

\begin{solutionbox}
\textbf{ટર્નસ્ટાઇલ એન્ટેના:}

\begin{answerdiagram}{Turnstile Antenna}
\begin{tikzpicture}
    \draw [thick] (-1.5, 0) -- (1.5, 0);
    \draw [thick] (0, -1.5) -- (0, 1.5);
    \node at (1.7, 0) {Dipole 1};
    \node at (0, 1.7) {Dipole 2};
    \node at (0,0) {Feed (90$^\circ$ phase)};
\end{tikzpicture}
\end{answerdiagram}

\textbf{સુપર ટર્નસ્ટાઇલ એન્ટેના:}
\begin{itemize}
    \item લંબચોરસ લૂપ્સ બનાવતા મલ્ટીપલ એલિમેન્ટ્સ સાથે સુધારો (Batwing shape).
\end{itemize}

\begin{tabulary}{\linewidth}{|L|L|}
\hline
\textbf{પ્રકાર} & \textbf{લક્ષણો} \\ \hline
\keyword{ટર્નસ્ટાઇલ} & કાટખૂણે બે આડા ડાયપોલ, $90^\circ$ ફેઝ શિફ્ટ સાથે ફીડ કરેલ \\ \hline
\keyword{સુપર ટર્નસ્ટાઇલ} & લંબચોરસ લૂપ્સ બનાવતા મલ્ટીપલ એલિમેન્ટ્સ સાથે સુધારો \\ \hline
\keyword{પેટર્ન} & આડા પ્લેનમાં સર્વવ્યાપી, ઊભા પ્લેનમાં ફિગર-8 \\ \hline
\keyword{પોલરાઇઝેશન} & આડું અથવા સર્ક્યુલર પોલરાઇઝેશન \\ \hline
\keyword{ઉપયોગો} & TV બ્રોડકાસ્ટિંગ, FM બ્રોડકાસ્ટિંગ, સેટેલાઇટ કમ્યુનિકેશન્સ \\ \hline
\end{tabulary}
\end{solutionbox}

\begin{mnemonicbox}
\mnemonic{"TOPS" - Turnstile Offers Perpendicular Symmetry}
\end{mnemonicbox}

\orquestionmarks{3(c)}{7}{પોલરાઇઝેશન શું છે? આકૃતિની મદદથી હેલીકલ એન્ટેના સમજાવો}

\begin{solutionbox}
\textbf{પોલરાઇઝેશન} એ અવકાશમાં પ્રસરણ કરતી વખતે ઇલેક્ટ્રોમેગ્નેટિક તરંગના ઇલેક્ટ્રિક ફિલ્ડ વેક્ટરનું અભિગમન છે.

\textbf{હેલીકલ એન્ટેના:}

\begin{answerdiagram}{Helical Antenna}
\begin{tikzpicture}[x=1cm, y=0.5cm]
    \draw [fill=gray!30] (0,-2) rectangle (0.5, 2); % Ground plane
    \node [below] at (0, -2) {Ground Plane};
    \draw [thick, decorate, decoration={coil, aspect=0.4, segment length=1cm, amplitude=1cm}] (0.5, 0) -- (5.5, 0);
    \draw [thick] (0.5, 0) -- (0.5, -1); % Coax feed
    \node [below] at (0.5, -1) {Coaxial Feed};
    \draw [->] (6, 0) -- (7, 0) node [right] {Axis of Radiation};
\end{tikzpicture}
\end{answerdiagram}

\begin{tabulary}{\linewidth}{|L|L|}
\hline
\textbf{પેરામીટર} & \textbf{વર્ણન} \\ \hline
\keyword{રચના} & ગ્રાઉન્ડ પ્લેન પર હેલિકલ આકારમાં વાયર વીંટાળેલો \\ \hline
\keyword{વ્યાસ} & સામાન્ય રીતે $\lambda/\pi$ \\ \hline
\keyword{પિચ} & વીંટાળા વચ્ચેનું અંતર, સામાન્ય રીતે $\lambda/4$ \\ \hline
\keyword{વીંટાળા} & ગેઇન જરૂરિયાતો આધારિત 3-10 વીંટાળા \\ \hline
\keyword{મોડ્સ} & નોર્મલ મોડ (બ્રોડસાઇડ) અથવા એક્સિયલ મોડ (એન્ડ-ફાયર) \\ \hline
\keyword{પોલરાઇઝેશન} & એક્સિયલ મોડમાં સર્ક્યુલર પોલરાઇઝેશન \\ \hline
\keyword{ઉપયોગો} & સેટેલાઇટ કમ્યુનિકેશન્સ, સ્પેસ ટેલિમેટ્રી, ટ્રેકિંગ \\ \hline
\end{tabulary}
\end{solutionbox}

\begin{mnemonicbox}
\mnemonic{"HASP" - Helical Antenna Supports Polarization}
\end{mnemonicbox}

\questionmarks{4(a)}{3}{ટ્રોપોસ્ફેરિક સ્કેટર્ડ પ્રોપોગેશન સમજાવો}

\begin{solutionbox}
\begin{tabulary}{\linewidth}{|L|L|}
\hline
\textbf{પાસું} & \textbf{વર્ણન} \\ \hline
\keyword{મિકેનિઝમ} & રેડિયો સિગ્નલ્સ ટ્રોપોસ્ફિયરિક અનિયમિતતાઓ અને રિફ્રેક્ટિવ ઇન્ડેક્સ વેરિએશન્સથી વિખેરાય છે \\ \hline
\keyword{ફ્રિક્વન્સી} & સામાન્ય રીતે VHF, UHF (100 MHz - 10 GHz) \\ \hline
\keyword{રેન્જ} & 100-800 km, લાઇન-ઓફ-સાઇટથી આગળ \\ \hline
\keyword{વિશ્વસનીયતા} & લાઇન-ઓફ-સાઇટ કરતાં હવામાનથી ઓછી અસરગ્રસ્ત; આયનોસ્ફેરિક કરતાં વધુ વિશ્વસનીય \\ \hline
\keyword{ઉપયોગો} & મિલિટરી કમ્યુનિકેશન્સ, દૂરસ્થ વિસ્તારો જ્યાં અન્ય સિસ્ટમ્સ વ્યવહારિક નથી \\ \hline
\end{tabulary}
\end{solutionbox}

\begin{mnemonicbox}
\mnemonic{"STRIP" - Scatter Through Refractive Index Patterns}
\end{mnemonicbox}

\questionmarks{4(b)}{4}{વ્યાખ્યા લખો: (1) વર્ચ્યુઅલ હાઇટ (2) મેક્સિમમ યુઝેબલ ફ્રિક્વન્સી - MUF (3) ક્રિટિકલ ફ્રિક્વન્સી}

\begin{solutionbox}
\begin{tabulary}{\linewidth}{|L|L|}
\hline
\textbf{શબ્દ} & \textbf{વ્યાખ્યા} \\ \hline
\keyword{વર્ચ્યુઅલ હાઇટ} & આયનોસ્ફિયરનું આભાસી ઊંચાઈ જે પૃથ્વી પર પાછા પરાવર્તિત થયેલા રેડિયો સિગ્નલના સમય વિલંબથી ગણવામાં આવે છે, જાણે કે પરાવર્તન એક જ બિંદુએ થયું હોય \\ \hline
\keyword{MUF} & નિર્દિષ્ટ પાથ અને સમય માટે આયનોસ્ફિયરિક પરાવર્તન દ્વારા વિશ્વસનીય કમ્યુનિકેશન માટે ઉપયોગ કરી શકાય તેવી ઉચ્ચતમ ફ્રિક્વન્સી \\ \hline
\keyword{ક્રિટિકલ ફ્રિક્વન્સી} & ઊભી દિશામાં આયનોસ્ફિયર તરફ પ્રસારિત થયા પછી પાછી પરાવર્તિત થઈ શકે તેવી ઉચ્ચતમ ફ્રિક્વન્સી (જ્યારે આપાત કોણ $90^\circ$ હોય) \\ \hline
\end{tabulary}
\end{solutionbox}

\begin{mnemonicbox}
\mnemonic{"VMC" - Virtual height Measures Critical reflection}
\end{mnemonicbox}

\questionmarks{4(c)}{7}{ઇલેક્ટ્રો મેગ્નેટીક વેવ પર ગ્રાઉંડની અસરો સમજાવો}

\begin{solutionbox}
\begin{answerdiagram}{Ground Wave Propagation Effects}
\begin{tikzpicture}
    % Ground
    \draw [thick] (-3, 0) -- (3, 0);
    \node [below] at (0, 0) {Ground};
    
    % TX RX
    \draw [thick] (-2, 0) -- (-2, 1) node [above] {TX};
    \draw [thick] (2, 0) -- (2, 1) node [above] {RX};
    
    % Direct
    \draw [thick, ->] (-2, 1) -- (2, 1) node [midway, above] {Direct Wave};
    
    % Reflected
    \draw [thick, dashed, ->] (-2, 1) -- (0, 0) -- (2, 1);
    \node [below] at (0, 0) {Reflection Point};
    \node at (1, 0.5) {Ground Reflected};
\end{tikzpicture}
\end{answerdiagram}

\begin{tabulary}{\linewidth}{|L|L|}
\hline
\textbf{અસર} & \textbf{વર્ણન} \\ \hline
\keyword{ગ્રાઉન્ડ રિફ્લેક્શન} & સિગ્નલ ગ્રાઉન્ડ પરથી પરાવર્તિત થાય છે, જેનાથી મલ્ટીપાથ રિસેપ્શન થાય છે \\ \hline
\keyword{ગ્રાઉન્ડ એબ્સોર્પશન} & સિગ્નલ ઊર્જાનો એક ભાગ ભૂમિ દ્વારા શોષાય છે, જેથી સિગ્નલ શક્તિ ઘટે છે \\ \hline
\keyword{ગ્રાઉન્ડ ડિફ્રેક્શન} & તરંગો અવરોધોની આસપાસ વળે છે, લાઇન-ઓફ-સાઇટથી આગળ કવરેજ વધારે છે \\ \hline
\keyword{પૃથ્વીની વક્રતા} & એન્ટેનાની ઊંચાઈના આધારે લાઇન-ઓફ-સાઇટ અંતરને મર્યાદિત કરે છે \\ \hline
\keyword{ગ્રાઉન્ડ કન્ડક્ટિવિટી} & ઉચ્ચ કન્ડક્ટિવિટી (પાણી) નબળા કન્ડક્ટર્સ (સૂકા) કરતાં વધુ સારો પ્રસરણ મંજૂરી આપે છે \\ \hline
\end{tabulary}

\textbf{તરંગ વર્તન સમીકરણ:} $d \approx 4.12(\sqrt{h_t} + \sqrt{h_r})$ km.
\end{solutionbox}

\begin{mnemonicbox}
\mnemonic{"RADAR" - Reflection Absorption Diffraction Affect Range}
\end{mnemonicbox}

\orquestionmarks{4(a)}{3}{ડક્ટ પ્રોપોગેશન સમજાવો}

\begin{solutionbox}
ડક્ટ પ્રોપોગેશન ત્યારે થાય છે જ્યારે રેડિયો તરંગો વિશેષ રિફ્રેક્ટિવ ગુણધર્મો સાથેના વાતાવરણીય સ્તરોમાં ફસાઈ જાય છે.

\begin{answerdiagram}{Atmospheric Duct}
\begin{tikzpicture}
    \draw [dashed] (0, 2) -- (6, 2);
    \draw [dashed] (0, 1) -- (6, 1);
    \node at (3, 1.5) {Duct Channel (Temperature Inversion)};
    \draw [decorate, decoration={coil, aspect=0, segment length=5pt, amplitude=5pt}] (1, 1.5) -- (5, 1.5);
    \node [left] at (1, 1.5) {TX};
    \node [right] at (5, 1.5) {RX};
    \node at (3, 0.5) {Normal Atmosphere};
    \node at (3, 2.5) {Normal Atmosphere};
\end{tikzpicture}
\end{answerdiagram}

\begin{itemize}
    \item \textbf{ફોર્મેશન}: તાપમાન વિપરીતતા અથવા ભેજ ગ્રેડિયન્ટ વાતાવરણીય ડક્ટ બનાવે છે.
    \item \textbf{અસર}: સિગ્નલ્સ ડક્ટની અંદર ફસાય છે, સામાન્ય રેન્જથી ઘણી દૂર સુધી પ્રસરણની મંજૂરી આપે છે.
    \item \textbf{ફ્રિક્વન્સી}: UHF અને માઇક્રોવેવ બેન્ડમાં સૌથી સામાન્ય.
\end{itemize}
\end{solutionbox}

\begin{mnemonicbox}
\mnemonic{"TIDE" - Trapped In Ducting Environment}
\end{mnemonicbox}

\orquestionmarks{4(b)}{4}{આઇનોસ્ફીયર ના જુદા જુદા સ્તરો સમજાવો}

\begin{solutionbox}
\begin{tabulary}{\linewidth}{|L|L|L|}
\hline
\textbf{સ્તર} & \textbf{ઊંચાઈ} & \textbf{લક્ષણો} \\ \hline
\textbf{D સ્તર} & 60-90 km & દિવસના સમયે HF તરંગોને શોષે છે, રાત્રે ગાયબ થઈ જાય છે \\ \hline
\textbf{E સ્તર} & 90-150 km & 10 MHz સુધીની આવૃત્તિઓને પરાવર્તિત કરે છે, સ્પોરેડિક E ઘટના \\ \hline
\textbf{F1 સ્તર} & 150-210 km & દિવસ દરમિયાન હાજર, રાત્રે F2 સાથે ભળી જાય છે \\ \hline
\textbf{F2 સ્તર} & 210-400+ km & મુખ્ય પરાવર્તન સ્તર, ઉચ્ચતમ ઇલેક્ટ્રોન ઘનતા, દિવસ અને રાત હાજર \\ \hline
\end{tabulary}
\end{solutionbox}

\begin{mnemonicbox}
\mnemonic{"DEAF" - D absorbs, E reflects, All merge, F2 persists}
\end{mnemonicbox}

\orquestionmarks{4(c)}{7}{ગ્રાઉંડ વેવ અને સ્કાય વેવ પ્રોપોગેશન સમજાવો}

\begin{solutionbox}
\textbf{ગ્રાઉન્ડ વેવ પ્રોપોગેશન:}
\begin{itemize}
    \item \textbf{ફ્રિક્વન્સી રેન્જ}: LF, MF (30 kHz - 3 MHz).
    \item \textbf{ઘટકો}: ડાયરેક્ટ, ગ્રાઉન્ડ-રિફ્લેક્ટેડ, સરફેસ વેવ્સ.
    \item \textbf{ઉપયોગો}: AM બ્રોડકાસ્ટિંગ, મેરીટાઇમ કમ્યુનિકેશન્સ.
\end{itemize}

\textbf{સ્કાય વેવ પ્રોપોગેશન:}
\begin{itemize}
    \item \textbf{ફ્રિક્વન્સી રેન્જ}: HF (3-30 MHz).
    \item \textbf{મિકેનિઝમ}: આયનોસ્ફિયર દ્વારા તરંગો પૃથ્વી પર પાછા વળે છે.
    \item \textbf{ઉપયોગો}: આંતરરાષ્ટ્રીય પ્રસારણ, એમેચ્યોર રેડિયો.
\end{itemize}

\begin{answerdiagram}{Ground vs Sky Wave}
\begin{tikzpicture}
    % Earth
    \draw [thick] (-3, 0) arc (170:10:3) node [midway, below] {Earth};
    % Ionosphere
    \draw [dashed] (-3, 3) arc (170:10:3) node [midway, above] {Ionosphere};
    
    % TX RX
    \coordinate (TX) at (-2, 0.5);
    \coordinate (RX) at (2, 0.5);
    \draw [thick] (TX) -- +(0, 0.2);
    \draw [thick] (RX) -- +(0, 0.2);
    
    % Sky wave
    \draw [->, thick, blue] (TX) -- (0, 2.5) -- (RX) node [midway, above] {Sky Wave};
    
    % Ground wave
    \draw [->, thick, red] (TX) to [bend left=10] (RX);
    \node [below, red] at (0, 0.8) {Ground Wave};
\end{tikzpicture}
\end{answerdiagram}
\end{solutionbox}

\begin{mnemonicbox}
\mnemonic{"GIST" - Ground-Interface Surface Transmission vs Ionospheric Sky Transmission}
\end{mnemonicbox}

\questionmarks{5(a)}{3}{ત્રણ જુદી જુદી જાતના ઉપગ્રહો સમજાવો}

\begin{solutionbox}
\begin{tabulary}{\linewidth}{|L|L|}
\hline
\textbf{ઉપગ્રહ પ્રકાર} & \textbf{લક્ષણો} \\ \hline
\keyword{LEO} (લો અર્થ ઓર્બિટ) & ઊંચાઈ: 160-2,000 km, અવધિ: ~90 મિનિટ, ઉપયોગો: પૃથ્વી નિરીક્ષણ, કમ્યુનિકેશન્સ \\ \hline
\keyword{MEO} (મીડિયમ અર્થ ઓર્બિટ) & ઊંચાઈ: 2,000-35,786 km, અવધિ: 2-24 કલાક, ઉપયોગો: નેવિગેશન (GPS) \\ \hline
\keyword{GEO} (જિઓસ્ટેશનરી ઓર્બિટ) & ઊંચાઈ: 35,786 km, અવધિ: 24 કલાક, ઉપયોગો: TV બ્રોડકાસ્ટિંગ, હવામાન નિરીક્ષણ \\ \hline
\end{tabulary}
\end{solutionbox}

\begin{mnemonicbox}
\mnemonic{"LMG" - Low Medium Geostationary}
\end{mnemonicbox}

\questionmarks{5(b)}{4}{સ્માર્ટ એન્ટેના શું છે? તેના બે ઉપયોગો જણાવો}

\begin{solutionbox}
સ્માર્ટ એન્ટેના એવી એન્ટેના સિસ્ટમ છે જે સ્પેશિયલ સિગ્નેચર્સને ઓળખવા અને ડાયનેમિકલી રેડિએશન પેટર્ન એડજસ્ટ કરવા માટે ડિજિટલ સિગ્નલ પ્રોસેસિંગ એલ્ગોરિધમનો ઉપયોગ કરે છે.

\begin{tabulary}{\linewidth}{|L|L|}
\hline
\textbf{ફીચર} & \textbf{વર્ણન} \\ \hline
\keyword{પ્રકારો} & સ્વિચ્ડ બીમ સિસ્ટમ્સ, એડેપ્ટિવ એરે સિસ્ટમ્સ \\ \hline
\keyword{ઓપરેશન} & બદલાતી પરિસ્થિતિઓને અનુકૂળ થવા માટે મલ્ટીપલ એન્ટેના એલિમેન્ટ્સ અને સિગ્નલ પ્રોસેસિંગનો ઉપયોગ કરે છે \\ \hline
\keyword{લાભો} & ક્ષમતા વધારી, કવરેજમાં સુધારો, દખલમાં ઘટાડો \\ \hline
\end{tabulary}

\textbf{ઉપયોગો:}
1. મોબાઇલ સેલ્યુલર નેટવર્ક્સ (4G, 5G).
2. સુધારેલા થ્રૂપુટ માટે વાયરલેસ LAN.
\end{solutionbox}

\begin{mnemonicbox}
\mnemonic{"SMART" - Signal Manipulation And Response Technology}
\end{mnemonicbox}

\questionmarks{5(c)}{7}{ઉપગ્રહ આધારિત સંદેશા વ્યવહાર શું છે? ડેટા કમ્યુનિકેશન વિશે સમજાવો.}

\begin{solutionbox}
\textbf{સેટેલાઇટ કમ્યુનિકેશન} એ પૃથ્વી પરના વિવિધ બિંદુઓ વચ્ચે કમ્યુનિકેશન લિંક્સ પ્રદાન કરવા માટે કૃત્રિમ ઉપગ્રહોનો ઉપયોગ છે.

\begin{answerdiagram}{Satellite Link}
\begin{tikzpicture}
    % Earth Stations
    \node [gtu block] (tx) {TX Earth Station};
    \node [gtu block, right=of tx, xshift=2cm] (rx) {RX Earth Station};
    
    % Satellite
    \node [draw, circle, minimum size=1.5cm, above=of tx, xshift=2.5cm, yshift=1cm] (sat) {Satellite};
    
    % Signals
    \draw [->, thick] (tx) -- node [left] {Uplink (6 GHz)} (sat);
    \draw [->, thick] (sat) -- node [right] {Downlink (4 GHz)} (rx);
\end{tikzpicture}
\end{answerdiagram}

\textbf{ઉપગ્રહ દ્વારા ડેટા કમ્યુનિકેશન:}
\begin{tabulary}{\linewidth}{|L|L|}
\hline
\textbf{ઘટક} & \textbf{કાર્ય} \\ \hline
\keyword{અર્થ સ્ટેશન} & ઉપગ્રહોને/થી સિગ્નલ્સ ટ્રાન્સમિટ/રિસીવ કરે છે \\ \hline
\keyword{ટ્રાન્સપોન્ડર} & અલગ-અલગ આવૃત્તિઓ પર સિગ્નલ્સ પ્રાપ્ત કરે છે, એમ્પલિફાય કરે છે અને ફરીથી પ્રસારિત કરે છે \\ \hline
\keyword{એક્સેસ મેથડ્સ} & FDMA, TDMA, CDMA મલ્ટિપલ યુઝર્સને ઉપગ્રહ ક્ષમતા શેર કરવાની મંજૂરી આપે છે \\ \hline
\keyword{ઉપયોગો} & ઇન્ટરનેટ બેકહોલ, VSAT નેટવર્ક્સ, IoT \\ \hline
\keyword{ફાયદા} & વિશાળ કવરેજ વિસ્તાર, ટેરેસ્ટ્રિયલ ઇન્ફ્રાસ્ટ્રક્ચરથી સ્વતંત્રતા \\ \hline
\end{tabulary}
\end{solutionbox}

\begin{mnemonicbox}
\mnemonic{"UPDATA" - Uplink Provides Data Access To All}
\end{mnemonicbox}

\orquestionmarks{5(a)}{3}{કેપલરના ઉપગ્રહ વિશેના નિયમો લખો}

\begin{solutionbox}
\begin{tabulary}{\linewidth}{|L|L|}
\hline
\textbf{કેપલરના નિયમો} & \textbf{વર્ણન} \\ \hline
\textbf{પ્રથમ નિયમ} & ઉપગ્રહો ઇલિપ્ટિકલ પાથમાં ભ્રમણ કરે છે જેમાં પૃથ્વી એલિપ્સના એક ફોકસ પર હોય છે \\ \hline
\textbf{બીજો નિયમ} & ઉપગ્રહ અને પૃથ્વીને જોડતી રેખા સમાન સમયમાં સમાન ક્ષેત્રફળ પસાર કરે છે (એન્ગ્યુલર મોમેન્ટમ સંરક્ષણ) \\ \hline
\textbf{ત્રીજો નિયમ} & કક્ષીય અવધિનો વર્ગ કક્ષાના અર્ધ-મેજર અક્ષના ઘનફળના સમપ્રમાણમાં હોય છે ($T^2 \propto a^3$) \\ \hline
\end{tabulary}
\end{solutionbox}

\begin{mnemonicbox}
\mnemonic{"ESP" - Elliptical orbits, Sweep equal areas, Period-distance relation}
\end{mnemonicbox}

\orquestionmarks{5(b)}{4}{બેઝ સ્ટેશન અને મોબાઇલ સ્ટેશન એન્ટેના વિશે સમજાવો}

\begin{solutionbox}
\textbf{બેઝ સ્ટેશન એન્ટેના:}
\begin{itemize}
    \item \textbf{પ્રકારો}: ઓમ્નિડાયરેક્શનલ, સેક્ટર, પેનલ એન્ટેના.
    \item \textbf{ગેઇન}: સામાન્ય રીતે 10-18 dBi.
    \item \textbf{માઉન્ટિંગ}: ટાવર અથવા છત પર ઇન્સ્ટોલેશન.
\end{itemize}

\textbf{મોબાઇલ સ્ટેશન એન્ટેના:}
\begin{itemize}
    \item \textbf{પ્રકારો}: ઇન્ટરનલ PIFA, પેચ, મોનોપોલ એન્ટેના.
    \item \textbf{ગેઇન}: લો ગેઇન (0-3 dBi).
    \item \textbf{સાઇઝ}: કોમ્પેક્ટ, ઘણી વખત ડિવાઇસની અંદર એકીકૃત.
\end{itemize}

\begin{answerdiagram}{Base vs Mobile}
\begin{tikzpicture}
    % Base
    \draw [thick] (0,0) -- (0, 2);
    \draw [thick] (-0.2, 2) -- (0.2, 2) -- (0, 3) -- (-0.2, 2); % Tower top
    \node [below] at (0,0) {Base Station};
    
    % Mobile
    \draw [thick, rounded corners] (3, 0) rectangle (3.5, 1);
    \draw [thick] (3.25, 1) -- (3.25, 1.3);
    \node [below] at (3.25, 0) {Mobile};
    
    \draw [dashed, <->] (0.2, 2.5) -- (3.25, 1.3);
    \node at (1.7, 2) {Wireless Link};
\end{tikzpicture}
\end{answerdiagram}
\end{solutionbox}

\begin{mnemonicbox}
\mnemonic{"BIMS" - Base stations Install Multiple Sectors, Mobile stations Stay small}
\end{mnemonicbox}

\orquestionmarks{5(c)}{7}{DTH રીસીવર સિસ્ટમ વિસ્તારથી સમજાવો}

\begin{solutionbox}
\textbf{DTH (ડાયરેક્ટ-ટુ-હોમ)} રિસીવર સિસ્ટમ ઉપગ્રહ દ્વારા સીધા વપરાશકર્તાઓને ટેલિવિઝન સિગ્નલ્સ પહોંચાડે છે.

\begin{answerdiagram}{DTH System}
\begin{tikzpicture}[node distance=1.5cm]
    \node [draw, circle] (sat) {Satellite};
    \node [gtu block, below=of sat] (dish) {Dish Antenna};
    \node [gtu block, below=of dish] (lnb) {LNB};
    \node [gtu block, right=of lnb] (stb) {Set-top Box};
    \node [gtu block, right=of stb] (tv) {TV};
    
    \draw [->, dashed] (sat) -- (dish);
    \draw [gtu arrow] (dish) -- (lnb);
    \draw [gtu arrow] (lnb) -- node [above] {Cable} (stb);
    \draw [gtu arrow] (stb) -- (tv);
\end{tikzpicture}
\end{answerdiagram}

\begin{tabulary}{\linewidth}{|L|L|}
\hline
\textbf{ઘટક} & \textbf{કાર્ય} \\ \hline
\keyword{ડિશ એન્ટેના} & ઉપગ્રહ સિગ્નલ્સ એકત્રિત કરવા માટે પેરાબોલિક રિફ્લેક્ટર (45-90 cm) \\ \hline
\keyword{LNB} & લો નોઈઝ બ્લોક ડાઉનકન્વર્ટર; સિગ્નલ એમ્પલિફાય કરે છે અને ફ્રિકવન્સી ઘટાડે છે \\ \hline
\keyword{સેટ-ટોપ બોક્સ} & ડિજિટલ સિગ્નલોને ડીકોડ કરે છે અને TV માટે ઓડિયો/વિડિયોમાં રૂપાંતરિત કરે છે \\ \hline
\keyword{TV} & ડિસ્પ્લે યુનિટ \\ \hline
\end{tabulary}
\end{solutionbox}

\end{document}
