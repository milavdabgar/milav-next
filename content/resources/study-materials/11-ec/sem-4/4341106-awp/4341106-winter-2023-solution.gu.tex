\documentclass{article}

% content/resources/templates/preamble.tex
\usepackage[margin=0.6in]{geometry}
\author{Milav Dabgar}
\usepackage{amsmath,amssymb,amsthm}
\usepackage{booktabs}
\usepackage{multirow}
\usepackage{xcolor}
\usepackage{tcolorbox}
\tcbuselibrary{breakable,skins}
\usepackage[colorlinks=true,linkcolor=blue]{hyperref}
\usepackage{titlesec}
\usepackage{enumitem}
\usepackage{tikz}
\usepackage{pgfplots}
\usepackage{circuitikz}
\usepackage[version=4]{mhchem}
\usepackage{longtable}
\usepackage{array}
\usepackage{float}
\usepackage{caption}
\usepackage{listings}

\lstset{
  basicstyle=\small\ttfamily,
  breaklines=true,
  breakatwhitespace=false,
  postbreak=\mbox{\textcolor{red}{$\hookrightarrow$}\space},
  float=false,
  numbers=left,
  numberstyle=\tiny\color{gray},
  numbersep=10pt,
  xleftmargin=2em,
  keywordstyle=\color{blue},
  commentstyle=\color{green!60!black},
  stringstyle=\color{purple},
  backgroundcolor=\color{gray!5},
  showstringspaces=false,
  tabsize=2,
  captionpos=b,
  keepspaces=true,
  columns=flexible
}

\pgfplotsset{compat=1.18}
\usetikzlibrary{shapes,arrows,positioning,calc,patterns,decorations.pathmorphing,decorations.markings,arrows.meta}

% Color scheme
\definecolor{headcolor}{RGB}{0,102,204}
\definecolor{keycolor}{RGB}{220,20,60}
\definecolor{solutioncolor}{RGB}{34,139,34}
\definecolor{mnemoniccolor}{RGB}{148,0,211}
\definecolor{codecolor}{RGB}{0,0,100}

% Spacing
\setlength{\parskip}{3pt}
\setlist[itemize]{nosep}
\setlist[enumerate]{nosep}

% Title formatting
\titleformat{\section}{\Large\bfseries\color{headcolor}}{\thesection}{1em}{}
\titleformat{\subsection}{\large\bfseries\color{headcolor}}{\thesubsection}{1em}{}

% Pandoc tightlist compatibility
\providecommand{\tightlist}{%
  \setlength{\itemsep}{0pt}\setlength{\parskip}{0pt}}

% Pandoc longtable compatibility
\newcounter{none}
\def\thenone{}


% content/resources/templates/gujarati-boxes.tex
\usepackage{fontspec}
\usepackage{polyglossia}

% Set Gujarati as main language (document is primarily in Gujarati)
% Note: gloss-gujarati.ldf doesn't exist in polyglossia, but it will use hyphenation patterns
\setdefaultlanguage{gujarati}
\setotherlanguage{english}

% Configure Gujarati font properly
% Use Language=Default to prevent polyglossia from trying to add language-specific features
% that don't exist for Gujarati, which causes "empty feature" warnings
\newfontfamily\gujaratifont[Script=Gujarati,AutoFakeBold=2.5,AutoFakeSlant=0.3]{Noto Sans Gujarati}
\setmainfont[Script=Gujarati,AutoFakeBold=2.5,AutoFakeSlant=0.3]{Noto Sans Gujarati}
% Use Noto Sans Gujarati for monospace to support Gujarati in text
\setmonofont[Scale=0.9]{Noto Sans Gujarati}

% Configure English to use the same font
\newfontfamily\englishfont[Script=Gujarati,AutoFakeBold=2.5,AutoFakeSlant=0.3]{Noto Sans Gujarati}

% Translations for polyglossia
\gappto\captionsgujarati{
  \renewcommand{\tablename}{કોષ્ટક}
  \renewcommand{\figurename}{આકૃતિ}
}

% Helper for TikZ nodes to ensure Gujarati font
\newcommand{\gu}[1]{{\gujaratifont #1}}

% Custom environments
\newtcolorbox{solutionbox}{
    breakable,
    enhanced,
    colback=solutioncolor!5!white,
    colframe=solutioncolor!75!black,
    fonttitle=\bfseries,
    title=જવાબ
}

\newtcolorbox{solutionboxnobreak}{
 colback=solutioncolor!5!white,
 colframe=solutioncolor!75!black,
 fonttitle=\bfseries,
 title=જવાબ
}

\newtcolorbox{keyformula}{
 breakable,
 enhanced,
 colback=keycolor!5!white,
 colframe=keycolor!75!black,
 fonttitle=\bfseries,
 title=રાસાયણિક સમીકરણ/સૂત્ર
}

\newtcolorbox{mnemonicbox}{
 breakable,
 enhanced,
 colback=mnemoniccolor!5!white,
 colframe=mnemoniccolor!75!black,
 fonttitle=\bfseries,
 title=મેમરી ટ્રીક
}


% Custom commands for GTU solutions
% This file defines semantic commands for consistent formatting

% Question command with automatic formatting
\newcommand{\question}[2]{%
  \section*{Question #1}%
  \textbf{#2}%
}

% OR question variant
\newcommand{\questionor}[2]{%
  \section*{Question #1 OR}%
  \textbf{#2}%
}

% Proper table environment with caption
\newenvironment{answertable}[1]{%
  \begin{table}[htbp]
  \centering
  \caption{#1}
}{%
  \end{table}
}

% Proper figure environment for diagrams
\newenvironment{answerdiagram}[1]{%
  \begin{figure}[htbp]
  \centering
  \caption{#1}
}{%
  \end{figure}
}

% Semantic markup for key terms
\newcommand{\keyword}[1]{\textbf{#1}}
\newcommand{\code}[1]{\texttt{#1}}
\newcommand{\classname}[1]{\texttt{#1}}
\newcommand{\methodname}[1]{\texttt{#1}}

% Proper quotation marks
\newcommand{\mnemonic}[1]{``#1''}


\title{એન્ટેના અને વેવ પ્રોપેગેશન (4341106) - વિન્ટર 2023 સોલ્યુશન}
\date{24 જાન્યુઆરી, 2024}

\begin{document}
\maketitle

\questionmarks{1(અ)}{3}{વ્યાખ્યાયિત કરો: (1) ડાયરેક્ટિવિટી, (2) ગેઇન, અને (3) HPBW}

\begin{solutionbox}
\begin{tabulary}{\linewidth}{|L|L|}
\hline
\textbf{પરિમાણ} & \textbf{વ્યાખ્યા} \\ \hline
\keyword{ડાયરેક્ટિવિટી} & અન્ટેનાની મહત્તમ વિકિરણ તીવ્રતા અને સરેરાશ વિકિરણ તીવ્રતાનો ગુણોત્તર. \\ \hline
\keyword{ગેઇન} & ચોક્કસ દિશામાં વિકિરિત થતી શક્તિ અને જે પાવર આઇસોટ્રોપિક અન્ટેના દ્વારા વિકિરિત થાય છે તેનો ગુણોત્તર. \\ \hline
\keyword{HPBW (હાફ પાવર બીમ વિડ્થ)} & કોણીય પહોળાઈ જ્યાં વિકિરણની તીવ્રતા મહત્તમ મૂલ્યના અડધા (3dB ઓછી) હોય છે. \\ \hline
\end{tabulary}
\end{solutionbox}

\begin{mnemonicbox}
\mnemonic{"DGH: દિશા ગેઇન હાફ-પાવર"}
\end{mnemonicbox}

\questionmarks{1(બ)}{4}{ઇલેક્ટ્રોમેગ્નેટિક તરંગોના ગુણધર્મોની સૂચિ બનાવો}

\begin{solutionbox}
\begin{tabulary}{\linewidth}{|L|L|}
\hline
\textbf{ગુણધર્મ} & \textbf{વર્ણન} \\ \hline
\keyword{ટ્રાન્સવર્સ તરંગો} & ઇલેક્ટ્રિક અને મેગ્નેટિક ક્ષેત્રો પ્રસરણની દિશાને લંબરૂપે હોય છે. \\ \hline
\keyword{વેગ} & નિર્વાતમાં પ્રકાશનો વેગ ($3 \times 10^8$ m/s). \\ \hline
\keyword{માધ્યમની જરૂર નથી} & યાંત્રિક તરંગોથી વિપરીત, નિર્વાતમાં પણ પ્રવાસ કરી શકે છે. \\ \hline
\keyword{ધ્રુવીકરણ} & ઇલેક્ટ્રિક ક્ષેત્ર વેક્ટરની દિશા દ્વારા વ્યાખ્યાયિત થાય છે. \\ \hline
\keyword{ઊર્જા વહન} & અવકાશમાં ઊર્જા વહન કરે છે. \\ \hline
\keyword{પરાવર્તન/વક્રીભવન} & સીમાઓ પર પરાવર્તિત અને વક્રીભૂત થઈ શકે છે. \\ \hline
\keyword{વ્યતિકરણ/વિવર્તન} & તરંગ જેવા ગુણધર્મો દર્શાવે છે. \\ \hline
\end{tabulary}
\end{solutionbox}

\begin{mnemonicbox}
\mnemonic{"TVNPER: ટ્રાન્સવર્સ વેગ નો-માધ્યમ પોલરાઇઝ્ડ એનર્જી રિફ્લેક્શન"}
\end{mnemonicbox}

\questionmarks{1(ક)}{7}{ઈલેક્ટ્રોમેગ્નેટિક તરંગોના નિર્માણનો ભૌતિક ખ્યાલ સમજાવો}

\begin{solutionbox}
\textbf{ખ્યાલ}: ઇલેક્ટ્રોમેગ્નેટિક તરંગો ત્વરિત ઇલેક્ટ્રિક ચાર્જ દ્વારા ઉત્પન્ન થાય છે.

\begin{answerdiagram}{EM તરંગોનું નિર્માણ}
\begin{tikzpicture}[node distance=1.5cm, auto]
    \node [gtu block] (charge) {ત્વરિત આવેશ};
    \node [gtu block, right=of charge] (efield) {સમય-ભિન્ન ઇલેક્ટ્રિક ક્ષેત્ર};
    \node [gtu block, right=of efield] (hfield) {સમય-ભિન્ન ચુંબકીય ક્ષેત્ર};
    \node [gtu block, below=of efield] (wave) {સ્વ-પ્રસરણશીલ EM તરંગ};

    \draw [gtu arrow] (charge) -- node {ઉત્પન્ન કરે છે} (efield);
    \draw [gtu arrow] (efield) -- node {ઉત્પન્ન કરે છે} (hfield);
    \draw [gtu arrow] (hfield) -- (efield);
    \draw [gtu arrow] (efield) -- (wave);
    \draw [gtu arrow] (hfield) -- (wave);
\end{tikzpicture}
\end{answerdiagram}

\begin{itemize}
    \item \textbf{આવેશનું ત્વરણ}: જ્યારે ઇલેક્ટ્રિક આવેશો ત્વરિત થાય છે (દા.ત. AC સર્કિટમાં), ત્યારે તેઓ બદલાતા ઇલેક્ટ્રિક ક્ષેત્રો ઉત્પન્ન કરે છે.
    \item \textbf{ક્ષેત્ર જોડાણ}: મેક્સવેલના સમીકરણો જણાવે છે કે બદલાતું ઇલેક્ટ્રિક ક્ષેત્ર ચુંબકીય ક્ષેત્ર ઉત્પન્ન કરે છે.
    \item \textbf{સ્વ-પ્રસરણ}: ક્ષેત્રોના આ ચક્રીય નિર્માણથી તરંગો સ્રોતથી અલગ થઈને અવકાશમાં પ્રવાસ કરે છે.
    \item \textbf{ક્ષેત્ર અભિમુખતા}: $E$ અને $H$ ક્ષેત્રો એકબીજાને અને પ્રસરણની દિશાને લંબરૂપ હોય છે.
\end{itemize}
\end{solutionbox}

\begin{mnemonicbox}
\mnemonic{"CASES: ચાર્જ એક્સેલરેશન સેલ્ફ-પ્રોપેગેટ્સ ઇલેક્ટ્રો-મેગ્નેટિક સિગ્નલ્સ"}
\end{mnemonicbox}

\orquestionmarks{1(ક)}{7}{સેન્ટર ફેડ ડાયપોલ માંથી ઇલેક્ટ્રોમેગ્નેટિક ક્ષેત્ર કેવી રીતે વિકિરણ થાય છે તે સમજાવો}

\begin{solutionbox}
\textbf{ડાયપોલનું રેડિએશન મિકેનિઝમ}:

\begin{answerdiagram}{ડાયપોલનું રેડિએશન}
\begin{tikzpicture}[node distance=1.5cm, auto]
    \node [gtu block] (input) {AC ઇનપુટ};
    \node [gtu block, right=of input] (charges) {આવર્તક આવેશો};
    \node [gtu block, right=of charges] (fields) {સમય-ભિન્ન E \& H ક્ષેત્રો};
    \node [gtu block, right=of fields] (rad) {EM વિકિરણ};

    \draw [gtu arrow] (input) -- (charges);
    \draw [gtu arrow] (charges) -- (fields);
    \draw [gtu arrow] (fields) -- (rad);

    % Dipole sketch
    \draw [thick] (0,-2) -- (0,-1) node[midway, left] {ઉપરનો હાથ};
    \draw [thick] (0,-3) -- (0,-4) node[midway, left] {નીચેનો હાથ};
    \node at (0,-2.5) {$\sim$};
    \node [right, blue] at (1,-2.5) {છૂટા પડતા લૂપ્સ};
\end{tikzpicture}
\end{answerdiagram}

\begin{itemize}
    \item \textbf{સેન્ટર ફીડિંગ}: કેન્દ્રમાં AC વોલ્ટેજ આપવામાં આવે છે, જેથી પ્રવાહ આગળ-પાછળ વહે છે.
    \item \textbf{આવેશ વિતરણ}: જેમે પ્રવાહ દોલન કરે છે, તેમ ડાયપોલના છેડા પર આવેશો જમા થાય છે.
    \item \textbf{ક્ષેત્ર નિર્માણ}: આવર્તક આવેશો ઇલેક્ટ્રિક ક્ષેત્ર બનાવે છે અને પ્રવાહ ચુંબકીય ક્ષેત્ર બનાવે છે.
    \item \textbf{વિકિરણ}: જ્યારે ધ્રુવીયતા બદલાય છે, ત્યારે ક્ષેત્રની રેખાઓ એન્ટેનાથી છૂટી પડીને બહાર તરફ જાય છે.
\end{itemize}
\end{solutionbox}

\begin{mnemonicbox}
\mnemonic{"CORONA: કરંટ ઓસિલેટ્સ, રેડિએશન ઓકર્સ, નીયર-ફાર એરિયાઝ"}
\end{mnemonicbox}

\questionmarks{2(અ)}{3}{રેઝોનન્ટ અને નોન-રેઝોનન્ટ એન્ટેનામાં તફાવત કરો}

\begin{solutionbox}
\begin{tabulary}{\linewidth}{|L|L|L|}
\hline
\textbf{લક્ષણ} & \textbf{રેઝોનન્ટ અન્ટેના} & \textbf{નોન-રેઝોનન્ટ અન્ટેના} \\ \hline
\keyword{લંબાઈ} & $\lambda/2$ નો પૂર્ણાંક ગુણાંક & તરંગલંબાઈ સાથે સીધો સંબંધ નથી \\ \hline
\keyword{સ્થાયી તરંગો} & હાજર & હાજર નથી (ટ્રાવેલિંગ વેવ્સ) \\ \hline
\keyword{પ્રતિબાધા} & વાસ્તવિક (Resistive) & જટિલ (Real + Imaginary) \\ \hline
\keyword{બેન્ડવિડ્થ} & સાંકડી & વિશાળ \\ \hline
\keyword{ઉદાહરણ} & અર્ધ-તરંગ ડાયપોલ & રોમ્બિક અન્ટેના \\ \hline
\end{tabulary}
\end{solutionbox}

\begin{mnemonicbox}
\mnemonic{"RESI: રેઝોનન્ટ એક્ઝિબિટ્સ સ્ટેન્ડિંગ-વેવ્સ ઇમ્પિડન્સ-રિયલ"}
\end{mnemonicbox}

\questionmarks{2(બ)}{4}{યાગી એન્ટેના સમજાવો અને તેની રેડિયેશન લાક્ષણિકતાઓની ચર્ચા કરો}

\begin{solutionbox}
\textbf{યાગી-ઉદા એન્ટેના}: ઉચ્ચ ગેઇન ધરાવતું દિશાત્મક એન્ટેના છે.

\begin{answerdiagram}{યાગી-ઉદા એન્ટેના}
\begin{tikzpicture}
    % Boom
    \draw [thick] (0,0) -- (6,0) node[right] {બૂમ (Boom)};
    
    % Reflector
    \draw [ultra thick, blue] (1, -1.5) -- (1, 1.5) node[above] {રીફ્લેક્ટર};
    
    % Driven Element
    \draw [ultra thick, red] (2, -1.2) -- (2, 1.2) node[above] {ડ્રાઈવન};
    \fill [white] (1.9, -0.1) rectangle (2.1, 0.1); 
    \draw (2, -0.1) -- (2, -0.5) node[below] {ફીડ};
    
    % Directors
    \draw [ultra thick, black] (3, -1.0) -- (3, 1.0);
    \draw [ultra thick, black] (4, -1.0) -- (4, 1.0);
    \draw [ultra thick, black] (5, -1.0) -- (5, 1.0) node[above] {ડાયરેક્ટર્સ};
    
    % Radiation Direction
    \draw [->, ultra thick, purple] (6.5, 0) -- (7.5, 0) node[right] {દિશા};
\end{tikzpicture}
\end{answerdiagram}

\begin{itemize}
    \item \textbf{સંરચના}: 1 રિફ્લેક્ટર, 1 ડ્રાઈવન એલિમેન્ટ, અનેક ડાયરેક્ટર્સ.
    \item \textbf{દિશાત્મકતા}: ડાયરેક્ટર્સની દિશામાં ઉચ્ચ (8-12 dB).
    \item \textbf{ગેઇન}: ડાયરેક્ટર્સની સંખ્યા વધવાથી ગેઇન વધે છે.
    \item \textbf{ઉપયોગ}: ટીવી રિસેપ્શન માટે.
\end{itemize}
\end{solutionbox}

\begin{mnemonicbox}
\mnemonic{"DRAGONS: ડાયરેક્શનલ રિફ્લેક્ટર એન્ડ ગેઇન-ઇમ્પ્રુવિંગ ડાયરેક્ટર્સ ઓફર નેરો સિગ્નલ્સ"}
\end{mnemonicbox}

\questionmarks{2(ક)}{7}{રેઝોનન્ટ વાયર એન્ટેનાની રેડિયેશન લાક્ષણિકતાઓનું વર્ણન કરો અને $\lambda/2$, $3\lambda/2$ અને $5\lambda/2$ એન્ટેનાનું કરંટ વિતરણ દોરો}

\begin{solutionbox}
રેઝોનન્ટ એન્ટેનામાં સ્ટેન્ડિંગ વેવ જોવા મળે છે.

\begin{answerdiagram}{કરંટ વિતરણ}
\begin{tikzpicture}[scale=0.8]
    % Lambda/2
    \node at (-2, 2) {$\lambda/2$};
    \draw [thick] (0, 2) -- (4, 2);
    \draw [thick, blue] (0, 2) sin (2, 3) cos (4, 2);
    \draw [thick, blue, dashed] (0, 2) sin (2, 1) cos (4, 2);
    \node [below] at (2, 1) {1 લૂપ};

    % 3 Lambda/2
    \node at (-2, -1) {$3\lambda/2$};
    \draw [thick] (0, -1) -- (6, -1);
    \draw [thick, blue] (0, -1) sin (1, 0) cos (2, -1) sin (3, -2) cos (4, -1) sin (5, 0) cos (6, -1);
    \node [below] at (3, -2) {3 લૂપ્સ};

    % 5 Lambda/2
    \node at (-2, -4) {$5\lambda/2$};
    \draw [thick] (0, -4) -- (8, -4);
    \foreach \x in {0, 1.6, 3.2, 4.8, 6.4} {
        \draw [thick, blue] (\x, -4) sin (\x+0.4, -3) cos (\x+0.8, -4) sin (\x+1.2, -5) cos (\x+1.6, -4);
    }
    \node [below] at (4, -5) {5 લૂપ્સ};
\end{tikzpicture}
\end{answerdiagram}

\begin{itemize}
    \item \textbf{હાફ-વેવ ($\lambda/2$)}: કેન્દ્રમાં પ્રવાહ મહત્તમ. પેટર્ન ફિગર-8 આકારની હોય છે.
    \item \textbf{$3\lambda/2$}: ત્રણ કરંટ લૂપ્સ. 6 લોબ્સ બને છે.
    \item \textbf{$5\lambda/2$}: પાંચ કરંટ લૂપ્સ. જેમ લંબાઈ વધે તેમ પેટર્ન વધુ જટિલ બને છે અને મુખ્ય લોબ્સ તારની નજીક સરકે છે.
\end{itemize}
\end{solutionbox}

\begin{mnemonicbox}
\mnemonic{"NODE: નંબર ઓફ ડિસ્ટ્રિબ્યુશન્સ ઇક્વલ્સ વેવલેન્થ-મલ્ટિપલ"}
\end{mnemonicbox}

\orquestionmarks{2(અ)}{3}{બ્રોડ સાઇડ અને એન્ડ ફાયર એરે એન્ટેનામાં તફાવત કરો}

\begin{solutionbox}
\begin{tabulary}{\linewidth}{|L|L|L|}
\hline
\textbf{લક્ષણ} & \textbf{બ્રોડસાઇડ એરે} & \textbf{એન્ડ ફાયર એરે} \\ \hline
\keyword{મહત્તમ વિકિરણ} & એરે અક્ષને લંબરૂપે ($90^\circ$) & એરે અક્ષની સાથે ($0^\circ, 180^\circ$) \\ \hline
\keyword{એલિમેન્ટ અંતર} & સામાન્ય રીતે $\lambda/2$ & સામાન્ય રીતે $\lambda/4$ \\ \hline
\keyword{ફેઝ તફાવત} & $0^\circ$ (સમાન ફેઝ) & $180^\circ$ (વિરુદ્ધ ફેઝ) \\ \hline
\keyword{પેટર્ન} & દ્વિદિશાત્મક & એકદિશાત્મક \\ \hline
\end{tabulary}
\end{solutionbox}

\begin{mnemonicbox}
\mnemonic{"PEPS: પરપેન્ડિક્યુલર એલિમેન્ટ્સ પ્રોડ્યુસ સાઇડવેઝ રેડિએશન"}
\end{mnemonicbox}

\orquestionmarks{2(બ)}{4}{લુપ એન્ટેના સમજાવો અને તેની રેડીયેસન લાક્ષણિકતાઓની ચર્ચા કરો}

\begin{solutionbox}
\textbf{લૂપ એન્ટેના}: એક બંધ વાહક લૂપ છે.

\begin{answerdiagram}{લૂપ એન્ટેના}
\begin{tikzpicture}
    % Small Loop
    \draw [ultra thick] (0,0) circle (1.5);
    \draw [fill=white] (-0.2, -1.5) rectangle (0.2, -1.3);
    \draw (0, -1.5) -- (0, -2) node[below] {ફીડ};
    \node at (0, 0) {નાની લૂપ ($C < \lambda/10$)};
    
    % Radiation Pattern (Figure 8)
    \draw [thick, blue] (5, 0) .. controls (6, 1) and (7, 1) .. (7, 0) .. controls (7, -1) and (6, -1) .. (5, 0);
    \draw [thick, blue] (5, 0) .. controls (4, 1) and (3, 1) .. (3, 0) .. controls (3, -1) and (4, -1) .. (5, 0);
    \node at (5, -1.5) {પેટર્ન (સમતલમાં)};
\end{tikzpicture}
\end{answerdiagram}

\begin{itemize}
    \item \textbf{નાની લૂપ}: મેગ્નેટિક ડાયપોલ તરીકે વર્તે છે. પેટર્ન ફિગર-8 જેવી હોય છે.
    \item \textbf{મોટી લૂપ ($C \approx \lambda$)}: રેઝોનન્ટ લૂપ. લૂપના સમતલને લંબરૂપે મહત્તમ રેડિએશન આપે છે.
    \item \textbf{ઉપયોગ}: દિશા શોધવા (Direction Finding) માટે.
\end{itemize}
\end{solutionbox}

\begin{mnemonicbox}
\mnemonic{"SPIRAL: સ્મોલ પેટર્ન્સ ઇન રિસીવિંગ એન્ડ લોકેટિંગ સિગ્નલ્સ"}
\end{mnemonicbox}

\orquestionmarks{2(ક)}{7}{નોન રેઝોનન્ટ વાયર એન્ટેનાની રેડિયેશન લાક્ષણિકતાઓનું વર્ણન કરો અને $\lambda/2$, $3\lambda/2$ અને $5\lambda/2$ એન્ટેનાની રેડિયેશન પેટર્ન દોરો}

\begin{solutionbox}
પ્રશ્નના સંદર્ભમાં, સામાન્ય રેઝોનન્ટ પેટર્ન દોરવામાં આવી છે કારણ કે લંબાઈ $\lambda/2$ ના ગુણાંકમાં છે.

\begin{answerdiagram}{વિકિરણ પેટર્ન}
\begin{tikzpicture}[scale=0.8]
    % Lambda/2
    \begin{scope}[shift={(0,0)}]
        \node at (0, 2) {$\lambda/2$};
        \draw [thick] (-1.5, 0) -- (1.5, 0);
        \draw [thick, red] (0,0) ellipse (0.5 and 1.5);
        \node [below] at (0, -2) {ફિગર-8};
    \end{scope}

    % 3 Lambda/2
    \begin{scope}[shift={(5,0)}]
        \node at (0, 2) {$3\lambda/2$};
        \draw [thick] (-1.5, 0) -- (1.5, 0);
        \foreach \ang in {45, 135, 225, 315} {
            \draw [thick, red, rotate=\ang] (0,0) ellipse (0.4 and 1.2);
        }
        \node [below] at (0, -2) {4 મુખ્ય લોબ્સ};
    \end{scope}

    % 5 Lambda/2
    \begin{scope}[shift={(10,0)}]
        \node at (0, 2) {$5\lambda/2$};
        \draw [thick] (-1.5, 0) -- (1.5, 0);
        \foreach \ang in {30, 150, 210, 330} {
             \draw [thick, red, rotate=\ang] (0,0) ellipse (0.3 and 1.5);
        }
        \node [below] at (0, -2) {તાર તરફ નમેલા લોબ્સ};
    \end{scope}
\end{tikzpicture}
\end{answerdiagram}

\begin{itemize}
    \item જેમ એન્ટેનાની લંબાઈ વધે છે, તેમ મુખ્ય બીમ સાંકડી થાય છે અને તારની અક્ષની નજીક આવે છે.
    \item ગૌણ લોબ્સ (Minor Lobes) ની સંખ્યા પણ વધે છે.
\end{itemize}
\end{solutionbox}

\begin{mnemonicbox}
\mnemonic{"TWIST: ટ્રાવેલિંગ વેવ્સ ઇન્ક્રીઝ સાઇડ-લોબ ટ્રાન્સમિશન"}
\end{mnemonicbox}


\questionmarks{3(અ)}{3}{માઇક્રો સ્ટ્રીપ (પેચ) એન્ટેના પર ટૂંકી નોંધ લખો}

\begin{solutionbox}
\textbf{માઇક્રોસ્ટ્રિપ (પેચ) એન્ટેના}: આધુનિક ઉપયોગો માટે લો-પ્રોફાઇલ એન્ટેના.

\begin{answerdiagram}{માઇક્રોસ્ટ્રિપ પેચ સંરચના}
\begin{tikzpicture}
    % Substrate
    \fill [gray!20] (0,0) rectangle (4, 1);
    \draw [thick] (0,0) rectangle (4, 1);
    \node [right] at (4, 0.5) {સબસ્ટ્રેટ ($\epsilon_r$)};

    % Ground Plane
    \draw [ultra thick] (0,0) -- (4, 0);
    \node [below] at (2,0) {ગ્રાઉન્ડ પ્લેન};

    % Patch
    \fill [orange!50] (1, 1) rectangle (3, 1.1);
    \draw [thick] (1, 1) rectangle (3, 1.1);
    \node [above] at (2, 1.1) {રેડિએટિંગ પેચ};
    
    % Feed
    \draw [thick] (1.5, 0) -- (1.5, 1);
    \node [right] at (1.5, 0.2) {પ્રોબ ફીડ};
    
    % Radiation
    \draw [->, red, decorate, decoration={snake}] (1, 1.2) -- (0.5, 2);
    \draw [->, red, decorate, decoration={snake}] (3, 1.2) -- (3.5, 2);
    \node [red] at (2, 2) {ફ્રિન્જિંગ ક્ષેત્રો};
\end{tikzpicture}
\end{answerdiagram}

\begin{itemize}
    \item \textbf{સંરચના}: ગ્રાઉન્ડવાળા ડાઇલેક્ટ્રિક સબસ્ટ્રેટ પર ધાતુનો પેચ હોય છે.
    \item \textbf{ફાયદા}: હળવું વજન, લો પ્રોફાઇલ, સસ્તું, કોઈપણ સપાટી પર લગાવી શકાય.
    \item \textbf{ગેરફાયદા}: સાંકડી બેન્ડવિડ્થ, ઓછી કાર્યક્ષમતા, ઓછો પાવર હેન્ડલિંગ.
    \item \textbf{ઉપયોગ}: મોબાઇલ ફોન, GPS, મિસાઇલ્સ, સેટેલાઇટ કોમ્યુનિકેશન.
\end{itemize}
\end{solutionbox}

\begin{mnemonicbox}
\mnemonic{"PSALM: પેચ સબસ્ટ્રેટ અબવ લેયર ઓફ મેટલ"}
\end{mnemonicbox}

\questionmarks{3(બ)}{4}{હેલિકલ એન્ટેના સમજાવો અને તેની રેડિયેશન લાક્ષણિકતાઓની ચર્ચા કરો}

\begin{solutionbox}
\textbf{હેલિકલ એન્ટેના}: સ્ક્રૂ આકારમાં વીંટળાયેલો તાર, જે વર્તુળાકાર ધ્રુવીકરણ (Circular Polarization) આપે છે.

\begin{answerdiagram}{હેલિકલ એન્ટેના}
\begin{tikzpicture}
    % Ground Plane
    \fill [gray!30] (0,-1.5) rectangle (1, 1.5);
    \draw [thick] (0,-1.5) -- (0, 1.5);
    \node [below] at (0.5, -1.5) {ગ્રાઉન્ડ};
    
    % Helix
    \draw [thick] (0, 0) 
    foreach \x in {0, 0.5, ..., 5} {
        -- ++(0.5, 0.5) -- ++(0.5, -0.5)
    };
    \node [above] at (2.5, 0.5) {હેલિક્સ વાયર};
    \draw [<->] (1, 0.6) -- (2, 0.6) node[midway, above] {$S$ (સ્પેસિંગ)};
    \draw [<->] (3, -0.6) -- (3, -1.1) node[midway, right] {$D$ (વ્યાસ)};

    % Radiation
    \draw [->, red, ultra thick] (5.5, 0) -- (7, 0) node[right] {એક્સિયલ રેડિએશન};
\end{tikzpicture}
\end{answerdiagram}

\begin{itemize}
    \item \textbf{નોર્મલ મોડ}: જો પરિમાણો $<< \lambda$, તો રેડિએશન અક્ષને લંબરૂપે હોય છે. ઓછી કાર્યક્ષમતા.
    \item \textbf{એક્સિયલ મોડ}: જો પરિધિ $C \approx \lambda$, તો રેડિએશન અક્ષની દિશામાં હોય છે. ઊંચો ગેઇન અને CP.
    \item \textbf{લાક્ષણિકતાઓ}: વિશાળ બેન્ડવિડ્થ (ઇમ્પિડન્સ રેઝિસ્ટિવ રહે છે).
    \item \textbf{ઉપયોગ}: સેટેલાઇટ ટ્રેકિંગ (CP ને કારણે).
\end{itemize}
\end{solutionbox}

\begin{mnemonicbox}
\mnemonic{"MOCHA: મોડ ઓફ સર્ક્યુલર હેલિક્સ એન્ટેનાઝ"}
\end{mnemonicbox}

\questionmarks{3(ક)}{7}{હોર્ન એન્ટેના સમજાવો અને તેની રેડિયેશન લાક્ષણિકતાઓની ચર્ચા કરો}

\begin{solutionbox}
\textbf{હોર્ન એન્ટેના}: ફ્લેર્ડ વેવગાઇડ જે વેવગાઇડ અને ફ્રી સ્પેસ વચ્ચે ઇમ્પિડન્સ મેચિંગ કરે છે.

\begin{answerdiagram}{હોર્ન એન્ટેનાના પ્રકારો}
\begin{tikzpicture}
    % Pyramidal
    \draw [thick] (0,0) rectangle (1,1); % Waveguide
    \draw [thick] (1,1) -- (3,2) -- (3,-1) -- (1,0); 
    \draw [thick] (3,2) -- (4,2) -- (4,-1) -- (3,-1); % Aperture
    \draw [dashed] (0,0) -- (1,0);
    \node [below] at (2, -1.2) {પિરામિડલ હોર્ન};
    \draw [->, red] (4, 0.5) -- (5.5, 0.5) node[right] {બીમ};

    % Conical
    \begin{scope}[shift={(6,0)}]
        \draw [thick] (0, 0.2) rectangle (1, 0.8);
        \draw [thick] (1, 0.8) -- (3, 1.5);
        \draw [thick] (1, 0.2) -- (3, -0.5);
        \draw [thick] (3, 0.5) ellipse (0.2 and 1); 
        \node [below] at (2, -1.2) {કોનિકલ હોર્ન};
    \end{scope}
\end{tikzpicture}
\end{answerdiagram}

\begin{itemize}
    \item \textbf{ઇમ્પિડન્સ મેચિંગ}: સ્મૂથ ફ્લેરને કારણે પરાવર્તન ઓછું થાય છે અને VSWR સુધરે છે.
    \item \textbf{બેન્ડવિડ્થ}: ખૂબ વિશાળ બેન્ડવિડ્થ.
    \item \textbf{દિશાત્મકતા}: મધ્યમ થી ઉચ્ચ (10-20 dB).
    \item \textbf{સાઇડ લોબ્સ}: એપર્ચર ડિસ્ટ્રિબ્યુશનના કારણે ખૂબ ઓછા સાઇડ લોબ્સ.
    \item \textbf{પ્રકારો}: સેક્ટોરલ (E કે H પ્લેન), પિરામિડલ, કોનિકલ.
    \item \textbf{ઉપયોગ}: પેરાબોલિક ડિશ માટે ફીડ તરીકે, રડાર, સ્ટાન્ડર્ડ ગેઇન રેફરન્સ.
\end{itemize}
\end{solutionbox}

\begin{mnemonicbox}
\mnemonic{"POWERS: પિરામિડલ ઓર વાઇડનિંગ એન્ડ રેડિએટ્સ સ્ટ્રોંગલી"}
\end{mnemonicbox}

\orquestionmarks{3(અ)}{3}{સ્લોટ એન્ટેના પર ટૂંકી નોંધ લખો}

\begin{solutionbox}
\textbf{સ્લોટ એન્ટેના}: વાહક સપાટી પર કાપવામાં આવેલો સ્લોટ.

\begin{answerdiagram}{સ્લોટ એન્ટેના}
\begin{tikzpicture}
    \fill [gray!20] (0,0) rectangle (4, 2);
    \draw [thick] (0,0) rectangle (4, 2);
    \fill [white] (1.5, 0.8) rectangle (2.5, 1.2);
    \draw [thick] (1.5, 0.8) rectangle (2.5, 1.2);
    \node at (2, 1) {સ્લોટ};
    \node [below] at (2, 0) {વાહક શીટ};
    
    % E-Fields
    \foreach \x in {1.6, 1.8, 2.0, 2.2, 2.4} {
        \draw [blue, ->] (\x, 0.8) -- (\x, 1.2);
    }
    \node [blue, right] at (2.5, 1) {E-ફિલ્ડ};
\end{tikzpicture}
\end{answerdiagram}

\begin{itemize}
    \item \textbf{બાબિનેટનો સિદ્ધાંત}: સ્લોટ એન્ટેના એ ડાયપોલનો "ડ્યુઅલ" છે. આડો સ્લોટ વર્ટિકલ પોલરાઈઝ્ડ તરંગો આપે છે.
    \item \textbf{ઇમ્પિડન્સ}: ડાયપોલ ઇમ્પિડન્સ સાથે સંબંધિત $Z_s Z_d = \frac{\eta^2}{4}$. ઉચ્ચ ઇમ્પિડન્સ (~500 $\Omega$).
    \item \textbf{ઉપયોગ}: વિમાન/મિસાઇલ્સ પર ફ્લશ માઉન્ટિંગ માટે (એરોડાયનેમિક ડ્રેગ ઘટાડવા).
\end{itemize}
\end{solutionbox}

\begin{mnemonicbox}
\mnemonic{"CROPS: કોમ્પ્લિમેન્ટરી રેડિએશન ઓપનિંગ પર્પેન્ડિક્યુલર ટુ સર્ફેસ"}
\end{mnemonicbox}

\orquestionmarks{3(બ)}{4}{પેરાબોલિક રિફ્લેક્ટર એન્ટેના સમજાવો અને તેની રેડિયેશન લાક્ષણિકતાઓની ચર્ચા કરો}

\begin{solutionbox}
\textbf{પેરાબોલિક રિફ્લેક્ટર}: બિંદુ સ્રોત (ફોકસ) માંથી spherical તરંગોને સમાંતર કિરણો (plane waves) માં ફેરવે છે.

\begin{answerdiagram}{પેરાબોલિક રિફ્લેક્ટર}
\begin{tikzpicture}
    % Parabola
    \draw [thick] (0, -2) parabola bend (1.5, 0) (0, 2);
    \node at (0.5, 1.5) {રિફ્લેક્ટર};
    
    % Focus
    \fill (2.5, 0) circle (0.1) node[below] {ફોકસ (ફીડ)};
    
    % Rays
    \draw [red, ->] (2.5, 0) -- (1, 1.5) -- (5, 1.5);
    \draw [red, ->] (2.5, 0) -- (1.5, 0) -- (5, 0);
    \draw [red, ->] (2.5, 0) -- (1, -1.5) -- (5, -1.5);
    
    \node [right, red] at (5, 0) {સમાંતર કિરણો};
\end{tikzpicture}
\end{answerdiagram}

\begin{itemize}
    \item \textbf{હાઈ ગેઇન}: અત્યંત ઊંચો ગેઇન (30-60 dB).
    \item \textbf{સાંકડી બીમવિડ્થ}: ખૂબ જ તીક્ષ્ણ "પેન્સિલ બીમ" બનાવે છે.
    \item \textbf{F/D રેશિયો}: ડિશની ઊંડાઈ અને ફોકલ લેન્થ નક્કી કરે છે.
    \item \textbf{કાર્યક્ષમતા}: સામાન્ય રીતે 55-65%. સ્પિલઓવર અને બ્લોકેજને કારણે ઘટે છે.
    \item \textbf{ઉપયોગ}: સેટેલાઇટ કોમ્યુનિકેશન, રેડિયો એસ્ટ્રોનોમી.
\end{itemize}
\end{solutionbox}

\begin{mnemonicbox}
\mnemonic{"DISH: ડાયરેક્ટિંગ ઇનકમિંગ સિગ્નલ્સ ટુ હબ"}
\end{mnemonicbox}

\orquestionmarks{3(ક)}{7}{V અને ઊંધી V એન્ટેનાનું વર્ણન કરો}

\begin{solutionbox}
બે તાર દ્વારા બનતા ટ્રાવેલિંગ વેવ એન્ટેના.

\begin{answerdiagram}{V અને ઇન્વર્ટેડ-V એન્ટેના}
\begin{tikzpicture}
    % V Antenna
    \begin{scope}[shift={(0,0)}]
        \node [font=\bfseries] at (2, 2.5) {V એન્ટેના};
        \draw [thick] (0,0) -- (3, 1.5);
        \draw [thick] (0,0) -- (3, -1.5);
        \draw [->] (-0.5, 0) -- (0,0) node[left] {ફીડ};
        \draw [red, ->] (0,0) -- (4,0) node[right] {મહત્તમ રેડિએશન};
        \node at (2, 0) {દ્વિ-દિશાત્મક};
    \end{scope}

    % Inverted V
    \begin{scope}[shift={(6,0)}]
        \node [font=\bfseries] at (2, 2.5) {ઇન્વર્ટેડ V};
        \draw [thick] (2, 2) -- (0, 0);
        \draw [thick] (2, 2) -- (4, 0);
        \draw [thick] (2, 2) -- (2, 0) node[midway, right] {માસ્ટ}; % Support
        \draw [->] (2, 2.5) -- (2, 2) node[above] {ફીડ};
        \draw [dashed] (-1,0) -- (5,0) node[right] {જમીન};
        \node at (2, -0.5) {ઓમ્નીડાયરેક્શનલ પેટર્ન};
    \end{scope}
\end{tikzpicture}
\end{answerdiagram}

\begin{tabulary}{\linewidth}{|L|L|L|}
\hline
\textbf{લક્ષણ} & \textbf{V એન્ટેના} & \textbf{ઇન્વર્ટેડ V એન્ટેના} \\ \hline
\keyword{રચના} & જમીનને સમાંતર આડો V આકાર & ઊંધો V આકાર (વર્ટિકલ) \\ \hline
\keyword{રેડિએશન} & અક્ષની દિશામાં દ્વિ-દિશાત્મક & લગભગ ઓમ્નીડાયરેક્શનલ (આડું) \\ \hline
\keyword{બાંધકામ} & અનેક સપોર્ટની જરૂર પડે & માત્ર એક મધ્ય સપોર્ટ (માસ્ટ) જોઈએ \\ \hline
\keyword{ઇમ્પિડન્સ} & ઊંચો (~600 $\Omega$) & નીચો (~50 $\Omega$) - મેચિંગ સરળ \\ \hline
\keyword{ઉપયોગ} & પોઇન્ટ-ટુ-પોઇન્ટ HF કોમ્યુનિકેશન & એમેચ્યોર રેડિયો (Ham), ઓછી જગ્યા \\ \hline
\end{tabulary}
\end{solutionbox}

\begin{mnemonicbox}
\mnemonic{"VIVA: V ઇઝ વર્ટિકલ અરેન્જમેન્ટ, ઇન્વર્ટેડ V એઇમ્સ ડાઉનવર્ડ"}
\end{mnemonicbox}


\questionmarks{4(અ)}{3}{વ્યાખ્યાયિત કરો: (1) રીફ્લેક્શન, (2) રીફ્રેક્શન અને (3) ડીફ્રેક્શન}

\begin{solutionbox}
\begin{tabulary}{\linewidth}{|L|L|}
\hline
\textbf{ઘટના} & \textbf{વ્યાખ્યા} \\ \hline
\keyword{પરાવર્તન (Reflection)} & જ્યારે તરંગો બે માધ્યમની સરહદ (જેમ કે જમીન, આયનોસ્ફિયર) પર અથડાય ત્યારે તેનું પાછું વળવું. \\ \hline
\keyword{વક્રીભવન (Refraction)} & જ્યારે તરંગો એક માધ્યમથી બીજા માધ્યમમાં જાય ત્યારે તેમની ગતિમાં ફેરફારને કારણે તેમનું વાંકા વળવું. \\ \hline
\keyword{વિવર્તન (Diffraction)} & અવરોધોની આસપાસથી તરંગોનું વળવું. આનાથી પર્વતની પાછળ પણ સિગ્નલ મળી શકે છે. \\ \hline
\end{tabulary}
\end{solutionbox}

\begin{mnemonicbox}
\mnemonic{"RRD: રિબાઉન્ડિંગ, રિડાયરેક્ટિંગ, ડિટૂર"}
\end{mnemonicbox}

\questionmarks{4(બ)}{4}{સંચાર માટે HAM રેડિયો એપ્લિકેશનની સૂચિ બનાવો}

\begin{solutionbox}
\begin{tabulary}{\linewidth}{|L|L|}
\hline
\textbf{એપ્લિકેશન} & \textbf{વર્ણન} \\ \hline
\keyword{ઇમરજન્સી કોમ.} & જ્યારે સામાન્ય નેટવર્ક ફેલ થાય ત્યારે આપત્તિ સમયે મદદ. \\ \hline
\keyword{DXing} & શોખ માટે લાંબા અંતરનો આંતરરાષ્ટ્રીય સંચાર. \\ \hline
\keyword{સેટેલાઇટ કોમ.} & રિલે માટે એમેચ્યોર ઉપગ્રહો (OSCAR) નો ઉપયોગ. \\ \hline
\keyword{ડિજિટલ મોડ્સ} & રેડિયો દ્વારા ડેટા મોકલવો (FT8, RTTY). \\ \hline
\keyword{મોર્સ કોડ} & નબળા સિગ્નલ માટે પરંપરાગત CW સંચાર. \\ \hline
\keyword{શિક્ષણ} & ઇલેક્ટ્રોનિક્સ અને રેડિયો ફિઝિક્સ શીખવા. \\ \hline
\end{tabulary}
\end{solutionbox}

\begin{mnemonicbox}
\mnemonic{"EDSDMVP: ઇમરજન્સી DX સેટેલાઇટ ડિજિટલ મોર્સ વોઇસ પબ્લિક-સર્વિસ"}
\end{mnemonicbox}

\questionmarks{4(ક)}{7}{આયનોસ્ફિયરના સ્તરો અને આકાશી તરંગોના પ્રસારને સમજાવો}

\begin{solutionbox}
\textbf{સ્કાય વેવ પ્રોપેગેશન}: પૃથ્વીના વાતાવરણના ઉપરના સ્તર (આયનોસ્ફિયર) નો ઉપયોગ કરીને સિગ્નલને પરાવર્તિત કરી લાંબા અંતર સુધી મોકલવાની પદ્ધતિ.

\begin{answerdiagram}{આયનોસ્ફેરિક સ્તરો}
\begin{tikzpicture}
    % Earth
    \draw [thick, brown] (-4, -1) .. controls (0, 0) .. (4, -1);
    \node [below] at (0, -0.5) {પૃથ્વી સપાટી};
    
    % Layers
    \draw [dotted] (-4, 1) -- (4, 1) node[right] {D (60-90km)};
    \draw [dotted] (-4, 2) -- (4, 2) node[right] {E (90-150km)};
    \draw [dotted] (-4, 3) -- (4, 3) node[right] {F1 (150-250km)};
    \draw [dotted] (-4, 4) -- (4, 4) node[right] {F2 (250-400km)};
    
    % Sun
    \fill [orange] (-4.5, 4.5) circle (0.3);
    \node [left] at (-4.5, 4.5) {દિવસ};
    
    % Rays
    \draw [red, ->] (-3, -0.5) -- (0, 4) -- (3, -0.5);
    \node [red, above] at (0, 4) {પરાવર્તન};
\end{tikzpicture}
\end{answerdiagram}

\begin{itemize}
    \item \textbf{D સ્તર}: માત્ર દિવસ દરમિયાન હોય છે. MF/HF સિગ્નલોનું શોષણ કરે છે.
    \item \textbf{E સ્તર}: કેટલાક HF તરંગો પરાવર્તિત કરે છે. Sporadic-E VHF DX માટે ઉપયોગી છે.
    \item \textbf{F1 સ્તર}: દિવસ દરમિયાન અસ્તિત્વમાં હોય છે.
    \item \textbf{F2 સ્તર}: લાંબા અંતરના HF સંચાર માટે સૌથી મહત્વનું. રાત્રે પણ હોય છે. સૌથી ઊંચી ફ્રિકવન્સી પરાવર્તિત કરે છે.
    \item \textbf{મિકેનિઝમ}: સૂર્યના UV કિરણો વાયુઓનું આયનીકરણ કરે છે. રેડિયો તરંગો વક્રીભવન પામી પૃથ્વી પર પાછા ફરે છે.
\end{itemize}
\end{solutionbox}

\begin{mnemonicbox}
\mnemonic{"DEFV: D-એબ્ઝોર્બ્સ, E-રિફ્લેક્ટ્સ, F-પ્રોવાઇડ્સ વેરી-લોંગ-ડિસ્ટન્સ"}
\end{mnemonicbox}

\orquestionmarks{4(અ)}{3}{વ્યાખ્યાયિત કરો: (1) MUF, (2) LUF અને (3) સ્કીપ અંતર}

\begin{solutionbox}
\begin{tabulary}{\linewidth}{|L|L|}
\hline
\textbf{શબ્દ} & \textbf{વ્યાખ્યા} \\ \hline
\keyword{MUF} & \textbf{મહત્તમ ઉપયોગી આવૃત્તિ}: સૌથી વધુ ફ્રિકવન્સી જે આયનોસ્ફિયર દ્વારા પરાવર્તિત થઈ શકે. $f_{MUF} = f_c \sec\theta$. \\ \hline
\keyword{LUF} & \textbf{ન્યૂનતમ ઉપયોગી આવૃત્તિ}: સૌથી ઓછી ફ્રિકવન્સી જ્યાં સિગ્નલ ઘોંઘાટ કરતા વધારે હોય. આનાથી નીચે શોષણ વધી જાય છે. \\ \hline
\keyword{સ્કીપ ડિસ્ટન્સ} & ટ્રાન્સમીટરથી તે ન્યૂનતમ અંતર જ્યાં સ્કાય વેવ પાછું આવે છે. આ ઝોનમાં સિગ્નલ મળતું નથી (સ્કીપ ઝોન). \\ \hline
\end{tabulary}
\end{solutionbox}

\begin{mnemonicbox}
\mnemonic{"MLS: મેક્સિમમ-હાયેસ્ટ, લોવેસ્ટ-મિનિમમ, સ્કિપ-નિયરેસ્ટ"}
\end{mnemonicbox}

\orquestionmarks{4(બ)}{4}{સંચારના HAM રેડિયો ડિજિટલ મોડ્સની સૂચિ બનાવો}

\begin{solutionbox}
\begin{tabulary}{\linewidth}{|L|L|}
\hline
\textbf{મોડ} & \textbf{લાક્ષણિકતાઓ} \\ \hline
\keyword{FT8} & ખૂબ નબળા સિગ્નલ માટે. 15-સેકન્ડ ઈન્ટરવલ. ઓટોમેટેડ. DX માટે લોકપ્રિય. \\ \hline
\keyword{PSK31} & ફેઝ શિફ્ટ કીઈંગ. સાંકડી બેન્ડવિડ્થ (31 Hz). ચેટિંગ જેવું ટાઈપિંગ. \\ \hline
\keyword{RTTY} & રેડિયો ટેલિટાઈપ. જૂનો અને મજબૂત ડિજિટલ મોડ. \\ \hline
\keyword{SSTV} & સ્લો સ્કેન ટીવી. ઓડિયો ટોન દ્વારા ઈમેજ મોકલવી. \\ \hline
\keyword{Packet} & ડેટા પેકેટ્સ (AX.25). APRS માટે વપરાય છે. \\ \hline
\keyword{JT65} & ડીપ સ્પેસ કોમ્યુનિકેશન માટે નબળા સિગ્નલ મોડ. \\ \hline
\end{tabulary}
\end{solutionbox}

\begin{mnemonicbox}
\mnemonic{"FIRST PAD: FT8 ઇઝ RTTY SSTV ધેન પેકેટ APRS ડિજિટલ-વોઇસ"}
\end{mnemonicbox}

\orquestionmarks{4(ક)}{7}{અવકાશ તરંગોના પ્રસારને સમજાવો}

\begin{solutionbox}
\textbf{સ્પેસ વેવ (ટ્રોપોસ્ફેરિક)}: VHF, UHF, માઈક્રોવેવ (> 30 MHz) માટે ડાયરેક્ટ Line-of-Sight સંચાર.

\begin{answerdiagram}{સ્પેસ વેવ પ્રોપેગેશન}
\begin{tikzpicture}
    % Earth
    \draw [thick, brown] (-4, -1) .. controls (0, 0.5) .. (4, -1);
    
    % Towers
    \draw [thick] (-3, -0.6) -- (-3, 0.5) node[above] {TX ($h_t$)};
    \draw [thick] (3, -0.6) -- (3, 0.5) node[above] {RX ($h_r$)};
    
    % Direct
    \draw [thick, blue, ->] (-3, 0.5) -- (3, 0.5) node[midway, above] {ડાયરેક્ટ વેવ};
    
    % Reflected
    \draw [thick, red, dashed, ->] (-3, 0.5) -- (0, 0.5) -- (3, 0.5);
    \node [red, below] at (0, 0.5) {ગ્રાઉન્ડ પરાવર્તન};

    % Range
    \draw [dashed] (-3, 0.5) -- (4, 0.5) node[right] {ક્ષિતિજ};
\end{tikzpicture}
\end{answerdiagram}

\begin{itemize}
    \item \textbf{ઘટકો}:
    1. \textbf{ડાયરેક્ટ વેવ}: TX થી RX સીધું જાય છે.
    2. \textbf{ગ્રાઉન્ડ રિફ્લેક્ટેડ}: જમીન પરથી પરાવર્તિત થઈને આવે છે (ફેઝ શિફ્ટ સાથે).
    \item \textbf{રેન્જ}: પૃથ્વીની વક્રતા/ક્ષિતિજ પર આધારિત.
    $$ d = 3.57 (\sqrt{h_t} + \sqrt{h_r}) \text{ km} $$
    \item \textbf{ટ્રોપોસ્ફેરિક સ્કેટર}: ક્ષિતિજથી થોડે દૂર સુધી સંચાર શક્ય બનાવે છે.
    \item \textbf{ડક્ટિંગ}: તાપમાન વ્યસ્તતા (Inversion) તરંગોને ટ્રેપ કરે છે, રેન્જ સેંકડો km વધારે છે.
\end{itemize}
\end{solutionbox}

\begin{mnemonicbox}
\mnemonic{"DRIFT: ડાયરેક્ટ રિફ્લેક્શન ઇન્વર્ઝન ફોરવર્ડ ટ્રોપોસ્ફેરિક"}
\end{mnemonicbox}

\questionmarks{5(અ)}{3}{વ્યાખ્યા કરો: (1) બીમ એરિયા (2) બીમ કાર્યક્ષમતા, અને (3) અસરકારક અપર્ચર}

\begin{solutionbox}
\begin{tabulary}{\linewidth}{|L|L|}
\hline
\textbf{પરિમાણ} & \textbf{વ્યાખ્યા} \\ \hline
\keyword{બીમ એરિયા} ($\Omega_A$) & તે ઘન કોણ જેમાંથી જો રેડિએશન ઇન્ટેન્સિટી અચળ હોય તો બધો પાવર પસાર થાય. \\ \hline
\keyword{બીમ કાર્યક્ષમતા} ($\epsilon_M$) & મુખ્ય બીમમાં રહેલા પાવરનો કુલ રેડિએટેડ પાવર (મુખ્ય + સાઇડ લોબ્સ) સાથેનો ગુણોત્તર. \\ \hline
\keyword{ઈફેક્ટિવ એપર્ચર} ($A_e$) & એક કાલ્પનિક વિસ્તાર જે આવતા મોજામાંથી ઉર્જા ગ્રહણ કરે છે. $A_e = \frac{\lambda^2}{4\pi} G$. \\ \hline
\end{tabulary}
\end{solutionbox}

\begin{mnemonicbox}
\mnemonic{"BEA: બીમ એફિશિયન્સી એપર્ચર"}
\end{mnemonicbox}

\questionmarks{5(બ)}{4}{સ્માર્ટ એન્ટેનાની જરૂરિયાતનું વર્ણન કરો}

\begin{solutionbox}
\textbf{જરૂરિયાત}: વાયરલેસ નેટવર્કમાં ક્ષમતા વધારવા અને ઇન્ટરફીયરન્સ ઘટાડવા.

\begin{answerdiagram}{સ્માર્ટ એન્ટેનાના ફાયદા}
\begin{tikzpicture}
    \node [gtu block, align=center] (center) {સ્માર્ટ એન્ટેના\\ફાયદા};
    
    \node [gtu state, above=of center] (cap) {ક્ષમતા $\uparrow$};
    \node [gtu state, right=of center] (cov) {કવરેજ $\uparrow$};
    \node [gtu state, below=of center] (pwr) {પાવર $\downarrow$};
    \node [gtu state, left=of center] (int) {ઇન્ટરફીયરન્સ $\downarrow$};
    
    \draw [gtu arrow] (center) -- (cap);
    \draw [gtu arrow] (center) -- (cov);
    \draw [gtu arrow] (center) -- (pwr);
    \draw [gtu arrow] (center) -- (int);
\end{tikzpicture}
\end{answerdiagram}

\begin{itemize}
    \item \textbf{ક્ષમતા}: SDMA દ્વારા એક જ ફ્રિકવન્સીનો ફરી ઉપયોગ થઈ શકે.
    \item \textbf{ઇન્ટરફીયરન્સ}: નલ સ્ટીયરિંગ (Null steering) દ્વારા અનિચ્છનીય સિગ્નલ રદ કરે છે.
    \item \textbf{રેન્જ}: હાઈ ગેઇન બીમ કવરેજ વધારે છે.
    \item \textbf{કાર્યક્ષમતા}: માત્ર જરૂરી દિશામાં જ પાવર મોકલે છે, બેટરી બચાવે છે.
\end{itemize}
\end{solutionbox}

\begin{mnemonicbox}
\mnemonic{"PRECISE: પાવર રિડક્શન, એન્હાન્સ્ડ કવરેજ, ઇન્ટરફેરન્સ સપ્રેશન, એન્હાન્સ્ડ સિગ્નલ"}
\end{mnemonicbox}

\questionmarks{5(ક)}{7}{DTH રીસીવર ઇન્ડોર અને આઉટડોર બ્લોક ડાયાગ્રામ દોરો અને તેના કાર્યોની ચર્ચા કરો}

\begin{solutionbox}
\begin{answerdiagram}{DTH સિસ્ટમ}
\begin{tikzpicture}[node distance=1.5cm, auto]
    % Outdoor
    \node [cloud, draw, aspect=2, blue] (sat) {સેટેલાઇટ (Ku બેન્ડ)};
    \node [draw, semicircle, rotate=90, minimum width=1cm, below=of sat] (dish) {}; 
    \node [right] at (dish) {ડિશ + LNB};
    \node [draw, dashed, fit=(dish), label=below:આઉટડોર યુનિટ] {};

    % Indoor
    \node [gtu block, below=2cm of dish] (tuner) {ટ્યુનર};
    \node [gtu block, right=of tuner] (demod) {ડીમોડયૂલેટર};
    \node [gtu block, right=of demod] (dec) {ડીકોડર};
    \node [gtu block, right=of dec] (av) {A/V આઉટ};
    
    \node [draw, dashed, fit=(tuner) (av), label=above:સેટ-ટોપ બોક્સ (ઇનડોર)] {};

    \draw [gtu arrow, dashed] (sat) -- (dish);
    \draw [gtu arrow] (dish) -- node[left] {કેબલ (IF)} (tuner);
    \draw [gtu arrow] (tuner) -- (demod);
    \draw [gtu arrow] (demod) -- (dec);
    \draw [gtu arrow] (dec) -- (av);
\end{tikzpicture}
\end{answerdiagram}

\begin{itemize}
    \item \textbf{આઉટડોર યુનિટ}:
    \begin{itemize}
        \item \textbf{ડિશ}: પેરાબોલિક રિફ્લેક્ટર નબળા સેટેલાઇટ સિગ્નલ (10-12 GHz) એકત્રિત કરે છે.
        \item \textbf{LNB}: હાઈ ફ્રિકવન્સીને લોઅર IF (950-2150 MHz) માં ફેરવે છે.
    \end{itemize}
    \item \textbf{ઇનડોર યુનિટ (STB)}:
    \begin{itemize}
        \item \textbf{ટ્યુનર}: ચોક્કસ ચેનલ સિલેક્ટ કરે છે.
        \item \textbf{ડીમોડ્યુલેટર}: ડિજિટલ સ્ટ્રીમ (QPSK) રિકવર કરે છે.
        \item \textbf{ડીકોડર}: સ્માર્ટ કાર્ડ વડે ચેનલ ડિક્રિપ્ટ કરે છે અને MPEG વિડીયો ડિકોડ કરે છે.
        \item \textbf{આઉટપુટ}: ટીવીને ઓડિયો/વિડિયો આપે છે.
    \end{itemize}
\end{itemize}
\end{solutionbox}

\begin{mnemonicbox}
\mnemonic{"COLD-TDUMS: કલેક્શન, ઓસિલેટર, લો-નોઇઝ, ડાઉનકન્વર્ઝન - ટ્યુનર ડિમોડ્યુલેટર અનસ્ક્રેમ્બલર MPEG સ્માર્ટ-કાર્ડ"}
\end{mnemonicbox}

\orquestionmarks{5(અ)}{3}{વ્યાખ્યાયિત કરો: (1) એન્ટેના, (2) ફોલ્ડેડ ડાયપોલ, અને (3) એન્ટેના એરે}

\begin{solutionbox}
\begin{tabulary}{\linewidth}{|L|L|}
\hline
\textbf{શબ્દ} & \textbf{વ્યાખ્યા} \\ \hline
\keyword{એન્ટેના} & એક ઉપકરણ જે ઇલેક્ટ્રિકલ સિગ્નલને ઇલેક્ટ્રોમેગ્નેટિક વેવ્સમાં (અને ઊલટું) રૂપાંતરિત કરે છે. \\ \hline
\keyword{ફોલ્ડેડ ડાયપોલ} & ડાયપોલ જેના છેડા એકબીજા સાથે જોડાયેલા હોય છે. ઇમ્પિડન્સ વધારે ($300\Omega$) અને બેન્ડવિડ્થ સારી હોય છે. \\ \hline
\keyword{એન્ટેના એરે} & અનેક એન્ટેનાઓની ગોઠવણી જે ઉચ્ચ દિશાત્મકતા અને ગેઇન મેળવવા માટે એકસાથે કામ કરે છે. \\ \hline
\end{tabulary}
\end{solutionbox}

\begin{mnemonicbox}
\mnemonic{"AFA: એન્ટેના ફોલ્ડેડ એરે"}
\end{mnemonicbox}

\orquestionmarks{5(બ)}{4}{સ્માર્ટ એન્ટેનાના ઉપયોગનું વર્ણન કરો}

\begin{solutionbox}
\begin{tabulary}{\linewidth}{|L|L|}
\hline
\textbf{એપ્લિકેશન} & \textbf{વર્ણન} \\ \hline
\keyword{સેલ્યુલર (4G/5G)} & બીમફોર્મિંગ/MIMO વડે યુઝર કેપેસિટી અને ડેટા રેટ વધારે છે. \\ \hline
\keyword{Wi-Fi (MIMO)} & રાઉટર્સ સિગ્નલ ફોકસ કરવા અને સ્પીડ વધારવા મલ્ટીપલ એન્ટેના વાપરે છે. \\ \hline
\keyword{રડાર} & ઇલેક્ટ્રોનિક સ્કેનિંગ (AESA) વડે ફરતા ભાગો વગર સ્કેનિંગ કરે છે. \\ \hline
\keyword{સેટેલાઇટ} & સ્પોટ બીમ એન્ટેના ચોક્કસ ભૌગોલિક વિસ્તાર પર ફોકસ કરે છે. \\ \hline
\keyword{વાહનો} & V2X કોમ્યુનિકેશન (ઓટોનોમસ ડ્રાઈવિંગ) માટે. \\ \hline
\end{tabulary}
\end{solutionbox}

\begin{mnemonicbox}
\mnemonic{"MBMRSWI: મોબાઇલ બેઝ MIMO રડાર સેટેલાઇટ Wi-Fi IoT"}
\end{mnemonicbox}

\orquestionmarks{5(ક)}{7}{ટેરેસ્ટ્રીયલ મોબાઈલ કોમ્યુનિકેશન એન્ટેના સમજાવો અને બેઝ સ્ટેશન અને મોબાઈલ સ્ટેશન એન્ટેના વિશે પણ ચર્ચા કરો}

\begin{solutionbox}
\textbf{મોબાઈલ કોમ્યુનિકેશન એન્ટેના}:

\begin{answerdiagram}{મોબાઇલ અને બેઝ એન્ટેના}
\begin{tikzpicture}
    \node [gtu block] (root) {એન્ટેના};
    
    \node [gtu block, below left=of root, xshift=-1cm] (bs) {બેઝ સ્ટેશન};
    \node [gtu block, below right=of root, xshift=1cm] (ms) {મોબાઈલ સ્ટેશન};
    
    \node [gtu state, below=of bs] (sec) {સેક્ટોરલ};
    \node [gtu state, below=of sec] (panel) {પેનલ};
    
    \node [gtu state, below=of ms] (pifa) {PIFA};
    \node [gtu state, below=of pifa] (whip) {વ્હીપ/મોનોપોલ};

    \draw [gtu arrow] (root) -- (bs);
    \draw [gtu arrow] (root) -- (ms);
    \draw [gtu arrow] (bs) -- (sec);
    \draw [gtu arrow] (sec) -- (panel);
    \draw [gtu arrow] (ms) -- (pifa);
    \draw [gtu arrow] (pifa) -- (whip);
\end{tikzpicture}
\end{answerdiagram}

\begin{enumerate}
    \item \textbf{બેઝ સ્ટેશન એન્ટેના (ટાવર)}:
    \begin{itemize}
        \item \textbf{સેક્ટર એન્ટેના}: વર્ટિકલ પેનલ જે $120^\circ$ કવરેજ આપે છે. હાઈ ગેઇન.
        \item \textbf{ઓમ્ની}: ગ્રામીણ વિસ્તારોમાં જ્યાં ટ્રાફિક ઓછો હોય.
        \item \textbf{લાક્ષણિકતાઓ}: હાઈ પાવર હેન્ડલિંગ, વેધર પ્રૂફ, ઇલેક્ટ્રિકલ ટિલ્ટ.
    \end{itemize}
    \item \textbf{મોબાઈલ સ્ટેશન એન્ટેના (યુઝર)}:
    \begin{itemize}
        \item \textbf{PIFA}: સ્માર્ટફોનની અંદર વપરાય છે. કોમ્પેક્ટ, લો પ્રોફાઇલ.
        \item \textbf{વ્હીપ/મોનોપોલ}: વાહનો પર લાગે છે. ઓમ્નીડાયરેક્શનલ પેટર્ન.
        \item \textbf{જરૂરિયાત}: નાનું કદ, ઓમ્નીડાયરેક્શનલ (કોઈપણ ખૂણેથી સિગ્નલ મેળવવા), ઓછો SAR.
    \end{itemize}
\end{enumerate}
\end{solutionbox}

\begin{mnemonicbox}
\mnemonic{"BEST: બેઝ-સ્ટેશન્સ એમ્પ્લોય સેક્ટર ટેકનોલોજી"}
\end{mnemonicbox}

\end{document}
