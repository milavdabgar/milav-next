\documentclass[10pt,a4paper]{article}

% content/resources/templates/preamble.tex
\usepackage[margin=0.6in]{geometry}
\author{Milav Dabgar}
\usepackage{amsmath,amssymb,amsthm}
\usepackage{booktabs}
\usepackage{multirow}
\usepackage{xcolor}
\usepackage{tcolorbox}
\tcbuselibrary{breakable,skins}
\usepackage[colorlinks=true,linkcolor=blue]{hyperref}
\usepackage{titlesec}
\usepackage{enumitem}
\usepackage{tikz}
\usepackage{pgfplots}
\usepackage{circuitikz}
\usepackage[version=4]{mhchem}
\usepackage{longtable}
\usepackage{array}
\usepackage{float}
\usepackage{caption}
\usepackage{listings}

\lstset{
  basicstyle=\small\ttfamily,
  breaklines=true,
  breakatwhitespace=false,
  postbreak=\mbox{\textcolor{red}{$\hookrightarrow$}\space},
  float=false,
  numbers=left,
  numberstyle=\tiny\color{gray},
  numbersep=10pt,
  xleftmargin=2em,
  keywordstyle=\color{blue},
  commentstyle=\color{green!60!black},
  stringstyle=\color{purple},
  backgroundcolor=\color{gray!5},
  showstringspaces=false,
  tabsize=2,
  captionpos=b,
  keepspaces=true,
  columns=flexible
}

\pgfplotsset{compat=1.18}
\usetikzlibrary{shapes,arrows,positioning,calc,patterns,decorations.pathmorphing,decorations.markings,arrows.meta}

% Color scheme
\definecolor{headcolor}{RGB}{0,102,204}
\definecolor{keycolor}{RGB}{220,20,60}
\definecolor{solutioncolor}{RGB}{34,139,34}
\definecolor{mnemoniccolor}{RGB}{148,0,211}
\definecolor{codecolor}{RGB}{0,0,100}

% Spacing
\setlength{\parskip}{3pt}
\setlist[itemize]{nosep}
\setlist[enumerate]{nosep}

% Title formatting
\titleformat{\section}{\Large\bfseries\color{headcolor}}{\thesection}{1em}{}
\titleformat{\subsection}{\large\bfseries\color{headcolor}}{\thesubsection}{1em}{}

% Pandoc tightlist compatibility
\providecommand{\tightlist}{%
  \setlength{\itemsep}{0pt}\setlength{\parskip}{0pt}}

% Pandoc longtable compatibility
\newcounter{none}
\def\thenone{}


% content/resources/templates/english-boxes.tex
% This file is currently empty - it exists to maintain consistency with the import structure.
% Add custom environments here if needed in the future.


\begin{document}

\begin{center}
{\Huge\bfseries\color{headcolor} Subject Name Solutions}\\[5pt]
{\LARGE 4341106 -- Summer 2024}\\[3pt]
{\large Semester 1 Study Material}\\[3pt]
{\normalsize\textit{Detailed Solutions and Explanations}}
\end{center}

\vspace{10pt}

\subsection*{Question 1(a) [3 marks]}\label{q1a}

\textbf{Define Beam Area and Beam Efficiency.}

\begin{solutionbox}

\textbf{Beam Area}: The solid angle through which all of the power
radiated by an antenna would flow if the radiation intensity was
constant throughout this angle and equal to the maximum value.

\textbf{Beam Efficiency}: The ratio of the power contained in the main
beam to the total power radiated by the antenna.

\textbf{Diagram}:

\begin{center}
\textbf{Mermaid Diagram (Code)}
\begin{verbatim}
{Shaded}
{Highlighting}[]
graph LR
    A[Beam Area] {-{-}{} B[Solid angle containing{}br /{}most of the radiated power]}
    C[Beam Efficiency] {-{-}{} D[Main Beam Power/Total Power]}
    D {-{-}{} E[Higher efficiency = Better antenna]}
{Highlighting}
{Shaded}
\end{verbatim}
\end{center}

\end{solutionbox}
\begin{mnemonicbox}
``BEAM: Better Efficiency Achieves Maximum
performance''

\end{mnemonicbox}
\subsection*{Question 1(b) [4 marks]}\label{q1b}

\textbf{What is EM field? Explain its radiation from center fed dipole.}

\begin{solutionbox}

EM field is a physical field produced by electrically charged objects
that affects charged particles with a force.

\textbf{Diagram}:

\begin{verbatim}
                        |
                        |
     E{-field            |            E{-}field}
    (vertical)          |           (vertical)
                        |
           ↑            |           ↑
           |   current  |           |
           |      ↓     |           |
     {-{-}{-}{-}{-}{-}+{-}{-}{-}{-}{-}{-}{-}{-}{-}{-}{-}{-}+{-}{-}{-}{-}{-}{-}{-}{-}{-}{-} dipole antenna}
           |      ↑     |           |
           |   current  |           |
           ↓            |           ↓
                        |
     H{-field            |            H{-}field}
    (circular)          |           (circular)
                        |
\end{verbatim}

\begin{itemize}
\tightlist
\item
  \textbf{Electric field}: Perpendicular to antenna axis, maximum at
  antenna ends
\item
  \textbf{Magnetic field}: Circular around antenna axis
\item
  \textbf{Radiation mechanism}: Alternating current creates time-varying
  fields
\item
  \textbf{Field behavior}: Near field (reactive) \rightarrow intermediate \rightarrow far
  field (radiating)
\end{itemize}

\end{solutionbox}
\begin{mnemonicbox}
``CERD: Current Excites Radiating Dipole''

\end{mnemonicbox}
\subsection*{Question 1(c) [7 marks]}\label{q1c}

\textbf{Explain Power radiated by elementary dipole using Poynting
Vector.}

\begin{solutionbox}

Power radiated by an elementary dipole can be calculated using the
Poynting vector, which represents power flow density.


{\def\LTcaptype{none} % do not increment counter
\vspace{-5pt}
\captionof{table}{Key Steps in Poynting Vector Analysis}
\vspace{-10pt}
\begin{longtable}[]{@{}ll@{}}
\toprule\noalign{}
Step & Description \\
\midrule\noalign{}
\endhead
\bottomrule\noalign{}
\endlastfoot
1 & Calculate E-field components (Eθ, Eφ) \\
2 & Calculate H-field components (Hθ, Hφ) \\
3 & Determine Poynting vector: P = E \times H \\
4 & Integrate over a spherical surface \\
\end{longtable}
}

\textbf{Diagram}:

\begin{center}
\textbf{Mermaid Diagram (Code)}
\begin{verbatim}
{Shaded}
{Highlighting}[]
graph LR
    A[Poynting Vector{br /{}P = E  H] {-}{-}{} B[Time{-}average{}br /{}power density]}
    B {-{-}{} C[Integrate over sphere{}br /{}P = ·ds]}
    C {-{-}{} D[Power radiated{}br /{}P = 80π^{2}I^{2}l^{2}/λ^{2}]}
{Highlighting}
{Shaded}
\end{verbatim}
\end{center}

\begin{itemize}
\tightlist
\item
  \textbf{Electric field}: E = (jη I_{0}dl/2λr) sin θ e^{-}ʲᵏʳ
\item
  \textbf{Magnetic field}: H = (j I_{0}dl/2λr) sin θ e^{-}ʲᵏʳ
\item
  \textbf{Poynting vector}: P = E \times H* =
  (η\textbar I_{0}\textbar^{2}\textbar dl\textbar^{2}/8π^{2}r^{2}) sin^{2} θ
\item
  \textbf{Total power}: P =
  (η\textbar I_{0}\textbar^{2}\textbar dl\textbar^{2}/12π) = 80π^{2}I^{2}l^{2}/λ^{2}
\end{itemize}

\end{solutionbox}
\begin{mnemonicbox}
``PEHP: Poynting Explains How Power propagates''

\end{mnemonicbox}
\subsection*{Question 1(c) OR [7
marks]}\label{q1c}

\textbf{Define Antenna, Radiation Pattern, Directivity, Gain, FBR,
Isotropic Radiator and Effective Aperture.}

\begin{solutionbox}


{\def\LTcaptype{none} % do not increment counter
\vspace{-5pt}
\captionof{table}{Key Antenna Parameters}
\vspace{-10pt}
\begin{longtable}[]{@{}
  >{\raggedright\arraybackslash}p{(\linewidth - 2\tabcolsep) * \real{0.4783}}
  >{\raggedright\arraybackslash}p{(\linewidth - 2\tabcolsep) * \real{0.5217}}@{}}
\toprule\noalign{}
\begin{minipage}[b]{\linewidth}\raggedright
Parameter
\end{minipage} & \begin{minipage}[b]{\linewidth}\raggedright
Definition
\end{minipage} \\
\midrule\noalign{}
\endhead
\bottomrule\noalign{}
\endlastfoot
Antenna & A device that converts guided electromagnetic waves to
free-space waves and vice versa \\
Radiation Pattern & Graphical representation of radiation properties as
a function of space coordinates \\
Directivity & Ratio of radiation intensity in a given direction to
average radiation intensity \\
Gain & Ratio of radiation intensity to that of an isotropic source with
same input power \\
FBR (Front-to-Back Ratio) & Ratio of power radiated in forward direction
to that in backward direction \\
Isotropic Radiator & Theoretical antenna that radiates equally in all
directions \\
Effective Aperture & Ratio of power received by antenna to incident
power density \\
\end{longtable}
}

\textbf{Diagram}:

\begin{verbatim}
pie
    title "Antenna Performance Factors"
    "Directivity" : 25
    "Gain" : 25
    "Effective Aperture" : 20
    "Radiation Pattern" : 15
    "FBR" : 15
\end{verbatim}

\end{solutionbox}
\begin{mnemonicbox}
``DIAGRAM: Directivity Improves Antenna Gain,
Radiation And More''

\end{mnemonicbox}
\subsection*{Question 2(a) [3 marks]}\label{q2a}

\textbf{Explain principle of pattern multiplication.}

\begin{solutionbox}

Pattern multiplication states that the radiation pattern of an array
equals the product of the element pattern and the array factor.

\textbf{Diagram}:

\begin{center}
\textbf{Mermaid Diagram (Code)}
\begin{verbatim}
{Shaded}
{Highlighting}[]
graph LR
    A[Array Pattern] {-{-}{} B["Element Pattern  Array Factor"]}
    B {-{-}{} C[Total Field Pattern]}
    C {-{-}{} D[Directivity Enhancement]}
{Highlighting}
{Shaded}
\end{verbatim}
\end{center}

\begin{itemize}
\tightlist
\item
  \textbf{Element pattern}: Radiation pattern of single element
\item
  \textbf{Array factor}: Pattern due to arrangement of elements
\item
  \textbf{Result}: Sharper beams, higher directivity
\end{itemize}

\end{solutionbox}
\begin{mnemonicbox}
``PEAM: Pattern Equals Array times Element Method''

\end{mnemonicbox}
\subsection*{Question 2(b) [4 marks]}\label{q2b}

\textbf{Draw \& Explain Loop antenna.}

\begin{solutionbox}

A loop antenna is a closed-circuit antenna consisting of one or more
complete turns of wire.

\textbf{Diagram}:

\begin{verbatim}
      ┌───────────┐
      │           │
      │           │
      │           │
  feed│           │
   ┌──┴──┐        │
   │     │        │
   └─────┘        │
      │           │
      │           │
      └───────────┘
\end{verbatim}

\begin{itemize}
\tightlist
\item
  \textbf{Small loop}: Circumference \textless{} λ/10, figure-8 pattern
\item
  \textbf{Large loop}: Circumference \approx λ, maximum radiation
  perpendicular to plane
\item
  \textbf{Applications}: Direction finding, AM radio reception
\item
  \textbf{Radiation resistance}: Proportional to (circumference/λ)^{4} for
  small loops
\end{itemize}

\end{solutionbox}
\begin{mnemonicbox}
``LOOP: Low Output, Orientation Precise''

\end{mnemonicbox}
\subsection*{Question 2(c) [7 marks]}\label{q2c}

\textbf{Design a Yagi-uda antenna and explain it.}

\begin{solutionbox}

Yagi-Uda is a directional antenna with driven element, reflector, and
directors.


{\def\LTcaptype{none} % do not increment counter
\vspace{-5pt}
\captionof{table}{Yagi-Uda Antenna Design Guidelines}
\vspace{-10pt}
\begin{longtable}[]{@{}lll@{}}
\toprule\noalign{}
Element & Length & Spacing from Driven Element \\
\midrule\noalign{}
\endhead
\bottomrule\noalign{}
\endlastfoot
Reflector & 0.5λ \times 1.05 & 0.15λ - 0.25λ \\
Driven Element & 0.5λ & Reference point \\
Director 1 & 0.5λ \times 0.95 & 0.1λ - 0.15λ \\
Director 2 & 0.5λ \times 0.92 & 0.2λ - 0.3λ \\
Additional Directors & Decreasing & 0.3λ - 0.4λ \\
\end{longtable}
}

\textbf{Diagram}:

\begin{verbatim}
          Director 2     Director 1     Driven      Reflector
             ┌┐             ┌┐          Element        ┌┐
             ││             ││            ┌┐           ││
             ││             ││            ││           ││
             ││             ││            ││           ││
     ────────┴┴─────────────┴┴────────────┴┴───────────┴┴────────►
             │{{-}{-}0.15λ{-}{-}│{-}{-}0.15λ{-}{-}│{-}{-}0.25λ{-}{-}│      Radiation}
             │{{-}{-}{-}{-}{-}{-}{-}{-}{-}{-}{-} Boom Length {-}{-}{-}{-}{-}{-}{-}{-}{-}{-}│      Direction}
\end{verbatim}

\begin{itemize}
\tightlist
\item
  \textbf{Function}: Reflector reflects signal, directors guide it
  forward
\item
  \textbf{Gain}: Increases with number of directors (diminishing
  returns)
\item
  \textbf{Impedance}: 20-30 ohms (typically matched with balun)
\item
  \textbf{Applications}: TV reception, point-to-point communication
\end{itemize}

\end{solutionbox}
\begin{mnemonicbox}
``YARD: Yagi Achieves Radical Directivity''

\end{mnemonicbox}
\subsection*{Question 2(a) OR [3
marks]}\label{q2a}

\textbf{Compare broad fire and end fire array antenna.}

\begin{solutionbox}


{\def\LTcaptype{none} % do not increment counter
\vspace{-5pt}
\captionof{table}{Broad Side vs End Fire Array}
\vspace{-10pt}
\begin{longtable}[]{@{}
  >{\raggedright\arraybackslash}p{(\linewidth - 4\tabcolsep) * \real{0.2500}}
  >{\raggedright\arraybackslash}p{(\linewidth - 4\tabcolsep) * \real{0.3864}}
  >{\raggedright\arraybackslash}p{(\linewidth - 4\tabcolsep) * \real{0.3636}}@{}}
\toprule\noalign{}
\begin{minipage}[b]{\linewidth}\raggedright
Parameter
\end{minipage} & \begin{minipage}[b]{\linewidth}\raggedright
Broad Side Array
\end{minipage} & \begin{minipage}[b]{\linewidth}\raggedright
End Fire Array
\end{minipage} \\
\midrule\noalign{}
\endhead
\bottomrule\noalign{}
\endlastfoot
Direction of Maximum Radiation & Perpendicular to array axis & Along
array axis \\
Phase Difference & 0^\circ & 180^\circ \pm βd \\
Beam Width & Narrower & Wider \\
Directivity & Higher & Lower \\
Applications & Broadcasting & Point-to-point links \\
\end{longtable}
}

\textbf{Diagram}:

\begin{center}
\textbf{Mermaid Diagram (Code)}
\begin{verbatim}
{Shaded}
{Highlighting}[]
graph LR
    A[Array Antennas] {-{-}{} B[Broad Side]}
    A {-{-}{} C[End Fire]}
    B {-{-}{} D[Max radiation perpendicular{}br /{}to array axis]}
    C {-{-}{} E[Max radiation along{}br /{}array axis]}
{Highlighting}
{Shaded}
\end{verbatim}
\end{center}

\end{solutionbox}
\begin{mnemonicbox}
``BEPS: Broadside Emits Perpendicularly, Sideways''

\end{mnemonicbox}
\subsection*{Question 2(b) OR [4
marks]}\label{q2b}

\textbf{Draw \& Explain Folded dipole antenna.}

\begin{solutionbox}

A folded dipole consists of a half-wavelength dipole with its ends
folded back and connected, forming a narrow loop.

\textbf{Diagram}:

\begin{verbatim}
       λ/2
    ┌───────┐
    │       │
    │       │
    │       │
feed│       │
 ┌──┴──┐    │
 │     │    │
 └─────┘    │
    │       │
    │       │
    └───────┘
\end{verbatim}

\begin{itemize}
\tightlist
\item
  \textbf{Impedance}: 4 times higher than standard dipole (\approx300Ω)
\item
  \textbf{Bandwidth}: Wider than simple dipole
\item
  \textbf{Applications}: TV antennas, FM receiving antennas
\item
  \textbf{Advantage}: Less susceptible to noise
\end{itemize}

\end{solutionbox}
\begin{mnemonicbox}
``FIBER: Folded Impedance Booster Enhances
Reception''

\end{mnemonicbox}
\subsection*{Question 2(c) OR [7
marks]}\label{q2c}

\textbf{Give names of Non-resonant antennas and explain any one in
detail with its radiation pattern.}

\begin{solutionbox}

Non-resonant antennas include Rhombic, V antenna, Terminated folded
dipole, Beverage, and Long-wire antennas.

\textbf{Rhombic Antenna in Detail:}

\textbf{Diagram}:

\begin{verbatim}
                    ┌───────┐
                   /         {}
                  /           {}
                 /             {}
                /               {}
               /                 {}
              /                   {}
             /                     {}
            /                       {}
           /                         {}
    ┌─────┘                           └─────┐
    │                                       │
 ───┴───                                 ───┴───
 Feeder                               Terminating
                                       Resistor
\end{verbatim}


{\def\LTcaptype{none} % do not increment counter
\vspace{-5pt}
\captionof{table}{Rhombic Antenna Characteristics}
\vspace{-10pt}
\begin{longtable}[]{@{}ll@{}}
\toprule\noalign{}
Parameter & Description \\
\midrule\noalign{}
\endhead
\bottomrule\noalign{}
\endlastfoot
Structure & Four long wires arranged in rhombus shape \\
Termination & Resistive load at far end (non-resonant) \\
Directivity & High (8-15 dB) \\
Frequency Range & Wide bandwidth (multi-octave) \\
Radiation Pattern & Unidirectional, cone-shaped \\
Applications & HF point-to-point communications \\
\end{longtable}
}

\begin{itemize}
\tightlist
\item
  \textbf{Advantages}: High gain, broad bandwidth, simple construction
\item
  \textbf{Disadvantages}: Large physical size, power loss in terminating
  resistor
\item
  \textbf{Pattern}: Main lobe along major axis of rhombus
\end{itemize}

\end{solutionbox}
\begin{mnemonicbox}
``RHOMBIC: Reliable High-Output Multi-Band Impressive
Communications''

\end{mnemonicbox}
\subsection*{Question 3(a) [3 marks]}\label{q3a}

\textbf{Compare radiation pattern of different resonant wire antennas.}

\begin{solutionbox}


{\def\LTcaptype{none} % do not increment counter
\vspace{-5pt}
\captionof{table}{Radiation Patterns of Resonant Wire Antennas}
\vspace{-10pt}
\begin{longtable}[]{@{}llll@{}}
\toprule\noalign{}
Antenna Type & Pattern Shape & Directivity & Polarization \\
\midrule\noalign{}
\endhead
\bottomrule\noalign{}
\endlastfoot
Half-Wave Dipole & Figure-8 (donut) & 2.15 dBi & Linear \\
Full-Wave Dipole & Four-lobed & 3.8 dBi & Linear \\
3λ/2 Dipole & Six-lobed & 4.2 dBi & Linear \\
2λ Dipole & Eight-lobed & 4.5 dBi & Linear \\
\end{longtable}
}

\textbf{Diagram}:

\begin{center}
\textbf{Mermaid Diagram (Code)}
\begin{verbatim}
{Shaded}
{Highlighting}[]
graph TD
    A[Resonant Wire Antennas] {-{-}{} B[Half{-}Wave Dipole{}br /{}Figure{-}8 Pattern]}
    A {-{-}{} C[Full{-}Wave Dipole{}br /{}Four{-}lobed Pattern]}
    A {-{-}{} D[Multi{-}wavelength Dipole{}br /{}Multi{-}lobed Pattern]}
{Highlighting}
{Shaded}
\end{verbatim}
\end{center}

\end{solutionbox}
\begin{mnemonicbox}
``MOLD: More wavelengths create Lots of Directivity
lobes''

\end{mnemonicbox}
\subsection*{Question 3(b) [4 marks]}\label{q3b}

\textbf{Draw V and Inverted V antenna with radiation Pattern.}

\begin{solutionbox}

\textbf{Diagram: V-Antenna}

\begin{verbatim}
        
        
      /{}
     /  {}
    /    {}
   /      {}
  /        {}
 /          {}
Feed        Feed
Point       Point
    
Radiation Pattern: Bidirectional along axis
\end{verbatim}

\textbf{Diagram: Inverted V-Antenna}

\begin{verbatim}
         Feed
         Point
           |
           V
          / {}
         /   {}
        /     {}
       /       {}
      /         {}
     Ground     Ground
     
Radiation Pattern: Omnidirectional with slight elevation
\end{verbatim}

\begin{itemize}
\tightlist
\item
  \textbf{V-Antenna}: Two wires forming V-shape, bidirectional pattern
\item
  \textbf{Inverted V}: Half-wave dipole with arms drooping down,
  omnidirectional
\item
  \textbf{Applications}: Amateur radio, FM reception
\item
  \textbf{Advantages}: Simple, flexible installation options
\end{itemize}

\end{solutionbox}
\begin{mnemonicbox}
``VIPS: V-shapes Improve Pattern Selectivity''

\end{mnemonicbox}
\subsection*{Question 3(c) [7 marks]}\label{q3c}

\textbf{Explain Morse Code and Practice Oscillator.}

\begin{solutionbox}

Morse code is a method of transmitting text using standardized sequences
of dots and dashes.


{\def\LTcaptype{none} % do not increment counter
\vspace{-5pt}
\captionof{table}{Basic Morse Code Elements}
\vspace{-10pt}
\begin{longtable}[]{@{}lll@{}}
\toprule\noalign{}
Element & Timing & Sound \\
\midrule\noalign{}
\endhead
\bottomrule\noalign{}
\endlastfoot
Dot (.) & 1 unit & Short beep \\
Dash (-) & 3 units & Long beep \\
Space between elements & 1 unit & Short silence \\
Space between letters & 3 units & Medium silence \\
Space between words & 7 units & Long silence \\
\end{longtable}
}

\textbf{Diagram: Simple Morse Code Practice Oscillator}

\begin{verbatim}
      +9V
       |
       R1
       |
      ┌┴┐
 C1   │ │                    Speaker
┌──┬──┤8├──┬────┬─────────────┬──┐
│  │  │5│  │    │             │  │
│  │  │5│  │    R2            C2 │
│  │  │5│  │    │             │  │
│  │  └┬┘  │    │             │  │
│  │   │   │    │             │  │
└──┴───┴───┴────┴─────────────┴──┘
  Key        Ground
\end{verbatim}

\begin{itemize}
\tightlist
\item
  \textbf{Components}: 555 timer, resistors, capacitors, key, speaker
\item
  \textbf{Operation}: Key closing completes circuit, creating
  oscillation
\item
  \textbf{Frequency}: Typically 600-800 Hz (adjustable with R2)
\item
  \textbf{Applications}: Ham radio training, emergency communications
\end{itemize}

\end{solutionbox}
\begin{mnemonicbox}
``TEMPO: Timing Elements Make Perfect Oscillation''

\end{mnemonicbox}
\subsection*{Question 3(a) OR [3
marks]}\label{q3a}

\textbf{Draw and Explain Microstrip Patch antenna.}

\begin{solutionbox}

A microstrip patch antenna consists of a metal patch on a grounded
substrate.

\textbf{Diagram}:

\begin{verbatim}
    ┌───────────────┐  {-{-} Patch (metal)}
    │               │
    │               │  Thickness
    │               │  ↕  
════════════════════════ {-{-} Substrate}
    |               |
    |               |  {-{-} Ground plane}
    └───────────────┘
    
    ↑               ↑
    Feed            Radiation
    point
\end{verbatim}

\begin{itemize}
\tightlist
\item
  \textbf{Structure}: Metal patch on dielectric substrate with ground
  plane
\item
  \textbf{Advantages}: Low profile, lightweight, easy fabrication,
  conformable
\item
  \textbf{Disadvantages}: Narrow bandwidth, low efficiency, low power
  handling
\item
  \textbf{Applications}: Mobile devices, RFID, satellite communications
\end{itemize}

\end{solutionbox}
\begin{mnemonicbox}
``MAPS: Microstrip Antenna Patches are Simple''

\end{mnemonicbox}
\subsection*{Question 3(b) OR [4
marks]}\label{q3b}

\textbf{Draw and Explain Horn antenna.}

\begin{solutionbox}

A horn antenna is a waveguide with flared open end that directs radio
waves in a beam.

\textbf{Diagram}:

\begin{verbatim}
            ┌───────────┐
            │           │
            │           │
       ┌────┤           │
       │    │           │
       │    │           │
Feed   │    │           │
point  │    │           │
       │    │           │
       └────┤           │
            │           │
            │           │
            └───────────┘
         Waveguide       Horn
\end{verbatim}

\begin{itemize}
\tightlist
\item
  \textbf{Types}: E-plane, H-plane, Pyramidal, Conical
\item
  \textbf{Frequency range}: Microwave (1-20 GHz)
\item
  \textbf{Advantages}: High gain, wide bandwidth, low VSWR
\item
  \textbf{Applications}: Satellite communications, radar, radio
  astronomy
\end{itemize}

\end{solutionbox}
\begin{mnemonicbox}
``HEWB: Horns Enhance Waveguide Beamwidth''

\end{mnemonicbox}
\subsection*{Question 3(c) OR [7
marks]}\label{q3c}

\textbf{List different feed system for Parabolic reflector antenna and
explain any one.}

\begin{solutionbox}


{\def\LTcaptype{none} % do not increment counter
\vspace{-5pt}
\captionof{table}{Parabolic Reflector Feed Systems}
\vspace{-10pt}
\begin{longtable}[]{@{}
  >{\raggedright\arraybackslash}p{(\linewidth - 4\tabcolsep) * \real{0.3333}}
  >{\raggedright\arraybackslash}p{(\linewidth - 4\tabcolsep) * \real{0.2564}}
  >{\raggedright\arraybackslash}p{(\linewidth - 4\tabcolsep) * \real{0.4103}}@{}}
\toprule\noalign{}
\begin{minipage}[b]{\linewidth}\raggedright
Feed System
\end{minipage} & \begin{minipage}[b]{\linewidth}\raggedright
Position
\end{minipage} & \begin{minipage}[b]{\linewidth}\raggedright
Characteristics
\end{minipage} \\
\midrule\noalign{}
\endhead
\bottomrule\noalign{}
\endlastfoot
Front Feed & At focus, in front of dish & Simple, some blockage \\
Cassegrain & Secondary reflector with feed at center of dish & Reduced
noise, compact \\
Gregorian & Secondary concave reflector & Better gain, larger size \\
Offset Feed & Feed offset from main axis & No blockage, asymmetric \\
Waveguide Feed & Direct waveguide at focus & Simple, limited
flexibility \\
\end{longtable}
}

\textbf{Front Feed System (Detailed):}

\textbf{Diagram}:

\begin{center}
\textbf{Mermaid Diagram (Code)}
\begin{verbatim}
{Shaded}
{Highlighting}[]
graph LR
    A[Parabolic Reflector] {-{-}{-} B[Focal Point]}
    B {-{-}{-} C[Feed Horn]}
    C {-{-}{-} D[Waveguide/Coax]}
    D {-{-}{-} E[Receiver/Transmitter]}
{Highlighting}
{Shaded}
\end{verbatim}
\end{center}

\begin{itemize}
\tightlist
\item
  \textbf{Operation}: Feed placed at focal point, illuminates reflector
\item
  \textbf{Advantages}: Simple design, easy alignment, maximum efficiency
\item
  \textbf{Disadvantages}: Feed and support structure block part of
  aperture
\item
  \textbf{Applications}: Satellite dishes, radio telescopes, radar
\end{itemize}

\end{solutionbox}
\begin{mnemonicbox}
``FACTS: Focused Aperture Captures Transmitted
Signals''

\end{mnemonicbox}
\subsection*{Question 4(a) [3 marks]}\label{q4a}

\textbf{Explain working principle of HAM radio.}

\begin{solutionbox}

HAM radio (Amateur Radio) operates on designated frequency bands for
non-commercial communications.

\textbf{Diagram}:

\begin{center}
\textbf{Mermaid Diagram (Code)}
\begin{verbatim}
{Shaded}
{Highlighting}[]
graph LR
    A[Transmitter] {-{-}{} B[Antenna]}
    B {-{-}{} C[Propagation Medium]}
    C {-{-}{} D[Receiver Antenna]}
    D {-{-}{} E[Receiver]}
{Highlighting}
{Shaded}
\end{verbatim}
\end{center}

\begin{itemize}
\tightlist
\item
  \textbf{Operation}: Transmitter generates RF signal, antenna radiates
  signal
\item
  \textbf{Frequency bands}: HF (3-30 MHz), VHF (30-300 MHz), UHF
  (300-3000 MHz)
\item
  \textbf{Modes}: AM, FM, SSB, CW (Morse), digital modes
\item
  \textbf{License}: Required for legal operation (levels based on
  skills)
\end{itemize}

\end{solutionbox}
\begin{mnemonicbox}
``TEAM: Transmission Enables Amateur Messages''

\end{mnemonicbox}
\subsection*{Question 4(b) [4 marks]}\label{q4b}

\textbf{Explain Duct Propagation.}

\begin{solutionbox}

Duct propagation occurs when radio waves are trapped within atmospheric
layers with varying refractive indices.

\textbf{Diagram}:

\begin{verbatim}
        {-{-}{-}{-}{-}{-}{-}{-}{-}{-}{-}{-}{-}{-}{-}{-}{-}{-}  -{-} Upper atmosphere}
       
         ===============  {-{-} Duct layer (temperature inversion)}
        /     Trapped    {}
       /       waves      {}
      /                    {}
     /                      {}
    /                        {}
   /                          {}
  Transmitter                Receiver
  
      ==================== Ground/Sea
\end{verbatim}

\begin{itemize}
\tightlist
\item
  \textbf{Formation}: Temperature inversion creates refractive index
  gradient
\item
  \textbf{Frequency range}: VHF, UHF, microwave frequencies
\item
  \textbf{Advantages}: Extended communication range (beyond horizon)
\item
  \textbf{Occurrence}: Common over oceans, varies with weather
  conditions
\end{itemize}

\end{solutionbox}
\begin{mnemonicbox}
``TRIP: Trapped Rays In atmospheric Paths''

\end{mnemonicbox}
\subsection*{Question 4(c) [7 marks]}\label{q4c}

\textbf{Explain Tropospheric Scattered Propagation in detail.}

\begin{solutionbox}

Tropospheric scatter uses the scattering properties of the troposphere
to enable beyond-horizon communications.


{\def\LTcaptype{none} % do not increment counter
\vspace{-5pt}
\captionof{table}{Tropospheric Scatter Characteristics}
\vspace{-10pt}
\begin{longtable}[]{@{}
  >{\raggedright\arraybackslash}p{(\linewidth - 2\tabcolsep) * \real{0.4583}}
  >{\raggedright\arraybackslash}p{(\linewidth - 2\tabcolsep) * \real{0.5417}}@{}}
\toprule\noalign{}
\begin{minipage}[b]{\linewidth}\raggedright
Parameter
\end{minipage} & \begin{minipage}[b]{\linewidth}\raggedright
Description
\end{minipage} \\
\midrule\noalign{}
\endhead
\bottomrule\noalign{}
\endlastfoot
Mechanism & Forward scattering of radio waves by tropospheric
irregularities \\
Frequency Range & 300 MHz to 10 GHz (UHF/SHF) \\
Range & 100-800 km \\
Path Loss & High (requires high-power transmitters) \\
Reliability & Affected by weather conditions \\
\end{longtable}
}

\textbf{Diagram}:

\begin{center}
\textbf{Mermaid Diagram (Code)}
\begin{verbatim}
{Shaded}
{Highlighting}[]
graph LR
    A[Transmitter] {-{-}{} B[High Gain Antenna]}
    B {-{-}{} C[Scattering Volume{}br /{}in Troposphere]}
    C {-{-}{} D[Receiving Antenna]}
    D {-{-}{} E[Receiver]}
    F[Factors] {-{-}{} G[Weather]}
    F {-{-}{} H[Frequency]}
    F {-{-}{} I[Antenna Size]}
{Highlighting}
{Shaded}
\end{verbatim}
\end{center}

\begin{itemize}
\tightlist
\item
  \textbf{Mechanism}: Signal scattered by refractive index
  irregularities
\item
  \textbf{Equipment}: High-power transmitters, large antennas, sensitive
  receivers
\item
  \textbf{Applications}: Military, backup communications, remote areas
\item
  \textbf{Advantages}: Beyond line-of-sight, relatively stable
\end{itemize}

\end{solutionbox}
\begin{mnemonicbox}
``STARS: Scatter Tropospheric Allows Range beyond
Sight''

\end{mnemonicbox}
\subsection*{Question 4(a) OR [3
marks]}\label{q4a}

\textbf{Draw turnstile and super turnstile antenna.}

\begin{solutionbox}

\textbf{Diagram: Turnstile Antenna}

\begin{verbatim}
          │   │
     ─────┼───┼─────
          │   │
          │   │
     ─────┼───┼─────
          │   │

  Two dipoles at 90^ fed with 90^ phase difference
\end{verbatim}

\textbf{Diagram: Super Turnstile (Batwing) Antenna}

\begin{verbatim}
      ┌───┬───┐
      │   │   │
      │   │   │
      │   │   │
    ──┼───┼───┼──
      │   │   │
      │   │   │
      │   │   │
      └───┴───┘

    Multiple elements for broadband operation
\end{verbatim}

\begin{itemize}
\tightlist
\item
  \textbf{Turnstile}: Two dipoles at right angles, circular polarization
\item
  \textbf{Super turnstile}: Multiple elements for increased bandwidth
\item
  \textbf{Applications}: TV broadcasting, FM broadcasting, satellite
  communications
\item
  \textbf{Advantage}: Omnidirectional horizontal pattern
\end{itemize}

\end{solutionbox}
\begin{mnemonicbox}
``TACO: Turnstile Antennas Create Omnidirectional
patterns''

\end{mnemonicbox}
\subsection*{Question 4(b) OR [4
marks]}\label{q4b}

\textbf{Give full form of MUF, LUF and OUF.}

\begin{solutionbox}


{\def\LTcaptype{none} % do not increment counter
\vspace{-5pt}
\captionof{table}{Ionospheric Propagation Parameters}
\vspace{-10pt}
\begin{longtable}[]{@{}
  >{\raggedright\arraybackslash}p{(\linewidth - 4\tabcolsep) * \real{0.3684}}
  >{\raggedright\arraybackslash}p{(\linewidth - 4\tabcolsep) * \real{0.2895}}
  >{\raggedright\arraybackslash}p{(\linewidth - 4\tabcolsep) * \real{0.3421}}@{}}
\toprule\noalign{}
\begin{minipage}[b]{\linewidth}\raggedright
Abbreviation
\end{minipage} & \begin{minipage}[b]{\linewidth}\raggedright
Full Form
\end{minipage} & \begin{minipage}[b]{\linewidth}\raggedright
Description
\end{minipage} \\
\midrule\noalign{}
\endhead
\bottomrule\noalign{}
\endlastfoot
MUF & Maximum Usable Frequency & Highest frequency that can be reflected
by ionosphere \\
LUF & Lowest Usable Frequency & Lowest frequency providing adequate
signal-to-noise ratio \\
OUF & Optimum Usable Frequency & Best working frequency (85\% of MUF) \\
\end{longtable}
}

\textbf{Diagram}:

\begin{center}
\textbf{Mermaid Diagram (Code)}
\begin{verbatim}
{Shaded}
{Highlighting}[]
graph TD
    A[Ionospheric Frequencies] {-{-}{} B[MUF]}
    A {-{-}{} C[LUF]}
    A {-{-}{} D[OUF]}
    B {-{-}{} E[Highest frequency{}br /{}that returns to Earth]}
    C {-{-}{} F[Lowest frequency{}br /{}with adequate SNR]}
    D {-{-}{} G[Best working frequency{}br /{}85\% of MUF]}
{Highlighting}
{Shaded}
\end{verbatim}
\end{center}

\end{solutionbox}
\begin{mnemonicbox}
``MLO: Maximum and Lowest determine Optimum''

\end{mnemonicbox}
\subsection*{Question 4(c) OR [7
marks]}\label{q4c}

\textbf{Explain virtual height, critical frequency and skip distance in
detail.}

\begin{solutionbox}


{\def\LTcaptype{none} % do not increment counter
\vspace{-5pt}
\captionof{table}{Key Ionospheric Propagation Parameters}
\vspace{-10pt}
\begin{longtable}[]{@{}
  >{\raggedright\arraybackslash}p{(\linewidth - 4\tabcolsep) * \real{0.2973}}
  >{\raggedright\arraybackslash}p{(\linewidth - 4\tabcolsep) * \real{0.3243}}
  >{\raggedright\arraybackslash}p{(\linewidth - 4\tabcolsep) * \real{0.3784}}@{}}
\toprule\noalign{}
\begin{minipage}[b]{\linewidth}\raggedright
Parameter
\end{minipage} & \begin{minipage}[b]{\linewidth}\raggedright
Definition
\end{minipage} & \begin{minipage}[b]{\linewidth}\raggedright
Significance
\end{minipage} \\
\midrule\noalign{}
\endhead
\bottomrule\noalign{}
\endlastfoot
Virtual Height & Apparent reflection height assuming straight-line
propagation & Determines maximum communication range \\
Critical Frequency & Maximum frequency reflected at vertical incidence &
Indicates ionization density \\
Skip Distance & Minimum distance where ionospheric signals can be
received & Creates ``skip zones'' with no reception \\
\end{longtable}
}

\textbf{Diagram}:

\begin{verbatim}
                  /|{}
                 / | {}
                /  |  {       Critical freq: Maximum}
               /   |   {      frequency at 90^ incidence}
              /    |    {}
             /     |     {}
            /      |      {}
Transmitter/       |       {Receiver}
           /       |        {}
          /        |         {}
         /         |          {}
        /          |           {}
       /           |            {}
      /            |             {}
     /             |              {}
    /              |               {}
   /               |                {}
  /                |                 {}
 /                 |                  {}
Earth              |                  Earth
       |{{-}{-}{-}{-}{-}{-} Skip Distance {-}{-}{-}{-}{-}{-}{-}{-}|}

Virtual height: Apparent reflection height
\end{verbatim}

\begin{itemize}
\tightlist
\item
  \textbf{Virtual height}: Typically 300-400 km for F layer, varies with
  time/season
\item
  \textbf{Critical frequency}: Usually 5-10 MHz for F2 layer, depends on
  solar activity
\item
  \textbf{Skip distance}: Given by D = 2h tan θ, where h is virtual
  height and θ is incidence angle
\end{itemize}

\end{solutionbox}
\begin{mnemonicbox}
``VCS: Virtual height Controls Skip distance''

\end{mnemonicbox}
\subsection*{Question 5(a) [3 marks]}\label{q5a}

\textbf{With neat figure show different Ionosphere layers.}

\begin{solutionbox}

\textbf{Diagram: Ionospheric Layers}

\begin{verbatim}
Height (km)
   \^{}
   |
400|                   F2 Layer
   |           {-{-}{-}{-}{-}{-}{-}{-}{-}{-}{-}{-}{-}{-}{-}{-}{-}{-}{-}{-}{-}{-}}
   |
300|                   F1 Layer (daytime)
   |           {-{-}{-}{-}{-}{-}{-}{-}{-}{-}{-}{-}{-}{-}{-}{-}{-}{-}{-}{-}{-}{-}}
   |
200|                   E Layer
   |           {-{-}{-}{-}{-}{-}{-}{-}{-}{-}{-}{-}{-}{-}{-}{-}{-}{-}{-}{-}{-}{-}}
   |
100|                   D Layer
   |           {-{-}{-}{-}{-}{-}{-}{-}{-}{-}{-}{-}{-}{-}{-}{-}{-}{-}{-}{-}{-}{-}}
   |
   +{-{-}{-}{-}{-}{-}{-}{-}{-}{-}{-}{-}{-}{-}{-}{-}{-}{-}{-}{-}{-}{-}{-}{-}{-}{-}{-}{-}{-}{-}{-}{-}{-}{-}{-}{-}{-}}
                 Electron Density
\end{verbatim}

\begin{itemize}
\tightlist
\item
  \textbf{D Layer}: 60-90 km, absorbs HF waves, disappears at night
\item
  \textbf{E Layer}: 90-150 km, reflects MF/lower HF, weakens at night
\item
  \textbf{F1 Layer}: 150-220 km, present in daytime only
\item
  \textbf{F2 Layer}: 220-400 km, main reflection layer, present
  day/night
\end{itemize}

\end{solutionbox}
\begin{mnemonicbox}
``DEAF: Down to up - D, E, And F layers''

\end{mnemonicbox}
\subsection*{Question 5(b) [4 marks]}\label{q5b}

\textbf{Give names of different types of satellite communication systems
and compare it.}

\begin{solutionbox}


{\def\LTcaptype{none} % do not increment counter
\vspace{-5pt}
\captionof{table}{Satellite Communication Systems}
\vspace{-10pt}
\begin{longtable}[]{@{}
  >{\raggedright\arraybackslash}p{(\linewidth - 6\tabcolsep) * \real{0.2167}}
  >{\raggedright\arraybackslash}p{(\linewidth - 6\tabcolsep) * \real{0.2667}}
  >{\raggedright\arraybackslash}p{(\linewidth - 6\tabcolsep) * \real{0.2333}}
  >{\raggedright\arraybackslash}p{(\linewidth - 6\tabcolsep) * \real{0.2833}}@{}}
\toprule\noalign{}
\begin{minipage}[b]{\linewidth}\raggedright
System Type
\end{minipage} & \begin{minipage}[b]{\linewidth}\raggedright
Frequency Bands
\end{minipage} & \begin{minipage}[b]{\linewidth}\raggedright
Applications
\end{minipage} & \begin{minipage}[b]{\linewidth}\raggedright
Characteristics
\end{minipage} \\
\midrule\noalign{}
\endhead
\bottomrule\noalign{}
\endlastfoot
Telecommunication & C, Ku, Ka bands & Phone, data, internet & Global
coverage, high capacity \\
Broadcasting & Ku, C bands & TV, radio transmission & High power, wide
coverage \\
Data Communication & L, S, Ka bands & IoT, VSAT, M2M & Low to medium
data rates \\
Military & X, EHF bands & Secure communications & Encrypted,
jam-resistant \\
Navigation & L band & GPS, GLONASS, Galileo & Precise timing,
positioning \\
\end{longtable}
}

\textbf{Diagram}:

\begin{verbatim}
pie
    title "Satellite Communication Systems"
    "Telecommunication" : 30
    "Broadcasting" : 25
    "Data Communication" : 20
    "Military" : 15
    "Navigation" : 10
\end{verbatim}

\end{solutionbox}
\begin{mnemonicbox}
``TBDMN: Telecom, Broadcasting, Data, Military,
Navigation''

\end{mnemonicbox}
\subsection*{Question 5(c) [7 marks]}\label{q5c}

\textbf{Draw and explain DTH receiver system.}

\begin{solutionbox}

DTH (Direct-to-Home) system delivers television programming directly to
viewers via satellite.

\textbf{Diagram}:

\begin{verbatim}
                     TV
                     |
                     V
                 Set{-top Box}
                     |
                     V
                  LNB/LNBF {{-}{-}{-}{-} Satellite signals}
                     |
                     V
                 Dish Antenna
                  (0.6{-1.2m)}
\end{verbatim}


{\def\LTcaptype{none} % do not increment counter
\vspace{-5pt}
\captionof{table}{DTH System Components}
\vspace{-10pt}
\begin{longtable}[]{@{}
  >{\raggedright\arraybackslash}p{(\linewidth - 4\tabcolsep) * \real{0.3056}}
  >{\raggedright\arraybackslash}p{(\linewidth - 4\tabcolsep) * \real{0.2778}}
  >{\raggedright\arraybackslash}p{(\linewidth - 4\tabcolsep) * \real{0.4167}}@{}}
\toprule\noalign{}
\begin{minipage}[b]{\linewidth}\raggedright
Component
\end{minipage} & \begin{minipage}[b]{\linewidth}\raggedright
Function
\end{minipage} & \begin{minipage}[b]{\linewidth}\raggedright
Specifications
\end{minipage} \\
\midrule\noalign{}
\endhead
\bottomrule\noalign{}
\endlastfoot
Dish Antenna & Collects satellite signals & 45-120 cm diameter \\
LNB (Low Noise Block) & Converts high frequency to lower IF & Noise
figure: 0.3-1.0 dB \\
Coaxial Cable & Carries IF signal to receiver & RG-6 type, 75 ohm \\
Set-top Box & Demodulates/decodes signals & MPEG-2/4 decoder \\
TV Set & Displays programming & HDMI/Component input \\
\end{longtable}
}

\begin{itemize}
\tightlist
\item
  \textbf{Frequency}: Ku-band (10.7-12.75 GHz) or C-band (3.7-4.2 GHz)
\item
  \textbf{Modulation}: QPSK or 8PSK digital modulation
\item
  \textbf{Signal processing}: Digital compression (MPEG-2/4)
\item
  \textbf{Features}: EPG (Electronic Program Guide), PVR (recording)
\end{itemize}

\end{solutionbox}
\begin{mnemonicbox}
``DOCS: Dish Obtains, Converts and Shows signals''

\end{mnemonicbox}
\subsection*{Question 5(a) OR [3
marks]}\label{q5a}

\textbf{What is the Need of Smart Antennas? Write its applications.}

\begin{solutionbox}

Smart antennas use adaptive signal processing to dynamically optimize
radiation patterns.

\textbf{Needs}:

\begin{itemize}
\tightlist
\item
  Increased capacity in congested networks
\item
  Improved signal quality and coverage
\item
  Reduced interference and multipath fading
\item
  Enhanced spectral efficiency
\end{itemize}

\textbf{Diagram}:

\begin{center}
\textbf{Mermaid Diagram (Code)}
\begin{verbatim}
{Shaded}
{Highlighting}[]
graph TD
    A[Smart Antenna] {-{-}{} B[Adaptive{}br /{}Beamforming]}
    A {-{-}{} C[Spatial{}br /{}Multiplexing]}
    A {-{-}{} D[Interference{}br /{}Suppression]}
{Highlighting}
{Shaded}
\end{verbatim}
\end{center}

\textbf{Applications}:

\begin{itemize}
\tightlist
\item
  Mobile communication networks (4G/5G)
\item
  MIMO systems for high data rates
\item
  Radar systems with enhanced target detection
\item
  Wireless LANs with improved coverage
\end{itemize}

\end{solutionbox}
\begin{mnemonicbox}
``SAFE: Smart Antennas For Efficiency''

\end{mnemonicbox}
\subsection*{Question 5(b) OR [4
marks]}\label{q5b}

\textbf{Explain Kepler's 3rd law.}

\begin{solutionbox}

Kepler's 3rd law relates the orbital period of a satellite to its
semi-major axis.

\textbf{Formula}: T^{2} = (4π^{2}/GM) \times a^{3}

Where:

\begin{itemize}
\tightlist
\item
  T = orbital period
\item
  a = semi-major axis
\item
  G = gravitational constant
\item
  M = mass of central body
\end{itemize}

\textbf{Diagram}:

\begin{center}
\textbf{Mermaid Diagram (Code)}
\begin{verbatim}
{Shaded}
{Highlighting}[]
graph LR
    A[Kepler{s 3rd Law] {-}{-}{} B["T^{2} ∝ a^{3}"]}
    B {-{-}{} C[T = orbital period]}
    B {-{-}{} D[a = semi{-}major axis]}
    E[Applications] {-{-}{} F[Satellite orbit determination]}
    E {-{-}{} G[Spacecraft mission planning]}
{Highlighting}
{Shaded}
\end{verbatim}
\end{center}

\begin{itemize}
\tightlist
\item
  \textbf{Meaning}: Larger orbits have longer periods
\item
  \textbf{Application}: Determines satellite orbit characteristics
\item
  \textbf{Geostationary orbit}: Period = 24 hours, altitude \approx 35,786 km
\end{itemize}

\end{solutionbox}
\begin{mnemonicbox}
``CAP: Cube of Axis equals Period squared''

\end{mnemonicbox}
\subsection*{Question 5(c) OR [7
marks]}\label{q5c}

\textbf{Identify the different types of Antennas for Terrestrial Mobile
communication and explain in detail.}

\begin{solutionbox}


{\def\LTcaptype{none} % do not increment counter
\vspace{-5pt}
\captionof{table}{Terrestrial Mobile Communication Antennas}
\vspace{-10pt}
\begin{longtable}[]{@{}
  >{\raggedright\arraybackslash}p{(\linewidth - 6\tabcolsep) * \real{0.2500}}
  >{\raggedright\arraybackslash}p{(\linewidth - 6\tabcolsep) * \real{0.2500}}
  >{\raggedright\arraybackslash}p{(\linewidth - 6\tabcolsep) * \real{0.2500}}
  >{\raggedright\arraybackslash}p{(\linewidth - 6\tabcolsep) * \real{0.2500}}@{}}
\toprule\noalign{}
\begin{minipage}[b]{\linewidth}\raggedright
Antenna Type
\end{minipage} & \begin{minipage}[b]{\linewidth}\raggedright
Typical Gain
\end{minipage} & \begin{minipage}[b]{\linewidth}\raggedright
Polarization
\end{minipage} & \begin{minipage}[b]{\linewidth}\raggedright
Applications
\end{minipage} \\
\midrule\noalign{}
\endhead
\bottomrule\noalign{}
\endlastfoot
Base Station Antennas & 10-18 dBi & Vertical/Dual & Cell towers, fixed
infrastructure \\
Mobile Station Antennas & 0-3 dBi & Vertical & Smartphones, vehicles,
portable devices \\
Repeater Antennas & 5-10 dBi & Circular/Dual & Signal boosting, coverage
extension \\
Diversity Antennas & Variable & Multiple & Multipath mitigation, MIMO
systems \\
\end{longtable}
}

\textbf{Base Station Antennas (Detailed)}:

\textbf{Diagram}:

\begin{verbatim}
        ┌────┐
        │    │
        │    │
        │    │      Array of
        │    │     radiating
        │    │     elements
        │    │
        │    │
        │    │
        │    │
        └────┘
          |
        Sector coverage
\end{verbatim}

\begin{itemize}
\tightlist
\item
  \textbf{Types}: Panel arrays, collinear arrays, sector antennas
\item
  \textbf{Characteristics}:

  \begin{itemize}
  \tightlist
  \item
    High gain (10-18 dBi)
  \item
    Directional radiation pattern (60^\circ-120^\circ sectors)
  \item
    Downtilt capability (electrical/mechanical)
  \item
    Multiple-band operation
  \end{itemize}
\item
  \textbf{Advanced features}:

  \begin{itemize}
  \tightlist
  \item
    Multiple-input multiple-output (MIMO)
  \item
    Remote electrical tilt (RET)
  \item
    Integrated diplexers/triplexers
  \end{itemize}
\end{itemize}

\textbf{Mobile Station Antennas}:

\begin{itemize}
\tightlist
\item
  Compact size (internal/external)
\item
  Omnidirectional pattern
\item
  Multiple band support (700-2600 MHz)
\item
  Implementations: PIFA, helical, monopole designs
\end{itemize}

\end{solutionbox}
\begin{mnemonicbox}
``BEST: Base-stations Employ Sector Technology''

\end{mnemonicbox}

\end{document}
