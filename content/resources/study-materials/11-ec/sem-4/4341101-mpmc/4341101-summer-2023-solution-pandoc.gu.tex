\documentclass[10pt,a4paper]{article}

% content/resources/templates/preamble.tex
\usepackage[margin=0.6in]{geometry}
\author{Milav Dabgar}
\usepackage{amsmath,amssymb,amsthm}
\usepackage{booktabs}
\usepackage{multirow}
\usepackage{xcolor}
\usepackage{tcolorbox}
\tcbuselibrary{breakable,skins}
\usepackage[colorlinks=true,linkcolor=blue]{hyperref}
\usepackage{titlesec}
\usepackage{enumitem}
\usepackage{tikz}
\usepackage{pgfplots}
\usepackage{circuitikz}
\usepackage[version=4]{mhchem}
\usepackage{longtable}
\usepackage{array}
\usepackage{float}
\usepackage{caption}
\usepackage{listings}

\lstset{
  basicstyle=\small\ttfamily,
  breaklines=true,
  breakatwhitespace=false,
  postbreak=\mbox{\textcolor{red}{$\hookrightarrow$}\space},
  float=false,
  numbers=left,
  numberstyle=\tiny\color{gray},
  numbersep=10pt,
  xleftmargin=2em,
  keywordstyle=\color{blue},
  commentstyle=\color{green!60!black},
  stringstyle=\color{purple},
  backgroundcolor=\color{gray!5},
  showstringspaces=false,
  tabsize=2,
  captionpos=b,
  keepspaces=true,
  columns=flexible
}

\pgfplotsset{compat=1.18}
\usetikzlibrary{shapes,arrows,positioning,calc,patterns,decorations.pathmorphing,decorations.markings,arrows.meta}

% Color scheme
\definecolor{headcolor}{RGB}{0,102,204}
\definecolor{keycolor}{RGB}{220,20,60}
\definecolor{solutioncolor}{RGB}{34,139,34}
\definecolor{mnemoniccolor}{RGB}{148,0,211}
\definecolor{codecolor}{RGB}{0,0,100}

% Spacing
\setlength{\parskip}{3pt}
\setlist[itemize]{nosep}
\setlist[enumerate]{nosep}

% Title formatting
\titleformat{\section}{\Large\bfseries\color{headcolor}}{\thesection}{1em}{}
\titleformat{\subsection}{\large\bfseries\color{headcolor}}{\thesubsection}{1em}{}

% Pandoc tightlist compatibility
\providecommand{\tightlist}{%
  \setlength{\itemsep}{0pt}\setlength{\parskip}{0pt}}

% Pandoc longtable compatibility
\newcounter{none}
\def\thenone{}


% content/resources/templates/gujarati-boxes.tex
\usepackage{fontspec}
\usepackage{polyglossia}

% Set Gujarati as main language (document is primarily in Gujarati)
% Note: gloss-gujarati.ldf doesn't exist in polyglossia, but it will use hyphenation patterns
\setdefaultlanguage{gujarati}
\setotherlanguage{english}

% Configure Gujarati font properly
% Use Language=Default to prevent polyglossia from trying to add language-specific features
% that don't exist for Gujarati, which causes "empty feature" warnings
\newfontfamily\gujaratifont[Script=Gujarati,AutoFakeBold=2.5,AutoFakeSlant=0.3]{Noto Sans Gujarati}
\setmainfont[Script=Gujarati,AutoFakeBold=2.5,AutoFakeSlant=0.3]{Noto Sans Gujarati}
% Use Noto Sans Gujarati for monospace to support Gujarati in text
\setmonofont[Scale=0.9]{Noto Sans Gujarati}

% Configure English to use the same font
\newfontfamily\englishfont[Script=Gujarati,AutoFakeBold=2.5,AutoFakeSlant=0.3]{Noto Sans Gujarati}

% Translations for polyglossia
\gappto\captionsgujarati{
  \renewcommand{\tablename}{કોષ્ટક}
  \renewcommand{\figurename}{આકૃતિ}
}

% Helper for TikZ nodes to ensure Gujarati font
\newcommand{\gu}[1]{{\gujaratifont #1}}

% Custom environments
\newtcolorbox{solutionbox}{
    breakable,
    enhanced,
    colback=solutioncolor!5!white,
    colframe=solutioncolor!75!black,
    fonttitle=\bfseries,
    title=જવાબ
}

\newtcolorbox{solutionboxnobreak}{
 colback=solutioncolor!5!white,
 colframe=solutioncolor!75!black,
 fonttitle=\bfseries,
 title=જવાબ
}

\newtcolorbox{keyformula}{
 breakable,
 enhanced,
 colback=keycolor!5!white,
 colframe=keycolor!75!black,
 fonttitle=\bfseries,
 title=રાસાયણિક સમીકરણ/સૂત્ર
}

\newtcolorbox{mnemonicbox}{
 breakable,
 enhanced,
 colback=mnemoniccolor!5!white,
 colframe=mnemoniccolor!75!black,
 fonttitle=\bfseries,
 title=મેમરી ટ્રીક
}


\begin{document}

\begin{center}
{\Huge\bfseries\color{headcolor} Subject Name (Gujarati)}\\[5pt]
{\LARGE 4341101 -- Summer 2023}\\[3pt]
{\large Semester 1 Study Material}\\[3pt]
{\normalsize\textit{Detailed Solutions and Explanations}}
\end{center}

\vspace{10pt}

\subsection*{પ્રશ્ન 1(અ) [3
ગુણ]}\label{uxaaauxab0uxab6uxaa8-1uxa85-3-uxa97uxaa3}

\textbf{માઇક્રોપ્રોસેસર અને માઇક્રોકન્ટ્રોલરની સરખામણી કરો.}

\begin{solutionbox}

{\def\LTcaptype{none} % do not increment counter
\begin{longtable}[]{@{}lll@{}}
\toprule\noalign{}
ફીચર & માઇક્રોપ્રોસેસર & માઇક્રોકન્ટ્રોલર \\
\midrule\noalign{}
\endhead
\bottomrule\noalign{}
\endlastfoot
વ્યાખ્યા & એકલ ચિપ પર CPU & એકલ ચિપ પર સંપૂર્ણ કમ્પ્યુટર \\
મેમરી & બાહ્ય RAM/ROM જરૂરી & અંદર જ RAM/ROM \\
ઉપયોગો & સામાન્ય કમ્પ્યુટિંગ, PC & એમ્બેડેડ સિસ્ટમ, IoT \\
ઉદાહરણો & Intel 8085, 8086 & 8051, Arduino, PIC \\
કિંમત & વધારે & ઓછી \\
\end{longtable}
}

\end{solutionbox}
\begin{mnemonicbox}
``પ્રોસેસર રામ માંગે, કન્ટ્રોલર રામ રાખે'' (પ્રોસેસરને બહારથી
રામ જોઈએ, કંટ્રોલરમાં રામ અંદર જ હોય છે)

\end{mnemonicbox}
\subsection*{પ્રશ્ન 1(બ) [4
ગુણ]}\label{uxaaauxab0uxab6uxaa8-1uxaac-4-uxa97uxaa3}

\textbf{RISC અને CISC ની સરખામણી કરો.}

\begin{solutionbox}

{\def\LTcaptype{none} % do not increment counter
\begin{longtable}[]{@{}
  >{\raggedright\arraybackslash}p{(\linewidth - 4\tabcolsep) * \real{0.1011}}
  >{\raggedright\arraybackslash}p{(\linewidth - 4\tabcolsep) * \real{0.4494}}
  >{\raggedright\arraybackslash}p{(\linewidth - 4\tabcolsep) * \real{0.4494}}@{}}
\toprule\noalign{}
\begin{minipage}[b]{\linewidth}\raggedright
ફીચર
\end{minipage} & \begin{minipage}[b]{\linewidth}\raggedright
RISC (રિડ્યુસ્ડ ઇન્સ્ટ્રક્શન સેટ કમ્પ્યુટર)
\end{minipage} & \begin{minipage}[b]{\linewidth}\raggedright
CISC (કોમ્પ્લેક્સ ઇન્સ્ટ્રક્શન સેટ કમ્પ્યુટર)
\end{minipage} \\
\midrule\noalign{}
\endhead
\bottomrule\noalign{}
\endlastfoot
ઇન્સ્ટ્રક્શન & થોડી, સરળ ઇન્સ્ટ્રક્શન & ઘણી, જટિલ ઇન્સ્ટ્રક્શન \\
એક્ઝિક્યુશન ટાઈમ & ફિક્સ્ડ (1 ક્લોક સાયકલ) & વેરિએબલ (ઘણી સાયકલ) \\
મેમરી એક્સેસ & માત્ર લોડ/સ્ટોર દ્વારા & ઘણા મેમરી એક્સેસ મોડ \\
પાઇપલાઇનિંગ & સરળ અમલીકરણ & મુશ્કેલ અમલીકરણ \\
ઉદાહરણો & ARM, MIPS & Intel x86, 8085 \\
હાર્ડવેર & સરળ, ઓછા ટ્રાન્ઝિસ્ટર & જટિલ, વધુ ટ્રાન્ઝિસ્ટર \\
રજિસ્ટર સેટ & વધુ રજિસ્ટર & ઓછા રજિસ્ટર \\
\end{longtable}
}

\end{solutionbox}
\begin{mnemonicbox}
``RISC ઝડપી, CISC બહોળું'' (RISC ઝડપી હોય છે, CISC માં
ઘણી ઇન્સ્ટ્રક્શન હોય છે)

\end{mnemonicbox}
\subsection*{પ્રશ્ન 1(ક) [7
ગુણ]}\label{uxaaauxab0uxab6uxaa8-1uxa95-7-uxa97uxaa3}

\textbf{વ્યાખ્યાયિત કરો: માઇક્રોપ્રોસેસર, ઓપરેન્ડ, ઈન્સ્ટ્રક્શન સાયકલ, ઓપકોડ, ALU,
મશીન સાયકલ, ટી-સ્ટેટ}

\begin{solutionbox}

{\def\LTcaptype{none} % do not increment counter
\begin{longtable}[]{@{}
  >{\raggedright\arraybackslash}p{(\linewidth - 2\tabcolsep) * \real{0.3333}}
  >{\raggedright\arraybackslash}p{(\linewidth - 2\tabcolsep) * \real{0.6667}}@{}}
\toprule\noalign{}
\begin{minipage}[b]{\linewidth}\raggedright
શબ્દ
\end{minipage} & \begin{minipage}[b]{\linewidth}\raggedright
વ્યાખ્યા
\end{minipage} \\
\midrule\noalign{}
\endhead
\bottomrule\noalign{}
\endlastfoot
\textbf{માઇક્રોપ્રોસેસર} & એક ઇન્ટિગ્રેટેડ સર્કિટ પર CPU જે ઇન્સ્ટ્રક્શન પ્રોસેસ કરે
છે \\
\textbf{ઓપરેન્ડ} & ઇન્સ્ટ્રક્શનમાં વપરાતી ડેટા વેલ્યુ \\
\textbf{ઈન્સ્ટ્રક્શન સાયકલ} & ઇન્સ્ટ્રક્શન ફેચ, ડિકોડ અને એક્ઝિક્યુટની સંપૂર્ણ
પ્રક્રિયા \\
\textbf{ઓપકોડ} & ઓપરેશન કોડ જે CPU ને કહે છે કે કયું ઓપરેશન કરવાનું છે \\
\textbf{ALU} & અર્થમેટિક લોજિક યુનિટ જે ગણિત ઓપરેશન કરે છે \\
\textbf{મશીન સાયકલ} & મૂળભૂત ઓપરેશન જેમ કે મેમરી રીડ/રાઈટ (ઇન્સ્ટ્રક્શન સાયકલનો
ભાગ) \\
\textbf{ટી-સ્ટેટ} & ટાઈમ સ્ટેટ - પ્રોસેસરમાં સમયનો સૌથી નાનો એકમ (ક્લોક
પીરિયડ) \\
\end{longtable}
}

\textbf{ડાયાગ્રામ:}

\begin{verbatim}
+{-{-}{-}{-}{-}{-}{-}{-}{-}{-}{-}{-}{-}{-}{-}{-}{-}{-}{-}+     +{-}{-}{-}{-}{-}{-}{-}{-}{-}{-}{-}{-}{-}{-}{-}{-}{-}{-}{-}+     +{-}{-}{-}{-}{-}{-}{-}{-}{-}{-}{-}{-}{-}{-}{-}{-}{-}{-}{-}+}
| FETCH             |     | DECODE            |     | EXECUTE           |
| Get instruction   |{-{-}{-}{-}| Interpret opcode  |{-}{-}{-}{-}| Perform operation |}
| from memory       |     | Identify operands |     | Store results     |
+{-{-}{-}{-}{-}{-}{-}{-}{-}{-}{-}{-}{-}{-}{-}{-}{-}{-}{-}+     +{-}{-}{-}{-}{-}{-}{-}{-}{-}{-}{-}{-}{-}{-}{-}{-}{-}{-}{-}+     +{-}{-}{-}{-}{-}{-}{-}{-}{-}{-}{-}{-}{-}{-}{-}{-}{-}{-}{-}+}
           \^{                                                 |}
           |                                                 |
           +{-{-}{-}{-}{-}{-}{-}{-}{-}{-}{-}{-}{-}{-}{-}{-}{-}{-}{-}{-}{-}{-}{-}{-}{-}{-}{-}{-}{-}{-}{-}{-}{-}{-}{-}{-}{-}{-}{-}{-}{-}{-}{-}{-}{-}{-}{-}{-}{-}+}
                        INSTRUCTION CYCLE
\end{verbatim}

\end{solutionbox}
\begin{mnemonicbox}
``મારો ઓલ્ડ Intel ચિપ ઓનલી મેક્સ ટ્રબલ'' (માઇક્રોપ્રોસેસર,
ઓપરેન્ડ, ઇન્સ્ટ્રક્શન, ઓપકોડ, ALU, મશીન, ટી-સ્ટેટ)

\end{mnemonicbox}
\subsection*{પ્રશ્ન 1(ક OR) [7
ગુણ]}\label{uxaaauxab0uxab6uxaa8-1uxa95-or-7-uxa97uxaa3}

\textbf{વોન-ન્યુમેન અને હાર્વર્ડ આર્કિટેક્ચરની તુલના કરો.}

\begin{solutionbox}

{\def\LTcaptype{none} % do not increment counter
\begin{longtable}[]{@{}
  >{\raggedright\arraybackslash}p{(\linewidth - 4\tabcolsep) * \real{0.1579}}
  >{\raggedright\arraybackslash}p{(\linewidth - 4\tabcolsep) * \real{0.4561}}
  >{\raggedright\arraybackslash}p{(\linewidth - 4\tabcolsep) * \real{0.3860}}@{}}
\toprule\noalign{}
\begin{minipage}[b]{\linewidth}\raggedright
ફીચર
\end{minipage} & \begin{minipage}[b]{\linewidth}\raggedright
વોન-ન્યુમેન આર્કિટેક્ચર
\end{minipage} & \begin{minipage}[b]{\linewidth}\raggedright
હાર્વર્ડ આર્કિટેક્ચર
\end{minipage} \\
\midrule\noalign{}
\endhead
\bottomrule\noalign{}
\endlastfoot
મેમરી બસ & ઇન્સ્ટ્રક્શન અને ડેટા માટે એક જ મેમરી બસ & પ્રોગ્રામ અને ડેટા મેમરી માટે અલગ
બસ \\
એક્ઝિક્યુશન & સિક્વેન્શિયલ એક્ઝિક્યુશન & પેરેલલ ફેચ અને એક્ઝિક્યુટ શક્ય \\
સ્પીડ & બસ બોટલનેક ને કારણે ધીમું & સમાંતર એક્સેસને કારણે ઝડપી \\
જટિલતા & સરળ ડિઝાઇન & વધુ જટિલ ડિઝાઇન \\
ઉપયોગો & સામાન્ય કમ્પ્યુટિંગ & DSP, માઇક્રોકન્ટ્રોલર, એમ્બેડેડ સિસ્ટમ \\
સિક્યોરિટી & ઓછી સુરક્ષિત (કોડ ડેટા તરીકે બદલી શકાય) & વધુ સુરક્ષિત (કોડ ડેટાથી
અલગ) \\
ઉદાહરણ & મોટાભાગના PC, 8085, 8086 & 8051, PIC, ARM Cortex-M \\
\end{longtable}
}

\textbf{ડાયાગ્રામ:}

\begin{verbatim}
Von{-Neumann:                    Harvard:}
+{-{-}{-}{-}{-}{-}{-}{-}{-}+                     +{-}{-}{-}{-}{-}{-}{-}{-}{-}+}
|         |                     |         |
|   CPU   |{{-}{-}{-}{-}| Memory |     |   CPU   |{-}{-}{-}{-}| Program Memory |}
|         |                     |         |
+{-{-}{-}{-}{-}{-}{-}{-}{-}+                     +{-}{-}{-}{-}{-}{-}{-}{-}{-}+}
                                     \^{}
                                     |
                                     v
                              | Data Memory |
\end{verbatim}

\end{solutionbox}
\begin{mnemonicbox}
``હાર્વર્ડ હંમેશા અલગ રસ્તા રાખે'' (હાર્વર્ડમાં મેમરી પાથ અલગ
હોય છે)

\end{mnemonicbox}
\subsection*{પ્રશ્ન 2(અ) [3
ગુણ]}\label{uxaaauxab0uxab6uxaa8-2uxa85-3-uxa97uxaa3}

\textbf{8085 માઇક્રોપ્રોસેસરનું ફ્લેગ રજીસ્ટર દોરો અને તેને સમજાવો.}

\begin{solutionbox}

\textbf{ડાયાગ્રામ:}

\begin{verbatim}
+{-{-}{-}+{-}{-}{-}+{-}{-}{-}+{-}{-}{-}+{-}{-}{-}+{-}{-}{-}+{-}{-}{-}+{-}{-}{-}+}
| S | Z | {- | AC| {-} | P | {-} | CY|}
+{-{-}{-}+{-}{-}{-}+{-}{-}{-}+{-}{-}{-}+{-}{-}{-}+{-}{-}{-}+{-}{-}{-}+{-}{-}{-}+}
  7   6   5   4   3   2   1   0
      FLAG REGISTER (F)
\end{verbatim}

{\def\LTcaptype{none} % do not increment counter
\begin{longtable}[]{@{}lll@{}}
\toprule\noalign{}
ફ્લેગ & નામ & કાર્ય \\
\midrule\noalign{}
\endhead
\bottomrule\noalign{}
\endlastfoot
S & સાઈન & જો પરિણામ નેગેટિવ હોય તો સેટ થાય (બિટ 7=1) \\
Z & ઝીરો & જો પરિણામ ઝીરો હોય તો સેટ થાય \\
AC & ઓક્ઝિલિયરી કૅરી & જો બિટ 3 થી બિટ 4 માં કૅરી થાય તો સેટ થાય \\
P & પેરિટી & જો પરિણામમાં ઇવન પેરિટી હોય તો સેટ થાય \\
CY & કૅરી & જો બિટ 7 થી કૅરી કે બોરો થાય તો સેટ થાય \\
\end{longtable}
}

\end{solutionbox}
\begin{mnemonicbox}
``સરસ ઝોમ્બી આજે પણ ચાલે'' (સાઈન, ઝીરો, ઓક્ઝિલિયરી,
પેરિટી, કૅરી)

\end{mnemonicbox}
\subsection*{પ્રશ્ન 2(બ) [4
ગુણ]}\label{uxaaauxab0uxab6uxaa8-2uxaac-4-uxa97uxaa3}

\textbf{8085 માઇક્રોપ્રોસેસર માટે એડ્રેસ અને ડેટાબસોનું ડી-મલ્ટીપ્લેક્સીંગ સમજાવો.}

\begin{solutionbox}

\textbf{ડાયાગ્રામ:}

\begin{verbatim}
                   +{-{-}{-}{-}{-}{-}{-}{-}{-}{-}+}
A15{-A8 {-}{-}{-}{-}{-}{-}{-}{-}{-}{-}{-}|          |}
                   |          |{-{-}{-}{-}{-}{-}{-}{-}{-} A15{-}A8 (Higher Address)}
                   |          |
AD7{-AD0 {-}{-}{-}{-}{-}{-}{-}{-}{-}| 8085 CPU |{-}{-}{-}{-}+}
                   |          |     |
                   |          |     |    +{-{-}{-}{-}{-}{-}{-}{-}+}
                   +{-{-}{-}{-}{-}{-}{-}{-}{-}{-}+     +{-}{-}{-}| 74LS373|{-}{-}{-}{-} A7{-}A0 (Lower Address)}
                        |                | Latch  |
                        |                +{-{-}{-}{-}{-}{-}{-}{-}+}
                        |                    \^{}
                        |                    |
                     ALE {-{-}{-}{-}{-}{-}{-}{-}{-}{-}{-}{-}{-}{-}{-}{-}{-}{-}{-}{-}}
\end{verbatim}

\begin{itemize}
\tightlist
\item
  \textbf{જરૂરિયાત}: 8085 માં પિન બચાવવા માટે મલ્ટીપ્લેક્સ્ડ પિન (AD0-AD7) હોય છે
\item
  \textbf{પ્રક્રિયા}:

  \begin{enumerate}
  \tightlist
  \item
    CPU એડ્રેસ AD0-AD7 પિન પર મૂકે છે
  \item
    ALE (એડ્રેસ લેચ એનેબલ) સિગ્નલ HIGH થાય છે
  \item
    એડ્રેસ લેચ (74LS373) લોઅર એડ્રેસ બિટ્સ પકડે છે
  \item
    ALE LOW થાય છે, એડ્રેસ લેચ થઈ જાય છે
  \item
    AD0-AD7 પિન હવે ડેટા ટ્રાન્સફર માટે ફ્રી થઈ જાય છે
  \end{enumerate}
\end{itemize}

\end{solutionbox}
\begin{mnemonicbox}
``ALE પહેલા, ડેટા પછી'' (એડ્રેસ લેચ એનેબલ પહેલા એડ્રેસ પકડે,
પછી ડેટા આવે)

\end{mnemonicbox}
\subsection*{પ્રશ્ન 2(ક) [7
ગુણ]}\label{uxaaauxab0uxab6uxaa8-2uxa95-7-uxa97uxaa3}

\textbf{આકૃતિની મદદથી 8085 માઇક્રોપ્રોસેસરના આર્કિટેક્ચરનું વર્ણન કરો.}

\begin{solutionbox}

\textbf{ડાયાગ્રામ:}

\begin{verbatim}
    +{-{-}{-}{-}{-}{-}{-}{-}{-}{-}{-}{-}{-}{-}{-}{-}{-}{-}{-}{-}{-}{-}{-}{-}{-}{-}{-}{-}{-}{-}{-}{-}{-}{-}{-}{-}{-}{-}{-}{-}{-}{-}{-}+}
    |               8085 CPU                    |
    |                                           |
    |  +{-{-}{-}{-}{-}{-}{-}{-}{-}{-}{-}{-}{-}+       +{-}{-}{-}{-}{-}{-}{-}{-}{-}{-}{-}{-}{-}{-}+   |}
    |  | REGISTERS   |       | CONTROL UNIT |   |
    |  |{-{-}{-}{-}{-}{-}{-}{-}{-}{-}{-}{-}{-}|       |{-}{-}{-}{-}{-}{-}{-}{-}{-}{-}{-}{-}{-}{-}|   |}
    |  | A (Accum.)  |{{-}{-}{-}{-}{-}| Instruction  |   |}
    |  | B,C,D,E,H,L |       | Decoder      |   |
    |  | SP, PC      |       |              |   |
    |  | Flags       |       | Timing \&     |   |
    |  +{-{-}{-}{-}{-}{-}{-}{-}{-}{-}{-}{-}{-}+       | Control      |   |}
    |       \^{                +{-}{-}{-}{-}{-}{-}{-}{-}{-}{-}{-}{-}{-}{-}+   |}
    |       |                      \^{            |}
    |       v                      |            |
    |  +{-{-}{-}{-}{-}{-}{-}{-}{-}{-}{-}{-}{-}+             |            |}
    |  |    ALU      |{{-}{-}{-}{-}{-}{-}{-}{-}{-}{-}{-}{-}+            |}
    |  +{-{-}{-}{-}{-}{-}{-}{-}{-}{-}{-}{-}{-}+                          |}
    |       \^{                                   |}
    +{-{-}{-}{-}{-}{-}{-}|{-}{-}{-}{-}{-}{-}{-}{-}{-}{-}{-}{-}{-}{-}{-}{-}{-}{-}{-}{-}{-}{-}{-}{-}{-}{-}{-}{-}{-}{-}{-}{-}{-}{-}{-}+}
            |
    +{-{-}{-}{-}{-}{-}{-}|{-}{-}{-}{-}{-}{-}{-}{-}{-}{-}{-}{-}{-}{-}{-}{-}{-}{-}{-}{-}{-}{-}{-}{-}{-}{-}{-}{-}{-}{-}{-}{-}{-}{-}{-}+}
    |       v                                   |
    |  +{-{-}{-}{-}{-}{-}{-}{-}{-}{-}{-}{-}{-}+       +{-}{-}{-}{-}{-}{-}{-}{-}{-}{-}{-}{-}{-}{-}+   |}
    |  | Address Bus |{-{-}{-}{-}{-}{-}| Memory \&     |   |}
    |  +{-{-}{-}{-}{-}{-}{-}{-}{-}{-}{-}{-}{-}+       | I/O Devices  |   |}
    |  | Data Bus    |{{-}{-}{-}{-}{-}|              |   |}
    |  +{-{-}{-}{-}{-}{-}{-}{-}{-}{-}{-}{-}{-}+       +{-}{-}{-}{-}{-}{-}{-}{-}{-}{-}{-}{-}{-}{-}+   |}
    +{-{-}{-}{-}{-}{-}{-}{-}{-}{-}{-}{-}{-}{-}{-}{-}{-}{-}{-}{-}{-}{-}{-}{-}{-}{-}{-}{-}{-}{-}{-}{-}{-}{-}{-}{-}{-}{-}{-}{-}{-}{-}{-}+}
\end{verbatim}

\begin{itemize}
\tightlist
\item
  \textbf{મુખ્ય ઘટકો}:

  \begin{itemize}
  \tightlist
  \item
    \textbf{રજિસ્ટર્સ}: સ્ટોરેજ લોકેશન (A, B-L, SP, PC, Flags)
  \item
    \textbf{ALU}: ગાણિતિક અને લોજિકલ ઓપરેશન કરે છે
  \item
    \textbf{કંટ્રોલ યુનિટ}: ટાઈમિંગ અને કંટ્રોલ સિગ્નલ જનરેટ કરે છે
  \item
    \textbf{બસ}: એડ્રેસ બસ (16-bit), ડેટા બસ (8-bit), કંટ્રોલ બસ
  \end{itemize}
\item
  \textbf{મુખ્ય ફીચર્સ}:

  \begin{itemize}
  \tightlist
  \item
    8-બિટ ડેટા બસ, 16-બિટ એડ્રેસ બસ (64KB એડ્રેસેબલ મેમરી)
  \item
    6 જનરલ-પર્પઝ રજિસ્ટર (B,C,D,E,H,L) અને એક્યુમુલેટર
  \item
    5 ફ્લેગ્સ સ્ટેટસ માહિતી માટે
  \end{itemize}
\end{itemize}

\end{solutionbox}
\begin{mnemonicbox}
``RABC'' - ``રજિસ્ટર, ALU, બસ, કંટ્રોલ'' (મુખ્ય ઘટકો)

\end{mnemonicbox}
\subsection*{પ્રશ્ન 2(અ OR) [3
ગુણ]}\label{uxaaauxab0uxab6uxaa8-2uxa85-or-3-uxa97uxaa3}

\textbf{8085 માઇક્રોપ્રોસેસરનું બસ ઓર્ગેનાઈઝેશન સમજાવો.}

\begin{solutionbox}

{\def\LTcaptype{none} % do not increment counter
\begin{longtable}[]{@{}lll@{}}
\toprule\noalign{}
બસ પ્રકાર & વિડ્થ & કાર્ય \\
\midrule\noalign{}
\endhead
\bottomrule\noalign{}
\endlastfoot
\textbf{એડ્રેસ બસ} & 16-બિટ (A0-A15) & મેમરી/I/O ડિવાઈસ એડ્રેસ લઈ જાય છે \\
\textbf{ડેટા બસ} & 8-બિટ (D0-D7) & CPU અને મેમરી/I/O વચ્ચે ડેટા ટ્રાન્સફર કરે
છે \\
\textbf{કંટ્રોલ બસ} & વિવિધ સિગ્નલ્સ & સિસ્ટમ ઓપરેશન કોઓર્ડિનેટ કરે છે \\
\end{longtable}
}

\textbf{મુખ્ય કંટ્રોલ સિગ્નલ્સ}:

\begin{itemize}
\tightlist
\item
  \textbf{RD̅}: રીડ સિગ્નલ (એક્ટિવ લો)
\item
  \textbf{WR̅}: રાઈટ સિગ્નલ (એક્ટિવ લો)
\item
  \textbf{ALE}: એડ્રેસ લેચ એનેબલ
\item
  \textbf{IO/M̅}: I/O (હાઈ) અને મેમરી (લો) ઓપરેશન વચ્ચે ભેદ પાડે છે
\end{itemize}

\textbf{ડાયાગ્રામ:}

\begin{verbatim}
                    +{-{-}{-}{-}{-}{-}{-}{-}{-}{-}+}
                    |   8085   |
                    |   CPU    |
                    +{-{-}{-}{-}{-}{-}{-}{-}{-}{-}+}
                       |  |  |
           +{-{-}{-}{-}{-}{-}{-}{-}{-}{-}{-}+  |  +{-}{-}{-}{-}{-}{-}{-}{-}{-}{-}{-}{-}+}
           |              |               |
    +{-{-}{-}{-}{-}{-}v{-}{-}{-}{-}{-}+  +{-}{-}{-}{-}{-}v{-}{-}{-}{-}{-}{-}+  +{-}{-}{-}{-}{-}v{-}{-}{-}{-}{-}{-}{-}+}
    | Address Bus|  |  Data Bus  |  | Control Bus |
    | (16{-bit)   |  |  (8{-}bit)   |  | RD,WR,ALE   |}
    +{-{-}{-}{-}{-}{-}{-}{-}{-}{-}{-}{-}+  +{-}{-}{-}{-}{-}{-}{-}{-}{-}{-}{-}{-}+  +{-}{-}{-}{-}{-}{-}{-}{-}{-}{-}{-}{-}{-}+}
\end{verbatim}

\end{solutionbox}
\begin{mnemonicbox}
``ADC'' - ``એડ્રેસ શોધે, ડેટા ફરે, કંટ્રોલ ચલાવે''

\end{mnemonicbox}
\subsection*{પ્રશ્ન 2(બ OR) [4
ગુણ]}\label{uxaaauxab0uxab6uxaa8-2uxaac-or-4-uxa97uxaa3}

\textbf{સમજાવો: પ્રોગ્રામ કાઉન્ટર અને સ્ટેક પોઈન્ટર}

\begin{solutionbox}

{\def\LTcaptype{none} % do not increment counter
\begin{longtable}[]{@{}
  >{\raggedright\arraybackslash}p{(\linewidth - 4\tabcolsep) * \real{0.3846}}
  >{\raggedright\arraybackslash}p{(\linewidth - 4\tabcolsep) * \real{0.2308}}
  >{\raggedright\arraybackslash}p{(\linewidth - 4\tabcolsep) * \real{0.3846}}@{}}
\toprule\noalign{}
\begin{minipage}[b]{\linewidth}\raggedright
રજિસ્ટર
\end{minipage} & \begin{minipage}[b]{\linewidth}\raggedright
સાઈઝ
\end{minipage} & \begin{minipage}[b]{\linewidth}\raggedright
કાર્ય
\end{minipage} \\
\midrule\noalign{}
\endhead
\bottomrule\noalign{}
\endlastfoot
\textbf{પ્રોગ્રામ કાઉન્ટર (PC)} & 16-બિટ & આગલા એક્ઝિક્યુટ થનાર ઇન્સ્ટ્રક્શનનું એડ્રેસ
રાખે છે \\
\textbf{સ્ટેક પોઈન્ટર (SP)} & 16-બિટ & મેમરીમાં સ્ટેકના ટોપને પોઇન્ટ કરે છે \\
\end{longtable}
}

\textbf{પ્રોગ્રામ કાઉન્ટર (PC)}:

\begin{itemize}
\tightlist
\item
  ઇન્સ્ટ્રક્શન ફેચ પછી ઓટોમેટિક વધે છે
\item
  જમ્પ/કોલ ઇન્સ્ટ્રક્શન દ્વારા બદલાય છે
\item
  પ્રોગ્રામ એક્ઝિક્યુશન સિક્વેન્સ કંટ્રોલ કરે છે
\item
  રીસેટ પર 0000H પર સેટ થાય છે
\end{itemize}

\textbf{સ્ટેક પોઈન્ટર (SP)}:

\begin{itemize}
\tightlist
\item
  સ્ટેક પર છેલ્લે પુશ કરેલ ડેટા આઈટમને પોઇન્ટ કરે છે
\item
  સ્ટેક LIFO (લાસ્ટ ઇન ફર્સ્ટ આઉટ) પ્રમાણે કામ કરે છે
\item
  સબરૂટિન કોલ અને ઇન્ટરપ્ટ દરમિયાન વપરાય છે
\item
  સ્ટેક મેમરીમાં નીચે તરફ વધે છે (ઘટાડાય છે)
\end{itemize}

\textbf{ડાયાગ્રામ:}

\begin{verbatim}
Memory:            PC:             SP:
+{-{-}{-}{-}{-}{-}{-}{-}{-}+        +{-}{-}{-}{-}{-}{-}{-}{-}+      +{-}{-}{-}{-}{-}{-}{-}{-}+}
| Instr 1 |{{-}{-}{-}{-}{-}{-}{-}| 2001H  |      | 3FFEH  |{-}{-}{-}+}
+{-{-}{-}{-}{-}{-}{-}{-}{-}+        +{-}{-}{-}{-}{-}{-}{-}{-}+      +{-}{-}{-}{-}{-}{-}{-}{-}+   |}
| Instr 2 |                                     |
+{-{-}{-}{-}{-}{-}{-}{-}{-}+        Stack:                       |}
    ...            ...                          |
+{-{-}{-}{-}{-}{-}{-}{-}+        +{-}{-}{-}{-}{-}{-}{-}{-}+                    |}
| Data 1 |        | Empty  |                    |
+{-{-}{-}{-}{-}{-}{-}{-}+        +{-}{-}{-}{-}{-}{-}{-}{-}+                    |}
| Data 2 |{{-}{-}{-}{-}{-}{-}{-}| Data A |{-}{-}{-}{-}{-}{-}{-}{-}{-}{-}{-}{-}{-}{-}{-}{-}{-}{-}{-}+}
+{-{-}{-}{-}{-}{-}{-}{-}+        +{-}{-}{-}{-}{-}{-}{-}{-}+}
\end{verbatim}

\end{solutionbox}
\begin{mnemonicbox}
``PC આગળ જુએ, SP સ્ટેક સંભાળે'' (PC આગલું ઇન્સ્ટ્રક્શન જુએ છે, SP
સ્ટેક મેનેજ કરે છે)

\end{mnemonicbox}
\subsection*{પ્રશ્ન 2(ક OR) [7
ગુણ]}\label{uxaaauxab0uxab6uxaa8-2uxa95-or-7-uxa97uxaa3}

\textbf{આકૃતિની મદદથી 8085 માઇક્રોપ્રોસેસરના પિન ડાયાગ્રામનું વર્ણન કરો.}

\begin{solutionbox}

\textbf{ડાયાગ્રામ:}

\begin{verbatim}
                 +{-{-}{-}{-}{-}{-}{-}{-}{-}{-}{-}{-}{-}{-}{-}{-}{-}{-}{-}+}
        +5V {-{-}{-}{-}|                   |{-}{-}{-}{-} GND}
                 |                   |
         X1 {-{-}{-}{-}|                   |{-}{-}{-}{-} X2}
                 |                   |
      RESET {-{-}{-}{-}|                   |{-}{-}{-}{-} READY}
                 |                   |
      HOLD {-{-}{-}{-}{-}|                   |{-}{-}{-}{-} CLK OUT}
                 |                   |
     HLDA {{-}{-}{-}{-}{-}{-}|      8085         |{-}{-}{-}{-} RESET IN}
                 |                   |
   INTR {-{-}{-}{-}{-}{-}{-}{-}|                   |{-}{-}{-}{-} RST 7.5}
                 |                   |
   INTA {{-}{-}{-}{-}{-}{-}{-}{-}|                   |{-}{-}{-}{-} RST 6.5}
                 |                   |
      SOD {{-}{-}{-}{-}{-}{-}|                   |{-}{-}{-}{-} RST 5.5}
                 |                   |
      SID {-{-}{-}{-}{-}{-}|                   |{-}{-}{-}{-} TRAP}
                 |                   |
     RD {{-}{-}{-}{-}{-}{-}{-}{-}|                   |}
                 |                   |
     WR {{-}{-}{-}{-}{-}{-}{-}{-}|                   |}
                 |                   |
    IO/M {{-}{-}{-}{-}{-}{-}{-}|                   |}
                 |                   |
     ALE {{-}{-}{-}{-}{-}{-}{-}|                   |}
                 |                   |
      S1 {{-}{-}{-}{-}{-}{-}{-}|                   |}
                 |                   |
      S0 {{-}{-}{-}{-}{-}{-}{-}|                   |}
                 |                   |
    A15{-A8 {-}{-}{-}{-}{-}|                   |}
                 |                   |
  AD7{-AD0 {-}{-}{-}{-}{-}|                   |}
                 +{-{-}{-}{-}{-}{-}{-}{-}{-}{-}{-}{-}{-}{-}{-}{-}{-}{-}{-}+}
\end{verbatim}

\textbf{પિન ગ્રુપ્સ}:

\begin{enumerate}
\tightlist
\item
  \textbf{પાવર \& ક્લોક}: Vcc, GND, X1, X2, CLK
\item
  \textbf{એડ્રેસ/ડેટા}: A8-A15, AD0-AD7 (મલ્ટીપ્લેક્સ્ડ)
\item
  \textbf{કંટ્રોલ}: ALE, RD̅, WR̅, IO/M̅
\item
  \textbf{ઇન્ટરપ્ટ}: INTR, INTA, RST 5.5/6.5/7.5, TRAP
\item
  \textbf{DMA}: HOLD, HLDA
\item
  \textbf{સિરિયલ I/O}: SID, SOD
\item
  \textbf{સ્ટેટસ}: S0, S1
\end{enumerate}

\end{solutionbox}
\begin{mnemonicbox}
``PACI-DHS'' (પાવર, એડ્રેસ, કંટ્રોલ, ઇન્ટરપ્ટ, DMA, હાર્ડવેર
સ્ટેટસ, સિરિયલ)

\end{mnemonicbox}
\subsection*{પ્રશ્ન 3(અ) [3
ગુણ]}\label{uxaaauxab0uxab6uxaa8-3uxa85-3-uxa97uxaa3}

\textbf{સ્ટેક, સ્ટેક પોઈન્ટર અને સ્ટેક ઓપરેશન સમજાવો.}

\begin{solutionbox}

{\def\LTcaptype{none} % do not increment counter
\begin{longtable}[]{@{}ll@{}}
\toprule\noalign{}
શબ્દ & વ્યાખ્યા \\
\midrule\noalign{}
\endhead
\bottomrule\noalign{}
\endlastfoot
\textbf{સ્ટેક} & LIFO ક્રમમાં કામચલાઉ સ્ટોરેજ માટે વપરાતી મેમરી એરિયા \\
\textbf{સ્ટેક પોઈન્ટર} & 16-બિટ રજિસ્ટર જે સ્ટેકમાં ટોપ આઈટમને પોઇન્ટ કરે છે \\
\textbf{PUSH} & ડેટાને સ્ટેક પર સ્ટોર કરવાનું ઓપરેશન (SP ઘટે છે) \\
\textbf{POP} & સ્ટેક પરથી ડેટા મેળવવાનું ઓપરેશન (SP વધે છે) \\
\end{longtable}
}

\textbf{ડાયાગ્રામ:}

\begin{verbatim}
Memory:                Stack Operations:

Before PUSH:           PUSH B (76H):           POP H:
+{-{-}{-}{-}{-}{-}{-}{-}+ {-}{-} 2000H   +{-}{-}{-}{-}{-}{-}{-}{-}+              +{-}{-}{-}{-}{-}{-}{-}{-}+}
|        |             |        |              |        |
+{-{-}{-}{-}{-}{-}{-}{-}+ {-}{-} 1FFFH   +{-}{-}{-}{-}{-}{-}{-}{-}+              +{-}{-}{-}{-}{-}{-}{-}{-}+}
|        |             |   76H  | {{-}{-} SP       |        | {-}{-} SP}
+{-{-}{-}{-}{-}{-}{-}{-}+ {-}{-} 1FFEH   +{-}{-}{-}{-}{-}{-}{-}{-}+              +{-}{-}{-}{-}{-}{-}{-}{-}+}
|        | {{-}{-} SP      |        |              |        |}
+{-{-}{-}{-}{-}{-}{-}{-}+             +{-}{-}{-}{-}{-}{-}{-}{-}+              +{-}{-}{-}{-}{-}{-}{-}{-}+}
\end{verbatim}

\end{solutionbox}
\begin{mnemonicbox}
``LIFO પુશ-પોપ કરે'' (છેલ્લો અંદર-પહેલો બહાર, પુશ અને પોપ
ઓપરેશન સાથે)

\end{mnemonicbox}
\subsection*{પ્રશ્ન 3(બ) [4
ગુણ]}\label{uxaaauxab0uxab6uxaa8-3uxaac-4-uxa97uxaa3}

\textbf{8051 માઇક્રોકન્ટ્રોલરનો ટાઈમર્સ/કાઉન્ટર્સનો લોજિક ડાયાગ્રામ દોરો અને તેને
સમજાવો.}

\begin{solutionbox}

\textbf{ડાયાગ્રામ:}

\begin{verbatim}
                            +{-{-}{-}{-}{-}{-}{-}{-}{-}{-}+}
Control Bits {-{-}{-}+           |          |}
                |           |  Control |
                +{-{-}{-}{-}{-}{-}{-}{-}{-}{-}|  Logic   |}
                            |          |
                            +{-{-}{-}{-}{-}{-}{-}{-}{-}{-}+}
                                 |
                                 v
Clock {-{-}{-}{-}{-}{-}{-} Prescaler {-}{-}{-} TLx {-}{-}{-} THx {-}{-}{-} TFx (Overflow Flag)}
                                 \^{}
External Input {-{-}{-}{-}{-}{-}{-}{-}{-}{-}{-}{-}{-}{-}{-}{-}{-}{-}|}
                            (Counter Mode)

                  |{-{-}{-}{-}Timer/Counter x{-}{-}{-}|}
\end{verbatim}

\begin{itemize}
\tightlist
\item
  \textbf{8051 માં 2 16-બિટ ટાઈમર/કાઉન્ટર છે}: ટાઈમર 0 અને ટાઈમર 1
\item
  \textbf{દરેક ટાઈમરમાં બે 8-બિટ રજિસ્ટર છે}: THx (હાઈ બાઈટ) અને TLx (લો બાઈટ)
\item
  \textbf{4 ઓપરેટિંગ મોડ્સ}:

  \begin{itemize}
  \tightlist
  \item
    મોડ 0: 13-બિટ ટાઈમર
  \item
    મોડ 1: 16-બિટ ટાઈમર
  \item
    મોડ 2: 8-બિટ ઓટો-રિલોડ
  \item
    મોડ 3: સ્પ્લિટ ટાઈમર મોડ
  \end{itemize}
\item
  \textbf{બે ફંક્શન્સ}:

  \begin{itemize}
  \tightlist
  \item
    ટાઈમર: આંતરિક ક્લોક પલ્સ ગણે છે
  \item
    કાઉન્ટર: બાહ્ય ઘટનાઓની ગણતરી કરે છે
  \end{itemize}
\end{itemize}

\end{solutionbox}
\begin{mnemonicbox}
``TIME-C'' (ટાઈમર ઈનપુટ, મોડ સિલેક્ટ, એક્સટર્નલ કાઉન્ટ)

\end{mnemonicbox}
\subsection*{પ્રશ્ન 3(ક) [7
ગુણ]}\label{uxaaauxab0uxab6uxaa8-3uxa95-7-uxa97uxaa3}

\textbf{આકૃતિની મદદથી 8051 માઇક્રોકન્ટ્રોલરનો પિન ડાયાગ્રામ સમજાવો.}

\begin{solutionbox}

\textbf{ડાયાગ્રામ:}

\begin{verbatim}
                 +{-{-}{-}{-}{-}{-}{-}{-}{-}{-}{-}{-}{-}{-}{-}{-}{-}{-}{-}+}
        VCC {-{-}{-}{-}|                   |{-}{-}{-}{-} GND}
                 |                   |
        P1.0 {{-}{-}|                   |{-}{-} P3.0 (RXD)}
        P1.1 {{-}{-}|                   |{-}{-} P3.1 (TXD)}
        P1.2 {{-}{-}|                   |{-}{-} P3.2 (INT0)}
        P1.3 {{-}{-}|                   |{-}{-} P3.3 (INT1)}
        P1.4 {{-}{-}|                   |{-}{-} P3.4 (T0)}
        P1.5 {{-}{-}|                   |{-}{-} P3.5 (T1)}
        P1.6 {{-}{-}|      8051         |{-}{-} P3.6 (WR)}
        P1.7 {{-}{-}|                   |{-}{-} P3.7 (RD)}
                 |                   |
        RST {-{-}{-}{-}|                   |{-}{-}{-}{-} XTAL2}
                 |                   |
        P2.0 {{-}{-}|                   |{-}{-}{-}{-} XTAL1}
        P2.1 {{-}{-}|                   |{-}{-} P0.0 (AD0)}
        P2.2 {{-}{-}|                   |{-}{-} P0.1 (AD1)}
        P2.3 {{-}{-}|                   |{-}{-} P0.2 (AD2)}
        P2.4 {{-}{-}|                   |{-}{-} P0.3 (AD3)}
        P2.5 {{-}{-}|                   |{-}{-} P0.4 (AD4)}
        P2.6 {{-}{-}|                   |{-}{-} P0.5 (AD5)}
        P2.7 {{-}{-}|                   |{-}{-} P0.6 (AD6)}
                 |                   |{{-}{-} P0.7 (AD7)}
        PSEN {-{-}{-}|                   |}
                 |                   |
        ALE {-{-}{-}{-}|                   |{-}{-}{-}{-} EA}
                 +{-{-}{-}{-}{-}{-}{-}{-}{-}{-}{-}{-}{-}{-}{-}{-}{-}{-}{-}+}
\end{verbatim}

\textbf{પિન ગ્રુપ્સ}:

\begin{enumerate}
\tightlist
\item
  \textbf{પોર્ટ પિન્સ}:

  \begin{itemize}
  \tightlist
  \item
    P0 (પોર્ટ 0): 8-બિટ બિડાયરેક્શનલ, મલ્ટીપ્લેક્સ્ડ એડ્રેસ/ડેટા
  \item
    P1 (પોર્ટ 1): 8-બિટ બિડાયરેક્શનલ I/O
  \item
    P2 (પોર્ટ 2): 8-બિટ બિડાયરેક્શનલ, હાયર એડ્રેસ બાઈટ
  \item
    P3 (પોર્ટ 3): 8-બિટ બિડાયરેક્શનલ ઓલ્ટરનેટ ફંક્શન સાથે
  \end{itemize}
\item
  \textbf{પાવર \& ક્લોક}: VCC, GND, XTAL1, XTAL2
\item
  \textbf{કંટ્રોલ સિગ્નલ્સ}:

  \begin{itemize}
  \tightlist
  \item
    RST: રીસેટ ઈનપુટ
  \item
    ALE: એડ્રેસ લેચ એનેબલ
  \item
    PSEN: પ્રોગ્રામ સ્ટોર એનેબલ
  \item
    EA: એક્સટર્નલ એક્સેસ
  \end{itemize}
\end{enumerate}

\end{solutionbox}
\begin{mnemonicbox}
``PORT-CAPS'' (પોર્ટ્સ 0-3, ક્લોક, એડ્રેસ લેચ, પ્રોગ્રામ
સ્ટોર, સપ્લાય)

\end{mnemonicbox}
\subsection*{પ્રશ્ન 3(અ OR) [3
ગુણ]}\label{uxaaauxab0uxab6uxaa8-3uxa85-or-3-uxa97uxaa3}

\textbf{8051 માઇક્રોકન્ટ્રોલર માટે સીરિયલ કોમ્યુનિકેશન મોડ્સ સમજાવો.}

\begin{solutionbox}

{\def\LTcaptype{none} % do not increment counter
\begin{longtable}[]{@{}
  >{\raggedright\arraybackslash}p{(\linewidth - 6\tabcolsep) * \real{0.1463}}
  >{\raggedright\arraybackslash}p{(\linewidth - 6\tabcolsep) * \real{0.3171}}
  >{\raggedright\arraybackslash}p{(\linewidth - 6\tabcolsep) * \real{0.2683}}
  >{\raggedright\arraybackslash}p{(\linewidth - 6\tabcolsep) * \real{0.2683}}@{}}
\toprule\noalign{}
\begin{minipage}[b]{\linewidth}\raggedright
મોડ
\end{minipage} & \begin{minipage}[b]{\linewidth}\raggedright
વર્ણન
\end{minipage} & \begin{minipage}[b]{\linewidth}\raggedright
બોડ રેટ
\end{minipage} & \begin{minipage}[b]{\linewidth}\raggedright
ડેટા બિટ્સ
\end{minipage} \\
\midrule\noalign{}
\endhead
\bottomrule\noalign{}
\endlastfoot
\textbf{મોડ 0} & શિફ્ટ રજિસ્ટર & ફિક્સ્ડ (FOSC/12) & 8 બિટ્સ \\
\textbf{મોડ 1} & 8-બિટ UART & વેરિએબલ & 10 બિટ્સ (8+સ્ટાર્ટ+સ્ટોપ) \\
\textbf{મોડ 2} & 9-બિટ UART & ફિક્સ્ડ (FOSC/32 અથવા FOSC/64) & 11 બિટ્સ
(9+સ્ટાર્ટ+સ્ટોપ) \\
\textbf{મોડ 3} & 9-બિટ UART & વેરિએબલ & 11 બિટ્સ (9+સ્ટાર્ટ+સ્ટોપ) \\
\end{longtable}
}

\textbf{મુખ્ય ઘટકો}:

\begin{itemize}
\tightlist
\item
  \textbf{SBUF}: સીરિયલ બફર રજિસ્ટર
\item
  \textbf{SCON}: સીરિયલ કંટ્રોલ રજિસ્ટર
\item
  \textbf{P3.0 (RXD)}: રિસીવ પિન
\item
  \textbf{P3.1 (TXD)}: ટ્રાન્સમિટ પિન
\end{itemize}

\textbf{ડાયાગ્રામ:}

\begin{verbatim}
+{-{-}{-}{-}{-}{-}{-}{-}{-}+     +{-}{-}{-}{-}{-}{-}{-}+     +{-}{-}{-}{-}{-}{-}{-}{-}{-}{-}{-}+}
| Timer 1 |{-{-}{-}{-}| Baud  |{-}{-}{-}{-}|           |     +{-}{-}{-}{-}{-}+}
+{-{-}{-}{-}{-}{-}{-}{-}{-}+     | Rate  |     |  Serial   |{-}{-}{-}{-}| TXD |{-}{-}}
                | Gen   |     |  Control  |     +{-{-}{-}{-}{-}+}
                +{-{-}{-}{-}{-}{-}{-}+     |  Logic    |}
                              |           |     +{-{-}{-}{-}{-}+}
                              |           |{{-}{-}{-}{-}| RXD |{-}{-}}
                              +{-{-}{-}{-}{-}{-}{-}{-}{-}{-}{-}+     +{-}{-}{-}{-}{-}+}
                                    |
                                    v
                                 +{-{-}{-}{-}{-}{-}+}
                                 | SBUF |
                                 +{-{-}{-}{-}{-}{-}+}
\end{verbatim}

\end{solutionbox}
\begin{mnemonicbox}
``SMART'' (સીરિયલ મોડ્સ આર રેટ એન્ડ ટાઈમિંગ પર આધારીત)

\end{mnemonicbox}
\subsection*{પ્રશ્ન 3(બ OR) [4
ગુણ]}\label{uxaaauxab0uxab6uxaa8-3uxaac-or-4-uxa97uxaa3}

\textbf{8051 માઇક્રોકન્ટ્રોલરનું ઈન્ટર્નલ રેમ ઓર્ગેનાઈઝેશન સમજાવો.}

\begin{solutionbox}

\textbf{ડાયાગ્રામ:}

\begin{verbatim}
8051 Internal RAM (128 bytes):
+{-{-}{-}{-}{-}{-}{-}{-}{-}{-}{-}{-}{-}{-}{-}{-}{-}{-}{-}+ 7FH}
|                   |
|  General Purpose  |
|       RAM         |
|                   |
+{-{-}{-}{-}{-}{-}{-}{-}{-}{-}{-}{-}{-}{-}{-}{-}{-}{-}{-}+ 30H}
|    Bit{-Addressable|}
|       Area        |
+{-{-}{-}{-}{-}{-}{-}{-}{-}{-}{-}{-}{-}{-}{-}{-}{-}{-}{-}+ 20H}
|                   |
|  Register Banks   |
|    (R0{-R7)        |}
|                   |
+{-{-}{-}{-}{-}{-}{-}{-}{-}{-}{-}{-}{-}{-}{-}{-}{-}{-}{-}+ 00H}
\end{verbatim}

{\def\LTcaptype{none} % do not increment counter
\begin{longtable}[]{@{}
  >{\raggedright\arraybackslash}p{(\linewidth - 4\tabcolsep) * \real{0.3488}}
  >{\raggedright\arraybackslash}p{(\linewidth - 4\tabcolsep) * \real{0.3488}}
  >{\raggedright\arraybackslash}p{(\linewidth - 4\tabcolsep) * \real{0.3023}}@{}}
\toprule\noalign{}
\begin{minipage}[b]{\linewidth}\raggedright
મેમરી રીજન
\end{minipage} & \begin{minipage}[b]{\linewidth}\raggedright
એડ્રેસ રેન્જ
\end{minipage} & \begin{minipage}[b]{\linewidth}\raggedright
વર્ણન
\end{minipage} \\
\midrule\noalign{}
\endhead
\bottomrule\noalign{}
\endlastfoot
\textbf{રજિસ્ટર બેન્ક્સ} & 00H-1FH & 8 રજિસ્ટર (R0-R7) ની ચાર બેન્ક (0-3) \\
\textbf{બિટ-એડ્રેસેબલ} & 20H-2FH & 16 બાઈટ્સ (128 બિટ્સ) વ્યક્તિગત રીતે એડ્રેસ કરી
શકાય \\
\textbf{જનરલ પર્પઝ} & 30H-7FH & વેરિએબલ્સ માટે સ્ક્રેચ પેડ RAM \\
\textbf{SFR} & 80H-FFH & સ્પેશિયલ ફંક્શન રજિસ્ટર્સ (RAM માં નથી) \\
\end{longtable}
}

\textbf{મુખ્ય લક્ષણો}:

\begin{itemize}
\tightlist
\item
  એક સમયે ફક્ત એક રજિસ્ટર બેન્ક એક્ટિવ હોય (PSW બિટ્સ દ્વારા પસંદ કરાય)
\item
  બિટ-એડ્રેસેબલ એરિયામાં દરેક બિટ પોતાનું એડ્રેસ ધરાવે છે (20H.0-2FH.7)
\item
  સ્ટેક આંતરિક RAM માં ક્યાંય પણ હોઈ શકે છે
\end{itemize}

\end{solutionbox}
\begin{mnemonicbox}
``RGB-S'' (રજિસ્ટર્સ, જનરલ પર્પઝ, બિટ-એડ્રેસેબલ, SFRs)

\end{mnemonicbox}
\subsection*{પ્રશ્ન 3(ક OR) [7
ગુણ]}\label{uxaaauxab0uxab6uxaa8-3uxa95-or-7-uxa97uxaa3}

\textbf{આકૃતિની મદદથી 8051 માઇક્રોકન્ટ્રોલરનું આર્કિટેક્ચર સમજાવો.}

\begin{solutionbox}

\textbf{ડાયાગ્રામ:}

\begin{verbatim}
+{-{-}{-}{-}{-}{-}{-}{-}{-}{-}{-}{-}{-}{-}{-}{-}{-}{-}{-}{-}{-}{-}{-}{-}{-}{-}{-}{-}{-}{-}{-}{-}{-}{-}{-}{-}{-}{-}{-}{-}{-}{-}{-}{-}{-}{-}{-}{-}{-}{-}{-}{-}{-}{-}{-}+}
|                    8051 ARCHITECTURE                  |
|                                                       |
|  +{-{-}{-}{-}{-}{-}{-}{-}{-}{-}+     +{-}{-}{-}{-}{-}{-}{-}{-}{-}{-}+     +{-}{-}{-}{-}{-}{-}{-}{-}{-}{-}{-}{-}+     |}
|  |          |     |          |     |            |     |
|  |   CPU    |{{-}{-}{-}|  Timers/ |     |  Interrupts|     |}
|  |          |     | Counters |     |            |     |
|  +{-{-}{-}{-}{-}{-}{-}{-}{-}{-}+     +{-}{-}{-}{-}{-}{-}{-}{-}{-}{-}+     +{-}{-}{-}{-}{-}{-}{-}{-}{-}{-}{-}{-}+     |}
|       \^{                                  \^{}            |}
|       |                                  |            |
|       v                                  v            |
|  +{-{-}{-}{-}{-}{-}{-}{-}{-}{-}+     +{-}{-}{-}{-}{-}{-}{-}{-}{-}{-}+     +{-}{-}{-}{-}{-}{-}{-}{-}{-}{-}{-}{-}+     |}
|  |          |     |          |     |            |     |
|  | Internal |{{-}{-}{-}|  Serial  |{-}{-}{-}| I/O Ports  |     |}
|  |   RAM    |     |   Port   |     |  P0,P1,P2,P3|    |
|  |          |     |          |     |            |     |
|  +{-{-}{-}{-}{-}{-}{-}{-}{-}{-}+     +{-}{-}{-}{-}{-}{-}{-}{-}{-}{-}+     +{-}{-}{-}{-}{-}{-}{-}{-}{-}{-}{-}{-}+     |}
|       \^{                                  \^{}            |}
|       |                                  |            |
|       v                                  v            |
|  +{-{-}{-}{-}{-}{-}{-}{-}{-}{-}+                      +{-}{-}{-}{-}{-}{-}{-}{-}{-}{-}{-}{-}+     |}
|  |          |                      |            |     |
|  | Internal |                      | External   |     |
|  |   ROM    |                      | Interface  |     |
|  |  (4K)    |                      |            |     |
|  +{-{-}{-}{-}{-}{-}{-}{-}{-}{-}+                      +{-}{-}{-}{-}{-}{-}{-}{-}{-}{-}{-}{-}+     |}
|                                         |             |
+{-{-}{-}{-}{-}{-}{-}{-}{-}{-}{-}{-}{-}{-}{-}{-}{-}{-}{-}{-}{-}{-}{-}{-}{-}{-}{-}{-}{-}{-}{-}{-}{-}{-}{-}{-}{-}{-}{-}{-}{-}{-}{-}{-}{-}{-}{-}{-}{-}{-}{-}{-}{-}{-}{-}+}
                                          |
                                          v
                                 External Memory \& Devices
\end{verbatim}

\textbf{મુખ્ય ઘટકો}:

\begin{itemize}
\tightlist
\item
  \textbf{CPU}: ALU, રજિસ્ટર્સ અને કંટ્રોલ લોજિક સાથે 8-બિટ પ્રોસેસર
\item
  \textbf{મેમરી}:

  \begin{itemize}
  \tightlist
  \item
    4KB આંતરિક ROM (પ્રોગ્રામ મેમરી)
  \item
    128 બાઈટ્સ આંતરિક RAM (ડેટા મેમરી)
  \end{itemize}
\item
  \textbf{I/O}: ચાર 8-બિટ I/O પોર્ટ્સ (P0-P3)
\item
  \textbf{ટાઈમર્સ}: બે 16-બિટ ટાઈમર/કાઉન્ટર
\item
  \textbf{સીરિયલ પોર્ટ}: ફુલ-ડુપ્લેક્સ UART
\item
  \textbf{ઇન્ટરપ્ટ્સ}: બે પ્રાયોરિટી લેવલ સાથે પાંચ ઇન્ટરપ્ટ સોર્સ
\end{itemize}

\end{solutionbox}
\begin{mnemonicbox}
``BASICS'' (બસ, આર્કિટેક્ચર વિથ CPU, સીરિયલ પોર્ટ, I/O
પોર્ટ્સ, કાઉન્ટર/ટાઈમર, સ્પેશિયલ ફંક્શન્સ)

\end{mnemonicbox}
\subsection*{પ્રશ્ન 4(અ) [3
ગુણ]}\label{uxaaauxab0uxab6uxaa8-4uxa85-3-uxa97uxaa3}

\textbf{રજિસ્ટર R5 અને R6 ના લોઅર નિબલને બદલવા માટે 8051 એસેમ્બલી લેંગ્વેજ પ્રોગ્રામ
લખો: R5 ના લોઅર નિબલને R6 માં અને R6 ના લોઅર નિબલને R5 માં મૂકો.}

\begin{solutionbox}

\begin{verbatim}
      ; Exchange lower nibbles of R5 and R6
      MOV A, R5    ; Copy R5 to accumulator
      ANL A, \#0FH  ; Mask upper nibble (keep only lower nibble)
      MOV B, A     ; Save R5{s lower nibble in B}
      
      MOV A, R6    ; Copy R6 to accumulator
      ANL A, \#0FH  ; Mask upper nibble (keep only lower nibble)
      MOV C, A     ; Save temporarily in a free register (R7)
      
      MOV A, R5    ; Get R5 again
      ANL A, \#F0H  ; Keep only upper nibble of R5
      ORL A, C     ; Combine with lower nibble from R6
      MOV R5, A    ; Store result back in R5
      
      MOV A, R6    ; Get R6 again
      ANL A, \#F0H  ; Keep only upper nibble of R6
      ORL A, B     ; Combine with lower nibble from R5
      MOV R6, A    ; Store result back in R6
\end{verbatim}

\textbf{ડાયાગ્રામ:}

\begin{verbatim}
Initially:         After Exchange:
R5: 1010 1100     R5: 1010 0011
    ↑↑↑↑ ↑↑↑↑         ↑↑↑↑ ↑↑↑↑
     |    |            |    |
     |    +{-{-}+         |    |}
     |       |         |    |
     |    +{-{-}v         |    v}
     |    |            |    From R6
     v    v            v
R6: 0011 0011     R6: 0011 1100
\end{verbatim}

\end{solutionbox}
\begin{mnemonicbox}
``MAMS'' (માસ્ક, એન્ડ, મુવ, સ્વેપ)

\end{mnemonicbox}
\subsection*{પ્રશ્ન 4(બ) [4
ગુણ]}\label{uxaaauxab0uxab6uxaa8-4uxaac-4-uxa97uxaa3}

\textbf{પોર્ટ P1.0 પર ઇન્ટરફેસ કરેલ LED ને 1ms ના સમય અંતરાલ પર બ્લિંક કરવા માટે
8051 એસેમ્બલી લેંગ્વેજ પ્રોગ્રામ લખો.}

\begin{solutionbox}

\begin{verbatim}
      ORG 0000H        ; Start at memory location 0000H
MAIN: CPL P1.0         ; Complement P1.0 (toggle LED)
      ACALL DELAY      ; Call delay subroutine
      SJMP MAIN        ; Loop forever

DELAY: MOV R7, \#2      ; Load R7 for outer loop (2)
DELAY1: MOV R6, \#250   ; Load R6 for inner loop (250)
DELAY2: NOP            ; No operation (consume time)
        NOP            ; Additional delay
        DJNZ R6, DELAY2 ; Decrement R6 \& loop until zero
        DJNZ R7, DELAY1 ; Decrement R7 \& loop until zero
        RET            ; Return from subroutine
\end{verbatim}

\textbf{ડાયાગ્રામ:}

\begin{verbatim}
+{-{-}{-}{-}{-}{-}{-}{-}{-}{-}+          +{-}{-}{-}{-}{-}{-}{-}{-}{-}{-}+}
|          |          |          |
| P1.0 = 1 |{-{-}{-}{-}{-}{-}{-}{-}{-}| P1.0 = 0 |}
|          |   1ms    |          |
+{-{-}{-}{-}{-}{-}{-}{-}{-}{-}+          +{-}{-}{-}{-}{-}{-}{-}{-}{-}{-}+}
      \^{                    |}
      |                    |
      |                    |
      +{-{-}{-}{-}{-}{-}{-}{-}{-}{-}{-}{-}{-}{-}{-}{-}{-}{-}{-}{-}+}
             1ms
\end{verbatim}

\end{solutionbox}
\begin{mnemonicbox}
``TCDL'' (ટોગલ, કોલ, ડિલે, લૂપ)

\end{mnemonicbox}
\subsection*{પ્રશ્ન 4(ક) [7
ગુણ]}\label{uxaaauxab0uxab6uxaa8-4uxa95-7-uxa97uxaa3}

\textbf{8051 માઇક્રોકન્ટ્રોલરના એડ્રેસિંગ મોડ્સની યાદી બનાવો અને ઓછામાં ઓછા એક
ઉદાહરણ સાથે તેમને સમજાવો.}

\begin{solutionbox}

{\def\LTcaptype{none} % do not increment counter
\begin{longtable}[]{@{}
  >{\raggedright\arraybackslash}p{(\linewidth - 4\tabcolsep) * \real{0.4359}}
  >{\raggedright\arraybackslash}p{(\linewidth - 4\tabcolsep) * \real{0.3333}}
  >{\raggedright\arraybackslash}p{(\linewidth - 4\tabcolsep) * \real{0.2308}}@{}}
\toprule\noalign{}
\begin{minipage}[b]{\linewidth}\raggedright
એડ્રેસિંગ મોડ
\end{minipage} & \begin{minipage}[b]{\linewidth}\raggedright
વર્ણન
\end{minipage} & \begin{minipage}[b]{\linewidth}\raggedright
ઉદાહરણ
\end{minipage} \\
\midrule\noalign{}
\endhead
\bottomrule\noalign{}
\endlastfoot
\textbf{રજિસ્ટર} & રજિસ્ટર્સ (R0-R7) વાપરે છે & \texttt{MOV\ A,\ R0} (R0 ને A
માં મુવ કરે) \\
\textbf{ડાયરેક્ટ} & ડાયરેક્ટ મેમરી એડ્રેસ વાપરે & \texttt{MOV\ A,\ 30H} (30H
પરથી ડેટા A માં મુવ કરે) \\
\textbf{રજિસ્ટર ઇન્ડાયરેક્ટ} & રજિસ્ટરને પોઇન્ટર તરીકે વાપરે &
\texttt{MOV\ A,\ @R0} (R0 માં રહેલા એડ્રેસ પરથી ડેટા A માં મુવ કરે) \\
\textbf{ઇમીડિયેટ} & કોન્સ્ટન્ટ ડેટા વાપરે & \texttt{MOV\ A,\ \#25H} (A માં 25H
લોડ કરે) \\
\textbf{ઇન્ડેક્સ્ડ} & બેઝ એડ્રેસ + ઓફસેટ & \texttt{MOVC\ A,\ @A+DPTR} (કોડ મેમરી
એક્સેસ) \\
\textbf{બિટ} & વ્યક્તિગત બિટ્સ પર ઓપરેશન કરે & \texttt{SETB\ P1.0} (પોર્ટ 1 ના
બિટ 0 ને સેટ કરે) \\
\textbf{ઇમ્પ્લાઈડ} & ઇમ્પ્લિસિટ ઓપરેન્ડ & \texttt{RRC\ A} (A ને રાઈટ થ્રુ કેરી
રોટેટ કરે) \\
\end{longtable}
}

\textbf{ડાયાગ્રામ:}

\begin{verbatim}
+{-{-}{-}{-}{-}{-}{-}{-}{-}{-}{-}{-}{-}{-}{-}{-}{-}{-}+    +{-}{-}{-}{-}{-}{-}{-}{-}{-}{-}{-}{-}{-}{-}{-}{-}{-}{-}+    +{-}{-}{-}{-}{-}{-}{-}{-}{-}{-}{-}{-}{-}{-}{-}{-}{-}{-}+}
| Register         |    | Direct           |    | Indirect         |
| MOV A, R5        |    | MOV A, 40H       |    | MOV A, @R1       |
| +{-{-}{-}+     +{-}{-}{-}+  |    | +{-}{-}{-}+     +{-}{-}{-}+  |    | +{-}{-}{-}+    +{-}{-}{-}+   |}
| | A |{{-}{-}{-}{-}| R5|  |    | | A |{-}{-}{-}{-}|40H|  |    | | A |{-}{-}{-}| X |   |}
| +{-{-}{-}+     +{-}{-}{-}+  |    | +{-}{-}{-}+     +{-}{-}{-}+  |    | +{-}{-}{-}+    +{-}{-}{-}+   |}
+{-{-}{-}{-}{-}{-}{-}{-}{-}{-}{-}{-}{-}{-}{-}{-}{-}{-}+    +{-}{-}{-}{-}{-}{-}{-}{-}{-}{-}{-}{-}{-}{-}{-}{-}{-}{-}+    |            \^{}     |}
                                                |            |     |
                                                |         +{-{-}+{-}{-}+  |}
                                                |         | R1=X|  |
                                                |         +{-{-}{-}{-}{-}+  |}
                                                +{-{-}{-}{-}{-}{-}{-}{-}{-}{-}{-}{-}{-}{-}{-}{-}{-}{-}+}
\end{verbatim}

\end{solutionbox}
\begin{mnemonicbox}
``RIDDIBM'' (રજિસ્ટર, ઇમીડિયેટ, ડાયરેક્ટ, ડેટા, ઇન્ડાયરેક્ટ,
બિટ, iમ્પ્લાઈડ)

\end{mnemonicbox}
\subsection*{પ્રશ્ન 4(અ OR) [3
ગુણ]}\label{uxaaauxab0uxab6uxaa8-4uxa85-or-3-uxa97uxaa3}

\textbf{રજિસ્ટર R2 અને R3 નાં બાઈટ નો સરવાળો કરવા માટે 8051 એસેમ્બલી લેંગ્વેજ
પ્રોગ્રામ લખો, પરિણામ બાહ્ય RAM માં 2040h (LSB) અને 2041h (MSB) મૂકો.}

\begin{solutionbox}

\begin{verbatim}
      MOV A, R2      ; Move R2 to accumulator
      ADD A, R3      ; Add R3 to accumulator
      MOV DPTR, \#2040H ; Set DPTR to external RAM address 2040H
      MOVX @DPTR, A  ; Store the result (LSB) at 2040H
      
      MOV A, \#00H    ; Clear accumulator
      ADDC A, \#00H   ; Add carry flag to accumulator
      INC DPTR       ; Increment DPTR to 2041H
      MOVX @DPTR, A  ; Store the result (MSB) at 2041H
\end{verbatim}

\textbf{ડાયાગ્રામ:}

\begin{verbatim}
  R2    R3           External RAM
+{-{-}{-}{-}+ +{-}{-}{-}{-}+        +{-}{-}{-}{-}{-}{-}{-}{-}{-}{-}{-}+}
| 25H| | 45H|        | 2040H: 6A |  (25H + 45H = 6AH)
+{-{-}{-}{-}+ +{-}{-}{-}{-}+        +{-}{-}{-}{-}{-}{-}{-}{-}{-}{-}{-}+}
   |      |          | 2041H: 00 |  (No carry)
   v      v          +{-{-}{-}{-}{-}{-}{-}{-}{-}{-}{-}+}
  +{-{-}{-}{-}{-}{-}{-}{-}+         }
  |   ADD  |
  +{-{-}{-}{-}{-}{-}{-}{-}+}
\end{verbatim}

\end{solutionbox}
\begin{mnemonicbox}
``MASIM'' (મુવ, એડ, સ્ટોર, ઇન્ક્રિમેન્ટ, મુવ એગેન)

\end{mnemonicbox}
\subsection*{પ્રશ્ન 4(બ OR) [4
ગુણ]}\label{uxaaauxab0uxab6uxaa8-4uxaac-or-4-uxa97uxaa3}

\textbf{12 MHz ની ક્રિસ્ટલ ફ્રિક્વન્સી સાથે 8051 માઇક્રોકન્ટ્રોલર માટે, 5ms નો ડિલે
જનરેટ કરો.}

\begin{solutionbox}

\begin{verbatim}
      ; Delay of 5ms with 12MHz Crystal (1 machine cycle = 1μs)
DELAY: MOV R7, \#5     ; 5 loops of 1ms each
LOOP1: MOV R6, \#250   ; 250 x 4μs = 1000μs = 1ms
LOOP2: NOP            ; 1μs
       NOP            ; 1μs
       DJNZ R6, LOOP2 ; 2μs (if jump taken)
       DJNZ R7, LOOP1 ; Repeat 5 times for 5ms
       RET            ; Return from subroutine
\end{verbatim}

\textbf{ડાયાગ્રામ:}

\begin{verbatim}
        +{-{-}{-}{-}{-}{-}{-}{-}{-}{-}+}
        | Start    |
        +{-{-}{-}{-}{-}{-}{-}{-}{-}{-}+}
             |
             v
      +{-{-}{-}{-}{-}{-}{-}{-}{-}{-}{-}{-}{-}+}
      | R7 = 5      |{{-}{-}{-}{-}{-}{-}{-}{-}{-}+}
      +{-{-}{-}{-}{-}{-}{-}{-}{-}{-}{-}{-}{-}+          |}
             |                 |
             v                 |
      +{-{-}{-}{-}{-}{-}{-}{-}{-}{-}{-}{-}{-}+          |}
      | R6 = 250    |{{-}{-}{-}{-}+    |}
      +{-{-}{-}{-}{-}{-}{-}{-}{-}{-}{-}{-}{-}+     |    |}
             |            |    |
             v            |    |
      +{-{-}{-}{-}{-}{-}{-}{-}{-}{-}{-}{-}{-}+     |    |}
      | 2 NOPs      |     |    |
      +{-{-}{-}{-}{-}{-}{-}{-}{-}{-}{-}{-}{-}+     |    |}
             |            |    |
             v            |    |
      +{-{-}{-}{-}{-}{-}{-}{-}{-}{-}{-}{-}{-}+     |    |}
      | Decrement R6|{-{-}{-}{-}{-}+    |}
      +{-{-}{-}{-}{-}{-}{-}{-}{-}{-}{-}{-}{-}+          |}
             |                 |
             v                 |
      +{-{-}{-}{-}{-}{-}{-}{-}{-}{-}{-}{-}{-}+          |}
      | Decrement R7|{-{-}{-}{-}{-}{-}{-}{-}{-}+}
      +{-{-}{-}{-}{-}{-}{-}{-}{-}{-}{-}{-}{-}+}
             |
             v
      +{-{-}{-}{-}{-}{-}{-}{-}{-}{-}{-}{-}{-}+}
      | Return      |
      +{-{-}{-}{-}{-}{-}{-}{-}{-}{-}{-}{-}{-}+}
\end{verbatim}

\textbf{ગણતરી}:

\begin{itemize}
\tightlist
\item
  12MHz ક્રિસ્ટલ = 1μs મશીન સાયકલ
\item
  ઇનર લૂપ: 2 NOPs (2μs) + DJNZ (2μs) = 4μs પ્રતિ ઇટરેશન
\item
  250 ઇટરેશન \times 4μs = 1000μs = 1ms
\item
  આઉટર લૂપ: 5 ઇટરેશન \times 1ms = 5ms
\end{itemize}

\end{solutionbox}
\begin{mnemonicbox}
``LOON-5'' (લૂપ નેસ્ટેડ ફોર 5ms)

\end{mnemonicbox}
\subsection*{પ્રશ્ન 4(ક OR) [7
ગુણ]}\label{uxaaauxab0uxab6uxaa8-4uxa95-or-7-uxa97uxaa3}

\textbf{8051 માઇક્રોકન્ટ્રોલર માટે કોઈપણ સાત એરિથમેટિક ઈન્સ્ટ્રક્શન ઉદાહરણ સાથે
સમજાવો.}

\begin{solutionbox}

{\def\LTcaptype{none} % do not increment counter
\begin{longtable}[]{@{}
  >{\raggedright\arraybackslash}p{(\linewidth - 6\tabcolsep) * \real{0.2826}}
  >{\raggedright\arraybackslash}p{(\linewidth - 6\tabcolsep) * \real{0.2174}}
  >{\raggedright\arraybackslash}p{(\linewidth - 6\tabcolsep) * \real{0.1957}}
  >{\raggedright\arraybackslash}p{(\linewidth - 6\tabcolsep) * \real{0.3043}}@{}}
\toprule\noalign{}
\begin{minipage}[b]{\linewidth}\raggedright
ઇન્સ્ટ્રક્શન
\end{minipage} & \begin{minipage}[b]{\linewidth}\raggedright
ફંક્શન
\end{minipage} & \begin{minipage}[b]{\linewidth}\raggedright
ઉદાહરણ
\end{minipage} & \begin{minipage}[b]{\linewidth}\raggedright
ફ્લેગ અસર
\end{minipage} \\
\midrule\noalign{}
\endhead
\bottomrule\noalign{}
\endlastfoot
\textbf{ADD A,src} & સોર્સને A માં ઉમેરે & \texttt{ADD\ A,R0} (A=A+R0) & C,
OV, AC \\
\textbf{ADDC A,src} & સોર્સ + કેરી A માં ઉમેરે & \texttt{ADDC\ A,\#25H}
(A=A+25H+C) & C, OV, AC \\
\textbf{SUBB A,src} & સોર્સ + બોરો A માંથી બાદ કરે & \texttt{SUBB\ A,@R1}
(A=A-@R1-C) & C, OV, AC \\
\textbf{INC} & 1 વધારે & \texttt{INC\ R3} (R3=R3+1) & કોઈ નહીં \\
\textbf{DEC} & 1 ઘટાડે & \texttt{DEC\ A} (A=A-1) & કોઈ નહીં \\
\textbf{MUL AB} & A અને B ગુણાકાર કરે & \texttt{MUL\ AB} (B:A=A\timesB) & C,
OV \\
\textbf{DIV AB} & A ને B વડે ભાગે & \texttt{DIV\ AB} (A=ભાગફળ, B=શેષ) & C,
OV \\
\end{longtable}
}

\textbf{ડાયાગ્રામ:}

\begin{verbatim}
+{-{-}{-}{-}{-}{-}{-}{-}{-}{-}{-}{-}{-}{-}{-}{-}{-}{-}{-}+     +{-}{-}{-}{-}{-}{-}{-}{-}{-}{-}{-}{-}{-}{-}{-}{-}{-}{-}{-}+     +{-}{-}{-}{-}{-}{-}{-}{-}{-}{-}{-}{-}{-}{-}{-}{-}{-}{-}{-}+}
| ADD A,R0          |     | MUL AB            |     | DIV AB            |
|                   |     |                   |     |                   |
|

A = 25H, R0 = 15H |     |

A = 05H,

B = 03H  |     |

A = 14H,

B = 05H  |

|

A = 25H + 15H     |     | B:A = 05H  03H   |     |

A = 14H  05H     |

|

A = 3AH           |     |

B = 00H,

A = 0FH  |     |

A = 04H,

B = 00H  |

+{-{-}{-}{-}{-}{-}{-}{-}{-}{-}{-}{-}{-}{-}{-}{-}{-}{-}{-}+     +{-}{-}{-}{-}{-}{-}{-}{-}{-}{-}{-}{-}{-}{-}{-}{-}{-}{-}{-}+     +{-}{-}{-}{-}{-}{-}{-}{-}{-}{-}{-}{-}{-}{-}{-}{-}{-}{-}{-}+}
\end{verbatim}

\end{solutionbox}
\begin{mnemonicbox}
``ACID-IBM'' (એડ, કેરી એડ, ઇન્ક, ડેક, મલ, બોરો સબ્ટ્રેક્ટ,
ડિવાઈડ)

\end{mnemonicbox}
\subsection*{પ્રશ્ન 5(અ) [3
ગુણ]}\label{uxaaauxab0uxab6uxaa8-5uxa85-3-uxa97uxaa3}

\textbf{વિવિધ ક્ષેત્રોમાં માઇક્રોકન્ટ્રોલરની એપ્લિકેશનોની સૂચિ બનાવો.}

\begin{solutionbox}

{\def\LTcaptype{none} % do not increment counter
\begin{longtable}[]{@{}ll@{}}
\toprule\noalign{}
ક્ષેત્ર & એપ્લિકેશન્સ \\
\midrule\noalign{}
\endhead
\bottomrule\noalign{}
\endlastfoot
\textbf{કન્ઝ્યુમર ઇલેક્ટ્રોનિક્સ} & ટીવી, વોશિંગ મશીન, માઇક્રોવેવ, રિમોટ કંટ્રોલ \\
\textbf{ઓટોમોટિવ} & એન્જિન કંટ્રોલ, એન્ટી-લોક બ્રેકિંગ, એરબેગ સિસ્ટમ \\
\textbf{ઇન્ડસ્ટ્રિયલ} & ઓટોમેશન, રોબોટિક્સ, પ્રોસેસ કંટ્રોલ \\
\textbf{મેડિકલ} & પેશન્ટ મોનિટરિંગ, મેડિકલ ઇન્સ્ટ્રુમેન્ટ્સ, ઇમ્પ્લાન્ટ્સ \\
\textbf{હોમ ઓટોમેશન} & સ્માર્ટ લાઇટિંગ, સિક્યુરિટી સિસ્ટમ, HVAC કંટ્રોલ \\
\textbf{કમ્યુનિકેશન} & મોબાઇલ ફોન, રાઉટર્સ, મોડેમ્સ \\
\textbf{એરોસ્પેસ} & નેવિગેશન સિસ્ટમ, ફ્લાઇટ કંટ્રોલ, સેટેલાઇટ સિસ્ટમ \\
\end{longtable}
}

\textbf{ડાયાગ્રામ:}

\begin{verbatim}
                   +{-{-}{-}{-}{-}{-}{-}{-}{-}{-}{-}{-}{-}{-}{-}{-}{-}{-}{-}+}
                   | Microcontroller   |
                   | Applications      |
                   +{-{-}{-}{-}{-}{-}{-}{-}{-}{-}{-}{-}{-}{-}{-}{-}{-}{-}{-}+}
                            |
         +{-{-}{-}{-}{-}{-}{-}{-}{-}{-}{-}{-}{-}{-}{-}{-}{-}{-}+{-}{-}{-}{-}{-}{-}{-}{-}{-}{-}{-}{-}{-}{-}{-}{-}{-}{-}+}
         |                  |                  |
+{-{-}{-}{-}{-}{-}{-}{-}{-}{-}{-}{-}{-}{-}{-}+  +{-}{-}{-}{-}{-}{-}{-}{-}{-}{-}{-}{-}{-}{-}{-}{-}+  +{-}{-}{-}{-}{-}{-}{-}{-}{-}{-}{-}{-}{-}{-}{-}+}
| Consumer      |  | Industrial     |  | Communication |
| Electronics   |  | Automation     |  | Systems       |
+{-{-}{-}{-}{-}{-}{-}{-}{-}{-}{-}{-}{-}{-}{-}+  +{-}{-}{-}{-}{-}{-}{-}{-}{-}{-}{-}{-}{-}{-}{-}{-}+  +{-}{-}{-}{-}{-}{-}{-}{-}{-}{-}{-}{-}{-}{-}{-}+}
         |                  |                  |
+{-{-}{-}{-}{-}{-}{-}{-}{-}{-}{-}{-}{-}{-}{-}+  +{-}{-}{-}{-}{-}{-}{-}{-}{-}{-}{-}{-}{-}{-}{-}{-}+  +{-}{-}{-}{-}{-}{-}{-}{-}{-}{-}{-}{-}{-}{-}{-}+}
| Automotive    |  | Medical        |  | Home          |
| Systems       |  | Devices        |  | Automation    |
+{-{-}{-}{-}{-}{-}{-}{-}{-}{-}{-}{-}{-}{-}{-}+  +{-}{-}{-}{-}{-}{-}{-}{-}{-}{-}{-}{-}{-}{-}{-}{-}+  +{-}{-}{-}{-}{-}{-}{-}{-}{-}{-}{-}{-}{-}{-}{-}+}
\end{verbatim}

\end{solutionbox}
\begin{mnemonicbox}
``CHAIM-MA'' (કન્ઝ્યુમર, હોમ, ઓટોમોટિવ, ઇન્ડસ્ટ્રિયલ,
મેડિકલ, મોબાઇલ, એરોસ્પેસ)

\end{mnemonicbox}
\subsection*{પ્રશ્ન 5(બ) [4
ગુણ]}\label{uxaaauxab0uxab6uxaa8-5uxaac-4-uxa97uxaa3}

\textbf{8051 માઇક્રોકન્ટ્રોલર સાથે રિલે ઇન્ટરફેસ કરો.}

\begin{solutionbox}

\textbf{ડાયાગ્રામ:}

\begin{verbatim}
+{-{-}{-}{-}{-}{-}{-}{-}+         +{-}{-}{-}{-}{-}{-}{-}{-}+         +{-}{-}{-}{-}{-}{-}{-}{-}{-}{-}{-}{-}+}
|        |         |        |         |            |
|        |         |        |         |            |
|  8051  |{-{-}{-}P1.0{-}{-}| Driver |{-}{-}{-}(+){-}{-}{-}| Relay Coil |}
|        |         | ULN2003|    |    |            |
|        |         |        |    |    |            |
+{-{-}{-}{-}{-}{-}{-}{-}+         +{-}{-}{-}{-}{-}{-}{-}{-}+    |    +{-}{-}{-}{-}{-}{-}{-}{-}{-}{-}{-}{-}+}
                                 |          |
                                 |          |
                        +{-{-}{-}{-}{-}{-}{-}{-}+          |}
                        |                   |
                        | +5V               |
                        +{-{-}{-}{-}{-}{-}{-}{-}{-}{-}{-}{-}{-}{-}{-}{-}{-}{-}{-}+}
                                 |
                            Protection
                              Diode
\end{verbatim}

\textbf{જરૂરી ઘટકો}:

\begin{itemize}
\tightlist
\item
  8051 માઇક્રોકન્ટ્રોલર
\item
  ULN2003 અથવા સમાન ડ્રાઇવર IC
\item
  રિલે (5V અથવા 12V)
\item
  પ્રોટેક્શન ડાયોડ (1N4007)
\item
  પાવર સપ્લાય
\end{itemize}

\textbf{કાર્યપ્રણાલી}:

\begin{enumerate}
\tightlist
\item
  8051 P1.0 થી કંટ્રોલ સિગ્નલ મોકલે છે
\item
  ડ્રાઇવર રિલે ચલાવવા માટે કરંટ એમ્પ્લિફાય કરે છે
\item
  પ્રોટેક્શન ડાયોડ બેક EMF નુકસાનથી બચાવે છે
\item
  રિલે કનેક્ટેડ ડિવાઇસ સ્વિચ કરે છે
\end{enumerate}

\end{solutionbox}
\begin{mnemonicbox}
``DRIPS'' (ડ્રાઇવર, રિલે, ઇનપુટ ફ્રોમ µC, પ્રોટેક્શન ડાયોડ,
સ્વિચિંગ)

\end{mnemonicbox}
\subsection*{પ્રશ્ન 5(ક) [7
ગુણ]}\label{uxaaauxab0uxab6uxaa8-5uxa95-7-uxa97uxaa3}

\textbf{8051 માઇક્રોકન્ટ્રોલર સાથે LCD ઇન્ટરફેસ કરો.}

\begin{solutionbox}

\textbf{ડાયાગ્રામ:}

\begin{verbatim}
        8051                    16x2 LCD
    +{-{-}{-}{-}{-}{-}{-}{-}{-}{-}+             +{-}{-}{-}{-}{-}{-}{-}{-}{-}{-}+}
    |          |             |          |
    |      P1.0|{-{-}{-}{-}{-}{-}{-}{-}{-}{-}{-}{-}|RS        |}
    |      P1.1|{-{-}{-}{-}{-}{-}{-}{-}{-}{-}{-}{-}|R/W       |}
    |      P1.2|{-{-}{-}{-}{-}{-}{-}{-}{-}{-}{-}{-}|E         |}
    |          |             |          |
    |      P2.0|{-{-}{-}{-}{-}{-}{-}{-}{-}{-}{-}{-}|D0        |}
    |      P2.1|{-{-}{-}{-}{-}{-}{-}{-}{-}{-}{-}{-}|D1        |}
    |      P2.2|{-{-}{-}{-}{-}{-}{-}{-}{-}{-}{-}{-}|D2        |}
    |      P2.3|{-{-}{-}{-}{-}{-}{-}{-}{-}{-}{-}{-}|D3        |}
    |      P2.4|{-{-}{-}{-}{-}{-}{-}{-}{-}{-}{-}{-}|D4        |}
    |      P2.5|{-{-}{-}{-}{-}{-}{-}{-}{-}{-}{-}{-}|D5        |}
    |      P2.6|{-{-}{-}{-}{-}{-}{-}{-}{-}{-}{-}{-}|D6        |}
    |      P2.7|{-{-}{-}{-}{-}{-}{-}{-}{-}{-}{-}{-}|D7        |}
    |          |             |          |
    +{-{-}{-}{-}{-}{-}{-}{-}{-}{-}+             +{-}{-}{-}{-}{-}{-}{-}{-}{-}{-}+}
                                 |
                             +{-{-}{-}+{-}{-}{-}+}
                             | VCC   |
                             | Pot   |
                             | GND   |
                             +{-{-}{-}{-}{-}{-}{-}+}
\end{verbatim}

\textbf{કનેક્શન્સ}:

\begin{itemize}
\tightlist
\item
  \textbf{કંટ્રોલ લાઇન્સ}:

  \begin{itemize}
  \tightlist
  \item
    P1.0 \rightarrow RS (રજિસ્ટર સિલેક્ટ)
  \item
    P1.1 \rightarrow R/W (રીડ/રાઈટ)
  \item
    P1.2 \rightarrow E (એનેબલ)
  \end{itemize}
\item
  \textbf{ડેટા લાઇન્સ}:

  \begin{itemize}
  \tightlist
  \item
    P2.0-P2.7 \rightarrow D0-D7 (8-બિટ ડેટા બસ)
  \end{itemize}
\end{itemize}

\textbf{LCD ઇનિશિયલાઇઝ કરવાનો કોડ}:

\begin{verbatim}
MOV A, \#38H      ; 2 lines, 5x7 matrix
ACALL COMMAND    ; Send command

MOV A, \#0EH      ; Display ON, cursor ON
ACALL COMMAND    ; Send command

MOV A, \#01H      ; Clear LCD
ACALL COMMAND    ; Send command

MOV A, \#06H      ; Increment cursor
ACALL COMMAND    ; Send command
\end{verbatim}

\end{solutionbox}
\begin{mnemonicbox}
``CIDER-8'' (કંટ્રોલ લાઇન્સ, ઇનિશિયલાઇઝ, ડેટા બસ, એનેબલ,
રજિસ્ટર સિલેક્ટ, 8-બિટ મોડ)

\end{mnemonicbox}
\subsection*{પ્રશ્ન 5(અ OR) [3
ગુણ]}\label{uxaaauxab0uxab6uxaa8-5uxa85-or-3-uxa97uxaa3}

\textbf{8051 માઇક્રોકન્ટ્રોલર સાથે LED નું ઇન્ટરફેસિંગ દોરો.}

\begin{solutionbox}

\textbf{ડાયાગ્રામ:}

\begin{verbatim}
       +5V
        |
        |
        R (220Ω)
        |
        |
        v
    +{-{-}{-}+{-}{-}{-}+}
    |       |
    | LED   |
    |       |
    +{-{-}{-}+{-}{-}{-}+}
        |
        |
        v
+{-{-}{-}{-}{-}{-}{-}{-}{-}{-}{-}{-}{-}{-}{-}{-}+}
|                |
|       P1.0     |
|                |
|      8051      |
|                |
+{-{-}{-}{-}{-}{-}{-}{-}{-}{-}{-}{-}{-}{-}{-}{-}+}
\end{verbatim}

\textbf{જરૂરી ઘટકો}:

\begin{itemize}
\tightlist
\item
  8051 માઇક્રોકન્ટ્રોલર
\item
  LED
\item
  કરંટ લિમિટિંગ રેસિસ્ટર (220Ω)
\item
  પાવર સપ્લાય
\end{itemize}

\textbf{કાર્યપ્રણાલી}:

\begin{itemize}
\tightlist
\item
  એક્ટિવ-લો કન્ફિગરેશન: પિન = 0 ત્યારે LED ON
\item
  P1.0 LED ને કરંટ લિમિટિંગ રેસિસ્ટર મારફતે ડ્રાઇવ કરે છે
\item
  મહત્તમ કરંટ પિન દીઠ 20mA નથી વધવો જોઈએ
\end{itemize}

\textbf{LED બ્લિંકિંગ માટે કોડ}:

\begin{verbatim}
MAIN: CLR P1.0    ; Turn ON LED (active low)
      CALL DELAY  ; Wait
      SETB P1.0   ; Turn OFF LED
      CALL DELAY  ; Wait
      SJMP MAIN   ; Repeat
\end{verbatim}

\end{solutionbox}
\begin{mnemonicbox}
``CIRCLE'' (કરંટ લિમિટિંગ રેસિસ્ટર, IO પિન, કેથોડ ટુ LED,
LED ટુ અર્થ/ગ્રાઉન્ડ)

\end{mnemonicbox}
\subsection*{પ્રશ્ન 5(બ OR) [4
ગુણ]}\label{uxaaauxab0uxab6uxaa8-5uxaac-or-4-uxa97uxaa3}

\textbf{8051 માઇક્રોકન્ટ્રોલર સાથે ડીસી મોટર ઇન્ટરફેસ કરો.}

\begin{solutionbox}

\textbf{ડાયાગ્રામ:}

\begin{verbatim}
+{-{-}{-}{-}{-}{-}{-}{-}+         +{-}{-}{-}{-}{-}{-}{-}{-}+        +{-}{-}{-}{-}{-}{-}{-}{-}{-}+}
|        |         |        |        |         |
|  8051  |{-{-}P1.0{-}{-}| L293D  |{-}{-}Out1{-}{-}+         |}
|        |         | Motor  |        |   DC    |
|        |{-{-}P1.1{-}{-}| Driver |{-}{-}Out2{-}{-}+  Motor  |}
|        |         |        |        |         |
+{-{-}{-}{-}{-}{-}{-}{-}+         +{-}{-}{-}{-}{-}{-}{-}{-}+        +{-}{-}{-}{-}{-}{-}{-}{-}{-}+}
                       |
                       |
                    +{-{-}+{-}{-}+}
                    | +5V |
                    +{-{-}{-}{-}{-}+}
\end{verbatim}

\textbf{જરૂરી ઘટકો}:

\begin{itemize}
\tightlist
\item
  8051 માઇક્રોકન્ટ્રોલર
\item
  L293D મોટર ડ્રાઇવર IC
\item
  ડીસી મોટર
\item
  પાવર સપ્લાય
\end{itemize}

\textbf{કંટ્રોલ લોજિક}:

{\def\LTcaptype{none} % do not increment counter
\begin{longtable}[]{@{}lll@{}}
\toprule\noalign{}
P1.0 & P1.1 & મોટર એક્શન \\
\midrule\noalign{}
\endhead
\bottomrule\noalign{}
\endlastfoot
0 & 0 & સ્ટોપ (બ્રેક) \\
0 & 1 & ક્લોકવાઇઝ \\
1 & 0 & કાઉન્ટર-ક્લોકવાઇઝ \\
1 & 1 & સ્ટોપ (ફ્રી-રનિંગ) \\
\end{longtable}
}

\textbf{મોટર કંટ્રોલ માટે કોડ}:

\begin{verbatim}
MOV P1, \#02H  ; P1.0=0, P1.1=1 (Clockwise)
CALL DELAY    ; Run for some time
MOV P1, \#01H  ; P1.0=1, P1.1=0 (Counter{-clockwise)}
CALL DELAY    ; Run for some time
MOV P1, \#00H  ; P1.0=0, P1.1=0 (Stop)
\end{verbatim}

\end{solutionbox}
\begin{mnemonicbox}
``DICER'' (ડ્રાઇવર ચિપ, ઇનપુટ ફ્રોમ µC, કંટ્રોલ લોજિક,
એનેબલ મોટર, રોટેશન)

\end{mnemonicbox}
\subsection*{પ્રશ્ન 5(ક OR) [7
ગુણ]}\label{uxaaauxab0uxab6uxaa8-5uxa95-or-7-uxa97uxaa3}

\textbf{8051 માઇક્રોકન્ટ્રોલર સાથે DAC0808 ઇન્ટરફેસ કરો.}

\begin{solutionbox}

\textbf{ડાયાગ્રામ:}

\begin{verbatim}
+{-{-}{-}{-}{-}{-}{-}{-}+              +{-}{-}{-}{-}{-}{-}{-}{-}+          +{-}{-}{-}{-}{-}{-}{-}{-}+}
|        |              |        |          |        |
|        |{-{-}P1.0{-}P1.7{-}{-}|D0{-}D7   |          |        |}
|  8051  |              |        |{-{-}Output{-}{-}| Filter |{-}{-}{-} Analog}
|        |              | DAC0808|          |        |     Output
|        |{-{-}P3.0{-}{-}{-}{-}{-}{-}{-}|CS      |          |        |}
|        |              |        |          |        |
+{-{-}{-}{-}{-}{-}{-}{-}+              +{-}{-}{-}{-}{-}{-}{-}{-}+          +{-}{-}{-}{-}{-}{-}{-}{-}+}
                            |
                        +{-{-}{-}+{-}{-}{-}+}
                        | {-5V   |}
                        | +5V   |
                        | GND   |
                        +{-{-}{-}{-}{-}{-}{-}+}
\end{verbatim}

\textbf{જરૂરી ઘટકો}:

\begin{itemize}
\tightlist
\item
  8051 માઇક્રોકન્ટ્રોલર
\item
  DAC0808 (8-બિટ ડિજિટલ-ટુ-એનાલોગ કન્વર્ટર)
\item
  ઓપરેશનલ એમ્પ્લિફાયર (આઉટપુટ બફરિંગ માટે)
\item
  RC ફિલ્ટર (સ્મુધિંગ માટે)
\item
  રેફરન્સ વોલ્ટેજ સોર્સ
\end{itemize}

\textbf{કનેક્શન્સ}:

\begin{itemize}
\tightlist
\item
  P1.0-P1.7 \rightarrow D0-D7 (8-બિટ ડિજિટલ ઇનપુટ)
\item
  P3.0 \rightarrow CS (ચિપ સિલેક્ટ)
\item
  DAC આઉટપુટ \rightarrow ફિલ્ટર \rightarrow ફાઇનલ એનાલોગ આઉટપુટ
\end{itemize}

\textbf{રેમ્પ સિગ્નલ જનરેશન માટે સેમ્પલ કોડ}:

\begin{verbatim}
START: MOV R0, \#00H      ; Start from 0
LOOP:  MOV P1, R0        ; Output value to DAC
       CALL DELAY        ; Wait
       INC R0            ; Increment value
       SJMP LOOP         ; Loop to create ramp
\end{verbatim}

\textbf{ઉપયોગો}:

\begin{itemize}
\tightlist
\item
  વેવફોર્મ જનરેશન
\item
  પ્રોગ્રામેબલ વોલ્ટેજ સોર્સ
\item
  મોટર સ્પીડ કંટ્રોલ
\item
  ઓડિયો એપ્લિકેશન્સ
\end{itemize}

\end{solutionbox}
\begin{mnemonicbox}
``DACR'' (ડિજિટલ ઇનપુટ, એનાલોગ આઉટપુટ, કન્વર્ઝન, રેફરન્સ
વોલ્ટેજ)

\end{mnemonicbox}

\end{document}
