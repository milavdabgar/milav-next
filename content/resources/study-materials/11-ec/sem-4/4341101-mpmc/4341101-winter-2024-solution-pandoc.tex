\documentclass[10pt,a4paper]{article}

% content/resources/templates/preamble.tex
\usepackage[margin=0.6in]{geometry}
\author{Milav Dabgar}
\usepackage{amsmath,amssymb,amsthm}
\usepackage{booktabs}
\usepackage{multirow}
\usepackage{xcolor}
\usepackage{tcolorbox}
\tcbuselibrary{breakable,skins}
\usepackage[colorlinks=true,linkcolor=blue]{hyperref}
\usepackage{titlesec}
\usepackage{enumitem}
\usepackage{tikz}
\usepackage{pgfplots}
\usepackage{circuitikz}
\usepackage[version=4]{mhchem}
\usepackage{longtable}
\usepackage{array}
\usepackage{float}
\usepackage{caption}
\usepackage{listings}

\lstset{
  basicstyle=\small\ttfamily,
  breaklines=true,
  breakatwhitespace=false,
  postbreak=\mbox{\textcolor{red}{$\hookrightarrow$}\space},
  float=false,
  numbers=left,
  numberstyle=\tiny\color{gray},
  numbersep=10pt,
  xleftmargin=2em,
  keywordstyle=\color{blue},
  commentstyle=\color{green!60!black},
  stringstyle=\color{purple},
  backgroundcolor=\color{gray!5},
  showstringspaces=false,
  tabsize=2,
  captionpos=b,
  keepspaces=true,
  columns=flexible
}

\pgfplotsset{compat=1.18}
\usetikzlibrary{shapes,arrows,positioning,calc,patterns,decorations.pathmorphing,decorations.markings,arrows.meta}

% Color scheme
\definecolor{headcolor}{RGB}{0,102,204}
\definecolor{keycolor}{RGB}{220,20,60}
\definecolor{solutioncolor}{RGB}{34,139,34}
\definecolor{mnemoniccolor}{RGB}{148,0,211}
\definecolor{codecolor}{RGB}{0,0,100}

% Spacing
\setlength{\parskip}{3pt}
\setlist[itemize]{nosep}
\setlist[enumerate]{nosep}

% Title formatting
\titleformat{\section}{\Large\bfseries\color{headcolor}}{\thesection}{1em}{}
\titleformat{\subsection}{\large\bfseries\color{headcolor}}{\thesubsection}{1em}{}

% Pandoc tightlist compatibility
\providecommand{\tightlist}{%
  \setlength{\itemsep}{0pt}\setlength{\parskip}{0pt}}

% Pandoc longtable compatibility
\newcounter{none}
\def\thenone{}


% content/resources/templates/english-boxes.tex
% This file is currently empty - it exists to maintain consistency with the import structure.
% Add custom environments here if needed in the future.


\begin{document}

\begin{center}
{\Huge\bfseries\color{headcolor} Subject Name Solutions}\\[5pt]
{\LARGE 4341101 -- Winter 2024}\\[3pt]
{\large Semester 1 Study Material}\\[3pt]
{\normalsize\textit{Detailed Solutions and Explanations}}
\end{center}

\vspace{10pt}

\subsection*{Question 1(a) [3 marks]}\label{q1a}

\textbf{List Common features of Microcontrollers.}

\begin{solutionbox}

{\def\LTcaptype{none} % do not increment counter
\begin{longtable}[]{@{}ll@{}}
\toprule\noalign{}
Feature & Purpose \\
\midrule\noalign{}
\endhead
\bottomrule\noalign{}
\endlastfoot
CPU Core & Process instructions \\
Memory (RAM/ROM) & Store program and data \\
I/O Ports & Interface with external devices \\
Timers/Counters & Measure time intervals \\
Interrupts & Handle asynchronous events \\
Serial Communication & Transfer data with other devices \\
\end{longtable}
}

\end{solutionbox}
\begin{mnemonicbox}
``CRITICS: CPU ROM I/O Timers Interrupts Comm
Serial''

\end{mnemonicbox}
\subsection*{Question 1(b) [4 marks]}\label{q1b}

\textbf{Explain the functions of ALU.}

\begin{solutionbox}

{\def\LTcaptype{none} % do not increment counter
\begin{longtable}[]{@{}ll@{}}
\toprule\noalign{}
Function & Description \\
\midrule\noalign{}
\endhead
\bottomrule\noalign{}
\endlastfoot
Arithmetic Operations & Addition, subtraction, increment, decrement \\
Logical Operations & AND, OR, XOR, NOT, comparison \\
Data Movement & Transfer between registers and memory \\
Flag Setting & Update status flags based on operation results \\
\end{longtable}
}

\textbf{Diagram:}

\begin{verbatim}
      +{-{-}{-}{-}{-}{-}{-}{-}{-}{-}+}
      |   ALU    |
      |          |
Input |          | Output
Data  |          | Data
{-{-}{-}{-}{-}|          |{-}{-}{-}{-}{-}}
      |          |
      +{-{-}{-}{-}{-}{-}{-}{-}{-}{-}+}
           |
           v
       Status Flags
       (Z, C, S, P)
\end{verbatim}

\end{solutionbox}
\begin{mnemonicbox}
``ALFS: Arithmetic Logic Flags Status''

\end{mnemonicbox}
\subsection*{Question 1(c) [7 marks]}\label{q1c}

\textbf{Define: Memory, Operand, Instruction Cycle, Opcode, CU, Machine
Cycle, CISC}

\begin{solutionbox}

{\def\LTcaptype{none} % do not increment counter
\begin{longtable}[]{@{}
  >{\raggedright\arraybackslash}p{(\linewidth - 2\tabcolsep) * \real{0.3333}}
  >{\raggedright\arraybackslash}p{(\linewidth - 2\tabcolsep) * \real{0.6667}}@{}}
\toprule\noalign{}
\begin{minipage}[b]{\linewidth}\raggedright
Term
\end{minipage} & \begin{minipage}[b]{\linewidth}\raggedright
Definition
\end{minipage} \\
\midrule\noalign{}
\endhead
\bottomrule\noalign{}
\endlastfoot
\textbf{Memory} & Storage unit that holds data and instructions \\
\textbf{Operand} & Data value or address used in an operation \\
\textbf{Instruction Cycle} & Complete process of fetching and executing
an instruction \\
\textbf{Opcode} & Operation code that specifies the instruction type \\
\textbf{CU} & Control Unit that coordinates processor operations \\
\textbf{Machine Cycle} & Basic operation cycle consisting of T-states \\
\textbf{CISC} & Complex Instruction Set Computer with rich instruction
set \\
\end{longtable}
}

\begin{itemize}
\tightlist
\item
  \textbf{Memory}: Organized array of storage cells with unique
  addresses
\item
  \textbf{Operand}: Data elements that instructions operate upon
\item
  \textbf{Instruction Cycle}: Fetch-decode-execute sequence for each
  instruction
\item
  \textbf{Opcode}: Binary code that tells processor what operation to
  perform
\end{itemize}

\textbf{Diagram:}

\begin{verbatim}
Instruction Cycle:
+{-{-}{-}{-}{-}{-}{-}{-}+    +{-}{-}{-}{-}{-}{-}{-}{-}+    +{-}{-}{-}{-}{-}{-}{-}{-}+}
| Fetch  |{-{-}{-}| Decode |{-}{-}{-}|Execute |}
+{-{-}{-}{-}{-}{-}{-}{-}+    +{-}{-}{-}{-}{-}{-}{-}{-}+    +{-}{-}{-}{-}{-}{-}{-}{-}+}
\end{verbatim}

\end{solutionbox}
\begin{mnemonicbox}
``MO-ICO-MC:
Memory-Operand-Instruction-Control-Operation-Machine-Complex''

\end{mnemonicbox}
\subsection*{Question 1(c OR) [7
marks]}\label{question-1c-or-7-marks}

\textbf{i) Define: Microprocessor. ii) Compare Von-Neumann and Harvard
architecture.}

\begin{solutionbox}

\textbf{i) Microprocessor Definition:}

\begin{verbatim}
An integrated circuit containing the CPU functionality of a computer,
capable of fetching, decoding, and executing instructions with ALU and
control circuitry on a single chip.
\end{verbatim}

\textbf{ii) Von-Neumann vs Harvard Architecture:}

{\def\LTcaptype{none} % do not increment counter
\begin{longtable}[]{@{}lll@{}}
\toprule\noalign{}
Feature & Von-Neumann & Harvard \\
\midrule\noalign{}
\endhead
\bottomrule\noalign{}
\endlastfoot
Memory & Single shared memory & Separate program \& data memory \\
Bus & Single bus for data \& instructions & Separate buses \\
Speed & Slower (memory bottleneck) & Faster (parallel access) \\
Complexity & Simpler design & More complex \\
Applications & General computing & Real-time systems \\
\end{longtable}
}

\textbf{Diagram:}

\begin{verbatim}
Von{-Neumann:}
+{-{-}{-}{-}{-}{-}{-}+         +{-}{-}{-}{-}{-}{-}{-}+}
| CPU   |{=======| Memory|}
+{-{-}{-}{-}{-}{-}{-}+         +{-}{-}{-}{-}{-}{-}{-}+}

Harvard:
+{-{-}{-}{-}{-}{-}{-}+         +{-}{-}{-}{-}{-}{-}{-}{-}{-}{-}{-}+}
| CPU   |========{| Program   |}
|       |         | Memory    |
|       |         +{-{-}{-}{-}{-}{-}{-}{-}{-}{-}{-}+}
|       |         +{-{-}{-}{-}{-}{-}{-}{-}{-}{-}{-}+}
|       |{=======| Data      |}
+{-{-}{-}{-}{-}{-}{-}+         | Memory    |}
                  +{-{-}{-}{-}{-}{-}{-}{-}{-}{-}{-}+}
\end{verbatim}

\end{solutionbox}
\begin{mnemonicbox}
``Harvard Has Separate Spaces''

\end{mnemonicbox}
\subsection*{Question 2(a) [3 marks]}\label{q2a}

\textbf{Explain various Registers of 8085 microprocessor.}

\begin{solutionbox}

{\def\LTcaptype{none} % do not increment counter
\begin{longtable}[]{@{}lll@{}}
\toprule\noalign{}
Register & Size & Function \\
\midrule\noalign{}
\endhead
\bottomrule\noalign{}
\endlastfoot
Accumulator (A) & 8-bit & Main register for arithmetic \& logic \\
General Purpose (B,C,D,E,H,L) & 8-bit & Temporary data storage \\
Program Counter (PC) & 16-bit & Address of next instruction \\
Stack Pointer (SP) & 16-bit & Points to top of stack \\
Flag Register & 8-bit & Status flags (Z,S,P,CY,AC) \\
\end{longtable}
}

\end{solutionbox}
\begin{mnemonicbox}
``AGSF: Accumulator-General-Stack-Flags''

\end{mnemonicbox}
\subsection*{Question 2(b) [4 marks]}\label{q2b}

\textbf{Explain Fetching, Decoding and Execution of Instruction.}

\begin{solutionbox}

{\def\LTcaptype{none} % do not increment counter
\begin{longtable}[]{@{}
  >{\raggedright\arraybackslash}p{(\linewidth - 4\tabcolsep) * \real{0.1944}}
  >{\raggedright\arraybackslash}p{(\linewidth - 4\tabcolsep) * \real{0.2778}}
  >{\raggedright\arraybackslash}p{(\linewidth - 4\tabcolsep) * \real{0.5278}}@{}}
\toprule\noalign{}
\begin{minipage}[b]{\linewidth}\raggedright
Phase
\end{minipage} & \begin{minipage}[b]{\linewidth}\raggedright
Activity
\end{minipage} & \begin{minipage}[b]{\linewidth}\raggedright
Hardware Involved
\end{minipage} \\
\midrule\noalign{}
\endhead
\bottomrule\noalign{}
\endlastfoot
Fetching & Get instruction from memory address in PC & PC, Address bus,
Memory \\
Decoding & Identify operation type and operands & Instruction Register,
Control Unit \\
Execution & Perform specified operation & ALU, Registers, Data bus \\
\end{longtable}
}

\textbf{Diagram:}

\begin{verbatim}
+{-{-}{-}{-}{-}{-}{-}{-}+    +{-}{-}{-}{-}{-}{-}{-}{-}+    +{-}{-}{-}{-}{-}{-}{-}{-}+}
| Fetch  |{-{-}{-}| Decode |{-}{-}{-}|Execute |}
+{-{-}{-}{-}{-}{-}{-}{-}+    +{-}{-}{-}{-}{-}{-}{-}{-}+    +{-}{-}{-}{-}{-}{-}{-}{-}+}
    \^{                           |}
    |                           |
    +{-{-}{-}{-}{-}{-}{-}{-}{-}{-}{-}{-}{-}{-}{-}{-}{-}{-}{-}{-}{-}{-}{-}{-}{-}{-}{-}+}
      Next Instruction Cycle
\end{verbatim}

\begin{itemize}
\tightlist
\item
  \textbf{Fetching}: PC sends address to memory, instruction loaded into
  IR
\item
  \textbf{Decoding}: Control unit interprets instruction opcode and
  addressing mode
\item
  \textbf{Execution}: ALU performs arithmetic/logic, data moves between
  registers/memory
\end{itemize}

\end{solutionbox}
\begin{mnemonicbox}
``FDE: First Get, Then Understand, Finally Do''

\end{mnemonicbox}
\subsection*{Question 2(c) [7 marks]}\label{q2c}

\textbf{Describe block diagram of 8085 microprocessor with the help of
neat diagram.}

\begin{solutionbox}

{\def\LTcaptype{none} % do not increment counter
\begin{longtable}[]{@{}ll@{}}
\toprule\noalign{}
Block & Function \\
\midrule\noalign{}
\endhead
\bottomrule\noalign{}
\endlastfoot
ALU & Arithmetic \& logical operations \\
Register Array & Temporary data storage \\
Instruction Register \& Decoder & Hold \& interpret instructions \\
Control \& Timing Unit & Generate control signals \\
Address Buffer & Interface with address bus \\
Data Buffer & Interface with data bus \\
Serial I/O & Communication with SID/SOD \\
Interrupt Control & Handle interrupt requests \\
\end{longtable}
}

\textbf{Diagram:}

\begin{verbatim}
+{-{-}{-}{-}{-}{-}{-}{-}{-}{-}{-}{-}{-}{-}{-}{-}{-}{-}{-}{-}{-}{-}{-}{-}{-}{-}{-}{-}{-}{-}{-}{-}{-}{-}{-}{-}{-}{-}{-}{-}{-}{-}{-}{-}{-}{-}{-}{-}{-}{-}{-}{-}{-}{-}+}
|                  8085 MICROPROCESSOR                 |
| +{-{-}{-}{-}{-}{-}{-}{-}{-}{-}{-}{-}{-}{-}{-}{-}+     +{-}{-}{-}{-}{-}{-}{-}{-}{-}{-}{-}{-}{-}{-}{-}{-}{-}{-}{-}{-}{-}{-}{-}{-}{-}+   |}
| | REGISTER ARRAY |     |                         |   |
| |B  C  D  E  H  L|{{-}{-}{-}|         ALU             |   |}
| +{-{-}{-}{-}{-}{-}{-}{-}{-}{-}{-}{-}{-}{-}{-}{-}+     |                         |   |}
| +{-{-}{-}{-}{-}{-}{-}{-}{-}{-}{-}{-}{-}{-}{-}{-}+     |                         |   |}
| | ACCUMULATOR    |{{-}{-}{-}|                         |   |}
| +{-{-}{-}{-}{-}{-}{-}{-}{-}{-}{-}{-}{-}{-}{-}{-}+     +{-}{-}{-}{-}{-}{-}{-}{-}{-}{-}{-}{-}{-}{-}{-}{-}{-}{-}{-}{-}{-}{-}{-}{-}{-}+   |}
|                                                      |
| +{-{-}{-}{-}{-}{-}{-}{-}{-}{-}{-}{-}{-}{-}{-}{-}+     +{-}{-}{-}{-}{-}{-}{-}{-}{-}{-}{-}{-}{-}{-}{-}{-}{-}{-}{-}{-}{-}{-}{-}{-}{-}+   |}
| |INSTR. REGISTER |{{-}{-}{-}|  CONTROL \& TIMING UNIT  |   |}
| | \& DECODER      |     |                         |   |
| +{-{-}{-}{-}{-}{-}{-}{-}{-}{-}{-}{-}{-}{-}{-}{-}+     +{-}{-}{-}{-}{-}{-}{-}{-}{-}{-}{-}{-}{-}{-}{-}{-}{-}{-}{-}{-}{-}{-}{-}{-}{-}+   |}
|                              |                       |
| +{-{-}{-}{-}{-}{-}{-}{-}{-}{-}{-}{-}{-}{-}{-}{-}+     +{-}{-}{-}{-}{-}|{-}{-}{-}{-}{-}{-}{-}{-}{-}{-}{-}{-}{-}{-}{-}{-}{-}{-}{-}+   |}
| |ADDRESS BUFFER  |{{-}{-}{-}|   INTERRUPT CONTROL     |   |}
| +{-{-}{-}{-}{-}{-}{-}{-}{-}{-}{-}{-}{-}{-}{-}{-}+     +{-}{-}{-}{-}{-}{-}{-}{-}{-}{-}{-}{-}{-}{-}{-}{-}{-}{-}{-}{-}{-}{-}{-}{-}{-}+   |}
| +{-{-}{-}{-}{-}{-}{-}{-}{-}{-}{-}{-}{-}{-}{-}{-}+     +{-}{-}{-}{-}{-}{-}{-}{-}{-}{-}{-}{-}{-}{-}{-}{-}{-}{-}{-}{-}{-}{-}{-}{-}{-}+   |}
| |  DATA BUFFER   |{{-}{-}{-}|      SERIAL I/O         |   |}
| +{-{-}{-}{-}{-}{-}{-}{-}{-}{-}{-}{-}{-}{-}{-}{-}+     +{-}{-}{-}{-}{-}{-}{-}{-}{-}{-}{-}{-}{-}{-}{-}{-}{-}{-}{-}{-}{-}{-}{-}{-}{-}+   |}
+{-{-}{-}{-}{-}{-}{-}{-}{-}{-}{-}{-}{-}{-}{-}{-}{-}{-}{-}{-}{-}{-}{-}{-}{-}{-}{-}{-}{-}{-}{-}{-}{-}{-}{-}{-}{-}{-}{-}{-}{-}{-}{-}{-}{-}{-}{-}{-}{-}{-}{-}{-}{-}{-}+}
\end{verbatim}

\begin{itemize}
\tightlist
\item
  \textbf{Core Components}: ALU and registers form processing core
\item
  \textbf{Control Path}: Instructions flow through register, decoder,
  control unit
\item
  \textbf{Data Path}: Data moves through buffers to/from external buses
\item
  \textbf{Timing}: Synchronizes all operations via internal clock
\end{itemize}

\end{solutionbox}
\begin{mnemonicbox}
``RAID: Registers-ALU-Instructions-Decoders''

\end{mnemonicbox}
\subsection*{Question 2(a OR) [3
marks]}\label{question-2a-or-3-marks}

\textbf{Compare Microprocessor \& Microcontroller.}

\begin{solutionbox}

{\def\LTcaptype{none} % do not increment counter
\begin{longtable}[]{@{}lll@{}}
\toprule\noalign{}
Feature & Microprocessor & Microcontroller \\
\midrule\noalign{}
\endhead
\bottomrule\noalign{}
\endlastfoot
Design & CPU only & CPU + peripherals \\
Memory & External & Internal (RAM/ROM) \\
I/O ports & Limited & Many built-in \\
Applications & General computing & Embedded systems \\
Cost & Higher & Lower \\
Example & Intel 8085/8086 & Intel 8051 \\
\end{longtable}
}

\end{solutionbox}
\begin{mnemonicbox}
``Micro-P Processes, Micro-C Controls''

\end{mnemonicbox}
\subsection*{Question 2(b OR) [4
marks]}\label{question-2b-or-4-marks}

\textbf{Explain De-multiplexing of Address and Data buses for 8085
Microprocessor.}

\begin{solutionbox}

{\def\LTcaptype{none} % do not increment counter
\begin{longtable}[]{@{}ll@{}}
\toprule\noalign{}
Step & Action \\
\midrule\noalign{}
\endhead
\bottomrule\noalign{}
\endlastfoot
1 & ALE signal goes high \\
2 & Lower address (A0-A7) appears on AD0-AD7 \\
3 & Latch captures address using ALE \\
4 & ALE goes low, AD0-AD7 now carries data \\
\end{longtable}
}

\textbf{Diagram:}

\begin{verbatim}
       8085            Latch           Memory
       {-{-}{-}{-}            {-}{-}{-}{-}{-}           {-}{-}{-}{-}{-}{-}}
                         |
AD0{-AD7 {-}{-}{-}{-}+{-}{-}{-}{-}{-}{-} |   |{-}{-}{-}{-}{-} A0{-}A7 {-}{-}{-} Memory}
            |         {-{-}{-}            Address}
            |          \^{}
            v          |
          Data Bus     ALE
\end{verbatim}

\begin{itemize}
\tightlist
\item
  \textbf{Multiplexing}: AD0-AD7 pins carry address and data at
  different times
\item
  \textbf{ALE Signal}: Address Latch Enable controls when address is
  captured
\item
  \textbf{8-bit Latch}: Holds lower address bits during entire machine
  cycle
\item
  \textbf{Timing}: Address valid only during high state of ALE pulse
\end{itemize}

\end{solutionbox}
\begin{mnemonicbox}
``ALAD: ALE Latches Address before Data''

\end{mnemonicbox}
\subsection*{Question 2(c OR) [7
marks]}\label{question-2c-or-7-marks}

\textbf{Describe Pin diagram of 8085 microprocessor with the help of
neat diagram.}

\begin{solutionbox}

{\def\LTcaptype{none} % do not increment counter
\begin{longtable}[]{@{}ll@{}}
\toprule\noalign{}
Pin Group & Function \\
\midrule\noalign{}
\endhead
\bottomrule\noalign{}
\endlastfoot
Address/Data & Multiplexed AD0-AD7, A8-A15 \\
Control & RD, WR, IO/M, S0, S1, ALE, CLK \\
Interrupts & INTR, RST 5.5-7.5, TRAP \\
DMA & HOLD, HLDA \\
Power & Vcc, Vss \\
Serial I/O & SID, SOD \\
Reset & RESET IN, RESET OUT \\
\end{longtable}
}

\textbf{Diagram:}

\begin{verbatim}
            +{-{-}{-}{-}{-}{-}{-}+}
      X1 {-{-}|1    40|{-}{-} Vcc}
      X2 {-{-}|2    39|{-}{-} HOLD}
RESET OUT{-{-}|3    38|{-}{-} HLDA}
RESET IN {-{-}|4    37|{-}{-} CLK}
    IO/M {-{-}|5    36|{-}{-} RESET IN}
      S1 {-{-}|6    35|{-}{-} READY}
      RD {-{-}|7    34|{-}{-} IO/M}
      WR {-{-}|8    33|{-}{-} S1}
     ALE {-{-}|9    32|{-}{-} RD}
      S0 {-{-}|10   31|{-}{-} WR}
     A15 {-{-}|11   30|{-}{-} ALE}
     A14 {-{-}|12   29|{-}{-} S0}
     A13 {-{-}|13   28|{-}{-} A15}
     A12 {-{-}|14   27|{-}{-} A14}
     A11 {-{-}|15   26|{-}{-} A13}
     A10 {-{-}|16   25|{-}{-} A12}
      A9 {-{-}|17   24|{-}{-} A11}
      A8 {-{-}|18   23|{-}{-} A10}
     AD7 {-{-}|19   22|{-}{-} A9}
     AD6 {-{-}|20   21|{-}{-} A8}
            +{-{-}{-}{-}{-}{-}{-}+}
\end{verbatim}

\begin{itemize}
\tightlist
\item
  \textbf{Address Pins}: A8-A15 (8) and multiplexed AD0-AD7 (8) for
  16-bit addressing
\item
  \textbf{Control Pins}: Generate timing and control signals for
  peripheral devices
\item
  \textbf{Interrupt Pins}: Hardware interrupt handling with priority
  levels
\item
  \textbf{Clock}: X1, X2 for crystal connection, CLK for synchronization
\item
  \textbf{Power}: Vcc (+5V) and Vss (Ground) for power supply
\end{itemize}

\end{solutionbox}
\begin{mnemonicbox}
``ACID-PS:
Address-Control-Interrupt-DMA-Power-Serial''

\end{mnemonicbox}
\subsection*{Question 3(a) [3 marks]}\label{q3a}

\textbf{Explain interrupts of 8051 microcontroller.}

\begin{solutionbox}

{\def\LTcaptype{none} % do not increment counter
\begin{longtable}[]{@{}llll@{}}
\toprule\noalign{}
Interrupt & Vector & Priority & Source \\
\midrule\noalign{}
\endhead
\bottomrule\noalign{}
\endlastfoot
External 0 & 0003H & 1 (IP.0) & Pin INT0 (P3.2) \\
Timer 0 & 000BH & 2 (IP.1) & Timer 0 overflow \\
External 1 & 0013H & 3 (IP.2) & Pin INT1 (P3.3) \\
Timer 1 & 001BH & 4 (IP.3) & Timer 1 overflow \\
Serial & 0023H & 5 (IP.4) & Serial port events \\
\end{longtable}
}

\textbf{Diagram:}

\begin{verbatim}
           +{-{-}{-}{-}{-}{-}+}
           | 8051 |
 INT0 {-{-}{-}{-}|      |}
           |      |
 INT1 {-{-}{-}{-}|      |    Interrupts are}
           |      |    prioritized and
TIMER0 {-{-}{-}|      |    can be enabled/disabled}
           |      |    individually
TIMER1 {-{-}{-}|      |}
           |      |
SERIAL {-{-}{-}|      |}
           +{-{-}{-}{-}{-}{-}+}
\end{verbatim}

\end{solutionbox}
\begin{mnemonicbox}
``ETTES: External-Timer-Timer-External-Serial''

\end{mnemonicbox}
\subsection*{Question 3(b) [4 marks]}\label{q3b}

\textbf{Draw Pin diagram of 8051 microcontroller.}

\begin{solutionbox}

\begin{verbatim}
          8051 Microcontroller
         +{-{-}{-}{-}{-}{-}{-}{-}{-}{-}{-}{-}{-}{-}{-}{-}{-}{-}{-}+}
   P1.0{-{-}| 1              40 |{-}{-}VCC}
   P1.1{-{-}| 2              39 |{-}{-}P0.0/AD0}
   P1.2{-{-}| 3              38 |{-}{-}P0.1/AD1}
   P1.3{-{-}| 4              37 |{-}{-}P0.2/AD2}
   P1.4{-{-}| 5              36 |{-}{-}P0.3/AD3}
   P1.5{-{-}| 6              35 |{-}{-}P0.4/AD4}
   P1.6{-{-}| 7              34 |{-}{-}P0.5/AD5}
   P1.7{-{-}| 8              33 |{-}{-}P0.6/AD6}
   RST {-{-}| 9              32 |{-}{-}P0.7/AD7}
 P3.0/RXD| 10             31 |{-{-}EA/VPP}
 P3.1/TXD| 11             30 |{-{-}ALE/PROG}
P3.2/INT0| 12             29 |{-{-}PSEN}
P3.3/INT1| 13             28 |{-{-}P2.7/A15}
 P3.4/T0{-| 14             27 |{-}{-}P2.6/A14}
 P3.5/T1{-| 15             26 |{-}{-}P2.5/A13}
 P3.6/WR{-| 16             25 |{-}{-}P2.4/A12}
 P3.7/RD{-| 17             24 |{-}{-}P2.3/A11}
 XTAL2 {-{-}| 18             23 |{-}{-}P2.2/A10}
 XTAL1 {-{-}| 19             22 |{-}{-}P2.1/A9}
   VSS {-{-}| 20             21 |{-}{-}P2.0/A8}
         +{-{-}{-}{-}{-}{-}{-}{-}{-}{-}{-}{-}{-}{-}{-}{-}{-}{-}{-}+}
\end{verbatim}

{\def\LTcaptype{none} % do not increment counter
\begin{longtable}[]{@{}ll@{}}
\toprule\noalign{}
Pin Group & Function \\
\midrule\noalign{}
\endhead
\bottomrule\noalign{}
\endlastfoot
P0 & Port 0, multiplexed with address/data \\
P1 & Port 1, general purpose I/O \\
P2 & Port 2, upper address and I/O \\
P3 & Port 3, special functions and I/O \\
XTAL & Crystal oscillator connections \\
Control & RST, EA, ALE, PSEN \\
\end{longtable}
}

\end{solutionbox}
\begin{mnemonicbox}
``PORT 0123: Data-General-Address-Special''

\end{mnemonicbox}
\subsection*{Question 3(c) [7 marks]}\label{q3c}

\textbf{Explain Internal RAM Organization of 8051 microcontroller.}

\begin{solutionbox}

{\def\LTcaptype{none} % do not increment counter
\begin{longtable}[]{@{}lll@{}}
\toprule\noalign{}
RAM Area & Address Range & Usage \\
\midrule\noalign{}
\endhead
\bottomrule\noalign{}
\endlastfoot
Register Banks & 00H-1FH & R0-R7 (4 banks) \\
Bit-addressable & 20H-2FH & 128 bits (0-7FH) \\
Scratch Pad & 30H-7FH & General purpose \\
SFRs & 80H-FFH & Control registers \\
\end{longtable}
}

\textbf{Diagram:}

\begin{verbatim}
8051 Internal RAM (128 bytes):
+{-{-}{-}{-}{-}{-}{-}{-}{-}{-}{-}{-}{-}{-}{-}{-}+ 00H}
| Register Bank 0|
+{-{-}{-}{-}{-}{-}{-}{-}{-}{-}{-}{-}{-}{-}{-}{-}+ 08H}
| Register Bank 1|
+{-{-}{-}{-}{-}{-}{-}{-}{-}{-}{-}{-}{-}{-}{-}{-}+ 10H}
| Register Bank 2|
+{-{-}{-}{-}{-}{-}{-}{-}{-}{-}{-}{-}{-}{-}{-}{-}+ 18H}
| Register Bank 3|
+{-{-}{-}{-}{-}{-}{-}{-}{-}{-}{-}{-}{-}{-}{-}{-}+ 20H}
| Bit{-addressable|}
+{-{-}{-}{-}{-}{-}{-}{-}{-}{-}{-}{-}{-}{-}{-}{-}+ 30H}
|                |
| Scratch Pad    |
|                |
+{-{-}{-}{-}{-}{-}{-}{-}{-}{-}{-}{-}{-}{-}{-}{-}+ 80H}

Special Function Registers:
+{-{-}{-}{-}{-}{-}{-}{-}{-}{-}{-}{-}{-}{-}{-}{-}+ 80H}
|                |
| SFRs           |
| (not all bytes |
| are used)      |
|                |
+{-{-}{-}{-}{-}{-}{-}{-}{-}{-}{-}{-}{-}{-}{-}{-}+ FFH}
\end{verbatim}

\begin{itemize}
\tightlist
\item
  \textbf{Register Banks}: Four banks of 8 registers (R0-R7) selectable
  by PSW
\item
  \textbf{Bit-addressable}: 16 bytes (128 bits) individually addressable
  as bits
\item
  \textbf{General Purpose}: User variables and stack space
\item
  \textbf{SFRs}: Control and status registers at higher addresses
\end{itemize}

\end{solutionbox}
\begin{mnemonicbox}
``RBBS: Registers Bits Buffer Special''

\end{mnemonicbox}
\subsection*{Question 3(a OR) [3
marks]}\label{question-3a-or-3-marks}

\textbf{List SFRs with their addresses.}

\begin{solutionbox}

{\def\LTcaptype{none} % do not increment counter
\begin{longtable}[]{@{}lll@{}}
\toprule\noalign{}
SFR & Address & Function \\
\midrule\noalign{}
\endhead
\bottomrule\noalign{}
\endlastfoot
P0 & 80H & Port 0 \\
SP & 81H & Stack Pointer \\
DPL & 82H & Data Pointer Low \\
DPH & 83H & Data Pointer High \\
PCON & 87H & Power Control \\
TCON & 88H & Timer Control \\
TMOD & 89H & Timer Mode \\
P1 & 90H & Port 1 \\
SCON & 98H & Serial Control \\
P2 & A0H & Port 2 \\
IE & A8H & Interrupt Enable \\
P3 & B0H & Port 3 \\
IP & B8H & Interrupt Priority \\
PSW & D0H & Program Status Word \\
ACC & E0H & Accumulator \\
B & F0H & B Register \\
\end{longtable}
}

\end{solutionbox}
\begin{mnemonicbox}
``PDPT-SP:
Ports-Data-Program-Timers-Serial-Prioritized''

\end{mnemonicbox}
\subsection*{Question 3(b OR) [4
marks]}\label{question-3b-or-4-marks}

\textbf{Explain Timers/Counters logic diagram of 8051 microcontroller.}

\begin{solutionbox}

\textbf{Timer/Counter Diagram:}

\begin{verbatim}
         +{-{-}{-}{-}{-}{-}{-}{-}{-}{-}{-}{-}+}
TLx {-{-}{-}{-}|  8{-}bit     |       +{-}{-}{-}{-}{-}{-}{-}{-}{-}{-}{-}{-}{-}+}
         |  Register  |{-{-}{-}{-}{-}{-}|  8{-}bit      |}
         +{-{-}{-}{-}{-}{-}{-}{-}{-}{-}{-}{-}+       |  Register   |{-}{-}{-}{-} Interrupt}
                              |  (THx)      |
         +{-{-}{-}{-}{-}{-}{-}{-}{-}{-}{-}{-}+       +{-}{-}{-}{-}{-}{-}{-}{-}{-}{-}{-}{-}{-}+}
TRx {-{-}{-}{-}| Control    |             \^{}}
         | Logic      |             |
         +{-{-}{-}{-}{-}{-}{-}{-}{-}{-}{-}{-}+             |}
                \^{                   |}
                |                   |
                v                   v
         +{-{-}{-}{-}{-}{-}{-}{-}{-}{-}{-}{-}{-}{-}{-}{-}{-}{-}{-}{-}{-}+}
C/T {-{-}{-}{-}| Mode Control Logic  |{-}{-}{-}{-}{-} GATE}
         +{-{-}{-}{-}{-}{-}{-}{-}{-}{-}{-}{-}{-}{-}{-}{-}{-}{-}{-}{-}{-}+}
                   \^{}
                   |
INTx {-{-}{-}{-}{-}{-}{-}{-}{-}{-}{-}{-}{-}{-}}
\end{verbatim}

{\def\LTcaptype{none} % do not increment counter
\begin{longtable}[]{@{}ll@{}}
\toprule\noalign{}
Component & Function \\
\midrule\noalign{}
\endhead
\bottomrule\noalign{}
\endlastfoot
TLx, THx & Timer low and high byte registers \\
C/T & Selects Timer (0) or Counter (1) mode \\
GATE & External enable control \\
TRx & Timer run control bit \\
Mode Control & Selects one of four operation modes \\
\end{longtable}
}

\begin{itemize}
\tightlist
\item
  \textbf{Timer}: Uses internal clock, counts machine cycles
\item
  \textbf{Counter}: Counts external events on T0/T1 pins
\item
  \textbf{Control Bits}: Set in TMOD and TCON registers
\item
  \textbf{Modes}: Different timer configurations (13/16/8-bit)
\end{itemize}

\end{solutionbox}
\begin{mnemonicbox}
``TCG: Timer-Counter-Gate''

\end{mnemonicbox}
\subsection*{Question 3(c OR) [7
marks]}\label{question-3c-or-7-marks}

\textbf{Explain block diagram of 8051 microcontroller.}

\begin{solutionbox}

{\def\LTcaptype{none} % do not increment counter
\begin{longtable}[]{@{}ll@{}}
\toprule\noalign{}
Component & Function \\
\midrule\noalign{}
\endhead
\bottomrule\noalign{}
\endlastfoot
CPU & 8-bit processor with ALU \\
Memory & 4K ROM, 128 bytes RAM \\
I/O Ports & Four 8-bit ports (P0-P3) \\
Timers & Two 16-bit timers/counters \\
Serial Port & Full-duplex UART \\
Interrupts & Five interrupt sources \\
Special Function Registers & Control registers \\
\end{longtable}
}

\textbf{Diagram:}

\begin{verbatim}
+{-{-}{-}{-}{-}{-}{-}{-}{-}{-}{-}{-}{-}{-}{-}{-}{-}{-}{-}{-}{-}{-}{-}{-}{-}{-}{-}{-}{-}{-}{-}{-}{-}{-}{-}{-}{-}{-}{-}{-}{-}{-}{-}{-}+}
|                 8051 MCU                   |
| +{-{-}{-}{-}{-}{-}{-}{-}{-}{-}{-}{-}{-}+         +{-}{-}{-}{-}{-}{-}{-}{-}{-}{-}{-}{-}{-}{-}+   |}
| |             |         |              |   |
| |    CPU      |{{-}{-}{-}{-}{-}{-}{-}| Program ROM  |   |}
| |             |         | (4K bytes)   |   |
| +{-{-}{-}{-}{-}{-}{-}{-}{-}{-}{-}{-}{-}+         +{-}{-}{-}{-}{-}{-}{-}{-}{-}{-}{-}{-}{-}{-}+   |}
|       \^{                                    |}
|       |                 +{-{-}{-}{-}{-}{-}{-}{-}{-}{-}{-}{-}{-}{-}+   |}
|       |                 |              |   |
|       +{-{-}{-}{-}{-}{-}{-}{-}{-}{-}{-}{-}{-}{-}{-}{-}| Internal RAM |   |}
|       |                 | (128 bytes)  |   |
|       v                 +{-{-}{-}{-}{-}{-}{-}{-}{-}{-}{-}{-}{-}{-}+   |}
| +{-{-}{-}{-}{-}{-}{-}{-}{-}{-}{-}{-}{-}+         +{-}{-}{-}{-}{-}{-}{-}{-}{-}{-}{-}{-}{-}{-}+   |}
| |             |         |              |   |
| |  SFRs       |{{-}{-}{-}{-}{-}{-}{-}| I/O Ports    |   |}
| |             |         | (P0,P1,P2,P3)|   |
| +{-{-}{-}{-}{-}{-}{-}{-}{-}{-}{-}{-}{-}+         +{-}{-}{-}{-}{-}{-}{-}{-}{-}{-}{-}{-}{-}{-}+   |}
|                                            |
| +{-{-}{-}{-}{-}{-}{-}{-}{-}{-}{-}{-}{-}+         +{-}{-}{-}{-}{-}{-}{-}{-}{-}{-}{-}{-}{-}{-}+   |}
| |             |         |              |   |
| | Timers/     |         | Serial Port  |   |
| | Counters    |         | (UART)       |   |
| +{-{-}{-}{-}{-}{-}{-}{-}{-}{-}{-}{-}{-}+         +{-}{-}{-}{-}{-}{-}{-}{-}{-}{-}{-}{-}{-}{-}+   |}
+{-{-}{-}{-}{-}{-}{-}{-}{-}{-}{-}{-}{-}{-}{-}{-}{-}{-}{-}{-}{-}{-}{-}{-}{-}{-}{-}{-}{-}{-}{-}{-}{-}{-}{-}{-}{-}{-}{-}{-}{-}{-}{-}{-}+}
\end{verbatim}

\begin{itemize}
\tightlist
\item
  \textbf{Harvard Architecture}: Separate program and data memory
\item
  \textbf{CISC Design}: Rich instruction set (over 100 instructions)
\item
  \textbf{In-built Peripherals}: No need for external components
\item
  \textbf{Single-chip Solution}: Complete system on one chip
\end{itemize}

\end{solutionbox}
\begin{mnemonicbox}
``CAPITALS: CPU Architecture Ports I/O Timer ALU
LS-Interface Serial''

\end{mnemonicbox}
\subsection*{Question 4(a) [3 marks]}\label{q4a}

\textbf{Write an 8051 Assembly Language Program to add two bytes of data
and store result in R4 register.}

\begin{solutionbox}

\begin{verbatim}
      MOV  A, \#25H    ; Load first value (25H) into accumulator
      MOV  R3, \#18H   ; Load second value (18H) into R3
      ADD  A, R3      ; Add R3 to accumulator
      MOV  R4, A      ; Store result in R4 register
\end{verbatim}

\textbf{Key Steps:}

\begin{itemize}
\tightlist
\item
  Load first operand into Accumulator
\item
  Load second operand into register R3
\item
  Perform addition using ADD instruction
\item
  Store result from Accumulator to R4
\end{itemize}

\end{solutionbox}
\begin{mnemonicbox}
``LLAS: Load-Load-Add-Store''

\end{mnemonicbox}
\subsection*{Question 4(b) [4 marks]}\label{q4b}

\textbf{Write an 8051 Assembly Language Program to OR the contents of
Port-1 and Port-2 then put the result in external RAM location 0200H.}

\begin{solutionbox}

\begin{verbatim}
      MOV  A, P1      ; Load contents of Port{-1 into accumulator}
      ORL  A, P2      ; OR Port{-2 contents with accumulator}
      MOV  DPTR, \#0200H ; Load DPTR with external RAM address
      MOVX @DPTR, A    ; Store result in external RAM location 0200H
\end{verbatim}

\textbf{Key Steps:}

\begin{itemize}
\tightlist
\item
  Read Port-1 into Accumulator
\item
  Perform OR operation with Port-2
\item
  Set up Data Pointer (DPTR) for external RAM
\item
  Write result to external memory
\end{itemize}

\end{solutionbox}
\begin{mnemonicbox}
``PORT: Port-OR-Register-Transfer''

\end{mnemonicbox}
\subsection*{Question 4(c) [7 marks]}\label{q4c}

\textbf{List Addressing Modes of 8051 Microcontroller and explain them
with at least one example.}

\begin{solutionbox}

{\def\LTcaptype{none} % do not increment counter
\begin{longtable}[]{@{}lll@{}}
\toprule\noalign{}
Addressing Mode & Example & Description \\
\midrule\noalign{}
\endhead
\bottomrule\noalign{}
\endlastfoot
Immediate & MOV A, \#25H & Data is in instruction \\
Register & MOV A, R0 & Data is in register \\
Direct & MOV A, 30H & Data is at RAM address \\
Indirect & MOV A, @R0 & R0/R1 contains address \\
Indexed & MOVC A, @A+DPTR & Access program memory \\
Bit & SETB P1.3 & Access individual bits \\
Relative & SJMP LABEL & Jumps with 8-bit offset \\
\end{longtable}
}

\textbf{Examples:}

\begin{itemize}
\tightlist
\item
  \textbf{Immediate}: \texttt{MOV\ A,\ \#55H} (Load A with 55H)
\item
  \textbf{Register}: \texttt{ADD\ A,\ R3} (Add R3 to A)
\item
  \textbf{Direct}: \texttt{MOV\ 40H,\ A} (Store A at address 40H)
\item
  \textbf{Indirect}: \texttt{MOV\ @R0,\ \#5} (Store 5 at address in R0)
\item
  \textbf{Indexed}: \texttt{MOVC\ A,\ @A+DPTR} (Read code memory)
\item
  \textbf{Bit}: \texttt{CLR\ C} (Clear carry flag)
\item
  \textbf{Relative}: \texttt{JZ\ LOOP} (Jump if A is zero)
\end{itemize}

\end{solutionbox}
\begin{mnemonicbox}
``I'M DIRBI: Immediate Register Direct Bit Indexed''

\end{mnemonicbox}
\subsection*{Question 4(a OR) [3
marks]}\label{question-4a-or-3-marks}

\textbf{Explain following instructions: (i) DJNZ (ii) POP (iii) CJNE.}

\begin{solutionbox}

{\def\LTcaptype{none} % do not increment counter
\begin{longtable}[]{@{}lll@{}}
\toprule\noalign{}
Instruction & Syntax & Operation \\
\midrule\noalign{}
\endhead
\bottomrule\noalign{}
\endlastfoot
DJNZ & DJNZ Rn, rel & Decrement register, Jump if Not Zero \\
POP & POP direct & Pop data from stack to direct address \\
CJNE & CJNE A, \#data, rel & Compare and Jump if Not Equal \\
\end{longtable}
}

\textbf{Examples and Explanation:}

\begin{itemize}
\tightlist
\item
  \textbf{DJNZ R7, LOOP}: Decrements R7 and jumps to LOOP if R7\neq0

  \begin{itemize}
  \tightlist
  \item
    Used for loop counters and delays
  \end{itemize}
\item
  \textbf{POP 30H}: Copies data from stack to address 30H

  \begin{itemize}
  \tightlist
  \item
    Increments SP after data retrieval
  \end{itemize}
\item
  \textbf{CJNE A, \#25H, NOTEQUAL}: Compares A with 25H, jumps if not
  equal

  \begin{itemize}
  \tightlist
  \item
    Also sets Carry flag if A \textless{} data
  \end{itemize}
\end{itemize}

\end{solutionbox}
\begin{mnemonicbox}
``DPC: Decrement-Pop-Compare''

\end{mnemonicbox}
\subsection*{Question 4(b OR) [4
marks]}\label{question-4b-or-4-marks}

\textbf{For 8051 Microcontroller with a crystal frequency of 12 MHz,
generate a delay of 4ms.}

\begin{solutionbox}

\begin{verbatim}
; Crystal frequency = 12 MHz
; Machine cycle = 1 μs
; Required delay = 4 ms = 4000 machine cycles

      MOV  R7, \#16    ; Outer loop counter (16 x 250 = 4000)
DELAY1: MOV  R6, \#250  ; Inner loop counter 
DELAY2: NOP            ; 1 machine cycle
        NOP            ; 1 machine cycle
        DJNZ R6, DELAY2 ; 2 machine cycles (250 x 4 = 1000 cycles)
        DJNZ R7, DELAY1 ; 16 x 250 = 4000 cycles total
        RET            ; Return from subroutine
\end{verbatim}

\textbf{Calculation:}

\begin{itemize}
\tightlist
\item
  Each inner loop: 4 cycles \times 250 iterations = 1000
\item
  Outer loop: 16 iterations \times 1000 cycles = 16,000 cycles
\item
  At 12MHz, 1 machine cycle = 1μs
\item
  Total delay = 4ms (4000 cycles)
\end{itemize}

\end{solutionbox}
\begin{mnemonicbox}
``LNDD: Load-NOP-Decrement-Decrement''

\end{mnemonicbox}
\subsection*{Question 4(c OR) [7
marks]}\label{question-4c-or-7-marks}

\textbf{Explain any seven Logical instructions with example for 8051
Microcontroller.}

\begin{solutionbox}

{\def\LTcaptype{none} % do not increment counter
\begin{longtable}[]{@{}lll@{}}
\toprule\noalign{}
Instruction & Example & Operation \\
\midrule\noalign{}
\endhead
\bottomrule\noalign{}
\endlastfoot
ANL & ANL A, \#3FH & Logical AND \\
ORL & ORL P1, \#80H & Logical OR \\
XRL & XRL A, R0 & Logical XOR \\
CLR & CLR A & Clear (set to 0) \\
CPL & CPL P1.0 & Complement (invert) \\
RL & RL A & Rotate left \\
RR & RR A & Rotate right \\
\end{longtable}
}

\textbf{Examples with Explanation:}

\begin{enumerate}
\tightlist
\item
  \textbf{ANL A, \#0FH}: Masks high nibble (A = A AND 0FH)

  \begin{itemize}
  \tightlist
  \item
Before:

A = 95H, After:

A = 05H

  \end{itemize}
\item
  \textbf{ORL 20H, A}: Sets bits in memory (20H = 20H OR A)

  \begin{itemize}
  \tightlist
  \item
Before: 20H = 55H,

A = 0AH, After: 20H = 5FH

  \end{itemize}
\item
  \textbf{XRL A, \#55H}: Toggles specific bits (A = A XOR 55H)

  \begin{itemize}
  \tightlist
  \item
Before:

A = AAH, After:

A = FFH

  \end{itemize}
\item
  \textbf{CLR C}: Clears carry flag (C = 0)

  \begin{itemize}
  \tightlist
  \item
    Used before subtraction operations
  \end{itemize}
\item
  \textbf{CPL A}: Inverts all bits (A = NOT A)

  \begin{itemize}
  \tightlist
  \item
Before:

A = 55H, After:

A = AAH

  \end{itemize}
\item
  \textbf{RL A}: Rotates accumulator left one bit

  \begin{itemize}
  \tightlist
  \item
Before:

A = 85H (10000101), After:

A = 0BH (00001011)

  \end{itemize}
\item
  \textbf{RR A}: Rotates accumulator right one bit

  \begin{itemize}
  \tightlist
  \item
Before:

A = 85H (10000101), After:

A = C2H (11000010)

  \end{itemize}
\end{enumerate}

\end{solutionbox}
\begin{mnemonicbox}
``A-OX-CCR: AND OR XOR Clear Complement Rotate''

\end{mnemonicbox}
\subsection*{Question 5(a) [3 marks]}\label{q5a}

\textbf{List Applications of microcontroller in various fields.}

\begin{solutionbox}

{\def\LTcaptype{none} % do not increment counter
\begin{longtable}[]{@{}ll@{}}
\toprule\noalign{}
Field & Applications \\
\midrule\noalign{}
\endhead
\bottomrule\noalign{}
\endlastfoot
Industrial & Motor control, automation, PLCs \\
Medical & Patient monitoring, diagnostic equipment \\
Consumer & Washing machines, microwaves, toys \\
Automotive & Engine control, ABS, airbag systems \\
Communication & Mobile phones, modems, routers \\
Security & Access control, alarm systems \\
\end{longtable}
}

\end{solutionbox}
\begin{mnemonicbox}
``I-MACS:
Industrial-Medical-Automotive-Consumer-Security''

\end{mnemonicbox}
\subsection*{Question 5(b) [4 marks]}\label{q5b}

\textbf{Interface Push button Switch and LED with 8051 microcontroller.}

\begin{solutionbox}

\textbf{Circuit Diagram:}

\begin{verbatim}
     Vcc
      |
      R (10K)
      |
P1.0{-{-}+{-}{-}{-}{-}{-}{-} Push Button {-}{-}{-}{-}{-}{-} GND}
      
      Vcc
       |
       R (330Ω)
       |
P1.7{-{-}{-}+{-}{-}{-}{-}{-}||{-}{-}{-}{-}{-}{-} LED {-}{-}{-}{-}{-}{-} GND}
\end{verbatim}

\textbf{Program:}

\begin{verbatim}
AGAIN:  JB    P1.0, LED\_OFF  ; If P1.0=1 (not pressed), LED off
        SETB  P1.7           ; If P1.0=0 (pressed), LED on
        SJMP  AGAIN          ; Repeat
LED\_OFF:CLR   P1.7           ; Turn LED off
        SJMP  AGAIN          ; Repeat
\end{verbatim}

{\def\LTcaptype{none} % do not increment counter
\begin{longtable}[]{@{}lll@{}}
\toprule\noalign{}
Component & Connection & Purpose \\
\midrule\noalign{}
\endhead
\bottomrule\noalign{}
\endlastfoot
Push Button & P1.0 (input) & User input, active-low \\
Pull-up Resistor & 10K to Vcc & Prevents floating input \\
LED & P1.7 (output) & Visual indicator \\
Current-limiting Resistor & 330Ω & Protects LED \\
\end{longtable}
}

\end{solutionbox}
\begin{mnemonicbox}
``PLIC: Push-LED-Input-Control''

\end{mnemonicbox}
\subsection*{Question 5(c) [7 marks]}\label{q5c}

\textbf{Interface LCD with microcontroller and write a program to
display ``HELLO''.}

\begin{solutionbox}

\textbf{Circuit Diagram:}

\begin{verbatim}
           8051                  16x2 LCD
       +{-{-}{-}{-}{-}{-}{-}{-}+              +{-}{-}{-}{-}{-}{-}{-}{-}{-}+}
       |        |              |         |
       |   P2.0 |{-{-}{-}{-}{-}{-}{-}{-}{-}{-}{-}{-}{-}|D0       |}
       |   P2.1 |{-{-}{-}{-}{-}{-}{-}{-}{-}{-}{-}{-}{-}|D1       |}
       |   P2.2 |{-{-}{-}{-}{-}{-}{-}{-}{-}{-}{-}{-}{-}|D2       |}
       |   P2.3 |{-{-}{-}{-}{-}{-}{-}{-}{-}{-}{-}{-}{-}|D3       |}
       |   P2.4 |{-{-}{-}{-}{-}{-}{-}{-}{-}{-}{-}{-}{-}|D4       |}
       |   P2.5 |{-{-}{-}{-}{-}{-}{-}{-}{-}{-}{-}{-}{-}|D5       |}
       |   P2.6 |{-{-}{-}{-}{-}{-}{-}{-}{-}{-}{-}{-}{-}|D6       |}
       |   P2.7 |{-{-}{-}{-}{-}{-}{-}{-}{-}{-}{-}{-}{-}|D7       |}
       |        |              |         |
       |   P3.0 |{-{-}{-}{-}{-}{-}{-}{-}{-}{-}{-}{-}{-}|RS       |}
       |   P3.1 |{-{-}{-}{-}{-}{-}{-}{-}{-}{-}{-}{-}{-}|R/W      |}
       |   P3.2 |{-{-}{-}{-}{-}{-}{-}{-}{-}{-}{-}{-}{-}|E        |}
       |        |              |         |
       +{-{-}{-}{-}{-}{-}{-}{-}+              +{-}{-}{-}{-}{-}{-}{-}{-}{-}+}
                                  |   |
                                 Vcc GND
\end{verbatim}

\textbf{Program:}

\begin{verbatim}
ORG 0000H               ; Start address
    
; Initialize LCD
    MOV A, \#38H         ; 8{-bit, 2 lines, 5x7 dots}
    ACALL COMMAND       ; Send command
    MOV A, \#0EH         ; Display ON, cursor ON
    ACALL COMMAND       ; Send command
    MOV A, \#01H         ; Clear display
    ACALL COMMAND       ; Send command
    MOV A, \#06H         ; Increment cursor
    ACALL COMMAND       ; Send command
    MOV A, \#80H         ; First line, first position
    ACALL COMMAND       ; Send command
    
; Display "HELLO"
    MOV A, \#{H         ; Load H}
    ACALL DISPLAY       ; Display it
    MOV A, \#{E         ; Load E}
    ACALL DISPLAY       ; Display it
    MOV A, \#{L         ; Load L}
    ACALL DISPLAY       ; Display it
    MOV A, \#{L         ; Load L}
    ACALL DISPLAY       ; Display it
    MOV A, \#{O         ; Load O}
    ACALL DISPLAY       ; Display it
    
    SJMP $              ; Stay here
    
; Command subroutine
COMMAND:
    MOV P2, A           ; Put command on data bus
    CLR P3.0            ; RS=0 for command
    CLR P3.1            ; R/W=0 for write
    SETB P3.2           ; E=1
    ACALL DELAY         ; Wait
    CLR P3.2            ; E=0
    RET                 ; Return
    
; Display subroutine
DISPLAY:
    MOV P2, A           ; Put data on data bus
    SETB P3.0           ; RS=1 for data
    CLR P3.1            ; R/W=0 for write
    SETB P3.2           ; E=1
    ACALL DELAY         ; Wait
    CLR P3.2            ; E=0
    RET                 ; Return
    
; Delay subroutine
DELAY:
    MOV R7, \#50         ; Load counter
DELAY\_LOOP:
    DJNZ R7, DELAY\_LOOP ; Loop until R7=0
    RET                 ; Return
    
END                     ; End of program
\end{verbatim}

{\def\LTcaptype{none} % do not increment counter
\begin{longtable}[]{@{}lll@{}}
\toprule\noalign{}
Component & Connection & Purpose \\
\midrule\noalign{}
\endhead
\bottomrule\noalign{}
\endlastfoot
Data Pins & P2.0-P2.7 & Transfer data/commands \\
RS (Register Select) & P3.0 & Select command (0) or data (1) \\
R/W (Read/Write) & P3.1 & Select write (0) or read (1) \\
E (Enable) & P3.2 & Latch data on falling edge \\
\end{longtable}
}

\begin{itemize}
\tightlist
\item
  \textbf{Initialization}: Configure LCD for 8-bit, 2 lines, cursor
  options
\item
  \textbf{Data Transfer}: Commands sent with RS=0, characters with RS=1
\item
  \textbf{Characters}: ASCII values sent one by one to display text
\item
  \textbf{Timing}: Delay routine ensures proper signal timing
\end{itemize}

\end{solutionbox}
\begin{mnemonicbox}
``DICE: Data-Instruction-Control-Enable''

\end{mnemonicbox}
\subsection*{Question 5(a OR) [3
marks]}\label{question-5a-or-3-marks}

\textbf{Draw Interfacing of LM35 with 8051 microcontroller.}

\begin{solutionbox}

\textbf{Circuit Diagram:}

\begin{verbatim}
                8051
        +{-{-}{-}{-}{-}{-}{-}{-}{-}{-}{-}{-}{-}{-}{-}{-}{-}{-}{-}+}
        |                   |
+5V{-{-}{-}{-}{-}+                   |}
        |                   |
 LM35   |      ADC0804      |
+{-{-}{-}+   |    +{-}{-}{-}{-}{-}{-}{-}{-}+     |}
|   |   |    |        |     |
|OUT+{-{-}{-}+{-}{-}{-}|Vin     |     |}
|   |   |    |        |     |
+{-{-}{-}+   |    |        |     |}
|       |    |        |     |
GND{-{-}{-}{-}{-}+{-}{-}{-}{-}+GND     |     |}
        |    |        |     |
        |    |     CS +{{-}{-}{-}{-}+P1.0}
        |    |     RD +{{-}{-}{-}{-}+P1.1}
        |    |     WR +{{-}{-}{-}{-}+P1.2}
        |    |    INT +{-{-}{-}{-}+P1.3}
        |    |        |     |
        |    |  D0{-D7 +{-}{-}{-}+P0.0{-}P0.7}
        |    +{-{-}{-}{-}{-}{-}{-}{-}+     |}
        |                   |
        +{-{-}{-}{-}{-}{-}{-}{-}{-}{-}{-}{-}{-}{-}{-}{-}{-}{-}{-}+}
\end{verbatim}

{\def\LTcaptype{none} % do not increment counter
\begin{longtable}[]{@{}ll@{}}
\toprule\noalign{}
Component & Function \\
\midrule\noalign{}
\endhead
\bottomrule\noalign{}
\endlastfoot
LM35 & Temperature sensor (10mV/^\circC) \\
ADC0804 & Analog-to-Digital Converter \\
8051 & Microcontroller to read temperature \\
\end{longtable}
}

\textbf{Key Points:}

\begin{itemize}
\tightlist
\item
  LM35 produces analog voltage proportional to temperature
\item
  ADC0804 converts analog voltage to digital value
\item
  8051 controls ADC and reads temperature data
\item
  Resolution: 10mV/^\circC \rightarrow \textasciitilde0.2^\circC resolution with 8-bit ADC
\end{itemize}

\end{solutionbox}
\begin{mnemonicbox}
``TAC: Temperature-Analog-Convert''

\end{mnemonicbox}
\subsection*{Question 5(b OR) [4
marks]}\label{question-5b-or-4-marks}

\textbf{Interface Stepper motor with 8051 microcontroller.}

\begin{solutionbox}

\textbf{Circuit Diagram:}

\begin{verbatim}
          8051                      ULN2003
       +{-{-}{-}{-}{-}{-}{-}{-}+                 +{-}{-}{-}{-}{-}{-}{-}{-}+}
       |        |           +{-{-}{-}{-}|IN1  OUT1|{-}{-}{-}+}
       |   P1.0 |{-{-}{-}{-}{-}{-}{-}{-}{-}{-}{-}|{-}{-}{-}{-}|IN2  OUT2|{-}{-}{-}+}
       |   P1.1 |{-{-}{-}{-}{-}{-}{-}{-}{-}{-}{-}|{-}{-}{-}{-}|IN3  OUT3|{-}{-}{-}+{-}{-}{-}{-} 4{-}wire}
       |   P1.2 |{-{-}{-}{-}{-}{-}{-}{-}{-}{-}{-}|{-}{-}{-}{-}|IN4  OUT4|{-}{-}{-}+      Stepper}
       |   P1.3 |{-{-}{-}{-}{-}{-}{-}{-}{-}{-}{-}|     |         |           Motor}
       |        |           |     |         |
       +{-{-}{-}{-}{-}{-}{-}{-}+           |     +{-}{-}{-}{-}{-}{-}{-}{-}+}
                            |
                          +5V
\end{verbatim}

\textbf{Program:}

\begin{verbatim}
      ORG 0000H

; Stepper Motor Sequence for Clockwise Rotation
SEQ:  DB 00001000B  ; Step 1
      DB 00001100B  ; Step 2
      DB 00000100B  ; Step 3
      DB 00000110B  ; Step 4
      DB 00000010B  ; Step 5
      DB 00000011B  ; Step 6
      DB 00000001B  ; Step 7
      DB 00001001B  ; Step 8

MAIN: MOV R0, \#00H  ; Initialize sequence pointer
      
STEP: MOV A, R0     ; Get current sequence number
      ANL A, \#07H   ; Keep within 0{-7 range (8 steps)}
      MOV DPTR, \#SEQ ; Point to sequence table
      MOVC A, @A+DPTR ; Get sequence pattern
      MOV P1, A     ; Output to stepper motor
      
      ACALL DELAY   ; Wait between steps
      
      INC R0        ; Next sequence
      SJMP STEP     ; Repeat
      
DELAY: MOV R6, \#250 ; Delay loop
LOOP:  MOV R7, \#250
LOOP2: DJNZ R7, LOOP2
       DJNZ R6, LOOP
       RET
       
      END
\end{verbatim}

{\def\LTcaptype{none} % do not increment counter
\begin{longtable}[]{@{}ll@{}}
\toprule\noalign{}
Component & Purpose \\
\midrule\noalign{}
\endhead
\bottomrule\noalign{}
\endlastfoot
ULN2003 & Driver IC with Darlington arrays \\
Port pins & P1.0-P1.3 for 4 motor phases \\
Power supply & Separate supply for motor \\
\end{longtable}
}

\textbf{Key Points:}

\begin{itemize}
\tightlist
\item
  Stepper motor requires specific sequence of pulses for rotation
\item
  ULN2003 provides current amplification for motor coils
\item
  8-step sequence provides smoother rotation
\item
  Delay between steps controls rotation speed
\end{itemize}

\end{solutionbox}
\begin{mnemonicbox}
``PDCS: Port-Driver-Current-Sequence''

\end{mnemonicbox}
\subsection*{Question 5(c OR) [7
marks]}\label{question-5c-or-7-marks}

\textbf{Interface ADC0804 with 8051 microcontroller.}

\begin{solutionbox}

\textbf{Circuit Diagram:}

\begin{verbatim}
                     8051
                 +{-{-}{-}{-}{-}{-}{-}{-}{-}{-}+}
                 |          |
 Analog Input{-{-}{-}| ADC0804  |}
 0{-5V        |   |          |}
             v   |          |
        +{-{-}{-}+{-}{-}{-}+|          |     +{-}{-}{-}{-}{-}{-}{-}{-}{-}+}
        |        |          |     |         |
Vref/2{-|        |          |     |         |}
        |        |          |{{-}{-}{-}|P0.0{-}P0.7|}
CS{-{-}{-}{-}{-}|        |          |     |         |}
RD{-{-}{-}{-}{-}|        |          |     |         |}
WR{-{-}{-}{-}{-}|        |          |     |         |}
INTR{-{-}{-}|        |          |{-}{-}{-}{-}|P3.2     |}
        |        |          |     |         |
        +{-{-}{-}{-}{-}{-}{-}{-}+          |     |         |}
                 +{-{-}{-}{-}{-}{-}{-}{-}{-}{-}+     +{-}{-}{-}{-}{-}{-}{-}{-}{-}+}
\end{verbatim}

\textbf{Program:}

\begin{verbatim}
ORG 0000H

START:  CLR P1.0      ; CS = 0 (Chip Select active)
        CLR P1.1      ; RD = 0 
        CLR P1.2      ; WR = 0
        SETB P1.2     ; WR = 1 (Start conversion)
        
WAIT:   JB P1.3, WAIT ; Wait for conversion (INTR = 0)
        
        CLR P1.1      ; RD = 0 (Read data)
        MOV A, P0     ; Read converted data into A
        MOV 30H, A    ; Store result in RAM
        
        SETB P1.0     ; CS = 1 (Chip deselect)
        
PROCESS:
        ; Process the data (e.g., display, compare, etc.)
        ; ...
        
        ACALL DELAY   ; Wait before next conversion
        SJMP START    ; Repeat
        
DELAY:  MOV R7, \#200  ; Delay routine
DLOOP:  DJNZ R7, DLOOP
        RET
        
END
\end{verbatim}

{\def\LTcaptype{none} % do not increment counter
\begin{longtable}[]{@{}lll@{}}
\toprule\noalign{}
Connection & 8051 Pin & ADC0804 Pin \\
\midrule\noalign{}
\endhead
\bottomrule\noalign{}
\endlastfoot
Data Bus & P0.0-P0.7 & D0-D7 \\
CS & P1.0 & CS \\
RD & P1.1 & RD \\
WR & P1.2 & WR \\
INTR & P1.3 & INTR \\
\end{longtable}
}

\textbf{ADC0804 Features:}

\begin{itemize}
\tightlist
\item
  8-bit resolution (256 steps)
\item
  0-5V input range
\item
  Single-channel operation
\item
  \textasciitilde100μs conversion time
\item
  Interface protocol:

  \begin{enumerate}
  \tightlist
  \item
    Activate CS, pulse WR to start conversion
  \item
    Wait for INTR to go low (conversion complete)
  \item
    Activate RD to read data
  \item
    Deactivate CS when done
  \end{enumerate}
\end{itemize}

\end{solutionbox}
\begin{mnemonicbox}
``CRIW: Control-Read-Interrupt-Write''

\end{mnemonicbox}

\end{document}
