\documentclass[10pt,a4paper]{article}

% content/resources/templates/preamble.tex
\usepackage[margin=0.6in]{geometry}
\author{Milav Dabgar}
\usepackage{amsmath,amssymb,amsthm}
\usepackage{booktabs}
\usepackage{multirow}
\usepackage{xcolor}
\usepackage{tcolorbox}
\tcbuselibrary{breakable,skins}
\usepackage[colorlinks=true,linkcolor=blue]{hyperref}
\usepackage{titlesec}
\usepackage{enumitem}
\usepackage{tikz}
\usepackage{pgfplots}
\usepackage{circuitikz}
\usepackage[version=4]{mhchem}
\usepackage{longtable}
\usepackage{array}
\usepackage{float}
\usepackage{caption}
\usepackage{listings}

\lstset{
  basicstyle=\small\ttfamily,
  breaklines=true,
  breakatwhitespace=false,
  postbreak=\mbox{\textcolor{red}{$\hookrightarrow$}\space},
  float=false,
  numbers=left,
  numberstyle=\tiny\color{gray},
  numbersep=10pt,
  xleftmargin=2em,
  keywordstyle=\color{blue},
  commentstyle=\color{green!60!black},
  stringstyle=\color{purple},
  backgroundcolor=\color{gray!5},
  showstringspaces=false,
  tabsize=2,
  captionpos=b,
  keepspaces=true,
  columns=flexible
}

\pgfplotsset{compat=1.18}
\usetikzlibrary{shapes,arrows,positioning,calc,patterns,decorations.pathmorphing,decorations.markings,arrows.meta}

% Color scheme
\definecolor{headcolor}{RGB}{0,102,204}
\definecolor{keycolor}{RGB}{220,20,60}
\definecolor{solutioncolor}{RGB}{34,139,34}
\definecolor{mnemoniccolor}{RGB}{148,0,211}
\definecolor{codecolor}{RGB}{0,0,100}

% Spacing
\setlength{\parskip}{3pt}
\setlist[itemize]{nosep}
\setlist[enumerate]{nosep}

% Title formatting
\titleformat{\section}{\Large\bfseries\color{headcolor}}{\thesection}{1em}{}
\titleformat{\subsection}{\large\bfseries\color{headcolor}}{\thesubsection}{1em}{}

% Pandoc tightlist compatibility
\providecommand{\tightlist}{%
  \setlength{\itemsep}{0pt}\setlength{\parskip}{0pt}}

% Pandoc longtable compatibility
\newcounter{none}
\def\thenone{}


% content/resources/templates/gujarati-boxes.tex
\usepackage{fontspec}
\usepackage{polyglossia}

% Set Gujarati as main language (document is primarily in Gujarati)
% Note: gloss-gujarati.ldf doesn't exist in polyglossia, but it will use hyphenation patterns
\setdefaultlanguage{gujarati}
\setotherlanguage{english}

% Configure Gujarati font properly
% Use Language=Default to prevent polyglossia from trying to add language-specific features
% that don't exist for Gujarati, which causes "empty feature" warnings
\newfontfamily\gujaratifont[Script=Gujarati,AutoFakeBold=2.5,AutoFakeSlant=0.3]{Noto Sans Gujarati}
\setmainfont[Script=Gujarati,AutoFakeBold=2.5,AutoFakeSlant=0.3]{Noto Sans Gujarati}
% Use Noto Sans Gujarati for monospace to support Gujarati in text
\setmonofont[Scale=0.9]{Noto Sans Gujarati}

% Configure English to use the same font
\newfontfamily\englishfont[Script=Gujarati,AutoFakeBold=2.5,AutoFakeSlant=0.3]{Noto Sans Gujarati}

% Translations for polyglossia
\gappto\captionsgujarati{
  \renewcommand{\tablename}{કોષ્ટક}
  \renewcommand{\figurename}{આકૃતિ}
}

% Helper for TikZ nodes to ensure Gujarati font
\newcommand{\gu}[1]{{\gujaratifont #1}}

% Custom environments
\newtcolorbox{solutionbox}{
    breakable,
    enhanced,
    colback=solutioncolor!5!white,
    colframe=solutioncolor!75!black,
    fonttitle=\bfseries,
    title=જવાબ
}

\newtcolorbox{solutionboxnobreak}{
 colback=solutioncolor!5!white,
 colframe=solutioncolor!75!black,
 fonttitle=\bfseries,
 title=જવાબ
}

\newtcolorbox{keyformula}{
 breakable,
 enhanced,
 colback=keycolor!5!white,
 colframe=keycolor!75!black,
 fonttitle=\bfseries,
 title=રાસાયણિક સમીકરણ/સૂત્ર
}

\newtcolorbox{mnemonicbox}{
 breakable,
 enhanced,
 colback=mnemoniccolor!5!white,
 colframe=mnemoniccolor!75!black,
 fonttitle=\bfseries,
 title=મેમરી ટ્રીક
}


\begin{document}

\begin{center}
{\Huge\bfseries\color{headcolor} Subject Name (Gujarati)}\\[5pt]
{\LARGE 4341101 -- Summer 2025}\\[3pt]
{\large Semester 1 Study Material}\\[3pt]
{\normalsize\textit{Detailed Solutions and Explanations}}
\end{center}

\vspace{10pt}

\subsection*{પ્રશ્ન 1(અ) [3
ગુણ]}\label{uxaaauxab0uxab6uxaa8-1uxa85-3-uxa97uxaa3}

\textbf{માઇક્રોપ્રોસેસરને વ્યાખ્યાયિત કરો અને તેનો બ્લોક ડાયાગ્રામ દોરો.}

\begin{solutionbox}
\textbf{માઇક્રોપ્રોસેસર} એક પ્રોગ્રામેબલ ડિજિટલ ઉપકરણ છે જે
સંગ્રહિત સૂચનાઓ અનુસાર ડેટા પર અંકગણિત અને તાર્કિક કામગીરી કરે છે.

\textbf{બ્લોક ડાયાગ્રામ:}

\begin{center}
\textbf{Mermaid Diagram (Code)}
\begin{verbatim}
{Shaded}
{Highlighting}[]
graph LR
    A[Input Device] {-{-}{} B[CPU]}
    B {-{-}{} C[Output Device]}
    B {{-}{-}{} D[Memory Unit]}
    B {-{-}{} E[Control Unit]}
    B {-{-}{} F[ALU]}
    E {-{-}{} G[Control Signals]}
    F {-{-}{} H[Arithmetic \& Logic Operations]}
{Highlighting}
{Shaded}
\end{verbatim}
\end{center}

\begin{itemize}
\tightlist
\item
  \textbf{CPU}: \textbf{સેન્ટ્રલ પ્રોસેસિંગ યુનિટ} બધી કામગીરી કરે છે
\item
  \textbf{મેમરી}: \textbf{પ્રોગ્રામ અને ડેટા} સંગ્રહ કરે છે
\item
  \textbf{કંટ્રોલ યુનિટ}: \textbf{સૂચના અમલીકરણ} ક્રમને નિયંત્રિત કરે છે
\end{itemize}

\textbf{યાદગાર વાક્ય}: ``મારું કમ્પ્યુટર પ્રોગ્રામ સમજે''
(મેમરી-CPU-પ્રોગ્રામ-સૂચનાઓ)

\end{solutionbox}
\subsection*{પ્રશ્ન 1(બ) [4
ગુણ]}\label{uxaaauxab0uxab6uxaa8-1uxaac-4-uxa97uxaa3}

\textbf{યોગ્ય instruction ના ઉદાહરણ સાથે ઓપરેન્ડ અને ઓપકોડ સમજાવો.}

\begin{solutionbox}
\textbf{ઓપકોડ} કરવાની કામગીરી સ્પષ્ટ કરે છે. \textbf{ઓપરેન્ડ}
કામગીરી થવાનો ડેટા સ્પષ્ટ કરે છે.

\textbf{ઉદાહરણ કોષ્ટક:}

{\def\LTcaptype{none} % do not increment counter
\begin{longtable}[]{@{}llll@{}}
\toprule\noalign{}
સૂચના & ઓપકોડ & ઓપરેન્ડ & કાર્ય \\
\midrule\noalign{}
\endhead
\bottomrule\noalign{}
\endlastfoot
MOV A,B & MOV & A,B & B ને A માં ખસેડો \\
ADD A,\#05H & ADD & A,\#05H & A માં 05H ઉમેરો \\
\end{longtable}
}

\begin{itemize}
\tightlist
\item
  \textbf{ઓપકોડ}: \textbf{ઓપરેશન કોડ} (MOV, ADD, SUB)
\item
  \textbf{ઓપરેન્ડ}: \textbf{ડેટા કે એડ્રેસ} (A, B, \#05H)
\item
  \textbf{ફોર્મેટ}: \textbf{ઓપકોડ + ઓપરેન્ડ = સંપૂર્ણ સૂચના}
\end{itemize}

\textbf{યાદગાર વાક્ય}: ``ઓપરેશન ઓન ડેટા'' (ઓપકોડ-ઓપરેન્ડ-ડેટા)

\end{solutionbox}
\subsection*{પ્રશ્ન 1(ક) [7
ગુણ]}\label{uxaaauxab0uxab6uxaa8-1uxa95-7-uxa97uxaa3}

\textbf{માઇક્રોપ્રોસેસર અને માઇક્રોકંટ્રોલરની સરખામણી કરો.}

\begin{solutionbox}

{\def\LTcaptype{none} % do not increment counter
\begin{longtable}[]{@{}lll@{}}
\toprule\noalign{}
પેરામીટર & માઇક્રોપ્રોસેસર & માઇક્રોકંટ્રોલર \\
\midrule\noalign{}
\endhead
\bottomrule\noalign{}
\endlastfoot
\textbf{વ્યાખ્યા} & માત્ર CPU & CPU + મેમરી + I/O \\
\textbf{મેમરી} & બાહ્ય RAM/ROM & આંતરિક RAM/ROM \\
\textbf{I/O પોર્ટ્સ} & બાહ્ય ઇન્ટરફેસ & બિલ્ટ-ઇન પોર્ટ્સ \\
\textbf{કિંમત} & વધુ સિસ્ટમ કિંમત & ઓછી સિસ્ટમ કિંમત \\
\textbf{પાવર} & વધુ વપરાશ & ઓછો વપરાશ \\
\textbf{ઝડપ} & ઝડપી પ્રક્રિયા & મધ્યમ ઝડપ \\
\textbf{ઉપયોગ} & કમ્પ્યુટર, લેપટોપ & વોશિંગ મશીન, માઇક્રોવેવ \\
\end{longtable}
}

\begin{itemize}
\tightlist
\item
  \textbf{માઇક્રોપ્રોસેસર}: \textbf{સામાન્ય હેતુ} કમ્પ્યુટિંગ
\item
  \textbf{માઇક્રોકંટ્રોલર}: \textbf{વિશિષ્ટ એમ્બેડેડ} એપ્લિકેશન્સ
\item
  \textbf{ઇન્ટિગ્રેશન}: \textbf{માઇક્રોકંટ્રોલર} માં બધું એક ચિપ પર
\end{itemize}

\textbf{યાદગાર વાક્ય}: ``માઇક્રો મીન્સ મોર ઇન્ટિગ્રેશન''
(માઇક્રોકંટ્રોલર-મેમરી-મોર-ઇન્ટેગ્રેશન)

\end{solutionbox}
\subsection*{પ્રશ્ન 1(ક અથવા) [7
ગુણ]}\label{uxaaauxab0uxab6uxaa8-1uxa95-uxa85uxaa5uxab5-7-uxa97uxaa3}

\textbf{RISC અને CISC ની સરખામણી કરો.}

\begin{solutionbox}

{\def\LTcaptype{none} % do not increment counter
\begin{longtable}[]{@{}lll@{}}
\toprule\noalign{}
પેરામીટર & RISC & CISC \\
\midrule\noalign{}
\endhead
\bottomrule\noalign{}
\endlastfoot
\textbf{સૂચનાઓ} & સરળ, ઓછી & જટિલ, વધુ \\
\textbf{સૂચના સાઇઝ} & નિશ્ચિત લંબાઇ & વેરિયેબલ લંબાઇ \\
\textbf{એક્ઝિક્યુશન ટાઇમ} & સિંગલ સાઇકલ & બહુવિધ સાઇકલ \\
\textbf{મેમરી એક્સેસ} & ફક્ત લોડ/સ્ટોર & કોઇપણ સૂચના \\
\textbf{રજિસ્ટર્સ} & વધુ રજિસ્ટર્સ & ઓછા રજિસ્ટર્સ \\
\textbf{પાઇપલાઇન} & કાર્યક્ષમ પાઇપલાઇનિંગ & જટિલ પાઇપલાઇનિંગ \\
\textbf{ઉદાહરણો} & ARM, MIPS & x86, 8085 \\
\end{longtable}
}

\begin{itemize}
\tightlist
\item
  \textbf{RISC}: \textbf{રિડ્યુસ્ડ ઇન્સ્ટ્રક્શન સેટ કમ્પ્યુટર}
\item
  \textbf{CISC}: \textbf{કોમ્પ્લેક્સ ઇન્સ્ટ્રક્શન સેટ કમ્પ્યુટર}
\item
  \textbf{પર્ફોર્મન્સ}: \textbf{RISC ઝડપી, CISC વધુ લવચીક}
\end{itemize}

\textbf{યાદગાર વાક્ય}: ``રિડ્યુસ્ડ ઇન્સ્ટ્રક્શન્સ સ્પીડ કમ્પ્યુટિંગ''
(RISC-ઇન્સ્ટ્રક્શન્સ-સ્પીડ-કમ્પ્યુટિંગ)

\end{solutionbox}
\subsection*{પ્રશ્ન 2(અ) [3
ગુણ]}\label{uxaaauxab0uxab6uxaa8-2uxa85-3-uxa97uxaa3}

\textbf{8085 માઇક્રોપ્રોસેસરનું બસ ઓર્ગેનાઇઝેશન સમજાવો.}

\begin{solutionbox}
8085 માં બાહ્ય ઉપકરણો સાથે સંચાર માટે \textbf{ત્રણ પ્રકારની} બસ
છે.

\textbf{બસ ઓર્ગેનાઇઝેશન કોષ્ટક:}

{\def\LTcaptype{none} % do not increment counter
\begin{longtable}[]{@{}lll@{}}
\toprule\noalign{}
બસ પ્રકાર & લાઇન્સ & કાર્ય \\
\midrule\noalign{}
\endhead
\bottomrule\noalign{}
\endlastfoot
\textbf{એડ્રેસ બસ} & 16 લાઇન્સ (A0-A15) & મેમરી એડ્રેસિંગ \\
\textbf{ડેટા બસ} & 8 લાઇન્સ (D0-D7) & ડેટા ટ્રાન્સફર \\
\textbf{કંટ્રોલ બસ} & બહુવિધ લાઇન્સ & કંટ્રોલ સિગ્નલ્સ \\
\end{longtable}
}

\begin{itemize}
\tightlist
\item
  \textbf{એડ્રેસ બસ}: \textbf{યુનિડાયરેક્શનલ}, 64KB મેમરી એડ્રેસિંગ
\item
  \textbf{ડેટા બસ}: \textbf{બાઇડાયરેક્શનલ}, 8-બિટ ડેટા ટ્રાન્સફર
\item
  \textbf{કંટ્રોલ બસ}: \textbf{રીડ, રાઇટ, IO/M સિગ્નલ્સ}
\end{itemize}

\textbf{યાદગાર વાક્ય}: ``એડ્રેસ ડેટા કંટ્રોલ'' (ADC)

\end{solutionbox}
\subsection*{પ્રશ્ન 2(બ) [4
ગુણ]}\label{uxaaauxab0uxab6uxaa8-2uxaac-4-uxa97uxaa3}

\textbf{ડાયાગ્રામ સાથે ALE સિગ્નલનું કાર્ય સમજાવો.}

\begin{solutionbox}
\textbf{ALE (એડ્રેસ લેચ એનેબલ)} મલ્ટિપ્લેક્સ્ડ બસ પર એડ્રેસ અને ડેટાને
અલગ કરે છે.

\textbf{ALE ટાઇમિંગ ડાયાગ્રામ:}

\begin{verbatim}
    ALE  \_\_\_\_      \_\_\_\_
        |    |\_\_\_\_|    |\_\_\_\_
        
  AD7{-0 ADDR |   DATA  | ADDR}
        \_\_\_\_\_|\_\_\_\_\_\_\_\_\_|\_\_\_\_\_
\end{verbatim}

\begin{itemize}
\tightlist
\item
  \textbf{હાઇ ALE}: \textbf{એડ્રેસ} AD0-AD7 પર ઉપલબ્ધ
\item
  \textbf{લો ALE}: \textbf{ડેટા} AD0-AD7 પર ઉપલબ્ધ
\item
  \textbf{કાર્ય}: \textbf{લોઅર એડ્રેસ બાઇટ} લેચ કરે છે
\item
  \textbf{ફ્રીક્વન્સી}: \textbf{ALE = Clock frequency \div 2}
\end{itemize}

\textbf{યાદગાર વાક્ય}: ``એડ્રેસ લેચ એનેબલ'' (ALE)

\end{solutionbox}
\subsection*{પ્રશ્ન 2(ક) [7
ગુણ]}\label{uxaaauxab0uxab6uxaa8-2uxa95-7-uxa97uxaa3}

\textbf{આકૃતિની મદદથી 8085 માઇક્રોપ્રોસેસરના આર્કિટેક્ચરનું વર્ણન કરો.}

\begin{solutionbox}

\begin{center}
\textbf{Mermaid Diagram (Code)}
\begin{verbatim}
{Shaded}
{Highlighting}[]
graph TD
    A[Accumulator A] {-{-}{} B[ALU]}
    C[Temp Register] {-{-}{} B}
    B {-{-}{} D[Flag Register]}
    E[B,C,D,E,H,L Registers] {-{-}{} F[Address Buffer]}
    G[Program Counter] {-{-}{} F}
    H[Stack Pointer] {-{-}{} F}
    F {-{-}{} I[Address Bus A0{-}A15]}
    J[Data/Address Buffer] {-{-}{} K[Data Bus AD0{-}AD7]}
    L[Instruction Register] {-{-}{} M[Instruction Decoder]}
    M {-{-}{} N[Control Unit]}
    N {-{-}{} O[Control Signals]}
{Highlighting}
{Shaded}
\end{verbatim}
\end{center}

\textbf{મુખ્ય ઘટકો:}

\begin{itemize}
\tightlist
\item
  \textbf{ALU}: \textbf{અંકગણિત અને તાર્કિક} કામગીરી કરે છે
\item
  \textbf{રજિસ્ટર્સ}: \textbf{અસ્થાયી ડેટા} સંગ્રહ કરે છે (A, B, C, D, E, H, L)
\item
  \textbf{પ્રોગ્રામ કાઉન્ટર}: \textbf{આગળની સૂચના} તરફ નિર્દેશ કરે છે
\item
  \textbf{સ્ટેક પોઇન્ટર}: \textbf{સ્ટેક ટોપ} તરફ નિર્દેશ કરે છે
\item
  \textbf{કંટ્રોલ યુનિટ}: \textbf{કંટ્રોલ સિગ્નલ્સ} જનરેટ કરે છે
\end{itemize}

\textbf{યાદગાર વાક્ય}: ``ઓલ રજિસ્ટર્સ પ્રોગ્રામ સ્ટેક કંટ્રોલ'' (A-R-P-S-C)

\end{solutionbox}
\subsection*{પ્રશ્ન 2(અ અથવા) [3
ગુણ]}\label{uxaaauxab0uxab6uxaa8-2uxa85-uxa85uxaa5uxab5-3-uxa97uxaa3}

\textbf{8085 માઇક્રોપ્રોસેસરનો ફ્લેગ રજિસ્ટર દોરો અને તેને સમજાવો.}

\begin{solutionbox}

\textbf{ફ્લેગ રજિસ્ટર ફોર્મેટ:}

\begin{verbatim}
 D7  D6  D5  D4  D3  D2  D1  D0
+{-{-}{-}+{-}{-}{-}+{-}{-}{-}+{-}{-}{-}+{-}{-}{-}+{-}{-}{-}+{-}{-}{-}+{-}{-}{-}+}
| S | Z | 0 |AC | 0 | P | 1 | C |
+{-{-}{-}+{-}{-}{-}+{-}{-}{-}+{-}{-}{-}+{-}{-}{-}+{-}{-}{-}+{-}{-}{-}+{-}{-}{-}+}
\end{verbatim}

\textbf{ફ્લેગ કાર્યો:}

\begin{itemize}
\tightlist
\item
  \textbf{S (સાઇન)}: \textbf{પરિણામ નેગેટિવ} હોય તો સેટ
\item
  \textbf{Z (ઝીરો)}: \textbf{પરિણામ શૂન્ય} હોય તો સેટ
\item
  \textbf{AC (ઓક્સિલિયરી કેરી)}: \textbf{BCD ઓપરેશન્સ} માટે સેટ
\item
  \textbf{P (પેરિટી)}: \textbf{ઇવન પેરિટી} માટે સેટ
\item
  \textbf{C (કેરી)}: \textbf{કેરી/બોરો} જ્યારે થાય તો સેટ
\end{itemize}

\textbf{યાદગાર વાક્ય}: ``સમ ઝીરો ઓક્સિલિયરી પેરિટી કેરી'' (SZAPC)

\end{solutionbox}
\subsection*{પ્રશ્ન 2(બ અથવા) [4
ગુણ]}\label{uxaaauxab0uxab6uxaa8-2uxaac-uxa85uxaa5uxab5-4-uxa97uxaa3}

\textbf{8085 માઇક્રોપ્રોસેસર માટે એડ્રેસ અને ડેટા બસોનું ડીમલ્ટિપ્લેક્સિંગ સમજાવો.}

\begin{solutionbox}
\textbf{ડીમલ્ટિપ્લેક્સિંગ} AD0-AD7 લાઇન્સમાંથી એડ્રેસ અને ડેટા
સિગ્નલ્સને અલગ કરે છે.

\textbf{ડીમલ્ટિપ્લેક્સિંગ સર્કિટ:}

\begin{verbatim}
AD0{-AD7 {-}{-}{-}{-}+{-}{-}{-}{-} D{-}Latch {-}{-}{-}{-} A0{-}A7 (Address)}
            |         \^{}
            |         |
            |       ALE
            |
            +{-{-}{-}{-} Data Buffer {-}{-}{-}{-} D0{-}D7 (Data)}
\end{verbatim}

\begin{itemize}
\tightlist
\item
  \textbf{ALE હાઇ}: \textbf{એડ્રેસ} બાહ્ય લેચમાં લેચ થાય છે
\item
  \textbf{ALE લો}: \textbf{ડેટા} બફર દ્વારા વહે છે
\item
  \textbf{74LS373}: \textbf{સામાન્ય લેચ IC} વપરાય છે
\item
  \textbf{ફાયદો}: \textbf{અલગ એડ્રેસ અને ડેટા બસ}
\end{itemize}

\textbf{યાદગાર વાક્ય}: ``એડ્રેસ લેચ એક્સ્ટર્નલ ડિમલ્ટિપ્લેક્સ'' (ALED)

\end{solutionbox}
\subsection*{પ્રશ્ન 2(ક અથવા) [7
ગુણ]}\label{uxaaauxab0uxab6uxaa8-2uxa95-uxa85uxaa5uxab5-7-uxa97uxaa3}

\textbf{આકૃતિની મદદથી 8085 માઇક્રોપ્રોસેસરના પિન ડાયાગ્રામનું વર્ણન કરો.}

\begin{solutionbox}

\begin{verbatim}
         8085 Microprocessor
        +{-{-}{-}{-}{-}{-}{-}{-}{-}{-}{-}{-}{-}{-}{-}{-}{-}{-}{-}+}
   X1 {-{-}| 1              40 |{-}{-} VCC}
   X2 {-{-}| 2              39 |{-}{-} HOLD}
RESET {-{-}| 3              38 |{-}{-} HLDA}
  SOD {-{-}| 4              37 |{-}{-} CLK}
  SID {-{-}| 5              36 |{-}{-} RESET}
 TRAP {-{-}| 6              35 |{-}{-} READY}
RST7.5{-{-}| 7              34 |{-}{-} IO/M}
RST6.5{-{-}| 8              33 |{-}{-} S1}
RST5.5{-{-}| 9              32 |{-}{-} RD}
 INTR {-{-}| 10             31 |{-}{-} WR}
 INTA {-{-}| 11             30 |{-}{-} ALE}
  AD0 {-{-}| 12             29 |{-}{-} S0}
  AD1 {-{-}| 13             28 |{-}{-} A15}
  AD2 {-{-}| 14             27 |{-}{-} A14}
  AD3 {-{-}| 15             26 |{-}{-} A13}
  AD4 {-{-}| 16             25 |{-}{-} A12}
  AD5 {-{-}| 17             24 |{-}{-} A11}
  AD6 {-{-}| 18             23 |{-}{-} A10}
  AD7 {-{-}| 19             22 |{-}{-} A9}
  VSS {-{-}| 20             21 |{-}{-} A8}
        +{-{-}{-}{-}{-}{-}{-}{-}{-}{-}{-}{-}{-}{-}{-}{-}{-}{-}{-}+}
\end{verbatim}

\textbf{પિન કેટેગરીઝ:}

\begin{itemize}
\tightlist
\item
  \textbf{પાવર}: \textbf{VCC, VSS}
\item
  \textbf{ક્લોક}: \textbf{X1, X2, CLK}
\item
  \textbf{એડ્રેસ/ડેટા}: \textbf{AD0-AD7, A8-A15}
\item
  \textbf{કંટ્રોલ}: \textbf{ALE, RD, WR, IO/M}
\item
  \textbf{ઇન્ટરપ્ટ}: \textbf{INTR, INTA, RST7.5, RST6.5, RST5.5, TRAP}
\end{itemize}

\textbf{યાદગાર વાક્ય}: ``પાવર ક્લોક એડ્રેસ કંટ્રોલ ઇન્ટરપ્ટ'' (PCACI)

\end{solutionbox}
\subsection*{પ્રશ્ન 3(અ) [3
ગુણ]}\label{uxaaauxab0uxab6uxaa8-3uxa85-3-uxa97uxaa3}

\textbf{DPTR અને PC નું કાર્ય લખો.}

\begin{solutionbox}

\textbf{કાર્યો કોષ્ટક:}

{\def\LTcaptype{none} % do not increment counter
\begin{longtable}[]{@{}lll@{}}
\toprule\noalign{}
રજિસ્ટર & કાર્ય & સાઇઝ \\
\midrule\noalign{}
\endhead
\bottomrule\noalign{}
\endlastfoot
\textbf{DPTR} & ડેટા પોઇન્ટર & 16-બિટ \\
\textbf{PC} & પ્રોગ્રામ કાઉન્ટર & 16-બિટ \\
\end{longtable}
}

\textbf{DPTR કાર્યો:}

\begin{itemize}
\tightlist
\item
  \textbf{બાહ્ય મેમરી}: \textbf{બાહ્ય ડેટા મેમરી} એક્સેસ કરે છે
\item
  \textbf{એડ્રેસિંગ}: \textbf{MOVX સૂચનાઓ} માટે 16-બિટ એડ્રેસ
\end{itemize}

\textbf{PC કાર્યો:}

\begin{itemize}
\tightlist
\item
  \textbf{ઇન્સ્ટ્રક્શન પોઇન્ટર}: \textbf{આગળની સૂચના} તરફ નિર્દેશ કરે છે
\item
  \textbf{ઓટો ઇન્ક્રિમેન્ટ}: \textbf{દરેક સૂચના ફેચ} પછી વધે છે
\end{itemize}

\textbf{યાદગાર વાક્ય}: ``ડેટા પ્રોગ્રામ કાઉન્ટર'' (DPC)

\end{solutionbox}
\subsection*{પ્રશ્ન 3(બ) [4
ગુણ]}\label{uxaaauxab0uxab6uxaa8-3uxaac-4-uxa97uxaa3}

\textbf{8051 નું PCON SFR દોરો અને દરેક બિટનું કાર્ય સમજાવો.}

\begin{solutionbox}

\textbf{PCON રજિસ્ટર (87H):}

\begin{verbatim}
 D7  D6  D5  D4  D3  D2  D1  D0
+{-{-}{-}+{-}{-}{-}+{-}{-}{-}+{-}{-}{-}+{-}{-}{-}+{-}{-}{-}+{-}{-}{-}+{-}{-}{-}+}
|SMOD| {- | {-} | {-} |GF1|GF0|PD |IDL|}
+{-{-}{-}+{-}{-}{-}+{-}{-}{-}+{-}{-}{-}+{-}{-}{-}+{-}{-}{-}+{-}{-}{-}+{-}{-}{-}+}
\end{verbatim}

\textbf{બિટ કાર્યો:}

\begin{itemize}
\tightlist
\item
  \textbf{SMOD}: \textbf{સીરિયલ પોર્ટ બોડ રેટ} ડબલર
\item
  \textbf{GF1, GF0}: \textbf{સામાન્ય હેતુ} ફ્લેગ્સ
\item
  \textbf{PD}: \textbf{પાવર ડાઉન મોડ} કંટ્રોલ
\item
  \textbf{IDL}: \textbf{આઇડલ મોડ} કંટ્રોલ
\end{itemize}

\textbf{પાવર મેનેજમેન્ટ:}

\begin{itemize}
\tightlist
\item
  \textbf{IDL = 1}: \textbf{CPU બંધ, પેરિફેરલ્સ} ચાલે છે
\item
  \textbf{PD = 1}: \textbf{સંપૂર્ણ પાવર ડાઉન}
\end{itemize}

\textbf{યાદગાર વાક્ય}: ``સીરિયલ જનરલ પાવર આઇડલ'' (SGPI)

\end{solutionbox}
\subsection*{પ્રશ્ન 3(ક) [7
ગુણ]}\label{uxaaauxab0uxab6uxaa8-3uxa95-7-uxa97uxaa3}

\textbf{આકૃતિની મદદથી 8051 માઇક્રોકંટ્રોલરનું આર્કિટેક્ચર સમજાવો.}

\begin{solutionbox}

\begin{center}
\textbf{Mermaid Diagram (Code)}
\begin{verbatim}
{Shaded}
{Highlighting}[]
graph TD
    A[CPU Core] {-{-}{} B[ALU]}
    A {-{-}{} C[Accumulator A]}
    A {-{-}{} D[B Register]}
    A {-{-}{} E[PSW]}
    F[Program Memory ROM] {-{-}{} G[Program Counter PC]}
    H[Data Memory RAM] {-{-}{} I[Data Pointer DPTR]}
    J[Timer 0] {-{-}{} K[Timer Control]}
    L[Timer 1] {-{-}{} K}
    M[Serial Port] {-{-}{} N[Serial Control]}
    O[Port 0] {-{-}{} P[I/O Control]}
    Q[Port 1] {-{-}{} P}
    R[Port 2] {-{-}{} P}
    S[Port 3] {-{-}{} P}
    T[Interrupt System] {-{-}{} U[Interrupt Control]}
{Highlighting}
{Shaded}
\end{verbatim}
\end{center}

\textbf{મુખ્ય બ્લોક્સ:}

\begin{itemize}
\tightlist
\item
  \textbf{CPU}: \textbf{ALU સાથે 8-બિટ} પ્રોસેસર
\item
  \textbf{મેમરી}: \textbf{4KB ROM, 128B RAM}
\item
  \textbf{ટાઇમર્સ}: \textbf{બે 16-બિટ} ટાઇમર્સ
\item
  \textbf{સીરિયલ પોર્ટ}: \textbf{ફુલ ડુપ્લેક્સ UART}
\item
  \textbf{I/O પોર્ટ્સ}: \textbf{ચાર 8-બિટ} પોર્ટ્સ
\item
  \textbf{ઇન્ટરપ્ટ્સ}: \textbf{5 ઇન્ટરપ્ટ} સોર્સ
\end{itemize}

\textbf{યાદગાર વાક્ય}: ``CPU મેમરી ટાઇમર સીરિયલ IO ઇન્ટરપ્ટ'' (CMTSII)

\end{solutionbox}
\subsection*{પ્રશ્ન 3(અ અથવા) [3
ગુણ]}\label{uxaaauxab0uxab6uxaa8-3uxa85-uxa85uxaa5uxab5-3-uxa97uxaa3}

\textbf{8051 માઇક્રોકંટ્રોલરના સામાન્ય ફીચર્સની યાદી બનાવો.}

\begin{solutionbox}

\textbf{સામાન્ય ફીચર્સ:}

\begin{itemize}
\tightlist
\item
  \textbf{CPU}: \textbf{8-બિટ માઇક્રોકંટ્રોલર}
\item
  \textbf{મેમરી}: \textbf{4KB ROM, 128B RAM}
\item
  \textbf{I/O પોર્ટ્સ}: \textbf{32 I/O લાઇન્સ} (4 પોર્ટ્સ)
\item
  \textbf{ટાઇમર્સ}: \textbf{બે 16-બિટ} ટાઇમર્સ/કાઉન્ટર્સ
\item
  \textbf{સીરિયલ પોર્ટ}: \textbf{ફુલ ડુપ્લેક્સ UART}
\item
  \textbf{ઇન્ટરપ્ટ્સ}: \textbf{5 ઇન્ટરપ્ટ} સોર્સ
\item
  \textbf{ક્લોક}: \textbf{12MHz મહત્તમ} ફ્રીક્વન્સી
\end{itemize}

\textbf{યાદગાર વાક્ય}: ``CPU મેમરી IO ટાઇમર સીરિયલ ઇન્ટરપ્ટ ક્લોક'' (CMITSIC)

\end{solutionbox}
\subsection*{પ્રશ્ન 3(બ અથવા) [4
ગુણ]}\label{uxaaauxab0uxab6uxaa8-3uxaac-uxa85uxaa5uxab5-4-uxa97uxaa3}

\textbf{8051 નું IP SFR દોરો અને દરેક બિટનું કાર્ય સમજાવો.}

\begin{solutionbox}

\textbf{IP રજિસ્ટર (B8H):}

\begin{verbatim}
 D7  D6  D5  D4  D3  D2  D1  D0
+{-{-}{-}+{-}{-}{-}+{-}{-}{-}+{-}{-}{-}+{-}{-}{-}+{-}{-}{-}+{-}{-}{-}+{-}{-}{-}+}
| {- | {-} | {-} |PS |PT1|PX1|PT0|PX0|}
+{-{-}{-}+{-}{-}{-}+{-}{-}{-}+{-}{-}{-}+{-}{-}{-}+{-}{-}{-}+{-}{-}{-}+{-}{-}{-}+}
\end{verbatim}

\textbf{બિટ કાર્યો:}

\begin{itemize}
\tightlist
\item
  \textbf{PS}: \textbf{સીરિયલ પોર્ટ ઇન્ટરપ્ટ} પ્રાઇઓરિટી
\item
  \textbf{PT1}: \textbf{ટાઇમર 1 ઇન્ટરપ્ટ} પ્રાઇઓરિટી
\item
  \textbf{PX1}: \textbf{એક્સ્ટર્નલ ઇન્ટરપ્ટ 1} પ્રાઇઓરિટી
\item
  \textbf{PT0}: \textbf{ટાઇમર 0 ઇન્ટરપ્ટ} પ્રાઇઓરિટી
\item
  \textbf{PX0}: \textbf{એક્સ્ટર્નલ ઇન્ટરપ્ટ 0} પ્રાઇઓરિટી
\end{itemize}

\textbf{પ્રાઇઓરિટી લેવલ્સ:}

\begin{itemize}
\tightlist
\item
  \textbf{1}: \textbf{હાઇ પ્રાઇઓરિટી}
\item
  \textbf{0}: \textbf{લો પ્રાઇઓરિટી}
\end{itemize}

\textbf{યાદગાર વાક્ય}: ``પ્રાઇઓરિટી સીરિયલ ટાઇમર એક્સ્ટર્નલ'' (PSTE)

\end{solutionbox}
\subsection*{પ્રશ્ન 3(ક અથવા) [7
ગુણ]}\label{uxaaauxab0uxab6uxaa8-3uxa95-uxa85uxaa5uxab5-7-uxa97uxaa3}

\textbf{આકૃતિની મદદથી 8051 માઇક્રોકંટ્રોલરનો પિન ડાયાગ્રામ સમજાવો.}

\begin{solutionbox}

\begin{verbatim}
          8051 Microcontroller
         +{-{-}{-}{-}{-}{-}{-}{-}{-}{-}{-}{-}{-}{-}{-}{-}{-}{-}{-}+}
   P1.0{-{-}| 1              40 |{-}{-}VCC}
   P1.1{-{-}| 2              39 |{-}{-}P0.0/AD0}
   P1.2{-{-}| 3              38 |{-}{-}P0.1/AD1}
   P1.3{-{-}| 4              37 |{-}{-}P0.2/AD2}
   P1.4{-{-}| 5              36 |{-}{-}P0.3/AD3}
   P1.5{-{-}| 6              35 |{-}{-}P0.4/AD4}
   P1.6{-{-}| 7              34 |{-}{-}P0.5/AD5}
   P1.7{-{-}| 8              33 |{-}{-}P0.6/AD6}
   RST {-{-}| 9              32 |{-}{-}P0.7/AD7}
 P3.0/RXD| 10             31 |{-{-}EA/VPP}
 P3.1/TXD| 11             30 |{-{-}ALE/PROG}
P3.2/INT0| 12             29 |{-{-}PSEN}
P3.3/INT1| 13             28 |{-{-}P2.7/A15}
 P3.4/T0{-| 14             27 |{-}{-}P2.6/A14}
 P3.5/T1{-| 15             26 |{-}{-}P2.5/A13}
 P3.6/WR{-| 16             25 |{-}{-}P2.4/A12}
 P3.7/RD{-| 17             24 |{-}{-}P2.3/A11}
 XTAL2 {-{-}| 18             23 |{-}{-}P2.2/A10}
 XTAL1 {-{-}| 19             22 |{-}{-}P2.1/A9}
   VSS {-{-}| 20             21 |{-}{-}P2.0/A8}
         +{-{-}{-}{-}{-}{-}{-}{-}{-}{-}{-}{-}{-}{-}{-}{-}{-}{-}{-}+}
\end{verbatim}

\textbf{પિન ગ્રુપ્સ:}

\begin{itemize}
\tightlist
\item
  \textbf{પાવર}: \textbf{VCC (40), VSS (20)}
\item
  \textbf{ક્લોક}: \textbf{XTAL1 (19), XTAL2 (18)}
\item
  \textbf{રીસેટ}: \textbf{RST (9)}
\item
  \textbf{પોર્ટ્સ}: \textbf{P0, P1, P2, P3}
\item
  \textbf{કંટ્રોલ}: \textbf{ALE, PSEN, EA}
\end{itemize}

\textbf{યાદગાર વાક્ય}: ``પાવર ક્લોક રીસેટ પોર્ટ્સ કંટ્રોલ'' (PCRPC)

\end{solutionbox}
\subsection*{પ્રશ્ન 4(અ) [3
ગુણ]}\label{uxaaauxab0uxab6uxaa8-4uxa85-3-uxa97uxaa3}

\textbf{એરિથમેટિક instruction ઉદાહરણ સાથે સમજાવો.}

\begin{solutionbox}

\textbf{અંકગણિત સૂચનાઓ:}

{\def\LTcaptype{none} % do not increment counter
\begin{longtable}[]{@{}lll@{}}
\toprule\noalign{}
સૂચના & કાર્ય & ઉદાહરણ \\
\midrule\noalign{}
\endhead
\bottomrule\noalign{}
\endlastfoot
\textbf{ADD} & બસ્તારણ & ADD A,\#10H \\
\textbf{SUBB} & બાદબાકી & SUBB A,R0 \\
\textbf{MUL} & ગુણાકાર & MUL AB \\
\textbf{DIV} & ભાગાકાર & DIV AB \\
\textbf{INC} & વૃદ્ધિ & INC A \\
\textbf{DEC} & ઘટાડો & DEC R1 \\
\end{longtable}
}

\begin{itemize}
\tightlist
\item
  \textbf{ADD A,\#10H}: \textbf{એક્યુમ્યુલેટરમાં 10H ઉમેરો}
\item
  \textbf{ફ્લેગ્સ}: \textbf{અંકગણિત કામગીરીથી} પ્રભાવિત થાય છે
\end{itemize}

\textbf{યાદગાર વાક્ય}: ``એડ સબ મલ ડિવ ઇન્ક ડેક'' (ASMIDI)

\end{solutionbox}
\subsection*{પ્રશ્ન 4(બ) [4
ગુણ]}\label{uxaaauxab0uxab6uxaa8-4uxaac-4-uxa97uxaa3}

\textbf{મેમરી લોકેશન 65H પર સંગ્રહિત મૂલ્યના 2's complement ને શોધવા માટે 8051
એસેમ્બલી લેંગ્વેજ પ્રોગ્રામ લખો તેમજ પરિણામ સમાન સ્થાન પર મૂકો.}

\begin{solutionbox}

\begin{verbatim}
ORG 0000H           ; પ્રોગ્રામ સ્ટાર્ટ એડ્રેસ
MOV A,65H           ; 65H લોકેશનથી વેલ્યુ લોડ કરો
CPL A               ; વેલ્યુનો કોમ્પ્લિમેન્ટ કરો (1{s complement)}
ADD A,\#01H          ; 2{s complement મેળવવા 1 ઉમેરો}
MOV 65H,A           ; પરિણામ પાછું 65H માં સ્ટોર કરો
SJMP $              ; પ્રોગ્રામ બંધ કરો
END
\end{verbatim}

\textbf{પ્રોગ્રામ સ્ટેપ્સ:}

\begin{itemize}
\tightlist
\item
  \textbf{લોડ}: \textbf{મેમરી લોકેશન 65H} થી વેલ્યુ મેળવો
\item
  \textbf{કોમ્પ્લિમેન્ટ}: \textbf{CPL વાપરીને 1's complement} જનરેટ કરો
\item
  \textbf{1 ઉમેરો}: \textbf{2's complement માં} કન્વર્ટ કરો
\item
  \textbf{સ્ટોર}: \textbf{પરિણામ સમાન લોકેશન} પર પાછું મૂકો
\end{itemize}

\textbf{યાદગાર વાક્ય}: ``લોડ કોમ્પ્લિમેન્ટ એડ સ્ટોર'' (LCAS)

\end{solutionbox}
\subsection*{પ્રશ્ન 4(ક) [7
ગુણ]}\label{uxaaauxab0uxab6uxaa8-4uxa95-7-uxa97uxaa3}

\textbf{8051 માઇક્રોકંટ્રોલરના એડ્રેસિંગ મોડ્સની યાદી બનાવો અને તેમને ઉદાહરણ સાથે
સમજાવો.}

\begin{solutionbox}

\textbf{એડ્રેસિંગ મોડ્સ કોષ્ટક:}

{\def\LTcaptype{none} % do not increment counter
\begin{longtable}[]{@{}llll@{}}
\toprule\noalign{}
મોડ & વર્ણન & ઉદાહરણ & ઉપયોગ \\
\midrule\noalign{}
\endhead
\bottomrule\noalign{}
\endlastfoot
\textbf{ઇમીડિયેટ} & સૂચનામાં ડેટા & MOV A,\#25H & કોન્સ્ટંટ ડેટા \\
\textbf{રજિસ્ટર} & રજિસ્ટરમાં ડેટા & MOV A,R0 & ઝડપી એક્સેસ \\
\textbf{ડાયરેક્ટ} & મેમરી એડ્રેસ & MOV A,30H & RAM એક્સેસ \\
\textbf{ઇન્ડાયરેક્ટ} & રજિસ્ટરમાં એડ્રેસ & MOV A,@R0 & પોઇન્ટર એક્સેસ \\
\textbf{ઇન્ડેક્સ્ડ} & બેઝ + ઓફસેટ & MOVC A,@A+DPTR & ટેબલ એક્સેસ \\
\textbf{રિલેટિવ} & PC + ઓફસેટ & SJMP LOOP & બ્રાન્ચ સૂચનાઓ \\
\textbf{બિટ} & બિટ એડ્રેસ & SETB P1.0 & બિટ ઓપરેશન્સ \\
\end{longtable}
}

\textbf{ઉદાહરણો:}

\begin{itemize}
\tightlist
\item
  \textbf{MOV A,\#25H}: \textbf{ઇમીડિયેટ વેલ્યુ 25H} લોડ કરો
\item
  \textbf{MOV A,@R0}: \textbf{R0 માં આપેલા એડ્રેસ} થી લોડ કરો
\item
  \textbf{SJMP LOOP}: \textbf{વર્તમાન PC ની સાપેક્ષે} જમ્પ કરો
\end{itemize}

\textbf{યાદગાર વાક્ય}: ``ઇમીડિયેટ રજિસ્ટર ડાયરેક્ટ ઇન્ડાયરેક્ટ ઇન્ડેક્સ્ડ રિલેટિવ
બિટ'' (IRDIIRB)

\end{solutionbox}
\subsection*{પ્રશ્ન 4(અ અથવા) [3
ગુણ]}\label{uxaaauxab0uxab6uxaa8-4uxa85-uxa85uxaa5uxab5-3-uxa97uxaa3}

\textbf{લોજીકલ instruction ઉદાહરણ સાથે સમજાવો.}

\begin{solutionbox}

\textbf{તાર્કિક સૂચનાઓ:}

{\def\LTcaptype{none} % do not increment counter
\begin{longtable}[]{@{}lll@{}}
\toprule\noalign{}
સૂચના & કાર્ય & ઉદાહરણ \\
\midrule\noalign{}
\endhead
\bottomrule\noalign{}
\endlastfoot
\textbf{ANL} & AND ઓપરેશન & ANL A,\#0FH \\
\textbf{ORL} & OR ઓપરેશન & ORL A,R1 \\
\textbf{XRL} & XOR ઓપરેશન & XRL A,\#55H \\
\textbf{CPL} & કોમ્પ્લિમેન્ટ & CPL A \\
\textbf{RL} & લેફ્ટ રોટેટ & RL A \\
\textbf{RR} & રાઇટ રોટેટ & RR A \\
\end{longtable}
}

\begin{itemize}
\tightlist
\item
  \textbf{ANL A,\#0FH}: \textbf{એક્યુમ્યુલેટરને 0FH સાથે AND} કરો (માસ્ક ઓપરેશન)
\item
  \textbf{એપ્લિકેશન્સ}: \textbf{બિટ માસ્કિંગ, ડેટા મેનિપ્યુલેશન}
\end{itemize}

\textbf{યાદગાર વાક્ય}: ``એન્ડ ઓર એક્સઓર કોમ્પ્લિમેન્ટ રોટેટ'' (AOXCR)

\end{solutionbox}
\subsection*{પ્રશ્ન 4(બ અથવા) [4
ગુણ]}\label{uxaaauxab0uxab6uxaa8-4uxaac-uxa85uxaa5uxab5-4-uxa97uxaa3}

\textbf{રજિસ્ટર R3 માં સંગ્રહિત સંખ્યાને રજિસ્ટર R0 માં સંગ્રહિત સંખ્યા વડે ગુણાકાર
કરવા માટે 8051 એસેમ્બલી લેંગ્વેજ પ્રોગ્રામ લખો અને પરિણામને ઇન્ટર્નલ RAM સ્થાન
10h(MSB) અને 11h(LSB) માં મૂકો.}

\begin{solutionbox}

\begin{verbatim}
ORG 0000H           ; પ્રોગ્રામ સ્ટાર્ટ એડ્રેસ
MOV A,R3            ; R3 ને એક્યુમ્યુલેટરમાં મૂવ કરો
MOV B,R0            ; R0 ને B રજિસ્ટરમાં મૂવ કરો
MUL AB              ; A અને B નો ગુણાકાર કરો
MOV 10H,B           ; MSB (B) ને લોકેશન 10H માં સ્ટોર કરો
MOV 11H,A           ; LSB (A) ને લોકેશન 11H માં સ્ટોર કરો
SJMP $              ; પ્રોગ્રામ બંધ કરો
END
\end{verbatim}

\textbf{પ્રોગ્રામ ફ્લો:}

\begin{itemize}
\tightlist
\item
  \textbf{લોડ}: \textbf{ગુણ્ય અને ગુણક} ને A અને B માં મૂવ કરો
\item
  \textbf{ગુણાકાર}: \textbf{MUL AB સૂચના} વાપરો
\item
  \textbf{સ્ટોર}: \textbf{MSB B રજિસ્ટરમાં, LSB A રજિસ્ટરમાં}
\item
  \textbf{પરિણામ}: \textbf{16-બિટ પરિણામ બે લોકેશન} માં સ્ટોર કર્યું
\end{itemize}

\textbf{યાદગાર વાક્ય}: ``લોડ મલ્ટિપ્લાય સ્ટોર રિઝલ્ટ'' (LMSR)

\end{solutionbox}
\subsection*{પ્રશ્ન 4(ક અથવા) [7
ગુણ]}\label{uxaaauxab0uxab6uxaa8-4uxa95-uxa85uxaa5uxab5-7-uxa97uxaa3}

\textbf{ઉદાહરણ સાથે ડેટા ટ્રાન્સફર instruction સમજાવો.}

\begin{solutionbox}

\textbf{ડેટા ટ્રાન્સફર સૂચનાઓ:}

{\def\LTcaptype{none} % do not increment counter
\begin{longtable}[]{@{}llll@{}}
\toprule\noalign{}
કેટેગરી & સૂચના & ઉદાહરણ & કાર્ય \\
\midrule\noalign{}
\endhead
\bottomrule\noalign{}
\endlastfoot
\textbf{રજિસ્ટર} & MOV & MOV A,R0 & રજિસ્ટર થી રજિસ્ટર \\
\textbf{ઇમીડિયેટ} & MOV & MOV A,\#25H & ઇમીડિયેટ થી રજિસ્ટર \\
\textbf{ડાયરેક્ટ} & MOV & MOV A,30H & મેમરી થી રજિસ્ટર \\
\textbf{ઇન્ડાયરેક્ટ} & MOV & MOV A,@R0 & ઇન્ડાયરેક્ટ એડ્રેસિંગ \\
\textbf{એક્સ્ટર્નલ} & MOVX & MOVX A,@DPTR & એક્સ્ટર્નલ મેમરી \\
\textbf{કોડ} & MOVC & MOVC A,@A+DPTR & કોડ મેમરી \\
\textbf{સ્ટેક} & PUSH/POP & PUSH ACC & સ્ટેક ઓપરેશન્સ \\
\end{longtable}
}

\textbf{ઉદાહરણો:}

\begin{itemize}
\tightlist
\item
  \textbf{MOV A,R0}: \textbf{R0 ની સામગ્રી એક્યુમ્યુલેટર} માં મૂવ કરો
\item
  \textbf{MOVX A,@DPTR}: \textbf{એક્સ્ટર્નલ ડેટા મેમરી} થી વાંચો
\item
  \textbf{PUSH ACC}: \textbf{એક્યુમ્યુલેટરને સ્ટેક} પર પુશ કરો
\end{itemize}

\textbf{ડેટા મૂવમેન્ટ:}

\begin{itemize}
\tightlist
\item
  \textbf{આંતરિક}: \textbf{8051 મેમરી સ્પેસ} અંદર
\item
  \textbf{બાહ્ય}: \textbf{એક્સ્ટર્નલ મેમરી} તરફ/થી
\item
  \textbf{કોડ}: \textbf{પ્રોગ્રામ મેમરી} થી
\end{itemize}

\textbf{યાદગાર વાક્ય}: ``મૂવ ડેટા બિટવીન લોકેશન્સ'' (MDBL)

\end{solutionbox}
\subsection*{પ્રશ્ન 5(અ) [3
ગુણ]}\label{uxaaauxab0uxab6uxaa8-5uxa85-3-uxa97uxaa3}

\textbf{PSW ફોર્મેટની મદદથી 8051 ફ્લેગ્સ સમજાવો.}

\begin{solutionbox}

\textbf{PSW રજિસ્ટર (D0H):}

\begin{verbatim}
 D7  D6  D5  D4  D3  D2  D1  D0
+{-{-}{-}+{-}{-}{-}+{-}{-}{-}+{-}{-}{-}+{-}{-}{-}+{-}{-}{-}+{-}{-}{-}+{-}{-}{-}+}
| C |AC | F0|RS1|RS0| OV| {- | P |}
+{-{-}{-}+{-}{-}{-}+{-}{-}{-}+{-}{-}{-}+{-}{-}{-}+{-}{-}{-}+{-}{-}{-}+{-}{-}{-}+}
\end{verbatim}

\textbf{ફ્લેગ કાર્યો:}

\begin{itemize}
\tightlist
\item
  \textbf{C (કેરી)}: \textbf{કેરી/બોરો} જ્યારે થાય તો સેટ
\item
  \textbf{AC (ઓક્સિલિયરી કેરી)}: \textbf{BCD અંકગણિત} માટે
\item
  \textbf{OV (ઓવરફ્લો)}: \textbf{સાઇન્ડ ઓવરફ્લો} થાય તો સેટ
\item
  \textbf{P (પેરિટી)}: \textbf{એક્યુમ્યુલેટરની ઇવન પેરિટી}
\item
  \textbf{RS1, RS0}: \textbf{રજિસ્ટર બેંક સિલેક્ટ} બિટ્સ
\end{itemize}

\textbf{યાદગાર વાક્ય}: ``કેરી ઓક્સિલિયરી ઓવરફ્લો પેરિટી રજિસ્ટર'' (CAOPR)

\end{solutionbox}
\subsection*{પ્રશ્ન 5(બ) [4
ગુણ]}\label{uxaaauxab0uxab6uxaa8-5uxaac-4-uxa97uxaa3}

\textbf{માઇક્રોકંટ્રોલર સાથે 7 સેગમેન્ટ ઇન્ટરફેસિંગ ડાયાગ્રામ દોરો અને સમજાવો.}

\begin{solutionbox}

\textbf{7-સેગમેન્ટ ઇન્ટરફેસ સર્કિટ:}

\begin{verbatim}
    8051           ULN2003        7{-Segment Display}
    P1.0 {-{-}{-}{-}{-}{-}{-}{-}{-} I1 {-}{-}{-}{-}{-} O1 {-}{-}{-}{-}{-} a}
    P1.1 {-{-}{-}{-}{-}{-}{-}{-}{-} I2 {-}{-}{-}{-}{-} O2 {-}{-}{-}{-}{-} b  }
    P1.2 {-{-}{-}{-}{-}{-}{-}{-}{-} I3 {-}{-}{-}{-}{-} O3 {-}{-}{-}{-}{-} c}
    P1.3 {-{-}{-}{-}{-}{-}{-}{-}{-} I4 {-}{-}{-}{-}{-} O4 {-}{-}{-}{-}{-} d}
    P1.4 {-{-}{-}{-}{-}{-}{-}{-}{-} I5 {-}{-}{-}{-}{-} O5 {-}{-}{-}{-}{-} e}
    P1.5 {-{-}{-}{-}{-}{-}{-}{-}{-} I6 {-}{-}{-}{-}{-} O6 {-}{-}{-}{-}{-} f}
    P1.6 {-{-}{-}{-}{-}{-}{-}{-}{-} I7 {-}{-}{-}{-}{-} O7 {-}{-}{-}{-}{-} g}
    P1.7 {-{-}{-}{-}{-}{-}{-}{-}{-} I8 {-}{-}{-}{-}{-} O8 {-}{-}{-}{-}{-} DP}
                                        |
                                    Common Cathode
                                        |
                                       GND
\end{verbatim}

\textbf{ઘટકો:}

\begin{itemize}
\tightlist
\item
  \textbf{ULN2003}: \textbf{કરંટ ડ્રાઇવર IC}
\item
  \textbf{રેઝિસ્ટર્સ}: \textbf{કરંટ લિમિટિંગ} (330Ω)
\item
  \textbf{ડિસ્પ્લે}: \textbf{કોમન કેથોડ} પ્રકાર
\end{itemize}

\textbf{કામકાજ}: \textbf{પોર્ટ ડેટા કરંટ ડ્રાઇવર} દ્વારા ડિસ્પ્લે સેગમેન્ટ્સ ચલાવે છે

\textbf{યાદગાર વાક્ય}: ``પોર્ટ ડ્રાઇવર ડિસ્પ્લે ગ્રાઉન્ડ'' (PDDG)

\end{solutionbox}
\subsection*{પ્રશ્ન 5(ક) [7
ગુણ]}\label{uxaaauxab0uxab6uxaa8-5uxa95-7-uxa97uxaa3}

\textbf{માઇક્રોકંટ્રોલર સાથે 8 LED ને ઇન્ટરફેસ કરો અને ચાલુ અને બંધ કરવા માટે
પ્રોગ્રામ લખો.}

\begin{solutionbox}

\textbf{LED ઇન્ટરફેસ સર્કિટ:}

\begin{verbatim}
    8051           Current Limiting      LEDs
    P1.0 {-{-}{-}{-}{-}{-}{-}{-}{-} 330Ω {-}{-}{-}{-}{-}{-}{-}{-}{-} LED0 {-}{-}{-}{-}{-} +5V}
    P1.1 {-{-}{-}{-}{-}{-}{-}{-}{-} 330Ω {-}{-}{-}{-}{-}{-}{-}{-}{-} LED1 {-}{-}{-}{-}{-} +5V}
    P1.2 {-{-}{-}{-}{-}{-}{-}{-}{-} 330Ω {-}{-}{-}{-}{-}{-}{-}{-}{-} LED2 {-}{-}{-}{-}{-} +5V}
    P1.3 {-{-}{-}{-}{-}{-}{-}{-}{-} 330Ω {-}{-}{-}{-}{-}{-}{-}{-}{-} LED3 {-}{-}{-}{-}{-} +5V}
    P1.4 {-{-}{-}{-}{-}{-}{-}{-}{-} 330Ω {-}{-}{-}{-}{-}{-}{-}{-}{-} LED4 {-}{-}{-}{-}{-} +5V}
    P1.5 {-{-}{-}{-}{-}{-}{-}{-}{-} 330Ω {-}{-}{-}{-}{-}{-}{-}{-}{-} LED5 {-}{-}{-}{-}{-} +5V}
    P1.6 {-{-}{-}{-}{-}{-}{-}{-}{-} 330Ω {-}{-}{-}{-}{-}{-}{-}{-}{-} LED6 {-}{-}{-}{-}{-} +5V}
    P1.7 {-{-}{-}{-}{-}{-}{-}{-}{-} 330Ω {-}{-}{-}{-}{-}{-}{-}{-}{-} LED7 {-}{-}{-}{-}{-} +5V}
\end{verbatim}

\textbf{એસેમ્બલી પ્રોગ્રામ:}

\begin{verbatim}
ORG 0000H           ; સ્ટાર્ટ એડ્રેસ
MAIN:
    MOV P1,\#0FFH    ; બધા LEDs ચાલુ કરો (logic 0)
    CALL DELAY      ; ડિલે સબરૂટિન કોલ કરો
    MOV P1,\#00H     ; બધા LEDs બંધ કરો (logic 1)
    CALL DELAY      ; ડિલે સબરૂટિન કોલ કરો
    SJMP MAIN       ; સતત રિપીટ કરો

DELAY:
    MOV R2,\#250     ; આઉટર લૂપ કાઉન્ટર
D1: MOV R3,\#250     ; ઇનર લૂપ કાઉન્ટર
D2: DJNZ R3,D2      ; R3 શૂન્ય થાય ત્યાં સુધી ઘટાડો
    DJNZ R2,D1      ; R2 શૂન્ય થાય ત્યાં સુધી ઘટાડો
    RET             ; સબરૂટિનથી રિટર્ન કરો
END
\end{verbatim}

\textbf{યાદગાર વાક્ય}: ``લાઇટ ઇમિટિંગ ડિસ્પ્લે ઇન્ટરફેસ'' (LEDI)

\end{solutionbox}
\subsection*{પ્રશ્ન 5(અ અથવા) [3
ગુણ]}\label{uxaaauxab0uxab6uxaa8-5uxa85-uxa85uxaa5uxab5-3-uxa97uxaa3}

\textbf{વિવિધ ક્ષેત્રોમાં માઇક્રોકંટ્રોલરની એપ્લિકેશનોની સૂચિ બનાવો.}

\begin{solutionbox}

\textbf{ક્ષેત્ર પ્રમાણે એપ્લિકેશન્સ:}

{\def\LTcaptype{none} % do not increment counter
\begin{longtable}[]{@{}ll@{}}
\toprule\noalign{}
ક્ષેત્ર & એપ્લિકેશન્સ \\
\midrule\noalign{}
\endhead
\bottomrule\noalign{}
\endlastfoot
\textbf{ઘર} & વોશિંગ મશીન, માઇક્રોવેવ, AC \\
\textbf{ઓટોમોટિવ} & એન્જિન કંટ્રોલ, ABS, એરબેગ \\
\textbf{ઇન્ડસ્ટ્રિયલ} & પ્રોસેસ કંટ્રોલ, રોબોટિક્સ \\
\textbf{મેડિકલ} & પેસમેકર, બ્લડ પ્રેશર મોનિટર \\
\textbf{કમ્યુનિકેશન} & મોબાઇલ ફોન્સ, મોડેમ્સ \\
\textbf{સિક્યુરિટી} & એક્સેસ કંટ્રોલ, બર્ગલર એલાર્મ \\
\textbf{એન્ટરટેનમેન્ટ} & ગેમિંગ કન્સોલ્સ, રિમોટ કંટ્રોલ \\
\end{longtable}
}

\textbf{યાદગાર વાક્ય}: ``હોમ ઓટો ઇન્ડસ્ટ્રિયલ મેડિકલ કમ્યુનિકેશન સિક્યુરિટી
એન્ટરટેનમેન્ટ'' (HAIMCSE)

\end{solutionbox}
\subsection*{પ્રશ્ન 5(બ અથવા) [4
ગુણ]}\label{uxaaauxab0uxab6uxaa8-5uxaac-uxa85uxaa5uxab5-4-uxa97uxaa3}

\textbf{8051 સાથે ડીસી મોટરનું ઇન્ટરફેસિંગ ડાયાગ્રામ દોરો અને સમજાવો.}

\begin{solutionbox}

\textbf{ડીસી મોટર ઇન્ટરફેસ:}

\begin{verbatim}
    8051       L293D Motor Driver         DC Motor
    P1.0 {-{-}{-}{-}{-}{-}{-} Enable Pin                 |}
    P1.1 {-{-}{-}{-}{-}{-}{-} Input 1  {-}{-}{-}{-}{-} Output 1 {-}{-}+}
    P1.2 {-{-}{-}{-}{-}{-}{-} Input 2  {-}{-}{-}{-}{-} Output 2 {-}{-}+}
                    |              |
                   VCC            GND
                    |              |
                  +12V           Motor
\end{verbatim}

\textbf{ઘટકો:}

\begin{itemize}
\tightlist
\item
  \textbf{L293D}: \textbf{ડ્યુઅલ H-બ્રિજ ડ્રાઇવર IC}
\item
  \textbf{મોટર}: \textbf{12V ડીસી મોટર}
\item
  \textbf{કંટ્રોલ}: \textbf{દિશા અને સ્પીડ કંટ્રોલ}
\end{itemize}

\textbf{કંટ્રોલ લોજિક:}

\begin{itemize}
\tightlist
\item
  \textbf{આગળ}: \textbf{P1.1=1, P1.2=0}
\item
  \textbf{પાછળ}: \textbf{P1.1=0, P1.2=1}
\item
  \textbf{બંધ}: \textbf{P1.1=0, P1.2=0}
\end{itemize}

\textbf{યાદગાર વાક્ય}: ``ડ્રાઇવર કંટ્રોલ મોટર ડાયરેક્શન'' (DCMD)

\end{solutionbox}
\subsection*{પ્રશ્ન 5(ક અથવા) [7
ગુણ]}\label{uxaaauxab0uxab6uxaa8-5uxa95-uxa85uxaa5uxab5-7-uxa97uxaa3}

\textbf{માઇક્રોકંટ્રોલર સાથે એલસીડી ઇન્ટરફેસ કરો અને ``માઇક્રોપ્રોસેસર અને
માઇક્રોકંટ્રોલર'' દર્શાવવા માટે એક પ્રોગ્રામ લખો.}

\begin{solutionbox}

\textbf{LCD ઇન્ટરફેસ:}

\begin{verbatim}
    8051                16x2 LCD
    P2.0 {-{-}{-}{-}{-}{-}{-}{-}{-}{-}{-}{-}{-}{-} RS (Register Select)}
    P2.1 {-{-}{-}{-}{-}{-}{-}{-}{-}{-}{-}{-}{-}{-} EN (Enable)  }
    P1.0{-P1.7 {-}{-}{-}{-}{-}{-}{-}{-}{-} D0{-}D7 (Data lines)}
    GND {-{-}{-}{-}{-}{-}{-}{-}{-}{-}{-}{-}{-}{-}{-} VSS, RW}
    +5V {-{-}{-}{-}{-}{-}{-}{-}{-}{-}{-}{-}{-}{-}{-} VDD, VEE (via 10kΩ pot)}
\end{verbatim}

\textbf{એસેમ્બલી પ્રોગ્રામ:}

\begin{verbatim}
ORG 0000H
    CALL LCD\_INIT       ; LCD ઇનિશિયલાઇઝ કરો
    MOV DPTR,\#MSG1      ; મેસેજ તરફ પોઇન્ટ કરો
    CALL DISPLAY\_MSG    ; મેસેજ ડિસ્પ્લે કરો
    SJMP $              ; બંધ કરો

LCD\_INIT:
    MOV P1,\#38H         ; Function set: 8{-bit, 2{-}line}
    CLR P2.0            ; RS=0 (command)
    SETB P2.1           ; EN=1
    CLR P2.1            ; EN=0 (pulse)
    CALL DELAY
    MOV P1,\#01H         ; Clear display
    CLR P2.0
    SETB P2.1
    CLR P2.1
    CALL DELAY
    RET

DISPLAY\_MSG:
    MOVC A,@A+DPTR      ; કેરેક્ટર મેળવો
    JZ EXIT             ; જો શૂન્ય હોય તો બહાર નીકળો
    MOV P1,A            ; કેરેક્ટર મોકલો
    SETB P2.0           ; RS=1 (data)
    SETB P2.1           ; EN=1
    CLR P2.1            ; EN=0
    CALL DELAY
    INC DPTR            ; આગળનો કેરેક્ટર
    SJMP DISPLAY\_MSG    ; ચાલુ રાખો
EXIT:
    RET

MSG1: DB "Microprocessor and Microcontroller",0

DELAY:
    MOV R1,\#50
D1: MOV R2,\#255
D2: DJNZ R2,D2
    DJNZ R1,D1
    RET
END
\end{verbatim}

\textbf{મુખ્ય પગલાઓ:}

\begin{itemize}
\tightlist
\item
  \textbf{LCD ઇનિશિયલાઇઝેશન}: \textbf{8-બિટ મોડ, 2-લાઇન ડિસ્પ્લે}
\item
  \textbf{મેસેજ ડિસ્પ્લે}: \textbf{કેરેક્ટર દ્વારા કેરેક્ટર}
\item
  \textbf{કંટ્રોલ સિગ્નલ્સ}: \textbf{RS અને EN સિગ્નલ્સ}
\end{itemize}

\textbf{યાદગાર વાક્ય}: ``લિક્વિડ ક્રિસ્ટલ ડિસ્પ્લે ઇન્ટરફેસ'' (LCDI)

\end{solutionbox}

\end{document}
