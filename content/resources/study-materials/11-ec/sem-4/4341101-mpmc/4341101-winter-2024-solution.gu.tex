\documentclass{article}

% content/resources/templates/preamble.tex
\usepackage[margin=0.6in]{geometry}
\author{Milav Dabgar}
\usepackage{amsmath,amssymb,amsthm}
\usepackage{booktabs}
\usepackage{multirow}
\usepackage{xcolor}
\usepackage{tcolorbox}
\tcbuselibrary{breakable,skins}
\usepackage[colorlinks=true,linkcolor=blue]{hyperref}
\usepackage{titlesec}
\usepackage{enumitem}
\usepackage{tikz}
\usepackage{pgfplots}
\usepackage{circuitikz}
\usepackage[version=4]{mhchem}
\usepackage{longtable}
\usepackage{array}
\usepackage{float}
\usepackage{caption}
\usepackage{listings}

\lstset{
  basicstyle=\small\ttfamily,
  breaklines=true,
  breakatwhitespace=false,
  postbreak=\mbox{\textcolor{red}{$\hookrightarrow$}\space},
  float=false,
  numbers=left,
  numberstyle=\tiny\color{gray},
  numbersep=10pt,
  xleftmargin=2em,
  keywordstyle=\color{blue},
  commentstyle=\color{green!60!black},
  stringstyle=\color{purple},
  backgroundcolor=\color{gray!5},
  showstringspaces=false,
  tabsize=2,
  captionpos=b,
  keepspaces=true,
  columns=flexible
}

\pgfplotsset{compat=1.18}
\usetikzlibrary{shapes,arrows,positioning,calc,patterns,decorations.pathmorphing,decorations.markings,arrows.meta}

% Color scheme
\definecolor{headcolor}{RGB}{0,102,204}
\definecolor{keycolor}{RGB}{220,20,60}
\definecolor{solutioncolor}{RGB}{34,139,34}
\definecolor{mnemoniccolor}{RGB}{148,0,211}
\definecolor{codecolor}{RGB}{0,0,100}

% Spacing
\setlength{\parskip}{3pt}
\setlist[itemize]{nosep}
\setlist[enumerate]{nosep}

% Title formatting
\titleformat{\section}{\Large\bfseries\color{headcolor}}{\thesection}{1em}{}
\titleformat{\subsection}{\large\bfseries\color{headcolor}}{\thesubsection}{1em}{}

% Pandoc tightlist compatibility
\providecommand{\tightlist}{%
  \setlength{\itemsep}{0pt}\setlength{\parskip}{0pt}}

% Pandoc longtable compatibility
\newcounter{none}
\def\thenone{}


% content/resources/templates/gujarati-boxes.tex
\usepackage{fontspec}
\usepackage{polyglossia}

% Set Gujarati as main language (document is primarily in Gujarati)
% Note: gloss-gujarati.ldf doesn't exist in polyglossia, but it will use hyphenation patterns
\setdefaultlanguage{gujarati}
\setotherlanguage{english}

% Configure Gujarati font properly
% Use Language=Default to prevent polyglossia from trying to add language-specific features
% that don't exist for Gujarati, which causes "empty feature" warnings
\newfontfamily\gujaratifont[Script=Gujarati,AutoFakeBold=2.5,AutoFakeSlant=0.3]{Noto Sans Gujarati}
\setmainfont[Script=Gujarati,AutoFakeBold=2.5,AutoFakeSlant=0.3]{Noto Sans Gujarati}
% Use Noto Sans Gujarati for monospace to support Gujarati in text
\setmonofont[Scale=0.9]{Noto Sans Gujarati}

% Configure English to use the same font
\newfontfamily\englishfont[Script=Gujarati,AutoFakeBold=2.5,AutoFakeSlant=0.3]{Noto Sans Gujarati}

% Translations for polyglossia
\gappto\captionsgujarati{
  \renewcommand{\tablename}{કોષ્ટક}
  \renewcommand{\figurename}{આકૃતિ}
}

% Helper for TikZ nodes to ensure Gujarati font
\newcommand{\gu}[1]{{\gujaratifont #1}}

% Custom environments
\newtcolorbox{solutionbox}{
    breakable,
    enhanced,
    colback=solutioncolor!5!white,
    colframe=solutioncolor!75!black,
    fonttitle=\bfseries,
    title=જવાબ
}

\newtcolorbox{solutionboxnobreak}{
 colback=solutioncolor!5!white,
 colframe=solutioncolor!75!black,
 fonttitle=\bfseries,
 title=જવાબ
}

\newtcolorbox{keyformula}{
 breakable,
 enhanced,
 colback=keycolor!5!white,
 colframe=keycolor!75!black,
 fonttitle=\bfseries,
 title=રાસાયણિક સમીકરણ/સૂત્ર
}

\newtcolorbox{mnemonicbox}{
 breakable,
 enhanced,
 colback=mnemoniccolor!5!white,
 colframe=mnemoniccolor!75!black,
 fonttitle=\bfseries,
 title=મેમરી ટ્રીક
}


% Custom commands for GTU solutions
% This file defines semantic commands for consistent formatting

% Question command with automatic formatting
\newcommand{\question}[2]{%
  \section*{Question #1}%
  \textbf{#2}%
}

% OR question variant
\newcommand{\questionor}[2]{%
  \section*{Question #1 OR}%
  \textbf{#2}%
}

% Proper table environment with caption
\newenvironment{answertable}[1]{%
  \begin{table}[htbp]
  \centering
  \caption{#1}
}{%
  \end{table}
}

% Proper figure environment for diagrams
\newenvironment{answerdiagram}[1]{%
  \begin{figure}[htbp]
  \centering
  \caption{#1}
}{%
  \end{figure}
}

% Semantic markup for key terms
\newcommand{\keyword}[1]{\textbf{#1}}
\newcommand{\code}[1]{\texttt{#1}}
\newcommand{\classname}[1]{\texttt{#1}}
\newcommand{\methodname}[1]{\texttt{#1}}

% Proper quotation marks
\newcommand{\mnemonic}[1]{``#1''}

\usetikzlibrary{mindmap,trees}

\title{માઇક્રોપ્રોસેસર અને માઇક્રોકન્ટ્રોલર (4341101) - વિન્ટર 2024 સોલ્યુશન}
\date{15 જાન્યુઆરી, 2024}

\begin{document}
\maketitle

\questionmarks{1(a)}{3}{માઇક્રોકંટ્રોલર્સનાં સામાન્ય ફીચર્સની સૂચિ બનાવો.}

\begin{solutionbox}
\textbf{જવાબ}:

\begin{center}
\captionof{table}{સામાન્ય ફીચર્સ}
\begin{tabulary}{\linewidth}{|l|J|}
\hline
\textbf{ફીચર} & \textbf{હેતુ} \\ \hline
\textbf{CPU કોર} & સૂચનાઓ પ્રોસેસ કરવા \\ \hline
\textbf{મેમરી (RAM/ROM)} & પ્રોગ્રામ અને ડેટા સ્ટોર કરવા \\ \hline
\textbf{I/O પોર્ટ્સ} & બાહ્ય ડિવાઇસ સાથે ઇન્ટરફેસ \\ \hline
\textbf{ટાઇમર/કાઉન્ટર} & સમય અંતરાલ માપવા \\ \hline
\textbf{ઇન્ટરપ્ટ} & અસિંક્રોનસ ઘટનાઓ સંભાળવા \\ \hline
\textbf{સીરિયલ કમ્યુનિકેશન} & અન્ય ડિવાઇસ સાથે ડેટા ટ્રાન્સફર \\ \hline
\end{tabulary}
\end{center}
\end{solutionbox}
\begin{mnemonicbox}
``CPU-TIS'' (CPU-RAM-I/O-Timer-Interrupt-Serial)
\end{mnemonicbox}

\questionmarks{1(b)}{4}{ALU ના કાર્યો સમજાવો.}

\begin{solutionbox}
\textbf{જવાબ}:

\begin{center}
\captionof{table}{ALU કાર્યો}
\begin{tabulary}{\linewidth}{|l|J|}
\hline
\textbf{કાર્ય} & \textbf{વર્ણન} \\ \hline
\textbf{ગણિત ઓપરેશન્સ} & સરવાળો, બાદબાકી, ઇન્ક્રિમેન્ટ, ડિક્રિમેન્ટ \\ \hline
\textbf{લોજિકલ ઓપરેશન્સ} & AND, OR, XOR, NOT, તુલના \\ \hline
\textbf{ડેટા મૂવમેન્ટ} & રજિસ્ટર અને મેમરી વચ્ચે ટ્રાન્સફર \\ \hline
\textbf{ફ્લેગ સેટિંગ} & ઓપરેશન પરિણામ પર આધારિત સ્ટેટસ ફ્લેગ અપડેટ \\ \hline
\end{tabulary}
\end{center}

\textbf{ડાયાગ્રામ:}

\begin{center}
\begin{tikzpicture}[auto, node distance=2cm]
    \node [gtu block, minimum width=3cm, minimum height=2cm] (alu) {\textbf{ALU}\\(Arithmetic \& Logic Unit)};
    \node [left of=alu, node distance=4cm] (input) {Input Data};
    \node [right of=alu, node distance=4cm] (output) {Output Data};
    \node [block, below of=alu, node distance=2.5cm] (flags) {Status Flags\\(Z, C, S, P)};
    
    \draw [gtu arrow] (input) -- (alu);
    \draw [gtu arrow] (alu) -- (output);
    \draw [gtu arrow] (alu) -- (flags);
\end{tikzpicture}
\end{center}
\end{solutionbox}
\begin{mnemonicbox}
``ALFS'' (Arithmetic-Logic-Flags-Status)
\end{mnemonicbox}

\questionmarks{1(c)}{7}{વ્યાખ્યાયિત કરો: મેમરી, ઓપરેન્ડ, ઈન્સ્ટ્રક્શન સાયકલ, ઓપકોડ, CU, મશીન સાયકલ, CISC}

\begin{solutionbox}
\textbf{જવાબ}:

\begin{center}
\captionof{table}{વ્યાખ્યાઓ}
\begin{tabulary}{\linewidth}{|l|J|}
\hline
\textbf{શબ્દ} & \textbf{વ્યાખ્યા} \\ \hline
\textbf{મેમરી} & ડેટા અને સૂચનાઓ સંગ્રહિત કરતું સ્ટોરેજ યુનિટ. \\ \hline
\textbf{ઓપરેન્ડ} & ઓપરેશનમાં વપરાતી ડેટા વેલ્યુ અથવા એડ્રેસ. \\ \hline
\textbf{ઈન્સ્ટ્રક્શન સાયકલ} & સૂચના ફેચ અને એક્ઝિક્યુટ કરવાની સંપૂર્ણ પ્રક્રિયા. \\ \hline
\textbf{ઓપકોડ} & સૂચનાનો પ્રકાર દર્શાવતો ઓપરેશન કોડ. \\ \hline
\textbf{CU} & પ્રોસેસર ઓપરેશન્સનું સંકલન કરતું કંટ્રોલ યુનિટ. \\ \hline
\textbf{મશીન સાયકલ} & T-સ્ટેટ્સથી બનેલી મૂળભૂત ઓપરેશન સાયકલ. \\ \hline
\textbf{CISC} & સમૃદ્ધ સૂચના સેટ સાથેનું કોમ્પ્લેક્સ ઇન્સ્ટ્રક્શન સેટ કમ્પ્યુટર. \\ \hline
\end{tabulary}
\end{center}

\begin{itemize}
    \item \textbf{મેમરી}: યુનિક એડ્રેસ સાથે સ્ટોરેજ સેલનો વ્યવસ્થિત એરે.
    \item \textbf{ઓપરેન્ડ}: સૂચનાઓ જેના પર ક્રિયા કરે છે તે ડેટા એલિમેન્ટ.
    \item \textbf{ઈન્સ્ટ્રક્શન સાયકલ}: દરેક સૂચના માટે ફેચ-ડિકોડ-એક્ઝિક્યુટ સિક્વન્સ.
    \item \textbf{ઓપકોડ}: પ્રોસેસરને કયું ઓપરેશન કરવાનું છે તે જણાવતો બાઇનરી કોડ.
\end{itemize}

\textbf{ડાયાગ્રામ:}

\begin{center}
\begin{tikzpicture}[node distance=3cm, auto]
    \node [gtu state] (fetch) {Fetch};
    \node [gtu state, right of=fetch] (decode) {Decode};
    \node [gtu state, right of=decode] (execute) {Execute};
    
    \draw [gtu arrow] (fetch) -- (decode);
    \draw [gtu arrow] (decode) -- (execute);
    \draw [gtu arrow] (execute.south) -- ++(0,-0.5) -| (fetch.south);
\end{tikzpicture}
\end{center}
\end{solutionbox}
\begin{mnemonicbox}
``MO-ICO-MC'' (Memory-Operand-Instruction-Control-Operation-Machine-Complex)
\end{mnemonicbox}

\orquestionmarks{1(c)}{7}{i) વ્યાખ્યાયિત કરો: માઇક્રોપ્રોસેસર. ii) વોન-ન્યુમેન અને હાર્વર્ડ આર્કિટેક્ચરની તુલના કરો.}

\begin{solutionbox}
\textbf{જવાબ}:

\textbf{i) માઇક્રોપ્રોસેસર વ્યાખ્યા:}
એક ઇન્ટિગ્રેટેડ સર્કિટ કે જેમાં કમ્પ્યુટરના CPU ફંક્શનાલિટી સમાવિષ્ટ હોય છે, જે સૂચનાઓને ફેચ, ડિકોડ અને એક્ઝિક્યુટ કરવા સક્ષમ છે અને ALU અને કંટ્રોલ સર્કિટરી એક જ ચિપ પર ધરાવે છે.

\textbf{ii) વોન-ન્યુમેન vs હાર્વર્ડ આર્કિટેક્ચર:}

\begin{center}
\captionof{table}{તુલના}
\begin{tabulary}{\linewidth}{|l|J|J|}
\hline
\textbf{લક્ષણ} & \textbf{વોન-ન્યુમેન} & \textbf{હાર્વર્ડ} \\ \hline
\textbf{મેમરી} & એક શેર્ડ મેમરી & અલગ પ્રોગ્રામ અને ડેટા મેમરી \\ \hline
\textbf{બસ} & ડેટા અને સૂચનાઓ માટે એક બસ & અલગ બસ \\ \hline
\textbf{સ્પીડ} & ધીમી (મેમરી બોટલનેક) & ઝડપી (પેરેલલ એક્સેસ) \\ \hline
\textbf{જટિલતા} & સરળ ડિઝાઇન & વધુ જટિલ \\ \hline
\textbf{ઉપયોગ} & જનરલ કમ્પ્યુટિંગ & રીયલ-ટાઇમ સિસ્ટમ \\ \hline
\end{tabulary}
\end{center}

\textbf{ડાયાગ્રામ:}

\begin{center}
\begin{minipage}{0.45\linewidth}
\centering
\textbf{Von-Neumann}
\begin{tikzpicture}[node distance=2cm, auto, scale=0.8, transform shape]
    \node [gtu block] (cpu) {CPU};
    \node [gtu block, right of=cpu, node distance=3cm] (mem) {Memory\\(Data + Code)};
    \draw [gtu arrow, <->] (cpu) -- (mem) node[midway, above] {Bus};
\end{tikzpicture}
\end{minipage}
\hfill
\begin{minipage}{0.45\linewidth}
\centering
\textbf{Harvard}
\begin{tikzpicture}[node distance=2cm, auto, scale=0.8, transform shape]
    \node [gtu block] (cpu) {CPU};
    \node [gtu block, right of=cpu, node distance=3cm, yshift=1cm] (prog) {Program\\Memory};
    \node [gtu block, right of=cpu, node distance=3cm, yshift=-1cm] (data) {Data\\Memory};
    
    \draw [gtu arrow, ->] (cpu.30) -- (prog.west);
    \draw [gtu arrow, <->] (cpu.-30) -- (data.west);
\end{tikzpicture}
\end{minipage}
\end{center}
\end{solutionbox}
\begin{mnemonicbox}
``Harvard Has Separate Spaces''
\end{mnemonicbox}

\questionmarks{2(a)}{3}{8085 માઇક્રોપ્રોસેસરના વિવિધ રજીસ્ટરો સમજાવો.}

\begin{solutionbox}
\textbf{જવાબ}:

\begin{center}
\captionof{table}{8085 રજિસ્ટરો}
\begin{tabulary}{\linewidth}{|l|l|J|}
\hline
\textbf{રજિસ્ટર} & \textbf{સાઇઝ} & \textbf{કાર્ય} \\ \hline
\textbf{એક્યુમુલેટર (A)} & 8-બિટ & ગાણિતિક અને લોજિક માટે મુખ્ય રજિસ્ટર. \\ \hline
\textbf{જનરલ પર્પઝ} & 8-બિટ & B, C, D, E, H, L (અસ્થાયી ડેટા સ્ટોરેજ). \\ \hline
\textbf{પ્રોગ્રામ કાઉન્ટર (PC)} & 16-બિટ & આગલી સૂચનાનું એડ્રેસ. \\ \hline
\textbf{સ્ટેક પોઇન્ટર (SP)} & 16-બિટ & સ્ટેકના ટોપને પોઇન્ટ કરે. \\ \hline
\textbf{ફ્લેગ રજિસ્ટર} & 8-બિટ & સ્ટેટસ ફ્લેગ્સ (Z, S, P, CY, AC). \\ \hline
\end{tabulary}
\end{center}
\end{solutionbox}
\begin{mnemonicbox}
``AGSF'' (Accumulator-General-Stack-Flags)
\end{mnemonicbox}

\questionmarks{2(b)}{4}{ઈન્સ્ટ્રક્શનનું ફેચિંગ, ડીકોડિંગ અને એક્ઝેક્યુશન સમજાવો.}

\begin{solutionbox}
\textbf{જવાબ}:

\begin{center}
\captionof{table}{ઇન્સ્ટ્રક્શન ફેઝ}
\begin{tabulary}{\linewidth}{|l|J|J|}
\hline
\textbf{ફેઝ} & \textbf{પ્રવૃત્તિ} & \textbf{સંબંધિત હાર્ડવેર} \\ \hline
\textbf{ફેચિંગ} & PC માંના એડ્રેસથી મેમરીમાંથી સૂચના મેળવવી. & PC, એડ્રેસ બસ, મેમરી \\ \hline
\textbf{ડીકોડિંગ} & ઓપરેશન પ્રકાર અને ઓપરેન્ડ ઓળખવા. & ઈન્સ્ટ્રક્શન રજિસ્ટર, કંટ્રોલ યુનિટ \\ \hline
\textbf{એક્ઝેક્યુશન} & નિર્દિષ્ટ ઓપરેશન કરવું. & ALU, રજિસ્ટર્સ, ડેટા બસ \\ \hline
\end{tabulary}
\end{center}

\textbf{ડાયાગ્રામ:}

\begin{center}
\begin{tikzpicture}[node distance=3cm, auto]
    \node [gtu state] (fetch) {Fetch};
    \node [gtu state, right of=fetch] (decode) {Decode};
    \node [gtu state, right of=decode] (execute) {Execute};
    
    \draw [gtu arrow] (fetch) -- (decode);
    \draw [gtu arrow] (decode) -- (execute);
    
    \coordinate (below) at ($(decode.south) + (0,-1)$);
    \draw [gtu arrow] (execute.south) |- (below) -| (fetch.south) node[pos=0.25, below] {Next Instruction};
\end{tikzpicture}
\end{center}

\begin{itemize}
    \item \textbf{ફેચિંગ}: PC મેમરીને એડ્રેસ મોકલે, સૂચના IR માં લોડ થાય.
    \item \textbf{ડીકોડિંગ}: કંટ્રોલ યુનિટ સૂચના ઓપકોડ અને એડ્રેસિંગ મોડ સમજે.
    \item \textbf{એક્ઝેક્યુશન}: ALU ગાણિતિક/લોજિક કાર્ય કરે, રજિસ્ટર/મેમરી વચ્ચે ડેટા ફેરફાર થાય.
\end{itemize}
\end{solutionbox}
\begin{mnemonicbox}
``FDE'' (Fetch-Decode-Execute)
\end{mnemonicbox}

\questionmarks{2(c)}{7}{આકૃતિની મદદથી 8085 માઇક્રોપ્રોસેસરના બ્લોક ડાયાગ્રામનું વર્ણન કરો.}

\begin{solutionbox}
\textbf{જવાબ}:

\textbf{ડાયાગ્રામ:}

\begin{center}
\begin{tikzpicture}[node distance=2.5cm, auto, scale=0.75, transform shape]
    % Main Blocks
    \node [gtu block, minimum width=3cm] (alu) {ALU (8-bit)};
    \node [gtu block, left of=alu, node distance=4cm, minimum width=2.5cm] (temp) {Accumulator (A)\\Temp Reg};
    \node [gtu block, right of=alu, node distance=4cm, minimum width=2.5cm] (flags) {Flag Register};
    
    \node [gtu block, above of=temp, node distance=3cm, minimum width=4cm] (regs) {Register Array\\(B, C, D, E, H, L)};
    \node [gtu block, right of=regs, node distance=5cm, minimum width=3cm] (pc) {PC (16)\\SP (16)};
    
    \node [gtu block, below of=alu, node distance=3cm, minimum width=5cm] (control) {Timing \& Control Unit};
    \node [gtu block, left of=control, node distance=5cm] (ir) {Instruction\\Register};
    \node [gtu block, above of=ir] (decoder) {Decoder};
    
    \node [gtu block, below of=control, node distance=2.5cm] (interrupt) {Interrupt Control};
    \node [gtu block, right of=control, node distance=5cm] (serial) {Serial I/O Control};
    
    % Buses
    \node [draw, dashed, fit=(alu) (control) (regs) (pc), inner sep=0.5cm] (cpu) {};
    
    % Connections
    \draw [gtu arrow, <->] (regs) -- (alu);
    \draw [gtu arrow, <->] (temp) -- (alu);
    \draw [gtu arrow, <->] (flags) -- (alu);
    \draw [gtu arrow] (decoder) -- (control);
    \draw [gtu arrow] (ir) -- (decoder);
    
    % External Interface
    \node [below of=pc, node distance=6cm] (buffers) {Address/Data Buffers};
    \draw [gtu arrow, <->] (pc) -- (buffers);
    \draw [gtu arrow, <->] (regs) -- (buffers);
    
\end{tikzpicture}
\end{center}

\begin{itemize}
    \item \textbf{ALU}: 8-બિટ ગાણિતિક અને લોજિકલ ઓપરેશન્સ કરે છે.
    \item \textbf{રજિસ્ટર એરે}: અસ્થાયી રજિસ્ટર્સ (B, C, D, E, H, L) અને SP, PC.
    \item \textbf{કંટ્રોલ યુનિટ}: ટાઇમિંગ અને કંટ્રોલ સિગ્નલ્સ (RD, WR, ALE) જનરેટ કરે છે.
    \item \textbf{ઈન્સ્ટ્રક્શન રજિસ્ટર}: ઓપકોડ ફેચ કરે છે અને ડિકોડ કરે છે.
    \item \textbf{ઇન્ટરપ્ટ કંટ્રોલ}: હાર્ડવેર ઇન્ટરપ્ટ (INTR, RST, TRAP) હેન્ડલ કરે છે.
\end{itemize}
\end{solutionbox}
\begin{mnemonicbox}
``RAID'' (Registers-ALU-Instructions-Decoders)
\end{mnemonicbox}

\orquestionmarks{2(a)}{3}{માઇક્રોપ્રોસેસર અને માઇક્રોકંટ્રોલરની સરખામણી કરો.}

\begin{solutionbox}
\textbf{જવાબ}:

\begin{center}
\captionof{table}{સરખામણી}
\begin{tabulary}{\linewidth}{|l|J|J|}
\hline
\textbf{લક્ષણ} & \textbf{માઇક્રોપ્રોસેસર} & \textbf{માઇક્રોકંટ્રોલર} \\ \hline
\textbf{ડિઝાઇન} & માત્ર CPU (બાહ્ય ઘટકો જરૂરી) & CPU + પેરિફેરલ્સ (SoC) \\ \hline
\textbf{મેમરી} & બાહ્ય RAM/ROM & આંતરિક RAM/ROM \\ \hline
\textbf{I/O પોર્ટ્સ} & મર્યાદિત/નહિવત & ઘણા બિલ્ટ-ઇન \\ \hline
\textbf{ઉપયોગ} & જનરલ કમ્પ્યુટિંગ & એમ્બેડેડ સિસ્ટમ \\ \hline
\textbf{કિંમત} & વધારે & ઓછી \\ \hline
\textbf{ઉદાહરણ} & 8085, Core i7 & 8051, AVR \\ \hline
\end{tabulary}
\end{center}
\end{solutionbox}
\begin{mnemonicbox}
``Micro-P Processes, Micro-C Controls''
\end{mnemonicbox}

\orquestionmarks{2(b)}{4}{8085 માઇક્રોપ્રોસેસર માટે એડ્રેસ અને ડેટા બસોનું ડી-મલ્ટીપ્લેક્સીંગ સમજાવો.}

\begin{solutionbox}
\textbf{જવાબ}:
A0-A7 અને D0-D7 મલ્ટિપ્લેક્સ્ડ (AD0-AD7) છે. મેમરી સાથે વાત કરવા માટે તેમને અલગ કરવા પડે.

\textbf{સ્ટેપ્સ:}
\begin{enumerate}
    \item \textbf{ALE High}: માઇક્રોપ્રોસેસર ALE=1 મોકલે. AD0-AD7 પર એડ્રેસ હોય.
    \item \textbf{Latch}: એક્સટર્નલ લેચ (74LS373) એડ્રેસ પકડે.
    \item \textbf{ALE Low}: ALE=0 થાય. લેચ એડ્રેસ હોલ્ડ કરે. AD0-AD7 હવે ડેટા (D0-D7) માટે ફ્રી છે.
\end{enumerate}

\textbf{ડાયાગ્રામ:}

\begin{center}
\begin{tikzpicture}[auto, node distance=3cm]
    \node [draw, minimum height=3cm] (mp) {8085};
    \node [left] at (mp.west) {}; 
    
    \node [gtu block, right of=mp, node distance=4cm] (latch) {Latch\\(74LS373)};
    \node [right of=latch, node distance=3cm] (mem) {Memory};
    
    % AD Bus
    \draw [thick] (mp.east) -- (latch.west) node[midway, above] {AD0-AD7};
    
    % Address Out
    \draw [->, thick] (latch.east) -- (mem.west) node[midway, above] {A0-A7};
    
    % Data Bus Tapping
    \draw [thick] ($(mp.east)!0.5!(latch.west)$) -- ++(0,-1.5) node[right] {D0-D7 (Data Bus)};
    
    % ALE
    \draw [->, dashed] ($(mp.east)+(0,1)$) -- ++(1,0) -| (latch.north);
    \node at ($(mp.east)+(0.5,1.2)$) {\footnotesize ALE};
    
    % High Address
    \draw [->, thick] ($(mp.east)+(0,1.5)$) -- ++(6.5,0) |- ($(mem.west)+(0,0.5)$);
    \node at ($(mp.east)+(1,1.7)$) {\footnotesize A8-A15};
\end{tikzpicture}
\end{center}
\end{solutionbox}
\begin{mnemonicbox}
``ALAD'' (ALE-Latches-Address-Data)
\end{mnemonicbox}

\orquestionmarks{2(c)}{7}{આકૃતિની મદદથી 8085 માઇક્રોપ્રોસેસરના પિન ડાયાગ્રામનું વર્ણન કરો.}

\begin{solutionbox}
\textbf{જવાબ}:

\begin{center}
\begin{tikzpicture}[scale=0.65, transform shape]
    \draw [thick] (0,0) rectangle (6,12);
    \node at (3,6) {\Large \textbf{8085}};
    \node at (3,11.5) {\small 40-Pin DIP};
    
    % Left Pins (1-20)
    \foreach \y/\label/\pin in {
        11/X1/1, 10.5/X2/2, 10/RESET OUT/3, 9.5/SOD/4, 9/SID/5, 
        8.5/TRAP/6, 8/RST7.5/7, 7.5/RST6.5/8, 7/RST5.5/9, 6.5/INTR/10, 
        6/INTA/11, 5.5/AD0/12, 5/AD1/13, 4.5/AD2/14, 4/AD3/15, 
        3.5/AD4/16, 3/AD5/17, 2.5/AD6/18, 2/AD7/19, 1.5/Vss/20} {
        \draw (-0.5, \y) -- (0, \y);
        \node [left] at (-0.5, \y) {\tiny \label};
        \node [right] at (0, \y) {\tiny \pin};
    }
    
    % Right Pins (21-40)
    \foreach \y/\label/\pin in {
        1.5/A8/21, 2/A9/22, 2.5/A10/23, 3/A11/24, 3.5/A12/25, 
        4/A13/26, 4.5/A14/27, 5/A15/28, 5.5/S0/29, 6/ALE/30, 
        6.5/WR/31, 7/RD/32, 7.5/S1/33, 8/IO/M/34, 8.5/READY/35, 
        9/RESET IN/36, 9.5/CLK/37, 10/HLDA/38, 10.5/HOLD/39, 11/Vcc/40} {
        \draw (6, \y) -- (6.5, \y);
        \node [right] at (6.5, \y) {\tiny \label};
        \node [left] at (6, \y) {\tiny \pin};
    }
\end{tikzpicture}
\end{center}

\begin{itemize}
    \item \textbf{એડ્રેસ/ડેટા}: AD0-AD7 (મલ્ટિપ્લેક્સ્ડ), A8-A15 (હાઈ એડ્રેસ).
    \item \textbf{કંટ્રોલ}: ALE, RD, WR, IO/M, S0, S1.
    \item \textbf{ઇન્ટરપ્ટ}: INTR, INTA, RST 5.5-7.5, TRAP.
    \item \textbf{સીરિયલ I/O}: SID (ઇનપુટ), SOD (આઉટપુટ).
    \item \textbf{પાવર/ક્લોક}: Vcc (+5V), Vss (GND), X1, X2.
\end{itemize}
\end{solutionbox}
\begin{mnemonicbox}
``ACID-PS'' (Address-Control-Interrupt-DMA-Power-Serial)
\end{mnemonicbox}

\questionmarks{3(a)}{3}{8051 માઇક્રોકંટ્રોલરનાં ઇંટરપ્ટ્સ સમજાવો.}

\begin{solutionbox}
\textbf{જવાબ}:

\begin{center}
\captionof{table}{8051 ઇન્ટરપ્ટ્સ}
\begin{tabulary}{\linewidth}{|l|l|l|J|}
\hline
\textbf{ઇન્ટરપ્ટ} & \textbf{વેક્ટર} & \textbf{પ્રાયોરિટી} & \textbf{સ્ત્રોત} \\ \hline
\textbf{External 0} & 0003H & 1 (હાઈ) & Pin INT0 (P3.2) \\ \hline
\textbf{Timer 0} & 000BH & 2 & TF0 \\ \hline
\textbf{External 1} & 0013H & 3 & Pin INT1 (P3.3) \\ \hline
\textbf{Timer 1} & 001BH & 4 & TF1 \\ \hline
\textbf{Serial} & 0023H & 5 (લો) & RI or TI \\ \hline
\end{tabulary}
\end{center}

\textbf{ડાયાગ્રામ:}

\begin{center}
\begin{tikzpicture}[node distance=2cm, auto]
    \node [gtu block] (cpu) {8051 CPU};
    \node [left of=cpu, node distance=3.5cm, yshift=1cm] (int0) {INT0};
    \node [left of=cpu, node distance=3.5cm, yshift=0.5cm] (int1) {INT1};
    \node [left of=cpu, node distance=3.5cm, yshift=0cm] (t0) {Timer 0};
    \node [left of=cpu, node distance=3.5cm, yshift=-0.5cm] (t1) {Timer 1};
    \node [left of=cpu, node distance=3.5cm, yshift=-1cm] (serial) {Serial};
    
    \draw [gtu arrow] (int0) -- (cpu);
    \draw [gtu arrow] (int1) -- (cpu);
    \draw [gtu arrow] (t0) -- (cpu);
    \draw [gtu arrow] (t1) -- (cpu);
    \draw [gtu arrow] (serial) -- (cpu);
\end{tikzpicture}
\end{center}
\end{solutionbox}
\begin{mnemonicbox}
``ETTES'' (External-Timer-Timer-External-Serial)
\end{mnemonicbox}

\questionmarks{3(b)}{4}{8051 માઇક્રોકંટ્રોલરનો પિન ડાયાગ્રામ દોરો.}

\begin{solutionbox}
\textbf{જવાબ}:

\begin{center}
\begin{tikzpicture}[scale=0.65, transform shape]
    \draw [thick] (0,0) rectangle (6,12);
    \node at (3,6) {\Large \textbf{8051}};
    \node at (3,11.5) {\small 40-Pin DIP};
    
    % Left Pins
    \foreach \y/\label/\pin in {
        11/P1.0/1, 10.5/P1.1/2, 10/P1.2/3, 9.5/P1.3/4, 9/P1.4/5, 
        8.5/P1.5/6, 8/P1.6/7, 7.5/P1.7/8, 7/RST/9, 6.5/P3.0 (RXD)/10, 
        6/P3.1 (TXD)/11, 5.5/P3.2 (INT0)/12, 5/P3.3 (INT1)/13, 
        4.5/P3.4 (T0)/14, 4/P3.5 (T1)/15, 3.5/P3.6 (WR)/16, 
        3/P3.7 (RD)/17, 2.5/XTAL2/18, 2/XTAL1/19, 1.5/Vss/20} {
        \draw (-0.5, \y) -- (0, \y);
        \node [left] at (-0.5, \y) {\tiny \label};
        \node [right] at (0, \y) {\tiny \pin};
    }
    
    % Right Pins
    \foreach \y/\label/\pin in {
        1.5/P2.0/21, 2/P2.1/22, 2.5/P2.2/23, 3/P2.3/24, 3.5/P2.4/25, 
        4/P2.5/26, 4.5/P2.6/27, 5/P2.7/28, 5.5/PSEN/29, 6/ALE/30, 
        6.5/EA/31, 7/P0.7/32, 7.5/P0.6/33, 8/P0.5/34, 8.5/P0.4/35, 
        9/P0.3/36, 9.5/P0.2/37, 10/P0.1/38, 10.5/P0.0/39, 11/Vcc/40} {
        \draw (6, \y) -- (6.5, \y);
        \node [right] at (6.5, \y) {\tiny \label};
        \node [left] at (6, \y) {\tiny \pin};
    }
\end{tikzpicture}
\end{center}

\begin{itemize}
    \item \textbf{P0-P3}: 4 I/O પોર્ટ્સ. પોર્ટ 0 એડ્રેસ/ડેટા પણ છે.
    \item \textbf{Power}: Vcc, Vss.
    \item \textbf{Control}: RST, ALE, PSEN, EA.
\end{itemize}
\end{solutionbox}
\begin{mnemonicbox}
``PORT 0123''
\end{mnemonicbox}

\questionmarks{3(c)}{7}{8051 માઇક્રોકંટ્રોલરનું આંતરિક રેમ ઓર્ગેનાઇઝેશન સમજાવો.}

\begin{solutionbox}
\textbf{જવાબ}:
128 બાઇટ્સ આંતરિક RAM (00H થી 7FH).

\begin{center}
\begin{tikzpicture}[node distance=0cm, auto]
    \node [draw, minimum width=4cm, minimum height=1cm, fill=black!5, align=center] (bank0) {Register Bank 0\\(R0-R7)};
    \node [right] at (bank0.east) {00H-07H};
    
    \node [draw, minimum width=4cm, minimum height=1cm, below=of bank0, align=center] (bank1) {Register Bank 1\\(R0-R7)};
    \node [right] at (bank1.east) {08H-0FH};
    
    \node [draw, minimum width=4cm, minimum height=1cm, below=of bank1, align=center] (bank2) {Register Bank 2\\(R0-R7)};
    \node [right] at (bank2.east) {10H-17H};
    
    \node [draw, minimum width=4cm, minimum height=1cm, below=of bank2, align=center] (bank3) {Register Bank 3\\(R0-R7)};
    \node [right] at (bank3.east) {18H-1FH};
    
    \node [draw, minimum width=4cm, minimum height=1.5cm, below=of bank3, fill=black!10, align=center] (bit) {Bit Addressable Area\\(16 Bytes)};
    \node [right] at (bit.east) {20H-2FH};
    
    \node [draw, minimum width=4cm, minimum height=2cm, below=of bit, align=center] (scratch) {General Purpose RAM\\(Scratch Pad)};
    \node [right] at (scratch.east) {30H-7FH};
    
    % SFRs
    \node [draw, minimum width=4cm, minimum height=1.5cm, below=of scratch, yshift=-0.5cm, dashed, align=center] (sfr) {Special Function Registers\\(SFRs)};
    \node [right] at (sfr.east) {80H-FFH};
    
\end{tikzpicture}
\end{center}

\begin{itemize}
    \item \textbf{રજિસ્ટર બેન્ક્સ}: 4 બેન્ક્સ (R0-R7).
    \item \textbf{બિટ એડ્રેસેબલ}: 16 બાઇટ્સ બિટ તરીકે એક્સેસ કરી શકાય.
    \item \textbf{જનરલ પર્પઝ}: યુઝર ડેટા માટે 80 બાઇટ્સ.
    \item \textbf{SFRs}: અપર 128 બાઇટ્સ રેન્જમાં મેપ થયેલ.
\end{itemize}
\end{solutionbox}
\begin{mnemonicbox}
``RBBS'' (Registers-Bits-Buffer-Special)
\end{mnemonicbox}

\orquestionmarks{3(a)}{3}{SFRs ને તેમના એડ્રેસ સાથે સૂચિબદ્ધ કરો.}

\begin{solutionbox}
\textbf{જવાબ}:

\begin{center}
\captionof{table}{SFRs}
\begin{tabulary}{\linewidth}{|l|l|J|}
\hline
\textbf{SFR} & \textbf{એડ્રેસ} & \textbf{કાર્ય} \\ \hline
\textbf{P0} & 80H & પોર્ટ 0 \\ \hline
\textbf{SP} & 81H & સ્ટેક પોઇન્ટર \\ \hline
\textbf{DPH:DPL} & 83H:82H & ડેટા પોઇન્ટર \\ \hline
\textbf{TCON} & 88H & ટાઇમર કંટ્રોલ \\ \hline
\textbf{TMOD} & 89H & ટાઇમર મોડ \\ \hline
\textbf{P1} & 90H & પોર્ટ 1 \\ \hline
\textbf{SCON} & 98H & સીરિયલ કંટ્રોલ \\ \hline
\textbf{P2} & A0H & પોર્ટ 2 \\ \hline
\textbf{IE} & A8H & ઇન્ટરપ્ટ એનેબલ \\ \hline
\textbf{P3} & B0H & પોર્ટ 3 \\ \hline
\textbf{IP} & B8H & ઇન્ટરપ્ટ પ્રાયોરિટી \\ \hline
\textbf{PSW} & D0H & સ્ટેટસ વર્ડ \\ \hline
\textbf{ACC (A)} & E0H & એક્યુમુલેટર \\ \hline
\textbf{B} & F0H & B રજિસ્ટર \\ \hline
\end{tabulary}
\end{center}
\end{solutionbox}
\begin{mnemonicbox}
``PDPT-SP'' (Ports-Data-Program-Timers-Serial-Prioritized)
\end{mnemonicbox}

\orquestionmarks{3(b)}{4}{8051 માઇક્રોકંટ્રોલરના ટાઇમર/કાઉન્ટર્સનો લોજિક ડાયાગ્રામ સમજાવો.}

\begin{solutionbox}
\textbf{જવાબ}:

\textbf{ડાયાગ્રામ:}

\begin{center}
\begin{tikzpicture}[auto, node distance=2.5cm]
    \node [draw, minimum width=2cm] (tl) {TLx (8-bit)};
    \node [draw, minimum width=2cm, right of=tl] (th) {THx (8-bit)};
    \draw [->] (tl) -- (th);
    \draw [->] (th) -- ++(1.5,0) node[right] {TFx};
    
    \node [draw, below of=tl, node distance=2cm] (control) {Control Logic};
    \draw [->] (control) -- (tl);
    
    \node [left of=control, node distance=3cm] (osc) {Osc/12};
    \node [below of=osc, node distance=1.5cm] (pin) {Tx Pin};
    
    \node [gtu block, right of=control, node distance=3cm] (mux) {C/T Mux};
    
    \draw [->] (osc) -| (mux.160);
    \draw [->] (pin) -| (mux.200);
    \draw [->] (mux) -- (control);
    
    \node [below of=mux, node distance=1cm] (gate) {GATE / TRx};
    \draw [->, dashed] (gate) -- (control);
    
\end{tikzpicture}
\end{center}

\begin{itemize}
    \item \textbf{રજિસ્ટર્સ}: TLx, THx (16-બિટ).
    \item \textbf{C/T}: ટાઇમર (0) અથવા કાઉન્ટર (1) પસંદ કરે.
    \item \textbf{કંટ્રોલ}: TRx અને GATE સ્ટાર્ટ/સ્ટોપ નિયંત્રિત કરે.
    \item \textbf{ઓવરફ્લો}: TFx ફ્લેગ સેટ થાય.
\end{itemize}
\end{solutionbox}
\begin{mnemonicbox}
``TCG'' (Timer-Counter-Gate)
\end{mnemonicbox}

\orquestionmarks{3(c)}{7}{8051 માઇક્રોકંટ્રોલરનો બ્લોક ડાયાગ્રામ સમજાવો.}

\begin{solutionbox}
\textbf{જવાબ}:

\textbf{ડાયાગ્રામ:}

\begin{center}
\begin{tikzpicture}[node distance=2.5cm, auto, scale=0.8, transform shape]
    \node [gtu block] (cpu) {8-Bit CPU};
    
    \node [gtu block, right of=cpu, node distance=3.5cm] (osc) {Oscillator};
    \node [gtu block, below of=cpu, node distance=2.5cm] (int) {Interrupt\\Control};
    \node [gtu block, right of=int, node distance=3.5cm] (bus) {Bus Control};
    
    \node [gtu block, left of=cpu, node distance=3.5cm] (rom) {4KB ROM};
    \node [gtu block, below of=rom, node distance=2.5cm] (ram) {128B RAM};
    
    \node [gtu block, right of=osc, node distance=3.5cm] (timers) {Timer 0\\Timer 1};
    \node [gtu block, below of=timers, node distance=2.5cm] (serial) {Serial Port\\(UART)};
    
    \node [gtu block, below of=int, node distance=3cm, minimum width=8cm] (ports) {I/O Ports (P0, P1, P2, P3)};
    
    % Connections
    \draw [gtu arrow, <->] (cpu) -- (rom);
    \draw [gtu arrow, <->] (cpu) -- (ram);
    \draw [gtu arrow] (osc) -- (cpu);
    \draw [gtu arrow] (int) -- (cpu);
    \draw [gtu arrow, <->] (cpu) -- (bus);
    
    \draw [gtu arrow, <->] (bus) -- (ports);
    \draw [gtu arrow, <->] (timers) -- (bus);
    \draw [gtu arrow, <->] (serial) -- (bus);
    
    \node [draw, dashed, fit=(cpu) (ports) (ram) (serial), inner sep=0.5cm] {};
    \node [above] at (current bounding box.north) {Internal System Bus};
\end{tikzpicture}
\end{center}

\begin{itemize}
    \item \textbf{CPU}: સિસ્ટમનું મગજ.
    \item \textbf{મેમરી}: 4KB ROM + 128B RAM.
    \item \textbf{I/O}: 4 પોર્ટ્સ.
    \item \textbf{ટાઇમર્સ}: 2 16-બિટ ટાઇમર્સ.
    \item \textbf{સીરિયલ}: કમ્યુનિકેશન માટે.
\end{itemize}
\end{solutionbox}
\begin{mnemonicbox}
``CAPITALS'' (CPU-Architecture-Ports-I/O-Timer-ALU-LS-Interface-Serial)
\end{mnemonicbox}

\questionmarks{4(a)}{3}{ડેટાના બે બાઇટ ઉમેરીને પરિણામ R4 રજિસ્ટરમાં સંગ્રહિત કરવા માટે 8051 એસેમ્બલી લેંગ્વેજ પ્રોગ્રામ લખો.}

\begin{solutionbox}
\textbf{જવાબ}:

\begin{lstlisting}[language={[x86masm]Assembler}]
MOV A, #25H       ; પ્રથમ મૂલ્ય લોડ કરો
MOV R3, #18H      ; બીજું મૂલ્ય R3માં લોડ કરો
ADD A, R3         ; સમેશન
MOV R4, A         ; પરિણામ R4માં સ્ટોર કરો
\end{lstlisting}
\end{solutionbox}
\begin{mnemonicbox}
``LLAS'' (Load-Load-Add-Store)
\end{mnemonicbox}

\questionmarks{4(b)}{4}{પોર્ટ-1 અને પોર્ટ-2ના કન્ટેન્ટને OR કરીને પછી પરિણામને બાહ્ય RAM સ્થાન 0200H માં મૂકવા માટે 8051 એસેમ્બલી લેંગ્વેજ પ્રોગ્રામ લખો.}

\begin{solutionbox}
\textbf{જવાબ}:

\begin{lstlisting}[language={[x86masm]Assembler}]
MOV A, P1         ; પોર્ટ 1 વાંચો
ORL A, P2         ; OR કરો પોર્ટ 2 સાથે
MOV DPTR, #0200H  ; એડ્રેસ સેટ કરો
MOVX @DPTR, A     ; બાહ્ય મેમરીમાં લખો
\end{lstlisting}
\end{solutionbox}
\begin{mnemonicbox}
``PORT'' (Port-OR-Register-Transfer)
\end{mnemonicbox}

\questionmarks{4(c)}{7}{8051 માઇક્રોકંટ્રોલરના એડ્રેસિંગ મોડ્સની યાદી બનાવો અને ઓછામાં ઓછા એક ઉદાહરણ સાથે તેમને સમજાવો.}

\begin{solutionbox}
\textbf{જવાબ}:

\begin{center}
\captionof{table}{એડ્રેસિંગ મોડ્સ}
\begin{tabulary}{\linewidth}{|l|l|J|}
\hline
\textbf{મોડ} & \textbf{ઉદાહરણ} & \textbf{વર્ણન} \\ \hline
\textbf{ઇમીડિયેટ} & \code{MOV A, \#25H} & ડેટા ઇન્સ્ટ્રક્શનમાં જ છે. \\ \hline
\textbf{રજિસ્ટર} & \code{MOV A, R0} & ડેટા રજિસ્ટરમાં છે. \\ \hline
\textbf{ડાયરેક્ટ} & \code{MOV A, 30H} & એડ્રેસ આપેલ છે. \\ \hline
\textbf{ઇનડાયરેક્ટ} & \code{MOV A, @R0} & એડ્રેસ રજિસ્ટરમાં છે (@). \\ \hline
\textbf{ઇન્ડેક્સ્ડ} & \code{MOVC A, @A+DPTR} & કોડ મેમરી એક્સેસ. \\ \hline
\textbf{બિટ} & \code{SETB P1.3} & સિંગલ બિટ ઓપરેશન. \\ \hline
\textbf{રિલેટિવ} & \code{SJMP LABEL} & ઓફસેટ જમ્પ. \\ \hline
\end{tabulary}
\end{center}
\end{solutionbox}
\begin{mnemonicbox}
``I'M DIRBI''
\end{mnemonicbox}

\orquestionmarks{4(a)}{3}{નીચેની ઈન્સ્ટ્રક્શન્સ સમજાવો: (i) DJNZ (ii) POP (iii) CJNE.}

\begin{solutionbox}
\textbf{જવાબ}:

\begin{itemize}
    \item \textbf{DJNZ (Decrement and Jump if Not Zero)}: રજિસ્ટર ઘટાડે અને 0 ન હોય તો જમ્પ કરે. લૂપ માટે વપરાય.
    \item \textbf{POP}: સ્ટેકમાંથી ડેટા પોપ કરે.
    \item \textbf{CJNE (Compare and Jump if Not Equal)}: એ બે વેલ્યુ સરખાવે અને અસમાન હોય તો જમ્પ કરે.
\end{itemize}
\end{solutionbox}
\begin{mnemonicbox}
``DPC''
\end{mnemonicbox}

\orquestionmarks{4(b)}{4}{12 MHz ની ક્રિસ્ટલ ફ્રિકવન્સી સાથે 8051 માઇક્રોકંટ્રોલર માટે, 4ms નો ડિલેય જનરેટ કરો.}

\begin{solutionbox}
\textbf{જવાબ}:
1 સાયકલ = 1 $\mu s$ (12MHz / 12). 4ms = 4000 સાયકલ.

\begin{lstlisting}[language={[x86masm]Assembler}]
      MOV R7, #08       ; Outer Loop
DELAY_LOOP:
      MOV R6, #250      ; Inner Loop (500us)
      DJNZ R6, $        ; 2 cycles per loop
      DJNZ R7, DELAY_LOOP ; 8 * 500 = 4000
      RET
\end{lstlisting}
\end{solutionbox}
\begin{mnemonicbox}
``LNDD''
\end{mnemonicbox}

\orquestionmarks{4(c)}{7}{8051 માઇક્રોકંટ્રોલર માટે કોઈપણ સાત લોજીકલ ઈન્સ્ટ્રક્શન ઉદાહરણ સાથે સમજાવો.}

\begin{solutionbox}
\textbf{જવાબ}:

\begin{center}
\captionof{table}{લોજીકલ ઈન્સ્ટ્રક્શન}
\begin{tabulary}{\linewidth}{|l|l|J|}
\hline
\textbf{ઇન્સ્ટ્રક્શન} & \textbf{ઉદાહરણ} & \textbf{ઓપરેશન} \\ \hline
\textbf{ANL} & \code{ANL A, \#0FH} & Logical AND. \\ \hline
\textbf{ORL} & \code{ORL P1, \#80H} & Logical OR. \\ \hline
\textbf{XRL} & \code{XRL A, R0} & Logical XOR. \\ \hline
\textbf{CLR} & \code{CLR A} & Clear. \\ \hline
\textbf{CPL} & \code{CPL A} & Complement. \\ \hline
\textbf{RL} & \code{RL A} & Rotate Left. \\ \hline
\textbf{RR} & \code{RR A} & Rotate Right. \\ \hline
\end{tabulary}
\end{center}
\end{solutionbox}
\begin{mnemonicbox}
``A-OX-CCR''
\end{mnemonicbox}

\questionmarks{5(a)}{3}{વિવિધ ક્ષેત્રોમાં માઇક્રોકંટ્રોલરની એપ્લિકેશનોની સૂચિ બનાવો.}

\begin{solutionbox}
\textbf{જવાબ}:
\begin{itemize}
    \item ઔદ્યોગિક (ઓટોમેશન)
    \item મેડિકલ (મોનિટરિંગ)
    \item કન્ઝ્યુમર (રમકડાં)
    \item ઓટોમોટિવ (ABS)
    \item કમ્યુનિકેશન (મોબાઇલ)
    \item સિક્યુરિટી (CCTV)
\end{itemize}
\end{solutionbox}
\begin{mnemonicbox}
``I-MACS''
\end{mnemonicbox}

\questionmarks{5(b)}{4}{8051 માઇક્રોકંટ્રોલર સાથે પુશ બટન સ્વિચ અને LED ઇન્ટરફેસ કરો.}

\begin{solutionbox}
\textbf{જવાબ}:

\textbf{ડાયાગ્રામ:}

\begin{center}
\begin{tikzpicture}[auto, node distance=2.5cm]
    \node [gtu block, minimum height=3cm] (mcu) {8051};
    
    % Switch
    \node [left of=mcu, yshift=1cm, node distance=2cm] (sw_pin) {};
    \draw (mcu.west |- sw_pin) -- ++(-1,0) node[circ] (node1) {} -- ++(0, 0.5) node[above] {Vcc} node[midway, right] {10k};
    \draw (node1) -- ++(0,-0.5) -- ++(-0.5,0) node[draw, rectangle] (sw) {SW} -- ++(0,-0.5) node[ground] {};
    \node [right] at (mcu.west |- sw_pin) {P1.0};
    
    % LED
    \node [left of=mcu, yshift=-1cm, node distance=2cm] (led_pin) {};
    \draw (mcu.west |- led_pin) -- ++(-1,0) to[R, l=330$\Omega$] ++(-1.5,0) node[circ] {} to[nativeLED] ++(-1,0) node[ground] {};
    \node [right] at (mcu.west |- led_pin) {P1.7};
    
\end{tikzpicture}
\end{center}

\textbf{પ્રોગ્રામ:}
\begin{lstlisting}[language={[x86masm]Assembler}]
AGAIN:
    JB P1.0, LED_OFF  ; જો હાઈ (ના દબાવેલ), જમ્પ
    SETB P1.7         ; જો લો (દબાવેલ), LED ચાલુ
    SJMP AGAIN
LED_OFF:
    CLR P1.7          ; LED બંધ
    SJMP AGAIN
\end{lstlisting}
\end{solutionbox}
\begin{mnemonicbox}
``PLIC''
\end{mnemonicbox}

\questionmarks{5(c)}{7}{માઇક્રોકંટ્રોલર સાથે LCD ઇન્ટરફેસ કરો અને "HELLO" દર્શાવવા માટે પ્રોગ્રામ લખો.}

\begin{solutionbox}
\textbf{જવાબ}:

\textbf{ડાયાગ્રામ:}

\begin{center}
\begin{tikzpicture}[auto, node distance=3cm]
    \node [gtu block, minimum height=3cm] (mcu) {8051};
    \node [gtu block, right of=mcu, node distance=5cm, minimum height=3cm] (lcd) {16x2 LCD};
    
    \draw [->, thick] (mcu.east) -- (lcd.west) node[midway, above] {Data (P2)};
    \draw [->] ($(mcu.east)+(0,-1)$) -- ($(lcd.west)+(0,-1)$) node[midway, above] {Ctrl (P3)};
\end{tikzpicture}
\end{center}

\textbf{પ્રોગ્રામ:} LCD (38H) ઇનિશિયલાઇઝ કરો, અક્ષર મોકલો.
\begin{lstlisting}[language={[x86masm]Assembler}]
    MOV A, #'H'
    ACALL DISP
    MOV A, #'E'
    ACALL DISP
    ; ... (L, L, O)
\end{lstlisting}
\end{solutionbox}
\begin{mnemonicbox}
``DICE''
\end{mnemonicbox}

\orquestionmarks{5(a)}{3}{8051 માઇક્રોકંટ્રોલર સાથે LM35 નું ઇન્ટરફેસિંગ દોરો.}

\begin{solutionbox}
\textbf{જવાબ}:

\begin{center}
\begin{tikzpicture}[auto, node distance=2.5cm]
    \node [gtu block] (mcu) {8051};
    \node [gtu block, left of=mcu, node distance=4cm] (adc) {ADC0804};
    \node [gtu block, left of=adc, node distance=3cm] (lm35) {LM35};
    
    \draw [->] (lm35) -- (adc);
    \draw [->, thick] (adc) -- (mcu);
\end{tikzpicture}
\end{center}
\end{solutionbox}
\begin{mnemonicbox}
``TAC''
\end{mnemonicbox}

\orquestionmarks{5(b)}{4}{8051 માઇક્રોકંટ્રોલર સાથે સ્ટેપર મોટર ઇન્ટરફેસ કરો.}

\begin{solutionbox}
\textbf{જવાબ}:

\textbf{ડાયાગ્રામ:}

\begin{center}
\begin{tikzpicture}[auto, node distance=3cm]
    \node [gtu block] (mcu) {8051};
    \node [gtu block, right of=mcu] (driver) {ULN2003};
    \node [gtu block, right of=driver] (motor) {Stepper};
    
    \draw [->, thick] (mcu) -- (driver);
    \draw [->, thick] (driver) -- (motor);
\end{tikzpicture}
\end{center}
\end{solutionbox}
\begin{mnemonicbox}
``PDCS''
\end{mnemonicbox}

\orquestionmarks{5(c)}{7}{8051 માઇક્રોકંટ્રોલર સાથે ADC0804 ઇન્ટરફેસ કરો.}

\begin{solutionbox}
\textbf{જવાબ}:

\textbf{ડાયાગ્રામ:}

\begin{center}
\begin{tikzpicture}[auto, node distance=4cm]
    \node [draw, minimum height=4cm, minimum width=2.5cm] (adc) {ADC0804};
    \node [draw, minimum height=4cm, minimum width=2.5cm, right of=adc] (mcu) {8051};
    
    \draw [->, thick] (adc.20) -- (mcu.160) node[midway, above] {Data};
    \draw [<-] (adc.-20) -- (mcu.-160) node[midway, below] {Ctrl (CS, WR, RD)};
    \draw [->] (adc.east) -- (mcu.west) node[midway, above] {INTR};
    
    \node [left] at (adc.west) {Vin};
    \draw [<-] (adc.west) -- ++(-1,0); 
\end{tikzpicture}
\end{center}
\end{solutionbox}
\begin{mnemonicbox}
``CRIW''
\end{mnemonicbox}

\end{document}
