\documentclass[10pt,a4paper]{article}

% content/resources/templates/preamble.tex
\usepackage[margin=0.6in]{geometry}
\author{Milav Dabgar}
\usepackage{amsmath,amssymb,amsthm}
\usepackage{booktabs}
\usepackage{multirow}
\usepackage{xcolor}
\usepackage{tcolorbox}
\tcbuselibrary{breakable,skins}
\usepackage[colorlinks=true,linkcolor=blue]{hyperref}
\usepackage{titlesec}
\usepackage{enumitem}
\usepackage{tikz}
\usepackage{pgfplots}
\usepackage{circuitikz}
\usepackage[version=4]{mhchem}
\usepackage{longtable}
\usepackage{array}
\usepackage{float}
\usepackage{caption}
\usepackage{listings}

\lstset{
  basicstyle=\small\ttfamily,
  breaklines=true,
  breakatwhitespace=false,
  postbreak=\mbox{\textcolor{red}{$\hookrightarrow$}\space},
  float=false,
  numbers=left,
  numberstyle=\tiny\color{gray},
  numbersep=10pt,
  xleftmargin=2em,
  keywordstyle=\color{blue},
  commentstyle=\color{green!60!black},
  stringstyle=\color{purple},
  backgroundcolor=\color{gray!5},
  showstringspaces=false,
  tabsize=2,
  captionpos=b,
  keepspaces=true,
  columns=flexible
}

\pgfplotsset{compat=1.18}
\usetikzlibrary{shapes,arrows,positioning,calc,patterns,decorations.pathmorphing,decorations.markings,arrows.meta}

% Color scheme
\definecolor{headcolor}{RGB}{0,102,204}
\definecolor{keycolor}{RGB}{220,20,60}
\definecolor{solutioncolor}{RGB}{34,139,34}
\definecolor{mnemoniccolor}{RGB}{148,0,211}
\definecolor{codecolor}{RGB}{0,0,100}

% Spacing
\setlength{\parskip}{3pt}
\setlist[itemize]{nosep}
\setlist[enumerate]{nosep}

% Title formatting
\titleformat{\section}{\Large\bfseries\color{headcolor}}{\thesection}{1em}{}
\titleformat{\subsection}{\large\bfseries\color{headcolor}}{\thesubsection}{1em}{}

% Pandoc tightlist compatibility
\providecommand{\tightlist}{%
  \setlength{\itemsep}{0pt}\setlength{\parskip}{0pt}}

% Pandoc longtable compatibility
\newcounter{none}
\def\thenone{}


% content/resources/templates/gujarati-boxes.tex
\usepackage{fontspec}
\usepackage{polyglossia}

% Set Gujarati as main language (document is primarily in Gujarati)
% Note: gloss-gujarati.ldf doesn't exist in polyglossia, but it will use hyphenation patterns
\setdefaultlanguage{gujarati}
\setotherlanguage{english}

% Configure Gujarati font properly
% Use Language=Default to prevent polyglossia from trying to add language-specific features
% that don't exist for Gujarati, which causes "empty feature" warnings
\newfontfamily\gujaratifont[Script=Gujarati,AutoFakeBold=2.5,AutoFakeSlant=0.3]{Noto Sans Gujarati}
\setmainfont[Script=Gujarati,AutoFakeBold=2.5,AutoFakeSlant=0.3]{Noto Sans Gujarati}
% Use Noto Sans Gujarati for monospace to support Gujarati in text
\setmonofont[Scale=0.9]{Noto Sans Gujarati}

% Configure English to use the same font
\newfontfamily\englishfont[Script=Gujarati,AutoFakeBold=2.5,AutoFakeSlant=0.3]{Noto Sans Gujarati}

% Translations for polyglossia
\gappto\captionsgujarati{
  \renewcommand{\tablename}{કોષ્ટક}
  \renewcommand{\figurename}{આકૃતિ}
}

% Helper for TikZ nodes to ensure Gujarati font
\newcommand{\gu}[1]{{\gujaratifont #1}}

% Custom environments
\newtcolorbox{solutionbox}{
    breakable,
    enhanced,
    colback=solutioncolor!5!white,
    colframe=solutioncolor!75!black,
    fonttitle=\bfseries,
    title=જવાબ
}

\newtcolorbox{solutionboxnobreak}{
 colback=solutioncolor!5!white,
 colframe=solutioncolor!75!black,
 fonttitle=\bfseries,
 title=જવાબ
}

\newtcolorbox{keyformula}{
 breakable,
 enhanced,
 colback=keycolor!5!white,
 colframe=keycolor!75!black,
 fonttitle=\bfseries,
 title=રાસાયણિક સમીકરણ/સૂત્ર
}

\newtcolorbox{mnemonicbox}{
 breakable,
 enhanced,
 colback=mnemoniccolor!5!white,
 colframe=mnemoniccolor!75!black,
 fonttitle=\bfseries,
 title=મેમરી ટ્રીક
}


\begin{document}

\begin{center}
{\Huge\bfseries\color{headcolor} Subject Name (Gujarati)}\\[5pt]
{\LARGE 4341101 -- Summer 2024}\\[3pt]
{\large Semester 1 Study Material}\\[3pt]
{\normalsize\textit{Detailed Solutions and Explanations}}
\end{center}

\vspace{10pt}

\subsection*{પ્રશ્ન 1(અ) [3
ગુણ]}\label{uxaaauxab0uxab6uxaa8-1uxa85-3-uxa97uxaa3}

\textbf{8051 માઇક્રોકન્ટ્રોલરના કોઈપણ એક પોર્ટ કન્ફિગરેશનનું વર્ણન કરો.}

\begin{solutionbox}

{\def\LTcaptype{none} % do not increment counter
\begin{longtable}[]{@{}
  >{\raggedright\arraybackslash}p{(\linewidth - 2\tabcolsep) * \real{0.5357}}
  >{\raggedright\arraybackslash}p{(\linewidth - 2\tabcolsep) * \real{0.4643}}@{}}
\toprule\noalign{}
\begin{minipage}[b]{\linewidth}\raggedright
કન્ફિગરેશન
\end{minipage} & \begin{minipage}[b]{\linewidth}\raggedright
વર્ણન
\end{minipage} \\
\midrule\noalign{}
\endhead
\bottomrule\noalign{}
\endlastfoot
\textbf{પોર્ટ 0} & ડ્યુઅલ-પર્પઝ પોર્ટ - 8-બિટ ઓપન ડ્રેન બિડાયરેક્શનલ I/O પોર્ટ અને
મલ્ટીપ્લેક્સ્ડ લો એડ્રેસ/ડેટા બસ. I/O ફંક્શન માટે બાહ્ય પુલ-અપ રેસિસ્ટર જરૂરી. \\
\end{longtable}
}

\textbf{ડાયાગ્રામ:}

\begin{verbatim}
                    8051
               +{-{-}{-}{-}{-}{-}{-}{-}{-}{-}{-}+}
               |           |
External       |           |    P0.0{-P0.7}
Pull{-ups       |           |    (AD0{-}AD7)}
  \_\_\_\_         |           |{{-}{-}{-}{-}{-}{-}{-}{-}{-}}
 |    |        |           |
 |    |{-{-}{-}{-}{-}{-}{-}|  PORT 0   |}
 |\_\_\_\_|        |           |
+5V            |           |
               +{-{-}{-}{-}{-}{-}{-}{-}{-}{-}{-}+}
\end{verbatim}

\end{solutionbox}
\begin{mnemonicbox}
``પોર્ટ 0-પ્લેડ'' (પોર્ટ 0 ને પુલ-અપ્સ જોઈએ, લેચ/એડ્રેસ/ડેટા
તરીકે કામ કરે)

\end{mnemonicbox}
\subsection*{પ્રશ્ન 1(બ) [4
ગુણ]}\label{uxaaauxab0uxab6uxaa8-1uxaac-4-uxa97uxaa3}

\textbf{માઇક્રોપ્રોસેસર આર્કિટેક્ચરનું વર્ણન કરો.}

\begin{solutionbox}

{\def\LTcaptype{none} % do not increment counter
\begin{longtable}[]{@{}ll@{}}
\toprule\noalign{}
ઘટક & કાર્ય \\
\midrule\noalign{}
\endhead
\bottomrule\noalign{}
\endlastfoot
\textbf{ALU} & ગાણિતિક અને લોજિકલ ઓપરેશન કરે છે \\
\textbf{રજિસ્ટર્સ} & ડેટા અને એડ્રેસ માટે કામચલાઉ સ્ટોરેજ \\
\textbf{કંટ્રોલ યુનિટ} & પ્રોસેસર ઓપરેશન અને ડેટા ફ્લો નિર્દેશિત કરે છે \\
\textbf{બસ} & ડેટા ટ્રાન્સફર માટે પાથવે (એડ્રેસ, ડેટા, કંટ્રોલ) \\
\end{longtable}
}

\textbf{ડાયાગ્રામ:}

\begin{verbatim}
     +{-{-}{-}{-}{-}{-}{-}{-}{-}{-}{-}{-}{-}{-}{-}{-}{-}{-}{-}{-}{-}{-}{-}{-}{-}{-}{-}{-}{-}{-}{-}{-}{-}{-}{-}{-}{-}{-}{-}{-}{-}{-}+}
     |          MICROPROCESSOR                  |
     |                                          |
     |  +{-{-}{-}{-}{-}{-}{-}{-}{-}{-}{-}{-}{-}+    +{-}{-}{-}{-}{-}{-}{-}{-}{-}{-}{-}{-}{-}{-}{-}{-}+   |}
     |  | REGISTERS   |    | CONTROL UNIT   |   |
     |  |{-{-}{-}{-}{-}{-}{-}{-}{-}{-}{-}{-}{-}|    |{-}{-}{-}{-}{-}{-}{-}{-}{-}{-}{-}{-}{-}{-}{-}{-}|   |}
     |  | A, B, C, D  |    | Instruction    |   |
     |  | H, L, SP, PC|{{-}{-}| Decoder        |   |}
     |  | Flags       |    | Timing \&       |   |
     |  +{-{-}{-}{-}{-}{-}\^{}{-}{-}{-}{-}{-}{-}+    | Control        |   |}
     |         |           +{-{-}{-}{-}{-}{-}{-}{-}\^{}{-}{-}{-}{-}{-}{-}{-}+   |}
     |         |                    |           |
     |         v                    |           |
     |  +{-{-}{-}{-}{-}{-}+{-}{-}{-}{-}{-}{-}+             |           |}
     |  |     ALU     |{{-}{-}{-}{-}{-}{-}{-}{-}{-}{-}{-}{-}+           |}
     |  +{-{-}{-}{-}{-}{-}\^{}{-}{-}{-}{-}{-}{-}+                         |}
     |         |                                |
     +{-{-}{-}{-}{-}{-}{-}{-}{-}|{-}{-}{-}{-}{-}{-}{-}{-}{-}{-}{-}{-}{-}{-}{-}{-}{-}{-}{-}{-}{-}{-}{-}{-}{-}{-}{-}{-}{-}{-}{-}{-}+}
               |
     +{-{-}{-}{-}{-}{-}{-}{-}{-}v{-}{-}{-}{-}{-}{-}{-}{-}{-}{-}{-}{-}{-}{-}{-}{-}{-}{-}{-}{-}{-}{-}{-}{-}{-}{-}{-}{-}{-}{-}{-}{-}+}
     |  ADDRESS, DATA \& CONTROL BUSES           |
     +{-{-}{-}{-}{-}{-}{-}{-}{-}{-}{-}{-}{-}{-}{-}{-}{-}{-}{-}{-}{-}{-}{-}{-}{-}{-}{-}{-}{-}{-}{-}{-}{-}{-}{-}{-}{-}{-}{-}{-}{-}{-}+}
\end{verbatim}

\end{solutionbox}
\begin{mnemonicbox}
``RABC'' - ``રજિસ્ટર, ALU, બસ, કંટ્રોલ''

\end{mnemonicbox}
\subsection*{પ્રશ્ન 1(ક) [7
ગુણ]}\label{uxaaauxab0uxab6uxaa8-1uxa95-7-uxa97uxaa3}

\textbf{વોન ન્યુમેન અને હાર્વર્ડ આર્કિટેક્ચરની તુલના કરો.}

\begin{solutionbox}

{\def\LTcaptype{none} % do not increment counter
\begin{longtable}[]{@{}
  >{\raggedright\arraybackslash}p{(\linewidth - 4\tabcolsep) * \real{0.1579}}
  >{\raggedright\arraybackslash}p{(\linewidth - 4\tabcolsep) * \real{0.4561}}
  >{\raggedright\arraybackslash}p{(\linewidth - 4\tabcolsep) * \real{0.3860}}@{}}
\toprule\noalign{}
\begin{minipage}[b]{\linewidth}\raggedright
ફીચર
\end{minipage} & \begin{minipage}[b]{\linewidth}\raggedright
વોન ન્યુમેન આર્કિટેક્ચર
\end{minipage} & \begin{minipage}[b]{\linewidth}\raggedright
હાર્વર્ડ આર્કિટેક્ચર
\end{minipage} \\
\midrule\noalign{}
\endhead
\bottomrule\noalign{}
\endlastfoot
મેમરી બસ & ઇન્સ્ટ્રક્શન અને ડેટા માટે એક જ મેમરી બસ & પ્રોગ્રામ અને ડેટા મેમરી માટે અલગ
બસ \\
એક્ઝિક્યુશન & સિક્વેન્શિયલ એક્ઝિક્યુશન & પેરેલલ ફેચ અને એક્ઝિક્યુટ શક્ય \\
સ્પીડ & બસ બોટલનેક ને કારણે ધીમું & સમાંતર એક્સેસને કારણે ઝડપી \\
મેમરી એક્સેસ & એક જ મેમરી સ્પેસ & અલગ મેમરી સ્પેસ \\
જટિલતા & સરળ ડિઝાઇન & વધુ જટિલ ડિઝાઇન \\
ઉપયોગો & સામાન્ય કમ્પ્યુટિંગ & DSP, માઇક્રોકન્ટ્રોલર, એમ્બેડેડ સિસ્ટમ \\
ઉદાહરણો & મોટાભાગના PC, 8085, 8086 & 8051, PIC, ARM Cortex-M \\
\end{longtable}
}

\textbf{ડાયાગ્રામ:}

\begin{verbatim}
Von Neumann:                    Harvard:
+{-{-}{-}{-}{-}{-}{-}{-}{-}+                     +{-}{-}{-}{-}{-}{-}{-}{-}{-}+}
|         |                     |         |
|   CPU   |{{-}{-}{-}{-}| Memory |     |   CPU   |{-}{-}{-}{-}| Program Memory |}
|         |                     |         |
+{-{-}{-}{-}{-}{-}{-}{-}{-}+                     +{-}{-}{-}{-}{-}{-}{-}{-}{-}+}
                                     \^{}
                                     |
                                     v
                                | Data Memory |
\end{verbatim}

\end{solutionbox}
\begin{mnemonicbox}
``હાર્વર્ડ હંમેશા અલગ રસ્તા રાખે'' (હાર્વર્ડમાં મેમરી પાથ અલગ
હોય છે)

\end{mnemonicbox}
\subsection*{પ્રશ્ન 1(ક OR) [7
ગુણ]}\label{uxaaauxab0uxab6uxaa8-1uxa95-or-7-uxa97uxaa3}

\textbf{RISC, CISC, Opcode, Operand, Instruction Cycle, Machine Cycle,
અને T State ને વ્યાખ્યાયિત કરો.}

\begin{solutionbox}

{\def\LTcaptype{none} % do not increment counter
\begin{longtable}[]{@{}
  >{\raggedright\arraybackslash}p{(\linewidth - 2\tabcolsep) * \real{0.3333}}
  >{\raggedright\arraybackslash}p{(\linewidth - 2\tabcolsep) * \real{0.6667}}@{}}
\toprule\noalign{}
\begin{minipage}[b]{\linewidth}\raggedright
શબ્દ
\end{minipage} & \begin{minipage}[b]{\linewidth}\raggedright
વ્યાખ્યા
\end{minipage} \\
\midrule\noalign{}
\endhead
\bottomrule\noalign{}
\endlastfoot
\textbf{RISC} & રિડ્યુસ્ડ ઇન્સ્ટ્રક્શન સેટ કમ્પ્યુટર - સરળ ઇન્સ્ટ્રક્શન સાથે સ્પીડ માટે
ઓપ્ટિમાઇઝ્ડ આર્કિટેક્ચર \\
\textbf{CISC} & કોમ્પ્લેક્સ ઇન્સ્ટ્રક્શન સેટ કમ્પ્યુટર - જટિલ, શક્તિશાળી ઇન્સ્ટ્રક્શન
સાથેનું આર્કિટેક્ચર \\
\textbf{Opcode} & ઓપરેશન કોડ - ઇન્સ્ટ્રક્શનનો ભાગ જે કયા ઓપરેશન કરવાના છે તે સ્પષ્ટ
કરે છે \\
\textbf{Operand} & ઓપરેશનમાં વપરાતો ડેટા વેલ્યુ અથવા એડ્રેસ \\
\textbf{Instruction Cycle} & ઇન્સ્ટ્રક્શન ફેચ, ડિકોડ અને એક્ઝિક્યુટની સંપૂર્ણ
પ્રક્રિયા \\
\textbf{Machine Cycle} & મૂળભૂત ઓપરેશન જેમ કે મેમરી રીડ/રાઈટ (ઇન્સ્ટ્રક્શન સાયકલનો
ભાગ) \\
\textbf{T-State} & ટાઈમ સ્ટેટ - પ્રોસેસરમાં સમયનો સૌથી નાનો એકમ (ક્લોક
પીરિયડ) \\
\end{longtable}
}

\textbf{ડાયાગ્રામ:}

\begin{verbatim}
Instruction Cycle:
+{-{-}{-}{-}{-}{-}{-}{-}{-}{-}+      +{-}{-}{-}{-}{-}{-}{-}{-}{-}{-}+      +{-}{-}{-}{-}{-}{-}{-}{-}{-}{-}+}
|  FETCH   |{-{-}{-}{-}{-}|  DECODE  |{-}{-}{-}{-}{-}| EXECUTE  |}
+{-{-}{-}{-}{-}{-}{-}{-}{-}{-}+      +{-}{-}{-}{-}{-}{-}{-}{-}{-}{-}+      +{-}{-}{-}{-}{-}{-}{-}{-}{-}{-}+}
      \^{                                  |}
      |                                  |
      +{-{-}{-}{-}{-}{-}{-}{-}{-}{-}{-}{-}{-}{-}{-}{-}{-}{-}{-}{-}{-}{-}{-}{-}{-}{-}{-}{-}{-}{-}{-}{-}{-}{-}+}

T{-States within Machine Cycle:}
+{-{-}{-}{-}{-}+{-}{-}{-}{-}{-}+{-}{-}{-}{-}{-}+{-}{-}{-}{-}{-}+}
| T1  | T2  | T3  | T4  | ...
+{-{-}{-}{-}{-}+{-}{-}{-}{-}{-}+{-}{-}{-}{-}{-}+{-}{-}{-}{-}{-}+}
 {{-}{-}{-} Machine Cycle {-}{-}{-}}
\end{verbatim}

\end{solutionbox}
\begin{mnemonicbox}
``RICO ITEM'' (RISC, CISC, Opcode, Instruction
cycle, T-state, Execute, Machine cycle)

\end{mnemonicbox}
\subsection*{પ્રશ્ન 2(અ) [3
ગુણ]}\label{uxaaauxab0uxab6uxaa8-2uxa85-3-uxa97uxaa3}

\textbf{ડેટા બસ, એડ્રેસ બસ અને કંટ્રોલ બસ વ્યાખ્યાયિત કરો.}

\begin{solutionbox}

{\def\LTcaptype{none} % do not increment counter
\begin{longtable}[]{@{}
  >{\raggedright\arraybackslash}p{(\linewidth - 2\tabcolsep) * \real{0.4545}}
  >{\raggedright\arraybackslash}p{(\linewidth - 2\tabcolsep) * \real{0.5455}}@{}}
\toprule\noalign{}
\begin{minipage}[b]{\linewidth}\raggedright
બસ પ્રકાર
\end{minipage} & \begin{minipage}[b]{\linewidth}\raggedright
વ્યાખ્યા
\end{minipage} \\
\midrule\noalign{}
\endhead
\bottomrule\noalign{}
\endlastfoot
\textbf{ડેટા બસ} & બિડાયરેક્શનલ પાથવે જે માઇક્રોપ્રોસેસર અને પેરિફેરલ ડિવાઇસ વચ્ચે
વાસ્તવિક ડેટા ટ્રાન્સફર કરે છે \\
\textbf{એડ્રેસ બસ} & યુનિડાયરેક્શનલ પાથવે જે એક્સેસ કરવાના મેમરી/IO ડિવાઇસ લોકેશન
ધરાવે છે \\
\textbf{કંટ્રોલ બસ} & સિગ્નલ લાઈનોનો ગ્રુપ જે સિસ્ટમ ઓપરેશનને કોઓર્ડિનેટ અને
સિન્ક્રોનાઇઝ કરે છે \\
\end{longtable}
}

\textbf{ડાયાગ્રામ:}

\begin{verbatim}
+{-{-}{-}{-}{-}{-}{-}{-}{-}{-}+     +{-}{-}{-}{-}{-}{-}{-}{-}{-}{-}{-}{-}{-}{-}{-}{-}{-}+}
|          |{-{-}{-}{-}| Address Bus     |}
|          |     | (Memory/IO loc) |
|  CPU     |     +{-{-}{-}{-}{-}{-}{-}{-}{-}{-}{-}{-}{-}{-}{-}{-}{-}+}
|          |{{-}{-}{-}+{-}{-}{-}{-}{-}{-}{-}{-}{-}{-}{-}{-}{-}{-}{-}{-}{-}+}
|          |     | Data Bus        |
|          |     | (Information)   |
|          |     +{-{-}{-}{-}{-}{-}{-}{-}{-}{-}{-}{-}{-}{-}{-}{-}{-}+}
|          |{{-}{-}{-}+{-}{-}{-}{-}{-}{-}{-}{-}{-}{-}{-}{-}{-}{-}{-}{-}{-}+}
|          |     | Control Bus     |
+{-{-}{-}{-}{-}{-}{-}{-}{-}{-}+     | (RD,WR,IO/M...) |}
                 +{-{-}{-}{-}{-}{-}{-}{-}{-}{-}{-}{-}{-}{-}{-}{-}{-}+}
\end{verbatim}

\end{solutionbox}
\begin{mnemonicbox}
``ADC'' - ``એડ્રેસ લોકેશન શોધે, ડેટા માહિતી લઈ જાય, કંટ્રોલ
ઓપરેશન કોઓર્ડિનેટ કરે''

\end{mnemonicbox}
\subsection*{પ્રશ્ન 2(બ) [4
ગુણ]}\label{uxaaauxab0uxab6uxaa8-2uxaac-4-uxa97uxaa3}

\textbf{માઇક્રોપ્રોસેસર અને માઇક્રોકન્ટ્રોલરની સરખામણી કરો.}

\begin{solutionbox}

{\def\LTcaptype{none} % do not increment counter
\begin{longtable}[]{@{}lll@{}}
\toprule\noalign{}
ફીચર & માઇક્રોપ્રોસેસર & માઇક્રોકન્ટ્રોલર \\
\midrule\noalign{}
\endhead
\bottomrule\noalign{}
\endlastfoot
વ્યાખ્યા & એકલ ચિપ પર CPU & એકલ ચિપ પર સંપૂર્ણ કમ્પ્યુટર સિસ્ટમ \\
મેમરી & બાહ્ય RAM/ROM જરૂરી & અંદર જ RAM/ROM \\
I/O પોર્ટ & મર્યાદિત અથવા ચિપ પર નથી & ચિપ પર ઘણા I/O પોર્ટ \\
પેરિફેરલ્સ & બાહ્ય પેરિફેરલ્સ જરૂરી & અંદર જ પેરિફેરલ્સ (ટાઈમર્સ, ADC, વગેરે) \\
ઉપયોગો & સામાન્ય કમ્પ્યુટિંગ, PC & એમ્બેડેડ સિસ્ટમ, IoT ડિવાઇસિસ \\
કિંમત & સંપૂર્ણ સિસ્ટમ માટે વધારે & ઓછી (ઓલ-ઇન-વન સોલ્યુશન) \\
પાવર કન્ઝમ્પશન & વધારે & ઓછું \\
\end{longtable}
}

\end{solutionbox}
\begin{mnemonicbox}
``MEMI-CAP'' (મેમરી બાહ્ય/આંતરિક, કિંમત, એપ્લિકેશન્સ,
પેરિફેરલ્સ)

\end{mnemonicbox}
\subsection*{પ્રશ્ન 2(ક) [7
ગુણ]}\label{uxaaauxab0uxab6uxaa8-2uxa95-7-uxa97uxaa3}

\textbf{8085 બ્લોક ડાયાગ્રામ સ્કેચ કરો અને સમજાવો.}

\begin{solutionbox}

\textbf{ડાયાગ્રામ:}

\begin{verbatim}
    +{-{-}{-}{-}{-}{-}{-}{-}{-}{-}{-}{-}{-}{-}{-}{-}{-}{-}{-}{-}{-}{-}{-}{-}{-}{-}{-}{-}{-}{-}{-}{-}{-}{-}{-}{-}{-}{-}{-}{-}{-}{-}{-}+}
    |               8085 CPU                    |
    |                                           |
    |  +{-{-}{-}{-}{-}{-}{-}{-}{-}{-}{-}{-}{-}+       +{-}{-}{-}{-}{-}{-}{-}{-}{-}{-}{-}{-}{-}{-}+   |}
    |  | REGISTER    |       | TIMING \&     |   |
    |  | ARRAY       |       | CONTROL      |   |
    |  |{-{-}{-}{-}{-}{-}{-}{-}{-}{-}{-}{-}{-}|       |{-}{-}{-}{-}{-}{-}{-}{-}{-}{-}{-}{-}{-}{-}|   |}
    |  | A,Flags     |       | Instruction  |   |
    |  | B,C,D,E,H,L |{{-}{-}{-}{-}{-}| Decoder      |   |}
    |  | SP, PC      |       | Interrupt    |   |
    |  | W,Z(Temp)   |       | Control      |   |
    |  +{-{-}{-}{-}{-}{-}\^{}{-}{-}{-}{-}{-}{-}+       +{-}{-}{-}{-}{-}{-}\^{}{-}{-}{-}{-}{-}{-}{-}+   |}
    |         |                     |           |
    |         v                     |           |
    |  +{-{-}{-}{-}{-}{-}+{-}{-}{-}{-}{-}{-}+              |           |}
    |  |     ALU     |{{-}{-}{-}{-}{-}{-}{-}{-}{-}{-}{-}{-}{-}+           |}
    |  +{-{-}{-}{-}{-}{-}\^{}{-}{-}{-}{-}{-}{-}+                          |}
    |         |                                 |
    |  +{-{-}{-}{-}{-}{-}+{-}{-}{-}{-}{-}{-}+                          |}
    |  |  INTERNAL   |                          |
    |  |  DATA BUS   |                          |
    |  +{-{-}{-}{-}{-}{-}\^{}{-}{-}{-}{-}{-}{-}+                          |}
    +{-{-}{-}{-}{-}{-}{-}{-}{-}|{-}{-}{-}{-}{-}{-}{-}{-}{-}{-}{-}{-}{-}{-}{-}{-}{-}{-}{-}{-}{-}{-}{-}{-}{-}{-}{-}{-}{-}{-}{-}{-}{-}+}
              |
    +{-{-}{-}{-}{-}{-}{-}{-}{-}v{-}{-}{-}{-}{-}{-}{-}{-}{-}{-}{-}{-}{-}{-}{-}{-}{-}{-}{-}{-}{-}{-}{-}{-}{-}{-}{-}{-}{-}{-}{-}{-}{-}+}
    |  ADDRESS, DATA \& CONTROL BUS INTERFACE    |
    +{-{-}{-}{-}{-}{-}\^{}{-}{-}{-}{-}{-}{-}{-}{-}{-}\^{}{-}{-}{-}{-}{-}{-}{-}{-}{-}{-}{-}\^{}{-}{-}{-}{-}{-}{-}{-}{-}{-}{-}{-}{-}{-}{-}+}
           |         |           |
           v         v           v
        ADDRESS     DATA      CONTROL
          BUS       BUS         BUS
\end{verbatim}

\textbf{મુખ્ય ઘટકો}:

\begin{itemize}
\tightlist
\item
  \textbf{રજિસ્ટર એરે}: A (એક્યુમુલેટર), ફ્લેગ્સ, B-L, SP, PC, ટેમ્પ રજિસ્ટર્સ
\item
  \textbf{ALU}: ગાણિતિક અને લોજિકલ ઓપરેશન કરે છે
\item
  \textbf{ટાઈમિંગ \& કંટ્રોલ}: કંટ્રોલ સિગ્નલ્સ જનરેટ કરે છે, ઇન્ટરપ્ટ હેન્ડલ કરે છે
\item
  \textbf{બસ ઇન્ટરફેસ}: CPU ને બાહ્ય ડિવાઇસ સાથે જોડે છે
\item
  \textbf{ઇન્ટરનલ ડેટા બસ}: આંતરિક ઘટકોને જોડે છે
\end{itemize}

\end{solutionbox}
\begin{mnemonicbox}
``RATBI'' - ``રજિસ્ટર્સ, ALU, ટાઈમિંગ, બસ, ઇન્ટરફેસ''

\end{mnemonicbox}
\subsection*{પ્રશ્ન 2(અ OR) [3
ગુણ]}\label{uxaaauxab0uxab6uxaa8-2uxa85-or-3-uxa97uxaa3}

\textbf{એક્યુમ્યુલેટર, પ્રોગ્રામ કાઉન્ટર અને સ્ટેક પોઇન્ટર સમજાવો.}

\begin{solutionbox}

{\def\LTcaptype{none} % do not increment counter
\begin{longtable}[]{@{}
  >{\raggedright\arraybackslash}p{(\linewidth - 2\tabcolsep) * \real{0.5000}}
  >{\raggedright\arraybackslash}p{(\linewidth - 2\tabcolsep) * \real{0.5000}}@{}}
\toprule\noalign{}
\begin{minipage}[b]{\linewidth}\raggedright
રજિસ્ટર
\end{minipage} & \begin{minipage}[b]{\linewidth}\raggedright
કાર્ય
\end{minipage} \\
\midrule\noalign{}
\endhead
\bottomrule\noalign{}
\endlastfoot
\textbf{એક્યુમ્યુલેટર (A)} & 8-બિટ રજિસ્ટર જે ગાણિતિક અને લોજિકલ ઓપરેશનના પરિણામો
સ્ટોર કરે છે \\
\textbf{પ્રોગ્રામ કાઉન્ટર (PC)} & 16-બિટ રજિસ્ટર જે આગલા એક્ઝિક્યુટ થનાર
ઇન્સ્ટ્રક્શનનું એડ્રેસ રાખે છે \\
\textbf{સ્ટેક પોઇન્ટર (SP)} & 16-બિટ રજિસ્ટર જે મેમરીમાં સ્ટેકના વર્તમાન ટોપને
પોઇન્ટ કરે છે \\
\end{longtable}
}

\textbf{ડાયાગ્રામ:}

\begin{verbatim}
Accumulator:      Program Counter:      Stack Pointer:
+{-{-}{-}{-}{-}{-}{-}{-}+        +{-}{-}{-}{-}{-}{-}{-}{-}{-}{-}{-}{-}{-}{-}+      +{-}{-}{-}{-}{-}{-}{-}{-}{-}{-}{-}{-}{-}{-}+}
|   A    | {{-}{-}{-}  |      PC      |{-}{-}{-}{-} |      SP      |{-}{-}{-}+}
+{-{-}{-}{-}{-}{-}{-}{-}+        +{-}{-}{-}{-}{-}{-}{-}{-}{-}{-}{-}{-}{-}{-}+      +{-}{-}{-}{-}{-}{-}{-}{-}{-}{-}{-}{-}{-}{-}+   |}
Data operations   Points to next        Points to          |
                  instruction           stack top          v
                                                      +{-{-}{-}{-}{-}{-}{-}{-}+}
                                                      | Stack  |
                                                      | Memory |
                                                      +{-{-}{-}{-}{-}{-}{-}{-}+}
\end{verbatim}

\end{solutionbox}
\begin{mnemonicbox}
``APS'' - ``એક્યુમ્યુલેટર પ્રોસેસ કરે, PC આગલું ઇન્સ્ટ્રક્શન જુએ,
SP સ્ટેક સંભાળે''

\end{mnemonicbox}
\subsection*{પ્રશ્ન 2(બ OR) [4
ગુણ]}\label{uxaaauxab0uxab6uxaa8-2uxaac-or-4-uxa97uxaa3}

\textbf{એડ્રેસ બસ અને ડેટા બસનું ડિમલ્ટિપ્લેક્સીંગ સ્કેચ કરો અને સમજાવો.}

\begin{solutionbox}

\textbf{ડાયાગ્રામ:}

\begin{verbatim}
                   +{-{-}{-}{-}{-}{-}{-}{-}{-}{-}+}
A15{-A8 {-}{-}{-}{-}{-}{-}{-}{-}{-}{-}{-}|          |}
                   |          |{-{-}{-}{-}{-}{-}{-}{-}{-} A15{-}A8 (Higher Address)}
                   |          |
AD7{-AD0 {-}{-}{-}{-}{-}{-}{-}{-}{-}| 8085 CPU |{-}{-}{-}{-}+}
                   |          |     |
                   |          |     |    +{-{-}{-}{-}{-}{-}{-}{-}+}
                   +{-{-}{-}{-}{-}{-}{-}{-}{-}{-}+     +{-}{-}{-}| 74LS373|{-}{-}{-}{-} A7{-}A0 (Lower Address)}
                        |                | Latch  |
                        |                +{-{-}{-}{-}{-}{-}{-}{-}+}
                        |                    \^{}
                        |                    |
                     ALE {-{-}{-}{-}{-}{-}{-}{-}{-}{-}{-}{-}{-}{-}{-}{-}{-}{-}{-}{-}}
\end{verbatim}

\textbf{પ્રક્રિયા}:

\begin{enumerate}
\tightlist
\item
  \textbf{મલ્ટિપ્લેક્સિંગ}: પિન કાઉન્ટ ઘટાડવા માટે AD0-AD7 પિન એડ્રેસ અને ડેટા
  સિગ્નલ શેર કરે છે
\item
  \textbf{ડિમલ્ટિપ્લેક્સિંગના સ્ટેપ્સ}:

  \begin{itemize}
  \tightlist
  \item
    CPU AD0-AD7 પિન પર એડ્રેસ મૂકે છે
  \item
    ALE (એડ્રેસ લેચ એનેબલ) સિગ્નલ HIGH થાય છે
  \item
    બાહ્ય લેચ (74LS373) લોઅર એડ્રેસ બિટ્સ પકડે છે
  \item
    ALE LOW થાય છે, એડ્રેસ લેચ થઈ જાય છે
  \item
    AD0-AD7 પિન હવે ડેટા લઈ જાય છે
  \end{itemize}
\end{enumerate}

\end{solutionbox}
\begin{mnemonicbox}
``ALAD'' - ``ALE એક્ટિવ, લેચ એડ્રેસ, આફ્ટર ડેટા''

\end{mnemonicbox}
\subsection*{પ્રશ્ન 2(ક OR) [7
ગુણ]}\label{uxaaauxab0uxab6uxaa8-2uxa95-or-7-uxa97uxaa3}

\textbf{8085 ની કોઈપણ સાત વિશેષતાઓની યાદી આપો.}

\begin{solutionbox}

{\def\LTcaptype{none} % do not increment counter
\begin{longtable}[]{@{}ll@{}}
\toprule\noalign{}
વિશેષતા & વર્ણન \\
\midrule\noalign{}
\endhead
\bottomrule\noalign{}
\endlastfoot
\textbf{8-બિટ ડેટા બસ} & 8 બિટ્સ ડેટા પેરેલલમાં ટ્રાન્સફર કરે છે \\
\textbf{16-બિટ એડ્રેસ બસ} & 64KB સુધીની મેમરી એડ્રેસ કરી શકે છે (2\^{}16) \\
\textbf{હાર્ડવેર ઇન્ટરપ્ટ} & 5 હાર્ડવેર ઇન્ટરપ્ટ (TRAP, RST 7.5, 6.5, 5.5,
INTR) \\
\textbf{સિરિયલ I/O} & સિરિયલ કમ્યુનિકેશન માટે SID અને SOD પિન \\
\textbf{ક્લોક જનરેશન} & ક્રિસ્ટલ સાથે ઓન-ચિપ ક્લોક જનરેટર \\
\textbf{ઇન્સ્ટ્રક્શન સેટ} & 74 ઓપરેશન કોડ્સ જે 246 ઇન્સ્ટ્રક્શન જનરેટ કરે છે \\
\textbf{રજિસ્ટર સેટ} & છ 8-બિટ રજિસ્ટર (B,C,D,E,H,L), એક્યુમુલેટર, ફ્લેગ્સ, SP,
PC \\
\end{longtable}
}

\textbf{ડાયાગ્રામ:}

\begin{verbatim}
           8085 Features
+{-{-}{-}{-}{-}{-}{-}{-}{-}{-}{-}{-}{-}{-}{-}{-}{-}{-}{-}{-}{-}{-}{-}{-}{-}{-}{-}{-}{-}{-}{-}{-}{-}+}
| +{-{-}{-}{-}{-}{-}{-}{-}+       +{-}{-}{-}{-}{-}{-}{-}{-}{-}{-}+   |}
| |  8{-bit |       | 5 HW     |   |}
| |  Data  |       | Interrupt|   |
| |  Bus   |       | Lines    |   |
| +{-{-}{-}{-}{-}{-}{-}{-}+       +{-}{-}{-}{-}{-}{-}{-}{-}{-}{-}+   |}
|                                 |
| +{-{-}{-}{-}{-}{-}{-}{-}+       +{-}{-}{-}{-}{-}{-}{-}{-}{-}+    |}
| | 16{-bit |       | 74      |    |}
| | Address|       | Opcodes |    |
| |  Bus   |       |         |    |
| +{-{-}{-}{-}{-}{-}{-}{-}+       +{-}{-}{-}{-}{-}{-}{-}{-}{-}+    |}
|                                 |
| +{-{-}{-}{-}{-}{-}{-}{-}+       +{-}{-}{-}{-}{-}{-}{-}{-}{-}+    |}
| | Serial |       | On{-chip |    |}
| |  I/O   |       | Clock   |    |
| |        |       |         |    |
| +{-{-}{-}{-}{-}{-}{-}{-}+       +{-}{-}{-}{-}{-}{-}{-}{-}{-}+    |}
+{-{-}{-}{-}{-}{-}{-}{-}{-}{-}{-}{-}{-}{-}{-}{-}{-}{-}{-}{-}{-}{-}{-}{-}{-}{-}{-}{-}{-}{-}{-}{-}{-}+}
\end{verbatim}

\end{solutionbox}
\begin{mnemonicbox}
``CHAIRS'' - ``ક્લોક, હાર્ડવેર ઇન્ટરપ્ટ, એડ્રેસ બસ,
ઇન્સ્ટ્રક્શન સેટ, રજિસ્ટર્સ, સિરિયલ I/O''

\end{mnemonicbox}
\subsection*{પ્રશ્ન 3(અ) [3
ગુણ]}\label{uxaaauxab0uxab6uxaa8-3uxa85-3-uxa97uxaa3}

\textbf{8051 ના કોઈપણ એક ટાઈમર મોડને સમજાવો.}

\begin{solutionbox}

\textbf{મોડ 1: 16-બિટ ટાઈમર/કાઉન્ટર}

{\def\LTcaptype{none} % do not increment counter
\begin{longtable}[]{@{}ll@{}}
\toprule\noalign{}
ફીચર & વર્ણન \\
\midrule\noalign{}
\endhead
\bottomrule\noalign{}
\endlastfoot
\textbf{ટાઈમર સ્ટ્રક્ચર} & THx અને TLx રજિસ્ટર્સ વાપરીને 16-બિટ ટાઈમર \\
\textbf{ઓપરેશન} & 0000H થી FFFFH સુધી ગણતરી કરે છે, પછી TF ફ્લેગ સેટ કરે છે \\
\textbf{કાઉન્ટર સાઈઝ} & ફુલ 16-બિટ કાઉન્ટર (2\^{}16 = 65,536 કાઉન્ટ્સ) \\
\textbf{રજિસ્ટર્સ} & THx (હાઈ બાઈટ) અને TLx (લો બાઈટ) \\
\end{longtable}
}

\textbf{ડાયાગ્રામ:}

\begin{verbatim}
                        TF (Timer Flag)
                              \^{}
                              |
+{-{-}{-}{-}{-}{-}{-}{-}+    +{-}{-}{-}{-}{-}{-}{-}{-}+   +{-}{-}{-}{-}{-}{-}{-}{-}{-}{-}+}
| Control|{-{-}{-}| Gate   |{-}{-}| Overflow |}
| Bits   |    | Control|   | Detect   |
+{-{-}{-}{-}{-}{-}{-}{-}+    +{-}{-}{-}{-}{-}{-}{-}{-}+   +{-}{-}{-}{-}{-}{-}{-}{-}{-}{-}+}
                 |
                 v
Clock Source {-{-} THx:TLx Counter {-}{-} TFx}
                (16{-bit counter)}
\end{verbatim}

\end{solutionbox}
\begin{mnemonicbox}
``MOGC'' - ``મોડ 1 ઓવરફ્લો ડિટેક્શન, ગેટ કંટ્રોલ, કમ્પ્લીટ
16-બિટ''

\end{mnemonicbox}
\subsection*{પ્રશ્ન 3(બ) [4
ગુણ]}\label{uxaaauxab0uxab6uxaa8-3uxaac-4-uxa97uxaa3}

\textbf{8051 માટે ALE, PSEN, RESET અને TXD પિનનું ફંક્શન લખો.}

\begin{solutionbox}

{\def\LTcaptype{none} % do not increment counter
\begin{longtable}[]{@{}
  >{\raggedright\arraybackslash}p{(\linewidth - 2\tabcolsep) * \real{0.3333}}
  >{\raggedright\arraybackslash}p{(\linewidth - 2\tabcolsep) * \real{0.6667}}@{}}
\toprule\noalign{}
\begin{minipage}[b]{\linewidth}\raggedright
પિન
\end{minipage} & \begin{minipage}[b]{\linewidth}\raggedright
ફંક્શન
\end{minipage} \\
\midrule\noalign{}
\endhead
\bottomrule\noalign{}
\endlastfoot
\textbf{ALE} & એડ્રેસ લેચ એનેબલ - પોર્ટ 0 માંથી એડ્રેસનો લો બાઈટ લેચ કરવા માટે
કંટ્રોલ સિગ્નલ પૂરું પાડે છે \\
\textbf{PSEN} & પ્રોગ્રામ સ્ટોર એનેબલ - બાહ્ય પ્રોગ્રામ મેમરી એક્સેસ માટે રીડ
સ્ટ્રોબ \\
\textbf{RESET} & રીસેટ ઇનપુટ - 2 મશીન સાયકલ સુધી HIGH રાખવાથી CPU ને પ્રારંભિક
સ્થિતિમાં ફોર્સ કરે છે \\
\textbf{TXD} & ટ્રાન્સમિટ ડેટા - સિરિયલ ડેટા ટ્રાન્સમિશન માટે સિરિયલ પોર્ટ આઉટપુટ
પિન \\
\end{longtable}
}

\textbf{ડાયાગ્રામ:}

\begin{verbatim}
   8051 Pin Functions
+{-{-}{-}{-}{-}{-}{-}{-}{-}{-}{-}{-}{-}{-}{-}{-}{-}{-}{-}{-}+}
|                    |
|      +{-{-}{-}{-}{-}{-}+      |}
| ALE {-|      |{-} TXD |}
|      |      |      |
| PSEN{-| 8051 |      |}
|      |      |      |
|RESET{-|      |      |}
|      +{-{-}{-}{-}{-}{-}+      |}
|                    |
+{-{-}{-}{-}{-}{-}{-}{-}{-}{-}{-}{-}{-}{-}{-}{-}{-}{-}{-}{-}+}
\end{verbatim}

\end{solutionbox}
\begin{mnemonicbox}
``APTR'' - ``એડ્રેસ લેચ, પ્રોગ્રામ સ્ટોર, ટોટલ રીસેટ,
ટ્રાન્સમિટ ડેટા''

\end{mnemonicbox}
\subsection*{પ્રશ્ન 3(ક) [7
ગુણ]}\label{uxaaauxab0uxab6uxaa8-3uxa95-7-uxa97uxaa3}

\textbf{8051 માઇક્રોકન્ટ્રોલરના દરેક બ્લોકના કાર્યો સમજાવો.}

\begin{solutionbox}

{\def\LTcaptype{none} % do not increment counter
\begin{longtable}[]{@{}ll@{}}
\toprule\noalign{}
બ્લોક & કાર્ય \\
\midrule\noalign{}
\endhead
\bottomrule\noalign{}
\endlastfoot
\textbf{CPU} & 8-બિટ પ્રોસેસર જે ઇન્સ્ટ્રક્શન ફેચ અને એક્ઝિક્યુટ કરે છે \\
\textbf{મેમરી} & 4KB ઇન્ટરનલ ROM અને 128 બાઈટ્સ ઇન્ટરનલ RAM \\
\textbf{I/O પોર્ટ્સ} & ચાર 8-બિટ બિડાયરેક્શનલ I/O પોર્ટ્સ (P0-P3) \\
\textbf{ટાઈમર/કાઉન્ટર} & ટાઈમિંગ અને કાઉન્ટિંગ માટે બે 16-બિટ ટાઈમર/કાઉન્ટર \\
\textbf{સિરિયલ પોર્ટ} & સિરિયલ કમ્યુનિકેશન માટે ફુલ-ડુપ્લેક્સ UART \\
\textbf{ઇન્ટરપ્ટ} & બે પ્રાયોરિટી લેવલ સાથે પાંચ ઇન્ટરપ્ટ સોર્સ \\
\textbf{ક્લોક સર્કિટ} & તમામ ઓપરેશન માટે ટાઈમિંગ પૂરું પાડે છે \\
\end{longtable}
}

\textbf{ડાયાગ્રામ:}

\begin{verbatim}
+{-{-}{-}{-}{-}{-}{-}{-}{-}{-}{-}{-}{-}{-}{-}{-}{-}{-}{-}{-}{-}{-}{-}{-}{-}{-}{-}{-}{-}{-}{-}{-}{-}{-}{-}{-}{-}{-}{-}{-}{-}{-}{-}{-}{-}{-}{-}{-}{-}{-}{-}{-}{-}{-}{-}+}
|                    8051 ARCHITECTURE                  |
|                                                       |
|  +{-{-}{-}{-}{-}{-}{-}{-}{-}{-}+     +{-}{-}{-}{-}{-}{-}{-}{-}{-}{-}+     +{-}{-}{-}{-}{-}{-}{-}{-}{-}{-}{-}{-}+     |}
|  |          |     |          |     |            |     |
|  |   CPU    |{{-}{-}{-}|  Timers/ |     |  Interrupts|     |}
|  |          |     | Counters |     |            |     |
|  +{-{-}{-}{-}{-}{-}{-}{-}{-}{-}+     +{-}{-}{-}{-}{-}{-}{-}{-}{-}{-}+     +{-}{-}{-}{-}{-}{-}{-}{-}{-}{-}{-}{-}+     |}
|       \^{                                  \^{}            |}
|       |                                  |            |
|       v                                  v            |
|  +{-{-}{-}{-}{-}{-}{-}{-}{-}{-}+     +{-}{-}{-}{-}{-}{-}{-}{-}{-}{-}+     +{-}{-}{-}{-}{-}{-}{-}{-}{-}{-}{-}{-}+     |}
|  |          |     |          |     |            |     |
|  | Memory   |{{-}{-}{-}|  Serial  |{-}{-}{-}| I/O Ports  |     |}
|  | RAM/ROM  |     |   Port   |     |P0,P1,P2,P3 |     |
|  +{-{-}{-}{-}{-}{-}{-}{-}{-}{-}+     +{-}{-}{-}{-}{-}{-}{-}{-}{-}{-}+     +{-}{-}{-}{-}{-}{-}{-}{-}{-}{-}{-}{-}+     |}
|                                                       |
|                  Clock Circuit                        |
+{-{-}{-}{-}{-}{-}{-}{-}{-}{-}{-}{-}{-}{-}{-}{-}{-}{-}{-}{-}{-}{-}{-}{-}{-}{-}{-}{-}{-}{-}{-}{-}{-}{-}{-}{-}{-}{-}{-}{-}{-}{-}{-}{-}{-}{-}{-}{-}{-}{-}{-}{-}{-}{-}{-}+}
\end{verbatim}

\end{solutionbox}
\begin{mnemonicbox}
``CRIMSON'' - ``CPU, RAM/ROM, I/O, મેમરી, સિરિયલ
પોર્ટ, ઓસિલેટર, ઇન્ટરપ્ટ''

\end{mnemonicbox}
\subsection*{પ્રશ્ન 3(અ OR) [3
ગુણ]}\label{uxaaauxab0uxab6uxaa8-3uxa85-or-3-uxa97uxaa3}

\textbf{8051 ના કોઈપણ એક સીરિયલ કોમ્યુનિકેશન મોડને સમજાવો.}

\begin{solutionbox}

\textbf{મોડ 1: 8-બિટ UART}

{\def\LTcaptype{none} % do not increment counter
\begin{longtable}[]{@{}ll@{}}
\toprule\noalign{}
ફીચર & વર્ણન \\
\midrule\noalign{}
\endhead
\bottomrule\noalign{}
\endlastfoot
\textbf{ફોર્મેટ} & 10 બિટ્સ (સ્ટાર્ટ બિટ, 8 ડેટા બિટ્સ, સ્ટોપ બિટ) \\
\textbf{બોડ રેટ} & વેરિએબલ, ટાઈમર 1 દ્વારા નક્કી થાય છે \\
\textbf{ડેટા ડાયરેક્શન} & ફુલ-ડુપ્લેક્સ (એક સાથે ટ્રાન્સમિટ અને રિસીવ) \\
\textbf{પિન્સ} & ટ્રાન્સમિટ માટે TXD (P3.1), રિસીવ માટે RXD (P3.0) \\
\end{longtable}
}

\textbf{ડાયાગ્રામ:}

\begin{verbatim}
                   SBUF            P3.1 (TXD)
                    |                  \^{}
                    v                  |
+{-{-}{-}{-}{-}{-}{-}{-}+    +{-}{-}{-}{-}{-}{-}{-}{-}{-}{-}{-}{-}+    +{-}{-}{-}{-}{-}{-}{-}{-}{-}{-}{-}+}
| Timer 1|{-{-}{-}| Baud Rate  |{-}{-}{-}| Transmit  |{-}{-}{-} Serial Out}
+{-{-}{-}{-}{-}{-}{-}{-}+    | Generator  |    | Shift Reg |}
              +{-{-}{-}{-}{-}{-}{-}{-}{-}{-}{-}{-}+    +{-}{-}{-}{-}{-}{-}{-}{-}{-}{-}{-}+}
                    |
                    v
              +{-{-}{-}{-}{-}{-}{-}{-}{-}{-}{-}{-}+}
              | Receive    |{{-}{-}{-} Serial In}
              | Shift Reg  |     P3.0 (RXD)
              +{-{-}{-}{-}{-}{-}{-}{-}{-}{-}{-}{-}+}
                    |
                    v
                   SBUF
\end{verbatim}

\end{solutionbox}
\begin{mnemonicbox}
``FADS'' - ``ફોર્મેટ 10-બિટ, ઓટો બોડ ટાઈમર 1 થી, ડુપ્લેક્સ
મોડ, સ્ટાન્ડર્ડ UART''

\end{mnemonicbox}
\subsection*{પ્રશ્ન 3(બ OR) [4
ગુણ]}\label{uxaaauxab0uxab6uxaa8-3uxaac-or-4-uxa97uxaa3}

\textbf{8051 માટે RXD, INT0, T0 અને PROG પિનનું ફંક્શન લખો.}

\begin{solutionbox}

{\def\LTcaptype{none} % do not increment counter
\begin{longtable}[]{@{}
  >{\raggedright\arraybackslash}p{(\linewidth - 2\tabcolsep) * \real{0.3333}}
  >{\raggedright\arraybackslash}p{(\linewidth - 2\tabcolsep) * \real{0.6667}}@{}}
\toprule\noalign{}
\begin{minipage}[b]{\linewidth}\raggedright
પિન
\end{minipage} & \begin{minipage}[b]{\linewidth}\raggedright
ફંક્શન
\end{minipage} \\
\midrule\noalign{}
\endhead
\bottomrule\noalign{}
\endlastfoot
\textbf{RXD (P3.0)} & રિસીવ ડેટા - સિરિયલ ડેટા રિસેપ્શન માટે સિરિયલ પોર્ટ ઇનપુટ
પિન \\
\textbf{INT0 (P3.2)} & એક્સટર્નલ ઇન્ટરપ્ટ 0 - બાહ્ય ઇન્ટરપ્ટ ટ્રિગર કરી શકે તેવો
ઇનપુટ \\
\textbf{T0 (P3.4)} & ટાઈમર 0 - ટાઈમર/કાઉન્ટર 0 માટે બાહ્ય કાઉન્ટ ઇનપુટ \\
\textbf{PROG (EA)} & પ્રોગ્રામ એનેબલ - જ્યારે LOW હોય, ત્યારે CPU ને બાહ્ય
મેમરીમાંથી કોડ ફેચ કરવા ફોર્સ કરે છે \\
\end{longtable}
}

\textbf{ડાયાગ્રામ:}

\begin{verbatim}
   8051 Pin Functions
+{-{-}{-}{-}{-}{-}{-}{-}{-}{-}{-}{-}{-}{-}{-}{-}{-}{-}{-}{-}+}
|                    |
|      +{-{-}{-}{-}{-}{-}+      |}
| RXD {-|      |{-} PROG|}
|      |      |      |
| INT0{-| 8051 |      |}
|      |      |      |
| T0  {-|      |      |}
|      +{-{-}{-}{-}{-}{-}+      |}
|                    |
+{-{-}{-}{-}{-}{-}{-}{-}{-}{-}{-}{-}{-}{-}{-}{-}{-}{-}{-}{-}+}
\end{verbatim}

\end{solutionbox}
\begin{mnemonicbox}
``RIPE'' - ``રિસીવ ડેટા, ઇન્ટરપ્ટ ટ્રિગર, પલ્સ કાઉન્ટિંગ,
એક્સટર્નલ મેમરી''

\end{mnemonicbox}
\subsection*{પ્રશ્ન 3(ક OR) [7
ગુણ]}\label{uxaaauxab0uxab6uxaa8-3uxa95-or-7-uxa97uxaa3}

\textbf{8051 માટે ALU, PC, DPTR, RS0, RS1, આંતરિક RAM અને આંતરિક ROM નું વર્ણન
કરો.}

\begin{solutionbox}

{\def\LTcaptype{none} % do not increment counter
\begin{longtable}[]{@{}
  >{\raggedright\arraybackslash}p{(\linewidth - 2\tabcolsep) * \real{0.4583}}
  >{\raggedright\arraybackslash}p{(\linewidth - 2\tabcolsep) * \real{0.5417}}@{}}
\toprule\noalign{}
\begin{minipage}[b]{\linewidth}\raggedright
ઘટક
\end{minipage} & \begin{minipage}[b]{\linewidth}\raggedright
વર્ણન
\end{minipage} \\
\midrule\noalign{}
\endhead
\bottomrule\noalign{}
\endlastfoot
\textbf{ALU} & અર્થમેટિક લોજિક યુનિટ - ગાણિતિક અને લોજિકલ ઓપરેશન કરે છે \\
\textbf{PC} & પ્રોગ્રામ કાઉન્ટર - 16-બિટ રજિસ્ટર જે આગલી ઇન્સ્ટ્રક્શનને પોઇન્ટ કરે
છે \\
\textbf{DPTR} & ડેટા પોઇન્ટર - 16-બિટ રજિસ્ટર (DPH+DPL) બાહ્ય મેમરી એડ્રેસિંગ
માટે \\
\textbf{RS0, RS1} & PSW માં રજિસ્ટર બેંક સિલેક્ટ બિટ્સ - ચાર રજિસ્ટર બેંક્સમાંથી એક
પસંદ કરે છે \\
\textbf{આંતરિક RAM} & 128 બાઈટ્સ ઓન-ચિપ RAM (00H-7FH) વેરિએબલ્સ અને સ્ટેક
માટે \\
\textbf{આંતરિક ROM} & 4KB ઓન-ચિપ ROM (0000H-0FFFH) પ્રોગ્રામ સ્ટોરેજ માટે \\
\end{longtable}
}

\textbf{ડાયાગ્રામ:}

\begin{verbatim}
8051 Memory Organization:
+{-{-}{-}{-}{-}{-}{-}{-}{-}{-}{-}{-}{-}{-}{-}{-}{-}{-}{-}+ 0FFFH}
|                   |
|  Internal ROM     |
|  (4KB)            |
|                   |
+{-{-}{-}{-}{-}{-}{-}{-}{-}{-}{-}{-}{-}{-}{-}{-}{-}{-}{-}+ 0000H}

Internal RAM:
+{-{-}{-}{-}{-}{-}{-}{-}{-}{-}{-}{-}{-}{-}{-}{-}{-}{-}{-}+ 7FH}
|  Scratch Pad      |
+{-{-}{-}{-}{-}{-}{-}{-}{-}{-}{-}{-}{-}{-}{-}{-}{-}{-}{-}+ 30H}
|  Bit{-addressable  |}
+{-{-}{-}{-}{-}{-}{-}{-}{-}{-}{-}{-}{-}{-}{-}{-}{-}{-}{-}+ 20H}
|  Register Banks   |
|  (RS0,RS1 select) |
+{-{-}{-}{-}{-}{-}{-}{-}{-}{-}{-}{-}{-}{-}{-}{-}{-}{-}{-}+ 00H}
\end{verbatim}

\end{solutionbox}
\begin{mnemonicbox}
``APRID'' - ``ALU પ્રોસેસ કરે, PC યાદ રાખે, રજિસ્ટર બેંક
સિલેક્ટ, ઇન્ટરનલ મેમરી, DPTR પોઇન્ટ કરે''

\end{mnemonicbox}
\subsection*{પ્રશ્ન 4(અ) [3
ગુણ]}\label{uxaaauxab0uxab6uxaa8-4uxa85-3-uxa97uxaa3}

\textbf{08H ને 02H થી વિભાજિત કરવા માટે એસેમ્બલી ભાષામાં પ્રોગ્રામ વિકસાવો.}

\begin{solutionbox}

\begin{verbatim}
      MOV A, \#08H    ; ડિવિડન્ડ 08H એક્યુમુલેટરમાં લોડ કરો
      MOV B, \#02H    ; ડિવાઇઝર 02H B રજિસ્ટરમાં લોડ કરો
DIV AB         ; A ને B વડે ભાગો (A=ભાગફળ,

B=શેષ)

      MOV R0, A      ; ભાગફળ R0 માં સ્ટોર કરો (04H)
      MOV R1, B      ; શેષ R1 માં સ્ટોર કરો (00H)
\end{verbatim}

\textbf{ડાયાગ્રામ:}

\begin{verbatim}
Before DIV AB:       After DIV AB:
+{-{-}{-}{-}{-}{-}{-}{-}+           +{-}{-}{-}{-}{-}{-}{-}{-}+}
| A: 08H |           | A: 04H | (Quotient)
+{-{-}{-}{-}{-}{-}{-}{-}+           +{-}{-}{-}{-}{-}{-}{-}{-}+}
+{-{-}{-}{-}{-}{-}{-}{-}+           +{-}{-}{-}{-}{-}{-}{-}{-}+}
| B: 02H |           | B: 00H | (Remainder)
+{-{-}{-}{-}{-}{-}{-}{-}+           +{-}{-}{-}{-}{-}{-}{-}{-}+}
\end{verbatim}

\end{solutionbox}
\begin{mnemonicbox}
``LDDS'' - ``લોડ ડિવિડન્ડ, ડિવાઇઝર B માં, ડિવાઇડ, સ્ટોર
રિઝલ્ટ''

\end{mnemonicbox}
\subsection*{પ્રશ્ન 4(બ) [4
ગુણ]}\label{uxaaauxab0uxab6uxaa8-4uxaac-4-uxa97uxaa3}

\textbf{76H અને 32H ઉમેરવા માટે એસેમ્બલી ભાષામાં પ્રોગ્રામ વિકસાવો.}

\begin{solutionbox}

\begin{verbatim}
      MOV A, \#76H    ; પહેલો નંબર 76H એક્યુમુલેટરમાં લોડ કરો
      MOV R0, \#32H   ; બીજો નંબર 32H R0 માં લોડ કરો
      ADD A, R0      ; R0 ને A માં ઉમેરો (76H + 32H = A8H)
      MOV R1, A      ; પરિણામ R1 માં સ્ટોર કરો (A8H)
      JNC DONE       ; જો કેરી ન આવે તો જમ્પ કરો
      MOV R2, \#01H   ; જો કેરી આવે તો, R2 માં 1 સ્ટોર કરો
DONE: NOP            ; પ્રોગ્રામ પૂરો કરો
\end{verbatim}

\textbf{ડાયાગ્રામ:}

\begin{verbatim}
+{-{-}{-}{-}{-}{-}+     +{-}{-}{-}{-}{-}{-}+     +{-}{-}{-}{-}{-}{-}{-}{-}+}
| 76H  | + ? | 32H  | = ? | A8H    | + Carry Flag
+{-{-}{-}{-}{-}{-}+     +{-}{-}{-}{-}{-}{-}+     +{-}{-}{-}{-}{-}{-}{-}{-}+}

Calculation:
   76H = 0111 0110
 + 32H = 0011 0010
{-{-}{-}{-}{-}{-}{-}{-}{-}{-}{-}{-}{-}{-}{-}{-}{-}}
   A8H = 1010 1000
\end{verbatim}

\end{solutionbox}
\begin{mnemonicbox}
``LASER'' - ``લોડ A, સ્ટોર સેકન્ડ નંબર, એક્ઝિક્યુટ એડિશન,
રિઝલ્ટ સ્ટોર''

\end{mnemonicbox}
\subsection*{પ્રશ્ન 4(ક) [7
ગુણ]}\label{uxaaauxab0uxab6uxaa8-4uxa95-7-uxa97uxaa3}

\textbf{એડ્રેસિંગ મોડ શું છે? તેને 8051 માટે વર્ગીકૃત કરો.}

\begin{solutionbox}

\textbf{એડ્રેસિંગ મોડ}: ઇન્સ્ટ્રક્શન માટે ઓપરેન્ડ/ડેટાનું સ્થાન સ્પષ્ટ કરવાની પદ્ધતિ.

{\def\LTcaptype{none} % do not increment counter
\begin{longtable}[]{@{}
  >{\raggedright\arraybackslash}p{(\linewidth - 4\tabcolsep) * \real{0.4359}}
  >{\raggedright\arraybackslash}p{(\linewidth - 4\tabcolsep) * \real{0.3333}}
  >{\raggedright\arraybackslash}p{(\linewidth - 4\tabcolsep) * \real{0.2308}}@{}}
\toprule\noalign{}
\begin{minipage}[b]{\linewidth}\raggedright
એડ્રેસિંગ મોડ
\end{minipage} & \begin{minipage}[b]{\linewidth}\raggedright
વર્ણન
\end{minipage} & \begin{minipage}[b]{\linewidth}\raggedright
ઉદાહરણ
\end{minipage} \\
\midrule\noalign{}
\endhead
\bottomrule\noalign{}
\endlastfoot
\textbf{રજિસ્ટર} & ઓપરેન્ડ રજિસ્ટરમાં & \texttt{MOV\ A,\ R0} (R0 ને A માં મુવ
કરે) \\
\textbf{ડાયરેક્ટ} & ઓપરેન્ડ ચોક્કસ મેમરી લોકેશન પર & \texttt{MOV\ A,\ 30H} (30H
પરથી ડેટા A માં મુવ કરે) \\
\textbf{રજિસ્ટર ઇન્ડાયરેક્ટ} & રજિસ્ટરમાં ઓપરેન્ડનું એડ્રેસ & \texttt{MOV\ A,\ @R0}
(R0 માં સ્ટોર એડ્રેસ પરથી ડેટા A માં મુવ કરે) \\
\textbf{ઈમીડિયેટ} & ઓપરેન્ડ ઇન્સ્ટ્રક્શનનો ભાગ છે & \texttt{MOV\ A,\ \#55H} (A
માં 55H લોડ કરે) \\
\textbf{ઇન્ડેક્સ્ડ} & બેઝ એડ્રેસ + ઓફસેટ & \texttt{MOVC\ A,\ @A+DPTR} (A+DPTR પર
કોડ બાઈટ મેળવે) \\
\textbf{બિટ} & વ્યક્તિગત બિટ એડ્રેસેબલ & \texttt{SETB\ P1.0} (પોર્ટ 1 ના બિટ 0
ને સેટ કરે) \\
\textbf{ઇમ્પ્લાઈડ/ઇનહેરન્ટ} & ઓપરેન્ડ ઇન્સ્ટ્રક્શન દ્વારા સૂચિત & \texttt{RRC\ A} (A
ને કેરી સાથે જમણી બાજુ રોટેટ કરે) \\
\end{longtable}
}

\textbf{ડાયાગ્રામ:}

\begin{verbatim}
+{-{-}{-}{-}{-}{-}{-}{-}{-}{-}{-}{-}{-}{-}{-}{-}{-}{-}+    +{-}{-}{-}{-}{-}{-}{-}{-}{-}{-}{-}{-}{-}{-}{-}{-}{-}{-}+    +{-}{-}{-}{-}{-}{-}{-}{-}{-}{-}{-}{-}{-}{-}{-}{-}{-}{-}+}
| Register         |    | Direct           |    | Indirect         |
| MOV A, R5        |    | MOV A, 40H       |    | MOV A, @R1       |
| +{-{-}{-}+     +{-}{-}{-}+  |    | +{-}{-}{-}+     +{-}{-}{-}+  |    | +{-}{-}{-}+    +{-}{-}{-}+   |}
| | A |{{-}{-}{-}{-}| R5|  |    | | A |{-}{-}{-}{-}|40H|  |    | | A |{-}{-}{-}| X |   |}
| +{-{-}{-}+     +{-}{-}{-}+  |    | +{-}{-}{-}+     +{-}{-}{-}+  |    | +{-}{-}{-}+    +{-}{-}{-}+   |}
+{-{-}{-}{-}{-}{-}{-}{-}{-}{-}{-}{-}{-}{-}{-}{-}{-}{-}+    +{-}{-}{-}{-}{-}{-}{-}{-}{-}{-}{-}{-}{-}{-}{-}{-}{-}{-}+    |            \^{}     |}
                                                |         +{-{-}+{-}{-}+  |}
                                                |         | R1=X|  |
                                                |         +{-{-}{-}{-}{-}+  |}
                                                +{-{-}{-}{-}{-}{-}{-}{-}{-}{-}{-}{-}{-}{-}{-}{-}{-}{-}+}
\end{verbatim}

\end{solutionbox}
\begin{mnemonicbox}
``RIDDIB'' - ``રજિસ્ટર, ઈમીડિયેટ, ડાયરેક્ટ, ડેટા
ઇન્ડાયરેક્ટ, ઇન્ડેક્સ્ડ, બિટ''

\end{mnemonicbox}
\subsection*{પ્રશ્ન 4(અ OR) [3
ગુણ]}\label{uxaaauxab0uxab6uxaa8-4uxa85-or-3-uxa97uxaa3}

\textbf{08H અને 02H નો ગુણાકાર કરવા માટે એસેમ્બલી ભાષામાં પ્રોગ્રામ વિકસાવો.}

\begin{solutionbox}

\begin{verbatim}
      MOV A, \#08H    ; પહેલો નંબર 08H એક્યુમુલેટરમાં લોડ કરો
      MOV B, \#02H    ; બીજો નંબર 02H B રજિસ્ટરમાં લોડ કરો
      MUL AB         ; A અને B નો ગુણાકાર કરો (B:A = પરિણામ)
      MOV R0, A      ; લો{-બાઈટ પરિણામ R0 માં સ્ટોર કરો (10H)}
      MOV R1, B      ; હાઈ{-બાઈટ પરિણામ R1 માં સ્ટોર કરો (00H)}
\end{verbatim}

\textbf{ડાયાગ્રામ:}

\begin{verbatim}
Before MUL AB:       After MUL AB:
+{-{-}{-}{-}{-}{-}{-}{-}+           +{-}{-}{-}{-}{-}{-}{-}{-}+}
| A: 08H |           | A: 10H | (08H  02H = 10H)
+{-{-}{-}{-}{-}{-}{-}{-}+           +{-}{-}{-}{-}{-}{-}{-}{-}+}
+{-{-}{-}{-}{-}{-}{-}{-}+           +{-}{-}{-}{-}{-}{-}{-}{-}+}
| B: 02H |           | B: 00H | (High byte = 00H)
+{-{-}{-}{-}{-}{-}{-}{-}+           +{-}{-}{-}{-}{-}{-}{-}{-}+}
\end{verbatim}

\end{solutionbox}
\begin{mnemonicbox}
``LMSR'' - ``લોડ નંબર્સ, મલ્ટિપ્લાય, સ્ટોર રિઝલ્ટ''

\end{mnemonicbox}
\subsection*{પ્રશ્ન 4(બ) [4
ગુણ]}\label{uxaaauxab0uxab6uxaa8-4uxaac-4-uxa97uxaa3-1}

\textbf{76H માંથી 32H બાદ કરવા માટે એસેમ્બલી ભાષામાં પ્રોગ્રામ વિકસાવો.}

\begin{solutionbox}

\begin{verbatim}
      MOV A, \#32H    ; 32H એક્યુમુલેટરમાં લોડ કરો
      MOV R0, \#76H   ; 76H R0 માં લોડ કરો
      CLR C          ; કેરી ફ્લેગ ક્લિયર કરો (બોરો ફ્લેગ)
      SUBB A, R0     ; A માંથી R0 બોરો સાથે બાદ કરો (32H {- 76H = BCH)}
      MOV R1, A      ; પરિણામ R1 માં સ્ટોર કરો (BCH, જે {-44H દર્શાવે છે)}
\end{verbatim}

\textbf{ડાયાગ્રામ:}

\begin{verbatim}
+{-{-}{-}{-}{-}{-}+     +{-}{-}{-}{-}{-}{-}+     +{-}{-}{-}{-}{-}{-}{-}{-}+}
| 32H  | {- ? | 76H  | = ? | BCH    | (represents {-}44H)}
+{-{-}{-}{-}{-}{-}+     +{-}{-}{-}{-}{-}{-}+     +{-}{-}{-}{-}{-}{-}{-}{-}+}

Calculation:
   32H = 0011 0010
 {- 76H = 0111 0110}
{-{-}{-}{-}{-}{-}{-}{-}{-}{-}{-}{-}{-}{-}{-}{-}{-}}
   BCH = 1011 1100 (two{s complement of 44H)}
\end{verbatim}

\end{solutionbox}
\begin{mnemonicbox}
``LESS'' - ``લોડ ફર્સ્ટ નંબર, એનેબલ બોરો (CLR C),
સબટ્રેક્ટ, સ્ટોર''

\end{mnemonicbox}
\subsection*{પ્રશ્ન 4(ક) [7
ગુણ]}\label{uxaaauxab0uxab6uxaa8-4uxa95-7-uxa97uxaa3-1}

\textbf{Instruction set ના પ્રકારોની સૂચિ બનાવો. કોઈપણ ત્રણને એક ઉદાહરણ સાથે
સમજાવો.}

\begin{solutionbox}

{\def\LTcaptype{none} % do not increment counter
\begin{longtable}[]{@{}
  >{\raggedright\arraybackslash}p{(\linewidth - 4\tabcolsep) * \real{0.4634}}
  >{\raggedright\arraybackslash}p{(\linewidth - 4\tabcolsep) * \real{0.3171}}
  >{\raggedright\arraybackslash}p{(\linewidth - 4\tabcolsep) * \real{0.2195}}@{}}
\toprule\noalign{}
\begin{minipage}[b]{\linewidth}\raggedright
ઇન્સ્ટ્રક્શન ગ્રુપ
\end{minipage} & \begin{minipage}[b]{\linewidth}\raggedright
વર્ણન
\end{minipage} & \begin{minipage}[b]{\linewidth}\raggedright
ઉદાહરણ
\end{minipage} \\
\midrule\noalign{}
\endhead
\bottomrule\noalign{}
\endlastfoot
\textbf{અર્થમેટિક} & ગાણિતિક ઓપરેશન & \texttt{ADD\ A,\ R0} (R0 ને A માં
ઉમેરો) \\
\textbf{લોજિકલ} & લોજિકલ ઓપરેશન & \texttt{ANL\ A,\ \#0FH} (A ને 0FH સાથે
AND કરો) \\
\textbf{ડેટા ટ્રાન્સફર} & લોકેશન વચ્ચે ડેટા ખસેડો & \texttt{MOV\ A,\ R7} (R7 ને A
માં ખસેડો) \\
\textbf{બ્રાન્ચ} & પ્રોગ્રામ ફ્લો બદલો & \texttt{JNZ\ LOOP} (જો A શૂન્ય ન હોય
તો જમ્પ કરો) \\
\textbf{બિટ મેનિપ્યુલેશન} & વ્યક્તિગત બિટ પર ઓપરેશન & \texttt{SETB\ P1.0} (પોર્ટ
1 ના બિટ 0 ને સેટ કરો) \\
\textbf{મશીન કંટ્રોલ} & પ્રોસેસર ઓપરેશન કંટ્રોલ & \texttt{NOP} (કોઈ ઓપરેશન
નહીં) \\
\end{longtable}
}

\textbf{સમજાવેલા ઇન્સ્ટ્રક્શન્સ}:

\begin{enumerate}
\tightlist
\item
  \textbf{ડેટા ટ્રાન્સફર ઇન્સ્ટ્રક્શન્સ}:

  \begin{itemize}
  \tightlist
  \item
    રજિસ્ટર્સ, મેમરી, અથવા I/O પોર્ટ્સ વચ્ચે ડેટા ખસેડે છે
  \item
    ઉદાહરણ: \texttt{MOV\ A,\ 30H} - મેમરી લોકેશન 30H માંથી એક્યુમુલેટરમાં ડેટા
    ખસેડે છે
  \item
    ઓપરેશન: \texttt{A\ \leftarrow\ [30H]}
  \end{itemize}
\item
  \textbf{અર્થમેટિક ઇન્સ્ટ્રક્શન્સ}:

  \begin{itemize}
  \tightlist
  \item
    ઉમેરવું, બાદ કરવું વગેરે જેવા ગાણિતિક ઓપરેશન કરે છે
  \item
    ઉદાહરણ: \texttt{ADD\ A,\ R0} - R0 ની સામગ્રી એક્યુમુલેટરમાં ઉમેરે છે
  \item
    ઓપરેશન: \texttt{A\ \leftarrow\ A\ +\ R0}
  \end{itemize}
\item
  \textbf{લોજિકલ ઇન્સ્ટ્રક્શન્સ}:

  \begin{itemize}
  \tightlist
  \item
    AND, OR, XOR, NOT જેવા લોજિકલ ઓપરેશન કરે છે
  \item
    ઉદાહરણ: \texttt{ANL\ A,\ \#0FH} - અપર નિબલ માસ્ક કરે છે (માત્ર લોઅર નિબલ
    રાખે છે)
  \item
    ઓપરેશન: \texttt{A\ \leftarrow\ A\ AND\ 0FH}
  \end{itemize}
\end{enumerate}

\textbf{ડાયાગ્રામ:}

\begin{verbatim}
             8051 Instruction Types
+{-{-}{-}{-}{-}{-}{-}{-}{-}{-}{-}{-}{-}{-}{-}{-}{-}{-}{-}{-}{-}{-}{-}{-}{-}{-}{-}{-}{-}{-}{-}{-}{-}{-}{-}{-}{-}{-}{-}+}
|                                       |
| +{-{-}{-}{-}{-}{-}{-}{-}{-}{-}{-}{-}{-}+     +{-}{-}{-}{-}{-}{-}{-}{-}{-}{-}{-}{-}{-}+   |}
| | Data        |     | Branch      |   |
| | Transfer    |     | Instructions|   |
| +{-{-}{-}{-}{-}{-}{-}{-}{-}{-}{-}{-}{-}+     +{-}{-}{-}{-}{-}{-}{-}{-}{-}{-}{-}{-}{-}+   |}
|                                       |
| +{-{-}{-}{-}{-}{-}{-}{-}{-}{-}{-}{-}{-}+     +{-}{-}{-}{-}{-}{-}{-}{-}{-}{-}{-}{-}{-}+   |}
| | Arithmetic  |     | Bit         |   |
| | Instructions|     | Manipulation|   |
| +{-{-}{-}{-}{-}{-}{-}{-}{-}{-}{-}{-}{-}+     +{-}{-}{-}{-}{-}{-}{-}{-}{-}{-}{-}{-}{-}+   |}
|                                       |
| +{-{-}{-}{-}{-}{-}{-}{-}{-}{-}{-}{-}{-}+     +{-}{-}{-}{-}{-}{-}{-}{-}{-}{-}{-}{-}{-}+   |}
| | Logical     |     | Machine     |   |
| | Instructions|     | Control     |   |
| +{-{-}{-}{-}{-}{-}{-}{-}{-}{-}{-}{-}{-}+     +{-}{-}{-}{-}{-}{-}{-}{-}{-}{-}{-}{-}{-}+   |}
+{-{-}{-}{-}{-}{-}{-}{-}{-}{-}{-}{-}{-}{-}{-}{-}{-}{-}{-}{-}{-}{-}{-}{-}{-}{-}{-}{-}{-}{-}{-}{-}{-}{-}{-}{-}{-}{-}{-}+}
\end{verbatim}

\end{solutionbox}
\begin{mnemonicbox}
``BALDM'' - ``બ્રાન્ચ, અર્થમેટિક, લોજિકલ, ડેટા ટ્રાન્સફર,
મશીન કંટ્રોલ''

\end{mnemonicbox}
\subsection*{પ્રશ્ન 5(અ) [3
ગુણ]}\label{uxaaauxab0uxab6uxaa8-5uxa85-3-uxa97uxaa3}

\textbf{8051 માઇક્રોકન્ટ્રોલર સાથે ચાર એલઇડીનું ઇન્ટરફેસિંગ દોરો.}

\begin{solutionbox}

\textbf{ડાયાગ્રામ:}

\begin{verbatim}
                      +5V
                       |
                       |
          +{-{-}{-}+      +{-}{-}{-}+      +{-}{-}{-}+      +{-}{-}{-}+}
          |   |      |   |      |   |      |   |
          R1  |      R2  |      R3  |      R4  |  (220Ω resistors)
          |   |      |   |      |   |      |   |
          +{-{-}{-}+      +{-}{-}{-}+      +{-}{-}{-}+      +{-}{-}{-}+}
            |          |          |          |
          +{-{-}{-}+      +{-}{-}{-}+      +{-}{-}{-}+      +{-}{-}{-}+}
          |   |      |   |      |   |      |   |
          LED1       LED2       LED3       LED4
          |   |      |   |      |   |      |   |
          +{-{-}{-}+      +{-}{-}{-}+      +{-}{-}{-}+      +{-}{-}{-}+}
            |          |          |          |
            |          |          |          |
            v          v          v          v
         +{-{-}{-}{-}{-}{-}{-}{-}{-}{-}{-}{-}{-}{-}{-}{-}{-}{-}{-}{-}{-}{-}{-}{-}{-}{-}{-}{-}{-}{-}{-}{-}{-}{-}{-}{-}{-}{-}{-}{-}{-}+}
         |    P1.0   P1.1       P1.2       P1.3    |
         |                                         |
         |                8051                     |
         |                                         |
         +{-{-}{-}{-}{-}{-}{-}{-}{-}{-}{-}{-}{-}{-}{-}{-}{-}{-}{-}{-}{-}{-}{-}{-}{-}{-}{-}{-}{-}{-}{-}{-}{-}{-}{-}{-}{-}{-}{-}{-}{-}+}
\end{verbatim}

\textbf{જરૂરી ઘટકો}:

\begin{itemize}
\tightlist
\item
  8051 માઇક્રોકન્ટ્રોલર
\item
  ચાર LED
\item
  ચાર કરંટ લિમિટિંગ રેસિસ્ટર (220Ω)
\item
  પાવર સપ્લાય
\end{itemize}

\end{solutionbox}
\begin{mnemonicbox}
``PALS'' - ``પોર્ટ પિન, એક્ટિવ-લો કંટ્રોલ, LED, સિમ્પલ
સર્કિટ''

\end{mnemonicbox}
\subsection*{પ્રશ્ન 5(બ) [4
ગુણ]}\label{uxaaauxab0uxab6uxaa8-5uxaac-4-uxa97uxaa3}

\textbf{8051 માઇક્રોકન્ટ્રોલર સાથે 7 સેગમેન્ટ એલઇડીનું ઇન્ટરફેસિંગ દોરો.}

\begin{solutionbox}

\textbf{ડાયાગ્રામ:}

\begin{verbatim}
                    +5V
                     |
                     v
               +{-{-}{-}{-}{-}{-}{-}{-}{-}{-}+}
    P1.0 {-{-}{-}{-}{-}|a         |}
               |         /|
    P1.1 {-{-}{-}{-}{-}|b      /  |}
               |     7    |
    P1.2 {-{-}{-}{-}{-}|c  segment|}
               |   display|
    P1.3 {-{-}{-}{-}{-}|d        |}
               |         {|}
    P1.4 {-{-}{-}{-}{-}|e         |}
               |          |
    P1.5 {-{-}{-}{-}{-}|f         |}
               |          |
    P1.6 {-{-}{-}{-}{-}|g         |}
               |          |
               +{-{-}{-}{-}{-}{-}{-}{-}{-}{-}+}
                    |
                    |
                   GND
        +{-{-}{-}{-}{-}{-}{-}{-}{-}{-}{-}{-}{-}{-}{-}{-}+}
        |                |
        |      8051      |
        |                |
        +{-{-}{-}{-}{-}{-}{-}{-}{-}{-}{-}{-}{-}{-}{-}{-}+}
\end{verbatim}

\textbf{જરૂરી ઘટકો}:

\begin{itemize}
\tightlist
\item
  8051 માઇક્રોકન્ટ્રોલર
\item
  7-સેગમેન્ટ LED ડિસ્પ્લે (કોમન કેથોડ)
\item
  સાત કરંટ લિમિટિંગ રેસિસ્ટર (નથી બતાવેલા)
\item
  પાવર સપ્લાય
\end{itemize}

\textbf{કોડ ઉદાહરણ}:

\begin{verbatim}
; 0{-9 અંકો માટે સેગમેન્ટ પેટર્ન ડિફાઇન કરો}
DIGITS: DB 3FH, 06H, 5BH, 4FH, 66H, 6DH, 7DH, 07H, 7FH, 6FH
  
; અંક 5 ડિસ્પ્લે કરો
MOV A, \#6DH      ; 5 માટે સેગમેન્ટ પેટર્ન
MOV P1, A        ; પોર્ટ P1 પર મોકલો
\end{verbatim}

\end{solutionbox}
\begin{mnemonicbox}
``SPACE-7'' - ``સેવન પિન્સ, પેટર્ન સેગમેન્ટ, અર્થિંગ કોમન,
ઇઝી ડિસ્પ્લે''

\end{mnemonicbox}
\subsection*{પ્રશ્ન 5(ક) [7
ગુણ]}\label{uxaaauxab0uxab6uxaa8-5uxa95-7-uxa97uxaa3}

\textbf{8051 માઇક્રોકન્ટ્રોલર સાથે DAC નું ઇન્ટરફેસિંગ સમજાવો અને જરૂરી પ્રોગ્રામ
લખો.}

\begin{solutionbox}

\textbf{ડાયાગ્રામ:}

\begin{verbatim}
+{-{-}{-}{-}{-}{-}{-}{-}+              +{-}{-}{-}{-}{-}{-}{-}{-}+          +{-}{-}{-}{-}{-}{-}{-}{-}+}
|        |              |        |          |        |
|        |{-{-}P1.0{-}P1.7{-}{-}|D0{-}D7   |          |        |}
|  8051  |              |        |{-{-}Output{-}{-}| Filter |{-}{-}{-} Analog}
|        |              | DAC0808|          |        |     Output
|        |{-{-}P3.0{-}{-}{-}{-}{-}{-}{-}|CS      |          |        |}
|        |              |        |          |        |
+{-{-}{-}{-}{-}{-}{-}{-}+              +{-}{-}{-}{-}{-}{-}{-}{-}+          +{-}{-}{-}{-}{-}{-}{-}{-}+}
                            |
                        +{-{-}{-}+{-}{-}{-}+}
                        | {-5V   |}
                        | +5V   |
                        | GND   |
                        +{-{-}{-}{-}{-}{-}{-}+}
\end{verbatim}

\textbf{જરૂરી ઘટકો}:

\begin{itemize}
\tightlist
\item
  8051 માઇક્રોકન્ટ્રોલર
\item
  DAC0808 (8-બિટ ડિજિટલ-ટુ-એનાલોગ કન્વર્ટર)
\item
  આઉટપુટ બફરિંગ માટે ઓપરેશનલ એમ્પ્લિફાયર
\item
  સ્મુધિંગ માટે RC ફિલ્ટર
\item
  પાવર સપ્લાય
\end{itemize}

\textbf{કનેક્શન્સ}:

\begin{itemize}
\tightlist
\item
  P1.0-P1.7 \rightarrow D0-D7 (8-બિટ ડિજિટલ ઇનપુટ)
\item
  P3.0 \rightarrow CS (ચિપ સિલેક્ટ)
\end{itemize}

\textbf{સોટૂથ વેવ જનરેટ કરવા માટે પ્રોગ્રામ}:

\begin{verbatim}
START:  MOV R0, \#00H     ; R0 ને 0 થી ઇનિશિયલાઇઝ કરો
LOOP:   MOV P1, R0       ; DAC પર વેલ્યુ આઉટપુટ કરો
        CALL DELAY       ; થોડો સમય રાહ જુઓ
        INC R0           ; વેલ્યુ વધારો
        SJMP LOOP        ; સોટૂથ વેવ બનાવવા રિપીટ કરો

DELAY:  MOV R7, \#50      ; ડિલે કાઉન્ટર લોડ કરો
DELAY1: MOV R6, \#255     ; ઇનર લૂપ કાઉન્ટર
DELAY2: DJNZ R6, DELAY2  ; R6 ને ઝીરો થાય ત્યાં સુધી ઘટાડો
        DJNZ R7, DELAY1  ; R7 ને ઝીરો થાય ત્યાં સુધી ઘટાડો
        RET              ; સબરૂટિનથી પાછા ફરો
\end{verbatim}

\textbf{કાર્યપ્રણાલી}:

\begin{enumerate}
\tightlist
\item
  ડિજિટલ વેલ્યુ પોર્ટ 1 પર આઉટપુટ કરવામાં આવે છે
\item
  DAC 8-બિટ ડિજિટલ વેલ્યુને પ્રપોર્શનલ એનાલોગ વોલ્ટેજમાં કન્વર્ટ કરે છે
\item
  ફિલ્ટર આઉટપુટ સિગ્નલને સ્મૂધ કરે છે
\item
  પ્રોગ્રામ આઉટપુટ વેલ્યુને વધારીને સોટૂથ વેવ બનાવે છે
\end{enumerate}

\end{solutionbox}
\begin{mnemonicbox}
``DICAF'' - ``ડિજિટલ ઇનપુટ, ઇન્ક્રિમેન્ટ, કન્વર્ટ ટુ એનાલોગ,
એમ્પ્લિફાય, ફિલ્ટર''

\end{mnemonicbox}
\subsection*{પ્રશ્ન 5(અ OR) [3
ગુણ]}\label{uxaaauxab0uxab6uxaa8-5uxa85-or-3-uxa97uxaa3}

\textbf{8051 માઇક્રોકન્ટ્રોલર સાથે ચાર સ્વીચોનું ઇન્ટરફેસિંગ દોરો.}

\begin{solutionbox}

\textbf{ડાયાગ્રામ:}

\begin{verbatim}
      +5V
       |
       |
      +{-+}
      | |
      | | 10KΩ Pull{-up}
      | | Resistors
      +{-+}
       |              S1        S2        S3        S4
       +{-{-}{-}{-}{-}+        |         |         |         |}
       |     |        |         |         |         |
       +{-{-}{-}{-}{-}+        |         |         |         |}
       |     |        |         |         |         |
       +{-{-}{-}{-}{-}+        |         |         |         |}
       |     |        |         |         |         |
       +{-{-}{-}{-}{-}+        |         |         |         |}
       |              v         v         v         v
    +{-{-}{-}{-}{-}{-}{-}{-}{-}{-}{-}{-}{-}{-}{-}{-}{-}{-}{-}{-}{-}{-}{-}{-}{-}{-}{-}{-}{-}{-}{-}{-}{-}{-}{-}{-}{-}{-}{-}{-}{-}{-}{-}{-}{-}{-}{-}{-}{-}{-}{-}+}
    |                P1.0      P1.1      P1.2    P1.3   |
    |                                                   |
    |                        8051                       |
    |                                                   |
    +{-{-}{-}{-}{-}{-}{-}{-}{-}{-}{-}{-}{-}{-}{-}{-}{-}{-}{-}{-}{-}{-}{-}{-}{-}{-}{-}{-}{-}{-}{-}{-}{-}{-}{-}{-}{-}{-}{-}{-}{-}{-}{-}{-}{-}{-}{-}{-}{-}{-}{-}+}
\end{verbatim}

\textbf{જરૂરી ઘટકો}:

\begin{itemize}
\tightlist
\item
  8051 માઇક્રોકન્ટ્રોલર
\item
  ચાર પુશ બટન (નોર્મલી ઓપન)
\item
  પુલ-અપ રેસિસ્ટર્સ (10KΩ)
\item
  પાવર સપ્લાય
\end{itemize}

\textbf{કાર્યપ્રણાલી}:

\begin{itemize}
\tightlist
\item
  સ્વિચ દબાવવા પર ગ્રાઉન્ડ સાથે જોડાય છે
\item
  સ્વિચ ઓપન હોય ત્યારે પોર્ટ પિન HIGH (1) વાંચે છે
\item
  સ્વિચ દબાવેલ હોય ત્યારે પોર્ટ પિન LOW (0) વાંચે છે
\end{itemize}

\end{solutionbox}
\begin{mnemonicbox}
``PIPS'' - ``પુલ-અપ્સ, ઇનપુટ પિન્સ, પ્રેસ ફોર ઝીરો,
સ્વિચિસ''

\end{mnemonicbox}
\subsection*{પ્રશ્ન 5(બ) [4
ગુણ]}\label{uxaaauxab0uxab6uxaa8-5uxaac-4-uxa97uxaa3-1}

\textbf{8051 માઇક્રોકન્ટ્રોલર સાથે સ્ટેપર મોટરનું ઇન્ટરફેસિંગ દોરો.}

\begin{solutionbox}

\textbf{ડાયાગ્રામ:}

\begin{verbatim}
                       +12V
                        |
                        v
+{-{-}{-}{-}{-}{-}{-}{-}+         +{-}{-}{-}{-}{-}{-}{-}{-}+        +{-}{-}{-}{-}{-}{-}{-}{-}{-}{-}+}
|        |         |        |        |          |
|  8051  |{-{-}P1.0{-}{-}| ULN2003|{-}{-}Out1{-}{-}+ A        |}
|        |         | Driver |        |          |
|        |{-{-}P1.1{-}{-}|        |{-}{-}Out2{-}{-}+ B        |}
|        |         |        |        |          |
|        |{-{-}P1.2{-}{-}|        |{-}{-}Out3{-}{-}+ C Stepper|}
|        |         |        |        |    Motor |
|        |{-{-}P1.3{-}{-}|        |{-}{-}Out4{-}{-}+ D        |}
|        |         |        |        |          |
+{-{-}{-}{-}{-}{-}{-}{-}+         +{-}{-}{-}{-}{-}{-}{-}{-}+        +{-}{-}{-}{-}{-}{-}{-}{-}{-}{-}+}
\end{verbatim}

\textbf{જરૂરી ઘટકો}:

\begin{itemize}
\tightlist
\item
  8051 માઇક્રોકન્ટ્રોલર
\item
  ULN2003 ડ્રાઇવર IC
\item
  સ્ટેપર મોટર (4-ફેઝ)
\item
  પાવર સપ્લાય
\end{itemize}

\textbf{એક્સાઇટેશન સિક્વન્સ}:

{\def\LTcaptype{none} % do not increment counter
\begin{longtable}[]{@{}llllll@{}}
\toprule\noalign{}
સ્ટેપ & P1.3 (D) & P1.2 (C) & P1.1 (B) & P1.0 (A) & હેક્સ વેલ્યુ \\
\midrule\noalign{}
\endhead
\bottomrule\noalign{}
\endlastfoot
1 & 0 & 0 & 0 & 1 & 01H \\
2 & 0 & 0 & 1 & 0 & 02H \\
3 & 0 & 1 & 0 & 0 & 04H \\
4 & 1 & 0 & 0 & 0 & 08H \\
\end{longtable}
}

\end{solutionbox}
\begin{mnemonicbox}
``CUPS'' - ``કંટ્રોલર આઉટપુટ સિક્વન્સ, ULN2003 એમ્પ્લિફાય,
ફેઝ એનર્જાઇઝ, સ્ટેપિંગ મોશન''

\end{mnemonicbox}
\subsection*{પ્રશ્ન 5(ક) [7
ગુણ]}\label{uxaaauxab0uxab6uxaa8-5uxa95-7-uxa97uxaa3-1}

\textbf{8051 માઇક્રોકન્ટ્રોલર સાથે ADC નું ઇન્ટરફેસિંગ સમજાવો અને જરૂરી પ્રોગ્રામ
લખો.}

\begin{solutionbox}

\textbf{ડાયાગ્રામ:}

\begin{verbatim}
           +5V
            |
            v
Analog    +{-{-}{-}{-}{-}{-}{-}{-}+              +{-}{-}{-}{-}{-}{-}{-}{-}{-}+}
Input{-{-}{-}{-}|        |              |         |}
          | ADC0804|{-D0{-}D7{-}{-}{-}{-}{-}{-}{-}|P1.0{-}P1.7|}
          |        |              |         |
VREF/2{-{-}{-}|        |              |         |}
          |        |              |   8051  |
+5V{-{-}{-}{-}{-}{-}|Vcc     |              |         |}
          |        |{{-}{-}{-}{-}{-}{-}CS{-}{-}{-}{-}{-}| P3.0    |}
GND{-{-}{-}{-}{-}{-}|GND     |              |         |}
          |        |{{-}{-}{-}{-}{-}{-}RD{-}{-}{-}{-}{-}| P3.1    |}
+5V{-{-}{-}{-}{-}{-}|INTR    |              |         |}
          |        |{{-}{-}{-}{-}{-}{-}WR{-}{-}{-}{-}{-}| P3.2    |}
          +{-{-}{-}{-}{-}{-}{-}{-}+              +{-}{-}{-}{-}{-}{-}{-}{-}{-}+}
\end{verbatim}

\textbf{જરૂરી ઘટકો}:

\begin{itemize}
\tightlist
\item
  8051 માઇક્રોકન્ટ્રોલર
\item
  ADC0804 (8-બિટ એનાલોગ-ટુ-ડિજિટલ કન્વર્ટર)
\item
  રેફરન્સ વોલ્ટેજ સોર્સ
\item
  ઇનપુટ કન્ડિશનિંગ સર્કિટ (નથી બતાવેલ)
\end{itemize}

\textbf{કનેક્શન્સ}:

\begin{itemize}
\tightlist
\item
  P1.0-P1.7 \leftarrow D0-D7 (ADC માંથી 8-બિટ ડિજિટલ આઉટપુટ)
\item
  P3.0 \rightarrow CS (ચિપ સિલેક્ટ)
\item
  P3.1 \rightarrow RD (રીડ)
\item
  P3.2 \rightarrow WR (રાઈટ)
\end{itemize}

\textbf{એનાલોગ ઇનપુટ વાંચવા માટે પ્રોગ્રામ}:

\begin{verbatim}
START:  MOV P1, \#0FFH    ; P1 ને ઇનપુટ પોર્ટ તરીકે કન્ફિગર કરો
        
READ:   CLR P3.0         ; ADC એનેબલ કરો (CS = 0)
        CLR P3.2         ; કન્વર્ઝન શરૂ કરો (WR = 0)
        NOP              ; નાનો ડિલે
        NOP
        SETB P3.2        ; WR = 1
        
WAIT:   JB P3.3, WAIT    ; કન્વર્ઝન માટે રાહ જુઓ (INTR = 0)
        
        CLR P3.1         ; ડેટા વાંચવા માટે RD = 0
        MOV A, P1        ; કન્વર્ટ કરેલી વેલ્યુ વાંચો
        SETB P3.1        ; RD = 1
        SETB P3.0        ; ADC ડિસેબલ કરો (CS = 1)
        
PROCESS:                 ; જરૂરિયાત મુજબ ડેટા પ્રોસેસ કરો
        ; ઉદાહરણ: R0 માં સ્ટોર કરો
        MOV R0, A
        
        SJMP READ        ; સતત કન્વર્ઝન માટે રિપીટ કરો
\end{verbatim}

\textbf{કાર્યપ્રણાલી}:

\begin{enumerate}
\tightlist
\item
  કંટ્રોલર સ્ટાર્ટ કન્વર્ઝન સિગ્નલ મોકલે છે
\item
  ADC એનાલોગ ઇનપુટને 8-બિટ ડિજિટલ વેલ્યુમાં કન્વર્ટ કરે છે
\item
  કંટ્રોલર કન્વર્ઝન પૂર્ણ થયા પછી ડિજિટલ વેલ્યુ વાંચે છે
\item
  પ્રોગ્રામ જરૂરિયાત મુજબ ડિજિટલ વેલ્યુને પ્રોસેસ કરે છે
\end{enumerate}

\end{solutionbox}
\begin{mnemonicbox}
``CARSW'' - ``કન્વર્ટ એનાલોગ, રીડ ડિજિટલ, સ્ટાર્ટ
કન્વર્ઝન, વેઇટ ફોર કમ્પ્લીશન''

\end{mnemonicbox}

\end{document}
