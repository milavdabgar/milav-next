\documentclass[10pt,a4paper]{article}

% content/resources/templates/preamble.tex
\usepackage[margin=0.6in]{geometry}
\author{Milav Dabgar}
\usepackage{amsmath,amssymb,amsthm}
\usepackage{booktabs}
\usepackage{multirow}
\usepackage{xcolor}
\usepackage{tcolorbox}
\tcbuselibrary{breakable,skins}
\usepackage[colorlinks=true,linkcolor=blue]{hyperref}
\usepackage{titlesec}
\usepackage{enumitem}
\usepackage{tikz}
\usepackage{pgfplots}
\usepackage{circuitikz}
\usepackage[version=4]{mhchem}
\usepackage{longtable}
\usepackage{array}
\usepackage{float}
\usepackage{caption}
\usepackage{listings}

\lstset{
  basicstyle=\small\ttfamily,
  breaklines=true,
  breakatwhitespace=false,
  postbreak=\mbox{\textcolor{red}{$\hookrightarrow$}\space},
  float=false,
  numbers=left,
  numberstyle=\tiny\color{gray},
  numbersep=10pt,
  xleftmargin=2em,
  keywordstyle=\color{blue},
  commentstyle=\color{green!60!black},
  stringstyle=\color{purple},
  backgroundcolor=\color{gray!5},
  showstringspaces=false,
  tabsize=2,
  captionpos=b,
  keepspaces=true,
  columns=flexible
}

\pgfplotsset{compat=1.18}
\usetikzlibrary{shapes,arrows,positioning,calc,patterns,decorations.pathmorphing,decorations.markings,arrows.meta}

% Color scheme
\definecolor{headcolor}{RGB}{0,102,204}
\definecolor{keycolor}{RGB}{220,20,60}
\definecolor{solutioncolor}{RGB}{34,139,34}
\definecolor{mnemoniccolor}{RGB}{148,0,211}
\definecolor{codecolor}{RGB}{0,0,100}

% Spacing
\setlength{\parskip}{3pt}
\setlist[itemize]{nosep}
\setlist[enumerate]{nosep}

% Title formatting
\titleformat{\section}{\Large\bfseries\color{headcolor}}{\thesection}{1em}{}
\titleformat{\subsection}{\large\bfseries\color{headcolor}}{\thesubsection}{1em}{}

% Pandoc tightlist compatibility
\providecommand{\tightlist}{%
  \setlength{\itemsep}{0pt}\setlength{\parskip}{0pt}}

% Pandoc longtable compatibility
\newcounter{none}
\def\thenone{}


% content/resources/templates/english-boxes.tex
% This file is currently empty - it exists to maintain consistency with the import structure.
% Add custom environments here if needed in the future.


\begin{document}

\begin{center}
{\Huge\bfseries\color{headcolor} Subject Name Solutions}\\[5pt]
{\LARGE 4341101 -- Winter 2023}\\[3pt]
{\large Semester 1 Study Material}\\[3pt]
{\normalsize\textit{Detailed Solutions and Explanations}}
\end{center}

\vspace{10pt}

\subsection*{Question 1(a) [3 marks]}\label{q1a}

\textbf{Compare RISC and CISC.}

\begin{solutionbox}

{\def\LTcaptype{none} % do not increment counter
\begin{longtable}[]{@{}lll@{}}
\toprule\noalign{}
Feature & RISC & CISC \\
\midrule\noalign{}
\endhead
\bottomrule\noalign{}
\endlastfoot
Instructions & Simple, fixed-length & Complex, variable-length \\
Execution & Single cycle & Multiple cycles \\
Addressing modes & Few & Many \\
Registers & Many & Few \\
Design focus & Hardware simplicity & Code density \\
\end{longtable}
}

\end{solutionbox}
\begin{mnemonicbox}
``RISCs Complete Instructions Simply''

\end{mnemonicbox}
\subsection*{Question 1(b) [4 marks]}\label{q1b}

\textbf{Compare Von-Neumann and Harvard architecture.}

\begin{solutionbox}

{\def\LTcaptype{none} % do not increment counter
\begin{longtable}[]{@{}lll@{}}
\toprule\noalign{}
Feature & Von-Neumann & Harvard \\
\midrule\noalign{}
\endhead
\bottomrule\noalign{}
\endlastfoot
Memory & Single shared memory & Separate program \& data memory \\
Bus & Single bus for data \& instructions & Separate buses \\
Speed & Slower (memory bottleneck) & Faster (parallel access) \\
Complexity & Simpler design & More complex \\
Applications & General computing & Real-time systems \\
\end{longtable}
}

\textbf{Diagram:}

\begin{verbatim}
Von{-Neumann:}
+{-{-}{-}{-}{-}{-}{-}+         +{-}{-}{-}{-}{-}{-}{-}+}
| CPU   |{=======| Memory|}
+{-{-}{-}{-}{-}{-}{-}+         +{-}{-}{-}{-}{-}{-}{-}+}

Harvard:
+{-{-}{-}{-}{-}{-}{-}+         +{-}{-}{-}{-}{-}{-}{-}{-}{-}{-}{-}+}
| CPU   |========{| Program   |}
|       |         | Memory    |
|       |         +{-{-}{-}{-}{-}{-}{-}{-}{-}{-}{-}+}
|       |         +{-{-}{-}{-}{-}{-}{-}{-}{-}{-}{-}+}
|       |{=======| Data      |}
+{-{-}{-}{-}{-}{-}{-}+         | Memory    |}
                  +{-{-}{-}{-}{-}{-}{-}{-}{-}{-}{-}+}
\end{verbatim}

\end{solutionbox}
\begin{mnemonicbox}
``Harvard Has Separate Spaces''

\end{mnemonicbox}
\subsection*{Question 1(c) [7 marks]}\label{q1c}

\textbf{Explain: 8085 Instruction Format, Control Unit, Machine Cycle,
ALU}

\begin{solutionbox}

\textbf{Instruction Format:}

\begin{verbatim}
+{-{-}{-}{-}{-}{-}{-}{-}+{-}{-}{-}{-}{-}{-}{-}{-}+{-}{-}{-}{-}{-}{-}{-}{-}+}
| Opcode |Operand1|Operand2|
+{-{-}{-}{-}{-}{-}{-}{-}+{-}{-}{-}{-}{-}{-}{-}{-}+{-}{-}{-}{-}{-}{-}{-}{-}+}
  1{-3 bytes total length}
\end{verbatim}

{\def\LTcaptype{none} % do not increment counter
\begin{longtable}[]{@{}ll@{}}
\toprule\noalign{}
Component & Function \\
\midrule\noalign{}
\endhead
\bottomrule\noalign{}
\endlastfoot
\textbf{Instruction Format} & 1-3 byte structure with opcode and
operands \\
\textbf{Control Unit} & Fetches, decodes instructions; generates
signals \\
\textbf{Machine Cycle} & Basic operation cycle (T-states) \\
\textbf{ALU} & Performs arithmetic and logical operations \\
\end{longtable}
}

\begin{itemize}
\tightlist
\item
  \textbf{Instruction Format}: Contains opcode (3-8 bits) and 0-2
  operands
\item
  \textbf{Control Unit}: Heart of processor that coordinates all
  operations
\item
  \textbf{Machine Cycle}: Consists of fetch, decode, execute phases
\item
  \textbf{ALU}: Handles ADD/SUB/AND/OR/XOR operations on data
\end{itemize}

\textbf{Diagram:}

\begin{verbatim}
+{-{-}{-}{-}{-}{-}{-}{-}{-}{-}{-}{-}{-}{-}{-}{-}{-}{-}{-}{-}{-}{-}{-}{-}{-}{-}{-}{-}{-}{-}{-}{-}{-}{-}+}
|              8085                |
|  +{-{-}{-}{-}{-}{-}{-}{-}{-}{-}{-}{-}+  +{-}{-}{-}{-}{-}{-}{-}{-}{-}{-}{-}{-}+  |}
|  |Control Unit|{-|Instruction |  |}
|  |(Sequencer) |{{-}|Register    |  |}
|  +{-{-}{-}{-}{-}{-}{-}{-}{-}{-}{-}{-}+  +{-}{-}{-}{-}{-}{-}{-}{-}{-}{-}{-}{-}+  |}
|        |               |         |
|        v               v         |
|  +{-{-}{-}{-}{-}{-}{-}{-}{-}{-}{-}{-}+  +{-}{-}{-}{-}{-}{-}{-}{-}{-}{-}{-}{-}+  |}
|  |   ALU      |{{-}|Registers   |  |}
|  |            |{-|            |  |}
|  +{-{-}{-}{-}{-}{-}{-}{-}{-}{-}{-}{-}+  +{-}{-}{-}{-}{-}{-}{-}{-}{-}{-}{-}{-}+  |}
+{-{-}{-}{-}{-}{-}{-}{-}{-}{-}{-}{-}{-}{-}{-}{-}{-}{-}{-}{-}{-}{-}{-}{-}{-}{-}{-}{-}{-}{-}{-}{-}{-}{-}+}
\end{verbatim}

\end{solutionbox}
\begin{mnemonicbox}
``CIMA: Control Interprets, Machine Acts''

\end{mnemonicbox}
\subsection*{Question 1(c OR) [7
marks]}\label{question-1c-or-7-marks}

\textbf{Compare Microprocessor and Microcontroller.}

\begin{solutionbox}

{\def\LTcaptype{none} % do not increment counter
\begin{longtable}[]{@{}lll@{}}
\toprule\noalign{}
Feature & Microprocessor & Microcontroller \\
\midrule\noalign{}
\endhead
\bottomrule\noalign{}
\endlastfoot
Design & CPU only & CPU + peripherals \\
Memory & External & Internal (RAM/ROM) \\
I/O ports & Limited & Many built-in \\
Cost & Higher & Lower \\
Applications & General computing & Embedded systems \\
Power consumption & Higher & Lower \\
Example & Intel 8085/8086 & Intel 8051 \\
\end{longtable}
}

\textbf{Diagram:}

\begin{verbatim}
Microprocessor System:
+{-{-}{-}{-}{-}{-}{-}+    +{-}{-}{-}{-}{-}{-}{-}+    +{-}{-}{-}{-}{-}{-}{-}+}
| CPU   |{{-}{-}| Memory|{-}{-}| I/O   |}
+{-{-}{-}{-}{-}{-}{-}+    +{-}{-}{-}{-}{-}{-}{-}+    +{-}{-}{-}{-}{-}{-}{-}+}
    Separate chips needed

Microcontroller:
+{-{-}{-}{-}{-}{-}{-}{-}{-}{-}{-}{-}{-}{-}{-}{-}{-}{-}{-}{-}{-}{-}{-}{-}+}
| +{-{-}{-}{-}{-}{-}{-}+  +{-}{-}{-}{-}{-}{-}{-}+   |}
| | CPU   |  | Memory|   |
| +{-{-}{-}{-}{-}{-}{-}+  +{-}{-}{-}{-}{-}{-}{-}+   |}
|       |       |        |
|       v       v        |
| +{-{-}{-}{-}{-}{-}{-}{-}{-}{-}{-}{-}{-}{-}{-}{-}{-}{-}{-}+  |}
| | I/O, Timers, etc. |  |
| +{-{-}{-}{-}{-}{-}{-}{-}{-}{-}{-}{-}{-}{-}{-}{-}{-}{-}{-}+  |}
+{-{-}{-}{-}{-}{-}{-}{-}{-}{-}{-}{-}{-}{-}{-}{-}{-}{-}{-}{-}{-}{-}{-}{-}+}
    All in one chip
\end{verbatim}

\end{solutionbox}
\begin{mnemonicbox}
``Micro-P Processes, Micro-C Controls''

\end{mnemonicbox}
\subsection*{Question 2(a) [3 marks]}\label{q2a}

\textbf{Explain Instruction Fetching, Decoding and Execution Operation
in microprocessor.}

\begin{solutionbox}

{\def\LTcaptype{none} % do not increment counter
\begin{longtable}[]{@{}ll@{}}
\toprule\noalign{}
Phase & Operation \\
\midrule\noalign{}
\endhead
\bottomrule\noalign{}
\endlastfoot
Fetching & CPU gets instruction from memory using PC \\
Decoding & Determines operation type and operands \\
Execution & Performs the actual operation \\
\end{longtable}
}

\textbf{Diagram:}

\begin{verbatim}
+{-{-}{-}{-}{-}{-}{-}{-}+    +{-}{-}{-}{-}{-}{-}{-}{-}+    +{-}{-}{-}{-}{-}{-}{-}{-}+}
| Fetch  |{-{-}{-}| Decode |{-}{-}{-}|Execute |}
+{-{-}{-}{-}{-}{-}{-}{-}+    +{-}{-}{-}{-}{-}{-}{-}{-}+    +{-}{-}{-}{-}{-}{-}{-}{-}+}
\end{verbatim}

\end{solutionbox}
\begin{mnemonicbox}
``FDE: First Get, Then Understand, Finally Do''

\end{mnemonicbox}
\subsection*{Question 2(b) [4 marks]}\label{q2b}

\textbf{Explain Bus Organization of 8085 microprocessor.}

\begin{solutionbox}

{\def\LTcaptype{none} % do not increment counter
\begin{longtable}[]{@{}lll@{}}
\toprule\noalign{}
Bus Type & Width & Function \\
\midrule\noalign{}
\endhead
\bottomrule\noalign{}
\endlastfoot
Address Bus & 16-bit & Carries memory addresses (A0-A15) \\
Data Bus & 8-bit & Transfers data (D0-D7) \\
Control Bus & Various lines & Manages data flow (RD, WR, IO/M) \\
Multiplexed & AD0-AD7 & Lower address bits + data bits \\
\end{longtable}
}

\textbf{Diagram:}

\begin{verbatim}
8085 Microprocessor
    |
    |{-{-}{-}{-} Address Bus (16{-}bit) {-}{-}{-}{-} Memory}
    |                                 Location
    |{-{-}{-}{-} Data Bus (8{-}bit) {-}{-}{-}{-}{-}{-}{-}{-} Data}
    |
    |{-{-}{-}{-} Control Bus {-}{-}{-}{-}{-}{-}{-}{-}{-}{-}{-}{-}{-} Control}
                                     Signals
\end{verbatim}

\end{solutionbox}
\begin{mnemonicbox}
``ADC: Address points, Data flows, Control directs''

\end{mnemonicbox}
\subsection*{Question 2(c) [7 marks]}\label{q2c}

\textbf{Describe architecture of 8085 microprocessor with the help of
neat diagram.}

\begin{solutionbox}

{\def\LTcaptype{none} % do not increment counter
\begin{longtable}[]{@{}ll@{}}
\toprule\noalign{}
Component & Function \\
\midrule\noalign{}
\endhead
\bottomrule\noalign{}
\endlastfoot
ALU & Arithmetic \& logical operations \\
Register Array & Temporary data storage (B,C,D,E,H,L) \\
Accumulator & Main register for arithmetic \\
Control Unit & Instruction control \& timing \\
Instruction Register & Holds current instruction \\
Timing \& Control & Generates timing signals \\
Address Buffer & Manages address bus \\
Data Buffer & Handles data bus transfers \\
\end{longtable}
}

\textbf{Diagram:}

\begin{verbatim}
+{-{-}{-}{-}{-}{-}{-}{-}{-}{-}{-}{-}{-}{-}{-}{-}{-}{-}{-}{-}{-}{-}{-}{-}{-}{-}{-}{-}{-}{-}{-}{-}{-}{-}{-}{-}{-}{-}{-}{-}{-}{-}{-}{-}{-}{-}{-}{-}{-}{-}{-}{-}{-}{-}+}
|                  8085 MICROPROCESSOR                 |
| +{-{-}{-}{-}{-}{-}{-}{-}{-}{-}{-}{-}{-}{-}{-}{-}+     +{-}{-}{-}{-}{-}{-}{-}{-}{-}{-}{-}{-}{-}{-}{-}{-}{-}{-}{-}{-}{-}{-}{-}{-}{-}+   |}
| | REGISTER ARRAY |     |                         |   |
| |B  C  D  E  H  L|{{-}{-}{-}|          ALU            |   |}
| +{-{-}{-}{-}{-}{-}{-}{-}{-}{-}{-}{-}{-}{-}{-}{-}+     |                         |   |}
| +{-{-}{-}{-}{-}{-}{-}{-}{-}{-}{-}{-}{-}{-}{-}{-}+     |                         |   |}
| | ACCUMULATOR    |{{-}{-}{-}|                         |   |}
| +{-{-}{-}{-}{-}{-}{-}{-}{-}{-}{-}{-}{-}{-}{-}{-}+     +{-}{-}{-}{-}{-}{-}{-}{-}{-}{-}{-}{-}{-}{-}{-}{-}{-}{-}{-}{-}{-}{-}{-}{-}{-}+   |}
|                                                      |
| +{-{-}{-}{-}{-}{-}{-}{-}{-}{-}{-}{-}{-}{-}{-}{-}+     +{-}{-}{-}{-}{-}{-}{-}{-}{-}{-}{-}{-}{-}{-}{-}{-}{-}{-}{-}{-}{-}{-}{-}{-}{-}+   |}
| |INSTRUCTION REG.|{{-}{-}{-}|     CONTROL UNIT        |   |}
| +{-{-}{-}{-}{-}{-}{-}{-}{-}{-}{-}{-}{-}{-}{-}{-}+     +{-}{-}{-}{-}{-}{-}{-}{-}{-}{-}{-}{-}{-}{-}{-}{-}{-}{-}{-}{-}{-}{-}{-}{-}{-}+   |}
|                              |                       |
| +{-{-}{-}{-}{-}{-}{-}{-}{-}{-}{-}{-}{-}{-}{-}{-}+     +{-}{-}{-}{-}{-}|{-}{-}{-}{-}{-}{-}{-}{-}{-}{-}{-}{-}{-}{-}{-}{-}{-}{-}{-}{-}+  |}
| |ADDRESS BUFFER  |{{-}{-}{-}|   TIMING AND CONTROL     |  |}
| +{-{-}{-}{-}{-}{-}{-}{-}{-}{-}{-}{-}{-}{-}{-}{-}+     +{-}{-}{-}{-}{-}{-}{-}{-}{-}{-}{-}{-}{-}{-}{-}{-}{-}{-}{-}{-}{-}{-}{-}{-}{-}{-}+  |}
| +{-{-}{-}{-}{-}{-}{-}{-}{-}{-}{-}{-}{-}{-}{-}{-}+                                   |}
| |  DATA BUFFER   |{{-}{-}{-}                              |}
| +{-{-}{-}{-}{-}{-}{-}{-}{-}{-}{-}{-}{-}{-}{-}{-}+                                   |}
+{-{-}{-}{-}{-}{-}{-}{-}{-}{-}{-}{-}{-}{-}{-}{-}{-}{-}{-}{-}{-}{-}{-}{-}{-}{-}{-}{-}{-}{-}{-}{-}{-}{-}{-}{-}{-}{-}{-}{-}{-}{-}{-}{-}{-}{-}{-}{-}{-}{-}{-}{-}{-}{-}+}
\end{verbatim}

\begin{itemize}
\tightlist
\item
  \textbf{ALU}: Performs arithmetic \& logical operations
\item
  \textbf{Control Unit}: Fetches \& decodes instructions
\item
  \textbf{Registers}: Store data temporarily during processing
\item
  \textbf{Buses}: Carry address, data and control signals
\end{itemize}

\end{solutionbox}
\begin{mnemonicbox}
``ARCBD: Architecture Registers Control Buses Data''

\end{mnemonicbox}
\subsection*{Question 2(a OR) [3
marks]}\label{question-2a-or-3-marks}

\textbf{Explain De-multiplexing of Address and Data buses for 8085
Microprocessor.}

\begin{solutionbox}

{\def\LTcaptype{none} % do not increment counter
\begin{longtable}[]{@{}ll@{}}
\toprule\noalign{}
Step & Action \\
\midrule\noalign{}
\endhead
\bottomrule\noalign{}
\endlastfoot
1 & ALE signal goes high \\
2 & Lower address (A0-A7) appears on AD0-AD7 \\
3 & Latch captures address using ALE \\
4 & ALE goes low, AD0-AD7 now carries data \\
\end{longtable}
}

\textbf{Diagram:}

\begin{verbatim}
AD0{-AD7 {-}{-}{-}{-}+{-}{-}{-}{-}{-}{-} Latch {-}{-}{-}{-}{-} A0{-}A7}
            |         \^{}
            |         |
            v         |
          Data       ALE
\end{verbatim}

\end{solutionbox}
\begin{mnemonicbox}
``ALAD: ALE Latches Address before Data''

\end{mnemonicbox}
\subsection*{Question 2(b OR) [4
marks]}\label{question-2b-or-4-marks}

\textbf{Draw Flag Register of 8085 microprocessor \& explain it.}

\begin{solutionbox}

\begin{verbatim}
Flag Register (8{-bit):}
+{-{-}{-}+{-}{-}{-}+{-}{-}{-}+{-}{-}{-}+{-}{-}{-}+{-}{-}{-}+{-}{-}{-}+{-}{-}{-}+}
| S | Z | 0 | AC| 0 | P | 1 | CY|
+{-{-}{-}+{-}{-}{-}+{-}{-}{-}+{-}{-}{-}+{-}{-}{-}+{-}{-}{-}+{-}{-}{-}+{-}{-}{-}+}
  7   6   5   4   3   2   1   0
\end{verbatim}

{\def\LTcaptype{none} % do not increment counter
\begin{longtable}[]{@{}lll@{}}
\toprule\noalign{}
Flag & Name & Set when \\
\midrule\noalign{}
\endhead
\bottomrule\noalign{}
\endlastfoot
S & Sign & Bit 7 of result is 1 (negative) \\
Z & Zero & Result is zero \\
AC & Auxiliary Carry & Carry from bit 3 to bit 4 \\
P & Parity & Result has even parity (even 1s) \\
CY & Carry & Carry generated from bit 7 \\
\end{longtable}
}

\end{solutionbox}
\begin{mnemonicbox}
``SuZie AC's Perfect CarrY''

\end{mnemonicbox}
\subsection*{Question 2(c OR) [7
marks]}\label{question-2c-or-7-marks}

\textbf{Describe Pin diagram of 8085 microprocessor with the help of
neat diagram.}

\begin{solutionbox}

{\def\LTcaptype{none} % do not increment counter
\begin{longtable}[]{@{}ll@{}}
\toprule\noalign{}
Pin Group & Function \\
\midrule\noalign{}
\endhead
\bottomrule\noalign{}
\endlastfoot
Address/Data & Multiplexed AD0-AD7, A8-A15 \\
Control & RD, WR, IO/M, S0, S1, ALE, CLK \\
Interrupts & INTR, RST 5.5-7.5, TRAP \\
DMA & HOLD, HLDA \\
Power & Vcc, Vss \\
Serial I/O & SID, SOD \\
Reset & RESET IN, RESET OUT \\
\end{longtable}
}

\textbf{Diagram:}

\begin{verbatim}
            +{-{-}{-}{-}{-}{-}{-}+}
      X1 {-{-}|1    40|{-}{-} Vcc}
      X2 {-{-}|2    39|{-}{-} HOLD}
RESET OUT{-{-}|3    38|{-}{-} HLDA}
RESET IN {-{-}|4    37|{-}{-} CLK}
    IO/M {-{-}|5    36|{-}{-} RESET IN}
      S1 {-{-}|6    35|{-}{-} READY}
      RD {-{-}|7    34|{-}{-} IO/M}
      WR {-{-}|8    33|{-}{-} S1}
     ALE {-{-}|9    32|{-}{-} RD}
      S0 {-{-}|10   31|{-}{-} WR}
     A15 {-{-}|11   30|{-}{-} ALE}
     A14 {-{-}|12   29|{-}{-} S0}
     A13 {-{-}|13   28|{-}{-} A15}
     A12 {-{-}|14   27|{-}{-} A14}
     A11 {-{-}|15   26|{-}{-} A13}
     A10 {-{-}|16   25|{-}{-} A12}
      A9 {-{-}|17   24|{-}{-} A11}
      A8 {-{-}|18   23|{-}{-} A10}
     AD7 {-{-}|19   22|{-}{-} A9}
     AD6 {-{-}|20   21|{-}{-} A8}
            +{-{-}{-}{-}{-}{-}{-}+}
\end{verbatim}

\begin{itemize}
\tightlist
\item
  \textbf{Address/Data Pins}: Multiplexed pins save physical pins
\item
  \textbf{Control Pins}: Coordinate memory and I/O operations
\item
  \textbf{Interrupt Pins}: Allow external device interrupts
\item
  \textbf{Serial Pins}: Provide basic serial communication
\end{itemize}

\end{solutionbox}
\begin{mnemonicbox}
``ACID-PS:
Address-Control-Interrupt-DMA-Power-Serial''

\end{mnemonicbox}
\subsection*{Question 3(a) [3 marks]}\label{q3a}

\textbf{Explain Stack, Stack Pointer and Stack operation.}

\begin{solutionbox}

{\def\LTcaptype{none} % do not increment counter
\begin{longtable}[]{@{}ll@{}}
\toprule\noalign{}
Term & Description \\
\midrule\noalign{}
\endhead
\bottomrule\noalign{}
\endlastfoot
Stack & LIFO memory area for temporary data storage \\
Stack Pointer & 16-bit register that points to stack top \\
Operations & PUSH (store), POP (retrieve) \\
\end{longtable}
}

\textbf{Diagram:}

\begin{verbatim}
Memory:      Stack Pointer:
+{-{-}{-}{-}{-}+      +{-}{-}{-}{-}{-}+}
|     |{{-}{-}{-}{-} | SP  |}
+{-{-}{-}{-}{-}+      +{-}{-}{-}{-}{-}+}
| Data|      
+{-{-}{-}{-}{-}+      PUSH: SP{-}{-}, M[SP]=data}
| Data|      POP:  data=M[SP], SP++
+{-{-}{-}{-}{-}+}
\end{verbatim}

\end{solutionbox}
\begin{mnemonicbox}
``SP Points to LIFO Lane''

\end{mnemonicbox}
\subsection*{Question 3(b) [4 marks]}\label{q3b}

\textbf{Draw Pin diagram of 8051 microcontroller.}

\begin{solutionbox}

\begin{verbatim}
          8051 Microcontroller
         +{-{-}{-}{-}{-}{-}{-}{-}{-}{-}{-}{-}{-}{-}{-}{-}{-}{-}{-}+}
   P1.0{-{-}| 1              40 |{-}{-}VCC}
   P1.1{-{-}| 2              39 |{-}{-}P0.0/AD0}
   P1.2{-{-}| 3              38 |{-}{-}P0.1/AD1}
   P1.3{-{-}| 4              37 |{-}{-}P0.2/AD2}
   P1.4{-{-}| 5              36 |{-}{-}P0.3/AD3}
   P1.5{-{-}| 6              35 |{-}{-}P0.4/AD4}
   P1.6{-{-}| 7              34 |{-}{-}P0.5/AD5}
   P1.7{-{-}| 8              33 |{-}{-}P0.6/AD6}
   RST {-{-}| 9              32 |{-}{-}P0.7/AD7}
 P3.0/RXD| 10             31 |{-{-}EA/VPP}
 P3.1/TXD| 11             30 |{-{-}ALE/PROG}
P3.2/INT0| 12             29 |{-{-}PSEN}
P3.3/INT1| 13             28 |{-{-}P2.7/A15}
 P3.4/T0{-| 14             27 |{-}{-}P2.6/A14}
 P3.5/T1{-| 15             26 |{-}{-}P2.5/A13}
 P3.6/WR{-| 16             25 |{-}{-}P2.4/A12}
 P3.7/RD{-| 17             24 |{-}{-}P2.3/A11}
 XTAL2 {-{-}| 18             23 |{-}{-}P2.2/A10}
 XTAL1 {-{-}| 19             22 |{-}{-}P2.1/A9}
   VSS {-{-}| 20             21 |{-}{-}P2.0/A8}
         +{-{-}{-}{-}{-}{-}{-}{-}{-}{-}{-}{-}{-}{-}{-}{-}{-}{-}{-}+}
\end{verbatim}

{\def\LTcaptype{none} % do not increment counter
\begin{longtable}[]{@{}ll@{}}
\toprule\noalign{}
Pin Group & Function \\
\midrule\noalign{}
\endhead
\bottomrule\noalign{}
\endlastfoot
P0 & Port 0, multiplexed with address/data \\
P1 & Port 1, general purpose I/O \\
P2 & Port 2, upper address and I/O \\
P3 & Port 3, special functions and I/O \\
\end{longtable}
}

\end{solutionbox}
\begin{mnemonicbox}
``PORT 0123: Data-General-Address-Special''

\end{mnemonicbox}
\subsection*{Question 3(c) [7 marks]}\label{q3c}

\textbf{Draw Timers/Counters logic diagram of 8051 microcontroller and
explain its operation in various modes.}

\begin{solutionbox}

\textbf{Timer/Counter Diagram:}

\begin{verbatim}
         +{-{-}{-}{-}{-}{-}{-}{-}{-}{-}{-}{-}+}
TLx {-{-}{-}{-}|  8{-}bit     |       +{-}{-}{-}{-}{-}{-}{-}{-}{-}{-}{-}{-}{-}+}
         |  Register  |{-{-}{-}{-}{-}{-}|  8{-}bit      |}
         +{-{-}{-}{-}{-}{-}{-}{-}{-}{-}{-}{-}+       |  Register   |{-}{-}{-}{-} Interrupt}
                              |  (THx)      |
         +{-{-}{-}{-}{-}{-}{-}{-}{-}{-}{-}{-}+       +{-}{-}{-}{-}{-}{-}{-}{-}{-}{-}{-}{-}{-}+}
TRx {-{-}{-}{-}| Control    |             \^{}}
         | Logic      |             |
         +{-{-}{-}{-}{-}{-}{-}{-}{-}{-}{-}{-}+             |}
                \^{                   |}
                |                   |
                v                   v
         +{-{-}{-}{-}{-}{-}{-}{-}{-}{-}{-}{-}{-}{-}{-}{-}{-}{-}{-}{-}{-}+}
C/T {-{-}{-}{-}| Mode Control Logic  |{-}{-}{-}{-}{-} GATE}
         +{-{-}{-}{-}{-}{-}{-}{-}{-}{-}{-}{-}{-}{-}{-}{-}{-}{-}{-}{-}{-}+}
                   \^{}
                   |
INTx {-{-}{-}{-}{-}{-}{-}{-}{-}{-}{-}{-}{-}{-}}
\end{verbatim}

{\def\LTcaptype{none} % do not increment counter
\begin{longtable}[]{@{}ll@{}}
\toprule\noalign{}
Mode & Operation \\
\midrule\noalign{}
\endhead
\bottomrule\noalign{}
\endlastfoot
Mode 0 & 13-bit timer (5-bit TL, 8-bit TH) \\
Mode 1 & 16-bit timer (8-bit TL, 8-bit TH) \\
Mode 2 & 8-bit auto-reload (TL counts, TH reloads) \\
Mode 3 & Split timer (Timer 0 only) \\
\end{longtable}
}

\begin{itemize}
\tightlist
\item
  \textbf{Timer}: Uses internal clock, counts machine cycles
\item
  \textbf{Counter}: Uses external input, counts external events
\item
  \textbf{Control Bits}: TMOD register sets mode, TCON controls
  operation
\item
  \textbf{Modes}: Different configurations for different timing needs
\end{itemize}

\end{solutionbox}
\begin{mnemonicbox}
``MARC: Mode Auto-Reload Count''

\end{mnemonicbox}
\subsection*{Question 3(a OR) [3
marks]}\label{question-3a-or-3-marks}

\textbf{List Common features of Microcontrollers.}

\begin{solutionbox}

{\def\LTcaptype{none} % do not increment counter
\begin{longtable}[]{@{}ll@{}}
\toprule\noalign{}
Feature & Purpose \\
\midrule\noalign{}
\endhead
\bottomrule\noalign{}
\endlastfoot
CPU Core & Process instructions \\
Memory (RAM/ROM) & Store program and data \\
I/O Ports & Interface with external devices \\
Timers/Counters & Measure time intervals \\
Interrupts & Handle asynchronous events \\
Serial Communication & Transfer data with other devices \\
\end{longtable}
}

\end{solutionbox}
\begin{mnemonicbox}
``CRITICS: CPU ROM I/O Timers Interrupts Comm
Serial''

\end{mnemonicbox}
\subsection*{Question 3(b OR) [4
marks]}\label{question-3b-or-4-marks}

\textbf{Explain Internal RAM Organization of 8051 microcontroller.}

\begin{solutionbox}

{\def\LTcaptype{none} % do not increment counter
\begin{longtable}[]{@{}lll@{}}
\toprule\noalign{}
RAM Area & Address Range & Usage \\
\midrule\noalign{}
\endhead
\bottomrule\noalign{}
\endlastfoot
Register Banks & 00H-1FH & R0-R7 (4 banks) \\
Bit-addressable & 20H-2FH & 128 bits (0-7FH) \\
Scratch Pad & 30H-7FH & General purpose \\
SFRs & 80H-FFH & Control registers \\
\end{longtable}
}

\textbf{Diagram:}

\begin{verbatim}
8051 Internal RAM (128 bytes):
+{-{-}{-}{-}{-}{-}{-}{-}{-}{-}{-}{-}{-}{-}{-}{-}+ 00H}
| Register Bank 0|
+{-{-}{-}{-}{-}{-}{-}{-}{-}{-}{-}{-}{-}{-}{-}{-}+ 08H}
| Register Bank 1|
+{-{-}{-}{-}{-}{-}{-}{-}{-}{-}{-}{-}{-}{-}{-}{-}+ 10H}
| Register Bank 2|
+{-{-}{-}{-}{-}{-}{-}{-}{-}{-}{-}{-}{-}{-}{-}{-}+ 18H}
| Register Bank 3|
+{-{-}{-}{-}{-}{-}{-}{-}{-}{-}{-}{-}{-}{-}{-}{-}+ 20H}
| Bit{-addressable|}
+{-{-}{-}{-}{-}{-}{-}{-}{-}{-}{-}{-}{-}{-}{-}{-}+ 30H}
|                |
| Scratch Pad    |
|                |
+{-{-}{-}{-}{-}{-}{-}{-}{-}{-}{-}{-}{-}{-}{-}{-}+ 80H}
\end{verbatim}

\end{solutionbox}
\begin{mnemonicbox}
``RBBS: Registers Bits Buffer Scratch''

\end{mnemonicbox}
\subsection*{Question 3(c OR) [7
marks]}\label{question-3c-or-7-marks}

\textbf{Explain architecture of 8051 microcontroller with the help of
neat diagram.}

\begin{solutionbox}

{\def\LTcaptype{none} % do not increment counter
\begin{longtable}[]{@{}ll@{}}
\toprule\noalign{}
Component & Function \\
\midrule\noalign{}
\endhead
\bottomrule\noalign{}
\endlastfoot
CPU & 8-bit processor with ALU \\
Memory & 4K ROM, 128 bytes RAM \\
I/O Ports & Four 8-bit ports (P0-P3) \\
Timers & Two 16-bit timers/counters \\
Serial Port & Full-duplex UART \\
Interrupts & Five interrupt sources \\
Special Function Registers & Control registers \\
\end{longtable}
}

\textbf{Diagram:}

\begin{verbatim}
+{-{-}{-}{-}{-}{-}{-}{-}{-}{-}{-}{-}{-}{-}{-}{-}{-}{-}{-}{-}{-}{-}{-}{-}{-}{-}{-}{-}{-}{-}{-}{-}{-}{-}{-}{-}{-}{-}{-}{-}{-}{-}{-}{-}+}
|                 8051 MCU                   |
| +{-{-}{-}{-}{-}{-}{-}{-}{-}{-}{-}{-}{-}+         +{-}{-}{-}{-}{-}{-}{-}{-}{-}{-}{-}{-}{-}{-}+   |}
| |             |         |              |   |
| |    CPU      |{{-}{-}{-}{-}{-}{-}{-}| Program ROM  |   |}
| |             |         | (4K bytes)   |   |
| +{-{-}{-}{-}{-}{-}{-}{-}{-}{-}{-}{-}{-}+         +{-}{-}{-}{-}{-}{-}{-}{-}{-}{-}{-}{-}{-}{-}+   |}
|       \^{                                    |}
|       |                 +{-{-}{-}{-}{-}{-}{-}{-}{-}{-}{-}{-}{-}{-}+   |}
|       |                 |              |   |
|       +{-{-}{-}{-}{-}{-}{-}{-}{-}{-}{-}{-}{-}{-}{-}{-}| Internal RAM |   |}
|       |                 | (128 bytes)  |   |
|       v                 +{-{-}{-}{-}{-}{-}{-}{-}{-}{-}{-}{-}{-}{-}+   |}
| +{-{-}{-}{-}{-}{-}{-}{-}{-}{-}{-}{-}{-}+         +{-}{-}{-}{-}{-}{-}{-}{-}{-}{-}{-}{-}{-}{-}+   |}
| |             |         |              |   |
| |  SFRs       |{{-}{-}{-}{-}{-}{-}{-}| I/O Ports    |   |}
| |             |         | (P0,P1,P2,P3)|   |
| +{-{-}{-}{-}{-}{-}{-}{-}{-}{-}{-}{-}{-}+         +{-}{-}{-}{-}{-}{-}{-}{-}{-}{-}{-}{-}{-}{-}+   |}
|                                            |
| +{-{-}{-}{-}{-}{-}{-}{-}{-}{-}{-}{-}{-}+         +{-}{-}{-}{-}{-}{-}{-}{-}{-}{-}{-}{-}{-}{-}+   |}
| |             |         |              |   |
| | Timers/     |         | Serial Port  |   |
| | Counters    |         | (UART)       |   |
| +{-{-}{-}{-}{-}{-}{-}{-}{-}{-}{-}{-}{-}+         +{-}{-}{-}{-}{-}{-}{-}{-}{-}{-}{-}{-}{-}{-}+   |}
+{-{-}{-}{-}{-}{-}{-}{-}{-}{-}{-}{-}{-}{-}{-}{-}{-}{-}{-}{-}{-}{-}{-}{-}{-}{-}{-}{-}{-}{-}{-}{-}{-}{-}{-}{-}{-}{-}{-}{-}{-}{-}{-}{-}+}
\end{verbatim}

\begin{itemize}
\tightlist
\item
  \textbf{Harvard Architecture}: Separate program and data memory
\item
  \textbf{CISC Design}: Rich instruction set (over 100 instructions)
\item
  \textbf{In-built Peripherals}: No need for external components
\item
  \textbf{Single-chip Solution}: Complete system on one chip
\end{itemize}

\end{solutionbox}
\begin{mnemonicbox}
``CAPITALS: CPU Architecture Ports I/O Timer ALU
LS-Interface Serial''

\end{mnemonicbox}
\subsection*{Question 4(a) [3 marks]}\label{q4a}

\textbf{Write an 8051 Assembly Language Program to Copy the data from
external RAM Location 0123h to TL0 and Data from external RAM location
0234h to TH0.}

\begin{solutionbox}

\begin{verbatim}
      MOV  DPTR, \#0123H   ; Load DPTR with source address 0123H
      MOVX A, @DPTR       ; Read data from external RAM
      MOV  TL0, A         ; Copy to Timer 0 low byte
      
      MOV  DPTR, \#0234H   ; Load DPTR with source address 0234H
      MOVX A, @DPTR       ; Read data from external RAM
      MOV  TH0, A         ; Copy to Timer 0 high byte
\end{verbatim}

\textbf{Key Steps:}

\begin{itemize}
\tightlist
\item
  Use DPTR to address external RAM
\item
  MOVX instruction for external memory access
\item
  Direct transfer to timer registers
\end{itemize}

\end{solutionbox}
\begin{mnemonicbox}
``DRAM: DPTR Read Address Move''

\end{mnemonicbox}
\subsection*{Question 4(b) [4 marks]}\label{q4b}

\textbf{Write an 8051 Assembly Language Program to blink LED interfaced
at port P1.3 at time interval of 1ms.}

\begin{solutionbox}

\begin{verbatim}
AGAIN:  SETB P1.3         ; Turn ON LED at P1.3
        ACALL DELAY       ; Call delay subroutine
        CLR  P1.3         ; Turn OFF LED at P1.3
        ACALL DELAY       ; Call delay subroutine
        SJMP AGAIN        ; Repeat forever

DELAY:  MOV  R7, \#250     ; Load R7 for outer loop
OUTER:  MOV  R6, \#1       ; Load R6 for inner loop
INNER:  DJNZ R6, INNER    ; Decrement R6 until zero
        DJNZ R7, OUTER    ; Decrement R7 until zero
        RET               ; Return from subroutine
\end{verbatim}

\textbf{Key Steps:}

\begin{itemize}
\tightlist
\item
  Toggle P1.3 pin to blink LED
\item
  Nested delay loop for timing
\item
  Infinite loop for continuous blinking
\end{itemize}

\end{solutionbox}
\begin{mnemonicbox}
``STACI: Set-Timer-And-Clear-Infinitely''

\end{mnemonicbox}
\subsection*{Question 4(c) [7 marks]}\label{q4c}

\textbf{List Addressing Modes of 8051 Microcontroller and explain all of
them with the help of example.}

\begin{solutionbox}

{\def\LTcaptype{none} % do not increment counter
\begin{longtable}[]{@{}lll@{}}
\toprule\noalign{}
Addressing Mode & Example & Description \\
\midrule\noalign{}
\endhead
\bottomrule\noalign{}
\endlastfoot
Immediate & MOV A, \#25H & Data is in instruction \\
Register & MOV A, R0 & Data is in register \\
Direct & MOV A, 30H & Data is at RAM address \\
Indirect & MOV A, @R0 & R0/R1 contains address \\
Indexed & MOVC A, @A+DPTR & Access program memory \\
Bit & SETB P1.3 & Access individual bits \\
Relative & SJMP LABEL & Jumps with 8-bit offset \\
\end{longtable}
}

\textbf{Examples:}

\begin{itemize}
\tightlist
\item
  \textbf{Immediate}: \texttt{MOV\ A,\ \#55H} (Load A with 55H)
\item
  \textbf{Register}: \texttt{ADD\ A,\ R3} (Add R3 to A)
\item
  \textbf{Direct}: \texttt{MOV\ 40H,\ A} (Store A at address 40H)
\item
  \textbf{Indirect}: \texttt{MOV\ @R0,\ \#5} (Store 5 at address in R0)
\item
  \textbf{Indexed}: \texttt{MOVC\ A,\ @A+DPTR} (Read code memory)
\item
  \textbf{Bit}: \texttt{CLR\ C} (Clear carry flag)
\item
  \textbf{Relative}: \texttt{JZ\ LOOP} (Jump if A is zero)
\end{itemize}

\end{solutionbox}
\begin{mnemonicbox}
``I'M DIRBI: Immediate Register Direct Bit Indexed''

\end{mnemonicbox}
\subsection*{Question 4(a OR) [3
marks]}\label{question-4a-or-3-marks}

\textbf{Write an 8051 Assembly Language Program to Subtract the content
of RAM location 11h from RAM location 14h; put result in RAM location
3Ch.}

\begin{solutionbox}

\begin{verbatim}
      MOV  A, 14H       ; Load content of RAM location 14H to A
      CLR  C            ; Clear carry flag
      SUBB A, 11H       ; Subtract content of 11H with borrow
      MOV  3CH, A       ; Store result in RAM location 3CH
\end{verbatim}

\textbf{Key Steps:}

\begin{itemize}
\tightlist
\item
  Load minuend into accumulator
\item
  Clear carry for correct subtraction
\item
  Use SUBB for subtraction with borrow
\item
  Store result in destination
\end{itemize}

\end{solutionbox}
\begin{mnemonicbox}
``LCSS: Load-Clear-Subtract-Store''

\end{mnemonicbox}
\subsection*{Question 4(b OR) [4
marks]}\label{question-4b-or-4-marks}

\textbf{Write an 8051 Assembly Language Program to generate a square
wave of 50\% duty cycle on bit 3 of Port 1 using Timer 0 in Mode 1.}

\begin{solutionbox}

\begin{verbatim}
      MOV  TMOD, \#01H   ; Timer 0, Mode 1 (16{-bit)}
AGAIN: MOV  TH0, \#0FCH   ; Load high byte
      MOV  TL0, \#18H    ; Load low byte ({-1000 in 16{-}bit)}
      SETB TR0          ; Start timer
      JNB  TF0, $       ; Wait for overflow
      CLR  TR0          ; Stop timer
      CLR  TF0          ; Clear timer flag
      CPL  P1.3         ; Toggle P1.3
      SJMP AGAIN        ; Repeat
\end{verbatim}

\textbf{Key Steps:}

\begin{itemize}
\tightlist
\item
  Configure Timer 0 in Mode 1
\item
  Preload timer with value for 1ms delay
\item
  Wait for timer overflow
\item
  Toggle output bit for square wave
\end{itemize}

\end{solutionbox}
\begin{mnemonicbox}
``MSTCCS: Mode-Set-Timer-Check-Clear-Switch''

\end{mnemonicbox}
\subsection*{Question 4(c OR) [7
marks]}\label{question-4c-or-7-marks}

\textbf{Explain any seven Logical Instructions with example for 8051
Microcontroller.}

\begin{solutionbox}

{\def\LTcaptype{none} % do not increment counter
\begin{longtable}[]{@{}lll@{}}
\toprule\noalign{}
Instruction & Example & Operation \\
\midrule\noalign{}
\endhead
\bottomrule\noalign{}
\endlastfoot
ANL & ANL A, \#3FH & Logical AND \\
ORL & ORL P1, \#80H & Logical OR \\
XRL & XRL A, R0 & Logical XOR \\
CLR & CLR A & Clear (set to 0) \\
CPL & CPL P1.0 & Complement (invert) \\
RL & RL A & Rotate left \\
RR & RR A & Rotate right \\
\end{longtable}
}

\textbf{Examples:}

\begin{itemize}
\tightlist
\item
  \textbf{ANL}: \texttt{ANL\ A,\ \#0FH} (A = A AND 0FH, masks high
  nibble)
\item
  \textbf{ORL}: \texttt{ORL\ 20H,\ A} (20H = 20H OR A, sets bits)
\item
  \textbf{XRL}: \texttt{XRL\ A,\ \#55H} (A = A XOR 55H, toggles bits)
\item
  \textbf{CLR}: \texttt{CLR\ C} (Clear carry flag, C = 0)
\item
  \textbf{CPL}: \texttt{CPL\ A} (Complement A, A = NOT A)
\item
  \textbf{RL}: \texttt{RL\ A} (Rotate A left one bit)
\item
  \textbf{RR}: \texttt{RR\ A} (Rotate A right one bit)
\end{itemize}

\end{solutionbox}
\begin{mnemonicbox}
``A-OX-CCR: AND OR XOR Clear Complement Rotate''

\end{mnemonicbox}
\subsection*{Question 5(a) [3 marks]}\label{q5a}

\textbf{Draw Interfacing of Push button Switch with 8051
microcontroller.}

\begin{solutionbox}

\begin{verbatim}
         Vcc
          |
          R (10K)
          |
P1.0 {-{-}{-}{-}{-}+{-}{-}{-}{-}{-}{-} Push Button {-}{-}{-}{-}{-}{-} GND}
\end{verbatim}

{\def\LTcaptype{none} % do not increment counter
\begin{longtable}[]{@{}ll@{}}
\toprule\noalign{}
Component & Connection \\
\midrule\noalign{}
\endhead
\bottomrule\noalign{}
\endlastfoot
Push Button & Between P1.0 and GND \\
Pull-up Resistor & 10K between P1.0 and VCC \\
Port Pin & P1.0 configured as input \\
\end{longtable}
}

\textbf{Key Points:}

\begin{itemize}
\tightlist
\item
  Active-low configuration (button press gives 0)
\item
  Pull-up resistor prevents floating input
\item
  Can connect to any I/O pin
\end{itemize}

\end{solutionbox}
\begin{mnemonicbox}
``PIP: Pull-up-Input-Press''

\end{mnemonicbox}
\subsection*{Question 5(b) [4 marks]}\label{q5b}

\textbf{Interface Relay with 8051 microcontroller.}

\begin{solutionbox}

\begin{verbatim}
                 5V
                 |
                 R (1K)
                 |
                 |   C (Diode)
                 |   |
P1.0 {-{-}{-}R(330){-}{-}{-}+{-}{-}{-}||{-}{-}{-}{-}+}
                     |      |
                     |      |
                 +{-{-}{-}+{-}{-}{-}+  |}
                 | NPN   |  |
                 |(BC547)|  |
                 +{-{-}{-}+{-}{-}{-}+  |}
                     |      |
                     |      |
                    GND     |
                            |
                         +{-{-}+{-}{-}+}
                         |Relay|{-{-}{-} Load}
                         +{-{-}{-}{-}{-}+}
\end{verbatim}

{\def\LTcaptype{none} % do not increment counter
\begin{longtable}[]{@{}ll@{}}
\toprule\noalign{}
Component & Purpose \\
\midrule\noalign{}
\endhead
\bottomrule\noalign{}
\endlastfoot
NPN Transistor & Current amplification \\
Diode & Back EMF protection \\
Resistors & Current limiting \\
Relay & High-power switching \\
\end{longtable}
}

\textbf{Key Steps:}

\begin{itemize}
\tightlist
\item
  Port pin drives transistor base
\item
  Transistor switches relay coil
\item
  Diode protects against back EMF
\item
  Relay contacts switch high-power load
\end{itemize}

\end{solutionbox}
\begin{mnemonicbox}
``TRIP: Transistor-Relay-Interface-Protection''

\end{mnemonicbox}
\subsection*{Question 5(c) [7 marks]}\label{q5c}

\textbf{Interface ADC0804 with 8051 microcontroller.}

\begin{solutionbox}

\textbf{Circuit Diagram:}

\begin{verbatim}
                     8051
                 +{-{-}{-}{-}{-}{-}{-}{-}{-}{-}+}
                 |          |
 Analog Input{-{-}{-}| ADC0804  |}
 0{-5V        |   |          |}
             v   |          |
        +{-{-}{-}+{-}{-}{-}+|          |     +{-}{-}{-}{-}{-}{-}{-}{-}{-}+}
        |        |          |     |         |
Vref/2{-|        |          |     |         |}
        |        |          |{{-}{-}{-}|P1.0{-}P1.7|}
CS{-{-}{-}{-}{-}|        |          |     |         |}
RD{-{-}{-}{-}{-}|        |          |     |         |}
WR{-{-}{-}{-}{-}|        |          |     |         |}
INTR{-{-}{-}|        |          |{-}{-}{-}{-}|P3.2     |}
        |        |          |     |         |
        +{-{-}{-}{-}{-}{-}{-}{-}+          |     |         |}
                 +{-{-}{-}{-}{-}{-}{-}{-}{-}{-}+     +{-}{-}{-}{-}{-}{-}{-}{-}{-}+}
\end{verbatim}

{\def\LTcaptype{none} % do not increment counter
\begin{longtable}[]{@{}lll@{}}
\toprule\noalign{}
Connection & 8051 Pin & ADC0804 Pin \\
\midrule\noalign{}
\endhead
\bottomrule\noalign{}
\endlastfoot
Data Bus & P1.0-P1.7 & D0-D7 \\
CS & P3.0 & CS \\
RD & P3.1 & RD \\
WR & P3.2 & WR \\
INTR & P3.3 & INTR \\
\end{longtable}
}

\begin{itemize}
\tightlist
\item
  \textbf{ADC0804}: 8-bit A/D converter with 0-5V input range
\item
  \textbf{Interface}: Connect data pins to Port 1, control to Port 3
\item
  \textbf{Operation}: Write to ADC to start conversion, wait for INTR,
  read result
\item
  \textbf{Resolution}: 8-bit (256 steps) for 0-5V gives
  \textasciitilde19.5mV per step
\end{itemize}

\end{solutionbox}
\begin{mnemonicbox}
``CRIW: Control-Read-Interrupt-Write''

\end{mnemonicbox}
\subsection*{Question 5(a OR) [3
marks]}\label{question-5a-or-3-marks}

\textbf{List Applications of microcontroller in various fields.}

\begin{solutionbox}

{\def\LTcaptype{none} % do not increment counter
\begin{longtable}[]{@{}ll@{}}
\toprule\noalign{}
Field & Applications \\
\midrule\noalign{}
\endhead
\bottomrule\noalign{}
\endlastfoot
Industrial & Motor control, automation, PLCs \\
Medical & Patient monitoring, diagnostic equipment \\
Consumer & Washing machines, microwaves, toys \\
Automotive & Engine control, ABS, airbag systems \\
Communication & Mobile phones, modems, routers \\
Security & Access control, alarm systems \\
\end{longtable}
}

\end{solutionbox}
\begin{mnemonicbox}
``I-MACS:
Industrial-Medical-Automotive-Consumer-Security''

\end{mnemonicbox}
\subsection*{Question 5(b OR) [4
marks]}\label{question-5b-or-4-marks}

\textbf{Interface Stepper motor with 8051 microcontroller.}

\begin{solutionbox}

\textbf{Circuit Diagram:}

\begin{verbatim}
          8051                      ULN2003
       +{-{-}{-}{-}{-}{-}{-}{-}+                 +{-}{-}{-}{-}{-}{-}{-}{-}{-}+}
       |        |           +{-{-}{-}{-}|IN1  OUT1|{-}{-}{-}+}
       |   P1.0 |{-{-}{-}{-}{-}{-}{-}{-}{-}{-}{-}|{-}{-}{-}{-}|IN2  OUT2|{-}{-}{-}+}
       |   P1.1 |{-{-}{-}{-}{-}{-}{-}{-}{-}{-}{-}|{-}{-}{-}{-}|IN3  OUT3|{-}{-}{-}+{-}{-}{-}{-} 4{-}wire}
       |   P1.2 |{-{-}{-}{-}{-}{-}{-}{-}{-}{-}{-}|{-}{-}{-}{-}|IN4  OUT4|{-}{-}{-}+      Stepper}
       |   P1.3 |{-{-}{-}{-}{-}{-}{-}{-}{-}{-}{-}|     |         |           Motor}
       |        |           |     |         |
       +{-{-}{-}{-}{-}{-}{-}{-}+           |     +{-}{-}{-}{-}{-}{-}{-}{-}{-}+}
                            |
                          +5V
\end{verbatim}

{\def\LTcaptype{none} % do not increment counter
\begin{longtable}[]{@{}ll@{}}
\toprule\noalign{}
Component & Purpose \\
\midrule\noalign{}
\endhead
\bottomrule\noalign{}
\endlastfoot
ULN2003 & Driver IC with Darlington arrays \\
Port pins & P1.0-P1.3 for 4 motor phases \\
Power supply & Separate supply for motor \\
\end{longtable}
}

\textbf{Code Structure:}

\begin{verbatim}
; Clockwise Rotation Sequence
STEP\_SEQ: DB 0000\_1000B  ; Step 1
          DB 0000\_1100B  ; Step 2
          DB 0000\_0100B  ; Step 3
          DB 0000\_0110B  ; Step 4
\end{verbatim}

\end{solutionbox}
\begin{mnemonicbox}
``PDCS: Port-Driver-Current-Sequence''

\end{mnemonicbox}
\subsection*{Question 5(c OR) [7
marks]}\label{question-5c-or-7-marks}

\textbf{Interface LCD with 8051 microcontroller.}

\begin{solutionbox}

\textbf{Circuit Diagram:}

\begin{verbatim}
           8051                  16x2 LCD
       +{-{-}{-}{-}{-}{-}{-}{-}+              +{-}{-}{-}{-}{-}{-}{-}{-}{-}+}
       |        |              |         |
       |   P2.0 |{-{-}{-}{-}{-}{-}{-}{-}{-}{-}{-}{-}{-}|D0       |}
       |   P2.1 |{-{-}{-}{-}{-}{-}{-}{-}{-}{-}{-}{-}{-}|D1       |}
       |   P2.2 |{-{-}{-}{-}{-}{-}{-}{-}{-}{-}{-}{-}{-}|D2       |}
       |   P2.3 |{-{-}{-}{-}{-}{-}{-}{-}{-}{-}{-}{-}{-}|D3       |}
       |   P2.4 |{-{-}{-}{-}{-}{-}{-}{-}{-}{-}{-}{-}{-}|D4       |}
       |   P2.5 |{-{-}{-}{-}{-}{-}{-}{-}{-}{-}{-}{-}{-}|D5       |}
       |   P2.6 |{-{-}{-}{-}{-}{-}{-}{-}{-}{-}{-}{-}{-}|D6       |}
       |   P2.7 |{-{-}{-}{-}{-}{-}{-}{-}{-}{-}{-}{-}{-}|D7       |}
       |        |              |         |
       |   P3.0 |{-{-}{-}{-}{-}{-}{-}{-}{-}{-}{-}{-}{-}|RS       |}
       |   P3.1 |{-{-}{-}{-}{-}{-}{-}{-}{-}{-}{-}{-}{-}|R/W      |}
       |   P3.2 |{-{-}{-}{-}{-}{-}{-}{-}{-}{-}{-}{-}{-}|E        |}
       |        |              |         |
       +{-{-}{-}{-}{-}{-}{-}{-}+              +{-}{-}{-}{-}{-}{-}{-}{-}{-}+}
                                  |   |
                                 Vcc GND
\end{verbatim}

{\def\LTcaptype{none} % do not increment counter
\begin{longtable}[]{@{}ll@{}}
\toprule\noalign{}
Connection & Purpose \\
\midrule\noalign{}
\endhead
\bottomrule\noalign{}
\endlastfoot
Data Pins (D0-D7) & Connect to P2.0-P2.7 \\
RS & Register Select (0=Command, 1=Data) \\
R/W & Read/Write (0=Write, 1=Read) \\
E & Enable signal (Active High) \\
\end{longtable}
}

\textbf{Basic Commands:}

\begin{verbatim}
0x01 - Clear Display
0x02 - Return Home
0x0C - Display ON, Cursor OFF
0x38 - 8-bit, 2 line, 5x7 dots
\end{verbatim}

\begin{itemize}
\tightlist
\item
  \textbf{Initialization}: Configure LCD for 8-bit mode, 2 lines
\item
  \textbf{Writing}: Send data with RS=1, control with RS=0
\item
  \textbf{Timing}: E pulse must meet timing requirements
\item
  \textbf{Contrast}: Adjust with potentiometer on VEE pin
\end{itemize}

\end{solutionbox}
\begin{mnemonicbox}
``DICE: Data-Instruction-Control-Enable''

\end{mnemonicbox}

\end{document}
