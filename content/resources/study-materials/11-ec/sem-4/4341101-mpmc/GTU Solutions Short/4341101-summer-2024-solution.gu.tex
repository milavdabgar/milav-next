\documentclass{article}

% content/resources/templates/preamble.tex
\usepackage[margin=0.6in]{geometry}
\author{Milav Dabgar}
\usepackage{amsmath,amssymb,amsthm}
\usepackage{booktabs}
\usepackage{multirow}
\usepackage{xcolor}
\usepackage{tcolorbox}
\tcbuselibrary{breakable,skins}
\usepackage[colorlinks=true,linkcolor=blue]{hyperref}
\usepackage{titlesec}
\usepackage{enumitem}
\usepackage{tikz}
\usepackage{pgfplots}
\usepackage{circuitikz}
\usepackage[version=4]{mhchem}
\usepackage{longtable}
\usepackage{array}
\usepackage{float}
\usepackage{caption}
\usepackage{listings}

\lstset{
  basicstyle=\small\ttfamily,
  breaklines=true,
  breakatwhitespace=false,
  postbreak=\mbox{\textcolor{red}{$\hookrightarrow$}\space},
  float=false,
  numbers=left,
  numberstyle=\tiny\color{gray},
  numbersep=10pt,
  xleftmargin=2em,
  keywordstyle=\color{blue},
  commentstyle=\color{green!60!black},
  stringstyle=\color{purple},
  backgroundcolor=\color{gray!5},
  showstringspaces=false,
  tabsize=2,
  captionpos=b,
  keepspaces=true,
  columns=flexible
}

\pgfplotsset{compat=1.18}
\usetikzlibrary{shapes,arrows,positioning,calc,patterns,decorations.pathmorphing,decorations.markings,arrows.meta}

% Color scheme
\definecolor{headcolor}{RGB}{0,102,204}
\definecolor{keycolor}{RGB}{220,20,60}
\definecolor{solutioncolor}{RGB}{34,139,34}
\definecolor{mnemoniccolor}{RGB}{148,0,211}
\definecolor{codecolor}{RGB}{0,0,100}

% Spacing
\setlength{\parskip}{3pt}
\setlist[itemize]{nosep}
\setlist[enumerate]{nosep}

% Title formatting
\titleformat{\section}{\Large\bfseries\color{headcolor}}{\thesection}{1em}{}
\titleformat{\subsection}{\large\bfseries\color{headcolor}}{\thesubsection}{1em}{}

% Pandoc tightlist compatibility
\providecommand{\tightlist}{%
  \setlength{\itemsep}{0pt}\setlength{\parskip}{0pt}}

% Pandoc longtable compatibility
\newcounter{none}
\def\thenone{}


% content/resources/templates/gujarati-boxes.tex
\usepackage{fontspec}
\usepackage{polyglossia}

% Set Gujarati as main language (document is primarily in Gujarati)
% Note: gloss-gujarati.ldf doesn't exist in polyglossia, but it will use hyphenation patterns
\setdefaultlanguage{gujarati}
\setotherlanguage{english}

% Configure Gujarati font properly
% Use Language=Default to prevent polyglossia from trying to add language-specific features
% that don't exist for Gujarati, which causes "empty feature" warnings
\newfontfamily\gujaratifont[Script=Gujarati,AutoFakeBold=2.5,AutoFakeSlant=0.3]{Noto Sans Gujarati}
\setmainfont[Script=Gujarati,AutoFakeBold=2.5,AutoFakeSlant=0.3]{Noto Sans Gujarati}
% Use Noto Sans Gujarati for monospace to support Gujarati in text
\setmonofont[Scale=0.9]{Noto Sans Gujarati}

% Configure English to use the same font
\newfontfamily\englishfont[Script=Gujarati,AutoFakeBold=2.5,AutoFakeSlant=0.3]{Noto Sans Gujarati}

% Translations for polyglossia
\gappto\captionsgujarati{
  \renewcommand{\tablename}{કોષ્ટક}
  \renewcommand{\figurename}{આકૃતિ}
}

% Helper for TikZ nodes to ensure Gujarati font
\newcommand{\gu}[1]{{\gujaratifont #1}}

% Custom environments
\newtcolorbox{solutionbox}{
    breakable,
    enhanced,
    colback=solutioncolor!5!white,
    colframe=solutioncolor!75!black,
    fonttitle=\bfseries,
    title=જવાબ
}

\newtcolorbox{solutionboxnobreak}{
 colback=solutioncolor!5!white,
 colframe=solutioncolor!75!black,
 fonttitle=\bfseries,
 title=જવાબ
}

\newtcolorbox{keyformula}{
 breakable,
 enhanced,
 colback=keycolor!5!white,
 colframe=keycolor!75!black,
 fonttitle=\bfseries,
 title=રાસાયણિક સમીકરણ/સૂત્ર
}

\newtcolorbox{mnemonicbox}{
 breakable,
 enhanced,
 colback=mnemoniccolor!5!white,
 colframe=mnemoniccolor!75!black,
 fonttitle=\bfseries,
 title=મેમરી ટ્રીક
}


% Custom commands for GTU solutions
% This file defines semantic commands for consistent formatting

% Question command with automatic formatting
\newcommand{\question}[2]{%
  \section*{Question #1}%
  \textbf{#2}%
}

% OR question variant
\newcommand{\questionor}[2]{%
  \section*{Question #1 OR}%
  \textbf{#2}%
}

% Proper table environment with caption
\newenvironment{answertable}[1]{%
  \begin{table}[htbp]
  \centering
  \caption{#1}
}{%
  \end{table}
}

% Proper figure environment for diagrams
\newenvironment{answerdiagram}[1]{%
  \begin{figure}[htbp]
  \centering
  \caption{#1}
}{%
  \end{figure}
}

% Semantic markup for key terms
\newcommand{\keyword}[1]{\textbf{#1}}
\newcommand{\code}[1]{\texttt{#1}}
\newcommand{\classname}[1]{\texttt{#1}}
\newcommand{\methodname}[1]{\texttt{#1}}

% Proper quotation marks
\newcommand{\mnemonic}[1]{``#1''}

\usetikzlibrary{mindmap,trees}

\title{માઇક્રોપ્રોસેસર અને માઇક્રોકન્ટ્રોલર (4341101) - સમર 2024 સોલ્યુશન}
\date{15 જૂન 2024}

\begin{document}
\maketitle

\questionmarks{1}{a}{3}
\textbf{8051 માઇક્રોકન્ટ્રોલરના કોઈપણ એક પોર્ટ કન્ફિગરેશનનું વર્ણન કરો.}

\begin{solutionbox}
\textbf{જવાબ}:

\begin{center}
\captionof{table}{પોર્ટ કન્ફિગરેશન}
\begin{tabulary}{\linewidth}{|l|J|}
\hline
\textbf{કન્ફિગરેશન} & \textbf{વર્ણન} \\ \hline
\textbf{પોર્ટ 0} & ડ્યુઅલ-પર્પઝ પોર્ટ - 8-બિટ ઓપન ડ્રેન બિડાયરેક્શનલ I/O પોર્ટ અને મલ્ટીપ્લેક્સ્ડ લો એડ્રેસ/ડેટા બસ. I/O ફંક્શન માટે બાહ્ય પુલ-અપ રેસિસ્ટર જરૂરી. \\ \hline
\end{tabulary}
\end{center}

\textbf{ડાયાગ્રામ:}

\begin{center}
\begin{tikzpicture}[node distance=2.5cm, auto]
    \node [gtu block, minimum height=3cm] (p0) {પોર્ટ 0};
    \node [left of=p0, node distance=3cm] (pin) {P0.x Pin};
    \node [above of=pin, node distance=1.5cm] (pullup) {બાહ્ય પુલ-અપ};
    
    \draw [->, thick] (p0) -- (pin);
    \draw [thick] (pullup) -- (pin);
    \draw (pullup) -- ++(0,0.5) node[above] {+5V};
    
    \node [right of=p0, node distance=3cm] (bus) {આંતરિક બસ};
    \draw [<->, thick] (p0) -- (bus);
    
    \node [below of=p0] {8051 માઇક્રોકન્ટ્રોલર};
    \draw [dashed] ($(p0.south west)+(-0.5,-0.5)$) rectangle ($(p0.north east)+(0.5,0.5)$);
\end{tikzpicture}
\end{center}
\end{solutionbox}
\begin{mnemonicbox}
``પોર્ટ 0-પ્લેડ'' (પોર્ટ 0 ને પુલ-અપ્સ જોઈએ, લેચ/એડ્રેસ/ડેટા તરીકે કામ કરે)
\end{mnemonicbox}

\questionmarks{1}{b}{4}
\textbf{માઇક્રોપ્રોસેસર આર્કિટેક્ચરનું વર્ણન કરો.}

\begin{solutionbox}
\textbf{જવાબ}:

\begin{center}
\captionof{table}{માઇક્રોપ્રોસેસર ઘટકો}
\begin{tabulary}{\linewidth}{|l|J|}
\hline
\textbf{ઘટક} & \textbf{કાર્ય} \\ \hline
\textbf{ALU} & ગાણિતિક અને લોજિકલ ઓપરેશન કરે છે \\ \hline
\textbf{રજિસ્ટર્સ} & ડેટા અને એડ્રેસ માટે કામચલાઉ સ્ટોરેજ \\ \hline
\textbf{કંટ્રોલ યુનિટ} & પ્રોસેસર ઓપરેશન અને ડેટા ફ્લો નિર્દેશિત કરે છે \\ \hline
\textbf{બસ} & ડેટા ટ્રાન્સફર માટે પાથવે (એડ્રેસ, ડેટા, કંટ્રોલ) \\ \hline
\end{tabulary}
\end{center}

\textbf{ડાયાગ્રામ:}

\begin{center}
\begin{tikzpicture}[node distance=2cm]
    % Outer block
    \node [draw, rectangle, minimum width=8cm, minimum height=6cm, fill=black!5, rounded corners, label=above:MICROPROCESSOR] (cpu) {};
    
    % Internal
    \node [gtu block, fill=white] (alu) at (0, -1.5) {ALU};
    \node [gtu block, fill=white] (regs) at (-2, 1.5) {રજિસ્ટર્સ\\(A, B, C, D...)};
    \node [gtu block, fill=white] (cu) at (2, 1.5) {કંટ્રોલ યુનિટ};
    
    % Bus
    \node [gtu block, fill=white, minimum width=6cm] (bus) at (0, -4) {સિસ્ટમ બસ (એડ્રેસ, ડેટા, કંટ્રોલ)};
    
    % Connections
    \draw [gtu arrow, <->] (regs) -- (alu);
    \draw [gtu arrow] (cu) -- (regs);
    \draw [gtu arrow] (cu) -- (alu);
    \draw [gtu arrow] (cu) -- ++(1.5,0) |- (bus.north east);
    \draw [gtu arrow, <->] (alu) -- (bus.north);
    
\end{tikzpicture}
\end{center}
\end{solutionbox}
\begin{mnemonicbox}
``RABC'' - ``રજિસ્ટર, ALU, બસ, કંટ્રોલ''
\end{mnemonicbox}

\questionmarks{1}{c}{7}
\textbf{વોન ન્યુમેન અને હાર્વર્ડ આર્કિટેક્ચરની તુલના કરો.}

\begin{solutionbox}
\textbf{જવાબ}:

\begin{center}
\captionof{table}{વોન ન્યુમેન વિરુદ્ધ હાર્વર્ડ}
\begin{tabulary}{\linewidth}{|l|J|J|}
\hline
\textbf{ફીચર} & \textbf{વોન ન્યુમેન આર્કિટેક્ચર} & \textbf{હાર્વર્ડ આર્કિટેક્ચર} \\ \hline
\textbf{મેમરી બસ} & ઇન્સ્ટ્રક્શન અને ડેટા માટે એક જ મેમરી બસ & પ્રોગ્રામ અને ડેટા મેમરી માટે અલગ બસ \\ \hline
\textbf{એક્ઝિક્યુશન} & સિક્વેન્શિયલ એક્ઝિક્યુશન & પેરેલલ ફેચ અને એક્ઝિક્યુટ શક્ય \\ \hline
\textbf{સ્પીડ} & બસ બોટલનેક ને કારણે ધીમું & સમાંતર એક્સેસને કારણે ઝડપી \\ \hline
\textbf{મેમરી એક્સેસ} & એક જ મેમરી સ્પેસ & અલગ મેમરી સ્પેસ \\ \hline
\textbf{જટિલતા} & સરળ ડિઝાઇન & વધુ જટિલ ડિઝાઇન \\ \hline
\textbf{ઉપયોગો} & સામાન્ય કમ્પ્યુટિંગ & DSP, માઇક્રોકન્ટ્રોલર, એમ્બેડેડ સિસ્ટમ \\ \hline
\textbf{ઉદાહરણો} & મોટાભાગના PC, 8085, 8086 & 8051, PIC, ARM Cortex-M \\ \hline
\end{tabulary}
\end{center}

\textbf{ડાયાગ્રામ:}

\begin{center}
\begin{tikzpicture}[node distance=2cm]
    % Von Neumann
    \node (v_cpu) [gtu block] at (0,0) {CPU};
    \node (v_mem) [gtu block, right of=v_cpu, node distance=3cm] {મેમરી\\(કોડ + ડેટા)};
    \draw [gtu arrow, <->] (v_cpu) -- node[above] {બસ} (v_mem);
    \node [below of=v_cpu, node distance=1.5cm] {વોન-ન્યુમેન};

    % Harvard
    \node (h_cpu) [gtu block] at (6,0) {CPU};
    \node (h_prog) [gtu block, right of=h_cpu, node distance=3cm, yshift=1cm] {પ્રોગ્રામ\\મેમરી};
    \node (h_data) [gtu block, right of=h_cpu, node distance=3cm, yshift=-1cm] {ડેટા\\મેમરી};
    \draw [gtu arrow, <->] (h_cpu) -- (h_prog);
    \draw [gtu arrow, <->] (h_cpu) -- (h_data);
    \node [below of=h_cpu, node distance=1.5cm] {હાર્વર્ડ};
\end{tikzpicture}
\end{center}
\end{solutionbox}
\begin{mnemonicbox}
``હાર્વર્ડ હંમેશા અલગ રસ્તા રાખે'' (હાર્વર્ડમાં મેમરી પાથ અલગ હોય છે)
\end{mnemonicbox}

\orquestionmarks{1}{c}{7}
\textbf{RISC, CISC, Opcode, Operand, Instruction Cycle, Machine Cycle, અને T State ને વ્યાખ્યાયિત કરો.}

\begin{solutionbox}
\textbf{જવાબ}:

\begin{center}
\captionof{table}{વ્યાખ્યાઓ}
\begin{tabulary}{\linewidth}{|l|J|}
\hline
\textbf{શબ્દ} & \textbf{વ્યાખ્યા} \\ \hline
\textbf{RISC} & રિડ્યુસ્ડ ઇન્સ્ટ્રક્શન સેટ કમ્પ્યુટર - સરળ ઇન્સ્ટ્રક્શન સાથે સ્પીડ માટે ઓપ્ટિમાઇઝ્ડ આર્કિટેક્ચર \\ \hline
\textbf{CISC} & કોમ્પ્લેક્સ ઇન્સ્ટ્રક્શન સેટ કમ્પ્યુટર - જટિલ, શક્તિશાળી ઇન્સ્ટ્રક્શન સાથેનું આર્કિટેક્ચર \\ \hline
\textbf{Opcode} & ઓપરેશન કોડ - ઇન્સ્ટ્રક્શનનો ભાગ જે કયા ઓપરેશન કરવાના છે તે સ્પષ્ટ કરે છે \\ \hline
\textbf{Operand} & ઓપરેશનમાં વપરાતો ડેટા વેલ્યુ અથવા એડ્રેસ \\ \hline
\textbf{Instruction Cycle} & ઇન્સ્ટ્રક્શન ફેચ, ડિકોડ અને એક્ઝિક્યુટની સંપૂર્ણ પ્રક્રિયા \\ \hline
\textbf{Machine Cycle} & મૂળભૂત ઓપરેશન જેમ કે મેમરી રીડ/રાઈટ (ઇન્સ્ટ્રક્શન સાયકલનો ભાગ) \\ \hline
\textbf{T-State} & ટાઈમ સ્ટેટ - પ્રોસેસરમાં સમયનો સૌથી નાનો એકમ (ક્લોક પીરિયડ) \\ \hline
\end{tabulary}
\end{center}

\textbf{ડાયાગ્રામ:}

\begin{center}
\begin{tikzpicture}[auto, node distance=2.5cm]
    \node [gtu block] (fetch) {FETCH};
    \node [gtu block, right of=fetch] (decode) {DECODE};
    \node [gtu block, right of=decode] (execute) {EXECUTE};
    
    \draw [gtu arrow] (fetch) -- (decode);
    \draw [gtu arrow] (decode) -- (execute);
    \draw [gtu arrow] (execute) -- ++(0,-1) -| (fetch);
    
    \node [below of=decode, node distance=1.5cm] {ઇન્સ્ટ્રક્શન સાયકલ};
\end{tikzpicture}
\end{center}
\end{solutionbox}
\begin{mnemonicbox}
``RICO ITEM'' (RISC, CISC, Opcode, Instruction cycle, T-state, Execute, Machine cycle)
\end{mnemonicbox}

\questionmarks{2}{a}{3}
\textbf{ડેટા બસ, એડ્રેસ બસ અને કંટ્રોલ બસ વ્યાખ્યાયિત કરો.}

\begin{solutionbox}
\textbf{જવાબ}:

\begin{center}
\captionof{table}{બસ પ્રકારો}
\begin{tabulary}{\linewidth}{|l|J|}
\hline
\textbf{બસ પ્રકાર} & \textbf{વ્યાખ્યા} \\ \hline
\textbf{ડેટા બસ} & બિડાયરેક્શનલ પાથવે જે માઇક્રોપ્રોસેસર અને પેરિફેરલ ડિવાઇસ વચ્ચે વાસ્તવિક ડેટા ટ્રાન્સફર કરે છે \\ \hline
\textbf{એડ્રેસ બસ} & યુનિડાયરેક્શનલ પાથવે જે એક્સેસ કરવાના મેમરી/IO ડિવાઇસ લોકેશન ધરાવે છે \\ \hline
\textbf{કંટ્રોલ બસ} & સિગ્નલ લાઈનોનો ગ્રુપ જે સિસ્ટમ ઓપરેશનને કોઓર્ડિનેટ અને સિન્ક્રોનાઇઝ કરે છે \\ \hline
\end{tabulary}
\end{center}

\textbf{ડાયાગ્રામ:}

\begin{center}
\begin{tikzpicture}[node distance=2cm, auto]
    \node [gtu block] (cpu) {CPU};
    \draw [->, thick] (cpu.south) ++(-1.5,0) -- ++(0,-1.5) node[below, align=center] {એડ્રેસ બસ\\(મેમરી/IO લોકેશન)};
    \draw [<->, thick] (cpu.south) -- ++(0,-1.5) node[below, align=center] {ડેટા બસ\\(માહિતી)};
    \draw [->, thick] (cpu.south) ++(1.5,0) -- ++(0,-1.5) node[below, align=center] {કંટ્રોલ બસ\\(RD, WR, etc.)};
\end{tikzpicture}
\end{center}
\end{solutionbox}
\begin{mnemonicbox}
``ADC'' - ``એડ્રેસ લોકેશન શોધે, ડેટા માહિતી લઈ જાય, કંટ્રોલ ઓપરેશન કોઓર્ડિનેટ કરે''
\end{mnemonicbox}

\questionmarks{2}{b}{4}
\textbf{માઇક્રોપ્રોસેસર અને માઇક્રોકન્ટ્રોલરની સરખામણી કરો.}

\begin{solutionbox}
\textbf{જવાબ}:

\begin{center}
\captionof{table}{સરખામણી}
\begin{tabulary}{\linewidth}{|l|J|J|}
\hline
\textbf{ફીચર} & \textbf{માઇક્રોપ્રોસેસર} & \textbf{માઇક્રોકન્ટ્રોલર} \\ \hline
\textbf{વ્યાખ્યા} & એકલ ચિપ પર CPU & એકલ ચિપ પર સંપૂર્ણ કમ્પ્યુટર સિસ્ટમ \\ \hline
\textbf{મેમરી} & બાહ્ય RAM/ROM જરૂરી & અંદર જ RAM/ROM \\ \hline
\textbf{I/O પોર્ટ} & મર્યાદિત અથવા ચિપ પર નથી & ચિપ પર ઘણા I/O પોર્ટ \\ \hline
\textbf{પેરિફેરલ્સ} & બાહ્ય પેરિફેરલ્સ જરૂરી & અંદર જ પેરિફેરલ્સ (ટાઈમર્સ, ADC, વગેરે) \\ \hline
\textbf{ઉપયોગો} & સામાન્ય કમ્પ્યુટિંગ, PC & એમ્બેડેડ સિસ્ટમ, IoT ડિવાઇસિસ \\ \hline
\textbf{કિંમત} & સંપૂર્ણ સિસ્ટમ માટે વધારે & ઓછી (ઓલ-ઇન-વન સોલ્યુશન) \\ \hline
\textbf{પાવર કન્ઝમ્પશન} & વધારે & ઓછું \\ \hline
\end{tabulary}
\end{center}
\end{solutionbox}
\begin{mnemonicbox}
``MEMI-CAP'' (મેમરી બાહ્ય/આંતરિક, કિંમત, એપ્લિકેશન્સ, પેરિફેરલ્સ)
\end{mnemonicbox}

\questionmarks{2}{c}{7}
\textbf{8085 બ્લોક ડાયાગ્રામ સ્કેચ કરો અને સમજાવો.}

\begin{solutionbox}
\textbf{જવાબ}:

\textbf{ડાયાગ્રામ:}

\begin{center}
\begin{tikzpicture}[node distance=2cm]
    % Outer block as CPU
    \node [draw, rectangle, minimum width=8cm, minimum height=6cm, fill=black!5, rounded corners] (cpu) {};
    \node [below right] at (cpu.north west) {\textbf{8085 CPU}};

    % Internal components
    \node [gtu block, fill=white] (regs) at ([xshift=-2cm, yshift=1.5cm]cpu.center) {રજિસ્ટર એરે\\A, Flags\\B,C,D,E,H,L\\SP, PC};
    \node [gtu block, fill=white, right of=regs, node distance=4cm] (control) {ટાઈમિંગ અને\\કંટ્રોલ\\(ડિકોડર)};
    \node [gtu block, fill=white, below of=regs, node distance=3cm] (alu) {ALU};
    
    % Connections
    \draw [gtu arrow, <->] (regs) -- (alu);
    \draw [gtu arrow] (control) -- (regs);
    \draw [gtu arrow] (control) -- (alu);
    
    % Buses
    \node [gtu block, fill=white] (bus) at ([yshift=-2.5cm]cpu.center) {એડ્રેસ/ડેટા બસ ઇન્ટરફેસ};
    \draw [gtu arrow, <->] (alu) -- (bus);
    
    % External connections
    \draw [gtu arrow] (bus.south) -- ++(0,-1) node[below] {એડ્રેસ અને ડેટા બસ};
    \draw [gtu arrow] (control.east) -- ++(1,0) node[right] {કંટ્રોલ સિગ્નલ્સ};

\end{tikzpicture}
\end{center}

\textbf{મુખ્ય ઘટકો}:
\begin{itemize}
    \item \textbf{રજિસ્ટર એરે}: A (એક્યુમુલેટર), ફ્લેગ્સ, B-L, SP, PC, ટેમ્પ રજિસ્ટર્સ
    \item \textbf{ALU}: ગાણિતિક અને લોજિકલ ઓપરેશન કરે છે
    \item \textbf{ટાઈમિંગ \& કંટ્રોલ}: કંટ્રોલ સિગ્નલ્સ જનરેટ કરે છે, ઇન્ટરપ્ટ હેન્ડલ કરે છે
    \item \textbf{બસ ઇન્ટરફેસ}: CPU ને બાહ્ય ડિવાઇસ સાથે જોડે છે
    \item \textbf{ઇન્ટરનલ ડેટા બસ}: આંતરિક ઘટકોને જોડે છે
\end{itemize}
\end{solutionbox}
\begin{mnemonicbox}
``RATBI'' - ``રજિસ્ટર્સ, ALU, ટાઈમિંગ, બસ, ઇન્ટરફેસ''
\end{mnemonicbox}

\orquestionmarks{2}{a}{3}
\textbf{એક્યુમ્યુલેટર, પ્રોગ્રામ કાઉન્ટર અને સ્ટેક પોઇન્ટર સમજાવો.}

\begin{solutionbox}
\textbf{જવાબ}:

\begin{center}
\captionof{table}{રજિસ્ટર વર્ણન}
\begin{tabulary}{\linewidth}{|l|J|}
\hline
\textbf{રજિસ્ટર} & \textbf{કાર્ય} \\ \hline
\textbf{એક્યુમ્યુલેટર (A)} & 8-બિટ રજિસ્ટર જે ગાણિતિક અને લોજિકલ ઓપરેશનના પરિણામો સ્ટોર કરે છે \\ \hline
\textbf{પ્રોગ્રામ કાઉન્ટર (PC)} & 16-બિટ રજિસ્ટર જે આગલા એક્ઝિક્યુટ થનાર ઇન્સ્ટ્રક્શનનું એડ્રેસ રાખે છે \\ \hline
\textbf{સ્ટેક પોઇન્ટર (SP)} & 16-બિટ રજિસ્ટર જે મેમરીમાં સ્ટેકના વર્તમાન ટોપને પોઇન્ટ કરે છે \\ \hline
\end{tabulary}
\end{center}

\textbf{ડાયાગ્રામ:}

\begin{center}
\begin{tikzpicture}[node distance=3cm]
    \node [gtu block] (acc) {A (8-bit)\\ડેટા પ્રોસેસિંગ};
    \node [gtu block, right of=acc] (pc) {PC (16-bit)\\આગલું ઇન્સ્ટ્રક્શન};
    \node [gtu block, right of=pc] (sp) {SP (16-bit)\\સ્ટેક ટોપ};
    
    \node [below of=sp, node distance=2cm, draw, rectangle] (stack) {સ્ટેક મેમરી};
    \draw [->] (sp) -- (stack);
\end{tikzpicture}
\end{center}
\end{solutionbox}
\begin{mnemonicbox}
``APS'' - ``એક્યુમ્યુલેટર પ્રોસેસ કરે, PC આગલું ઇન્સ્ટ્રક્શન જુએ, SP સ્ટેક સંભાળે''
\end{mnemonicbox}

\orquestionmarks{2}{b}{4}
\textbf{એડ્રેસ બસ અને ડેટા બસનું ડિમલ્ટિપ્લેક્સીંગ સ્કેચ કરો અને સમજાવો.}

\begin{solutionbox}
\textbf{જવાબ}:

\textbf{ડાયાગ્રામ:}

\begin{center}
\begin{tikzpicture}[auto, node distance=2.5cm]
    \node [gtu block, minimum width=3cm, minimum height=4cm] (cpu) {8085 CPU};
    \node [gtu block, right of=cpu, node distance=5cm, yshift=-1.5cm] (latch) {74LS373\\Latch};
    
    % Higher Order Address
    \draw [->, thick] (cpu.north east) ++(0,-0.5) -- ++(2,0) node[right] {A15-A8 (Higher Address)};
    
    % Address/Data Bus
    \draw [thick] (cpu.east) ++(0,-1.5) -- ++(1.5,0) coordinate (split);
    \draw [thick] (split) -- (latch.west) node[midway, above] {AD7-AD0};
    \draw [<->, thick] (split) -- ++(0, -1.5) node[below] {ડેટા બસ D7-D0};
    
    % ALE
    \draw [->] (cpu.east) ++(0,-2.5) -- ++(1.5,0) -- (latch.south) node[midway, right] {ALE};
    
    % Latch output
    \draw [->, thick] (latch.east) -- ++(1.5,0) node[right] {A7-A0 (Lower Address)};
\end{tikzpicture}
\end{center}

\textbf{પ્રક્રિયા}:
\begin{itemize}
    \item \textbf{મલ્ટિપ્લેક્સિંગ}: પિન કાઉન્ટ ઘટાડવા માટે AD0-AD7 પિન એડ્રેસ અને ડેટા સિગ્નલ શેર કરે છે.
    \item \textbf{ડિમલ્ટિપ્લેક્સિંગના સ્ટેપ્સ}:
    \begin{enumerate}
        \item CPU AD0-AD7 પિન પર એડ્રેસ મૂકે છે.
        \item ALE (એડ્રેસ લેચ એનેબલ) સિગ્નલ HIGH થાય છે.
        \item બાહ્ય લેચ (74LS373) લોઅર એડ્રેસ બિટ્સ પકડે છે.
        \item ALE LOW થાય છે, એડ્રેસ લેચ થઈ જાય છે.
        \item AD0-AD7 પિન હવે ડેટા લઈ જાય છે.
    \end{enumerate}
\end{itemize}
\end{solutionbox}
\begin{mnemonicbox}
``ALAD'' - ``ALE એક્ટિવ, લેચ એડ્રેસ, આફ્ટર ડેટા''
\end{mnemonicbox}

\orquestionmarks{2}{c}{7}
\textbf{8085 ની કોઈપણ સાત વિશેષતાઓની યાદી આપો.}

\begin{solutionbox}
\textbf{જવાબ}:

\begin{center}
\captionof{table}{8085 વિશેષતાઓ}
\begin{tabulary}{\linewidth}{|l|J|}
\hline
\textbf{વિશેષતા} & \textbf{વર્ણન} \\ \hline
\textbf{8-બિટ ડેટા બસ} & 8 બિટ્સ ડેટા પેરેલલમાં ટ્રાન્સફર કરે છે \\ \hline
\textbf{16-બિટ એડ્રેસ બસ} & 64KB સુધીની મેમરી એડ્રેસ કરી શકે છે ($2^{16}$) \\ \hline
\textbf{હાર્ડવેર ઇન્ટરપ્ટ} & 5 હાર્ડવેર ઇન્ટરપ્ટ (TRAP, RST 7.5, 6.5, 5.5, INTR) \\ \hline
\textbf{સિરિયલ I/O} & સિરિયલ કમ્યુનિકેશન માટે SID અને SOD પિન \\ \hline
\textbf{ક્લોક જનરેશન} & ક્રિસ્ટલ સાથે ઓન-ચિપ ક્લોક જનરેટર \\ \hline
\textbf{ઇન્સ્ટ્રક્શન સેટ} & 74 ઓપરેશન કોડ્સ જે 246 ઇન્સ્ટ્રક્શન જનરેટ કરે છે \\ \hline
\textbf{રજિસ્ટર સેટ} & છ 8-બિટ રજિસ્ટર (B,C,D,E,H,L), એક્યુમુલેટર, ફ્લેગ્સ, SP, PC \\ \hline
\end{tabulary}
\end{center}

\textbf{ડાયાગ્રામ:}

\begin{center}
\begin{tikzpicture}
  \path[mindmap,concept color=black!10,text=black]
    node[concept] {8085 Features}
    [clockwise from=0]
    child { node[concept] {8-bit Data Bus} }
    child { node[concept] {16-bit Address Bus} }
    child { node[concept] {5 HW Interrupts} }
    child { node[concept] {Serial I/O} }
    child { node[concept] {On-chip Clock} }
    child { node[concept] {Registers} };
\end{tikzpicture}
\end{center}
\end{solutionbox}
\begin{mnemonicbox}
``CHAIRS'' - ``ક્લોક, હાર્ડવેર ઇન્ટરપ્ટ, એડ્રેસ બસ, ઇન્સ્ટ્રક્શન સેટ, રજિસ્ટર્સ, સિરિયલ I/O''
\end{mnemonicbox}

\questionmarks{3}{a}{3}
\textbf{8051 ના કોઈપણ એક ટાઈમર મોડને સમજાવો.}

\begin{solutionbox}
\textbf{જવાબ}:

\textbf{મોડ 1: 16-બિટ ટાઈમર/કાઉન્ટર}

\begin{center}
\captionof{table}{ટાઈમર મોડ 1}
\begin{tabulary}{\linewidth}{|l|J|}
\hline
\textbf{ફીચર} & \textbf{વર્ણન} \\ \hline
\textbf{ટાઈમર સ્ટ્રક્ચર} & THx અને TLx રજિસ્ટર્સ વાપરીને 16-બિટ ટાઈમર \\ \hline
\textbf{ઓપરેશન} & 0000H થી FFFFH સુધી ગણતરી કરે છે, પછી TF ફ્લેગ સેટ કરે છે \\ \hline
\textbf{કાઉન્ટર સાઈઝ} & ફુલ 16-બિટ કાઉન્ટર ($2^{16} = 65,536$ કાઉન્ટ્સ) \\ \hline
\textbf{રજિસ્ટર્સ} & THx (હાઈ બાઈટ) અને TLx (લો બાઈટ) \\ \hline
\end{tabulary}
\end{center}

\textbf{ડાયાગ્રામ:}

\begin{center}
\begin{tikzpicture}[auto, node distance=2.5cm]
    \node [gtu block] (control) {Control Bits};
    \node [gtu block, right of=control, node distance=3cm] (gate) {Gate Control};
    \node [gtu block, right of=gate, node distance=3cm] (overflow) {Overflow\\Detect};
    
    \node [right of=overflow, node distance=2.5cm] (flag) {TF (Flag)};
    \draw [->, thick] (overflow) -- (flag);
    
    \draw [->, thick] (control) -- (gate);
    \draw [->, thick] (gate) -- (overflow);
    
    \node [gtu block, below of=gate, node distance=2cm] (counter) {THx:TLx Counter\\(16-bit)};
    \node [left of=counter, node distance=3cm] (clock) {Clock Source};
    
    \draw [->, thick] (clock) -- (counter);
    \draw [->, thick] (counter) -| (overflow);
    \draw [dashed] (gate) -- (counter);

\end{tikzpicture}
\end{center}
\end{solutionbox}
\begin{mnemonicbox}
``MOGC'' - ``મોડ 1 ઓવરફ્લો ડિટેક્શન, ગેટ કંટ્રોલ, કમ્પ્લીટ 16-બિટ''
\end{mnemonicbox}

\questionmarks{3}{b}{4}
\textbf{8051 માટે ALE, PSEN, RESET અને TXD પિનનું ફંક્શન લખો.}

\begin{solutionbox}
\textbf{જવાબ}:

\begin{center}
\captionof{table}{પિન ફંક્શન્સ}
\begin{tabulary}{\linewidth}{|l|J|}
\hline
\textbf{પિન} & \textbf{ફંક્શન} \\ \hline
\textbf{ALE} & એડ્રેસ લેચ એનેબલ - પોર્ટ 0 માંથી એડ્રેસનો લો બાઈટ લેચ કરવા માટે કંટ્રોલ સિગ્નલ પૂરું પાડે છે \\ \hline
\textbf{PSEN} & પ્રોગ્રામ સ્ટોર એનેબલ - બાહ્ય પ્રોગ્રામ મેમરી એક્સેસ માટે રીડ સ્ટ્રોબ \\ \hline
\textbf{RESET} & રીસેટ ઇનપુટ - 2 મશીન સાયકલ સુધી HIGH રાખવાથી CPU ને પ્રારંભિક સ્થિતિમાં ફોર્સ કરે છે \\ \hline
\textbf{TXD} & ટ્રાન્સમિટ ડેટા - સિરિયલ ડેટા ટ્રાન્સમિશન માટે સિરિયલ પોર્ટ આઉટપુટ પિન \\ \hline
\end{tabulary}
\end{center}

\textbf{ડાયાગ્રામ:}

\begin{center}
\begin{tikzpicture}[scale=0.8, transform shape]
    \draw[thick, fill=black!5] (0,0) rectangle (4,4);
    \node at (2,2) {\textbf{8051}};
    
    \draw (-1, 3) -- (0, 3) node[midway, above] {ALE};
    \draw (-1, 2) -- (0, 2) node[midway, above] {PSEN};
    \draw (-1, 1) -- (0, 1) node[midway, above] {RESET};
    \draw (4, 3) -- (5, 3) node[midway, above] {TXD};
    
    \node at (-0.2, 3) {$\bullet$};
    \node at (-0.2, 2) {$\bullet$};
    \node at (-0.2, 1) {$\bullet$};
    \node at (4.2, 3) {$\bullet$};
\end{tikzpicture}
\end{center}
\end{solutionbox}
\begin{mnemonicbox}
``APTR'' - ``એડ્રેસ લેચ, પ્રોગ્રામ સ્ટોર, ટોટલ રીસેટ, ટ્રાન્સમિટ ડેટા''
\end{mnemonicbox}

\questionmarks{3}{c}{7}
\textbf{8051 માઇક્રોકન્ટ્રોલરના દરેક બ્લોકના કાર્યો સમજાવો.}

\begin{solutionbox}
\textbf{જવાબ}:

\begin{center}
\captionof{table}{8051 બ્લોક્સ}
\begin{tabulary}{\linewidth}{|l|J|}
\hline
\textbf{બ્લોક} & \textbf{કાર્ય} \\ \hline
\textbf{CPU} & 8-બિટ પ્રોસેસર જે ઇન્સ્ટ્રક્શન ફેચ અને એક્ઝિક્યુટ કરે છે \\ \hline
\textbf{મેમરી} & 4KB ઇન્ટરનલ ROM અને 128 બાઈટ્સ ઇન્ટરનલ RAM \\ \hline
\textbf{I/O પોર્ટ્સ} & ચાર 8-બિટ બિડાયરેક્શનલ I/O પોર્ટ્સ (P0-P3) \\ \hline
\textbf{ટાઈમર/કાઉન્ટર} & ટાઈમિંગ અને કાઉન્ટિંગ માટે બે 16-બિટ ટાઈમર/કાઉન્ટર \\ \hline
\textbf{સિરિયલ પોર્ટ} & સિરિયલ કમ્યુનિકેશન માટે ફુલ-ડુપ્લેક્સ UART \\ \hline
\textbf{ઇન્ટરપ્ટ} & બે પ્રાયોરિટી લેવલ સાથે પાંચ ઇન્ટરપ્ટ સોર્સ \\ \hline
\textbf{ક્લોક સર્કિટ} & તમામ ઓપરેશન માટે ટાઈમિંગ પૂરું પાડે છે \\ \hline
\end{tabulary}
\end{center}

\textbf{ડાયાગ્રામ:}

\begin{center}
\begin{tikzpicture}[node distance=2.5cm, auto]
    % CPU
    \node [gtu block, fill=white] (cpu) {CPU};
    
    % Peripherals
    \node [gtu block, right of=cpu, node distance=4cm] (timers) {Timers/\\Counters};
    \node [gtu block, right of=timers, node distance=3.5cm] (interrupts) {Interrupts};
    
    \node [gtu block, below of=cpu, node distance=3cm] (memory) {Memory\\RAM/ROM};
    \node [gtu block, right of=memory, node distance=4cm] (serial) {Serial\\Port};
    \node [gtu block, right of=serial, node distance=3.5cm] (ports) {I/O Ports\\P0-P3};
    
    % Clock
    \node [gtu block, minimum width=8cm, below of=memory, node distance=2.5cm, xshift=2cm] (clock) {Clock Circuit};
    
    % Connections
    \draw [gtu arrow, <->] (cpu) -- (timers);
    \draw [gtu arrow, <->] (cpu) -| (interrupts);
    \draw [gtu arrow, <->] (cpu) -- (memory);
    \draw [gtu arrow, <->] (memory) -- (serial);
    \draw [gtu arrow, <->] (serial) -- (ports);
    \draw [gtu arrow, ->] (clock) -- ++(0,1.5);
    
    \node [draw, dashed, fit=(cpu) (ports) (clock)] {};
    \node [above] at (cpu.north) {\textbf{8051 ARCHITECTURE}};

\end{tikzpicture}
\end{center}
\end{solutionbox}
\begin{mnemonicbox}
``CRIMSON'' - ``CPU, RAM/ROM, I/O, મેમરી, સિરિયલ પોર્ટ, ઓસિલેટર, ઇન્ટરપ્ટ''
\end{mnemonicbox}

\orquestionmarks{3}{a}{3}
\textbf{8051 ના કોઈપણ એક સીરિયલ કોમ્યુનિકેશન મોડને સમજાવો.}

\begin{solutionbox}
\textbf{જવાબ}:

\textbf{મોડ 1: 8-બિટ UART}

\begin{center}
\captionof{table}{સિરિયલ મોડ 1}
\begin{tabulary}{\linewidth}{|l|J|}
\hline
\textbf{ફીચર} & \textbf{વર્ણન} \\ \hline
\textbf{ફોર્મેટ} & 10 બિટ્સ (સ્ટાર્ટ બિટ, 8 ડેટા બિટ્સ, સ્ટોપ બિટ) \\ \hline
\textbf{બોડ રેટ} & વેરિએબલ, ટાઈમર 1 દ્વારા નક્કી થાય છે \\ \hline
\textbf{ડેટા ડાયરેક્શન} & ફુલ-ડુપ્લેક્સ (એક સાથે ટ્રાન્સમિટ અને રિસીવ) \\ \hline
\textbf{પિન્સ} & ટ્રાન્સમિટ માટે TXD (P3.1), રિસીવ માટે RXD (P3.0) \\ \hline
\end{tabulary}
\end{center}

\textbf{ડાયાગ્રામ:}

\begin{center}
\begin{tikzpicture}[node distance=2.5cm, auto]
    \node [gtu block] (timer) {Timer 1};
    \node [gtu block, right of=timer, node distance=3.5cm] (baud) {Baud Rate\\Generator};
    \node [gtu block, right of=baud, node distance=3.5cm] (tx) {Transmit\\Shift Reg};
    
    \draw [->, thick] (timer) -- (baud);
    \draw [->, thick] (baud) -- (tx);
    \draw [->, thick] (tx) -- ++(2,0) node[right] {Serial Out (TXD)};
    
    \node [gtu block, below of=tx, node distance=2.5cm] (rx) {Receive\\Shift Reg};
    \draw [<-, thick] (rx) -- ++(2,0) node[right] {Serial In (RXD)};
    
    \node [gtu block, above of=tx, node distance=2cm] (sbuf) {SBUF};
    \draw [->, thick] (sbuf) -- (tx);
    \draw [->, thick] (rx) -- ++(-2,0) |- (sbuf);

\end{tikzpicture}
\end{center}
\end{solutionbox}
\begin{mnemonicbox}
``FADS'' - ``ફોર્મેટ 10-બિટ, ઓટો બોડ ટાઈમર 1 થી, ડુપ્લેક્સ મોડ, સ્ટાન્ડર્ડ UART''
\end{mnemonicbox}

\orquestionmarks{3}{b}{4}
\textbf{8051 માટે RXD, INT0, T0 અને PROG પિનનું ફંક્શન લખો.}

\begin{solutionbox}
\textbf{જવાબ}:

\begin{center}
\captionof{table}{પિન ફંક્શન્સ}
\begin{tabulary}{\linewidth}{|l|J|}
\hline
\textbf{પિન} & \textbf{ફંક્શન} \\ \hline
\textbf{RXD (P3.0)} & રિસીવ ડેટા - સિરિયલ ડેટા રિસેપ્શન માટે સિરિયલ પોર્ટ ઇનપુટ પિન \\ \hline
\textbf{INT0 (P3.2)} & એક્સટર્નલ ઇન્ટરપ્ટ 0 - બાહ્ય ઇન્ટરપ્ટ ટ્રિગર કરી શકે તેવો ઇનપુટ \\ \hline
\textbf{T0 (P3.4)} & ટાઈમર 0 - ટાઈમર/કાઉન્ટર 0 માટે બાહ્ય કાઉન્ટ ઇનપુટ \\ \hline
\textbf{PROG (EA)} & પ્રોગ્રામ એનેબલ - જ્યારે LOW હોય, ત્યારે CPU ને બાહ્ય મેમરીમાંથી કોડ ફેચ કરવા ફોર્સ કરે છે \\ \hline
\end{tabulary}
\end{center}

\textbf{ડાયાગ્રામ:}

\begin{center}
\begin{tikzpicture}[scale=0.8, transform shape]
    \draw[thick, fill=black!5] (0,0) rectangle (4,4);
    \node at (2,2) {\textbf{8051}};
    
    \draw (-1, 3) -- (0, 3) node[midway, above] {RXD};
    \draw (-1, 2) -- (0, 2) node[midway, above] {INT0};
    \draw (-1, 1) -- (0, 1) node[midway, above] {T0};
    \draw (4, 3) -- (5, 3) node[midway, above] {PROG};
    
    \node at (-0.2, 3) {$\bullet$};
    \node at (-0.2, 2) {$\bullet$};
    \node at (-0.2, 1) {$\bullet$};
    \node at (4.2, 3) {$\bullet$};
\end{tikzpicture}
\end{center}
\end{solutionbox}
\begin{mnemonicbox}
``RIPE'' - ``રિસીવ ડેટા, ઇન્ટરપ્ટ ટ્રિગર, પલ્સ કાઉન્ટિંગ, એક્સટર્નલ મેમરી''
\end{mnemonicbox}

\orquestionmarks{3}{c}{7}
\textbf{8051 માટે ALU, PC, DPTR, RS0, RS1, આંતરિક RAM અને આંતરિક ROM નું વર્ણન કરો.}

\begin{solutionbox}
\textbf{જવાબ}:

\begin{center}
\captionof{table}{રજિસ્ટર/મેમરી વર્ણન}
\begin{tabulary}{\linewidth}{|l|J|}
\hline
\textbf{ઘટક} & \textbf{વર્ણન} \\ \hline
\textbf{ALU} & અર્થમેટિક લોજિક યુનિટ - ગાણિતિક અને લોજિકલ ઓપરેશન કરે છે \\ \hline
\textbf{PC} & પ્રોગ્રામ કાઉન્ટર - 16-બિટ રજિસ્ટર જે આગલી ઇન્સ્ટ્રક્શનને પોઇન્ટ કરે છે \\ \hline
\textbf{DPTR} & ડેટા પોઇન્ટર - 16-બિટ રજિસ્ટર (DPH+DPL) બાહ્ય મેમરી એડ્રેસિંગ માટે \\ \hline
\textbf{RS0, RS1} & PSW માં રજિસ્ટર બેંક સિલેક્ટ બિટ્સ - ચાર રજિસ્ટર બેંક્સમાંથી એક પસંદ કરે છે \\ \hline
\textbf{આંતરિક RAM} & 128 બાઈટ્સ ઓન-ચિપ RAM (00H-7FH) વેરિએબલ્સ અને સ્ટેક માટે \\ \hline
\textbf{આંતરિક ROM} & 4KB ઓન-ચિપ ROM (0000H-0FFFH) પ્રોગ્રામ સ્ટોરેજ માટે \\ \hline
\end{tabulary}
\end{center}

\textbf{ડાયાગ્રામ:}

\begin{center}
\begin{tikzpicture}[node distance=2cm]
    % ROM
    \node [draw, rectangle, minimum width=3cm, minimum height=4cm, label=above:Internal ROM] (rom) at (0,0) {};
    \node at (0, 1.5) {0FFFH};
    \node at (0, -1.5) {0000H};
    \node at (0, 0) {4KB};
    
    % RAM
    \node [draw, rectangle, minimum width=3cm, minimum height=4cm, label=above:Internal RAM] (ram) at (5,0) {};
    \node at (5, 1.8) {7FH};
    \node at (5, 0.5) {Scratch Pad};
    \draw (3.5, 0) -- (6.5, 0);
    \node at (5, -0.5) {Bit-Addressable};
    \draw (3.5, -1) -- (6.5, -1);
    \node at (5, -1.5) {Banks (RS0,RS1)};
    \node at (5, -1.8) {00H};
    
\end{tikzpicture}
\end{center}
\end{solutionbox}
\begin{mnemonicbox}
``APRID'' - ``ALU પ્રોસેસ કરે, PC યાદ રાખે, રજિસ્ટર બેંક સિલેક્ટ, ઇન્ટરનલ મેમરી, DPTR પોઇન્ટ કરે''
\end{mnemonicbox}

\questionmarks{4}{a}{3}
\textbf{08H ને 02H થી વિભાજિત કરવા માટે એસેમ્બલી ભાષામાં પ્રોગ્રામ વિકસાવો.}

\begin{solutionbox}
\textbf{જવાબ}:

\begin{minipage}{\linewidth}
\begin{lstlisting}[language={[x86masm]Assembler}]
      MOV A, #08H    ; ડિવિડન્ડ 08H એક્યુમુલેટરમાં લોડ કરો
      MOV B, #02H    ; ડિવાઇઝર 02H B રજિસ્ટરમાં લોડ કરો
      DIV AB         ; A ને B વડે ભાગો (A=ભાગફળ, B=શેષ)
      MOV R0, A      ; ભાગફળ R0 માં સ્ટોર કરો (04H)
      MOV R1, B      ; શેષ R1 માં સ્ટોર કરો (00H)
\end{lstlisting}
\end{minipage}

\textbf{ડાયાગ્રામ:}

\begin{center}
\begin{tikzpicture}[node distance=2cm]
    \node [draw, rectangle, minimum width=2cm, minimum height=1cm] (before) {A: 08H\\B: 02H};
    \node [draw, rectangle, minimum width=2cm, minimum height=1cm, right of=before, node distance=4cm] (after) {A: 04H (Q)\\B: 00H (R)};
    \draw [->, thick] (before) -- (after) node[midway, above] {DIV AB};
\end{tikzpicture}
\end{center}
\end{solutionbox}
\begin{mnemonicbox}
``LDDS'' - ``લોડ ડિવિડન્ડ, ડિવાઇઝર B માં, ડિવાઇડ, સ્ટોર રિઝલ્ટ''
\end{mnemonicbox}

\questionmarks{4}{b}{4}
\textbf{76H અને 32H ઉમેરવા માટે એસેમ્બલી ભાષામાં પ્રોગ્રામ વિકસાવો.}

\begin{solutionbox}
\textbf{જવાબ}:

\begin{minipage}{\linewidth}
\begin{lstlisting}[language={[x86masm]Assembler}]
      MOV A, #76H    ; પહેલો નંબર 76H એક્યુમુલેટરમાં લોડ કરો
      MOV R0, #32H   ; બીજો નંબર 32H R0 માં લોડ કરો
      ADD A, R0      ; R0 ને A માં ઉમેરો (76H + 32H = A8H)
      MOV R1, A      ; પરિણામ R1 માં સ્ટોર કરો (A8H)
      JNC DONE       ; જો કેરી ન આવે તો જમ્પ કરો
      MOV R2, #01H   ; જો કેરી આવે તો, R2 માં 1 સ્ટોર કરો
DONE: NOP            ; પ્રોગ્રામ પૂરો કરો
\end{lstlisting}
\end{minipage}

\textbf{ડાયાગ્રામ:}

\begin{center}
\begin{tikzpicture}[node distance=1.5cm]
    \node (eq) {76H + 32H = A8H};
    \node [below of=eq] {કેરી ફ્લેગ (CY) = 0};
\end{tikzpicture}
\end{center}
\end{solutionbox}
\begin{mnemonicbox}
``LASER'' - ``લોડ A, સ્ટોર સેકન્ડ નંબર, એક્ઝિક્યુટ એડિશન, રિઝલ્ટ સ્ટોર''
\end{mnemonicbox}

\questionmarks{4}{c}{7}
\textbf{એડ્રેસિંગ મોડ શું છે? તેને 8051 માટે વર્ગીકૃત કરો.}

\begin{solutionbox}
\textbf{જવાબ}:

\textbf{એડ્રેસિંગ મોડ}: ઇન્સ્ટ્રક્શન માટે ઓપરેન્ડ/ડેટાનું સ્થાન સ્પષ્ટ કરવાની પદ્ધતિ.

\begin{center}
\captionof{table}{એડ્રેસિંગ મોડ્સ}
\begin{tabulary}{\linewidth}{|l|l|J|}
\hline
\textbf{એડ્રેસિંગ મોડ} & \textbf{ઉદાહરણ} & \textbf{વર્ણન} \\ \hline
\textbf{રજિસ્ટર} & \code{MOV A, R0} & ઓપરેન્ડ રજિસ્ટરમાં \\ \hline
\textbf{ડાયરેક્ટ} & \code{MOV A, 30H} & ઓપરેન્ડ ચોક્કસ મેમરી લોકેશન પર (30H) \\ \hline
\textbf{રજિસ્ટર ઇન્ડાયરેક્ટ} & \code{MOV A, @R0} & રજિસ્ટરમાં ઓપરેન્ડનું એડ્રેસ \\ \hline
\textbf{ઈમીડિયેટ} & \code{MOV A, \#55H} & ઓપરેન્ડ ઇન્સ્ટ્રક્શનનો ભાગ છે \\ \hline
\textbf{ઇન્ડેક્સ્ડ} & \code{MOVC A, @A+DPTR} & બેઝ એડ્રેસ (DPTR) + ઓફસેટ (A) \\ \hline
\textbf{બિટ} & \code{SETB P1.0} & વ્યક્તિગત બિટ એડ્રેસેબલ \\ \hline
\textbf{ઇમ્પ્લાઈડ} & \code{RRC A} & ઓપરેન્ડ ઇન્સ્ટ્રક્શન દ્વારા સૂચિત \\ \hline
\end{tabulary}
\end{center}

\textbf{ડાયાગ્રામ:}

\begin{center}
\begin{tikzpicture}[node distance=3cm]
    \node [gtu block] (immed) {Immediate\\\#Data};
    \node [gtu block, right of=immed] (direct) {Direct\\Address};
    \node [gtu block, right of=direct] (reg) {Register\\Rn};
    
    \node [gtu block, below of=immed, node distance=2cm] (indirect) {Indirect\\@Rn};
    \node [gtu block, right of=indirect] (index) {Indexed\\@A+DPTR};
    
    \node [draw, fit=(immed) (index), label=above:Address Modes] {};
\end{tikzpicture}
\end{center}
\end{solutionbox}
\begin{mnemonicbox}
``RIDDIB'' - ``રજિસ્ટર, ઈમીડિયેટ, ડાયરેક્ટ, ડેટા ઇન્ડાયરેક્ટ, ઇન્ડેક્સ્ડ, બિટ''
\end{mnemonicbox}

\orquestionmarks{4}{a}{3}
\textbf{08H અને 02H નો ગુણાકાર કરવા માટે એસેમ્બલી ભાષામાં પ્રોગ્રામ વિકસાવો.}

\begin{solutionbox}
\textbf{જવાબ}:

\begin{minipage}{\linewidth}
\begin{lstlisting}[language={[x86masm]Assembler}]
      MOV A, #08H    ; પહેલો નંબર 08H એક્યુમુલેટરમાં લોડ કરો
      MOV B, #02H    ; બીજો નંબર 02H B રજિસ્ટરમાં લોડ કરો
      MUL AB         ; A અને B નો ગુણાકાર કરો (B:A = પરિણામ)
      MOV R0, A      ; લો-બાઈટ પરિણામ R0 માં સ્ટોર કરો (10H)
      MOV R1, B      ; હાઈ-બાઈટ પરિણામ R1 માં સ્ટોર કરો (00H)
\end{lstlisting}
\end{minipage}

\textbf{ડાયાગ્રામ:}

\begin{center}
\begin{tikzpicture}[node distance=2cm]
    \node [draw, rectangle, minimum width=2cm, minimum height=1cm] (before) {A: 08H\\B: 02H};
    \node [draw, rectangle, minimum width=2cm, minimum height=1cm, right of=before, node distance=4cm] (after) {A: 10H (Low)\\B: 00H (High)};
    \draw [->, thick] (before) -- (after) node[midway, above] {MUL AB};
\end{tikzpicture}
\end{center}
\end{solutionbox}
\begin{mnemonicbox}
``LMSR'' - ``લોડ નંબર્સ, મલ્ટિપ્લાય, સ્ટોર રિઝલ્ટ''
\end{mnemonicbox}

\orquestionmarks{4}{b}{4}
\textbf{76H માંથી 32H બાદ કરવા માટે એસેમ્બલી ભાષામાં પ્રોગ્રામ વિકસાવો.}

\begin{solutionbox}
\textbf{જવાબ}:

\begin{minipage}{\linewidth}
\begin{lstlisting}[language={[x86masm]Assembler}]
      MOV A, #32H    ; 32H એક્યુમુલેટરમાં લોડ કરો
      MOV R0, #76H   ; 76H R0 માં લોડ કરો
      CLR C          ; કેરી ફ્લેગ ક્લિયર કરો (બોરો ફ્લેગ)
      SUBB A, R0     ; A માંથી R0 બોરો સાથે બાદ કરો (32H - 76H = BCH)
      MOV R1, A      ; પરિણામ R1 માં સ્ટોર કરો (BCH, જે -44H દર્શાવે છે)
\end{lstlisting}
\end{minipage}

\textbf{ડાયાગ્રામ:}

\begin{center}
\begin{tikzpicture}[node distance=1.5cm]
    \node (eq) {32H - 76H = BCH (-44H)};
    \node [below of=eq] {બોરો ફ્લેગ (CY) = 1};
\end{tikzpicture}
\end{center}
\end{solutionbox}
\begin{mnemonicbox}
``LESS'' - ``લોડ ફર્સ્ટ નંબર, એનેબલ બોરો (CLR C), સબટ્રેક્ટ, સ્ટોર''
\end{mnemonicbox}

\orquestionmarks{4}{c}{7}
\textbf{Instruction set ના પ્રકારોની સૂચિ બનાવો. કોઈપણ ત્રણને એક ઉદાહરણ સાથે સમજાવો.}

\begin{solutionbox}
\textbf{જવાબ}:

\begin{center}
\captionof{table}{ઇન્સ્ટ્રક્શન પ્રકારો}
\begin{tabulary}{\linewidth}{|l|J|l|}
\hline
\textbf{ઇન્સ્ટ્રક્શન ગ્રુપ} & \textbf{વર્ણન} & \textbf{ઉદાહરણ} \\ \hline
\textbf{અર્થમેટિક} & ગાણિતિક ઓપરેશન & \code{ADD A, R0} \\ \hline
\textbf{લોજિકલ} & લોજિકલ ઓપરેશન & \code{ANL A, \#0FH} \\ \hline
\textbf{ડેટા ટ્રાન્સફર} & લોકેશન વચ્ચે ડેટા ખસેડો & \code{MOV A, R7} \\ \hline
\textbf{બ્રાન્ચ} & પ્રોગ્રામ ફ્લો બદલો & \code{JNZ LOOP} \\ \hline
\textbf{બિટ મેનિપ્યુલેશન} & વ્યક્તિગત બિટ પર ઓપરેશન & \code{SETB P1.0} \\ \hline
\textbf{મશીન કંટ્રોલ} & પ્રોસેસર ઓપરેશન કંટ્રોલ & \code{NOP} \\ \hline
\end{tabulary}
\end{center}

\textbf{સમજૂતી}:
\begin{enumerate}
    \item \textbf{ડેટા ટ્રાન્સફર ઇન્સ્ટ્રક્શન્સ}:
    \begin{itemize}
        \item રજિસ્ટર્સ, મેમરી, અથવા I/O પોર્ટ્સ વચ્ચે ડેટા ખસેડે છે.
        \item ઉદાહરણ: \code{MOV A, 30H} (મેમરી લોકેશન 30H માંથી એક્યુમુલેટરમાં ડેટા ખસેડે છે).
    \end{itemize}
    \item \textbf{અર્થમેટિક ઇન્સ્ટ્રક્શન્સ}:
    \begin{itemize}
        \item ગાણિતિક ઓપરેશન કરે છે.
        \item ઉદાહરણ: \code{ADD A, R0} (R0 ની સામગ્રી એક્યુમુલેટરમાં ઉમેરે છે).
    \end{itemize}
    \item \textbf{લોજિકલ ઇન્સ્ટ્રક્શન્સ}:
    \begin{itemize}
        \item લોજિકલ ઓપરેશન (AND, OR, XOR) કરે છે.
        \item ઉદાહરણ: \code{ANL A, \#0FH} (A ને 0FH સાથે AND કરે છે).
    \end{itemize}
\end{enumerate}

\textbf{ડાયાગ્રામ:}

\begin{center}
\begin{tikzpicture}
  \path[mindmap,concept color=black!10,text=black]
    node[concept] {Instruction Examples}
    [clockwise from=0]
    child { node[concept] {Data Transfer\\(MOV)} }
    child { node[concept] {Arithmetic\\(ADD)} }
    child { node[concept] {Logical\\(ANL)} }
    child { node[concept] {Branch\\(JNZ)} }
    child { node[concept] {Bit\\(SETB)} };
\end{tikzpicture}
\end{center}
\end{solutionbox}
\begin{mnemonicbox}
``BALDM'' - ``બ્રાન્ચ, અર્થમેટિક, લોજિકલ, ડેટા ટ્રાન્સફર, મશીન કંટ્રોલ''
\end{mnemonicbox}

\questionmarks{5}{a}{3}
\textbf{8051 માઇક્રોકન્ટ્રોલર સાથે ચાર એલઇડીનું ઇન્ટરફેસિંગ દોરો.}

\begin{solutionbox}
\textbf{જવાબ}:

\textbf{ડાયાગ્રામ:}

\begin{center}
\begin{tikzpicture}[node distance=2.5cm]
    \node [gtu block, minimum width=6cm] (8051) {8051 Microcontroller};
    
    \foreach \i in {0,1,2,3} {
        \node (p\i) at (-1.5 + \i, 1) {\scriptsize P1.\i};
        \draw [<-] (p\i.north) -- ++(0, 0.5) coordinate (mid\i);
        \draw (mid\i) -- ++(0, 0.5) node[midway, draw, rectangle, minimum width=0.3cm, minimum height=0.6cm, fill=white] {} coordinate (led\i); % LED
        \node at (led\i) {\tiny LED};
        \draw (led\i) ++(0, 0.3) -- ++(0, 0.5) node[midway, draw, rectangle, minimum width=0.2cm, minimum height=0.5cm] {}; % Resistor
        \node at ($(led\i)+(0.5,0.6)$) {\tiny $220\Omega$};
        \draw ($(led\i)+(0, 0.8)$) -- ++(0, 0.2) node[above] {+5V};
    }
\end{tikzpicture}
\end{center}

\textbf{વિગતો}:
\begin{itemize}
    \item LEDs પોર્ટ 1 (P1.0-P1.3) સાથે જોડાયેલ છે.
    \item કોમન એનોડ કન્ફિગરેશન (એક્ટિવ લો).
    \item $220\Omega$ રેસિસ્ટર કરંટ લિમિટિંગ માટે છે.
\end{itemize}
\end{solutionbox}
\begin{mnemonicbox}
``PALS'' - ``Port pins, Active-low control, LEDs, Simple circuit''
\end{mnemonicbox}

\questionmarks{5}{b}{4}
\textbf{8051 માઇક્રોકન્ટ્રોલર સાથે 7 સેગમેન્ટ LED નું ઇન્ટરફેસિંગ દોરો.}

\begin{solutionbox}
\textbf{જવાબ}:

\textbf{ડાયાગ્રામ:}

\begin{center}
\begin{tikzpicture}[node distance=2.5cm, auto]
    \node [gtu block, minimum height=3cm] (cpu) {8051\\(Port 1)};
    \node [gtu block, minimum height=3cm, right of=cpu, node distance=4cm] (disp) {7-Segment\\Display\\(Common\\Cathode)};
    
    \draw [->] (cpu.east) -- (disp.west) node[midway, above] {a-g lines};
    \draw [->] (disp.south) -- ++(0, -1) node[below] {GND};
    
    \node [above of=disp] {Display};
\end{tikzpicture}
\end{center}

\textbf{કોડ ઉદાહરણ}:
\begin{minipage}{\linewidth}
\begin{lstlisting}[language={[x86masm]Assembler}]
; Display digit 5 (Pattern: 6DH)
MOV A, #6DH      ; 5 માટે સેગમેન્ટ પેટર્ન (a,c,d,f,g ON)
MOV P1, A        ; સેગમેન્ટ ડ્રાઇવ કરવા માટે પોર્ટ P1 પર મોકલો
\end{lstlisting}
\end{minipage}
\end{solutionbox}
\begin{mnemonicbox}
``SPACE-7'' - ``Seven Pins, Active segments, Common ground, Easy display''
\end{mnemonicbox}

\questionmarks{5}{c}{7}
\textbf{8051 માઇક્રોકન્ટ્રોલર સાથે DAC નું ઇન્ટરફેસિંગ સમજાવો અને જરૂરી પ્રોગ્રામ લખો.}

\begin{solutionbox}
\textbf{જવાબ}:

\textbf{ડાયાગ્રામ:}

\begin{center}
\begin{tikzpicture}[node distance=2.5cm, auto]
    \node [gtu block] (cpu) {8051};
    \node [gtu block, right of=cpu, node distance=3.5cm] (dac) {DAC0808};
    \node [gtu block, right of=dac, node distance=3cm] (filter) {Op-Amp\\Filter};
    
    \draw [->, thick] (cpu) -- (dac) node[midway, above] {P1 (Data)};
    \draw [->, dashed] (cpu) to[bend right] node[below] {Control} (dac);
    \draw [->, thick] (dac) -- (filter) node[midway, above] {Iout};
    \draw [->, thick] (filter) -- ++(2,0) node[right] {Analog Out};
\end{tikzpicture}
\end{center}

\textbf{પ્રોગ્રામ (સોટૂથ વેવ)}:
\begin{minipage}{\linewidth}
\begin{lstlisting}[language={[x86masm]Assembler}]
START:  MOV R0, #00H     ; R0 ને 0 થી શરૂ કરો
LOOP:   MOV P1, R0       ; કિંમત DAC પર આઉટપુટ કરો
        CALL DELAY       ; થોડી વાર રાહ જુઓ
        INC R0           ; કિંમત વધારો (00->FF->00)
        SJMP LOOP        ; સોટૂથ વેવ બનાવવા માટે પુનરાવર્તન કરો

DELAY:  MOV R7, #50      ; ડિલે કાઉન્ટર લોડ કરો
        DJNZ R7, $       ; લૂપ
        RET
\end{lstlisting}
\end{minipage}
\end{solutionbox}
\begin{mnemonicbox}
``DICAF'' - ``Digital input, Increment, Convert to analog, Amplify, Filter''
\end{mnemonicbox}

\orquestionmarks{5}{a}{3}
\textbf{8051 માઇક્રોકન્ટ્રોલર સાથે ચાર સ્વિચનું ઇન્ટરફેસિંગ દોરો.}

\begin{solutionbox}
\textbf{જવાબ}:

\textbf{ડાયાગ્રામ:}

\begin{center}
\begin{tikzpicture}[auto]
    \node [gtu block] (cpu) {8051};
    
    \foreach \i in {0,1,2,3} {
        \coordinate (pin\i) at ($(cpu.west)+(0, 1.0 - \i*0.6)$);
        \node [left] at (pin\i) {P1.\i};
        \draw (pin\i) -- ++(-1, 0) coordinate (node\i);
        
        % Pull up
        \draw (node\i) -- ++(0, 0.5) node[draw, rectangle, minimum width=0.2cm, minimum height=0.4cm] {} -- ++(0, 0.3) node[above] {+5V};
        
        % Switch
        \draw (node\i) -- ++(0, -0.5) coordinate (sw\i);
        \draw (sw\i) -- ++(-0.2, 0) -- ++(0.4, 0.2); % Open switch symbol
        \node at ($(sw\i)+(-0.3,-0.2)$) {SW};
        \draw ($(sw\i)+(0,-0.3)$) -- ++(0, -0.2) node[below] {GND};
    }
\end{tikzpicture}
\end{center}
\end{solutionbox}
\begin{mnemonicbox}
``PIPS'' - ``Pull-ups, Input pins, Press for zero, Switches''
\end{mnemonicbox}

\orquestionmarks{5}{b}{4}
\textbf{8051 માઇક્રોકન્ટ્રોલર સાથે સ્ટેપર મોટરનું ઇન્ટરફેસિંગ દોરો.}

\begin{solutionbox}
\textbf{જવાબ}:

\textbf{ડાયાગ્રામ:}

\begin{center}
\begin{tikzpicture}[node distance=2.5cm, auto]
    \node [gtu block] (cpu) {8051};
    \node [gtu block, right of=cpu, node distance=3.5cm] (driver) {ULN2003\\Driver};
    \node [gtu block, right of=driver, node distance=3.5cm] (motor) {Stepper\\Motor};
    
    \draw [->, thick] (cpu) -- (driver) node[midway, above] {P1.0-P1.3};
    \draw [->, thick] (driver) -- (motor) node[midway, above] {A, B, C, D};
    
    \node [above of=driver] {+12V Supply};
    \draw [->] ($(driver)+(0,1)$) -- (driver);
\end{tikzpicture}
\end{center}
\end{solutionbox}
\begin{mnemonicbox}
``CUPS'' - ``Controller outputs sequence, ULN2003 amplifies, Phases energized, Stepping motion''
\end{mnemonicbox}

\orquestionmarks{5}{c}{7}
\textbf{8051 માઇક્રોકન્ટ્રોલર સાથે ADC નું ઇન્ટરફેસિંગ સમજાવો અને જરૂરી પ્રોગ્રામ લખો.}

\begin{solutionbox}
\textbf{જવાબ}:

\textbf{ડાયાગ્રામ:}

\begin{center}
\begin{tikzpicture}[node distance=3cm, auto]
    \node [gtu block] (adc) {ADC0804};
    \node [gtu block, right of=adc, node distance=4cm] (cpu) {8051};
    
    \draw [->, thick] (adc) -- (cpu) node[midway, above] {Data (D0-D7)};
    \draw [->, thick] (cpu.south) ++(-1,0) -- ++(0, -1) -| (adc.south) node[midway, below] {Control (RD, WR, CS)};
    
    \node [left of=adc, node distance=2.5cm] {Analog In};
    \draw [->] ($(adc)+(-2.5,0)$) -- (adc);
\end{tikzpicture}
\end{center}

\textbf{પ્રોગ્રામ}:
\begin{minipage}{\linewidth}
\begin{lstlisting}[language={[x86masm]Assembler}]
START:  MOV P1, #0FFH    ; P1 ને ઇનપુટ તરીકે કન્ફિગર કરો
READ:   CLR P3.0         ; CS = 0 (ADC પસંદ કરો)
        CLR P3.2         ; WR = 0 (કન્વર્ઝન શરૂ કરો)
        NOP
        SETB P3.2        ; WR = 1 (Latch Start)
        
WAIT:   JB P3.3, WAIT    ; INTR = 0 (EOC) ની રાહ જુઓ
        
        CLR P3.1         ; RD = 0 (ડેટા આઉટપુટ એનેબલ)
        MOV A, P1        ; ડેટા વાંચો
        SETB P3.1        ; RD = 1
        SETB P3.0        ; CS = 1 (Deselect)
        
        MOV R0, A        ; ડેટા સ્ટોર કરો
        SJMP READ        ; પુનરાવર્તન કરો
\end{lstlisting}
\end{minipage}
\end{solutionbox}
\begin{mnemonicbox}
``CARSW'' - ``Convert Analog, Read Digital, Start conversion, Wait for completion''
\end{mnemonicbox}

\end{document}
