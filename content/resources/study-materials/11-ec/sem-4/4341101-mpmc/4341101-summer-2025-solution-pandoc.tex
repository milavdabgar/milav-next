\documentclass[10pt,a4paper]{article}

% content/resources/templates/preamble.tex
\usepackage[margin=0.6in]{geometry}
\author{Milav Dabgar}
\usepackage{amsmath,amssymb,amsthm}
\usepackage{booktabs}
\usepackage{multirow}
\usepackage{xcolor}
\usepackage{tcolorbox}
\tcbuselibrary{breakable,skins}
\usepackage[colorlinks=true,linkcolor=blue]{hyperref}
\usepackage{titlesec}
\usepackage{enumitem}
\usepackage{tikz}
\usepackage{pgfplots}
\usepackage{circuitikz}
\usepackage[version=4]{mhchem}
\usepackage{longtable}
\usepackage{array}
\usepackage{float}
\usepackage{caption}
\usepackage{listings}

\lstset{
  basicstyle=\small\ttfamily,
  breaklines=true,
  breakatwhitespace=false,
  postbreak=\mbox{\textcolor{red}{$\hookrightarrow$}\space},
  float=false,
  numbers=left,
  numberstyle=\tiny\color{gray},
  numbersep=10pt,
  xleftmargin=2em,
  keywordstyle=\color{blue},
  commentstyle=\color{green!60!black},
  stringstyle=\color{purple},
  backgroundcolor=\color{gray!5},
  showstringspaces=false,
  tabsize=2,
  captionpos=b,
  keepspaces=true,
  columns=flexible
}

\pgfplotsset{compat=1.18}
\usetikzlibrary{shapes,arrows,positioning,calc,patterns,decorations.pathmorphing,decorations.markings,arrows.meta}

% Color scheme
\definecolor{headcolor}{RGB}{0,102,204}
\definecolor{keycolor}{RGB}{220,20,60}
\definecolor{solutioncolor}{RGB}{34,139,34}
\definecolor{mnemoniccolor}{RGB}{148,0,211}
\definecolor{codecolor}{RGB}{0,0,100}

% Spacing
\setlength{\parskip}{3pt}
\setlist[itemize]{nosep}
\setlist[enumerate]{nosep}

% Title formatting
\titleformat{\section}{\Large\bfseries\color{headcolor}}{\thesection}{1em}{}
\titleformat{\subsection}{\large\bfseries\color{headcolor}}{\thesubsection}{1em}{}

% Pandoc tightlist compatibility
\providecommand{\tightlist}{%
  \setlength{\itemsep}{0pt}\setlength{\parskip}{0pt}}

% Pandoc longtable compatibility
\newcounter{none}
\def\thenone{}


% content/resources/templates/english-boxes.tex
% This file is currently empty - it exists to maintain consistency with the import structure.
% Add custom environments here if needed in the future.


\begin{document}

\begin{center}
{\Huge\bfseries\color{headcolor} Subject Name Solutions}\\[5pt]
{\LARGE 4341101 -- Summer 2025}\\[3pt]
{\large Semester 1 Study Material}\\[3pt]
{\normalsize\textit{Detailed Solutions and Explanations}}
\end{center}

\vspace{10pt}

\subsection*{Question 1(a) [3 marks]}\label{q1a}

\textbf{Define Microprocessor and draw its block diagram.}

\begin{solutionbox}
A \textbf{microprocessor} is a programmable digital
device that performs arithmetic and logical operations on data according
to stored instructions.

\textbf{Block Diagram:}

\begin{center}
\textbf{Mermaid Diagram (Code)}
\begin{verbatim}
{Shaded}
{Highlighting}[]
graph LR
    A[Input Device] {-{-}{} B[CPU]}
    B {-{-}{} C[Output Device]}
    B {{-}{-}{} D[Memory Unit]}
    B {-{-}{} E[Control Unit]}
    B {-{-}{} F[ALU]}
    E {-{-}{} G[Control Signals]}
    F {-{-}{} H[Arithmetic \& Logic Operations]}
{Highlighting}
{Shaded}
\end{verbatim}
\end{center}

\begin{itemize}
\tightlist
\item
  \textbf{CPU}: Central Processing Unit performs all operations
\item
  \textbf{Memory}: Stores programs and data
\item
  \textbf{Control Unit}: Controls instruction execution sequence
\end{itemize}

\end{solutionbox}
\begin{mnemonicbox}
``My Computer Processes Instructions''
(Memory-CPU-Program-Instructions)

\end{mnemonicbox}
\subsection*{Question 1(b) [4 marks]}\label{q1b}

\textbf{Explain operand and opcode with proper instruction example.}

\begin{solutionbox}
\textbf{Opcode} specifies the operation to be
performed. \textbf{Operand} specifies the data on which operation is
performed.

\textbf{Example Table:}

{\def\LTcaptype{none} % do not increment counter
\begin{longtable}[]{@{}llll@{}}
\toprule\noalign{}
Instruction & Opcode & Operand & Function \\
\midrule\noalign{}
\endhead
\bottomrule\noalign{}
\endlastfoot
MOV A,B & MOV & A,B & Move B to A \\
ADD A,\#05H & ADD & A,\#05H & Add 05H to A \\
\end{longtable}
}

\begin{itemize}
\tightlist
\item
  \textbf{Opcode}: Operation code (MOV, ADD, SUB)
\item
  \textbf{Operand}: Data or address (A, B, \#05H)
\item
  \textbf{Format}: Opcode + Operand = Complete Instruction
\end{itemize}

\end{solutionbox}
\begin{mnemonicbox}
``Operation On Data'' (Opcode-Operand-Data)

\end{mnemonicbox}
\subsection*{Question 1(c) [7 marks]}\label{q1c}

\textbf{Compare Microprocessor and Microcontroller.}

\begin{solutionbox}

{\def\LTcaptype{none} % do not increment counter
\begin{longtable}[]{@{}lll@{}}
\toprule\noalign{}
Parameter & Microprocessor & Microcontroller \\
\midrule\noalign{}
\endhead
\bottomrule\noalign{}
\endlastfoot
\textbf{Definition} & CPU only & CPU + Memory + I/O \\
\textbf{Memory} & External RAM/ROM & Internal RAM/ROM \\
\textbf{I/O Ports} & External interface & Built-in ports \\
\textbf{Cost} & Higher system cost & Lower system cost \\
\textbf{Power} & Higher consumption & Lower consumption \\
\textbf{Speed} & Faster processing & Moderate speed \\
\textbf{Applications} & Computers, laptops & Washing machine,
microwave \\
\end{longtable}
}

\begin{itemize}
\tightlist
\item
  \textbf{Microprocessor}: General purpose computing
\item
  \textbf{Microcontroller}: Specific embedded applications
\item
  \textbf{Integration}: Microcontroller has everything on single chip
\end{itemize}

\end{solutionbox}
\begin{mnemonicbox}
``Micro Means More Integration''
(Microcontroller-Memory-More-Integration)

\end{mnemonicbox}
\subsection*{Question 1(c OR) [7
marks]}\label{question-1c-or-7-marks}

\textbf{Compare RISC and CISC.}

\begin{solutionbox}

{\def\LTcaptype{none} % do not increment counter
\begin{longtable}[]{@{}lll@{}}
\toprule\noalign{}
Parameter & RISC & CISC \\
\midrule\noalign{}
\endhead
\bottomrule\noalign{}
\endlastfoot
\textbf{Instructions} & Simple, few & Complex, many \\
\textbf{Instruction Size} & Fixed length & Variable length \\
\textbf{Execution Time} & Single cycle & Multiple cycles \\
\textbf{Memory Access} & Load/Store only & Any instruction \\
\textbf{Registers} & More registers & Fewer registers \\
\textbf{Pipeline} & Efficient pipelining & Complex pipelining \\
\textbf{Examples} & ARM, MIPS & x86, 8085 \\
\end{longtable}
}

\begin{itemize}
\tightlist
\item
  \textbf{RISC}: Reduced Instruction Set Computer
\item
  \textbf{CISC}: Complex Instruction Set Computer
\item
  \textbf{Performance}: RISC faster, CISC more flexible
\end{itemize}

\end{solutionbox}
\begin{mnemonicbox}
``Reduced Instructions Speed Computing''
(RISC-Instructions-Speed-Computing)

\end{mnemonicbox}
\subsection*{Question 2(a) [3 marks]}\label{q2a}

\textbf{Explain Bus Organization of 8085 microprocessor.}

\begin{solutionbox}
8085 has \textbf{three types} of buses for
communication with external devices.

\textbf{Bus Organization Table:}

{\def\LTcaptype{none} % do not increment counter
\begin{longtable}[]{@{}lll@{}}
\toprule\noalign{}
Bus Type & Lines & Function \\
\midrule\noalign{}
\endhead
\bottomrule\noalign{}
\endlastfoot
\textbf{Address Bus} & 16 lines (A0-A15) & Memory addressing \\
\textbf{Data Bus} & 8 lines (D0-D7) & Data transfer \\
\textbf{Control Bus} & Multiple lines & Control signals \\
\end{longtable}
}

\begin{itemize}
\tightlist
\item
  \textbf{Address Bus}: Unidirectional, 64KB memory addressing
\item
  \textbf{Data Bus}: Bidirectional, 8-bit data transfer
\item
  \textbf{Control Bus}: Read, Write, IO/M signals
\end{itemize}

\end{solutionbox}
\begin{mnemonicbox}
``Address Data Control'' (ADC)

\end{mnemonicbox}
\subsection*{Question 2(b) [4 marks]}\label{q2b}

\textbf{Explain function of ALE signal with diagram.}

\begin{solutionbox}
\textbf{ALE (Address Latch Enable)} separates address
and data on multiplexed bus.

\textbf{ALE Timing Diagram:}

\begin{verbatim}
    ALE  \_\_\_\_      \_\_\_\_
        |    |\_\_\_\_|    |\_\_\_\_
        
  AD7{-0 ADDR |   DATA  | ADDR}
        \_\_\_\_\_|\_\_\_\_\_\_\_\_\_|\_\_\_\_\_
\end{verbatim}

\begin{itemize}
\tightlist
\item
  \textbf{High ALE}: Address is available on AD0-AD7
\item
  \textbf{Low ALE}: Data is available on AD0-AD7
\item
  \textbf{Function}: Latches lower address byte
\item
  \textbf{Frequency}: ALE = Clock frequency \div 2
\end{itemize}

\end{solutionbox}
\begin{mnemonicbox}
``Address Latch Enable'' (ALE)

\end{mnemonicbox}
\subsection*{Question 2(c) [7 marks]}\label{q2c}

\textbf{Describe architecture of 8085 microprocessor with the help of
neat diagram.}

\begin{solutionbox}

\begin{center}
\textbf{Mermaid Diagram (Code)}
\begin{verbatim}
{Shaded}
{Highlighting}[]
graph TD
    A[Accumulator A] {-{-}{} B[ALU]}
    C[Temp Register] {-{-}{} B}
    B {-{-}{} D[Flag Register]}
    E[B,C,D,E,H,L Registers] {-{-}{} F[Address Buffer]}
    G[Program Counter] {-{-}{} F}
    H[Stack Pointer] {-{-}{} F}
    F {-{-}{} I[Address Bus A0{-}A15]}
    J[Data/Address Buffer] {-{-}{} K[Data Bus AD0{-}AD7]}
    L[Instruction Register] {-{-}{} M[Instruction Decoder]}
    M {-{-}{} N[Control Unit]}
    N {-{-}{} O[Control Signals]}
{Highlighting}
{Shaded}
\end{verbatim}
\end{center}

\textbf{Key Components:}

\begin{itemize}
\tightlist
\item
  \textbf{ALU}: Performs arithmetic and logical operations
\item
  \textbf{Registers}: Store temporary data (A, B, C, D, E, H, L)
\item
  \textbf{Program Counter}: Points to next instruction
\item
  \textbf{Stack Pointer}: Points to stack top
\item
  \textbf{Control Unit}: Generates control signals
\end{itemize}

\end{solutionbox}
\begin{mnemonicbox}
``All Registers Program Stack Control'' (A-R-P-S-C)

\end{mnemonicbox}
\subsection*{Question 2(a OR) [3
marks]}\label{question-2a-or-3-marks}

\textbf{Draw Flag Register of 8085 microprocessor \& explain it.}

\begin{solutionbox}

\textbf{Flag Register Format:}

\begin{verbatim}
 D7  D6  D5  D4  D3  D2  D1  D0
+{-{-}{-}+{-}{-}{-}+{-}{-}{-}+{-}{-}{-}+{-}{-}{-}+{-}{-}{-}+{-}{-}{-}+{-}{-}{-}+}
| S | Z | 0 |AC | 0 | P | 1 | C |
+{-{-}{-}+{-}{-}{-}+{-}{-}{-}+{-}{-}{-}+{-}{-}{-}+{-}{-}{-}+{-}{-}{-}+{-}{-}{-}+}
\end{verbatim}

\textbf{Flag Functions:}

\begin{itemize}
\tightlist
\item
  \textbf{S (Sign)}: Set if result is negative
\item
  \textbf{Z (Zero)}: Set if result is zero
\item
  \textbf{AC (Auxiliary Carry)}: Set for BCD operations
\item
  \textbf{P (Parity)}: Set for even parity
\item
  \textbf{C (Carry)}: Set when carry/borrow occurs
\end{itemize}

\end{solutionbox}
\begin{mnemonicbox}
``Some Zero Auxiliary Parity Carry'' (SZAPC)

\end{mnemonicbox}
\subsection*{Question 2(b OR) [4
marks]}\label{question-2b-or-4-marks}

\textbf{Explain De-multiplexing of Address and Data buses for 8085
Microprocessor.}

\begin{solutionbox}
\textbf{De-multiplexing} separates address and data
signals from AD0-AD7 lines.

\textbf{De-multiplexing Circuit:}

\begin{verbatim}
AD0{-AD7 {-}{-}{-}{-}+{-}{-}{-}{-} D{-}Latch {-}{-}{-}{-} A0{-}A7 (Address)}
            |         \^{}
            |         |
            |       ALE
            |
            +{-{-}{-}{-} Data Buffer {-}{-}{-}{-} D0{-}D7 (Data)}
\end{verbatim}

\begin{itemize}
\tightlist
\item
  \textbf{ALE High}: Address latched in external latch
\item
  \textbf{ALE Low}: Data flows through buffer
\item
  \textbf{74LS373}: Common latch IC used
\item
  \textbf{Benefit}: Separate address and data buses
\end{itemize}

\end{solutionbox}
\begin{mnemonicbox}
``Address Latch External Demultiplex'' (ALED)

\end{mnemonicbox}
\subsection*{Question 2(c OR) [7
marks]}\label{question-2c-or-7-marks}

\textbf{Describe Pin diagram of 8085 microprocessor with the help of
neat diagram.}

\begin{solutionbox}

\begin{verbatim}
         8085 Microprocessor
        +{-{-}{-}{-}{-}{-}{-}{-}{-}{-}{-}{-}{-}{-}{-}{-}{-}{-}{-}+}
   X1 {-{-}| 1              40 |{-}{-} VCC}
   X2 {-{-}| 2              39 |{-}{-} HOLD}
RESET {-{-}| 3              38 |{-}{-} HLDA}
  SOD {-{-}| 4              37 |{-}{-} CLK}
  SID {-{-}| 5              36 |{-}{-} RESET}
 TRAP {-{-}| 6              35 |{-}{-} READY}
RST7.5{-{-}| 7              34 |{-}{-} IO/M}
RST6.5{-{-}| 8              33 |{-}{-} S1}
RST5.5{-{-}| 9              32 |{-}{-} RD}
 INTR {-{-}| 10             31 |{-}{-} WR}
 INTA {-{-}| 11             30 |{-}{-} ALE}
  AD0 {-{-}| 12             29 |{-}{-} S0}
  AD1 {-{-}| 13             28 |{-}{-} A15}
  AD2 {-{-}| 14             27 |{-}{-} A14}
  AD3 {-{-}| 15             26 |{-}{-} A13}
  AD4 {-{-}| 16             25 |{-}{-} A12}
  AD5 {-{-}| 17             24 |{-}{-} A11}
  AD6 {-{-}| 18             23 |{-}{-} A10}
  AD7 {-{-}| 19             22 |{-}{-} A9}
  VSS {-{-}| 20             21 |{-}{-} A8}
        +{-{-}{-}{-}{-}{-}{-}{-}{-}{-}{-}{-}{-}{-}{-}{-}{-}{-}{-}+}
\end{verbatim}

\textbf{Pin Categories:}

\begin{itemize}
\tightlist
\item
  \textbf{Power}: VCC, VSS
\item
  \textbf{Clock}: X1, X2, CLK
\item
  \textbf{Address/Data}: AD0-AD7, A8-A15
\item
  \textbf{Control}: ALE, RD, WR, IO/M
\item
  \textbf{Interrupt}: INTR, INTA, RST7.5, RST6.5, RST5.5, TRAP
\end{itemize}

\end{solutionbox}
\begin{mnemonicbox}
``Power Clock Address Control Interrupt'' (PCACI)

\end{mnemonicbox}
\subsection*{Question 3(a) [3 marks]}\label{q3a}

\textbf{Write a function of DPTR and PC.}

\begin{solutionbox}

\textbf{Functions Table:}

{\def\LTcaptype{none} % do not increment counter
\begin{longtable}[]{@{}lll@{}}
\toprule\noalign{}
Register & Function & Size \\
\midrule\noalign{}
\endhead
\bottomrule\noalign{}
\endlastfoot
\textbf{DPTR} & Data Pointer & 16-bit \\
\textbf{PC} & Program Counter & 16-bit \\
\end{longtable}
}

\textbf{DPTR Functions:}

\begin{itemize}
\tightlist
\item
  \textbf{External Memory}: Access external data memory
\item
  \textbf{Addressing}: 16-bit address for MOVX instructions
\end{itemize}

\textbf{PC Functions:}

\begin{itemize}
\tightlist
\item
  \textbf{Instruction Pointer}: Points to next instruction
\item
  \textbf{Auto Increment}: Increments after each instruction fetch
\end{itemize}

\end{solutionbox}
\begin{mnemonicbox}
``Data Program Counter'' (DPC)

\end{mnemonicbox}
\subsection*{Question 3(b) [4 marks]}\label{q3b}

\textbf{Draw PCON SFR of 8051and Explain function of each bit.}

\begin{solutionbox}

\textbf{PCON Register (87H):}

\begin{verbatim}
 D7  D6  D5  D4  D3  D2  D1  D0
+{-{-}{-}+{-}{-}{-}+{-}{-}{-}+{-}{-}{-}+{-}{-}{-}+{-}{-}{-}+{-}{-}{-}+{-}{-}{-}+}
|SMOD| {- | {-} | {-} |GF1|GF0|PD |IDL|}
+{-{-}{-}+{-}{-}{-}+{-}{-}{-}+{-}{-}{-}+{-}{-}{-}+{-}{-}{-}+{-}{-}{-}+{-}{-}{-}+}
\end{verbatim}

\textbf{Bit Functions:}

\begin{itemize}
\tightlist
\item
  \textbf{SMOD}: Serial port baud rate doubler
\item
  \textbf{GF1, GF0}: General purpose flags
\item
  \textbf{PD}: Power Down mode control\\
\item
  \textbf{IDL}: Idle mode control
\end{itemize}

\textbf{Power Management:}

\begin{itemize}
\tightlist
\item
  \textbf{IDL = 1}: CPU stops, peripherals run
\item
  \textbf{PD = 1}: Complete power down
\end{itemize}

\end{solutionbox}
\begin{mnemonicbox}
``Serial General Power Idle'' (SGPI)

\end{mnemonicbox}
\subsection*{Question 3(c) [7 marks]}\label{q3c}

\textbf{Explain architecture of 8051 microcontroller with the help of
neat diagram.}

\begin{solutionbox}

\begin{center}
\textbf{Mermaid Diagram (Code)}
\begin{verbatim}
{Shaded}
{Highlighting}[]
graph TD
    A[CPU Core] {-{-}{} B[ALU]}
    A {-{-}{} C[Accumulator A]}
    A {-{-}{} D[B Register]}
    A {-{-}{} E[PSW]}
    F[Program Memory ROM] {-{-}{} G[Program Counter PC]}
    H[Data Memory RAM] {-{-}{} I[Data Pointer DPTR]}
    J[Timer 0] {-{-}{} K[Timer Control]}
    L[Timer 1] {-{-}{} K}
    M[Serial Port] {-{-}{} N[Serial Control]}
    O[Port 0] {-{-}{} P[I/O Control]}
    Q[Port 1] {-{-}{} P}
    R[Port 2] {-{-}{} P}
    S[Port 3] {-{-}{} P}
    T[Interrupt System] {-{-}{} U[Interrupt Control]}
{Highlighting}
{Shaded}
\end{verbatim}
\end{center}

\textbf{Major Blocks:}

\begin{itemize}
\tightlist
\item
  \textbf{CPU}: 8-bit processor with ALU
\item
  \textbf{Memory}: 4KB ROM, 128B RAM
\item
  \textbf{Timers}: Two 16-bit timers
\item
  \textbf{Serial Port}: Full duplex UART
\item
  \textbf{I/O Ports}: Four 8-bit ports
\item
  \textbf{Interrupts}: 5 interrupt sources
\end{itemize}

\end{solutionbox}
\begin{mnemonicbox}
``CPU Memory Timer Serial IO Interrupt'' (CMTSII)

\end{mnemonicbox}
\subsection*{Question 3(a OR) [3
marks]}\label{question-3a-or-3-marks}

\textbf{List common features of 8051 microcontroller.}

\begin{solutionbox}

\textbf{Common Features:}

\begin{itemize}
\tightlist
\item
  \textbf{CPU}: 8-bit microcontroller
\item
  \textbf{Memory}: 4KB ROM, 128B RAM
\item
  \textbf{I/O Ports}: 32 I/O lines (4 ports)
\item
  \textbf{Timers}: Two 16-bit timers/counters
\item
  \textbf{Serial Port}: Full duplex UART
\item
  \textbf{Interrupts}: 5 interrupt sources
\item
  \textbf{Clock}: 12MHz maximum frequency
\end{itemize}

\end{solutionbox}
\begin{mnemonicbox}
``CPU Memory IO Timer Serial Interrupt Clock''
(CMITSIC)

\end{mnemonicbox}
\subsection*{Question 3(b OR) [4
marks]}\label{question-3b-or-4-marks}

\textbf{Draw IP SFR of 8051 and Explain function of each bit.}

\begin{solutionbox}

\textbf{IP Register (B8H):}

\begin{verbatim}
 D7  D6  D5  D4  D3  D2  D1  D0
+{-{-}{-}+{-}{-}{-}+{-}{-}{-}+{-}{-}{-}+{-}{-}{-}+{-}{-}{-}+{-}{-}{-}+{-}{-}{-}+}
| {- | {-} | {-} |PS |PT1|PX1|PT0|PX0|}
+{-{-}{-}+{-}{-}{-}+{-}{-}{-}+{-}{-}{-}+{-}{-}{-}+{-}{-}{-}+{-}{-}{-}+{-}{-}{-}+}
\end{verbatim}

\textbf{Bit Functions:}

\begin{itemize}
\tightlist
\item
  \textbf{PS}: Serial port interrupt priority
\item
  \textbf{PT1}: Timer 1 interrupt priority
\item
  \textbf{PX1}: External interrupt 1 priority
\item
  \textbf{PT0}: Timer 0 interrupt priority
\item
  \textbf{PX0}: External interrupt 0 priority
\end{itemize}

\textbf{Priority Levels:}

\begin{itemize}
\tightlist
\item
  \textbf{1}: High priority
\item
  \textbf{0}: Low priority
\end{itemize}

\end{solutionbox}
\begin{mnemonicbox}
``Priority Serial Timer External'' (PSTE)

\end{mnemonicbox}
\subsection*{Question 3(c OR) [7
marks]}\label{question-3c-or-7-marks}

\textbf{With the help of neat diagram explain Pin diagram of 8051
microcontroller.}

\begin{solutionbox}

\begin{verbatim}
         8051 Microcontroller
         +{-{-}{-}{-}{-}{-}{-}{-}{-}{-}{-}{-}{-}{-}{-}{-}{-}{-}{-}+}
   P1.0{-{-}| 1              40 |{-}{-}VCC}
   P1.1{-{-}| 2              39 |{-}{-}P0.0/AD0}
   P1.2{-{-}| 3              38 |{-}{-}P0.1/AD1}
   P1.3{-{-}| 4              37 |{-}{-}P0.2/AD2}
   P1.4{-{-}| 5              36 |{-}{-}P0.3/AD3}
   P1.5{-{-}| 6              35 |{-}{-}P0.4/AD4}
   P1.6{-{-}| 7              34 |{-}{-}P0.5/AD5}
   P1.7{-{-}| 8              33 |{-}{-}P0.6/AD6}
   RST {-{-}| 9              32 |{-}{-}P0.7/AD7}
 P3.0/RXD| 10             31 |{-{-}EA/VPP}
 P3.1/TXD| 11             30 |{-{-}ALE/PROG}
P3.2/INT0| 12             29 |{-{-}PSEN}
P3.3/INT1| 13             28 |{-{-}P2.7/A15}
 P3.4/T0{-| 14             27 |{-}{-}P2.6/A14}
 P3.5/T1{-| 15             26 |{-}{-}P2.5/A13}
 P3.6/WR{-| 16             25 |{-}{-}P2.4/A12}
 P3.7/RD{-| 17             24 |{-}{-}P2.3/A11}
 XTAL2 {-{-}| 18             23 |{-}{-}P2.2/A10}
 XTAL1 {-{-}| 19             22 |{-}{-}P2.1/A9}
   VSS {-{-}| 20             21 |{-}{-}P2.0/A8}
         +{-{-}{-}{-}{-}{-}{-}{-}{-}{-}{-}{-}{-}{-}{-}{-}{-}{-}{-}+}
\end{verbatim}

\textbf{Pin Groups:}

\begin{itemize}
\tightlist
\item
  \textbf{Power}: VCC (40), VSS (20)
\item
  \textbf{Clock}: XTAL1 (19), XTAL2 (18)
\item
  \textbf{Reset}: RST (9)
\item
  \textbf{Ports}: P0, P1, P2, P3
\item
  \textbf{Control}: ALE, PSEN, EA
\end{itemize}

\end{solutionbox}
\begin{mnemonicbox}
``Power Clock Reset Ports Control'' (PCRPC)

\end{mnemonicbox}
\subsection*{Question 4(a) [3 marks]}\label{q4a}

\textbf{Explain arithmetic instructions with example.}

\begin{solutionbox}

\textbf{Arithmetic Instructions:}

{\def\LTcaptype{none} % do not increment counter
\begin{longtable}[]{@{}lll@{}}
\toprule\noalign{}
Instruction & Function & Example \\
\midrule\noalign{}
\endhead
\bottomrule\noalign{}
\endlastfoot
\textbf{ADD} & Addition & ADD A,\#10H \\
\textbf{SUBB} & Subtraction & SUBB A,R0 \\
\textbf{MUL} & Multiplication & MUL AB \\
\textbf{DIV} & Division & DIV AB \\
\textbf{INC} & Increment & INC A \\
\textbf{DEC} & Decrement & DEC R1 \\
\end{longtable}
}

\begin{itemize}
\tightlist
\item
  \textbf{ADD A,\#10H}: Add 10H to accumulator
\item
  \textbf{Flags}: Affected by arithmetic operations
\end{itemize}

\end{solutionbox}
\begin{mnemonicbox}
``Add Subtract Multiply Divide Increment Decrement''
(ASMIDI)

\end{mnemonicbox}
\subsection*{Question 4(b) [4 marks]}\label{q4b}

\textbf{Write an 8051 Assembly Language Program to Find 2's complement
of a value stored at memory location 65H. Put the result on same
location.}

\begin{solutionbox}

\begin{verbatim}
ORG 0000H           ; Program start address
MOV A,65H           ; Load value from location 65H
CPL A               ; Complement the value (1{s complement)}
ADD A,\#01H          ; Add 1 to get 2{s complement}
MOV 65H,A           ; Store result back to 65H
SJMP $              ; Stop program
END
\end{verbatim}

\textbf{Program Steps:}

\begin{itemize}
\tightlist
\item
  \textbf{Load}: Get value from memory location 65H
\item
  \textbf{Complement}: Generate 1's complement using CPL
\item
  \textbf{Add 1}: Convert to 2's complement
\item
  \textbf{Store}: Put result back to same location
\end{itemize}

\end{solutionbox}
\begin{mnemonicbox}
``Load Complement Add Store'' (LCAS)

\end{mnemonicbox}
\subsection*{Question 4(c) [7 marks]}\label{q4c}

\textbf{List Addressing Modes of 8051 Microcontroller and explain them
with example.}

\begin{solutionbox}

\textbf{Addressing Modes Table:}

{\def\LTcaptype{none} % do not increment counter
\begin{longtable}[]{@{}llll@{}}
\toprule\noalign{}
Mode & Description & Example & Usage \\
\midrule\noalign{}
\endhead
\bottomrule\noalign{}
\endlastfoot
\textbf{Immediate} & Data in instruction & MOV A,\#25H & Constant
data \\
\textbf{Register} & Data in register & MOV A,R0 & Fast access \\
\textbf{Direct} & Memory address & MOV A,30H & RAM access \\
\textbf{Indirect} & Address in register & MOV A,@R0 & Pointer access \\
\textbf{Indexed} & Base + offset & MOVC A,@A+DPTR & Table access \\
\textbf{Relative} & PC + offset & SJMP LOOP & Branch instructions \\
\textbf{Bit} & Bit address & SETB P1.0 & Bit operations \\
\end{longtable}
}

\textbf{Examples:}

\begin{itemize}
\tightlist
\item
  \textbf{MOV A,\#25H}: Load immediate value 25H
\item
  \textbf{MOV A,@R0}: Load from address in R0
\item
  \textbf{SJMP LOOP}: Jump relative to current PC
\end{itemize}

\end{solutionbox}
\begin{mnemonicbox}
``Immediate Register Direct Indirect Indexed Relative
Bit'' (IRDIIRB)

\end{mnemonicbox}
\subsection*{Question 4(a OR) [3
marks]}\label{question-4a-or-3-marks}

\textbf{Explain logical instruction with example.}

\begin{solutionbox}

\textbf{Logical Instructions:}

{\def\LTcaptype{none} % do not increment counter
\begin{longtable}[]{@{}lll@{}}
\toprule\noalign{}
Instruction & Function & Example \\
\midrule\noalign{}
\endhead
\bottomrule\noalign{}
\endlastfoot
\textbf{ANL} & AND operation & ANL A,\#0FH \\
\textbf{ORL} & OR operation & ORL A,R1 \\
\textbf{XRL} & XOR operation & XRL A,\#55H \\
\textbf{CPL} & Complement & CPL A \\
\textbf{RL} & Rotate left & RL A \\
\textbf{RR} & Rotate right & RR A \\
\end{longtable}
}

\begin{itemize}
\tightlist
\item
  \textbf{ANL A,\#0FH}: AND accumulator with 0FH (mask operation)
\item
  \textbf{Applications}: Bit masking, data manipulation
\end{itemize}

\end{solutionbox}
\begin{mnemonicbox}
``AND OR XOR Complement Rotate'' (AOXCR)

\end{mnemonicbox}
\subsection*{Question 4(b OR) [4
marks]}\label{question-4b-or-4-marks}

\textbf{Write an 8051 Assembly Language Program to Multiply the number
in register R3 by the number in register R0 and put the result in
internal RAM location 10h(MSB) and 11h(LSB).}

\begin{solutionbox}

\begin{verbatim}
ORG 0000H           ; Program start address
MOV A,R3            ; Move R3 to accumulator
MOV B,R0            ; Move R0 to B register
MUL AB              ; Multiply A and B
MOV 10H,B           ; Store MSB (B) to location 10H
MOV 11H,A           ; Store LSB (A) to location 11H  
SJMP $              ; Stop program
END
\end{verbatim}

\textbf{Program Flow:}

\begin{itemize}
\tightlist
\item
  \textbf{Load}: Move multiplicand and multiplier to A and B
\item
  \textbf{Multiply}: Use MUL AB instruction
\item
  \textbf{Store}: MSB in B register, LSB in A register
\item
  \textbf{Result}: 16-bit result stored in two locations
\end{itemize}

\end{solutionbox}
\begin{mnemonicbox}
``Load Multiply Store Result'' (LMSR)

\end{mnemonicbox}
\subsection*{Question 4(c OR) [7
marks]}\label{question-4c-or-7-marks}

\textbf{Explain data transfer instruction with example.}

\begin{solutionbox}

\textbf{Data Transfer Instructions:}

{\def\LTcaptype{none} % do not increment counter
\begin{longtable}[]{@{}llll@{}}
\toprule\noalign{}
Category & Instruction & Example & Function \\
\midrule\noalign{}
\endhead
\bottomrule\noalign{}
\endlastfoot
\textbf{Register} & MOV & MOV A,R0 & Register to register \\
\textbf{Immediate} & MOV & MOV A,\#25H & Immediate to register \\
\textbf{Direct} & MOV & MOV A,30H & Memory to register \\
\textbf{Indirect} & MOV & MOV A,@R0 & Indirect addressing \\
\textbf{External} & MOVX & MOVX A,@DPTR & External memory \\
\textbf{Code} & MOVC & MOVC A,@A+DPTR & Code memory \\
\textbf{Stack} & PUSH/POP & PUSH ACC & Stack operations \\
\end{longtable}
}

\textbf{Examples:}

\begin{itemize}
\tightlist
\item
  \textbf{MOV A,R0}: Move R0 content to accumulator
\item
  \textbf{MOVX A,@DPTR}: Read from external data memory
\item
  \textbf{PUSH ACC}: Push accumulator to stack
\end{itemize}

\textbf{Data Movement:}

\begin{itemize}
\tightlist
\item
  \textbf{Internal}: Within 8051 memory space
\item
  \textbf{External}: To/from external memory
\item
  \textbf{Code}: From program memory
\end{itemize}

\end{solutionbox}
\begin{mnemonicbox}
``Move Data Between Locations'' (MDBL)

\end{mnemonicbox}
\subsection*{Question 5(a) [3 marks]}\label{q5a}

\textbf{Explain the 8051 flags with the help of PSW format.}

\begin{solutionbox}

\textbf{PSW Register (D0H):}

\begin{verbatim}
 D7  D6  D5  D4  D3  D2  D1  D0
+{-{-}{-}+{-}{-}{-}+{-}{-}{-}+{-}{-}{-}+{-}{-}{-}+{-}{-}{-}+{-}{-}{-}+{-}{-}{-}+}
| C |AC | F0|RS1|RS0| OV| {- | P |}
+{-{-}{-}+{-}{-}{-}+{-}{-}{-}+{-}{-}{-}+{-}{-}{-}+{-}{-}{-}+{-}{-}{-}+{-}{-}{-}+}
\end{verbatim}

\textbf{Flag Functions:}

\begin{itemize}
\tightlist
\item
  \textbf{C (Carry)}: Set when carry/borrow occurs
\item
  \textbf{AC (Auxiliary Carry)}: For BCD arithmetic
\item
  \textbf{OV (Overflow)}: Set when signed overflow
\item
  \textbf{P (Parity)}: Even parity of accumulator
\item
  \textbf{RS1, RS0}: Register bank select bits
\end{itemize}

\end{solutionbox}
\begin{mnemonicbox}
``Carry Auxiliary Overflow Parity Register'' (CAOPR)

\end{mnemonicbox}
\subsection*{Question 5(b) [4 marks]}\label{q5b}

\textbf{Draw and explain diagram Interfacing 7 segment with
microcontroller.}

\begin{solutionbox}

\textbf{7-Segment Interface Circuit:}

\begin{verbatim}
    8051           ULN2003        7{-Segment Display}
    P1.0 {-{-}{-}{-}{-}{-}{-}{-}{-} I1 {-}{-}{-}{-}{-} O1 {-}{-}{-}{-}{-} a}
    P1.1 {-{-}{-}{-}{-}{-}{-}{-}{-} I2 {-}{-}{-}{-}{-} O2 {-}{-}{-}{-}{-} b  }
    P1.2 {-{-}{-}{-}{-}{-}{-}{-}{-} I3 {-}{-}{-}{-}{-} O3 {-}{-}{-}{-}{-} c}
    P1.3 {-{-}{-}{-}{-}{-}{-}{-}{-} I4 {-}{-}{-}{-}{-} O4 {-}{-}{-}{-}{-} d}
    P1.4 {-{-}{-}{-}{-}{-}{-}{-}{-} I5 {-}{-}{-}{-}{-} O5 {-}{-}{-}{-}{-} e}
    P1.5 {-{-}{-}{-}{-}{-}{-}{-}{-} I6 {-}{-}{-}{-}{-} O6 {-}{-}{-}{-}{-} f}
    P1.6 {-{-}{-}{-}{-}{-}{-}{-}{-} I7 {-}{-}{-}{-}{-} O7 {-}{-}{-}{-}{-} g}
    P1.7 {-{-}{-}{-}{-}{-}{-}{-}{-} I8 {-}{-}{-}{-}{-} O8 {-}{-}{-}{-}{-} DP}
                                        |
                                    Common Cathode
                                        |
                                       GND
\end{verbatim}

\textbf{Components:}

\begin{itemize}
\tightlist
\item
  \textbf{ULN2003}: Current driver IC
\item
  \textbf{Resistors}: Current limiting (330Ω)
\item
  \textbf{Display}: Common cathode type
\end{itemize}

\textbf{Working}: Port data drives display segments through current
driver

\end{solutionbox}
\begin{mnemonicbox}
``Port Driver Display Ground'' (PDDG)

\end{mnemonicbox}
\subsection*{Question 5(c) [7 marks]}\label{q5c}

\textbf{Interface 8 LEDs with microcontroller and write a program to
turn on and off.}

\begin{solutionbox}

\textbf{LED Interface Circuit:}

\begin{verbatim}
    8051           Current Limiting      LEDs
    P1.0 {-{-}{-}{-}{-}{-}{-}{-}{-} 330Ω {-}{-}{-}{-}{-}{-}{-}{-}{-} LED0 {-}{-}{-}{-}{-} +5V}
    P1.1 {-{-}{-}{-}{-}{-}{-}{-}{-} 330Ω {-}{-}{-}{-}{-}{-}{-}{-}{-} LED1 {-}{-}{-}{-}{-} +5V}
    P1.2 {-{-}{-}{-}{-}{-}{-}{-}{-} 330Ω {-}{-}{-}{-}{-}{-}{-}{-}{-} LED2 {-}{-}{-}{-}{-} +5V}
    P1.3 {-{-}{-}{-}{-}{-}{-}{-}{-} 330Ω {-}{-}{-}{-}{-}{-}{-}{-}{-} LED3 {-}{-}{-}{-}{-} +5V}
    P1.4 {-{-}{-}{-}{-}{-}{-}{-}{-} 330Ω {-}{-}{-}{-}{-}{-}{-}{-}{-} LED4 {-}{-}{-}{-}{-} +5V}
    P1.5 {-{-}{-}{-}{-}{-}{-}{-}{-} 330Ω {-}{-}{-}{-}{-}{-}{-}{-}{-} LED5 {-}{-}{-}{-}{-} +5V}
    P1.6 {-{-}{-}{-}{-}{-}{-}{-}{-} 330Ω {-}{-}{-}{-}{-}{-}{-}{-}{-} LED6 {-}{-}{-}{-}{-} +5V}
    P1.7 {-{-}{-}{-}{-}{-}{-}{-}{-} 330Ω {-}{-}{-}{-}{-}{-}{-}{-}{-} LED7 {-}{-}{-}{-}{-} +5V}
\end{verbatim}

\textbf{Assembly Program:}

\begin{verbatim}
ORG 0000H           ; Start address
MAIN:
    MOV P1,\#0FFH    ; Turn on all LEDs (logic 0)
    CALL DELAY      ; Call delay subroutine
    MOV P1,\#00H     ; Turn off all LEDs (logic 1)
    CALL DELAY      ; Call delay subroutine
    SJMP MAIN       ; Repeat continuously

DELAY:
    MOV R2,\#250     ; Outer loop counter
D1: MOV R3,\#250     ; Inner loop counter  
D2: DJNZ R3,D2      ; Decrement R3 until zero
    DJNZ R2,D1      ; Decrement R2 until zero
    RET             ; Return from subroutine
END
\end{verbatim}

\end{solutionbox}
\begin{mnemonicbox}
``Light Emitting Display Interface'' (LEDI)

\end{mnemonicbox}
\subsection*{Question 5(a OR) [3
marks]}\label{question-5a-or-3-marks}

\textbf{List Applications of microcontroller in various fields.}

\begin{solutionbox}

\textbf{Applications by Field:}

{\def\LTcaptype{none} % do not increment counter
\begin{longtable}[]{@{}ll@{}}
\toprule\noalign{}
Field & Applications \\
\midrule\noalign{}
\endhead
\bottomrule\noalign{}
\endlastfoot
\textbf{Home} & Washing machine, Microwave, AC \\
\textbf{Automotive} & Engine control, ABS, Airbag \\
\textbf{Industrial} & Process control, Robotics \\
\textbf{Medical} & Pacemaker, Blood pressure monitor \\
\textbf{Communication} & Mobile phones, Modems \\
\textbf{Security} & Access control, Burglar alarm \\
\textbf{Entertainment} & Gaming consoles, Remote control \\
\end{longtable}
}

\end{solutionbox}
\begin{mnemonicbox}
``Home Auto Industrial Medical Communication Security
Entertainment'' (HAIMCSE)

\end{mnemonicbox}
\subsection*{Question 5(b OR) [4
marks]}\label{question-5b-or-4-marks}

\textbf{Draw and explain diagram interfacing of DC motor with 8051.}

\begin{solutionbox}

\textbf{DC Motor Interface:}

\begin{verbatim}
    8051       L293D Motor Driver         DC Motor
    P1.0 {-{-}{-}{-}{-}{-}{-} Enable Pin                 |}
    P1.1 {-{-}{-}{-}{-}{-}{-} Input 1  {-}{-}{-}{-}{-} Output 1 {-}{-}+}
    P1.2 {-{-}{-}{-}{-}{-}{-} Input 2  {-}{-}{-}{-}{-} Output 2 {-}{-}+}
                    |              |
                   VCC            GND
                    |              |
                  +12V           Motor
\end{verbatim}

\textbf{Components:}

\begin{itemize}
\tightlist
\item
  \textbf{L293D}: Dual H-bridge driver IC
\item
  \textbf{Motor}: 12V DC motor
\item
  \textbf{Control}: Direction and speed control
\end{itemize}

\textbf{Control Logic:}

\begin{itemize}
\tightlist
\item
  \textbf{Forward}: P1.1=1, P1.2=0
\item
  \textbf{Reverse}: P1.1=0, P1.2=1
\item
  \textbf{Stop}: P1.1=0, P1.2=0
\end{itemize}

\end{solutionbox}
\begin{mnemonicbox}
``Driver Control Motor Direction'' (DCMD)

\end{mnemonicbox}
\subsection*{Question 5(c OR) [7
marks]}\label{question-5c-or-7-marks}

\textbf{Interface LCD with microcontroller and write a program to
display ``Microprocessor and Microcontroller''.}

\begin{solutionbox}

\textbf{LCD Interface:}

\begin{verbatim}
    8051                16x2 LCD
    P2.0 {-{-}{-}{-}{-}{-}{-}{-}{-}{-}{-}{-}{-}{-} RS (Register Select)}
    P2.1 {-{-}{-}{-}{-}{-}{-}{-}{-}{-}{-}{-}{-}{-} EN (Enable)  }
    P1.0{-P1.7 {-}{-}{-}{-}{-}{-}{-}{-}{-} D0{-}D7 (Data lines)}
    GND {-{-}{-}{-}{-}{-}{-}{-}{-}{-}{-}{-}{-}{-}{-} VSS, RW}
    +5V {-{-}{-}{-}{-}{-}{-}{-}{-}{-}{-}{-}{-}{-}{-} VDD, VEE (via 10kΩ pot)}
\end{verbatim}

\textbf{Assembly Program:}

\begin{verbatim}
ORG 0000H
    CALL LCD\_INIT       ; Initialize LCD
    MOV DPTR,\#MSG1      ; Point to message
    CALL DISPLAY\_MSG    ; Display message
    SJMP $              ; Stop

LCD\_INIT:
    MOV P1,\#38H         ; Function set: 8{-bit, 2{-}line}
    CLR P2.0            ; RS=0 (command)
    SETB P2.1           ; EN=1
    CLR P2.1            ; EN=0 (pulse)
    CALL DELAY
    MOV P1,\#01H         ; Clear display
    CLR P2.0
    SETB P2.1
    CLR P2.1
    CALL DELAY
    RET

DISPLAY\_MSG:
    MOV P1,A            ; Send character
    SETB P2.0           ; RS=1 (data)
    SETB P2.1           ; EN=1
    CLR P2.1            ; EN=0
    CALL DELAY
    RET

MSG1: DB "Microprocessor and Microcontroller",0

DELAY:
    MOV R1,\#50
D1: MOV R2,\#255
D2: DJNZ R2,D2
    DJNZ R1,D1
    RET
END
\end{verbatim}

\end{solutionbox}
\begin{mnemonicbox}
``Liquid Crystal Display Interface'' (LCDI)

\end{mnemonicbox}

\end{document}
