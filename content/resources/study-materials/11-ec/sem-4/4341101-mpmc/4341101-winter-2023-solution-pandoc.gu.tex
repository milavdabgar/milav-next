\documentclass[10pt,a4paper]{article}

% content/resources/templates/preamble.tex
\usepackage[margin=0.6in]{geometry}
\author{Milav Dabgar}
\usepackage{amsmath,amssymb,amsthm}
\usepackage{booktabs}
\usepackage{multirow}
\usepackage{xcolor}
\usepackage{tcolorbox}
\tcbuselibrary{breakable,skins}
\usepackage[colorlinks=true,linkcolor=blue]{hyperref}
\usepackage{titlesec}
\usepackage{enumitem}
\usepackage{tikz}
\usepackage{pgfplots}
\usepackage{circuitikz}
\usepackage[version=4]{mhchem}
\usepackage{longtable}
\usepackage{array}
\usepackage{float}
\usepackage{caption}
\usepackage{listings}

\lstset{
  basicstyle=\small\ttfamily,
  breaklines=true,
  breakatwhitespace=false,
  postbreak=\mbox{\textcolor{red}{$\hookrightarrow$}\space},
  float=false,
  numbers=left,
  numberstyle=\tiny\color{gray},
  numbersep=10pt,
  xleftmargin=2em,
  keywordstyle=\color{blue},
  commentstyle=\color{green!60!black},
  stringstyle=\color{purple},
  backgroundcolor=\color{gray!5},
  showstringspaces=false,
  tabsize=2,
  captionpos=b,
  keepspaces=true,
  columns=flexible
}

\pgfplotsset{compat=1.18}
\usetikzlibrary{shapes,arrows,positioning,calc,patterns,decorations.pathmorphing,decorations.markings,arrows.meta}

% Color scheme
\definecolor{headcolor}{RGB}{0,102,204}
\definecolor{keycolor}{RGB}{220,20,60}
\definecolor{solutioncolor}{RGB}{34,139,34}
\definecolor{mnemoniccolor}{RGB}{148,0,211}
\definecolor{codecolor}{RGB}{0,0,100}

% Spacing
\setlength{\parskip}{3pt}
\setlist[itemize]{nosep}
\setlist[enumerate]{nosep}

% Title formatting
\titleformat{\section}{\Large\bfseries\color{headcolor}}{\thesection}{1em}{}
\titleformat{\subsection}{\large\bfseries\color{headcolor}}{\thesubsection}{1em}{}

% Pandoc tightlist compatibility
\providecommand{\tightlist}{%
  \setlength{\itemsep}{0pt}\setlength{\parskip}{0pt}}

% Pandoc longtable compatibility
\newcounter{none}
\def\thenone{}


% content/resources/templates/gujarati-boxes.tex
\usepackage{fontspec}
\usepackage{polyglossia}

% Set Gujarati as main language (document is primarily in Gujarati)
% Note: gloss-gujarati.ldf doesn't exist in polyglossia, but it will use hyphenation patterns
\setdefaultlanguage{gujarati}
\setotherlanguage{english}

% Configure Gujarati font properly
% Use Language=Default to prevent polyglossia from trying to add language-specific features
% that don't exist for Gujarati, which causes "empty feature" warnings
\newfontfamily\gujaratifont[Script=Gujarati,AutoFakeBold=2.5,AutoFakeSlant=0.3]{Noto Sans Gujarati}
\setmainfont[Script=Gujarati,AutoFakeBold=2.5,AutoFakeSlant=0.3]{Noto Sans Gujarati}
% Use Noto Sans Gujarati for monospace to support Gujarati in text
\setmonofont[Scale=0.9]{Noto Sans Gujarati}

% Configure English to use the same font
\newfontfamily\englishfont[Script=Gujarati,AutoFakeBold=2.5,AutoFakeSlant=0.3]{Noto Sans Gujarati}

% Translations for polyglossia
\gappto\captionsgujarati{
  \renewcommand{\tablename}{કોષ્ટક}
  \renewcommand{\figurename}{આકૃતિ}
}

% Helper for TikZ nodes to ensure Gujarati font
\newcommand{\gu}[1]{{\gujaratifont #1}}

% Custom environments
\newtcolorbox{solutionbox}{
    breakable,
    enhanced,
    colback=solutioncolor!5!white,
    colframe=solutioncolor!75!black,
    fonttitle=\bfseries,
    title=જવાબ
}

\newtcolorbox{solutionboxnobreak}{
 colback=solutioncolor!5!white,
 colframe=solutioncolor!75!black,
 fonttitle=\bfseries,
 title=જવાબ
}

\newtcolorbox{keyformula}{
 breakable,
 enhanced,
 colback=keycolor!5!white,
 colframe=keycolor!75!black,
 fonttitle=\bfseries,
 title=રાસાયણિક સમીકરણ/સૂત્ર
}

\newtcolorbox{mnemonicbox}{
 breakable,
 enhanced,
 colback=mnemoniccolor!5!white,
 colframe=mnemoniccolor!75!black,
 fonttitle=\bfseries,
 title=મેમરી ટ્રીક
}


\begin{document}

\begin{center}
{\Huge\bfseries\color{headcolor} Subject Name (Gujarati)}\\[5pt]
{\LARGE 4341101 -- Winter 2023}\\[3pt]
{\large Semester 1 Study Material}\\[3pt]
{\normalsize\textit{Detailed Solutions and Explanations}}
\end{center}

\vspace{10pt}

\subsection*{પ્રશ્ન 1(અ) [3
ગુણ]}\label{uxaaauxab0uxab6uxaa8-1uxa85-3-uxa97uxaa3}

\textbf{RISC અને CISC ની સરખામણી કરો.}

\begin{solutionbox}

{\def\LTcaptype{none} % do not increment counter
\begin{longtable}[]{@{}lll@{}}
\toprule\noalign{}
લક્ષણ & RISC & CISC \\
\midrule\noalign{}
\endhead
\bottomrule\noalign{}
\endlastfoot
સૂચનાઓ & સરળ, નિશ્ચિત લંબાઈ & જટિલ, અલગ-અલગ લંબાઈ \\
અમલીકરણ & સિંગલ સાયકલ & મલ્ટીપલ સાયકલ \\
એડ્રેસિંગ મોડ & ઓછા & ઘણા \\
રજિસ્ટર્સ & વધારે & ઓછા \\
ડિઝાઇન ફોકસ & હાર્ડવેર સરળતા & કોડ ડેન્સિટી \\
\end{longtable}
}

\textbf{યાદ રાખવા માટે:} ``RISC સરળતાથી સૂચનાઓ પૂર્ણ કરે છે''

\end{solutionbox}
\subsection*{પ્રશ્ન 1(બ) [4
ગુણ]}\label{uxaaauxab0uxab6uxaa8-1uxaac-4-uxa97uxaa3}

\textbf{વોન-ન્યુમેન અને હાર્વર્ડ આર્કિટેક્ચરની તુલના કરો.}

\begin{solutionbox}

{\def\LTcaptype{none} % do not increment counter
\begin{longtable}[]{@{}lll@{}}
\toprule\noalign{}
લક્ષણ & વોન-ન્યુમેન & હાર્વર્ડ \\
\midrule\noalign{}
\endhead
\bottomrule\noalign{}
\endlastfoot
મેમરી & એક શેર્ડ મેમરી & અલગ પ્રોગ્રામ અને ડેટા મેમરી \\
બસ & ડેટા અને સૂચનાઓ માટે એક બસ & અલગ બસ \\
સ્પીડ & ધીમી (મેમરી બોટલનેક) & ઝડપી (પેરેલલ એક્સેસ) \\
જટિલતા & સરળ ડિઝાઇન & વધુ જટિલ \\
ઉપયોગ & જનરલ કમ્પ્યુટિંગ & રીયલ-ટાઇમ સિસ્ટમ \\
\end{longtable}
}

\textbf{ડાયાગ્રામ:}

\begin{verbatim}
Von{-Neumann:}
+{-{-}{-}{-}{-}{-}{-}+         +{-}{-}{-}{-}{-}{-}{-}+}
| CPU   |{=======| Memory|}
+{-{-}{-}{-}{-}{-}{-}+         +{-}{-}{-}{-}{-}{-}{-}+}

Harvard:
+{-{-}{-}{-}{-}{-}{-}+         +{-}{-}{-}{-}{-}{-}{-}{-}{-}{-}{-}+}
| CPU   |========{| Program   |}
|       |         | Memory    |
|       |         +{-{-}{-}{-}{-}{-}{-}{-}{-}{-}{-}+}
|       |         +{-{-}{-}{-}{-}{-}{-}{-}{-}{-}{-}+}
|       |{=======| Data      |}
+{-{-}{-}{-}{-}{-}{-}+         | Memory    |}
                  +{-{-}{-}{-}{-}{-}{-}{-}{-}{-}{-}+}
\end{verbatim}

\textbf{યાદ રાખવા માટે:} ``હાર્વર્ડ પાસે અલગ જગ્યાઓ છે''

\end{solutionbox}
\subsection*{પ્રશ્ન 1(ક) [7
ગુણ]}\label{uxaaauxab0uxab6uxaa8-1uxa95-7-uxa97uxaa3}

\textbf{સમજાવો: 8085 ઈન્સ્ટ્રક્શન ફોર્મેટ, કંટ્રોલ યુનિટ, મશીન સાયકલ, ALU}

\begin{solutionbox}

\textbf{ઈન્સ્ટ્રક્શન ફોર્મેટ:}

\begin{verbatim}
+{-{-}{-}{-}{-}{-}{-}{-}+{-}{-}{-}{-}{-}{-}{-}{-}+{-}{-}{-}{-}{-}{-}{-}{-}+}
| Opcode |Operand1|Operand2|
+{-{-}{-}{-}{-}{-}{-}{-}+{-}{-}{-}{-}{-}{-}{-}{-}+{-}{-}{-}{-}{-}{-}{-}{-}+}
  1{-3 bytes total length}
\end{verbatim}

{\def\LTcaptype{none} % do not increment counter
\begin{longtable}[]{@{}ll@{}}
\toprule\noalign{}
કમ્પોનન્ટ & કાર્ય \\
\midrule\noalign{}
\endhead
\bottomrule\noalign{}
\endlastfoot
\textbf{ઈન્સ્ટ્રક્શન ફોર્મેટ} & 1-3 બાઇટ સ્ટ્રક્ચર ઓપકોડ અને ઓપરેન્ડ સાથે \\
\textbf{કંટ્રોલ યુનિટ} & સૂચનાઓ ફેચ અને ડિકોડ કરે; સિગ્નલ પેદા કરે \\
\textbf{મશીન સાયકલ} & મૂળભૂત ઓપરેશન સાયકલ (T-સ્ટેટ્સ) \\
\textbf{ALU} & ગાણિતિક અને લોજિકલ ઓપરેશન કરે \\
\end{longtable}
}

\begin{itemize}
\tightlist
\item
  \textbf{ઈન્સ્ટ્રક્શન ફોર્મેટ}: ઓપકોડ (3-8 બિટ્સ) અને 0-2 ઓપરેન્ડ્સ ધરાવે છે
\item
  \textbf{કંટ્રોલ યુનિટ}: પ્રોસેસરનું હૃદય જે બધા ઓપરેશન્સનું સંચાલન કરે છે
\item
  \textbf{મશીન સાયકલ}: ફેચ, ડિકોડ, એક્ઝિક્યુટ ફેઝ ધરાવે છે
\item
  \textbf{ALU}: ડેટા પર ADD/SUB/AND/OR/XOR ઓપરેશન કરે છે
\end{itemize}

\textbf{ડાયાગ્રામ:}

\begin{verbatim}
+{-{-}{-}{-}{-}{-}{-}{-}{-}{-}{-}{-}{-}{-}{-}{-}{-}{-}{-}{-}{-}{-}{-}{-}{-}{-}{-}{-}{-}{-}{-}{-}{-}{-}+}
|              8085                |
|  +{-{-}{-}{-}{-}{-}{-}{-}{-}{-}{-}{-}+  +{-}{-}{-}{-}{-}{-}{-}{-}{-}{-}{-}{-}+  |}
|  |Control Unit|{-|Instruction |  |}
|  |(Sequencer) |{{-}|Register    |  |}
|  +{-{-}{-}{-}{-}{-}{-}{-}{-}{-}{-}{-}+  +{-}{-}{-}{-}{-}{-}{-}{-}{-}{-}{-}{-}+  |}
|        |               |         |
|        v               v         |
|  +{-{-}{-}{-}{-}{-}{-}{-}{-}{-}{-}{-}+  +{-}{-}{-}{-}{-}{-}{-}{-}{-}{-}{-}{-}+  |}
|  |   ALU      |{{-}|Registers   |  |}
|  |            |{-|            |  |}
|  +{-{-}{-}{-}{-}{-}{-}{-}{-}{-}{-}{-}+  +{-}{-}{-}{-}{-}{-}{-}{-}{-}{-}{-}{-}+  |}
+{-{-}{-}{-}{-}{-}{-}{-}{-}{-}{-}{-}{-}{-}{-}{-}{-}{-}{-}{-}{-}{-}{-}{-}{-}{-}{-}{-}{-}{-}{-}{-}{-}{-}+}
\end{verbatim}

\textbf{યાદ રાખવા માટે:} ``CIMA: કંટ્રોલ સમજે, મશીન ક્રિયા કરે''

\end{solutionbox}
\subsection*{પ્રશ્ન 1(ક OR) [7
ગુણ]}\label{uxaaauxab0uxab6uxaa8-1uxa95-or-7-uxa97uxaa3}

\textbf{માઇક્રોપ્રોસેસર અને માઇક્રોકંટ્રોલરની સરખામણી કરો.}

\begin{solutionbox}

{\def\LTcaptype{none} % do not increment counter
\begin{longtable}[]{@{}lll@{}}
\toprule\noalign{}
લક્ષણ & માઇક્રોપ્રોસેસર & માઇક્રોકંટ્રોલર \\
\midrule\noalign{}
\endhead
\bottomrule\noalign{}
\endlastfoot
ડિઝાઇન & માત્ર CPU & CPU + પેરિફેરલ્સ \\
મેમરી & બાહ્ય & આંતરિક (RAM/ROM) \\
I/O પોર્ટ્સ & મર્યાદિત & બિલ્ટ-ઇન ઘણા \\
કિંમત & વધારે & ઓછી \\
ઉપયોગ & જનરલ કમ્પ્યુટિંગ & એમ્બેડેડ સિસ્ટમ \\
પાવર ખપત & વધારે & ઓછો \\
ઉદાહરણ & Intel 8085/8086 & Intel 8051 \\
\end{longtable}
}

\textbf{ડાયાગ્રામ:}

\begin{verbatim}
Microprocessor System:
+{-{-}{-}{-}{-}{-}{-}+    +{-}{-}{-}{-}{-}{-}{-}+    +{-}{-}{-}{-}{-}{-}{-}+}
| CPU   |{{-}{-}| Memory|{-}{-}| I/O   |}
+{-{-}{-}{-}{-}{-}{-}+    +{-}{-}{-}{-}{-}{-}{-}+    +{-}{-}{-}{-}{-}{-}{-}+}
    Separate chips needed

Microcontroller:
+{-{-}{-}{-}{-}{-}{-}{-}{-}{-}{-}{-}{-}{-}{-}{-}{-}{-}{-}{-}{-}{-}{-}{-}+}
| +{-{-}{-}{-}{-}{-}{-}+  +{-}{-}{-}{-}{-}{-}{-}+   |}
| | CPU   |  | Memory|   |
| +{-{-}{-}{-}{-}{-}{-}+  +{-}{-}{-}{-}{-}{-}{-}+   |}
|       |       |        |
|       v       v        |
| +{-{-}{-}{-}{-}{-}{-}{-}{-}{-}{-}{-}{-}{-}{-}{-}{-}{-}{-}+  |}
| | I/O, Timers, etc. |  |
| +{-{-}{-}{-}{-}{-}{-}{-}{-}{-}{-}{-}{-}{-}{-}{-}{-}{-}{-}+  |}
+{-{-}{-}{-}{-}{-}{-}{-}{-}{-}{-}{-}{-}{-}{-}{-}{-}{-}{-}{-}{-}{-}{-}{-}+}
    All in one chip
\end{verbatim}

\textbf{યાદ રાખવા માટે:} ``માઇક્રો-P પ્રોસેસ કરે, માઇક્રો-C કંટ્રોલ કરે''

\end{solutionbox}
\subsection*{પ્રશ્ન 2(અ) [3
ગુણ]}\label{uxaaauxab0uxab6uxaa8-2uxa85-3-uxa97uxaa3}

\textbf{માઇક્રોપ્રોસેસરમાં ઇન્સ્ટ્રક્શન ફેચિંગ, ડીકોડિંગ અને એક્ઝેક્યુશન ઓપરેશન સમજાવો.}

\begin{solutionbox}

{\def\LTcaptype{none} % do not increment counter
\begin{longtable}[]{@{}ll@{}}
\toprule\noalign{}
ફેઝ & ઓપરેશન \\
\midrule\noalign{}
\endhead
\bottomrule\noalign{}
\endlastfoot
ફેચિંગ & CPU PC નો ઉપયોગ કરી મેમરીમાંથી સૂચના મેળવે \\
ડીકોડિંગ & ઓપરેશન પ્રકાર અને ઓપરેન્ડ નક્કી કરે \\
એક્ઝેક્યુશન & ખરેખર ઓપરેશન કરે \\
\end{longtable}
}

\textbf{ડાયાગ્રામ:}

\begin{verbatim}
+{-{-}{-}{-}{-}{-}{-}{-}+    +{-}{-}{-}{-}{-}{-}{-}{-}+    +{-}{-}{-}{-}{-}{-}{-}{-}+}
| Fetch  |{-{-}{-}| Decode |{-}{-}{-}|Execute |}
+{-{-}{-}{-}{-}{-}{-}{-}+    +{-}{-}{-}{-}{-}{-}{-}{-}+    +{-}{-}{-}{-}{-}{-}{-}{-}+}
\end{verbatim}

\textbf{યાદ રાખવા માટે:} ``FDE: પહેલા લે, પછી સમજે, અંતે કરે''

\end{solutionbox}
\subsection*{પ્રશ્ન 2(બ) [4
ગુણ]}\label{uxaaauxab0uxab6uxaa8-2uxaac-4-uxa97uxaa3}

\textbf{8085 માઇક્રોપ્રોસેસરનું બસ ઓર્ગેનાઇઝેશન સમજાવો.}

\begin{solutionbox}

{\def\LTcaptype{none} % do not increment counter
\begin{longtable}[]{@{}lll@{}}
\toprule\noalign{}
બસ પ્રકાર & પહોળાઈ & કાર્ય \\
\midrule\noalign{}
\endhead
\bottomrule\noalign{}
\endlastfoot
એડ્રેસ બસ & 16-બિટ & મેમરી એડ્રેસ ટ્રાન્સફર કરે (A0-A15) \\
ડેટા બસ & 8-બિટ & ડેટા ટ્રાન્સફર કરે (D0-D7) \\
કંટ્રોલ બસ & વિવિધ લાઇન્સ & ડેટા ફ્લો મેનેજ કરે (RD, WR, IO/M) \\
મલ્ટિપ્લેક્સ્ડ & AD0-AD7 & લોઅર એડ્રેસ બિટ્સ + ડેટા બિટ્સ \\
\end{longtable}
}

\textbf{ડાયાગ્રામ:}

\begin{verbatim}
8085 Microprocessor
    |
    |{-{-}{-}{-} Address Bus (16{-}bit) {-}{-}{-}{-} Memory}
    |                                 Location
    |{-{-}{-}{-} Data Bus (8{-}bit) {-}{-}{-}{-}{-}{-}{-}{-} Data}
    |
    |{-{-}{-}{-} Control Bus {-}{-}{-}{-}{-}{-}{-}{-}{-}{-}{-}{-}{-} Control}
                                     Signals
\end{verbatim}

\textbf{યાદ રાખવા માટે:} ``ADC: એડ્રેસ બતાવે, ડેટા વહે, કંટ્રોલ દિશા આપે''

\end{solutionbox}
\subsection*{પ્રશ્ન 2(ક) [7
ગુણ]}\label{uxaaauxab0uxab6uxaa8-2uxa95-7-uxa97uxaa3}

\textbf{આકૃતિની મદદથી 8085 માઇક્રોપ્રોસેસરના આર્કિટેક્ચરનું વર્ણન કરો.}

\begin{solutionbox}

{\def\LTcaptype{none} % do not increment counter
\begin{longtable}[]{@{}ll@{}}
\toprule\noalign{}
કમ્પોનન્ટ & કાર્ય \\
\midrule\noalign{}
\endhead
\bottomrule\noalign{}
\endlastfoot
ALU & ગાણિતિક અને લોજિકલ ઓપરેશન્સ \\
રજિસ્ટર એરે & અસ્થાયી ડેટા સ્ટોરેજ (B,C,D,E,H,L) \\
એક્યુમુલેટર & ગાણિતિક માટે મુખ્ય રજિસ્ટર \\
કંટ્રોલ યુનિટ & સૂચના કંટ્રોલ અને ટાઇમિંગ \\
ઈન્સ્ટ્રક્શન રજિસ્ટર & વર્તમાન સૂચના ધરાવે \\
ટાઇમિંગ \& કંટ્રોલ & ટાઇમિંગ સિગ્નલ્સ જનરેટ કરે \\
એડ્રેસ બફર & એડ્રેસ બસ મેનેજ કરે \\
ડેટા બફર & ડેટા બસ ટ્રાન્સફર મેનેજ કરે \\
\end{longtable}
}

\textbf{ડાયાગ્રામ:}

\begin{verbatim}
+{-{-}{-}{-}{-}{-}{-}{-}{-}{-}{-}{-}{-}{-}{-}{-}{-}{-}{-}{-}{-}{-}{-}{-}{-}{-}{-}{-}{-}{-}{-}{-}{-}{-}{-}{-}{-}{-}{-}{-}{-}{-}{-}{-}{-}{-}{-}{-}{-}{-}{-}{-}{-}{-}+}
|                  8085 MICROPROCESSOR                 |
| +{-{-}{-}{-}{-}{-}{-}{-}{-}{-}{-}{-}{-}{-}{-}{-}+     +{-}{-}{-}{-}{-}{-}{-}{-}{-}{-}{-}{-}{-}{-}{-}{-}{-}{-}{-}{-}{-}{-}{-}{-}{-}+   |}
| | REGISTER ARRAY |     |                         |   |
| |B  C  D  E  H  L|{{-}{-}{-}|          ALU            |   |}
| +{-{-}{-}{-}{-}{-}{-}{-}{-}{-}{-}{-}{-}{-}{-}{-}+     |                         |   |}
| +{-{-}{-}{-}{-}{-}{-}{-}{-}{-}{-}{-}{-}{-}{-}{-}+     |                         |   |}
| | ACCUMULATOR    |{{-}{-}{-}|                         |   |}
| +{-{-}{-}{-}{-}{-}{-}{-}{-}{-}{-}{-}{-}{-}{-}{-}+     +{-}{-}{-}{-}{-}{-}{-}{-}{-}{-}{-}{-}{-}{-}{-}{-}{-}{-}{-}{-}{-}{-}{-}{-}{-}+   |}
|                                                      |
| +{-{-}{-}{-}{-}{-}{-}{-}{-}{-}{-}{-}{-}{-}{-}{-}+     +{-}{-}{-}{-}{-}{-}{-}{-}{-}{-}{-}{-}{-}{-}{-}{-}{-}{-}{-}{-}{-}{-}{-}{-}{-}+   |}
| |INSTRUCTION REG.|{{-}{-}{-}|     CONTROL UNIT        |   |}
| +{-{-}{-}{-}{-}{-}{-}{-}{-}{-}{-}{-}{-}{-}{-}{-}+     +{-}{-}{-}{-}{-}{-}{-}{-}{-}{-}{-}{-}{-}{-}{-}{-}{-}{-}{-}{-}{-}{-}{-}{-}{-}+   |}
|                              |                       |
| +{-{-}{-}{-}{-}{-}{-}{-}{-}{-}{-}{-}{-}{-}{-}{-}+     +{-}{-}{-}{-}{-}|{-}{-}{-}{-}{-}{-}{-}{-}{-}{-}{-}{-}{-}{-}{-}{-}{-}{-}{-}{-}+  |}
| |ADDRESS BUFFER  |{{-}{-}{-}|   TIMING AND CONTROL     |  |}
| +{-{-}{-}{-}{-}{-}{-}{-}{-}{-}{-}{-}{-}{-}{-}{-}+     +{-}{-}{-}{-}{-}{-}{-}{-}{-}{-}{-}{-}{-}{-}{-}{-}{-}{-}{-}{-}{-}{-}{-}{-}{-}{-}+  |}
| +{-{-}{-}{-}{-}{-}{-}{-}{-}{-}{-}{-}{-}{-}{-}{-}+                                   |}
| |  DATA BUFFER   |{{-}{-}{-}                              |}
| +{-{-}{-}{-}{-}{-}{-}{-}{-}{-}{-}{-}{-}{-}{-}{-}+                                   |}
+{-{-}{-}{-}{-}{-}{-}{-}{-}{-}{-}{-}{-}{-}{-}{-}{-}{-}{-}{-}{-}{-}{-}{-}{-}{-}{-}{-}{-}{-}{-}{-}{-}{-}{-}{-}{-}{-}{-}{-}{-}{-}{-}{-}{-}{-}{-}{-}{-}{-}{-}{-}{-}{-}+}
\end{verbatim}

\begin{itemize}
\tightlist
\item
  \textbf{ALU}: ગાણિતિક અને લોજિકલ ઓપરેશન્સ કરે છે
\item
  \textbf{કંટ્રોલ યુનિટ}: સૂચનાઓને ફેચ અને ડિકોડ કરે છે
\item
  \textbf{રજિસ્ટર્સ}: પ્રોસેસિંગ દરમિયાન ડેટા અસ્થાયી રૂપે સ્ટોર કરે છે
\item
  \textbf{બસેસ}: એડ્રેસ, ડેટા અને કંટ્રોલ સિગ્નલ્સ ટ્રાન્સફર કરે છે
\end{itemize}

\textbf{યાદ રાખવા માટે:} ``ARCBD: આર્કિટેક્ચર રજિસ્ટર કંટ્રોલ બસ ડેટા''

\end{solutionbox}
\subsection*{પ્રશ્ન 2(અ OR) [3
ગુણ]}\label{uxaaauxab0uxab6uxaa8-2uxa85-or-3-uxa97uxaa3}

\textbf{8085 માઇક્રોપ્રોસેસર માટે એડ્રેસ અને ડેટા બસોનું ડી-મલ્ટીપ્લેક્સીંગ સમજાવો.}

\begin{solutionbox}

{\def\LTcaptype{none} % do not increment counter
\begin{longtable}[]{@{}ll@{}}
\toprule\noalign{}
સ્ટેપ & ક્રિયા \\
\midrule\noalign{}
\endhead
\bottomrule\noalign{}
\endlastfoot
1 & ALE સિગ્નલ હાઈ થાય \\
2 & AD0-AD7 પર લોઅર એડ્રેસ (A0-A7) દેખાય \\
3 & લેચ ALE નો ઉપયોગ કરી એડ્રેસ પકડે \\
4 & ALE લો થાય, AD0-AD7 હવે ડેટા ટ્રાન્સફર કરે \\
\end{longtable}
}

\textbf{ડાયાગ્રામ:}

\begin{verbatim}
AD0{-AD7 {-}{-}{-}{-}+{-}{-}{-}{-}{-}{-} Latch {-}{-}{-}{-}{-} A0{-}A7}
            |         \^{}
            |         |
            v         |
          Data       ALE
\end{verbatim}

\textbf{યાદ રાખવા માટે:} ``ALAD: ALE ડેટા પહેલા એડ્રેસ લેચ કરે''

\end{solutionbox}
\subsection*{પ્રશ્ન 2(બ OR) [4
ગુણ]}\label{uxaaauxab0uxab6uxaa8-2uxaac-or-4-uxa97uxaa3}

\textbf{8085 માઇક્રોપ્રોસેસરનું ફ્લેગ રજિસ્ટર દોરો અને તેને સમજાવો.}

\begin{solutionbox}

\begin{verbatim}
Flag Register (8{-bit):}
+{-{-}{-}+{-}{-}{-}+{-}{-}{-}+{-}{-}{-}+{-}{-}{-}+{-}{-}{-}+{-}{-}{-}+{-}{-}{-}+}
| S | Z | 0 | AC| 0 | P | 1 | CY|
+{-{-}{-}+{-}{-}{-}+{-}{-}{-}+{-}{-}{-}+{-}{-}{-}+{-}{-}{-}+{-}{-}{-}+{-}{-}{-}+}
  7   6   5   4   3   2   1   0
\end{verbatim}

{\def\LTcaptype{none} % do not increment counter
\begin{longtable}[]{@{}lll@{}}
\toprule\noalign{}
ફ્લેગ & નામ & સેટ થાય ત્યારે \\
\midrule\noalign{}
\endhead
\bottomrule\noalign{}
\endlastfoot
S & સાઇન & પરિણામના બિટ 7 માં 1 હોય (નેગેટિવ) \\
Z & ઝીરો & પરિણામ શૂન્ય છે \\
AC & ઓક્ઝિલરી કેરી & બિટ 3 થી બિટ 4 માં કેરી આવે \\
P & પેરિટી & પરિણામમાં `1' ની સંખ્યા એવન (બેકી) હોય \\
CY & કેરી & બિટ 7 માંથી કેરી જનરેટ થાય \\
\end{longtable}
}

\textbf{યાદ રાખવા માટે:} ``સુઝી ACની પરફેક્ટ કેરી''

\end{solutionbox}
\subsection*{પ્રશ્ન 2(ક OR) [7
ગુણ]}\label{uxaaauxab0uxab6uxaa8-2uxa95-or-7-uxa97uxaa3}

\textbf{આકૃતિની મદદથી 8085 માઇક્રોપ્રોસેસરના પિન ડાયાગ્રામનું વર્ણન કરો.}

\begin{solutionbox}

{\def\LTcaptype{none} % do not increment counter
\begin{longtable}[]{@{}ll@{}}
\toprule\noalign{}
પિન ગ્રુપ & કાર્ય \\
\midrule\noalign{}
\endhead
\bottomrule\noalign{}
\endlastfoot
એડ્રેસ/ડેટા & મલ્ટિપ્લેક્સ્ડ AD0-AD7, A8-A15 \\
કંટ્રોલ & RD, WR, IO/M, S0, S1, ALE, CLK \\
ઇન્ટરપ્ટ & INTR, RST 5.5-7.5, TRAP \\
DMA & HOLD, HLDA \\
પાવર & Vcc, Vss \\
સીરિયલ I/O & SID, SOD \\
રીસેટ & RESET IN, RESET OUT \\
\end{longtable}
}

\textbf{ડાયાગ્રામ:}

\begin{verbatim}
            +{-{-}{-}{-}{-}{-}{-}+}
      X1 {-{-}|1    40|{-}{-} Vcc}
      X2 {-{-}|2    39|{-}{-} HOLD}
RESET OUT{-{-}|3    38|{-}{-} HLDA}
RESET IN {-{-}|4    37|{-}{-} CLK}
    IO/M {-{-}|5    36|{-}{-} RESET IN}
      S1 {-{-}|6    35|{-}{-} READY}
      RD {-{-}|7    34|{-}{-} IO/M}
      WR {-{-}|8    33|{-}{-} S1}
     ALE {-{-}|9    32|{-}{-} RD}
      S0 {-{-}|10   31|{-}{-} WR}
     A15 {-{-}|11   30|{-}{-} ALE}
     A14 {-{-}|12   29|{-}{-} S0}
     A13 {-{-}|13   28|{-}{-} A15}
     A12 {-{-}|14   27|{-}{-} A14}
     A11 {-{-}|15   26|{-}{-} A13}
     A10 {-{-}|16   25|{-}{-} A12}
      A9 {-{-}|17   24|{-}{-} A11}
      A8 {-{-}|18   23|{-}{-} A10}
     AD7 {-{-}|19   22|{-}{-} A9}
     AD6 {-{-}|20   21|{-}{-} A8}
            +{-{-}{-}{-}{-}{-}{-}+}
\end{verbatim}

\begin{itemize}
\tightlist
\item
  \textbf{એડ્રેસ/ડેટા પિન્સ}: મલ્ટિપ્લેક્સ્ડ પિન્સ ભૌતિક પિન બચાવે છે
\item
  \textbf{કંટ્રોલ પિન્સ}: મેમરી અને I/O ઓપરેશન્સ કોઓર્ડિનેટ કરે છે
\item
  \textbf{ઇન્ટરપ્ટ પિન્સ}: બાહ્ય ડિવાઇસને ઇન્ટરપ્ટ કરવા દે છે
\item
  \textbf{સીરિયલ પિન્સ}: બેઝિક સીરિયલ કમ્યુનિકેશન પૂરું પાડે છે
\end{itemize}

\textbf{યાદ રાખવા માટે:} ``ACID-PS: એડ્રેસ-કંટ્રોલ-ઇન્ટરપ્ટ-DMA-પાવર-સીરિયલ''

\end{solutionbox}
\subsection*{પ્રશ્ન 3(અ) [3
ગુણ]}\label{uxaaauxab0uxab6uxaa8-3uxa85-3-uxa97uxaa3}

\textbf{સ્ટેક, સ્ટેક પોઇન્ટર અને સ્ટેક ઓપરેશન સમજાવો.}

\begin{solutionbox}

{\def\LTcaptype{none} % do not increment counter
\begin{longtable}[]{@{}ll@{}}
\toprule\noalign{}
શબ્દ & વર્ણન \\
\midrule\noalign{}
\endhead
\bottomrule\noalign{}
\endlastfoot
સ્ટેક & LIFO મેમરી એરિયા અસ્થાયી ડેટા સ્ટોરેજ માટે \\
સ્ટેક પોઇન્ટર & 16-બિટ રજિસ્ટર જે સ્ટેક ટોપને પોઇન્ટ કરે \\
ઓપરેશન્સ & PUSH (સ્ટોર), POP (રીટ્રીવ) \\
\end{longtable}
}

\textbf{ડાયાગ્રામ:}

\begin{verbatim}
Memory:      Stack Pointer:
+{-{-}{-}{-}{-}+      +{-}{-}{-}{-}{-}+}
|     |{{-}{-}{-}{-} | SP  |}
+{-{-}{-}{-}{-}+      +{-}{-}{-}{-}{-}+}
| Data|      
+{-{-}{-}{-}{-}+      PUSH: SP{-}{-}, M[SP]=data}
| Data|      POP:  data=M[SP], SP++
+{-{-}{-}{-}{-}+}
\end{verbatim}

\textbf{યાદ રાખવા માટે:} ``SP LIFO લેનને પોઇન્ટ કરે છે''

\end{solutionbox}
\subsection*{પ્રશ્ન 3(બ) [4
ગુણ]}\label{uxaaauxab0uxab6uxaa8-3uxaac-4-uxa97uxaa3}

\textbf{8051 માઇક્રોકંટ્રોલરનો પિન ડાયાગ્રામ દોરો.}

\begin{solutionbox}

\begin{verbatim}
          8051 Microcontroller
         +{-{-}{-}{-}{-}{-}{-}{-}{-}{-}{-}{-}{-}{-}{-}{-}{-}{-}{-}+}
   P1.0{-{-}| 1              40 |{-}{-}VCC}
   P1.1{-{-}| 2              39 |{-}{-}P0.0/AD0}
   P1.2{-{-}| 3              38 |{-}{-}P0.1/AD1}
   P1.3{-{-}| 4              37 |{-}{-}P0.2/AD2}
   P1.4{-{-}| 5              36 |{-}{-}P0.3/AD3}
   P1.5{-{-}| 6              35 |{-}{-}P0.4/AD4}
   P1.6{-{-}| 7              34 |{-}{-}P0.5/AD5}
   P1.7{-{-}| 8              33 |{-}{-}P0.6/AD6}
   RST {-{-}| 9              32 |{-}{-}P0.7/AD7}
 P3.0/RXD| 10             31 |{-{-}EA/VPP}
 P3.1/TXD| 11             30 |{-{-}ALE/PROG}
P3.2/INT0| 12             29 |{-{-}PSEN}
P3.3/INT1| 13             28 |{-{-}P2.7/A15}
 P3.4/T0{-| 14             27 |{-}{-}P2.6/A14}
 P3.5/T1{-| 15             26 |{-}{-}P2.5/A13}
 P3.6/WR{-| 16             25 |{-}{-}P2.4/A12}
 P3.7/RD{-| 17             24 |{-}{-}P2.3/A11}
 XTAL2 {-{-}| 18             23 |{-}{-}P2.2/A10}
 XTAL1 {-{-}| 19             22 |{-}{-}P2.1/A9}
   VSS {-{-}| 20             21 |{-}{-}P2.0/A8}
         +{-{-}{-}{-}{-}{-}{-}{-}{-}{-}{-}{-}{-}{-}{-}{-}{-}{-}{-}+}
\end{verbatim}

{\def\LTcaptype{none} % do not increment counter
\begin{longtable}[]{@{}ll@{}}
\toprule\noalign{}
પિન ગ્રુપ & કાર્ય \\
\midrule\noalign{}
\endhead
\bottomrule\noalign{}
\endlastfoot
P0 & પોર્ટ 0, એડ્રેસ/ડેટા સાથે મલ્ટિપ્લેક્સ્ડ \\
P1 & પોર્ટ 1, જનરલ પર્પઝ I/O \\
P2 & પોર્ટ 2, અપર એડ્રેસ અને I/O \\
P3 & પોર્ટ 3, સ્પેશિયલ ફંક્શન્સ અને I/O \\
\end{longtable}
}

\textbf{યાદ રાખવા માટે:} ``PORT 0123: ડેટા-જનરલ-એડ્રેસ-સ્પેશિયલ''

\end{solutionbox}
\subsection*{પ્રશ્ન 3(ક) [7 ગુણ]
(ચાલુ)}\label{uxaaauxab0uxab6uxaa8-3uxa95-7-uxa97uxaa3-uxa9auxab2}

\textbf{8051 માઇક્રોકંટ્રોલરનો ટાઇમર્સ/કાઉન્ટર્સ લોજિક ડાયાગ્રામ દોરો અને વિવિધ
મોડમાં તેની કામગીરી સમજાવો.}

\begin{solutionbox}

\textbf{Timer/Counter Diagram:}

\begin{verbatim}
         +{-{-}{-}{-}{-}{-}{-}{-}{-}{-}{-}{-}+}
TLx {-{-}{-}{-}|  8{-}bit     |       +{-}{-}{-}{-}{-}{-}{-}{-}{-}{-}{-}{-}{-}+}
         |  Register  |{-{-}{-}{-}{-}{-}|  8{-}bit      |}
         +{-{-}{-}{-}{-}{-}{-}{-}{-}{-}{-}{-}+       |  Register   |{-}{-}{-}{-} Interrupt}
                              |  (THx)      |
         +{-{-}{-}{-}{-}{-}{-}{-}{-}{-}{-}{-}+       +{-}{-}{-}{-}{-}{-}{-}{-}{-}{-}{-}{-}{-}+}
TRx {-{-}{-}{-}| Control    |             \^{}}
         | Logic      |             |
         +{-{-}{-}{-}{-}{-}{-}{-}{-}{-}{-}{-}+             |}
                \^{                   |}
                |                   |
                v                   v
         +{-{-}{-}{-}{-}{-}{-}{-}{-}{-}{-}{-}{-}{-}{-}{-}{-}{-}{-}{-}{-}+}
C/T {-{-}{-}{-}| Mode Control Logic  |{-}{-}{-}{-}{-} GATE}
         +{-{-}{-}{-}{-}{-}{-}{-}{-}{-}{-}{-}{-}{-}{-}{-}{-}{-}{-}{-}{-}+}
                   \^{}
                   |
INTx {-{-}{-}{-}{-}{-}{-}{-}{-}{-}{-}{-}{-}{-}}
\end{verbatim}

{\def\LTcaptype{none} % do not increment counter
\begin{longtable}[]{@{}ll@{}}
\toprule\noalign{}
મોડ & ઓપરેશન \\
\midrule\noalign{}
\endhead
\bottomrule\noalign{}
\endlastfoot
મોડ 0 & 13-બિટ ટાઇમર (5-બિટ TL, 8-બિટ TH) \\
મોડ 1 & 16-બિટ ટાઇમર (8-બિટ TL, 8-બિટ TH) \\
મોડ 2 & 8-બિટ ઓટો-રિલોડ (TL કાઉન્ટ, TH રીલોડ) \\
મોડ 3 & સ્પ્લિટ ટાઇમર (માત્ર ટાઇમર 0) \\
\end{longtable}
}

\begin{itemize}
\tightlist
\item
  \textbf{ટાઇમર}: આંતરિક ક્લોક વાપરે, મશીન સાયકલ ગણે
\item
  \textbf{કાઉન્ટર}: બાહ્ય ઇનપુટ વાપરે, બાહ્ય ઘટનાઓ ગણે
\item
  \textbf{કંટ્રોલ બિટ્સ}: TMOD રજિસ્ટર મોડ સેટ કરે, TCON ઓપરેશન કંટ્રોલ કરે
\item
  \textbf{મોડ્સ}: વિવિધ ટાઇમિંગ જરૂરિયાતો માટે અલગ-અલગ કોન્ફિગરેશન
\end{itemize}

\textbf{યાદ રાખવા માટે:} ``MARC: મોડ ઓટો-રિલોડ કાઉન્ટ''

\end{solutionbox}
\subsection*{પ્રશ્ન 3(અ OR) [3
ગુણ]}\label{uxaaauxab0uxab6uxaa8-3uxa85-or-3-uxa97uxaa3}

\textbf{માઇક્રોકંટ્રોલર્સનાં કોમન ફીચર્સની સૂચિ બનાવો.}

\begin{solutionbox}

{\def\LTcaptype{none} % do not increment counter
\begin{longtable}[]{@{}ll@{}}
\toprule\noalign{}
ફીચર & હેતુ \\
\midrule\noalign{}
\endhead
\bottomrule\noalign{}
\endlastfoot
CPU કોર & સૂચનાઓ પ્રોસેસ કરવા \\
મેમરી (RAM/ROM) & પ્રોગ્રામ અને ડેટા સ્ટોર કરવા \\
I/O પોર્ટ્સ & બાહ્ય ડિવાઇસ સાથે ઇન્ટરફેસ \\
ટાઇમર/કાઉન્ટર & સમય અંતરાલ માપવા \\
ઇન્ટરપ્ટ & અસિંક્રોનસ ઘટનાઓ સંભાળવા \\
સીરિયલ કમ્યુનિકેશન & અન્ય ડિવાઇસ સાથે ડેટા ટ્રાન્સફર \\
\end{longtable}
}

\textbf{યાદ રાખવા માટે:} ``CPU-TIS: CPU-RAM-I/O-ટાઇમર-ઇન્ટરપ્ટ-સીરિયલ''

\end{solutionbox}
\subsection*{પ્રશ્ન 3(બ OR) [4
ગુણ]}\label{uxaaauxab0uxab6uxaa8-3uxaac-or-4-uxa97uxaa3}

\textbf{8051 માઇક્રોકંટ્રોલરનું ઈન્ટરનલ રેમ ઓર્ગેનાઇઝેશન સમજાવો.}

\begin{solutionbox}

{\def\LTcaptype{none} % do not increment counter
\begin{longtable}[]{@{}lll@{}}
\toprule\noalign{}
RAM એરિયા & એડ્રેસ રેન્જ & ઉપયોગ \\
\midrule\noalign{}
\endhead
\bottomrule\noalign{}
\endlastfoot
રજિસ્ટર બેન્ક્સ & 00H-1FH & R0-R7 (4 બેન્ક્સ) \\
બિટ-એડ્રેસેબલ & 20H-2FH & 128 બિટ્સ (0-7FH) \\
સ્ક્રેચ પેડ & 30H-7FH & જનરલ પર્પઝ \\
SFRs & 80H-FFH & કંટ્રોલ રજિસ્ટર્સ \\
\end{longtable}
}

\textbf{ડાયાગ્રામ:}

\begin{verbatim}
8051 Internal RAM (128 bytes):
+{-{-}{-}{-}{-}{-}{-}{-}{-}{-}{-}{-}{-}{-}{-}{-}+ 00H}
| Register Bank 0|
+{-{-}{-}{-}{-}{-}{-}{-}{-}{-}{-}{-}{-}{-}{-}{-}+ 08H}
| Register Bank 1|
+{-{-}{-}{-}{-}{-}{-}{-}{-}{-}{-}{-}{-}{-}{-}{-}+ 10H}
| Register Bank 2|
+{-{-}{-}{-}{-}{-}{-}{-}{-}{-}{-}{-}{-}{-}{-}{-}+ 18H}
| Register Bank 3|
+{-{-}{-}{-}{-}{-}{-}{-}{-}{-}{-}{-}{-}{-}{-}{-}+ 20H}
| Bit{-addressable|}
+{-{-}{-}{-}{-}{-}{-}{-}{-}{-}{-}{-}{-}{-}{-}{-}+ 30H}
|                |
| Scratch Pad    |
|                |
+{-{-}{-}{-}{-}{-}{-}{-}{-}{-}{-}{-}{-}{-}{-}{-}+ 80H}
\end{verbatim}

\textbf{યાદ રાખવા માટે:} ``RBBS: રજિસ્ટર્સ-બિટ્સ-બફર-સ્ક્રેચ''

\end{solutionbox}
\subsection*{પ્રશ્ન 3(ક OR) [7
ગુણ]}\label{uxaaauxab0uxab6uxaa8-3uxa95-or-7-uxa97uxaa3}

\textbf{આકૃતિની મદદથી 8051 માઇક્રોકંટ્રોલરનું આર્કિટેક્ચર સમજાવો.}

\begin{solutionbox}

{\def\LTcaptype{none} % do not increment counter
\begin{longtable}[]{@{}ll@{}}
\toprule\noalign{}
કમ્પોનન્ટ & કાર્ય \\
\midrule\noalign{}
\endhead
\bottomrule\noalign{}
\endlastfoot
CPU & 8-બિટ પ્રોસેસર ALU સાથે \\
મેમરી & 4K ROM, 128 બાઇટ્સ RAM \\
I/O પોર્ટ્સ & ચાર 8-બિટ પોર્ટ્સ (P0-P3) \\
ટાઇમર્સ & બે 16-બિટ ટાઇમર/કાઉન્ટર \\
સીરિયલ પોર્ટ & ફુલ-ડુપ્લેક્સ UART \\
ઇન્ટરપ્ટ & પાંચ ઇન્ટરપ્ટ સોર્સ \\
સ્પેશિયલ ફંક્શન રજિસ્ટર્સ & કંટ્રોલ રજિસ્ટર્સ \\
\end{longtable}
}

\textbf{ડાયાગ્રામ:}

\begin{verbatim}
+{-{-}{-}{-}{-}{-}{-}{-}{-}{-}{-}{-}{-}{-}{-}{-}{-}{-}{-}{-}{-}{-}{-}{-}{-}{-}{-}{-}{-}{-}{-}{-}{-}{-}{-}{-}{-}{-}{-}{-}{-}{-}{-}{-}+}
|                 8051 MCU                   |
| +{-{-}{-}{-}{-}{-}{-}{-}{-}{-}{-}{-}{-}+         +{-}{-}{-}{-}{-}{-}{-}{-}{-}{-}{-}{-}{-}{-}+   |}
| |             |         |              |   |
| |    CPU      |{{-}{-}{-}{-}{-}{-}{-}| Program ROM  |   |}
| |             |         | (4K bytes)   |   |
| +{-{-}{-}{-}{-}{-}{-}{-}{-}{-}{-}{-}{-}+         +{-}{-}{-}{-}{-}{-}{-}{-}{-}{-}{-}{-}{-}{-}+   |}
|       \^{                                    |}
|       |                 +{-{-}{-}{-}{-}{-}{-}{-}{-}{-}{-}{-}{-}{-}+   |}
|       |                 |              |   |
|       +{-{-}{-}{-}{-}{-}{-}{-}{-}{-}{-}{-}{-}{-}{-}{-}| Internal RAM |   |}
|       |                 | (128 bytes)  |   |
|       v                 +{-{-}{-}{-}{-}{-}{-}{-}{-}{-}{-}{-}{-}{-}+   |}
| +{-{-}{-}{-}{-}{-}{-}{-}{-}{-}{-}{-}{-}+         +{-}{-}{-}{-}{-}{-}{-}{-}{-}{-}{-}{-}{-}{-}+   |}
| |             |         |              |   |
| |  SFRs       |{{-}{-}{-}{-}{-}{-}{-}| I/O Ports    |   |}
| |             |         | (P0,P1,P2,P3)|   |
| +{-{-}{-}{-}{-}{-}{-}{-}{-}{-}{-}{-}{-}+         +{-}{-}{-}{-}{-}{-}{-}{-}{-}{-}{-}{-}{-}{-}+   |}
|                                            |
| +{-{-}{-}{-}{-}{-}{-}{-}{-}{-}{-}{-}{-}+         +{-}{-}{-}{-}{-}{-}{-}{-}{-}{-}{-}{-}{-}{-}+   |}
| |             |         |              |   |
| | Timers/     |         | Serial Port  |   |
| | Counters    |         | (UART)       |   |
| +{-{-}{-}{-}{-}{-}{-}{-}{-}{-}{-}{-}{-}+         +{-}{-}{-}{-}{-}{-}{-}{-}{-}{-}{-}{-}{-}{-}+   |}
+{-{-}{-}{-}{-}{-}{-}{-}{-}{-}{-}{-}{-}{-}{-}{-}{-}{-}{-}{-}{-}{-}{-}{-}{-}{-}{-}{-}{-}{-}{-}{-}{-}{-}{-}{-}{-}{-}{-}{-}{-}{-}{-}{-}+}
\end{verbatim}

\begin{itemize}
\tightlist
\item
  \textbf{હાર્વર્ડ આર્કિટેક્ચર}: અલગ પ્રોગ્રામ અને ડેટા મેમરી
\item
  \textbf{CISC ડિઝાઇન}: સમૃદ્ધ ઇન્સ્ટ્રકશન સેટ (100થી વધુ સૂચનાઓ)
\item
  \textbf{બિલ્ટ-ઇન પેરિફેરલ્સ}: બાહ્ય કમ્પોનન્ટ્સની જરૂર નથી
\item
  \textbf{સિંગલ-ચિપ સોલ્યુશન}: એક જ ચિપ પર સંપૂર્ણ સિસ્ટમ
\end{itemize}

\textbf{યાદ રાખવા માટે:} ``CAPITALS: CPU આર્કિટેક્ચર પોર્ટ્સ I/O ટાઇમર ALU
ઇન્ટરફેસ સીરિયલ''

\end{solutionbox}
\subsection*{પ્રશ્ન 4(અ) [3
ગુણ]}\label{uxaaauxab0uxab6uxaa8-4uxa85-3-uxa97uxaa3}

\textbf{બાહ્ય RAM સ્થાન 0123h થી TL0 અને બાહ્ય RAM સ્થાન 0234h થી TH0 ડેટાને
કોપી કરવા માટે 8051 એસેમ્બલી લેંગ્વેજ પ્રોગ્રામ લખો.}

\begin{solutionbox}

\begin{verbatim}
      MOV  DPTR, \#0123H   ; DPTR મા સોર્સ એડ્રેસ 0123H લોડ કરો
      MOVX A, @DPTR       ; બાહ્ય RAM માંથી ડેટા વાંચો
      MOV  TL0, A         ; ટાઇમર 0 લો બાઇટમાં કોપી કરો
      
      MOV  DPTR, \#0234H   ; DPTR મા સોર્સ એડ્રેસ 0234H લોડ કરો
      MOVX A, @DPTR       ; બાહ્ય RAM માંથી ડેટા વાંચો
      MOV  TH0, A         ; ટાઇમર 0 હાઈ બાઇટમાં કોપી કરો
\end{verbatim}

\textbf{મુખ્ય સ્ટેપ્સ:}

\begin{itemize}
\tightlist
\item
  બાહ્ય RAM એડ્રેસ માટે DPTR વાપરો
\item
  બાહ્ય મેમરી એક્સેસ માટે MOVX સૂચના
\item
  ટાઇમર રજિસ્ટર્સમાં સીધો ટ્રાન્સફર
\end{itemize}

\textbf{યાદ રાખવા માટે:} ``DRAM: DPTR વાંચો એડ્રેસ હલાવો''

\end{solutionbox}
\subsection*{પ્રશ્ન 4(બ) [4
ગુણ]}\label{uxaaauxab0uxab6uxaa8-4uxaac-4-uxa97uxaa3}

\textbf{પોર્ટ P1.3 પર ઇન્ટરફેસ કરેલ LED ને 1ms ના સમય અંતરાલ પર બ્લિંક કરવા માટે
8051 એસેમ્બલી લેંગ્વેજ પ્રોગ્રામ લખો.}

\begin{solutionbox}

\begin{verbatim}
AGAIN:  SETB P1.3         ; P1.3 પર LED ચાલુ કરો
        ACALL DELAY       ; ડિલે સબરૂટીન કોલ કરો
        CLR  P1.3         ; P1.3 પર LED બંધ કરો
        ACALL DELAY       ; ડિલે સબરૂટીન કોલ કરો
        SJMP AGAIN        ; હંમેશા રિપીટ કરો

DELAY:  MOV  R7, \#250     ; આઉટર લૂપ માટે R7 લોડ કરો
OUTER:  MOV  R6, \#1       ; ઇનર લૂપ માટે R6 લોડ કરો
INNER:  DJNZ R6, INNER    ; R6 ઝીરો થાય ત્યાં સુધી ઘટાડો
        DJNZ R7, OUTER    ; R7 ઝીરો થાય ત્યાં સુધી ઘટાડો
        RET               ; સબરૂટીનમાંથી પાછા ફરો
\end{verbatim}

\textbf{મુખ્ય સ્ટેપ્સ:}

\begin{itemize}
\tightlist
\item
  LED બ્લિંક કરવા માટે P1.3 પીન ટોગલ કરો
\item
  ટાઇમિંગ માટે નેસ્ટેડ ડિલે લૂપ
\item
  સતત બ્લિંકિંગ માટે અનંત લૂપ
\end{itemize}

\textbf{યાદ રાખવા માટે:} ``STACI: સેટ-ટાઇમર-એન્ડ-ક્લિયર-ઇન્ફિનિટલી''

\end{solutionbox}
\subsection*{પ્રશ્ન 4(ક) [7
ગુણ]}\label{uxaaauxab0uxab6uxaa8-4uxa95-7-uxa97uxaa3}

\textbf{8051 માઇક્રોકંટ્રોલરના એડ્રેસિંગ મોડ્સની યાદી બનાવો અને ઉદાહરણની મદદથી તે
બધાને સમજાવો.}

\begin{solutionbox}

{\def\LTcaptype{none} % do not increment counter
\begin{longtable}[]{@{}lll@{}}
\toprule\noalign{}
એડ્રેસિંગ મોડ & ઉદાહરણ & વર્ણન \\
\midrule\noalign{}
\endhead
\bottomrule\noalign{}
\endlastfoot
ઇમીડિયેટ & MOV A, \#25H & ડેટા સૂચનામાં છે \\
રજિસ્ટર & MOV A, R0 & ડેટા રજિસ્ટરમાં છે \\
ડાયરેક્ટ & MOV A, 30H & ડેટા RAM એડ્રેસ પર છે \\
ઇનડાયરેક્ટ & MOV A, @R0 & R0/R1 એડ્રેસ ધરાવે છે \\
ઇન્ડેક્સ્ડ & MOVC A, @A+DPTR & પ્રોગ્રામ મેમરી એક્સેસ \\
બિટ & SETB P1.3 & વ્યક્તિગત બિટ્સ એક્સેસ \\
રિલેટિવ & SJMP LABEL & 8-બિટ ઑફસેટ સાથે જમ્પ \\
\end{longtable}
}

\textbf{ઉદાહરણો:}

\begin{itemize}
\tightlist
\item
  \textbf{ઇમીડિયેટ}: \texttt{MOV\ A,\ \#55H} (A માં 55H લોડ કરો)
\item
  \textbf{રજિસ્ટર}: \texttt{ADD\ A,\ R3} (A માં R3 ઉમેરો)
\item
  \textbf{ડાયરેક્ટ}: \texttt{MOV\ 40H,\ A} (A ને એડ્રેસ 40H પર સ્ટોર કરો)
\item
  \textbf{ઇનડાયરેક્ટ}: \texttt{MOV\ @R0,\ \#5} (R0 માં રહેલા એડ્રેસ પર 5 સ્ટોર
  કરો)
\item
  \textbf{ઇન્ડેક્સ્ડ}: \texttt{MOVC\ A,\ @A+DPTR} (કોડ મેમરી વાંચો)
\item
  \textbf{બિટ}: \texttt{CLR\ C} (કેરી ફ્લેગ સાફ કરો)
\item
  \textbf{રિલેટિવ}: \texttt{JZ\ LOOP} (જો A ઝીરો હોય તો જમ્પ કરો)
\end{itemize}

\textbf{યાદ રાખવા માટે:} ``I'M DIRBI: ઇમીડિયેટ રજિસ્ટર ડાયરેક્ટ બિટ ઇન્ડેક્સ્ડ''

\end{solutionbox}
\subsection*{પ્રશ્ન 4(અ OR) [3
ગુણ]}\label{uxaaauxab0uxab6uxaa8-4uxa85-or-3-uxa97uxaa3}

\textbf{RAM સ્થાન 14h માંથી RAM સ્થાન 11h નાં ડેટાને બાદ કરવા માટે 8051 એસેમ્બલી
લેંગ્વેજ પ્રોગ્રામ લખો; RAM સ્થાન 3Ch માં પરિણામ મૂકો.}

\begin{solutionbox}

\begin{verbatim}
      MOV  A, 14H       ; RAM લોકેશન 14H નો કન્ટેન્ટ A માં લોડ કરો
      CLR  C            ; કેરી ફ્લેગ સાફ કરો
      SUBB A, 11H       ; બોરો સાથે 11H ના કન્ટેન્ટ બાદ કરો
      MOV  3CH, A       ; પરિણામને RAM લોકેશન 3CH માં સ્ટોર કરો
\end{verbatim}

\textbf{મુખ્ય સ્ટેપ્સ:}

\begin{itemize}
\tightlist
\item
  એક્યુમુલેટરમાં મિન્યુએન્ડ લોડ કરો
\item
  સાચા સબટ્રેક્શન માટે કેરી સાફ કરો
\item
  બોરો સાથે સબટ્રેક્શન માટે SUBB વાપરો
\item
  પરિણામને ડેસ્ટિનેશનમાં સ્ટોર કરો
\end{itemize}

\textbf{યાદ રાખવા માટે:} ``LCSS: લોડ-ક્લિયર-સબટ્રેક્ટ-સ્ટોર''

\end{solutionbox}
\subsection*{પ્રશ્ન 4(બ OR) [4
ગુણ]}\label{uxaaauxab0uxab6uxaa8-4uxaac-or-4-uxa97uxaa3}

\textbf{મોડ 1 માં ટાઈમર 0 નો ઉપયોગ કરીને પોર્ટ 1 ના બીટ 3 પર 50\% ડ્યુટી
સાયકલની સ્ક્વેર વેવ જનરેટ કરવા માટે 8051 એસેમ્બલી લેંગ્વેજ પ્રોગ્રામ લખો.}

\begin{solutionbox}

\begin{verbatim}
      MOV  TMOD, \#01H   ; ટાઇમર 0, મોડ 1 (16{-બિટ)}
AGAIN: MOV  TH0, \#0FCH   ; હાઈ બાઇટ લોડ કરો
      MOV  TL0, \#18H    ; લો બાઇટ લોડ કરો ({-1000 16{-}બિટમાં)}
      SETB TR0          ; ટાઇમર ચાલુ કરો
      JNB  TF0, $       ; ઓવરફ્લો માટે રાહ જુઓ
      CLR  TR0          ; ટાઇમર બંધ કરો
      CLR  TF0          ; ટાઇમર ફ્લેગ સાફ કરો
      CPL  P1.3         ; P1.3 ટોગલ કરો
      SJMP AGAIN        ; રિપીટ કરો
\end{verbatim}

\textbf{મુખ્ય સ્ટેપ્સ:}

\begin{itemize}
\tightlist
\item
  મોડ 1 માં ટાઇમર 0 કોન્ફિગર કરો
\item
  1ms ડિલે માટે ટાઇમરમાં વેલ્યુ પ્રીલોડ કરો
\item
  ટાઇમર ઓવરફ્લો માટે રાહ જુઓ
\item
  સ્ક્વેર વેવ માટે આઉટપુટ બિટ ટોગલ કરો
\end{itemize}

\textbf{યાદ રાખવા માટે:} ``MSTCCS: મોડ-સેટ-ટાઇમર-ચેક-ક્લિયર-સ્વિચ''

\end{solutionbox}
\subsection*{પ્રશ્ન 4(ક OR) [7
ગુણ]}\label{uxaaauxab0uxab6uxaa8-4uxa95-or-7-uxa97uxaa3}

\textbf{8051 માઇક્રોકંટ્રોલર માટે કોઈપણ સાત લોજીકલ ઈન્સ્ટ્રક્શન ઉદાહરણ સાથે
સમજાવો.}

\begin{solutionbox}

{\def\LTcaptype{none} % do not increment counter
\begin{longtable}[]{@{}lll@{}}
\toprule\noalign{}
ઈન્સ્ટ્રક્શન & ઉદાહરણ & ઓપરેશન \\
\midrule\noalign{}
\endhead
\bottomrule\noalign{}
\endlastfoot
ANL & ANL A, \#3FH & લોજિકલ AND \\
ORL & ORL P1, \#80H & લોજિકલ OR \\
XRL & XRL A, R0 & લોજિકલ XOR \\
CLR & CLR A & ક્લિયર (0 સેટ) \\
CPL & CPL P1.0 & કોમ્પ્લિમેન્ટ (ઇન્વર્ટ) \\
RL & RL A & રોટેટ લેફ્ટ \\
RR & RR A & રોટેટ રાઇટ \\
\end{longtable}
}

\textbf{ઉદાહરણો:}

\begin{itemize}
\tightlist
\item
  \textbf{ANL}: \texttt{ANL\ A,\ \#0FH} (A = A AND 0FH, હાઈ નિબલ માસ્ક)
\item
  \textbf{ORL}: \texttt{ORL\ 20H,\ A} (20H = 20H OR A, બિટ્સ સેટ)
\item
  \textbf{XRL}: \texttt{XRL\ A,\ \#55H} (A = A XOR 55H, બિટ્સ ટોગલ)
\item
  \textbf{CLR}: \texttt{CLR\ C} (કેરી ફ્લેગ ક્લિયર, C = 0)
\item
  \textbf{CPL}: \texttt{CPL\ A} (A ને કોમ્પ્લિમેન્ટ, A = NOT A)
\item
  \textbf{RL}: \texttt{RL\ A} (A ને એક બિટ લેફ્ટ રોટેટ)
\item
  \textbf{RR}: \texttt{RR\ A} (A ને એક બિટ રાઇટ રોટેટ)
\end{itemize}

\textbf{યાદ રાખવા માટે:} ``A-OX-CCR: AND OR XOR ક્લિયર કોમ્પ્લિમેન્ટ રોટેટ''

\end{solutionbox}
\subsection*{પ્રશ્ન 5(અ) [3
ગુણ]}\label{uxaaauxab0uxab6uxaa8-5uxa85-3-uxa97uxaa3}

\textbf{8051 માઇક્રોકંટ્રોલર સાથે Push button Switch નું ઇન્ટરફેસિંગ દોરો.}

\begin{solutionbox}

\begin{verbatim}
         Vcc
          |
          R (10K)
          |
P1.0 {-{-}{-}{-}{-}+{-}{-}{-}{-}{-}{-} Push Button {-}{-}{-}{-}{-}{-} GND}
\end{verbatim}

{\def\LTcaptype{none} % do not increment counter
\begin{longtable}[]{@{}ll@{}}
\toprule\noalign{}
કમ્પોનન્ટ & કનેક્શન \\
\midrule\noalign{}
\endhead
\bottomrule\noalign{}
\endlastfoot
પુશ બટન & P1.0 અને GND વચ્ચે \\
પુલ-અપ રેસિસ્ટર & P1.0 અને VCC વચ્ચે 10K \\
પોર્ટ પિન & P1.0 ઇનપુટ તરીકે કોન્ફિગર \\
\end{longtable}
}

\textbf{મુખ્ય પોઇન્ટ્સ:}

\begin{itemize}
\tightlist
\item
  એક્ટિવ-લો કોન્ફિગરેશન (બટન દબાવવાથી 0 મળે)
\item
  પુલ-અપ રેસિસ્ટર ફ્લોટિંગ ઇનપુટ રોકે
\item
  કોઈપણ I/O પિન સાથે જોડી શકાય
\end{itemize}

\textbf{યાદ રાખવા માટે:} ``PIP: પુલ-અપ-ઇનપુટ-પ્રેસ''

\end{solutionbox}
\subsection*{પ્રશ્ન 5(બ) [4
ગુણ]}\label{uxaaauxab0uxab6uxaa8-5uxaac-4-uxa97uxaa3}

\textbf{8051 માઇક્રોકંટ્રોલર સાથે રિલે ઇન્ટરફેસ કરો.}

\begin{solutionbox}

\begin{verbatim}
                 5V
                 |
                 R (1K)
                 |
                 |   C (Diode)
                 |   |
P1.0 {-{-}{-}R(330){-}{-}{-}+{-}{-}{-}||{-}{-}{-}{-}+}
                     |      |
                     |      |
                 +{-{-}{-}+{-}{-}{-}+  |}
                 | NPN   |  |
                 |(BC547)|  |
                 +{-{-}{-}+{-}{-}{-}+  |}
                     |      |
                     |      |
                    GND     |
                            |
                         +{-{-}+{-}{-}+}
                         |Relay|{-{-}{-} Load}
                         +{-{-}{-}{-}{-}+}
\end{verbatim}

{\def\LTcaptype{none} % do not increment counter
\begin{longtable}[]{@{}ll@{}}
\toprule\noalign{}
કમ્પોનન્ટ & હેતુ \\
\midrule\noalign{}
\endhead
\bottomrule\noalign{}
\endlastfoot
NPN ટ્રાન્ઝિસ્ટર & કરંટ એમ્પ્લિફિકેશન \\
ડાયોડ & બેક EMF પ્રોટેક્શન \\
રેસિસ્ટર્સ & કરંટ લિમિટિંગ \\
રિલે & હાઈ-પાવર સ્વિચિંગ \\
\end{longtable}
}

\textbf{મુખ્ય સ્ટેપ્સ:}

\begin{itemize}
\tightlist
\item
  પોર્ટ પિન ટ્રાન્ઝિસ્ટર બેઝ ડ્રાઇવ કરે
\item
  ટ્રાન્ઝિસ્ટર રિલે કોઇલ સ્વિચ કરે
\item
  ડાયોડ બેક EMF સામે રક્ષણ આપે
\item
  રિલે કોન્ટેક્ટ હાઈ-પાવર લોડ સ્વિચ કરે
\end{itemize}

\textbf{યાદ રાખવા માટે:} ``TRIP: ટ્રાન્ઝિસ્ટર-રિલે-ઇન્ટરફેસ-પ્રોટેક્શન''

\end{solutionbox}
\subsection*{પ્રશ્ન 5(ક) [7 ગુણ]
(ચાલુ)}\label{uxaaauxab0uxab6uxaa8-5uxa95-7-uxa97uxaa3-uxa9auxab2}

\textbf{8051 માઇક્રોકંટ્રોલર સાથે ADC0804 ઇન્ટરફેસ કરો.}

\begin{solutionbox}

\textbf{સર્કિટ ડાયાગ્રામ:}

\begin{verbatim}
                     8051
                 +{-{-}{-}{-}{-}{-}{-}{-}{-}{-}+}
                 |          |
 Analog Input{-{-}{-}| ADC0804  |}
 0{-5V        |   |          |}
             v   |          |
        +{-{-}{-}+{-}{-}{-}+|          |     +{-}{-}{-}{-}{-}{-}{-}{-}{-}+}
        |        |          |     |         |
Vref/2{-|        |          |     |         |}
        |        |          |{{-}{-}{-}|P1.0{-}P1.7|}
CS{-{-}{-}{-}{-}|        |          |     |         |}
RD{-{-}{-}{-}{-}|        |          |     |         |}
WR{-{-}{-}{-}{-}|        |          |     |         |}
INTR{-{-}{-}|        |          |{-}{-}{-}{-}|P3.2     |}
        |        |          |     |         |
        +{-{-}{-}{-}{-}{-}{-}{-}+          |     |         |}
                 +{-{-}{-}{-}{-}{-}{-}{-}{-}{-}+     +{-}{-}{-}{-}{-}{-}{-}{-}{-}+}
\end{verbatim}

{\def\LTcaptype{none} % do not increment counter
\begin{longtable}[]{@{}lll@{}}
\toprule\noalign{}
કનેક્શન & 8051 પિન & ADC0804 પિન \\
\midrule\noalign{}
\endhead
\bottomrule\noalign{}
\endlastfoot
ડેટા બસ & P1.0-P1.7 & D0-D7 \\
CS & P3.0 & CS \\
RD & P3.1 & RD \\
WR & P3.2 & WR \\
INTR & P3.3 & INTR \\
\end{longtable}
}

\begin{itemize}
\tightlist
\item
  \textbf{ADC0804}: 8-બિટ A/D કન્વર્ટર 0-5V ઇનપુટ રેન્જ સાથે
\item
  \textbf{ઇન્ટરફેસ}: ડેટા પિન પોર્ટ 1 સાથે, કંટ્રોલ પોર્ટ 3 સાથે જોડો
\item
  \textbf{ઓપરેશન}: કન્વર્ઝન શરૂ કરવા ADC ને લખો, INTR માટે રાહ જુઓ, રિઝલ્ટ વાંચો
\item
  \textbf{રેઝોલ્યુશન}: 8-બિટ (256 સ્ટેપ) 0-5V માટે \textasciitilde19.5mV પ્રતિ
  સ્ટેપ
\end{itemize}

\textbf{યાદ રાખવા માટે:} ``CRIW: કંટ્રોલ-રીડ-ઇન્ટરપ્ટ-રાઇટ''

\end{solutionbox}
\subsection*{પ્રશ્ન 5(અ OR) [3
ગુણ]}\label{uxaaauxab0uxab6uxaa8-5uxa85-or-3-uxa97uxaa3}

\textbf{વિવિધ ક્ષેત્રોમાં માઇક્રોકંટ્રોલરની એપ્લિકેશનોની સૂચિ બનાવો.}

\begin{solutionbox}

{\def\LTcaptype{none} % do not increment counter
\begin{longtable}[]{@{}ll@{}}
\toprule\noalign{}
ક્ષેત્ર & એપ્લિકેશન્સ \\
\midrule\noalign{}
\endhead
\bottomrule\noalign{}
\endlastfoot
ઔદ્યોગિક & મોટર કંટ્રોલ, ઓટોમેશન, PLCs \\
મેડિકલ & પેશન્ટ મોનિટરિંગ, ડાયગ્નોસ્ટિક ઉપકરણો \\
કન્ઝ્યુમર & વોશિંગ મશીન, માઇક્રોવેવ, રમકડાં \\
ઓટોમોટિવ & એન્જિન કંટ્રોલ, ABS, એરબેગ સિસ્ટમ \\
કમ્યુનિકેશન & મોબાઇલ ફોન, મોડેમ, રાઉટર \\
સિક્યુરિટી & એક્સેસ કંટ્રોલ, અલાર્મ સિસ્ટમ \\
\end{longtable}
}

\textbf{યાદ રાખવા માટે:} ``I-MACS:
ઇન્ડસ્ટ્રિયલ-મેડિકલ-ઓટોમોટિવ-કન્ઝ્યુમર-સિક્યુરિટી''

\end{solutionbox}
\subsection*{પ્રશ્ન 5(બ OR) [4
ગુણ]}\label{uxaaauxab0uxab6uxaa8-5uxaac-or-4-uxa97uxaa3}

\textbf{8051 માઇક્રોકંટ્રોલર સાથે સ્ટેપર મોટર ઇન્ટરફેસ કરો.}

\begin{solutionbox}

\textbf{સર્કિટ ડાયાગ્રામ:}

\begin{verbatim}
          8051                      ULN2003
       +{-{-}{-}{-}{-}{-}{-}{-}+                 +{-}{-}{-}{-}{-}{-}{-}{-}{-}+}
       |        |           +{-{-}{-}{-}|IN1  OUT1|{-}{-}{-}+}
       |   P1.0 |{-{-}{-}{-}{-}{-}{-}{-}{-}{-}{-}|{-}{-}{-}{-}|IN2  OUT2|{-}{-}{-}+}
       |   P1.1 |{-{-}{-}{-}{-}{-}{-}{-}{-}{-}{-}|{-}{-}{-}{-}|IN3  OUT3|{-}{-}{-}+{-}{-}{-}{-} 4{-}wire}
       |   P1.2 |{-{-}{-}{-}{-}{-}{-}{-}{-}{-}{-}|{-}{-}{-}{-}|IN4  OUT4|{-}{-}{-}+      Stepper}
       |   P1.3 |{-{-}{-}{-}{-}{-}{-}{-}{-}{-}{-}|     |         |           Motor}
       |        |           |     |         |
       +{-{-}{-}{-}{-}{-}{-}{-}+           |     +{-}{-}{-}{-}{-}{-}{-}{-}{-}+}
                            |
                          +5V
\end{verbatim}

{\def\LTcaptype{none} % do not increment counter
\begin{longtable}[]{@{}ll@{}}
\toprule\noalign{}
કમ્પોનન્ટ & હેતુ \\
\midrule\noalign{}
\endhead
\bottomrule\noalign{}
\endlastfoot
ULN2003 & ડ્રાઇવર IC ડાર્લિંગટન એરે સાથે \\
પોર્ટ પિન & P1.0-P1.3 4 મોટર ફેઝ માટે \\
પાવર સપ્લાય & મોટર માટે અલગ સપ્લાય \\
\end{longtable}
}

\textbf{કોડ સ્ટ્રક્ચર:}

\begin{verbatim}
; ક્લોકવાઇઝ રોટેશન સિક્વન્સ
STEP\_SEQ: DB 0000\_1000B  ; સ્ટેપ 1
          DB 0000\_1100B  ; સ્ટેપ 2
          DB 0000\_0100B  ; સ્ટેપ 3
          DB 0000\_0110B  ; સ્ટેપ 4
\end{verbatim}

\textbf{યાદ રાખવા માટે:} ``PDCS: પોર્ટ-ડ્રાઇવર-કરંટ-સિક્વન્સ''

\end{solutionbox}
\subsection*{પ્રશ્ન 5(ક OR) [7
ગુણ]}\label{uxaaauxab0uxab6uxaa8-5uxa95-or-7-uxa97uxaa3}

\textbf{8051 માઇક્રોકંટ્રોલર સાથે LCD ઇન્ટરફેસ કરો.}

\begin{solutionbox}

\textbf{સર્કિટ ડાયાગ્રામ:}

\begin{verbatim}
           8051                  16x2 LCD
       +{-{-}{-}{-}{-}{-}{-}{-}+              +{-}{-}{-}{-}{-}{-}{-}{-}{-}+}
       |        |              |         |
       |   P2.0 |{-{-}{-}{-}{-}{-}{-}{-}{-}{-}{-}{-}{-}|D0       |}
       |   P2.1 |{-{-}{-}{-}{-}{-}{-}{-}{-}{-}{-}{-}{-}|D1       |}
       |   P2.2 |{-{-}{-}{-}{-}{-}{-}{-}{-}{-}{-}{-}{-}|D2       |}
       |   P2.3 |{-{-}{-}{-}{-}{-}{-}{-}{-}{-}{-}{-}{-}|D3       |}
       |   P2.4 |{-{-}{-}{-}{-}{-}{-}{-}{-}{-}{-}{-}{-}|D4       |}
       |   P2.5 |{-{-}{-}{-}{-}{-}{-}{-}{-}{-}{-}{-}{-}|D5       |}
       |   P2.6 |{-{-}{-}{-}{-}{-}{-}{-}{-}{-}{-}{-}{-}|D6       |}
       |   P2.7 |{-{-}{-}{-}{-}{-}{-}{-}{-}{-}{-}{-}{-}|D7       |}
       |        |              |         |
       |   P3.0 |{-{-}{-}{-}{-}{-}{-}{-}{-}{-}{-}{-}{-}|RS       |}
       |   P3.1 |{-{-}{-}{-}{-}{-}{-}{-}{-}{-}{-}{-}{-}|R/W      |}
       |   P3.2 |{-{-}{-}{-}{-}{-}{-}{-}{-}{-}{-}{-}{-}|E        |}
       |        |              |         |
       +{-{-}{-}{-}{-}{-}{-}{-}+              +{-}{-}{-}{-}{-}{-}{-}{-}{-}+}
                                  |   |
                                 Vcc GND
\end{verbatim}

{\def\LTcaptype{none} % do not increment counter
\begin{longtable}[]{@{}ll@{}}
\toprule\noalign{}
કનેક્શન & હેતુ \\
\midrule\noalign{}
\endhead
\bottomrule\noalign{}
\endlastfoot
ડેટા પિન (D0-D7) & P2.0-P2.7 સાથે જોડો \\
RS & રજિસ્ટર સિલેક્ટ (0=કમાન્ડ, 1=ડેટા) \\
R/W & રીડ/રાઇટ (0=રાઇટ, 1=રીડ) \\
E & એનેબલ સિગ્નલ (એક્ટિવ હાઈ) \\
\end{longtable}
}

\textbf{બેઝિક કમાન્ડ્સ:}

\begin{verbatim}
0x01 - ડિસ્પ્લે ક્લિયર
0x02 - હોમ પોઝિશન
0x0C - ડિસ્પ્લે ON, કર્સર OFF
0x38 - 8-બિટ, 2 લાઇન, 5x7 ડોટ્સ
\end{verbatim}

\begin{itemize}
\tightlist
\item
  \textbf{ઇનિશિયલાઇઝેશન}: LCD ને 8-બિટ મોડ, 2 લાઇન માટે કોન્ફિગર કરો
\item
  \textbf{રાઇટિંગ}: RS=1 સાથે ડેટા, RS=0 સાથે કંટ્રોલ મોકલો
\item
  \textbf{ટાઇમિંગ}: E પલ્સ ટાઇમિંગ જરૂરિયાતો પૂરી કરવી જોઈએ
\item
  \textbf{કોન્ટ્રાસ્ટ}: VEE પિન પર પોટેન્શિયોમીટર સાથે એડજસ્ટ કરો
\end{itemize}

\textbf{યાદ રાખવા માટે:} ``DICE: ડેટા-ઇન્સ્ટ્રક્શન-કંટ્રોલ-એનેબલ''

\end{solutionbox}

\end{document}
