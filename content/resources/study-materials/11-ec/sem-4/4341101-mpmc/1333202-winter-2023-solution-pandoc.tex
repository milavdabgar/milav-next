\documentclass[10pt,a4paper]{article}

% content/resources/templates/preamble.tex
\usepackage[margin=0.6in]{geometry}
\author{Milav Dabgar}
\usepackage{amsmath,amssymb,amsthm}
\usepackage{booktabs}
\usepackage{multirow}
\usepackage{xcolor}
\usepackage{tcolorbox}
\tcbuselibrary{breakable,skins}
\usepackage[colorlinks=true,linkcolor=blue]{hyperref}
\usepackage{titlesec}
\usepackage{enumitem}
\usepackage{tikz}
\usepackage{pgfplots}
\usepackage{circuitikz}
\usepackage[version=4]{mhchem}
\usepackage{longtable}
\usepackage{array}
\usepackage{float}
\usepackage{caption}
\usepackage{listings}

\lstset{
  basicstyle=\small\ttfamily,
  breaklines=true,
  breakatwhitespace=false,
  postbreak=\mbox{\textcolor{red}{$\hookrightarrow$}\space},
  float=false,
  numbers=left,
  numberstyle=\tiny\color{gray},
  numbersep=10pt,
  xleftmargin=2em,
  keywordstyle=\color{blue},
  commentstyle=\color{green!60!black},
  stringstyle=\color{purple},
  backgroundcolor=\color{gray!5},
  showstringspaces=false,
  tabsize=2,
  captionpos=b,
  keepspaces=true,
  columns=flexible
}

\pgfplotsset{compat=1.18}
\usetikzlibrary{shapes,arrows,positioning,calc,patterns,decorations.pathmorphing,decorations.markings,arrows.meta}

% Color scheme
\definecolor{headcolor}{RGB}{0,102,204}
\definecolor{keycolor}{RGB}{220,20,60}
\definecolor{solutioncolor}{RGB}{34,139,34}
\definecolor{mnemoniccolor}{RGB}{148,0,211}
\definecolor{codecolor}{RGB}{0,0,100}

% Spacing
\setlength{\parskip}{3pt}
\setlist[itemize]{nosep}
\setlist[enumerate]{nosep}

% Title formatting
\titleformat{\section}{\Large\bfseries\color{headcolor}}{\thesection}{1em}{}
\titleformat{\subsection}{\large\bfseries\color{headcolor}}{\thesubsection}{1em}{}

% Pandoc tightlist compatibility
\providecommand{\tightlist}{%
  \setlength{\itemsep}{0pt}\setlength{\parskip}{0pt}}

% Pandoc longtable compatibility
\newcounter{none}
\def\thenone{}


% content/resources/templates/english-boxes.tex
% This file is currently empty - it exists to maintain consistency with the import structure.
% Add custom environments here if needed in the future.


\begin{document}

\begin{center}
{\Huge\bfseries\color{headcolor} Subject Name Solutions}\\[5pt]
{\LARGE 1333202 -- Winter 2023}\\[3pt]
{\large Semester 1 Study Material}\\[3pt]
{\normalsize\textit{Detailed Solutions and Explanations}}
\end{center}

\vspace{10pt}

\subsubsection{Q1a: Define
Microprocessor}\label{q1a-define-microprocessor}

\textbf{Definition of a Microprocessor}

A microprocessor is a single integrated circuit (IC) that incorporates
the core functions of a computer's central processing unit (CPU).

\begin{itemize}
\item
  \textbf{Key Points:}

  \begin{itemize}
  \tightlist
  \item
    \textbf{The ``Brain'' of a Computer:} It executes instructions,
    performs calculations, and manages the flow of data within a
    computer system.
  \item
    \textbf{Small and Powerful:} Microprocessors pack millions or even
    billions of transistors into a tiny chip, enabling complex
    processing in compact devices.
  \item
    \textbf{Essential for Modern Devices:} They power a vast range of
    devices from smartphones and laptops to cars, appliances, and
    industrial equipment.
  \item
    \textbf{Components:} Typical components of a microprocessor include:

    \begin{itemize}
    \tightlist
    \item
      Arithmetic Logic Unit (ALU) - Performs arithmetic and logical
      operations
    \item
      Control Unit (CU) - Decodes instructions and coordinates the
      operations of other units
    \item
      Registers - Small, high-speed memory locations for temporary data
      storage
    \end{itemize}
  \end{itemize}
\end{itemize}

\subsubsection{Q1b: Explain Flag register of 8085
microprocessor}\label{q1b-explain-flag-register-of-8085-microprocessor}

\textbf{The Flag Register}

The Flag register in the 8085 is an 8-bit register, with only 5 bits
actively used as flags. These flags act as individual flip-flops that
are set (1) or reset (0) to reflect specific conditions arising from
arithmetic, logical, and other operations performed by the ALU
(Arithmetic and Logic Unit).

\textbf{The 5 Flags:}

\begin{enumerate}
\item
  \textbf{Sign Flag (S):}

  \begin{itemize}
  \tightlist
  \item
    Set (1) if the result of an operation is negative (the Most
    Significant Bit, or MSB, of the result is 1).
  \item
    Reset (0) if the result is positive.
  \end{itemize}
\item
  \textbf{Zero Flag (Z):}

  \begin{itemize}
  \tightlist
  \item
    Set (1) if the result of an operation is zero.
  \item
    Reset (0) if the result is not zero.
  \end{itemize}
\item
  \textbf{Auxiliary Carry Flag (AC):}

  \begin{itemize}
  \tightlist
  \item
    Set (1) if there is a carry-out from the lower nibble (lower 4 bits)
    into the upper nibble (upper 4 bits) of a result.
  \item
    Used primarily in instructions that perform decimal arithmetic.
  \end{itemize}
\item
  \textbf{Parity Flag (P):}

  \begin{itemize}
  \tightlist
  \item
    Set (1) if the result has even parity (contains an even number of
    1s).
  \item
    Reset (0) if the result has odd parity.
  \end{itemize}
\item
  \textbf{Carry Flag (CY):}

  \begin{itemize}
  \tightlist
  \item
    Set (1) if there is a carry-out from the most significant bit (MSB)
    of a result during addition, or a borrow during subtraction.
  \item
    Reset (0) otherwise.
  \end{itemize}
\end{enumerate}

\textbf{How the Flags are Used:}

\begin{itemize}
\tightlist
\item
  \textbf{Conditional Jumps:} Instructions like JZ (Jump if Zero), JNZ
  (Jump if Not Zero), JC (Jump if Carry), etc. use the status of these
  flags to determine whether to branch to different parts of the
  program.
\item
  \textbf{Decision Making:} The processor can examine flag states to
  modify calculations or behaviors based on previous operations.
\end{itemize}

\textbf{Example:}

\begin{verbatim}
; Assume the accumulator (A) holds the value 50
SUB B  ; Subtract the value in register B from the accumulator
JZ LABEL  ; If the result is zero, jump to the code section marked as LABEL
\end{verbatim}

\subsubsection{Q1c: Explain format of instruction of 8085 microprocessor
with
example}\label{q1c-explain-format-of-instruction-of-8085-microprocessor-with-example}

\textbf{8085 Instruction Formats}

Instructions in the 8085 microprocessor can be 1, 2, or 3 bytes long.
The structure varies depending on the specific instruction and the
addressing modes used.

\textbf{General Structure}

\begin{itemize}
\item
  \textbf{Opcode (Operation Code):} The first byte of an instruction. It
  specifies the operation to be performed (e.g., MOV, ADD, JMP).
\item
  \textbf{Operands (Optional):} The second and third bytes, if present,
  provide data or addresses required by the operation. Operands can be:

  \begin{itemize}
  \tightlist
  \item
    \textbf{Registers:} 8-bit registers within the 8085 (B, C, D, E, H,
    L, or the accumulator A).
  \item
    \textbf{Immediate Data:} 8-bit or 16-bit data embedded directly into
    the instruction.
  \item
    \textbf{Memory Addresses:} 16-bit addresses of memory locations.
  \end{itemize}
\end{itemize}

\textbf{8085 Instruction Examples}

\begin{enumerate}
\item
  \textbf{Single-Byte Instruction (No Operands):}

  \begin{itemize}
  \tightlist
  \item
    \textbf{Instruction:} NOP (No Operation)
  \item
    \textbf{Opcode:} 00000000
  \item
    \textbf{Explanation:} Does nothing - the processor simply moves to
    the next instruction.
  \end{itemize}
\item
  \textbf{Two-Byte Instruction (Immediate Data):}

  \begin{itemize}
  \tightlist
  \item
    \textbf{Instruction:} MVI A, 42H (Move Immediate to Accumulator)
  \item
    \textbf{Opcode:} 00111110
  \item
    \textbf{Operand:} 42H (Hexadecimal value to be loaded)
  \item
    \textbf{Explanation:} Loads the value 42H into the accumulator.
  \end{itemize}
\item
  \textbf{Three-Byte Instruction (16-bit Memory Address):}

  \begin{itemize}
  \tightlist
  \item
    \textbf{Instruction:} LDA 2050H (Load Accumulator Direct)
  \item
    \textbf{Opcode:} 00111010
  \item
    \textbf{Operand:} 2050H (16-bit memory address)
  \item
    \textbf{Explanation:} Loads the content of the memory location at
    address 2050H into the accumulator.
  \end{itemize}
\end{enumerate}

\textbf{Addressing Modes}

The way operands are specified determines the ``addressing mode'' of the
instruction. The 8085 supports modes like:

\begin{itemize}
\tightlist
\item
  \textbf{Register addressing:} The operand is a register.
\item
  \textbf{Direct addressing:} The operand is a 16-bit memory address.
\item
  \textbf{Immediate addressing:} The operand is data within the
  instruction.
\item
  \textbf{Register indirect addressing:} The operand's address is held
  within a register pair.
\end{itemize}

\textbf{Remember:}

\begin{itemize}
\tightlist
\item
  The specific format depends on the instruction and the addressing mode
  used.
\item
  Opcodes and addressing modes are how the 8085 interprets the bytes
  that make up an instruction.
\end{itemize}

\subsubsection{Q1c: Explain function of ALU, Control Unit and CPU of
8085
microprocessor}\label{q1c-explain-function-of-alu-control-unit-and-cpu-of-8085-microprocessor}

\textbf{1. ALU (Arithmetic Logic Unit)}

\begin{itemize}
\tightlist
\item
  \textbf{Heart of Calculations:} The ALU performs the core arithmetic
  and logical operations within the microprocessor.
\item
  \textbf{Operations:}

  \begin{itemize}
  \tightlist
  \item
    Arithmetic: Addition, subtraction, increment, decrement, etc.
  \item
    Logical: AND, OR, XOR, NOT, comparisons, etc.
  \end{itemize}
\item
  \textbf{Flags:} Sets status flags (Carry, Zero, Sign, Parity) based on
  the results of its operations. These flags are used for conditional
  branching and decision-making by the processor.
\end{itemize}

\textbf{2. Control Unit}

\begin{itemize}
\tightlist
\item
  \textbf{The Orchestrator:} Governs the overall operation of the
  microprocessor.
\item
  \textbf{Key Functions:}

  \begin{itemize}
  \tightlist
  \item
    \textbf{Instruction Decoding:} Interprets the opcode of the current
    instruction fetched from memory.
  \item
    \textbf{Control Signal Generation:} Produces control signals that
    synchronize and manage the actions of all other units within the
    microprocessor (ALU, registers, memory interface, etc.).
  \item
    \textbf{Data Flow Management:} Coordinates the movement of data
    between the ALU, registers, and memory/I/O devices.
  \end{itemize}
\end{itemize}

\textbf{3. CPU (Central Processing Unit)}

\begin{itemize}
\tightlist
\item
  \textbf{The Brain of the System:} The CPU is the combination of the
  ALU and the Control Unit.
\item
  \textbf{Responsibilities:}

  \begin{itemize}
  \tightlist
  \item
    \textbf{Instruction Execution:} Fetches instructions from memory,
    decodes them using the control unit, and executes them using the ALU
    and other components.
  \item
    \textbf{Program Control:} Manages the flow of instructions within a
    program, including branching and jumps based on conditions.
  \item
    \textbf{System Management:} Handles communication with external
    devices and responds to interrupts.
  \end{itemize}
\end{itemize}

\textbf{How They Work Together}

\begin{enumerate}
\tightlist
\item
  The Control Unit fetches an instruction from memory.
\item
  The Control Unit decodes the instruction and generates the necessary
  control signals.
\item
  The ALU, if needed, performs the required arithmetic or logical
  operation.
\item
  Results may be stored in registers, written to memory, or sent to
  output devices.
\item
  The Control Unit directs the processor to fetch the next instruction,
  continuing the cycle.
\end{enumerate}

\subsubsection{Q2a: Explain function of ALE signal with
diagram}\label{q2a-explain-function-of-ale-signal-with-diagram}

\textbf{What is the ALE Signal?}

\begin{itemize}
\tightlist
\item
  The ALE signal is a control signal generated by the 8085
  microprocessor.
\item
  It is a positive-going pulse that occurs during the first clock cycle
  (T1 state) of each machine cycle.
\end{itemize}

\textbf{Purpose of the ALE Signal}

The primary function of the ALE signal is to demultiplex the lower-order
address/data bus (AD0-AD7). This bus is shared (multiplexed) to carry
both:

\begin{enumerate}
\tightlist
\item
  \textbf{Lower 8-bits of the Address (during T1 state):} The 8085 needs
  to send out the 16-bit address of a memory location or I/O port. The
  lower 8 bits of the address are carried on lines AD0-AD7.
\item
  \textbf{Data (during subsequent states):} The same lines are used to
  transmit or receive actual data to/from the memory or I/O device.
\end{enumerate}

\textbf{How ALE Demultiplexes the Bus}

\begin{enumerate}
\item
  \textbf{T1 State:}

  \begin{itemize}
  \tightlist
  \item
    The ALE signal goes high.
  \item
    The 8085 places the lower 8 bits of the address on lines AD0-AD7.
  \item
    An external latch (usually an 8282 or 8283 octal latch) connected to
    these lines ``latches'' or captures this address information.
  \end{itemize}
\item
  \textbf{Subsequent States (T2, T3, \ldots):}

  \begin{itemize}
  \tightlist
  \item
    ALE goes low.
  \item
    The lower-order address lines (AD0-AD7) are now free to be used as a
    data bus for transferring data.
  \end{itemize}
\end{enumerate}

\textbf{Diagram}

A simple timing diagram can help visualize this:

\begin{verbatim}
          ______         ______
ALE      |      |_______|      |______
          _____           _____
AD0-AD7  |Addr |_______| Data |________
         (T1)      (T2, T3, ...)
\end{verbatim}

\textbf{Key Points:}

\begin{itemize}
\tightlist
\item
  The ALE signal is crucial for the 8085 to correctly interface with
  memory and I/O devices.
\item
  The external latch holds the lower order address bits, freeing the
  8085 to continue its fetch or write operation.
\end{itemize}

\subsubsection{Q2b: Compare microprocessor and
microcontroller}\label{q2b-compare-microprocessor-and-microcontroller}

{\def\LTcaptype{none} % do not increment counter
\begin{longtable}[]{@{}
  >{\raggedright\arraybackslash}p{(\linewidth - 4\tabcolsep) * \real{0.1583}}
  >{\raggedright\arraybackslash}p{(\linewidth - 4\tabcolsep) * \real{0.4167}}
  >{\raggedright\arraybackslash}p{(\linewidth - 4\tabcolsep) * \real{0.4250}}@{}}
\toprule\noalign{}
\begin{minipage}[b]{\linewidth}\raggedright
Feature
\end{minipage} & \begin{minipage}[b]{\linewidth}\raggedright
Microprocessor
\end{minipage} & \begin{minipage}[b]{\linewidth}\raggedright
Microcontroller
\end{minipage} \\
\midrule\noalign{}
\endhead
\bottomrule\noalign{}
\endlastfoot
\textbf{Complexity} & Less complex & More complex \\
\textbf{Instruction Set} & Larger and more versatile & Smaller and more
application-specific \\
\textbf{Memory} & Requires external memory (RAM, ROM) & Has built-in
memory (RAM, ROM, Flash) \\
\textbf{Peripherals} & Requires external peripherals (e.g., display,
I/O) & Has built-in peripherals (e.g., timers, ADCs, DACs) \\
\textbf{Cost} & Generally lower cost & Generally higher cost \\
\textbf{Applications} & General-purpose computing & Embedded systems,
specific tasks \\
\textbf{Examples} & Intel 8085, Intel x86 & Atmel AVR, PIC, ARM
Cortex-M \\
\end{longtable}
}

\subsubsection{Q2c: Draw \& explain block diagram of 8085
microprocessor}\label{q2c-draw-explain-block-diagram-of-8085-microprocessor}

\textbf{Block Diagram}

\begin{figure}
\centering
\pandocbounded{\includegraphics[keepaspectratio,alt={Block Diagram of 8085}]{./imgs/8085_architeccher.png}}
\caption{Block Diagram of 8085}
\end{figure}

\textbf{Key Components and their Functions}

\begin{enumerate}
\item
  \textbf{Accumulator:} An 8-bit register that's central to arithmetic
  and logical operations performed by the ALU.
\item
  \textbf{Arithmetic and Logic Unit (ALU):} Performs arithmetic
  operations (addition, subtraction, etc.) and logical operations (AND,
  OR, NOT, etc.). It sets flags (Carry, Zero, Sign, etc.) based on the
  results.
\item
  \textbf{Temporary Register:} A temporary holding location for data
  used during instruction execution.
\item
  \textbf{Instruction Register:} Holds the currently fetched
  instruction.
\item
  \textbf{Instruction Decoder and Machine Cycle Encoder:} Decodes the
  instruction in the instruction register and generates control signals
  to coordinate the microprocessor's actions during a machine cycle.
\item
  \textbf{Register Array:} Contains six general purpose 8-bit registers
  (B, C, D, E, H, and L), which can be used individually or in pairs
  (BC, DE, HL) for 16-bit operations.
\item
  \textbf{Program Counter (PC):} A 16-bit register that holds the memory
  address of the next instruction to be fetched.
\item
  \textbf{Stack Pointer (SP):} A 16-bit register pointing to the top of
  the stack in memory. The stack is used for storing return addresses of
  subroutines and temporarily storing data.
\item
  \textbf{Timing and Control Unit:} Generates timing and control signals
  for all operations within the microprocessor and synchronizes with
  external devices.
\item
  \textbf{Interrupt Control:} Handles incoming interrupt requests (if
  any), acknowledging them and allowing them to temporarily disrupt the
  current program execution.
\item
  \textbf{Serial I/O Control:} Facilitates serial input and output,
  useful for slower communication with certain types of peripherals.
\item
  \textbf{Address Bus (A8 - A15):} The upper 8-bits of the 16-bit
  address bus, used to send the most significant portion of an address.
\item
  \textbf{Address/Data Bus (AD0 - AD7):} A multiplexed bus. It carries
  the lower 8 bits of an address during the beginning of a machine cycle
  and data during data transfer operations.
\end{enumerate}

\textbf{How it Works (Simplified)}

\begin{enumerate}
\tightlist
\item
  \textbf{Fetch:} The PC provides an address; the instruction is fetched
  from memory and placed into the Instruction Register.
\item
  \textbf{Decode:} The Instruction Decoder decodes the instruction to
  understand what needs to be done.
\item
  \textbf{Execute:} The Control Unit generates signals to coordinate the
  ALU, registers, and other components as they perform the necessary
  operations.
\item
  \textbf{Repeat:} The process continues, fetching and executing
  instructions sequentially.
\end{enumerate}

\subsubsection{Q2a: Explain 16 bits registers of 8085
microprocessor}\label{q2a-explain-16-bits-registers-of-8085-microprocessor}

\textbf{16-Bit Registers in the 8085}

The 8085 microprocessor, while primarily an 8-bit processor, features
several 16-bit registers that are crucial for memory addressing and
specific operations:

\begin{itemize}
\item
  \textbf{Program Counter (PC):}

  \begin{itemize}
  \tightlist
  \item
    Holds the 16-bit memory address of the next instruction to be
    fetched and executed by the processor.
  \item
    Essential for maintaining the correct sequence of program execution.
  \end{itemize}
\item
  \textbf{Stack Pointer (SP):}

  \begin{itemize}
  \tightlist
  \item
    Points to the current top of the stack in memory.
  \item
    The stack is a Last-In, First-Out (LIFO) data structure used for
    storing return addresses during subroutine calls, temporary data,
    and interrupt handling.
  \end{itemize}
\item
  \textbf{Register Pairs (BC, DE, HL):}

  \begin{itemize}
  \tightlist
  \item
    While B, C, D, E, H, and L are individual 8-bit registers, they can
    be paired together to form 16-bit registers:

    \begin{itemize}
    \tightlist
    \item
      BC
    \item
      DE
    \item
      HL
    \end{itemize}
  \item
    These register pairs allow for operations on 16-bit data and for
    holding 16-bit memory addresses.
  \end{itemize}
\end{itemize}

\textbf{Key Functions of 16-bit Registers}

\begin{enumerate}
\item
  \textbf{Memory Addressing:} The 8085 has a 16-bit address bus, meaning
  it can address up to 64KB of memory. The 16-bit registers are used to
  store and manipulate memory addresses for data storage and retrieval.
\item
  \textbf{Subroutine Calls and Returns:} When a subroutine is called
  (using instructions like CALL), the processor needs to store the
  address where it should return to after the subroutine is finished.
  The Program Counter is pushed onto the stack for safekeeping.
\item
  \textbf{Data Manipulation:} Some instructions treat these register
  pairs as a single unit for performing 16-bit operations (e.g.,
  addition, loading immediate 16-bit values).
\end{enumerate}

\subsubsection{Q2b: Explain de-multiplexing lower order address and data
lines with diagram of 8085
microprocessor}\label{q2b-explain-de-multiplexing-lower-order-address-and-data-lines-with-diagram-of-8085-microprocessor}

\textbf{Why Demultiplexing is Needed}

The Intel 8085 utilizes a multiplexed address/data bus to reduce the
number of pins required. The lower 8 lines (AD0-AD7) carry two types of
information:

\begin{enumerate}
\item
  \textbf{Address (during T1 state):} During the first clock cycle of a
  machine cycle, these lines hold the lower 8 bits of a 16-bit memory or
  I/O address.
\item
  \textbf{Data (during subsequent states):} In the remaining clock
  cycles, those same lines transmit or receive the actual data being
  sent to or from a memory location or I/O device.
\end{enumerate}

\textbf{Demultiplexing Process}

Demultiplexing is the process of separating the address and data
information so the 8085 and external devices can operate correctly.
Here's how it's achieved:

\begin{enumerate}
\item
  \textbf{The ALE Signal:} During the first clock cycle (T1), the 8085
  asserts the ALE (Address Latch Enable) control signal. This signal
  goes high.
\item
  \textbf{External Latch:} An external latch circuit (e.g., 8282 or
  74LS373 octal latch) is connected to the AD0-AD7 lines. When the ALE
  signal goes high, this latch captures and holds the lower 8 bits of
  the address.
\item
  \textbf{Address Decoded:} The latched lower-order address bits, along
  with the higher-order address bits (A8-A15), provide the complete
  16-bit address for memory or I/O devices.
\item
  \textbf{Data Bus Freed:} After the T1 state, the ALE signal goes low.
  The AD0-AD7 lines are now free to be used as a data bus for the
  remainder of the machine cycle.
\end{enumerate}

\textbf{Diagram}

\begin{figure}
\centering
\pandocbounded{\includegraphics[keepaspectratio,alt={Demultiplexing of Lower Order Address Bus \& Data Bus}]{./imgs/de-multiplexing.png}}
\caption{Demultiplexing of Lower Order Address Bus \& Data Bus}
\end{figure}

\textbf{Key Points}

\begin{itemize}
\tightlist
\item
  Demultiplexing enables the 8085 to interface with memory and I/O
  devices correctly by separating the address and data functions of the
  same physical bus lines.
\item
  The ALE signal plays a crucial role in timing the latching of address
  information.
\end{itemize}

\subsubsection{Q2c: Draw and explain pin diagram of
8085}\label{q2c-draw-and-explain-pin-diagram-of-8085}

\textbf{Pin Diagram}

\begin{figure}
\centering
\pandocbounded{\includegraphics[keepaspectratio,alt={8085 microprocessor pin diagram}]{./imgs/8085-pin-diagram.png}}
\caption{8085 microprocessor pin diagram}
\end{figure}

\textbf{Explanation of Pin Groups}

\begin{enumerate}
\item
  \textbf{Address Bus (A8-A15):} The upper 8-bits of the 16-bit address
  bus used for addressing memory and I/O devices.
\item
  \textbf{Multiplexed Address/Data Bus (AD0-AD7):} These pins serve two
  functions:

  \begin{itemize}
  \tightlist
  \item
    During the first clock state (T1), they carry the lower 8-bits of
    the address.
  \item
    During subsequent clock states, they serve as the data bus for data
    transfer.
  \end{itemize}
\item
  \textbf{Control and Status Signals}

  \begin{itemize}
  \tightlist
  \item
    \textbf{ALE (Address Latch Enable):} Indicates that the AD0-AD7
    lines contain a valid address.
  \item
    \textbf{RD (Read):} Indicates a read operation from memory or I/O.
  \item
    \textbf{WR (Write):} Indicates a write operation to memory or I/O.
  \item
    \textbf{IO/M (IO/Memory Select):} Distinguishes between memory (IO/M
    = 0) and I/O (IO/M = 1) operations.
  \item
    \textbf{S0, S1 (Status signals):} These, along with IO/M, indicate
    the type of machine cycle (opcode fetch, memory read, I/O write,
    etc.).
  \end{itemize}
\item
  \textbf{Power Supply and Clock}

  \begin{itemize}
  \tightlist
  \item
    \textbf{VCC:} +5V power supply.
  \item
    \textbf{VSS:} Ground (0V).
  \item
    \textbf{X1, X2:} Connections for a crystal or external clock source
    to drive the internal clock generator.
  \item
    \textbf{CLK (OUT):} Clock output signal for synchronizing external
    devices.
  \end{itemize}
\item
  \textbf{Interrupts}

  \begin{itemize}
  \tightlist
  \item
    \textbf{TRAP:} Highest priority non-maskable interrupt.
  \item
    \textbf{RST 7.5, RST 6.5, RST 5.5:} Maskable interrupts with
    decreasing priority.
  \item
    \textbf{INTR:} General maskable interrupt.
  \item
    \textbf{INTA:} Interrupt acknowledge signal sent by the 8085.
  \end{itemize}
\item
  \textbf{Serial I/O}

  \begin{itemize}
  \tightlist
  \item
    \textbf{SID (Serial Input Data):} Input line for serial data.
  \item
    \textbf{SOD (Serial Output Data):} Output line for serial data.
  \end{itemize}
\item
  \textbf{Reset}

  \begin{itemize}
  \tightlist
  \item
    \textbf{RESET IN:} When low, resets the microprocessor, clearing the
    program counter and registers.
  \item
    \textbf{RESET OUT:} Indicates that the microprocessor is being
    reset.
  \end{itemize}
\item
  \textbf{DMA (Direct Memory Access)}

  \begin{itemize}
  \tightlist
  \item
    \textbf{HOLD:} Input from a DMA device to request control of buses.
  \item
    \textbf{HLDA:} Acknowledge signal, indicating the 8085 has
    relinquished control of buses.
  \end{itemize}
\end{enumerate}

\subsubsection{Q3a: Draw clock and reset circuit of 8051
microcontroller}\label{q3a-draw-clock-and-reset-circuit-of-8051-microcontroller}

Absolutely! Let's break down the clock and reset circuit of a typical
8051 microcontroller.

\textbf{Clock Circuit}

\begin{itemize}
\item
  \textbf{Crystal Oscillator:} The foundation of the timing for the 8051
  is a crystal oscillator connected to the XTAL1 and XTAL2 pins of the
  microcontroller.

  \begin{itemize}
  \tightlist
  \item
    The crystal, along with small capacitors (usually in the 20-30pF
    range), provides a stable and precise clock frequency.
  \item
    Common crystal frequencies for 8051 systems are 11.0592 MHz or 12
    MHz.
  \end{itemize}
\item
  \textbf{Internal Clock Generation:} The 8051 has an internal clock
  generator that takes the external crystal oscillator's signal and
  divides it down. This ensures that the microcontroller and its various
  components operate at the correct internal clock speed.
\end{itemize}

\textbf{Diagram}

\begin{figure}
\centering
\pandocbounded{\includegraphics[keepaspectratio,alt={clock-circuit}]{./imgs/8051-Clock-Circuit.png}}
\caption{clock-circuit}
\end{figure}

\textbf{Reset Circuit}

\begin{itemize}
\item
  \textbf{RC Network:} A simple resistor-capacitor (RC) network is often
  used for the reset circuit.

  \begin{itemize}
  \tightlist
  \item
    When power is first applied, the capacitor begins to charge. This
    holds the RESET pin low for a short period, guaranteeing the 8051
    starts in a known state.
  \item
    Once the capacitor voltage reaches a threshold, the RESET pin goes
    high, allowing the microcontroller to begin executing code.
  \end{itemize}
\item
  \textbf{Supervisory Circuit (Optional):} For more robust reset
  control, a dedicated supervisory circuit/IC provides more precise
  monitoring of the power supply voltage. This ensures reliable resets
  if the power supply fluctuates or becomes unstable.
\end{itemize}

\textbf{Diagram}

\begin{figure}
\centering
\pandocbounded{\includegraphics[keepaspectratio,alt={reset-circuit}]{./imgs/8051-reset-circuit.png}}
\caption{reset-circuit}
\end{figure}

\textbf{Explanation}

\begin{enumerate}
\tightlist
\item
  \textbf{Power On:} When the system powers on, the capacitor of the
  reset circuit is initially discharged, holding the RESET pin low.
\item
  \textbf{Reset:} This low level on the RESET pin forces the 8051
  microcontroller into a reset state. Internal registers are cleared,
  and the Program Counter begins at address 0000H.
\item
  \textbf{Capacitor Charging:} The capacitor in the reset circuit starts
  charging through the resistor.
\item
  \textbf{Reset Released:} Once the capacitor charges beyond the RESET
  pin's threshold voltage, the pin goes high. The 8051 starts executing
  code from the beginning of its program memory.
\item
  \textbf{Clock Stabilization:} While the reset circuit is active, the
  crystal oscillator begins to oscillate and the clock stabilizes. The
  8051's internal clock generator uses this signal to provide the
  necessary timing for the microcontroller's operation.
\end{enumerate}

\textbf{Key Points}

\begin{itemize}
\tightlist
\item
  The clock and reset circuits are essential for the correct
  initialization and operation of an 8051 microcontroller system.
\item
  Simple and inexpensive reset circuits can be designed using just a
  capacitor and resistor.
\item
  Supervisory circuits offer improved power monitoring and enhanced
  reset reliability.
\end{itemize}

\subsubsection{Q3b: Explain internal RAM of
8051}\label{q3b-explain-internal-ram-of-8051}

\textbf{Internal RAM Organization}

The 8051 family of microcontrollers typically includes 128 bytes of
internal RAM, although some derivatives like the 8052 offer an extended
256 bytes. This internal RAM is organized into several distinct
sections:

\begin{enumerate}
\item
  \textbf{Register Banks (00H - 1FH):}

  \begin{itemize}
  \tightlist
  \item
    Four banks of eight general-purpose registers (R0-R7).
  \item
    Each bank can be selected using two bits in the Program Status Word
    (PSW) register.
  \item
    Used for storing temporary data and intermediate results during
    calculations.
  \end{itemize}
\item
  \textbf{Bit-Addressable Area (20H to 2FH):}

  \begin{itemize}
  \tightlist
  \item
    16 bytes of RAM where each bit can be individually addressed (128
    individual bits in total).
  \item
    Useful for storing single-bit variables (like flags or control
    signals).
  \end{itemize}
\item
  \textbf{General Purpose RAM (30H - 7FH):}

  \begin{itemize}
  \tightlist
  \item
    The remaining 80 bytes of general-purpose RAM.
  \item
    Used for variable storage, temporary data, and even as a small stack
    if needed.
  \end{itemize}
\end{enumerate}

\textbf{Key Points}

\begin{itemize}
\tightlist
\item
  \textbf{Speed:} Internal RAM is extremely fast to access compared to
  external RAM, as it's located directly on the microcontroller chip.
\item
  \textbf{Limited Size:} The internal RAM in 8051 is limited. Programs
  with larger data requirements often need external RAM.
\item
  \textbf{Flexibility:} The bit-addressable area provides fine-grained
  control over individual bits, ideal for control and status flags.
\end{itemize}

\textbf{How Internal RAM Is Used}

\begin{itemize}
\tightlist
\item
  \textbf{Arithmetic and Logical Operations}: The register banks are
  heavily used by the ALU for arithmetic and logical operations.
\item
  \textbf{Temporary Storage:} All sections of the internal RAM can be
  used for temporarily storing data during calculation or program
  execution.
\item
  \textbf{Stack:} Although the 8051 has a dedicated hardware stack, the
  general-purpose RAM can also be used as a stack area in constrained
  situations.
\item
  \textbf{Flags and Control:} The bit-addressable area often houses
  individual control flags and status bits for the 8051 or its
  peripherals.
\end{itemize}

\textbf{Example}

\begin{verbatim}
MOV R1, \#50H  ; Move the value 50H into register R1
ADD A, R1     ; Add the value in R1 to the accumulator
MOV 35H, A    ; Store the result in general{-purpose RAM location 35H}
SETB PSW.2    ; Set bit 2 (Carry flag) in the Program Status Word
\end{verbatim}

\subsubsection{Q3c: Explain block diagram of
8051}\label{q3c-explain-block-diagram-of-8051}

\textbf{Detailed Block Diagram Breakdown}

\begin{figure}
\centering
\pandocbounded{\includegraphics[keepaspectratio,alt={Simplified 8051 Block Diagram}]{./imgs/a9f6d83e89bf83f5.jpg}}
\caption{Simplified 8051 Block Diagram}
\end{figure}

\textbf{Central Processing Unit (CPU)}

\begin{itemize}
\tightlist
\item
  \textbf{Accumulator:} A core 8-bit register involved in most
  arithmetic and logical operations with the ALU.
\item
  \textbf{B Register:} A temporary register that can be used for
  multiplication, division, or as extra data storage.
\item
  \textbf{Program Status Word (PSW):} Holds important status flags like
  Carry, Overflow, Parity, and register bank selection bits.
\item
  \textbf{Stack Pointer (SP):} An 8-bit register pointing to the current
  top of the stack in RAM.
\item
  \textbf{Program Counter (PC):} 16-bit register keeping track of the
  memory address of the next instruction to be fetched.
\item
  \textbf{Instruction Register:} Holds the currently fetched
  instruction.
\item
  \textbf{Instruction Decoder:} Decodes the instruction in the
  instruction register, generating control signals that orchestrate what
  happens within the other blocks.
\item
  \textbf{Timing and Control:} Governs the fetch-decode-execute cycle of
  the CPU, synchronizes actions, and interfaces with external signals.
\end{itemize}

\textbf{Memory}

\begin{itemize}
\tightlist
\item
  \textbf{Internal RAM (128 bytes):}

  \begin{itemize}
  \tightlist
  \item
    \textbf{Register Banks 0-3:} Four sets of eight 8-bit
    general-purpose registers (R0-R7).
  \item
    \textbf{Bit-addressable area (20h-2Fh):} 16 bytes with individually
    addressable bits.
  \item
    \textbf{General-purpose area (30h-7Fh):} Remaining 80 bytes of RAM
    for data and variables.
  \end{itemize}
\item
  \textbf{Internal ROM (typically 4KB):} Non-volatile memory for storing
  the 8051's program code.
\end{itemize}

\textbf{Input/Output (I/O)}

\begin{itemize}
\tightlist
\item
  \textbf{Ports 0 through 3 (P0 - P3):} Four bi-directional 8-bit I/O
  ports that can be individually configured as input or output.
\end{itemize}

\textbf{Timers/Counters}

\begin{itemize}
\tightlist
\item
  \textbf{Timer/Counter 0 and 1 (T0, T1):} 16-bit timers/counters with
  various modes of operation (counting external events, generating time
  intervals, etc.).
\end{itemize}

\textbf{Serial Port (UART)}

\begin{itemize}
\tightlist
\item
  \textbf{TXD: Transmit Data:} The line used for sending serial data out
  from the 8051.
\item
  \textbf{RXD: Receive Data:} The line used for receiving serial data
  into the 8051.
\item
  \textbf{SBUF: Serial Data Buffer:} A temporary register for holding
  data during serial transmission or reception.
\end{itemize}

\textbf{Interrupts}

\begin{itemize}
\tightlist
\item
  \textbf{External Interrupts (INT0, INT1):} Triggered by signals on
  external pins.
\item
  \textbf{Timer Interrupts (TF0, TF1):} Triggered when the
  timers/counters overflow or reach a specific value.
\item
  \textbf{Serial Interrupt (RI, TI):} Triggered by events related to
  UART transmission/reception.
\item
  \textbf{Interrupt Control Logic:} Handles enabling/disabling,
  prioritizing, and managing these interrupts.
\end{itemize}

\textbf{Additional Notes}

\begin{itemize}
\tightlist
\item
  \textbf{Bus Structure:} Notice the internal data bus that connects the
  CPU, memory, and I/O blocks. Instructions and data flow along this bus
  under the control of the CPU.
\item
  \textbf{Reset:} The RESET input initializes the 8051, setting
  registers and the Program Counter to their starting states.
\item
  \textbf{Oscillator:} The XTAL1 and XTAL2 inputs are for connecting the
  crystal and other components that form the clock circuit for the
  microcontroller.
\end{itemize}

\subsubsection{Q3a: Explain different timer modes of 8051
microcontroller}\label{q3a-explain-different-timer-modes-of-8051-microcontroller}

\textbf{Mode 0: 13-Bit Timer}

\begin{itemize}
\tightlist
\item
  \textbf{Configuration:} The timer register is split into two parts:

  \begin{itemize}
  \tightlist
  \item
    Five high-order bits (THx)
  \item
    Eight low-order bits (TLx), with the top 3 bits of TLx written as
    zeroes.
  \end{itemize}
\item
  \textbf{Operation:} The 5 bits of TLx are automatically incremented.
  When TLx overflows, it increments THx. This forms a 13-bit timer.
\item
  \textbf{Use Cases:} Often used for event counting or generating baud
  rates in serial communication, particularly when interfacing with
  legacy systems.
\end{itemize}

\textbf{Mode 1: 16-Bit Timer}

\begin{itemize}
\tightlist
\item
  \textbf{Configuration:} The full 16-bits of the Timer register (THx
  and TLx) function as a single timer unit.
\item
  \textbf{Operation:} Each clock pulse increments the entire register.
\item
  \textbf{Use Cases} General-purpose time delays, long interval
  measurements, anything requiring 16-bit precision timing.
\end{itemize}

\textbf{Mode 2: 8-Bit Auto-Reload Timer}

\begin{itemize}
\tightlist
\item
  \textbf{Configuration:}

  \begin{itemize}
  \tightlist
  \item
    THx holds a fixed reload value.
  \item
    TLx operates as the 8-bit timer.
  \end{itemize}
\item
  \textbf{Operation:}

  \begin{itemize}
  \tightlist
  \item
    TLx counts up. When it overflows, it's automatically reloaded with
    the value stored in THx.
  \item
    This creates a recurring time interval.
  \end{itemize}
\item
  \textbf{Use Cases:} Generating fixed, predictable time delays or
  timing periodic events.
\end{itemize}

\textbf{Mode 3: Split 8-bit Timers}

\begin{itemize}
\tightlist
\item
  \textbf{Configuration:}

  \begin{itemize}
  \tightlist
  \item
    Timer 0 is split into two independent 8-bit timers/counters: TL0 and
    TH0.
  \item
    Timer 1 remains as a 16-bit timer if needed.
  \end{itemize}
\item
  \textbf{Operation:}

  \begin{itemize}
  \tightlist
  \item
    TL0 and TH0 function as two separate timers, often with TL0 used as
    a timer and TH0 used as a counter.
  \end{itemize}
\item
  \textbf{Use Cases:}

  \begin{itemize}
  \tightlist
  \item
    Situations requiring two independent timers
  \item
    Generating baud rates (TL0) while counting external events (TH0)
  \end{itemize}
\end{itemize}

\textbf{Key Control Registers}

\begin{itemize}
\tightlist
\item
  \textbf{TMOD (Timer Mode):} This register selects the operating mode
  for Timer 0 and Timer 1.
\item
  \textbf{TCON (Timer Control):} Contains flags and start/stop control
  bits for the timers.
\end{itemize}

\textbf{How to Select a Mode}

Mode selection depends on:

\begin{itemize}
\tightlist
\item
  \textbf{Timing Precision:} 16-bit vs.~8-bit
\item
  \textbf{Recurring Intervals:} Auto-reload mode vs.~manual restart.
\item
  \textbf{Number of Timers Needed:} Split timer mode provides two
  independent 8-bit timers if needed within Timer 0.
\end{itemize}

\subsubsection{Q3b: Explain function of DPTR and
PC}\label{q3b-explain-function-of-dptr-and-pc}

\textbf{DPTR (Data Pointer)}

\begin{itemize}
\tightlist
\item
  \textbf{Type:} 16-bit register, often visualized as two conjoined
  8-bit registers DPH (Higher Byte) and DPL (Lower Byte).
\item
  \textbf{Primary Function:} Pointing to memory locations within the
  8051's data memory space. This includes both the internal and external
  RAM.
\end{itemize}

\textbf{Key Uses of DPTR}

\begin{enumerate}
\item
  \textbf{Accessing External Memory:} The 8051 can access up to 64KB of
  external data memory. The DPTR holds the address when instructions
  like these are used:

  \begin{itemize}
  \tightlist
  \item
    \texttt{MOVX\ A,@DPTR} (Read a byte from external RAM pointed to by
    DPTR)
  \item
    \texttt{MOVX\ @DPTR,A} (Write a byte to external RAM pointed to by
    DPTR)
  \end{itemize}
\item
  \textbf{Lookup Tables and Buffers:} DPTR can be used to conveniently
  access data stored within tables or buffers located in memory. You can
  change the value in DPTR to point to different areas of these data
  structures.
\item
  \textbf{Passing Parameters:} If a function or subroutine requires data
  stored in memory, the memory address can be passed using the DPTR
  register.
\end{enumerate}

\textbf{PC (Program Counter)}

\begin{itemize}
\tightlist
\item
  \textbf{Type:} 16-bit Register
\item
  \textbf{Primary Function:} Keeping track of the memory address of the
  next instruction to be executed by the 8051's CPU.
\end{itemize}

\textbf{How PC Works}

\begin{enumerate}
\item
  \textbf{Fetch:} The PC provides the address from which the next
  instruction is fetched from program memory (ROM).
\item
  \textbf{Increment:} After the instruction is fetched, the PC is
  automatically incremented to point to the next sequential instruction,
  ensuring instructions are normally executed in order.
\item
  \textbf{Modifiying Program Flow:} Instructions like jumps, calls, and
  returns alter the value in the PC, changing the execution flow of the
  program.
\end{enumerate}

\textbf{Key Points}

\begin{itemize}
\tightlist
\item
  \textbf{DPTR is primarily for data; PC is for code:} DPTR is designed
  for accessing data in memory, while PC is for tracking instructions in
  the program memory.
\item
  \textbf{PC mostly automatic; DPTR is programmer-controlled:} The PC
  increments automatically with regular program execution, while the
  programmer generally controls and sets the value in DPTR.
\end{itemize}

\subsubsection{Q3c: Explain interrupts of 8051
microcontroller}\label{q3c-explain-interrupts-of-8051-microcontroller}

\textbf{What is an Interrupt?}

\begin{itemize}
\tightlist
\item
  An interrupt is an event that temporarily suspends the normal
  execution of a program and forces the 8051 to execute a special
  routine called an Interrupt Service Routine (ISR).
\item
  Interrupts allow the microcontroller to respond quickly to important
  events (e.g., button presses, timer overflow, data received) without
  needing to constantly poll for them in the main code.
\end{itemize}

\textbf{Types of Interrupts in the 8051}

\begin{enumerate}
\item
  \textbf{External Interrupts:}

  \begin{itemize}
  \tightlist
  \item
    \textbf{INT0 (Pin P3.2):} Triggered by a low-to-high transition on
    the INT0 pin.
  \item
    \textbf{INT1 (Pin P3.3):} Triggered by a low-to-high transition on
    the INT1 pin.
  \end{itemize}
\item
  \textbf{Timer Interrupts:}

  \begin{itemize}
  \tightlist
  \item
    \textbf{TF0 (Timer 0 Overflow):} Triggered when Timer 0 overflows.
  \item
    \textbf{TF1 (Timer 1 Overflow):} Triggered when Timer 1 overflows.
  \end{itemize}
\item
  \textbf{Serial Interrupt:}

  \begin{itemize}
  \tightlist
  \item
    \textbf{RI/TI (Receive Interrupt/Transmit Interrupt):} Triggered
    when the serial port finishes receiving a byte (RI) or transmitting
    a byte (TI).
  \end{itemize}
\end{enumerate}

\textbf{Interrupt Process}

\begin{enumerate}
\tightlist
\item
  \textbf{Trigger:} An interrupt source (external pin, timer overflow,
  etc.) is triggered.
\item
  \textbf{Completion of Current Instruction:} The 8051 completes
  executing its current instruction.
\item
  \textbf{Saving State:} The microcontroller automatically pushes the
  current Program Counter (PC) onto the stack.
\item
  \textbf{Jump to ISR:} The 8051 jumps to the pre-determined memory
  address of the corresponding Interrupt Service Routine (ISR).
\item
  \textbf{ISR Execution:} The ISR code executes, handling the event that
  triggered the interrupt.
\item
  \textbf{Returning:} After the ISR completes, a `RETI' instruction pops
  the PC value from the stack, resuming the original program flow.
\end{enumerate}

\textbf{Interrupt Control Registers}

\begin{itemize}
\tightlist
\item
  \textbf{IE (Interrupt Enable):} Enables or disables specific
  interrupts globally and individually within the system.
\item
  \textbf{IP (Interrupt Priority):} Assigns priority levels to each
  interrupt source. If multiple interrupts occur simultaneously, the one
  with higher priority is serviced first.
\end{itemize}

\textbf{Key Points}

\begin{itemize}
\tightlist
\item
  \textbf{Priority:} The 8051 has a fixed interrupt priority structure
  (e.g., INT0 has the highest priority).
\item
  \textbf{Masking:} Interrupts can be turned on or off selectively using
  the IE register.
\item
  \textbf{Nesting:} Interrupts can potentially interrupt other
  interrupts, depending on their priority.
\end{itemize}

\textbf{Example}

Imagine an 8051 system monitoring a sensor. A timer interrupt might
trigger periodically to read the sensor value, while an external
interrupt could signal a critical threshold being exceeded, requiring
immediate action.

\subsubsection{Q4a: Explain data transfer instruction with example for
8051}\label{q4a-explain-data-transfer-instruction-with-example-for-8051}

\textbf{Data Transfer Instructions in the 8051}

These instructions move data between various registers, internal RAM,
external RAM, and I/O ports of the 8051. Here's a breakdown of the key
types:

\textbf{1. Register-to-Register Transfers}

\begin{itemize}
\tightlist
\item
  \textbf{MOV instruction:} The most versatile data transfer
  instruction.
\item
  \textbf{Examples:}

  \begin{itemize}
  \tightlist
  \item
    \texttt{MOV\ A,\ R5} - Copies the contents of register R5 into the
    accumulator.
  \item
    \texttt{MOV\ R2,\ \#45H} - Loads the immediate value 45H into
    register R2.
  \item
    \texttt{MOV\ P1,\ A} - Copies the accumulator's contents to port P1
    (output).
  \end{itemize}
\end{itemize}

\textbf{2. Direct Addressing}

\begin{itemize}
\tightlist
\item
  \textbf{MOV instruction with `direct' addressing mode:} Accesses
  internal RAM or Special Function Registers (SFRs).
\item
  \textbf{Examples:}

  \begin{itemize}
  \tightlist
  \item
    \texttt{MOV\ 50H,\ A} - Stores the value in the accumulator to
    internal RAM location 50H.
  \item
    \texttt{MOV\ ACC,\ 55H} - Loads the byte from internal RAM location
    55H into the accumulator.
  \item
    \texttt{MOV\ TMOD,\ \#01H} - Sets Timer 0 into mode 1.
  \end{itemize}
\end{itemize}

\textbf{3. Indirect Addressing}

\begin{itemize}
\tightlist
\item
  \textbf{MOV instruction using registers as pointers:} The register
  holds the address of the data.
\item
  \textbf{Examples:}

  \begin{itemize}
  \tightlist
  \item
    \texttt{MOV\ A,\ @R0} - Copies the byte pointed to by register R0
    into the accumulator.
  \item
    \texttt{MOV\ @R1,\ 33H} - Stores the value 33H at the address
    pointed to by R1.
  \end{itemize}
\end{itemize}

\textbf{4. External Memory Transfers}

\begin{itemize}
\tightlist
\item
  \textbf{MOVX instruction:} Used to access external RAM.
\item
  \textbf{Examples:}

  \begin{itemize}
  \tightlist
  \item
    \texttt{MOVX\ A,\ @DPTR} -- Copies a byte from external RAM (address
    in DPTR) into the accumulator.
  \item
    \texttt{MOVX\ @DPTR,\ A} -- Copies the contents of the accumulator
    into external RAM (address in DPTR).
  \end{itemize}
\end{itemize}

\textbf{5. Special Data Transfers}

\begin{itemize}
\tightlist
\item
  \textbf{PUSH instruction:} Pushes data onto the stack (internal RAM).
\item
  \textbf{POP instruction:} Pops data off the stack.
\end{itemize}

\textbf{Example: Data Sorting Routine}

Consider a simple routine to sort three numbers stored at internal RAM
locations 40H, 41H, and 42H:

\begin{verbatim}
COMPARE: MOV A, 40H    ; Load the first number
         MOVC A, @A+DPTR ; Load the second number (assuming DPTR points to RAM)
         JC  SWAP        ; Jump to SWAP if the first is greater than the second

         MOV A, 41H     ; Load the second number
         MOVC A, @A+DPTR ; Load the third number
         JC  SWAP        ; Jump to SWAP if the second is greater than the third
         ; ... (rest of your code)

SWAP:    ; ... (Code to swap values)
\end{verbatim}

\subsubsection{Q4b: List and explain different addressing modes of 8051
microcontroller}\label{q4b-list-and-explain-different-addressing-modes-of-8051-microcontroller}

\textbf{Key Addressing Modes in the 8051}

\begin{enumerate}
\item
  \textbf{Register Addressing}

  \begin{itemize}
  \tightlist
  \item
    \textbf{How it Works:} The operand of the instruction directly
    specifies one of the 8051's registers (A, B, R0-R7).
  \item
    \textbf{Example:} \texttt{MOV\ A,\ R2} (Copy the contents of R2 into
    the accumulator)
  \item
    \textbf{Fast and Efficient:} No additional memory accesses are
    needed.
  \end{itemize}
\item
  \textbf{Direct Addressing}

  \begin{itemize}
  \tightlist
  \item
    \textbf{How it Works:} The instruction contains an 8-bit address
    that directly points to a location in the internal RAM or Special
    Function Registers (SFRs).
  \item
    \textbf{Example:} \texttt{MOV\ 45H,\ A} (Store the value in the
    accumulator into internal RAM location 45H)
  \item
    \textbf{Accesses only first 256 bytes:} Limited to accessing the
    lower portion of internal RAM and SFRs.
  \end{itemize}
\item
  \textbf{Indirect Addressing}

  \begin{itemize}
  \tightlist
  \item
    \textbf{How it Works:} The instruction specifies a register (R0 or
    R1) that holds the memory address of where the data actually
    resides.
  \item
    \textbf{Example:} \texttt{MOV\ A,\ @R0} (Copy the byte pointed to by
    the address in R0 into the accumulator).
  \item
    \textbf{Flexibility:} Allows dynamic calculation of data locations.
  \end{itemize}
\item
  \textbf{Immediate Addressing}

  \begin{itemize}
  \tightlist
  \item
    \textbf{How it Works:} The data to be used is embedded directly
    within the instruction itself. Preceded by the `\#' symbol.
  \item
    \textbf{Example:} \texttt{MOV\ A,\ \#60H} (Load the value 60H into
    the accumulator).
  \item
    \textbf{Convenient for constants:} Useful for loading fixed values.
  \end{itemize}
\item
  \textbf{Base Relative Addressing (with DPTR)}

  \begin{itemize}
  \tightlist
  \item
    \textbf{How it Works}: Used for accessing external RAM. The Data
    Pointer (DPTR) provides a 16-bit base address, and an 8-bit offset
    within the instruction specifies a location relative to that base.
  \item
    \textbf{Example}: \texttt{MOVX\ A,\ @DPTR} (Copy byte from external
    RAM pointed to by DPTR into the accumulator)
  \item
    \textbf{Expanded Memory:} Access up to 64KB of external memory
  \end{itemize}
\end{enumerate}

\textbf{Important Notes}

\begin{itemize}
\tightlist
\item
  Different instructions support different combinations of addressing
  modes.
\item
  The chosen addressing mode affects the instruction's length and
  execution time.
\item
  Choosing the right addressing mode is a balance between flexibility
  and efficiency.
\end{itemize}

\subsubsection{Q4c: Write a program to copy block of 8 data starting
from location 100h to
200h}\label{q4c-write-a-program-to-copy-block-of-8-data-starting-from-location-100h-to-200h}

Here's an assembly program for the 8051 microcontroller to copy a block
of 8 bytes of data from starting location 100H to destination location
200H:

\begin{verbatim}
ORG 0000H  ; Program starts at memory location 0000H

; Initialization
MOV DPTR, \#100H   ; Set DPTR to point to the source block (100H)
MOV R0, \#200H     ; Set R0 to point to the destination block (200H)
MOV R1, \#08H      ; Set R1 as the loop counter (8 bytes to copy)

COPY\_LOOP:
MOVX A, @DPTR   ; Read a byte from the source using DPTR
MOVX @R0, A     ; Write the byte to the destination using R0
INC DPTR        ; Increment DPTR to point to the next source byte
INC R0          ; Increment R0 to point to the next destination byte
DJNZ R1, COPY\_LOOP ; Decrement R1 and jump if not zero

; End of Program (You can add more code here or an infinite loop)
END
\end{verbatim}

\textbf{Explanation}

\begin{enumerate}
\item
  \textbf{ORG 0000H:} This directive tells the assembler to place the
  code starting from memory location 0000H.
\item
  \textbf{Initialization:}

  \begin{itemize}
  \tightlist
  \item
    We load the Data Pointer (DPTR) with the starting address of the
    source block (100H).
  \item
    Register R0 is loaded with the starting address of the destination
    block (200H).
  \item
    Register R1 is initialized to 8, which is the number of bytes we
    want to copy.
  \end{itemize}
\item
  \textbf{COPY\_LOOP:}

  \begin{itemize}
  \tightlist
  \item
    \texttt{MOVX\ A,\ @DPTR}: Reads a byte from external RAM pointed to
    by DPTR and stores it in the accumulator.
  \item
    \texttt{MOVX\ @R0,\ A}: Writes the byte from the accumulator to
    external RAM pointed to by R0.
  \item
    \texttt{INC\ DPTR}, \texttt{INC\ R0}: Increment both DPTR and R0 to
    move to the next memory locations.
  \item
    \texttt{DJNZ\ R1,\ COPY\_LOOP}: Decrement R1 and jump back to the
    `COPY\_LOOP' label if R1 is not zero (meaning we haven't copied all
    8 bytes yet).
  \end{itemize}
\item
  \textbf{END:} Signifies the end of the assembly program.
\end{enumerate}

\textbf{Key Points}

\begin{itemize}
\tightlist
\item
  This assumes you have external RAM where you are storing the data.
\item
  You may need to adapt the addresses (100H and 200H) if your data is
  stored elsewhere.
\end{itemize}

\subsubsection{Q4a: Write a program to add two bytes of data and store
result in R0
register}\label{q4a-write-a-program-to-add-two-bytes-of-data-and-store-result-in-r0-register}

Here's the 8051 assembly code to add two bytes of data and store the
result in register R0:

\begin{verbatim}
; Data initialization {- you might load these from memory in a real program}
MOV A, \#56H   ; Load the first byte of data into the accumulator
MOV B, \#23H   ; Load the second byte of data into register B

; Addition
ADD A, B       ; Add the value in register B to the accumulator
MOV R0, A     ; Store the result (which is now in the accumulator) into R0

; End of program (you might do something with the result or add an infinite loop here)
END
\end{verbatim}

\textbf{Explanation}

\begin{enumerate}
\item
  \textbf{Data Initialization:}

  \begin{itemize}
  \tightlist
  \item
    \texttt{MOV\ A,\ \#56H}: Loads the immediate value 56H (hexadecimal)
    into the accumulator (A register).
  \item
    \texttt{MOV\ B,\ \#23H}: Loads the immediate value 23H into register
    B.
  \end{itemize}
\item
  \textbf{Addition:}

  \begin{itemize}
  \tightlist
  \item
    \texttt{ADD\ A,\ B}: Adds the value in register B to the value in
    the accumulator. The result remains in the accumulator.
  \end{itemize}
\item
  \textbf{Storing the Result:}

  \begin{itemize}
  \tightlist
  \item
    \texttt{MOV\ R0,\ A}: Moves the value from the accumulator (which
    holds the sum) into register R0.
  \end{itemize}
\end{enumerate}

\textbf{Important Points}

\begin{itemize}
\tightlist
\item
  You can replace the \texttt{MOV} instructions with ways to get data
  from other sources (memory, user input, etc.).
\item
  Make sure that the sum of your two data bytes can fit into 8 bits to
  avoid overflow.
\end{itemize}

\subsubsection{Q4b: Explain indexed addressing mode with
example}\label{q4b-explain-indexed-addressing-mode-with-example}

\textbf{What is Indexed Addressing Mode?}

\begin{itemize}
\tightlist
\item
  \textbf{Combination:} Indexed addressing mode combines the use of a
  base register and an offset to calculate the effective address of
  data.

  \begin{itemize}
  \tightlist
  \item
    \textbf{Base Register:} Can be either the Data Pointer (DPTR) or the
    Program Counter (PC).
  \item
    \textbf{Offset:} The accumulator (A) holds the offset value.
  \end{itemize}
\item
  \textbf{Purpose:} Primarily used to access elements within data
  structures like arrays or lookup tables located in program memory
  (ROM).
\end{itemize}

\textbf{How Indexed Addressing Works}

\begin{enumerate}
\tightlist
\item
  The base register (DPTR or PC) is loaded with the starting address of
  the data structure (e.g., the array).
\item
  The accumulator (A) is loaded with an offset indicating the position
  of a specific element relative to the base address.
\item
  The effective address is calculated by adding the contents of the base
  register and the accumulator.
\item
  The instruction accesses the data at this calculated effective
  address.
\end{enumerate}

\textbf{Example: Accessing an Array Element}

Suppose you have an array of bytes stored in program memory starting at
address 2000H. Here's how to access the 5th element using indexed
addressing:

\begin{verbatim}
MOV DPTR, \#2000H  ; Load DPTR with the base address of the array
MOV A, \#04H       ; Load offset (index 4, since arrays are zero{-based)}

; Accessing the 5th element (assuming you want to load it into the accumulator)
MOVC A, @A+DPTR   ; Calculate effective address and fetch the data
\end{verbatim}

\textbf{Explanation of the Example}

\begin{itemize}
\tightlist
\item
  \textbf{MOV DPTR, \#2000H:} Sets the DPTR as the base register
  pointing to the beginning of the array.
\item
  \textbf{MOV A, \#04H:} Loads accumulator with offset 4, indicating we
  want the 5th element (remember: zero-based indexing).
\item
  \textbf{MOVC A, @A+DPTR:}

  \begin{itemize}
  \tightlist
  \item
    The 8051 adds the offset in A (4) to the base address in DPTR
    (2000H), resulting in the effective address 2004H.
  \item
    The `MOVC' instruction fetches the byte from program memory location
    2004H and loads it into the accumulator.
  \end{itemize}
\end{itemize}

\textbf{Key Points}

\begin{itemize}
\tightlist
\item
  Indexed addressing requires ROM access, so the source operand can only
  be program memory.
\item
  It makes sequentially accessing elements within arrays or tables
  convenient.
\end{itemize}

\subsubsection{Q4c: Explain stack operation of 8051 microcontroller,
PUSH and POP
instruction}\label{q4c-explain-stack-operation-of-8051-microcontroller-push-and-pop-instruction}

\textbf{The Stack in the 8051}

\begin{itemize}
\tightlist
\item
  \textbf{Purpose:} A Last-In, First-Out (LIFO) data structure residing
  in the internal RAM.
\item
  \textbf{Stack Pointer (SP):} A dedicated 8-bit register that always
  points to the current top of the stack.
\item
  \textbf{Growth:} The 8051 stack grows downward in memory. The SP is
  decremented when data is pushed, and incremented when data is popped.
\end{itemize}

\textbf{PUSH Instruction}

\begin{enumerate}
\tightlist
\item
  \textbf{Decrement Stack Pointer:} The SP is decremented by one.
\item
  \textbf{Write Data:} The byte to be pushed is written to the internal
  RAM location now pointed to by the SP.
\end{enumerate}

\textbf{Example:}

\begin{verbatim}
MOV R5, \#37H  ; Load the value 37H into register R5
PUSH R5       ; Push the contents of R5 onto the stack
\end{verbatim}

\textbf{POP Instruction}

\begin{enumerate}
\tightlist
\item
  \textbf{Read Data:} The byte pointed to by the SP is read from
  internal RAM.
\item
  \textbf{Increment Stack Pointer:} The SP is incremented by one.
\end{enumerate}

\textbf{Example:}

\begin{verbatim}
POP R6    ; Pop the top value from the stack into register R6
\end{verbatim}

\textbf{Common Uses of the Stack}

\begin{itemize}
\item
  \textbf{Temporary Storage:} Storing the contents of registers during
  calculations when there aren't enough registers available.
\item
  \textbf{Subroutine Calls:} When a subroutine (function) is called
  using the `CALL' instruction, the return address (next instruction
  after the call) is automatically pushed onto the stack. The `RET'
  instruction pops this return address, so execution continues
  correctly.
\item
  \textbf{Interrupt Handling:} When an interrupt occurs, the 8051
  automatically pushes the Program Counter (PC) onto the stack, allowing
  seamless return to the interrupted code after the interrupt service
  routine.
\end{itemize}

\textbf{Important Points}

\begin{itemize}
\tightlist
\item
  \textbf{Stack Size:} The 8051's internal RAM for the stack is limited;
  it's crucial to prevent stack overflow.
\item
  \textbf{Initialization:} The SP is initialized to 07H when the 8051
  resets; your code often needs to set it to a custom location.
\end{itemize}

\textbf{Example: Swapping Two Numbers}

\begin{verbatim}
MOV SP, \#70H   ; Initialize Stack Pointer (assuming safe RAM space)

MOV A, \#25H    ; Load the first number into the accumulator
PUSH A         ; Push the first number onto the stack

MOV A, \#30H    ; Load the second number into the accumulator
PUSH A         ; Push the second number onto the stack

POP B         ; Pop the top (second) number into register B
POP A          ; Pop the original (first) number into the accumulator
\end{verbatim}

\subsubsection{Q5a: Explain branching instruction with
example}\label{q5a-explain-branching-instruction-with-example}

\textbf{What are Branching Instructions?}

Branching instructions, often also called jump instructions, allow you
to alter the normal sequential flow of program execution. They cause the
Program Counter (PC) to jump to a different memory location, breaking
the usual `execute the next instruction' pattern.

\textbf{Types of Branching Instructions}

\begin{enumerate}
\item
  \textbf{Unconditional Branching}

  \begin{itemize}
  \tightlist
  \item
    \textbf{LJMP (Long Jump):} Jumps to the specified 16-bit address.

    \begin{itemize}
    \tightlist
    \item
      \emph{Example:} \texttt{LJMP\ 2050H} (jumps to memory location
      2050H)
    \end{itemize}
  \end{itemize}
\item
  \textbf{Conditional Branching}

  \begin{itemize}
  \tightlist
  \item
    \textbf{These depend on the status of flags (Carry, Parity,
    Overflow, etc.) set by previous operations}
  \item
    \textbf{Examples:}

    \begin{itemize}
    \tightlist
    \item
      \texttt{JC\ LABEL} (Jump if Carry flag is set)
    \item
      \texttt{JNC\ LABEL} (Jump if Carry flag is not set)
    \item
      \texttt{JZ\ LABEL} (Jump if Zero flag is set)
    \item
      \texttt{JNZ\ LABEL} (Jump if Zero flag is not set)
    \end{itemize}
  \end{itemize}
\item
  \textbf{Short Jump (Relative Jump)}

  \begin{itemize}
  \tightlist
  \item
    \textbf{SJMP:} Jumps to an address within a limited range relative
    (+127 or -128 bytes) to the current instruction.
  \item
    \emph{Example:} \texttt{SJMP\ LOOP\_START} (jumps to a label
    relatively nearby)
  \end{itemize}
\end{enumerate}

\textbf{Example: Conditional Loop}

\begin{verbatim}
MOV R0, \#10    ; Initialize a counter
LOOP:
   ; ... some code here ...
   DJNZ R0, LOOP  ; Decrement and jump if not zero
\end{verbatim}

\textbf{Explanation}

\begin{enumerate}
\tightlist
\item
  The counter register R0 is loaded with 10.
\item
  The code in the \texttt{LOOP} section executes.
\item
  \texttt{DJNZ\ R0,\ LOOP}

  \begin{itemize}
  \tightlist
  \item
    Decrements R0 by one.
  \item
    If the Zero flag is NOT set (R0 is not zero), jumps back to the
    \texttt{LOOP} label.
  \end{itemize}
\end{enumerate}

\textbf{Key Points}

\begin{itemize}
\tightlist
\item
  Branching instructions are core to creating loops, decision structures
  (if-else), and subroutines within programs.
\item
  The destination of a jump can be an explicit address (e.g.,
  \texttt{LJMP\ 2050H}) or often a label that the assembler translates
  to the correct address.
\end{itemize}

\subsubsection{Q5b: Interface 8 leds with 8051 microcontroller and write
a program to turn on and
off}\label{q5b-interface-8-leds-with-8051-microcontroller-and-write-a-program-to-turn-on-and-off}

\textbf{Hardware Setup}

\begin{enumerate}
\tightlist
\item
  \textbf{8 LEDs:} Choose standard LEDs considering the current
  requirements of the 8051's I/O ports.
\item
  \textbf{Current-Limiting Resistors:} Calculate the appropriate
  resistor values for your specific LEDs to prevent damage (search for
  an online ``LED resistor calculator'' if needed).
\item
  \textbf{8051 Microcontroller:} We'll assume you have an 8051
  development board with an I/O port (e.g., Port 1).
\item
  \textbf{Connections:}

  \begin{itemize}
  \tightlist
  \item
    Connect one leg of each LED to a separate pin on Port 1 of the 8051
    (P1.0 to P1.7).
  \item
    Connect the other leg of each LED through a current-limiting
    resistor to ground.
  \end{itemize}
\end{enumerate}

\textbf{Programming (Assembly)}

Here's a simple 8051 assembly program to repeatedly turn the LEDs on and
then off with a delay:

\begin{verbatim}
ORG 0000H  ; Program starts at address 0000H

MAIN\_LOOP:
    MOV A, \#FFH   ; Set all port pins high (LEDs on)
    MOV P1, A     ; Send data to Port 1
    CALL DELAY    ; Call a delay subroutine

    MOV A, \#00H   ; Set all port pins low (LEDs off)
    MOV P1, A
    CALL DELAY

    SJMP MAIN\_LOOP  ; Jump back to the beginning

DELAY:  ; Simple delay subroutine {- adjust for desired time}
    MOV R0, \#200   ; Adjust these values for timing
    MOV R1, \#150
DLY\_LOOP: DJNZ R1, DLY\_LOOP
          DJNZ R0, DLY\_LOOP
          RET  ; Return from subroutine

END
\end{verbatim}

\textbf{Explanation}

\begin{enumerate}
\item
  \textbf{MAIN\_LOOP:}

  \begin{itemize}
  \tightlist
  \item
    \texttt{MOV\ A,\ \#FFH}: Loads the accumulator with FFH (all bits
    1), which will turn all LEDs on.
  \item
    \texttt{MOV\ P1,\ A}: Sends this value to Port 1.
  \item
    \texttt{CALL\ DELAY}: Calls a subroutine to create a delay.
  \item
    \texttt{MOV\ A,\ \#00H}: Loads the accumulator with 00H (all bits
    0), which will turn all LEDs off.
  \item
    \texttt{SJMP\ MAIN\_LOOP}: Creates an infinite loop.
  \end{itemize}
\item
  \textbf{DELAY Subroutine:}

  \begin{itemize}
  \tightlist
  \item
    Provides a simple software delay using nested loops. You might want
    a more precise timer-based delay in a real application.
  \end{itemize}
\end{enumerate}

\textbf{Key Points:}

\begin{itemize}
\tightlist
\item
  \textbf{Port Choice:} I used Port 1(P1); adapt the code if you connect
  the LEDs to a different port.
\item
  \textbf{LED Polarity:} If your LEDs light up in the opposite manner,
  reverse the logic (use 00H to turn them on and FFH to turn them off)
\item
  \textbf{Delay Adjustment:} Modify the values in the DELAY subroutine
  to change the on and off duration.
\end{itemize}

\subsubsection{Q5c: Interface LCD with 8051 microcontroller and write a
program to display
``welcome''}\label{q5c-interface-lcd-with-8051-microcontroller-and-write-a-program-to-display-welcome}

\textbf{Hardware Setup}

\begin{itemize}
\tightlist
\item
  \textbf{LCD:} We'll assume a standard 16x2 character LCD module with a
  common HD44780 compatible controller.
\item
  \textbf{Connections:}

  \begin{itemize}
  \tightlist
  \item
    \textbf{Data Pins (D0-D7):} Connect to an 8051 I/O Port (e.g., Port
    2)
  \item
    \textbf{Control Pins:}

    \begin{itemize}
    \tightlist
    \item
      RS (Register Select): Connect to an 8051 pin (e.g., P1.0)
    \item
      RW (Read/Write): Connect to an 8051 pin (e.g., P1.1)
    \item
      E (Enable): Connect to an 8051 pin (e.g., P1.2)
    \end{itemize}
  \item
    \textbf{Contrast Adjustment (Vo):} Connect to a potentiometer for
    controlling display contrast.
  \item
    \textbf{Backlight (if present):} Power according to its
    requirements.
  \end{itemize}
\end{itemize}

\textbf{Assembly Programming}

Here's an 8051 assembly program to initialize the LCD and display
``Welcome'':

\begin{verbatim}
ORG 0000H

; Constants
LCD\_PORT EQU P2
RS\_PIN   EQU P1.0
RW\_PIN   EQU P1.1
E\_PIN    EQU P1.2

; {-{-}{-} LCD Initialization {-}{-}{-}}
LCD\_INIT:
    MOV A, \#38H  ; Function set: 8{-bit, 2 lines, 5x7 font}
    CALL LCD\_CMD
    MOV A, \#0CH  ; Display on, cursor off, no blinking
    CALL LCD\_CMD
    MOV A, \#01H  ; Clear display
    CALL LCD\_CMD
    MOV A, \#06H  ; Entry mode: Increment cursor
    CALL LCD\_CMD
    RET

; {-{-}{-} Send Command to LCD  {-}{-}{-}}
LCD\_CMD:
    CLR RS\_PIN
    CLR RW\_PIN
    MOV LCD\_PORT, A
    SETB E\_PIN     ; Pulse Enable
    CLR E\_PIN
    CALL DELAY     ; Small delay (important for LCD)
    RET

; {-{-}{-} Send Data (Character) to LCD {-}{-}{-}}
LCD\_DATA:
    SETB RS\_PIN
    CLR RW\_PIN
    MOV LCD\_PORT, A
    SETB E\_PIN      ; Pulse Enable
    CLR E\_PIN
    CALL DELAY      ; Small delay
    RET

; {-{-}{-} Simple Delay Subroutine {-}{-}{-}}
DELAY:
    MOV R5, \#50    ; Adjust these for approximate delay
    DLOOP: MOV R6, \#200
           DJNZ R6, DLOOP
           DJNZ R5, DLOOP
           RET

; {-{-} Main Program {-}{-}}
MAIN:
    CALL LCD\_INIT

    MOV A, \#80H     ; Set cursor to first line, first position
    CALL LCD\_CMD

    MOV A, \#{W    ; Load characters to send}
    CALL LCD\_DATA
    MOV A, \#{e}
    CALL LCD\_DATA
    MOV A, \#{l}
    CALL LCD\_DATA
    ; ... (Send rest of "come")

END ; Add an infinite loop if needed for display to stay
\end{verbatim}

\textbf{Explanation}

\begin{itemize}
\tightlist
\item
  \textbf{Constants:} \texttt{LCD\_PORT} is defined to make the code
  adaptable if your connections change.
\item
  \textbf{Subroutines:} These encapsulate the interaction details with
  the LCD (sending commands, sending data), making the main program
  cleaner. You'll need to fill in the details of these subroutines
  according to your LCD's datasheet.
\item
  \textbf{Main Program:}

  \begin{itemize}
  \tightlist
  \item
    Calls \texttt{LCD\_INIT} to configure the LCD.
  \item
    Sends commands to select the desired display mode (2 lines, 5x7
    font) and clear the display.
  \item
    Sends each character of the message ``Welcome'' to the LCD using
    \texttt{LCD\_DATA}.
  \end{itemize}
\end{itemize}

\textbf{Key Points}

\begin{itemize}
\tightlist
\item
  \textbf{LCD Datasheet:} You MUST adapt the initialization and command
  sequences based on your specific LCD module.
\item
  \textbf{8051 Timing:} You might need short delays within the
  subroutines to ensure the LCD processes commands correctly.
\item
  \textbf{Subroutine Implementation:} The core logic of sending
  commands/data to the LCD involves setting the RS/RW lines, placing
  data on the data port, and pulsing the Enable pin.
\end{itemize}

\subsubsection{Q5a: Explain logical instruction with
example}\label{q5a-explain-logical-instruction-with-example}

\textbf{What are Logical Instructions?}

Logical instructions perform bitwise operations on individual bits
within registers or between a register and an immediate value. These
include:

\begin{itemize}
\tightlist
\item
  \textbf{AND:} Bitwise logical AND operation.
\item
  \textbf{OR:} Bitwise logical OR operation.
\item
  \textbf{XOR:} Bitwise logical XOR (Exclusive OR) operation.
\item
  \textbf{NOT:} Bitwise inversion (Complement)
\item
  \textbf{Rotate/Shift:} Move bits within a register or memory location
\end{itemize}

\textbf{How they work}

Each bit of the first operand is compared to the corresponding bit of
the second operand according to the following truth tables:

{\def\LTcaptype{none} % do not increment counter
\begin{longtable}[]{@{}lllll@{}}
\toprule\noalign{}
Input A & Input B & AND & OR & XOR \\
\midrule\noalign{}
\endhead
\bottomrule\noalign{}
\endlastfoot
0 & 0 & 0 & 0 & 0 \\
0 & 1 & 0 & 1 & 1 \\
1 & 0 & 0 & 1 & 1 \\
1 & 1 & 1 & 1 & 0 \\
\end{longtable}
}

\textbf{Key Takeaways:}

\begin{itemize}
\tightlist
\item
  \textbf{AND:} Outputs 1 only when both inputs are 1.
\item
  \textbf{OR:} Outputs 1 when at least one input is 1.
\item
  \textbf{XOR:} Outputs 1 when the inputs are different.
\end{itemize}

\textbf{Examples}

\begin{enumerate}
\item
  \textbf{Masking Bits (AND)}

\begin{verbatim}
MOV A, \#53H  ; A = 0101 0011
ANL A, \#0FH  ; AND with 0000 1111 (mask to keep only the lower 4 bits)
; A now holds 0000 0011
\end{verbatim}
\item
  \textbf{Setting Bits (OR)}

\begin{verbatim}
MOV A, \#7BH  ; A = 0111 1011
ORL A, \#80H  ; OR with 1000 0000 (set the most significant bit)
; A now holds 1111 1011
\end{verbatim}
\item
  \textbf{Toggling Bits (XOR)}

\begin{verbatim}
MOV A, \#96H  ; A = 1001 0110
XOR A, \#05H  ; XOR with 0000 0101 (toggle specific bits)
; A now holds 1001 0011
\end{verbatim}
\item
  \textbf{Rotating Bits}

\begin{verbatim}
MOV A, \#0AH  ; A = 0000 1010
RL A         ; Rotate left through carry (assume Carry flag is 0)
; A now holds 0001 0100
\end{verbatim}
\end{enumerate}

\textbf{Common Uses}

\begin{itemize}
\tightlist
\item
  \textbf{Testing if specific bits are set or clear.}
\item
  \textbf{Manipulating flags (e.g., setting the Carry flag).}
\item
  \textbf{Isolating sections of data within a byte.}
\item
  \textbf{Implementing simple cryptographic functions.}
\end{itemize}

\subsubsection{Q5b: Interface 7 segment with 8051
microcontroller}\label{q5b-interface-7-segment-with-8051-microcontroller}

\textbf{Assumptions}

\begin{itemize}
\tightlist
\item
  \textbf{Common Anode Display:} The segments have a common positive
  connection. We'll control them by sinking current (connecting to
  ground) through the 8051's pins.

  \begin{itemize}
  \tightlist
  \item
    Adapt the segment patterns if you have a Common Cathode display.
  \end{itemize}
\item
  \textbf{Single Digit:} We'll interface a single digit display. This
  can be extended for multiple digits using multiplexing techniques.
\item
  \textbf{Connections:} We'll assume you'll connect the 7-segment pins
  (a through g) to a port of the 8051 (e.g., Port 1).
\end{itemize}

\textbf{Hardware Setup}

\begin{enumerate}
\tightlist
\item
  \textbf{7-Segment Display:} Choose a common anode 7-segment LED
  display.
\item
  \textbf{Current-Limiting Resistors:} Calculate and use resistors in
  series with each segment LED to prevent damage. Search for a ``LED
  resistor calculator'' to find the right values.
\item
  \textbf{Connections:}

  \begin{itemize}
  \tightlist
  \item
    Connect the anodes of all segments (a through g) to the
    corresponding pins of Port 1 (P1.0 through P1.6) of the 8051.
  \item
    Connect the common anode pin to the power supply (+5V).
  \item
    Connect each segment's cathode through the resistor to ground.
  \end{itemize}
\end{enumerate}

\textbf{Lookup Table}

Create a lookup table in your program memory that maps the digit you
want to display (0-9) to the corresponding segment patterns:

\begin{verbatim}
SEGMENT\_PATTERNS:
    DB 0C0H ; Pattern for 0 (abcdefg)
    DB 0F9H ; Pattern for 1
    DB 0A4H ; Pattern for 2
     ; ... Add patterns for 3{-9}
\end{verbatim}

Note: For a common anode display, `1' means the segment should be ON, so
it's connected to ground.

\textbf{Assembly Code Example}

\begin{verbatim}
ORG 0000H

; Assume display is connected to Port 1
DISPLAY\_PORT EQU P1

; ... (Segment patterns lookup table from above)

MAIN\_LOOP:
    MOV R0, \#2   ; Example: Load the digit 2 to display
    MOV A, @R0   ; Point to the segment pattern in the table
    ADD A, SEGMENT\_PATTERNS  ; Calculate the address
    MOVC A, @A+DPTR          ; Fetch the segment pattern
    MOV DISPLAY\_PORT, A      ;  Send the pattern to the display port

    ; ... (Add display refreshing if you want to multiplex multiple digits)
END
\end{verbatim}

\textbf{Explanation}

\begin{itemize}
\tightlist
\item
  \textbf{Table Usage:} The code loads the digit to be displayed into
  R0, uses indirect addressing to get the corresponding pattern from the
  lookup table, and sends it to the display port.
\item
  \textbf{Multiplexing:} If you have multiple 7-segment displays, you
  need to switch between them rapidly and update the display port
  accordingly to create the illusion they are all on simultaneously.
\end{itemize}

\textbf{Important:}

\begin{itemize}
\tightlist
\item
  \textbf{Port Output:} Ensure the port you use is configured as output.
\item
  \textbf{Resistors:} Don't forget the current-limiting resistors!
\end{itemize}

\subsubsection{Q5c: Interface LM 35 with 8051 microcontroller and
explain block diagram of temperature
controller}\label{q5c-interface-lm-35-with-8051-microcontroller-and-explain-block-diagram-of-temperature-controller}

\textbf{Interfacing LM35 with 8051}

\begin{enumerate}
\item
  \textbf{Connections:}

  \begin{itemize}
  \tightlist
  \item
    Connect the Vout pin of the LM35 to one of the 8051's analog input
    channels (ADC).
  \item
    Connect the VSS pin of the LM35 to ground.
  \item
    Connect the VS pin of the LM35 to the power supply (+5V).
  \end{itemize}
\item
  \textbf{ADC Configuration:}

  \begin{itemize}
  \tightlist
  \item
    Select the ADC channel connected to the LM35.
  \item
    Set the ADC's resolution (e.g., 10-bit).
  \item
    In your program, initiate the ADC conversion process.
  \end{itemize}
\item
  \textbf{Reading and Conversion:}

  \begin{itemize}
  \tightlist
  \item
    After the conversion completes, read the digital value from the ADC
    data register.
  \item
    The LM35 outputs 10mV per degree Celsius. To convert the digital ADC
    reading to temperature:

    \begin{itemize}
    \tightlist
    \item
      Scale the ADC value based on its resolution and reference voltage.
    \item
      Divide by 10 to get the temperature in Celsius.
    \end{itemize}
  \end{itemize}
\end{enumerate}

\textbf{Code Snippet (Illustrative)}

\begin{verbatim}
; ... (ADC Initialization)
START\_CONVERSION:
    SETB ADC\_START\_BIT  ; Trigger ADC conversion
    JBC ADC\_BUSY\_BIT, START\_CONVERSION ; Wait for conversion to complete

    MOV A, ADC\_DATA\_REG   ; Read ADC result
    ; ... (Calculate temperature from ADC value)
\end{verbatim}

\textbf{Block Diagram: Temperature Controller}

\begin{figure}
\centering
\pandocbounded{\includegraphics[keepaspectratio,alt={Block Diagram of 8051 based Temperature Control}]{./imgs/75b4db4005f407f7.png}}
\caption{Block Diagram of 8051 based Temperature Control}
\end{figure}

\begin{itemize}
\tightlist
\item
  \textbf{Temperature Sensor (LM35):} Measures the ambient temperature
  and generates an analog voltage proportional to the temperature.
\item
  \textbf{ADC (Analog-to-Digital Converter):} Part of the 8051
  microcontroller, it converts the analog voltage from the LM35 into a
  digital value.
\item
  \textbf{8051 Microcontroller:}

  \begin{itemize}
  \tightlist
  \item
    Reads the temperature from the ADC.
  \item
    Compares the measured temperature with a desired setpoint.
  \item
    Generates control signals based on the comparison.
  \end{itemize}
\item
  \textbf{Control Output (Relay, etc.):} Controls a device (e.g.,
  heater, fan) to regulate the temperature. Could be a simple on/off
  relay or more complex control like PWM.
\item
  \textbf{Display (Optional):} A display (LCD, 7-segment) to show the
  current temperature or setpoint.
\end{itemize}

\textbf{How it Works}

\begin{enumerate}
\tightlist
\item
  The LM35 senses temperature and sends the analog signal to the ADC.
\item
  The ADC converts the analog signal to a digital value.
\item
  The 8051 microcontroller reads this value, calculates the temperature,
  and compares it to the desired setpoint.
\item
  If the temperature deviates from the setpoint, the 8051 sends control
  signals to turn a heater or cooler on or off, aiming to bring the
  temperature back to the setpoint.
\end{enumerate}


\end{document}
