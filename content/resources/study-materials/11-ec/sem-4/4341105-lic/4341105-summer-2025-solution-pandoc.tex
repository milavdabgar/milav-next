\documentclass[10pt,a4paper]{article}

% content/resources/templates/preamble.tex
\usepackage[margin=0.6in]{geometry}
\author{Milav Dabgar}
\usepackage{amsmath,amssymb,amsthm}
\usepackage{booktabs}
\usepackage{multirow}
\usepackage{xcolor}
\usepackage{tcolorbox}
\tcbuselibrary{breakable,skins}
\usepackage[colorlinks=true,linkcolor=blue]{hyperref}
\usepackage{titlesec}
\usepackage{enumitem}
\usepackage{tikz}
\usepackage{pgfplots}
\usepackage{circuitikz}
\usepackage[version=4]{mhchem}
\usepackage{longtable}
\usepackage{array}
\usepackage{float}
\usepackage{caption}
\usepackage{listings}

\lstset{
  basicstyle=\small\ttfamily,
  breaklines=true,
  breakatwhitespace=false,
  postbreak=\mbox{\textcolor{red}{$\hookrightarrow$}\space},
  float=false,
  numbers=left,
  numberstyle=\tiny\color{gray},
  numbersep=10pt,
  xleftmargin=2em,
  keywordstyle=\color{blue},
  commentstyle=\color{green!60!black},
  stringstyle=\color{purple},
  backgroundcolor=\color{gray!5},
  showstringspaces=false,
  tabsize=2,
  captionpos=b,
  keepspaces=true,
  columns=flexible
}

\pgfplotsset{compat=1.18}
\usetikzlibrary{shapes,arrows,positioning,calc,patterns,decorations.pathmorphing,decorations.markings,arrows.meta}

% Color scheme
\definecolor{headcolor}{RGB}{0,102,204}
\definecolor{keycolor}{RGB}{220,20,60}
\definecolor{solutioncolor}{RGB}{34,139,34}
\definecolor{mnemoniccolor}{RGB}{148,0,211}
\definecolor{codecolor}{RGB}{0,0,100}

% Spacing
\setlength{\parskip}{3pt}
\setlist[itemize]{nosep}
\setlist[enumerate]{nosep}

% Title formatting
\titleformat{\section}{\Large\bfseries\color{headcolor}}{\thesection}{1em}{}
\titleformat{\subsection}{\large\bfseries\color{headcolor}}{\thesubsection}{1em}{}

% Pandoc tightlist compatibility
\providecommand{\tightlist}{%
  \setlength{\itemsep}{0pt}\setlength{\parskip}{0pt}}

% Pandoc longtable compatibility
\newcounter{none}
\def\thenone{}


% content/resources/templates/english-boxes.tex
% This file is currently empty - it exists to maintain consistency with the import structure.
% Add custom environments here if needed in the future.


\begin{document}

\begin{center}
{\Huge\bfseries\color{headcolor} Subject Name Solutions}\\[5pt]
{\LARGE 4341105 -- Summer 2025}\\[3pt]
{\large Semester 1 Study Material}\\[3pt]
{\normalsize\textit{Detailed Solutions and Explanations}}
\end{center}

\vspace{10pt}

\subsection*{Question 1(a) [3 marks]}\label{q1a}

\textbf{Explain effect of negative feedback on gain and stability.}

\begin{solutionbox}
Negative feedback significantly improves amplifier
performance by reducing gain but enhancing stability and other
parameters.


{\def\LTcaptype{none} % do not increment counter
\begin{longtable}[]{@{}ll@{}}
\toprule\noalign{}
Parameter & Effect of Negative Feedback \\
\midrule\noalign{}
\endhead
\bottomrule\noalign{}
\endlastfoot
\textbf{Gain} & Reduces overall gain \\
\textbf{Stability} & Increases stability \\
\textbf{Bandwidth} & Increases bandwidth \\
\end{longtable}
}

\begin{itemize}
\tightlist
\item
  \textbf{Gain reduction}: Makes amplifier more predictable
\item
  \textbf{Stability improvement}: Reduces oscillations and distortions
\item
  \textbf{Better control}: Provides consistent performance
\end{itemize}

\end{solutionbox}
\begin{mnemonicbox}
``Gain Goes Down, Stability Stays Strong''

\end{mnemonicbox}
\subsection*{Question 1(b) [4 marks]}\label{q1b}

\textbf{State different types of feedback amplifier and advantages of
negative feedback amplifier.}

\begin{solutionbox}
Four basic feedback topologies exist based on input and
output connections.


{\def\LTcaptype{none} % do not increment counter
\begin{longtable}[]{@{}lll@{}}
\toprule\noalign{}
Type & Input Connection & Output Connection \\
\midrule\noalign{}
\endhead
\bottomrule\noalign{}
\endlastfoot
\textbf{Voltage Series} & Series & Voltage \\
\textbf{Voltage Shunt} & Shunt & Voltage \\
\textbf{Current Series} & Series & Current \\
\textbf{Current Shunt} & Shunt & Current \\
\end{longtable}
}

\textbf{Advantages:}

\begin{itemize}
\tightlist
\item
  \textbf{Reduced distortion}: Minimizes harmonic content
\item
  \textbf{Increased bandwidth}: Better frequency response
\item
  \textbf{Improved stability}: Consistent operation
\end{itemize}

\end{solutionbox}
\begin{mnemonicbox}
``Very Smart Current Control''

\end{mnemonicbox}
\subsection*{Question 1(c) [7 marks]}\label{q1c}

\textbf{Derive an equation for overall gain of negative feedback voltage
amplifier.}

\begin{solutionbox}
For negative feedback amplifier, output is fed back to
input in opposite phase.

\textbf{Circuit Analysis:} Let A = Open loop gain, β = Feedback factor

\textbf{Diagram:}

\begin{center}
\textbf{Mermaid Diagram (Code)}
\begin{verbatim}
{Shaded}
{Highlighting}[]
graph LR
    A[Input Vi] {-{-}{} B[Amplifier A]}
    B {-{-}{} C[Output Vo]}
    C {-{-}{} D[Feedback β]}
    D {-{-}{} E[Summing Junction]}
    A {-{-}{} E}
{Highlighting}
{Shaded}
\end{verbatim}
\end{center}

\textbf{Derivation:}

\begin{itemize}
\tightlist
\item
  Input to amplifier: Vi - βVo
\item
  Output: Vo = A(Vi - βVo)
\item
  Vo = AVi - AβVo
\item
  Vo + AβVo = AVi
\item
  Vo(1 + Aβ) = AVi
\item
  \textbf{Overall Gain: Af = A/(1 + Aβ)}
\end{itemize}

\textbf{Key Points:}

\begin{itemize}
\tightlist
\item
  \textbf{Denominator (1 + Aβ)}: Called loop gain
\item
  \textbf{Stability factor}: Determines system response
\item
  \textbf{Gain reduction}: Traded for better performance
\end{itemize}

\end{solutionbox}
\begin{mnemonicbox}
``Always Divide by (1 + Loop)''

\end{mnemonicbox}
\subsection*{Question 1(c OR) [7
marks]}\label{question-1c-or-7-marks}

\textbf{Draw and explain current shunt type negative feedback amplifier
and Derive the formula of input impedance and output impedance of it.}

\begin{solutionbox}
Current shunt feedback samples output current and feeds
back voltage in shunt with input.

\textbf{Circuit Diagram:}

\begin{center}
\textbf{Mermaid Diagram (Code)}
\begin{verbatim}
{Shaded}
{Highlighting}[]
graph LR
    A[Vi] {-{-}{} B["{}+"]}
    B {-{-}{} C[Amplifier A]}
    C {-{-}{} D[Ro]}
    D {-{-}{} E[RL]}
    D {-{-}{} F[Feedback Network β]}
    F {-{-}{} G["{}{-}"]}
    B {-{-}{} G}
{Highlighting}
{Shaded}
\end{verbatim}
\end{center}

\textbf{Analysis:}

\begin{itemize}
\tightlist
\item
  \textbf{Feedback type}: Current sampling, voltage mixing
\item
  \textbf{Input impedance}: Decreases due to shunt feedback
\item
  \textbf{Output impedance}: Decreases due to current sampling
\end{itemize}

\textbf{Formulas:}

\begin{itemize}
\tightlist
\item
  \textbf{Input Impedance: Zif = Zi/(1 + Aβ)}
\item
  \textbf{Output Impedance: Zof = Zo/(1 + Aβ)}
\end{itemize}

\textbf{Characteristics:}

\begin{itemize}
\tightlist
\item
  \textbf{Low input impedance}: Good for current sources
\item
  \textbf{Low output impedance}: Good for voltage output
\item
  \textbf{Current-to-voltage converter}: Useful in applications
\end{itemize}

\end{solutionbox}
\begin{mnemonicbox}
``Current Shunt Lowers Both Impedances''

\end{mnemonicbox}
\subsection*{Question 2(a) [3 marks]}\label{q2a}

\textbf{Explain Barkhausen criteria for oscillations.}

\begin{solutionbox}
For sustained oscillations in feedback circuits, two
conditions must be satisfied simultaneously.


{\def\LTcaptype{none} % do not increment counter
\begin{longtable}[]{@{}lll@{}}
\toprule\noalign{}
Criteria & Condition & Description \\
\midrule\noalign{}
\endhead
\bottomrule\noalign{}
\endlastfoot
\textbf{Magnitude} & \textbar Aβ\textbar{} = 1 & Loop gain unity \\
\textbf{Phase} & ∠Aβ = 0^\circ or 360^\circ & Zero phase shift \\
\end{longtable}
}

\begin{itemize}
\tightlist
\item
  \textbf{Unity loop gain}: Ensures signal maintains amplitude
\item
  \textbf{Zero phase shift}: Ensures positive feedback
\item
  \textbf{Sustained oscillation}: Both conditions create self-sustaining
  signals
\end{itemize}

\end{solutionbox}
\begin{mnemonicbox}
``One Magnitude, Zero Phase''

\end{mnemonicbox}
\subsection*{Question 2(b) [4 marks]}\label{q2b}

\textbf{Explain use of tank circuit with neat diagram.}

\begin{solutionbox}
Tank circuit provides frequency selective positive
feedback for oscillator circuits.

\textbf{Circuit Diagram:}

\begin{verbatim}
    +{-{-}{-}L{-}{-}{-}+}
    |       |
    C       R
    |       |
    +{-{-}{-}{-}{-}{-}{-}+}
\end{verbatim}

\textbf{Operation:} At resonant frequency, LC tank circuit exhibits:


{\def\LTcaptype{none} % do not increment counter
\begin{longtable}[]{@{}lll@{}}
\toprule\noalign{}
Parameter & Value & Effect \\
\midrule\noalign{}
\endhead
\bottomrule\noalign{}
\endlastfoot
\textbf{Reactance} & XL = XC & Resonance \\
\textbf{Impedance} & Maximum & High selectivity \\
\textbf{Phase} & 0^\circ & Unity feedback \\
\end{longtable}
}

\begin{itemize}
\tightlist
\item
  \textbf{Energy storage}: L and C exchange energy
\item
  \textbf{Frequency selection}: Sharp resonance characteristic
\item
  \textbf{Oscillation sustenance}: Provides positive feedback
\end{itemize}

\end{solutionbox}
\begin{mnemonicbox}
``Tank Stores Energy, Selects Frequency''

\end{mnemonicbox}
\subsection*{Question 2(c) [7 marks]}\label{q2c}

\textbf{Draw and explain the Hartley Oscillator. Also state equation of
oscillation frequency of it.}

\begin{solutionbox}
Hartley oscillator uses tapped inductor in tank circuit
for frequency generation.

\textbf{Circuit Diagram:}

\begin{center}
\textbf{Mermaid Diagram (Code)}
\begin{verbatim}
{Shaded}
{Highlighting}[]
graph LR
    A[Vcc] {-{-}{} B[RFC]}
    B {-{-}{} C[Collector]}
    C {-{-}{} D[L1]}
    D {-{-}{} E[L2]}
    E {-{-}{} F[Emitter]}
    D {-{-}{} G[C]}
    G {-{-}{} E}
    C {-{-}{} H[Output]}
{Highlighting}
{Shaded}
\end{verbatim}
\end{center}

\textbf{Operation:}

\begin{itemize}
\tightlist
\item
  \textbf{Tapped inductor}: L1 and L2 provide feedback
\item
  \textbf{Tank circuit}: L1+L2 with C determines frequency
\item
  \textbf{Positive feedback}: Phase shift through L1-L2 coupling
\end{itemize}

\textbf{Frequency Formula:} \textbf{f = 1/[2π\sqrt((L1+L2)C)]}

\textbf{Key Features:}

\begin{itemize}
\tightlist
\item
  \textbf{Good frequency stability}: Inductor-based tuning
\item
  \textbf{Easy tuning}: Variable inductor or capacitor
\item
  \textbf{RF applications}: Suitable for high frequencies
\end{itemize}

\end{solutionbox}
\begin{mnemonicbox}
``Hartley Has Tapped inductor''

\end{mnemonicbox}
\subsection*{Question 2(a OR) [3
marks]}\label{question-2a-or-3-marks}

\textbf{Explain term oscillator as positive feedback amplifier.}

\begin{solutionbox}
Oscillator generates AC signals using positive feedback
without external input signal.


{\def\LTcaptype{none} % do not increment counter
\begin{longtable}[]{@{}lll@{}}
\toprule\noalign{}
Parameter & Amplifier & Oscillator \\
\midrule\noalign{}
\endhead
\bottomrule\noalign{}
\endlastfoot
\textbf{Input} & External signal & No external input \\
\textbf{Feedback} & May use negative & Uses positive \\
\textbf{Output} & Amplified input & Self-generated AC \\
\end{longtable}
}

\begin{itemize}
\tightlist
\item
  \textbf{Self-sustaining}: Positive feedback maintains oscillation
\item
  \textbf{Barkhausen criteria}: Loop gain = 1, phase = 0^\circ
\item
  \textbf{Signal generation}: Creates AC from DC supply
\end{itemize}

\end{solutionbox}
\begin{mnemonicbox}
``Positive feedback Powers Perpetual signals''

\end{mnemonicbox}
\subsection*{Question 2(b OR) [4
marks]}\label{question-2b-or-4-marks}

\textbf{Draw and explain the Crystal Oscillator.}

\begin{solutionbox}
Crystal oscillator uses piezoelectric effect of quartz
crystal for high stability.

\textbf{Circuit Diagram:}

\begin{verbatim}
         Vcc
          |
          R
          |
    +{-{-}{-}{-}{-}+{-}{-}{-}{-}{-}+}
    |           |
    |     Q     |
    |           |
    +{-{-}{-}{-}{-}+{-}{-}{-}{-}{-}+}
          |
        XTAL
          |
         GND
\end{verbatim}

\textbf{Characteristics:}


{\def\LTcaptype{none} % do not increment counter
\begin{longtable}[]{@{}lll@{}}
\toprule\noalign{}
Property & Value & Advantage \\
\midrule\noalign{}
\endhead
\bottomrule\noalign{}
\endlastfoot
\textbf{Stability} & \pm0.01\% & Very high \\
\textbf{Q factor} & \textgreater10,000 & Sharp resonance \\
\textbf{Temperature} & Low drift & Stable frequency \\
\end{longtable}
}

\begin{itemize}
\tightlist
\item
  \textbf{Piezoelectric effect}: Mechanical vibration creates electrical
  signal
\item
  \textbf{High Q}: Very stable frequency generation
\item
  \textbf{Clock applications}: Used in digital systems
\end{itemize}

\end{solutionbox}
\begin{mnemonicbox}
``Crystal Creates Constant frequency''

\end{mnemonicbox}
\subsection*{Question 2(c OR) [7
marks]}\label{question-2c-or-7-marks}

\textbf{Draw the Structure, symbol, equivalent circuit of UJT and
explain it in brief.}

\begin{solutionbox}
UJT (Unijunction Transistor) is three-terminal device
with unique switching characteristics.

\textbf{Structure:}

\begin{verbatim}
    B2 +{-{-}{-}{-}{-}{-}{-}+}
       |   n   |
       |       |
    E  +   p   +
       |       |
       |   n   |
    B1 +{-{-}{-}{-}{-}{-}{-}+}
\end{verbatim}

\textbf{Symbol:}

\begin{verbatim}
    B2
     |
     +
    /|
   / |
  /  +{-{-}{-} E}
     |
     +
     |
    B1
\end{verbatim}

\textbf{Equivalent Circuit:}

\begin{verbatim}
    B2 +{-{-}{-}R2{-}{-}{-}+}
               |
    E  +{-{-}{-}{-}{-}{-}{-}+}
               |
    B1 +{-{-}{-}R1{-}{-}{-}+}
\end{verbatim}

\textbf{Operation:}

\begin{itemize}
\tightlist
\item
  \textbf{Intrinsic standoff ratio}: η = R1/(R1+R2)
\item
  \textbf{Peak point voltage}: VP = ηVBB + VD
\item
  \textbf{Negative resistance}: After peak point
\end{itemize}

\textbf{Applications:}

\begin{itemize}
\tightlist
\item
  \textbf{Relaxation oscillator}: Sawtooth wave generation
\item
  \textbf{Trigger circuits}: SCR firing circuits
\item
  \textbf{Timing applications}: RC charging circuits
\end{itemize}

\end{solutionbox}
\begin{mnemonicbox}
``UJT Uses Unique Junction Technology''

\end{mnemonicbox}
\subsection*{Question 3(a) [3 marks]}\label{q3a}

\textbf{Classify power amplifier based on operating point.}

\begin{solutionbox}
Power amplifiers are classified based on transistor
conduction angle and bias point.


{\def\LTcaptype{none} % do not increment counter
\begin{longtable}[]{@{}llll@{}}
\toprule\noalign{}
Class & Conduction Angle & Efficiency & Application \\
\midrule\noalign{}
\endhead
\bottomrule\noalign{}
\endlastfoot
\textbf{Class A} & 360^\circ & 25-50\% & Audio, low power \\
\textbf{Class B} & 180^\circ & 78.5\% & Push-pull \\
\textbf{Class AB} & 180^\circ-360^\circ & 60-70\% & Audio power \\
\textbf{Class C} & \textless180^\circ & \textgreater90\% & RF, tuned \\
\end{longtable}
}

\begin{itemize}
\tightlist
\item
  \textbf{Bias point}: Determines operating class
\item
  \textbf{Efficiency trade-off}: Higher efficiency, more distortion
\item
  \textbf{Application specific}: Choose based on requirements
\end{itemize}

\end{solutionbox}
\begin{mnemonicbox}
``All Big Amplifiers Can deliver power''

\end{mnemonicbox}
\subsection*{Question 3(b) [4 marks]}\label{q3b}

\textbf{Draw and Explain Complementary symmetry push-pull power
amplifier.}

\begin{solutionbox}
Uses NPN and PNP transistors for efficient power
amplification without center-tapped transformer.

\textbf{Circuit Diagram:}

\begin{center}
\textbf{Mermaid Diagram (Code)}
\begin{verbatim}
{Shaded}
{Highlighting}[]
graph LR
    A[+Vcc] {-{-}{} B[NPN Q1]}
    B {-{-}{} C[Output]}
    C {-{-}{} D[RL]}
    D {-{-}{} E[PNP Q2]}
    E {-{-}{} F[{-}Vcc]}
    G[Input] {-{-}{} B}
    G {-{-}{} E}
{Highlighting}
{Shaded}
\end{verbatim}
\end{center}

\textbf{Operation:}

\begin{itemize}
\tightlist
\item
  \textbf{Positive half-cycle}: NPN conducts, PNP off
\item
  \textbf{Negative half-cycle}: PNP conducts, NPN off
\item
  \textbf{Complementary action}: Both transistors handle alternate
  half-cycles
\end{itemize}

\textbf{Advantages:}

\begin{itemize}
\tightlist
\item
  \textbf{No transformer}: Direct coupling to load
\item
  \textbf{High efficiency}: Class B operation
\item
  \textbf{Compact design}: Fewer components
\item
  \textbf{Good power transfer}: Direct coupling
\end{itemize}

\end{solutionbox}
\begin{mnemonicbox}
``Complementary transistors Complete the cycle''

\end{mnemonicbox}
\subsection*{Question 3(c) [7 marks]}\label{q3c}

\textbf{Derive an equation for Efficiency of class B push pull
amplifier.}

\begin{solutionbox}
Class B push-pull amplifier has each transistor
conducting for 180^\circ of input cycle.

\textbf{Analysis:} For sinusoidal input: Vi = Vm sin ωt

\textbf{Output Power:}

\begin{itemize}
\tightlist
\item
  Peak output voltage: Vom = Vcc
\item
  RMS output voltage: Vo(rms) = Vcc/\sqrt2
\item
  \textbf{Po = Vo^{2}(rms)/RL = Vcc^{2}/2RL}
\end{itemize}

\textbf{Input Power:}

\begin{itemize}
\tightlist
\item
  DC current (average): Idc = 2Im/π
\item
  Where Im = Vcc/RL
\item
  \textbf{Pin = Vcc \times Idc = 2VccIm/π = 2Vcc^{2}/πRL}
\end{itemize}

\textbf{Efficiency Calculation:} \textbf{η = Po/Pin =
(Vcc^{2}/2RL)/(2Vcc^{2}/πRL)} \textbf{η = π/4 = 0.785 = 78.5\%}

\textbf{Key Points:}

\begin{itemize}
\tightlist
\item
  \textbf{Maximum theoretical efficiency}: 78.5\%
\item
  \textbf{Class B advantage}: Much higher than Class A (25\%)
\item
  \textbf{Practical efficiency}: Slightly lower due to losses
\end{itemize}

\end{solutionbox}
\begin{mnemonicbox}
``Push-Pull Provides Pi/4 efficiency''

\end{mnemonicbox}
\subsection*{Question 3(a OR) [3
marks]}\label{question-3a-or-3-marks}

\textbf{Differentiate between voltage and power amplifier.}

\begin{solutionbox}
Voltage and power amplifiers serve different purposes
in electronic systems.


{\def\LTcaptype{none} % do not increment counter
\begin{longtable}[]{@{}lll@{}}
\toprule\noalign{}
Parameter & Voltage Amplifier & Power Amplifier \\
\midrule\noalign{}
\endhead
\bottomrule\noalign{}
\endlastfoot
\textbf{Purpose} & Increase voltage & Increase power \\
\textbf{Load} & High impedance & Low impedance \\
\textbf{Efficiency} & Not critical & Very important \\
\textbf{Distortion} & Must be low & Moderate acceptable \\
\textbf{Coupling} & RC/Direct & Transformer \\
\end{longtable}
}

\begin{itemize}
\tightlist
\item
  \textbf{Design priority}: Voltage gain vs power delivery
\item
  \textbf{Application}: Signal processing vs driving loads
\item
  \textbf{Circuit complexity}: Simple vs complex power stages
\end{itemize}

\end{solutionbox}
\begin{mnemonicbox}
``Voltage amplifies signal, Power drives load''

\end{mnemonicbox}
\subsection*{Question 3(b OR) [4
marks]}\label{question-3b-or-4-marks}

\textbf{Explain Class AB power amplifier with diagram.}

\begin{solutionbox}
Class AB operates between Class A and Class B, reducing
crossover distortion.

\textbf{Circuit Diagram:}

\begin{verbatim}
    +Vcc
      |
      R
      |
    +{-+{-}+}
    |   |
   Q1  Q2  
    |   |
    +{-{-}{-}+{-}{-}{-} Output}
    |   |
   D1  D2
    |   |
    +{-{-}{-}+}
      |
      R
      |
    {-Vcc}
\end{verbatim}

\textbf{Operation:}

\begin{itemize}
\tightlist
\item
  \textbf{Slight forward bias}: Both transistors slightly on
\item
  \textbf{Conduction angle}: \textgreater180^\circ but \textless360^\circ
\item
  \textbf{Overlap conduction}: Eliminates crossover distortion
\end{itemize}

\textbf{Characteristics:}


{\def\LTcaptype{none} % do not increment counter
\begin{longtable}[]{@{}lll@{}}
\toprule\noalign{}
Parameter & Value & Benefit \\
\midrule\noalign{}
\endhead
\bottomrule\noalign{}
\endlastfoot
\textbf{Efficiency} & 60-70\% & Better than Class A \\
\textbf{Distortion} & Low & Better than Class B \\
\textbf{Bias} & Slight forward & Compromise solution \\
\end{longtable}
}

\end{solutionbox}
\begin{mnemonicbox}
``AB Avoids Bad crossover distortion''

\end{mnemonicbox}
\subsection*{Question 3(c OR) [7
marks]}\label{question-3c-or-7-marks}

\textbf{Derive an equation for Efficiency of series fed class A power
amplifier.}

\begin{solutionbox}
Series fed Class A amplifier has DC supply connected in
series with load.

\textbf{Circuit Analysis:}

\begin{itemize}
\tightlist
\item
  \textbf{DC supply voltage}: Vcc
\item
  \textbf{Quiescent current}: Icq = Vcc/2RL (for maximum power)
\item
  \textbf{Quiescent voltage}: Vceq = Vcc/2
\end{itemize}

\textbf{AC Analysis:}

\begin{itemize}
\tightlist
\item
  \textbf{Maximum output voltage swing}: Vom = Vcc/2
\item
  \textbf{Output power}: Po = Vom^{2}/2RL = Vcc^{2}/8RL
\end{itemize}

\textbf{DC Power:}

\begin{itemize}
\tightlist
\item
  \textbf{DC current}: Idc = Icq = Vcc/2RL
\item
  \textbf{Input power}: Pin = Vcc \times Idc = Vcc^{2}/2RL
\end{itemize}

\textbf{Efficiency:} \textbf{η = Po/Pin = (Vcc^{2}/8RL)/(Vcc^{2}/2RL)}
\textbf{η = 1/4 = 0.25 = 25\%}

\textbf{Key Points:}

\begin{itemize}
\tightlist
\item
  \textbf{Maximum theoretical efficiency}: 25\%
\item
  \textbf{Power wastage}: 75\% lost as heat
\item
  \textbf{Design limitation}: Poor efficiency but good linearity
\end{itemize}

\end{solutionbox}
\begin{mnemonicbox}
``Class A Achieves quarter efficiency''

\end{mnemonicbox}
\subsection*{Question 4(a) [3 marks]}\label{q4a}

\textbf{Draw pin diagram of IC 741 OP-AMP and explain it.}

\begin{solutionbox}
IC 741 is 8-pin dual-in-line package operational
amplifier with industry standard pinout.

\textbf{Pin Diagram:}

\begin{verbatim}
    +{-{-}{-}U{-}{-}{-}+}
  1 |       | 8
    |  741  |
  2 |       | 7
    |       |
  3 |       | 6
    |       |
  4 |       | 5
    +{-{-}{-}{-}{-}{-}{-}+}
\end{verbatim}

\textbf{Pin Configuration:}


{\def\LTcaptype{none} % do not increment counter
\begin{longtable}[]{@{}lll@{}}
\toprule\noalign{}
Pin & Function & Description \\
\midrule\noalign{}
\endhead
\bottomrule\noalign{}
\endlastfoot
\textbf{1} & Offset Null & Offset adjustment \\
\textbf{2} & Inverting Input & Negative input \\
\textbf{3} & Non-inverting Input & Positive input \\
\textbf{4} & -Vcc & Negative supply \\
\textbf{5} & Offset Null & Offset adjustment \\
\textbf{6} & Output & Amplifier output \\
\textbf{7} & +Vcc & Positive supply \\
\textbf{8} & NC & No connection \\
\end{longtable}
}

\end{solutionbox}
\begin{mnemonicbox}
``Null, Negative, Positive, Negative supply, Null,
Output, Positive supply, Nothing''

\end{mnemonicbox}
\subsection*{Question 4(b) [4 marks]}\label{q4b}

\textbf{Define the following OP-AMP parameters. 1. Input offset voltage
2. CMRR}

\begin{solutionbox}
These parameters define the non-ideal characteristics
of practical operational amplifiers.

\textbf{1. Input Offset Voltage (Vio):}

\begin{itemize}
\tightlist
\item
  \textbf{Definition}: DC voltage applied between inputs to make output
  zero
\item
  \textbf{Typical value}: 1-5 mV for 741
\item
  \textbf{Cause}: Mismatch in input transistors
\item
  \textbf{Effect}: Output error in DC applications
\end{itemize}

\textbf{2. Common Mode Rejection Ratio (CMRR):}

\begin{itemize}
\tightlist
\item
  \textbf{Definition}: Ability to reject common signals at both inputs
\item
  \textbf{Formula}: CMRR = Ad/Acm
\item
  \textbf{Typical value}: 90 dB for 741
\item
  \textbf{Importance}: Noise immunity
\end{itemize}


{\def\LTcaptype{none} % do not increment counter
\begin{longtable}[]{@{}lllll@{}}
\toprule\noalign{}
Parameter & Symbol & Unit & Ideal & 741 Typical \\
\midrule\noalign{}
\endhead
\bottomrule\noalign{}
\endlastfoot
\textbf{Input Offset Voltage} & Vio & mV & 0 & 2 \\
\textbf{CMRR} & - & dB & \infty & 90 \\
\end{longtable}
}

\end{solutionbox}
\begin{mnemonicbox}
``Offset creates Output error, CMRR Rejects common
signals''

\end{mnemonicbox}
\subsection*{Question 4(c) [7 marks]}\label{q4c}

\textbf{Explain inverting amplifier using IC 741 in detail.}

\begin{solutionbox}
Inverting amplifier uses negative feedback with input
applied to inverting terminal.

\textbf{Circuit Diagram:}

\begin{center}
\textbf{Mermaid Diagram (Code)}
\begin{verbatim}
{Shaded}
{Highlighting}[]
graph LR
    A[Vin] {-{-}{} B[R1]}
    B {-{-}{} C["{}{-}"]}
    D["{+"] {-}{-}{} E[Ground]}
    C {-{-}{} F[IC 741]}
    F {-{-}{} G[Vout]}
    G {-{-}{} H[Rf]}
    H {-{-}{} C}
{Highlighting}
{Shaded}
\end{verbatim}
\end{center}

\textbf{Analysis:} Using virtual short concept:

\begin{itemize}
\tightlist
\item
  \textbf{V+ = V- = 0V} (virtual ground)
\item
  \textbf{Input current}: I1 = Vin/R1
\item
  \textbf{Feedback current}: If = Vout/Rf
\item
  \textbf{Current balance}: I1 = If (no current into op-amp)
\end{itemize}

\textbf{Derivation:}

\begin{itemize}
\tightlist
\item
  Vin/R1 = -Vout/Rf
\item
  \textbf{Voltage Gain: Av = -Rf/R1}
\end{itemize}

\textbf{Characteristics:}


{\def\LTcaptype{none} % do not increment counter
\begin{longtable}[]{@{}lll@{}}
\toprule\noalign{}
Parameter & Expression & Notes \\
\midrule\noalign{}
\endhead
\bottomrule\noalign{}
\endlastfoot
\textbf{Voltage Gain} & -Rf/R1 & Negative sign \\
\textbf{Input Impedance} & R1 & Low impedance \\
\textbf{Output Impedance} & \textasciitilde0Ω & Very low \\
\textbf{Bandwidth} & f = GBW/\textbar Av\textbar{} & Gain-bandwidth
product \\
\end{longtable}
}

\textbf{Applications:}

\begin{itemize}
\tightlist
\item
  \textbf{Signal inversion}: Phase reversal
\item
  \textbf{Scale factor}: Programmable gain
\item
  \textbf{AC amplification}: With coupling capacitors
\end{itemize}

\end{solutionbox}
\begin{mnemonicbox}
``Inverting Input gives Inverted output''

\end{mnemonicbox}
\subsection*{Question 4(a OR) [3
marks]}\label{question-4a-or-3-marks}

\textbf{List characteristics of ideal OP-AMP.}

\begin{solutionbox}
Ideal op-amp represents perfect amplifier with
theoretical limits for all parameters.


{\def\LTcaptype{none} % do not increment counter
\begin{longtable}[]{@{}lll@{}}
\toprule\noalign{}
Parameter & Ideal Value & Practical Impact \\
\midrule\noalign{}
\endhead
\bottomrule\noalign{}
\endlastfoot
\textbf{Open Loop Gain} & \infty & Perfect amplification \\
\textbf{Input Impedance} & \infty & No input current \\
\textbf{Output Impedance} & 0Ω & Perfect voltage source \\
\textbf{Bandwidth} & \infty & No frequency limitation \\
\textbf{CMRR} & \infty & Perfect noise rejection \\
\textbf{Slew Rate} & \infty & No slew rate limiting \\
\textbf{Input Offset} & 0V & No DC errors \\
\end{longtable}
}

\begin{itemize}
\tightlist
\item
  \textbf{Perfect performance}: All parameters optimized
\item
  \textbf{Design simplification}: Analysis becomes easier
\item
  \textbf{Practical approximation}: Close to ideal in many applications
\end{itemize}

\end{solutionbox}
\begin{mnemonicbox}
``Infinite Input, Zero Output, Perfect Performance''

\end{mnemonicbox}
\subsection*{Question 4(b OR) [4
marks]}\label{question-4b-or-4-marks}

\textbf{Draw and explain summing amplifier using Op-amp.}

\begin{solutionbox}
Summing amplifier adds multiple input voltages with
programmable gain for each input.

\textbf{Circuit Diagram:}

\begin{verbatim}
V1 {-{-}{-}R1{-}{-}{-}+}
           |
V2 {-{-}{-}R2{-}{-}{-}+{-}{-}{-} ({-}) IC741 {-}{-}{-} Vout}
           |             |
V3 {-{-}{-}R3{-}{-}{-}+             |}
                        Rf
           (+) {-{-}{-}{-}{-}{-}{-}{-}GND}
\end{verbatim}

\textbf{Analysis:} Using virtual ground concept (V- = 0V):

\begin{itemize}
\tightlist
\item
  \textbf{Current through R1}: I1 = V1/R1
\item
  \textbf{Current through R2}: I2 = V2/R2\\
\item
  \textbf{Current through R3}: I3 = V3/R3
\item
  \textbf{Total input current}: Iin = I1 + I2 + I3
\end{itemize}

\textbf{Output Equation:} \textbf{Vout = -Rf(V1/R1 + V2/R2 + V3/R3)}

\textbf{Special Cases:}

\begin{itemize}
\tightlist
\item
  \textbf{Equal resistors}: Vout = -(Rf/R)(V1 + V2 + V3)
\item
  \textbf{Unity gain}: Rf = R, Vout = -(V1 + V2 + V3)
\end{itemize}

\textbf{Applications:}

\begin{itemize}
\tightlist
\item
  \textbf{Audio mixing}: Multiple signal combination
\item
  \textbf{Digital-to-analog}: Weighted resistor DAC
\item
  \textbf{Signal processing}: Mathematical operations
\end{itemize}

\end{solutionbox}
\begin{mnemonicbox}
``Sum inputs, Scale by resistor ratios''

\end{mnemonicbox}
\subsection*{Question 4(c OR) [7
marks]}\label{question-4c-or-7-marks}

\textbf{Explain differential amplifier using IC 741 in detail.}

\begin{solutionbox}
Differential amplifier amplifies the difference between
two input signals while rejecting common signals.

\textbf{Circuit Diagram:}

\begin{verbatim}
V1 {-{-}{-}R1{-}{-}{-} ({-}) }
            IC741 {-{-}{-} Vout}
V2 {-{-}{-}R2{-}{-}{-} (+)  |}
              |  |
             R3  Rf
              |  |
             GND +
\end{verbatim}

\textbf{Analysis:} For the non-inverting input:

\begin{itemize}
\tightlist
\item
  \textbf{V+ = V2 \times R3/(R2+R3)}
\end{itemize}

For the inverting input using virtual short:

\begin{itemize}
\tightlist
\item
  \textbf{V- = V+ = V2 \times R3/(R2+R3)}
\end{itemize}

Using current balance:

\begin{itemize}
\tightlist
\item
  \textbf{(V1-V-)/R1 = (V--Vout)/Rf}
\end{itemize}

\textbf{Output Equation:} When R1 = R2 and R3 = Rf: \textbf{Vout =
(Rf/R1)(V2 - V1)}

\textbf{Key Features:}


{\def\LTcaptype{none} % do not increment counter
\begin{longtable}[]{@{}lll@{}}
\toprule\noalign{}
Parameter & Value & Advantage \\
\midrule\noalign{}
\endhead
\bottomrule\noalign{}
\endlastfoot
\textbf{Differential Gain} & Rf/R1 & Amplifies difference \\
\textbf{Common Mode Gain} & \textasciitilde0 & Rejects common signals \\
\textbf{CMRR} & Very high & Excellent noise immunity \\
\end{longtable}
}

\textbf{Applications:}

\begin{itemize}
\tightlist
\item
  \textbf{Instrumentation}: Sensor signal processing
\item
  \textbf{Noise rejection}: Differential signal transmission
\item
  \textbf{Bridge circuits}: Strain gauge measurements
\end{itemize}

\end{solutionbox}
\begin{mnemonicbox}
``Difference amplified, Common rejected''

\end{mnemonicbox}
\subsection*{Question 5(a) [3 marks]}\label{q5a}

\textbf{Draw the circuit of integrator using Op-amp and its input and
output waveforms.}

\begin{solutionbox}
Op-amp integrator performs mathematical integration of
input signal using RC feedback.

\textbf{Circuit Diagram:}

\begin{verbatim}
Vin {-{-}{-}R{-}{-}{-} ({-}) IC741 {-}{-}{-} Vout}
              |        |
             (+)       C
              |        |
             GND      /
\end{verbatim}

\textbf{Waveforms:}

\begin{verbatim}
Input (Square Wave):
     +V |‾‾|\_\_|‾‾|\_\_|‾‾
        |  |  |  |  |
      0 +{-{-}+{-}{-}+{-}{-}+{-}{-}+{-}{-} t}
        |  |  |  |  |
     {-V    |\_\_|  |\_\_|}

Output (Triangular):
      0 +  /{  /  {-}{-} t}
        | /  {/  }
     {-V +/        }
\end{verbatim}

\textbf{Operation:}

\begin{itemize}
\tightlist
\item
  \textbf{Integration function}: Vout = -(1/RC)\intVin dt
\item
  \textbf{Square wave input}: Produces triangular output
\item
  \textbf{Ramp generation}: Constant input gives linear ramp
\end{itemize}

\end{solutionbox}
\begin{mnemonicbox}
``Integration creates Triangular from square''

\end{mnemonicbox}
\subsection*{Question 5(b) [4 marks]}\label{q5b}

\textbf{State advantage and disadvantage of push-pull arrangement of
power amplifier}

\begin{solutionbox}
Push-pull configuration uses two transistors operating
in complementary fashion for power amplification.

\textbf{Advantages:}


{\def\LTcaptype{none} % do not increment counter
\begin{longtable}[]{@{}
  >{\raggedright\arraybackslash}p{(\linewidth - 4\tabcolsep) * \real{0.3333}}
  >{\raggedright\arraybackslash}p{(\linewidth - 4\tabcolsep) * \real{0.2727}}
  >{\raggedright\arraybackslash}p{(\linewidth - 4\tabcolsep) * \real{0.3939}}@{}}
\toprule\noalign{}
\begin{minipage}[b]{\linewidth}\raggedright
Advantage
\end{minipage} & \begin{minipage}[b]{\linewidth}\raggedright
Benefit
\end{minipage} & \begin{minipage}[b]{\linewidth}\raggedright
Application
\end{minipage} \\
\midrule\noalign{}
\endhead
\bottomrule\noalign{}
\endlastfoot
\textbf{High Efficiency} & Up to 78.5\% & Battery operated \\
\textbf{No Transformer} & Compact design & Portable devices \\
\textbf{Low Distortion} & Better linearity & Audio systems \\
\textbf{Heat Distribution} & Shared between transistors & Thermal
management \\
\end{longtable}
}

\textbf{Disadvantages:}

{\def\LTcaptype{none} % do not increment counter
\begin{longtable}[]{@{}
  >{\raggedright\arraybackslash}p{(\linewidth - 4\tabcolsep) * \real{0.4062}}
  >{\raggedright\arraybackslash}p{(\linewidth - 4\tabcolsep) * \real{0.2812}}
  >{\raggedright\arraybackslash}p{(\linewidth - 4\tabcolsep) * \real{0.3125}}@{}}
\toprule\noalign{}
\begin{minipage}[b]{\linewidth}\raggedright
Disadvantage
\end{minipage} & \begin{minipage}[b]{\linewidth}\raggedright
Problem
\end{minipage} & \begin{minipage}[b]{\linewidth}\raggedright
Solution
\end{minipage} \\
\midrule\noalign{}
\endhead
\bottomrule\noalign{}
\endlastfoot
\textbf{Crossover Distortion} & Dead zone at zero crossing & Class AB
bias \\
\textbf{Component Matching} & Requires matched transistors & Careful
selection \\
\textbf{Thermal Runaway} & Temperature coefficient mismatch & Thermal
coupling \\
\end{longtable}
}

\textbf{Applications:}

\begin{itemize}
\tightlist
\item
  \textbf{Audio amplifiers}: High fidelity systems
\item
  \textbf{Motor drivers}: DC motor control
\item
  \textbf{RF amplifiers}: Communication systems
\end{itemize}

\end{solutionbox}
\begin{mnemonicbox}
``Push-Pull Provides Power but Problems exist''

\end{mnemonicbox}
\subsection*{Question 5(c) [7 marks]}\label{q5c}

\textbf{Draw and explain astable multivibrator using 555 timer IC.}

\begin{solutionbox}
Astable multivibrator generates continuous square wave
output without external trigger using 555 timer.

\textbf{Circuit Diagram:}

\begin{verbatim}
    +Vcc
      |
      RA
      |
   +{-{-}+{-}{-}+ (7)}
   |     |
   RB   (2)+(6) 555 (3){-{-}{-} Output}
   |         |     |
   C        (1)   (4)
   |         |     |
  GND       GND   +Vcc
\end{verbatim}

\textbf{Pin Connections:}

\begin{itemize}
\tightlist
\item
  \textbf{Pin 1}: Ground
\item
  \textbf{Pin 2}: Trigger (connected to pin 6)
\item
  \textbf{Pin 3}: Output
\item
  \textbf{Pin 4}: Reset (+Vcc)
\item
  \textbf{Pin 6}: Threshold
\item
  \textbf{Pin 7}: Discharge
\item
  \textbf{Pin 8}: +Vcc
\end{itemize}

\textbf{Operation:}

\begin{enumerate}
\tightlist
\item
  \textbf{Charging phase}: C charges through RA + RB
\item
  \textbf{Threshold reached}: At 2/3 Vcc, output goes LOW
\item
  \textbf{Discharging phase}: C discharges through RB
\item
  \textbf{Trigger reached}: At 1/3 Vcc, output goes HIGH
\item
  \textbf{Cycle repeats}: Continuous oscillation
\end{enumerate}

\textbf{Timing Equations:}

\begin{itemize}
\tightlist
\item
  \textbf{HIGH time}: t1 = 0.693(RA + RB)C
\item
  \textbf{LOW time}: t2 = 0.693(RB)C\\
\item
  \textbf{Total period}: T = t1 + t2 = 0.693(RA + 2RB)C
\item
  \textbf{Frequency}: f = 1.44/[(RA + 2RB)C]
\item
  \textbf{Duty cycle}: D = (RA + RB)/(RA + 2RB) \times 100\%
\end{itemize}

\textbf{Applications:}

\begin{itemize}
\tightlist
\item
  \textbf{Clock generation}: Digital systems
\item
  \textbf{LED flasher}: Blinking circuits
\item
  \textbf{Tone generation}: Audio oscillators
\item
  \textbf{PWM generation}: Motor speed control
\end{itemize}

\end{solutionbox}
\begin{mnemonicbox}
``Astable Always oscillates Automatically''

\end{mnemonicbox}
\subsection*{Question 5(a OR) [3
marks]}\label{question-5a-or-3-marks}

\textbf{Draw the block diagram of Op-amp and explain it.}

\begin{solutionbox}
Op-amp internal structure consists of multiple stages
for high gain and performance.

\textbf{Block Diagram:}

\begin{center}
\textbf{Mermaid Diagram (Code)}
\begin{verbatim}
{Shaded}
{Highlighting}[]
graph LR
    A[V+] {-{-}{} B[Differential Amplifier]}
    C[V{-] {-}{-}{} B}
    B {-{-}{} D[Intermediate Amplifier]}
    D {-{-}{} E[Output Stage]}
    E {-{-}{} F[Output]}
    G[Level Shifter] {-{-}{} E}
    D {-{-}{} G}
{Highlighting}
{Shaded}
\end{verbatim}
\end{center}

\textbf{Stage Functions:}


{\def\LTcaptype{none} % do not increment counter
\begin{longtable}[]{@{}
  >{\raggedright\arraybackslash}p{(\linewidth - 4\tabcolsep) * \real{0.2121}}
  >{\raggedright\arraybackslash}p{(\linewidth - 4\tabcolsep) * \real{0.3030}}
  >{\raggedright\arraybackslash}p{(\linewidth - 4\tabcolsep) * \real{0.4848}}@{}}
\toprule\noalign{}
\begin{minipage}[b]{\linewidth}\raggedright
Stage
\end{minipage} & \begin{minipage}[b]{\linewidth}\raggedright
Function
\end{minipage} & \begin{minipage}[b]{\linewidth}\raggedright
Characteristics
\end{minipage} \\
\midrule\noalign{}
\endhead
\bottomrule\noalign{}
\endlastfoot
\textbf{Differential Input} & High input impedance & Low offset, high
CMRR \\
\textbf{Intermediate Amplifier} & High voltage gain & Most of the
gain \\
\textbf{Level Shifter} & DC level adjustment & Couples AC stages \\
\textbf{Output Stage} & Low output impedance & Current buffer \\
\end{longtable}
}

\begin{itemize}
\tightlist
\item
  \textbf{High gain}: Typically 100,000 or more
\item
  \textbf{Wide bandwidth}: MHz range capability\\
\item
  \textbf{Low output impedance}: Drives various loads
\end{itemize}

\end{solutionbox}
\begin{mnemonicbox}
``Differential Input, Intermediate gain, Level shift,
Output buffer''

\end{mnemonicbox}
\subsection*{Question 5(b OR) [4
marks]}\label{question-5b-or-4-marks}

\textbf{Explain about the terms related to power amplifier. i)
Efficiency ii) Distortion.}

\begin{solutionbox}
These parameters determine power amplifier performance
and suitability for applications.

\textbf{i) Efficiency (η):}

\begin{itemize}
\tightlist
\item
  \textbf{Definition}: Ratio of AC output power to DC input power
\item
  \textbf{Formula}: η = Po(AC)/Pin(DC) \times 100\%
\item
  \textbf{Importance}: Determines heat dissipation and battery life
\end{itemize}

\textbf{Efficiency Comparison:}


{\def\LTcaptype{none} % do not increment counter
\begin{longtable}[]{@{}lll@{}}
\toprule\noalign{}
Class & Efficiency & Application \\
\midrule\noalign{}
\endhead
\bottomrule\noalign{}
\endlastfoot
\textbf{A} & 25\% & Low power, high fidelity \\
\textbf{B} & 78.5\% & Push-pull amplifiers \\
\textbf{AB} & 60-70\% & Audio amplifiers \\
\textbf{C} & \textgreater90\% & RF applications \\
\end{longtable}
}

\textbf{ii) Distortion:}

\begin{itemize}
\tightlist
\item
  \textbf{Definition}: Unwanted changes in output signal shape
\item
  \textbf{Types}: Harmonic, intermodulation, crossover
\item
  \textbf{Measurement}: Total Harmonic Distortion (THD)
\end{itemize}

\textbf{Distortion Sources:}

\begin{itemize}
\tightlist
\item
  \textbf{Nonlinearity}: Transistor characteristics
\item
  \textbf{Crossover}: Dead zone in push-pull
\item
  \textbf{Thermal effects}: Temperature variations
\end{itemize}

\end{solutionbox}
\begin{mnemonicbox}
``Efficiency measures Energy use, Distortion shows
signal Degradation''

\end{mnemonicbox}
\subsection*{Question 5(c OR) [7
marks]}\label{question-5c-or-7-marks}

\textbf{Draw pin diagram of 555 timer IC. Also draw circuit diagram of
two stage sequential timer using 555 timer IC.}

\begin{solutionbox}
555 timer is versatile IC used for timing applications
with standard 8-pin package.

\textbf{Pin Diagram:}

\begin{verbatim}
    +{-{-}{-}U{-}{-}{-}+}
  1 |       | 8  +Vcc
GND |  555  | 7  Discharge
    |       |
  2 |       | 6  Threshold
Trig|       |
    |       | 5  Control
  3 |       |
Out |       | 4  Reset
    +{-{-}{-}{-}{-}{-}{-}+}
\end{verbatim}

\textbf{Pin Functions:}


{\def\LTcaptype{none} % do not increment counter
\begin{longtable}[]{@{}lll@{}}
\toprule\noalign{}
Pin & Name & Function \\
\midrule\noalign{}
\endhead
\bottomrule\noalign{}
\endlastfoot
\textbf{1} & Ground & Common ground \\
\textbf{2} & Trigger & Starts timing cycle \\
\textbf{3} & Output & Timer output \\
\textbf{4} & Reset & Resets timer \\
\textbf{5} & Control & Voltage reference \\
\textbf{6} & Threshold & Stops timing cycle \\
\textbf{7} & Discharge & Discharges timing capacitor \\
\textbf{8} & Vcc & Supply voltage \\
\end{longtable}
}

\textbf{Two Stage Sequential Timer Circuit:}

\begin{verbatim}
First Stage (555A):
    +Vcc
      |
      R1
      |
   +{-{-}+{-}{-}+ (7)}
   |     |
   R2   (2)+(6) 555A (3){-{-}{-}+}
   |         |      |      |
   C1       (1)    (4)     |
   |         |      |      |
  GND       GND    +Vcc    |
                           |
Second Stage (555B):       |
    +Vcc                   |
      |                    |
      R3                   |
      |                    |
   +{-{-}+{-}{-}+ (7)             |}
   |     |                 |
   R4   (2) 555B (3){-{-}{-} Output}
   |     |   |      |
   C2   (6) (1)    (4)
   |     |   |      |
  GND{-{-}{-}{-}+{-}{-}GND    +Vcc}
         |
         +{-{-}{-}{-}{-}{-}{-}{-}{-}{-}+}
\end{verbatim}

\textbf{Operation:}

\begin{enumerate}
\tightlist
\item
  \textbf{First timer}: Operates in monostable mode
\item
  \textbf{Trigger applied}: First timer gives output pulse
\item
  \textbf{Output duration}: T1 = 1.1 \times R2 \times C1
\item
  \textbf{Second timer}: Triggered by first timer's output
\item
  \textbf{Sequential operation}: Second timer starts after first
  completes
\item
  \textbf{Total delay}: T1 + T2 where T2 = 1.1 \times R4 \times C2
\end{enumerate}

\textbf{Applications:}

\begin{itemize}
\tightlist
\item
  \textbf{Delay circuits}: Sequential switching
\item
  \textbf{Traffic lights}: Timed sequence control
\item
  \textbf{Industrial automation}: Process timing
\item
  \textbf{Motor control}: Start-stop sequences
\end{itemize}

\textbf{Timing Equations:}

\begin{itemize}
\tightlist
\item
  \textbf{Stage 1 delay}: T1 = 1.1 R2 C1
\item
  \textbf{Stage 2 delay}: T2 = 1.1 R4 C2
\item
  \textbf{Total sequence time}: Ttotal = T1 + T2
\end{itemize}

\textbf{Key Features:}

\begin{itemize}
\tightlist
\item
  \textbf{Independent timing}: Each stage separately adjustable
\item
  \textbf{Sequential operation}: No overlap between stages
\item
  \textbf{Reliable switching}: Clean digital transitions
\item
  \textbf{Easy design}: Simple component calculation
\end{itemize}

\end{solutionbox}
\begin{mnemonicbox}
``Sequential Stages Start Separately''

\end{mnemonicbox}

\end{document}
