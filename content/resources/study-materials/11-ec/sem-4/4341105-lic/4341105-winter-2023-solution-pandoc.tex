\documentclass[10pt,a4paper]{article}

% content/resources/templates/preamble.tex
\usepackage[margin=0.6in]{geometry}
\author{Milav Dabgar}
\usepackage{amsmath,amssymb,amsthm}
\usepackage{booktabs}
\usepackage{multirow}
\usepackage{xcolor}
\usepackage{tcolorbox}
\tcbuselibrary{breakable,skins}
\usepackage[colorlinks=true,linkcolor=blue]{hyperref}
\usepackage{titlesec}
\usepackage{enumitem}
\usepackage{tikz}
\usepackage{pgfplots}
\usepackage{circuitikz}
\usepackage[version=4]{mhchem}
\usepackage{longtable}
\usepackage{array}
\usepackage{float}
\usepackage{caption}
\usepackage{listings}

\lstset{
  basicstyle=\small\ttfamily,
  breaklines=true,
  breakatwhitespace=false,
  postbreak=\mbox{\textcolor{red}{$\hookrightarrow$}\space},
  float=false,
  numbers=left,
  numberstyle=\tiny\color{gray},
  numbersep=10pt,
  xleftmargin=2em,
  keywordstyle=\color{blue},
  commentstyle=\color{green!60!black},
  stringstyle=\color{purple},
  backgroundcolor=\color{gray!5},
  showstringspaces=false,
  tabsize=2,
  captionpos=b,
  keepspaces=true,
  columns=flexible
}

\pgfplotsset{compat=1.18}
\usetikzlibrary{shapes,arrows,positioning,calc,patterns,decorations.pathmorphing,decorations.markings,arrows.meta}

% Color scheme
\definecolor{headcolor}{RGB}{0,102,204}
\definecolor{keycolor}{RGB}{220,20,60}
\definecolor{solutioncolor}{RGB}{34,139,34}
\definecolor{mnemoniccolor}{RGB}{148,0,211}
\definecolor{codecolor}{RGB}{0,0,100}

% Spacing
\setlength{\parskip}{3pt}
\setlist[itemize]{nosep}
\setlist[enumerate]{nosep}

% Title formatting
\titleformat{\section}{\Large\bfseries\color{headcolor}}{\thesection}{1em}{}
\titleformat{\subsection}{\large\bfseries\color{headcolor}}{\thesubsection}{1em}{}

% Pandoc tightlist compatibility
\providecommand{\tightlist}{%
  \setlength{\itemsep}{0pt}\setlength{\parskip}{0pt}}

% Pandoc longtable compatibility
\newcounter{none}
\def\thenone{}


% content/resources/templates/english-boxes.tex
% This file is currently empty - it exists to maintain consistency with the import structure.
% Add custom environments here if needed in the future.


\begin{document}

\begin{center}
{\Huge\bfseries\color{headcolor} Subject Name Solutions}\\[5pt]
{\LARGE 4341105 -- Winter 2023}\\[3pt]
{\large Semester 1 Study Material}\\[3pt]
{\normalsize\textit{Detailed Solutions and Explanations}}
\end{center}

\vspace{10pt}

\subsection*{Question 1(a) [3 marks]}\label{q1a}

\textbf{What is negative feedback? List out advantages and disadvantages
of negative feedback.}

\begin{solutionbox}
Negative feedback is feeding a portion of output signal
back to the input with 180^\circ phase shift to reduce the input signal.

{\def\LTcaptype{none} % do not increment counter
\begin{longtable}[]{@{}ll@{}}
\toprule\noalign{}
Advantages & Disadvantages \\
\midrule\noalign{}
\endhead
\bottomrule\noalign{}
\endlastfoot
Increased stability & Reduced gain \\
Reduced distortion & Complex circuit design \\
Increased bandwidth & More components required \\
Reduced noise & Higher power consumption \\
\end{longtable}
}

\end{solutionbox}
\begin{mnemonicbox}
``SIRS'' - Stability Improved, Reduced distortion,
Sensitivity decreased

\end{mnemonicbox}
\subsection*{Question 1(b) [4 marks]}\label{q1b}

\textbf{Describe the effect of negative feedback on frequency response
and distortion of an amplifier.}

\begin{solutionbox}
Negative feedback improves both frequency response and
reduces distortion in amplifiers.

\textbf{Diagram:}

\begin{center}
\textbf{Mermaid Diagram (Code)}
\begin{verbatim}
{Shaded}
{Highlighting}[]
graph TD
    A[Amplifier without feedback] {-{-}{} B[Narrow bandwidth]}
    C[Amplifier with negative feedback] {-{-}{} D[Wider bandwidth]}
    E[Input with harmonics] {-{-}{} F[Amplifier without feedback] {-}{-}{} G[Output with more harmonics]}
    E {-{-}{} H[Amplifier with negative feedback] {-}{-}{} I[Output with fewer harmonics]}
{Highlighting}
{Shaded}
\end{verbatim}
\end{center}

{\def\LTcaptype{none} % do not increment counter
\begin{longtable}[]{@{}lll@{}}
\toprule\noalign{}
Effect on & Without feedback & With negative feedback \\
\midrule\noalign{}
\endhead
\bottomrule\noalign{}
\endlastfoot
Frequency response & Narrow bandwidth & Wider bandwidth \\
Distortion & Higher harmonics & Reduced harmonics \\
\end{longtable}
}

\end{solutionbox}
\begin{mnemonicbox}
``WIDE'' - With negative feedback, Improved response,
Distortion reduced, Extended bandwidth

\end{mnemonicbox}
\subsection*{Question 1(c) [7 marks]}\label{q1c}

\textbf{Derive an equation for overall gain of negative feedback voltage
amplifier.}

\begin{solutionbox}
The equation for overall gain of negative feedback
voltage amplifier can be derived as follows:

\textbf{Diagram:}

\begin{verbatim}
    Input +{-{-}{-}{-}{-}+      +{-}{-}{-}{-}{-}{-}{-}+}
    Vi {-{-}|  Σ  |{-}{-}{-}{-}{-}|       |{-}{-}{-}{-} Vo (Output)}
          +{-{-}{-}{-}{-}+      |   A   |}
             \^{         |       |}
             |         +{-{-}{-}{-}{-}{-}{-}+}
             |             |
             |         +{-{-}{-}{-}{-}{-}{-}+}
             +{-{-}{-}{-}{-}{-}{-}{-}{-}|   β   |}
                       +{-{-}{-}{-}{-}{-}{-}+}
\end{verbatim}

\begin{itemize}
\tightlist
\item
  \textbf{Input equation}: V' = Vi - βVo
\item
  \textbf{Output equation}: Vo = AV'
\item
  \textbf{Substituting}: Vo = A(Vi - βVo)
\item
  \textbf{Solving for Vo}: Vo = AVi - AβVo
\item
  \textbf{Rearranging}: Vo(1 + Aβ) = AVi
\item
  \textbf{Final equation}: Vo/Vi = A/(1 + Aβ) = Af
\end{itemize}

\end{solutionbox}
\begin{mnemonicbox}
``LOOP'' - Look at Original Open-loop gain and
Proceed with feedback

\end{mnemonicbox}
\subsection*{Question 1(c) OR [7
marks]}\label{q1c}

\textbf{Compare voltage shunt amplifier and current series amplifier.}

\begin{solutionbox}

{\def\LTcaptype{none} % do not increment counter
\begin{longtable}[]{@{}lll@{}}
\toprule\noalign{}
Parameter & Voltage Shunt Amplifier & Current Series Amplifier \\
\midrule\noalign{}
\endhead
\bottomrule\noalign{}
\endlastfoot
Input & Voltage & Current \\
Output & Current & Voltage \\
Feedback network connection & Parallel at input & Series at input \\
Input impedance & Decreased & Increased \\
Output impedance & Increased & Decreased \\
Gain & Current gain decreases & Voltage gain decreases \\
Application & Current amplification & Voltage amplification \\
\end{longtable}
}

\textbf{Diagram:}

\begin{verbatim}
graph TB
    subgraph "Voltage Shunt"
        A1[Input voltage] {-{-} B1[Shunt connected β]}
        B1 {-{-} C1[Amplifier]}
        C1 {-{-} D1[Output current]}
    end
    subgraph "Current Series"
        A2[Input current] {-{-} B2[Series connected β]}
        B2 {-{-} C2[Amplifier]}
        C2 {-{-} D2[Output voltage]}
    end
\end{verbatim}

\end{solutionbox}
\begin{mnemonicbox}
``VICS'' - Voltage shunt In, Current out Series has
opposite

\end{mnemonicbox}
\subsection*{Question 2(a) [3 marks]}\label{q2a}

\textbf{Discuss Barkhausen's criteria for oscillation.}

\begin{solutionbox}
Barkhausen's criteria states that for sustained
oscillations, the following conditions must be met:

{\def\LTcaptype{none} % do not increment counter
\begin{longtable}[]{@{}ll@{}}
\toprule\noalign{}
Criteria & Requirement \\
\midrule\noalign{}
\endhead
\bottomrule\noalign{}
\endlastfoot
Loop gain & \textbar Aβ\textbar{} = 1 (magnitude equals 1) \\
Phase shift & Total phase shift around loop = 0^\circ or 360^\circ \\
\end{longtable}
}

\textbf{Diagram:}

\begin{verbatim}
    +{-{-}{-}{-}{-}{-}{-}+      +{-}{-}{-}{-}{-}{-}{-}+}
    |       |{-{-}{-}{-}{-}|       |{-}{-}+}
    |   A   |      |   β   |  |
    |       |{{-}{-}{-}{-}{-}|       |{-}+}
    +{-{-}{-}{-}{-}{-}{-}+      +{-}{-}{-}{-}{-}{-}{-}+}
\end{verbatim}

\end{solutionbox}
\begin{mnemonicbox}
``LOOP'' - Loop gain One, Oscillation needs Phase
shift zero

\end{mnemonicbox}
\subsection*{Question 2(b) [4 marks]}\label{q2b}

\textbf{Draw circuit diagram of Hartley oscillator and Colpitts
oscillator.}

\begin{solutionbox}

\textbf{Hartley Oscillator:}

\begin{verbatim}
    +{-{-}{-}{-}{-}+     +{-}{-}{-}{-}||{-}{-}{-}{-}{-}+}
    |     |     |           |
    +     |     C1          |
   ===    +{-{-}{-}{-}{-}+           |}
   GND    |     |           |
          |     Z    +{-{-}{-}{-}{-}{-}+}
          |     Z    |      |
          |     Z    |      |
          +{-{-}{-}{-}{-}+    |      |}
            L1  |    |  L2  |
                |    |      |
                +{-{-}{-}{-}+{-}{-}{-}{-}{-}{-}+}
                |    |
                |    |
          +{-{-}{-}{-}{-}+    +{-}{-}{-}{-}{-}+}
          |     |    |     |
          |  Q  |    |     |
          |     |    |     |
          +{-{-}+{-}{-}+    |     |}
             |       |     |
             +{-{-}{-}{-}{-}{-}{-}+     |}
             |             |
            === GND        |
                           |
    +{-{-}{-}{-}||{-}{-}{-}{-}{-}{-}{-}{-}{-}{-}{-}{-}{-}{-}{-}{-}+}
    |       C2
    |
   === GND
\end{verbatim}

\textbf{Colpitts Oscillator:}

\begin{verbatim}
    +{-{-}{-}{-}{-}+     +{-}{-}{-}{-}||{-}{-}{-}{-}{-}+}
    |     |     |     C1    |
    +     |     |           |
   ===    +{-{-}{-}{-}{-}+           |}
   GND    |     |           |
          |     |    +{-{-}{-}{-}{-}{-}+}
          |     |    |      |
          |     Z    |      |
          +{-{-}{-}{-}{-}Z    |      |}
            L   Z    |      |
                |    |      |
                |    |      |
                +{-{-}{-}{-}+{-}{-}{-}{-}{-}{-}+}
                |    |
                |    |
          +{-{-}{-}{-}{-}+    +{-}{-}{-}{-}{-}+}
          |     |    |     |
          |  Q  |    |     |
          |     |    |     |
          +{-{-}+{-}{-}+    |     |}
             |       |     |
             +{-{-}{-}{-}{-}{-}{-}+     |}
             |             |
            === GND        |
             |             |
             +{-{-}{-}{-}{-}{-}{-}{-}{-}{-}{-}{-}{-}+}
             |
            ===
            C2
            ===
            GND
\end{verbatim}

\end{solutionbox}
\begin{mnemonicbox}
``HaLs CoCs'' - Hartley has inductors in series,
Colpitts has Capacitors in series

\end{mnemonicbox}
\subsection*{Question 2(c) [7 marks]}\label{q2c}

\textbf{Explain UJT as a relaxation oscillator.}

\begin{solutionbox}
UJT (Unijunction Transistor) works as a relaxation
oscillator by repeatedly charging and discharging a capacitor.

\textbf{Diagram:}

\begin{verbatim}
         RB1
    B2 +{-{-}{-}///{-}{-}{-}+ VCC}
         |          |
         |          |
         |          |
    +{-{-}{-}{-}|          |}
    |    |          |
    |    |  UJT     |
    |    |          |
    |    |          |
    |    |          |
    |    +{-{-}{-}{-}{-}{-}{-}{-}{-}{-}+ B1}
    |    |
    |    |
    |    |
    C    R
    |    |
    |    |
    +{-{-}{-}{-}+{-}{-}{-}{-}+ GND}
\end{verbatim}

{\def\LTcaptype{none} % do not increment counter
\begin{longtable}[]{@{}
  >{\raggedright\arraybackslash}p{(\linewidth - 2\tabcolsep) * \real{0.3500}}
  >{\raggedright\arraybackslash}p{(\linewidth - 2\tabcolsep) * \real{0.6500}}@{}}
\toprule\noalign{}
\begin{minipage}[b]{\linewidth}\raggedright
Phase
\end{minipage} & \begin{minipage}[b]{\linewidth}\raggedright
Description
\end{minipage} \\
\midrule\noalign{}
\endhead
\bottomrule\noalign{}
\endlastfoot
Charging & Capacitor charges through R until voltage reaches VP (peak
voltage) \\
Firing & UJT turns ON when emitter voltage reaches VP \\
Discharge & Capacitor discharges rapidly through UJT \\
Reset & Voltage falls below valley voltage, UJT turns OFF, cycle
repeats \\
\end{longtable}
}

\begin{itemize}
\tightlist
\item
  \textbf{Intrinsic standoff ratio}: η = RB1/(RB1+RB2)
\item
  \textbf{Peak voltage}: VP = η\timesVBB + VD
\item
  \textbf{Frequency}: f = 1/[R\timesC\timesln(1/(1-η))]
\end{itemize}

\end{solutionbox}
\begin{mnemonicbox}
``CFDR'' - Charge, Fire, Discharge, Repeat

\end{mnemonicbox}
\subsection*{Question 2(a) OR [3
marks]}\label{q2a}

\textbf{Classify Oscillators.}

\begin{solutionbox}

{\def\LTcaptype{none} % do not increment counter
\begin{longtable}[]{@{}ll@{}}
\toprule\noalign{}
Classification & Types \\
\midrule\noalign{}
\endhead
\bottomrule\noalign{}
\endlastfoot
Based on feedback & RC, LC, Crystal \\
Based on waveform & Sinusoidal, Non-sinusoidal \\
Based on frequency & Audio, Radio, VHF, UHF \\
Based on circuit & Hartley, Colpitts, Wien-bridge, RC-phase shift \\
\end{longtable}
}

\textbf{Diagram:}

\begin{center}
\textbf{Mermaid Diagram (Code)}
\begin{verbatim}
{Shaded}
{Highlighting}[]
graph TD
    A[Oscillators] {-{-}{} B[RC Oscillators]}
    A {-{-}{} C[LC Oscillators]}
    A {-{-}{} D[Crystal Oscillators]}
    A {-{-}{} E[Relaxation Oscillators]}
    B {-{-}{} F[Wien Bridge]}
    B {-{-}{} G[Phase Shift]}
    C {-{-}{} H[Hartley]}
    C {-{-}{} I[Colpitts]}
    C {-{-}{} J[Clapp]}
    E {-{-}{} K[UJT based]}
    E {-{-}{} L[IC 555 based]}
{Highlighting}
{Shaded}
\end{verbatim}
\end{center}

\end{solutionbox}
\begin{mnemonicbox}
``SRLC'' - Sine waves from RC, LC, and Crystal
oscillators

\end{mnemonicbox}
\subsection*{Question 2(b) OR [4
marks]}\label{q2b}

\textbf{Explain construction of UJT with its symbol.}

\begin{solutionbox}
UJT (Unijunction Transistor) consists of a lightly
doped N-type silicon bar with electrical connections at both ends
(bases) and a P-type emitter junction.

\textbf{Diagram:}

\begin{verbatim}
    Symbol:               Structure:
    
      B2                     B2
       |                      |
       |                  +{-{-}{-}+{-}{-}{-}+}
       |                  |   |   |
       +{-{-}{-}+          +{-}{-}{-}+{-}{-}{-}+{-}{-}{-}+{-}{-}{-}+}
           |          |   |   |   |   |
           +          |   | N{-type    |}
           |          |   |   |   |   |
       +{-{-}{-}+          |   +{-}{-}{-}+{-}{-}{-}+   |}
       |              |       |       |
       |              |       |       |
       E              |       | P     |
       |              |       |       |
       |              +{-{-}{-}{-}{-}{-}{-}+{-}{-}{-}{-}{-}{-}{-}+}
       |                      |
       |                      |
       |                      |
      B1                     B1
\end{verbatim}

{\def\LTcaptype{none} % do not increment counter
\begin{longtable}[]{@{}ll@{}}
\toprule\noalign{}
Component & Description \\
\midrule\noalign{}
\endhead
\bottomrule\noalign{}
\endlastfoot
Base 1 (B1) & Connected to one end of N-type bar \\
Base 2 (B2) & Connected to other end of N-type bar \\
Emitter (E) & Connected to P-type region diffused into N-type bar \\
RB1 & Resistance between emitter and B1 \\
RB2 & Resistance between emitter and B2 \\
\end{longtable}
}

\end{solutionbox}
\begin{mnemonicbox}
``BEB'' - Bases at Ends, Emitter in Between

\end{mnemonicbox}
\subsection*{Question 2(c) OR [7
marks]}\label{q2c}

\textbf{Explain working of Wien Bridge oscillator circuit. List out its
application.}

\begin{solutionbox}
Wien Bridge oscillator produces sine waves using RC
network for positive feedback and negative feedback for amplitude
stability.

\textbf{Diagram:}

\begin{center}
\textbf{Mermaid Diagram (Code)}
\begin{verbatim}
{Shaded}
{Highlighting}[]
graph TD
    subgraph "Positive Feedback"
        R1 {-{-}{-} C1}
        R2 {-{-}{-} C2}
    end
    subgraph "Negative Feedback"
        R3
        R4
    end
    A[Op{-Amp] {-}{-}{} Output}
    R1 \& C1 \& R2 \& C2 {-{-}{} A}
    A {-{-}{} R3 {-}{-}{} R4 {-}{-}{} A}
{Highlighting}
{Shaded}
\end{verbatim}
\end{center}

{\def\LTcaptype{none} % do not increment counter
\begin{longtable}[]{@{}ll@{}}
\toprule\noalign{}
Component & Function \\
\midrule\noalign{}
\endhead
\bottomrule\noalign{}
\endlastfoot
R1, C1 (series) & Positive feedback, phase lead \\
R2, C2 (parallel) & Positive feedback, phase lag \\
R3, R4 & Negative feedback, amplitude control \\
Op-Amp & Active amplifier element \\
\end{longtable}
}

\textbf{Applications:}

\begin{itemize}
\tightlist
\item
  Audio signal generators
\item
  Function generators
\item
  Musical instrument tuning
\item
  Test equipment
\item
  Filter circuits
\end{itemize}

\end{solutionbox}
\begin{mnemonicbox}
``APPS'' - Audio Production, Pure Sine waves, Stable
frequency

\end{mnemonicbox}
\subsection*{Question 3(a) [3 marks]}\label{q3a}

\textbf{Differentiate between voltage and power amplifier.}

\begin{solutionbox}

{\def\LTcaptype{none} % do not increment counter
\begin{longtable}[]{@{}lll@{}}
\toprule\noalign{}
Parameter & Voltage Amplifier & Power Amplifier \\
\midrule\noalign{}
\endhead
\bottomrule\noalign{}
\endlastfoot
Primary function & Increases voltage level & Increases power level \\
Output & Low current capability & High current capability \\
Efficiency & Not critical & Critical parameter \\
Heat dissipation & Low & High, needs heat sink \\
Biasing & Class A typically & Class A, B, AB, or C \\
Applications & Pre-amplification stages & Driving speakers, motors \\
\end{longtable}
}

\end{solutionbox}
\begin{mnemonicbox}
``VICE'' - Voltage amplifiers Increase voltage,
Current not important, Efficiency not critical

\end{mnemonicbox}
\subsection*{Question 3(b) [4 marks]}\label{q3b}

\textbf{Derive an equation for Efficiency of class B push pull
amplifier.}

\begin{solutionbox}
Efficiency (η) of a Class B push-pull amplifier is
derived as follows:

\textbf{Diagram:}

\begin{verbatim}
          +VCC
           |
           |
    +{-{-}{-}{-}{-}{-}+{-}{-}{-}{-}{-}{-}+}
    |             |
    |      T1     |
   +++            |
    |             |
   +++     +{-{-}{-}{-}{-}{-}+{-}{-}{-}{-}{-}{-}+}
    |      |      |      |
Input+{-{-}{-}+ |      |      +{-}{-}{-}+Output}
    |      |      |      |
   +++     +{-{-}{-}{-}{-}{-}+{-}{-}{-}{-}{-}{-}+}
    |             |
    |      T2     |
    +{-{-}{-}{-}{-}{-}+{-}{-}{-}{-}{-}{-}+}
           |
           |
          {-VCC}
\end{verbatim}

\begin{itemize}
\tightlist
\item
  \textbf{AC power output}: P_{0} = Vrms \times Irms = (Vm/\sqrt2) \times (Im/\sqrt2) = Vm \times
  Im/2
\item
  \textbf{DC power input}: PDC = VCC \times IDC = VCC \times (2\timesIm/π)
\item
  \textbf{Efficiency}: η = P_{0}/PDC = (Vm\timesIm/2)/(VCC\times2\timesIm/π) =
  (Vm\timesπ)/(4\timesVCC)
\item
  \textbf{For maximum swing}: Vm = VCC, so η = π/4 = 78.5\%
\end{itemize}

\end{solutionbox}
\begin{mnemonicbox}
``POP'' - Push-pull Output Power = π/4 or 78.5\%

\end{mnemonicbox}
\subsection*{Question 3(c) [7 marks]}\label{q3c}

\textbf{Explain working of Class-B Push Pull Amplifiers along with
waveform.}

\begin{solutionbox}
Class B push-pull amplifier uses two transistors to
amplify opposite halves of the input waveform.

\textbf{Diagram:}

\begin{center}
\textbf{Mermaid Diagram (Code)}
\begin{verbatim}
{Shaded}
{Highlighting}[]
graph LR
    A[Input Signal] {-{-}{} B[Driver Stage]}
    B {-{-}{} C[Upper Transistor]}
    B {-{-}{} D[Lower Transistor]}
    C {-{-}{} E[Output Transformer]}
    D {-{-}{} E}
    E {-{-}{} F[Output Signal]}

    subgraph "Waveforms"
    direction LR
    G[Input] {-{-}{-} H[T1 Conducts] {-}{-}{-} I[T2 Conducts]}
    end
{Highlighting}
{Shaded}
\end{verbatim}
\end{center}

{\def\LTcaptype{none} % do not increment counter
\begin{longtable}[]{@{}ll@{}}
\toprule\noalign{}
Phase & Description \\
\midrule\noalign{}
\endhead
\bottomrule\noalign{}
\endlastfoot
Positive half & Upper transistor (T1) conducts, T2 is off \\
Negative half & Lower transistor (T2) conducts, T1 is off \\
Crossover & Both transistors are near cutoff, causing distortion \\
\end{longtable}
}

\textbf{Key points:}

\begin{itemize}
\tightlist
\item
  \textbf{Efficiency}: Approximately 78.5\% (π/4)
\item
  \textbf{Conduction angle}: 180^\circ for each transistor
\item
  \textbf{Crossover distortion}: Due to both transistors being off near
  zero crossing
\item
  \textbf{Advantages}: Higher efficiency, less heat, suitable for high
  power
\end{itemize}

\end{solutionbox}
\begin{mnemonicbox}
``HOPE'' - Half cycle Operation, Push-pull,
Efficiency high

\end{mnemonicbox}
\subsection*{Question 3(a) OR [3
marks]}\label{q3a}

\textbf{Explain Classification of Power amplifier.}

\begin{solutionbox}

{\def\LTcaptype{none} % do not increment counter
\begin{longtable}[]{@{}llll@{}}
\toprule\noalign{}
Class & Conduction Angle & Efficiency & Distortion \\
\midrule\noalign{}
\endhead
\bottomrule\noalign{}
\endlastfoot
Class A & 360^\circ & 25-30\% & Low \\
Class B & 180^\circ & 78.5\% & Medium \\
Class AB & 180^\circ-360^\circ & 50-78.5\% & Low-Medium \\
Class C & \textless180^\circ & \textgreater78.5\% & High \\
\end{longtable}
}

\textbf{Diagram:}

\begin{center}
\textbf{Mermaid Diagram (Code)}
\begin{verbatim}
{Shaded}
{Highlighting}[]
graph TD
    A[Power Amplifiers] {-{-}{} B[Class A]}
    A {-{-}{} C[Class B]}
    A {-{-}{} D[Class AB]}
    A {-{-}{} E[Class C]}
    B {-{-}{} F[Low distortion, Low efficiency]}
    C {-{-}{} G[Medium distortion, High efficiency]}
    D {-{-}{} H[Low distortion, Medium efficiency]}
    E {-{-}{} I[High distortion, Very high efficiency]}
{Highlighting}
{Shaded}
\end{verbatim}
\end{center}

\end{solutionbox}
\begin{mnemonicbox}
``ABCE'' - As Biasing Changes, Efficiency increases

\end{mnemonicbox}
\subsection*{Question 3(b) OR [4
marks]}\label{q3b}

\textbf{Derive an equation for Efficiency of class A power amplifier.}

\begin{solutionbox}
Efficiency of Class A power amplifier is derived as
follows:

\textbf{Diagram:}

\begin{verbatim}
     +VCC
       |
       |
       Z
       Z RL
       Z
       |
       +{-{-}{-}+Output}
       |
       |
       Q
       |
       |
     Input
       |
      GND
\end{verbatim}

\begin{itemize}
\tightlist
\item
  \textbf{Maximum AC power output}: P_{0} = (Vrms)^{2}/RL = (VCC/2\sqrt2)^{2}/RL =
  VCC^{2}/8RL
\item
  \textbf{DC power input}: PDC = VCC \times IDC = VCC \times (VCC/2RL) = VCC^{2}/2RL
\item
  \textbf{Efficiency}: η = P_{0}/PDC = (VCC^{2}/8RL)/(VCC^{2}/2RL) = 1/4 = 25\%
\end{itemize}

\end{solutionbox}
\begin{mnemonicbox}
``ONE'' - Output Never Exceeds 25\% efficiency in
Class A

\end{mnemonicbox}
\subsection*{Question 3(c) OR [7
marks]}\label{q3c}

\textbf{Explain working of Class-A transformer coupled Amplifiers along
with waveform.}

\begin{solutionbox}
Class A transformer coupled amplifier conducts for the
full input cycle (360^\circ) using a transformer for output coupling.

\textbf{Diagram:}

\begin{verbatim}
     +VCC
       |
       |
    +{-{-}+{-}{-}+}
    |     |
    | Pri |
    |     |
    +{-{-}+{-}{-}+}
       |
       +{-{-}{-}+}
       |   |
    Q  |   |
       |   |
       |   +{-{-}+Output}
       |      |
      === C   +{-{-}+{-}{-}+}
       |      |     |
      GND     | Sec |
              |     |
              +{-{-}+{-}{-}+}
                 |
                GND
\end{verbatim}

{\def\LTcaptype{none} % do not increment counter
\begin{longtable}[]{@{}ll@{}}
\toprule\noalign{}
Component & Function \\
\midrule\noalign{}
\endhead
\bottomrule\noalign{}
\endlastfoot
Transformer & Matches impedance, removes DC, provides isolation \\
Transistor & Conducts for full 360^\circ cycle \\
Capacitor & AC coupling \\
VCC & DC power supply \\
\end{longtable}
}

\textbf{Waveform characteristics:}

\begin{itemize}
\tightlist
\item
  Input and output waveforms are in phase
\item
  No crossover distortion
\item
  Full cycle amplification
\item
  Low efficiency (25\%)
\item
  Low distortion
\end{itemize}

\end{solutionbox}
\begin{mnemonicbox}
``FACT'' - Full cycle Amplification in Class-a with
Transformer

\end{mnemonicbox}
\subsection*{Question 4(a) [3 marks]}\label{q4a}

\textbf{Define (i) CMRR (ii) Slew Rate}

\begin{solutionbox}

{\def\LTcaptype{none} % do not increment counter
\begin{longtable}[]{@{}
  >{\raggedright\arraybackslash}p{(\linewidth - 4\tabcolsep) * \real{0.2895}}
  >{\raggedright\arraybackslash}p{(\linewidth - 4\tabcolsep) * \real{0.3158}}
  >{\raggedright\arraybackslash}p{(\linewidth - 4\tabcolsep) * \real{0.3947}}@{}}
\toprule\noalign{}
\begin{minipage}[b]{\linewidth}\raggedright
Parameter
\end{minipage} & \begin{minipage}[b]{\linewidth}\raggedright
Definition
\end{minipage} & \begin{minipage}[b]{\linewidth}\raggedright
Typical Value
\end{minipage} \\
\midrule\noalign{}
\endhead
\bottomrule\noalign{}
\endlastfoot
CMRR & Common Mode Rejection Ratio, the ratio of differential gain to
common mode gain & 90 dB (IC 741) \\
Slew Rate & Maximum rate of change of output voltage per unit of time &
0.5 V/μs (IC 741) \\
\end{longtable}
}

\textbf{CMRR}: CMRR = 20 log_{1}_{0}(Ad/Acm) where Ad is differential gain and
Acm is common mode gain

\textbf{Slew Rate}: SR = dVout/dt (V/μs)

\end{solutionbox}
\begin{mnemonicbox}
``CRiSp'' - CMRR Rejects common signals, Slew Rate
limits speed

\end{mnemonicbox}
\subsection*{Question 4(b) [4 marks]}\label{q4b}

\textbf{Explain inverting amplifier of operational amplifiers with
sketch.}

\begin{solutionbox}
Inverting amplifier provides gain with 180^\circ phase shift
using negative feedback.

\textbf{Diagram:}

\begin{verbatim}
        Rf
    +{-{-}{-}///{-}{-}{-}+}
    |            |
    |            |
    |    +{-{-}{-}{-}{-}{-}{-}+}
    |    |       |
    |    |   +   |
Vin +{-{-}{-}{-}+{-}{-}{-}+   +{-}{-}{-}{-}+ Vout}
    Ri   |   {-   |}
         |       |
         +{-{-}{-}{-}{-}{-}{-}+}
              |
              |
             === GND
\end{verbatim}

{\def\LTcaptype{none} % do not increment counter
\begin{longtable}[]{@{}ll@{}}
\toprule\noalign{}
Component & Function \\
\midrule\noalign{}
\endhead
\bottomrule\noalign{}
\endlastfoot
Ri & Input resistor \\
Rf & Feedback resistor \\
Op-Amp & Amplifies signal with high gain \\
\end{longtable}
}

\textbf{Key equations:}

\begin{itemize}
\tightlist
\item
  \textbf{Gain}: A = -Rf/Ri
\item
  \textbf{Input impedance}: Z = Ri
\item
  \textbf{Bandwidth}: Depends on op-amp and gain
\end{itemize}

\end{solutionbox}
\begin{mnemonicbox}
``IRON'' - Inverting, Resistance ratio gives gain,
Output Negative phase

\end{mnemonicbox}
\subsection*{Question 4(c) [7 marks]}\label{q4c}

\textbf{Explain Op-amp as a Summing amplifier.}

\begin{solutionbox}
Summing amplifier adds multiple input signals with
weighted contributions.

\textbf{Diagram:}

\begin{center}
\textbf{Mermaid Diagram (Code)}
\begin{verbatim}
{Shaded}
{Highlighting}[]
graph LR
    V1[V1] {-{-}{}|R1| A(({}+))}
    V2[V2] {-{-}{}|R2| A}
    V3[V3] {-{-}{}|R3| A}
    A {-{-}{-} B[Op{-}Amp]}
    B {-{-}{-} C[Vout]}
    C {-.{-}{}|Rf| A}
{Highlighting}
{Shaded}
\end{verbatim}
\end{center}

\textbf{Circuit:}

\begin{verbatim}
       R1             Rf
    +{-{-}///{-}{-}+{-}{-}{-}///{-}{-}{-}+}
    |          |            |
V1  |          |            |
    +          |    +{-{-}{-}{-}{-}{-}{-}+}
               |    |       |
    +{-{-}///{-}{-}+    |   +   |}
    |          |{-{-}{-}{-}+{-}{-}{-}+   +{-}{-}{-}{-}+ Vout}
V2  +   R2     |    |   {-   |}
               |    |       |
    +{-{-}///{-}{-}+    +{-}{-}{-}{-}{-}{-}{-}+}
    |          |        |
V3  +   R3     |        |
               |       === GND
              === GND
\end{verbatim}

{\def\LTcaptype{none} % do not increment counter
\begin{longtable}[]{@{}ll@{}}
\toprule\noalign{}
Parameter & Value \\
\midrule\noalign{}
\endhead
\bottomrule\noalign{}
\endlastfoot
Output voltage & Vout = -(Rf/R1)V1 - (Rf/R2)V2 - (Rf/R3)V3 \ldots{} \\
Gain for each input & -Rf/Rn where Rn is input resistor \\
Equal weight summing & All input resistors equal: R1 = R2 = R3 = Rf \\
\end{longtable}
}

\textbf{Applications:}

\begin{itemize}
\tightlist
\item
  Audio mixers
\item
  Signal processing
\item
  Analog computers
\item
  Weighted averages
\end{itemize}

\end{solutionbox}
\begin{mnemonicbox}
``SARI'' - Summing Amplifier Requires Inverting
configuration

\end{mnemonicbox}
\subsection*{Question 4(a) OR [3
marks]}\label{q4a}

\textbf{Sketch basic Block diagram of an operational amplifier.}

\begin{solutionbox}

\textbf{Diagram:}

\begin{center}
\textbf{Mermaid Diagram (Code)}
\begin{verbatim}
{Shaded}
{Highlighting}[]
graph LR
    A[Input Differential Stage] {-{-}{} B[Intermediate Stage]}
    B {-{-}{} C[Level Shifter]}
    C {-{-}{} D[Output Stage]}
    E[Bias Circuit] {-{-}{} A}
    E {-{-}{} B}
    E {-{-}{} C}
    E {-{-}{} D}
{Highlighting}
{Shaded}
\end{verbatim}
\end{center}

{\def\LTcaptype{none} % do not increment counter
\begin{longtable}[]{@{}
  >{\raggedright\arraybackslash}p{(\linewidth - 2\tabcolsep) * \real{0.4118}}
  >{\raggedright\arraybackslash}p{(\linewidth - 2\tabcolsep) * \real{0.5882}}@{}}
\toprule\noalign{}
\begin{minipage}[b]{\linewidth}\raggedright
Stage
\end{minipage} & \begin{minipage}[b]{\linewidth}\raggedright
Function
\end{minipage} \\
\midrule\noalign{}
\endhead
\bottomrule\noalign{}
\endlastfoot
Input differential stage & High input impedance, rejects common mode
signals \\
Intermediate stage & High gain, frequency compensation \\
Level shifter & Shifts DC level for output stage \\
Output stage & Low output impedance, current amplification \\
Bias circuit & Provides proper operating points \\
\end{longtable}
}

\end{solutionbox}
\begin{mnemonicbox}
``DILO'' - Differential Input, Level shifting, Output
amplification

\end{mnemonicbox}
\subsection*{Question 4(b) OR [4
marks]}\label{q4b}

\textbf{Explain non inverting amplifier of operational amplifiers with
sketch.}

\begin{solutionbox}
Non-inverting amplifier provides gain without phase
inversion using negative feedback.

\textbf{Diagram:}

\begin{verbatim}
              +{-{-}{-}{-}{-}{-}{-}+}
              |       |
              |   +   |
Vin +{-{-}{-}{-}{-}{-}{-}{-}{-}){-}{-}{-}+   +{-}{-}{-}{-}+ Vout}
              |   {-   |}
              |       |
              +{-{-}{-}+{-}{-}{-}+}
                  |
                  |
         Ri       |
    +{-{-}{-}///{-}{-}{-}{-}+}
    |              
    |              
    |    Rf        
    +{-{-}{-}///{-}{-}{-}{-}+}
    |             |
    |             |
   === GND        |
                  |
                  +
\end{verbatim}

{\def\LTcaptype{none} % do not increment counter
\begin{longtable}[]{@{}ll@{}}
\toprule\noalign{}
Parameter & Value \\
\midrule\noalign{}
\endhead
\bottomrule\noalign{}
\endlastfoot
Gain & A = 1 + Rf/Ri \\
Input impedance & Very high (depends on op-amp) \\
Phase & In-phase with input \\
Common application & Voltage follower (when Rf=0, Ri=\infty) \\
\end{longtable}
}

\end{solutionbox}
\begin{mnemonicbox}
``NIPS'' - Non-inverting, Input and output In Phase,
Same polarity

\end{mnemonicbox}
\subsection*{Question 4(c) OR [7
marks]}\label{q4c}

\textbf{Explain Op-amp as an Integrator.}

\begin{solutionbox}
Op-amp integrator produces output proportional to the
time integral of the input.

\textbf{Diagram:}

\begin{verbatim}
           C
    +{-{-}{-}{-}{-}{-}||{-}{-}{-}{-}{-}{-}+}
    |              |
    |              |
    |      +{-{-}{-}{-}{-}{-}{-}+}
    |      |       |
    |      |   +   |
Vin +{-{-}{-}{-}{-}{-}+{-}{-}{-}+   +{-}{-}{-}{-}+ Vout}
    R      |   {-   |}
           |       |
           +{-{-}{-}{-}{-}{-}{-}+}
                |
                |
               === GND
\end{verbatim}

{\def\LTcaptype{none} % do not increment counter
\begin{longtable}[]{@{}ll@{}}
\toprule\noalign{}
Parameter & Formula \\
\midrule\noalign{}
\endhead
\bottomrule\noalign{}
\endlastfoot
Output voltage & Vout = -(1/RC)\intVin dt \\
Transfer function & Vout/Vin = -1/(sRC) in Laplace domain \\
Gain & Decreases at 20dB/decade with frequency \\
Phase shift & -90^\circ (ideally) \\
\end{longtable}
}

\textbf{Applications:}

\begin{itemize}
\tightlist
\item
  Analog computers
\item
  Waveform generators
\item
  PID controllers
\item
  Active filters
\item
  Signal processing
\end{itemize}

\end{solutionbox}
\begin{mnemonicbox}
``TIME'' - Takes Input and Makes time-dependent
Effect

\end{mnemonicbox}
\subsection*{Question 5(a) [3 marks]}\label{q5a}

\textbf{Draw Pin Diagram of IC 555.}

\begin{solutionbox}

\textbf{Diagram:}

\begin{verbatim}
     +{-{-}{-}{-}{-}{-}{-}+}
  1 {-|       |{-} 8}
     |       |
  2 {-|       |{-} 7}
     |  555  |
  3 {-|       |{-} 6}
     |       |
  4 {-|       |{-} 5}
     +{-{-}{-}{-}{-}{-}{-}+}
\end{verbatim}

{\def\LTcaptype{none} % do not increment counter
\begin{longtable}[]{@{}lll@{}}
\toprule\noalign{}
Pin Number & Name & Function \\
\midrule\noalign{}
\endhead
\bottomrule\noalign{}
\endlastfoot
1 & GND & Ground \\
2 & TRIGGER & Starts timing cycle \\
3 & OUTPUT & Timer output \\
4 & RESET & Resets timer \\
5 & CONTROL & Modifies timing \\
6 & THRESHOLD & Ends timing cycle \\
7 & DISCHARGE & Discharges timing capacitor \\
8 & VCC & Positive supply \\
\end{longtable}
}

\end{solutionbox}
\begin{mnemonicbox}
``GTOR-CTD'' - Ground, Trigger, Output, Reset,
Control, Threshold, Discharge

\end{mnemonicbox}
\subsection*{Question 5(b) [4 marks]}\label{q5b}

\textbf{Explain astable multivibrator of timer IC 555.}

\begin{solutionbox}
Astable multivibrator using IC 555 generates continuous
square wave output without any external trigger.

\textbf{Diagram:}

\begin{center}
\textbf{Mermaid Diagram (Code)}
\begin{verbatim}
{Shaded}
{Highlighting}[]
graph LR
    A[VCC] {-{-}{} B[R1]}
    B {-{-}{} C[Pin 7]}
    B {-{-}{} D[Pin 6/2]}
    C {-{-}{} E[IC 555]}
    D {-{-}{} E}
    F[R2] {-{-}{} D}
    F {-{-}{} G[Pin 7]}
    G {-{-}{} E}
    H[C] {-{-}{} D}
    H {-{-}{} I[GND]}
    E {-{-}{} J[Output Pin 3]}
{Highlighting}
{Shaded}
\end{verbatim}
\end{center}

{\def\LTcaptype{none} % do not increment counter
\begin{longtable}[]{@{}ll@{}}
\toprule\noalign{}
Parameter & Formula \\
\midrule\noalign{}
\endhead
\bottomrule\noalign{}
\endlastfoot
Charging time & t_{1} = 0.693(R_{1}+R_{2})C \\
Discharging time & t_{2} = 0.693(R_{2})C \\
Frequency & f = 1.44/((R_{1}+2R_{2})C) \\
Duty cycle & D = (R_{1}+R_{2})/(R_{1}+2R_{2}) \\
\end{longtable}
}

\end{solutionbox}
\begin{mnemonicbox}
``FREE'' - FREquency Established by External RC
network

\end{mnemonicbox}
\subsection*{Question 5(c) [7 marks]}\label{q5c}

\textbf{Explain working of Complementary symmetry Push Pull Amplifiers.}

\begin{solutionbox}
Complementary symmetry push-pull amplifier uses
complementary transistors (NPN and PNP) to amplify both halves of the
waveform.

\textbf{Diagram:}

\begin{verbatim}
            VCC
             |
             |
        Q1  /+{  NPN}
             |
             +{-{-}{-}{-}{-}{-}+}
             |      |
Input +{-{-}{-}{-}{-}{-}+      +{-}{-}{-}+ Output}
             |      |
             +{-{-}{-}{-}{-}{-}+}
             |
        Q2  {{-}/  PNP}
             |
             |
            GND
\end{verbatim}

{\def\LTcaptype{none} % do not increment counter
\begin{longtable}[]{@{}lll@{}}
\toprule\noalign{}
Transistor & Conduction & Current Flow \\
\midrule\noalign{}
\endhead
\bottomrule\noalign{}
\endlastfoot
Q1 (NPN) & Positive half-cycle & Source to load \\
Q2 (PNP) & Negative half-cycle & Sink from load \\
\end{longtable}
}

\textbf{Key features:}

\begin{itemize}
\tightlist
\item
  \textbf{No center-tapped transformer}: Simpler design than
  transformer-coupled push-pull
\item
  \textbf{Crossover distortion}: Requires biasing to minimize
\item
  \textbf{Efficiency}: About 78.5\% (Class B operation)
\item
  \textbf{Thermal runaway}: Risk if not properly designed
\item
  \textbf{Applications}: Audio power amplifiers, output stages of
  op-amps
\end{itemize}

\end{solutionbox}
\begin{mnemonicbox}
``COPS'' - Complementary Opposing Pair of transistors
for Symmetrical operation

\end{mnemonicbox}
\subsection*{Question 5(a) OR [3
marks]}\label{q5a}

\textbf{Draw the diagram of Sequential Timer.}

\begin{solutionbox}

\textbf{Diagram:}

\begin{center}
\textbf{Mermaid Diagram (Code)}
\begin{verbatim}
{Shaded}
{Highlighting}[]
graph LR
    A[Start] {-{-}{} B[555 Timer 1]}
    B {-{-}{} C[555 Timer 2]}
    C {-{-}{} D[555 Timer 3]}
    D {-{-}{} E[Optional additional timers]}
    B {-{-}{} B1[Output 1]}
    C {-{-}{} C1[Output 2]}
    D {-{-}{} D1[Output 3]}
{Highlighting}
{Shaded}
\end{verbatim}
\end{center}

\begin{verbatim}
   +{-{-}{-}{-}{-}+      +{-}{-}{-}{-}{-}+      +{-}{-}{-}{-}{-}+}
   |     |      |     |      |     |
   | 555 |      | 555 |      | 555 |
   |  1  |      |  2  |      |  3  |
   |     |      |     |      |     |
   +{-{-}|{-}{-}+      +{-}{-}|{-}{-}+      +{-}{-}|{-}{-}+}
      |            |            |
      v            v            v
   Output 1     Output 2     Output 3
      |            |            |
   Start        Trigger      Trigger
   Input          from         from
                Timer 1      Timer 2
\end{verbatim}

\end{solutionbox}
\begin{mnemonicbox}
``SET'' - Sequential Events Triggered one after
another

\end{mnemonicbox}
\subsection*{Question 5(b) OR [4
marks]}\label{q5b}

\textbf{Explain bistable multivibrator of timer IC 555.}

\begin{solutionbox}
Bistable multivibrator using IC 555 has two stable
states and changes state only when triggered.

\textbf{Diagram:}

\begin{verbatim}
        VCC
         |
     +{-{-}{-}+{-}{-}{-}+}
     |       |
     +{-{-}{-}+{-}{-}{-}+         +{-}{-}{-}{-}{-}+}
     |   |R  |         |     |
     |   +{-{-}{-}+{-}{-}{-}{-}+{-}{-}{-}{-}+ 555 |}
     |           |    4|     |
     |         +{-+     |     |}
     |   Set   | |     |     |
     +{-{-}{-}{-}o{-}{-}{-}{-}+ +{-}{-}{-}{-}{-}+  3  +{-}{-}{-}{-} Output}
               |       |     |
     +{-{-}{-}{-}o{-}{-}{-}{-}+ +{-}{-}{-}{-}{-}+     |}
     |   Reset | |    2|     |
     |         +{-+     +{-}{-}{-}{-}{-}+}
     |             |      |
     |             |      |
    === GND       === GND  
\end{verbatim}

{\def\LTcaptype{none} % do not increment counter
\begin{longtable}[]{@{}
  >{\raggedright\arraybackslash}p{(\linewidth - 4\tabcolsep) * \real{0.3226}}
  >{\raggedright\arraybackslash}p{(\linewidth - 4\tabcolsep) * \real{0.3226}}
  >{\raggedright\arraybackslash}p{(\linewidth - 4\tabcolsep) * \real{0.3548}}@{}}
\toprule\noalign{}
\begin{minipage}[b]{\linewidth}\raggedright
Terminal
\end{minipage} & \begin{minipage}[b]{\linewidth}\raggedright
Function
\end{minipage} & \begin{minipage}[b]{\linewidth}\raggedright
Operation
\end{minipage} \\
\midrule\noalign{}
\endhead
\bottomrule\noalign{}
\endlastfoot
Pin 2 (TRIGGER) & SET input & When pulled below 1/3 VCC, output goes
HIGH \\
Pin 4 (RESET) & RESET input & When pulled LOW, output goes LOW \\
Pin 3 & Output & Remains in last state until triggered \\
\end{longtable}
}

\end{solutionbox}
\begin{mnemonicbox}
``FLIP'' - Firmly Latched In Position until triggered

\end{mnemonicbox}
\subsection*{Question 5(c) OR [7
marks]}\label{q5c}

\textbf{Compare different types of power Amplifiers.}

\begin{solutionbox}

{\def\LTcaptype{none} % do not increment counter
\begin{longtable}[]{@{}
  >{\raggedright\arraybackslash}p{(\linewidth - 8\tabcolsep) * \real{0.2292}}
  >{\raggedright\arraybackslash}p{(\linewidth - 8\tabcolsep) * \real{0.1875}}
  >{\raggedright\arraybackslash}p{(\linewidth - 8\tabcolsep) * \real{0.1875}}
  >{\raggedright\arraybackslash}p{(\linewidth - 8\tabcolsep) * \real{0.2083}}
  >{\raggedright\arraybackslash}p{(\linewidth - 8\tabcolsep) * \real{0.1875}}@{}}
\toprule\noalign{}
\begin{minipage}[b]{\linewidth}\raggedright
Parameter
\end{minipage} & \begin{minipage}[b]{\linewidth}\raggedright
Class A
\end{minipage} & \begin{minipage}[b]{\linewidth}\raggedright
Class B
\end{minipage} & \begin{minipage}[b]{\linewidth}\raggedright
Class AB
\end{minipage} & \begin{minipage}[b]{\linewidth}\raggedright
Class C
\end{minipage} \\
\midrule\noalign{}
\endhead
\bottomrule\noalign{}
\endlastfoot
Conduction angle & 360^\circ & 180^\circ & 180^\circ-360^\circ & \textless180^\circ \\
Efficiency & 25-30\% & 78.5\% & 50-78.5\% & \textgreater78.5\% \\
Distortion & Very low & Moderate & Low & High \\
Biasing & Above cutoff & At cutoff & Slightly above cutoff & Below
cutoff \\
Circuit complexity & Low & Medium & Medium & Low \\
Heat dissipation & High & Medium & Medium & Low \\
Applications & High fidelity audio & Audio power amps & Audio power amps
& RF transmitters \\
\end{longtable}
}

\textbf{Diagram:}

\begin{center}
\textbf{Mermaid Diagram (Code)}
\begin{verbatim}
{Shaded}
{Highlighting}[]
graph TD
    A[Power Amplifier Classes] {-{-}{} B[Class A: 360^ conduction]}
    A {-{-}{} C[Class B: 180^ conduction]}
    A {-{-}{} D[Class AB: 180^{-}360^ conduction]}
    A {-{-}{} E[Class C: {}180^ conduction]}
    B {-{-}{-} B1[25{-}30\% efficient]}
    C {-{-}{-} C1[78.5\% efficient]}
    D {-{-}{-} D1[50{-}78.5\% efficient]}
    E {-{-}{-} E1[{}78.5\% efficient]}
{Highlighting}
{Shaded}
\end{verbatim}
\end{center}

\end{solutionbox}
\begin{mnemonicbox}
``ABCE'' - As Biasing Condition changes, Efficiency
increases

\end{mnemonicbox}

\end{document}
