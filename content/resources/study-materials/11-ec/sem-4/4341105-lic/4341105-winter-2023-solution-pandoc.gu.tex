\documentclass[10pt,a4paper]{article}

% content/resources/templates/preamble.tex
\usepackage[margin=0.6in]{geometry}
\author{Milav Dabgar}
\usepackage{amsmath,amssymb,amsthm}
\usepackage{booktabs}
\usepackage{multirow}
\usepackage{xcolor}
\usepackage{tcolorbox}
\tcbuselibrary{breakable,skins}
\usepackage[colorlinks=true,linkcolor=blue]{hyperref}
\usepackage{titlesec}
\usepackage{enumitem}
\usepackage{tikz}
\usepackage{pgfplots}
\usepackage{circuitikz}
\usepackage[version=4]{mhchem}
\usepackage{longtable}
\usepackage{array}
\usepackage{float}
\usepackage{caption}
\usepackage{listings}

\lstset{
  basicstyle=\small\ttfamily,
  breaklines=true,
  breakatwhitespace=false,
  postbreak=\mbox{\textcolor{red}{$\hookrightarrow$}\space},
  float=false,
  numbers=left,
  numberstyle=\tiny\color{gray},
  numbersep=10pt,
  xleftmargin=2em,
  keywordstyle=\color{blue},
  commentstyle=\color{green!60!black},
  stringstyle=\color{purple},
  backgroundcolor=\color{gray!5},
  showstringspaces=false,
  tabsize=2,
  captionpos=b,
  keepspaces=true,
  columns=flexible
}

\pgfplotsset{compat=1.18}
\usetikzlibrary{shapes,arrows,positioning,calc,patterns,decorations.pathmorphing,decorations.markings,arrows.meta}

% Color scheme
\definecolor{headcolor}{RGB}{0,102,204}
\definecolor{keycolor}{RGB}{220,20,60}
\definecolor{solutioncolor}{RGB}{34,139,34}
\definecolor{mnemoniccolor}{RGB}{148,0,211}
\definecolor{codecolor}{RGB}{0,0,100}

% Spacing
\setlength{\parskip}{3pt}
\setlist[itemize]{nosep}
\setlist[enumerate]{nosep}

% Title formatting
\titleformat{\section}{\Large\bfseries\color{headcolor}}{\thesection}{1em}{}
\titleformat{\subsection}{\large\bfseries\color{headcolor}}{\thesubsection}{1em}{}

% Pandoc tightlist compatibility
\providecommand{\tightlist}{%
  \setlength{\itemsep}{0pt}\setlength{\parskip}{0pt}}

% Pandoc longtable compatibility
\newcounter{none}
\def\thenone{}


% content/resources/templates/gujarati-boxes.tex
\usepackage{fontspec}
\usepackage{polyglossia}

% Set Gujarati as main language (document is primarily in Gujarati)
% Note: gloss-gujarati.ldf doesn't exist in polyglossia, but it will use hyphenation patterns
\setdefaultlanguage{gujarati}
\setotherlanguage{english}

% Configure Gujarati font properly
% Use Language=Default to prevent polyglossia from trying to add language-specific features
% that don't exist for Gujarati, which causes "empty feature" warnings
\newfontfamily\gujaratifont[Script=Gujarati,AutoFakeBold=2.5,AutoFakeSlant=0.3]{Noto Sans Gujarati}
\setmainfont[Script=Gujarati,AutoFakeBold=2.5,AutoFakeSlant=0.3]{Noto Sans Gujarati}
% Use Noto Sans Gujarati for monospace to support Gujarati in text
\setmonofont[Scale=0.9]{Noto Sans Gujarati}

% Configure English to use the same font
\newfontfamily\englishfont[Script=Gujarati,AutoFakeBold=2.5,AutoFakeSlant=0.3]{Noto Sans Gujarati}

% Translations for polyglossia
\gappto\captionsgujarati{
  \renewcommand{\tablename}{કોષ્ટક}
  \renewcommand{\figurename}{આકૃતિ}
}

% Helper for TikZ nodes to ensure Gujarati font
\newcommand{\gu}[1]{{\gujaratifont #1}}

% Custom environments
\newtcolorbox{solutionbox}{
    breakable,
    enhanced,
    colback=solutioncolor!5!white,
    colframe=solutioncolor!75!black,
    fonttitle=\bfseries,
    title=જવાબ
}

\newtcolorbox{solutionboxnobreak}{
 colback=solutioncolor!5!white,
 colframe=solutioncolor!75!black,
 fonttitle=\bfseries,
 title=જવાબ
}

\newtcolorbox{keyformula}{
 breakable,
 enhanced,
 colback=keycolor!5!white,
 colframe=keycolor!75!black,
 fonttitle=\bfseries,
 title=રાસાયણિક સમીકરણ/સૂત્ર
}

\newtcolorbox{mnemonicbox}{
 breakable,
 enhanced,
 colback=mnemoniccolor!5!white,
 colframe=mnemoniccolor!75!black,
 fonttitle=\bfseries,
 title=મેમરી ટ્રીક
}


\begin{document}

\begin{center}
{\Huge\bfseries\color{headcolor} Subject Name (Gujarati)}\\[5pt]
{\LARGE 4341105 -- Winter 2023}\\[3pt]
{\large Semester 1 Study Material}\\[3pt]
{\normalsize\textit{Detailed Solutions and Explanations}}
\end{center}

\vspace{10pt}

\subsection*{પ્રશ્ન 1(અ) [3
ગુણ]}\label{uxaaauxab0uxab6uxaa8-1uxa85-3-uxa97uxaa3}

\textbf{નેગેટિવ ફિડબેક શું છે? નેગેટિવ ફિડબેકના ફાયદા અને ગેરફાયદાની સૂચિ બનાવો.}

\begin{solutionbox}
નેગેટિવ ફિડબેક એટલે આઉટપુટ સિગ્નલનો એક ભાગ 180^\circ ફેઝ શિફ્ટ સાથે
ઇનપુટમાં પાછો મોકલવો જેથી ઇનપુટ સિગ્નલમાં ઘટાડો થાય.

{\def\LTcaptype{none} % do not increment counter
\begin{longtable}[]{@{}ll@{}}
\toprule\noalign{}
ફાયદા & ગેરફાયદા \\
\midrule\noalign{}
\endhead
\bottomrule\noalign{}
\endlastfoot
સ્થિરતામાં વધારો & ગેઇનમાં ઘટાડો \\
ડિસ્ટોર્શનમાં ઘટાડો & જટિલ સર્કિટ ડિઝાઇન \\
બેન્ડવિડ્થમાં વધારો & વધુ ઘટકોની જરૂર \\
નોઈઝમાં ઘટાડો & વધુ પાવર વપરાશ \\
\end{longtable}
}

\end{solutionbox}
\begin{mnemonicbox}
``SIRS'' - Stability Improved, Reduced distortion,
Sensitivity decreased

\end{mnemonicbox}
\subsection*{પ્રશ્ન 1(બ) [4
ગુણ]}\label{uxaaauxab0uxab6uxaa8-1uxaac-4-uxa97uxaa3}

\textbf{એમ્પલિફાયરના ફ્રિક્વન્સી રિસ્પોન્સ અને ડિસ્ટોર્શન ઉપર નેગેટિવ ફિડબેકની અસર
સમજાવો.}

\begin{solutionbox}
નેગેટિવ ફિડબેક એમ્પલિફાયરમાં ફ્રિક્વન્સી રિસ્પોન્સ સુધારે છે અને
ડિસ્ટોર્શન ઘટાડે છે.

\textbf{આકૃતિ:}

\begin{center}
\textbf{Mermaid Diagram (Code)}
\begin{verbatim}
{Shaded}
{Highlighting}[]
graph TD
    A[Feedback વગરનો એમ્પલિફાયર] {-{-}{} B[સાંકડી બેન્ડવિડ્થ]}
    C[નેગેટિવ ફિડબેક સાથેનો એમ્પલિફાયર] {-{-}{} D[વધુ પહોળી બેન્ડવિડ્થ]}
    E[હાર્મોનિક્સ સાથેનું ઇનપુટ] {-{-}{} F[Feedback વગરનો એમ્પલિફાયર] {-}{-}{} G[વધુ હાર્મોનિક્સ સાથેનું આઉટપુટ]}
    E {-{-}{} H[નેગેટિવ ફિડબેક સાથેનો એમ્પલિફાયર] {-}{-}{} I[ઓછા હાર્મોનિક્સ સાથેનું આઉટપુટ]}
{Highlighting}
{Shaded}
\end{verbatim}
\end{center}

{\def\LTcaptype{none} % do not increment counter
\begin{longtable}[]{@{}lll@{}}
\toprule\noalign{}
અસર & ફિડબેક વગર & નેગેટિવ ફિડબેક સાથે \\
\midrule\noalign{}
\endhead
\bottomrule\noalign{}
\endlastfoot
ફ્રિક્વન્સી રિસ્પોન્સ & સાંકડી બેન્ડવિડ્થ & વધુ પહોળી બેન્ડવિડ્થ \\
ડિસ્ટોર્શન & વધુ હાર્મોનિક્સ & ઓછા હાર્મોનિક્સ \\
\end{longtable}
}

\end{solutionbox}
\begin{mnemonicbox}
``WIDE'' - With negative feedback, Improved
response, Distortion reduced, Extended bandwidth

\end{mnemonicbox}
\subsection*{પ્રશ્ન 1(ક) [7
ગુણ]}\label{uxaaauxab0uxab6uxaa8-1uxa95-7-uxa97uxaa3}

\textbf{નેગેટિવ ફિડબેક વોલ્ટેજ એમ્પલિફાયરના ઓવરઓલ ગેઇન માટે સમીકરણ તારવો.}

\begin{solutionbox}
નેગેટિવ ફિડબેક વોલ્ટેજ એમ્પલિફાયરના ઓવરઓલ ગેઇન માટેનું સમીકરણ નીચે
મુજબ તારવી શકાય:

\textbf{આકૃતિ:}

\begin{verbatim}
    Input +{-{-}{-}{-}{-}+      +{-}{-}{-}{-}{-}{-}{-}+}
    Vi {-{-}|  Σ  |{-}{-}{-}{-}{-}|       |{-}{-}{-}{-} Vo (Output)}
          +{-{-}{-}{-}{-}+      |   A   |}
             \^{         |       |}
             |         +{-{-}{-}{-}{-}{-}{-}+}
             |             |
             |         +{-{-}{-}{-}{-}{-}{-}+}
             +{-{-}{-}{-}{-}{-}{-}{-}{-}|   β   |}
                       +{-{-}{-}{-}{-}{-}{-}+}
\end{verbatim}

\begin{itemize}
\tightlist
\item
  \textbf{ઇનપુટ સમીકરણ}: V' = Vi - βVo
\item
  \textbf{આઉટપુટ સમીકરણ}: Vo = AV'
\item
  \textbf{બંનેને જોડતા}: Vo = A(Vi - βVo)
\item
  \textbf{Vo માટે ઉકેલતા}: Vo = AVi - AβVo
\item
  \textbf{ફેરવીને}: Vo(1 + Aβ) = AVi
\item
  \textbf{અંતિમ સમીકરણ}: Vo/Vi = A/(1 + Aβ) = Af
\end{itemize}

\end{solutionbox}
\begin{mnemonicbox}
``LOOP'' - Look at Original Open-loop gain and
Proceed with feedback

\end{mnemonicbox}
\subsection*{પ્રશ્ન 1(ક) OR [7
ગુણ]}\label{uxaaauxab0uxab6uxaa8-1uxa95-or-7-uxa97uxaa3}

\textbf{વોલ્ટેજ શંટ એમ્પ્લીફાયર અને વર્તમાન શ્રેણીના એમ્પ્લીફાયરની તુલના કરો.}

\begin{solutionbox}

{\def\LTcaptype{none} % do not increment counter
\begin{longtable}[]{@{}lll@{}}
\toprule\noalign{}
પેરામીટર & વોલ્ટેજ શંટ એમ્પ્લીફાયર & વર્તમાન શ્રેણી એમ્પ્લીફાયર \\
\midrule\noalign{}
\endhead
\bottomrule\noalign{}
\endlastfoot
ઇનપુટ & વોલ્ટેજ & વર્તમાન \\
આઉટપુટ & વર્તમાન & વોલ્ટેજ \\
ફિડબેક નેટવર્ક જોડાણ & ઇનપુટ પર સમાંતર & ઇનપુટ પર શ્રેણીમાં \\
ઇનપુટ ઇમ્પિડન્સ & ઘટાડો & વધારો \\
આઉટપુટ ઇમ્પિડન્સ & વધારો & ઘટાડો \\
ગેઇન & વર્તમાન ગેઇનમાં ઘટાડો & વોલ્ટેજ ગેઇનમાં ઘટાડો \\
એપ્લિકેશન & વર્તમાન એમ્પલિફિકેશન & વોલ્ટેજ એમ્પલિફિકેશન \\
\end{longtable}
}

\textbf{આકૃતિ:}

\begin{verbatim}
graph TB
    subgraph "વોલ્ટેજ શંટ"
        A1[ઇનપુટ વોલ્ટેજ] {-{-} B1[શંટ જોડાયેલ β]}
        B1 {-{-} C1[એમ્પલિફાયર]}
        C1 {-{-} D1[આઉટપુટ વર્તમાન]}
    end
    subgraph "વર્તમાન શ્રેણી"
        A2[ઇનપુટ વર્તમાન] {-{-} B2[શ્રેણીમાં જોડાયેલ β]}
        B2 {-{-} C2[એમ્પલિફાયર]}
        C2 {-{-} D2[આઉટપુટ વોલ્ટેજ]}
    end
\end{verbatim}

\end{solutionbox}
\begin{mnemonicbox}
``VICS'' - Voltage shunt In, Current out; Series has
opposite

\end{mnemonicbox}
\subsection*{પ્રશ્ન 2(અ) [3
ગુણ]}\label{uxaaauxab0uxab6uxaa8-2uxa85-3-uxa97uxaa3}

\textbf{ઓસિલેશન માટે Barkhausen's criteriaની ચર્ચા કરો.}

\begin{solutionbox}
Barkhausen's criteria અનુસાર સતત ઓસિલેશન માટે, નીચેની શરતો
પૂરી થવી જોઈએ:

{\def\LTcaptype{none} % do not increment counter
\begin{longtable}[]{@{}ll@{}}
\toprule\noalign{}
ક્રાઇટેરિયા & જરૂરિયાત \\
\midrule\noalign{}
\endhead
\bottomrule\noalign{}
\endlastfoot
લૂપ ગેઇન & \textbar Aβ\textbar{} = 1 (મેગ્નિટ્યુડ 1 જેટલી) \\
ફેઝ શિફ્ટ & લૂપમાં કુલ ફેઝ શિફ્ટ = 0^\circ અથવા 360^\circ \\
\end{longtable}
}

\textbf{આકૃતિ:}

\begin{verbatim}
    +{-{-}{-}{-}{-}{-}{-}+      +{-}{-}{-}{-}{-}{-}{-}+}
    |       |{-{-}{-}{-}{-}|       |{-}{-}+}
    |   A   |      |   β   |  |
    |       |{{-}{-}{-}{-}{-}|       |{-}+}
    +{-{-}{-}{-}{-}{-}{-}+      +{-}{-}{-}{-}{-}{-}{-}+}
\end{verbatim}

\end{solutionbox}
\begin{mnemonicbox}
``LOOP'' - Loop gain One, Oscillation needs Phase
shift zero

\end{mnemonicbox}
\subsection*{પ્રશ્ન 2(બ) [4
ગુણ]}\label{uxaaauxab0uxab6uxaa8-2uxaac-4-uxa97uxaa3}

\textbf{હાર્ટલી ઓસીલેટર અને કોલપીટ્સ ઓસીલેટરનો સર્કિટ ડાયાગ્રામ દોરો.}

\begin{solutionbox}

\textbf{હાર્ટલી ઓસીલેટર:}

\begin{verbatim}
    +{-{-}{-}{-}{-}+     +{-}{-}{-}{-}||{-}{-}{-}{-}{-}+}
    |     |     |           |
    +     |     C1          |
   ===    +{-{-}{-}{-}{-}+           |}
   GND    |     |           |
          |     Z    +{-{-}{-}{-}{-}{-}+}
          |     Z    |      |
          |     Z    |      |
          +{-{-}{-}{-}{-}+    |      |}
            L1  |    |  L2  |
                |    |      |
                +{-{-}{-}{-}+{-}{-}{-}{-}{-}{-}+}
                |    |
                |    |
          +{-{-}{-}{-}{-}+    +{-}{-}{-}{-}{-}+}
          |     |    |     |
          |  Q  |    |     |
          |     |    |     |
          +{-{-}+{-}{-}+    |     |}
             |       |     |
             +{-{-}{-}{-}{-}{-}{-}+     |}
             |             |
            === GND        |
                           |
    +{-{-}{-}{-}||{-}{-}{-}{-}{-}{-}{-}{-}{-}{-}{-}{-}{-}{-}{-}{-}+}
    |       C2
    |
   === GND
\end{verbatim}

\textbf{કોલપીટ્સ ઓસીલેટર:}

\begin{verbatim}
    +{-{-}{-}{-}{-}+     +{-}{-}{-}{-}||{-}{-}{-}{-}+}
    |     |     |     C1    |
    +     |     |           |
   ===    +{-{-}{-}{-}{-}+           |}
   GND    |     |           |
          |     |    +{-{-}{-}{-}{-}{-}+}
          |     |    |      |
          |     Z    |      |
          +{-{-}{-}{-}{-}Z    |      |}
            L   Z    |      |
                |    |      |
                |    |      |
                +{-{-}{-}{-}+{-}{-}{-}{-}{-}{-}+}
                |    |
                |    |
          +{-{-}{-}{-}{-}+    +{-}{-}{-}{-}{-}+}
          |     |    |     |
          |  Q  |    |     |
          |     |    |     |
          +{-{-}+{-}{-}+    |     |}
             |       |     |
             +{-{-}{-}{-}{-}{-}{-}+     |}
             |             |
            === GND        |
             |             |
             +{-{-}{-}{-}{-}{-}{-}{-}{-}{-}{-}{-}{-}+}
             |
            ===
            C2
            ===
            GND
\end{verbatim}

\end{solutionbox}
\begin{mnemonicbox}
``HaLs CoCs'' - Hartley has inductors in series,
Colpitts has Capacitors in series

\end{mnemonicbox}
\subsection*{પ્રશ્ન 2(ક) [7
ગુણ]}\label{uxaaauxab0uxab6uxaa8-2uxa95-7-uxa97uxaa3}

\textbf{UJT ને રિલેક્સેશન ઓસિલેટર તરીકે સમજાવો}

\begin{solutionbox}
UJT (Unijunction Transistor) કૅપેસિટરને વારંવાર ચાર્જ અને
ડિસ્ચાર્જ કરીને રિલેક્સેશન ઓસિલેટર તરીકે કામ કરે છે.

\textbf{આકૃતિ:}

\begin{verbatim}
         RB1
    B2 +{-{-}{-}///{-}{-}{-}+ VCC}
         |          |
         |          |
         |          |
    +{-{-}{-}{-}|          |}
    |    |          |
    |    |  UJT     |
    |    |          |
    |    |          |
    |    |          |
    |    +{-{-}{-}{-}{-}{-}{-}{-}{-}{-}+ B1}
    |    |
    |    |
    |    |
    C    R
    |    |
    |    |
    +{-{-}{-}{-}+{-}{-}{-}{-}+ GND}
\end{verbatim}

{\def\LTcaptype{none} % do not increment counter
\begin{longtable}[]{@{}
  >{\raggedright\arraybackslash}p{(\linewidth - 2\tabcolsep) * \real{0.3500}}
  >{\raggedright\arraybackslash}p{(\linewidth - 2\tabcolsep) * \real{0.6500}}@{}}
\toprule\noalign{}
\begin{minipage}[b]{\linewidth}\raggedright
ફેઝ
\end{minipage} & \begin{minipage}[b]{\linewidth}\raggedright
વર્ણન
\end{minipage} \\
\midrule\noalign{}
\endhead
\bottomrule\noalign{}
\endlastfoot
ચાર્જિંગ & કેપેસિટર R દ્વારા ચાર્જ થાય છે જ્યાં સુધી વોલ્ટેજ VP (પીક વોલ્ટેજ) સુધી ન
પહોંચે \\
ફાયરિંગ & જ્યારે એમિટર વોલ્ટેજ VP પર પહોંચે ત્યારે UJT ચાલુ થાય છે \\
ડિસ્ચાર્જ & કેપેસિટર UJT દ્વારા ઝડપથી ડિસ્ચાર્જ થાય છે \\
રીસેટ & વોલ્ટેજ વેલી વોલ્ટેજ કરતાં નીચે જાય છે, UJT બંધ થાય છે, ચક્ર ફરીથી શરૂ થાય
છે \\
\end{longtable}
}

\begin{itemize}
\tightlist
\item
  \textbf{ઇન્ટ્રિન્સિક સ્ટેન્ડઓફ રેશિયો}: η = RB1/(RB1+RB2)
\item
  \textbf{પીક વોલ્ટેજ}: VP = η\timesVBB + VD
\item
  \textbf{ફ્રિક્વન્સી}: f = 1/[R\timesC\timesln(1/(1-η))]
\end{itemize}

\end{solutionbox}
\begin{mnemonicbox}
``CFDR'' - Charge, Fire, Discharge, Repeat

\end{mnemonicbox}
\subsection*{પ્રશ્ન 2(અ) OR [3
ગુણ]}\label{uxaaauxab0uxab6uxaa8-2uxa85-or-3-uxa97uxaa3}

\textbf{ઓસિલેટરનું વર્ગીકરણ કરો.}

\begin{solutionbox}

{\def\LTcaptype{none} % do not increment counter
\begin{longtable}[]{@{}ll@{}}
\toprule\noalign{}
વર્ગીકરણ & પ્રકાર \\
\midrule\noalign{}
\endhead
\bottomrule\noalign{}
\endlastfoot
ફિડબેક આધારિત & RC, LC, ક્રિસ્ટલ \\
વેવફોર્મ આધારિત & સાઇન્યુસોઇડલ, નોન-સાઇન્યુસોઇડલ \\
ફ્રિક્વન્સી આધારિત & ઓડિયો, રેડિયો, VHF, UHF \\
સર્કિટ આધારિત & હાર્ટલી, કોલપીટ્સ, વિએન-બ્રિજ, RC-ફેઝ શિફ્ટ \\
\end{longtable}
}

\textbf{આકૃતિ:}

\begin{center}
\textbf{Mermaid Diagram (Code)}
\begin{verbatim}
{Shaded}
{Highlighting}[]
graph TD
    A[ઓસિલેટર્સ] {-{-}{} B[RC ઓસિલેટર્સ]}
    A {-{-}{} C[LC ઓસિલેટર્સ]}
    A {-{-}{} D[ક્રિસ્ટલ ઓસિલેટર્સ]}
    A {-{-}{} E[રિલેક્સેશન ઓસિલેટર્સ]}
    B {-{-}{} F[વિએન બ્રિજ]}
    B {-{-}{} G[ફેઝ શિફ્ટ]}
    C {-{-}{} H[હાર્ટલી]}
    C {-{-}{} I[કોલપીટ્સ]}
    C {-{-}{} J[ક્લેપ]}
    E {-{-}{} K[UJT આધારિત]}
    E {-{-}{} L[IC 555 આધારિત]}
{Highlighting}
{Shaded}
\end{verbatim}
\end{center}

\end{solutionbox}
\begin{mnemonicbox}
``SRLC'' - Sine waves from RC, LC, and Crystal
oscillators

\end{mnemonicbox}
\subsection*{પ્રશ્ન 2(બ) OR [4
ગુણ]}\label{uxaaauxab0uxab6uxaa8-2uxaac-or-4-uxa97uxaa3}

\textbf{UJT નું બાંધકામ તેના પ્રતીક (સિમ્બોલ) સાથે સમજાવો.}

\begin{solutionbox}
UJT (Unijunction Transistor) માં હલકા ડોપ્ડ N-પ્રકારના
સિલિકોન બાર હોય છે જેમાં બંને છેડે ઇલેક્ટ્રિકલ કનેક્શન (બેઝિસ) અને P-પ્રકારના એમિટર
જંક્શન હોય છે.

\textbf{આકૃતિ:}

\begin{verbatim}
    સિમ્બોલ:               સ્ટ્રક્ચર:
    
      B2                     B2
       |                      |
       |                  +{-{-}{-}+{-}{-}{-}+}
       |                  |   |   |
       +{-{-}{-}+          +{-}{-}{-}+{-}{-}{-}+{-}{-}{-}+{-}{-}{-}+}
           |          |   |   |   |   |
           +          |   | N{-type    |}
           |          |   |   |   |   |
       +{-{-}{-}+          |   +{-}{-}{-}+{-}{-}{-}+   |}
       |              |       |       |
       |              |       |       |
       E              |       | P     |
       |              |       |       |
       |              +{-{-}{-}{-}{-}{-}{-}+{-}{-}{-}{-}{-}{-}{-}+}
       |                      |
       |                      |
       |                      |
      B1                     B1
\end{verbatim}

{\def\LTcaptype{none} % do not increment counter
\begin{longtable}[]{@{}ll@{}}
\toprule\noalign{}
ઘટક & વર્ણન \\
\midrule\noalign{}
\endhead
\bottomrule\noalign{}
\endlastfoot
બેઝ 1 (B1) & N-પ્રકારના બારના એક છેડા સાથે જોડાયેલ \\
બેઝ 2 (B2) & N-પ્રકારના બારના બીજા છેડા સાથે જોડાયેલ \\
એમિટર (E) & N-પ્રકારના બારમાં ડિફ્યુઝ થયેલ P-પ્રકારના ક્ષેત્ર સાથે જોડાયેલ \\
RB1 & એમિટર અને B1 વચ્ચેનો રેઝિસ્ટન્સ \\
RB2 & એમિટર અને B2 વચ્ચેનો રેઝિસ્ટન્સ \\
\end{longtable}
}

\end{solutionbox}
\begin{mnemonicbox}
``BEB'' - Bases at Ends, Emitter in Between

\end{mnemonicbox}
\subsection*{પ્રશ્ન 2(ક) OR [7
ગુણ]}\label{uxaaauxab0uxab6uxaa8-2uxa95-or-7-uxa97uxaa3}

\textbf{વેન બ્રિજ ઓસિલેટર સર્કિટનું કાર્ય સમજાવો.તેની એપ્લિકેશનની યાદી બનાવો.}

\begin{solutionbox}
વેન બ્રિજ ઓસિલેટર પોઝિટિવ ફિડબેક માટે RC નેટવર્ક અને એમ્પ્લિટ્યુડ
સ્ટેબિલિટી માટે નેગેટિવ ફિડબેક વાપરીને સાઇન વેવ્સ ઉત્પન્ન કરે છે.

\textbf{આકૃતિ:}

\begin{center}
\textbf{Mermaid Diagram (Code)}
\begin{verbatim}
{Shaded}
{Highlighting}[]
graph TD
    subgraph "પોઝિટિવ ફિડબેક"
        R1 {-{-}{-} C1}
        R2 {-{-}{-} C2}
    end
    subgraph "નેગેટિવ ફિડબેક"
        R3
        R4
    end
    A[ઓપ{-એમ્પ] {-}{-}{} Output}
    R1 \& C1 \& R2 \& C2 {-{-}{} A}
    A {-{-}{} R3 {-}{-}{} R4 {-}{-}{} A}
{Highlighting}
{Shaded}
\end{verbatim}
\end{center}

{\def\LTcaptype{none} % do not increment counter
\begin{longtable}[]{@{}ll@{}}
\toprule\noalign{}
ઘટક & કાર્ય \\
\midrule\noalign{}
\endhead
\bottomrule\noalign{}
\endlastfoot
R1, C1 (શ્રેણીમાં) & પોઝિટિવ ફિડબેક, ફેઝ લીડ \\
R2, C2 (સમાંતર) & પોઝિટિવ ફિડબેક, ફેઝ લેગ \\
R3, R4 & નેગેટિવ ફિડબેક, એમ્પ્લિટ્યુડ નિયંત્રણ \\
ઓપ-એમ્પ & એક્ટિવ એમ્પ્લિફાયર એલિમેન્ટ \\
\end{longtable}
}

\textbf{એપ્લિકેશન્સ:}

\begin{itemize}
\tightlist
\item
  ઓડિયો સિગ્નલ જનરેટર્સ
\item
  ફંક્શન જનરેટર્સ
\item
  મ્યુઝિકલ ઇન્સ્ટ્રુમેન્ટ ટ્યુનિંગ
\item
  ટેસ્ટ ઇક્વિપમેન્ટ
\item
  ફિલ્ટર સર્કિટ્સ
\end{itemize}

\end{solutionbox}
\begin{mnemonicbox}
``APPS'' - Audio Production, Pure Sine waves, Stable
frequency

\end{mnemonicbox}
\subsection*{પ્રશ્ન 3(અ) [3
ગુણ]}\label{uxaaauxab0uxab6uxaa8-3uxa85-3-uxa97uxaa3}

\textbf{વોલ્ટેજ અને પાવર એમ્પ્લીફાયર વચ્ચે તફાવત કરો.}

\begin{solutionbox}

{\def\LTcaptype{none} % do not increment counter
\begin{longtable}[]{@{}lll@{}}
\toprule\noalign{}
પેરામીટર & વોલ્ટેજ એમ્પ્લિફાયર & પાવર એમ્પ્લિફાયર \\
\midrule\noalign{}
\endhead
\bottomrule\noalign{}
\endlastfoot
મુખ્ય કાર્ય & વોલ્ટેજ લેવલ વધારે છે & પાવર લેવલ વધારે છે \\
આઉટપુટ & ઓછી વર્તમાન ક્ષમતા & ઉચ્ચ વર્તમાન ક્ષમતા \\
કાર્યક્ષમતા & મહત્વપૂર્ણ નથી & અત્યંત મહત્વપૂર્ણ \\
હીટ ડિસિપેશન & ઓછું & ઉચ્ચ, હીટ સિંક જરૂરી \\
બાયસિંગ & સામાન્ય રીતે ક્લાસ A & ક્લાસ A, B, AB, અથવા C \\
એપ્લિકેશન્સ & પ્રી-એમ્પ્લિફિકેશન સ્ટેજ & સ્પીકર્સ, મોટર્સ ડ્રાઇવિંગ \\
\end{longtable}
}

\end{solutionbox}
\begin{mnemonicbox}
``VICE'' - Voltage amplifiers Increase voltage,
Current not important, Efficiency not critical

\end{mnemonicbox}
\subsection*{પ્રશ્ન 3(બ) [4
ગુણ]}\label{uxaaauxab0uxab6uxaa8-3uxaac-4-uxa97uxaa3}

\textbf{વર્ગ B પુશ પુલ એમ્પ્લીફાયરની કાર્યક્ષમતા માટે સમીકરણ મેળવો.}

\begin{solutionbox}
વર્ગ B પુશ-પુલ એમ્પ્લિફાયરની કાર્યક્ષમતા (η) નીચે મુજબ મેળવવામાં આવે
છે:

\textbf{આકૃતિ:}

\begin{verbatim}
          +VCC
           |
           |
    +{-{-}{-}{-}{-}{-}+{-}{-}{-}{-}{-}{-}+}
    |             |
    |      T1     |
   +++            |
    |             |
   +++     +{-{-}{-}{-}{-}{-}+{-}{-}{-}{-}{-}{-}+}
    |      |      |      |
Input+{-{-}{-}+ |      |      +{-}{-}{-}+Output}
    |      |      |      |
   +++     +{-{-}{-}{-}{-}{-}+{-}{-}{-}{-}{-}{-}+}
    |             |
    |      T2     |
    +{-{-}{-}{-}{-}{-}+{-}{-}{-}{-}{-}{-}+}
           |
           |
          {-VCC}
\end{verbatim}

\begin{itemize}
\tightlist
\item
  \textbf{AC પાવર આઉટપુટ}: P_{0} = Vrms \times Irms = (Vm/\sqrt2) \times (Im/\sqrt2) = Vm \times
  Im/2
\item
  \textbf{DC પાવર ઇનપુટ}: PDC = VCC \times IDC = VCC \times (2\timesIm/π)
\item
  \textbf{કાર્યક્ષમતા}: η = P_{0}/PDC = (Vm\timesIm/2)/(VCC\times2\timesIm/π) =
  (Vm\timesπ)/(4\timesVCC)
\item
  \textbf{મહત્તમ સ્વિંગ માટે}: Vm = VCC, તેથી η = π/4 = 78.5\%
\end{itemize}

\end{solutionbox}
\begin{mnemonicbox}
``POP'' - Push-pull Output Power = π/4 or 78.5\%

\end{mnemonicbox}
\subsection*{પ્રશ્ન 3(ક) [7
ગુણ]}\label{uxaaauxab0uxab6uxaa8-3uxa95-7-uxa97uxaa3}

\textbf{વેવફોર્મ અને તેની કાર્યક્ષમતા સાથે વર્ગ-બી પુશ પુલ એમ્પ્લીફાયરનું કાર્ય
સમજાવો.}

\begin{solutionbox}
વર્ગ B પુશ-પુલ એમ્પ્લિફાયર ઇનપુટ વેવફોર્મના વિપરીત અર્ધચક્રોને
એમ્પ્લિફાય કરવા માટે બે ટ્રાન્ઝિસ્ટર્સનો ઉપયોગ કરે છે.

\textbf{આકૃતિ:}

\begin{center}
\textbf{Mermaid Diagram (Code)}
\begin{verbatim}
{Shaded}
{Highlighting}[]
graph LR
    A[ઇનપુટ સિગ્નલ] {-{-}{} B[ડ્રાઇવર સ્ટેજ]}
    B {-{-}{} C[ઉપરનો ટ્રાન્ઝિસ્ટર]}
    B {-{-}{} D[નીચેનો ટ્રાન્ઝિસ્ટર]}
    C {-{-}{} E[આઉટપુટ ટ્રાન્સફોર્મર]}
    D {-{-}{} E}
    E {-{-}{} F[આઉટપુટ સિગ્નલ]}

    subgraph "વેવફોર્મ્સ"
    direction LR
    G[ઇનપુટ] {-{-}{-} H[T1 કન્ડક્ટ કરે છે] {-}{-}{-} I[T2 કન્ડક્ટ કરે છે]}
    end
{Highlighting}
{Shaded}
\end{verbatim}
\end{center}

{\def\LTcaptype{none} % do not increment counter
\begin{longtable}[]{@{}ll@{}}
\toprule\noalign{}
ફેઝ & વર્ણન \\
\midrule\noalign{}
\endhead
\bottomrule\noalign{}
\endlastfoot
પોઝિટિવ અર્ધચક્ર & ઉપરનો ટ્રાન્ઝિસ્ટર (T1) કન્ડક્ટ કરે છે, T2 બંધ હોય છે \\
નેગેટિવ અર્ધચક્ર & નીચેનો ટ્રાન્ઝિસ્ટર (T2) કન્ડક્ટ કરે છે, T1 બંધ હોય છે \\
ક્રોસઓવર & બંને ટ્રાન્ઝિસ્ટર્સ કટઓફ નજીક હોય છે, જેનાથી ડિસ્ટોર્શન થાય છે \\
\end{longtable}
}

\textbf{મુખ્ય મુદ્દાઓ:}

\begin{itemize}
\tightlist
\item
  \textbf{કાર્યક્ષમતા}: આશરે 78.5\% (π/4)
\item
  \textbf{કન્ડક્શન એંગલ}: દરેક ટ્રાન્ઝિસ્ટર માટે 180^\circ
\item
  \textbf{ક્રોસઓવર ડિસ્ટોર્શન}: શૂન્ય ક્રોસિંગ નજીક બંને ટ્રાન્ઝિસ્ટર્સ બંધ હોવાને કારણે
\item
  \textbf{ફાયદા}: ઉચ્ચ કાર્યક્ષમતા, ઓછી ગરમી, ઉચ્ચ પાવર માટે યોગ્ય
\end{itemize}

\end{solutionbox}
\begin{mnemonicbox}
``HOPE'' - Half cycle Operation, Push-pull,
Efficiency high

\end{mnemonicbox}
\subsection*{પ્રશ્ન 3(અ) OR [3
ગુણ]}\label{uxaaauxab0uxab6uxaa8-3uxa85-or-3-uxa97uxaa3}

\textbf{પાવર એમ્પ્લીફાયરનું વર્ગીકરણ સમજાવો.}

\begin{solutionbox}

{\def\LTcaptype{none} % do not increment counter
\begin{longtable}[]{@{}llll@{}}
\toprule\noalign{}
વર્ગ & કન્ડક્શન એંગલ & કાર્યક્ષમતા & ડિસ્ટોર્શન \\
\midrule\noalign{}
\endhead
\bottomrule\noalign{}
\endlastfoot
વર્ગ A & 360^\circ & 25-30\% & ઓછું \\
વર્ગ B & 180^\circ & 78.5\% & મધ્યમ \\
વર્ગ AB & 180^\circ-360^\circ & 50-78.5\% & ઓછું-મધ્યમ \\
વર્ગ C & \textless180^\circ & \textgreater78.5\% & ઉચ્ચ \\
\end{longtable}
}

\textbf{આકૃતિ:}

\begin{center}
\textbf{Mermaid Diagram (Code)}
\begin{verbatim}
{Shaded}
{Highlighting}[]
graph TD
    A[પાવર એમ્પ્લિફાયર્સ] {-{-}{} B[વર્ગ A]}
    A {-{-}{} C[વર્ગ B]}
    A {-{-}{} D[વર્ગ AB]}
    A {-{-}{} E[વર્ગ C]}
    B {-{-}{} F[ઓછું ડિસ્ટોર્શન, ઓછી કાર્યક્ષમતા]}
    C {-{-}{} G[મધ્યમ ડિસ્ટોર્શન, ઉચ્ચ કાર્યક્ષમતા]}
    D {-{-}{} H[ઓછું ડિસ્ટોર્શન, મધ્યમ કાર્યક્ષમતા]}
    E {-{-}{} I[ઉચ્ચ ડિસ્ટોર્શન, અત્યંત ઉચ્ચ કાર્યક્ષમતા]}
{Highlighting}
{Shaded}
\end{verbatim}
\end{center}

\end{solutionbox}
\begin{mnemonicbox}
``ABCE'' - As Biasing Changes, Efficiency increases

\end{mnemonicbox}
\subsection*{પ્રશ્ન 3(બ) OR [4
ગુણ]}\label{uxaaauxab0uxab6uxaa8-3uxaac-or-4-uxa97uxaa3}

\textbf{વર્ગ A પાવર એમ્પ્લીફાયરની કાર્યક્ષમતા માટે સમીકરણ મેળવો.}

\begin{solutionbox}
વર્ગ A પાવર એમ્પ્લિફાયરની કાર્યક્ષમતા નીચે મુજબ મેળવવામાં આવે છે:

\textbf{આકૃતિ:}

\begin{verbatim}
     +VCC
       |
       |
       Z
       Z RL
       Z
       |
       +{-{-}{-}+Output}
       |
       |
       Q
       |
       |
     Input
       |
      GND
\end{verbatim}

\begin{itemize}
\tightlist
\item
  \textbf{મહત્તમ AC પાવર આઉટપુટ}: P_{0} = (Vrms)^{2}/RL = (VCC/2\sqrt2)^{2}/RL =
  VCC^{2}/8RL
\item
  \textbf{DC પાવર ઇનપુટ}: PDC = VCC \times IDC = VCC \times (VCC/2RL) = VCC^{2}/2RL
\item
  \textbf{કાર્યક્ષમતા}: η = P_{0}/PDC = (VCC^{2}/8RL)/(VCC^{2}/2RL) = 1/4 = 25\%
\end{itemize}

\end{solutionbox}
\begin{mnemonicbox}
``ONE'' - Output Never Exceeds 25\% efficiency in
Class A

\end{mnemonicbox}
\subsection*{પ્રશ્ન 3(ક) OR [7
ગુણ]}\label{uxaaauxab0uxab6uxaa8-3uxa95-or-7-uxa97uxaa3}

\textbf{વેવફોર્મ અને તેની કાર્યક્ષમતા સાથે વર્ગ-A ટ્રાન્સફોર્મર કપલ્ડ એમ્પ્લીફાયરનું
કાર્ય સમજાવો.}

\begin{solutionbox}
વર્ગ A ટ્રાન્સફોર્મર કપલ્ડ એમ્પ્લિફાયર આઉટપુટ કપલિંગ માટે
ટ્રાન્સફોર્મરનો ઉપયોગ કરીને સંપૂર્ણ ઇનપુટ સાયકલ (360^\circ) માટે કન્ડક્ટ કરે છે.

\textbf{આકૃતિ:}

\begin{verbatim}
     +VCC
       |
       |
    +{-{-}+{-}{-}+}
    |     |
    | Pri |
    |     |
    +{-{-}+{-}{-}+}
       |
       +{-{-}{-}+}
       |   |
    Q  |   |
       |   |
       |   +{-{-}+Output}
       |      |
      === C   +{-{-}+{-}{-}+}
       |      |     |
      GND     | Sec |
              |     |
              +{-{-}+{-}{-}+}
                 |
                GND
\end{verbatim}

{\def\LTcaptype{none} % do not increment counter
\begin{longtable}[]{@{}ll@{}}
\toprule\noalign{}
ઘટક & કાર્ય \\
\midrule\noalign{}
\endhead
\bottomrule\noalign{}
\endlastfoot
ટ્રાન્સફોર્મર & ઇમ્પિડન્સ મેચિંગ, DC દૂર કરે, આઇસોલેશન આપે \\
ટ્રાન્ઝિસ્ટર & સંપૂર્ણ 360^\circ સાયકલ માટે કન્ડક્ટ કરે \\
કેપેસિટર & AC કપલિંગ \\
VCC & DC પાવર સપ્લાય \\
\end{longtable}
}

\textbf{વેવફોર્મ લક્ષણો:}

\begin{itemize}
\tightlist
\item
  ઇનપુટ અને આઉટપુટ વેવફોર્મ્સ ફેઝમાં હોય છે
\item
  ક્રોસઓવર ડિસ્ટોર્શન નથી
\item
  સંપૂર્ણ સાયકલ એમ્પ્લિફિકેશન
\item
  ઓછી કાર્યક્ષમતા (25\%)
\item
  ઓછું ડિસ્ટોર્શન
\end{itemize}

\end{solutionbox}
\begin{mnemonicbox}
``FACT'' - Full cycle Amplification in Class-a with
Transformer

\end{mnemonicbox}
\subsection*{પ્રશ્ન 4(અ) [3
ગુણ]}\label{uxaaauxab0uxab6uxaa8-4uxa85-3-uxa97uxaa3}

\textbf{વ્યાખ્યાયિત કરો (i) CMRR (ii) સ્લ્યુ રેટ}

\begin{solutionbox}

{\def\LTcaptype{none} % do not increment counter
\begin{longtable}[]{@{}
  >{\raggedright\arraybackslash}p{(\linewidth - 4\tabcolsep) * \real{0.2895}}
  >{\raggedright\arraybackslash}p{(\linewidth - 4\tabcolsep) * \real{0.3158}}
  >{\raggedright\arraybackslash}p{(\linewidth - 4\tabcolsep) * \real{0.3947}}@{}}
\toprule\noalign{}
\begin{minipage}[b]{\linewidth}\raggedright
પેરામીટર
\end{minipage} & \begin{minipage}[b]{\linewidth}\raggedright
વ્યાખ્યા
\end{minipage} & \begin{minipage}[b]{\linewidth}\raggedright
પ્રમાણભૂત મૂલ્ય
\end{minipage} \\
\midrule\noalign{}
\endhead
\bottomrule\noalign{}
\endlastfoot
CMRR & કોમન મોડ રિજેક્શન રેશિયો, ડિફરેન્શિયલ ગેઇનનો કોમન મોડ ગેઇન સાથેનો ગુણોત્તર
& 90 dB (IC 741) \\
સ્લ્યુ રેટ & આઉટપુટ વોલ્ટેજના પરિવર્તનનો સમય એકમ દીઠ મહત્તમ દર & 0.5 V/μs (IC
741) \\
\end{longtable}
}

\textbf{CMRR}: CMRR = 20 log_{1}_{0}(Ad/Acm) જ્યાં Ad એ ડિફરેન્શિયલ ગેઇન અને Acm એ
કોમન મોડ ગેઇન છે

\textbf{સ્લ્યુ રેટ}: SR = dVout/dt (V/μs)

\end{solutionbox}
\begin{mnemonicbox}
``CRiSp'' - CMRR Rejects common signals, Slew Rate
limits speed

\end{mnemonicbox}
\subsection*{પ્રશ્ન 4(બ) [4
ગુણ]}\label{uxaaauxab0uxab6uxaa8-4uxaac-4-uxa97uxaa3}

\textbf{સ્કેચ સાથે ઓપરેશનલ એમ્પ્લીફાયરના ઇન્વર્ટિંગ એમ્પ્લીફાયર સમજાવો.}

\begin{solutionbox}
ઇન્વર્ટિંગ એમ્પ્લિફાયર નેગેટિવ ફિડબેકનો ઉપયોગ કરીને 180^\circ ફેઝ શિફ્ટ
સાથે ગેઇન પ્રદાન કરે છે.

\textbf{આકૃતિ:}

\begin{verbatim}
        Rf
    +{-{-}{-}///{-}{-}{-}+}
    |            |
    |            |
    |    +{-{-}{-}{-}{-}{-}{-}+}
    |    |       |
    |    |   +   |
Vin +{-{-}{-}{-}+{-}{-}{-}+   +{-}{-}{-}{-}+ Vout}
    Ri   |   {-   |}
         |       |
         +{-{-}{-}{-}{-}{-}{-}+}
              |
              |
             === GND
\end{verbatim}

{\def\LTcaptype{none} % do not increment counter
\begin{longtable}[]{@{}ll@{}}
\toprule\noalign{}
ઘટક & કાર્ય \\
\midrule\noalign{}
\endhead
\bottomrule\noalign{}
\endlastfoot
Ri & ઇનપુટ રેઝિસ્ટર \\
Rf & ફિડબેક રેઝિસ્ટર \\
ઓપ-એમ્પ & ઉચ્ચ ગેઇન સાથે સિગ્નલને એમ્પ્લિફાય કરે \\
\end{longtable}
}

\textbf{મુખ્ય સમીકરણો:}

\begin{itemize}
\tightlist
\item
  \textbf{ગેઇન}: A = -Rf/Ri
\item
  \textbf{ઇનપુટ ઇમ્પિડન્સ}: Z = Ri
\item
  \textbf{બેન્ડવિડ્થ}: ઓપ-એમ્પ અને ગેઇન પર આધારિત
\end{itemize}

\end{solutionbox}
\begin{mnemonicbox}
``IRON'' - Inverting, Resistance ratio gives gain,
Output Negative phase

\end{mnemonicbox}
\subsection*{પ્રશ્ન 4(ક) [7
ગુણ]}\label{uxaaauxab0uxab6uxaa8-4uxa95-7-uxa97uxaa3}

\textbf{Op-amp ને સમિંગ એમ્પ્લીફાયર તરીકે સમજાવો.}

\begin{solutionbox}
સમિંગ એમ્પ્લિફાયર ભારિત યોગદાન સાથે બહુવિધ ઇનપુટ સિગ્નલોને ઉમેરે છે.

\textbf{આકૃતિ:}

\begin{center}
\textbf{Mermaid Diagram (Code)}
\begin{verbatim}
{Shaded}
{Highlighting}[]
graph LR
    V1[V1] {-{-}{}|R1| A(({}+))}
    V2[V2] {-{-}{}|R2| A}
    V3[V3] {-{-}{}|R3| A}
    A {-{-}{-} B[Op{-}Amp]}
    B {-{-}{-} C[Vout]}
    C {-.{-}{}|Rf| A}
{Highlighting}
{Shaded}
\end{verbatim}
\end{center}

\textbf{સર્કિટ:}

\begin{verbatim}
       R1             Rf
    +{-{-}///{-}{-}+{-}{-}{-}///{-}{-}{-}+}
    |          |            |
V1  |          |            |
    +          |    +{-{-}{-}{-}{-}{-}{-}+}
               |    |       |
    +{-{-}///{-}{-}+    |   +   |}
    |          |{-{-}{-}{-}+{-}{-}{-}+   +{-}{-}{-}{-}+ Vout}
V2  +   R2     |    |   {-   |}
               |    |       |
    +{-{-}///{-}{-}+    +{-}{-}{-}{-}{-}{-}{-}+}
    |          |        |
V3  +   R3     |        |
               |       === GND
              === GND
\end{verbatim}

{\def\LTcaptype{none} % do not increment counter
\begin{longtable}[]{@{}ll@{}}
\toprule\noalign{}
પેરામીટર & મૂલ્ય \\
\midrule\noalign{}
\endhead
\bottomrule\noalign{}
\endlastfoot
આઉટપુટ વોલ્ટેજ & Vout = -(Rf/R1)V1 - (Rf/R2)V2 - (Rf/R3)V3 \ldots{} \\
દરેક ઇનપુટ માટે ગેઇન & -Rf/Rn જ્યાં Rn ઇનપુટ રેઝિસ્ટર છે \\
સમાન ભારિત સમિંગ & બધા ઇનપુટ રેઝિસ્ટર્સ સમાન: R1 = R2 = R3 = Rf \\
\end{longtable}
}

\textbf{એપ્લિકેશન્સ:}

\begin{itemize}
\tightlist
\item
  ઓડિયો મિક્સર્સ
\item
  સિગ્નલ પ્રોસેસિંગ
\item
  એનેલોગ કમ્પ્યુટર્સ
\item
  ભારિત સરેરાશ
\end{itemize}

\end{solutionbox}
\begin{mnemonicbox}
``SARI'' - Summing Amplifier Requires Inverting
configuration

\end{mnemonicbox}
\subsection*{પ્રશ્ન 4(અ) OR [3
ગુણ]}\label{uxaaauxab0uxab6uxaa8-4uxa85-or-3-uxa97uxaa3}

\textbf{ઓપરેશનલ એમ્પ્લિફાયરના મૂળભૂત બ્લોક ડાયાગ્રામનું સ્કેચ કરો.}

\begin{solutionbox}

\textbf{આકૃતિ:}

\begin{center}
\textbf{Mermaid Diagram (Code)}
\begin{verbatim}
{Shaded}
{Highlighting}[]
graph LR
    A[ઇનપુટ ડિફરેન્શિયલ સ્ટેજ] {-{-}{} B[ઇન્ટરમીડિયેટ સ્ટેજ]}
    B {-{-}{} C[લેવલ શિફ્ટર]}
    C {-{-}{} D[આઉટપુટ સ્ટેજ]}
    E[બાયસ સર્કિટ] {-{-}{} A}
    E {-{-}{} B}
    E {-{-}{} C}
    E {-{-}{} D}
{Highlighting}
{Shaded}
\end{verbatim}
\end{center}

{\def\LTcaptype{none} % do not increment counter
\begin{longtable}[]{@{}ll@{}}
\toprule\noalign{}
સ્ટેજ & કાર્ય \\
\midrule\noalign{}
\endhead
\bottomrule\noalign{}
\endlastfoot
ઇનપુટ ડિફરેન્શિયલ સ્ટેજ & ઉચ્ચ ઇનપુટ ઇમ્પિડન્સ, કોમન મોડ સિગ્નલોને રિજેક્ટ કરે \\
ઇન્ટરમીડિયેટ સ્ટેજ & ઉચ્ચ ગેઇન, ફ્રિક્વન્સી કમ્પેનસેશન \\
લેવલ શિફ્ટર & આઉટપુટ સ્ટેજ માટે DC લેવલ શિફ્ટ કરે \\
આઉટપુટ સ્ટેજ & ઓછી આઉટપુટ ઇમ્પિડન્સ, વર્તમાન એમ્પ્લિફિકેશન \\
બાયસ સર્કિટ & યોગ્ય ઓપરેટિંગ પોઇન્ટ્સ પ્રદાન કરે \\
\end{longtable}
}

\end{solutionbox}
\begin{mnemonicbox}
``DILO'' - Differential Input, Level shifting,
Output amplification

\end{mnemonicbox}
\subsection*{પ્રશ્ન 4(બ) OR [4
ગુણ]}\label{uxaaauxab0uxab6uxaa8-4uxaac-or-4-uxa97uxaa3}

\textbf{ઓપરેશનલ એમ્પ્લીફાયરના નોન ઇન્વર્ટીંગ એમ્પ્લીફાયરને સ્કેચ સાથે સમજાવો.}

\begin{solutionbox}
નોન-ઇન્વર્ટિંગ એમ્પ્લિફાયર નેગેટિવ ફિડબેકનો ઉપયોગ કરીને ફેઝ ઇન્વર્ઝન
વગર ગેઇન પ્રદાન કરે છે.

\textbf{આકૃતિ:}

\begin{verbatim}
              +{-{-}{-}{-}{-}{-}{-}+}
              |       |
              |   +   |
Vin +{-{-}{-}{-}{-}{-}{-}{-}{-}){-}{-}{-}+   +{-}{-}{-}{-}+ Vout}
              |   {-   |}
              |       |
              +{-{-}{-}+{-}{-}{-}+}
                  |
                  |
         Ri       |
    +{-{-}{-}///{-}{-}{-}{-}+}
    |              
    |              
    |    Rf        
    +{-{-}{-}///{-}{-}{-}{-}+}
    |             |
    |             |
   === GND        |
                  |
                  +
\end{verbatim}

{\def\LTcaptype{none} % do not increment counter
\begin{longtable}[]{@{}ll@{}}
\toprule\noalign{}
પેરામીટર & મૂલ્ય \\
\midrule\noalign{}
\endhead
\bottomrule\noalign{}
\endlastfoot
ગેઇન & A = 1 + Rf/Ri \\
ઇનપુટ ઇમ્પિડન્સ & અત્યંત ઉચ્ચ (ઓપ-એમ્પ પર આધારિત) \\
ફેઝ & ઇનપુટ સાથે ફેઝમાં \\
સામાન્ય એપ્લિકેશન & વોલ્ટેજ ફોલોવર (જ્યારે Rf=0, Ri=\infty) \\
\end{longtable}
}

\end{solutionbox}
\begin{mnemonicbox}
``NIPS'' - Non-inverting, Input and output In Phase,
Same polarity

\end{mnemonicbox}
\subsection*{પ્રશ્ન 4(ક) OR [7
ગુણ]}\label{uxaaauxab0uxab6uxaa8-4uxa95-or-7-uxa97uxaa3}

\textbf{Op-amp ને ઇન્ટિગ્રેટર તરીકે સમજાવો.}

\begin{solutionbox}
ઓપ-એમ્પ ઇન્ટિગ્રેટર ઇનપુટના સમય ઇન્ટિગ્રલના પ્રમાણમાં આઉટપુટ ઉત્પન્ન
કરે છે.

\textbf{આકૃતિ:}

\begin{verbatim}
           C
    +{-{-}{-}{-}{-}{-}||{-}{-}{-}{-}{-}{-}+}
    |              |
    |              |
    |      +{-{-}{-}{-}{-}{-}{-}+}
    |      |       |
    |      |   +   |
Vin +{-{-}{-}{-}{-}{-}+{-}{-}{-}+   +{-}{-}{-}{-}+ Vout}
    R      |   {-   |}
           |       |
           +{-{-}{-}{-}{-}{-}{-}+}
                |
                |
               === GND
\end{verbatim}

{\def\LTcaptype{none} % do not increment counter
\begin{longtable}[]{@{}ll@{}}
\toprule\noalign{}
પેરામીટર & સૂત્ર \\
\midrule\noalign{}
\endhead
\bottomrule\noalign{}
\endlastfoot
આઉટપુટ વોલ્ટેજ & Vout = -(1/RC)\intVin dt \\
ટ્રાન્સફર ફંક્શન & Vout/Vin = -1/(sRC) in Laplace domain \\
ગેઇન & ફ્રિક્વન્સી સાથે 20dB/decade ઘટે છે \\
ફેઝ શિફ્ટ & -90^\circ (આદર્શ રીતે) \\
\end{longtable}
}

\textbf{એપ્લિકેશન્સ:}

\begin{itemize}
\tightlist
\item
  એનેલોગ કમ્પ્યુટર્સ
\item
  વેવફોર્મ જનરેટર્સ
\item
  PID કન્ટ્રોલર્સ
\item
  એક્ટિવ ફિલ્ટર્સ
\item
  સિગ્નલ પ્રોસેસિંગ
\end{itemize}

\end{solutionbox}
\begin{mnemonicbox}
``TIME'' - Takes Input and Makes time-dependent
Effect

\end{mnemonicbox}
\subsection*{પ્રશ્ન 5(અ) [3
ગુણ]}\label{uxaaauxab0uxab6uxaa8-5uxa85-3-uxa97uxaa3}

\textbf{IC 555 નો પિન ડાયાગ્રામ દોરો.}

\begin{solutionbox}

\textbf{આકૃતિ:}

\begin{verbatim}
     +{-{-}{-}{-}{-}{-}{-}+}
  1 {-|       |{-} 8}
     |       |
  2 {-|       |{-} 7}
     |  555  |
  3 {-|       |{-} 6}
     |       |
  4 {-|       |{-} 5}
     +{-{-}{-}{-}{-}{-}{-}+}
\end{verbatim}

{\def\LTcaptype{none} % do not increment counter
\begin{longtable}[]{@{}lll@{}}
\toprule\noalign{}
પિન નંબર & નામ & કાર્ય \\
\midrule\noalign{}
\endhead
\bottomrule\noalign{}
\endlastfoot
1 & GND & ગ્રાઉન્ડ \\
2 & TRIGGER & ટાઇમિંગ સાયકલ શરૂ કરે \\
3 & OUTPUT & ટાઇમર આઉટપુટ \\
4 & RESET & ટાઇમર રીસેટ કરે \\
5 & CONTROL & ટાઇમિંગમાં ફેરફાર કરે \\
6 & THRESHOLD & ટાઇમિંગ સાયકલ સમાપ્ત કરે \\
7 & DISCHARGE & ટાઇમિંગ કેપેસિટર ડિસ્ચાર્જ કરે \\
8 & VCC & પોઝિટિવ સપ્લાય \\
\end{longtable}
}

\end{solutionbox}
\begin{mnemonicbox}
``GTOR-CTD'' - Ground, Trigger, Output, Reset,
Control, Threshold, Discharge

\end{mnemonicbox}
\subsection*{પ્રશ્ન 5(બ) [4
ગુણ]}\label{uxaaauxab0uxab6uxaa8-5uxaac-4-uxa97uxaa3}

\textbf{ટાઈમર IC 555ના એસ્ટેબલ મલ્ટિવાઈબ્રેટર સમજાવો.}

\begin{solutionbox}
IC 555 નો ઉપયોગ કરતો એસ્ટેબલ મલ્ટિવાઈબ્રેટર કોઈપણ બાહ્ય ટ્રિગર
વગર સતત સ્ક્વેર વેવ આઉટપુટ ઉત્પન્ન કરે છે.

\textbf{આકૃતિ:}

\begin{center}
\textbf{Mermaid Diagram (Code)}
\begin{verbatim}
{Shaded}
{Highlighting}[]
graph LR
    A[VCC] {-{-}{} B[R1]}
    B {-{-}{} C[Pin 7]}
    B {-{-}{} D[Pin 6/2]}
    C {-{-}{} E[IC 555]}
    D {-{-}{} E}
    F[R2] {-{-}{} D}
    F {-{-}{} G[Pin 7]}
    G {-{-}{} E}
    H[C] {-{-}{} D}
    H {-{-}{} I[GND]}
    E {-{-}{} J[આઉટપુટ Pin 3]}
{Highlighting}
{Shaded}
\end{verbatim}
\end{center}

{\def\LTcaptype{none} % do not increment counter
\begin{longtable}[]{@{}ll@{}}
\toprule\noalign{}
પેરામીટર & સૂત્ર \\
\midrule\noalign{}
\endhead
\bottomrule\noalign{}
\endlastfoot
ચાર્જિંગ સમય & t_{1} = 0.693(R_{1}+R_{2})C \\
ડિસ્ચાર્જિંગ સમય & t_{2} = 0.693(R_{2})C \\
ફ્રિક્વન્સી & f = 1.44/((R_{1}+2R_{2})C) \\
ડ્યુટી સાયકલ & D = (R_{1}+R_{2})/(R_{1}+2R_{2}) \\
\end{longtable}
}

\end{solutionbox}
\begin{mnemonicbox}
``FREE'' - FREquency Established by External RC
network

\end{mnemonicbox}
\subsection*{પ્રશ્ન 5(ક) [7
ગુણ]}\label{uxaaauxab0uxab6uxaa8-5uxa95-7-uxa97uxaa3}

\textbf{Complementary symmetry પુશ પુલ એમ્પ્લીફાયર્સનું કાર્ય સમજાવો.}

\begin{solutionbox}
Complementary symmetry પુશ-પુલ એમ્પ્લિફાયર વેવફોર્મના બંને
અર્ધભાગોને એમ્પ્લિફાય કરવા માટે કોમ્પ્લિમેન્ટરી ટ્રાન્ઝિસ્ટર્સ (NPN અને PNP) નો ઉપયોગ
કરે છે.

\textbf{આકૃતિ:}

\begin{verbatim}
            VCC
             |
             |
        Q1  /+{  NPN}
             |
             +{-{-}{-}{-}{-}{-}+}
             |      |
Input +{-{-}{-}{-}{-}{-}+      +{-}{-}{-}+ Output}
             |      |
             +{-{-}{-}{-}{-}{-}+}
             |
        Q2  {{-}/  PNP}
             |
             |
            GND
\end{verbatim}

{\def\LTcaptype{none} % do not increment counter
\begin{longtable}[]{@{}lll@{}}
\toprule\noalign{}
ટ્રાન્ઝિસ્ટર & કન્ડક્શન & વર્તમાન પ્રવાહ \\
\midrule\noalign{}
\endhead
\bottomrule\noalign{}
\endlastfoot
Q1 (NPN) & પોઝિટિવ અર્ધ-સાયકલ & સોર્સથી લોડ તરફ \\
Q2 (PNP) & નેગેટિવ અર્ધ-સાયકલ & લોડથી સિંક તરફ \\
\end{longtable}
}

\textbf{મુખ્ય લક્ષણો:}

\begin{itemize}
\tightlist
\item
  \textbf{સેન્ટર-ટેપ્ડ ટ્રાન્સફોર્મર નથી}: ટ્રાન્સફોર્મર-કપલ્ડ પુશ-પુલ કરતાં સરળ
  ડિઝાઇન
\item
  \textbf{ક્રોસઓવર ડિસ્ટોર્શન}: ઓછું કરવા માટે બાયસિંગની જરૂર પડે છે
\item
  \textbf{કાર્યક્ષમતા}: આશરે 78.5\% (વર્ગ B ઓપરેશન)
\item
  \textbf{થર્મલ રનઅવે}: યોગ્ય રીતે ડિઝાઇન ન થયેલ હોય તો જોખમ
\item
  \textbf{એપ્લિકેશન્સ}: ઓડિયો પાવર એમ્પ્લિફાયર્સ, ઓપ-એમ્પ્સના આઉટપુટ સ્ટેજ
\end{itemize}

\end{solutionbox}
\begin{mnemonicbox}
``COPS'' - Complementary Opposing Pair of
transistors for Symmetrical operation

\end{mnemonicbox}
\subsection*{પ્રશ્ન 5(અ) OR [3
ગુણ]}\label{uxaaauxab0uxab6uxaa8-5uxa85-or-3-uxa97uxaa3}

\textbf{સિક્વન્શિયલ ટાઈમરનો આકૃતિ દોરો.}

\begin{solutionbox}

\textbf{આકૃતિ:}

\begin{center}
\textbf{Mermaid Diagram (Code)}
\begin{verbatim}
{Shaded}
{Highlighting}[]
graph LR
    A[Start] {-{-}{} B[555 Timer 1]}
    B {-{-}{} C[555 Timer 2]}
    C {-{-}{} D[555 Timer 3]}
    D {-{-}{} E[વૈકલ્પિક વધારાના ટાઈમર્સ]}
    B {-{-}{} B1[આઉટપુટ 1]}
    C {-{-}{} C1[આઉટપુટ 2]}
    D {-{-}{} D1[આઉટપુટ 3]}
{Highlighting}
{Shaded}
\end{verbatim}
\end{center}

\begin{verbatim}
   +{-{-}{-}{-}{-}+      +{-}{-}{-}{-}{-}+      +{-}{-}{-}{-}{-}+}
   |     |      |     |      |     |
   | 555 |      | 555 |      | 555 |
   |  1  |      |  2  |      |  3  |
   |     |      |     |      |     |
   +{-{-}|{-}{-}+      +{-}{-}|{-}{-}+      +{-}{-}|{-}{-}+}
      |            |            |
      v            v            v
   Output 1     Output 2     Output 3
      |            |            |
   Start        Trigger      Trigger
   Input          from         from
                Timer 1      Timer 2
\end{verbatim}

\end{solutionbox}
\begin{mnemonicbox}
``SET'' - Sequential Events Triggered one after
another

\end{mnemonicbox}
\subsection*{પ્રશ્ન 5(બ) OR [4
ગુણ]}\label{uxaaauxab0uxab6uxaa8-5uxaac-or-4-uxa97uxaa3}

\textbf{ટાઈમર IC 555 ના બાયસ્ટેબલ મલ્ટિવાઈબ્રેટર સમજાવો.}

\begin{solutionbox}
IC 555નો ઉપયોગ કરતો બાયસ્ટેબલ મલ્ટિવાઈબ્રેટરમાં બે સ્થિર અવસ્થાઓ
હોય છે અને માત્ર ટ્રિગર થાય ત્યારે જ અવસ્થા બદલે છે.

\textbf{આકૃતિ:}

\begin{verbatim}
        VCC
         |
     +{-{-}{-}+{-}{-}{-}+}
     |       |
     +{-{-}{-}+{-}{-}{-}+         +{-}{-}{-}{-}{-}+}
     |   |R  |         |     |
     |   +{-{-}{-}+{-}{-}{-}{-}+{-}{-}{-}{-}+ 555 |}
     |           |    4|     |
     |         +{-+     |     |}
     |   Set   | |     |     |
     +{-{-}{-}{-}o{-}{-}{-}{-}+ +{-}{-}{-}{-}{-}+  3  +{-}{-}{-}{-} આઉટપુટ}
               |       |     |
     +{-{-}{-}{-}o{-}{-}{-}{-}+ +{-}{-}{-}{-}{-}+     |}
     |   Reset | |    2|     |
     |         +{-+     +{-}{-}{-}{-}{-}+}
     |             |      |
     |             |      |
    === GND       === GND  
\end{verbatim}

{\def\LTcaptype{none} % do not increment counter
\begin{longtable}[]{@{}
  >{\raggedright\arraybackslash}p{(\linewidth - 4\tabcolsep) * \real{0.3226}}
  >{\raggedright\arraybackslash}p{(\linewidth - 4\tabcolsep) * \real{0.3226}}
  >{\raggedright\arraybackslash}p{(\linewidth - 4\tabcolsep) * \real{0.3548}}@{}}
\toprule\noalign{}
\begin{minipage}[b]{\linewidth}\raggedright
ટર્મિનલ
\end{minipage} & \begin{minipage}[b]{\linewidth}\raggedright
કાર્ય
\end{minipage} & \begin{minipage}[b]{\linewidth}\raggedright
ઓપરેશન
\end{minipage} \\
\midrule\noalign{}
\endhead
\bottomrule\noalign{}
\endlastfoot
Pin 2 (TRIGGER) & SET ઇનપુટ & જ્યારે 1/3 VCC થી નીચે ખેંચાય, આઉટપુટ HIGH થાય \\
Pin 4 (RESET) & RESET ઇનપુટ & જ્યારે LOW ખેંચાય, આઉટપુટ LOW થાય \\
Pin 3 & આઉટપુટ & ટ્રિગર ન થાય ત્યાં સુધી છેલ્લી અવસ્થામાં રહે \\
\end{longtable}
}

\end{solutionbox}
\begin{mnemonicbox}
``FLIP'' - Firmly Latched In Position until
triggered

\end{mnemonicbox}
\subsection*{પ્રશ્ન 5(ક) OR [7
ગુણ]}\label{uxaaauxab0uxab6uxaa8-5uxa95-or-7-uxa97uxaa3}

\textbf{વિવિધ પ્રકારના પાવર એમ્પ્લીફાયરની સરખામણી કરો.}

\begin{solutionbox}

{\def\LTcaptype{none} % do not increment counter
\begin{longtable}[]{@{}
  >{\raggedright\arraybackslash}p{(\linewidth - 8\tabcolsep) * \real{0.2292}}
  >{\raggedright\arraybackslash}p{(\linewidth - 8\tabcolsep) * \real{0.1875}}
  >{\raggedright\arraybackslash}p{(\linewidth - 8\tabcolsep) * \real{0.1875}}
  >{\raggedright\arraybackslash}p{(\linewidth - 8\tabcolsep) * \real{0.2083}}
  >{\raggedright\arraybackslash}p{(\linewidth - 8\tabcolsep) * \real{0.1875}}@{}}
\toprule\noalign{}
\begin{minipage}[b]{\linewidth}\raggedright
પેરામીટર
\end{minipage} & \begin{minipage}[b]{\linewidth}\raggedright
વર્ગ A
\end{minipage} & \begin{minipage}[b]{\linewidth}\raggedright
વર્ગ B
\end{minipage} & \begin{minipage}[b]{\linewidth}\raggedright
વર્ગ AB
\end{minipage} & \begin{minipage}[b]{\linewidth}\raggedright
વર્ગ C
\end{minipage} \\
\midrule\noalign{}
\endhead
\bottomrule\noalign{}
\endlastfoot
કન્ડક્શન એંગલ & 360^\circ & 180^\circ & 180^\circ-360^\circ & \textless180^\circ \\
કાર્યક્ષમતા & 25-30\% & 78.5\% & 50-78.5\% & \textgreater78.5\% \\
ડિસ્ટોર્શન & અત્યંત ઓછું & મધ્યમ & ઓછું & ઉચ્ચ \\
બાયસિંગ & કટઓફથી ઉપર & કટઓફ પર & કટઓફથી થોડું ઉપર & કટઓફથી નીચે \\
સર્કિટ જટિલતા & ઓછી & મધ્યમ & મધ્યમ & ઓછી \\
હીટ ડિસિપેશન & ઉચ્ચ & મધ્યમ & મધ્યમ & ઓછું \\
એપ્લિકેશન્સ & હાઈ ફિડેલિટી ઓડિયો & ઓડિયો પાવર એમ્પ્સ & ઓડિયો પાવર એમ્પ્સ & RF
ટ્રાન્સમિટર્સ \\
\end{longtable}
}

\textbf{આકૃતિ:}

\begin{center}
\textbf{Mermaid Diagram (Code)}
\begin{verbatim}
{Shaded}
{Highlighting}[]
graph TD
    A[પાવર એમ્પ્લિફાયર વર્ગો] {-{-}{} B[વર્ગ A: 360^ કન્ડક્શન]}
    A {-{-}{} C[વર્ગ B: 180^ કન્ડક્શન]}
    A {-{-}{} D[વર્ગ AB: 180^{-}360^ કન્ડક્શન]}
    A {-{-}{} E[વર્ગ C: {}180^ કન્ડક્શન]}
    B {-{-}{-} B1[25{-}30\% કાર્યક્ષમતા]}
    C {-{-}{-} C1[78.5\% કાર્યક્ષમતા]}
    D {-{-}{-} D1[50{-}78.5\% કાર્યક્ષમતા]}
    E {-{-}{-} E1[{}78.5\% કાર્યક્ષમતા]}
{Highlighting}
{Shaded}
\end{verbatim}
\end{center}

\end{solutionbox}
\begin{mnemonicbox}
``ABCE'' - As Biasing Condition changes, Efficiency
increases

\end{mnemonicbox}

\end{document}
