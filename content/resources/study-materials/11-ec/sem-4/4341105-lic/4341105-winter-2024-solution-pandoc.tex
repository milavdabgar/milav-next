\documentclass[10pt,a4paper]{article}

% content/resources/templates/preamble.tex
\usepackage[margin=0.6in]{geometry}
\author{Milav Dabgar}
\usepackage{amsmath,amssymb,amsthm}
\usepackage{booktabs}
\usepackage{multirow}
\usepackage{xcolor}
\usepackage{tcolorbox}
\tcbuselibrary{breakable,skins}
\usepackage[colorlinks=true,linkcolor=blue]{hyperref}
\usepackage{titlesec}
\usepackage{enumitem}
\usepackage{tikz}
\usepackage{pgfplots}
\usepackage{circuitikz}
\usepackage[version=4]{mhchem}
\usepackage{longtable}
\usepackage{array}
\usepackage{float}
\usepackage{caption}
\usepackage{listings}

\lstset{
  basicstyle=\small\ttfamily,
  breaklines=true,
  breakatwhitespace=false,
  postbreak=\mbox{\textcolor{red}{$\hookrightarrow$}\space},
  float=false,
  numbers=left,
  numberstyle=\tiny\color{gray},
  numbersep=10pt,
  xleftmargin=2em,
  keywordstyle=\color{blue},
  commentstyle=\color{green!60!black},
  stringstyle=\color{purple},
  backgroundcolor=\color{gray!5},
  showstringspaces=false,
  tabsize=2,
  captionpos=b,
  keepspaces=true,
  columns=flexible
}

\pgfplotsset{compat=1.18}
\usetikzlibrary{shapes,arrows,positioning,calc,patterns,decorations.pathmorphing,decorations.markings,arrows.meta}

% Color scheme
\definecolor{headcolor}{RGB}{0,102,204}
\definecolor{keycolor}{RGB}{220,20,60}
\definecolor{solutioncolor}{RGB}{34,139,34}
\definecolor{mnemoniccolor}{RGB}{148,0,211}
\definecolor{codecolor}{RGB}{0,0,100}

% Spacing
\setlength{\parskip}{3pt}
\setlist[itemize]{nosep}
\setlist[enumerate]{nosep}

% Title formatting
\titleformat{\section}{\Large\bfseries\color{headcolor}}{\thesection}{1em}{}
\titleformat{\subsection}{\large\bfseries\color{headcolor}}{\thesubsection}{1em}{}

% Pandoc tightlist compatibility
\providecommand{\tightlist}{%
  \setlength{\itemsep}{0pt}\setlength{\parskip}{0pt}}

% Pandoc longtable compatibility
\newcounter{none}
\def\thenone{}


% content/resources/templates/english-boxes.tex
% This file is currently empty - it exists to maintain consistency with the import structure.
% Add custom environments here if needed in the future.


\begin{document}

\begin{center}
{\Huge\bfseries\color{headcolor} Subject Name Solutions}\\[5pt]
{\LARGE 4341105 -- Winter 2024}\\[3pt]
{\large Semester 1 Study Material}\\[3pt]
{\normalsize\textit{Detailed Solutions and Explanations}}
\end{center}

\vspace{10pt}

\subsection*{Question 1(a) [3 marks]}\label{q1a}

\textbf{List advantages and disadvantages of negative feedback}

\begin{solutionbox}

{\def\LTcaptype{none} % do not increment counter
\begin{longtable}[]{@{}
  >{\raggedright\arraybackslash}p{(\linewidth - 2\tabcolsep) * \real{0.4848}}
  >{\raggedright\arraybackslash}p{(\linewidth - 2\tabcolsep) * \real{0.5152}}@{}}
\toprule\noalign{}
\begin{minipage}[b]{\linewidth}\raggedright
Advantages of Negative Feedback
\end{minipage} & \begin{minipage}[b]{\linewidth}\raggedright
Disadvantages of Negative Feedback
\end{minipage} \\
\midrule\noalign{}
\endhead
\bottomrule\noalign{}
\endlastfoot
Increases bandwidth & Reduces gain \\
Improves stability & More components required \\
Reduces distortion & Complex circuit design \\
Decreases noise & Possibility of oscillations if improperly designed \\
Provides better input/output impedance control & Increased power
consumption \\
\end{longtable}
}

\end{solutionbox}
\begin{mnemonicbox}
``STAND'' - Stability, linearity, Amplitude
reduction, Noise reduction, Distortion reduction

\end{mnemonicbox}
\subsection*{Question 1(b) [4 marks]}\label{q1b}

\textbf{Explain effect of negative feedback on gain and stability}

\begin{solutionbox}

{\def\LTcaptype{none} % do not increment counter
\begin{longtable}[]{@{}
  >{\raggedright\arraybackslash}p{(\linewidth - 2\tabcolsep) * \real{0.4412}}
  >{\raggedright\arraybackslash}p{(\linewidth - 2\tabcolsep) * \real{0.5588}}@{}}
\toprule\noalign{}
\begin{minipage}[b]{\linewidth}\raggedright
Effect on Gain
\end{minipage} & \begin{minipage}[b]{\linewidth}\raggedright
Effect on Stability
\end{minipage} \\
\midrule\noalign{}
\endhead
\bottomrule\noalign{}
\endlastfoot
Reduces gain by factor (1+Aβ) & Increases stability against temperature
variations \\
Gain equation: A' = A/(1+Aβ) & Reduces sensitivity to component
parameter changes \\
More predictable gain values & Prevents oscillations in normal operating
conditions \\
Less variation in gain with temperature & Makes circuit performance more
consistent over time \\
\end{longtable}
}

\textbf{Diagram:}

\begin{center}
\textbf{Mermaid Diagram (Code)}
\begin{verbatim}
{Shaded}
{Highlighting}[]
graph LR
    A[Input] {-{-}{} B[Amplifier A]}
    B {-{-}{} C[Output]}
    C {-{-}{} D[Feedback Network β]}
    D {-{-}{} E[Subtractor]}
    A {-{-}{} E}
    E {-{-}{} B}
{Highlighting}
{Shaded}
\end{verbatim}
\end{center}

\end{solutionbox}
\begin{mnemonicbox}
``GRIP'' - Gain Reduction, Improved stability,
Predictable performance

\end{mnemonicbox}
\subsection*{Question 1(c) [7 marks]}\label{q1c}

\textbf{Derive an equation for overall gain of negative feedback voltage
amplifier.}

\begin{solutionbox}

{\def\LTcaptype{none} % do not increment counter
\begin{longtable}[]{@{}
  >{\raggedright\arraybackslash}p{(\linewidth - 4\tabcolsep) * \real{0.2069}}
  >{\raggedright\arraybackslash}p{(\linewidth - 4\tabcolsep) * \real{0.3448}}
  >{\raggedright\arraybackslash}p{(\linewidth - 4\tabcolsep) * \real{0.4483}}@{}}
\toprule\noalign{}
\begin{minipage}[b]{\linewidth}\raggedright
Step
\end{minipage} & \begin{minipage}[b]{\linewidth}\raggedright
Equation
\end{minipage} & \begin{minipage}[b]{\linewidth}\raggedright
Description
\end{minipage} \\
\midrule\noalign{}
\endhead
\bottomrule\noalign{}
\endlastfoot
1 & Vi = Vs - Vf & Input voltage equals source minus feedback \\
2 & Vf = β \times Vo & Feedback voltage is β times output voltage \\
3 & Vo = A \times Vi & Output voltage is amplifier gain times input
voltage \\
4 & Vo = A \times (Vs - β \times Vo) & Substituting (1) and (2) into (3) \\
5 & Vo + A \times β \times Vo = A \times Vs & Rearranging terms \\
6 & Vo(1 + Aβ) = A \times Vs & Factoring Vo \\
7 & Vo/Vs = A/(1+Aβ) & Overall gain equation \\
\end{longtable}
}

\textbf{Diagram:}

\begin{center}
\textbf{Mermaid Diagram (Code)}
\begin{verbatim}
{Shaded}
{Highlighting}[]
graph LR
    Vs[Vs Source] {-{-}{} Sum((+/{-}))}
    Sum {-{-}{} A[Amplifier A]}
    A {-{-}{} Vo[Vo Output]}
    Vo {-{-}{} FB[Feedback β]}
    FB {-{-}{} Sum}
{Highlighting}
{Shaded}
\end{verbatim}
\end{center}

\end{solutionbox}
\begin{mnemonicbox}
``SAFE'' - Source, Amplifier, Feedback, Equation
A/(1+Aβ)

\end{mnemonicbox}
\subsection*{Question 1(c-OR) [7
marks]}\label{question-1c-or-7-marks}

\textbf{Compare voltage shunt amplifier, voltage series, current shunt
and current series amplifier.}

\begin{solutionbox}

{\def\LTcaptype{none} % do not increment counter
\begin{longtable}[]{@{}
  >{\raggedright\arraybackslash}p{(\linewidth - 8\tabcolsep) * \real{0.1549}}
  >{\raggedright\arraybackslash}p{(\linewidth - 8\tabcolsep) * \real{0.2113}}
  >{\raggedright\arraybackslash}p{(\linewidth - 8\tabcolsep) * \real{0.1972}}
  >{\raggedright\arraybackslash}p{(\linewidth - 8\tabcolsep) * \real{0.2254}}
  >{\raggedright\arraybackslash}p{(\linewidth - 8\tabcolsep) * \real{0.2113}}@{}}
\toprule\noalign{}
\begin{minipage}[b]{\linewidth}\raggedright
Parameter
\end{minipage} & \begin{minipage}[b]{\linewidth}\raggedright
Voltage Series
\end{minipage} & \begin{minipage}[b]{\linewidth}\raggedright
Voltage Shunt
\end{minipage} & \begin{minipage}[b]{\linewidth}\raggedright
Current Series
\end{minipage} & \begin{minipage}[b]{\linewidth}\raggedright
Current Shunt
\end{minipage} \\
\midrule\noalign{}
\endhead
\bottomrule\noalign{}
\endlastfoot
\textbf{Input Signal} & Voltage & Voltage & Current & Current \\
\textbf{Output Signal} & Voltage & Current & Voltage & Current \\
\textbf{Input Configuration} & Series & Parallel & Series & Parallel \\
\textbf{Output Configuration} & Series & Series & Parallel & Parallel \\
\textbf{Input Impedance} & Increases & Decreases & Decreases &
Increases \\
\textbf{Output Impedance} & Decreases & Decreases & Increases &
Increases \\
\textbf{Application} & Voltage amplifiers & Transconductance amplifiers
& Transresistance amplifiers & Current amplifiers \\
\end{longtable}
}

\textbf{Diagram:}

\begin{verbatim}
+{-{-}{-}{-}{-}{-}{-}{-}{-}{-}{-}{-}{-}{-}{-}{-}{-}{-}{-}{-}{-}+       +{-}{-}{-}{-}{-}{-}{-}{-}{-}{-}{-}{-}{-}{-}{-}{-}{-}{-}{-}{-}{-}+}
|                     |       |                     |
| Voltage Series      |       | Voltage Shunt       |
| Zi↑ Zo↓             |       | Zi↓ Zo↓             |  
| Av↓                 |       | Av↓                 |
|                     |       |                     |
+{-{-}{-}{-}{-}{-}{-}{-}{-}{-}{-}{-}{-}{-}{-}{-}{-}{-}{-}{-}{-}+       +{-}{-}{-}{-}{-}{-}{-}{-}{-}{-}{-}{-}{-}{-}{-}{-}{-}{-}{-}{-}{-}+}

+{-{-}{-}{-}{-}{-}{-}{-}{-}{-}{-}{-}{-}{-}{-}{-}{-}{-}{-}{-}{-}+       +{-}{-}{-}{-}{-}{-}{-}{-}{-}{-}{-}{-}{-}{-}{-}{-}{-}{-}{-}{-}{-}+}
|                     |       |                     |
| Current Series      |       | Current Shunt       |
| Zi↓ Zo↑             |       | Zi↑ Zo↑             |
| Ai↓                 |       | Ai↓                 |
|                     |       |                     |
+{-{-}{-}{-}{-}{-}{-}{-}{-}{-}{-}{-}{-}{-}{-}{-}{-}{-}{-}{-}{-}+       +{-}{-}{-}{-}{-}{-}{-}{-}{-}{-}{-}{-}{-}{-}{-}{-}{-}{-}{-}{-}{-}+}
\end{verbatim}

\end{solutionbox}
\begin{mnemonicbox}
``VISC'' - Voltage In (Series/shunt), Signal Current
(series/shunt)

\end{mnemonicbox}
\subsection*{Question 2(a) [3 marks]}\label{q2a}

\textbf{Write application of UJT.}

\begin{solutionbox}

{\def\LTcaptype{none} % do not increment counter
\begin{longtable}[]{@{}l@{}}
\toprule\noalign{}
Applications of UJT \\
\midrule\noalign{}
\endhead
\bottomrule\noalign{}
\endlastfoot
Relaxation oscillators \\
Timing circuits \\
Trigger circuits for SCR and TRIAC \\
Sawtooth wave generators \\
Pulse generators \\
Phase control in power electronics \\
\end{longtable}
}

\end{solutionbox}
\begin{mnemonicbox}
``ROBOTS'' - Relaxation Oscillators, Bistable
circuits, Oscillators, Timing, Switching

\end{mnemonicbox}
\subsection*{Question 2(b) [4 marks]}\label{q2b}

\textbf{Draw circuit diagram of Wein bridge oscillator and Heartly
oscillator.}

\begin{solutionbox}

\textbf{Wein Bridge Oscillator:}

\begin{verbatim}
      R1
      ┌──┐
      │  │
┌─────┤  ├─────┬─────────┐
│     └──┘     │         │
│              │        ┌┴┐
│      C1     ┌┴┐ R2    │ │
│     ┌──┐    │ │       │ │ R3
│ ┌───┤  ├────┘ │       │ │
│ │   └──┘      │       └┬┘
│ │             │        │
│ │    R4       │        │
┌┴┐┌──┐         │        │
│ ││  │         │        │
│ ││  │        ┌┴┐       │
└┬┘└──┘        │ │       │
 │             │ │Op{-amp │}
 └─────────────┤ ├───────┘
               └┬┘
                │
                │ C2
             ┌──┤
             │  │
             │  │
             └──┘
\end{verbatim}

\textbf{Hartley Oscillator:}

\begin{verbatim}
                   C1
            ┌───┤ ├────┐
            │         ┌┴┐
            │         │ │
            │         │ │ RFC
            │         │ │
            │         └┬┘
            │   ┌──────┴───┐
            │   │          │
            │   │  Q       │
            │   │    ┌─────┤
            │   └────┴─────┘
            │    │    │
           ┌┴┐  ┌┴┐  ┌┴┐
L1         │ │  │ │  │ │ L2
┌───┐      │ │  │ │  │ │ ┌───┐
│   ├──────┘ │  │ │  │ └─┤   │
│   │        │  │ │  │   │   │
└───┘        └──┴─┴──┘   └───┘
            L tap point
               │  │
              ┌┴┐ │
              │ │ │
              │ │ │  C2
              │ │ └─┤ ├─┐
              └┬┘       │
               │        │
               └────────┘
\end{verbatim}

\end{solutionbox}
\begin{mnemonicbox}
``WH-RC-LC'' - Wein uses RC, Hartley uses LC

\end{mnemonicbox}
\subsection*{Question 2(c) [7 marks]}\label{q2c}

\textbf{Draw and explain the structure, working and characteristics of
UJT.}

\begin{solutionbox}

\textbf{Structure of UJT:}

\begin{verbatim}
              Base 2 (B2)
                 │
                 ▼
              ┌─────┐
              │     │
              │  N  │
              │     │
              ├─────┤
              │     │◄── Emitter (E)
              │  P  │
              │     │
              ├─────┤
              │     │
              │  N  │
              │     │
              └─────┘
                 │
                 ▼
              Base 1 (B1)
\end{verbatim}

{\def\LTcaptype{none} % do not increment counter
\begin{longtable}[]{@{}
  >{\raggedright\arraybackslash}p{(\linewidth - 4\tabcolsep) * \real{0.2973}}
  >{\raggedright\arraybackslash}p{(\linewidth - 4\tabcolsep) * \real{0.2432}}
  >{\raggedright\arraybackslash}p{(\linewidth - 4\tabcolsep) * \real{0.4595}}@{}}
\toprule\noalign{}
\begin{minipage}[b]{\linewidth}\raggedright
Structure
\end{minipage} & \begin{minipage}[b]{\linewidth}\raggedright
Working
\end{minipage} & \begin{minipage}[b]{\linewidth}\raggedright
Characteristics
\end{minipage} \\
\midrule\noalign{}
\endhead
\bottomrule\noalign{}
\endlastfoot
N-type silicon bar with P-type junction & Acts as voltage divider with
intrinsic stand-off ratio η & Negative resistance region in V-I curve \\
Three terminals: Base1, Base2, Emitter & When VE \textgreater{} ηVBB, it
conducts & Peak point and valley point \\
Single P-N junction & Internal resistance decreases rapidly & Stable
switching operation \\
Single junction but two bases & Generates relaxation oscillations &
Temperature sensitivity \\
\end{longtable}
}

\textbf{V-I Characteristics:}

\begin{center}
\textbf{Mermaid Diagram (Code)}
\begin{verbatim}
{Shaded}
{Highlighting}[]
graph LR
    Peak[Peak point] {-{-}{} Valley[Valley point]}
    style Peak fill:\#f9f,stroke:\#333,stroke{-width:2px}
    style Valley fill:\#bbf,stroke:\#333,stroke{-width:2px}
{Highlighting}
{Shaded}
\end{verbatim}
\end{center}

\end{solutionbox}
\begin{mnemonicbox}
``PNVB'' - P-N junction, Negative resistance, Valley
point, Bases two

\end{mnemonicbox}
\subsection*{Question 2(a-OR) [3
marks]}\label{question-2a-or-3-marks}

\textbf{Classify oscillators based on component used and operating
frequency.}

\begin{solutionbox}

{\def\LTcaptype{none} % do not increment counter
\begin{longtable}[]{@{}
  >{\raggedright\arraybackslash}p{(\linewidth - 2\tabcolsep) * \real{0.4167}}
  >{\raggedright\arraybackslash}p{(\linewidth - 2\tabcolsep) * \real{0.5833}}@{}}
\toprule\noalign{}
\begin{minipage}[b]{\linewidth}\raggedright
Based on Components
\end{minipage} & \begin{minipage}[b]{\linewidth}\raggedright
Based on Operating Frequency
\end{minipage} \\
\midrule\noalign{}
\endhead
\bottomrule\noalign{}
\endlastfoot
RC Oscillators (Wien bridge, Phase shift) & Audio Frequency
(20Hz-20kHz) \\
LC Oscillators (Hartley, Colpitts, Clapp) & Radio Frequency
(20kHz-30MHz) \\
Crystal Oscillators (Quartz crystal) & Very High Frequency
(30MHz-300MHz) \\
Relaxation Oscillators (UJT based) & Ultra High Frequency
(300MHz-3GHz) \\
Negative Resistance Oscillators (Tunnel diode) & Microwave Frequency
(\textgreater3GHz) \\
\end{longtable}
}

\end{solutionbox}
\begin{mnemonicbox}
``RCLCN'' - RC, LC, Crystal, Negative resistance

\end{mnemonicbox}
\subsection*{Question 2(b-OR) [4
marks]}\label{question-2b-or-4-marks}

\textbf{Explain UJT as a relaxation oscillator}

\begin{solutionbox}

{\def\LTcaptype{none} % do not increment counter
\begin{longtable}[]{@{}
  >{\raggedright\arraybackslash}p{(\linewidth - 2\tabcolsep) * \real{0.5517}}
  >{\raggedright\arraybackslash}p{(\linewidth - 2\tabcolsep) * \real{0.4483}}@{}}
\toprule\noalign{}
\begin{minipage}[b]{\linewidth}\raggedright
Operation Stage
\end{minipage} & \begin{minipage}[b]{\linewidth}\raggedright
Description
\end{minipage} \\
\midrule\noalign{}
\endhead
\bottomrule\noalign{}
\endlastfoot
Charging Phase & Capacitor charges through resistor R \\
Threshold Point & When capacitor voltage reaches peak point voltage
(ηVBB), UJT turns ON \\
Discharge Phase & Capacitor discharges rapidly through UJT's low
resistance \\
Reset & UJT turns OFF after capacitor voltage falls below valley
point \\
\end{longtable}
}

\textbf{Circuit Diagram:}

\begin{verbatim}
        VBB
         │
         ▼
        ┌┴┐
        │ │
        │ │ R1
        │ │
        └┬┘
         │     B2
         └───┬───┐
             │   │
             │   │
         R   │ UJT
      ┌──┐   │   │
  Vcc │  │   │   │
  ────┤  ├───┤   │
      └──┘   │   │
         │   │   │
         │   └───┘
         │     │
         │     │ B1
        ┌┴┐    │
  C     │ │    │
        │ │    │
        └┬┘    │
         │     │
         └─────┘
          GND
\end{verbatim}

\end{solutionbox}
\begin{mnemonicbox}
``CTDR'' - Charge, Threshold, Discharge, Repeat

\end{mnemonicbox}
\subsection*{Question 2(c-OR) [7
marks]}\label{question-2c-or-7-marks}

\textbf{Sketch the circuit of Colpitts oscillator and explain working of
it in brief}

\begin{solutionbox}

\textbf{Colpitts Oscillator Circuit:}

\begin{verbatim}
                    Vcc
                     │
                     ▼
                    ┌┴┐
                    │ │
                    │ │ RFC
                    │ │
                    └┬┘
          ┌──────────┴───────┐
          │                  │
          │    ┌─────────┐   │
          │    │         │   │
          │    │    Q    │   │
          │    │         │   │
          │    └─┬─────┬─┘   │
          │      │     │     │
          │      │     │     │
C1      ┌─┴─┐   ┌┴┐   ┌┴┐    │ C2
┌──┐    │   │   │ │   │ │    │ ┌──┐
│  ├────┤   │   │ │   │ │    ├─┤  │
│  │    │   │   │ │   │ │    │ │  │
└──┘    └─┬─┘   └┬┘   └┬┘    │ └──┘
          │      │     │     │
          │      └─────┘     │
          │        │         │
          │       ┌┴┐        │
          │       │ │        │
          │       │ │ L      │
          │       │ │        │
          │       └┬┘        │
          │        │         │
          └────────┴─────────┘
\end{verbatim}

{\def\LTcaptype{none} % do not increment counter
\begin{longtable}[]{@{}ll@{}}
\toprule\noalign{}
Component & Function \\
\midrule\noalign{}
\endhead
\bottomrule\noalign{}
\endlastfoot
C1 and C2 & Voltage divider network that provides feedback \\
Inductor L & Forms LC tank circuit with C1 and C2 \\
Transistor Q & Provides amplification \\
RFC (Radio Frequency Choke) & Blocks AC while allowing DC \\
\end{longtable}
}

\textbf{Working:}

\begin{enumerate}
\tightlist
\item
  Tank circuit (L with C1+C2) determines oscillation frequency
\item
  Frequency formula: f = 1/(2π\sqrt(L\times(C1\timesC2)/(C1+C2)))
\item
  Feedback through capacitive voltage divider
\item
  Transistor amplifies and sustains oscillations
\item
  Phase shift of 180^\circ through transistor, 180^\circ through feedback network
\end{enumerate}

\end{solutionbox}
\begin{mnemonicbox}
``COLTS'' - Capacitors form Oscillations with L-Tank
circuit Sustainably

\end{mnemonicbox}
\subsection*{Question 3(a) [3 marks]}\label{q3a}

\textbf{Define the terms related to power amplifier:} \textbf{i)
collector Efficiency ii) Distortion iii) power dissipation capability}

\begin{solutionbox}

{\def\LTcaptype{none} % do not increment counter
\begin{longtable}[]{@{}
  >{\raggedright\arraybackslash}p{(\linewidth - 2\tabcolsep) * \real{0.3333}}
  >{\raggedright\arraybackslash}p{(\linewidth - 2\tabcolsep) * \real{0.6667}}@{}}
\toprule\noalign{}
\begin{minipage}[b]{\linewidth}\raggedright
Term
\end{minipage} & \begin{minipage}[b]{\linewidth}\raggedright
Definition
\end{minipage} \\
\midrule\noalign{}
\endhead
\bottomrule\noalign{}
\endlastfoot
\textbf{Collector Efficiency} & Ratio of AC output power to DC power
supplied by the collector battery (η = P\_out/P\_DC \times 100\%) \\
\textbf{Distortion} & Unwanted change in waveform shape from input to
output (measured as THD - Total Harmonic Distortion) \\
\textbf{Power Dissipation Capability} & Maximum power that amplifier can
safely dissipate as heat without damage (P\_D = V\_CE \times I\_C) \\
\end{longtable}
}

\end{solutionbox}
\begin{mnemonicbox}
``EDP'' - Efficiency measures DC-to-AC conversion,
Distortion alters signal, Power dissipation limits operation

\end{mnemonicbox}
\subsection*{Question 3(b) [4 marks]}\label{q3b}

\textbf{Derive efficiency of class-A power amplifier.}

\begin{solutionbox}

{\def\LTcaptype{none} % do not increment counter
\begin{longtable}[]{@{}
  >{\raggedright\arraybackslash}p{(\linewidth - 4\tabcolsep) * \real{0.2069}}
  >{\raggedright\arraybackslash}p{(\linewidth - 4\tabcolsep) * \real{0.3448}}
  >{\raggedright\arraybackslash}p{(\linewidth - 4\tabcolsep) * \real{0.4483}}@{}}
\toprule\noalign{}
\begin{minipage}[b]{\linewidth}\raggedright
Step
\end{minipage} & \begin{minipage}[b]{\linewidth}\raggedright
Equation
\end{minipage} & \begin{minipage}[b]{\linewidth}\raggedright
Description
\end{minipage} \\
\midrule\noalign{}
\endhead
\bottomrule\noalign{}
\endlastfoot
1 & P\_DC = V\_CC \times I\_C & DC power input \\
2 & P\_out = (V\_peak \times I\_peak)/2 & AC power output \\
3 & V\_peak = V\_CC & Maximum voltage swing \\
4 & I\_peak = I\_C & Maximum current swing \\
5 & P\_out = (V\_CC \times I\_C)/2 & Substituting max values \\
6 & η = (P\_out/P\_DC) \times 100\% & Definition of efficiency \\
7 & η = ((V\_CC \times I\_C)/2)/(V\_CC \times I\_C) \times 100\% & Substituting power
values \\
8 & η = 50\% & Maximum theoretical efficiency \\
\end{longtable}
}

\textbf{Diagram:}

\begin{center}
\textbf{Mermaid Diagram (Code)}
\begin{verbatim}
{Shaded}
{Highlighting}[]
graph LR
    A[Class A] {-{-}{} B["Maximum η = 25{-}30\%"]}
    B {-{-}{} C["Practical η {} 50\%"]}
    style A fill:\#f9f,stroke:\#333,stroke{-width:2px}
{Highlighting}
{Shaded}
\end{verbatim}
\end{center}

\end{solutionbox}
\begin{mnemonicbox}
``HALF'' - Highest Achievable Level Fifty percent

\end{mnemonicbox}
\subsection*{Question 3(c) [7 marks]}\label{q3c}

\textbf{Explain operation of Complementary symmetry push-pull amplifier}

\begin{solutionbox}

\textbf{Circuit Diagram:}

\begin{verbatim}
              Vcc
               │
               ▼
              ┌┴┐
              │ │
              │ │ Rc1
              │ │
              └┬┘
               │
               ├────────┐
               │        │
               │   NPN  │
               │  Q1    │
               │   ┌────┴─┐
           R1  │   │      │ Output
       ┌──┐    │   │      ├─┬────►
 Input │  ├────┴───┤      │ │
 ──────┤  │        └──────┘ │
       └──┘             ┌───┴───┐
                        │       │
                        │  PNP  │
                        │  Q2   │
                        │       │
                        └───┬───┘
                            │
                           ┌┴┐
                           │ │
                           │ │ Rc2
                           │ │
                           └┬┘
                            │
                            ▼
                           {-Vcc}
\end{verbatim}

{\def\LTcaptype{none} % do not increment counter
\begin{longtable}[]{@{}
  >{\raggedright\arraybackslash}p{(\linewidth - 2\tabcolsep) * \real{0.4583}}
  >{\raggedright\arraybackslash}p{(\linewidth - 2\tabcolsep) * \real{0.5417}}@{}}
\toprule\noalign{}
\begin{minipage}[b]{\linewidth}\raggedright
Operation
\end{minipage} & \begin{minipage}[b]{\linewidth}\raggedright
Description
\end{minipage} \\
\midrule\noalign{}
\endhead
\bottomrule\noalign{}
\endlastfoot
\textbf{Positive Half Cycle} & NPN transistor Q1 conducts, PNP
transistor Q2 is OFF \\
\textbf{Negative Half Cycle} & PNP transistor Q2 conducts, NPN
transistor Q1 is OFF \\
\textbf{Crossover Region} & Both transistors are almost OFF, causing
crossover distortion \\
\textbf{Bias Circuit} & Reduces crossover distortion by providing slight
forward bias \\
\textbf{Efficiency} & Higher than Class A (theoretically up to
78.5\%) \\
\textbf{Heat Dissipation} & Better than Class A as only one transistor
conducts at a time \\
\end{longtable}
}

\end{solutionbox}
\begin{mnemonicbox}
``COPS'' - Complementary transistors, Opposite
conducting cycles, Push-pull operation, Symmetrical output

\end{mnemonicbox}
\subsection*{Question 3(a-OR) [3
marks]}\label{question-3a-or-3-marks}

\textbf{Give classification of Power amplifier}

\begin{solutionbox}

{\def\LTcaptype{none} % do not increment counter
\begin{longtable}[]{@{}
  >{\raggedright\arraybackslash}p{(\linewidth - 2\tabcolsep) * \real{0.7500}}
  >{\raggedright\arraybackslash}p{(\linewidth - 2\tabcolsep) * \real{0.2500}}@{}}
\toprule\noalign{}
\begin{minipage}[b]{\linewidth}\raggedright
Classification Basis
\end{minipage} & \begin{minipage}[b]{\linewidth}\raggedright
Types
\end{minipage} \\
\midrule\noalign{}
\endhead
\bottomrule\noalign{}
\endlastfoot
\textbf{Based on Biasing} & Class A, Class B, Class AB, Class C \\
\textbf{Based on Configuration} & Single-ended, Push-pull, Complementary
symmetry \\
\textbf{Based on Coupling} & RC coupled, Transformer coupled, Direct
coupled \\
\textbf{Based on Frequency Range} & Audio power amplifier, RF power
amplifier \\
\textbf{Based on Operating Mode} & Linear, Switching (Class D, E, F) \\
\end{longtable}
}

\end{solutionbox}
\begin{mnemonicbox}
``ABCDE'' - A, B, C classes, Direct/transformer
coupling, Efficiency increases from A to C

\end{mnemonicbox}
\subsection*{Question 3(b-OR) [4
marks]}\label{question-3b-or-4-marks}

\textbf{Derive efficiency of class B push pull amplifier}

\begin{solutionbox}

{\def\LTcaptype{none} % do not increment counter
\begin{longtable}[]{@{}
  >{\raggedright\arraybackslash}p{(\linewidth - 4\tabcolsep) * \real{0.2069}}
  >{\raggedright\arraybackslash}p{(\linewidth - 4\tabcolsep) * \real{0.3448}}
  >{\raggedright\arraybackslash}p{(\linewidth - 4\tabcolsep) * \real{0.4483}}@{}}
\toprule\noalign{}
\begin{minipage}[b]{\linewidth}\raggedright
Step
\end{minipage} & \begin{minipage}[b]{\linewidth}\raggedright
Equation
\end{minipage} & \begin{minipage}[b]{\linewidth}\raggedright
Description
\end{minipage} \\
\midrule\noalign{}
\endhead
\bottomrule\noalign{}
\endlastfoot
1 & P\_DC = (2 \times V\_CC \times I\_max)/π & DC power input (each transistor
conducts for half cycle) \\
2 & P\_out = (V\_CC \times I\_max)/2 & AC power output \\
3 & η = (P\_out/P\_DC) \times 100\% & Definition of efficiency \\
4 & η = ((V\_CC \times I\_max)/2)/((2 \times V\_CC \times I\_max)/π) \times 100\% &
Substituting power values \\
5 & η = (π/4) \times 100\% & Simplifying \\
6 & η = 78.5\% & Maximum theoretical efficiency \\
\end{longtable}
}

\textbf{Diagram:}

\begin{center}
\textbf{Mermaid Diagram (Code)}
\begin{verbatim}
{Shaded}
{Highlighting}[]
graph LR
    A[Class B] {-{-}{} B["Maximum η = 78.5\%"]}
    B {-{-}{} C["π/4  100\%"]}
    style A fill:\#bbf,stroke:\#333,stroke{-width:2px}
{Highlighting}
{Shaded}
\end{verbatim}
\end{center}

\end{solutionbox}
\begin{mnemonicbox}
``PIPE'' - Pi divided by four Equals efficiency

\end{mnemonicbox}
\subsection*{Question 3(c-OR) [7
marks]}\label{question-3c-or-7-marks}

\textbf{Differentiate between class A, B, C and AB power amplifier.}

\begin{solutionbox}

{\def\LTcaptype{none} % do not increment counter
\begin{longtable}[]{@{}
  >{\raggedright\arraybackslash}p{(\linewidth - 8\tabcolsep) * \real{0.2292}}
  >{\raggedright\arraybackslash}p{(\linewidth - 8\tabcolsep) * \real{0.1875}}
  >{\raggedright\arraybackslash}p{(\linewidth - 8\tabcolsep) * \real{0.1875}}
  >{\raggedright\arraybackslash}p{(\linewidth - 8\tabcolsep) * \real{0.2083}}
  >{\raggedright\arraybackslash}p{(\linewidth - 8\tabcolsep) * \real{0.1875}}@{}}
\toprule\noalign{}
\begin{minipage}[b]{\linewidth}\raggedright
Parameter
\end{minipage} & \begin{minipage}[b]{\linewidth}\raggedright
Class A
\end{minipage} & \begin{minipage}[b]{\linewidth}\raggedright
Class B
\end{minipage} & \begin{minipage}[b]{\linewidth}\raggedright
Class AB
\end{minipage} & \begin{minipage}[b]{\linewidth}\raggedright
Class C
\end{minipage} \\
\midrule\noalign{}
\endhead
\bottomrule\noalign{}
\endlastfoot
\textbf{Conduction Angle} & 360^\circ & 180^\circ & 180^\circ-360^\circ & \textless180^\circ \\
\textbf{Bias Point} & At center of load line & At cutoff & Slightly
above cutoff & Below cutoff \\
\textbf{Efficiency} & 25-30\% & 78.5\% & 50-78.5\% & Up to 90\% \\
\textbf{Distortion} & Lowest & High (crossover) & Low & Very high \\
\textbf{Linearity} & Best & Poor & Good & Poor \\
\textbf{Power Output} & Low & Medium & Medium & High \\
\textbf{Applications} & High-fidelity audio & Audio power amplifiers &
Audio power amplifiers & RF power amplifiers \\
\end{longtable}
}

\textbf{Waveform Comparison:}

\begin{verbatim}
Class A:      Class B:      Class AB:     Class C:
   ┌───┐         ┌───┐        ┌───┐         ┌───┐
   │   │         │   │        │   │         │   │
───┘   └───   ───┘   │      ───┘   │      ───┘   │
                └───┐         └───┐         └───┐
                    │            │            │
                    └───         └───         └───
\end{verbatim}

\end{solutionbox}
\begin{mnemonicbox}
``ABCE'' - Angle decreases, Bias moves to cutoff,
Conduction decreases, Efficiency increases

\end{mnemonicbox}
\subsection*{Question 4(a) [3 marks]}\label{q4a}

\textbf{Define (i) CMRR (ii) Slew rate}

\begin{solutionbox}

{\def\LTcaptype{none} % do not increment counter
\begin{longtable}[]{@{}
  >{\raggedright\arraybackslash}p{(\linewidth - 4\tabcolsep) * \real{0.2895}}
  >{\raggedright\arraybackslash}p{(\linewidth - 4\tabcolsep) * \real{0.3158}}
  >{\raggedright\arraybackslash}p{(\linewidth - 4\tabcolsep) * \real{0.3947}}@{}}
\toprule\noalign{}
\begin{minipage}[b]{\linewidth}\raggedright
Parameter
\end{minipage} & \begin{minipage}[b]{\linewidth}\raggedright
Definition
\end{minipage} & \begin{minipage}[b]{\linewidth}\raggedright
Typical Value
\end{minipage} \\
\midrule\noalign{}
\endhead
\bottomrule\noalign{}
\endlastfoot
\textbf{CMRR (Common Mode Rejection Ratio)} & Ratio of differential mode
gain to common mode gain, expressed in dB & 90-120 dB \\
& CMRR = 20 log(Ad/Acm) & Higher is better \\
\textbf{Slew Rate} & Maximum rate of change of output voltage per unit
time & 0.5-10 V/μs \\
& SR = dVo/dt & Higher means faster response \\
\end{longtable}
}

\end{solutionbox}
\begin{mnemonicbox}
``CRSR'' - Common Rejection Slope Rate

\end{mnemonicbox}
\subsection*{Question 4(b) [4 marks]}\label{q4b}

\textbf{Explain Op-amp as a Summing amplifier.}

\begin{solutionbox}

\textbf{Circuit Diagram:}

\begin{verbatim}
         R\_f
      ┌──────┐
      │      │
      │      │
      │    ┌─┴─┐
R1    │    │   │
┌──────┐   │   │
│      │   │   │
V1─────┤    {──┼──── V\_out}
       │   │   │
└──────┘   │   │
      │    │   │
R2    │    └─┬─┘
┌──────┐     │
│      │     │
V2─────┤     │
       │     │
└──────┘     │
      │      │
      └──────┘
\end{verbatim}

{\def\LTcaptype{none} % do not increment counter
\begin{longtable}[]{@{}
  >{\raggedright\arraybackslash}p{(\linewidth - 2\tabcolsep) * \real{0.4583}}
  >{\raggedright\arraybackslash}p{(\linewidth - 2\tabcolsep) * \real{0.5417}}@{}}
\toprule\noalign{}
\begin{minipage}[b]{\linewidth}\raggedright
Operation
\end{minipage} & \begin{minipage}[b]{\linewidth}\raggedright
Description
\end{minipage} \\
\midrule\noalign{}
\endhead
\bottomrule\noalign{}
\endlastfoot
\textbf{Working Principle} & Virtual ground concept - inverting input
maintained at ground potential \\
\textbf{Output Equation} & V\_out = -(R\_f/R1 \times V1 + R\_f/R2 \times V2 +
\ldots{} + R\_f/Rn \times Vn) \\
\textbf{Special Case} & When all input resistors equal
(R1=R2=\ldots=Rn=R), V\_out = -(R\_f/R) \times (V1+V2+\ldots+Vn) \\
\textbf{Applications} & Audio mixers, Analog computers, Signal
conditioning circuits \\
\end{longtable}
}

\end{solutionbox}
\begin{mnemonicbox}
``SWAP'' - Summing With Amplification Property

\end{mnemonicbox}
\subsection*{Question 4(c) [7 marks]}\label{q4c}

\textbf{Draw noninverting amplifier using op Amp and Derive equation of
voltage Gain. Also draw input and output waveform for it}

\begin{solutionbox}

\textbf{Circuit Diagram:}

\begin{verbatim}
         ┌───────────┐
         │           │
         │           │
         │         ┌─┴─┐
         │         │   │
         │   R\_f   │   │
     ┌───┴───┐     │   │
     │       │     │   │
     │       │     │   │ V\_out
     └───┬───┘     └─┬─┼───────►
         │           │ │
         │           │ │
         └───────────┘ │
                       │
                       │
     V\_in              │
     ───────────┬──────┘
                │
                │
                │  R1
               ┌┴┐
               │ │
               │ │
               └┬┘
                │
                │
                ▼
               GND
\end{verbatim}

{\def\LTcaptype{none} % do not increment counter
\begin{longtable}[]{@{}ll@{}}
\toprule\noalign{}
Parameter & Description \\
\midrule\noalign{}
\endhead
\bottomrule\noalign{}
\endlastfoot
\textbf{Voltage Gain Equation} & A\_v = 1 + (R\_f/R1) \\
\textbf{Input Impedance} & Very high (typically \textgreater10^{6} Ω) \\
\textbf{Output Impedance} & Very low (typically \textless100 Ω) \\
\textbf{Phase Shift} & 0^\circ (in phase) \\
\end{longtable}
}

\textbf{Input and Output Waveforms:}

\begin{verbatim}
Input:               Output:
      ┌───┐                ┌───────┐
      │   │                │       │
      │   │                │       │
      │   │                │       │
\_\_\_\_\_\_│   │\_\_\_\_\_\_    \_\_\_\_\_\_│       │\_\_\_\_\_\_
      └───┘                └───────┘

Gain = 1 + (R\_f/R1) { 1}
\end{verbatim}

\textbf{Derivation of Voltage Gain:}

\begin{enumerate}
\tightlist
\item
  Voltage at both input pins is equal (V^{+} = V^{-})
\item
  In an ideal op-amp, voltage at the inverting input, V^{-} = V\_in
\item
  The feedback network forms a voltage divider: V^{-} = V\_out \times
  [R1/(R1+R\_f)]
\item
  Equating the above two equations: V\_in = V\_out \times [R1/(R1+R\_f)]
\item
  Rearranging: V\_out/V\_in = (R1+R\_f)/R1 = 1 + (R\_f/R1)
\item
  Therefore, A\_v = 1 + (R\_f/R1)
\end{enumerate}

\textbf{Characteristics of Non-inverting Amplifier:}

\begin{itemize}
\tightlist
\item
  Output is in phase with input (0^\circ phase shift)
\item
  High input impedance makes it ideal as voltage amplifier
\item
  Gain is always greater than 1
\item
  Noise rejection is lower than inverting amplifier
\end{itemize}

\end{solutionbox}
\begin{mnemonicbox}
``UPON'' - Unity Plus One plus Noninverting gain

\end{mnemonicbox}
\subsection*{Question 4(a-OR) [3
marks]}\label{question-4a-or-3-marks}

\textbf{Draw symbol of operational amplifier. Draw pin diagram of IC
741.}

\begin{solutionbox}

\textbf{Op-Amp Symbol:}

\begin{verbatim}
             ┌───────────────┐
             │               │
  Non{-inv    │               │}
  Input ─────┤+              │
             │               │
             │      Op{-Amp   ├───── Output}
             │               │
  Inverting  │               │
  Input ─────┤{-              │}
             │               │
             └───┬─────┬─────┘
                 │     │
                 │     │
            V+   ▼     ▼   V{-}
           Supply voltages
\end{verbatim}

\textbf{IC 741 Pin Diagram:}

\begin{verbatim}
        ┌────┐
Offset  │1  8│  NC
Null  1 ├────┤
        │    │
     {-  │2  7│  V+}
  Input ├────┤
        │    │
     +  │3  6│  Output
  Input ├────┤
        │    │
     V{- │4  5│  Offset}
        ├────┤     Null 2
        └────┘
\end{verbatim}

\end{solutionbox}
\begin{mnemonicbox}
``7-PIN'' - 741 Pinout INcludes power, inputs, null,
output

\end{mnemonicbox}
\subsection*{Question 4(b-OR) [4
marks]}\label{question-4b-or-4-marks}

\textbf{Draw and explain inverting configuration of op-amp with
derivation of voltage gain.}

\begin{solutionbox}

\textbf{Inverting Amplifier Circuit:}

\begin{verbatim}
       R\_f
     ┌─────┐
     │     │
     │     │
     │   ┌─┴─┐
     │   │   │
R\_i  │   │   │
 ┌───┴───┤   │
 │       │   │
Vin      │   │ V\_out
 └───────┤   ├───────►
         └─┬─┘
   ┌─────┐ │
   │     │ │
   │     │ │
   └─────┘ │
     │     │
     ▼     │
    GND    │
           └─────────
\end{verbatim}

{\def\LTcaptype{none} % do not increment counter
\begin{longtable}[]{@{}ll@{}}
\toprule\noalign{}
Step & Description \\
\midrule\noalign{}
\endhead
\bottomrule\noalign{}
\endlastfoot
1 & Apply virtual ground concept (V^{-} \approx 0) \\
2 & Current through R\_i: I\_i = V\_in/R\_i \\
3 & Current through R\_f: I\_f = -V\_out/R\_f \\
4 & By Kirchhoff's current law: I\_i + I\_f = 0 \\
5 & Therefore, V\_in/R\_i = V\_out/R\_f \\
6 & Voltage gain: A\_v = V\_out/V\_in = -R\_f/R\_i \\
\end{longtable}
}

\end{solutionbox}
\begin{mnemonicbox}
``IRON'' - Inverting Ratio Of Negative feedback

\end{mnemonicbox}
\subsection*{Question 4(c-OR) [7
marks]}\label{question-4c-or-7-marks}

\textbf{Explain Op-amp as an Integrator.}

\begin{solutionbox}

\textbf{Integrator Circuit:}

\begin{verbatim}
         R
    ┌─────────┐
    │         │
Vin │         │      C
    └─────────┤  ┌─────────┐
              │  │         │
              │  │         │
              └──┤       ┌─┴─┐
                 │       │   │
                 │       │   │
                 └───────┤   │ V\_out
                         │   ├───────►
                         │   │
                         └─┬─┘
                     ┌─────┴───┐
                     │         │
                     │         │
                     └─────────┘
                          │
                          │
                          ▼
                         GND
\end{verbatim}

{\def\LTcaptype{none} % do not increment counter
\begin{longtable}[]{@{}
  >{\raggedright\arraybackslash}p{(\linewidth - 2\tabcolsep) * \real{0.4583}}
  >{\raggedright\arraybackslash}p{(\linewidth - 2\tabcolsep) * \real{0.5417}}@{}}
\toprule\noalign{}
\begin{minipage}[b]{\linewidth}\raggedright
Parameter
\end{minipage} & \begin{minipage}[b]{\linewidth}\raggedright
Description
\end{minipage} \\
\midrule\noalign{}
\endhead
\bottomrule\noalign{}
\endlastfoot
\textbf{Transfer Function} & V\_out = -(1/RC) \intV\_in dt \\
\textbf{Input Signal} & Any waveform (DC, sine, square, etc.) \\
\textbf{Output for Constant Input} & Ramp (linearly
increasing/decreasing) \\
\textbf{Output for Square Wave} & Triangular wave \\
\textbf{Output for Sine Wave} & Cosine wave (90^\circ phase shift) \\
\end{longtable}
}

\textbf{Waveform Transformations:}

\begin{verbatim}
Input:               Output:
DC:                  Ramp:
──────                   ∕
                        ∕
                       ∕
                      ∕

Square Wave:         Triangular Wave:
      ┌───┐                  ∕{}
      │   │                 ∕  {}
      │   │                ∕    {}
\_\_\_\_\_\_│   │\_\_\_\_\_\_         ∕      {\_\_\_\_}
      └───┘

Sine Wave:           Cosine Wave:
     ┌─┐                   ┌─┐
    ∕   {                 │   │}
   ∕     {               ∕     }
──┘       └──          ─┘       └─
\end{verbatim}

\textbf{Practical Considerations:}

\begin{itemize}
\tightlist
\item
  Need for reset switch across capacitor
\item
  Saturation due to input offset voltage
\item
  Limited frequency range due to op-amp bandwidth
\end{itemize}

\end{solutionbox}
\begin{mnemonicbox}
``SIRT'' - Signal Integration Results in Time-domain
transformation

\end{mnemonicbox}
\subsection*{Question 5(a) [3 marks]}\label{q5a}

\textbf{Draw the diagram of Sequential Timer.}

\begin{solutionbox}

\textbf{Sequential Timer Circuit using IC 555:}

\begin{verbatim}
                        Vcc
                         │
                         ▼
                ┌────────┬────────┐
                │        │        │
               ┌┴┐      ┌┴┐      ┌┴┐
               │ │      │ │      │ │
          R1   │ │      │ │ R2   │ │ R3
               │ │      │ │      │ │
               └┬┘      └┬┘      └┬┘
                │        │        │
                │   ┌────┼────┐   │   ┌────────┐
                │   │    │ 8  │   │   │        │
          ┌─────┼───┤7   └────┤3  ├───┤7       │
          │     │   │         │   │   │        │
          │     │   │         │   │   │        │
 C1      ┌┴┐    │   │  555    │   │   │  555   │
┌──┐     │ │    │   │  (1)    │   │   │  (2)   │
│  ├─────┤ ├────┼───┤6        │   └───┤2       │
│  │     │ │    │   │         │       │        │
└──┘     └┬┘    └───┤2        │       │        │    More stages
          │         │         │       │        │       can be
          │         │   GND   │       │Output  │       added
          │         └────┬────┘       └────┬───┘
          │              │                 │
          └──────────────┘                 ▼
\end{verbatim}

\end{solutionbox}
\begin{mnemonicbox}
``STTR'' - Sequential Timing Through Relay-like
operation

\end{mnemonicbox}
\subsection*{Question 5(b) [4 marks]}\label{q5b}

\textbf{Explain working of timer IC 555 using block diagram}

\begin{solutionbox}

\textbf{Block Diagram of IC 555:}

\begin{center}
\textbf{Mermaid Diagram (Code)}
\begin{verbatim}
{Shaded}
{Highlighting}[]
graph LR
    A[Threshold Comparator] {-{-}{} C[SR Flip{-}Flop]}
    B[Trigger Comparator] {-{-}{} C}
    C {-{-}{} D[Output Stage]}
    C {-{-}{} E[Discharge Transistor]}
    F[Voltage Divider] {-{-}{} A}
    F {-{-}{} B}
    style C fill:\#f9f,stroke:\#333,stroke{-width:2px}
{Highlighting}
{Shaded}
\end{verbatim}
\end{center}

{\def\LTcaptype{none} % do not increment counter
\begin{longtable}[]{@{}
  >{\raggedright\arraybackslash}p{(\linewidth - 2\tabcolsep) * \real{0.4118}}
  >{\raggedright\arraybackslash}p{(\linewidth - 2\tabcolsep) * \real{0.5882}}@{}}
\toprule\noalign{}
\begin{minipage}[b]{\linewidth}\raggedright
Block
\end{minipage} & \begin{minipage}[b]{\linewidth}\raggedright
Function
\end{minipage} \\
\midrule\noalign{}
\endhead
\bottomrule\noalign{}
\endlastfoot
\textbf{Voltage Divider} & Creates reference voltages of (2/3)VCC and
(1/3)VCC \\
\textbf{Threshold Comparator} & Compares threshold pin voltage with
(2/3)VCC \\
\textbf{Trigger Comparator} & Compares trigger pin voltage with
(1/3)VCC \\
\textbf{SR Flip-Flop} & Controls output state based on comparator
inputs \\
\textbf{Output Stage} & Provides current to drive external loads \\
\textbf{Discharge Transistor} & Discharges timing capacitor when output
is low \\
\end{longtable}
}

\end{solutionbox}
\begin{mnemonicbox}
``VTTDO'' - Voltage divider, Two comparators, Toggle
flip-flop, Discharge, Output

\end{mnemonicbox}
\subsection*{Question 5(c) [7 marks]}\label{q5c}

\textbf{Explain astable multivibrator of timer IC 555.}

\begin{solutionbox}

\textbf{Astable Multivibrator Circuit:}

\begin{verbatim}
               Vcc
                │
                ▼
               ┌┴┐
               │ │
               │ │ Ra
               │ │
               └┬┘
                │        ┌──────────┐
                ├────────┤8  Vcc    │
                │        │          │
                │       ┌┴┐         │
                │       │ │         │
                │       │ │ Rb      │   Output
                │       │ │  ┌──────┼───────►
                │       └┬┘  │      │3
                │ ┌──────┼───┤7 Dis │
                │ │      │   │      │
                │ │      ├───┤6 Thr │
              ┌─┴─┐      │   │      │
              │   │      |   │ 555  │
         C    │   │      │   │      │
        ┌──┐  │   │      ├───┤2 Trg │
        │  ├──┤   │      │   │      │
        │  │  │   │      │   │      │
        └──┘  └───┘      │   │      │
               │         │   │      │
               │         │ ┌─┴─┐    │
               │         └─┤5  │    │
               │           │   │    │
               │           └───┘    │
               │        ┌───────────┘
               │        │
               └────────┤1 GND
                        │
                        ▼
\end{verbatim}

{\def\LTcaptype{none} % do not increment counter
\begin{longtable}[]{@{}
  >{\raggedright\arraybackslash}p{(\linewidth - 4\tabcolsep) * \real{0.3333}}
  >{\raggedright\arraybackslash}p{(\linewidth - 4\tabcolsep) * \real{0.2727}}
  >{\raggedright\arraybackslash}p{(\linewidth - 4\tabcolsep) * \real{0.3939}}@{}}
\toprule\noalign{}
\begin{minipage}[b]{\linewidth}\raggedright
Parameter
\end{minipage} & \begin{minipage}[b]{\linewidth}\raggedright
Formula
\end{minipage} & \begin{minipage}[b]{\linewidth}\raggedright
Description
\end{minipage} \\
\midrule\noalign{}
\endhead
\bottomrule\noalign{}
\endlastfoot
\textbf{Charging Time (HIGH)} & t_{1} = 0.693 \times (Ra + Rb) \times C & Output HIGH
duration \\
\textbf{Discharging Time (LOW)} & t_{2} = 0.693 \times Rb \times C & Output LOW
duration \\
\textbf{Total Period} & T = t_{1} + t_{2} = 0.693 \times (Ra + 2Rb) \times C & Complete
cycle time \\
\textbf{Frequency} & f = 1.44/((Ra + 2Rb) \times C) & Number of cycles per
second \\
\textbf{Duty Cycle} & D = (Ra + Rb)/(Ra + 2Rb) & Ratio of HIGH time to
total period \\
\end{longtable}
}

\textbf{Waveforms:}

\begin{verbatim}
Capacitor Voltage:       Output Voltage:
      ┌─┐ ┌─┐ ┌─┐              ┌───┐ ┌───┐ ┌───┐
     ∕ │∕ │∕ │∕ │             │   │ │   │ │   │
    ∕  │  │  │  │             │   │ │   │ │   │
   ∕   │  │  │  │             │   │ │   │ │   │
──┘    └─┘ └─┘ └─┘     \_\_\_\_\_\_\_│   │\_│   │\_│   │\_\_\_\_
  2/3Vcc                       t_{1    t_{2}}
  1/3Vcc
\end{verbatim}

\end{solutionbox}
\begin{mnemonicbox}
``FREE'' - Frequency Related to External Elements

\end{mnemonicbox}
\subsection*{Question 5(a-OR) [3
marks]}\label{question-5a-or-3-marks}

\textbf{Draw Pin Diagram of IC 555.}

\begin{solutionbox}

\textbf{IC 555 Pin Configuration:}

\begin{verbatim}
          ┌───────┐
  GND   1 │       │ 8   Vcc
          │       │
TRIGGER 2 │       │ 7   DISCHARGE
          │  555  │
 OUTPUT 3 │       │ 6   THRESHOLD
          │       │
 RESET  4 │       │ 5   CONTROL
          └───────┘
\end{verbatim}

{\def\LTcaptype{none} % do not increment counter
\begin{longtable}[]{@{}lll@{}}
\toprule\noalign{}
Pin Name & Pin Number & Function \\
\midrule\noalign{}
\endhead
\bottomrule\noalign{}
\endlastfoot
GND & 1 & Ground reference \\
TRIGGER & 2 & Starts timing cycle when \textless{} 1/3 VCC \\
OUTPUT & 3 & Output terminal \\
RESET & 4 & Resets timing cycle when LOW \\
CONTROL & 5 & Controls threshold and trigger levels \\
THRESHOLD & 6 & Ends timing cycle when \textgreater{} 2/3 VCC \\
DISCHARGE & 7 & Discharges timing capacitor \\
VCC & 8 & Positive supply voltage (4.5V-18V) \\
\end{longtable}
}

\end{solutionbox}
\begin{mnemonicbox}
``GTORCTDV'' - Ground, Trigger, Output, Reset,
Control, Threshold, Discharge, Vcc

\end{mnemonicbox}
\subsection*{Question 5(b-OR) [4
marks]}\label{question-5b-or-4-marks}

\textbf{Explain monostable multivibrator of timer IC 555.}

\begin{solutionbox}

\textbf{Monostable Multivibrator Circuit:}

\begin{verbatim}
               Vcc
                │
                ▼
               ┌┴┐
               │ │
               │ │ R
               │ │
               └┬┘
                │        ┌──────────┐
                ├────────┤8  Vcc    │
                │        │          │
                │        │          │
                │        │          │
                ├────────┤7 Dis     │   Output
                │        │       ┌──┼───────►
                │        │       │  │3
                │        │       │  │
                │ ┌──────┼───────┤  │
                │ │      │       │  │
                │ │      └───────┤6 Thr │
              ┌─┴─┐              │      │
              │   │              │ 555  │
         C    │   │              │      │
        ┌──┐  │   │          ┌───┤2 Trg │
        │  ├──┤   │  Trigger │   │      │
        │  │  │   │   ───────┘   │      │
        └──┘  └───┘              │      │
               │                ┌┴┐     │
               │                │ │     │
               │                │ │     │
               │                └┬┘     │
               │        ┌────────┼──────┘
               │        │        │
               └────────┤1 GND   │4 RST
                        │        │
                        ▼        ▼
\end{verbatim}

{\def\LTcaptype{none} % do not increment counter
\begin{longtable}[]{@{}
  >{\raggedright\arraybackslash}p{(\linewidth - 2\tabcolsep) * \real{0.4583}}
  >{\raggedright\arraybackslash}p{(\linewidth - 2\tabcolsep) * \real{0.5417}}@{}}
\toprule\noalign{}
\begin{minipage}[b]{\linewidth}\raggedright
Parameter
\end{minipage} & \begin{minipage}[b]{\linewidth}\raggedright
Description
\end{minipage} \\
\midrule\noalign{}
\endhead
\bottomrule\noalign{}
\endlastfoot
\textbf{Trigger} & Negative edge triggered at pin 2 (\textless1/3
VCC) \\
\textbf{Pulse Width} & T = 1.1 \times R \times C seconds \\
\textbf{Operating States} & Stable state (output LOW) and quasi-stable
state (output HIGH) \\
\textbf{Reset} & Can be terminated early by applying LOW to Reset pin \\
\end{longtable}
}

\textbf{Monostable Operation:}

\begin{enumerate}
\tightlist
\item
  Output normally LOW
\item
  Negative trigger pulse initiates timing cycle
\item
  Output goes HIGH for duration T
\item
  After time T, output returns to LOW
\item
  Circuit ignores additional trigger pulses during timing cycle
\end{enumerate}

\end{solutionbox}
\begin{mnemonicbox}
``OPTS'' - One Pulse Timed by Single trigger

\end{mnemonicbox}
\subsection*{Question 5(c-OR) [7
marks]}\label{question-5c-or-7-marks}

\textbf{Explain bistable multivibrator of timer IC 555.}

\begin{solutionbox}

\textbf{Bistable Multivibrator Circuit:}

\begin{verbatim}
               Vcc
                │
                ▼
               ┌┴┐
               │ │
               │ │ R1
               │ │
               └┬┘
                │        ┌──────────┐
                ├────────┤8  Vcc    │
                │        │          │
                │        │          │
                │        │          │
                ├────────┤4 RST     │   Output
                │        │       ┌──┼───────►
                │        │       │  │3
                │        │       │  │
                │        │       │  │
                │        │       │  │
                │        │       │  │
                │        │  555  │  │
                │        │       │  │
  Reset       ┌─┴─┐      │       │  │
  Switch      │ o │──────┤6 THR  │  │
  ───────── ──┤ o │      │       │  │
              └───┘      │       │  │
                │        │       │  │
                │        │       │  │
              ┌─┴─┐      │       │  │
  Set         │ o │──────┤2 TRG  │  │
  Switch      │ o │      │       │  │
  ────────────┤   │      │       │  │
              └───┘      │       │  │
               │         │       │  │
               │         │       │  │
               │      ┌──┴───────┘  │
               │      │             │
               └──────┤1 GND        │
                      │             │
                      ▼             │
\end{verbatim}

{\def\LTcaptype{none} % do not increment counter
\begin{longtable}[]{@{}
  >{\raggedright\arraybackslash}p{(\linewidth - 4\tabcolsep) * \real{0.2692}}
  >{\raggedright\arraybackslash}p{(\linewidth - 4\tabcolsep) * \real{0.4231}}
  >{\raggedright\arraybackslash}p{(\linewidth - 4\tabcolsep) * \real{0.3077}}@{}}
\toprule\noalign{}
\begin{minipage}[b]{\linewidth}\raggedright
State
\end{minipage} & \begin{minipage}[b]{\linewidth}\raggedright
Condition
\end{minipage} & \begin{minipage}[b]{\linewidth}\raggedright
Output
\end{minipage} \\
\midrule\noalign{}
\endhead
\bottomrule\noalign{}
\endlastfoot
\textbf{Set State} & Trigger pin (2) momentarily pulled below 1/3 VCC &
HIGH \\
\textbf{Reset State} & Reset pin (4) momentarily pulled LOW & LOW \\
\textbf{Memory Function} & Maintains state until changed by input &
Stable in either state \\
\end{longtable}
}

\textbf{Bistable Operation:}

\begin{enumerate}
\tightlist
\item
  Circuit has two stable states (HIGH or LOW)
\item
  SET input (Trigger) makes output HIGH
\item
  RESET input makes output LOW
\item
  No timing components needed
\item
  Functions as a basic latch or flip-flop
\end{enumerate}

\textbf{Applications:}

\begin{itemize}
\tightlist
\item
  Toggle switches
\item
  Memory elements
\item
  Bounce-free switching
\item
  Level shifting
\item
  Push-button ON/OFF control
\end{itemize}

\end{solutionbox}
\begin{mnemonicbox}
``SRSS'' - Set-Reset Stable States

\end{mnemonicbox}

\end{document}
