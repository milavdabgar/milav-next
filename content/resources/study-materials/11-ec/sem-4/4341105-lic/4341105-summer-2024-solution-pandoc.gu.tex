\documentclass[10pt,a4paper]{article}

% content/resources/templates/preamble.tex
\usepackage[margin=0.6in]{geometry}
\author{Milav Dabgar}
\usepackage{amsmath,amssymb,amsthm}
\usepackage{booktabs}
\usepackage{multirow}
\usepackage{xcolor}
\usepackage{tcolorbox}
\tcbuselibrary{breakable,skins}
\usepackage[colorlinks=true,linkcolor=blue]{hyperref}
\usepackage{titlesec}
\usepackage{enumitem}
\usepackage{tikz}
\usepackage{pgfplots}
\usepackage{circuitikz}
\usepackage[version=4]{mhchem}
\usepackage{longtable}
\usepackage{array}
\usepackage{float}
\usepackage{caption}
\usepackage{listings}

\lstset{
  basicstyle=\small\ttfamily,
  breaklines=true,
  breakatwhitespace=false,
  postbreak=\mbox{\textcolor{red}{$\hookrightarrow$}\space},
  float=false,
  numbers=left,
  numberstyle=\tiny\color{gray},
  numbersep=10pt,
  xleftmargin=2em,
  keywordstyle=\color{blue},
  commentstyle=\color{green!60!black},
  stringstyle=\color{purple},
  backgroundcolor=\color{gray!5},
  showstringspaces=false,
  tabsize=2,
  captionpos=b,
  keepspaces=true,
  columns=flexible
}

\pgfplotsset{compat=1.18}
\usetikzlibrary{shapes,arrows,positioning,calc,patterns,decorations.pathmorphing,decorations.markings,arrows.meta}

% Color scheme
\definecolor{headcolor}{RGB}{0,102,204}
\definecolor{keycolor}{RGB}{220,20,60}
\definecolor{solutioncolor}{RGB}{34,139,34}
\definecolor{mnemoniccolor}{RGB}{148,0,211}
\definecolor{codecolor}{RGB}{0,0,100}

% Spacing
\setlength{\parskip}{3pt}
\setlist[itemize]{nosep}
\setlist[enumerate]{nosep}

% Title formatting
\titleformat{\section}{\Large\bfseries\color{headcolor}}{\thesection}{1em}{}
\titleformat{\subsection}{\large\bfseries\color{headcolor}}{\thesubsection}{1em}{}

% Pandoc tightlist compatibility
\providecommand{\tightlist}{%
  \setlength{\itemsep}{0pt}\setlength{\parskip}{0pt}}

% Pandoc longtable compatibility
\newcounter{none}
\def\thenone{}


% content/resources/templates/gujarati-boxes.tex
\usepackage{fontspec}
\usepackage{polyglossia}

% Set Gujarati as main language (document is primarily in Gujarati)
% Note: gloss-gujarati.ldf doesn't exist in polyglossia, but it will use hyphenation patterns
\setdefaultlanguage{gujarati}
\setotherlanguage{english}

% Configure Gujarati font properly
% Use Language=Default to prevent polyglossia from trying to add language-specific features
% that don't exist for Gujarati, which causes "empty feature" warnings
\newfontfamily\gujaratifont[Script=Gujarati,AutoFakeBold=2.5,AutoFakeSlant=0.3]{Noto Sans Gujarati}
\setmainfont[Script=Gujarati,AutoFakeBold=2.5,AutoFakeSlant=0.3]{Noto Sans Gujarati}
% Use Noto Sans Gujarati for monospace to support Gujarati in text
\setmonofont[Scale=0.9]{Noto Sans Gujarati}

% Configure English to use the same font
\newfontfamily\englishfont[Script=Gujarati,AutoFakeBold=2.5,AutoFakeSlant=0.3]{Noto Sans Gujarati}

% Translations for polyglossia
\gappto\captionsgujarati{
  \renewcommand{\tablename}{કોષ્ટક}
  \renewcommand{\figurename}{આકૃતિ}
}

% Helper for TikZ nodes to ensure Gujarati font
\newcommand{\gu}[1]{{\gujaratifont #1}}

% Custom environments
\newtcolorbox{solutionbox}{
    breakable,
    enhanced,
    colback=solutioncolor!5!white,
    colframe=solutioncolor!75!black,
    fonttitle=\bfseries,
    title=જવાબ
}

\newtcolorbox{solutionboxnobreak}{
 colback=solutioncolor!5!white,
 colframe=solutioncolor!75!black,
 fonttitle=\bfseries,
 title=જવાબ
}

\newtcolorbox{keyformula}{
 breakable,
 enhanced,
 colback=keycolor!5!white,
 colframe=keycolor!75!black,
 fonttitle=\bfseries,
 title=રાસાયણિક સમીકરણ/સૂત્ર
}

\newtcolorbox{mnemonicbox}{
 breakable,
 enhanced,
 colback=mnemoniccolor!5!white,
 colframe=mnemoniccolor!75!black,
 fonttitle=\bfseries,
 title=મેમરી ટ્રીક
}


\begin{document}

\begin{center}
{\Huge\bfseries\color{headcolor} Subject Name (Gujarati)}\\[5pt]
{\LARGE 4341105 -- Summer 2024}\\[3pt]
{\large Semester 1 Study Material}\\[3pt]
{\normalsize\textit{Detailed Solutions and Explanations}}
\end{center}

\vspace{10pt}

\subsection*{પ્રશ્ન 1(અ) [3
ગુણ]}\label{uxaaauxab0uxab6uxaa8-1uxa85-3-uxa97uxaa3}

\textbf{આકૃતિ સાથે પોઝિટિવ અને નેગેટિવ ફીડબેક વચ્ચેનો તફાવત જણાવો અને સમજાવો.}

\begin{solutionbox}

{\def\LTcaptype{none} % do not increment counter
\begin{longtable}[]{@{}
  >{\raggedright\arraybackslash}p{(\linewidth - 4\tabcolsep) * \real{0.2245}}
  >{\raggedright\arraybackslash}p{(\linewidth - 4\tabcolsep) * \real{0.3878}}
  >{\raggedright\arraybackslash}p{(\linewidth - 4\tabcolsep) * \real{0.3878}}@{}}
\toprule\noalign{}
\begin{minipage}[b]{\linewidth}\raggedright
પરિમાણ
\end{minipage} & \begin{minipage}[b]{\linewidth}\raggedright
નેગેટિવ ફીડબેક
\end{minipage} & \begin{minipage}[b]{\linewidth}\raggedright
પોઝિટિવ ફીડબેક
\end{minipage} \\
\midrule\noalign{}
\endhead
\bottomrule\noalign{}
\endlastfoot
સિગ્નલ & આઉટપુટ સિગ્નલ વિરુદ્ધ તબક્કા સાથે ઇનપુટ પર પાછો ફીડ કરવામાં આવે છે & આઉટપુટ
સિગ્નલ સમાન તબક્કા સાથે ઇનપુટ પર પાછો ફીડ કરવામાં આવે છે \\
ગેઇન & ઘટાડે છે & વધારે છે \\
સ્થિરતા & સુધારે છે & ઘટાડે છે \\
ઉપયોગો & એમ્પલિફાયર્સ & ઓસિલેટર્સ \\
\end{longtable}
}

\textbf{આકૃતિ:}

\begin{center}
\textbf{Mermaid Diagram (Code)}
\begin{verbatim}
{Shaded}
{Highlighting}[]
graph LR
    A[Input] {-{-}{} B[Amplifier]}
    B {-{-}{} C[Output]}
    C {-{-}{} D\{Feedback Network\}}

    \%\% Negative Feedback
    subgraph Negative Feedback
    D {-{-}{}|180^ Phase Shift| E[Subtractor]}
    E {-{-}{} B}
    end
    
    \%\% Positive Feedback
    subgraph Positive Feedback
    D {-{-}{}|0^ Phase Shift| F[Adder]}
    F {-{-}{} B}
    end
{Highlighting}
{Shaded}
\end{verbatim}
\end{center}

\begin{itemize}
\tightlist
\item
  \textbf{ફેઝ સંબંધ}: નેગેટિવ ફીડબેકમાં, સિગ્નલ 180^\circ આઉટ ઓફ ફેઝ હોય છે જ્યારે
  પોઝિટિવ ફીડબેકમાં, સિગ્નલ ઇન ફેઝ હોય છે
\item
  \textbf{હેતુ}: નેગેટિવ ફીડબેક સિસ્ટમને સ્થિર કરે છે જ્યારે પોઝિટિવ ફીડબેક ઓસિલેશન
  ઉત્પન્ન કરે છે
\end{itemize}

\end{solutionbox}
\begin{mnemonicbox}
``નેગેટિવ નિયમિતતા માંગે, પોઝિટિવ પરિવર્તન આપે''

\end{mnemonicbox}
\subsection*{પ્રશ્ન 1(બ) [4
ગુણ]}\label{uxaaauxab0uxab6uxaa8-1uxaac-4-uxa97uxaa3}

\textbf{એમ્પલીફાયરના ઇનપુટ ઇમ્પીડન્સ પર નેગેટિવ ફીડબેક ની અસર સમજાવો.}

\begin{solutionbox}

{\def\LTcaptype{none} % do not increment counter
\begin{longtable}[]{@{}lll@{}}
\toprule\noalign{}
ફીડબેકનો પ્રકાર & ઇનપુટ ઇમ્પિડન્સ પર અસર & સૂત્ર \\
\midrule\noalign{}
\endhead
\bottomrule\noalign{}
\endlastfoot
વોલ્ટેજ સિરીઝ & વધારે છે & Z(in-f) = Z(in)(1+Aβ) \\
કરંટ સિરીઝ & વધારે છે & Z(in-f) = Z(in)(1+Aβ) \\
વોલ્ટેજ શંટ & ઘટાડે છે & Z(in-f) = Z(in)/(1+Aβ) \\
કરંટ શંટ & ઘટાડે છે & Z(in-f) = Z(in)/(1+Aβ) \\
\end{longtable}
}

\textbf{આકૃતિ:}

\begin{center}
\textbf{Mermaid Diagram (Code)}
\begin{verbatim}
{Shaded}
{Highlighting}[]
graph LR
    A[Input Signal] {-{-}{} B[Input Impedance]}
    B {-{-}{} C[Amplifier]}
    C {-{-}{} D[Output]}
    D {-{-}{} E[Feedback Network]}
    E {-{-}{} F[Summing Point]}
    F {-{-}{} B}
    style B fill:\#f9f,stroke:\#333,stroke{-width:2px}
{Highlighting}
{Shaded}
\end{verbatim}
\end{center}

\begin{itemize}
\tightlist
\item
  \textbf{સિરીઝ ફીડબેક}: જ્યારે ફીડબેક સિગ્નલ ઇનપુટની સાથે સિરીઝમાં હોય, ઇનપુટ
  ઇમ્પિડન્સ વધે છે
\item
  \textbf{શંટ ફીડબેક}: જ્યારે ફીડબેક સિગ્નલ ઇનપુટની સમાંતર હોય, ઇનપુટ ઇમ્પિડન્સ ઘટે
  છે
\item
  \textbf{મેગ્નિટ્યુડ}: ફેરફાર (1+Aβ)ના પ્રમાણમાં હોય છે જ્યાં A એ ગેઇન અને β એ
  ફીડબેક ફેક્ટર છે
\end{itemize}

\end{solutionbox}
\begin{mnemonicbox}
``સિરીઝ સંવર્ધન કરે, શંટ સંકોચન કરે''

\end{mnemonicbox}
\subsection*{પ્રશ્ન 1(ક) [7
ગુણ]}\label{uxaaauxab0uxab6uxaa8-1uxa95-7-uxa97uxaa3}

\textbf{નેગેટિવ ફીડબેકના ફાયદા અને ગેરફાયદાની યાદી બનાવો.}

\begin{solutionbox}

{\def\LTcaptype{none} % do not increment counter
\begin{longtable}[]{@{}
  >{\raggedright\arraybackslash}p{(\linewidth - 2\tabcolsep) * \real{0.4444}}
  >{\raggedright\arraybackslash}p{(\linewidth - 2\tabcolsep) * \real{0.5556}}@{}}
\toprule\noalign{}
\begin{minipage}[b]{\linewidth}\raggedright
ફાયદા
\end{minipage} & \begin{minipage}[b]{\linewidth}\raggedright
ગેરફાયદા
\end{minipage} \\
\midrule\noalign{}
\endhead
\bottomrule\noalign{}
\endlastfoot
ગેઇન સ્થિર કરે છે & સમગ્ર ગેઇન ઘટાડે છે \\
બેન્ડવિડ્થ વધારે છે & વધારાના ઘટકોની જરૂર પડે છે \\
ડિસ્ટોર્શન ઘટાડે છે & યોગ્ય રીતે ડિઝાઇન ન કરવામાં આવે તો ઓસિલેશન થઈ શકે છે \\
નોઇઝ ઘટાડે છે & કાળજીપૂર્વક ફેઝ કોમ્પેન્સેશનની જરૂર પડે છે \\
ઇનપુટ/આઉટપુટ ઇમ્પિડન્સ સુધારે છે & પાવર કન્ઝમ્પશન વધારે છે \\
તાપમાન સંવેદનશીલતા ઘટાડે છે & સર્કિટ વધુ જટિલ બનાવે છે \\
ફ્રિક્વન્સી રિસ્પોન્સ નિયંત્રિત કરે છે & કેટલાક કિસ્સાઓમાં સિગ્નલ-ટુ-નોઇઝ રેશિયો ઘટાડી
શકે છે \\
\end{longtable}
}

\textbf{આકૃતિ:}

\begin{center}
\textbf{Mermaid Diagram (Code)}
\begin{verbatim}
{Shaded}
{Highlighting}[]
graph TD
    A[Negative Feedback] {-{-}{} B[Advantages]}
    A {-{-}{} C[Disadvantages]}

    B {-{-}{} D[Stable Gain]}
    B {-{-}{} E[Wider Bandwidth]}
    B {-{-}{} F[Lower Distortion]}
    B {-{-}{} G[Better Impedance]}
    
    C {-{-}{} H[Reduced Gain]}
    C {-{-}{} I[More Components]}
    C {-{-}{} J[Complex Design]}
{Highlighting}
{Shaded}
\end{verbatim}
\end{center}

\begin{itemize}
\tightlist
\item
  \textbf{પર્ફોર્મન્સ ટ્રેડઓફ}: બેહતર સ્થિરતા અને લિનિયરિટી મેળવવા માટે ગેઇનનો ત્યાગ
  કરે છે
\item
  \textbf{ફ્રિક્વન્સી વિચારણા}: ઉચ્ચ ફ્રિક્વન્સી પર ઓસિલેશન રોકવા માટે કોમ્પેન્સેશનની
  જરૂર પડી શકે છે
\item
  \textbf{ડિઝાઇન જટિલતા}: યોગ્ય રીતે ડિઝાઇન કરવું વધુ જટિલ છે પરંતુ લાંબા ગાળે
  બેહતર કામગીરી આપે છે
\end{itemize}

\end{solutionbox}
\begin{mnemonicbox}
``ગેઇન ગુમાવી, સ્થિરતા મેળવી''

\end{mnemonicbox}
\subsection*{પ્રશ્ન 1(ક) અથવા [7
ગુણ]}\label{uxaaauxab0uxab6uxaa8-1uxa95-uxa85uxaa5uxab5-7-uxa97uxaa3}

\textbf{વોલ્ટેજ શ્રેણી ફીડબેક એમ્પ્લીફાયરને બ્લોક ડાયગ્રામ દોરી વિગતવાર સમજાવો અને
પ્રાયોગિક વોલ્ટેજ શ્રેણી ફિડબક સર્કિટ દોરો.}

\begin{solutionbox}

{\def\LTcaptype{none} % do not increment counter
\begin{longtable}[]{@{}ll@{}}
\toprule\noalign{}
પરિમાણ & વોલ્ટેજ સિરીઝ ફીડબેકમાં અસર \\
\midrule\noalign{}
\endhead
\bottomrule\noalign{}
\endlastfoot
ઇનપુટ સિગ્નલ & વોલ્ટેજ \\
ફીડબેક સિગ્નલ & વોલ્ટેજ \\
ઇનપુટ ઇમ્પિડન્સ & વધે છે \\
આઉટપુટ ઇમ્પિડન્સ & ઘટે છે \\
ગેઇન સ્થિરતા & સુધરે છે \\
બેન્ડવિડ્થ & વધે છે \\
\end{longtable}
}

\textbf{આકૃતિ:}

\begin{center}
\textbf{Mermaid Diagram (Code)}
\begin{verbatim}
{Shaded}
{Highlighting}[]
graph LR
    A[Input Vi] {-{-}{} B["{}+"]}
    B {-{-}{} C[Amplifier A]}
    C {-{-}{} D[Output Vo]}
    D {-{-}{} E[Feedback Network β]}
    E {-{-}{} F["{}{-}"]}
    F {-{-}{} B}

    style C fill:\#bbf,stroke:\#333,stroke{-width:1px}
    style E fill:\#fbb,stroke:\#333,stroke{-width:1px}
{Highlighting}
{Shaded}
\end{verbatim}
\end{center}

\textbf{પ્રાયોગિક સર્કિટ:}

\begin{verbatim}
          +Vcc
            |
            R2
            |
            +{-{-}{-}{-}{-}+}
            |     |
Vin o{-{-}{-}R1{-}{-}+{-}{-}+  |}
            |  |  |
            C1 | C2
            |  |  |
            |  +{-{-}+{-}{-}{-}{-}o Vout}
            |     |
            RE   RC
            |     |
            +{-{-}{-}{-}{-}+}
            |
           GND
\end{verbatim}

\begin{itemize}
\tightlist
\item
  \textbf{સેમ્પલિંગ પદ્ધતિ}: આઉટપુટ વોલ્ટેજ સેમ્પલ કરવામાં આવે છે અને ઇનપુટ પર પાછો
  ફીડ કરવામાં આવે છે
\item
  \textbf{મિક્સિંગ પદ્ધતિ}: ફીડબેક સિગ્નલ ઇનપુટ સિગ્નલ સાથે શ્રેણીમાં મિક્સ કરવામાં
  આવે છે
\item
  \textbf{કાર્ય સિદ્ધાંત}: સુધારેલી સ્થિરતા અને લિનિયરિટી માટે ગેઇન ઘટાડે છે
\item
  \textbf{અનુપ્રયોગો}: ઓડિયો એમ્પલિફાયર્સ, ઇન્સ્ટ્રુમેન્ટેશન એમ્પલિફાયર્સ
\end{itemize}

\end{solutionbox}
\begin{mnemonicbox}
``વોલ્ટેજ સિરીઝ - ઇમ્પિડન્સ ઇન ઉપર, આઉટ નીચે''

\end{mnemonicbox}
\subsection*{પ્રશ્ન 2(અ) [3
ગુણ]}\label{uxaaauxab0uxab6uxaa8-2uxa85-3-uxa97uxaa3}

\textbf{કોલપીટ્સ ઓસીલેટર સર્કિટ પર ટૂંકી નોંધ લખો.}

\begin{solutionbox}

{\def\LTcaptype{none} % do not increment counter
\begin{longtable}[]{@{}ll@{}}
\toprule\noalign{}
ઘટક & કાર્ય \\
\midrule\noalign{}
\endhead
\bottomrule\noalign{}
\endlastfoot
LC ટેંક & ઓસિલેશન ફ્રિક્વન્સી નક્કી કરે છે \\
કેપેસિટીવ વોલ્ટેજ ડિવાઇડર & ફીડબેક પ્રદાન કરે છે \\
સક્રિય ઉપકરણ & ઓસિલેશન જાળવી રાખવા માટે ગેઇન પ્રદાન કરે છે \\
\end{longtable}
}

\textbf{આકૃતિ:}

\begin{verbatim}
     +Vcc
       |
       R1
       |
       +{-{-}{-}{-}{-}+}
       |     |
       |     C3
       |     |
       +{-{-}+{-}{-}+{-}{-}{-}o Output}
       |  |  |
       L1 |  |
       |  |  |
       +{-{-}+  |}
       |     |
       C1    |
       |     |
       +{-{-}{-}{-}{-}+}
       |     |
       C2    |
       |     |
      GND   GND
\end{verbatim}

\begin{itemize}
\tightlist
\item
  \textbf{ફ્રિક્વન્સી સૂત્ર}: f = 1/(2π\sqrt(L\times(C1\timesC2)/(C1+C2)))
\item
  \textbf{ફીડબેક}: કેપેસિટીવ વોલ્ટેજ ડિવાઇડર (C1 અને C2) દ્વારા પ્રદાન કરવામાં આવે
  છે
\item
  \textbf{અનુપ્રયોગો}: RF ઓસિલેટર્સ, કમ્યુનિકેશન સર્કિટ્સ
\end{itemize}

\end{solutionbox}
\begin{mnemonicbox}
``કોલપીટ્સમાં કેપેસિટિવ ડિવાઇડર છે''

\end{mnemonicbox}
\subsection*{પ્રશ્ન 2(બ) [4
ગુણ]}\label{uxaaauxab0uxab6uxaa8-2uxaac-4-uxa97uxaa3}

\textbf{ઓસીલેટરની જરૂરિયાત સમજાવો. i) બાર્કસન માપદંડ. ii) ટેન્ક સર્કિટ. iii)
એમ્પ્લીફાયર.}

\begin{solutionbox}

{\def\LTcaptype{none} % do not increment counter
\begin{longtable}[]{@{}
  >{\raggedright\arraybackslash}p{(\linewidth - 4\tabcolsep) * \real{0.3611}}
  >{\raggedright\arraybackslash}p{(\linewidth - 4\tabcolsep) * \real{0.2778}}
  >{\raggedright\arraybackslash}p{(\linewidth - 4\tabcolsep) * \real{0.3611}}@{}}
\toprule\noalign{}
\begin{minipage}[b]{\linewidth}\raggedright
જરૂરિયાત
\end{minipage} & \begin{minipage}[b]{\linewidth}\raggedright
કાર્ય
\end{minipage} & \begin{minipage}[b]{\linewidth}\raggedright
સમજૂતી
\end{minipage} \\
\midrule\noalign{}
\endhead
\bottomrule\noalign{}
\endlastfoot
બાર્કસન માપદંડ & સતત ઓસિલેશન સુનિશ્ચિત કરે છે & લૂપ ગેઇન = 1, ફેઝ શિફ્ટ = 0^\circ અથવા
360^\circ \\
ટેંક સર્કિટ & ફ્રિક્વન્સી નક્કી કરે છે & ઊર્જા સંગ્રહ કરતી રેઝોનન્ટ LC સર્કિટ \\
એમ્પલિફાયર & ગેઇન પ્રદાન કરે છે & સર્કિટ ખોટને ભરપાઈ કરે છે \\
\end{longtable}
}

\textbf{આકૃતિ:}

\begin{center}
\textbf{Mermaid Diagram (Code)}
\begin{verbatim}
{Shaded}
{Highlighting}[]
graph TD
    A[Oscillator] {-{-}{} B[Barkhausen Criterion]}
    A {-{-}{} C[Tank Circuit]}
    A {-{-}{} D[Amplifier]}

    B {-{-}{} E[Loop Gain = 1]}
    B {-{-}{} F[Phase Shift = 0^ or 360^]}
    
    C {-{-}{} G[Energy Storage]}
    C {-{-}{} H[Frequency Determination]}
    
    D {-{-}{} I[Overcome Losses]}
    D {-{-}{} J[Maintain Amplitude]}
{Highlighting}
{Shaded}
\end{verbatim}
\end{center}

\begin{itemize}
\tightlist
\item
  \textbf{બાર્કસન માપદંડ}: ડેમ્પિંગ વિના સતત ઓસિલેશન માટેની ગાણિતિક શરત
\item
  \textbf{ટેંક સર્કિટ}: ઓસિલેશનની ફ્રિક્વન્સી નક્કી કરતી LC સર્કિટ
\item
  \textbf{એમ્પલિફાયર}: ઓસિલેશન જાળવવા માટે ઊર્જા પ્રદાન કરતું સક્રિય ઉપકરણ
\end{itemize}

\end{solutionbox}
\begin{mnemonicbox}
``BAT - બાર્કસન એમ્પલિફાયર ટેંક''

\end{mnemonicbox}
\subsection*{પ્રશ્ન 2(ક) [7
ગુણ]}\label{uxaaauxab0uxab6uxaa8-2uxa95-7-uxa97uxaa3}

\textbf{UJT ના બાંધકામ, કાર્ય અને V-I લાક્ષણિકતાઓ સમજાવો.}

\begin{solutionbox}

{\def\LTcaptype{none} % do not increment counter
\begin{longtable}[]{@{}ll@{}}
\toprule\noalign{}
પરિમાણ & વર્ણન \\
\midrule\noalign{}
\endhead
\bottomrule\noalign{}
\endlastfoot
બાંધકામ & બે બેઝ કનેક્શન અને એક એમિટર સાથેનો સિલિકોન બાર \\
સિમ્બોલ & એક બાજુએ એમિટર સાથેનો ત્રિકોણ અને બે બેઝ \\
સમકક્ષ સર્કિટ & ડાયોડ સાથેનો વોલ્ટેજ ડિવાઇડર \\
મુખ્ય પરિમાણ & ઇન્ટ્રિન્સિક સ્ટેંડઓફ રેશિયો (η) \\
\end{longtable}
}

\textbf{આકૃતિ:}

\begin{verbatim}
         E
         |
         v
    +{-{-}{-}{-}+{-}{-}{-}{-}+}
    |    |    |
    |    D    |
    |    |    |
B1 o+{-{-}{-}www{-}{-}{-}+o B2}
         R1    R2
     
UJT Symbol \& Equivalent Circuit
\end{verbatim}

\textbf{V-I લાક્ષણિક કર્વ:}

\begin{verbatim}
  I
  \^{}
  |
  |       Peak point
  |         o
  |        /|
  |       / |
  |      /  |
  |     /   |
  |    /    |
  |   /     |
  |  /      |
  | /       |
  |/        |
  +{-{-}{-}{-}{-}{-}{-}{-}{-} V}
  |
  | Valley point
\end{verbatim}

\begin{itemize}
\tightlist
\item
  \textbf{બાંધકામ}: P-ટાઇપ એમિટર જંક્શન સાથેનો N-ટાઇપ સિલિકોન બાર
\item
  \textbf{કાર્ય સિદ્ધાંત}: જ્યારે એમિટર વોલ્ટેજ \textgreater{} (η\timesVBB), ડિવાઇસ
  કન્ડક્ટ કરે છે
\item
  \textbf{ઓપરેશનના વિસ્તારો}: કટ-ઓફ, નેગેટિવ રેસિસ્ટન્સ, અને સેચુરેશન
\item
  \textbf{અનુપ્રયોગો}: રિલેક્સેશન ઓસિલેટર્સ, ટાઇમિંગ સર્કિટ્સ, ટ્રિગરિંગ ડિવાઇસીસ
\end{itemize}

\end{solutionbox}
\begin{mnemonicbox}
``UJT પહેલા ઉંચું પછી નીચું - નકારાત્મક પ્રતિરોધ રાજ કરે''

\end{mnemonicbox}
\subsection*{પ્રશ્ન 2(અ) અથવા [3
ગુણ]}\label{uxaaauxab0uxab6uxaa8-2uxa85-uxa85uxaa5uxab5-3-uxa97uxaa3}

\textbf{હાર્ટલી ઓસીલેટરના ફાયદા, ગેરફાયદા અને એપ્લીકેશન જણાવો.}

\begin{solutionbox}

{\def\LTcaptype{none} % do not increment counter
\begin{longtable}[]{@{}lll@{}}
\toprule\noalign{}
ફાયદા & ગેરફાયદા & અનુપ્રયોગો \\
\midrule\noalign{}
\endhead
\bottomrule\noalign{}
\endlastfoot
સરળ ટ્યુનિંગ & ભારે ઇન્ડક્ટર્સ & RF જનરેટર્સ \\
વિશાળ ફ્રિક્વન્સી રેન્જ & મ્યુચ્યુઅલ ઇન્ડક્ટન્સ સમસ્યાઓ & રેડિયો રિસીવર્સ \\
સરળ ડિઝાઇન & ઉચ્ચ ફ્રિક્વન્સી પર મુશ્કેલ & એમેચ્યોર રેડિયો \\
સારી ફ્રિક્વન્સી સ્થિરતા & સેન્ટર-ટેપ્ડ કોઇલની જરૂર પડે છે & કમ્યુનિકેશન ઇક્વિપમેન્ટ \\
\end{longtable}
}

\textbf{આકૃતિ:}

\begin{verbatim}
            +Vcc
              |
              R1
              |
              +{-{-}{-}{-}{-}+}
              |     |
              |     C2
              |     |
              +{-{-}+{-}{-}+{-}{-}{-}{-}o Output}
              |  |  |
              L1 |  |
              |  |  |
              L2 |  |
              |  |  |
              +{-{-}+  |}
              |     |
              C1    |
              |     |
             GND   GND
\end{verbatim}

\begin{itemize}
\tightlist
\item
  \textbf{મુખ્ય લક્ષણ}: ફીડબેક માટે ટેપ્ડ ઇન્ડક્ટર વાપરે છે
\item
  \textbf{ફ્રિક્વન્સી સૂત્ર}: f = 1/(2π\sqrt(C\times(L1+L2)))
\item
  \textbf{ખાસ લક્ષણ}: ફીડબેક માટે ઇન્ડક્ટિવ વોલ્ટેજ ડિવાઇડર
\end{itemize}

\end{solutionbox}
\begin{mnemonicbox}
``હાર્ટલીમાં હંમેશા ટેપ્ડ ઇન્ડક્ટર''

\end{mnemonicbox}
\subsection*{પ્રશ્ન 2(બ) અથવા [4
ગુણ]}\label{uxaaauxab0uxab6uxaa8-2uxaac-uxa85uxaa5uxab5-4-uxa97uxaa3}

\textbf{UJT ને રિલેક્સેસન ઓસીલેટર તરીકે સમજાવો.}

\begin{solutionbox}

{\def\LTcaptype{none} % do not increment counter
\begin{longtable}[]{@{}ll@{}}
\toprule\noalign{}
ઘટક & કાર્ય \\
\midrule\noalign{}
\endhead
\bottomrule\noalign{}
\endlastfoot
UJT & સ્વીચિંગ ક્રિયા પ્રદાન કરે છે \\
કેપેસિટર & ટાઇમિંગ ઘટક \\
રેસિસ્ટર & ચાર્જિંગ રેટ નિયંત્રિત કરે છે \\
આઉટપુટ & સોટૂથ વેવફોર્મ \\
\end{longtable}
}

\textbf{આકૃતિ:}

\begin{verbatim}
      +Vcc
        |
        R
        |
        +{-{-}{-}{-}{-}{-}{-}+}
        |       |
        |       |
        +{-{-}| |{-}{-}+{-}{-}{-}{-}o Output}
        |   C   |
        |       |
        E       |
       UJT      |
       B1  B2   |
        |   |   |
        +{-{-}{-}+{-}{-}{-}+}
            |
           GND
\end{verbatim}

\textbf{વેવફોર્મ્સ:}

\begin{verbatim}
  Vc
  \^{}
  |  /|  /|  /|
  | / | / | / |
  |/  |/  |/  |
  +{-{-}{-}{-}{-}{-}{-}{-}{-}{-}{-}{-} t}

  Vo
  \^{}
  |
  |  \_   \_   \_
  | | | | | | |
  |\_| |\_| |\_| |\_
  +{-{-}{-}{-}{-}{-}{-}{-}{-}{-}{-}{-} t}
\end{verbatim}

\begin{itemize}
\tightlist
\item
  \textbf{ઓપરેટિંગ પ્રિન્સિપલ}: કેપેસિટર UJT ફાયરિંગ વોલ્ટેજ સુધી ચાર્જ થાય ત્યાં
  સુધી, પછી ઝડપથી ડિસ્ચાર્જ થાય છે
\item
  \textbf{ફ્રિક્વન્સી સૂત્ર}: f \approx 1/(RC\timesln(1/(1-η)))
\item
  \textbf{અનુપ્રયોગો}: ટાઇમિંગ સર્કિટ્સ, પલ્સ જનરેટર્સ, કંટ્રોલ સિસ્ટમ્સ
\end{itemize}

\end{solutionbox}
\begin{mnemonicbox}
``ચાર્જ-ફાયર-રિપીટ - સોટૂથની ધબક''

\end{mnemonicbox}
\subsection*{પ્રશ્ન 2(ક) અથવા [7
ગુણ]}\label{uxaaauxab0uxab6uxaa8-2uxa95-uxa85uxaa5uxab5-7-uxa97uxaa3}

\textbf{વેઇનબ્રિજ ઓસિલેટરનું કાર્ય સુઘડ રેખાકૃતિ સાથે સમજાવો, તેના માટે ફાયદા,
ગેરફાયદા અને એપ્લિકેશન પણ જણાવો.}

\begin{solutionbox}

{\def\LTcaptype{none} % do not increment counter
\begin{longtable}[]{@{}ll@{}}
\toprule\noalign{}
પરિમાણ & વર્ણન \\
\midrule\noalign{}
\endhead
\bottomrule\noalign{}
\endlastfoot
રચના & બ્રિજ ફોર્મેશનમાં RC ફીડબેક નેટવર્ક \\
ફ્રિક્વન્સી સૂત્ર &

f = 1/(2πRC) જ્યારે R1=R3 અને C2=C4 \\

ફીડબેક & RC નેટવર્ક મારફતે પોઝિટિવ ફીડબેક \\
ફેઝ શિફ્ટ & રેઝોનન્ટ ફ્રિક્વન્સી પર 0^\circ \\
\end{longtable}
}

\textbf{આકૃતિ:}

\begin{center}
\textbf{Mermaid Diagram (Code)}
\begin{verbatim}
{Shaded}
{Highlighting}[]
graph LR
    A[Amplifier] {-{-}{} B[RC Bridge]}
    B {-{-}{} A}

    subgraph "Wien Bridge Network"
    direction LR
    C[R1] {-{-}{-} D[C1]}
    D {-{-}{-} E[R2]}
    E {-{-}{-} F[C2]}
    F {-{-}{-} C}
    end
{Highlighting}
{Shaded}
\end{verbatim}
\end{center}

\textbf{સર્કિટ:}

\begin{verbatim}
                +Vcc
                  |
                  |
                  v
    +{-{-}{-}R2{-}{-}{-}+{-}{-}{-}{-}+{-}{-}{-}{-}+}
    |        |         |
    C2       |        R4
    |        |         |
    +{-{-}{-}+    +    +{-}{-}{-}{-}+}
    |   |    |    |    |
    |   +{-{-}{-}{-}+{-}{-}{-}{-}+    |}
    |        |         |
    R1       +        R3
    |        |         |
    C1       v        R5
    |       Op{-Amp     |}
    +{-{-}{-}{-}{-}{-}{-}{-}+         |}
             |         |
             +{-{-}{-}{-}{-}{-}{-}{-}{-}+}
                  |
                 GND
\end{verbatim}

\textbf{ફાયદા:}

\begin{itemize}
\tightlist
\item
  ઉચ્ચ ફ્રિક્વન્સી સ્થિરતા
\item
  ઓછા ડિસ્ટોર્શન આઉટપુટ
\item
  સરળ RC ઘટકો
\item
  સરળતાથી ટ્યુન કરી શકાય
\end{itemize}

\textbf{ગેરફાયદા:}

\begin{itemize}
\tightlist
\item
  મર્યાદિત ફ્રિક્વન્સી રેન્જ
\item
  એમ્પલિટ્યુડ સ્ટેબિલાઇઝેશનની જરૂર
\item
  ઘટક વેરિએશન પ્રત્યે સંવેદનશીલ
\item
  ઓસિલેશન શરૂ કરવા મુશ્કેલ
\end{itemize}

\textbf{અનુપ્રયોગો:}

\begin{itemize}
\tightlist
\item
  ઓડિયો ટેસ્ટ ઇક્વિપમેન્ટ
\item
  ફંક્શન જનરેટર્સ
\item
  સંગીત વાદ્યો
\item
  લેબોરેટરી સિગ્નલ સોર્સીસ
\end{itemize}

\end{solutionbox}
\begin{mnemonicbox}
``વાઇન વર્ક્સ એટ R1C1=R2C2 ફ્રિક્વન્સી''

\end{mnemonicbox}
\subsection*{પ્રશ્ન 3(અ) [3
ગુણ]}\label{uxaaauxab0uxab6uxaa8-3uxa85-3-uxa97uxaa3}

\textbf{પાવર એમ્પલીફાયરનું વર્ગીકરણ આપો.}

\begin{solutionbox}

{\def\LTcaptype{none} % do not increment counter
\begin{longtable}[]{@{}ll@{}}
\toprule\noalign{}
વર્ગીકરણ આધાર & પ્રકારો \\
\midrule\noalign{}
\endhead
\bottomrule\noalign{}
\endlastfoot
કન્ડક્શન એંગલ પર આધારિત & ક્લાસ A, B, AB, C \\
રચના પર આધારિત & સિંગલ-એન્ડેડ, પુશ-પુલ, કોમ્પ્લિમેન્ટરી \\
કપલિંગ પર આધારિત & RC કપલ્ડ, ટ્રાન્સફોર્મર કપલ્ડ, ડાયરેક્ટ કપલ્ડ \\
ઓપરેશન પર આધારિત & લિનિયર, સ્વિચિંગ \\
\end{longtable}
}

\textbf{આકૃતિ:}

\begin{center}
\textbf{Mermaid Diagram (Code)}
\begin{verbatim}
{Shaded}
{Highlighting}[]
graph TD
    A[Power Amplifiers]
    A {-{-}{} B[Class A {-} 360^]}
    A {-{-}{} C[Class B {-} 180^]}
    A {-{-}{} D[Class AB {-} 180^{-}360^]}
    A {-{-}{} E[Class C {} 180^]}

    style B fill:\#d4f0f0,stroke:\#333
    style C fill:\#d4f0f0,stroke:\#333
    style D fill:\#d4f0f0,stroke:\#333
    style E fill:\#d4f0f0,stroke:\#333
{Highlighting}
{Shaded}
\end{verbatim}
\end{center}

\begin{itemize}
\tightlist
\item
  \textbf{ક્લાસ A}: સંપૂર્ણ 360^\circ સાયકલ માટે કન્ડક્ટ કરે છે, સૌથી વધુ લિનિયરિટી,
  સૌથી ઓછી કાર્યક્ષમતા
\item
  \textbf{ક્લાસ B}: 180^\circ સાયકલ માટે કન્ડક્ટ કરે છે, મધ્યમ ડિસ્ટોર્શન, મધ્યમ
  કાર્યક્ષમતા
\item
  \textbf{ક્લાસ AB}: 180^\circ-360^\circ સાયકલ માટે કન્ડક્ટ કરે છે, સારી લિનિયરિટી, સારી
  કાર્યક્ષમતા
\item
  \textbf{ક્લાસ C}: \textless180^\circ સાયકલ માટે કન્ડક્ટ કરે છે, સૌથી વધુ ડિસ્ટોર્શન,
  સૌથી વધુ કાર્યક્ષમતા
\end{itemize}

\end{solutionbox}
\begin{mnemonicbox}
``A આખો સમય, B અર્ધો, AB લગભગ અર્ધો, C વધુ કાપે''

\end{mnemonicbox}
\subsection*{પ્રશ્ન 3(બ) [4
ગુણ]}\label{uxaaauxab0uxab6uxaa8-3uxaac-4-uxa97uxaa3}

\textbf{વર્ગ A પાવર એમ્પલિફાયર સમજાવો.}

\begin{solutionbox}

{\def\LTcaptype{none} % do not increment counter
\begin{longtable}[]{@{}ll@{}}
\toprule\noalign{}
પરિમાણ & ક્લાસ A એમ્પલિફાયર \\
\midrule\noalign{}
\endhead
\bottomrule\noalign{}
\endlastfoot
કન્ડક્શન એંગલ & 360^\circ (પૂર્ણ સાયકલ) \\
બાયસિંગ & લોડ લાઇનના કેન્દ્રમાં Q-પોઇન્ટ \\
કાર્યક્ષમતા & ઓછી (25-30\% મહત્તમ) \\
ડિસ્ટોર્શન & ખૂબ ઓછું \\
\end{longtable}
}

\textbf{આકૃતિ:}

\begin{verbatim}
         +Vcc
           |
           |
         Rcollector
           |
           +{-{-}{-}{-}{-}+}
           |     |
           |     +{-{-}{-} Output}
           |     |
       +{-{-}{-}+     |}
       |   |     |
   In {-+   Q1    |}
       |   |     |
       +{-{-}{-}+     |}
           |     |
         Remitter|
           |     |
           +{-{-}{-}{-}{-}+}
           |
          GND
\end{verbatim}

\textbf{લોડ લાઇન:}

\begin{verbatim}
 Ic
  \^{}
  |          Load Line
  |         /
  |        /
  |       /
  |      *  Q{-point}
  |     /
  |    /
  |   /
  |  /
  | /
  |/
  +{-{-}{-}{-}{-}{-}{-}{-}{-}{-}{-}{-}{-}{-}{-} Vce}
\end{verbatim}

\begin{itemize}
\tightlist
\item
  \textbf{ઓપરેટિંગ પ્રિન્સિપલ}: ટ્રાન્ઝિસ્ટર સમગ્ર ઇનપુટ સાયકલ માટે કન્ડક્ટ કરે છે
\item
  \textbf{કાર્યક્ષમતા ગણતરી}: મહત્તમ સૈદ્ધાંતિક કાર્યક્ષમતા = 50\%
\item
  \textbf{વ્યવહારિક કાર્યક્ષમતા}: સામાન્ય રીતે ખોટ કારણે 25-30\%
\item
  \textbf{અનુપ્રયોગો}: ઓડિયો પ્રી-એમ્પલિફાયર્સ, ઓછી પાવરના એમ્પલિફાયર્સ જ્યાં
  કાર્યક્ષમતા કરતાં ગુણવત્તા વધુ મહત્વની છે
\end{itemize}

\end{solutionbox}
\begin{mnemonicbox}
``ક્લાસ A - હંમેશાં કન્ડકટિંગ, આખો સાયકલ''

\end{mnemonicbox}
\subsection*{પ્રશ્ન 3(ક) [7
ગુણ]}\label{uxaaauxab0uxab6uxaa8-3uxa95-7-uxa97uxaa3}

\textbf{પુશ પુલ એમ્પલીફાયરનો સિદ્ધાંત સમજાવો અને વર્ગ B પુશ પુલ એમ્પલીફાયર પર ટૂંકી
નોંધ લખો.}

\begin{solutionbox}

{\def\LTcaptype{none} % do not increment counter
\begin{longtable}[]{@{}
  >{\raggedright\arraybackslash}p{(\linewidth - 2\tabcolsep) * \real{0.5250}}
  >{\raggedright\arraybackslash}p{(\linewidth - 2\tabcolsep) * \real{0.4750}}@{}}
\toprule\noalign{}
\begin{minipage}[b]{\linewidth}\raggedright
પુશ-પુલ સિદ્ધાંત
\end{minipage} & \begin{minipage}[b]{\linewidth}\raggedright
ક્લાસ B પુશ-પુલ
\end{minipage} \\
\midrule\noalign{}
\endhead
\bottomrule\noalign{}
\endlastfoot
બે પૂરક ઉપકરણો વાપરે છે & દરેક ટ્રાન્ઝિસ્ટર અર્ધા સાયકલ માટે કન્ડક્ટ કરે છે \\
ઇવન હાર્મોનિક ડિસ્ટોર્શન ઘટાડે છે & ઉચ્ચ કાર્યક્ષમતા (78.5\% સૈદ્ધાંતિક) \\
ટ્રાન્સફોર્મરમાં DC મેગ્નેટાઇઝેશનને રદ કરે છે & ક્રોસઓવર ડિસ્ટોર્શનથી પીડાય છે \\
ઉચ્ચ આઉટપુટ પાવર પ્રદાન કરે છે & ડિસ્ટોર્શન ઘટાડવા માટે યોગ્ય બાયસિંગની જરૂર પડે
છે \\
\end{longtable}
}

\textbf{આકૃતિ:}

\begin{verbatim}
           +Vcc
             |
             |
        +{-{-}{-}{-}+{-}{-}{-}{-}+}
        |         |
        Q1        Q2
        |         |
        +{-{-}{-}{-}+{-}{-}{-}{-}+}
             |
             +{-{-}{-}{-}{-}{-} Output}
             |
             R
             |
            GND
\end{verbatim}

\textbf{વેવફોર્મ્સ:}

\begin{verbatim}
  Input      Q1 Current    Q2 Current     Output
    \^{            \^{}             \^{}             \^{}}
    |            |             |             |
    |  /{        |  /         |    /       |  /}
    | /  {       | /          |   /        | /  }
{-{-}{-}{-}+{-}{-}{-}{-}{-}{-}     {-}+{-}{-}{-}{-}{-}{-}      {-}+{-}{-}{-}{-}{-}{-}{-}     {-}+{-}{-}{-}{-}{-}{-}}
    |    {       |             |            |    }
    |     {      |             |            |     }
    |      {     |             |            |      }
    v       v    v             v       v     v       v
\end{verbatim}

\begin{itemize}
\tightlist
\item
  \textbf{કાર્ય સિદ્ધાંત}: દરેક ટ્રાન્ઝિસ્ટર વૈકલ્પિક અર્ધ-સાયકલ માટે કન્ડક્ટ કરે છે
\item
  \textbf{ફાયદા}: ઉચ્ચ કાર્યક્ષમતા, ઓછા ઇવન હાર્મોનિક્સ, ઓછી ગરમી ઉત્પન્ન થાય છે
\item
  \textbf{ગેરફાયદા}: ટ્રાન્ઝિશન પોઇન્ટ્સ પર ક્રોસઓવર ડિસ્ટોર્શન
\item
  \textbf{અનુપ્રયોગો}: ઓડિયો પાવર એમ્પલિફાયર્સ, ઉચ્ચ-પાવર સિસ્ટમના આઉટપુટ સ્ટેજ
\end{itemize}

\end{solutionbox}
\begin{mnemonicbox}
``પુશ-પુલ: જોડીએ પ્રોસેસ કરે અલગ પલસેશન''

\end{mnemonicbox}
\subsection*{પ્રશ્ન 3(અ) અથવા [3
ગુણ]}\label{uxaaauxab0uxab6uxaa8-3uxa85-uxa85uxaa5uxab5-3-uxa97uxaa3}

\textbf{પુશ પુલ એમ્પલીફાયરમાં ક્રોસઓવર ડિસ્ટોરશન ની ચર્ચા કરો. તેને કેવી રીતે દૂર
કરી શકાય છે.}

\begin{solutionbox}

{\def\LTcaptype{none} % do not increment counter
\begin{longtable}[]{@{}ll@{}}
\toprule\noalign{}
ક્રોસઓવર ડિસ્ટોર્શન & ઉકેલ પદ્ધતિઓ \\
\midrule\noalign{}
\endhead
\bottomrule\noalign{}
\endlastfoot
સિગ્નલ ક્રોસઓવર પોઇન્ટ્સ પર થાય છે & નાનો બાયસ વોલ્ટેજ લાગુ કરો (ક્લાસ AB) \\
ટ્રાન્ઝિસ્ટરના નોન-લિનિયર રીજન કારણે & ડાયોડ કોમ્પેન્સેશન નેટવર્ક વાપરો \\
શૂન્યની આસપાસ ``ડેડ ઝોન'' બનાવે છે & ફીડબેક કરેક્શન લાગુ કરો \\
નાના સિગ્નલ્સને વધુ અસર કરે છે & કોમ્પ્લિમેન્ટરી એમિટર-ફોલોઅર સ્ટેજ વાપરો \\
\end{longtable}
}

\textbf{આકૃતિ:}

\begin{verbatim}
  Input          Output with Distortion
    \^{                  \^{}}
    |                  |
    |  /{              |   /}
    | /  {             |  /  }
{-{-}{-}{-}+{-}{-}{-}{-}{-}{-}           {-}+{-}{-}{-}{-}{-}{-}}
    |    {             |     }
    |     {            | gap  }
    |      {           |       }
    v       v          v        v
\end{verbatim}

\textbf{કરેક્શન સર્કિટ:}

\begin{verbatim}
          +Vcc
            |
            |
       +{-{-}{-}{-}+{-}{-}{-}{-}+}
       |    |    |
       |    R    |
       |    |    |
       |    D1   |
       |    |    |
       Q1   +    Q2
       |    |    |
       |    D2   |
       |    |    |
       +{-{-}{-}{-}+{-}{-}{-}{-}+}
            |
            R
            |
           GND
\end{verbatim}

\begin{itemize}
\tightlist
\item
  \textbf{કારણ}: ટ્રાન્ઝિસ્ટર્સને ચાલુ થવા માટે \textasciitilde0.7V જરૂરી છે, જે
  ડેડ ઝોન બનાવે છે
\item
  \textbf{અસર}: ડિસ્ટોર્શન ખાસ કરીને ઓછા વોલ્યુમ પર નોંધપાત્ર રીતે જોવા મળે છે
\item
  \textbf{ઉકેલ}: ડાયોડ્સ અથવા VBE મલ્ટિપ્લાયર સાથે ક્લાસ AB બાયસિંગ
\item
  \textbf{પરિણામ}: પોઝિટિવ અને નેગેટિવ હાફ-સાયકલ વચ્ચે સરળ ટ્રાન્ઝિશન
\end{itemize}

\end{solutionbox}
\begin{mnemonicbox}
``ક્લાસ AB ગેપને સરળ બનાવે''

\end{mnemonicbox}
\subsection*{પ્રશ્ન 3(બ) અથવા [4
ગુણ]}\label{uxaaauxab0uxab6uxaa8-3uxaac-uxa85uxaa5uxab5-4-uxa97uxaa3}

\textbf{કોંપલિમેંટરી સિમેટરી પુશ-પુલ એમ્પલીફાયર સમજાવો.}

\begin{solutionbox}

{\def\LTcaptype{none} % do not increment counter
\begin{longtable}[]{@{}ll@{}}
\toprule\noalign{}
ઘટક & હેતુ \\
\midrule\noalign{}
\endhead
\bottomrule\noalign{}
\endlastfoot
NPN ટ્રાન્ઝિસ્ટર & પોઝિટિવ હાફ-સાયકલ સંભાળે છે \\
PNP ટ્રાન્ઝિસ્ટર & નેગેટિવ હાફ-સાયકલ સંભાળે છે \\
બાયસિંગ નેટવર્ક & ક્રોસઓવર ડિસ્ટોર્શન ઘટાડે છે \\
આઉટપુટ કપલિંગ & લોડમાં ડાયરેક્ટ કપલિંગ \\
\end{longtable}
}

\textbf{આકૃતિ:}

\begin{verbatim}
          +Vcc
            |
            |
            Q1 (NPN)
            |
      R1    |
       +{-{-}{-}{-}+}
       |    |
Input  |    +{-{-}{-}{-}{-}o Output}
       |    |
       +{-{-}{-}{-}+}
            |
            Q2 (PNP)
            |
            |
           GND
\end{verbatim}

\textbf{કાર્ય સિદ્ધાંત:}

\begin{center}
\textbf{Mermaid Diagram (Code)}
\begin{verbatim}
{Shaded}
{Highlighting}[]
graph LR
    A[Input Signal] {-{-}{} B\{Voltage Polarity\}}
    B {-{-}{}|Positive| C[NPN Conducts]}
    B {-{-}{}|Negative| D[PNP Conducts]}
    C {-{-}{} E[Output]}
    D {-{-}{} E}
{Highlighting}
{Shaded}
\end{verbatim}
\end{center}

\begin{itemize}
\tightlist
\item
  \textbf{મુખ્ય લક્ષણ}: પુશ-પુલ ઓપરેશન માટે પૂરક ટ્રાન્ઝિસ્ટર્સ (NPN અને PNP) વાપરે છે
\item
  \textbf{ફાયદો}: આઉટપુટ ટ્રાન્સફોર્મરની જરૂર નથી, લોડમાં ડાયરેક્ટ કપલિંગ
\item
  \textbf{કાર્યક્ષમતા}: સામાન્ય રીતે 78.5\% સૈદ્ધાંતિક મહત્તમ
\item
  \textbf{અનુપ્રયોગો}: ઓડિયો એમ્પલિફાયર્સ, પાવર આઉટપુટ સ્ટેજ
\end{itemize}

\end{solutionbox}
\begin{mnemonicbox}
``NPN ઉપર તાણે, PNP નીચે તાણે''

\end{mnemonicbox}
\subsection*{પ્રશ્ન 3(ક) અથવા [7
ગુણ]}\label{uxaaauxab0uxab6uxaa8-3uxa95-uxa85uxaa5uxab5-7-uxa97uxaa3}

\textbf{વર્ગ B પુશ પુલ એમ્પલીફાયર માટે કાર્યક્ષમતાનું સમીકરણ મેળવો.}

\begin{solutionbox}

{\def\LTcaptype{none} % do not increment counter
\begin{longtable}[]{@{}lll@{}}
\toprule\noalign{}
પરિમાણ & સૂત્ર & વર્ણન \\
\midrule\noalign{}
\endhead
\bottomrule\noalign{}
\endlastfoot
DC ઇનપુટ પાવર & PDC = 2VCC\timesIDC & સપ્લાયમાંથી લેવામાં આવતી પાવર \\
AC આઉટપુટ પાવર & PAC = Vrms^{2}/RL & લોડમાં ડેલિવર થતી પાવર \\
મહત્તમ કાર્યક્ષમતા &

η = (π/4)\times100\% = 78.5\% & સૈદ્ધાંતિક મહત્તમ \\

વ્યવહારિક કાર્યક્ષમતા & 60-70\% & ખોટને ધ્યાનમાં લેતા \\
\end{longtable}
}

\textbf{ગાણિતિક વ્યુત્પત્તિ:}

સાઇનસોઇડલ ઇનપુટ માટે: v(t) = Vm sin(ωt)

\textbf{સ્ટેપ 1}: DC ઇનપુટ પાવર

\begin{itemize}
\tightlist
\item
  પ્રતિ ટ્રાન્ઝિસ્ટર ઇનપુટ કરંટ: Im/π
\item
  કુલ DC ઇનપુટ પાવર: PDC = 2VCC\timesIm/π
\end{itemize}

\textbf{સ્ટેપ 2}: AC આઉટપુટ પાવર

\begin{itemize}
\tightlist
\item
  RMS આઉટપુટ વોલ્ટેજ: Vrms = Vm/\sqrt2
\item
  મહત્તમ આઉટપુટ વોલ્ટેજ: Vm = VCC
\item
  આઉટપુટ પાવર: PAC = Vrms^{2}/RL = Vm^{2}/2RL
\end{itemize}

\textbf{સ્ટેપ 3}: કાર્યક્ષમતા ગણતરી

\begin{itemize}
\tightlist
\item
  η = (PAC/PDC)\times100\%
\item
  η = ((Vm^{2}/2RL)/(2VCC\timesIm/π))\times100\%
\item
  જ્યારે Vm = VCC અને Im = VCC/RL
\item
  η = (π/4)\times100\% = 78.5\%
\end{itemize}

\textbf{આકૃતિ:}

\begin{verbatim}
 Vm=VCC
    \^{}
    |         /{}
    |        /  {}
    |       /    {}
    |      /      {}
    |     /        {}
0   +{-{-}{-}{-}/{-}{-}{-}{-}{-}{-}{-}{-}{-}{-}{-}{-}{-}{-}{-} t}
    |   /            {}
    |  /              {}
    | /                {}
    |/                  {}
    v
\end{verbatim}

\begin{itemize}
\tightlist
\item
  \textbf{પાવર ડિસિપેશન}: આઉટપુટ વોલ્ટેજ સ્વિંગ VCC નજીક પહોંચે ત્યારે સૌથી વધુ
  કાર્યક્ષમ
\item
  \textbf{કન્ડક્શન એંગલ}: દરેક ટ્રાન્ઝિસ્ટર ચોક્કસ 180^\circ માટે કન્ડક્ટ કરે છે
\item
  \textbf{વ્યવહારિક પરિબળો}: બાયસિંગ કરંટ, સેચુરેશન વોલ્ટેજ અને અન્ય ખોટ કાર્યક્ષમતા
  ઘટાડે છે
\item
  \textbf{તુલના}: ક્લાસ A (25-30\%) કરતાં ઘણી ઊંચી, ક્લાસ C
  (\textgreater80\%) કરતાં ઓછી
\end{itemize}

\end{solutionbox}
\begin{mnemonicbox}
``પાઈ-ડિવાઈડ-બાય-4 આપે 78.5\% - ક્લાસ B નું બેસ્ટ''

\end{mnemonicbox}
\subsection*{પ્રશ્ન 4(અ) [3
ગુણ]}\label{uxaaauxab0uxab6uxaa8-4uxa85-3-uxa97uxaa3}

\textbf{વ્યાખ્યાયિત કરો. (i) CMRR (ii)સ્લ્યુ રેટ. (iii)ઇનપુટ ઓફસેટ પ્રવાહ.}

\begin{solutionbox}

{\def\LTcaptype{none} % do not increment counter
\begin{longtable}[]{@{}lll@{}}
\toprule\noalign{}
પરિમાણ & વ્યાખ્યા & સામાન્ય મૂલ્યો \\
\midrule\noalign{}
\endhead
\bottomrule\noalign{}
\endlastfoot
CMRR & ડિફરન્શિયલ ગેઇનનો કોમન-મોડ ગેઇનના ગુણોત્તર & 80-120 dB \\
સ્લ્યુ રેટ & આઉટપુટ વોલ્ટેજના પરિવર્તનનો મહત્તમ દર & 0.5-20 V/μs \\
ઇનપુટ ઓફસેટ કરંટ & બે ઇનપુટ્સમાં જતા કરંટનો તફાવત & 1-100 nA \\
\end{longtable}
}

\textbf{આકૃતિ:}

\begin{center}
\textbf{Mermaid Diagram (Code)}
\begin{verbatim}
{Shaded}
{Highlighting}[]
graph TD
    A[Op{-Amp Parameters]}
    A {-{-}{} B[CMRR = Ad/Acm]}
    A {-{-}{} C[Slew Rate = dVo/dt]}
    A {-{-}{} D["IOS = |I+ {-} I{-}|"]}
    
    style B fill:\#f9f9f9,stroke:\#333
    style C fill:\#f9f9f9,stroke:\#333
    style D fill:\#f9f9f9,stroke:\#333
{Highlighting}
{Shaded}
\end{verbatim}
\end{center}

\begin{itemize}
\tightlist
\item
  \textbf{CMRR}: ઓપ-એમ્પની કોમન-મોડ સિગ્નલ્સને નકારવાની ક્ષમતા માપે છે
\item
  \textbf{સ્લ્યુ રેટ}: અવિકૃત આઉટપુટ માટે મહત્તમ ફ્રિક્વન્સીને મર્યાદિત કરે છે
\item
  \textbf{ઇનપુટ ઓફસેટ કરંટ}: સમાન ઇનપુટ્સ હોવા છતાં આઉટપુટ એરર કરાવે છે
\end{itemize}

\end{solutionbox}
\begin{mnemonicbox}
``ભૂલો રદ કરવા રેશિયો જોઈએ''

\end{mnemonicbox}
\subsection*{પ્રશ્ન 4(બ) [4
ગુણ]}\label{uxaaauxab0uxab6uxaa8-4uxaac-4-uxa97uxaa3}

\textbf{ઓપરેશનલ એમ્પલીફાયરનો મૂળભૂત બ્લોક ડાયાગ્રામ દોરો અને સમજાવો.}

\begin{solutionbox}

{\def\LTcaptype{none} % do not increment counter
\begin{longtable}[]{@{}ll@{}}
\toprule\noalign{}
સ્ટેજ & કાર્ય \\
\midrule\noalign{}
\endhead
\bottomrule\noalign{}
\endlastfoot
ડિફરન્શિયલ ઇનપુટ & ઇનપુટ્સ વચ્ચેના તફાવતને સ્વીકારે અને એમ્પલિફાય કરે છે \\
હાઈ-ગેઇન ઇન્ટરમીડિયેટ & વોલ્ટેજ એમ્પલિફિકેશન પ્રદાન કરે છે \\
લેવલ શિફ્ટર & આઉટપુટ સ્ટેજ માટે DC લેવલ શિફ્ટ કરે છે \\
આઉટપુટ બફર & ઓછો આઉટપુટ ઇમ્પિડન્સ પ્રદાન કરે છે \\
\end{longtable}
}

\textbf{આકૃતિ:}

\begin{center}
\textbf{Mermaid Diagram (Code)}
\begin{verbatim}
{Shaded}
{Highlighting}[]
graph LR
    A[Inverting Input] {-{-}{} B[Differential Input Stage]}
    C[Non{-inverting Input] {-}{-}{} B}
    B {-{-}{} D[High{-}Gain Intermediate Stage]}
    D {-{-}{} E[Level Shifter]}
    E {-{-}{} F[Output Buffer]}
    F {-{-}{} G[Output]}
    
    style B fill:\#d4f0f0,stroke:\#333
    style D fill:\#d4f0f0,stroke:\#333
    style E fill:\#d4f0f0,stroke:\#333
    style F fill:\#d4f0f0,stroke:\#333
{Highlighting}
{Shaded}
\end{verbatim}
\end{center}

\begin{itemize}
\tightlist
\item
  \textbf{ડિફરન્શિયલ ઇનપુટ સ્ટેજ}: ડિફરન્શિયલ ઇનપુટને સિંગલ-એન્ડેડ આઉટપુટમાં કન્વર્ટ
  કરે છે
\item
  \textbf{હાઈ-ગેઇન સ્ટેજ}: મોટાભાગનો ઓપન-લૂપ ગેઇન પ્રદાન કરે છે
\item
  \textbf{લેવલ શિફ્ટર}: યોગ્ય આઉટપુટ ઓપરેશન માટે સિગ્નલ લેવલ શિફ્ટ કરે છે
\item
  \textbf{આઉટપુટ સ્ટેજ}: કરંટ ગેઇન અને ઓછો આઉટપુટ ઇમ્પિડન્સ પ્રદાન કરે છે
\end{itemize}

\end{solutionbox}
\begin{mnemonicbox}
``ડિફ-એમ્પ ગેઇન શિફ્ટ આઉટ''

\end{mnemonicbox}
\subsection*{પ્રશ્ન 4(ક) [7
ગુણ]}\label{uxaaauxab0uxab6uxaa8-4uxa95-7-uxa97uxaa3}

\textbf{ઇન્ટિગ્રેટર તરીકે ઓપરેશનલ એમ્પલીફાયરને વિગતવાર સમજાવો.}

\begin{solutionbox}

{\def\LTcaptype{none} % do not increment counter
\begin{longtable}[]{@{}lll@{}}
\toprule\noalign{}
પરિમાણ & વર્ણન & સૂત્ર \\
\midrule\noalign{}
\endhead
\bottomrule\noalign{}
\endlastfoot
સર્કિટ & ફીડબેકમાં કેપેસિટર સાથે ઓપ-એમ્પ & - \\
ટ્રાન્સફર ફંક્શન & આઉટપુટ ઇનપુટના ઇન્ટિગ્રલને પ્રમાણસર & Vo = -(1/RC)\intVi dt \\
ફ્રિક્વન્સી રિસ્પોન્સ & લો-પાસ ફિલ્ટર તરીકે કાર્ય કરે છે & ગેઇન = 1/(jωRC) \\
ફેઝ શિફ્ટ & -90^\circ & - \\
\end{longtable}
}

\textbf{આકૃતિ:}

\begin{verbatim}
              C
       +{-{-}{-}{-}{-}{-}||{-}{-}{-}{-}{-}{-}+}
       |              |
       |    +{-{-}{-}{-}{-}+   |}
       |    |     |   |
       +{-{-}{-}{-}+  {-}  |   |}
       |    |     |   |
Vin o{-{-}+{-}{-}R{-}+     +{-}{-}{-}+{-}{-}o Vout}
            |  +  |
            |     |
            +{-{-}{-}{-}{-}+}
              |
             GND
\end{verbatim}

\textbf{ઇનપુટ/આઉટપુટ વેવફોર્મ્સ:}

\begin{verbatim}
 Input Square Wave     Output Triangle Wave
       \_\_\_                      /|
      |   |                    / |
      |   |                   /  |
  \_\_\_\_|   |\_\_\_\_         \_\_\_\_/    |{\_\_\_\_}
      |   |                      |
      |   |                      |
      |\_\_\_|                      |
                                {|}
\end{verbatim}

\begin{itemize}
\tightlist
\item
  \textbf{કાર્ય સિદ્ધાંત}: કેપેસિટર સમય સાથે કરંટને ઇન્ટિગ્રેટ કરે છે
\item
  \textbf{ગાણિતિક આધાર}: Vo(t) = -(1/RC)\intVi(t)dt + Vo(0)
\item
  \textbf{મર્યાદાઓ}: કેપેસિટર લીકેજ, ઓપ-એમ્પ ઇનપુટ બાયસ કરંટ ડ્રિફ્ટ ઉત્પન્ન કરે છે
\item
  \textbf{અનુપ્રયોગો}: વેવફોર્મ જનરેટર્સ, એનાલોગ કમ્પ્યુટર્સ, એક્ટિવ ફિલ્ટર્સ
\end{itemize}

\end{solutionbox}
\begin{mnemonicbox}
``સ્ક્વેર-ઇન ટ્રાયેંગલ-આઉટ, RC સેટ્સ ધ સ્લોપ''

\end{mnemonicbox}
\subsection*{પ્રશ્ન 4(અ) અથવા [3
ગુણ]}\label{uxaaauxab0uxab6uxaa8-4uxa85-uxa85uxaa5uxab5-3-uxa97uxaa3}

\textbf{ઓપરેશનલ એમ્પલીફાયરને સમિંગ એમ્પલીફાયર તરીકે સમજાવો.}

\begin{solutionbox}

{\def\LTcaptype{none} % do not increment counter
\begin{longtable}[]{@{}
  >{\raggedright\arraybackslash}p{(\linewidth - 4\tabcolsep) * \real{0.3333}}
  >{\raggedright\arraybackslash}p{(\linewidth - 4\tabcolsep) * \real{0.3939}}
  >{\raggedright\arraybackslash}p{(\linewidth - 4\tabcolsep) * \real{0.2727}}@{}}
\toprule\noalign{}
\begin{minipage}[b]{\linewidth}\raggedright
પરિમાણ
\end{minipage} & \begin{minipage}[b]{\linewidth}\raggedright
વર્ણન
\end{minipage} & \begin{minipage}[b]{\linewidth}\raggedright
સૂત્ર
\end{minipage} \\
\midrule\noalign{}
\endhead
\bottomrule\noalign{}
\endlastfoot
સર્કિટ & સમાન ફીડબેક સાથે મલ્ટિપલ ઇનપુટ્સ & Vo = -(R_{1}/R_{1}\timesV_{1} + R_{1}/R_{2}\timesV_{2} +
\ldots) \\
સમાન રેસિસ્ટર્સ & સરળ યોગ/સરેરાશ & Vo = -(V_{1} + V_{2} + \ldots{} + V_{n}) \\
વેઇટેડ સમ & અલગ ઇનપુટ રેસિસ્ટર્સ & Vo = -(K_{1}V_{1} + K_{2}V_{2} + \ldots{} + K_{n}V_{n}) \\
ઇન્વર્ટિંગ & ઇનપુટ્સથી આઉટપુટ ઇન્વર્ટેડ થયેલો & - \\
\end{longtable}
}

\textbf{આકૃતિ:}

\begin{verbatim}
        R1
V1 o{-{-}{-}www{-}{-}{-}+}
             |
        R2   |    +{-{-}{-}{-}{-}+}
V2 o{-{-}{-}www{-}{-}{-}+{-}{-}{-}{-}+     |}
             |    |  {-  |}
        R3   |    |     |
V3 o{-{-}{-}www{-}{-}{-}+{-}{-}{-}{-}+     +{-}{-}{-}o Vout}
                  |  +  |
                  |     |
                  +{-{-}{-}{-}{-}+}
                    |
          Rf        |
        +{-{-}{-}www{-}{-}{-}{-}{-}+}
        |
       GND
\end{verbatim}

\begin{itemize}
\tightlist
\item
  \textbf{કાર્ય સિદ્ધાંત}: દરેક ઇનપુટ સમિંગ જંક્શનમાં કરંટ યોગદાન આપે છે
\item
  \textbf{અનુપ્રયોગો}: ઓડિયો મિક્સર્સ, સિગ્નલ પ્રોસેસિંગ, એનાલોગ કમ્પ્યુટર્સ
\item
  \textbf{વર્ચ્યુઅલ ગ્રાઉન્ડ}: સમિંગ પોઇન્ટ લગભગ-શૂન્ય વોલ્ટેજ જાળવે છે
\item
  \textbf{વેરિએશન્સ}: ઇન્વર્ટિંગ, નોન-ઇન્વર્ટિંગ અને ડિફરન્શિયલ સમર
\end{itemize}

\end{solutionbox}
\begin{mnemonicbox}
``ઘણા ઇનપુટ, એક આઉટપુટ - બધું બેરેબાર''

\end{mnemonicbox}
\subsection*{પ્રશ્ન 4(બ) અથવા [4
ગુણ]}\label{uxaaauxab0uxab6uxaa8-4uxaac-uxa85uxaa5uxab5-4-uxa97uxaa3}

\textbf{ઓપરેશનલ એમ્પલીફાયરના ઉપયોગો જણાવો.}

\begin{solutionbox}

{\def\LTcaptype{none} % do not increment counter
\begin{longtable}[]{@{}ll@{}}
\toprule\noalign{}
અનુપ્રયોગ કેટેગરી & ઉદાહરણો \\
\midrule\noalign{}
\endhead
\bottomrule\noalign{}
\endlastfoot
સિગ્નલ પ્રોસેસિંગ & એમ્પલિફાયર્સ, ફિલ્ટર્સ, બફર્સ \\
ગાણિતિક ઓપરેશન્સ & એડર્સ, સબટ્રેક્ટર્સ, ઇન્ટિગ્રેટર્સ, ડિફરન્શિએટર્સ \\
વેવફોર્મ જનરેટર્સ & સાઇન, સ્ક્વેર, ટ્રાયેંગલ, પલ્સ જનરેટર્સ \\
ઇન્સ્ટ્રુમેન્ટેશન & ઇન્સ્ટ્રુમેન્ટેશન એમ્પલિફાયર્સ, કરંટ-ટુ-વોલ્ટેજ કન્વર્ટર્સ \\
કોમ્પેરેટર્સ & ઝીરો ક્રોસિંગ ડિટેક્ટર્સ, વિન્ડો કોમ્પેરેટર્સ \\
પ્રિસિઝન રેક્ટિફાયર્સ & ફુલ-વેવ, હાફ-વેવ રેક્ટિફાયર્સ \\
વોલ્ટેજ રેગ્યુલેટર્સ & સિરીઝ રેગ્યુલેટર્સ, શંટ રેગ્યુલેટર્સ \\
\end{longtable}
}

\textbf{આકૃતિ:}

\begin{center}
\textbf{Mermaid Diagram (Code)}
\begin{verbatim}
{Shaded}
{Highlighting}[]
graph TD
    A[Op{-Amp Applications]}
    A {-{-}{} B[Signal Processing]}
    A {-{-}{} C[Math Operations]}
    A {-{-}{} D[Waveform Generators]}
    A {-{-}{} E[Instrumentation]}
    A {-{-}{} F[Comparators]}
    A {-{-}{} G[Rectifiers]}
    A {-{-}{} H[Regulators]}
{Highlighting}
{Shaded}
\end{verbatim}
\end{center}

\begin{itemize}
\tightlist
\item
  \textbf{લિનિયર અનુપ્રયોગો}: એમ્પલિફિકેશન, ફિલ્ટરિંગ માટે લિનિયર રીજનમાં ઓપ-એમ્પ
  વાપરે છે
\item
  \textbf{નોન-લિનિયર અનુપ્રયોગો}: કમ્પેરિઝન, લિમિટેશન માટે સેચુરેશન લક્ષણો વાપરે છે
\item
  \textbf{એનાલોગ કોમ્પ્યુટેશન}: એનાલોગ સિગ્નલ પર ગાણિતિક ઓપરેશન્સ કરવા
\item
  \textbf{સિગ્નલ કન્ડિશનિંગ}: એનાલોગ-ટુ-ડિજિટલ કન્વર્ઝન માટે સિગ્નલ્સ અડેપ્ટ કરવા
\end{itemize}

\end{solutionbox}
\begin{mnemonicbox}
``SMWIG-CR: સિગ્નલ, મેથ, વેવ, ઇન્સ્ટ્રુમેન્ટ, ગેટ, કન્વર્ટ,
રેગ્યુલેટ''

\end{mnemonicbox}
\subsection*{પ્રશ્ન 4(ક) અથવા [7
ગુણ]}\label{uxaaauxab0uxab6uxaa8-4uxa95-uxa85uxaa5uxab5-7-uxa97uxaa3}

\textbf{ઓપરેશનલ એંપ્લિફાયર ને ઇનવરટિંગ અને નોન-ઈનવરટિંગ અંપ્લિફાયર તરીકે સમજાવો.}

\begin{solutionbox}

{\def\LTcaptype{none} % do not increment counter
\begin{longtable}[]{@{}lll@{}}
\toprule\noalign{}
પરિમાણ & ઇન્વર્ટિંગ એમ્પલિફાયર & નોન-ઇન્વર્ટિંગ એમ્પલિફાયર \\
\midrule\noalign{}
\endhead
\bottomrule\noalign{}
\endlastfoot
સર્કિટ કન્ફિગરેશન & નેગેટિવ ટર્મિનલ પર ઇનપુટ & પોઝિટિવ ટર્મિનલ પર ઇનપુટ \\
ગેઇન ફોર્મ્યુલા &

A = -Rf/Rin &

A = 1 + Rf/Rin \\

ઇનપુટ ઇમ્પિડન્સ & = Rin & ખૂબ ઊંચી (\approx 10^{9} ohms) \\
ફેઝ શિફ્ટ & 180^\circ & 0^\circ \\
વર્ચ્યુઅલ ગ્રાઉન્ડ & નેગેટિવ ઇનપુટ પર & લાગુ પડતું નથી \\
\end{longtable}
}

\textbf{ઇન્વર્ટિંગ એમ્પલિફાયર:}

\begin{verbatim}
            Rf
        +{-{-}{-}{-}www{-}{-}{-}+}
        |          |
        |   +{-{-}{-}{-}{-}+|}
        |   |     ||
Vin o{-{-}{-}w{-}{-}{-}+  {-}  ||}
       Rin  |     ||
            |     +{-{-}{-}o Vout}
            |  +  |
            |     |
            +{-{-}{-}{-}{-}+}
              |
             GND
\end{verbatim}

\textbf{નોન-ઇન્વર્ટિંગ એમ્પલિફાયર:}

\begin{verbatim}
                Rf
            +{-{-}{-}{-}www{-}{-}{-}+}
            |          |
            |   +{-{-}{-}{-}{-}+|}
            |   |     ||
Vin o{-{-}{-}{-}{-}{-}{-}+{-}{-}{-}+  +  ||}
            |   |     ||
            |   |     +{-{-}{-}o Vout}
            |   |  {-  |}
            |   |     |
            |   +{-{-}{-}{-}{-}+}
            |     |
            |    GND
            |
            +{-{-}{-}{-}www{-}{-}{-}{-}{-}+}
                 Rin     |
                        GND
\end{verbatim}

\textbf{ઇન્વર્ટિંગ મોડ:}

\begin{itemize}
\tightlist
\item
  \textbf{ગેઇન સમીકરણ}: Vout = -(Rf/Rin)\timesVin
\item
  \textbf{વર્ચ્યુઅલ ગ્રાઉન્ડ}: નેગેટિવ ઇનપુટ \textasciitilde0V પર જાળવવામાં આવે છે
\item
  \textbf{અનુપ્રયોગો}: સિગ્નલ ઇન્વર્ઝન, નિયંત્રિત ગેઇન, સમિંગ
\end{itemize}

\textbf{નોન-ઇન્વર્ટિંગ મોડ:}

\begin{itemize}
\tightlist
\item
  \textbf{ગેઇન સમીકરણ}: Vout = (1 + Rf/Rin)\timesVin
\item
  \textbf{લઘુત્તમ ગેઇન}: હંમેશા \geq 1
\item
  \textbf{અનુપ્રયોગો}: બફરિંગ, ઊંચા ઇનપુટ ઇમ્પિડન્સ સાથે વોલ્ટેજ એમ્પલિફિકેશન
\end{itemize}

\end{solutionbox}
\begin{mnemonicbox}
``ઇન્વર્ટ: નેગેટિવ ઇનપુટ લે, નોન-ઇન્વર્ટ: પોઝિટિવ સિગ્નલ લે''

\end{mnemonicbox}
\subsection*{પ્રશ્ન 5(અ) [3
ગુણ]}\label{uxaaauxab0uxab6uxaa8-5uxa85-3-uxa97uxaa3}

\textbf{IC555 નું પિન વર્ણન આપો.}

\begin{solutionbox}

{\def\LTcaptype{none} % do not increment counter
\begin{longtable}[]{@{}lll@{}}
\toprule\noalign{}
પિન નંબર & પિન નામ & વર્ણન \\
\midrule\noalign{}
\endhead
\bottomrule\noalign{}
\endlastfoot
1 & ગ્રાઉન્ડ & સર્કિટ ગ્રાઉન્ડ સાથે જોડાયેલ \\
2 & ટ્રિગર & \textless{} 1/3 VCC હોય ત્યારે ટાઇમિંગ સાયકલ શરૂ કરે છે \\
3 & આઉટપુટ & આઉટપુટ સિગ્નલ પ્રદાન કરે છે \\
4 & રીસેટ & LOW હોય ત્યારે ટાઇમિંગ સમાપ્ત કરે છે \\
5 & કંટ્રોલ વોલ્ટેજ & થ્રેશોલ્ડ વોલ્ટેજ એડજસ્ટ કરે છે \\
6 & થ્રેશોલ્ડ & \textgreater{} 2/3 VCC હોય ત્યારે ટાઇમિંગ સાયકલ સમાપ્ત કરે છે \\
7 & ડિસ્ચાર્જ & ટાઇમિંગ કેપેસિટર સાથે જોડાયેલ \\
8 & VCC & પોઝિટિવ સપ્લાય વોલ્ટેજ (5-15V) \\
\end{longtable}
}

\textbf{આકૃતિ:}

\begin{verbatim}
    +{-{-}{-}{-}{-}{-}{-}{-}+}
  8 |        | 7
+{-{-}{-}+ VCC    | DISCHARGE +{-}{-}{-}+}
    |        |               |
  7 |        | 6             |
+{-{-}{-}+ DISCHARGE THRESHOLD +{-}{-}+}
    |        |               |
  6 |        | 5             |
+{-{-}{-}+ THRESHOLD CONTROL   +{-}{-}+}
    |        |               |
  5 |        | 4             |
+{-{-}{-}+ CONTROL  RESET     +{-}{-}{-}+}
    |        |               |
  4 |        | 3             |
+{-{-}{-}+ RESET   OUTPUT    +{-}{-}{-}{-}+}
    |        |               |
  3 |        | 2             |
+{-{-}{-}+ OUTPUT  TRIGGER   +{-}{-}{-}{-}+}
    |        |               |
  2 |        | 1             |
+{-{-}{-}+ TRIGGER GND       +{-}{-}{-}{-}+}
    |        |
    +{-{-}{-}{-}{-}{-}{-}{-}+}
\end{verbatim}

\begin{itemize}
\tightlist
\item
  \textbf{ઇનપુટ પિન્સ}: ટ્રિગર, રીસેટ, થ્રેશોલ્ડ, કંટ્રોલ વોલ્ટેજ
\item
  \textbf{આઉટપુટ પિન્સ}: આઉટપુટ, ડિસ્ચાર્જ
\item
  \textbf{પાવર પિન્સ}: VCC, ગ્રાઉન્ડ
\item
  \textbf{આંતરિક સ્ટ્રક્ચર}: કોમ્પેરેટર્સ, ફ્લિપ-ફ્લોપ, ડિસ્ચાર્જ ટ્રાન્ઝિસ્ટરથી બનેલું છે
\end{itemize}

\end{solutionbox}
\begin{mnemonicbox}
``ગ્રાઉન્ડ ટ્રિગર આઉટપુટ રીસેટ કંટ્રોલ થ્રેશોલ્ડ ડિસ્ચાર્જ
વોલ્ટેજ''

\end{mnemonicbox}
\subsection*{પ્રશ્ન 5(બ) [4
ગુણ]}\label{uxaaauxab0uxab6uxaa8-5uxaac-4-uxa97uxaa3}

\textbf{દિફ્ફેરેંટિયાટર તરીકે op-amp સમજાવો.}

\begin{solutionbox}

{\def\LTcaptype{none} % do not increment counter
\begin{longtable}[]{@{}
  >{\raggedright\arraybackslash}p{(\linewidth - 4\tabcolsep) * \real{0.3333}}
  >{\raggedright\arraybackslash}p{(\linewidth - 4\tabcolsep) * \real{0.3939}}
  >{\raggedright\arraybackslash}p{(\linewidth - 4\tabcolsep) * \real{0.2727}}@{}}
\toprule\noalign{}
\begin{minipage}[b]{\linewidth}\raggedright
પરિમાણ
\end{minipage} & \begin{minipage}[b]{\linewidth}\raggedright
વર્ણન
\end{minipage} & \begin{minipage}[b]{\linewidth}\raggedright
સૂત્ર
\end{minipage} \\
\midrule\noalign{}
\endhead
\bottomrule\noalign{}
\endlastfoot
સર્કિટ & ઇનપુટમાં કેપેસિટર સાથેનો ઓપ-એમ્પ & Vo = -RC(dVi/dt) \\
ટ્રાન્સફર ફંક્શન & આઉટપુટ પરિવર્તનના દરને પ્રમાણસર & H(s) = -sRC \\
ફ્રિક્વન્સી રિસ્પોન્સ & હાઈ-પાસ ફિલ્ટર તરીકે કાર્ય કરે છે & ગેઇન ફ્રિક્વન્સી સાથે વધે
છે \\
ફેઝ શિફ્ટ & +90^\circ & - \\
\end{longtable}
}

\textbf{આકૃતિ:}

\begin{verbatim}
                 R
            +{-{-}{-}{-}www{-}{-}{-}{-}+}
            |           |
            |    +{-{-}{-}{-}{-}+|}
            |    |     ||
Vin o{-{-}{-}{-}{-}{-}{-}|{-}{-}{-}{-}+  {-}  ||}
            ||   |     ||
            ||   |     +{-{-}{-}{-}o Vout}
            |C   |  +  |
            ||   |     |
            |    +{-{-}{-}{-}{-}+}
            |      |
           GND    GND
\end{verbatim}

\textbf{ઇનપુટ/આઉટપુટ વેવફોર્મ્સ:}

\begin{verbatim}
  Triangle Input       Square Output
       /{                 \_}
      /  {               | |}
     /    {              | |}
\_\_\_\_/      {\_\_\_\_    \_\_\_\_\_| |\_\_\_\_\_}
               {         | |}
                {        | |}
                 {       | |}
                  {     \_| |\_}
\end{verbatim}

\begin{itemize}
\tightlist
\item
  \textbf{કાર્ય સિદ્ધાંત}: આઉટપુટ વોલ્ટેજ ઇનપુટના પરિવર્તન દરને પ્રમાણસર છે
\item
  \textbf{ગાણિતિક આધાર}: Vo = -RC(dVin/dt)
\item
  \textbf{વ્યવહારિક મર્યાદાઓ}: ઉચ્ચ-આવૃત્તિના નોઇઝ પ્રત્યે સંવેદનશીલ
\item
  \textbf{અનુપ્રયોગો}: વેવફોર્મ જનરેશન, એજ ડિટેક્શન, રેટ-ઓફ-ચેન્જ ઇન્ડિકેટર
\end{itemize}

\end{solutionbox}
\begin{mnemonicbox}
``ડિફરન્શિએટર ડેરિવેટિવ્સ આપે - RC સ્પીડ નક્કી કરે''

\end{mnemonicbox}
\subsection*{પ્રશ્ન 5(ક) [7
ગુણ]}\label{uxaaauxab0uxab6uxaa8-5uxa95-7-uxa97uxaa3}

\textbf{IC 555 ને અસ્ટેબલ અને મોનોસ્ટેબલ મલ્ટિવાઇબ્રેટર તરીકે સમજાવો.}

\begin{solutionbox}

{\def\LTcaptype{none} % do not increment counter
\begin{longtable}[]{@{}lll@{}}
\toprule\noalign{}
પરિમાણ & અસ્ટેબલ મલ્ટિવાઇબ્રેટર & મોનોસ્ટેબલ મલ્ટિવાઇબ્રેટર \\
\midrule\noalign{}
\endhead
\bottomrule\noalign{}
\endlastfoot
વ્યાખ્યા & ફ્રી-રનિંગ ઓસિલેટર & વન-શોટ પલ્સ જનરેટર \\
સ્ટેબલ સ્ટેટ્સ & કોઈ નહીં (સતત ઓસિલેટ) & એક સ્ટેબલ સ્ટેટ \\
ટાઇમિંગ &

T = 0.693(RA+2RB)C &

T = 1.1RC \\

ટ્રિગર & સેલ્ફ-ટ્રિગરિંગ & બાહ્ય ટ્રિગરની જરૂર \\
આઉટપુટ & સતત સ્ક્વેર વેવ & ફિક્સ્ડ પહોળાઈનો સિંગલ પલ્સ \\
\end{longtable}
}

\textbf{અસ્ટેબલ સર્કિટ:}

\begin{verbatim}
        +Vcc
         |
         |
      +{-{-}+{-}{-}+}
      |     |
      R1    |
      |     |
      +{-{-}+{-}{-}+{-}{-}{-}{-}{-}{-}{-}{-}+}
      |  |           |
      |  +{-{-}+        |}
      |     |   8    |     7
      R2    +{-{-}{-}+{-}{-}{-}{-}{-}{-}{-}+{-}{-}{-}+}
      |         |       |   |
      +{-{-}{-}{-}{-}{-}+  | 555   |   |}
      |      |  |       |   |
      C1     |  |       |   |
      |      |  |       |   |
      +{-{-}{-}{-}{-}{-}+{-}{-}+{-}{-}{-}{-}{-}{-}{-}+{-}{-}{-}+}
             |  |       |   |
           2 |  |   3   |   |
      +{-{-}{-}{-}{-}{-}+{-}{-}+{-}{-}{-}{-}{-}{-}{-}+   |}
      |         |           |
      |         |           |
      +{-{-}{-}{-}{-}{-}{-}{-}{-}+{-}{-}{-}{-}{-}{-}{-}{-}{-}{-}{-}+}
                |
              Output
\end{verbatim}

\textbf{મોનોસ્ટેબલ સર્કિટ:}

\begin{verbatim}
     +Vcc
      |
      |
      R
      |
      +{-{-}{-}{-}{-}{-}{-}{-}+{-}{-}+}
      |        |  |
      |    8   |  |
      +{-{-}{-}{-}+{-}{-}{-}{-}{-}{-}{-}+{-}{-}{-}{-}+}
      |    |       |    |
      |    | 555   |    |
      |    |       |    |
      |    |       |    |
      |  4 |       | 7  |
      +{-{-}{-}{-}+{-}{-}{-}{-}{-}{-}{-}+{-}{-}{-}{-}+}
           |       |    |
         2 |       |    |
      +{-{-}{-}{-}+       |    |}
      |    |       |    |
      |  3 |       |    |
      +{-{-}{-}{-}+{-}{-}{-}{-}{-}{-}{-}+    |}
           |            |
         Output         |
           |            |
           +{-{-}{-}{-}{-}+{-}{-}{-}{-}{-}{-}+}
                 |
                 C
                 |
                GND
\end{verbatim}

\textbf{અસ્ટેબલ ઓપરેશન:}

\begin{itemize}
\tightlist
\item
  \textbf{કાર્ય}: કેપેસિટર RA+RB મારફતે ચાર્જ થાય છે અને RB મારફતે ડિસ્ચાર્જ થાય છે
\item
  \textbf{ડ્યુટી સાયકલ}: RA અને RB ના યોગ્ય પસંદગીથી એડજસ્ટ કરી શકાય છે
\item
  \textbf{ફ્રિક્વન્સી}: f = 1.44/((RA+2RB)C)
\item
  \textbf{અનુપ્રયોગો}: LED ફ્લેશર્સ, ટોન જનરેટર્સ, ક્લોક પલ્સ જનરેટર્સ
\end{itemize}

\textbf{મોનોસ્ટેબલ ઓપરેશન:}

\begin{itemize}
\tightlist
\item
  \textbf{કાર્ય}: પિન 2 પર ફોલિંગ એજથી ટ્રિગર થાય છે, સમય T માટે HIGH આઉટપુટ
  આપે છે
\item
  \textbf{સમય અવધિ}: T = 1.1RC
\item
  \textbf{અનુપ્રયોગો}: ટાઇમ ડિલે, પલ્સ વિડ્થ મોડ્યુલેશન, ડિબાઉન્સિંગ
\end{itemize}

\end{solutionbox}
\begin{mnemonicbox}
``અસ્ટેબલ હંમેશાં બદલે, મોનોસ્ટેબલ એક પલ્સ બનાવે''

\end{mnemonicbox}
\subsection*{પ્રશ્ન 5(અ) અથવા [3
ગુણ]}\label{uxaaauxab0uxab6uxaa8-5uxa85-uxa85uxaa5uxab5-3-uxa97uxaa3}

\textbf{IC555 ને બાયસ્ટેબલ માલતિવાયબરેટર તરીકે સમજાવો.}

\begin{solutionbox}

{\def\LTcaptype{none} % do not increment counter
\begin{longtable}[]{@{}ll@{}}
\toprule\noalign{}
પરિમાણ & વર્ણન \\
\midrule\noalign{}
\endhead
\bottomrule\noalign{}
\endlastfoot
વ્યાખ્યા & બે સ્ટેબલ સ્ટેટ્સ ધરાવતી ફ્લિપ-ફ્લોપ સર્કિટ \\
ટ્રિગરિંગ & ટ્રિગર પિન (2) દ્વારા SET, રીસેટ પિન (4) દ્વારા RESET \\
સ્ટેબલ સ્ટેટ્સ & બે (HIGH અથવા LOW) \\
સમય અવધિ & ટાઇમિંગ ઘટકોની જરૂર નથી \\
\end{longtable}
}

\textbf{આકૃતિ:}

\begin{verbatim}
           +Vcc
            |
            |
       +{-{-}{-}{-}|{-}{-}{-}{-}+}
       |    |    |
       |  8 |  4 |
   +{-{-}{-}+{-}{-}{-}{-}+{-}{-}{-}{-}+{-}{-}{-}+}
   |   |         |   |
   |   |   555   |   |
   |   |         |   |
   |   |         |   |
   | 2 |         | 3 |
   +{-{-}{-}+{-}{-}{-}{-}{-}{-}{-}{-}{-}+{-}{-}{-}+}
       |         |
    Trigger    Output
       |         |
      GND       GND
\end{verbatim}

\textbf{ટ્રુથ ટેબલ:}

{\def\LTcaptype{none} % do not increment counter
\begin{longtable}[]{@{}lll@{}}
\toprule\noalign{}
ટ્રિગર (પિન 2) & રીસેટ (પિન 4) & આઉટપુટ (પિન 3) \\
\midrule\noalign{}
\endhead
\bottomrule\noalign{}
\endlastfoot
\textless{} 1/3 VCC & HIGH & HIGH \\
\textgreater{} 1/3 VCC & HIGH & No change \\
Any & LOW & LOW \\
\end{longtable}
}

\begin{itemize}
\tightlist
\item
  \textbf{SET ઓપરેશન}: ટ્રિગર પિન 1/3 VCC કરતાં નીચે જાય ત્યારે થાય છે
\item
  \textbf{RESET ઓપરેશન}: રીસેટ પિન LOW ખેંચવામાં આવે ત્યારે થાય છે
\item
  \textbf{અનુપ્રયોગો}: લેચિંગ સ્વિચ, મેમરી એલિમેન્ટ્સ, ફ્લિપ-ફ્લોપ્સ
\item
  \textbf{લક્ષણો}: ટાઇમિંગ ઘટકો (R, C) ની જરૂર નથી
\end{itemize}

\end{solutionbox}
\begin{mnemonicbox}
``બાયસ્ટેબલ બે સ્ટેટમાં આવજા કરે''

\end{mnemonicbox}
\subsection*{પ્રશ્ન 5(બ) અથવા [4
ગુણ]}\label{uxaaauxab0uxab6uxaa8-5uxaac-uxa85uxaa5uxab5-4-uxa97uxaa3}

\textbf{આંતરિક બ્લોક ડાયાગ્રામ સાથે IC555 ની મૂળભૂત કામગીરી સમજાવો.}

\begin{solutionbox}

{\def\LTcaptype{none} % do not increment counter
\begin{longtable}[]{@{}ll@{}}
\toprule\noalign{}
બ્લોક & કાર્ય \\
\midrule\noalign{}
\endhead
\bottomrule\noalign{}
\endlastfoot
કોમ્પેરેટર્સ & ટ્રિગર અને થ્રેશોલ્ડ વોલ્ટેજનું મોનિટરિંગ કરે છે \\
ફ્લિપ-ફ્લોપ & આઉટપુટ સ્ટેટને નિયંત્રિત કરે છે \\
ડિસ્ચાર્જ ટ્રાન્ઝિસ્ટર & ટાઇમિંગ કેપેસિટરને ડિસ્ચાર્જ કરે છે \\
વોલ્ટેજ ડિવાઇડર & રેફરન્સ વોલ્ટેજ સ્થાપિત કરે છે \\
\end{longtable}
}

\textbf{આંતરિક બ્લોક ડાયાગ્રામ:}

\begin{verbatim}
                         +Vcc (8)
                            |
                            v
             +{-{-}{-}{-}{-}{-}{-}{-}{-}{-}{-}{-}{-}{-}{-}{-}{-}{-}{-}{-}{-}{-}{-}{-}{-}{-}{-}{-}{-}+}
             |        Voltage Divider      |
             |     +{-{-}{-}{-}{-}{-}{-}+{-}{-}{-}{-}{-}{-}{-}+       |}
             |     |       |       |       |
             |     R       R       R       |
             |     |       |       |       |
             |     +       +       +       |
             |     |       |       |       |
Control (5){-{-}+{-}{-}{-}{-}{-}+       |       |       |}
             |             |       |       |
Threshold(6){-+{-}{-}{-}{-}{-}|{-}{-}{-}{-}{-}{-}+       |       |}
             |      |+             |       |
             |    Comp2            |       |
             |      |              |       |
             |      v              |       |
             |    +{-+{-}+            |       |}
Trigger (2){-{-}+{-}{-}{-}|   |        S   |       |}
             |    |+  |{-{-}{-}{-}{-}{-}{-}{-}{-}{-}Q+{-}{-}{-}{-}{-}{-}{-}+{-}{-}Output (3)}
             |    Comp1         FF |       |
             |      |          R   |       |
             |      |          \^{   |       |}
Reset (4){-{-}{-}{-}+{-}{-}{-}{-}{-}{-}|{-}{-}{-}{-}{-}{-}{-}{-}{-}{-}|{-}{-}{-}+       |}
             |      |          |   |       |
             |      |          |   |       |
             |      v          |   |       |
             |    Transistor   |   |       |
             |      |          |   |       |
Discharge(7){-+{-}{-}{-}{-}{-}{-}+{-}{-}{-}{-}{-}{-}{-}{-}{-}{-}+{-}{-}{-}+       |}
             |                             |
             |                             |
             |                             |
             +{-{-}{-}{-}{-}{-}{-}{-}{-}{-}{-}{-}{-}{-}{-}{-}{-}{-}{-}{-}{-}{-}{-}{-}{-}{-}{-}{-}{-}+}
                            |
                           GND (1)
\end{verbatim}

\textbf{મૂળભૂત ઓપરેશન:}

\begin{enumerate}
\tightlist
\item
  \textbf{વોલ્ટેજ ડિવાઇડર}: 2/3 VCC અને 1/3 VCC રેફરન્સ પોઇન્ટ્સ બનાવે છે
\item
  \textbf{કોમ્પેરેટર 1}: પિન 2, 1/3 VCC થી નીચે જાય ત્યારે ટ્રિગર થાય છે
\item
  \textbf{કોમ્પેરેટર 2}: પિન 6, 2/3 VCC થી ઉપર જાય ત્યારે રીસેટ થાય છે
\item
  \textbf{ફ્લિપ-ફ્લોપ}: કોમ્પેરેટર ઇનપુટ્સના આધારે આઉટપુટ સ્ટેટને નિયંત્રિત કરે છે
\item
  \textbf{ડિસ્ચાર્જ ટ્રાન્ઝિસ્ટર}: આઉટપુટ LOW હોય ત્યારે પિન 7ને ગ્રાઉન્ડ સાથે જોડે છે
\end{enumerate}

\begin{itemize}
\tightlist
\item
  \textbf{વર્સેટિલિટી}: મલ્ટિપલ મોડ્સમાં કોન્ફિગર કરી શકાય છે (અસ્ટેબલ, મોનોસ્ટેબલ,
  બાયસ્ટેબલ)
\item
  \textbf{ટાઇમિંગ પ્રિસિઝન}: બાહ્ય RC ઘટકો દ્વારા નક્કી થાય છે
\item
  \textbf{વિશાળ સપ્લાય રેન્જ}: 4.5V થી 16V સુધી કાર્ય કરે છે
\end{itemize}

\end{solutionbox}
\begin{mnemonicbox}
``કોમ્પેરેટર્સ કંટ્રોલ ફ્લિપ-ફ્લોપ ફોર ટાઇમિંગ''

\end{mnemonicbox}
\subsection*{પ્રશ્ન 5(ક) અથવા [7
ગુણ]}\label{uxaaauxab0uxab6uxaa8-5uxa95-uxa85uxaa5uxab5-7-uxa97uxaa3}

\textbf{વર્ગ A, ક્લાસ B, ક્લાસ C અને ક્લાસ AB પાવર એમ્પલીફાયરને તેમના Q પોઇન્ટ
સ્થાનના આધારે લોડ લાઇન પર, રેખાકૃતિ સાથે કેવી રીતે વર્ગીકૃત કરવામાં આવે છે તે
સમજાવો.}

\begin{solutionbox}

{\def\LTcaptype{none} % do not increment counter
\begin{longtable}[]{@{}llll@{}}
\toprule\noalign{}
એમ્પલિફાયર ક્લાસ & Q-પોઇન્ટ સ્થાન & કન્ડક્શન એંગલ & કાર્યક્ષમતા \\
\midrule\noalign{}
\endhead
\bottomrule\noalign{}
\endlastfoot
ક્લાસ A & લોડ લાઇનના કેન્દ્રમાં & 360^\circ & 25-30\% \\
ક્લાસ B & કટ-ઓફ પોઇન્ટ & 180^\circ & 78.5\% \\
ક્લાસ AB & કટ-ઓફથી થોડું ઉપર & 180^\circ-360^\circ & 50-78.5\% \\
ક્લાસ C & કટ-ઓફથી નીચે & \textless180^\circ & \textgreater80\% \\
\end{longtable}
}

\textbf{ડાયાગ્રામ લોડ લાઇન:}

\begin{verbatim}
     Ic
      \^{}
      |
      |               Load Line
      |              /
      |             /
      |            /
  IcQ1|       A   /
      |        *  /
      |         {/}
  IcQ2|     AB * /
      |         {/}
      |      B * /
  IcQ3|         {/}
      |         /
      |    C * /
      |       /
      |      /
      |     /
      |    /
      |   /
      |  /
      | /
      |/
      +{-{-}{-}{-}{-}{-}{-}{-}{-}{-}{-}{-}{-}{-}{-}{-}{-}{-}{-}{-}{-}{-}{-}{-}{-}{-}{-}{-}{-}{-}{-}{-} Vce}
          VceC   VceB   VceAB   VceA
\end{verbatim}

\textbf{ઇનપુટ/આઉટપુટ વેવફોર્મ્સ:}

\begin{verbatim}
   Input          Class A         Class B          Class AB         Class C
     \^{              \^{}               \^{}                \^{}                \^{}}
     |              |               |                |                |
     |  /{          |  /           |  /            |  /            |  /}
     | /  {         | /            | /             | /             | /  }
{-{-}{-}{-}{-}+{-}{-}{-}{-}{-}{-}       {-}+{-}{-}{-}{-}{-}{-}        {-}+{-}{-}{-}{-}{-}{-}         {-}+{-}{-}{-}{-}{-}{-}         {-}+{-}{-}{-}{-}{-}{-}}
     |    {         |              |                |               |}
     |     {        |              |                |               |}
     |      {       |              |               |               |}
     v       v      v       v       v       v        v       v        v       v
                                    Distortion       Small Distortion
\end{verbatim}

\textbf{ક્લાસ A લક્ષણો:}

\begin{itemize}
\tightlist
\item
  \textbf{Q-પોઇન્ટ}: લોડ લાઇનના કેન્દ્રમાં
\item
  \textbf{બાયસ}: સમગ્ર સાયકલ માટે કન્ડક્શન જાળવવા માટે ફિક્સ્ડ બાયસ
\item
  \textbf{લિનિયરિટી}: ઉત્કૃષ્ટ લિનિયરિટી, ન્યૂનતમ ડિસ્ટોર્શન
\item
  \textbf{કાર્યક્ષમતા}: નબળી (25-30\%)
\end{itemize}

\textbf{ક્લાસ B લક્ષણો:}

\begin{itemize}
\tightlist
\item
  \textbf{Q-પોઇન્ટ}: કટઓફ પોઇન્ટ પર
\item
  \textbf{બાયસ}: કટઓફ પર બાયસ, દરેક ડિવાઇસ અર્ધા-સાયકલ માટે કન્ડક્ટ કરે છે
\item
  \textbf{ડિસ્ટોર્શન}: ઝીરો-ક્રોસિંગ પર ક્રોસઓવર ડિસ્ટોર્શન
\item
  \textbf{કાર્યક્ષમતા}: સારી (78.5\% સૈદ્ધાંતિક)
\end{itemize}

\textbf{ક્લાસ AB લક્ષણો:}

\begin{itemize}
\tightlist
\item
  \textbf{Q-પોઇન્ટ}: કટઓફથી થોડું ઉપર
\item
  \textbf{બાયસ}: ક્રોસઓવર ડિસ્ટોર્શન દૂર કરવા માટે નાનો બાયસ કરંટ
\item
  \textbf{લિનિયરિટી}: A અને B વચ્ચે સારો સમાધાન
\item
  \textbf{કાર્યક્ષમતા}: મધ્યમ (50-78.5\%)
\end{itemize}

\textbf{ક્લાસ C લક્ષણો:}

\begin{itemize}
\tightlist
\item
  \textbf{Q-પોઇન્ટ}: કટઓફથી નીચે
\item
  \textbf{બાયસ}: અર્ધા-સાયકલથી ઓછા માટે કન્ડક્ટ કરે છે
\item
  \textbf{ડિસ્ટોર્શન}: ગંભીર ડિસ્ટોર્શન, ટ્યુન્ડ સર્કિટની જરૂર
\item
  \textbf{કાર્યક્ષમતા}: ઉત્કૃષ્ટ (\textgreater80\%)
\end{itemize}

\end{solutionbox}
\begin{mnemonicbox}
``કેન્દ્રથી ઉપર, કેન્દ્રથી નીચે, કટ-ઓફ પોઇન્ટ, નીચે બિલકુલ -
ABCD ક્રમ Q-પોઇન્ટ સ્થાન માટે''

\end{mnemonicbox}

\end{document}
