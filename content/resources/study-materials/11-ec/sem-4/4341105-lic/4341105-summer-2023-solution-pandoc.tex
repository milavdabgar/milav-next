\documentclass[10pt,a4paper]{article}

% content/resources/templates/preamble.tex
\usepackage[margin=0.6in]{geometry}
\author{Milav Dabgar}
\usepackage{amsmath,amssymb,amsthm}
\usepackage{booktabs}
\usepackage{multirow}
\usepackage{xcolor}
\usepackage{tcolorbox}
\tcbuselibrary{breakable,skins}
\usepackage[colorlinks=true,linkcolor=blue]{hyperref}
\usepackage{titlesec}
\usepackage{enumitem}
\usepackage{tikz}
\usepackage{pgfplots}
\usepackage{circuitikz}
\usepackage[version=4]{mhchem}
\usepackage{longtable}
\usepackage{array}
\usepackage{float}
\usepackage{caption}
\usepackage{listings}

\lstset{
  basicstyle=\small\ttfamily,
  breaklines=true,
  breakatwhitespace=false,
  postbreak=\mbox{\textcolor{red}{$\hookrightarrow$}\space},
  float=false,
  numbers=left,
  numberstyle=\tiny\color{gray},
  numbersep=10pt,
  xleftmargin=2em,
  keywordstyle=\color{blue},
  commentstyle=\color{green!60!black},
  stringstyle=\color{purple},
  backgroundcolor=\color{gray!5},
  showstringspaces=false,
  tabsize=2,
  captionpos=b,
  keepspaces=true,
  columns=flexible
}

\pgfplotsset{compat=1.18}
\usetikzlibrary{shapes,arrows,positioning,calc,patterns,decorations.pathmorphing,decorations.markings,arrows.meta}

% Color scheme
\definecolor{headcolor}{RGB}{0,102,204}
\definecolor{keycolor}{RGB}{220,20,60}
\definecolor{solutioncolor}{RGB}{34,139,34}
\definecolor{mnemoniccolor}{RGB}{148,0,211}
\definecolor{codecolor}{RGB}{0,0,100}

% Spacing
\setlength{\parskip}{3pt}
\setlist[itemize]{nosep}
\setlist[enumerate]{nosep}

% Title formatting
\titleformat{\section}{\Large\bfseries\color{headcolor}}{\thesection}{1em}{}
\titleformat{\subsection}{\large\bfseries\color{headcolor}}{\thesubsection}{1em}{}

% Pandoc tightlist compatibility
\providecommand{\tightlist}{%
  \setlength{\itemsep}{0pt}\setlength{\parskip}{0pt}}

% Pandoc longtable compatibility
\newcounter{none}
\def\thenone{}


% content/resources/templates/english-boxes.tex
% This file is currently empty - it exists to maintain consistency with the import structure.
% Add custom environments here if needed in the future.


\begin{document}

\begin{center}
{\Huge\bfseries\color{headcolor} Subject Name Solutions}\\[5pt]
{\LARGE 4341105 -- Summer 2023}\\[3pt]
{\large Semester 1 Study Material}\\[3pt]
{\normalsize\textit{Detailed Solutions and Explanations}}
\end{center}

\vspace{10pt}

\subsection*{Question 1(a) [3 marks]}\label{q1a}

\textbf{Write advantages and disadvantages of negative feedback
amplifier}

\begin{solutionbox}

{\def\LTcaptype{none} % do not increment counter
\begin{longtable}[]{@{}
  >{\raggedright\arraybackslash}p{(\linewidth - 2\tabcolsep) * \real{0.4444}}
  >{\raggedright\arraybackslash}p{(\linewidth - 2\tabcolsep) * \real{0.5556}}@{}}
\toprule\noalign{}
\begin{minipage}[b]{\linewidth}\raggedright
Advantages
\end{minipage} & \begin{minipage}[b]{\linewidth}\raggedright
Disadvantages
\end{minipage} \\
\midrule\noalign{}
\endhead
\bottomrule\noalign{}
\endlastfoot
Increases bandwidth & Reduces gain \\
Stabilizes gain & Requires more components \\
Reduces distortion & Increases cost \\
Increases input impedance (voltage series) & May cause oscillations if
improperly designed \\
Decreases output impedance (voltage series) & Requires careful phase
compensation \\
\end{longtable}
}

\end{solutionbox}
\begin{mnemonicbox}
``GRASS Grows Better Despite Dry Soil'' (Gain
Reduction, Amplifies Stability, Stops distortion, Better impedance)

\end{mnemonicbox}
\subsection*{Question 1(b) [4 marks]}\label{q1b}

\textbf{Derive the equation of overall gain with negative feedback in
amplifier and give application of negative feedback.}

\begin{solutionbox}

\textbf{Derivation of overall gain with negative feedback:}

\begin{verbatim}
flowchart LR
    I[Input] {-{-} S[SummingnPoint]}
    S {-{-} A[AmplifiernA]}
    A {-{-} O[Output]}
    O {-{-} F[FeedbacknNetwork β]}
    F {-{-} S}
\end{verbatim}

\begin{itemize}
\tightlist
\item
  For an amplifier with gain A and feedback factor β:

  \begin{itemize}
  \tightlist
  \item
    Input signal = Vin
  \item
    Feedback signal = βVout
  \item
    Actual input to amplifier = Vin - βVout
  \item
    Output = A(Vin - βVout)
  \item
    Therefore, Vout = A(Vin - βVout)
  \item
    Vout + AβVout = AVin
  \item
    Vout(1 + Aβ) = AVin
  \item
    \textbf{Overall gain = Vout/Vin = A/(1 + Aβ)}
  \end{itemize}
\end{itemize}

\textbf{Applications of negative feedback:}

\begin{itemize}
\tightlist
\item
  Operational amplifiers
\item
  Voltage regulators
\item
  Audio amplifiers
\item
  Instrumentation amplifiers
\end{itemize}

\end{solutionbox}
\begin{mnemonicbox}
``AVOI'' (Amplifiers, Voltage regulators, Oscillation
control, Instrumentation)

\end{mnemonicbox}
\subsection*{Question 1(c) [7 marks]}\label{q1c}

\textbf{Draw and Explain current shunt type negative feedback amplifier
and Derive the formula of input impedance and output impedance of it.}

\begin{solutionbox}

\textbf{Current Shunt Negative Feedback Amplifier:}

\begin{verbatim}
flowchart LR
    I[Input] {-{-} S[CurrentnSampling]}
    S {-{-} A[Amplifier]}
    A {-{-} O[Output]}
    O {-{-} F[FeedbacknNetwork]}
    F {-{-}|Feedback Current| S}
\end{verbatim}

In current shunt feedback, the output voltage is sampled and converted
to a current that is subtracted from the input current.

\textbf{Circuit Diagram:}

\begin{verbatim}
                     +Vcc
                       |
                       R
                       |
                       |
  Iin  o{-{-}{-}{-}+{-}{-}{-}{-}{-}{-}{-}{-}{-}{-}|{-}{-}{-}{-}{-}{-}{-}o Vout}
            |          |
            |          |
           Zin        Zo
            |          |
            |          |
      +{-{-}{-}{-}{-}+          |}
      |                |
   Feedback            |
   Network(β)          |
      |                |
      +{-{-}{-}{-}{-}{-}{-}{-}{-}{-}{-}{-}{-}{-}{-}{-}+}
            |
            |
           GND
\end{verbatim}

\textbf{Characteristics:}

\begin{itemize}
\tightlist
\item
  \textbf{Feedback type}: Current sampling at input, shunt mixing at
  input
\item
  \textbf{Samples}: Output voltage
\item
  \textbf{Feedback to}: Input current
\end{itemize}

\textbf{Derivation of Input Impedance:}

\begin{itemize}
\tightlist
\item
  Without feedback: Zin
\item
  With current shunt feedback: Zin' = Zin/(1 + Aβ)
\item
  \textbf{Therefore, input impedance decreases by factor (1 + Aβ)}
\end{itemize}

\textbf{Derivation of Output Impedance:}

\begin{itemize}
\tightlist
\item
  Without feedback: Zo
\item
  With current shunt feedback: Zo' = Zo/(1 + Aβ)
\item
  \textbf{Therefore, output impedance decreases by factor (1 + Aβ)}
\end{itemize}

\end{solutionbox}
\begin{mnemonicbox}
``DISCO'' (Decreased Impedances with Shunt Current
Operation)

\end{mnemonicbox}
\subsection*{Question 1(c) OR [7
marks]}\label{q1c}

\textbf{Draw and Explain voltage series type negative feedback amplifier
and Derive the formula of input impedance and output impedance of it.}

\begin{solutionbox}

\textbf{Voltage Series Negative Feedback Amplifier:}

\begin{verbatim}
flowchart LR
    I[Input] {-{-} S[VoltagenSampling]}
    S {-{-} A[Amplifier]}
    A {-{-} O[Output]}
    O {-{-} F[FeedbacknNetwork β]}
    F {-{-}|Feedback Voltage| S}
\end{verbatim}

In voltage series feedback, the output voltage is sampled and fed back
in series with the input voltage.

\textbf{Circuit Diagram:}

\begin{verbatim}
                     +Vcc
                       |
                       R
                       |
                       |
  Vin  o{-{-}+{-}{-}{-}{-}{-}{-}{-}+{-}{-}{-}{-}+{-}{-}{-}{-}{-}{-}{-}o Vout}
          |       |    |
          Z       |    Z
          i       |    o
          n       |    |
          |       |    |
          +{-{-}{-}+   |    |}
              |   |    |
              |   |    |
           Feedback    |
           Network(β)  |
              |        |
              +{-{-}{-}{-}{-}{-}{-}{-}+}
              |
             GND
\end{verbatim}

\textbf{Characteristics:}

\begin{itemize}
\tightlist
\item
  \textbf{Feedback type}: Voltage sampling at output, series mixing at
  input
\item
  \textbf{Samples}: Output voltage
\item
  \textbf{Feedback to}: Input voltage
\end{itemize}

\textbf{Derivation of Input Impedance:}

\begin{itemize}
\tightlist
\item
  Without feedback: Zin
\item
  With voltage series feedback: Zin' = Zin \times (1 + Aβ)
\item
  \textbf{Therefore, input impedance increases by factor (1 + Aβ)}
\end{itemize}

\textbf{Derivation of Output Impedance:}

\begin{itemize}
\tightlist
\item
  Without feedback: Zo
\item
  With voltage series feedback: Zo' = Zo/(1 + Aβ)
\item
  \textbf{Therefore, output impedance decreases by factor (1 + Aβ)}
\end{itemize}

\end{solutionbox}
\begin{mnemonicbox}
``ISDO'' (Increased input impedance, Series feedback,
Decreased output impedance, Output voltage sampled)

\end{mnemonicbox}
\subsection*{Question 2(a) [3 marks]}\label{q2a}

\textbf{Draw and Explain the circuit diagram of UJT as a relaxation
oscillator.}

\begin{solutionbox}

\textbf{UJT Relaxation Oscillator:}

\begin{verbatim}
flowchart TB
    A[UJT Relaxation Oscillator]
    A {-{-}{-} B[Produces pulses withnRC charging circuit]}
    A {-{-}{-} C[UJT triggers whenncapacitor voltage reaches peak]}
    A {-{-}{-} D[Simple timing circuit]}
\end{verbatim}

\textbf{Circuit Diagram:}

\begin{verbatim}
    +Vcc
     |
     R1
     |
     +{-{-}{-}{-}{-}{-}+}
     |      |
     |      |
     |     B2
     |      |   UJT
     |      |
  C1 {-{-}{-} E  |}
     |      |
     |     B1
     |      |
     |      |
     +{-{-}{-}{-}{-}{-}+}
     |
    GND
\end{verbatim}

In this circuit:

\begin{itemize}
\tightlist
\item
  C1 charges through R1
\item
  When capacitor voltage reaches UJT's peak point, UJT turns on
\item
  Capacitor discharges rapidly through UJT
\item
  Process repeats creating oscillations
\end{itemize}

\end{solutionbox}
\begin{mnemonicbox}
``CURD'' (Capacitor charges Until Reaching Discharge
point)

\end{mnemonicbox}
\subsection*{Question 2(b) [4 marks]}\label{q2b}

\textbf{Draw circuit diagram of Colpitts oscillator and explain in
brief. Give the advantages and disadvantages of it.}

\begin{solutionbox}

\textbf{Colpitts Oscillator:}

\textbf{Circuit Diagram:}

\begin{verbatim}
                  +Vcc
                    |
                    RL
                    |
                    |
    +{-{-}{-}{-}{-}{-}{-}{-}{-}{-}{-}{-}{-}{-}{-}+{-}{-}{-}{-}{-}{-}{-}{-}{-}{-}{-}{-}{-}{-}{-}+}
    |               |               |
    |              C3               |
    |               |               |
    |              C1               |
    |              C2               |
    |               +{-{-}{-}+           |}
    |               |   |           |
    +{-{-}{-}+           +{-}{-}{-}+           |}
    |   |{-{-}{-}{-}+      |   |           |}
    +{-{-}{-}+    |      +{-}{-}{-}+           |}
              |                     |
              +{-{-}{-}{-}{-}{-}{-}{-}{-}{-}{-}{-}{-}{-}{-}{-}{-}{-}{-}{-}{-}+}
              |
             GND
\end{verbatim}

\textbf{Working:}

\begin{itemize}
\tightlist
\item
  Uses LC tank circuit with capacitive voltage divider (C1 and C2)
\item
  Transistor amplifies and provides energy to tank circuit
\item
  Oscillation frequency: f = 1/[2π\sqrt(L\times(C1\timesC2)/(C1+C2))]
\end{itemize}

{\def\LTcaptype{none} % do not increment counter
\begin{longtable}[]{@{}
  >{\raggedright\arraybackslash}p{(\linewidth - 2\tabcolsep) * \real{0.4444}}
  >{\raggedright\arraybackslash}p{(\linewidth - 2\tabcolsep) * \real{0.5556}}@{}}
\toprule\noalign{}
\begin{minipage}[b]{\linewidth}\raggedright
Advantages
\end{minipage} & \begin{minipage}[b]{\linewidth}\raggedright
Disadvantages
\end{minipage} \\
\midrule\noalign{}
\endhead
\bottomrule\noalign{}
\endlastfoot
Good frequency stability & Requires two capacitors (C1, C2) \\
Works well at high frequencies & More difficult to tune than some
oscillators \\
Lower harmonics & Sensitive to transistor parameters \\
Simple design & Limited frequency range \\
\end{longtable}
}

\end{solutionbox}
\begin{mnemonicbox}
``FAST Circuits'' (Frequency stable, Appropriate for
high frequencies, Simple design, Two capacitors needed)

\end{mnemonicbox}
\subsection*{Question 2(c) [7 marks]}\label{q2c}

\textbf{Explain the Crystal Oscillator.}

\begin{solutionbox}

\textbf{Crystal Oscillator:}

\begin{verbatim}
flowchart TD
    A[Crystal Oscillator] {-{-} B[Uses piezoelectric crystal]}
    A {-{-} C[Extremely stable frequency]}
    A {-{-} D[High Q factor]}
    A {-{-} E[Precise timing applications]}
\end{verbatim}

\textbf{Circuit Diagram:}

\begin{verbatim}
                +Vcc
                  |
                  RL
                  |
    +{-{-}{-}{-}{-}{-}{-}{-}{-}{-}{-}{-}{-}+{-}{-}{-}{-}{-}{-}{-}{-}{-}{-}{-}{-}{-}+}
    |             |             |
    |            C3             |
    |             |             |
    |      +{-{-}{-}{-}{-}{-}+{-}{-}{-}{-}{-}{-}+      |}
    |      |      |      |      |
    |     C1    XTAL     C2     |
    |      |      |      |      |
    |      +{-{-}{-}{-}{-}{-}+{-}{-}{-}{-}{-}{-}+      |}
    |                           |
    +{-{-}{-}{-}{-}{-}{-}{-}{-}{-}{-}{-}{-}{-}{-}{-}{-}{-}{-}{-}{-}{-}{-}{-}{-}{-}{-}+}
                |
               GND
\end{verbatim}

\textbf{Working Principle:}

\begin{itemize}
\tightlist
\item
  Based on piezoelectric effect of quartz crystal
\item
  Crystal vibrates at natural resonant frequency when voltage applied
\item
  Acts as very stable resonator with extremely high Q factor
\item
  Provides feedback at precise frequency
\end{itemize}

\textbf{Characteristics:}

\begin{itemize}
\tightlist
\item
  \textbf{Resonant frequency}: Determined by crystal cut and dimensions
\item
  \textbf{Q factor}: Typically 10,000-100,000 (much higher than LC
  circuits)
\item
  \textbf{Frequency stability}: Typically 0.001\% to 0.01\%
\item
  \textbf{Temperature coefficient}: Usually low, can be specially cut
  for zero temp coefficient
\end{itemize}

\textbf{Applications:}

\begin{itemize}
\tightlist
\item
  Clock generation in computers
\item
  Frequency standards
\item
  Radio transmitters/receivers
\item
  Digital watches and clocks
\item
  Microcontroller timing
\end{itemize}

\end{solutionbox}
\begin{mnemonicbox}
``STOP Precisely'' (Stable, Temperature-resistant,
Oscillates, Piezoelectric, Precisely)

\end{mnemonicbox}
\subsection*{Question 2(a) OR [3
marks]}\label{q2a}

\textbf{Draw and explain the Hartley Oscillator.}

\begin{solutionbox}

\textbf{Hartley Oscillator:}

\begin{verbatim}
flowchart TB
    A[Hartley Oscillator] {-{-} B[Uses tapped inductor]}
    A {-{-} C[LC tank circuit]}
    A {-{-} D[RF applications]}
\end{verbatim}

\textbf{Circuit Diagram:}

\begin{verbatim}
                +Vcc
                  |
                  R
                  |
    +{-{-}{-}{-}{-}{-}{-}{-}{-}{-}{-}{-}{-}+{-}{-}{-}{-}{-}{-}{-}{-}{-}{-}{-}{-}{-}+}
    |             |             |
    |             C3            |
    |             |             |
    |      +{-{-}{-}{-}{-}{-}+{-}{-}{-}{-}{-}{-}+      |}
    |      |      |      |      |
    |     C1      L1     L2     |
    |      |      |      |      |
    |      +{-{-}{-}{-}{-}{-}+{-}{-}{-}{-}{-}{-}+      |}
    |                           |
    +{-{-}{-}{-}{-}{-}{-}{-}{-}{-}{-}{-}{-}{-}{-}{-}{-}{-}{-}{-}{-}{-}{-}{-}{-}{-}{-}+}
                |
               GND
\end{verbatim}

\textbf{Working:}

\begin{itemize}
\tightlist
\item
  Uses LC tank circuit with tapped inductor (L1 and L2)
\item
  Transistor amplifies and provides energy to tank circuit
\item
Oscillation frequency:

f = 1/[2π\sqrt(L\timesC)] where

L = L1 + L2

\item
  Feedback through inductive coupling
\end{itemize}

\end{solutionbox}
\begin{mnemonicbox}
``TIC'' (Tapped inductor Circuit)

\end{mnemonicbox}
\subsection*{Question 2(b) OR [4
marks]}\label{q2b}

\textbf{Draw and explain Wien Bridge oscillator.}

\begin{solutionbox}

\textbf{Wien Bridge Oscillator:}

\begin{verbatim}
flowchart TD
    A[Wien Bridge Oscillator] {-{-} B[Uses RC network]}
    A {-{-} C[Audio frequency range]}
    A {-{-} D[Low distortion]}
    A {-{-} E[Stable output]}
\end{verbatim}

\textbf{Circuit Diagram:}

\begin{verbatim}
          +{-{-}{-}{-}{-}{-}{-}{-}+{-}{-}{-}{-}{-}{-}{-}{-}+}
          |        |        |
          |        R1       |
          |        |        |
     C1   |        |        |    R3
    ||{-{-}{-}{-}+        +{-}{-}{-}///{-}{-}{-}{-}+}
    ||    |        |             |
          |        |             |
          |        |             |
     R2   |       Op{-Amp         |}
    /{/{-}{-}+        |             |}
          |        |             |
          |        |             |
    +{-{-}{-}{-}{-}+        +{-}{-}{-}{-}{-}{-}{-}{-}{-}{-}{-}{-}{-}+}
    |     |        |             |
    C2    |        |             |
    |     |        |       R4    |
    +{-{-}{-}{-}{-}+{-}{-}{-}{-}{-}{-}{-}{-}+{-}{-}{-}{-}{-}{-}///{-}+}
          |
         GND
\end{verbatim}

\textbf{Working:}

\begin{itemize}
\tightlist
\item
  Uses RC Wien bridge network as frequency-selective feedback
\item
  R1=R2 and C1=C2 for simplest design
\item
  Oscillation frequency: f = 1/(2πRC)
\item
  Gain must be \geq 3 for sustained oscillations
\item
  Used for audio frequency generation with low distortion
\end{itemize}

\end{solutionbox}
\begin{mnemonicbox}
``FEAR'' (Frequency selective, Equal RC components,
Audio range, Reduced distortion)

\end{mnemonicbox}
\subsection*{Question 2(c) OR [7
marks]}\label{q2c}

\textbf{Draw the Structure, symbol, equivalent circuit of UJT and
explain in brief.}

\begin{solutionbox}

\textbf{Unijunction Transistor (UJT):}

\textbf{Structure:}

\begin{verbatim}
              Base 2 (B2)
                  |
                  |
                  v
                +{-{-}{-}+}
                |   |
                |   |
                |   |
                |   |
                |   |
                |   |
                |   |
                |   |
   Emitter (E) {|   | Base 1 (B1)}
                +{-{-}{-}+}
\end{verbatim}

\textbf{Symbol:}

\begin{verbatim}
                  B2
                  |
                  |
                  |
                  o
                 /|
                / |
               /  |
              /   |
     E o{-{-}{-}{-}o     |}
              {   |}
               {  |}
                { |}
                 {|}
                  o
                  |
                  |
                  |
                  B1
\end{verbatim}

\textbf{Equivalent Circuit:}

\begin{verbatim}
                B2
                 |
                 |
                 R
                /{/}
                 |
                 |
      E o{-{-}{-}+{-}{-}{-}||{-}{-}{-}+ B1}
             |         |
             |         |
             R         |
            /{/       |}
             |         |
             +{-{-}{-}{-}{-}{-}{-}{-}{-}+}
\end{verbatim}

\textbf{Working Principle:}

\begin{itemize}
\tightlist
\item
  UJT is a three-terminal device with one emitter and two bases
\item
  N-type silicon bar with P-type emitter junction
\item
  Forms a voltage divider with internal resistances RB1 and RB2
\item
  Emitter current starts flowing when VE \textgreater{} η\timesVBB + VD
\item
  Where η is intrinsic standoff ratio = RB1/(RB1+RB2)
\end{itemize}

\textbf{Characteristics:}

\begin{itemize}
\tightlist
\item
  \textbf{Intrinsic standoff ratio (η)}: Typically 0.5 to 0.8
\item
  \textbf{Negative resistance region}: Current increases as voltage
  decreases
\item
  \textbf{Peak point}: Beginning of negative resistance region
\item
  \textbf{Valley point}: End of negative resistance region
\end{itemize}

\textbf{Applications:}

\begin{itemize}
\tightlist
\item
  Relaxation oscillators
\item
  Timing circuits
\item
  Trigger generators
\item
  SCR triggering circuits
\item
  Sawtooth generators
\end{itemize}

\end{solutionbox}
\begin{mnemonicbox}
``NEVER'' (Negative resistance, Emitter-triggered,
Valley and peak points, Easily timed, Relaxation oscillator)

\end{mnemonicbox}
\subsection*{Question 3(a) [3 marks]}\label{q3a}

\textbf{Differentiate between voltage and power amplifier.}

\begin{solutionbox}

{\def\LTcaptype{none} % do not increment counter
\begin{longtable}[]{@{}lll@{}}
\toprule\noalign{}
Parameter & Voltage Amplifier & Power Amplifier \\
\midrule\noalign{}
\endhead
\bottomrule\noalign{}
\endlastfoot
Purpose & Amplifies voltage & Delivers power to load \\
Output impedance & High & Low \\
Input impedance & High & Relatively low \\
Efficiency & Not important & Very important \\
Heat dissipation & Low & High (requires heat sink) \\
Position in circuit & Early stages & Final stage \\
\end{longtable}
}

\end{solutionbox}
\begin{mnemonicbox}
``PEHIP'' (Power for Efficiency and Heat, Impedance
matters, Position differs)

\end{mnemonicbox}
\subsection*{Question 3(b) [4 marks]}\label{q3b}

\textbf{Explain class-B push pull power amplifier in detail.}

\begin{solutionbox}

\textbf{Class-B Push-Pull Amplifier:}

\begin{verbatim}
flowchart TB
    A[Class{-B Push{-}Pull] {-}{-} B[Uses two transistors]}
    A {-{-} C[Each handles half cycle]}
    A {-{-} D[Higher efficiency 78\%]}
    A {-{-} E[Crossover distortion]}
\end{verbatim}

\textbf{Circuit Diagram:}

\begin{verbatim}
         +Vcc
           |
           |
        +{-{-}+{-}{-}+}
        |     |
        Q1    |
        |     |
Input o{-+     +{-}{-}{-}o Output}
        |     |
        Q2    |
        |     |
        +{-{-}+{-}{-}+}
           |
           |
          GND
\end{verbatim}

\textbf{Working:}

\begin{itemize}
\tightlist
\item
  Uses two complementary transistors
\item
  Q1 conducts during positive half-cycle
\item
  Q2 conducts during negative half-cycle
\item
  Each transistor conducts for 180^\circ of input cycle
\item
  Theoretical efficiency: 78.5\%
\end{itemize}

\end{solutionbox}
\begin{mnemonicbox}
``ECHO'' (Efficiency high, Crossover distortion,
Half-cycle operation, Output high power)

\end{mnemonicbox}
\subsection*{Question 3(c) [7 marks]}\label{q3c}

\textbf{Draw and Explain Complementary symmetry push-pull power
amplifier in detail also list the disadvantages of it.}

\begin{solutionbox}

\textbf{Complementary Symmetry Push-Pull Amplifier:}

\begin{verbatim}
flowchart TD
    A[Complementary Push{-Pull] {-}{-} B[Uses NPN and PNP]}
    A {-{-} C[Direct coupling possible]}
    A {-{-} D[No center{-}tapped transformer]}
    A {-{-} E[Class B or AB operation]}
\end{verbatim}

\textbf{Circuit Diagram:}

\begin{verbatim}
               +Vcc
                |
                |
                Q1 (NPN)
                |
   +{-{-}{-}{-}{-}+{-}{-}{-}{-}{-}{-}+{-}{-}{-}{-}{-}+}
   |     |            |
   |     |            |
   |     |            |
Input   Bias       Output
   |     |            |
   |     |            |
   +{-{-}{-}{-}{-}+{-}{-}{-}{-}{-}{-}+{-}{-}{-}{-}{-}+}
                |
                |
                Q2 (PNP)
                |
                |
               GND
\end{verbatim}

\textbf{Working:}

\begin{itemize}
\tightlist
\item
  Uses complementary pair (NPN and PNP transistors)
\item
  No need for center-tapped transformer
\item
  NPN handles positive half-cycle
\item
  PNP handles negative half-cycle
\item
  Biasing network reduces crossover distortion
\item
  Direct coupling to speaker possible
\end{itemize}

\textbf{Disadvantages:}

\begin{itemize}
\tightlist
\item
  Thermal runaway if not properly biased
\item
  Requires complementary matched transistors
\item
  Crossover distortion in Class-B operation
\item
  Needs both positive and negative power supplies
\item
  Difficulty finding exact complementary pairs
\end{itemize}

\end{solutionbox}
\begin{mnemonicbox}
``MATCH Precisely'' (Matched transistors, Avoids
transformers, Thermal issues, Crossover distortion, Heat dissipation
needed)

\end{mnemonicbox}
\subsection*{Question 3(a) OR [3
marks]}\label{q3a}

\textbf{Define the terms related to power amplifier. i)Efficiency
ii)Distortion iii)power dissipation capability}

\begin{solutionbox}

{\def\LTcaptype{none} % do not increment counter
\begin{longtable}[]{@{}
  >{\raggedright\arraybackslash}p{(\linewidth - 2\tabcolsep) * \real{0.3333}}
  >{\raggedright\arraybackslash}p{(\linewidth - 2\tabcolsep) * \real{0.6667}}@{}}
\toprule\noalign{}
\begin{minipage}[b]{\linewidth}\raggedright
Term
\end{minipage} & \begin{minipage}[b]{\linewidth}\raggedright
Definition
\end{minipage} \\
\midrule\noalign{}
\endhead
\bottomrule\noalign{}
\endlastfoot
\textbf{Efficiency} & Ratio of AC output power delivered to the load to
the DC input power drawn from the supply. Mathematically: η = (Pout/Pin)
\times 100\%. Higher efficiency means less power wasted as heat. \\
\textbf{Distortion} & Unwanted alteration of the output waveform
compared to input waveform. Measured as Total Harmonic Distortion (THD).
Includes harmonic, intermodulation, crossover, and amplitude
distortion. \\
\textbf{Power Dissipation Capability} & Maximum power that can be
dissipated by the amplifier without damage. Depends on heat sink,
thermal resistance, and maximum junction temperature of transistors. \\
\end{longtable}
}

\end{solutionbox}
\begin{mnemonicbox}
``EDP'' (Efficiency converts, Distortion deforms,
Power capability protects)

\end{mnemonicbox}
\subsection*{Question 3(b) OR [4
marks]}\label{q3b}

\textbf{Classify the power amplifier for mode of operation and explain
working of different type power amplifier}

\begin{solutionbox}

\textbf{Classification of Power Amplifiers:}

\begin{verbatim}
flowchart TB
    A[Power Amplifiers] {-{-} B[Class A]}
    A {-{-} C[Class B]}
    A {-{-} D[Class AB]}
    A {-{-} E[Class C]}
\end{verbatim}

{\def\LTcaptype{none} % do not increment counter
\begin{longtable}[]{@{}
  >{\raggedright\arraybackslash}p{(\linewidth - 4\tabcolsep) * \real{0.2059}}
  >{\raggedright\arraybackslash}p{(\linewidth - 4\tabcolsep) * \real{0.5294}}
  >{\raggedright\arraybackslash}p{(\linewidth - 4\tabcolsep) * \real{0.2647}}@{}}
\toprule\noalign{}
\begin{minipage}[b]{\linewidth}\raggedright
Class
\end{minipage} & \begin{minipage}[b]{\linewidth}\raggedright
Conduction Angle
\end{minipage} & \begin{minipage}[b]{\linewidth}\raggedright
Working
\end{minipage} \\
\midrule\noalign{}
\endhead
\bottomrule\noalign{}
\endlastfoot
\textbf{Class A} & 360^\circ & Amplifier conducts for entire input cycle.
Output signal is exact replica of input but amplified. Linear but
inefficient (25-30\%). \\
\textbf{Class B} & 180^\circ & Two transistors each conduct for half cycle.
One handles positive half, other handles negative half. More efficient
(70-80\%) but has crossover distortion. \\
\textbf{Class AB} & 180^\circ-360^\circ & Compromise between Class A and B. Slight
bias to reduce crossover distortion. Good efficiency (50-70\%) with
acceptable distortion. \\
\textbf{Class C} & \textless180^\circ & Conducts for less than half cycle.
Very efficient (\textgreater80\%) but highly distorted. Used mainly in
RF tuned amplifiers. \\
\end{longtable}
}

\end{solutionbox}
\begin{mnemonicbox}
``ABCE'' (A-all cycle, B-both halves separately,
C-compromise solution, E-efficiency with distortion)

\end{mnemonicbox}
\subsection*{Question 3(c) OR [7
marks]}\label{q3c}

\textbf{Derive efficiency of class-B push pull power amplifier.}

\begin{solutionbox}

\textbf{Derivation of Class-B Push-Pull Amplifier Efficiency:}

\begin{verbatim}
flowchart TB
    A[Class{-B Efficiency] {-}{-} B[Based on power ratio]}
    A {-{-} C[Each transistor conducts half cycle]}
    A {-{-} D[Theoretical max efficiency: 78.5\%]}
\end{verbatim}

\textbf{Circuit Diagram:}

\begin{verbatim}
         +Vcc
           |
           |
        +{-{-}+{-}{-}+}
        |     |
        Q1    |
        |     |
Input o{-+     +{-}{-}{-}o Output}
        |     |
        Q2    |
        |     |
        +{-{-}+{-}{-}+}
           |
           |
          GND
\end{verbatim}

\textbf{Efficiency Calculation:}

\begin{enumerate}
\tightlist
\item
  \textbf{DC power input calculation:}

  \begin{itemize}
  \tightlist
  \item
    Each transistor conducts for half cycle
  \item
    Average DC current: Idc = Imax/π
  \item
    DC power input: Pdc = Vcc \times Idc = Vcc \times Imax/π
  \end{itemize}
\item
  \textbf{AC power output calculation:}

  \begin{itemize}
  \tightlist
  \item
    RMS value of current: Irms = Imax/2
  \item
    AC power output: Pac = (Irms)^{2} \times RL = (Imax/2)^{2} \times RL
  \item
    For maximum power: Imax \times RL = Vcc
  \item
    Therefore: Pac = (Vcc)^{2}/(2π \times RL)
  \end{itemize}
\item
  \textbf{Efficiency calculation:}

  \begin{itemize}
  \tightlist
  \item
    η = (Pac/Pdc) \times 100\%
  \item
    η = [(Vcc)^{2}/(2π \times RL)] \div [Vcc \times Imax/π] \times 100\%
  \item
    η = [(Vcc)^{2}/(2π \times RL)] \div [Vcc \times Vcc/(π \times RL)] \times 100\%
  \item
    η = [(Vcc)^{2}/(2π \times RL)] \times [π \times RL/Vcc^{2}] \times 100\%
  \item
    η = π/4 \times 100\% \approx 78.5\%
  \end{itemize}
\end{enumerate}

\textbf{Maximum theoretical efficiency of Class-B push-pull amplifier is
78.5\%}

\end{solutionbox}
\begin{mnemonicbox}
``PIPE'' (Power ratio, Input DC vs output AC, Pi in
formula, Efficiency maximum 78.5\%)

\end{mnemonicbox}
\subsection*{Question 4(a) [3 marks]}\label{q4a}

\textbf{Draw pin diagram and Schematic symbol of IC 741 and explain it
in detail.}

\begin{solutionbox}

\textbf{IC 741 Op-Amp Pin Diagram and Symbol:}

\textbf{Pin Diagram:}

\begin{verbatim}
        +{-{-}{-}{-}{-}{-}{-}+}
  1 o{-{-}{-}|       |{-}{-}{-}o 8}
        |       |
  2 o{-{-}{-}|  741  |{-}{-}{-}o 7}
        |       |
  3 o{-{-}{-}|       |{-}{-}{-}o 6}
        |       |
  4 o{-{-}{-}|       |{-}{-}{-}o 5}
        +{-{-}{-}{-}{-}{-}{-}+}
\end{verbatim}

\textbf{Schematic Symbol:}

\begin{verbatim}
            
        |{ }
        | {}
   +{-{-}{-}o|  }
        |   {{-}{-}{-}o Output}
   {-o{-}{-}{-}| /}
        |/
            
\end{verbatim}

\textbf{Pin Description:}

\begin{enumerate}
\tightlist
\item
  Offset Null (NC1)
\item
  Inverting Input (-)
\item
  Non-inverting Input (+)
\item
  Negative Supply (-Vcc)
\item
  Offset Null (NC2)\\
\item
  Output
\item
  Positive Supply (+Vcc)
\item
  NC (No Connection)
\end{enumerate}

\end{solutionbox}
\begin{mnemonicbox}
``ON-INO'' (Offset Null, Inverting input, Negative
supply, Input non-inverting, Output, No connection)

\end{mnemonicbox}
\subsection*{Question 4(b) [4 marks]}\label{q4b}

\textbf{Explain differential Amplifier using OPAMP.}

\begin{solutionbox}

\textbf{Differential Amplifier Using Op-Amp:}

\begin{verbatim}
flowchart TD
    A[Differential Amplifier] {-{-} B[Amplifies difference]}
    A {-{-} C[Rejects common mode]}
    A {-{-} D[Four equal resistors]}
    A {-{-} E[Gain = R2/R1]}
\end{verbatim}

\textbf{Circuit Diagram:}

\begin{verbatim}
             R2
   v1 o{-{-}{-}///{-}{-}{-}+}
                   |
                   |
      R1           |      R2
   +{-{-}///{-}{-}+{-}{-}{-}o|+     ///{-}{-}o Vout}
   |           |    |{-}
   |           |    |
   |           |    |
   +{-{-}///{-}{-}+{-}{-}{-}{-}+}
      R1           |
                   |
   v2 o{-{-}{-}///{-}{-}{-}+}
             R2
\end{verbatim}

\textbf{Working:}

\begin{itemize}
\tightlist
\item
  Output is proportional to difference between inputs
\item
  If R1 = R3 and R2 = R4, then: Vout = (R2/R1)(V2-V1)
\item
  Rejects signals common to both inputs (common-mode rejection)
\item
  Used in instrumentation applications
\end{itemize}

\end{solutionbox}
\begin{mnemonicbox}
``CARE'' (Common-mode rejection, Amplifies
difference, Resistor matching important, Equal resistors for balance)

\end{mnemonicbox}
\subsection*{Question 4(c) [7 marks]}\label{q4c}

\textbf{Explain the following parameters of an OP-Amp: 1)Input offset
voltage 2) Output Offset Voltage 3) Input Offset Current 4)Input Bias
Current 5) CMRR 6) Slew rate 7) Gain.}

\begin{solutionbox}

\textbf{Parameters of an Op-Amp:}

{\def\LTcaptype{none} % do not increment counter
\begin{longtable}[]{@{}
  >{\raggedright\arraybackslash}p{(\linewidth - 4\tabcolsep) * \real{0.2292}}
  >{\raggedright\arraybackslash}p{(\linewidth - 4\tabcolsep) * \real{0.2708}}
  >{\raggedright\arraybackslash}p{(\linewidth - 4\tabcolsep) * \real{0.5000}}@{}}
\toprule\noalign{}
\begin{minipage}[b]{\linewidth}\raggedright
Parameter
\end{minipage} & \begin{minipage}[b]{\linewidth}\raggedright
Description
\end{minipage} & \begin{minipage}[b]{\linewidth}\raggedright
Typical Value for 741
\end{minipage} \\
\midrule\noalign{}
\endhead
\bottomrule\noalign{}
\endlastfoot
\textbf{Input Offset Voltage} & Voltage needed at input to zero the
output & 1-5 mV \\
\textbf{Output Offset Voltage} & Output voltage when inputs are grounded
& Depends on input offset and gain \\
\textbf{Input Offset Current} & Difference between input bias currents &
3-30 nA \\
\textbf{Input Bias Current} & Average of the two input currents & 30-500
nA \\
\textbf{CMRR} & Ability to reject common-mode signals & 70-100 dB \\
\textbf{Slew Rate} & Maximum rate of output voltage change & 0.5 V/μs \\
\textbf{Gain (Aol)} & Open-loop voltage gain & 104-106 (80-120 dB) \\
\end{longtable}
}

\textbf{Diagram for Input Offset Voltage:}

\begin{verbatim}
                 Vos
                  |
                  v
      +{-{-}{-}{-}{-}+     |     +{-}{-}{-}{-}{-}+}
      |     |     |     |     |
   {-{-}{-}+  +  +{-}{-}{-}{-}{-}+{-}{-}{-}{-}{-}+     +{-}{-}{-}}
      |     |           |     |
      +{-{-}{-}{-}{-}+           +{-}{-}{-}{-}{-}+}
\end{verbatim}

\end{solutionbox}
\begin{mnemonicbox}
``VICS BGR'' (Voltage offset at Input, Current
offset, Slew rate, Bias current, Gain, Rejection ratio)

\end{mnemonicbox}
\subsection*{Question 4(a) OR [3
marks]}\label{q4a}

\textbf{List characteristics of ideal op-amp.}

\begin{solutionbox}

{\def\LTcaptype{none} % do not increment counter
\begin{longtable}[]{@{}ll@{}}
\toprule\noalign{}
Characteristic & Ideal Value \\
\midrule\noalign{}
\endhead
\bottomrule\noalign{}
\endlastfoot
\textbf{Open-loop gain} & Infinite \\
\textbf{Input impedance} & Infinite \\
\textbf{Output impedance} & Zero \\
\textbf{Bandwidth} & Infinite \\
\textbf{CMRR} & Infinite \\
\textbf{Slew rate} & Infinite \\
\textbf{Offset voltage} & Zero \\
\textbf{Noise} & Zero \\
\end{longtable}
}

\end{solutionbox}
\begin{mnemonicbox}
``ZINC BOSS'' (Zero offset, Infinite bandwidth, No
noise, CMRR infinite, Bandwidth unlimited, Output impedance zero, Slew
rate unlimited, Speed unlimited)

\end{mnemonicbox}
\subsection*{Question 4(b) OR [4
marks]}\label{q4b}

\textbf{Draw and explain the block diagram of the Operational Amplifier
(OPAMP) in detail.}

\begin{solutionbox}

\textbf{Op-Amp Block Diagram:}

\begin{verbatim}
flowchart LR
    A[Input Stage] {-{-} B[Intermediate Stage]}
    B {-{-} C[Output Stage]}
    D[Biasing Circuit] {-{-} A}
    D {-{-} B}
    D {-{-} C}
    E[Compensation Network] {-{-} B}
\end{verbatim}

\textbf{Detailed Block Diagram:}

\begin{verbatim}
                                Power Supply
                                    |
                                    v
 +{-{-}{-}{-}{-}{-}+     +{-}{-}{-}{-}{-}{-}{-}{-}{-}{-}{-}{-}+     +{-}{-}{-}{-}{-}{-}{-}{-}{-}{-}+     +{-}{-}{-}{-}{-}{-}+}
 |      |     |            |     |          |     |      |
 | Input|{-{-}{-}{-}|Differential|{-}{-}{-}{-}|  Voltage |{-}{-}{-}{-}|Output|{-}{-}{-} Output}
 | Pins |     |   Stage    |     |   Gain   |     | Stage|
 |      |     |            |     |   Stage  |     |      |
 +{-{-}{-}{-}{-}{-}+     +{-}{-}{-}{-}{-}{-}{-}{-}{-}{-}{-}{-}+     +{-}{-}{-}{-}{-}{-}{-}{-}{-}{-}+     +{-}{-}{-}{-}{-}{-}+}
                  \^{                 \^{}               \^{}}
                  |                 |               |
              +{-{-}{-}+{-}{-}{-}{-}+            |               |}
              |        |            |               |
              | Biasing|{-{-}{-}{-}{-}{-}{-}{-}{-}{-}{-}{-}+{-}{-}{-}{-}{-}{-}{-}{-}{-}{-}{-}{-}{-}{-}{-}+}
              | Circuit|
              |        |
              +{-{-}{-}{-}{-}{-}{-}{-}+}
                  \^{}
                  |
            Power Supply
\end{verbatim}

\textbf{Working of Blocks:}

\begin{enumerate}
\tightlist
\item
  \textbf{Input Stage}: Differential amplifier with high input impedance
\item
  \textbf{Intermediate Stage}: High-gain voltage amplifier with
  frequency compensation
\item
  \textbf{Output Stage}: Low output impedance buffer, provides current
  gain
\item
  \textbf{Biasing Circuit}: Provides proper DC levels to all stages
\item
  \textbf{Compensation Network}: Prevents oscillation, ensures stability
\end{enumerate}

\end{solutionbox}
\begin{mnemonicbox}
``DISCO'' (Differential stage Input, Second stage
amplifies, Compensation network, Output buffer)

\end{mnemonicbox}
\subsection*{Question 4(c) OR [7
marks]}\label{q4c}

\textbf{Draw \& explain Inverting and Non-inverting Op-amp amplifier
with the derivation of voltage gain.}

\begin{solutionbox}

\textbf{Inverting Amplifier:}

\begin{verbatim}
flowchart TB
    A[Inverting Amplifier] {-{-} B[Output 180^ out of phase]}
    A {-{-} C[Gain = {-}Rf/Rin]}
    A {-{-} D[Virtual ground at inverting input]}
\end{verbatim}

\textbf{Circuit Diagram:}

\begin{verbatim}
                 Rf
           +{-{-}{-}{-}///{-}{-}{-}{-}+}
           |              |
           |              |
     Rin   |    +{        |}
Vin o{-{-}///{-}{-}{-}|{-}       |}
           |    |  {      |}
           |    |   {{-}{-}{-}{-}{-}o Vout}
           |    |  /
           |    |{-/}
           |    +/
           |    |
           |    |
           +{-{-}{-}{-}+}
                |
               GND
\end{verbatim}

\textbf{Gain Derivation:}

\begin{itemize}
\tightlist
\item
  Using virtual ground concept (V- \approx 0)
\item
  Current through Rin: Iin = Vin/Rin
\item
  Current through Rf: If = Iin (no current into op-amp input)
\item
  Voltage across Rf: Vout = -If \times Rf = -Iin \times Rf = -Vin \times Rf/Rin
\item
  Therefore, Gain = Vout/Vin = -Rf/Rin
\end{itemize}

\textbf{Non-Inverting Amplifier:}

\begin{verbatim}
flowchart TB
    A[Non{-Inverting Amplifier] {-}{-} B[Output in phase with input]}
    A {-{-} C[Gain = 1 + Rf/Rin]}
    A {-{-} D[Higher input impedance than inverting]}
\end{verbatim}

\textbf{Circuit Diagram:}

\begin{verbatim}
                 Rf
           +{-{-}{-}{-}///{-}{-}{-}{-}+}
           |              |
           |              |
           |    +{        |}
Vin o{-{-}{-}{-}{-}{-}+{-}{-}{-}{-}|{-}       |}
           |    |  {      |}
           |    |   {{-}{-}{-}{-}{-}o Vout}
           |    |  /
           |    |{-/}
           |    +/
           |    |
           |    |
           +{-{-}///{-}{-}+}
              Rin     |
                      |
                     GND
\end{verbatim}

\textbf{Gain Derivation:}

\begin{itemize}
\tightlist
\item
  Due to negative feedback, V- \approx V+ = Vin
\item
  Voltage across Rin: V- = Vin
\item
  Current through Rin: IRin = V-/Rin = Vin/Rin
\item
  Same current flows through Rf: IRf = IRin
\item
  Voltage across Rf: VRf = IRf \times Rf = Vin \times Rf/Rin
\item
  Output voltage: Vout = V- + VRf = Vin + Vin \times Rf/Rin = Vin(1 + Rf/Rin)
\item
  Therefore, Gain = Vout/Vin = 1 + Rf/Rin
\end{itemize}

\textbf{Comparison:}

{\def\LTcaptype{none} % do not increment counter
\begin{longtable}[]{@{}lll@{}}
\toprule\noalign{}
Parameter & Inverting Amplifier & Non-Inverting Amplifier \\
\midrule\noalign{}
\endhead
\bottomrule\noalign{}
\endlastfoot
\textbf{Gain formula} & -Rf/Rin & 1 + Rf/Rin \\
\textbf{Phase shift} & 180^\circ & 0^\circ \\
\textbf{Input impedance} & Equal to Rin & Very high (\approx infinite) \\
\textbf{Min. possible gain} & Can be \textless1 & Always \geq1 \\
\end{longtable}
}

\end{solutionbox}
\begin{mnemonicbox}
``PING-PONG'' (Phase Inverted Negative Gain vs
Positive Output Non-inverted Gain)

\end{mnemonicbox}
\subsection*{Question 5(a) [3 marks]}\label{q5a}

\textbf{Draw and explain integrator using Op-Amp.}

\begin{solutionbox}

\textbf{Op-Amp Integrator:}

\begin{verbatim}
flowchart TD
    A[Integrator] {-{-} B[RC circuit in feedback]}
    A {-{-} C[Output is integral of input]}
    A {-{-} D[Acts as low{-}pass filter]}
\end{verbatim}

\textbf{Circuit Diagram:}

\begin{verbatim}
                  C
              +{-{-}{-}||{-}{-}{-}+}
              |        |
              |        |
   Vin o{-{-}///{-}{-}+{-}{-}{-}o|+      |}
            R     |    |{-      +{-}{-}{-}o Vout}
                  |    |       |
                  |    |
                  +{-{-}{-}{-}+}
                  |
                 GND
\end{verbatim}

\textbf{Working:}

\begin{itemize}
\tightlist
\item
  Output voltage is proportional to integral of input
\item
  Vout = -1/RC \intVin dt
\item
  Used in waveform generators, analog computers
\item
  Acts as low-pass filter with -20dB/decade slope
\end{itemize}

\end{solutionbox}
\begin{mnemonicbox}
``TIME'' (Takes Input and Makes integral over time
Exactly)

\end{mnemonicbox}
\subsection*{Question 5(b) [4 marks]}\label{q5b}

\textbf{Compare different types of power amplifier.}

\begin{solutionbox}

{\def\LTcaptype{none} % do not increment counter
\begin{longtable}[]{@{}
  >{\raggedright\arraybackslash}p{(\linewidth - 8\tabcolsep) * \real{0.2292}}
  >{\raggedright\arraybackslash}p{(\linewidth - 8\tabcolsep) * \real{0.1875}}
  >{\raggedright\arraybackslash}p{(\linewidth - 8\tabcolsep) * \real{0.1875}}
  >{\raggedright\arraybackslash}p{(\linewidth - 8\tabcolsep) * \real{0.2083}}
  >{\raggedright\arraybackslash}p{(\linewidth - 8\tabcolsep) * \real{0.1875}}@{}}
\toprule\noalign{}
\begin{minipage}[b]{\linewidth}\raggedright
Parameter
\end{minipage} & \begin{minipage}[b]{\linewidth}\raggedright
Class A
\end{minipage} & \begin{minipage}[b]{\linewidth}\raggedright
Class B
\end{minipage} & \begin{minipage}[b]{\linewidth}\raggedright
Class AB
\end{minipage} & \begin{minipage}[b]{\linewidth}\raggedright
Class C
\end{minipage} \\
\midrule\noalign{}
\endhead
\bottomrule\noalign{}
\endlastfoot
\textbf{Conduction angle} & 360^\circ & 180^\circ & 180^\circ-360^\circ & \textless180^\circ \\
\textbf{Efficiency} & 25-30\% & 70-80\% & 50-70\% & \textgreater80\% \\
\textbf{Distortion} & Very low & High (crossover) & Low & Very high \\
\textbf{Biasing} & Above cutoff & At cutoff & Slightly above cutoff &
Below cutoff \\
\textbf{Applications} & High fidelity audio & General purpose & Audio
amplifiers & RF amplifiers \\
\end{longtable}
}

\end{solutionbox}
\begin{mnemonicbox}
``CABINET'' (Conduction angle, Amplification quality,
Biasing, Ideal applications, Noise/distortion, Efficiency, Temperature
concerns)

\end{mnemonicbox}
\subsection*{Question 5(c) [7 marks]}\label{q5c}

\textbf{List applications of IC555 and explain any one in detail.}

\begin{solutionbox}

\textbf{Applications of IC 555:}

\begin{enumerate}
\tightlist
\item
  Astable multivibrator
\item
  Monostable multivibrator
\item
  Bistable multivibrator
\item
  Pulse width modulator
\item
  Sequential timer
\item
  Frequency divider
\item
  Tone generator
\end{enumerate}

\textbf{Astable Multivibrator Using IC 555:}

\begin{verbatim}
flowchart TB
    A[555 Astable] {-{-} B[Free{-}running oscillator]}
    A {-{-} C[No stable state]}
    A {-{-} D[Output continuously switches]}
    A {-{-} E[Frequency determined by R1, R2, C]}
\end{verbatim}

\textbf{Circuit Diagram:}

\begin{verbatim}
                  +Vcc
                    |
                    |
             +{-{-}{-}{-}{-}{-}+{-}{-}{-}{-}{-}{-}+}
             |      |      |
             |     R1      |
             |      |      |
             +{-{-}{-}{-}{-}{-}+{-}{-}{-}{-}{-}{-}+}
             |      |      |
             |     R2      |
             |      |      |
    +{-{-}{-}{-}{-}{-}{-}{-}+{-}{-}{-}{-}{-}{-}+{-}{-}{-}{-}{-}{-}+{-}{-}{-}{-}{-}{-}{-}{-}+}
    |        |      |      |        |
    |   +{-{-}{-}{-}+    8 |      |        |}
    |   |    |      |      |        |
    |   |    |    7 |      |        |
    |   |    |      |      |        |
    |   |    |    6 |      |        |
    |  C1    |      | 555  |        |
    |   |    |    5 +{-{-}{-}{-}{-}{-}+        |}
    |   |    |      |      |        |
    |   |    |    4 |      |        |
    |   |    |      |      |        |
    |   |    |    3 |      +{-{-}{-}o Output}
    |   |    |      |      |        |
    |   |    |    2 |      |        |
    |   |    |      |      |        |
    |   |    |    1 |      |        |
    |   |    +{-{-}{-}{-}{-}{-}+{-}{-}{-}{-}{-}{-}+        |}
    |   |           |               |
    +{-{-}{-}+{-}{-}{-}{-}{-}{-}{-}{-}{-}{-}{-}+{-}{-}{-}{-}{-}{-}{-}{-}{-}{-}{-}{-}{-}{-}{-}+}
        |           |
       GND         GND
\end{verbatim}

\textbf{Working:}

\begin{itemize}
\tightlist
\item
  R1, R2, and C determine frequency
\item
  Output oscillates between HIGH and LOW
\item
  Charging time: t1 = 0.693(R1+R2)C
\item
  Discharging time: t2 = 0.693(R2)C
\item
Total period:

T = t1 + t2 = 0.693(R1+2R2)C

\item
  Frequency: f = 1.44/[(R1+2R2)C]
\item
  Duty cycle: D = (R1+R2)/(R1+2R2)
\end{itemize}

\textbf{Applications:}

\begin{itemize}
\tightlist
\item
  LED flashers
\item
  Clock generators
\item
  Tone generators
\item
  Pulse generation
\end{itemize}

\end{solutionbox}
\begin{mnemonicbox}
``FREE'' (Frequency determined by Resistors and
capacitor, Endless oscillation, Easy to configure)

\end{mnemonicbox}
\subsection*{Question 5(a) OR [3
marks]}\label{q5a}

\textbf{Draw and explain summing amplifier using Op-Amp.}

\begin{solutionbox}

\textbf{Summing Amplifier Using Op-Amp:}

\begin{verbatim}
flowchart TB
    A[Summing Amplifier] {-{-} B[Adds multiple inputs]}
    A {-{-} C[Weighted sum possible]}
    A {-{-} D[Inverting configuration]}
\end{verbatim}

\textbf{Circuit Diagram:}

\begin{verbatim}
              Rf
        +{-{-}{-}///{-}{-}{-}+}
        |            |
        |            |
  R1    |    +{      |}
V1 o{-{-}///{-}{-}|{-}     |}
        |    |  {    |}
  R2    |    |   {{-}{-}{-}o Vout}
V2 o{-{-}///{-}{-}+   /}
        |    |  /
  R3    |    |{-/}
V3 o{-{-}///{-}{-}+/}
        |     |
        |     |
        +{-{-}{-}{-}{-}+}
              |
             GND
\end{verbatim}

\textbf{Working:}

\begin{itemize}
\tightlist
\item
  Uses inverting configuration with multiple inputs
\item
  Each input contributes to output based on its resistance
\item
  If R1 = R2 = R3 = R, and Rf = R, then Vout = -(V1 + V2 + V3)
\item
  If resistors differ, weighted sum is produced: Vout = -Rf(V1/R1 +
  V2/R2 + V3/R3)
\item
  Virtual ground at inverting input simplifies analysis
\end{itemize}

\end{solutionbox}
\begin{mnemonicbox}
``SWIM'' (Summing Weighted Inputs with Mixing)

\end{mnemonicbox}
\subsection*{Question 5(b) OR [4
marks]}\label{q5b}

\textbf{Compare between push-pull amplifier and Complementary push-pull
power amplifier.}

\begin{solutionbox}

{\def\LTcaptype{none} % do not increment counter
\begin{longtable}[]{@{}
  >{\raggedright\arraybackslash}p{(\linewidth - 4\tabcolsep) * \real{0.1642}}
  >{\raggedright\arraybackslash}p{(\linewidth - 4\tabcolsep) * \real{0.3134}}
  >{\raggedright\arraybackslash}p{(\linewidth - 4\tabcolsep) * \real{0.5224}}@{}}
\toprule\noalign{}
\begin{minipage}[b]{\linewidth}\raggedright
Parameter
\end{minipage} & \begin{minipage}[b]{\linewidth}\raggedright
Push-Pull Amplifier
\end{minipage} & \begin{minipage}[b]{\linewidth}\raggedright
Complementary Push-Pull Amplifier
\end{minipage} \\
\midrule\noalign{}
\endhead
\bottomrule\noalign{}
\endlastfoot
\textbf{Transistors used} & Same type (NPN or PNP) & Complementary pair
(NPN and PNP) \\
\textbf{Input transformer} & Required (center-tapped) & Not required \\
\textbf{Output transformer} & Required & Not required \\
\textbf{Circuit complexity} & More complex & Simpler \\
\textbf{Cost} & Higher due to transformers & Lower \\
\textbf{Frequency response} & Limited by transformers & Better (wider
range) \\
\textbf{Phase distortion} & Higher & Lower \\
\textbf{Power supply} & Single polarity & Dual polarity usually
required \\
\end{longtable}
}

\end{solutionbox}
\begin{mnemonicbox}
``TONIC'' (Transformers vs None, One type vs
complementary, Nice frequency response, Improved distortion, Cost
effectiveness)

\end{mnemonicbox}
\subsection*{Question 5(c) OR [7
marks]}\label{q5c}

\textbf{Draw pin diagram and block diagram of IC555 and explain in
detail.}

\begin{solutionbox}

\textbf{IC 555 Timer:}

\textbf{Pin Diagram:}

\begin{verbatim}
        +{-{-}{-}{-}{-}{-}{-}+}
  1 o{-{-}{-}|       |{-}{-}{-}o 8}
        |       |
  2 o{-{-}{-}|  555  |{-}{-}{-}o 7}
        |       |
  3 o{-{-}{-}|       |{-}{-}{-}o 6}
        |       |
  4 o{-{-}{-}|       |{-}{-}{-}o 5}
        +{-{-}{-}{-}{-}{-}{-}+}
\end{verbatim}

\textbf{Pin Description:}

\begin{enumerate}
\tightlist
\item
  Ground - Connected to circuit ground
\item
  Trigger - Starts the timing cycle when voltage falls below 1/3 Vcc
\item
  Output - Provides the output signal, can source or sink up to 200mA
\item
  Reset - Terminates timing cycle when pulled low
\item
  Control Voltage - Allows access to internal voltage divider (2/3 Vcc)
\item
  Threshold - Ends timing cycle when voltage exceeds 2/3 Vcc
\item
  Discharge - Connected to open collector of internal transistor
\item
  Vcc - Positive supply voltage (4.5V to 16V)
\end{enumerate}

\textbf{Block Diagram:}

\begin{verbatim}
    8                               
    o{-{-}{-}{-}{-}{-}+{-}{-}{-}{-}{-}{-}{-}{-}{-}{-}{-}{-}{-}{-}{-}{-}{-}{-}{-}{-}{-}+  }
    Vcc    |                     |  
           |    +{-{-}{-}{-}{-}{-}{-}{-}{-}{-}{-}+    |  }
    5      |    |           |    |  
    o{-{-}{-}{-}{-}{-}+{-}{-}{-}{-}| Voltage   |    |  }
    Control|    | Divider   |    |  
           |    |           |    |  
           |    +{-{-}{-}{-}{-}{-}{-}{-}{-}{-}{-}+    |  }
           |      |     |        |  
           |      |     |        |  
    2      |    +{-v{-}+ +{-}v{-}+      |  }
    o{-{-}{-}{-}{-}{-}+{-}{-}{-}|   | |   |      |  }
    Trigger     |Comp| |Comp|    |  
                |   | |   |      |  
    6           +{-+{-}+ +{-}+{-}+      |  }
    o{-{-}{-}{-}{-}{-}{-}{-}{-}{-}{-}{-}+|     |+{-}{-}{-}{-}{-}{-}{-}+  }
    Threshold     |     |           
                +{-v{-}{-}{-}{-}{-}v{-}+         }
                |         |         
                | Flip    |         
    4           | Flop    |         
    o{-{-}{-}{-}{-}{-}{-}{-}{-}{-}|         |         }
    Reset       +{-+{-}{-}{-}{-}{-}+{-}+         }
                  |     |           
                  |     |           
                +{-v{-}+ +{-}v{-}+         }
                |   | |   |         
                |Buf| |Out|         
                |   | |   |         
                +{-+{-}+ +{-}+{-}+         }
                  |     |           
    7             |     |  3        
    o{-{-}{-}{-}{-}{-}{-}{-}{-}{-}{-}{-}{-}+     +{-}{-}o        }
    Discharge               Output  
                                    
    1                               
    o{-{-}{-}{-}{-}{-}{-}{-}{-}{-}{-}{-}{-}{-}{-}{-}{-}{-}{-}{-}{-}{-}{-}{-}{-}{-}{-}{-}+  }
    GND                          |  
                                 |  
    +{-{-}{-}{-}{-}{-}{-}{-}{-}{-}{-}{-}{-}{-}{-}{-}{-}{-}{-}{-}{-}{-}{-}{-}{-}{-}{-}{-}+  }
\end{verbatim}

\textbf{Working:}

\begin{enumerate}
\tightlist
\item
  \textbf{Voltage Divider}: Creates reference voltages at 1/3 and 2/3 of
  Vcc
\item
  \textbf{Comparators}: Compare input voltages with reference voltages
\item
  \textbf{Flip-Flop}: Stores timing state based on comparator outputs
\item
  \textbf{Output Stage}: Buffers and amplifies flip-flop output
\item
  \textbf{Discharge Transistor}: Controlled by flip-flop to discharge
  timing capacitor
\end{enumerate}

\textbf{Operating Modes:}

\begin{enumerate}
\tightlist
\item
  \textbf{Monostable}: One-shot timer triggered by input pulse
\item
  \textbf{Astable}: Free-running oscillator for pulse generation
\item
  \textbf{Bistable}: Flip-flop with set and reset functionality
\end{enumerate}

\textbf{Applications:}

\begin{itemize}
\tightlist
\item
  Pulse generation
\item
  Time delays
\item
  Oscillators
\item
  PWM controllers
\item
  Sequential timers
\end{itemize}

\end{solutionbox}
\begin{mnemonicbox}
``VICTOR'' (Voltage divider, Internal comparators,
Control flip-flop, Timing capabilities, Output buffer, Reset function)

\end{mnemonicbox}

\end{document}
