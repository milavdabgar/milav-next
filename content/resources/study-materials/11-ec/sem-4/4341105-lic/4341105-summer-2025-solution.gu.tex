\documentclass{article}

% content/resources/templates/preamble.tex
\usepackage[margin=0.6in]{geometry}
\author{Milav Dabgar}
\usepackage{amsmath,amssymb,amsthm}
\usepackage{booktabs}
\usepackage{multirow}
\usepackage{xcolor}
\usepackage{tcolorbox}
\tcbuselibrary{breakable,skins}
\usepackage[colorlinks=true,linkcolor=blue]{hyperref}
\usepackage{titlesec}
\usepackage{enumitem}
\usepackage{tikz}
\usepackage{pgfplots}
\usepackage{circuitikz}
\usepackage[version=4]{mhchem}
\usepackage{longtable}
\usepackage{array}
\usepackage{float}
\usepackage{caption}
\usepackage{listings}

\lstset{
  basicstyle=\small\ttfamily,
  breaklines=true,
  breakatwhitespace=false,
  postbreak=\mbox{\textcolor{red}{$\hookrightarrow$}\space},
  float=false,
  numbers=left,
  numberstyle=\tiny\color{gray},
  numbersep=10pt,
  xleftmargin=2em,
  keywordstyle=\color{blue},
  commentstyle=\color{green!60!black},
  stringstyle=\color{purple},
  backgroundcolor=\color{gray!5},
  showstringspaces=false,
  tabsize=2,
  captionpos=b,
  keepspaces=true,
  columns=flexible
}

\pgfplotsset{compat=1.18}
\usetikzlibrary{shapes,arrows,positioning,calc,patterns,decorations.pathmorphing,decorations.markings,arrows.meta}

% Color scheme
\definecolor{headcolor}{RGB}{0,102,204}
\definecolor{keycolor}{RGB}{220,20,60}
\definecolor{solutioncolor}{RGB}{34,139,34}
\definecolor{mnemoniccolor}{RGB}{148,0,211}
\definecolor{codecolor}{RGB}{0,0,100}

% Spacing
\setlength{\parskip}{3pt}
\setlist[itemize]{nosep}
\setlist[enumerate]{nosep}

% Title formatting
\titleformat{\section}{\Large\bfseries\color{headcolor}}{\thesection}{1em}{}
\titleformat{\subsection}{\large\bfseries\color{headcolor}}{\thesubsection}{1em}{}

% Pandoc tightlist compatibility
\providecommand{\tightlist}{%
  \setlength{\itemsep}{0pt}\setlength{\parskip}{0pt}}

% Pandoc longtable compatibility
\newcounter{none}
\def\thenone{}


% content/resources/templates/gujarati-boxes.tex
\usepackage{fontspec}
\usepackage{polyglossia}

% Set Gujarati as main language (document is primarily in Gujarati)
% Note: gloss-gujarati.ldf doesn't exist in polyglossia, but it will use hyphenation patterns
\setdefaultlanguage{gujarati}
\setotherlanguage{english}

% Configure Gujarati font properly
% Use Language=Default to prevent polyglossia from trying to add language-specific features
% that don't exist for Gujarati, which causes "empty feature" warnings
\newfontfamily\gujaratifont[Script=Gujarati,AutoFakeBold=2.5,AutoFakeSlant=0.3]{Noto Sans Gujarati}
\setmainfont[Script=Gujarati,AutoFakeBold=2.5,AutoFakeSlant=0.3]{Noto Sans Gujarati}
% Use Noto Sans Gujarati for monospace to support Gujarati in text
\setmonofont[Scale=0.9]{Noto Sans Gujarati}

% Configure English to use the same font
\newfontfamily\englishfont[Script=Gujarati,AutoFakeBold=2.5,AutoFakeSlant=0.3]{Noto Sans Gujarati}

% Translations for polyglossia
\gappto\captionsgujarati{
  \renewcommand{\tablename}{કોષ્ટક}
  \renewcommand{\figurename}{આકૃતિ}
}

% Helper for TikZ nodes to ensure Gujarati font
\newcommand{\gu}[1]{{\gujaratifont #1}}

% Custom environments
\newtcolorbox{solutionbox}{
    breakable,
    enhanced,
    colback=solutioncolor!5!white,
    colframe=solutioncolor!75!black,
    fonttitle=\bfseries,
    title=જવાબ
}

\newtcolorbox{solutionboxnobreak}{
 colback=solutioncolor!5!white,
 colframe=solutioncolor!75!black,
 fonttitle=\bfseries,
 title=જવાબ
}

\newtcolorbox{keyformula}{
 breakable,
 enhanced,
 colback=keycolor!5!white,
 colframe=keycolor!75!black,
 fonttitle=\bfseries,
 title=રાસાયણિક સમીકરણ/સૂત્ર
}

\newtcolorbox{mnemonicbox}{
 breakable,
 enhanced,
 colback=mnemoniccolor!5!white,
 colframe=mnemoniccolor!75!black,
 fonttitle=\bfseries,
 title=મેમરી ટ્રીક
}


% Custom commands for GTU solutions
% This file defines semantic commands for consistent formatting

% Question command with automatic formatting
\newcommand{\question}[2]{%
  \section*{Question #1}%
  \textbf{#2}%
}

% OR question variant
\newcommand{\questionor}[2]{%
  \section*{Question #1 OR}%
  \textbf{#2}%
}

% Proper table environment with caption
\newenvironment{answertable}[1]{%
  \begin{table}[htbp]
  \centering
  \caption{#1}
}{%
  \end{table}
}

% Proper figure environment for diagrams
\newenvironment{answerdiagram}[1]{%
  \begin{figure}[htbp]
  \centering
  \caption{#1}
}{%
  \end{figure}
}

% Semantic markup for key terms
\newcommand{\keyword}[1]{\textbf{#1}}
\newcommand{\code}[1]{\texttt{#1}}
\newcommand{\classname}[1]{\texttt{#1}}
\newcommand{\methodname}[1]{\texttt{#1}}

% Proper quotation marks
\newcommand{\mnemonic}[1]{``#1''}


\title{Linear Integrated Circuit (4341105) - Summer 2025 Solution}
\date{May 17, 2025}

\begin{document}
\maketitle
\solutiontitle

% Question 1(a) [3 marks]
\questionmarks{1}{a}{3}
\textbf{ગેઈન અને સ્ટેબિલિટી પર નેગેટિવ ફીડબેકની અસર સમજાવો.}

\begin{solutionbox}
    \textbf{જવાબ}:
    નેગેટિવ ફીડબેક એમ્પ્લીફાયરની કામગીરીને નોંધપાત્ર રીતે સુધારે છે.
    
    \begin{tabulary}{\textwidth}{L|L}
        \toprule
        \textbf{પરિમાણ} & \textbf{નેગેટિવ ફીડબેકની અસર} \\
        \midrule
        \textbf{ગેઈન} & એકુલ ગેઈન ઘટાડે છે \\
        \textbf{સ્ટેબિલિટી} & સ્થિરતા વધારે છે \\
        \textbf{બેન્ડવિડ્થ} & બેન્ડવિડ્થ વધારે છે \\
        \bottomrule
    \end{tabulary}
    
    \begin{itemize}
        \item \textbf{ગેઈન ઘટાડો}: એમ્પ્લીફાયરને વધુ અનુમાનિત બનાવે છે
        \item \textbf{સ્થિરતા સુધારો}: ઓસિલેશન અને વિકૃતિ ઘટાડે છે
        \item \textbf{સારું નિયંત્રણ}: સતત કામગીરી પ્રદાન કરે છે
    \end{itemize}
    
    \begin{mnemonicbox}
        \mnemonic{Gain Goes Down, Stability Stays Strong}
    \end{mnemonicbox}
\end{solutionbox}

% Question 1(b) [4 marks]
\questionmarks{1}{b}{4}
\textbf{ફીડબેક એમ્પ્લીફાયરના જુદા જુદા પ્રકારો અને નેગેટિવ ફીડબેકના એમ્પ્લીફાયરના ફાયદા જણાવો.}

\begin{solutionbox}
    \textbf{જવાબ}:
    ઇનપુટ અને આઉટપુટ કનેક્શનના આધારે ચાર મૂળભૂત ફીડબેક પ્રકારો છે.
    
    \begin{tabulary}{\textwidth}{L|L|L}
        \toprule
        \textbf{પ્રકાર} & \textbf{ઇનપુટ કનેક્શન} & \textbf{આઉટપુટ કનેક્શન} \\
        \midrule
        \textbf{વોલ્ટેજ સીરીઝ} & સીરીઝ & વોલ્ટેજ \\
        \textbf{વોલ્ટેજ શન્ટ} & શન્ટ & વોલ્ટેજ \\
        \textbf{કરંટ સીરીઝ} & સીરીઝ & કરંટ \\
        \textbf{કરંટ શન્ટ} & શન્ટ & કરંટ \\
        \bottomrule
    \end{tabulary}
    
    \textbf{ફાયદા:}
    \begin{itemize}
        \item \textbf{વિકૃતિ ઘટાડો}: હાર્મોનિક કન્ટેન્ટ ઘટાડે છે
        \item \textbf{બેન્ડવિડ્થ વૃદ્ધિ}: સારી ફ્રીક્વન્સી રિસ્પોન્સ
        \item \textbf{સુધારેલી સ્થિરતા}: સતત ઓપરેશન
    \end{itemize}
    
    \begin{mnemonicbox}
        \mnemonic{Very Smart Current Control}
    \end{mnemonicbox}
\end{solutionbox}

% Question 1(c) [7 marks]
\questionmarks{1}{c}{7}
\textbf{નેગેટીવ ફીડબેક વોલ્ટેજ એમ્પ્લીફાયરનું ઓવરઓલ ગેઈનનું સૂત્ર મેળવો.}

\begin{solutionbox}
    \textbf{જવાબ}:
    નેગેટિવ ફીડબેક એમ્પ્લીફાયરમાં આઉટપુટ ઇનપુટમાં વિપરીત ફેઝમાં ફીડ થાય છે.
    
    \textbf{સર્કિટ વિશ્લેષણ:}
    ધારો કે $A$ = ઓપન લૂપ ગેઈન, $\beta$ = ફીડબેક ફેક્ટર
    
    \textbf{ડાયગ્રામ:}
    \begin{center}
    \begin{tikzpicture}[node distance=2.5cm, auto, >=latex]
        \node [gtu input] (input) {$V_i$};
        \node [draw, circle, right of=input] (sum) {$\Sigma$};
        \node [gtu block, right of=sum] (amp) {Amplifier A};
        \node [gtu output, right of=amp] (output) {$V_o$};
        \node [gtu block, below of=amp] (feedback) {Feedback $\beta$};
        
        \draw [gtu arrow] (input) -- (sum);
        \draw [gtu arrow] (sum) -- node {$V_i - \beta V_o$} (amp);
        \draw [gtu arrow] (amp) -- (output);
        \draw [gtu arrow] (output) |- (feedback);
        \draw [gtu arrow] (feedback) -| node[pos=0.99] {$-$} (sum);
    \end{tikzpicture}
    \end{center}
    
    \textbf{વ્યુત્પત્તિ:}
    \begin{itemize}
        \item એમ્પ્લીફાયરનું ઇનપુટ: $V_i - \beta V_o$
        \item આઉટપુટ: $V_o = A(V_i - \beta V_o)$
        \item $V_o = AV_i - A\beta V_o$
        \item $V_o + A\beta V_o = AV_i$
        \item $V_o(1 + A\beta) = AV_i$
        \item \textbf{એકુલ ગેઈન}: $A_f = \frac{A}{1 + A\beta}$
    \end{itemize}
    
    \textbf{મુખ્ય મુદ્દા:}
    \begin{itemize}
        \item \textbf{હર (1 + A$\beta$)}: લૂપ ગેઈન કહેવાય છે
        \item \textbf{સ્થિરતા ફેક્ટર}: સિસ્ટમ રિસ્પોન્સ નક્કી કરે છે
        \item \textbf{ગેઈન ઘટાડો}: સારી કામગીરી માટે ગેઈન આપવામાં આવે છે
    \end{itemize}
    
    \begin{mnemonicbox}
        \mnemonic{Always Divide by (1 + Loop)}
    \end{mnemonicbox}
\end{solutionbox}

% Question 1(c OR) [7 marks]
\questionmarks{1}{c}{7}
\textbf{કરંટ શન્ટ પ્રકારના નેગેટીવ ફીડબેક એમ્પ્લીફાયર દોરો અને સમજાવો અને તેના ઇનપુટ અને આઉટપુટ ઇમ્પીડન્સના સૂત્ર મેળવો.}

\begin{solutionbox}
    \textbf{જવાબ}:
    કરંટ શન્ટ ફીડબેક આઉટપુટ કરંટ સેમ્પલ કરે છે અને ઇનપુટ સાથે શન્ટમાં વોલ્ટેજ ફીડ કરે છે.
    
    \textbf{સર્કિટ ડાયગ્રામ:}
    \begin{center}
    \begin{tikzpicture}[node distance=2cm, auto, >=latex]
        \node [gtu input] (input) {$I_i$};
        \node [draw, circle, right of=input] (sum) {$\Sigma$};
        \node [gtu block, right of=sum] (amp) {Amplifier A};
        \node [gtu output, right of=amp] (output) {$I_o$};
        \node [gtu block, below of=amp] (feedback) {Feedback $\beta$};
        
        \draw [gtu arrow] (input) -- (sum);
        \draw [gtu arrow] (sum) -- (amp);
        \draw [gtu arrow] (amp) -- (output); 
        \draw [gtu arrow] (amp) |- (feedback);
        \draw [gtu arrow] (feedback) -| (sum);
    \end{tikzpicture}
    \end{center}
    
    \textbf{વિશ્લેષણ:}
    \begin{itemize}
        \item \textbf{ફીડબેક પ્રકાર}: કરંટ સેમ્પલિંગ, વોલ્ટેજ મિક્સિંગ
        \item \textbf{ઇનપુટ ઇમ્પીડન્સ}: શન્ટ ફીડબેકને કારણે ઘટે છે
        \item \textbf{આઉટપુટ ઇમ્પીડન્સ}: કરંટ સેમ્પલિંગને કારણે ઘટે છે (MDX મુજબ)
    \end{itemize}
    
    \textbf{સૂત્રો:}
    \begin{itemize}
        \item \textbf{ઇનપુટ ઇમ્પીડન્સ}: $Z_{if} = \frac{Z_i}{1 + A\beta}$
        \item \textbf{આઉટપુટ ઇમ્પીડન્સ}: $Z_{of} = \frac{Z_o}{1 + A\beta}$
    \end{itemize}
    
    \textbf{લાક્ષણિકતાઓ:}
    \begin{itemize}
        \item \textbf{નીચું ઇનપુટ ઇમ્પીડન્સ}: કરંટ સોર્સ માટે સારું
        \item \textbf{નીચું આઉટપુટ ઇમ્પીડન્સ}: વોલ્ટેજ આઉટપુટ માટે સારું
        \item \textbf{કરંટ-ટુ-વોલ્ટેજ કન્વર્ટર}: એપ્લીકેશનમાં ઉપયોગી
    \end{itemize}
    
    \begin{mnemonicbox}
        \mnemonic{Current Shunt Lowers Both Impedances}
    \end{mnemonicbox}
\end{solutionbox}

% Question 2(a) [3 marks]
\questionmarks{2}{a}{3}
\textbf{ઓસિલેટર માટે બારખૌસન ક્રાઈટેરીઆ સમજાવો.}

\begin{solutionbox}
    \textbf{જવાબ}:
    ફીડબેક સર્કિટમાં સતત ઓસિલેશન માટે બે શરતો એક સાથે પૂરી થવી જોઈએ.
    
    \begin{tabulary}{\textwidth}{L|L|L}
        \toprule
        \textbf{ક્રાઈટેરીઆ} & \textbf{શરત} & \textbf{વર્ણન} \\
        \midrule
        \textbf{મેગ્નિટ્યુડ} & $|A\beta| = 1$ & લૂપ ગેઈન એકમ \\
        \textbf{ફેઝ} & $\angle A\beta = 0^\circ$ અથવા $360^\circ$ & શૂન્ય ફેઝ શિફ્ટ \\
        \bottomrule
    \end{tabulary}
    
    \begin{itemize}
        \item \textbf{એકમ લૂપ ગેઈન}: સિગ્નલ એમ્પ્લિટ્યુડ જાળવે છે
        \item \textbf{શૂન્ય ફેઝ શિફ્ટ}: પોઝીટીવ ફીડબેક સુનિશ્ચિત કરે છે
        \item \textbf{સતત ઓસિલેશન}: બંને શરતો સ્વ-ટકાઉ સિગ્નલ બનાવે છે
    \end{itemize}
    
    \begin{mnemonicbox}
        \mnemonic{One Magnitude, Zero Phase}
    \end{mnemonicbox}
\end{solutionbox}

% Question 2(b) [4 marks]
\questionmarks{2}{b}{4}
\textbf{સ્વચ્છ ડાયગ્રામની મદદથી ટેન્ક સર્કિટ સમજાવો.}

\begin{solutionbox}
    \textbf{જવાબ}:
    ટેન્ક સર્કિટ ઓસિલેટર સર્કિટ માટે ફ્રીક્વન્સી સિલેક્ટિવ પોઝીટીવ ફીડબેક પ્રદાન કરે છે.
    
    \textbf{સર્કિટ ડાયગ્રામ:}
    \begin{center}
    \begin{circuitikz}[american]
        \draw (0,0) to[L, l=$L$] (0,2) -- (2,2) to[C, l=$C$] (2,0) -- (0,0);
        \draw (2,2) -- (3,2) to[R, l=$R$] (3,0) -- (2,0);
    \end{circuitikz}
    \end{center}
    
    \textbf{ઓપરેશન:}
    રેઝોનન્ટ ફ્રીક્વન્સી પર, LC ટેન્ક સર્કિટ દર્શાવે છે:
    
    \begin{tabulary}{\textwidth}{L|L|L}
        \toprule
        \textbf{પેરામીટર} & \textbf{મૂલ્ય} & \textbf{અસર} \\
        \midrule
        \textbf{રીએક્ટન્સ} & $X_L = X_C$ & રેઝોનન્સ \\
        \textbf{ઇમ્પીડન્સ} & મહત્તમ & ઉચ્ચ સિલેક્ટિવિટી \\
        \textbf{ફેઝ} & $0^\circ$ & એકમ ફીડબેક \\
        \bottomrule
    \end{tabulary}
    
    \begin{itemize}
        \item \textbf{ઊર્જા સંગ્રહ}: L અને C ઊર્જાની આપ-લે કરે છે
        \item \textbf{ફ્રીક્વન્સી પસંદગી}: તીક્ષ્ણ રેઝોનન્સ લાક્ષણિકતા
        \item \textbf{ઓસિલેશન ટકાવી રાખવું}: પોઝીટીવ ફીડબેક પ્રદાન કરે છે
    \end{itemize}
    
    \begin{mnemonicbox}
        \mnemonic{Tank Stores Energy, Selects Frequency}
    \end{mnemonicbox}
\end{solutionbox}

% Question 2(c) [7 marks]
\questionmarks{2}{c}{7}
\textbf{હાર્ટલી ઓસિલેટર દોરો અને સમજાવો. ઉપરાંત હાર્ટલી ઓસિલેટરની ઓસિલેશનની ફ્રીક્વન્સીનું સૂત્ર જણાવો.}

\begin{solutionbox}
    \textbf{જવાબ}:
    હાર્ટલી ઓસિલેટર ફ્રીક્વન્સી જનરેશન માટે ટેન્ક સર્કિટમાં ટેપ્ડ ઇન્ડક્ટરનો ઉપયોગ કરે છે.
    
    \textbf{સર્કિટ ડાયગ્રામ:}
    \begin{center}
    \begin{circuitikz}[american]
        \draw (0,0) node[npn](Q){Q}
        (Q.E) to[R, l=$R_E$] (0,-2) node[ground]{}
        (Q.C) to[L, l=$L_{RFC}$] (0,2) node[vcc]{$V_{CC}$}
        (Q.C) to[C, l=$C_{out}$] (2,0) -- (2,-2)
        
        % Tank
        (4,0) to[C, l=$C$] (4,-2)
        (5,0) to[L, l=$L_1$] (5,-1) to[L, l=$L_2$] (5,-2)
        (5,-1) to[short] (4,-1) % Tap
        (2,-2) -- (5,-2) node[ground]{}
        
        % Feedback
        (5,-1) -- (5,-2.5) -- (-1,-2.5) -- (-1,0) to[short] (Q.B)
        (Q.B) to[R, l=$R_1$] (-1,2) to[short] (0,2)
        (Q.B) to[R, l=$R_2$] (-1,-2) node[ground]{};
    \end{circuitikz}
    \end{center}
    
    \textbf{ઓપરેશન:}
    \begin{itemize}
        \item \textbf{ટેપ્ડ ઇન્ડક્ટર}: $L_1$ અને $L_2$ ફીડબેક પ્રદાન કરે છે
        \item \textbf{ટેન્ક સર્કિટ}: $L_1+L_2$ સાથે $C$ ફ્રીક્વન્સી નક્કી કરે છે
        \item \textbf{પોઝીટીવ ફીડબેક}: $L_1-L_2$ કપલિંગ દ્વારા ફેઝ શિફ્ટ
    \end{itemize}
    
    \textbf{ફ્રીક્વન્સી સૂત્ર:}
    \[ f = \frac{1}{2\pi\sqrt{(L_1+L_2)C}} \]
    
    \textbf{મુખ્ય લાક્ષણિકતાઓ:}
    \begin{itemize}
        \item \textbf{સારી ફ્રીક્વન્સી સ્થિરતા}: ઇન્ડક્ટર-આધારિત ટ્યુનિંગ
        \item \textbf{સરળ ટ્યુનિંગ}: વેરિયેબલ ઇન્ડક્ટર અથવા કેપેસિટર
        \item \textbf{RF એપ્લીકેશન}: ઉચ્ચ ફ્રીક્વન્સી માટે યોગ્ય
    \end{itemize}
    
    \begin{mnemonicbox}
        \mnemonic{Hartley Has Tapped inductor}
    \end{mnemonicbox}
\end{solutionbox}

% Question 2(a OR) [3 marks]
\questionmarks{2}{a}{3}
\textbf{ઓસિલેટરના પદને પોઝીટીવ ફીડબેક એમ્પ્લીફાયર તરીકે સમજાવો.}

\begin{solutionbox}
    \textbf{જવાબ}:
    ઓસિલેટર બાહ્ય ઇનપુટ સિગ્નલ વિના પોઝીટીવ ફીડબેકનો ઉપયોગ કરીને AC સિગ્નલ ઉત્પન્ન કરે છે.
    
    \begin{tabulary}{\textwidth}{L|L|L}
        \toprule
        \textbf{પેરામીટર} & \textbf{એમ્પ્લીફાયર} & \textbf{ઓસિલેટર} \\
        \midrule
        \textbf{ઇનપુટ} & બાહ્ય સિગ્નલ & બાહ્ય ઇનપુટ નહીં \\
        \textbf{ફીડબેક} & નેગેટિવ ઉપયોગ કરી શકે & પોઝીટીવ ઉપયોગ કરે \\
        \textbf{આઉટપુટ} & એમ્પ્લિફાઇડ ઇનપુટ & સ્વ-ઉત્પન્ન AC \\
        \bottomrule
    \end{tabulary}
    
    \begin{itemize}
        \item \textbf{સ્વ-ટકાઉ}: પોઝીટીવ ફીડબેક ઓસિલેશન જાળવે છે
        \item \textbf{બારખૌસન ક્રાઈટેરીઆ}: લૂપ ગેઈન = 1, ફેઝ = $0^\circ$
        \item \textbf{સિગ્નલ જનરેશન}: DC સપ્લાયમાંથી AC બનાવે છે
    \end{itemize}
    
    \begin{mnemonicbox}
        \mnemonic{Positive feedback Powers Perpetual signals}
    \end{mnemonicbox}
\end{solutionbox}

% Question 2(b OR) [4 marks]
\questionmarks{2}{b}{4}
\textbf{ક્રિસ્ટલ ઓસિલેટર દોરો અને સમજાવો.}

\begin{solutionbox}
    \textbf{જવાબ}:
    ક્રિસ્ટલ ઓસિલેટર ઉચ્ચ સ્થિરતા માટે ક્વાર્ટ્ઝ ક્રિસ્ટલના પીઝોઇલેક્ટ્રિક ઇફેક્ટનો ઉપયોગ કરે છે.
    
    \textbf{સર્કિટ ડાયગ્રામ:}
    \begin{center}
    \begin{circuitikz}[american]
        \draw (0,0) node[npn](Q){Q}
        (Q.E) to[R, l=$R_E$] (0,-2) node[ground]{}
        (Q.C) to[L, l=$L$] (0,2) node[vcc]{$V_{CC}$}
        (Q.C) to[C, l=$C_1$] (0,-2) 
        (Q.B) to[R, l=$R_1$] (-1,2) to[short] (0,2)
        (Q.B) to[R, l=$R_2$] (-1,-2) node[ground]{}
        (Q.B) to[short] (-2,0) to[piezoelectric, l=XTAL] (-2,-2) node[ground]{};
    \end{circuitikz}
    \end{center}
    
    \textbf{લાક્ષણિકતાઓ:}
    \begin{tabulary}{\textwidth}{L|L|L}
        \toprule
        \textbf{ગુણધર્મ} & \textbf{મૂલ્ય} & \textbf{ફાયદો} \\
        \midrule
        \textbf{સ્થિરતા} & ±0.01\% & ખૂબ ઉચ્ચી \\
        \textbf{Q ફેક્ટર} & >10,000 & તીક્ષ્ણ રેઝોનન્સ \\
        \textbf{તાપમાન} & નીચું ડ્રિફ્ટ & સ્થિર ફ્રીક્વન્સી \\
        \bottomrule
    \end{tabulary}
    
    \begin{itemize}
        \item \textbf{પીઝોઇલેક્ટ્રિક ઇફેક્ટ}: મિકેનિકલ વાઇબ્રેશન ઇલેક્ટ્રિકલ સિગ્નલ બનાવે છે
        \item \textbf{ઉચ્ચ Q}: ખૂબ સ્થિર ફ્રીક્વન્સી જનરેશન
        \item \textbf{ક્લોક એપ્લીકેશન}: ડિજિટલ સિસ્ટમમાં ઉપયોગ
    \end{itemize}
    
    \begin{mnemonicbox}
        \mnemonic{Crystal Creates Constant frequency}
    \end{mnemonicbox}
\end{solutionbox}

% Question 2(c OR) [7 marks]
\questionmarks{2}{c}{7}
\textbf{UJTની રચના, સિમ્બોલ તથા ઇક્વિવેલેન્ટ સર્કિટ દોરો અને તેને વિસ્તૃતમાં સમજાવો.}

\begin{solutionbox}
    \textbf{જવાબ}:
    UJT (Unijunction Transistor) અનોખી સ્વિચિંગ લાક્ષણિકતાઓ ધરાવતું ત્રણ-ટર્મિનલ ડિવાઇસ છે.
    
    \textbf{રચના:}
    \begin{center}
    \begin{tikzpicture}
        \draw (0,0) rectangle (2,3);
        \node at (1,0.5) {N-type Bar};
        \draw [fill=white] (1.5, 1.5) circle (0.3);
        \node at (1.5, 1.5) {P};
        \node at (0, 1.5) {E};
        \draw (0, 1.5) -- (1.2, 1.5);
        \node at (1, 2.8) {B2};
        \node at (1, 0.2) {B1};
    \end{tikzpicture}
    \end{center}
    
    \textbf{ઇક્વિવેલેન્ટ સર્કિટ:}
    \begin{center}
    \begin{circuitikz}
        \draw (0,3) node[above]{B2} to[R, l=$R_{B2}$] (0,1.5) to[R, l=$R_{B1}$] (0,0) node[below]{B1};
        \draw (-2,1.5) node[left]{E} to[D, l=$D$] (0,1.5);
    \end{circuitikz}
    \end{center}
    
    \textbf{ઓપરેશન:}
    \begin{itemize}
        \item \textbf{ઇન્ટ્રિન્સિક સ્ટેન્ડઓફ રેશિયો}: $\eta = R_1/(R_1+R_2)$
        \item \textbf{પીક પોઇન્ટ વોલ્ટેજ}: $V_P = \eta V_{BB} + V_D$
        \item \textbf{નેગેટિવ રેઝિસ્ટન્સ}: પીક પોઇન્ટ પછી
    \end{itemize}
    
    \textbf{એપ્લીકેશન:}
    \begin{itemize}
        \item \textbf{રિલેક્સેશન ઓસિલેટર}: સોટૂથ વેવ જનરેશન
        \item \textbf{ટ્રિગર સર્કિટ}: SCR ફાયરિંગ સર્કિટ
        \item \textbf{ટાઇમિંગ એપ્લીકેશન}: RC ચાર્જિંગ સર્કિટ
    \end{itemize}
    
    \begin{mnemonicbox}
        \mnemonic{UJT Uses Unique Junction Technology}
    \end{mnemonicbox}
\end{solutionbox}

% Question 3(a) [3 marks]
\questionmarks{3}{a}{3}
\textbf{ઓપરેટિંગ પોઇન્ટના આધારે પાવર એમ્પ્લીફાયરને વર્ગીકૃત કરો.}

\begin{solutionbox}
    \textbf{જવાબ}:
    પાવર એમ્પ્લીફાયર ટ્રાન્ઝિસ્ટર કન્ડક્શન એંગલ અને બાયસ પોઇન્ટના આધારે વર્ગીકૃત થાય છે.
    
    \begin{tabulary}{\textwidth}{L|L|L|L}
        \toprule
        \textbf{ક્લાસ} & \textbf{કન્ડક્શન એંગલ} & \textbf{કાર્યક્ષમતા} & \textbf{એપ્લીકેશન} \\
        \midrule
        \textbf{ક્લાસ A} & $360^\circ$ & 25-50\% & ઓડિયો, લો પાવર \\
        \textbf{ક્લાસ B} & $180^\circ$ & 78.5\% & પુશ-પુલ \\
        \textbf{ક્લાસ AB} & $180^\circ$-$360^\circ$ & 60-70\% & ઓડિયો પાવર \\
        \textbf{ક્લાસ C} & $<180^\circ$ & >90\% & RF, ટ્યુન્ડ \\
        \bottomrule
    \end{tabulary}
    
    \begin{itemize}
        \item \textbf{બાયસ પોઇન્ટ}: ઓપરેટિંગ ક્લાસ નક્કી કરે છે
        \item \textbf{કાર્યક્ષમતા ટ્રેડ-ઓફ}: ઉચ્ચ કાર્યક્ષમતા, વધુ વિકૃતિ
        \item \textbf{એપ્લીકેશન સ્પેસિફિક}: જરૂરિયાત પ્રમાણે પસંદગી
    \end{itemize}
    
    \begin{mnemonicbox}
        \mnemonic{All Big Amplifiers Can deliver power}
    \end{mnemonicbox}
\end{solutionbox}

% Question 3(b) [4 marks]
\questionmarks{3}{b}{4}
\textbf{કોમ્પ્લીમેંટરી સિમેટ્રી પુશ પુલ પાવર એમ્પ્લીફાયરને દોરો અને સમજાવો.}

\begin{solutionbox}
    \textbf{જવાબ}:
    સેન્ટર-ટેપ્ડ ટ્રાન્સફોર્મર વિના કાર્યક્ષમ પાવર એમ્પ્લિફિકેશન માટે NPN અને PNP ટ્રાન્ઝિસ્ટરનો ઉપયોગ કરે છે.
    
    \textbf{સર્કિટ ડાયગ્રામ:}
    \begin{center}
    \begin{circuitikz}[american]
        \draw (0,2) node[vcc]{$+V_{CC}$} to[short] (0,1) node[npn](Q1){Q1};
        \draw (0,-2) node[vcc]{$-V_{CC}$} to[short] (0,-1) node[pnp, anchor=C](Q2){Q2};
        \draw (0,1) node[npn](NPN){Q1};
        \draw (0,-1) node[pnp](PNP){Q2};
        \draw (NPN.E) -- (PNP.E);
        \draw (NPN.E) to[short] (1,0) to[R, l=$R_L$] (1,-2) node[ground]{};
        \draw (NPN.B) -- (PNP.B) to[short] (-1,0) node[left]{Input};
        \draw (NPN.C) node[vcc]{$+V_{CC}$};
        \draw (PNP.C) node[vcc]{$-V_{CC}$};
    \end{circuitikz}
    \end{center}
    
    \textbf{ઓપરેશન:}
    \begin{itemize}
        \item \textbf{પોઝીટીવ હાફ-સાયકલ}: NPN કન્ડક્ટ કરે, PNP બંધ
        \item \textbf{નેગેટિવ હાફ-સાયકલ}: PNP કન્ડક્ટ કરે, NPN બંધ
        \item \textbf{કોમ્પ્લીમેંટરી એક્શન}: બંને ટ્રાન્ઝિસ્ટર વૈકલ્પિક હાફ-સાયકલ હેન્ડલ કરે
    \end{itemize}
    
    \textbf{ફાયદા:}
    \begin{itemize}
        \item \textbf{ટ્રાન્સફોર્મર નહીં}: ડાયરેક્ટ કપલિંગ ટુ લોડ
        \item \textbf{ઉચ્ચ કાર્યક્ષમતા}: ક્લાસ B ઓપરેશન
        \item \textbf{કોમ્પેક્ટ ડિઝાઇન}: ઓછા કોમ્પોનન્ટ્સ
        \item \textbf{સારું પાવર ટ્રાન્સફર}: ડાયરેક્ટ કપલિંગ
    \end{itemize}
    
    \begin{mnemonicbox}
        \mnemonic{Complementary transistors Complete the cycle}
    \end{mnemonicbox}
\end{solutionbox}

% Question 3(c) [7 marks]
\questionmarks{3}{c}{7}
\textbf{ક્લાસ-B પુશ પુલ એમ્પ્લીફાયરની કાર્યક્ષમતાનું સૂત્ર મેળવો.}

\begin{solutionbox}
    \textbf{જવાબ}:
    ક્લાસ B પુશ-પુલ એમ્પ્લીફાયરમાં દરેક ટ્રાન્ઝિસ્ટર ઇનપુટ સાયકલના 180° માટે કન્ડક્ટ કરે છે.
    
    \textbf{વિશ્લેષણ:}
    સાઇનુસોઇડલ ઇનપુટ માટે: $V_{in} = V_m \sin \omega t$
    
    \textbf{આઉટપુટ પાવર:}
    \begin{itemize}
        \item પીક આઉટપુટ વોલ્ટેજ: $V_{om} = V_{CC}$
        \item RMS આઉટપુટ વોલ્ટેજ: $V_{o(rms)} = V_{CC}/\sqrt{2}$
        \item \textbf{આઉટપુટ પાવર}: $P_o = V_{o(rms)}^2/R_L = V_{CC}^2/2R_L$
    \end{itemize}
    
    \textbf{ઇનપુટ પાવર:}
    \begin{itemize}
        \item DC કરંટ (એવરેજ): $I_{dc} = 2I_m/\pi$
        \item જ્યાં $I_m = V_{CC}/R_L$
        \item \textbf{ઇનપુટ પાવર}: $P_{in} = V_{CC} \times I_{dc} = 2V_{CC}I_m/\pi = 2V_{CC}^2/\pi R_L$
    \end{itemize}
    
    \textbf{કાર્યક્ષમતા ગણતરી:}
    \[ \eta = \frac{P_o}{P_{in}} = \frac{V_{CC}^2/2R_L}{2V_{CC}^2/\pi R_L} \]
    \[ \eta = \frac{\pi}{4} = 0.785 = 78.5\% \]
    
    \textbf{મુખ્ય મુદ્દા:}
    \begin{itemize}
        \item \textbf{મહત્તમ સૈદ્ધાંતિક કાર્યક્ષમતા}: 78.5\%
        \item \textbf{ક્લાસ B ફાયદો}: ક્લાસ A (25\%) કરતાં ખૂબ ઊંચી
        \item \textbf{પ્રેક્ટિકલ કાર્યક્ષમતા}: નુકસાનને કારણે થોડી ઓછી
    \end{itemize}
    
    \begin{mnemonicbox}
        \mnemonic{Push-Pull Provides Pi/4 efficiency}
    \end{mnemonicbox}
\end{solutionbox}

% Question 3(a OR) [3 marks]
\questionmarks{3}{a}{3}
\textbf{વોલ્ટેજ અને પાવર એમ્પ્લીફાયર વચ્ચેનો તફાવત કરો.}

\begin{solutionbox}
    \textbf{જવાબ}:
    વોલ્ટેજ અને પાવર એમ્પ્લીફાયર ઇલેક્ટ્રોનિક સિસ્ટમમાં જુદા હેતુઓ સેવે છે.
    
    \begin{tabulary}{\textwidth}{L|L|L}
        \toprule
        \textbf{પેરામીટર} & \textbf{વોલ્ટેજ એમ્પ્લીફાયર} & \textbf{પાવર એમ્પ્લીફાયર} \\
        \midrule
        \textbf{હેતુ} & વોલ્ટેજ વધારવું & પાવર વધારવું \\
        \textbf{લોડ} & ઉચ્ચ ઇમ્પીડન્સ & નીચું ઇમ્પીડન્સ \\
        \textbf{કાર્યક્ષમતા} & મહત્વપૂર્ણ નથી & ખૂબ મહત્વપૂર્ણ \\
        \textbf{વિકૃતિ} & ઓછી હોવી જોઈએ & મધ્યમ સ્વીકાર્ય \\
        \textbf{કપલિંગ} & RC/ડાયરેક્ટ & ટ્રાન્સફોર્મર \\
        \bottomrule
    \end{tabulary}
    
    \begin{itemize}
        \item \textbf{ડિઝાઇન પ્રાથમિકતા}: વોલ્ટેજ ગેઈન વર્સીસ પાવર ડિલિવરી
        \item \textbf{એપ્લીકેશન}: સિગ્નલ પ્રોસેસિંગ વર્સીસ લોડ ડ્રાઇવિંગ
        \item \textbf{સર્કિટ જટિલતા}: સરળ વર્સીસ જટિલ પાવર સ્ટેજ
    \end{itemize}
    
    \begin{mnemonicbox}
        \mnemonic{Voltage amplifies signal, Power drives load}
    \end{mnemonicbox}
\end{solutionbox}

% Question 3(b OR) [4 marks]
\questionmarks{3}{b}{4}
\textbf{ક્લાસ AB પાવર એમ્પ્લીફાયર ડાયગ્રામ સાથે સમજાવો.}

\begin{solutionbox}
    \textbf{જવાબ}:
    ક્લાસ AB ક્લાસ A અને ક્લાસ B વચ્ચે ઓપરેટ કરે છે, ક્રોસઓવર ડિસ્ટોર્શન ઘટાડે છે.
    
    \textbf{સર્કિટ ડાયગ્રામ:}
    \begin{center}
    \begin{circuitikz}[american]
        \draw (0,2) node[vcc]{$+V_{CC}$} to[R, l=$R$] (0,1);
        \draw (0,1) to[D*, l=$D_1$] (0,0) to[D*, l=$D_2$] (0,-1);
        \draw (0,-1) to[R, l=$R$] (0,-2) node[vcc]{$-V_{CC}$};
        
        \draw (2,1) node[npn](Q1){Q1};
        \draw (2,-1) node[pnp](Q2){Q2};
        \draw (0,1) -- (Q1.B);
        \draw (0,-1) -- (Q2.B);
        \draw (Q1.C) node[vcc]{$+V_{CC}$};
        \draw (Q2.C) node[vcc]{$-V_{CC}$};
        
        \draw (Q1.E) -- (Q2.E) -- (3,0) to[R, l=$R_L$] (3,-2) node[ground]{};
        \draw (-1,0) to[C] (0,0); % Input coupling
        \node at (-1.2,0) {Input};
    \end{circuitikz}
    \end{center}
    
    \textbf{ઓપરેશન:}
    \begin{itemize}
        \item \textbf{થોડું ફોરવર્ડ બાયસ}: બંને ટ્રાન્ઝિસ્ટર થોડા ઓન
        \item \textbf{કન્ડક્શન એંગલ}: >$180^\circ$ પણ <$360^\circ$
        \item \textbf{ઓવરલેપ કન્ડક્શન}: ક્રોસઓવર ડિસ્ટોર્શન દૂર કરે છે
    \end{itemize}
    
    \textbf{લાક્ષણિકતાઓ:}
    \begin{tabulary}{\textwidth}{L|L|L}
        \toprule
        \textbf{પેરામીટર} & \textbf{મૂલ્ય} & \textbf{ફાયદો} \\
        \midrule
        \textbf{કાર્યક્ષમતા} & 60-70\% & ક્લાસ A કરતાં સારી \\
        \textbf{વિકૃતિ} & ઓછી & ક્લાસ B કરતાં સારી \\
        \textbf{બાયસ} & થોડું ફોરવર્ડ & સમાધાનકારી ઉકેલ \\
        \bottomrule
    \end{tabulary}
    
    \begin{mnemonicbox}
        \mnemonic{AB Avoids Bad crossover distortion}
    \end{mnemonicbox}
\end{solutionbox}

% Question 3(c OR) [7 marks]
\questionmarks{3}{c}{7}
\textbf{સીરીજ ફેડ ક્લાસ-A પાવર એમ્પ્લીફાયરની કાર્યક્ષમતાનું સૂત્ર મેળવો.}

\begin{solutionbox}
    \textbf{જવાબ}:
    સીરીજ ફેડ ક્લાસ A એમ્પ્લીફાયરમાં DC સપ્લાય લોડ સાથે સીરીજમાં જોડાયેલું હોય છે.
    
    \textbf{સર્કિટ વિશ્લેષણ:}
    \begin{itemize}
        \item \textbf{DC સપ્લાય વોલ્ટેજ}: $V_{CC}$
        \item \textbf{ક્વિસન્ટ કરંટ}: $I_{CQ} = V_{CC}/2R_L$ (મહત્તમ પાવર માટે)
        \item \textbf{ક્વિસન્ટ વોલ્ટેજ}: $V_{CEQ} = V_{CC}/2$
    \end{itemize}
    
    \textbf{AC વિશ્લેષણ:}
    \begin{itemize}
        \item \textbf{મહત્તમ આઉટપુટ વોલ્ટેજ સ્વિંગ}: $V_{om} = V_{CC}/2$
        \item \textbf{આઉટપુટ પાવર}: $P_o = V_{om}^2/2R_L = V_{CC}^2/8R_L$
    \end{itemize}
    
    \textbf{DC પાવર:}
    \begin{itemize}
        \item \textbf{DC કરંટ}: $I_{dc} = I_{CQ} = V_{CC}/2R_L$
        \item \textbf{ઇનપુટ પાવર}: $P_{in} = V_{CC} \times I_{dc} = V_{CC}^2/2R_L$
    \end{itemize}
    
    \textbf{કાર્યક્ષમતા:}
    \[ \eta = \frac{P_o}{P_{in}} = \frac{V_{CC}^2/8R_L}{V_{CC}^2/2R_L} \]
    \[ \eta = \frac{1}{4} = 0.25 = 25\% \]
    
    \textbf{મુખ્ય મુદ્દા:}
    \begin{itemize}
        \item \textbf{મહત્તમ સૈદ્ધાંતિક કાર્યક્ષમતા}: 25\%
        \item \textbf{પાવર બર્બાદી}: 75\% ગરમીમાં ખોવાય છે
        \item \textbf{ડિઝાઇન મર્યાદા}: નબળી કાર્યક્ષમતા પણ સારી લીનિયરિટી
    \end{itemize}
    
    \begin{mnemonicbox}
        \mnemonic{Class A Achieves quarter efficiency}
    \end{mnemonicbox}
\end{solutionbox}

% Question 4(a) [3 marks]
\questionmarks{4}{a}{3}
\textbf{IC 741 OP-AMPનો પિન ડાયગ્રામ દોરો અને સમજાવો.}

\begin{solutionbox}
    \textbf{જવાબ}:
    IC 741 ઇન્ડસ્ટ્રી સ્ટાન્ડર્ડ પિનઆઉટ સાથે 8-પિન ડ્યુઅલ-ઇન-લાઇન પેકેજ ઓપરેશનલ એમ્પ્લીફાયર છે.
    
    \textbf{પિન ડાયગ્રામ:}
    \begin{center}
    \begin{tikzpicture}
        \draw[thick] (0,0) rectangle (3,4);
        \draw (1.5,4) arc (180:360:0.2);
        \foreach \y in {1,2,3,4} {
            \draw (-0.5, 4.5-\y) -- (0, 4.5-\y) node[right, font=\tiny] {\y};
            \draw (3, 4.5-\y) -- (3.5, 4.5-\y) node[left, font=\tiny] {\the\numexpr9-\y};
        }
        % Labels
        \node[right, font=\tiny] at (0, 3.5) {Offset Null};
        \node[right, font=\tiny] at (0, 2.5) {Inv In (-)};
        \node[right, font=\tiny] at (0, 1.5) {Non-Inv (+)};
        \node[right, font=\tiny] at (0, 0.5) {$V_{EE}$ (-)};
        
        \node[left, font=\tiny] at (3, 3.5) {NC};
        \node[left, font=\tiny] at (3, 2.5) {$V_{CC}$ (+)};
        \node[left, font=\tiny] at (3, 1.5) {Output};
        \node[left, font=\tiny] at (3, 0.5) {Offset Null};
    \end{tikzpicture}
    \end{center}
    
    \textbf{પિન કન્ફિગરેશન:}
    \begin{tabulary}{\textwidth}{C|L|L}
        \toprule
        \textbf{પિન} & \textbf{ફંક્શન} & \textbf{વર્ણન} \\
        \midrule
        \textbf{1} & ઓફસેટ નલ & ઓફસેટ એડજસ્ટમેન્ટ \\
        \textbf{2} & ઇન્વર્ટિંગ ઇનપુટ & નેગેટિવ ઇનપુટ \\
        \textbf{3} & નોન-ઇન્વર્ટિંગ ઇનપુટ & પોઝિટિવ ઇનપુટ \\
        \textbf{4} & $-V_{CC}$ & નેગેટિવ સપ્લાય \\
        \textbf{5} & ઓફસેટ નલ & ઓફસેટ એડજસ્ટમેન્ટ \\
        \textbf{6} & આઉટપુટ & એમ્પ્લીફાયર આઉટપુટ \\
        \textbf{7} & $+V_{CC}$ & પોઝિટિવ સપ્લાય \\
        \textbf{8} & NC & કોઈ કનેક્શન નહીં \\
        \bottomrule
    \end{tabulary}
    
    \begin{mnemonicbox}
        \mnemonic{Null, Negative, Positive, Negative supply, Null, Output, Positive supply, Nothing}
    \end{mnemonicbox}
\end{solutionbox}

% Question 4(b) [4 marks]
\questionmarks{4}{b}{4}
\textbf{OP-AMPના નીચેના પરિમાણ વ્યાખ્યાયિત કરો. ૧. ઇનપુટ ઓફસેટ વોલ્ટેજ ૨. સી.એમ.આર.આર}

\begin{solutionbox}
    \textbf{જવાબ}:
    આ પેરામીટર્સ પ્રેક્ટિકલ ઓપરેશનલ એમ્પ્લીફાયરની નોન-આઇડીયલ લાક્ષણિકતાઓ વ્યાખ્યાયિત કરે છે.
    
    \textbf{૧. ઇનપુટ ઓફસેટ વોલ્ટેજ ($V_{io}$):}
    \begin{itemize}
        \item \textbf{વ્યાખ્યા}: આઉટપુટ શૂન્ય બનાવવા માટે ઇનપુટ્સ વચ્ચે લાગુ કરવામાં આવતું DC વોલ્ટેજ
        \item \textbf{સામાન્ય મૂલ્ય}: 741 માટે 1-5 mV
        \item \textbf{કારણ}: ઇનપુટ ટ્રાન્ઝિસ્ટરમાં મિસમેચ
        \item \textbf{અસર}: DC એપ્લીકેશનમાં આઉટપુટ એરર
    \end{itemize}
    
    \textbf{૨. કોમન મોડ રિજેક્શન રેશિયો (CMRR):}
    \begin{itemize}
        \item \textbf{વ્યાખ્યા}: બંને ઇનપુટ્સ પર કોમન સિગ્નલ રિજેક્ટ કરવાની ક્ષમતા
        \item \textbf{સૂત્ર}: $CMRR = A_d/A_{cm}$
        \item \textbf{સામાન્ય મૂલ્ય}: 741 માટે 90 dB
        \item \textbf{મહત્વ}: નોઇઝ ઇમ્યુનિટી
    \end{itemize}
    
    \begin{tabulary}{\textwidth}{L|L|L|L|L}
        \toprule
        \textbf{પેરામીટર} & \textbf{સિમ્બોલ} & \textbf{એકમ} & \textbf{આઇડીયલ} & \textbf{741 સામાન્ય} \\
        \midrule
        \textbf{ઇનપુટ ઓફસેટ વોલ્ટેજ} & $V_{io}$ & mV & 0 & 2 \\
        \textbf{CMRR} & - & dB & $\infty$ & 90 \\
        \bottomrule
    \end{tabulary}
    
    \begin{mnemonicbox}
        \mnemonic{Offset creates Output error, CMRR Rejects common signals}
    \end{mnemonicbox}
\end{solutionbox}

% Question 4(c) [7 marks]
\questionmarks{4}{c}{7}
\textbf{IC 741ની મદદથી ઇન્વર્ટિંગ એમ્પ્લીફાયર વિસ્તૃતમાં સમજાવો.}

\begin{solutionbox}
    \textbf{જવાબ}:
    ઇન્વર્ટિંગ એમ્પ્લીફાયર ઇન્વર્ટિંગ ટર્મિનલ પર લાગુ ઇનપુટ સાથે નેગેટિવ ફીડબેકનો ઉપયોગ કરે છે.
    
    \textbf{સર્કિટ ડાયગ્રામ:}
    \begin{center}
    \begin{circuitikz}[american]
        \draw (0,0) node[op amp](opamp){}
        (opamp.+) node[ground]{}
        (opamp.-) to[R, l=$R_1$] (-2, 0.5) node[left]{$V_{in}$}
        (opamp.-) -- (0, 0.5) -- (0, 1.5) to[R, l=$R_f$] (2, 1.5) -- (opamp.out)
        (opamp.out) -- (3,0) node[right]{$V_{out}$};
    \end{circuitikz}
    \end{center}
    
    \textbf{વિશ્લેષણ:}
    વર્ચ્યુઅલ શોર્ટ કોન્સેપ્ટનો ઉપયોગ કરીને:
    \begin{itemize}
        \item \textbf{$V_+ = V_- = 0V$} (વર્ચ્યુઅલ ગ્રાઉન્ડ)
        \item \textbf{ઇનપુટ કરંટ}: $I_1 = V_{in}/R_1$
        \item \textbf{ફીડબેક કરંટ}: $I_f = V_{out}/R_f$
        \item \textbf{કરંટ બેલેન્સ}: $I_1 = I_f$ (ઓપ-એમ્પમાં કોઈ કરંટ નહીં)
    \end{itemize}
    
    \textbf{વ્યુત્પત્તિ:}
    \begin{itemize}
        \item $V_{in}/R_1 = -V_{out}/R_f$
        \item \textbf{વોલ્ટેજ ગેઈન: $A_v = -R_f/R_1$}
    \end{itemize}
    
    \textbf{લાક્ષણિકતાઓ:}
    \begin{tabulary}{\textwidth}{L|L|L}
        \toprule
        \textbf{પેરામીટર} & \textbf{એક્સપ્રેશન} & \textbf{નોંધ} \\
        \midrule
        \textbf{વોલ્ટેજ ગેઈન} & $-R_f/R_1$ & નેગેટિવ સાઇન \\
        \textbf{ઇનપુટ ઇમ્પીડન્સ} & $R_1$ & નીચું ઇમ્પીડન્સ \\
        \textbf{આઉટપુટ ઇમ્પીડન્સ} & $\approx 0\Omega$ & ખૂબ નીચું \\
        \textbf{બેન્ડવિડ્થ} & $f = GBW/|A_v|$ & ગેઈન-બેન્ડવિડ્થ પ્રોડક્ટ \\
        \bottomrule
    \end{tabulary}
    
    \textbf{એપ્લીકેશન:}
    \begin{itemize}
        \item \textbf{સિગ્નલ ઇન્વર્શન}: ફેઝ રિવર્સલ
        \item \textbf{સ્કેલ ફેક્ટર}: પ્રોગ્રામેબલ ગેઈન
        \item \textbf{AC એમ્પ્લિફિકેશન}: કપલિંગ કેપેસિટર સાથે
    \end{itemize}
    
    \begin{mnemonicbox}
        \mnemonic{Inverting Input gives Inverted output}
    \end{mnemonicbox}
\end{solutionbox}

% Question 4(a OR) [3 marks]
\questionmarks{4}{a}{3}
\textbf{Ideal OP-AMPની લાક્ષણિકતાની સૂચિ બનાવો.}

\begin{solutionbox}
    \textbf{જવાબ}:
    આઇડીયલ ઓપ-એમ્પ બધા પેરામીટર્સ માટે સૈદ્ધાંતિક મર્યાદા સાથે સંપૂર્ણ એમ્પ્લીફાયરનું પ્રતિનિધિત્વ કરે છે.
    
    \begin{tabulary}{\textwidth}{L|L|L}
        \toprule
        \textbf{પેરામીટર} & \textbf{આઇડીયલ મૂલ્ય} & \textbf{પ્રેક્ટિકલ ઇમ્પેક્ટ} \\
        \midrule
        \textbf{ઓપન લૂપ ગેઈન} & $\infty$ & સંપૂર્ણ એમ્પ્લિફિકેશન \\
        \textbf{ઇનપુટ ઇમ્પીડન્સ} & $\infty$ & કોઈ ઇનપુટ કરંટ નહીં \\
        \textbf{આઉટપુટ ઇમ્પીડન્સ} & $0\Omega$ & સંપૂર્ણ વોલ્ટેજ સોર્સ \\
        \textbf{બેન્ડવિડ્થ} & $\infty$ & કોઈ ફ્રીક્વન્સી મર્યાદા નહીં \\
        \textbf{CMRR} & $\infty$ & સંપૂર્ણ નોઇઝ રિજેક્શન \\
        \textbf{સ્લ્યુ રેટ} & $\infty$ & કોઈ સ્લ્યુ રેટ લિમિટિંગ નહીં \\
        \textbf{ઇનપુટ ઓફસેટ} & 0V & કોઈ DC એરર નહીં \\
        \bottomrule
    \end{tabulary}
    
    \begin{itemize}
        \item \textbf{સંપૂર્ણ કામગીરી}: બધા પેરામીટર્સ ઓપ્ટિમાઇઝ્ડ
        \item \textbf{ડિઝાઇન સરળીકરણ}: વિશ્લેષણ સરળ બને છે
        \item \textbf{પ્રેક્ટિકલ અપ્રોક્સિમેશન}: ઘણી એપ્લીકેશનમાં આઇડીયલની નજીક
    \end{itemize}
    
    \begin{mnemonicbox}
        \mnemonic{Infinite Input, Zero Output, Perfect Performance}
    \end{mnemonicbox}
\end{solutionbox}

% Question 4(b OR) [4 marks]
\questionmarks{4}{b}{4}
\textbf{Op-ampની મદદથી સમિંગ એમ્પ્લીફાયર દોરો અને સમજાવો.}

\begin{solutionbox}
    \textbf{જવાબ}:
    સમિંગ એમ્પ્લીફાયર દરેક ઇનપુટ માટે પ્રોગ્રામેબલ ગેઈન સાથે બહુવિધ ઇનપુટ વોલ્ટેજ ઉમેરે છે.
    
    \textbf{સર્કિટ ડાયગ્રામ:}
    \begin{center}
    \begin{circuitikz}[american]
        \draw (0,0) node[op amp](opamp){}
        (opamp.+) node[ground]{}
        (opamp.-) -- (-1, 0.5) 
        (-1, 0.5) to[R, l=$R_1$] (-3, 1.5) node[left]{$V_1$}
        (-1, 0.5) to[R, l=$R_2$] (-3, 0.5) node[left]{$V_2$}
        (-1, 0.5) to[R, l=$R_3$] (-3, -0.5) node[left]{$V_3$}
        (opamp.-) -- (0, 0.5) -- (0, 1.5) to[R, l=$R_f$] (2, 1.5) -- (opamp.out)
        (opamp.out) -- (3,0) node[right]{$V_{out}$};
    \end{circuitikz}
    \end{center}
    
    \textbf{વિશ્લેષણ:}
    વર્ચ્યુઅલ ગ્રાઉન્ડ કોન્સેપ્ટનો ઉપયોગ કરીને ($V_- = 0V$):
    \begin{itemize}
        \item \textbf{R1 દ્વારા કરંટ}: $I_1 = V_1/R_1$
        \item \textbf{R2 દ્વારા કરંટ}: $I_2 = V_2/R_2$
        \item \textbf{R3 દ્વારા કરંટ}: $I_3 = V_3/R_3$
        \item \textbf{કુલ ઇનપુટ કરંટ}: $I_{in} = I_1 + I_2 + I_3$
    \end{itemize}
    
    \textbf{આઉટપુટ સમીકરણ:}
    \[ V_{out} = -R_f(V_1/R_1 + V_2/R_2 + V_3/R_3) \]
    
    \textbf{વિશેષ કેસો:}
    \begin{itemize}
        \item \textbf{સમાન રેઝિસ્ટર}: $V_{out} = -(R_f/R)(V_1 + V_2 + V_3)$
        \item \textbf{યુનિટી ગેઈન}: $R_f = R$, $V_{out} = -(V_1 + V_2 + V_3)$
    \end{itemize}
    
    \begin{mnemonicbox}
        \mnemonic{Sum inputs, Scale by resistor ratios}
    \end{mnemonicbox}
\end{solutionbox}

% Question 4(c OR) [7 marks]
\questionmarks{4}{c}{7}
\textbf{IC741ની મદદથી ડિફરેન્શિયલ એમ્પ્લીફાયર વિસ્તૃતમાં સમજાવો.}

\begin{solutionbox}
    \textbf{જવાબ}:
    ડિફરેન્શિયલ એમ્પ્લીફાયર કોમન સિગ્નલ રિજેક્ટ કરતાં બે ઇનપુટ સિગ્નલ વચ્ચેનો તફાવત એમ્પ્લિફાઇ કરે છે.
    
    \textbf{સર્કિટ ડાયગ્રામ:}
    \begin{center}
    \begin{circuitikz}[american]
        \draw (0,0) node[op amp](opamp){}
        (opamp.-) to[R, l=$R_1$] (-2, 0.5) node[left]{$V_1$}
        (opamp.+) to[R, l=$R_2$] (-2, -0.5) node[left]{$V_2$}
        (opamp.+) to[R, l=$R_3$] (0, -2) node[ground]{}
        (opamp.-) -- (0, 0.5) -- (0, 1.5) to[R, l=$R_f$] (2, 1.5) -- (opamp.out)
        (opamp.out) -- (3,0) node[right]{$V_{out}$};
    \end{circuitikz}
    \end{center}
    
    \textbf{વિશ્લેષણ:}
    નોન-ઇન્વર્ટિંગ ઇનપુટ માટે:
    \begin{itemize}
        \item $V_+ = V_2 \times \frac{R_3}{R_2+R_3}$
    \end{itemize}
    
    ઇન્વર્ટિંગ ઇનપુટ માટે વર્ચ્યુઅલ શોર્ટનો ઉપયોગ કરીને:
    \begin{itemize}
        \item $V_- = V_+ = V_2 \times \frac{R_3}{R_2+R_3}$
    \end{itemize}
    
    કરંટ બેલેન્સનો ઉપયોગ કરીને:
    \begin{itemize}
        \item $\frac{V_1-V_-}{R_1} = \frac{V_--V_{out}}{R_f}$
    \end{itemize}
    
    \textbf{આઉટપુટ સમીકરણ:}
    જ્યારે $R_1 = R_2$ અને $R_3 = R_f$:
    \[ V_{out} = \frac{R_f}{R_1}(V_2 - V_1) \]
    
    \textbf{મુખ્ય લાક્ષણિકતાઓ:}
    \begin{tabulary}{\textwidth}{L|L|L}
        \toprule
        \textbf{પેરામીટર} & \textbf{મૂલ્ય} & \textbf{ફાયદો} \\
        \midrule
        \textbf{ડિફરેન્શિયલ ગેઈન} & $R_f/R_1$ & તફાવત એમ્પ્લિફાઇ કરે \\
        \textbf{કોમન મોડ ગેઈન} & $\approx 0$ & કોમન સિગ્નલ રિજેક્ટ કરે \\
        \textbf{CMRR} & ખૂબ ઊંચું & શ્રેષ્ઠ નોઇઝ ઇમ્યુનિટી \\
        \bottomrule
    \end{tabulary}
    
    \begin{mnemonicbox}
        \mnemonic{Difference amplified, Common rejected}
    \end{mnemonicbox}
\end{solutionbox}

% Question 5(a) [3 marks]
\questionmarks{5}{a}{3}
\textbf{OP-AMPની મદદથી ઇન્ટીગ્રેટર સર્કિટ દોરો અને તેના ઇનપુટ અને આઉટપુટ વેવફોર્મ દોરો.}

\begin{solutionbox}
    \textbf{જવાબ}:
    ઓપ-એમ્પ ઇન્ટીગ્રેટર RC ફીડબેકનો ઉપયોગ કરીને ઇનપુટ સિગ્નલનું ગાણિતિક ઇન્ટીગ્રેશન કરે છે.
    
    \textbf{સર્કિટ ડાયગ્રામ:}
    \begin{center}
    \begin{circuitikz}[american]
        \draw (0,0) node[op amp](opamp){}
        (opamp.-) to[R, l=$R$] (-3, 0.5) node[left]{$V_{in}$}
        (opamp.+) node[ground]{}
        (opamp.-) -- (-0.5, 0.5) -- (-0.5, 1.5) to[C, l=$C$] (2, 1.5) -- (opamp.out)
        (opamp.out) -- (3,0) node[right]{$V_{out}$};
    \end{circuitikz}
    \end{center}
    
    \textbf{વેવફોર્મ:}
    \begin{center}
    \begin{tikzpicture}
        % Input waveform
        \begin{axis}[
            width=6cm, height=3cm,
            axis lines=middle,
            xlabel={\small $t$}, ylabel={\small $V_{in}$},
            xmin=0, xmax=4, ymin=-1.5, ymax=1.5,
            xtick=\empty, ytick=\empty,
            title={\small ઇનપુટ (સ્ક્વેર વેવ)}
        ]
            \draw[thick, blue] (0,1) -- (1,1) -- (1,-1) -- (2,-1) -- (2,1) -- (3,1) -- (3,-1) -- (4,-1);
        \end{axis}
    \end{tikzpicture}
    
    \begin{tikzpicture}
        % Output waveform
        \begin{axis}[
            width=6cm, height=3cm,
            axis lines=middle,
            xlabel={\small $t$}, ylabel={\small $V_{out}$},
            xmin=0, xmax=4, ymin=-1.5, ymax=1.5,
            xtick=\empty, ytick=\empty,
            title={\small આઉટપુટ (ત્રિકોણાકાર)}
        ]
            \draw[thick, red] (0,0) -- (1,-1) -- (2,0) -- (3,1) -- (4,0);
        \end{axis}
    \end{tikzpicture}
    \end{center}
    
    \textbf{ઓપરેશન:}
    \begin{itemize}
        \item \textbf{ઇન્ટીગ્રેશન ફંક્શન}: $V_{out} = -\frac{1}{RC}\int V_{in} dt$
        \item \textbf{સ્ક્વેર વેવ ઇનપુટ}: ત્રિકોણાકાર આઉટપુટ ઉત્પન્ન કરે છે
        \item \textbf{રેમ્પ જનરેશન}: કોન્સ્ટન્ટ ઇનપુટ લીનિયર રેમ્પ આપે છે
    \end{itemize}
    
    \begin{mnemonicbox}
        \mnemonic{ઇન્ટીગ્રેશન સ્ક્વેરમાંથી ત્રિકોણાકાર બનાવે}
    \end{mnemonicbox}
\end{solutionbox}

% Question 5(b) [4 marks]
\questionmarks{5}{b}{4}
\textbf{પુશ પુલ એરેન્જમેન્ટ પાવર એમ્પ્લીફાયરના ફાયદા તથા ગેરફાયદા જણાવો.}

\begin{solutionbox}
    \textbf{જવાબ}:
    પુશ-પુલ કન્ફિગરેશન પાવર એમ્પ્લિફિકેશન માટે કમ્પ્લીમેન્ટરી રીતે ઓપરેટ કરતા બે ટ્રાન્ઝિસ્ટરનો ઉપયોગ કરે છે.
    
    \textbf{ફાયદા:}
    \begin{center}
    \captionof{table}{પુશ-પુલ એમ્પ્લીફાયરના ફાયદા}
    \begin{tabulary}{\textwidth}{L|L|L}
        \toprule
        \textbf{ફાયદો} & \textbf{લાભ} & \textbf{એપ્લીકેશન} \\
        \midrule
        \textbf{ઊંચી કાર્યક્ષમતા} & 78.5\% સુધી & બેટરી ઓપરેટેડ \\
        \textbf{ટ્રાન્સફોર્મર નહીં} & કોમ્પેક્ટ ડિઝાઇન & પોર્ટેબલ ડિવાઇસ \\
        \textbf{ઓછી વિકૃતિ} & સારી લીનિયરિટી & ઓડિયો સિસ્ટમ \\
        \textbf{ગરમીનું વિતરણ} & ટ્રાન્ઝિસ્ટર વચ્ચે વહેંચાયેલું & થર્મલ મેનેજમેન્ટ \\
        \bottomrule
    \end{tabulary}
    \end{center}
    
    \textbf{ગેરફાયદા:}
    \begin{center}
    \captionof{table}{પુશ-પુલ એમ્પ્લીફાયરના ગેરફાયદા}
    \begin{tabulary}{\textwidth}{L|L|L}
        \toprule
        \textbf{ગેરફાયદો} & \textbf{સમસ્યા} & \textbf{ઉકેલ} \\
        \midrule
        \textbf{ક્રોસઓવર ડિસ્ટોર્શન} & શૂન્ય ક્રોસિંગ પર ડેડ ઝોન & ક્લાસ AB બાયસ \\
        \textbf{કોમ્પોનન્ટ મેચિંગ} & મેચ્ડ ટ્રાન્ઝિસ્ટરની જરૂર & કાળજીપૂર્વક પસંદગી \\
        \textbf{થર્મલ રનઅવે} & તાપમાન કોઇફિશન્ટ મિસમેચ & થર્મલ કપલિંગ \\
        \bottomrule
    \end{tabulary}
    \end{center}
    
    \textbf{એપ્લીકેશન:}
    \begin{itemize}
        \item \textbf{ઓડિયો એમ્પ્લીફાયર}: હાઇ ફિડેલિટી સિસ્ટમ
        \item \textbf{મોટર ડ્રાઇવર}: DC મોટર કંટ્રોલ
        \item \textbf{RF એમ્પ્લીફાયર}: કમ્યુનિકેશન સિસ્ટમ
    \end{itemize}
    
    \begin{mnemonicbox}
        \mnemonic{પુશ-પુલ પાવર પ્રદાન કરે પણ સમસ્યાઓ છે}
    \end{mnemonicbox}
\end{solutionbox}

% Question 5(c) [7 marks]
\questionmarks{5}{c}{7}
\textbf{555 ટાઇમર ICની મદદથી એસ્ટેબલ મલ્ટીવાઇબ્રેટર દોરો અને સમજાવો.}

\begin{solutionbox}
    \textbf{જવાબ}:
    એસ્ટેબલ મલ્ટીવાઇબ્રેટર 555 ટાઇમરનો ઉપયોગ કરીને બાહ્ય ટ્રિગર વિના સતત સ્ક્વેર વેવ આઉટપુટ ઉત્પન્ન કરે છે.
    
    \textbf{સર્કિટ ડાયગ્રામ:}
    \begin{center}
    \begin{circuitikz}
        \draw (0,0) node[ground]{} to[C, l=$C$] (0,2);
        \draw (0,2) to[R, l=$R_B$] (0,4);
        \draw (0,4) to[R, l=$R_A$] (0,6) node[vcc]{$+V_{cc}$};
        
        \draw (2,0) node{GND} rectangle (4,6) node[pos=0.5]{555};
        \draw (0,2) -- (2,2) node[left, xshift=2cm]{2,6};
        \draw (0,4) -- (2,4) node[left, xshift=2cm]{7};
        \draw (0,6) -- (2,6) node[left, xshift=2cm]{4,8};
        \draw (4,3) -- (5,3) node[right]{3: Output};
    \end{circuitikz}
    \end{center}
    
    \textbf{પિન કનેક્શન:}
    \begin{itemize}
        \item \textbf{પિન 1}: ગ્રાઉન્ડ
        \item \textbf{પિન 2}: ટ્રિગર (પિન 6 સાથે કનેક્ટેડ)
        \item \textbf{પિન 3}: આઉટપુટ
        \item \textbf{પિન 4}: રીસેટ (+Vcc)
        \item \textbf{પિન 6}: થ્રેશોલ્ડ
        \item \textbf{પિન 7}: ડિસચાર્જ
        \item \textbf{પિન 8}: +Vcc
    \end{itemize}
    
    \textbf{ઓપરેશન:}
    \begin{enumerate}
        \item \textbf{ચાર્જિંગ ફેઝ}: C એ $R_A + R_B$ દ્વારા ચાર્જ થાય છે
        \item \textbf{થ્રેશોલ્ડ પહોંચ્યું}: 2/3 Vcc પર, આઉટપુટ LOW જાય છે
        \item \textbf{ડિસચાર્જિંગ ફેઝ}: C એ $R_B$ દ્વારા ડિસચાર્જ થાય છે
        \item \textbf{ટ્રિગર પહોંચ્યું}: 1/3 Vcc પર, આઉટપુટ HIGH જાય છે
        \item \textbf{સાયકલ રિપીટ}: સતત ઓસિલેશન
    \end{enumerate}
    
    \textbf{ટાઇમિંગ સમીકરણો:}
    \begin{itemize}
        \item \textbf{HIGH સમય}: $t_1 = 0.693(R_A + R_B)C$
        \item \textbf{LOW સમય}: $t_2 = 0.693(R_B)C$
        \item \textbf{કુલ પીરિયડ}: $T = t_1 + t_2 = 0.693(R_A + 2R_B)C$
        \item \textbf{ફ્રીક્વન્સી}: $f = \frac{1.44}{(R_A + 2R_B)C}$
        \item \textbf{ડ્યુટી સાયકલ}: $D = \frac{R_A + R_B}{R_A + 2R_B} \times 100\%$
    \end{itemize}
    
    \textbf{એપ્લીકેશન:}
    \begin{itemize}
        \item \textbf{ક્લોક જનરેશન}: ડિજિટલ સિસ્ટમ
        \item \textbf{LED ફ્લેશર}: બ્લિંકિંગ સર્કિટ
        \item \textbf{ટોન જનરેશન}: ઓડિયો ઓસિલેટર
        \item \textbf{PWM જનરેશન}: મોટર સ્પીડ કંટ્રોલ
    \end{itemize}
    
    \begin{mnemonicbox}
        \mnemonic{એસ્ટેબલ હંમેશા ઓટોમેટિક ઓસિલેટ કરે}
    \end{mnemonicbox}
\end{solutionbox}

% Question 5(a OR) [3 marks]
\questionmarks{5}{a}{3}
\textbf{Op-ampનો બ્લોક ડાયગ્રામ દોરો અને તેને સમજાવો.}

\begin{solutionbox}
    \textbf{જવાબ}:
    ઓપ-એમ્પની આંતરિક રચના ઉચ્ચ ગેઈન અને કામગીરી માટે બહુવિધ સ્ટેજનો સમાવેશ કરે છે.
    
    \textbf{બ્લોક ડાયગ્રામ:}
    \begin{center}
    \begin{tikzpicture}[node distance=2cm, gtu block]
        \node (input) [gtu block] {Differential\\Amplifier};
        \node (inter) [gtu block, right of=input, xshift=1.5cm] {Intermediate\\Amplifier};
        \node (level) [gtu block, below of=inter] {Level\\Shifter};
        \node (output) [gtu block, right of=inter, xshift=1.5cm] {Output\\Stage};
        
        \node[left of=input, xshift=-1cm] (vplus) {$V_+$};
        \node[below of=vplus, yshift=1cm] (vminus) {$V_-$};
        \node[right of=output, xshift=1cm] (out) {Output};
        
        \draw[gtu flow] (vplus) -- (input);
        \draw[gtu flow] (vminus) -- (input);
        \draw[gtu flow] (input) -- (inter);
        \draw[gtu flow] (inter) -- (level);
        \draw[gtu flow] (level) -- (output);
        \draw[gtu flow] (inter) -- (output);
        \draw[gtu flow] (output) -- (out);
    \end{tikzpicture}
    \end{center}
    
    \textbf{સ્ટેજ ફંક્શન:}
    \begin{center}
    \captionof{table}{ઓપ-એમ્પ સ્ટેજ ફંક્શન}
    \begin{tabulary}{\textwidth}{L|L|L}
        \toprule
        \textbf{સ્ટેજ} & \textbf{ફંક્શન} & \textbf{લાક્ષણિકતાઓ} \\
        \midrule
        \textbf{ડિફરેન્શિયલ ઇનપુટ} & ઉચ્ચ ઇનપુટ ઇમ્પીડન્સ & નીચું ઓફસેટ, ઉચ્ચ CMRR \\
        \textbf{ઇન્ટરમીડિયેટ એમ્પ્લીફાયર} & ઉચ્ચ વોલ્ટેજ ગેઈન & મોટાભાગનું ગેઈન \\
        \textbf{લેવલ શિફ્ટર} & DC લેવલ એડજસ્ટમેન્ટ & AC સ્ટેજ કપલ કરે છે \\
        \textbf{આઉટપુટ સ્ટેજ} & નીચું આઉટપુટ ઇમ્પીડન્સ & કરંટ બફર \\
        \bottomrule
    \end{tabulary}
    \end{center}
    
    \textbf{મુખ્ય લાક્ષણિકતાઓ:}
    \begin{itemize}
        \item \textbf{ઉચ્ચ ગેઈન}: સામાન્ય રીતે 100,000 અથવા વધુ
        \item \textbf{વાઇડ બેન્ડવિડ્થ}: MHz રેન્જ ક્ષમતા
        \item \textbf{નીચું આઉટપુટ ઇમ્પીડન્સ}: વિવિધ લોડ ડ્રાઇવ કરે છે
    \end{itemize}
    
    \begin{mnemonicbox}
        \mnemonic{ડિફરેન્શિયલ ઇનપુટ, ઇન્ટરમીડિયેટ ગેઈન, લેવલ શિફ્ટ, આઉટપુટ બફર}
    \end{mnemonicbox}
\end{solutionbox}

% Question 5(b OR) [4 marks]
\questionmarks{5}{b}{4}
\textbf{પાવર એમ્પ્લીફાયરના સંદર્ભમાં પદો વિશે સમજાવો. i) કાર્યક્ષમતા ii) ડિસ્ટોર્શન.}

\begin{solutionbox}
    \textbf{જવાબ}:
    આ પેરામીટર્સ પાવર એમ્પ્લીફાયરની કામગીરી અને એપ્લીકેશન માટે યોગ્યતા નક્કી કરે છે.
    
    \textbf{i) કાર્યક્ષમતા ($\eta$):}
    \begin{itemize}
        \item \textbf{વ્યાખ્યા}: AC આઉટપુટ પાવર અને DC ઇનપુટ પાવરનો ગુણોત્તર
        \item \textbf{સૂત્ર}: $\eta = \frac{P_o(AC)}{P_{in}(DC)} \times 100\%$
        \item \textbf{મહત્વ}: ગરમી વિસર્જન અને બેટરી લાઇફ નક્કી કરે છે
    \end{itemize}
    
    \textbf{કાર્યક્ષમતા સરખામણી:}
    \begin{center}
    \captionof{table}{પાવર એમ્પ્લીફાયર ક્લાસ કાર્યક્ષમતા}
    \begin{tabulary}{\textwidth}{L|L|L}
        \toprule
        \textbf{ક્લાસ} & \textbf{કાર્યક્ષમતા} & \textbf{એપ્લીકેશન} \\
        \midrule
        \textbf{A} & 25\% & લો પાવર, હાઇ ફિડેલિટી \\
        \textbf{B} & 78.5\% & પુશ-પુલ એમ્પ્લીફાયર \\
        \textbf{AB} & 60-70\% & ઓડિયો એમ્પ્લીફાયર \\
        \textbf{C} & >90\% & RF એપ્લીકેશન \\
        \bottomrule
    \end{tabulary}
    \end{center}
    
    \textbf{ii) ડિસ્ટોર્શન:}
    \begin{itemize}
        \item \textbf{વ્યાખ્યા}: આઉટપુટ સિગ્નલ શેપમાં અનિચ્છનીય ફેરફારો
        \item \textbf{પ્રકારો}: હાર્મોનિક, ઇન્ટરમોડ્યુલેશન, ક્રોસઓવર
        \item \textbf{મેઝરમેન્ટ}: ટોટલ હાર્મોનિક ડિસ્ટોર્શન (THD)
    \end{itemize}
    
    \textbf{ડિસ્ટોર્શન સોર્સ:}
    \begin{itemize}
        \item \textbf{નોનલીનિયરિટી}: ટ્રાન્ઝિસ્ટર લાક્ષણિકતાઓ
        \item \textbf{ક્રોસઓવર}: પુશ-પુલમાં ડેડ ઝોન
        \item \textbf{થર્મલ ઇફેક્ટ}: તાપમાન વેરિયેશન
    \end{itemize}
    
    \begin{mnemonicbox}
        \mnemonic{કાર્યક્ષમતા ઊર્જા ઉપયોગ માપે, ડિસ્ટોર્શન સિગ્નલ ડિગ્રેડેશન દર્શાવે}
    \end{mnemonicbox}
\end{solutionbox}

% Question 5(c OR) [7 marks]
\questionmarks{5}{c}{7}
\textbf{555 ટાઇમર IC નો પિન ડાયગ્રામ દોરો. ઉપરાંત 555 ટાઇમર ICની મદદથી બે સ્ટેજવાળું સિક્વન્સિયલ ટાઇમર દોરો.}

\begin{solutionbox}
    \textbf{જવાબ}:
    555 ટાઇમર સ્ટાન્ડર્ડ 8-પિન પેકેજ સાથે ટાઇમિંગ એપ્લીકેશન માટે વર્સેટાઇલ IC છે.
    
    \textbf{પિન ડાયગ્રામ:}
    \begin{center}
    \begin{tikzpicture}
        \draw (0,0) rectangle (3,4);
        \node at (1.5,2) {\large 555};
        
        % Left pins
        \draw (-0.5,3.5) node[left]{1 GND} -- (0,3.5);
        \draw (-0.5,2.5) node[left]{2 Trigger} -- (0,2.5);
        \draw (-0.5,1.5) node[left]{3 Output} -- (0,1.5);
        \draw (-0.5,0.5) node[left]{4 Reset} -- (0,0.5);
        
        % Right pins
        \draw (3,3.5) -- (3.5,3.5) node[right]{8 +Vcc};
        \draw (3,2.5) -- (3.5,2.5) node[right]{7 Discharge};
        \draw (3,1.5) -- (3.5,1.5) node[right]{6 Threshold};
        \draw (3,0.5) -- (3.5,0.5) node[right]{5 Control};
    \end{tikzpicture}
    \end{center}
    
    \textbf{પિન ફંક્શન:}
    \begin{center}
    \captionof{table}{555 ટાઇમર IC પિન ફંક્શન}
    \begin{tabulary}{\textwidth}{L|L|L}
        \toprule
        \textbf{પિન} & \textbf{નામ} & \textbf{ફંક્શન} \\
        \midrule
        \textbf{1} & ગ્રાઉન્ડ & કોમન ગ્રાઉન્ડ \\
        \textbf{2} & ટ્રિગર & ટાઇમિંગ સાયકલ શરૂ કરે \\
        \textbf{3} & આઉટપુટ & ટાઇમર આઉટપુટ \\
        \textbf{4} & રીસેટ & ટાઇમર રીસેટ કરે \\
        \textbf{5} & કંટ્રોલ & વોલ્ટેજ રેફરન્સ \\
        \textbf{6} & થ્રેશોલ્ડ & ટાઇમિંગ સાયકલ બંધ કરે \\
        \textbf{7} & ડિસચાર્જ & ટાઇમિંગ કેપેસિટર ડિસચાર્જ કરે \\
        \textbf{8} & Vcc & સપ્લાય વોલ્ટેજ \\
        \bottomrule
    \end{tabulary}
    \end{center}
    
    \textbf{બે સ્ટેજ સિક્વન્સિયલ ટાઇમર સર્કિટ:}
    \begin{center}
    \begin{tikzpicture}
        % First timer
        \draw (0,2) rectangle (2,4) node[pos=0.5]{555A};
        \draw (0,3.5) -- (-0.5,3.5) node[left]{Trigger};
        \draw (2,3) -- (2.5,3) node[right]{Out1};
        \draw (0.5,4) -- (0.5,4.5) node[above]{$+V_{cc}$};
        \draw (0.5,4) to[R, l=$R_1$] (0.5,5.5);
        \draw (-1,3) to[C, l=$C_1$] (-1,1.5) node[ground]{};
        \draw (-1,3) -- (0,3);
        
        % Second timer  
        \draw (4,2) rectangle (6,4) node[pos=0.5]{555B};
        \draw (2.5,3) -- (4,3.5);
        \draw (6,3) -- (6.5,3) node[right]{Out2};
        \draw (4.5,4) -- (4.5,4.5) node[above]{$+V_{cc}$};
        \draw (4.5,4) to[R, l=$R_2$] (4.5,5.5);
        \draw (3,3) to[C, l=$C_2$] (3,1.5) node[ground]{};
        \draw (3,3) -- (4,3);
    \end{tikzpicture}
    \end{center}
    
    \textbf{ઓપરેશન:}
    \begin{enumerate}
        \item \textbf{પ્રથમ ટાઇમર}: મોનોસ્ટેબલ મોડમાં ઓપરેટ કરે છે
        \item \textbf{ટ્રિગર લાગુ}: પ્રથમ ટાઇમર આઉટપુટ પલ્સ આપે છે
        \item \textbf{આઉટપુટ અવધિ}: $T_1 = 1.1 \times R_2 \times C_1$
        \item \textbf{બીજું ટાઇમર}: પ્રથમ ટાઇમરના આઉટપુટ દ્વારા ટ્રિગર થાય છે
        \item \textbf{સિક્વન્સિયલ ઓપરેશન}: પ્રથમ પૂર્ણ થયા પછી બીજું શરૂ થાય છે
        \item \textbf{કુલ વિલંબ}: $T_1 + T_2$ જ્યાં $T_2 = 1.1 \times R_4 \times C_2$
    \end{enumerate}
    
    \textbf{એપ્લીકેશન:}
    \begin{itemize}
        \item \textbf{ડિલે સર્કિટ}: સિક્વન્સિયલ સ્વિચિંગ
        \item \textbf{ટ્રાફિક લાઇટ}: ટાઇમ્ડ સિક્વન્સ કંટ્રોલ
        \item \textbf{ઇન્ડસ્ટ્રિયલ ઓટોમેશન}: પ્રોસેસ ટાઇમિંગ
        \item \textbf{મોટર કંટ્રોલ}: સ્ટાર્ટ-સ્ટોપ સિક્વન્સ
    \end{itemize}
    
    \textbf{ટાઇમિંગ સમીકરણો:}
    \begin{itemize}
        \item \textbf{સ્ટેજ 1 વિલંબ}: $T_1 = 1.1 R_2 C_1$
        \item \textbf{સ્ટેજ 2 વિલંબ}: $T_2 = 1.1 R_4 C_2$
        \item \textbf{કુલ સિક્વન્સ સમય}: $T_{total} = T_1 + T_2$
    \end{itemize}
    
    \textbf{મુખ્ય લાક્ષણિકતાઓ:}
    \begin{itemize}
        \item \textbf{સ્વતંત્ર ટાઇમિંગ}: દરેક સ્ટેજ અલગથી એડજસ્ટેબલ
        \item \textbf{સિક્વન્સિયલ ઓપરેશન}: સ્ટેજ વચ્ચે કોઈ ઓવરલેપ નહીં
        \item \textbf{વિશ્વસનીય સ્વિચિંગ}: સ્વચ્છ ડિજિટલ ટ્રાન્ઝિશન
        \item \textbf{સરળ ડિઝાઇન}: સરળ કોમ્પોનન્ટ ગણતરી
    \end{itemize}
    
    \begin{mnemonicbox}
        \mnemonic{સિક્વન્સિયલ સ્ટેજ અલગથી શરૂ થાય}
    \end{mnemonicbox}
\end{solutionbox}

\end{document}
