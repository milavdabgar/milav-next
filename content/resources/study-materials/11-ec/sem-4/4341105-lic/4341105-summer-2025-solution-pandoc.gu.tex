\documentclass[10pt,a4paper]{article}

% content/resources/templates/preamble.tex
\usepackage[margin=0.6in]{geometry}
\author{Milav Dabgar}
\usepackage{amsmath,amssymb,amsthm}
\usepackage{booktabs}
\usepackage{multirow}
\usepackage{xcolor}
\usepackage{tcolorbox}
\tcbuselibrary{breakable,skins}
\usepackage[colorlinks=true,linkcolor=blue]{hyperref}
\usepackage{titlesec}
\usepackage{enumitem}
\usepackage{tikz}
\usepackage{pgfplots}
\usepackage{circuitikz}
\usepackage[version=4]{mhchem}
\usepackage{longtable}
\usepackage{array}
\usepackage{float}
\usepackage{caption}
\usepackage{listings}

\lstset{
  basicstyle=\small\ttfamily,
  breaklines=true,
  breakatwhitespace=false,
  postbreak=\mbox{\textcolor{red}{$\hookrightarrow$}\space},
  float=false,
  numbers=left,
  numberstyle=\tiny\color{gray},
  numbersep=10pt,
  xleftmargin=2em,
  keywordstyle=\color{blue},
  commentstyle=\color{green!60!black},
  stringstyle=\color{purple},
  backgroundcolor=\color{gray!5},
  showstringspaces=false,
  tabsize=2,
  captionpos=b,
  keepspaces=true,
  columns=flexible
}

\pgfplotsset{compat=1.18}
\usetikzlibrary{shapes,arrows,positioning,calc,patterns,decorations.pathmorphing,decorations.markings,arrows.meta}

% Color scheme
\definecolor{headcolor}{RGB}{0,102,204}
\definecolor{keycolor}{RGB}{220,20,60}
\definecolor{solutioncolor}{RGB}{34,139,34}
\definecolor{mnemoniccolor}{RGB}{148,0,211}
\definecolor{codecolor}{RGB}{0,0,100}

% Spacing
\setlength{\parskip}{3pt}
\setlist[itemize]{nosep}
\setlist[enumerate]{nosep}

% Title formatting
\titleformat{\section}{\Large\bfseries\color{headcolor}}{\thesection}{1em}{}
\titleformat{\subsection}{\large\bfseries\color{headcolor}}{\thesubsection}{1em}{}

% Pandoc tightlist compatibility
\providecommand{\tightlist}{%
  \setlength{\itemsep}{0pt}\setlength{\parskip}{0pt}}

% Pandoc longtable compatibility
\newcounter{none}
\def\thenone{}


% content/resources/templates/gujarati-boxes.tex
\usepackage{fontspec}
\usepackage{polyglossia}

% Set Gujarati as main language (document is primarily in Gujarati)
% Note: gloss-gujarati.ldf doesn't exist in polyglossia, but it will use hyphenation patterns
\setdefaultlanguage{gujarati}
\setotherlanguage{english}

% Configure Gujarati font properly
% Use Language=Default to prevent polyglossia from trying to add language-specific features
% that don't exist for Gujarati, which causes "empty feature" warnings
\newfontfamily\gujaratifont[Script=Gujarati,AutoFakeBold=2.5,AutoFakeSlant=0.3]{Noto Sans Gujarati}
\setmainfont[Script=Gujarati,AutoFakeBold=2.5,AutoFakeSlant=0.3]{Noto Sans Gujarati}
% Use Noto Sans Gujarati for monospace to support Gujarati in text
\setmonofont[Scale=0.9]{Noto Sans Gujarati}

% Configure English to use the same font
\newfontfamily\englishfont[Script=Gujarati,AutoFakeBold=2.5,AutoFakeSlant=0.3]{Noto Sans Gujarati}

% Translations for polyglossia
\gappto\captionsgujarati{
  \renewcommand{\tablename}{કોષ્ટક}
  \renewcommand{\figurename}{આકૃતિ}
}

% Helper for TikZ nodes to ensure Gujarati font
\newcommand{\gu}[1]{{\gujaratifont #1}}

% Custom environments
\newtcolorbox{solutionbox}{
    breakable,
    enhanced,
    colback=solutioncolor!5!white,
    colframe=solutioncolor!75!black,
    fonttitle=\bfseries,
    title=જવાબ
}

\newtcolorbox{solutionboxnobreak}{
 colback=solutioncolor!5!white,
 colframe=solutioncolor!75!black,
 fonttitle=\bfseries,
 title=જવાબ
}

\newtcolorbox{keyformula}{
 breakable,
 enhanced,
 colback=keycolor!5!white,
 colframe=keycolor!75!black,
 fonttitle=\bfseries,
 title=રાસાયણિક સમીકરણ/સૂત્ર
}

\newtcolorbox{mnemonicbox}{
 breakable,
 enhanced,
 colback=mnemoniccolor!5!white,
 colframe=mnemoniccolor!75!black,
 fonttitle=\bfseries,
 title=મેમરી ટ્રીક
}


\begin{document}

\begin{center}
{\Huge\bfseries\color{headcolor} Subject Name (Gujarati)}\\[5pt]
{\LARGE 4341105 -- Summer 2025}\\[3pt]
{\large Semester 1 Study Material}\\[3pt]
{\normalsize\textit{Detailed Solutions and Explanations}}
\end{center}

\vspace{10pt}

\subsection*{પ્રશ્ન ૧(અ) [૩
ગુણ]}\label{uxaaauxab0uxab6uxaa8-uxae7uxa85-uxae9-uxa97uxaa3}

\textbf{ગેઈન અને સ્ટેબિલિટી પર નેગેટિવ ફીડબેકની અસર સમજાવો.}

\begin{solutionbox}
નેગેટિવ ફીડબેક એમ્પ્લીફાયરની કામગીરીને નોંધપાત્ર રીતે સુધારે છે.

\textbf{ટેબલ:}

{\def\LTcaptype{none} % do not increment counter
\begin{longtable}[]{@{}ll@{}}
\toprule\noalign{}
પરિમાણ & નેગેટિવ ફીડબેકની અસર \\
\midrule\noalign{}
\endhead
\bottomrule\noalign{}
\endlastfoot
\textbf{ગેઈન} & એકુલ ગેઈન ઘટાડે છે \\
\textbf{સ્ટેબિલિટી} & સ્થિરતા વધારે છે \\
\textbf{બેન્ડવિડ્થ} & બેન્ડવિડ્થ વધારે છે \\
\end{longtable}
}

\begin{itemize}
\tightlist
\item
  \textbf{ગેઈન ઘટાડો}: એમ્પ્લીફાયરને વધુ અનુમાનિત બનાવે છે
\item
  \textbf{સ્થિરતા સુધારો}: ઓસિલેશન અને વિકૃતિ ઘટાડે છે
\item
  \textbf{સારું નિયંત્રણ}: સતત કામગીરી પ્રદાન કરે છે
\end{itemize}

\end{solutionbox}
\begin{mnemonicbox}
``ગેઈન ઘટે, સ્ટેબિલિટી સારી''

\end{mnemonicbox}
\subsection*{પ્રશ્ન ૧(બ) [૪
ગુણ]}\label{uxaaauxab0uxab6uxaa8-uxae7uxaac-uxaea-uxa97uxaa3}

\textbf{ફીડબેક એમ્પ્લીફાયરના જુદા જુદા પ્રકારો અને નેગેટિવ ફીડબેકના એમ્પ્લીફાયરના
ફાયદા જણાવો.}

\begin{solutionbox}
ઇનપુટ અને આઉટપુટ કનેક્શનના આધારે ચાર મૂળભૂત ફીડબેક પ્રકારો છે.

\textbf{ટેબલ:}

{\def\LTcaptype{none} % do not increment counter
\begin{longtable}[]{@{}lll@{}}
\toprule\noalign{}
પ્રકાર & ઇનપુટ કનેક્શન & આઉટપુટ કનેક્શન \\
\midrule\noalign{}
\endhead
\bottomrule\noalign{}
\endlastfoot
\textbf{વોલ્ટેજ સીરીઝ} & સીરીઝ & વોલ્ટેજ \\
\textbf{વોલ્ટેજ શન્ટ} & શન્ટ & વોલ્ટેજ \\
\textbf{કરંટ સીરીઝ} & સીરીઝ & કરંટ \\
\textbf{કરંટ શન્ટ} & શન્ટ & કરંટ \\
\end{longtable}
}

\textbf{ફાયદા:}

\begin{itemize}
\tightlist
\item
  \textbf{વિકૃતિ ઘટાડો}: હાર્મોનિક કન્ટેન્ટ ઘટાડે છે
\item
  \textbf{બેન્ડવિડ્થ વૃદ્ધિ}: સારી ફ્રીક્વન્સી રિસ્પોન્સ
\item
  \textbf{સુધારેલી સ્થિરતા}: સતત ઓપરેશન
\end{itemize}

\end{solutionbox}
\begin{mnemonicbox}
``ખૂબ સ્માર્ટ કરંટ કંટ્રોલ''

\end{mnemonicbox}
\subsection*{પ્રશ્ન ૧(ક) [૭
ગુણ]}\label{uxaaauxab0uxab6uxaa8-uxae7uxa95-uxaed-uxa97uxaa3}

\textbf{નેગેટીવ ફીડબેક વોલ્ટેજ એમ્પ્લીફાયરનું ઓવરઓલ ગેઈનનું સૂત્ર મેળવો.}

\begin{solutionbox}
નેગેટિવ ફીડબેક એમ્પ્લીફાયરમાં આઉટપુટ ઇનપુટમાં વિપરીત ફેઝમાં ફીડ થાય
છે.

\textbf{ડાયગ્રામ:}

\begin{center}
\textbf{Mermaid Diagram (Code)}
\begin{verbatim}
{Shaded}
{Highlighting}[]
graph LR
    A[Input Vi] {-{-}{} B[Amplifier A]}
    B {-{-}{} C[Output Vo]}
    C {-{-}{} D[Feedback β]}
    D {-{-}{} E[Summing Junction]}
    A {-{-}{} E}
{Highlighting}
{Shaded}
\end{verbatim}
\end{center}

\textbf{વ્યુત્પત્તિ:}

\begin{itemize}
\tightlist
\item
  એમ્પ્લીફાયરનું ઇનપુટ: Vi - βVo
\item
  આઉટપુટ: Vo = A(Vi - βVo)
\item
  Vo = AVi - AβVo
\item
  Vo + AβVo = AVi
\item
  Vo(1 + Aβ) = AVi
\item
  \textbf{એકુલ ગેઈન: Af = A/(1 + Aβ)}
\end{itemize}

\textbf{મુખ્ય મુદ્દા:}

\begin{itemize}
\tightlist
\item
  \textbf{હર (1 + Aβ)}: લૂપ ગેઈન કહેવાય છે
\item
  \textbf{સ્થિરતા ફેક્ટર}: સિસ્ટમ રિસ્પોન્સ નક્કી કરે છે
\item
  \textbf{ગેઈન ઘટાડો}: સારી કામગીરી માટે ગેઈન આપવામાં આવે છે
\end{itemize}

\end{solutionbox}
\begin{mnemonicbox}
``હંમેશા (1 + લૂપ) થી ભાગો''

\end{mnemonicbox}
\subsection*{પ્રશ્ન ૧(ક અથવા) [૭
ગુણ]}\label{uxaaauxab0uxab6uxaa8-uxae7uxa95-uxa85uxaa5uxab5-uxaed-uxa97uxaa3}

\textbf{કરંટ શન્ટ પ્રકારના નેગેટીવ ફીડબેક એમ્પ્લીફાયર દોરો અને સમજાવો અને તેના ઇનપુટ
અને આઉટપુટ ઇમ્પીડન્સના સૂત્ર મેળવો.}

\begin{solutionbox}
કરંટ શન્ટ ફીડબેક આઉટપુટ કરંટ સેમ્પલ કરે છે અને ઇનપુટ સાથે શન્ટમાં વોલ્ટેજ
ફીડ કરે છે.

\textbf{સર્કિટ ડાયગ્રામ:}

\begin{center}
\textbf{Mermaid Diagram (Code)}
\begin{verbatim}
{Shaded}
{Highlighting}[]
graph LR
    A[Vi] {-{-}{} B["{}+"]}
    B {-{-}{} C[Amplifier A]}
    C {-{-}{} D[Ro]}
    D {-{-}{} E[RL]}
    D {-{-}{} F[Feedback Network β]}
    F {-{-}{} G["{}{-}"]}
    B {-{-}{} G}
{Highlighting}
{Shaded}
\end{verbatim}
\end{center}

\textbf{વિશ્લેષણ:}

\begin{itemize}
\tightlist
\item
  \textbf{ફીડબેક પ્રકાર}: કરંટ સેમ્પલિંગ, વોલ્ટેજ મિક્સિંગ
\item
  \textbf{ઇનપુટ ઇમ્પીડન્સ}: શન્ટ ફીડબેકને કારણે ઘટે છે
\item
  \textbf{આઉટપુટ ઇમ્પીડન્સ}: કરંટ સેમ્પલિંગને કારણે ઘટે છે
\end{itemize}

\textbf{સૂત્રો:}

\begin{itemize}
\tightlist
\item
  \textbf{ઇનપુટ ઇમ્પીડન્સ: Zif = Zi/(1 + Aβ)}
\item
  \textbf{આઉટપુટ ઇમ્પીડન્સ: Zof = Zo/(1 + Aβ)}
\end{itemize}

\textbf{લાક્ષણિકતાઓ:}

\begin{itemize}
\tightlist
\item
  \textbf{નીચું ઇનપુટ ઇમ્પીડન્સ}: કરંટ સોર્સ માટે સારું
\item
  \textbf{નીચું આઉટપુટ ઇમ્પીડન્સ}: વોલ્ટેજ આઉટપુટ માટે સારું
\item
  \textbf{કરંટ-ટુ-વોલ્ટેજ કન્વર્ટર}: એપ્લીકેશનમાં ઉપયોગી
\end{itemize}

\end{solutionbox}
\begin{mnemonicbox}
``કરંટ શન્ટ બંને ઇમ્પીડન્સ ઘટાડે''

\end{mnemonicbox}
\subsection*{પ્રશ્ન ૨(અ) [૩
ગુણ]}\label{uxaaauxab0uxab6uxaa8-uxae8uxa85-uxae9-uxa97uxaa3}

\textbf{ઓસિલેટર માટે બારખૌસન ક્રાઈટેરીઆ સમજાવો.}

\begin{solutionbox}
ફીડબેક સર્કિટમાં સતત ઓસિલેશન માટે બે શરતો એક સાથે પૂરી થવી જોઈએ.

\textbf{ટેબલ:}

{\def\LTcaptype{none} % do not increment counter
\begin{longtable}[]{@{}lll@{}}
\toprule\noalign{}
ક્રાઈટેરીઆ & શરત & વર્ણન \\
\midrule\noalign{}
\endhead
\bottomrule\noalign{}
\endlastfoot
\textbf{મેગ્નિટ્યુડ} & \textbar Aβ\textbar{} = 1 & લૂપ ગેઈન એકમ \\
\textbf{ફેઝ} & ∠Aβ = 0^\circ અથવા 360^\circ & શૂન્ય ફેઝ શિફ્ટ \\
\end{longtable}
}

\begin{itemize}
\tightlist
\item
  \textbf{એકમ લૂપ ગેઈન}: સિગ્નલ એમ્પ્લિટ્યુડ જાળવે છે
\item
  \textbf{શૂન્ય ફેઝ શિફ્ટ}: પોઝીટીવ ફીડબેક સુનિશ્ચિત કરે છે
\item
  \textbf{સતત ઓસિલેશન}: બંને શરતો સ્વ-ટકાઉ સિગ્નલ બનાવે છે
\end{itemize}

\end{solutionbox}
\begin{mnemonicbox}
``એક મેગ્નિટ્યુડ, શૂન્ય ફેઝ''

\end{mnemonicbox}
\subsection*{પ્રશ્ન ૨(બ) [૪
ગુણ]}\label{uxaaauxab0uxab6uxaa8-uxae8uxaac-uxaea-uxa97uxaa3}

\textbf{સ્વચ્છ ડાયગ્રામની મદદથી ટેન્ક સર્કિટ સમજાવો.}

\begin{solutionbox}
ટેન્ક સર્કિટ ઓસિલેટર સર્કિટ માટે ફ્રીક્વન્સી સિલેક્ટિવ પોઝીટીવ ફીડબેક
પ્રદાન કરે છે.

\textbf{સર્કિટ ડાયગ્રામ:}

\begin{verbatim}
    +{-{-}{-}L{-}{-}{-}+}
    |       |
    C       R
    |       |
    +{-{-}{-}{-}{-}{-}{-}+}
\end{verbatim}

\textbf{ઓપરેશન:} રેઝોનન્ટ ફ્રીક્વન્સી પર, LC ટેન્ક સર્કિટ દર્શાવે છે:

\textbf{ટેબલ:}

{\def\LTcaptype{none} % do not increment counter
\begin{longtable}[]{@{}lll@{}}
\toprule\noalign{}
પેરામીટર & મૂલ્ય & અસર \\
\midrule\noalign{}
\endhead
\bottomrule\noalign{}
\endlastfoot
\textbf{રીએક્ટન્સ} & XL = XC & રેઝોનન્સ \\
\textbf{ઇમ્પીડન્સ} & મહત્તમ & ઉચ્ચ સિલેક્ટિવિટી \\
\textbf{ફેઝ} & 0^\circ & એકમ ફીડબેક \\
\end{longtable}
}

\begin{itemize}
\tightlist
\item
  \textbf{ઊર્જા સંગ્રહ}: L અને C ઊર્જાની આપ-લે કરે છે
\item
  \textbf{ફ્રીક્વન્સી પસંદગી}: તીક્ષ્ણ રેઝોનન્સ લાક્ષણિકતા
\item
  \textbf{ઓસિલેશન ટકાવી રાખવું}: પોઝીટીવ ફીડબેક પ્રદાન કરે છે
\end{itemize}

\end{solutionbox}
\begin{mnemonicbox}
``ટેન્ક ઊર્જા સંગ્રહે, ફ્રીક્વન્સી પસંદ કરે''

\end{mnemonicbox}
\subsection*{પ્રશ્ન ૨(ક) [૭
ગુણ]}\label{uxaaauxab0uxab6uxaa8-uxae8uxa95-uxaed-uxa97uxaa3}

\textbf{હાર્ટલી ઓસિલેટર દોરો અને સમજાવો. ઉપરાંત હાર્ટલી ઓસિલેટરની ઓસિલેશનની
ફ્રીક્વન્સીનું સૂત્ર જણાવો.}

\begin{solutionbox}
હાર્ટલી ઓસિલેટર ફ્રીક્વન્સી જનરેશન માટે ટેન્ક સર્કિટમાં ટેપ્ડ ઇન્ડક્ટરનો
ઉપયોગ કરે છે.

\textbf{સર્કિટ ડાયગ્રામ:}

\begin{center}
\textbf{Mermaid Diagram (Code)}
\begin{verbatim}
{Shaded}
{Highlighting}[]
graph LR
    A[Vcc] {-{-}{} B[RFC]}
    B {-{-}{} C[Collector]}
    C {-{-}{} D[L1]}
    D {-{-}{} E[L2]}
    E {-{-}{} F[Emitter]}
    D {-{-}{} G[C]}
    G {-{-}{} E}
    C {-{-}{} H[Output]}
{Highlighting}
{Shaded}
\end{verbatim}
\end{center}

\textbf{ઓપરેશન:}

\begin{itemize}
\tightlist
\item
  \textbf{ટેપ્ડ ઇન્ડક્ટર}: L1 અને L2 ફીડબેક પ્રદાન કરે છે
\item
  \textbf{ટેન્ક સર્કિટ}: L1+L2 સાથે C ફ્રીક્વન્સી નક્કી કરે છે
\item
  \textbf{પોઝીટીવ ફીડબેક}: L1-L2 કપલિંગ દ્વારા ફેઝ શિફ્ટ
\end{itemize}

\textbf{ફ્રીક્વન્સી સૂત્ર:} \textbf{f = 1/[2π\sqrt((L1+L2)C)]}

\textbf{મુખ્ય લાક્ષણિકતાઓ:}

\begin{itemize}
\tightlist
\item
  \textbf{સારી ફ્રીક્વન્સી સ્થિરતા}: ઇન્ડક્ટર-આધારિત ટ્યુનિંગ
\item
  \textbf{સરળ ટ્યુનિંગ}: વેરિયેબલ ઇન્ડક્ટર અથવા કેપેસિટર
\item
  \textbf{RF એપ્લીકેશન}: ઉચ્ચ ફ્રીક્વન્સી માટે યોગ્ય
\end{itemize}

\end{solutionbox}
\begin{mnemonicbox}
``હાર્ટલીમાં ટેપ્ડ ઇન્ડક્ટર હોય છે''

\end{mnemonicbox}
\subsection*{પ્રશ્ન ૨(અ અથવા) [૩
ગુણ]}\label{uxaaauxab0uxab6uxaa8-uxae8uxa85-uxa85uxaa5uxab5-uxae9-uxa97uxaa3}

\textbf{ઓસિલેટરના પદને પોઝીટીવ ફીડબેક એમ્પ્લીફાયર તરીકે સમજાવો.}

\begin{solutionbox}
ઓસિલેટર બાહ્ય ઇનપુટ સિગ્નલ વિના પોઝીટીવ ફીડબેકનો ઉપયોગ કરીને AC
સિગ્નલ ઉત્પન્ન કરે છે.

\textbf{ટેબલ:}

{\def\LTcaptype{none} % do not increment counter
\begin{longtable}[]{@{}lll@{}}
\toprule\noalign{}
પેરામીટર & એમ્પ્લીફાયર & ઓસિલેટર \\
\midrule\noalign{}
\endhead
\bottomrule\noalign{}
\endlastfoot
\textbf{ઇનપુટ} & બાહ્ય સિગ્નલ & બાહ્ય ઇનપુટ નહીં \\
\textbf{ફીડબેક} & નેગેટિવ ઉપયોગ કરી શકે & પોઝીટીવ ઉપયોગ કરે \\
\textbf{આઉટપુટ} & એમ્પ્લિફાઇડ ઇનપુટ & સ્વ-ઉત્પન્ન AC \\
\end{longtable}
}

\begin{itemize}
\tightlist
\item
  \textbf{સ્વ-ટકાઉ}: પોઝીટીવ ફીડબેક ઓસિલેશન જાળવે છે
\item
  \textbf{બારખૌસન ક્રાઈટેરીઆ}: લૂપ ગેઈન = 1, ફેઝ = 0^\circ
\item
  \textbf{સિગ્નલ જનરેશન}: DC સપ્લાયમાંથી AC બનાવે છે
\end{itemize}

\end{solutionbox}
\begin{mnemonicbox}
``પોઝીટીવ ફીડબેક સતત સિગ્નલ ચલાવે''

\end{mnemonicbox}
\subsection*{પ્રશ્ન ૨(બ અથવા) [૪
ગુણ]}\label{uxaaauxab0uxab6uxaa8-uxae8uxaac-uxa85uxaa5uxab5-uxaea-uxa97uxaa3}

\textbf{ક્રિસ્ટલ ઓસિલેટર દોરો અને સમજાવો.}

\begin{solutionbox}
ક્રિસ્ટલ ઓસિલેટર ઉચ્ચ સ્થિરતા માટે ક્વાર્ટ્ઝ ક્રિસ્ટલના પીઝોઇલેક્ટ્રિક
ઇફેક્ટનો ઉપયોગ કરે છે.

\textbf{સર્કિટ ડાયગ્રામ:}

\begin{verbatim}
         Vcc
          |
          R
          |
    +{-{-}{-}{-}{-}+{-}{-}{-}{-}{-}+}
    |           |
    |     Q     |
    |           |
    +{-{-}{-}{-}{-}+{-}{-}{-}{-}{-}+}
          |
        XTAL
          |
         GND
\end{verbatim}

\textbf{લાક્ષણિકતાઓ:}

\textbf{ટેબલ:}

{\def\LTcaptype{none} % do not increment counter
\begin{longtable}[]{@{}lll@{}}
\toprule\noalign{}
ગુણધર્મ & મૂલ્ય & ફાયદો \\
\midrule\noalign{}
\endhead
\bottomrule\noalign{}
\endlastfoot
\textbf{સ્થિરતા} & \pm0.01\% & ખૂબ ઉચ્ચી \\
\textbf{Q ફેક્ટર} & \textgreater10,000 & તીક્ષ્ણ રેઝોનન્સ \\
\textbf{તાપમાન} & નીચું ડ્રિફ્ટ & સ્થિર ફ્રીક્વન્સી \\
\end{longtable}
}

\begin{itemize}
\tightlist
\item
  \textbf{પીઝોઇલેક્ટ્રિક ઇફેક્ટ}: મિકેનિકલ વાઇબ્રેશન ઇલેક્ટ્રિકલ સિગ્નલ બનાવે છે
\item
  \textbf{ઉચ્ચ Q}: ખૂબ સ્થિર ફ્રીક્વન્સી જનરેશન
\item
  \textbf{ક્લોક એપ્લીકેશન}: ડિજિટલ સિસ્ટમમાં ઉપયોગ
\end{itemize}

\end{solutionbox}
\begin{mnemonicbox}
``ક્રિસ્ટલ સતત ફ્રીક્વન્સી બનાવે''

\end{mnemonicbox}
\subsection*{પ્રશ્ન ૨(ક અથવા) [૭
ગુણ]}\label{uxaaauxab0uxab6uxaa8-uxae8uxa95-uxa85uxaa5uxab5-uxaed-uxa97uxaa3}

\textbf{UJTની રચના, સિમ્બોલ તથા ઇક્વિવેલેન્ટ સર્કિટ દોરો અને તેને વિસ્તૃતમાં
સમજાવો.}

\begin{solutionbox}
UJT (Unijunction Transistor) અનોખી સ્વિચિંગ લાક્ષણિકતાઓ
ધરાવતું ત્રણ-ટર્મિનલ ડિવાઇસ છે.

\textbf{રચના:}

\begin{verbatim}
    B2 +{-{-}{-}{-}{-}{-}{-}+}
       |   n   |
       |       |
    E  +   p   +
       |       |
       |   n   |
    B1 +{-{-}{-}{-}{-}{-}{-}+}
\end{verbatim}

\textbf{સિમ્બોલ:}

\begin{verbatim}
    B2
     |
     +
    /|
   / |
  /  +{-{-}{-} E}
     |
     +
     |
    B1
\end{verbatim}

\textbf{ઇક્વિવેલેન્ટ સર્કિટ:}

\begin{verbatim}
    B2 +{-{-}{-}R2{-}{-}{-}+}
               |
    E  +{-{-}{-}{-}{-}{-}{-}+}
               |
    B1 +{-{-}{-}R1{-}{-}{-}+}
\end{verbatim}

\textbf{ઓપરેશન:}

\begin{itemize}
\tightlist
\item
  \textbf{ઇન્ટ્રિન્સિક સ્ટેન્ડઓફ રેશિયો}: η = R1/(R1+R2)
\item
  \textbf{પીક પોઇન્ટ વોલ્ટેજ}: VP = ηVBB + VD
\item
  \textbf{નેગેટિવ રેઝિસ્ટન્સ}: પીક પોઇન્ટ પછી
\end{itemize}

\textbf{એપ્લીકેશન:}

\begin{itemize}
\tightlist
\item
  \textbf{રિલેક્સેશન ઓસિલેટર}: સોટૂથ વેવ જનરેશન
\item
  \textbf{ટ્રિગર સર્કિટ}: SCR ફાયરિંગ સર્કિટ
\item
  \textbf{ટાઇમિંગ એપ્લીકેશન}: RC ચાર્જિંગ સર્કિટ
\end{itemize}

\end{solutionbox}
\begin{mnemonicbox}
``UJT અનોખી જંક્શન ટેકનોલોજી વાપરે''

\end{mnemonicbox}
\subsection*{પ્રશ્ન ૩(અ) [૩
ગુણ]}\label{uxaaauxab0uxab6uxaa8-uxae9uxa85-uxae9-uxa97uxaa3}

\textbf{ઓપરેટિંગ પોઇન્ટના આધારે પાવર એમ્પ્લીફાયરને વર્ગીકૃત કરો.}

\begin{solutionbox}
પાવર એમ્પ્લીફાયર ટ્રાન્ઝિસ્ટર કન્ડક્શન એંગલ અને બાયસ પોઇન્ટના આધારે
વર્ગીકૃત થાય છે.

\textbf{ટેબલ:}

{\def\LTcaptype{none} % do not increment counter
\begin{longtable}[]{@{}llll@{}}
\toprule\noalign{}
ક્લાસ & કન્ડક્શન એંગલ & કાર્યક્ષમતા & એપ્લીકેશન \\
\midrule\noalign{}
\endhead
\bottomrule\noalign{}
\endlastfoot
\textbf{ક્લાસ A} & 360^\circ & 25-50\% & ઓડિયો, લો પાવર \\
\textbf{ક્લાસ B} & 180^\circ & 78.5\% & પુશ-પુલ \\
\textbf{ક્લાસ AB} & 180^\circ-360^\circ & 60-70\% & ઓડિયો પાવર \\
\textbf{ક્લાસ C} & \textless180^\circ & \textgreater90\% & RF, ટ્યુન્ડ \\
\end{longtable}
}

\begin{itemize}
\tightlist
\item
  \textbf{બાયસ પોઇન્ટ}: ઓપરેટિંગ ક્લાસ નક્કી કરે છે
\item
  \textbf{કાર્યક્ષમતા ટ્રેડ-ઓફ}: ઉચ્ચ કાર્યક્ષમતા, વધુ વિકૃતિ
\item
  \textbf{એપ્લીકેશન સ્પેસિફિક}: જરૂરિયાત પ્રમાણે પસંદગી
\end{itemize}

\end{solutionbox}
\begin{mnemonicbox}
``બધા મોટા એમ્પ્લીફાયર પાવર આપી શકે''

\end{mnemonicbox}
\subsection*{પ્રશ્ન ૩(બ) [૪
ગુણ]}\label{uxaaauxab0uxab6uxaa8-uxae9uxaac-uxaea-uxa97uxaa3}

\textbf{કોમ્પ્લીમેંટરી સિમેટ્રી પુશ પુલ પાવર એમ્પ્લીફાયરને દોરો અને સમજાવો.}

\begin{solutionbox}
સેન્ટર-ટેપ્ડ ટ્રાન્સફોર્મર વિના કાર્યક્ષમ પાવર એમ્પ્લિફિકેશન માટે NPN
અને PNP ટ્રાન્ઝિસ્ટરનો ઉપયોગ કરે છે.

\textbf{સર્કિટ ડાયગ્રામ:}

\begin{center}
\textbf{Mermaid Diagram (Code)}
\begin{verbatim}
{Shaded}
{Highlighting}[]
graph LR
    A[+Vcc] {-{-}{} B[NPN Q1]}
    B {-{-}{} C[Output]}
    C {-{-}{} D[RL]}
    D {-{-}{} E[PNP Q2]}
    E {-{-}{} F[{-}Vcc]}
    G[Input] {-{-}{} B}
    G {-{-}{} E}
{Highlighting}
{Shaded}
\end{verbatim}
\end{center}

\textbf{ઓપરેશન:}

\begin{itemize}
\tightlist
\item
  \textbf{પોઝીટીવ હાફ-સાયકલ}: NPN કન્ડક્ટ કરે, PNP બંધ
\item
  \textbf{નેગેટિવ હાફ-સાયકલ}: PNP કન્ડક્ટ કરે, NPN બંધ
\item
  \textbf{કોમ્પ્લીમેંટરી એક્શન}: બંને ટ્રાન્ઝિસ્ટર વૈકલ્પિક હાફ-સાયકલ હેન્ડલ કરે
\end{itemize}

\textbf{ફાયદા:}

\begin{itemize}
\tightlist
\item
  \textbf{ટ્રાન્સફોર્મર નહીં}: ડાયરેક્ટ કપલિંગ ટુ લોડ
\item
  \textbf{ઉચ્ચ કાર્યક્ષમતા}: ક્લાસ B ઓપરેશન
\item
  \textbf{કોમ્પેક્ટ ડિઝાઇન}: ઓછા કોમ્પોનન્ટ્સ
\item
  \textbf{સારું પાવર ટ્રાન્સફર}: ડાયરેક્ટ કપલિંગ
\end{itemize}

\end{solutionbox}
\begin{mnemonicbox}
``કોમ્પ્લીમેંટરી ટ્રાન્ઝિસ્ટર સાયકલ પૂરું કરે''

\end{mnemonicbox}
\subsection*{પ્રશ્ન ૩(ક) [૭
ગુણ]}\label{uxaaauxab0uxab6uxaa8-uxae9uxa95-uxaed-uxa97uxaa3}

\textbf{ક્લાસ-B પુશ પુલ એમ્પ્લીફાયરની કાર્યક્ષમતાનું સૂત્ર મેળવો.}

\begin{solutionbox}
ક્લાસ B પુશ-પુલ એમ્પ્લીફાયરમાં દરેક ટ્રાન્ઝિસ્ટર ઇનપુટ સાયકલના 180^\circ
માટે કન્ડક્ટ કરે છે.

\textbf{વિશ્લેષણ:} સાઇનુસોઇડલ ઇનપુટ માટે: Vi = Vm sin ωt

\textbf{આઉટપુટ પાવર:}

\begin{itemize}
\tightlist
\item
  પીક આઉટપુટ વોલ્ટેજ: Vom = Vcc
\item
  RMS આઉટપુટ વોલ્ટેજ: Vo(rms) = Vcc/\sqrt2
\item
  \textbf{Po = Vo^{2}(rms)/RL = Vcc^{2}/2RL}
\end{itemize}

\textbf{ઇનપુટ પાવર:}

\begin{itemize}
\tightlist
\item
  DC કરંટ (એવરેજ): Idc = 2Im/π
\item
  જ્યાં Im = Vcc/RL
\item
  \textbf{Pin = Vcc \times Idc = 2VccIm/π = 2Vcc^{2}/πRL}
\end{itemize}

\textbf{કાર્યક્ષમતા ગણતરી:} \textbf{η = Po/Pin = (Vcc^{2}/2RL)/(2Vcc^{2}/πRL)}
\textbf{η = π/4 = 0.785 = 78.5\%}

\textbf{મુખ્ય મુદ્દા:}

\begin{itemize}
\tightlist
\item
  \textbf{મહત્તમ સૈદ્ધાંતિક કાર્યક્ષમતા}: 78.5\%
\item
  \textbf{ક્લાસ B ફાયદો}: ક્લાસ A (25\%) કરતાં ખૂબ ઊંચી
\item
  \textbf{પ્રેક્ટિકલ કાર્યક્ષમતા}: નુકસાનને કારણે થોડી ઓછી
\end{itemize}

\end{solutionbox}
\begin{mnemonicbox}
``પુશ-પુલ π/4 કાર્યક્ષમતા આપે''

\end{mnemonicbox}
\subsection*{પ્રશ્ન ૩(અ અથવા) [૩
ગુણ]}\label{uxaaauxab0uxab6uxaa8-uxae9uxa85-uxa85uxaa5uxab5-uxae9-uxa97uxaa3}

\textbf{વોલ્ટેજ અને પાવર એમ્પ્લીફાયર વચ્ચેનો તફાવત કરો.}

\begin{solutionbox}
વોલ્ટેજ અને પાવર એમ્પ્લીફાયર ઇલેક્ટ્રોનિક સિસ્ટમમાં જુદા હેતુઓ સેવે છે.

\textbf{ટેબલ:}

{\def\LTcaptype{none} % do not increment counter
\begin{longtable}[]{@{}lll@{}}
\toprule\noalign{}
પેરામીટર & વોલ્ટેજ એમ્પ્લીફાયર & પાવર એમ્પ્લીફાયર \\
\midrule\noalign{}
\endhead
\bottomrule\noalign{}
\endlastfoot
\textbf{હેતુ} & વોલ્ટેજ વધારવું & પાવર વધારવું \\
\textbf{લોડ} & ઉચ્ચ ઇમ્પીડન્સ & નીચું ઇમ્પીડન્સ \\
\textbf{કાર્યક્ષમતા} & મહત્વપૂર્ણ નથી & ખૂબ મહત્વપૂર્ણ \\
\textbf{વિકૃતિ} & ઓછી હોવી જોઈએ & મધ્યમ સ્વીકાર્ય \\
\textbf{કપલિંગ} & RC/ડાયરેક્ટ & ટ્રાન્સફોર્મર \\
\end{longtable}
}

\begin{itemize}
\tightlist
\item
  \textbf{ડિઝાઇન પ્રાથમિકતા}: વોલ્ટેજ ગેઈન વર્સીસ પાવર ડિલિવરી
\item
  \textbf{એપ્લીકેશન}: સિગ્નલ પ્રોસેસિંગ વર્સીસ લોડ ડ્રાઇવિંગ
\item
  \textbf{સર્કિટ જટિલતા}: સરળ વર્સીસ જટિલ પાવર સ્ટેજ
\end{itemize}

\end{solutionbox}
\begin{mnemonicbox}
``વોલ્ટેજ સિગ્નલ વધારે, પાવર લોડ ચલાવે''

\end{mnemonicbox}
\subsection*{પ્રશ્ન ૩(બ અથવા) [૪
ગુણ]}\label{uxaaauxab0uxab6uxaa8-uxae9uxaac-uxa85uxaa5uxab5-uxaea-uxa97uxaa3}

\textbf{ક્લાસ AB પાવર એમ્પ્લીફાયર ડાયગ્રામ સાથે સમજાવો.}

\begin{solutionbox}
ક્લાસ AB ક્લાસ A અને ક્લાસ B વચ્ચે ઓપરેટ કરે છે, ક્રોસઓવર ડિસ્ટોર્શન
ઘટાડે છે.

\textbf{સર્કિટ ડાયગ્રામ:}

\begin{verbatim}
    +Vcc
      |
      R
      |
    +{-+{-}+}
    |   |
   Q1  Q2  
    |   |
    +{-{-}{-}+{-}{-}{-} Output}
    |   |
   D1  D2
    |   |
    +{-{-}{-}+}
      |
      R
      |
    {-Vcc}
\end{verbatim}

\textbf{ઓપરેશન:}

\begin{itemize}
\tightlist
\item
  \textbf{થોડું ફોરવર્ડ બાયસ}: બંને ટ્રાન્ઝિસ્ટર થોડા ઓન
\item
  \textbf{કન્ડક્શન એંગલ}: \textgreater180^\circ પણ \textless360^\circ
\item
  \textbf{ઓવરલેપ કન્ડક્શન}: ક્રોસઓવર ડિસ્ટોર્શન દૂર કરે છે
\end{itemize}

\textbf{લાક્ષણિકતાઓ:}

\textbf{ટેબલ:}

{\def\LTcaptype{none} % do not increment counter
\begin{longtable}[]{@{}lll@{}}
\toprule\noalign{}
પેરામીટર & મૂલ્ય & ફાયદો \\
\midrule\noalign{}
\endhead
\bottomrule\noalign{}
\endlastfoot
\textbf{કાર્યક્ષમતા} & 60-70\% & ક્લાસ A કરતાં સારી \\
\textbf{વિકૃતિ} & ઓછી & ક્લાસ B કરતાં સારી \\
\textbf{બાયસ} & થોડું ફોરવર્ડ & સમાધાનકારી ઉકેલ \\
\end{longtable}
}

\end{solutionbox}
\begin{mnemonicbox}
``AB ખરાબ ક્રોસઓવર ડિસ્ટોર્શન ટાળે''

\end{mnemonicbox}
\subsection*{પ્રશ્ન ૩(ક અથવા) [૭
ગુણ]}\label{uxaaauxab0uxab6uxaa8-uxae9uxa95-uxa85uxaa5uxab5-uxaed-uxa97uxaa3}

\textbf{સીરીજ ફેડ ક્લાસ-A પાવર એમ્પ્લીફાયરની કાર્યક્ષમતાનું સૂત્ર મેળવો.}

\begin{solutionbox}
સીરીજ ફેડ ક્લાસ A એમ્પ્લીફાયરમાં DC સપ્લાય લોડ સાથે સીરીજમાં
જોડાયેલું હોય છે.

\textbf{સર્કિટ વિશ્લેષણ:}

\begin{itemize}
\tightlist
\item
  \textbf{DC સપ્લાય વોલ્ટેજ}: Vcc
\item
  \textbf{ક્વિસન્ટ કરંટ}: Icq = Vcc/2RL (મહત્તમ પાવર માટે)
\item
  \textbf{ક્વિસન્ટ વોલ્ટેજ}: Vceq = Vcc/2
\end{itemize}

\textbf{AC વિશ્લેષણ:}

\begin{itemize}
\tightlist
\item
  \textbf{મહત્તમ આઉટપુટ વોલ્ટેજ સ્વિંગ}: Vom = Vcc/2
\item
  \textbf{આઉટપુટ પાવર}: Po = Vom^{2}/2RL = Vcc^{2}/8RL
\end{itemize}

\textbf{DC પાવર:}

\begin{itemize}
\tightlist
\item
  \textbf{DC કરંટ}: Idc = Icq = Vcc/2RL
\item
  \textbf{ઇનપુટ પાવર}: Pin = Vcc \times Idc = Vcc^{2}/2RL
\end{itemize}

\textbf{કાર્યક્ષમતા:} \textbf{η = Po/Pin = (Vcc^{2}/8RL)/(Vcc^{2}/2RL)}
\textbf{η = 1/4 = 0.25 = 25\%}

\textbf{મુખ્ય મુદ્દા:}

\begin{itemize}
\tightlist
\item
  \textbf{મહત્તમ સૈદ્ધાંતિક કાર્યક્ષમતા}: 25\%
\item
  \textbf{પાવર બર્બાદી}: 75\% ગરમીમાં ખોવાય છે
\item
  \textbf{ડિઝાઇન મર્યાદા}: નબળી કાર્યક્ષમતા પણ સારી લીનિયરિટી
\end{itemize}

\end{solutionbox}
\begin{mnemonicbox}
``ક્લાસ A ક્વાર્ટર કાર્યક્ષમતા મેળવે''

\end{mnemonicbox}
\subsection*{પ્રશ્ન ૪(અ) [૩
ગુણ]}\label{uxaaauxab0uxab6uxaa8-uxaeauxa85-uxae9-uxa97uxaa3}

\textbf{IC 741 OP-AMPનો પિન ડાયગ્રામ દોરો અને સમજાવો.}

\begin{solutionbox}
IC 741 ઇન્ડસ્ટ્રી સ્ટાન્ડર્ડ પિનઆઉટ સાથે 8-પિન ડ્યુઅલ-ઇન-લાઇન પેકેજ
ઓપરેશનલ એમ્પ્લીફાયર છે.

\textbf{પિન ડાયગ્રામ:}

\begin{verbatim}
    +{-{-}{-}U{-}{-}{-}+}
  1 |       | 8
    |  741  |
  2 |       | 7
    |       |
  3 |       | 6
    |       |
  4 |       | 5
    +{-{-}{-}{-}{-}{-}{-}+}
\end{verbatim}

\textbf{પિન કન્ફિગરેશન:}

\textbf{ટેબલ:}

{\def\LTcaptype{none} % do not increment counter
\begin{longtable}[]{@{}lll@{}}
\toprule\noalign{}
પિન & ફંક્શન & વર્ણન \\
\midrule\noalign{}
\endhead
\bottomrule\noalign{}
\endlastfoot
\textbf{1} & ઓફસેટ નલ & ઓફસેટ એડજસ્ટમેન્ટ \\
\textbf{2} & ઇન્વર્ટિંગ ઇનપુટ & નેગેટિવ ઇનપુટ \\
\textbf{3} & નોન-ઇન્વર્ટિંગ ઇનપુટ & પોઝિટિવ ઇનપુટ \\
\textbf{4} & -Vcc & નેગેટિવ સપ્લાય \\
\textbf{5} & ઓફસેટ નલ & ઓફસેટ એડજસ્ટમેન્ટ \\
\textbf{6} & આઉટપુટ & એમ્પ્લીફાયર આઉટપુટ \\
\textbf{7} & +Vcc & પોઝિટિવ સપ્લાય \\
\textbf{8} & NC & કોઈ કનેક્શન નહીં \\
\end{longtable}
}

\end{solutionbox}
\begin{mnemonicbox}
``નલ, નેગેટિવ, પોઝિટિવ, નેગેટિવ સપ્લાય, નલ, આઉટપુટ,
પોઝિટિવ સપ્લાય, કંઈ નહીં''

\end{mnemonicbox}
\subsection*{પ્રશ્ન ૪(બ) [૪
ગુણ]}\label{uxaaauxab0uxab6uxaa8-uxaeauxaac-uxaea-uxa97uxaa3}

\textbf{OP-AMPના નીચેના પરિમાણ વ્યાખ્યાયિત કરો. ૧. ઇનપુટ ઓફસેટ વોલ્ટેજ ૨.
સી.એમ.આર.આર}

\begin{solutionbox}
આ પેરામીટર્સ પ્રેક્ટિકલ ઓપરેશનલ એમ્પ્લીફાયરની નોન-આઇડીયલ
લાક્ષણિકતાઓ વ્યાખ્યાયિત કરે છે.

\textbf{૧. ઇનપુટ ઓફસેટ વોલ્ટેજ (Vio):}

\begin{itemize}
\tightlist
\item
  \textbf{વ્યાખ્યા}: આઉટપુટ શૂન્ય બનાવવા માટે ઇનપુટ્સ વચ્ચે લાગુ કરવામાં આવતું DC
  વોલ્ટેજ
\item
  \textbf{સામાન્ય મૂલ્ય}: 741 માટે 1-5 mV
\item
  \textbf{કારણ}: ઇનપુટ ટ્રાન્ઝિસ્ટરમાં મિસમેચ
\item
  \textbf{અસર}: DC એપ્લીકેશનમાં આઉટપુટ એરર
\end{itemize}

\textbf{૨. કોમન મોડ રિજેક્શન રેશિયો (CMRR):}

\begin{itemize}
\tightlist
\item
  \textbf{વ્યાખ્યા}: બંને ઇનપુટ્સ પર કોમન સિગ્નલ રિજેક્ટ કરવાની ક્ષમતા
\item
  \textbf{સૂત્ર}: CMRR = Ad/Acm
\item
  \textbf{સામાન્ય મૂલ્ય}: 741 માટે 90 dB
\item
  \textbf{મહત્વ}: નોઇઝ ઇમ્યુનિટી
\end{itemize}

\textbf{ટેબલ:}

{\def\LTcaptype{none} % do not increment counter
\begin{longtable}[]{@{}lllll@{}}
\toprule\noalign{}
પેરામીટર & સિમ્બોલ & એકમ & આઇડીયલ & 741 સામાન્ય \\
\midrule\noalign{}
\endhead
\bottomrule\noalign{}
\endlastfoot
\textbf{ઇનપુટ ઓફસેટ વોલ્ટેજ} & Vio & mV & 0 & 2 \\
\textbf{CMRR} & - & dB & \infty & 90 \\
\end{longtable}
}

\end{solutionbox}
\begin{mnemonicbox}
``ઓફસેટ આઉટપુટ એરર બનાવે, CMRR કોમન સિગ્નલ રિજેક્ટ કરે''

\end{mnemonicbox}
\subsection*{પ્રશ્ન ૪(ક) [૭
ગુણ]}\label{uxaaauxab0uxab6uxaa8-uxaeauxa95-uxaed-uxa97uxaa3}

\textbf{IC 741ની મદદથી ઇન્વર્ટિંગ એમ્પ્લીફાયર વિસ્તૃતમાં સમજાવો.}

\begin{solutionbox}
ઇન્વર્ટિંગ એમ્પ્લીફાયર ઇન્વર્ટિંગ ટર્મિનલ પર લાગુ ઇનપુટ સાથે નેગેટિવ
ફીડબેકનો ઉપયોગ કરે છે.

\textbf{સર્કિટ ડાયગ્રામ:}

\begin{center}
\textbf{Mermaid Diagram (Code)}
\begin{verbatim}
{Shaded}
{Highlighting}[]
graph LR
    A[Vin] {-{-}{} B[R1]}
    B {-{-}{} C["{}{-}"]}
    D["{+"] {-}{-}{} E[Ground]}
    C {-{-}{} F[IC 741]}
    F {-{-}{} G[Vout]}
    G {-{-}{} H[Rf]}
    H {-{-}{} C}
{Highlighting}
{Shaded}
\end{verbatim}
\end{center}

\textbf{વિશ્લેષણ:} વર્ચ્યુઅલ શોર્ટ કોન્સેપ્ટનો ઉપયોગ કરીને:

\begin{itemize}
\tightlist
\item
  \textbf{V+ = V- = 0V} (વર્ચ્યુઅલ ગ્રાઉન્ડ)
\item
  \textbf{ઇનપુટ કરંટ}: I1 = Vin/R1
\item
  \textbf{ફીડબેક કરંટ}: If = Vout/Rf
\item
  \textbf{કરંટ બેલેન્સ}: I1 = If (ઓપ-એમ્પમાં કોઈ કરંટ નહીં)
\end{itemize}

\textbf{વ્યુત્પત્તિ:}

\begin{itemize}
\tightlist
\item
  Vin/R1 = -Vout/Rf
\item
  \textbf{વોલ્ટેજ ગેઈન: Av = -Rf/R1}
\end{itemize}

\textbf{લાક્ષણિકતાઓ:}

\textbf{ટેબલ:}

{\def\LTcaptype{none} % do not increment counter
\begin{longtable}[]{@{}lll@{}}
\toprule\noalign{}
પેરામીટર & એક્સપ્રેશન & નોંધ \\
\midrule\noalign{}
\endhead
\bottomrule\noalign{}
\endlastfoot
\textbf{વોલ્ટેજ ગેઈન} & -Rf/R1 & નેગેટિવ સાઇન \\
\textbf{ઇનપુટ ઇમ્પીડન્સ} & R1 & નીચું ઇમ્પીડન્સ \\
\textbf{આઉટપુટ ઇમ્પીડન્સ} & \textasciitilde0Ω & ખૂબ નીચું \\
\textbf{બેન્ડવિડ્થ} & f = GBW/\textbar Av\textbar{} & ગેઈન-બેન્ડવિડ્થ પ્રોડક્ટ \\
\end{longtable}
}

\textbf{એપ્લીકેશન:}

\begin{itemize}
\tightlist
\item
  \textbf{સિગ્નલ ઇન્વર્શન}: ફેઝ રિવર્સલ
\item
  \textbf{સ્કેલ ફેક્ટર}: પ્રોગ્રામેબલ ગેઈન
\item
  \textbf{AC એમ્પ્લિફિકેશન}: કપલિંગ કેપેસિટર સાથે
\end{itemize}

\end{solutionbox}
\begin{mnemonicbox}
``ઇન્વર્ટિંગ ઇનપુટ ઇન્વર્ટેડ આઉટપુટ આપે''

\end{mnemonicbox}
\subsection*{પ્રશ્ન ૪(અ અથવા) [૩
ગુણ]}\label{uxaaauxab0uxab6uxaa8-uxaeauxa85-uxa85uxaa5uxab5-uxae9-uxa97uxaa3}

\textbf{Ideal OP-AMPની લાક્ષણિકતાની સૂચિ બનાવો.}

\begin{solutionbox}
આઇડીયલ ઓપ-એમ્પ બધા પેરામીટર્સ માટે સૈદ્ધાંતિક મર્યાદા સાથે સંપૂર્ણ
એમ્પ્લીફાયરનું પ્રતિનિધિત્વ કરે છે.

\textbf{ટેબલ:}

{\def\LTcaptype{none} % do not increment counter
\begin{longtable}[]{@{}lll@{}}
\toprule\noalign{}
પેરામીટર & આઇડીયલ મૂલ્ય & પ્રેક્ટિકલ ઇમ્પેક્ટ \\
\midrule\noalign{}
\endhead
\bottomrule\noalign{}
\endlastfoot
\textbf{ઓપન લૂપ ગેઈન} & \infty & સંપૂર્ણ એમ્પ્લિફિકેશન \\
\textbf{ઇનપુટ ઇમ્પીડન્સ} & \infty & કોઈ ઇનપુટ કરંટ નહીં \\
\textbf{આઉટપુટ ઇમ્પીડન્સ} & 0Ω & સંપૂર્ણ વોલ્ટેજ સોર્સ \\
\textbf{બેન્ડવિડ્થ} & \infty & કોઈ ફ્રીક્વન્સી મર્યાદા નહીં \\
\textbf{CMRR} & \infty & સંપૂર્ણ નોઇઝ રિજેક્શન \\
\textbf{સ્લ્યુ રેટ} & \infty & કોઈ સ્લ્યુ રેટ લિમિટિંગ નહીં \\
\textbf{ઇનપુટ ઓફસેટ} & 0V & કોઈ DC એરર નહીં \\
\end{longtable}
}

\begin{itemize}
\tightlist
\item
  \textbf{સંપૂર્ણ કામગીરી}: બધા પેરામીટર્સ ઓપ્ટિમાઇઝ્ડ
\item
  \textbf{ડિઝાઇન સરળીકરણ}: વિશ્લેષણ સરળ બને છે
\item
  \textbf{પ્રેક્ટિકલ અપ્રોક્સિમેશન}: ઘણી એપ્લીકેશનમાં આઇડીયલની નજીક
\end{itemize}

\end{solutionbox}
\begin{mnemonicbox}
``અનંત ઇનપુટ, શૂન્ય આઉટપુટ, સંપૂર્ણ કામગીરી''

\end{mnemonicbox}
\subsection*{પ્રશ્ન ૪(બ અથવા) [૪
ગુણ]}\label{uxaaauxab0uxab6uxaa8-uxaeauxaac-uxa85uxaa5uxab5-uxaea-uxa97uxaa3}

\textbf{Op-ampની મદદથી સમિંગ એમ્પ્લીફાયર દોરો અને સમજાવો.}

\begin{solutionbox}
સમિંગ એમ્પ્લીફાયર દરેક ઇનપુટ માટે પ્રોગ્રામેબલ ગેઈન સાથે બહુવિધ ઇનપુટ
વોલ્ટેજ ઉમેરે છે.

\textbf{સર્કિટ ડાયગ્રામ:}

\begin{verbatim}
V1 {-{-}{-}R1{-}{-}{-}+}
           |
V2 {-{-}{-}R2{-}{-}{-}+{-}{-}{-} ({-}) IC741 {-}{-}{-} Vout}
           |             |
V3 {-{-}{-}R3{-}{-}{-}+             |}
                        Rf
           (+) {-{-}{-}{-}{-}{-}{-}{-}GND}
\end{verbatim}

\textbf{વિશ્લેષણ:} વર્ચ્યુઅલ ગ્રાઉન્ડ કોન્સેપ્ટનો ઉપયોગ કરીને (V- = 0V):

\begin{itemize}
\tightlist
\item
  \textbf{R1 દ્વારા કરંટ}: I1 = V1/R1
\item
  \textbf{R2 દ્વારા કરંટ}: I2 = V2/R2\\
\item
  \textbf{R3 દ્વારા કરંટ}: I3 = V3/R3
\item
  \textbf{કુલ ઇનપુટ કરંટ}: Iin = I1 + I2 + I3
\end{itemize}

\textbf{આઉટપુટ સમીકરણ:} \textbf{Vout = -Rf(V1/R1 + V2/R2 + V3/R3)}

\textbf{વિશેષ કેસો:}

\begin{itemize}
\tightlist
\item
  \textbf{સમાન રેઝિસ્ટર}: Vout = -(Rf/R)(V1 + V2 + V3)
\item
  \textbf{યુનિટી ગેઈન}: Rf = R, Vout = -(V1 + V2 + V3)
\end{itemize}

\textbf{એપ્લીકેશન:}

\begin{itemize}
\tightlist
\item
  \textbf{ઓડિયો મિક્સિંગ}: બહુવિધ સિગ્નલ કમ્બિનેશન
\item
  \textbf{ડિજિટલ-ટુ-એનાલોગ}: વેઈટેડ રેઝિસ્ટર DAC
\item
  \textbf{સિગ્નલ પ્રોસેસિંગ}: ગણિતીય ઓપરેશન
\end{itemize}

\end{solutionbox}
\begin{mnemonicbox}
``ઇનપુટ્સ સરવાળો, રેઝિસ્ટર રેશિયો દ્વારા સ્કેલ કરો''

\end{mnemonicbox}
\subsection*{પ્રશ્ન ૪(ક અથવા) [૭
ગુણ]}\label{uxaaauxab0uxab6uxaa8-uxaeauxa95-uxa85uxaa5uxab5-uxaed-uxa97uxaa3}

\textbf{IC741ની મદદથી ડિફરેન્શિયલ એમ્પ્લીફાયર વિસ્તૃતમાં સમજાવો.}

\begin{solutionbox}
ડિફરેન્શિયલ એમ્પ્લીફાયર કોમન સિગ્નલ રિજેક્ટ કરતાં બે ઇનપુટ સિગ્નલ
વચ્ચેનો તફાવત એમ્પ્લિફાઇ કરે છે.

\textbf{સર્કિટ ડાયગ્રામ:}

\begin{verbatim}
V1 {-{-}{-}R1{-}{-}{-} ({-}) }
            IC741 {-{-}{-} Vout}
V2 {-{-}{-}R2{-}{-}{-} (+)  |}
              |  |
             R3  Rf
              |  |
             GND +
\end{verbatim}

\textbf{વિશ્લેષણ:} નોન-ઇન્વર્ટિંગ ઇનપુટ માટે:

\begin{itemize}
\tightlist
\item
  \textbf{V+ = V2 \times R3/(R2+R3)}
\end{itemize}

ઇન્વર્ટિંગ ઇનપુટ માટે વર્ચ્યુઅલ શોર્ટનો ઉપયોગ કરીને:

\begin{itemize}
\tightlist
\item
  \textbf{V- = V+ = V2 \times R3/(R2+R3)}
\end{itemize}

કરંટ બેલેન્સનો ઉપયોગ કરીને:

\begin{itemize}
\tightlist
\item
  \textbf{(V1-V-)/R1 = (V--Vout)/Rf}
\end{itemize}

\textbf{આઉટપુટ સમીકરણ:} જ્યારે R1 = R2 અને R3 = Rf: \textbf{Vout =
(Rf/R1)(V2 - V1)}

\textbf{મુખ્ય લાક્ષણિકતાઓ:}

\textbf{ટેબલ:}

{\def\LTcaptype{none} % do not increment counter
\begin{longtable}[]{@{}lll@{}}
\toprule\noalign{}
પેરામીટર & મૂલ્ય & ફાયદો \\
\midrule\noalign{}
\endhead
\bottomrule\noalign{}
\endlastfoot
\textbf{ડિફરેન્શિયલ ગેઈન} & Rf/R1 & તફાવત એમ્પ્લિફાઇ કરે \\
\textbf{કોમન મોડ ગેઈન} & \textasciitilde0 & કોમન સિગ્નલ રિજેક્ટ કરે \\
\textbf{CMRR} & ખૂબ ઊંચું & શ્રેષ્ઠ નોઇઝ ઇમ્યુનિટી \\
\end{longtable}
}

\textbf{એપ્લીકેશન:}

\begin{itemize}
\tightlist
\item
  \textbf{ઇન્સ્ટ્રુમેન્ટેશન}: સેન્સર સિગ્નલ પ્રોસેસિંગ
\item
  \textbf{નોઇઝ રિજેક્શન}: ડિફરેન્શિયલ સિગ્નલ ટ્રાન્સમિશ
\item
  \textbf{બ્રિજ સર્કિટ}: સ્ટ્રેઇન ગેજ મેઝરમેન્ટ
\end{itemize}

\end{solutionbox}
\begin{mnemonicbox}
``તફાવત એમ્પ્લિફાઇડ, કોમન રિજેક્ટેડ''

\end{mnemonicbox}
\subsection*{પ્રશ્ન ૫(અ) [૩
ગુણ]}\label{uxaaauxab0uxab6uxaa8-uxaebuxa85-uxae9-uxa97uxaa3}

\textbf{OP-AMPની મદદથી ઇન્ટીગ્રેટર સર્કિટ દોરો અને તેના ઇનપુટ અને આઉટપુટ વેવફોર્મ
દોરો.}

\begin{solutionbox}
ઓપ-એમ્પ ઇન્ટીગ્રેટર RC ફીડબેકનો ઉપયોગ કરીને ઇનપુટ સિગ્નલનું ગાણિતિક
ઇન્ટીગ્રેશન કરે છે.

\textbf{સર્કિટ ડાયગ્રામ:}

\begin{verbatim}
Vin {-{-}{-}R{-}{-}{-} ({-}) IC741 {-}{-}{-} Vout}
              |        |
             (+)       C
              |        |
             GND      /
\end{verbatim}

\textbf{વેવફોર્મ:}

\begin{verbatim}
ઇનપુટ (સ્ક્વેર વેવ):
     +V |‾‾|\_\_|‾‾|\_\_|‾‾
        |  |  |  |  |
      0 +{-{-}+{-}{-}+{-}{-}+{-}{-}+{-}{-} t}
        |  |  |  |  |
     {-V    |\_\_|  |\_\_|}

આઉટપુટ (ત્રિકોણાકાર):
      0 +  /{  /  {-}{-} t}
        | /  {/  }
     {-V +/        }
\end{verbatim}

\textbf{ઓપરેશન:}

\begin{itemize}
\tightlist
\item
  \textbf{ઇન્ટીગ્રેશન ફંક્શન}: Vout = -(1/RC)\intVin dt
\item
  \textbf{સ્ક્વેર વેવ ઇનપુટ}: ત્રિકોણાકાર આઉટપુટ ઉત્પન્ન કરે છે
\item
  \textbf{રેમ્પ જનરેશન}: કોન્સ્ટન્ટ ઇનપુટ લીનિયર રેમ્પ આપે છે
\end{itemize}

\end{solutionbox}
\begin{mnemonicbox}
``ઇન્ટીગ્રેશન સ્ક્વેરમાંથી ત્રિકોણાકાર બનાવે''

\end{mnemonicbox}
\subsection*{પ્રશ્ન ૫(બ) [૪
ગુણ]}\label{uxaaauxab0uxab6uxaa8-uxaebuxaac-uxaea-uxa97uxaa3}

\textbf{પુશ પુલ એરેન્જમેન્ટ પાવર એમ્પ્લીફાયરના ફાયદા તથા ગેરફાયદા જણાવો.}

\begin{solutionbox}
પુશ-પુલ કન્ફિગરેશન પાવર એમ્પ્લિફિકેશન માટે કમ્પ્લીમેન્ટરી રીતે ઓપરેટ
કરતા બે ટ્રાન્ઝિસ્ટરનો ઉપયોગ કરે છે.

\textbf{ફાયદા:}

\textbf{ટેબલ:}

{\def\LTcaptype{none} % do not increment counter
\begin{longtable}[]{@{}lll@{}}
\toprule\noalign{}
ફાયદો & લાભ & એપ્લીકેશન \\
\midrule\noalign{}
\endhead
\bottomrule\noalign{}
\endlastfoot
\textbf{ঊચ્ચ કાર્યక્ષમતા} & 78.5\% સુધી & બેટરી ઓપરેટેડ \\
\textbf{ટ્રાન્સફોર્મર નહીં} & કોમ્પેક્ટ ડિઝાઇન & પોર્ટેબલ ડિવાઇસ \\
\textbf{ઓછી વિકૃતિ} & સારી લીનિયરિટી & ઓડિયો સિસ્ટમ \\
\textbf{ગરમીનું વિતરણ} & ટ્રાન્ઝિસ્ટર વચ્ચે વહેંચાયેલું & થર્મલ મેનેજમેન્ટ \\
\end{longtable}
}

\textbf{ગેરફાયદા:}

{\def\LTcaptype{none} % do not increment counter
\begin{longtable}[]{@{}lll@{}}
\toprule\noalign{}
ગેરફાયદો & સમસ્યા & ઉકેલ \\
\midrule\noalign{}
\endhead
\bottomrule\noalign{}
\endlastfoot
\textbf{ક્રોસઓવર ડિસ્ટોર્શન} & શૂન્ય ક્રોસિંગ પર ડેડ ઝોન & ક્લાસ AB બાયસ \\
\textbf{કોમ્પોનન્ટ મેચિંગ} & મેચ્ડ ટ્રાન્ઝિસ્ટરની જરૂર & કાળજીપૂર્વક પસંદગી \\
\textbf{થર્મલ રનઅવે} & તાપમાન કોઇફિશન્ટ મિસમેચ & થર્મલ કપલિંગ \\
\end{longtable}
}

\textbf{એપ્લીકેશન:}

\begin{itemize}
\tightlist
\item
  \textbf{ઓડિયો એમ્પ્લીફાયર}: હાઇ ફિડેલિટી સિસ્ટમ
\item
  \textbf{મોટર ડ્રાઇવર}: DC મોટર કંટ્રોલ
\item
  \textbf{RF એમ્પ્લીફાયર}: કમ્યુનિકેશન સિસ્ટમ
\end{itemize}

\end{solutionbox}
\begin{mnemonicbox}
``પુશ-પુલ પાવર પ્રદાન કરે પણ સમસ્યાઓ છે''

\end{mnemonicbox}
\subsection*{પ્રશ્ન ૫(ક) [૭
ગુણ]}\label{uxaaauxab0uxab6uxaa8-uxaebuxa95-uxaed-uxa97uxaa3}

\textbf{555 ટાઇમર ICની મદદથી એસ્ટેબલ મલ્ટીવાઇબ્રેટર દોરો અને સમજાવો.}

\begin{solutionbox}
એસ્ટેબલ મલ્ટીવાઇબ્રેટર 555 ટાઇમરનો ઉપયોગ કરીને બાહ્ય ટ્રિગર વિના
સતત સ્ક્વેર વેવ આઉટપુટ ઉત્પન્ન કરે છે.

\textbf{સર્કિટ ડાયગ્રામ:}

\begin{verbatim}
    +Vcc
      |
      RA
      |
   +{-{-}+{-}{-}+ (7)}
   |     |
   RB   (2)+(6) 555 (3){-{-}{-} Output}
   |         |     |
   C        (1)   (4)
   |         |     |
  GND       GND   +Vcc
\end{verbatim}

\textbf{પિન કનેક્શન:}

\begin{itemize}
\tightlist
\item
  \textbf{પિન 1}: ગ્રાઉન્ડ
\item
  \textbf{પિન 2}: ટ્રિગર (પિન 6 સાથે કનેક્ટેડ)
\item
  \textbf{પિન 3}: આઉટપુટ
\item
  \textbf{પિન 4}: રીસેટ (+Vcc)
\item
  \textbf{પિન 6}: થ્રેશોલ્ડ
\item
  \textbf{પિન 7}: ડિસચાર્જ
\item
  \textbf{પિન 8}: +Vcc
\end{itemize}

\textbf{ઓપરેશન:}

\begin{enumerate}
\tightlist
\item
  \textbf{ચાર્જિંગ ફેઝ}: C એ RA + RB દ્વારા ચાર્જ થાય છે
\item
  \textbf{થ્રેશોલ્ડ પહોંચ્યું}: 2/3 Vcc પર, આઉટપુટ LOW જાય છે
\item
  \textbf{ડિસચાર્જિંગ ફેઝ}: C એ RB દ્વારા ડિસચાર્જ થાય છે
\item
  \textbf{ટ્રિગર પહોંચ્યું}: 1/3 Vcc પર, આઉટપુટ HIGH જાય છે
\item
  \textbf{સાયકલ રિપીટ}: સતત ઓસિલેશન
\end{enumerate}

\textbf{ટાઇમિંગ સમીકરણો:}

\begin{itemize}
\tightlist
\item
  \textbf{HIGH સમય}: t1 = 0.693(RA + RB)C
\item
  \textbf{LOW સમય}: t2 = 0.693(RB)C\\
\item
  \textbf{કુલ પીરિયડ}: T = t1 + t2 = 0.693(RA + 2RB)C
\item
  \textbf{ફ્રીક્વન્સી}: f = 1.44/[(RA + 2RB)C]
\item
  \textbf{ડ્યુટી સાયકલ}: D = (RA + RB)/(RA + 2RB) \times 100\%
\end{itemize}

\textbf{એપ્લીકેશન:}

\begin{itemize}
\tightlist
\item
  \textbf{ક્લોક જનરેશન}: ડિજિટલ સિસ્ટમ
\item
  \textbf{LED ફ્લેશર}: બ્લિંકિંગ સર્કિટ
\item
  \textbf{ટોન જનરેશન}: ઓડિયો ઓસિલેટર
\item
  \textbf{PWM જનરેશન}: મોટર સ્પીડ કંટ્રોલ
\end{itemize}

\end{solutionbox}
\begin{mnemonicbox}
``એસ્ટેબલ હંમેશા ઓટોમેટિક ઓસિલેટ કરે''

\end{mnemonicbox}
\subsection*{પ્રશ્ન ૫(અ અથવા) [૩
ગુણ]}\label{uxaaauxab0uxab6uxaa8-uxaebuxa85-uxa85uxaa5uxab5-uxae9-uxa97uxaa3}

\textbf{Op-ampનો બ્લોક ડાયગ્રામ દોરો અને તેને સમજાવો.}

\begin{solutionbox}
ઓપ-એમ્પની આંતરિક રચના ઉચ્ચ ગેઈન અને કામગીરી માટે બહુવિધ સ્ટેજનો
સમાવેશ કરે છે.

\textbf{બ્લોક ડાયગ્રામ:}

\begin{center}
\textbf{Mermaid Diagram (Code)}
\begin{verbatim}
{Shaded}
{Highlighting}[]
graph LR
    A[V+] {-{-}{} B[Differential Amplifier]}
    C[V{-] {-}{-}{} B}
    B {-{-}{} D[Intermediate Amplifier]}
    D {-{-}{} E[Output Stage]}
    E {-{-}{} F[Output]}
    G[Level Shifter] {-{-}{} E}
    D {-{-}{} G}
{Highlighting}
{Shaded}
\end{verbatim}
\end{center}

\textbf{સ્ટેજ ફંક્શન:}

\textbf{ટેબલ:}

{\def\LTcaptype{none} % do not increment counter
\begin{longtable}[]{@{}lll@{}}
\toprule\noalign{}
સ્ટેજ & ફંક્શન & લાક્ષણિકતાઓ \\
\midrule\noalign{}
\endhead
\bottomrule\noalign{}
\endlastfoot
\textbf{ડિફરેન્શિયલ ઇનપુટ} & ઉચ્ચ ઇનપુટ ઇમ્પીડન્સ & નીચું ઓફસેટ, ઉચ્ચ CMRR \\
\textbf{ઇન્ટરમીડિયેટ એમ્પ્લીફાયર} & ઉચ્ચ વોલ્ટેજ ગેઈન & મોટાભાગનું ગેઈન \\
\textbf{લેવલ શિફ્ટર} & DC લેવલ એડજસ્ટમેન્ટ & AC સ્ટેજ કપલ કરે છે \\
\textbf{આઉટપુટ સ્ટેજ} & નીચું આઉટપુટ ઇમ્પીડન્સ & કરંટ બફર \\
\end{longtable}
}

\begin{itemize}
\tightlist
\item
  \textbf{ઉચ્ચ ગેઈન}: સામાન્ય રીતે 100,000 અથવા વધુ
\item
  \textbf{વાઇડ બેન્ડવિડ્થ}: MHz રેન્જ ક્ષમતા\\
\item
  \textbf{નીચું આઉટપુટ ઇમ્પીડન્સ}: વિવિધ લોડ ડ્રાઇવ કરે છે
\end{itemize}

\end{solutionbox}
\begin{mnemonicbox}
``ડિફરેન્શિયલ ઇનપુટ, ઇન્ટરમીડિયેટ ગેઈન, લેવલ શિફ્ટ, આઉટપુટ
બફર''

\end{mnemonicbox}
\subsection*{પ્રશ્ન ૫(બ અથવા) [૪
ગુણ]}\label{uxaaauxab0uxab6uxaa8-uxaebuxaac-uxa85uxaa5uxab5-uxaea-uxa97uxaa3}

\textbf{પાવર એમ્પ્લીફાયરના સંદર્ભમાં પદો વિશે સમજાવો.i) કાર્યક્ષમતા ii)
ડિસ્ટોર્શન.}

\begin{solutionbox}
આ પેરામીટર્સ પાવર એમ્પ્લીફાયરની કામગીરી અને એપ્લીકેશન માટે યોગ્યતા
નક્કી કરે છે.

\textbf{i) કાર્યક્ષમતા (η):}

\begin{itemize}
\tightlist
\item
  \textbf{વ્યાખ્યા}: AC આઉટપુટ પાવર અને DC ઇનપુટ પાવરનો ગુણોત્તર
\item
  \textbf{સૂત્ર}: η = Po(AC)/Pin(DC) \times 100\%
\item
  \textbf{મહત્વ}: ગરમી વિસર્જન અને બેટરી લાઇફ નક્કી કરે છે
\end{itemize}

\textbf{કાર્યક્ષમતા સરખામણી:}

\textbf{ટેબલ:}

{\def\LTcaptype{none} % do not increment counter
\begin{longtable}[]{@{}lll@{}}
\toprule\noalign{}
ક્લાસ & કાર્યક્ષમતા & એપ્લીકેશન \\
\midrule\noalign{}
\endhead
\bottomrule\noalign{}
\endlastfoot
\textbf{A} & 25\% & લો પાવર, હાઇ ફિડેલિટી \\
\textbf{B} & 78.5\% & પુશ-પુલ એમ્પ્લીફાયર \\
\textbf{AB} & 60-70\% & ઓડિયો એમ્પ્લીફાયર \\
\textbf{C} & \textgreater90\% & RF એપ્લીકેશન \\
\end{longtable}
}

\textbf{ii) ડિસ્ટોર્શન:}

\begin{itemize}
\tightlist
\item
  \textbf{વ્યાખ્યા}: આઉટપુટ સિગ્નલ શેપમાં અનિચ્છનીય ફેરફારો
\item
  \textbf{પ્રકારો}: હાર્મોનિક, ઇન્ટરમોડ્યુલેશન, ક્રોસઓવર
\item
  \textbf{મેઝરમેન્ટ}: ટોટલ હાર્મોનિક ડિસ્ટોર્શન (THD)
\end{itemize}

\textbf{ડિસ્ટોર્શન સોર્સ:}

\begin{itemize}
\tightlist
\item
  \textbf{નોનલીનિયરિટી}: ટ્રાન્ઝિસ્ટર લાક્ષણિકતાઓ
\item
  \textbf{ક્રોસઓવર}: પુશ-પુલમાં ડેડ ઝોન
\item
  \textbf{થર્મલ ઇફેક્ટ}: તાપમાન વેરિયેશન
\end{itemize}

\end{solutionbox}
\begin{mnemonicbox}
``કાર્યક્ષમતા ઊર્જા ઉપયોગ માપે, ડિસ્ટોર્શન સિગ્નલ ડિગ્રેડેશન
દર્શાવે''

\end{mnemonicbox}
\subsection*{પ્રશ્ન ૫(ક અથવા) [૭
ગુણ]}\label{uxaaauxab0uxab6uxaa8-uxaebuxa95-uxa85uxaa5uxab5-uxaed-uxa97uxaa3}

\textbf{555 ટાઇમર IC નો પિન ડાયગ્રામ દોરો. ઉપરાંત 555 ટાઇમર ICની મદદથી બે
સ્ટેજવાળું સિક્વન્સિયલ ટાઇમર દોરો.}

\begin{solutionbox}
555 ટાઇમર સ્ટાન્ડર્ડ 8-પિન પેકેજ સાથે ટાઇમિંગ એપ્લીકેશન માટે
વર્સેટાઇલ IC છે.

\textbf{પિન ડાયગ્રામ:}

\begin{verbatim}
    +{-{-}{-}U{-}{-}{-}+}
  1 |       | 8  +Vcc
GND |  555  | 7  Discharge
    |       |
  2 |       | 6  Threshold
Trig|       |
    |       | 5  Control
  3 |       |
Out |       | 4  Reset
    +{-{-}{-}{-}{-}{-}{-}+}
\end{verbatim}

\textbf{પિન ફંક્શન:}

\textbf{ટેબલ:}

{\def\LTcaptype{none} % do not increment counter
\begin{longtable}[]{@{}lll@{}}
\toprule\noalign{}
પિન & નામ & ફંક્શન \\
\midrule\noalign{}
\endhead
\bottomrule\noalign{}
\endlastfoot
\textbf{1} & ગ્રાઉન્ડ & કોમન ગ્રાઉન્ડ \\
\textbf{2} & ટ્રિગર & ટાઇમિંગ સાયકલ શરૂ કરે \\
\textbf{3} & આઉટપુટ & ટાઇમર આઉટપુટ \\
\textbf{4} & રીસેટ & ટાઇમર રીસેટ કરે \\
\textbf{5} & કંટ્રોલ & વોલ્ટેજ રેફરન્સ \\
\textbf{6} & થ્રેશોલ્ડ & ટાઇમિંગ સાયકલ બંધ કરે \\
\textbf{7} & ડિસચાર્જ & ટાઇમિંગ કેપેસિટર ડિસચાર્જ કરે \\
\textbf{8} & Vcc & સપ્લાય વોલ્ટેજ \\
\end{longtable}
}

\textbf{બે સ્ટેજ સિક્વન્સિયલ ટાઇમર સર્કિટ:}

\begin{verbatim}
First Stage (555A):
    +Vcc
      |
      R1
      |
   +{-{-}+{-}{-}+ (7)}
   |     |
   R2   (2)+(6) 555A (3){-{-}{-}+}
   |         |      |      |
   C1       (1)    (4)     |
   |         |      |      |
  GND       GND    +Vcc    |
                           |
Second Stage (555B):       |
    +Vcc                   |
      |                    |
      R3                   |
      |                    |
   +{-{-}+{-}{-}+ (7)             |}
   |     |                 |
   R4   (2) 555B (3){-{-}{-} Output}
   |     |   |      |
   C2   (6) (1)    (4)
   |     |   |      |
  GND{-{-}{-}{-}+{-}{-}GND    +Vcc}
         |
         +{-{-}{-}{-}{-}{-}{-}{-}{-}{-}+}
\end{verbatim}

\textbf{ઓપરેશન:}

\begin{enumerate}
\tightlist
\item
  \textbf{પ્રથમ ટાઇમર}: મોનોસ્ટેબલ મોડમાં ઓપરેટ કરે છે
\item
  \textbf{ટ્રિગર લાગુ}: પ્રથમ ટાઇમર આઉટપુટ પલ્સ આપે છે
\item
  \textbf{આઉટપુટ અવધિ}: T1 = 1.1 \times R2 \times C1
\item
  \textbf{બીજું ટાઇમર}: પ્રથમ ટાઇમરના આઉટપુટ દ્વારા ટ્રિગર થાય છે
\item
  \textbf{સિક્વન્સિયલ ઓપરેશન}: પ્રથમ પૂર્ણ થયા પછી બીજું શરૂ થાય છે
\item
  \textbf{કુલ વિલંબ}: T1 + T2 જ્યાં T2 = 1.1 \times R4 \times C2
\end{enumerate}

\textbf{એપ્લીકેશન:}

\begin{itemize}
\tightlist
\item
  \textbf{ડિલે સર્કિટ}: સિક્વન્સિયલ સ્વિચિંગ
\item
  \textbf{ટ્રાફિક લાઇટ}: ટાઇમ્ડ સિક્વન્સ કંટ્રોલ
\item
  \textbf{ઇન્ડસ્ટ્રિયલ ઓટોમેશન}: પ્રોસેસ ટાઇમિંગ
\item
  \textbf{મોટર કંટ્રોલ}: સ્ટાર્ટ-સ્ટોપ સિક્વન્સ
\end{itemize}

\textbf{ટાઇમિંગ સમીકરણો:}

\begin{itemize}
\tightlist
\item
  \textbf{સ્ટેજ 1 વિલંબ}: T1 = 1.1 R2 C1
\item
  \textbf{સ્ટેજ 2 વિલંબ}: T2 = 1.1 R4 C2
\item
  \textbf{કુલ સિક્વન્સ સમય}: Ttotal = T1 + T2
\end{itemize}

\textbf{મુખ્ય લાક્ષણિકતાઓ:}

\begin{itemize}
\tightlist
\item
  \textbf{સ્વતંત્ર ટાઇમિંગ}: દરેક સ્ટેજ અલગથી એડજસ્ટેબલ
\item
  \textbf{સિક્વન્સિયલ ઓપરેશન}: સ્ટેજ વચ્ચે કોઈ ઓવરલેપ નહીં
\item
  \textbf{વિશ્વસનીય સ્વિચિંગ}: સ્વચ્છ ડિજિટલ ટ્રાન્ઝિશન
\item
  \textbf{સરળ ડિઝાઇન}: સરળ કોમ્પોનન્ટ ગણતરી
\end{itemize}

\end{solutionbox}
\begin{mnemonicbox}
``સિક્વન્સિયલ સ્ટેજ અલગથી શરૂ થાય''

\end{mnemonicbox}

\end{document}
