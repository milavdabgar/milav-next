\documentclass[10pt,a4paper]{article}

% content/resources/templates/preamble.tex
\usepackage[margin=0.6in]{geometry}
\author{Milav Dabgar}
\usepackage{amsmath,amssymb,amsthm}
\usepackage{booktabs}
\usepackage{multirow}
\usepackage{xcolor}
\usepackage{tcolorbox}
\tcbuselibrary{breakable,skins}
\usepackage[colorlinks=true,linkcolor=blue]{hyperref}
\usepackage{titlesec}
\usepackage{enumitem}
\usepackage{tikz}
\usepackage{pgfplots}
\usepackage{circuitikz}
\usepackage[version=4]{mhchem}
\usepackage{longtable}
\usepackage{array}
\usepackage{float}
\usepackage{caption}
\usepackage{listings}

\lstset{
  basicstyle=\small\ttfamily,
  breaklines=true,
  breakatwhitespace=false,
  postbreak=\mbox{\textcolor{red}{$\hookrightarrow$}\space},
  float=false,
  numbers=left,
  numberstyle=\tiny\color{gray},
  numbersep=10pt,
  xleftmargin=2em,
  keywordstyle=\color{blue},
  commentstyle=\color{green!60!black},
  stringstyle=\color{purple},
  backgroundcolor=\color{gray!5},
  showstringspaces=false,
  tabsize=2,
  captionpos=b,
  keepspaces=true,
  columns=flexible
}

\pgfplotsset{compat=1.18}
\usetikzlibrary{shapes,arrows,positioning,calc,patterns,decorations.pathmorphing,decorations.markings,arrows.meta}

% Color scheme
\definecolor{headcolor}{RGB}{0,102,204}
\definecolor{keycolor}{RGB}{220,20,60}
\definecolor{solutioncolor}{RGB}{34,139,34}
\definecolor{mnemoniccolor}{RGB}{148,0,211}
\definecolor{codecolor}{RGB}{0,0,100}

% Spacing
\setlength{\parskip}{3pt}
\setlist[itemize]{nosep}
\setlist[enumerate]{nosep}

% Title formatting
\titleformat{\section}{\Large\bfseries\color{headcolor}}{\thesection}{1em}{}
\titleformat{\subsection}{\large\bfseries\color{headcolor}}{\thesubsection}{1em}{}

% Pandoc tightlist compatibility
\providecommand{\tightlist}{%
  \setlength{\itemsep}{0pt}\setlength{\parskip}{0pt}}

% Pandoc longtable compatibility
\newcounter{none}
\def\thenone{}


% content/resources/templates/gujarati-boxes.tex
\usepackage{fontspec}
\usepackage{polyglossia}

% Set Gujarati as main language (document is primarily in Gujarati)
% Note: gloss-gujarati.ldf doesn't exist in polyglossia, but it will use hyphenation patterns
\setdefaultlanguage{gujarati}
\setotherlanguage{english}

% Configure Gujarati font properly
% Use Language=Default to prevent polyglossia from trying to add language-specific features
% that don't exist for Gujarati, which causes "empty feature" warnings
\newfontfamily\gujaratifont[Script=Gujarati,AutoFakeBold=2.5,AutoFakeSlant=0.3]{Noto Sans Gujarati}
\setmainfont[Script=Gujarati,AutoFakeBold=2.5,AutoFakeSlant=0.3]{Noto Sans Gujarati}
% Use Noto Sans Gujarati for monospace to support Gujarati in text
\setmonofont[Scale=0.9]{Noto Sans Gujarati}

% Configure English to use the same font
\newfontfamily\englishfont[Script=Gujarati,AutoFakeBold=2.5,AutoFakeSlant=0.3]{Noto Sans Gujarati}

% Translations for polyglossia
\gappto\captionsgujarati{
  \renewcommand{\tablename}{કોષ્ટક}
  \renewcommand{\figurename}{આકૃતિ}
}

% Helper for TikZ nodes to ensure Gujarati font
\newcommand{\gu}[1]{{\gujaratifont #1}}

% Custom environments
\newtcolorbox{solutionbox}{
    breakable,
    enhanced,
    colback=solutioncolor!5!white,
    colframe=solutioncolor!75!black,
    fonttitle=\bfseries,
    title=જવાબ
}

\newtcolorbox{solutionboxnobreak}{
 colback=solutioncolor!5!white,
 colframe=solutioncolor!75!black,
 fonttitle=\bfseries,
 title=જવાબ
}

\newtcolorbox{keyformula}{
 breakable,
 enhanced,
 colback=keycolor!5!white,
 colframe=keycolor!75!black,
 fonttitle=\bfseries,
 title=રાસાયણિક સમીકરણ/સૂત્ર
}

\newtcolorbox{mnemonicbox}{
 breakable,
 enhanced,
 colback=mnemoniccolor!5!white,
 colframe=mnemoniccolor!75!black,
 fonttitle=\bfseries,
 title=મેમરી ટ્રીક
}


\begin{document}

\begin{center}
{\Huge\bfseries\color{headcolor} Subject Name (Gujarati)}\\[5pt]
{\LARGE 4341105 -- Summer 2023}\\[3pt]
{\large Semester 1 Study Material}\\[3pt]
{\normalsize\textit{Detailed Solutions and Explanations}}
\end{center}

\vspace{10pt}

\subsection*{પ્રશ્ન 1(અ) [3
ગુણ]}\label{uxaaauxab0uxab6uxaa8-1uxa85-3-uxa97uxaa3}

\textbf{નેગેટીવ ફીડબેક એમ્પ્લીફાયરના ફાયદા અને ગેરફાયદા લખો.}

\begin{solutionbox}

{\def\LTcaptype{none} % do not increment counter
\begin{longtable}[]{@{}
  >{\raggedright\arraybackslash}p{(\linewidth - 2\tabcolsep) * \real{0.4444}}
  >{\raggedright\arraybackslash}p{(\linewidth - 2\tabcolsep) * \real{0.5556}}@{}}
\toprule\noalign{}
\begin{minipage}[b]{\linewidth}\raggedright
ફાયદા
\end{minipage} & \begin{minipage}[b]{\linewidth}\raggedright
ગેરફાયદા
\end{minipage} \\
\midrule\noalign{}
\endhead
\bottomrule\noalign{}
\endlastfoot
બેન્ડવિડ્થ વધારે છે & ગેઇન ઘટાડે છે \\
ગેઇન સ્થિર કરે છે & વધારે કોમ્પોનન્ટ્સ જરૂરી પડે છે \\
ડિસ્ટોર્શન ઘટાડે છે & ખર્ચ વધારે છે \\
ઇનપુટ ઇમ્પીડન્સ વધારે છે (વોલ્ટેજ સીરીઝ) & જો યોગ્ય રીતે ડિઝાઇન ન કરવામાં આવે તો
ઓસિલેશન થઈ શકે છે \\
આઉટપુટ ઇમ્પીડન્સ ઘટાડે છે (વોલ્ટેજ સીરીઝ) & કાળજીપૂર્વક ફેઝ કમ્પેન્સેશન જરૂરી છે \\
\end{longtable}
}

\end{solutionbox}
\begin{mnemonicbox}
``GRASS ઊગે પણ ડ્રાય સોઇલ પર'' (Gain Reduction,
Amplifies Stability, Stops distortion, Better impedance)

\end{mnemonicbox}
\subsection*{પ્રશ્ન 1(બ) [4
ગુણ]}\label{uxaaauxab0uxab6uxaa8-1uxaac-4-uxa97uxaa3}

\textbf{નેગેટીવ ફીડબેક એમ્પ્લીફાયરનુ ઓવરઓલ ગેઇન સૂત્ર મેળવો અને નેગેટીવ ફીડબેકની
એપ્લીકેશન જણાવો.}

\begin{solutionbox}

\textbf{નેગેટીવ ફીડબેક સાથે ઓવરઓલ ગેઇનની મેળવણી:}

\begin{verbatim}
flowchart LR
    I[Input] {-{-} S[SummingnPoint]}
    S {-{-} A[AmplifiernA]}
    A {-{-} O[Output]}
    O {-{-} F[FeedbacknNetwork β]}
    F {-{-} S}
\end{verbatim}

\begin{itemize}
\tightlist
\item
  એમ્પ્લીફાયર ગેઇન A અને ફીડબેક ફેક્ટર β માટે:

  \begin{itemize}
  \tightlist
  \item
    ઇનપુટ સિગ્નલ = Vin
  \item
    ફીડબેક સિગ્નલ = βVout
  \item
    એમ્પ્લીફાયરમાં વાસ્તવિક ઇનપુટ = Vin - βVout
  \item
    આઉટપુટ = A(Vin - βVout)
  \item
    આથી, Vout = A(Vin - βVout)
  \item
    Vout + AβVout = AVin
  \item
    Vout(1 + Aβ) = AVin
  \item
    \textbf{ઓવરઓલ ગેઇન = Vout/Vin = A/(1 + Aβ)}
  \end{itemize}
\end{itemize}

\textbf{નેગેટીવ ફીડબેકની એપ્લીકેશન:}

\begin{itemize}
\tightlist
\item
  ઓપરેશનલ એમ્પ્લીફાયર
\item
  વોલ્ટેજ રેગ્યુલેટર્સ
\item
  ઓડિયો એમ્પ્લીફાયર્સ
\item
  ઇન્સ્ટ્રુમેન્ટેશન એમ્પ્લીફાયર્સ
\end{itemize}

\end{solutionbox}
\begin{mnemonicbox}
``AVOI'' (Amplifiers, Voltage regulators,
Oscillation control, Instrumentation)

\end{mnemonicbox}
\subsection*{પ્રશ્ન 1(ક) [7
ગુણ]}\label{uxaaauxab0uxab6uxaa8-1uxa95-7-uxa97uxaa3}

\textbf{કરંટ શન્ટ નેગેટીવ ફીડબેક એમ્પ્લીફાયર દોરી ને સમજાવો અને ઈનપુટ અને આઉટપુટ
ઈમ્પપીડન્સ નું સૂત્ર મેળવો.}

\begin{solutionbox}

\textbf{કરંટ શન્ટ નેગેટીવ ફીડબેક એમ્પ્લીફાયર:}

\begin{verbatim}
flowchart LR
    I[Input] {-{-} S[CurrentnSampling]}
    S {-{-} A[Amplifier]}
    A {-{-} O[Output]}
    O {-{-} F[FeedbacknNetwork]}
    F {-{-}|Feedback Current| S}
\end{verbatim}

કરંટ શન્ટ ફીડબેકમાં, આઉટપુટ વોલ્ટેજનું સેમ્પલિંગ કરવામાં આવે છે અને તેને કરંટમાં રૂપાંતરિત
કરીને ઇનપુટ કરંટમાંથી બાદ કરવામાં આવે છે.

\textbf{સર્કિટ ડાયાગ્રામ:}

\begin{verbatim}
                     +Vcc
                       |
                       R
                       |
                       |
  Iin  o{-{-}{-}{-}+{-}{-}{-}{-}{-}{-}{-}{-}{-}{-}|{-}{-}{-}{-}{-}{-}{-}o Vout}
            |          |
            |          |
           Zin        Zo
            |          |
            |          |
      +{-{-}{-}{-}{-}+          |}
      |                |
   Feedback            |
   Network(β)          |
      |                |
      +{-{-}{-}{-}{-}{-}{-}{-}{-}{-}{-}{-}{-}{-}{-}{-}+}
            |
            |
           GND
\end{verbatim}

\textbf{લાક્ષણિકતાઓ:}

\begin{itemize}
\tightlist
\item
  \textbf{ફીડબેક પ્રકાર}: ઇનપુટ પર કરંટ સેમ્પલિંગ, ઇનપુટ પર શન્ટ મિક્સિંગ
\item
  \textbf{સેમ્પલ્સ}: આઉટપુટ વોલ્ટેજ
\item
  \textbf{ફીડબેક ટુ}: ઇનપુટ કરંટ
\end{itemize}

\textbf{ઇનપુટ ઇમ્પીડન્સનું સૂત્ર:}

\begin{itemize}
\tightlist
\item
  ફીડબેક વિના: Zin
\item
  કરંટ શન્ટ ફીડબેક સાથે: Zin' = Zin/(1 + Aβ)
\item
  \textbf{આથી, ઇનપુટ ઇમ્પીડન્સ (1 + Aβ) ફેક્ટર દ્વારા ઘટે છે}
\end{itemize}

\textbf{આઉટપુટ ઇમ્પીડન્સનું સૂત્ર:}

\begin{itemize}
\tightlist
\item
  ફીડબેક વિના: Zo
\item
  કરંટ શન્ટ ફીડબેક સાથે: Zo' = Zo/(1 + Aβ)
\item
  \textbf{આથી, આઉટપુટ ઇમ્પીડન્સ (1 + Aβ) ફેક્ટર દ્વારા ઘટે છે}
\end{itemize}

\end{solutionbox}
\begin{mnemonicbox}
``DISCO'' (Decreased Impedances with Shunt Current
Operation)

\end{mnemonicbox}
\subsection*{પ્રશ્ન 1(ક) OR [7
ગુણ]}\label{uxaaauxab0uxab6uxaa8-1uxa95-or-7-uxa97uxaa3}

\textbf{વોલ્ટેજ સીરીઝ નેગેટીવ ફીડબેક એમ્પ્લીફાયર દોરી ને સમજાવો અને ઈનપુટ અને આઉટપુટ
ઈમ્પપીડન્સ નું સૂત્ર મેળવો.}

\begin{solutionbox}

\textbf{વોલ્ટેજ સીરીઝ નેગેટીવ ફીડબેક એમ્પ્લીફાયર:}

\begin{verbatim}
flowchart LR
    I[Input] {-{-} S[VoltagenSampling]}
    S {-{-} A[Amplifier]}
    A {-{-} O[Output]}
    O {-{-} F[FeedbacknNetwork β]}
    F {-{-}|Feedback Voltage| S}
\end{verbatim}

વોલ્ટેજ સીરીઝ ફીડબેકમાં, આઉટપુટ વોલ્ટેજનું સેમ્પલિંગ કરવામાં આવે છે અને તેને ઇનપુટ વોલ્ટેજ
સાથે સીરીઝમાં ફીડબેક કરવામાં આવે છે.

\textbf{સર્કિટ ડાયાગ્રામ:}

\begin{verbatim}
                     +Vcc
                       |
                       R
                       |
                       |
  Vin  o{-{-}+{-}{-}{-}{-}{-}{-}{-}+{-}{-}{-}{-}+{-}{-}{-}{-}{-}{-}{-}o Vout}
          |       |    |
          Z       |    Z
          i       |    o
          n       |    |
          |       |    |
          +{-{-}{-}+   |    |}
              |   |    |
              |   |    |
           Feedback    |
           Network(β)  |
              |        |
              +{-{-}{-}{-}{-}{-}{-}{-}+}
              |
             GND
\end{verbatim}

\textbf{લાક્ષણિકતાઓ:}

\begin{itemize}
\tightlist
\item
  \textbf{ફીડબેક પ્રકાર}: આઉટપુટ પર વોલ્ટેજ સેમ્પલિંગ, ઇનપુટ પર સીરીઝ મિક્સિંગ
\item
  \textbf{સેમ્પલ્સ}: આઉટપુટ વોલ્ટેજ
\item
  \textbf{ફીડબેક ટુ}: ઇનપુટ વોલ્ટેજ
\end{itemize}

\textbf{ઇનપુટ ઇમ્પીડન્સનું સૂત્ર:}

\begin{itemize}
\tightlist
\item
  ફીડબેક વિના: Zin
\item
  વોલ્ટેજ સીરીઝ ફીડબેક સાથે: Zin' = Zin \times (1 + Aβ)
\item
  \textbf{આથી, ઇનપુટ ઇમ્પીડન્સ (1 + Aβ) ફેક્ટર દ્વારા વધે છે}
\end{itemize}

\textbf{આઉટપુટ ઇમ્પીડન્સનું સૂત્ર:}

\begin{itemize}
\tightlist
\item
  ફીડબેક વિના: Zo
\item
  વોલ્ટેજ સીરીઝ ફીડબેક સાથે: Zo' = Zo/(1 + Aβ)
\item
  \textbf{આથી, આઉટપુટ ઇમ્પીડન્સ (1 + Aβ) ફેક્ટર દ્વારા ઘટે છે}
\end{itemize}

\end{solutionbox}
\begin{mnemonicbox}
``ISDO'' (Increased input impedance, Series
feedback, Decreased output impedance, Output voltage sampled)

\end{mnemonicbox}
\subsection*{પ્રશ્ન 2(અ) [3
ગુણ]}\label{uxaaauxab0uxab6uxaa8-2uxa85-3-uxa97uxaa3}

\textbf{UJT રીલેક્ષેશન ઓસીલેટરનો સરકીટ ડાયાગ્રામ દોરીને સમજાવો.}

\begin{solutionbox}

\textbf{UJT રીલેક્ષેશન ઓસીલેટર:}

\begin{verbatim}
flowchart TB
    A[UJT Relaxation Oscillator]
    A {-{-}{-} B[RC ચાર્જિંગ સર્કિટnસાથે પલ્સ ઉત્પન્ન કરે છે]}
    A {-{-}{-} C[કેપેસિટર વોલ્ટેજnપીક પર પહોંચે ત્યારે UJT ટ્રિગર થાય છે]}
    A {-{-}{-} D[સિમ્પલ ટાઇમિંગ સર્કિટ]}
\end{verbatim}

\textbf{સર્કિટ ડાયાગ્રામ:}

\begin{verbatim}
    +Vcc
     |
     R1
     |
     +{-{-}{-}{-}{-}{-}+}
     |      |
     |      |
     |     B2
     |      |   UJT
     |      |
  C1 {-{-}{-} E  |}
     |      |
     |     B1
     |      |
     |      |
     +{-{-}{-}{-}{-}{-}+}
     |
    GND
\end{verbatim}

આ સર્કિટમાં:

\begin{itemize}
\tightlist
\item
  C1 ચાર્જ થાય છે R1 દ્વારા
\item
  જ્યારે કેપેસિટર વોલ્ટેજ UJT ના પીક પોઇન્ટ સુધી પહોંચે છે, UJT ચાલુ થાય છે
\item
  કેપેસિટર UJT દ્વારા ઝડપથી ડિસ્ચાર્જ થાય છે
\item
  પ્રક્રિયા પુનરાવર્તિત થાય છે અને ઓસિલેશન ઉત્પન્ન થાય છે
\end{itemize}

\end{solutionbox}
\begin{mnemonicbox}
``CURD'' (Capacitor charges Until Reaching Discharge
point)

\end{mnemonicbox}
\subsection*{પ્રશ્ન 2(અ) OR [3
ગુણ]}\label{uxaaauxab0uxab6uxaa8-2uxa85-or-3-uxa97uxaa3}

\textbf{હાર્ટલી ઓસીલેટર દોરી ને સમજાવો.}

\begin{solutionbox}

\textbf{હાર્ટલી ઓસીલેટર:}

\begin{verbatim}
flowchart TB
    A[હાર્ટલી ઓસીલેટર] {-{-} B[ટેપ્ડ ઇન્ડક્ટર વાપરે છે]}
    A {-{-} C[LC ટેન્ક સર્કિટ]}
    A {-{-} D[RF એપ્લિકેશન્સ]}
\end{verbatim}

\textbf{સર્કિટ ડાયાગ્રામ:}

\begin{verbatim}
                +Vcc
                  |
                  R
                  |
    +{-{-}{-}{-}{-}{-}{-}{-}{-}{-}{-}{-}{-}+{-}{-}{-}{-}{-}{-}{-}{-}{-}{-}{-}{-}{-}+}
    |             |             |
    |             C3            |
    |             |             |
    |      +{-{-}{-}{-}{-}{-}+{-}{-}{-}{-}{-}{-}+      |}
    |      |      |      |      |
    |     C1      L1     L2     |
    |      |      |      |      |
    |      +{-{-}{-}{-}{-}{-}+{-}{-}{-}{-}{-}{-}+      |}
    |                           |
    +{-{-}{-}{-}{-}{-}{-}{-}{-}{-}{-}{-}{-}{-}{-}{-}{-}{-}{-}{-}{-}{-}{-}{-}{-}{-}{-}+}
                |
               GND
\end{verbatim}

\textbf{કાર્યપ્રણાલી:}

\begin{itemize}
\tightlist
\item
  LC ટેન્ક સર્કિટ સાથે ટેપ્ડ ઇન્ડક્ટર (L1 અને L2) વાપરે છે
\item
  ટ્રાન્ઝિસ્ટર એમ્પ્લિફાય કરે છે અને ટેન્ક સર્કિટને ઊર્જા પૂરી પાડે છે
\item
ઓસિલેશન ફ્રીક્વન્સી:

f = 1/[2π\sqrt(L\timesC)] જ્યાં

L = L1 + L2

\item
  ઇન્ડક્ટિવ કપલિંગ દ્વારા ફીડબેક
\end{itemize}

\end{solutionbox}
\begin{mnemonicbox}
``TIC'' (Tapped inductor Circuit)

\end{mnemonicbox}
\subsection*{પ્રશ્ન 2(બ) [4
ગુણ]}\label{uxaaauxab0uxab6uxaa8-2uxaac-4-uxa97uxaa3}

\textbf{કોલપીટ ઓસીલેટરનો સરકીટ ડાયાગ્રામ દોરો અને વિસ્તૃત માં સમજાવો. તેના
ફાયદા અને ગેરફાયદા પણ જણાવો.}

\begin{solutionbox}

\textbf{કોલપીટ્સ ઓસીલેટર:}

\textbf{સર્કિટ ડાયાગ્રામ:}

\begin{verbatim}
                  +Vcc
                    |
                    RL
                    |
                    |
    +{-{-}{-}{-}{-}{-}{-}{-}{-}{-}{-}{-}{-}{-}{-}+{-}{-}{-}{-}{-}{-}{-}{-}{-}{-}{-}{-}{-}{-}{-}+}
    |               |               |
    |              C3               |
    |               |               |
    |              C1               |
    |              C2               |
    |               +{-{-}{-}+           |}
    |               |   |           |
    +{-{-}{-}+           +{-}{-}{-}+           |}
    |   |{-{-}{-}{-}+      |   |           |}
    +{-{-}{-}+    |      +{-}{-}{-}+           |}
              |                     |
              +{-{-}{-}{-}{-}{-}{-}{-}{-}{-}{-}{-}{-}{-}{-}{-}{-}{-}{-}{-}{-}+}
              |
             GND
\end{verbatim}

\textbf{કાર્યપ્રણાલી:}

\begin{itemize}
\tightlist
\item
  કેપેસિટિવ વોલ્ટેજ ડિવાઇડર (C1 અને C2) સાથે LC ટેન્ક સર્કિટ વાપરે છે
\item
  ટ્રાન્ઝિસ્ટર એમ્પ્લિફાય કરે છે અને ટેન્ક સર્કિટને ઊર્જા પૂરી પાડે છે
\item
  ઓસિલેશન ફ્રીક્વન્સી: f = 1/[2π\sqrt(L\times(C1\timesC2)/(C1+C2))]
\end{itemize}

{\def\LTcaptype{none} % do not increment counter
\begin{longtable}[]{@{}ll@{}}
\toprule\noalign{}
ફાયદા & ગેરફાયદા \\
\midrule\noalign{}
\endhead
\bottomrule\noalign{}
\endlastfoot
સારી ફ્રીક્વન્સી સ્થિરતા & બે કેપેસિટર (C1, C2) જરૂરી છે \\
ઉચ્ચ ફ્રીક્વન્સી પર સારું કામ કરે છે & અન્ય ઓસિલેટર કરતાં ટ્યુન કરવું વધુ મુશ્કેલ છે \\
ઓછા હાર્મોનિક્સ & ટ્રાન્ઝિસ્ટર પેરામીટર્સ પ્રત્યે સંવેદનશીલ \\
સરળ ડિઝાઇન & સીમિત ફ્રીક્વન્સી રેન્જ \\
\end{longtable}
}

\end{solutionbox}
\begin{mnemonicbox}
``FAST Circuits'' (Frequency stable, Appropriate for
high frequencies, Simple design, Two capacitors needed)

\end{mnemonicbox}
\subsection*{પ્રશ્ન 2(બ) OR [4
ગુણ]}\label{uxaaauxab0uxab6uxaa8-2uxaac-or-4-uxa97uxaa3}

\textbf{વિએન બ્રીજ ઓસીલેટર દોરીને સમજાવો.}

\begin{solutionbox}

\textbf{વિએન બ્રીજ ઓસીલેટર:}

\begin{verbatim}
flowchart TD
    A[વિએન બ્રીજ ઓસીલેટર] {-{-} B[RC નેટવર્ક વાપરે છે]}
    A {-{-} C[ઓડિયો ફ્રીક્વન્સી રેન્જ]}
    A {-{-} D[ઓછું ડિસ્ટોર્શન]}
    A {-{-} E[સ્થાયી આઉટપુટ]}
\end{verbatim}

\textbf{સર્કિટ ડાયાગ્રામ:}

\begin{verbatim}
          +{-{-}{-}{-}{-}{-}{-}{-}+{-}{-}{-}{-}{-}{-}{-}{-}+}
          |        |        |
          |        R1       |
          |        |        |
     C1   |        |        |    R3
    ||{-{-}{-}{-}+        +{-}{-}{-}///{-}{-}{-}{-}+}
    ||    |        |             |
          |        |             |
          |        |             |
     R2   |       Op{-Amp         |}
    /{/{-}{-}+        |             |}
          |        |             |
          |        |             |
    +{-{-}{-}{-}{-}+        +{-}{-}{-}{-}{-}{-}{-}{-}{-}{-}{-}{-}{-}+}
    |     |        |             |
    C2    |        |             |
    |     |        |       R4    |
    +{-{-}{-}{-}{-}+{-}{-}{-}{-}{-}{-}{-}{-}+{-}{-}{-}{-}{-}{-}///{-}+}
          |
         GND
\end{verbatim}

\textbf{કાર્યપ્રણાલી:}

\begin{itemize}
\tightlist
\item
  ફ્રીક્વન્સી-સિલેક્ટિવ ફીડબેક તરીકે RC વિએન બ્રીજ નેટવર્ક વાપરે છે
\item
  સૌથી સરળ ડિઝાઇન માટે R1=R2 અને C1=C2
\item
  ઓસિલેશન ફ્રીક્વન્સી: f = 1/(2πRC)
\item
  સતત ઓસિલેશન માટે ગેઇન \geq 3 હોવું જરૂરી છે
\item
  ઓછા ડિસ્ટોર્શન સાથે ઓડિયો ફ્રીક્વન્સી જનરેશન માટે વપરાય છે
\end{itemize}

\end{solutionbox}
\begin{mnemonicbox}
``FEAR'' (Frequency selective, Equal RC components,
Audio range, Reduced distortion)

\end{mnemonicbox}
\subsection*{પ્રશ્ન 2(ક) [7
ગુણ]}\label{uxaaauxab0uxab6uxaa8-2uxa95-7-uxa97uxaa3}

\textbf{Crystal ઓસીલેટર સમજાવો.}

\begin{solutionbox}

\textbf{ક્રિસ્ટલ ઓસીલેટર:}

\begin{verbatim}
flowchart TD
    A[ક્રિસ્ટલ ઓસીલેટર] {-{-} B[પિઝોઇલેક્ટ્રિક ક્રિસ્ટલ વાપરે છે]}
    A {-{-} C[અત્યંત સ્થાયી ફ્રીક્વન્સી]}
    A {-{-} D[ઉચ્ચ Q ફેક્ટર]}
    A {-{-} E[સચોટ ટાઇમિંગ એપ્લિકેશન્સ]}
\end{verbatim}

\textbf{સર્કિટ ડાયાગ્રામ:}

\begin{verbatim}
                +Vcc
                  |
                  RL
                  |
    +{-{-}{-}{-}{-}{-}{-}{-}{-}{-}{-}{-}{-}+{-}{-}{-}{-}{-}{-}{-}{-}{-}{-}{-}{-}{-}+}
    |             |             |
    |            C3             |
    |             |             |
    |      +{-{-}{-}{-}{-}{-}+{-}{-}{-}{-}{-}{-}+      |}
    |      |      |      |      |
    |     C1    XTAL     C2     |
    |      |      |      |      |
    |      +{-{-}{-}{-}{-}{-}+{-}{-}{-}{-}{-}{-}+      |}
    |                           |
    +{-{-}{-}{-}{-}{-}{-}{-}{-}{-}{-}{-}{-}{-}{-}{-}{-}{-}{-}{-}{-}{-}{-}{-}{-}{-}{-}+}
                |
               GND
\end{verbatim}

\textbf{કાર્યપ્રણાલી સિદ્ધાંત:}

\begin{itemize}
\tightlist
\item
  ક્વાર્ટ્ઝ ક્રિસ્ટલના પિઝોઇલેક્ટ્રિક ઇફેક્ટ પર આધારિત છે
\item
  જ્યારે વોલ્ટેજ લાગુ કરવામાં આવે ત્યારે ક્રિસ્ટલ તેની કુદરતી રેઝોનન્ટ ફ્રીક્વન્સી પર કંપન
  કરે છે
\item
  અત્યંત ઊંચા Q ફેક્ટર સાથે ખૂબ જ સ્થાયી રેઝોનેટર તરીકે કામ કરે છે
\item
  સચોટ ફ્રીક્વન્સી પર ફીડબેક પ્રદાન કરે છે
\end{itemize}

\textbf{લાક્ષણિકતાઓ:}

\begin{itemize}
\tightlist
\item
  \textbf{રેઝોનન્ટ ફ્રીક્વન્સી}: ક્રિસ્ટલ કટ અને પરિમાણો દ્વારા નક્કી થાય છે
\item
  \textbf{Q ફેક્ટર}: સામાન્ય રીતે 10,000-100,000 (LC સર્કિટ્સ કરતાં ઘણું વધારે)
\item
  \textbf{ફ્રીક્વન્સી સ્થિરતા}: સામાન્ય રીતે 0.001\% થી 0.01\%
\item
  \textbf{તાપમાન કોએફિશિયન્ટ}: સામાન્ય રીતે ઓછો, ઝીરો તાપમાન કોએફિશિયન્ટ માટે
  વિશેષ રીતે કાપી શકાય છે
\end{itemize}

\textbf{એપ્લિકેશન્સ:}

\begin{itemize}
\tightlist
\item
  કમ્પ્યુટર્સમાં ક્લોક જનરેશન
\item
  ફ્રીક્વન્સી સ્ટાન્ડર્ડ્સ
\item
  રેડિયો ટ્રાન્સમિટર/રિસીવર
\item
  ડિજિટલ ઘડિયાળ અને ક્લોક્સ
\item
  માઇક્રોકન્ટ્રોલર ટાઇમિંગ
\end{itemize}

\end{solutionbox}
\begin{mnemonicbox}
``STOP Precisely'' (Stable, Temperature-resistant,
Oscillates, Piezoelectric, Precisely)

\end{mnemonicbox}
\subsection*{પ્રશ્ન 2(ક) OR [7
ગુણ]}\label{uxaaauxab0uxab6uxaa8-2uxa95-or-7-uxa97uxaa3}

\textbf{UJT નું સ્ટ્રક્ચર, સીમ્બોલ, એક્વીવેલેન્ટ સરકીટ દોરો અને સમજાવો.}

\begin{solutionbox}

\textbf{યુનિજંક્શન ટ્રાન્ઝિસ્ટર (UJT):}

\textbf{સ્ટ્રક્ચર:}

\begin{verbatim}
              Base 2 (B2)
                  |
                  |
                  v
                +{-{-}{-}+}
                |   |
                |   |
                |   |
                |   |
                |   |
                |   |
                |   |
                |   |
   Emitter (E) {|   | Base 1 (B1)}
                +{-{-}{-}+}
\end{verbatim}

\textbf{સિમ્બોલ:}

\begin{verbatim}
                  B2
                  |
                  |
                  |
                  o
                 /|
                / |
               /  |
              /   |
     E o{-{-}{-}{-}o     |}
              {   |}
               {  |}
                { |}
                 {|}
                  o
                  |
                  |
                  |
                  B1
\end{verbatim}

\textbf{એક્વિવેલેન્ટ સર્કિટ:}

\begin{verbatim}
                B2
                 |
                 |
                 R
                /{/}
                 |
                 |
      E o{-{-}{-}+{-}{-}{-}||{-}{-}{-}+ B1}
             |         |
             |         |
             R         |
            /{/       |}
             |         |
             +{-{-}{-}{-}{-}{-}{-}{-}{-}+}
\end{verbatim}

\textbf{કાર્યપ્રણાલી સિદ્ધાંત:}

\begin{itemize}
\tightlist
\item
  UJT એ એક એમિટર અને બે બેઝ સાથેનું ત્રણ-ટર્મિનલ ડિવાઇસ છે
\item
  P-ટાઇપ એમિટર જંક્શન સાથે N-ટાઇપ સિલિકોન બાર
\item
  આંતરિક રેસિસ્ટન્સ RB1 અને RB2 સાથે વોલ્ટેજ ડિવાઇડર બનાવે છે
\item
  એમિટર કરંટ વહેવાનું શરૂ થાય છે જ્યારે VE \textgreater{} η\timesVBB + VD
\item
  જ્યાં η ઇન્ટ્રિન્સિક સ્ટેન્ડઓફ રેશિયો = RB1/(RB1+RB2)
\end{itemize}

\textbf{લાક્ષણિકતાઓ:}

\begin{itemize}
\tightlist
\item
  \textbf{ઇન્ટ્રિન્સિક સ્ટેન્ડઓફ રેશિયો (η)}: સામાન્ય રીતે 0.5 થી 0.8
\item
  \textbf{નેગેટિવ રેઝિસ્ટન્સ રીજન}: વોલ્ટેજ ઘટે છે ત્યારે કરંટ વધે છે
\item
  \textbf{પીક પોઇન્ટ}: નેગેટિવ રેઝિસ્ટન્સ રીજનની શરૂઆત
\item
  \textbf{વેલી પોઇન્ટ}: નેગેટિવ રેઝિસ્ટન્સ રીજનનો અંત
\end{itemize}

\textbf{એપ્લિકેશન્સ:}

\begin{itemize}
\tightlist
\item
  રિલેક્ઝેશન ઓસિલેટર્સ
\item
  ટાઇમિંગ સર્કિટ્સ
\item
  ટ્રિગર જનરેટર્સ
\item
  SCR ટ્રિગરિંગ સર્કિટ્સ
\item
  સૉટૂથ જનરેટર્સ
\end{itemize}

\end{solutionbox}
\begin{mnemonicbox}
``NEVER'' (Negative resistance, Emitter-triggered,
Valley and peak points, Easily timed, Relaxation oscillator)

\end{mnemonicbox}
\subsection*{પ્રશ્ન 3(અ) [3
ગુણ]}\label{uxaaauxab0uxab6uxaa8-3uxa85-3-uxa97uxaa3}

\textbf{વોલ્ટેજ અને પાવર એમ્પ્લીફાયર વચ્ચેનો તફાવત સમજાવો.}

\begin{solutionbox}

{\def\LTcaptype{none} % do not increment counter
\begin{longtable}[]{@{}lll@{}}
\toprule\noalign{}
પેરામીટર & વોલ્ટેજ એમ્પ્લીફાયર & પાવર એમ્પ્લીફાયર \\
\midrule\noalign{}
\endhead
\bottomrule\noalign{}
\endlastfoot
ઉદ્દેશ & વોલ્ટેજને એમ્પ્લિફાય કરે છે & લોડને પાવર પહોંચાડે છે \\
આઉટપુટ ઇમ્પીડન્સ & ઊંચી & નીચી \\
ઇનપુટ ઇમ્પીડન્સ & ઊંચી & તુલનાત્મક રીતે નીચી \\
કાર્યક્ષમતા & મહત્વપૂર્ણ નથી & ખૂબ મહત્વપૂર્ણ છે \\
હીટ ડિસિપેશન & ઓછી & ઊંચી (હીટ સિંક જરૂરી) \\
સર્કિટમાં સ્થાન & શરૂઆતના તબક્કામાં & છેલ્લા તબક્કામાં \\
\end{longtable}
}

\end{solutionbox}
\begin{mnemonicbox}
``PEHIP'' (Power for Efficiency and Heat, Impedance
matters, Position differs)

\end{mnemonicbox}
\subsection*{પ્રશ્ન 3(અ) OR [3
ગુણ]}\label{uxaaauxab0uxab6uxaa8-3uxa85-or-3-uxa97uxaa3}

\textbf{વ્યાખ્યા આપો: 1) Efficiency 2) Distortion 3) Power dissipation
capability}

\begin{solutionbox}

{\def\LTcaptype{none} % do not increment counter
\begin{longtable}[]{@{}
  >{\raggedright\arraybackslash}p{(\linewidth - 2\tabcolsep) * \real{0.3333}}
  >{\raggedright\arraybackslash}p{(\linewidth - 2\tabcolsep) * \real{0.6667}}@{}}
\toprule\noalign{}
\begin{minipage}[b]{\linewidth}\raggedright
શબ્દ
\end{minipage} & \begin{minipage}[b]{\linewidth}\raggedright
વ્યાખ્યા
\end{minipage} \\
\midrule\noalign{}
\endhead
\bottomrule\noalign{}
\endlastfoot
\textbf{Efficiency} & લોડને પહોંચાડવામાં આવતી AC આઉટપુટ પાવરનો સપ્લાયમાંથી
લેવામાં આવતી DC ઇનપુટ પાવર સાથેનો ગુણોત્તર. ગાણિતિક રીતે: η = (Pout/Pin) \times
100\%. ઉચ્ચ કાર્યક્ષમતા એટલે ઓછી પાવર ગરમી તરીકે વેડફાય છે. \\
\textbf{Distortion} & ઇનપુટ વેવફોર્મની તુલનામાં આઉટપુટ વેવફોર્મમાં અનિચ્છનીય
ફેરફાર. Total Harmonic Distortion (THD) તરીકે માપવામાં આવે છે. હાર્મોનિક,
ઇન્ટરમોડ્યુલેશન, ક્રોસઓવર અને એમ્પ્લિટ્યુડ ડિસ્ટોર્શન શામેલ છે. \\
\textbf{Power Dissipation Capability} & નુકસાન વિના એમ્પ્લિફાયર દ્વારા વેડફી
શકાતી મહત્તમ પાવર. હીટ સિંક, થર્મલ રેઝિસ્ટન્સ અને ટ્રાન્ઝિસ્ટરના મહત્તમ જંક્શન તાપમાન
પર આધાર રાખે છે. \\
\end{longtable}
}

\end{solutionbox}
\begin{mnemonicbox}
``EDP'' (Efficiency converts, Distortion deforms,
Power capability protects)

\end{mnemonicbox}
\subsection*{પ્રશ્ન 3(બ) [4
ગુણ]}\label{uxaaauxab0uxab6uxaa8-3uxaac-4-uxa97uxaa3}

\textbf{ક્લાસ-બી પુશ પુલ પાવર એમ્પ્લીફાયર સમજાવો.}

\begin{solutionbox}

\textbf{ક્લાસ-B પુશ-પુલ એમ્પ્લિફાયર:}

\begin{verbatim}
flowchart TB
    A[ક્લાસ{-B પુશ{-}પુલ] {-}{-} B[બે ટ્રાન્ઝિસ્ટર વાપરે છે]}
    A {-{-} C[દરેક અર્ધ સાયકલ સંભાળે છે]}
    A {-{-} D[ઊંચી કાર્યક્ષમતા 78\%]}
    A {-{-} E[ક્રોસઓવર ડિસ્ટોર્શન]}
\end{verbatim}

\textbf{સર્કિટ ડાયાગ્રામ:}

\begin{verbatim}
         +Vcc
           |
           |
        +{-{-}+{-}{-}+}
        |     |
        Q1    |
        |     |
Input o{-+     +{-}{-}{-}o Output}
        |     |
        Q2    |
        |     |
        +{-{-}+{-}{-}+}
           |
           |
          GND
\end{verbatim}

\textbf{કાર્યપ્રણાલી:}

\begin{itemize}
\tightlist
\item
  બે કોમ્પ્લિમેન્ટરી ટ્રાન્ઝિસ્ટરનો ઉપયોગ કરે છે
\item
  Q1 પોઝિટિવ અર્ધ-સાયકલ દરમિયાન કન્ડક્ટ કરે છે
\item
  Q2 નેગેટિવ અર્ધ-સાયકલ દરમિયાન કન્ડક્ટ કરે છે
\item
  દરેક ટ્રાન્ઝિસ્ટર ઇનપુટ સાયકલના 180^\circ માટે કન્ડક્ટ કરે છે
\item
  સૈદ્ધાંતિક કાર્યક્ષમતા: 78.5\%
\end{itemize}

\end{solutionbox}
\begin{mnemonicbox}
``ECHO'' (Efficiency high, Crossover distortion,
Half-cycle operation, Output high power)

\end{mnemonicbox}
\subsection*{પ્રશ્ન 3(બ) OR [4
ગુણ]}\label{uxaaauxab0uxab6uxaa8-3uxaac-or-4-uxa97uxaa3}

\textbf{ઓપરેશન મોડ નાં આધારે પાવર એમ્પ્લીફાયરનું વર્ગીકરણ કરો અને વિવિધ પ્રકારના
પાવર એમ્પ્લીફાયરની કામગીરી સમજાવો.}

\begin{solutionbox}

\textbf{પાવર એમ્પ્લિફાયરનું વર્ગીકરણ:}

\begin{verbatim}
flowchart TB
    A[પાવર એમ્પ્લિફાયર્સ] {-{-} B[ક્લાસ A]}
    A {-{-} C[ક્લાસ B]}
    A {-{-} D[ક્લાસ AB]}
    A {-{-} E[ક્લાસ C]}
\end{verbatim}

{\def\LTcaptype{none} % do not increment counter
\begin{longtable}[]{@{}
  >{\raggedright\arraybackslash}p{(\linewidth - 4\tabcolsep) * \real{0.2059}}
  >{\raggedright\arraybackslash}p{(\linewidth - 4\tabcolsep) * \real{0.5294}}
  >{\raggedright\arraybackslash}p{(\linewidth - 4\tabcolsep) * \real{0.2647}}@{}}
\toprule\noalign{}
\begin{minipage}[b]{\linewidth}\raggedright
ક્લાસ
\end{minipage} & \begin{minipage}[b]{\linewidth}\raggedright
કન્ડક્શન એંગલ
\end{minipage} & \begin{minipage}[b]{\linewidth}\raggedright
કાર્યપ્રણાલી
\end{minipage} \\
\midrule\noalign{}
\endhead
\bottomrule\noalign{}
\endlastfoot
\textbf{ક્લાસ A} & 360^\circ & એમ્પ્લિફાયર સંપૂર્ણ ઇનપુટ સાયકલ માટે કન્ડક્ટ કરે છે. આઉટપુટ
સિગ્નલ ઇનપુટની સચોટ પ્રતિકૃતિ હોય છે પરંતુ એમ્પ્લિફાય થયેલી. લિનિયર પરંતુ અકાર્યક્ષમ
(25-30\%). \\
\textbf{ક્લાસ B} & 180^\circ & બે ટ્રાન્ઝિસ્ટર દરેક અર્ધ સાયકલ માટે કન્ડક્ટ કરે છે. એક
પોઝિટિવ અર્ધ, બીજો નેગેટિવ અર્ધ સંભાળે છે. વધુ કાર્યક્ષમ (70-80\%) પરંતુ ક્રોસઓવર
ડિસ્ટોર્શન છે. \\
\textbf{ક્લાસ AB} & 180^\circ-360^\circ & ક્લાસ A અને B વચ્ચેનો સમાધાન. ક્રોસઓવર
ડિસ્ટોર્શન ઘટાડવા માટે થોડું બાયસ. સારી કાર્યક્ષમતા (50-70\%) સાથે સ્વીકાર્ય
ડિસ્ટોર્શન. \\
\textbf{ક્લાસ C} & \textless180^\circ & અર્ધ સાયકલથી ઓછા સમય માટે કન્ડક્ટ કરે છે. ખૂબ
કાર્યક્ષમ (\textgreater80\%) પરંતુ અત્યંત ડિસ્ટોર્ટેડ. મુખ્યત્વે RF ટ્યૂન્ડ એમ્પ્લિફાયર્સમાં
વપરાય છે. \\
\end{longtable}
}

\end{solutionbox}
\begin{mnemonicbox}
``ABCE'' (A-all cycle, B-both halves separately,
C-compromise solution, E-efficiency with distortion)

\end{mnemonicbox}
\subsection*{પ્રશ્ન 3(ક) [7
ગુણ]}\label{uxaaauxab0uxab6uxaa8-3uxa95-7-uxa97uxaa3}

\textbf{Complementary symmetry પુશ પુલ પાવર એમ્પ્લીફાયર દોરી ને સમજાવો અને તેના
ગેરફાયદા લખો.}

\begin{solutionbox}

\textbf{કોમ્પ્લિમેન્ટરી સિમેટ્રી પુશ-પુલ એમ્પ્લિફાયર:}

\begin{verbatim}
flowchart TD
    A[કોમ્પ્લિમેન્ટરી પુશ{-પુલ] {-}{-} B[NPN અને PNP વાપરે છે]}
    A {-{-} C[ડાયરેક્ટ કપલિંગ શક્ય]}
    A {-{-} D[સેન્ટર{-}ટેપ્ડ ટ્રાન્સફોર્મર જરૂરી નથી]}
    A {-{-} E[ક્લાસ B અથવા AB ઓપરેશન]}
\end{verbatim}

\textbf{સર્કિટ ડાયાગ્રામ:}

\begin{verbatim}
               +Vcc
                |
                |
                Q1 (NPN)
                |
   +{-{-}{-}{-}{-}+{-}{-}{-}{-}{-}{-}+{-}{-}{-}{-}{-}+}
   |     |            |
   |     |            |
   |     |            |
Input   Bias       Output
   |     |            |
   |     |            |
   +{-{-}{-}{-}{-}+{-}{-}{-}{-}{-}{-}+{-}{-}{-}{-}{-}+}
                |
                |
                Q2 (PNP)
                |
                |
               GND
\end{verbatim}

\textbf{કાર્યપ્રણાલી:}

\begin{itemize}
\tightlist
\item
  કોમ્પ્લિમેન્ટરી પેર (NPN અને PNP ટ્રાન્ઝિસ્ટર) વાપરે છે
\item
  સેન્ટર-ટેપ્ડ ટ્રાન્સફોર્મરની જરૂર નથી
\item
  NPN પોઝિટિવ અર્ધ-સાયકલ સંભાળે છે
\item
  PNP નેગેટિવ અર્ધ-સાયકલ સંભાળે છે
\item
  બાયસિંગ નેટવર્ક ક્રોસઓવર ડિસ્ટોર્શન ઘટાડે છે
\item
  સ્પીકર સાથે ડાયરેક્ટ કપલિંગ શક્ય છે
\end{itemize}

\textbf{ગેરફાયદા:}

\begin{itemize}
\tightlist
\item
  યોગ્ય રીતે બાયસ ન થાય તો થર્મલ રનવે
\item
  કોમ્પ્લિમેન્ટરી મેચ્ડ ટ્રાન્ઝિસ્ટર જરૂરી છે
\item
  ક્લાસ-B ઓપરેશનમાં ક્રોસઓવર ડિસ્ટોર્શન
\item
  પોઝિટિવ અને નેગેટિવ બંને પાવર સપ્લાય જરૂરી છે
\item
  સચોટ કોમ્પ્લિમેન્ટરી પેર શોધવામાં મુશ્કેલી
\end{itemize}

\end{solutionbox}
\begin{mnemonicbox}
``MATCH Precisely'' (Matched transistors, Avoids
transformers, Thermal issues, Crossover distortion, Heat dissipation
needed)

\end{mnemonicbox}
\subsection*{પ્રશ્ન 3(ક) OR [7
ગુણ]}\label{uxaaauxab0uxab6uxaa8-3uxa95-or-7-uxa97uxaa3}

\textbf{ક્લાસ-બી પુશ પુલ પાવર એમ્પ્લીફાયરનું કાર્યક્ષમતાનું સમીકરણ મેળવો.}

\begin{solutionbox}

\textbf{ક્લાસ-B પુશ-પુલ એમ્પ્લિફાયર કાર્યક્ષમતાની મેળવણી:}

\begin{verbatim}
flowchart TB
    A[ક્લાસ{-B કાર્યક્ષમતા] {-}{-} B[પાવર રેશિયો પર આધારિત]}
    A {-{-} C[દરેક ટ્રાન્ઝિસ્ટર અર્ધ સાયકલ કન્ડક્ટ કરે છે]}
    A {-{-} D[સૈદ્ધાંતિક મેક્સ કાર્યક્ષમતા: 78.5\%]}
\end{verbatim}

\textbf{સર્કિટ ડાયાગ્રામ:}

\begin{verbatim}
         +Vcc
           |
           |
        +{-{-}+{-}{-}+}
        |     |
        Q1    |
        |     |
Input o{-+     +{-}{-}{-}o Output}
        |     |
        Q2    |
        |     |
        +{-{-}+{-}{-}+}
           |
           |
          GND
\end{verbatim}

\textbf{કાર્યક્ષમતા ગણતરી:}

\begin{enumerate}
\tightlist
\item
  \textbf{DC પાવર ઇનપુટ ગણતરી:}

  \begin{itemize}
  \tightlist
  \item
    દરેક ટ્રાન્ઝિસ્ટર અર્ધ સાયકલ માટે કન્ડક્ટ કરે છે
  \item
    એવરેજ DC કરંટ: Idc = Imax/π
  \item
    DC પાવર ઇનપુટ: Pdc = Vcc \times Idc = Vcc \times Imax/π
  \end{itemize}
\item
  \textbf{AC પાવર આઉટપુટ ગણતરી:}

  \begin{itemize}
  \tightlist
  \item
    કરંટની RMS વેલ્યુ: Irms = Imax/2
  \item
    AC પાવર આઉટપુટ: Pac = (Irms)^{2} \times RL = (Imax/2)^{2} \times RL
  \item
    મહત્તમ પાવર માટે: Imax \times RL = Vcc
  \item
    આથી: Pac = (Vcc)^{2}/(2π \times RL)
  \end{itemize}
\item
  \textbf{કાર્યક્ષમતા ગણતરી:}

  \begin{itemize}
  \tightlist
  \item
    η = (Pac/Pdc) \times 100\%
  \item
    η = [(Vcc)^{2}/(2π \times RL)] \div [Vcc \times Imax/π] \times 100\%
  \item
    η = [(Vcc)^{2}/(2π \times RL)] \div [Vcc \times Vcc/(π \times RL)] \times 100\%
  \item
    η = [(Vcc)^{2}/(2π \times RL)] \times [π \times RL/Vcc^{2}] \times 100\%
  \item
    η = π/4 \times 100\% \approx 78.5\%
  \end{itemize}
\end{enumerate}

\textbf{ક્લાસ-B પુશ-પુલ એમ્પ્લિફાયરની મહત્તમ સૈદ્ધાંતિક કાર્યક્ષમતા 78.5\% છે}

\end{solutionbox}
\begin{mnemonicbox}
``PIPE'' (Power ratio, Input DC vs output AC, Pi in
formula, Efficiency maximum 78.5\%)

\end{mnemonicbox}
\subsection*{પ્રશ્ન 4(અ) [3
ગુણ]}\label{uxaaauxab0uxab6uxaa8-4uxa85-3-uxa97uxaa3}

\textbf{IC 741 નો પીન ડાયાગ્રામ અને યોજનાકીય પ્રતિક દોરો અને તેને વિગતવાર
સમજાવો.}

\begin{solutionbox}

\textbf{IC 741 ઓપ-એમ્પ પીન ડાયાગ્રામ અને સિમ્બોલ:}

\textbf{પીન ડાયાગ્રામ:}

\begin{verbatim}
        +{-{-}{-}{-}{-}{-}{-}+}
  1 o{-{-}{-}|       |{-}{-}{-}o 8}
        |       |
  2 o{-{-}{-}|  741  |{-}{-}{-}o 7}
        |       |
  3 o{-{-}{-}|       |{-}{-}{-}o 6}
        |       |
  4 o{-{-}{-}|       |{-}{-}{-}o 5}
        +{-{-}{-}{-}{-}{-}{-}+}
\end{verbatim}

\textbf{સ્કેમેટિક સિમ્બોલ:}

\begin{verbatim}
            
        |{ }
        | {}
   +{-{-}{-}o|  }
        |   {{-}{-}{-}o Output}
   {-o{-}{-}{-}| /}
        |/
            
\end{verbatim}

\textbf{પીન વિગત:}

\begin{enumerate}
\tightlist
\item
  ઓફસેટ નલ (NC1)
\item
  ઇન્વર્ટિંગ ઇનપુટ (-)
\item
  નોન-ઇન્વર્ટિંગ ઇનપુટ (+)
\item
  નેગેટિવ સપ્લાય (-Vcc)
\item
  ઓફસેટ નલ (NC2)\\
\item
  આઉટપુટ
\item
  પોઝિટિવ સપ્લાય (+Vcc)
\item
  NC (નો કનેક્શન)
\end{enumerate}

\end{solutionbox}
\begin{mnemonicbox}
``ON-INO'' (Offset Null, Inverting input, Negative
supply, Input non-inverting, Output, No connection)

\end{mnemonicbox}
\subsection*{પ્રશ્ન 4(અ) OR [3
ગુણ]}\label{uxaaauxab0uxab6uxaa8-4uxa85-or-3-uxa97uxaa3}

\textbf{Ideal Op-amp ની લાક્ષણિકતાની યાદી બનાવો.}

\begin{solutionbox}

{\def\LTcaptype{none} % do not increment counter
\begin{longtable}[]{@{}ll@{}}
\toprule\noalign{}
લાક્ષણિકતા & આદર્શ મૂલ્ય \\
\midrule\noalign{}
\endhead
\bottomrule\noalign{}
\endlastfoot
\textbf{ઓપન-લૂપ ગેઇન} & અનંત \\
\textbf{ઇનપુટ ઇમ્પીડન્સ} & અનંત \\
\textbf{આઉટપુટ ઇમ્પીડન્સ} & શૂન્ય \\
\textbf{બેન્ડવિડ્થ} & અનંત \\
\textbf{CMRR} & અનંત \\
\textbf{સ્લ્યુ રેટ} & અનંત \\
\textbf{ઓફસેટ વોલ્ટેજ} & શૂન્ય \\
\textbf{નોઇઝ} & શૂન્ય \\
\end{longtable}
}

\end{solutionbox}
\begin{mnemonicbox}
``ZINC BOSS'' (Zero offset, Infinite bandwidth, No
noise, CMRR infinite, Bandwidth unlimited, Output impedance zero, Slew
rate unlimited, Speed unlimited)

\end{mnemonicbox}
\subsection*{પ્રશ્ન 4(બ) [4
ગુણ]}\label{uxaaauxab0uxab6uxaa8-4uxaac-4-uxa97uxaa3}

\textbf{OPAMP નો ઉપયોગ કરીને differential એમ્પ્લીફાયર સમજાવો.}

\begin{solutionbox}

\textbf{ઓપ-એમ્પનો ઉપયોગ કરીને ડિફરેન્શિયલ એમ્પ્લિફાયર:}

\begin{verbatim}
flowchart TD
    A[ડિફરેન્શિયલ એમ્પ્લિફાયર] {-{-} B[તફાવતને એમ્પ્લિફાય કરે છે]}
    A {-{-} C[કોમન મોડ રિજેક્ટ કરે છે]}
    A {-{-} D[ચાર સમાન રેસિસ્ટર્સ]}
    A {-{-} E[ગેઇન = R2/R1]}
\end{verbatim}

\textbf{સર્કિટ ડાયાગ્રામ:}

\begin{verbatim}
             R2
   v1 o{-{-}{-}///{-}{-}{-}+}
                   |
                   |
      R1           |      R2
   +{-{-}///{-}{-}+{-}{-}{-}o|+     ///{-}{-}o Vout}
   |           |    |{-}
   |           |    |
   |           |    |
   +{-{-}///{-}{-}+{-}{-}{-}{-}+}
      R1           |
                   |
   v2 o{-{-}{-}///{-}{-}{-}+}
             R2
\end{verbatim}

\textbf{કાર્યપ્રણાલી:}

\begin{itemize}
\tightlist
\item
  આઉટપુટ ઇનપુટ્સ વચ્ચેના તફાવતને પ્રપોર્શનલ હોય છે
\item
  જો R1 = R3 અને R2 = R4, તો: Vout = (R2/R1)(V2-V1)
\item
  બંને ઇનપુટ્સ માટે સામાન્ય સિગ્નલ્સને રિજેક્ટ કરે છે (કોમન-મોડ રિજેક્શન)
\item
  ઇન્સ્ટ્રુમેન્ટેશન એપ્લિકેશન્સમાં વપરાય છે
\end{itemize}

\end{solutionbox}
\begin{mnemonicbox}
``CARE'' (Common-mode rejection, Amplifies
difference, Resistor matching important, Equal resistors for balance)

\end{mnemonicbox}
\subsection*{પ્રશ્ન 4(બ) OR [4
ગુણ]}\label{uxaaauxab0uxab6uxaa8-4uxaac-or-4-uxa97uxaa3}

\textbf{ઓપરેશનલ એમ્પ્લીફાયર (OP-AMP) નો બ્લોક ડાયાગ્રામ દોરીને વિસ્તૃતમાં
સમજાવો.}

\begin{solutionbox}

\textbf{ઓપ-એમ્પ બ્લોક ડાયાગ્રામ:}

\begin{verbatim}
flowchart LR
    A[ઇનપુટ સ્ટેજ] {-{-} B[ઇન્ટરમીડિયેટ સ્ટેજ]}
    B {-{-} C[આઉટપુટ સ્ટેજ]}
    D[બાયસિંગ સર્કિટ] {-{-} A}
    D {-{-} B}
    D {-{-} C}
    E[કોમ્પેન્સેશન નેટવર્ક] {-{-} B}
\end{verbatim}

\textbf{વિગતવાર બ્લોક ડાયાગ્રામ:}

\begin{verbatim}
                                Power Supply
                                    |
                                    v
 +{-{-}{-}{-}{-}{-}+     +{-}{-}{-}{-}{-}{-}{-}{-}{-}{-}{-}{-}+     +{-}{-}{-}{-}{-}{-}{-}{-}{-}{-}+     +{-}{-}{-}{-}{-}{-}+}
 |      |     |            |     |          |     |      |
 | Input|{-{-}{-}{-}|Differential|{-}{-}{-}{-}|  Voltage |{-}{-}{-}{-}|Output|{-}{-}{-} Output}
 | Pins |     |   Stage    |     |   Gain   |     | Stage|
 |      |     |            |     |   Stage  |     |      |
 +{-{-}{-}{-}{-}{-}+     +{-}{-}{-}{-}{-}{-}{-}{-}{-}{-}{-}{-}+     +{-}{-}{-}{-}{-}{-}{-}{-}{-}{-}+     +{-}{-}{-}{-}{-}{-}+}
                  \^{                 \^{}               \^{}}
                  |                 |               |
              +{-{-}{-}+{-}{-}{-}{-}+            |               |}
              |        |            |               |
              | Biasing|{-{-}{-}{-}{-}{-}{-}{-}{-}{-}{-}{-}+{-}{-}{-}{-}{-}{-}{-}{-}{-}{-}{-}{-}{-}{-}{-}+}
              | Circuit|
              |        |
              +{-{-}{-}{-}{-}{-}{-}{-}+}
                  \^{}
                  |
            Power Supply
\end{verbatim}

\textbf{બ્લોક્સની કાર્યપ્રણાલી:}

\begin{enumerate}
\tightlist
\item
  \textbf{ઇનપુટ સ્ટેજ}: ઊંચા ઇનપુટ ઇમ્પીડન્સ સાથે ડિફરેન્શિયલ એમ્પ્લિફાયર
\item
  \textbf{ઇન્ટરમીડિયેટ સ્ટેજ}: ફ્રીક્વન્સી કોમ્પેન્સેશન સાથે હાઇ-ગેઇન વોલ્ટેજ એમ્પ્લિફાયર
\item
  \textbf{આઉટપુટ સ્ટેજ}: ઓછા આઉટપુટ ઇમ્પીડન્સ બફર, કરંટ ગેઇન પ્રદાન કરે છે
\item
  \textbf{બાયસિંગ સર્કિટ}: બધા સ્ટેજને યોગ્ય DC સ્તર પ્રદાન કરે છે
\item
  \textbf{કોમ્પેન્સેશન નેટવર્ક}: ઓસિલેશન અટકાવે છે, સ્થિરતા સુનિશ્ચિત કરે છે
\end{enumerate}

\end{solutionbox}
\begin{mnemonicbox}
``DISCO'' (Differential stage Input, Second stage
amplifies, Compensation network, Output buffer)

\end{mnemonicbox}
\subsection*{પ્રશ્ન 4(ક) [7
ગુણ]}\label{uxaaauxab0uxab6uxaa8-4uxa95-7-uxa97uxaa3}

\textbf{OP-Amp પેરામીટર સમજાવો: 1) ઈનપુટ ઓફસેટ વોલ્ટેજ 2) આઉટપુટ ઓફસેટ વોલ્ટેજ 3)
ઈનપુટ ઓફસેટ કરંટ 4) ઈનપુટ બાયસ કરંટ 5) CMRR 6) સ્લુ રેટ 7) ગેઇન.}

\begin{solutionbox}

\textbf{ઓપ-એમ્પના પેરામીટર્સ:}

{\def\LTcaptype{none} % do not increment counter
\begin{longtable}[]{@{}
  >{\raggedright\arraybackslash}p{(\linewidth - 4\tabcolsep) * \real{0.2292}}
  >{\raggedright\arraybackslash}p{(\linewidth - 4\tabcolsep) * \real{0.2708}}
  >{\raggedright\arraybackslash}p{(\linewidth - 4\tabcolsep) * \real{0.5000}}@{}}
\toprule\noalign{}
\begin{minipage}[b]{\linewidth}\raggedright
પેરામીટર
\end{minipage} & \begin{minipage}[b]{\linewidth}\raggedright
વર્ણન
\end{minipage} & \begin{minipage}[b]{\linewidth}\raggedright
741 માટે ટિપિકલ વેલ્યુ
\end{minipage} \\
\midrule\noalign{}
\endhead
\bottomrule\noalign{}
\endlastfoot
\textbf{ઇનપુટ ઓફસેટ વોલ્ટેજ} & આઉટપુટને શૂન્ય કરવા માટે ઇનપુટ પર જરૂરી વોલ્ટેજ & 1-5
mV \\
\textbf{આઉટપુટ ઓફસેટ વોલ્ટેજ} & ઇનપુટ્સ ગ્રાઉન્ડ કરવામાં આવે ત્યારે આઉટપુટ વોલ્ટેજ &
ઇનપુટ ઓફસેટ અને ગેઇન પર આધારિત \\
\textbf{ઇનપુટ ઓફસેટ કરંટ} & ઇનપુટ બાયસ કરંટ્સ વચ્ચેનો તફાવત & 3-30 nA \\
\textbf{ઇનપુટ બાયસ કરંટ} & બે ઇનપુટ કરંટ્સની સરેરાશ & 30-500 nA \\
\textbf{CMRR} & કોમન-મોડ સિગ્નલ્સને રિજેક્ટ કરવાની ક્ષમતા & 70-100 dB \\
\textbf{સ્લ્યુ રેટ} & આઉટપુટ વોલ્ટેજ પરિવર્તનનો મહત્તમ દર & 0.5 V/μs \\
\textbf{ગેઇન (Aol)} & ઓપન-લૂપ વોલ્ટેજ ગેઇન & 104-106 (80-120 dB) \\
\end{longtable}
}

\textbf{ઇનપુટ ઓફસેટ વોલ્ટેજ માટે ડાયાગ્રામ:}

\begin{verbatim}
                 Vos
                  |
                  v
      +{-{-}{-}{-}{-}+     |     +{-}{-}{-}{-}{-}+}
      |     |     |     |     |
   {-{-}{-}+  +  +{-}{-}{-}{-}{-}+{-}{-}{-}{-}{-}+     +{-}{-}{-}}
      |     |           |     |
      +{-{-}{-}{-}{-}+           +{-}{-}{-}{-}{-}+}
\end{verbatim}

\end{solutionbox}
\begin{mnemonicbox}
``VICS BGR'' (Voltage offset at Input, Current
offset, Slew rate, Bias current, Gain, Rejection ratio)

\end{mnemonicbox}
\subsection*{પ્રશ્ન 4(ક) OR [7
ગુણ]}\label{uxaaauxab0uxab6uxaa8-4uxa95-or-7-uxa97uxaa3}

\textbf{Inverting અને Non-inverting Op-amp એમ્પ્લીફાયર આકૃતિ દોરી વોલ્ટેજ ગેઇન
નું સૂત્ર તારવી સમજાવો.}

\begin{solutionbox}

\textbf{ઇન્વર્ટિંગ એમ્પ્લિફાયર:}

\begin{verbatim}
flowchart TB
    A[ઇન્વર્ટિંગ એમ્પ્લિફાયર] {-{-} B[આઉટપુટ 180^ આઉટ ઓફ ફેઝ]}
    A {-{-} C[ગેઇન = {-}Rf/Rin]}
    A {-{-} D[ઇન્વર્ટિંગ ઇનપુટ પર વર્ચ્યુઅલ ગ્રાઉન્ડ]}
\end{verbatim}

\textbf{સર્કિટ ડાયાગ્રામ:}

\begin{verbatim}
                 Rf
           +{-{-}{-}{-}///{-}{-}{-}{-}+}
           |              |
           |              |
     Rin   |    +{        |}
Vin o{-{-}///{-}{-}{-}|{-}       |}
           |    |  {      |}
           |    |   {{-}{-}{-}{-}{-}o Vout}
           |    |  /
           |    |{-/}
           |    +/
           |    |
           |    |
           +{-{-}{-}{-}+}
                |
               GND
\end{verbatim}

\textbf{ગેઇન મેળવણી:}

\begin{itemize}
\tightlist
\item
  વર્ચ્યુઅલ ગ્રાઉન્ડ કન્સેપ્ટનો ઉપયોગ (V- \approx 0)
\item
  Rin દ્વારા કરંટ: Iin = Vin/Rin
\item
  Rf દ્વારા કરંટ: If = Iin (ઓપ-એમ્પ ઇનપુટમાં કોઈ કરંટ નથી)
\item
  Rf પર વોલ્ટેજ: Vout = -If \times Rf = -Iin \times Rf = -Vin \times Rf/Rin
\item
  આથી, ગેઇન = Vout/Vin = -Rf/Rin
\end{itemize}

\textbf{નોન-ઇન્વર્ટિંગ એમ્પ્લિફાયર:}

\begin{verbatim}
flowchart TB
    A[નોન{-ઇન્વર્ટિંગ એમ્પ્લિફાયર] {-}{-} B[આઉટપુટ ઇનપુટ સાથે ઇન ફેઝ]}
    A {-{-} C[ગેઇન = 1 + Rf/Rin]}
    A {-{-} D[ઇન્વર્ટિંગ કરતાં ઊંચા ઇનપુટ ઇમ્પીડન્સ]}
\end{verbatim}

\textbf{સર્કિટ ડાયાગ્રામ:}

\begin{verbatim}
                 Rf
           +{-{-}{-}{-}///{-}{-}{-}{-}+}
           |              |
           |              |
           |    +{        |}
Vin o{-{-}{-}{-}{-}{-}+{-}{-}{-}{-}|{-}       |}
           |    |  {      |}
           |    |   {{-}{-}{-}{-}{-}o Vout}
           |    |  /
           |    |{-/}
           |    +/
           |    |
           |    |
           +{-{-}///{-}{-}+}
              Rin     |
                      |
                     GND
\end{verbatim}

\textbf{ગેઇન મેળવણી:}

\begin{itemize}
\tightlist
\item
  નેગેટિવ ફીડબેકને કારણે, V- \approx V+ = Vin
\item
  Rin પર વોલ્ટેજ: V- = Vin
\item
  Rin દ્વારા કરંટ: IRin = V-/Rin = Vin/Rin
\item
  સમાન કરંટ Rf દ્વારા વહે છે: IRf = IRin
\item
  Rf પર વોલ્ટેજ: VRf = IRf \times Rf = Vin \times Rf/Rin
\item
  આઉટપુટ વોલ્ટેજ: Vout = V- + VRf = Vin + Vin \times Rf/Rin = Vin(1 + Rf/Rin)
\item
  આથી, ગેઇન = Vout/Vin = 1 + Rf/Rin
\end{itemize}

\textbf{તુલના:}

{\def\LTcaptype{none} % do not increment counter
\begin{longtable}[]{@{}lll@{}}
\toprule\noalign{}
પેરામીટર & ઇન્વર્ટિંગ એમ્પ્લિફાયર & નોન-ઇન્વર્ટિંગ એમ્પ્લિફાયર \\
\midrule\noalign{}
\endhead
\bottomrule\noalign{}
\endlastfoot
\textbf{ગેઇન ફોર્મ્યુલા} & -Rf/Rin & 1 + Rf/Rin \\
\textbf{ફેઝ શિફ્ટ} & 180^\circ & 0^\circ \\
\textbf{ઇનપુટ ઇમ્પીડન્સ} & Rin ની બરાબર & ખૂબ ઊંચી (\approx અનંત) \\
\textbf{ન્યૂનતમ સંભવિત ગેઇન} & \textless1 હોઈ શકે છે & હંમેશા \geq1 હોય છે \\
\end{longtable}
}

\end{solutionbox}
\begin{mnemonicbox}
``PING-PONG'' (Phase Inverted Negative Gain vs
Positive Output Non-inverted Gain)

\end{mnemonicbox}
\subsection*{પ્રશ્ન 5(અ) [3
ગુણ]}\label{uxaaauxab0uxab6uxaa8-5uxa85-3-uxa97uxaa3}

\textbf{Op-Amp નો ઉપયોગ કરીને ઇન્ટીગ્રેટર દોરો અને સમજાવો.}

\begin{solutionbox}

\textbf{ઓપ-એમ્પ ઇન્ટીગ્રેટર:}

\begin{verbatim}
flowchart TD
    A[ઇન્ટિગ્રેટર] {-{-} B[ફીડબેકમાં RC સર્કિટ]}
    A {-{-} C[આઉટપુટ ઇનપુટનો ઇન્ટિગ્રલ છે]}
    A {-{-} D[લો{-}પાસ ફિલ્ટર તરીકે કાર્ય કરે છે]}
\end{verbatim}

\textbf{સર્કિટ ડાયાગ્રામ:}

\begin{verbatim}
                  C
              +{-{-}{-}||{-}{-}{-}+}
              |        |
              |        |
   Vin o{-{-}///{-}{-}+{-}{-}{-}o|+      |}
            R     |    |{-      +{-}{-}{-}o Vout}
                  |    |       |
                  |    |
                  +{-{-}{-}{-}+}
                  |
                 GND
\end{verbatim}

\textbf{કાર્યપ્રણાલી:}

\begin{itemize}
\tightlist
\item
  આઉટપુટ વોલ્ટેજ ઇનપુટના ઇન્ટિગ્રલને પ્રપોર્શનલ છે
\item
  Vout = -1/RC \intVin dt
\item
  વેવફોર્મ જનરેટર્સ, એનાલોગ કમ્પ્યુટર્સમાં વપરાય છે
\item
  -20dB/decade સ્લોપ સાથે લો-પાસ ફિલ્ટર તરીકે કાર્ય કરે છે
\end{itemize}

\end{solutionbox}
\begin{mnemonicbox}
``TIME'' (Takes Input and Makes integral over time
Exactly)

\end{mnemonicbox}
\subsection*{પ્રશ્ન 5(અ) OR [3
ગુણ]}\label{uxaaauxab0uxab6uxaa8-5uxa85-or-3-uxa97uxaa3}

\textbf{Op-Amp નો ઉપયોગ કરી સમિંગ એમ્પ્લીફાયર દોરો અને સમજાવો.}

\begin{solutionbox}

\textbf{ઓપ-એમ્પ સમિંગ એમ્પ્લિફાયર:}

\begin{verbatim}
flowchart TB
    A[સમિંગ એમ્પ્લિફાયર] {-{-} B[મલ્ટિપલ ઇનપુટ્સ એડ કરે છે]}
    A {-{-} C[વેઇટેડ સમ શક્ય]}
    A {-{-} D[ઇન્વર્ટિંગ કોન્ફિગરેશન]}
\end{verbatim}

\textbf{સર્કિટ ડાયાગ્રામ:}

\begin{verbatim}
              Rf
        +{-{-}{-}///{-}{-}{-}+}
        |            |
        |            |
  R1    |    +{      |}
V1 o{-{-}///{-}{-}|{-}     |}
        |    |  {    |}
  R2    |    |   {{-}{-}{-}o Vout}
V2 o{-{-}///{-}{-}+   /}
        |    |  /
  R3    |    |{-/}
V3 o{-{-}///{-}{-}+/}
        |     |
        |     |
        +{-{-}{-}{-}{-}+}
              |
             GND
\end{verbatim}

\textbf{કાર્યપ્રણાલી:}

\begin{itemize}
\tightlist
\item
  મલ્ટિપલ ઇનપુટ્સ સાથે ઇન્વર્ટિંગ કોન્ફિગરેશન વાપરે છે
\item
  દરેક ઇનપુટ તેના રેઝિસ્ટન્સના આધારે આઉટપુટમાં યોગદાન આપે છે
\item
  જો R1 = R2 = R3 = R અને Rf = R, તો Vout = -(V1 + V2 + V3)
\item
  જો રેઝિસ્ટર્સ અલગ હોય, તો વેઇટેડ સમ ઉત્પન્ન થાય છે: Vout = -Rf(V1/R1 + V2/R2 +
  V3/R3)
\item
  ઇન્વર્ટિંગ ઇનપુટ પર વર્ચ્યુઅલ ગ્રાઉન્ડ એનાલિસિસને સરળ બનાવે છે
\end{itemize}

\end{solutionbox}
\begin{mnemonicbox}
``SWIM'' (Summing Weighted Inputs with Mixing)

\end{mnemonicbox}
\subsection*{પ્રશ્ન 5(બ) [4
ગુણ]}\label{uxaaauxab0uxab6uxaa8-5uxaac-4-uxa97uxaa3}

\textbf{વિવિધ પ્રકારના પાવર એમ્પ્લીફાયરની સરખામણી કરો.}

\begin{solutionbox}

{\def\LTcaptype{none} % do not increment counter
\begin{longtable}[]{@{}
  >{\raggedright\arraybackslash}p{(\linewidth - 8\tabcolsep) * \real{0.2292}}
  >{\raggedright\arraybackslash}p{(\linewidth - 8\tabcolsep) * \real{0.1875}}
  >{\raggedright\arraybackslash}p{(\linewidth - 8\tabcolsep) * \real{0.1875}}
  >{\raggedright\arraybackslash}p{(\linewidth - 8\tabcolsep) * \real{0.2083}}
  >{\raggedright\arraybackslash}p{(\linewidth - 8\tabcolsep) * \real{0.1875}}@{}}
\toprule\noalign{}
\begin{minipage}[b]{\linewidth}\raggedright
પેરામીટર
\end{minipage} & \begin{minipage}[b]{\linewidth}\raggedright
ક્લાસ A
\end{minipage} & \begin{minipage}[b]{\linewidth}\raggedright
ક્લાસ B
\end{minipage} & \begin{minipage}[b]{\linewidth}\raggedright
ક્લાસ AB
\end{minipage} & \begin{minipage}[b]{\linewidth}\raggedright
ક્લાસ C
\end{minipage} \\
\midrule\noalign{}
\endhead
\bottomrule\noalign{}
\endlastfoot
\textbf{કન્ડક્શન એંગલ} & 360^\circ & 180^\circ & 180^\circ-360^\circ & \textless180^\circ \\
\textbf{કાર્યક્ષમતા} & 25-30\% & 70-80\% & 50-70\% & \textgreater80\% \\
\textbf{ડિસ્ટોર્શન} & ખૂબ ઓછું & ઊંચું (ક્રોસઓવર) & ઓછું & ખૂબ ઊંચું \\
\textbf{બાયસિંગ} & કટઓફ ઉપર & કટઓફ પર & કટઓફથી થોડું ઉપર & કટઓફથી નીચે \\
\textbf{એપ્લિકેશન્સ} & હાઇ ફિડેલિટી ઓડિયો & જનરલ પર્પઝ & ઓડિયો એમ્પ્લિફાયર્સ &
RF એમ્પ્લિફાયર્સ \\
\end{longtable}
}

\end{solutionbox}
\begin{mnemonicbox}
``CABINET'' (Conduction angle, Amplification
quality, Biasing, Ideal applications, Noise/distortion, Efficiency,
Temperature concerns)

\end{mnemonicbox}
\subsection*{પ્રશ્ન 5(બ) OR [4
ગુણ]}\label{uxaaauxab0uxab6uxaa8-5uxaac-or-4-uxa97uxaa3}

\textbf{પુશ પુલ એમ્પ્લીફાયર અને કોમ્પ્લીમેન્ટરી પુશ પુલ એમ્પ્લીફાયર ની સરખામણી કરો.}

\begin{solutionbox}

{\def\LTcaptype{none} % do not increment counter
\begin{longtable}[]{@{}
  >{\raggedright\arraybackslash}p{(\linewidth - 4\tabcolsep) * \real{0.1642}}
  >{\raggedright\arraybackslash}p{(\linewidth - 4\tabcolsep) * \real{0.3134}}
  >{\raggedright\arraybackslash}p{(\linewidth - 4\tabcolsep) * \real{0.5224}}@{}}
\toprule\noalign{}
\begin{minipage}[b]{\linewidth}\raggedright
પેરામીટર
\end{minipage} & \begin{minipage}[b]{\linewidth}\raggedright
પુશ-પુલ એમ્પ્લિફાયર
\end{minipage} & \begin{minipage}[b]{\linewidth}\raggedright
કોમ્પ્લિમેન્ટરી પુશ-પુલ એમ્પ્લિફાયર
\end{minipage} \\
\midrule\noalign{}
\endhead
\bottomrule\noalign{}
\endlastfoot
\textbf{વપરાતા ટ્રાન્ઝિસ્ટર્સ} & સમાન પ્રકાર (NPN અથવા PNP) & કોમ્પ્લિમેન્ટરી
જોડી (NPN અને PNP) \\
\textbf{ઇનપુટ ટ્રાન્સફોર્મર} & જરૂરી (સેન્ટર-ટેપ્ડ) & જરૂરી નથી \\
\textbf{આઉટપુટ ટ્રાન્સફોર્મર} & જરૂરી & જરૂરી નથી \\
\textbf{સર્કિટ જટિલતા} & વધુ જટિલ & સરળ \\
\textbf{ખર્ચ} & ટ્રાન્સફોર્મર્સને કારણે ઊંચો & નીચો \\
\textbf{ફ્રીક્વન્સી રિસ્પોન્સ} & ટ્રાન્સફોર્મર્સ દ્વારા મર્યાદિત & વધુ સારું (વિશાળ
રેન્જ) \\
\textbf{ફેઝ ડિસ્ટોર્શન} & ઊંચું & નીચું \\
\textbf{પાવર સપ્લાય} & સિંગલ પોલારિટી & સામાન્ય રીતે ડ્યુઅલ પોલારિટી જરૂરી \\
\end{longtable}
}

\end{solutionbox}
\begin{mnemonicbox}
``TONIC'' (Transformers vs None, One type vs
complementary, Nice frequency response, Improved distortion, Cost
effectiveness)

\end{mnemonicbox}
\subsection*{પ્રશ્ન 5(ક) [7
ગુણ]}\label{uxaaauxab0uxab6uxaa8-5uxa95-7-uxa97uxaa3}

\textbf{IC555 ના ઉપયોગો લખો અને કોઈ પણ એક વિસ્તૃતમાં સમજાવો.}

\begin{solutionbox}

\textbf{IC 555 ના એપ્લિકેશન્સ:}

\begin{enumerate}
\tightlist
\item
  એસ્ટેબલ મલ્ટિવાયબ્રેટર
\item
  મોનોસ્ટેબલ મલ્ટિવાયબ્રેટર
\item
  બાયસ્ટેબલ મલ્ટિવાયબ્રેટર
\item
  પલ્સ વિડ્થ મોડુલેટર
\item
  સિક્વેન્શિયલ ટાઇમર
\item
  ફ્રીક્વન્સી ડિવાઇડર
\item
  ટોન જનરેટર
\end{enumerate}

\textbf{IC 555 નો ઉપયોગ કરીને એસ્ટેબલ મલ્ટિવાયબ્રેટર:}

\begin{verbatim}
flowchart TB
    A[555 એસ્ટેબલ] {-{-} B[ફ્રી{-}રનિંગ ઓસિલેટર]}
    A {-{-} C[કોઈ સ્થાયી સ્ટેટ નથી]}
    A {-{-} D[આઉટપુટ સતત સ્વિચ કરે છે]}
    A {-{-} E[ફ્રીક્વન્સી R1, R2, C દ્વારા નક્કી થાય છે]}
\end{verbatim}

\textbf{સર્કિટ ડાયાગ્રામ:}

\begin{verbatim}
                  +Vcc
                    |
                    |
             +{-{-}{-}{-}{-}{-}+{-}{-}{-}{-}{-}{-}+}
             |      |      |
             |     R1      |
             |      |      |
             +{-{-}{-}{-}{-}{-}+{-}{-}{-}{-}{-}{-}+}
             |      |      |
             |     R2      |
             |      |      |
    +{-{-}{-}{-}{-}{-}{-}{-}+{-}{-}{-}{-}{-}{-}+{-}{-}{-}{-}{-}{-}+{-}{-}{-}{-}{-}{-}{-}{-}+}
    |        |      |      |        |
    |   +{-{-}{-}{-}+    8 |      |        |}
    |   |    |      |      |        |
    |   |    |    7 |      |        |
    |   |    |      |      |        |
    |   |    |    6 |      |        |
    |  C1    |      | 555  |        |
    |   |    |    5 +{-{-}{-}{-}{-}{-}+        |}
    |   |    |      |      |        |
    |   |    |    4 |      |        |
    |   |    |      |      |        |
    |   |    |    3 |      +{-{-}{-}o Output}
    |   |    |      |      |        |
    |   |    |    2 |      |        |
    |   |    |      |      |        |
    |   |    |    1 |      |        |
    |   |    +{-{-}{-}{-}{-}{-}+{-}{-}{-}{-}{-}{-}+        |}
    |   |           |               |
    +{-{-}{-}+{-}{-}{-}{-}{-}{-}{-}{-}{-}{-}{-}+{-}{-}{-}{-}{-}{-}{-}{-}{-}{-}{-}{-}{-}{-}{-}+}
        |           |
       GND         GND
\end{verbatim}

\textbf{કાર્યપ્રણાલી:}

\begin{itemize}
\tightlist
\item
  R1, R2 અને C ફ્રીક્વન્સી નક્કી કરે છે
\item
  આઉટપુટ HIGH અને LOW વચ્ચે ઓસિલેટ કરે છે
\item
  ચાર્જિંગ ટાઇમ: t1 = 0.693(R1+R2)C
\item
  ડિસ્ચાર્જિંગ ટાઇમ: t2 = 0.693(R2)C
\item
કુલ પીરિયડ:

T = t1 + t2 = 0.693(R1+2R2)C

\item
  ફ્રીક્વન્સી: f = 1.44/[(R1+2R2)C]
\item
  ડ્યુટી સાયકલ: D = (R1+R2)/(R1+2R2)
\end{itemize}

\textbf{એપ્લિકેશન્સ:}

\begin{itemize}
\tightlist
\item
  LED ફ્લેશર્સ
\item
  ક્લોક જનરેટર્સ
\item
  ટોન જનરેટર્સ
\item
  પલ્સ જનરેશન
\end{itemize}

\end{solutionbox}
\begin{mnemonicbox}
``FREE'' (Frequency determined by Resistors and
capacitor, Endless oscillation, Easy to configure)

\end{mnemonicbox}
\subsection*{પ્રશ્ન 5(ક) OR [7
ગુણ]}\label{uxaaauxab0uxab6uxaa8-5uxa95-or-7-uxa97uxaa3}

\textbf{IC555 નો પીન ડાયાગ્રામ અને બ્લોક ડાયાગ્રામ દોરો અને વિસ્તૃતમાં સમજાવો.}

\begin{solutionbox}

\textbf{IC 555 ટાઇમર:}

\textbf{પીન ડાયાગ્રામ:}

\begin{verbatim}
        +{-{-}{-}{-}{-}{-}{-}+}
  1 o{-{-}{-}|       |{-}{-}{-}o 8}
        |       |
  2 o{-{-}{-}|  555  |{-}{-}{-}o 7}
        |       |
  3 o{-{-}{-}|       |{-}{-}{-}o 6}
        |       |
  4 o{-{-}{-}|       |{-}{-}{-}o 5}
        +{-{-}{-}{-}{-}{-}{-}+}
\end{verbatim}

\textbf{પીન વિગત:}

\begin{enumerate}
\tightlist
\item
  ગ્રાઉન્ડ - સર્કિટ ગ્રાઉન્ડથી જોડાયેલ
\item
  ટ્રિગર - વોલ્ટેજ 1/3 Vcc થી નીચે પડે ત્યારે ટાઇમિંગ સાયકલ શરૂ કરે છે
\item
  આઉટપુટ - આઉટપુટ સિગ્નલ પ્રદાન કરે છે, 200mA સુધી સોર્સ અથવા સિંક કરી શકે છે
\item
  રીસેટ - લો પર ખેંચવામાં આવે ત્યારે ટાઇમિંગ સાયકલ સમાપ્ત કરે છે
\item
  કન્ટ્રોલ વોલ્ટેજ - આંતરિક વોલ્ટેજ ડિવાઇડર (2/3 Vcc) ને ઍક્સેસ કરવાની મંજૂરી આપે છે
\item
  થ્રેશોલ્ડ - વોલ્ટેજ 2/3 Vcc થી વધે ત્યારે ટાઇમિંગ સાયકલ સમાપ્ત કરે છે
\item
  ડિસ્ચાર્જ - આંતરિક ટ્રાન્ઝિસ્ટરના ઓપન કલેક્ટરથી જોડાયેલ
\item
  Vcc - પોઝિટિવ સપ્લાય વોલ્ટેજ (4.5V થી 16V)
\end{enumerate}

\textbf{બ્લોક ડાયાગ્રામ:}

\begin{verbatim}
    8                               
    o{-{-}{-}{-}{-}{-}+{-}{-}{-}{-}{-}{-}{-}{-}{-}{-}{-}{-}{-}{-}{-}{-}{-}{-}{-}{-}{-}+  }
    Vcc    |                     |  
           |    +{-{-}{-}{-}{-}{-}{-}{-}{-}{-}{-}+    |  }
    5      |    |           |    |  
    o{-{-}{-}{-}{-}{-}+{-}{-}{-}{-}| Voltage   |    |  }
    Control|    | Divider   |    |  
           |    |           |    |  
           |    +{-{-}{-}{-}{-}{-}{-}{-}{-}{-}{-}+    |  }
           |      |     |        |  
           |      |     |        |  
    2      |    +{-v{-}+ +{-}v{-}+      |  }
    o{-{-}{-}{-}{-}{-}+{-}{-}{-}|   | |   |      |  }
    Trigger     |Comp| |Comp|    |  
                |   | |   |      |  
    6           +{-+{-}+ +{-}+{-}+      |  }
    o{-{-}{-}{-}{-}{-}{-}{-}{-}{-}{-}{-}+|     |+{-}{-}{-}{-}{-}{-}{-}+  }
    Threshold     |     |           
                +{-v{-}{-}{-}{-}{-}v{-}+         }
                |         |         
                | Flip    |         
    4           | Flop    |         
    o{-{-}{-}{-}{-}{-}{-}{-}{-}{-}|         |         }
    Reset       +{-+{-}{-}{-}{-}{-}+{-}+         }
                  |     |           
                  |     |           
                +{-v{-}+ +{-}v{-}+         }
                |   | |   |         
                |Buf| |Out|         
                |   | |   |         
                +{-+{-}+ +{-}+{-}+         }
                  |     |           
    7             |     |  3        
    o{-{-}{-}{-}{-}{-}{-}{-}{-}{-}{-}{-}{-}+     +{-}{-}o        }
    Discharge               Output  
                                    
    1                               
    o{-{-}{-}{-}{-}{-}{-}{-}{-}{-}{-}{-}{-}{-}{-}{-}{-}{-}{-}{-}{-}{-}{-}{-}{-}{-}{-}{-}+  }
    GND                          |  
                                 |  
    +{-{-}{-}{-}{-}{-}{-}{-}{-}{-}{-}{-}{-}{-}{-}{-}{-}{-}{-}{-}{-}{-}{-}{-}{-}{-}{-}{-}+  }
\end{verbatim}

\textbf{કાર્યપ્રણાલી:}

\begin{enumerate}
\tightlist
\item
  \textbf{વોલ્ટેજ ડિવાઇડર}: Vcc ના 1/3 અને 2/3 પર રેફરન્સ વોલ્ટેજ બનાવે છે
\item
  \textbf{કમ્પેરેટર્સ}: ઇનપુટ વોલ્ટેજને રેફરન્સ વોલ્ટેજ સાથે સરખાવે છે
\item
  \textbf{ફ્લિપ-ફ્લોપ}: કમ્પેરેટર્સના આઉટપુટના આધારે ટાઇમિંગ સ્ટેટ સ્ટોર કરે છે
\item
  \textbf{આઉટપુટ સ્ટેજ}: ફ્લિપ-ફ્લોપ આઉટપુટને બફર અને એમ્પ્લિફાય કરે છે
\item
  \textbf{ડિસ્ચાર્જ ટ્રાન્ઝિસ્ટર}: ટાઇમિંગ કેપેસિટર ડિસ્ચાર્જ કરવા માટે ફ્લિપ-ફ્લોપ
  દ્વારા નિયંત્રિત
\end{enumerate}

\textbf{ઓપરેટિંગ મોડ્સ:}

\begin{enumerate}
\tightlist
\item
  \textbf{મોનોસ્ટેબલ}: ઇનપુટ પલ્સ દ્વારા ટ્રિગર થયેલ વન-શોટ ટાઇમર
\item
  \textbf{એસ્ટેબલ}: પલ્સ જનરેશન માટે ફ્રી-રનિંગ ઓસિલેટર
\item
  \textbf{બાયસ્ટેબલ}: સેટ અને રીસેટ ફંક્શનાલિટી સાથે ફ્લિપ-ફ્લોપ
\end{enumerate}

\textbf{એપ્લિકેશન્સ:}

\begin{itemize}
\tightlist
\item
  પલ્સ જનરેશન
\item
  ટાઇમ ડિલે
\item
  ઓસિલેટર્સ
\item
  PWM કન્ટ્રોલર્સ
\item
  સિક્વેન્શિયલ ટાઇમર્સ
\end{itemize}

\end{solutionbox}
\begin{mnemonicbox}
``VICTOR'' (Voltage divider, Internal comparators,
Control flip-flop, Timing capabilities, Output buffer, Reset function)

\end{mnemonicbox}

\end{document}
