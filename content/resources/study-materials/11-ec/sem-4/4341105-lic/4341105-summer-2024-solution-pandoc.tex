\documentclass[10pt,a4paper]{article}

% content/resources/templates/preamble.tex
\usepackage[margin=0.6in]{geometry}
\author{Milav Dabgar}
\usepackage{amsmath,amssymb,amsthm}
\usepackage{booktabs}
\usepackage{multirow}
\usepackage{xcolor}
\usepackage{tcolorbox}
\tcbuselibrary{breakable,skins}
\usepackage[colorlinks=true,linkcolor=blue]{hyperref}
\usepackage{titlesec}
\usepackage{enumitem}
\usepackage{tikz}
\usepackage{pgfplots}
\usepackage{circuitikz}
\usepackage[version=4]{mhchem}
\usepackage{longtable}
\usepackage{array}
\usepackage{float}
\usepackage{caption}
\usepackage{listings}

\lstset{
  basicstyle=\small\ttfamily,
  breaklines=true,
  breakatwhitespace=false,
  postbreak=\mbox{\textcolor{red}{$\hookrightarrow$}\space},
  float=false,
  numbers=left,
  numberstyle=\tiny\color{gray},
  numbersep=10pt,
  xleftmargin=2em,
  keywordstyle=\color{blue},
  commentstyle=\color{green!60!black},
  stringstyle=\color{purple},
  backgroundcolor=\color{gray!5},
  showstringspaces=false,
  tabsize=2,
  captionpos=b,
  keepspaces=true,
  columns=flexible
}

\pgfplotsset{compat=1.18}
\usetikzlibrary{shapes,arrows,positioning,calc,patterns,decorations.pathmorphing,decorations.markings,arrows.meta}

% Color scheme
\definecolor{headcolor}{RGB}{0,102,204}
\definecolor{keycolor}{RGB}{220,20,60}
\definecolor{solutioncolor}{RGB}{34,139,34}
\definecolor{mnemoniccolor}{RGB}{148,0,211}
\definecolor{codecolor}{RGB}{0,0,100}

% Spacing
\setlength{\parskip}{3pt}
\setlist[itemize]{nosep}
\setlist[enumerate]{nosep}

% Title formatting
\titleformat{\section}{\Large\bfseries\color{headcolor}}{\thesection}{1em}{}
\titleformat{\subsection}{\large\bfseries\color{headcolor}}{\thesubsection}{1em}{}

% Pandoc tightlist compatibility
\providecommand{\tightlist}{%
  \setlength{\itemsep}{0pt}\setlength{\parskip}{0pt}}

% Pandoc longtable compatibility
\newcounter{none}
\def\thenone{}


% content/resources/templates/english-boxes.tex
% This file is currently empty - it exists to maintain consistency with the import structure.
% Add custom environments here if needed in the future.


\begin{document}

\begin{center}
{\Huge\bfseries\color{headcolor} Subject Name Solutions}\\[5pt]
{\LARGE 4341105 -- Summer 2024}\\[3pt]
{\large Semester 1 Study Material}\\[3pt]
{\normalsize\textit{Detailed Solutions and Explanations}}
\end{center}

\vspace{10pt}

\subsection*{Question 1(a) [3 marks]}\label{q1a}

\textbf{State and explain the difference between positive and negative
feedback with diagram.}

\begin{solutionbox}

{\def\LTcaptype{none} % do not increment counter
\begin{longtable}[]{@{}
  >{\raggedright\arraybackslash}p{(\linewidth - 4\tabcolsep) * \real{0.2245}}
  >{\raggedright\arraybackslash}p{(\linewidth - 4\tabcolsep) * \real{0.3878}}
  >{\raggedright\arraybackslash}p{(\linewidth - 4\tabcolsep) * \real{0.3878}}@{}}
\toprule\noalign{}
\begin{minipage}[b]{\linewidth}\raggedright
Parameter
\end{minipage} & \begin{minipage}[b]{\linewidth}\raggedright
Negative Feedback
\end{minipage} & \begin{minipage}[b]{\linewidth}\raggedright
Positive Feedback
\end{minipage} \\
\midrule\noalign{}
\endhead
\bottomrule\noalign{}
\endlastfoot
Signal & Output signal is fed back to input with opposite phase & Output
signal is fed back to input with same phase \\
Gain & Decreases & Increases \\
Stability & Improves & Reduces \\
Applications & Amplifiers & Oscillators \\
\end{longtable}
}

\textbf{Diagram:}

\begin{center}
\textbf{Mermaid Diagram (Code)}
\begin{verbatim}
{Shaded}
{Highlighting}[]
graph LR
    A[Input] {-{-}{} B[Amplifier]}
    B {-{-}{} C[Output]}
    C {-{-}{} D\{Feedback Network\}}
    
    \%\% Negative Feedback
    subgraph Negative Feedback
    D {-{-}{}|180^ Phase Shift| E[Subtractor]}
    E {-{-}{} B}
    end
    
    \%\% Positive Feedback
    subgraph Positive Feedback
    D {-{-}{}|0^ Phase Shift| F[Adder]}
    F {-{-}{} B}
    end
{Highlighting}
{Shaded}
\end{verbatim}
\end{center}

\begin{itemize}
\tightlist
\item
  \textbf{Phase relationship}: In negative feedback, signal is 180^\circ out
  of phase while in positive feedback, signal is in phase
\item
  \textbf{Purpose}: Negative feedback stabilizes system while positive
  feedback creates oscillations
\end{itemize}

\end{solutionbox}
\begin{mnemonicbox}
``Negative Needs Stability, Positive Produces
Oscillations''

\end{mnemonicbox}
\subsection*{Question 1(b) [4 marks]}\label{q1b}

\textbf{Explain the effect of negative feedback on input impedance of
the Amplifier.}

\begin{solutionbox}

{\def\LTcaptype{none} % do not increment counter
\begin{longtable}[]{@{}lll@{}}
\toprule\noalign{}
Type of Feedback & Effect on Input Impedance & Formula \\
\midrule\noalign{}
\endhead
\bottomrule\noalign{}
\endlastfoot
Voltage Series & Increases & Z(in-f) = Z(in)(1+Aβ) \\
Current Series & Increases & Z(in-f) = Z(in)(1+Aβ) \\
Voltage Shunt & Decreases & Z(in-f) = Z(in)/(1+Aβ) \\
Current Shunt & Decreases & Z(in-f) = Z(in)/(1+Aβ) \\
\end{longtable}
}

\textbf{Diagram:}

\begin{center}
\textbf{Mermaid Diagram (Code)}
\begin{verbatim}
{Shaded}
{Highlighting}[]
graph LR
    A[Input Signal] {-{-}{} B[Input Impedance]}
    B {-{-}{} C[Amplifier]}
    C {-{-}{} D[Output]}
    D {-{-}{} E[Feedback Network]}
    E {-{-}{} F[Summing Point]}
    F {-{-}{} B}
    style B fill:\#f9f,stroke:\#333,stroke{-width:2px}
{Highlighting}
{Shaded}
\end{verbatim}
\end{center}

\begin{itemize}
\tightlist
\item
  \textbf{Series feedback}: When feedback signal is in series with
  input, input impedance increases
\item
  \textbf{Shunt feedback}: When feedback signal is in parallel with
  input, input impedance decreases
\item
  \textbf{Magnitude}: Change is proportional to (1+Aβ) where A is gain
  and β is feedback factor
\end{itemize}

\end{solutionbox}
\begin{mnemonicbox}
``Series Soars, Shunt Shrinks''

\end{mnemonicbox}
\subsection*{Question 1(c) [7 marks]}\label{q1c}

\textbf{List the advantages and Disadvantages of negative feedback.}

\begin{solutionbox}

{\def\LTcaptype{none} % do not increment counter
\begin{longtable}[]{@{}
  >{\raggedright\arraybackslash}p{(\linewidth - 2\tabcolsep) * \real{0.4444}}
  >{\raggedright\arraybackslash}p{(\linewidth - 2\tabcolsep) * \real{0.5556}}@{}}
\toprule\noalign{}
\begin{minipage}[b]{\linewidth}\raggedright
Advantages
\end{minipage} & \begin{minipage}[b]{\linewidth}\raggedright
Disadvantages
\end{minipage} \\
\midrule\noalign{}
\endhead
\bottomrule\noalign{}
\endlastfoot
Stabilizes gain & Reduces overall gain \\
Increases bandwidth & Requires additional components \\
Reduces distortion & May cause oscillation if improperly designed \\
Reduces noise & Requires careful phase compensation \\
Improves input/output impedance & Increases power consumption \\
Reduces temperature sensitivity & Makes circuit more complex \\
Controls frequency response & May reduce signal-to-noise ratio in some
cases \\
\end{longtable}
}

\textbf{Diagram:}

\begin{center}
\textbf{Mermaid Diagram (Code)}
\begin{verbatim}
{Shaded}
{Highlighting}[]
graph TD
    A[Negative Feedback] {-{-}{} B[Advantages]}
    A {-{-}{} C[Disadvantages]}
    
    B {-{-}{} D[Stable Gain]}
    B {-{-}{} E[Wider Bandwidth]}
    B {-{-}{} F[Lower Distortion]}
    B {-{-}{} G[Better Impedance]}
    
    C {-{-}{} H[Reduced Gain]}
    C {-{-}{} I[More Components]}
    C {-{-}{} J[Complex Design]}
{Highlighting}
{Shaded}
\end{verbatim}
\end{center}

\begin{itemize}
\tightlist
\item
  \textbf{Performance tradeoff}: Sacrifices gain to achieve better
  stability and linearity
\item
  \textbf{Frequency considerations}: May require compensation to prevent
  oscillations at high frequencies
\item
  \textbf{Design complexity}: More complex to design properly but offers
  better long-term performance
\end{itemize}

\end{solutionbox}
\begin{mnemonicbox}
``Stability Grows As Gain Drops''

\end{mnemonicbox}
\subsection*{Question 1(c) OR [7
marks]}\label{q1c}

\textbf{Explain Voltage series feedback amplifier in detail with block
diagram and draw the Practical voltage series feedback circuit.}

\begin{solutionbox}

{\def\LTcaptype{none} % do not increment counter
\begin{longtable}[]{@{}ll@{}}
\toprule\noalign{}
Parameter & Effect in Voltage Series Feedback \\
\midrule\noalign{}
\endhead
\bottomrule\noalign{}
\endlastfoot
Input signal & Voltage \\
Feedback signal & Voltage \\
Input impedance & Increases \\
Output impedance & Decreases \\
Gain stability & Improves \\
Bandwidth & Increases \\
\end{longtable}
}

\textbf{Diagram:}

\begin{center}
\textbf{Mermaid Diagram (Code)}
\begin{verbatim}
{Shaded}
{Highlighting}[]
graph LR
    A[Input Vi] {-{-}{} B["{}+"]}
    B {-{-}{} C[Amplifier A]}
    C {-{-}{} D[Output Vo]}
    D {-{-}{} E[Feedback Network β]}
    E {-{-}{} F["{}{-}"]}
    F {-{-}{} B}

    style C fill:\#bbf,stroke:\#333,stroke{-width:1px}
    style E fill:\#fbb,stroke:\#333,stroke{-width:1px}
{Highlighting}
{Shaded}
\end{verbatim}
\end{center}

\textbf{Practical Circuit:}

\begin{verbatim}
          +Vcc
            |
            R2
            |
            +{-{-}{-}{-}{-}+}
            |     |
Vin o{-{-}{-}R1{-}{-}+{-}{-}+  |}
            |  |  |
            C1 | C2
            |  |  |
            |  +{-{-}+{-}{-}{-}{-}o Vout}
            |     |
            RE   RC
            |     |
            +{-{-}{-}{-}{-}+}
            |
           GND
\end{verbatim}

\begin{itemize}
\tightlist
\item
  \textbf{Sampling method}: Output voltage is sampled and fed back to
  input
\item
  \textbf{Mixing method}: Feedback signal is mixed in series with input
  signal
\item
  \textbf{Working principle}: Reduces gain for improved stability and
  linearity
\item
  \textbf{Applications}: Audio amplifiers, instrumentation amplifiers
\end{itemize}

\end{solutionbox}
\begin{mnemonicbox}
``Voltage Series - Impedance In Up, Out Down''

\end{mnemonicbox}
\subsection*{Question 2(a) [3 marks]}\label{q2a}

\textbf{Write short note on Colpitts oscillator circuit.}

\begin{solutionbox}

{\def\LTcaptype{none} % do not increment counter
\begin{longtable}[]{@{}ll@{}}
\toprule\noalign{}
Component & Function \\
\midrule\noalign{}
\endhead
\bottomrule\noalign{}
\endlastfoot
LC Tank & Determines oscillation frequency \\
Capacitive Voltage Divider & Provides feedback \\
Active Device & Provides gain to sustain oscillations \\
\end{longtable}
}

\textbf{Diagram:}

\begin{verbatim}
     +Vcc
       |
       R1
       |
       +{-{-}{-}{-}{-}+}
       |     |
       |     C3
       |     |
       +{-{-}+{-}{-}+{-}{-}{-}o Output}
       |  |  |
       L1 |  |
       |  |  |
       +{-{-}+  |}
       |     |
       C1    |
       |     |
       +{-{-}{-}{-}{-}+}
       |     |
       C2    |
       |     |
      GND   GND
\end{verbatim}

\begin{itemize}
\tightlist
\item
  \textbf{Frequency formula}: f = 1/(2π\sqrt(L\times(C1\timesC2)/(C1+C2)))
\item
  \textbf{Feedback}: Provided by capacitive voltage divider (C1 and C2)
\item
  \textbf{Applications}: RF oscillators, communication circuits
\end{itemize}

\end{solutionbox}
\begin{mnemonicbox}
``Colpitts Contains Capacitive divider''

\end{mnemonicbox}
\subsection*{Question 2(b) [4 marks]}\label{q2b}

\textbf{Explain requirement of oscillator. i) Barkhausen Criterion. ii)
Tank circuit. iii) Amplifier.}

\begin{solutionbox}

{\def\LTcaptype{none} % do not increment counter
\begin{longtable}[]{@{}
  >{\raggedright\arraybackslash}p{(\linewidth - 4\tabcolsep) * \real{0.3611}}
  >{\raggedright\arraybackslash}p{(\linewidth - 4\tabcolsep) * \real{0.2778}}
  >{\raggedright\arraybackslash}p{(\linewidth - 4\tabcolsep) * \real{0.3611}}@{}}
\toprule\noalign{}
\begin{minipage}[b]{\linewidth}\raggedright
Requirement
\end{minipage} & \begin{minipage}[b]{\linewidth}\raggedright
Function
\end{minipage} & \begin{minipage}[b]{\linewidth}\raggedright
Explanation
\end{minipage} \\
\midrule\noalign{}
\endhead
\bottomrule\noalign{}
\endlastfoot
Barkhausen Criterion & Ensures sustained oscillation & Loop gain = 1,
Phase shift = 0^\circ or 360^\circ \\
Tank Circuit & Determines frequency & Resonant LC circuit that stores
energy \\
Amplifier & Provides gain & Compensates for circuit losses \\
\end{longtable}
}

\textbf{Diagram:}

\begin{center}
\textbf{Mermaid Diagram (Code)}
\begin{verbatim}
{Shaded}
{Highlighting}[]
graph TD
    A[Oscillator] {-{-}{} B[Barkhausen Criterion]}
    A {-{-}{} C[Tank Circuit]}
    A {-{-}{} D[Amplifier]}
    
    B {-{-}{} E[Loop Gain = 1]}
    B {-{-}{} F[Phase Shift = 0^ or 360^]}
    
    C {-{-}{} G[Energy Storage]}
    C {-{-}{} H[Frequency Determination]}
    
    D {-{-}{} I[Overcome Losses]}
    D {-{-}{} J[Maintain Amplitude]}
{Highlighting}
{Shaded}
\end{verbatim}
\end{center}

\begin{itemize}
\tightlist
\item
  \textbf{Barkhausen Criterion}: Mathematical condition for sustained
  oscillations without damping
\item
  \textbf{Tank Circuit}: LC circuit that determines frequency of
  oscillations
\item
  \textbf{Amplifier}: Active device that provides energy to maintain
  oscillations
\end{itemize}

\end{solutionbox}
\begin{mnemonicbox}
``BAT - Barkhausen Amplifies Tank''

\end{mnemonicbox}
\subsection*{Question 2(c) [7 marks]}\label{q2c}

\textbf{Explain construction, working and V-I characteristics of UJT.}

\begin{solutionbox}

{\def\LTcaptype{none} % do not increment counter
\begin{longtable}[]{@{}ll@{}}
\toprule\noalign{}
Parameter & Description \\
\midrule\noalign{}
\endhead
\bottomrule\noalign{}
\endlastfoot
Construction & Silicon bar with two base connections and one emitter \\
Symbol & Triangle with emitter on one side and two bases \\
Equivalent Circuit & Voltage divider with diode \\
Key Parameter & Intrinsic standoff ratio (η) \\
\end{longtable}
}

\textbf{Diagram:}

\begin{verbatim}
         E
         |
         v
    +{-{-}{-}{-}+{-}{-}{-}{-}+}
    |    |    |
    |    D    |
    |    |    |
B1 o+{-{-}{-}www{-}{-}{-}+o B2}
         R1    R2
     
UJT Symbol \& Equivalent Circuit
\end{verbatim}

\textbf{V-I Characteristic Curve:}

\begin{verbatim}
  I
  \^{}
  |
  |       Peak point
  |         o
  |        /|
  |       / |
  |      /  |
  |     /   |
  |    /    |
  |   /     |
  |  /      |
  | /       |
  |/        |
  +{-{-}{-}{-}{-}{-}{-}{-}{-} V}
  |
  | Valley point
\end{verbatim}

\begin{itemize}
\tightlist
\item
  \textbf{Construction}: N-type silicon bar with P-type emitter junction
\item
  \textbf{Working principle}: When emitter voltage \textgreater{}
  (η\timesVBB), device conducts
\item
  \textbf{Regions of operation}: Cut-off, negative resistance, and
  saturation
\item
  \textbf{Applications}: Relaxation oscillators, timing circuits,
  triggering devices
\end{itemize}

\end{solutionbox}
\begin{mnemonicbox}
``UJT Peaks Then Valleys - Negative Resistance
Rules''

\end{mnemonicbox}
\subsection*{Question 2(a) OR [3
marks]}\label{q2a}

\textbf{State the advantages, disadvantages and applications of Hartley
oscillator.}

\begin{solutionbox}

{\def\LTcaptype{none} % do not increment counter
\begin{longtable}[]{@{}
  >{\raggedright\arraybackslash}p{(\linewidth - 4\tabcolsep) * \real{0.2927}}
  >{\raggedright\arraybackslash}p{(\linewidth - 4\tabcolsep) * \real{0.3659}}
  >{\raggedright\arraybackslash}p{(\linewidth - 4\tabcolsep) * \real{0.3415}}@{}}
\toprule\noalign{}
\begin{minipage}[b]{\linewidth}\raggedright
Advantages
\end{minipage} & \begin{minipage}[b]{\linewidth}\raggedright
Disadvantages
\end{minipage} & \begin{minipage}[b]{\linewidth}\raggedright
Applications
\end{minipage} \\
\midrule\noalign{}
\endhead
\bottomrule\noalign{}
\endlastfoot
Easy tuning & Bulky inductors & RF generators \\
Wide frequency range & Mutual inductance issues & Radio receivers \\
Simple design & Difficult at high frequencies & Amateur radio \\
Good frequency stability & Requires center-tapped coil & Communication
equipment \\
\end{longtable}
}

\textbf{Diagram:}

\begin{verbatim}
            +Vcc
              |
              R1
              |
              +{-{-}{-}{-}{-}+}
              |     |
              |     C2
              |     |
              +{-{-}+{-}{-}+{-}{-}{-}{-}o Output}
              |  |  |
              L1 |  |
              |  |  |
              L2 |  |
              |  |  |
              +{-{-}+  |}
              |     |
              C1    |
              |     |
             GND   GND
\end{verbatim}

\begin{itemize}
\tightlist
\item
  \textbf{Key feature}: Uses tapped inductor for feedback
\item
  \textbf{Frequency formula}: f = 1/(2π\sqrt(C\times(L1+L2)))
\item
  \textbf{Distinguishing characteristic}: Inductive voltage divider for
  feedback
\end{itemize}

\end{solutionbox}
\begin{mnemonicbox}
``Hartley Has tapped Inductor''

\end{mnemonicbox}
\subsection*{Question 2(b) OR [4
marks]}\label{q2b}

\textbf{Explain UJT as relaxation oscillator.}

\begin{solutionbox}

{\def\LTcaptype{none} % do not increment counter
\begin{longtable}[]{@{}ll@{}}
\toprule\noalign{}
Component & Function \\
\midrule\noalign{}
\endhead
\bottomrule\noalign{}
\endlastfoot
UJT & Provides switching action \\
Capacitor & Timing element \\
Resistor & Controls charging rate \\
Output & Sawtooth waveform \\
\end{longtable}
}

\textbf{Diagram:}

\begin{verbatim}
      +Vcc
        |
        R
        |
        +{-{-}{-}{-}{-}{-}{-}+}
        |       |
        |       |
        +{-{-}| |{-}{-}+{-}{-}{-}{-}o Output}
        |   C   |
        |       |
        E       |
       UJT      |
       B1  B2   |
        |   |   |
        +{-{-}{-}+{-}{-}{-}+}
            |
           GND
\end{verbatim}

\textbf{Waveforms:}

\begin{verbatim}
  Vc
  \^{}
  |  /|  /|  /|
  | / | / | / |
  |/  |/  |/  |
  +{-{-}{-}{-}{-}{-}{-}{-}{-}{-}{-}{-} t}

  Vo
  \^{}
  |
  |  \_   \_   \_
  | | | | | | |
  |\_| |\_| |\_| |\_
  +{-{-}{-}{-}{-}{-}{-}{-}{-}{-}{-}{-} t}
\end{verbatim}

\begin{itemize}
\tightlist
\item
  \textbf{Operating principle}: Capacitor charges until UJT firing
  voltage, then rapidly discharges
\item
  \textbf{Frequency formula}: f \approx 1/(RC\timesln(1/(1-η)))
\item
  \textbf{Applications}: Timing circuits, pulse generators, control
  systems
\end{itemize}

\end{solutionbox}
\begin{mnemonicbox}
``Charge-Fire-Repeat - Sawtooth's Beat''

\end{mnemonicbox}
\subsection*{Question 2(c) OR [7
marks]}\label{q2c}

\textbf{Explain working of weinbridge oscillator with neat diagram also
state the advantage, disadvantage and application for the same.}

\begin{solutionbox}

{\def\LTcaptype{none} % do not increment counter
\begin{longtable}[]{@{}ll@{}}
\toprule\noalign{}
Parameter & Description \\
\midrule\noalign{}
\endhead
\bottomrule\noalign{}
\endlastfoot
Configuration & RC feedback network in bridge formation \\
Frequency Formula &

f = 1/(2πRC) when R1=R3 and C2=C4 \\

Feedback & Positive feedback through RC network \\
Phase Shift & 0^\circ at resonant frequency \\
\end{longtable}
}

\textbf{Diagram:}

\begin{center}
\textbf{Mermaid Diagram (Code)}
\begin{verbatim}
{Shaded}
{Highlighting}[]
graph LR
    A[Amplifier] {-{-}{} B[RC Bridge]}
    B {-{-}{} A}
    
    subgraph "Wien Bridge Network"
    direction LR
    C[R1] {-{-}{-} D[C1]}
    D {-{-}{-} E[R2]}
    E {-{-}{-} F[C2]}
    F {-{-}{-} C}
    end
{Highlighting}
{Shaded}
\end{verbatim}
\end{center}

\textbf{Circuit:}

\begin{verbatim}
                +Vcc
                  |
                  |
                  v
    +{-{-}{-}R2{-}{-}{-}+{-}{-}{-}{-}+{-}{-}{-}{-}+}
    |        |         |
    C2       |        R4
    |        |         |
    +{-{-}{-}+    +    +{-}{-}{-}{-}+}
    |   |    |    |    |
    |   +{-{-}{-}{-}+{-}{-}{-}{-}+    |}
    |        |         |
    R1       +        R3
    |        |         |
    C1       v        R5
    |       Op{-Amp     |}
    +{-{-}{-}{-}{-}{-}{-}{-}+         |}
             |         |
             +{-{-}{-}{-}{-}{-}{-}{-}{-}+}
                  |
                 GND
\end{verbatim}

\textbf{Advantages:}

\begin{itemize}
\tightlist
\item
  High frequency stability
\item
  Low distortion output
\item
  Simple RC components
\item
  Easy to tune
\end{itemize}

\textbf{Disadvantages:}

\begin{itemize}
\tightlist
\item
  Limited frequency range
\item
  Amplitude stabilization needed
\item
  Sensitive to component variations
\item
  Difficult to start oscillations
\end{itemize}

\textbf{Applications:}

\begin{itemize}
\tightlist
\item
  Audio test equipment
\item
  Function generators
\item
  Musical instruments
\item
  Laboratory signal sources
\end{itemize}

\end{solutionbox}
\begin{mnemonicbox}
``Wien Works at R1C1=R2C2 frequency''

\end{mnemonicbox}
\subsection*{Question 3(a) [3 marks]}\label{q3a}

\textbf{Give classification of power Amplifier.}

\begin{solutionbox}

{\def\LTcaptype{none} % do not increment counter
\begin{longtable}[]{@{}ll@{}}
\toprule\noalign{}
Classification Basis & Types \\
\midrule\noalign{}
\endhead
\bottomrule\noalign{}
\endlastfoot
Based on Conduction Angle & Class A, B, AB, C \\
Based on Configuration & Single-ended, Push-pull, Complementary \\
Based on Coupling & RC coupled, Transformer coupled, Direct coupled \\
Based on Operation & Linear, Switching \\
\end{longtable}
}

\textbf{Diagram:}

\begin{center}
\textbf{Mermaid Diagram (Code)}
\begin{verbatim}
{Shaded}
{Highlighting}[]
graph TD
    A[Power Amplifiers]
    A {-{-}{} B[Class A {-} 360^]}
    A {-{-}{} C[Class B {-} 180^]}
    A {-{-}{} D[Class AB {-} 180^{-}360^]}
    A {-{-}{} E[Class C {} 180^]}
    
    style B fill:\#d4f0f0,stroke:\#333
    style C fill:\#d4f0f0,stroke:\#333
    style D fill:\#d4f0f0,stroke:\#333
    style E fill:\#d4f0f0,stroke:\#333
{Highlighting}
{Shaded}
\end{verbatim}
\end{center}

\begin{itemize}
\tightlist
\item
  \textbf{Class A}: Conducts for full 360^\circ cycle, highest linearity,
  lowest efficiency
\item
  \textbf{Class B}: Conducts for 180^\circ cycle, medium distortion, medium
  efficiency
\item
  \textbf{Class AB}: Conducts for 180^\circ-360^\circ cycle, good linearity, good
  efficiency
\item
  \textbf{Class C}: Conducts for \textless180^\circ cycle, highest
  distortion, highest efficiency
\end{itemize}

\end{solutionbox}
\begin{mnemonicbox}
``A All-time, B Bisects, AB Almost-Bisects, C
Cuts-more''

\end{mnemonicbox}
\subsection*{Question 3(b) [4 marks]}\label{q3b}

\textbf{Explain class A power amplifier.}

\begin{solutionbox}

{\def\LTcaptype{none} % do not increment counter
\begin{longtable}[]{@{}ll@{}}
\toprule\noalign{}
Parameter & Class A Amplifier \\
\midrule\noalign{}
\endhead
\bottomrule\noalign{}
\endlastfoot
Conduction Angle & 360^\circ (full cycle) \\
Biasing & Q-point at center of load line \\
Efficiency & Low (25-30\% max) \\
Distortion & Very low \\
\end{longtable}
}

\textbf{Diagram:}

\begin{verbatim}
         +Vcc
           |
           |
         Rcollector
           |
           +{-{-}{-}{-}{-}+}
           |     |
           |     +{-{-}{-} Output}
           |     |
       +{-{-}{-}+     |}
       |   |     |
   In {-+   Q1    |}
       |   |     |
       +{-{-}{-}+     |}
           |     |
         Remitter|
           |     |
           +{-{-}{-}{-}{-}+}
           |
          GND
\end{verbatim}

\textbf{Load Line:}

\begin{verbatim}
 Ic
  \^{}
  |          Load Line
  |         /
  |        /
  |       /
  |      *  Q{-point}
  |     /
  |    /
  |   /
  |  /
  | /
  |/
  +{-{-}{-}{-}{-}{-}{-}{-}{-}{-}{-}{-}{-}{-}{-} Vce}
\end{verbatim}

\begin{itemize}
\tightlist
\item
  \textbf{Operating principle}: Transistor conducts for entire input
  cycle
\item
  \textbf{Efficiency calculation}: Maximum theoretical efficiency = 50\%
\item
  \textbf{Practical efficiency}: Typically 25-30\% due to losses
\item
  \textbf{Applications}: Audio pre-amplifiers, low-power amplifiers
  where quality matters more than efficiency
\end{itemize}

\end{solutionbox}
\begin{mnemonicbox}
``Class A - Always conducting, All cycle''

\end{mnemonicbox}
\subsection*{Question 3(c) [7 marks]}\label{q3c}

\textbf{Explain the principle of push pull amplifiers and write short
note on class B push pull amplifier.}

\begin{solutionbox}

{\def\LTcaptype{none} % do not increment counter
\begin{longtable}[]{@{}
  >{\raggedright\arraybackslash}p{(\linewidth - 2\tabcolsep) * \real{0.5250}}
  >{\raggedright\arraybackslash}p{(\linewidth - 2\tabcolsep) * \real{0.4750}}@{}}
\toprule\noalign{}
\begin{minipage}[b]{\linewidth}\raggedright
Push-Pull Principle
\end{minipage} & \begin{minipage}[b]{\linewidth}\raggedright
Class B Push-Pull
\end{minipage} \\
\midrule\noalign{}
\endhead
\bottomrule\noalign{}
\endlastfoot
Uses two complementary devices & Each transistor conducts for half
cycle \\
Reduces even harmonic distortion & Higher efficiency (78.5\%
theoretical) \\
Cancels DC magnetization in transformer & Suffers from crossover
distortion \\
Provides higher output power & Requires proper biasing to minimize
distortion \\
\end{longtable}
}

\textbf{Diagram:}

\begin{verbatim}
           +Vcc
             |
             |
        +{-{-}{-}{-}+{-}{-}{-}{-}+}
        |         |
        Q1        Q2
        |         |
        +{-{-}{-}{-}+{-}{-}{-}{-}+}
             |
             +{-{-}{-}{-}{-}{-} Output}
             |
             R
             |
            GND
\end{verbatim}

\textbf{Waveforms:}

\begin{verbatim}
  Input      Q1 Current    Q2 Current     Output
    \^{            \^{}             \^{}             \^{}}
    |            |             |             |
    |  /{        |  /         |    /       |  /}
    | /  {       | /          |   /        | /  }
{-{-}{-}{-}+{-}{-}{-}{-}{-}{-}     {-}+{-}{-}{-}{-}{-}{-}      {-}+{-}{-}{-}{-}{-}{-}{-}     {-}+{-}{-}{-}{-}{-}{-}}
    |    {       |             |            |    }
    |     {      |             |            |     }
    |      {     |             |            |      }
    v       v    v             v       v     v       v
\end{verbatim}

\begin{itemize}
\tightlist
\item
  \textbf{Working principle}: Each transistor conducts for alternate
  half-cycles
\item
  \textbf{Advantages}: Higher efficiency, reduced even harmonics, lower
  heat generation
\item
  \textbf{Disadvantages}: Crossover distortion at transition points
\item
  \textbf{Applications}: Audio power amplifiers, output stages of
  high-power systems
\end{itemize}

\end{solutionbox}
\begin{mnemonicbox}
``Push-Pull: Pair Processes alternate Pulses''

\end{mnemonicbox}
\subsection*{Question 3(a) OR [3
marks]}\label{q3a}

\textbf{Discuss crossover distortion in push pull amplifier. How it can
be removed.}

\begin{solutionbox}

{\def\LTcaptype{none} % do not increment counter
\begin{longtable}[]{@{}
  >{\raggedright\arraybackslash}p{(\linewidth - 2\tabcolsep) * \real{0.5500}}
  >{\raggedright\arraybackslash}p{(\linewidth - 2\tabcolsep) * \real{0.4500}}@{}}
\toprule\noalign{}
\begin{minipage}[b]{\linewidth}\raggedright
Crossover Distortion
\end{minipage} & \begin{minipage}[b]{\linewidth}\raggedright
Solution Methods
\end{minipage} \\
\midrule\noalign{}
\endhead
\bottomrule\noalign{}
\endlastfoot
Occurs at signal crossover points & Apply small bias voltage (Class
AB) \\
Due to transistor's non-linear region & Use diode compensation
networks \\
Creates ``dead zone'' around zero & Implement feedback correction \\
Affects small signals more & Use complementary emitter-follower stage \\
\end{longtable}
}

\textbf{Diagram:}

\begin{verbatim}
  Input          Output with Distortion
    \^{                  \^{}}
    |                  |
    |  /{              |   /}
    | /  {             |  /  }
{-{-}{-}{-}+{-}{-}{-}{-}{-}{-}           {-}+{-}{-}{-}{-}{-}{-}}
    |    {             |     }
    |     {            | gap  }
    |      {           |       }
    v       v          v        v
\end{verbatim}

\textbf{Correction Circuit:}

\begin{verbatim}
          +Vcc
            |
            |
       +{-{-}{-}{-}+{-}{-}{-}{-}+}
       |    |    |
       |    R    |
       |    |    |
       |    D1   |
       |    |    |
       Q1   +    Q2
       |    |    |
       |    D2   |
       |    |    |
       +{-{-}{-}{-}+{-}{-}{-}{-}+}
            |
            R
            |
           GND
\end{verbatim}

\begin{itemize}
\tightlist
\item
  \textbf{Cause}: Transistors require \textasciitilde0.7V to turn on,
  creating dead zone
\item
  \textbf{Effect}: Distortion particularly noticeable at low volumes
\item
  \textbf{Solution}: Class AB biasing with diodes or VBE multiplier
\item
  \textbf{Result}: Smoother transition between positive and negative
  half-cycles
\end{itemize}

\end{solutionbox}
\begin{mnemonicbox}
``Cross to Class AB Smooths the Gap''

\end{mnemonicbox}
\subsection*{Question 3(b) OR [4
marks]}\label{q3b}

\textbf{Explain complimentary symmetry push-pull amplifier.}

\begin{solutionbox}

{\def\LTcaptype{none} % do not increment counter
\begin{longtable}[]{@{}ll@{}}
\toprule\noalign{}
Component & Purpose \\
\midrule\noalign{}
\endhead
\bottomrule\noalign{}
\endlastfoot
NPN Transistor & Handles positive half-cycle \\
PNP Transistor & Handles negative half-cycle \\
Biasing Network & Reduces crossover distortion \\
Output Coupling & Direct coupling to load \\
\end{longtable}
}

\textbf{Diagram:}

\begin{verbatim}
          +Vcc
            |
            |
            Q1 (NPN)
            |
      R1    |
       +{-{-}{-}{-}+}
       |    |
Input  |    +{-{-}{-}{-}{-}o Output}
       |    |
       +{-{-}{-}{-}+}
            |
            Q2 (PNP)
            |
            |
           GND
\end{verbatim}

\textbf{Working Principle:}

\begin{center}
\textbf{Mermaid Diagram (Code)}
\begin{verbatim}
{Shaded}
{Highlighting}[]
graph LR
    A[Input Signal] {-{-}{} B\{Voltage Polarity\}}
    B {-{-}{}|Positive| C[NPN Conducts]}
    B {-{-}{}|Negative| D[PNP Conducts]}
    C {-{-}{} E[Output]}
    D {-{-}{} E}
{Highlighting}
{Shaded}
\end{verbatim}
\end{center}

\begin{itemize}
\tightlist
\item
  \textbf{Key feature}: Uses complementary transistors (NPN and PNP) for
  push-pull operation
\item
  \textbf{Advantage}: No output transformer required, direct coupling to
  load
\item
  \textbf{Efficiency}: Typically 78.5\% theoretical maximum
\item
  \textbf{Applications}: Audio amplifiers, power output stages
\end{itemize}

\end{solutionbox}
\begin{mnemonicbox}
``NPN Pulls-up, PNP Pulls-down''

\end{mnemonicbox}
\subsection*{Question 3(c) OR [7
marks]}\label{q3c}

\textbf{Derive the equation of efficiency for class B push pull
Amplifier.}

\begin{solutionbox}

{\def\LTcaptype{none} % do not increment counter
\begin{longtable}[]{@{}lll@{}}
\toprule\noalign{}
Parameter & Formula & Description \\
\midrule\noalign{}
\endhead
\bottomrule\noalign{}
\endlastfoot
DC Input Power & PDC = 2VCC\timesIDC & Power drawn from supply \\
AC Output Power & PAC = Vrms^{2}/RL & Power delivered to load \\
Maximum Efficiency &

η = (π/4)\times100\% = 78.5\% & Theoretical maximum \\

Practical Efficiency & 60-70\% & Considering losses \\
\end{longtable}
}

\textbf{Mathematical Derivation:}

For a sinusoidal input: v(t) = Vm sin(ωt)

\textbf{Step 1}: DC Input Power

\begin{itemize}
\tightlist
\item
  Input current per transistor: Im/π
\item
  Total DC input power: PDC = 2VCC\timesIm/π
\end{itemize}

\textbf{Step 2}: AC Output Power

\begin{itemize}
\tightlist
\item
  RMS output voltage: Vrms = Vm/\sqrt2
\item
  Maximum output voltage: Vm = VCC
\item
  Output power: PAC = Vrms^{2}/RL = Vm^{2}/2RL
\end{itemize}

\textbf{Step 3}: Efficiency Calculation

\begin{itemize}
\tightlist
\item
  η = (PAC/PDC)\times100\%
\item
  η = ((Vm^{2}/2RL)/(2VCC\timesIm/π))\times100\%
\item
  Since Vm = VCC and Im = VCC/RL
\item
  η = (π/4)\times100\% = 78.5\%
\end{itemize}

\textbf{Diagram:}

\begin{verbatim}
 Vm=VCC
    \^{}
    |         /{}
    |        /  {}
    |       /    {}
    |      /      {}
    |     /        {}
0   +{-{-}{-}{-}/{-}{-}{-}{-}{-}{-}{-}{-}{-}{-}{-}{-}{-}{-}{-} t}
    |   /            {}
    |  /              {}
    | /                {}
    |/                  {}
    v
\end{verbatim}

\begin{itemize}
\tightlist
\item
  \textbf{Power dissipation}: Most efficient when output voltage swing
  approaches VCC
\item
  \textbf{Conduction angle}: Each transistor conducts for exactly 180^\circ
\item
  \textbf{Practical factors}: Biasing current, saturation voltage, and
  other losses reduce efficiency
\item
  \textbf{Comparison}: Much higher than Class A (25-30\%), less than
  Class C (\textgreater80\%)
\end{itemize}

\end{solutionbox}
\begin{mnemonicbox}
``Pi-over-4 gives 78.5\% - Class B's best''

\end{mnemonicbox}
\subsection*{Question 4(a) [3 marks]}\label{q4a}

\textbf{Define.(i) CMRR (ii)slew rate.(iii)Input offset Current.}

\begin{solutionbox}

{\def\LTcaptype{none} % do not increment counter
\begin{longtable}[]{@{}
  >{\raggedright\arraybackslash}p{(\linewidth - 4\tabcolsep) * \real{0.2821}}
  >{\raggedright\arraybackslash}p{(\linewidth - 4\tabcolsep) * \real{0.3077}}
  >{\raggedright\arraybackslash}p{(\linewidth - 4\tabcolsep) * \real{0.4103}}@{}}
\toprule\noalign{}
\begin{minipage}[b]{\linewidth}\raggedright
Parameter
\end{minipage} & \begin{minipage}[b]{\linewidth}\raggedright
Definition
\end{minipage} & \begin{minipage}[b]{\linewidth}\raggedright
Typical Values
\end{minipage} \\
\midrule\noalign{}
\endhead
\bottomrule\noalign{}
\endlastfoot
CMRR & Ratio of differential gain to common-mode gain & 80-120 dB \\
Slew Rate & Maximum rate of change of output voltage & 0.5-20 V/μs \\
Input Offset Current & Difference between currents into the two inputs &
1-100 nA \\
\end{longtable}
}

\textbf{Diagram:}

\begin{center}
\textbf{Mermaid Diagram (Code)}
\begin{verbatim}
{Shaded}
{Highlighting}[]
graph TD
    A[Op{-Amp Parameters]}
    A {-{-}{} B[CMRR = Ad/Acm]}
    A {-{-}{} C[Slew Rate = dVo/dt]}
    A {-{-}{} D["IOS = |I+ {-} I{-}|"]}
    
    style B fill:\#f9f9f9,stroke:\#333
    style C fill:\#f9f9f9,stroke:\#333
    style D fill:\#f9f9f9,stroke:\#333
{Highlighting}
{Shaded}
\end{verbatim}
\end{center}

\begin{itemize}
\tightlist
\item
  \textbf{CMRR}: Measures op-amp's ability to reject common-mode signals
\item
  \textbf{Slew Rate}: Limits maximum frequency for undistorted output
\item
  \textbf{Input Offset Current}: Causes output error even with identical
  inputs
\end{itemize}

\end{solutionbox}
\begin{mnemonicbox}
``Cancelling Mistakes Requires Ratios''

\end{mnemonicbox}
\subsection*{Question 4(b) [4 marks]}\label{q4b}

\textbf{Draw and explain the basic block diagram of an operational
amplifier.}

\begin{solutionbox}

{\def\LTcaptype{none} % do not increment counter
\begin{longtable}[]{@{}ll@{}}
\toprule\noalign{}
Stage & Function \\
\midrule\noalign{}
\endhead
\bottomrule\noalign{}
\endlastfoot
Differential Input & Accepts and amplifies difference between inputs \\
High-Gain Intermediate & Provides voltage amplification \\
Level Shifter & Shifts DC level for output stage \\
Output Buffer & Provides low output impedance \\
\end{longtable}
}

\textbf{Diagram:}

\begin{center}
\textbf{Mermaid Diagram (Code)}
\begin{verbatim}
{Shaded}
{Highlighting}[]
graph LR
    A[Inverting Input] {-{-}{} B[Differential Input Stage]}
    C[Non{-inverting Input] {-}{-}{} B}
    B {-{-}{} D[High{-}Gain Intermediate Stage]}
    D {-{-}{} E[Level Shifter]}
    E {-{-}{} F[Output Buffer]}
    F {-{-}{} G[Output]}
    
    style B fill:\#d4f0f0,stroke:\#333
    style D fill:\#d4f0f0,stroke:\#333
    style E fill:\#d4f0f0,stroke:\#333
    style F fill:\#d4f0f0,stroke:\#333
{Highlighting}
{Shaded}
\end{verbatim}
\end{center}

\begin{itemize}
\tightlist
\item
  \textbf{Differential input stage}: Converts differential input to
  single-ended output
\item
  \textbf{High-gain stage}: Provides most of the open-loop gain
\item
  \textbf{Level shifter}: Shifts signal level for proper output
  operation
\item
  \textbf{Output stage}: Provides current gain and low output impedance
\end{itemize}

\end{solutionbox}
\begin{mnemonicbox}
``Diff-Amp Gain Shift Out''

\end{mnemonicbox}
\subsection*{Question 4(c) [7 marks]}\label{q4c}

\textbf{Explain in detail operational amplifier as integrator.}

\begin{solutionbox}

{\def\LTcaptype{none} % do not increment counter
\begin{longtable}[]{@{}
  >{\raggedright\arraybackslash}p{(\linewidth - 4\tabcolsep) * \real{0.3333}}
  >{\raggedright\arraybackslash}p{(\linewidth - 4\tabcolsep) * \real{0.3939}}
  >{\raggedright\arraybackslash}p{(\linewidth - 4\tabcolsep) * \real{0.2727}}@{}}
\toprule\noalign{}
\begin{minipage}[b]{\linewidth}\raggedright
Parameter
\end{minipage} & \begin{minipage}[b]{\linewidth}\raggedright
Description
\end{minipage} & \begin{minipage}[b]{\linewidth}\raggedright
Formula
\end{minipage} \\
\midrule\noalign{}
\endhead
\bottomrule\noalign{}
\endlastfoot
Circuit & Op-amp with capacitor in feedback & - \\
Transfer Function & Output proportional to integral of input & Vo =
-(1/RC)\intVi dt \\
Frequency Response & Acts as low-pass filter & Gain = 1/(jωRC) \\
Phase Shift & -90^\circ & - \\
\end{longtable}
}

\textbf{Diagram:}

\begin{verbatim}
              C
       +{-{-}{-}{-}{-}{-}||{-}{-}{-}{-}{-}{-}+}
       |              |
       |    +{-{-}{-}{-}{-}+   |}
       |    |     |   |
       +{-{-}{-}{-}+  {-}  |   |}
       |    |     |   |
Vin o{-{-}+{-}{-}R{-}+     +{-}{-}{-}+{-}{-}o Vout}
            |  +  |
            |     |
            +{-{-}{-}{-}{-}+}
              |
             GND
\end{verbatim}

\textbf{Input/Output Waveforms:}

\begin{verbatim}
 Input Square Wave     Output Triangle Wave
       \_\_\_                      /|
      |   |                    / |
      |   |                   /  |
  \_\_\_\_|   |\_\_\_\_         \_\_\_\_/    |{\_\_\_\_}
      |   |                      |
      |   |                      |
      |\_\_\_|                      |
                                {|}
\end{verbatim}

\begin{itemize}
\tightlist
\item
  \textbf{Working principle}: Capacitor integrates current over time
\item
  \textbf{Mathematical basis}: Vo(t) = -(1/RC)\intVi(t)dt + Vo(0)
\item
  \textbf{Limitations}: Capacitor leakage, op-amp input bias current
  cause drift
\item
  \textbf{Applications}: Waveform generators, analog computers, active
  filters
\end{itemize}

\end{solutionbox}
\begin{mnemonicbox}
``Square-In Triangle-Out, RC sets the Slope''

\end{mnemonicbox}
\subsection*{Question 4(a) OR [3
marks]}\label{q4a}

\textbf{Explain operational amplifier as summing amplifier.}

\begin{solutionbox}

{\def\LTcaptype{none} % do not increment counter
\begin{longtable}[]{@{}
  >{\raggedright\arraybackslash}p{(\linewidth - 4\tabcolsep) * \real{0.3333}}
  >{\raggedright\arraybackslash}p{(\linewidth - 4\tabcolsep) * \real{0.3939}}
  >{\raggedright\arraybackslash}p{(\linewidth - 4\tabcolsep) * \real{0.2727}}@{}}
\toprule\noalign{}
\begin{minipage}[b]{\linewidth}\raggedright
Parameter
\end{minipage} & \begin{minipage}[b]{\linewidth}\raggedright
Description
\end{minipage} & \begin{minipage}[b]{\linewidth}\raggedright
Formula
\end{minipage} \\
\midrule\noalign{}
\endhead
\bottomrule\noalign{}
\endlastfoot
Circuit & Multiple inputs with same feedback & Vo = -(R_{1}/R_{1}\timesV_{1} +
R_{1}/R_{2}\timesV_{2} + \ldots) \\
Equal Resistors & Simple addition/averaging & Vo = -(V_{1} + V_{2} + \ldots{}
+ V_{n}) \\
Weighted Sum & Different input resistors & Vo = -(K_{1}V_{1} + K_{2}V_{2} + \ldots{}
+ K_{n}V_{n}) \\
Inverting & Output inverted from inputs & - \\
\end{longtable}
}

\textbf{Diagram:}

\begin{verbatim}
        R1
V1 o{-{-}{-}www{-}{-}{-}+}
             |
        R2   |    +{-{-}{-}{-}{-}+}
V2 o{-{-}{-}www{-}{-}{-}+{-}{-}{-}{-}+     |}
             |    |  {-  |}
        R3   |    |     |
V3 o{-{-}{-}www{-}{-}{-}+{-}{-}{-}{-}+     +{-}{-}{-}o Vout}
                  |  +  |
                  |     |
                  +{-{-}{-}{-}{-}+}
                    |
          Rf        |
        +{-{-}{-}www{-}{-}{-}{-}{-}+}
        |
       GND
\end{verbatim}

\begin{itemize}
\tightlist
\item
  \textbf{Working principle}: Each input contributes current to summing
  junction
\item
  \textbf{Applications}: Audio mixers, signal processing, analog
  computers
\item
  \textbf{Virtual ground}: Summing point maintains near-zero voltage
\item
  \textbf{Variations}: Inverting, non-inverting, and differential summer
\end{itemize}

\end{solutionbox}
\begin{mnemonicbox}
``Many Inputs, One Output - Sum It All''

\end{mnemonicbox}
\subsection*{Question 4(b) OR [4
marks]}\label{q4b}

\textbf{State the applications of operational amplifier.}

\begin{solutionbox}

{\def\LTcaptype{none} % do not increment counter
\begin{longtable}[]{@{}
  >{\raggedright\arraybackslash}p{(\linewidth - 2\tabcolsep) * \real{0.6875}}
  >{\raggedright\arraybackslash}p{(\linewidth - 2\tabcolsep) * \real{0.3125}}@{}}
\toprule\noalign{}
\begin{minipage}[b]{\linewidth}\raggedright
Application Category
\end{minipage} & \begin{minipage}[b]{\linewidth}\raggedright
Examples
\end{minipage} \\
\midrule\noalign{}
\endhead
\bottomrule\noalign{}
\endlastfoot
Signal Processing & Amplifiers, Filters, Buffers \\
Mathematical Operations & Adders, Subtractors, Integrators,
Differentiators \\
Waveform Generators & Sine, Square, Triangle, Pulse generators \\
Instrumentation & Instrumentation amplifiers, Current-to-voltage
converters \\
Comparators & Zero crossing detectors, Window comparators \\
Precision Rectifiers & Full-wave, Half-wave rectifiers \\
Voltage Regulators & Series regulators, Shunt regulators \\
\end{longtable}
}

\textbf{Diagram:}

\begin{center}
\textbf{Mermaid Diagram (Code)}
\begin{verbatim}
{Shaded}
{Highlighting}[]
graph TD
    A[Op{-Amp Applications]}
    A {-{-}{} B[Signal Processing]}
    A {-{-}{} C[Math Operations]}
    A {-{-}{} D[Waveform Generators]}
    A {-{-}{} E[Instrumentation]}
    A {-{-}{} F[Comparators]}
    A {-{-}{} G[Rectifiers]}
    A {-{-}{} H[Regulators]}
{Highlighting}
{Shaded}
\end{verbatim}
\end{center}

\begin{itemize}
\tightlist
\item
  \textbf{Linear applications}: Utilize op-amp in linear region for
  amplification, filtering
\item
  \textbf{Non-linear applications}: Use saturation characteristics for
  comparison, limitation
\item
  \textbf{Analog computation}: Perform mathematical operations on analog
  signals
\item
  \textbf{Signal conditioning}: Adapt signals for analog-to-digital
  conversion
\end{itemize}

\end{solutionbox}
\begin{mnemonicbox}
``SMWIG-CR: Signal, Math, Wave, Instrument, Gate,
Convert, Regulate''

\end{mnemonicbox}
\subsection*{Question 4(c) OR [7
marks]}\label{q4c}

\textbf{Explain op-amp as inverting and non-inverting amplifier.}

\begin{solutionbox}

{\def\LTcaptype{none} % do not increment counter
\begin{longtable}[]{@{}
  >{\raggedright\arraybackslash}p{(\linewidth - 4\tabcolsep) * \real{0.1930}}
  >{\raggedright\arraybackslash}p{(\linewidth - 4\tabcolsep) * \real{0.3684}}
  >{\raggedright\arraybackslash}p{(\linewidth - 4\tabcolsep) * \real{0.4386}}@{}}
\toprule\noalign{}
\begin{minipage}[b]{\linewidth}\raggedright
Parameter
\end{minipage} & \begin{minipage}[b]{\linewidth}\raggedright
Inverting Amplifier
\end{minipage} & \begin{minipage}[b]{\linewidth}\raggedright
Non-Inverting Amplifier
\end{minipage} \\
\midrule\noalign{}
\endhead
\bottomrule\noalign{}
\endlastfoot
Circuit Configuration & Input to negative terminal & Input to positive
terminal \\
Gain Formula &

A = -Rf/Rin &

A = 1 + Rf/Rin \\

Input Impedance & = Rin & Very high (\approx 10^{9} ohms) \\
Phase Shift & 180^\circ & 0^\circ \\
Virtual Ground & At negative input & Not applicable \\
\end{longtable}
}

\textbf{Inverting Amplifier:}

\begin{verbatim}
            Rf
        +{-{-}{-}{-}www{-}{-}{-}+}
        |          |
        |   +{-{-}{-}{-}{-}+|}
        |   |     ||
Vin o{-{-}{-}w{-}{-}{-}+  {-}  ||}
       Rin  |     ||
            |     +{-{-}{-}o Vout}
            |  +  |
            |     |
            +{-{-}{-}{-}{-}+}
              |
             GND
\end{verbatim}

\textbf{Non-Inverting Amplifier:}

\begin{verbatim}
                Rf
            +{-{-}{-}{-}www{-}{-}{-}+}
            |          |
            |   +{-{-}{-}{-}{-}+|}
            |   |     ||
Vin o{-{-}{-}{-}{-}{-}{-}+{-}{-}{-}+  +  ||}
            |   |     ||
            |   |     +{-{-}{-}o Vout}
            |   |  {-  |}
            |   |     |
            |   +{-{-}{-}{-}{-}+}
            |     |
            |    GND
            |
            +{-{-}{-}{-}www{-}{-}{-}{-}{-}+}
                 Rin     |
                        GND
\end{verbatim}

\textbf{Inverting Mode:}

\begin{itemize}
\tightlist
\item
  \textbf{Gain equation}: Vout = -(Rf/Rin)\timesVin
\item
  \textbf{Virtual ground}: Negative input maintained at
  \textasciitilde0V
\item
  \textbf{Applications}: Signal inversion, controlled gain, summing
\end{itemize}

\textbf{Non-Inverting Mode:}

\begin{itemize}
\tightlist
\item
  \textbf{Gain equation}: Vout = (1 + Rf/Rin)\timesVin
\item
  \textbf{Minimum gain}: Always \geq 1
\item
  \textbf{Applications}: Buffering, voltage amplification with high
  input impedance
\end{itemize}

\end{solutionbox}
\begin{mnemonicbox}
``Invert: Negative is Input, Non-invert: Positive
gets signal''

\end{mnemonicbox}
\subsection*{Question 5(a) [3 marks]}\label{q5a}

\textbf{Give pin description of IC555.}

\begin{solutionbox}

{\def\LTcaptype{none} % do not increment counter
\begin{longtable}[]{@{}lll@{}}
\toprule\noalign{}
Pin Number & Pin Name & Description \\
\midrule\noalign{}
\endhead
\bottomrule\noalign{}
\endlastfoot
1 & Ground & Connected to circuit ground \\
2 & Trigger & Starts timing cycle when \textless{} 1/3 VCC \\
3 & Output & Provides output signal \\
4 & Reset & Terminates timing when LOW \\
5 & Control Voltage & Adjusts threshold voltage \\
6 & Threshold & Ends timing cycle when \textgreater{} 2/3 VCC \\
7 & Discharge & Connected to timing capacitor \\
8 & VCC & Positive supply voltage (5-15V) \\
\end{longtable}
}

\textbf{Diagram:}

\begin{verbatim}
    +{-{-}{-}{-}{-}{-}{-}{-}+}
  8 |        | 7
+{-{-}{-}+ VCC    | DISCHARGE +{-}{-}{-}+}
    |        |               |
  7 |        | 6             |
+{-{-}{-}+ DISCHARGE THRESHOLD +{-}{-}+}
    |        |               |
  6 |        | 5             |
+{-{-}{-}+ THRESHOLD CONTROL   +{-}{-}+}
    |        |               |
  5 |        | 4             |
+{-{-}{-}+ CONTROL  RESET     +{-}{-}{-}+}
    |        |               |
  4 |        | 3             |
+{-{-}{-}+ RESET   OUTPUT    +{-}{-}{-}{-}+}
    |        |               |
  3 |        | 2             |
+{-{-}{-}+ OUTPUT  TRIGGER   +{-}{-}{-}{-}+}
    |        |               |
  2 |        | 1             |
+{-{-}{-}+ TRIGGER GND       +{-}{-}{-}{-}+}
    |        |
    +{-{-}{-}{-}{-}{-}{-}{-}+}
\end{verbatim}

\begin{itemize}
\tightlist
\item
  \textbf{Input pins}: Trigger, Reset, Threshold, Control Voltage
\item
  \textbf{Output pins}: Output, Discharge
\item
  \textbf{Power pins}: VCC, Ground
\item
  \textbf{Internal structure}: Composed of comparators, flip-flop,
  discharge transistor
\end{itemize}

\end{solutionbox}
\begin{mnemonicbox}
``Ground Triggers Output Reset Control Threshold
Discharges Voltage''

\end{mnemonicbox}
\subsection*{Question 5(b) [4 marks]}\label{q5b}

\textbf{Explain op-amp as differentiator.}

\begin{solutionbox}

{\def\LTcaptype{none} % do not increment counter
\begin{longtable}[]{@{}
  >{\raggedright\arraybackslash}p{(\linewidth - 4\tabcolsep) * \real{0.3333}}
  >{\raggedright\arraybackslash}p{(\linewidth - 4\tabcolsep) * \real{0.3939}}
  >{\raggedright\arraybackslash}p{(\linewidth - 4\tabcolsep) * \real{0.2727}}@{}}
\toprule\noalign{}
\begin{minipage}[b]{\linewidth}\raggedright
Parameter
\end{minipage} & \begin{minipage}[b]{\linewidth}\raggedright
Description
\end{minipage} & \begin{minipage}[b]{\linewidth}\raggedright
Formula
\end{minipage} \\
\midrule\noalign{}
\endhead
\bottomrule\noalign{}
\endlastfoot
Circuit & Op-amp with capacitor in input & Vo = -RC(dVi/dt) \\
Transfer Function & Output proportional to rate of change & H(s) =
-sRC \\
Frequency Response & Acts as high-pass filter & Gain increases with
frequency \\
Phase Shift & +90^\circ & - \\
\end{longtable}
}

\textbf{Diagram:}

\begin{verbatim}
                 R
            +{-{-}{-}{-}www{-}{-}{-}{-}+}
            |           |
            |    +{-{-}{-}{-}{-}+|}
            |    |     ||
Vin o{-{-}{-}{-}{-}{-}{-}|{-}{-}{-}{-}+  {-}  ||}
            ||   |     ||
            ||   |     +{-{-}{-}{-}o Vout}
            |C   |  +  |
            ||   |     |
            |    +{-{-}{-}{-}{-}+}
            |      |
           GND    GND
\end{verbatim}

\textbf{Input/Output Waveforms:}

\begin{verbatim}
  Triangle Input       Square Output
       /{                 \_}
      /  {               | |}
     /    {              | |}
\_\_\_\_/      {\_\_\_\_    \_\_\_\_\_| |\_\_\_\_\_}
               {         | |}
                {        | |}
                 {       | |}
                  {     \_| |\_}
\end{verbatim}

\begin{itemize}
\tightlist
\item
  \textbf{Working principle}: Output voltage proportional to rate of
  change of input
\item
  \textbf{Mathematical basis}: Vo = -RC(dVin/dt)
\item
  \textbf{Practical limitations}: Sensitive to high-frequency noise
\item
  \textbf{Applications}: Waveform generation, edge detection,
  rate-of-change indicator
\end{itemize}

\end{solutionbox}
\begin{mnemonicbox}
``Differentiator Delivers Derivatives - RC determines
speed''

\end{mnemonicbox}
\subsection*{Question 5(c) [7 marks]}\label{q5c}

\textbf{Explain IC 555 as astable and Monostable multivibrator.}

\begin{solutionbox}

{\def\LTcaptype{none} % do not increment counter
\begin{longtable}[]{@{}lll@{}}
\toprule\noalign{}
Parameter & Astable Multivibrator & Monostable Multivibrator \\
\midrule\noalign{}
\endhead
\bottomrule\noalign{}
\endlastfoot
Definition & Free-running oscillator & One-shot pulse generator \\
Stable States & None (continuously oscillates) & One stable state \\
Timing &

T = 0.693(RA+2RB)C &

T = 1.1RC \\

Trigger & Self-triggering & External trigger required \\
Output & Continuous square wave & Single pulse of fixed width \\
\end{longtable}
}

\textbf{Astable Circuit:}

\begin{verbatim}
        +Vcc
         |
         |
      +{-{-}+{-}{-}+}
      |     |
      R1    |
      |     |
      +{-{-}+{-}{-}+{-}{-}{-}{-}{-}{-}{-}{-}+}
      |  |           |
      |  +{-{-}+        |}
      |     |   8    |     7
      R2    +{-{-}{-}+{-}{-}{-}{-}{-}{-}{-}+{-}{-}{-}+}
      |         |       |   |
      +{-{-}{-}{-}{-}{-}+  | 555   |   |}
      |      |  |       |   |
      C1     |  |       |   |
      |      |  |       |   |
      +{-{-}{-}{-}{-}{-}+{-}{-}+{-}{-}{-}{-}{-}{-}{-}+{-}{-}{-}+}
             |  |       |   |
           2 |  |   3   |   |
      +{-{-}{-}{-}{-}{-}+{-}{-}+{-}{-}{-}{-}{-}{-}{-}+   |}
      |         |           |
      |         |           |
      +{-{-}{-}{-}{-}{-}{-}{-}{-}+{-}{-}{-}{-}{-}{-}{-}{-}{-}{-}{-}+}
                |
              Output
\end{verbatim}

\textbf{Monostable Circuit:}

\begin{verbatim}
     +Vcc
      |
      |
      R
      |
      +{-{-}{-}{-}{-}{-}{-}{-}+{-}{-}+}
      |        |  |
      |    8   |  |
      +{-{-}{-}{-}+{-}{-}{-}{-}{-}{-}{-}+{-}{-}{-}{-}+}
      |    |       |    |
      |    | 555   |    |
      |    |       |    |
      |    |       |    |
      |  4 |       | 7  |
      +{-{-}{-}{-}+{-}{-}{-}{-}{-}{-}{-}+{-}{-}{-}{-}+}
           |       |    |
         2 |       |    |
      +{-{-}{-}{-}+       |    |}
      |    |       |    |
      |  3 |       |    |
      +{-{-}{-}{-}+{-}{-}{-}{-}{-}{-}{-}+    |}
           |            |
         Output         |
           |            |
           +{-{-}{-}{-}{-}+{-}{-}{-}{-}{-}{-}+}
                 |
                 C
                 |
                GND
\end{verbatim}

\textbf{Astable Operation:}

\begin{itemize}
\tightlist
\item
  \textbf{Working}: Capacitor charges through RA+RB and discharges
  through RB
\item
  \textbf{Duty cycle}: Can be adjusted by proper selection of RA and RB
\item
  \textbf{Frequency}: f = 1.44/((RA+2RB)C)
\item
  \textbf{Applications}: LED flashers, tone generators, clock pulse
  generators
\end{itemize}

\textbf{Monostable Operation:}

\begin{itemize}
\tightlist
\item
  \textbf{Working}: Triggered by falling edge on pin 2, outputs HIGH for
  time T
\item
  \textbf{Time period}: T = 1.1RC
\item
  \textbf{Applications}: Time delays, pulse width modulation, debouncing
\end{itemize}

\end{solutionbox}
\begin{mnemonicbox}
``Astable Always Alternates, Monostable Makes One
pulse''

\end{mnemonicbox}
\subsection*{Question 5(a) OR [3
marks]}\label{q5a}

\textbf{Explain IC555 as Bistable multivibrator.}

\begin{solutionbox}

{\def\LTcaptype{none} % do not increment counter
\begin{longtable}[]{@{}ll@{}}
\toprule\noalign{}
Parameter & Description \\
\midrule\noalign{}
\endhead
\bottomrule\noalign{}
\endlastfoot
Definition & Flip-flop circuit with two stable states \\
Triggering & SET by trigger pin (2), RESET by reset pin (4) \\
Stable States & Two (HIGH or LOW) \\
Time Period & No timing components needed \\
\end{longtable}
}

\textbf{Diagram:}

\begin{verbatim}
           +Vcc
            |
            |
       +{-{-}{-}{-}|{-}{-}{-}{-}+}
       |    |    |
       |  8 |  4 |
   +{-{-}{-}+{-}{-}{-}{-}+{-}{-}{-}{-}+{-}{-}{-}+}
   |   |         |   |
   |   |   555   |   |
   |   |         |   |
   |   |         |   |
   | 2 |         | 3 |
   +{-{-}{-}+{-}{-}{-}{-}{-}{-}{-}{-}{-}+{-}{-}{-}+}
       |         |
    Trigger    Output
       |         |
      GND       GND
\end{verbatim}

\textbf{Truth Table:}

{\def\LTcaptype{none} % do not increment counter
\begin{longtable}[]{@{}lll@{}}
\toprule\noalign{}
Trigger (Pin 2) & Reset (Pin 4) & Output (Pin 3) \\
\midrule\noalign{}
\endhead
\bottomrule\noalign{}
\endlastfoot
\textless{} 1/3 VCC & HIGH & HIGH \\
\textgreater{} 1/3 VCC & HIGH & No change \\
Any & LOW & LOW \\
\end{longtable}
}

\begin{itemize}
\tightlist
\item
  \textbf{SET operation}: Occurs when trigger pin falls below 1/3 VCC
\item
  \textbf{RESET operation}: Occurs when reset pin is pulled LOW
\item
  \textbf{Applications}: Latching switches, memory elements, flip-flops
\item
  \textbf{Features}: No timing components (R, C) required
\end{itemize}

\end{solutionbox}
\begin{mnemonicbox}
``Bistable Bounces Between two states''

\end{mnemonicbox}
\subsection*{Question 5(b) OR [4
marks]}\label{q5b}

\textbf{Explain the basic operation of IC555 with internal block
diagram.}

\begin{solutionbox}

{\def\LTcaptype{none} % do not increment counter
\begin{longtable}[]{@{}ll@{}}
\toprule\noalign{}
Block & Function \\
\midrule\noalign{}
\endhead
\bottomrule\noalign{}
\endlastfoot
Comparators & Monitor trigger and threshold voltages \\
Flip-Flop & Controls output state \\
Discharge Transistor & Discharges timing capacitor \\
Voltage Divider & Establishes reference voltages \\
\end{longtable}
}

\textbf{Internal Block Diagram:}

\begin{verbatim}
                         +Vcc (8)
                            |
                            v
             +{-{-}{-}{-}{-}{-}{-}{-}{-}{-}{-}{-}{-}{-}{-}{-}{-}{-}{-}{-}{-}{-}{-}{-}{-}{-}{-}{-}{-}+}
             |        Voltage Divider      |
             |     +{-{-}{-}{-}{-}{-}{-}+{-}{-}{-}{-}{-}{-}{-}+       |}
             |     |       |       |       |
             |     R       R       R       |
             |     |       |       |       |
             |     +       +       +       |
             |     |       |       |       |
Control (5){-{-}+{-}{-}{-}{-}{-}+       |       |       |}
             |             |       |       |
Threshold(6){-+{-}{-}{-}{-}{-}|{-}{-}{-}{-}{-}{-}+       |       |}
             |      |+             |       |
             |    Comp2            |       |
             |      |              |       |
             |      v              |       |
             |    +{-+{-}+            |       |}
Trigger (2){-{-}+{-}{-}{-}|   |        S   |       |}
             |    |+  |{-{-}{-}{-}{-}{-}{-}{-}{-}{-}Q+{-}{-}{-}{-}{-}{-}{-}+{-}{-}Output (3)}
             |    Comp1         FF |       |
             |      |          R   |       |
             |      |          \^{   |       |}
Reset (4){-{-}{-}{-}+{-}{-}{-}{-}{-}{-}|{-}{-}{-}{-}{-}{-}{-}{-}{-}{-}|{-}{-}{-}+       |}
             |      |          |   |       |
             |      |          |   |       |
             |      v          |   |       |
             |    Transistor   |   |       |
             |      |          |   |       |
Discharge(7){-+{-}{-}{-}{-}{-}{-}+{-}{-}{-}{-}{-}{-}{-}{-}{-}{-}+{-}{-}{-}+       |}
             |                             |
             |                             |
             |                             |
             +{-{-}{-}{-}{-}{-}{-}{-}{-}{-}{-}{-}{-}{-}{-}{-}{-}{-}{-}{-}{-}{-}{-}{-}{-}{-}{-}{-}{-}+}
                            |
                           GND (1)
\end{verbatim}

\textbf{Basic Operation:}

\begin{enumerate}
\tightlist
\item
  \textbf{Voltage divider}: Creates 2/3 VCC and 1/3 VCC reference points
\item
  \textbf{Comparator 1}: Triggers when pin 2 goes below 1/3 VCC
\item
  \textbf{Comparator 2}: Resets when pin 6 goes above 2/3 VCC
\item
  \textbf{Flip-flop}: Controls output state based on comparator inputs
\item
  \textbf{Discharge transistor}: Connects pin 7 to ground when output is
  LOW
\end{enumerate}

\begin{itemize}
\tightlist
\item
  \textbf{Versatility}: Can be configured in multiple modes (astable,
  monostable, bistable)
\item
  \textbf{Timing precision}: Determined by external RC components
\item
  \textbf{Wide supply range}: Functions from 4.5V to 16V
\end{itemize}

\end{solutionbox}
\begin{mnemonicbox}
``Comparators Control Flip-flop For Timing''

\end{mnemonicbox}
\subsection*{Question 5(c) OR [7
marks]}\label{q5c}

\textbf{Explain how class A, Class B, Class C and Class AB Power
amplifier are classified based on their Q Point location on load line,
with diagram.}

\begin{solutionbox}

{\def\LTcaptype{none} % do not increment counter
\begin{longtable}[]{@{}llll@{}}
\toprule\noalign{}
Amplifier Class & Q-Point Location & Conduction Angle & Efficiency \\
\midrule\noalign{}
\endhead
\bottomrule\noalign{}
\endlastfoot
Class A & Center of load line & 360^\circ & 25-30\% \\
Class B & Cut-off point & 180^\circ & 78.5\% \\
Class AB & Slightly above cut-off & 180^\circ-360^\circ & 50-78.5\% \\
Class C & Below cut-off & \textless180^\circ & \textgreater80\% \\
\end{longtable}
}

\textbf{Diagram Load Line:}

\begin{verbatim}
     Ic
      \^{}
      |
      |               Load Line
      |              /
      |             /
      |            /
  IcQ1|       A   /
      |        *  /
      |         {/}
  IcQ2|     AB * /
      |         {/}
      |      B * /
  IcQ3|         {/}
      |         /
      |    C * /
      |       /
      |      /
      |     /
      |    /
      |   /
      |  /
      | /
      |/
      +{-{-}{-}{-}{-}{-}{-}{-}{-}{-}{-}{-}{-}{-}{-}{-}{-}{-}{-}{-}{-}{-}{-}{-}{-}{-}{-}{-}{-}{-}{-}{-} Vce}
          VceC   VceB   VceAB   VceA
\end{verbatim}

\textbf{Input/Output Waveforms:}

\begin{verbatim}
   Input          Class A         Class B          Class AB         Class C
     \^{              \^{}               \^{}                \^{}                \^{}}
     |              |               |                |                |
     |  /{          |  /           |  /            |  /            |  /}
     | /  {         | /            | /             | /             | /  }
{-{-}{-}{-}{-}+{-}{-}{-}{-}{-}{-}       {-}+{-}{-}{-}{-}{-}{-}        {-}+{-}{-}{-}{-}{-}{-}         {-}+{-}{-}{-}{-}{-}{-}         {-}+{-}{-}{-}{-}{-}{-}}
     |    {         |              |                |               |}
     |     {        |              |                |               |}
     |      {       |              |               |               |}
     v       v      v       v       v       v        v       v        v       v
                                    Distortion       Small Distortion
\end{verbatim}

\textbf{Class A Characteristics:}

\begin{itemize}
\tightlist
\item
  \textbf{Q-point}: Center of load line
\item
  \textbf{Bias}: Fixed bias maintains conduction for entire cycle
\item
  \textbf{Linearity}: Excellent linearity, minimal distortion
\item
  \textbf{Efficiency}: Poor (25-30\%)
\end{itemize}

\textbf{Class B Characteristics:}

\begin{itemize}
\tightlist
\item
  \textbf{Q-point}: At cutoff point
\item
  \textbf{Bias}: Biased at cutoff, each device conducts for half-cycle
\item
  \textbf{Distortion}: Crossover distortion at zero-crossing
\item
  \textbf{Efficiency}: Good (78.5\% theoretical)
\end{itemize}

\textbf{Class AB Characteristics:}

\begin{itemize}
\tightlist
\item
  \textbf{Q-point}: Slightly above cutoff
\item
  \textbf{Bias}: Small bias current eliminates crossover distortion
\item
  \textbf{Linearity}: Good compromise between A and B
\item
  \textbf{Efficiency}: Moderate (50-78.5\%)
\end{itemize}

\textbf{Class C Characteristics:}

\begin{itemize}
\tightlist
\item
  \textbf{Q-point}: Below cutoff
\item
  \textbf{Bias}: Conducts for less than half-cycle
\item
  \textbf{Distortion}: Severe distortion, requires tuned circuit
\item
  \textbf{Efficiency}: Excellent (\textgreater80\%)
\end{itemize}

\end{solutionbox}
\begin{mnemonicbox}
``Above center, Below center, Cut-off point, Down
below - ABCD order for Q-point location''

\end{mnemonicbox}

\end{document}
