\documentclass[10pt,a4paper]{article}

% content/resources/templates/preamble.tex
\usepackage[margin=0.6in]{geometry}
\author{Milav Dabgar}
\usepackage{amsmath,amssymb,amsthm}
\usepackage{booktabs}
\usepackage{multirow}
\usepackage{xcolor}
\usepackage{tcolorbox}
\tcbuselibrary{breakable,skins}
\usepackage[colorlinks=true,linkcolor=blue]{hyperref}
\usepackage{titlesec}
\usepackage{enumitem}
\usepackage{tikz}
\usepackage{pgfplots}
\usepackage{circuitikz}
\usepackage[version=4]{mhchem}
\usepackage{longtable}
\usepackage{array}
\usepackage{float}
\usepackage{caption}
\usepackage{listings}

\lstset{
  basicstyle=\small\ttfamily,
  breaklines=true,
  breakatwhitespace=false,
  postbreak=\mbox{\textcolor{red}{$\hookrightarrow$}\space},
  float=false,
  numbers=left,
  numberstyle=\tiny\color{gray},
  numbersep=10pt,
  xleftmargin=2em,
  keywordstyle=\color{blue},
  commentstyle=\color{green!60!black},
  stringstyle=\color{purple},
  backgroundcolor=\color{gray!5},
  showstringspaces=false,
  tabsize=2,
  captionpos=b,
  keepspaces=true,
  columns=flexible
}

\pgfplotsset{compat=1.18}
\usetikzlibrary{shapes,arrows,positioning,calc,patterns,decorations.pathmorphing,decorations.markings,arrows.meta}

% Color scheme
\definecolor{headcolor}{RGB}{0,102,204}
\definecolor{keycolor}{RGB}{220,20,60}
\definecolor{solutioncolor}{RGB}{34,139,34}
\definecolor{mnemoniccolor}{RGB}{148,0,211}
\definecolor{codecolor}{RGB}{0,0,100}

% Spacing
\setlength{\parskip}{3pt}
\setlist[itemize]{nosep}
\setlist[enumerate]{nosep}

% Title formatting
\titleformat{\section}{\Large\bfseries\color{headcolor}}{\thesection}{1em}{}
\titleformat{\subsection}{\large\bfseries\color{headcolor}}{\thesubsection}{1em}{}

% Pandoc tightlist compatibility
\providecommand{\tightlist}{%
  \setlength{\itemsep}{0pt}\setlength{\parskip}{0pt}}

% Pandoc longtable compatibility
\newcounter{none}
\def\thenone{}


% content/resources/templates/gujarati-boxes.tex
\usepackage{fontspec}
\usepackage{polyglossia}

% Set Gujarati as main language (document is primarily in Gujarati)
% Note: gloss-gujarati.ldf doesn't exist in polyglossia, but it will use hyphenation patterns
\setdefaultlanguage{gujarati}
\setotherlanguage{english}

% Configure Gujarati font properly
% Use Language=Default to prevent polyglossia from trying to add language-specific features
% that don't exist for Gujarati, which causes "empty feature" warnings
\newfontfamily\gujaratifont[Script=Gujarati,AutoFakeBold=2.5,AutoFakeSlant=0.3]{Noto Sans Gujarati}
\setmainfont[Script=Gujarati,AutoFakeBold=2.5,AutoFakeSlant=0.3]{Noto Sans Gujarati}
% Use Noto Sans Gujarati for monospace to support Gujarati in text
\setmonofont[Scale=0.9]{Noto Sans Gujarati}

% Configure English to use the same font
\newfontfamily\englishfont[Script=Gujarati,AutoFakeBold=2.5,AutoFakeSlant=0.3]{Noto Sans Gujarati}

% Translations for polyglossia
\gappto\captionsgujarati{
  \renewcommand{\tablename}{કોષ્ટક}
  \renewcommand{\figurename}{આકૃતિ}
}

% Helper for TikZ nodes to ensure Gujarati font
\newcommand{\gu}[1]{{\gujaratifont #1}}

% Custom environments
\newtcolorbox{solutionbox}{
    breakable,
    enhanced,
    colback=solutioncolor!5!white,
    colframe=solutioncolor!75!black,
    fonttitle=\bfseries,
    title=જવાબ
}

\newtcolorbox{solutionboxnobreak}{
 colback=solutioncolor!5!white,
 colframe=solutioncolor!75!black,
 fonttitle=\bfseries,
 title=જવાબ
}

\newtcolorbox{keyformula}{
 breakable,
 enhanced,
 colback=keycolor!5!white,
 colframe=keycolor!75!black,
 fonttitle=\bfseries,
 title=રાસાયણિક સમીકરણ/સૂત્ર
}

\newtcolorbox{mnemonicbox}{
 breakable,
 enhanced,
 colback=mnemoniccolor!5!white,
 colframe=mnemoniccolor!75!black,
 fonttitle=\bfseries,
 title=મેમરી ટ્રીક
}


\begin{document}

\begin{center}
{\Huge\bfseries\color{headcolor} Subject Name (Gujarati)}\\[5pt]
{\LARGE 4341107 -- Winter 2023}\\[3pt]
{\large Semester 1 Study Material}\\[3pt]
{\normalsize\textit{Detailed Solutions and Explanations}}
\end{center}

\vspace{10pt}

\subsection*{પ્રશ્ન 1(અ) [3
marks]}\label{uxaaauxab0uxab6uxaa8-1uxa85-3-marks}

\textbf{મેઇંટેનન્સ ના ભિન્ન પ્રકારો ટૂંકમા સમજાવો.}

\begin{solutionbox}

{\def\LTcaptype{none} % do not increment counter
\begin{longtable}[]{@{}
  >{\raggedright\arraybackslash}p{(\linewidth - 2\tabcolsep) * \real{0.5000}}
  >{\raggedright\arraybackslash}p{(\linewidth - 2\tabcolsep) * \real{0.5000}}@{}}
\toprule\noalign{}
\begin{minipage}[b]{\linewidth}\raggedright
મેઇંટેનન્સનો પ્રકાર
\end{minipage} & \begin{minipage}[b]{\linewidth}\raggedright
વિગત
\end{minipage} \\
\midrule\noalign{}
\endhead
\bottomrule\noalign{}
\endlastfoot
\textbf{પ્રિવેન્ટિવ મેઇંટેનન્સ} & નિયમિત ચકાસણી અને સર્વિસિંગ દ્વારા બ્રેકડાઉન
અટકાવવા \\
\textbf{કરેક્ટિવ મેઇંટેનન્સ} & ઉપકરણ ખરાબ થયા પછી કામગીરી પુનઃસ્થાપિત કરવા \\
\textbf{પ્રિડિક્ટિવ મેઇંટેનન્સ} & સ્થિતિ મોનિટરિંગનો ઉપયોગ કરીને મેઇંટેનન્સની જરૂર પડશે
તે અગાઉથી નક્કી કરવું \\
\end{longtable}
}

\end{solutionbox}
\begin{mnemonicbox}
``PCPro'' - પ્રિવેન્ટિવ પ્રતિબંધિત કરે છે, કરેક્ટિવ સુધારે છે,
પ્રિડિક્ટિવ આગાહી કરે છે

\end{mnemonicbox}
\subsection*{પ્રશ્ન 1(બ) [4
marks]}\label{uxaaauxab0uxab6uxaa8-1uxaac-4-marks}

\textbf{વોશિંગ મશીનના મેઇંટેનન્સની પ્રક્રિયા સમજાવો.}

\begin{solutionbox}

\textbf{વોશિંગ મશીનની મેઇંટેનન્સ પ્રક્રિયા:}

\begin{verbatim}
flowchart LR
    A[Regular Inspection] {-{-} B[Clean Filter]}
    B {-{-} C[Check Hoses]}
    C {-{-} D[Balance Load]}
    D {-{-} E[Clean Drum]}
\end{verbatim}

\begin{itemize}
\tightlist
\item
  \textbf{ફિલ્ટર સફાઈ}: દર મહિને લિન્ટ ફિલ્ટર કાઢીને સાફ કરવું
\item
  \textbf{હોસ નિરીક્ષણ}: દર 3 મહિને તિરાડો અને લીકેજ માટે તપાસ કરવી
\item
  \textbf{લોડ વિતરણ}: કંપન અટકાવવા માટે યોગ્ય સંતુલન સુનિશ્ચિત કરવું
\item
  \textbf{ડ્રમ સફાઈ}: ત્રિમાસિક ખાલી ગરમ પાણીના ચક્ર સાથે વિનેગર ચલાવવું
\end{itemize}

\end{solutionbox}
\begin{mnemonicbox}
``FHLD'' - ફિલ્ટર્સ, હોસેસ, લોડ્સ, ડ્રમને નિયમિત ધ્યાન
આપવાની જરૂર છે

\end{mnemonicbox}
\subsection*{પ્રશ્ન 1(ક) [7
marks]}\label{uxaaauxab0uxab6uxaa8-1uxa95-7-marks}

\textbf{માઇક્રોવેવ ઓવન ના મેઇંટેનન્સ અને ટ્રબલશૂટિંગની પ્રક્રિયા સમજાવો.}

\begin{solutionbox}

\textbf{માઇક્રોવેવ ઓવનનું મેઇંટેનન્સ અને ટ્રબલશૂટિંગ:}

{\def\LTcaptype{none} % do not increment counter
\begin{longtable}[]{@{}lll@{}}
\toprule\noalign{}
મેઇંટેનન્સ કાર્ય & પ્રક્રિયા & આવર્તન \\
\midrule\noalign{}
\endhead
\bottomrule\noalign{}
\endlastfoot
બાહ્ય સફાઈ & હળવા ડિટર્જન્ટથી સાફ કરવું & સાપ્તાહિક \\
આંતરિક સફાઈ & ખોરાકના કણો અને ગ્રીસ સાફ કરવા & દરેક છલકાય પછી \\
દરવાજાની સીલ ચેક & નુકસાન અથવા લીકેજ માટે તપાસ & માસિક \\
વેન્ટિલેશન ચેક & વેન્ટ્સ અવરોધિત ન હોય તે સુનિશ્ચિત કરવું & માસિક \\
\end{longtable}
}

\textbf{ટ્રબલશૂટિંગ પ્રક્રિયા:}

\begin{verbatim}
flowchart TD
    A[No Power] {-{-}|Check| B[Power Connection]}
    C[Not Heating] {-{-}|Check| D[Door Switch \& Magnetron]}
    E[Uneven Cooking] {-{-}|Check| F[Turntable Mechanism]}
    G[Sparking] {-{-}|Check| H[Metal Objects/Damaged Cavity]}
    I[Unusual Noise] {-{-}|Check| J[Fan \& Turntable Motor]}
\end{verbatim}

\begin{itemize}
\tightlist
\item
  \textbf{પાવર સમસ્યાઓ}: ફ્યુઝ, સર્કિટ બ્રેકર, અને કોર્ડ ચેક કરો
\item
  \textbf{હીટિંગ સમસ્યાઓ}: દરવાજા સ્વિચ, હાઈ વોલ્ટેજ કેપેસિટર, મેગ્નેટ્રોન ટેસ્ટ કરો
\item
  \textbf{સલામતી પ્રથમ}: ક્યારેય ડેમેજ્ડ દરવાજા અથવા સીલ સાથે ઓપરેટ ન કરો
\end{itemize}

\end{solutionbox}
\begin{mnemonicbox}
``POWER'' - પાવર, ઓવન ઇન્ટીરિયર, વાયરિંગ, ઇલેક્ટ્રોનિક્સ,
રેડિએશન સીલ્સ

\end{mnemonicbox}
\subsection*{પ્રશ્ન 1(ક OR) [7
marks]}\label{uxaaauxab0uxab6uxaa8-1uxa95-or-7-marks}

\textbf{પ્રોજેક્ટર ના મેઇંટેનન્સ અનેટ્રબલશૂટિંગની પ્રક્રિયા સમજાવો.}

\begin{solutionbox}

\textbf{પ્રોજેક્ટરનું મેઇંટેનન્સ અને ટ્રબલશૂટિંગ:}

{\def\LTcaptype{none} % do not increment counter
\begin{longtable}[]{@{}lll@{}}
\toprule\noalign{}
મેઇંટેનન્સ કાર્ય & પ્રક્રિયા & આવર્તન \\
\midrule\noalign{}
\endhead
\bottomrule\noalign{}
\endlastfoot
લેન્સ સફાઈ & લેન્સ ક્લોથ અને સોલ્યુશન વાપરવું & માસિક \\
ફિલ્ટર સફાઈ & કાઢીને ધૂળ સાફ કરવી & દર 100 કલાકે \\
લેમ્પ ઇન્સ્પેક્શન & ડિસ્કલરેશન/ડિમિંગ માટે તપાસ & દર 300 કલાકે \\
વેન્ટિલેશન & યોગ્ય એરફ્લો સુનિશ્ચિત કરવું & દરેક ઉપયોગ પહેલાં \\
\end{longtable}
}

\textbf{ટ્રબલશૂટિંગ પ્રક્રિયા:}

\begin{verbatim}
flowchart TD
    A[No Power] {-{-}|Check| B[Power Supply \& Cable]}
    C[No Image] {-{-}|Check| D[Source Connection \& Input Selection]}
    E[Poor Image] {-{-}|Check| F[Focus \& Lens]}
    G[Overheating] {-{-}|Check| H[Ventilation \& Filters]}
    I[Lamp Failure] {-{-}|Check| J[Lamp Life \& Replacement]}
\end{verbatim}

\begin{itemize}
\tightlist
\item
  \textbf{ઇમેજ સમસ્યાઓ}: ફોકસ, રેઝોલ્યુશન, કીસ્ટોન કરેક્શન એડજસ્ટ કરવું
\item
  \textbf{લેમ્પ સમસ્યાઓ}: લેમ્પ કલાકો ચેક કરો, મર્યાદા વટાવી જાય તો બદલો
\item
  \textbf{કનેક્ટિવિટી}: ઇનપુટ સોર્સ અને કેબલ કનેક્શનો ચકાસો
\item
  \textbf{થર્મલ સમસ્યાઓ}: ફિલ્ટર્સ સાફ કરો અને યોગ્ય વેન્ટિલેશન સુનિશ્ચિત કરો
\end{itemize}

\end{solutionbox}
\begin{mnemonicbox}
``FLAMVE'' - ફિલ્ટર્સ, લેમ્પ, એરફ્લો, માઉન્ટિંગ, વોલ્ટેજ,
એન્વાયરમેન્ટ

\end{mnemonicbox}
\subsection*{પ્રશ્ન 2(અ) [3
marks]}\label{uxaaauxab0uxab6uxaa8-2uxa85-3-marks}

\textbf{પદો ટૂંક મા સમજાવો (1) હ્યુ (2) બ્રાઈટનેસ}

\begin{solutionbox}

{\def\LTcaptype{none} % do not increment counter
\begin{longtable}[]{@{}
  >{\raggedright\arraybackslash}p{(\linewidth - 2\tabcolsep) * \real{0.5000}}
  >{\raggedright\arraybackslash}p{(\linewidth - 2\tabcolsep) * \real{0.5000}}@{}}
\toprule\noalign{}
\begin{minipage}[b]{\linewidth}\raggedright
પદ
\end{minipage} & \begin{minipage}[b]{\linewidth}\raggedright
વિગત
\end{minipage} \\
\midrule\noalign{}
\endhead
\bottomrule\noalign{}
\endlastfoot
\textbf{હ્યુ} & શુદ્ધ રંગ લક્ષણ જે રંગોને અલગ પાડે છે (લાલ, લીલો, વાદળી, વગેરે) પ્રકાશ
તરંગલંબાઈના આધારે \\
\textbf{બ્રાઈટનેસ} & રંગમાંથી ઉત્સર્જિત અથવા પરાવર્તિત પ્રકાશની માત્રા, જે નક્કી કરે
છે કે તે કેટલો પ્રકાશિત અથવા અંધકારમય દેખાય છે \\
\end{longtable}
}

\textbf{ડાયાગ્રામ:}

\begin{verbatim}
             Hue
          (Color Type)
              ↑
              |
Saturation {-{-}+{-}{-} Brightness}
(Intensity)   |   (Lightness)
              ↓
            Value
\end{verbatim}

\end{solutionbox}
\begin{mnemonicbox}
``HB-WC'' - હ્યુ નક્કી કરે છે કયો રંગ, બ્રાઈટનેસ નક્કી કરે છે
સફેદથી કાળા સ્તર

\end{mnemonicbox}
\subsection*{પ્રશ્ન 2(બ) [4
marks]}\label{uxaaauxab0uxab6uxaa8-2uxaac-4-marks}

\textbf{એલસીડી ટીવી પર ટૂંકનોંધ લખો}

\begin{solutionbox}

\textbf{એલસીડી ટીવી ટેકનોલોજી:}

\begin{verbatim}
flowchart LR
    A[Backlight] {-{-} B[Polarizing Filter]}
    B {-{-} C[Liquid Crystal Layer]}
    C {-{-} D[Color Filter]}
    D {-{-} E[Screen]}
\end{verbatim}

\begin{itemize}
\tightlist
\item
  \textbf{કાર્ય સિદ્ધાંત}: લિક્વિડ ક્રિસ્ટલનો ઉપયોગ કરે છે જે પ્રકાશને પાસ/બ્લોક
  કરવા માટે ટ્વિસ્ટ/અનટ્વિસ્ટ થાય છે
\item
  \textbf{મુખ્ય ઘટકો}: બેકલાઇટ, પોલરાઇઝિંગ ફિલ્ટર્સ, લિક્વિડ ક્રિસ્ટલ મેટ્રિક્સ, કલર
  ફિલ્ટર્સ
\item
  \textbf{ફાયદાઓ}: પાતળી પ્રોફાઇલ, ઊર્જા કાર્યક્ષમ, કોઈ રેડિએશન નહીં, તીક્ષ્ણ
  છબી
\item
  \textbf{મર્યાદાઓ}: મર્યાદિત વ્યૂઇંગ એંગલ, નવી ટેક્નોલોજી કરતાં ધીમો રિસ્પોન્સ
  ટાઇમ
\end{itemize}

\end{solutionbox}
\begin{mnemonicbox}
``BPLCS'' - બેકલાઇટ પાસ લાઇટ થ્રુ ક્રિસ્ટલ્સ ટુ સ્ક્રીન

\end{mnemonicbox}
\subsection*{પ્રશ્ન 2(ક) [7
marks]}\label{uxaaauxab0uxab6uxaa8-2uxa95-7-marks}

\textbf{ડીટીએચ રિસિવર નો બ્લોક ડાયેગ્રામ દોરો અને સમજાવો.}

\begin{solutionbox}

\textbf{DTH રિસીવર બ્લોક ડાયાગ્રામ:}

\begin{verbatim}
flowchart LR
    A[Satellite Dish] {-{-} B[LNB]}
    B {-{-} C[Tuner]}
    C {-{-} D[Demodulator]}
    D {-{-} E[MPEG Decoder]}
    E {-{-} F[Video/Audio Processor]}
    F {-{-} G[TV Display]}
    H[Smart Card] {-{-} I[Conditional Access Module]}
    I {-{-} D}
    J[User Interface] {-{-} K[Microcontroller]}
    K {-{-} C}
    K {-{-} E}
\end{verbatim}

\begin{itemize}
\tightlist
\item
  \textbf{સેટેલાઇટ ડિશ}: સેટેલાઇટથી સિગ્નલ્સ કેપ્ચર કરે છે
\item
  \textbf{LNB (લો નોઇઝ બ્લોક)}: ઉચ્ચ ફ્રિક્વન્સી સિગ્નલ્સને નીચી ફ્રિક્વન્સીમાં
  કન્વર્ટ કરે છે
\item
  \textbf{ટ્યુનર}: ચોક્કસ ચેનલ ફ્રિક્વન્સી પસંદ કરે છે
\item
  \textbf{ડિમોડ્યુલેટર}: કેરિયર સિગ્નલમાંથી ડિજિટલ માહિતી એક્સટ્રેક્ટ કરે છે
\item
  \textbf{MPEG ડિકોડર}: વિડિયો/ઓડિયો ડેટા ડિકમ્પ્રેસ કરે છે
\item
  \textbf{કન્ડિશનલ એક્સેસ મોડ્યુલ}: સબ્સ્ક્રિપ્શન ઍક્સેસ નિયંત્રિત કરે છે
\item
  \textbf{માઇક્રોકન્ટ્રોલર}: સમગ્ર ઓપરેશન અને યુઝર ઇનપુટ્સ નિયંત્રિત કરે છે
\end{itemize}

\end{solutionbox}
\begin{mnemonicbox}
``SLTDMP'' - સેટેલાઇટ, LNB, ટ્યુનર, ડિમોડ્યુલેટર, MPEG,
પ્રોસેસર

\end{mnemonicbox}
\subsection*{પ્રશ્ન 2(અ OR) [3
marks]}\label{uxaaauxab0uxab6uxaa8-2uxa85-or-3-marks}

\textbf{પદો ટૂંક મા સમજાવો (1) લ્યુમિનેન્સ (2) ક્રોમિનેન્સ}

\begin{solutionbox}

{\def\LTcaptype{none} % do not increment counter
\begin{longtable}[]{@{}
  >{\raggedright\arraybackslash}p{(\linewidth - 2\tabcolsep) * \real{0.5000}}
  >{\raggedright\arraybackslash}p{(\linewidth - 2\tabcolsep) * \real{0.5000}}@{}}
\toprule\noalign{}
\begin{minipage}[b]{\linewidth}\raggedright
પદ
\end{minipage} & \begin{minipage}[b]{\linewidth}\raggedright
વિગત
\end{minipage} \\
\midrule\noalign{}
\endhead
\bottomrule\noalign{}
\endlastfoot
\textbf{લ્યુમિનેન્સ} & વિડિયો સિગ્નલનો બ્રાઇટનેસ અથવા તીવ્રતા ઘટક (Y) જે બ્લેક અને
વ્હાઇટ માહિતી લઈ જાય છે \\
\textbf{ક્રોમિનેન્સ} & વિડિયો સિગ્નલનો રંગ ઘટક (Cb, Cr) જે હ્યુ અને સેચુરેશન માહિતી
લઈ જાય છે \\
\end{longtable}
}

\textbf{ડાયાગ્રામ:}

\begin{verbatim}
Video Signal
    |
    +{-{-}{-}{-}{-}{-}{-}{-}{-}{-}+{-}{-}{-}{-}{-}{-}{-}{-}{-}{-}+}
    |                     |
Luminance (Y)      Chrominance (C)
(Brightness)         /         {}
                    /           {}
                   /             {}
            Blue{-Y (Cb)       Red{-}Y (Cr)}
            (Blue diff)       (Red diff)
\end{verbatim}

\end{solutionbox}
\begin{mnemonicbox}
``LC-BH'' - લ્યુમિનેન્સ બ્રાઇટનેસ નિયંત્રિત કરે છે, ક્રોમિનેન્સ
હ્યુ નિયંત્રિત કરે છે

\end{mnemonicbox}
\subsection*{પ્રશ્ન 2(બ OR) [4
marks]}\label{uxaaauxab0uxab6uxaa8-2uxaac-or-4-marks}

\textbf{ગ્રાસમેનનો નિયમ સમજાવો.}

\begin{solutionbox}

\textbf{ગ્રાસમેનના રંગ મિશ્રણના નિયમો:}

{\def\LTcaptype{none} % do not increment counter
\begin{longtable}[]{@{}
  >{\raggedright\arraybackslash}p{(\linewidth - 2\tabcolsep) * \real{0.5000}}
  >{\raggedright\arraybackslash}p{(\linewidth - 2\tabcolsep) * \real{0.5000}}@{}}
\toprule\noalign{}
\begin{minipage}[b]{\linewidth}\raggedright
નિયમ
\end{minipage} & \begin{minipage}[b]{\linewidth}\raggedright
વિગત
\end{minipage} \\
\midrule\noalign{}
\endhead
\bottomrule\noalign{}
\endlastfoot
\textbf{સિમેટ્રી} & જો રંગ A, રંગ B સાથે મેળ ખાય છે, તો B, A સાથે મેળ ખાય છે \\
\textbf{પ્રોપોર્શનલિટી} & જો A, B સાથે મેળ ખાય છે, તો nA, nB સાથે મેળ ખાય છે
(કોઈપણ તીવ્રતા n માટે) \\
\textbf{એડિટિવિટી} & જો A, B સાથે મેળ ખાય છે અને C, D સાથે મેળ ખાય છે, તો A+C,
B+D સાથે મેળ ખાય છે \\
\end{longtable}
}

\begin{itemize}
\tightlist
\item
  \textbf{એપ્લિકેશન}: ડિસ્પ્લેમાં RGB રંગ મોડેલનો આધાર બને છે
\item
  \textbf{મહત્વ}: ત્રણ પ્રાથમિક રંગોને મિશ્રિત કરીને કોઈપણ રંગ બનાવવાની મંજૂરી આપે
  છે
\item
  \textbf{મર્યાદા}: માત્ર પ્રકાશ (એડિટિવ મિક્સિંગ) માટે લાગુ પડે છે, પિગમેન્ટ્સ માટે
  નહીં
\end{itemize}

\end{solutionbox}
\begin{mnemonicbox}
``SPA Color'' - સિમેટ્રી, પ્રોપોર્શનલિટી, એડિટિવિટી રંગ
મેચિંગ માટેના નિયમો

\end{mnemonicbox}
\subsection*{પ્રશ્ન 2(ક OR) [7
marks]}\label{uxaaauxab0uxab6uxaa8-2uxa95-or-7-marks}

\textbf{કલર ટીવી રિસિવર નો બ્લોક ડાયેગ્રામ દોરો અને સમજાવો.}

\begin{solutionbox}

\textbf{કલર ટીવી રિસીવર બ્લોક ડાયાગ્રામ:}

\begin{verbatim}
flowchart LR
    A[Antenna] {-{-} B[Tuner]}
    B {-{-} C[IF Amplifier]}
    C {-{-} D[Video Detector]}
    D {-{-} E[Video Amplifier]}
    E {-{-} F[Color Processor]}
    F {-{-} G[RGB Matrix]}
    G {-{-} H[Picture Tube/Display]}
    D {-{-} I[Sound IF]}
    I {-{-} J[Sound Demodulator]}
    J {-{-} K[Audio Amplifier]}
    K {-{-} L[Speaker]}
    M[Sync Separator] {-{-} N[Deflection Circuits]}
    N {-{-} H}
    D {-{-} M}
\end{verbatim}

\begin{itemize}
\tightlist
\item
  \textbf{ટ્યુનર}: ઇચ્છિત ચેનલ ફ્રિક્વન્સી પસંદ કરે છે
\item
  \textbf{IF એમ્પ્લિફાયર}: ઇન્ટરમીડિયેટ ફ્રિક્વન્સી સિગ્નલ્સને એમ્પ્લિફાય કરે છે
\item
  \textbf{વિડિયો ડિટેક્ટર}: વિડિયો અને ઓડિયો માહિતી એક્સટ્રેક્ટ કરે છે
\item
  \textbf{કલર પ્રોસેસર}: લ્યુમિનન્સ અને ક્રોમિનન્સને અલગ કરે છે
\item
  \textbf{RGB મેટ્રિક્સ}: કલર સિગ્નલ્સને રેડ, ગ્રીન, બ્લુમાં કન્વર્ટ કરે છે
\item
  \textbf{સિન્ક સેપરેટર}: હોરિઝોન્ટલ અને વર્ટિકલ સિન્ક એક્સટ્રેક્ટ કરે છે
\item
  \textbf{ડિફ્લેક્શન સર્કિટ્સ}: ઇલેક્ટ્રોન બીમ સ્કેનિંગ નિયંત્રિત કરે છે
\end{itemize}

\end{solutionbox}
\begin{mnemonicbox}
``TIVCRDS'' - ટ્યુનર, IF, વિડિયો, કલર, RGB, ડિફ્લેક્શન,
સ્પીકર

\end{mnemonicbox}
\subsection*{પ્રશ્ન 3(અ) [3
marks]}\label{uxaaauxab0uxab6uxaa8-3uxa85-3-marks}

\textbf{સોલર પાવર સિસ્ટમના મેઇન કોમ્પોનન્ટો અને સોલર પાવર સિસ્ટમના સ્પેસિફિકેશનો
લખો.}

\begin{solutionbox}

\textbf{સોલર પાવર સિસ્ટમના મુખ્ય ઘટકો:}

{\def\LTcaptype{none} % do not increment counter
\begin{longtable}[]{@{}ll@{}}
\toprule\noalign{}
ઘટક & કાર્ય \\
\midrule\noalign{}
\endhead
\bottomrule\noalign{}
\endlastfoot
\textbf{સોલર પેનલ્સ} & સૂર્યપ્રકાશને વીજળીમાં રૂપાંતરિત કરે છે \\
\textbf{ચાર્જ કન્ટ્રોલર} & બેટરી ચાર્જિંગ નિયંત્રિત કરે છે \\
\textbf{બેટરી બેંક} & વીજ ઊર્જા સંગ્રહિત કરે છે \\
\textbf{ઈન્વર્ટર} & DCને AC વીજળીમાં રૂપાંતરિત કરે છે \\
\textbf{માઉન્ટિંગ સ્ટ્રક્ચર} & પેનલને ટેકો આપે છે અને પોઝિશન આપે છે \\
\end{longtable}
}

\textbf{સ્પેસિફિકેશનો:}

\begin{itemize}
\tightlist
\item
  \textbf{પેનલ રેટિંગ}: 100-400W પ્રતિ પેનલ
\item
  \textbf{બેટરી કેપેસિટી}: 100-200Ah
\item
  \textbf{ઈન્વર્ટર રેટિંગ}: 500-5000W
\item
  \textbf{સિસ્ટમ વોલ્ટેજ}: 12/24/48V
\end{itemize}

\end{solutionbox}
\begin{mnemonicbox}
``SCBIM'' - સોલર પેનલ્સ, કન્ટ્રોલર, બેટરી, ઈન્વર્ટર,
માઉન્ટિંગ

\end{mnemonicbox}
\subsection*{પ્રશ્ન 3(બ) [4
marks]}\label{uxaaauxab0uxab6uxaa8-3uxaac-4-marks}

\textbf{માઇક્રોવેવ ઓવન ના પ્રકારો, એપ્લિકેશનો અને ટેક્નિકલ સ્પેસિફિકેશનો લખો.}

\begin{solutionbox}

\textbf{માઇક્રોવેવ ઓવનના પ્રકારો:}

{\def\LTcaptype{none} % do not increment counter
\begin{longtable}[]{@{}ll@{}}
\toprule\noalign{}
પ્રકાર & વિશેષતાઓ \\
\midrule\noalign{}
\endhead
\bottomrule\noalign{}
\endlastfoot
\textbf{સોલો} & માત્ર બેઝિક હીટિંગ અને ડિફ્રોસ્ટિંગ \\
\textbf{ગ્રિલ} & વધારાનું ગ્રિલિંગ એલિમેન્ટ \\
\textbf{કન્વેક્શન} & હીટિંગ એલિમેન્ટ અને બેકિંગ માટે ફેન ધરાવે છે \\
\textbf{કોમ્બિનેશન} & માઇક્રોવેવ, ગ્રિલ અને કન્વેક્શન એકીકૃત કરે છે \\
\end{longtable}
}

\textbf{એપ્લિકેશનો:}

\begin{itemize}
\tightlist
\item
  ફૂડ રીહીટિંગ
\item
  ડિફ્રોસ્ટિંગ
\item
  કુકિંગ
\item
  બેકિંગ (કન્વેક્શન મોડેલ્સ)
\end{itemize}

\textbf{ટેક્નિકલ સ્પેસિફિકેશનો:}

\begin{itemize}
\tightlist
\item
  \textbf{પાવર}: 700-1200 વોટ્સ
\item
  \textbf{કેપેસિટી}: 20-40 લિટર
\item
  \textbf{ફ્રિક્વન્સી}: 2.45 GHz
\item
  \textbf{વોલ્ટેજ}: 220-240V AC
\end{itemize}

\end{solutionbox}
\begin{mnemonicbox}
``SGCC'' - સોલો, ગ્રિલ, કન્વેક્શન, કોમ્બો ઓવન્સ વિવિધ કુકિંગ
જરૂરિયાતો માટે

\end{mnemonicbox}
\subsection*{પ્રશ્ન 3(ક) [7
marks]}\label{uxaaauxab0uxab6uxaa8-3uxa95-7-marks}

\textbf{એર કંડીશનર અને રેફ્રિજરેટરની કાર્યપધ્ધતિ સમજાવો}

\begin{solutionbox}

\textbf{એર કંડીશનર અને રેફ્રિજરેટરનો કાર્ય સિદ્ધાંત:}

\begin{verbatim}
flowchart LR
    A[Compressor] {-{-}|High pressure hot gas| B[Condenser]}
    B {-{-}|High pressure liquid| C[Expansion Valve]}
    C {-{-}|Low pressure liquid| D[Evaporator]}
    D {-{-}|Low pressure gas| A}
\end{verbatim}

\textbf{સામાન્ય ઘટકો:}

\begin{itemize}
\tightlist
\item
  \textbf{કમ્પ્રેસર}: રેફ્રિજરન્ટ ગેસને દબાણ આપે છે
\item
  \textbf{કન્ડેન્સર}: ગરમી છોડે છે, ગેસને પ્રવાહીમાં રૂપાંતરિત કરે છે
\item
  \textbf{એક્સપાન્શન વાલ્વ}: પ્રવાહી રેફ્રિજરન્ટનું દબાણ ઘટાડે છે
\item
  \textbf{ઇવેપોરેટર}: ગરમી શોષે છે, પ્રવાહીને ગેસમાં રૂપાંતરિત કરે છે
\end{itemize}

\textbf{તફાવતો:}

{\def\LTcaptype{none} % do not increment counter
\begin{longtable}[]{@{}lll@{}}
\toprule\noalign{}
પાસું & એર કંડીશનર & રેફ્રિજરેટર \\
\midrule\noalign{}
\endhead
\bottomrule\noalign{}
\endlastfoot
\textbf{હેતુ} & સમગ્ર રૂમને ઠંડુ કરે છે & ઇન્સ્યુલેટેડ કેબિનેટમાં ઠંડક જાળવે છે \\
\textbf{તાપમાન} & સામાન્ય રીતે 18-26^\circC & 2-8^\circC (ફ્રિજ), -18^\circC (ફ્રીઝર) \\
\textbf{નિયંત્રણ} & રિમોટ સાથે થર્મોસ્ટેટ & મેન્યુઅલ અથવા ડિજિટલ થર્મોસ્ટેટ \\
\end{longtable}
}

\end{solutionbox}
\begin{mnemonicbox}
``CEVA'' - કમ્પ્રેશન, એક્સપાન્શન, વેપરાઇઝેશન, એબ્સોર્પશન
સાયકલ

\end{mnemonicbox}
\subsection*{પ્રશ્ન 3(અ OR) [3
marks]}\label{uxaaauxab0uxab6uxaa8-3uxa85-or-3-marks}

\textbf{એર કંડીશનર અને રેફ્રિજરેટર ના ટેક્નિકલ સ્પેસિફિકેશનો લખો.}

\begin{solutionbox}

\textbf{ટેક્નિકલ સ્પેસિફિકેશનો:}

{\def\LTcaptype{none} % do not increment counter
\begin{longtable}[]{@{}lll@{}}
\toprule\noalign{}
સ્પેસિફિકેશન & એર કંડીશનર & રેફ્રિજરેટર \\
\midrule\noalign{}
\endhead
\bottomrule\noalign{}
\endlastfoot
\textbf{કૂલિંગ કેપેસિટી} & 1-2 ટન (12,000-24,000 BTU) & 100-500 લિટર
કેપેસિટી \\
\textbf{પાવર કન્ઝમ્પશન} & 1000-2500 વોટ્સ & 100-400 વોટ્સ \\
\textbf{એનર્જી એફિશિયન્સી} & ISEER/સ્ટાર રેટિંગ 3-5 & BEE સ્ટાર રેટિંગ 3-5 \\
\textbf{રેફ્રિજરન્ટ પ્રકાર} & R32, R410A & R600a, R134a \\
\textbf{વોલ્ટેજ/ફ્રિક્વન્સી} & 220-240V/50Hz & 220-240V/50Hz \\
\end{longtable}
}

\end{solutionbox}
\begin{mnemonicbox}
``CPERS'' - કેપેસિટી, પાવર, એફિશિયન્સી, રેફ્રિજરન્ટ, સપ્લાય
સ્પેસિફિકેશન્સ

\end{mnemonicbox}
\subsection*{પ્રશ્ન 3(બ OR) [4
marks]}\label{uxaaauxab0uxab6uxaa8-3uxaac-or-4-marks}

\textbf{વોશિંગ મશીન માટે ઇલેક્ટ્રોનિક કંટ્રોલર સમજાવો.}

\begin{solutionbox}

\textbf{વોશિંગ મશીન માટે ઇલેક્ટ્રોનિક કંટ્રોલર:}

\begin{verbatim}
flowchart TD
    A[User Interface] {-{-} B[Microcontroller]}
    B {-{-} C[Motor Driver]}
    B {-{-} D[Water Valve Control]}
    B {-{-} E[Temperature Sensor]}
    B {-{-} F[Water Level Sensor]}
    B {-{-} G[Door Lock Control]}
    B {-{-} H[Drain Pump Control]}
\end{verbatim}

\begin{itemize}
\tightlist
\item
  \textbf{માઇક્રોકન્ટ્રોલર}: બધી ઓપરેશન નિયંત્રિત કરતું સેન્ટ્રલ પ્રોસેસિંગ યુનિટ
\item
  \textbf{સેન્સર્સ}: વોટર લેવલ, તાપમાન, લોડ બેલેન્સ, દરવાજાની સ્થિતિ
\item
  \textbf{એક્ચુએટર્સ}: મોટર ડ્રાઇવર, વોટર વાલ્વ, હીટર, ડ્રેઇન પમ્પ
\item
  \textbf{યુઝર ઇન્ટરફેસ}: પ્રોગ્રામ સિલેક્શન, તાપમાન, સ્પિન સ્પીડ સેટિંગ્સ
\end{itemize}

\end{solutionbox}
\begin{mnemonicbox}
``MIST-WAD'' - માઇક્રોકન્ટ્રોલર ઇન્ટિગ્રેટ્સ સેન્સર્સ અને
ટાઇમર્સ ફોર વોટર, એજિટેશન એન્ડ ડ્રેનેજ

\end{mnemonicbox}
\subsection*{પ્રશ્ન 3(ક OR) [7
marks]}\label{uxaaauxab0uxab6uxaa8-3uxa95-or-7-marks}

\textbf{માઇક્રોવેવ ઓવન નો બ્લોક ડાયેગ્રામ દોરો અને સમજાવો. માઇક્રોવેવ ઓવન માટે
વાયરિંગ અને સેફ્ટી ઇન્સ્ટ્રક્શન લખો.}

\begin{solutionbox}

\textbf{માઇક્રોવેવ ઓવન બ્લોક ડાયાગ્રામ:}

\begin{verbatim}
flowchart LR
    A[Control Panel] {-{-} B[Control Circuit]}
    B {-{-} C[High Voltage Transformer]}
    C {-{-} D[High Voltage Capacitor]}
    D {-{-} E[Magnetron]}
    E {-{-} F[Waveguide]}
    F {-{-} G[Cooking Cavity]}
    B {-{-} H[Turntable Motor]}
    B {-{-} I[Fan Motor]}
    B {-{-} J[Door Interlock Switches]}
\end{verbatim}

\begin{itemize}
\tightlist
\item
  \textbf{કન્ટ્રોલ સર્કિટ}: યુઝર ઇનપુટ્સ પ્રોસેસ કરે છે અને ટાઇમિંગ નિયંત્રિત કરે છે
\item
  \textbf{હાઈ વોલ્ટેજ ટ્રાન્સફોર્મર}: વોલ્ટેજને 2000-4000V સુધી સ્ટેપ અપ કરે છે
\item
  \textbf{મેગ્નેટ્રોન}: 2.45 GHz પર માઇક્રોવેવ રેડિએશન ઉત્પન્ન કરે છે
\item
  \textbf{વેવગાઇડ}: માઇક્રોવેવ્સને કુકિંગ કેવિટીમાં દોરે છે
\item
  \textbf{ટર્નટેબલ}: રોટેશન દ્વારા સમાન કુકિંગ સુનિશ્ચિત કરે છે
\end{itemize}

\textbf{સેફ્ટી ઇન્સ્ટ્રક્શન્સ:}

\begin{itemize}
\tightlist
\item
  દરવાજો ખુલ્લો અથવા ડેમેજ્ડ હોય ત્યારે ક્યારેય ઓપરેટ ન કરો
\item
  યોગ્ય ગ્રાઉન્ડિંગ સુનિશ્ચિત કરો
\item
  સેફ્ટી ઇન્ટરલૉક્સને ઓવરરાઇડ ન કરો
\item
  ફક્ત માઇક્રોવેવ-સેફ કન્ટેનર વાપરો
\end{itemize}

\textbf{વાયરિંગ ઇન્સ્ટ્રક્શન્સ:}

\begin{itemize}
\tightlist
\item
  યોગ્ય ગેજ પાવર કેબલ વાપરો (સામાન્ય રીતે 14-16 AWG)
\item
  15-20A સર્કિટ સાથે જોડો
\item
  યોગ્ય ગ્રાઉન્ડ કનેક્શન સુનિશ્ચિત કરો
\item
  વાયરિંગને હીટ સોર્સથી દૂર રાખો
\end{itemize}

\end{solutionbox}
\begin{mnemonicbox}
``MAGIC'' - મેગ્નેટ્રોન એન્ડ ગાઇડેડ વેવ્સ ઇનટુ કેવિટી

\end{mnemonicbox}
\subsection*{પ્રશ્ન 4(અ) [3
marks]}\label{uxaaauxab0uxab6uxaa8-4uxa85-3-marks}

\textbf{ફોટોકોપિયર નો બ્લોક ડાયેગ્રામ દોરો.}

\begin{solutionbox}

\textbf{ફોટોકોપિયર બ્લોક ડાયાગ્રામ:}

\begin{verbatim}
flowchart LR
    A[Document Scanner] {-{-} B[Image Processor]}
    B {-{-} C[Laser Unit]}
    C {-{-} D[Photosensitive Drum]}
    E[Charging Unit] {-{-} D}
    D {-{-} F[Developer Unit]}
    F {-{-} G[Transfer Unit]}
    G {-{-} H[Paper Feed]}
    G {-{-} I[Fusing Unit]}
    I {-{-} J[Output Tray]}
\end{verbatim}

\begin{itemize}
\tightlist
\item
  \textbf{સ્કેનર}: મૂળ દસ્તાવેજની છબી કેપ્ચર કરે છે
\item
  \textbf{ડ્રમ}: ઇલેક્ટ્રોસ્ટેટિક ઇમેજ ધારણ કરે છે
\item
  \textbf{ડેવલપર}: ચાર્જ થયેલા એરિયા પર ટોનર લાગુ કરે છે
\item
  \textbf{ટ્રાન્સફર}: ટોનરને પેપર પર ટ્રાન્સફર કરે છે
\item
  \textbf{ફ્યુઝર}: ટોનરને કાયમી રીતે પેપર પર પિગળાવે છે
\end{itemize}

\end{solutionbox}
\begin{mnemonicbox}
``SDTFO'' - સ્કેન, ડેવલપ, ટ્રાન્સફર, ફ્યુઝ, આઉટપુટ

\end{mnemonicbox}
\subsection*{પ્રશ્ન 4(બ) [4
marks]}\label{uxaaauxab0uxab6uxaa8-4uxaac-4-marks}

\textbf{એમએફ પ્રિંટર અને CCTV ના સ્પેસિફિકેશનો લખો.}

\begin{solutionbox}

\textbf{સ્પેસિફિકેશનો:}

{\def\LTcaptype{none} % do not increment counter
\begin{longtable}[]{@{}ll@{}}
\toprule\noalign{}
MF પ્રિંટર સ્પેસિફિકેશનો & CCTV સ્પેસિફિકેશનો \\
\midrule\noalign{}
\endhead
\bottomrule\noalign{}
\endlastfoot
\textbf{પ્રિન્ટ રેઝોલ્યુશન}: 600-1200 dpi & \textbf{કેમેરા રેઝોલ્યુશન}: 2-8 MP \\
\textbf{પ્રિન્ટ સ્પીડ}: 15-40 ppm & \textbf{ફ્રેમ રેટ}: 15-30 fps \\
\textbf{સ્કેન રેઝોલ્યુશન}: 300-600 dpi & \textbf{સ્ટોરેજ}: 1-8 TB HDD/NVR \\
\textbf{પેપર કેપેસિટી}: 150-500 શીટ્સ & \textbf{નાઇટ વિઝન}: 10-30m રેન્જ \\
\textbf{કનેક્ટિવિટી}: USB, ઇથરનેટ, Wi-Fi & \textbf{કનેક્ટિવિટી}:
કોએક્સિયલ/IP/વાયરલેસ \\
\textbf{ફંક્શન્સ}: પ્રિન્ટ, સ્કેન, કોપી, ફેક્સ & \textbf{વિડિયો ફોર્મેટ}:
H.264/H.265 \\
\end{longtable}
}

\end{solutionbox}
\begin{mnemonicbox}
``RSCPF'' - રેઝોલ્યુશન, સ્પીડ, કેપેસિટી, પ્રોટોકોલ, ફંક્શન
સ્પેસિફિકેશન્સ

\end{mnemonicbox}
\subsection*{પ્રશ્ન 4(ક) [7
marks]}\label{uxaaauxab0uxab6uxaa8-4uxa95-7-marks}

\textbf{લેસર પ્રિંટરની કાર્યપધ્ધતિ બ્લોક ડાયેગ્રામ સાથે સમજાવો}

\begin{solutionbox}

\textbf{લેસર પ્રિંટર કાર્યપધ્ધતિ:}

\begin{verbatim}
flowchart LR
    A[Data Processing] {-{-} B[Laser Unit]}
    B {-{-} C[Photosensitive Drum]}
    D[Primary Corona] {-{-} C}
    C {-{-} E[Developer Unit]}
    E {-{-} F[Transfer Corona]}
    F {-{-} G[Paper Transport]}
    G {-{-} H[Fusing Unit]}
    H {-{-} I[Output]}
    J[Cleaning Unit] {-{-} C}
\end{verbatim}

\textbf{કાર્ય પ્રક્રિયા:}

\begin{enumerate}
\tightlist
\item
  \textbf{ચાર્જિંગ}: કોરોના વાયર ડ્રમને યુનિફોર્મ નેગેટિવ ચાર્જ આપે છે
\item
  \textbf{રાઇટિંગ}: લેસર ઇમેજ બનાવવા માટે ડ્રમ પરના ચાર્જને ન્યુટ્રલાઇઝ કરે છે
\item
  \textbf{ડેવલપિંગ}: ટોનર ડ્રમના ડિસ્ચાર્જ થયેલા વિસ્તારો પર ચોંટે છે
\item
  \textbf{ટ્રાન્સફર}: પેપરને પોઝિટિવ ચાર્જ મળે છે, ટોનરને આકર્ષે છે
\item
  \textbf{ફ્યુઝિંગ}: હીટ અને પ્રેશર ટોનરને પેપર પર પિગળાવે છે
\item
  \textbf{ક્લીનિંગ}: ડ્રમ પરથી બાકી ટોનર દૂર કરવામાં આવે છે
\end{enumerate}

\begin{itemize}
\tightlist
\item
  \textbf{રેઝોલ્યુશન}: લેસર પ્રિસિઝન દ્વારા નક્કી થાય છે (600-1200 dpi)
\item
  \textbf{સ્પીડ}: ડ્રમ રોટેશન અને પેપર ટ્રાન્સપોર્ટ પર આધારિત છે (15-40 ppm)
\end{itemize}

\end{solutionbox}
\begin{mnemonicbox}
``CWTFC'' - ચાર્જ, રાઇટ, ટ્રાન્સફર, ફ્યુઝ, ક્લીન સાયકલ

\end{mnemonicbox}
\subsection*{પ્રશ્ન 4(અ OR) [3
marks]}\label{uxaaauxab0uxab6uxaa8-4uxa85-or-3-marks}

\textbf{CCTV નો બ્લોક ડાયેગ્રામ દોરો.}

\begin{solutionbox}

\textbf{CCTV સિસ્ટમ બ્લોક ડાયાગ્રામ:}

\begin{verbatim}
flowchart LR
    A[Cameras] {-{-} B[Video Transmission]}
    B {-{-} C[Digital Video Recorder]}
    C {-{-} D[Storage HDD]}
    C {-{-} E[Monitor Display]}
    F[Power Supply] {-{-} A}
    F {-{-} C}
    G[Network Switch] {-{-} C}
    C {-{-} H[Remote Access]}
\end{verbatim}

\begin{itemize}
\tightlist
\item
  \textbf{કેમેરા}: વિડિયો ફુટેજ કેપ્ચર કરે છે
\item
  \textbf{ટ્રાન્સમિશન}: કોએક્સિયલ કેબલ/IP નેટવર્ક/વાયરલેસ
\item
  \textbf{DVR/NVR}: વિડિયો પ્રોસેસ અને રેકોર્ડ કરે છે
\item
  \textbf{સ્ટોરેજ}: ફુટેજ રિટેન્શન માટે હાર્ડ ડ્રાઇવ
\item
  \textbf{મોનિટર}: લાઇવ અથવા રેકોર્ડેડ ફુટેજ દર્શાવે છે
\end{itemize}

\end{solutionbox}
\begin{mnemonicbox}
``CTDSM'' - કેમેરા, ટ્રાન્સમિશન, DVR, સ્ટોરેજ, મોનિટર
સિસ્ટમ

\end{mnemonicbox}
\subsection*{પ્રશ્ન 4(બ OR) [4
marks]}\label{uxaaauxab0uxab6uxaa8-4uxaac-or-4-marks}

\textbf{ઇંક જેટ પ્રિંટર અને ફોટોકોપિયર ના સ્પેસિફિકેશનો લખો.}

\begin{solutionbox}

\textbf{સ્પેસિફિકેશનો:}

{\def\LTcaptype{none} % do not increment counter
\begin{longtable}[]{@{}
  >{\raggedright\arraybackslash}p{(\linewidth - 2\tabcolsep) * \real{0.5000}}
  >{\raggedright\arraybackslash}p{(\linewidth - 2\tabcolsep) * \real{0.5000}}@{}}
\toprule\noalign{}
\begin{minipage}[b]{\linewidth}\raggedright
ઇંક જેટ પ્રિંટર સ્પેસિફિકેશનો
\end{minipage} & \begin{minipage}[b]{\linewidth}\raggedright
ફોટોકોપિયર સ્પેસિફિકેશનો
\end{minipage} \\
\midrule\noalign{}
\endhead
\bottomrule\noalign{}
\endlastfoot
\textbf{પ્રિન્ટ રેઝોલ્યુશન}: 1200-4800 dpi & \textbf{કોપી રેઝોલ્યુશન}: 600-1200
dpi \\
\textbf{પ્રિન્ટ સ્પીડ}: 8-20 ppm & \textbf{કોપી સ્પીડ}: 20-60 cpm \\
\textbf{ઇન્ક પ્રકાર}: ડાય/પિગમેન્ટ & \textbf{ટોનર પ્રકાર}: ડ્રાય/લિક્વિડ \\
\textbf{પેપર કેપેસિટી}: 100-250 શીટ્સ & \textbf{પેપર કેપેસિટી}: 250-2000
શીટ્સ \\
\textbf{કનેક્ટિવિટી}: USB, Wi-Fi & \textbf{ફંક્શન્સ}: કોપી, સ્કેન, પ્રિન્ટ,
ફેક્સ \\
\textbf{ડ્યુટી સાયકલ}: 1,000-5,000 પેજ/મહિનો & \textbf{ડ્યુટી સાયકલ}:
10,000-100,000 પેજ/મહિનો \\
\end{longtable}
}

\end{solutionbox}
\begin{mnemonicbox}
``RSIPCD'' - રેઝોલ્યુશન, સ્પીડ, ઇન્ક/ટોનર, પેપર કેપેસિટી,
કનેક્ટિવિટી, ડ્યુટી સાયકલ

\end{mnemonicbox}
\subsection*{પ્રશ્ન 4(ક OR) [7
marks]}\label{uxaaauxab0uxab6uxaa8-4uxa95-or-7-marks}

\textbf{એલસીડી પ્રોજેક્ટરની કાર્યપધ્ધતિ બ્લોક ડાયેગ્રામ સાથે સમજાવો અને તેના ટેક્નિકલ
સ્પેસિફિકેશનો લખો.}

\begin{solutionbox}

\textbf{LCD પ્રોજેક્ટર કાર્યપધ્ધતિ:}

\begin{verbatim}
flowchart LR
    A[Input Source] {-{-} B[Signal Processor]}
    B {-{-} C[Lamp/Light Source]}
    C {-{-} D[Condenser Lens]}
    D {-{-} E[Dichroic Mirrors]}
    E {-{-}|Red| F[Red LCD Panel]}
    E {-{-}|Green| G[Green LCD Panel]}
    E {-{-}|Blue| H[Blue LCD Panel]}
    F {-{-} I[Prism]}
    G {-{-} I}
    H {-{-} I}
    I {-{-} J[Projection Lens]}
    J {-{-} K[Screen]}
\end{verbatim}

\textbf{કાર્ય પ્રક્રિયા:}

\begin{enumerate}
\tightlist
\item
  \textbf{લાઇટ જનરેશન}: હાઈ-ઇન્ટેન્સિટી લેમ્પ વ્હાઈટ લાઇટ ઉત્પન્ન કરે છે
\item
  \textbf{કલર સેપરેશન}: ડાયક્રોઇક મિરર લાઇટને RGB માં વિભાજિત કરે છે
\item
  \textbf{મોડ્યુલેશન}: LCD પેનલ દરેક રંગ માટે લાઇટ ઇન્ટેન્સિટી નિયંત્રિત કરે છે
\item
  \textbf{રિકોમ્બિનેશન}: પ્રિઝમ RGB ઇમેજને ફરીથી એકત્રિત કરે છે
\item
  \textbf{પ્રોજેક્શન}: લેન્સ સિસ્ટમ ઇમેજને સ્ક્રીન પર પ્રોજેક્ટ કરે છે
\end{enumerate}

\textbf{સ્પેસિફિકેશનો:}

\begin{itemize}
\tightlist
\item
  \textbf{રેઝોલ્યુશન}: XGA (1024\times768), WXGA (1280\times800), FHD (1920\times1080)
\item
  \textbf{બ્રાઇટનેસ}: 2000-5000 ANSI લુમેન્સ
\item
  \textbf{કોન્ટ્રાસ્ટ રેશિયો}: 2000:1 થી 20000:1
\item
  \textbf{લેમ્પ લાઇફ}: 3000-6000 કલાક
\item
  \textbf{થ્રો રેશિયો}: 0.5:1 થી 2.0:1
\item
  \textbf{કનેક્ટિવિટી}: HDMI, VGA, USB, Wi-Fi
\end{itemize}

\end{solutionbox}
\begin{mnemonicbox}
``LSPMPS'' - લેમ્પ, સ્પ્લિટ, પેનલ્સ, મોડ્યુલેટ, પ્રિઝમ, સ્ક્રીન

\end{mnemonicbox}
\subsection*{પ્રશ્ન 5(અ) [3
marks]}\label{uxaaauxab0uxab6uxaa8-5uxa85-3-marks}

\textbf{પીએ સિસ્ટમનો બ્લોક ડાયેગ્રામ દોરો}

\begin{solutionbox}

\textbf{પબ્લિક એડ્રેસ (PA) સિસ્ટમ બ્લોક ડાયાગ્રામ:}

\begin{verbatim}
flowchart LR
    A[Microphone] {-{-} B[Pre{-}amplifier]}
    B {-{-} C[Mixer]}
    D[Audio Source] {-{-} C}
    C {-{-} E[Equalizer]}
    E {-{-} F[Power Amplifier]}
    F {-{-} G[Speaker Network]}
    H[Volume Control] {-{-} C}
\end{verbatim}

\begin{itemize}
\tightlist
\item
  \textbf{માઇક્રોફોન}: ધ્વનિને ઇલેક્ટ્રિકલ સિગ્નલમાં રૂપાંતરિત કરે છે
\item
  \textbf{પ્રી-એમ્પ્લિફાયર}: માઇક્રોફોન સિગ્નલને બૂસ્ટ કરે છે
\item
  \textbf{મિક્સર}: મલ્ટિપલ ઓડિયો સોર્સને જોડે છે
\item
  \textbf{ઇક્વલાઇઝર}: ફ્રિક્વન્સી રિસ્પોન્સ એડજસ્ટ કરે છે
\item
  \textbf{પાવર એમ્પ્લિફાયર}: સિગ્નલ પાવર વધારે છે
\item
  \textbf{સ્પીકર્સ}: ઇલેક્ટ્રિકલ સિગ્નલને પાછા ધ્વનિમાં કન્વર્ટ કરે છે
\end{itemize}

\end{solutionbox}
\begin{mnemonicbox}
``MMEPS'' - માઇક્રોફોન, મિક્સર, ઇક્વલાઇઝર, પાવર એમ્પ,
સ્પીકર્સ

\end{mnemonicbox}
\subsection*{પ્રશ્ન 5(બ) [4
marks]}\label{uxaaauxab0uxab6uxaa8-5uxaac-4-marks}

\textbf{ટ્વીટર અને વૂફર સમજાવો}

\begin{solutionbox}

\textbf{સ્પીકર કોમ્પોનન્ટ્સ:}

{\def\LTcaptype{none} % do not increment counter
\begin{longtable}[]{@{}lll@{}}
\toprule\noalign{}
ફીચર & ટ્વીટર & વૂફર \\
\midrule\noalign{}
\endhead
\bottomrule\noalign{}
\endlastfoot
\textbf{ફ્રિક્વન્સી રેન્જ} & હાઈ (2kHz-20kHz) & લો (20Hz-2kHz) \\
\textbf{સાઇઝ} & સ્મોલ (0.5''-1.5'') & લાર્જ (4''-15'') \\
\textbf{ડાયાફ્રામ} & લાઇટ, રિજિડ (ડોમ/કોન) & હેવી, ફ્લેક્સિબલ કોન \\
\textbf{વોઇસ કોઇલ} & સ્મોલ ડાયામીટર & લાર્જ ડાયામીટર \\
\textbf{કેબિનેટ ડિઝાઇન} & હોર્ન/સીલ્ડ & પોર્ટેડ/સીલ્ડ/બાસ રિફ્લેક્સ \\
\end{longtable}
}

\textbf{કાર્ય સિદ્ધાંત:}

\begin{verbatim}
flowchart LR
    A[Audio Signal] {-{-} B[Crossover Network]}
    B {-{-}|High Frequencies| C[Tweeter]}
    B {-{-}|Low Frequencies| D[Woofer]}
    C {-{-} E[High{-}Frequency Sound Waves]}
    D {-{-} F[Low{-}Frequency Sound Waves]}
\end{verbatim}

\begin{itemize}
\tightlist
\item
  \textbf{ટ્વીટર}: ઉચ્ચ આવૃત્તિઓને સ્પષ્ટતા અને વિગતવાર રીતે રીપ્રોડ્યુસ કરે છે
\item
  \textbf{વૂફર}: ઓછી આવૃત્તિઓને પાવર અને ડેપ્થ સાથે રીપ્રોડ્યુસ કરે છે
\end{itemize}

\end{solutionbox}
\begin{mnemonicbox}
``THSL'' - ટ્વીટર્સ હેન્ડલ હાઇસ, સ્મોલ એન્ડ લાઇટ; વૂફર્સ
હેન્ડલ લોસ

\end{mnemonicbox}
\subsection*{પ્રશ્ન 5(ક) [7
marks]}\label{uxaaauxab0uxab6uxaa8-5uxa95-7-marks}

\textbf{માઇક્રોફોનની વ્યાખ્યા આપો. માઇક્રોફોનના પ્રકારો લખો અને કોઇ પણ એક
માઇક્રોફોનની કાર્યપધ્ધતિ સમજાવો.}

\begin{solutionbox}

\textbf{માઇક્રોફોનની વ્યાખ્યા:} માઇક્રોફોન એક ઇલેક્ટ્રોએકોસ્ટિક ટ્રાન્સડ્યુસર છે જે
ધ્વનિ તરંગોને ઇલેક્ટ્રિકલ સિગ્નલમાં રૂપાંતરિત કરે છે.

\textbf{માઇક્રોફોનના પ્રકારો:}

{\def\LTcaptype{none} % do not increment counter
\begin{longtable}[]{@{}lll@{}}
\toprule\noalign{}
પ્રકાર & કાર્ય સિદ્ધાંત & એપ્લિકેશન્સ \\
\midrule\noalign{}
\endhead
\bottomrule\noalign{}
\endlastfoot
\textbf{ડાયનેમિક} & ઇલેક્ટ્રોમેગ્નેટિક ઇન્ડક્શન & લાઇવ પરફોર્મન્સ, બ્રોડકાસ્ટિંગ \\
\textbf{કન્ડેન્સર} & ઇલેક્ટ્રોસ્ટેટિક પ્રિન્સિપલ & સ્ટુડિયો રેકોર્ડિંગ, સ્માર્ટફોન \\
\textbf{રિબન} & ઇલેક્ટ્રોમેગ્નેટિક ઇન્ડક્શન & સ્ટુડિયો વોકલ્સ, ઇન્સ્ટ્રુમેન્ટ્સ \\
\textbf{કાર્બન} & રેઝિસ્ટન્સ વેરિએશન & જૂના ટેલિફોન \\
\textbf{પિઝોઇલેક્ટ્રિક} & પિઝોઇલેક્ટ્રિક ઇફેક્ટ & કોન્ટેક્ટ માઇક, ઇન્સ્ટ્રુમેન્ટ્સ \\
\textbf{MEMS} & માઇક્રો-ઇલેક્ટ્રોમિકેનિકલ & લેપટોપ, નાના ડિવાઇસ \\
\end{longtable}
}

\textbf{ડાયનેમિક માઇક્રોફોન કાર્યપધ્ધતિ:}

\begin{verbatim}
flowchart LR
    A[Sound Waves] {-{-} B[Diaphragm]}
    B {-{-} C[Attached Coil]}
    C {-{-} D[Movement in Magnetic Field]}
    D {-{-} E[Induced Voltage]}
    E {-{-} F[Electrical Signal Output]}
\end{verbatim}

\begin{itemize}
\tightlist
\item
  \textbf{સાઉન્ડ કેપ્ચર}: ડાયાફ્રામ ધ્વનિ તરંગો સાથે કંપન કરે છે
\item
  \textbf{ટ્રાન્સડક્શન}: ડાયાફ્રામ સાથે જોડાયેલી કોઇલ ચુંબકીય ક્ષેત્રમાં હલનચલન કરે છે
\item
  \textbf{સિગ્નલ જનરેશન}: હલનચલન ધ્વનિની તીવ્રતાના પ્રમાણમાં વોલ્ટેજ પ્રેરિત કરે છે
\item
  \textbf{આઉટપુટ}: ઓછા ઇમ્પિડન્સ, મજબૂત સિગ્નલ જેને ન્યૂનતમ એમ્પ્લિફિકેશનની જરૂર પડે છે
\item
  \textbf{ફાયદાઓ}: ટકાઉ, ઉચ્ચ SPL સંભાળી શકે છે, બાહ્ય પાવરની જરૂર નથી
\end{itemize}

\end{solutionbox}
\begin{mnemonicbox}
``DDCMIO'' - ડાયાફ્રામ ડિસ્પ્લેસિસ કોઇલ ઇન મેગ્નેટિક ફિલ્ડ
ઇન્ડ્યુસિંગ આઉટપુટ

\end{mnemonicbox}
\subsection*{પ્રશ્ન 5(અ OR) [3
marks]}\label{uxaaauxab0uxab6uxaa8-5uxa85-or-3-marks}

\textbf{વ્યાખ્યા આપો: (૧) પિચ (૨) લાઉડસ્પીકર (3) રીવર્બરેશન}

\begin{solutionbox}

\textbf{વ્યાખ્યાઓ:}

{\def\LTcaptype{none} % do not increment counter
\begin{longtable}[]{@{}
  >{\raggedright\arraybackslash}p{(\linewidth - 2\tabcolsep) * \real{0.5000}}
  >{\raggedright\arraybackslash}p{(\linewidth - 2\tabcolsep) * \real{0.5000}}@{}}
\toprule\noalign{}
\begin{minipage}[b]{\linewidth}\raggedright
પદ
\end{minipage} & \begin{minipage}[b]{\linewidth}\raggedright
વ્યાખ્યા
\end{minipage} \\
\midrule\noalign{}
\endhead
\bottomrule\noalign{}
\endlastfoot
\textbf{પિચ} & ધ્વનિની અનુભવાતી આવૃત્તિ જે નક્કી કરે છે કે તે કેટલો ``ઊંચો'' અથવા
``નીચો'' સંભળાય છે \\
\textbf{લાઉડસ્પીકર} & એક ઇલેક્ટ્રોએકોસ્ટિક ટ્રાન્સડ્યુસર જે ઇલેક્ટ્રિકલ સિગ્નલને ધ્વનિ
તરંગોમાં રૂપાંતરિત કરે છે \\
\textbf{રીવર્બરેશન} & મૂળ ધ્વનિ બંધ થયા પછી પણ બહુવિધ પરાવર્તનોને કારણે ધ્વનિની
સાતત્યતા \\
\end{longtable}
}

\textbf{ડાયાગ્રામ:}

\begin{verbatim}
Reverberation
    |
    v
Original Sound {-{-}{-}{-}{-} Early Reflections {-}{-}{-}{-}{-} Late Reflections}
    \^{                       |                       |}
    |                       v                       v
Direct Sound           Clarity (80ms)       Spaciousness ({80ms)}
\end{verbatim}

\end{solutionbox}
\begin{mnemonicbox}
``PLR Sound'' - પિચ ટોન વ્યાખ્યાયિત કરે છે, લાઉડસ્પીકર તેને
ઉત્પન્ન કરે છે, રીવર્બરેશન તેને વિસ્તારે છે

\end{mnemonicbox}
\subsection*{પ્રશ્ન 5(બ OR) [4
marks]}\label{uxaaauxab0uxab6uxaa8-5uxaac-or-4-marks}

\textbf{હોમ થિયેટર સાઉંડ સિસ્ટમ નો બ્લોક ડાયેગ્રામ દોરો અને ટૂંકમા સમજાવો.}

\begin{solutionbox}

\textbf{હોમ થિયેટર સાઉંડ સિસ્ટમ:}

\begin{verbatim}
flowchart TD
    A[Audio/Video Source] {-{-} B[AV Receiver/Amplifier]}
    B {-{-} C[Front Speakers]}
    B {-{-} D[Center Speaker]}
    B {-{-} E[Surround Speakers]}
    B {-{-} F[Subwoofer]}
    B {-{-} G[Video Display]}
    H[Remote Control] {-{-} B}
\end{verbatim}

\begin{itemize}
\tightlist
\item
  \textbf{AV રિસીવર}: ઓડિયો/વિડિયો સિગ્નલ પ્રોસેસ કરતું સેન્ટ્રલ હબ
\item
  \textbf{ફ્રન્ટ સ્પીકર્સ}: સ્ટીરિયો સાઉન્ડ માટે લેફ્ટ અને રાઇટ ચેનલ
\item
  \textbf{સેન્ટર સ્પીકર}: ડાયલોગ અને સેન્ટ્રલ સાઉન્ડ ડેલિવર કરે છે
\item
  \textbf{સરાઉન્ડ સ્પીકર્સ}: એમ્બિયન્ટ સાઉન્ડ સાથે ઇમર્સિવ વાતાવરણ બનાવે છે
\item
  \textbf{સબવૂફર}: 120Hz નીચેના લો-ફ્રિક્વન્સી ઇફેક્ટ્સ (LFE) રીપ્રોડ્યુસ કરે છે
\item
  \textbf{કોન્ફિગરેશન}: સામાન્ય સેટઅપમાં 2.1, 5.1, 7.1, અથવા 9.1 ચેનલ સિસ્ટમ
  શામેલ છે
\end{itemize}

\end{solutionbox}
\begin{mnemonicbox}
``AFSCS'' - એમ્પ્લિફાયર ડ્રાઇવ્સ ફ્રન્ટ, સરાઉન્ડ, સેન્ટર
સ્પીકર્સ એન્ડ સબવૂફર

\end{mnemonicbox}
\subsection*{પ્રશ્ન 5(ક OR) [7
marks]}\label{uxaaauxab0uxab6uxaa8-5uxa95-or-7-marks}

\textbf{ઇલેક્ટ્રોસ્ટેટિક લાઉડસ્પીકર અને પરમેનેન્ટ મેગ્નેટ લાઉડસ્પીકર સમજાવો.}

\begin{solutionbox}

\textbf{લાઉડસ્પીકર પ્રકારોની તુલના:}

{\def\LTcaptype{none} % do not increment counter
\begin{longtable}[]{@{}
  >{\raggedright\arraybackslash}p{(\linewidth - 4\tabcolsep) * \real{0.3333}}
  >{\raggedright\arraybackslash}p{(\linewidth - 4\tabcolsep) * \real{0.3333}}
  >{\raggedright\arraybackslash}p{(\linewidth - 4\tabcolsep) * \real{0.3333}}@{}}
\toprule\noalign{}
\begin{minipage}[b]{\linewidth}\raggedright
ફીચર
\end{minipage} & \begin{minipage}[b]{\linewidth}\raggedright
ઇલેક્ટ્રોસ્ટેટિક સ્પીકર
\end{minipage} & \begin{minipage}[b]{\linewidth}\raggedright
પરમેનેન્ટ મેગ્નેટ સ્પીકર
\end{minipage} \\
\midrule\noalign{}
\endhead
\bottomrule\noalign{}
\endlastfoot
\textbf{કાર્ય સિદ્ધાંત} & પ્લેટ્સ વચ્ચે ઇલેક્ટ્રોસ્ટેટિક બળ & ઇલેક્ટ્રોમેગ્નેટિક ઇન્ડક્શન \\
\textbf{બંધારણ} & સ્ટેટર પ્લેટ્સ વચ્ચે પાતળું ડાયાફ્રામ & ચુંબકીય ક્ષેત્રમાં વોઇસ કોઇલ
સાથે જોડાયેલું કોન \\
\textbf{પાવર રિક્વાયરમેન્ટ} & ઉચ્ચ વોલ્ટેજ પોલરાઇઝિંગ સપ્લાયની જરૂર & સિગ્નલ સિવાય
બાહ્ય પાવરની જરૂર નથી \\
\textbf{ફ્રિક્વન્સી રિસ્પોન્સ} & ઉત્કૃષ્ટ મિડ/હાઇ ફ્રિક્વન્સી & યોગ્ય ડિઝાઇન સાથે
સંપૂર્ણ રેન્જમાં સારું \\
\textbf{એફિશિયન્સી} & ઓછી (1-3\%) & મધ્યમ (2-5\%) \\
\textbf{ડિસ્ટોર્શન} & ખૂબ ઓછું & મધ્યમ \\
\end{longtable}
}

\textbf{ઇલેક્ટ્રોસ્ટેટિક સ્પીકર કાર્યપધ્ધતિ:}

\begin{verbatim}
flowchart LR
    A[Audio Signal] {-{-} B[Step{-}up Transformer]}
    C[High Voltage DC Supply] {-{-} D[Charged Diaphragm]}
    B {-{-} E[Conductive Stator Plates]}
    E {-{-} F[Electrostatic Force]}
    F {-{-} D}
    D {-{-} G[Sound Waves]}
\end{verbatim}

\begin{itemize}
\tightlist
\item
  \textbf{ડાયાફ્રામ}: કન્ડક્ટિવ કોટિંગ સાથે પાતળું, હલકું મેમ્બ્રેન
\item
  \textbf{ઓપરેશન}: ઓડિયો સિગ્નલ સ્ટેટર પ્લેટ્સ પરના ચાર્જમાં ફેરફાર કરે છે, જે
  ડાયાફ્રામ પર બદલાતું બળ ઉત્પન્ન કરે છે
\end{itemize}

\textbf{પરમેનેન્ટ મેગ્નેટ સ્પીકર કાર્યપધ્ધતિ:}

\begin{verbatim}
flowchart LR
    A[Audio Signal] {-{-} B[Voice Coil]}
    C[Permanent Magnet] {-{-} D[Magnetic Field]}
    B {-{-} E[Current in Coil]}
    E {-{-} F[Electromagnetic Force]}
    F {-{-} G[Cone Displacement]}
    G {-{-} H[Air Movement]}
    H {-{-} I[Sound Waves]}
\end{verbatim}

\begin{itemize}
\tightlist
\item
  \textbf{વોઇસ કોઇલ}: સ્પીકર કોન સાથે જોડાયેલી તારની વાઇન્ડિંગ
\item
  \textbf{ઓપરેશન}: કોઇલ મારફતે વીજપ્રવાહ ચુંબકીય ક્ષેત્ર ઉત્પન્ન કરે છે જે પરમેનેન્ટ
  મેગ્નેટ સાથે ઇન્ટરેક્ટ કરે છે
\item
  \textbf{ફાયદાઓ}: મજબૂત ડિઝાઇન, સારી પાવર હેન્ડલિંગ, ઉચ્ચ વોલ્ટેજની જરૂર નથી
\item
  \textbf{એપ્લિકેશન્સ}: સામાન્ય ઓડિયો રીપ્રોડક્શન માટે સૌથી સામાન્ય સ્પીકર ડિઝાઇન
\end{itemize}

\end{solutionbox}
\begin{mnemonicbox}
``ESPM'' - ઇલેક્ટ્રોસ્ટેટિક યુઝિસ સ્ટેટિક ચાર્જિસ, પરમેનેન્ટ
મેગ્નેટ યુઝિસ મેગ્નેટિક ફોર્સિસ

\end{mnemonicbox}

\end{document}
