\documentclass[10pt,a4paper]{article}

% content/resources/templates/preamble.tex
\usepackage[margin=0.6in]{geometry}
\author{Milav Dabgar}
\usepackage{amsmath,amssymb,amsthm}
\usepackage{booktabs}
\usepackage{multirow}
\usepackage{xcolor}
\usepackage{tcolorbox}
\tcbuselibrary{breakable,skins}
\usepackage[colorlinks=true,linkcolor=blue]{hyperref}
\usepackage{titlesec}
\usepackage{enumitem}
\usepackage{tikz}
\usepackage{pgfplots}
\usepackage{circuitikz}
\usepackage[version=4]{mhchem}
\usepackage{longtable}
\usepackage{array}
\usepackage{float}
\usepackage{caption}
\usepackage{listings}

\lstset{
  basicstyle=\small\ttfamily,
  breaklines=true,
  breakatwhitespace=false,
  postbreak=\mbox{\textcolor{red}{$\hookrightarrow$}\space},
  float=false,
  numbers=left,
  numberstyle=\tiny\color{gray},
  numbersep=10pt,
  xleftmargin=2em,
  keywordstyle=\color{blue},
  commentstyle=\color{green!60!black},
  stringstyle=\color{purple},
  backgroundcolor=\color{gray!5},
  showstringspaces=false,
  tabsize=2,
  captionpos=b,
  keepspaces=true,
  columns=flexible
}

\pgfplotsset{compat=1.18}
\usetikzlibrary{shapes,arrows,positioning,calc,patterns,decorations.pathmorphing,decorations.markings,arrows.meta}

% Color scheme
\definecolor{headcolor}{RGB}{0,102,204}
\definecolor{keycolor}{RGB}{220,20,60}
\definecolor{solutioncolor}{RGB}{34,139,34}
\definecolor{mnemoniccolor}{RGB}{148,0,211}
\definecolor{codecolor}{RGB}{0,0,100}

% Spacing
\setlength{\parskip}{3pt}
\setlist[itemize]{nosep}
\setlist[enumerate]{nosep}

% Title formatting
\titleformat{\section}{\Large\bfseries\color{headcolor}}{\thesection}{1em}{}
\titleformat{\subsection}{\large\bfseries\color{headcolor}}{\thesubsection}{1em}{}

% Pandoc tightlist compatibility
\providecommand{\tightlist}{%
  \setlength{\itemsep}{0pt}\setlength{\parskip}{0pt}}

% Pandoc longtable compatibility
\newcounter{none}
\def\thenone{}


% content/resources/templates/english-boxes.tex
% This file is currently empty - it exists to maintain consistency with the import structure.
% Add custom environments here if needed in the future.


\begin{document}

\begin{center}
{\Huge\bfseries\color{headcolor} Subject Name Solutions}\\[5pt]
{\LARGE 4341107 -- Summer 2023}\\[3pt]
{\large Semester 1 Study Material}\\[3pt]
{\normalsize\textit{Detailed Solutions and Explanations}}
\end{center}

\vspace{10pt}

\subsection*{Question 1(a) [3 marks]}\label{q1a}

\textbf{Describe maintenance procedure of CCTV.}

\begin{solutionbox}


{\def\LTcaptype{none} % do not increment counter
\vspace{-5pt}
\captionof{table}{CCTV Maintenance Procedure}
\vspace{-10pt}
\begin{longtable}[]{@{}lll@{}}
\toprule\noalign{}
Step & Procedure & Details \\
\midrule\noalign{}
\endhead
\bottomrule\noalign{}
\endlastfoot
1 & \textbf{Camera Cleaning} & Clean lenses and housings monthly \\
2 & \textbf{Cable Inspection} & Check for damage/exposure quarterly \\
3 & \textbf{Recording Check} & Verify data storage and playback
monthly \\
4 & \textbf{Firmware Updates} & Update software when available \\
5 & \textbf{Angle Adjustment} & Realign cameras as needed \\
\end{longtable}
}

\end{solutionbox}
\begin{mnemonicbox}
``CCRU: Clean, Check, Record, Update''

\end{mnemonicbox}
\subsection*{Question 1(b) [4 marks]}\label{q1b}

\textbf{List the types of maintenance and explain in brief.}

\begin{solutionbox}


{\def\LTcaptype{none} % do not increment counter
\vspace{-5pt}
\captionof{table}{Types of Maintenance}
\vspace{-10pt}
\begin{longtable}[]{@{}
  >{\raggedright\arraybackslash}p{(\linewidth - 6\tabcolsep) * \real{0.1333}}
  >{\raggedright\arraybackslash}p{(\linewidth - 6\tabcolsep) * \real{0.2889}}
  >{\raggedright\arraybackslash}p{(\linewidth - 6\tabcolsep) * \real{0.3556}}
  >{\raggedright\arraybackslash}p{(\linewidth - 6\tabcolsep) * \real{0.2222}}@{}}
\toprule\noalign{}
\begin{minipage}[b]{\linewidth}\raggedright
Type
\end{minipage} & \begin{minipage}[b]{\linewidth}\raggedright
Description
\end{minipage} & \begin{minipage}[b]{\linewidth}\raggedright
When Performed
\end{minipage} & \begin{minipage}[b]{\linewidth}\raggedright
Benefits
\end{minipage} \\
\midrule\noalign{}
\endhead
\bottomrule\noalign{}
\endlastfoot
\textbf{Preventive} & Regular checks before failure & Scheduled
intervals & Reduces unexpected downtime \\
\textbf{Corrective} & Repairs after equipment breaks & After failure
occurs & Restores functionality \\
\textbf{Predictive} & Uses data to predict failures & Based on analysis
& Optimizes maintenance timing \\
\textbf{Condition-based} & Monitors actual equipment state & When
conditions indicate & Reduces unnecessary maintenance \\
\end{longtable}
}

\begin{center}
\textbf{Mermaid Diagram (Code)}
\begin{verbatim}
{Shaded}
{Highlighting}[]
graph TD
    A[Maintenance Types] {-{-}{} B[Preventive]}
    A {-{-}{} C[Corrective]}
    A {-{-}{} D[Predictive]}
    A {-{-}{} E[Condition{-}based]}
    B {-{-}{} F[Scheduled checks]}
    C {-{-}{} G[Repairs after breakdown]}
    D {-{-}{} H[Data{-}based forecasting]}
    E {-{-}{} I[Based on equipment condition]}
{Highlighting}
{Shaded}
\end{verbatim}
\end{center}

\end{solutionbox}
\begin{mnemonicbox}
``PCPC: Prevent, Correct, Predict, Condition''

\end{mnemonicbox}
\subsection*{Question 1(c) [7 marks]}\label{q1c}

\textbf{Explain maintenance and troubleshooting procedure of Washing
Machine.}

\begin{solutionbox}


{\def\LTcaptype{none} % do not increment counter
\vspace{-5pt}
\captionof{table}{Washing Machine Maintenance and Troubleshooting}
\vspace{-10pt}
\begin{longtable}[]{@{}
  >{\raggedright\arraybackslash}p{(\linewidth - 4\tabcolsep) * \real{0.1957}}
  >{\raggedright\arraybackslash}p{(\linewidth - 4\tabcolsep) * \real{0.3261}}
  >{\raggedright\arraybackslash}p{(\linewidth - 4\tabcolsep) * \real{0.4783}}@{}}
\toprule\noalign{}
\begin{minipage}[b]{\linewidth}\raggedright
Problem
\end{minipage} & \begin{minipage}[b]{\linewidth}\raggedright
Possible Cause
\end{minipage} & \begin{minipage}[b]{\linewidth}\raggedright
Troubleshooting Steps
\end{minipage} \\
\midrule\noalign{}
\endhead
\bottomrule\noalign{}
\endlastfoot
\textbf{Machine not starting} & Power issue, door lock & Check power
supply, ensure door is closed properly \\
\textbf{Not filling with water} & Water supply, inlet valve & Check
water taps, inspect inlet hoses for blocks \\
\textbf{Not draining} & Clogged filter, drain pump & Clean filter, check
drain hose for kinks \\
\textbf{Excessive vibration} & Unbalanced load, shipping bolts &
Redistribute clothes, check if shipping bolts removed \\
\textbf{Leaking water} & Damaged hoses, loose connections & Inspect and
tighten connections, replace damaged hoses \\
\end{longtable}
}

\textbf{Regular Maintenance:}

\begin{itemize}
\tightlist
\item
  \textbf{Monthly}: Clean detergent drawer and door seal
\item
  \textbf{Quarterly}: Run empty hot cycle with vinegar/cleaner
\item
  \textbf{Bi-annually}: Check hoses for cracks, clean filter
\end{itemize}

\begin{verbatim}
flowchart LR
    A[Problem Detected] {-{-} B\{Machine Starts?\}}
    B {-{-}|No| C[Check Power \& Door Lock]}
    B {-{-}|Yes| D\{Fills with Water?\}}
    D {-{-}|No| E[Check Water Supply \& Inlet Valve]}
    D {-{-}|Yes| F\{Drains Properly?\}}
    F {-{-}|No| G[Check Filter \& Drain Pump]}
    F {-{-}|Yes| H\{Excessive Vibration?\}}
    H {-{-}|Yes| I[Check Load Balance \& Shipping Bolts]}
    H {-{-}|No| J\{Water Leakage?\}}
    J {-{-}|Yes| K[Check Hoses \& Connections]}
\end{verbatim}

\end{solutionbox}
\begin{mnemonicbox}
``POWER: Power, Observe, Water, Examine, Repair''

\end{mnemonicbox}
\subsection*{Question 1(c OR) [7
marks]}\label{question-1c-or-7-marks}

\textbf{Explain maintenance and troubleshooting procedure of Digital
TV.}

\begin{solutionbox}


{\def\LTcaptype{none} % do not increment counter
\vspace{-5pt}
\captionof{table}{Digital TV Maintenance and Troubleshooting}
\vspace{-10pt}
\begin{longtable}[]{@{}
  >{\raggedright\arraybackslash}p{(\linewidth - 4\tabcolsep) * \real{0.1957}}
  >{\raggedright\arraybackslash}p{(\linewidth - 4\tabcolsep) * \real{0.3261}}
  >{\raggedright\arraybackslash}p{(\linewidth - 4\tabcolsep) * \real{0.4783}}@{}}
\toprule\noalign{}
\begin{minipage}[b]{\linewidth}\raggedright
Problem
\end{minipage} & \begin{minipage}[b]{\linewidth}\raggedright
Possible Cause
\end{minipage} & \begin{minipage}[b]{\linewidth}\raggedright
Troubleshooting Steps
\end{minipage} \\
\midrule\noalign{}
\endhead
\bottomrule\noalign{}
\endlastfoot
\textbf{No power} & Power supply issue & Check power cord, wall outlet,
try different socket \\
\textbf{No picture} & Input/source selection & Verify correct input
selected, check source device \\
\textbf{Poor reception} & Antenna/cable issue & Check cable connections,
reposition antenna \\
\textbf{Distorted colors} & Display settings & Reset picture settings to
default \\
\textbf{Remote not working} & Battery issue, sensor blocked & Replace
batteries, ensure IR sensor not blocked \\
\end{longtable}
}

\textbf{Regular Maintenance:}

\begin{itemize}
\tightlist
\item
  \textbf{Weekly}: Dust screen carefully with microfiber cloth
\item
  \textbf{Monthly}: Check and tighten cable connections
\item
  \textbf{Annually}: Update firmware if available
\end{itemize}

\begin{verbatim}
flowchart LR
    A[TV Problem] {-{-} B\{Powers On?\}}
    B {-{-}|No| C[Check Power Supply]}
    B {-{-}|Yes| D\{Picture Visible?\}}
    D {-{-}|No| E[Check Input Source]}
    D {-{-}|Yes| F\{Good Reception?\}}
    F {-{-}|No| G[Check Antenna/Cable]}
    F {-{-}|Yes| H\{Correct Colors?\}}
    H {-{-}|No| I[Reset Picture Settings]}
    H {-{-}|Yes| J\{Remote Working?\}}
    J {-{-}|No| K[Check Batteries/Sensor]}
\end{verbatim}

\end{solutionbox}
\begin{mnemonicbox}
``SPIRE: Supply, Picture, Input, Reception,
Electronics''

\end{mnemonicbox}
\subsection*{Question 2(a) [3 marks]}\label{q2a}

\textbf{Define: (1) Brightness (2) Luminance (3) Chrominance}

\begin{solutionbox}


{\def\LTcaptype{none} % do not increment counter
\vspace{-5pt}
\captionof{table}{Key TV Display Terms}
\vspace{-10pt}
\begin{longtable}[]{@{}
  >{\raggedright\arraybackslash}p{(\linewidth - 4\tabcolsep) * \real{0.1935}}
  >{\raggedright\arraybackslash}p{(\linewidth - 4\tabcolsep) * \real{0.3871}}
  >{\raggedright\arraybackslash}p{(\linewidth - 4\tabcolsep) * \real{0.4194}}@{}}
\toprule\noalign{}
\begin{minipage}[b]{\linewidth}\raggedright
Term
\end{minipage} & \begin{minipage}[b]{\linewidth}\raggedright
Definition
\end{minipage} & \begin{minipage}[b]{\linewidth}\raggedright
Measured In
\end{minipage} \\
\midrule\noalign{}
\endhead
\bottomrule\noalign{}
\endlastfoot
\textbf{Brightness} & Perceived intensity of light output from display &
Subjective perception (nits) \\
\textbf{Luminance} & Objective measurement of light intensity per unit
area & Candela per square meter (cd/m^{2}) \\
\textbf{Chrominance} & Color information in video signal independent of
brightness & U and V components \\
\end{longtable}
}

\end{solutionbox}
\begin{mnemonicbox}
``BLC: Brightness is Light perception, Luminance is
Calculated light, Chrominance is Color information''

\end{mnemonicbox}
\subsection*{Question 2(b) [4 marks]}\label{q2b}

\textbf{Draw and explain block diagram of DTH receiver.}

\begin{solutionbox}

\textbf{DTH Receiver Block Diagram:}

\begin{center}
\textbf{Mermaid Diagram (Code)}
\begin{verbatim}
{Shaded}
{Highlighting}[]
graph LR
    A[Satellite Dish] {-{-}{} B[LNB]}
    B {-{-}{} C[Tuner]}
    C {-{-}{} D[Demodulator]}
    D {-{-}{} E[MPEG Decoder]}
    E {-{-}{} F[Video Processor]}
    E {-{-}{} G[Audio Processor]}
    F {-{-}{} H[TV Display]}
    G {-{-}{} I[Speakers]}
    J[Smart Card] {-{-}{} K[Conditional Access Module]}
    K {-{-}{} D}
    L[User Interface] {-{-}{} M[Microcontroller]}
    M {-{-}{} C}
    M {-{-}{} E}
{Highlighting}
{Shaded}
\end{verbatim}
\end{center}


{\def\LTcaptype{none} % do not increment counter
\vspace{-5pt}
\captionof{table}{DTH Receiver Components}
\vspace{-10pt}
\begin{longtable}[]{@{}
  >{\raggedright\arraybackslash}p{(\linewidth - 2\tabcolsep) * \real{0.5238}}
  >{\raggedright\arraybackslash}p{(\linewidth - 2\tabcolsep) * \real{0.4762}}@{}}
\toprule\noalign{}
\begin{minipage}[b]{\linewidth}\raggedright
Component
\end{minipage} & \begin{minipage}[b]{\linewidth}\raggedright
Function
\end{minipage} \\
\midrule\noalign{}
\endhead
\bottomrule\noalign{}
\endlastfoot
\textbf{Satellite Dish} & Receives satellite signals from space \\
\textbf{LNB (Low Noise Block)} & Converts high-frequency signals to
lower frequency \\
\textbf{Tuner} & Selects specific channel frequency \\
\textbf{Demodulator} & Extracts digital data from carrier signal \\
\textbf{MPEG Decoder} & Decompresses audio/video data \\
\textbf{Conditional Access Module} & Controls subscription access \\
\end{longtable}
}

\end{solutionbox}
\begin{mnemonicbox}
``SLTDM: Satellite captures, LNB converts, Tuner
selects, Demodulator extracts, MPEG decodes''

\end{mnemonicbox}
\subsection*{Question 2(c) [7 marks]}\label{q2c}

\textbf{Draw and explain block diagram of colour TV receiver.}

\begin{solutionbox}

\textbf{Colour TV Receiver Block Diagram:}

\begin{center}
\textbf{Mermaid Diagram (Code)}
\begin{verbatim}
{Shaded}
{Highlighting}[]
graph LR
    A[Antenna] {-{-}{} B[Tuner]}
    B {-{-}{} C[IF Amplifier]}
    C {-{-}{} D[Video Detector]}
    D {-{-}{} E[Video Amplifier]}
    D {-{-}{} F[Sound IF \& Detector]}
    E {-{-}{} G[Y Signal Processing]}
    E {-{-}{} H[Chrominance Bandpass]}
    H {-{-}{} I[Chroma Demodulator]}
    I {-{-}{} J[R{-}Y Signal]}
    I {-{-}{} K[B{-}Y Signal]}
    G {-{-}{} L[RGB Matrix]}
    J {-{-}{} L}
    K {-{-}{} L}
    L {-{-}{} M[Picture Tube/Display]}
    F {-{-}{} N[Audio Amplifier]}
    N {-{-}{} O[Speaker]}
    P[Power Supply] {-{-}{} B}
    P {-{-}{} C}
    P {-{-}{} E}
    P {-{-}{} H}
    P {-{-}{} N}
{Highlighting}
{Shaded}
\end{verbatim}
\end{center}


{\def\LTcaptype{none} % do not increment counter
\vspace{-5pt}
\captionof{table}{Colour TV Components and Functions}
\vspace{-10pt}
\begin{longtable}[]{@{}
  >{\raggedright\arraybackslash}p{(\linewidth - 4\tabcolsep) * \real{0.2571}}
  >{\raggedright\arraybackslash}p{(\linewidth - 4\tabcolsep) * \real{0.2857}}
  >{\raggedright\arraybackslash}p{(\linewidth - 4\tabcolsep) * \real{0.4571}}@{}}
\toprule\noalign{}
\begin{minipage}[b]{\linewidth}\raggedright
Section
\end{minipage} & \begin{minipage}[b]{\linewidth}\raggedright
Function
\end{minipage} & \begin{minipage}[b]{\linewidth}\raggedright
Key Components
\end{minipage} \\
\midrule\noalign{}
\endhead
\bottomrule\noalign{}
\endlastfoot
\textbf{Tuner} & Selects desired channel & RF amplifier, mixer, local
oscillator \\
\textbf{IF Amplifier} & Amplifies intermediate frequency & Bandpass
filters, amplifiers \\
\textbf{Video Detector} & Extracts video signal & Diode detector,
filters \\
\textbf{Chrominance Section} & Processes color information & Bandpass
filter, color demodulator \\
\textbf{Luminance Section} & Processes brightness information & Y signal
amplifier \\
\textbf{RGB Matrix} & Combines signals for display & Mixing circuits \\
\textbf{Audio Section} & Processes sound & Sound IF, detector,
amplifier \\
\end{longtable}
}

\end{solutionbox}
\begin{mnemonicbox}
``TIVACRL: Tuner tunes, IF amplifies, Video detects,
Audio separates, Chrominance demodulates, RGB mixes, Light displays''

\end{mnemonicbox}
\subsection*{Question 2(a OR) [3
marks]}\label{question-2a-or-3-marks}

\textbf{Write a short note on LED TV.}

\begin{solutionbox}


{\def\LTcaptype{none} % do not increment counter
\vspace{-5pt}
\captionof{table}{LED TV Technology}
\vspace{-10pt}
\begin{longtable}[]{@{}
  >{\raggedright\arraybackslash}p{(\linewidth - 2\tabcolsep) * \real{0.3810}}
  >{\raggedright\arraybackslash}p{(\linewidth - 2\tabcolsep) * \real{0.6190}}@{}}
\toprule\noalign{}
\begin{minipage}[b]{\linewidth}\raggedright
Aspect
\end{minipage} & \begin{minipage}[b]{\linewidth}\raggedright
Description
\end{minipage} \\
\midrule\noalign{}
\endhead
\bottomrule\noalign{}
\endlastfoot
\textbf{Basic Technology} & Uses Light Emitting Diodes for display
backlighting \\
\textbf{Types} & Edge-lit (LEDs at edges), Direct-lit (LEDs behind
screen), Full-array (with local dimming) \\
\textbf{Advantages} & Thinner profile, energy efficient, better contrast
ratio, longer lifespan than LCD \\
\textbf{Display Panel} & Still uses LCD panel; LEDs are only for
backlighting \\
\end{longtable}
}

\end{solutionbox}
\begin{mnemonicbox}
``BEST: Backlighting with LEDs, Energy efficient,
Slim design, True colors''

\end{mnemonicbox}
\subsection*{Question 2(b OR) [4
marks]}\label{question-2b-or-4-marks}

\textbf{Briefly explain the terms: (1) Hue (2) Saturation}

\begin{solutionbox}


{\def\LTcaptype{none} % do not increment counter
\vspace{-5pt}
\captionof{table}{Color Properties}
\vspace{-10pt}
\begin{longtable}[]{@{}
  >{\raggedright\arraybackslash}p{(\linewidth - 6\tabcolsep) * \real{0.1765}}
  >{\raggedright\arraybackslash}p{(\linewidth - 6\tabcolsep) * \real{0.3529}}
  >{\raggedright\arraybackslash}p{(\linewidth - 6\tabcolsep) * \real{0.2059}}
  >{\raggedright\arraybackslash}p{(\linewidth - 6\tabcolsep) * \real{0.2647}}@{}}
\toprule\noalign{}
\begin{minipage}[b]{\linewidth}\raggedright
Term
\end{minipage} & \begin{minipage}[b]{\linewidth}\raggedright
Definition
\end{minipage} & \begin{minipage}[b]{\linewidth}\raggedright
Range
\end{minipage} & \begin{minipage}[b]{\linewidth}\raggedright
Example
\end{minipage} \\
\midrule\noalign{}
\endhead
\bottomrule\noalign{}
\endlastfoot
\textbf{Hue} & Actual color wavelength (red, blue, green, etc.) & 0-360
degrees on color wheel & Red=0^\circ, Green=120^\circ, Blue=240^\circ \\
\textbf{Saturation} & Intensity or purity of color (how vivid) & 0-100\%
(gray to pure color) & 0\%=grayscale, 100\%=vivid color \\
\end{longtable}
}

\begin{center}
\textbf{Mermaid Diagram (Code)}
\begin{verbatim}
{Shaded}
{Highlighting}[]
graph LR
    A[Color Properties] {-{-}{} B[Hue]}
    A {-{-}{} C[Saturation]}
    B {-{-}{} D[Wavelength of color]}
    C {-{-}{} E[Purity/vividness]}
    D {-{-}{} F[Measured in degrees on color wheel]}
    E {-{-}{} G[Measured in percentage]}
{Highlighting}
{Shaded}
\end{verbatim}
\end{center}

\end{solutionbox}
\begin{mnemonicbox}
``HS: Hue is the color Shade, Saturation is the color
Strength''

\end{mnemonicbox}
\subsection*{Question 2(c OR) [7
marks]}\label{question-2c-or-7-marks}

\textbf{Explain additive colour mixing using colour circle diagram and
Grassman's law.}

\begin{solutionbox}


{\def\LTcaptype{none} % do not increment counter
\vspace{-5pt}
\captionof{table}{Additive Color Mixing Principles}
\vspace{-10pt}
\begin{longtable}[]{@{}lll@{}}
\toprule\noalign{}
Color Combination & Result & RGB Value \\
\midrule\noalign{}
\endhead
\bottomrule\noalign{}
\endlastfoot
\textbf{Red + Green} & Yellow & (255,255,0) \\
\textbf{Green + Blue} & Cyan & (0,255,255) \\
\textbf{Blue + Red} & Magenta & (255,0,255) \\
\textbf{Red + Green + Blue} & White & (255,255,255) \\
\textbf{No colors} & Black & (0,0,0) \\
\end{longtable}
}

\textbf{Grassman's Laws:}

\begin{itemize}
\tightlist
\item
  \textbf{Law 1}: Any color can be created by mixing three primary
  colors
\item
  \textbf{Law 2}: The appearance of a color depends only on its
  tristimulus values
\item
  \textbf{Law 3}: In additive mixing, the tristimulus values add
  together
\end{itemize}

\begin{center}
\textbf{Mermaid Diagram (Code)}
\begin{verbatim}
{Shaded}
{Highlighting}[]
graph LR
    A[Additive Color Mixing] {-{-}{} B[Primary Colors]}
    B {-{-}{} C[Red]}
    B {-{-}{} D[Green]}
    B {-{-}{} E[Blue]}
    C {-{-}{} F[Red + Green = Yellow]}
    D {-{-}{} F}
    D {-{-}{} G[Green + Blue = Cyan]}
    E {-{-}{} G}
    E {-{-}{} H[Blue + Red = Magenta]}
    C {-{-}{} H}
    C {-{-}{} I[Red + Green + Blue = White]}
    D {-{-}{} I}
    E {-{-}{} I}
{Highlighting}
{Shaded}
\end{verbatim}
\end{center}

\textbf{Color Circle Diagram:}

\begin{verbatim}
    Yellow
      /{}
     /  {}
    /    {}
Red {-{-}{-}{-}{-}{-}Green}
    {    /}
     {  /}
      {/}
   Magenta{-{-}{-}{-}Cyan}
       {    /}
        {  /}
         {/}
        Blue
\end{verbatim}

\end{solutionbox}
\begin{mnemonicbox}
``RGB-CMY-W: Red, Green, Blue make Cyan, Magenta,
Yellow, and White''

\end{mnemonicbox}
\subsection*{Question 3(a) [3 marks]}\label{q3a}

\textbf{List wiring and safety instructions for microwave oven.}

\begin{solutionbox}


{\def\LTcaptype{none} % do not increment counter
\vspace{-5pt}
\captionof{table}{Microwave Oven Wiring and Safety Instructions}
\vspace{-10pt}
\begin{longtable}[]{@{}
  >{\raggedright\arraybackslash}p{(\linewidth - 2\tabcolsep) * \real{0.4167}}
  >{\raggedright\arraybackslash}p{(\linewidth - 2\tabcolsep) * \real{0.5833}}@{}}
\toprule\noalign{}
\begin{minipage}[b]{\linewidth}\raggedright
Category
\end{minipage} & \begin{minipage}[b]{\linewidth}\raggedright
Instructions
\end{minipage} \\
\midrule\noalign{}
\endhead
\bottomrule\noalign{}
\endlastfoot
\textbf{Wiring} & Use grounded outlet with dedicated 15-20A circuit \\
\textbf{Power} & Ensure voltage matches rating (typically 220-240V) \\
\textbf{Installation} & Allow 5cm clearance on all sides for
ventilation \\
\textbf{Safety} & Never operate empty, never bypass door interlocks \\
\textbf{Maintenance} & Disconnect power before servicing, discharge
capacitor \\
\end{longtable}
}

\end{solutionbox}
\begin{mnemonicbox}
``POWER: Proper Outlet, Wiring check, Empty operation
avoided, Repairs by professionals''

\end{mnemonicbox}
\subsection*{Question 3(b) [4 marks]}\label{q3b}

\textbf{Explain working of Air conditioner.}

\begin{solutionbox}


{\def\LTcaptype{none} % do not increment counter
\vspace{-5pt}
\captionof{table}{Air Conditioner Working Cycle}
\vspace{-10pt}
\begin{longtable}[]{@{}
  >{\raggedright\arraybackslash}p{(\linewidth - 4\tabcolsep) * \real{0.3667}}
  >{\raggedright\arraybackslash}p{(\linewidth - 4\tabcolsep) * \real{0.3333}}
  >{\raggedright\arraybackslash}p{(\linewidth - 4\tabcolsep) * \real{0.3000}}@{}}
\toprule\noalign{}
\begin{minipage}[b]{\linewidth}\raggedright
Component
\end{minipage} & \begin{minipage}[b]{\linewidth}\raggedright
Function
\end{minipage} & \begin{minipage}[b]{\linewidth}\raggedright
Process
\end{minipage} \\
\midrule\noalign{}
\endhead
\bottomrule\noalign{}
\endlastfoot
\textbf{Compressor} & Pressurizes refrigerant & Converts low-pressure
gas to high-pressure gas \\
\textbf{Condenser} & Releases heat outside & Converts gas to liquid,
expels heat \\
\textbf{Expansion Valve} & Regulates refrigerant flow & Reduces pressure
of liquid \\
\textbf{Evaporator} & Absorbs heat from room & Converts liquid to gas,
cools air \\
\textbf{Thermostat} & Controls temperature & Regulates compressor
operation \\
\end{longtable}
}

\begin{verbatim}
flowchart LR
    A[Compressor] {-{-}|High{-}pressure gas| B[Condenser]}
    B {-{-}|Liquid| C[Expansion Valve]}
    C {-{-}|Low{-}pressure liquid| D[Evaporator]}
    D {-{-}|Low{-}pressure gas| A}
    E[Room Air] {-{-} D}
    D {-{-} F[Cool Air]}
    G[Outside Air] {-{-} B}
    B {-{-} H[Hot Air]}
\end{verbatim}

\end{solutionbox}
\begin{mnemonicbox}
``CELT: Compress gas, Expel heat, Lower pressure,
Take in heat''

\end{mnemonicbox}
\subsection*{Question 3(c) [7 marks]}\label{q3c}

\textbf{Explain electronic controller for washing machine and fuzzy
logic washing machine. Also list technical specifications of washing
machine.}

\begin{solutionbox}


{\def\LTcaptype{none} % do not increment counter
\vspace{-5pt}
\captionof{table}{Electronic Controller in Washing Machines}
\vspace{-10pt}
\begin{longtable}[]{@{}
  >{\raggedright\arraybackslash}p{(\linewidth - 2\tabcolsep) * \real{0.5238}}
  >{\raggedright\arraybackslash}p{(\linewidth - 2\tabcolsep) * \real{0.4762}}@{}}
\toprule\noalign{}
\begin{minipage}[b]{\linewidth}\raggedright
Component
\end{minipage} & \begin{minipage}[b]{\linewidth}\raggedright
Function
\end{minipage} \\
\midrule\noalign{}
\endhead
\bottomrule\noalign{}
\endlastfoot
\textbf{Microcontroller} & Central processing unit controlling all
operations \\
\textbf{Sensors} & Detect water level, temperature, load balance, door
status \\
\textbf{Input Interface} & Buttons/touch panel for program selection \\
\textbf{Display} & Shows program status, time remaining, error codes \\
\textbf{Actuator Drivers} & Control motor, valves, heater, pump \\
\end{longtable}
}

\textbf{Fuzzy Logic in Washing Machines:}

\begin{itemize}
\tightlist
\item
  Uses artificial intelligence for optimal washing
\item
  Adjusts water level, wash time, and spin speed based on load
\item
  Makes decisions using approximate reasoning instead of precise values
\item
  Adapts to different fabric types and soil levels automatically
\end{itemize}

\textbf{Technical Specifications:}

\begin{itemize}
\tightlist
\item
  \textbf{Capacity}: 6-10 kg (front load), 5-8 kg (top load)
\item
  \textbf{Energy Rating}: A+++ to B (EU standards)
\item
  \textbf{Water Consumption}: 40-70 liters per cycle
\item
  \textbf{Spin Speed}: 800-1600 RPM
\item
  \textbf{Cycle Options}: 8-16 programs
\end{itemize}

\begin{center}
\textbf{Mermaid Diagram (Code)}
\begin{verbatim}
{Shaded}
{Highlighting}[]
graph TD
    A[Electronic Controller] {-{-}{} B[Microcontroller]}
    B {-{-}{} C[Sensor Inputs]}
    B {-{-}{} D[User Interface]}
    B {-{-}{} E[Actuator Control]}
    C {-{-}{} F[Water Level Sensor]}
    C {-{-}{} G[Temperature Sensor]}
    C {-{-}{} H[Load Balance Sensor]}
    C {-{-}{} I[Door Lock Sensor]}
    E {-{-}{} J[Motor Driver]}
    E {-{-}{} K[Water Valve Control]}
    E {-{-}{} L[Drain Pump Control]}
    E {-{-}{} M[Heater Control]}
    N[Fuzzy Logic] {-{-}{} B}
    N {-{-}{} O[Adaptive Control]}
    O {-{-}{} P[Water Level Adjustment]}
    O {-{-}{} Q[Wash Time Optimization]}
    O {-{-}{} R[Spin Speed Adjustment]}
{Highlighting}
{Shaded}
\end{verbatim}
\end{center}

\end{solutionbox}
\begin{mnemonicbox}
``SCRAM: Sensors detect, Controller processes, Rules
applied, Actuators operate, Machine adapts''

\end{mnemonicbox}
\subsection*{Question 3(a OR) [3
marks]}\label{question-3a-or-3-marks}

\textbf{State main components of solar power system and specifications
of solar power system.}

\begin{solutionbox}


{\def\LTcaptype{none} % do not increment counter
\vspace{-5pt}
\captionof{table}{Solar Power System Components}
\vspace{-10pt}
\begin{longtable}[]{@{}ll@{}}
\toprule\noalign{}
Component & Function \\
\midrule\noalign{}
\endhead
\bottomrule\noalign{}
\endlastfoot
\textbf{Solar Panels} & Convert sunlight to DC electricity \\
\textbf{Inverter} & Converts DC to AC power \\
\textbf{Battery Bank} & Stores energy for later use \\
\textbf{Charge Controller} & Prevents battery overcharging \\
\textbf{Mounting Structure} & Supports and angles panels optimally \\
\end{longtable}
}

\textbf{Specifications:}

\begin{itemize}
\tightlist
\item
  \textbf{Panel Capacity}: 250-400 Watts per panel
\item
  \textbf{System Size}: 1-10 kW (residential)
\item
  \textbf{Battery Capacity}: 100-200 Ah
\item
  \textbf{Inverter Efficiency}: 90-97\%
\item
  \textbf{Expected Lifespan}: 25-30 years (panels)
\end{itemize}

\end{solutionbox}
\begin{mnemonicbox}
``PIBCM: Panels collect, Inverter converts, Batteries
store, Controller protects, Mounts support''

\end{mnemonicbox}
\subsection*{Question 3(b OR) [4
marks]}\label{question-3b-or-4-marks}

\textbf{Explain working of Refrigerator.}

\begin{solutionbox}


{\def\LTcaptype{none} % do not increment counter
\vspace{-5pt}
\captionof{table}{Refrigerator Working Cycle}
\vspace{-10pt}
\begin{longtable}[]{@{}
  >{\raggedright\arraybackslash}p{(\linewidth - 6\tabcolsep) * \real{0.1429}}
  >{\raggedright\arraybackslash}p{(\linewidth - 6\tabcolsep) * \real{0.1837}}
  >{\raggedright\arraybackslash}p{(\linewidth - 6\tabcolsep) * \real{0.2245}}
  >{\raggedright\arraybackslash}p{(\linewidth - 6\tabcolsep) * \real{0.4490}}@{}}
\toprule\noalign{}
\begin{minipage}[b]{\linewidth}\raggedright
Stage
\end{minipage} & \begin{minipage}[b]{\linewidth}\raggedright
Process
\end{minipage} & \begin{minipage}[b]{\linewidth}\raggedright
Component
\end{minipage} & \begin{minipage}[b]{\linewidth}\raggedright
State of Refrigerant
\end{minipage} \\
\midrule\noalign{}
\endhead
\bottomrule\noalign{}
\endlastfoot
1 & Compression & Compressor & Low pressure gas \rightarrow High pressure gas \\
2 & Condensation & Condenser coils & High pressure gas \rightarrow High pressure
liquid \\
3 & Expansion & Expansion valve & High pressure liquid \rightarrow Low pressure
liquid \\
4 & Evaporation & Evaporator coils & Low pressure liquid \rightarrow Low pressure
gas \\
\end{longtable}
}

\begin{verbatim}
flowchart LR
    A[Compressor] {-{-}|High{-}pressure gas| B[Condenser]}
    B {-{-}|Heat released outside| C[High{-}pressure liquid]}
    C {-{-} D[Expansion Valve]}
    D {-{-}|Sudden pressure drop| E[Low{-}pressure liquid]}
    E {-{-} F[Evaporator]}
    F {-{-}|Heat absorbed from inside| G[Low{-}pressure gas]}
    G {-{-} A}
    H[Thermostat] {-{-} A}
\end{verbatim}

\end{solutionbox}
\begin{mnemonicbox}
``CEHE: Compress gas, Expel heat, Halve pressure,
Extract heat''

\end{mnemonicbox}
\subsection*{Question 3(c OR) [7
marks]}\label{question-3c-or-7-marks}

\textbf{Draw and explain block diagram of Microwave oven. List types,
applications and technical specifications of microwave oven.}

\begin{solutionbox}

\textbf{Microwave Oven Block Diagram:}

\begin{center}
\textbf{Mermaid Diagram (Code)}
\begin{verbatim}
{Shaded}
{Highlighting}[]
graph LR
    A[Power Supply] {-{-}{} B[Control Panel/Timer]}
    A {-{-}{} C[High Voltage Transformer]}
    B {-{-}{} D[Door Interlock Switches]}
    B {-{-}{} E[Control Circuit/Microcontroller]}
    E {-{-}{} F[Magnetron Driver]}
    C {-{-}{} F}
    F {-{-}{} G[Magnetron]}
    G {-{-}{} H[Waveguide]}
    H {-{-}{} I[Cooking Cavity]}
    E {-{-}{} J[Turntable Motor]}
    J {-{-}{} K[Turntable]}
    E {-{-}{} L[Cooling Fan]}
{Highlighting}
{Shaded}
\end{verbatim}
\end{center}

\textbf{Types of Microwave Ovens:}

\begin{itemize}
\tightlist
\item
  \textbf{Solo}: Basic heating and defrosting only
\item
  \textbf{Grill}: Has additional grilling element
\item
  \textbf{Convection}: Combines microwave with convection heating
\item
  \textbf{Over-the-Range (OTR)}: Includes ventilation system
\item
  \textbf{Built-in}: Designed for cabinet installation
\end{itemize}

\textbf{Applications:}

\begin{itemize}
\tightlist
\item
  \textbf{Cooking}: Fast meal preparation
\item
  \textbf{Reheating}: Leftover foods
\item
  \textbf{Defrosting}: Frozen foods
\item
  \textbf{Sterilization}: Small items
\item
  \textbf{Commercial}: Food service industry
\end{itemize}

\textbf{Technical Specifications:}

\begin{itemize}
\tightlist
\item
  \textbf{Capacity}: 20-40 liters
\item
  \textbf{Power Output}: 700-1200 watts
\item
  \textbf{Power Consumption}: 1100-1500 watts
\item
  \textbf{Frequency}: 2.45 GHz
\item
  \textbf{Voltage}: 220-240V AC
\end{itemize}

\end{solutionbox}
\begin{mnemonicbox}
``MICROWAVES: Magnetron generates, Interior receives,
Control regulates, Rotating turntable, Oven cavity, Waveguide directs,
Alternating current powers, Ventilation cools, Electronic timer, Safety
interlocks''

\end{mnemonicbox}
\subsection*{Question 4(a) [3 marks]}\label{q4a}

\textbf{List specifications of MF printer and LCD projector.}

\begin{solutionbox}


{\def\LTcaptype{none} % do not increment counter
\vspace{-5pt}
\captionof{table}{Multi-Function Printer Specifications}
\vspace{-10pt}
\begin{longtable}[]{@{}ll@{}}
\toprule\noalign{}
Specification & Typical Range \\
\midrule\noalign{}
\endhead
\bottomrule\noalign{}
\endlastfoot
\textbf{Print Resolution} & 600-4800 dpi \\
\textbf{Print Speed} & 20-40 ppm (black), 15-30 ppm (color) \\
\textbf{Scan Resolution} & 600-1200 dpi \\
\textbf{Connectivity} & Wi-Fi, Ethernet, USB, Cloud \\
\textbf{Paper Capacity} & 100-500 sheets \\
\end{longtable}
}


{\def\LTcaptype{none} % do not increment counter
\vspace{-5pt}
\captionof{table}{LCD Projector Specifications}
\vspace{-10pt}
\begin{longtable}[]{@{}ll@{}}
\toprule\noalign{}
Specification & Typical Range \\
\midrule\noalign{}
\endhead
\bottomrule\noalign{}
\endlastfoot
\textbf{Brightness} & 2000-5000 lumens \\
\textbf{Resolution} & XGA (1024\times768) to 4K (3840\times2160) \\
\textbf{Contrast Ratio} & 2000:1 to 100,000:1 \\
\textbf{Lamp Life} & 4000-8000 hours \\
\textbf{Connectivity} & HDMI, VGA, USB, Wireless \\
\end{longtable}
}

\end{solutionbox}
\begin{mnemonicbox}
``PSCPL: Print resolution, Speed, Connectivity,
Projection brightness, Lamp life''

\end{mnemonicbox}
\subsection*{Question 4(b) [4 marks]}\label{q4b}

\textbf{Draw block diagram of Inkjet printer and explain its working in
brief.}

\begin{solutionbox}

\textbf{Inkjet Printer Block Diagram:}

\begin{center}
\textbf{Mermaid Diagram (Code)}
\begin{verbatim}
{Shaded}
{Highlighting}[]
graph TD
    A[Power Supply] {-{-}{} B[Control Board/CPU]}
    B {-{-}{} C[Paper Feed Motor]}
    B {-{-}{} D[Printhead Motor]}
    B {-{-}{} E[Printhead Controller]}
    E {-{-}{} F[Ink Cartridges]}
    F {-{-}{} G[Printhead Nozzles]}
    B {-{-}{} H[Input Interface]}
    I[Computer] {-{-}{} H}
    C {-{-}{} J[Paper Feed Mechanism]}
    D {-{-}{} K[Carriage Assembly]}
    K {-{-}{} F}
    B {-{-}{} L[Sensors]}
    L {-{-}{} M[Paper Sensors]}
    L {-{-}{} N[Ink Level Sensors]}
{Highlighting}
{Shaded}
\end{verbatim}
\end{center}

\textbf{Working of Inkjet Printer:}

\begin{enumerate}
\tightlist
\item
  \textbf{Document Processing}: Control board receives data and converts
  to printer commands
\item
  \textbf{Paper Loading}: Feed motor pulls paper from tray
\item
  \textbf{Printing}: Printhead moves across paper while ejecting tiny
  ink droplets
\item
  \textbf{Droplet Formation}: Thermal or piezoelectric method forces ink
  droplets onto paper
\item
  \textbf{Paper Advancement}: Paper advances line by line until printing
  completes
\end{enumerate}

\end{solutionbox}
\begin{mnemonicbox}
``PIPES: Paper feeds, Ink ejects, Printhead moves,
Electronic control, Sheet advances''

\end{mnemonicbox}
\subsection*{Question 4(c) [7 marks]}\label{q4c}

\textbf{Explain working of Photocopier with block diagram and list its
specifications.}

\begin{solutionbox}

\textbf{Photocopier Block Diagram:}

\begin{center}
\textbf{Mermaid Diagram (Code)}
\begin{verbatim}
{Shaded}
{Highlighting}[]
graph TD
    A[Control Panel] {-{-}{} B[Main Control Board]}
    B {-{-}{} C[Scanning System]}
    C {-{-}{} D[Light Source]}
    C {-{-}{} E[Mirrors and Lenses]}
    C {-{-}{} F[CCD/Image Sensor]}
    B {-{-}{} G[Imaging System]}
    G {-{-}{} H[Photosensitive Drum]}
    G {-{-}{} I[Charging Corona]}
    G {-{-}{} J[Developing Unit]}
    G {-{-}{} K[Transfer Corona]}
    G {-{-}{} L[Fusing Unit]}
    B {-{-}{} M[Paper Feed System]}
    M {-{-}{} N[Paper Trays]}
    M {-{-}{} O[Feed Rollers]}
    M {-{-}{} P[Registration Rollers]}
    B {-{-}{} Q[Power Supply]}
{Highlighting}
{Shaded}
\end{verbatim}
\end{center}

\textbf{Working of Photocopier:}

\begin{enumerate}
\tightlist
\item
  \textbf{Charging}: Photosensitive drum receives uniform electrostatic
  charge
\item
  \textbf{Exposure}: Original document scanned, creating light pattern
  on drum
\item
  \textbf{Developing}: Toner particles attracted to charged areas on
  drum
\item
  \textbf{Transfer}: Toner image transferred from drum to paper
\item
  \textbf{Fusing}: Heat and pressure melt toner permanently onto paper
\item
  \textbf{Cleaning}: Drum cleaned for next cycle
\end{enumerate}

\textbf{Technical Specifications:}

\begin{itemize}
\tightlist
\item
  \textbf{Speed}: 20-60 pages per minute
\item
  \textbf{Resolution}: 600-1200 dpi
\item
  \textbf{Paper Capacity}: 250-2000 sheets
\item
  \textbf{Maximum Paper Size}: A3/11\times17 inches
\item
  \textbf{Zoom Range}: 25-400\%
\item
  \textbf{Memory}: 512MB-2GB
\item
  \textbf{Connectivity}: Ethernet, USB, Wi-Fi
\end{itemize}

\end{solutionbox}
\begin{mnemonicbox}
``CETFC: Charge drum, Expose image, Transfer toner,
Fuse permanently, Clean drum''

\end{mnemonicbox}
\subsection*{Question 4(a OR) [3
marks]}\label{question-4a-or-3-marks}

\textbf{Write a short note on CCTV.}

\begin{solutionbox}


{\def\LTcaptype{none} % do not increment counter
\vspace{-5pt}
\captionof{table}{CCTV System Overview}
\vspace{-10pt}
\begin{longtable}[]{@{}ll@{}}
\toprule\noalign{}
Aspect & Description \\
\midrule\noalign{}
\endhead
\bottomrule\noalign{}
\endlastfoot
\textbf{Full Form} & Closed-Circuit Television \\
\textbf{Purpose} & Security monitoring and surveillance \\
\textbf{Components} & Cameras, DVR/NVR, monitors, cables, power
supply \\
\textbf{Types} & Analog, IP (digital), Wireless, HD-CVI/TVI/SDI \\
\textbf{Features} & Motion detection, night vision, remote viewing \\
\end{longtable}
}

\textbf{Key Applications:}

\begin{itemize}
\tightlist
\item
  Security monitoring of buildings
\item
  Traffic monitoring
\item
  Retail loss prevention
\item
  Public area surveillance
\item
  Home security
\end{itemize}

\end{solutionbox}
\begin{mnemonicbox}
``SCRAM: Security monitoring, Closed circuit,
Recording footage, Access restricted, Monitoring continuous''

\end{mnemonicbox}
\subsection*{Question 4(b OR) [4
marks]}\label{question-4b-or-4-marks}

\textbf{Explain working of LCD projector with block diagram.}

\begin{solutionbox}

\textbf{LCD Projector Block Diagram:}

\begin{center}
\textbf{Mermaid Diagram (Code)}
\begin{verbatim}
{Shaded}
{Highlighting}[]
graph LR
    A[Power Supply] {-{-}{} B[Control Circuit]}
    B {-{-}{} C[Lamp/Light Source]}
    C {-{-}{} D[Cooling System]}
    C {-{-}{} E[Reflector]}
    E {-{-}{} F[Condenser Lens]}
    F {-{-}{} G[Dichroic Mirrors]}
    G {-{-}{}|Red| H[Red LCD Panel]}
    G {-{-}{}|Green| I[Green LCD Panel]}
    G {-{-}{}|Blue| J[Blue LCD Panel]}
    H {-{-}{} K[Combining Prism]}
    I {-{-}{} K}
    J {-{-}{} K}
    K {-{-}{} L[Projection Lens]}
    L {-{-}{} M[Screen]}
    B {-{-}{} N[Input Interfaces]}
    B {-{-}{} O[Keystone Correction]}
    B {-{-}{} P[Focus Control]}
{Highlighting}
{Shaded}
\end{verbatim}
\end{center}

\textbf{Working of LCD Projector:}

\begin{enumerate}
\tightlist
\item
  \textbf{Light Generation}: High-intensity lamp produces white light
\item
  \textbf{Color Separation}: Dichroic mirrors split light into RGB
  components
\item
  \textbf{Image Formation}: LCD panels modulate light based on input
  signal
\item
  \textbf{Recombination}: Prism combines RGB images into full-color
  image
\item
  \textbf{Projection}: Lens system projects final image onto screen
\end{enumerate}

\end{solutionbox}
\begin{mnemonicbox}
``LSCIP: Light source generates, Split into colors,
Control with LCDs, Image combined, Projected on screen''

\end{mnemonicbox}
\subsection*{Question 4(c OR) [7
marks]}\label{question-4c-or-7-marks}

\textbf{Explain working of laser printer with block diagram.}

\begin{solutionbox}

\textbf{Laser Printer Block Diagram:}

\begin{center}
\textbf{Mermaid Diagram (Code)}
\begin{verbatim}
{Shaded}
{Highlighting}[]
graph TD
    A[Control Board] {-{-}{} B[Laser Diode]}
    A {-{-}{} C[Polygon Mirror Motor]}
    B {-{-}{} D[Polygon Mirror]}
    D {-{-}{} E[Focusing Lenses]}
    E {-{-}{} F[Photosensitive Drum]}
    A {-{-}{} G[Primary Corona]}
    G {-{-}{} F}
    A {-{-}{} H[Developer Unit]}
    H {-{-}{} F}
    A {-{-}{} I[Transfer Corona]}
    I {-{-}{} F}
    A {-{-}{} J[Fusing Unit]}
    A {-{-}{} K[Paper Feed Mechanism]}
    K {-{-}{} L[Paper Path]}
    L {-{-}{} J}
    A {-{-}{} M[Power Supply]}
    A {-{-}{} N[Interface]}
{Highlighting}
{Shaded}
\end{verbatim}
\end{center}

\textbf{Laser Printing Process:}


{\def\LTcaptype{none} % do not increment counter
\vspace{-5pt}
\captionof{table}{Six Steps of Laser Printing}
\vspace{-10pt}
\begin{longtable}[]{@{}
  >{\raggedright\arraybackslash}p{(\linewidth - 6\tabcolsep) * \real{0.1667}}
  >{\raggedright\arraybackslash}p{(\linewidth - 6\tabcolsep) * \real{0.2500}}
  >{\raggedright\arraybackslash}p{(\linewidth - 6\tabcolsep) * \real{0.3056}}
  >{\raggedright\arraybackslash}p{(\linewidth - 6\tabcolsep) * \real{0.2778}}@{}}
\toprule\noalign{}
\begin{minipage}[b]{\linewidth}\raggedright
Step
\end{minipage} & \begin{minipage}[b]{\linewidth}\raggedright
Process
\end{minipage} & \begin{minipage}[b]{\linewidth}\raggedright
Component
\end{minipage} & \begin{minipage}[b]{\linewidth}\raggedright
Function
\end{minipage} \\
\midrule\noalign{}
\endhead
\bottomrule\noalign{}
\endlastfoot
1 & \textbf{Cleaning} & Cleaning blade & Removes residual toner from
drum \\
2 & \textbf{Charging} & Primary corona & Applies uniform negative charge
to drum \\
3 & \textbf{Writing} & Laser \& mirror & Creates electrostatic image on
drum \\
4 & \textbf{Developing} & Developer unit & Applies toner to charged
areas of drum \\
5 & \textbf{Transferring} & Transfer corona & Moves toner from drum to
paper \\
6 & \textbf{Fusing} & Fuser unit & Melts toner permanently onto paper \\
\end{longtable}
}

\textbf{Technical Specifications:}

\begin{itemize}
\tightlist
\item
  \textbf{Print Speed}: 20-50 ppm
\item
  \textbf{Resolution}: 600-2400 dpi
\item
  \textbf{Memory}: 128MB-1GB
\item
  \textbf{Duty Cycle}: 10,000-150,000 pages/month
\item
  \textbf{Connectivity}: USB, Ethernet, Wi-Fi
\end{itemize}

\end{solutionbox}
\begin{mnemonicbox}
``CCWDTF: Clean drum, Charge uniformly, Write with
laser, Develop with toner, Transfer to paper, Fuse permanently''

\end{mnemonicbox}
\subsection*{Question 5(a) [3 marks]}\label{q5a}

\textbf{Define: (1) Pitch (2) Reverberation (3) Microphone.}

\begin{solutionbox}


{\def\LTcaptype{none} % do not increment counter
\vspace{-5pt}
\captionof{table}{Audio Terminology}
\vspace{-10pt}
\begin{longtable}[]{@{}
  >{\raggedright\arraybackslash}p{(\linewidth - 4\tabcolsep) * \real{0.1935}}
  >{\raggedright\arraybackslash}p{(\linewidth - 4\tabcolsep) * \real{0.3871}}
  >{\raggedright\arraybackslash}p{(\linewidth - 4\tabcolsep) * \real{0.4194}}@{}}
\toprule\noalign{}
\begin{minipage}[b]{\linewidth}\raggedright
Term
\end{minipage} & \begin{minipage}[b]{\linewidth}\raggedright
Definition
\end{minipage} & \begin{minipage}[b]{\linewidth}\raggedright
Measured In
\end{minipage} \\
\midrule\noalign{}
\endhead
\bottomrule\noalign{}
\endlastfoot
\textbf{Pitch} & Perceived frequency of sound; how high or low a tone
seems & Hertz (Hz) \\
\textbf{Reverberation} & Persistence of sound after source stops; caused
by reflections & Seconds (RT60) \\
\textbf{Microphone} & Transducer that converts sound waves into
electrical signals & Sensitivity in dB/mV/Pa \\
\end{longtable}
}

\end{solutionbox}
\begin{mnemonicbox}
``PRM: Pitch is frequency, Reverberation is
reflection, Microphone is converter''

\end{mnemonicbox}
\subsection*{Question 5(b) [4 marks]}\label{q5b}

\textbf{Draw and explain block diagram of PA system.}

\begin{solutionbox}

\textbf{PA System Block Diagram:}

\begin{center}
\textbf{Mermaid Diagram (Code)}
\begin{verbatim}
{Shaded}
{Highlighting}[]
graph LR
    A[Microphone] {-{-}{} B[Pre{-}amplifier]}
    B {-{-}{} C[Mixer]}
    D[Audio Source] {-{-}{} C}
    E[Equalizer] {-{-}{} C}
    C {-{-}{} F[Power Amplifier]}
    F {-{-}{} G[Speaker System]}
    H[Control System] {-{-}{} C}
    H {-{-}{} F}
{Highlighting}
{Shaded}
\end{verbatim}
\end{center}


{\def\LTcaptype{none} % do not increment counter
\vspace{-5pt}
\captionof{table}{PA System Components}
\vspace{-10pt}
\begin{longtable}[]{@{}ll@{}}
\toprule\noalign{}
Component & Function \\
\midrule\noalign{}
\endhead
\bottomrule\noalign{}
\endlastfoot
\textbf{Microphone} & Captures sound and converts to electrical
signals \\
\textbf{Pre-amplifier} & Boosts weak microphone signals to line level \\
\textbf{Mixer} & Combines multiple audio sources, adjusts levels \\
\textbf{Equalizer} & Adjusts frequency response for optimal sound \\
\textbf{Power Amplifier} & Increases signal strength to drive
speakers \\
\textbf{Speaker System} & Converts electrical signals back to sound
waves \\
\end{longtable}
}

\end{solutionbox}
\begin{mnemonicbox}
``MPMEPA: Microphone Picks, Preamp Magnifies,
Equalizer adjusts, Power Amplifier drives, Audience hears''

\end{mnemonicbox}
\subsection*{Question 5(c) [7 marks]}\label{q5c}

\textbf{Explain Crystal microphone.}

\begin{solutionbox}


{\def\LTcaptype{none} % do not increment counter
\vspace{-5pt}
\captionof{table}{Crystal Microphone Characteristics}
\vspace{-10pt}
\begin{longtable}[]{@{}
  >{\raggedright\arraybackslash}p{(\linewidth - 2\tabcolsep) * \real{0.5517}}
  >{\raggedright\arraybackslash}p{(\linewidth - 2\tabcolsep) * \real{0.4483}}@{}}
\toprule\noalign{}
\begin{minipage}[b]{\linewidth}\raggedright
Characteristic
\end{minipage} & \begin{minipage}[b]{\linewidth}\raggedright
Description
\end{minipage} \\
\midrule\noalign{}
\endhead
\bottomrule\noalign{}
\endlastfoot
\textbf{Operating Principle} & Piezoelectric effect \\
\textbf{Construction} & Crystal element (Rochelle salt) between metal
plates \\
\textbf{Response} & High output, moderate frequency response \\
\textbf{Impedance} & Very high (typically \textgreater{} 1 MΩ) \\
\textbf{Durability} & Sensitive to heat and humidity \\
\end{longtable}
}

\textbf{Working Principle:} When sound waves strike the diaphragm, they
create pressure on the crystal element. Due to the piezoelectric effect,
the crystal generates a voltage proportional to the mechanical stress.
This voltage is the electrical representation of the sound.

\begin{center}
\textbf{Mermaid Diagram (Code)}
\begin{verbatim}
{Shaded}
{Highlighting}[]
graph LR
    A[Sound Waves] {-{-}{} B[Diaphragm]}
    B {-{-}{} C[Mechanical Stress on Crystal]}
    C {-{-}{} D[Piezoelectric Effect]}
    D {-{-}{} E[Voltage Generation]}
    E {-{-}{} F[Electrical Output]}
{Highlighting}
{Shaded}
\end{verbatim}
\end{center}

\textbf{Applications:}

\begin{itemize}
\tightlist
\item
  Telephone receivers
\item
  Contact pickups for acoustic instruments
\item
  Low-cost recording devices
\item
  Public address systems
\end{itemize}

\textbf{Advantages and Limitations:}

{\def\LTcaptype{none} % do not increment counter
\begin{longtable}[]{@{}ll@{}}
\toprule\noalign{}
Advantages & Limitations \\
\midrule\noalign{}
\endhead
\bottomrule\noalign{}
\endlastfoot
High output voltage & Poor frequency response \\
No external power needed & Sensitive to temperature/humidity \\
Simple construction & Higher distortion \\
Low cost & Fragile crystal element \\
\end{longtable}
}

\end{solutionbox}
\begin{mnemonicbox}
``PIES: Pressure applied, Impedance high, Electricity
generated, Sound converted''

\end{mnemonicbox}
\subsection*{Question 5(a OR) [3
marks]}\label{question-5a-or-3-marks}

\textbf{Draw block diagram of Home theatre sound system.}

\begin{solutionbox}

\textbf{Home Theatre Sound System Block Diagram:}

\begin{center}
\textbf{Mermaid Diagram (Code)}
\begin{verbatim}
{Shaded}
{Highlighting}[]
graph TD
    A[Audio/Video Source] {-{-}{} B[AV Receiver/Amplifier]}
    B {-{-}{} C[Front Left Speaker]}
    B {-{-}{} D[Center Speaker]}
    B {-{-}{} E[Front Right Speaker]}
    B {-{-}{} F[Surround Left Speaker]}
    B {-{-}{} G[Surround Right Speaker]}
    B {-{-}{} H[Subwoofer]}
    I[Remote Control] {-{-}{} B}
    J[TV/Display] {-{-}{} B}
    B {-{-}{} J}
    K[Streaming Module] {-{-}{} B}
{Highlighting}
{Shaded}
\end{verbatim}
\end{center}

\end{solutionbox}
\begin{mnemonicbox}
``SAVS: Source provides, Amplifier processes, Various
speakers deliver, Surround experience created''

\end{mnemonicbox}
\subsection*{Question 5(b OR) [4
marks]}\label{question-5b-or-4-marks}

\textbf{Explain optical sound recording.}

\begin{solutionbox}


{\def\LTcaptype{none} % do not increment counter
\vspace{-5pt}
\captionof{table}{Optical Sound Recording Process}
\vspace{-10pt}
\begin{longtable}[]{@{}
  >{\raggedright\arraybackslash}p{(\linewidth - 4\tabcolsep) * \real{0.2308}}
  >{\raggedright\arraybackslash}p{(\linewidth - 4\tabcolsep) * \real{0.3462}}
  >{\raggedright\arraybackslash}p{(\linewidth - 4\tabcolsep) * \real{0.4231}}@{}}
\toprule\noalign{}
\begin{minipage}[b]{\linewidth}\raggedright
Step
\end{minipage} & \begin{minipage}[b]{\linewidth}\raggedright
Process
\end{minipage} & \begin{minipage}[b]{\linewidth}\raggedright
Component
\end{minipage} \\
\midrule\noalign{}
\endhead
\bottomrule\noalign{}
\endlastfoot
1 & \textbf{Sound Capture} & Microphone converts sound to electrical
signals \\
2 & \textbf{Modulation} & Signal modulates light source intensity or
area \\
3 & \textbf{Exposure} & Modulated light exposes photographic film \\
4 & \textbf{Development} & Film processed to create visible sound
track \\
5 & \textbf{Playback} & Light passes through track, photodetector
converts to electrical signal \\
\end{longtable}
}

\textbf{Types of Optical Sound Tracks:}

\begin{itemize}
\tightlist
\item
  \textbf{Variable Density}: Light intensity varies (darker/lighter
  areas)
\item
  \textbf{Variable Area}: Transparent area width varies against opaque
  background
\end{itemize}

\begin{center}
\textbf{Mermaid Diagram (Code)}
\begin{verbatim}
{Shaded}
{Highlighting}[]
graph LR
    A[Sound Input] {-{-}{} B[Microphone]}
    B {-{-}{} C[Amplifier]}
    C {-{-}{} D[Light Modulator]}
    E[Light Source] {-{-}{} D}
    D {-{-}{} F[Optical System]}
    F {-{-}{} G[Moving Film]}
    H[Developed Film] {-{-}{} I[Playback Light Source]}
    I {-{-}{} J[Photocell/Detector]}
    J {-{-}{} K[Amplifier]}
    K {-{-}{} L[Speaker]}
{Highlighting}
{Shaded}
\end{verbatim}
\end{center}

\end{solutionbox}
\begin{mnemonicbox}
``CAREP: Capture sound, Amplify signal, Record
optically, Expose film, Play back''

\end{mnemonicbox}
\subsection*{Question 5(c OR) [7
marks]}\label{question-5c-or-7-marks}

\textbf{Define loudspeaker. List types of loudspeakers and explain
working of any one type of loudspeaker.}

\begin{solutionbox}

\textbf{Definition:} A loudspeaker is an electroacoustic transducer that
converts electrical signals into sound waves by moving a diaphragm that
creates air pressure variations.


{\def\LTcaptype{none} % do not increment counter
\vspace{-5pt}
\captionof{table}{Types of Loudspeakers}
\vspace{-10pt}
\begin{longtable}[]{@{}
  >{\raggedright\arraybackslash}p{(\linewidth - 6\tabcolsep) * \real{0.1071}}
  >{\raggedright\arraybackslash}p{(\linewidth - 6\tabcolsep) * \real{0.3393}}
  >{\raggedright\arraybackslash}p{(\linewidth - 6\tabcolsep) * \real{0.3036}}
  >{\raggedright\arraybackslash}p{(\linewidth - 6\tabcolsep) * \real{0.2500}}@{}}
\toprule\noalign{}
\begin{minipage}[b]{\linewidth}\raggedright
Type
\end{minipage} & \begin{minipage}[b]{\linewidth}\raggedright
Working Principle
\end{minipage} & \begin{minipage}[b]{\linewidth}\raggedright
Frequency Range
\end{minipage} & \begin{minipage}[b]{\linewidth}\raggedright
Applications
\end{minipage} \\
\midrule\noalign{}
\endhead
\bottomrule\noalign{}
\endlastfoot
\textbf{Dynamic/Moving Coil} & Electromagnetic induction & 20Hz-20kHz &
Most common, general purpose \\
\textbf{Electrostatic} & Electrostatic force between plates &
100Hz-20kHz & High-fidelity audio systems \\
\textbf{Piezoelectric} & Piezoelectric effect & 1kHz-25kHz & Tweeters,
alarms, buzzers \\
\textbf{Ribbon} & Current through ribbon in magnetic field & 2kHz-50kHz
& High-frequency reproduction \\
\textbf{Planar Magnetic} & Magnetic force on conductor sheet &
30Hz-20kHz & Audiophile headphones, speakers \\
\end{longtable}
}

\textbf{Working of Dynamic/Moving Coil Loudspeaker:}

\begin{center}
\textbf{Mermaid Diagram (Code)}
\begin{verbatim}
{Shaded}
{Highlighting}[]
graph LR
    A[Audio Signal] {-{-}{} B[Voice Coil]}
    B {-{-}{} C[Electromagnetic Field]}
    D[Permanent Magnet] {-{-}{} C}
    C {-{-}{} E[Movement of Voice Coil]}
    E {-{-}{} F[Cone/Diaphragm Movement]}
    F {-{-}{} G[Air Pressure Variations]}
    G {-{-}{} H[Sound Waves]}
{Highlighting}
{Shaded}
\end{verbatim}
\end{center}

\textbf{Working Process:}

\begin{enumerate}
\tightlist
\item
  Audio current flows through voice coil
\item
  Current creates electromagnetic field
\item
  Electromagnetic field interacts with permanent magnet
\item
  Voice coil moves forward/backward based on signal polarity
\item
  Attached cone/diaphragm moves, creating air pressure variations
\item
  Air pressure variations propagate as sound waves
\end{enumerate}

\textbf{Components:}

\begin{itemize}
\tightlist
\item
  \textbf{Cone/Diaphragm}: Moves air to create sound
\item
  \textbf{Voice Coil}: Carries audio signal current
\item
  \textbf{Magnet}: Creates static magnetic field
\item
  \textbf{Suspension}: Keeps cone centered, allows movement
\item
  \textbf{Frame/Basket}: Holds components in proper alignment
\end{itemize}

\end{solutionbox}
\begin{mnemonicbox}
``SEPVADICS: Signal Enters, Produces Vibrations,
Activates Diaphragm, In Coordination with Suspension''

\end{mnemonicbox}

\end{document}
