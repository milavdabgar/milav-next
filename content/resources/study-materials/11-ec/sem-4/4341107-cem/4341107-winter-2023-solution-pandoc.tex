\documentclass[10pt,a4paper]{article}

% content/resources/templates/preamble.tex
\usepackage[margin=0.6in]{geometry}
\author{Milav Dabgar}
\usepackage{amsmath,amssymb,amsthm}
\usepackage{booktabs}
\usepackage{multirow}
\usepackage{xcolor}
\usepackage{tcolorbox}
\tcbuselibrary{breakable,skins}
\usepackage[colorlinks=true,linkcolor=blue]{hyperref}
\usepackage{titlesec}
\usepackage{enumitem}
\usepackage{tikz}
\usepackage{pgfplots}
\usepackage{circuitikz}
\usepackage[version=4]{mhchem}
\usepackage{longtable}
\usepackage{array}
\usepackage{float}
\usepackage{caption}
\usepackage{listings}

\lstset{
  basicstyle=\small\ttfamily,
  breaklines=true,
  breakatwhitespace=false,
  postbreak=\mbox{\textcolor{red}{$\hookrightarrow$}\space},
  float=false,
  numbers=left,
  numberstyle=\tiny\color{gray},
  numbersep=10pt,
  xleftmargin=2em,
  keywordstyle=\color{blue},
  commentstyle=\color{green!60!black},
  stringstyle=\color{purple},
  backgroundcolor=\color{gray!5},
  showstringspaces=false,
  tabsize=2,
  captionpos=b,
  keepspaces=true,
  columns=flexible
}

\pgfplotsset{compat=1.18}
\usetikzlibrary{shapes,arrows,positioning,calc,patterns,decorations.pathmorphing,decorations.markings,arrows.meta}

% Color scheme
\definecolor{headcolor}{RGB}{0,102,204}
\definecolor{keycolor}{RGB}{220,20,60}
\definecolor{solutioncolor}{RGB}{34,139,34}
\definecolor{mnemoniccolor}{RGB}{148,0,211}
\definecolor{codecolor}{RGB}{0,0,100}

% Spacing
\setlength{\parskip}{3pt}
\setlist[itemize]{nosep}
\setlist[enumerate]{nosep}

% Title formatting
\titleformat{\section}{\Large\bfseries\color{headcolor}}{\thesection}{1em}{}
\titleformat{\subsection}{\large\bfseries\color{headcolor}}{\thesubsection}{1em}{}

% Pandoc tightlist compatibility
\providecommand{\tightlist}{%
  \setlength{\itemsep}{0pt}\setlength{\parskip}{0pt}}

% Pandoc longtable compatibility
\newcounter{none}
\def\thenone{}


% content/resources/templates/english-boxes.tex
% This file is currently empty - it exists to maintain consistency with the import structure.
% Add custom environments here if needed in the future.


\begin{document}

\begin{center}
{\Huge\bfseries\color{headcolor} Subject Name Solutions}\\[5pt]
{\LARGE 4341107 -- Winter 2023}\\[3pt]
{\large Semester 1 Study Material}\\[3pt]
{\normalsize\textit{Detailed Solutions and Explanations}}
\end{center}

\vspace{10pt}

\subsection*{Question 1(a) [3 marks]}\label{q1a}

\textbf{Explain different types of maintenance in brief.}

\begin{solutionbox}

{\def\LTcaptype{none} % do not increment counter
\begin{longtable}[]{@{}
  >{\raggedright\arraybackslash}p{(\linewidth - 2\tabcolsep) * \real{0.5000}}
  >{\raggedright\arraybackslash}p{(\linewidth - 2\tabcolsep) * \real{0.5000}}@{}}
\toprule\noalign{}
\begin{minipage}[b]{\linewidth}\raggedright
Type of Maintenance
\end{minipage} & \begin{minipage}[b]{\linewidth}\raggedright
Description
\end{minipage} \\
\midrule\noalign{}
\endhead
\bottomrule\noalign{}
\endlastfoot
\textbf{Preventive Maintenance} & Scheduled regular inspection and
servicing to prevent breakdowns \\
\textbf{Corrective Maintenance} & Repairs performed after equipment
failure to restore functionality \\
\textbf{Predictive Maintenance} & Uses condition monitoring to predict
when maintenance will be needed \\
\end{longtable}
}

\end{solutionbox}
\begin{mnemonicbox}
``PCPro'' - Preventive prevents, Corrective cures,
Predictive predicts

\end{mnemonicbox}
\subsection*{Question 1(b) [4 marks]}\label{q1b}

\textbf{Explain maintenance procedure of Washing Machine.}

\begin{solutionbox}

\textbf{Maintenance Procedure for Washing Machine:}

\begin{verbatim}
flowchart LR
    A[Regular Inspection] {-{-} B[Clean Filter]}
    B {-{-} C[Check Hoses]}
    C {-{-} D[Balance Load]}
    D {-{-} E[Clean Drum]}
\end{verbatim}

\begin{itemize}
\tightlist
\item
  \textbf{Filter Cleaning}: Remove and clean lint filter monthly
\item
  \textbf{Hose Inspection}: Check for cracks and leaks every 3 months
\item
  \textbf{Load Distribution}: Ensure proper balancing to prevent
  vibration
\item
  \textbf{Drum Cleaning}: Run empty hot water cycle with vinegar
  quarterly
\end{itemize}

\end{solutionbox}
\begin{mnemonicbox}
``FHLD'' - Filters, Hoses, Loads, Drum need regular
attention

\end{mnemonicbox}
\subsection*{Question 1(c) [7 marks]}\label{q1c}

\textbf{Explain maintenance and troubleshooting procedure of Microwave
Oven.}

\begin{solutionbox}

\textbf{Maintenance and Troubleshooting for Microwave Oven:}

{\def\LTcaptype{none} % do not increment counter
\begin{longtable}[]{@{}
  >{\raggedright\arraybackslash}p{(\linewidth - 4\tabcolsep) * \real{0.3333}}
  >{\raggedright\arraybackslash}p{(\linewidth - 4\tabcolsep) * \real{0.3333}}
  >{\raggedright\arraybackslash}p{(\linewidth - 4\tabcolsep) * \real{0.3333}}@{}}
\toprule\noalign{}
\begin{minipage}[b]{\linewidth}\raggedright
Maintenance Task
\end{minipage} & \begin{minipage}[b]{\linewidth}\raggedright
Procedure
\end{minipage} & \begin{minipage}[b]{\linewidth}\raggedright
Frequency
\end{minipage} \\
\midrule\noalign{}
\endhead
\bottomrule\noalign{}
\endlastfoot
External Cleaning & Wipe with mild detergent & Weekly \\
Internal Cleaning & Clean food particles and grease & After each
spill \\
Door Seal Check & Inspect for damage or leakage & Monthly \\
Ventilation Check & Ensure vents are unobstructed & Monthly \\
\end{longtable}
}

\textbf{Troubleshooting Procedure:}

\begin{verbatim}
flowchart TD
    A[No Power] {-{-}|Check| B[Power Connection]}
    C[Not Heating] {-{-}|Check| D[Door Switch \& Magnetron]}
    E[Uneven Cooking] {-{-}|Check| F[Turntable Mechanism]}
    G[Sparking] {-{-}|Check| H[Metal Objects/Damaged Cavity]}
    I[Unusual Noise] {-{-}|Check| J[Fan \& Turntable Motor]}
\end{verbatim}

\begin{itemize}
\tightlist
\item
  \textbf{Power Issues}: Check fuse, circuit breaker, and cord
\item
  \textbf{Heating Problems}: Test door switch, high voltage capacitor,
  magnetron
\item
  \textbf{Safety First}: Never operate with damaged door or seals
\end{itemize}

\end{solutionbox}
\begin{mnemonicbox}
``POWER'' - Power, Oven interior, Wiring,
Electronics, Radiation seals

\end{mnemonicbox}
\subsection*{Question 1(c OR) [7
marks]}\label{question-1c-or-7-marks}

\textbf{Explain maintenance and troubleshooting procedure of projector.}

\begin{solutionbox}

\textbf{Maintenance and Troubleshooting for Projector:}

{\def\LTcaptype{none} % do not increment counter
\begin{longtable}[]{@{}lll@{}}
\toprule\noalign{}
Maintenance Task & Procedure & Frequency \\
\midrule\noalign{}
\endhead
\bottomrule\noalign{}
\endlastfoot
Lens Cleaning & Use lens cloth and solution & Monthly \\
Filter Cleaning & Remove and clean dust & Every 100 hours \\
Lamp Inspection & Check for discoloration/dimming & Every 300 hours \\
Ventilation & Ensure proper airflow & Before each use \\
\end{longtable}
}

\textbf{Troubleshooting Procedure:}

\begin{verbatim}
flowchart TD
    A[No Power] {-{-}|Check| B[Power Supply \& Cable]}
    C[No Image] {-{-}|Check| D[Source Connection \& Input Selection]}
    E[Poor Image] {-{-}|Check| F[Focus \& Lens]}
    G[Overheating] {-{-}|Check| H[Ventilation \& Filters]}
    I[Lamp Failure] {-{-}|Check| J[Lamp Life \& Replacement]}
\end{verbatim}

\begin{itemize}
\tightlist
\item
  \textbf{Image Issues}: Adjust focus, resolution, keystone correction
\item
  \textbf{Lamp Problems}: Check lamp hours, replace if exceeding limit
\item
  \textbf{Connectivity}: Verify input source and cable connections
\item
  \textbf{Thermal Issues}: Clean filters and ensure proper ventilation
\end{itemize}

\end{solutionbox}
\begin{mnemonicbox}
``FLAMVE'' - Filters, Lamp, Airflow, Mounting,
Voltage, Environment

\end{mnemonicbox}
\subsection*{Question 2(a) [3 marks]}\label{q2a}

\textbf{Explain the terms in brief: (1) Hue (2) Brightness}

\begin{solutionbox}

{\def\LTcaptype{none} % do not increment counter
\begin{longtable}[]{@{}
  >{\raggedright\arraybackslash}p{(\linewidth - 2\tabcolsep) * \real{0.5000}}
  >{\raggedright\arraybackslash}p{(\linewidth - 2\tabcolsep) * \real{0.5000}}@{}}
\toprule\noalign{}
\begin{minipage}[b]{\linewidth}\raggedright
Term
\end{minipage} & \begin{minipage}[b]{\linewidth}\raggedright
Description
\end{minipage} \\
\midrule\noalign{}
\endhead
\bottomrule\noalign{}
\endlastfoot
\textbf{Hue} & The pure color attribute that distinguishes colors (red,
green, blue, etc.) based on light wavelength \\
\textbf{Brightness} & The amount of light emitted or reflected from a
color, determining how light or dark it appears \\
\end{longtable}
}

\textbf{Diagram:}

\begin{verbatim}
             Hue
          (Color Type)
              ↑
              |
Saturation {-{-}+{-}{-} Brightness}
(Intensity)   |   (Lightness)
              ↓
            Value
\end{verbatim}

\end{solutionbox}
\begin{mnemonicbox}
``HB-WC'' - Hue determines What Color, Brightness
determines White-to-black level

\end{mnemonicbox}
\subsection*{Question 2(b) [4 marks]}\label{q2b}

\textbf{Write a short note on LCD TV.}

\begin{solutionbox}

\textbf{LCD TV Technology:}

\begin{verbatim}
flowchart LR
    A[Backlight] {-{-} B[Polarizing Filter]}
    B {-{-} C[Liquid Crystal Layer]}
    C {-{-} D[Color Filter]}
    D {-{-} E[Screen]}
\end{verbatim}

\begin{itemize}
\tightlist
\item
  \textbf{Working Principle}: Uses liquid crystals that twist/untwist to
  allow/block light
\item
  \textbf{Key Components}: Backlight, polarizing filters, liquid crystal
  matrix, color filters
\item
  \textbf{Advantages}: Thin profile, energy efficient, no radiation,
  sharp image
\item
  \textbf{Limitations}: Limited viewing angle, slower response time than
  newer technologies
\end{itemize}

\end{solutionbox}
\begin{mnemonicbox}
``BPLCS'' - Backlight Passes Light through Crystals
to Screen

\end{mnemonicbox}
\subsection*{Question 2(c) [7 marks]}\label{q2c}

\textbf{Draw and explain block diagram of DTH receiver.}

\begin{solutionbox}

\textbf{DTH Receiver Block Diagram:}

\begin{verbatim}
flowchart LR
    A[Satellite Dish] {-{-} B[LNB]}
    B {-{-} C[Tuner]}
    C {-{-} D[Demodulator]}
    D {-{-} E[MPEG Decoder]}
    E {-{-} F[Video/Audio Processor]}
    F {-{-} G[TV Display]}
    H[Smart Card] {-{-} I[Conditional Access Module]}
    I {-{-} D}
    J[User Interface] {-{-} K[Microcontroller]}
    K {-{-} C}
    K {-{-} E}
\end{verbatim}

\begin{itemize}
\tightlist
\item
  \textbf{Satellite Dish}: Captures signals from satellite
\item
  \textbf{LNB (Low Noise Block)}: Converts high frequency signals to
  lower frequency
\item
  \textbf{Tuner}: Selects specific channel frequency
\item
  \textbf{Demodulator}: Extracts digital information from carrier signal
\item
  \textbf{MPEG Decoder}: Decompresses video/audio data
\item
  \textbf{Conditional Access Module}: Controls subscription access
\item
  \textbf{Microcontroller}: Controls overall operation and user inputs
\end{itemize}

\end{solutionbox}
\begin{mnemonicbox}
``SLTDMP'' - Satellite, LNB, Tuner, Demodulator,
MPEG, Processor

\end{mnemonicbox}
\subsection*{Question 2(a OR) [3
marks]}\label{question-2a-or-3-marks}

\textbf{Explain the terms in brief: (1) Luminance (2) chrominance}

\begin{solutionbox}

{\def\LTcaptype{none} % do not increment counter
\begin{longtable}[]{@{}
  >{\raggedright\arraybackslash}p{(\linewidth - 2\tabcolsep) * \real{0.5000}}
  >{\raggedright\arraybackslash}p{(\linewidth - 2\tabcolsep) * \real{0.5000}}@{}}
\toprule\noalign{}
\begin{minipage}[b]{\linewidth}\raggedright
Term
\end{minipage} & \begin{minipage}[b]{\linewidth}\raggedright
Description
\end{minipage} \\
\midrule\noalign{}
\endhead
\bottomrule\noalign{}
\endlastfoot
\textbf{Luminance} & The brightness or intensity component of a video
signal (Y) that carries black and white information \\
\textbf{Chrominance} & The color component of a video signal (Cb, Cr)
that carries hue and saturation information \\
\end{longtable}
}

\textbf{Diagram:}

\begin{verbatim}
Video Signal
    |
    +{-{-}{-}{-}{-}{-}{-}{-}{-}{-}+{-}{-}{-}{-}{-}{-}{-}{-}{-}{-}+}
    |                     |
Luminance (Y)      Chrominance (C)
(Brightness)         /         {}
                    /           {}
                   /             {}
            Blue{-Y (Cb)       Red{-}Y (Cr)}
            (Blue diff)       (Red diff)
\end{verbatim}

\end{solutionbox}
\begin{mnemonicbox}
``LC-BH'' - Luminance controls Brightness,
Chrominance controls Hue

\end{mnemonicbox}
\subsection*{Question 2(b OR) [4
marks]}\label{question-2b-or-4-marks}

\textbf{Explain Grassman's law.}

\begin{solutionbox}

\textbf{Grassman's Laws of Color Mixing:}

{\def\LTcaptype{none} % do not increment counter
\begin{longtable}[]{@{}
  >{\raggedright\arraybackslash}p{(\linewidth - 2\tabcolsep) * \real{0.5000}}
  >{\raggedright\arraybackslash}p{(\linewidth - 2\tabcolsep) * \real{0.5000}}@{}}
\toprule\noalign{}
\begin{minipage}[b]{\linewidth}\raggedright
Law
\end{minipage} & \begin{minipage}[b]{\linewidth}\raggedright
Description
\end{minipage} \\
\midrule\noalign{}
\endhead
\bottomrule\noalign{}
\endlastfoot
\textbf{Symmetry} & If color A matches color B, then B matches A \\
\textbf{Proportionality} & If A matches B, then nA matches nB (for any
intensity n) \\
\textbf{Additivity} & If A matches B and C matches D, then A+C matches
B+D \\
\end{longtable}
}

\begin{itemize}
\tightlist
\item
  \textbf{Application}: Forms the basis of RGB color model in displays
\item
  \textbf{Significance}: Allows creating any color by mixing three
  primary colors
\item
  \textbf{Limitation}: Applies only to light (additive mixing), not
  pigments
\end{itemize}

\end{solutionbox}
\begin{mnemonicbox}
``SPA Color'' - Symmetry, Proportionality, Additivity
laws for Color matching

\end{mnemonicbox}
\subsection*{Question 2(c OR) [7
marks]}\label{question-2c-or-7-marks}

\textbf{Draw and explain block diagram of colour TV receiver.}

\begin{solutionbox}

\textbf{Colour TV Receiver Block Diagram:}

\begin{verbatim}
flowchart LR
    A[Antenna] {-{-} B[Tuner]}
    B {-{-} C[IF Amplifier]}
    C {-{-} D[Video Detector]}
    D {-{-} E[Video Amplifier]}
    E {-{-} F[Color Processor]}
    F {-{-} G[RGB Matrix]}
    G {-{-} H[Picture Tube/Display]}
    D {-{-} I[Sound IF]}
    I {-{-} J[Sound Demodulator]}
    J {-{-} K[Audio Amplifier]}
    K {-{-} L[Speaker]}
    M[Sync Separator] {-{-} N[Deflection Circuits]}
    N {-{-} H}
    D {-{-} M}
\end{verbatim}

\begin{itemize}
\tightlist
\item
  \textbf{Tuner}: Selects desired channel frequency
\item
  \textbf{IF Amplifier}: Amplifies intermediate frequency signals
\item
  \textbf{Video Detector}: Extracts video and audio information
\item
  \textbf{Color Processor}: Separates luminance and chrominance
\item
  \textbf{RGB Matrix}: Converts color signals to red, green, blue
\item
  \textbf{Sync Separator}: Extracts horizontal and vertical sync
\item
  \textbf{Deflection Circuits}: Control electron beam scanning
\end{itemize}

\end{solutionbox}
\begin{mnemonicbox}
``TIVCRDS'' - Tuner, IF, Video, Color, RGB,
Deflection, Speaker

\end{mnemonicbox}
\subsection*{Question 3(a) [3 marks]}\label{q3a}

\textbf{State main components of solar power system and specifications
of solar power system.}

\begin{solutionbox}

\textbf{Main Components of Solar Power System:}

{\def\LTcaptype{none} % do not increment counter
\begin{longtable}[]{@{}ll@{}}
\toprule\noalign{}
Component & Function \\
\midrule\noalign{}
\endhead
\bottomrule\noalign{}
\endlastfoot
\textbf{Solar Panels} & Convert sunlight to electricity \\
\textbf{Charge Controller} & Regulates battery charging \\
\textbf{Battery Bank} & Stores electrical energy \\
\textbf{Inverter} & Converts DC to AC electricity \\
\textbf{Mounting Structure} & Supports and positions panels \\
\end{longtable}
}

\textbf{Specifications:}

\begin{itemize}
\tightlist
\item
  \textbf{Panel Rating}: 100-400W per panel
\item
  \textbf{Battery Capacity}: 100-200Ah
\item
  \textbf{Inverter Rating}: 500-5000W
\item
  \textbf{System Voltage}: 12/24/48V
\end{itemize}

\end{solutionbox}
\begin{mnemonicbox}
``SCBIM'' - Solar panels, Controller, Battery,
Inverter, Mounting

\end{mnemonicbox}
\subsection*{Question 3(b) [4 marks]}\label{q3b}

\textbf{List types, applications and technical specifications of
microwave oven.}

\begin{solutionbox}

\textbf{Types of Microwave Ovens:}

{\def\LTcaptype{none} % do not increment counter
\begin{longtable}[]{@{}ll@{}}
\toprule\noalign{}
Type & Features \\
\midrule\noalign{}
\endhead
\bottomrule\noalign{}
\endlastfoot
\textbf{Solo} & Basic heating and defrosting only \\
\textbf{Grill} & Additional grilling element \\
\textbf{Convection} & Has heating element and fan for baking \\
\textbf{Combination} & Integrates microwave, grill and convection \\
\end{longtable}
}

\textbf{Applications:}

\begin{itemize}
\tightlist
\item
  Food reheating
\item
  Defrosting
\item
  Cooking
\item
  Baking (convection models)
\end{itemize}

\textbf{Technical Specifications:}

\begin{itemize}
\tightlist
\item
  \textbf{Power}: 700-1200 Watts
\item
  \textbf{Capacity}: 20-40 Liters
\item
  \textbf{Frequency}: 2.45 GHz
\item
  \textbf{Voltage}: 220-240V AC
\end{itemize}

\end{solutionbox}
\begin{mnemonicbox}
``SGCC'' - Solo, Grill, Convection, Combo ovens for
various cooking needs

\end{mnemonicbox}
\subsection*{Question 3(c) [7 marks]}\label{q3c}

\textbf{Explain working of Air conditioner and Refrigerator}

\begin{solutionbox}

\textbf{Working Principle of Air Conditioner and Refrigerator:}

\begin{verbatim}
flowchart LR
    A[Compressor] {-{-}|High pressure hot gas| B[Condenser]}
    B {-{-}|High pressure liquid| C[Expansion Valve]}
    C {-{-}|Low pressure liquid| D[Evaporator]}
    D {-{-}|Low pressure gas| A}
\end{verbatim}

\textbf{Common Components:}

\begin{itemize}
\tightlist
\item
  \textbf{Compressor}: Pressurizes refrigerant gas
\item
  \textbf{Condenser}: Releases heat, converts gas to liquid
\item
  \textbf{Expansion Valve}: Reduces pressure of liquid refrigerant
\item
  \textbf{Evaporator}: Absorbs heat, converts liquid to gas
\end{itemize}

\textbf{Differences:}

{\def\LTcaptype{none} % do not increment counter
\begin{longtable}[]{@{}
  >{\raggedright\arraybackslash}p{(\linewidth - 4\tabcolsep) * \real{0.3333}}
  >{\raggedright\arraybackslash}p{(\linewidth - 4\tabcolsep) * \real{0.3333}}
  >{\raggedright\arraybackslash}p{(\linewidth - 4\tabcolsep) * \real{0.3333}}@{}}
\toprule\noalign{}
\begin{minipage}[b]{\linewidth}\raggedright
Aspect
\end{minipage} & \begin{minipage}[b]{\linewidth}\raggedright
Air Conditioner
\end{minipage} & \begin{minipage}[b]{\linewidth}\raggedright
Refrigerator
\end{minipage} \\
\midrule\noalign{}
\endhead
\bottomrule\noalign{}
\endlastfoot
\textbf{Purpose} & Cools entire room & Maintains cold in insulated
cabinet \\
\textbf{Temperature} & 18-26^\circC typically & 2-8^\circC (fridge), -18^\circC
(freezer) \\
\textbf{Control} & Thermostat with remote & Manual or digital
thermostat \\
\end{longtable}
}

\end{solutionbox}
\begin{mnemonicbox}
``CEVA'' - Compression, Expansion, Vaporization,
Absorption cycle

\end{mnemonicbox}
\subsection*{Question 3(a OR) [3
marks]}\label{question-3a-or-3-marks}

\textbf{List technical specifications of Air conditioner and
Refrigerator}

\begin{solutionbox}

\textbf{Technical Specifications:}

{\def\LTcaptype{none} % do not increment counter
\begin{longtable}[]{@{}
  >{\raggedright\arraybackslash}p{(\linewidth - 4\tabcolsep) * \real{0.3333}}
  >{\raggedright\arraybackslash}p{(\linewidth - 4\tabcolsep) * \real{0.3333}}
  >{\raggedright\arraybackslash}p{(\linewidth - 4\tabcolsep) * \real{0.3333}}@{}}
\toprule\noalign{}
\begin{minipage}[b]{\linewidth}\raggedright
Specification
\end{minipage} & \begin{minipage}[b]{\linewidth}\raggedright
Air Conditioner
\end{minipage} & \begin{minipage}[b]{\linewidth}\raggedright
Refrigerator
\end{minipage} \\
\midrule\noalign{}
\endhead
\bottomrule\noalign{}
\endlastfoot
\textbf{Cooling Capacity} & 1-2 ton (12,000-24,000 BTU) & 100-500 liters
capacity \\
\textbf{Power Consumption} & 1000-2500 watts & 100-400 watts \\
\textbf{Energy Efficiency} & ISEER/Star Rating 3-5 & BEE Star Rating
3-5 \\
\textbf{Refrigerant Type} & R32, R410A & R600a, R134a \\
\textbf{Voltage/Frequency} & 220-240V/50Hz & 220-240V/50Hz \\
\end{longtable}
}

\end{solutionbox}
\begin{mnemonicbox}
``CPERS'' - Capacity, Power, Efficiency, Refrigerant,
Supply specifications

\end{mnemonicbox}
\subsection*{Question 3(b OR) [4
marks]}\label{question-3b-or-4-marks}

\textbf{Explain electronic controller for washing machine.}

\begin{solutionbox}

\textbf{Electronic Controller for Washing Machine:}

\begin{verbatim}
flowchart TD
    A[User Interface] {-{-} B[Microcontroller]}
    B {-{-} C[Motor Driver]}
    B {-{-} D[Water Valve Control]}
    B {-{-} E[Temperature Sensor]}
    B {-{-} F[Water Level Sensor]}
    B {-{-} G[Door Lock Control]}
    B {-{-} H[Drain Pump Control]}
\end{verbatim}

\begin{itemize}
\tightlist
\item
  \textbf{Microcontroller}: Central processing unit that controls all
  operations
\item
  \textbf{Sensors}: Water level, temperature, load balance, door
  position
\item
  \textbf{Actuators}: Motor driver, water valves, heater, drain pump
\item
  \textbf{User Interface}: Program selection, temperature, spin speed
  settings
\end{itemize}

\end{solutionbox}
\begin{mnemonicbox}
``MIST-WAD'' - Microcontroller Integrates Sensors and
Timers for Water, Agitation and Drainage

\end{mnemonicbox}
\subsection*{Question 3(c OR) [7
marks]}\label{question-3c-or-7-marks}

\textbf{Draw and explain block diagram of Microwave oven. List wiring
and safety instructions for microwave oven}

\begin{solutionbox}

\textbf{Microwave Oven Block Diagram:}

\begin{verbatim}
flowchart LR
    A[Control Panel] {-{-} B[Control Circuit]}
    B {-{-} C[High Voltage Transformer]}
    C {-{-} D[High Voltage Capacitor]}
    D {-{-} E[Magnetron]}
    E {-{-} F[Waveguide]}
    F {-{-} G[Cooking Cavity]}
    B {-{-} H[Turntable Motor]}
    B {-{-} I[Fan Motor]}
    B {-{-} J[Door Interlock Switches]}
\end{verbatim}

\begin{itemize}
\tightlist
\item
  \textbf{Control Circuit}: Processes user inputs and controls timing
\item
  \textbf{High Voltage Transformer}: Steps up voltage to 2000-4000V
\item
  \textbf{Magnetron}: Generates microwave radiation at 2.45 GHz
\item
  \textbf{Waveguide}: Directs microwaves into cooking cavity
\item
  \textbf{Turntable}: Ensures even cooking through rotation
\end{itemize}

\textbf{Safety Instructions:}

\begin{itemize}
\tightlist
\item
  Never operate with door open or damaged
\item
  Ensure proper grounding
\item
  Don't override safety interlocks
\item
  Use microwave-safe containers only
\end{itemize}

\textbf{Wiring Instructions:}

\begin{itemize}
\tightlist
\item
  Use appropriate gauge power cable (typically 14-16 AWG)
\item
  Connect to dedicated 15-20A circuit
\item
  Ensure proper ground connection
\item
  Keep wiring away from heat sources
\end{itemize}

\end{solutionbox}
\begin{mnemonicbox}
``MAGIC'' - Magnetron And Guided waves Into Cavity

\end{mnemonicbox}
\subsection*{Question 4(a) [3 marks]}\label{q4a}

\textbf{Draw block diagram of Photocopier.}

\begin{solutionbox}

\textbf{Photocopier Block Diagram:}

\begin{verbatim}
flowchart LR
    A[Document Scanner] {-{-} B[Image Processor]}
    B {-{-} C[Laser Unit]}
    C {-{-} D[Photosensitive Drum]}
    E[Charging Unit] {-{-} D}
    D {-{-} F[Developer Unit]}
    F {-{-} G[Transfer Unit]}
    G {-{-} H[Paper Feed]}
    G {-{-} I[Fusing Unit]}
    I {-{-} J[Output Tray]}
\end{verbatim}

\begin{itemize}
\tightlist
\item
  \textbf{Scanner}: Captures original document image
\item
  \textbf{Drum}: Holds electrostatic image
\item
  \textbf{Developer}: Applies toner to charged areas
\item
  \textbf{Transfer}: Transfers toner to paper
\item
  \textbf{Fuser}: Melts toner permanently onto paper
\end{itemize}

\end{solutionbox}
\begin{mnemonicbox}
``SDTFO'' - Scan, Develop, Transfer, Fuse, Output

\end{mnemonicbox}
\subsection*{Question 4(b) [4 marks]}\label{q4b}

\textbf{List specifications of MF printer and CCTV.}

\begin{solutionbox}

\textbf{Specifications:}

{\def\LTcaptype{none} % do not increment counter
\begin{longtable}[]{@{}
  >{\raggedright\arraybackslash}p{(\linewidth - 2\tabcolsep) * \real{0.5000}}
  >{\raggedright\arraybackslash}p{(\linewidth - 2\tabcolsep) * \real{0.5000}}@{}}
\toprule\noalign{}
\begin{minipage}[b]{\linewidth}\raggedright
MF Printer Specifications
\end{minipage} & \begin{minipage}[b]{\linewidth}\raggedright
CCTV Specifications
\end{minipage} \\
\midrule\noalign{}
\endhead
\bottomrule\noalign{}
\endlastfoot
\textbf{Print Resolution}: 600-1200 dpi & \textbf{Camera Resolution}:
2-8 MP \\
\textbf{Print Speed}: 15-40 ppm & \textbf{Frame Rate}: 15-30 fps \\
\textbf{Scan Resolution}: 300-600 dpi & \textbf{Storage}: 1-8 TB
HDD/NVR \\
\textbf{Paper Capacity}: 150-500 sheets & \textbf{Night Vision}: 10-30m
range \\
\textbf{Connectivity}: USB, Ethernet, Wi-Fi & \textbf{Connectivity}:
Coaxial/IP/Wireless \\
\textbf{Functions}: Print, Scan, Copy, Fax & \textbf{Video Format}:
H.264/H.265 \\
\end{longtable}
}

\end{solutionbox}
\begin{mnemonicbox}
``RSCPF'' - Resolution, Speed, Capacity, Protocol,
Function specifications

\end{mnemonicbox}
\subsection*{Question 4(c) [7 marks]}\label{q4c}

\textbf{Explain working of laser printer with block diagram.}

\begin{solutionbox}

\textbf{Laser Printer Working:}

\begin{verbatim}
flowchart LR
    A[Data Processing] {-{-} B[Laser Unit]}
    B {-{-} C[Photosensitive Drum]}
    D[Primary Corona] {-{-} C}
    C {-{-} E[Developer Unit]}
    E {-{-} F[Transfer Corona]}
    F {-{-} G[Paper Transport]}
    G {-{-} H[Fusing Unit]}
    H {-{-} I[Output]}
    J[Cleaning Unit] {-{-} C}
\end{verbatim}

\textbf{Working Process:}

\begin{enumerate}
\tightlist
\item
  \textbf{Charging}: Corona wire gives drum uniform negative charge
\item
  \textbf{Writing}: Laser neutralizes charges on drum to form image
\item
  \textbf{Developing}: Toner adheres to discharged areas of drum
\item
  \textbf{Transfer}: Paper receives positive charge, attracts toner
\item
  \textbf{Fusing}: Heat and pressure melt toner onto paper
\item
  \textbf{Cleaning}: Residual toner is removed from drum
\end{enumerate}

\begin{itemize}
\tightlist
\item
  \textbf{Resolution}: Determined by laser precision (600-1200 dpi)
\item
  \textbf{Speed}: Based on drum rotation and paper transport (15-40 ppm)
\end{itemize}

\end{solutionbox}
\begin{mnemonicbox}
``CWTFC'' - Charge, Write, Transfer, Fuse, Clean
cycle

\end{mnemonicbox}
\subsection*{Question 4(a OR) [3
marks]}\label{question-4a-or-3-marks}

\textbf{Draw block diagram of CCTV.}

\begin{solutionbox}

\textbf{CCTV System Block Diagram:}

\begin{verbatim}
flowchart LR
    A[Cameras] {-{-} B[Video Transmission]}
    B {-{-} C[Digital Video Recorder]}
    C {-{-} D[Storage HDD]}
    C {-{-} E[Monitor Display]}
    F[Power Supply] {-{-} A}
    F {-{-} C}
    G[Network Switch] {-{-} C}
    C {-{-} H[Remote Access]}
\end{verbatim}

\begin{itemize}
\tightlist
\item
  \textbf{Cameras}: Capture video footage
\item
  \textbf{Transmission}: Coaxial cable/IP network/Wireless
\item
  \textbf{DVR/NVR}: Processes and records video
\item
  \textbf{Storage}: Hard drives for footage retention
\item
  \textbf{Monitor}: Displays live or recorded footage
\end{itemize}

\end{solutionbox}
\begin{mnemonicbox}
``CTDSM'' - Camera, Transmission, DVR, Storage,
Monitor system

\end{mnemonicbox}
\subsection*{Question 4(b OR) [4
marks]}\label{question-4b-or-4-marks}

\textbf{List specifications of inkjet printer and Photocopier.}

\begin{solutionbox}

\textbf{Specifications:}

{\def\LTcaptype{none} % do not increment counter
\begin{longtable}[]{@{}
  >{\raggedright\arraybackslash}p{(\linewidth - 2\tabcolsep) * \real{0.5000}}
  >{\raggedright\arraybackslash}p{(\linewidth - 2\tabcolsep) * \real{0.5000}}@{}}
\toprule\noalign{}
\begin{minipage}[b]{\linewidth}\raggedright
Inkjet Printer Specifications
\end{minipage} & \begin{minipage}[b]{\linewidth}\raggedright
Photocopier Specifications
\end{minipage} \\
\midrule\noalign{}
\endhead
\bottomrule\noalign{}
\endlastfoot
\textbf{Print Resolution}: 1200-4800 dpi & \textbf{Copy Resolution}:
600-1200 dpi \\
\textbf{Print Speed}: 8-20 ppm & \textbf{Copy Speed}: 20-60 cpm \\
\textbf{Ink Type}: Dye/Pigment & \textbf{Toner Type}: Dry/Liquid \\
\textbf{Paper Capacity}: 100-250 sheets & \textbf{Paper Capacity}:
250-2000 sheets \\
\textbf{Connectivity}: USB, Wi-Fi & \textbf{Functions}: Copy, Scan,
Print, Fax \\
\textbf{Duty Cycle}: 1,000-5,000 pages/month & \textbf{Duty Cycle}:
10,000-100,000 pages/month \\
\end{longtable}
}

\end{solutionbox}
\begin{mnemonicbox}
``RSIPCD'' - Resolution, Speed, Ink/toner, Paper
capacity, Connectivity, Duty cycle

\end{mnemonicbox}
\subsection*{Question 4(c OR) [7
marks]}\label{question-4c-or-7-marks}

\textbf{Explain working of LCD projector with block diagram and list its
specifications.}

\begin{solutionbox}

\textbf{LCD Projector Working:}

\begin{verbatim}
flowchart LR
    A[Input Source] {-{-} B[Signal Processor]}
    B {-{-} C[Lamp/Light Source]}
    C {-{-} D[Condenser Lens]}
    D {-{-} E[Dichroic Mirrors]}
    E {-{-}|Red| F[Red LCD Panel]}
    E {-{-}|Green| G[Green LCD Panel]}
    E {-{-}|Blue| H[Blue LCD Panel]}
    F {-{-} I[Prism]}
    G {-{-} I}
    H {-{-} I}
    I {-{-} J[Projection Lens]}
    J {-{-} K[Screen]}
\end{verbatim}

\textbf{Working Process:}

\begin{enumerate}
\tightlist
\item
  \textbf{Light Generation}: High-intensity lamp produces white light
\item
  \textbf{Color Separation}: Dichroic mirrors split light into RGB
\item
  \textbf{Modulation}: LCD panels control light intensity for each color
\item
  \textbf{Recombination}: Prism reassembles RGB images
\item
  \textbf{Projection}: Lens system projects image onto screen
\end{enumerate}

\textbf{Specifications:}

\begin{itemize}
\tightlist
\item
  \textbf{Resolution}: XGA (1024\times768), WXGA (1280\times800), FHD (1920\times1080)
\item
  \textbf{Brightness}: 2000-5000 ANSI lumens
\item
  \textbf{Contrast Ratio}: 2000:1 to 20000:1
\item
  \textbf{Lamp Life}: 3000-6000 hours
\item
  \textbf{Throw Ratio}: 0.5:1 to 2.0:1
\item
  \textbf{Connectivity}: HDMI, VGA, USB, Wi-Fi
\end{itemize}

\end{solutionbox}
\begin{mnemonicbox}
``LSPMPS'' - Lamp, Split, Panels, Modulate, Prism,
Screen

\end{mnemonicbox}
\subsection*{Question 5(a) [3 marks]}\label{q5a}

\textbf{Draw block diagram of PA system.}

\begin{solutionbox}

\textbf{Public Address (PA) System Block Diagram:}

\begin{verbatim}
flowchart LR
    A[Microphone] {-{-} B[Pre{-}amplifier]}
    B {-{-} C[Mixer]}
    D[Audio Source] {-{-} C}
    C {-{-} E[Equalizer]}
    E {-{-} F[Power Amplifier]}
    F {-{-} G[Speaker Network]}
    H[Volume Control] {-{-} C}
\end{verbatim}

\begin{itemize}
\tightlist
\item
  \textbf{Microphone}: Converts sound to electrical signals
\item
  \textbf{Pre-amplifier}: Boosts microphone signal
\item
  \textbf{Mixer}: Combines multiple audio sources
\item
  \textbf{Equalizer}: Adjusts frequency response
\item
  \textbf{Power Amplifier}: Increases signal power
\item
  \textbf{Speakers}: Convert electrical signals back to sound
\end{itemize}

\end{solutionbox}
\begin{mnemonicbox}
``MMEPS'' - Microphone, Mixer, Equalizer, Power amp,
Speakers

\end{mnemonicbox}
\subsection*{Question 5(b) [4 marks]}\label{q5b}

\textbf{Explain tweeter and woofer.}

\begin{solutionbox}

\textbf{Speaker Components:}

{\def\LTcaptype{none} % do not increment counter
\begin{longtable}[]{@{}lll@{}}
\toprule\noalign{}
Feature & Tweeter & Woofer \\
\midrule\noalign{}
\endhead
\bottomrule\noalign{}
\endlastfoot
\textbf{Frequency Range} & High (2kHz-20kHz) & Low (20Hz-2kHz) \\
\textbf{Size} & Small (0.5''-1.5'') & Large (4''-15'') \\
\textbf{Diaphragm} & Light, rigid (dome/cone) & Heavy, flexible cone \\
\textbf{Voice Coil} & Small diameter & Large diameter \\
\textbf{Cabinet Design} & Horn/sealed & Ported/sealed/bass reflex \\
\end{longtable}
}

\textbf{Working Principle:}

\begin{verbatim}
flowchart LR
    A[Audio Signal] {-{-} B[Crossover Network]}
    B {-{-}|High Frequencies| C[Tweeter]}
    B {-{-}|Low Frequencies| D[Woofer]}
    C {-{-} E[High{-}Frequency Sound Waves]}
    D {-{-} F[Low{-}Frequency Sound Waves]}
\end{verbatim}

\begin{itemize}
\tightlist
\item
  \textbf{Tweeter}: Reproduces high frequencies with clarity and detail
\item
  \textbf{Woofer}: Reproduces low frequencies with power and depth
\end{itemize}

\end{solutionbox}
\begin{mnemonicbox}
``THSL'' - Tweeters handle Highs, Small and Light;
Woofers handle Lows

\end{mnemonicbox}
\subsection*{Question 5(c) [7 marks]}\label{q5c}

\textbf{Define microphone. List types of microphone and explain working
of any one type of microphone.}

\begin{solutionbox}

\textbf{Microphone Definition:} A microphone is an electroacoustic
transducer that converts sound waves into electrical signals.

\textbf{Types of Microphones:}

{\def\LTcaptype{none} % do not increment counter
\begin{longtable}[]{@{}
  >{\raggedright\arraybackslash}p{(\linewidth - 4\tabcolsep) * \real{0.3333}}
  >{\raggedright\arraybackslash}p{(\linewidth - 4\tabcolsep) * \real{0.3333}}
  >{\raggedright\arraybackslash}p{(\linewidth - 4\tabcolsep) * \real{0.3333}}@{}}
\toprule\noalign{}
\begin{minipage}[b]{\linewidth}\raggedright
Type
\end{minipage} & \begin{minipage}[b]{\linewidth}\raggedright
Working Principle
\end{minipage} & \begin{minipage}[b]{\linewidth}\raggedright
Applications
\end{minipage} \\
\midrule\noalign{}
\endhead
\bottomrule\noalign{}
\endlastfoot
\textbf{Dynamic} & Electromagnetic induction & Live performance,
broadcasting \\
\textbf{Condenser} & Electrostatic principles & Studio recording,
smartphones \\
\textbf{Ribbon} & Electromagnetic induction & Studio vocals,
instruments \\
\textbf{Carbon} & Resistance variation & Old telephones \\
\textbf{Piezoelectric} & Piezoelectric effect & Contact mics,
instruments \\
\textbf{MEMS} & Micro-electromechanical & Laptops, tiny devices \\
\end{longtable}
}

\textbf{Dynamic Microphone Working:}

\begin{verbatim}
flowchart LR
    A[Sound Waves] {-{-} B[Diaphragm]}
    B {-{-} C[Attached Coil]}
    C {-{-} D[Movement in Magnetic Field]}
    D {-{-} E[Induced Voltage]}
    E {-{-} F[Electrical Signal Output]}
\end{verbatim}

\begin{itemize}
\tightlist
\item
  \textbf{Sound Capture}: Diaphragm vibrates with sound waves
\item
  \textbf{Transduction}: Coil attached to diaphragm moves within
  magnetic field
\item
  \textbf{Signal Generation}: Movement induces voltage proportional to
  sound intensity
\item
  \textbf{Output}: Low impedance, strong signal requiring minimal
  amplification
\item
  \textbf{Advantages}: Durable, handles high SPL, no external power
  needed
\end{itemize}

\end{solutionbox}
\begin{mnemonicbox}
``DDCMIO'' - Diaphragm Displaces Coil in Magnetic
field Inducing Output

\end{mnemonicbox}
\subsection*{Question 5(a OR) [3
marks]}\label{question-5a-or-3-marks}

\textbf{Define: (1) Pitch (2) Loudspeaker (3) Reverberation.}

\begin{solutionbox}

\textbf{Definitions:}

{\def\LTcaptype{none} % do not increment counter
\begin{longtable}[]{@{}
  >{\raggedright\arraybackslash}p{(\linewidth - 2\tabcolsep) * \real{0.5000}}
  >{\raggedright\arraybackslash}p{(\linewidth - 2\tabcolsep) * \real{0.5000}}@{}}
\toprule\noalign{}
\begin{minipage}[b]{\linewidth}\raggedright
Term
\end{minipage} & \begin{minipage}[b]{\linewidth}\raggedright
Definition
\end{minipage} \\
\midrule\noalign{}
\endhead
\bottomrule\noalign{}
\endlastfoot
\textbf{Pitch} & The perceived frequency of a sound that determines how
``high'' or ``low'' it sounds \\
\textbf{Loudspeaker} & An electroacoustic transducer that converts
electrical signals into sound waves \\
\textbf{Reverberation} & The persistence of sound after the original
sound has stopped due to multiple reflections \\
\end{longtable}
}

\textbf{Diagram:}

\begin{verbatim}
Reverberation
    |
    v
Original Sound {-{-}{-}{-}{-} Early Reflections {-}{-}{-}{-}{-} Late Reflections}
    \^{                       |                       |}
    |                       v                       v
Direct Sound           Clarity (80ms)       Spaciousness ({80ms)}
\end{verbatim}

\end{solutionbox}
\begin{mnemonicbox}
``PLR Sound'' - Pitch defines tone, Loudspeaker
produces it, Reverberation extends it

\end{mnemonicbox}
\subsection*{Question 5(b OR) [4
marks]}\label{question-5b-or-4-marks}

\textbf{Draw block diagram of Home theatre sound system and explain in
brief.}

\begin{solutionbox}

\textbf{Home Theatre Sound System:}

\begin{verbatim}
flowchart TD
    A[Audio/Video Source] {-{-} B[AV Receiver/Amplifier]}
    B {-{-} C[Front Speakers]}
    B {-{-} D[Center Speaker]}
    B {-{-} E[Surround Speakers]}
    B {-{-} F[Subwoofer]}
    B {-{-} G[Video Display]}
    H[Remote Control] {-{-} B}
\end{verbatim}

\begin{itemize}
\tightlist
\item
  \textbf{AV Receiver}: Central hub that processes audio/video signals
\item
  \textbf{Front Speakers}: Left and right channels for stereo sound
\item
  \textbf{Center Speaker}: Delivers dialog and central sounds
\item
  \textbf{Surround Speakers}: Create immersive environment with ambient
  sounds
\item
  \textbf{Subwoofer}: Reproduces low-frequency effects (LFE) below 120Hz
\item
  \textbf{Configuration}: Common setups include 2.1, 5.1, 7.1, or 9.1
  channel systems
\end{itemize}

\end{solutionbox}
\begin{mnemonicbox}
``AFSCS'' - Amplifier drives Front, Surround, Center
Speakers and Subwoofer

\end{mnemonicbox}
\subsection*{Question 5(c OR) [7
marks]}\label{question-5c-or-7-marks}

\textbf{Explain Electrostatic loudspeaker and permanent magnet
loudspeaker.}

\begin{solutionbox}

\textbf{Comparison of Loudspeaker Types:}

{\def\LTcaptype{none} % do not increment counter
\begin{longtable}[]{@{}
  >{\raggedright\arraybackslash}p{(\linewidth - 4\tabcolsep) * \real{0.3333}}
  >{\raggedright\arraybackslash}p{(\linewidth - 4\tabcolsep) * \real{0.3333}}
  >{\raggedright\arraybackslash}p{(\linewidth - 4\tabcolsep) * \real{0.3333}}@{}}
\toprule\noalign{}
\begin{minipage}[b]{\linewidth}\raggedright
Feature
\end{minipage} & \begin{minipage}[b]{\linewidth}\raggedright
Electrostatic Speaker
\end{minipage} & \begin{minipage}[b]{\linewidth}\raggedright
Permanent Magnet Speaker
\end{minipage} \\
\midrule\noalign{}
\endhead
\bottomrule\noalign{}
\endlastfoot
\textbf{Working Principle} & Electrostatic forces between plates &
Electromagnetic induction \\
\textbf{Construction} & Thin diaphragm between stator plates & Cone
attached to voice coil in magnetic field \\
\textbf{Power Requirements} & Needs high voltage polarizing supply & No
external power beyond signal \\
\textbf{Frequency Response} & Excellent mid/high frequency & Good across
full range with proper design \\
\textbf{Efficiency} & Low (1-3\%) & Moderate (2-5\%) \\
\textbf{Distortion} & Very low & Moderate \\
\end{longtable}
}

\textbf{Electrostatic Speaker Working:}

\begin{verbatim}
flowchart LR
    A[Audio Signal] {-{-} B[Step{-}up Transformer]}
    C[High Voltage DC Supply] {-{-} D[Charged Diaphragm]}
    B {-{-} E[Conductive Stator Plates]}
    E {-{-} F[Electrostatic Force]}
    F {-{-} D}
    D {-{-} G[Sound Waves]}
\end{verbatim}

\begin{itemize}
\tightlist
\item
  \textbf{Diaphragm}: Thin, lightweight membrane with conductive coating
\item
  \textbf{Operation}: Audio signal varies charge on stator plates,
  creating varying force on diaphragm
\end{itemize}

\textbf{Permanent Magnet Speaker Working:}

\begin{verbatim}
flowchart LR
    A[Audio Signal] {-{-} B[Voice Coil]}
    C[Permanent Magnet] {-{-} D[Magnetic Field]}
    B {-{-} E[Current in Coil]}
    E[Current in Coil] {-{-} F[Electromagnetic Force]}
    F {-{-} G[Cone Displacement]}
    G {-{-} H[Air Movement]}
    H {-{-} I[Sound Waves]}
\end{verbatim}

\begin{itemize}
\tightlist
\item
  \textbf{Voice Coil}: Winding of wire attached to speaker cone
\item
  \textbf{Operation}: Current through coil creates magnetic field that
  interacts with permanent magnet
\item
  \textbf{Advantages}: Robust design, good power handling, no high
  voltage required
\item
  \textbf{Applications}: Most common speaker design for general audio
  reproduction
\end{itemize}

\end{solutionbox}
\begin{mnemonicbox}
``ESPM'' - Electrostatic uses Static charges,
Permanent Magnet uses Magnetic forces

\end{mnemonicbox}

\end{document}
