\documentclass[10pt,a4paper]{article}

% content/resources/templates/preamble.tex
\usepackage[margin=0.6in]{geometry}
\author{Milav Dabgar}
\usepackage{amsmath,amssymb,amsthm}
\usepackage{booktabs}
\usepackage{multirow}
\usepackage{xcolor}
\usepackage{tcolorbox}
\tcbuselibrary{breakable,skins}
\usepackage[colorlinks=true,linkcolor=blue]{hyperref}
\usepackage{titlesec}
\usepackage{enumitem}
\usepackage{tikz}
\usepackage{pgfplots}
\usepackage{circuitikz}
\usepackage[version=4]{mhchem}
\usepackage{longtable}
\usepackage{array}
\usepackage{float}
\usepackage{caption}
\usepackage{listings}

\lstset{
  basicstyle=\small\ttfamily,
  breaklines=true,
  breakatwhitespace=false,
  postbreak=\mbox{\textcolor{red}{$\hookrightarrow$}\space},
  float=false,
  numbers=left,
  numberstyle=\tiny\color{gray},
  numbersep=10pt,
  xleftmargin=2em,
  keywordstyle=\color{blue},
  commentstyle=\color{green!60!black},
  stringstyle=\color{purple},
  backgroundcolor=\color{gray!5},
  showstringspaces=false,
  tabsize=2,
  captionpos=b,
  keepspaces=true,
  columns=flexible
}

\pgfplotsset{compat=1.18}
\usetikzlibrary{shapes,arrows,positioning,calc,patterns,decorations.pathmorphing,decorations.markings,arrows.meta}

% Color scheme
\definecolor{headcolor}{RGB}{0,102,204}
\definecolor{keycolor}{RGB}{220,20,60}
\definecolor{solutioncolor}{RGB}{34,139,34}
\definecolor{mnemoniccolor}{RGB}{148,0,211}
\definecolor{codecolor}{RGB}{0,0,100}

% Spacing
\setlength{\parskip}{3pt}
\setlist[itemize]{nosep}
\setlist[enumerate]{nosep}

% Title formatting
\titleformat{\section}{\Large\bfseries\color{headcolor}}{\thesection}{1em}{}
\titleformat{\subsection}{\large\bfseries\color{headcolor}}{\thesubsection}{1em}{}

% Pandoc tightlist compatibility
\providecommand{\tightlist}{%
  \setlength{\itemsep}{0pt}\setlength{\parskip}{0pt}}

% Pandoc longtable compatibility
\newcounter{none}
\def\thenone{}


% content/resources/templates/gujarati-boxes.tex
\usepackage{fontspec}
\usepackage{polyglossia}

% Set Gujarati as main language (document is primarily in Gujarati)
% Note: gloss-gujarati.ldf doesn't exist in polyglossia, but it will use hyphenation patterns
\setdefaultlanguage{gujarati}
\setotherlanguage{english}

% Configure Gujarati font properly
% Use Language=Default to prevent polyglossia from trying to add language-specific features
% that don't exist for Gujarati, which causes "empty feature" warnings
\newfontfamily\gujaratifont[Script=Gujarati,AutoFakeBold=2.5,AutoFakeSlant=0.3]{Noto Sans Gujarati}
\setmainfont[Script=Gujarati,AutoFakeBold=2.5,AutoFakeSlant=0.3]{Noto Sans Gujarati}
% Use Noto Sans Gujarati for monospace to support Gujarati in text
\setmonofont[Scale=0.9]{Noto Sans Gujarati}

% Configure English to use the same font
\newfontfamily\englishfont[Script=Gujarati,AutoFakeBold=2.5,AutoFakeSlant=0.3]{Noto Sans Gujarati}

% Translations for polyglossia
\gappto\captionsgujarati{
  \renewcommand{\tablename}{કોષ્ટક}
  \renewcommand{\figurename}{આકૃતિ}
}

% Helper for TikZ nodes to ensure Gujarati font
\newcommand{\gu}[1]{{\gujaratifont #1}}

% Custom environments
\newtcolorbox{solutionbox}{
    breakable,
    enhanced,
    colback=solutioncolor!5!white,
    colframe=solutioncolor!75!black,
    fonttitle=\bfseries,
    title=જવાબ
}

\newtcolorbox{solutionboxnobreak}{
 colback=solutioncolor!5!white,
 colframe=solutioncolor!75!black,
 fonttitle=\bfseries,
 title=જવાબ
}

\newtcolorbox{keyformula}{
 breakable,
 enhanced,
 colback=keycolor!5!white,
 colframe=keycolor!75!black,
 fonttitle=\bfseries,
 title=રાસાયણિક સમીકરણ/સૂત્ર
}

\newtcolorbox{mnemonicbox}{
 breakable,
 enhanced,
 colback=mnemoniccolor!5!white,
 colframe=mnemoniccolor!75!black,
 fonttitle=\bfseries,
 title=મેમરી ટ્રીક
}


\begin{document}

\begin{center}
{\Huge\bfseries\color{headcolor} Subject Name (Gujarati)}\\[5pt]
{\LARGE 4341107 -- Summer 2023}\\[3pt]
{\large Semester 1 Study Material}\\[3pt]
{\normalsize\textit{Detailed Solutions and Explanations}}
\end{center}

\vspace{10pt}

\subsection*{પ્રશ્ન 1(અ) [3
ગુણ]}\label{uxaaauxab0uxab6uxaa8-1uxa85-3-uxa97uxaa3}

\textbf{CCTV ના મેઇંટેનન્સ ની પ્રક્રિયા વર્ણવો.}

\begin{solutionbox}


{\def\LTcaptype{none} % do not increment counter
\vspace{-5pt}
\captionof{table}{CCTV મેઇંટેનન્સ પ્રક્રિયા}
\vspace{-10pt}
\begin{longtable}[]{@{}lll@{}}
\toprule\noalign{}
સ્ટેપ & પ્રક્રિયા & વિગત \\
\midrule\noalign{}
\endhead
\bottomrule\noalign{}
\endlastfoot
1 & \textbf{કેમેરા ક્લીનિંગ} & મહિને એક વાર લેન્સ અને હાઉસિંગ સાફ કરો \\
2 & \textbf{કેબલ ઇન્સ્પેક્શન} & ત્રિમાસિક નુકસાન/એક્સપોઝર તપાસો \\
3 & \textbf{રેકોર્ડિંગ ચેક} & માસિક ડેટા સંગ્રહ અને પ્લેબેક ચકાસો \\
4 & \textbf{ફર્મવેર અપડેટ} & ઉપલબ્ધ હોય ત્યારે સૉફ્ટવેર અપડેટ કરો \\
5 & \textbf{એંગલ એડજસ્ટમેન્ટ} & જરૂર મુજબ કેમેરા ફરીથી ગોઠવો \\
\end{longtable}
}

\end{solutionbox}
\begin{mnemonicbox}
``CCRU: ક્લીન, ચેક, રેકોર્ડ, અપડેટ''

\end{mnemonicbox}
\subsection*{પ્રશ્ન 1(બ) [4
ગુણ]}\label{uxaaauxab0uxab6uxaa8-1uxaac-4-uxa97uxaa3}

\textbf{મેઇંટેનન્સ ના પ્રકારો લખો અને ટૂંકમા સમજાવો.}

\begin{solutionbox}


{\def\LTcaptype{none} % do not increment counter
\vspace{-5pt}
\captionof{table}{મેઇંટેનન્સના પ્રકારો}
\vspace{-10pt}
\begin{longtable}[]{@{}
  >{\raggedright\arraybackslash}p{(\linewidth - 6\tabcolsep) * \real{0.1333}}
  >{\raggedright\arraybackslash}p{(\linewidth - 6\tabcolsep) * \real{0.2889}}
  >{\raggedright\arraybackslash}p{(\linewidth - 6\tabcolsep) * \real{0.3556}}
  >{\raggedright\arraybackslash}p{(\linewidth - 6\tabcolsep) * \real{0.2222}}@{}}
\toprule\noalign{}
\begin{minipage}[b]{\linewidth}\raggedright
પ્રકાર
\end{minipage} & \begin{minipage}[b]{\linewidth}\raggedright
વર્ણન
\end{minipage} & \begin{minipage}[b]{\linewidth}\raggedright
ક્યારે કરવામાં આવે છે
\end{minipage} & \begin{minipage}[b]{\linewidth}\raggedright
ફાયદા
\end{minipage} \\
\midrule\noalign{}
\endhead
\bottomrule\noalign{}
\endlastfoot
\textbf{પ્રિવેન્ટિવ} & નિયમિત તપાસ ખરાબી પહેલાં & નિર્ધારિત સમયાંતરે & અનપેક્ષિત
ડાઉનટાઇમ ઘટાડે છે \\
\textbf{કરેક્ટિવ} & ઉપકરણ તૂટી જાય ત્યારે રિપેર & નિષ્ફળતા પછી & કાર્યક્ષમતા
પુનઃસ્થાપિત કરે છે \\
\textbf{પ્રિડિક્ટિવ} & ડેટાનો ઉપયોગ નિષ્ફળતાની આગાહી કરવા & વિશ્લેષણ આધારિત &
મેઇંટેનન્સનો સમય અનુકૂળ કરે છે \\
\textbf{કન્ડિશન-બેઝ્ડ} & વાસ્તવિક ઉપકરણની સ્થિતિ મોનિટર કરે છે & સ્થિતિ સૂચવે ત્યારે
& બિનજરૂરી મેઇંટેનન્સ ઘટાડે છે \\
\end{longtable}
}

\begin{center}
\textbf{Mermaid Diagram (Code)}
\begin{verbatim}
{Shaded}
{Highlighting}[]
graph TD
    A[મેઇંટેનન્સના પ્રકારો] {-{-}{} B[પ્રિવેન્ટિવ]}
    A {-{-}{} C[કરેક્ટિવ]}
    A {-{-}{} D[પ્રિડિક્ટિવ]}
    A {-{-}{} E[કન્ડિશન{-}બેઝ્ડ]}
    B {-{-}{} F[નિયમિત તપાસ]}
    C {-{-}{} G[બ્રેકડાઉન પછી રિપેર]}
    D {-{-}{} H[ડેટા આધારિત આગાહી]}
    E {-{-}{} I[ઉપકરણની સ્થિતિ આધારિત]}
{Highlighting}
{Shaded}
\end{verbatim}
\end{center}

\end{solutionbox}
\begin{mnemonicbox}
``PCPC: પ્રિવેન્ટ, કરેક્ટ, પ્રિડિક્ટ, કન્ડિશન''

\end{mnemonicbox}
\subsection*{પ્રશ્ન 1(ક) [7
ગુણ]}\label{uxaaauxab0uxab6uxaa8-1uxa95-7-uxa97uxaa3}

\textbf{વોશીંગ મશીનના મેઇંટેનન્સ અને ટ્રબલશૂટીંગ ની પ્રક્રિયા સમજાવો.}

\begin{solutionbox}


{\def\LTcaptype{none} % do not increment counter
\vspace{-5pt}
\captionof{table}{વોશીંગ મશીન મેઇંટેનન્સ અને ટ્રબલશૂટિંગ}
\vspace{-10pt}
\begin{longtable}[]{@{}
  >{\raggedright\arraybackslash}p{(\linewidth - 4\tabcolsep) * \real{0.1957}}
  >{\raggedright\arraybackslash}p{(\linewidth - 4\tabcolsep) * \real{0.3261}}
  >{\raggedright\arraybackslash}p{(\linewidth - 4\tabcolsep) * \real{0.4783}}@{}}
\toprule\noalign{}
\begin{minipage}[b]{\linewidth}\raggedright
સમસ્યા
\end{minipage} & \begin{minipage}[b]{\linewidth}\raggedright
સંભવિત કારણ
\end{minipage} & \begin{minipage}[b]{\linewidth}\raggedright
ટ્રબલશૂટિંગ સ્ટેપ્સ
\end{minipage} \\
\midrule\noalign{}
\endhead
\bottomrule\noalign{}
\endlastfoot
\textbf{મશીન ચાલુ ન થવું} & પાવર સમસ્યા, ડોર લોક & પાવર સપ્લાય તપાસો, ડોર
બરાબર બંધ છે તે ખાતરી કરો \\
\textbf{પાણી ન ભરાવું} & પાણીનો પુરવઠો, ઇનલેટ વાલ્વ & પાણીના નળ તપાસો, ઇનલેટ
હોઝમાં બ્લોક તપાસો \\
\textbf{પાણી ન નીકળવું} & બ્લોક થયેલ ફિલ્ટર, ડ્રેન પંપ & ફિલ્ટર સાફ કરો, ડ્રેન હોઝ
વળાંક માટે તપાસો \\
\textbf{વધુ વાઇબ્રેશન} & અસંતુલિત લોડ, શિપિંગ બોલ્ટ્સ & કપડાં પુનઃવિતરિત કરો,
શિપિંગ બોલ્ટ્સ દૂર કર્યા છે તે તપાસો \\
\textbf{પાણી લીકેજ} & ક્ષતિગ્રસ્ત હોઝ, ઢીલા કનેક્શન & કનેક્શન તપાસો અને કસો,
ક્ષતિગ્રસ્ત હોઝ બદલો \\
\end{longtable}
}

\textbf{નિયમિત મેઇંટેનન્સ:}

\begin{itemize}
\tightlist
\item
  \textbf{માસિક}: ડિટરજન્ટ ડ્રોઅર અને ડોર સીલ સાફ કરો
\item
  \textbf{ત્રિમાસિક}: ખાલી ગરમ સાયકલ વિનેગર/ક્લીનર સાથે ચલાવો
\item
  \textbf{અર્ધવાર્ષિક}: હોઝમાં તિરાડો તપાસો, ફિલ્ટર સાફ કરો
\end{itemize}

\begin{verbatim}
flowchart LR
    A[સમસ્યા મળી] {-{-} B\{મશીન ચાલુ થાય છે?\}}
    B {-{-}|ના| C[પાવર અને ડોર લોક તપાસો]}
    B {-{-}|હા| D\{પાણી ભરાય છે?\}}
    D {-{-}|ના| E[પાણીનો પુરવઠો અને ઇનલેટ વાલ્વ તપાસો]}
    D {-{-}|હા| F\{પાણી બરાબર નીકળે છે?\}}
    F {-{-}|ના| G[ફિલ્ટર અને ડ્રેન પંપ તપાસો]}
    F {-{-}|હા| H\{વધુ વાઇબ્રેશન?\}}
    H {-{-}|હા| I[લોડ બેલેન્સ અને શિપિંગ બોલ્ટ્સ તપાસો]}
    H {-{-}|ના| J\{પાણી લીકેજ?\}}
    J {-{-}|હા| K[હોઝ અને કનેક્શન તપાસો]}
\end{verbatim}

\end{solutionbox}
\begin{mnemonicbox}
``POWER: પાવર, ઑબ્ઝર્વ, વોટર, એક્ઝામિન, રિપેર''

\end{mnemonicbox}
\subsection*{પ્રશ્ન 1(ક OR) [7
ગુણ]}\label{uxaaauxab0uxab6uxaa8-1uxa95-or-7-uxa97uxaa3}

\textbf{ડીજીટલ ટીવી ના મેઇંટેનન્સ અને ટ્રબલશૂટીંગ ની પ્રક્રિયા સમજાવો.}

\begin{solutionbox}


{\def\LTcaptype{none} % do not increment counter
\vspace{-5pt}
\captionof{table}{ડિજિટલ ટીવી મેઇંટેનન્સ અને ટ્રબલશૂટિંગ}
\vspace{-10pt}
\begin{longtable}[]{@{}
  >{\raggedright\arraybackslash}p{(\linewidth - 4\tabcolsep) * \real{0.1957}}
  >{\raggedright\arraybackslash}p{(\linewidth - 4\tabcolsep) * \real{0.3261}}
  >{\raggedright\arraybackslash}p{(\linewidth - 4\tabcolsep) * \real{0.4783}}@{}}
\toprule\noalign{}
\begin{minipage}[b]{\linewidth}\raggedright
સમસ્યા
\end{minipage} & \begin{minipage}[b]{\linewidth}\raggedright
સંભવિત કારણ
\end{minipage} & \begin{minipage}[b]{\linewidth}\raggedright
ટ્રબલશૂટિંગ સ્ટેપ્સ
\end{minipage} \\
\midrule\noalign{}
\endhead
\bottomrule\noalign{}
\endlastfoot
\textbf{પાવર ન આવવો} & પાવર સપ્લાય સમસ્યા & પાવર કોર્ડ, વોલ આઉટલેટ તપાસો,
જુદા સોકેટમાં પ્રયાસ કરો \\
\textbf{ચિત્ર ન દેખાવું} & ઇનપુટ/સોર્સ પસંદગી & યોગ્ય ઇનપુટ પસંદ કર્યું છે તે તપાસો,
સોર્સ ઉપકરણ તપાસો \\
\textbf{નબળું રિસેપ્શન} & એન્ટેના/કેબલ સમસ્યા & કેબલ કનેક્શન તપાસો, એન્ટેના સ્થિતિ
બદલો \\
\textbf{વિકૃત રંગો} & ડિસ્પ્લે સેટિંગ્સ & પિક્ચર સેટિંગ્સ ડિફોલ્ટ પર રીસેટ કરો \\
\textbf{રિમોટ કામ ન કરવું} & બેટરી સમસ્યા, સેન્સર બ્લોક & બેટરી બદલો, IR સેન્સર
બ્લોક નથી તેની ખાતરી કરો \\
\end{longtable}
}

\textbf{નિયમિત મેઇંટેનન્સ:}

\begin{itemize}
\tightlist
\item
  \textbf{સાપ્તાહિક}: માઇક્રોફાઇબર કપડાથી સ્ક્રીન સાવચેતીથી સાફ કરો
\item
  \textbf{માસિક}: કેબલ કનેક્શન તપાસો અને કસો
\item
  \textbf{વાર્ષિક}: જો ઉપલબ્ધ હોય તો ફર્મવેર અપડેટ કરો
\end{itemize}

\begin{verbatim}
flowchart LR
    A[ટીવી સમસ્યા] {-{-} B\{પાવર ચાલુ થાય છે?\}}
    B {-{-}|ના| C[પાવર સપ્લાય તપાસો]}
    B {-{-}|હા| D\{ચિત્ર દેખાય છે?\}}
    D {-{-}|ના| E[ઇનપુટ સોર્સ તપાસો]}
    D {-{-}|હા| F\{સારું રિસેપ્શન?\}}
    F {-{-}|ના| G[એન્ટેના/કેબલ તપાસો]}
    F {-{-}|હા| H\{યોગ્ય રંગો?\}}
    H {-{-}|ના| I[પિક્ચર સેટિંગ્સ રીસેટ કરો]}
    H {-{-}|હા| J\{રિમોટ કામ કરે છે?\}}
    J {-{-}|ના| K[બેટરી/સેન્સર તપાસો]}
\end{verbatim}

\end{solutionbox}
\begin{mnemonicbox}
``SPIRE: સપ્લાય, પિક્ચર, ઇનપુટ, રિસેપ્શન, ઇલેક્ટ્રોનિક્સ''

\end{mnemonicbox}
\subsection*{પ્રશ્ન 2(અ) [3
ગુણ]}\label{uxaaauxab0uxab6uxaa8-2uxa85-3-uxa97uxaa3}

\textbf{વ્યાખ્યા આપો: (૧) બ્રાઈટનેસ (૨) લ્યુમિનેન્સ (3) ક્રોમિનેન્સ}

\begin{solutionbox}


{\def\LTcaptype{none} % do not increment counter
\vspace{-5pt}
\captionof{table}{ટીવી ડિસ્પ્લે ટર્મ્સ}
\vspace{-10pt}
\begin{longtable}[]{@{}
  >{\raggedright\arraybackslash}p{(\linewidth - 4\tabcolsep) * \real{0.1935}}
  >{\raggedright\arraybackslash}p{(\linewidth - 4\tabcolsep) * \real{0.3871}}
  >{\raggedright\arraybackslash}p{(\linewidth - 4\tabcolsep) * \real{0.4194}}@{}}
\toprule\noalign{}
\begin{minipage}[b]{\linewidth}\raggedright
પદ
\end{minipage} & \begin{minipage}[b]{\linewidth}\raggedright
વ્યાખ્યા
\end{minipage} & \begin{minipage}[b]{\linewidth}\raggedright
માપન એકમ
\end{minipage} \\
\midrule\noalign{}
\endhead
\bottomrule\noalign{}
\endlastfoot
\textbf{બ્રાઈટનેસ} & ડિસ્પ્લેમાંથી પ્રકાશની તીવ્રતાનું અનુભવાતું મૂલ્ય & સબ્જેક્ટિવ પર્સેપ્શન
(નિટ્સ) \\
\textbf{લ્યુમિનેન્સ} & પ્રતિ એકમ ક્ષેત્રફળ માટે પ્રકાશની તીવ્રતાનું ઓબ્જેક્ટિવ માપન &
કેન્ડેલા પ્રતિ ચોરસ મીટર (cd/m^{2}) \\
\textbf{ક્રોમિનેન્સ} & વિડિઓ સિગ્નલમાં બ્રાઈટનેસથી સ્વતંત્ર રંગ માહિતી & U અને V
કોમ્પોનન્ટ્સ \\
\end{longtable}
}

\end{solutionbox}
\begin{mnemonicbox}
``BLC: બ્રાઈટનેસ એટલે પ્રકાશ અનુભવ, લ્યુમિનેન્સ એટલે ગણિત
પ્રકાશ, ક્રોમિનેન્સ એટલે રંગ માહિતી''

\end{mnemonicbox}
\subsection*{પ્રશ્ન 2(બ) [4
ગુણ]}\label{uxaaauxab0uxab6uxaa8-2uxaac-4-uxa97uxaa3}

\textbf{ડીટીએચ રિસિવર નો બ્લોક ડાયેગ્રામ દોરો અને સમજાવો.}

\begin{solutionbox}

\textbf{DTH રિસિવર બ્લોક ડાયાગ્રામ:}

\begin{center}
\textbf{Mermaid Diagram (Code)}
\begin{verbatim}
{Shaded}
{Highlighting}[]
graph LR
    A[સેટેલાઈટ ડિશ] {-{-}{} B[LNB]}
    B {-{-}{} C[ટ્યુનર]}
    C {-{-}{} D[ડિમોડ્યુલેટર]}
    D {-{-}{} E[MPEG ડિકોડર]}
    E {-{-}{} F[વિડિઓ પ્રોસેસર]}
    E {-{-}{} G[ઓડિઓ પ્રોસેસર]}
    F {-{-}{} H[ટીવી ડિસ્પ્લે]}
    G {-{-}{} I[સ્પીકર્સ]}
    J[સ્માર્ટ કાર્ડ] {-{-}{} K[કન્ડિશનલ એક્સેસ મોડ્યુલ]}
    K {-{-}{} D}
    L[યુઝર ઇન્ટરફેસ] {-{-}{} M[માઇક્રોકન્ટ્રોલર]}
    M {-{-}{} C}
    M {-{-}{} E}
{Highlighting}
{Shaded}
\end{verbatim}
\end{center}


{\def\LTcaptype{none} % do not increment counter
\vspace{-5pt}
\captionof{table}{DTH રિસિવર કોમ્પોનન્ટ્સ}
\vspace{-10pt}
\begin{longtable}[]{@{}
  >{\raggedright\arraybackslash}p{(\linewidth - 2\tabcolsep) * \real{0.5238}}
  >{\raggedright\arraybackslash}p{(\linewidth - 2\tabcolsep) * \real{0.4762}}@{}}
\toprule\noalign{}
\begin{minipage}[b]{\linewidth}\raggedright
કોમ્પોનન્ટ
\end{minipage} & \begin{minipage}[b]{\linewidth}\raggedright
કાર્ય
\end{minipage} \\
\midrule\noalign{}
\endhead
\bottomrule\noalign{}
\endlastfoot
\textbf{સેટેલાઈટ ડિશ} & અવકાશમાંથી સેટેલાઈટ સિગ્નલ્સ મેળવે છે \\
\textbf{LNB (લો નોઈઝ બ્લોક)} & ઉચ્ચ-આવૃત્તિના સિગ્નલ્સને નીચી આવૃત્તિમાં પરિવર્તિત
કરે છે \\
\textbf{ટ્યુનર} & ચોક્કસ ચેનલ ફ્રિક્વન્સી પસંદ કરે છે \\
\textbf{ડિમોડ્યુલેટર} & કેરિયર સિગ્નલમાંથી ડિજિટલ ડેટા કાઢે છે \\
\textbf{MPEG ડિકોડર} & ઓડિઓ/વિડિઓ ડેટા ડિકમ્પ્રેસ કરે છે \\
\textbf{કન્ડિશનલ એક્સેસ મોડ્યુલ} & સબ્સ્ક્રિપ્શન એક્સેસ નિયંત્રિત કરે છે \\
\end{longtable}
}

\end{solutionbox}
\begin{mnemonicbox}
``SLTDM: સેટેલાઈટ કેપ્ચર કરે, LNB કન્વર્ટ કરે, ટ્યુનર સિલેક્ટ
કરે, ડિમોડ્યુલેટર એક્સટ્રેક્ટ કરે, MPEG ડિકોડ કરે''

\end{mnemonicbox}
\subsection*{પ્રશ્ન 2(ક) [7
ગુણ]}\label{uxaaauxab0uxab6uxaa8-2uxa95-7-uxa97uxaa3}

\textbf{કલર ટીવી રિસિવર નો બ્લોક ડાયેગ્રામ દોરો અને સમજાવો.}

\begin{solutionbox}

\textbf{કલર ટીવી રિસિવર બ્લોક ડાયાગ્રામ:}

\begin{center}
\textbf{Mermaid Diagram (Code)}
\begin{verbatim}
{Shaded}
{Highlighting}[]
graph TD
    A[એન્ટેના] {-{-}{} B[ટ્યુનર]}
    B {-{-}{} C[IF એમ્પ્લિફાયર]}
    C {-{-}{} D[વિડિઓ ડિટેક્ટર]}
    D {-{-}{} E[વિડિઓ એમ્પ્લિફાયર]}
    D {-{-}{} F[સાઉન્ડ IF અને ડિટેક્ટર]}
    E {-{-}{} G[Y સિગ્નલ પ્રોસેસિંગ]}
    E {-{-}{} H[ક્રોમિનન્સ બેન્ડપાસ]}
    H {-{-}{} I[ક્રોમા ડિમોડ્યુલેટર]}
    I {-{-}{} J[R{-}Y સિગ્નલ]}
    I {-{-}{} K[B{-}Y સિગ્નલ]}
    G {-{-}{} L[RGB મેટ્રિક્સ]}
    J {-{-}{} L}
    K {-{-}{} L}
    L {-{-}{} M[પિક્ચર ટ્યુબ/ડિસ્પ્લે]}
    F {-{-}{} N[ઓડિઓ એમ્પ્લિફાયર]}
    N {-{-}{} O[સ્પીકર]}
    P[પાવર સપ્લાય] {-{-}{} B}
    P {-{-}{} C}
    P {-{-}{} E}
    P {-{-}{} H}
    P {-{-}{} N}
{Highlighting}
{Shaded}
\end{verbatim}
\end{center}


{\def\LTcaptype{none} % do not increment counter
\vspace{-5pt}
\captionof{table}{કલર ટીવી કોમ્પોનન્ટ્સ અને ફંક્શન્સ}
\vspace{-10pt}
\begin{longtable}[]{@{}
  >{\raggedright\arraybackslash}p{(\linewidth - 4\tabcolsep) * \real{0.2571}}
  >{\raggedright\arraybackslash}p{(\linewidth - 4\tabcolsep) * \real{0.2857}}
  >{\raggedright\arraybackslash}p{(\linewidth - 4\tabcolsep) * \real{0.4571}}@{}}
\toprule\noalign{}
\begin{minipage}[b]{\linewidth}\raggedright
સેક્શન
\end{minipage} & \begin{minipage}[b]{\linewidth}\raggedright
ફંક્શન
\end{minipage} & \begin{minipage}[b]{\linewidth}\raggedright
મુખ્ય કોમ્પોનન્ટ્સ
\end{minipage} \\
\midrule\noalign{}
\endhead
\bottomrule\noalign{}
\endlastfoot
\textbf{ટ્યુનર} & ઇચ્છિત ચેનલ પસંદ કરે છે & RF એમ્પ્લિફાયર, મિક્સર, લોકલ ઓસિલેટર \\
\textbf{IF એમ્પ્લિફાયર} & ઇન્ટરમીડિયેટ ફ્રિક્વન્સી એમ્પ્લિફાય કરે છે & બેન્ડપાસ
ફિલ્ટર્સ, એમ્પ્લિફાયર્સ \\
\textbf{વિડિઓ ડિટેક્ટર} & વિડિઓ સિગ્નલ એક્સટ્રેક્ટ કરે છે & ડાયોડ ડિટેક્ટર,
ફિલ્ટર્સ \\
\textbf{ક્રોમિનન્સ સેક્શન} & રંગ માહિતી પ્રોસેસ કરે છે & બેન્ડપાસ ફિલ્ટર, કલર
ડિમોડ્યુલેટર \\
\textbf{લ્યુમિનન્સ સેક્શન} & બ્રાઈટનેસ માહિતી પ્રોસેસ કરે છે & Y સિગ્નલ એમ્પ્લિફાયર \\
\textbf{RGB મેટ્રિક્સ} & ડિસ્પ્લે માટે સિગ્નલ્સ ભેગા કરે છે & મિક્સિંગ સર્કિટ્સ \\
\textbf{ઓડિઓ સેક્શન} & અવાજ પ્રોસેસ કરે છે & સાઉન્ડ IF, ડિટેક્ટર, એમ્પ્લિફાયર \\
\end{longtable}
}

\end{solutionbox}
\begin{mnemonicbox}
``TIVACRL: ટ્યુનર ટ્યુન કરે, IF એમ્પ્લિફાય કરે, વિડિઓ ડિટેક્ટ
કરે, ઓડિઓ અલગ કરે, ક્રોમિનન્સ ડિમોડ્યુલેટ કરે, RGB મિક્સ કરે, લાઈટ ડિસ્પ્લે કરે''

\end{mnemonicbox}
\subsection*{પ્રશ્ન 2(અ OR) [3
ગુણ]}\label{uxaaauxab0uxab6uxaa8-2uxa85-or-3-uxa97uxaa3}

\textbf{એલઇડી ટીવી પર ટૂંકનોંધ લખો.}

\begin{solutionbox}


{\def\LTcaptype{none} % do not increment counter
\vspace{-5pt}
\captionof{table}{LED ટીવી ટેક્નોલોજી}
\vspace{-10pt}
\begin{longtable}[]{@{}
  >{\raggedright\arraybackslash}p{(\linewidth - 2\tabcolsep) * \real{0.3810}}
  >{\raggedright\arraybackslash}p{(\linewidth - 2\tabcolsep) * \real{0.6190}}@{}}
\toprule\noalign{}
\begin{minipage}[b]{\linewidth}\raggedright
પાસું
\end{minipage} & \begin{minipage}[b]{\linewidth}\raggedright
વર્ણન
\end{minipage} \\
\midrule\noalign{}
\endhead
\bottomrule\noalign{}
\endlastfoot
\textbf{મૂળભૂત ટેક્નોલોજી} & ડિસ્પ્લે બેકલાઈટિંગ માટે લાઈટ એમિટિંગ ડાયોડ્સનો ઉપયોગ
કરે છે \\
\textbf{પ્રકારો} & એજ-લિટ (કિનારે LED), ડાયરેક્ટ-લિટ (સ્ક્રીન પાછળ LED), ફુલ-એરે
(લોકલ ડિમિંગ સાથે) \\
\textbf{ફાયદા} & પાતળી પ્રોફાઇલ, ઊર્જા કાર્યક્ષમ, વધુ સારો કોન્ટ્રાસ્ટ રેશિયો,
LCD કરતાં લાંબો જીવનકાળ \\
\textbf{ડિસ્પ્લે પેનલ} & હજુ પણ LCD પેનલનો ઉપયોગ કરે છે; LED ફક્ત બેકલાઈટિંગ માટે
છે \\
\end{longtable}
}

\end{solutionbox}
\begin{mnemonicbox}
``BEST: બેકલાઈટિંગ LED સાથે, એનર્જી અસરકારક, સ્લિમ
ડિઝાઇન, ટ્રુ કલર્સ''

\end{mnemonicbox}
\subsection*{પ્રશ્ન 2(બ OR) [4
ગુણ]}\label{uxaaauxab0uxab6uxaa8-2uxaac-or-4-uxa97uxaa3}

\textbf{પદો ટૂંક મા સમજાવો: (૧)હ્યુ (૨) સેચ્યુરેશન}

\begin{solutionbox}


{\def\LTcaptype{none} % do not increment counter
\vspace{-5pt}
\captionof{table}{રંગ ગુણધર્મો}
\vspace{-10pt}
\begin{longtable}[]{@{}
  >{\raggedright\arraybackslash}p{(\linewidth - 6\tabcolsep) * \real{0.1765}}
  >{\raggedright\arraybackslash}p{(\linewidth - 6\tabcolsep) * \real{0.3529}}
  >{\raggedright\arraybackslash}p{(\linewidth - 6\tabcolsep) * \real{0.2059}}
  >{\raggedright\arraybackslash}p{(\linewidth - 6\tabcolsep) * \real{0.2647}}@{}}
\toprule\noalign{}
\begin{minipage}[b]{\linewidth}\raggedright
પદ
\end{minipage} & \begin{minipage}[b]{\linewidth}\raggedright
વ્યાખ્યા
\end{minipage} & \begin{minipage}[b]{\linewidth}\raggedright
રેન્જ
\end{minipage} & \begin{minipage}[b]{\linewidth}\raggedright
ઉદાહરણ
\end{minipage} \\
\midrule\noalign{}
\endhead
\bottomrule\noalign{}
\endlastfoot
\textbf{હ્યુ} & વાસ્તવિક રંગ તરંગ લંબાઈ (લાલ, વાદળી, લીલો, વગેરે) & કલર વ્હીલ પર
0-360 ડિગ્રી & લાલ=0^\circ, લીલો=120^\circ, વાદળી=240^\circ \\
\textbf{સેચ્યુરેશન} & રંગની તીવ્રતા અથવા શુદ્ધતા (કેટલો જીવંત) & 0-100\% (ગ્રે થી
શુદ્ધ રંગ) & 0\%=ગ્રેસ્કેલ, 100\%=જીવંત રંગ \\
\end{longtable}
}

\begin{center}
\textbf{Mermaid Diagram (Code)}
\begin{verbatim}
{Shaded}
{Highlighting}[]
graph LR
    A[રંગ ગુણધર્મો] {-{-}{} B[હ્યુ]}
    A {-{-}{} C[સેચ્યુરેશન]}
    B {-{-}{} D[રંગની તરંગલંબાઈ]}
    C {-{-}{} E[શુદ્ધતા/જીવંતતા]}
    D {-{-}{} F[કલર વ્હીલ પર ડિગ્રીમાં માપવામાં આવે છે]}
    E {-{-}{} G[ટકાવારીમાં માપવામાં આવે છે]}
{Highlighting}
{Shaded}
\end{verbatim}
\end{center}

\end{solutionbox}
\begin{mnemonicbox}
``HS: હ્યુ એટલે રંગનો શેડ, સેચ્યુરેશન એટલે રંગની સ્ટ્રેન્થ''

\end{mnemonicbox}
\subsection*{પ્રશ્ન 2(ક OR) [7
ગુણ]}\label{uxaaauxab0uxab6uxaa8-2uxa95-or-7-uxa97uxaa3}

\textbf{કલર સર્કલ ડાયેગ્રામ અને ગ્રાસમેનના નિયમ ની મદદ થી એડીટીવ કલર મિક્સિંગ
સમજાવો.}

\begin{solutionbox}


{\def\LTcaptype{none} % do not increment counter
\vspace{-5pt}
\captionof{table}{એડિટિવ કલર મિક્સિંગ પ્રિન્સિપલ્સ}
\vspace{-10pt}
\begin{longtable}[]{@{}lll@{}}
\toprule\noalign{}
રંગનું સંયોજન & પરિણામ & RGB મૂલ્ય \\
\midrule\noalign{}
\endhead
\bottomrule\noalign{}
\endlastfoot
\textbf{લાલ + લીલો} & પીળો & (255,255,0) \\
\textbf{લીલો + વાદળી} & સિયાન & (0,255,255) \\
\textbf{વાદળી + લાલ} & મેજેન્ટા & (255,0,255) \\
\textbf{લાલ + લીલો + વાદળી} & સફેદ & (255,255,255) \\
\textbf{કોઈ રંગ નહીં} & કાળો & (0,0,0) \\
\end{longtable}
}

\textbf{ગ્રાસમેનના નિયમો:}

\begin{itemize}
\tightlist
\item
  \textbf{નિયમ 1}: કોઈપણ રંગ ત્રણ પ્રાથમિક રંગો મિશ્ર કરીને બનાવી શકાય છે
\item
  \textbf{નિયમ 2}: રંગનો દેખાવ માત્ર તેના ટ્રિસ્ટિમ્યુલસ મૂલ્યો પર આધારિત છે
\item
  \textbf{નિયમ 3}: એડિટિવ મિક્સિંગમાં, ટ્રિસ્ટિમ્યુલસ મૂલ્યો એકસાથે ઉમેરાય છે
\end{itemize}

\begin{center}
\textbf{Mermaid Diagram (Code)}
\begin{verbatim}
{Shaded}
{Highlighting}[]
graph LR
    A[એડિટિવ કલર મિક્સિંગ] {-{-}{} B[પ્રાથમિક રંગો]}
    B {-{-}{} C[લાલ]}
    B {-{-}{} D[લીલો]}
    B {-{-}{} E[વાદળી]}
    C {-{-}{} F[લાલ + લીલો = પીળો]}
    D {-{-}{} F}
    D {-{-}{} G[લીલો + વાદળી = સિયાન]}
    E {-{-}{} G}
    E {-{-}{} H[વાદળી + લાલ = મેજેન્ટા]}
    C {-{-}{} H}
    C {-{-}{} I[લાલ + લીલો + વાદળી = સફેદ]}
    D {-{-}{} I}
    E {-{-}{} I}
{Highlighting}
{Shaded}
\end{verbatim}
\end{center}

\textbf{કલર સર્કલ ડાયાગ્રામ:}

\begin{verbatim}
    Yellow
      /{}
     /  {}
    /    {}
Red {-{-}{-}{-}{-}{-}Green}
    {    /}
     {  /}
      {/}
   Magenta{-{-}{-}{-}Cyan}
       {    /}
        {  /}
         {/}
        Blue
\end{verbatim}

\end{solutionbox}
\begin{mnemonicbox}
``RGB-CMY-W: લાલ, લીલો, વાદળી, સિયાન, મેજેન્ટા, પીળો,
અને સફેદ બનાવે છે''

\end{mnemonicbox}
\subsection*{પ્રશ્ન 3(અ) [3
ગુણ]}\label{uxaaauxab0uxab6uxaa8-3uxa85-3-uxa97uxaa3}

\textbf{માઇક્રોવેવ ઓવન માટે વાયરિંગ અને સેફ્ટી ઇંસ્ટ્રક્શન લખો.}

\begin{solutionbox}


{\def\LTcaptype{none} % do not increment counter
\vspace{-5pt}
\captionof{table}{માઇક્રોવેવ ઓવન વાયરિંગ અને સેફ્ટી ઇન્સ્ટ્રક્શન્સ}
\vspace{-10pt}
\begin{longtable}[]{@{}ll@{}}
\toprule\noalign{}
કેટેગરી & સૂચનાઓ \\
\midrule\noalign{}
\endhead
\bottomrule\noalign{}
\endlastfoot
\textbf{વાયરિંગ} & 15-20A સર્કિટ સાથે ગ્રાઉન્ડેડ આઉટલેટનો ઉપયોગ કરો \\
\textbf{પાવર} & વોલ્ટેજ રેટિંગ સાથે મેળ ખાય તેની ખાતરી કરો (સામાન્ય રીતે
220-240V) \\
\textbf{ઇન્સ્ટોલેશન} & વેન્ટિલેશન માટે તમામ બાજુએ 5 સેમી જગ્યા રાખો \\
\textbf{સેફ્ટી} & ક્યારેય ખાલી ન ચલાવો, ક્યારેય ડોર ઇન્ટરલોક્સ બાયપાસ ન કરો \\
\textbf{મેઇંટેનન્સ} & સર્વિસિંગ પહેલાં પાવર ડિસ્કનેક્ટ કરો, કેપેસિટર ડિસ્ચાર્જ કરો \\
\end{longtable}
}

\end{solutionbox}
\begin{mnemonicbox}
``POWER: પ્રોપર આઉટલેટ, વાયરિંગ ચેક, એમ્પ્ટી ઓપરેશન અવોઇડેડ,
રિપેર્સ બાય પ્રોફેશનલ્સ''

\end{mnemonicbox}
\subsection*{પ્રશ્ન 3(બ) [4
ગુણ]}\label{uxaaauxab0uxab6uxaa8-3uxaac-4-uxa97uxaa3}

\textbf{એર કંડીશનર ની કાર્યપધ્ધતિ સમજાવો.}

\begin{solutionbox}


{\def\LTcaptype{none} % do not increment counter
\vspace{-5pt}
\captionof{table}{એર કન્ડિશનર વર્કિંગ સાયકલ}
\vspace{-10pt}
\begin{longtable}[]{@{}
  >{\raggedright\arraybackslash}p{(\linewidth - 4\tabcolsep) * \real{0.3667}}
  >{\raggedright\arraybackslash}p{(\linewidth - 4\tabcolsep) * \real{0.3333}}
  >{\raggedright\arraybackslash}p{(\linewidth - 4\tabcolsep) * \real{0.3000}}@{}}
\toprule\noalign{}
\begin{minipage}[b]{\linewidth}\raggedright
કોમ્પોનન્ટ
\end{minipage} & \begin{minipage}[b]{\linewidth}\raggedright
ફંક્શન
\end{minipage} & \begin{minipage}[b]{\linewidth}\raggedright
પ્રક્રિયા
\end{minipage} \\
\midrule\noalign{}
\endhead
\bottomrule\noalign{}
\endlastfoot
\textbf{કમ્પ્રેસર} & રેફ્રિજરન્ટ પ્રેશરાઇઝ કરે છે & ઓછા દબાણવાળી ગેસને ઉચ્ચ દબાણવાળી
ગેસમાં પરિવર્તિત કરે છે \\
\textbf{કન્ડેન્સર} & બહાર ગરમી છોડે છે & ગેસને પ્રવાહીમાં પરિવર્તિત કરે છે, ગરમી કાઢે
છે \\
\textbf{એક્સપાન્શન વાલ્વ} & રેફ્રિજરન્ટનો પ્રવાહ નિયંત્રિત કરે છે & પ્રવાહીનું દબાણ
ઘટાડે છે \\
\textbf{ઇવેપોરેટર} & રૂમમાંથી ગરમી શોષે છે & પ્રવાહીને ગેસમાં પરિવર્તિત કરે છે, હવા
ઠંડી કરે છે \\
\textbf{થર્મોસ્ટેટ} & તાપમાન નિયંત્રિત કરે છે & કમ્પ્રેસર ઓપરેશન રેગ્યુલેટ કરે છે \\
\end{longtable}
}

\begin{verbatim}
flowchart LR
    A[કમ્પ્રેસર] {-{-}|ઉચ્ચ{-}દબાણવાળી ગેસ| B[કન્ડેન્સર]}
    B {-{-}|પ્રવાહી| C[એક્સપાન્શન વાલ્વ]}
    C {-{-}|ઓછા{-}દબાણવાળી પ્રવાહી| D[ઇવેપોરેટર]}
    D {-{-}|ઓછા{-}દબાણવાળી ગેસ| A}
    E[રૂમ એર] {-{-} D}
    D {-{-} F[કૂલ એર]}
    G[આઉટસાઇડ એર] {-{-} B}
    B {-{-} H[હોટ એર]}
\end{verbatim}

\end{solutionbox}
\begin{mnemonicbox}
``CELT: કમ્પ્રેસ ગેસ, એક્સપેલ હીટ, લોઅર પ્રેશર, ટેક ઇન હીટ''

\end{mnemonicbox}
\subsection*{પ્રશ્ન 3(ક) [7
ગુણ]}\label{uxaaauxab0uxab6uxaa8-3uxa95-7-uxa97uxaa3}

\textbf{વોશિંગ મશીન માટે ઇલેક્ટ્રોનિક કંટ્રોલર અને ફજી લોજીક વોશિંગ મશીન સમજાવો.
વોશિંગ મશીન ના ટેકનીકલ સ્પેસીફીકેશનો પણ લખો.}

\begin{solutionbox}


{\def\LTcaptype{none} % do not increment counter
\vspace{-5pt}
\captionof{table}{વોશિંગ મશીનમાં ઇલેક્ટ્રોનિક કંટ્રોલર}
\vspace{-10pt}
\begin{longtable}[]{@{}ll@{}}
\toprule\noalign{}
કોમ્પોનન્ટ & ફંક્શન \\
\midrule\noalign{}
\endhead
\bottomrule\noalign{}
\endlastfoot
\textbf{માઇક્રોકંટ્રોલર} & બધા ઓપરેશન્સ નિયંત્રિત કરતું સેન્ટ્રલ પ્રોસેસિંગ યુનિટ \\
\textbf{સેન્સર્સ} & વોટર લેવલ, તાપમાન, લોડ બેલેન્સ, ડોર સ્ટેટસ ડિટેક્ટ કરે છે \\
\textbf{ઇનપુટ ઇન્ટરફેસ} & પ્રોગ્રામ પસંદગી માટે બટન/ટચ પેનલ \\
\textbf{ડિસ્પ્લે} & પ્રોગ્રામ સ્ટેટસ, બાકી સમય, એરર કોડ્સ બતાવે છે \\
\textbf{એક્ચ્યુએટર ડ્રાઇવર્સ} & મોટર, વાલ્વ, હીટર, પંપ નિયંત્રિત કરે છે \\
\end{longtable}
}

\textbf{ફજી લોજિક વોશિંગ મશીન:}

\begin{itemize}
\tightlist
\item
  શ્રેષ્ઠ વોશિંગ માટે આર્ટિફિશિયલ ઇન્ટેલિજન્સનો ઉપયોગ કરે છે
\item
  લોડના આધારે વોટર લેવલ, વોશ ટાઇમ અને સ્પિન સ્પીડ એડજસ્ટ કરે છે
\item
  ચોક્કસ મૂલ્યોને બદલે અંદાજિત તર્ક વડે નિર્ણયો લે છે
\item
  વિવિધ ફેબ્રિક પ્રકારો અને મેલના સ્તરો સાથે આપોઆપ અનુકૂલન કરે છે
\end{itemize}

\textbf{ટેકનિકલ સ્પેસિફિકેશન્સ:}

\begin{itemize}
\tightlist
\item
  \textbf{ક્ષમતા}: 6-10 કિલો (ફ્રન્ટ લોડ), 5-8 કિલો (ટોપ લોડ)
\item
  \textbf{એનર્જી રેટિંગ}: A+++ થી B (EU સ્ટાન્ડર્ડ)
\item
  \textbf{વોટર કન્ઝમ્પશન}: સાયકલ દીઠ 40-70 લિટર
\item
  \textbf{સ્પિન સ્પીડ}: 800-1600 RPM
\item
  \textbf{સાયકલ ઓપ્શન્સ}: 8-16 પ્રોગ્રામ્સ
\end{itemize}

\begin{center}
\textbf{Mermaid Diagram (Code)}
\begin{verbatim}
{Shaded}
{Highlighting}[]
graph TD
    A[ઇલેક્ટ્રોનિક કંટ્રોલર] {-{-}{} B[માઇક્રોકંટ્રોલર]}
    B {-{-}{} C[સેન્સર ઇનપુટ્સ]}
    B {-{-}{} D[યુઝર ઇન્ટરફેસ]}
    B {-{-}{} E[એક્ચ્યુએટર કંટ્રોલ]}
    C {-{-}{} F[વોટર લેવલ સેન્સર]}
    C {-{-}{} G[ટેમ્પરેચર સેન્સર]}
    C {-{-}{} H[લોડ બેલેન્સ સેન્સર]}
    C {-{-}{} I[ડોર લોક સેન્સર]}
    E {-{-}{} J[મોટર ડ્રાઇવર]}
    E {-{-}{} K[વોટર વાલ્વ કંટ્રોલ]}
    E {-{-}{} L[ડ્રેન પંપ કંટ્રોલ]}
    E {-{-}{} M[હીટર કંટ્રોલ]}
    N[ફજી લોજિક] {-{-}{} B}
    N {-{-}{} O[એડેપ્ટિવ કંટ્રોલ]}
    O {-{-}{} P[વોટર લેવલ એડજસ્ટમેન્ટ]}
    O {-{-}{} Q[વોશ ટાઇમ ઓપ્ટિમાઇઝેશન]}
    O {-{-}{} R[સ્પિન સ્પીડ એડજસ્ટમેન્ટ]}
{Highlighting}
{Shaded}
\end{verbatim}
\end{center}

\end{solutionbox}
\begin{mnemonicbox}
``SCRAM: સેન્સર્સ ડિટેક્ટ, કંટ્રોલર પ્રોસેસ, રૂલ્સ એપ્લાઇડ,
એક્ચ્યુએટર્સ ઓપરેટ, મશીન એડેપ્ટ''

\end{mnemonicbox}
\subsection*{પ્રશ્ન 3(અ OR) [3
ગુણ]}\label{uxaaauxab0uxab6uxaa8-3uxa85-or-3-uxa97uxaa3}

\textbf{સોલર પાવર સીસ્ટમના મેઇન કોમ્પોનન્ટો અને સોલર પાવર સીસ્ટમના સ્પેસીફીકેશનો
લખો.}

\begin{solutionbox}


{\def\LTcaptype{none} % do not increment counter
\vspace{-5pt}
\captionof{table}{સોલર પાવર સિસ્ટમ કોમ્પોનન્ટ્સ}
\vspace{-10pt}
\begin{longtable}[]{@{}ll@{}}
\toprule\noalign{}
કોમ્પોનન્ટ & ફંક્શન \\
\midrule\noalign{}
\endhead
\bottomrule\noalign{}
\endlastfoot
\textbf{સોલર પેનલ્સ} & સૂર્યપ્રકાશને DC વીજળીમાં રૂપાંતરિત કરે છે \\
\textbf{ઇન્વર્ટર} & DC પાવરને AC પાવરમાં રૂપાંતરિત કરે છે \\
\textbf{બેટરી બેંક} & પછીના ઉપયોગ માટે ઊર્જા સંગ્રહિત કરે છે \\
\textbf{ચાર્જ કંટ્રોલર} & બેટરીના ઓવરચાર્જિંગને અટકાવે છે \\
\textbf{માઉન્ટિંગ સ્ટ્રક્ચર} & પેનલોને ટેકો આપે છે અને શ્રેષ્ઠ રીતે એંગલ કરે છે \\
\end{longtable}
}

\textbf{સ્પેસિફિકેશન્સ:}

\begin{itemize}
\tightlist
\item
  \textbf{પેનલ કેપેસિટી}: પેનલ દીઠ 250-400 વોટ
\item
  \textbf{સિસ્ટમ સાઇઝ}: 1-10 kW (રહેણાંક)
\item
  \textbf{બેટરી કેપેસિટી}: 100-200 Ah
\item
  \textbf{ઇન્વર્ટર એફિશિયન્સી}: 90-97\%
\item
  \textbf{અપેક્ષિત જીવનકાળ}: 25-30 વર્ષ (પેનલ)
\end{itemize}

\end{solutionbox}
\begin{mnemonicbox}
``PIBCM: પેનલ કલેક્ટ, ઇન્વર્ટર કન્વર્ટ, બેટરી સ્ટોર, કંટ્રોલર
પ્રોટેક્ટ, માઉન્ટ્સ સપોર્ટ''

\end{mnemonicbox}
\subsection*{પ્રશ્ન 3(બ OR) [4
ગુણ]}\label{uxaaauxab0uxab6uxaa8-3uxaac-or-4-uxa97uxaa3}

\textbf{રેફ્રીજરેટર ની કાર્યપધ્ધતિ સમજાવો.}

\begin{solutionbox}


{\def\LTcaptype{none} % do not increment counter
\vspace{-5pt}
\captionof{table}{રેફ્રિજરેટર વર્કિંગ સાયકલ}
\vspace{-10pt}
\begin{longtable}[]{@{}
  >{\raggedright\arraybackslash}p{(\linewidth - 6\tabcolsep) * \real{0.1429}}
  >{\raggedright\arraybackslash}p{(\linewidth - 6\tabcolsep) * \real{0.1837}}
  >{\raggedright\arraybackslash}p{(\linewidth - 6\tabcolsep) * \real{0.2245}}
  >{\raggedright\arraybackslash}p{(\linewidth - 6\tabcolsep) * \real{0.4490}}@{}}
\toprule\noalign{}
\begin{minipage}[b]{\linewidth}\raggedright
સ્ટેજ
\end{minipage} & \begin{minipage}[b]{\linewidth}\raggedright
પ્રક્રિયા
\end{minipage} & \begin{minipage}[b]{\linewidth}\raggedright
કોમ્પોનન્ટ
\end{minipage} & \begin{minipage}[b]{\linewidth}\raggedright
રેફ્રિજરન્ટની સ્થિતિ
\end{minipage} \\
\midrule\noalign{}
\endhead
\bottomrule\noalign{}
\endlastfoot
1 & કમ્પ્રેશન & કમ્પ્રેસર & ઓછા દબાણવાળી ગેસ \rightarrow ઉચ્ચ દબાણવાળી ગેસ \\
2 & કન્ડેન્સેશન & કન્ડેન્સર કોઇલ્સ & ઉચ્ચ દબાણવાળી ગેસ \rightarrow ઉચ્ચ દબાણવાળી પ્રવાહી \\
3 & એક્સપાન્શન & એક્સપાન્શન વાલ્વ & ઉચ્ચ દબાણવાળી પ્રવાહી \rightarrow ઓછા દબાણવાળી
પ્રવાહી \\
4 & ઇવેપોરેશન & ઇવેપોરેટર કોઇલ્સ & ઓછા દબાણવાળી પ્રવાહી \rightarrow ઓછા દબાણવાળી ગેસ \\
\end{longtable}
}

\begin{verbatim}
flowchart LR
    A[કમ્પ્રેસર] {-{-}|ઉચ્ચ{-}દબાણવાળી ગેસ| B[કન્ડેન્સર]}
    B {-{-}|બહાર ગરમી છોડે| C[ઉચ્ચ{-}દબાણવાળી પ્રવાહી]}
    C {-{-} D[એક્સપાન્શન વાલ્વ]}
    D {-{-}|અચાનક દબાણ ઘટાડો| E[ઓછા{-}દબાણવાળી પ્રવાહી]}
    E {-{-} F[ઇવેપોરેટર]}
    F {-{-}|અંદરથી ગરમી શોષે| G[ઓછા{-}દબાણવાળી ગેસ]}
    G {-{-} A}
    H[થર્મોસ્ટેટ] {-{-} A}
\end{verbatim}

\end{solutionbox}
\begin{mnemonicbox}
``CEHE: કમ્પ્રેસ ગેસ, એક્સપેલ હીટ, હાલ્વ પ્રેશર, એક્સટ્રેક્ટ
હીટ''

\end{mnemonicbox}
\subsection*{પ્રશ્ન 3(ક OR) [7
ગુણ]}\label{uxaaauxab0uxab6uxaa8-3uxa95-or-7-uxa97uxaa3}

\textbf{માઇક્રોવેવ ઓવન નો બ્લોક ડાયેગ્રામ દોરો અને સમજાવો. માઇક્રોવેવ ઓવન ના
પ્રકારો, એપ્લીકેશનો અને ટેકનીકલ સ્પેસીફીકેશનો લખો.}

\begin{solutionbox}

\textbf{માઇક્રોવેવ ઓવન બ્લોક ડાયાગ્રામ:}

\begin{center}
\textbf{Mermaid Diagram (Code)}
\begin{verbatim}
{Shaded}
{Highlighting}[]
graph LR
    A[પાવર સપ્લાય] {-{-}{} B[કંટ્રોલ પેનલ/ટાઇમર]}
    A {-{-}{} C[હાઇ વોલ્ટેજ ટ્રાન્સફોર્મર]}
    B {-{-}{} D[ડોર ઇન્ટરલોક સ્વિચ]}
    B {-{-}{} E[કંટ્રોલ સર્કિટ/માઇક્રોકંટ્રોલર]}
    E {-{-}{} F[મેગ્નેટ્રોન ડ્રાઇવર]}
    C {-{-}{} F}
    F {-{-}{} G[મેગ્નેટ્રોન]}
    G {-{-}{} H[વેવગાઇડ]}
    H {-{-}{} I[કુકિંગ કેવિટી]}
    E {-{-}{} J[ટર્નટેબલ મોટર]}
    J {-{-}{} K[ટર્નટેબલ]}
    E {-{-}{} L[કૂલિંગ ફેન]}
{Highlighting}
{Shaded}
\end{verbatim}
\end{center}

\textbf{માઇક્રોવેવ ઓવનના પ્રકારો:}

\begin{itemize}
\tightlist
\item
  \textbf{સોલો}: ફક્ત બેઝિક હીટિંગ અને ડિફ્રોસ્ટિંગ
\item
  \textbf{ગ્રિલ}: વધારાના ગ્રિલિંગ એલિમેન્ટ સાથે
\item
  \textbf{કન્વેક્શન}: માઇક્રોવેવ સાથે કન્વેક્શન હીટિંગ
\item
  \textbf{ઓવર-ધ-રેન્જ (OTR)}: વેન્ટિલેશન સિસ્ટમ સાથે
\item
  \textbf{બિલ્ટ-ઇન}: કેબિનેટ ઇન્સ્ટોલેશન માટે ડિઝાઇન કરેલ
\end{itemize}

\textbf{એપ્લિકેશન્સ:}

\begin{itemize}
\tightlist
\item
  \textbf{કુકિંગ}: ઝડપી ભોજન તૈયારી
\item
  \textbf{રિહીટિંગ}: બચેલા ખોરાક
\item
  \textbf{ડિફ્રોસ્ટિંગ}: ફ્રોઝન ફૂડ
\item
  \textbf{સ્ટેરિલાઇઝેશન}: નાની વસ્તુઓ
\item
  \textbf{કોમર્શિયલ}: ફૂડ સર્વિસ ઇન્ડસ્ટ્રી
\end{itemize}

\textbf{ટેકનિકલ સ્પેસિફિકેશન્સ:}

\begin{itemize}
\tightlist
\item
  \textbf{કેપેસિટી}: 20-40 લિટર
\item
  \textbf{પાવર આઉટપુટ}: 700-1200 વોટ
\item
  \textbf{પાવર કન્ઝમ્પશન}: 1100-1500 વોટ
\item
  \textbf{ફ્રિક્વન્સી}: 2.45 GHz
\item
  \textbf{વોલ્ટેજ}: 220-240V AC
\end{itemize}

\end{solutionbox}
\begin{mnemonicbox}
``MICROWAVES: મેગ્નેટ્રોન જનરેટ કરે, ઇન્ટીરિયર રિસીવ કરે,
કંટ્રોલ રેગ્યુલેટ કરે, રોટેટિંગ ટર્નટેબલ, ઓવન કેવિટી, વેવગાઇડ ડાયરેક્ટ કરે, AC પાવર આપે,
વેન્ટિલેશન કૂલ કરે, ઇલેક્ટ્રોનિક ટાઇમર, સેફ્ટી ઇન્ટરલોક્સ''

\end{mnemonicbox}
\subsection*{પ્રશ્ન 4(અ) [3
ગુણ]}\label{uxaaauxab0uxab6uxaa8-4uxa85-3-uxa97uxaa3}

\textbf{એમએફ પ્રિંટર અને એલસીડી પ્રોજેક્ટર ના સ્પેસીફીકેશનો લખો.}

\begin{solutionbox}


{\def\LTcaptype{none} % do not increment counter
\vspace{-5pt}
\captionof{table}{મલ્ટી-ફંક્શન પ્રિંટર સ્પેસિફિકેશન્સ}
\vspace{-10pt}
\begin{longtable}[]{@{}ll@{}}
\toprule\noalign{}
સ્પેસિફિકેશન & સામાન્ય રેન્જ \\
\midrule\noalign{}
\endhead
\bottomrule\noalign{}
\endlastfoot
\textbf{પ્રિન્ટ રિઝોલ્યુશન} & 600-4800 dpi \\
\textbf{પ્રિન્ટ સ્પીડ} & 20-40 ppm (બ્લેક), 15-30 ppm (કલર) \\
\textbf{સ્કેન રિઝોલ્યુશન} & 600-1200 dpi \\
\textbf{કનેક્ટિવિટી} & Wi-Fi, ઇથરનેટ, USB, ક્લાઉડ \\
\textbf{પેપર કેપેસિટી} & 100-500 શીટ્સ \\
\end{longtable}
}


{\def\LTcaptype{none} % do not increment counter
\vspace{-5pt}
\captionof{table}{LCD પ્રોજેક્ટર સ્પેસિફિકેશન્સ}
\vspace{-10pt}
\begin{longtable}[]{@{}ll@{}}
\toprule\noalign{}
સ્પેસિફિકેશન & સામાન્ય રેન્જ \\
\midrule\noalign{}
\endhead
\bottomrule\noalign{}
\endlastfoot
\textbf{બ્રાઈટનેસ} & 2000-5000 લુમેન્સ \\
\textbf{રિઝોલ્યુશન} & XGA (1024\times768) થી 4K (3840\times2160) \\
\textbf{કોન્ટ્રાસ્ટ રેશિયો} & 2000:1 થી 100,000:1 \\
\textbf{લેમ્પ લાઇફ} & 4000-8000 કલાક \\
\textbf{કનેક્ટિવિટી} & HDMI, VGA, USB, વાયરલેસ \\
\end{longtable}
}

\end{solutionbox}
\begin{mnemonicbox}
``PSCPL: પ્રિન્ટ રિઝોલ્યુશન, સ્પીડ, કનેક્ટિવિટી, પ્રોજેક્શન
બ્રાઈટનેસ, લેમ્પ લાઇફ''

\end{mnemonicbox}
\subsection*{પ્રશ્ન 4(બ) [4
ગુણ]}\label{uxaaauxab0uxab6uxaa8-4uxaac-4-uxa97uxaa3}

\textbf{ઇન્કજેટ પ્રિંટર નો બ્લોક ડાયેગ્રામ દોરો અને તેની કાર્યપધ્ધતિ ટૂંક મા સમજાવો}

\begin{solutionbox}

\textbf{ઇન્કજેટ પ્રિંટર બ્લોક ડાયાગ્રામ:}

\begin{center}
\textbf{Mermaid Diagram (Code)}
\begin{verbatim}
{Shaded}
{Highlighting}[]
graph LR
    A[પાવર સપ્લાય] {-{-}{} B[કંટ્રોલ બોર્ડ/CPU]}
    B {-{-}{} C[પેપર ફીડ મોટર]}
    B {-{-}{} D[પ્રિન્ટહેડ મોટર]}
    B {-{-}{} E[પ્રિન્ટહેડ કંટ્રોલર]}
    E {-{-}{} F[ઇન્ક કાર્ટ્રિજ]}
    F {-{-}{} G[પ્રિન્ટહેડ નોઝલ્સ]}
    B {-{-}{} H[ઇનપુટ ઇન્ટરફેસ]}
    I[કમ્પ્યુટર] {-{-}{} H}
    C {-{-}{} J[પેપર ફીડ મેકેનિઝમ]}
    D {-{-}{} K[કેરેજ એસેમ્બલી]}
    K {-{-}{} F}
    B {-{-}{} L[સેન્સર્સ]}
    L {-{-}{} M[પેપર સેન્સર્સ]}
    L {-{-}{} N[ઇન્ક લેવલ સેન્સર્સ]}
{Highlighting}
{Shaded}
\end{verbatim}
\end{center}

\textbf{ઇન્કજેટ પ્રિંટરની કાર્યપદ્ધતિ:}

\begin{enumerate}
\tightlist
\item
  \textbf{ડોક્યુમેન્ટ પ્રોસેસિંગ}: કંટ્રોલ બોર્ડ ડેટા મેળવે છે અને પ્રિન્ટર કમાન્ડમાં
  રૂપાંતરિત કરે છે
\item
  \textbf{પેપર લોડિંગ}: ફીડ મોટર ટ્રેમાંથી પેપર ખેંચે છે
\item
  \textbf{પ્રિન્ટિંગ}: પ્રિન્ટહેડ પેપર પર ચાલે છે અને નાના ઇન્ક ડ્રોપલેટ્સ છોડે છે
\item
  \textbf{ડ્રોપલેટ ફોર્મેશન}: થર્મલ અથવા પિઝોઇલેક્ટ્રિક પદ્ધતિ દ્વારા ઇન્ક ડ્રોપલેટ્સને
  પેપર પર મોકલે છે
\item
  \textbf{પેપર એડવાન્સમેન્ટ}: પ્રિન્ટિંગ પૂર્ણ થાય ત્યાં સુધી પેપર લાઇન બાય લાઇન આગળ
  વધે છે
\end{enumerate}

\end{solutionbox}
\begin{mnemonicbox}
``PIPES: પેપર ફીડ્સ, ઇન્ક ઇજેક્ટ્સ, પ્રિન્ટહેડ મૂવ્સ, ઇલેક્ટ્રોનિક
કંટ્રોલ, શીટ એડવાન્સીસ''

\end{mnemonicbox}
\subsection*{પ્રશ્ન 4(ક) [7
ગુણ]}\label{uxaaauxab0uxab6uxaa8-4uxa95-7-uxa97uxaa3}

\textbf{ફોટોકોપીયર ની કાર્યપધ્ધતિ બ્લોક ડાયેગ્રામ સાથે સમજાવો અને તેના ટેકનીકલ
સ્પેસીફીકેશનો લખો.}

\begin{solutionbox}

\textbf{ફોટોકોપીયર બ્લોક ડાયાગ્રામ:}

\begin{center}
\textbf{Mermaid Diagram (Code)}
\begin{verbatim}
{Shaded}
{Highlighting}[]
graph TD
    A[કંટ્રોલ પેનલ] {-{-}{} B[મેઇન કંટ્રોલ બોર્ડ]}
    B {-{-}{} C[સ્કેનિંગ સિસ્ટમ]}
    C {-{-}{} D[લાઇટ સોર્સ]}
    C {-{-}{} E[મિરર્સ અને લેન્સ]}
    C {-{-}{} F[CCD/ઇમેજ સેન્સર]}
    B {-{-}{} G[ઇમેજિંગ સિસ્ટમ]}
    G {-{-}{} H[ફોટોસેન્સિટિવ ડ્રમ]}
    G {-{-}{} I[ચાર્જિંગ કોરોના]}
    G {-{-}{} J[ડેવેલપિંગ યુનિટ]}
    G {-{-}{} K[ટ્રાન્સફર કોરોના]}
    G {-{-}{} L[ફ્યુઝિંગ યુનિટ]}
    B {-{-}{} M[પેપર ફીડ સિસ્ટમ]}
    M {-{-}{} N[પેપર ટ્રે]}
    M {-{-}{} O[ફીડ રોલર્સ]}
    M {-{-}{} P[રજિસ્ટ્રેશન રોલર્સ]}
    B {-{-}{} Q[પાવર સપ્લાય]}
{Highlighting}
{Shaded}
\end{verbatim}
\end{center}

\textbf{ફોટોકોપીયરની કાર્યપદ્ધતિ:}

\begin{enumerate}
\tightlist
\item
  \textbf{ચાર્જિંગ}: ફોટોસેન્સિટિવ ડ્રમને યુનિફોર્મ ઇલેક્ટ્રોસ્ટેટિક ચાર્જ આપવામાં આવે છે
\item
  \textbf{એક્સપોઝર}: ઓરિજિનલ ડોક્યુમેન્ટ સ્કેન થાય છે, ડ્રમ પર પ્રકાશ પેટર્ન બનાવે છે
\item
  \textbf{ડેવેલપિંગ}: ટોનર કણો ડ્રમ પર ચાર્જ કરેલા ક્ષેત્રો તરફ આકર્ષાય છે
\item
  \textbf{ટ્રાન્સફર}: ટોનર ઇમેજ ડ્રમ પરથી પેપર પર ટ્રાન્સફર થાય છે
\item
  \textbf{ફ્યુઝિંગ}: હીટ અને પ્રેશરથી ટોનર કાયમી રીતે પેપર પર ફિક્સ થાય છે
\item
  \textbf{ક્લીનિંગ}: આગલા સાયકલ માટે ડ્રમ સાફ કરવામાં આવે છે
\end{enumerate}

\textbf{ટેકનિકલ સ્પેસિફિકેશન્સ:}

\begin{itemize}
\tightlist
\item
  \textbf{સ્પીડ}: 20-60 પેજ પ્રતિ મિનિટ
\item
  \textbf{રિઝોલ્યુશન}: 600-1200 dpi
\item
  \textbf{પેપર કેપેસિટી}: 250-2000 શીટ્સ
\item
  \textbf{મેક્સિમમ પેપર સાઇઝ}: A3/11\times17 ઇંચ
\item
  \textbf{ઝૂમ રેન્જ}: 25-400\%
\item
  \textbf{મેમરી}: 512MB-2GB
\item
  \textbf{કનેક્ટિવિટી}: ઇથરનેટ, USB, Wi-Fi
\end{itemize}

\end{solutionbox}
\begin{mnemonicbox}
``CETFC: ચાર્જ ડ્રમ, એક્સપોઝ ઇમેજ, ટ્રાન્સફર ટોનર, ફ્યુઝ
પર્મેનન્ટલી, ક્લીન ડ્રમ''

\end{mnemonicbox}
\subsection*{પ્રશ્ન 4(અ OR) [3
ગુણ]}\label{uxaaauxab0uxab6uxaa8-4uxa85-or-3-uxa97uxaa3}

\textbf{CCTV ઉપર ટૂંક નોંધ લખો.}

\begin{solutionbox}


{\def\LTcaptype{none} % do not increment counter
\vspace{-5pt}
\captionof{table}{CCTV સિસ્ટમ ઓવરવ્યુ}
\vspace{-10pt}
\begin{longtable}[]{@{}ll@{}}
\toprule\noalign{}
પાસું & વર્ણન \\
\midrule\noalign{}
\endhead
\bottomrule\noalign{}
\endlastfoot
\textbf{ફુલ ફોર્મ} & ક્લોઝ્ડ-સર્કિટ ટેલિવિઝન \\
\textbf{હેતુ} & સિક્યુરિટી મોનિટરિંગ અને સર્વેલન્સ \\
\textbf{કોમ્પોનન્ટ્સ} & કેમેરા, DVR/NVR, મોનિટર્સ, કેબલ્સ, પાવર સપ્લાય \\
\textbf{પ્રકારો} & એનાલોગ, IP (ડિજિટલ), વાયરલેસ, HD-CVI/TVI/SDI \\
\textbf{ફીચર્સ} & મોશન ડિટેક્શન, નાઇટ વિઝન, રિમોટ વ્યુઇંગ \\
\end{longtable}
}

\textbf{કી એપ્લિકેશન્સ:}

\begin{itemize}
\tightlist
\item
  બિલ્ડિંગ્સનું સિક્યુરિટી મોનિટરિંગ
\item
  ટ્રાફિક મોનિટરિંગ
\item
  રિટેલ લોસ પ્રિવેન્શન
\item
  પબ્લિક એરિયા સર્વેલન્સ
\item
  હોમ સિક્યુરિટી
\end{itemize}

\end{solutionbox}
\begin{mnemonicbox}
``SCRAM: સિક્યુરિટી મોનિટરિંગ, ક્લોઝ્ડ સર્કિટ, રેકોર્ડિંગ
ફુટેજ, એક્સેસ રેસ્ટ્રિક્ટેડ, મોનિટરિંગ કન્ટિન્યુઅસ''

\end{mnemonicbox}
\subsection*{પ્રશ્ન 4(બ OR) [4
ગુણ]}\label{uxaaauxab0uxab6uxaa8-4uxaac-or-4-uxa97uxaa3}

\textbf{એલસીડી પ્રોજેક્ટર ની કાર્યપધ્ધતિ બ્લોક ડાયેગ્રામ સાથે સમજાવો}

\begin{solutionbox}

\textbf{LCD પ્રોજેક્ટર બ્લોક ડાયાગ્રામ:}

\begin{center}
\textbf{Mermaid Diagram (Code)}
\begin{verbatim}
{Shaded}
{Highlighting}[]
graph LR
    A[પાવર સપ્લાય] {-{-}{} B[કંટ્રોલ સર્કિટ]}
    B {-{-}{} C[લેમ્પ/લાઇટ સોર્સ]}
    C {-{-}{} D[કૂલિંગ સિસ્ટમ]}
    C {-{-}{} E[રિફ્લેક્ટર]}
    E {-{-}{} F[કન્ડેન્સર લેન્સ]}
    F {-{-}{} G[ડિક્રોઇક મિરર્સ]}
    G {-{-}{}|લાલ| H[રેડ LCD પેનલ]}
    G {-{-}{}|લીલો| I[ગ્રીન LCD પેનલ]}
    G {-{-}{}|વાદળી| J[બ્લુ LCD પેનલ]}
    H {-{-}{} K[કમ્બાઇનિંગ પ્રિઝમ]}
    I {-{-}{} K}
    J {-{-}{} K}
    K {-{-}{} L[પ્રોજેક્શન લેન્સ]}
    L {-{-}{} M[સ્ક્રીન]}
    B {-{-}{} N[ઇનપુટ ઇન્ટરફેસ]}
    B {-{-}{} O[કીસ્ટોન કરેક્શન]}
    B {-{-}{} P[ફોકસ કંટ્રોલ]}
{Highlighting}
{Shaded}
\end{verbatim}
\end{center}

\textbf{LCD પ્રોજેક્ટરની કાર્યપદ્ધતિ:}

\begin{enumerate}
\tightlist
\item
  \textbf{લાઇટ જનરેશન}: હાઇ-ઇન્ટેન્સિટી લેમ્પ સફેદ પ્રકાશ ઉત્પન્ન કરે છે
\item
  \textbf{કલર સેપરેશન}: ડિક્રોઇક મિરર્સ પ્રકાશને RGB કોમ્પોનન્ટ્સમાં વિભાજિત કરે છે
\item
  \textbf{ઇમેજ ફોર્મેશન}: LCD પેનલ્સ ઇનપુટ સિગ્નલના આધારે પ્રકાશને મોડ્યુલેટ કરે છે
\item
  \textbf{રિકમ્બિનેશન}: પ્રિઝમ RGB ઇમેજને ફુલ-કલર ઇમેજમાં જોડે છે
\item
  \textbf{પ્રોજેક્શન}: લેન્સ સિસ્ટમ અંતિમ ઇમેજને સ્ક્રીન પર પ્રોજેક્ટ કરે છે
\end{enumerate}

\end{solutionbox}
\begin{mnemonicbox}
``LSCIP: લાઇટ સોર્સ જનરેટ્સ, સ્પ્લિટ ઇન્ટુ કલર્સ, કંટ્રોલ વિથ
LCDs, ઇમેજ કંબાઇન્ડ, પ્રોજેક્ટેડ ઓન સ્ક્રીન''

\end{mnemonicbox}
\subsection*{પ્રશ્ન 4(ક OR) [7
ગુણ]}\label{uxaaauxab0uxab6uxaa8-4uxa95-or-7-uxa97uxaa3}

\textbf{લેસર પ્રિંટર ની કાર્યપધ્ધતિ બ્લોક ડાયેગ્રામ સાથે સમજાવો}

\begin{solutionbox}

\textbf{લેસર પ્રિંટર બ્લોક ડાયાગ્રામ:}

\begin{center}
\textbf{Mermaid Diagram (Code)}
\begin{verbatim}
{Shaded}
{Highlighting}[]
graph TD
    A[કંટ્રોલ બોર્ડ] {-{-}{} B[લેસર ડાયોડ]}
    A {-{-}{} C[પોલીગોન મિરર મોટર]}
    B {-{-}{} D[પોલીગોન મિરર]}
    D {-{-}{} E[ફોકસિંગ લેન્સ]}
    E {-{-}{} F[ફોટોસેન્સિટિવ ડ્રમ]}
    A {-{-}{} G[પ્રાઇમરી કોરોના]}
    G {-{-}{} F}
    A {-{-}{} H[ડેવેલપર યુનિટ]}
    H {-{-}{} F}
    A {-{-}{} I[ટ્રાન્સફર કોરોના]}
    I {-{-}{} F}
    A {-{-}{} J[ફ્યુઝિંગ યુનિટ]}
    A {-{-}{} K[પેપર ફીડ મેકેનિઝમ]}
    K {-{-}{} L[પેપર પાથ]}
    L {-{-}{} J}
    A {-{-}{} M[પાવર સપ્લાય]}
    A {-{-}{} N[ઇન્ટરફેસ]}
{Highlighting}
{Shaded}
\end{verbatim}
\end{center}

\textbf{લેસર પ્રિન્ટિંગ પ્રોસેસ:}


{\def\LTcaptype{none} % do not increment counter
\vspace{-5pt}
\captionof{table}{લેસર પ્રિન્ટિંગના છ સ્ટેપ્સ}
\vspace{-10pt}
\begin{longtable}[]{@{}llll@{}}
\toprule\noalign{}
સ્ટેપ & પ્રક્રિયા & કોમ્પોનન્ટ & ફંક્શન \\
\midrule\noalign{}
\endhead
\bottomrule\noalign{}
\endlastfoot
1 & \textbf{ક્લીનિંગ} & ક્લીનિંગ બ્લેડ & ડ્રમ પરથી બાકી ટોનર દૂર કરે છે \\
2 & \textbf{ચાર્જિંગ} & પ્રાઇમરી કોરોના & ડ્રમને યુનિફોર્મ નેગેટિવ ચાર્જ આપે છે \\
3 & \textbf{રાઇટિંગ} & લેસર અને મિરર & ડ્રમ પર ઇલેક્ટ્રોસ્ટેટિક ઇમેજ બનાવે છે \\
4 & \textbf{ડેવેલપિંગ} & ડેવેલપર યુનિટ & ડ્રમના ચાર્જ કરેલા ક્ષેત્રોમાં ટોનર લગાવે
છે \\
5 & \textbf{ટ્રાન્સફરિંગ} & ટ્રાન્સફર કોરોના & ડ્રમથી પેપર પર ટોનર ખસેડે છે \\
6 & \textbf{ફ્યુઝિંગ} & ફ્યુઝર યુનિટ & ટોનરને કાયમી રીતે પેપર પર પિગળાવે છે \\
\end{longtable}
}

\textbf{ટેકનિકલ સ્પેસિફિકેશન્સ:}

\begin{itemize}
\tightlist
\item
  \textbf{પ્રિન્ટ સ્પીડ}: 20-50 ppm
\item
  \textbf{રિઝોલ્યુશન}: 600-2400 dpi
\item
  \textbf{મેમરી}: 128MB-1GB
\item
  \textbf{ડ્યુટી સાયકલ}: 10,000-150,000 પેજ/મહિનો
\item
  \textbf{કનેક્ટિવિટી}: USB, ઇથરનેટ, Wi-Fi
\end{itemize}

\end{solutionbox}
\begin{mnemonicbox}
``CCWDTF: ક્લીન ડ્રમ, ચાર્જ યુનિફોર્મલી, રાઇટ વિથ લેસર,
ડેવેલપ વિથ ટોનર, ટ્રાન્સફર ટુ પેપર, ફ્યુઝ પર્મેનન્ટલી''

\end{mnemonicbox}
\subsection*{પ્રશ્ન 5(અ) [3
ગુણ]}\label{uxaaauxab0uxab6uxaa8-5uxa85-3-uxa97uxaa3}

\textbf{વ્યાખ્યા આપો: (૧) પીચ (૨) રીવબર્રેશન (3) માઇક્રોફોન}

\begin{solutionbox}


{\def\LTcaptype{none} % do not increment counter
\vspace{-5pt}
\captionof{table}{ઓડિઓ ટર્મિનોલોજી}
\vspace{-10pt}
\begin{longtable}[]{@{}
  >{\raggedright\arraybackslash}p{(\linewidth - 4\tabcolsep) * \real{0.1935}}
  >{\raggedright\arraybackslash}p{(\linewidth - 4\tabcolsep) * \real{0.3871}}
  >{\raggedright\arraybackslash}p{(\linewidth - 4\tabcolsep) * \real{0.4194}}@{}}
\toprule\noalign{}
\begin{minipage}[b]{\linewidth}\raggedright
પદ
\end{minipage} & \begin{minipage}[b]{\linewidth}\raggedright
વ્યાખ્યા
\end{minipage} & \begin{minipage}[b]{\linewidth}\raggedright
માપન એકમ
\end{minipage} \\
\midrule\noalign{}
\endhead
\bottomrule\noalign{}
\endlastfoot
\textbf{પીચ} & ધ્વનિની અનુભવાતી આવૃત્તિ; ટોન કેટલો ઊંચો કે નીચો લાગે છે & હર્ટ્ઝ
(Hz) \\
\textbf{રીવબર્રેશન} & સ્ત્રોત બંધ થયા પછી ધ્વનિનું સાતત્ય; પરાવર્તનને કારણે થાય છે &
સેકન્ડ (RT60) \\
\textbf{માઇક્રોફોન} & ટ્રાન્સડ્યુસર જે ધ્વનિ તરંગોને ઇલેક્ટ્રિકલ સિગ્નલમાં રૂપાંતરિત કરે
છે & સેન્સિટિવિટી dB/mV/Pa માં \\
\end{longtable}
}

\end{solutionbox}
\begin{mnemonicbox}
``PRM: પીચ એટલે ફ્રિક્વન્સી, રીવબર્રેશન એટલે રિફ્લેક્શન,
માઇક્રોફોન એટલે કન્વર્ટર''

\end{mnemonicbox}
\subsection*{પ્રશ્ન 5(બ) [4
ગુણ]}\label{uxaaauxab0uxab6uxaa8-5uxaac-4-uxa97uxaa3}

\textbf{પીએ સિસ્ટમનો બ્લોક ડાયેગ્રામ દોરો અને સમજાવો}

\begin{solutionbox}

\textbf{PA સિસ્ટમ બ્લોક ડાયાગ્રામ:}

\begin{center}
\textbf{Mermaid Diagram (Code)}
\begin{verbatim}
{Shaded}
{Highlighting}[]
graph LR
    A[માઇક્રોફોન] {-{-}{} B[પ્રી{-}એમ્પ્લિફાયર]}
    B {-{-}{} C[મિક્સર]}
    D[ઓડિઓ સોર્સ] {-{-}{} C}
    E[ઇક્વલાઇઝર] {-{-}{} C}
    C {-{-}{} F[પાવર એમ્પ્લિફાયર]}
    F {-{-}{} G[સ્પીકર સિસ્ટમ]}
    H[કંટ્રોલ સિસ્ટમ] {-{-}{} C}
    H {-{-}{} F}
{Highlighting}
{Shaded}
\end{verbatim}
\end{center}


{\def\LTcaptype{none} % do not increment counter
\vspace{-5pt}
\captionof{table}{PA સિસ્ટમ કોમ્પોનન્ટ્સ}
\vspace{-10pt}
\begin{longtable}[]{@{}ll@{}}
\toprule\noalign{}
કોમ્પોનન્ટ & ફંક્શન \\
\midrule\noalign{}
\endhead
\bottomrule\noalign{}
\endlastfoot
\textbf{માઇક્રોફોન} & અવાજ કેપ્ચર કરે છે અને ઇલેક્ટ્રિકલ સિગ્નલમાં કન્વર્ટ કરે છે \\
\textbf{પ્રી-એમ્પ્લિફાયર} & નબળા માઇક્રોફોન સિગ્નલને લાઇન લેવલ સુધી બૂસ્ટ કરે છે \\
\textbf{મિક્સર} & મલ્ટિપલ ઓડિઓ સોર્સ કમ્બાઇન કરે છે, લેવલ્સ એડજસ્ટ કરે છે \\
\textbf{ઇક્વલાઇઝર} & શ્રેષ્ઠ સાઉન્ડ માટે ફ્રિક્વન્સી રિસ્પોન્સ એડજસ્ટ કરે છે \\
\textbf{પાવર એમ્પ્લિફાયર} & સ્પીકર્સને ડ્રાઇવ કરવા માટે સિગ્નલ સ્ટ્રેન્થ વધારે છે \\
\textbf{સ્પીકર સિસ્ટમ} & ઇલેક્ટ્રિકલ સિગ્નલને પાછા ધ્વનિ તરંગોમાં કન્વર્ટ કરે છે \\
\end{longtable}
}

\end{solutionbox}
\begin{mnemonicbox}
``MPMEPA: માઇક્રોફોન પિક્સ, પ્રીએમ્પ મેગ્નિફાઇઝ, ઇક્વલાઇઝર
એડજસ્ટ્સ, પાવર એમ્પ્લિફાયર ડ્રાઇવ્સ, ઓડિયન્સ હિયર્સ''

\end{mnemonicbox}
\subsection*{પ્રશ્ન 5(ક) [7
ગુણ]}\label{uxaaauxab0uxab6uxaa8-5uxa95-7-uxa97uxaa3}

\textbf{ક્રિસ્ટલ માઇક્રોફોન સમજાવો.}

\begin{solutionbox}


{\def\LTcaptype{none} % do not increment counter
\vspace{-5pt}
\captionof{table}{ક્રિસ્ટલ માઇક્રોફોન ખાસિયતો}
\vspace{-10pt}
\begin{longtable}[]{@{}ll@{}}
\toprule\noalign{}
ખાસિયત & વર્ણન \\
\midrule\noalign{}
\endhead
\bottomrule\noalign{}
\endlastfoot
\textbf{ઓપરેટિંગ પ્રિન્સિપલ} & પિએઝોઇલેક્ટ્રિક ઇફેક્ટ \\
\textbf{રચના} & મેટલ પ્લેટ્સ વચ્ચે ક્રિસ્ટલ એલિમેન્ટ (રોશેલ સોલ્ટ) \\
\textbf{રિસ્પોન્સ} & હાઇ આઉટપુટ, મોડરેટ ફ્રિક્વન્સી રિસ્પોન્સ \\
\textbf{ઇમ્પીડન્સ} & ખૂબ ઊંચી (સામાન્ય રીતે \textgreater{} 1 MΩ) \\
\textbf{ટકાઉપણું} & હીટ અને ભેજ પ્રત્યે સંવેદનશીલ \\
\end{longtable}
}

\textbf{કાર્યપ્રણાલી:} જ્યારે ધ્વનિ તરંગો ડાયાફ્રામ પર આઘાત કરે છે, ત્યારે તેઓ
ક્રિસ્ટલ એલિમેન્ટ પર દબાણ ઉત્પન્ન કરે છે. પિએઝોઇલેક્ટ્રિક અસરને કારણે, ક્રિસ્ટલ મિકેનિકલ
સ્ટ્રેસના પ્રમાણમાં વોલ્ટેજ ઉત્પન્ન કરે છે. આ વોલ્ટેજ ધ્વનિનું ઇલેક્ટ્રિકલ પ્રતિનિધિત્વ છે.

\begin{center}
\textbf{Mermaid Diagram (Code)}
\begin{verbatim}
{Shaded}
{Highlighting}[]
graph LR
    A[ધ્વનિ તરંગો] {-{-}{} B[ડાયાફ્રામ]}
    B {-{-}{} C[ક્રિસ્ટલ પર યાંત્રિક તણાવ]}
    C {-{-}{} D[પિએઝોઇલેક્ટ્રિક ઇફેક્ટ]}
    D {-{-}{} E[વોલ્ટેજ જનરેશન]}
    E {-{-}{} F[ઇલેક્ટ્રિકલ આઉટપુટ]}
{Highlighting}
{Shaded}
\end{verbatim}
\end{center}

\textbf{એપ્લિકેશન્સ:}

\begin{itemize}
\tightlist
\item
  ટેલિફોન રિસીવર્સ
\item
  એકુસ્ટિક ઇન્સ્ટ્રુમેન્ટ્સ માટે કોન્ટેક્ટ પિકઅપ્સ
\item
  ઓછી કિંમતના રેકોર્ડિંગ ડિવાઇસીસ
\item
  પબ્લિક એડ્રેસ સિસ્ટમ્સ
\end{itemize}

\textbf{ફાયદા અને મર્યાદાઓ:}

{\def\LTcaptype{none} % do not increment counter
\begin{longtable}[]{@{}ll@{}}
\toprule\noalign{}
ફાયદા & મર્યાદાઓ \\
\midrule\noalign{}
\endhead
\bottomrule\noalign{}
\endlastfoot
ઉચ્ચ આઉટપુટ વોલ્ટેજ & નબળી ફ્રિક્વન્સી રિસ્પોન્સ \\
બાહ્ય પાવર જરૂરી નથી & તાપમાન/ભેજ પ્રત્યે સંવેદનશીલ \\
સરળ રચના & ઉચ્ચ ડિસ્ટોર્શન \\
ઓછી કિંમત & નાજુક ક્રિસ્ટલ એલિમેન્ટ \\
\end{longtable}
}

\end{solutionbox}
\begin{mnemonicbox}
``PIES: પ્રેશર અપ્લાઇડ, ઇમ્પીડન્સ હાઇ, ઇલેક્ટ્રિસિટી જનરેટેડ,
સાઉન્ડ કન્વર્ટેડ''

\end{mnemonicbox}
\subsection*{પ્રશ્ન 5(અ OR) [3
ગુણ]}\label{uxaaauxab0uxab6uxaa8-5uxa85-or-3-uxa97uxaa3}

\textbf{હોમ થીયેટર સાઉંડ સિસ્ટમ નો બ્લોક ડાયેગ્રામ દોરો.}

\begin{solutionbox}

\textbf{હોમ થીયેટર સાઉન્ડ સિસ્ટમ બ્લોક ડાયાગ્રામ:}

\begin{center}
\textbf{Mermaid Diagram (Code)}
\begin{verbatim}
{Shaded}
{Highlighting}[]
graph TD
    A[ઓડિઓ/વિડિઓ સોર્સ] {-{-}{} B[AV રિસીવર/એમ્પ્લિફાયર]}
    B {-{-}{} C[ફ્રન્ટ લેફ્ટ સ્પીકર]}
    B {-{-}{} D[સેન્ટર સ્પીકર]}
    B {-{-}{} E[ફ્રન્ટ રાઇટ સ્પીકર]}
    B {-{-}{} F[સરાઉન્ડ લેફ્ટ સ્પીકર]}
    B {-{-}{} G[સરાઉન્ડ રાઇટ સ્પીકર]}
    B {-{-}{} H[સબવૂફર]}
    I[રિમોટ કંટ્રોલ] {-{-}{} B}
    J[TV/ડિસ્પ્લે] {-{-}{} B}
    B {-{-}{} J}
    K[સ્ટ્રીમિંગ મોડ્યુલ] {-{-}{} B}
{Highlighting}
{Shaded}
\end{verbatim}
\end{center}

\end{solutionbox}
\begin{mnemonicbox}
``SAVS: સોર્સ પ્રોવાઇડ્સ, એમ્પ્લિફાયર પ્રોસેસીસ, વેરિયસ
સ્પીકર્સ ડિલિવર, સરાઉન્ડ એક્સપીરિયન્સ ક્રિએટેડ''

\end{mnemonicbox}
\subsection*{પ્રશ્ન 5(બ OR) [4
ગુણ]}\label{uxaaauxab0uxab6uxaa8-5uxaac-or-4-uxa97uxaa3}

\textbf{ઓપ્ટિકલ સાઉન્ડ રેકોર્ડિંગ સમજાવો.}

\begin{solutionbox}


{\def\LTcaptype{none} % do not increment counter
\vspace{-5pt}
\captionof{table}{ઓપ્ટિકલ સાઉન્ડ રેકોર્ડિંગ પ્રક્રિયા}
\vspace{-10pt}
\begin{longtable}[]{@{}
  >{\raggedright\arraybackslash}p{(\linewidth - 4\tabcolsep) * \real{0.2308}}
  >{\raggedright\arraybackslash}p{(\linewidth - 4\tabcolsep) * \real{0.3462}}
  >{\raggedright\arraybackslash}p{(\linewidth - 4\tabcolsep) * \real{0.4231}}@{}}
\toprule\noalign{}
\begin{minipage}[b]{\linewidth}\raggedright
સ્ટેપ
\end{minipage} & \begin{minipage}[b]{\linewidth}\raggedright
પ્રક્રિયા
\end{minipage} & \begin{minipage}[b]{\linewidth}\raggedright
કોમ્પોનન્ટ
\end{minipage} \\
\midrule\noalign{}
\endhead
\bottomrule\noalign{}
\endlastfoot
1 & \textbf{સાઉન્ડ કેપ્ચર} & માઇક્રોફોન ધ્વનિને ઇલેક્ટ્રિકલ સિગ્નલમાં રૂપાંતરિત કરે
છે \\
2 & \textbf{મોડ્યુલેશન} & સિગ્નલ લાઇટ સોર્સની તીવ્રતા અથવા એરિયા મોડ્યુલેટ કરે
છે \\
3 & \textbf{એક્સપોઝર} & મોડ્યુલેટેડ લાઇટ ફોટોગ્રાફિક ફિલ્મને એક્સપોઝ કરે છે \\
4 & \textbf{ડેવેલપમેન્ટ} & દૃશ્યમાન સાઉન્ડ ટ્રેક બનાવવા માટે ફિલ્મ પ્રોસેસ કરવામાં આવે
છે \\
5 & \textbf{પ્લેબેક} & લાઇટ ટ્રેક મારફતે પસાર થાય છે, ફોટોડિટેક્ટર ઇલેક્ટ્રિકલ
સિગ્નલમાં રૂપાંતરિત કરે છે \\
\end{longtable}
}

\textbf{ઓપ્ટિકલ સાઉન્ડ ટ્રેક્સના પ્રકારો:}

\begin{itemize}
\tightlist
\item
  \textbf{વેરિએબલ ડેન્સિટી}: લાઇટની તીવ્રતા બદલાય છે (ઘાટા/પાતળા ક્ષેત્રો)
\item
  \textbf{વેરિએબલ એરિયા}: અપારદર્શક પૃષ્ઠભૂમિ સામે પારદર્શક ક્ષેત્રની પહોળાઈ બદલાય
  છે
\end{itemize}

\begin{center}
\textbf{Mermaid Diagram (Code)}
\begin{verbatim}
{Shaded}
{Highlighting}[]
graph LR
    A[સાઉન્ડ ઇનપુટ] {-{-}{} B[માઇક્રોફોન]}
    B {-{-}{} C[એમ્પ્લિફાયર]}
    C {-{-}{} D[લાઇટ મોડ્યુલેટર]}
    E[લાઇટ સોર્સ] {-{-}{} D}
    D {-{-}{} F[ઓપ્ટિકલ સિસ્ટમ]}
    F {-{-}{} G[મુવિંગ ફિલ્મ]}
    H[ડેવેલપ કરેલી ફિલ્મ] {-{-}{} I[પ્લેબેક લાઇટ સોર્સ]}
    I {-{-}{} J[ફોટોસેલ/ડિટેક્ટર]}
    J {-{-}{} K[એમ્પ્લિફાયર]}
    K {-{-}{} L[સ્પીકર]}
{Highlighting}
{Shaded}
\end{verbatim}
\end{center}

\end{solutionbox}
\begin{mnemonicbox}
``CAREP: કેપ્ચર સાઉન્ડ, એમ્પ્લિફાય સિગ્નલ, રેકોર્ડ ઓપ્ટિકલી,
એક્સપોઝ ફિલ્મ, પ્લે બેક''

\end{mnemonicbox}
\subsection*{પ્રશ્ન 5(ક OR) [7
ગુણ]}\label{uxaaauxab0uxab6uxaa8-5uxa95-or-7-uxa97uxaa3}

\textbf{લાઉડસ્પીકર ની વ્યાખ્યા આપો. લાઉડસ્પીકર ના પ્રકારો લખો અને કોઇ પણ એક
લાઉડસ્પીકર ની કાર્યપધ્ધતિ સમજાવો.}

\begin{solutionbox}

\textbf{વ્યાખ્યા:} લાઉડસ્પીકર એ ઇલેક્ટ્રોએકુસ્ટિક ટ્રાન્સડ્યુસર છે જે ઇલેક્ટ્રિકલ સિગ્નલને
ધ્વનિ તરંગોમાં રૂપાંતરિત કરે છે, જેમાં ડાયાફ્રામ હલનચલન કરીને વાયુના દબાણમાં ફેરફાર કરે
છે.


{\def\LTcaptype{none} % do not increment counter
\vspace{-5pt}
\captionof{table}{લાઉડસ્પીકરના પ્રકારો}
\vspace{-10pt}
\begin{longtable}[]{@{}
  >{\raggedright\arraybackslash}p{(\linewidth - 6\tabcolsep) * \real{0.1071}}
  >{\raggedright\arraybackslash}p{(\linewidth - 6\tabcolsep) * \real{0.3393}}
  >{\raggedright\arraybackslash}p{(\linewidth - 6\tabcolsep) * \real{0.3036}}
  >{\raggedright\arraybackslash}p{(\linewidth - 6\tabcolsep) * \real{0.2500}}@{}}
\toprule\noalign{}
\begin{minipage}[b]{\linewidth}\raggedright
પ્રકાર
\end{minipage} & \begin{minipage}[b]{\linewidth}\raggedright
કાર્યસિદ્ધાંત
\end{minipage} & \begin{minipage}[b]{\linewidth}\raggedright
ફ્રિક્વન્સી રેન્જ
\end{minipage} & \begin{minipage}[b]{\linewidth}\raggedright
એપ્લિકેશન્સ
\end{minipage} \\
\midrule\noalign{}
\endhead
\bottomrule\noalign{}
\endlastfoot
\textbf{ડાયનેમિક/મુવિંગ કોઇલ} & ઇલેક્ટ્રોમેગ્નેટિક ઇન્ડક્શન & 20Hz-20kHz & સૌથી
સામાન્ય, જનરલ પર્પઝ \\
\textbf{ઇલેક્ટ્રોસ્ટેટિક} & પ્લેટ્સ વચ્ચે ઇલેક્ટ્રોસ્ટેટિક ફોર્સ & 100Hz-20kHz &
હાઇ-ફિડેલિટી ઓડિઓ સિસ્ટમ્સ \\
\textbf{પિએઝોઇલેક્ટ્રિક} & પિએઝોઇલેક્ટ્રિક ઇફેક્ટ & 1kHz-25kHz & ટ્વીટર્સ, અલાર્મ્સ,
બઝર્સ \\
\textbf{રિબન} & મેગ્નેટિક ફિલ્ડમાં રિબન મારફતે કરંટ & 2kHz-50kHz & હાઇ-ફ્રિક્વન્સી
રિપ્રોડક્શન \\
\textbf{પ્લેનર મેગ્નેટિક} & કન્ડક્ટર શીટ પર મેગ્નેટિક ફોર્સ & 30Hz-20kHz &
ઓડિયોફાઇલ હેડફોન્સ, સ્પીકર્સ \\
\end{longtable}
}

\textbf{ડાયનેમિક/મુવિંગ કોઇલ લાઉડસ્પીકરની કાર્યપદ્ધતિ:}

\begin{center}
\textbf{Mermaid Diagram (Code)}
\begin{verbatim}
{Shaded}
{Highlighting}[]
graph LR
    A[ઓડિઓ સિગ્નલ] {-{-}{} B[વોઇસ કોઇલ]}
    B {-{-}{} C[ઇલેક્ટ્રોમેગ્નેટિક ફિલ્ડ]}
    D[પર્મેનન્ટ મેગ્નેટ] {-{-}{} C}
    C {-{-}{} E[વોઇસ કોઇલની હલનચલન]}
    E {-{-}{} F[કોન/ડાયાફ્રામનું હલનચલન]}
    F {-{-}{} G[વાયુના દબાણમાં ફેરફાર]}
    G {-{-}{} H[ધ્વનિ તરંગો]}
{Highlighting}
{Shaded}
\end{verbatim}
\end{center}

\textbf{કાર્યપદ્ધતિ:}

\begin{enumerate}
\tightlist
\item
  ઓડિઓ કરંટ વોઇસ કોઇલમાંથી પસાર થાય છે
\item
  કરંટ ઇલેક્ટ્રોમેગ્નેટિક ફિલ્ડ ઉત્પન્ન કરે છે
\item
  ઇલેક્ટ્રોમેગ્નેટિક ફિલ્ડ પર્મેનન્ટ મેગ્નેટ સાથે ઇન્ટરેક્ટ કરે છે
\item
  સિગ્નલ પોલેરિટીના આધારે વોઇસ કોઇલ આગળ/પાછળ ખસે છે
\item
  જોડાયેલ કોન/ડાયાફ્રામ ખસે છે, જે વાયુના દબાણમાં ફેરફાર કરે છે
\item
  વાયુના દબાણના ફેરફારો ધ્વનિ તરંગો તરીકે ફેલાય છે
\end{enumerate}

\textbf{કોમ્પોનન્ટ્સ:}

\begin{itemize}
\tightlist
\item
  \textbf{કોન/ડાયાફ્રામ}: ધ્વનિ ઉત્પન્ન કરવા માટે વાયુને ખસેડે છે
\item
  \textbf{વોઇસ કોઇલ}: ઓડિઓ સિગ્નલ કરંટ વહન કરે છે
\item
  \textbf{મેગ્નેટ}: સ્ટેટિક મેગ્નેટિક ફિલ્ડ ઉત્પન્ન કરે છે
\item
  \textbf{સસ્પેન્શન}: કોનને કેન્દ્રિત રાખે છે, હલનચલનની મંજૂરી આપે છે
\item
  \textbf{ફ્રેમ/બાસ્કેટ}: કોમ્પોનન્ટ્સને યોગ્ય એલાઇનમેન્ટમાં રાખે છે
\end{itemize}

\end{solutionbox}
\begin{mnemonicbox}
``SEPVADICS: સિગ્નલ એન્ટર્સ, પ્રોડ્યુસેસ વાઇબ્રેશન્સ, એક્ટિવેટ્સ
ડાયાફ્રામ, ઇન કોઓર્ડીનેશન વિથ સસ્પેન્શન''

\end{mnemonicbox}

\end{document}
