\documentclass[10pt,a4paper]{article}

% content/resources/templates/preamble.tex
\usepackage[margin=0.6in]{geometry}
\author{Milav Dabgar}
\usepackage{amsmath,amssymb,amsthm}
\usepackage{booktabs}
\usepackage{multirow}
\usepackage{xcolor}
\usepackage{tcolorbox}
\tcbuselibrary{breakable,skins}
\usepackage[colorlinks=true,linkcolor=blue]{hyperref}
\usepackage{titlesec}
\usepackage{enumitem}
\usepackage{tikz}
\usepackage{pgfplots}
\usepackage{circuitikz}
\usepackage[version=4]{mhchem}
\usepackage{longtable}
\usepackage{array}
\usepackage{float}
\usepackage{caption}
\usepackage{listings}

\lstset{
  basicstyle=\small\ttfamily,
  breaklines=true,
  breakatwhitespace=false,
  postbreak=\mbox{\textcolor{red}{$\hookrightarrow$}\space},
  float=false,
  numbers=left,
  numberstyle=\tiny\color{gray},
  numbersep=10pt,
  xleftmargin=2em,
  keywordstyle=\color{blue},
  commentstyle=\color{green!60!black},
  stringstyle=\color{purple},
  backgroundcolor=\color{gray!5},
  showstringspaces=false,
  tabsize=2,
  captionpos=b,
  keepspaces=true,
  columns=flexible
}

\pgfplotsset{compat=1.18}
\usetikzlibrary{shapes,arrows,positioning,calc,patterns,decorations.pathmorphing,decorations.markings,arrows.meta}

% Color scheme
\definecolor{headcolor}{RGB}{0,102,204}
\definecolor{keycolor}{RGB}{220,20,60}
\definecolor{solutioncolor}{RGB}{34,139,34}
\definecolor{mnemoniccolor}{RGB}{148,0,211}
\definecolor{codecolor}{RGB}{0,0,100}

% Spacing
\setlength{\parskip}{3pt}
\setlist[itemize]{nosep}
\setlist[enumerate]{nosep}

% Title formatting
\titleformat{\section}{\Large\bfseries\color{headcolor}}{\thesection}{1em}{}
\titleformat{\subsection}{\large\bfseries\color{headcolor}}{\thesubsection}{1em}{}

% Pandoc tightlist compatibility
\providecommand{\tightlist}{%
  \setlength{\itemsep}{0pt}\setlength{\parskip}{0pt}}

% Pandoc longtable compatibility
\newcounter{none}
\def\thenone{}


% content/resources/templates/english-boxes.tex
% This file is currently empty - it exists to maintain consistency with the import structure.
% Add custom environments here if needed in the future.


\begin{document}

\begin{center}
{\Huge\bfseries\color{headcolor} Subject Name Solutions}\\[5pt]
{\LARGE 4341107 -- Summer 2024}\\[3pt]
{\large Semester 1 Study Material}\\[3pt]
{\normalsize\textit{Detailed Solutions and Explanations}}
\end{center}

\vspace{10pt}

\subsection*{Question 1(a) [3 marks]}\label{q1a}

\textbf{Give definition (Only) of Loudness, Fidelity and Reverberation}

\begin{solutionbox}

\begin{itemize}
\tightlist
\item
  \textbf{Loudness}: The subjective perception of sound intensity by the
  human ear, measured in decibels (dB).
\item
  \textbf{Fidelity}: The degree to which a system reproduces sound that
  is faithful to the original input signal.
\item
  \textbf{Reverberation}: The persistence of sound after the original
  sound source has stopped, caused by multiple reflections within an
  enclosed space.
\end{itemize}

\end{solutionbox}
\begin{mnemonicbox}
``LFR: Listen Faithfully to Room echoes''

\end{mnemonicbox}
\subsection*{Question 1(b) [4 marks]}\label{q1b}

\textbf{Draw and explain block diagram of PA system}

\begin{solutionbox}

\textbf{Diagram:}

\begin{center}
\textbf{Mermaid Diagram (Code)}
\begin{verbatim}
{Shaded}
{Highlighting}[]
graph LR
    A[Microphone] {-{-}{} B[Preamplifier]}
    B {-{-}{} C[Mixer]}
    C {-{-}{} D[Power Amplifier]}
    D {-{-}{} E[Loudspeaker]}
    F[Audio Input] {-{-}{} C}
    G[Equalizer] {-{-}{} C}
{Highlighting}
{Shaded}
\end{verbatim}
\end{center}

\begin{itemize}
\tightlist
\item
  \textbf{Microphone}: Converts sound waves into electrical signals
\item
  \textbf{Preamplifier}: Boosts weak microphone signals to line level
\item
  \textbf{Mixer}: Combines multiple audio signals and adjusts levels
\item
  \textbf{Power Amplifier}: Increases signal power to drive loudspeakers
\item
  \textbf{Loudspeaker}: Converts electrical signals back into sound
  waves
\end{itemize}

\end{solutionbox}
\begin{mnemonicbox}
``MPMEL: Many People Make Excellent Listeners''

\end{mnemonicbox}
\subsection*{Question 1(c) [7 marks]}\label{q1c}

\textbf{Discuss any two characteristic of Microphone and Explain
wireless microphone}

\begin{solutionbox}

\textbf{Microphone Characteristics:}

{\def\LTcaptype{none} % do not increment counter
\begin{longtable}[]{@{}
  >{\raggedright\arraybackslash}p{(\linewidth - 2\tabcolsep) * \real{0.5357}}
  >{\raggedright\arraybackslash}p{(\linewidth - 2\tabcolsep) * \real{0.4643}}@{}}
\toprule\noalign{}
\begin{minipage}[b]{\linewidth}\raggedright
Characteristic
\end{minipage} & \begin{minipage}[b]{\linewidth}\raggedright
Description
\end{minipage} \\
\midrule\noalign{}
\endhead
\bottomrule\noalign{}
\endlastfoot
\textbf{Sensitivity} & Measures how efficiently microphone converts
acoustic pressure to electrical output (mV/Pa) \\
\textbf{Directional Pattern} & Defines pickup area (omnidirectional,
cardioid, hypercardioid, bidirectional) \\
\end{longtable}
}

\textbf{Wireless Microphone:}

\begin{center}
\textbf{Mermaid Diagram (Code)}
\begin{verbatim}
{Shaded}
{Highlighting}[]
graph LR
    A[Microphone Element] {-{-}{} B[Audio Processor]}
    B {-{-}{} C[RF Transmitter]}
    C {-{-}{}|Radio Waves| D[RF Receiver]}
    D {-{-}{} E[Audio Output]}
{Highlighting}
{Shaded}
\end{verbatim}
\end{center}

\begin{itemize}
\tightlist
\item
  \textbf{Microphone Element}: Captures sound and converts to electrical
  signals
\item
  \textbf{RF Transmitter}: Modulates audio onto radio frequency carrier
\item
  \textbf{Transmission}: Typical frequency bands are UHF (470-698 MHz)
  or VHF (174-216 MHz)
\item
  \textbf{RF Receiver}: Demodulates signal back to audio
\item
  \textbf{Advantages}: Mobility, no cable restrictions, reduces stage
  clutter
\end{itemize}

\end{solutionbox}
\begin{mnemonicbox}
``SMART: Sensitivity Measures Audio Response Truly''

\end{mnemonicbox}
\subsection*{Question 1(c) OR [7
marks]}\label{q1c}

\textbf{Discuss any two characteristics of loudspeaker and explain
permanent magnet loudspeaker.}

\begin{solutionbox}

\textbf{Loudspeaker Characteristics:}

{\def\LTcaptype{none} % do not increment counter
\begin{longtable}[]{@{}
  >{\raggedright\arraybackslash}p{(\linewidth - 2\tabcolsep) * \real{0.5357}}
  >{\raggedright\arraybackslash}p{(\linewidth - 2\tabcolsep) * \real{0.4643}}@{}}
\toprule\noalign{}
\begin{minipage}[b]{\linewidth}\raggedright
Characteristic
\end{minipage} & \begin{minipage}[b]{\linewidth}\raggedright
Description
\end{minipage} \\
\midrule\noalign{}
\endhead
\bottomrule\noalign{}
\endlastfoot
\textbf{Frequency Response} & Range of frequencies (Hz) speaker can
reproduce (typically 20Hz-20kHz) \\
\textbf{Impedance} & Electrical resistance (ohms) that affects power
transfer from amplifier (typically 4-8Ω) \\
\end{longtable}
}

\textbf{Permanent Magnet Loudspeaker:}

\begin{center}
\textbf{Mermaid Diagram (Code)}
\begin{verbatim}
{Shaded}
{Highlighting}[]
graph LR
    A[Permanent Magnet] {-{-}{} B[Voice Coil]}
    B {-{-}{} C[Cone/Diaphragm]}
    D[Audio Input] {-{-}{} B}
    C {-{-}{} E[Sound Waves]}
{Highlighting}
{Shaded}
\end{verbatim}
\end{center}

\begin{itemize}
\tightlist
\item
  \textbf{Permanent Magnet}: Creates fixed magnetic field (usually
  ferrite or neodymium)
\item
  \textbf{Voice Coil}: Wire coil that carries audio current, creating
  variable magnetic field
\item
  \textbf{Cone/Diaphragm}: Moves in response to voice coil movement
\item
  \textbf{Working Principle}: Interaction between fixed magnetic field
  and varying field from voice coil creates mechanical movement
\item
  \textbf{Advantages}: More efficient, no field coil power required,
  compact design
\end{itemize}

\end{solutionbox}
\begin{mnemonicbox}
``FIRM: Frequency Impedance Require Magnets''

\end{mnemonicbox}
\subsection*{Question 2(a) [3 marks]}\label{q2a}

\textbf{Define only: Aspect ratio, Luminance and chrominance}

\begin{solutionbox}

\begin{itemize}
\tightlist
\item
  \textbf{Aspect Ratio}: The ratio of width to height of a television
  screen (commonly 16:9 for HDTV, 4:3 for older TVs).
\item
  \textbf{Luminance}: The brightness component of a video signal that
  carries intensity information (represented as Y).
\item
  \textbf{Chrominance}: The color component of a video signal that
  carries color information (represented as U and V or Cb and Cr).
\end{itemize}

\end{solutionbox}
\begin{mnemonicbox}
``ALC: All Light Contains color''

\end{mnemonicbox}
\subsection*{Question 2(b) [4 marks]}\label{q2b}

\textbf{Draw PAL-D decoder only and explain separation of U and V
component of chroma signal.}

\begin{solutionbox}

\textbf{Diagram:}

\begin{center}
\textbf{Mermaid Diagram (Code)}
\begin{verbatim}
{Shaded}
{Highlighting}[]
graph LR
    A[Composite Video Input] {-{-}{} B[Comb Filter]}
    B {-{-}{}|Y Signal| C[Luminance Processing]}
    B {-{-}{}|Chroma Signal| D[Delay Line]}
    D {-{-}{} E[Phase Alternating Switch]}
    E {-{-}{} F[Synchronous Demodulator]}
    F {-{-}{} G[U Signal {-} Blue{-}Luminance]}
    F {-{-}{} H[V Signal {-} Red{-}Luminance]}
{Highlighting}
{Shaded}
\end{verbatim}
\end{center}

\begin{itemize}
\tightlist
\item
  \textbf{Comb Filter}: Separates luminance (Y) from chrominance signal
\item
  \textbf{Delay Line}: Delays chroma signal by one line period (64μs)
\item
  \textbf{Phase Alternating Switch}: Inverts V component on alternate
  lines
\item
  \textbf{Synchronous Demodulator}: Uses subcarrier reference to extract
  U and V components
\item
  \textbf{U Component}: Represents Blue-minus-Luminance (B-Y)
\item
  \textbf{V Component}: Represents Red-minus-Luminance (R-Y)
\end{itemize}

\end{solutionbox}
\begin{mnemonicbox}
``CODES: Chrominance Only Decodes Extracting
Signals''

\end{mnemonicbox}
\subsection*{Question 2(c) [7 marks]}\label{q2c}

\textbf{Explain in detail working of LCD television. Give any two
technical specifications of it.}

\begin{solutionbox}

\textbf{LCD Television Working:}

\begin{center}
\textbf{Mermaid Diagram (Code)}
\begin{verbatim}
{Shaded}
{Highlighting}[]
graph LR
    A[Backlight] {-{-}{} B[Polarizing Filter 1]}
    B {-{-}{} C[Liquid Crystal Layer]}
    C {-{-}{} D[Color Filter]}
    D {-{-}{} E[Polarizing Filter 2]}
    F[Video Signal] {-{-}{} G[Control Circuit]}
    G {-{-}{} H[TFT Matrix]}
    H {-{-}{} C}
{Highlighting}
{Shaded}
\end{verbatim}
\end{center}

\textbf{Working Process:}

\begin{enumerate}
\tightlist
\item
  \textbf{Backlight}: CCFL or LED provides white light source
\item
  \textbf{TFT Matrix}: Thin-film transistors control voltage to each
  pixel
\item
  \textbf{Liquid Crystal Layer}: Molecules twist based on applied
  voltage
\item
  \textbf{Polarizers}: First filter aligns light, second passes only
  rotated light
\item
  \textbf{Color Filters}: RGB filters create colored pixels
\item
  \textbf{Image Formation}: Varying voltage controls light passage
  through each pixel
\end{enumerate}

\textbf{Technical Specifications:}

\begin{itemize}
\tightlist
\item
  \textbf{Resolution}: 1920\times1080 (Full HD) or 3840\times2160 (4K UHD)
\item
  \textbf{Refresh Rate}: 60Hz, 120Hz, or 240Hz
\end{itemize}

\end{solutionbox}
\begin{mnemonicbox}
``BALTIC: Backlight Activates Liquid To Illuminate
Colors''

\end{mnemonicbox}
\subsection*{Question 2(a) OR [3
marks]}\label{q2a}

\textbf{State Grassmens law \& explain it with concept of additive
mixing.}

\begin{solutionbox}

\textbf{Grassmann's Law:} Any color can be matched by a linear
combination of three primary colors.

\textbf{Additive Color Mixing Explanation:}

\begin{center}
\textbf{Mermaid Diagram (Code)}
\begin{verbatim}
{Shaded}
{Highlighting}[]
graph TD
    A[Red Primary] {-{-}{} D[Red + Green = Yellow]}
    B[Green Primary] {-{-}{} D}
    B {-{-}{} E[Green + Blue = Cyan]}
    C[Blue Primary] {-{-}{} E}
    C {-{-}{} F[Blue + Red = Magenta]}
    A {-{-}{} F}
    D {-{-}{} G[R + G + B = White]}
    E {-{-}{} G}
    F {-{-}{} G}
{Highlighting}
{Shaded}
\end{verbatim}
\end{center}

\begin{itemize}
\tightlist
\item
  \textbf{Principle}: Adding light of different colors creates new
  colors
\item
  \textbf{Primary Colors}: Red, Green, and Blue
\item
  \textbf{Secondary Colors}: Yellow (R+G), Cyan (G+B), Magenta (B+R)
\item
  \textbf{Example}: Equal intensities of RGB create white light
\end{itemize}

\end{solutionbox}
\begin{mnemonicbox}
``RGB-ACM: Red Green Blue - Additive Creates More''

\end{mnemonicbox}
\subsection*{Question 2(b) OR [4
marks]}\label{q2b}

\textbf{Draw block diagram of DTH receiver and explain it.}

\begin{solutionbox}

\textbf{Diagram:}

\begin{center}
\textbf{Mermaid Diagram (Code)}
\begin{verbatim}
{Shaded}
{Highlighting}[]
graph LR
    A[Satellite Dish] {-{-}{} B[LNB]}
    B {-{-}{} C[Tuner]}
    C {-{-}{} D[Demodulator]}
    D {-{-}{} E[MPEG Decoder]}
    E {-{-}{} F[Video/Audio Processor]}
    F {-{-}{} G[TV Output]}
    H[Conditional Access Module] {-{-}{} E}
    I[Smart Card] {-{-}{} H}
{Highlighting}
{Shaded}
\end{verbatim}
\end{center}

\begin{itemize}
\tightlist
\item
  \textbf{Satellite Dish}: Collects weak satellite signals (10.7-12.75
  GHz)
\item
  \textbf{LNB} (Low Noise Block): Amplifies and converts signal to lower
  frequency (950-2150 MHz)
\item
  \textbf{Tuner}: Selects desired transponder frequency
\item
  \textbf{Demodulator}: Extracts digital data from carrier signal
\item
  \textbf{MPEG Decoder}: Decompresses audio/video data
\item
  \textbf{CAM \& Smart Card}: Provide decryption and subscription
  verification
\item
  \textbf{Output}: Processes signals for display on television
\end{itemize}

\end{solutionbox}
\begin{mnemonicbox}
``SLTD-MCS: Satellites Link Through Decoders Making
Clear Signals''

\end{mnemonicbox}
\subsection*{Question 2(c) OR [7
marks]}\label{q2c}

\textbf{State following frequency/standard (used in color TV system)}

\begin{solutionbox}

{\def\LTcaptype{none} % do not increment counter
\begin{longtable}[]{@{}ll@{}}
\toprule\noalign{}
Parameter & Frequency/Standard \\
\midrule\noalign{}
\endhead
\bottomrule\noalign{}
\endlastfoot
\textbf{VIF (Video Intermediate Frequency)} & 38.9 MHz (PAL-B/G) \\
\textbf{SIF (Sound Intermediate Frequency)} & 33.4 MHz (PAL-B/G) \\
\textbf{Color Sub-carrier Frequency} & 4.43361875 MHz (PAL) \\
\textbf{Vertical Blanking Frequency} & 50 Hz (PAL) \\
\textbf{Horizontal Synchronizing Frequency} & 15.625 kHz (PAL) \\
\textbf{Inter Carrier Sound Signal Frequency} & 5.5 MHz (PAL-B/G) \\
\textbf{One Channel Bandwidth} & 7 MHz (VHF), 8 MHz (UHF) \\
\end{longtable}
}

\end{solutionbox}
\begin{mnemonicbox}
``Very Special Colors Vertically Harmonize In One
Channel''

\end{mnemonicbox}
\subsection*{Question 3(a) [3 marks]}\label{q3a}

\textbf{What is fuzzy logic? Explain its usage in washing machine.}

\begin{solutionbox}

\textbf{Fuzzy Logic}: A mathematical approach that deals with
approximate reasoning rather than fixed, binary logic, allowing for
degrees of truth values between 0 and 1.

\textbf{Usage in Washing Machine:}

\begin{center}
\textbf{Mermaid Diagram (Code)}
\begin{verbatim}
{Shaded}
{Highlighting}[]
graph LR
    A[Sensors] {-{-}{} B[Fuzzy Controller]}
    B {-{-}{} C[Decision Making]}
    C {-{-}{} D[Control Actions]}
    E[Load Size] {-{-}{} A}
    F[Fabric Type] {-{-}{} A}
    G[Dirt Level] {-{-}{} A}
{Highlighting}
{Shaded}
\end{verbatim}
\end{center}

\begin{itemize}
\tightlist
\item
  \textbf{Input Variables}: Load weight, fabric type, water hardness,
  dirt level
\item
  \textbf{Processing}: Controller evaluates multiple conditions
  simultaneously
\item
  \textbf{Output}: Adjusts water level, wash time, rinse cycles, spin
  speed
\end{itemize}

\end{solutionbox}
\begin{mnemonicbox}
``FIND: Fuzzy Intelligence Navigates Decisions''

\end{mnemonicbox}
\subsection*{Question 3(b) [4 marks]}\label{q3b}

\textbf{Define air conditioning. Explain working of fridge. State its
technical specification.}

\begin{solutionbox}

\textbf{Air Conditioning}: The process of removing heat and moisture
from indoor air to improve comfort.

\textbf{Refrigerator Working:}

\begin{center}
\textbf{Mermaid Diagram (Code)}
\begin{verbatim}
{Shaded}
{Highlighting}[]
graph LR
    A[Compressor] {-{-}{}|High Pressure Vapor| B[Condenser]}
    B {-{-}{}|High Pressure Liquid| C[Expansion Valve]}
    C {-{-}{}|Low Pressure Liquid| D[Evaporator]}
    D {-{-}{}|Low Pressure Vapor| A}
{Highlighting}
{Shaded}
\end{verbatim}
\end{center}

\textbf{Working Cycle:}

\begin{enumerate}
\tightlist
\item
  \textbf{Compressor}: Compresses refrigerant gas, raising temperature
\item
  \textbf{Condenser}: Hot gas releases heat to outside, becomes liquid
\item
  \textbf{Expansion Valve}: Liquid expands, cools rapidly
\item
  \textbf{Evaporator}: Cold refrigerant absorbs heat from inside cabinet
\end{enumerate}

\textbf{Technical Specifications:}

\begin{itemize}
\tightlist
\item
  \textbf{Capacity}: 150-500 liters
\item
  \textbf{Energy Rating}: 3-5 Star
\item
  \textbf{Power Consumption}: 100-300 kWh/year
\end{itemize}

\end{solutionbox}
\begin{mnemonicbox}
``CEVA: Compress, Expel heat, Valve expands, Absorb
heat''

\end{mnemonicbox}
\subsection*{Question 3(c) [7 marks]}\label{q3c}

\textbf{Explain working principle of Microwave oven using functional
block diagram. State its technical specifications.}

\begin{solutionbox}

\textbf{Microwave Oven Working:}

\begin{center}
\textbf{Mermaid Diagram (Code)}
\begin{verbatim}
{Shaded}
{Highlighting}[]
graph LR
    A[Power Supply] {-{-}{} B[Control Panel]}
    B {-{-}{} C[Timer \& Controller]}
    C {-{-}{} D[Magnetron]}
    D {-{-}{} E[Waveguide]}
    E {-{-}{} F[Cooking Cavity]}
    G[Turntable Motor] {-{-}{} F}
    C {-{-}{} G}
    H[Door Safety Interlocks] {-{-}{} C}
{Highlighting}
{Shaded}
\end{verbatim}
\end{center}

\textbf{Working Principle:}

\begin{enumerate}
\tightlist
\item
  \textbf{Magnetron}: Generates microwaves at 2.45 GHz frequency
\item
  \textbf{Waveguide}: Directs microwaves into cooking cavity
\item
  \textbf{Water Molecules}: Microwaves cause water molecules to vibrate
\item
  \textbf{Heat Generation}: Molecular vibration creates friction and
  heat
\item
  \textbf{Turntable}: Rotates food for even cooking
\item
  \textbf{Safety Interlocks}: Prevent operation when door is open
\end{enumerate}

\textbf{Technical Specifications:}

\begin{itemize}
\tightlist
\item
  \textbf{Power Output}: 700-1200 watts
\item
  \textbf{Frequency}: 2.45 GHz
\item
  \textbf{Capacity}: 20-40 liters
\item
  \textbf{Cooking Modes}: Microwave, Grill, Convection, Combination
\end{itemize}

\end{solutionbox}
\begin{mnemonicbox}
``MICRO: Magnetron Initiates Cooking by Rotating
Oscillations''

\end{mnemonicbox}
\subsection*{Question 3(a) OR [3
marks]}\label{q3a}

\textbf{Give technical specification of solar panel. State advantages
and disadvantages of solar roof top system}

\begin{solutionbox}

\textbf{Solar Panel Technical Specifications:}

\begin{itemize}
\tightlist
\item
  \textbf{Power Rating}: 250-400 Wp (Watt peak)
\item
  \textbf{Efficiency}: 15-22\%
\item
  \textbf{Cell Type}: Monocrystalline, Polycrystalline, or Thin Film
\end{itemize}

\textbf{Advantages and Disadvantages:}

{\def\LTcaptype{none} % do not increment counter
\begin{longtable}[]{@{}ll@{}}
\toprule\noalign{}
Advantages & Disadvantages \\
\midrule\noalign{}
\endhead
\bottomrule\noalign{}
\endlastfoot
\textbf{Renewable Energy Source} & \textbf{High Initial Cost} \\
\textbf{Reduces Electricity Bills} & \textbf{Weather Dependent} \\
\textbf{Low Maintenance Cost} & \textbf{Requires Large Space} \\
\textbf{No Noise Pollution} & \textbf{Limited Nighttime Generation} \\
\end{longtable}
}

\end{solutionbox}
\begin{mnemonicbox}
``SERLN: Solar Energy Reduces Long-term Numbers''

\end{mnemonicbox}
\subsection*{Question 3(b) OR [4
marks]}\label{q3b}

\textbf{State various types of washing machine. Compare frontload and
top load washing machine.}

\begin{solutionbox}

\textbf{Types of Washing Machines:}

\begin{itemize}
\tightlist
\item
  Top Load (Agitator \& Impeller)
\item
  Front Load
\item
  Semi-Automatic
\item
  Fully Automatic
\end{itemize}

\textbf{Comparison:}

{\def\LTcaptype{none} % do not increment counter
\begin{longtable}[]{@{}
  >{\raggedright\arraybackslash}p{(\linewidth - 4\tabcolsep) * \real{0.3333}}
  >{\raggedright\arraybackslash}p{(\linewidth - 4\tabcolsep) * \real{0.3636}}
  >{\raggedright\arraybackslash}p{(\linewidth - 4\tabcolsep) * \real{0.3030}}@{}}
\toprule\noalign{}
\begin{minipage}[b]{\linewidth}\raggedright
Parameter
\end{minipage} & \begin{minipage}[b]{\linewidth}\raggedright
Front Load
\end{minipage} & \begin{minipage}[b]{\linewidth}\raggedright
Top Load
\end{minipage} \\
\midrule\noalign{}
\endhead
\bottomrule\noalign{}
\endlastfoot
\textbf{Water Consumption} & Lower (40-60 liters) & Higher (80-120
liters) \\
\textbf{Energy Efficiency} & Higher & Lower \\
\textbf{Cleaning Performance} & Better & Good \\
\textbf{Space Requirement} & Can be stacked & Needs top clearance \\
\textbf{Cost} & Higher & Lower \\
\textbf{Cycle Duration} & Longer (60-120 min) & Shorter (30-60 min) \\
\end{longtable}
}

\end{solutionbox}
\begin{mnemonicbox}
``FTEST: Front-loaders Take Extra Space but Triumph
in efficiency''

\end{mnemonicbox}
\subsection*{Question 3(c) OR [7
marks]}\label{q3c}

\textbf{Give classification of solar rooftop system. Explain working of
solar rooftop system (Grid connected online) with suitable diagram.
State steps to maintain solar roof top system.}

\begin{solutionbox}

\textbf{Classification of Solar Rooftop Systems:}

\begin{itemize}
\tightlist
\item
  \textbf{Grid-Connected} (On-grid)
\item
  \textbf{Off-Grid} (Standalone)
\item
  \textbf{Hybrid} (With battery backup)
\end{itemize}

\textbf{Grid-Connected Solar System:}

\begin{center}
\textbf{Mermaid Diagram (Code)}
\begin{verbatim}
{Shaded}
{Highlighting}[]
graph LR
    A[Solar Panels] {-{-}{}|DC Current| B[DC Junction Box]}
    B {-{-}{} C[Solar Inverter]}
    C {-{-}{}|AC Current| D[AC Distribution Box]}
    D {-{-}{} E[Home Loads]}
    D {-{-}{} F[Bi{-}directional Meter]}
    F {-{-}{} G[Grid Connection]}
    G {-{-}{} F}
{Highlighting}
{Shaded}
\end{verbatim}
\end{center}

\textbf{Working:}

\begin{enumerate}
\tightlist
\item
  \textbf{Solar Panels}: Convert sunlight to DC electricity
\item
  \textbf{Junction Box}: Combines outputs, provides protection
\item
  \textbf{Inverter}: Converts DC to grid-compatible AC
\item
  \textbf{Distribution Box}: Distributes power to loads
\item
  \textbf{Bi-directional Meter}: Measures import/export of electricity
\item
  \textbf{Excess Generation}: Feeds back to grid (Net metering)
\end{enumerate}

\textbf{Maintenance Steps:}

\begin{enumerate}
\tightlist
\item
  Regular cleaning of panels (dust, bird droppings)
\item
  Checking electrical connections for corrosion
\item
  Monitoring system performance via inverter data
\item
  Trimming nearby trees to prevent shading
\item
  Annual inspection by qualified technician
\end{enumerate}

\end{solutionbox}
\begin{mnemonicbox}
``SPICED: Solar Panels Invert Current for Electrical
Distribution''

\end{mnemonicbox}
\subsection*{Question 4(a) [3 marks]}\label{q4a}

\textbf{Explain in brief working principle of photo copier machine with
concept of latent image.}

\begin{solutionbox}

\textbf{Photocopier Working Principle:}

\begin{center}
\textbf{Mermaid Diagram (Code)}
\begin{verbatim}
{Shaded}
{Highlighting}[]
graph LR
    A[Charging] {-{-}{} B[Exposure]}
    B {-{-}{} C[Developing]}
    C {-{-}{} D[Transfer]}
    D {-{-}{} E[Fusing]}
    E {-{-}{} F[Cleaning]}
{Highlighting}
{Shaded}
\end{verbatim}
\end{center}

\textbf{Latent Image Concept:}

\begin{itemize}
\tightlist
\item
  \textbf{Charging}: Photosensitive drum receives uniform positive
  charge
\item
  \textbf{Exposure}: Light reflects from original document onto drum
\item
  \textbf{Latent Image}: Light areas discharge drum creating invisible
  electrostatic image
\item
  \textbf{Development}: Negatively charged toner particles attracted to
  positive areas
\item
  \textbf{Transfer}: Toner transferred to paper through electrical
  attraction
\item
  \textbf{Fusing}: Heat and pressure permanently bond toner to paper
\end{itemize}

\end{solutionbox}
\begin{mnemonicbox}
``CEDTFC: Charging Exposure Develops The Final Copy''

\end{mnemonicbox}
\subsection*{Question 4(b) [4 marks]}\label{q4b}

\textbf{Explain working of Laser printer with suitable diagram}

\begin{solutionbox}

\textbf{Laser Printer Working:}

\begin{center}
\textbf{Mermaid Diagram (Code)}
\begin{verbatim}
{Shaded}
{Highlighting}[]
graph LR
    A[Data Processing] {-{-}{} B[Laser Scanning Unit]}
    B {-{-}{} C[Photosensitive Drum]}
    D[Primary Corona] {-{-}{} C}
    C {-{-}{} E[Developer Unit]}
    E {-{-}{} F[Transfer Unit]}
    F {-{-}{} G[Fusing Unit]}
    G {-{-}{} H[Paper Output]}
    I[Cleaning Unit] {-{-}{} C}
{Highlighting}
{Shaded}
\end{verbatim}
\end{center}

\textbf{Working Process:}

\begin{enumerate}
\tightlist
\item
  \textbf{Raster Image Processing}: Computer data converted to bitmap
\item
  \textbf{Charging}: Corona wire gives drum uniform negative charge
\item
  \textbf{Writing}: Laser beam neutralizes charge in pattern of image
\item
  \textbf{Developing}: Toner attracted to neutralized areas
\item
  \textbf{Transfer}: Paper given positive charge to attract toner
\item
  \textbf{Fusing}: Heat rollers melt toner permanently onto paper
\item
  \textbf{Cleaning}: Excess toner removed from drum for next cycle
\end{enumerate}

\end{solutionbox}
\begin{mnemonicbox}
``RASTER: Raster-image Attracts Static Toner,
Electricity Releases''

\end{mnemonicbox}
\subsection*{Question 4(c) [7 marks]}\label{q4c}

\textbf{Draw and explain block diagram of CCTV system using Digital IP
camera connected with internet. List at least five different camera used
in CCTV system. What is meaning of POE cable?}

\begin{solutionbox}

\textbf{IP CCTV System:}

\begin{center}
\textbf{Mermaid Diagram (Code)}
\begin{verbatim}
{Shaded}
{Highlighting}[]
graph LR
    A[IP Cameras] {-{-}{}|Ethernet/POE| B[Network Switch]}
    B {-{-}{} C[Network Video Recorder]}
    C {-{-}{} D[Storage]}
    C {-{-}{} E[Router/Internet Gateway]}
    E {-{-}{}|WAN| F[Remote Viewing Devices]}
    G[Local Monitor] {-{-}{} C}
{Highlighting}
{Shaded}
\end{verbatim}
\end{center}

\textbf{Working:}

\begin{enumerate}
\tightlist
\item
  \textbf{IP Cameras}: Capture and digitize video
\item
  \textbf{Network Infrastructure}: Transmits data via TCP/IP protocols
\item
  \textbf{NVR}: Records, manages, and processes video streams
\item
  \textbf{Storage}: Hard drives store recorded footage
\item
  \textbf{Router}: Provides secure internet access for remote viewing
\end{enumerate}

\textbf{Camera Types:}

\begin{enumerate}
\tightlist
\item
  \textbf{Dome Cameras}: Indoor ceiling-mounted, vandal-resistant
\item
  \textbf{Bullet Cameras}: Outdoor wall-mounted, long-range
\item
  \textbf{PTZ Cameras}: Pan, tilt, zoom capabilities for wide coverage
\item
  \textbf{Fisheye Cameras}: 360^\circ panoramic view with single lens
\item
  \textbf{Thermal Cameras}: Detect heat signatures in darkness
\end{enumerate}

\textbf{POE Cable}: Power Over Ethernet - A technology that carries both
power and data over a single Ethernet cable, eliminating the need for
separate power cables.

\end{solutionbox}
\begin{mnemonicbox}
``INSPIRE: Internet Networking Secures Places In
Remote Environments''

\end{mnemonicbox}
\subsection*{Question 4(a) OR [3
marks]}\label{q4a}

\textbf{Discuss pros and cons of internet based Digital IP camera CCTV
system}

\begin{solutionbox}

\textbf{Pros and Cons of IP Camera CCTV Systems:}

{\def\LTcaptype{none} % do not increment counter
\begin{longtable}[]{@{}ll@{}}
\toprule\noalign{}
Pros & Cons \\
\midrule\noalign{}
\endhead
\bottomrule\noalign{}
\endlastfoot
\textbf{Higher Resolution} (1080p to 4K) & \textbf{Higher Initial
Cost} \\
\textbf{Remote Viewing} via internet & \textbf{Bandwidth
Requirements} \\
\textbf{Scalability} \& easy expansion & \textbf{Cybersecurity Risks} \\
\textbf{Power Over Ethernet} (POE) & \textbf{Network Dependency} \\
\textbf{Advanced Analytics} capabilities & \textbf{Complex
Configuration} \\
\end{longtable}
}

\end{solutionbox}
\begin{mnemonicbox}
``HIGHER: High-resolution Images Give Higher
Evaluation Remotely''

\end{mnemonicbox}
\subsection*{Question 4(b) OR [4
marks]}\label{q4b}

\textbf{Explain working of inkjet printer with suitable diagram}

\begin{solutionbox}

\textbf{Inkjet Printer Working:}

\begin{center}
\textbf{Mermaid Diagram (Code)}
\begin{verbatim}
{Shaded}
{Highlighting}[]
graph LR
    A[Print Data] {-{-}{} B[Controller]}
    B {-{-}{} C[Print Head Assembly]}
    C {-{-}{} D[Ink Cartridges]}
    D {-{-}{} E[Nozzles]}
    E {-{-}{} F[Paper]}
    G[Paper Feed Mechanism] {-{-}{} F}
    B {-{-}{} G}
{Highlighting}
{Shaded}
\end{verbatim}
\end{center}

\textbf{Working Process:}

\begin{enumerate}
\tightlist
\item
  \textbf{Data Processing}: Controller converts digital data to nozzle
  instructions
\item
  \textbf{Paper Loading}: Feed rollers position paper correctly
\item
  \textbf{Print Head Movement}: Carriage moves printhead across paper
\item
  \textbf{Ink Ejection}: Two methods:

  \begin{itemize}
  \tightlist
  \item
    Thermal: Tiny resistors heat ink to create bubbles, forcing droplets
  \item
    Piezoelectric: Crystal elements flex to push ink through nozzles
  \end{itemize}
\item
  \textbf{Drying}: Ink adheres to paper surface
\end{enumerate}

\end{solutionbox}
\begin{mnemonicbox}
``PRINT: Paper Receives Ink through Numerous
Tiny-nozzles''

\end{mnemonicbox}
\subsection*{Question 4(c) OR [7
marks]}\label{q4c}

\textbf{Draw and explain block diagram of CCTV system using simple
analog camera and DVR. List types of cable used in CCTV system. Discuss
at least four different categories of camera used in modern CCTV
system.}

\begin{solutionbox}

\textbf{Analog CCTV System:}

\begin{center}
\textbf{Mermaid Diagram (Code)}
\begin{verbatim}
{Shaded}
{Highlighting}[]
graph LR
    A[Analog Cameras] {-{-}{}|Coaxial Cable| B[DVR]}
    B {-{-}{} C[Hard Disk Storage]}
    B {-{-}{} D[Monitor]}
    B {-{-}{} E[Router]}
    E {-{-}{}|Internet| F[Remote Viewing]}
    G[Power Supply] {-{-}{} A}
{Highlighting}
{Shaded}
\end{verbatim}
\end{center}

\textbf{Working:}

\begin{enumerate}
\tightlist
\item
  \textbf{Analog Cameras}: Capture video as continuous analog signals
\item
  \textbf{DVR}: Converts analog signals to digital format for recording
\item
  \textbf{Storage}: Records footage on internal hard drives
\item
  \textbf{Viewing}: Local monitors and remote access options
\end{enumerate}

\textbf{Cable Types:}

\begin{enumerate}
\tightlist
\item
  \textbf{Coaxial Cable} (RG59, RG6): Traditional analog camera
  connection
\item
  \textbf{Twisted Pair} (CAT5/6): For IP cameras or with baluns
\item
  \textbf{Power Cable}: Supplies electricity to cameras
\item
  \textbf{Fiber Optic}: For long-distance transmissions
\item
  \textbf{Siamese Cable}: Combined coaxial and power cable
\end{enumerate}

\textbf{Camera Categories:}

\begin{enumerate}
\tightlist
\item
  \textbf{Fixed Cameras}: Constant view angle, no movement
\item
  \textbf{Varifocal Cameras}: Adjustable lens for different focal
  lengths
\item
  \textbf{Night Vision Cameras}: IR illuminators for low-light
  conditions
\item
  \textbf{High Dynamic Range (HDR)}: Balanced exposure in mixed lighting
\end{enumerate}

\end{solutionbox}
\begin{mnemonicbox}
``CARD: Coaxial Analog Recording Devices''

\end{mnemonicbox}
\subsection*{Question 5(a) [3 marks]}\label{q5a}

\textbf{Define only: Maintenance, Preventive maintenance and Predictive
maintenance.}

\begin{solutionbox}

\begin{itemize}
\tightlist
\item
  \textbf{Maintenance}: The process of preserving equipment in proper
  operating condition through regular inspection, cleaning, and repair.
\item
  \textbf{Preventive Maintenance}: Scheduled maintenance activities
  performed to prevent equipment failures before they occur.
\item
  \textbf{Predictive Maintenance}: Condition-based maintenance approach
  that uses data analysis and monitoring techniques to predict when
  equipment failure might occur.
\end{itemize}

\end{solutionbox}
\begin{mnemonicbox}
``MPP: Maintain Proactively, Predict problems''

\end{mnemonicbox}
\subsection*{Question 5(b) [4 marks]}\label{q5b}

\textbf{Discuss maintenance of public address system.}

\begin{solutionbox}

\textbf{PA System Maintenance:}

{\def\LTcaptype{none} % do not increment counter
\begin{longtable}[]{@{}
  >{\raggedright\arraybackslash}p{(\linewidth - 2\tabcolsep) * \real{0.3667}}
  >{\raggedright\arraybackslash}p{(\linewidth - 2\tabcolsep) * \real{0.6333}}@{}}
\toprule\noalign{}
\begin{minipage}[b]{\linewidth}\raggedright
Component
\end{minipage} & \begin{minipage}[b]{\linewidth}\raggedright
Maintenance Tasks
\end{minipage} \\
\midrule\noalign{}
\endhead
\bottomrule\noalign{}
\endlastfoot
\textbf{Microphones} & • Clean windscreens and grilles• Check cables for
damage• Test for proper sensitivity \\
\textbf{Amplifiers} & • Clean cooling vents• Check power connections•
Inspect for overheating \\
\textbf{Speakers} & • Inspect mounting brackets• Test for distortion•
Check wiring connections \\
\textbf{Cables \& Connections} & • Test continuity• Replace damaged
cables• Secure loose connections \\
\end{longtable}
}

\textbf{Periodic Maintenance:}

\begin{itemize}
\tightlist
\item
  Weekly: Basic operations check
\item
  Monthly: Signal path testing
\item
  Quarterly: Comprehensive inspection
\item
  Annually: Professional service
\end{itemize}

\end{solutionbox}
\begin{mnemonicbox}
``MACS: Microphones, Amplifiers, Connections,
Speakers''

\end{mnemonicbox}
\subsection*{Question 5(c) [7 marks]}\label{q5c}

\textbf{State any three faults of washing machine. Discuss in general
maintenance of washing machine.}

\begin{solutionbox}

\textbf{Common Washing Machine Faults:}

\begin{enumerate}
\tightlist
\item
  \textbf{Water Not Filling}: Faulty inlet valve, clogged filter, water
  pressure issues
\item
  \textbf{Not Spinning}: Belt issues, motor problems, unbalanced load
\item
  \textbf{Excessive Vibration}: Uneven feet, suspension issues, drum
  damage
\end{enumerate}

\textbf{General Maintenance:}

{\def\LTcaptype{none} % do not increment counter
\begin{longtable}[]{@{}
  >{\raggedright\arraybackslash}p{(\linewidth - 2\tabcolsep) * \real{0.3143}}
  >{\raggedright\arraybackslash}p{(\linewidth - 2\tabcolsep) * \real{0.6857}}@{}}
\toprule\noalign{}
\begin{minipage}[b]{\linewidth}\raggedright
Component
\end{minipage} & \begin{minipage}[b]{\linewidth}\raggedright
Maintenance Procedure
\end{minipage} \\
\midrule\noalign{}
\endhead
\bottomrule\noalign{}
\endlastfoot
\textbf{Drum/Tub} & • Clean monthly to remove residue• Check for foreign
objects• Run cleaning cycle with white vinegar \\
\textbf{Filters} & • Clean lint filter after each use• Clean pump filter
monthly• Check water inlet filters quarterly \\
\textbf{Hoses} & • Inspect for cracks or leaks• Replace every 3-5 years•
Ensure proper connection \\
\textbf{Door Seal} & • Wipe after use to prevent mold• Check for tears•
Keep door ajar when not in use \\
\textbf{Dispensers} & • Remove and clean monthly• Check for blockages•
Remove detergent buildup \\
\end{longtable}
}

\end{solutionbox}
\begin{mnemonicbox}
``WATCH: Water And Tub Cleaning Helps''

\end{mnemonicbox}
\subsection*{Question 5(a) OR [3
marks]}\label{q5a}

\textbf{Compare predictive and preventive maintenance.}

\begin{solutionbox}

\textbf{Comparison of Predictive vs.~Preventive Maintenance:}

{\def\LTcaptype{none} % do not increment counter
\begin{longtable}[]{@{}
  >{\raggedright\arraybackslash}p{(\linewidth - 4\tabcolsep) * \real{0.1833}}
  >{\raggedright\arraybackslash}p{(\linewidth - 4\tabcolsep) * \real{0.4000}}
  >{\raggedright\arraybackslash}p{(\linewidth - 4\tabcolsep) * \real{0.4167}}@{}}
\toprule\noalign{}
\begin{minipage}[b]{\linewidth}\raggedright
Parameter
\end{minipage} & \begin{minipage}[b]{\linewidth}\raggedright
Predictive Maintenance
\end{minipage} & \begin{minipage}[b]{\linewidth}\raggedright
Preventive Maintenance
\end{minipage} \\
\midrule\noalign{}
\endhead
\bottomrule\noalign{}
\endlastfoot
\textbf{Approach} & Condition-based & Time-based \\
\textbf{Timing} & When needed based on data & Fixed schedule regardless
of condition \\
\textbf{Techniques} & Vibration analysis, thermal imaging, oil analysis
& Visual inspection, cleaning, lubrication \\
\textbf{Cost} & Higher initial setup, lower long-term & Lower initial
cost, potentially higher long-term \\
\textbf{Downtime} & Minimized, planned ahead & Regular scheduled
downtime \\
\textbf{Equipment Usage} & Maximized lifespan & Some components replaced
prematurely \\
\end{longtable}
}

\end{solutionbox}
\begin{mnemonicbox}
``TIMED: Testing Identifies Maintenance Exactly when
Due''

\end{mnemonicbox}
\subsection*{Question 5(b) OR [4
marks]}\label{q5b}

\textbf{Discuss maintenance and troubleshooting of LCD TV.}

\begin{solutionbox}

\textbf{LCD TV Maintenance:}

{\def\LTcaptype{none} % do not increment counter
\begin{longtable}[]{@{}
  >{\raggedright\arraybackslash}p{(\linewidth - 2\tabcolsep) * \real{0.3667}}
  >{\raggedright\arraybackslash}p{(\linewidth - 2\tabcolsep) * \real{0.6333}}@{}}
\toprule\noalign{}
\begin{minipage}[b]{\linewidth}\raggedright
Component
\end{minipage} & \begin{minipage}[b]{\linewidth}\raggedright
Maintenance Tasks
\end{minipage} \\
\midrule\noalign{}
\endhead
\bottomrule\noalign{}
\endlastfoot
\textbf{Screen} & • Clean with microfiber cloth• Avoid liquid cleaners•
Check for dead pixels \\
\textbf{Ventilation} & • Remove dust from vents• Ensure proper airflow•
Check fan operation \\
\textbf{Connections} & • Verify cable connections• Check for corrosion•
Test HDMI ports \\
\textbf{Software} & • Update firmware regularly• Reset settings if
needed \\
\end{longtable}
}

\textbf{Common Troubleshooting Issues:}

{\def\LTcaptype{none} % do not increment counter
\begin{longtable}[]{@{}
  >{\raggedright\arraybackslash}p{(\linewidth - 2\tabcolsep) * \real{0.3214}}
  >{\raggedright\arraybackslash}p{(\linewidth - 2\tabcolsep) * \real{0.6786}}@{}}
\toprule\noalign{}
\begin{minipage}[b]{\linewidth}\raggedright
Problem
\end{minipage} & \begin{minipage}[b]{\linewidth}\raggedright
Possible Solutions
\end{minipage} \\
\midrule\noalign{}
\endhead
\bottomrule\noalign{}
\endlastfoot
\textbf{No Power} & Check power cord, outlet, internal fuse \\
\textbf{No Picture} & Verify input source, backlight failure, T-Con
board \\
\textbf{Lines on Screen} & Check ribbon cables, screen damage, T-Con
board \\
\textbf{Audio Issues} & Speaker connections, audio settings, amplifier
board \\
\end{longtable}
}

\end{solutionbox}
\begin{mnemonicbox}
``PVCS: Pixels, Ventilation, Connections, Software''

\end{mnemonicbox}
\subsection*{Question 5(c) OR [7
marks]}\label{q5c}

\textbf{Explain installation of laser printers in your computer system.
Discuss its maintenance and troubleshooting procedure.}

\begin{solutionbox}

\textbf{Laser Printer Installation:}

\begin{center}
\textbf{Mermaid Diagram (Code)}
\begin{verbatim}
{Shaded}
{Highlighting}[]
graph LR
    A[Unpacking] {-{-}{} B[Hardware Setup]}
    B {-{-}{} C[Cartridge Installation]}
    C {-{-}{} D[Power Connection]}
    D {-{-}{} E[Interface Connection]}
    E {-{-}{} F[Driver Installation]}
    F {-{-}{} G[Test Print]}
{Highlighting}
{Shaded}
\end{verbatim}
\end{center}

\textbf{Installation Steps:}

\begin{enumerate}
\tightlist
\item
  \textbf{Setup Location}: Flat, stable surface with proper ventilation
\item
  \textbf{Remove Packaging}: Remove tape, protective films, shipping
  locks
\item
  \textbf{Install Consumables}: Toner cartridge, imaging drum if
  separate
\item
  \textbf{Connect Power}: Plug into grounded outlet
\item
  \textbf{Connect Interface}: USB, Ethernet, or Wi-Fi setup
\item
  \textbf{Install Driver}: From included CD or manufacturer website
\item
  \textbf{Configure Settings}: Network parameters, paper size, default
  tray
\end{enumerate}

\textbf{Maintenance:}

{\def\LTcaptype{none} % do not increment counter
\begin{longtable}[]{@{}ll@{}}
\toprule\noalign{}
Component & Maintenance Task \\
\midrule\noalign{}
\endhead
\bottomrule\noalign{}
\endlastfoot
\textbf{Paper Path} & Clean with compressed air monthly \\
\textbf{Toner Area} & Vacuum carefully when replacing toner \\
\textbf{Rollers} & Clean with isopropyl alcohol quarterly \\
\textbf{Exterior} & Wipe with damp cloth as needed \\
\end{longtable}
}

\textbf{Troubleshooting:}

{\def\LTcaptype{none} % do not increment counter
\begin{longtable}[]{@{}
  >{\raggedright\arraybackslash}p{(\linewidth - 2\tabcolsep) * \real{0.4737}}
  >{\raggedright\arraybackslash}p{(\linewidth - 2\tabcolsep) * \real{0.5263}}@{}}
\toprule\noalign{}
\begin{minipage}[b]{\linewidth}\raggedright
Problem
\end{minipage} & \begin{minipage}[b]{\linewidth}\raggedright
Solution
\end{minipage} \\
\midrule\noalign{}
\endhead
\bottomrule\noalign{}
\endlastfoot
\textbf{Paper Jams} & Check paper path, clean rollers, verify paper
specifications \\
\textbf{Streaking} & Clean corona wire, replace drum if worn \\
\textbf{Light Printing} & Adjust density settings, replace toner \\
\textbf{Connection Issues} & Check cables, reinstall drivers, reset
printer \\
\end{longtable}
}

\end{solutionbox}
\begin{mnemonicbox}
``SECURE: Setup, Execute drivers, Clean Regularly,
Update, Replace consumables, Examine problems''

\end{mnemonicbox}

\end{document}
