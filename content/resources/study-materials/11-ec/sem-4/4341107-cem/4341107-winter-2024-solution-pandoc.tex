\documentclass[10pt,a4paper]{article}

% content/resources/templates/preamble.tex
\usepackage[margin=0.6in]{geometry}
\author{Milav Dabgar}
\usepackage{amsmath,amssymb,amsthm}
\usepackage{booktabs}
\usepackage{multirow}
\usepackage{xcolor}
\usepackage{tcolorbox}
\tcbuselibrary{breakable,skins}
\usepackage[colorlinks=true,linkcolor=blue]{hyperref}
\usepackage{titlesec}
\usepackage{enumitem}
\usepackage{tikz}
\usepackage{pgfplots}
\usepackage{circuitikz}
\usepackage[version=4]{mhchem}
\usepackage{longtable}
\usepackage{array}
\usepackage{float}
\usepackage{caption}
\usepackage{listings}

\lstset{
  basicstyle=\small\ttfamily,
  breaklines=true,
  breakatwhitespace=false,
  postbreak=\mbox{\textcolor{red}{$\hookrightarrow$}\space},
  float=false,
  numbers=left,
  numberstyle=\tiny\color{gray},
  numbersep=10pt,
  xleftmargin=2em,
  keywordstyle=\color{blue},
  commentstyle=\color{green!60!black},
  stringstyle=\color{purple},
  backgroundcolor=\color{gray!5},
  showstringspaces=false,
  tabsize=2,
  captionpos=b,
  keepspaces=true,
  columns=flexible
}

\pgfplotsset{compat=1.18}
\usetikzlibrary{shapes,arrows,positioning,calc,patterns,decorations.pathmorphing,decorations.markings,arrows.meta}

% Color scheme
\definecolor{headcolor}{RGB}{0,102,204}
\definecolor{keycolor}{RGB}{220,20,60}
\definecolor{solutioncolor}{RGB}{34,139,34}
\definecolor{mnemoniccolor}{RGB}{148,0,211}
\definecolor{codecolor}{RGB}{0,0,100}

% Spacing
\setlength{\parskip}{3pt}
\setlist[itemize]{nosep}
\setlist[enumerate]{nosep}

% Title formatting
\titleformat{\section}{\Large\bfseries\color{headcolor}}{\thesection}{1em}{}
\titleformat{\subsection}{\large\bfseries\color{headcolor}}{\thesubsection}{1em}{}

% Pandoc tightlist compatibility
\providecommand{\tightlist}{%
  \setlength{\itemsep}{0pt}\setlength{\parskip}{0pt}}

% Pandoc longtable compatibility
\newcounter{none}
\def\thenone{}


% content/resources/templates/english-boxes.tex
% This file is currently empty - it exists to maintain consistency with the import structure.
% Add custom environments here if needed in the future.


\begin{document}

\begin{center}
{\Huge\bfseries\color{headcolor} Subject Name Solutions}\\[5pt]
{\LARGE 4341107 -- Winter 2024}\\[3pt]
{\large Semester 1 Study Material}\\[3pt]
{\normalsize\textit{Detailed Solutions and Explanations}}
\end{center}

\vspace{10pt}

\subsection*{Question 1(a) [3 marks]}\label{q1a}

\textbf{Define only: 1. Loudness 2. Timbre 3. Echo}

\begin{solutionbox}

{\def\LTcaptype{none} % do not increment counter
\begin{longtable}[]{@{}
  >{\raggedright\arraybackslash}p{(\linewidth - 2\tabcolsep) * \real{0.3333}}
  >{\raggedright\arraybackslash}p{(\linewidth - 2\tabcolsep) * \real{0.6667}}@{}}
\toprule\noalign{}
\begin{minipage}[b]{\linewidth}\raggedright
Term
\end{minipage} & \begin{minipage}[b]{\linewidth}\raggedright
Definition
\end{minipage} \\
\midrule\noalign{}
\endhead
\bottomrule\noalign{}
\endlastfoot
\textbf{Loudness} & The subjective perception of sound intensity that
depends on sound pressure and frequency \\
\textbf{Timbre} & The quality of sound that distinguishes different
instruments or voices playing the same note \\
\textbf{Echo} & A sound reflection that arrives at the listener with a
delay greater than 50ms after the direct sound \\
\end{longtable}
}

\end{solutionbox}
\begin{mnemonicbox}
``LTE: Loudness measures strength, Timbre gives
uniqueness, Echo comes back delayed''

\end{mnemonicbox}
\subsection*{Question 1(b) [4 marks]}\label{q1b}

\textbf{List Type of loudspeaker and explain any one of them}

\begin{solutionbox}

\textbf{Types of Loudspeakers:}

{\def\LTcaptype{none} % do not increment counter
\begin{longtable}[]{@{}ll@{}}
\toprule\noalign{}
Type & Key Feature \\
\midrule\noalign{}
\endhead
\bottomrule\noalign{}
\endlastfoot
Dynamic/Moving Coil & Uses electromagnetic coil \\
Electrostatic & Uses charged diaphragm \\
Ribbon & Uses thin metal ribbon \\
Piezoelectric & Uses crystals that vibrate \\
Horn & Uses acoustic horn for amplification \\
Planar Magnetic & Uses magnetic strips on diaphragm \\
\end{longtable}
}

\textbf{Dynamic/Moving Coil Loudspeaker:}

\begin{verbatim}
flowchart LR
    A[Audio Signal] {-{-} B[Voice Coil]}
    B {-{-} C[Electromagnetic Field]}
    C {-{-} D[Coil Movement]}
    D {-{-} E[Cone/Diaphragm Vibration]}
    E {-{-} F[Sound Waves]}
\end{verbatim}

\begin{itemize}
\tightlist
\item
  \textbf{Magnetic Structure}: Permanent magnet creates static magnetic
  field
\item
  \textbf{Voice Coil}: Receives audio current and creates varying
  magnetic field
\item
  \textbf{Diaphragm/Cone}: Attached to voice coil, vibrates to produce
  sound waves
\end{itemize}

\end{solutionbox}
\begin{mnemonicbox}
``COPPER-D: Coil Oscillates, Permanent magnet
Pulls/Pushes, Emitting Resonance through Diaphragm''

\end{mnemonicbox}
\subsection*{Question 1(c) [7 marks]}\label{q1c}

\textbf{List types of Microphone. State its Characteristics and explain
Wireless Microphone in detail.}

\begin{solutionbox}

\textbf{Types of Microphones:}

{\def\LTcaptype{none} % do not increment counter
\begin{longtable}[]{@{}ll@{}}
\toprule\noalign{}
Type & Operating Principle \\
\midrule\noalign{}
\endhead
\bottomrule\noalign{}
\endlastfoot
Dynamic & Moving coil in magnetic field \\
Condenser & Variable capacitance \\
Carbon & Variable resistance \\
Ribbon & Ribbon movement in magnetic field \\
Crystal/Piezoelectric & Crystal deformation \\
Electret & Permanently charged material \\
MEMS & Micro-Electro-Mechanical Systems \\
\end{longtable}
}

\textbf{Microphone Characteristics:}

\begin{itemize}
\tightlist
\item
  \textbf{Sensitivity}: Output level for given sound pressure
\item
  \textbf{Frequency Response}: Range of frequencies captured
\item
  \textbf{Directional Pattern}: Pickup pattern (omnidirectional,
  cardioid, etc.)
\item
  \textbf{Impedance}: Electrical resistance to AC signals
\item
  \textbf{Signal-to-Noise Ratio}: Desired signal vs.~background noise
\end{itemize}

\textbf{Wireless Microphone System:}

\begin{verbatim}
flowchart LR
    A[Sound Input] {-{-} B[Microphone Element]}
    B {-{-} C[Preamplifier]}
    C {-{-} D[Compressor]}
    D {-{-} E[RF Transmitter]}
    E {-{-} "Radio Waves" {-}{-} F[RF Receiver]}
    F {-{-} G[Demodulator]}
    G {-{-} H[Expander]}
    H {-{-} I[Output Signal]}
\end{verbatim}

\begin{itemize}
\tightlist
\item
  \textbf{Microphone Element}: Converts sound to electrical signals
\item
  \textbf{Transmitter}: Modulates audio onto radio frequency carrier
\item
  \textbf{Receiver}: Captures RF signal and demodulates to recover audio
\item
  \textbf{Operating Frequency}: Uses VHF (30-300 MHz) or UHF (300-3000
  MHz) bands
\item
  \textbf{Battery Operation}: Requires power source for transmitter
\end{itemize}

\end{solutionbox}
\begin{mnemonicbox}
``WIRED: Wireless Is Radio-Enabled Device''

\end{mnemonicbox}
\subsection*{Question 1(c OR) [7
marks]}\label{question-1c-or-7-marks}

\textbf{State characteristics of Loudspeakers and explain pearmeant
magnet loudspeaker with its advantages and disadvantages.}

\begin{solutionbox}

\textbf{Loudspeaker Characteristics:}

{\def\LTcaptype{none} % do not increment counter
\begin{longtable}[]{@{}
  >{\raggedright\arraybackslash}p{(\linewidth - 2\tabcolsep) * \real{0.5517}}
  >{\raggedright\arraybackslash}p{(\linewidth - 2\tabcolsep) * \real{0.4483}}@{}}
\toprule\noalign{}
\begin{minipage}[b]{\linewidth}\raggedright
Characteristic
\end{minipage} & \begin{minipage}[b]{\linewidth}\raggedright
Description
\end{minipage} \\
\midrule\noalign{}
\endhead
\bottomrule\noalign{}
\endlastfoot
Frequency Response & Range of frequencies reproduced (20Hz-20kHz
ideal) \\
Sensitivity & Sound pressure level (dB) with 1W input at 1m distance \\
Impedance & Electrical resistance (typically 4, 8, or 16 ohms) \\
Power Handling & Maximum power without damage (watts) \\
Directivity & Sound dispersion pattern \\
Distortion & Unwanted alteration of the original signal \\
\end{longtable}
}

\textbf{Permanent Magnet Loudspeaker:}

\begin{verbatim}
flowchart LR
    A[Audio Signal] {-{-} B[Voice Coil]}
    B {{-}{-} C[Magnetic Field]}
    C {-{-}{-} D[Permanent Magnet]}
    B {-{-} E[Diaphragm Movement]}
    E {-{-} F[Sound Waves]}
\end{verbatim}

\textbf{Working Principle:}

\begin{itemize}
\tightlist
\item
  Voice coil receives electrical audio signals
\item
  Magnetic field interactions cause coil movement
\item
  Attached diaphragm vibrates to produce sound
\item
  Permanent magnet provides constant magnetic field
\end{itemize}

\textbf{Advantages:}

\begin{itemize}
\tightlist
\item
  \textbf{Cost-effective}: No external power for magnetic field
\item
  \textbf{Reliable}: Simple design with fewer failure points
\item
  \textbf{Compact}: No field coil or power supply needed
\item
  \textbf{Efficient}: Good power-to-sound conversion
\end{itemize}

\textbf{Disadvantages:}

\begin{itemize}
\tightlist
\item
  \textbf{Limited Power}: Magnetic field strength is fixed
\item
  \textbf{Magnet Deterioration}: Can weaken over time
\item
  \textbf{Weight}: Strong magnets can make unit heavy
\item
  \textbf{Heat Sensitivity}: Performance affected by temperature
\end{itemize}

\end{solutionbox}
\begin{mnemonicbox}
``PMLS: Permanent Magnet Loudly Speaks''

\end{mnemonicbox}
\subsection*{Question 2(a) [3 marks]}\label{q2a}

\textbf{Define 1. Aspect ratio 2. Chrominance 3. Additive Mixing}

\begin{solutionbox}

{\def\LTcaptype{none} % do not increment counter
\begin{longtable}[]{@{}
  >{\raggedright\arraybackslash}p{(\linewidth - 2\tabcolsep) * \real{0.3333}}
  >{\raggedright\arraybackslash}p{(\linewidth - 2\tabcolsep) * \real{0.6667}}@{}}
\toprule\noalign{}
\begin{minipage}[b]{\linewidth}\raggedright
Term
\end{minipage} & \begin{minipage}[b]{\linewidth}\raggedright
Definition
\end{minipage} \\
\midrule\noalign{}
\endhead
\bottomrule\noalign{}
\endlastfoot
\textbf{Aspect Ratio} & The ratio of width to height of a television or
display screen (e.g., 16:9, 4:3) \\
\textbf{Chrominance} & The color information in a video signal,
independent of the luminance or brightness \\
\textbf{Additive Mixing} & The process of combining different colored
lights to create new colors, where mixing all primary colors produces
white \\
\end{longtable}
}

\end{solutionbox}
\begin{mnemonicbox}
``ACA: Aspect sets dimensions, Chrominance adds
color, Additive mixing creates brightness''

\end{mnemonicbox}
\subsection*{Question 2(b) [4 marks]}\label{q2b}

\textbf{Explain interlace scanning}

\begin{solutionbox}

\textbf{Interlace Scanning:}

\begin{verbatim}
flowchart LR
    A[Complete Frame] {-{-} B[Odd Lines]}
    A {-{-} C[Even Lines]}
    B {-{-} D[First Field]}
    C {-{-} E[Second Field]}
    D {-{-} F[Display]}
    E {-{-} F}
\end{verbatim}

\textbf{Process:}

\begin{itemize}
\tightlist
\item
  Frame divided into two fields: odd-numbered lines and even-numbered
  lines
\item
  First field displays all odd-numbered lines (1,3,5\ldots)
\item
  Second field displays all even-numbered lines (2,4,6\ldots)
\item
  Fields displayed alternately, creating illusion of full frame
\item
  Standard rate: 50/60 fields per second (25/30 frames per second)
\end{itemize}

\textbf{Key Benefit}: Reduces bandwidth while maintaining perceived
vertical resolution

\end{solutionbox}
\begin{mnemonicbox}
``ODD-EVEN: One Display, then Delayed Extra Visual
Enhancement Next''

\end{mnemonicbox}
\subsection*{Question 2(c) [7 marks]}\label{q2c}

\textbf{Discuss working principle of LED Television. State its
advantages and compare it with LCD television.}

\begin{solutionbox}

\textbf{LED TV Working Principle:}

\begin{verbatim}
flowchart LR
    A[Input Signal] {-{-} B[Signal Processing]}
    B {-{-} C[LCD Panel]}
    D[LED Backlight] {-{-} C}
    C {-{-} E[Polarizing Filters]}
    E {-{-} F[Color Filters]}
    F {-{-} G[Screen Display]}
\end{verbatim}

\textbf{Key Components:}

\begin{itemize}
\tightlist
\item
  \textbf{LED Backlight}: Light source (edge-lit or full-array)
\item
  \textbf{LCD Panel}: Liquid crystal layer controls light passage
\item
  \textbf{TFT Matrix}: Thin-film transistors control each pixel
\item
  \textbf{Color Filters}: Create RGB colors from white backlight
\item
  \textbf{Polarizing Filters}: Control light direction and intensity
\end{itemize}

\textbf{Advantages of LED TV:}

\begin{itemize}
\tightlist
\item
  \textbf{Energy Efficient}: Consumes less power
\item
  \textbf{Thinner Design}: Allows for slim profile
\item
  \textbf{Better Contrast}: Especially with local dimming
\item
  \textbf{Longer Lifespan}: LEDs last 50,000-100,000 hours
\item
  \textbf{Eco-Friendly}: No mercury content
\end{itemize}

\textbf{Comparison with LCD TV:}

{\def\LTcaptype{none} % do not increment counter
\begin{longtable}[]{@{}lll@{}}
\toprule\noalign{}
Feature & LED TV & LCD TV \\
\midrule\noalign{}
\endhead
\bottomrule\noalign{}
\endlastfoot
Backlight & LED lights & CCFL (Cold Cathode Fluorescent Lamps) \\
Thickness & Thinner (25-40mm) & Thicker (100-150mm) \\
Power Consumption & Lower & Higher \\
Contrast Ratio & Better (3000:1-8000:1) & Lower (1000:1-2000:1) \\
Color Reproduction & More vibrant & Less vibrant \\
Lifespan & 50,000-100,000 hours & 30,000-60,000 hours \\
Cost & Higher & Lower \\
\end{longtable}
}

\end{solutionbox}
\begin{mnemonicbox}
``LEDGE: Light Emitting Diodes Give Excellence''

\end{mnemonicbox}
\subsection*{Question 2(a) [3 marks]}\label{q2a}

\textbf{State any six standards of Color television system.}

\begin{solutionbox}

{\def\LTcaptype{none} % do not increment counter
\begin{longtable}[]{@{}
  >{\raggedright\arraybackslash}p{(\linewidth - 2\tabcolsep) * \real{0.3704}}
  >{\raggedright\arraybackslash}p{(\linewidth - 2\tabcolsep) * \real{0.6296}}@{}}
\toprule\noalign{}
\begin{minipage}[b]{\linewidth}\raggedright
Standard
\end{minipage} & \begin{minipage}[b]{\linewidth}\raggedright
Region/Features
\end{minipage} \\
\midrule\noalign{}
\endhead
\bottomrule\noalign{}
\endlastfoot
\textbf{PAL} (Phase Alternating Line) & Europe, Australia, 625 lines, 25
fps \\
\textbf{NTSC} (National Television System Committee) & North America,
Japan, 525 lines, 30 fps \\
\textbf{SECAM} (Sequential Color with Memory) & France, Russia, 625
lines, 25 fps \\
\textbf{PAL-M} & Brazil, 525 lines, 30 fps \\
\textbf{PAL-N} & Argentina, Paraguay, Uruguay \\
\textbf{ATSC} (Advanced Television Systems Committee) & Digital
standard, North America \\
\textbf{DVB-T} (Digital Video Broadcasting-Terrestrial) & Digital
standard, Europe \\
\textbf{ISDB} (Integrated Services Digital Broadcasting) & Digital
standard, Japan, Brazil \\
\end{longtable}
}

\end{solutionbox}
\begin{mnemonicbox}
``PANS-ADI: PAL, ATSC, NTSC, SECAM - All Display
Images''

\end{mnemonicbox}
\subsection*{Question 2(b) [4 marks]}\label{q2b}

\textbf{Explain working of LCD Television.}

\begin{solutionbox}

\textbf{LCD Television Working:}

\begin{verbatim}
flowchart LR
    A[Input Signal] {-{-} B[Signal Processor]}
    B {-{-} C[LCD Driver Circuits]}
    D[Backlight] {-{-} E[Diffuser]}
    E {-{-} F[Polarizing Filter 1]}
    F {-{-} G[LCD Panel]}
    C {-{-} G}
    G {-{-} H[Polarizing Filter 2]}
    H {-{-} I[Color Filters]}
    I {-{-} J[Screen Display]}
\end{verbatim}

\textbf{Operating Principle:}

\begin{itemize}
\tightlist
\item
  \textbf{Backlight}: Provides white light source
\item
  \textbf{Polarizing Filters}: Two filters at 90^\circ to each other
\item
  \textbf{Liquid Crystals}: Twist/untwist to control light passage
\item
  \textbf{TFT Array}: Controls voltage to each pixel
\item
  \textbf{Color Filters}: Create RGB colors from white light
\end{itemize}

\end{solutionbox}
\begin{mnemonicbox}
``BPLTC: Backlight Passes through Liquid crystals
That Color''

\end{mnemonicbox}
\subsection*{Question 2(c) [7 marks]}\label{q2c}

\textbf{Draw and Explain block diagram of PAL-D decoder.}

\begin{solutionbox}

\textbf{PAL-D Decoder:}

\begin{verbatim}
flowchart LR
    A[Composite Video Input] {-{-} B[Y/C Separator]}
    B {-{-} C[Luminance Y Processing]}
    B {-{-} D[Chrominance Processing]}
    D {-{-} E[Delay Line]}
    D {-{-} F[PAL Switch]}
    E {-{-} F}
    F {-{-} G[U/V Demodulator]}
    G {-{-} H[U Signal]}
    G {-{-} I[V Signal]}
    C {-{-} J[RGB Matrix]}
    H {-{-} J}
    I {-{-} J}
    J {-{-} K[RGB Output]}
\end{verbatim}

\textbf{PAL-D Decoder Components:}

\begin{itemize}
\tightlist
\item
  \textbf{Y/C Separator}: Separates luminance (Y) from chrominance (C)
\item
  \textbf{Luminance Processing}: Enhances brightness and contrast
\item
  \textbf{Chrominance Processing}: Extracts color subcarrier
\item
  \textbf{Delay Line}: Delays signal by one line (64µs)
\item
  \textbf{PAL Switch}: Reverses phase of V signal on alternate lines
\item
  \textbf{U/V Demodulator}: Extracts U (B-Y) and V (R-Y) color
  difference signals
\item
  \textbf{RGB Matrix}: Combines Y, U, V to produce RGB signals
\end{itemize}

\textbf{Key Feature}: Phase alternation corrects phase errors by
averaging consecutive lines

\end{solutionbox}
\begin{mnemonicbox}
``PAL Decodes Color Right By Switching, Delaying,
Unscrambling Variations''

\end{mnemonicbox}
\subsection*{Question 3(a) [3 marks]}\label{q3a}

\textbf{Give classification of rooftop Solar power plant and explain any
one plant.}

\begin{solutionbox}

\textbf{Rooftop Solar Power Plant Types:}

{\def\LTcaptype{none} % do not increment counter
\begin{longtable}[]{@{}ll@{}}
\toprule\noalign{}
Type & Description \\
\midrule\noalign{}
\endhead
\bottomrule\noalign{}
\endlastfoot
\textbf{Grid-Connected} & Connected to utility grid, no batteries \\
\textbf{Off-Grid} & Standalone system with battery storage \\
\textbf{Hybrid} & Can operate in both grid-connected and off-grid
modes \\
\end{longtable}
}

\textbf{Grid-Connected System:}

\begin{verbatim}
flowchart LR
    A[Solar Panels] {-{-} B[DC{-}AC Inverter]}
    B {-{-} C[Bi{-}directional Meter]}
    C {-{-} D[Utility Grid]}
    C {-{-} E[Home Load]}
\end{verbatim}

\begin{itemize}
\tightlist
\item
  \textbf{Solar Panels}: Convert sunlight to DC electricity
\item
  \textbf{Inverter}: Converts DC to grid-compatible AC
\item
  \textbf{Meter}: Measures power exported/imported
\item
  \textbf{Grid Connection}: Excess power fed to grid
\end{itemize}

\end{solutionbox}
\begin{mnemonicbox}
``GOH: Grid connects, Off-grid stores, Hybrid does
both''

\end{mnemonicbox}
\subsection*{Question 3(b) [4 marks]}\label{q3b}

\textbf{Give at least four technical specification of Refrigerator and
split Air condition each.}

\begin{solutionbox}

\textbf{Refrigerator Specifications:}

{\def\LTcaptype{none} % do not increment counter
\begin{longtable}[]{@{}ll@{}}
\toprule\noalign{}
Specification & Typical Range/Description \\
\midrule\noalign{}
\endhead
\bottomrule\noalign{}
\endlastfoot
\textbf{Capacity} & 150-750 liters \\
\textbf{Energy Rating} & Star rating (1-5 stars) \\
\textbf{Power Consumption} & 100-400 kWh per year \\
\textbf{Compressor Type} & Reciprocating or inverter \\
\textbf{Defrost System} & Manual, frost-free, or direct cool \\
\textbf{Refrigerant Type} & R-600a, R-134a \\
\textbf{Temperature Range} & 2-8^\circC (refrigerator), -18 to -24^\circC
(freezer) \\
\end{longtable}
}

\textbf{Split Air Conditioner Specifications:}

{\def\LTcaptype{none} % do not increment counter
\begin{longtable}[]{@{}
  >{\raggedright\arraybackslash}p{(\linewidth - 2\tabcolsep) * \real{0.3571}}
  >{\raggedright\arraybackslash}p{(\linewidth - 2\tabcolsep) * \real{0.6429}}@{}}
\toprule\noalign{}
\begin{minipage}[b]{\linewidth}\raggedright
Specification
\end{minipage} & \begin{minipage}[b]{\linewidth}\raggedright
Typical Range/Description
\end{minipage} \\
\midrule\noalign{}
\endhead
\bottomrule\noalign{}
\endlastfoot
\textbf{Cooling Capacity} & 1-2 tons (12,000-24,000 BTU/hr) \\
\textbf{Energy Efficiency Ratio (EER)} & 2.8-3.5 W/W \\
\textbf{ISEER Rating} & Star rating (1-5 stars) \\
\textbf{Power Consumption} & 800-2500 watts \\
\textbf{Refrigerant Type} & R-32, R-410A \\
\textbf{Noise Level} & 30-55 dB \\
\textbf{Operating Temperature Range} & 18-32^\circC (indoor), -5 to 55^\circC
(outdoor) \\
\end{longtable}
}

\end{solutionbox}
\begin{mnemonicbox}
``CERT: Capacity, Efficiency, Refrigerant Type,
Temperature''

\end{mnemonicbox}
\subsection*{Question 3(c) [7 marks]}\label{q3c}

\textbf{Explain working of Microwave oven with respect to its working
principle, functional block diagram and its safety precautions while in
operative condition.}

\begin{solutionbox}

\textbf{Microwave Oven Working Principle:} Food contains water
molecules, which are polar. Microwaves cause these molecules to rotate
rapidly (2.45 GHz), creating friction and generating heat throughout the
food.

\textbf{Functional Block Diagram:}

\begin{verbatim}
flowchart LR
    A[Control Panel] {-{-} B[Control Circuit]}
    B {-{-} C[Timer]}
    B {-{-} D[Power Control]}
    D {-{-} E[High Voltage Transformer]}
    E {-{-} F[High Voltage Capacitor]}
    E {-{-} G[High Voltage Diode]}
    F {-{-} H[Magnetron]}
    G {-{-} H}
    H {-{-} I[Waveguide]}
    I {-{-} J[Cooking Cavity]}
    K[Turntable Motor] {-{-} L[Turntable]}
    B {-{-} K}
    L {-{-} J}
\end{verbatim}

\textbf{Key Components:}

\begin{itemize}
\tightlist
\item
  \textbf{Magnetron}: Generates microwave radiation (2.45 GHz)
\item
  \textbf{Waveguide}: Directs microwaves to cooking cavity
\item
  \textbf{Turntable}: Ensures even cooking
\item
  \textbf{Control Circuit}: Manages timing and power
\item
  \textbf{High Voltage Circuit}: Powers the magnetron
\end{itemize}

\textbf{Safety Precautions:}

\begin{itemize}
\tightlist
\item
  \textbf{Door Interlocks}: Multiple switches prevent operation when
  door is open
\item
  \textbf{Monitoring Circuit}: Shuts down if interlocks fail
\item
  \textbf{Cavity Mesh Screen}: Blocks microwaves from escaping
\item
  \textbf{Never Operate Empty}: Can damage magnetron
\item
  \textbf{No Metal Objects}: Can cause arcing and damage
\item
  \textbf{Regular Cleaning}: Prevents food buildup and arcing
\item
  \textbf{Avoid Damaged Seals}: May allow microwave leakage
\end{itemize}

\end{solutionbox}
\begin{mnemonicbox}
``MICROWAVE: Magnetron Initiates Cooking, Radiation
Only Within Authorized Vessel Environment''

\end{mnemonicbox}
\subsection*{Question 3(a OR) [3
marks]}\label{question-3a-or-3-marks}

\textbf{State various hardware used in rooftop solar power plant and
explain solar panels used in it.}

\begin{solutionbox}

\textbf{Rooftop Solar Power Plant Hardware:}

{\def\LTcaptype{none} % do not increment counter
\begin{longtable}[]{@{}
  >{\raggedright\arraybackslash}p{(\linewidth - 2\tabcolsep) * \real{0.5238}}
  >{\raggedright\arraybackslash}p{(\linewidth - 2\tabcolsep) * \real{0.4762}}@{}}
\toprule\noalign{}
\begin{minipage}[b]{\linewidth}\raggedright
Component
\end{minipage} & \begin{minipage}[b]{\linewidth}\raggedright
Function
\end{minipage} \\
\midrule\noalign{}
\endhead
\bottomrule\noalign{}
\endlastfoot
\textbf{Solar Panels} & Convert sunlight to DC electricity \\
\textbf{Mounting Structure} & Supports panels at optimal angle \\
\textbf{Inverter} & Converts DC to AC power \\
\textbf{Batteries} (optional) & Store energy for later use \\
\textbf{Charge Controller} & Regulates battery charging (in off-grid
systems) \\
\textbf{Junction Boxes} & Provide connection points and protection \\
\textbf{Meters} & Measure power generation/consumption \\
\textbf{Cables \& Connectors} & Transmit power between components \\
\end{longtable}
}

\textbf{Solar Panels:}

\begin{verbatim}
flowchart LR
    A[Sunlight] {-{-} B[Tempered Glass]}
    B {-{-} C[Anti{-}Reflective Coating]}
    C {-{-} D[EVA Encapsulant]}
    D {-{-} E[Silicon Solar Cells]}
    E {-{-} F[Backsheet]}
    G[Aluminum Frame] {-{-} H[Complete Panel]}
    F {-{-} H}
\end{verbatim}

\begin{itemize}
\tightlist
\item
  \textbf{Monocrystalline}: Higher efficiency (15-22\%), darker color,
  longer lifespan
\item
  \textbf{Polycrystalline}: Lower cost, blue appearance, 13-17\%
  efficiency
\item
  \textbf{Thin-Film}: Flexible, lightweight, lower efficiency (10-12\%)
\item
  \textbf{Typical Output}: 250-400W per panel
\item
  \textbf{Lifespan}: 25-30 years with warranty
\end{itemize}

\end{solutionbox}
\begin{mnemonicbox}
``SIMPLE: Solar panels Integrate Multiple
Photovoltaic Layers Efficiently''

\end{mnemonicbox}
\subsection*{Question 3(b OR) [4
marks]}\label{question-3b-or-4-marks}

\textbf{Give at least four technical specification of Microwave oven and
washing machine each.}

\begin{solutionbox}

\textbf{Microwave Oven Specifications:}

{\def\LTcaptype{none} % do not increment counter
\begin{longtable}[]{@{}ll@{}}
\toprule\noalign{}
Specification & Typical Range/Description \\
\midrule\noalign{}
\endhead
\bottomrule\noalign{}
\endlastfoot
\textbf{Power Output} & 700-1200 watts \\
\textbf{Capacity} & 15-42 liters \\
\textbf{Frequency} & 2.45 GHz \\
\textbf{Operating Modes} & Microwave, grill, convection, combo \\
\textbf{Control Type} & Mechanical, digital, touch panel \\
\textbf{Power Consumption} & 1000-1500 watts \\
\textbf{Timer Range} & 0-60 minutes \\
\end{longtable}
}

\textbf{Washing Machine Specifications:}

{\def\LTcaptype{none} % do not increment counter
\begin{longtable}[]{@{}ll@{}}
\toprule\noalign{}
Specification & Typical Range/Description \\
\midrule\noalign{}
\endhead
\bottomrule\noalign{}
\endlastfoot
\textbf{Capacity} & 5-12 kg \\
\textbf{Washing Technology} & Agitator, impeller, drum \\
\textbf{Spin Speed} & 700-1600 RPM \\
\textbf{Water Consumption} & 30-80 liters per cycle \\
\textbf{Energy Rating} & Star rating (1-5 stars) \\
\textbf{Program Options} & 8-16 programs \\
\textbf{Motor Type} & Universal, inverter, direct drive \\
\end{longtable}
}

\end{solutionbox}
\begin{mnemonicbox}
``CPFWS: Capacity, Power, Frequency, Washing
technology, Spin speed''

\end{mnemonicbox}
\subsection*{Question 3(c OR) [7
marks]}\label{question-3c-or-7-marks}

\textbf{Give classification of washing machine. Explain working of top
load washing machine with respect to its functional block diagram and
working strategy/steps to wash clothes.}

\begin{solutionbox}

\textbf{Washing Machine Classification:}

{\def\LTcaptype{none} % do not increment counter
\begin{longtable}[]{@{}lll@{}}
\toprule\noalign{}
Type & Subtype & Key Features \\
\midrule\noalign{}
\endhead
\bottomrule\noalign{}
\endlastfoot
\textbf{Top Load} & Agitator & Central post that rotates \\
& Impeller & Rotating disk at bottom \\
\textbf{Front Load} & Horizontal Axis & Tumbling action, water
efficient \\
\textbf{By Automation} & Fully Automatic & Complete cycle automation \\
& Semi-Automatic & Manual intervention required \\
\textbf{By Function} & Washer Only & Washing function only \\
& Washer-Dryer & Combined washing and drying \\
\end{longtable}
}

\textbf{Top Load Washing Machine Functional Block Diagram:}

\begin{verbatim}
flowchart LR
    A[Control Panel] {-{-} B[Main Control Board]}
    B {-{-} C[Water Inlet Valve]}
    B {-{-} D[Water Level Sensor]}
    B {-{-} E[Motor Controller]}
    E {-{-} F[Main Motor]}
    F {-{-} G[Transmission]}
    G {-{-} H[Agitator/Impeller]}
    G {-{-} I[Spin Basket]}
    B {-{-} J[Drain Pump]}
    B {-{-} K[Timer]}
\end{verbatim}

\textbf{Working Strategy/Steps:}

\begin{enumerate}
\tightlist
\item
  \textbf{Fill Phase}:

  \begin{itemize}
  \tightlist
  \item
    Water inlet valve opens
  \item
    Tub fills to preset level
  \item
    Detergent mixed with water
  \end{itemize}
\item
  \textbf{Wash Phase}:

  \begin{itemize}
  \tightlist
  \item
    Motor drives agitator/impeller
  \item
    Creates water currents
  \item
    Clothing moves through soapy water
  \item
    Dirt loosened by mechanical action
  \end{itemize}
\item
  \textbf{Drain Phase}:

  \begin{itemize}
  \tightlist
  \item
    Drain pump activates
  \item
    Soapy water removed
  \end{itemize}
\item
  \textbf{Rinse Phase}:

  \begin{itemize}
  \tightlist
  \item
    Fresh water enters
  \item
    Agitator/impeller removes soap residue
  \item
    May repeat multiple times
  \end{itemize}
\item
  \textbf{Spin Phase}:

  \begin{itemize}
  \tightlist
  \item
    Basket rotates at high speed
  \item
    Centrifugal force removes water
  \item
    Clothes partially dried
  \end{itemize}
\end{enumerate}

\end{solutionbox}
\begin{mnemonicbox}
``FWDRS: Fill, Wash, Drain, Rinse, Spin''

\end{mnemonicbox}
\subsection*{Question 4(a) [3 marks]}\label{q4a}

\textbf{Explain working principle of laser printer. Give its technical
specifications.}

\begin{solutionbox}

\textbf{Laser Printer Working Principle:} Based on electrophotography
where a laser beam creates an electrostatic image on a photosensitive
drum, which attracts toner particles that are then transferred to paper
and fused with heat.

\textbf{Technical Specifications:}

{\def\LTcaptype{none} % do not increment counter
\begin{longtable}[]{@{}ll@{}}
\toprule\noalign{}
Specification & Typical Range/Values \\
\midrule\noalign{}
\endhead
\bottomrule\noalign{}
\endlastfoot
\textbf{Print Resolution} & 600-1200 dpi \\
\textbf{Print Speed} & 20-50 ppm (pages per minute) \\
\textbf{Duty Cycle} & 10,000-100,000 pages/month \\
\textbf{Memory} & 64-512 MB \\
\textbf{Connectivity} & USB, Ethernet, Wi-Fi \\
\textbf{Paper Capacity} & 250-500 sheets \\
\textbf{Power Consumption} & 300-800W (active), \textless10W
(standby) \\
\end{longtable}
}

\end{solutionbox}
\begin{mnemonicbox}
``RSCDCP: Resolution, Speed, Cycle, Duty,
Connectivity, Power''

\end{mnemonicbox}
\subsection*{Question 4(b) [4 marks]}\label{q4b}

\textbf{Explain working principle of Photo copier machine. State its
technical specifications.}

\begin{solutionbox}

\textbf{Photocopier Working Principle:} Uses xerography (dry copying)
process where light reflects off the original document onto a charged
photoreceptor drum, creating an electrical image that attracts toner
particles which are transferred and fused to paper.

\begin{verbatim}
flowchart LR
    A[Document Scanning] {-{-} B[Charging]}
    B {-{-} C[Exposure]}
    C {-{-} D[Development]}
    D {-{-} E[Transfer]}
    E {-{-} F[Fusing]}
    F {-{-} G[Final Copy]}
\end{verbatim}

\textbf{Technical Specifications:}

{\def\LTcaptype{none} % do not increment counter
\begin{longtable}[]{@{}ll@{}}
\toprule\noalign{}
Specification & Typical Values \\
\midrule\noalign{}
\endhead
\bottomrule\noalign{}
\endlastfoot
\textbf{Copy Speed} & 20-60 cpm (copies per minute) \\
\textbf{Resolution} & 600-1200 dpi \\
\textbf{Paper Size Support} & A5 to A3 \\
\textbf{Zoom Range} & 25\%-400\% \\
\textbf{Paper Capacity} & 250-2000 sheets \\
\textbf{Warm-up Time} & 10-30 seconds \\
\textbf{Multiple Copy} & 1-999 copies \\
\textbf{Power Consumption} & 1.0-1.5 kW (operating) \\
\end{longtable}
}

\end{solutionbox}
\begin{mnemonicbox}
``CRSPWMP: Copy speed, Resolution, Size, Paper
capacity, Warm-up, Multiple copy, Power''

\end{mnemonicbox}
\subsection*{Question 4(c) [7 marks]}\label{q4c}

\textbf{Draw and explain schematic of wireless CCTV camera system.
Explain Network video recorder. State types of camera used in CCTV
system and explain any one of them.}

\begin{solutionbox}

\textbf{Wireless CCTV Camera System:}

\begin{verbatim}
flowchart LR
    A[Camera with Image Sensor] {-{-} B[Signal Processor]}
    B {-{-} C[Compression Module]}
    C {-{-} D[Wireless Transmitter]}
    D {-{-} "Wi{-}Fi/RF Signal" {-}{-} E[Wireless Receiver]}
    E {-{-} F[Network Video Recorder]}
    F {-{-} G[Storage HDD]}
    F {-{-} H[Router]}
    H {-{-} I[Internet]}
    H {-{-} J[Monitoring Devices]}
\end{verbatim}

\textbf{Network Video Recorder (NVR):}

\begin{itemize}
\tightlist
\item
  \textbf{Function}: Records video streams from IP cameras
\item
  \textbf{Key Components}:

  \begin{itemize}
  \tightlist
  \item
    CPU: Processes multiple video streams
  \item
    Storage: Multiple hard drives (2-16TB typical)
  \item
    Network Interface: Connects to cameras and network
  \item
    Video Management Software: Controls recording schedules
  \end{itemize}
\item
  \textbf{Features}:

  \begin{itemize}
  \tightlist
  \item
    Motion detection recording
  \item
    Remote access capabilities
  \item
    Video analytics
  \item
    Simultaneous recording and playback
  \end{itemize}
\end{itemize}

\textbf{Types of CCTV Cameras:}

{\def\LTcaptype{none} % do not increment counter
\begin{longtable}[]{@{}ll@{}}
\toprule\noalign{}
Camera Type & Key Features \\
\midrule\noalign{}
\endhead
\bottomrule\noalign{}
\endlastfoot
\textbf{Dome Camera} & Ceiling mounted, vandal-resistant \\
\textbf{Bullet Camera} & Long-range viewing, weather-resistant \\
\textbf{PTZ Camera} & Pan, tilt, zoom capabilities \\
\textbf{Box Camera} & Customizable lens options \\
\textbf{Thermal Camera} & Heat detection, works in darkness \\
\textbf{Fisheye/360^\circ Camera} & Wide-angle panoramic view \\
\end{longtable}
}

\textbf{IP Camera Explained:}

\begin{itemize}
\tightlist
\item
  Uses digital signal processing
\item
  Connects directly to network
\item
  Has built-in web server
\item
  Higher resolution (2-8MP typical)
\item
  Power over Ethernet (PoE) capability
\item
  Two-way audio communication
\item
  Advanced analytics capabilities
\end{itemize}

\end{solutionbox}
\begin{mnemonicbox}
``WISP-NET: Wireless Images Securely Processed,
Networked, Enabling Tracking''

\end{mnemonicbox}
\subsection*{Question 4(a OR) [3
marks]}\label{question-4a-or-3-marks}

\textbf{Explain working principle of inkjet printer. Give its technical
specifications.}

\begin{solutionbox}

\textbf{Inkjet Printer Working Principle:} Creates images by propelling
tiny droplets of liquid ink onto paper. The printhead contains hundreds
of microscopic nozzles that eject ink droplets precisely where needed to
form text and images.

\begin{verbatim}
flowchart LR
    A[Print Command] {-{-} B[Controller Circuit]}
    B {-{-} C[Printhead Carriage Movement]}
    B {-{-} D[Paper Feed]}
    B {-{-} E[Ink Ejection]}
    E {-{-} F[Droplet Formation]}
    F {-{-} G[Image Creation on Paper]}
\end{verbatim}

\textbf{Technical Specifications:}

{\def\LTcaptype{none} % do not increment counter
\begin{longtable}[]{@{}ll@{}}
\toprule\noalign{}
Specification & Typical Values \\
\midrule\noalign{}
\endhead
\bottomrule\noalign{}
\endlastfoot
\textbf{Print Resolution} & 1200-4800 dpi \\
\textbf{Print Speed} & 8-20 ppm (black), 4-15 ppm (color) \\
\textbf{Ink Type} & Dye-based or pigment-based \\
\textbf{Connectivity} & USB, Wi-Fi, Ethernet \\
\textbf{Paper Capacity} & 100-250 sheets \\
\textbf{Droplet Size} & 1-3 picoliters \\
\textbf{Color System} & 4-8 ink cartridges \\
\end{longtable}
}

\end{solutionbox}
\begin{mnemonicbox}
``RIPS-CCD: Resolution, Ink type, Print speed, Size
of droplet, Connectivity, Capacity, Droplet''

\end{mnemonicbox}
\subsection*{Question 4(b OR) [4
marks]}\label{question-4b-or-4-marks}

\textbf{Explain maintenance and trouble shooting of television receiver
and Washing machine.}

\begin{solutionbox}

\textbf{Television Maintenance:}

{\def\LTcaptype{none} % do not increment counter
\begin{longtable}[]{@{}ll@{}}
\toprule\noalign{}
Maintenance Task & Frequency \\
\midrule\noalign{}
\endhead
\bottomrule\noalign{}
\endlastfoot
Dust cleaning & Monthly \\
Software updates & As available \\
Screen cleaning & Weekly \\
Ventilation check & Monthly \\
Brightness/contrast adjustment & As needed \\
\end{longtable}
}

\textbf{Television Troubleshooting:}

{\def\LTcaptype{none} % do not increment counter
\begin{longtable}[]{@{}
  >{\raggedright\arraybackslash}p{(\linewidth - 2\tabcolsep) * \real{0.3333}}
  >{\raggedright\arraybackslash}p{(\linewidth - 2\tabcolsep) * \real{0.6667}}@{}}
\toprule\noalign{}
\begin{minipage}[b]{\linewidth}\raggedright
Problem
\end{minipage} & \begin{minipage}[b]{\linewidth}\raggedright
Possible Solution
\end{minipage} \\
\midrule\noalign{}
\endhead
\bottomrule\noalign{}
\endlastfoot
No power & Check power cable, outlet, fuse \\
No picture but sound works & Check video cable, picture settings \\
No sound but picture works & Check audio settings, speaker
connections \\
Poor picture quality & Adjust settings, check signal strength \\
Remote not working & Replace batteries, clean IR sensor \\
\end{longtable}
}

\textbf{Washing Machine Maintenance:}

{\def\LTcaptype{none} % do not increment counter
\begin{longtable}[]{@{}ll@{}}
\toprule\noalign{}
Maintenance Task & Frequency \\
\midrule\noalign{}
\endhead
\bottomrule\noalign{}
\endlastfoot
Clean drum and gasket & Monthly \\
Check/clean filter & Monthly \\
Clean detergent drawer & Monthly \\
Run empty hot cycle & Quarterly \\
Check hoses for leaks & Quarterly \\
\end{longtable}
}

\textbf{Washing Machine Troubleshooting:}

{\def\LTcaptype{none} % do not increment counter
\begin{longtable}[]{@{}ll@{}}
\toprule\noalign{}
Problem & Possible Solution \\
\midrule\noalign{}
\endhead
\bottomrule\noalign{}
\endlastfoot
Not spinning & Check load balance, door lock \\
Leaking water & Check hoses, door seal, drain pump \\
Not draining & Clean filter, check drain hose \\
Excessive vibration & Level machine, check suspension \\
Door won't open & Wait for safety lock release \\
\end{longtable}
}

\end{solutionbox}
\begin{mnemonicbox}
``CREST: Clean Regularly, Examine connections,
Service filters, Test functions''

\end{mnemonicbox}
\subsection*{Question 4(c OR) [7
marks]}\label{question-4c-or-7-marks}

\textbf{Define CCTV. Explain with schematic CCTV camera system installed
in a home. Describe analog camera, Digital camera and IP camera and
differentiate them.}

\begin{solutionbox}

\textbf{CCTV (Closed-Circuit Television):} A video surveillance system
that transmits signals to a specific, limited set of monitors, unlike
broadcast television. It's used for surveillance and security monitoring
in homes, businesses, and public spaces.

\textbf{Home CCTV System Schematic:}

\begin{verbatim}
flowchart LR
    A[Cameras] {-{-} B[DVR/NVR]}
    B {-{-} C[Storage HDD]}
    B {-{-} D[Monitor]}
    B {-{-} E[Router]}
    E {-{-} F[Internet]}
    F {-{-} G[Remote Viewing Devices]}
    E {-{-} G}
    H[Power Supply] {-{-} A}
    H {-{-} B}
\end{verbatim}

\textbf{Camera Types:}

\textbf{Analog Camera:}

\begin{itemize}
\tightlist
\item
  Transmits continuous analog signal via coaxial cable
\item
  Typically 720\times576 resolution (standard definition)
\item
  Requires DVR (Digital Video Recorder) for recording
\item
  Limited cable run distance (300-500m)
\item
  Simpler installation, lower cost
\end{itemize}

\textbf{Digital Camera:}

\begin{itemize}
\tightlist
\item
  Converts analog signal to digital at camera
\item
  Uses coaxial cable or twisted pair for transmission
\item
  Better resolution than analog (up to 2MP)
\item
  Improved image quality and stability
\item
  Works with traditional DVR systems
\end{itemize}

\textbf{IP Camera:}

\begin{itemize}
\tightlist
\item
  Fully digital from capture to transmission
\item
  Connects directly to network via Ethernet/Wi-Fi
\item
  High resolution (2-8MP or higher)
\item
  Uses NVR (Network Video Recorder) for recording
\item
  Advanced features: remote viewing, analytics, PoE
\end{itemize}

\textbf{Comparison Table:}

{\def\LTcaptype{none} % do not increment counter
\begin{longtable}[]{@{}llll@{}}
\toprule\noalign{}
Feature & Analog Camera & Digital Camera & IP Camera \\
\midrule\noalign{}
\endhead
\bottomrule\noalign{}
\endlastfoot
Signal & Analog & Analog-to-Digital & Digital \\
Resolution & SD (up to 700 TVL) & HD (up to 2MP) & HD/UHD (2-12MP) \\
Cabling & Coaxial & Coaxial/Twisted pair & Ethernet/Wi-Fi \\
Recorder & DVR & DVR & NVR \\
Setup Complexity & Low & Medium & High \\
Price & Lower & Medium & Higher \\
Remote Access & Limited & Limited & Advanced \\
\end{longtable}
}

\end{solutionbox}
\begin{mnemonicbox}
``ADI: Analog uses Decaying technology, IP represents
Innovation''

\end{mnemonicbox}
\subsection*{Question 5(a) [3 marks]}\label{q5a}

\textbf{Define maintenance. State its types. Explain any one of them.}

\begin{solutionbox}

\textbf{Maintenance:} The process of preserving equipment in operational
condition by regular inspection, servicing, repair, and replacement of
components to prevent failures and extend equipment life.

\textbf{Types of Maintenance:}

{\def\LTcaptype{none} % do not increment counter
\begin{longtable}[]{@{}
  >{\raggedright\arraybackslash}p{(\linewidth - 2\tabcolsep) * \real{0.3158}}
  >{\raggedright\arraybackslash}p{(\linewidth - 2\tabcolsep) * \real{0.6842}}@{}}
\toprule\noalign{}
\begin{minipage}[b]{\linewidth}\raggedright
Type
\end{minipage} & \begin{minipage}[b]{\linewidth}\raggedright
Description
\end{minipage} \\
\midrule\noalign{}
\endhead
\bottomrule\noalign{}
\endlastfoot
\textbf{Preventive} & Scheduled regular maintenance to prevent
failures \\
\textbf{Predictive} & Based on monitoring and data analysis to predict
failures \\
\textbf{Corrective/Breakdown} & Performed after equipment failure
occurs \\
\textbf{Condition-based} & Based on actual equipment condition \\
\textbf{Reliability-centered} & Focuses on maintaining system
function \\
\end{longtable}
}

\textbf{Preventive Maintenance:}

\begin{itemize}
\tightlist
\item
  Conducted at scheduled intervals regardless of equipment condition
\item
  Includes cleaning, lubricating, adjusting, and replacing wear
  components
\item
  Aims to prevent unexpected failures and extend equipment life
\item
  Follows manufacturer's service recommendations
\item
  Examples: filter changes, belt replacements, calibration, lubrication
\end{itemize}

\end{solutionbox}
\begin{mnemonicbox}
``PPCR: Prevent Problems through Checkups Regularly''

\end{mnemonicbox}
\subsection*{Question 5(b) [4 marks]}\label{q5b}

\textbf{Explaining maintenance of PA systems and Home theatre system.}

\begin{solutionbox}

\textbf{PA System Maintenance:}

{\def\LTcaptype{none} % do not increment counter
\begin{longtable}[]{@{}
  >{\raggedright\arraybackslash}p{(\linewidth - 2\tabcolsep) * \real{0.3793}}
  >{\raggedright\arraybackslash}p{(\linewidth - 2\tabcolsep) * \real{0.6207}}@{}}
\toprule\noalign{}
\begin{minipage}[b]{\linewidth}\raggedright
Component
\end{minipage} & \begin{minipage}[b]{\linewidth}\raggedright
Maintenance Task
\end{minipage} \\
\midrule\noalign{}
\endhead
\bottomrule\noalign{}
\endlastfoot
\textbf{Speakers} & Check connections, inspect for damage, clean dust \\
\textbf{Amplifiers} & Clean cooling vents, check for overheating,
inspect cables \\
\textbf{Microphones} & Clean grilles, check cables, test for proper
operation \\
\textbf{Cables} & Inspect for damage, verify connections are tight \\
\textbf{Mixers} & Clean faders/knobs, check input/output levels \\
\end{longtable}
}

\textbf{Key Procedures:}

\begin{itemize}
\tightlist
\item
  Verify proper grounding to avoid noise
\item
  Test system at low volume before use
\item
  Keep equipment dry and dust-free
\item
  Follow manufacturer's cleaning instructions
\item
  Document any issues for troubleshooting
\end{itemize}

\textbf{Home Theatre System Maintenance:}

{\def\LTcaptype{none} % do not increment counter
\begin{longtable}[]{@{}
  >{\raggedright\arraybackslash}p{(\linewidth - 2\tabcolsep) * \real{0.3793}}
  >{\raggedright\arraybackslash}p{(\linewidth - 2\tabcolsep) * \real{0.6207}}@{}}
\toprule\noalign{}
\begin{minipage}[b]{\linewidth}\raggedright
Component
\end{minipage} & \begin{minipage}[b]{\linewidth}\raggedright
Maintenance Task
\end{minipage} \\
\midrule\noalign{}
\endhead
\bottomrule\noalign{}
\endlastfoot
\textbf{AV Receiver} & Keep ventilated, update firmware, check
connections \\
\textbf{Speakers} & Check connections, clean dust, verify positioning \\
\textbf{Subwoofer} & Check for rattling, adjust placement for optimal
sound \\
\textbf{Display Device} & Clean screen properly, check settings \\
\textbf{Source Devices} & Clean optical drives, update firmware \\
\end{longtable}
}

\textbf{Key Procedures:}

\begin{itemize}
\tightlist
\item
  Calibrate audio settings periodically
\item
  Verify proper HDMI connections
\item
  Keep remote controls clean and with fresh batteries
\item
  Maintain proper ventilation for all components
\item
  Run speaker test tones to verify all channels
\end{itemize}

\end{solutionbox}
\begin{mnemonicbox}
``CAVS: Clean, Adjust, Verify connections, Service
regularly''

\end{mnemonicbox}
\subsection*{Question 5(c) [7 marks]}\label{q5c}

\textbf{Draw and Explain block diagram of DTH technology. Discuss
hardware components used in DTH system. Discuss various modern features
currently provided in current DTH system.}

\begin{solutionbox}

\textbf{DTH (Direct To Home) Technology Block Diagram:}

\begin{verbatim}
flowchart LR
    A[Broadcaster] {-{-} B[Uplink Center]}
    B {-{-} C[Satellite]}
    C {-{-} D[Dish Antenna]}
    D {-{-} E[LNB]}
    E {-{-} F[Set{-}Top Box]}
    F {-{-} G[Television]}
    H[Remote Control] {-{-} F}
\end{verbatim}

\textbf{DTH Hardware Components:}

\begin{enumerate}
\tightlist
\item
  \textbf{Satellite Dish Antenna}:

  \begin{itemize}
  \tightlist
  \item
    Parabolic reflector that captures satellite signals
  \item
    Size typically 45-90cm diameter
  \item
    Must be accurately aligned to satellite position
  \end{itemize}
\item
  \textbf{LNB (Low Noise Block)}:

  \begin{itemize}
  \tightlist
  \item
    Receives signals reflected by dish
  \item
    Amplifies weak signals while minimizing noise
  \item
    Converts high frequency signals to lower frequency
  \item
    Typical frequency: 10.7-12.75 GHz down to 950-2150 MHz
  \end{itemize}
\item
  \textbf{Coaxial Cable}:

  \begin{itemize}
  \tightlist
  \item
    Connects LNB to set-top box
  \item
    RG-6 type with F-connectors
  \item
    Minimal signal loss characteristics
  \end{itemize}
\item
  \textbf{Set-Top Box (STB)}:

  \begin{itemize}
  \tightlist
  \item
    Demodulates and decodes satellite signals
  \item
    Contains conditional access system
  \item
    Processes MPEG-2/MPEG-4/H.264 video
  \item
    Provides user interface and program guide
  \end{itemize}
\item
  \textbf{Smart Card}:

  \begin{itemize}
  \tightlist
  \item
    Contains subscriber information
  \item
    Enables decryption of encrypted channels
  \item
    Stores subscription details
  \end{itemize}
\end{enumerate}

\textbf{Modern Features of DTH Systems:}

{\def\LTcaptype{none} % do not increment counter
\begin{longtable}[]{@{}
  >{\raggedright\arraybackslash}p{(\linewidth - 2\tabcolsep) * \real{0.4091}}
  >{\raggedright\arraybackslash}p{(\linewidth - 2\tabcolsep) * \real{0.5909}}@{}}
\toprule\noalign{}
\begin{minipage}[b]{\linewidth}\raggedright
Feature
\end{minipage} & \begin{minipage}[b]{\linewidth}\raggedright
Description
\end{minipage} \\
\midrule\noalign{}
\endhead
\bottomrule\noalign{}
\endlastfoot
\textbf{HD \& 4K Channels} & High-definition and ultra-high-definition
content \\
\textbf{Interactive TV} & On-demand content, voting, games \\
\textbf{Multi-room Viewing} & Same subscription on multiple TVs \\
\textbf{Recording Capability} & Built-in or external DVR
functionality \\
\textbf{Mobile App Control} & Remote control via smartphone \\
\textbf{Voice Control} & Voice-activated commands \\
\textbf{Catch-up TV} & Watch missed programs for several days \\
\textbf{OTT Integration} & Access to Netflix, Prime Video, etc. \\
\textbf{Content Recommendation} & AI-based personalized suggestions \\
\textbf{Parental Controls} & Content restriction based on ratings \\
\end{longtable}
}

\end{solutionbox}
\begin{mnemonicbox}
``DISH-STB: Direct Information Satellite Hub -
Signals Transmitted to Box''

\end{mnemonicbox}
\subsection*{Question 5(a OR) [3
marks]}\label{question-5a-or-3-marks}

\textbf{Differentiate between predictive and preventive maintenance.}

\begin{solutionbox}

{\def\LTcaptype{none} % do not increment counter
\begin{longtable}[]{@{}
  >{\raggedright\arraybackslash}p{(\linewidth - 4\tabcolsep) * \real{0.1429}}
  >{\raggedright\arraybackslash}p{(\linewidth - 4\tabcolsep) * \real{0.4286}}
  >{\raggedright\arraybackslash}p{(\linewidth - 4\tabcolsep) * \real{0.4286}}@{}}
\toprule\noalign{}
\begin{minipage}[b]{\linewidth}\raggedright
Aspect
\end{minipage} & \begin{minipage}[b]{\linewidth}\raggedright
Predictive Maintenance
\end{minipage} & \begin{minipage}[b]{\linewidth}\raggedright
Preventive Maintenance
\end{minipage} \\
\midrule\noalign{}
\endhead
\bottomrule\noalign{}
\endlastfoot
\textbf{Basis} & Equipment condition & Time or usage intervals \\
\textbf{Approach} & Data-driven monitoring & Pre-scheduled service \\
\textbf{Timing} & Just before failure predicted & Regular intervals
regardless of condition \\
\textbf{Tools Used} & Sensors, vibration analysis, thermal imaging &
Maintenance schedules, checklists \\
\textbf{Cost} & Higher initial setup, lower long-term & Lower initial,
potentially higher long-term \\
\textbf{Downtime} & Minimal, planned & Regular planned downtime \\
\textbf{Resource Efficiency} & Higher (service only when needed) & Lower
(may service unnecessarily) \\
\textbf{Example} & Oil analysis showing degradation triggers change &
Oil changed every 5,000 km regardless of condition \\
\end{longtable}
}

\end{solutionbox}
\begin{mnemonicbox}
``TIME vs DATA: Timed Intervals Maintenance
Everywhere vs Data Analysis Triggers Action''

\end{mnemonicbox}
\subsection*{Question 5(b OR) [4
marks]}\label{question-5b-or-4-marks}

\textbf{Describe troubleshooting procedure and safety precautions for
microwave oven.}

\begin{solutionbox}

\textbf{Microwave Oven Troubleshooting Procedure:}

\begin{enumerate}
\tightlist
\item
  \textbf{Initial Assessment}:

  \begin{itemize}
  \tightlist
  \item
    Verify power connection and outlet
  \item
    Check display/lights for power indication
  \item
    Listen for normal operational sounds
  \end{itemize}
\item
  \textbf{Common Issues and Checks}:

  \begin{itemize}
  \tightlist
  \item
    \textbf{No Power}: Check fuse, door switches, control board
  \item
    \textbf{No Heating}: Check magnetron, high voltage components
  \item
    \textbf{Turntable Not Working}: Check motor, drive coupling
  \item
    \textbf{Noisy Operation}: Inspect fan, magnetron, turntable
  \item
    \textbf{Sparking}: Look for metal objects, damaged rack/cavity
  \end{itemize}
\item
  \textbf{Diagnostic Steps}:

  \begin{itemize}
  \tightlist
  \item
    Check error codes on display
  \item
    Test door interlock switches
  \item
    Verify proper voltage at components
  \item
    Inspect for burnt components or wiring
  \end{itemize}
\end{enumerate}

\textbf{Safety Precautions:}

{\def\LTcaptype{none} % do not increment counter
\begin{longtable}[]{@{}ll@{}}
\toprule\noalign{}
Precaution & Reason \\
\midrule\noalign{}
\endhead
\bottomrule\noalign{}
\endlastfoot
\textbf{Unplug Before Service} & Prevents electric shock \\
\textbf{Discharge Capacitor} & Stores lethal voltage even when
unplugged \\
\textbf{Wait 60 Seconds} & Allows capacitor to discharge naturally \\
\textbf{Never Run Empty} & Can damage magnetron \\
\textbf{Check Microwave Leakage} & Using calibrated leakage detector \\
\textbf{Don't Defeat Interlocks} & Essential safety feature \\
\textbf{Wear Insulated Gloves} & Protection from electrical shock \\
\textbf{Verify Repairs} & Test thoroughly before returning to service \\
\end{longtable}
}

\end{solutionbox}
\begin{mnemonicbox}
``DUEL-SAFE: Disconnect power, Use discharge tool,
Examine systematically, Look for damage - Safety Always First, Every
time''

\end{mnemonicbox}
\subsection*{Question 5(c OR) [7
marks]}\label{question-5c-or-7-marks}

\textbf{Draw and explain block diagram of PA system. Discuss design
parameters while designing for auditorium. Draw connection diagram of
four 8 Ohm speakers to PA system amplifier having 8 Ohm as output
impedance.}

\begin{solutionbox}

\textbf{PA System Block Diagram:}

\begin{verbatim}
flowchart LR
    A[Input Sources] {-{-} B[Mixer/Preamplifier]}
    B {-{-} C[Equalizer]}
    C {-{-} D[Power Amplifier]}
    D {-{-} E[Crossover Network]}
    E {-{-} F[Speakers]}
    G[Microphones] {-{-} A}
    H[Line Level Sources] {-{-} A}
    I[Feedback Suppressor] {-{-} B}
\end{verbatim}

\textbf{PA System Components:}

\begin{itemize}
\tightlist
\item
  \textbf{Input Sources}: Microphones, instruments, media players
\item
  \textbf{Mixer/Preamplifier}: Combines and adjusts input signals
\item
  \textbf{Equalizer}: Adjusts frequency response
\item
  \textbf{Power Amplifier}: Increases signal power to drive speakers
\item
  \textbf{Crossover Network}: Divides frequencies for appropriate
  speakers
\item
  \textbf{Speakers}: Converts electrical signals to sound
\item
  \textbf{Feedback Suppressor}: Prevents audio feedback
\end{itemize}

\textbf{Auditorium Design Parameters:}

{\def\LTcaptype{none} % do not increment counter
\begin{longtable}[]{@{}
  >{\raggedright\arraybackslash}p{(\linewidth - 2\tabcolsep) * \real{0.4231}}
  >{\raggedright\arraybackslash}p{(\linewidth - 2\tabcolsep) * \real{0.5769}}@{}}
\toprule\noalign{}
\begin{minipage}[b]{\linewidth}\raggedright
Parameter
\end{minipage} & \begin{minipage}[b]{\linewidth}\raggedright
Consideration
\end{minipage} \\
\midrule\noalign{}
\endhead
\bottomrule\noalign{}
\endlastfoot
\textbf{Room Acoustics} & Reverberation time (1.0-2.0s optimal), echo
control \\
\textbf{Speaker Placement} & Coverage angle, distance, height,
minimizing feedback \\
\textbf{Power Requirements} & 1-2W per person for speech, 2-3W for
music \\
\textbf{Frequency Response} & 100Hz-12kHz for speech, 40Hz-16kHz for
music \\
\textbf{Speech Intelligibility} & STI (Speech Transmission Index)
\textgreater{} 0.60 \\
\textbf{Ambient Noise} & NC-25 to NC-30 (Noise Criterion) \\
\textbf{Sound Pressure Level} & 85-95dB for optimal listening \\
\textbf{Line Array vs.~Point Source} & Based on room size and shape \\
\end{longtable}
}

\textbf{Connection Diagram for 8Ω Speakers to 8Ω Amplifier:}

\textbf{Series-Parallel Connection:}

\begin{verbatim}
     Amplifier
  Output (8 Ohm)
        |
        |
   +{-{-}{-}{-}+{-}{-}{-}{-}+}
   |         |
   |         |
   v         v
 Speaker1  Speaker3
 (8 Ohm)   (8 Ohm)
   |         |
   |         |
   v         v
 Speaker2  Speaker4
 (8 Ohm)   (8 Ohm)
   |         |
   |         |
   +{-{-}{-}{-}{-}{-}{-}{-}{-}+}
\end{verbatim}

\begin{itemize}
\tightlist
\item
  Two parallel branches of two speakers in series
\item
  Each series branch = 16Ω (8Ω + 8Ω)
\item
  Two 16Ω branches in parallel = 8Ω total (16Ω \div 2)
\item
  Maintains proper impedance match with amplifier
\item
  Distributes power evenly to all speakers
\end{itemize}

\end{solutionbox}
\begin{mnemonicbox}
``PASS: Proper Amplification, Speaker placement,
Series-parallel wiring''

\end{mnemonicbox}

\end{document}
