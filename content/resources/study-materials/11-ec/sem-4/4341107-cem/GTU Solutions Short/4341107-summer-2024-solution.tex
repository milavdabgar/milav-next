\documentclass{article}

% content/resources/templates/preamble.tex
\usepackage[margin=0.6in]{geometry}
\author{Milav Dabgar}
\usepackage{amsmath,amssymb,amsthm}
\usepackage{booktabs}
\usepackage{multirow}
\usepackage{xcolor}
\usepackage{tcolorbox}
\tcbuselibrary{breakable,skins}
\usepackage[colorlinks=true,linkcolor=blue]{hyperref}
\usepackage{titlesec}
\usepackage{enumitem}
\usepackage{tikz}
\usepackage{pgfplots}
\usepackage{circuitikz}
\usepackage[version=4]{mhchem}
\usepackage{longtable}
\usepackage{array}
\usepackage{float}
\usepackage{caption}
\usepackage{listings}

\lstset{
  basicstyle=\small\ttfamily,
  breaklines=true,
  breakatwhitespace=false,
  postbreak=\mbox{\textcolor{red}{$\hookrightarrow$}\space},
  float=false,
  numbers=left,
  numberstyle=\tiny\color{gray},
  numbersep=10pt,
  xleftmargin=2em,
  keywordstyle=\color{blue},
  commentstyle=\color{green!60!black},
  stringstyle=\color{purple},
  backgroundcolor=\color{gray!5},
  showstringspaces=false,
  tabsize=2,
  captionpos=b,
  keepspaces=true,
  columns=flexible
}

\pgfplotsset{compat=1.18}
\usetikzlibrary{shapes,arrows,positioning,calc,patterns,decorations.pathmorphing,decorations.markings,arrows.meta}

% Color scheme
\definecolor{headcolor}{RGB}{0,102,204}
\definecolor{keycolor}{RGB}{220,20,60}
\definecolor{solutioncolor}{RGB}{34,139,34}
\definecolor{mnemoniccolor}{RGB}{148,0,211}
\definecolor{codecolor}{RGB}{0,0,100}

% Spacing
\setlength{\parskip}{3pt}
\setlist[itemize]{nosep}
\setlist[enumerate]{nosep}

% Title formatting
\titleformat{\section}{\Large\bfseries\color{headcolor}}{\thesection}{1em}{}
\titleformat{\subsection}{\large\bfseries\color{headcolor}}{\thesubsection}{1em}{}

% Pandoc tightlist compatibility
\providecommand{\tightlist}{%
  \setlength{\itemsep}{0pt}\setlength{\parskip}{0pt}}

% Pandoc longtable compatibility
\newcounter{none}
\def\thenone{}


% content/resources/templates/english-boxes.tex
% This file is currently empty - it exists to maintain consistency with the import structure.
% Add custom environments here if needed in the future.


% Custom commands for GTU solutions
% This file defines semantic commands for consistent formatting

% Question command with automatic formatting
\newcommand{\question}[2]{%
  \section*{Question #1}%
  \textbf{#2}%
}

% OR question variant
\newcommand{\questionor}[2]{%
  \section*{Question #1 OR}%
  \textbf{#2}%
}

% Proper table environment with caption
\newenvironment{answertable}[1]{%
  \begin{table}[htbp]
  \centering
  \caption{#1}
}{%
  \end{table}
}

% Proper figure environment for diagrams
\newenvironment{answerdiagram}[1]{%
  \begin{figure}[htbp]
  \centering
  \caption{#1}
}{%
  \end{figure}
}

% Semantic markup for key terms
\newcommand{\keyword}[1]{\textbf{#1}}
\newcommand{\code}[1]{\texttt{#1}}
\newcommand{\classname}[1]{\texttt{#1}}
\newcommand{\methodname}[1]{\texttt{#1}}

% Proper quotation marks
\newcommand{\mnemonic}[1]{``#1''}


\title{Consumer Electronics \& Maintenance (4341107) - Summer 2024 Solution}
\date{June 21, 2024}

\begin{document}
\maketitle

\questionmarks{1(a)}{3}{Give definition (Only) of Loudness, Fidelity and Reverberation}
\begin{solutionbox}
    \begin{itemize}
        \item \textbf{Loudness}: The subjective perception of sound intensity by the human ear, measured in decibels (dB).
        \item \textbf{Fidelity}: The degree to which a system reproduces sound that is faithful to the original input signal.
        \item \textbf{Reverberation}: The persistence of sound after the original sound source has stopped, caused by multiple reflections within an enclosed space.
    \end{itemize}

    \begin{mnemonicbox}
        \mnemonic{LFR: Listen Faithfully to Room echoes}
    \end{mnemonicbox}
\end{solutionbox}

\questionmarks{1(b)}{4}{Draw and explain block diagram of PA system}
\begin{solutionbox}
    \textbf{PA System Block Diagram:}

    \begin{figure}[H]
        \centering
        \begin{tikzpicture}[gtu flow]
            \node (Mic) [gtu block] {Microphone};
            \node (Pre) [gtu block, right=of Mic] {Preamplifier};
            \node (Mixer) [gtu block, right=of Pre] {Mixer};
            
            \node (Input) [gtu block, above=of Mixer] {Audio Input};
            \node (EQ) [gtu block, below=of Mixer] {Equalizer};
            
            \node (Power) [gtu block, right=of Mixer] {Power Amplifier};
            \node (Speaker) [gtu block, right=of Power] {Loudspeaker};
            
            \draw [gtu arrow] (Mic) -- (Pre);
            \draw [gtu arrow] (Pre) -- (Mixer);
            \draw [gtu arrow] (Input) -- (Mixer);
            \draw [gtu arrow] (Mixer) -- (Power);
            \draw [gtu arrow] (Power) -- (Speaker);
            \draw [gtu arrow] (EQ) -- (Mixer);
        \end{tikzpicture}
        \caption{Public Address System}
    \end{figure}

    \textbf{Explanation:}
    \begin{itemize}
        \item \textbf{Microphone}: Converts sound waves into electrical signals.
        \item \textbf{Preamplifier}: Boosts weak microphone signals to line level.
        \item \textbf{Mixer}: Combines multiple audio signals and adjusts levels.
        \item \textbf{Power Amplifier}: Increases signal power to drive loudspeakers.
        \item \textbf{Loudspeaker}: Converts electrical signals back into sound waves.
    \end{itemize}

    \begin{mnemonicbox}
        \mnemonic{MPMEL: Many People Make Excellent Listeners}
    \end{mnemonicbox}
\end{solutionbox}

\questionmarks{1(c)}{7}{Discuss any two characteristic of Microphone and Explain wireless microphone}
\begin{solutionbox}
    \textbf{Microphone Characteristics:}
    \textbf{Table: Microphone Characteristics} \\
    \begin{tabulary}{\linewidth}{|l|L|}
        \hline
        \textbf{Characteristic} & \textbf{Description} \\
        \hline
        \textbf{Sensitivity} & Measures how efficiently microphone converts acoustic pressure to electrical output (mV/Pa) \\
        \hline
        \textbf{Directional Pattern} & Defines pickup area (omnidirectional, cardioid, hypercardioid, bidirectional) \\
        \hline
    \end{tabulary}

    \textbf{Wireless Microphone Working:}

    \begin{figure}[H]
        \centering
        \begin{tikzpicture}[gtu flow]
            \node (Mic) [gtu block] {Mic Element};
            \node (Proc) [gtu block, right=of Mic] {Audio Processor};
            \node (Tx) [gtu block, right=of Proc] {RF Transmitter};
            \node (Ant1) [gtu block, above=of Tx, minimum width=1cm, minimum height=1cm] {Antenna};
            
            \node (Ant2) [gtu block, right=3cm of Ant1, minimum width=1cm, minimum height=1cm] {Antenna};
            \node (Rx) [gtu block, below=of Ant2] {RF Receiver};
            \node (Out) [gtu block, right=of Rx] {Audio Output};
            
            \draw [gtu arrow] (Mic) -- (Proc);
            \draw [gtu arrow] (Proc) -- (Tx);
            \draw [gtu arrow] (Tx) -- (Ant1);
            
            \draw [gtu arrow, dashed] (Ant1) -- node[above] {Radio Waves} (Ant2);
            \draw [gtu arrow] (Ant2) -- (Rx);
            \draw [gtu arrow] (Rx) -- (Out);
        \end{tikzpicture}
        \caption{Wireless Microphone System}
    \end{figure}

    \begin{itemize}
        \item \textbf{Microphone Element}: Captures sound and converts to electrical signals.
        \item \textbf{RF Transmitter}: Modulates audio onto radio frequency carrier.
        \item \textbf{Transmission}: Typical frequency bands are UHF (470-698 MHz) or VHF (174-216 MHz).
        \item \textbf{RF Receiver}: Demodulates signal back to audio.
        \item \textbf{Advantages}: Mobility, no cable restrictions, reduces stage clutter.
    \end{itemize}

    \begin{mnemonicbox}
        \mnemonic{SMART: Sensitivity Measures Audio Response Truly}
    \end{mnemonicbox}
\end{solutionbox}

\questionmarks{1(c) OR}{7}{Discuss any two characteristics of loudspeaker and explain permanent magnet loudspeaker.}
\begin{solutionbox}
    \textbf{Loudspeaker Characteristics:}
    \textbf{Table: Loudspeaker Specs} \\
    \begin{tabulary}{\linewidth}{|l|L|}
        \hline
        \textbf{Characteristic} & \textbf{Description} \\
        \hline
        \textbf{Frequency Response} & Range of frequencies (Hz) speaker can reproduce (typically 20Hz-20kHz) \\
        \hline
        \textbf{Impedance} & Electrical resistance (ohms) that affects power transfer from amplifier (typically 4-8$\Omega$) \\
        \hline
    \end{tabulary}

    \textbf{Permanent Magnet Loudspeaker:}

    \begin{figure}[H]
        \centering
        \begin{tikzpicture}[gtu flow]
            \node (Input) [gtu block] {Audio Input};
            \node (Coil) [gtu block, right=of Input] {Voice Coil};
            \node (Magnet) [gtu block, above=of Coil] {Permanent Magnet};
            \node (Cone) [gtu block, right=of Coil] {Cone/Diaphragm};
            \node (Sound) [gtu block, right=of Cone] {Sound Waves};
            
            \draw [gtu arrow] (Input) -- (Coil);
            \draw [gtu arrow] (Magnet) -- (Coil);
            \draw [gtu arrow] (Coil) -- (Cone);
            \draw [gtu arrow] (Cone) -- (Sound);
        \end{tikzpicture}
        \caption{Permanent Magnet Loudspeaker}
    \end{figure}

    \begin{itemize}
        \item \textbf{Permanent Magnet}: Creates fixed magnetic field (usually ferrite or neodymium).
        \item \textbf{Voice Coil}: Wire coil that carries audio current, creating variable magnetic field.
        \item \textbf{Cone/Diaphragm}: Moves in response to voice coil movement.
        \item \textbf{Working Principle}: Interaction between fixed magnetic field and varying field from voice coil creates mechanical movement.
        \item \textbf{Advantages}: More efficient, no field coil power required, compact design.
    \end{itemize}

    \begin{mnemonicbox}
        \mnemonic{FIRM: Frequency Impedance Require Magnets}
    \end{mnemonicbox}
\end{solutionbox}

\questionmarks{2(a)}{3}{Define only: Aspect ratio, Luminance and chrominance}
\begin{solutionbox}
    \begin{itemize}
        \item \textbf{Aspect Ratio}: The ratio of width to height of a television screen (commonly 16:9 for HDTV, 4:3 for older TVs).
        \item \textbf{Luminance}: The brightness component of a video signal that carries intensity information (represented as Y).
        \item \textbf{Chrominance}: The color component of a video signal that carries color information (represented as U and V or Cb and Cr).
    \end{itemize}

    \begin{mnemonicbox}
        \mnemonic{ALC: All Light Contains color}
    \end{mnemonicbox}
\end{solutionbox}

\questionmarks{2(b)}{4}{Draw PAL-D decoder only and explain separation of U and V component of chroma signal.}
\begin{solutionbox}
    \textbf{PAL-D Decoder Diagram:}

    \begin{figure}[H]
        \centering
        \begin{tikzpicture}[gtu flow]
            \node (Input) [gtu block] {Composite Video};
            \node (Filter) [gtu block, right=of Input] {Comb Filter};
            
            \node (Y) [gtu block, above right=1cm of Filter] {Y Processing};
            \node (Delay) [gtu block, below right=1cm of Filter] {Delay Line (64$\mu$s)};
            
            \node (Switch) [gtu block, right=of Delay] {Phase Switch};
            \node (Demod) [gtu block, below=of Switch] {Sync Demodulator};
            
            \node (U) [gtu block, right=of Demod] {U (B-Y)};
            \node (V) [gtu block, left=of Demod] {V (R-Y)};
            
            \draw [gtu arrow] (Input) -- (Filter);
            \draw [gtu arrow] (Filter) |- (Y);
            \draw [gtu arrow] (Filter) |- node[above right, font=\footnotesize] {Chroma} (Delay);
            \draw [gtu arrow] (Delay) -- (Switch);
            \draw [gtu arrow] (Switch) -- (Demod);
            \draw [gtu arrow] (Demod) -- (U);
            \draw [gtu arrow] (Demod) -- (V);
        \end{tikzpicture}
        \caption{PAL-D Decoder}
    \end{figure}

    \begin{itemize}
        \item \textbf{Comb Filter}: Separates luminance (Y) from chrominance signal.
        \item \textbf{Delay Line}: Delays chroma signal by one line period (64$\mu$s).
        \item \textbf{Phase Alternating Switch}: Inverts V component on alternate lines.
        \item \textbf{Synchronous Demodulator}: Uses subcarrier reference to extract U and V components.
        \item \textbf{U Component}: Represents Blue-minus-Luminance (B-Y).
        \item \textbf{V Component}: Represents Red-minus-Luminance (R-Y).
    \end{itemize}

    \begin{mnemonicbox}
        \mnemonic{CODES: Chrominance Only Decodes Extracting Signals}
    \end{mnemonicbox}
\end{solutionbox}

\questionmarks{2(c)}{7}{Explain in detail working of LCD television. Give any two technical specifications of it.}
\begin{solutionbox}
    \textbf{LCD Television Working:}

    \begin{figure}[H]
        \centering
        \begin{tikzpicture}[gtu flow]
            \node (Back) [gtu block] {Backlight};
            \node (Pol1) [gtu block, right=of Back] {Polarizer 1};
            \node (LC) [gtu block, right=of Pol1] {Liquid Crystal};
            \node (Color) [gtu block, right=of LC] {Color Filter};
            \node (Pol2) [gtu block, right=of Color] {Polarizer 2};
            \node (Screen) [gtu block, right=of Pol2] {Screen};
            
            \node (Signal) [gtu block, below=of LC] {Video Signal};
            \node (TFT) [gtu block, above=of LC] {TFT Matrix};
            
            \draw [gtu arrow] (Back) -- (Pol1);
            \draw [gtu arrow] (Pol1) -- (LC);
            \draw [gtu arrow] (LC) -- (Color);
            \draw [gtu arrow] (Color) -- (Pol2);
            \draw [gtu arrow] (Pol2) -- (Screen);
            
            \draw [gtu arrow] (Signal) -- (LC);
            \draw [gtu arrow] (TFT) -- (LC);
        \end{tikzpicture}
        \caption{LCD Panel Structure}
    \end{figure}

    \textbf{Working Process:}
    \begin{enumerate}
        \item \textbf{Backlight}: CCFL or LED provides white light source.
        \item \textbf{TFT Matrix}: Thin-film transistors control voltage to each pixel.
        \item \textbf{Liquid Crystal Layer}: Molecules twist based on applied voltage.
        \item \textbf{Polarizers}: First filter aligns light, second passes only rotated light.
        \item \textbf{Color Filters}: RGB filters create colored pixels.
        \item \textbf{Image Formation}: Varying voltage controls light passage through each pixel.
    \end{enumerate}

    \textbf{Technical Specifications:}
    \begin{itemize}
        \item \textbf{Resolution}: 1920$\times$1080 (Full HD) or 3840$\times$2160 (4K UHD)
        \item \textbf{Refresh Rate}: 60Hz, 120Hz, or 240Hz
    \end{itemize}

    \begin{mnemonicbox}
        \mnemonic{BALTIC: Backlight Activates Liquid To Illuminate Colors}
    \end{mnemonicbox}
\end{solutionbox}

\questionmarks{2(a) OR}{3}{State Grassmens law \& explain it with concept of additive mixing.}
\begin{solutionbox}
    \textbf{Grassmann's Law:}
    Any color can be matched by a linear combination of three primary colors.

    \textbf{Additive Color Mixing:}

    \begin{figure}[H]
        \centering
        \begin{tikzpicture}[gtu flow]
            \node (R) [gtu block, fill=red!20] {Red};
            \node (G) [gtu block, fill=green!20, right=2cm of R] {Green};
            \node (B) [gtu block, fill=blue!20, right=2cm of G] {Blue};
            
            \node (Y) [gtu block, fill=yellow!20, below right=1cm of R] {Yellow (R+G)};
            \node (C) [gtu block, fill=cyan!20, below right=1cm of G] {Cyan (G+B)};
            \node (M) [gtu block, fill=magenta!20, below left=1cm of B] {Magenta (B+R)};
            
            \node (W) [gtu block, fill=gray!10, below=3cm of G] {White (R+G+B)};
            
            \draw [gtu arrow] (R) -- (Y);
            \draw [gtu arrow] (G) -- (Y);
            
            \draw [gtu arrow] (G) -- (C);
            \draw [gtu arrow] (B) -- (C);
            
            \draw [gtu arrow] (R) -- (M);
            \draw [gtu arrow] (B) -- (M);
            
            \draw [gtu arrow] (Y) -- (W);
            \draw [gtu arrow] (C) -- (W);
            \draw [gtu arrow] (M) -- (W);
        \end{tikzpicture}
        \caption{Additive Color Mixing}
    \end{figure}

    \begin{itemize}
        \item \textbf{Principle}: Adding light of different colors creates new colors.
        \item \textbf{Primary Colors}: Red, Green, and Blue.
        \item \textbf{Secondary Colors}: Yellow (R+G), Cyan (G+B), Magenta (B+R).
        \item \textbf{Example}: Equal intensities of RGB create white light.
    \end{itemize}

    \begin{mnemonicbox}
        \mnemonic{RGB-ACM: Red Green Blue - Additive Creates More}
    \end{mnemonicbox}
\end{solutionbox}

\questionmarks{2(b) OR}{4}{Draw block diagram of DTH receiver and explain it.}
\begin{solutionbox}
    \textbf{DTH Receiver Block Diagram:}

    \begin{figure}[H]
        \centering
        \begin{tikzpicture}[gtu flow]
            \node (Dish) [gtu block] {Dish/LNB};
            \node (Tuner) [gtu block, right=of Dish] {Tuner};
            \node (Demod) [gtu block, right=of Tuner] {Demodulator};
            \node (Decoder) [gtu block, below=of Demod] {MPEG Decoder};
            
            \node (Proc) [gtu block, left=of Decoder] {IV/Audio Proc};
            \node (TV) [gtu block, left=of Proc] {TV Output};
            
            \node (CAM) [gtu block, right=of Decoder] {CAM};
            \node (Card) [gtu block, below=of CAM] {Smart Card};
            
            \draw [gtu arrow] (Dish) -- (Tuner);
            \draw [gtu arrow] (Tuner) -- (Demod);
            \draw [gtu arrow] (Demod) -- (Decoder);
            \draw [gtu arrow] (Decoder) -- (Proc);
            \draw [gtu arrow] (Proc) -- (TV);
            
            \draw [gtu arrow] (CAM) -- (Decoder);
            \draw [gtu arrow] (Card) -- (CAM);
        \end{tikzpicture}
        \caption{DTH Receiver}
    \end{figure}

    \begin{itemize}
        \item \textbf{Satellite Dish}: Collects weak satellite signals (10.7-12.75 GHz).
        \item \textbf{LNB} (Low Noise Block): Amplifies and converts signal to lower frequency (950-2150 MHz).
        \item \textbf{Tuner}: Selects desired transponder frequency.
        \item \textbf{Demodulator}: Extracts digital data from carrier signal.
        \item \textbf{MPEG Decoder}: Decompresses audio/video data.
        \item \textbf{CAM \& Smart Card}: Provide decryption and subscription verification.
        \item \textbf{Output}: Processes signals for display on television.
    \end{itemize}

    \begin{mnemonicbox}
        \mnemonic{SLTD-MCS: Satellites Link Through Decoders Making Clear Signals}
    \end{mnemonicbox}
\end{solutionbox}

\questionmarks{2(c) OR}{7}{State following frequency/standard (used in color TV system)}
\begin{solutionbox}
    \textbf{Table: Color TV Standards (PAL-B/G)} \\
    \begin{tabulary}{\linewidth}{|L|L|}
        \hline
        \textbf{Parameter} & \textbf{Frequency/Standard} \\
        \hline
        \textbf{VIF (Video Intermediate Frequency)} & 38.9 MHz \\
        \hline
        \textbf{SIF (Sound Intermediate Frequency)} & 33.4 MHz \\
        \hline
        \textbf{Color Sub-carrier Frequency} & 4.43361875 MHz \\
        \hline
        \textbf{Vertical Blanking Frequency} & 50 Hz \\
        \hline
        \textbf{Horizontal Synchronizing Frequency} & 15.625 kHz \\
        \hline
        \textbf{Inter Carrier Sound Signal Frequency} & 5.5 MHz \\
        \hline
        \textbf{One Channel Bandwidth} & 7 MHz (VHF), 8 MHz (UHF) \\
        \hline
    \end{tabulary}

    \begin{mnemonicbox}
        \mnemonic{Very Special Colors Vertically Harmonize In One Channel}
    \end{mnemonicbox}
\end{solutionbox}

\questionmarks{3(a)}{3}{What is fuzzy logic? Explain its usage in washing machine.}
\begin{solutionbox}
    \textbf{Fuzzy Logic}: A mathematical approach that deals with approximate reasoning rather than fixed, binary logic, allowing for degrees of truth values between 0 and 1.

    \textbf{Usage in Washing Machine:}

    \begin{figure}[H]
        \centering
        \begin{tikzpicture}[gtu flow]
            \node (Sensors) [gtu block] {Sensors (Load/Dirt)};
            \node (Controller) [gtu block, right=of Sensors] {Fuzzy Controller};
            \node (Decision) [gtu block, right=of Controller] {Decision Making};
            \node (Action) [gtu block, right=of Decision] {Control Actions};
            
            \node (Inputs) [gtu block, above=of Sensors] {Input Variables};
            \node (Outputs) [gtu block, below=of Action] {Output: Wash Cycle};
            
            \draw [gtu arrow] (Sensors) -- (Controller);
            \draw [gtu arrow] (Controller) -- (Decision);
            \draw [gtu arrow] (Decision) -- (Action);
            
            \draw [gtu arrow] (Inputs) -- (Sensors);
            \draw [gtu arrow] (Action) -- (Outputs);
        \end{tikzpicture}
        \caption{Fuzzy Logic in Washing Machine}
    \end{figure}

    \begin{itemize}
        \item \textbf{Input Variables}: Load weight, fabric type, water hardness, dirt level.
        \item \textbf{Processing}: Controller evaluates multiple conditions simultaneously.
        \item \textbf{Output}: Adjusts water level, wash time, rinse cycles, spin speed.
    \end{itemize}

    \begin{mnemonicbox}
        \mnemonic{FIND: Fuzzy Intelligence Navigates Decisions}
    \end{mnemonicbox}
\end{solutionbox}

\questionmarks{3(b)}{4}{Define air conditioning. Explain working of fridge. State its technical specification.}
\begin{solutionbox}
    \textbf{Air Conditioning}: The process of removing heat and moisture from indoor air to improve comfort.

    \textbf{Refrigerator Working Cycle:}

    \begin{figure}[H]
        \centering
        \begin{tikzpicture}[gtu flow]
            \node (Comp) [gtu block] {Compressor};
            \node (Cond) [gtu block, right=of Comp] {Condenser};
            \node (Valve) [gtu block, below=of Cond] {Expansion Valve};
            \node (Evap) [gtu block, left=of Valve] {Evaporator};
            
            \draw [gtu arrow] (Comp) -- node[above, font=\footnotesize] {High Pressure Vapor} (Cond);
            \draw [gtu arrow] (Cond) -- node[right, font=\footnotesize] {High Pressure Liquid} (Valve);
            \draw [gtu arrow] (Valve) -- node[below, font=\footnotesize] {Low Pressure Liquid} (Evap);
            \draw [gtu arrow] (Evap) -- node[left, font=\footnotesize] {Low Pressure Vapor} (Comp);
        \end{tikzpicture}
        \caption{Refrigeration Cycle}
    \end{figure}

    \textbf{Working Steps:}
    \begin{enumerate}
        \item \textbf{Compressor}: Compresses refrigerant gas, raising temperature.
        \item \textbf{Condenser}: Hot gas releases heat to outside, becomes liquid.
        \item \textbf{Expansion Valve}: Liquid expands, cools rapidly.
        \item \textbf{Evaporator}: Cold refrigerant absorbs heat from inside cabinet.
    \end{enumerate}

    \textbf{Technical Specifications:}
    \begin{itemize}
        \item \textbf{Capacity}: 150-500 liters
        \item \textbf{Energy Rating}: 3-5 Star
        \item \textbf{Power Consumption}: 100-300 kWh/year
    \end{itemize}

    \begin{mnemonicbox}
        \mnemonic{CEVA: Compress, Expel heat, Valve expands, Absorb heat}
    \end{mnemonicbox}
\end{solutionbox}

\questionmarks{3(c)}{7}{Explain working principle of Microwave oven using functional block diagram. State its technical specifications.}
\begin{solutionbox}
    \textbf{Microwave Oven Block Diagram:}

    \begin{figure}[H]
        \centering
        \begin{tikzpicture}[gtu flow]
            \node (Power) [gtu block] {Power Supply};
            \node (Control) [gtu block, right=of Power] {Control Panel};
            \node (Timer) [gtu block, right=of Control] {Timer/Controller};
            
            \node (Mag) [gtu block, below=of Timer] {Magnetron};
            \node (Guide) [gtu block, left=of Mag] {Waveguide};
            \node (Cavity) [gtu block, left=of Guide, minimum size=2cm] {Cooking Cavity};
            
            \node (Motor) [gtu block, below=of Cavity] {Turntable Motor};
            \node (Safety) [gtu block, above=of Timer] {Safety Interlocks};
            
            \draw [gtu arrow] (Power) -- (Control);
            \draw [gtu arrow] (Control) -- (Timer);
            \draw [gtu arrow] (Timer) -- (Mag);
            \draw [gtu arrow] (Mag) -- (Guide);
            \draw [gtu arrow] (Guide) -- (Cavity);
            
            \draw [gtu arrow] (Timer) |- (Motor);
            \draw [gtu arrow] (Safety) -- (Timer);
            \draw [gtu arrow] (Motor) -- (Cavity);
        \end{tikzpicture}
        \caption{Microwave Oven System}
    \end{figure}

    \textbf{Working Principle:}
    \begin{enumerate}
        \item \textbf{Magnetron}: Generates microwaves at 2.45 GHz frequency.
        \item \textbf{Waveguide}: Directs microwaves into cooking cavity.
        \item \textbf{Water Molecules}: Microwaves cause water molecules to vibrate.
        \item \textbf{Heat Generation}: Molecular vibration creates friction and heat.
        \item \textbf{Turntable}: Rotates food for even cooking.
        \item \textbf{Safety Interlocks}: Prevent operation when door is open.
    \end{enumerate}

    \textbf{Technical Specifications:}
    \begin{itemize}
        \item \textbf{Power Output}: 700-1200 watts
        \item \textbf{Frequency}: 2.45 GHz
        \item \textbf{Capacity}: 20-40 liters
        \item \textbf{Cooking Modes}: Microwave, Grill, Convection, Combination
    \end{itemize}

    \begin{mnemonicbox}
        \mnemonic{MICRO: Magnetron Initiates Cooking by Rotating Oscillations}
    \end{mnemonicbox}
\end{solutionbox}

\questionmarks{3(a) OR}{3}{Give technical specification of solar panel. State advantages and disadvantages of solar roof top system}
\begin{solutionbox}
    \textbf{Solar Panel Technical Specifications:}
    \begin{itemize}
        \item \textbf{Power Rating}: 250-400 Wp (Watt peak)
        \item \textbf{Efficiency}: 15-22\%
        \item \textbf{Cell Type}: Monocrystalline, Polycrystalline, or Thin Film
    \end{itemize}

    \textbf{Advantages and Disadvantages:}
    \begin{tabulary}{\linewidth}{|L|L|}
        \hline
        \textbf{Advantages} & \textbf{Disadvantages} \\
        \hline
        Renewable Energy Source & High Initial Cost \\
        \hline
        Reduces Electricity Bills & Weather Dependent \\
        \hline
        Low Maintenance Cost & Requires Large Space \\
        \hline
        No Noise Pollution & Limited Nighttime Generation \\
        \hline
    \end{tabulary}

    \begin{mnemonicbox}
        \mnemonic{SERLN: Solar Energy Reduces Long-term Numbers}
    \end{mnemonicbox}
\end{solutionbox}

\questionmarks{3(b) OR}{4}{State various types of washing machine. Compare frontload and top load washing machine.}
\begin{solutionbox}
    \textbf{Types of Washing Machines:}
    \begin{itemize}
        \item Top Load (Agitator \& Impeller)
        \item Front Load
        \item Semi-Automatic
        \item Fully Automatic
    \end{itemize}

    \textbf{Comparison:}
    \begin{tabulary}{\linewidth}{|l|L|L|}
        \hline
        \textbf{Parameter} & \textbf{Front Load} & \textbf{Top Load} \\
        \hline
        \textbf{Water Consumption} & Lower (40-60 liters) & Higher (80-120 liters) \\
        \hline
        \textbf{Energy Efficiency} & Higher & Lower \\
        \hline
        \textbf{Cleaning Performance} & Better & Good \\
        \hline
        \textbf{Space Requirement} & Can be stacked & Needs top clearance \\
        \hline
        \textbf{Cost} & Higher & Lower \\
        \hline
        \textbf{Cycle Duration} & Longer (60-120 min) & Shorter (30-60 min) \\
        \hline
    \end{tabulary}

    \begin{mnemonicbox}
        \mnemonic{FTEST: Front-loaders Take Extra Space but Triumph in efficiency}
    \end{mnemonicbox}
\end{solutionbox}

\questionmarks{3(c) OR}{7}{Give classification of solar rooftop system. Explain working of solar rooftop system (Grid connected online) with suitable diagram. State steps to maintain solar roof top system.}
\begin{solutionbox}
    \textbf{Classification:} Grid-Connected (On-grid), Off-Grid (Standalone), Hybrid.

    \textbf{Grid-Connected Solar System Diagram:}

    \begin{figure}[H]
        \centering
        \begin{tikzpicture}[gtu flow]
            \node (Panels) [gtu block] {Solar Panels};
            \node (DCBox) [gtu block, right=of Panels] {DC Box};
            \node (Inverter) [gtu block, right=of DCBox] {Solar Inverter};
            
            \node (ACBox) [gtu block, below=of Inverter] {AC Box};
            \node (Load) [gtu block, left=of ACBox] {Home Loads};
            \node (Meter) [gtu block, right=of ACBox] {Bi-dir Meter};
            \node (Grid) [gtu block, right=of Meter] {Grid};
            
            \draw [gtu arrow] (Panels) -- node[above] {DC} (DCBox);
            \draw [gtu arrow] (DCBox) -- (Inverter);
            \draw [gtu arrow] (Inverter) -- node[right] {AC} (ACBox);
            \draw [gtu arrow] (ACBox) -- (Load);
            \draw [gtu arrow] (ACBox) -- (Meter);
            \draw [gtu arrow] (Meter) <-> (Grid);
        \end{tikzpicture}
        \caption{On-Grid Solar System}
    \end{figure}

    \textbf{Working:}
    \begin{enumerate}
        \item \textbf{Solar Panels}: Convert sunlight to DC electricity.
        \item \textbf{Junction Box}: Combines outputs, provides protection.
        \item \textbf{Inverter}: Converts DC to grid-compatible AC.
        \item \textbf{Bi-directional Meter}: Measures import/export of electricity.
        \item \textbf{Excess Generation}: Feeds back to grid (Net metering).
    \end{enumerate}

    \textbf{Maintenance Steps:}
    \begin{itemize}
        \item Regular cleaning of panels (dust, bird droppings).
        \item Checking electrical connections for corrosion.
        \item Monitoring system performance via inverter data.
        \item Trimming nearby trees to prevent shading.
        \item Annual inspection by qualified technician.
    \end{itemize}

    \begin{mnemonicbox}
        \mnemonic{SPICED: Solar Panels Invert Current for Electrical Distribution}
    \end{mnemonicbox}
\end{solutionbox}

\questionmarks{4(a)}{3}{Explain in brief working principle of photo copier machine with concept of latent image.}
\begin{solutionbox}
    \textbf{Photocopier Process:}

    \begin{figure}[H]
        \centering
        \begin{tikzpicture}[gtu flow]
            \node (Charge) [gtu block] {1. Charging};
            \node (Expose) [gtu block, right=of Charge] {2. Exposure};
            \node (Develop) [gtu block, right=of Expose] {3. Developing};
            \node (Transfer) [gtu block, below=of Develop] {4. Transfer};
            \node (Fuse) [gtu block, left=of Transfer] {5. Fusing};
            \node (Clean) [gtu block, left=of Fuse] {6. Cleaning};
            
            \draw [gtu arrow] (Charge) -- (Expose);
            \draw [gtu arrow] (Expose) -- (Develop);
            \draw [gtu arrow] (Develop) -- (Transfer);
            \draw [gtu arrow] (Transfer) -- (Fuse);
            \draw [gtu arrow] (Fuse) -- (Clean);
            \draw [gtu arrow] (Clean) -- (Charge);
        \end{tikzpicture}
        \caption{Xerography Cycle}
    \end{figure}

    \textbf{Latent Image Concept:}
    \begin{itemize}
        \item \textbf{Charging}: Photosensitive drum receives uniform positive charge.
        \item \textbf{Exposure}: Light reflects from original document onto drum.
        \item \textbf{Latent Image}: Light areas discharge drum creating invisible electrostatic image.
        \item \textbf{Development}: Negatively charged toner particles attracted to positive areas.
        \item \textbf{Transfer}: Toner transferred to paper through electrical attraction.
        \item \textbf{Fusing}: Heat and pressure permanently bond toner to paper.
    \end{itemize}

    \begin{mnemonicbox}
        \mnemonic{CEDTFC: Charging Exposure Develops The Final Copy}
    \end{mnemonicbox}
\end{solutionbox}

\questionmarks{4(b)}{4}{Explain working of Laser printer with suitable diagram}
\begin{solutionbox}
    \textbf{Laser Printer Diagram:}

    \begin{figure}[H]
        \centering
        \begin{tikzpicture}[gtu flow]
            \node (Data) [gtu block] {Data Processing};
            \node (Laser) [gtu block, right=of Data] {Laser Unit};
            \node (Drum) [gtu block, right=of Laser] {Photo Drum};
            \node (Corona) [gtu block, above=of Drum] {Primary Corona};
            
            \node (Dev) [gtu block, below=of Drum] {Developer};
            \node (Trans) [gtu block, left=of Dev] {Transfer};
            \node (Fuse) [gtu block, left=of Trans] {Fuser};
            \node (Out) [gtu block, left=of Fuse] {Output};
            
            \draw [gtu arrow] (Data) -- (Laser);
            \draw [gtu arrow] (Laser) -- (Drum);
            \draw [gtu arrow] (Corona) -- (Drum);
            \draw [gtu arrow] (Drum) -- (Dev);
            \draw [gtu arrow] (Dev) -- (Trans);
            \draw [gtu arrow] (Trans) -- (Fuse);
            \draw [gtu arrow] (Fuse) -- (Out);
        \end{tikzpicture}
        \caption{Laser Printer Mechanism}
    \end{figure}

    \textbf{Working Process:}
    \begin{itemize}
        \item \textbf{Raster Image Processing}: Computer data converted to bitmap.
        \item \textbf{Charging}: Corona wire gives drum uniform negative charge.
        \item \textbf{Writing}: Laser beam neutralizes charge in pattern of image.
        \item \textbf{Developing}: Toner attracted to neutralized areas.
        \item \textbf{Transfer}: Paper given positive charge to attract toner.
        \item \textbf{Fusing}: Heat rollers melt toner permanently onto paper.
    \end{itemize}

    \begin{mnemonicbox}
        \mnemonic{RASTER: Raster-image Attracts Static Toner, Electricity Releases}
    \end{mnemonicbox}
\end{solutionbox}

\questionmarks{4(c)}{7}{Draw and explain block diagram of CCTV system using Digital IP camera connected with internet...}
\begin{solutionbox}
    \textbf{IP CCTV System Diagram:}

    \begin{figure}[H]
        \centering
        \begin{tikzpicture}[gtu flow]
            \node (Cam1) [gtu block] {IP Camera 1};
            \node (Cam2) [gtu block, below=0.5cm of Cam1] {IP Camera 2};
            \node (Switch) [gtu block, right=of Cam1] {Network Switch (PoE)};
            
            \node (NVR) [gtu block, right=of Switch] {NVR};
            \node (Storage) [gtu block, below=of NVR] {Storage HDD};
            \node (Monitor) [gtu block, above=of NVR] {Local Monitor};
            
            \node (Router) [gtu block, right=of NVR] {Router};
            \node (Web) [gtu block, right=of Router] {Internet};
            \node (Remote) [gtu block, below=of Web] {Remote View};
            
            \draw [gtu arrow] (Cam1) -- (Switch);
            \draw [gtu arrow] (Cam2) -- (Switch);
            \draw [gtu arrow] (Switch) -- (NVR);
            \draw [gtu arrow] (NVR) -- (Storage);
            \draw [gtu arrow] (NVR) -- (Monitor);
            \draw [gtu arrow] (NVR) -- (Router);
            \draw [gtu arrow] (Router) -- (Web);
            \draw [gtu arrow] (Web) -- (Remote);
        \end{tikzpicture}
        \caption{IP CCTV Architecture}
    \end{figure}

    \textbf{Working:}
    \begin{itemize}
        \item \textbf{IP Cameras}: Capture and digitize video.
        \item \textbf{Network Infrastructure}: Transmits data via TCP/IP protocols.
        \item \textbf{NVR}: Records, manages, and processes video streams.
        \item \textbf{Router}: Provides secure internet access for remote viewing.
    \end{itemize}
    
    \textbf{Camera Types:} Dome, Bullet, PTZ, Fisheye, Thermal.
    
    \textbf{PoE Cable}: Power Over Ethernet carries both power and data on a single cable.

    \begin{mnemonicbox}
        \mnemonic{INSPIRE: Internet Networking Secures Places In Remote Environments}
    \end{mnemonicbox}
\end{solutionbox}

\questionmarks{4(a) OR}{3}{Discuss pros and cons of internet based Digital IP camera CCTV system}
\begin{solutionbox}
    \begin{tabulary}{\linewidth}{|L|L|}
        \hline
        \textbf{Pros} & \textbf{Cons} \\
        \hline
        Higher Resolution (1080p to 4K) & Higher Initial Cost \\
        \hline
        Remote Viewing via internet & Bandwidth Requirements \\
        \hline
        Scalability \& easy expansion & Cybersecurity Risks \\
        \hline
        Power Over Ethernet (POE) & Network Dependency \\
        \hline
        Advanced Analytics capabilities & Complex Configuration \\
        \hline
    \end{tabulary}

    \begin{mnemonicbox}
        \mnemonic{HIGHER: High-resolution Images Give Higher Evaluation Remotely}
    \end{mnemonicbox}
\end{solutionbox}

\questionmarks{4(b) OR}{4}{Explain working of inkjet printer with suitable diagram}
\begin{solutionbox}
    \textbf{Inkjet Printer Diagram:}

    \begin{figure}[H]
        \centering
        \begin{tikzpicture}[gtu flow]
            \node (Data) [gtu block] {Print Data};
            \node (Control) [gtu block, right=of Data] {Controller};
            \node (Head) [gtu block, right=of Control] {Print Head};
            \node (Cartridge) [gtu block, above=of Head] {Ink Tank};
            
            \node (Nozzle) [gtu block, right=of Head] {Nozzles};
            \node (Paper) [gtu block, below=of Nozzle] {Paper};
            \node (Feed) [gtu block, left=of Paper] {Feed Mech};
            
            \draw [gtu arrow] (Data) -- (Control);
            \draw [gtu arrow] (Control) -- (Head);
            \draw [gtu arrow] (Cartridge) -- (Head);
            \draw [gtu arrow] (Head) -- (Nozzle);
            \draw [gtu arrow] (Nozzle) -- (Paper);
            \draw [gtu arrow] (Control) -- (Feed);
            \draw [gtu arrow] (Feed) -- (Paper);
        \end{tikzpicture}
        \caption{Inkjet Working}
    \end{figure}

    \textbf{Working Process:}
    \begin{itemize}
        \item \textbf{Data Processing}: Controller converts digital data to nozzle instructions.
        \item \textbf{Ink Ejection}:
            \begin{itemize}
                \item Thermal: Resistors heat ink to create bubbles.
                \item Piezoelectric: Crystals flex to push ink.
            \end{itemize}
        \item \textbf{Drying}: Ink adheres to paper surface.
    \end{itemize}

    \begin{mnemonicbox}
        \mnemonic{PRINT: Paper Receives Ink through Numerous Tiny-nozzles}
    \end{mnemonicbox}
\end{solutionbox}

\questionmarks{4(c) OR}{7}{Draw and explain block diagram of CCTV system using simple analog camera and DVR...}
\begin{solutionbox}
    \textbf{Analog CCTV Diagram:}

    \begin{figure}[H]
        \centering
        \begin{tikzpicture}[gtu flow]
            \node (Cam1) [gtu block] {Analog Cam 1};
            \node (Cam2) [gtu block, below=0.5cm of Cam1] {Analog Cam 2};
            \node (DVR) [gtu block, right=of Cam1] {DVR};
            \node (HDD) [gtu block, below=of DVR] {Storage};
            
            \node (Monitor) [gtu block, above=of DVR] {Monitor};
            \node (Router) [gtu block, right=of DVR] {Router};
            \node (Remote) [gtu block, right=of Router] {Remote View};
            
            \node (Power) [gtu block, left=of Cam2] {Power Supply};
            
            \draw [gtu arrow] (Cam1) -- node[above, font=\tiny] {Coax} (DVR);
            \draw [gtu arrow] (Cam2) -- node[below, font=\tiny] {Coax} (DVR);
            \draw [gtu arrow] (DVR) -- (HDD);
            \draw [gtu arrow] (DVR) -- (Monitor);
            \draw [gtu arrow] (DVR) -- (Router);
            \draw [gtu arrow] (Router) -- (Remote);
            \draw [gtu arrow] (Power) -- (Cam1);
            \draw [gtu arrow] (Power) -- (Cam2);
        \end{tikzpicture}
        \caption{Analog CCTV System}
    \end{figure}

    \textbf{Cable Types:} Coaxial (RG59), Twisted Pair (CAT5/6), Power Cable, Fiber Optic, Siamese Cable.
    
    \textbf{Camera Categories:} Fixed, Varifocal, Night Vision, HDR.

    \begin{mnemonicbox}
        \mnemonic{CARD: Coaxial Analog Recording Devices}
    \end{mnemonicbox}
\end{solutionbox}

\questionmarks{5(a)}{3}{Define only: Maintenance, Preventive maintenance and Predictive maintenance.}
\begin{solutionbox}
    \begin{itemize}
        \item \textbf{Maintenance}: Process of preserving equipment in proper operating condition.
        \item \textbf{Preventive Maintenance}: Scheduled activities to prevent failures before they occur.
        \item \textbf{Predictive Maintenance}: Condition-based maintenance using data to predict failure timing.
    \end{itemize}

    \begin{mnemonicbox}
        \mnemonic{MPP: Maintain Proactively, Predict problems}
    \end{mnemonicbox}
\end{solutionbox}

\questionmarks{5(b)}{4}{Discuss maintenance of public address system.}
\begin{solutionbox}
    \begin{tabulary}{\linewidth}{|l|L|}
        \hline
        \textbf{Component} & \textbf{Maintenance Tasks} \\
        \hline
        \textbf{Microphones} & Clean windscreens, Check cables, Test sensitivity \\
        \hline
        \textbf{Amplifiers} & Clean vents, Check power, Inspect overheating \\
        \hline
        \textbf{Speakers} & Inspect brackets, Test for distortion, Check wiring \\
        \hline
        \textbf{Cables} & Test continuity, Replace damaged cables \\
        \hline
    \end{tabulary}

    \begin{mnemonicbox}
        \mnemonic{MACS: Microphones, Amplifiers, Connections, Speakers}
    \end{mnemonicbox}
\end{solutionbox}

\questionmarks{5(c)}{7}{State any three faults of washing machine. Discuss in general maintenance of washing machine.}
\begin{solutionbox}
    \textbf{Common Faults:}
    \begin{enumerate}
        \item \textbf{Water Not Filling}: Faulty valve, clogged filter.
        \item \textbf{Not Spinning}: Belt issues, motor problems.
        \item \textbf{Excessive Vibration}: Uneven feet, suspension issues.
    \end{enumerate}

    \textbf{Maintenance Procedures:}
    \begin{tabulary}{\linewidth}{|l|L|}
        \hline
        \textbf{Component} & \textbf{Tasks} \\
        \hline
        \textbf{Drum} & Clean monthly, remove residue, check foreign objects \\
        \hline
        \textbf{Filters} & Clean lint filter after use, pump filter monthly \\
        \hline
        \textbf{Hoses} & Inspect cracks, replace every 3-5 years \\
        \hline
        \textbf{Door Seal} & Wipe to prevent mold, check for tears \\
        \hline
    \end{tabulary}

    \begin{mnemonicbox}
        \mnemonic{WATCH: Water And Tub Cleaning Helps}
    \end{mnemonicbox}
\end{solutionbox}

\questionmarks{5(a) OR}{3}{Compare predictive and preventive maintenance.}
\begin{solutionbox}
    \begin{tabulary}{\linewidth}{|l|L|L|}
        \hline
        \textbf{Param} & \textbf{Predictive} & \textbf{Preventive} \\
        \hline
        \textbf{Timing} & As needed (condition-based) & Fixed schedule \\
        \hline
        \textbf{Tech} & Vibration/Thermal analysis & Visual inspection/Cleaning \\
        \hline
        \textbf{Cost} & High initial, low long-term & Low initial, maybe high long-term \\
        \hline
        \textbf{Downtime} & Minimized/Planned & Systematic scheduled \\
        \hline
    \end{tabulary}

    \begin{mnemonicbox}
        \mnemonic{TIMED: Testing Identifies Maintenance Exactly when Due}
    \end{mnemonicbox}
\end{solutionbox}

\questionmarks{5(b) OR}{4}{Discuss maintenance and troubleshooting of LCD TV.}
\begin{solutionbox}
    \textbf{Maintenance:}
    \begin{itemize}
        \item \textbf{Screen}: Clean with microfiber, no liquids.
        \item \textbf{Ventilation}: Remove dust, ensure airflow.
        \item \textbf{Connections}: Verify cables, check corrosion.
    \end{itemize}

    \textbf{Troubleshooting:}
    \begin{itemize}
        \item \textbf{No Power}: Check cord, fuse.
        \item \textbf{No Picture}: Verify backlight, T-Con board.
        \item \textbf{Lines on Screen}: Ribbon cables, screen damage.
    \end{itemize}

    \begin{mnemonicbox}
        \mnemonic{PVCS: Pixels, Ventilation, Connections, Software}
    \end{mnemonicbox}
\end{solutionbox}

\questionmarks{5(c) OR}{7}{Explain installation of laser printers in your computer system. Discuss its maintenance and troubleshooting procedure.}
\begin{solutionbox}
    \textbf{Installation Diagram:}

    \begin{figure}[H]
        \centering
        \begin{tikzpicture}[gtu flow]
            \node (Unpack) [gtu block] {Unpacking};
            \node (HW) [gtu block, right=of Unpack] {Hardware Setup};
            \node (Toner) [gtu block, right=of HW] {Install Toner};
            \node (Power) [gtu block, below=of Toner] {Connect Power};
            \node (Cable) [gtu block, left=of Power] {Connect Data};
            \node (Driver) [gtu block, left=of Cable] {Install Driver};
            \node (Test) [gtu block, below=of Driver] {Test Print};
            
            \draw [gtu arrow] (Unpack) -- (HW);
            \draw [gtu arrow] (HW) -- (Toner);
            \draw [gtu arrow] (Toner) -- (Power);
            \draw [gtu arrow] (Power) -- (Cable);
            \draw [gtu arrow] (Cable) -- (Driver);
            \draw [gtu arrow] (Driver) -- (Test);
        \end{tikzpicture}
        \caption{Printer Installation}
    \end{figure}

    \textbf{Maintenance:}
    \begin{itemize}
        \item \textbf{Paper Path}: Clean with compressed air.
        \item \textbf{Rollers}: Clean with isopropyl alcohol.
        \item \textbf{Toner Area}: Vacuum carefully.
    \end{itemize}

    \textbf{Troubleshooting:} Paper jams (Clear path), Streaking (Clean corona), Light print (Replace toner), Connection issues (Reinstall driver).

    \begin{mnemonicbox}
        \mnemonic{SECURE: Setup, Execute drivers, Clean Regularly, Update, Replace consumables, Examine problems}
    \end{mnemonicbox}
\end{solutionbox}

\end{document}
