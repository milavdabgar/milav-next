\documentclass{article}

% content/resources/templates/preamble.tex
\usepackage[margin=0.6in]{geometry}
\author{Milav Dabgar}
\usepackage{amsmath,amssymb,amsthm}
\usepackage{booktabs}
\usepackage{multirow}
\usepackage{xcolor}
\usepackage{tcolorbox}
\tcbuselibrary{breakable,skins}
\usepackage[colorlinks=true,linkcolor=blue]{hyperref}
\usepackage{titlesec}
\usepackage{enumitem}
\usepackage{tikz}
\usepackage{pgfplots}
\usepackage{circuitikz}
\usepackage[version=4]{mhchem}
\usepackage{longtable}
\usepackage{array}
\usepackage{float}
\usepackage{caption}
\usepackage{listings}

\lstset{
  basicstyle=\small\ttfamily,
  breaklines=true,
  breakatwhitespace=false,
  postbreak=\mbox{\textcolor{red}{$\hookrightarrow$}\space},
  float=false,
  numbers=left,
  numberstyle=\tiny\color{gray},
  numbersep=10pt,
  xleftmargin=2em,
  keywordstyle=\color{blue},
  commentstyle=\color{green!60!black},
  stringstyle=\color{purple},
  backgroundcolor=\color{gray!5},
  showstringspaces=false,
  tabsize=2,
  captionpos=b,
  keepspaces=true,
  columns=flexible
}

\pgfplotsset{compat=1.18}
\usetikzlibrary{shapes,arrows,positioning,calc,patterns,decorations.pathmorphing,decorations.markings,arrows.meta}

% Color scheme
\definecolor{headcolor}{RGB}{0,102,204}
\definecolor{keycolor}{RGB}{220,20,60}
\definecolor{solutioncolor}{RGB}{34,139,34}
\definecolor{mnemoniccolor}{RGB}{148,0,211}
\definecolor{codecolor}{RGB}{0,0,100}

% Spacing
\setlength{\parskip}{3pt}
\setlist[itemize]{nosep}
\setlist[enumerate]{nosep}

% Title formatting
\titleformat{\section}{\Large\bfseries\color{headcolor}}{\thesection}{1em}{}
\titleformat{\subsection}{\large\bfseries\color{headcolor}}{\thesubsection}{1em}{}

% Pandoc tightlist compatibility
\providecommand{\tightlist}{%
  \setlength{\itemsep}{0pt}\setlength{\parskip}{0pt}}

% Pandoc longtable compatibility
\newcounter{none}
\def\thenone{}


% content/resources/templates/gujarati-boxes.tex
\usepackage{fontspec}
\usepackage{polyglossia}

% Set Gujarati as main language (document is primarily in Gujarati)
% Note: gloss-gujarati.ldf doesn't exist in polyglossia, but it will use hyphenation patterns
\setdefaultlanguage{gujarati}
\setotherlanguage{english}

% Configure Gujarati font properly
% Use Language=Default to prevent polyglossia from trying to add language-specific features
% that don't exist for Gujarati, which causes "empty feature" warnings
\newfontfamily\gujaratifont[Script=Gujarati,AutoFakeBold=2.5,AutoFakeSlant=0.3]{Noto Sans Gujarati}
\setmainfont[Script=Gujarati,AutoFakeBold=2.5,AutoFakeSlant=0.3]{Noto Sans Gujarati}
% Use Noto Sans Gujarati for monospace to support Gujarati in text
\setmonofont[Scale=0.9]{Noto Sans Gujarati}

% Configure English to use the same font
\newfontfamily\englishfont[Script=Gujarati,AutoFakeBold=2.5,AutoFakeSlant=0.3]{Noto Sans Gujarati}

% Translations for polyglossia
\gappto\captionsgujarati{
  \renewcommand{\tablename}{કોષ્ટક}
  \renewcommand{\figurename}{આકૃતિ}
}

% Helper for TikZ nodes to ensure Gujarati font
\newcommand{\gu}[1]{{\gujaratifont #1}}

% Custom environments
\newtcolorbox{solutionbox}{
    breakable,
    enhanced,
    colback=solutioncolor!5!white,
    colframe=solutioncolor!75!black,
    fonttitle=\bfseries,
    title=જવાબ
}

\newtcolorbox{solutionboxnobreak}{
 colback=solutioncolor!5!white,
 colframe=solutioncolor!75!black,
 fonttitle=\bfseries,
 title=જવાબ
}

\newtcolorbox{keyformula}{
 breakable,
 enhanced,
 colback=keycolor!5!white,
 colframe=keycolor!75!black,
 fonttitle=\bfseries,
 title=રાસાયણિક સમીકરણ/સૂત્ર
}

\newtcolorbox{mnemonicbox}{
 breakable,
 enhanced,
 colback=mnemoniccolor!5!white,
 colframe=mnemoniccolor!75!black,
 fonttitle=\bfseries,
 title=મેમરી ટ્રીક
}


% Custom commands for GTU solutions
% This file defines semantic commands for consistent formatting

% Question command with automatic formatting
\newcommand{\question}[2]{%
  \section*{Question #1}%
  \textbf{#2}%
}

% OR question variant
\newcommand{\questionor}[2]{%
  \section*{Question #1 OR}%
  \textbf{#2}%
}

% Proper table environment with caption
\newenvironment{answertable}[1]{%
  \begin{table}[htbp]
  \centering
  \caption{#1}
}{%
  \end{table}
}

% Proper figure environment for diagrams
\newenvironment{answerdiagram}[1]{%
  \begin{figure}[htbp]
  \centering
  \caption{#1}
}{%
  \end{figure}
}

% Semantic markup for key terms
\newcommand{\keyword}[1]{\textbf{#1}}
\newcommand{\code}[1]{\texttt{#1}}
\newcommand{\classname}[1]{\texttt{#1}}
\newcommand{\methodname}[1]{\texttt{#1}}

% Proper quotation marks
\newcommand{\mnemonic}[1]{``#1''}


\title{Consumer Electronics and Maintenance (4341107) - Summer 2023 Solution}
\date{July 24, 2023}

\begin{document}
\maketitle

\questionmarks{1(a)}{3}{CCTV ના મેઇંટેનન્સ ની પ્રક્રિયા વર્ણવો.}
\begin{solutionbox}
    \textbf{Table: CCTV મેઇંટેનન્સ પ્રક્રિયા} \\
    \begin{longtable}{|c|l|l|}
        \hline
        \textbf{સ્ટેપ} & \textbf{પ્રક્રિયા} & \textbf{વિગત} \\
        \hline
        1 & \textbf{કેમેરા ક્લીનિંગ} & મહિને એક વાર લેન્સ અને હાઉસિંગ સાફ કરો \\
        \hline
        2 & \textbf{કેબલ ઇન્સ્પેક્શન} & ત્રિમાસિક નુકસાન/એક્સપોઝર તપાસો \\
        \hline
        3 & \textbf{રેકોર્ડિંગ ચેક} & માસિક ડેટા સંગ્રહ અને પ્લેબેક ચકાસો \\
        \hline
        4 & \textbf{ફર્મવેર અપડેટ} & ઉપલબ્ધ હોય ત્યારે સૉફ્ટવેર અપડેટ કરો \\
        \hline
        5 & \textbf{એંગલ એડજસ્ટમેન્ટ} & જરૂર મુજબ કેમેરા ફરીથી ગોઠવો \\
        \hline
    \end{longtable}

    \begin{mnemonicbox}
        \mnemonic{CCRU: ક્લીન, ચેક, રેકોર્ડ, અપડેટ}
    \end{mnemonicbox}
\end{solutionbox}

\questionmarks{1(b)}{4}{મેઇંટેનન્સ ના પ્રકારો લખો અને ટૂંકમા સમજાવો.}
\begin{solutionbox}
    \textbf{Table: મેઇંટેનન્સના પ્રકારો} \\
    \begin{tabulary}{\linewidth}{|l|L|L|L|}
        \hline
        \textbf{પ્રકાર} & \textbf{વર્ણન} & \textbf{ક્યારે કરવામાં આવે છે} & \textbf{ફાયદા} \\
        \hline
        \textbf{પ્રિવેન્ટિવ} & નિયમિત તપાસ ખરાબી પહેલાં & નિર્ધારિત સમયાંતરે & અનપેક્ષિત ડાઉનટાઇમ ઘટાડે છે \\
        \hline
        \textbf{કરેક્ટિવ} & ઉપકરણ તૂટી જાય ત્યારે રિપેર & નિષ્ફળતા પછી & કાર્યક્ષમતા પુનઃસ્થાપિત કરે છે \\
        \hline
        \textbf{પ્રિડિક્ટિવ} & ડેટાનો ઉપયોગ નિષ્ફળતાની આગાહી કરવા & વિશ્લેષણ આધારિત & મેઇંટેનન્સનો સમય અનુકૂળ કરે છે \\
        \hline
        \textbf{કન્ડિશન-બેઝ્ડ} & વાસ્તવિક ઉપકરણની સ્થિતિ મોનિટર કરે છે & સ્થિતિ સૂચવે ત્યારે & બિનજરૂરી મેઇંટેનન્સ ઘટાડે છે \\
        \hline
    \end{tabulary}

    \begin{figure}[H]
        \centering
        \begin{tikzpicture}[gtu flow]
            \node (A) [gtu block] {મેઇંટેનન્સના પ્રકારો};
            \node (B) [gtu block, below left=1cm and 1cm of A] {પ્રિવેન્ટિવ};
            \node (C) [gtu block, below right=1cm and 1cm of A] {કરેક્ટિવ};
            \node (D) [gtu block, below left=3cm and 0.5cm of A] {પ્રિડિક્ટિવ};
            \node (E) [gtu block, below right=3cm and 0.5cm of A] {કન્ડિશન-બેઝ્ડ};
            
            \draw [gtu arrow] (A) -- (B);
            \draw [gtu arrow] (A) -- (C);
            \draw [gtu arrow] (A) -- (D);
            \draw [gtu arrow] (A) -- (E);

            \node (F) [below=0.2cm of B, align=center, font=\footnotesize] {નિયમિત તપાસ};
            \node (G) [below=0.2cm of C, align=center, font=\footnotesize] {બ્રેકડાઉન પછી રિપેર};
            \node (H) [below=0.2cm of D, align=center, font=\footnotesize] {ડેટા આધારિત આગાહી};
            \node (I) [below=0.2cm of E, align=center, font=\footnotesize] {ઉપકરણની સ્થિતિ આધારિત};
        \end{tikzpicture}
        \caption{\foreignlanguage{english}{Maintenance Types}}
    \end{figure}

    \begin{mnemonicbox}
        \mnemonic{PCPC: પ્રિવેન્ટ, કરેક્ટ, પ્રિડિક્ટ, કન્ડિશન}
    \end{mnemonicbox}
\end{solutionbox}

\questionmarks{1(c)}{7}{વોશીંગ મશીનના મેઇંટેનન્સ અને ટ્રબલશૂટીંગ ની પ્રક્રિયા સમજાવો.}
\begin{solutionbox}
    \textbf{Table: વોશીંગ મશીન મેઇંટેનન્સ અને ટ્રબલશૂટિંગ} \\
    \begin{tabulary}{\linewidth}{|L|L|L|}
        \hline
        \textbf{સમસ્યા} & \textbf{સંભવિત કારણ} & \textbf{ટ્રબલશૂટિંગ સ્ટેપ્સ} \\
        \hline
        \textbf{મશીન ચાલુ ન થવું} & પાવર સમસ્યા, ડોર લોક & પાવર સપ્લાય તપાસો, ડોર બરાબર બંધ છે તે ખાતરી કરો \\
        \hline
        \textbf{પાણી ન ભરાવું} & પાણીનો પુરવઠો, ઇનલેટ વાલ્વ & પાણીના નળ તપાસો, ઇનલેટ હોઝમાં બ્લોક તપાસો \\
        \hline
        \textbf{પાણી ન નીકળવું} & બ્લોક થયેલ ફિલ્ટર, ડ્રેન પંપ & ફિલ્ટર સાફ કરો, ડ્રેન હોઝ વળાંક માટે તપાસો \\
        \hline
        \textbf{વધુ વાઇબ્રેશન} & અસંતુલિત લોડ, શિપિંગ બોલ્ટ્સ & કપડાં પુનઃવિતરિત કરો, શિપિંગ બોલ્ટ્સ દૂર કર્યા છે તે તપાસો \\
        \hline
        \textbf{પાણી લીકેજ} & ક્ષતિગ્રસ્ત હોઝ, ઢીલા કનેક્શન & કનેક્શન તપાસો અને કસો, ક્ષતિગ્રસ્ત હોઝ બદલો \\
        \hline
    \end{tabulary}

    \textbf{નિયમિત મેઇંટેનન્સ:}
    \begin{itemize}
        \item \textbf{માસિક}: ડિટરજન્ટ ડ્રોઅર અને ડોર સીલ સાફ કરો
        \item \textbf{ત્રિમાસિક}: ખાલી ગરમ સાયકલ વિનેગર/ક્લીનર સાથે ચલાવો
        \item \textbf{અર્ધવાર્ષિક}: હોઝમાં તિરાડો તપાસો, ફિલ્ટર સાફ કરો
    \end{itemize}

    \begin{figure}[H]
        \centering
        \begin{tikzpicture}[gtu flow]
            \node (A) [gtu block] {સમસ્યા મળી};
            \node (B) [gtu state, right=of A] {મશીન ચાલુ થાય?};
            
            \node (C) [gtu block, below=of B] {પાવર/ડોર લોક તપાસો};
            \node (D) [gtu state, right=of B] {પાણી ભરાય છે?};
            
            \node (E) [gtu block, below=of D] {ઇનલેટ વાલ્વ તપાસો};
            \node (F) [gtu state, right=of D] {પાણી નીકળે છે?};
            
            \node (G) [gtu block, below=of F] {ફિલ્ટર/પંપ તપાસો};
            \node (H) [gtu state, right=of F] {વધુ વાઇબ્રેશન?};
            
            \node (I) [gtu block, below=of H] {લોડ બેલેન્સ તપાસો};
            \node (J) [gtu state, right=of H] {પાણી લીકેજ?};
            
            \node (K) [gtu block, below=of J] {હોઝ/કનેક્શન તપાસો};

            \draw [gtu arrow] (A) -- (B);
            \draw [gtu arrow] (B) -- node[left] {ના} (C);
            \draw [gtu arrow] (B) -- node[above] {હા} (D);
            
            \draw [gtu arrow] (D) -- node[left] {ના} (E);
            \draw [gtu arrow] (D) -- node[above] {હા} (F);
            
            \draw [gtu arrow] (F) -- node[left] {ના} (G);
            \draw [gtu arrow] (F) -- node[above] {હા} (H);
            
            \draw [gtu arrow] (H) -- node[left] {હા} (I);
            \draw [gtu arrow] (H) -- node[above] {ના} (J);
            
            \draw [gtu arrow] (J) -- node[left] {હા} (K);
        \end{tikzpicture}
        \caption{\foreignlanguage{english}{Washing Machine Troubleshooting}}
    \end{figure}

    \begin{mnemonicbox}
        \mnemonic{POWER: પાવર, ઑબ્ઝર્વ, વોટર, એક્ઝામિન, રિપેર}
    \end{mnemonicbox}
\end{solutionbox}

\questionmarks{1(c) OR}{7}{ડીજીટલ ટીવી ના મેઇંટેનન્સ અને ટ્રબલશૂટીંગ ની પ્રક્રિયા સમજાવો.}
\begin{solutionbox}
    \textbf{Table: ડિજિટલ ટીવી મેઇંટેનન્સ અને ટ્રબલશૂટિંગ} \\
    \begin{tabulary}{\linewidth}{|L|L|L|}
        \hline
        \textbf{સમસ્યા} & \textbf{સંભવિત કારણ} & \textbf{ટ્રબલશૂટિંગ સ્ટેપ્સ} \\
        \hline
        \textbf{પાવર ન આવવો} & પાવર સપ્લાય સમસ્યા & પાવર કોર્ડ, વોલ આઉટલેટ તપાસો, જુદા સોકેટમાં પ્રયાસ કરો \\
        \hline
        \textbf{ચિત્ર ન દેખાવું} & ઇનપુટ/સોર્સ પસંદગી & યોગ્ય ઇનપુટ પસંદ કર્યું છે તે તપાસો, સોર્સ ઉપકરણ તપાસો \\
        \hline
        \textbf{નબળું રિસેપ્શન} & એન્ટેના/કેબલ સમસ્યા & કેબલ કનેક્શન તપાસો, એન્ટેના સ્થિતિ બદલો \\
        \hline
        \textbf{વિકૃત રંગો} & ડિસ્પ્લે સેટિંગ્સ & પિક્ચર સેટિંગ્સ ડિફોલ્ટ પર રીસેટ કરો \\
        \hline
        \textbf{રિમોટ કામ ન કરવું} & બેટરી સમસ્યા, સેન્સર બ્લોક & બેટરી બદલો, IR સેન્સર બ્લોક નથી તેની ખાતરી કરો \\
        \hline
    \end{tabulary}

    \textbf{નિયમિત મેઇંટેનન્સ:}
    \begin{itemize}
        \item \textbf{સાપ્તાહિક}: માઇક્રોફાઇબર કપડાથી સ્ક્રીન સાવચેતીથી સાફ કરો
        \item \textbf{માસિક}: કેબલ કનેક્શન તપાસો અને કસો
        \item \textbf{વાર્ષિક}: જો ઉપલબ્ધ હોય તો ફર્મવેર અપડેટ કરો
    \end{itemize}

    \begin{figure}[H]
        \centering
        \begin{tikzpicture}[gtu flow]
            \node (A) [gtu block] {ટીવી સમસ્યા};
            \node (B) [gtu state, right=of A] {પાવર ચાલુ થાય?};
            \node (C) [gtu block, below=of B] {પાવર સપ્લાય તપાસો};
            
            \node (D) [gtu state, right=of B] {ચિત્ર દેખાય?};
            \node (E) [gtu block, below=of D] {ઇનપુટ સોર્સ તપાસો};
            
            \node (F) [gtu state, right=of D] {હારૂં રિસેપ્શન?};
            \node (G) [gtu block, below=of F] {એન્ટેના/કેબલ તપાસો};
            
            \node (H) [gtu state, right=of F] {યોગ્ય રંગો?};
            \node (I) [gtu block, below=of H] {સેટિંગ્સ રીસેટ કરો};
            
            \node (J) [gtu state, right=of H] {રિમોટ કામ કરે?};
            \node (K) [gtu block, below=of J] {બેટરી/સેન્સર તપાસો};

            \draw [gtu arrow] (A) -- (B);
            \draw [gtu arrow] (B) -- node[right] {ના} (C);
            \draw [gtu arrow] (B) -- node[above] {હા} (D);
            
            \draw [gtu arrow] (D) -- node[right] {ના} (E);
            \draw [gtu arrow] (D) -- node[above] {હા} (F);
            
            \draw [gtu arrow] (F) -- node[right] {ના} (G);
            \draw [gtu arrow] (F) -- node[above] {હા} (H);
            
            \draw [gtu arrow] (H) -- node[right] {ના} (I);
            \draw [gtu arrow] (H) -- node[above] {હા} (J);
            
            \draw [gtu arrow] (J) -- node[right] {ના} (K);
        \end{tikzpicture}
        \caption{\foreignlanguage{english}{TV Troubleshooting}}
    \end{figure}

    \begin{mnemonicbox}
        \mnemonic{SPIRE: સપ્લાય, પિક્ચર, ઇનપુટ, રિસેપ્શન, ઇલેક્ટ્રોનિક્સ}
    \end{mnemonicbox}
\end{solutionbox}

\questionmarks{2(a)}{3}{વ્યાખ્યા આપો: (૧) બ્રાઈટનેસ (૨) લ્યુમિનેન્સ (3) ક્રોમિનેન્સ}
\begin{solutionbox}
    \textbf{Table: ટીવી ડિસ્પ્લે ટર્મ્સ} \\
    \begin{tabulary}{\linewidth}{|l|L|L|}
        \hline
        \textbf{પદ} & \textbf{વ્યાખ્યા} & \textbf{માપન એકમ} \\
        \hline
        \textbf{બ્રાઈટનેસ} & ડિસ્પ્લેમાંથી પ્રકાશની તીવ્રતાનું અનુભવાતું મૂલ્ય & સબ્જેક્ટિવ પર્સેપ્શન (નિટ્સ) \\
        \hline
        \textbf{લ્યુમિનેન્સ} & પ્રતિ એકમ ક્ષેત્રફળ માટે પ્રકાશની તીવ્રતાનું ઓબ્જેક્ટિવ માપન & કેન્ડેલા પ્રતિ ચોરસ મીટર (cd/m$^2$) \\
        \hline
        \textbf{ક્રોમિનેન્સ} & વિડિઓ સિગ્નલમાં બ્રાઈટનેસથી સ્વતંત્ર રંગ માહિતી & U અને V કોમ્પોનન્ટ્સ \\
        \hline
    \end{tabulary}

    \begin{mnemonicbox}
        \mnemonic{BLC: બ્રાઈટનેસ એટલે પ્રકાશ અનુભવ, લ્યુમિનેન્સ એટલે ગણિત પ્રકાશ, ક્રોમિનેન્સ એટલે રંગ માહિતી}
    \end{mnemonicbox}
\end{solutionbox}

\questionmarks{2(b)}{4}{ડીટીએચ રિસિવર નો બ્લોક ડાયેગ્રામ દોરો અને સમજાવો.}
\begin{solutionbox}
    \textbf{DTH રિસિવર બ્લોક ડાયાગ્રામ:}
    
    \begin{figure}[H]
        \centering
        \begin{tikzpicture}[gtu flow]
            \node (A) [gtu block] {સેટેલાઈટ ડિશ};
            \node (B) [gtu block, right=of A] {LNB};
            \node (C) [gtu block, right=of B] {ટ્યુનર};
            \node (D) [gtu block, below=of C] {ડિમોડ્યુલેટર};
            \node (E) [gtu block, left=of D] {MPEG ડિકોડર};
            
            \node (F) [gtu block, left=of E] {વિડિઓ પ્રોસેસર};
            \node (G) [gtu block, below=of E] {ઓડિઓ પ્રોસેસર};
            
            \node (H) [gtu block, left=of F] {ટીવી ડિસ્પ્લે};
            \node (I) [gtu block, left=of G] {સ્પીકર્સ};
            
            \node (J) [gtu block, below=of D] {સ્માર્ટ કાર્ડ};
            \node (K) [gtu block, below=of J] {CAM};
            
            \node (L) [gtu block, right=of C] {યુઝર ઇન્ટરફેસ};
            \node (M) [gtu block, below=of L] {માઇક્રોકન્ટ્રોલર};

            \draw [gtu arrow] (A) -- (B);
            \draw [gtu arrow] (B) -- (C);
            \draw [gtu arrow] (C) -- (D);
            \draw [gtu arrow] (D) -- (E);
            
            \draw [gtu arrow] (E) -- (F);
            \draw [gtu arrow] (E) -- (G);
            \draw [gtu arrow] (F) -- (H);
            \draw [gtu arrow] (G) -- (I);
            
            \draw [gtu arrow] (J) -- (K);
            \draw [gtu arrow] (K) -| (D);
            
            \draw [gtu arrow] (L) -- (M);
            \draw [gtu arrow] (M) -- (C);
            \draw [gtu arrow] (M) |- (E);
        \end{tikzpicture}
        \caption{\foreignlanguage{english}{DTH Receiver}}
    \end{figure}

    \textbf{Table: DTH રિસિવર કોમ્પોનન્ટ્સ} \\
    \begin{tabulary}{\linewidth}{|L|L|}
        \hline
        \textbf{કોમ્પોનન્ટ} & \textbf{કાર્ય} \\
        \hline
        \textbf{સેટેલાઈટ ડિશ} & અવકાશમાંથી સેટેલાઈટ સિગ્નલ્સ મેળવે છે \\
        \hline
        \textbf{LNB (લો નોઈઝ બ્લોક)} & ઉચ્ચ-આવૃત્તિના સિગ્નલ્સને નીચી આવૃત્તિમાં પરિવર્તિત કરે છે \\
        \hline
        \textbf{ટ્યુનર} & ચોક્કસ ચેનલ ફ્રિક્વન્સી પસંદ કરે છે \\
        \hline
        \textbf{ડિમોડ્યુલેટર} & કેરિયર સિગ્નલમાંથી ડિજિટલ ડેટા કાઢે છે \\
        \hline
        \textbf{MPEG ડિકોડર} & ઓડિઓ/વિડિઓ ડેટા ડિકમ્પ્રેસ કરે છે \\
        \hline
        \textbf{કન્ડિશનલ એક્સેસ મોડ્યુલ} & સબ્સ્ક્રિપ્શન એક્સેસ નિયંત્રિત કરે છે \\
        \hline
    \end{tabulary}

    \begin{mnemonicbox}
        \mnemonic{SLTDM: સેટેલાઈટ કેપ્ચર કરે, LNB કન્વર્ટ કરે, ટ્યુનર સિલેક્ટ કરે, ડિમોડ્યુલેટર એક્સટ્રેક્ટ કરે, MPEG ડિકોડ કરે}
    \end{mnemonicbox}
\end{solutionbox}

\questionmarks{2(c)}{7}{કલર ટીવી રિસિવર નો બ્લોક ડાયેગ્રામ દોરો અને સમજાવો.}
\begin{solutionbox}
    \textbf{કલર ટીવી રિસિવર બ્લોક ડાયાગ્રામ:}

    \begin{figure}[H]
        \centering
        \begin{tikzpicture}[gtu flow]
            \node (A) [gtu block] {એન્ટેના};
            \node (B) [gtu block, right=of A] {ટ્યુનર};
            \node (C) [gtu block, right=of B] {IF એમ્પ્લિફાયર};
            \node (D) [gtu block, right=of C] {વિડિઓ ડિટેક્ટર};
            
            \node (E) [gtu block, below=of D] {વિડિઓ એમ્પ્લિફાયર};
            \node (F) [gtu block, right=of D] {સાઉન્ડ IF \& ડિટેક્ટર};
            
            \node (G) [gtu block, left=of E] {Y સિગ્નલ પ્રોસેસિંગ};
            \node (H) [gtu block, below=of E] {ક્રોમિનન્સ બેન્ડપાસ};
            \node (I) [gtu block, below=of H] {ક્રોમા ડિમોડ્યુલેટર};
            
            \node (J) [gtu block, left=of I] {R-Y સિગ્નલ};
            \node (K) [gtu block, below=of I] {B-Y સિગ્નલ};
            
            \node (L) [gtu block, left=of G, text width=2.5cm] {RGB મેટ્રિક્સ};
            \node (M) [gtu block, left=of L] {પિક્ચર ટ્યુબ};
            
            \node (N) [gtu block, right=of F] {ઓડિઓ એમ્પ્લિફાયર};
            \node (O) [gtu block, right=of N] {સ્પીકર};
            \node (P) [gtu block, right=of B] {પાવર સપ્લાય};

            \draw [gtu arrow] (A) -- (B);
            \draw [gtu arrow] (B) -- (C);
            \draw [gtu arrow] (C) -- (D);
            \draw [gtu arrow] (D) -- (E);
            \draw [gtu arrow] (D) -- (F);
            
            \draw [gtu arrow] (E) -- (G);
            \draw [gtu arrow] (E) -- (H);
            \draw [gtu arrow] (H) -- (I);
            
            \draw [gtu arrow] (I) -- (J);
            \draw [gtu arrow] (I) -- (K);
            
            \draw [gtu arrow] (G) -- (L);
            \draw [gtu arrow] (J) -- (L);
            \draw [gtu arrow] (K) -| (L);
            
            \draw [gtu arrow] (L) -- (M);
            \draw [gtu arrow] (F) -- (N);
            \draw [gtu arrow] (N) -- (O);
            
            \draw [gtu arrow, dashed] (P) to[bend left] (C);
        \end{tikzpicture}
        \caption{\foreignlanguage{english}{Colour TV Receiver}}
    \end{figure}

    \textbf{Table: કલર ટીવી કોમ્પોનન્ટ્સ અને ફંક્શન્સ} \\
    \begin{tabulary}{\linewidth}{|l|L|L|}
        \hline
        \textbf{સેક્શન} & \textbf{ફંક્શન} & \textbf{મુખ્ય કોમ્પોનન્ટ્સ} \\
        \hline
        \textbf{ટ્યુનર} & ઇચ્છિત ચેનલ પસંદ કરે છે & RF એમ્પ્લિફાયર, મિક્સર, લોકલ ઓસિલેટર \\
        \hline
        \textbf{IF એમ્પ્લિફાયર} & ઇન્ટરમીડિયેટ ફ્રિક્વન્સી એમ્પ્લિફાય કરે છે & બેન્ડપાસ ફિલ્ટર્સ, એમ્પ્લિફાયર્સ \\
        \hline
        \textbf{વિડિઓ ડિટેક્ટર} & વિડિઓ સિગ્નલ એક્સટ્રેક્ટ કરે છે & ડાયોડ ડિટેક્ટર, ફિલ્ટર્સ \\
        \hline
        \textbf{ક્રોમિનન્સ સેક્શન} & રંગ માહિતી પ્રોસેસ કરે છે & બેન્ડપાસ ફિલ્ટર, કલર ડિમોડ્યુલેટર \\
        \hline
        \textbf{લ્યુમિનન્સ સેક્શન} & બ્રાઈટનેસ માહિતી પ્રોસેસ કરે છે & Y સિગ્નલ એમ્પ્લિફાયર \\
        \hline
        \textbf{RGB મેટ્રિક્સ} & ડિસ્પ્લે માટે સિગ્નલ્સ ભેગા કરે છે & મિક્સિંગ સર્કિટ્સ \\
        \hline
        \textbf{ઓડિઓ સેક્શન} & અવાજ પ્રોસેસ કરે છે & સાઉન્ડ IF, ડિટેક્ટર, એમ્પ્લિફાયર \\
        \hline
    \end{tabulary}

    \begin{mnemonicbox}
        \mnemonic{TIVACRL: ટ્યુનર ટ્યુન કરે, IF એમ્પ્લિફાય કરે, વિડિઓ ડિટેક્ટ કરે, ઓડિઓ અલગ કરે, ક્રોમિનન્સ ડિમોડ્યુલેટ કરે, RGB મિક્સ કરે, લાઈટ ડિસ્પ્લે કરે}
    \end{mnemonicbox}
\end{solutionbox}

\questionmarks{2(a) OR}{3}{એલઇડી ટીવી પર ટૂંકનોંધ લખો.}
\begin{solutionbox}
    \textbf{Table: LED ટીવી ટેક્નોલોજી} \\
    \begin{tabulary}{\linewidth}{|l|L|}
        \hline
        \textbf{પાસું} & \textbf{વર્ણન} \\
        \hline
        \textbf{મૂળભૂત ટેક્નોલોજી} & ડિસ્પ્લે બેકલાઈટિંગ માટે લાઈટ એમિટિંગ ડાયોડ્સનો ઉપયોગ કરે છે \\
        \hline
        \textbf{પ્રકારો} & એજ-લિટ (કિનારે LED), ડાયરેક્ટ-લિટ (સ્ક્રીન પાછળ LED), ફુલ-એરે (લોકલ ડિમિંગ સાથે) \\
        \hline
        \textbf{ફાયદા} & પાતળી પ્રોફાઇલ, ઊર્જા કાર્યક્ષમ, વધુ સારો કોન્ટ્રાસ્ટ રેશિયો, LCD કરતાં લાંબો જીવનકાળ \\
        \hline
        \textbf{ડિસ્પ્લે પેનલ} & હજુ પણ LCD પેનલનો ઉપયોગ કરે છે; LED ફક્ત બેકલાઈટિંગ માટે છે \\
        \hline
    \end{tabulary}

    \begin{mnemonicbox}
        \mnemonic{BEST: બેકલાઈટિંગ LED સાથે, એનર્જી અસરકારક, સ્લિમ ડિઝાઇન, ટ્રુ કલર્સ}
    \end{mnemonicbox}
\end{solutionbox}

\questionmarks{2(b) OR}{4}{પદો ટૂંક મા સમજાવો: (૧)હ્યુ (૨) સેચ્યુરેશન}
\begin{solutionbox}
    \textbf{Table: રંગ ગુણધર્મો} \\
    \begin{tabulary}{\linewidth}{|l|L|L|L|}
        \hline
        \textbf{પદ} & \textbf{વ્યાખ્યા} & \textbf{રેન્જ} & \textbf{ઉદાહરણ} \\
        \hline
        \textbf{હ્યુ} & વાસ્તવિક રંગ તરંગ લંબાઈ (લાલ, વાદળી, લીલો, વગેરે) & કલર વ્હીલ પર 0-360 ડિગ્રી & લાલ=0$^\circ$, લીલો=120$^\circ$, વાદળી=240$^\circ$ \\
        \hline
        \textbf{સેચ્યુરેશન} & રંગની તીવ્રતા અથવા શુદ્ધતા (કેટલો જીવંત) & 0-100\% (ગ્રે થી શુદ્ધ રંગ) & 0\%=ગ્રેસ્કેલ, 100\%=જીવંત રંગ \\
        \hline
    \end{tabulary}

    \begin{figure}[H]
        \centering
        \begin{tikzpicture}[gtu flow]
            \node (A) [gtu block] {રંગ ગુણધર્મો};
            \node (B) [gtu block, above right=of A] {હ્યુ};
            \node (C) [gtu block, below right=of A] {સેચ્યુરેશન};
            
            \node (D) [gtu block, right=of B] {તરંગલંબાઈ};
            \node (E) [gtu block, right=of C] {શુદ્ધતા};
            
            \node (F) [right=of D, font=\footnotesize] {0-360$^\circ$};
            \node (G) [right=of E, font=\footnotesize] {0-100\%};

            \draw [gtu arrow] (A) -- (B);
            \draw [gtu arrow] (A) -- (C);
            \draw [gtu arrow] (B) -- (D);
            \draw [gtu arrow] (C) -- (E);
            \draw [gtu arrow] (D) -- (F);
            \draw [gtu arrow] (E) -- (G);
        \end{tikzpicture}
        \caption{\foreignlanguage{english}{Hue \& Saturation}}
    \end{figure}

    \begin{mnemonicbox}
        \mnemonic{HS: હ્યુ એટલે રંગનો શેડ, સેચ્યુરેશન એટલે રંગની સ્ટ્રેન્થ}
    \end{mnemonicbox}
\end{solutionbox}

\questionmarks{2(c) OR}{7}{કલર સર્કલ ડાયેગ્રામ અને ગ્રાસમેનના નિયમ ની મદદ થી એડીટીવ કલર મિક્સિંગ સમજાવો.}
\begin{solutionbox}
    \textbf{Table: એડિટિવ કલર મિક્સિંગ પ્રિન્સિપલ્સ} \\
    \begin{tabulary}{\linewidth}{|L|l|l|}
        \hline
        \textbf{રંગનું સંયોજન} & \textbf{પરિણામ} & \textbf{RGB મૂલ્ય} \\
        \hline
        \textbf{લાલ + લીલો} & પીળો & (255,255,0) \\
        \hline
        \textbf{લીલો + વાદળી} & સિયાન & (0,255,255) \\
        \hline
        \textbf{વાદળી + લાલ} & મેજેન્ટા & (255,0,255) \\
        \hline
        \textbf{લાલ + લીલો + વાદળી} & સફેદ & (255,255,255) \\
        \hline
        \textbf{કોઈ રંગ નહીં} & કાળો & (0,0,0) \\
        \hline
    \end{tabulary}

    \textbf{ગ્રાસમેનના નિયમો:}
    \begin{itemize}
        \item \textbf{નિયમ 1}: કોઈપણ રંગ ત્રણ પ્રાથમિક રંગો મિશ્ર કરીને બનાવી શકાય છે
        \item \textbf{નિયમ 2}: રંગનો દેખાવ માત્ર તેના ટ્રિસ્ટિમ્યુલસ મૂલ્યો પર આધારિત છે
        \item \textbf{નિયમ 3}: એડિટિવ મિક્સિંગમાં, ટ્રિસ્ટિમ્યુલસ મૂલ્યો એકસાથે ઉમેરાય છે
    \end{itemize}

    \begin{figure}[H]
        \centering
        \begin{tikzpicture}[gtu flow]
            \node (P) [gtu block] {પ્રાથમિક રંગો};
            \node (R) [gtu block, fill=red!20, below left=1.5cm and 2cm of P] {લાલ};
            \node (G) [gtu block, fill=green!20, below=1.5cm of P] {લીલો};
            \node (B) [gtu block, fill=blue!20, below right=1.5cm and 2cm of P] {વાદળી};
            
            \draw [gtu arrow] (P) -- (R);
            \draw [gtu arrow] (P) -- (G);
            \draw [gtu arrow] (P) -- (B);
            
            \node (Y) [gtu block, fill=yellow!20, below left=1.5cm of G] {પીળો};
            \node (C) [gtu block, fill=cyan!20, below right=1.5cm of G] {સિયાન};
            \node (M) [gtu block, fill=magenta!20, below right=1.5cm and -1.5cm of Y] {મેજેન્ટા};
            
            \draw [gtu arrow] (R) -- (Y);
            \draw [gtu arrow] (G) -- (Y);
            
            \draw [gtu arrow] (G) -- (C);
            \draw [gtu arrow] (B) -- (C);
            
            \draw [gtu arrow] (B) -- (M);
            \draw [gtu arrow] (R) -- (M);
            
            \node (W) [gtu block, fill=white, draw=black, below=2cm of G] {સફેદ};
            \draw [gtu arrow] (R) -- (W);
            \draw [gtu arrow] (G) -- (W);
            \draw [gtu arrow] (B) -- (W);
        \end{tikzpicture}
        \caption{\foreignlanguage{english}{Additive Mixing Flow}}
    \end{figure}

    \textbf{કલર સર્કલ ડાયાગ્રામ:}
    \begin{figure}[H]
        \centering
        \begin{tikzpicture}
            \def\R{2}
            \coordinate (Center) at (0,0);
            
            \node (Red) at (90:\R) [circle, fill=red, text=white, minimum size=1.5cm] {Red};
            \node (Green) at (330:\R) [circle, fill=green, text=black, minimum size=1.5cm] {Green};
            \node (Blue) at (210:\R) [circle, fill=blue, text=white, minimum size=1.5cm] {Blue};
            
            \node (Yellow) at (30:{\R*0.6}) [circle, fill=yellow, text=black, minimum size=1.2cm] {Yellow};
            \node (Cyan) at (270:{\R*0.6}) [circle, fill=cyan, text=black, minimum size=1.2cm] {Cyan};
            \node (Magenta) at (150:{\R*0.6}) [circle, fill=magenta, text=white, minimum size=1.2cm] {Magenta};
            
            \node (White) at (0,0) [circle, fill=white, draw=black, minimum size=1cm] {White};
            
            \draw [thick, ->] (Red) -- (Yellow);
            \draw [thick, ->] (Green) -- (Yellow);
            
            \draw [thick, ->] (Green) -- (Cyan);
            \draw [thick, ->] (Blue) -- (Cyan);
            
            \draw [thick, ->] (Blue) -- (Magenta);
            \draw [thick, ->] (Red) -- (Magenta);
        \end{tikzpicture}
        \caption{\foreignlanguage{english}{Color Mixing Circle}}
    \end{figure}

    \begin{mnemonicbox}
        \mnemonic{RGB-CMY-W: લાલ, લીલો, વાદળી, સિયાન, મેજેન્ટા, પીળો, અને સફેદ બનાવે છે}
    \end{mnemonicbox}
\end{solutionbox}

\questionmarks{3(a)}{3}{માઇક્રોવેવ ઓવન માટે વાયરિંગ અને સેફ્ટી ઇંસ્ટ્રક્શન લખો.}
\begin{solutionbox}
    \textbf{Table: માઇક્રોવેવ ઓવન વાયરિંગ અને સેફ્ટી ઇન્સ્ટ્રક્શન્સ} \\
    \begin{tabulary}{\linewidth}{|l|L|}
        \hline
        \textbf{કેટેગરી} & \textbf{સૂચનાઓ} \\
        \hline
        \textbf{વાયરિંગ} & 15-20A સર્કિટ સાથે ગ્રાઉન્ડેડ આઉટલેટનો ઉપયોગ કરો \\
        \hline
        \textbf{પાવર} & વોલ્ટેજ રેટિંગ સાથે મેળ ખાય તેની ખાતરી કરો (સામાન્ય રીતે 220-240V) \\
        \hline
        \textbf{ઇન્સ્ટોલેશન} & વેન્ટિલેશન માટે તમામ બાજુએ 5 સેમી જગ્યા રાખો \\
        \hline
        \textbf{સેફ્ટી} & ક્યારેય ખાલી ન ચલાવો, ક્યારેય ડોર ઇન્ટરલોક્સ બાયપાસ ન કરો \\
        \hline
        \textbf{મેઇંટેનન્સ} & સર્વિસિંગ પહેલાં પાવર ડિસ્કનેક્ટ કરો, કેપેસિટર ડિસ્ચાર્જ કરો \\
        \hline
    \end{tabulary}

    \begin{mnemonicbox}
        \mnemonic{POWER: પ્રોપર આઉટલેટ, વાયરિંગ ચેક, એમ્પ્ટી ઓપરેશન અવોઇડેડ, રિપેર્સ બાય પ્રોફેશનલ્સ}
    \end{mnemonicbox}
\end{solutionbox}

\questionmarks{3(b)}{4}{એર કંડીશનર ની કાર્યપધ્ધતિ સમજાવો.}
\begin{solutionbox}
    \textbf{Table: એર કન્ડિશનર વર્કિંગ સાયકલ} \\
    \begin{tabulary}{\linewidth}{|l|L|L|}
        \hline
        \textbf{કોમ્પોનન્ટ} & \textbf{ફંક્શન} & \textbf{પ્રક્રિયા} \\
        \hline
        \textbf{કમ્પ્રેસર} & રેફ્રિજરન્ટ પ્રેશરાઇઝ કરે છે & ઓછા દબાણવાળી ગેસને ઉચ્ચ દબાણવાળી ગેસમાં પરિવર્તિત કરે છે \\
        \hline
        \textbf{કન્ડેન્સર} & બહાર ગરમી છોડે છે & ગેસને પ્રવાહીમાં પરિવર્તિત કરે છે, ગરમી કાઢે છે \\
        \hline
        \textbf{એક્સપાન્શન વાલ્વ} & રેફ્રિજરન્ટનો પ્રવાહ નિયંત્રિત કરે છે & પ્રવાહીનું દબાણ ઘટાડે છે \\
        \hline
        \textbf{ઇવેપોરેટર} & રૂમમાંથી ગરમી શોષે છે & પ્રવાહીને ગેસમાં પરિવર્તિત કરે છે, હવા ઠંડી કરે છે \\
        \hline
        \textbf{થર્મોસ્ટેટ} & તાપમાન નિયંત્રિત કરે છે & કમ્પ્રેસર ઓપરેશન રેગ્યુલેટ કરે છે \\
        \hline
    \end{tabulary}

    \begin{figure}[H]
        \centering
        \begin{tikzpicture}[gtu flow]
            \node (Comp) [gtu block] {કમ્પ્રેસર};
            \node (Cond) [gtu block, right=2cm of Comp] {કન્ડેન્સર};
            \node (Exp) [gtu block, below=2cm of Cond] {એક્સપાન્શન વાલ્વ};
            \node (Evap) [gtu block, below=2cm of Comp] {ઇવેપોરેટર};
            
            \draw [gtu arrow] (Comp) -- node[above, font=\footnotesize] {High-P ગેસ} (Cond);
            \draw [gtu arrow] (Cond) -- node[right, font=\footnotesize] {પ્રવાહી} (Exp);
            \draw [gtu arrow] (Exp) -- node[below, font=\footnotesize] {Low-P પ્રવાહી} (Evap);
            \draw [gtu arrow] (Evap) -- node[left, font=\footnotesize] {Low-P ગેસ} (Comp);
            
            \node (In) [left=of Evap, align=center] {રૂમ એર};
            \node (Out) [right=of Evap, align=center] {કૂલ એર};
            \draw [gtu arrow] (In) -- (Evap);
            \draw [gtu arrow] (Evap) -- (Out);
            
            \node (Outside) [left=of Cond, align=center] {લે-આઉટસાઇડ એર};
            \node (Hot) [right=of Cond, align=center] {હોટ એર};
            \draw [gtu arrow] (Outside) -- (Cond);
            \draw [gtu arrow] (Cond) -- (Hot);
        \end{tikzpicture}
        \caption{\foreignlanguage{english}{Air Conditioner Cycle}}
    \end{figure}

    \begin{mnemonicbox}
        \mnemonic{CELT: કમ્પ્રેસ ગેસ, એક્સપેલ હીટ, લોઅર પ્રેશર, ટેક ઇન હીટ}
    \end{mnemonicbox}
\end{solutionbox}

\questionmarks{3(c)}{7}{વોશિંગ મશીન માટે ઇલેક્ટ્રોનિક કંટ્રોલર અને ફજી લોજીક વોશિંગ મશીન સમજાવો. વોશિંગ મશીન ના ટેકનીકલ સ્પેસીફીકેશનો પણ લખો.}
\begin{solutionbox}
    \textbf{Table: વોશિંગ મશીનમાં ઇલેક્ટ્રોનિક કંટ્રોલર} \\
    \begin{tabulary}{\linewidth}{|l|L|}
        \hline
        \textbf{કોમ્પોનન્ટ} & \textbf{ફંક્શન} \\
        \hline
        \textbf{માઇક્રોકંટ્રોલર} & બધા ઓપરેશન્સ નિયંત્રિત કરતું સેન્ટ્રલ પ્રોસેસિંગ યુનિટ \\
        \hline
        \textbf{સેન્સર્સ} & વોટર લેવલ, તાપમાન, લોડ બેલેન્સ, ડોર સ્ટેટસ ડિટેક્ટ કરે છે \\
        \hline
        \textbf{ઇનપુટ ઇન્ટરફેસ} & પ્રોગ્રામ પસંદગી માટે બટન/ટચ પેનલ \\
        \hline
        \textbf{ડિસ્પ્લે} & પ્રોગ્રામ સ્ટેટસ, બાકી સમય, એરર કોડ્સ બતાવે છે \\
        \hline
        \textbf{એક્ચ્યુએટર ડ્રાઇવર્સ} & મોટર, વાલ્વ, હીટર, પંપ નિયંત્રિત કરે છે \\
        \hline
    \end{tabulary}

    \textbf{ફજી લોજિક વોશિંગ મશીન:}
    \begin{itemize}
        \item શ્રેષ્ઠ વોશિંગ માટે આર્ટિફિશિયલ ઇન્ટેલિજન્સનો ઉપયોગ કરે છે
        \item લોડના આધારે વોટર લેવલ, વોશ ટાઇમ અને સ્પિન સ્પીડ એડજસ્ટ કરે છે
        \item ચોક્કસ મૂલ્યોને બદલે અંદાજિત તર્ક વડે નિર્ણયો લે છે
        \item વિવિધ ફેબ્રિક પ્રકારો અને મેલના સ્તરો સાથે આપોઆપ અનુકૂલન કરે છે
    \end{itemize}

    \textbf{ટેકનિકલ સ્પેસિફિકેશન્સ:}
    \begin{itemize}
        \item \textbf{ક્ષમતા}: 6-10 કિલો (ફ્રન્ટ લોડ), 5-8 કિલો (ટોપ લોડ)
        \item \textbf{એનર્જી રેટિંગ}: A+++ થી B (EU સ્ટાન્ડર્ડ)
        \item \textbf{વોટર કન્ઝમ્પશન}: સાયકલ દીઠ 40-70 લિટર
        \item \textbf{સ્પિન સ્પીડ}: 800-1600 RPM
        \item \textbf{સાયકલ ઓપ્શન્સ}: 8-16 પ્રોગ્રામ્સ
    \end{itemize}

    \begin{figure}[H]
        \centering
        \begin{tikzpicture}[gtu flow]
            \node (MC) [gtu block, minimum width=3cm, minimum height=2cm] {ઇલેક્ટ્રોનિક કંટ્રોલર\\(માઇક્રોકંટ્રોલર)};
            
            \node (Sens) [gtu block, left=2cm of MC] {સેન્સર્સ};
            \node (UI) [gtu block, above=1cm of MC] {યુઝર ઇન્ટરફેસ};
            \node (Fuzzy) [gtu block, below=1cm of MC] {ફજી લોજિક};
            \node (Act) [gtu block, right=2cm of MC] {એક્ચ્યુએટર્સ};
            
            \draw [gtu arrow] (Sens) -- (MC);
            \draw [gtu arrow] (UI) -- (MC);
            \draw [gtu arrow] (Fuzzy) -- (MC);
            \draw [gtu arrow] (MC) -- (Act);
            
            % Detail inputs/outputs
            \node [left=0.2cm of Sens, align=right, font=\footnotesize] {વોટર લેવલ\\ટેમ્પ\\લોડ બેલેન્સ\\ડોર લોક};
            \node [right=0.2cm of Act, align=left, font=\footnotesize] {મોટર\\વોટર વાલ્વ\\ડ્રેન પંપ\\હીટર};
        \end{tikzpicture}
        \caption{\foreignlanguage{english}{Washing Machine Controller}}
    \end{figure}

    \begin{mnemonicbox}
        \mnemonic{SCRAM: સેન્સર્સ ડિટેક્ટ, કંટ્રોલર પ્રોસેસ, રૂલ્સ એપ્લાઇડ, એક્ચ્યુએટર્સ ઓપરેટ, મશીન એડેપ્ટ}
    \end{mnemonicbox}
\end{solutionbox}

\questionmarks{3(a) OR}{3}{સોલર પાવર સીસ્ટમના મેઇન કોમ્પોનન્ટો અને સોલર પાવર સીસ્ટમના સ્પેસીફીકેશનો લખો.}
\begin{solutionbox}
    \textbf{Table: સોલર પાવર સિસ્ટમ કોમ્પોનન્ટ્સ} \\
    \begin{tabulary}{\linewidth}{|l|L|}
        \hline
        \textbf{કોમ્પોનન્ટ} & \textbf{ફંક્શન} \\
        \hline
        \textbf{સોલર પેનલ્સ} & સૂર્યપ્રકાશને DC વીજળીમાં રૂપાંતરિત કરે છે \\
        \hline
        \textbf{ઇન્વર્ટર} & DC પાવરને AC પાવરમાં રૂપાંતરિત કરે છે \\
        \hline
        \textbf{બેટરી બેંક} & પછીના ઉપયોગ માટે ઊર્જા સંગ્રહિત કરે છે \\
        \hline
        \textbf{ચાર્જ કંટ્રોલર} & બેટરીના ઓવરચાર્જિંગને અટકાવે છે \\
        \hline
        \textbf{માઉન્ટિંગ સ્ટ્રક્ચર} & પેનલોને ટેકો આપે છે અને શ્રેષ્ઠ રીતે એંગલ કરે છે \\
        \hline
    \end{tabulary}

    \textbf{સ્પેસિફિકેશન્સ:}
    \begin{itemize}
        \item \textbf{પેનલ કેપેસિટી}: પેનલ દીઠ 250-400 વોટ
        \item \textbf{સિસ્ટમ સાઇઝ}: 1-10 kW (રહેણાંક)
        \item \textbf{બેટરી કેપેસિટી}: 100-200 Ah
        \item \textbf{ઇન્વર્ટર એફિશિયન્સી}: 90-97\%
        \item \textbf{અપેક્ષિત જીવનકાળ}: 25-30 વર્ષ (પેનલ)
    \end{itemize}

    \begin{mnemonicbox}
        \mnemonic{PIBCM: પેનલ કલેક્ટ, ઇન્વર્ટર કન્વર્ટ, બેટરી સ્ટોર, કંટ્રોલર પ્રોટેક્ટ, માઉન્ટ્સ સપોર્ટ}
    \end{mnemonicbox}
\end{solutionbox}

\questionmarks{3(b) OR}{4}{રેફ્રીજરેટર ની કાર્યપધ્ધતિ સમજાવો.}
\begin{solutionbox}
    \textbf{Table: રેફ્રિજરેટર વર્કિંગ સાયકલ} \\
    \begin{tabulary}{\linewidth}{|c|l|l|L|}
        \hline
        \textbf{સ્ટેજ} & \textbf{પ્રક્રિયા} & \textbf{કોમ્પોનન્ટ} & \textbf{રેફ્રિજરન્ટની સ્થિતિ} \\
        \hline
        1 & કમ્પ્રેશન & કમ્પ્રેસર & ઓછા દબાણવાળી ગેસ $\rightarrow$ ઉચ્ચ દબાણવાળી ગેસ \\
        \hline
        2 & કન્ડેન્સર & કન્ડેન્સર કોઇલ્સ & ઉચ્ચ દબાણવાળી ગેસ $\rightarrow$ ઉચ્ચ દબાણવાળી પ્રવાહી \\
        \hline
        3 & એક્સપાન્શન & એક્સપાન્શન વાલ્વ & ઉચ્ચ દબાણવાળી પ્રવાહી $\rightarrow$ ઓછા દબાણવાળી પ્રવાહી \\
        \hline
        4 & ઇવેપોરેશન & ઇવેપોરેટર કોઇલ્સ & ઓછા દબાણવાળી પ્રવાહી $\rightarrow$ ઓછા દબાણવાળી ગેસ \\
        \hline
    \end{tabulary}

    \begin{figure}[H]
        \centering
        \begin{tikzpicture}[gtu flow]
            \node (Comp) [gtu block] {કમ્પ્રેસર};
            \node (Cond) [gtu block, right=2cm of Comp] {કન્ડેન્સર};
            \node (Exp) [gtu block, below=2cm of Cond] {એક્સપાન્શન વાલ્વ};
            \node (Evap) [gtu block, below=2cm of Comp] {ઇવેપોરેટર};
            
            \draw [gtu arrow] (Comp) -- node[above, font=\footnotesize] {High-P ગેસ} (Cond);
            \draw [gtu arrow] (Cond) -- node[right, font=\footnotesize] {High-P પ્રવાહી} (Exp);
            \draw [gtu arrow] (Exp) -- node[below, font=\footnotesize] {Low-P પ્રવાહી} (Evap);
            \draw [gtu arrow] (Evap) -- node[left, font=\footnotesize] {Low-P ગેસ} (Comp);
            
            \node (HeatOut) [above=0.5cm of Cond, font=\footnotesize] {ગરમી બહાર છોડે};
            \draw [->, thick] (Cond) -- (HeatOut);
            
            \node (HeatIn) [below=0.5cm of Evap, font=\footnotesize] {ગરમી શોષે};
            \draw [->, thick] (HeatIn) -- (Evap);
            
            \node (Thermo) [left=of Comp, font=\footnotesize] {થર્મોસ્ટેટ};
            \draw [gtu arrow] (Thermo) -- (Comp);
        \end{tikzpicture}
        \caption{\foreignlanguage{english}{Refrigerator Cycle}}
    \end{figure}

    \begin{mnemonicbox}
        \mnemonic{CEHE: કમ્પ્રેસ ગેસ, એક્સપેલ હીટ, હાલ્વ પ્રેશર, એક્સટ્રેક્ટ હીટ}
    \end{mnemonicbox}
\end{solutionbox}

\questionmarks{3(c) OR}{7}{માઇક્રોવેવ ઓવન નો બ્લોક ડાયેગ્રામ દોરો અને સમજાવો. માઇક્રોવેવ ઓવન ના પ્રકારો, એપ્લીકેશનો અને ટેકનીકલ સ્પેસીફીકેશનો લખો.}
\begin{solutionbox}
    \textbf{માઇક્રોવેવ ઓવન બ્લોક ડાયાગ્રામ:}

    \begin{figure}[H]
        \centering
        \begin{tikzpicture}[gtu flow]
            \node (PS) [gtu block] {પાવર સપ્લાય};
            \node (Control) [gtu block, below=of PS] {કંટ્રોલ પેનલ};
            \node (HV) [gtu block, right=of PS] {HV ટ્રાન્સફોર્મર};
            \node (Circuit) [gtu block, below=of HV] {કંટ્રોલ સર્કિટ};
            
            \node (Magnetron) [gtu block, right=of HV] {મેગ્નેટ્રોન};
            \node (Wave) [gtu block, right=of Magnetron] {વેવગાઇડ};
            \node (Cavity) [gtu block, below=of Wave] {કુકિંગ કેવિટી};
            
            \node (Motor) [gtu block, below=of Cavity] {ટર્નટેબલ મોટર};
            
            \draw [gtu arrow] (PS) -- (HV);
            \draw [gtu arrow] (PS) -- (Control);
            \draw [gtu arrow] (Control) -- (Circuit);
            
            \draw [gtu arrow] (HV) -- (Magnetron);
            \draw [gtu arrow] (Magnetron) -- (Wave);
            \draw [gtu arrow] (Wave) -- (Cavity);
            
            \draw [gtu arrow] (Circuit) -| (Motor);
            \draw [gtu arrow] (Circuit) -- node[right, font=\footnotesize] {કંટ્રોલ} (HV);
        \end{tikzpicture}
        \caption{\foreignlanguage{english}{Microwave Oven Block Diagram}}
    \end{figure}

    \textbf{માઇક્રોવેવ ઓવનના પ્રકારો:}
    \begin{itemize}
        \item \textbf{સોલો}: ફક્ત બેઝિક હીટિંગ અને ડિફ્રોસ્ટિંગ
        \item \textbf{ગ્રિલ}: વધારાના ગ્રિલિંગ એલિમેન્ટ સાથે
        \item \textbf{કન્વેક્શન}: માઇક્રોવેવ સાથે કન્વેક્શન હીટિંગ
        \item \textbf{ઓવર-ધ-રેન્જ (OTR)}: વેન્ટિલેશન સિસ્ટમ સાથે
        \item \textbf{બિલ્ટ-ઇન}: કેબિનેટ ઇન્સ્ટોલેશન માટે ડિઝાઇન કરેલ
    \end{itemize}

    \textbf{એપ્લિકેશન્સ:}
    \begin{itemize}
        \item \textbf{કુકિંગ}: ઝડપી ભોજન તૈયારી
        \item \textbf{રિહીટિંગ}: બચેલા ખોરાક
        \item \textbf{ડિફ્રોસ્ટિંગ}: ફ્રોઝન ફૂડ
        \item \textbf{સ્ટેરિલાઇઝેશન}: નાની વસ્તુઓ
        \item \textbf{કોમર્શિયલ}: ફૂડ સર્વિસ ઇન્ડસ્ટ્રી
    \end{itemize}

    \textbf{ટેકનિકલ સ્પેસિફિકેશન્સ:}
    \begin{itemize}
        \item \textbf{કેપેસિટી}: 20-40 લિટર
        \item \textbf{પાવર આઉટપુટ}: 700-1200 વોટ
        \item \textbf{પાવર કન્ઝમ્પશન}: 1100-1500 વોટ
        \item \textbf{ફ્રિક્વન્સી}: 2.45 GHz
        \item \textbf{વોલ્ટેજ}: 220-240V AC
    \end{itemize}

    \begin{mnemonicbox}
        \mnemonic{MICROWAVES: મેગ્નેટ્રોન જનરેટ કરે, ઇન્ટીરિયર રિસીવ કરે, કંટ્રોલ રેગ્યુલેટ કરે, રોટેટિંગ ટર્નટેબલ, ઓવન કેવિટી, વેવગાઇડ ડાયરેક્ટ કરે, AC પાવર આપે, વેન્ટિલેશન કૂલ કરે, ઇલેક્ટ્રોનિક ટાઇમર, સેફ્ટી ઇન્ટરલોક્સ}
    \end{mnemonicbox}
\end{solutionbox}

\questionmarks{4(a)}{3}{એમએફ પ્રિંટર અને એલસીડી પ્રોજેક્ટર ના સ્પેસીફીકેશનો લખો.}
\begin{solutionbox}
    \textbf{Table: મલ્ટી-ફંક્શન પ્રિંટર સ્પેસિફિકેશન્સ} \\
    \begin{tabulary}{\linewidth}{|l|l|}
        \hline
        \textbf{સ્પેસિફિકેશન} & \textbf{સામાન્ય રેન્જ} \\
        \hline
        \textbf{પ્રિન્ટ રિઝોલ્યુશન} & 600-4800 dpi \\
        \hline
        \textbf{પ્રિન્ટ સ્પીડ} & 20-40 ppm (બ્લેક), 15-30 ppm (કલર) \\
        \hline
        \textbf{સ્કેન રિઝોલ્યુશન} & 600-1200 dpi \\
        \hline
        \textbf{કનેક્ટિવિટી} & Wi-Fi, ઇથરનેટ, USB, ક્લાઉડ \\
        \hline
        \textbf{પેપર કેપેસિટી} & 100-500 શીટ્સ \\
        \hline
    \end{tabulary}

    \textbf{Table: LCD પ્રોજેક્ટર સ્પેસિફિકેશન્સ} \\
    \begin{tabulary}{\linewidth}{|l|l|}
        \hline
        \textbf{સ્પેસિફિકેશન} & \textbf{સામાન્ય રેન્જ} \\
        \hline
        \textbf{બ્રાઈટનેસ} & 2000-5000 લુમેન્સ \\
        \hline
        \textbf{રિઝોલ્યુશન} & XGA (1024$\times$768) થી 4K (3840$\times$2160) \\
        \hline
        \textbf{કોન્ટ્રાસ્ટ રેશિયો} & 2000:1 થી 100,000:1 \\
        \hline
        \textbf{લેમ્પ લાઇફ} & 4000-8000 કલાક \\
        \hline
        \textbf{કનેક્ટિવિટી} & HDMI, VGA, USB, વાયરલેસ \\
        \hline
    \end{tabulary}

    \begin{mnemonicbox}
        \mnemonic{PSCPL: પ્રિન્ટ રિઝોલ્યુશન, સ્પીડ, કનેક્ટિવિટી, પ્રોજેક્શન બ્રાઈટનેસ, લેમ્પ લાઇફ}
    \end{mnemonicbox}
\end{solutionbox}

\questionmarks{4(b)}{4}{ઇન્કજેટ પ્રિંટર નો બ્લોક ડાયેગ્રામ દોરો અને તેની કાર્યપધ્ધતિ ટૂંક મા સમજાવો}
\begin{solutionbox}
    \textbf{ઇન્કજેટ પ્રિંટર બ્લોક ડાયાગ્રામ:}

    \begin{figure}[H]
        \centering
        \begin{tikzpicture}[gtu flow]
            \node (Control) [gtu block] {કંટ્રોલ બોર્ડ/CPU};
            \node (PS) [gtu block, left=of Control] {પાવર સપ્લાય};
            \node (Interface) [gtu block, right=of Control] {ઇન્ટરફેસ};
            \node (PC) [gtu block, right=of Interface] {કમ્પ્યુટર};
            
            \node (PaperMotor) [gtu block, below left=1.5cm of Control] {પેપર ફીડ મોટર};
            \node (PrintMotor) [gtu block, below=1.5cm of Control] {પ્રિન્ટહેડ મોટર};
            \node (HeadControl) [gtu block, below right=1.5cm of Control] {પ્રિન્ટહેડ કંટ્રોલર};
            
            \node (Mech) [gtu block, below=of PaperMotor] {મેકેનિઝમ};
            \node (Carriage) [gtu block, below=of PrintMotor] {કેરેજ};
            \node (Cartridge) [gtu block, below=of HeadControl] {ઇન્ક કાર્ટ્રિજ};
            \node (Nozzle) [gtu block, below=of Cartridge] {નોઝલ્સ};
            
            \draw [gtu arrow] (PS) -- (Control);
            \draw [gtu arrow] (PC) -- (Interface);
            \draw [gtu arrow] (Interface) -- (Control);
            
            \draw [gtu arrow] (Control) -- (PaperMotor);
            \draw [gtu arrow] (Control) -- (PrintMotor);
            \draw [gtu arrow] (Control) -- (HeadControl);
            
            \draw [gtu arrow] (PaperMotor) -- (Mech);
            \draw [gtu arrow] (PrintMotor) -- (Carriage);
            \draw [gtu arrow] (HeadControl) -- (Cartridge);
            \draw [gtu arrow] (Cartridge) -- (Nozzle);
            \draw [gtu arrow] (Carriage) -- (Cartridge);
        \end{tikzpicture}
        \caption{\foreignlanguage{english}{Inkjet Printer}}
    \end{figure}

    \textbf{ઇન્કજેટ પ્રિંટરની કાર્યપદ્ધતિ:}
    \begin{enumerate}
        \item \textbf{ડોક્યુમેન્ટ પ્રોસેસિંગ}: કંટ્રોલ બોર્ડ ડેટા મેળવે છે અને પ્રિન્ટર કમાન્ડમાં રૂપાંતરિત કરે છે
        \item \textbf{પેપર લોડિંગ}: ફીડ મોટર ટ્રેમાંથી પેપર ખેંચે છે
        \item \textbf{પ્રિન્ટિંગ}: પ્રિન્ટહેડ પેપર પર ચાલે છે અને નાના ઇન્ક ડ્રોપલેટ્સ છોડે છે
        \item \textbf{ડ્રોપલેટ ફોર્મેશન}: થર્મલ અથવા પિઝોઇલેક્ટ્રિક પદ્ધતિ દ્વારા ઇન્ક ડ્રોપલેટ્સને પેપર પર મોકલે છે
        \item \textbf{પેપર એડવાન્સમેન્ટ}: પ્રિન્ટિંગ પૂર્ણ થાય ત્યાં સુધી પેપર લાઇન બાય લાઇન આગળ વધે છે
    \end{enumerate}

    \begin{mnemonicbox}
        \mnemonic{PIPES: પેપર ફીડ્સ, ઇન્ક ઇજેક્ટ્સ, પ્રિન્ટહેડ મૂવ્સ, ઇલેક્ટ્રોનિક કંટ્રોલ, શીટ એડવાન્સીસ}
    \end{mnemonicbox}
\end{solutionbox}

\questionmarks{4(c)}{7}{ફોટોકોપીયર ની કાર્યપધ્ધતિ બ્લોક ડાયેગ્રામ સાથે સમજાવો અને તેના ટેકનીકલ સ્પેસીફીકેશનો લખો.}
\begin{solutionbox}
    \textbf{ફોટોકોપીયર બ્લોક ડાયાગ્રામ:}

    \begin{figure}[H]
        \centering
        \begin{tikzpicture}[gtu flow]
            \node (Main) [gtu block] {મેઇન કંટ્રોલ};
            \node (Scan) [gtu block, left=of Main] {સ્કેનર/ઓપ્ટિક્સ};
            \node (Image) [gtu block, right=of Main] {ઇમેજિંગ સિસ્ટમ};
            \node (Paper) [gtu block, below=of Main] {પેપર ફીડ};
            
            \node (Mirror) [gtu block, below=of Scan] {મિરર્સ/લેન્સ};
            \node (Drum) [gtu block, below=of Image] {ફોટોસેન્સિટિવ ડ્રમ};
            
            \node (Charge) [gtu block, right=of Drum] {ચાર્જિંગ};
            \node (Dev) [gtu block, below=of Drum] {ડેવેલપર};
            \node (Trans) [gtu block, left=of Drum] {ટ્રાન્સફર};
            \node (Fuse) [gtu block, below=of Trans] {ફ્યુઝર};
            
            \draw [gtu arrow] (Main) -- (Scan);
            \draw [gtu arrow] (Main) -- (Image);
            \draw [gtu arrow] (Main) -- (Paper);
            
            \draw [gtu arrow] (Scan) -- (Mirror);
            \draw [gtu arrow] (Mirror) -- (Drum);
            
            \draw [gtu arrow] (Charge) -- (Drum);
            \draw [gtu arrow] (Dev) -- (Drum);
            \draw [gtu arrow] (Drum) -- (Trans);
            \draw [gtu arrow] (Paper) -| (Trans);
            \draw [gtu arrow] (Trans) -- (Fuse);
        \end{tikzpicture}
        \caption{\foreignlanguage{english}{Photocopier System}}
    \end{figure}

    \textbf{ફોટોકોપીયરની કાર્યપદ્ધતિ:}
    \begin{enumerate}
        \item \textbf{ચાર્જિંગ}: ફોટોસેન્સિટિવ ડ્રમને યુનિફોર્મ ઇલેક્ટ્રોસ્ટેટિક ચાર્જ આપવામાં આવે છે
        \item \textbf{એક્સપોઝર}: ઓરિજિનલ ડોક્યુમેન્ટ સ્કેન થાય છે, ડ્રમ પર પ્રકાશ પેટર્ન બનાવે છે
        \item \textbf{ડેવેલપિંગ}: ટોનર કણો ડ્રમ પર ચાર્જ કરેલા ક્ષેત્રો તરફ આકર્ષાય છે
        \item \textbf{ટ્રાન્સફર}: ટોનર ઇમેજ ડ્રમ પરથી પેપર પર ટ્રાન્સફર થાય છે
        \item \textbf{ફ્યુઝિંગ}: હીટ અને પ્રેશરથી ટોનર કાયમી રીતે પેપર પર ફિક્સ થાય છે
        \item \textbf{ક્લીનિંગ}: આગલા સાયકલ માટે ડ્રમ સાફ કરવામાં આવે છે
    \end{enumerate}

    \textbf{ટેકનિકલ સ્પેસિફિકેશન્સ:}
    \begin{itemize}
        \item \textbf{સ્પીડ}: 20-60 પેજ પ્રતિ મિનિટ
        \item \textbf{રિઝોલ્યુશન}: 600-1200 dpi
        \item \textbf{પેપર કેપેસિટી}: 250-2000 શીટ્સ
        \item \textbf{મેક્સિમમ પેપર સાઇઝ}: A3/11$\times$17 ઇંચ
        \item \textbf{ઝૂમ રેન્જ}: 25-400\%
        \item \textbf{મેમરી}: 512MB-2GB
        \item \textbf{કનેક્ટિવિટી}: Ethernet, USB, Wi-Fi
    \end{itemize}

    \begin{mnemonicbox}
        \mnemonic{CETFC: ચાર્જ ડ્રમ, એક્સપોઝ ઇમેજ, ટ્રાન્સફર ટોનર, ફ્યુઝ પર્મેનન્ટલી, ક્લીન ડ્રમ}
    \end{mnemonicbox}
\end{solutionbox}

\questionmarks{4(a) OR}{3}{CCTV ઉપર ટૂંક નોંધ લખો.}
\begin{solutionbox}
    \textbf{Table: CCTV સિસ્ટમ ઓવરવ્યુ} \\
    \begin{tabulary}{\linewidth}{|l|L|}
        \hline
        \textbf{પાસું} & \textbf{વર્ણન} \\
        \hline
        \textbf{ફુલ ફોર્મ} & ક્લોઝ્ડ-સર્કિટ ટેલિવિઝન \\
        \hline
        \textbf{હેતુ} & સિક્યુરિટી મોનિટરિંગ અને સર્વેલન્સ \\
        \hline
        \textbf{કોમ્પોનન્ટ્સ} & કેમેરા, DVR/NVR, મોનિટર્સ, કેબલ્સ, પાવર સપ્લાય \\
        \hline
        \textbf{પ્રકારો} & એનાલોગ, IP (ડિજિટલ), વાયરલેસ, HD-CVI/TVI/SDI \\
        \hline
        \textbf{ફીચર્સ} & મોશન ડિટેક્શન, નાઇટ વિઝન, રિમોટ વ્યુઇંગ \\
        \hline
    \end{tabulary}

    \textbf{કી એપ્લિકેશન્સ:}
    \begin{itemize}
        \item બિલ્ડિંગ્સનું સિક્યુરિટી મોનિટરિંગ
        \item ટ્રાફિક મોનિટરિંગ
        \item રિટેલ લોસ પ્રિવેન્શન
        \item પબ્લિક એરિયા સર્વેલન્સ
        \item હોમ સિક્યુરિટી
    \end{itemize}

    \begin{mnemonicbox}
        \mnemonic{SCRAM: સિક્યુરિટી મોનિટરિંગ, ક્લોઝ્ડ સર્કિટ, રેકોર્ડિંગ ફુટેજ, એક્સેસ રેસ્ટ્રિક્ટેડ, મોનિટરિંગ કન્ટિન્યુઅસ}
    \end{mnemonicbox}
\end{solutionbox}

\questionmarks{4(b) OR}{4}{એલસીડી પ્રોજેક્ટર ની કાર્યપધ્ધતિ બ્લોક ડાયેગ્રામ સાથે સમજાવો}
\begin{solutionbox}
    \textbf{LCD પ્રોજેક્ટર બ્લોક ડાયાગ્રામ:}

    \begin{figure}[H]
        \centering
        \begin{tikzpicture}[gtu flow]
            \node (Lamp) [gtu block] {લેમ્પ};
            \node (Mirror) [gtu block, right=of Lamp] {ડિક્રોઇક મિરર્સ};
            
            \node (R) [gtu block, right=of Mirror, fill=red!20] {Red LCD};
            \node (G) [gtu block, above=of Mirror, fill=green!20] {Green LCD};
            \node (B) [gtu block, below=of Mirror, fill=blue!20] {Blue LCD};
            
            \node (Prism) [gtu block, right=3cm of Mirror] {પ્રિઝમ};
            \node (Lens) [gtu block, right=of Prism] {લેન્સ};
            \node (Screen) [gtu block, right=of Lens] {સ્ક્રીન};
            
            \draw [gtu arrow] (Lamp) -- (Mirror);
            \draw [gtu arrow] (Mirror) -- (R);
            \draw [gtu arrow] (Mirror) -- (G);
            \draw [gtu arrow] (Mirror) -- (B);
            
            \draw [gtu arrow] (R) -- (Prism);
            \draw [gtu arrow] (G) -- (Prism);
            \draw [gtu arrow] (B) -- (Prism);
            
            \draw [gtu arrow] (Prism) -- (Lens);
            \draw [gtu arrow] (Lens) -- (Screen);
            
            \node (Control) [gtu block, below=of Prism] {કંટ્રોલ સર્કિટ};
            \draw [gtu arrow] (Control) -- (R);
            \draw [gtu arrow] (Control) -- (G);
            \draw [gtu arrow] (Control) -- (B);
        \end{tikzpicture}
        \caption{\foreignlanguage{english}{LCD Projector}}
    \end{figure}

    \textbf{LCD પ્રોજેક્ટરની કાર્યપદ્ધતિ:}
    \begin{enumerate}
        \item \textbf{લાઇટ જનરેશન}: હાઇ-ઇન્ટેન્સિટી લેમ્પ સફેદ પ્રકાશ ઉત્પન્ન કરે છે
        \item \textbf{કલર સેપરેશન}: ડિક્રોઇક મિરર્સ પ્રકાશને RGB કોમ્પોનન્ટ્સમાં વિભાજિત કરે છે
        \item \textbf{ઇમેજ ફોર્મેશન}: LCD પેનલ્સ ઇનપુટ સિગ્નલના આધારે પ્રકાશને મોડ્યુલેટ કરે છે
        \item \textbf{રિકમ્બિનેશન}: પ્રિઝમ RGB ઇમેજને ફુલ-કલર ઇમેજમાં જોડે છે
        \item \textbf{પ્રોજેક્શન}: લેન્સ સિસ્ટમ અંતિમ ઇમેજને સ્ક્રીન પર પ્રોજેક્ટ કરે છે
    \end{enumerate}

    \begin{mnemonicbox}
        \mnemonic{LSCIP: લાઇટ સોર્સ જનરેટ્સ, સ્પ્લિટ ઇન્ટુ કલર્સ, કંટ્રોલ વિથ LCDs, ઇમેજ કંબાઇન્ડ, પ્રોજેક્ટેડ ઓન સ્ક્રીન}
    \end{mnemonicbox}
\end{solutionbox}

\questionmarks{4(c) OR}{7}{લેસર પ્રિંટર ની કાર્યપધ્ધતિ બ્લોક ડાયેગ્રામ સાથે સમજાવો}
\begin{solutionbox}
    \textbf{લેસર પ્રિંટર બ્લોક ડાયાગ્રામ:}

    \begin{figure}[H]
        \centering
        \begin{tikzpicture}[gtu flow]
            \node (Control) [gtu block] {કંટ્રોલ બોર્ડ};
            
            \node (Laser) [gtu block, right=of Control] {લેસર ડાયોડ};
            \node (Mirror) [gtu block, right=of Laser] {પોલીગોન મિરર};
            \node (Drum) [gtu block, below=of Mirror] {ફોટોસેન્સિટિવ ડ્રમ};
            
            \node (Charge) [gtu block, right=of Drum] {ચાર્જિંગ};
            \node (Dev) [gtu block, below=of Drum] {ડેવેલપર};
            \node (Trans) [gtu block, left=of Drum] {ટ્રાન્સફર};
            \node (Fuse) [gtu block, below=of Trans] {ફ્યુઝર};
            
            \draw [gtu arrow] (Control) -- (Laser);
            \draw [gtu arrow] (Laser) -- (Mirror);
            \draw [gtu arrow] (Mirror) -- (Drum);
            
            \draw [gtu arrow] (Charge) -- (Drum);
            \draw [gtu arrow] (Dev) -- (Drum);
            \draw [gtu arrow] (Drum) -- (Trans);
            \draw [gtu arrow] (Trans) -- (Fuse);
            
            \node (Paper) [gtu block, left=of Trans] {પેપર પાથ};
            \draw [gtu arrow] (Paper) -- (Trans);
        \end{tikzpicture}
        \caption{\foreignlanguage{english}{Laser Printer}}
    \end{figure}

    \textbf{લેસર પ્રિન્ટિંગ પ્રોસેસ:}
    \textbf{Table: લેસર પ્રિન્ટિંગના છ સ્ટેપ્સ} \\
    \begin{tabulary}{\linewidth}{|c|L|L|L|}
         \hline
        \textbf{સ્ટેપ} & \textbf{પ્રક્રિયા} & \textbf{કોમ્પોનન્ટ} & \textbf{ફંક્શન} \\
        \hline
        1 & \textbf{ક્લીનિંગ} & ક્લીનિંગ બ્લેડ & ડ્રમ પરથી બાકી ટોનર દૂર કરે છે \\
        \hline
        2 & \textbf{ચાર્જિંગ} & પ્રાઇમરી કોરોના & ડ્રમને યુનિફોર્મ નેગેટિવ ચાર્જ આપે છે \\
        \hline
        3 & \textbf{રાઇટિંગ} & લેસર અને મિરર & ડ્રમ પર ઇલેક્ટ્રોસ્ટેટિક ઇમેજ બનાવે છે \\
        \hline
        4 & \textbf{ડેવેલપિંગ} & ડેવેલપર યુનિટ & ડ્રમના ચાર્જ કરેલા ક્ષેત્રોમાં ટોનર લગાવે છે \\
        \hline
        5 & \textbf{ટ્રાન્સફરિંગ} & ટ્રાન્સફર કોરોના & ડ્રમથી પેપર પર ટોનર ખસેડે છે \\
        \hline
        6 & \textbf{ફ્યુઝિંગ} & ફ્યુઝર યુનિટ & ટોનરને કાયમી રીતે પેપર પર પિગળાવે છે \\
        \hline
    \end{tabulary}

    \textbf{ટેકનિકલ સ્પેસિફિકેશન્સ:}
    \begin{itemize}
        \item \textbf{પ્રિન્ટ સ્પીડ}: 20-50 ppm
        \item \textbf{રિઝોલ્યુશન}: 600-2400 dpi
        \item \textbf{મેમરી}: 128MB-1GB
        \item \textbf{ડ્યુટી સાયકલ}: 10,000-150,000 પેજ/મહિનો
        \item \textbf{કનેક્ટિવિટી}: USB, ઇથરનેટ, Wi-Fi
    \end{itemize}

    \begin{mnemonicbox}
        \mnemonic{CCWDTF: ક્લીન ડ્રમ, ચાર્જ યુનિફોર્મલી, રાઇટ વિથ લેસર, ડેવેલપ વિથ ટોનર, ટ્રાન્સફર ટુ પેપર, ફ્યુઝ પર્મેનન્ટલી}
    \end{mnemonicbox}
\end{solutionbox}

\questionmarks{5(a)}{3}{વ્યાખ્યા આપો: (૧) પીચ (૨) રીવબર્રેશન (3) માઇક્રોફોન}
\begin{solutionbox}
    \textbf{Table: ઓડિઓ ટર્મિનોલોજી} \\
    \begin{tabulary}{\linewidth}{|l|L|l|}
        \hline
        \textbf{પદ} & \textbf{વ્યાખ્યા} & \textbf{માપન એકમ} \\
        \hline
        \textbf{પીચ} & ધ્વનિની અનુભવાતી આવૃત્તિ; ટોન કેટલો ઊંચો કે નીચો લાગે છે & હર્ટ્ઝ (Hz) \\
        \hline
        \textbf{રીવબર્રેશન} & સ્ત્રોત બંધ થયા પછી ધ્વનિનું સાતત્ય; પરાવર્તનને કારણે થાય છે & સેકન્ડ (RT60) \\
        \hline
        \textbf{માઇક્રોફોન} & ટ્રાન્સડ્યુસર જે ધ્વનિ તરંગોને ઇલેક્ટ્રિકલ સિગ્નલમાં રૂપાંતરિત કરે છે & સેન્સિટિવિટી dB/mV/Pa માં \\
        \hline
    \end{tabulary}

    \begin{mnemonicbox}
        \mnemonic{PRM: પીચ એટલે ફ્રિક્વન્સી, રીવબર્રેશન એટલે રિફ્લેક્શન, માઇક્રોફોન એટલે કન્વર્ટર}
    \end{mnemonicbox}
\end{solutionbox}

\questionmarks{5(b)}{4}{પીએ સિસ્ટમનો બ્લોક ડાયેગ્રામ દોરો અને સમજાવો}
\begin{solutionbox}
    \textbf{PA સિસ્ટમ બ્લોક ડાયાગ્રામ:}

    \begin{figure}[H]
        \centering
        \begin{tikzpicture}[gtu flow]
            \node (Mic) [gtu block] {માઇક્રોફોન};
            \node (Pre) [gtu block, right=of Mic] {પ્રી-એમ્પ્લિફાયર};
            \node (Mixer) [gtu block, right=of Pre] {મિક્સર};
            \node (Source) [gtu block, above=of Mixer] {ઓડિઓ સોર્સ};
            \node (EQ) [gtu block, right=of Mixer] {ઇક્વલાઇઝર};
            \node (Power) [gtu block, right=of EQ] {પાવર એમ્પ્લિફાયર};
            \node (Speaker) [gtu block, right=of Power] {સ્પીકર સિસ્ટમ};
            \node (Control) [gtu block, below=of Mixer] {કંટ્રોલ સિસ્ટમ};
            
            \draw [gtu arrow] (Mic) -- (Pre);
            \draw [gtu arrow] (Pre) -- (Mixer);
            \draw [gtu arrow] (Source) -- (Mixer);
            \draw [gtu arrow] (Mixer) -- (EQ);
            \draw [gtu arrow] (EQ) -- (Power);
            \draw [gtu arrow] (Power) -- (Speaker);
            \draw [gtu arrow] (Control) -| (Power);
            \draw [gtu arrow] (Control) -- (Mixer);
        \end{tikzpicture}
        \caption{\foreignlanguage{english}{Public Address System}}
    \end{figure}

    \textbf{Table: PA સિસ્ટમ કોમ્પોનન્ટ્સ} \\
    \begin{tabulary}{\linewidth}{|l|L|}
        \hline
        \textbf{કોમ્પોનન્ટ} & \textbf{કાર્ય} \\
        \hline
        \textbf{માઇક્રોફોન} & અવાજ મેળવે છે અને ઇલેક્ટ્રિકલ સિગ્નલમાં રૂપાંતરિત કરે છે \\
        \hline
        \textbf{પ્રી-એમ્પ્લિફાયર} & નબળા માઇક્રોફોન સિગ્નલને લાઇન લેવલ સુધી વધારે છે \\
        \hline
        \textbf{મિક્સર} & અનેક ઓડિઓ સોર્સ ભેગા કરે છે, લેવલ એડજસ્ટ કરે છે \\
        \hline
        \textbf{ઇક્વલાઇઝર} & શ્રેષ્ઠ અવાજ માટે ફ્રિક્વન્સી રિસ્પોન્સ એડજસ્ટ કરે છે \\
        \hline
        \textbf{પાવર એમ્પ્લિફાયર} & સ્પીકર્સ ડ્રાઇવ કરવા માટે સિગ્નલ સ્ટ્રેન્થ વધારે છે \\
        \hline
        \textbf{સ્પીકર સિસ્ટમ} & ઇલેક્ટ્રિકલ સિગ્નલને ફરીથી ધ્વનિ તરંગોમાં રૂપાંતરિત કરે છે \\
        \hline
    \end{tabulary}

    \begin{mnemonicbox}
        \mnemonic{MPMEPA: માઇક્રોફોન પિક્સ, પ્રીએમ્પ મેગ્નિફાઇઝ, ઇક્વલાઇઝર એડજસ્ટ્સ, પાવર એમ્પ્લિફાયર ડ્રાઇવ્સ, ઓડિયન્સ હિયર્સ}
    \end{mnemonicbox}
\end{solutionbox}

\questionmarks{5(c)}{7}{ક્રિસ્ટલ માઇક્રોફોન સમજાવો.}
\begin{solutionbox}
    \textbf{Table: ક્રિસ્ટલ માઇક્રોફોન લાક્ષણિકતાઓ} \\
    \begin{tabulary}{\linewidth}{|l|L|}
        \hline
        \textbf{લાક્ષણિકતા} & \textbf{વર્ણન} \\
        \hline
        \textbf{ઓપરેટિંગ સિદ્ધાંત} & પીઝોઇલેક્ટ્રિક અસર \\
        \hline
        \textbf{રચના} & મેટલ પ્લેટો વચ્ચે ક્રિસ્ટલ એલિમેન્ટ (રોશેલ સોલ્ટ) \\
        \hline
        \textbf{રિસ્પોન્સ} & ઉચ્ચ આઉટપુટ, મધ્યમ ફ્રીક્વન્સી રિસ્પોન્સ \\
        \hline
        \textbf{ઇમ્પીડન્સ} & ખૂબ વધારે (સામાન્ય રીતે $>$ 1 M$\Omega$) \\
        \hline
        \textbf{ટકાઉપણું} & ગરમી અને ભેજ માટે સંવેદનશીલ \\
        \hline
    \end{tabulary}

    \textbf{વર્કિંગ પ્રિન્સિપલ:}
    જ્યારે ધ્વનિ તરંગો ડાયાફ્રેમ પર અથડાય છે, ત્યારે તેઓ ક્રિસ્ટલ તત્વ પર દબાણ બનાવે છે. પીઝોઇલેક્ટ્રિક અસરને કારણે, ક્રિસ્ટલ યાંત્રિક તાણના પ્રમાણમાં વોલ્ટેજ ઉત્પન્ન કરે છે. આ વોલ્ટેજ અવાજનું વિદ્યુત પ્રતિનિધિત્વ છે.

    \begin{figure}[H]
        \centering
        \begin{tikzpicture}[gtu flow]
            \node (Sound) [gtu block] {ધ્વનિ તરંગો};
            \node (Diaphragm) [gtu block, right=of Sound] {ડાયાફ્રેમ};
            \node (Stress) [gtu block, right=of Diaphragm] {યાંત્રિક દબાણ};
            \node (Piezo) [gtu block, below=of Stress] {પીઝો ઇફેક્ટ};
            \node (Voltage) [gtu block, left=of Piezo] {વોલ્ટેજ જનરેશન};
            \node (Output) [gtu block, left=of Voltage] {ઇલેક્ટ્રિકલ આઉટપુટ};
            
            \draw [gtu arrow] (Sound) -- (Diaphragm);
            \draw [gtu arrow] (Diaphragm) -- (Stress);
            \draw [gtu arrow] (Stress) -- (Piezo);
            \draw [gtu arrow] (Piezo) -- (Voltage);
            \draw [gtu arrow] (Voltage) -- (Output);
        \end{tikzpicture}
        \caption{\foreignlanguage{english}{Crystal Microphone Working}}
    \end{figure}

    \textbf{એપ્લિકેશન્સ:}
    \begin{itemize}
        \item ટેલિફોન રિસિવર્સ
        \item એકોસ્ટિક ઇન્સ્ટ્રુમેન્ટ્સ માટે કોન્ટેક્ટ પિકઅપ્સ
        \item ઓછા ખર્ચે રેકોર્ડિંગ ઉપકરણો
        \item પબ્લિક એડ્રેસ સિસ્ટમ્સ
    \end{itemize}

    \textbf{ફાયદા અને મર્યાદાઓ:}
    \textbf{Table: Pros and Cons} \\
    \begin{tabulary}{\linewidth}{|L|L|}
        \hline
        \textbf{ફાયદા} & \textbf{મર્યાદાઓ} \\
        \hline
        ઉચ્ચ આઉટપુટ વોલ્ટેજ & ખરાબ ફ્રીક્વન્સી રિસ્પોન્સ \\
        \hline
        બાહ્ય પાવરની જરૂર નથી & તાપમાન/ભેજ માટે સંવેદનશીલ \\
        \hline
        સરળ બાંધકામ & વધુ ડિસ્ટોર્શન \\
        \hline
        ઓછી કિંમત & નાજુક ક્રિસ્ટલ તત્વ \\
        \hline
    \end{tabulary}

    \begin{mnemonicbox}
        \mnemonic{PIES: પ્રેશર એપ્લાઇડ, ઇમ્પીડન્સ હાઇ, ઇલેક્ટ્રિસિટી જનરેટેડ, સાઉન્ડ કન્વર્ટેડ}
    \end{mnemonicbox}
\end{solutionbox}

\questionmarks{5(a) OR}{3}{હોમ થિયેટર સાઉન્ડ સિસ્ટમનો બ્લોક ડાયાગ્રામ દોરો.}
\begin{solutionbox}
    \textbf{હોમ થિયેટર સાઉન્ડ સિસ્ટમ બ્લોક ડાયાગ્રામ:}

    \begin{figure}[H]
        \centering
        \begin{tikzpicture}[gtu flow]
            \node (AV) [gtu block] {AV રિસિવર};
            \node (Source) [gtu block, above=of AV] {સોર્સ};
            
            \node (Center) [gtu block, below=1cm of AV] {સેન્ટર};
            \node (FL) [gtu block, left=of Center] {ફ્રન્ટ લેફ્ટ};
            \node (FR) [gtu block, right=of Center] {ફ્રન્ટ રાઇટ};
            
            \node (Sub) [gtu block, below=of Center] {સબવૂફર};
            \node (SL) [gtu block, left=of Sub] {સરાઉન્ડ લેફ્ટ};
            \node (SR) [gtu block, right=of Sub] {સરાઉન્ડ રાઇટ};
            
            \draw [gtu arrow] (Source) -- (AV);
            \draw [gtu arrow] (AV) -- (Center);
            \draw [gtu arrow] (AV) -- (FL);
            \draw [gtu arrow] (AV) -- (FR);
            \draw [gtu arrow] (AV) -- (Sub);
            \draw [gtu arrow] (AV) -- (SL);
            \draw [gtu arrow] (AV) -- (SR);
        \end{tikzpicture}
        \caption{\foreignlanguage{english}{5.1 Home Theatre System}}
    \end{figure}

    \begin{mnemonicbox}
        \mnemonic{SAVS: સોર્સ પ્રોવાઇડ્સ, એમ્પ્લિફાયર પ્રોસેસિસ, વેરિયસ સ્પીકર્સ ડિલીવર, સરાઉન્ડ એક્સપિરિયન્સ ક્રિએટેડ}
    \end{mnemonicbox}
\end{solutionbox}

\questionmarks{5(b) OR}{4}{ઓપ્ટિકલ સાઉન્ડ રેકોર્ડિંગ સમજાવો.}
\begin{solutionbox}
    \textbf{Table: ઓપ્ટિકલ સાઉન્ડ રેકોર્ડિંગ પ્રક્રિયા} \\
    \begin{tabulary}{\linewidth}{|c|l|L|}
        \hline
        \textbf{સ્ટેપ} & \textbf{પ્રક્રિયા} & \textbf{કોમ્પોનન્ટ} \\
        \hline
        1 & \textbf{સાઉન્ડ કેપ્ચર} & માઇક્રોફોન અવાજને ઇલેક્ટ્રિકલ સિગ્નલમાં પરિવર્તિત કરે છે \\
        \hline
        2 & \textbf{મોડ્યુલેશન} & સિગ્નલ પ્રકાશ સ્રોતની તીવ્રતા અથવા વિસ્તારને મોડ્યુલેટ કરે છે \\
        \hline
        3 & \textbf{એક્સપોઝર} & મોડ્યુલેટેડ લાઇટ ફોટોગ્રાફિક ફિલ્મને એક્સપોઝ કરે છે \\
        \hline
        4 & \textbf{ડેવેલપમેન્ટ} & દૃશ્યમાન સાઉન્ડ ટ્રેક બનાવવા માટે ફિલ્મ પ્રોસેસ થાય છે \\
        \hline
        5 & \textbf{પ્લેબેક} & પ્રકાશ ટ્રેકમાંથી પસાર થાય છે, ફોટોડિટેક્ટર ઇલેક્ટ્રિકલ સિગ્નલમાં પરિવર્તિત કરે છે \\
        \hline
    \end{tabulary}

    \textbf{ઓપ્ટિકલ સાઉન્ડ ટ્રેક્સના પ્રકારો:}
    \begin{itemize}
        \item \textbf{વેરિયેબલ ડેન્સિટી}: પ્રકાશની તીવ્રતા બદલાય છે (ઘેરા/આછા વિસ્તારો)
        \item \textbf{વેરિયેબલ એરિયા}: અપારદર્શક પૃષ્ઠભૂમિ સામે પારદર્શક વિસ્તારની પહોળાઈ બદલાય છે
    \end{itemize}

    \begin{figure}[H]
        \centering
        \begin{tikzpicture}[gtu flow]
            \node (Sound) [gtu block] {સાઉન્ડ ઇનપુટ};
            \node (Mic) [gtu block, right=of Sound] {માઇક્રોફોન};
            \node (Amp) [gtu block, right=of Mic] {એમ્પ્લિફાયર};
            \node (Mod) [gtu block, right=of Amp] {લાઇટ મોડ્યુલેટર};
            \node (Source) [gtu block, above=of Mod] {લાઇટ સોર્સ};
            
            \node (Optics) [gtu block, below=of Mod] {ઓપ્ટિકલ સિસ્ટમ};
            \node (Film) [gtu block, left=of Optics] {ફિલ્મ};
            
            \draw [gtu arrow] (Sound) -- (Mic);
            \draw [gtu arrow] (Mic) -- (Amp);
            \draw [gtu arrow] (Amp) -- (Mod);
            \draw [gtu arrow] (Source) -- (Mod);
            \draw [gtu arrow] (Mod) -- (Optics);
            \draw [gtu arrow] (Optics) -- (Film);
        \end{tikzpicture}
        \caption{\foreignlanguage{english}{Optical Recording}}
    \end{figure}

    \begin{mnemonicbox}
        \mnemonic{CAREP: કેપ્ચર સાઉન્ડ, એમ્પ્લિફાય સિગ્નલ, રેકોર્ડ ઓપ્ટિકલી, એક્સપોઝ ફિલ્મ, પ્લે બેક}
    \end{mnemonicbox}
\end{solutionbox}

\questionmarks{5(c) OR}{7}{લાઉડસ્પીકર વ્યાખ્યાયિત કરો. લાઉડસ્પીકરના પ્રકારોની યાદી આપો અને કોઈપણ એક પ્રકારના લાઉડસ્પીકરનું કાર્ય સમજાવો.}
\begin{solutionbox}
    \textbf{વ્યાખ્યા:}
    લાઉડસ્પીકર એક ઇલેક્ટ્રોએકોસ્ટિક ટ્રાન્સડ્યુસર છે જે ડાયાફ્રેમને ખસેડીને ઇલેક્ટ્રિકલ સિગ્નલોને ધ્વનિ તરંગોમાં પરિવર્તિત કરે છે, જે હવાના દબાણમાં ફેરફાર કરે છે.

    \textbf{Table: લાઉડસ્પીકરના પ્રકારો} \\
    \begin{tabulary}{\linewidth}{|l|L|l|L|}
        \hline
        \textbf{પ્રકાર} & \textbf{કાર્યકારી સિદ્ધાંત} & \textbf{ફ્રિક્વન્સી રેન્જ} & \textbf{એપ્લિકેશન્સ} \\
        \hline
        \textbf{ડાયનેમિક/મુવિંગ કોઇલ} & ઇલેક્ટ્રોમેગ્નેટિક ઇન્ડક્શન & 20Hz-20kHz & સૌથી સામાન્ય, સામાન્ય હેતુ \\
        \hline
        \textbf{ઇલેક્ટ્રોસ્ટેટિક} & પ્લેટો વચ્ચે ઇલેક્ટ્રોસ્ટેટિક ફોર્સ & 100Hz-20kHz & હાઇ-ફિડેલિટી ઓડિઓ સિસ્ટમ્સ \\
        \hline
        \textbf{પીઝોઇલેક્ટ્રિક} & પીઝોઇલેક્ટ્રિક અસર & 1kHz-25kHz & ટ્વીટર્સ, એલાર્મ્સ, બઝર્સ \\
        \hline
        \textbf{રિબન} & ચુંબકીય ક્ષેત્રમાં રિબન દ્વારા પ્રવાહ & 2kHz-50kHz & ઉચ્ચ ફ્રીક્વન્સી રિપ્રોડક્શન \\
        \hline
        \textbf{પ્લેનર મેગ્નેટિક} & કંડક્ટર શીટ પર મેગ્નેટિક ફોર્સ & 30Hz-20kHz & ઓડિઓફાઇલ હેડફોન્સ, સ્પીકર્સ \\
        \hline
    \end{tabulary}

    \textbf{ડાયનેમિક/મુવિંગ કોઇલ લાઉડસ્પીકરનું કાર્ય:}

    \begin{figure}[H]
        \centering
        \begin{tikzpicture}[gtu flow]
            \node (Signal) [gtu block] {ઓડિઓ સિગ્નલ};
            \node (Coil) [gtu block, right=of Signal] {વોઇસ કોઇલ};
            \node (Field) [gtu block, right=of Coil] {EM ક્ષેત્ર};
            \node (Magnet) [gtu block, above=of Field] {કાયમી ચુંબક};
            
            \node (Move) [gtu block, below=of Field] {કોઇલ હલનચલન};
            \node (Cone) [gtu block, left=of Move] {કોન હલનચલન};
            \node (Sound) [gtu block, left=of Cone] {ધ્વનિ તરંગો};
            
            \draw [gtu arrow] (Signal) -- (Coil);
            \draw [gtu arrow] (Coil) -- (Field);
            \draw [gtu arrow] (Magnet) -- (Field);
            \draw [gtu arrow] (Field) -- (Move);
            \draw [gtu arrow] (Move) -- (Cone);
            \draw [gtu arrow] (Cone) -- (Sound);
        \end{tikzpicture}
        \caption{\foreignlanguage{english}{Dynamic Loudspeaker working}}
    \end{figure}

    \textbf{વર્કિંગ પ્રક્રિયા:}
    \begin{enumerate}
        \item ઓડિઓ પ્રવાહ વોઇસ કોઇલમાંથી વહે છે
        \item પ્રવાહ ઇલેક્ટ્રોમેગ્નેટિક ક્ષેત્ર બનાવે છે
        \item ઇલેક્ટ્રોમેગ્નેટિક ક્ષેત્ર કાયમી ચુંબક સાથે ક્રિયાપ્રતિક્રિયા કરે છે
        \item સિગ્નલ પોલેરિટીના આધારે વોઇસ કોઇલ આગળ/પાછળ ખસે છે
        \item જોડાયેલ કોન/ડાયાફ્રેમ ખસે છે, જે હવાના દબાણમાં ફેરફાર કરે છે
        \item હવાના દબાણમાં ફેરફાર ધ્વનિ તરંગો તરીકે ફેલાય છે
    \end{enumerate}

    \textbf{કોમ્પોનન્ટ્સ:}
    \begin{itemize}
        \item \textbf{કોન/ડાયાફ્રેમ}: અવાજ બનાવવા માટે હવાને ખસેડે છે
        \item \textbf{વોઇસ કોઇલ}: ઓડિઓ સિગ્નલ પ્રવાહ વહન કરે છે
        \item \textbf{ચુંબક}: સ્થિર ચુંબકીય ક્ષેત્ર બનાવે છે
        \item \textbf{સસ્પેન્શન}: કોનને કેન્દ્રમાં રાખે છે, હલનચલન કરવાની મંજૂરી આપે છે
        \item \textbf{ફ્રેમ/બાસ્કેટ}: કોમ્પોનન્ટ્સને યોગ્ય ગોઠવણીમાં રાખે છે
    \end{itemize}

    \begin{mnemonicbox}
        \mnemonic{SEPVADICS: સિગ્નલ એન્ટર્સ, પ્રોડ્યુસિસ વાઇબ્રેશન્સ, એક્ટિવેટ્સ ડાયાફ્રેમ, ઇન કોઓર્ડિનેશન વિથ સસ્પેન્શન}
    \end{mnemonicbox}
\end{solutionbox}

\end{document}

