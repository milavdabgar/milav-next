\documentclass{article}

% content/resources/templates/preamble.tex
\usepackage[margin=0.6in]{geometry}
\author{Milav Dabgar}
\usepackage{amsmath,amssymb,amsthm}
\usepackage{booktabs}
\usepackage{multirow}
\usepackage{xcolor}
\usepackage{tcolorbox}
\tcbuselibrary{breakable,skins}
\usepackage[colorlinks=true,linkcolor=blue]{hyperref}
\usepackage{titlesec}
\usepackage{enumitem}
\usepackage{tikz}
\usepackage{pgfplots}
\usepackage{circuitikz}
\usepackage[version=4]{mhchem}
\usepackage{longtable}
\usepackage{array}
\usepackage{float}
\usepackage{caption}
\usepackage{listings}

\lstset{
  basicstyle=\small\ttfamily,
  breaklines=true,
  breakatwhitespace=false,
  postbreak=\mbox{\textcolor{red}{$\hookrightarrow$}\space},
  float=false,
  numbers=left,
  numberstyle=\tiny\color{gray},
  numbersep=10pt,
  xleftmargin=2em,
  keywordstyle=\color{blue},
  commentstyle=\color{green!60!black},
  stringstyle=\color{purple},
  backgroundcolor=\color{gray!5},
  showstringspaces=false,
  tabsize=2,
  captionpos=b,
  keepspaces=true,
  columns=flexible
}

\pgfplotsset{compat=1.18}
\usetikzlibrary{shapes,arrows,positioning,calc,patterns,decorations.pathmorphing,decorations.markings,arrows.meta}

% Color scheme
\definecolor{headcolor}{RGB}{0,102,204}
\definecolor{keycolor}{RGB}{220,20,60}
\definecolor{solutioncolor}{RGB}{34,139,34}
\definecolor{mnemoniccolor}{RGB}{148,0,211}
\definecolor{codecolor}{RGB}{0,0,100}

% Spacing
\setlength{\parskip}{3pt}
\setlist[itemize]{nosep}
\setlist[enumerate]{nosep}

% Title formatting
\titleformat{\section}{\Large\bfseries\color{headcolor}}{\thesection}{1em}{}
\titleformat{\subsection}{\large\bfseries\color{headcolor}}{\thesubsection}{1em}{}

% Pandoc tightlist compatibility
\providecommand{\tightlist}{%
  \setlength{\itemsep}{0pt}\setlength{\parskip}{0pt}}

% Pandoc longtable compatibility
\newcounter{none}
\def\thenone{}


% content/resources/templates/english-boxes.tex

% Custom environments
\newtcolorbox{solutionbox}{
 breakable,
 enhanced,
 colback=solutioncolor!5!white,
 colframe=solutioncolor!75!black,
 fonttitle=\bfseries,
 title=Solution
}

\newtcolorbox{solutionboxnobreak}{
 colback=solutioncolor!5!white,
 colframe=solutioncolor!75!black,
 fonttitle=\bfseries,
 title=Solution
}

\newtcolorbox{keyformula}{
 breakable,
 enhanced,
 colback=keycolor!5!white,
 colframe=keycolor!75!black,
 fonttitle=\bfseries,
 title=Key Formula
}

\newtcolorbox{mnemonicboxenv}{
 breakable,
 enhanced,
 colback=mnemoniccolor!5!white,
 colframe=mnemoniccolor!75!black,
 fonttitle=\bfseries,
 title=Mnemonic
}

\newcommand{\mnemonicbox}[1]{%
  \begin{mnemonicboxenv}
    #1
  \end{mnemonicboxenv}
}


% Custom commands for GTU solutions
% This file defines semantic commands for consistent formatting

% Question command with automatic formatting
\newcommand{\question}[2]{%
  \section*{Question #1}%
  \textbf{#2}%
}

% OR question variant
\newcommand{\questionor}[2]{%
  \section*{Question #1 OR}%
  \textbf{#2}%
}

% Proper table environment with caption
\newenvironment{answertable}[1]{%
  \begin{table}[htbp]
  \centering
  \caption{#1}
}{%
  \end{table}
}

% Proper figure environment for diagrams
\newenvironment{answerdiagram}[1]{%
  \begin{figure}[htbp]
  \centering
  \caption{#1}
}{%
  \end{figure}
}

% Semantic markup for key terms
\newcommand{\keyword}[1]{\textbf{#1}}
\newcommand{\code}[1]{\texttt{#1}}
\newcommand{\classname}[1]{\texttt{#1}}
\newcommand{\methodname}[1]{\texttt{#1}}

% Proper quotation marks
\newcommand{\mnemonic}[1]{``#1''}


\title{Consumer Electronics \& Maintenance (4341107) - Winter 2023 Solution}
\date{January 29, 2024}

\begin{document}
\maketitle

\questionmarks{1(a)}{3}{Explain different types of maintenance in brief.}
\begin{solutionbox}
    \begin{tabulary}{\linewidth}{|l|L|}
        \hline
        \textbf{Type of Maintenance} & \textbf{Description} \\
        \hline
        \textbf{Preventive Maintenance} & Scheduled regular inspection and servicing to prevent breakdowns \\
        \hline
        \textbf{Corrective Maintenance} & Repairs performed after equipment failure to restore functionality \\
        \hline
        \textbf{Predictive Maintenance} & Uses condition monitoring to predict when maintenance will be needed \\
        \hline
    \end{tabulary}

    \begin{mnemonicbox}
        \mnemonic{PCPro: Preventive prevents, Corrective cures, Predictive predicts}
    \end{mnemonicbox}
\end{solutionbox}

\questionmarks{1(b)}{4}{Explain maintenance procedure of Washing Machine.}
\begin{solutionbox}
    \textbf{Maintenance Procedure:}
    
    \begin{figure}[H]
        \centering
        \begin{tikzpicture}[gtu flow]
            \node (Insp) [gtu block] {Regular Inspection};
            \node (Filter) [gtu block, right=of Insp] {Clean Filter};
            \node (Hose) [gtu block, right=of Filter] {Check Hoses};
            \node (Load) [gtu block, below=of Filter] {Balance Load};
            \node (Drum) [gtu block, left=of Load] {Clean Drum};

            \draw [gtu arrow] (Insp) -- (Filter);
            \draw [gtu arrow] (Filter) -- (Hose);
            \draw [gtu arrow] (Hose) -- (Load);
            \draw [gtu arrow] (Load) -- (Drum);
        \end{tikzpicture}
        \caption{Washing Machine Maintenance Steps}
    \end{figure}

    \begin{itemize}
        \item \textbf{Filter Cleaning}: Remove and clean lint filter monthly
        \item \textbf{Hose Inspection}: Check for cracks and leaks every 3 months
        \item \textbf{Load Distribution}: Ensure proper balancing to prevent vibration
        \item \textbf{Drum Cleaning}: Run empty hot water cycle with vinegar quarterly
    \end{itemize}

    \begin{mnemonicbox}
        \mnemonic{FHLD: Filters, Hoses, Loads, Drum need regular attention}
    \end{mnemonicbox}
\end{solutionbox}

\questionmarks{1(c)}{7}{Explain maintenance and troubleshooting procedure of Microwave Oven.}
\begin{solutionbox}
    \textbf{Maintenance Procedures:}
    \begin{tabulary}{\linewidth}{|l|L|l|}
        \hline
        \textbf{Task} & \textbf{Procedure} & \textbf{Frequency} \\
        \hline
        External Cleaning & Wipe with mild detergent & Weekly \\
        \hline
        Internal Cleaning & Clean food particles and grease & After each spill \\
        \hline
        Door Seal Check & Inspect for damage or leakage & Monthly \\
        \hline
        Ventilation Check & Ensure vents are unobstructed & Monthly \\
        \hline
    \end{tabulary}

    \textbf{Troubleshooting:}
    
    \begin{figure}[H]
        \centering
        \begin{tikzpicture}[gtu flow]
            \node (Power) [gtu decision] {No Power?};
            \node (ChkPwr) [gtu block, right=of Power] {Check Connection};
            \node (Heat) [gtu decision, below=2cm of Power] {Not Heating?};
            \node (ChkHeat) [gtu block, right=of Heat] {Check Magnetron};
            \node (Uneven) [gtu decision, below=2cm of Heat] {Uneven Cooking?};
            \node (ChkTurn) [gtu block, right=of Uneven] {Check Turntable};

            \draw [gtu arrow] (Power) -- node[above] {Yes} (ChkPwr);
            \draw [gtu arrow] (Heat) -- node[above] {Yes} (ChkHeat);
            \draw [gtu arrow] (Uneven) -- node[above] {Yes} (ChkTurn);
        \end{tikzpicture}
        \caption{Microwave Troubleshooting Flow}
    \end{figure}

    \begin{itemize}
        \item \textbf{Power Issues}: Check fuse, circuit breaker, and cord
        \item \textbf{Heating Problems}: Test door switch, high voltage capacitor, magnetron
        \item \textbf{Safety First}: Never operate with damaged door or seals
    \end{itemize}

    \begin{mnemonicbox}
        \mnemonic{POWER: Power, Oven interior, Wiring, Electronics, Radiation seals}
    \end{mnemonicbox}
\end{solutionbox}

\questionmarks{1(c) OR}{7}{Explain maintenance and troubleshooting procedure of projector.}
\begin{solutionbox}
    \textbf{Maintenance Procedures:}
    \begin{tabulary}{\linewidth}{|l|L|l|}
        \hline
        \textbf{Task} & \textbf{Procedure} & \textbf{Frequency} \\
        \hline
        Lens Cleaning & Use lens cloth and solution & Monthly \\
        \hline
        Filter Cleaning & Remove and clean dust & Every 100 hours \\
        \hline
        Lamp Inspection & Check for discoloration/dimming & Every 300 hours \\
        \hline
        Ventilation & Ensure proper airflow & Before each use \\
        \hline
    \end{tabulary}

    \textbf{Troubleshooting:}
    \begin{itemize}
        \item \textbf{Image Issues}: Adjust focus, resolution, keystone correction
        \item \textbf{Lamp Problems}: Check lamp hours, replace if exceeding limit
        \item \textbf{Connectivity}: Verify input source and cable connections
        \item \textbf{Thermal Issues}: Clean filters and ensure proper ventilation
    \end{itemize}

    \begin{mnemonicbox}
        \mnemonic{FLAMVE: Filters, Lamp, Airflow, Mounting, Voltage, Environment}
    \end{mnemonicbox}
\end{solutionbox}

\questionmarks{2(a)}{3}{Explain the terms in brief: (1) Hue (2) Brightness}
\begin{solutionbox}
    \begin{tabulary}{\linewidth}{|l|L|}
        \hline
        \textbf{Term} & \textbf{Description} \\
        \hline
        \textbf{Hue} & The pure color attribute that distinguishes colors (red, green, blue, etc.) based on light wavelength \\
        \hline
        \textbf{Brightness} & The amount of light emitted or reflected from a color, determining how light or dark it appears \\
        \hline
    \end{tabulary}

    \begin{figure}[H]
        \centering
        \begin{tikzpicture}[gtu flow]
            \draw[<->, thick] (-2,0) -- (2,0) node[right] {Brightness (Lightness)};
            \draw[<->, thick] (0,-2) -- (0,2) node[above] {Hue (Color Type)};
            \node at (0,0) {Color Space};
        \end{tikzpicture}
        \caption{Hue vs Brightness}
    \end{figure}

    \begin{mnemonicbox}
        \mnemonic{HB-WC: Hue determines What Color, Brightness determines White-to-black level}
    \end{mnemonicbox}
\end{solutionbox}


\questionmarks{2(b)}{4}{Write a short note on LCD TV.}
\begin{solutionbox}
    \textbf{LCD TV Technology:}
    \begin{itemize}
        \item \textbf{Working Principle}: Uses liquid crystals that twist/untwist to allow/block light provided by a backend backlight source.
        \item \textbf{Key Components}: Backlight, polarizing filters, liquid crystal matrix, color filters.
        \item \textbf{Advantages}: Thin profile, energy efficient, no radiation, sharp image.
        \item \textbf{Limitations}: Limited viewing angle, slower response time than newer technologies (OLED).
    \end{itemize}

    \begin{figure}[H]
        \centering
        \begin{tikzpicture}[gtu flow]
            \node (Backlight) [gtu block] {Backlight};
            \node (Pol1) [gtu block, right=of Backlight] {Polarizing Filter};
            \node (LC) [gtu block, right=of Pol1] {Liquid Crystal};
            \node (Color) [gtu block, right=of LC] {Color Filter};
            \node (Screen) [gtu block, right=of Color] {Screen};

            \draw [gtu arrow] (Backlight) -- (Pol1);
            \draw [gtu arrow] (Pol1) -- (LC);
            \draw [gtu arrow] (LC) -- (Color);
            \draw [gtu arrow] (Color) -- (Screen);
        \end{tikzpicture}
        \caption{LCD TV Layers}
    \end{figure}

    \begin{mnemonicbox}
        \mnemonic{BPLCS: Backlight Passes Light through Crystals to Screen}
    \end{mnemonicbox}
\end{solutionbox}

\questionmarks{2(c)}{7}{Draw and explain block diagram of DTH receiver.}
\begin{solutionbox}
    \textbf{DTH Receiver Block Diagram:}
    
    \begin{figure}[H]
        \centering
        \begin{tikzpicture}[gtu flow]
            \node (Dish) [gtu block] {Satellite Dish};
            \node (LNB) [gtu block, right=of Dish] {LNB};
            \node (Tuner) [gtu block, right=of LNB] {Tuner};
            \node (Demod) [gtu block, below=of Tuner] {Demodulator};
            \node (MPEG) [gtu block, left=of Demod] {MPEG Decoder};
            \node (Proc) [gtu block, left=of MPEG] {Audio/Video Proc};
            \node (TV) [gtu block, left=of Proc] {TV Display};

            \draw [gtu arrow] (Dish) -- (LNB);
            \draw [gtu arrow] (LNB) -- (Tuner);
            \draw [gtu arrow] (Tuner) -- (Demod);
            \draw [gtu arrow] (Demod) -- (MPEG);
            \draw [gtu arrow] (MPEG) -- (Proc);
            \draw [gtu arrow] (Proc) -- (TV);
        \end{tikzpicture}
        \caption{DTH Receiver System}
    \end{figure}

    \begin{itemize}
        \item \textbf{Satellite Dish}: Captures signals from satellite.
        \item \textbf{LNB (Low Noise Block)}: Converts high frequency signals to lower frequency.
        \item \textbf{Tuner}: Selects specific channel frequency.
        \item \textbf{Demodulator}: Extracts digital information from carrier signal.
        \item \textbf{MPEG Decoder}: Decompresses video/audio data.
        \item \textbf{Conditional Access Module}: Controls subscription access.
        \item \textbf{Microcontroller}: Controls overall operation and user inputs.
    \end{itemize}

    \begin{mnemonicbox}
        \mnemonic{SLTDMP: Satellite, LNB, Tuner, Demodulator, MPEG, Processor}
    \end{mnemonicbox}
\end{solutionbox}

\questionmarks{2(a) OR}{3}{Explain the terms in brief: (1) Luminance (2) chrominance}
\begin{solutionbox}
    \begin{tabulary}{\linewidth}{|l|L|}
        \hline
        \textbf{Term} & \textbf{Description} \\
        \hline
        \textbf{Luminance} & The brightness or intensity component of a video signal (Y) that carries black and white information. \\
        \hline
        \textbf{Chrominance} & The color component of a video signal (Cb, Cr) that carries hue and saturation information. \\
        \hline
    \end{tabulary}

    \begin{mnemonicbox}
        \mnemonic{LC-BH: Luminance controls Brightness, Chrominance controls Hue}
    \end{mnemonicbox}
\end{solutionbox}

\questionmarks{2(b) OR}{4}{Explain Grassman's law.}
\begin{solutionbox}
    \textbf{Grassman's Laws of Color Mixing:}
    \begin{tabulary}{\linewidth}{|l|L|}
        \hline
        \textbf{Law} & \textbf{Description} \\
        \hline
        \textbf{Symmetry} & If color A matches color B, then B matches A \\
        \hline
        \textbf{Proportionality} & If A matches B, then nA matches nB (for any intensity n) \\
        \hline
        \textbf{Additivity} & If A matches B and C matches D, then A+C matches B+D \\
        \hline
    \end{tabulary}
    
    \begin{itemize}
        \item Forms the basis of RGB color model in displays as it applies to additive light mixing.
        \item Allows creating any color by mixing three primary colors properly.
    \end{itemize}

    \begin{mnemonicbox}
        \mnemonic{SPA Color: Symmetry, Proportionality, Additivity laws for Color matching}
    \end{mnemonicbox}
\end{solutionbox}

\questionmarks{2(c) OR}{7}{Draw and explain block diagram of colour TV receiver.}
\begin{solutionbox}
    \textbf{Block Diagram:}
    
    \begin{figure}[H]
        \centering
        \begin{tikzpicture}[gtu flow]
            \node (Ant) [gtu block] {Antenna};
            \node (Tuner) [gtu block, right=of Ant] {Tuner};
            \node (IF) [gtu block, right=of Tuner] {IF Amp};
            \node (VidDet) [gtu block, below=of IF] {Video Detect};
            \node (VidAmp) [gtu block, left=of VidDet] {Video Amp};
            \node (Color) [gtu block, left=of VidAmp] {Color Proc};
            \node (RGB) [gtu block, below=of Color] {RGB Matrix};
            \node (CRT) [gtu block, right=of RGB] {Picture Tube};

            \draw [gtu arrow] (Ant) -- (Tuner);
            \draw [gtu arrow] (Tuner) -- (IF);
            \draw [gtu arrow] (IF) -- (VidDet);
            \draw [gtu arrow] (VidDet) -- (VidAmp);
            \draw [gtu arrow] (VidAmp) -- (Color);
            \draw [gtu arrow] (Color) -- (RGB);
            \draw [gtu arrow] (RGB) -- (CRT);
            
            % Sound path
            \node (SoundIF) [gtu block, below=of VidDet] {Sound IF};
            \node (Spk) [gtu block, left=of SoundIF] {Speaker};
            \draw [gtu arrow] (VidDet) -- (SoundIF);
            \draw [gtu arrow] (SoundIF) -- (Spk);

        \end{tikzpicture}
        \caption{Color TV Receiver}
    \end{figure}

    \begin{itemize}
        \item \textbf{Tuner}: Selects desired channel frequency.
        \item \textbf{IF Amplifier}: Amplifies intermediate frequency signals.
        \item \textbf{Video Detector}: Extracts video and audio information.
        \item \textbf{Color Processor}: Separates luminance and chrominance.
        \item \textbf{RGB Matrix}: Converts color signals to red, green, blue drivers.
        \item \textbf{Deflection Circuits}: Control electron beam scanning (H-sync, V-sync).
    \end{itemize}
\end{solutionbox}

\questionmarks{3(a)}{3}{State main components of solar power system and specifications of solar power system.}
\begin{solutionbox}
    \textbf{Main Components:}
    \begin{tabulary}{\linewidth}{|l|L|}
        \hline
        \textbf{Component} & \textbf{Function} \\
        \hline
        \textbf{Solar Panels} & Convert sunlight to electricity \\
        \hline
        \textbf{Charge Controller} & Regulates battery charging \\
        \hline
        \textbf{Battery Bank} & Stores electrical energy \\
        \hline
        \textbf{Inverter} & Converts DC to AC electricity \\
        \hline
    \end{tabulary}

    \textbf{Specifications:}
    \begin{itemize}
        \item \textbf{Panel Rating}: 100-400W per panel
        \item \textbf{Battery Capacity}: 100-200Ah
        \item \textbf{Inverter Rating}: 500-5000W
        \item \textbf{System Voltage}: 12/24/48V
    \end{itemize}

    \begin{mnemonicbox}
        \mnemonic{SCBIM: Solar panels, Controller, Battery, Inverter, Mounting}
    \end{mnemonicbox}
\end{solutionbox}

\questionmarks{3(b)}{4}{List types, applications and technical specifications of microwave oven.}
\begin{solutionbox}
    \begin{tabulary}{\linewidth}{|l|L|}
        \hline
        \textbf{Type} & \textbf{Features} \\
        \hline
        \textbf{Solo} & Basic heating and defrosting only \\
        \hline
        \textbf{Grill} & Additional grilling element \\
        \hline
        \textbf{Convection} & Has heating element and fan for baking \\
        \hline
        \textbf{Combination} & Integrates microwave, grill and convection \\
        \hline
    \end{tabulary}

    \textbf{Applications}: Food reheating, Defrosting, Cooking, Baking.
    
    \textbf{Specs}: Power (700-1200W), Capacity (20-40L), Frequency (2.45 GHz).
\end{solutionbox}

\questionmarks{3(c)}{7}{Explain working of Air conditioner and Refrigerator}
\begin{solutionbox}
    \textbf{Working Principle:}
    
    \begin{figure}[H]
        \centering
        \begin{tikzpicture}[gtu flow]
            \node (Comp) [gtu block] {Compressor};
            \node (Cond) [gtu block, right=of Comp] {Condenser};
            \node (Exp) [gtu block, below=of Cond] {Expansion Valve};
            \node (Evap) [gtu block, left=of Exp] {Evaporator};

            \draw [gtu arrow] (Comp) -- node[above, font=\tiny] {Hot Gas} (Cond);
            \draw [gtu arrow] (Cond) -- node[right, font=\tiny] {Liquid} (Exp);
            \draw [gtu arrow] (Exp) -- node[below, font=\tiny] {Low Pres} (Evap);
            \draw [gtu arrow] (Evap) -- node[left, font=\tiny] {Gas} (Comp);
        \end{tikzpicture}
        \caption{Refrigeration Cycle (AC/Fridge)}
    \end{figure}

    \textbf{Cycle Components:}
    \begin{itemize}
        \item \textbf{Compressor}: Pressurizes refrigerant gas.
        \item \textbf{Condenser}: Releases heat, converts gas to liquid.
        \item \textbf{Expansion Valve}: Reduces pressure/temperature.
        \item \textbf{Evaporator}: Absorbs heat from room/box, converts liquid to gas.
    \end{itemize}

    \textbf{Differences:} AC cools room (18-26$^{\circ}$C), Fridge cools cabinet (2-8$^{\circ}$C).
\end{solutionbox}

\questionmarks{3(a) OR}{3}{List technical specifications of Air conditioner and Refrigerator}
\begin{solutionbox}
    \begin{tabulary}{\linewidth}{|l|L|L|}
        \hline
        \textbf{Spec} & \textbf{Air Conditioner} & \textbf{Refrigerator} \\
        \hline
        \textbf{Capacity} & 1-2 ton (12k-24k BTU) & 100-500 liters \\
        \hline
        \textbf{Power} & 1000-2500 watts & 100-400 watts \\
        \hline
        \textbf{Efficiency} & ISEER/Star Rating 3-5 & BEE Star Rating 3-5 \\
        \hline
        \textbf{Gas} & R32, R410A & R600a, R134a \\
        \hline
    \end{tabulary}
\end{solutionbox}

\questionmarks{3(b) OR}{4}{Explain electronic controller for washing machine.}
\begin{solutionbox}
    \begin{figure}[H]
        \centering
        \begin{tikzpicture}[gtu flow]
            \node (Micro) [gtu block] {Microcontroller};
            \node (UI) [gtu block, above=of Micro] {User Interface};
            \node (Sensors) [gtu block, left=of Micro] {Sensors (Temp/Level)};
            \node (Motor) [gtu block, right=of Micro] {Motor/Valves};
            
            \draw [gtu arrow] (UI) -- (Micro);
            \draw [gtu arrow] (Sensors) -- (Micro);
            \draw [gtu arrow] (Micro) -- (Motor);
        \end{tikzpicture}
        \caption{Electronic Control System}
    \end{figure}

    \begin{itemize}
        \item \textbf{Microcontroller}: Central CPU managing operations.
        \item \textbf{Sensors}: Water level, temperature, load balance.
        \item \textbf{Actuators}: Motor driver, water valves, drain pump.
    \end{itemize}

    \begin{mnemonicbox}
        \mnemonic{MIST-WAD: Microcontroller Integrates Sensors and Timers for Water, Agitation and Drainage}
    \end{mnemonicbox}
\end{solutionbox}

\questionmarks{3(c) OR}{7}{Draw and explain block diagram of Microwave oven. List wiring and safety instructions}
\begin{solutionbox}
    \textbf{Block Diagram:}
    
    \begin{figure}[H]
        \centering
        \begin{tikzpicture}[gtu flow]
            \node (Ctrl) [gtu block] {Control Unit};
            \node (HVTrans) [gtu block, right=of Ctrl] {HV Transformer};
            \node (HVCap) [gtu block, right=of HVTrans] {HV Capacitor};
            \node (Mag) [gtu block, below=of HVCap] {Magnetron};
            \node (Cavity) [gtu block, left=of Mag] {Cooking Cavity};
            
            \draw [gtu arrow] (Ctrl) -- (HVTrans);
            \draw [gtu arrow] (HVTrans) -- (HVCap);
            \draw [gtu arrow] (HVCap) -- (Mag);
            \draw [gtu arrow] (Mag) -- (Cavity);
        \end{tikzpicture}
        \caption{Microwave Internal System}
    \end{figure}

    \begin{itemize}
        \item \textbf{Magnetron}: Generates microwaves (2.45 GHz).
        \item \textbf{HV Transformer}: Steps up voltage to 2-4kV.
        \item \textbf{Safety}: Never operate with open door; ensure grounding; don't override interlocks.
        \item \textbf{Wiring}: Use 15-20A dedicated circuit with proper ground.
    \end{itemize}
\end{solutionbox}

\questionmarks{4(a)}{3}{Draw block diagram of Photocopier.}
\begin{solutionbox}
    \begin{figure}[H]
        \centering
        \begin{tikzpicture}[gtu flow]
            \node (Scan) [gtu block] {Scanner};
            \node (Img) [gtu block, right=of Scan] {Image Proc};
            \node (Laser) [gtu block, right=of Img] {Laser Unit};
            \node (Drum) [gtu block, below=of Laser] {Drum};
            \node (Dev) [gtu block, left=of Drum] {Developer};
            \node (Trans) [gtu block, left=of Dev] {Transfer};
            \node (Fuse) [gtu block, left=of Trans] {Fuser};

            \draw [gtu arrow] (Scan) -- (Img);
            \draw [gtu arrow] (Img) -- (Laser);
            \draw [gtu arrow] (Laser) -- (Drum);
            \draw [gtu arrow] (Drum) -- (Dev);
            \draw [gtu arrow] (Dev) -- (Trans);
            \draw [gtu arrow] (Trans) -- (Fuse);
        \end{tikzpicture}
        \caption{Photocopier Process}
    \end{figure}
\end{solutionbox}

\questionmarks{4(b)}{4}{List specifications of MF printer and CCTV.}
\begin{solutionbox}
    \begin{tabulary}{\linewidth}{|l|L|}
        \hline
        \textbf{MF Printer} & \textbf{CCTV} \\
        \hline
        Res: 600-1200 dpi & Res: 2-8 MP \\
        \hline
        Speed: 15-40 ppm & FPS: 15-30 fps \\
        \hline
        Scan: 300-600 dpi & Night Vision: 10-30m \\
        \hline
        Conn: USB, WiFi & Storage: 1-8 TB \\
        \hline
    \end{tabulary}
\end{solutionbox}

\questionmarks{4(c)}{7}{Explain working of laser printer with block diagram.}
\begin{solutionbox}
    \begin{figure}[H]
        \centering
        \begin{tikzpicture}[gtu flow]
            \node (Data) [gtu block] {Data};
            \node (Laser) [gtu block, right=of Data] {Laser};
            \node (Drum) [gtu block, right=of Laser] {Drum};
            \node (Dev) [gtu block, below=of Drum] {Developer};
            \node (Trans) [gtu block, left=of Dev] {Transfer};
            \node (Fuse) [gtu block, left=of Trans] {Fuser/Output};

            \draw [gtu arrow] (Data) -- (Laser);
            \draw [gtu arrow] (Laser) -- (Drum);
            \draw [gtu arrow] (Drum) -- (Dev);
            \draw [gtu arrow] (Dev) -- (Trans);
            \draw [gtu arrow] (Trans) -- (Fuse);
        \end{tikzpicture}
        \caption{Laser Printer Cycle}
    \end{figure}

    \textbf{Process Stages:}
    \begin{enumerate}
        \item \textbf{Charging}: Drum gets uniform charge.
        \item \textbf{Writing}: Laser discharges image areas.
        \item \textbf{Developing}: Toner sticks to discharged areas.
        \item \textbf{Transfer}: Toner moves to paper.
        \item \textbf{Fusing}: Heat melts toner onto paper.
        \item \textbf{Cleaning}: Residual toner removed.
    \end{enumerate}

    \begin{mnemonicbox}
        \mnemonic{CWTFC: Charge, Write, Transfer, Fuse, Clean cycle}
    \end{mnemonicbox}
\end{solutionbox}

\questionmarks{4(a) OR}{3}{Draw block diagram of CCTV.}
\begin{solutionbox}
    \begin{figure}[H]
        \centering
        \begin{tikzpicture}[gtu flow]
            \node (Cam) [gtu block] {Camera};
            \node (DVR) [gtu block, right=of Cam] {DVR/NVR};
            \node (HDD) [gtu block, right=of DVR] {HDD Storage};
            \node (Mon) [gtu block, below=of DVR] {Monitor};
            
            \draw [gtu arrow] (Cam) -- (DVR);
            \draw [gtu arrow] (DVR) -- (HDD);
            \draw [gtu arrow] (DVR) -- (Mon);
        \end{tikzpicture}
        \caption{Basic CCTV System}
    \end{figure}
\end{solutionbox}

\questionmarks{4(b) OR}{4}{List specifications of inkjet printer and Photocopier.}
\begin{solutionbox}
    \begin{tabulary}{\linewidth}{|l|L|}
        \hline
        \textbf{Inkjet Printer} & \textbf{Photocopier} \\
        \hline
        Res: 1200-4800 dpi & Res: 600-1200 dpi \\
        \hline
        Speed: 8-20 ppm & Speed: 20-60 cpm \\
        \hline
        Ink: Dye/Pigment & Toner: Dry Powder \\
        \hline
        Duty: 1-5k pages/mo & Duty: 10k-100k pg/mo \\
        \hline
    \end{tabulary}
\end{solutionbox}

\questionmarks{4(c) OR}{7}{Explain working of LCD projector with block diagram and list its specifications.}
\begin{solutionbox}
    \textbf{Working Process:}
    \begin{figure}[H]
        \centering
        \begin{tikzpicture}[gtu flow]
            \node (Lamp) [gtu block] {Lamp};
            \node (Mirror) [gtu block, right=of Lamp] {Dichroic Mirrors};
            \node (RGB) [gtu block, right=of Mirror] {RGB LCDs};
            \node (Prism) [gtu block, below=of RGB] {Prism};
            \node (Lens) [gtu block, left=of Prism] {Lens};
            \node (Screen) [gtu block, left=of Lens] {Screen};

            \draw [gtu arrow] (Lamp) -- (Mirror);
            \draw [gtu arrow] (Mirror) -- (RGB);
            \draw [gtu arrow] (RGB) -- (Prism);
            \draw [gtu arrow] (Prism) -- (Lens);
            \draw [gtu arrow] (Lens) -- (Screen);
        \end{tikzpicture}
        \caption{LCD Projector}
    \end{figure}

    \begin{itemize}
        \item \textbf{Lamp}: High intensity source.
        \item \textbf{Mirrors}: Split light into Red, Green, Blue.
        \item \textbf{LCDs}: Modulate light for each color.
        \item \textbf{Prism}: Recombines light beams.
    \end{itemize}

    \textbf{Specs}: Res (XGA/FHD), Brightness (2000-5000 Lumens), Lamp Life (3000-6000 hrs).
\end{solutionbox}

\questionmarks{5(a)}{3}{Draw block diagram of PA system.}
\begin{solutionbox}
    \begin{figure}[H]
        \centering
        \begin{tikzpicture}[gtu flow]
            \node (Mic) [gtu block] {Microphone};
            \node (PreAmp) [gtu block, right=of Mic] {Pre-Amp};
            \node (Mixer) [gtu block, right=of PreAmp] {Mixer};
            \node (EQ) [gtu block, below=of Mixer] {Equalizer};
            \node (PwrAmp) [gtu block, left=of EQ] {Power Amp};
            \node (Spk) [gtu block, left=of PwrAmp] {Speaker};
            
            \draw [gtu arrow] (Mic) -- (PreAmp);
            \draw [gtu arrow] (PreAmp) -- (Mixer);
            \draw [gtu arrow] (Mixer) -- (EQ);
            \draw [gtu arrow] (EQ) -- (PwrAmp);
            \draw [gtu arrow] (PwrAmp) -- (Spk);
            
            \node (Source) [gtu block, above=of Mixer] {Line In};
            \draw [gtu arrow] (Source) -- (Mixer);
        \end{tikzpicture}
        \caption{Public Address System}
    \end{figure}

    \begin{mnemonicbox}
        \mnemonic{MMEPS: Microphone, Mixer, Equalizer, Power amp, Speakers}
    \end{mnemonicbox}
\end{solutionbox}

\questionmarks{5(b)}{4}{Explain tweeter and woofer.}
\begin{solutionbox}
    \begin{tabulary}{\linewidth}{|l|L|L|}
        \hline
        \textbf{Feature} & \textbf{Tweeter} & \textbf{Woofer} \\
        \hline
        Frequency & High (2kHz-20kHz) & Low (20Hz-2kHz) \\
        \hline
        Size & Small (0.5"-1.5") & Large (4"-15") \\
        \hline
        Diaphragm & Light, rigid & Heavy, flexible \\
        \hline
        Role & Treble/Detail & Bass/Power \\
        \hline
    \end{tabulary}

    \begin{figure}[H]
        \centering
        \begin{tikzpicture}[gtu flow]
            \node (In) [gtu block] {Signal};
            \node (Xover) [gtu block, right=of In] {Crossover};
            \node (Tweet) [gtu block, right=of Xover, yshift=1cm] {Tweeter};
            \node (Woof) [gtu block, right=of Xover, yshift=-1cm] {Woofer};

            \draw [gtu arrow] (In) -- (Xover);
            \draw [gtu arrow] (Xover) -- node[above, sloped] {High Freq} (Tweet);
            \draw [gtu arrow] (Xover) -- node[below, sloped] {Low Freq} (Woof);
        \end{tikzpicture}
        \caption{Speaker Crossover}
    \end{figure}

    \begin{mnemonicbox}
        \mnemonic{THSL: Tweeters catch Highs (Small/Light), Woofers catch Lows}
    \end{mnemonicbox}
\end{solutionbox}

\questionmarks{5(c)}{7}{Define microphone. List types of microphone and explain working of any one type of microphone.}
\begin{solutionbox}
    \textbf{Definition:} Electroacoustic transducer converting sound waves into electrical signals.
    
    \textbf{Types:} Dynamic, Condenser, Ribbon, Carbon, Piezo, MEMS.

    \textbf{Dynamic Microphone Working:}
    
    \begin{figure}[H]
        \centering
        \begin{tikzpicture}[gtu flow]
            \node (Sound) [gtu block] {Sound Wave};
            \node (Diaph) [gtu block, right=of Sound] {Diaphragm};
            \node (Coil) [gtu block, right=of Diaph] {Voice Coil};
            \node (Mag) [gtu block, right=of Coil] {Magnetic Field};
            \node (Volt) [gtu block, right=of Mag] {Induced Voltage};

            \draw [gtu arrow] (Sound) -- (Diaph);
            \draw [gtu arrow] (Diaph) -- (Coil);
            \draw [gtu arrow] (Coil) -- (Mag);
            \draw [gtu arrow] (Mag) -- (Volt);
        \end{tikzpicture}
        \caption{Dynamic Mic Principle}
    \end{figure}

    \begin{itemize}
        \item \textbf{Sound Capture}: Sound waves hit diaphragm.
        \item \textbf{Transduction}: Coil moves in magnetic field.
        \item \textbf{Output}: Movement induces voltage (Faraday's Law).
        \item \textbf{Pros}: Rugged, no power needed, high acoustic handling.
    \end{itemize}

    \begin{mnemonicbox}
        \mnemonic{DDCMIO: Diaphragm Displaces Coil in Magnetic field Inducing Output}
    \end{mnemonicbox}
\end{solutionbox}

\questionmarks{5(a) OR}{3}{Define: (1) Pitch (2) Loudspeaker (3) Reverberation.}
\begin{solutionbox}
    \begin{itemize}
        \item \textbf{Pitch}: Perceived frequency of sound (High/Low tone).
        \item \textbf{Loudspeaker}: Transducer converting electrical signals to sound waves.
        \item \textbf{Reverberation}: Persistence of sound after source stops due to reflections.
    \end{itemize}

    \begin{figure}[H]
        \centering
        \begin{tikzpicture}[gtu flow]
            \node (Src) [gtu block] {Source};
            \node (Direct) [gtu block, right=of Src] {Direct Sound};
            \node (Early) [gtu block, right=of Direct] {Early Refl};
            \node (Late) [gtu block, right=of Early] {Reverb};

            \draw [gtu arrow] (Src) -- (Direct);
            \draw [gtu arrow] (Direct) -- (Early);
            \draw [gtu arrow] (Early) -- (Late);
        \end{tikzpicture}
        \caption{Sound Propagation}
    \end{figure}

    \begin{mnemonicbox}
        \mnemonic{PLR Sound: Pitch(Tone), Loudspeaker(Producer), Reverb(Echo)}
    \end{mnemonicbox}
\end{solutionbox}

\questionmarks{5(b) OR}{4}{Draw block diagram of Home theatre sound system and explain in brief.}
\begin{solutionbox}
    \begin{figure}[H]
        \centering
        \begin{tikzpicture}[gtu flow]
            \node (AV) [gtu block] {AV Receiver};
            \node (Center) [gtu block, right=of AV] {Center};
            \node (Front) [gtu block, above=of Center] {Front L/R};
            \node (Surr) [gtu block, below=of Center] {Surround L/R};
            \node (Sub) [gtu block, right=of Center] {Subwoofer};
            \node (Src) [gtu block, left=of AV] {Source (TV/BluRay)};

            \draw [gtu arrow] (Src) -- (AV);
            \draw [gtu arrow] (AV) -- (Center);
            \draw [gtu arrow] (AV) -- (Front);
            \draw [gtu arrow] (AV) -- (Surr);
            \draw [gtu arrow] (AV) -- (Sub);
        \end{tikzpicture}
        \caption{5.1 Home Theatre System}
    \end{figure}

    \begin{itemize}
        \item \textbf{Receiver}: Processes amplification and decoding.
        \item \textbf{Center}: Dialog clarity.
        \item \textbf{Front/Surround}: Stereo and ambient effects.
        \item \textbf{Subwoofer}: Low Frequency Effects (LFE).
    \end{itemize}
\end{solutionbox}

\questionmarks{5(c) OR}{7}{Explain Electrostatic loudspeaker and permanent magnet loudspeaker.}
\begin{solutionbox}
    \begin{tabulary}{\linewidth}{|l|L|L|}
        \hline
        \textbf{Feature} & \textbf{Electrostatic} & \textbf{Permanent Magnet} \\
        \hline
        Principle & Electrostatic force (Capacitive) & Electromagnetic induction \\
        \hline
        Parts & Stator plates, Charged film & Magnet, Voice Coil, Cone \\
        \hline
        Power & Needs HV Bias Supply & Driven by signal only \\
        \hline
        Quality & Low distortion, fast transient & Good bass, efficient \\
        \hline
    \end{tabulary}

    \textbf{Permanent Magnet Working:}
    \begin{figure}[H]
        \centering
        \begin{tikzpicture}[gtu flow]
            \node (Sig) [gtu block] {Signal};
            \node (Coil) [gtu block, right=of Sig] {Voice Coil};
            \node (Mag) [gtu block, right=of Coil] {Magnetic Field};
            \node (Force) [gtu block, below=of Mag] {Force on Cone};
            \node (Sound) [gtu block, left=of Force] {Sound Wave};

            \draw [gtu arrow] (Sig) -- (Coil);
            \draw [gtu arrow] (Coil) -- (Mag);
            \draw [gtu arrow] (Mag) -- (Force);
            \draw [gtu arrow] (Force) -- (Sound);
        \end{tikzpicture}
        \caption{Moving Coil Speaker}
    \end{figure}

    \begin{mnemonicbox}
        \mnemonic{ESPM: Electrostatic(Static Charge), Permanent Magnet(Magnetic Coil)}
    \end{mnemonicbox}
\end{solutionbox}

\end{document}
