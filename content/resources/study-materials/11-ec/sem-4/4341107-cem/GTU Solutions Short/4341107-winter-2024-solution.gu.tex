\documentclass{article}

% Imports

% content/resources/templates/preamble.tex
\usepackage[margin=0.6in]{geometry}
\author{Milav Dabgar}
\usepackage{amsmath,amssymb,amsthm}
\usepackage{booktabs}
\usepackage{multirow}
\usepackage{xcolor}
\usepackage{tcolorbox}
\tcbuselibrary{breakable,skins}
\usepackage[colorlinks=true,linkcolor=blue]{hyperref}
\usepackage{titlesec}
\usepackage{enumitem}
\usepackage{tikz}
\usepackage{pgfplots}
\usepackage{circuitikz}
\usepackage[version=4]{mhchem}
\usepackage{longtable}
\usepackage{array}
\usepackage{float}
\usepackage{caption}
\usepackage{listings}

\lstset{
  basicstyle=\small\ttfamily,
  breaklines=true,
  breakatwhitespace=false,
  postbreak=\mbox{\textcolor{red}{$\hookrightarrow$}\space},
  float=false,
  numbers=left,
  numberstyle=\tiny\color{gray},
  numbersep=10pt,
  xleftmargin=2em,
  keywordstyle=\color{blue},
  commentstyle=\color{green!60!black},
  stringstyle=\color{purple},
  backgroundcolor=\color{gray!5},
  showstringspaces=false,
  tabsize=2,
  captionpos=b,
  keepspaces=true,
  columns=flexible
}

\pgfplotsset{compat=1.18}
\usetikzlibrary{shapes,arrows,positioning,calc,patterns,decorations.pathmorphing,decorations.markings,arrows.meta}

% Color scheme
\definecolor{headcolor}{RGB}{0,102,204}
\definecolor{keycolor}{RGB}{220,20,60}
\definecolor{solutioncolor}{RGB}{34,139,34}
\definecolor{mnemoniccolor}{RGB}{148,0,211}
\definecolor{codecolor}{RGB}{0,0,100}

% Spacing
\setlength{\parskip}{3pt}
\setlist[itemize]{nosep}
\setlist[enumerate]{nosep}

% Title formatting
\titleformat{\section}{\Large\bfseries\color{headcolor}}{\thesection}{1em}{}
\titleformat{\subsection}{\large\bfseries\color{headcolor}}{\thesubsection}{1em}{}

% Pandoc tightlist compatibility
\providecommand{\tightlist}{%
  \setlength{\itemsep}{0pt}\setlength{\parskip}{0pt}}

% Pandoc longtable compatibility
\newcounter{none}
\def\thenone{}


% content/resources/templates/gujarati-boxes.tex
\usepackage{fontspec}
\usepackage{polyglossia}

% Set Gujarati as main language (document is primarily in Gujarati)
% Note: gloss-gujarati.ldf doesn't exist in polyglossia, but it will use hyphenation patterns
\setdefaultlanguage{gujarati}
\setotherlanguage{english}

% Configure Gujarati font properly
% Use Language=Default to prevent polyglossia from trying to add language-specific features
% that don't exist for Gujarati, which causes "empty feature" warnings
\newfontfamily\gujaratifont[Script=Gujarati,AutoFakeBold=2.5,AutoFakeSlant=0.3]{Noto Sans Gujarati}
\setmainfont[Script=Gujarati,AutoFakeBold=2.5,AutoFakeSlant=0.3]{Noto Sans Gujarati}
% Use Noto Sans Gujarati for monospace to support Gujarati in text
\setmonofont[Scale=0.9]{Noto Sans Gujarati}

% Configure English to use the same font
\newfontfamily\englishfont[Script=Gujarati,AutoFakeBold=2.5,AutoFakeSlant=0.3]{Noto Sans Gujarati}

% Translations for polyglossia
\gappto\captionsgujarati{
  \renewcommand{\tablename}{કોષ્ટક}
  \renewcommand{\figurename}{આકૃતિ}
}

% Helper for TikZ nodes to ensure Gujarati font
\newcommand{\gu}[1]{{\gujaratifont #1}}

% Custom environments
\newtcolorbox{solutionbox}{
    breakable,
    enhanced,
    colback=solutioncolor!5!white,
    colframe=solutioncolor!75!black,
    fonttitle=\bfseries,
    title=જવાબ
}

\newtcolorbox{solutionboxnobreak}{
 colback=solutioncolor!5!white,
 colframe=solutioncolor!75!black,
 fonttitle=\bfseries,
 title=જવાબ
}

\newtcolorbox{keyformula}{
 breakable,
 enhanced,
 colback=keycolor!5!white,
 colframe=keycolor!75!black,
 fonttitle=\bfseries,
 title=રાસાયણિક સમીકરણ/સૂત્ર
}

\newtcolorbox{mnemonicbox}{
 breakable,
 enhanced,
 colback=mnemoniccolor!5!white,
 colframe=mnemoniccolor!75!black,
 fonttitle=\bfseries,
 title=મેમરી ટ્રીક
}


% Custom commands for GTU solutions
% This file defines semantic commands for consistent formatting

% Question command with automatic formatting
\newcommand{\question}[2]{%
  \section*{Question #1}%
  \textbf{#2}%
}

% OR question variant
\newcommand{\questionor}[2]{%
  \section*{Question #1 OR}%
  \textbf{#2}%
}

% Proper table environment with caption
\newenvironment{answertable}[1]{%
  \begin{table}[htbp]
  \centering
  \caption{#1}
}{%
  \end{table}
}

% Proper figure environment for diagrams
\newenvironment{answerdiagram}[1]{%
  \begin{figure}[htbp]
  \centering
  \caption{#1}
}{%
  \end{figure}
}

% Semantic markup for key terms
\newcommand{\keyword}[1]{\textbf{#1}}
\newcommand{\code}[1]{\texttt{#1}}
\newcommand{\classname}[1]{\texttt{#1}}
\newcommand{\methodname}[1]{\texttt{#1}}

% Proper quotation marks
\newcommand{\mnemonic}[1]{``#1''}


\title{કન્ઝ્યુમર ઇલેક્ટ્રોનિક્સ એન્ડ મેઇન્ટેનન્સ (4341107) - વિન્ટર 2024 સોલ્યુશન}
\date{૧૮ એપ્રિલ, ૨૦૨૪}

\begin{document}
\maketitle
\solutiontitle


% Question 1(a)
\questionmarks{1(a)}{3}{ફક્ત વ્યાખ્યા આપો. : 1. લાઉડનેસ 2.ટીમ્બર 3. ઇકો}
\begin{solutionbox}
    \begin{tabulary}{\linewidth}{|l|L|}
        \hline
        \textbf{શબ્દ} & \textbf{વ્યાખ્યા} \\
        \hline
        \textbf{લાઉડનેસ} & અવાજની તીવ્રતાની સબજેક્ટિવ સમજ જે અવાજના દબાણ અને આવૃત્તિ પર આધારિત છે \\
        \hline
        \textbf{ટીમ્બર} & અવાજની ગુણવત્તા જે વિવિધ વાદ્ય યંત્રો અથવા અવાજને એક જ સૂર વગાડતી વખતે અલગ કરે છે \\
        \hline
        \textbf{ઇકો} & અવાજનું પરાવર્તન જે શ્રોતા પાસે સીધા અવાજ પછી 50ms કરતાં વધુ વિલંબ સાથે પહોંચે છે \\
        \hline
    \end{tabulary}

    \begin{mnemonicbox}
        \mnemonic{LTE: લાઉડનેસ શક્તિ માપે છે, ટીમ્બર વિશિષ્ટતા આપે છે, ઇકો વિલંબિત પરત આવે છે}
    \end{mnemonicbox}
\end{solutionbox}

% Question 1(b)
\questionmarks{1(b)}{4}{લાઉડસ્પીકરના પ્રકારોની યાદી બનાવો અને તેમાંથી કોઈપણ એક સમજાવો}
\begin{solutionbox}
    \textbf{લાઉડસ્પીકરના પ્રકારો:}
    \begin{itemize}
        \item ડાયનામિક/મૂવિંગ કોઇલ (ઇલેક્ટ્રોમેગ્નેટિક)
        \item ઇલેક્ટ્રોસ્ટેટિક (ચાર્જ્ડ ડાયાફ્રામ)
        \item રિબન (પાતળી ધાતુ રિબન)
        \item પિઝોઇલેક્ટ્રિક (ક્રિસ્ટલ કંપન)
        \item હોર્ન (એકોસ્ટિક એમ્પ્લિફિકેશન)
        \item પ્લેનર મેગ્નેટિક (મેગ્નેટિક સ્ટ્રિપ્સ)
    \end{itemize}

    \textbf{ડાયનામિક/મૂવિંગ કોઇલ લાઉડસ્પીકર:}
    \begin{figure}[H]
        \centering
        \begin{tikzpicture}[gtu flow]
            \node (Sig) [gtu block] {ઓડિયો સિગ્નલ};
            \node (Coil) [gtu block, right=of Sig] {વોઇસ કોઇલ};
            \node (Mag) [gtu block, right=of Coil] {EM ફિલ્ડ};
            \node (Move) [gtu block, below=of Mag] {કોઇલ મૂવમેન્ટ};
            \node (Cone) [gtu block, left=of Move] {કોન/ડાયાફ્રામ};
            \node (Sound) [gtu block, left=of Cone] {ધ્વનિ તરંગો};

            \draw [gtu arrow] (Sig) -- (Coil);
            \draw [gtu arrow] (Coil) -- (Mag);
            \draw [gtu arrow] (Mag) -- (Move);
            \draw [gtu arrow] (Move) -- (Cone);
            \draw [gtu arrow] (Cone) -- (Sound);
        \end{tikzpicture}
        \caption{ડાયનામિક લાઉડસ્પીકર કાર્ય}
    \end{figure}

    \begin{itemize}
        \item \textbf{મેગ્નેટિક સ્ટ્રક્ચર}: પર્મેનન્ટ મેગ્નેટ સ્થિર મેગ્નેટિક ફિલ્ડ બનાવે છે.
        \item \textbf{વોઇસ કોઇલ}: ઓડિયો કરંટ મેળવે છે અને બદલાતા મેગ્નેટિક ફિલ્ડ બનાવે છે.
        \item \textbf{ડાયાફ્રામ/કોન}: વોઇસ કોઇલ સાથે જોડાયેલ છે, કંપન કરીને ધ્વનિ તરંગો પેદા કરે છે.
    \end{itemize}

    \begin{mnemonicbox}
        \mnemonic{COPPER-D: કોઇલ ઓસીલેટ્સ, પર્મેનન્ટ મેગ્નેટ પુલ/પુશ કરે છે, ડાયાફ્રામ દ્વારા રેઝોનન્સ ઉત્સર્જિત થાય છે}
    \end{mnemonicbox}
\end{solutionbox}

% Question 1(c)
\questionmarks{1(c)}{7}{માઇક્રોફોનના પ્રકારોની સૂચિ બનાવો. તેની લાક્ષણિકતાઓ જણાવો અને વાયરલેસ માઇક્રોફોનને વિગતવાર સમજાવો}
\begin{solutionbox}
    \textbf{માઇક્રોફોનના પ્રકારો:}
    \begin{itemize}
        \item ડાયનામિક, કન્ડેન્સર, કાર્બન, રિબન, ક્રિસ્ટલ/પિઝોઇલેક્ટ્રિક, ઇલેક્ટ્રેટ, MEMS.
    \end{itemize}

    \textbf{લાક્ષણિકતાઓ:} સેન્સિટિવિટી, ફ્રિક્વન્સી રિસ્પોન્સ, દિશાત્મક પેટર્ન, ઇમ્પીડન્સ, સિગ્નલ-ટુ-નોઇઝ રેશિયો.

    \textbf{વાયરલેસ માઇક્રોફોન સિસ્ટમ:}
    \begin{figure}[H]
        \centering
        \begin{tikzpicture}[gtu flow]
            \node (Sound) [gtu block] {સાઉન્ડ};
            \node (Mic) [gtu block, right=of Sound] {Mic એલિમેન્ટ};
            \node (Pre) [gtu block, right=of Mic] {પ્રિ-એમ્પ};
            \node (Comp) [gtu block, right=of Pre] {કમ્પ્રેસર};
            \node (Tx) [gtu block, below=of Comp] {RF Tx};
            \node (Rx) [gtu block, left=of Tx] {RF Rx};
            \node (Demod) [gtu block, left=of Rx] {ડીમોડ્યુલેટર};
            \node (Exp) [gtu block, left=of Demod] {એક્સપેન્ડર};
            \node (Out) [gtu block, below=of Exp] {આઉટપુટ};

            \draw [gtu arrow] (Sound) -- (Mic);
            \draw [gtu arrow] (Mic) -- (Pre);
            \draw [gtu arrow] (Pre) -- (Comp);
            \draw [gtu arrow] (Comp) -- (Tx);
            \draw [gtu arrow] (Tx) -- node[above] {રેડિયો વેવ્સ} (Rx);
            \draw [gtu arrow] (Rx) -- (Demod);
            \draw [gtu arrow] (Demod) -- (Exp);
            \draw [gtu arrow] (Exp) |- (Out);
        \end{tikzpicture}
        \caption{વાયરલેસ Mic સિસ્ટમ}
    \end{figure}

    \begin{itemize}
        \item \textbf{માઇક્રોફોન એલિમેન્ટ}: ધ્વનિને ઇલેક્ટ્રિકલ સિગ્નલ્સમાં રૂપાંતરિત કરે છે.
        \item \textbf{ટ્રાન્સમિટર}: ઓડિયોને રેડિયો ફ્રિક્વન્સી કેરિયર પર મોડ્યુલેટ કરે છે.
        \item \textbf{રિસીવર}: RF સિગ્નલ કેપ્ચર કરે છે અને ઓડિયો રિકવર કરવા માટે ડીમોડ્યુલેટ કરે છે.
        \item \textbf{કમ્પેન્ડર}: નોઇઝ ઘટાડવા માટે Tx પર સિગ્નલને કમ્પ્રેસ અને Rx પર એક્સપાન્ડ કરે છે.
    \end{itemize}

    \begin{mnemonicbox}
        \mnemonic{WIRED: વાયરલેસ ઇઝ રેડિયો-એનેબલ્ડ ડિવાઇસ}
    \end{mnemonicbox}
\end{solutionbox}

% Question 1(c) OR
\questionmarks{1(c) OR}{7}{લાઉડસ્પીકર્સની લાક્ષણિકતાઓ જણાવો અને પરમેનેન્ટ મેગ્નેટ લાઉડસ્પીકરને તેના ફાયદા અને ગેરફાયદા સાથે સમજાવો.}
\begin{solutionbox}
    \textbf{લાઉડસ્પીકરની લાક્ષણિકતાઓ:}
    \begin{itemize}
        \item ફ્રિક્વન્સી રિસ્પોન્સ, સેન્સિટિવિટી, ઇમ્પીડન્સ, પાવર હેન્ડલિંગ, દિશાત્મકતા, વિકૃતિ.
    \end{itemize}

    \textbf{પર્મેનન્ટ મેગ્નેટ લાઉડસ્પીકર:}
    \begin{figure}[H]
        \centering
        \begin{tikzpicture}[gtu flow]
            \node (Sig) [gtu block] {સિગ્નલ};
            \node (Coil) [gtu block, right=of Sig] {વોઇસ કોઇલ};
            \node (Field) [gtu block, right=of Coil] {મેગ્નેટિક ફિલ્ડ};
            \node (Mag) [gtu block, right=of Field] {પર્મેનન્ટ મેગ્નેટ};
            \node (Diaph) [gtu block, below=of Field] {ડાયાફ્રામ};
            \node (Sound) [gtu block, left=of Diaph] {ધ્વનિ તરંગો};

            \draw [gtu arrow] (Sig) -- (Coil);
            \draw [gtu arrow] (Coil) <-> (Field);
            \draw [gtu arrow] (Mag) -- (Field);
            \draw [gtu arrow] (Coil) -- (Diaph);
            \draw [gtu arrow] (Diaph) -- (Sound);
        \end{tikzpicture}
        \caption{PM લાઉડસ્પીકર સિદ્ધાંત}
    \end{figure}

    \textbf{ફાયદા:} સસ્તા-અસરકારક, વિશ્વસનીય, કોમ્પેક્ટ, કાર્યક્ષમ.\\
    \textbf{ગેરફાયદા:} મર્યાદિત પાવર, મેગ્નેટ ડિટીરિયોરેશન, વજન, હીટ સેન્સિટિવિટી.

    \begin{mnemonicbox}
        \mnemonic{PMLS: પર્મેનન્ટ મેગ્નેટ જોરથી બોલે છે}
    \end{mnemonicbox}
\end{solutionbox}

% Question 2(a)
\questionmarks{2(a)}{3}{વ્યાખ્યાયિત કરો 1. આસ્પેક્ટ રેશિયો 2. ક્રોમિનેન્સ 3. એડિટિવ મિક્સિંગ}
\begin{solutionbox}
    \begin{tabulary}{\linewidth}{|l|L|}
        \hline
        \textbf{શબ્દ} & \textbf{વ્યાખ્યા} \\
        \hline
        \textbf{આસ્પેક્ટ રેશિયો} & ટેલિવિઝન અથવા ડિસ્પ્લે સ્ક્રીનની પહોળાઈનો ઊંચાઈ સાથેનો ગુણોત્તર (દા.ત., 16:9) \\
        \hline
        \textbf{ક્રોમિનેન્સ} & વિડિયો સિગ્નલમાં રંગની માહિતી, લ્યુમિનન્સ અથવા બ્રાઇટનેસથી સ્વતંત્ર \\
        \hline
        \textbf{એડિટિવ મિક્સિંગ} & વિવિધ રંગીન પ્રકાશને ભેગા કરીને નવા રંગો બનાવવાની પ્રક્રિયા, જ્યાં બધા પ્રાથમિક રંગોને મિક્સ કરવાથી સફેદ રંગ ઉત્પન્ન થાય છે \\
        \hline
    \end{tabulary}

    \begin{mnemonicbox}
        \mnemonic{ACA: આસ્પેક્ટ પરિમાણો નક્કી કરે છે, ક્રોમિનન્સ રંગ ઉમેરે છે, એડિટિવ મિક્સિંગ પ્રકાશ બનાવે છે}
    \end{mnemonicbox}
\end{solutionbox}

% Question 2(b)
\questionmarks{2(b)}{4}{ઇન્ટરલેસ સ્કેનિંગ સમજાવો}
\begin{solutionbox}
    \textbf{ખ્યાલ:} વિડિયો ફ્રેમને બે ફિલ્ડ્સ (ઓડ અને ઇવન લાઇન્સ)માં વિભાજિત કરીને બેન્ડવિડ્થ ઘટાડવી. સ્ટાન્ડર્ડ રેટ 50/60 ફિલ્ડ્સ/સેકન્ડ.

    \begin{figure}[H]
        \centering
        \begin{tikzpicture}[gtu flow]
            \node (Frame) [gtu block] {સંપૂર્ણ ફ્રેમ};
            \node (Odd) [gtu block, below left=of Frame] {ઓડ લાઇન્સ};
            \node (Even) [gtu block, below right=of Frame] {ઇવન લાઇન્સ};
            \node (F1) [gtu block, below=of Odd] {ફિલ્ડ 1};
            \node (F2) [gtu block, below=of Even] {ફિલ્ડ 2};
            \node (Disp) [gtu block, below right=of F1] {ઇન્ટરલેસ ડિસ્પ્લે};

            \draw [gtu arrow] (Frame) -- (Odd);
            \draw [gtu arrow] (Frame) -- (Even);
            \draw [gtu arrow] (Odd) -- (F1);
            \draw [gtu arrow] (Even) -- (F2);
            \draw [gtu arrow] (F1) -- (Disp);
            \draw [gtu arrow] (F2) -- (Disp);
        \end{tikzpicture}
        \caption{ઇન્ટરલેસ્ડ સ્કેનિંગ પ્રક્રિયા}
    \end{figure}

    \begin{mnemonicbox}
        \mnemonic{ODD-EVEN: એક ડિસ્પ્લે, પછી વિલંબિત વધારાની વિઝ્યુઅલ એન્હાન્સમેન્ટ નેક્સ્ટ}
    \end{mnemonicbox}
\end{solutionbox}

% Question 2(c)
\questionmarks{2(c)}{7}{LED ટેલિવિઝનના કાર્યકારી સિદ્ધાંતની ચર્ચા કરો. તેના ફાયદા જણાવો અને LCD ટેલિવિઝન સાથે તેની સરખામણી કરો.}
\begin{solutionbox}
    \textbf{કાર્યકારી સિદ્ધાંત:} LED TV એક LCD TV છે જે CCFLs ને બદલે બેકલાઇટિંગ માટે LEDs નો ઉપયોગ કરે છે.
    \begin{figure}[H]
        \centering
        \begin{tikzpicture}[gtu flow]
            \node (Sig) [gtu block] {સિગ્નલ};
            \node (Proc) [gtu block, right=of Sig] {પ્રોસેસિંગ};
            \node (LED) [gtu block, right=of Proc] {LED બેકલાઇટ};
            \node (LCD) [gtu block, below=of LED] {LCD પેનલ};
            \node (Pol) [gtu block, left=of LCD] {પોલરાઇઝર્સ};
            \node (Color) [gtu block, left=of Pol] {કલર ફિલ્ટર્સ};
            \node (Disp) [gtu block, left=of Color] {ડિસ્પ્લે};

            \draw [gtu arrow] (Sig) -- (Proc);
            \draw [gtu arrow] (Proc) -- (LCD);
            \draw [gtu arrow] (LED) -- (LCD);
            \draw [gtu arrow] (LCD) -- (Pol);
            \draw [gtu arrow] (Pol) -- (Color);
            \draw [gtu arrow] (Color) -- (Disp);
        \end{tikzpicture}
        \caption{LED TV આર્કિટેક્ચર}
    \end{figure}

    \textbf{ફાયદા:} ઉર્જા કાર્યક્ષમ, પાતળી ડિઝાઇન, સારો કોન્ટ્રાસ્ટ (લોકલ ડિમિંગ), લાંબુ આયુષ્ય, મર્ક્યુરી-ફ્રી.

    \textbf{સરખામણી (LED vs LCD):}
    \begin{tabulary}{\linewidth}{|l|L|L|}
        \hline
        \textbf{લક્ષણ} & \textbf{LED TV} & \textbf{LCD TV} \\
        \hline
        બેકલાઇટ & LEDs & CCFL ટ્યુબ્સ \\
        \hline
        જાડાઈ & પાતળું (સ્લિમ) & જાડું \\
        \hline
        પાવર & ઓછો & વધારે \\
        \hline
        કોન્ટ્રાસ્ટ & સારો & ઓછો \\
        \hline
    \end{tabulary}

    \begin{mnemonicbox}
        \mnemonic{LEDGE: લાઇટ એમિટિંગ ડાયોડ્સ ગિવ એક્સેલન્સ (Light Emitting Diodes Give Excellence)}
    \end{mnemonicbox}
\end{solutionbox}

% Question 2(a) OR
\questionmarks{2(a) OR}{3}{કલર ટેલિવિઝન સિસ્ટમના કોઈપણ છ ધોરણો જણાવો.}
\begin{solutionbox}
    \begin{itemize}
        \item \textbf{PAL} (Phase Alternating Line)
        \item \textbf{NTSC} (National Television System Committee)
        \item \textbf{SECAM} (Sequential Color with Memory)
        \item \textbf{PAL-M} (Brazil variant)
        \item \textbf{ATSC} (Digital - N. America)
        \item \textbf{DVB-T} (Digital - Europe)
        \item \textbf{ISDB} (Digital - Japan)
    \end{itemize}

    \begin{mnemonicbox}
        \mnemonic{PANS-ADI: PAL, ATSC, NTSC, SECAM - All Display Images}
    \end{mnemonicbox}
\end{solutionbox}

% Question 2(b) OR
\questionmarks{2(b) OR}{4}{LCD ટેલિવિઝનનું કામ સમજાવો.}
\begin{solutionbox}
    \begin{figure}[H]
        \centering
        \begin{tikzpicture}[gtu flow]
            \node (In) [gtu block] {સિગ્નલ};
            \node (Driver) [gtu block, right=of In] {LCD ડ્રાઇવર્સ};
            \node (Back) [gtu block, right=of Driver] {બેકલાઇટ};
            \node (Diff) [gtu block, below=of Back] {ડિફ્યુઝર};
            \node (Pol1) [gtu block, left=of Diff] {પોલરાઇઝર 1};
            \node (LCD) [gtu block, left=of Pol1] {લિક્વિડ ક્રિસ્ટલ્સ};
            \node (Pol2) [gtu block, left=of LCD] {પોલરાઇઝર 2};
            \node (RGB) [gtu block, below=of LCD] {RGB ફિલ્ટર્સ};
            \node (Scrn) [gtu block, right=of RGB] {સ્ક્રીન};

            \draw [gtu arrow] (In) -- (Driver);
            \draw [gtu arrow] (Driver) -- (LCD);
            \draw [gtu arrow] (Back) -- (Diff);
            \draw [gtu arrow] (Diff) -- (Pol1);
            \draw [gtu arrow] (Pol1) -- (LCD);
            \draw [gtu arrow] (LCD) -- (Pol2);
            \draw [gtu arrow] (Pol2) -- (RGB);
            \draw [gtu arrow] (RGB) -- (Scrn);
        \end{tikzpicture}
        \caption{LCD TV સ્ટેક}
    \end{figure}

    \textbf{કાર્ય:} બેકલાઇટ પોલરાઇઝર 1 માંથી પસાર થાય છે. લિક્વિડ ક્રિસ્ટલ્સ વોલ્ટેજ (TFT) પર આધારિત ટ્વિસ્ટ/અનટ્વિસ્ટ થાય છે જે પોલરાઇઝર 2 દ્વારા પ્રકાશને રોકવા અથવા પસાર કરવા માટે છે. પછી પ્રકાશ રંગીન પિક્સેલ્સ બનાવવા માટે RGB ફિલ્ટર્સમાંથી પસાર થાય છે.

    \begin{mnemonicbox}
        \mnemonic{BPLTC: બેકલાઇટ પાસીસ થ્રુ લિક્વિડ ક્રિસ્ટલ્સ ધેટ કલર (Backlight Passes through Liquid crystals That Color)}
    \end{mnemonicbox}
\end{solutionbox}

% Question 2(c) OR
\questionmarks{2(c) OR}{7}{PAL-D ડીકોડરનો બ્લોક ડાયાગ્રામ દોરો અને સમજાવો.}
\begin{solutionbox}
    \begin{figure}[H]
        \centering
        \begin{tikzpicture}[gtu flow]
            \node (Comp) [gtu block] {કમ્પોઝિટ વિડિયો};
            \node (Sep) [gtu block, right=of Comp] {Y/C સેપરેટર};
            \node (Y) [gtu block, right=of Sep] {Lum (Y)};
            \node (C) [gtu block, below=of Sep] {Chrom (C)};
            \node (Delay) [gtu block, below=of C] {ડીલે લાઇન (64$\mu$s)};
            \node (Switch) [gtu block, right=of Delay] {PAL સ્વિચ};
            \node (Demod) [gtu block, right=of Switch] {U/V ડિમોડ};
            \node (Mat) [gtu block, right=of Y] {RGB મેટ્રિક્સ};
            \node (Out) [gtu block, right=of Mat] {RGB આઉટ};

            \draw [gtu arrow] (Comp) -- (Sep);
            \draw [gtu arrow] (Sep) -- (Y);
            \draw [gtu arrow] (Sep) -- (C);
            \draw [gtu arrow] (C) -- (Delay);
            \draw [gtu arrow] (C) -- (Switch);
            \draw [gtu arrow] (Delay) -- (Switch);
            \draw [gtu arrow] (Switch) -- (Demod);
            \draw [gtu arrow] (Demod) -- node[right] {U,V} (Mat);
            \draw [gtu arrow] (Y) -- (Mat);
            \draw [gtu arrow] (Mat) -- (Out);
        \end{tikzpicture}
        \caption{PAL-D ડીકોડર બ્લોક ડાયાગ્રામ}
    \end{figure}

    \begin{itemize}
        \item \textbf{Y/C સેપરેટર}: બ્રાઇટનેસ (Y) અને કલર (C) ને અલગ કરે છે.
        \item \textbf{ડીલે લાઇન}: ફેઝ એરર્સને એવરેજ કરવા માટે 64$\mu$s (એક લાઇન) સિગ્નલને વિલંબિત કરે છે.
        \item \textbf{PAL સ્વિચ}: વૈકલ્પિક રેખાઓ પર V-સિગ્નલ ફેઝને ઉલટાવે છે.
        \item \textbf{U/V ડીકોડર}: કલર ડિફરન્સ સિગ્નલ્સ એક્સટ્રેક્ટ કરે છે.
        \item \textbf{RGB મેટ્રિક્સ}: Y, U, V ને જોડીને Red, Green, Blue આઉટપુટ આપે છે.
    \end{itemize}

    \begin{mnemonicbox}
        \mnemonic{PAL ડીકોડ્સ કલર રાઈટ બાય સ્વિચિંગ, ડીલેઈંગ, અનસ્ક્રામ્બલીંગ વેરિએશન્સ (PAL Decodes Color Right By Switching, Delaying, Unscrambling Variations)}
    \end{mnemonicbox}
\end{solutionbox}



% Question 3(a)
\questionmarks{3(a)}{3}{રૂફટોપ સોલાર પાવર પ્લાન્ટનું વર્ગીકરણ આપો અને તેમાંથી કોઈપણ એક પ્લાન્ટ સમજાવો.}
\begin{solutionbox}
    \textbf{વર્ગીકરણ:}
    \begin{itemize}
        \item \textbf{ગ્રિડ-કનેક્ટેડ/ઓન-ગ્રિડ}: યુટિલિટી ગ્રિડ સાથે સીધું જોડાયેલું, બેટરી વિના.
        \item \textbf{ઓફ-ગ્રિડ/સ્ટેન્ડઅલોન}: પાવર સ્ટોર કરવા માટે બેટરીનો ઉપયોગ કરે છે, ગ્રિડ સાથે જોડાયેલ નથી.
        \item \textbf{હાઇબ્રિડ}: ગ્રિડ કનેક્શન અને બેટરી બેકઅપ બંનેને જોડે છે.
    \end{itemize}

    \textbf{ગ્રિડ-કનેક્ટેડ સિસ્ટમ:}
    \begin{figure}[H]
        \centering
        \begin{tikzpicture}[gtu flow]
            \node (Panels) [gtu block] {સોલાર પેનલ્સ};
            \node (Inv) [gtu block, right=of Panels] {DC-AC ઇન્વર્ટર};
            \node (Meter) [gtu block, right=of Inv] {બાય-ડિરેક્શનલ મીટર};
            \node (Grid) [gtu block, right=of Meter] {યુટિલિટી ગ્રિડ};
            \node (Load) [gtu block, below=of Meter] {ઘરનો લોડ};

            \draw [gtu arrow] (Panels) -- (Inv);
            \draw [gtu arrow] (Inv) -- (Meter);
            \draw [gtu arrow] (Meter) -- (Grid);
            \draw [gtu arrow] (Meter) -- (Load);
        \end{tikzpicture}
        \caption{ગ્રિડ-કનેક્ટેડ રૂફટોપ સિસ્ટમ}
    \end{figure}

    \begin{itemize}
        \item સોલાર પેનલ્સ સૂર્યપ્રકાશમાંથી DC પાવર જનરેટ કરે છે.
        \item ઇન્વર્ટર DC ને ગ્રિડ સાથે સિંક્રનસ AC માં રૂપાંતરિત કરે છે.
        \item બાય-ડિરેક્શનલ મીટર આયાત (વપરાશ) અને નિકાસ (ઉત્પાદન) રેકોર્ડ કરે છે.
        \item વધારાની શક્તિ ગ્રીડને આપવામાં આવે છે (નેટ મીટરિંગ).
    \end{itemize}

    \begin{mnemonicbox}
        \mnemonic{GOH: ગ્રિડ કનેક્ટ કરે છે, ઓફ-ગ્રિડ સ્ટોર કરે છે, હાઇબ્રિડ બંને કરે છે}
    \end{mnemonicbox}
\end{solutionbox}

% Question 3(b)
\questionmarks{3(b)}{4}{રેફ્રિજરેટર અને સ્પ્લિટ એર કન્ડિશન, (દરેકના) ના ઓછામાં ઓછા ચાર ટેકનિકલ સ્પેસિફિકેશન આપો.}
\begin{solutionbox}
    \textbf{રેફ્રિજરેટર સ્પેસિફિકેશન:}
    \begin{tabulary}{\linewidth}{|l|L|}
        \hline
        \textbf{સ્પેસિફિકેશન} & \textbf{સામાન્ય રેન્જ} \\
        \hline
        Capacity & 150-750 લિટર \\
        \hline
        Power Consumption & 100-400 kWh/વર્ષ \\
        \hline
        Refrigerant & R-600a, R-134a \\
        \hline
        Compressor & રેસિપ્રોકેટિંગ અથવા ઇન્વર્ટર \\
        \hline
    \end{tabulary}

    \textbf{સ્પ્લિટ એર કન્ડિશનર સ્પેસિફિકેશન:}
    \begin{tabulary}{\linewidth}{|l|L|}
        \hline
        \textbf{સ્પેસિફિકેશન} & \textbf{સામાન્ય રેન્જ} \\
        \hline
        Cooling Capacity & 1.0 - 2.0 ટન (12000-24000 BTU) \\
        \hline
        ISEER Rating & 3.0 - 5.0 સ્ટાર \\
        \hline
        Refrigerant & R-32, R-410A \\
        \hline
        Noise Level & 30-45 dB (ઇનડોર યુનિટ) \\
        \hline
    \end{tabulary}

    \begin{mnemonicbox}
        \mnemonic{CERT: કેપેસિટી, એફિશિયન્સી, રેફ્રિજરન્ટ ટાઇપ, ટેમ્પરેચર}
    \end{mnemonicbox}
\end{solutionbox}

% Question 3(c)
\questionmarks{3(c)}{7}{માઇક્રોવેવ ઓવનને તેના કાર્યકારી સિદ્ધાંત, કાર્યકારી બ્લોક ડાયાગ્રામ અને ઓપરેટિવ સ્થિતિમાં હોય ત્યારે તેની સલામતીની સાવચેતીઓના સંદર્ભમાં સમજાવો.}
\begin{solutionbox}
    \textbf{કાર્યકારી સિદ્ધાંત:} મેગ્નેટ્રોન ઉચ્ચ-આવર્તન માઇક્રોવેવ્સ (2.45 GHz) જનરેટ કરે છે જે ખોરાકમાં પાણીના અણુઓને હલાવે છે. આ કંપન ઘર્ષણ પેદા કરે છે, જે ગરમી ઉત્પન્ન કરે છે અને ખોરાકને અંદરથી રાંધે છે.

    \begin{figure}[H]
        \centering
        \begin{tikzpicture}[gtu flow]
            \node (Panel) [gtu block] {કંટ્રોલ પેનલ};
            \node (Control) [gtu block, right=of Panel] {કંટ્રોલ યુનિટ};
            \node (Driver) [gtu block, right=of Control] {પાવર ડ્રાઇવર};
            \node (HV) [gtu block, below=of Driver] {HV ટ્રાન્સફોર્મર};
            \node (HVDiode) [gtu block, left=of HV] {HV ડાયોડ/Cap};
            \node (Mag) [gtu block, left=of HVDiode] {મેગ્નેટ્રોન};
            \node (Guide) [gtu block, left=of Mag] {વેવગાઇડ};
            \node (Cavity) [gtu block, below=of Guide] {કુકિંગ કેવિટી};
            \node (Motor) [gtu block, right=of Cavity] {ટર્નટેબલ મોટર};

            \draw [gtu arrow] (Panel) -- (Control);
            \draw [gtu arrow] (Control) -- (Driver);
            \draw [gtu arrow] (Control) -- (Motor);
            \draw [gtu arrow] (Driver) -- (HV);
            \draw [gtu arrow] (HV) -- (HVDiode);
            \draw [gtu arrow] (HVDiode) -- (Mag);
            \draw [gtu arrow] (Mag) -- (Guide);
            \draw [gtu arrow] (Guide) -- (Cavity);
            \draw [gtu arrow] (Motor) -- (Cavity);
        \end{tikzpicture}
        \caption{માઇક્રોવેવ ઓવન બ્લોક ડાયાગ્રામ}
    \end{figure}

    \textbf{સલામતી સાવચેતીઓ:}
    \begin{itemize}
        \item \textbf{ડોર ઇન્ટરલોક્સ}: ખાતરી કરો કે દરવાજો ખુલ્લો હોય તો ઓવન કામ ન કરે.
        \item \textbf{RF શિલ્ડિંગ}: દરવાજા પર મેટલ મેશ માઇક્રોવેવ લીકેજ અટકાવે છે.
        \item \textbf{કેપેસિટર ડિસ્ચાર્જ}: હાઇ વોલ્ટેજ કેપેસિટર ચાર્જ જાળવી રાખે છે; સર્વિસ દરમિયાન ડિસ્ચાર્જ કરવું જરૂરી છે.
        \item \textbf{નો મેટલ}: આર્કિંગ અટકાવવા અંદર મેટલ કન્ટેનર ન વાપરો.
        \item \textbf{ખાલી ન ચલાવો}: પરાવર્તિત તરંગોને કારણે મેગ્નેટ્રોનને નુકસાન થઈ શકે છે.
    \end{itemize}

    \begin{mnemonicbox}
        \mnemonic{MICROWAVE: મેગ્નેટ્રોન ઇનિશિએટ્સ કુકિંગ, રેડિએશન ઓન્લી વિધિન ઓથોરાઇઝ્ડ વેસલ એન્વાયરમેન્ટ}
    \end{mnemonicbox}
\end{solutionbox}

% Question 3(a) OR
\questionmarks{3(a) OR}{3}{રૂફટોપ સોલાર પાવર પ્લાન્ટમાં વપરાતા વિવિધ હાર્ડવેરનાં નામ લખો અને તેમાં વપરાતી સોલાર પેનલ સમજાવો.}
\begin{solutionbox}
    \textbf{હાર્ડવેર:} સોલાર પેનલ્સ, ઇન્વર્ટર, માઉન્ટિંગ સ્ટ્રક્ચર, બેટરીઓ (વૈકલ્પિક), ચાર્જ કંટ્રોલર, AC/DC ડિસ્ટ્રિબ્યુશન બોક્સ, કેબલ્સ.

    \textbf{સોલાર પેનલ્સ:}
    \begin{figure}[H]
        \centering
        \begin{tikzpicture}[gtu flow]
            \node (Sun) [gtu block] {સૂર્યપ્રકાશ};
            \node (Glass) [gtu block, right=of Sun] {ગ્લાસ};
            \node (EVA1) [gtu block, right=of Glass] {EVA};
            \node (Cell) [gtu block, right=of EVA1] {સિલિકોન સેલ};
            \node (EVA2) [gtu block, below=of Cell] {EVA};
            \node (Sheet) [gtu block, left=of EVA2] {બેકશીટ};
            \node (Frame) [gtu block, left=of Sheet] {Al ફ્રેમ};

            \draw [gtu arrow] (Sun) -- (Glass);
            \draw [gtu arrow] (Glass) -- (EVA1);
            \draw [gtu arrow] (EVA1) -- (Cell);
            \draw [gtu arrow] (Cell) -- (EVA2);
            \draw [gtu arrow] (EVA2) -- (Sheet);
            \draw [gtu arrow] (Sheet) -- (Frame);
        \end{tikzpicture}
        \caption{સોલાર પેનલ લેયર્સ}
    \end{figure}

    સોલાર PV પેનલ્સ ગ્લાસ અને બેકશીટ વચ્ચે એન્કેપ્સ્યુલેટેડ સેમિકન્ડક્ટર સેલ્સ (સિલિકોન) ધરાવે છે. તેઓ ફોટોવોલ્ટેઇક ઇફેક્ટ દ્વારા ફોટોન એનર્જીને DC ઇલેક્ટ્રિકલ એનર્જીમાં ફેરવે છે. પ્રકાર: મોનોક્રિસ્ટલાઇન (ઉચ્ચ કાર્યક્ષમતા), પોલીક્રિસ્ટલાઇન (ઓછી કિંમત).

    \begin{mnemonicbox}
        \mnemonic{SIMPLE: સોલાર પેનલ્સ ઇન્ટિગ્રેટ મલ્ટિપલ ફોટોવોલ્ટેઇક લેયર્સ એફિશિયન્ટલી}
    \end{mnemonicbox}
\end{solutionbox}

% Question 3(b) OR
\questionmarks{3(b) OR}{4}{માઇક્રોવેવ ઓવન અને વોશિંગ મશીનના પ્રત્યેકના ઓછામાં ઓછા ચાર ટેકનિકલ સ્પેસિફિકેશન આપો}
\begin{solutionbox}
    \textbf{માઇક્રોવેવ ઓવન:}
    \begin{itemize}
        \item Power Output: 700 - 1200 Watts
        \item Frequency: 2.45 GHz
        \item Capacity: 20 - 32 Liters
        \item Control: Digital/Touchpad/Knob
    \end{itemize}

    \textbf{વોશિંગ મશીન:}
    \begin{itemize}
        \item Capacity: 6 kg - 10 kg
        \item Spin Speed: 800 - 1400 RPM
        \item Type: Top Load / Front Load
        \item Energy Rating: 5 Star
    \end{itemize}

    \begin{mnemonicbox}
        \mnemonic{CPFWS: કેપેસિટી, પાવર, ફ્રિક્વન્સી, વોશિંગ ટેક્નોલોજી, સ્પિન સ્પીડ}
    \end{mnemonicbox}
\end{solutionbox}

% Question 3(c) OR
\questionmarks{3(c) OR}{7}{વોશિંગ મશીનનું વર્ગીકરણ આપો. ટોપ લોડ વોશિંગ મશીનની કામગીરી, કાર્યકારી બ્લોક ડાયાગ્રામ અને કામ કરવાની વ્યૂહરચના/કપડા ધોવાના પગલાંઓ સંદર્ભે સમજાવો}
\begin{solutionbox}
    \textbf{વર્ગીકરણ:} લોડિંગ દ્વારા (ટોપ/ફ્રન્ટ), ઓટોમેશન દ્વારા (સેમી/ફુલી), ટેકનોલોજી દ્વારા (એજિટેટર/ઇમ્પેલર).

    \textbf{ફંક્શનલ બ્લોક ડાયાગ્રામ (ટોપ લોડ):}
    \begin{figure}[H]
        \centering
        \begin{tikzpicture}[gtu flow]
            \node (Control) [gtu block] {કંટ્રોલર/PCB};
            \node (Inlet) [gtu block, below left=of Control] {વોટર ઇનલેટ};
            \node (Level) [gtu block, above left=of Control] {લેવલ સેન્સર};
            \node (Motor) [gtu block, below right=of Control] {AC મોટર};
            \node (Trans) [gtu block, right=of Motor] {ટ્રાન્સમિશન/ક્લચ};
            \node (Drum) [gtu block, right=of Trans] {ડ્રમ/એજિટેટર};
            \node (Pump) [gtu block, above right=of Control] {ડ્રેન પમ્પ};

            \draw [gtu arrow] (Control) -- (Inlet);
            \draw [gtu arrow] (Control) <-> (Level);
            \draw [gtu arrow] (Control) -- (Motor);
            \draw [gtu arrow] (Motor) -- (Trans);
            \draw [gtu arrow] (Trans) -- (Drum);
            \draw [gtu arrow] (Control) -- (Pump);
        \end{tikzpicture}
        \caption{વોશિંગ મશીન બ્લોક્સ}
    \end{figure}

    \textbf{કાર્યકારી પગલાં:}
    \begin{itemize}
        \item \textbf{ફિલ}: વાલ્વ ખુલે છે, ટબ ભરાય છે.
        \item \textbf{વોશ}: મોટર કપડાંને સાફ કરવા એજિટેટરને ફેરવે છે.
        \item \textbf{ડ્રેન}: પમ્પ ગંદુ પાણી દૂર કરે છે.
        \item \textbf{રિન્સ}: સ્વચ્છ પાણી ભરાય છે, સાબુ દૂર કરવા માટે હલાવે છે, પછી ડ્રેઇન કરે છે.
        \item \textbf{સ્પિન}: પાણી કાઢવા માટે ડ્રમ ઝડપથી ફરે છે.
    \end{itemize}

    \begin{mnemonicbox}
        \mnemonic{FWDRS: ફિલ, વોશ, ડ્રેન, રિન્સ, સ્પિન}
    \end{mnemonicbox}
\end{solutionbox}

% Question 4(a)
\questionmarks{4(a)}{3}{લેસર પ્રિન્ટરના કાર્ય સિદ્ધાંતને સમજાવો. તેની ટેકનિકલ સ્પેસિફિકેશન આપો.}
\begin{solutionbox}
    \textbf{કાર્યકારી સિદ્ધાંત:} લેસર પ્રિન્ટર ઝેરોગ્રાફિક સિદ્ધાંત પર કાર્ય કરે છે. ફરતા ડ્રમ પર લેસર બીમ દ્વારા ઇલેક્ટ્રોસ્ટેટિક ઇમેજ બનાવવામાં આવે છે, જે ટોનર પાવડરને આકર્ષે છે અને પછી કાગળ પર ટ્રાન્સફર અને ફ્યુઝ થાય છે.

    \textbf{ટેક્નિકલ સ્પેસિફિકેશન:}
    \begin{tabulary}{\linewidth}{|l|L|}
        \hline
        \textbf{સ્પેસિફિકેશન} & \textbf{સામાન્ય મૂલ્યો} \\
        \hline
        પ્રિન્ટ રિઝોલ્યુશન & 600 - 1200 DPI \\
        \hline
        પ્રિન્ટ સ્પીડ & 20 - 50 PPM \\
        \hline
        મેમરી & 64 MB - 512 MB \\
        \hline
        ડ્યુટી સાયકલ & 10,000 - 100,000 પેજ/મહિનો \\
        \hline
    \end{tabulary}

    \begin{mnemonicbox}
        \mnemonic{RSCD: Resolution, Speed, Cycle, Duty}
    \end{mnemonicbox}
\end{solutionbox}

% Question 4(b)
\questionmarks{4(b)}{4}{ફોટો કોપીયર મશીનના કાર્યકારી સિદ્ધાંતને સમજાવો. તેના ટેકનિકલ સ્પેસિફિકેશન આપો.}
\begin{solutionbox}
    \textbf{કાર્યકારી સિદ્ધાંત (ઝેરોગ્રાફી):}
    \begin{figure}[H]
        \centering
        \begin{tikzpicture}[gtu flow]
            \node (Scan) [gtu block] {સ્કેનિંગ};
            \node (Charge) [gtu block, right=of Scan] {ચાર્જિંગ};
            \node (Exp) [gtu block, right=of Charge] {એક્સપોઝર};
            \node (Dev) [gtu block, below=of Exp] {ડેવલપમેન્ટ};
            \node (Trans) [gtu block, left=of Dev] {ટ્રાન્સફર};
            \node (Fuse) [gtu block, left=of Trans] {ફ્યુઝિંગ};
            \node (Copy) [gtu block, left=of Fuse] {ફાઇનલ કોપી};

            \draw [gtu arrow] (Scan) -- (Charge);
            \draw [gtu arrow] (Charge) -- (Exp);
            \draw [gtu arrow] (Exp) -- (Dev);
            \draw [gtu arrow] (Dev) -- (Trans);
            \draw [gtu arrow] (Trans) -- (Fuse);
            \draw [gtu arrow] (Fuse) -- (Copy);
        \end{tikzpicture}
        \caption{ફોટોકોપિયર પ્રક્રિયા}
    \end{figure}

    \textbf{સ્પેસિફિકેશન:} કોપી સ્પીડ (20-60 cpm), રિઝોલ્યુશન (600 dpi), પેપર સાઇઝ (A3-A5), ઝૂમ (25-400\%), વોર્મ-અપ ટાઇમ (<30s).

    \begin{mnemonicbox}
        \mnemonic{SCEDTF: Scan, Charge, Expose, Develop, Transfer, Fuse}
    \end{mnemonicbox}
\end{solutionbox}

% Question 4(c)
\questionmarks{4(c)}{7}{વાયરલેસ સીસીટીવી કેમેરા સિસ્ટમની યોજના દોરો અને સમજાવો. નેટવર્ક વિડિયો રેકોર્ડર સમજાવો. CCTV સિસ્ટમમાં ઉપયોગમાં લેવાતા વિવિધ કેમેરાના પ્રકાર લખો અને તેમાંથી કોઈપણ એક સમજાવો.}
\begin{solutionbox}
    \textbf{વાયરલેસ CCTV સિસ્ટમ:}
    \begin{figure}[H]
        \centering
        \begin{tikzpicture}[gtu flow]
            \node (Cam) [gtu block] {કેમેરા (સેન્સર)};
            \node (Proc) [gtu block, right=of Cam] {પ્રોસેસર};
            \node (Tx) [gtu block, right=of Proc] {વાયરલેસ Tx};
            \node (Rx) [gtu block, below=of Tx] {વાયરલેસ Rx};
            \node (NVR) [gtu block, left=of Rx] {NVR};
            \node (HDD) [gtu block, below=of NVR] {HDD};
            \node (Router) [gtu block, left=of NVR] {રાઉટર};
            \node (Inet) [gtu block, left=of Router] {ઇન્ટરનેટ};

            \draw [gtu arrow] (Cam) -- (Proc);
            \draw [gtu arrow] (Proc) -- (Tx);
            \draw [gtu arrow] (Tx) -- node[right] {RF/Wi-Fi} (Rx);
            \draw [gtu arrow] (Rx) -- (NVR);
            \draw [gtu arrow] (NVR) -- (HDD);
            \draw [gtu arrow] (NVR) -- (Router);
            \draw [gtu arrow] (Router) -- (Inet);
        \end{tikzpicture}
        \caption{વાયરલેસ CCTV સિસ્ટમ}
    \end{figure}

    \textbf{નેટવર્ક વિડિયો રેકોર્ડર (NVR):} IP કેમેરા સ્ટ્રીમ્સને રેકોર્ડ કરે છે. ફીચર્સ: રિમોટ એક્સેસ, મોશન ડિટેક્શન, હાઇ કેપેસિટી સ્ટોરેજ.

    \textbf{કેમેરા પ્રકારો:} ડોમ (વેન્ડલ પ્રૂફ), બુલેટ (લોંગ રેન્જ), PTZ (પેન-ટિલ્ટ-ઝૂમ), થર્મલ (નાઇટ વિઝન), IP (નેટવર્ક).

    \textbf{IP કેમેરા:} ડિજિટલ સિગ્નલ મોકલે છે, ઉચ્ચ રિઝોલ્યુશન, PoE પાવર, બિલ્ટ-ઇન વેબ સર્વર.

    \begin{mnemonicbox}
        \mnemonic{WISP-NET: Wireless Images Securely Processed, Networked}
    \end{mnemonicbox}
\end{solutionbox}

% Question 4(a) OR
\questionmarks{4(a) OR}{3}{ઇંકજેટ પ્રિન્ટરના કાર્યકારી સિદ્ધાંતને સમજાવો. તેની તકનીકી લાક્ષણિકતાઓ આપો.}
\begin{solutionbox}
    \textbf{કાર્યકારી સિદ્ધાંત:} નોઝલ દ્વારા શાહીના નાના ટીપાંને કાગળ પર ચોકસાઈપૂર્વક ફેંકીને ઇમેજ બનાવે છે.

    \textbf{ટેકનિકલ સ્પેસિફિકેશન:}
    \begin{itemize}
        \item રિઝોલ્યુશન: 1200-4800 dpi
        \item સ્પીડ: 8-20 ppm (Black), 4-15 ppm (Color)
        \item ઇંક ટાઇપ: ડાય અથવા પિગમેન્ટ
        \item કનેક્ટિવિટી: Wi-Fi, USB
    \end{itemize}

    \begin{mnemonicbox}
        \mnemonic{RIPS: રિઝોલ્યુશન, ઇંક, પ્રિન્ટ સ્પીડ, સાઇઝ}
    \end{mnemonicbox}
\end{solutionbox}

% Question 4(b) OR
\questionmarks{4(b) OR}{4}{ટેલિવિઝન રીસીવર અને વોશિંગ મશીનની જાળવણી અને રિપેરિંગ સમજાવો.}
\begin{solutionbox}
    \textbf{ટેલિવિઝન:}
    \begin{itemize}
        \item \textbf{જાળવણી}: ડસ્ટ ક્લીનિંગ (વેન્ટ્સ), સ્ક્રીન વાઇપિંગ, કેબલ ચેક.
        \item \textbf{ટ્રબલશૂટિંગ}: નો પાવર (ફ્યુઝ ચેક), નો પિક્ચર (કેબલ ચેક), રિમોટ (બેટરી).
    \end{itemize}
    \textbf{વોશિંગ મશીન:}
    \begin{itemize}
        \item \textbf{જાળવણી}: ફિલ્ટર ક્લીનિંગ, ડ્રમ ડિસ્કેલિંગ, હોઝ ચેક.
        \item \textbf{ટ્રબલશૂટિંગ}: નો ડ્રેઇન (બ્લોક્ડ પમ્પ), નો સ્પિન (અનબેલેન્સ), લીકેજ (સીલ/હોઝ).
    \end{itemize}

    \begin{mnemonicbox}
        \mnemonic{CREST: Clean, Repair, Examine, Service, Test}
    \end{mnemonicbox}
\end{solutionbox}

% Question 4(c) OR
\questionmarks{4(c) OR}{7}{સીસીટીવી વ્યાખ્યાયિત કરો. ઘરમાં સ્થાપિત સીસીટીવી કેમેરા સિસ્ટમને schematic દોરીને સમજાવો. એનાલોગ કેમેરા, ડિજિટલ કેમેરા અને IP કેમેરાનું વર્ણન કરો અને તેમનાં વચ્ચેનો તફાવત આપો.}
\begin{solutionbox}
    \textbf{CCTV:} ક્લોઝ્ડ-સર્કિટ ટેલિવિઝન, સુરક્ષા અને સર્વેલન્સ માટે વપરાય છે.

    \begin{figure}[H]
        \centering
        \begin{tikzpicture}[gtu flow]
            \node (Cam) [gtu block] {કેમેરાઓ};
            \node (DVR) [gtu block, right=of Cam] {DVR/NVR};
            \node (HDD) [gtu block, below=of DVR] {HDD};
            \node (Mon) [gtu block, above=of DVR] {મોનિટર};
            \node (Router) [gtu block, right=of DVR] {રાઉટર};
            \node (Int) [gtu block, right=of Router] {ઇન્ટરનેટ};
            \node (Rem) [gtu block, below=of Int] {મોબાઇલ/PC};

            \draw [gtu arrow] (Cam) -- (DVR);
            \draw [gtu arrow] (DVR) -- (HDD);
            \draw [gtu arrow] (DVR) -- (Mon);
            \draw [gtu arrow] (DVR) -- (Router);
            \draw [gtu arrow] (Router) -- (Int);
            \draw [gtu arrow] (Int) -- (Rem);
        \end{tikzpicture}
        \caption{ઘરેલું CCTV સ્કેમેટિક}
    \end{figure}

    \textbf{તફાવત:}
    \begin{itemize}
        \item \textbf{એનાલોગ}: કોએક્સિયલ કેબલ, ઓછું રિઝોલ્યુશન, DVR ની જરૂર છે.
        \item \textbf{ડિજિટલ}: ડિજિટલ સિગ્નલ, કોએક્સિયલ વાપરે છે, સારું રિઝોલ્યુશન.
        \item \textbf{IP કેમેરા}: નેટવર્ક કેબલ (CAT6), હાઇ રિઝોલ્યુશન (4K+), NVR વાપરે છે, સ્માર્ટ ફીચર્સ.
    \end{itemize}

    \begin{mnemonicbox}
        \mnemonic{ADI: Analog is old, Digital is better, IP is best}
    \end{mnemonicbox}
\end{solutionbox}

% Question 5(a)
\questionmarks{5(a)}{3}{જાળવણીને વ્યાખ્યાયિત કરો. તેના પ્રકારો જણાવો. તેમાંથી કોઈપણ એક સમજાવો}
\begin{solutionbox}
    \textbf{વ્યાખ્યા:} સાધનોને કાર્યરત સ્થિતિમાં રાખવા માટેની પ્રક્રિયા.
    \textbf{પ્રકારો:} પ્રિવેન્ટિવ, કરેક્ટિવ (બ્રેકડાઉન), પ્રેડિક્ટિવ.

    \textbf{પ્રિવેન્ટિવ મેઇન્ટેનન્સ:} નિષ્ફળતા અટકાવવા માટે નિયમિત સર્વિસિંગ (દા.ત., સમયાંતરે ઓઇલિંગ, ક્લીનિંગ). આયુષ્ય વધારે છે.

    \begin{mnemonicbox}
        \mnemonic{PCP: Preventive, Corrective, Predictive}
    \end{mnemonicbox}
\end{solutionbox}

% Question 5(b)
\questionmarks{5(b)}{4}{PA સિસ્ટમ્સ અને હોમ થિયેટર સિસ્ટમની જાળવણી વિશે સમજાવો.}
\begin{solutionbox}
    \textbf{PA સિસ્ટમ:} કેબલ કનેક્શન્સ તપાસો, માઇકને સાફ રાખો, ફીડબેક ટાળો, ગ્રાઉન્ડિંગ ચેક કરો.
    \textbf{હોમ થિયેટર:} સ્પીકર પ્લેસમેન્ટ, વેન્ટિલેશન, સ્ક્રીન ક્લીનિંગ, ફર્મવેર અપડેટ્સ.

    \begin{mnemonicbox}
        \mnemonic{CAVS: Clean, Adjust, Verify, Service}
    \end{mnemonicbox}
\end{solutionbox}

% Question 5(c)
\questionmarks{5(c)}{7}{DTH ટેકનોલોજીનો બ્લોક ડાયાગ્રામ દોરો અને સમજાવો. DTH સિસ્ટમમાં વપરાતા હાર્ડવેર ઘટકોની ચર્ચા કરો. વર્તમાન DTH સિસ્ટમમાં હાલમાં પ્રદાન કરવામાં આવતી વિવિધ આધુનિક સુવિધાઓની ચર્ચા કરો.}
\begin{solutionbox}
    \begin{figure}[H]
        \centering
        \begin{tikzpicture}[gtu flow]
            \node (Broad) [gtu block] {બ્રોડકાસ્ટર};
            \node (Up) [gtu block, right=of Broad] {અપલિંક};
            \node (Sat) [gtu block, right=of Up] {સેટેલાઇટ};
            \node (Dish) [gtu block, below=of Sat] {ડિશ};
            \node (LNB) [gtu block, below=of Dish] {LNB};
            \node (STB) [gtu block, left=of LNB] {STB};
            \node (TV) [gtu block, left=of STB] {TV};

            \draw [gtu arrow] (Broad) -- (Up);
            \draw [gtu arrow] (Up) -- node[above] {Ku બેન્ડ} (Sat);
            \draw [gtu arrow] (Sat) -- (Dish);
            \draw [gtu arrow] (Dish) -- (LNB);
            \draw [gtu arrow] (LNB) -- node[below] {IF} (STB);
            \draw [gtu arrow] (STB) -- (TV);
        \end{tikzpicture}
        \caption{DTH સિસ્ટમ}
    \end{figure}

    \textbf{ઘટકો:} ડિશ (રિફ્લેક્ટર), LNB (ફ્રિક્વન્સી ડાઉનકન્વર્ટર), STB (ડીકોડર), સ્માર્ટ કાર્ડ (ઓથોરાઇઝેશન).
    \textbf{સુવિધાઓ:} HD/4K, રેકોર્ડિંગ, VOD, પોઝ/રિવાઇન્ડ, પેરેન્ટલ લોક.

    \begin{mnemonicbox}
        \mnemonic{Direct To Home: Satellite transmits High-quality signals}
    \end{mnemonicbox}
\end{solutionbox}

% Question 5(a) OR
\questionmarks{5(a) OR}{3}{Differentiate between predictive and preventive maintenance.}
\begin{solutionbox}
    \begin{tabulary}{\linewidth}{|l|L|L|}
        \hline
        \textbf{Aspect} & \textbf{Preventive} & \textbf{Predictive} \\
        \hline
        Basis & Time/Schedule & Actual Condition \\
        \hline
        Trigger & Fixed Interval & Data/Warning Signs \\
        \hline
        Cost & Medium (may replace good parts) & Low long-term (max life) \\
        \hline
        Example & Change oil every 5000km & Change oil when sensor detects dirt \\
        \hline
    \end{tabulary}

    \begin{mnemonicbox}
        \mnemonic{TIME vs DATA: Timed Intervals Maintenance Everywhere vs Data Analysis Triggers Action}
    \end{mnemonicbox}
\end{solutionbox}

% Question 5(b) OR
\questionmarks{5(b) OR}{4}{Describe troubleshooting procedure and safety precautions for microwave oven.}
\begin{solutionbox}
    \textbf{Troubleshooting Procedure:}
    \begin{enumerate}
        \item \textbf{No Power}: Check fuse, thermal cutout, door switches.
        \item \textbf{Not Heating}: Check magnetron, HV diode, HV capacitor.
        \item \textbf{Sparks/Arcing}: Check waveguide cover, remove metal objects, check paint damage.
    \end{enumerate}

    \textbf{Safety Precautions:}
    \begin{itemize}
        \item Always discharge HV capacitor before touching components (store 2000V+).
        \item Check for radiation leakage after reassembly.
        \item Never bypass door interlock switches.
        \item Do not operate with door open.
    \end{itemize}

    \begin{mnemonicbox}
        \mnemonic{DUEL-SAFE: Disconnect power, Use discharge tool, Examine systematically, Look for damage - Safety Always First, Every time}
    \end{mnemonicbox}
\end{solutionbox}

% Question 5(a) OR
\questionmarks{5(a) OR}{3}{અનુમાનિત અને નિવારક જાળવણી વચ્ચે તફાવત કરો.}
\begin{solutionbox}
    \begin{tabulary}{\linewidth}{|l|L|L|}
        \hline
        \textbf{પાસાં} & \textbf{પ્રેડિક્ટિવ} & \textbf{પ્રિવેન્ટિવ} \\
        \hline
        આધાર & કન્ડિશન/ડેટા & સમય/શેડ્યુલ \\
        \hline
        સાધનો & સેન્સર્સ/મોનિટરિંગ & ચેકલિસ્ટ/મેન્યુઅલ \\
        \hline
        ખર્ચ & શરૂઆતમાં વધુ & મધ્યમ \\
        \hline
    \end{tabulary}
    \begin{mnemonicbox}
        \mnemonic{Data vs Time}
    \end{mnemonicbox}
\end{solutionbox}

% Question 5(b) OR
\questionmarks{5(b) OR}{4}{માઇક્રોવેવ ઓવન માટે મુશ્કેલી નિવારણ પ્રક્રિયા અને સલામતીની સાવચેતીઓનું વર્ણન કરો.}
\begin{solutionbox}
    \textbf{ટ્રબલશૂટિંગ:} પાવર ચેક કરો, ફ્યુઝ, ડોર ઇન્ટરલોક, મેગ્નેટ્રોન. સ્પાર્કિંગ માટે મેટલ છે કે નહીં તે જુઓ.
    \textbf{સલામતી:} કેપેસિટર ડિસ્ચાર્જ કરો (HV શોક), ડોર સ્વિચ બાયપાસ ન કરો, લીકેજ ટેસ્ટ કરો.

    \begin{mnemonicbox}
        \mnemonic{Safety First: Discharge Capacitor}
    \end{mnemonicbox}
\end{solutionbox}

% Question 5(c) OR
\questionmarks{5(c) OR}{7}{PA સિસ્ટમનો બ્લોક ડાયાગ્રામ દોરો અને સમજાવો. ઓડિટોરિયમ માટે ડિઝાઇન કરતી વખતે ડિઝાઇન પરિમાણોની ચર્ચા કરો. આઉટપુટ ઇમ્પીડેન્સ તરીકે 8 ઓહ્મ ધરાવતા PA સિસ્ટમ એમ્પ્લિફાયર સાથે ચાર 8 ઓહ્મ સ્પીકર્સનું જોડાણનો ડાયાગ્રામ દોરો.}
\begin{solutionbox}
    \begin{figure}[H]
        \centering
        \begin{tikzpicture}[gtu flow]
            \node (Mic) [gtu block] {માઇક/સોર્સ};
            \node (Mix) [gtu block, right=of Mic] {મિક્સર};
            \node (Eq) [gtu block, right=of Mix] {EQ};
            \node (Amp) [gtu block, below=of Eq] {પાવર એમ્પ્લાયફાયર};
            \node (Spk) [gtu block, left=of Amp] {સ્પીકર્સ};

            \draw [gtu arrow] (Mic) -- (Mixer);
            \draw [gtu arrow] (Mixer) -- (Eq);
            \draw [gtu arrow] (Eq) -- (Amp);
            \draw [gtu arrow] (Amp) -- (Spk);
        \end{tikzpicture}
        \caption{PA સિસ્ટમ બ્લોક્સ}
    \end{figure}

    \textbf{ડિઝાઇન પરિમાણો:} એકોસ્ટિક્સ, પાવર (2-3W/વ્યક્તિ), કવરેજ, ફ્રિક્વન્સી રિસ્પોન્સ, ફીડબેક કંટ્રોલ.

    \textbf{સ્પીકર કનેક્શન (8$\Omega$ એમ્પ સાથે 4x 8$\Omega$ સ્પીકર્સ):}
    સિરીઝ-પેરેલલ કનેક્શન જરૂરી છે.
    - 2 સ્પીકર્સ સિરીઝમાં (16$\Omega$)
    - બીજી 2 સ્પીકર્સ સિરીઝમાં (16$\Omega$)
    - આ બે સેટ્સ પેરેલલમાં = 8$\Omega$ ટોટલ ઇમ્પીડન્સ.

    \begin{figure}[H]
        \centering
        \begin{tikzpicture}[gtu flow]
            \node (Amp) [gtu block] {Amp output};
            \node (S1) [gtu block, below left=of Amp] {S1 (8$\Omega$)};
            \node (S2) [gtu block, below=of S1] {S2 (8$\Omega$)};
            \node (S3) [gtu block, below right=of Amp] {S3 (8$\Omega$)};
            \node (S4) [gtu block, below=of S3] {S4 (8$\Omega$)};

            \draw [gtu arrow] (Amp) -| (S1);
            \draw [gtu arrow] (S1) -- (S2);
            \draw [gtu arrow] (S2) -- ++(0,-1) -| (Amp);

            \draw [gtu arrow] (Amp) -| (S3);
            \draw [gtu arrow] (S3) -- (S4);
            \draw [gtu arrow] (S4) -- ++(0,-1) -| (Amp);
        \end{tikzpicture}
        \caption{સિરીઝ-પેરેલલ કનેક્શન}
    \end{figure}

    \begin{mnemonicbox}
        \mnemonic{Match Impedance for Max Power}
    \end{mnemonicbox}
\end{solutionbox}

\end{document}

