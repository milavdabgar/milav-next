\documentclass{article}

% Imports

% content/resources/templates/preamble.tex
\usepackage[margin=0.6in]{geometry}
\author{Milav Dabgar}
\usepackage{amsmath,amssymb,amsthm}
\usepackage{booktabs}
\usepackage{multirow}
\usepackage{xcolor}
\usepackage{tcolorbox}
\tcbuselibrary{breakable,skins}
\usepackage[colorlinks=true,linkcolor=blue]{hyperref}
\usepackage{titlesec}
\usepackage{enumitem}
\usepackage{tikz}
\usepackage{pgfplots}
\usepackage{circuitikz}
\usepackage[version=4]{mhchem}
\usepackage{longtable}
\usepackage{array}
\usepackage{float}
\usepackage{caption}
\usepackage{listings}

\lstset{
  basicstyle=\small\ttfamily,
  breaklines=true,
  breakatwhitespace=false,
  postbreak=\mbox{\textcolor{red}{$\hookrightarrow$}\space},
  float=false,
  numbers=left,
  numberstyle=\tiny\color{gray},
  numbersep=10pt,
  xleftmargin=2em,
  keywordstyle=\color{blue},
  commentstyle=\color{green!60!black},
  stringstyle=\color{purple},
  backgroundcolor=\color{gray!5},
  showstringspaces=false,
  tabsize=2,
  captionpos=b,
  keepspaces=true,
  columns=flexible
}

\pgfplotsset{compat=1.18}
\usetikzlibrary{shapes,arrows,positioning,calc,patterns,decorations.pathmorphing,decorations.markings,arrows.meta}

% Color scheme
\definecolor{headcolor}{RGB}{0,102,204}
\definecolor{keycolor}{RGB}{220,20,60}
\definecolor{solutioncolor}{RGB}{34,139,34}
\definecolor{mnemoniccolor}{RGB}{148,0,211}
\definecolor{codecolor}{RGB}{0,0,100}

% Spacing
\setlength{\parskip}{3pt}
\setlist[itemize]{nosep}
\setlist[enumerate]{nosep}

% Title formatting
\titleformat{\section}{\Large\bfseries\color{headcolor}}{\thesection}{1em}{}
\titleformat{\subsection}{\large\bfseries\color{headcolor}}{\thesubsection}{1em}{}

% Pandoc tightlist compatibility
\providecommand{\tightlist}{%
  \setlength{\itemsep}{0pt}\setlength{\parskip}{0pt}}

% Pandoc longtable compatibility
\newcounter{none}
\def\thenone{}


% content/resources/templates/english-boxes.tex

% Custom environments
\newtcolorbox{solutionbox}{
 breakable,
 enhanced,
 colback=solutioncolor!5!white,
 colframe=solutioncolor!75!black,
 fonttitle=\bfseries,
 title=Solution
}

\newtcolorbox{solutionboxnobreak}{
 colback=solutioncolor!5!white,
 colframe=solutioncolor!75!black,
 fonttitle=\bfseries,
 title=Solution
}

\newtcolorbox{keyformula}{
 breakable,
 enhanced,
 colback=keycolor!5!white,
 colframe=keycolor!75!black,
 fonttitle=\bfseries,
 title=Key Formula
}

\newtcolorbox{mnemonicboxenv}{
 breakable,
 enhanced,
 colback=mnemoniccolor!5!white,
 colframe=mnemoniccolor!75!black,
 fonttitle=\bfseries,
 title=Mnemonic
}

\newcommand{\mnemonicbox}[1]{%
  \begin{mnemonicboxenv}
    #1
  \end{mnemonicboxenv}
}


% Custom commands for GTU solutions
% This file defines semantic commands for consistent formatting

% Question command with automatic formatting
\newcommand{\question}[2]{%
  \section*{Question #1}%
  \textbf{#2}%
}

% OR question variant
\newcommand{\questionor}[2]{%
  \section*{Question #1 OR}%
  \textbf{#2}%
}

% Proper table environment with caption
\newenvironment{answertable}[1]{%
  \begin{table}[htbp]
  \centering
  \caption{#1}
}{%
  \end{table}
}

% Proper figure environment for diagrams
\newenvironment{answerdiagram}[1]{%
  \begin{figure}[htbp]
  \centering
  \caption{#1}
}{%
  \end{figure}
}

% Semantic markup for key terms
\newcommand{\keyword}[1]{\textbf{#1}}
\newcommand{\code}[1]{\texttt{#1}}
\newcommand{\classname}[1]{\texttt{#1}}
\newcommand{\methodname}[1]{\texttt{#1}}

% Proper quotation marks
\newcommand{\mnemonic}[1]{``#1''}


\title{Consumer Electronics \& Maintenance (4341107) - Winter 2024 Solution}
\date{April 18, 2024}

\begin{document}
\maketitle
\solutiontitle


% Question 1(a)
\questionmarks{1(a)}{3}{Define only: 1. Loudness 2. Timbre 3. Echo}
\begin{solutionbox}
    \begin{tabulary}{\linewidth}{|l|L|}
        \hline
        \textbf{Term} & \textbf{Definition} \\
        \hline
        \textbf{Loudness} & The subjective perception of sound intensity that depends on sound pressure and frequency \\
        \hline
        \textbf{Timbre} & The quality of sound that distinguishes different instruments or voices playing the same note \\
        \hline
        \textbf{Echo} & A sound reflection that arrives at the listener with a delay greater than 50ms after the direct sound \\
        \hline
    \end{tabulary}

    \begin{mnemonicbox}
        \mnemonic{LTE: Loudness measures strength, Timbre gives uniqueness, Echo comes back delayed}
    \end{mnemonicbox}
\end{solutionbox}

% Question 1(b)
\questionmarks{1(b)}{4}{List Type of loudspeaker and explain any one of them}
\begin{solutionbox}
    \textbf{Types of Loudspeakers:}
    \begin{itemize}
        \item Dynamic/Moving Coil (Electromagnetic)
        \item Electrostatic (Charged diaphragm)
        \item Ribbon (Thin metal ribbon)
        \item Piezoelectric (Vibrating crystals)
        \item Horn (Acoustic amplification)
        \item Planar Magnetic (Magnetic strips)
    \end{itemize}

    \textbf{Dynamic/Moving Coil Loudspeaker:}
    \begin{figure}[H]
        \centering
        \begin{tikzpicture}[gtu flow]
            \node (Sig) [gtu block] {Audio Signal};
            \node (Coil) [gtu block, right=of Sig] {Voice Coil};
            \node (Mag) [gtu block, right=of Coil] {EM Field};
            \node (Move) [gtu block, below=of Mag] {Coil Movement};
            \node (Cone) [gtu block, left=of Move] {Cone/Diaphragm};
            \node (Sound) [gtu block, left=of Cone] {Sound Waves};

            \draw [gtu arrow] (Sig) -- (Coil);
            \draw [gtu arrow] (Coil) -- (Mag);
            \draw [gtu arrow] (Mag) -- (Move);
            \draw [gtu arrow] (Move) -- (Cone);
            \draw [gtu arrow] (Cone) -- (Sound);
        \end{tikzpicture}
        \caption{Dynamic Loudspeaker Working}
    \end{figure}

    \begin{itemize}
        \item \textbf{Magnetic Structure}: Permanent magnet creates static magnetic field.
        \item \textbf{Voice Coil}: Receives audio current and creates varying magnetic field.
        \item \textbf{Diaphragm/Cone}: Attached to voice coil, vibrates to produce sound waves.
    \end{itemize}

    \begin{mnemonicbox}
        \mnemonic{COPPER-D: Coil Oscillates, Permanent magnet Pulls/Pushes, Emitting Resonance through Diaphragm}
    \end{mnemonicbox}
\end{solutionbox}

% Question 1(c)
\questionmarks{1(c)}{7}{List types of Microphone. State its Characteristics and explain Wireless Microphone in detail.}
\begin{solutionbox}
    \textbf{Types of Microphones:}
    \begin{itemize}
        \item Dynamic, Condenser, Carbon, Ribbon, Crystal/Piezoelectric, Electret, MEMS.
    \end{itemize}

    \textbf{Characteristics:} Sensitivity, Frequency Response, Directional Pattern, Impedance, Signal-to-Noise Ratio.

    \textbf{Wireless Microphone System:}
    \begin{figure}[H]
        \centering
        \begin{tikzpicture}[gtu flow]
            \node (Sound) [gtu block] {Sound};
            \node (Mic) [gtu block, right=of Sound] {Mic Element};
            \node (Pre) [gtu block, right=of Mic] {Pre-Amp};
            \node (Comp) [gtu block, right=of Pre] {Compressor};
            \node (Tx) [gtu block, below=of Comp] {RF Tx};
            \node (Rx) [gtu block, left=of Tx] {RF Rx};
            \node (Demod) [gtu block, left=of Rx] {Demodulator};
            \node (Exp) [gtu block, left=of Demod] {Expander};
            \node (Out) [gtu block, below=of Exp] {Output};

            \draw [gtu arrow] (Sound) -- (Mic);
            \draw [gtu arrow] (Mic) -- (Pre);
            \draw [gtu arrow] (Pre) -- (Comp);
            \draw [gtu arrow] (Comp) -- (Tx);
            \draw [gtu arrow] (Tx) -- node[above] {Radio Waves} (Rx);
            \draw [gtu arrow] (Rx) -- (Demod);
            \draw [gtu arrow] (Demod) -- (Exp);
            \draw [gtu arrow] (Exp) |- (Out);
        \end{tikzpicture}
        \caption{Wireless Mic System}
    \end{figure}

    \begin{itemize}
        \item \textbf{Mic Element}: Converts sound to electrical signals.
        \item \textbf{Transmitter}: Modulates audio onto VHF/UHF carrier (FM/Digital).
        \item \textbf{Receiver}: Captures RF signal and demodulates audio.
        \item \textbf{Compander}: Compresses signal at Tx and expands at Rx for noise reduction.
    \end{itemize}

    \begin{mnemonicbox}
        \mnemonic{WIRED: Wireless Is Radio-Enabled Device}
    \end{mnemonicbox}
\end{solutionbox}

% Question 1(c) OR
\questionmarks{1(c) OR}{7}{State characteristics of Loudspeakers and explain pearmeant magnet loudspeaker with its advantages and disadvantages.}
\begin{solutionbox}
    \textbf{Loudspeaker Characteristics:}
    \begin{itemize}
        \item Frequency Response, Sensitivity, Impedance, Power Handling, Directivity, Distortion.
    \end{itemize}

    \textbf{Permanent Magnet Loudspeaker:}
    \begin{figure}[H]
        \centering
        \begin{tikzpicture}[gtu flow]
            \node (Sig) [gtu block] {Signal};
            \node (Coil) [gtu block, right=of Sig] {Voice Coil};
            \node (Field) [gtu block, right=of Coil] {Magnetic Field};
            \node (Mag) [gtu block, right=of Field] {Permanent Magnet};
            \node (Diaph) [gtu block, below=of Field] {Diaphragm};
            \node (Sound) [gtu block, left=of Diaph] {Sound Waves};

            \draw [gtu arrow] (Sig) -- (Coil);
            \draw [gtu arrow] (Coil) <-> (Field);
            \draw [gtu arrow] (Mag) -- (Field);
            \draw [gtu arrow] (Coil) -- (Diaph);
            \draw [gtu arrow] (Diaph) -- (Sound);
        \end{tikzpicture}
        \caption{PM Loudspeaker Principle}
    \end{figure}

    \textbf{Advantages:} Cost-effective, Reliable, Compact, Efficient.\\
    \textbf{Disadvantages:} Limited Power (fixed field), Magnet Deterioration, Weight, Heat Sensitivity.

    \begin{mnemonicbox}
        \mnemonic{PMLS: Permanent Magnet Loudly Speaks}
    \end{mnemonicbox}
\end{solutionbox}

% Question 2(a)
\questionmarks{2(a)}{3}{Define 1. Aspect ratio 2. Chrominance 3. Additive Mixing}
\begin{solutionbox}
    \begin{tabulary}{\linewidth}{|l|L|}
        \hline
        \textbf{Term} & \textbf{Definition} \\
        \hline
        \textbf{Aspect Ratio} & Ratio of width to height of a display screen (e.g., 16:9) \\
        \hline
        \textbf{Chrominance} & Color information in video signal, separate from brightness (Luminance) \\
        \hline
        \textbf{Additive Mixing} & Combining colored lights (RGB) to create new colors; all together make white \\
        \hline
    \end{tabulary}

    \begin{mnemonicbox}
        \mnemonic{ACA: Aspect sets dimensions, Chrominance adds color, Additive mixing creates brightness}
    \end{mnemonicbox}
\end{solutionbox}

% Question 2(b)
\questionmarks{2(b)}{4}{Explain interlace scanning}
\begin{solutionbox}
    \textbf{Concept:} Dividing a video frame into two fields (odd and even lines) to reduce bandwidth. Use standard rate 50/60 fields/sec.

    \begin{figure}[H]
        \centering
        \begin{tikzpicture}[gtu flow]
            \node (Frame) [gtu block] {Full Frame};
            \node (Odd) [gtu block, below left=of Frame] {Odd Lines (1,3,5)};
            \node (Even) [gtu block, below right=of Frame] {Even Lines (2,4,6)};
            \node (F1) [gtu block, below=of Odd] {Field 1};
            \node (F2) [gtu block, below=of Even] {Field 2};
            \node (Disp) [gtu block, below right=of F1] {Interlaced Display};

            \draw [gtu arrow] (Frame) -- (Odd);
            \draw [gtu arrow] (Frame) -- (Even);
            \draw [gtu arrow] (Odd) -- (F1);
            \draw [gtu arrow] (Even) -- (F2);
            \draw [gtu arrow] (F1) -- (Disp);
            \draw [gtu arrow] (F2) -- (Disp);
        \end{tikzpicture}
        \caption{Interlaced Scanning Process}
    \end{figure}

    \begin{mnemonicbox}
        \mnemonic{ODD-EVEN: One Display, then Delayed Extra Visual Enhancement Next}
    \end{mnemonicbox}
\end{solutionbox}

% Question 2(c)
\questionmarks{2(c)}{7}{Discuss working principle of LED Television. State its advantages and compare it with LCD television.}
\begin{solutionbox}
    \textbf{Working Principle:} LED TV is an LCD TV that uses LEDs for backlighting instead of CCFLs.
    \begin{figure}[H]
        \centering
        \begin{tikzpicture}[gtu flow]
            \node (Sig) [gtu block] {Signal};
            \node (Proc) [gtu block, right=of Sig] {Processing};
            \node (LED) [gtu block, right=of Proc] {LED Backlight};
            \node (LCD) [gtu block, below=of LED] {LCD Panel};
            \node (Pol) [gtu block, left=of LCD] {Polarizers};
            \node (Color) [gtu block, left=of Pol] {Color Filters};
            \node (Disp) [gtu block, left=of Color] {Display};

            \draw [gtu arrow] (Sig) -- (Proc);
            \draw [gtu arrow] (Proc) -- (LCD);
            \draw [gtu arrow] (LED) -- (LCD);
            \draw [gtu arrow] (LCD) -- (Pol);
            \draw [gtu arrow] (Pol) -- (Color);
            \draw [gtu arrow] (Color) -- (Disp);
        \end{tikzpicture}
        \caption{LED TV Architecture}
    \end{figure}

    \textbf{Advantages:} Energy Efficient, Thinner Design, Better Contrast (Local Dimming), Longer Lifespan, Mercury-free.

    \textbf{Comparison (LED vs LCD):}
    \begin{tabulary}{\linewidth}{|l|L|L|}
        \hline
        \textbf{Feature} & \textbf{LED TV} & \textbf{LCD TV} \\
        \hline
        Backlight & LEDs & CCFL Tubes \\
        \hline
        Thickness & Thinner (Slim) & Thicker \\
        \hline
        Power & Lower & Higher \\
        \hline
        Contrast & Better & Lower \\
        \hline
    \end{tabulary}

    \begin{mnemonicbox}
        \mnemonic{LEDGE: Light Emitting Diodes Give Excellence}
    \end{mnemonicbox}
\end{solutionbox}

% Question 2(a) OR
\questionmarks{2(a) OR}{3}{State any six standards of Color television system.}
\begin{solutionbox}
    \begin{itemize}
        \item \textbf{PAL} (Phase Alternating Line)
        \item \textbf{NTSC} (National Television System Committee)
        \item \textbf{SECAM} (Sequential Color with Memory)
        \item \textbf{PAL-M} (Brazil variant)
        \item \textbf{ATSC} (Digital - N. America)
        \item \textbf{DVB-T} (Digital - Europe)
        \item \textbf{ISDB} (Digital - Japan)
    \end{itemize}

    \begin{mnemonicbox}
        \mnemonic{PANS-ADI: PAL, ATSC, NTSC, SECAM - All Display Images}
    \end{mnemonicbox}
\end{solutionbox}

% Question 2(b) OR
\questionmarks{2(b) OR}{4}{Explain working of LCD Television.}
\begin{solutionbox}
    \begin{figure}[H]
        \centering
        \begin{tikzpicture}[gtu flow]
            \node (In) [gtu block] {Signal};
            \node (Driver) [gtu block, right=of In] {LCD Drivers};
            \node (Back) [gtu block, right=of Driver] {Backlight};
            \node (Diff) [gtu block, below=of Back] {Diffuser};
            \node (Pol1) [gtu block, left=of Diff] {Polarizer 1};
            \node (LCD) [gtu block, left=of Pol1] {Liquid Crystals};
            \node (Pol2) [gtu block, left=of LCD] {Polarizer 2};
            \node (RGB) [gtu block, below=of LCD] {RGB Filters};
            \node (Scrn) [gtu block, right=of RGB] {Screen};

            \draw [gtu arrow] (In) -- (Driver);
            \draw [gtu arrow] (Driver) -- (LCD);
            \draw [gtu arrow] (Back) -- (Diff);
            \draw [gtu arrow] (Diff) -- (Pol1);
            \draw [gtu arrow] (Pol1) -- (LCD);
            \draw [gtu arrow] (LCD) -- (Pol2);
            \draw [gtu arrow] (Pol2) -- (RGB);
            \draw [gtu arrow] (RGB) -- (Scrn);
        \end{tikzpicture}
        \caption{LCD TV Stack}
    \end{figure}

    \textbf{Working:} Backlight passes through Polarizer 1. Liquid crystals twist/untwist based on voltage (TFT) to block or pass light through Polarizer 2. Light then passes through RGB filters to create color pixels.

    \begin{mnemonicbox}
        \mnemonic{BPLTC: Backlight Passes through Liquid crystals That Color}
    \end{mnemonicbox}
\end{solutionbox}

% Question 2(c) OR
\questionmarks{2(c) OR}{7}{Draw and Explain block diagram of PAL-D decoder.}
\begin{solutionbox}
    \begin{figure}[H]
        \centering
        \begin{tikzpicture}[gtu flow]
            \node (Comp) [gtu block] {Composite Video};
            \node (Sep) [gtu block, right=of Comp] {Y/C Sep};
            \node (Y) [gtu block, right=of Sep] {Lum (Y)};
            \node (C) [gtu block, below=of Sep] {Chrom (C)};
            \node (Delay) [gtu block, below=of C] {Delay Line (64$\mu$s)};
            \node (Switch) [gtu block, right=of Delay] {PAL Switch};
            \node (Demod) [gtu block, right=of Switch] {U/V Demod};
            \node (Mat) [gtu block, right=of Y] {RGB Matrix};
            \node (Out) [gtu block, right=of Mat] {RGB Out};

            \draw [gtu arrow] (Comp) -- (Sep);
            \draw [gtu arrow] (Sep) -- (Y);
            \draw [gtu arrow] (Sep) -- (C);
            \draw [gtu arrow] (C) -- (Delay);
            \draw [gtu arrow] (C) -- (Switch); % Direct path for U usually, simplified here
            \draw [gtu arrow] (Delay) -- (Switch);
            \draw [gtu arrow] (Switch) -- (Demod);
            \draw [gtu arrow] (Demod) -- node[right] {U,V} (Mat);
            \draw [gtu arrow] (Y) -- (Mat);
            \draw [gtu arrow] (Mat) -- (Out);
        \end{tikzpicture}
        \caption{PAL-D Decoder Block Diagram}
    \end{figure}

    \begin{itemize}
        \item \textbf{Y/C Separator}: Splits brightness (Y) and color (C).
        \item \textbf{Delay Line}: Delays signal by 64$\mu$s (one line) to average phase errors.
        \item \textbf{PAL Switch}: Reverses V-signal phase on alternate lines.
        \item \textbf{U/V Demodulator}: Extracts color difference signals.
        \item \textbf{RGB Matrix}: Combines Y, U, V to output Red, Green, Blue.
    \end{itemize}

    \begin{mnemonicbox}
        \mnemonic{PAL Decodes Color Right By Switching, Delaying, Unscrambling Variations}
    \end{mnemonicbox}
\end{solutionbox}



% Question 3(a)
\questionmarks{3(a)}{3}{Give classification of rooftop Solar power plant and explain any one plant.}
\begin{solutionbox}
    \textbf{Classification:}
    \begin{itemize}
        \item \textbf{Grid-Connected/On-Grid}: Connected directly to utility grid without batteries.
        \item \textbf{Off-Grid/Standalone}: Uses batteries to store power, not connected to grid.
        \item \textbf{Hybrid}: Combines both grid connection and battery backup.
    \end{itemize}

    \textbf{Grid-Connected System:}
    \begin{figure}[H]
        \centering
        \begin{tikzpicture}[gtu flow]
            \node (Panels) [gtu block] {Solar Panels};
            \node (Inv) [gtu block, right=of Panels] {DC-AC Inverter};
            \node (Meter) [gtu block, right=of Inv] {Bi-directional Meter};
            \node (Grid) [gtu block, right=of Meter] {Utility Grid};
            \node (Load) [gtu block, below=of Meter] {Home Load};

            \draw [gtu arrow] (Panels) -- (Inv);
            \draw [gtu arrow] (Inv) -- (Meter);
            \draw [gtu arrow] (Meter) -- (Grid);
            \draw [gtu arrow] (Meter) -- (Load);
        \end{tikzpicture}
        \caption{Grid-Connected Rooftop System}
    \end{figure}

    \begin{itemize}
        \item Solar panels generate DC power from sunlight.
        \item Inverter converts DC to AC synchronous with grid.
        \item Bi-directional meter records import (consumption) and export (generation).
        \item Excess power is fed to the grid (net metering).
    \end{itemize}

    \begin{mnemonicbox}
        \mnemonic{GOH: Grid connects, Off-grid stores, Hybrid does both}
    \end{mnemonicbox}
\end{solutionbox}

% Question 3(b)
\questionmarks{3(b)}{4}{Give at least four technical specification of Refrigerator and split Air condition each.}
\begin{solutionbox}
    \textbf{Refrigerator Specifications:}
    \begin{tabulary}{\linewidth}{|l|L|}
        \hline
        \textbf{Specification} & \textbf{Typical Range} \\
        \hline
        Capacity & 150-750 liters \\
        \hline
        Power Consumption & 100-400 kWh/year \\
        \hline
        Refrigerant & R-600a, R-134a \\
        \hline
        Compressor & Reciprocating or Inverter \\
        \hline
    \end{tabulary}

    \textbf{Split Air Conditioner Specifications:}
    \begin{tabulary}{\linewidth}{|l|L|}
        \hline
        \textbf{Specification} & \textbf{Typical Range} \\
        \hline
        Cooling Capacity & 1.0 - 2.0 Tons (12000-24000 BTU) \\
        \hline
        ISEER Rating & 3.0 - 5.0 Stars \\
        \hline
        Refrigerant & R-32, R-410A \\
        \hline
        Noise Level & 30-45 dB (Indoor unit) \\
        \hline
    \end{tabulary}

    \begin{mnemonicbox}
        \mnemonic{CERT: Capacity, Efficiency, Refrigerant Type, Temperature}
    \end{mnemonicbox}
\end{solutionbox}

% Question 3(c)
\questionmarks{3(c)}{7}{Explain working of Microwave oven with respect to its working principle, functional block diagram and its safety precautions while in operative condition.}
\begin{solutionbox}
    \textbf{Working Principle:} The magnetron generates high-frequency microwaves (2.45 GHz) which agitate water molecules in food. This vibration creates friction, generating heat that cooks the food from within.

    \begin{figure}[H]
        \centering
        \begin{tikzpicture}[gtu flow]
            \node (Panel) [gtu block] {Control Panel};
            \node (Control) [gtu block, right=of Panel] {Control Unit};
            \node (Driver) [gtu block, right=of Control] {Power Driver};
            \node (HV) [gtu block, below=of Driver] {HV Transformer};
            \node (HVDiode) [gtu block, left=of HV] {HV Diode/Cap};
            \node (Mag) [gtu block, left=of HVDiode] {Magnetron};
            \node (Guide) [gtu block, left=of Mag] {Waveguide};
            \node (Cavity) [gtu block, below=of Guide] {Cooking Cavity};
            \node (Motor) [gtu block, right=of Cavity] {Turntable Motor};

            \draw [gtu arrow] (Panel) -- (Control);
            \draw [gtu arrow] (Control) -- (Driver);
            \draw [gtu arrow] (Control) -- (Motor);
            \draw [gtu arrow] (Driver) -- (HV);
            \draw [gtu arrow] (HV) -- (HVDiode);
            \draw [gtu arrow] (HVDiode) -- (Mag);
            \draw [gtu arrow] (Mag) -- (Guide);
            \draw [gtu arrow] (Guide) -- (Cavity);
            \draw [gtu arrow] (Motor) -- (Cavity);
        \end{tikzpicture}
        \caption{Microwave Oven Block Diagram}
    \end{figure}

    \textbf{Safety Precautions:}
    \begin{itemize}
        \item \textbf{Door Interlocks}: Ensure oven cannot operate if door is open.
        \item \textbf{RF Shielding}: Metal mesh on door prevents microwave leakage.
        \item \textbf{Capacitor Discharge}: High voltage capacitor retains charge; needs discharge during service.
        \item \textbf{No Metal}: Do not use metal containers inside to prevent arcing.
        \item \textbf{Never Run Empty}: Can damage magnetron due to reflected waves.
    \end{itemize}

    \begin{mnemonicbox}
        \mnemonic{MICROWAVE: Magnetron Initiates Cooking, Radiation Only Within Authorized Vessel Environment}
    \end{mnemonicbox}
\end{solutionbox}

% Question 3(a) OR
\questionmarks{3(a) OR}{3}{State various hardware used in rooftop solar power plant and explain solar panels used in it.}
\begin{solutionbox}
    \textbf{Hardware:} Solar Panels, Inverter, Mounting Structure, Batteries (optional), Charge Controller, AC/DC Distribution Boxes, Cables.

    \textbf{Solar Panels Explanation:}
    \begin{figure}[H]
        \centering
        \begin{tikzpicture}[gtu flow]
            \node (Sun) [gtu block] {Sunlight};
            \node (Glass) [gtu block, right=of Sun] {Glass};
            \node (EVA1) [gtu block, right=of Glass] {EVA};
            \node (Cell) [gtu block, right=of EVA1] {Silicon Cell};
            \node (EVA2) [gtu block, below=of Cell] {EVA};
            \node (Sheet) [gtu block, left=of EVA2] {Backsheet};
            \node (Frame) [gtu block, left=of Sheet] {Al Frame};

            \draw [gtu arrow] (Sun) -- (Glass);
            \draw [gtu arrow] (Glass) -- (EVA1);
            \draw [gtu arrow] (EVA1) -- (Cell);
            \draw [gtu arrow] (Cell) -- (EVA2);
            \draw [gtu arrow] (EVA2) -- (Sheet);
            \draw [gtu arrow] (Sheet) -- (Frame);
        \end{tikzpicture}
        \caption{Solar Panel Layers}
    \end{figure}

    Solar PV panels consist of semiconductor cells (Silicon) encapsulated between glass and backsheet. They convert photon energy into DC electrical energy via photovoltaic effect. Types: Monocrystalline (high eff), Polycrystalline (lower cost).

    \begin{mnemonicbox}
        \mnemonic{SIMPLE: Solar panels Integrate Multiple Photovoltaic Layers Efficiently}
    \end{mnemonicbox}
\end{solutionbox}

% Question 3(b) OR
\questionmarks{3(b) OR}{4}{Give at least four technical specification of Microwave oven and washing machine each.}
\begin{solutionbox}
    \textbf{Microwave Oven:}
    \begin{itemize}
        \item Power Output: 700 - 1200 Watts
        \item Frequency: 2.45 GHz
        \item Capacity: 20 - 32 Liters
        \item Control: Digital/Touchpad/Knob
    \end{itemize}

    \textbf{Washing Machine:}
    \begin{itemize}
        \item Capacity: 6 kg - 10 kg
        \item Spin Speed: 800 - 1400 RPM
        \item Type: Top Load / Front Load
        \item Energy Rating: 5 Star
    \end{itemize}

    \begin{mnemonicbox}
        \mnemonic{CPFWS: Capacity, Power, Frequency, Washing technology, Spin speed}
    \end{mnemonicbox}
\end{solutionbox}

% Question 3(c) OR
\questionmarks{3(c) OR}{7}{Give classification of washing machine. Explain working of top load washing machine with respect to its functional block diagram and working strategy/steps to wash clothes.}
\begin{solutionbox}
    \textbf{Classification:} By loading (Top/Front), By automation (Semi/Fully), By technology (Agitator/Impeller).

    \textbf{Functional Block Diagram (Top Load):}
    \begin{figure}[H]
        \centering
        \begin{tikzpicture}[gtu flow]
            \node (Control) [gtu block] {Controller/PCB};
            \node (Inlet) [gtu block, below left=of Control] {Water Inlet};
            \node (Level) [gtu block, above left=of Control] {Level Sensor};
            \node (Motor) [gtu block, below right=of Control] {AC Motor};
            \node (Trans) [gtu block, right=of Motor] {Transmission/Clutch};
            \node (Drum) [gtu block, right=of Trans] {Drum/Agitator};
            \node (Pump) [gtu block, above right=of Control] {Drain Pump};

            \draw [gtu arrow] (Control) -- (Inlet);
            \draw [gtu arrow] (Control) <-> (Level);
            \draw [gtu arrow] (Control) -- (Motor);
            \draw [gtu arrow] (Motor) -- (Trans);
            \draw [gtu arrow] (Trans) -- (Drum);
            \draw [gtu arrow] (Control) -- (Pump);
        \end{tikzpicture}
        \caption{Washing Machine Blocks}
    \end{figure}

    \textbf{Working Steps:}
    \begin{itemize}
        \item \textbf{Fill}: Water valve opens, fills tub to set level.
        \item \textbf{Wash}: Motor rotates agitator back and forth to clean clothes.
        \item \textbf{Drain}: Pump removes dirty water.
        \item \textbf{Rinse}: Clean water fills, agitates to remove soap, then drains.
        \item \textbf{Spin}: Drum spins at high speed to extract water centrifugally.
    \end{itemize}

    \begin{mnemonicbox}
        \mnemonic{FWDRS: Fill, Wash, Drain, Rinse, Spin}
    \end{mnemonicbox}
\end{solutionbox}

% Question 4(a)
\questionmarks{4(a)}{3}{Explain working principle of laser printer. Give its technical specifications.}
\begin{solutionbox}
    \textbf{Working Principle:} Laser printers use the xerographic principle. A laser beam scans back and forth across a rotating drum, creating a static electricity pattern (latent image). The drum attracts toner powder to these charged areas. The toner is then transferred to paper and fused with heat and pressure.

    \textbf{Technical Specifications:}
    \begin{tabulary}{\linewidth}{|l|L|}
        \hline
        \textbf{Specification} & \textbf{Typical Range} \\
        \hline
        Resolution & 600 - 1200 DPI \\
        \hline
        Print Speed & 20 - 50 PPM \\
        \hline
        Memory & 64 MB - 512 MB \\
        \hline
        Duty Cycle & 10,000 - 100,000 pages/month \\
        \hline
    \end{tabulary}

    \begin{mnemonicbox}
        \mnemonic{RSCDCP: Resolution, Speed, Cycle, Duty, Connectivity, Power}
    \end{mnemonicbox}
\end{solutionbox}

% Question 4(b)
\questionmarks{4(b)}{4}{Explain working principle of Photo copier machine. State its technical specifications.}
\begin{solutionbox}
    \textbf{Working Principle:} Photocopiers use electrophotography (Xerography). Light reflected from the document discharges a charged drum in light areas, leaving a charge image in dark areas (text). Toner sticks to charged areas and transfers to paper.

    \begin{figure}[H]
        \centering
        \begin{tikzpicture}[gtu flow]
            \node (Scan) [gtu block] {Scan Document};
            \node (Charge) [gtu block, right=of Scan] {Charge Drum};
            \node (Expose) [gtu block, right=of Charge] {Expose Drum};
            \node (Develop) [gtu block, below=of Expose] {Develop (Toner)};
            \node (Transfer) [gtu block, left=of Develop] {Transfer to Paper};
            \node (Fuse) [gtu block, left=of Transfer] {Fuse/Fix};
            \node (Clean) [gtu block, left=of Fuse] {Clean Drum};

            \draw [gtu arrow] (Scan) -- (Charge);
            \draw [gtu arrow] (Charge) -- (Expose);
            \draw [gtu arrow] (Expose) -- (Develop);
            \draw [gtu arrow] (Develop) -- (Transfer);
            \draw [gtu arrow] (Transfer) -- (Fuse);
            \draw [gtu arrow] (Fuse) -- (Clean);
            \draw [gtu arrow] (Clean) |- (Charge);
        \end{tikzpicture}
        \caption{Photocopy Process Cycle}
    \end{figure}

    \textbf{Specifications:} Copy Speed (20-60 CPM), Zoom (25-400\%), Resolution (600 DPI), Paper Size (A5-A3), Warm-up time (<30s).

    \begin{mnemonicbox}
        \mnemonic{CRSPWMP: Copy speed, Resolution, Size, Paper capacity, Warm-up, Multiple copy, Power}
    \end{mnemonicbox}
\end{solutionbox}

% Question 4(c)
\questionmarks{4(c)}{7}{Draw and explain schematic of wireless CCTV camera system. Explain Network video recorder. State types of camera used in CCTV system and explain any one of them.}
\begin{solutionbox}
    \textbf{Wireless CCTV Schematic:}
    \begin{figure}[H]
        \centering
        \begin{tikzpicture}[gtu flow]
            \node (Cam) [gtu block] {IP Camera};
            \node (Proc) [gtu block, right=of Cam] {Signal Proc};
            \node (Enc) [gtu block, right=of Proc] {Encoder};
            \node (Tx) [gtu block, right=of Enc] {Wi-Fi Tx};
            \node (Rx) [gtu block, below=of Tx] {Wi-Fi Rx};
            \node (Router) [gtu block, left=of Rx] {Wireless Router};
            \node (NVR) [gtu block, left=of Router] {NVR};
            \node (Monitor) [gtu block, left=of NVR] {Monitor};
            \node (Net) [gtu block, below=of Router] {Internet};
            \node (Remote) [gtu block, right=of Net] {Remote User};

            \draw [gtu arrow] (Cam) -- (Proc);
            \draw [gtu arrow] (Proc) -- (Enc);
            \draw [gtu arrow] (Enc) -- (Tx);
            \draw [gtu arrow] (Tx) -- node[right] {RF} (Rx);
            \draw [gtu arrow] (Rx) -- (Router);
            \draw [gtu arrow] (Router) -- (NVR);
            \draw [gtu arrow] (NVR) -- (Monitor);
            \draw [gtu arrow] (Router) -- (Net);
            \draw [gtu arrow] (Net) -- (Remote);
        \end{tikzpicture}
        \caption{Wireless CCTV System}
    \end{figure}

    \textbf{NVR (Network Video Recorder):} Records digital streams from IP cameras over network. Features: Remote access, high resolution support, intelligent analytics, storage management.

    \textbf{Camera Types:} Dome, Bullet, PTZ, Box, Thermal, 360-Fisheye.\\
    \textbf{IP Camera:} Connects to network, has own IP address, processes images digitally, supports PoE, higher resolution than analog.

    \begin{mnemonicbox}
        \mnemonic{WISP-NET: Wireless Images Securely Processed, Networked, Enabling Tracking}
    \end{mnemonicbox}
\end{solutionbox}

% Question 4(a) OR
\questionmarks{4(a) OR}{3}{Explain working principle of inkjet printer. Give its technical specifications.}
\begin{solutionbox}
    \textbf{Working Principle:} Propels tiny droplets of liquid ink onto paper.
    \begin{figure}[H]
        \centering
        \begin{tikzpicture}[gtu flow]
            \node (Data) [gtu block] {Print Data};
            \node (Control) [gtu block, right=of Data] {Controller};
            \node (Head) [gtu block, right=of Control] {Printhead};
            \node (Nozzle) [gtu block, right=of Head] {Nozzles};
            \node (Paper) [gtu block, below=of Nozzle] {Paper};
            \node (Image) [gtu block, left=of Paper] {Image};

            \draw [gtu arrow] (Data) -- (Control);
            \draw [gtu arrow] (Control) -- (Head);
            \draw [gtu arrow] (Head) -- (Nozzle);
            \draw [gtu arrow] (Nozzle) -- node[right] {Ink Drops} (Paper);
            \draw [gtu arrow] (Paper) -- (Image);
        \end{tikzpicture}
        \caption{Inkjet Printing Process}
    \end{figure}
    Includes thermal bubble or piezoelectric technology to eject ink.

    \textbf{Specifications:} Resolution (4800 DPI), Speed (10-20 PPM), Ink (Dye/Pigment), Connectivity (USB/Wi-Fi).

    \begin{mnemonicbox}
        \mnemonic{RIPS-CCD: Resolution, Ink type, Print speed, Size of droplet, Connectivity, Capacity, Droplet}
    \end{mnemonicbox}
\end{solutionbox}

% Question 4(b) OR
\questionmarks{4(b) OR}{4}{Explain maintenance and trouble shooting of television receiver and Washing machine.}
\begin{solutionbox}
    \textbf{Television Maintenance:}
    \begin{itemize}
        \item Clean screen with microfiber cloth.
        \item Ensure proper ventilation to prevent overheating.
        \item Check cable connections periodically.
    \end{itemize}
    \textit{Troubleshooting:} No power? Check fuse/cable. No sound? Check mute/speakers. Poor picture? Adjust antenna/settings.

    \textbf{Washing Machine Maintenance:}
    \begin{itemize}
        \item Clean lint filter and door seal regularly.
        \item Descale drum to remove limescale.
        \item Keep door open after use to prevent mold.
    \end{itemize}
    \textit{Troubleshooting:} Not draining? Check pump filter. Leaking? Check hoses. Vibration? Level the feet.

    \begin{mnemonicbox}
        \mnemonic{CREST: Clean Regularly, Examine connections, Service filters, Test functions}
    \end{mnemonicbox}
\end{solutionbox}

% Question 4(c) OR
\questionmarks{4(c) OR}{7}{Define CCTV. Explain with schematic CCTV camera system installed in a home. Describe analog camera, Digital camera and IP camera and differentiate them.}
\begin{solutionbox}
    \textbf{CCTV:} Closed-Circuit Television, a system where video is transmitted to a limited set of monitors for surveillance.

    \textbf{Home System Schematic:}
    \begin{figure}[H]
        \centering
        \begin{tikzpicture}[gtu flow]
            \node (Cam1) [gtu block] {Cam 1};
            \node (Cam2) [gtu block, below=of Cam1] {Cam 2};
            \node (DVR) [gtu block, right=of Cam1] {DVR/NVR}; % Simplified placement
            \node (HDD) [gtu block, above=of DVR] {Storage};
            \node (Mon) [gtu block, right=of DVR] {Monitor};
            \node (Router) [gtu block, below=of DVR] {Router};
            \node (Mobile) [gtu block, right=of Router] {Phone App};

            \draw [gtu arrow] (Cam1) -- (DVR);
            \draw [gtu arrow] (Cam2) -| (DVR);
            \draw [gtu arrow] (DVR) <-> (HDD);
            \draw [gtu arrow] (DVR) -- (Mon);
            \draw [gtu arrow] (DVR) -- (Router);
            \draw [gtu arrow] (Router) -- (Mobile);
        \end{tikzpicture}
        \caption{Home CCTV Layout}
    \end{figure}

    \begin{tabulary}{\linewidth}{|l|L|L|L|}
        \hline
        \textbf{Feature} & \textbf{Analog} & \textbf{Digital (HD-TVI)} & \textbf{IP Camera} \\
        \hline
        Signal & Analog (Coax) & Digital via Coax & Digital (Ethernet) \\
        \hline
        Resolution & SD (Low) & HD (Medium) & UHD (High) \\
        \hline
        Cabling & Coax (RG59) & Coax & Cat5e/Cat6 \\
        \hline
        Intelligent & No & Limited & Advanced Analytics \\
        \hline
    \end{tabulary}

    \begin{mnemonicbox}
        \mnemonic{ADI: Analog uses Decaying technology, IP represents Innovation}
    \end{mnemonicbox}
\end{solutionbox}

% Question 5(a)
\questionmarks{5(a)}{3}{Define maintenance. State its types. Explain any one of them.}
\begin{solutionbox}
    \textbf{Maintenance:} Process of preserving equipment in good working order to prevent failure.

    \textbf{Types:} Preventive, Predictive, Corrective (Breakdown), Condition-based.

    \textbf{Preventive Maintenance:} Scheduled servicing performed at regular intervals (regardless of condition) to prevent unexpected failures. Example: Weekly cleaning, monthly oiling.

    \begin{mnemonicbox}
        \mnemonic{PPCR: Prevent Problems through Checkups Regularly}
    \end{mnemonicbox}
\end{solutionbox}

% Question 5(b)
\questionmarks{5(b)}{4}{Explaining maintenance of PA systems and Home theatre system.}
\begin{solutionbox}
    \textbf{PA System Maintenance:}
    \begin{itemize}
        \item \textbf{Cables}: Check for cuts/loose connectors. Coil properly.
        \item \textbf{Mics}: Clean grills, check for moisture damage.
        \item \textbf{Amps}: Clean vents to prevent overheating.
    \end{itemize}

    \textbf{Home Theatre Maintenance:}
    \begin{itemize}
        \item \textbf{Dusting}: Accumulation causes heat/static.
        \item \textbf{Ventilation}: Ensure receivers have airflow.
        \item \textbf{Connections}: Re-plug HDMI/Audio cables to remove oxidation.
        \item \textbf{Calibration}: Re-run auto-setup periodically.
    \end{itemize}

    \begin{mnemonicbox}
        \mnemonic{CAVS: Clean, Adjust, Verify connections, Service regularly}
    \end{mnemonicbox}
\end{solutionbox}

% Question 5(c)
\questionmarks{5(c)}{7}{Draw and Explain block diagram of DTH technology. Discuss hardware components used in DTH system. Discuss various modern features currently provided in current DTH system.}
\begin{solutionbox}
    \textbf{DTH Block Diagram:}
    \begin{figure}[H]
        \centering
        \begin{tikzpicture}[gtu flow]
            \node (Source) [gtu block] {Broadcaster};
            \node (Uplink) [gtu block, right=of Source] {Uplink Station};
            \node (Sat) [gtu block, right=of Uplink] {Satellite (Ku Band)};
            \node (Dish) [gtu block, below=of Sat] {Dish Antenna};
            \node (STB) [gtu block, left=of Dish] {Set-Top Box};
            \node (TV) [gtu block, left=of STB] {TV};

            \draw [gtu arrow] (Source) -- (Uplink);
            \draw [gtu arrow] (Uplink) -- node[above] {Tx (14 GHz)} (Sat);
            \draw [gtu arrow] (Sat) -- node[right] {Rx (11-12 GHz)} (Dish);
            \draw [gtu arrow] (Dish) -- (STB);
            \draw [gtu arrow] (STB) -- (TV);
        \end{tikzpicture}
        \caption{DTH Transmission System}
    \end{figure}

    \textbf{Hardware:} Dish Antenna (Parabolic reflector), LNB (Low Noise Block downconverter), Coaxial Cable, Set-Top Box (Decoder + Smart Card), Remote.

    \textbf{Features:} HD/4K support, Recording (DVR), Pause/Rewind Live TV, Interactive Apps, Video on Demand.

    \begin{mnemonicbox}
        \mnemonic{DISH-STB: Direct Information Satellite Hub - Signals Transmitted to Box}
    \end{mnemonicbox}
\end{solutionbox}

% Question 5(a) OR
\questionmarks{5(a) OR}{3}{Differentiate between predictive and preventive maintenance.}
\begin{solutionbox}
    \begin{tabulary}{\linewidth}{|l|L|L|}
        \hline
        \textbf{Aspect} & \textbf{Preventive} & \textbf{Predictive} \\
        \hline
        Basis & Time/Schedule & Actual Condition \\
        \hline
        Trigger & Fixed Interval & Data/Warning Signs \\
        \hline
        Cost & Medium (may replace good parts) & Low long-term (max life) \\
        \hline
        Example & Change oil every 5000km & Change oil when sensor detects dirt \\
        \hline
    \end{tabulary}

    \begin{mnemonicbox}
        \mnemonic{TIME vs DATA: Timed Intervals Maintenance Everywhere vs Data Analysis Triggers Action}
    \end{mnemonicbox}
\end{solutionbox}

% Question 5(b) OR
\questionmarks{5(b) OR}{4}{Describe troubleshooting procedure and safety precautions for microwave oven.}
\begin{solutionbox}
    \textbf{Troubleshooting Procedure:}
    \begin{enumerate}
        \item \textbf{No Power}: Check fuse, thermal cutout, door switches.
        \item \textbf{Not Heating}: Check magnetron, HV diode, HV capacitor.
        \item \textbf{Sparks/Arcing}: Check waveguide cover, remove metal objects, check paint damage.
    \end{enumerate}

    \textbf{Safety Precautions:}
    \begin{itemize}
        \item Always discharge HV capacitor before touching components (store 2000V+).
        \item Check for radiation leakage after reassembly.
        \item Never bypass door interlock switches.
        \item Do not operate with door open.
    \end{itemize}

    \begin{mnemonicbox}
        \mnemonic{DUEL-SAFE: Disconnect power, Use discharge tool, Examine systematically, Look for damage - Safety Always First, Every time}
    \end{mnemonicbox}
\end{solutionbox}

% Question 5(c) OR
\questionmarks{5(c) OR}{7}{Draw and explain block diagram of PA system. Discuss design parameters while designing for auditorium. Draw connection diagram of four 8 Ohm speakers to PA system amplifier having 8 Ohm as output impedance.}
\begin{solutionbox}
    \textbf{PA System Block Diagram:}
    \begin{figure}[H]
        \centering
        \begin{tikzpicture}[gtu flow]
            \node (Mics) [gtu block] {Mics/Inputs};
            \node (Mixer) [gtu block, right=of Mics] {Mixer/Pre-Amp};
            \node (Eq) [gtu block, right=of Mixer] {Equalizer};
            \node (Amp) [gtu block, below=of Eq] {Power Amp};
            \node (Spk) [gtu block, left=of Amp] {Speakers};

            \draw [gtu arrow] (Mics) -- (Mixer);
            \draw [gtu arrow] (Mixer) -- (Eq);
            \draw [gtu arrow] (Eq) -- (Amp);
            \draw [gtu arrow] (Amp) -- (Spk);
        \end{tikzpicture}
        \caption{Public Address System}
    \end{figure}

    \textbf{Auditorium Design Parameters:} Acoustics (Reverb time), Coverage (Speaker placement), Intelligibility (STI), Power (Watts per seat), Feedback control.

    \textbf{Speaker Connection (Series-Parallel):}
    Target: 8$\Omega$ total load from four 8$\Omega$ speakers.
    \begin{figure}[H]
        \centering
        \begin{tikzpicture}[gtu flow]
            \node (AmpOut) [gtu block] {Amp Output (8$\Omega$)};
            % Branch 1
            \node (S1) [gtu block, below left=of AmpOut] {S1 (8$\Omega$)};
            \node (S2) [gtu block, below=of S1] {S2 (8$\Omega$)};
            % Branch 2
            \node (S3) [gtu block, below right=of AmpOut] {S3 (8$\Omega$)};
            \node (S4) [gtu block, below=of S3] {S4 (8$\Omega$)};

            % Wires
            \draw [gtu arrow] (AmpOut) -- (0,-1) -| (S1); % Simplified wiring viz
            \draw [gtu arrow] (AmpOut) -- (0,-1) -| (S3);
            \draw [gtu arrow] (S1) -- (S2);
            \draw [gtu arrow] (S3) -- (S4);
            % Return path implied or drawn explicitly
        \end{tikzpicture}
        \caption{Series-Parallel: (8+8) || (8+8) = 16 || 16 = 8$\Omega$}
    \end{figure}
    Connect pair 1 in series ($8+8=16\Omega$). Connect pair 2 in series ($8+8=16\Omega$). Connect these two pairs in parallel ($16 || 16 = 8\Omega$). Matches amp impedance perfectly.

    \begin{mnemonicbox}
        \mnemonic{PASS: Proper Amplification, Speaker placement, Series-parallel wiring}
    \end{mnemonicbox}
\end{solutionbox}

\end{document}

