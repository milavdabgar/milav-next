\documentclass{article}

% content/resources/templates/preamble.tex
\usepackage[margin=0.6in]{geometry}
\author{Milav Dabgar}
\usepackage{amsmath,amssymb,amsthm}
\usepackage{booktabs}
\usepackage{multirow}
\usepackage{xcolor}
\usepackage{tcolorbox}
\tcbuselibrary{breakable,skins}
\usepackage[colorlinks=true,linkcolor=blue]{hyperref}
\usepackage{titlesec}
\usepackage{enumitem}
\usepackage{tikz}
\usepackage{pgfplots}
\usepackage{circuitikz}
\usepackage[version=4]{mhchem}
\usepackage{longtable}
\usepackage{array}
\usepackage{float}
\usepackage{caption}
\usepackage{listings}

\lstset{
  basicstyle=\small\ttfamily,
  breaklines=true,
  breakatwhitespace=false,
  postbreak=\mbox{\textcolor{red}{$\hookrightarrow$}\space},
  float=false,
  numbers=left,
  numberstyle=\tiny\color{gray},
  numbersep=10pt,
  xleftmargin=2em,
  keywordstyle=\color{blue},
  commentstyle=\color{green!60!black},
  stringstyle=\color{purple},
  backgroundcolor=\color{gray!5},
  showstringspaces=false,
  tabsize=2,
  captionpos=b,
  keepspaces=true,
  columns=flexible
}

\pgfplotsset{compat=1.18}
\usetikzlibrary{shapes,arrows,positioning,calc,patterns,decorations.pathmorphing,decorations.markings,arrows.meta}

% Color scheme
\definecolor{headcolor}{RGB}{0,102,204}
\definecolor{keycolor}{RGB}{220,20,60}
\definecolor{solutioncolor}{RGB}{34,139,34}
\definecolor{mnemoniccolor}{RGB}{148,0,211}
\definecolor{codecolor}{RGB}{0,0,100}

% Spacing
\setlength{\parskip}{3pt}
\setlist[itemize]{nosep}
\setlist[enumerate]{nosep}

% Title formatting
\titleformat{\section}{\Large\bfseries\color{headcolor}}{\thesection}{1em}{}
\titleformat{\subsection}{\large\bfseries\color{headcolor}}{\thesubsection}{1em}{}

% Pandoc tightlist compatibility
\providecommand{\tightlist}{%
  \setlength{\itemsep}{0pt}\setlength{\parskip}{0pt}}

% Pandoc longtable compatibility
\newcounter{none}
\def\thenone{}


% content/resources/templates/gujarati-boxes.tex
\usepackage{fontspec}
\usepackage{polyglossia}

% Set Gujarati as main language (document is primarily in Gujarati)
% Note: gloss-gujarati.ldf doesn't exist in polyglossia, but it will use hyphenation patterns
\setdefaultlanguage{gujarati}
\setotherlanguage{english}

% Configure Gujarati font properly
% Use Language=Default to prevent polyglossia from trying to add language-specific features
% that don't exist for Gujarati, which causes "empty feature" warnings
\newfontfamily\gujaratifont[Script=Gujarati,AutoFakeBold=2.5,AutoFakeSlant=0.3]{Noto Sans Gujarati}
\setmainfont[Script=Gujarati,AutoFakeBold=2.5,AutoFakeSlant=0.3]{Noto Sans Gujarati}
% Use Noto Sans Gujarati for monospace to support Gujarati in text
\setmonofont[Scale=0.9]{Noto Sans Gujarati}

% Configure English to use the same font
\newfontfamily\englishfont[Script=Gujarati,AutoFakeBold=2.5,AutoFakeSlant=0.3]{Noto Sans Gujarati}

% Translations for polyglossia
\gappto\captionsgujarati{
  \renewcommand{\tablename}{કોષ્ટક}
  \renewcommand{\figurename}{આકૃતિ}
}

% Helper for TikZ nodes to ensure Gujarati font
\newcommand{\gu}[1]{{\gujaratifont #1}}

% Custom environments
\newtcolorbox{solutionbox}{
    breakable,
    enhanced,
    colback=solutioncolor!5!white,
    colframe=solutioncolor!75!black,
    fonttitle=\bfseries,
    title=જવાબ
}

\newtcolorbox{solutionboxnobreak}{
 colback=solutioncolor!5!white,
 colframe=solutioncolor!75!black,
 fonttitle=\bfseries,
 title=જવાબ
}

\newtcolorbox{keyformula}{
 breakable,
 enhanced,
 colback=keycolor!5!white,
 colframe=keycolor!75!black,
 fonttitle=\bfseries,
 title=રાસાયણિક સમીકરણ/સૂત્ર
}

\newtcolorbox{mnemonicbox}{
 breakable,
 enhanced,
 colback=mnemoniccolor!5!white,
 colframe=mnemoniccolor!75!black,
 fonttitle=\bfseries,
 title=મેમરી ટ્રીક
}


% Custom commands for GTU solutions
% This file defines semantic commands for consistent formatting

% Question command with automatic formatting
\newcommand{\question}[2]{%
  \section*{Question #1}%
  \textbf{#2}%
}

% OR question variant
\newcommand{\questionor}[2]{%
  \section*{Question #1 OR}%
  \textbf{#2}%
}

% Proper table environment with caption
\newenvironment{answertable}[1]{%
  \begin{table}[htbp]
  \centering
  \caption{#1}
}{%
  \end{table}
}

% Proper figure environment for diagrams
\newenvironment{answerdiagram}[1]{%
  \begin{figure}[htbp]
  \centering
  \caption{#1}
}{%
  \end{figure}
}

% Semantic markup for key terms
\newcommand{\keyword}[1]{\textbf{#1}}
\newcommand{\code}[1]{\texttt{#1}}
\newcommand{\classname}[1]{\texttt{#1}}
\newcommand{\methodname}[1]{\texttt{#1}}

% Proper quotation marks
\newcommand{\mnemonic}[1]{``#1''}


\title{Consumer Electronics \& Maintenance (4341107) - વિન્ટર ૨૦૨૩ સોલ્યુશન}
\date{૨૯ જાન્યુઆરી, ૨૦૨૪}

\begin{document}
\maketitle


\questionmarks{1(a)}{3}{વિવિધ પ્રકારના મેન્ટેનન્સ વિશે સંક્ષિપ્તમાં સમજાવો.}
\begin{solutionbox}
    \begin{tabulary}{\linewidth}{|l|L|}
        \hline
        \textbf{મેન્ટેનન્સનો પ્રકાર} & \textbf{વર્ણન} \\
        \hline
        \textbf{પ્રિવેન્ટિવ મેન્ટેનન્સ} & બ્રેકડાઉન અટકાવવા માટે નિયમિત તપાસ અને સર્વિસિંગ \\
        \hline
        \textbf{કરેક્ટિવ મેન્ટેનન્સ} & સાધનોની નિષ્ફળતા પછી કાર્યક્ષમતા પુનઃસ્થાપિત કરવા માટે કરવામાં આવતી મરામત \\
        \hline
        \textbf{પ્રેડિક્ટિવ મેન્ટેનન્સ} & મેન્ટેનન્સ ક્યારે જરૂરી છે તે આગાહી કરવા માટે કંડિશન મોનિટરિંગનો ઉપયોગ કરે છે \\
        \hline
    \end{tabulary}

    \begin{mnemonicbox}
        \mnemonic{PCPro: પ્રિવેન્ટિવ અટકાવે છે, કરેક્ટિવ ઈલાજ કરે છે, પ્રેડિક્ટિવ આગાહી કરે છે}
    \end{mnemonicbox}
\end{solutionbox}

\questionmarks{1(b)}{4}{વોશિંગ મશીનની મેન્ટેનન્સ પ્રક્રિયા સમજાવો.}
\begin{solutionbox}
    \textbf{મેન્ટેનન્સ પ્રક્રિયા:}
    
    \begin{figure}[H]
        \centering
        \begin{tikzpicture}[gtu flow]
            \node (Insp) [gtu block] {નિયમિત તપાસ};
            \node (Filter) [gtu block, right=of Insp] {ફિલ્ટર સાફ કરો};
            \node (Hose) [gtu block, right=of Filter] {હોસ તપાસો};
            \node (Load) [gtu block, below=of Filter] {લોડ બેલેન્સ કરો};
            \node (Drum) [gtu block, left=of Load] {ડ્રમ સાફ કરો};

            \draw [gtu arrow] (Insp) -- (Filter);
            \draw [gtu arrow] (Filter) -- (Hose);
            \draw [gtu arrow] (Hose) -- (Load);
            \draw [gtu arrow] (Load) -- (Drum);
        \end{tikzpicture}
        \caption{વોશિંગ મશીન મેન્ટેનન્સ સ્ટેપ્સ}
    \end{figure}

    \begin{itemize}
        \item \textbf{ફિલ્ટર સફાઈ}: દર મહિને લિંટ ફિલ્ટર કાઢો અને સાફ કરો
        \item \textbf{હોસ તપાસ}: દર 3 મહિને તિરાડો અને લીકેજ માટે તપાસો
        \item \textbf{લોડ વિતરણ}: કંપન અટકાવવા માટે યોગ્ય સંતુલન સુનિશ્ચિત કરો
        \item \textbf{ડ્રમ સફાઈ}: સરકા (vinegar) સાથે ખાલી ગરમ પાણીની સાયકલ ચલાવો
    \end{itemize}

    \begin{mnemonicbox}
        \mnemonic{FHLD: ફિલ્ટર્સ, હોસિસ, લોડ્સ, ડ્રમ નિયમિત ધ્યાન માંગે છે}
    \end{mnemonicbox}
\end{solutionbox}

\questionmarks{1(c)}{7}{માઇક્રોવેવ ઓવનની મેન્ટેનન્સ અને ટ્રબલશૂટિંગ પ્રક્રિયા સમજાવો.}
\begin{solutionbox}
    \textbf{મેન્ટેનન્સ પ્રક્રિયાઓ:}
    \begin{tabulary}{\linewidth}{|l|L|l|}
        \hline
        \textbf{કાર્ય} & \textbf{પ્રક્રિયા} & \textbf{આવર્તન} \\
        \hline
        બાહ્ય સફાઈ & હળવા ડિટર્જન્ટથી સાફ કરો & સાપ્તાહિક \\
        \hline
        આંતરિક સફાઈ & ખોરાકના કણો અને ગ્રીસ સાફ કરો & દરેક સ્પિલ પછી \\
        \hline
        દરવાજાના સીલની તપાસ & નુકસાન અથવા લીકેજ માટે તપાસો & માસિક \\
        \hline
        વેન્ટિલેશન તપાસ & ખાતરી કરો કે વેન્ટ્સ અવરોધાયેલ નથી & માસિક \\
        \hline
    \end{tabulary}

    \textbf{ટ્રબલશૂટિંગ:}
    
    \begin{figure}[H]
        \centering
        \begin{tikzpicture}[gtu flow]
            \node (Power) [gtu decision] {પાવર નથી?};
            \node (ChkPwr) [gtu block, right=of Power] {કનેક્શન તપાસો};
            \node (Heat) [gtu decision, below=2cm of Power] {ગરમ નથી થતું?};
            \node (ChkHeat) [gtu block, right=of Heat] {મેગ્નેટ્રોન તપાસો};
            \node (Uneven) [gtu decision, below=2cm of Heat] {અસમાન રસોઈ?};
            \node (ChkTurn) [gtu block, right=of Uneven] {ટર્નટેબલ તપાસો};

            \draw [gtu arrow] (Power) -- node[above] {હા} (ChkPwr);
            \draw [gtu arrow] (Heat) -- node[above] {હા} (ChkHeat);
            \draw [gtu arrow] (Uneven) -- node[above] {હા} (ChkTurn);
        \end{tikzpicture}
        \caption{માઇક્રોવેવ ટ્રબલશૂટિંગ ફ્લો}
    \end{figure}

    \begin{itemize}
        \item \textbf{પાવર સમસ્યાઓ}: ફ્યુઝ, સર્કિટ બ્રેકર અને કોર્ડ તપાસો
        \item \textbf{હીટિંગ સમસ્યાઓ}: ડોર સ્વિચ, હાઈ વોલ્ટેજ કેપેસિટર, મેગ્નેટ્રોન ટેસ્ટ કરો
        \item \textbf{સલામતી પ્રથમ}: ક્ષતિગ્રસ્ત દરવાજા અથવા સીલ સાથે ક્યારેય ઓપરેટ કરશો નહીં
    \end{itemize}

    \begin{mnemonicbox}
        \mnemonic{POWER: પાવર, ઓવન ઈન્ટીરીયર, વાયરીંગ, ઈલેક્ટ્રોનિક્સ, રેડિયેશન સીલ}
    \end{mnemonicbox}
\end{solutionbox}

\questionmarks{1(c) OR}{7}{પ્રોજેક્ટરની મેન્ટેનન્સ અને ટ્રબલશૂટિંગ પ્રક્રિયા સમજાવો.}
\begin{solutionbox}
    \textbf{મેન્ટેનન્સ પ્રક્રિયાઓ:}
    \begin{tabulary}{\linewidth}{|l|L|l|}
        \hline
        \textbf{કાર્ય} & \textbf{પ્રક્રિયા} & \textbf{આવર્તન} \\
        \hline
        લેન્સ સફાઈ & લેન્સ ક્લોથ અને સોલ્યુશન વાપરો & માસિક \\
        \hline
        ફિલ્ટર સફાઈ & ધૂળ કાઢો અને સાફ કરો & દર 100 કલાકે \\
        \hline
        લેમ્પ તપાસ & ડિસ્કલરેશન/ઝાંખાપણું તપાસો & દર 300 કલાકે \\
        \hline
        વેન્ટિલેશન & યોગ્ય એરફ્લો સુનિશ્ચિત કરો & દરેક ઉપયોગ પહેલાં \\
        \hline
    \end{tabulary}

    \textbf{ટ્રબલશૂટિંગ:}
    \begin{itemize}
        \item \textbf{ઇમેજ સમસ્યાઓ}: ફોકસ, રિઝોલ્યુશન, કીસ્ટોન કરેક્શન એડજસ્ટ કરો
        \item \textbf{લેમ્પ સમસ્યાઓ}: લેમ્પ કલાકો તપાસો, જો મર્યાદા વટાવી ગયેલ હોય તો બદલો
        \item \textbf{કનેક્ટિવિટી}: ઇનપુટ સોર્સ અને કેબલ કનેક્શન ચકાસો
        \item \textbf{થર્મલ સમસ્યાઓ}: ફિલ્ટર્સ સાફ કરો અને યોગ્ય વેન્ટિલેશન સુનિશ્ચિત કરો
    \end{itemize}

    \begin{mnemonicbox}
        \mnemonic{FLAMVE: ફિલ્ટર્સ, લેમ્પ, એરફ્લો, માઉન્ટિંગ, વોલ્ટેજ, એન્વાયર્નમેન્ટ}
    \end{mnemonicbox}
\end{solutionbox}

\questionmarks{2(a)}{3}{ટૂંકમાં સમજાવો: (1) હ્યુ (Hue) (2) બ્રાઇટનેસ (Brightness)}
\begin{solutionbox}
    \begin{tabulary}{\linewidth}{|l|L|}
        \hline
        \textbf{શબ્દ} & \textbf{વર્ણન} \\
        \hline
        \textbf{હ્યુ (Hue)} & શુદ્ધ રંગ લક્ષણ જે પ્રકાશની તરંગલંબાઇના આધારે રંગો (લાલ, લીલો, વાદળી, વગેરે) ને અલગ પાડે છે \\
        \hline
        \textbf{બ્રાઇટનેસ} & રંગમાંથી ઉત્સર્જિત અથવા પરાવર્તિત પ્રકાશનું પ્રમાણ, જે નક્કી કરે છે કે તે કેટલો હળવો કે ઘેરો દેખાય છે \\
        \hline
    \end{tabulary}

    \begin{figure}[H]
        \centering
        \begin{tikzpicture}[gtu flow]
            \draw[<->, thick] (-2,0) -- (2,0) node[right] {બ્રાઇટનેસ (લાઇટનેસ)};
            \draw[<->, thick] (0,-2) -- (0,2) node[above] {હ્યુ (કલર ટાઇપ)};
            \node at (0,0) {કલર સ્પેસ};
        \end{tikzpicture}
        \caption{હ્યુ વિ બ્રાઇટનેસ}
    \end{figure}

    \begin{mnemonicbox}
        \mnemonic{HB-WC: હ્યુ નક્કી કરે છે કયો કલર, બ્રાઇટનેસ નક્કી કરે છે વ્હાઇટ-ટુ-બ્લેક લેવલ}
    \end{mnemonicbox}
\end{solutionbox}

\questionmarks{2(b)}{4}{LCD TV પર ટૂંકનોંધ લખો.}
\begin{solutionbox}
    \textbf{LCD TV ટેકનોલોજી:}
    \begin{itemize}
        \item \textbf{કાર્ય સિદ્ધાંત}: લિક્વિડ ક્રિસ્ટલ્સનો ઉપયોગ કરે છે જે બેકએન્ડ બેકલાઇટ સોર્સ દ્વારા પૂરા પાડવામાં આવેલ પ્રકાશને પસાર થવા/રોકવા માટે ટ્વિસ્ટ/અનટ્વિસ્ટ થાય છે.
        \item \textbf{મુખ્ય ઘટકો}: બેકલાઇટ, પોલરાઇઝિંગ ફિલ્ટર્સ, લિક્વિડ ક્રિસ્ટલ મેટ્રિક્સ, કલર ફિલ્ટર્સ.
        \item \textbf{ફાયદા}: પાતળી પ્રોફાઇલ, ઊર્જા કાર્યક્ષમ, રેડિયેશન નથી, શાર્પ ઇમેજ.
        \item \textbf{મર્યાદાઓ}: મર્યાદિત વ્યુઇંગ એંગલ, નવી ટેકનોલોજી (OLED) કરતાં ધીમો રિસ્પોન્સ ટાઇમ.
    \end{itemize}

    \begin{figure}[H]
        \centering
        \begin{tikzpicture}[gtu flow]
            \node (Backlight) [gtu block] {બેકલાઇટ};
            \node (Pol1) [gtu block, right=of Backlight] {પોલરાઇઝિંગ ફિલ્ટર};
            \node (LC) [gtu block, right=of Pol1] {લિક્વિડ ક્રિસ્ટલ};
            \node (Color) [gtu block, right=of LC] {કલર ફિલ્ટર};
            \node (Screen) [gtu block, right=of Color] {સ્ક્રીન};

            \draw [gtu arrow] (Backlight) -- (Pol1);
            \draw [gtu arrow] (Pol1) -- (LC);
            \draw [gtu arrow] (LC) -- (Color);
            \draw [gtu arrow] (Color) -- (Screen);
        \end{tikzpicture}
        \caption{LCD TV લેયર્સ}
    \end{figure}

    \begin{mnemonicbox}
        \mnemonic{BPLCS: બેકલાઇટ પાસિસ લાઈટ થ્રુ ક્રિસ્ટલ્સ ટુ સ્ક્રીન}
    \end{mnemonicbox}
\end{solutionbox}

\questionmarks{2(c)}{7}{DTH રિસીવરનો બ્લોક ડાયાગ્રામ દોરો અને સમજાવો.}
\begin{solutionbox}
    \textbf{DTH રિસીવર બ્લોક ડાયાગ્રામ:}
    
    \begin{figure}[H]
        \centering
        \begin{tikzpicture}[gtu flow]
            \node (Dish) [gtu block] {સેટેલાઇટ ડિશ};
            \node (LNB) [gtu block, right=of Dish] {LNB};
            \node (Tuner) [gtu block, right=of LNB] {ટ્યુનર};
            \node (Demod) [gtu block, below=of Tuner] {ડિમોડ્યુલેટર};
            \node (MPEG) [gtu block, left=of Demod] {MPEG ડીકોડર};
            \node (Proc) [gtu block, left=of MPEG] {ઓડિયો/વિડિયો પ્રોસેસર};
            \node (TV) [gtu block, left=of Proc] {TV ડિસ્પ્લે};

            \draw [gtu arrow] (Dish) -- (LNB);
            \draw [gtu arrow] (LNB) -- (Tuner);
            \draw [gtu arrow] (Tuner) -- (Demod);
            \draw [gtu arrow] (Demod) -- (MPEG);
            \draw [gtu arrow] (MPEG) -- (Proc);
            \draw [gtu arrow] (Proc) -- (TV);
        \end{tikzpicture}
        \caption{DTH રિસીવર સિસ્ટમ}
    \end{figure}

    \begin{itemize}
        \item \textbf{સેટેલાઇટ ડિશ}: ઉપગ્રહમાંથી સિગ્નલ કેપ્ચર કરે છે.
        \item \textbf{LNB (Low Noise Block)}: ઉચ્ચ આવર્તન સિગ્નલોને ઓછી આવર્તનમાં રૂપાંતરિત કરે છે.
        \item \textbf{ટ્યુનર}: ચોક્કસ ચેનલ આવર્તન પસંદ કરે છે.
        \item \textbf{ડિમોડ્યુલેટર}: કેરિયર સિગ્નલમાંથી ડિજિટલ માહિતી મેળવે છે.
        \item \textbf{MPEG ડીકોડર}: વિડિયો/ઓડિયો ડેટાને ડીકોમ્પ્રેસ કરે છે.
        \item \textbf{કંડિશનલ એક્સેસ મોડ્યુલ}: સબસ્ક્રિપ્શન એક્સેસ નિયંત્રિત કરે છે.
        \item \textbf{માઈક્રોકંટ્રોલર}: સમગ્ર કામગીરી અને વપરાશકર્તા ઇનપુટ્સનું નિયંત્રણ કરે છે.
    \end{itemize}

    \begin{mnemonicbox}
        \mnemonic{SLTDMP: સેટેલાઇટ, LNB, ટ્યુનર, ડિમોડ્યુલેટર, MPEG, પ્રોસેસર}
    \end{mnemonicbox}
\end{solutionbox}

\questionmarks{2(a) OR}{3}{ટૂંકમાં સમજાવો: (1) લ્યુમિનન્સ (2) ક્રોમિનન્સ}
\begin{solutionbox}
    \begin{tabulary}{\linewidth}{|l|L|}
        \hline
        \textbf{શબ્દ} & \textbf{વર્ણન} \\
        \hline
        \textbf{લ્યુમિનન્સ} & વિડિયો સિગ્નલ (Y) નો બ્રાઇટનેસ અથવા તીવ્રતાનો ઘટક જે બ્લેક અને વ્હાઇટ માહિતી વહન કરે છે. \\
        \hline
        \textbf{ક્રોમિનન્સ} & વિડિયો સિગ્નલ (Cb, Cr) નો રંગ ઘટક જે હ્યુ અને સેચ્યુરેશન માહિતી વહન કરે છે. \\
        \hline
    \end{tabulary}

    \begin{mnemonicbox}
        \mnemonic{LC-BH: લ્યુમિનન્સ કંટ્રોલ્સ બ્રાઇટનેસ, ક્રોમિનન્સ કંટ્રોલ્સ હ્યુ}
    \end{mnemonicbox}
\end{solutionbox}

\questionmarks{2(b) OR}{4}{ગ્રાસમેનનો નિયમ (Grassman's law) સમજાવો.}
\begin{solutionbox}
    \textbf{કલર મિક્સિંગના ગ્રાસમેનના નિયમો:}
    \begin{tabulary}{\linewidth}{|l|L|}
        \hline
        \textbf{નિયમ} & \textbf{વર્ણન} \\
        \hline
        \textbf{સપ્રમાણતા (Symmetry)} & જો કલર A કલર B સાથે મેચ થાય, તો B એ A સાથે મેચ થાય \\
        \hline
        \textbf{પ્રોપોર્શનાલિટી} & જો A એ B સાથે મેચ થાય, તો nA એ nB સાથે મેચ થાય (કોઈપણ તીવ્રતા n માટે) \\
        \hline
        \textbf{એડિટિવિટી} & જો A એ B સાથે અને C એ D સાથે મેચ થાય, તો A+C એ B+D સાથે મેચ થાય \\
        \hline
    \end{tabulary}
    
    \begin{itemize}
        \item ડિસ્પ્લેમાં RGB કલર મોડેલનો આધાર બનાવે છે કારણ કે તે એડિટિવ લાઇટ મિક્સિંગને લાગુ પડે છે.
        \item ત્રણ પ્રાથમિક રંગોને યોગ્ય રીતે મિશ્રિત કરીને કોઈપણ રંગ બનાવવાની મંજૂરી આપે છે.
    \end{itemize}

    \begin{mnemonicbox}
        \mnemonic{SPA Color: સિમેટ્રી, પ્રોપોર્શનાલિટી, એડિટિવિટી નિયમો કલર મેચિંગ માટે}
    \end{mnemonicbox}
\end{solutionbox}

\questionmarks{2(c) OR}{7}{કલર TV રિસીવરનો બ્લોક ડાયાગ્રામ દોરો અને સમજાવો.}
\begin{solutionbox}
    \textbf{બ્લોક ડાયાગ્રામ:}
    
    \begin{figure}[H]
        \centering
        \begin{tikzpicture}[gtu flow]
            \node (Ant) [gtu block] {એન્ટેના};
            \node (Tuner) [gtu block, right=of Ant] {ટ્યુનર};
            \node (IF) [gtu block, right=of Tuner] {IF Amp};
            \node (VidDet) [gtu block, below=of IF] {વિડિયો ડિટેક્ટર};
            \node (VidAmp) [gtu block, left=of VidDet] {વિડિયો Amp};
            \node (Color) [gtu block, left=of VidAmp] {કલર પ્રોસેસર};
            \node (RGB) [gtu block, below=of Color] {RGB મેટ્રિક્સ};
            \node (CRT) [gtu block, right=of RGB] {પિક્ચર ટ્યુબ};

            \draw [gtu arrow] (Ant) -- (Tuner);
            \draw [gtu arrow] (Tuner) -- (IF);
            \draw [gtu arrow] (IF) -- (VidDet);
            \draw [gtu arrow] (VidDet) -- (VidAmp);
            \draw [gtu arrow] (VidAmp) -- (Color);
            \draw [gtu arrow] (Color) -- (RGB);
            \draw [gtu arrow] (RGB) -- (CRT);
            
            % Sound path
            \node (SoundIF) [gtu block, below=of VidDet] {સાઉન્ડ IF};
            \node (Spk) [gtu block, left=of SoundIF] {સ્પીકર};
            \draw [gtu arrow] (VidDet) -- (SoundIF);
            \draw [gtu arrow] (SoundIF) -- (Spk);

        \end{tikzpicture}
        \caption{કલર TV રિસીવર}
    \end{figure}

    \begin{itemize}
        \item \textbf{ટ્યુનર}: ઇચ્છિત ચેનલ ફ્રિકવન્સી પસંદ કરે છે.
        \item \textbf{IF એમ્પ્લીફાયર}: ઇન્ટરમીડિયેટ ફ્રિકવન્સી સિગ્નલોને એમ્પ્લીફાય કરે છે.
        \item \textbf{વિડિયો ડિટેક્ટર}: વિડિયો અને ઓડિયો માહિતી અલગ કરે છે.
        \item \textbf{કલર પ્રોસેસર}: લ્યુમિનન્સ અને ક્રોમિનન્સને અલગ કરે છે.
        \item \textbf{RGB મેટ્રિક્સ}: કલર સિગ્નલોને લાલ, લીલા, વાદળી ડ્રાઇવરોમાં રૂપાંતરિત કરે છે.
        \item \textbf{ડિફ્લેક્શન સર્કિટ્સ}: ઇલેક્ટ્રોન બીમ સ્કેનિંગ (H-sync, V-sync) ને નિયંત્રિત કરે છે.
    \end{itemize}
\end{solutionbox}


\questionmarks{3(a)}{3}{સોલાર પાવર સિસ્ટમના મુખ્ય ઘટકો અને સોલાર પાવર સિસ્ટમના સ્પેસિફિકેશન્સ જણાવો.}
\begin{solutionbox}
    \textbf{મુખ્ય ઘટકો:}
    \begin{tabulary}{\linewidth}{|l|L|}
        \hline
        \textbf{ઘટક} & \textbf{કાર્ય} \\
        \hline
        \textbf{સોલાર પેનલ્સ} & સૂર્યપ્રકાશને વીજળીમાં રૂપાંતરિત કરે છે \\
        \hline
        \textbf{ચાર્જ કંટ્રોલર} & બેટરી ચાર્જિંગનું નિયમન કરે છે \\
        \hline
        \textbf{બેટરી બેંક} & વિદ્યુત ઊર્જા સંગ્રહિત કરે છે \\
        \hline
        \textbf{ઇન્વર્ટર} & DC ને AC વીજળીમાં રૂપાંતરિત કરે છે \\
        \hline
    \end{tabulary}

    \textbf{સ્પેસિફિકેશન્સ:}
    \begin{itemize}
        \item \textbf{પેનલ રેટિંગ}: પેનલ દીઠ 100-400W
        \item \textbf{બેટરી ક્ષમતા}: 100-200Ah
        \item \textbf{ઇન્વર્ટર રેટિંગ}: 500-5000W
        \item \textbf{સિસ્ટમ વોલ્ટેજ}: 12/24/48V
    \end{itemize}

    \begin{mnemonicbox}
        \mnemonic{SCBIM: સોલાર પેનલ્સ, કંટ્રોલર, બેટરી, ઇન્વર્ટર, માઉન્ટિંગ}
    \end{mnemonicbox}
\end{solutionbox}

\questionmarks{3(b)}{4}{માઇક્રોવેવ ઓવનના પ્રકારો, ઉપયોગો અને ટેકનિકલ સ્પેસિફિકેશન્સની યાદી આપો.}
\begin{solutionbox}
    \begin{tabulary}{\linewidth}{|l|L|}
        \hline
        \textbf{પ્રકાર} & \textbf{વિશેષતાઓ} \\
        \hline
        \textbf{સોલો} & માત્ર બેઝિક હીટિંગ અને ડિફ્રોસ્ટિંગ \\
        \hline
        \textbf{ગ્રીલ} & વધારાનું ગ્રીલિંગ એલિમેન્ટ \\
        \hline
        \textbf{કન્વેક્શન} & બેકિંગ માટે હીટિંગ એલિમેન્ટ અને પંખો ધરાવે છે \\
        \hline
        \textbf{કોમ્બિનેશન} & માઇક્રોવેવ, ગ્રીલ અને કન્વેક્શનને સાંકળે છે \\
        \hline
    \end{tabulary}

    \textbf{ઉપયોગો}: ખોરાક ગરમ કરવો, ડિફ્રોસ્ટિંગ, રસોઈ, બેકિંગ.
    
    \textbf{સ્પેક્સ}: પાવર (700-1200W), ક્ષમતા (20-40L), ફ્રિકવન્સી (2.45 GHz).
\end{solutionbox}

\questionmarks{3(c)}{7}{એર કંડિશનર અને રેફ્રિજરેટરની કાર્યપધ્ધતિ સમજાવો.}
\begin{solutionbox}
    \textbf{કાર્ય સિદ્ધાંત:}
    
    \begin{figure}[H]
        \centering
        \begin{tikzpicture}[gtu flow]
            \node (Comp) [gtu block] {કોમ્પ્રેસર};
            \node (Cond) [gtu block, right=of Comp] {કન્ડેન્સર};
            \node (Exp) [gtu block, below=of Cond] {એક્સપાન્શન વાલ્વ};
            \node (Evap) [gtu block, left=of Exp] {ઇવેપોરેટર};

            \draw [gtu arrow] (Comp) -- node[above, font=\tiny] {હોટ ગેસ} (Cond);
            \draw [gtu arrow] (Cond) -- node[right, font=\tiny] {લિક્વિડ} (Exp);
            \draw [gtu arrow] (Exp) -- node[below, font=\tiny] {લો પ્રેશર} (Evap);
            \draw [gtu arrow] (Evap) -- node[left, font=\tiny] {ગેસ} (Comp);
        \end{tikzpicture}
        \caption{રેફ્રિજરેશન સાયકલ (AC/ફ્રિજ)}
    \end{figure}

    \textbf{સાયકલ ઘટકો:}
    \begin{itemize}
        \item \textbf{કોમ્પ્રેસર}: રેફ્રિજરેન્ટ ગેસને દબાણયુક્ત કરે છે.
        \item \textbf{કન્ડેન્સર}: ગરમી મુક્ત કરે છે, ગેસને પ્રવાહીમાં ફેરવે છે.
        \item \textbf{એક્સપાન્શન વાલ્વ}: દબાણ/તાપમાન ઘટાડે છે.
        \item \textbf{ઇવેપોરેટર}: રૂમ/બોક્સમાંથી ગરમી શોષી લે છે, પ્રવાહીને ગેસમાં ફેરવે છે.
    \end{itemize}

    \textbf{તફાવતો}: AC રૂમને ઠંડુ કરે છે (18-26$^{\circ}$C), ફ્રિજ કેબિનેટને ઠંડુ કરે છે (2-8$^{\circ}$C).
\end{solutionbox}

\questionmarks{3(a) OR}{3}{એર કંડિશનર અને રેફ્રિજરેટરના ટેકનિકલ સ્પેસિફિકેશન્સની યાદી આપો.}
\begin{solutionbox}
    \begin{tabulary}{\linewidth}{|l|L|L|}
        \hline
        \textbf{સ્પેક} & \textbf{એર કંડિશનર} & \textbf{રેફ્રિજરેટર} \\
        \hline
        \textbf{ક્ષમતા} & 1-2 ટન (12k-24k BTU) & 100-500 લિટર \\
        \hline
        \textbf{પાવર} & 1000-2500 વોટ્સ & 100-400 વોટ્સ \\
        \hline
        \textbf{કાર્યક્ષમતા} & ISEER/Star રેટિંગ 3-5 & BEE Star રેટિંગ 3-5 \\
        \hline
        \textbf{ગેસ} & R32, R410A & R600a, R134a \\
        \hline
    \end{tabulary}
\end{solutionbox}

\questionmarks{3(b) OR}{4}{વોશિંગ મશીન માટે ઇલેક્ટ્રોનિક કંટ્રોલર સમજાવો.}
\begin{solutionbox}
    \begin{figure}[H]
        \centering
        \begin{tikzpicture}[gtu flow]
            \node (Micro) [gtu block] {માઈક્રોકંટ્રોલર};
            \node (UI) [gtu block, above=of Micro] {યુઝર ઇન્ટરફેસ};
            \node (Sensors) [gtu block, left=of Micro] {સેન્સર્સ (તાપમાન/લેવલ)};
            \node (Motor) [gtu block, right=of Micro] {મોટર/વાલ્વ};
            
            \draw [gtu arrow] (UI) -- (Micro);
            \draw [gtu arrow] (Sensors) -- (Micro);
            \draw [gtu arrow] (Micro) -- (Motor);
        \end{tikzpicture}
        \caption{ઇલેક્ટ્રોનિક કંટ્રોલ સિસ્ટમ}
    \end{figure}

    \begin{itemize}
        \item \textbf{માઈક્રોકંટ્રોલર}: કામગીરીનું સંચાલન કરતું કેન્દ્રીય CPU.
        \item \textbf{સેન્સર્સ}: પાણીનું સ્તર, તાપમાન, લોડ બેલેન્સ.
        \item \textbf{એક્ટ્યુએટર્સ}: મોટર ડ્રાઈવર, વોટર વાલ્વ, ડ્રેઇન પંપ.
    \end{itemize}

    \begin{mnemonicbox}
        \mnemonic{MIST-WAD: માઈક્રોકંટ્રોલર, ઈન્ટીગ્રેટ્સ, સેન્સર્સ અને ટાઈમર્સ ફોર વોટર, એજીટેશન અને ડ્રેનેજ}
    \end{mnemonicbox}
\end{solutionbox}

\questionmarks{3(c) OR}{7}{માઇક્રોવેવ ઓવનનો બ્લોક ડાયાગ્રામ દોરો અને સમજાવો. વાયરિંગ અને સેફ્ટી સૂચનાઓની યાદી આપો.}
\begin{solutionbox}
    \textbf{બ્લોક ડાયાગ્રામ:}
    
    \begin{figure}[H]
        \centering
        \begin{tikzpicture}[gtu flow]
            \node (Ctrl) [gtu block] {કંટ્રોલ યુનિટ};
            \node (HVTrans) [gtu block, right=of Ctrl] {HV ટ્રાન્સફોર્મર};
            \node (HVCap) [gtu block, right=of HVTrans] {HV કેપેસિટર};
            \node (Mag) [gtu block, below=of HVCap] {મેગ્નેટ્રોન};
            \node (Cavity) [gtu block, left=of Mag] {કુકિંગ કેવિટી};
            
            \draw [gtu arrow] (Ctrl) -- (HVTrans);
            \draw [gtu arrow] (HVTrans) -- (HVCap);
            \draw [gtu arrow] (HVCap) -- (Mag);
            \draw [gtu arrow] (Mag) -- (Cavity);
        \end{tikzpicture}
        \caption{માઇક્રોવેવ આંતરિક સિસ્ટમ}
    \end{figure}

    \begin{itemize}
        \item \textbf{મેગ્નેટ્રોન}: માઇક્રોવેવ્સ (2.45 GHz) ઉત્પન્ન કરે છે.
        \item \textbf{HV ટ્રાન્સફોર્મર}: વોલ્ટેજ 2-4kV સુધી વધારે છે.
        \item \textbf{સલામતી}: ખુલ્લા દરવાજા સાથે ક્યારેય ઓપરેટ કરશો નહીં; ગ્રાઉન્ડિંગ સુનિશ્ચિત કરો; ઇન્ટરલોકને ઓવરરાઇડ કરશો નહીં.
        \item \textbf{વાયરિંગ}: યોગ્ય ગ્રાઉન્ડ સાથે 15-20A ડેડિકેટેડ સર્કિટ વાપરો.
    \end{itemize}
\end{solutionbox}

\questionmarks{4(a)}{3}{ફોટોકોપીયરનો બ્લોક ડાયાગ્રામ દોરો.}
\begin{solutionbox}
    \begin{figure}[H]
        \centering
        \begin{tikzpicture}[gtu flow]
            \node (Scan) [gtu block] {સ્કેનર};
            \node (Img) [gtu block, right=of Scan] {ઇમેજ પ્રોસેસિંગ};
            \node (Laser) [gtu block, right=of Img] {લેસર યુનિટ};
            \node (Drum) [gtu block, below=of Laser] {ડ્રમ};
            \node (Dev) [gtu block, left=of Drum] {ડેવલપર};
            \node (Trans) [gtu block, left=of Dev] {ટ્રાન્સફર};
            \node (Fuse) [gtu block, left=of Trans] {ફ્યુઝર};

            \draw [gtu arrow] (Scan) -- (Img);
            \draw [gtu arrow] (Img) -- (Laser);
            \draw [gtu arrow] (Laser) -- (Drum);
            \draw [gtu arrow] (Drum) -- (Dev);
            \draw [gtu arrow] (Dev) -- (Trans);
            \draw [gtu arrow] (Trans) -- (Fuse);
        \end{tikzpicture}
        \caption{ફોટોકોપીયર પ્રક્રિયા}
    \end{figure}
\end{solutionbox}

\questionmarks{4(b)}{4}{MF પ્રિન્ટર અને CCTV ના સ્પેસિફિકેશન્સની યાદી આપો.}
\begin{solutionbox}
    \begin{tabulary}{\linewidth}{|l|L|}
        \hline
        \textbf{MF પ્રિન્ટર} & \textbf{CCTV} \\
        \hline
        Res: 600-1200 dpi & Res: 2-8 MP \\
        \hline
        Speed: 15-40 ppm & FPS: 15-30 fps \\
        \hline
        Scan: 300-600 dpi & Night Vision: 10-30m \\
        \hline
        Conn: USB, WiFi & Storage: 1-8 TB \\
        \hline
    \end{tabulary}
\end{solutionbox}


\questionmarks{4(c)}{7}{લેસર પ્રિન્ટરની કાર્યપધ્ધતિ બ્લોક ડાયાગ્રામ સાથે સમજાવો.}
\begin{solutionbox}
    \begin{figure}[H]
        \centering
        \begin{tikzpicture}[gtu flow]
            \node (Data) [gtu block] {ડેટા};
            \node (Laser) [gtu block, right=of Data] {લેસર};
            \node (Drum) [gtu block, right=of Laser] {ડ્રમ};
            \node (Dev) [gtu block, below=of Drum] {ડેવલપર};
            \node (Trans) [gtu block, left=of Dev] {ટ્રાન્સફર};
            \node (Fuse) [gtu block, left=of Trans] {ફ્યુઝર/આઉટપુટ};

            \draw [gtu arrow] (Data) -- (Laser);
            \draw [gtu arrow] (Laser) -- (Drum);
            \draw [gtu arrow] (Drum) -- (Dev);
            \draw [gtu arrow] (Dev) -- (Trans);
            \draw [gtu arrow] (Trans) -- (Fuse);
        \end{tikzpicture}
        \caption{લેસર પ્રિન્ટર સાયકલ}
    \end{figure}

    \textbf{પ્રક્રિયા તબક્કા:}
    \begin{enumerate}
        \item \textbf{ચાર્જિંગ}: ડ્રમ એકસમાન ચાર્જ મેળવે છે.
        \item \textbf{રાઈટિંગ}: લેસર ઇમેજ વિસ્તારોને ડિસ્ચાર્જ કરે છે.
        \item \textbf{ડેવલપિંગ}: ટોનર ડિસ્ચાર્જ થયેલા વિસ્તારોને ચોંટી જાય છે.
        \item \textbf{ટ્રાન્સફર}: ટોનર કાગળ પર જાય છે.
        \item \textbf{ફ્યુઝિંગ}: ગરમી કાગળ પર ટોનર ઓગાળે છે.
        \item \textbf{ક્લીનિંગ}: બાકી રહેલ ટોનર દૂર કરવામાં આવે છે.
    \end{enumerate}

    \begin{mnemonicbox}
        \mnemonic{CWTFC: ચાર્જ, રાઈટ, ટ્રાન્સફર, ફ્યુઝ, ક્લીન સાયકલ}
    \end{mnemonicbox}
\end{solutionbox}

\questionmarks{4(a) OR}{3}{CCTV નો બ્લોક ડાયાગ્રામ દોરો.}
\begin{solutionbox}
    \begin{figure}[H]
        \centering
        \begin{tikzpicture}[gtu flow]
            \node (Cam) [gtu block] {કેમેરા};
            \node (DVR) [gtu block, right=of Cam] {DVR/NVR};
            \node (HDD) [gtu block, right=of DVR] {HDD સ્ટોરેજ};
            \node (Mon) [gtu block, below=of DVR] {મોનિટર};
            
            \draw [gtu arrow] (Cam) -- (DVR);
            \draw [gtu arrow] (DVR) -- (HDD);
            \draw [gtu arrow] (DVR) -- (Mon);
        \end{tikzpicture}
        \caption{બેઝિક CCTV સિસ્ટમ}
    \end{figure}
\end{solutionbox}

\questionmarks{4(b) OR}{4}{ઇંકજેટ પ્રિન્ટર અને ફોટોકોપીયરના સ્પેસિફિકેશન્સની યાદી આપો.}
\begin{solutionbox}
    \begin{tabulary}{\linewidth}{|l|L|}
        \hline
        \textbf{ઇંકજેટ પ્રિન્ટર} & \textbf{ફોટોકોપીયર} \\
        \hline
        Res: 1200-4800 dpi & Res: 600-1200 dpi \\
        \hline
        Speed: 8-20 ppm & Speed: 20-60 cpm \\
        \hline
        Ink: ડાય/પિગમેન્ટ & Toner: ડ્રાય પાવડર \\
        \hline
        Duty: 1-5k પેજ/મહિને & Duty: 10k-100k પેજ/મહિને \\
        \hline
    \end{tabulary}
\end{solutionbox}

\questionmarks{4(c) OR}{7}{LCD પ્રોજેક્ટરની કાર્યપધ્ધતિ બ્લોક ડાયાગ્રામ સાથે સમજાવો અને તેના સ્પેસિફિકેશન્સની યાદી આપો.}
\begin{solutionbox}
    \textbf{કાર્ય પ્રક્રિયા:}
    \begin{figure}[H]
        \centering
        \begin{tikzpicture}[gtu flow]
            \node (Lamp) [gtu block] {લેમ્પ};
            \node (Mirror) [gtu block, right=of Lamp] {ડાઇક્રોઇક મિરર્સ};
            \node (RGB) [gtu block, right=of Mirror] {RGB LCDs};
            \node (Prism) [gtu block, below=of RGB] {પ્રિઝમ};
            \node (Lens) [gtu block, left=of Prism] {Lens};
            \node (Screen) [gtu block, left=of Lens] {સ્ક્રીન};

            \draw [gtu arrow] (Lamp) -- (Mirror);
            \draw [gtu arrow] (Mirror) -- (RGB);
            \draw [gtu arrow] (RGB) -- (Prism);
            \draw [gtu arrow] (Prism) -- (Lens);
            \draw [gtu arrow] (Lens) -- (Screen);
        \end{tikzpicture}
        \caption{LCD પ્રોજેક્ટર}
    \end{figure}

    \begin{itemize}
        \item \textbf{લેમ્પ}: ઉચ્ચ તીવ્રતા સ્રોત.
        \item \textbf{મિરર્સ}: પ્રકાશને લાલ, લીલા, વાદળીમાં વિભાજીત કરે છે.
        \item \textbf{LCDs}: દરેક રંગ માટે પ્રકાશને મોડ્યુલેટ કરે છે.
        \item \textbf{પ્રિઝમ}: પ્રકાશના બીમને ફરીથી સંયોજિત કરે છે.
    \end{itemize}

    \textbf{સ્પેક્સ}: Res (XGA/FHD), Brightness (2000-5000 Lumens), Lamp Life (3000-6000 hrs).
\end{solutionbox}

\questionmarks{5(a)}{3}{PA સિસ્ટમનો બ્લોક ડાયાગ્રામ દોરો.}
\begin{solutionbox}
    \begin{figure}[H]
        \centering
        \begin{tikzpicture}[gtu flow]
            \node (Mic) [gtu block] {માઇક્રોફોન};
            \node (PreAmp) [gtu block, right=of Mic] {પ્રિ-એમ્પ};
            \node (Mixer) [gtu block, right=of PreAmp] {મિક્સર};
            \node (EQ) [gtu block, below=of Mixer] {ઇક્વલાઇઝર};
            \node (PwrAmp) [gtu block, left=of EQ] {પાવર એમ્પ};
            \node (Spk) [gtu block, left=of PwrAmp] {સ્પીકર};
            
            \draw [gtu arrow] (Mic) -- (PreAmp);
            \draw [gtu arrow] (PreAmp) -- (Mixer);
            \draw [gtu arrow] (Mixer) -- (EQ);
            \draw [gtu arrow] (EQ) -- (PwrAmp);
            \draw [gtu arrow] (PwrAmp) -- (Spk);
            
            \node (Source) [gtu block, above=of Mixer] {લાઈન ઇન};
            \draw [gtu arrow] (Source) -- (Mixer);
        \end{tikzpicture}
        \caption{પબ્લિક એડ્રેસ સિસ્ટમ}
    \end{figure}

    \begin{mnemonicbox}
        \mnemonic{MMEPS: માઇક્રોફોન, મિક્સર, ઇક્વલાઇઝર, પાવર એમ્પ, સ્પીકર્સ}
    \end{mnemonicbox}
\end{solutionbox}

\questionmarks{5(b)}{4}{ટ્વીટર અને વૂફર સમજાવો.}
\begin{solutionbox}
    \begin{tabulary}{\linewidth}{|l|L|L|}
        \hline
        \textbf{લક્ષણ} & \textbf{ટ્વીટર} & \textbf{વૂફર} \\
        \hline
        ફ્રિકવન્સી & હાઈ (2kHz-20kHz) & લો (20Hz-2kHz) \\
        \hline
        કદ & નાનું (0.5"-1.5") & મોટું (4"-15") \\
        \hline
        ડાયાફ્રેમ & હળવા, કઠોર & ભારે, લવચીક \\
        \hline
        ભૂમિકા & ટ્રબલ/ડિટેલ & બાસ/પાવર \\
        \hline
    \end{tabulary}

    \begin{figure}[H]
        \centering
        \begin{tikzpicture}[gtu flow]
            \node (In) [gtu block] {સિગ્નલ};
            \node (Xover) [gtu block, right=of In] {ક્રોસઓવર};
            \node (Tweet) [gtu block, right=of Xover, yshift=1cm] {ટ્વીટર};
            \node (Woof) [gtu block, right=of Xover, yshift=-1cm] {વૂફર};

            \draw [gtu arrow] (In) -- (Xover);
            \draw [gtu arrow] (Xover) -- node[above, sloped] {હાઈ ફ્રિ.} (Tweet);
            \draw [gtu arrow] (Xover) -- node[below, sloped] {લો ફ્રિ.} (Woof);
        \end{tikzpicture}
        \caption{સ્પીકર ક્રોસઓવર}
    \end{figure}

    \begin{mnemonicbox}
        \mnemonic{THSL: ટ્વીટર્સ કેચ હાઈઝ (સ્મોલ/લાઈટ), વૂફર્સ કેચ લોઝ}
    \end{mnemonicbox}
\end{solutionbox}

\questionmarks{5(c)}{7}{માઇક્રોફોનની વ્યાખ્યા આપો. માઇક્રોફોનના પ્રકારોની યાદી આપો અને કોઈપણ એક પ્રકારના માઇક્રોફોનની કાર્યપધ્ધતિ સમજાવો.}
\begin{solutionbox}
    \textbf{વ્યાખ્યા:} ઇલેક્ટ્રોએકોસ્ટિક ટ્રાન્સડ્યુસર જે ધ્વનિ તરંગોને વિદ્યુત સિગ્નલોમાં રૂપાંતરિત કરે છે.
    
    \textbf{પ્રકારો:} ડાયનેમિક, કન્ડેન્સર, રિબન, કાર્બન, પીઝો, MEMS.

    \textbf{ડાયનેમિક માઇક્રોફોન કાર્ય:}
    
    \begin{figure}[H]
        \centering
        \begin{tikzpicture}[gtu flow]
            \node (Sound) [gtu block] {સાઉન્ડ વેવ};
            \node (Diaph) [gtu block, right=of Sound] {ડાયાફ્રેમ};
            \node (Coil) [gtu block, right=of Diaph] {વોઈસ કોઈલ};
            \node (Mag) [gtu block, right=of Coil] {મેગ્નેટિક ફીલ્ડ};
            \node (Volt) [gtu block, right=of Mag] {ઈન્ડ્યુસ્ડ વોલ્ટેજ};

            \draw [gtu arrow] (Sound) -- (Diaph);
            \draw [gtu arrow] (Diaph) -- (Coil);
            \draw [gtu arrow] (Coil) -- (Mag);
            \draw [gtu arrow] (Mag) -- (Volt);
        \end{tikzpicture}
        \caption{ડાયનેમિક માઈક સિદ્ધાંત}
    \end{figure}

    \begin{itemize}
        \item \textbf{સાઉન્ડ કેપ્ચર}: ધ્વનિ તરંગો ડાયાફ્રેમને અથડાય છે.
        \item \textbf{ટ્રાન્સડક્શન}: કોઈલ ચુંબકીય ક્ષેત્રમાં હલે છે.
        \item \textbf{આઉટપુટ}: હલનચલન વોલ્ટેજ પ્રેરિત કરે છે (ફેરાડેનો નિયમ).
        \item \textbf{ફાયદા}: મજબૂત, પાવરની જરૂર નથી, ઉચ્ચ એકોસ્ટિક હેન્ડલિંગ.
    \end{itemize}

    \begin{mnemonicbox}
        \mnemonic{DDCMIO: ડાયાફ્રેમ ડિસ્પ્લેસિસ કોઈલ ઇન મેગ્નેટિક ફિલ્ડ ઈન્ડ્યુસિંગ આઉટપુટ}
    \end{mnemonicbox}
\end{solutionbox}

\questionmarks{5(a) OR}{3}{વ્યાખ્યા આપો: (1) પિચ (Pitch) (2) લાઉડસ્પીકર (3) રિવર્બરેશન (Reverberation).}
\begin{solutionbox}
    \begin{itemize}
        \item \textbf{પિચ}: અવાજની અનુભવાતી આવર્તન (હાઈ/લો ટોન).
        \item \textbf{લાઉડસ્પીકર}: ટ્રાન્સડ્યુસર જે વિદ્યુત સિગ્નલોને ધ્વનિ તરંગોમાં રૂપાંતરિત કરે છે.
        \item \textbf{રિવર્બરેશન}: પરાવર્તનને કારણે સ્ત્રોત બંધ થયા પછી અવાજની હાજરી.
    \end{itemize}

    \begin{figure}[H]
        \centering
        \begin{tikzpicture}[gtu flow]
            \node (Src) [gtu block] {સોર્સ};
            \node (Direct) [gtu block, right=of Src] {ડાયરેક્ટ સાઉન્ડ};
            \node (Early) [gtu block, right=of Direct] {અર્લી રિફ્લેક્શન};
            \node (Late) [gtu block, right=of Early] {રિવર્બ};

            \draw [gtu arrow] (Src) -- (Direct);
            \draw [gtu arrow] (Direct) -- (Early);
            \draw [gtu arrow] (Early) -- (Late);
        \end{tikzpicture}
        \caption{સાઉન્ડ પ્રોપેગેશન}
    \end{figure}

    \begin{mnemonicbox}
        \mnemonic{PLR Sound: પિચ(ટોન), લાઉડસ્પીકર(પ્રોડ્યુસર), રિવર્બ(ઈકો)}
    \end{mnemonicbox}
\end{solutionbox}

\questionmarks{5(b) OR}{4}{હોમ થિયેટર સાઉન્ડ સિસ્ટમનો બ્લોક ડાયાગ્રામ દોરો અને ટૂંકમાં સમજાવો.}
\begin{solutionbox}
    \begin{figure}[H]
        \centering
        \begin{tikzpicture}[gtu flow]
            \node (AV) [gtu block] {AV રિસીવર};
            \node (Center) [gtu block, right=of AV] {સેન્ટર};
            \node (Front) [gtu block, above=of Center] {ફ્રન્ટ L/R};
            \node (Surr) [gtu block, below=of Center] {સરાઉન્ડ L/R};
            \node (Sub) [gtu block, right=of Center] {સબવૂફર};
            \node (Src) [gtu block, left=of AV] {સોર્સ (TV/BluRay)};

            \draw [gtu arrow] (Src) -- (AV);
            \draw [gtu arrow] (AV) -- (Center);
            \draw [gtu arrow] (AV) -- (Front);
            \draw [gtu arrow] (AV) -- (Surr);
            \draw [gtu arrow] (AV) -- (Sub);
        \end{tikzpicture}
        \caption{5.1 હોમ થિયેટર સિસ્ટમ}
    \end{figure}

    \begin{itemize}
        \item \textbf{રિસીવર}: એમ્પ્લીફિકેશન અને ડીકોડિંગ પ્રક્રિયા કરે છે.
        \item \textbf{સેન્ટર}: ડાયલોગ સ્પષ્ટતા.
        \item \textbf{ફ્રન્ટ/સરાઉન્ડ}: સ્ટીરિયો અને એમ્બિયન્ટ ઇફેક્ટ્સ.
        \item \textbf{સબવૂફર}: લો ફ્રિકવન્સી ઇફેક્ટ્સ (LFE).
    \end{itemize}
\end{solutionbox}

\questionmarks{5(c) OR}{7}{ઇલેક્ટ્રોસ્ટેટિક લાઉડસ્પીકર અને પરમેનન્ટ મેગ્નેટ લાઉડસ્પીકર સમજાવો.}
\begin{solutionbox}
    \begin{tabulary}{\linewidth}{|l|L|L|}
        \hline
        \textbf{લક્ષણ} & \textbf{ઇલેક્ટ્રોસ્ટેટિક} & \textbf{પરમેનન્ટ મેગ્નેટ} \\
        \hline
        સિદ્ધાંત & ઇલેક્ટ્રોસ્ટેટિક બળ (કેપેસિટીવ) & ઇલેક્ટ્રોમેગ્નેટિક ઇન્ડક્શન \\
        \hline
        ભાગો & સ્ટેટર પ્લેટ્સ, ચાર્જડ ફિલ્મ & મેગ્નેટ, વોઈસ કોઈલ, કોન \\
        \hline
        પાવર & HV બાયસ સપ્લાયની જરૂર છે & માત્ર સિગ્નલ દ્વારા ચાલે છે \\
        \hline
        ગુણવત્તા & ઓછું ડિસ્ટોર્શન, ઝડપી ટ્રાન્ઝિયન્ટ & સારો બાસ, કાર્યક્ષમ \\
        \hline
    \end{tabulary}

    \textbf{પરમેનન્ટ મેગ્નેટ કાર્ય:}
    \begin{figure}[H]
        \centering
        \begin{tikzpicture}[gtu flow]
            \node (Sig) [gtu block] {સિગ્નલ};
            \node (Coil) [gtu block, right=of Sig] {વોઈસ કોઈલ};
            \node (Mag) [gtu block, right=of Coil] {મેગ્નેટિક ફીલ્ડ};
            \node (Force) [gtu block, below=of Mag] {કોન પર બળ};
            \node (Sound) [gtu block, left=of Force] {સાઉન્ડ વેવ};

            \draw [gtu arrow] (Sig) -- (Coil);
            \draw [gtu arrow] (Coil) -- (Mag);
            \draw [gtu arrow] (Mag) -- (Force);
            \draw [gtu arrow] (Force) -- (Sound);
        \end{tikzpicture}
        \caption{મૂવિંગ કોઈલ સ્પીકર}
    \end{figure}

    \begin{mnemonicbox}
        \mnemonic{ESPM: ઇલેક્ટ્રોસ્ટેટિક(સ્ટેટિક ચાર્જ), પરમેનન્ટ મેગ્નેટ(મેગ્નેટિક કોઈલ)}
    \end{mnemonicbox}
\end{solutionbox}

\end{document}
