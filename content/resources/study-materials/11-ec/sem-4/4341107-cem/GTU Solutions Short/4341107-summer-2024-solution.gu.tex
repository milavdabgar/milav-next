\documentclass{article}

% content/resources/templates/preamble.tex
\usepackage[margin=0.6in]{geometry}
\author{Milav Dabgar}
\usepackage{amsmath,amssymb,amsthm}
\usepackage{booktabs}
\usepackage{multirow}
\usepackage{xcolor}
\usepackage{tcolorbox}
\tcbuselibrary{breakable,skins}
\usepackage[colorlinks=true,linkcolor=blue]{hyperref}
\usepackage{titlesec}
\usepackage{enumitem}
\usepackage{tikz}
\usepackage{pgfplots}
\usepackage{circuitikz}
\usepackage[version=4]{mhchem}
\usepackage{longtable}
\usepackage{array}
\usepackage{float}
\usepackage{caption}
\usepackage{listings}

\lstset{
  basicstyle=\small\ttfamily,
  breaklines=true,
  breakatwhitespace=false,
  postbreak=\mbox{\textcolor{red}{$\hookrightarrow$}\space},
  float=false,
  numbers=left,
  numberstyle=\tiny\color{gray},
  numbersep=10pt,
  xleftmargin=2em,
  keywordstyle=\color{blue},
  commentstyle=\color{green!60!black},
  stringstyle=\color{purple},
  backgroundcolor=\color{gray!5},
  showstringspaces=false,
  tabsize=2,
  captionpos=b,
  keepspaces=true,
  columns=flexible
}

\pgfplotsset{compat=1.18}
\usetikzlibrary{shapes,arrows,positioning,calc,patterns,decorations.pathmorphing,decorations.markings,arrows.meta}

% Color scheme
\definecolor{headcolor}{RGB}{0,102,204}
\definecolor{keycolor}{RGB}{220,20,60}
\definecolor{solutioncolor}{RGB}{34,139,34}
\definecolor{mnemoniccolor}{RGB}{148,0,211}
\definecolor{codecolor}{RGB}{0,0,100}

% Spacing
\setlength{\parskip}{3pt}
\setlist[itemize]{nosep}
\setlist[enumerate]{nosep}

% Title formatting
\titleformat{\section}{\Large\bfseries\color{headcolor}}{\thesection}{1em}{}
\titleformat{\subsection}{\large\bfseries\color{headcolor}}{\thesubsection}{1em}{}

% Pandoc tightlist compatibility
\providecommand{\tightlist}{%
  \setlength{\itemsep}{0pt}\setlength{\parskip}{0pt}}

% Pandoc longtable compatibility
\newcounter{none}
\def\thenone{}


% content/resources/templates/gujarati-boxes.tex
\usepackage{fontspec}
\usepackage{polyglossia}

% Set Gujarati as main language (document is primarily in Gujarati)
% Note: gloss-gujarati.ldf doesn't exist in polyglossia, but it will use hyphenation patterns
\setdefaultlanguage{gujarati}
\setotherlanguage{english}

% Configure Gujarati font properly
% Use Language=Default to prevent polyglossia from trying to add language-specific features
% that don't exist for Gujarati, which causes "empty feature" warnings
\newfontfamily\gujaratifont[Script=Gujarati,AutoFakeBold=2.5,AutoFakeSlant=0.3]{Noto Sans Gujarati}
\setmainfont[Script=Gujarati,AutoFakeBold=2.5,AutoFakeSlant=0.3]{Noto Sans Gujarati}
% Use Noto Sans Gujarati for monospace to support Gujarati in text
\setmonofont[Scale=0.9]{Noto Sans Gujarati}

% Configure English to use the same font
\newfontfamily\englishfont[Script=Gujarati,AutoFakeBold=2.5,AutoFakeSlant=0.3]{Noto Sans Gujarati}

% Translations for polyglossia
\gappto\captionsgujarati{
  \renewcommand{\tablename}{કોષ્ટક}
  \renewcommand{\figurename}{આકૃતિ}
}

% Helper for TikZ nodes to ensure Gujarati font
\newcommand{\gu}[1]{{\gujaratifont #1}}

% Custom environments
\newtcolorbox{solutionbox}{
    breakable,
    enhanced,
    colback=solutioncolor!5!white,
    colframe=solutioncolor!75!black,
    fonttitle=\bfseries,
    title=જવાબ
}

\newtcolorbox{solutionboxnobreak}{
 colback=solutioncolor!5!white,
 colframe=solutioncolor!75!black,
 fonttitle=\bfseries,
 title=જવાબ
}

\newtcolorbox{keyformula}{
 breakable,
 enhanced,
 colback=keycolor!5!white,
 colframe=keycolor!75!black,
 fonttitle=\bfseries,
 title=રાસાયણિક સમીકરણ/સૂત્ર
}

\newtcolorbox{mnemonicbox}{
 breakable,
 enhanced,
 colback=mnemoniccolor!5!white,
 colframe=mnemoniccolor!75!black,
 fonttitle=\bfseries,
 title=મેમરી ટ્રીક
}


% Custom commands for GTU solutions
% This file defines semantic commands for consistent formatting

% Question command with automatic formatting
\newcommand{\question}[2]{%
  \section*{Question #1}%
  \textbf{#2}%
}

% OR question variant
\newcommand{\questionor}[2]{%
  \section*{Question #1 OR}%
  \textbf{#2}%
}

% Proper table environment with caption
\newenvironment{answertable}[1]{%
  \begin{table}[htbp]
  \centering
  \caption{#1}
}{%
  \end{table}
}

% Proper figure environment for diagrams
\newenvironment{answerdiagram}[1]{%
  \begin{figure}[htbp]
  \centering
  \caption{#1}
}{%
  \end{figure}
}

% Semantic markup for key terms
\newcommand{\keyword}[1]{\textbf{#1}}
\newcommand{\code}[1]{\texttt{#1}}
\newcommand{\classname}[1]{\texttt{#1}}
\newcommand{\methodname}[1]{\texttt{#1}}

% Proper quotation marks
\newcommand{\mnemonic}[1]{``#1''}


\title{કન્ઝ્યુમર ઇલેક્ટ્રોનિક્સ એન્ડ મેઇન્ટેનન્સ (4341107) - સમર 2024 સોલ્યુશન}
\date{૨૧ જૂન, ૨૦૨૪}

\begin{document}
\maketitle

\questionmarks{1(અ)}{3}{લાઉડનેસ, ફાઈડાલીટી અને રીવાર્બેરાશનની માત્ર વ્યાખ્યા આપો.}
\begin{solutionbox}
    \begin{itemize}
        \item \textbf{લાઉડનેસ}: માનવ કાન દ્વારા ધ્વનિની તીવ્રતાની આત્મલક્ષી ધારણા, જે ડેસિબલ (dB)માં માપવામાં આવે છે.
        \item \textbf{ફાઈડાલીટી}: એક સિસ્ટમ મૂળ ઇનપુટ સિગ્નલને કેટલી સચોટતાથી પુનઃઉત્પાદિત કરે છે તેનું માપ.
        \item \textbf{રીવાર્બેરાશન}: મૂળ ધ્વનિ સ્રોત બંધ થયા પછી પણ ધ્વનિનું ચાલુ રહેવું, જે બંધ જગ્યામાં અનેક પરાવર્તનોને કારણે થાય છે.
    \end{itemize}

    \begin{mnemonicbox}
        \mnemonic{LFR: ધ્વનિને વિશ્વાસપૂર્વક સાંભળો અને રૂમના પડઘાઓને સમજો}
    \end{mnemonicbox}
\end{solutionbox}

\questionmarks{1(બ)}{4}{પીએ સિસ્ટમને તેના બ્લોક ડાયાગ્રામ વડે સમજાવો.}
\begin{solutionbox}
    \textbf{પીએ સિસ્ટમ ડાયાગ્રામ:}

    \begin{figure}[H]
        \centering
        \begin{tikzpicture}[gtu flow]
            \node (Mic) [gtu block] {માઈક્રોફોન};
            \node (Pre) [gtu block, right=of Mic] {પ્રિએમ્પલિફાયર};
            \node (Mixer) [gtu block, right=of Pre] {મિક્સર};
            
            \node (Input) [gtu block, above=of Mixer] {ઓડિયો ઇનપુટ};
            \node (EQ) [gtu block, below=of Mixer] {ઇક્વલાઈઝર};
            
            \node (Power) [gtu block, right=of Mixer] {પાવર એમ્પલિફાયર};
            \node (Speaker) [gtu block, right=of Power] {લાઉડસ્પીકર};
            
            \draw [gtu arrow] (Mic) -- (Pre);
            \draw [gtu arrow] (Pre) -- (Mixer);
            \draw [gtu arrow] (Input) -- (Mixer);
            \draw [gtu arrow] (Mixer) -- (Power);
            \draw [gtu arrow] (Power) -- (Speaker);
            \draw [gtu arrow] (EQ) -- (Mixer);
        \end{tikzpicture}
        \caption{પબ્લિક એડ્રેસ સિસ્ટમ}
    \end{figure}

    \textbf{સમજૂતી:}
    \begin{itemize}
        \item \textbf{માઈક્રોફોન}: ધ્વનિ તરંગોને ઇલેક્ટ્રિકલ સિગ્નલમાં રૂપાંતરિત કરે છે.
        \item \textbf{પ્રિએમ્પલિફાયર}: નબળા માઈક્રોફોન સિગ્નલ્સને લાઈન લેવલ સુધી વધારે છે.
        \item \textbf{મિક્સર}: અનેક ઓડિયો સિગ્નલ્સને ભેગા કરે છે અને લેવલ એડજસ્ટ કરે છે.
        \item \textbf{પાવર એમ્પલિફાયર}: લાઉડસ્પીકર ચલાવવા માટે સિગ્નલની પાવર વધારે છે.
        \item \textbf{લાઉડસ્પીકર}: ઇલેક્ટ્રિકલ સિગ્નલને પાછા ધ્વનિ તરંગોમાં રૂપાંતરિત કરે છે.
    \end{itemize}

    \begin{mnemonicbox}
        \mnemonic{MPMEL: ઘણા લોકો ઉત્તમ શ્રોતાઓ બનાવે છે}
    \end{mnemonicbox}
\end{solutionbox}

\questionmarks{1(ક)}{7}{માઈક્રોફોનની કોઈ પણ બે લાક્ષણિકતાઓ સમજાવી વાયરલેસ માઈક્રોફોન સમજાવો.}
\begin{solutionbox}
    \textbf{માઈક્રોફોનની લાક્ષણિકતાઓ:}
    \textbf{ટેબલ: માઈક્રોફોન લાક્ષણિકતાઓ} \\
    \begin{tabulary}{\linewidth}{|l|L|}
        \hline
        \textbf{લાક્ષણિકતા} & \textbf{વર્ણન} \\
        \hline
        \textbf{સેન્સિટિવિટી} & માઈક્રોફોન કેટલી કાર્યક્ષમતાથી ધ્વનિ દબાણને ઇલેક્ટ્રિકલ આઉટપુટમાં રૂપાંતરિત કરે છે તે માપે છે (mV/Pa) \\
        \hline
        \textbf{દિશાત્મક પેટર્ન} & પિકઅપ એરિયા નક્કી કરે છે (ઓમ્નિડાયરેક્શનલ, કાર્ડિયોઇડ, હાયપરકાર્ડિયોઇડ, બાયડાયરેક્શનલ) \\
        \hline
    \end{tabulary}

    \textbf{વાયરલેસ માઈક્રોફોન:}

    \begin{figure}[H]
        \centering
        \begin{tikzpicture}[gtu flow]
            \node (Mic) [gtu block] {માઈક્રોફોન એલિમેન્ટ};
            \node (Proc) [gtu block, right=of Mic] {ઓડિયો પ્રોસેસર};
            \node (Tx) [gtu block, right=of Proc] {RF ટ્રાન્સમિટર};
            \node (Ant1) [gtu block, above=of Tx, minimum width=1cm, minimum height=1cm] {એન્ટેના};
            
            \node (Ant2) [gtu block, right=3cm of Ant1, minimum width=1cm, minimum height=1cm] {એન્ટેના};
            \node (Rx) [gtu block, below=of Ant2] {RF રિસીવર};
            \node (Out) [gtu block, right=of Rx] {ઓડિયો આઉટપુટ};
            
            \draw [gtu arrow] (Mic) -- (Proc);
            \draw [gtu arrow] (Proc) -- (Tx);
            \draw [gtu arrow] (Tx) -- (Ant1);
            
            \draw [gtu arrow, dashed] (Ant1) -- node[above] {રેડિયો વેવ્સ} (Ant2);
            \draw [gtu arrow] (Ant2) -- (Rx);
            \draw [gtu arrow] (Rx) -- (Out);
        \end{tikzpicture}
        \caption{વાયરલેસ માઈક્રોફોન સિસ્ટમ}
    \end{figure}

    \begin{itemize}
        \item \textbf{માઈક્રોફોન એલિમેન્ટ}: ધ્વનિ પકડી તેને ઇલેક્ટ્રિકલ સિગ્નલમાં રૂપાંતરિત કરે છે.
        \item \textbf{RF ટ્રાન્સમિટર}: ઓડિયોને રેડિયો ફ્રિક્વન્સી કેરિયર પર મોડ્યુલેટ કરે છે.
        \item \textbf{ટ્રાન્સમિશન}: સામાન્ય ફ્રિક્વન્સી બેન્ડ UHF (470-698 MHz) અથવા VHF (174-216 MHz) છે.
        \item \textbf{RF રિસીવર}: સિગ્નલને ફરીથી ઓડિયોમાં ડિમોડ્યુલેટ કરે છે.
        \item \textbf{ફાયદાઓ}: ગતિશીલતા, કેબલ પ્રતિબંધો નથી, સ્ટેજ પર ગરબડ ઘટાડે છે.
    \end{itemize}

    \begin{mnemonicbox}
        \mnemonic{SMART: સેન્સિટિવિટી ધ્વનિની પ્રતિક્રિયાને સાચી રીતે માપે છે}
    \end{mnemonicbox}
\end{solutionbox}

\questionmarks{1(ક) OR}{7}{લાઉડસ્પીકરની કોઈ પણ બે લાક્ષણિકતાઓ સમજાવી પરમેનેન્ટ મેગ્નેટ લાઉડસ્પીકર સમજાવો.}
\begin{solutionbox}
    \textbf{લાઉડસ્પીકરની લાક્ષણિકતાઓ:}
    \textbf{ટેબલ: લાઉડસ્પીકર સ્પેસિફિકેશન્સ} \\
    \begin{tabulary}{\linewidth}{|l|L|}
        \hline
        \textbf{લાક્ષણિકતા} & \textbf{વર્ણન} \\
        \hline
        \textbf{ફ્રિક્વન્સી રિસ્પોન્સ} & સ્પીકર કયા ફ્રિક્વન્સી રેન્જ (Hz) ફરીથી ઉત્પન્ન કરી શકે છે (સામાન્ય રીતે 20Hz-20kHz) \\
        \hline
        \textbf{ઇમ્પીડન્સ} & ઇલેક્ટ્રિકલ રેઝિસ્ટન્સ (ઓહ્મ) જે એમ્પલિફાયરથી પાવર ટ્રાન્સફરને અસર કરે છે (સામાન્ય રીતે 4-8$\Omega$) \\
        \hline
    \end{tabulary}

    \textbf{પરમેનેન્ટ મેગ્નેટ લાઉડસ્પીકર:}

    \begin{figure}[H]
        \centering
        \begin{tikzpicture}[gtu flow]
            \node (Input) [gtu block] {ઓડિયો ઇનપુટ};
            \node (Coil) [gtu block, right=of Input] {વોઇસ કોઇલ};
            \node (Magnet) [gtu block, above=of Coil] {પરમેનેન્ટ મેગ્નેટ};
            \node (Cone) [gtu block, right=of Coil] {કોન/ડાયાફ્રામ};
            \node (Sound) [gtu block, right=of Cone] {ધ્વનિ તરંગો};
            
            \draw [gtu arrow] (Input) -- (Coil);
            \draw [gtu arrow] (Magnet) -- (Coil);
            \draw [gtu arrow] (Coil) -- (Cone);
            \draw [gtu arrow] (Cone) -- (Sound);
        \end{tikzpicture}
        \caption{પરમેનેન્ટ મેગ્નેટ લાઉડસ્પીકર}
    \end{figure}

    \begin{itemize}
        \item \textbf{પરમેનેન્ટ મેગ્નેટ}: સ્થિર ચુંબકીય ક્ષેત્ર બનાવે છે (સામાન્ય રીતે ફેરાઇટ અથવા નિયોડિમિયમ).
        \item \textbf{વોઇસ કોઇલ}: તાર કોઇલ જે ઓડિયો કરંટ વહન કરે છે, ચલિત ચુંબકીય ક્ષેત્ર બનાવે છે.
        \item \textbf{કોન/ડાયાફ્રામ}: વોઇસ કોઇલની ગતિના જવાબમાં ખસે છે.
        \item \textbf{કાર્યસિદ્ધાંત}: સ્થિર ચુંબકીય ક્ષેત્ર અને વોઇસ કોઇલના ચલિત ક્ષેત્ર વચ્ચેની ક્રિયા-પ્રતિક્રિયા યાંત્રિક ગતિ ઉત્પન્ન કરે છે.
        \item \textbf{ફાયદાઓ}: વધુ કાર્યક્ષમ, ફિલ્ડ કોઇલ પાવરની જરૂર નથી, કોમ્પેક્ટ ડિઝાઇન.
    \end{itemize}

    \begin{mnemonicbox}
        \mnemonic{FIRM: ફ્રિક્વન્સી ઇમ્પીડન્સને મેગ્નેટની જરૂર પડે છે}
    \end{mnemonicbox}
\end{solutionbox}

\questionmarks{2(અ)}{3}{આસ્પેક્ટ રેશીઓ, લ્યુમિનેન્સ અને ક્રોમિનેન્સની માત્ર વ્યાખ્યા આપો.}
\begin{solutionbox}
    \begin{itemize}
        \item \textbf{આસ્પેક્ટ રેશીઓ}: ટેલિવિઝન સ્ક્રીનની પહોળાઈથી ઊંચાઈનો ગુણોત્તર (સામાન્ય રીતે HDTV માટે 16:9, જૂના TV માટે 4:3).
        \item \textbf{લ્યુમિનેન્સ}: વિડિયો સિગ્નલનો બ્રાઇટનેસ ઘટક જે તીવ્રતાની માહિતી વહન કરે છે (Y તરીકે દર્શાવાય છે).
        \item \textbf{ક્રોમિનેન્સ}: વિડિયો સિગ્નલનો રંગ ઘટક જે રંગની માહિતી વહન કરે છે (U અને V અથવા Cb અને Cr તરીકે દર્શાવાય છે).
    \end{itemize}

    \begin{mnemonicbox}
        \mnemonic{ALC: બધા પ્રકાશમાં રંગ હોય છે}
    \end{mnemonicbox}
\end{solutionbox}

\questionmarks{2(બ)}{4}{પાલ –ડી ડીકોડરનો ફક્ત ડાયાગ્રામ દોરો. ક્રોમા સિગ્નલનાં બે ઘટકો યુ અને વી ને કેવી રીતે છુટા પાડવામાં આવે છે?}
\begin{solutionbox}
    \textbf{PAL-D ડીકોડર ડાયાગ્રામ:}

    \begin{figure}[H]
        \centering
        \begin{tikzpicture}[gtu flow]
            \node (Input) [gtu block] {કમ્પોઝિટ વિડિયો};
            \node (Filter) [gtu block, right=of Input] {કોમ્બ ફિલ્ટર};
            
            \node (Y) [gtu block, above right=1cm of Filter] {Y પ્રોસેસિંગ};
            \node (Delay) [gtu block, below right=1cm of Filter] {ડિલે લાઇન (64$\mu$s)};
            
            \node (Switch) [gtu block, right=of Delay] {ફેઝ સ્વિચ};
            \node (Demod) [gtu block, below=of Switch] {સિંક ડિમોડ્યુલેટર};
            
            \node (U) [gtu block, right=of Demod] {U (B-Y)};
            \node (V) [gtu block, left=of Demod] {V (R-Y)};
            
            \draw [gtu arrow] (Input) -- (Filter);
            \draw [gtu arrow] (Filter) |- (Y);
            \draw [gtu arrow] (Filter) |- node[above right, font=\footnotesize] {ક્રોમા} (Delay);
            \draw [gtu arrow] (Delay) -- (Switch);
            \draw [gtu arrow] (Switch) -- (Demod);
            \draw [gtu arrow] (Demod) -- (U);
            \draw [gtu arrow] (Demod) -- (V);
        \end{tikzpicture}
        \caption{PAL-D ડીકોડર}
    \end{figure}

    \begin{itemize}
        \item \textbf{કોમ્બ ફિલ્ટર}: લ્યુમિનન્સ (Y)ને ક્રોમિનન્સ સિગ્નલથી અલગ કરે છે.
        \item \textbf{ડિલે લાઇન}: ક્રોમા સિગ્નલને એક લાઇન પીરિયડ (64$\mu$s) સુધી વિલંબિત કરે છે.
        \item \textbf{ફેઝ ઓલ્ટરનેટિંગ સ્વિચ}: વૈકલ્પિક લાઈનો પર V ઘટકને ઉલટાવે છે.
        \item \textbf{સિંક્રોનસ ડિમોડ્યુલેટર}: U અને V ઘટકોને કાઢવા માટે સબકેરિયર રેફરન્સનો ઉપયોગ કરે છે.
        \item \textbf{U ઘટક}: બ્લુ-માઈનસ-લ્યુમિનન્સ (B-Y) રજૂ કરે છે.
        \item \textbf{V ઘટક}: રેડ-માઈનસ-લ્યુમિનન્સ (R-Y) રજૂ કરે છે.
    \end{itemize}

    \begin{mnemonicbox}
        \mnemonic{CODES: ક્રોમિનન્સ માત્ર સિગ્નલ્સ કાઢીને ડિકોડિંગ કરે છે}
    \end{mnemonicbox}
\end{solutionbox}

\questionmarks{2(ક)}{7}{એલસીડી ટીવીની કાર્યપદ્ધતિ સમજાવો. કોઈ પણ બે ટેકનીકલ સ્પેસિફિકેશન લખો.}
\begin{solutionbox}
    \textbf{LCD ટેલિવિઝન કાર્યપદ્ધતિ:}

    \begin{figure}[H]
        \centering
        \begin{tikzpicture}[gtu flow]
            \node (Back) [gtu block] {બેકલાઇટ};
            \node (Pol1) [gtu block, right=of Back] {પોલરાઇઝર 1};
            \node (LC) [gtu block, right=of Pol1] {લિક્વિડ ક્રિસ્ટલ};
            \node (Color) [gtu block, right=of LC] {કલર ફિલ્ટર};
            \node (Pol2) [gtu block, right=of Color] {પોલરાઇઝર 2};
            \node (Screen) [gtu block, right=of Pol2] {સ્ક્રીન};
            
            \node (Signal) [gtu block, below=of LC] {વિડિયો સિગ્નલ};
            \node (TFT) [gtu block, above=of LC] {TFT મેટ્રિક્સ};
            
            \draw [gtu arrow] (Back) -- (Pol1);
            \draw [gtu arrow] (Pol1) -- (LC);
            \draw [gtu arrow] (LC) -- (Color);
            \draw [gtu arrow] (Color) -- (Pol2);
            \draw [gtu arrow] (Pol2) -- (Screen);
            
            \draw [gtu arrow] (Signal) -- (LC);
            \draw [gtu arrow] (TFT) -- (LC);
        \end{tikzpicture}
        \caption{LCD પેનલ સ્ટ્રક્ચર}
    \end{figure}

    \textbf{કાર્યપ્રક્રિયા:}
    \begin{enumerate}
        \item \textbf{બેકલાઇટ}: CCFL અથવા LED સફેદ પ્રકાશનો સ્ત્રોત પૂરો પાડે છે.
        \item \textbf{TFT મેટ્રિક્સ}: થિન-ફિલ્મ ટ્રાન્ઝિસ્ટર્સ દરેક પિક્સેલ પર વોલ્ટેજને નિયંત્રિત કરે છે.
        \item \textbf{લિક્વિડ ક્રિસ્ટલ લેયર}: અણુઓ લાગુ વોલ્ટેજના આધારે વળે છે.
        \item \textbf{પોલરાઇઝર્સ}: પ્રથમ ફિલ્ટર પ્રકાશને સંરેખિત કરે છે, બીજો માત્ર ફેરવેલા પ્રકાશને પસાર કરે છે.
        \item \textbf{કલર ફિલ્ટર્સ}: RGB ફિલ્ટર્સ રંગીન પિક્સેલ બનાવે છે.
        \item \textbf{ઇમેજ ફોર્મેશન}: વેરિંગ વોલ્ટેજ દરેક પિક્સેલ દ્વારા પ્રકાશના માર્ગને નિયંત્રિત કરે છે.
    \end{enumerate}

    \textbf{ટેકનીકલ સ્પેસિફિકેશન:}
    \begin{itemize}
        \item \textbf{રેઝોલ્યુશન}: 1920$\times$1080 (ફુલ HD) અથવા 3840$\times$2160 (4K UHD)
        \item \textbf{રિફ્રેશ રેટ}: 60Hz, 120Hz, અથવા 240Hz
    \end{itemize}

    \begin{mnemonicbox}
        \mnemonic{BALTIC: બેકલાઇટ રંગોને પ્રકાશિત કરવા માટે તરલ પદાર્થને સક્રિય કરે છે}
    \end{mnemonicbox}
\end{solutionbox}

\questionmarks{2(અ) OR}{3}{ગ્રાસમેનનો નિયમ લખી તેને એડીટીવ મિક્સિંગના કોન્સેપ્ટથી સમજાવો.}
\begin{solutionbox}
    \textbf{ગ્રાસમેનનો નિયમ:}
    કોઈપણ રંગને ત્રણ પ્રાથમિક રંગોના રૈખિક સંયોજન દ્વારા મેળવી શકાય છે.

    \textbf{એડિટિવ કલર મિક્સિંગ:}

    \begin{figure}[H]
        \centering
        \begin{tikzpicture}[gtu flow]
            \node (R) [gtu block, fill=red!20] {લાલ};
            \node (G) [gtu block, fill=green!20, right=2cm of R] {લીલો};
            \node (B) [gtu block, fill=blue!20, right=2cm of G] {વાદળી};
            
            \node (Y) [gtu block, fill=yellow!20, below right=1cm of R] {પીળો (R+G)};
            \node (C) [gtu block, fill=cyan!20, below right=1cm of G] {સાયન (G+B)};
            \node (M) [gtu block, fill=magenta!20, below left=1cm of B] {મેજેન્ટા (B+R)};
            
            \node (W) [gtu block, fill=gray!10, below=3cm of G] {સફેદ (R+G+B)};
            
            \draw [gtu arrow] (R) -- (Y);
            \draw [gtu arrow] (G) -- (Y);
            
            \draw [gtu arrow] (G) -- (C);
            \draw [gtu arrow] (B) -- (C);
            
            \draw [gtu arrow] (R) -- (M);
            \draw [gtu arrow] (B) -- (M);
            
            \draw [gtu arrow] (Y) -- (W);
            \draw [gtu arrow] (C) -- (W);
            \draw [gtu arrow] (M) -- (W);
        \end{tikzpicture}
        \caption{એડિટિવ કલર મિક્સિંગ}
    \end{figure}

    \begin{itemize}
        \item \textbf{સિદ્ધાંત}: અલગ-અલગ રંગોનો પ્રકાશ ઉમેરવાથી નવા રંગો ઉત્પન્ન થાય છે.
        \item \textbf{પ્રાથમિક રંગો}: લાલ, લીલો, અને વાદળી.
        \item \textbf{ગૌણ રંગો}: પીળો (R+G), સાયન (G+B), મેજેન્ટા (B+R).
        \item \textbf{ઉદાહરણ}: RGB ની સમાન તીવ્રતા સફેદ પ્રકાશ બનાવે છે.
    \end{itemize}

    \begin{mnemonicbox}
        \mnemonic{RGB-ACM: લાલ લીલો વાદળી - ઉમેરણ વધુ રંગો બનાવે છે}
    \end{mnemonicbox}
\end{solutionbox}

\questionmarks{2(બ) OR}{4}{ડીટીએચ રિસિવરનો બ્લોક ડાયાગ્રામ દોરો અને સમજાવો.}
\begin{solutionbox}
    \textbf{ડીટીએચ રિસિવર ડાયાગ્રામ:}

    \begin{figure}[H]
        \centering
        \begin{tikzpicture}[gtu flow]
            \node (Dish) [gtu block] {ડિશ/LNB};
            \node (Tuner) [gtu block, right=of Dish] {ટ્યુનર};
            \node (Demod) [gtu block, right=of Tuner] {ડિમોડ્યુલેટર};
            \node (Decoder) [gtu block, below=of Demod] {MPEG ડિકોડર};
            
            \node (Proc) [gtu block, left=of Decoder] {IV/ઓડિયો પ્રોસેસર};
            \node (TV) [gtu block, left=of Proc] {TV આઉટપુટ};
            
            \node (CAM) [gtu block, right=of Decoder] {CAM};
            \node (Card) [gtu block, below=of CAM] {સ્માર્ટ કાર્ડ};
            
            \draw [gtu arrow] (Dish) -- (Tuner);
            \draw [gtu arrow] (Tuner) -- (Demod);
            \draw [gtu arrow] (Demod) -- (Decoder);
            \draw [gtu arrow] (Decoder) -- (Proc);
            \draw [gtu arrow] (Proc) -- (TV);
            
            \draw [gtu arrow] (CAM) -- (Decoder);
            \draw [gtu arrow] (Card) -- (CAM);
        \end{tikzpicture}
        \caption{DTH રિસિવર}
    \end{figure}

    \begin{itemize}
        \item \textbf{સેટેલાઇટ ડિશ}: નબળા સેટેલાઇટ સિગ્નલ્સ એકત્રિત કરે છે (10.7-12.75 GHz).
        \item \textbf{LNB} (લો નોઇઝ બ્લોક): સિગ્નલને એમ્પલિફાય કરે છે અને ઓછી ફ્રિક્વન્સીમાં રૂપાંતરિત કરે છે (950-2150 MHz).
        \item \textbf{ટ્યુનર}: ઇચ્છિત ટ્રાન્સપોન્ડર ફ્રિક્વન્સી પસંદ કરે છે.
        \item \textbf{ડિમોડ્યુલેટર}: કેરિયર સિગ્નલમાંથી ડિજિટલ ડેટા કાઢે છે.
        \item \textbf{MPEG ડિકોડર}: ઓડિયો/વિડિયો ડેટાને ડિકોમ્પ્રેસ કરે છે.
        \item \textbf{CAM અને સ્માર્ટ કાર્ડ}: ડિક્રિપ્શન અને સબ્સ્ક્રિપ્શન વેરિફિકેશન પૂરા પાડે છે.
        \item \textbf{આઉટપુટ}: ટેલિવિઝન પર પ્રદર્શિત કરવા માટે સિગ્નલ્સ પ્રોસેસ કરે છે.
    \end{itemize}

    \begin{mnemonicbox}
        \mnemonic{SLTD-MCS: સેટેલાઇટ્સ ડિકોડર્સ મારફતે ક્લિયર સિગ્નલ્સ જોડે છે}
    \end{mnemonicbox}
\end{solutionbox}

\questionmarks{2(ક) OR}{7}{નીચે દશાર્વ્યા મુજબની ફ્રીક્વન્શી આપો. (used in color TV system)}
\begin{solutionbox}
    \textbf{ટેબલ: કલર ટીવી સ્ટાન્ડર્ડ્સ (PAL-B/G)} \\
    \begin{tabulary}{\linewidth}{|L|L|}
        \hline
        \textbf{પેરામીટર} & \textbf{ફ્રિક્વન્સી/સ્ટાન્ડર્ડ} \\
        \hline
        \textbf{VIF (વિડિયો ઇન્ટરમીડિયેટ ફ્રિક્વન્સી)} & 38.9 MHz \\
        \hline
        \textbf{SIF (સાઉન્ડ ઇન્ટરમીડિયેટ ફ્રિક્વન્સી)} & 33.4 MHz \\
        \hline
        \textbf{કલર સબ કેરિયર ફ્રિક્વન્સી} & 4.43361875 MHz \\
        \hline
        \textbf{વર્ટિકલ બ્લેન્કિંગ ફ્રિક્વન્સી} & 50 Hz \\
        \hline
        \textbf{હોરિઝોન્ટલ સિંક ફ્રિક્વન્સી} & 15.625 kHz \\
        \hline
        \textbf{ઇન્ટર કેરિયર સાઉન્ડ સિગ્નલ ફ્રિક્વન્સી} & 5.5 MHz \\
        \hline
        \textbf{એક ચેનલની બેન્ડવીથ} & 7 MHz (VHF), 8 MHz (UHF) \\
        \hline
    \end{tabulary}

    \begin{mnemonicbox}
        \mnemonic{વિડિયો સ્પેશિયલ કલર વર્ટિકલી હોરિઝોન્ટલી ઇન્ટર ચેનલ}
    \end{mnemonicbox}
\end{solutionbox}

\questionmarks{3(અ)}{3}{ફઝી લોજીક એટલે શું? વોશિંગ મશીનમાં તેનો ઉપયોગ સમજાવો.}
\begin{solutionbox}
    \textbf{ફઝી લોજીક}: ગાણિતિક અભિગમ જે નિશ્ચિત, બાઇનરી લોજિકને બદલે આશરે તર્ક સાથે કામ કરે છે, 0 અને 1 વચ્ચે સત્ય મૂલ્યોની ડિગ્રીની મંજૂરી આપે છે.

    \textbf{વોશિંગ મશીનમાં ઉપયોગ:}

    \begin{figure}[H]
        \centering
        \begin{tikzpicture}[gtu flow]
            \node (Sensors) [gtu block] {સેન્સર્સ};
            \node (Controller) [gtu block, right=of Sensors] {ફઝી કંટ્રોલર};
            \node (Decision) [gtu block, right=of Controller] {નિર્ણય લેવો};
            \node (Action) [gtu block, right=of Decision] {કંટ્રોલ ક્રિયાઓ};
            
            \node (Inputs) [gtu block, above=of Sensors] {ઇનપુટ વેરિએબલ્સ};
            \node (Outputs) [gtu block, below=of Action] {આઉટપુટ};
            
            \draw [gtu arrow] (Sensors) -- (Controller);
            \draw [gtu arrow] (Controller) -- (Decision);
            \draw [gtu arrow] (Decision) -- (Action);
            
            \draw [gtu arrow] (Inputs) -- (Sensors);
            \draw [gtu arrow] (Action) -- (Outputs);
        \end{tikzpicture}
        \caption{વોશિંગ મશીનમાં ફઝી લોજીક}
    \end{figure}

    \begin{itemize}
        \item \textbf{ઇનપુટ વેરિએબલ્સ}: લોડ વજન, ફેબ્રિક પ્રકાર, પાણીની કઠોરતા, ગંદકી સ્તર.
        \item \textbf{પ્રોસેસિંગ}: કંટ્રોલર એકસાથે બહુવિધ સ્થિતિઓનું મૂલ્યાંકન કરે છે.
        \item \textbf{આઉટપુટ}: પાણીનું સ્તર, ધોવાનો સમય, રિન્સ સાયકલ, સ્પિન સ્પીડ સમાયોજિત કરે છે.
    \end{itemize}

    \begin{mnemonicbox}
        \mnemonic{FIND: ફઝી ઇન્ટેલિજન્સ નિર્ણયોનું નેવિગેશન કરે છે}
    \end{mnemonicbox}
\end{solutionbox}

\questionmarks{3(બ)}{4}{એર કન્ડીશનીંગની વ્યાખ્યા આપો. ફ્રિજની કાર્યપધ્ધતિ સમજાવો. ફ્રિજનાં ટેકનીકલ સ્પેસિફિકેશન લખો.}
\begin{solutionbox}
    \textbf{એર કન્ડીશનીંગ}: આરામ સુધારવા માટે ઇનડોર હવામાંથી ગરમી અને ભેજ દૂર કરવાની પ્રક્રિયા.

    \textbf{ફ્રિજ કાર્યપધ્ધતિ:}

    \begin{figure}[H]
        \centering
        \begin{tikzpicture}[gtu flow]
            \node (Comp) [gtu block] {કમ્પ્રેસર};
            \node (Cond) [gtu block, right=of Comp] {કન્ડેન્સર};
            \node (Valve) [gtu block, below=of Cond] {એક્સપાન્શન વાલ્વ};
            \node (Evap) [gtu block, left=of Valve] {ઇવેપોરેટર};
            
            \draw [gtu arrow] (Comp) -- node[above, font=\footnotesize] {ઉચ્ચ દબાણ વરાળ} (Cond);
            \draw [gtu arrow] (Cond) -- node[right, font=\footnotesize] {ઉચ્ચ દબાણ પ્રવાહી} (Valve);
            \draw [gtu arrow] (Valve) -- node[below, font=\footnotesize] {નીચા દબાણ પ્રવાહી} (Evap);
            \draw [gtu arrow] (Evap) -- node[left, font=\footnotesize] {નીચા દબાણ વરાળ} (Comp);
        \end{tikzpicture}
        \caption{રેફ્રિજરેશન સાયકલ}
    \end{figure}

    \textbf{કાર્ય સાયકલ:}
    \begin{enumerate}
        \item \textbf{કમ્પ્રેસર}: રેફ્રિજરન્ટ ગેસને કોમ્પ્રેસ કરે છે, તાપમાન વધારે છે.
        \item \textbf{કન્ડેન્સર}: ગરમ ગેસ બહારની હવામાં ગરમી છોડે છે, પ્રવાહી બની જાય છે.
        \item \textbf{એક્સપાન્શન વાલ્વ}: પ્રવાહી વિસ્તરે છે, ઝડપથી ઠંડું થાય છે.
        \item \textbf{ઇવેપોરેટર}: ઠંડું રેફ્રિજરન્ટ કેબિનેટની અંદરથી ગરમી શોષે છે.
    \end{enumerate}

    \textbf{ટેકનીકલ સ્પેસિફિકેશન્સ:}
    \begin{itemize}
        \item \textbf{કેપેસિટી}: 150-500 લિટર્સ
        \item \textbf{એનર્જી રેટિંગ}: 3-5 સ્ટાર
        \item \textbf{પાવર કન્ઝમ્પશન}: 100-300 kWh/વર્ષ
    \end{itemize}

    \begin{mnemonicbox}
        \mnemonic{CEVA: Compress, Expel heat, Valve expands, Absorb heat}
    \end{mnemonicbox}
\end{solutionbox}

\questionmarks{3(ક)}{7}{ફન્કશનલ ડાયાગ્રામ વડે માઈક્રોવેવ ઓવનની કાર્યપધ્ધતી સમજાવી તેના ટેકનીકલ સ્પેસિફિકેશન લખો.}
\begin{solutionbox}
    \textbf{માઈક્રોવેવ ઓવન કાર્યપધ્ધતિ:}

    \begin{figure}[H]
        \centering
        \begin{tikzpicture}[gtu flow]
            \node (Power) [gtu block] {પાવર સપ્લાય};
            \node (Control) [gtu block, right=of Power] {કંટ્રોલ પેનલ};
            \node (Timer) [gtu block, right=of Control] {ટાઇમર/કંટ્રોલર};
            
            \node (Mag) [gtu block, below=of Timer] {મેગ્નેટ્રોન};
            \node (Guide) [gtu block, left=of Mag] {વેવગાઇડ};
            \node (Cavity) [gtu block, left=of Guide, minimum size=2cm] {કુકિંગ કેવિટી};
            
            \node (Motor) [gtu block, below=of Cavity] {ટર્નટેબલ મોટર};
            \node (Safety) [gtu block, above=of Timer] {સેફ્ટી ઇન્ટરલોક્સ};
            
            \draw [gtu arrow] (Power) -- (Control);
            \draw [gtu arrow] (Control) -- (Timer);
            \draw [gtu arrow] (Timer) -- (Mag);
            \draw [gtu arrow] (Mag) -- (Guide);
            \draw [gtu arrow] (Guide) -- (Cavity);
            
            \draw [gtu arrow] (Timer) |- (Motor);
            \draw [gtu arrow] (Safety) -- (Timer);
            \draw [gtu arrow] (Motor) -- (Cavity);
        \end{tikzpicture}
        \caption{માઈક્રોવેવ ઓવન સિસ્ટમ}
    \end{figure}

    \textbf{કાર્યસિદ્ધાંત:}
    \begin{enumerate}
        \item \textbf{મેગ્નેટ્રોન}: 2.45 GHz ફ્રિક્વન્સી પર માઇક્રોવેવ્સ ઉત્પન્ન કરે છે.
        \item \textbf{વેવગાઇડ}: કુકિંગ કેવિટીમાં માઇક્રોવેવ્સનું માર્ગદર્શન કરે છે.
        \item \textbf{પાણીના અણુઓ}: માઇક્રોવેવ્સ પાણીના અણુઓને કંપિત કરે છે.
        \item \textbf{ગરમી ઉત્પાદન}: આણ્વિક કંપન ઘર્ષણ અને ગરમી પેદા કરે છે.
        \item \textbf{ટર્નટેબલ}: સમાન રાંધવા માટે ખોરાક ફેરવે છે.
        \item \textbf{સેફ્ટી ઇન્ટરલોક્સ}: ડોર ખુલ્લો હોય ત્યારે ઓપરેશન અટકાવે છે.
    \end{enumerate}

    \textbf{ટેકનીકલ સ્પેસિફિકેશન્સ:}
    \begin{itemize}
        \item \textbf{પાવર આઉટપુટ}: 700-1200 વોટ
        \item \textbf{ફ્રિક્વન્સી}: 2.45 GHz
        \item \textbf{કેપેસિટી}: 20-40 લિટર્સ
        \item \textbf{કુકિંગ મોડ્સ}: માઇક્રોવેવ, ગ્રિલ, કન્વેક્શન, કોમ્બિનેશન
    \end{itemize}

    \begin{mnemonicbox}
        \mnemonic{MICRO: મેગ્નેટ્રોન કંપિત આંદોલનો દ્વારા રાંધવાની શરૂઆત કરે છે}
    \end{mnemonicbox}
\end{solutionbox}

\questionmarks{3(અ) OR}{3}{કોલાર પેનલના ટેકનીકલ સ્પેસિફિકેશન આપો. સોલાર રૂફ ટોપ સીસ્ટમનાં ફાયદા અને ગેરફાયદા આપો.}
\begin{solutionbox}
    \textbf{સોલાર પેનલ ટેકનીકલ સ્પેસિફિકેશન્સ:}
    \begin{itemize}
        \item \textbf{પાવર રેટિંગ}: 250-400 Wp (વોટ પીક)
        \item \textbf{કાર્યક્ષમતા}: 15-22\%
        \item \textbf{સેલ પ્રકાર}: મોનોક્રિસ્ટલાઇન, પોલિક્રિસ્ટલાઇન, અથવા થિન ફિલ્મ
    \end{itemize}

    \textbf{ફાયદા અને ગેરફાયદા:}
    \begin{tabulary}{\linewidth}{|L|L|}
        \hline
        \textbf{ફાયદા} & \textbf{ગેરફાયદા} \\
        \hline
        નવીકરણીય ઊર્જા સ્ત્રોત & ઉચ્ચ પ્રારંભિક ખર્ચ \\
        \hline
        વીજળી બિલમાં ઘટાડો & હવામાન પર આધારિત \\
        \hline
        ઓછો જાળવણી ખર્ચ & મોટી જગ્યાની જરૂર \\
        \hline
        અવાજ પ્રદૂષણ નથી & રાત્રે મર્યાદિત ઉત્પાદન \\
        \hline
    \end{tabulary}

    \begin{mnemonicbox}
        \mnemonic{SERLN: સોલાર એનર્જી લાંબા ગાળે ખર્ચ ઘટાડે છે}
    \end{mnemonicbox}
\end{solutionbox}

\questionmarks{3(બ) OR}{4}{વોશિંગ મશીનનાં અલગ અલગ પ્રકારો જણાવી ફ્રન્ટલોડ અને ટોપ લોડ પ્રકારના વોશિંગ મશીન ની સરખામણી કરો.}
\begin{solutionbox}
    \textbf{વોશિંગ મશીનના પ્રકારો:}
    \begin{itemize}
        \item ટોપ લોડ (એજિટેટર અને ઇમ્પેલર)
        \item ફ્રન્ટ લોડ
        \item સેમી-ઓટોમેટિક
        \item ફુલી ઓટોમેટિક
    \end{itemize}

    \textbf{સરખામણી:}
    \begin{tabulary}{\linewidth}{|l|L|L|}
        \hline
        \textbf{પેરામીટર} & \textbf{ફ્રન્ટ લોડ} & \textbf{ટોપ લોડ} \\
        \hline
        \textbf{પાણીનો વપરાશ} & ઓછો (40-60 લિટર) & વધારે (80-120 લિટર) \\
        \hline
        \textbf{ઊર્જા કાર્યક્ષમતા} & ઉચ્ચ & નીચી \\
        \hline
        \textbf{સફાઈ પ્રદર્શન} & વધુ સારું & સારું \\
        \hline
        \textbf{જગ્યાની જરૂરિયાત} & સ્ટેક કરી શકાય છે & ઉપર ક્લિયરન્સની જરૂર છે \\
        \hline
        \textbf{કિંમત} & ઉચ્ચ & નીચી \\
        \hline
        \textbf{સાયકલ સમયગાળો} & લાંબો (60-120 મિનિટ) & ટૂંકો (30-60 મિનિટ) \\
        \hline
    \end{tabulary}

    \begin{mnemonicbox}
        \mnemonic{FTEST: ફ્રન્ટ-લોડર વધારાની જગ્યા લે છે પરંતુ કાર્યક્ષમતામાં વિજય મેળવે છે}
    \end{mnemonicbox}
\end{solutionbox}

\questionmarks{3(ક) OR}{7}{સોલાર રૂફ ટોપ સીસ્ટમને વર્ગીકૃત કરો. ગ્રીડ ક્નેકટેડ સોલાર રૂફ ટોપ સીસ્ટમને યોગ્ય ડાયાગ્રામ વડે સમજાવો. સોલાર રૂફ ટોપ સીસ્ટમની જાળવણી માટેના પગલા જણાવો.}
\begin{solutionbox}
    \textbf{સોલાર રૂફટોપ સિસ્ટમનું વર્ગીકરણ:}
    ગ્રિડ-કનેક્ટેડ (ઓન-ગ્રિડ), ઓફ-ગ્રિડ (સ્ટેન્ડઅલોન), હાઇબ્રિડ.

    \textbf{ગ્રિડ-કનેક્ટેડ સોલાર સિસ્ટમ:}

    \begin{figure}[H]
        \centering
        \begin{tikzpicture}[gtu flow]
            \node (Panels) [gtu block] {સોલાર પેનલ્સ};
            \node (DCBox) [gtu block, right=of Panels] {DC બોક્સ};
            \node (Inverter) [gtu block, right=of DCBox] {સોલાર ઇન્વર્ટર};
            
            \node (ACBox) [gtu block, below=of Inverter] {AC બોક્સ};
            \node (Load) [gtu block, left=of ACBox] {ઘરનાં લોડ્સ};
            \node (Meter) [gtu block, right=of ACBox] {બાય-ડાયરેક્શનલ મીટર};
            \node (Grid) [gtu block, right=of Meter] {ગ્રિડ};
            
            \draw [gtu arrow] (Panels) -- node[above] {DC} (DCBox);
            \draw [gtu arrow] (DCBox) -- (Inverter);
            \draw [gtu arrow] (Inverter) -- node[right] {AC} (ACBox);
            \draw [gtu arrow] (ACBox) -- (Load);
            \draw [gtu arrow] (ACBox) -- (Meter);
            \draw [gtu arrow] (Meter) <-> (Grid);
        \end{tikzpicture}
        \caption{ઓન-ગ્રિડ સોલાર સિસ્ટમ}
    \end{figure}

    \textbf{કાર્યપ્રણાલી:}
    \begin{enumerate}
        \item \textbf{સોલાર પેનલ્સ}: સૂર્યપ્રકાશને DC વીજળીમાં રૂપાંતરિત કરે છે.
        \item \textbf{જંક્શન બોક્સ}: આઉટપુટ્સને જોડે છે, સુરક્ષા પ્રદાન કરે છે.
        \item \textbf{ઇન્વર્ટર}: DC ને ગ્રિડ-સંગત AC માં રૂપાંતરિત કરે છે.
        \item \textbf{બાય-ડાયરેક્શનલ મીટર}: વીજળીના આયાત/નિકાસને માપે છે.
        \item \textbf{વધારાનું ઉત્પાદન}: ગ્રિડમાં પાછું ફીડ કરે છે (નેટ મીટરિંગ).
    \end{enumerate}

    \textbf{જાળવણી પગલાં:}
    \begin{itemize}
        \item પેનલોની નિયમિત સફાઈ (ધૂળ, પક્ષીઓનો કચરો).
        \item ક્ષારના લીધે ઇલેક્ટ્રિકલ કનેક્શન તપાસવા.
        \item ઇન્વર્ટર ડેટા મારફતે સિસ્ટમ પરફોર્મન્સ મોનિટરિંગ.
        \item છાંયડો અટકાવવા નજીકના વૃક્ષોની છટણી.
        \item લાયક ટેકનિશિયન દ્વારા વાર્ષિક નિરીક્ષણ.
    \end{itemize}

    \begin{mnemonicbox}
        \mnemonic{SPICED: સોલાર પેનલ્સ ઇન્વર્ટ કરંટ ઇલેક્ટ્રિકલ ડિસ્ટ્રિબ્યુશન માટે}
    \end{mnemonicbox}
\end{solutionbox}

\questionmarks{4(અ)}{3}{ફોટો કોપીયર મશીનનો કાર્યસિદ્ધાંત લેટેન્ટ ઇમેજના કોન્સેપ્ટ વડે ટૂંકમાં સમજાવો.}
\begin{solutionbox}
    \textbf{ફોટોકોપિયર પ્રક્રિયા:}

    \begin{figure}[H]
        \centering
        \begin{tikzpicture}[gtu flow]
            \node (Charge) [gtu block] {1. ચાર્જિંગ};
            \node (Expose) [gtu block, right=of Charge] {2. એક્સ્પોઝર};
            \node (Develop) [gtu block, right=of Expose] {3. ડેવલપિંગ};
            \node (Transfer) [gtu block, below=of Develop] {4. ટ્રાન્સફર};
            \node (Fuse) [gtu block, left=of Transfer] {5. ફ્યુઝિંગ};
            \node (Clean) [gtu block, left=of Fuse] {6. ક્લીનિંગ};
            
            \draw [gtu arrow] (Charge) -- (Expose);
            \draw [gtu arrow] (Expose) -- (Develop);
            \draw [gtu arrow] (Develop) -- (Transfer);
            \draw [gtu arrow] (Transfer) -- (Fuse);
            \draw [gtu arrow] (Fuse) -- (Clean);
            \draw [gtu arrow] (Clean) -- (Charge);
        \end{tikzpicture}
        \caption{ઝેરોગ્રાફી સાયકલ}
    \end{figure}

    \textbf{લેટેન્ટ ઈમેજ કોન્સેપ્ટ:}
    \begin{itemize}
        \item \textbf{ચાર્જિંગ}: ફોટોસેન્સિટિવ ડ્રમને સમાન પોઝિટિવ ચાર્જ મળે છે.
        \item \textbf{એક્સ્પોઝર}: પ્રકાશ મૂળ દસ્તાવેજમાંથી ડ્રમ પર પ્રતિબિંબિત થાય છે.
        \item \textbf{લેટેન્ટ ઈમેજ}: પ્રકાશિત વિસ્તારો ડ્રમને ડિસ્ચાર્જ કરે છે, અદૃશ્ય ઇલેક્ટ્રોસ્ટેટિક ઈમેજ બનાવે છે.
        \item \textbf{ડેવલપમેન્ટ}: નેગેટિવ ચાર્જ્ડ ટોનર કણો પોઝિટિવ એરિયા તરફ આકર્ષાય છે.
        \item \textbf{ટ્રાન્સફર}: ઇલેક્ટ્રિકલ આકર્ષણ દ્વારા ટોનર કાગળ પર ટ્રાન્સફર થાય છે.
        \item \textbf{ફ્યુઝિંગ}: ગરમી અને દબાણ ટોનરને કાગળ સાથે કાયમી રીતે જોડે છે.
    \end{itemize}

    \begin{mnemonicbox}
        \mnemonic{CEDTFC: Charging Exposure Develops The Final Copy}
    \end{mnemonicbox}
\end{solutionbox}

\questionmarks{4(બ)}{4}{યોગ્ય ડાયાગ્રામ વડે લેસર પ્રિન્ટરનો કાર્યસિદ્ધાંત સમજાવો.}
\begin{solutionbox}
    \textbf{લેસર પ્રિન્ટર ડાયાગ્રામ:}

    \begin{figure}[H]
        \centering
        \begin{tikzpicture}[gtu flow]
            \node (Data) [gtu block] {ડેટા પ્રોસેસિંગ};
            \node (Laser) [gtu block, right=of Data] {લેસર યુનિટ};
            \node (Drum) [gtu block, right=of Laser] {ફોટો ડ્રમ};
            \node (Corona) [gtu block, above=of Drum] {પ્રાયમરી કોરોના};
            
            \node (Dev) [gtu block, below=of Drum] {ડેવલપર};
            \node (Trans) [gtu block, left=of Dev] {ટ્રાન્સફર};
            \node (Fuse) [gtu block, left=of Trans] {ફ્યુઝર};
            \node (Out) [gtu block, left=of Fuse] {આઉટપુટ};
            
            \draw [gtu arrow] (Data) -- (Laser);
            \draw [gtu arrow] (Laser) -- (Drum);
            \draw [gtu arrow] (Corona) -- (Drum);
            \draw [gtu arrow] (Drum) -- (Dev);
            \draw [gtu arrow] (Dev) -- (Trans);
            \draw [gtu arrow] (Trans) -- (Fuse);
            \draw [gtu arrow] (Fuse) -- (Out);
        \end{tikzpicture}
        \caption{લેસર પ્રિન્ટર મિકેનિઝમ}
    \end{figure}

    \textbf{કાર્યપ્રક્રિયા:}
    \begin{itemize}
        \item \textbf{રાસ્ટર ઈમેજ પ્રોસેસિંગ}: કમ્પ્યુટર ડેટા બિટમેપમાં રૂપાંતરિત થાય છે.
        \item \textbf{ચાર્જિંગ}: કોરોના વાયર ડ્રમને એકસરખો નેગેટિવ ચાર્જ આપે છે.
        \item \textbf{રાઇટિંગ}: લેસર બીમ ઈમેજના પેટર્નમાં ચાર્જને ન્યુટ્રલાઈઝ કરે છે.
        \item \textbf{ડેવલપિંગ}: ટોનર ન્યુટ્રલાઈઝડ એરિયા તરફ આકર્ષાય છે.
        \item \textbf{ટ્રાન્સફર}: ટોનરને આકર્ષિત કરવા કાગળને પોઝિટિવ ચાર્જ આપવામાં આવે છે.
        \item \textbf{ફ્યુઝિંગ}: હીટ રોલર્સ ટોનરને કાગળ પર કાયમી રીતે પિગળાવે છે.
    \end{itemize}

    \begin{mnemonicbox}
        \mnemonic{RASTER: રાસ્ટર-ઈમેજ સ્ટેટિક ટોનર આકર્ષે છે, ઇલેક્ટ્રિસિટી રિલીઝ કરે છે}
    \end{mnemonicbox}
\end{solutionbox}

\questionmarks{4(ક)}{7}{ઈંટરનેટ સાથે ક્નેક્ટેડ ડીજીટલ આઈપી કેમેરાવાળો સીસીટીવી સીસ્ટમનો ડાયાગ્રામ દોરીને સમજાવો. અલગ અલગ પાંચ કેમેરાનાં નામ આપો. પીઓઈ કેબલ એટલે શું?}
\begin{solutionbox}
    \textbf{IP CCTV સિસ્ટમ ડાયાગ્રામ:}

    \begin{figure}[H]
        \centering
        \begin{tikzpicture}[gtu flow]
            \node (Cam1) [gtu block] {IP કેમેરા 1};
            \node (Cam2) [gtu block, below=0.5cm of Cam1] {IP કેમેરા 2};
            \node (Switch) [gtu block, right=of Cam1] {નેટવર્ક સ્વિચ (PoE)};
            
            \node (NVR) [gtu block, right=of Switch] {NVR};
            \node (Storage) [gtu block, below=of NVR] {સ્ટોરેજ HDD};
            \node (Monitor) [gtu block, above=of NVR] {લોકલ મોનિટર};
            
            \node (Router) [gtu block, right=of NVR] {રાઉટર};
            \node (Web) [gtu block, right=of Router] {ઈન્ટરનેટ};
            \node (Remote) [gtu block, below=of Web] {રિમોટ વ્યુ};
            
            \draw [gtu arrow] (Cam1) -- (Switch);
            \draw [gtu arrow] (Cam2) -- (Switch);
            \draw [gtu arrow] (Switch) -- (NVR);
            \draw [gtu arrow] (NVR) -- (Storage);
            \draw [gtu arrow] (NVR) -- (Monitor);
            \draw [gtu arrow] (NVR) -- (Router);
            \draw [gtu arrow] (Router) -- (Web);
            \draw [gtu arrow] (Web) -- (Remote);
        \end{tikzpicture}
        \caption{IP CCTV આર્કિટેક્ચર}
    \end{figure}

    \textbf{કાર્યપદ્ધતિ:}
    \begin{itemize}
        \item \textbf{IP કેમેરા}: વિડિયો કેપ્ચર કરી ડિજિટાઈઝ કરે છે.
        \item \textbf{નેટવર્ક ઇન્ફ્રાસ્ટ્રક્ચર}: TCP/IP પ્રોટોકોલ દ્વારા ડેટા ટ્રાન્સમિટ કરે છે.
        \item \textbf{NVR}: વિડિયો સ્ટ્રીમ રેકોર્ડ, મેનેજ અને પ્રોસેસ કરે છે.
        \item \textbf{રાઉટર}: રિમોટ વ્યુઇંગ માટે સુરક્ષિત ઇન્ટરનેટ એક્સેસ પ્રદાન કરે છે.
    \end{itemize}
    
    \textbf{કેમેરાના પ્રકારો:} ડોમ, બુલેટ, PTZ, ફિશઆઈ, થર્મલ.
    
    \textbf{POE કેબલ}: પાવર ઓવર ઈથરનેટ - એક ટેકનોલોજી જે એક જ ઈથરનેટ કેબલ પર પાવર અને ડેટા બંને વહન કરે છે.

    \begin{mnemonicbox}
        \mnemonic{INSPIRE: ઇન્ટરનેટ નેટવર્કિંગ રિમોટ વાતાવરણમાં જગ્યાઓ સુરક્ષિત કરે છે}
    \end{mnemonicbox}
\end{solutionbox}

\questionmarks{4(અ) OR}{3}{ઈંટરનેટ સાથે ક્નેક્ટેડ ડીજીટલ આઈપી કેમેરા વાળી સીસીટીવી સીસ્ટમનાં ફાયદા અને ગેરફાયદા આપો.}
\begin{solutionbox}
    \begin{tabulary}{\linewidth}{|L|L|}
        \hline
        \textbf{ફાયદા} & \textbf{ગેરફાયદા} \\
        \hline
        ઉચ્ચ રેઝોલ્યુશન (1080p થી 4K) & ઉચ્ચ પ્રારંભિક ખર્ચ \\
        \hline
        રિમોટ વ્યુઇંગ ઇન્ટરનેટ દ્વારા & બેન્ડવિડ્થ જરૂરિયાતો \\
        \hline
        સ્કેલેબિલિટી & સરળ વિસ્તરણ & સાયબર સુરક્ષા જોખમો \\
        \hline
        પાવર ઓવર ઈથરનેટ (POE) & નેટવર્ક ડિપેન્ડન્સી \\
        \hline
        એડવાન્સ્ડ એનાલિટિક્સ ક્ષમતાઓ & જટિલ કોન્ફિગરેશન \\
        \hline
    \end{tabulary}

    \begin{mnemonicbox}
        \mnemonic{HIGHER: હાઈ-રેઝોલ્યુશન ઇમેજ ગિવ્સ હાયર ઇવેલ્યુએશન રિમોટલી}
    \end{mnemonicbox}
\end{solutionbox}

\questionmarks{4(બ) OR}{4}{ઈન્કજેટ પ્રિન્ટરને યોગ્ય ડાયાગ્રામ વડે સમજાવો.}
\begin{solutionbox}
    \textbf{ઇન્કજેટ પ્રિન્ટર ડાયાગ્રામ:}

    \begin{figure}[H]
        \centering
        \begin{tikzpicture}[gtu flow]
            \node (Data) [gtu block] {પ્રિન્ટ ડેટા};
            \node (Control) [gtu block, right=of Data] {કંટ્રોલર};
            \node (Head) [gtu block, right=of Control] {પ્રિન્ટ હેડ};
            \node (Cartridge) [gtu block, above=of Head] {ઇન્ક ટેન્ક};
            
            \node (Nozzle) [gtu block, right=of Head] {નોઝલ};
            \node (Paper) [gtu block, below=of Nozzle] {પેપર};
            \node (Feed) [gtu block, left=of Paper] {પેપર ફીડ};
            
            \draw [gtu arrow] (Data) -- (Control);
            \draw [gtu arrow] (Control) -- (Head);
            \draw [gtu arrow] (Cartridge) -- (Head);
            \draw [gtu arrow] (Head) -- (Nozzle);
            \draw [gtu arrow] (Nozzle) -- (Paper);
            \draw [gtu arrow] (Control) -- (Feed);
            \draw [gtu arrow] (Feed) -- (Paper);
        \end{tikzpicture}
        \caption{ઇન્કજેટ પ્રિન્ટર}
    \end{figure}

    \textbf{કાર્યપ્રક્રિયા:}
    \begin{itemize}
        \item \textbf{ડેટા પ્રોસેસિંગ}: કંટ્રોલર ડિજિટલ ડેટાને નોઝલ ઇન્સ્ટ્રક્શન્સમાં રૂપાંતરિત કરે છે.
        \item \textbf{ઇન્ક ઇજેક્શન}:
            \begin{itemize}
                \item થર્મલ: રેઝિસ્ટર્સ ઇન્કને ગરમ કરીને બબલ્સ બનાવે છે.
                \item પિઝોઇલેક્ટ્રિક: સ્ફટિકો ઇન્કને ધકેલવા માટે ફ્લેક્સ થાય છે.
            \end{itemize}
        \item \textbf{સૂકવણી}: ઇન્ક પેપરની સપાટી પર ચોંટી જાય છે.
    \end{itemize}

    \begin{mnemonicbox}
        \mnemonic{PRINT: પેપર રિસીવ્સ ઇન્ક થ્રુ ન્યુમરસ ટાઇની-નોઝલ}
    \end{mnemonicbox}
\end{solutionbox}

\questionmarks{4(ક) OR}{7}{સાદા કેમેરા અને ડીવીઆર વાળી સીસીટીવી સીસ્ટમનો ડાયાગ્રામ દોરો અને સમજાવો. વપરાતા અલગ અલગ પ્રકારનાં કેબલોની યાદી આપો. આધુનિક સીસીટીવી સીસ્ટમમાં વપરાતા અલગ અલગ પ્રકારનાં ચાર કેમેરાઓની ચર્ચા કરો.}
\begin{solutionbox}
    \textbf{એનાલોગ CCTV સિસ્ટમ:}

    \begin{figure}[H]
        \centering
        \begin{tikzpicture}[gtu flow]
            \node (Cam1) [gtu block] {એનાલોગ કેમેરા 1};
            \node (Cam2) [gtu block, below=0.5cm of Cam1] {એનાલોગ કેમેરા 2};
            \node (DVR) [gtu block, right=of Cam1] {DVR};
            \node (HDD) [gtu block, below=of DVR] {સ્ટોરેજ};
            
            \node (Monitor) [gtu block, above=of DVR] {મોનિટર};
            \node (Router) [gtu block, right=of DVR] {રાઉટર};
            \node (Remote) [gtu block, right=of Router] {રિમોટ વ્યુ};
            
            \node (Power) [gtu block, left=of Cam2] {પાવર સપ્લાય};
            
            \draw [gtu arrow] (Cam1) -- node[above, font=\tiny] {Coax} (DVR);
            \draw [gtu arrow] (Cam2) -- node[below, font=\tiny] {Coax} (DVR);
            \draw [gtu arrow] (DVR) -- (HDD);
            \draw [gtu arrow] (DVR) -- (Monitor);
            \draw [gtu arrow] (DVR) -- (Router);
            \draw [gtu arrow] (Router) -- (Remote);
            \draw [gtu arrow] (Power) -- (Cam1);
            \draw [gtu arrow] (Power) -- (Cam2);
        \end{tikzpicture}
        \caption{એનાલોગ CCTV સિસ્ટમ}
    \end{figure}

    \textbf{કેબલના પ્રકારો:} કોએક્સિયલ (RG59), ટ્વિસ્ટેડ પેર (CAT5/6), પાવર કેબલ, ફાઇબર ઓપ્ટિક, સાયમીઝ કેબલ.
    
    \textbf{કેમેરા કેટેગરીઝ:} ફિક્સ્ડ, વેરિફોકલ, નાઇટ વિઝન, HDR.

    \begin{mnemonicbox}
        \mnemonic{CARD: કોએક્સિયલ એનાલોગ રેકોર્ડિંગ ડિવાઇસીસ}
    \end{mnemonicbox}
\end{solutionbox}

\questionmarks{5(અ)}{3}{વ્યાખ્યા આપો: મેન્ટેનન્સ, પ્રિવેન્ટિવ મેન્ટેનન્સ અને પ્રિડિક્ટિવ મેન્ટેનન્સ.}
\begin{solutionbox}
    \begin{itemize}
        \item \textbf{મેન્ટેનન્સ (જાળવણી)}: સાધનોને યોગ્ય કાર્યકારી સ્થિતિમાં જાળવવાની પ્રક્રિયા.
        \item \textbf{પ્રિવેન્ટિવ મેન્ટેનન્સ (નિવારક)}: નિષ્ફળતા થાય તે પહેલાં તેને રોકવા માટેની સુનિશ્ચિત પ્રવૃત્તિઓ.
        \item \textbf{પ્રિડિક્ટિવ મેન્ટેનન્સ (આગાહીક)}: નિષ્ફળતાના સમયની આગાહી કરવા માટે ડેટાનો ઉપયોગ કરીને સ્થિતિ-આધારિત જાળવણી.
    \end{itemize}

    \begin{mnemonicbox}
        \mnemonic{MPP: Maintain Proactively, Predict problems}
    \end{mnemonicbox}
\end{solutionbox}

\questionmarks{5(બ)}{4}{પબ્લિક એડ્રેસ સિસ્ટમનાં મેઈન્ટેનન્સની ચર્ચા કરો.}
\begin{solutionbox}
    \begin{tabulary}{\linewidth}{|l|L|}
        \hline
        \textbf{ઘટક} & \textbf{જાળવણી કાર્યો} \\
        \hline
        \textbf{માઇક્રોફોન} & વિન્ડસ્ક્રીન સાફ કરો, કેબલ્સ તપાસો, સંવેદનશીલતા ટેસ્ટ કરો \\
        \hline
        \textbf{એમ્પ્લીફાયર} & વેન્ટ્સ સાફ કરો, પાવર તપાસો, ઓવરહિટીંગ તપાસો \\
        \hline
        \textbf{સ્પીકર્સ} & બ્રેકેટ્સ તપાસો, ડિસ્ટોર્શન માટે ટેસ્ટ કરો, વાયરિંગ તપાસો \\
        \hline
        \textbf{કેબલ્સ} & કન્ટીન્યુટી ટેસ્ટ કરો, ક્ષતિગ્રસ્ત કેબલ્સ બદલો \\
        \hline
    \end{tabulary}

    \begin{mnemonicbox}
        \mnemonic{MACS: Microphones, Amplifiers, Connections, Speakers}
    \end{mnemonicbox}
\end{solutionbox}

\questionmarks{5(ક)}{7}{વોશિંગ મશીનનાં કોઈ પણ ત્રણ ફોલ્ટ જણાવો. વોશિંગ મશીનનાં મેઈન્ટેનન્સની સામાન્ય ચર્ચા કરો.}
\begin{solutionbox}
    \textbf{સામાન્ય ખામીઓ:}
    \begin{enumerate}
        \item \textbf{પાણી ભરાતું નથી}: ખામીયુક્ત વાલ્વ, ભરાયેલું ફિલ્ટર.
        \item \textbf{સ્પિન થતું નથી}: બેલ્ટની સમસ્યાઓ, મોટર સમસ્યાઓ.
        \item \textbf{અતિશય કંપન}: અસમાન પગ, સસ્પેન્શન સમસ્યાઓ.
    \end{enumerate}

    \textbf{જાળવણી પ્રક્રિયાઓ:}
    \begin{tabulary}{\linewidth}{|l|L|}
        \hline
        \textbf{ઘટક} & \textbf{કાર્યો} \\
        \hline
        \textbf{ડ્રમ} & દર મહિને સાફ કરો, અવશેષો દૂર કરો, વિદેશી વસ્તુઓ તપાસો \\
        \hline
        \textbf{ફિલ્ટર્સ} & ઉપયોગ પછી લિન્ટ ફિલ્ટર સાફ કરો, પંપ ફિલ્ટર દર મહિને \\
        \hline
        \textbf{હોઝીસ (પાઈપો)} & તિરાડો તપાસો, દર 3-5 વર્ષે બદલો \\
        \hline
        \textbf{ડોર સીલ} & મોલ્ડ અટકાવવા સાફ કરો, તિરાડો માટે તપાસો \\
        \hline
    \end{tabulary}

    \begin{mnemonicbox}
        \mnemonic{WATCH: Water And Tub Cleaning Helps}
    \end{mnemonicbox}
\end{solutionbox}

\questionmarks{5(અ) OR}{3}{પ્રિડિક્ટિવ મેન્ટેનન્સ અને પ્રિવેન્ટિવ મેન્ટેનન્સની સરખામણી કરો.}
\begin{solutionbox}
    \begin{tabulary}{\linewidth}{|l|L|L|}
        \hline
        \textbf{પેરામીટર} & \textbf{પ્રિડિક્ટિવ} & \textbf{પ્રિવેન્ટિવ} \\
        \hline
        \textbf{સમય} & જરૂર મુજબ (સ્થિતિ-આધારિત) & નિશ્ચિત સમયપત્રક \\
        \hline
        \textbf{તકનીક} & વાઇબ્રેશન/થર્મલ એનાલિસિસ & વિઝ્યુઅલ નિરીક્ષણ/સફાઈ \\
        \hline
        \textbf{ખર્ચ} & ઉચ્ચ પ્રારંભિક, ઓછો લાંબા ગાળે & ઓછો પ્રારંભિક, કદાચ વધુ લાંબા ગાળે \\
        \hline
        \textbf{ડાઉનટાઇમ} & ઓછો/આયોજિત & વ્યવસ્થિત અનુસૂચિત \\
        \hline
    \end{tabulary}

    \begin{mnemonicbox}
        \mnemonic{TIMED: Testing Identifies Maintenance Exactly when Due}
    \end{mnemonicbox}
\end{solutionbox}

\questionmarks{5(બ) OR}{4}{એલસીડી ટીવીનાં મેઈન્ટેનન્સ અને ટ્રબલશુટીંગની ચર્ચા કરો.}
\begin{solutionbox}
    \textbf{જાળવણી:}
    \begin{itemize}
        \item \textbf{સ્ક્રીન}: માઇક્રોફાઇબરથી સાફ કરો, પ્રવાહી નહીં.
        \item \textbf{વેન્ટિલેશન}: ધૂળ દૂર કરો, એરફ્લો ખાતરી કરો.
        \item \textbf{કનેક્શન્સ}: કેબલ્સ ચકાસો, કાટ તપાસો.
    \end{itemize}

    \textbf{ટ્રબલશૂટિંગ:}
    \begin{itemize}
        \item \textbf{પાવર નથી}: કોર્ડ, ફ્યુઝ તપાસો.
        \item \textbf{ચિત્ર નથી}: બેકલાઇટ, ટી-કોન બોર્ડ ચકાસો.
        \item \textbf{સ્ક્રીન પર લાઈન્સ}: રિબન કેબલ્સ, સ્ક્રીન નુકસાન.
    \end{itemize}

    \begin{mnemonicbox}
        \mnemonic{PVCS: Pixels, Ventilation, Connections, Software}
    \end{mnemonicbox}
\end{solutionbox}

\questionmarks{5(ક) OR}{7}{તમારી કોમ્પ્યુટર સીસ્ટમમાં લેસર પ્રિન્ટરનાં ઈન્સ્ટોલેશન વિશે સમજાવો. તેનાં મેઈન્ટેનન્સ અને ટ્રબલશુટીંગ પ્રોસીજરની ચર્ચા કરો.}
\begin{solutionbox}
    \textbf{ઇન્સ્ટોલેશન ડાયાગ્રામ:}

    \begin{figure}[H]
        \centering
        \begin{tikzpicture}[gtu flow]
            \node (Unpack) [gtu block] {અનપેકિંગ};
            \node (HW) [gtu block, right=of Unpack] {હાર્ડવેર સેટઅપ};
            \node (Toner) [gtu block, right=of HW] {ટોનર ઇન્સ્ટોલ};
            \node (Power) [gtu block, below=of Toner] {પાવર કનેક્ટ};
            \node (Cable) [gtu block, left=of Power] {ડેટા કનેક્ટ};
            \node (Driver) [gtu block, left=of Cable] {ડ્રાઇવર ઇન્સ્ટોલ};
            \node (Test) [gtu block, below=of Driver] {ટેસ્ટ પ્રિન્ટ};
            
            \draw [gtu arrow] (Unpack) -- (HW);
            \draw [gtu arrow] (HW) -- (Toner);
            \draw [gtu arrow] (Toner) -- (Power);
            \draw [gtu arrow] (Power) -- (Cable);
            \draw [gtu arrow] (Cable) -- (Driver);
            \draw [gtu arrow] (Driver) -- (Test);
        \end{tikzpicture}
        \caption{પ્રિન્ટર ઇન્સ્ટોલેશન}
    \end{figure}

    \textbf{જાળવણી:}
    \begin{itemize}
        \item \textbf{પેપર પાથ}: કોમ્પ્રેસ્ડ એરથી સાફ કરો.
        \item \textbf{રોલર્સ}: આઇસોપ્રોપીલ આલ્કોહોલથી સાફ કરો.
        \item \textbf{ટોનર એરિયા}: કાળજીપૂર્વક વેક્યુમ કરો.
    \end{itemize}

    \textbf{ટ્રબલશૂટિંગ:} પેપર જામ (પાથ ક્લિયર કરો), સ્ટ્રીકિંગ (કોરોના સાફ કરો), લાઈટ પ્રિન્ટ (ટોનર બદલો), કનેક્શન સમસ્યાઓ (ડ્રાઈવર ફરી ઇન્સ્ટોલ કરો).

    \begin{mnemonicbox}
        \mnemonic{SECURE: Setup, Execute drivers, Clean Regularly, Update, Replace consumables, Examine problems}
    \end{mnemonicbox}
\end{solutionbox}

\end{document}
