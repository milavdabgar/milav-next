\documentclass[10pt,a4paper]{article}

% content/resources/templates/preamble.tex
\usepackage[margin=0.6in]{geometry}
\author{Milav Dabgar}
\usepackage{amsmath,amssymb,amsthm}
\usepackage{booktabs}
\usepackage{multirow}
\usepackage{xcolor}
\usepackage{tcolorbox}
\tcbuselibrary{breakable,skins}
\usepackage[colorlinks=true,linkcolor=blue]{hyperref}
\usepackage{titlesec}
\usepackage{enumitem}
\usepackage{tikz}
\usepackage{pgfplots}
\usepackage{circuitikz}
\usepackage[version=4]{mhchem}
\usepackage{longtable}
\usepackage{array}
\usepackage{float}
\usepackage{caption}
\usepackage{listings}

\lstset{
  basicstyle=\small\ttfamily,
  breaklines=true,
  breakatwhitespace=false,
  postbreak=\mbox{\textcolor{red}{$\hookrightarrow$}\space},
  float=false,
  numbers=left,
  numberstyle=\tiny\color{gray},
  numbersep=10pt,
  xleftmargin=2em,
  keywordstyle=\color{blue},
  commentstyle=\color{green!60!black},
  stringstyle=\color{purple},
  backgroundcolor=\color{gray!5},
  showstringspaces=false,
  tabsize=2,
  captionpos=b,
  keepspaces=true,
  columns=flexible
}

\pgfplotsset{compat=1.18}
\usetikzlibrary{shapes,arrows,positioning,calc,patterns,decorations.pathmorphing,decorations.markings,arrows.meta}

% Color scheme
\definecolor{headcolor}{RGB}{0,102,204}
\definecolor{keycolor}{RGB}{220,20,60}
\definecolor{solutioncolor}{RGB}{34,139,34}
\definecolor{mnemoniccolor}{RGB}{148,0,211}
\definecolor{codecolor}{RGB}{0,0,100}

% Spacing
\setlength{\parskip}{3pt}
\setlist[itemize]{nosep}
\setlist[enumerate]{nosep}

% Title formatting
\titleformat{\section}{\Large\bfseries\color{headcolor}}{\thesection}{1em}{}
\titleformat{\subsection}{\large\bfseries\color{headcolor}}{\thesubsection}{1em}{}

% Pandoc tightlist compatibility
\providecommand{\tightlist}{%
  \setlength{\itemsep}{0pt}\setlength{\parskip}{0pt}}

% Pandoc longtable compatibility
\newcounter{none}
\def\thenone{}


% content/resources/templates/gujarati-boxes.tex
\usepackage{fontspec}
\usepackage{polyglossia}

% Set Gujarati as main language (document is primarily in Gujarati)
% Note: gloss-gujarati.ldf doesn't exist in polyglossia, but it will use hyphenation patterns
\setdefaultlanguage{gujarati}
\setotherlanguage{english}

% Configure Gujarati font properly
% Use Language=Default to prevent polyglossia from trying to add language-specific features
% that don't exist for Gujarati, which causes "empty feature" warnings
\newfontfamily\gujaratifont[Script=Gujarati,AutoFakeBold=2.5,AutoFakeSlant=0.3]{Noto Sans Gujarati}
\setmainfont[Script=Gujarati,AutoFakeBold=2.5,AutoFakeSlant=0.3]{Noto Sans Gujarati}
% Use Noto Sans Gujarati for monospace to support Gujarati in text
\setmonofont[Scale=0.9]{Noto Sans Gujarati}

% Configure English to use the same font
\newfontfamily\englishfont[Script=Gujarati,AutoFakeBold=2.5,AutoFakeSlant=0.3]{Noto Sans Gujarati}

% Translations for polyglossia
\gappto\captionsgujarati{
  \renewcommand{\tablename}{કોષ્ટક}
  \renewcommand{\figurename}{આકૃતિ}
}

% Helper for TikZ nodes to ensure Gujarati font
\newcommand{\gu}[1]{{\gujaratifont #1}}

% Custom environments
\newtcolorbox{solutionbox}{
    breakable,
    enhanced,
    colback=solutioncolor!5!white,
    colframe=solutioncolor!75!black,
    fonttitle=\bfseries,
    title=જવાબ
}

\newtcolorbox{solutionboxnobreak}{
 colback=solutioncolor!5!white,
 colframe=solutioncolor!75!black,
 fonttitle=\bfseries,
 title=જવાબ
}

\newtcolorbox{keyformula}{
 breakable,
 enhanced,
 colback=keycolor!5!white,
 colframe=keycolor!75!black,
 fonttitle=\bfseries,
 title=રાસાયણિક સમીકરણ/સૂત્ર
}

\newtcolorbox{mnemonicbox}{
 breakable,
 enhanced,
 colback=mnemoniccolor!5!white,
 colframe=mnemoniccolor!75!black,
 fonttitle=\bfseries,
 title=મેમરી ટ્રીક
}


\begin{document}

\begin{center}
{\Huge\bfseries\color{headcolor} Subject Name (Gujarati)}\\[5pt]
{\LARGE 4311101 -- Winter 2023}\\[3pt]
{\large Semester 1 Study Material}\\[3pt]
{\normalsize\textit{Detailed Solutions and Explanations}}
\end{center}

\vspace{10pt}

\subsection*{પ્રશ્ન 1(a) [3
માર્ક્સ]}\label{q1a}

\textbf{પાવર અને એનર્જી વ્યાખ્યાયિત કરો.}

\begin{solutionbox}

\begin{itemize}
\tightlist
\item
  \textbf{પાવર}: કાર્ય કરવાનો દર અથવા એકમ સમય દીઠ ઊર્જાનો વપરાશ. વોટ્સ
  (W)માં માપવામાં આવે છે.
\item
  \textbf{એનર્જી}: કાર્ય કરવાની ક્ષમતા અથવા કરેલ કાર્ય. જૂલ (J) અથવા વોટ-કલાક
  (Wh)માં માપવામાં આવે છે.
\end{itemize}


{\def\LTcaptype{none} % do not increment counter
\vspace{-5pt}
\captionof{table}{પાવર vs એનર્જી}
\vspace{-10pt}
\begin{longtable}[]{@{}llll@{}}
\toprule\noalign{}
પેરામીટર & વ્યાખ્યા & ફોર્મ્યુલા & એકમ \\
\midrule\noalign{}
\endhead
\bottomrule\noalign{}
\endlastfoot
પાવર & ઊર્જા ટ્રાન્સફરનો દર & P = W/t & વોટ (W) \\
એનર્જી & કાર્ય કરવાની ક્ષમતા & E = P \times t & જૂલ (J) અથવા વોટ-કલાક (Wh) \\
\end{longtable}
}

\end{solutionbox}
\begin{mnemonicbox}
``પાવર પ્રવૃત્તિ કરે, એનર્જી એકત્રિત થાય''

\end{mnemonicbox}
\subsection*{પ્રશ્ન 1(b) [4
માર્ક્સ]}\label{q1b}

\textbf{વિદ્યુત્પ્રવાહ અને વિદ્યુત પોટેંશિયલ વ્યાખ્યાયિત કરો.}

\begin{solutionbox}

\textbf{આકૃતિ:}

\begin{verbatim}
flowchart LR
    A[Electron Flow] {-{-}|Rate of Flow| B[Current]}
    C[Potential Energy] {-{-}|Per Unit Charge| D[Voltage]}
\end{verbatim}

\begin{itemize}
\tightlist
\item
  \textbf{વિદ્યુત્પ્રવાહ}: એકમ સમય દીઠ વહેતો વિદ્યુત ચાર્જ. એમ્પિયર (A)માં માપવામાં
  આવે છે.
\item
  \textbf{વિદ્યુત પોટેંશિયલ}: એક બિંદુથી બીજા બિંદુ પર ચાર્જ ખસેડવા માટે એકમ ચાર્જ
  દીઠ કરવામાં આવતું કાર્ય. વોલ્ટ (V)માં માપવામાં આવે છે.
\end{itemize}

\end{solutionbox}
\begin{mnemonicbox}
``કરંટ ચાર્જનું વહન, પોટેંશિયલ પ્રેરણા''

\end{mnemonicbox}
\subsection*{પ્રશ્ન 1(c) [7
માર્ક્સ]}\label{q1c}

\textbf{ઉદાહરણો સાથે કેસીએલ અને કેવીએલ સમજાવો.}

\begin{solutionbox}

\textbf{આકૃતિ:}

\begin{verbatim}
+{-{-}{-}{-}{-}+        i1}
      |        ↓
      |    R1
      +{-{-}{-}{-}{-}///{-}{-}{-}{-}+}
      |               |
      |               |
     +++              |
     | | V1       R2  |
     +++          /{//}
      |               |
      |               |
      +{-{-}{-}{-}{-}{-}{-}{-}{-}{-}{-}{-}{-}{-}{-}+}
         i2 ↑    ↑ i3
            {    /}
             {  /}
              {/}
              R3
              |
              |
             {-{-}{-}}
              {-}
\end{verbatim}

\textbf{કિરચોફનો કરંટ નિયમ (KCL):}

\begin{itemize}
\tightlist
\item
  નોડમાં પ્રવેશતા કરંટનો સરવાળો તેમાંથી બહાર નીકળતા કરંટના સરવાળા સમાન હોય છે.
\item
  ઉદાહરણ: નોડ X પર, i1 + i2 = i3
\end{itemize}

\textbf{કિરચોફનો વોલ્ટેજ નિયમ (KVL):}

\begin{itemize}
\tightlist
\item
  કોઈપણ બંધ લૂપમાં વોલ્ટેજ ડ્રોપ્સનો સરવાળો શૂન્ય છે.
\item
  ઉદાહરણ: V1 - V(R1) - V(R2) = 0
\end{itemize}

\end{solutionbox}
\begin{mnemonicbox}
``કરંટ આવે-જાય, વોલ્ટેજ લૂપ-સરવાળો શૂન્ય થાય''

\end{mnemonicbox}
\subsection*{પ્રશ્ન 1(c) OR [7
માર્ક્સ]}\label{q1c}

\textbf{રેસિસ્ટર્સ માટે વિવિધ પ્રકારનાં જોડાણો સમજાવો.}

\begin{solutionbox}

\textbf{આકૃતિ:}

\begin{center}
\textbf{Mermaid Diagram (Code)}
\begin{verbatim}
{Shaded}
{Highlighting}[]
graph TD
    subgraph "Series Connection"
    A[R1] {-{-}{-} B[R2] {-}{-}{-} C[R3]}
    end
    subgraph "Parallel Connection"
    D[R1]
    E[R2]
    F[R3]
    G {-{-}{-} D \& E \& F {-}{-}{-} H}
    end
{Highlighting}
{Shaded}
\end{verbatim}
\end{center}


{\def\LTcaptype{none} % do not increment counter
\vspace{-5pt}
\captionof{table}{શ્રેણી vs સમાંતર જોડાણ}
\vspace{-10pt}
\begin{longtable}[]{@{}
  >{\raggedright\arraybackslash}p{(\linewidth - 4\tabcolsep) * \real{0.2157}}
  >{\raggedright\arraybackslash}p{(\linewidth - 4\tabcolsep) * \real{0.3725}}
  >{\raggedright\arraybackslash}p{(\linewidth - 4\tabcolsep) * \real{0.4118}}@{}}
\toprule\noalign{}
\begin{minipage}[b]{\linewidth}\raggedright
પેરામીટર
\end{minipage} & \begin{minipage}[b]{\linewidth}\raggedright
શ્રેણી જોડાણ
\end{minipage} & \begin{minipage}[b]{\linewidth}\raggedright
સમાંતર જોડાણ
\end{minipage} \\
\midrule\noalign{}
\endhead
\bottomrule\noalign{}
\endlastfoot
કુલ અવરોધ & Req = R1 + R2 + R3 + \ldots{} & 1/Req = 1/R1 + 1/R2 + 1/R3 +
\ldots{} \\
કરંટ & બધા અવરોધો માટે સમાન & દરેક માર્ગમાં વહેંચાય છે \\
વોલ્ટેજ & અવરોધો વચ્ચે વહેંચાય છે & બધા અવરોધો માટે સમાન \\
ઉપયોગ & વોલ્ટેજ ડિવાઇડર & કરંટ વહેંચણી \\
\end{longtable}
}

\end{solutionbox}
\begin{mnemonicbox}
``શ્રેણી સરવાળો, સમાંતર ભાગાકાર''

\end{mnemonicbox}
\subsection*{પ્રશ્ન 2(a) [3
માર્ક્સ]}\label{q2a}

\textbf{અવરોધ અને અવરોધકતાને વ્યાખ્યાયિત કરો. તેમના એકમો પણ જણાવો.}

\begin{solutionbox}

\begin{itemize}
\tightlist
\item
  \textbf{અવરોધ}: કરંટ પ્રવાહમાં અડચણ, ઓહ્મ (Ω)માં માપવામાં આવે છે. R = V/I.
\item
  \textbf{અવરોધકતા}: પદાર્થની એક ગુણધર્મ જે એકમ દિમેન્શન દીઠ અવરોધ દર્શાવે છે,
  ઓહ્મ-મીટર (Ω·m)માં માપવામાં આવે છે. ρ = RA/L.
\end{itemize}

\end{solutionbox}
\begin{mnemonicbox}
``અવરોધ અટકાવે, અવરોધકતા અભિલક્ષણ''

\end{mnemonicbox}
\subsection*{પ્રશ્ન 2(b) [4
માર્ક્સ]}\label{q2b}

\textbf{વિદ્યુત કોષને વ્યાખ્યાયિત કરો અને વિવિધ પ્રકારના વિદ્યુત કોષના નામ લખો.}

\begin{solutionbox}

\textbf{આકૃતિ:}

\begin{verbatim}
    +{-{-}{-}{-}{-}{-}{-}{-}+}
    |        |
    | +    {- |}
    |  {  /  |}
    |   {/   |}
    |        |
    +{-{-}{-}{-}{-}{-}{-}{-}+}
      Battery
\end{verbatim}

\begin{itemize}
\tightlist
\item
  \textbf{વિદ્યુત કોષ}: એક ઉપકરણ જે રાસાયણિક ઊર્જાને વિદ્યુત ઊર્જામાં રૂપાંતરિત
  કરીને વોલ્ટેજ ઉત્પન્ન કરે છે.
\end{itemize}

\textbf{વિદ્યુત કોષના પ્રકારો:}

\begin{enumerate}
\tightlist
\item
  \textbf{પ્રાથમિક કોષ}: ડ્રાય સેલ, આલ્કલાઇન સેલ, મર્ક્યુરી સેલ
\item
  \textbf{દ્વિતીય કોષ}: લેડ-એસિડ, નિકલ-કેડમિયમ, લિથિયમ-આયન
\end{enumerate}

\end{solutionbox}
\begin{mnemonicbox}
``પ્રાથમિક એક વાર પ્રવૃત્તિ, દ્વિતીય વારંવાર પુનઃચાર્જ''

\end{mnemonicbox}
\subsection*{પ્રશ્ન 2(c) [7
માર્ક્સ]}\label{q2c}

\textbf{ઉપરોક્ત સર્કિટના કુલ સમકક્ષ અવરોધની ગણતરી કરો જેમા R1=5Ω, R2=3Ω,
R3=4Ω, R4=1Ω, R5=2Ω લો.}

\begin{solutionbox}

\textbf{આકૃતિ:}

\begin{verbatim}
                  R1
                /{//}
       +{-{-}{-}{-}{-}{-}{-}+      +{-}{-}{-}{-}{-}{-}+}
       |                     |
       |                     |
       |                     |
    R2 /{          R3       / R5}
       {/         ///    /}
       |       +{-{-}+      +{-}{-}+}
       |       |            |
       +{-{-}{-}{-}{-}{-}{-}+            |}
                            |
                R4          |
               /{//       |}
       +{-{-}{-}{-}{-}{-}{-}+      +{-}{-}{-}{-}{-}+}
       |                    |
       +{-{-}{-}{-}{-}{-}{-}{-}{-}{-}{-}{-}{-}{-}{-}{-}{-}{-}{-}{-}+}
\end{verbatim}

\textbf{પગલાવાર ઉકેલ:}

\begin{enumerate}
\tightlist
\item
  R2 અને R3 શ્રેણીમાં છે: R23 = R2 + R3 = 3Ω + 4Ω = 7Ω
\item
  R23 અને R4 સમાંતરમાં છે: 1/R234 = 1/7 + 1/1 = (1+7)/7 = 8/7 આથી, R234 =
  7/8 = 0.875Ω
\item
  R1, R234, અને R5 શ્રેણીમાં છે: Req = R1 + R234 + R5 = 5Ω + 0.875Ω + 2Ω =
  7.875Ω
\end{enumerate}

\textbf{આથી, સમકક્ષ અવરોધ = 7.875Ω}

\end{solutionbox}
\begin{mnemonicbox}
``શ્રેણી-સરવાળો, સમાંતર-ગુણાકાર ભાગ્યા સરવાળો''

\end{mnemonicbox}
\subsection*{પ્રશ્ન 2(a) OR [3
માર્ક્સ]}\label{q2a}

\textbf{જો 100 વોટનો બલ્બ 30 દિવસ માટે દરરોજ 10 કલાક ચલાવે તો એનર્જીની કિંમત
શોધો. એનર્જી નો દર રૂપિયા 5/એકમ છે.}

\begin{solutionbox}


{\def\LTcaptype{none} % do not increment counter
\vspace{-5pt}
\captionof{table}{એનર્જી ગણતરી}
\vspace{-10pt}
\begin{longtable}[]{@{}lll@{}}
\toprule\noalign{}
પેરામીટર & મૂલ્ય & ગણતરી \\
\midrule\noalign{}
\endhead
\bottomrule\noalign{}
\endlastfoot
પાવર & 100W = 0.1kW & આપેલ છે \\
ઓપરેટિંગ કલાકો & 10 કલાક/દિવસ \times 30 દિવસ = 300 કલાક & આપેલ છે \\
વપરાયેલ એનર્જી & 0.1kW \times 300h = 30kWh = 30 એકમ &

E = P \times t \\

દર & રૂ. 5/એકમ & આપેલ છે \\
કુલ કિંમત & 30 એકમ \times રૂ. 5/એકમ = રૂ. 150 & કિંમત = એકમો \times દર \\
\end{longtable}
}

\textbf{આથી, એનર્જીની કિંમત = રૂ. 150}

\end{solutionbox}
\begin{mnemonicbox}
``એનર્જી \times દર = વીજળી બિલનો ભાર''

\end{mnemonicbox}
\subsection*{પ્રશ્ન 2(b) OR [4
માર્ક્સ]}\label{q2b}

\textbf{ઓહમનો નિયમ લખો અને કોઈપણ સર્કિટમાં કરંટની ગણતરી કરવા માટે ઓહ્મના નિયમ
નો ઉપયોગ સમજાવો.}

\begin{solutionbox}

\textbf{આકૃતિ:}

\begin{center}
\textbf{Mermaid Diagram (Code)}
\begin{verbatim}
{Shaded}
{Highlighting}[]
graph LR
    A[Voltage] {-{-}{}|"V = IR"| B[Current]}
    C[Resistance] {-{-}{} B}
{Highlighting}
{Shaded}
\end{verbatim}
\end{center}

\textbf{ઓહમનો નિયમ:} વાહકમાંથી વહેતો કરંટ વોલ્ટેજના સીધા પ્રમાણમાં અને અવરોધના
વ્યસ્ત પ્રમાણમાં હોય છે.

\textbf{ફોર્મ્યુલા: V = IR અથવા I = V/R અથવા R = V/I}

\textbf{ઉપયોગ:} સર્કિટમાં કરંટ શોધવા માટે, ઘટક પરના વોલ્ટેજને તેના અવરોધ વડે ભાગો
(I = V/R).

\end{solutionbox}
\begin{mnemonicbox}
``વોલ્ટેજ ઇન્વાઇટ કરે, અવરોધ અટકાવે''

\end{mnemonicbox}
\subsection*{પ્રશ્ન 2(c) OR [7
માર્ક્સ]}\label{q2c}

\textbf{સાબિત કરો કે સંપૂર્ણ કેપેસિટીવ સર્કિટમાં કરંટ વોલ્ટેજ થી 90^\circ આગળ હોઇ છે, અને
સંપૂર્ણ રીતે ઇંડક્ટીવ સર્કિટમાં કરંટ વોલ્ટેજ થી 90^\circ પાછળ હોઇ છે.}

\begin{solutionbox}

\textbf{આકૃતિઓ:}

\begin{center}
\textbf{Mermaid Diagram (Code)}
\begin{verbatim}
{Shaded}
{Highlighting}[]
graph TD
    subgraph "Capacitive Circuit"
    A[Voltage] {-{-}{-} B["Voltage = V sin(ωt)"]}
    C[Current] {-{-}{-} D["Current = I sin(ωt + 90^)"]}
    end
    subgraph "Inductive Circuit"
    E[Voltage] {-{-}{-} F["Voltage = V sin(ωt)"]}
    G[Current] {-{-}{-} H["Current = I sin(ωt {-} 90^)"]}
    end
{Highlighting}
{Shaded}
\end{verbatim}
\end{center}

\textbf{કેપેસિટીવ સર્કિટ માટે:}

\begin{itemize}
\tightlist
\item
  વોલ્ટેજ સમીકરણ: v = V sin(ωt)
\item
કરંટ:

i = C \times dv/dt = ωCV cos(ωt) = I sin(ωt + 90^\circ)

\item
  કરંટ વોલ્ટેજથી 90^\circ આગળ હોય છે
\end{itemize}

\textbf{ઇંડક્ટીવ સર્કિટ માટે:}

\begin{itemize}
\tightlist
\item
વોલ્ટેજ સમીકરણ:

v = L \times di/dt = ωLI cos(ωt) = V sin(ωt + 90^\circ)

\item
  કરંટ: i = I sin(ωt)
\item
  કરંટ વોલ્ટેજથી 90^\circ પાછળ હોય છે
\end{itemize}

\end{solutionbox}
\begin{mnemonicbox}
``ELI the ICE man'' - EL (ઇન્ડક્ટર)માં, I લગ્સ E; ICE
(કેપેસિટર)માં, I લીડ્સ E

\end{mnemonicbox}
\subsection*{પ્રશ્ન 3(a) [3
માર્ક્સ]}\label{q3a}

\textbf{સાયકલ, ફોર્મ ફેક્ટર અને એમ્પ્લિટ્યુડને વ્યાખ્યાયિત કરો.}

\begin{solutionbox}

\textbf{આકૃતિ:}

\begin{verbatim}
    \^{}
    |    /{      /}
    |   /  {    /  }
A{-{-}{-}|{-}{-}/{-}{-}{-}{-}{-}{-}/{-}{-}{-}{-}{-}{-}}
    | /      {/      }
    |/                {}
    +{-{-}{-}{-}{-}{-}{-}{-}{-}{-}{-}{-}{-}{-}{-}{-}{-}{-}{-}{-}}
         |{-{-}{-}{-}{-}{-}|}
          cycle
\end{verbatim}

\begin{itemize}
\tightlist
\item
  \textbf{સાયકલ}: વેવફોર્મનું એક સંપૂર્ણ પુનરાવર્તન.
\item
  \textbf{ફોર્મ ફેક્ટર}: RMS મૂલ્યનો સરેરાશ મૂલ્ય સાથેનો ગુણોત્તર. સાઇન વેવ માટે =
  1.11.
\item
  \textbf{એમ્પ્લિટ્યુડ}: વેવફોર્મનું તેના સરેરાશ સ્થાનથી મહત્તમ વિચલન.
\end{itemize}

\end{solutionbox}
\begin{mnemonicbox}
``સાયકલ સંપૂર્ણ, ફોર્મ ફેક્ટર ફોર્મ્યુલા, એમ્પ્લિટ્યુડ ઉચ્ચતમ''

\end{mnemonicbox}
\subsection*{પ્રશ્ન 3(b) [4
માર્ક્સ]}\label{q3b}

\textbf{આરએમએસ અને સરેરાશ મૂલ્ય વ્યાખ્યાયિત કરો. સાઇન વેવફોર્મનું આરએમએસ અને સરેરાશ
મૂલ્ય નુ સૂત્ર લખો.}

\begin{solutionbox}


{\def\LTcaptype{none} % do not increment counter
\vspace{-5pt}
\captionof{table}{RMS vs સરેરાશ મૂલ્ય}
\vspace{-10pt}
\begin{longtable}[]{@{}
  >{\raggedright\arraybackslash}p{(\linewidth - 4\tabcolsep) * \real{0.2444}}
  >{\raggedright\arraybackslash}p{(\linewidth - 4\tabcolsep) * \real{0.2667}}
  >{\raggedright\arraybackslash}p{(\linewidth - 4\tabcolsep) * \real{0.4889}}@{}}
\toprule\noalign{}
\begin{minipage}[b]{\linewidth}\raggedright
પેરામીટર
\end{minipage} & \begin{minipage}[b]{\linewidth}\raggedright
વ્યાખ્યા
\end{minipage} & \begin{minipage}[b]{\linewidth}\raggedright
સાઇન વેવ માટે ફોર્મ્યુલા
\end{minipage} \\
\midrule\noalign{}
\endhead
\bottomrule\noalign{}
\endlastfoot
RMS મૂલ્ય & વર્ગ કરેલા મૂલ્યોના સરેરાશનો વર્ગમૂળ & Vrms = Vm/\sqrt2 = 0.707 Vm \\
સરેરાશ મૂલ્ય & અર્ધ સાયકલ પર તમામ ક્ષણિક મૂલ્યોની સરેરાશ & Vavg = 2Vm/π = 0.637
Vm \\
\end{longtable}
}

\begin{itemize}
\tightlist
\item
  \textbf{RMS (રૂટ મીન સ્ક્વેર)}: સમાન હીટિંગ અસર ઉત્પન્ન કરતું સમકક્ષ DC મૂલ્ય.
\item
  \textbf{સરેરાશ મૂલ્ય}: અર્ધ સાયકલ પર તમામ ક્ષણિક મૂલ્યોની સરેરાશ.
\end{itemize}

\end{solutionbox}
\begin{mnemonicbox}
``RMS રિલેટ્સ ટુ હીટિંગ, એવરેજ એડ્સ એન્ડ ડિવાઇડ્સ''

\end{mnemonicbox}
\subsection*{પ્રશ્ન 3(c) [7
માર્ક્સ]}\label{q3c}

\textbf{એપરંટ પાવર, ટ્રુ પાવર અને રિયેક્ટીવ પાવર સમજાવો. તેમના માપનના એકમ
જણાવો.}

\begin{solutionbox}

\textbf{આકૃતિ:}

\begin{center}
\textbf{Mermaid Diagram (Code)}
\begin{verbatim}
{Shaded}
{Highlighting}[]
graph TD
    subgraph "Power Triangle"
    A[True Power P] {-{-}{-} B[Apparent Power S]}
    C[Reactive Power Q] {-{-}{-} B}
    end
{Highlighting}
{Shaded}
\end{verbatim}
\end{center}


{\def\LTcaptype{none} % do not increment counter
\vspace{-5pt}
\captionof{table}{પાવરના પ્રકારો}
\vspace{-10pt}
\begin{longtable}[]{@{}
  >{\raggedright\arraybackslash}p{(\linewidth - 6\tabcolsep) * \real{0.3077}}
  >{\raggedright\arraybackslash}p{(\linewidth - 6\tabcolsep) * \real{0.3077}}
  >{\raggedright\arraybackslash}p{(\linewidth - 6\tabcolsep) * \real{0.2308}}
  >{\raggedright\arraybackslash}p{(\linewidth - 6\tabcolsep) * \real{0.1538}}@{}}
\toprule\noalign{}
\begin{minipage}[b]{\linewidth}\raggedright
પાવર પ્રકાર
\end{minipage} & \begin{minipage}[b]{\linewidth}\raggedright
વ્યાખ્યા
\end{minipage} & \begin{minipage}[b]{\linewidth}\raggedright
ફોર્મ્યુલા
\end{minipage} & \begin{minipage}[b]{\linewidth}\raggedright
એકમ
\end{minipage} \\
\midrule\noalign{}
\endhead
\bottomrule\noalign{}
\endlastfoot
એપરંટ પાવર (S) & કુલ પૂરો પાડેલો પાવર & S = VI & VA (વોલ્ટ-એમ્પિયર) \\
ટ્રુ પાવર (P) & ખરેખર વપરાયેલો પાવર & P = VI cos φ & W (વોટ) \\
રિયેક્ટીવ પાવર (Q) & સ્ત્રોત અને લોડ વચ્ચે આવતો-જતો પાવર & Q = VI sin φ & VAR
(વોલ્ટ-એમ્પિયર રિયેક્ટીવ) \\
\end{longtable}
}

\textbf{પાવર ટ્રાયએંગલ:} S^{2} = P^{2} + Q^{2}

\end{solutionbox}
\begin{mnemonicbox}
``એક્ટિવ પરફોર્મ્સ વર્ક, રિયેક્ટીવ રિટર્ન્સ એનર્જી, એપરંટ એડ્સ
વેક્ટર્સ''

\end{mnemonicbox}
\subsection*{પ્રશ્ન 3(a) OR [3
માર્ક્સ]}\label{q3a}

\textbf{3-ફેઝ વોલ્ટેજના ગાણિતિક અભિવ્યક્તિઓ લખો.}

\begin{solutionbox}

\textbf{થ્રી-ફેઝ વોલ્ટેજની અભિવ્યક્તિઓ:}


{\def\LTcaptype{none} % do not increment counter
\vspace{-5pt}
\captionof{table}{3-ફેઝ વોલ્ટેજ}
\vspace{-10pt}
\begin{longtable}[]{@{}ll@{}}
\toprule\noalign{}
ફેઝ & અભિવ્યક્તિ \\
\midrule\noalign{}
\endhead
\bottomrule\noalign{}
\endlastfoot
R-ફેઝ & VR = Vm sin(ωt) \\
Y-ફેઝ & VY = Vm sin(ωt - 120^\circ) \\
B-ફેઝ & VB = Vm sin(ωt - 240^\circ) \\
\end{longtable}
}

જ્યાં Vm મહત્તમ વોલ્ટેજ છે અને ω એન્ગ્યુલર ફ્રિક્વન્સી છે.

\end{solutionbox}
\begin{mnemonicbox}
``લાલ લીડર, પીળો 120^\circ પાછળ, વાદળી 240^\circ પાછળ''

\end{mnemonicbox}
\subsection*{પ્રશ્ન 3(b) OR [4
માર્ક્સ]}\label{q3b}

\textbf{ક્રેસ્ટ ફેક્ટર વ્યાખ્યાયિત કરો અને સાઇન વેવ માટે ક્રેસ્ટ ફેક્ટર ની કિમત લખો.}

\begin{solutionbox}

\textbf{આકૃતિ:}

\begin{verbatim}
    \^{}
    |    /{      /}
    |   /  {    /  }
{-{-}{-}{-}|{-}{-}/{-}{-}{-}{-}{-}{-}/{-}{-}{-}{-}{-}{-}}
    | /      {/      }
    |/                {}
    +{-{-}{-}{-}{-}{-}{-}{-}{-}{-}{-}{-}{-}{-}{-}{-}{-}{-}{-}{-}}
    
    Peak value
    |{-{-}{-}{-}{-}{-}{-}{-}|}
    |   RMS  |
    |   value|
\end{verbatim}

\begin{itemize}
\tightlist
\item
  \textbf{ક્રેસ્ટ ફેક્ટર}: વેવફોર્મના પીક મૂલ્યનો RMS મૂલ્ય સાથેનો ગુણોત્તર.
\item
  \textbf{ફોર્મ્યુલા}: ક્રેસ્ટ ફેક્ટર = પીક મૂલ્ય / RMS મૂલ્ય
\item
  \textbf{સાઇન વેવ માટે}: ક્રેસ્ટ ફેક્ટર = 1/0.707 = 1.414
\end{itemize}

\end{solutionbox}
\begin{mnemonicbox}
``ક્રેસ્ટ કમ્પેર્સ પીક ટુ RMS''

\end{mnemonicbox}
\subsection*{પ્રશ્ન 3(c) OR [7
માર્ક્સ]}\label{q3c}

\textbf{વિવિધ 3-ફેઝ વિદ્યુત જોડાણોનું વર્ણન કરો.}

\begin{solutionbox}

\textbf{આકૃતિ:}

\begin{center}
\textbf{Mermaid Diagram (Code)}
\begin{verbatim}
{Shaded}
{Highlighting}[]
graph TD
    subgraph "Star Connection"
    A1[R] {-{-}{-} D[Neutral]}
    B1[Y] {-{-}{-} D}
    C1[B] {-{-}{-} D}
    end

    subgraph "Delta Connection"
    A2[R] {-{-}{-} B2[Y]}
    B2 {-{-}{-} C2[B]}
    C2 {-{-}{-} A2}
    end
{Highlighting}
{Shaded}
\end{verbatim}
\end{center}


{\def\LTcaptype{none} % do not increment counter
\vspace{-5pt}
\captionof{table}{સ્ટાર vs ડેલ્ટા જોડાણ}
\vspace{-10pt}
\begin{longtable}[]{@{}lll@{}}
\toprule\noalign{}
પેરામીટર & સ્ટાર (Y) જોડાણ & ડેલ્ટા (Δ) જોડાણ \\
\midrule\noalign{}
\endhead
\bottomrule\noalign{}
\endlastfoot
લાઇન વોલ્ટેજ (VL) & \sqrt3 \times ફેઝ વોલ્ટેજ & ફેઝ વોલ્ટેજ જેટલું જ \\
લાઇન કરંટ (IL) & ફેઝ કરંટ જેટલો જ & \sqrt3 \times ફેઝ કરંટ \\
ન્યુટ્રલ વાયર & હાજર & ગેરહાજર \\
ઉપયોગ & અસંતુલિત લોડ્સ, રહેણાંક & સંતુલિત લોડ્સ, ઔદ્યોગિક \\
\end{longtable}
}

\end{solutionbox}
\begin{mnemonicbox}
``સ્ટાર શોઝ ન્યુટ્રલ, ડેલ્ટા ડિલિવર્સ હાયર કરંટ''

\end{mnemonicbox}
\subsection*{પ્રશ્ન 4(a) [3
માર્ક્સ]}\label{q4a}

\textbf{જો આરએમએસ મૂલ્ય 230V હોય તો સાઇનયુસાઇડલ વોલ્ટેજની પીક-ટુ-પીક કિંમતની
ગણતરી કરો.}

\begin{solutionbox}


{\def\LTcaptype{none} % do not increment counter
\vspace{-5pt}
\captionof{table}{ગણતરીના પગલાં}
\vspace{-10pt}
\begin{longtable}[]{@{}lll@{}}
\toprule\noalign{}
પેરામીટર & ફોર્મ્યુલા & ગણતરી \\
\midrule\noalign{}
\endhead
\bottomrule\noalign{}
\endlastfoot
RMS મૂલ્ય & આપેલ છે & 230V \\
પીક મૂલ્ય & Vm = \sqrt2 \times Vrms & Vm = \sqrt2 \times 230 = 325.27V \\
પીક-ટુ-પીક મૂલ્ય & Vp-p = 2 \times Vm & Vp-p = 2 \times 325.27 = 650.54V \\
\end{longtable}
}

\textbf{આથી, પીક-ટુ-પીક મૂલ્ય = 650.54V}

\end{solutionbox}
\begin{mnemonicbox}
``RMS થી પીક - \sqrt2 વડે ગુણો, પીક થી પીક-ટુ-પીક - બમણું
કરો''

\end{mnemonicbox}
\subsection*{પ્રશ્ન 4(b) [4
માર્ક્સ]}\label{q4b}

\textbf{આપેલા એસી પ્રવાહ i = 142.14sin628t માટે ફ્રીક્વંસી અને ટાઇમ પિરિયડ
શોધો.}

\begin{solutionbox}


{\def\LTcaptype{none} % do not increment counter
\vspace{-5pt}
\captionof{table}{ગણતરીના પગલાં}
\vspace{-10pt}
\begin{longtable}[]{@{}lll@{}}
\toprule\noalign{}
પેરામીટર & ફોર્મ્યુલા & ગણતરી \\
\midrule\noalign{}
\endhead
\bottomrule\noalign{}
\endlastfoot
આપેલ સમીકરણ &

i = 142.14 sin(628t) &

ω = 628 rad/s \\

ફ્રીક્વંસી &

f = ω/(2π) &

f = 628/(2π) = 100 Hz \\

ટાઇમ પિરિયડ &

T = 1/f &

T = 1/100 = 0.01

s = 10 ms \\

\end{longtable}
}

\textbf{આથી, ફ્રીક્વંસી = 100 Hz અને ટાઇમ પિરિયડ = 0.01 s}

\end{solutionbox}
\begin{mnemonicbox}
``ફ્રીક્વંસી ફ્રોમ ઓમેગા ડિવાઇડ 2π, ટાઇમ ટેક્સ ઇન્વર્સ''

\end{mnemonicbox}
\subsection*{પ્રશ્ન 4(c) [7
માર્ક્સ]}\label{q4c}

\textbf{ફ્લેમિંગના ડાબા હાથનો નિયમ અને જમણા હાથનો નિયમ સમજાવો.}

\begin{solutionbox}

\textbf{આકૃતિ:}

\begin{verbatim}
Left Hand Rule           Right Hand Rule
    F                        F
    \^{                        \^{}}
    |                        |
    |                        |
    +{-{-}B                    +{-}{-}B}
   /                        /
  /                        /
 I                        I
\end{verbatim}

\textbf{ફ્લેમિંગનો ડાબા હાથનો નિયમ (મોટર):}

\begin{itemize}
\tightlist
\item
  ચુંબકીય ક્ષેત્રમાં વિદ્યુત પ્રવાહ વહનકર્તા પર લાગતા \textbf{બળ}ની દિશા નક્કી કરવા
  માટે વપરાય છે.
\item
  ડાબા હાથને અંગૂઠો, પ્રથમ અને મધ્ય આંગળીઓને કાટખૂણે રાખો.
\item
  અંગૂઠો: ગતિ (બળ)
\item
  પ્રથમ આંગળી: ચુંબકીય ક્ષેત્ર
\item
  મધ્ય આંગળી: વિદ્યુત પ્રવાહ
\end{itemize}

\textbf{ફ્લેમિંગનો જમણા હાથનો નિયમ (જનરેટર):}

\begin{itemize}
\tightlist
\item
  જ્યારે વાહક ચુંબકીય ક્ષેત્રમાં ગતિ કરે છે ત્યારે \textbf{પ્રેરિત વિદ્યુત પ્રવાહ}ની દિશા
  નક્કી કરવા માટે વપરાય છે.
\item
  જમણા હાથને અંગૂઠો, પ્રથમ અને મધ્ય આંગળીઓને કાટખૂણે રાખો.
\item
  અંગૂઠો: વાહકની ગતિ
\item
  પ્રથમ આંગળી: ચુંબકીય ક્ષેત્ર
\item
  મધ્ય આંગળી: પ્રેરિત વિદ્યુત પ્રવાહ
\end{itemize}

\end{solutionbox}
\begin{mnemonicbox}
``ડાબો દર્શાવે મોટર, જમણો જણાવે જનરેટર''

\end{mnemonicbox}
\subsection*{પ્રશ્ન 4(a) OR [3
માર્ક્સ]}\label{q4a}

\textbf{0.6 ટેસ્લાના મેગ્નેટિક ફીલ્ડમાં 30 મીટર/સેકંડ ગતિ સાથે 1 મીટરની લંબાઈ નો
વાહક ક્ષેત્ર સાથે 30^\circ નો કોણ બનાવે છે. તેમાં ઉત્ત્પન્ન થતુ ડાયનેમીક ઇએમએફની ગણતરી
કરો. (sin 30^\circ=0.5 નો ઉપયોગ કરો)}

\begin{solutionbox}


{\def\LTcaptype{none} % do not increment counter
\vspace{-5pt}
\captionof{table}{આપેલ પેરામીટર્સ}
\vspace{-10pt}
\begin{longtable}[]{@{}ll@{}}
\toprule\noalign{}
પેરામીટર & મૂલ્ય \\
\midrule\noalign{}
\endhead
\bottomrule\noalign{}
\endlastfoot
લંબાઈ (l) & 1 મીટર \\
ગતિ (v) & 30 m/s \\
ચુંબકીય ક્ષેત્ર (B) & 0.6 Tesla \\
કોણ (θ) & 30^\circ \\
\end{longtable}
}

\textbf{ફોર્મ્યુલા:} E = Blv sin θ

\textbf{ગણતરી:} E = 0.6 \times 1 \times 30 \times 0.5 = 9 volts

\textbf{આથી, પ્રેરિત EMF = 9 volts}

\end{solutionbox}
\begin{mnemonicbox}
``EMF ઈમર્જિસ ફ્રોમ ફિલ્ડ, વેલોસિટી એન્ડ લેન્થ વિથ એંગલ''

\end{mnemonicbox}
\subsection*{પ્રશ્ન 4(b) OR [4
માર્ક્સ]}\label{q4b}

\textbf{લેન્ઝનો નિયમ લખો અને સમજાવો.}

\begin{solutionbox}

\textbf{આકૃતિ:}

\begin{verbatim}
    +{-{-}{-}{-}{-}{-}{-}{-}+}
    |   N    |  Moving
    |   |    |  Magnet
    |   v    |
    +{-{-}{-}{-}{-}{-}{-}{-}+}
        |
        v
    +{-{-}{-}{-}{-}{-}{-}{-}+}
    |        |  Induced
    |   ↺    |  Current
    |        |
    +{-{-}{-}{-}{-}{-}{-}{-}+}
      Coil
\end{verbatim}

\textbf{લેન્ઝનો નિયમ:} પ્રેરિત EMF અથવા વિદ્યુત પ્રવાહની દિશા હંમેશા એવી હોય છે કે
તે તેને ઉત્પન્ન કરતા કારણનો વિરોધ કરે છે.

\textbf{ઉપયોગ:} જ્યારે ચુંબક કોઈલની નજીક આવે છે, ત્યારે પ્રેરિત વિદ્યુત પ્રવાહ એક
ચુંબકીય ક્ષેત્ર બનાવે છે જે આવતા ચુંબકને પાછો ધક્કો મારે છે.

\end{solutionbox}
\begin{mnemonicbox}
``લેન્ઝ લાઇક્સ ટુ ઓપોઝ''

\end{mnemonicbox}
\subsection*{પ્રશ્ન 4(c) OR [7
માર્ક્સ]}\label{q4c}

\textbf{સ્થિર અને ગતિશીલ રીતે પ્રેરિત ઇએમએફ સમજાવો.}

\begin{solutionbox}


{\def\LTcaptype{none} % do not increment counter
\vspace{-5pt}
\captionof{table}{સ્થિર vs ગતિશીલ પ્રેરિત EMF}
\vspace{-10pt}
\begin{longtable}[]{@{}
  >{\raggedright\arraybackslash}p{(\linewidth - 4\tabcolsep) * \real{0.1803}}
  >{\raggedright\arraybackslash}p{(\linewidth - 4\tabcolsep) * \real{0.3934}}
  >{\raggedright\arraybackslash}p{(\linewidth - 4\tabcolsep) * \real{0.4262}}@{}}
\toprule\noalign{}
\begin{minipage}[b]{\linewidth}\raggedright
પેરામીટર
\end{minipage} & \begin{minipage}[b]{\linewidth}\raggedright
સ્થિર પ્રેરિત EMF
\end{minipage} & \begin{minipage}[b]{\linewidth}\raggedright
ગતિશીલ પ્રેરિત EMF
\end{minipage} \\
\midrule\noalign{}
\endhead
\bottomrule\noalign{}
\endlastfoot
વ્યાખ્યા & કરંટ/ફ્લક્સમાં ફેરફાર થવાથી પ્રેરિત EMF & ચુંબકીય ક્ષેત્રમાં વાહકની ગતિથી
પ્રેરિત EMF \\
ભૌતિક ક્રિયા & સ્થિર વાહક, બદલાતું ક્ષેત્ર & સ્થિર ક્ષેત્રમાં ગતિશીલ વાહક \\
ઉદાહરણ & ટ્રાન્સફોર્મર & જનરેટર \\
ફોર્મ્યુલા &

e = -N dΦ/dt &

e = Blv sin θ \\

\end{longtable}
}

\end{solutionbox}
\begin{mnemonicbox}
``સ્ટેટિક સ્ટેઝ બટ ફ્લક્સ ચેન્જીસ, ડાયનેમિક ડ્રાઇવ્ઝ થ્રુ ફિલ્ડ''

\end{mnemonicbox}
\subsection*{પ્રશ્ન 5(a) [3
માર્ક્સ]}\label{q5a}

\textbf{પીવી સેલ સમજાવો.}

\begin{solutionbox}

\textbf{આકૃતિ:}

\begin{verbatim}
    Sun Rays
       |||
       vvv
    +{-{-}{-}{-}{-}{-}{-}+}
    |  N    |
    |{-{-}{-}{-}{-}{-}{-}| {-} P{-}N Junction}
    |  P    |
    +{-{-}{-}{-}{-}{-}{-}+}
       | |
       | |
       Load
\end{verbatim}

\begin{itemize}
\tightlist
\item
  \textbf{PV સેલ}: ફોટોવોલ્ટિક અસરનો ઉપયોગ કરીને સૂર્યપ્રકાશને સીધા વીજળીમાં
  રૂપાંતરિત કરતું ઉપકરણ.
\item
  \textbf{કાર્યપ્રણાલી}: સૂર્યપ્રકાશ અર્ધવાહક પદાર્થમાં ઇલેક્ટ્રોન્સને ઉત્તેજિત કરે છે,
  જેનાથી વોલ્ટેજ તફાવત ઉત્પન્ન થાય છે.
\item
  \textbf{સામગ્રી}: સામાન્ય રીતે P-N જંક્શન સાથે સિલિકોનમાંથી બનાવવામાં આવે છે.
\end{itemize}

\end{solutionbox}
\begin{mnemonicbox}
``ફોટોન્સ વિઝિટ, કરંટ ક્રિએટેડ''

\end{mnemonicbox}
\subsection*{પ્રશ્ન 5(b) [4
માર્ક્સ]}\label{q5b}

\textbf{પીવી સોલર પેનલ અને એરેસ સમજાવો.}

\begin{solutionbox}

\textbf{આકૃતિ:}

\begin{center}
\textbf{Mermaid Diagram (Code)}
\begin{verbatim}
{Shaded}
{Highlighting}[]
graph LR
    A[Solar Cell] {-{-}{}|"Multiple cells in series"| B[Solar Panel]}
    B {-{-}{}|"Multiple panels connected"| C[Solar Array]}
{Highlighting}
{Shaded}
\end{verbatim}
\end{center}


{\def\LTcaptype{none} % do not increment counter
\vspace{-5pt}
\captionof{table}{સોલર સિસ્ટમ હાયરાર્કી}
\vspace{-10pt}
\begin{longtable}[]{@{}ll@{}}
\toprule\noalign{}
ઘટક & વર્ણન \\
\midrule\noalign{}
\endhead
\bottomrule\noalign{}
\endlastfoot
PV સેલ & સૂર્યપ્રકાશને વીજળીમાં રૂપાંતરિત કરતું મૂળભૂત એકમ (0.5V - 0.6V) \\
PV પેનલ & શ્રેણી/સમાંતરમાં જોડાયેલા અનેક સેલ (સામાન્ય રીતે 12V, 24V) \\
PV એરે & જરૂરી વોલ્ટેજ/કરંટ મેળવવા માટે જોડાયેલા અનેક પેનલ \\
\end{longtable}
}

\end{solutionbox}
\begin{mnemonicbox}
``સેલ્સ કમ્બાઇન ઇન્ટુ પેનલ્સ, પેનલ્સ પ્રોડ્યુસ એરેસ''

\end{mnemonicbox}
\subsection*{પ્રશ્ન 5(c) [7
માર્ક્સ]}\label{q5c}

\textbf{વિન્ડ પાવર સિસ્ટમનો બ્લોક ડાયાગ્રામ દોરો અને સમજાવો.}

\begin{solutionbox}

\textbf{આકૃતિ:}

\begin{verbatim}
flowchart LR
    A[Wind Turbine] {-{-}|"Mechanical energy"| B[Gearbox]}
    B {-{-}|"High speed rotation"| C[Generator]}
    C {-{-}|"AC power"| D[Power Electronics]}
    D {-{-}|"Controlled output"| E[Transformer]}
    E {-{-}|"Grid{-}compatible power"| F[Grid/Load]}
    G[Control System] {-.{-} A \& C \& D}
\end{verbatim}

\textbf{વિન્ડ પાવર સિસ્ટમના ઘટકો:}

\begin{enumerate}
\tightlist
\item
  \textbf{વિન્ડ ટર્બાઇન}: પવનની ઊર્જાને યાંત્રિક ઊર્જામાં રૂપાંતરિત કરે છે
\item
  \textbf{ગિયરબોક્સ}: જનરેટર માટે રોટેશનલ સ્પીડ વધારે છે
\item
  \textbf{જનરેટર}: યાંત્રિક ઊર્જાને વિદ્યુત ઊર્જામાં રૂપાંતરિત કરે છે
\item
  \textbf{પાવર ઇલેક્ટ્રોનિક્સ}: વિદ્યુત આઉટપુટને નિયંત્રિત અને નિયમિત કરે છે
\item
  \textbf{ટ્રાન્સફોર્મર}: ટ્રાન્સમિશન/ડિસ્ટ્રિબ્યુશન માટે વોલ્ટેજ વધારે/ઘટાડે છે
\item
  \textbf{કંટ્રોલ સિસ્ટમ}: સમગ્ર ઓપરેશનનું મોનિટરિંગ અને ઓપ્ટિમાઇઝેશન કરે છે
\end{enumerate}

\end{solutionbox}
\begin{mnemonicbox}
``વિન્ડ ટર્ન્સ ગિયર્સ, જનરેટિંગ ઇલેક્ટ્રિકલ રિટર્ન્સ''

\end{mnemonicbox}
\subsection*{પ્રશ્ન 5(a) OR [3
માર્ક્સ]}\label{q5a}

\textbf{ગ્રીન એનર્જી ના ફાયદા જણાવો.}

\begin{solutionbox}


{\def\LTcaptype{none} % do not increment counter
\vspace{-5pt}
\captionof{table}{ગ્રીન એનર્જીના ફાયદા}
\vspace{-10pt}
\begin{longtable}[]{@{}ll@{}}
\toprule\noalign{}
ફાયદા શ્રેણી & ઉદાહરણો \\
\midrule\noalign{}
\endhead
\bottomrule\noalign{}
\endlastfoot
પર્યાવરણીય & પ્રદૂષણ ઘટાડે છે, કાર્બન ફૂટપ્રિન્ટ ઘટાડે છે \\
આર્થિક & નોકરીઓ સર્જે છે, ઊર્જા પર આધારિતતા ઘટાડે છે \\
આરોગ્ય & હવાની ગુણવત્તા સુધારે છે, આરોગ્ય સમસ્યાઓ ઘટાડે છે \\
ટકાઉપણું & નવીનીકરણીય, અખૂટ સ્ત્રોત \\
\end{longtable}
}

\end{solutionbox}
\begin{mnemonicbox}
``ક્લીન એનર્જી ક્રિએટ્સ ઇકોનોમિક સેલ્વેશન''

\end{mnemonicbox}
\subsection*{પ્રશ્ન 5(b) OR [4
માર્ક્સ]}\label{q5b}

\textbf{સોલર PV ના ઉપયોગો ટુંકમા સમજાવો.}

\begin{solutionbox}

\textbf{આકૃતિ:}

\begin{center}
\textbf{Mermaid Diagram (Code)}
\begin{verbatim}
{Shaded}
{Highlighting}[]
graph TD
    A[Solar PV Applications] {-{-}{} B[Residential]}
    A {-{-}{} C[Commercial]}
    A {-{-}{} D[Industrial]}
    A {-{-}{} E[Utility Scale]}
    A {-{-}{} F[Off{-}grid]}
{Highlighting}
{Shaded}
\end{verbatim}
\end{center}

\textbf{સોલર PV ઉપયોગો:}

\begin{enumerate}
\tightlist
\item
  \textbf{રહેણાંક}: રૂફટોપ સિસ્ટમ, સોલર વોટર હીટર
\item
  \textbf{વ્યાપારી}: બિલ્ડિંગ ઇન્ટીગ્રેટેડ PV, સોલર પાર્કિંગ
\item
  \textbf{ઔદ્યોગિક}: પ્રોસેસ હીટિંગ, પાવર જનરેશન
\item
  \textbf{યુટિલિટી સ્કેલ}: સોલર ફાર્મ, ગ્રીડ સપોર્ટ
\item
  \textbf{ઓફ-ગ્રિડ}: ગ્રામીણ વિદ્યુતીકરણ, રિમોટ એપ્લિકેશન્સ
\end{enumerate}

\end{solutionbox}
\begin{mnemonicbox}
``રેસિડેન્સીસ, કોમર્સ, ઇન્ડસ્ટ્રી યુટિલાઇઝ સોલર''

\end{mnemonicbox}
\subsection*{પ્રશ્ન 5(c) OR [7
માર્ક્સ]}\label{q5c}

\textbf{ગ્રીન એનર્જી ના વિવિધ પ્રકારો સમજાવો.}

\begin{solutionbox}


{\def\LTcaptype{none} % do not increment counter
\vspace{-5pt}
\captionof{table}{ગ્રીન એનર્જીના પ્રકારો}
\vspace{-10pt}
\begin{longtable}[]{@{}lll@{}}
\toprule\noalign{}
પ્રકાર & સ્ત્રોત & ઉપયોગો \\
\midrule\noalign{}
\endhead
\bottomrule\noalign{}
\endlastfoot
સોલર & સૂર્ય & PV સિસ્ટમ, થર્મલ પ્લાન્ટ \\
વિન્ડ & હવાની ગતિ & વિન્ડ ટર્બાઇન, વિન્ડમિલ \\
હાઇડ્રો & વહેતા પાણી & ડેમ, રન-ઓફ-રિવર સિસ્ટમ \\
બાયોમાસ & જૈવિક પદાર્થ & દહન, બાયોગેસ ઉત્પાદન \\
જીયોથર્મલ & પૃથ્વીની ગરમી & ડાયરેક્ટ હીટિંગ, પાવર પ્લાન્ટ \\
ટાઇડલ & સમુદ્રના ભરતી-ઓટ & બેરેજ સિસ્ટમ, ટાઇડલ ટર્બાઇન \\
\end{longtable}
}

\textbf{આકૃતિ:}

\begin{verbatim}
pie title "Green Energy Sources"
    "Solar" : 30
    "Wind" : 25
    "Hydro" : 20
    "Biomass" : 15
    "Geothermal" : 7
    "Tidal" : 3
\end{verbatim}

\end{solutionbox}
\begin{mnemonicbox}
``સૂર્ય, પવન, જળ, બાયોમાસ, જીયોથર્મલ, ટાઇડલ - સરળ માર્ગે
હરિત ભવિષ્ય''

\end{mnemonicbox}

\end{document}
