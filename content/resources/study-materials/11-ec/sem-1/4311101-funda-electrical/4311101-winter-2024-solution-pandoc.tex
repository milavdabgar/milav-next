\documentclass[10pt,a4paper]{article}

% content/resources/templates/preamble.tex
\usepackage[margin=0.6in]{geometry}
\author{Milav Dabgar}
\usepackage{amsmath,amssymb,amsthm}
\usepackage{booktabs}
\usepackage{multirow}
\usepackage{xcolor}
\usepackage{tcolorbox}
\tcbuselibrary{breakable,skins}
\usepackage[colorlinks=true,linkcolor=blue]{hyperref}
\usepackage{titlesec}
\usepackage{enumitem}
\usepackage{tikz}
\usepackage{pgfplots}
\usepackage{circuitikz}
\usepackage[version=4]{mhchem}
\usepackage{longtable}
\usepackage{array}
\usepackage{float}
\usepackage{caption}
\usepackage{listings}

\lstset{
  basicstyle=\small\ttfamily,
  breaklines=true,
  breakatwhitespace=false,
  postbreak=\mbox{\textcolor{red}{$\hookrightarrow$}\space},
  float=false,
  numbers=left,
  numberstyle=\tiny\color{gray},
  numbersep=10pt,
  xleftmargin=2em,
  keywordstyle=\color{blue},
  commentstyle=\color{green!60!black},
  stringstyle=\color{purple},
  backgroundcolor=\color{gray!5},
  showstringspaces=false,
  tabsize=2,
  captionpos=b,
  keepspaces=true,
  columns=flexible
}

\pgfplotsset{compat=1.18}
\usetikzlibrary{shapes,arrows,positioning,calc,patterns,decorations.pathmorphing,decorations.markings,arrows.meta}

% Color scheme
\definecolor{headcolor}{RGB}{0,102,204}
\definecolor{keycolor}{RGB}{220,20,60}
\definecolor{solutioncolor}{RGB}{34,139,34}
\definecolor{mnemoniccolor}{RGB}{148,0,211}
\definecolor{codecolor}{RGB}{0,0,100}

% Spacing
\setlength{\parskip}{3pt}
\setlist[itemize]{nosep}
\setlist[enumerate]{nosep}

% Title formatting
\titleformat{\section}{\Large\bfseries\color{headcolor}}{\thesection}{1em}{}
\titleformat{\subsection}{\large\bfseries\color{headcolor}}{\thesubsection}{1em}{}

% Pandoc tightlist compatibility
\providecommand{\tightlist}{%
  \setlength{\itemsep}{0pt}\setlength{\parskip}{0pt}}

% Pandoc longtable compatibility
\newcounter{none}
\def\thenone{}


% content/resources/templates/english-boxes.tex
% This file is currently empty - it exists to maintain consistency with the import structure.
% Add custom environments here if needed in the future.


\begin{document}

\begin{center}
{\Huge\bfseries\color{headcolor} Subject Name Solutions}\\[5pt]
{\LARGE 4311101 -- Winter 2024}\\[3pt]
{\large Semester 1 Study Material}\\[3pt]
{\normalsize\textit{Detailed Solutions and Explanations}}
\end{center}

\vspace{10pt}

\subsection*{Question 1(a) [3 marks]}\label{q1a}

\textbf{Define current, electric Power and energy.}

\begin{solutionbox}

{\def\LTcaptype{none} % do not increment counter
\begin{longtable}[]{@{}
  >{\raggedright\arraybackslash}p{(\linewidth - 2\tabcolsep) * \real{0.3333}}
  >{\raggedright\arraybackslash}p{(\linewidth - 2\tabcolsep) * \real{0.6667}}@{}}
\toprule\noalign{}
\begin{minipage}[b]{\linewidth}\raggedright
Term
\end{minipage} & \begin{minipage}[b]{\linewidth}\raggedright
Definition
\end{minipage} \\
\midrule\noalign{}
\endhead
\bottomrule\noalign{}
\endlastfoot
\textbf{Current} & The rate of flow of electric charge through a
conductor (measured in amperes, A) \\
\textbf{Electric Power} & The rate at which electrical energy is
transferred or consumed (measured in watts, W) \\
\textbf{Energy} & The capacity to do work, measured as power multiplied
by time (measured in joules or watt-hours) \\
\end{longtable}
}

\end{solutionbox}
\begin{mnemonicbox}
``CPE: Charge-Per-second, Product-of-VI,
Energy-over-time''

\end{mnemonicbox}
\subsection*{Question 1(b) [4 marks]}\label{q1b}

\textbf{Explain the effect of temperature on the value of resistance of
pure metal, alloys and insulators.}

\begin{solutionbox}

{\def\LTcaptype{none} % do not increment counter
\begin{longtable}[]{@{}
  >{\raggedright\arraybackslash}p{(\linewidth - 4\tabcolsep) * \real{0.3409}}
  >{\raggedright\arraybackslash}p{(\linewidth - 4\tabcolsep) * \real{0.4318}}
  >{\raggedright\arraybackslash}p{(\linewidth - 4\tabcolsep) * \real{0.2273}}@{}}
\toprule\noalign{}
\begin{minipage}[b]{\linewidth}\raggedright
Material Type
\end{minipage} & \begin{minipage}[b]{\linewidth}\raggedright
Temperature Effect
\end{minipage} & \begin{minipage}[b]{\linewidth}\raggedright
Equation
\end{minipage} \\
\midrule\noalign{}
\endhead
\bottomrule\noalign{}
\endlastfoot
\textbf{Pure Metals} & Resistance increases with temperature & R_{2} =
R_{1}[1 + α(T_{2}-T_{1})] \\
\textbf{Alloys} & Slight increase with temperature (low α) & R_{2} = R_{1}[1
+ α(T_{2}-T_{1})] \\
\textbf{Insulators} & Resistance decreases with temperature & R_{2} =
R_{1}e\^{}(β(1/T_{2}-1/T_{1})) \\
\end{longtable}
}

where α is temperature coefficient, T is temperature, and R is
resistance

\end{solutionbox}
\begin{mnemonicbox}
``MAI: Metals Add, Alloys Increase-little, Insulators
Invert''

\end{mnemonicbox}
\subsection*{Question 1(c) [7 marks]}\label{q1c}

\textbf{State and explain KCL and KVL with examples.}

\begin{solutionbox}

\textbf{Kirchhoff's Laws:}

{\def\LTcaptype{none} % do not increment counter
\begin{longtable}[]{@{}
  >{\raggedright\arraybackslash}p{(\linewidth - 6\tabcolsep) * \real{0.1190}}
  >{\raggedright\arraybackslash}p{(\linewidth - 6\tabcolsep) * \real{0.2619}}
  >{\raggedright\arraybackslash}p{(\linewidth - 6\tabcolsep) * \real{0.2381}}
  >{\raggedright\arraybackslash}p{(\linewidth - 6\tabcolsep) * \real{0.3810}}@{}}
\toprule\noalign{}
\begin{minipage}[b]{\linewidth}\raggedright
Law
\end{minipage} & \begin{minipage}[b]{\linewidth}\raggedright
Statement
\end{minipage} & \begin{minipage}[b]{\linewidth}\raggedright
Equation
\end{minipage} & \begin{minipage}[b]{\linewidth}\raggedright
Example Circuit
\end{minipage} \\
\midrule\noalign{}
\endhead
\bottomrule\noalign{}
\endlastfoot
\textbf{KCL} & Sum of currents entering a node equals sum of currents
leaving the node & \sumIin = \sumIout &
\texttt{mermaid\ graph\ TD;\ A((Node));\ I1-\/-\textgreater{}A;\ I2-\/-\textgreater{}A;\ A-\/-\textgreater{}I3;\ A-\/-\textgreater{}I4;} \\
\textbf{KVL} & Sum of voltage drops equals sum of voltage rises in a
closed loop & \sumV = 0 &
\texttt{mermaid\ graph\ LR;\ A((+))-\/-\textgreater{}B((-)));\ B-\/-\textgreater{}C((+));\ C-\/-\textgreater{}D((+));\ D-\/-\textgreater{}A;\ linkStyle\ 0\ stroke:red,stroke-width:2px;\ linkStyle\ 1\ stroke:green,stroke-width:2px;\ linkStyle\ 2\ stroke:blue,stroke-width:2px;\ linkStyle\ 3\ stroke:orange,stroke-width:2px;} \\
\end{longtable}
}

\textbf{Example}:

\begin{itemize}
\tightlist
\item
  \textbf{KCL}: At node A, if I_{1} = 5A and I_{2} = 3A entering, then I_{3} + I_{4}
  = 8A must be leaving
\item
  \textbf{KVL}: In a loop with battery 12V and resistors R_{1}(4Ω) and
  R_{2}(8Ω), 12V = I\times(4Ω+8Ω)
\end{itemize}

\end{solutionbox}
\begin{mnemonicbox}
``CLAN: Currents Leave And eNter equally, Voltage
Around Loop is Null''

\end{mnemonicbox}
\subsection*{Question 1(c) OR [7
marks]}\label{q1c}

\textbf{Explain series and parallel connections of resistors with
necessary equations.}

\begin{solutionbox}

{\def\LTcaptype{none} % do not increment counter
\begin{longtable}[]{@{}
  >{\raggedright\arraybackslash}p{(\linewidth - 6\tabcolsep) * \real{0.1846}}
  >{\raggedright\arraybackslash}p{(\linewidth - 6\tabcolsep) * \real{0.2615}}
  >{\raggedright\arraybackslash}p{(\linewidth - 6\tabcolsep) * \real{0.1538}}
  >{\raggedright\arraybackslash}p{(\linewidth - 6\tabcolsep) * \real{0.4000}}@{}}
\toprule\noalign{}
\begin{minipage}[b]{\linewidth}\raggedright
Connection
\end{minipage} & \begin{minipage}[b]{\linewidth}\raggedright
Circuit Diagram
\end{minipage} & \begin{minipage}[b]{\linewidth}\raggedright
Equation
\end{minipage} & \begin{minipage}[b]{\linewidth}\raggedright
Current/Voltage Relation
\end{minipage} \\
\midrule\noalign{}
\endhead
\bottomrule\noalign{}
\endlastfoot
\textbf{Series} &
\texttt{mermaid\ graph\ LR;\ A-\/-\/-B[(R_{1})]-\/-\/-C[(R_{2})]-\/-\/-D[(R_{3})]-\/-\/-E;}
& Req = R_{1} + R_{2} + R_{3} + \ldots{} + Rn & Same current through all
resistors \\
\textbf{Parallel} &
\texttt{mermaid\ graph\ TD;\ A-\/-\/-B;\ A-\/-\/-C[(R_{1})]-\/-\/-B;\ A-\/-\/-D[(R_{2})]-\/-\/-B;\ A-\/-\/-E[(R_{3})]-\/-\/-B;}
& 1/Req = 1/R_{1} + 1/R_{2} + 1/R_{3} + \ldots{} + 1/Rn & Same voltage across all
resistors \\
\end{longtable}
}

\begin{itemize}
\tightlist
\item
  \textbf{Series}: Total resistance increases, current decreases
\item
  \textbf{Parallel}: Total resistance decreases, current increases
\end{itemize}

\end{solutionbox}
\begin{mnemonicbox}
``SPARC: Series Plus All Resistors, parallel Combines
with reciprocals''

\end{mnemonicbox}
\subsection*{Question 2(a) [3 marks]}\label{q2a}

\textbf{Write factors affecting the Resistance value.}

\begin{solutionbox}

{\def\LTcaptype{none} % do not increment counter
\begin{longtable}[]{@{}lll@{}}
\toprule\noalign{}
Factor & Effect on Resistance & Relation \\
\midrule\noalign{}
\endhead
\bottomrule\noalign{}
\endlastfoot
\textbf{Length (l)} & Directly proportional & R ∝ l \\
\textbf{Cross-sectional Area (A)} & Inversely proportional & R ∝ 1/A \\
\textbf{Material (ρ)} & Depends on resistivity & R ∝ ρ \\
\textbf{Temperature (T)} & Usually increases with temperature & R ∝ T \\
\end{longtable}
}

\end{solutionbox}
\begin{mnemonicbox}
``LAMT: Length Adds, Area Minimizes, Material
matters, Temperature transforms''

\end{mnemonicbox}
\subsection*{Question 2(b) [4 marks]}\label{q2b}

\textbf{Draw power triangle and define active and reactive power.}

\begin{solutionbox}

\textbf{Power Triangle:}

\begin{center}
\textbf{Mermaid Diagram (Code)}
\begin{verbatim}
{Shaded}
{Highlighting}[]
graph LR;
    A((O)){-{-}{}B((P));}
    A{-{-}{}C((S));}
    A{-{-}{}D((Q));}
    linkStyle 0 stroke:green,stroke{-width:2px;}
    linkStyle 1 stroke:red,stroke{-width:2px;}
    linkStyle 2 stroke:blue,stroke{-width:2px;}
{Highlighting}
{Shaded}
\end{verbatim}
\end{center}

{\def\LTcaptype{none} % do not increment counter
\begin{longtable}[]{@{}
  >{\raggedright\arraybackslash}p{(\linewidth - 6\tabcolsep) * \real{0.3077}}
  >{\raggedright\arraybackslash}p{(\linewidth - 6\tabcolsep) * \real{0.3077}}
  >{\raggedright\arraybackslash}p{(\linewidth - 6\tabcolsep) * \real{0.1538}}
  >{\raggedright\arraybackslash}p{(\linewidth - 6\tabcolsep) * \real{0.2308}}@{}}
\toprule\noalign{}
\begin{minipage}[b]{\linewidth}\raggedright
Power Type
\end{minipage} & \begin{minipage}[b]{\linewidth}\raggedright
Definition
\end{minipage} & \begin{minipage}[b]{\linewidth}\raggedright
Unit
\end{minipage} & \begin{minipage}[b]{\linewidth}\raggedright
Formula
\end{minipage} \\
\midrule\noalign{}
\endhead
\bottomrule\noalign{}
\endlastfoot
\textbf{Active Power (P)} & Actual power consumed by device & Watt (W) &
P = VI cos φ \\
\textbf{Reactive Power (Q)} & Power oscillating between source and load
& VAR & Q = VI sin φ \\
\textbf{Apparent Power (S)} & Vector sum of active and reactive power &
VA & S = VI \\
\end{longtable}
}

\end{solutionbox}
\begin{mnemonicbox}
``PAWS: Power Active Works, Apparent is
Slant-hypotenuse, reactive Qoscillates''

\end{mnemonicbox}
\subsection*{Question 2(c) [7 marks]}\label{q2c}

\textbf{Explain concept of cell and battery. List out various rating and
types of battery.}

\begin{solutionbox}

\textbf{Cell vs Battery:}

{\def\LTcaptype{none} % do not increment counter
\begin{longtable}[]{@{}
  >{\raggedright\arraybackslash}p{(\linewidth - 2\tabcolsep) * \real{0.3333}}
  >{\raggedright\arraybackslash}p{(\linewidth - 2\tabcolsep) * \real{0.6667}}@{}}
\toprule\noalign{}
\begin{minipage}[b]{\linewidth}\raggedright
Term
\end{minipage} & \begin{minipage}[b]{\linewidth}\raggedright
Definition
\end{minipage} \\
\midrule\noalign{}
\endhead
\bottomrule\noalign{}
\endlastfoot
\textbf{Cell} & Basic electrochemical unit that converts chemical energy
to electrical energy \\
\textbf{Battery} & Collection of one or more cells connected in series
or parallel \\
\end{longtable}
}

\textbf{Battery Ratings:}

{\def\LTcaptype{none} % do not increment counter
\begin{longtable}[]{@{}lll@{}}
\toprule\noalign{}
Rating & Description & Unit \\
\midrule\noalign{}
\endhead
\bottomrule\noalign{}
\endlastfoot
\textbf{Voltage} & Potential difference & Volts (V) \\
\textbf{Capacity} & Amount of charge stored & Ampere-hour (Ah) \\
\textbf{Energy} & Total energy available & Watt-hour (Wh) \\
\textbf{C-Rate} & Discharge/charge rate & C \\
\textbf{Cycle Life} & Number of charge/discharge cycles & - \\
\end{longtable}
}

\textbf{Battery Types:}

\begin{center}
\textbf{Mermaid Diagram (Code)}
\begin{verbatim}
{Shaded}
{Highlighting}[]
graph TD;
    A[Battery Types]{-{-}{}B[Primary];}
    A{-{-}{}C[Secondary];}
    B{-{-}{}D[Alkaline];}
    B{-{-}{}E[Zinc{-}Carbon];}
    B{-{-}{}F[Lithium];}
    C{-{-}{}G[Lead{-}Acid];}
    C{-{-}{}H[Li{-}ion];}
    C{-{-}{}I[Ni{-}MH];}
{Highlighting}
{Shaded}
\end{verbatim}
\end{center}

\end{solutionbox}
\begin{mnemonicbox}
``CAVE: Cells Are Voltage Elements, batteries Bundle
And TallY Energy''

\end{mnemonicbox}
\subsection*{Question 2(a) OR [3
marks]}\label{q2a}

\textbf{Define the terms resistance, conductance and conductivity.}

\begin{solutionbox}

{\def\LTcaptype{none} % do not increment counter
\begin{longtable}[]{@{}
  >{\raggedright\arraybackslash}p{(\linewidth - 6\tabcolsep) * \real{0.1818}}
  >{\raggedright\arraybackslash}p{(\linewidth - 6\tabcolsep) * \real{0.3636}}
  >{\raggedright\arraybackslash}p{(\linewidth - 6\tabcolsep) * \real{0.1818}}
  >{\raggedright\arraybackslash}p{(\linewidth - 6\tabcolsep) * \real{0.2727}}@{}}
\toprule\noalign{}
\begin{minipage}[b]{\linewidth}\raggedright
Term
\end{minipage} & \begin{minipage}[b]{\linewidth}\raggedright
Definition
\end{minipage} & \begin{minipage}[b]{\linewidth}\raggedright
Unit
\end{minipage} & \begin{minipage}[b]{\linewidth}\raggedright
Formula
\end{minipage} \\
\midrule\noalign{}
\endhead
\bottomrule\noalign{}
\endlastfoot
\textbf{Resistance (R)} & Opposition to current flow & Ohm (Ω) & R =
ρl/A \\
\textbf{Conductance (G)} & Ease of current flow & Siemens (S) & G =
1/R \\
\textbf{Conductivity (σ)} & Material property of allowing current flow &
S/m & σ = 1/ρ \\
\end{longtable}
}

where ρ is resistivity, l is length, and A is cross-sectional area

\end{solutionbox}
\begin{mnemonicbox}
``RCG: Resist Current Gladly, Conduct Generously, σ
Gets current through''

\end{mnemonicbox}
\subsection*{Question 2(b) OR [4
marks]}\label{q2b}

\textbf{Prove that for pure inductive circuit, the current lags applied
voltage by 90^\circ.}

\begin{solutionbox}

\textbf{For pure inductive circuit:}

\begin{center}
\textbf{Mermaid Diagram (Code)}
\begin{verbatim}
{Shaded}
{Highlighting}[]
graph LR;
    A((AC Source)){-{-}{}B((L))}
{Highlighting}
{Shaded}
\end{verbatim}
\end{center}

\textbf{Mathematical Proof:}

\begin{itemize}
\tightlist
\item
  Applied voltage: v = Vm sin(ωt)
\item
  For inductor: v = L(di/dt)
\item
  Therefore: L(di/dt) = Vm sin(ωt)
\item
Integrating:

i = -(Vm/ωL)cos(ωt) = (Vm/ωL)sin(ωt-90^\circ)

\end{itemize}

\textbf{Waveform:}

\begin{verbatim}
    v    i
    |    |
    |{  /|}
    | {/ |}
    | /{ |}
    |/  {|}
    |    |
    |    |
    |    |
    +{-{-}{-}{-}+}
       t
\end{verbatim}

\end{solutionbox}
\begin{mnemonicbox}
``ELI: Voltage Leads current In inductor by 90
degrees''

\end{mnemonicbox}
\subsection*{Question 2(c) OR [7
marks]}\label{q2c}

\textbf{Describe Resistor, Inductor and Capacitor with their formula.}

\begin{solutionbox}

{\def\LTcaptype{none} % do not increment counter
\begin{longtable}[]{@{}
  >{\raggedright\arraybackslash}p{(\linewidth - 8\tabcolsep) * \real{0.1964}}
  >{\raggedright\arraybackslash}p{(\linewidth - 8\tabcolsep) * \real{0.1429}}
  >{\raggedright\arraybackslash}p{(\linewidth - 8\tabcolsep) * \real{0.2321}}
  >{\raggedright\arraybackslash}p{(\linewidth - 8\tabcolsep) * \real{0.1607}}
  >{\raggedright\arraybackslash}p{(\linewidth - 8\tabcolsep) * \real{0.2679}}@{}}
\toprule\noalign{}
\begin{minipage}[b]{\linewidth}\raggedright
Component
\end{minipage} & \begin{minipage}[b]{\linewidth}\raggedright
Symbol
\end{minipage} & \begin{minipage}[b]{\linewidth}\raggedright
Description
\end{minipage} & \begin{minipage}[b]{\linewidth}\raggedright
Formula
\end{minipage} & \begin{minipage}[b]{\linewidth}\raggedright
Energy Storage
\end{minipage} \\
\midrule\noalign{}
\endhead
\bottomrule\noalign{}
\endlastfoot
\textbf{Resistor} &
\texttt{mermaid\ graph\ LR;\ A-\/-\/-B[(\_\_\_/\textbackslash{}/\textbackslash{}/\textbackslash{}\_\_\_)]-\/-\/-C}
& Opposes current flow & V = IR & No storage \\
\textbf{Inductor} &
\texttt{mermaid\ graph\ LR;\ A-\/-\/-B[(\_mmmmm\_)]-\/-\/-C} &
Opposes change in current &

V = L(di/dt) &

E = ½LI^{2} \\

\textbf{Capacitor} &
\texttt{mermaid\ graph\ LR;\ A-\/-\/-B[(\_⎥⎥\_)]-\/-\/-C} & Opposes
change in voltage &

I = C(dv/dt) &

E = ½CV^{2} \\

\end{longtable}
}

\textbf{Effect on AC Circuit:}

\begin{itemize}
\tightlist
\item
  \textbf{Resistor}: Current in phase with voltage (cos θ = 1)
\item
  \textbf{Inductor}: Current lags voltage by 90^\circ (cos θ = 0)
\item
  \textbf{Capacitor}: Current leads voltage by 90^\circ (cos θ = 0)
\end{itemize}

\end{solutionbox}
\begin{mnemonicbox}
``RIC: Resistor Impedes Current, Inductor Catches
current-changes, Capacitor Controls voltage-changes''

\end{mnemonicbox}
\subsection*{Question 3(a) [3 marks]}\label{q3a}

\textbf{Define and explain R.M.S value and average value of AC signal.}

\begin{solutionbox}

{\def\LTcaptype{none} % do not increment counter
\begin{longtable}[]{@{}
  >{\raggedright\arraybackslash}p{(\linewidth - 6\tabcolsep) * \real{0.1373}}
  >{\raggedright\arraybackslash}p{(\linewidth - 6\tabcolsep) * \real{0.2353}}
  >{\raggedright\arraybackslash}p{(\linewidth - 6\tabcolsep) * \real{0.4314}}
  >{\raggedright\arraybackslash}p{(\linewidth - 6\tabcolsep) * \real{0.1961}}@{}}
\toprule\noalign{}
\begin{minipage}[b]{\linewidth}\raggedright
Value
\end{minipage} & \begin{minipage}[b]{\linewidth}\raggedright
Definition
\end{minipage} & \begin{minipage}[b]{\linewidth}\raggedright
Formula for Sine Wave
\end{minipage} & \begin{minipage}[b]{\linewidth}\raggedright
Relation
\end{minipage} \\
\midrule\noalign{}
\endhead
\bottomrule\noalign{}
\endlastfoot
\textbf{RMS Value} & Square root of mean of squared values & Vrms =
Vmax/\sqrt2 = 0.707Vmax & Gives equivalent heating effect of DC \\
\textbf{Average Value} & Mean of rectified signal over half cycle & Vavg
= 2Vmax/π = 0.637Vmax & Used for battery charging applications \\
\end{longtable}
}

\end{solutionbox}
\begin{mnemonicbox}
``RAM: Rms-Average Method: Root-mean-square And
Mean-of-absolute''

\end{mnemonicbox}
\subsection*{Question 3(b) [4 marks]}\label{q3b}

\textbf{With necessary diagrams explain how alternating EMF is
generated?}

\begin{solutionbox}

\textbf{Alternating EMF Generation:}

\begin{center}
\textbf{Mermaid Diagram (Code)}
\begin{verbatim}
{Shaded}
{Highlighting}[]
graph LR;
    A[Rotating Coil]{-{-}{}B[Magnetic Field];}
    B{-{-}{}C[EMF Induced];}
    C{-{-}{}D[Direction Changes];}
    D{-{-}{}E[AC Waveform];}
{Highlighting}
{Shaded}
\end{verbatim}
\end{center}

\textbf{Diagram:}

\begin{verbatim}
    N       S
    |       |
    +{-{-}{-}{-}{-}{-}{-}+}
      |   |
      |   |
      |\_\_\_|
       { /}
        |
        v
\end{verbatim}

\begin{itemize}
\tightlist
\item
  Coil rotates in uniform magnetic field
\item
  EMF = NBAlω sin(ωt)
\item
  As coil rotates, cutting flux changes direction
\item
  Generating sinusoidal waveform e = Emax sin(ωt)
\end{itemize}

\end{solutionbox}
\begin{mnemonicbox}
``FARM: Flux And Rotation Make alternating voltage''

\end{mnemonicbox}
\subsection*{Question 3(c) [7 marks]}\label{q3c}

\textbf{Explain A.C analysis of purely resistive AC circuit.}

\begin{solutionbox}

\textbf{Purely Resistive Circuit:}

\begin{center}
\textbf{Mermaid Diagram (Code)}
\begin{verbatim}
{Shaded}
{Highlighting}[]
graph LR;
    A(({)){-}{-}{}B[(R)]{-}{-}{}C}
{Highlighting}
{Shaded}
\end{verbatim}
\end{center}

{\def\LTcaptype{none} % do not increment counter
\begin{longtable}[]{@{}lll@{}}
\toprule\noalign{}
Parameter & Formula & Waveform Relationship \\
\midrule\noalign{}
\endhead
\bottomrule\noalign{}
\endlastfoot
\textbf{Applied Voltage} & v = Vm sin(ωt) & Current and voltage in
phase \\
\textbf{Current} & i = v/R = (Vm/R)sin(ωt) & Follows Ohm's Law \\
\textbf{Power} & p = vi = Vm Im sin^{2}(ωt) & Always positive \\
\textbf{Average Power} & P = Vrms \times Irms = V^{2}/R & Constant value \\
\end{longtable}
}

\textbf{Waveform:}

\begin{verbatim}
    v,i  p
    |    |
    |{  /"}
    | {/ | }
    | /{ | /}
    |/  {|/}
    |    |
    |    |
    |    |
    +{-{-}{-}{-}+}
       t
\end{verbatim}

\end{solutionbox}
\begin{mnemonicbox}
``VIPS: Voltage In-Phase with current, Same waveform,
Power always Positive''

\end{mnemonicbox}
\subsection*{Question 3(a) OR [3
marks]}\label{q3a}

\textbf{Alternating current is given by I = 28.28sin(2Π50t). Find R.M.S
value of current.}

\begin{solutionbox}

\textbf{Given:}

\begin{itemize}
\tightlist
\item
  I = 28.28sin(2Π50t)
\item
  Therefore, Im = 28.28A
\end{itemize}

\textbf{Solution:} \textbar{} Step \textbar{} Calculation \textbar{}
\textbar------\textbar-------------\textbar{} \textbar{} 1. Identify
peak value \textbar{} Im = 28.28A \textbar{} \textbar{} 2. Apply RMS
formula \textbar{} Irms = Im/\sqrt2 \textbar{} \textbar{} 3. Calculate
\textbar{} Irms = 28.28/\sqrt2 = 28.28/1.414 = 20A \textbar{}

\textbf{Therefore, RMS value of current = 20A}

\end{solutionbox}
\begin{mnemonicbox}
``PER: Peak to Effective by Root-2''

\end{mnemonicbox}
\subsection*{Question 3(b) OR [4
marks]}\label{q3b}

\textbf{Find maximum value and R.M.S value of sinusoidal voltage if
Vav=60V.}

\begin{solutionbox}

\textbf{Given:}

\begin{itemize}
\tightlist
\item
  Average value (Vav) = 60V
\end{itemize}

\textbf{Solution:}

{\def\LTcaptype{none} % do not increment counter
\begin{longtable}[]{@{}
  >{\raggedright\arraybackslash}p{(\linewidth - 4\tabcolsep) * \real{0.2143}}
  >{\raggedright\arraybackslash}p{(\linewidth - 4\tabcolsep) * \real{0.3214}}
  >{\raggedright\arraybackslash}p{(\linewidth - 4\tabcolsep) * \real{0.4643}}@{}}
\toprule\noalign{}
\begin{minipage}[b]{\linewidth}\raggedright
Step
\end{minipage} & \begin{minipage}[b]{\linewidth}\raggedright
Formula
\end{minipage} & \begin{minipage}[b]{\linewidth}\raggedright
Calculation
\end{minipage} \\
\midrule\noalign{}
\endhead
\bottomrule\noalign{}
\endlastfoot
1. Relation between Vav and Vm & Vav = 2Vm/π = 0.637Vm & Vm = Vav/0.637
= 60/0.637 \\
2. Calculate maximum value & Vm = Vav \times (π/2) & Vm = 60 \times (π/2) = 60 \times
1.57 = 94.2V \\
3. Calculate RMS value & Vrms = Vm/\sqrt2 = 0.707Vm & Vrms = 0.707 \times 94.2 =
66.6V \\
\end{longtable}
}

\textbf{Therefore, maximum value = 94.2V and RMS value = 66.6V}

\end{solutionbox}
\begin{mnemonicbox}
``AVR: Average to peak Via multiplying by (π/2), Rms
is peak/\sqrt2''

\end{mnemonicbox}
\subsection*{Question 3(c) OR [7
marks]}\label{q3c}

\textbf{Derive equation of line and phase voltage for balanced star
connected load with help of phasor diagram.}

\begin{solutionbox}

\textbf{Star Connection:}

\begin{center}
\textbf{Mermaid Diagram (Code)}
\begin{verbatim}
{Shaded}
{Highlighting}[]
graph LR;
    A((R)){-{-}{}N((N));}
    B((Y)){-{-}{}N;}
    C((B)){-{-}{}N;}
    R[Load]{-{-}{}A;}
    Y[Load]{-{-}{}B;}
    B[Load]{-{-}{}C;}
{Highlighting}
{Shaded}
\end{verbatim}
\end{center}

\textbf{Phasor Diagram:}

\begin{verbatim}
     VRY
      \^{}
     /|{}
    / | {}
   /  |  {}
  /   |   {}
VRB   |    VYB
\end{verbatim}

\textbf{Derivation:}

\begin{itemize}
\tightlist
\item
  Phase voltages: VRN, VYN, VBN (120^\circ apart)
\item
  Line voltages: VRY = VRN - VYN
\item
  For balanced system with magnitude Vp for phase voltage:
\item
  VRY = VRN - VYN = Vp∠0^\circ - Vp∠-120^\circ = Vp(1 - ∠-120^\circ) = \sqrt3Vp∠30^\circ
\end{itemize}

\textbf{Relation:}

\begin{itemize}
\tightlist
\item
  Line voltage (VL) = \sqrt3 \times Phase voltage (Vp)
\item
  Line voltage leads phase voltage by 30^\circ
\end{itemize}

\end{solutionbox}
\begin{mnemonicbox}
``PALS: Phase to Line in Star: multiply by
Square-root-3''

\end{mnemonicbox}
\subsection*{Question 4(a) [3 marks]}\label{q4a}

\textbf{Write statement of Faraday's law and Lenz's law with
expression.}

\begin{solutionbox}

{\def\LTcaptype{none} % do not increment counter
\begin{longtable}[]{@{}
  >{\raggedright\arraybackslash}p{(\linewidth - 4\tabcolsep) * \real{0.1786}}
  >{\raggedright\arraybackslash}p{(\linewidth - 4\tabcolsep) * \real{0.3929}}
  >{\raggedright\arraybackslash}p{(\linewidth - 4\tabcolsep) * \real{0.4286}}@{}}
\toprule\noalign{}
\begin{minipage}[b]{\linewidth}\raggedright
Law
\end{minipage} & \begin{minipage}[b]{\linewidth}\raggedright
Statement
\end{minipage} & \begin{minipage}[b]{\linewidth}\raggedright
Expression
\end{minipage} \\
\midrule\noalign{}
\endhead
\bottomrule\noalign{}
\endlastfoot
\textbf{Faraday's Law} & EMF induced is directly proportional to rate of
change of magnetic flux & e = -N(dΦ/dt) \\
\textbf{Lenz's Law} & Induced EMF opposes the cause producing it
(negative sign in formula) & Direction opposes flux change \\
\end{longtable}
}

\end{solutionbox}
\begin{mnemonicbox}
``FORC: Faraday's flux Over Rate Change, Lenz Opposes
the Reason for Change''

\end{mnemonicbox}
\subsection*{Question 4(b) [4 marks]}\label{q4b}

\textbf{State any four advantage of 3-phase supply over single-phase
supply.}

\begin{solutionbox}

{\def\LTcaptype{none} % do not increment counter
\begin{longtable}[]{@{}
  >{\raggedright\arraybackslash}p{(\linewidth - 2\tabcolsep) * \real{0.7547}}
  >{\raggedright\arraybackslash}p{(\linewidth - 2\tabcolsep) * \real{0.2453}}@{}}
\toprule\noalign{}
\begin{minipage}[b]{\linewidth}\raggedright
Advantages of 3-Phase Over Single-Phase
\end{minipage} & \begin{minipage}[b]{\linewidth}\raggedright
Explanation
\end{minipage} \\
\midrule\noalign{}
\endhead
\bottomrule\noalign{}
\endlastfoot
\textbf{Higher Power Density} & 3-phase delivers 1.732 times more power
with same wire size \\
\textbf{Constant Power Delivery} & No pulsation in power as in
single-phase \\
\textbf{Smaller Conductors} & Less copper required for same power
transfer \\
\textbf{Self-Starting Motors} & No starting mechanism needed for
motors \\
\end{longtable}
}

\textbf{Additional: More efficient transmission, reduced harmonics,
balanced loading}

\end{solutionbox}
\begin{mnemonicbox}
``PCCS: Power higher, Constant delivery, Copper less,
Self-starting motors''

\end{mnemonicbox}
\subsection*{Question 4(c) [7 marks]}\label{q4c}

\textbf{Explain Fleming's right-hand rule for generators and left-hand
rule for motors.}

\begin{solutionbox}

\textbf{Fleming's Hand Rules:}

{\def\LTcaptype{none} % do not increment counter
\begin{longtable}[]{@{}
  >{\raggedright\arraybackslash}p{(\linewidth - 6\tabcolsep) * \real{0.1429}}
  >{\raggedright\arraybackslash}p{(\linewidth - 6\tabcolsep) * \real{0.3095}}
  >{\raggedright\arraybackslash}p{(\linewidth - 6\tabcolsep) * \real{0.3333}}
  >{\raggedright\arraybackslash}p{(\linewidth - 6\tabcolsep) * \real{0.2143}}@{}}
\toprule\noalign{}
\begin{minipage}[b]{\linewidth}\raggedright
Rule
\end{minipage} & \begin{minipage}[b]{\linewidth}\raggedright
Application
\end{minipage} & \begin{minipage}[b]{\linewidth}\raggedright
Hand Position
\end{minipage} & \begin{minipage}[b]{\linewidth}\raggedright
Diagram
\end{minipage} \\
\midrule\noalign{}
\endhead
\bottomrule\noalign{}
\endlastfoot
\textbf{Right-Hand Rule (Generator)} & Determines direction of induced
EMF & \textbf{Thumb}: Motion\textbf{Forefinger}: Field\textbf{Middle
finger}: Current/EMF & ```goat \\
\end{longtable}
}

\begin{verbatim}
F ^
  |
--+-- > M
  |
  v
  C
``` |
\end{verbatim}

\textbf{Left-Hand Rule (Motor)} \textbar{} Determines direction of
motion/force \textbar{} \textbf{Thumb}: Motion/Force\textbf{Forefinger}:
Field\textbf{Middle finger}: Current \textbar{}
\texttt{goat\ \ \ \ F\ \^{}\ \ \ \ \ \ \textbar{}\ \ \ \ -\/-+-\/-\ \textgreater{}\ M\ \ \ \ \ \ \textbar{}\ \ \ \ \ \ v\ \ \ \ \ \ C}
\textbar{}

\begin{itemize}
\tightlist
\item
  \textbf{Generator}: Mechanical energy converted to electrical energy
\item
  \textbf{Motor}: Electrical energy converted to mechanical energy
\end{itemize}

\end{solutionbox}
\begin{mnemonicbox}
``FBI-MFC: Field-B-Induced current for right hand,
Motion-Field-Current for left''

\end{mnemonicbox}
\subsection*{Question 4(a) OR [3
marks]}\label{q4a}

\textbf{Describe phenomenon of electromagnetic induction.}

\begin{solutionbox}

\textbf{Electromagnetic Induction:}

\begin{center}
\textbf{Mermaid Diagram (Code)}
\begin{verbatim}
{Shaded}
{Highlighting}[]
graph LR;
    A[Changing Magnetic Field/Flux]{-{-}{}B[Induces EMF in Conductor];}
    B{-{-}{}C[Causes Current to Flow];}
    C{-{-}{}D[Creates Secondary Magnetic Field];}
{Highlighting}
{Shaded}
\end{verbatim}
\end{center}

\textbf{Key Factors:}

\begin{itemize}
\tightlist
\item
  Requires relative motion or changing flux
\item
  EMF proportional to rate of change of flux
\item
  Direction determined by Lenz's law
\end{itemize}

\end{solutionbox}
\begin{mnemonicbox}
``MICE: Motion Induces Current via Electromagnetic
induction''

\end{mnemonicbox}
\subsection*{Question 4(b) OR [4
marks]}\label{q4b}

\textbf{Explain the generation of 3-phase alternating EMF.}

\begin{solutionbox}

\textbf{3-Phase EMF Generation:}

\begin{center}
\textbf{Mermaid Diagram (Code)}
\begin{verbatim}
{Shaded}
{Highlighting}[]
graph LR;
    A[3 Coils at 120^ Apart]{-{-}{}B[Rotating Magnetic Field];}
    B{-{-}{}C[3 EMFs Generated at 120^ Phase Difference];}
    C{-{-}{}D[Balanced 3{-}Phase Supply];}
{Highlighting}
{Shaded}
\end{verbatim}
\end{center}

\textbf{Three Phase Waveform:}

\begin{verbatim}
    +       R
    |   /{    /}
    |  /  {  /  }
    | /    {/    }
{-{-}{-}{-}+{-}{-}{-}{-}{-}{-}{-}{-}{-}{-}{-}{-}{-}{-}{-}}
    |{    /    /}
    | {  /    /}
    |  {/    /  }
    +       Y
    |   /{    /}
    |  /  {  /  }
    | /    {/    }
{-{-}{-}{-}+{-}{-}{-}{-}{-}{-}{-}{-}{-}{-}{-}{-}{-}{-}{-}}
    |{    /    /}
    | {  /    /}
    |  {/    /}
    +       B
\end{verbatim}

\begin{itemize}
\tightlist
\item
  Three identical coils displaced 120^\circ spatially
\item
  Produces three identical EMFs displaced 120^\circ in time
\item
  EMFs: eR = Emax sin(ωt), eY = Emax sin(ωt-120^\circ), eB = Emax
  sin(ωt-240^\circ)
\end{itemize}

\end{solutionbox}
\begin{mnemonicbox}
``CPS: Coils Produce Shifted waveforms at 120
degrees''

\end{mnemonicbox}
\subsection*{Question 4(c) OR [7
marks]}\label{q4c}

\textbf{Differentiate statically and dynamically induced E.M.F.}

\begin{solutionbox}

{\def\LTcaptype{none} % do not increment counter
\begin{longtable}[]{@{}
  >{\raggedright\arraybackslash}p{(\linewidth - 4\tabcolsep) * \real{0.1833}}
  >{\raggedright\arraybackslash}p{(\linewidth - 4\tabcolsep) * \real{0.4000}}
  >{\raggedright\arraybackslash}p{(\linewidth - 4\tabcolsep) * \real{0.4167}}@{}}
\toprule\noalign{}
\begin{minipage}[b]{\linewidth}\raggedright
Parameter
\end{minipage} & \begin{minipage}[b]{\linewidth}\raggedright
Statically Induced EMF
\end{minipage} & \begin{minipage}[b]{\linewidth}\raggedright
Dynamically Induced EMF
\end{minipage} \\
\midrule\noalign{}
\endhead
\bottomrule\noalign{}
\endlastfoot
\textbf{Definition} & EMF induced due to change in flux linking with
stationary conductor & EMF induced due to conductor moving in a magnetic
field \\
\textbf{Movement} & No relative motion between conductor and field &
Relative motion exists \\
\textbf{Change Source} & Changing current in primary circuit & Physical
movement of conductor \\
\textbf{Examples} & Transformer, inductor & Generator, alternator \\
\textbf{Mathematical Expression} & e = -N(dΦ/dt) due to changing current
&

e = Blv (B=flux density,

l=length,

v=velocity) \\

\end{longtable}
}

\end{solutionbox}
\begin{mnemonicbox}
``SMCE: Static-Moving, Change-External: static has
changing flux, moving has constant flux''

\end{mnemonicbox}
\subsection*{Question 5(a) [3 marks]}\label{q5a}

\textbf{Differentiate HAWT and VAWT.}

\begin{solutionbox}

{\def\LTcaptype{none} % do not increment counter
\begin{longtable}[]{@{}
  >{\raggedright\arraybackslash}p{(\linewidth - 4\tabcolsep) * \real{0.1325}}
  >{\raggedright\arraybackslash}p{(\linewidth - 4\tabcolsep) * \real{0.4458}}
  >{\raggedright\arraybackslash}p{(\linewidth - 4\tabcolsep) * \real{0.4217}}@{}}
\toprule\noalign{}
\begin{minipage}[b]{\linewidth}\raggedright
Parameter
\end{minipage} & \begin{minipage}[b]{\linewidth}\raggedright
HAWT (Horizontal Axis Wind Turbine)
\end{minipage} & \begin{minipage}[b]{\linewidth}\raggedright
VAWT (Vertical Axis Wind Turbine)
\end{minipage} \\
\midrule\noalign{}
\endhead
\bottomrule\noalign{}
\endlastfoot
\textbf{Orientation} & Blades rotate on horizontal axis & Blades rotate
on vertical axis \\
\textbf{Wind Direction} & Needs to face wind direction & Works with wind
from any direction \\
\textbf{Installation} & Tall tower, high off ground & Lower to ground,
easier access \\
\end{longtable}
}

\textbf{Diagram:}

\begin{verbatim}
   HAWT       VAWT
    /|{        \_|\_}
   / | {      | | |}
  /\_\_|\_\_{     |\_|\_|}
     |          |
    \_|\_        \_|\_
\end{verbatim}

\end{solutionbox}
\begin{mnemonicbox}
``HV-DIT: Horizontal-Vertical,
Directional-Independent, Tall-lower''

\end{mnemonicbox}
\subsection*{Question 5(b) [4 marks]}\label{q5b}

\textbf{Classification of green energy.}

\begin{solutionbox}

\textbf{Green Energy Classifications:}

\begin{center}
\textbf{Mermaid Diagram (Code)}
\begin{verbatim}
{Shaded}
{Highlighting}[]
graph TD;
    A[Green Energy Sources]{-{-}{}B[Solar Energy];}
    A{-{-}{}C[Wind Energy];}
    A{-{-}{}D[Hydro Energy];}
    A{-{-}{}E[Geothermal];}
    A{-{-}{}F[Biomass Energy];}
    A{-{-}{}G[Tidal/Wave Energy];}
{Highlighting}
{Shaded}
\end{verbatim}
\end{center}

{\def\LTcaptype{none} % do not increment counter
\begin{longtable}[]{@{}lll@{}}
\toprule\noalign{}
Source & Primary Principle & Application \\
\midrule\noalign{}
\endhead
\bottomrule\noalign{}
\endlastfoot
\textbf{Solar} & Photovoltaic effect & Solar panels, thermal
collectors \\
\textbf{Wind} & Kinetic energy of air & Wind turbines \\
\textbf{Hydro} & Potential energy of water & Dams, run-of-river \\
\textbf{Geothermal} & Earth's internal heat & Heat pumps, power
plants \\
\end{longtable}
}

\end{solutionbox}
\begin{mnemonicbox}
``SWHGBT: Sun Wind Hydro Geo Bio Tidal - Sources With
Huge Green Benefits Today''

\end{mnemonicbox}
\subsection*{Question 5(c) [7 marks]}\label{q5c}

\textbf{Explain wind power system.}

\begin{solutionbox}

\textbf{Wind Power System:}

\begin{center}
\textbf{Mermaid Diagram (Code)}
\begin{verbatim}
{Shaded}
{Highlighting}[]
graph LR;
    A[Wind]{-{-}{}B[Turbine];}
    B{-{-}{}C[Gearbox];}
    C{-{-}{}D[Generator];}
    D{-{-}{}E[Transformer];}
    E{-{-}{}F[Grid Connection];}
    D{-{-}{}G[Controller];}
{Highlighting}
{Shaded}
\end{verbatim}
\end{center}

\textbf{Components:}

\begin{itemize}
\tightlist
\item
  \textbf{Wind Turbine}: Converts wind energy to mechanical rotation
\item
  \textbf{Gearbox}: Increases rotation speed for generator
\item
  \textbf{Generator}: Converts mechanical to electrical energy
\item
  \textbf{Controller}: Regulates output and safety functions
\item
  \textbf{Transformer}: Steps up voltage for transmission
\item
  \textbf{Tower}: Elevates turbine to capture stronger winds
\end{itemize}

\textbf{Working Principle:}

\begin{enumerate}
\tightlist
\item
  Wind turns blades (kinetic to mechanical)
\item
  Gearbox increases RPM
\item
  Generator produces AC power
\item
  Controller regulates output
\item
  Transformer prepares for grid connection
\end{enumerate}

\end{solutionbox}
\begin{mnemonicbox}
``WINGER: Wind In, Gearbox Enhances Rotation,
Generator outputs''

\end{mnemonicbox}
\subsection*{Question 5(a) OR [3
marks]}\label{q5a}

\textbf{List any three needs of green energy.}

\begin{solutionbox}

{\def\LTcaptype{none} % do not increment counter
\begin{longtable}[]{@{}
  >{\raggedright\arraybackslash}p{(\linewidth - 2\tabcolsep) * \real{0.6176}}
  >{\raggedright\arraybackslash}p{(\linewidth - 2\tabcolsep) * \real{0.3824}}@{}}
\toprule\noalign{}
\begin{minipage}[b]{\linewidth}\raggedright
Need for Green Energy
\end{minipage} & \begin{minipage}[b]{\linewidth}\raggedright
Explanation
\end{minipage} \\
\midrule\noalign{}
\endhead
\bottomrule\noalign{}
\endlastfoot
\textbf{Environmental Protection} & Reduces pollution and greenhouse gas
emissions \\
\textbf{Resource Conservation} & Preserves finite fossil fuel
resources \\
\textbf{Energy Security} & Reduces dependence on imported fuels and
price volatility \\
\end{longtable}
}

\textbf{Other Needs}: Climate change mitigation, sustainable
development, economic benefits

\end{solutionbox}
\begin{mnemonicbox}
``ECO: Environment protected, Conservation of
resources, Oil-independence''

\end{mnemonicbox}
\subsection*{Question 5(b) OR [4
marks]}\label{q5b}

\textbf{Write short note on PV cell.}

\begin{solutionbox}

\textbf{Photovoltaic (PV) Cell:}

\begin{center}
\textbf{Mermaid Diagram (Code)}
\begin{verbatim}
{Shaded}
{Highlighting}[]
graph TD;
    A[Sunlight]{-{-}{}B[PV Cell];}
    B{-{-}{}C[DC Electricity];}
    B{-{-}{}D[Construction: P{-}N Junction];}
    B{-{-}{}E[Materials: Silicon, Thin Film];}
{Highlighting}
{Shaded}
\end{verbatim}
\end{center}

\textbf{Working Principle:}

\begin{itemize}
\tightlist
\item
  Based on photovoltaic effect
\item
  Converts sunlight directly to electricity
\item
  Uses semiconductor material (usually silicon)
\item
  Creates electron flow when photons hit P-N junction
\end{itemize}

\textbf{Types}: Monocrystalline, Polycrystalline, Thin-film

\textbf{Efficiency}: Typically 15-22\% for commercial cells

\end{solutionbox}
\begin{mnemonicbox}
``SPEC: Sunlight Produces Electricity through Cells
with p-n junctions''

\end{mnemonicbox}
\subsection*{Question 5(c) OR [7
marks]}\label{q5c}

\textbf{Explain solar system.}

\begin{solutionbox}

\textbf{Solar Power System:}

\begin{center}
\textbf{Mermaid Diagram (Code)}
\begin{verbatim}
{Shaded}
{Highlighting}[]
graph LR;
    A[Solar Panels]{-{-}{}B[Charge Controller];}
    B{-{-}{}C[Battery Bank];}
    C{-{-}{}D[Inverter];}
    D{-{-}{}E[AC Loads];}
    A{-{-}{}F[On{-}Grid System]{-}{-}{}G[Grid Tie Inverter]{-}{-}{}H[Electric Grid];}
{Highlighting}
{Shaded}
\end{verbatim}
\end{center}

\textbf{Components:}

\begin{itemize}
\tightlist
\item
  \textbf{Solar Panels}: Convert sunlight to DC electricity
\item
  \textbf{Charge Controller}: Regulates battery charging
\item
  \textbf{Battery Bank}: Stores electrical energy (off-grid)
\item
  \textbf{Inverter}: Converts DC to AC for household use
\item
  \textbf{Distribution Panel}: Connects to home electrical system
\end{itemize}

\textbf{Types:}

\begin{itemize}
\tightlist
\item
  \textbf{Grid-Connected}: Feeds excess power to grid
\item
  \textbf{Off-Grid}: Independent with battery storage
\item
  \textbf{Hybrid}: Combination of both systems
\end{itemize}

\textbf{Applications:} Home power, water pumping, street lighting,
industrial use

\end{solutionbox}
\begin{mnemonicbox}
``SCBID: Solar Cells produce, Battery stores,
Inverter converts, Distribution supplies''

\end{mnemonicbox}

\end{document}
