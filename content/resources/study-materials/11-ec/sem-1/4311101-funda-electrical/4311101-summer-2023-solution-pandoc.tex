\documentclass[10pt,a4paper]{article}

% content/resources/templates/preamble.tex
\usepackage[margin=0.6in]{geometry}
\author{Milav Dabgar}
\usepackage{amsmath,amssymb,amsthm}
\usepackage{booktabs}
\usepackage{multirow}
\usepackage{xcolor}
\usepackage{tcolorbox}
\tcbuselibrary{breakable,skins}
\usepackage[colorlinks=true,linkcolor=blue]{hyperref}
\usepackage{titlesec}
\usepackage{enumitem}
\usepackage{tikz}
\usepackage{pgfplots}
\usepackage{circuitikz}
\usepackage[version=4]{mhchem}
\usepackage{longtable}
\usepackage{array}
\usepackage{float}
\usepackage{caption}
\usepackage{listings}

\lstset{
  basicstyle=\small\ttfamily,
  breaklines=true,
  breakatwhitespace=false,
  postbreak=\mbox{\textcolor{red}{$\hookrightarrow$}\space},
  float=false,
  numbers=left,
  numberstyle=\tiny\color{gray},
  numbersep=10pt,
  xleftmargin=2em,
  keywordstyle=\color{blue},
  commentstyle=\color{green!60!black},
  stringstyle=\color{purple},
  backgroundcolor=\color{gray!5},
  showstringspaces=false,
  tabsize=2,
  captionpos=b,
  keepspaces=true,
  columns=flexible
}

\pgfplotsset{compat=1.18}
\usetikzlibrary{shapes,arrows,positioning,calc,patterns,decorations.pathmorphing,decorations.markings,arrows.meta}

% Color scheme
\definecolor{headcolor}{RGB}{0,102,204}
\definecolor{keycolor}{RGB}{220,20,60}
\definecolor{solutioncolor}{RGB}{34,139,34}
\definecolor{mnemoniccolor}{RGB}{148,0,211}
\definecolor{codecolor}{RGB}{0,0,100}

% Spacing
\setlength{\parskip}{3pt}
\setlist[itemize]{nosep}
\setlist[enumerate]{nosep}

% Title formatting
\titleformat{\section}{\Large\bfseries\color{headcolor}}{\thesection}{1em}{}
\titleformat{\subsection}{\large\bfseries\color{headcolor}}{\thesubsection}{1em}{}

% Pandoc tightlist compatibility
\providecommand{\tightlist}{%
  \setlength{\itemsep}{0pt}\setlength{\parskip}{0pt}}

% Pandoc longtable compatibility
\newcounter{none}
\def\thenone{}


% content/resources/templates/english-boxes.tex
% This file is currently empty - it exists to maintain consistency with the import structure.
% Add custom environments here if needed in the future.


\begin{document}

\begin{center}
{\Huge\bfseries\color{headcolor} Subject Name Solutions}\\[5pt]
{\LARGE 4311101 -- Summer 2023}\\[3pt]
{\large Semester 1 Study Material}\\[3pt]
{\normalsize\textit{Detailed Solutions and Explanations}}
\end{center}

\vspace{10pt}

\subsection*{Question 1(a) [3 marks]}\label{q1a}

\textbf{Define the following term (1) Resistance (2) Electrical energy
(3) Electrical Power}

\begin{solutionbox}

{\def\LTcaptype{none} % do not increment counter
\begin{longtable}[]{@{}
  >{\raggedright\arraybackslash}p{(\linewidth - 2\tabcolsep) * \real{0.3333}}
  >{\raggedright\arraybackslash}p{(\linewidth - 2\tabcolsep) * \real{0.6667}}@{}}
\toprule\noalign{}
\begin{minipage}[b]{\linewidth}\raggedright
Term
\end{minipage} & \begin{minipage}[b]{\linewidth}\raggedright
Definition
\end{minipage} \\
\midrule\noalign{}
\endhead
\bottomrule\noalign{}
\endlastfoot
\textbf{Resistance} & The property of a material that opposes the flow
of electric current, measured in ohms (Ω) \\
\textbf{Electrical Energy} & The ability to do work by electrical means,
measured in joules (J) or kilowatt-hours (kWh) \\
\textbf{Electrical Power} & The rate at which electrical energy is
transferred or converted, measured in watts (W) \\
\end{longtable}
}

\end{solutionbox}
\begin{mnemonicbox}
``RIP'' - Resistance Impedes Path, Energy Is
Potential, Power Is Performance

\end{mnemonicbox}
\subsection*{Question 1(b) [4 marks]}\label{q1b}

\textbf{State and Explain Ohm's Law \& write limitations of it.}

\begin{solutionbox}

\textbf{Ohm's Law}: The current flowing through a conductor is directly
proportional to the voltage across the conductor and inversely
proportional to its resistance.

Mathematically: V = IR, where:

\begin{itemize}
\tightlist
\item
  V = Voltage (volts)
\item
  I = Current (amperes)
\item
  R = Resistance (ohms)
\end{itemize}

\begin{verbatim}
flowchart LR
    V[Voltage] {-{-} I[Current]}
    R[Resistance] {-{-}Limits{-}{-} I}
\end{verbatim}

\textbf{Limitations of Ohm's Law}:

\begin{itemize}
\tightlist
\item
  Not applicable to non-linear devices (semiconductors, gas discharge
  tubes)
\item
  Doesn't hold at high temperatures
\item
  Not valid for unilateral elements (diodes)
\item
  Fails for time-varying currents
\end{itemize}

\end{solutionbox}
\begin{mnemonicbox}
``VIRO'' - Voltage Is Resistance times Output current

\end{mnemonicbox}
\subsection*{Question 1(c) [7 marks]}\label{q1c}

\textbf{Explain series and parallel connection of batteries.}

\begin{solutionbox}

\textbf{Series Connection of Batteries:}

\begin{verbatim}
flowchart LR
    B1[Battery 1] {-{-} B2[Battery 2] {-}{-} B3[Battery 3] {-}{-} L[Load]}
    L {-{-} B1}
\end{verbatim}

\textbf{Characteristics of Series Connection:}

\begin{itemize}
\tightlist
\item
  \textbf{Total Voltage} = Sum of individual voltages (V = V_{1} + V_{2} +
  \ldots{} + V_{n})
\item
  \textbf{Current} = Same through all batteries
\item
  \textbf{Applications}: Higher voltage requirements
\item
  \textbf{Internal Resistance}: Increases (R_{s} = r_{1} + r_{2} + \ldots{} + r_{n})
\end{itemize}

\textbf{Parallel Connection of Batteries:}

\begin{verbatim}
flowchart LR
    B1[Battery 1] {-{-} L[Load]}
    B2[Battery 2] {-{-} L}
    B3[Battery 3] {-{-} L}
    L {-{-} B1}
    L {-{-} B2}
    L {-{-} B3}
\end{verbatim}

\textbf{Characteristics of Parallel Connection:}

\begin{itemize}
\tightlist
\item
  \textbf{Voltage} = Same as individual battery (if identical)
\item
  \textbf{Total Current} = Sum of individual currents (I = I_{1} + I_{2} +
  \ldots{} + I_{n})
\item
  \textbf{Applications}: Higher current capacity required
\item
  \textbf{Internal Resistance}: Decreases (1/R_{p} = 1/r_{1} + 1/r_{2} + \ldots{}
  + 1/r_{n})
\end{itemize}

\end{solutionbox}
\begin{mnemonicbox}
``VSCP'' - Voltage Sums in Series, Current Parallels

\end{mnemonicbox}
\subsection*{Question 1(c) OR [7
marks]}\label{q1c}

\textbf{Explain series and parallel connection of Resistors.}

\begin{solutionbox}

\textbf{Series Connection of Resistors:}

\begin{verbatim}
flowchart LR
    S[Source] {-{-} R1[R1] {-}{-} R2[R2] {-}{-} R3[R3] {-}{-} S}
\end{verbatim}

\textbf{Characteristics of Series Connection:}

\begin{itemize}
\tightlist
\item
  \textbf{Equivalent Resistance} = Sum of individual resistances (R_{s} =
  R_{1} + R_{2} + \ldots{} + R_{n})
\item
  \textbf{Current} = Same through all resistors
\item
  \textbf{Voltage} = Divided across resistors proportional to resistance
  values
\item
  \textbf{Power} divided as per voltage distribution
\end{itemize}

\textbf{Parallel Connection of Resistors:}

\begin{verbatim}
flowchart LR
    S[Source] {-{-} R1[R1]}
    S {-{-} R2[R2]}
    S {-{-} R3[R3]}
    R1 {-{-} S}
    R2 {-{-} S}
    R3 {-{-} S}
\end{verbatim}

\textbf{Characteristics of Parallel Connection:}

\begin{itemize}
\tightlist
\item
  \textbf{Equivalent Resistance}: 1/R_{p} = 1/R_{1} + 1/R_{2} + \ldots{} + 1/R_{n}
\item
  \textbf{Voltage} = Same across all resistors
\item
  \textbf{Current} = Divided inverse-proportionally to resistance values
\item
  \textbf{Total Current} = Sum of individual currents
\end{itemize}

\end{solutionbox}
\begin{mnemonicbox}
``RISE-VICE'' - Resistance Increases in Series,
Voltage Is Constant in Every parallel

\end{mnemonicbox}
\subsection*{Question 2(a) [3 marks]}\label{q2a}

\textbf{Define: (1) Amplitude (2) Frequency (3) Time period}

\begin{solutionbox}

{\def\LTcaptype{none} % do not increment counter
\begin{longtable}[]{@{}
  >{\raggedright\arraybackslash}p{(\linewidth - 2\tabcolsep) * \real{0.3333}}
  >{\raggedright\arraybackslash}p{(\linewidth - 2\tabcolsep) * \real{0.6667}}@{}}
\toprule\noalign{}
\begin{minipage}[b]{\linewidth}\raggedright
Term
\end{minipage} & \begin{minipage}[b]{\linewidth}\raggedright
Definition
\end{minipage} \\
\midrule\noalign{}
\endhead
\bottomrule\noalign{}
\endlastfoot
\textbf{Amplitude} & Maximum displacement of a waveform from its mean
position, measured in volts or amperes \\
\textbf{Frequency} & Number of complete cycles occurring in one second,
measured in hertz (Hz) \\
\textbf{Time Period} & Time taken to complete one cycle of waveform,
measured in seconds (s) \\
\end{longtable}
}

\end{solutionbox}
\begin{mnemonicbox}
``AFT'' - Amplitude is the Full height, Time period
is the Total cycle

\end{mnemonicbox}
\subsection*{Question 2(b) [4 marks]}\label{q2b}

\textbf{10Ω, 20Ω and 30Ω resistors are connected in series and 100V
supply is given to them. Find (1) Equivalent resistance (2) Circuit
current (3) Voltage drop across each Resistor (4) Power loss in each
resistor.}

\begin{solutionbox}

\textbf{Diagram:}

\begin{verbatim}
     +{-{-}[10Ω]{-}{-}[20Ω]{-}{-}[30Ω]{-}{-}+}
     |                        |
   (100V)                     |
     |                        |
     +{-{-}{-}{-}{-}{-}{-}{-}{-}{-}{-}{-}{-}{-}{-}{-}{-}{-}{-}{-}{-}{-}{-}{-}+}
\end{verbatim}

\textbf{Solution:}

{\def\LTcaptype{none} % do not increment counter
\begin{longtable}[]{@{}lll@{}}
\toprule\noalign{}
Parameter & Calculation & Result \\
\midrule\noalign{}
\endhead
\bottomrule\noalign{}
\endlastfoot
Equivalent Resistance & R = 10Ω + 20Ω + 30Ω & 60Ω \\
Circuit Current & I = 100V/60Ω & 1.67A \\
Voltage across 10Ω & V_{1} = 1.67A \times 10Ω & 16.7V \\
Voltage across 20Ω & V_{2} = 1.67A \times 20Ω & 33.3V \\
Voltage across 30Ω & V_{3} = 1.67A \times 30Ω & 50.0V \\
Power in 10Ω & P_{1} = 1.67^{2} \times 10 & 27.8W \\
Power in 20Ω & P_{2} = 1.67^{2} \times 20 & 55.6W \\
Power in 30Ω & P_{3} = 1.67^{2} \times 30 & 83.4W \\
\end{longtable}
}

\end{solutionbox}
\begin{mnemonicbox}
``REÇVP'' - Resistances Equivalent Causes Voltage and
Power division

\end{mnemonicbox}
\subsection*{Question 2(c) [7 marks]}\label{q2c}

\textbf{Explain A.C Through pure Resistor with wave form \& vector
diagram.}

\begin{solutionbox}

In a pure resistive circuit with AC supply:

\textbf{Key Characteristics:}

\begin{itemize}
\tightlist
\item
  Current and voltage are \textbf{in phase} with each other
\item
  Circuit follows Ohm's Law: V = IR
\item
  Power is always positive (P = VI)
\item
  No reactive power consumed
\item
Power factor = 1 (cos

φ = 1)

\end{itemize}

\textbf{Waveform:}

\begin{verbatim}
    │    ╭─╮   ╭─╮   ╭─╮   ╭─╮
    │   ╱   ╲ ╱   ╲ ╱   ╲ ╱   ╲
    │  ╱     V     V     V     ╲
────┼─╱───────────────────────╲─────
    │╱       ╱╲       ╱╲       ╲
    V       V  ╲     V  ╲       V
    │      ╱    ╲   ╱    ╲
    │     ╱      ╲ ╱      ╲
    │    ╰─╯     ╰─╯      ╰─╯

    {-{-}{-} Voltage waveform}
    {-{-}{-} Current waveform (identical phase)}
\end{verbatim}

\textbf{Vector Diagram:}

\begin{verbatim}
         │
         │
         V (voltage)
         │
         │
─────────┼────────
         │        I (current)
         │
         │
\end{verbatim}

\end{solutionbox}
\begin{mnemonicbox}
``PARVIP'' - Pure AC Resistor has Voltage In Phase
with current

\end{mnemonicbox}
\subsection*{Question 2(a) OR [3
marks]}\label{q2a}

\textbf{Define: (1) cycle (2) Form factor (3) Peak factor}

\begin{solutionbox}

{\def\LTcaptype{none} % do not increment counter
\begin{longtable}[]{@{}
  >{\raggedright\arraybackslash}p{(\linewidth - 2\tabcolsep) * \real{0.3333}}
  >{\raggedright\arraybackslash}p{(\linewidth - 2\tabcolsep) * \real{0.6667}}@{}}
\toprule\noalign{}
\begin{minipage}[b]{\linewidth}\raggedright
Term
\end{minipage} & \begin{minipage}[b]{\linewidth}\raggedright
Definition
\end{minipage} \\
\midrule\noalign{}
\endhead
\bottomrule\noalign{}
\endlastfoot
\textbf{Cycle} & One complete repetition of a periodic waveform from
start point to same point again \\
\textbf{Form Factor} & Ratio of RMS value to average value of AC
waveform (For sine wave = 1.11) \\
\textbf{Peak Factor} & Ratio of maximum value to RMS value of AC
waveform (For sine wave = 1.414) \\
\end{longtable}
}

\end{solutionbox}
\begin{mnemonicbox}
``CFP'' - Cycle Finishes a Pattern, Form Factor =
Vrms/Vavg, Peak Factor = Vmax/Vrms

\end{mnemonicbox}
\subsection*{Question 2(b) OR [4
marks]}\label{q2b}

\textbf{20Ω, 30Ω and 50Ω resistors are connected in parallel and 60V
supply is given to them. Find (1) Current in each Resistor. (2) Total
current (3) Equivalent resistance (4) Power loss in each resistor.}

\begin{solutionbox}

\textbf{Diagram:}

\begin{verbatim}
         ┌─[20Ω]─┐
         │       │
     +───┼───────┼───+
     │   │       │   │
    (60V) ├─[30Ω]─┤  │
     │   │       │   │
     │   └─[50Ω]─┘   │
     │               │
     +───────────────+
\end{verbatim}

\textbf{Solution:}

{\def\LTcaptype{none} % do not increment counter
\begin{longtable}[]{@{}lll@{}}
\toprule\noalign{}
Parameter & Calculation & Result \\
\midrule\noalign{}
\endhead
\bottomrule\noalign{}
\endlastfoot
Current in 20Ω & I_{1} = 60V/20Ω & 3A \\
Current in 30Ω & I_{2} = 60V/30Ω & 2A \\
Current in 50Ω & I_{3} = 60V/50Ω & 1.2A \\
Total Current & I = 3A + 2A + 1.2A & 6.2A \\
Equivalent Resistance & 1/Req = 1/20 + 1/30 + 1/50 & 9.68Ω \\
Power in 20Ω & P_{1} = 60V \times 3A & 180W \\
Power in 30Ω & P_{2} = 60V \times 2A & 120W \\
Power in 50Ω & P_{3} = 60V \times 1.2A & 72W \\
\end{longtable}
}

\end{solutionbox}
\begin{mnemonicbox}
``VICTIM'' - Voltage Is Constant, Total current Is
the Measure (in parallel)

\end{mnemonicbox}
\subsection*{Question 2(c) OR [7
marks]}\label{q2c}

\textbf{Explain A.C Through pure capacitor with wave form \& vector
diagram.}

\begin{solutionbox}

In a pure capacitive circuit with AC supply:

\textbf{Key Characteristics:}

\begin{itemize}
\tightlist
\item
  Current \textbf{leads} voltage by 90^\circ
\item
  Capacitive reactance Xc = 1/(2πfC)
\item
  Only reactive power (no active power)
\item
  Power factor = 0 (lagging)
\item
  Average power over complete cycle = 0
\end{itemize}

\textbf{Waveform:}

\begin{verbatim}
           Current
    │      ╭─╮     ╭─╮     ╭─╮     ╭─╮
    │     ╱   ╲   ╱   ╲   ╱   ╲   ╱   ╲
    │    ╱     ╲ ╱     ╲ ╱     ╲ ╱     ╲
────┼───╱───────V───────V───────V───────╲─
    │  ╱         ╲       ╲       ╲       ╲
    │ ╱           ╲       ╲       ╲       ╲
    │╱             ╲       ╲       ╲       ╲
    V               ╰─╮     ╰─╮     ╰─╮     ╰
    │                 │       │       │
    │                 │       │       │
    │                 │       │       │
    │                 V       V       V     
    │                ╱ ╲     ╱ ╲     ╱ ╲   Voltage
    │               ╱   ╲   ╱   ╲   ╱   ╲ 
    │              ╱     ╲ ╱     ╲ ╱     ╲
\end{verbatim}

\textbf{Vector Diagram:}

\begin{verbatim}
         │ I (current)
         │
         │
         │
─────────┼────────
         │
         │
         │
         V V (voltage)
\end{verbatim}

\end{solutionbox}
\begin{mnemonicbox}
``CLEAR-90'' - Capacitive Load has Electrical Angle
Reaching 90^\circ (current leads voltage)

\end{mnemonicbox}
\subsection*{Question 3(a) [3 marks]}\label{q3a}

\textbf{Define RMS value and average value related to alternating
waveform write formula of it.}

\begin{solutionbox}

{\def\LTcaptype{none} % do not increment counter
\begin{longtable}[]{@{}
  >{\raggedright\arraybackslash}p{(\linewidth - 4\tabcolsep) * \real{0.2222}}
  >{\raggedright\arraybackslash}p{(\linewidth - 4\tabcolsep) * \real{0.4444}}
  >{\raggedright\arraybackslash}p{(\linewidth - 4\tabcolsep) * \real{0.3333}}@{}}
\toprule\noalign{}
\begin{minipage}[b]{\linewidth}\raggedright
Term
\end{minipage} & \begin{minipage}[b]{\linewidth}\raggedright
Definition
\end{minipage} & \begin{minipage}[b]{\linewidth}\raggedright
Formula
\end{minipage} \\
\midrule\noalign{}
\endhead
\bottomrule\noalign{}
\endlastfoot
\textbf{RMS Value} & Root Mean Square value - equivalent DC value
producing the same heating effect & Vrms = 0.707 \times Vmax for sine wave \\
\textbf{Average Value} & Mean value of all instantaneous values over
half cycle & Vavg = 0.637 \times Vmax for sine wave \\
\end{longtable}
}

\end{solutionbox}
\begin{mnemonicbox}
``RAM'' - RMS Averages the Mean square (RMS =
0.707\timesVmax, AVG = 0.637\timesVmax)

\end{mnemonicbox}
\subsection*{Question 3(b) [4 marks]}\label{q3b}

\textbf{If A.C. current is represented by equation i=25 sin(314t).
Calculate (1) R.m.s. value (2) Average value (3) Frequency (4) Time
period}

\begin{solutionbox}

\textbf{Given equation:} i = 25 sin(314t)

{\def\LTcaptype{none} % do not increment counter
\begin{longtable}[]{@{}lll@{}}
\toprule\noalign{}
Parameter & Calculation & Result \\
\midrule\noalign{}
\endhead
\bottomrule\noalign{}
\endlastfoot
Maximum value & Imax = 25 A & 25 A \\
RMS value & Irms = Imax/\sqrt2 = 25/1.414 & 17.68 A \\
Average value & Iavg = 2Imax/π = 2\times25/3.14 & 15.92 A \\
Angular frequency & ω = 314 rad/s & 314 rad/s \\
Frequency &

f = ω/2π = 314/6.28 & 50 Hz \\

Time period &

T = 1/f = 1/50 & 0.02 s \\

\end{longtable}
}

\end{solutionbox}
\begin{mnemonicbox}
``SMART'' - Sine's Maximum divided by root 2 equals
RMS Then 2/π for Average

\end{mnemonicbox}
\subsection*{Question 3(c) [7 marks]}\label{q3c}

\textbf{Explain star connection of resistors and Derive equation shows
relationship between voltage and current in star connection.}

\begin{solutionbox}

\textbf{Star (Y) Connection:}

\begin{center}
\textbf{Mermaid Diagram (Code)}
\begin{verbatim}
{Shaded}
{Highlighting}[]
graph TD
    N((N)) {-{-}{-} R1[R1] {-}{-}{-} L1((L1))}
    N {-{-}{-} R2[R2] {-}{-}{-} L2((L2))}
    N {-{-}{-} R3[R3] {-}{-}{-} L3((L3))}
    N((Neutral))
{Highlighting}
{Shaded}
\end{verbatim}
\end{center}

\textbf{Characteristics of Star Connection:}

\begin{itemize}
\tightlist
\item
  Three resistors connected at common point (neutral)
\item
  Line voltage (VL) = \sqrt3 \times Phase voltage (Vph)
\item
  Line current (IL) = Phase current (Iph)
\item
  For balanced load: IL = Iph
\item
  Total power = 3 \times Phase power
\end{itemize}

\textbf{Mathematical Relationship:}

\begin{itemize}
\tightlist
\item
  Phase voltage: Vph = VL/\sqrt3
\item
  Phase current: Iph = IL
\item
  For balanced resistive load: Iph = Vph/R
\item
  Therefore: IL = VL/(\sqrt3\timesR)
\end{itemize}

\end{solutionbox}
\begin{mnemonicbox}
``SLIP-3'' - Star Line current Is Phase current, Line
voltage is Phase voltage times root-3

\end{mnemonicbox}
\subsection*{Question 3(a) OR [3
marks]}\label{q3a}

\textbf{Explain generation of alternating E.M.F.}

\begin{solutionbox}

\textbf{Generation of Alternating EMF:}

\begin{center}
\textbf{Mermaid Diagram (Code)}
\begin{verbatim}
{Shaded}
{Highlighting}[]
graph LR
    subgraph "Rotating Coil in Magnetic Field"
    N[N] {-{-}{-} M((Magnet)) {-}{-}{-} S[S]}
    end
    M {-{-}{-} R[Rotating Coil]}
    R {-{-}{-} EMF[EMF Output]}
{Highlighting}
{Shaded}
\end{verbatim}
\end{center}

\textbf{Process:}

\begin{itemize}
\tightlist
\item
  Coil rotates in uniform magnetic field
\item
  Flux linkage changes with angle of rotation
\item
  Rate of change of flux induces EMF
\item
  EMF follows sinusoidal pattern: e = Emax sin(ωt)
\item
  Frequency depends on rotation speed
\end{itemize}

\end{solutionbox}
\begin{mnemonicbox}
``FRAME'' - Flux Rotation Alternates Magnetic EMF

\end{mnemonicbox}
\subsection*{Question 3(b) OR [4
marks]}\label{q3b}

\textbf{An alternating EMF is expressed by e= 100 sin2π50t. Find out (1)
Max value of EMF (2) Frequency (3) Time period (4) Angular Frequency}

\begin{solutionbox}

\textbf{Given equation:} e = 100 sin2π50t

{\def\LTcaptype{none} % do not increment counter
\begin{longtable}[]{@{}lll@{}}
\toprule\noalign{}
Parameter & Calculation & Result \\
\midrule\noalign{}
\endhead
\bottomrule\noalign{}
\endlastfoot
Maximum EMF & Emax = 100 V & 100 V \\
Angular Frequency &

ω = 2π50 = 314 rad/s & 314 rad/s \\

Frequency & f = 50 Hz (directly from equation) & 50 Hz \\
Time Period &

T = 1/f = 1/50 & 0.02 s \\

\end{longtable}
}

\end{solutionbox}
\begin{mnemonicbox}
``FAST'' - Frequency And period are reciprocals,
Sin's Top value is maximum

\end{mnemonicbox}
\subsection*{Question 3(c) OR [7
marks]}\label{q3c}

\textbf{Explain star connection and Derive equation shows relationship
between voltage and current in delta connection.}

\begin{solutionbox}

\textbf{Delta (Δ) Connection:}

\begin{center}
\textbf{Mermaid Diagram (Code)}
\begin{verbatim}
{Shaded}
{Highlighting}[]
graph LR
    L1((L1)) {-{-}{-} R1[R1] {-}{-}{-} L2((L2))}
    L2 {-{-}{-} R2[R2] {-}{-}{-} L3((L3))}
    L3 {-{-}{-} R3[R3] {-}{-}{-} L1}
{Highlighting}
{Shaded}
\end{verbatim}
\end{center}

\textbf{Characteristics of Delta Connection:}

\begin{itemize}
\tightlist
\item
  Three resistors connected in closed loop
\item
  Line voltage (VL) = Phase voltage (Vph)
\item
  Line current (IL) = \sqrt3 \times Phase current (Iph)
\item
  For balanced load: Vph = VL
\item
  Total power = 3 \times Phase power
\end{itemize}

\textbf{Mathematical Relationship:}

\begin{itemize}
\tightlist
\item
  Phase voltage: Vph = VL
\item
  Phase current: Iph = Vph/R
\item
  Line current: IL = \sqrt3 \times Iph
\item
  Therefore: IL = \sqrt3 \times VL/R
\end{itemize}

\end{solutionbox}
\begin{mnemonicbox}
``DELVIr3'' - Delta Equal Line Voltage, Its line
current equals phase current times root-3

\end{mnemonicbox}
\subsection*{Question 4(a) [3 marks]}\label{q4a}

\textbf{Define (1) M.M.F. (2) Reluctance (3) flux}

\begin{solutionbox}

{\def\LTcaptype{none} % do not increment counter
\begin{longtable}[]{@{}
  >{\raggedright\arraybackslash}p{(\linewidth - 2\tabcolsep) * \real{0.3333}}
  >{\raggedright\arraybackslash}p{(\linewidth - 2\tabcolsep) * \real{0.6667}}@{}}
\toprule\noalign{}
\begin{minipage}[b]{\linewidth}\raggedright
Term
\end{minipage} & \begin{minipage}[b]{\linewidth}\raggedright
Definition
\end{minipage} \\
\midrule\noalign{}
\endhead
\bottomrule\noalign{}
\endlastfoot
\textbf{M.M.F. (Magnetomotive Force)} & The force that produces magnetic
flux in a magnetic circuit, measured in ampere-turns (AT) \\
\textbf{Reluctance} & The magnetic equivalent of resistance, opposition
to magnetic flux, measured in AT/Wb \\
\textbf{Flux} & The total magnetic field passing through a surface,
measured in webers (Wb) \\
\end{longtable}
}

\end{solutionbox}
\begin{mnemonicbox}
``MFR'' - MMF Flows against Reluctance like current
flows against resistance

\end{mnemonicbox}
\subsection*{Question 4(b) [4 marks]}\label{q4b}

\textbf{Explain Apparent, Active and Reactive power in A.C circuits.}

\begin{solutionbox}

{\def\LTcaptype{none} % do not increment counter
\begin{longtable}[]{@{}
  >{\raggedright\arraybackslash}p{(\linewidth - 4\tabcolsep) * \real{0.3077}}
  >{\raggedright\arraybackslash}p{(\linewidth - 4\tabcolsep) * \real{0.3846}}
  >{\raggedright\arraybackslash}p{(\linewidth - 4\tabcolsep) * \real{0.3077}}@{}}
\toprule\noalign{}
\begin{minipage}[b]{\linewidth}\raggedright
Power Type
\end{minipage} & \begin{minipage}[b]{\linewidth}\raggedright
Symbol \& Unit
\end{minipage} & \begin{minipage}[b]{\linewidth}\raggedright
Definition
\end{minipage} \\
\midrule\noalign{}
\endhead
\bottomrule\noalign{}
\endlastfoot
\textbf{Apparent Power} & S (VA) & Vector sum of active and reactive
power \\
\textbf{Active Power} & P (W) & Actual work-producing power consumed by
the load \\
\textbf{Reactive Power} & Q (VAR) & Power that oscillates between source
and load \\
\end{longtable}
}

\textbf{Power Triangle:}

\begin{verbatim}
          \^{ Q (Reactive Power)}
          │
          │
          │
          │           S (Apparent Power)
          │         /
          │        /
          │       /
          │      /
          │     /
          │    θ
          │   /
          └──/───────────{}
             P (Active Power)
\end{verbatim}

\textbf{Relationships:}

\begin{itemize}
\tightlist
\item
  S = \sqrt(P^{2} + Q^{2})
\item
  P = S \times cos θ
\item
  Q = S \times sin θ
\item
Power factor = cos

θ = P/S

\end{itemize}

\end{solutionbox}
\begin{mnemonicbox}
``SPARQ'' - S is Power Apparent, Real is P, Q is
reactive

\end{mnemonicbox}
\subsection*{Question 4(c) [7 marks]}\label{q4c}

\textbf{Compare electric and magnetic circuit.}

\begin{solutionbox}

{\def\LTcaptype{none} % do not increment counter
\begin{longtable}[]{@{}lll@{}}
\toprule\noalign{}
Parameter & Electric Circuit & Magnetic Circuit \\
\midrule\noalign{}
\endhead
\bottomrule\noalign{}
\endlastfoot
\textbf{Force} & EMF (V) & MMF (AT) \\
\textbf{Opposition} & Resistance (Ω) & Reluctance (AT/Wb) \\
\textbf{Flow} & Current (A) & Flux (Wb) \\
\textbf{Ohm's Law} & V = IR & MMF = Φ \times S \\
\textbf{Medium} & Conductor & Ferromagnetic material \\
\textbf{Energy} & Stored in electric field & Stored in magnetic field \\
\textbf{Leakage} & Negligible & Significant \\
\textbf{Path} & Conductors & Usually closed loop \\
\textbf{Material Property} & Conductivity & Permeability \\
\textbf{Current Flow} & Electron flow & No particle flow \\
\end{longtable}
}

\end{solutionbox}
\begin{mnemonicbox}
``VIRO-MSΦS'' - Voltage Is to Resistance as MMF is to
Reluctance, Our φ flows Similar

\end{mnemonicbox}
\subsection*{Question 4(a) OR [3
marks]}\label{q4a}

\textbf{State and explain Fleming's left hand rule.}

\begin{solutionbox}

\textbf{Fleming's Left Hand Rule:} Used to find the direction of the
force experienced by a current-carrying conductor placed in a magnetic
field.

\begin{center}
\textbf{Mermaid Diagram (Code)}
\begin{verbatim}
{Shaded}
{Highlighting}[]
graph TD
    subgraph "Fleming{s Left Hand Rule"}
    T[Thumb: Force] {-{-}{-} F[Forefinger: Field] {-}{-}{-} M[Middle finger: Current]}
    end
{Highlighting}
{Shaded}
\end{verbatim}
\end{center}

\textbf{Application:}

\begin{itemize}
\tightlist
\item
  Thumb \rightarrow Direction of Force (F)
\item
  Forefinger \rightarrow Direction of magnetic Field (B)
\item
  Middle finger \rightarrow Direction of Current (I)
\item
  Only works when fingers are perpendicular to each other
\end{itemize}

\end{solutionbox}
\begin{mnemonicbox}
``FBI-Left'' - Force, B-field, and I-current
directions are shown by the Left hand

\end{mnemonicbox}
\subsection*{Question 4(b) OR [4
marks]}\label{q4b}

\textbf{Draw power triangle and explain each component of it.}

\begin{solutionbox}

\textbf{Power Triangle:}

\begin{center}
\textbf{Mermaid Diagram (Code)}
\begin{verbatim}
{Shaded}
{Highlighting}[]
graph LR
    O {-{-}{-} P[Active Power P]}
    O {-{-}{-} S[Hypotenuse: Apparent Power S]}
    P {-{-}{-} Q[Reactive Power Q]}
    P {-.{-} A[Power Factor Angle φ]}
{Highlighting}
{Shaded}
\end{verbatim}
\end{center}

\textbf{Components:}

{\def\LTcaptype{none} % do not increment counter
\begin{longtable}[]{@{}
  >{\raggedright\arraybackslash}p{(\linewidth - 6\tabcolsep) * \real{0.3235}}
  >{\raggedright\arraybackslash}p{(\linewidth - 6\tabcolsep) * \real{0.2353}}
  >{\raggedright\arraybackslash}p{(\linewidth - 6\tabcolsep) * \real{0.1765}}
  >{\raggedright\arraybackslash}p{(\linewidth - 6\tabcolsep) * \real{0.2647}}@{}}
\toprule\noalign{}
\begin{minipage}[b]{\linewidth}\raggedright
Component
\end{minipage} & \begin{minipage}[b]{\linewidth}\raggedright
Symbol
\end{minipage} & \begin{minipage}[b]{\linewidth}\raggedright
Unit
\end{minipage} & \begin{minipage}[b]{\linewidth}\raggedright
Meaning
\end{minipage} \\
\midrule\noalign{}
\endhead
\bottomrule\noalign{}
\endlastfoot
\textbf{Active Power} & P & Watt (W) & Real power doing useful work \\
\textbf{Reactive Power} & Q & VAR & Power oscillating between source and
load \\
\textbf{Apparent Power} & S & VA & Vector sum of P and Q \\
\textbf{Power Factor} & cos φ & - & Ratio of active to apparent power
(P/S) \\
\end{longtable}
}

\textbf{Relationships:}

\begin{itemize}
\tightlist
\item
  S^{2} = P^{2} + Q^{2}
\item
  P = S \times cos φ
\item
  Q = S \times sin φ
\end{itemize}

\end{solutionbox}
\begin{mnemonicbox}
``SPQR'' - S is Pythagoras of P and Q, Ratio of P/S
is power factor

\end{mnemonicbox}
\subsection*{Question 4(c) OR [7
marks]}\label{q4c}

\textbf{Differentiate statically and dynamically induced E.M.F.}

\begin{solutionbox}

{\def\LTcaptype{none} % do not increment counter
\begin{longtable}[]{@{}
  >{\raggedright\arraybackslash}p{(\linewidth - 4\tabcolsep) * \real{0.1803}}
  >{\raggedright\arraybackslash}p{(\linewidth - 4\tabcolsep) * \real{0.3934}}
  >{\raggedright\arraybackslash}p{(\linewidth - 4\tabcolsep) * \real{0.4262}}@{}}
\toprule\noalign{}
\begin{minipage}[b]{\linewidth}\raggedright
Parameter
\end{minipage} & \begin{minipage}[b]{\linewidth}\raggedright
Statically Induced EMF
\end{minipage} & \begin{minipage}[b]{\linewidth}\raggedright
Dynamically Induced EMF
\end{minipage} \\
\midrule\noalign{}
\endhead
\bottomrule\noalign{}
\endlastfoot
\textbf{Definition} & EMF induced due to change in current in the
primary coil & EMF induced due to relative motion between conductor and
magnetic field \\
\textbf{Mechanism} & Change in linkage flux & Cutting of magnetic
flux \\
\textbf{Movement} & No physical movement required & Requires relative
motion \\
\textbf{Examples} & Transformer, inductor & Generator, motor \\
\textbf{Faraday's Law} & e = -N(dΦ/dt) & e = Blv \\
\textbf{Application} & Power transfer without motion & Power generation
through motion \\
\textbf{Energy Conversion} & Electrical to magnetic and back &
Mechanical to electrical or vice versa \\
\end{longtable}
}

\end{solutionbox}
\begin{mnemonicbox}
``STIM-DMOV'' - STatically Induced needs Magnetic
flux change, Dynamically needs MOVement

\end{mnemonicbox}
\subsection*{Question 5(a) [3 marks]}\label{q5a}

\textbf{Define (1) solar cell (2) solar panel (3) solar array}

\begin{solutionbox}

{\def\LTcaptype{none} % do not increment counter
\begin{longtable}[]{@{}
  >{\raggedright\arraybackslash}p{(\linewidth - 2\tabcolsep) * \real{0.3333}}
  >{\raggedright\arraybackslash}p{(\linewidth - 2\tabcolsep) * \real{0.6667}}@{}}
\toprule\noalign{}
\begin{minipage}[b]{\linewidth}\raggedright
Term
\end{minipage} & \begin{minipage}[b]{\linewidth}\raggedright
Definition
\end{minipage} \\
\midrule\noalign{}
\endhead
\bottomrule\noalign{}
\endlastfoot
\textbf{Solar Cell} & Basic photovoltaic unit that converts sunlight
directly into electricity through semiconductor material \\
\textbf{Solar Panel} & Collection of solar cells connected in
series/parallel in a frame \\
\textbf{Solar Array} & Multiple solar panels connected together to form
a larger electricity-generating unit \\
\end{longtable}
}

\end{solutionbox}
\begin{mnemonicbox}
``CPA'' - Cell Produces electricity, Panel Arrays
cells, Array is collection of panels

\end{mnemonicbox}
\subsection*{Question 5(b) [4 marks]}\label{q5b}

\textbf{Differentiate HAWT and VAWT.}

\begin{solutionbox}

{\def\LTcaptype{none} % do not increment counter
\begin{longtable}[]{@{}
  >{\raggedright\arraybackslash}p{(\linewidth - 4\tabcolsep) * \real{0.1310}}
  >{\raggedright\arraybackslash}p{(\linewidth - 4\tabcolsep) * \real{0.4405}}
  >{\raggedright\arraybackslash}p{(\linewidth - 4\tabcolsep) * \real{0.4286}}@{}}
\toprule\noalign{}
\begin{minipage}[b]{\linewidth}\raggedright
Parameter
\end{minipage} & \begin{minipage}[b]{\linewidth}\raggedright
Horizontal Axis Wind Turbine (HAWT)
\end{minipage} & \begin{minipage}[b]{\linewidth}\raggedright
Vertical Axis Wind Turbine (VAWT)
\end{minipage} \\
\midrule\noalign{}
\endhead
\bottomrule\noalign{}
\endlastfoot
\textbf{Axis Orientation} & Parallel to ground & Perpendicular to
ground \\
\textbf{Efficiency} & Higher (35-45\%) & Lower (15-30\%) \\
\textbf{Wind Direction} & Needs to face the wind & Works with wind from
any direction \\
\textbf{Generator Location} & At the top of tower & Can be placed at
ground level \\
\textbf{Space Required} & More & Less \\
\textbf{Noise} & Higher & Lower \\
\textbf{Examples} & Propeller-type, widely used commercially & Darrieus,
Savonius designs \\
\end{longtable}
}

\end{solutionbox}
\begin{mnemonicbox}
``HAVE'' - Horizontal Aligns with wind, Vertical
Enjoys omnidirectional wind

\end{mnemonicbox}
\subsection*{Question 5(c) [7 marks]}\label{q5c}

\textbf{Draw and explain the Block diagram of solar power system.}

\begin{solutionbox}

\textbf{Solar Power System Block Diagram:}

\begin{verbatim}
flowchart LR
    S[Solar Panel] {-{-} C[Charge Controller]}
    C {-{-} B[Battery Bank]}
    B {-{-} I[Inverter]}
    I {-{-} L[AC Load]}
    B {-{-} D[DC Load]}
\end{verbatim}

\textbf{Components:}

\begin{enumerate}
\tightlist
\item
  \textbf{Solar Panels}: Convert sunlight to DC electricity
\item
  \textbf{Charge Controller}: Regulates battery charging, prevents
  overcharging
\item
  \textbf{Battery Bank}: Stores energy for use when sunlight isn't
  available
\item
  \textbf{Inverter}: Converts DC to AC power for household appliances
\item
  \textbf{Loads}: AC loads (appliances) and DC loads (LED lights, etc.)
\end{enumerate}

\textbf{Optional Components:}

\begin{itemize}
\tightlist
\item
  \textbf{Monitoring System}: Tracks power generation/consumption
\item
  \textbf{Grid Connection}: Allows selling excess electricity
\end{itemize}

\end{solutionbox}
\begin{mnemonicbox}
``SCBIL'' - Solar Collects, Battery Inverts for Loads

\end{mnemonicbox}
\subsection*{Question 5(a) OR [3
marks]}\label{q5a}

\textbf{Explain the need of green energy for our planet.}

\begin{solutionbox}

\textbf{Need for Green Energy:}

\begin{enumerate}
\tightlist
\item
  \textbf{Sustainability}: Renewable sources won't deplete unlike fossil
  fuels
\item
  \textbf{Pollution Reduction}: Minimizes air and water pollution from
  burning fossil fuels
\item
  \textbf{Climate Change}: Reduces greenhouse gas emissions that cause
  global warming
\item
  \textbf{Energy Security}: Decreases dependence on imported fuels
\item
  \textbf{Economic Benefits}: Creates jobs and reduces health costs
  related to pollution
\end{enumerate}

\end{solutionbox}
\begin{mnemonicbox}
``SPECS'' - Sustainable, Pollution-free, Economic,
Climate-friendly, Secure

\end{mnemonicbox}
\subsection*{Question 5(b) OR [4
marks]}\label{q5b}

\textbf{Classify green energy and explain any one in detail.}

\begin{solutionbox}

\textbf{Classification of Green Energy Sources:}

\begin{verbatim}
mindmap
  root((Green Energy))
    Solar
    Wind
    Hydro
    Biomass
    Geothermal
    Tidal
\end{verbatim}

\textbf{Solar Energy in Detail:}

\begin{itemize}
\tightlist
\item
  \textbf{Working Principle}: Photovoltaic effect converts sunlight to
  electricity
\item
  \textbf{Components}: Solar cells, panels, inverters, batteries
\item
  \textbf{Applications}: Residential power, industrial use,
  transportation
\item
  \textbf{Advantages}: No pollution, abundant source, low maintenance
\item
  \textbf{Limitations}: Weather dependent, requires storage, initial
  cost
\end{itemize}

\end{solutionbox}
\begin{mnemonicbox}
``SWHBGT'' - Sun Wind Hydro Biomass Geothermal Tidal
are green energy types

\end{mnemonicbox}
\subsection*{Question 5(c) OR [7
marks]}\label{q5c}

\textbf{Explain block diagram of wind power system and explain the
operation of wind power system.}

\begin{solutionbox}

\textbf{Wind Power System Block Diagram:}

\begin{verbatim}
flowchart LR
    W[Wind Turbine] {-{-} G[Generator]}
    G {-{-} C[Controller]}
    C {-{-} B[Battery Storage]}
    C {-{-} I[Inverter]}
    I {-{-} L[Load]}
    C {-{-} GR[Grid Connection]}
\end{verbatim}

\textbf{Operation:}

\begin{enumerate}
\tightlist
\item
  \textbf{Wind Turbine}: Converts wind's kinetic energy to mechanical
  energy
\item
  \textbf{Generator}: Transforms mechanical rotation to electrical
  energy
\item
  \textbf{Controller}: Regulates power output and protects from high
  winds
\item
  \textbf{Battery}: Stores excess energy (for off-grid systems)
\item
  \textbf{Inverter}: Converts DC to AC for consumption
\item
  \textbf{Grid Connection}: Feeds excess power to grid or draws when
  needed
\end{enumerate}

\textbf{Types of Wind Turbines:}

\begin{itemize}
\tightlist
\item
  Horizontal Axis (HAWT): Main commercial type
\item
  Vertical Axis (VAWT): Better for urban settings
\end{itemize}

\textbf{Wind Speed Requirements:}

\begin{itemize}
\tightlist
\item
  Cut-in speed: 3-5 m/s
\item
  Rated output: 12-15 m/s
\item
  Cut-out speed: 25 m/s (for safety)
\end{itemize}

\end{solutionbox}
\begin{mnemonicbox}
``WGCBIL'' - Wind Generates, Controller Balances,
Inverter Loads

\end{mnemonicbox}

\end{document}
