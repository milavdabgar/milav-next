\documentclass[10pt,a4paper]{article}

% content/resources/templates/preamble.tex
\usepackage[margin=0.6in]{geometry}
\author{Milav Dabgar}
\usepackage{amsmath,amssymb,amsthm}
\usepackage{booktabs}
\usepackage{multirow}
\usepackage{xcolor}
\usepackage{tcolorbox}
\tcbuselibrary{breakable,skins}
\usepackage[colorlinks=true,linkcolor=blue]{hyperref}
\usepackage{titlesec}
\usepackage{enumitem}
\usepackage{tikz}
\usepackage{pgfplots}
\usepackage{circuitikz}
\usepackage[version=4]{mhchem}
\usepackage{longtable}
\usepackage{array}
\usepackage{float}
\usepackage{caption}
\usepackage{listings}

\lstset{
  basicstyle=\small\ttfamily,
  breaklines=true,
  breakatwhitespace=false,
  postbreak=\mbox{\textcolor{red}{$\hookrightarrow$}\space},
  float=false,
  numbers=left,
  numberstyle=\tiny\color{gray},
  numbersep=10pt,
  xleftmargin=2em,
  keywordstyle=\color{blue},
  commentstyle=\color{green!60!black},
  stringstyle=\color{purple},
  backgroundcolor=\color{gray!5},
  showstringspaces=false,
  tabsize=2,
  captionpos=b,
  keepspaces=true,
  columns=flexible
}

\pgfplotsset{compat=1.18}
\usetikzlibrary{shapes,arrows,positioning,calc,patterns,decorations.pathmorphing,decorations.markings,arrows.meta}

% Color scheme
\definecolor{headcolor}{RGB}{0,102,204}
\definecolor{keycolor}{RGB}{220,20,60}
\definecolor{solutioncolor}{RGB}{34,139,34}
\definecolor{mnemoniccolor}{RGB}{148,0,211}
\definecolor{codecolor}{RGB}{0,0,100}

% Spacing
\setlength{\parskip}{3pt}
\setlist[itemize]{nosep}
\setlist[enumerate]{nosep}

% Title formatting
\titleformat{\section}{\Large\bfseries\color{headcolor}}{\thesection}{1em}{}
\titleformat{\subsection}{\large\bfseries\color{headcolor}}{\thesubsection}{1em}{}

% Pandoc tightlist compatibility
\providecommand{\tightlist}{%
  \setlength{\itemsep}{0pt}\setlength{\parskip}{0pt}}

% Pandoc longtable compatibility
\newcounter{none}
\def\thenone{}


% content/resources/templates/english-boxes.tex
% This file is currently empty - it exists to maintain consistency with the import structure.
% Add custom environments here if needed in the future.


\begin{document}

\begin{center}
{\Huge\bfseries\color{headcolor} Subject Name Solutions}\\[5pt]
{\LARGE 4311101 -- Winter 2023}\\[3pt]
{\large Semester 1 Study Material}\\[3pt]
{\normalsize\textit{Detailed Solutions and Explanations}}
\end{center}

\vspace{10pt}

\subsection*{Question 1(a) [3 marks]}\label{q1a}

\textbf{Define Power \& Energy.}

\begin{solutionbox}

\begin{itemize}
\tightlist
\item
  \textbf{Power}: Rate of doing work or energy consumption per unit
  time. Measured in Watts (W).
\item
  \textbf{Energy}: Ability to do work or the work done. Measured in
  Joules (J) or Watt-hours (Wh).
\end{itemize}


{\def\LTcaptype{none} % do not increment counter
\vspace{-5pt}
\captionof{table}{Power vs Energy}
\vspace{-10pt}
\begin{longtable}[]{@{}
  >{\raggedright\arraybackslash}p{(\linewidth - 6\tabcolsep) * \real{0.2895}}
  >{\raggedright\arraybackslash}p{(\linewidth - 6\tabcolsep) * \real{0.3158}}
  >{\raggedright\arraybackslash}p{(\linewidth - 6\tabcolsep) * \real{0.2368}}
  >{\raggedright\arraybackslash}p{(\linewidth - 6\tabcolsep) * \real{0.1579}}@{}}
\toprule\noalign{}
\begin{minipage}[b]{\linewidth}\raggedright
Parameter
\end{minipage} & \begin{minipage}[b]{\linewidth}\raggedright
Definition
\end{minipage} & \begin{minipage}[b]{\linewidth}\raggedright
Formula
\end{minipage} & \begin{minipage}[b]{\linewidth}\raggedright
Unit
\end{minipage} \\
\midrule\noalign{}
\endhead
\bottomrule\noalign{}
\endlastfoot
Power & Rate of energy transfer & P = W/t & Watt (W) \\
Energy & Capacity to do work & E = P \times t & Joule (J) or Watt-hour
(Wh) \\
\end{longtable}
}

\end{solutionbox}
\begin{mnemonicbox}
``Power Performs, Energy Endures''

\end{mnemonicbox}
\subsection*{Question 1(b) [4 marks]}\label{q1b}

\textbf{Define current and electrical potential.}

\begin{solutionbox}

\textbf{Diagram:}

\begin{verbatim}
flowchart LR
    A[Electron Flow] {-{-}|Rate of Flow| B[Current]}
    C[Potential Energy] {-{-}|Per Unit Charge| D[Voltage]}
\end{verbatim}

\begin{itemize}
\tightlist
\item
  \textbf{Current}: Flow of electric charge per unit time. Measured in
  Amperes (A).
\item
  \textbf{Electrical Potential}: Work done per unit charge to move a
  charge from one point to another. Measured in Volts (V).
\end{itemize}

\end{solutionbox}
\begin{mnemonicbox}
``Current Charges, Potential Pushes''

\end{mnemonicbox}
\subsection*{Question 1(c) [7 marks]}\label{q1c}

\textbf{Explain KCL and KVL with examples.}

\begin{solutionbox}

\textbf{Diagram:}

\begin{verbatim}
+{-{-}{-}{-}{-}+        i1}
      |        ↓
      |    R1
      +{-{-}{-}{-}{-}///{-}{-}{-}{-}+}
      |               |
      |               |
     +++              |
     | | V1       R2  |
     +++          /{//}
      |               |
      |               |
      +{-{-}{-}{-}{-}{-}{-}{-}{-}{-}{-}{-}{-}{-}{-}+}
         i2 ↑    ↑ i3
            {    /}
             {  /}
              {/}
              R3
              |
              |
             {-{-}{-}}
              {-}
\end{verbatim}

\textbf{Kirchhoff's Current Law (KCL):}

\begin{itemize}
\tightlist
\item
  Sum of currents entering a node equals sum of currents leaving it.
\item
  Example: At node X, i1 + i2 = i3
\end{itemize}

\textbf{Kirchhoff's Voltage Law (KVL):}

\begin{itemize}
\tightlist
\item
  Sum of voltage drops around any closed loop equals zero.
\item
  Example: V1 - V(R1) - V(R2) = 0
\end{itemize}

\end{solutionbox}
\begin{mnemonicbox}
``Currents Come-Leave, Voltages Voyage-Loop''

\end{mnemonicbox}
\subsection*{Question 1(c) OR [7
marks]}\label{q1c}

\textbf{Explain different types of connections for Resistors.}

\begin{solutionbox}

\textbf{Diagram:}

\begin{center}
\textbf{Mermaid Diagram (Code)}
\begin{verbatim}
{Shaded}
{Highlighting}[]
graph TD
    subgraph "Series Connection"
    A[R1] {-{-}{-} B[R2] {-}{-}{-} C[R3]}
    end
    subgraph "Parallel Connection"
    D[R1] 
    E[R2]
    F[R3]
    G {-{-}{-} D \& E \& F {-}{-}{-} H}
    end
{Highlighting}
{Shaded}
\end{verbatim}
\end{center}


{\def\LTcaptype{none} % do not increment counter
\vspace{-5pt}
\captionof{table}{Series vs Parallel Connection}
\vspace{-10pt}
\begin{longtable}[]{@{}
  >{\raggedright\arraybackslash}p{(\linewidth - 4\tabcolsep) * \real{0.2157}}
  >{\raggedright\arraybackslash}p{(\linewidth - 4\tabcolsep) * \real{0.3725}}
  >{\raggedright\arraybackslash}p{(\linewidth - 4\tabcolsep) * \real{0.4118}}@{}}
\toprule\noalign{}
\begin{minipage}[b]{\linewidth}\raggedright
Parameter
\end{minipage} & \begin{minipage}[b]{\linewidth}\raggedright
Series Connection
\end{minipage} & \begin{minipage}[b]{\linewidth}\raggedright
Parallel Connection
\end{minipage} \\
\midrule\noalign{}
\endhead
\bottomrule\noalign{}
\endlastfoot
Total Resistance & Req = R1 + R2 + R3 + \ldots{} & 1/Req = 1/R1 + 1/R2 +
1/R3 + \ldots{} \\
Current & Same through all resistors & Divides through each path \\
Voltage & Divides across resistors & Same across all resistors \\
Application & Voltage dividers & Current division \\
\end{longtable}
}

\end{solutionbox}
\begin{mnemonicbox}
``Series Sum, Parallel Parts''

\end{mnemonicbox}
\subsection*{Question 2(a) [3 marks]}\label{q2a}

\textbf{Define Resistance and Resistivity. Also state their unit of
measurement.}

\begin{solutionbox}

\begin{itemize}
\tightlist
\item
  \textbf{Resistance}: Opposition to current flow, measured in Ohms (Ω).
  R = V/I.
\item
  \textbf{Resistivity}: Material property indicating resistance per unit
  dimension, measured in Ohm-meter (Ω·m). ρ = RA/L.
\end{itemize}

\end{solutionbox}
\begin{mnemonicbox}
``Resistance Restricts, Resistivity Relates to
material''

\end{mnemonicbox}
\subsection*{Question 2(b) [4 marks]}\label{q2b}

\textbf{Define cell and write names of different types of cell.}

\begin{solutionbox}

\textbf{Diagram:}

\begin{verbatim}
    +{-{-}{-}{-}{-}{-}{-}{-}+}
    |        |
    | +    {- |}
    |  {  /  |}
    |   {/   |}
    |        |
    +{-{-}{-}{-}{-}{-}{-}{-}+}
      Battery
\end{verbatim}

\begin{itemize}
\tightlist
\item
  \textbf{Cell}: Device that converts chemical energy into electrical
  energy creating a voltage.
\end{itemize}

\textbf{Types of Cells:}

\begin{enumerate}
\tightlist
\item
  \textbf{Primary cells}: Dry cell, Alkaline cell, Mercury cell
\item
  \textbf{Secondary cells}: Lead-acid, Nickel-Cadmium, Lithium-ion
\end{enumerate}

\end{solutionbox}
\begin{mnemonicbox}
``Primary Produces once, Secondary Serves
repeatedly''

\end{mnemonicbox}
\subsection*{Question 2(c) [7 marks]}\label{q2c}

\textbf{Calculate total equivalent resistance of the above circuit if
R1=5Ω, R2=3Ω, R3=4Ω, R4=1Ω, R5=2Ω.}

\begin{solutionbox}

\textbf{Diagram:}

\begin{verbatim}
                  R1
                /{//}
       +{-{-}{-}{-}{-}{-}{-}+      +{-}{-}{-}{-}{-}{-}+}
       |                     |
       |                     |
       |                     |
    R2 /{          R3       / R5}
       {/         ///    /}
       |       +{-{-}+      +{-}{-}+}
       |       |            |
       +{-{-}{-}{-}{-}{-}{-}+            |}
                            |
                R4          |
               /{//       |}
       +{-{-}{-}{-}{-}{-}{-}+      +{-}{-}{-}{-}{-}+}
       |                    |
       +{-{-}{-}{-}{-}{-}{-}{-}{-}{-}{-}{-}{-}{-}{-}{-}{-}{-}{-}{-}+}
\end{verbatim}

\textbf{Step-by-step solution:}

\begin{enumerate}
\tightlist
\item
  R2 and R3 are in series: R23 = R2 + R3 = 3Ω + 4Ω = 7Ω
\item
  R23 and R4 are in parallel: 1/R234 = 1/7 + 1/1 = (1+7)/7 = 8/7
  Therefore, R234 = 7/8 = 0.875Ω
\item
  R1, R234, and R5 are in series: Req = R1 + R234 + R5 = 5Ω + 0.875Ω +
  2Ω = 7.875Ω
\end{enumerate}

\textbf{Therefore, equivalent resistance = 7.875Ω}

\end{solutionbox}
\begin{mnemonicbox}
``Series-Sum, Parallel-Product over Sum''

\end{mnemonicbox}
\subsection*{Question 2(a) OR [3
marks]}\label{q2a}

\textbf{Find the cost of energy if 100W bulb operated 10 hours daily for
30 days. Rate of energy is Rupees 5/unit.}

\begin{solutionbox}


{\def\LTcaptype{none} % do not increment counter
\vspace{-5pt}
\captionof{table}{Energy Calculation}
\vspace{-10pt}
\begin{longtable}[]{@{}lll@{}}
\toprule\noalign{}
Parameter & Value & Calculation \\
\midrule\noalign{}
\endhead
\bottomrule\noalign{}
\endlastfoot
Power & 100W = 0.1kW & Given \\
Operating hours & 10 hours/day \times 30 days = 300 hours & Given \\
Energy consumed & 0.1kW \times 300h = 30kWh = 30 units &

E = P \times t \\

Rate & Rs. 5/unit & Given \\
Total cost & 30 units \times Rs. 5/unit = Rs. 150 & Cost = Units \times Rate \\
\end{longtable}
}

\textbf{Therefore, cost of energy = Rs. 150}

\end{solutionbox}
\begin{mnemonicbox}
``Energy \times Rate = Electric bill fate''

\end{mnemonicbox}
\subsection*{Question 2(b) OR [4
marks]}\label{q2b}

\textbf{State ohm's law and explain the use ohm's law to calculate
current in any circuit.}

\begin{solutionbox}

\textbf{Diagram:}

\begin{center}
\textbf{Mermaid Diagram (Code)}
\begin{verbatim}
{Shaded}
{Highlighting}[]
graph LR
    A[Voltage] {-{-}{}|"V = IR"| B[Current]}
    C[Resistance] {-{-}{} B}
{Highlighting}
{Shaded}
\end{verbatim}
\end{center}

\textbf{Ohm's Law:} Current flowing through a conductor is directly
proportional to voltage and inversely proportional to resistance.

\textbf{Formula: V = IR or I = V/R or R = V/I}

\textbf{Application:} To find current in a circuit, measure voltage
across a component and divide by its resistance (I = V/R).

\end{solutionbox}
\begin{mnemonicbox}
``Volts Invite current, Resistance Restricts''

\end{mnemonicbox}
\subsection*{Question 2(c) OR [7
marks]}\label{q2c}

\textbf{Show that the current in a purely capacitive circuit leads the
applied voltage by 90^\circ and the current in a purely inductive circuit
lags the applied voltage by 90^\circ.}

\begin{solutionbox}

\textbf{Diagrams:}

\begin{center}
\textbf{Mermaid Diagram (Code)}
\begin{verbatim}
{Shaded}
{Highlighting}[]
graph TD
    subgraph "Capacitive Circuit"
    A[Voltage] {-{-}{-} B["Voltage = V sin(ωt)"]}
    C[Current] {-{-}{-} D["Current = I sin(ωt + 90^)"]}
    end
    subgraph "Inductive Circuit"
    E[Voltage] {-{-}{-} F["Voltage = V sin(ωt)"]}
    G[Current] {-{-}{-} H["Current = I sin(ωt {-} 90^)"]}
    end
{Highlighting}
{Shaded}
\end{verbatim}
\end{center}

\textbf{For Capacitive Circuit:}

\begin{itemize}
\tightlist
\item
  Voltage equation: v = V sin(ωt)
\item
Current:

i = C \times dv/dt = ωCV cos(ωt) = I sin(ωt + 90^\circ)

\item
  Current leads voltage by 90^\circ
\end{itemize}

\textbf{For Inductive Circuit:}

\begin{itemize}
\tightlist
\item
Voltage equation:

v = L \times di/dt = ωLI cos(ωt) = V sin(ωt + 90^\circ)

\item
  Current: i = I sin(ωt)
\item
  Current lags voltage by 90^\circ
\end{itemize}

\end{solutionbox}
\begin{mnemonicbox}
``ELI the ICE man'' - In EL (inductor), I lags E; in
ICE (capacitor), I leads E

\end{mnemonicbox}
\subsection*{Question 3(a) [3 marks]}\label{q3a}

\textbf{Define cycle, form factor and amplitude.}

\begin{solutionbox}

\textbf{Diagram:}

\begin{verbatim}
    \^{}
    |    /{      /}
    |   /  {    /  }
A{-{-}{-}|{-}{-}/{-}{-}{-}{-}{-}{-}/{-}{-}{-}{-}{-}{-}}
    | /      {/      }
    |/                {}
    +{-{-}{-}{-}{-}{-}{-}{-}{-}{-}{-}{-}{-}{-}{-}{-}{-}{-}{-}{-}}
         |{-{-}{-}{-}{-}{-}|}
          cycle
\end{verbatim}

\begin{itemize}
\tightlist
\item
  \textbf{Cycle}: One complete repetition of a waveform.
\item
  \textbf{Form Factor}: Ratio of RMS value to average value. For sine
  wave = 1.11.
\item
  \textbf{Amplitude}: Maximum displacement of a waveform from its mean
  position.
\end{itemize}

\end{solutionbox}
\begin{mnemonicbox}
``Cycles Complete, Form Factors Find ratio, Amplitude
Achieves maximum''

\end{mnemonicbox}
\subsection*{Question 3(b) [4 marks]}\label{q3b}

\textbf{Define RMS and Average value. Write expression of RMS and
average value of sinusoidal waveform.}

\begin{solutionbox}


{\def\LTcaptype{none} % do not increment counter
\vspace{-5pt}
\captionof{table}{RMS vs Average Value}
\vspace{-10pt}
\begin{longtable}[]{@{}
  >{\raggedright\arraybackslash}p{(\linewidth - 4\tabcolsep) * \real{0.2444}}
  >{\raggedright\arraybackslash}p{(\linewidth - 4\tabcolsep) * \real{0.2667}}
  >{\raggedright\arraybackslash}p{(\linewidth - 4\tabcolsep) * \real{0.4889}}@{}}
\toprule\noalign{}
\begin{minipage}[b]{\linewidth}\raggedright
Parameter
\end{minipage} & \begin{minipage}[b]{\linewidth}\raggedright
Definition
\end{minipage} & \begin{minipage}[b]{\linewidth}\raggedright
Formula for Sine Wave
\end{minipage} \\
\midrule\noalign{}
\endhead
\bottomrule\noalign{}
\endlastfoot
RMS Value & Square root of mean of squared values & Vrms = Vm/\sqrt2 = 0.707
Vm \\
Average Value & Mean of all instantaneous values over half cycle & Vavg
= 2Vm/π = 0.637 Vm \\
\end{longtable}
}

\begin{itemize}
\tightlist
\item
  \textbf{RMS (Root Mean Square)}: Equivalent DC value that produces
  same heating effect.
\item
  \textbf{Average Value}: Mean of all instantaneous values over a half
  cycle.
\end{itemize}

\end{solutionbox}
\begin{mnemonicbox}
``RMS Relates to heating, Average Adds and divides''

\end{mnemonicbox}
\subsection*{Question 3(c) [7 marks]}\label{q3c}

\textbf{Explain the terms Apparent power, True Power and Reactive power.
State their unit of measurement.}

\begin{solutionbox}

\textbf{Diagram:}

\begin{center}
\textbf{Mermaid Diagram (Code)}
\begin{verbatim}
{Shaded}
{Highlighting}[]
graph TD
    subgraph "Power Triangle"
    A[True Power P] {-{-}{-} B[Apparent Power S]}
    C[Reactive Power Q] {-{-}{-} B}
    end
{Highlighting}
{Shaded}
\end{verbatim}
\end{center}


{\def\LTcaptype{none} % do not increment counter
\vspace{-5pt}
\captionof{table}{Types of Power}
\vspace{-10pt}
\begin{longtable}[]{@{}
  >{\raggedright\arraybackslash}p{(\linewidth - 6\tabcolsep) * \real{0.3077}}
  >{\raggedright\arraybackslash}p{(\linewidth - 6\tabcolsep) * \real{0.3077}}
  >{\raggedright\arraybackslash}p{(\linewidth - 6\tabcolsep) * \real{0.2308}}
  >{\raggedright\arraybackslash}p{(\linewidth - 6\tabcolsep) * \real{0.1538}}@{}}
\toprule\noalign{}
\begin{minipage}[b]{\linewidth}\raggedright
Power Type
\end{minipage} & \begin{minipage}[b]{\linewidth}\raggedright
Definition
\end{minipage} & \begin{minipage}[b]{\linewidth}\raggedright
Formula
\end{minipage} & \begin{minipage}[b]{\linewidth}\raggedright
Unit
\end{minipage} \\
\midrule\noalign{}
\endhead
\bottomrule\noalign{}
\endlastfoot
Apparent Power (S) & Total power supplied & S = VI & VA (Volt-Ampere) \\
True Power (P) & Actual power consumed & P = VI cos φ & W (Watt) \\
Reactive Power (Q) & Power oscillating between source and load & Q = VI
sin φ & VAR (Volt-Ampere Reactive) \\
\end{longtable}
}

\textbf{Power Triangle:} S^{2} = P^{2} + Q^{2}

\end{solutionbox}
\begin{mnemonicbox}
``Active Performs work, Reactive Returns energy,
Apparent Adds vectors''

\end{mnemonicbox}
\subsection*{Question 3(a) OR [3
marks]}\label{q3a}

\textbf{Write mathematical expressions of 3-phase voltages.}

\begin{solutionbox}

\textbf{Three-phase voltage expressions:}


{\def\LTcaptype{none} % do not increment counter
\vspace{-5pt}
\captionof{table}{3-Phase Voltages}
\vspace{-10pt}
\begin{longtable}[]{@{}ll@{}}
\toprule\noalign{}
Phase & Expression \\
\midrule\noalign{}
\endhead
\bottomrule\noalign{}
\endlastfoot
R-phase & VR = Vm sin(ωt) \\
Y-phase & VY = Vm sin(ωt - 120^\circ) \\
B-phase & VB = Vm sin(ωt - 240^\circ) \\
\end{longtable}
}

Where Vm is the maximum voltage and ω is the angular frequency.

\end{solutionbox}
\begin{mnemonicbox}
``Red phase Reference, Yellow lags 120^\circ, Blue brings
up 240^\circ''

\end{mnemonicbox}
\subsection*{Question 3(b) OR [4
marks]}\label{q3b}

\textbf{Define crest factor and state value of crest factor for sine
wave.}

\begin{solutionbox}

\textbf{Diagram:}

\begin{verbatim}
    \^{}
    |    /{      /}
    |   /  {    /  }
{-{-}{-}{-}|{-}{-}/{-}{-}{-}{-}{-}{-}/{-}{-}{-}{-}{-}{-}}
    | /      {/      }
    |/                {}
    +{-{-}{-}{-}{-}{-}{-}{-}{-}{-}{-}{-}{-}{-}{-}{-}{-}{-}{-}{-}}
    
    Peak value
    |{-{-}{-}{-}{-}{-}{-}{-}|}
    |   RMS  |
    |   value|
\end{verbatim}

\begin{itemize}
\tightlist
\item
  \textbf{Crest Factor}: Ratio of peak value to RMS value of a waveform.
\item
  \textbf{Formula}: Crest Factor = Peak Value / RMS Value
\item
  \textbf{For sine wave}: Crest Factor = 1/0.707 = 1.414
\end{itemize}

\end{solutionbox}
\begin{mnemonicbox}
``Crest Compares peak to RMS''

\end{mnemonicbox}
\subsection*{Question 3(c) OR [7
marks]}\label{q3c}

\textbf{Describe different three phase electrical connections.}

\begin{solutionbox}

\textbf{Diagram:}

\begin{center}
\textbf{Mermaid Diagram (Code)}
\begin{verbatim}
{Shaded}
{Highlighting}[]
graph TD
    subgraph "Star Connection"
    A1[R] {-{-}{-} D[Neutral]}
    B1[Y] {-{-}{-} D}
    C1[B] {-{-}{-} D}
    end
    
    subgraph "Delta Connection"
    A2[R] {-{-}{-} B2[Y]}
    B2 {-{-}{-} C2[B]}
    C2 {-{-}{-} A2}
    end
{Highlighting}
{Shaded}
\end{verbatim}
\end{center}


{\def\LTcaptype{none} % do not increment counter
\vspace{-5pt}
\captionof{table}{Star vs Delta Connection}
\vspace{-10pt}
\begin{longtable}[]{@{}
  >{\raggedright\arraybackslash}p{(\linewidth - 4\tabcolsep) * \real{0.2037}}
  >{\raggedright\arraybackslash}p{(\linewidth - 4\tabcolsep) * \real{0.3889}}
  >{\raggedright\arraybackslash}p{(\linewidth - 4\tabcolsep) * \real{0.4074}}@{}}
\toprule\noalign{}
\begin{minipage}[b]{\linewidth}\raggedright
Parameter
\end{minipage} & \begin{minipage}[b]{\linewidth}\raggedright
Star (Y) Connection
\end{minipage} & \begin{minipage}[b]{\linewidth}\raggedright
Delta (Δ) Connection
\end{minipage} \\
\midrule\noalign{}
\endhead
\bottomrule\noalign{}
\endlastfoot
Line Voltage (VL) & \sqrt3 \times Phase Voltage & Same as Phase Voltage \\
Line Current (IL) & Same as Phase Current & \sqrt3 \times Phase Current \\
Neutral Wire & Present & Absent \\
Application & Unbalanced loads, Residential & Balanced loads,
Industrial \\
\end{longtable}
}

\end{solutionbox}
\begin{mnemonicbox}
``Star Shows neutral, Delta Delivers higher current''

\end{mnemonicbox}
\subsection*{Question 4(a) [3 marks]}\label{q4a}

\textbf{Calculate the peak to peak value of a sinusoidal voltage if RMS
value is 230V.}

\begin{solutionbox}


{\def\LTcaptype{none} % do not increment counter
\vspace{-5pt}
\captionof{table}{Calculation Steps}
\vspace{-10pt}
\begin{longtable}[]{@{}lll@{}}
\toprule\noalign{}
Parameter & Formula & Calculation \\
\midrule\noalign{}
\endhead
\bottomrule\noalign{}
\endlastfoot
RMS Value & Given & 230V \\
Peak Value & Vm = \sqrt2 \times Vrms & Vm = \sqrt2 \times 230 = 325.27V \\
Peak-to-Peak Value & Vp-p = 2 \times Vm & Vp-p = 2 \times 325.27 = 650.54V \\
\end{longtable}
}

\textbf{Therefore, peak-to-peak value = 650.54V}

\end{solutionbox}
\begin{mnemonicbox}
``RMS to Peak - multiply by \sqrt2, Peak to Peak - double
it''

\end{mnemonicbox}
\subsection*{Question 4(b) [4 marks]}\label{q4b}

\textbf{An alternating current is given by i=142.14sin628t find
frequency and time period.}

\begin{solutionbox}


{\def\LTcaptype{none} % do not increment counter
\vspace{-5pt}
\captionof{table}{Calculation Steps}
\vspace{-10pt}
\begin{longtable}[]{@{}lll@{}}
\toprule\noalign{}
Parameter & Formula & Calculation \\
\midrule\noalign{}
\endhead
\bottomrule\noalign{}
\endlastfoot
Given equation &

i = 142.14 sin(628t) &

ω = 628 rad/s \\

Frequency &

f = ω/(2π) &

f = 628/(2π) = 100 Hz \\

Time Period &

T = 1/f &

T = 1/100 = 0.01

s = 10 ms \\

\end{longtable}
}

\textbf{Therefore, frequency = 100 Hz and time period = 0.01 s}

\end{solutionbox}
\begin{mnemonicbox}
``Frequency From omega divide 2π, Time takes
inverse''

\end{mnemonicbox}
\subsection*{Question 4(c) [7 marks]}\label{q4c}

\textbf{State and explain Fleming's left hand rule and right hand rule.}

\begin{solutionbox}

\textbf{Diagram:}

\begin{verbatim}
Left Hand Rule           Right Hand Rule
    F                        F
    \^{                        \^{}}
    |                        |
    |                        |
    +{-{-}B                    +{-}{-}B}
   /                        /
  /                        /
 I                        I
\end{verbatim}

\textbf{Fleming's Left Hand Rule (Motor):}

\begin{itemize}
\tightlist
\item
  Used to determine direction of \textbf{force} on a current-carrying
  conductor in a magnetic field.
\item
  Hold left hand with thumb, fore and middle fingers at right angles.
\item
  Thumb: Motion (Force)
\item
  Forefinger: Magnetic field
\item
  Middle finger: Current
\end{itemize}

\textbf{Fleming's Right Hand Rule (Generator):}

\begin{itemize}
\tightlist
\item
  Used to determine direction of \textbf{induced current} when a
  conductor moves in a magnetic field.
\item
  Hold right hand with thumb, fore and middle fingers at right angles.
\item
  Thumb: Motion of conductor
\item
  Forefinger: Magnetic field
\item
  Middle finger: Induced current
\end{itemize}

\end{solutionbox}
\begin{mnemonicbox}
``Left Lifts motors, Right Raises generators''

\end{mnemonicbox}
\subsection*{Question 4(a) OR [3
marks]}\label{q4a}

\textbf{A conductor of length 1 metre moves with speed of 30m/s in
magnetic field of 0.6 Tesla making angle of 30^\circ with the field.
Calculate dynamically EMF induced in it. (use sin 30^\circ=0.5)}

\begin{solutionbox}


{\def\LTcaptype{none} % do not increment counter
\vspace{-5pt}
\captionof{table}{Given Parameters}
\vspace{-10pt}
\begin{longtable}[]{@{}ll@{}}
\toprule\noalign{}
Parameter & Value \\
\midrule\noalign{}
\endhead
\bottomrule\noalign{}
\endlastfoot
Length (l) & 1 meter \\
Speed (v) & 30 m/s \\
Magnetic Field (B) & 0.6 Tesla \\
Angle (θ) & 30^\circ \\
\end{longtable}
}

\textbf{Formula:} E = Blv sin θ

\textbf{Calculation:} E = 0.6 \times 1 \times 30 \times 0.5 = 9 volts

\textbf{Therefore, induced EMF = 9 volts}

\end{solutionbox}
\begin{mnemonicbox}
``EMF Emerges from Field, velocity and Length with
angle''

\end{mnemonicbox}
\subsection*{Question 4(b) OR [4
marks]}\label{q4b}

\textbf{State \& explain Lenz's law.}

\begin{solutionbox}

\textbf{Diagram:}

\begin{verbatim}
    +{-{-}{-}{-}{-}{-}{-}{-}+}
    |   N    |  Moving
    |   |    |  Magnet
    |   v    |
    +{-{-}{-}{-}{-}{-}{-}{-}+}
        |
        v
    +{-{-}{-}{-}{-}{-}{-}{-}+}
    |        |  Induced
    |   ↺    |  Current
    |        |
    +{-{-}{-}{-}{-}{-}{-}{-}+}
      Coil
\end{verbatim}

\textbf{Lenz's Law:} The direction of induced EMF or current is always
such that it opposes the cause that produces it.

\textbf{Application:} When a magnet approaches a coil, induced current
creates a magnetic field that repels the approaching magnet.

\end{solutionbox}
\begin{mnemonicbox}
``Lenz Likes to Oppose''

\end{mnemonicbox}
\subsection*{Question 4(c) OR [7
marks]}\label{q4c}

\textbf{Explain Statically and dynamically induced EMF.}

\begin{solutionbox}


{\def\LTcaptype{none} % do not increment counter
\vspace{-5pt}
\captionof{table}{Statically vs Dynamically Induced EMF}
\vspace{-10pt}
\begin{longtable}[]{@{}
  >{\raggedright\arraybackslash}p{(\linewidth - 4\tabcolsep) * \real{0.1803}}
  >{\raggedright\arraybackslash}p{(\linewidth - 4\tabcolsep) * \real{0.3934}}
  >{\raggedright\arraybackslash}p{(\linewidth - 4\tabcolsep) * \real{0.4262}}@{}}
\toprule\noalign{}
\begin{minipage}[b]{\linewidth}\raggedright
Parameter
\end{minipage} & \begin{minipage}[b]{\linewidth}\raggedright
Statically Induced EMF
\end{minipage} & \begin{minipage}[b]{\linewidth}\raggedright
Dynamically Induced EMF
\end{minipage} \\
\midrule\noalign{}
\endhead
\bottomrule\noalign{}
\endlastfoot
Definition & EMF induced due to change in current/flux & EMF induced due
to movement of conductor in magnetic field \\
Physical action & Fixed conductor, changing field & Moving conductor in
fixed field \\
Example & Transformer & Generator \\
Formula &

e = -N dΦ/dt &

e = Blv sin θ \\

\end{longtable}
}

\end{solutionbox}
\begin{mnemonicbox}
``Static Stays but flux Changes, Dynamic Drives
through field''

\end{mnemonicbox}
\subsection*{Question 5(a) [3 marks]}\label{q5a}

\textbf{Explain PV Cell.}

\begin{solutionbox}

\textbf{Diagram:}

\begin{verbatim}
    Sun Rays
       |||
       vvv
    +{-{-}{-}{-}{-}{-}{-}+}
    |  N    |
    |{-{-}{-}{-}{-}{-}{-}| {-} P{-}N Junction}
    |  P    |
    +{-{-}{-}{-}{-}{-}{-}+}
       | |
       | |
       Load
\end{verbatim}

\begin{itemize}
\tightlist
\item
  \textbf{PV Cell}: Device that converts sunlight directly into
  electricity using photovoltaic effect.
\item
  \textbf{Working}: Sunlight excites electrons in semiconductor
  material, creating voltage difference.
\item
  \textbf{Material}: Typically made from silicon with P-N junction.
\end{itemize}

\end{solutionbox}
\begin{mnemonicbox}
``Photons Visit, Current Created''

\end{mnemonicbox}
\subsection*{Question 5(b) [4 marks]}\label{q5b}

\textbf{Explain the solar PV panel and arrays.}

\begin{solutionbox}

\textbf{Diagram:}

\begin{center}
\textbf{Mermaid Diagram (Code)}
\begin{verbatim}
{Shaded}
{Highlighting}[]
graph LR
    A[Solar Cell] {-{-}{}|"Multiple cells in series"| B[Solar Panel]}
    B {-{-}{}|"Multiple panels connected"| C[Solar Array]}
{Highlighting}
{Shaded}
\end{verbatim}
\end{center}


{\def\LTcaptype{none} % do not increment counter
\vspace{-5pt}
\captionof{table}{Solar System Hierarchy}
\vspace{-10pt}
\begin{longtable}[]{@{}
  >{\raggedright\arraybackslash}p{(\linewidth - 2\tabcolsep) * \real{0.4583}}
  >{\raggedright\arraybackslash}p{(\linewidth - 2\tabcolsep) * \real{0.5417}}@{}}
\toprule\noalign{}
\begin{minipage}[b]{\linewidth}\raggedright
Component
\end{minipage} & \begin{minipage}[b]{\linewidth}\raggedright
Description
\end{minipage} \\
\midrule\noalign{}
\endhead
\bottomrule\noalign{}
\endlastfoot
PV Cell & Basic unit that converts sunlight to electricity (0.5V -
0.6V) \\
PV Panel & Multiple cells connected in series/parallel (typically 12V,
24V) \\
PV Array & Multiple panels connected to achieve required
voltage/current \\
\end{longtable}
}

\end{solutionbox}
\begin{mnemonicbox}
``Cells Combine into Panels, Panels Produce Arrays''

\end{mnemonicbox}
\subsection*{Question 5(c) [7 marks]}\label{q5c}

\textbf{Draw and explain block diagram of wind power system.}

\begin{solutionbox}

\textbf{Diagram:}

\begin{verbatim}
flowchart LR
    A[Wind Turbine] {-{-}|"Mechanical energy"| B[Gearbox]}
    B {-{-}|"High speed rotation"| C[Generator]}
    C {-{-}|"AC power"| D[Power Electronics]}
    D {-{-}|"Controlled output"| E[Transformer]}
    E {-{-}|"Grid{-}compatible power"| F[Grid/Load]}
    G[Control System] {-.{-} A \& C \& D}
\end{verbatim}

\textbf{Components of Wind Power System:}

\begin{enumerate}
\tightlist
\item
  \textbf{Wind Turbine}: Converts wind energy to mechanical energy
\item
  \textbf{Gearbox}: Increases rotational speed for generator
\item
  \textbf{Generator}: Converts mechanical energy to electrical energy
\item
  \textbf{Power Electronics}: Controls and regulates electrical output
\item
  \textbf{Transformer}: Steps up/down voltage for
  transmission/distribution
\item
  \textbf{Control System}: Monitors and optimizes overall operation
\end{enumerate}

\end{solutionbox}
\begin{mnemonicbox}
``Wind Turns Gears, Generating Electrical Returns''

\end{mnemonicbox}
\subsection*{Question 5(a) OR [3
marks]}\label{q5a}

\textbf{State the benefits of green energy.}

\begin{solutionbox}


{\def\LTcaptype{none} % do not increment counter
\vspace{-5pt}
\captionof{table}{Benefits of Green Energy}
\vspace{-10pt}
\begin{longtable}[]{@{}ll@{}}
\toprule\noalign{}
Benefit Category & Examples \\
\midrule\noalign{}
\endhead
\bottomrule\noalign{}
\endlastfoot
Environmental & Reduces pollution, Minimizes carbon footprint \\
Economic & Creates jobs, Reduces energy dependency \\
Health & Improves air quality, Reduces health issues \\
Sustainability & Renewable, Inexhaustible sources \\
\end{longtable}
}

\end{solutionbox}
\begin{mnemonicbox}
``Clean Energy Creates Economic Salvation''

\end{mnemonicbox}
\subsection*{Question 5(b) OR [4
marks]}\label{q5b}

\textbf{Explain Solar PV applications in brief.}

\begin{solutionbox}

\textbf{Diagram:}

\begin{center}
\textbf{Mermaid Diagram (Code)}
\begin{verbatim}
{Shaded}
{Highlighting}[]
graph TD
    A[Solar PV Applications] {-{-}{} B[Residential]}
    A {-{-}{} C[Commercial]}
    A {-{-}{} D[Industrial]}
    A {-{-}{} E[Utility Scale]}
    A {-{-}{} F[Off{-}grid]}
{Highlighting}
{Shaded}
\end{verbatim}
\end{center}

\textbf{Solar PV Applications:}

\begin{enumerate}
\tightlist
\item
  \textbf{Residential}: Rooftop systems, Solar water heaters
\item
  \textbf{Commercial}: Building integrated PV, Solar parking
\item
  \textbf{Industrial}: Process heating, Power generation
\item
  \textbf{Utility Scale}: Solar farms, Grid support
\item
  \textbf{Off-grid}: Rural electrification, Remote applications
\end{enumerate}

\end{solutionbox}
\begin{mnemonicbox}
``Residences, Commerce, Industry Utilize Solar''

\end{mnemonicbox}
\subsection*{Question 5(c) OR [7
marks]}\label{q5c}

\textbf{Explain different types of Green energy.}

\begin{solutionbox}


{\def\LTcaptype{none} % do not increment counter
\vspace{-5pt}
\captionof{table}{Types of Green Energy}
\vspace{-10pt}
\begin{longtable}[]{@{}lll@{}}
\toprule\noalign{}
Type & Source & Applications \\
\midrule\noalign{}
\endhead
\bottomrule\noalign{}
\endlastfoot
Solar & Sun & PV systems, Thermal plants \\
Wind & Moving air & Wind turbines, Windmills \\
Hydro & Flowing water & Dams, Run-of-river systems \\
Biomass & Organic matter & Combustion, Biogas production \\
Geothermal & Earth's heat & Direct heating, Power plants \\
Tidal & Ocean tides & Barrage systems, Tidal turbines \\
\end{longtable}
}

\textbf{Diagram:}

\begin{verbatim}
pie title "Green Energy Sources"
    "Solar" : 30
    "Wind" : 25
    "Hydro" : 20
    "Biomass" : 15
    "Geothermal" : 7
    "Tidal" : 3
\end{verbatim}

\end{solutionbox}
\begin{mnemonicbox}
``Sun, Wind, Hydro, Biomass, Geothermal, Tidal -
Simple Ways Humans Build Green Tomorrow''

\end{mnemonicbox}

\end{document}
