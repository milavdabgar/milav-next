\documentclass[10pt,a4paper]{article}

% content/resources/templates/preamble.tex
\usepackage[margin=0.6in]{geometry}
\author{Milav Dabgar}
\usepackage{amsmath,amssymb,amsthm}
\usepackage{booktabs}
\usepackage{multirow}
\usepackage{xcolor}
\usepackage{tcolorbox}
\tcbuselibrary{breakable,skins}
\usepackage[colorlinks=true,linkcolor=blue]{hyperref}
\usepackage{titlesec}
\usepackage{enumitem}
\usepackage{tikz}
\usepackage{pgfplots}
\usepackage{circuitikz}
\usepackage[version=4]{mhchem}
\usepackage{longtable}
\usepackage{array}
\usepackage{float}
\usepackage{caption}
\usepackage{listings}

\lstset{
  basicstyle=\small\ttfamily,
  breaklines=true,
  breakatwhitespace=false,
  postbreak=\mbox{\textcolor{red}{$\hookrightarrow$}\space},
  float=false,
  numbers=left,
  numberstyle=\tiny\color{gray},
  numbersep=10pt,
  xleftmargin=2em,
  keywordstyle=\color{blue},
  commentstyle=\color{green!60!black},
  stringstyle=\color{purple},
  backgroundcolor=\color{gray!5},
  showstringspaces=false,
  tabsize=2,
  captionpos=b,
  keepspaces=true,
  columns=flexible
}

\pgfplotsset{compat=1.18}
\usetikzlibrary{shapes,arrows,positioning,calc,patterns,decorations.pathmorphing,decorations.markings,arrows.meta}

% Color scheme
\definecolor{headcolor}{RGB}{0,102,204}
\definecolor{keycolor}{RGB}{220,20,60}
\definecolor{solutioncolor}{RGB}{34,139,34}
\definecolor{mnemoniccolor}{RGB}{148,0,211}
\definecolor{codecolor}{RGB}{0,0,100}

% Spacing
\setlength{\parskip}{3pt}
\setlist[itemize]{nosep}
\setlist[enumerate]{nosep}

% Title formatting
\titleformat{\section}{\Large\bfseries\color{headcolor}}{\thesection}{1em}{}
\titleformat{\subsection}{\large\bfseries\color{headcolor}}{\thesubsection}{1em}{}

% Pandoc tightlist compatibility
\providecommand{\tightlist}{%
  \setlength{\itemsep}{0pt}\setlength{\parskip}{0pt}}

% Pandoc longtable compatibility
\newcounter{none}
\def\thenone{}


% content/resources/templates/gujarati-boxes.tex
\usepackage{fontspec}
\usepackage{polyglossia}

% Set Gujarati as main language (document is primarily in Gujarati)
% Note: gloss-gujarati.ldf doesn't exist in polyglossia, but it will use hyphenation patterns
\setdefaultlanguage{gujarati}
\setotherlanguage{english}

% Configure Gujarati font properly
% Use Language=Default to prevent polyglossia from trying to add language-specific features
% that don't exist for Gujarati, which causes "empty feature" warnings
\newfontfamily\gujaratifont[Script=Gujarati,AutoFakeBold=2.5,AutoFakeSlant=0.3]{Noto Sans Gujarati}
\setmainfont[Script=Gujarati,AutoFakeBold=2.5,AutoFakeSlant=0.3]{Noto Sans Gujarati}
% Use Noto Sans Gujarati for monospace to support Gujarati in text
\setmonofont[Scale=0.9]{Noto Sans Gujarati}

% Configure English to use the same font
\newfontfamily\englishfont[Script=Gujarati,AutoFakeBold=2.5,AutoFakeSlant=0.3]{Noto Sans Gujarati}

% Translations for polyglossia
\gappto\captionsgujarati{
  \renewcommand{\tablename}{કોષ્ટક}
  \renewcommand{\figurename}{આકૃતિ}
}

% Helper for TikZ nodes to ensure Gujarati font
\newcommand{\gu}[1]{{\gujaratifont #1}}

% Custom environments
\newtcolorbox{solutionbox}{
    breakable,
    enhanced,
    colback=solutioncolor!5!white,
    colframe=solutioncolor!75!black,
    fonttitle=\bfseries,
    title=જવાબ
}

\newtcolorbox{solutionboxnobreak}{
 colback=solutioncolor!5!white,
 colframe=solutioncolor!75!black,
 fonttitle=\bfseries,
 title=જવાબ
}

\newtcolorbox{keyformula}{
 breakable,
 enhanced,
 colback=keycolor!5!white,
 colframe=keycolor!75!black,
 fonttitle=\bfseries,
 title=રાસાયણિક સમીકરણ/સૂત્ર
}

\newtcolorbox{mnemonicbox}{
 breakable,
 enhanced,
 colback=mnemoniccolor!5!white,
 colframe=mnemoniccolor!75!black,
 fonttitle=\bfseries,
 title=મેમરી ટ્રીક
}


\begin{document}

\begin{center}
{\Huge\bfseries\color{headcolor} Subject Name (Gujarati)}\\[5pt]
{\LARGE 4311101 -- Winter 2024}\\[3pt]
{\large Semester 1 Study Material}\\[3pt]
{\normalsize\textit{Detailed Solutions and Explanations}}
\end{center}

\vspace{10pt}

\subsection*{પ્રશ્ન ૧(અ) [૩
ગુણ]}\label{uxaaauxab0uxab6uxaa8-uxae7uxa85-uxae9-uxa97uxaa3}

\textbf{વિદ્યુત પ્રવાહ, પાવર, અને ઊર્જા ની વ્યાખ્યા આપો.}

\begin{solutionbox}

{\def\LTcaptype{none} % do not increment counter
\begin{longtable}[]{@{}
  >{\raggedright\arraybackslash}p{(\linewidth - 2\tabcolsep) * \real{0.3333}}
  >{\raggedright\arraybackslash}p{(\linewidth - 2\tabcolsep) * \real{0.6667}}@{}}
\toprule\noalign{}
\begin{minipage}[b]{\linewidth}\raggedright
શબ્દ
\end{minipage} & \begin{minipage}[b]{\linewidth}\raggedright
વ્યાખ્યા
\end{minipage} \\
\midrule\noalign{}
\endhead
\bottomrule\noalign{}
\endlastfoot
\textbf{વિદ્યુત પ્રવાહ} & વાહક દ્વારા વિદ્યુત ચાર્જનો પ્રવાહ દર (એમ્પિયર, A માં
માપવામાં આવે છે) \\
\textbf{વિદ્યુત પાવર} & વિદ્યુત ઊર્જાના ટ્રાન્સફર અથવા વપરાશનો દર (વોટ, W માં
માપવામાં આવે છે) \\
\textbf{ઊર્જા} & કાર્ય કરવાની ક્ષમતા, પાવર ગુણાકાર સમય તરીકે માપવામાં આવે છે (જૂલ
અથવા વોટ-કલાક) \\
\end{longtable}
}

\end{solutionbox}
\begin{mnemonicbox}
``CPE: Charge-Per-second, Product-of-VI,
Energy-over-time''

\end{mnemonicbox}
\subsection*{પ્રશ્ન ૧(બ) [૪
ગુણ]}\label{uxaaauxab0uxab6uxaa8-uxae7uxaac-uxaea-uxa97uxaa3}

\textbf{વાહક, અવાહક અને મિશ્ર ધાતુના અવરોધના મૂલ્ય પર તાપમાનની અસર સમજાવો.}

\begin{solutionbox}

{\def\LTcaptype{none} % do not increment counter
\begin{longtable}[]{@{}
  >{\raggedright\arraybackslash}p{(\linewidth - 4\tabcolsep) * \real{0.3409}}
  >{\raggedright\arraybackslash}p{(\linewidth - 4\tabcolsep) * \real{0.4318}}
  >{\raggedright\arraybackslash}p{(\linewidth - 4\tabcolsep) * \real{0.2273}}@{}}
\toprule\noalign{}
\begin{minipage}[b]{\linewidth}\raggedright
મટીરિયલનો પ્રકાર
\end{minipage} & \begin{minipage}[b]{\linewidth}\raggedright
તાપમાનની અસર
\end{minipage} & \begin{minipage}[b]{\linewidth}\raggedright
સમીકરણ
\end{minipage} \\
\midrule\noalign{}
\endhead
\bottomrule\noalign{}
\endlastfoot
\textbf{શુદ્ધ ધાતુઓ} & તાપમાન વધતાં અવરોધ વધે છે & R_{2} = R_{1}[1 +
α(T_{2}-T_{1})] \\
\textbf{મિશ્ર ધાતુઓ} & તાપમાન સાથે થોડોક વધારો (ઓછો α) & R_{2} = R_{1}[1 +
α(T_{2}-T_{1})] \\
\textbf{અવાહકો} & તાપમાન વધતાં અવરોધ ઘટે છે & R_{2} = R_{1}e\^{}(β(1/T_{2}-1/T_{1})) \\
\end{longtable}
}

જ્યાં α તાપમાન ગુણાંક, T તાપમાન, અને R અવરોધ છે

\end{solutionbox}
\begin{mnemonicbox}
``MAI: Metals Add, Alloys Increase-little,
Insulators Invert''

\end{mnemonicbox}
\subsection*{પ્રશ્ન ૧(ક) [૭
ગુણ]}\label{uxaaauxab0uxab6uxaa8-uxae7uxa95-uxaed-uxa97uxaa3}

\textbf{KVL અને KCL ઉદાહરણ સાથે સમજાવો.}

\begin{solutionbox}

\textbf{કિરચોફના નિયમો:}

{\def\LTcaptype{none} % do not increment counter
\begin{longtable}[]{@{}
  >{\raggedright\arraybackslash}p{(\linewidth - 6\tabcolsep) * \real{0.1190}}
  >{\raggedright\arraybackslash}p{(\linewidth - 6\tabcolsep) * \real{0.2619}}
  >{\raggedright\arraybackslash}p{(\linewidth - 6\tabcolsep) * \real{0.2381}}
  >{\raggedright\arraybackslash}p{(\linewidth - 6\tabcolsep) * \real{0.3810}}@{}}
\toprule\noalign{}
\begin{minipage}[b]{\linewidth}\raggedright
નિયમ
\end{minipage} & \begin{minipage}[b]{\linewidth}\raggedright
વિધાન
\end{minipage} & \begin{minipage}[b]{\linewidth}\raggedright
સમીકરણ
\end{minipage} & \begin{minipage}[b]{\linewidth}\raggedright
સર્કિટ ઉદાહરણ
\end{minipage} \\
\midrule\noalign{}
\endhead
\bottomrule\noalign{}
\endlastfoot
\textbf{KCL} & નોડમાં પ્રવેશતા કરંટનો સરવાળો નોડમાંથી નીકળતા કરંટના સરવાળા
બરાબર હોય છે & \sumIin = \sumIout &
\texttt{mermaid\ graph\ TD;\ A((Node));\ I1-\/-\textgreater{}A;\ I2-\/-\textgreater{}A;\ A-\/-\textgreater{}I3;\ A-\/-\textgreater{}I4;} \\
\textbf{KVL} & બંધ લૂપમાં વોલ્ટેજ ડ્રોપનો સરવાળો વોલ્ટેજ રાઈઝના સરવાળા બરાબર હોય
છે & \sumV = 0 &
\texttt{mermaid\ graph\ LR;\ A((+))-\/-\textgreater{}B((-)));\ B-\/-\textgreater{}C((+));\ C-\/-\textgreater{}D((+));\ D-\/-\textgreater{}A;\ linkStyle\ 0\ stroke:red,stroke-width:2px;\ linkStyle\ 1\ stroke:green,stroke-width:2px;\ linkStyle\ 2\ stroke:blue,stroke-width:2px;\ linkStyle\ 3\ stroke:orange,stroke-width:2px;} \\
\end{longtable}
}

\textbf{ઉદાહરણ}:

\begin{itemize}
\tightlist
\item
  \textbf{KCL}: નોડ A પર, જો I_{1} = 5A અને I_{2} = 3A પ્રવેશે છે, તો I_{3} + I_{4} = 8A
  બહાર નીકળવું જોઈએ
\item
  \textbf{KVL}: જો 12V બેટરી અને રેઝિસ્ટર R_{1}(4Ω) અને R_{2}(8Ω)ના લૂપમાં, 12V =
  I\times(4Ω+8Ω)
\end{itemize}

\end{solutionbox}
\begin{mnemonicbox}
``CLAN: Currents Leave And eNter equally, Voltage
Around Loop is Null''

\end{mnemonicbox}
\subsection*{પ્રશ્ન ૧(ક) OR [૭
ગુણ]}\label{uxaaauxab0uxab6uxaa8-uxae7uxa95-or-uxaed-uxa97uxaa3}

\textbf{જરૂરી સૂત્ર સાથે અવરોધનું શ્રેણી અને સમાંતર જોડાણ સમજાવો.}

\begin{solutionbox}

{\def\LTcaptype{none} % do not increment counter
\begin{longtable}[]{@{}
  >{\raggedright\arraybackslash}p{(\linewidth - 6\tabcolsep) * \real{0.1846}}
  >{\raggedright\arraybackslash}p{(\linewidth - 6\tabcolsep) * \real{0.2615}}
  >{\raggedright\arraybackslash}p{(\linewidth - 6\tabcolsep) * \real{0.1538}}
  >{\raggedright\arraybackslash}p{(\linewidth - 6\tabcolsep) * \real{0.4000}}@{}}
\toprule\noalign{}
\begin{minipage}[b]{\linewidth}\raggedright
જોડાણ
\end{minipage} & \begin{minipage}[b]{\linewidth}\raggedright
સર્કિટ ડાયાગ્રામ
\end{minipage} & \begin{minipage}[b]{\linewidth}\raggedright
સમીકરણ
\end{minipage} & \begin{minipage}[b]{\linewidth}\raggedright
કરંટ/વોલ્ટેજ સંબંધ
\end{minipage} \\
\midrule\noalign{}
\endhead
\bottomrule\noalign{}
\endlastfoot
\textbf{શ્રેણી} &
\texttt{mermaid\ graph\ LR;\ A-\/-\/-B[(R_{1})]-\/-\/-C[(R_{2})]-\/-\/-D[(R_{3})]-\/-\/-E;}
& Req = R_{1} + R_{2} + R_{3} + \ldots{} + Rn & બધા અવરોધોમાં સમાન કરંટ \\
\textbf{સમાંતર} &
\texttt{mermaid\ graph\ TD;\ A-\/-\/-B;\ A-\/-\/-C[(R_{1})]-\/-\/-B;\ A-\/-\/-D[(R_{2})]-\/-\/-B;\ A-\/-\/-E[(R_{3})]-\/-\/-B;}
& 1/Req = 1/R_{1} + 1/R_{2} + 1/R_{3} + \ldots{} + 1/Rn & બધા અવરોધોમાં સમાન
વોલ્ટેજ \\
\end{longtable}
}

\begin{itemize}
\tightlist
\item
  \textbf{શ્રેણી}: કુલ અવરોધ વધે છે, કરંટ ઘટે છે
\item
  \textbf{સમાંતર}: કુલ અવરોધ ઘટે છે, કરંટ વધે છે
\end{itemize}

\end{solutionbox}
\begin{mnemonicbox}
``SPARC: Series Plus All Resistors, parallel
Combines with reciprocals''

\end{mnemonicbox}
\subsection*{પ્રશ્ન ૨(અ) [૩
ગુણ]}\label{uxaaauxab0uxab6uxaa8-uxae8uxa85-uxae9-uxa97uxaa3}

\textbf{અવરોધના મૂલ્યને અસર કરતાં પરિબળો લખો.}

\begin{solutionbox}

{\def\LTcaptype{none} % do not increment counter
\begin{longtable}[]{@{}lll@{}}
\toprule\noalign{}
પરિબળ & અવરોધ પર અસર & સંબંધ \\
\midrule\noalign{}
\endhead
\bottomrule\noalign{}
\endlastfoot
\textbf{લંબાઈ (l)} & સીધો સંબંધ & R ∝ l \\
\textbf{ક્રોસ-સેક્શનલ એરિયા (A)} & વ્યસ્ત સંબંધ & R ∝ 1/A \\
\textbf{મટિરિયલ (ρ)} & રેઝિસ્ટિવિટી પર આધાર રાખે છે & R ∝ ρ \\
\textbf{તાપમાન (T)} & સામાન્ય રીતે તાપમાન સાથે વધે છે & R ∝ T \\
\end{longtable}
}

\end{solutionbox}
\begin{mnemonicbox}
``LAMT: Length Adds, Area Minimizes, Material
matters, Temperature transforms''

\end{mnemonicbox}
\subsection*{પ્રશ્ન ૨(બ) [૪
ગુણ]}\label{uxaaauxab0uxab6uxaa8-uxae8uxaac-uxaea-uxa97uxaa3}

\textbf{પાવર ત્રિકોણ દોરી એક્ટિવ અને રીઍક્ટિવ પાવરની વ્યાખ્યા આપો.}

\begin{solutionbox}

\textbf{પાવર ત્રિકોણ:}

\begin{center}
\textbf{Mermaid Diagram (Code)}
\begin{verbatim}
{Shaded}
{Highlighting}[]
graph LR;
    A((O)){-{-}{}B((P));}
    A{-{-}{}C((S));}
    A{-{-}{}D((Q));}
    linkStyle 0 stroke:green,stroke{-width:2px;}
    linkStyle 1 stroke:red,stroke{-width:2px;}
    linkStyle 2 stroke:blue,stroke{-width:2px;}
{Highlighting}
{Shaded}
\end{verbatim}
\end{center}

{\def\LTcaptype{none} % do not increment counter
\begin{longtable}[]{@{}
  >{\raggedright\arraybackslash}p{(\linewidth - 6\tabcolsep) * \real{0.3077}}
  >{\raggedright\arraybackslash}p{(\linewidth - 6\tabcolsep) * \real{0.3077}}
  >{\raggedright\arraybackslash}p{(\linewidth - 6\tabcolsep) * \real{0.1538}}
  >{\raggedright\arraybackslash}p{(\linewidth - 6\tabcolsep) * \real{0.2308}}@{}}
\toprule\noalign{}
\begin{minipage}[b]{\linewidth}\raggedright
પાવરનો પ્રકાર
\end{minipage} & \begin{minipage}[b]{\linewidth}\raggedright
વ્યાખ્યા
\end{minipage} & \begin{minipage}[b]{\linewidth}\raggedright
એકમ
\end{minipage} & \begin{minipage}[b]{\linewidth}\raggedright
ફોર્મ્યુલા
\end{minipage} \\
\midrule\noalign{}
\endhead
\bottomrule\noalign{}
\endlastfoot
\textbf{એક્ટિવ પાવર (P)} & ઉપકરણ દ્વારા વપરાતી વાસ્તવિક પાવર & વોટ (W) & P
= VI cos φ \\
\textbf{રીઍક્ટિવ પાવર (Q)} & સ્ત્રોત અને લોડ વચ્ચે આંદોલિત થતી પાવર & VAR & Q =
VI sin φ \\
\textbf{એપેરન્ટ પાવર (S)} & એક્ટિવ અને રીઍક્ટિવ પાવરનો વેક્ટર સરવાળો & VA & S =
VI \\
\end{longtable}
}

\end{solutionbox}
\begin{mnemonicbox}
``PAWS: Power Active Works, Apparent is
Slant-hypotenuse, reactive Qoscillates''

\end{mnemonicbox}
\subsection*{પ્રશ્ન ૨(ક) [૭
ગુણ]}\label{uxaaauxab0uxab6uxaa8-uxae8uxa95-uxaed-uxa97uxaa3}

\textbf{સેલ અને બેટરી સમજાવો. વિવિધ રેટિંગ અને બેટરીના પ્રકારોની યાદી બનાવો.}

\begin{solutionbox}

\textbf{સેલ અને બેટરી:}

{\def\LTcaptype{none} % do not increment counter
\begin{longtable}[]{@{}
  >{\raggedright\arraybackslash}p{(\linewidth - 2\tabcolsep) * \real{0.3333}}
  >{\raggedright\arraybackslash}p{(\linewidth - 2\tabcolsep) * \real{0.6667}}@{}}
\toprule\noalign{}
\begin{minipage}[b]{\linewidth}\raggedright
શબ્દ
\end{minipage} & \begin{minipage}[b]{\linewidth}\raggedright
વ્યાખ્યા
\end{minipage} \\
\midrule\noalign{}
\endhead
\bottomrule\noalign{}
\endlastfoot
\textbf{સેલ} & મૂળભૂત ઇલેક્ટ્રોકેમિકલ એકમ જે રાસાયણિક ઊર્જાને વિદ્યુત ઊર્જામાં રૂપાંતરિત
કરે છે \\
\textbf{બેટરી} & શ્રેણી અથવા સમાંતરમાં જોડાયેલા એક કે વધુ સેલનો સમૂહ \\
\end{longtable}
}

\textbf{બેટરી રેટિંગ:}

{\def\LTcaptype{none} % do not increment counter
\begin{longtable}[]{@{}lll@{}}
\toprule\noalign{}
રેટિંગ & વર્ણન & એકમ \\
\midrule\noalign{}
\endhead
\bottomrule\noalign{}
\endlastfoot
\textbf{વોલ્ટેજ} & પોટેન્શિયલ ડિફરન્સ & વોલ્ટ (V) \\
\textbf{કેપેસિટી} & સંગ્રહિત ચાર્જની માત્રા & એમ્પિયર-કલાક (Ah) \\
\textbf{ઊર્જા} & કુલ ઉપલબ્ધ ઊર્જા & વોટ-કલાક (Wh) \\
\textbf{C-રેટ} & ડિસ્ચાર્જ/ચાર્જ દર & C \\
\textbf{સાયકલ લાઇફ} & ચાર્જ/ડિસ્ચાર્જ સાયકલની સંખ્યા & - \\
\end{longtable}
}

\textbf{બેટરીના પ્રકારો:}

\begin{center}
\textbf{Mermaid Diagram (Code)}
\begin{verbatim}
{Shaded}
{Highlighting}[]
graph TD;
    A[Battery Types]{-{-}{}B[Primary];}
    A{-{-}{}C[Secondary];}
    B{-{-}{}D[Alkaline];}
    B{-{-}{}E[Zinc{-}Carbon];}
    B{-{-}{}F[Lithium];}
    C{-{-}{}G[Lead{-}Acid];}
    C{-{-}{}H[Li{-}ion];}
    C{-{-}{}I[Ni{-}MH];}
{Highlighting}
{Shaded}
\end{verbatim}
\end{center}

\end{solutionbox}
\begin{mnemonicbox}
``CAVE: Cells Are Voltage Elements, batteries Bundle
And TallY Energy''

\end{mnemonicbox}
\subsection*{પ્રશ્ન ૨(અ) OR [૩
ગુણ]}\label{uxaaauxab0uxab6uxaa8-uxae8uxa85-or-uxae9-uxa97uxaa3}

\textbf{અવરોધ, વહન અને વાહકતાની વ્યાખ્યા આપો.}

\begin{solutionbox}

{\def\LTcaptype{none} % do not increment counter
\begin{longtable}[]{@{}
  >{\raggedright\arraybackslash}p{(\linewidth - 6\tabcolsep) * \real{0.1818}}
  >{\raggedright\arraybackslash}p{(\linewidth - 6\tabcolsep) * \real{0.3636}}
  >{\raggedright\arraybackslash}p{(\linewidth - 6\tabcolsep) * \real{0.1818}}
  >{\raggedright\arraybackslash}p{(\linewidth - 6\tabcolsep) * \real{0.2727}}@{}}
\toprule\noalign{}
\begin{minipage}[b]{\linewidth}\raggedright
શબ્દ
\end{minipage} & \begin{minipage}[b]{\linewidth}\raggedright
વ્યાખ્યા
\end{minipage} & \begin{minipage}[b]{\linewidth}\raggedright
એકમ
\end{minipage} & \begin{minipage}[b]{\linewidth}\raggedright
ફોર્મ્યુલા
\end{minipage} \\
\midrule\noalign{}
\endhead
\bottomrule\noalign{}
\endlastfoot
\textbf{અવરોધ (R)} & વિદ્યુત પ્રવાહનો વિરોધ & ઓહ્મ (Ω) & R = ρl/A \\
\textbf{વહન (G)} & વિદ્યુત પ્રવાહની સરળતા & સિમેન્સ (S) & G = 1/R \\
\textbf{વાહકતા (σ)} & વિદ્યુત પ્રવાહને પસાર કરવાની મટિરિયલની ક્ષમતા & S/m & σ
= 1/ρ \\
\end{longtable}
}

જ્યાં ρ રેઝિસ્ટિવિટી, l લંબાઈ, અને A ક્રોસ-સેક્શનલ એરિયા છે

\end{solutionbox}
\begin{mnemonicbox}
``RCG: Resist Current Gladly, Conduct Generously, σ
Gets current through''

\end{mnemonicbox}
\subsection*{પ્રશ્ન ૨(બ) OR [૪
ગુણ]}\label{uxaaauxab0uxab6uxaa8-uxae8uxaac-or-uxaea-uxa97uxaa3}

\textbf{શુદ્ધ ઈંડક્ટિવ સર્કિટ માટે સાબિત કરો કે કરંટ એ વોલ્ટેજ કરતા 90^\circ પાછળ હોય
છે.}

\begin{solutionbox}

\textbf{શુદ્ધ ઈંડક્ટિવ સર્કિટ માટે:}

\begin{center}
\textbf{Mermaid Diagram (Code)}
\begin{verbatim}
{Shaded}
{Highlighting}[]
graph LR;
    A((AC Source)){-{-}{}B((L))}
{Highlighting}
{Shaded}
\end{verbatim}
\end{center}

\textbf{ગાણિતિક સાબિતી:}

\begin{itemize}
\tightlist
\item
  આપેલ વોલ્ટેજ: v = Vm sin(ωt)
\item
  ઇન્ડક્ટર માટે: v = L(di/dt)
\item
  આથી: L(di/dt) = Vm sin(ωt)
\item
ઇન્ટિગ્રેટ કરતાં:

i = -(Vm/ωL)cos(ωt) = (Vm/ωL)sin(ωt-90^\circ)

\end{itemize}

\textbf{વેવફોર્મ:}

\begin{verbatim}
    v    i
    |    |
    |{  /|}
    | {/ |}
    | /{ |}
    |/  {|}
    |    |
    |    |
    |    |
    +{-{-}{-}{-}+}
       t
\end{verbatim}

\end{solutionbox}
\begin{mnemonicbox}
``ELI: Voltage Leads current In inductor by 90
degrees''

\end{mnemonicbox}
\subsection*{પ્રશ્ન ૨(ક) OR [૭
ગુણ]}\label{uxaaauxab0uxab6uxaa8-uxae8uxa95-or-uxaed-uxa97uxaa3}

\textbf{અવરોધ, ઈંડક્ટર અને કેપેસીટર તેમના સૂત્ર સાથે સમજાવો.}

\begin{solutionbox}

{\def\LTcaptype{none} % do not increment counter
\begin{longtable}[]{@{}
  >{\raggedright\arraybackslash}p{(\linewidth - 8\tabcolsep) * \real{0.1964}}
  >{\raggedright\arraybackslash}p{(\linewidth - 8\tabcolsep) * \real{0.1429}}
  >{\raggedright\arraybackslash}p{(\linewidth - 8\tabcolsep) * \real{0.2321}}
  >{\raggedright\arraybackslash}p{(\linewidth - 8\tabcolsep) * \real{0.1607}}
  >{\raggedright\arraybackslash}p{(\linewidth - 8\tabcolsep) * \real{0.2679}}@{}}
\toprule\noalign{}
\begin{minipage}[b]{\linewidth}\raggedright
ઘટક
\end{minipage} & \begin{minipage}[b]{\linewidth}\raggedright
સિમ્બોલ
\end{minipage} & \begin{minipage}[b]{\linewidth}\raggedright
વર્ણન
\end{minipage} & \begin{minipage}[b]{\linewidth}\raggedright
ફોર્મ્યુલા
\end{minipage} & \begin{minipage}[b]{\linewidth}\raggedright
ઊર્જા સંગ્રહ
\end{minipage} \\
\midrule\noalign{}
\endhead
\bottomrule\noalign{}
\endlastfoot
\textbf{અવરોધ} &
\texttt{mermaid\ graph\ LR;\ A-\/-\/-B[(\_\_\_/\textbackslash{}/\textbackslash{}/\textbackslash{}\_\_\_)]-\/-\/-C}
& કરંટ પ્રવાહનો વિરોધ કરે છે & V = IR & સંગ્રહ નથી \\
\textbf{ઈંડક્ટર} &
\texttt{mermaid\ graph\ LR;\ A-\/-\/-B[(\_mmmmm\_)]-\/-\/-C} & કરંટમાં
ફેરફારનો વિરોધ કરે છે &

V = L(di/dt) &

E = ½LI^{2} \\

\textbf{કેપેસીટર} &
\texttt{mermaid\ graph\ LR;\ A-\/-\/-B[(\_⎥⎥\_)]-\/-\/-C} & વોલ્ટેજમાં
ફેરફારનો વિરોધ કરે છે &

I = C(dv/dt) &

E = ½CV^{2} \\

\end{longtable}
}

\textbf{AC સર્કિટ પર અસર:}

\begin{itemize}
\tightlist
\item
  \textbf{અવરોધ}: કરંટ વોલ્ટેજ સાથે એક ફેઝમાં (cos θ = 1)
\item
  \textbf{ઈંડક્ટર}: કરંટ વોલ્ટેજથી 90^\circ પાછળ (cos θ = 0)
\item
  \textbf{કેપેસીટર}: કરંટ વોલ્ટેજથી 90^\circ આગળ (cos θ = 0)
\end{itemize}

\end{solutionbox}
\begin{mnemonicbox}
``RIC: Resistor Impedes Current, Inductor Catches
current-changes, Capacitor Controls voltage-changes''

\end{mnemonicbox}
\subsection*{પ્રશ્ન ૩(અ) [૩
ગુણ]}\label{uxaaauxab0uxab6uxaa8-uxae9uxa85-uxae9-uxa97uxaa3}

\textbf{A.C. સિગ્નલની R.M.S અને એવરેજ મૂલ્યની વ્યાખ્યા આપો અને સમજાવો.}

\begin{solutionbox}

{\def\LTcaptype{none} % do not increment counter
\begin{longtable}[]{@{}
  >{\raggedright\arraybackslash}p{(\linewidth - 6\tabcolsep) * \real{0.1373}}
  >{\raggedright\arraybackslash}p{(\linewidth - 6\tabcolsep) * \real{0.2353}}
  >{\raggedright\arraybackslash}p{(\linewidth - 6\tabcolsep) * \real{0.4314}}
  >{\raggedright\arraybackslash}p{(\linewidth - 6\tabcolsep) * \real{0.1961}}@{}}
\toprule\noalign{}
\begin{minipage}[b]{\linewidth}\raggedright
મૂલ્ય
\end{minipage} & \begin{minipage}[b]{\linewidth}\raggedright
વ્યાખ્યા
\end{minipage} & \begin{minipage}[b]{\linewidth}\raggedright
સાઇન વેવ માટે ફોર્મ્યુલા
\end{minipage} & \begin{minipage}[b]{\linewidth}\raggedright
સંબંધ
\end{minipage} \\
\midrule\noalign{}
\endhead
\bottomrule\noalign{}
\endlastfoot
\textbf{RMS મૂલ્ય} & સ્ક્વેર કરેલા મૂલ્યોના મીનનો સ્ક્વેર રૂટ & Vrms = Vmax/\sqrt2 =
0.707Vmax & DC સમાન હીટિંગ ઇફેક્ટ આપે છે \\
\textbf{એવરેજ મૂલ્ય} & અર્ધ સાયકલ પર રેક્ટિફાઇડ સિગ્નલનું મીન & Vavg = 2Vmax/π =
0.637Vmax & બેટરી ચાર્જિંગ એપ્લિકેશન માટે ઉપયોગી \\
\end{longtable}
}

\end{solutionbox}
\begin{mnemonicbox}
``RAM: Rms-Average Method: Root-mean-square And
Mean-of-absolute''

\end{mnemonicbox}
\subsection*{પ્રશ્ન ૩(બ) [૪
ગુણ]}\label{uxaaauxab0uxab6uxaa8-uxae9uxaac-uxaea-uxa97uxaa3}

\textbf{વૈકલ્પિક EMF કેવી રીતે ઉત્પન્ન થાય છે તે જરૂરી આકૃતિ સાથે સમજાવો.}

\begin{solutionbox}

\textbf{વૈકલ્પિક EMF ઉત્પાદન:}

\begin{center}
\textbf{Mermaid Diagram (Code)}
\begin{verbatim}
{Shaded}
{Highlighting}[]
graph LR;
    A[Rotating Coil]{-{-}{}B[Magnetic Field];}
    B{-{-}{}C[EMF Induced];}
    C{-{-}{}D[Direction Changes];}
    D{-{-}{}E[AC Waveform];}
{Highlighting}
{Shaded}
\end{verbatim}
\end{center}

\textbf{આકૃતિ:}

\begin{verbatim}
    N       S
    |       |
    +{-{-}{-}{-}{-}{-}{-}+}
      |   |
      |   |
      |\_\_\_|
       { /}
        |
        v
\end{verbatim}

\begin{itemize}
\tightlist
\item
  કોઈલ સમાન ચુંબકીય ક્ષેત્રમાં ફરે છે
\item
  EMF = NBAlω sin(ωt)
\item
  કોઈલ ફરતી વખતે, ફ્લક્સ કટિંગની દિશા બદલાય છે
\item
  સાઇન વેવ ઉત્પન્ન થાય છે e = Emax sin(ωt)
\end{itemize}

\end{solutionbox}
\begin{mnemonicbox}
``FARM: Flux And Rotation Make alternating voltage''

\end{mnemonicbox}
\subsection*{પ્રશ્ન ૩(ક) [૭
ગુણ]}\label{uxaaauxab0uxab6uxaa8-uxae9uxa95-uxaed-uxa97uxaa3}

\textbf{શુધ્ધ આવરોધીય AC સરકીટનું એસી એનાલિસિસ કરો.}

\begin{solutionbox}

\textbf{શુદ્ધ અવરોધિય સર્કિટ:}

\begin{center}
\textbf{Mermaid Diagram (Code)}
\begin{verbatim}
{Shaded}
{Highlighting}[]
graph LR;
    A(({)){-}{-}{}B[(R)]{-}{-}{}C}
{Highlighting}
{Shaded}
\end{verbatim}
\end{center}

{\def\LTcaptype{none} % do not increment counter
\begin{longtable}[]{@{}lll@{}}
\toprule\noalign{}
પેરામીટર & ફોર્મ્યુલા & વેવફોર્મ સંબંધ \\
\midrule\noalign{}
\endhead
\bottomrule\noalign{}
\endlastfoot
\textbf{આપેલ વોલ્ટેજ} & v = Vm sin(ωt) & કરંટ અને વોલ્ટેજ એક ફેઝમાં \\
\textbf{કરંટ} & i = v/R = (Vm/R)sin(ωt) & ઓહ્મના નિયમનું પાલન કરે છે \\
\textbf{પાવર} & p = vi = Vm Im sin^{2}(ωt) & હંમેશા સકારાત્મક \\
\textbf{એવરેજ પાવર} & P = Vrms \times Irms = V^{2}/R & સ્થિર મૂલ્ય \\
\end{longtable}
}

\textbf{વેવફોર્મ:}

\begin{verbatim}
    v,i  p
    |    |
    |{  /"}
    | {/ | }
    | /{ | /}
    |/  {|/}
    |    |
    |    |
    |    |
    +{-{-}{-}{-}+}
       t
\end{verbatim}

\end{solutionbox}
\begin{mnemonicbox}
``VIPS: Voltage In-Phase with current, Same
waveform, Power always Positive''

\end{mnemonicbox}
\subsection*{પ્રશ્ન ૩(અ) OR [૩
ગુણ]}\label{uxaaauxab0uxab6uxaa8-uxae9uxa85-or-uxae9-uxa97uxaa3}

\textbf{એસી વિદ્યુતપ્રવાહ I=28.28sin(2Π50t). વિદ્યુત પ્રવાહનું RMS મૂલ્ય શોધો.}

\begin{solutionbox}

\textbf{આપેલુ:}

\begin{itemize}
\tightlist
\item
  I = 28.28sin(2Π50t)
\item
  તેથી, Im = 28.28A
\end{itemize}

\textbf{ઉકેલ:}

{\def\LTcaptype{none} % do not increment counter
\begin{longtable}[]{@{}ll@{}}
\toprule\noalign{}
સ્ટેપ & કેલ્ક્યુલેશન \\
\midrule\noalign{}
\endhead
\bottomrule\noalign{}
\endlastfoot
1. પીક વેલ્યૂ ઓળખો & Im = 28.28A \\
2. RMS ફોર્મ્યુલા લાગુ કરો & Irms = Im/\sqrt2 \\
3. ગણતરી કરો & Irms = 28.28/\sqrt2 = 28.28/1.414 = 20A \\
\end{longtable}
}

\textbf{આથી, કરંટની RMS મૂલ્ય = 20A}

\end{solutionbox}
\begin{mnemonicbox}
``PER: Peak to Effective by Root-2''

\end{mnemonicbox}
\subsection*{પ્રશ્ન ૩(બ) OR [૪
ગુણ]}\label{uxaaauxab0uxab6uxaa8-uxae9uxaac-or-uxaea-uxa97uxaa3}

\textbf{જો Vav=60 V હોય તો વૉલ્ટેજનું RMS અને મહત્તમ મૂલ્ય શોધો.}

\begin{solutionbox}

\textbf{આપેલુ:}

\begin{itemize}
\tightlist
\item
  એવરેજ મૂલ્ય (Vav) = 60V
\end{itemize}

\textbf{ઉકેલ:}

{\def\LTcaptype{none} % do not increment counter
\begin{longtable}[]{@{}
  >{\raggedright\arraybackslash}p{(\linewidth - 4\tabcolsep) * \real{0.2143}}
  >{\raggedright\arraybackslash}p{(\linewidth - 4\tabcolsep) * \real{0.3214}}
  >{\raggedright\arraybackslash}p{(\linewidth - 4\tabcolsep) * \real{0.4643}}@{}}
\toprule\noalign{}
\begin{minipage}[b]{\linewidth}\raggedright
સ્ટેપ
\end{minipage} & \begin{minipage}[b]{\linewidth}\raggedright
ફોર્મ્યુલા
\end{minipage} & \begin{minipage}[b]{\linewidth}\raggedright
કેલ્ક્યુલેશન
\end{minipage} \\
\midrule\noalign{}
\endhead
\bottomrule\noalign{}
\endlastfoot
1. Vav અને Vm વચ્ચેનો સંબંધ & Vav = 2Vm/π = 0.637Vm & Vm = Vav/0.637 =
60/0.637 \\
2. મહત્તમ મૂલ્ય ગણો & Vm = Vav \times (π/2) & Vm = 60 \times (π/2) = 60 \times 1.57 =
94.2V \\
3. RMS મૂલ્ય ગણો & Vrms = Vm/\sqrt2 = 0.707Vm & Vrms = 0.707 \times 94.2 = 66.6V \\
\end{longtable}
}

\textbf{આથી, મહત્તમ મૂલ્ય = 94.2V અને RMS મૂલ્ય = 66.6V}

\end{solutionbox}
\begin{mnemonicbox}
``AVR: Average to peak Via multiplying by (π/2), Rms
is peak/\sqrt2''

\end{mnemonicbox}
\subsection*{પ્રશ્ન ૩(ક) OR [૭
ગુણ]}\label{uxaaauxab0uxab6uxaa8-uxae9uxa95-or-uxaed-uxa97uxaa3}

\textbf{ફેઈઝ ડાયાગ્રામની મદદથી સ્ટાર જોડાણનું લાઈન અને ફેઈસ વૉલ્ટેજનું સમીકરણ
તારવો.}

\begin{solutionbox}

\textbf{સ્ટાર જોડાણ:}

\begin{center}
\textbf{Mermaid Diagram (Code)}
\begin{verbatim}
{Shaded}
{Highlighting}[]
graph LR;
    A((R)){-{-}{}N((N));}
    B((Y)){-{-}{}N;}
    C((B)){-{-}{}N;}
    R[Load]{-{-}{}A;}
    Y[Load]{-{-}{}B;}
    B[Load]{-{-}{}C;}
{Highlighting}
{Shaded}
\end{verbatim}
\end{center}

\textbf{ફેઝર ડાયાગ્રામ:}

\begin{verbatim}
     VRY
      \^{}
     /|{}
    / | {}
   /  |  {}
  /   |   {}
VRB   |    VYB
\end{verbatim}

\textbf{ડેરિવેશન:}

\begin{itemize}
\tightlist
\item
  ફેઝ વોલ્ટેજ: VRN, VYN, VBN (120^\circ અલગ)
\item
  લાઈન વોલ્ટેજ: VRY = VRN - VYN
\item
  બેલેન્સ સિસ્ટમ માટે ફેઝ વોલ્ટેજનું મેગ્નિટ્યૂડ Vp સાથે:
\item
  VRY = VRN - VYN = Vp∠0^\circ - Vp∠-120^\circ = Vp(1 - ∠-120^\circ) = \sqrt3Vp∠30^\circ
\end{itemize}

\textbf{સંબંધ:}

\begin{itemize}
\tightlist
\item
  લાઈન વોલ્ટેજ (VL) = \sqrt3 \times ફેઝ વોલ્ટેજ (Vp)
\item
  લાઈન વોલ્ટેજ ફેઝ વોલ્ટેજથી 30^\circ આગળ રહે છે
\end{itemize}

\end{solutionbox}
\begin{mnemonicbox}
``PALS: Phase to Line in Star: multiply by
Square-root-3''

\end{mnemonicbox}
\subsection*{પ્રશ્ન ૪(અ) [૩
ગુણ]}\label{uxaaauxab0uxab6uxaa8-uxaeauxa85-uxae9-uxa97uxaa3}

\textbf{Faraday અને Lenzનો નિયમ તેના સૂત્ર સાથે લખો.}

\begin{solutionbox}

{\def\LTcaptype{none} % do not increment counter
\begin{longtable}[]{@{}
  >{\raggedright\arraybackslash}p{(\linewidth - 4\tabcolsep) * \real{0.1786}}
  >{\raggedright\arraybackslash}p{(\linewidth - 4\tabcolsep) * \real{0.3929}}
  >{\raggedright\arraybackslash}p{(\linewidth - 4\tabcolsep) * \real{0.4286}}@{}}
\toprule\noalign{}
\begin{minipage}[b]{\linewidth}\raggedright
નિયમ
\end{minipage} & \begin{minipage}[b]{\linewidth}\raggedright
વિધાન
\end{minipage} & \begin{minipage}[b]{\linewidth}\raggedright
સમીકરણ
\end{minipage} \\
\midrule\noalign{}
\endhead
\bottomrule\noalign{}
\endlastfoot
\textbf{ફેરાડેનો નિયમ} & પ્રેરિત EMF ચુંબકીય ફ્લક્સના પરિવર્તનના દરના સમપ્રમાણમાં
હોય છે & e = -N(dΦ/dt) \\
\textbf{લેન્ઝનો નિયમ} & પ્રેરિત EMF તેને ઉત્પન્ન કરતા કારણનો વિરોધ કરે છે (સૂત્રમાં
નેગેટિવ સાઇન) & ફ્લક્સ પરિવર્તનની વિરુદ્ધ દિશા \\
\end{longtable}
}

\end{solutionbox}
\begin{mnemonicbox}
``FORC: Faraday's flux Over Rate Change, Lenz
Opposes the Reason for Change''

\end{mnemonicbox}
\subsection*{પ્રશ્ન ૪(બ) [૪
ગુણ]}\label{uxaaauxab0uxab6uxaa8-uxaeauxaac-uxaea-uxa97uxaa3}

\textbf{સિંગલ ફેઈસ સપ્લાયની સરખામણીમાં 3-ફેઈસ સપ્લાયના 4 ફાયદા લખો.}

\begin{solutionbox}

{\def\LTcaptype{none} % do not increment counter
\begin{longtable}[]{@{}ll@{}}
\toprule\noalign{}
3-ફેઈસ સપ્લાયના સિંગલ-ફેઈસ કરતાં ફાયદા & સમજૂતી \\
\midrule\noalign{}
\endhead
\bottomrule\noalign{}
\endlastfoot
\textbf{ઉચ્ચ પાવર ઘનત્વ} & 3-ફેઈસ સમાન વાયર સાઈઝ સાથે 1.732 ગણો વધુ પાવર આપે
છે \\
\textbf{સ્થિર પાવર ડિલિવરી} & સિંગલ-ફેઈસની જેમ પાવરમાં ઉછાળા નહીં \\
\textbf{નાના કન્ડક્ટર} & સમાન પાવર ટ્રાન્સફર માટે ઓછા કોપરની જરૂર \\
\textbf{સેલ્ફ-સ્ટાર્ટિંગ મોટર} & મોટર માટે કોઈ સ્ટાર્ટિંગ મેકેનિઝમની જરૂર નથી \\
\end{longtable}
}

\textbf{વધારાના: વધુ કાર્યક્ષમ ટ્રાન્સમિશન, ઓછા હાર્મોનિક્સ, બેલેન્સ્ડ લોડિંગ}

\end{solutionbox}
\begin{mnemonicbox}
``PCCS: Power higher, Constant delivery, Copper
less, Self-starting motors''

\end{mnemonicbox}
\subsection*{પ્રશ્ન ૪(ક) [૭
ગુણ]}\label{uxaaauxab0uxab6uxaa8-uxaeauxa95-uxaed-uxa97uxaa3}

\textbf{Flemingનો જમણા હાથનો અને ડાબા હાથનો નિયમ સમજાવો.}

\begin{solutionbox}

\textbf{ફ્લેમિંગના હાથના નિયમો:}

{\def\LTcaptype{none} % do not increment counter
\begin{longtable}[]{@{}
  >{\raggedright\arraybackslash}p{(\linewidth - 6\tabcolsep) * \real{0.1429}}
  >{\raggedright\arraybackslash}p{(\linewidth - 6\tabcolsep) * \real{0.3095}}
  >{\raggedright\arraybackslash}p{(\linewidth - 6\tabcolsep) * \real{0.3333}}
  >{\raggedright\arraybackslash}p{(\linewidth - 6\tabcolsep) * \real{0.2143}}@{}}
\toprule\noalign{}
\begin{minipage}[b]{\linewidth}\raggedright
નિયમ
\end{minipage} & \begin{minipage}[b]{\linewidth}\raggedright
ઉપયોગ
\end{minipage} & \begin{minipage}[b]{\linewidth}\raggedright
હાથની સ્થિતિ
\end{minipage} & \begin{minipage}[b]{\linewidth}\raggedright
આકૃતિ
\end{minipage} \\
\midrule\noalign{}
\endhead
\bottomrule\noalign{}
\endlastfoot
\textbf{જમણા હાથનો નિયમ (જનરેટર)} & પ્રેરિત EMFની દિશા નક્કી કરે છે &
\textbf{અંગૂઠો}: ગતિ\textbf{તર્જની}: ક્ષેત્ર\textbf{મધ્યમા}: કરંટ/EMF &
```goat \\
\end{longtable}
}

\begin{verbatim}
F ^
  |
--+-- > M
  |
  v
  C
``` |
\end{verbatim}

\textbf{ડાબા હાથનો નિયમ (મોટર)} \textbar{} ગતિ/બળની દિશા નક્કી કરે છે
\textbar{} \textbf{અંગૂઠો}: ગતિ/બળ\textbf{તર્જની}: ક્ષેત્ર\textbf{મધ્યમા}: કરંટ
\textbar{}
\texttt{goat\ \ \ \ F\ \^{}\ \ \ \ \ \ \textbar{}\ \ \ \ -\/-+-\/-\ \textgreater{}\ M\ \ \ \ \ \ \textbar{}\ \ \ \ \ \ v\ \ \ \ \ \ C}
\textbar{}

\begin{itemize}
\tightlist
\item
  \textbf{જનરેટર}: યાંત્રિક ઊર્જાનું વિદ્યુત ઊર્જામાં રૂપાંતર
\item
  \textbf{મોટર}: વિદ્યુત ઊર્જાનું યાંત્રિક ઊર્જામાં રૂપાંતર
\end{itemize}

\end{solutionbox}
\begin{mnemonicbox}
``FBI-MFC: Field-B-Induced current for right hand,
Motion-Field-Current for left''

\end{mnemonicbox}
\subsection*{પ્રશ્ન ૪(અ) OR [૩
ગુણ]}\label{uxaaauxab0uxab6uxaa8-uxaeauxa85-or-uxae9-uxa97uxaa3}

\textbf{ઈલેક્ટ્રોમેગ્નેટિક ઈન્ડક્સનની ઘટના સમજાવો.}

\begin{solutionbox}

\textbf{ઈલેક્ટ્રોમેગ્નેટિક ઈન્ડક્શન:}

\begin{center}
\textbf{Mermaid Diagram (Code)}
\begin{verbatim}
{Shaded}
{Highlighting}[]
graph LR;
    A[Changing Magnetic Field/Flux]{-{-}{}B[Induces EMF in Conductor];}
    B{-{-}{}C[Causes Current to Flow];}
    C{-{-}{}D[Creates Secondary Magnetic Field];}
{Highlighting}
{Shaded}
\end{verbatim}
\end{center}

\textbf{મુખ્ય પરિબળો:}

\begin{itemize}
\tightlist
\item
  સાપેક્ષ ગતિ અથવા ફ્લક્સમાં ફેરફારની જરૂર
\item
  EMF ફ્લક્સના ફેરફારના દરના સમપ્રમાણમાં
\item
  દિશા લેન્ઝના નિયમ દ્વારા નિર્ધારિત
\end{itemize}

\end{solutionbox}
\begin{mnemonicbox}
``MICE: Motion Induces Current via Electromagnetic
induction''

\end{mnemonicbox}
\subsection*{પ્રશ્ન ૪(બ) OR [૪
ગુણ]}\label{uxaaauxab0uxab6uxaa8-uxaeauxaac-or-uxaea-uxa97uxaa3}

\textbf{3-ફેઈસ વૈકલ્પિક ઈ. એમ. એફ. કેવી રીતે ઉત્પન થાય છે સમજાવો.}

\begin{solutionbox}

\textbf{3-ફેઈસ EMF ઉત્પાદન:}

\begin{center}
\textbf{Mermaid Diagram (Code)}
\begin{verbatim}
{Shaded}
{Highlighting}[]
graph LR;
    A[3 Coils at 120^ Apart]{-{-}{}B[Rotating Magnetic Field];}
    B{-{-}{}C[3 EMFs Generated at 120^ Phase Difference];}
    C{-{-}{}D[Balanced 3{-}Phase Supply];}
{Highlighting}
{Shaded}
\end{verbatim}
\end{center}

\textbf{થ્રી ફેઈસ વેવફોર્મ:}

\begin{verbatim}
    +       R
    |   /{    /}
    |  /  {  /  }
    | /    {/    }
{-{-}{-}{-}+{-}{-}{-}{-}{-}{-}{-}{-}{-}{-}{-}{-}{-}{-}{-}}
    |{    /    /}
    | {  /    /}
    |  {/    /  }
    +       Y
    |   /{    /}
    |  /  {  /  }
    | /    {/    }
{-{-}{-}{-}+{-}{-}{-}{-}{-}{-}{-}{-}{-}{-}{-}{-}{-}{-}{-}}
    |{    /    /}
    | {  /    /}
    |  {/    /}
    +       B
\end{verbatim}

\begin{itemize}
\tightlist
\item
  ત્રણ સમાન કોઈલ્સ 120^\circ અંતરે ગોઠવાયેલી
\item
  ત્રણ સમાન EMF ઉત્પન્ન કરે છે જે સમયમાં 120^\circ અંતરે હોય છે
\item
  EMFs: eR = Emax sin(ωt), eY = Emax sin(ωt-120^\circ), eB = Emax
  sin(ωt-240^\circ)
\end{itemize}

\end{solutionbox}
\begin{mnemonicbox}
``CPS: Coils Produce Shifted waveforms at 120
degrees''

\end{mnemonicbox}
\subsection*{પ્રશ્ન ૪(ક) OR [૭
ગુણ]}\label{uxaaauxab0uxab6uxaa8-uxaeauxa95-or-uxaed-uxa97uxaa3}

\textbf{Statically induced E.M.F અને dynamically induced E.M.F વચ્ચેનો
તફાવત લખો.}

\begin{solutionbox}

{\def\LTcaptype{none} % do not increment counter
\begin{longtable}[]{@{}
  >{\raggedright\arraybackslash}p{(\linewidth - 4\tabcolsep) * \real{0.1833}}
  >{\raggedright\arraybackslash}p{(\linewidth - 4\tabcolsep) * \real{0.4000}}
  >{\raggedright\arraybackslash}p{(\linewidth - 4\tabcolsep) * \real{0.4167}}@{}}
\toprule\noalign{}
\begin{minipage}[b]{\linewidth}\raggedright
પેરામીટર
\end{minipage} & \begin{minipage}[b]{\linewidth}\raggedright
સ્ટેટિકલી ઈન્ડ્યુસ્ડ EMF
\end{minipage} & \begin{minipage}[b]{\linewidth}\raggedright
ડાયનેમિકલી ઈન્ડ્યુસ્ડ EMF
\end{minipage} \\
\midrule\noalign{}
\endhead
\bottomrule\noalign{}
\endlastfoot
\textbf{વ્યાખ્યા} & સ્થિર વાહકમાં ફ્લક્સના ફેરફારને કારણે પ્રેરિત EMF & ચુંબકીય
ક્ષેત્રમાં વાહકની ગતિને કારણે પ્રેરિત EMF \\
\textbf{ગતિ} & વાહક અને ક્ષેત્ર વચ્ચે કોઈ સાપેક્ષ ગતિ નહીં & સાપેક્ષ ગતિ હાજર \\
\textbf{ફેરફારનો સ્ત્રોત} & પ્રાથમિક સર્કિટમાં કરંટમાં ફેરફાર & વાહકની ભૌતિક
ગતિ \\
\textbf{ઉદાહરણો} & ટ્રાન્સફોર્મર, ઈન્ડક્ટર & જનરેટર, આલ્ટરનેટર \\
\textbf{ગાણિતિક સમીકરણ} & e = -N(dΦ/dt) કરંટમાં ફેરફારને કારણે & e = Blv
(B=ફ્લક્સ ઘનતા,

l=લંબાઈ,

v=વેગ) \\

\end{longtable}
}

\end{solutionbox}
\begin{mnemonicbox}
``SMCE: Static-Moving, Change-External: static has
changing flux, moving has constant flux''

\end{mnemonicbox}
\subsection*{પ્રશ્ન ૫(અ) [૩
ગુણ]}\label{uxaaauxab0uxab6uxaa8-uxaebuxa85-uxae9-uxa97uxaa3}

\textbf{HAWT અને VAWT વચ્ચેનો તફાવત લખો.}

\begin{solutionbox}

{\def\LTcaptype{none} % do not increment counter
\begin{longtable}[]{@{}
  >{\raggedright\arraybackslash}p{(\linewidth - 4\tabcolsep) * \real{0.1325}}
  >{\raggedright\arraybackslash}p{(\linewidth - 4\tabcolsep) * \real{0.4458}}
  >{\raggedright\arraybackslash}p{(\linewidth - 4\tabcolsep) * \real{0.4217}}@{}}
\toprule\noalign{}
\begin{minipage}[b]{\linewidth}\raggedright
પેરામીટર
\end{minipage} & \begin{minipage}[b]{\linewidth}\raggedright
HAWT (હોરિઝોન્ટલ એક્સિસ વિન્ડ ટર્બાઈન)
\end{minipage} & \begin{minipage}[b]{\linewidth}\raggedright
VAWT (વર્ટિકલ એક્સિસ વિન્ડ ટર્બાઈન)
\end{minipage} \\
\midrule\noalign{}
\endhead
\bottomrule\noalign{}
\endlastfoot
\textbf{ઓરિએન્ટેશન} & બ્લેડ્સ ક્ષૈતિજ અક્ષ પર ફરે છે & બ્લેડ્સ ઊભી અક્ષ પર ફરે છે \\
\textbf{પવનની દિશા} & પવનની દિશા તરફ મોંઢું રાખવાની જરૂર & કોઈપણ દિશાના પવન
સાથે કામ કરે છે \\
\textbf{ઇન્સ્ટોલેશન} & ઊંચા ટાવર, જમીનથી ઊંચે & જમીનથી નીચે, સરળ એક્સેસ \\
\end{longtable}
}

\textbf{આકૃતિ:}

\begin{verbatim}
   HAWT       VAWT
    /|{        \_|\_}
   / | {      | | |}
  /\_\_|\_\_{     |\_|\_|}
     |          |
    \_|\_        \_|\_
\end{verbatim}

\end{solutionbox}
\begin{mnemonicbox}
``HV-DIT: Horizontal-Vertical,
Directional-Independent, Tall-lower''

\end{mnemonicbox}
\subsection*{પ્રશ્ન ૫(બ) [૪
ગુણ]}\label{uxaaauxab0uxab6uxaa8-uxaebuxaac-uxaea-uxa97uxaa3}

\textbf{Green energyનું વર્ગીકરણ કરો.}

\begin{solutionbox}

\textbf{ગ્રીન એનર્જી વર્ગીકરણ:}

\begin{center}
\textbf{Mermaid Diagram (Code)}
\begin{verbatim}
{Shaded}
{Highlighting}[]
graph TD;
    A[Green Energy Sources]{-{-}{}B[Solar Energy];}
    A{-{-}{}C[Wind Energy];}
    A{-{-}{}D[Hydro Energy];}
    A{-{-}{}E[Geothermal];}
    A{-{-}{}F[Biomass Energy];}
    A{-{-}{}G[Tidal/Wave Energy];}
{Highlighting}
{Shaded}
\end{verbatim}
\end{center}

{\def\LTcaptype{none} % do not increment counter
\begin{longtable}[]{@{}lll@{}}
\toprule\noalign{}
સ્ત્રોત & મુખ્ય સિદ્ધાંત & ઉપયોગ \\
\midrule\noalign{}
\endhead
\bottomrule\noalign{}
\endlastfoot
\textbf{સોલાર} & ફોટોવોલ્ટિક ઇફેક્ટ & સોલાર પેનલ્સ, થર્મલ કલેક્ટર્સ \\
\textbf{વિન્ડ} & હવાની ગતિશીલ ઊર્જા & વિન્ડ ટર્બાઈન \\
\textbf{હાઇડ્રો} & પાણીની સ્થિતિજ ઊર્જા & ડેમ, રન-ઓફ-રિવર \\
\textbf{જિયોથર્મલ} & પૃથ્વીની આંતરિક ગરમી & હીટ પમ્પ, પાવર પ્લાન્ટ \\
\end{longtable}
}

\end{solutionbox}
\begin{mnemonicbox}
``SWHGBT: Sun Wind Hydro Geo Bio Tidal - Sources
With Huge Green Benefits Today''

\end{mnemonicbox}
\subsection*{પ્રશ્ન ૫(ક) [૭
ગુણ]}\label{uxaaauxab0uxab6uxaa8-uxaebuxa95-uxaed-uxa97uxaa3}

\textbf{વિન્ડ પાવર સિસ્ટમ સમજાવો.}

\begin{solutionbox}

\textbf{વિન્ડ પાવર સિસ્ટમ:}

\begin{center}
\textbf{Mermaid Diagram (Code)}
\begin{verbatim}
{Shaded}
{Highlighting}[]
graph LR;
    A[Wind]{-{-}{}B[Turbine];}
    B{-{-}{}C[Gearbox];}
    C{-{-}{}D[Generator];}
    D{-{-}{}E[Transformer];}
    E{-{-}{}F[Grid Connection];}
    D{-{-}{}G[Controller];}
{Highlighting}
{Shaded}
\end{verbatim}
\end{center}

\textbf{ઘટકો:}

\begin{itemize}
\tightlist
\item
  \textbf{વિન્ડ ટર્બાઈન}: પવનની ઊર્જાને યાંત્રિક રોટેશનમાં રૂપાંતરિત કરે છે
\item
  \textbf{ગિયરબોક્સ}: જનરેટર માટે રોટેશન સ્પીડ વધારે છે
\item
  \textbf{જનરેટર}: યાંત્રિકને વિદ્યુત ઊર્જામાં રૂપાંતરિત કરે છે
\item
  \textbf{કંટ્રોલર}: આઉટપુટ અને સેફ્ટી ફંક્શન્સ નિયંત્રિત કરે છે
\item
  \textbf{ટ્રાન્સફોર્મર}: ટ્રાન્સમિશન માટે વોલ્ટેજ વધારે છે
\item
  \textbf{ટાવર}: વધુ મજબૂત પવન પકડવા માટે ટર્બાઈનને ઊંચે રાખે છે
\end{itemize}

\textbf{કાર્ય સિદ્ધાંત:}

\begin{enumerate}
\tightlist
\item
  પવન બ્લેડ્સને ફેરવે છે (ગતિશીલથી યાંત્રિક)
\item
  ગિયરબોક્સ RPM વધારે છે
\item
  જનરેટર AC પાવર ઉત્પન્ન કરે છે
\item
  કંટ્રોલર આઉટપુટ નિયંત્રિત કરે છે
\item
  ટ્રાન્સફોર્મર ગ્રિડ કનેક્શન માટે તૈયાર કરે છે
\end{enumerate}

\end{solutionbox}
\begin{mnemonicbox}
``WINGER: Wind In, Gearbox Enhances Rotation,
Generator outputs''

\end{mnemonicbox}
\subsection*{પ્રશ્ન ૫(અ) OR [૩
ગુણ]}\label{uxaaauxab0uxab6uxaa8-uxaebuxa85-or-uxae9-uxa97uxaa3}

\textbf{ગ્રીન ઊર્જાની કોઈપણ ત્રણ જરૂરિયાત લખો.}

\begin{solutionbox}

{\def\LTcaptype{none} % do not increment counter
\begin{longtable}[]{@{}ll@{}}
\toprule\noalign{}
ગ્રીન એનર્જીની જરૂરિયાત & સમજૂતી \\
\midrule\noalign{}
\endhead
\bottomrule\noalign{}
\endlastfoot
\textbf{પર્યાવરણ સંરક્ષણ} & પ્રદૂષણ અને ગ્રીનહાઉસ ગેસ ઉત્સર્જન ઘટાડે છે \\
\textbf{સંસાધન સંરક્ષણ} & સીમિત ફોસિલ ફ્યુઅલ સંસાધનોનું સંરક્ષણ કરે છે \\
\textbf{ઊર્જા સુરક્ષા} & આયાત કરેલા ઈંધણ પર નિર્ભરતા અને ભાવમાં અસ્થિરતા ઘટાડે
છે \\
\end{longtable}
}

\textbf{અન્ય જરૂરિયાતો}: જળવાયુ પરિવર્તન શમન, ટકાઉ વિકાસ, આર્થિક લાભો

\end{solutionbox}
\begin{mnemonicbox}
``ECO: Environment protected, Conservation of
resources, Oil-independence''

\end{mnemonicbox}
\subsection*{પ્રશ્ન ૫(બ) OR [૪
ગુણ]}\label{uxaaauxab0uxab6uxaa8-uxaebuxaac-or-uxaea-uxa97uxaa3}

\textbf{PV સેલ પર ટૂંક નોંધ લખો.}

\begin{solutionbox}

\textbf{ફોટોવોલ્ટિક (PV) સેલ:}

\begin{center}
\textbf{Mermaid Diagram (Code)}
\begin{verbatim}
{Shaded}
{Highlighting}[]
graph TD;
    A[Sunlight]{-{-}{}B[PV Cell];}
    B{-{-}{}C[DC Electricity];}
    B{-{-}{}D[Construction: P{-}N Junction];}
    B{-{-}{}E[Materials: Silicon, Thin Film];}
{Highlighting}
{Shaded}
\end{verbatim}
\end{center}

\textbf{કાર્ય સિદ્ધાંત:}

\begin{itemize}
\tightlist
\item
  ફોટોવોલ્ટિક ઇફેક્ટ પર આધારિત
\item
  સૂર્યપ્રકાશને સીધો વિદ્યુતમાં રૂપાંતરિત કરે છે
\item
  સેમીકન્ડક્ટર મટિરિયલ (સામાન્ય રીતે સિલિકોન) વાપરે છે
\item
  ફોટોન્સ P-N જંક્શન પર પડવાથી ઈલેક્ટ્રોન ફ્લો બનાવે છે
\end{itemize}

\textbf{પ્રકારો}: મોનોક્રિસ્ટલાઈન, પોલીક્રિસ્ટલાઈન, થિન-ફિલ્મ

\textbf{કાર્યક્ષમતા}: વ્યાવસાયિક સેલ માટે સામાન્ય રીતે 15-22\%

\end{solutionbox}
\begin{mnemonicbox}
``SPEC: Sunlight Produces Electricity through Cells
with p-n junctions''

\end{mnemonicbox}
\subsection*{પ્રશ્ન ૫(ક) OR [૭
ગુણ]}\label{uxaaauxab0uxab6uxaa8-uxaebuxa95-or-uxaed-uxa97uxaa3}

\textbf{સોલાર પાવર પદ્ધતિ સમજાવો.}

\begin{solutionbox}

\textbf{સોલાર પાવર સિસ્ટમ:}

\begin{center}
\textbf{Mermaid Diagram (Code)}
\begin{verbatim}
{Shaded}
{Highlighting}[]
graph LR;
    A[Solar Panels]{-{-}{}B[Charge Controller];}
    B{-{-}{}C[Battery Bank];}
    C{-{-}{}D[Inverter];}
    D{-{-}{}E[AC Loads];}
    A{-{-}{}F[On{-}Grid System]{-}{-}{}G[Grid Tie Inverter]{-}{-}{}H[Electric Grid];}
{Highlighting}
{Shaded}
\end{verbatim}
\end{center}

\textbf{ઘટકો:}

\begin{itemize}
\tightlist
\item
  \textbf{સોલાર પેનલ્સ}: સૂર્યપ્રકાશને DC વિદ્યુતમાં રૂપાંતરિત કરે છે
\item
  \textbf{ચાર્જ કંટ્રોલર}: બેટરી ચાર્જિંગ નિયંત્રિત કરે છે
\item
  \textbf{બેટરી બેંક}: વિદ્યુત ઊર્જા સંગ્રહિત કરે છે (ઓફ-ગ્રિડ)
\item
  \textbf{ઇન્વર્ટર}: ઘરેલુ ઉપયોગ માટે DCને ACમાં રૂપાંતરિત કરે છે
\item
  \textbf{ડિસ્ટ્રિબ્યુશન પેનલ}: ઘરની વિદ્યુત પ્રણાલી સાથે જોડાણ કરે છે
\end{itemize}

\textbf{પ્રકારો:}

\begin{itemize}
\tightlist
\item
  \textbf{ગ્રિડ-કનેક્ટેડ}: વધારાની પાવર ગ્રિડમાં ફીડ કરે છે
\item
  \textbf{ઓફ-ગ્રિડ}: બેટરી સ્ટોરેજ સાથે સ્વતંત્ર
\item
  \textbf{હાઇબ્રિડ}: બંને સિસ્ટમનું સંયોજન
\end{itemize}

\textbf{એપ્લિકેશન્સ:} ઘર પાવર, વોટર પમ્પિંગ, સ્ટ્રીટ લાઇટિંગ, ઔદ્યોગિક ઉપયોગ

\end{solutionbox}
\begin{mnemonicbox}
``SCBID: Solar Cells produce, Battery stores,
Inverter converts, Distribution supplies''

\end{mnemonicbox}

\end{document}
