\documentclass{article}

% content/resources/templates/preamble.tex
\usepackage[margin=0.6in]{geometry}
\author{Milav Dabgar}
\usepackage{amsmath,amssymb,amsthm}
\usepackage{booktabs}
\usepackage{multirow}
\usepackage{xcolor}
\usepackage{tcolorbox}
\tcbuselibrary{breakable,skins}
\usepackage[colorlinks=true,linkcolor=blue]{hyperref}
\usepackage{titlesec}
\usepackage{enumitem}
\usepackage{tikz}
\usepackage{pgfplots}
\usepackage{circuitikz}
\usepackage[version=4]{mhchem}
\usepackage{longtable}
\usepackage{array}
\usepackage{float}
\usepackage{caption}
\usepackage{listings}

\lstset{
  basicstyle=\small\ttfamily,
  breaklines=true,
  breakatwhitespace=false,
  postbreak=\mbox{\textcolor{red}{$\hookrightarrow$}\space},
  float=false,
  numbers=left,
  numberstyle=\tiny\color{gray},
  numbersep=10pt,
  xleftmargin=2em,
  keywordstyle=\color{blue},
  commentstyle=\color{green!60!black},
  stringstyle=\color{purple},
  backgroundcolor=\color{gray!5},
  showstringspaces=false,
  tabsize=2,
  captionpos=b,
  keepspaces=true,
  columns=flexible
}

\pgfplotsset{compat=1.18}
\usetikzlibrary{shapes,arrows,positioning,calc,patterns,decorations.pathmorphing,decorations.markings,arrows.meta}

% Color scheme
\definecolor{headcolor}{RGB}{0,102,204}
\definecolor{keycolor}{RGB}{220,20,60}
\definecolor{solutioncolor}{RGB}{34,139,34}
\definecolor{mnemoniccolor}{RGB}{148,0,211}
\definecolor{codecolor}{RGB}{0,0,100}

% Spacing
\setlength{\parskip}{3pt}
\setlist[itemize]{nosep}
\setlist[enumerate]{nosep}

% Title formatting
\titleformat{\section}{\Large\bfseries\color{headcolor}}{\thesection}{1em}{}
\titleformat{\subsection}{\large\bfseries\color{headcolor}}{\thesubsection}{1em}{}

% Pandoc tightlist compatibility
\providecommand{\tightlist}{%
  \setlength{\itemsep}{0pt}\setlength{\parskip}{0pt}}

% Pandoc longtable compatibility
\newcounter{none}
\def\thenone{}


% content/resources/templates/gujarati-boxes.tex
\usepackage{fontspec}
\usepackage{polyglossia}

% Set Gujarati as main language (document is primarily in Gujarati)
% Note: gloss-gujarati.ldf doesn't exist in polyglossia, but it will use hyphenation patterns
\setdefaultlanguage{gujarati}
\setotherlanguage{english}

% Configure Gujarati font properly
% Use Language=Default to prevent polyglossia from trying to add language-specific features
% that don't exist for Gujarati, which causes "empty feature" warnings
\newfontfamily\gujaratifont[Script=Gujarati,AutoFakeBold=2.5,AutoFakeSlant=0.3]{Noto Sans Gujarati}
\setmainfont[Script=Gujarati,AutoFakeBold=2.5,AutoFakeSlant=0.3]{Noto Sans Gujarati}
% Use Noto Sans Gujarati for monospace to support Gujarati in text
\setmonofont[Scale=0.9]{Noto Sans Gujarati}

% Configure English to use the same font
\newfontfamily\englishfont[Script=Gujarati,AutoFakeBold=2.5,AutoFakeSlant=0.3]{Noto Sans Gujarati}

% Translations for polyglossia
\gappto\captionsgujarati{
  \renewcommand{\tablename}{કોષ્ટક}
  \renewcommand{\figurename}{આકૃતિ}
}

% Helper for TikZ nodes to ensure Gujarati font
\newcommand{\gu}[1]{{\gujaratifont #1}}

% Custom environments
\newtcolorbox{solutionbox}{
    breakable,
    enhanced,
    colback=solutioncolor!5!white,
    colframe=solutioncolor!75!black,
    fonttitle=\bfseries,
    title=જવાબ
}

\newtcolorbox{solutionboxnobreak}{
 colback=solutioncolor!5!white,
 colframe=solutioncolor!75!black,
 fonttitle=\bfseries,
 title=જવાબ
}

\newtcolorbox{keyformula}{
 breakable,
 enhanced,
 colback=keycolor!5!white,
 colframe=keycolor!75!black,
 fonttitle=\bfseries,
 title=રાસાયણિક સમીકરણ/સૂત્ર
}

\newtcolorbox{mnemonicbox}{
 breakable,
 enhanced,
 colback=mnemoniccolor!5!white,
 colframe=mnemoniccolor!75!black,
 fonttitle=\bfseries,
 title=મેમરી ટ્રીક
}


% Custom commands for GTU solutions
% This file defines semantic commands for consistent formatting

% Question command with automatic formatting
\newcommand{\question}[2]{%
  \section*{Question #1}%
  \textbf{#2}%
}

% OR question variant
\newcommand{\questionor}[2]{%
  \section*{Question #1 OR}%
  \textbf{#2}%
}

% Proper table environment with caption
\newenvironment{answertable}[1]{%
  \begin{table}[htbp]
  \centering
  \caption{#1}
}{%
  \end{table}
}

% Proper figure environment for diagrams
\newenvironment{answerdiagram}[1]{%
  \begin{figure}[htbp]
  \centering
  \caption{#1}
}{%
  \end{figure}
}

% Semantic markup for key terms
\newcommand{\keyword}[1]{\textbf{#1}}
\newcommand{\code}[1]{\texttt{#1}}
\newcommand{\classname}[1]{\texttt{#1}}
\newcommand{\methodname}[1]{\texttt{#1}}

% Proper quotation marks
\newcommand{\mnemonic}[1]{``#1''}


\title{Fundamentals of Electrical Engineering (4311101) - Winter 2024 Solution}
\date{January 10, 2024}

\begin{document}
\maketitle

% Question 1
\questionmarks{1(a)}{3}{વિદ્યુત પ્રવાહ, પાવર, અને ઊર્જા ની વ્યાખ્યા આપો.}

\begin{solutionbox}
\textbf{જવાબ}:

\begin{center}
\captionof{table}{મૂળભૂત શરતો}
\begin{tabulary}{\linewidth}{|L|L|}
\hline
\textbf{શબ્દ} & \textbf{વ્યાખ્યા} \\ \hline
\textbf{વિદ્યુત પ્રવાહ (Current)} & વાહક દ્વારા વિદ્યુત ચાર્જનો પ્રવાહ દર (એમ્પિયર, A માં). \\ \hline
\textbf{વિદ્યુત પાવર (Power)} & વિદ્યુત ઊર્જાના ટ્રાન્સફર અથવા વપરાશનો દર (વોટ, W માં). \\ \hline
\textbf{ઊર્જા (Energy)} & કાર્ય કરવાની ક્ષમતા, પાવર $\times$ સમય (જૂલ અથવા વોટ-કલાક). \\ \hline
\end{tabulary}
\end{center}
\end{solutionbox}

\begin{mnemonicbox}
\mnemonic{CPE: Charge-Per-second, Product-of-VI, Energy-over-time}
\end{mnemonicbox}

\questionmarks{1(b)}{4}{વાહક, અવાહક અને મિશ્ર ધાતુના અવરોધના મૂલ્ય પર તાપમાનની અસર સમજાવો.}

\begin{solutionbox}
\textbf{જવાબ}:

\begin{center}
\captionof{table}{તાપમાનની અસર}
\begin{tabulary}{\linewidth}{|L|L|L|}
\hline
\textbf{મટીરિયલનો પ્રકાર} & \textbf{તાપમાનની અસર} & \textbf{સમીકરણ} \\ \hline
\textbf{શુદ્ધ ધાતુઓ} & તાપમાન વધતાં અવરોધ વધે છે (Positive Temp Coeff). & $R_2 = R_1[1 + \alpha(T_2-T_1)]$ \\ \hline
\textbf{મિશ્ર ધાતુઓ} & તાપમાન સાથે થોડોક વધારો (Low $\alpha$). & $R_2 = R_1[1 + \alpha(T_2-T_1)]$ \\ \hline
\textbf{અવાહકો} & તાપમાન વધતાં અવરોધ ઘટે છે (Negative Temp Coeff). & $R_2 = R_1 e^{\beta(1/T_2-1/T_1)}$ \\ \hline
\end{tabulary}
\end{center}
\end{solutionbox}

\begin{mnemonicbox}
\mnemonic{MAI: Metals Add, Alloys Increase-little, Insulators Invert}
\end{mnemonicbox}

\questionmarks{1(c)}{7}{KVL અને KCL ઉદાહરણ સાથે સમજાવો.}

\begin{solutionbox}
\textbf{જવાબ}:

\textbf{કિરચોફના નિયમો:}

\begin{center}
\captionof{table}{KCL vs KVL}
\begin{tabulary}{\linewidth}{|L|L|L|}
\hline
\textbf{નિયમ} & \textbf{વિધાન} & \textbf{સમીકરણ} \\ \hline
\textbf{KCL} & નોડમાં પ્રવેશતા કરંટનો સરવાળો = નોડમાંથી નીકળતા કરંટનો સરવાળો. & $\sum I_{in} = \sum I_{out}$ \\ \hline
\textbf{KVL} & બંધ લૂપમાં વોલ્ટેજ ડ્રોપનો સરવાળો = વોલ્ટેજ રાઈઝનો સરવાળો. & $\sum V = 0$ \\ \hline
\end{tabulary}
\end{center}

\begin{answerdiagram}{KCL and KVL Illustrations}
\begin{tikzpicture}
    % KCL
    \begin{scope}[xshift=0cm, yshift=0cm]
        \node[circle, fill=black, inner sep=2pt, label=below:A] (A) at (0,0) {};
        \draw[<-] (A) -- (-1.5, 1) node[left] {$I_1 (5A)$};
        \draw[<-] (A) -- (-1.5, -1) node[left] {$I_2 (3A)$};
        \draw[->] (A) -- (1.5, 1) node[right] {$I_3$};
        \draw[->] (A) -- (1.5, -1) node[right] {$I_4$};
        \node at (0, -2) {KCL: $I_1+I_2 = I_3+I_4$};
        \node at (0, -2.5) {$8A$ Leaving};
    \end{scope}

    % KVL
    \begin{scope}[xshift=6cm, yshift=0cm]
        \draw (0,0) to[battery1, l=12V] (0,3) -- (3,3) to[R, l=$R_1(4\Omega)$] (3,1.5) to[R, l=$R_2(8\Omega)$] (3,0) -- (0,0);
        \node at (1.5, -2) {KVL: $12V = I(4+8)$};
        \node at (1.5, -2.5) {$\sum V = 0$};
    \end{scope}
\end{tikzpicture}
\end{answerdiagram}
\end{solutionbox}

\begin{mnemonicbox}
\mnemonic{CLAN: Currents Leave And eNter equally, Voltage Around Loop is Null}
\end{mnemonicbox}

% Question 1 OR
\questionmarks{1(c) OR}{7}{જરૂરી સૂત્ર સાથે અવરોધનું શ્રેણી અને સમાંતર જોડાણ સમજાવો.}

\begin{solutionbox}
\textbf{જવાબ}:

\begin{center}
\captionof{table}{શ્રેણી vs સમાંતર જોડાણ}
\begin{tabulary}{\linewidth}{|L|L|L|}
\hline
\textbf{જોડાણ} & \textbf{સમીકરણ} & \textbf{લાક્ષણિકતા} \\ \hline
\textbf{શ્રેણી (Series)} & $R_{eq} = R_1 + R_2 + \dots$ & કરંટ સમાન રહે છે. કુલ અવરોધ વધે છે. \\ \hline
\textbf{સમાંતર (Parallel)} & $\frac{1}{R_{eq}} = \frac{1}{R_1} + \frac{1}{R_2} + \dots$ & વોલ્ટેજ સમાન રહે છે. કુલ અવરોધ ઘટે છે. \\ \hline
\end{tabulary}
\end{center}

\begin{answerdiagram}{Resistor Connections}
\begin{circuitikz}[scale=0.8, transform shape]
    % Series
    \draw (0,2) to[R, l=$R_1$] (2,2) to[R, l=$R_2$] (4,2) to[R, l=$R_3$] (6,2);
    \node at (3, 1) {Series: Current $I$ constant};

    % Parallel
    \begin{scope}[yshift=-2cm]
        \draw (0,0) -- (1,0) -- (1,1.5) to[R, l=$R_1$] (5,1.5) -- (5,0) -- (6,0);
        \draw (1,0) -- (1,0) to[R, l=$R_2$] (5,0) -- (5,0);
        \draw (1,0) -- (1,-1.5) to[R, l=$R_3$] (5,-1.5) -- (5,0);
        \node at (3, -2.5) {Parallel: Voltage $V$ constant};
    \end{scope}
\end{circuitikz}
\end{answerdiagram}
\end{solutionbox}

\begin{mnemonicbox}
\mnemonic{SPARC: Series Plus All Resistors, parallel Combines with reciprocals}
\end{mnemonicbox}

% Question 2
\questionmarks{2(a)}{3}{અવરોધના મૂલ્યને અસર કરતાં પરિબળો લખો.}

\begin{solutionbox}
\textbf{જવાબ}:

અવરોધ $R$ નીચેના પરિબળો પર આધાર રાખે છે:

\begin{center}
\captionof{table}{અવરોધ પર અસર કરતાં પરિબળો}
\begin{tabulary}{\linewidth}{|L|L|L|}
\hline
\textbf{પરિબળ} & \textbf{અસર} & \textbf{સંબંધ} \\ \hline
\textbf{લંબાઈ ($l$)} & સમપ્રમાણમાં & $R \propto l$ \\ \hline
\textbf{આડછેદનું ક્ષેત્રફળ ($A$)} & વ્યસ્ત પ્રમાણમાં & $R \propto 1/A$ \\ \hline
\textbf{મટીરિયલ ($\rho$)} & રેઝિસ્ટિવિટી પર આધારિત & $R \propto \rho$ \\ \hline
\textbf{તાપમાન ($T$)} & તાપમાન સાથે વધે છે & $R \propto T$ \\ \hline
\end{tabulary}
\end{center}

\keyword{સૂત્ર}: $R = \rho \frac{l}{A}$
\end{solutionbox}

\begin{mnemonicbox}
\mnemonic{LAMT: Length Adds, Area Minimizes, Material matters, Temperature transforms}
\end{mnemonicbox}

\questionmarks{2(b)}{4}{પાવર ત્રિકોણ દોરી એક્ટિવ અને રીઍક્ટિવ પાવરની વ્યાખ્યા આપો.}

\begin{solutionbox}
\textbf{જવાબ}:

\begin{center}
\captionof{table}{પાવરના પ્રકારો}
\begin{tabulary}{\linewidth}{|L|L|L|L|}
\hline
\textbf{પ્રકાર} & \textbf{વ્યાખ્યા} & \textbf{એકમ} & \textbf{ફોર્મ્યુલા} \\ \hline
\textbf{એક્ટિવ પાવર (P)} & ઉપયોગી કાર્ય કરતી વાસ્તવિક પાવર. & Watt (W) & $P = VI \cos \phi$ \\ \hline
\textbf{રીઍક્ટિવ પાવર (Q)} & સ્ત્રોત અને લોડ વચ્ચે આંદોલિત થતી પાવર. & VAR & $Q = VI \sin \phi$ \\ \hline
\textbf{એપેરન્ટ પાવર (S)} & એક્ટિવ અને રીઍક્ટિવ પાવરનો વેક્ટર સરવાળો. & VA & $S = VI$ \\ \hline
\end{tabulary}
\end{center}

\begin{answerdiagram}{Power Triangle}
\begin{tikzpicture}[scale=1.2]
    \draw[thick] (0,0) -- (4,0) node[midway, below] {Active Power $P$ (W)};
    \draw[thick] (4,0) -- (4,3) node[midway, right] {Reactive Power $Q$ (VAR)};
    \draw[thick] (0,0) -- (4,3) node[midway, above left] {Apparent Power $S$ (VA)};
    \draw (0.5,0) arc (0:36.87:0.5);
    \node at (0.8, 0.3) {$\phi$};
\end{tikzpicture}
\end{answerdiagram}
\end{solutionbox}

\begin{mnemonicbox}
\mnemonic{PAWS: Power Active Works, Apparent is Slant-hypotenuse, reactive Qoscillates}
\end{mnemonicbox}

\questionmarks{2(c)}{7}{સેલ અને બેટરી સમજાવો. વિવિધ રેટિંગ અને બેટરીના પ્રકારોની યાદી બનાવો.}

\begin{solutionbox}
\textbf{જવાબ}:

\textbf{તફાવત:}
\begin{itemize}
    \item \keyword{Cell (સેલ)}: રાસાયણિક ઊર્જાને વિદ્યુત ઊર્જામાં રૂપાંતરિત કરતું એકમ.
    \item \keyword{Battery (બેટરી)}: શ્રેણી અથવા સમાંતરમાં જોડાયેલા સેલનો સમૂહ.
\end{itemize}

\textbf{બેટરી રેટિંગ્સ:}
\begin{itemize}
    \item \keyword{Voltage}: પોટેન્શિયલ ડિફરન્સ (Volts).
    \item \keyword{Capacity}: સંગ્રહિત ચાર્જ (Ah).
    \item \keyword{Energy}: કુલ ઊર્જા (Wh).
    \item \keyword{C-Rate}: ચાર્જ/ડિસ્ચાર્જ દર.
\end{itemize}

\begin{answerdiagram}{Battery Types Hierarchy}
\begin{tikzpicture}[
    level 1/.style={sibling distance=4cm},
    level 2/.style={sibling distance=1.5cm}
]
\node [gtu root] {Battery Types}
    child { node [gtu child] {Primary\\(Non-rechargeable)}
        child { node [gtu child] {Alkaline} }
        child { node [gtu child] {Zinc-Carbon} }
        child { node [gtu child] {Lithium} }
    }
    child { node [gtu child] {Secondary\\(Rechargeable)}
        child { node [gtu child] {Lead-Acid} }
        child { node [gtu child] {Li-ion} }
        child { node [gtu child] {Ni-MH} }
    };
\end{tikzpicture}
\end{answerdiagram}
\end{solutionbox}

\begin{mnemonicbox}
\mnemonic{CAVE: Cells Are Voltage Elements, batteries Bundle And TallY Energy}
\end{mnemonicbox}

% Question 2 OR
\questionmarks{2(a) OR}{3}{અવરોધ, વહન અને વાહકતાની વ્યાખ્યા આપો.}

\begin{solutionbox}
\textbf{જવાબ}:

\begin{center}
\captionof{table}{વ્યાખ્યાઓ}
\begin{tabulary}{\linewidth}{|L|L|L|L|}
\hline
\textbf{શબ્દ} & \textbf{વ્યાખ્યા} & \textbf{એકમ} & \textbf{સૂત્ર} \\ \hline
\textbf{અવરોધ (R)} & વિદ્યુત પ્રવાહનો વિરોધ. & Ohm ($\Omega$) & $R = \rho l/A$ \\ \hline
\textbf{વહન (G)} & વિદ્યુત પ્રવાહની સરળતા (અવરોધનો વ્યસ્ત). & Siemens (S) & $G = 1/R$ \\ \hline
\textbf{વાહકતા ($\sigma$)} & કરંટ પસાર કરવાની મટીરિયલની ક્ષમતા. & S/m & $\sigma = 1/\rho$ \\ \hline
\end{tabulary}
\end{center}
\end{solutionbox}

\begin{mnemonicbox}
\mnemonic{RCG: Resist Current Gladly, Conduct Generously, Sigma Gets current through}
\end{mnemonicbox}

\questionmarks{2(b) OR}{4}{શુદ્ધ ઈંડક્ટિવ સર્કિટ માટે સાબિત કરો કે કરંટ એ વોલ્ટેજ કરતા 90° પાછળ હોય છે.}

\begin{solutionbox}
\textbf{જવાબ}:

\begin{itemize}
    \item આપેલ વોલ્ટેજ: $v = V_m \sin(\omega t)$
    \item ઇન્ડક્ટર માટે: $v = L \frac{di}{dt}$
    \item તેથી: $di = \frac{V_m}{L} \sin(\omega t) dt$
    \item સંકલન (Integration) કરતાં:
    \[ i = -\frac{V_m}{\omega L} \cos(\omega t) = \frac{V_m}{\omega L} \sin(\omega t - 90^\circ) \]
    \item આ સાબિત કરે છે કે કરંટ $i$ વોલ્ટેજ $v$ કરતાં $90^\circ$ પાછળ છે.
\end{itemize}

\begin{answerdiagram}{Inductive Circuit Waveforms}
\begin{tikzpicture}
    \begin{axis}[
        width=8cm, height=4cm,
        axis lines=middle,
        xtick={0, 1.57, 3.14, 4.71, 6.28},
        xticklabels={0, $\pi/2$, $\pi$, $3\pi/2$, $2\pi$},
        ytick=\empty,
        xlabel=$\omega t$,
        ymin=-1.2, ymax=1.2
    ]
    \addplot[blue, thick, domain=0:6.5, samples=100] {sin(deg(x))} node[right] {$v$};
    \addplot[red, thick, dashed, domain=0:6.5, samples=100] {sin(deg(x)-90)} node[right] {$i$};
    \end{axis}
\end{tikzpicture}
\end{answerdiagram}
\end{solutionbox}

\begin{mnemonicbox}
\mnemonic{ELI: Voltage Leads current In inductor by 90 degrees}
\end{mnemonicbox}

\questionmarks{2(c) OR}{7}{અવરોધ, ઈંડક્ટર અને કેપેસીટર તેમના સૂત્ર સાથે સમજાવો.}

\begin{solutionbox}
\textbf{જવાબ}:

\begin{center}
\captionof{table}{પેસિવ ઘટકો}
\begin{tabulary}{\linewidth}{|L|L|L|L|}
\hline
\textbf{ઘટક} & \textbf{વર્ણન} & \textbf{સૂત્ર} & \textbf{ઊર્જા} \\ \hline
\textbf{અવરોધ (R)} & કરંટનો વિરોધ કરે છે. & $V = IR$ & વ્યય પામે છે \\ \hline
\textbf{ઈંડક્ટર (L)} & કરંટના ફેરફારનો વિરોધ કરે છે. & $V = L\frac{di}{dt}$ & $E = \frac{1}{2}LI^2$ \\ \hline
\textbf{કેપેસીટર (C)} & વોલ્ટેજના ફેરફારનો વિરોધ કરે છે. & $I = C\frac{dv}{dt}$ & $E = \frac{1}{2}CV^2$ \\ \hline
\end{tabulary}
\end{center}

\begin{answerdiagram}{R, L, C Symbols}
\begin{circuitikz}
    \draw (0,2) to[R, l=$R$] (2,2);
    \draw (3,2) to[L, l=$L$] (5,2);
    \draw (6,2) to[C, l=$C$] (8,2);
\end{circuitikz}
\end{answerdiagram}
\end{solutionbox}

\begin{mnemonicbox}
\mnemonic{RIC: Resistor Impedes Current, Inductor Catches current-changes, Capacitor Controls voltage-changes}
\end{mnemonicbox}

% Question 3
\questionmarks{3(a)}{3}{A.C. સિગ્નલની R.M.S અને એવરેજ મૂલ્યની વ્યાખ્યા આપો અને સમજાવો.}

\begin{solutionbox}
\textbf{જવાબ}:

\begin{center}
\captionof{table}{RMS અને એવરેજ મૂલ્ય}
\begin{tabulary}{\linewidth}{|L|L|L|L|}
\hline
\textbf{મૂલ્ય} & \textbf{વ્યાખ્યા} & \textbf{સૂત્ર (સાઇન વેવ)} & \textbf{સંબંધ} \\ \hline
\textbf{RMS મૂલ્ય} & વર્ગ કરેલા મૂલ્યોના સરેરાશનું વર્ગમૂળ. & $V_{rms} = \frac{V_{max}}{\sqrt{2}} = 0.707 V_{max}$ & DC સમકક્ષ હીટિંગ અસર. \\ \hline
\textbf{એવરેજ મૂલ્ય} & અર્ધ સાયકલ પર સરેરાશ મૂલ્ય. & $V_{avg} = \frac{2V_{max}}{\pi} = 0.637 V_{max}$ & બેટરી ચાર્જિંગ માટે ઉપયોગી. \\ \hline
\end{tabulary}
\end{center}
\end{solutionbox}

\begin{mnemonicbox}
\mnemonic{RAM: Rms-Average Method: Root-mean-square And Mean-of-absolute}
\end{mnemonicbox}

\questionmarks{3(b)}{4}{વૈકલ્પિક EMF કેવી રીતે ઉત્પન્ન થાય છે તે જરૂરી આકૃતિ સાથે સમજાવો.}

\begin{solutionbox}
\textbf{જવાબ}:

\textbf{સિદ્ધાંત}: જ્યારે કોઈલ સમાન ચુંબકીય ક્ષેત્રમાં ફરે છે, ત્યારે ફ્લક્સ લિંકેજ બદલાય છે અને EMF પ્રેરિત થાય છે (ફેરાડેનો નિયમ).

\begin{answerdiagram}{AC EMF Generation}
\begin{tikzpicture}
    % Magnets
    \draw[fill=red!20] (-3,-1) rectangle (-1,1) node[midway] {N};
    \draw[fill=blue!20] (1,-1) rectangle (3,1) node[midway] {S};
    
    % Field Lines
    \foreach \y in {-0.8,-0.4,0,0.4,0.8}
        \draw[dashed, ->] (-1,\y) -- (1,\y);

    % Coil
    \draw[thick] (-0.5,-0.8) rectangle (0.5,0.8);
    \draw[->, thick] (0,0.8) arc (90:0:0.5) node[right] {$\omega$};
    
    % Waveform
    \begin{scope}[xshift=4cm, yshift=-1cm]
        \draw[->] (0,1) -- (4,1) node[right] {$\omega t$};
        \draw[->] (0,0) -- (0,2.5) node[above] {$e$};
        \draw[blue, thick, smooth, samples=100, domain=0:3.5] plot (\x, {1 + sin(\x*180)});
        \node at (2, -0.5) {Sine Waveform};
    \end{scope}
\end{tikzpicture}
\end{answerdiagram}

\begin{itemize}
    \item કોઈલ ફરે છે, ફ્લક્સ ($\phi$) કાપે છે.
    \item $e = -N \frac{d\phi}{dt} = N B A \omega \sin(\omega t)$.
    \item દરેક અર્ધ સાયકલે દિશા બદલાય છે.
\end{itemize}
\end{solutionbox}

\begin{mnemonicbox}
\mnemonic{FARM: Flux And Rotation Make alternating voltage}
\end{mnemonicbox}

\questionmarks{3(c)}{7}{શુધ્ધ આવરોધીય AC સરકીટનું એસી એનાલિસિસ કરો.}

\begin{solutionbox}
\textbf{જવાબ}:

\textbf{શુદ્ધ અવરોધીય સર્કિટ:}
\begin{itemize}
    \item વોલ્ટેજ: $v = V_m \sin \omega t$
    \item કરંટ: $i = \frac{v}{R} = \frac{V_m}{R} \sin \omega t = I_m \sin \omega t$
\end{itemize}

\begin{center}
\captionof{table}{અવરોધીય સર્કિટ એનાલિસિસ}
\begin{tabulary}{\linewidth}{|L|L|L|}
\hline
\textbf{પેરામીટર} & \textbf{સૂત્ર} & \textbf{સંબંધ} \\ \hline
\textbf{વોલ્ટેજ} & $v = V_m \sin \omega t$ & કરંટ સાથે ફેઝમાં \\ \hline
\textbf{કરંટ} & $i = I_m \sin \omega t$ & ઓહ્મના નિયમ મુજબ \\ \hline
\textbf{પાવર} & $p = vi = V_m I_m \sin^2 \omega t$ & હંમેશા ધન (Positive) \\ \hline
\textbf{સરેરાશ પાવર} & $P = V_{rms} I_{rms} = I^2R$ & અચળ હીટિંગ \\ \hline
\end{tabulary}
\end{center}

\begin{answerdiagram}{Resistive Circuit Waveforms}
\begin{tikzpicture}
    \begin{axis}[
        width=8cm, height=4cm,
        axis lines=middle,
        xtick={0, 3.14, 6.28},
        xticklabels={0, $\pi$, $2\pi$},
        xlabel=$\omega t$,
        ymin=-1.2, ymax=1.2
    ]
    \addplot[blue, thick, domain=0:6.28, samples=100] {sin(deg(x))} node[right] {$v$};
    \addplot[red, thick, dashed, domain=0:6.28, samples=100] {0.7*sin(deg(x))} node[right] {$i$};
    \end{axis}
\end{tikzpicture}
\end{answerdiagram}
\end{solutionbox}

\begin{mnemonicbox}
\mnemonic{VIPS: Voltage In-Phase with current, Same waveform, Power always Positive}
\end{mnemonicbox}

% Question 3 OR
\questionmarks{3(a) OR}{3}{એસી વિદ્યુતપ્રવાહ I=28.28sin(2$\pi$50t). વિદ્યુત પ્રવાહનું RMS મૂલ્ય શોધો.}

\begin{solutionbox}
\textbf{જવાબ}:

\textbf{આપેલ}: $I = 28.28 \sin(2\pi 50 t)$, $I = I_m \sin(\omega t)$ સાથે સરખાવતા.
\begin{itemize}
    \item $I_m = 28.28$ A
\end{itemize}

\textbf{ગણતરી}:
\[ I_{rms} = \frac{I_m}{\sqrt{2}} = \frac{28.28}{1.414} = 20 \text{ A} \]

\textbf{પરિણામ}: RMS કરંટ = 20 A
\end{solutionbox}

\begin{mnemonicbox}
\mnemonic{PER: Peak to Effective by Root-2}
\end{mnemonicbox}

\questionmarks{3(b) OR}{4}{જો Vav=60 V હોય તો વૉલ્ટેજનું RMS અને મહત્તમ મૂલ્ય શોધો.}

\begin{solutionbox}
\textbf{જવાબ}:

\textbf{આપેલ}: $V_{av} = 60$ V.

\begin{center}
\captionof{table}{ગણતરી}
\begin{tabulary}{\linewidth}{|L|L|L|}
\hline
\textbf{પગલું} & \textbf{સૂત્ર} & \textbf{ગણતરી} \\ \hline
\textbf{Max શોધો ($V_m$)} & $V_{av} = 0.637 V_m \implies V_m = \frac{V_{av}}{0.637}$ & $V_m = \frac{60}{0.637} = 94.2$ V \\ \hline
\textbf{RMS શોધો ($V_{rms}$)} & $V_{rms} = 0.707 V_m$ & $V_{rms} = 0.707 \times 94.2 = 66.6$ V \\ \hline
\end{tabulary}
\end{center}

\textbf{પરિણામ}: મહત્તમ મૂલ્ય = 94.2 V, RMS મૂલ્ય = 66.6 V
\end{solutionbox}

\begin{mnemonicbox}
\mnemonic{AVR: Average to peak Via multiplying by (pi/2), Rms is peak/root2}
\end{mnemonicbox}

\questionmarks{3(c) OR}{7}{ફેઈઝ ડાયાગ્રામની મદદથી સ્ટાર જોડાણનું લાઈન અને ફેઈસ વૉલ્ટેજનું સમીકરણ તારવો.}

\begin{solutionbox}
\textbf{જવાબ}:

\begin{answerdiagram}{Star Connection and Phasor}
\begin{tikzpicture}[scale=0.8]
    % Star Load
    \draw (0,0) node[anchor=north]{N} to[R] (0,2) node[anchor=south]{R};
    \draw (0,0) to[R] (-1.73,-1) node[anchor=north]{Y};
    \draw (0,0) to[R] (1.73,-1) node[anchor=north]{B};
    
    % Phasor
    \begin{scope}[xshift=5cm, yshift=0cm]
        \draw[->] (0,0) -- (0,2) node[above] {$V_{RN}$};
        \draw[->] (0,0) -- (-1.73,-1) node[left] {$V_{YN}$};
        \draw[->] (0,0) -- (1.73,-1) node[right] {$V_{BN}$};
        
        \draw[dashed, ->] (0,0) -- (1.73,1) node[right] {$-V_{YN}$};
        \draw[thick, ->, red] (0,0) -- (1.73,3) node[right] {$V_{RY} (Line)$};
        \draw (0,0.5) arc (90:60:0.5) node[midway, above] {$30^\circ$};
    \end{scope}
\end{tikzpicture}
\end{answerdiagram}

\textbf{તારવણી}:
\begin{itemize}
    \item લાઈન વોલ્ટેજ $V_{RY}$ એ $V_{RN}$ અને $V_{YN}$ નો વેક્ટર તફાવત છે.
    \item $V_{RY} = V_{RN} - V_{YN}$
    \item મૂલ્ય: $V_L = \sqrt{V_P^2 + V_P^2 + 2V_PV_P \cos(60^\circ)} = \sqrt{3V_P^2}$
    \item \keyword{પરિણામ}: $V_L = \sqrt{3} V_P$
    \item લાઈન વોલ્ટેજ ફેઝ વોલ્ટેજ કરતાં $30^\circ$ આગળ હોય છે.
\end{itemize}
\end{solutionbox}

\begin{mnemonicbox}
\mnemonic{PALS: Phase to Line in Star: multiply by Square-root-3}
\end{mnemonicbox}

% Question 4
\questionmarks{4(a)}{3}{Faraday અને Lenzનો નિયમ તેના સૂત્ર સાથે લખો.}

\begin{solutionbox}
\textbf{જવાબ}:

\begin{center}
\captionof{table}{ઇન્ડક્શનના નિયમો}
\begin{tabulary}{\linewidth}{|L|L|L|}
\hline
\textbf{નિયમ} & \textbf{વિધાન} & \textbf{સમીકરણ} \\ \hline
\textbf{ફેરાડેનો નિયમ} & પ્રેરિત EMF ચુંબકીય ફ્લક્સના ફેરફારના દરના સમપ્રમાણમાં હોય છે. & $e = -N \frac{d\phi}{dt}$ \\ \hline
\textbf{લેન્ઝનો નિયમ} & પ્રેરિત EMF ની દિશા તેને ઉત્પન્ન કરતા કારણનો વિરોધ કરે છે. & ઋણ ચિહ્ન (-) \\ \hline
\end{tabulary}
\end{center}
\end{solutionbox}

\begin{mnemonicbox}
\mnemonic{FORC: Faraday's flux Over Rate Change, Lenz Opposes the Reason for Change}
\end{mnemonicbox}

\questionmarks{4(b)}{4}{સિંગલ ફેઈસ સપ્લાયની સરખામણીમાં 3-ફેઈસ સપ્લાયના 4 ફાયદા લખો.}

\begin{solutionbox}
\textbf{જવાબ}:

\begin{itemize}
    \item \keyword{ઉચ્ચ પાવર ઘનત્વ}: સમાન કદ માટે, 3-ફેઝ મશીન વધુ પાવર આપે છે.
    \item \keyword{અચળ પાવર}: 3-ફેઝ પાવર અચળ (નોન-પલ્સેટિંગ) હોય છે.
    \item \keyword{મટીરિયલ બચત}: સમાન પાવર ટ્રાન્સમિશન માટે ઓછા કોપરની જરૂર પડે છે.
    \item \keyword{સેલ્ફ-સ્ટાર્ટિંગ}: 3-ફેઝ મોટર્સ ફરતા ચુંબકીય ક્ષેત્રને કારણે સેલ્ફ-સ્ટાર્ટિંગ હોય છે.
\end{itemize}
\end{solutionbox}

\begin{mnemonicbox}
\mnemonic{PCCS: Power higher, Constant delivery, Copper less, Self-starting motors}
\end{mnemonicbox}

\questionmarks{4(c)}{7}{Flemingનો જમણા હાથનો અને ડાબા હાથનો નિયમ સમજાવો.}

\begin{solutionbox}
\textbf{જવાબ}:

\begin{center}
\captionof{table}{ફ્લેમિંગના નિયમોની સરખામણી}
\begin{tabulary}{\linewidth}{|L|L|L|}
\hline
\textbf{લાક્ષણિકતા} & \textbf{જમણા હાથનો નિયમ (જનરેટર)} & \textbf{ડાબા હાથનો નિયમ (મોટર)} \\ \hline
\textbf{હેતુ} & પ્રેરિત EMF/કરંટની દિશા શોધવા & બળ/ગતિની દિશા શોધવા \\ \hline
\textbf{અંગૂઠો} & વાહકની ગતિ & ગતિ/બળ \\ \hline
\textbf{તર્જની} & ચુંબકીય ક્ષેત્ર (N થી S) & ચુંબકીય ક્ષેત્ર (N થી S) \\ \hline
\textbf{મધ્યમા} & પ્રેરિત કરંટ & કરંટ \\ \hline
\end{tabulary}
\end{center}

\begin{answerdiagram}{Fleming's Hand Rules}
\begin{tikzpicture}
    % Right Hand
    \begin{scope}[xshift=0cm]
        \draw[->, Ultra Thick, blue] (0,0) -- (0,2) node[above] {ગતિ};
        \draw[->, Ultra Thick, red] (0,0) -- (2,0) node[right] {ક્ષેત્ર};
        \draw[->, Ultra Thick, purple] (0,0) -- (0,0,-2) node[left] {પ્રેરિત કરંટ};
        \node at (0,-1) {જમણા હાથ (જનરેટર)};
    \end{scope}

    % Left Hand
    \begin{scope}[xshift=5cm]
        \draw[->, Ultra Thick, blue] (0,0) -- (0,2) node[above] {બળ/ગતિ};
        \draw[->, Ultra Thick, red] (0,0) -- (2,0) node[right] {ક્ષેત્ર};
        \draw[->, Ultra Thick, green!60!black] (0,0) -- (0,0,-2) node[left] {કરંટ};
        \node at (0,-1) {ડાબા હાથ (મોટર)};
    \end{scope}
\end{tikzpicture}
\end{answerdiagram}
\end{solutionbox}

\begin{mnemonicbox}
\mnemonic{FBI-MFC: Field-B-Induced current for right hand, Motion-Field-Current for left}
\end{mnemonicbox}

% Question 4 OR
\questionmarks{4(a) OR}{3}{ઈલેક્ટ્રોમેગ્નેટિક ઈન્ડક્સનની ઘટના સમજાવો.}

\begin{solutionbox}
\textbf{જવાબ}:

\keyword{ઈલેક્ટ્રોમેગ્નેટિક ઈન્ડક્શન}: બદલાતા ચુંબકીય ક્ષેત્રમાં મુકેલા વાહકમાં EMF ઉત્પન્ન થવાની પ્રક્રિયાને ઈલેક્ટ્રોમેગ્નેટિક ઈન્ડક્શન કહે છે.

\begin{answerdiagram}{Induction Flow}
\begin{tikzpicture}[node distance=1.5cm, auto]
    \node [gtu block] (flux) {બદલાતું ફ્લક્સ};
    \node [gtu block, right=of flux] (emf) {પ્રેરિત EMF};
    \node [gtu block, right=of emf] (current) {પ્રેરિત કરંટ};
    
    \path [gtu arrow] (flux) -- (emf);
    \path [gtu arrow] (emf) -- (current);
\end{tikzpicture}
\end{answerdiagram}
\end{solutionbox}

\begin{mnemonicbox}
\mnemonic{MICE: Motion Induces Current via Electromagnetic induction}
\end{mnemonicbox}

\questionmarks{4(b) OR}{4}{3-ફેઈસ વૈકલ્પિક ઈ. એમ. એફ. કેવી રીતે ઉત્પન થાય છે સમજાવો.}

\begin{solutionbox}
\textbf{જવાબ}:

\textbf{ઉત્પાદન સિદ્ધાંત}:
\begin{itemize}
    \item ત્રણ કોઈલ અવકાશમાં $120^\circ$ ના વિદ્યુત અંતરે મુકવામાં આવે છે.
    \item આ કોઈલને ચુંબકીય ક્ષેત્રમાં ફેરવવાથી ત્રણ EMF ઉત્પન્ન થાય છે.
    \item આ EMF સમાન મૂલ્ય અને આવૃત્તિ ધરાવે છે પરંતુ એકબીજાથી $120^\circ$ ફેઝ શિફ્ટ ધરાવે છે.
\end{itemize}

\begin{answerdiagram}{3-Phase Waveforms}
\begin{tikzpicture}
    \begin{axis}[
        width=8cm, height=4cm,
        axis lines=middle,
        xtick={0, 120, 240, 360},
        xticklabels={0, $120^\circ$, $240^\circ$, $360^\circ$},
        xlabel=$\omega t$,
        ylabel=$e$,
        ymin=-1.2, ymax=1.5,
        legend style={at={(0.5,-0.3)}, anchor=north, legend columns=-1}
    ]
    \addplot[red, thick, domain=0:360, samples=100] {sin(x)}; \addlegendentry{R}
    \addplot[yellow!80!black, thick, domain=0:360, samples=100] {sin(x-120)}; \addlegendentry{Y}
    \addplot[blue, thick, domain=0:360, samples=100] {sin(x-240)}; \addlegendentry{B}
    \end{axis}
\end{tikzpicture}
\end{answerdiagram}
\end{solutionbox}

\begin{mnemonicbox}
\mnemonic{CPS: Coils Produce Shifted waveforms at 120 degrees}
\end{mnemonicbox}

\questionmarks{4(c) OR}{7}{Statically induced E.M.F અને dynamically induced E.M.F વચ્ચેનો તફાવત લખો.}

\begin{solutionbox}
\textbf{જવાબ}:

\begin{center}
\captionof{table}{સ્ટેટિકલી vs ડાયનેમિકલી EMF}
\begin{tabulary}{\linewidth}{|L|L|L|}
\hline
\textbf{પેરામીટર} & \textbf{સ્ટેટિકલી પ્રેરિત EMF} & \textbf{ડાયનેમિકલી પ્રેરિત EMF} \\ \hline
\textbf{વ્યાખ્યા} & હલનચલન વગર પ્રેરિત EMF (ફ્લક્સ લિંકેજમાં ફેરફાર). & વાહક અને ક્ષેત્ર વચ્ચે સાપેક્ષ ગતિથી પ્રેરિત EMF. \\ \hline
\textbf{ગતિ} & સ્થિર વાહક અને ક્ષેત્ર. & ગતિમાન વાહક અથવા ક્ષેત્ર. \\ \hline
\textbf{સૂત્ર} & $e = -N \frac{d\phi}{dt}$ & $e = Blv \sin \theta$ \\ \hline
\textbf{ઉદાહરણ} & ટ્રાન્સફોર્મર & જનરેટર, ડાયનેમો \\ \hline
\end{tabulary}
\end{center}
\end{solutionbox}

\begin{mnemonicbox}
\mnemonic{SMCE: Static-Moving, Change-External: static has changing flux, moving has constant flux}
\end{mnemonicbox}

% Question 5
\questionmarks{5(a)}{3}{HAWT અને VAWT વચ્ચેનો તફાવત લખો.}

\begin{solutionbox}
\textbf{જવાબ}:

\begin{center}
\captionof{table}{HAWT vs VAWT}
\begin{tabulary}{\linewidth}{|L|L|L|}
\hline
\textbf{પેરામીટર} & \textbf{HAWT (Horizontal Axis)} & \textbf{VAWT (Vertical Axis)} \\ \hline
\textbf{ધરી (Axis)} & આડી (જમીનને સમાંતર) & ઊભી (જમીનને લંબ) \\ \hline
\textbf{પવન દિશા} & પવન તરફ મુખ રાખવું પડે (Yaw mechanism). & કોઈપણ દિશામાંથી પવન સ્વીકારે છે. \\ \hline
\textbf{ઉત્પાદન} & ટાવરની ટોચ પર ઘટકો. & જનરેટર જમીન પર હોઈ શકે છે. \\ \hline
\end{tabulary}
\end{center}

\begin{answerdiagram}{Wind Turbine Types}
\begin{tikzpicture}
    % HAWT
    \begin{scope}[xshift=0cm]
        \draw[thick] (0,0) -- (0,2); % Tower
        \draw[fill=white] (-0.2,2) rectangle (0.2,2.3); % Nacelle
        \draw[thick] (0,2.15) -- (-1,3); % Blade 1
        \draw[thick] (0,2.15) -- (1,3); % Blade 2
        \draw[thick] (0,2.15) -- (0,1); % Blade 3
        \node at (0,-0.5) {HAWT};
    \end{scope}

    % VAWT
    \begin{scope}[xshift=4cm]
        \draw[thick] (0,0) -- (0,2.5); % Shaft
        \draw[thick] (-0.8,0.5) to[out=90,in=270] (-0.8,2); % Blade L
        \draw[thick] (0.8,0.5) to[out=90,in=270] (0.8,2); % Blade R
        \draw (-0.8,0.5) -- (0.8,0.5);
        \draw (-0.8,2) -- (0.8,2);
        \node at (0,-0.5) {VAWT};
    \end{scope}
\end{tikzpicture}
\end{answerdiagram}
\end{solutionbox}

\begin{mnemonicbox}
\mnemonic{HV-DIT: Horizontal-Vertical, Directional-Independent, Tall-lower}
\end{mnemonicbox}

\questionmarks{5(b)}{4}{Green energyનું વર્ગીકરણ કરો.}

\begin{solutionbox}
\textbf{જવાબ}:

\begin{itemize}
    \item \keyword{સોલાર એનર્જી}: ફોટોવોલ્ટિક, થર્મલ.
    \item \keyword{વિન્ડ એનર્જી}: ઓનશોર, ઓફશોર.
    \item \keyword{હાઈડ્રો એનર્જી}: ડેમ, ટાઈડલ, વેવ.
    \item \keyword{જિયોથર્મલ}: પૃથ્વીની ગરમી.
    \item \keyword{બાયોમાસ}: જૈવિક કચરો.
\end{itemize}

\begin{answerdiagram}{Green Energy Classification}
\begin{tikzpicture}[
    level 1/.style={sibling distance=2.2cm, level distance=1.5cm},
    edge from parent fork down
]
\node [gtu root] {Green Energy}
    child { node [gtu child] {Solar} }
    child { node [gtu child] {Wind} }
    child { node [gtu child] {Hydro} }
    child { node [gtu child] {Bio} }
    child { node [gtu child] {Geo} };
\end{tikzpicture}
\end{answerdiagram}
\end{solutionbox}

\begin{mnemonicbox}
\mnemonic{SWHGBT: Sun Wind Hydro Geo Bio Tidal - Sources With Huge Green Benefits Today}
\end{mnemonicbox}

\questionmarks{5(c)}{7}{વિન્ડ પાવર સિસ્ટમ સમજાવો.}

\begin{solutionbox}
\textbf{જવાબ}:

\textbf{ઘટકો}:
\begin{enumerate}
    \item \keyword{Blades (બ્લેડ)}: પવન ઊર્જા મેળવે છે.
    \item \keyword{Rotor (રોટર)}: બ્લેડને જોડતી હબ.
    \item \keyword{Gearbox (ગિયરબોક્સ)}: જનરેટર માટે સ્પીડ વધારે છે.
    \item \keyword{Generator (જનરેટર)}: યાંત્રિક ગતિને વિદ્યુતમાં ફેરવે છે.
    \item \keyword{Yaw Drive}: ટર્બાઈનને પવન તરફ રાખે છે.
    \item \keyword{Tower (ટાવર)}: ઊંચાઈ પર સપોર્ટ આપે છે.
\end{enumerate}

\begin{answerdiagram}{Wind Power Block Diagram}
\begin{tikzpicture}[node distance=1.2cm, auto]
    \node [gtu block] (wind) {Wind};
    \node [gtu block, right=of wind] (blade) {Blades/Rotor};
    \node [gtu block, right=of blade] (gear) {Gearbox};
    \node [gtu block, right=of gear] (gen) {Generator};
    \node [gtu block, below=of gen] (grid) {Grid};
    
    \path [gtu arrow] (wind) -- (blade);
    \path [gtu arrow] (blade) -- (gear);
    \path [gtu arrow] (gear) -- (gen);
    \path [gtu arrow] (gen) -- (grid);
    
    \node [draw, dashed, fit=(blade) (gear) (gen)] (nacelle) {};
    \node [above] at (nacelle.north) {Nacelle};
\end{tikzpicture}
\end{answerdiagram}
\end{solutionbox}

\begin{mnemonicbox}
\mnemonic{WINGER: Wind In, Gearbox Enhances Rotation, Generator outputs}
\end{mnemonicbox}

% Question 5 OR
\questionmarks{5(a) OR}{3}{ગ્રીન ઊર્જાની કોઈપણ ત્રણ જરૂરિયાત લખો.}

\begin{solutionbox}
\textbf{જવાબ}:

\begin{itemize}
    \item \keyword{પર્યાવરણ સંરક્ષણ}: કાર્બન ફૂટપ્રિન્ટ અને પ્રદૂષણ ઘટાડવા.
    \item \keyword{ટકાઉપણું (Sustainability)}: અશ્મિભૂત ઇંધણની સરખામણીમાં અખૂટ સ્ત્રોત.
    \item \keyword{ઊર્જા સુરક્ષા}: આયાતી ઇંધણ પર નિર્ભરતા ઘટાડવા.
\end{itemize}
\end{solutionbox}

\begin{mnemonicbox}
\mnemonic{ECO: Environment protected, Conservation of resources, Oil-independence}
\end{mnemonicbox}

\questionmarks{5(b) OR}{4}{PV સેલ પર ટૂંક નોંધ લખો.}

\begin{solutionbox}
\textbf{જવાબ}:

\textbf{ફોટોવોલ્ટિક (PV) સેલ}:
\begin{itemize}
    \item સોલાર સિસ્ટમનું મૂળભૂત એકમ.
    \item સેમીકન્ડક્ટર (સિલિકોન) થી બનેલું.
    \item \keyword{ફોટોવોલ્ટિક ઇફેક્ટ} પર કામ કરે છે: ફોટોન PN જંક્શન પર પડે $\to$ ઈલેક્ટ્રોન-હોલ જોડી $\to$ કરંટ.
    \item આઉટપુટ: DC વોલ્ટેજ (~0.5-0.6V પ્રતિ સેલ).
\end{itemize}

\begin{answerdiagram}{PV Cell Construction}
\begin{tikzpicture}
    \draw[fill=blue!10] (0,0) rectangle (4,1) node[midway] {P-Type Si};
    \draw[fill=blue!30] (0,1) rectangle (4,1.5) node[midway] {N-Type Si};
    \draw[thick] (2,1.5) -- (2,2.5); % Wire
    \draw[thick] (2,-0.5) -- (2,0);
    \foreach \x in {0.5, 1.5, 2.5, 3.5}
        \draw[->, orange, decorate, decoration={snake}] (\x,3) -- (\x,1.5);
    \node at (2, 3.2) {Sunlight};
\end{tikzpicture}
\end{answerdiagram}
\end{solutionbox}

\begin{mnemonicbox}
\mnemonic{SPEC: Sunlight Produces Electricity through Cells with p-n junctions}
\end{mnemonicbox}

\questionmarks{5(c) OR}{7}{સોલાર પાવર પદ્ધતિ સમજાવો.}

\begin{solutionbox}
\textbf{જવાબ}:

\textbf{સોલાર પાવર સિસ્ટમ}:
\begin{enumerate}
    \item \keyword{Solar Array}: DC ઉત્પન્ન કરવા માટે PV પેનલ્સનો સંગ્રહ.
    \item \keyword{Charge Controller}: બેટરી ચાર્જિંગનું નિયમન કરે છે.
    \item \keyword{Battery Bank}: ઊર્જા સંગ્રહ કરે છે (ઓફ-ગ્રિડ).
    \item \keyword{Inverter}: ઉપકરણો માટે DC ને AC માં ફેરવે છે.
    \item \keyword{Load}: વિદ્યુત ઉપકરણો.
\end{enumerate}

\begin{answerdiagram}{Solar System Block Diagram}
\begin{tikzpicture}[node distance=1.5cm, auto]
    \node [gtu block] (panel) {PV Panels};
    \node [gtu block, right=of panel] (cc) {Charge Controller};
    \node [gtu block, below=of cc] (batt) {Battery};
    \node [gtu block, right=of cc] (inv) {Inverter};
    \node [gtu block, right=of inv] (load) {AC Load};
    
    \path [gtu arrow] (panel) -- (cc);
    \path [gtu arrow] (cc) -- (batt);
    \path [gtu arrow] (cc) -- (inv);
    \path [gtu arrow] (inv) -- (load);
    \path [gtu arrow] (batt) -- (cc);
\end{tikzpicture}
\end{answerdiagram}
\end{solutionbox}

\begin{mnemonicbox}
\mnemonic{SCBID: Solar Cells produce, Battery stores, Inverter converts, Distribution supplies}
\end{mnemonicbox}

\end{document}


