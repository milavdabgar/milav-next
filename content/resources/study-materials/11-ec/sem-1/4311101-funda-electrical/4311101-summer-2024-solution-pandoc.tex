\documentclass[10pt,a4paper]{article}

% content/resources/templates/preamble.tex
\usepackage[margin=0.6in]{geometry}
\author{Milav Dabgar}
\usepackage{amsmath,amssymb,amsthm}
\usepackage{booktabs}
\usepackage{multirow}
\usepackage{xcolor}
\usepackage{tcolorbox}
\tcbuselibrary{breakable,skins}
\usepackage[colorlinks=true,linkcolor=blue]{hyperref}
\usepackage{titlesec}
\usepackage{enumitem}
\usepackage{tikz}
\usepackage{pgfplots}
\usepackage{circuitikz}
\usepackage[version=4]{mhchem}
\usepackage{longtable}
\usepackage{array}
\usepackage{float}
\usepackage{caption}
\usepackage{listings}

\lstset{
  basicstyle=\small\ttfamily,
  breaklines=true,
  breakatwhitespace=false,
  postbreak=\mbox{\textcolor{red}{$\hookrightarrow$}\space},
  float=false,
  numbers=left,
  numberstyle=\tiny\color{gray},
  numbersep=10pt,
  xleftmargin=2em,
  keywordstyle=\color{blue},
  commentstyle=\color{green!60!black},
  stringstyle=\color{purple},
  backgroundcolor=\color{gray!5},
  showstringspaces=false,
  tabsize=2,
  captionpos=b,
  keepspaces=true,
  columns=flexible
}

\pgfplotsset{compat=1.18}
\usetikzlibrary{shapes,arrows,positioning,calc,patterns,decorations.pathmorphing,decorations.markings,arrows.meta}

% Color scheme
\definecolor{headcolor}{RGB}{0,102,204}
\definecolor{keycolor}{RGB}{220,20,60}
\definecolor{solutioncolor}{RGB}{34,139,34}
\definecolor{mnemoniccolor}{RGB}{148,0,211}
\definecolor{codecolor}{RGB}{0,0,100}

% Spacing
\setlength{\parskip}{3pt}
\setlist[itemize]{nosep}
\setlist[enumerate]{nosep}

% Title formatting
\titleformat{\section}{\Large\bfseries\color{headcolor}}{\thesection}{1em}{}
\titleformat{\subsection}{\large\bfseries\color{headcolor}}{\thesubsection}{1em}{}

% Pandoc tightlist compatibility
\providecommand{\tightlist}{%
  \setlength{\itemsep}{0pt}\setlength{\parskip}{0pt}}

% Pandoc longtable compatibility
\newcounter{none}
\def\thenone{}


% content/resources/templates/english-boxes.tex
% This file is currently empty - it exists to maintain consistency with the import structure.
% Add custom environments here if needed in the future.


\begin{document}

\begin{center}
{\Huge\bfseries\color{headcolor} Subject Name Solutions}\\[5pt]
{\LARGE 4311101 -- Summer 2024}\\[3pt]
{\large Semester 1 Study Material}\\[3pt]
{\normalsize\textit{Detailed Solutions and Explanations}}
\end{center}

\vspace{10pt}

\subsection*{Question 1(a) [3 marks]}\label{q1a}

\textbf{Define EMF, electric current and power. Also write their units.}

\begin{solutionbox}

{\def\LTcaptype{none} % do not increment counter
\begin{longtable}[]{@{}
  >{\raggedright\arraybackslash}p{(\linewidth - 4\tabcolsep) * \real{0.2500}}
  >{\raggedright\arraybackslash}p{(\linewidth - 4\tabcolsep) * \real{0.5000}}
  >{\raggedright\arraybackslash}p{(\linewidth - 4\tabcolsep) * \real{0.2500}}@{}}
\toprule\noalign{}
\begin{minipage}[b]{\linewidth}\raggedright
Term
\end{minipage} & \begin{minipage}[b]{\linewidth}\raggedright
Definition
\end{minipage} & \begin{minipage}[b]{\linewidth}\raggedright
Unit
\end{minipage} \\
\midrule\noalign{}
\endhead
\bottomrule\noalign{}
\endlastfoot
\textbf{EMF (Electromotive Force)} & The energy supplied by a source per
unit charge & Volt (V) \\
\textbf{Electric Current} & The rate of flow of electric charge & Ampere
(A) \\
\textbf{Power} & The rate at which electrical energy is transferred &
Watt (W) \\
\end{longtable}
}

\end{solutionbox}
\begin{mnemonicbox}
``EVA'' - EMF in Volts, Current in Amperes, Power in
Watts

\end{mnemonicbox}
\subsection*{Question 1(b) [4 marks]}\label{q1b}

\textbf{Three resistors having resistances of 1000 Ω, 2000 Ω and 3000 Ω
respectively are connected in series. Find the equivalent resistance of
this series connection. Now these three resistors are connected in
parallel. Find the equivalent resistance of this parallel connection.}

\begin{solutionbox}

\textbf{For Series Connection:}

\begin{verbatim}
Req = R1 + R2 + R3
Req = 1000 Ω + 2000 Ω + 3000 Ω
Req = 6000 Ω
\end{verbatim}

\textbf{For Parallel Connection:}

\begin{verbatim}
1/Req = 1/R1 + 1/R2 + 1/R3
1/Req = 1/1000 + 1/2000 + 1/3000
1/Req = 0.001 + 0.0005 + 0.00033
1/Req = 0.00183
Req = 545.45 Ω
\end{verbatim}

\textbf{Diagram:}

\begin{center}
\textbf{Mermaid Diagram (Code)}
\begin{verbatim}
{Shaded}
{Highlighting}[]
graph LR
    A[Input] {-{-}{-} B[1000 Ω]}
    B {-{-}{-} C[2000 Ω]}
    C {-{-}{-} D[3000 Ω]}
    D {-{-}{-} E[Output]}

    F[Input] {-{-}{-} G[1000 Ω] {-}{-}{-} H[Output]}
    F {-{-}{-} I[2000 Ω] {-}{-}{-} H}
    F {-{-}{-} J[3000 Ω] {-}{-}{-} H}
{Highlighting}
{Shaded}
\end{verbatim}
\end{center}

\end{solutionbox}
\begin{mnemonicbox}
``Series Sum, Parallel Product/Sum'' - In series add
directly, in parallel take reciprocal sum

\end{mnemonicbox}
\subsection*{Question 1(c) [7 marks]}\label{q1c}

\textbf{Write the definition of Resistor, Capacitor and Inductor. Draw
their symbols and write their units. Also write the use of each device
in electrical circuit.}

\begin{solutionbox}

{\def\LTcaptype{none} % do not increment counter
\begin{longtable}[]{@{}
  >{\raggedright\arraybackslash}p{(\linewidth - 8\tabcolsep) * \real{0.2075}}
  >{\raggedright\arraybackslash}p{(\linewidth - 8\tabcolsep) * \real{0.2264}}
  >{\raggedright\arraybackslash}p{(\linewidth - 8\tabcolsep) * \real{0.1509}}
  >{\raggedright\arraybackslash}p{(\linewidth - 8\tabcolsep) * \real{0.1132}}
  >{\raggedright\arraybackslash}p{(\linewidth - 8\tabcolsep) * \real{0.3019}}@{}}
\toprule\noalign{}
\begin{minipage}[b]{\linewidth}\raggedright
Component
\end{minipage} & \begin{minipage}[b]{\linewidth}\raggedright
Definition
\end{minipage} & \begin{minipage}[b]{\linewidth}\raggedright
Symbol
\end{minipage} & \begin{minipage}[b]{\linewidth}\raggedright
Unit
\end{minipage} & \begin{minipage}[b]{\linewidth}\raggedright
Use in Circuit
\end{minipage} \\
\midrule\noalign{}
\endhead
\bottomrule\noalign{}
\endlastfoot
\textbf{Resistor} & A component that opposes the flow of electric
current & ⊥⊥⊥ & Ohm (Ω) & Limits current, divides voltage, generates
heat \\
\textbf{Capacitor} & A component that stores electric charge & ⊢⊣ &
Farad (F) & Blocks DC, passes AC, energy storage, filtering \\
\textbf{Inductor} & A component that stores energy in magnetic field &
\otimes\otimes\otimes & Henry (H) & Blocks AC, passes DC, energy storage, filtering \\
\end{longtable}
}

\textbf{Diagram:}

\begin{verbatim}
+{-{-}{-}{-}{-}+    +{-}{-}{-}{-}{-}+     +{-}{-}{-}{-}{-}+}
|     |    |     |     |    |
| ⊥⊥⊥ |    | ⊢⊣ |     |    |
|     |    |     |     |    |
+{-{-}{-}{-}{-}+    +{-}{-}{-}{-}{-}+     +{-}{-}{-}{-}{-}+}
Resistor   Capacitor   Inductor
\end{verbatim}

\end{solutionbox}
\begin{mnemonicbox}
``RCI'' - Resistor Controls current, Capacitor stores
charge, Inductor stores magnetic energy

\end{mnemonicbox}
\subsection*{Question 1(c OR) [7
marks]}\label{question-1c-or-7-marks}

\textbf{State Ohm's law and write the equation of Ohm's law with circuit
diagram. Write applications of Ohm's law. Also write the limitation of
Ohm's law.}

\begin{solutionbox}

\textbf{Ohm's Law:} The current flowing through a conductor is directly
proportional to the voltage across it and inversely proportional to its
resistance.

\textbf{Equation:} V = I \times R

\textbf{Circuit Diagram:}

\begin{center}
\textbf{Mermaid Diagram (Code)}
\begin{verbatim}
{Shaded}
{Highlighting}[]
graph LR
    A[Voltage Source V] {-{-}{-} B[Resistor R]}
    B {-{-}{-} C[Current I]}
    C {-{-}{-} A}
{Highlighting}
{Shaded}
\end{verbatim}
\end{center}

\textbf{Applications of Ohm's Law:}

\begin{itemize}
\tightlist
\item
  Calculating current, voltage, or resistance in circuits
\item
  Designing electrical and electronic circuits
\item
Power calculations (P = V \times

I = I^{2} \times

R = V^{2}/R)

\item
  Circuit analysis using voltage divider and current divider
\end{itemize}

\textbf{Limitations of Ohm's Law:}

\begin{itemize}
\tightlist
\item
  Not applicable for non-linear devices (diodes, transistors)
\item
  Not valid for high-frequency AC circuits
\item
  Not valid for non-metallic conductors
\item
  Does not apply during transient conditions
\end{itemize}

\end{solutionbox}
\begin{mnemonicbox}
``VIR'' - Voltage equals current times resistance

\end{mnemonicbox}
\subsection*{Question 2(a) [3 marks]}\label{q2a}

\textbf{Explain the generation of alternating EMF with the help of
necessary diagram and equation.}

\begin{solutionbox}

Alternating EMF is generated when a conductor rotates in a magnetic
field.

\textbf{Equation:} e = E_{0} sin(ωt) = E_{0} sin(2πft)

Where:

\begin{itemize}
\tightlist
\item
  e = instantaneous EMF
\item
  E_{0} = maximum EMF
\item
  ω = angular velocity (2πf)
\item
  f = frequency
\item
  t = time
\end{itemize}

\textbf{Diagram:}

\begin{center}
\textbf{Mermaid Diagram (Code)}
\begin{verbatim}
{Shaded}
{Highlighting}[]
graph LR
    A[Magnetic Field] {-{-}{-} B[Rotating Coil]}
    B {-{-}{-} C[Slip Rings]}
    C {-{-}{-} D[Brushes]}
    D {-{-}{-} E[AC Output]}
{Highlighting}
{Shaded}
\end{verbatim}
\end{center}

\end{solutionbox}
\begin{mnemonicbox}
``RCBS'' - Rotation of Coil in magnetic field
produces sinusoidal EMF

\end{mnemonicbox}
\subsection*{Question 2(b) [4 marks]}\label{q2b}

\textbf{Explain the behavior of pure capacitor with AC supply with
necessary circuit diagram and equation.}

\begin{solutionbox}

\textbf{Behavior of Pure Capacitor with AC:}

\begin{itemize}
\tightlist
\item
  Current leads voltage by 90^\circ in a pure capacitor
\item
  Capacitive reactance (Xc) = 1/(2πfC)
\item
  As frequency increases, reactance decreases
\item
  Stores energy in electric field during charging
\end{itemize}

\textbf{Circuit and Waveform:}

\begin{verbatim}
    +       +
    |       |
 AC |       | C
    |       |
    +       +

 Voltage
    |    /{}
    |   /  {}
    |  /    {    Current}
    | /      {    /}
    |/        {  /  }
{-{-}{-}{-}+{-}{-}{-}{-}{-}{-}{-}{-}{-}{-}/{-}{-}{-}{-}{-}{-}{-}{-}+{-}{-}{-}{-}}
    |{        /|        /}
    | {      / |       /}
    |  {    /  |}
    |   {  /   |}
    |    {/    |}
\end{verbatim}

\textbf{Equation:} I = C \times dV/dt

\end{solutionbox}
\begin{mnemonicbox}
``CIVIC'' - Capacitor's current Is ahead of Voltage
by 90^\circ In Circuit

\end{mnemonicbox}
\subsection*{Question 2(c) [7 marks]}\label{q2c}

\textbf{An AC voltage is expressed as 300 Sin (628t) V. Find (i)
Amplitude (ii) Frequency (iii) Time period (iv) Average value (v) RMS
Value (vi) Form Factor and (vii) Peak Factor for this AC voltage.}

\begin{solutionbox}

Given: v = 300 Sin(628t) V

{\def\LTcaptype{none} % do not increment counter
\begin{longtable}[]{@{}llll@{}}
\toprule\noalign{}
Parameter & Formula & Calculation & Result \\
\midrule\noalign{}
\endhead
\bottomrule\noalign{}
\endlastfoot
\textbf{Amplitude} & V_{0} & 300 V & 300 V \\
\textbf{Angular Frequency} & ω & 628 rad/s & 628 rad/s \\
\textbf{Frequency} & f = ω/2π & 628/2π = 628/6.28 & 100 Hz \\
\textbf{Time Period} & T = 1/f & 1/100 & 0.01 s \\
\textbf{Average Value} & Vavg = 2V_{0}/π & 2\times300/π = 600/3.14 & 191 V \\
\textbf{RMS Value} & Vrms = V_{0}/\sqrt2 & 300/1.414 & 212.16 V \\
\textbf{Form Factor} & FF = Vrms/Vavg & 212.16/191 & 1.11 \\
\textbf{Peak Factor} & PF = V_{0}/Vrms & 300/212.16 & 1.414 \\
\end{longtable}
}

\end{solutionbox}
\begin{mnemonicbox}
``FART FAFP'' - Frequency is Angular frequency
divided by 2π, RMS is peak divided by root 2, Time period is 1/f, Form
factor is 1.11, Average is 2V_{m}/π, Peak factor is 1.414

\end{mnemonicbox}
\subsection*{Question 2(a OR) [3
marks]}\label{question-2a-or-3-marks}

\textbf{Explain the generation of 3-phase alternating EMF.}

\begin{solutionbox}

3-phase alternating EMF is generated using three separate coils placed
120^\circ apart in a magnetic field.

\textbf{Key Points:}

\begin{itemize}
\tightlist
\item
  Three identical coils are placed 120^\circ apart
\item
  Each coil produces sinusoidal EMF
\item
  Phases are labeled as R, Y, and B (or U, V, W)
\item
  Phase difference between any two phases is 120^\circ
\end{itemize}

\textbf{Diagram:}

\begin{center}
\textbf{Mermaid Diagram (Code)}
\begin{verbatim}
{Shaded}
{Highlighting}[]
graph LR
    A[Rotating Magnet] {-{-}{-} B[Three Coils 120^ Apart]}
    B {-{-}{-} C[Three{-}Phase Output]}
    D[Time] {-{-}{-} E[Three Phase Waveforms]}
{Highlighting}
{Shaded}
\end{verbatim}
\end{center}

\end{solutionbox}
\begin{mnemonicbox}
``THREE'' - Three coils Have 120^\circ Rotating EMF Each

\end{mnemonicbox}
\subsection*{Question 2(b OR) [4
marks]}\label{question-2b-or-4-marks}

\textbf{Explain the behavior of pure inductor with AC supply with
necessary circuit diagram and equation.}

\begin{solutionbox}

\textbf{Behavior of Pure Inductor with AC:}

\begin{itemize}
\tightlist
\item
  Current lags voltage by 90^\circ in a pure inductor
\item
  Inductive reactance (XL) = 2πfL
\item
  As frequency increases, reactance increases
\item
  Stores energy in magnetic field
\end{itemize}

\textbf{Circuit and Waveform:}

\begin{verbatim}
    +       +
    |       |
 AC |       | L
    |       |
    +       +

 Voltage
    |    /{}
    |   /  {}
    |  /    {}
    | /      {    Current}
    |/        {    /}
{-{-}{-}{-}+{-}{-}{-}{-}{-}{-}{-}{-}{-}{-}{-}{-}/{-}{-}{-}{-}{-}{-}+{-}{-}{-}{-}}
    |{          /      /}
    | {         /|     /}
    |  {       / |}
    |   {     /  |}
    |    {   /   |}
    |     { /    |}
    |      V     |
\end{verbatim}

\textbf{Equation:} V = L \times dI/dt

\end{solutionbox}
\begin{mnemonicbox}
``VLIC'' - Voltage Leads current by 90^\circ In inductor
Circuit

\end{mnemonicbox}
\subsection*{Question 2(c OR) [7
marks]}\label{question-2c-or-7-marks}

\textbf{Define phase voltage, line voltage, phase current and line
current for 3-phase AC. (i) Calculate the line voltage for star (Y)
connection if the phase voltage is 100V. Also find the line current for
star (Y) connection if the phase current is 5A (ii) Calculate the line
voltage for delta (Δ) connection if the phase voltage is 100V. Also find
the line current for delta (Δ) connection if the phase current is 5A.}

\begin{solutionbox}

{\def\LTcaptype{none} % do not increment counter
\begin{longtable}[]{@{}ll@{}}
\toprule\noalign{}
Term & Definition \\
\midrule\noalign{}
\endhead
\bottomrule\noalign{}
\endlastfoot
\textbf{Phase Voltage} & Voltage across a single phase element \\
\textbf{Line Voltage} & Voltage between any two lines \\
\textbf{Phase Current} & Current flowing through a phase element \\
\textbf{Line Current} & Current flowing through a line \\
\end{longtable}
}

\textbf{Star (Y) Connection:}

\begin{itemize}
\tightlist
\item
  Line voltage = \sqrt3 \times Phase voltage
\item
  Line current = Phase current
\end{itemize}

Calculations:

\begin{itemize}
\tightlist
\item
  Line voltage = \sqrt3 \times 100 = 173.2 V
\item
  Line current = 5 A
\end{itemize}

\textbf{Delta (Δ) Connection:}

\begin{itemize}
\tightlist
\item
  Line voltage = Phase voltage
\item
  Line current = \sqrt3 \times Phase current
\end{itemize}

Calculations:

\begin{itemize}
\tightlist
\item
  Line voltage = 100 V
\item
  Line current = \sqrt3 \times 5 = 8.66 A
\end{itemize}

\textbf{Diagram:}

\begin{center}
\textbf{Mermaid Diagram (Code)}
\begin{verbatim}
{Shaded}
{Highlighting}[]
graph TD
    subgraph Star Connection
    A1((R)) {-{-}{-} B1((Y))}
    B1 {-{-}{-} C1((B))}
    C1 {-{-}{-} A1}
    D1((N)) {-{-}{-} A1}
    D1 {-{-}{-} B1}
    D1 {-{-}{-} C1}
    end

    subgraph Delta Connection
    A2((R)) {-{-}{-} B2((Y))}
    B2 {-{-}{-} C2((B))}
    C2 {-{-}{-} A2}
    end
{Highlighting}
{Shaded}
\end{verbatim}
\end{center}

\end{solutionbox}
\begin{mnemonicbox}
``SLIP'' - Star: Line voltage is \sqrt3 times Phase
voltage, In Delta: Phase voltage equals Line voltage

\end{mnemonicbox}
\subsection*{Question 3(a) [3 marks]}\label{q3a}

\textbf{State and explain Faraday's laws of electromagnetic induction
with necessary diagram and equations.}

\begin{solutionbox}

\textbf{Faraday's Laws:}

\begin{enumerate}
\tightlist
\item
  \textbf{First Law:} When a conductor cuts magnetic flux, EMF is
  induced
\item
  \textbf{Second Law:} The magnitude of induced EMF is proportional to
  the rate of change of magnetic flux
\end{enumerate}

\textbf{Equation:} e = -N \times (dΦ/dt) Where: e = induced EMF, N = number
of turns, dΦ/dt = rate of change of flux

\textbf{Diagram:}

\begin{center}
\textbf{Mermaid Diagram (Code)}
\begin{verbatim}
{Shaded}
{Highlighting}[]
graph LR
    A[Moving Magnet] {-{-}{-} B[Coil]}
    B {-{-}{-} C[Galvanometer]}
    D[Changing Magnetic Field] {-{-}{-} E[Induced EMF]}
{Highlighting}
{Shaded}
\end{verbatim}
\end{center}

\end{solutionbox}
\begin{mnemonicbox}
``FIRE'' - Flux change Induces Rapid EMF

\end{mnemonicbox}
\subsection*{Question 3(b) [4 marks]}\label{q3b}

\textbf{Define amplitude, frequency, time duration and RMS value for
alternating quantity.}

\begin{solutionbox}

{\def\LTcaptype{none} % do not increment counter
\begin{longtable}[]{@{}
  >{\raggedright\arraybackslash}p{(\linewidth - 4\tabcolsep) * \real{0.3438}}
  >{\raggedright\arraybackslash}p{(\linewidth - 4\tabcolsep) * \real{0.3750}}
  >{\raggedright\arraybackslash}p{(\linewidth - 4\tabcolsep) * \real{0.2812}}@{}}
\toprule\noalign{}
\begin{minipage}[b]{\linewidth}\raggedright
Parameter
\end{minipage} & \begin{minipage}[b]{\linewidth}\raggedright
Definition
\end{minipage} & \begin{minipage}[b]{\linewidth}\raggedright
Formula
\end{minipage} \\
\midrule\noalign{}
\endhead
\bottomrule\noalign{}
\endlastfoot
\textbf{Amplitude} & Maximum value of the alternating quantity & V_{m} \\
\textbf{Frequency} & Number of complete cycles per second & f = 1/T \\
\textbf{Time Period} & Time taken to complete one cycle & T = 1/f \\
\textbf{RMS Value} & Effective value, equivalent to DC causing same
heating & Vrms = V_{m}/\sqrt2 = 0.707V_{m} \\
\end{longtable}
}

\textbf{Diagram:}

\begin{verbatim}
    Amplitude
        \^{}
        |    /|{}
        |   / | {}
        |  /  |  {}
        | /   |   {}
        |/    |    {}
{-{-}{-}{-}{-}{-}{-}{-}+{-}{-}{-}{-}{-}+{-}{-}{-}{-}{-}{-}{-}{-}{-}{-}{-}{-}+{-}{-}{-}{-}}
        |{    |     /      |}
        | {   |    /       |}
        |  {  |   /        |}
        |   { |  /         |}
        |    {|/           |}
        |                 { |}
        |                  {|}
        |                   
        |{{-}Time Period T {-}|}
\end{verbatim}

\end{solutionbox}
\begin{mnemonicbox}
``AFTR'' - Amplitude is peak, Frequency is cycles per
second, Time period is 1/f, RMS is 0.707 times peak

\end{mnemonicbox}
\subsection*{Question 3(c) [7 marks]}\label{q3c}

\textbf{Explain self inductance and mutual inductance. (i) Find the self
induction of the coil if total magnetic flux linked with the coil is 5
μWb-turns (micro Wb-turns) for 2 A current given to the coil (ii) Find
the self induction of the coil, if the parameters of the coils are as
follows: number of turns is 10, relative permeability of the material
used for coil is 3, length of the coil is 5 cm and cross sectional area
of coil is 2 cm^{2}.}

\begin{solutionbox}

\textbf{Self Inductance:} Property of a coil to oppose change in current
through it by inducing EMF in itself.

\textbf{Mutual Inductance:} Property of one coil to induce EMF in
another coil due to change in current.

\textbf{Part (i):}

\begin{verbatim}
Self inductance (L) = Flux linkage / Current
L = 5 μWb-turns / 2 A
L = 2.5 μH
\end{verbatim}

\textbf{Part (ii):}

\begin{verbatim}
L = (μ_{o} \times μᵣ \times N^{2} \times A) / l
L = (4π \times 10^{-}^{7} \times 3 \times 10^{2} \times 2 \times 10^{-}^{4}) / (5 \times 10^{-}^{2})
L = (4π \times 3 \times 100 \times 2 \times 10^{-}^{7}) / (5 \times 10^{-}^{2})
L = (24π \times 10^{-}^{5}) / (5 \times 10^{-}^{2})
L = 24π \times 10^{-}^{3} / 5
L = 4.8π \times 10^{-}^{3}
L = 15.07 μH
\end{verbatim}

\textbf{Diagram:}

\begin{center}
\textbf{Mermaid Diagram (Code)}
\begin{verbatim}
{Shaded}
{Highlighting}[]
graph TD
    subgraph Self Inductance
    A[Current in Coil] {-{-}{} B[Magnetic Field]}
    B {-{-}{} C[EMF in Same Coil]}
    end

    subgraph Mutual Inductance
    D[Current in Coil 1] {-{-}{} E[Magnetic Field]}
    E {-{-}{} F[EMF in Coil 2]}
    end
{Highlighting}
{Shaded}
\end{verbatim}
\end{center}

\end{solutionbox}
\begin{mnemonicbox}
``SLIM'' - Self inductance Linked with own flux,
Induction Mutual between two coils

\end{mnemonicbox}
\subsection*{Question 3(a OR) [3
marks]}\label{question-3a-or-3-marks}

\textbf{Define dynamically induced EMF. Explain it with the help of
necessary diagram and equation.}

\begin{solutionbox}

\textbf{Dynamically Induced EMF:} EMF induced in a conductor due to
relative motion between the conductor and magnetic field.

\textbf{Equation:} e = Blv Where: e = induced EMF, B = magnetic flux
density,

l = length of conductor,

v = velocity of conductor


\textbf{Diagram:}

\begin{center}
\textbf{Mermaid Diagram (Code)}
\begin{verbatim}
{Shaded}
{Highlighting}[]
graph LR
    A[Magnetic Field B] {-{-}{-} B[Moving Conductor]}
    B {-{-}{-} C[Induced EMF e]}
    D[Motion with velocity v] {-{-}{-} B}
{Highlighting}
{Shaded}
\end{verbatim}
\end{center}

\end{solutionbox}
\begin{mnemonicbox}
``MOVE'' - Motion Of conductor in magnetic field
produces Voltage Effect

\end{mnemonicbox}
\subsection*{Question 3(b OR) [4
marks]}\label{question-3b-or-4-marks}

\textbf{Define cycle, Form Factor and Peak Factor for alternating
quantity. Write the value of Form Factor and Peak Factor for sinusoidal
alternating quantity.}

\begin{solutionbox}

{\def\LTcaptype{none} % do not increment counter
\begin{longtable}[]{@{}lll@{}}
\toprule\noalign{}
Term & Definition & Value for Sinusoidal Wave \\
\midrule\noalign{}
\endhead
\bottomrule\noalign{}
\endlastfoot
\textbf{Cycle} & One complete oscillation of an alternating quantity &
- \\
\textbf{Form Factor} & Ratio of RMS value to average value & 1.11 \\
\textbf{Peak Factor} & Ratio of maximum value to RMS value & 1.414 \\
\end{longtable}
}

\textbf{Diagram:}

\begin{verbatim}
    \^{}
    |    /|{}
    |   / | {     One Cycle}
    |  /  |  {    {-}{-}{-}{-}{-}{-}{-}{-}{-}{-}{-}}
    | /   |   {}
    |/    |    {}
{-{-}{-}{-}+{-}{-}{-}{-}{-}+{-}{-}{-}{-}{-}{-}{-}{-}{-}{-}{-}{-}+{-}{-}{-}{-}}
    |{    |     /      |}
    | {   |    /       |}
    |  {  |   /        |}
    |   { |  /         |}
    |    {|/           |}
    |                 { |}
    |                  {|}
    
    Form Factor = Vrms/Vavg = 1.11
    Peak Factor = Vm/Vrms = 1.414
\end{verbatim}

\end{solutionbox}
\begin{mnemonicbox}
``CFP'' - Cycle is one oscillation, Form factor is
1.11, Peak factor is 1.414

\end{mnemonicbox}
\subsection*{Question 3(c OR) [7
marks]}\label{question-3c-or-7-marks}

\textbf{State and explain Lenz's law. State and explain Fleming's right
hand rule for generator. Find the energy stored in inductor having self
inductance of 4 μH, if 3 A of current is flowing through the inductor.}

\begin{solutionbox}

\textbf{Lenz's Law:} The direction of induced EMF is such that it
opposes the change in magnetic flux that produces it.

\textbf{Fleming's Right Hand Rule:}

\begin{itemize}
\tightlist
\item
  Thumb: Direction of motion of conductor
\item
  Index finger: Direction of magnetic field
\item
  Middle finger: Direction of induced current
\end{itemize}

\textbf{Energy Calculation:}

\begin{verbatim}
Energy stored in inductor (W) = (1/2) \times L \times I^{2}
W = (1/2) \times 4 \times 10^{-}^{6} \times 3^{2}
W = (1/2) \times 4 \times 10^{-}^{6} \times 9
W = 18 \times 10^{-}^{6} / 2
W = 9 \times 10^{-}^{6} Joules
W = 9 μJ
\end{verbatim}

\textbf{Diagram:}

\begin{verbatim}
   Fleming{s Right Hand Rule:}
   
   Thumb (Motion) 
   Index (Field) ↑
   Middle (Current) ↻
   
   Lenz{s Law:}
   
   N[==={]S    (Conductor)}
   Induced current opposes motion
\end{verbatim}

\end{solutionbox}
\begin{mnemonicbox}
``LOF'' - Lenz's law Opposes Flux change, Fleming's
rule - thumb Motion, index Field, middle Current

\end{mnemonicbox}
\subsection*{Question 4(a) [3 marks]}\label{q4a}

\textbf{Define PV cell. Explain the function of PV cell.}

\begin{solutionbox}

\textbf{PV Cell:} Photovoltaic cell is a semiconductor device that
converts light energy directly into electrical energy.

\textbf{Function:}

\begin{itemize}
\tightlist
\item
  Absorbs photons from sunlight
\item
  Creates electron-hole pairs in semiconductor
\item
  Generates potential difference at p-n junction
\item
  Converts solar energy to electrical energy
\end{itemize}

\textbf{Diagram:}

\begin{center}
\textbf{Mermaid Diagram (Code)}
\begin{verbatim}
{Shaded}
{Highlighting}[]
graph LR
    A[Sunlight] {-{-}{} B[PV Cell]}
    B {-{-}{} C[DC Electricity]}
    D[P{-type Silicon] {-}{-}{-} E[N{-}type Silicon]}
{Highlighting}
{Shaded}
\end{verbatim}
\end{center}

\end{solutionbox}
\begin{mnemonicbox}
``PASE'' - PV cell Absorbs Sunlight to generate
Electricity

\end{mnemonicbox}
\subsection*{Question 4(b) [4 marks]}\label{q4b}

\textbf{Explain the classification of green energy.}

\begin{solutionbox}

{\def\LTcaptype{none} % do not increment counter
\begin{longtable}[]{@{}lll@{}}
\toprule\noalign{}
Green Energy Type & Source & Example Applications \\
\midrule\noalign{}
\endhead
\bottomrule\noalign{}
\endlastfoot
\textbf{Solar Energy} & Sun & PV panels, solar thermal \\
\textbf{Wind Energy} & Air currents & Wind turbines \\
\textbf{Hydro Energy} & Flowing water & Dams, tidal, wave \\
\textbf{Biomass Energy} & Organic matter & Biofuels, biogas \\
\textbf{Geothermal Energy} & Earth's heat & Geothermal plants \\
\end{longtable}
}

\textbf{Diagram:}

\begin{center}
\textbf{Mermaid Diagram (Code)}
\begin{verbatim}
{Shaded}
{Highlighting}[]
graph TD
    A[Green Energy] {-{-}{} B[Solar]}
    A {-{-}{} C[Wind]}
    A {-{-}{} D[Hydro]}
    A {-{-}{} E[Biomass]}
    A {-{-}{} F[Geothermal]}
{Highlighting}
{Shaded}
\end{verbatim}
\end{center}

\end{solutionbox}
\begin{mnemonicbox}
``SWHBG'' - Sun, Wind, Hydro, Biomass, Geothermal
energy sources

\end{mnemonicbox}
\subsection*{Question 4(c) [7 marks]}\label{q4c}

\textbf{Draw and explain the block diagram of solar power system.}

\begin{solutionbox}

\textbf{Solar Power System Components:}

{\def\LTcaptype{none} % do not increment counter
\begin{longtable}[]{@{}
  >{\raggedright\arraybackslash}p{(\linewidth - 2\tabcolsep) * \real{0.5238}}
  >{\raggedright\arraybackslash}p{(\linewidth - 2\tabcolsep) * \real{0.4762}}@{}}
\toprule\noalign{}
\begin{minipage}[b]{\linewidth}\raggedright
Component
\end{minipage} & \begin{minipage}[b]{\linewidth}\raggedright
Function
\end{minipage} \\
\midrule\noalign{}
\endhead
\bottomrule\noalign{}
\endlastfoot
\textbf{Solar Panel} & Converts sunlight to DC electricity \\
\textbf{Charge Controller} & Regulates battery charging and prevents
overcharging \\
\textbf{Battery Bank} & Stores electricity for later use \\
\textbf{Inverter} & Converts DC to AC for household appliances \\
\textbf{Distribution Panel} & Distributes electricity to loads \\
\textbf{Grid Connection} & Optional connection to utility grid \\
\end{longtable}
}

\textbf{Block Diagram:}

\begin{verbatim}
flowchart LR
    A[Solar Panels] {-{-} B[Charge Controller]}
    B {-{-} C[Battery Bank]}
    C {-{-} D[Inverter]}
    D {-{-} E[Distribution Panel]}
    E {-{-} F[Home Appliances]}
    E {-.{-} G[Grid Connection]}
\end{verbatim}

\end{solutionbox}
\begin{mnemonicbox}
``SCBIDG'' - Solar panels, Charge controller,
Batteries, Inverter, Distribution, Grid

\end{mnemonicbox}
\subsection*{Question 4(a OR) [3
marks]}\label{question-4a-or-3-marks}

\textbf{Define green energy, conventional energy and renewable energy.}

\begin{solutionbox}

{\def\LTcaptype{none} % do not increment counter
\begin{longtable}[]{@{}
  >{\raggedright\arraybackslash}p{(\linewidth - 2\tabcolsep) * \real{0.3333}}
  >{\raggedright\arraybackslash}p{(\linewidth - 2\tabcolsep) * \real{0.6667}}@{}}
\toprule\noalign{}
\begin{minipage}[b]{\linewidth}\raggedright
Term
\end{minipage} & \begin{minipage}[b]{\linewidth}\raggedright
Definition
\end{minipage} \\
\midrule\noalign{}
\endhead
\bottomrule\noalign{}
\endlastfoot
\textbf{Green Energy} & Energy from naturally replenished sources with
minimal environmental impact \\
\textbf{Conventional Energy} & Energy from traditional fossil fuel
sources like coal, oil, and natural gas \\
\textbf{Renewable Energy} & Energy from sources that are naturally
replenished on a human timescale \\
\end{longtable}
}

\textbf{Diagram:}

\begin{center}
\textbf{Mermaid Diagram (Code)}
\begin{verbatim}
{Shaded}
{Highlighting}[]
graph LR
    A[Energy Sources] {-{-}{} B[Green/Renewable]}
    A {-{-}{} C[Conventional/Non{-}renewable]}

    B {-{-}{} D[Solar, Wind, Hydro, etc.]}
    C {-{-}{} E[Coal, Oil, Natural Gas]}
{Highlighting}
{Shaded}
\end{verbatim}
\end{center}

\end{solutionbox}
\begin{mnemonicbox}
``GCR'' - Green is Clean, Conventional is
Carbon-emitting, Renewable is Replenished

\end{mnemonicbox}
\subsection*{Question 4(b OR) [4
marks]}\label{question-4b-or-4-marks}

\textbf{Explain the need of green energy.}

\begin{solutionbox}

\textbf{Need for Green Energy:}

{\def\LTcaptype{none} % do not increment counter
\begin{longtable}[]{@{}
  >{\raggedright\arraybackslash}p{(\linewidth - 2\tabcolsep) * \real{0.3158}}
  >{\raggedright\arraybackslash}p{(\linewidth - 2\tabcolsep) * \real{0.6842}}@{}}
\toprule\noalign{}
\begin{minipage}[b]{\linewidth}\raggedright
Need
\end{minipage} & \begin{minipage}[b]{\linewidth}\raggedright
Explanation
\end{minipage} \\
\midrule\noalign{}
\endhead
\bottomrule\noalign{}
\endlastfoot
\textbf{Environmental Protection} & Reduces pollution and greenhouse gas
emissions \\
\textbf{Resource Conservation} & Preserves limited fossil fuel
resources \\
\textbf{Energy Security} & Reduces dependence on imported fuels \\
\textbf{Economic Benefits} & Creates jobs and reduces energy costs
long-term \\
\textbf{Sustainable Development} & Meets present needs without
compromising future generations \\
\end{longtable}
}

\textbf{Diagram:}

\begin{center}
\textbf{Mermaid Diagram (Code)}
\begin{verbatim}
{Shaded}
{Highlighting}[]
graph TD
    A[Need for Green Energy] {-{-}{} B[Environmental Protection]}
    A {-{-}{} C[Resource Conservation]}
    A {-{-}{} D[Energy Security]}
    A {-{-}{} E[Economic Benefits]}
    A {-{-}{} F[Sustainable Development]}
{Highlighting}
{Shaded}
\end{verbatim}
\end{center}

\end{solutionbox}
\begin{mnemonicbox}
``ERESS'' - Environment, Resources, Energy security,
Savings, Sustainability

\end{mnemonicbox}
\subsection*{Question 4(c OR) [7
marks]}\label{question-4c-or-7-marks}

\textbf{Draw and explain the block diagram of wind power system with
types of turbines.}

\begin{solutionbox}

\textbf{Wind Power System Components:}

{\def\LTcaptype{none} % do not increment counter
\begin{longtable}[]{@{}ll@{}}
\toprule\noalign{}
Component & Function \\
\midrule\noalign{}
\endhead
\bottomrule\noalign{}
\endlastfoot
\textbf{Wind Turbine} & Converts wind energy to mechanical energy \\
\textbf{Gearbox} & Increases the rotational speed \\
\textbf{Generator} & Converts mechanical energy to electrical energy \\
\textbf{Controller} & Monitors and controls the system \\
\textbf{Transformer} & Steps up voltage for transmission \\
\textbf{Grid Connection} & Connects to the utility grid \\
\end{longtable}
}

\textbf{Types of Wind Turbines:}

\begin{enumerate}
\tightlist
\item
  \textbf{Horizontal Axis Wind Turbine (HAWT)} - Blades rotate around
  horizontal axis
\item
  \textbf{Vertical Axis Wind Turbine (VAWT)} - Blades rotate around
  vertical axis
\end{enumerate}

\textbf{Block Diagram:}

\begin{verbatim}
flowchart LR
    A[Wind] {-{-} B[Wind Turbine]}
    B {-{-} C[Gearbox]}
    C {-{-} D[Generator]}
    D {-{-} E[Controller]}
    E {-{-} F[Transformer]}
    F {-{-} G[Grid]}

    subgraph "Types of Turbines"
    H[Horizontal Axis]
    I[Vertical Axis]
    end
\end{verbatim}

\end{solutionbox}
\begin{mnemonicbox}
``WGGTC'' - Wind turns turbine, Gearbox speeds up,
Generator produces electricity, Transformer steps up, Controller manages

\end{mnemonicbox}
\subsection*{Question 5(a) [3 marks]}\label{q5a}

\textbf{Explain the factors affecting the value of resistance of a
resistor.}

\begin{solutionbox}

\textbf{Factors Affecting Resistance:}

{\def\LTcaptype{none} % do not increment counter
\begin{longtable}[]{@{}
  >{\raggedright\arraybackslash}p{(\linewidth - 2\tabcolsep) * \real{0.5000}}
  >{\raggedright\arraybackslash}p{(\linewidth - 2\tabcolsep) * \real{0.5000}}@{}}
\toprule\noalign{}
\begin{minipage}[b]{\linewidth}\raggedright
Factor
\end{minipage} & \begin{minipage}[b]{\linewidth}\raggedright
Effect
\end{minipage} \\
\midrule\noalign{}
\endhead
\bottomrule\noalign{}
\endlastfoot
\textbf{Temperature} & Resistance increases with temperature in
metals \\
\textbf{Length} & Resistance is directly proportional to length \\
\textbf{Cross-sectional Area} & Resistance is inversely proportional to
area \\
\textbf{Material} & Different materials have different resistivities \\
\end{longtable}
}

\textbf{Equation:} R = ρ \times (l/A)

Where:

\begin{itemize}
\tightlist
\item
  R = Resistance
\item
  ρ = Resistivity
\item
  l = Length
\item
  A = Cross-sectional area
\end{itemize}

\end{solutionbox}
\begin{mnemonicbox}
``TLAM'' - Temperature, Length, Area, Material affect
resistance

\end{mnemonicbox}
\subsection*{Question 5(b) [4 marks]}\label{q5b}

\textbf{Define active power, reactive power, apparent power and power
factor with the help of power triangle. Write their units.}

\begin{solutionbox}

{\def\LTcaptype{none} % do not increment counter
\begin{longtable}[]{@{}
  >{\raggedright\arraybackslash}p{(\linewidth - 6\tabcolsep) * \real{0.3077}}
  >{\raggedright\arraybackslash}p{(\linewidth - 6\tabcolsep) * \real{0.3077}}
  >{\raggedright\arraybackslash}p{(\linewidth - 6\tabcolsep) * \real{0.2308}}
  >{\raggedright\arraybackslash}p{(\linewidth - 6\tabcolsep) * \real{0.1538}}@{}}
\toprule\noalign{}
\begin{minipage}[b]{\linewidth}\raggedright
Power Type
\end{minipage} & \begin{minipage}[b]{\linewidth}\raggedright
Definition
\end{minipage} & \begin{minipage}[b]{\linewidth}\raggedright
Formula
\end{minipage} & \begin{minipage}[b]{\linewidth}\raggedright
Unit
\end{minipage} \\
\midrule\noalign{}
\endhead
\bottomrule\noalign{}
\endlastfoot
\textbf{Active Power (P)} & Actual power consumed & P = VI cosφ & Watt
(W) \\
\textbf{Reactive Power (Q)} & Power oscillating between source and load
& Q = VI sinφ & Volt-Ampere Reactive (VAR) \\
\textbf{Apparent Power (S)} & Product of voltage and current & S = VI &
Volt-Ampere (VA) \\
\textbf{Power Factor (PF)} & Ratio of active power to apparent power &
PF = P/S = cosφ & No unit (0 to 1) \\
\end{longtable}
}

\textbf{Power Triangle:}

\begin{verbatim}
            Q (VAR)
            |
            |
            |
            |       S (VA)
            |     /
            |   /
            | /
            +{-{-}{-}{-}{-}{-}{-}{-}{-}{-}{-}{-}{-}{-}{-} P (W)}
           /|
       PF=cosφ
\end{verbatim}

\end{solutionbox}
\begin{mnemonicbox}
``ARSP'' - Active is Real power in Watts, Reactive is
Stored power in VAR, S is total VA, PF is cosφ

\end{mnemonicbox}
\subsection*{Question 5(c) [7 marks]}\label{q5c}

\textbf{State and explain Kirchhoff's Voltage Law (KVL) and Kirchhoff's
Current Law (KCL) with the help of circuit diagram.}

\begin{solutionbox}

\textbf{Kirchhoff's Voltage Law (KVL):} The algebraic sum of all
voltages around any closed loop in a circuit is zero.

\textbf{Kirchhoff's Current Law (KCL):} The algebraic sum of all
currents entering and leaving a node is zero.

{\def\LTcaptype{none} % do not increment counter
\begin{longtable}[]{@{}lll@{}}
\toprule\noalign{}
Law & Equation & Application \\
\midrule\noalign{}
\endhead
\bottomrule\noalign{}
\endlastfoot
\textbf{KVL} & \sumV = 0 & Finding voltage in complex circuits \\
\textbf{KCL} & \sumI = 0 & Finding current distribution \\
\end{longtable}
}

\textbf{Circuit Diagrams:}

\begin{center}
\textbf{Mermaid Diagram (Code)}
\begin{verbatim}
{Shaded}
{Highlighting}[]
graph TD
    subgraph KVL
    A1(({+)) {-}{-}{-} B1[R1]}
    B1 {-{-}{-} C1[R2]}
    C1 {-{-}{-} D1[R3]}
    D1 {-{-}{-} A1}
    end

    subgraph KCL
    A2((Node)) {-{-}{-} B2[I1]}
    A2 {-{-}{-} C2[I2]}
    A2 {-{-}{-} D2[I3]}
    A2 {-{-}{-} E2[I4]}
    end
{Highlighting}
{Shaded}
\end{verbatim}
\end{center}

\textbf{KVL Example:} V_{1} + V_{2} + V_{3} = 0

\textbf{KCL Example:} I_{1} + I_{2} = I_{3} + I_{4}

\end{solutionbox}
\begin{mnemonicbox}
``VCL'' - Voltage around Closed Loop is zero,
Currents at a point sum to zero

\end{mnemonicbox}
\subsection*{Question 5(a OR) [3
marks]}\label{question-5a-or-3-marks}

\textbf{Write the difference between EMF and potential difference. Also
write the difference between cell and battery.}

\begin{solutionbox}

{\def\LTcaptype{none} % do not increment counter
\begin{longtable}[]{@{}
  >{\raggedright\arraybackslash}p{(\linewidth - 2\tabcolsep) * \real{0.6250}}
  >{\raggedright\arraybackslash}p{(\linewidth - 2\tabcolsep) * \real{0.3750}}@{}}
\toprule\noalign{}
\begin{minipage}[b]{\linewidth}\raggedright
EMF vs.~Potential Difference
\end{minipage} & \begin{minipage}[b]{\linewidth}\raggedright
Cell vs.~Battery
\end{minipage} \\
\midrule\noalign{}
\endhead
\bottomrule\noalign{}
\endlastfoot
\textbf{EMF}: Energy supplied by source per unit charge & \textbf{Cell}:
Single unit that converts chemical energy to electrical energy \\
\textbf{Potential Difference}: Energy consumed in external circuit &
\textbf{Battery}: Collection of two or more cells connected in series or
parallel \\
EMF exists even in open circuit & Cell has lower voltage (typically 1.5V
or 2V) \\
Potential difference exists only in closed circuit & Battery has higher
voltage output \\
\end{longtable}
}

\textbf{Diagram:}

\begin{verbatim}
EMF Source          Cell vs Battery
  +{-{-}{-}+              +{-}{-}{-}+    +{-}{-}{-}+{-}{-}{-}+{-}{-}{-}+}
  |   |              |   |    |   |   |   |
  | E |              | 1 |    | 1 | 2 | 3 |
  |   |              |   |    |   |   |   |
  +{-{-}{-}+              +{-}{-}{-}+    +{-}{-}{-}+{-}{-}{-}+{-}{-}{-}+}
                     Cell     Battery (Series)
\end{verbatim}

\end{solutionbox}
\begin{mnemonicbox}
``ESOP'' - EMF is Source energy, Open circuit too;
Potential difference is Operating energy

\end{mnemonicbox}
\subsection*{Question 5(b OR) [4
marks]}\label{question-5b-or-4-marks}

\textbf{Write the relation between AC voltage and AC current for pure
resistor, pure capacitor and pure inductor. Draw the vector diagram of
AC voltage and AC current for pure resistor, pure capacitor and pure
inductor. Also write the value of power factor for pure resistor, pure
capacitor and pure inductor.}

\begin{solutionbox}

{\def\LTcaptype{none} % do not increment counter
\begin{longtable}[]{@{}
  >{\raggedright\arraybackslash}p{(\linewidth - 6\tabcolsep) * \real{0.2075}}
  >{\raggedright\arraybackslash}p{(\linewidth - 6\tabcolsep) * \real{0.1887}}
  >{\raggedright\arraybackslash}p{(\linewidth - 6\tabcolsep) * \real{0.3396}}
  >{\raggedright\arraybackslash}p{(\linewidth - 6\tabcolsep) * \real{0.2642}}@{}}
\toprule\noalign{}
\begin{minipage}[b]{\linewidth}\raggedright
Component
\end{minipage} & \begin{minipage}[b]{\linewidth}\raggedright
Relation
\end{minipage} & \begin{minipage}[b]{\linewidth}\raggedright
Phase Difference
\end{minipage} & \begin{minipage}[b]{\linewidth}\raggedright
Power Factor
\end{minipage} \\
\midrule\noalign{}
\endhead
\bottomrule\noalign{}
\endlastfoot
\textbf{Pure Resistor} & V = IR & In phase (0^\circ) & 1 \\
\textbf{Pure Capacitor} & I = C(dV/dt) & Current leads voltage by 90^\circ &
0 (leading) \\
\textbf{Pure Inductor} & V = L(dI/dt) & Current lags voltage by 90^\circ & 0
(lagging) \\
\end{longtable}
}

\textbf{Vector Diagrams:}

\begin{verbatim}
   Resistor         Capacitor        Inductor
     V,I               I               V
      \^{                \^{}               \^{}}
      |                |               |
      |                |               |
      |                |               |
      +{-{-}{-}{-}{-}{-}        {-}+{-}           {-}{-}+{-}{-}}
                       V               I
\end{verbatim}

\end{solutionbox}
\begin{mnemonicbox}
``RCI'' - Resistor Current In phase, Capacitor
current leads, Inductor current lags

\end{mnemonicbox}
\subsection*{Question 5(c OR) [7
marks]}\label{question-5c-or-7-marks}

\textbf{Define temperature coefficient of material and write its unit.
Explain the effect of temperature on resistance of conductor with the
help of temperature coefficient of conductor.}

\begin{solutionbox}

\textbf{Temperature Coefficient:} The fractional change in resistance
per degree change in temperature.

\textbf{Unit:} per degree Celsius (^\circC^{-}^{1}) or per Kelvin (K^{-}^{1})

\textbf{Effect of Temperature on Resistance:}

\textbf{Equation:} R_{2} = R_{1}[1 + α(T_{2} - T_{1})]

Where:

\begin{itemize}
\tightlist
\item
  R_{1} = Resistance at temperature T_{1}
\item
  R_{2} = Resistance at temperature T_{2}
\item
  α = Temperature coefficient
\item
  T_{1}, T_{2} = Initial and final temperatures
\end{itemize}

\textbf{For Conductors (Metals):}

\begin{itemize}
\tightlist
\item
  Resistance increases with temperature (positive α)
\item
  Resistance decreases when temperature decreases
\end{itemize}

\textbf{For Semiconductors:}

\begin{itemize}
\tightlist
\item
  Resistance decreases with temperature (negative α)
\end{itemize}


{\def\LTcaptype{none} % do not increment counter
\begin{longtable}[]{@{}lll@{}}
\toprule\noalign{}
Material & Temperature Coefficient (α) per ^\circC & Behavior \\
\midrule\noalign{}
\endhead
\bottomrule\noalign{}
\endlastfoot
Copper & 0.0043 & Resistance increases with temperature \\
Aluminum & 0.0039 & Resistance increases with temperature \\
Nichrome & 0.0004 & Small change with temperature \\
Silicon & -0.07 & Resistance decreases with temperature \\
\end{longtable}
}

\textbf{Diagram:}

\begin{center}
\textbf{Mermaid Diagram (Code)}
\begin{verbatim}
{Shaded}
{Highlighting}[]
graph LR
    A[Temperature Increase] {-{-}{} B[Increased Atomic Vibrations]}
    B {-{-}{} C[More Electron Collisions]}
    C {-{-}{} D[Increased Resistance in Metals]}
{Highlighting}
{Shaded}
\end{verbatim}
\end{center}

\end{solutionbox}
\begin{mnemonicbox}
``TRIP'' - Temperature Raises resistance In
Proportion to coefficient

\end{mnemonicbox}

\end{document}
