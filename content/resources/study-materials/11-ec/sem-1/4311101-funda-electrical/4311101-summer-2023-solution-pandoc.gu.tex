\documentclass[10pt,a4paper]{article}

% content/resources/templates/preamble.tex
\usepackage[margin=0.6in]{geometry}
\author{Milav Dabgar}
\usepackage{amsmath,amssymb,amsthm}
\usepackage{booktabs}
\usepackage{multirow}
\usepackage{xcolor}
\usepackage{tcolorbox}
\tcbuselibrary{breakable,skins}
\usepackage[colorlinks=true,linkcolor=blue]{hyperref}
\usepackage{titlesec}
\usepackage{enumitem}
\usepackage{tikz}
\usepackage{pgfplots}
\usepackage{circuitikz}
\usepackage[version=4]{mhchem}
\usepackage{longtable}
\usepackage{array}
\usepackage{float}
\usepackage{caption}
\usepackage{listings}

\lstset{
  basicstyle=\small\ttfamily,
  breaklines=true,
  breakatwhitespace=false,
  postbreak=\mbox{\textcolor{red}{$\hookrightarrow$}\space},
  float=false,
  numbers=left,
  numberstyle=\tiny\color{gray},
  numbersep=10pt,
  xleftmargin=2em,
  keywordstyle=\color{blue},
  commentstyle=\color{green!60!black},
  stringstyle=\color{purple},
  backgroundcolor=\color{gray!5},
  showstringspaces=false,
  tabsize=2,
  captionpos=b,
  keepspaces=true,
  columns=flexible
}

\pgfplotsset{compat=1.18}
\usetikzlibrary{shapes,arrows,positioning,calc,patterns,decorations.pathmorphing,decorations.markings,arrows.meta}

% Color scheme
\definecolor{headcolor}{RGB}{0,102,204}
\definecolor{keycolor}{RGB}{220,20,60}
\definecolor{solutioncolor}{RGB}{34,139,34}
\definecolor{mnemoniccolor}{RGB}{148,0,211}
\definecolor{codecolor}{RGB}{0,0,100}

% Spacing
\setlength{\parskip}{3pt}
\setlist[itemize]{nosep}
\setlist[enumerate]{nosep}

% Title formatting
\titleformat{\section}{\Large\bfseries\color{headcolor}}{\thesection}{1em}{}
\titleformat{\subsection}{\large\bfseries\color{headcolor}}{\thesubsection}{1em}{}

% Pandoc tightlist compatibility
\providecommand{\tightlist}{%
  \setlength{\itemsep}{0pt}\setlength{\parskip}{0pt}}

% Pandoc longtable compatibility
\newcounter{none}
\def\thenone{}


% content/resources/templates/gujarati-boxes.tex
\usepackage{fontspec}
\usepackage{polyglossia}

% Set Gujarati as main language (document is primarily in Gujarati)
% Note: gloss-gujarati.ldf doesn't exist in polyglossia, but it will use hyphenation patterns
\setdefaultlanguage{gujarati}
\setotherlanguage{english}

% Configure Gujarati font properly
% Use Language=Default to prevent polyglossia from trying to add language-specific features
% that don't exist for Gujarati, which causes "empty feature" warnings
\newfontfamily\gujaratifont[Script=Gujarati,AutoFakeBold=2.5,AutoFakeSlant=0.3]{Noto Sans Gujarati}
\setmainfont[Script=Gujarati,AutoFakeBold=2.5,AutoFakeSlant=0.3]{Noto Sans Gujarati}
% Use Noto Sans Gujarati for monospace to support Gujarati in text
\setmonofont[Scale=0.9]{Noto Sans Gujarati}

% Configure English to use the same font
\newfontfamily\englishfont[Script=Gujarati,AutoFakeBold=2.5,AutoFakeSlant=0.3]{Noto Sans Gujarati}

% Translations for polyglossia
\gappto\captionsgujarati{
  \renewcommand{\tablename}{કોષ્ટક}
  \renewcommand{\figurename}{આકૃતિ}
}

% Helper for TikZ nodes to ensure Gujarati font
\newcommand{\gu}[1]{{\gujaratifont #1}}

% Custom environments
\newtcolorbox{solutionbox}{
    breakable,
    enhanced,
    colback=solutioncolor!5!white,
    colframe=solutioncolor!75!black,
    fonttitle=\bfseries,
    title=જવાબ
}

\newtcolorbox{solutionboxnobreak}{
 colback=solutioncolor!5!white,
 colframe=solutioncolor!75!black,
 fonttitle=\bfseries,
 title=જવાબ
}

\newtcolorbox{keyformula}{
 breakable,
 enhanced,
 colback=keycolor!5!white,
 colframe=keycolor!75!black,
 fonttitle=\bfseries,
 title=રાસાયણિક સમીકરણ/સૂત્ર
}

\newtcolorbox{mnemonicbox}{
 breakable,
 enhanced,
 colback=mnemoniccolor!5!white,
 colframe=mnemoniccolor!75!black,
 fonttitle=\bfseries,
 title=મેમરી ટ્રીક
}


\begin{document}

\begin{center}
{\Huge\bfseries\color{headcolor} Subject Name (Gujarati)}\\[5pt]
{\LARGE 4311101 -- Summer 2023}\\[3pt]
{\large Semester 1 Study Material}\\[3pt]
{\normalsize\textit{Detailed Solutions and Explanations}}
\end{center}

\vspace{10pt}

\subsection*{પ્રશ્ન 1(અ) [3
માર્ક્સ]}\label{uxaaauxab0uxab6uxaa8-1uxa85-3-uxaaeuxab0uxa95uxab8}

\textbf{નીચેનાની વ્યાખ્યા સમજાવો. (૧) રેસીસ્તંસ (૨) ઈલેક્ટ્રીકલ એનર્જી (૩)
ઈલેક્ટ્રીકલ પાવર}

\begin{solutionbox}

{\def\LTcaptype{none} % do not increment counter
\begin{longtable}[]{@{}
  >{\raggedright\arraybackslash}p{(\linewidth - 2\tabcolsep) * \real{0.3333}}
  >{\raggedright\arraybackslash}p{(\linewidth - 2\tabcolsep) * \real{0.6667}}@{}}
\toprule\noalign{}
\begin{minipage}[b]{\linewidth}\raggedright
શબ્દ
\end{minipage} & \begin{minipage}[b]{\linewidth}\raggedright
વ્યાખ્યા
\end{minipage} \\
\midrule\noalign{}
\endhead
\bottomrule\noalign{}
\endlastfoot
\textbf{રેસીસ્તંસ} & પદાર્થનો ગુણ જે વીજ પ્રવાહના પ્રવાહનો વિરોધ કરે છે, ઓહમ (Ω)માં
માપવામાં આવે છે \\
\textbf{ઈલેક્ટ્રીકલ એનર્જી} & વીજળી દ્વારા કાર્ય કરવાની ક્ષમતા, જૂલ (J) અથવા
કિલોવોટ-કલાક (kWh)માં માપવામાં આવે છે \\
\textbf{ઈલેક્ટ્રીકલ પાવર} & વીજળીની ઊર્જાના સ્થાનાંતરણ અથવા રૂપાંતરણનો દર, વોટ
(W)માં માપવામાં આવે છે \\
\end{longtable}
}

\end{solutionbox}
\begin{mnemonicbox}
``RIP'' - Resistance Impedes Path, Energy Is
Potential, Power Is Performance

\end{mnemonicbox}
\subsection*{પ્રશ્ન 1(બ) [4
માર્ક્સ]}\label{uxaaauxab0uxab6uxaa8-1uxaac-4-uxaaeuxab0uxa95uxab8}

\textbf{ઓહ્મ ના નિયમ નું વિધાન લખી સમજાઓ. તેની મર્યાદા લખો.}

\begin{solutionbox}

\textbf{ઓહ્મનો નિયમ}: કોઈ વાહક મારફતે વહેતો પ્રવાહ વાહકના બે છેડા વચ્ચેના
વિભવાંતરના સમપ્રમાણમાં અને વાહકના અવરોધના વ્યસ્ત પ્રમાણમાં હોય છે.

ગાણિતિક રીતે: V = IR, જ્યાં:

\begin{itemize}
\tightlist
\item
  V = વોલ્ટેજ (વોલ્ટ)
\item
  I = પ્રવાહ (એમ્પિયર)
\item
  R = અવરોધ (ઓહમ)
\end{itemize}

\begin{verbatim}
flowchart LR
    V[Voltage] {-{-} I[Current]}
    R[Resistance] {-{-}Limits{-}{-} I}
\end{verbatim}

\textbf{ઓહ્મના નિયમની મર્યાદાઓ}:

\begin{itemize}
\tightlist
\item
  બિન-રેખીય ઉપકરણો (અર્ધવાહકો, ગેસ ડિસ્ચાર્જ ટ્યુબ) માટે લાગુ પડતો નથી
\item
  ઉચ્ચ તાપમાને લાગુ પડતો નથી
\item
  એકતરફી તત્વો (ડાયોડ) માટે માન્ય નથી
\item
  સમય-પરિવર્તિત પ્રવાહો માટે નિષ્ફળ જાય છે
\end{itemize}

\end{solutionbox}
\begin{mnemonicbox}
``VIRO'' - Voltage Is Resistance times Output
current

\end{mnemonicbox}
\subsection*{પ્રશ્ન 1(ક) [7
માર્ક્સ]}\label{uxaaauxab0uxab6uxaa8-1uxa95-7-uxaaeuxab0uxa95uxab8}

\textbf{બેટ્રીની શ્રેણી અને સમાંતર જોડાણ સમજાવો.}

\begin{solutionbox}

\textbf{બેટ્રીનું શ્રેણી જોડાણ:}

\begin{verbatim}
flowchart LR
    B1[Battery 1] {-{-} B2[Battery 2] {-}{-} B3[Battery 3] {-}{-} L[Load]}
    L {-{-} B1}
\end{verbatim}

\textbf{શ્રેણી જોડાણની લાક્ષણિકતાઓ:}

\begin{itemize}
\tightlist
\item
  \textbf{કુલ વોલ્ટેજ} = વ્યક્તિગત વોલ્ટેજનો સરવાળો (V = V_{1} + V_{2} + \ldots{} +
  V_{n})
\item
  \textbf{પ્રવાહ} = બધી બેટરીઓમાં સમાન
\item
  \textbf{ઉપયોગો}: ઉચ્ચ વોલ્ટેજની જરૂરિયાતો
\item
  \textbf{આંતરિક અવરોધ}: વધે છે (R_{s} = r_{1} + r_{2} + \ldots{} + r_{n})
\end{itemize}

\textbf{બેટ્રીનું સમાંતર જોડાણ:}

\begin{verbatim}
flowchart LR
    B1[Battery 1] {-{-} L[Load]}
    B2[Battery 2] {-{-} L}
    B3[Battery 3] {-{-} L}
    L {-{-} B1}
    L {-{-} B2}
    L {-{-} B3}
\end{verbatim}

\textbf{સમાંતર જોડાણની લાક્ષણિકતાઓ:}

\begin{itemize}
\tightlist
\item
  \textbf{વોલ્ટેજ} = વ્યક્તિગત બેટરી જેટલું જ (જો સમાન હોય તો)
\item
  \textbf{કુલ પ્રવાહ} = વ્યક્તિગત પ્રવાહોનો સરવાળો (I = I_{1} + I_{2} + \ldots{} +
  I_{n})
\item
  \textbf{ઉપયોગો}: વધુ પ્રવાહ ક્ષમતાની જરૂર છે
\item
  \textbf{આંતરિક અવરોધ}: ઘટે છે (1/R_{p} = 1/r_{1} + 1/r_{2} + \ldots{} + 1/r_{n})
\end{itemize}

\end{solutionbox}
\begin{mnemonicbox}
``VSCP'' - Voltage Sums in Series, Current Parallels

\end{mnemonicbox}
\subsection*{પ્રશ્ન 1(ક) OR [7
માર્ક્સ]}\label{uxaaauxab0uxab6uxaa8-1uxa95-or-7-uxaaeuxab0uxa95uxab8}

\textbf{રેસિસ્ટરની શ્રેણી અને સમાંતર જોડાણ સમજાવો.}

\begin{solutionbox}

\textbf{રેસિસ્ટરનું શ્રેણી જોડાણ:}

\begin{verbatim}
flowchart LR
    S[Source] {-{-} R1[R1] {-}{-} R2[R2] {-}{-} R3[R3] {-}{-} S}
\end{verbatim}

\textbf{શ્રેણી જોડાણની લાક્ષણિકતાઓ:}

\begin{itemize}
\tightlist
\item
  \textbf{સમતુલ્ય અવરોધ} = વ્યક્તિગત અવરોધોનો સરવાળો (R_{s} = R_{1} + R_{2} +
  \ldots{} + R_{n})
\item
  \textbf{પ્રવાહ} = બધા રેસિસ્ટરોમાં સમાન
\item
  \textbf{વોલ્ટેજ} = અવરોધના મૂલ્યોના પ્રમાણમાં રેસિસ્ટરો પર વિભાજિત
\item
  \textbf{પાવર} વોલ્ટેજ વિતરણ અનુસાર વહેંચાયેલો
\end{itemize}

\textbf{રેસિસ્ટરનું સમાંતર જોડાણ:}

\begin{verbatim}
flowchart LR
    S[Source] {-{-} R1[R1]}
    S {-{-} R2[R2]}
    S {-{-} R3[R3]}
    R1 {-{-} S}
    R2 {-{-} S}
    R3 {-{-} S}
\end{verbatim}

\textbf{સમાંતર જોડાણની લાક્ષણિકતાઓ:}

\begin{itemize}
\tightlist
\item
  \textbf{સમતુલ્ય અવરોધ}: 1/R_{p} = 1/R_{1} + 1/R_{2} + \ldots{} + 1/R_{n}
\item
  \textbf{વોલ્ટેજ} = બધા રેસિસ્ટરોમાં સમાન
\item
  \textbf{પ્રવાહ} = અવરોધના મૂલ્યોના વ્યસ્ત પ્રમાણમાં વિભાજિત
\item
  \textbf{કુલ પ્રવાહ} = વ્યક્તિગત પ્રવાહોનો સરવાળો
\end{itemize}

\end{solutionbox}
\begin{mnemonicbox}
``RISE-VICE'' - Resistance Increases in Series,
Voltage Is Constant in Every parallel

\end{mnemonicbox}
\subsection*{પ્રશ્ન 2(અ) [3
માર્ક્સ]}\label{uxaaauxab0uxab6uxaa8-2uxa85-3-uxaaeuxab0uxa95uxab8}

\textbf{વ્યાખ્યા આપો (૧) એમ્પલીટ્યુડ (૨) આવૃત્તિ (૩) ટાઈમ પિરીયડ}

\begin{solutionbox}

{\def\LTcaptype{none} % do not increment counter
\begin{longtable}[]{@{}
  >{\raggedright\arraybackslash}p{(\linewidth - 2\tabcolsep) * \real{0.3333}}
  >{\raggedright\arraybackslash}p{(\linewidth - 2\tabcolsep) * \real{0.6667}}@{}}
\toprule\noalign{}
\begin{minipage}[b]{\linewidth}\raggedright
શબ્દ
\end{minipage} & \begin{minipage}[b]{\linewidth}\raggedright
વ્યાખ્યા
\end{minipage} \\
\midrule\noalign{}
\endhead
\bottomrule\noalign{}
\endlastfoot
\textbf{એમ્પલીટ્યુડ} & વેવફોર્મનું તેના મધ્ય સ્થાનથી મહત્તમ વિચલન, વોલ્ટ અથવા
એમ્પિયરમાં માપવામાં આવે છે \\
\textbf{આવૃત્તિ} & એક સેકન્ડમાં થતા પૂર્ણ ચક્રોની સંખ્યા, હર્ટઝ (Hz)માં માપવામાં આવે
છે \\
\textbf{ટાઈમ પિરીયડ} & વેવફોર્મના એક ચક્રને પૂર્ણ કરવા માટે લાગતો સમય, સેકન્ડ
(s)માં માપવામાં આવે છે \\
\end{longtable}
}

\end{solutionbox}
\begin{mnemonicbox}
``AFT'' - Amplitude is the Full height, Time period
is the Total cycle

\end{mnemonicbox}
\subsection*{પ્રશ્ન 2(બ) [4
માર્ક્સ]}\label{uxaaauxab0uxab6uxaa8-2uxaac-4-uxaaeuxab0uxa95uxab8}

\textbf{10Ω, 20Ω અને 30Ω રેસિસ્ટર શ્રેણીમાં જોડાયેલા છે અને તેમને 100V સપ્લાય આપવામાં
આવે છે. શોધો (1) સમતુલ્ય પ્રતિરોધ (2) સર્કિટ કરંટ (3) દરેક રેસિસ્ટરમાં વોલ્ટેજ ડ્રોપ.
(4) દરેક રેસિસ્ટરમાં પાવર લોસ.}

\begin{solutionbox}

\textbf{આકૃતિ:}

\begin{verbatim}
     +{-{-}[10Ω]{-}{-}[20Ω]{-}{-}[30Ω]{-}{-}+}
     |                        |
   (100V)                     |
     |                        |
     +{-{-}{-}{-}{-}{-}{-}{-}{-}{-}{-}{-}{-}{-}{-}{-}{-}{-}{-}{-}{-}{-}{-}{-}+}
\end{verbatim}

\textbf{ઉકેલ:}

{\def\LTcaptype{none} % do not increment counter
\begin{longtable}[]{@{}lll@{}}
\toprule\noalign{}
પરિમાણ & ગણતરી & પરિણામ \\
\midrule\noalign{}
\endhead
\bottomrule\noalign{}
\endlastfoot
સમતુલ્ય અવરોધ & R = 10Ω + 20Ω + 30Ω & 60Ω \\
સર્કિટ કરંટ & I = 100V/60Ω & 1.67A \\
10Ω માં વોલ્ટેજ & V_{1} = 1.67A \times 10Ω & 16.7V \\
20Ω માં વોલ્ટેજ & V_{2} = 1.67A \times 20Ω & 33.3V \\
30Ω માં વોલ્ટેજ & V_{3} = 1.67A \times 30Ω & 50.0V \\
10Ω માં પાવર & P_{1} = 1.67^{2} \times 10 & 27.8W \\
20Ω માં પાવર & P_{2} = 1.67^{2} \times 20 & 55.6W \\
30Ω માં પાવર & P_{3} = 1.67^{2} \times 30 & 83.4W \\
\end{longtable}
}

\end{solutionbox}
\begin{mnemonicbox}
``REÇVP'' - Resistances Equivalent Causes Voltage
and Power division

\end{mnemonicbox}
\subsection*{પ્રશ્ન 2(ક) [7
માર્ક્સ]}\label{uxaaauxab0uxab6uxaa8-2uxa95-7-uxaaeuxab0uxa95uxab8}

\textbf{વેવ ફોર્મ અને વેક્ટર ડાયાગ્રામ સાથે શુદ્ધ રેસિસ્ટર માં A.C સમજાવો.}

\begin{solutionbox}

શુદ્ધ અવરોધી સર્કિટમાં AC સપ્લાય સાથે:

\textbf{મુખ્ય લાક્ષણિકતાઓ:}

\begin{itemize}
\tightlist
\item
  કરંટ અને વોલ્ટેજ એકબીજા સાથે \textbf{ઇન-ફેઝ} (એક-તબક્કામાં) હોય છે
\item
  સર્કિટ ઓહ્મના નિયમનું પાલન કરે છે: V = IR
\item
  પાવર હંમેશા હકારાત્મક હોય છે (P = VI)
\item
  કોઈ રિએક્ટિવ પાવરનો વપરાશ નથી
\item
પાવર ફેક્ટર = 1 (cos

φ = 1)

\end{itemize}

\textbf{વેવફોર્મ:}

\begin{verbatim}
    │    ╭─╮   ╭─╮   ╭─╮   ╭─╮
    │   ╱   ╲ ╱   ╲ ╱   ╲ ╱   ╲
    │  ╱     V     V     V     ╲
────┼─╱───────────────────────╲─────
    │╱       ╱╲       ╱╲       ╲
    V       V  ╲     V  ╲       V
    │      ╱    ╲   ╱    ╲
    │     ╱      ╲ ╱      ╲
    │    ╰─╯     ╰─╯      ╰─╯

    {-{-}{-} Voltage waveform}
    {-{-}{-} Current waveform (identical phase)}
\end{verbatim}

\textbf{વેક્ટર ડાયાગ્રામ:}

\begin{verbatim}
         │
         │
         V (voltage)
         │
         │
─────────┼────────
         │        I (current)
         │
         │
\end{verbatim}

\end{solutionbox}
\begin{mnemonicbox}
``PARVIP'' - Pure AC Resistor has Voltage In Phase
with current

\end{mnemonicbox}
\subsection*{પ્રશ્ન 2(અ) OR [3
માર્ક્સ]}\label{uxaaauxab0uxab6uxaa8-2uxa85-or-3-uxaaeuxab0uxa95uxab8}

\textbf{વ્યાખ્યાયિત કરો: (1) સાઈકલ (2) ફોર્મ ફેક્ટર (3) પીક ફેક્ટર}

\begin{solutionbox}

{\def\LTcaptype{none} % do not increment counter
\begin{longtable}[]{@{}
  >{\raggedright\arraybackslash}p{(\linewidth - 2\tabcolsep) * \real{0.3333}}
  >{\raggedright\arraybackslash}p{(\linewidth - 2\tabcolsep) * \real{0.6667}}@{}}
\toprule\noalign{}
\begin{minipage}[b]{\linewidth}\raggedright
શબ્દ
\end{minipage} & \begin{minipage}[b]{\linewidth}\raggedright
વ્યાખ્યા
\end{minipage} \\
\midrule\noalign{}
\endhead
\bottomrule\noalign{}
\endlastfoot
\textbf{સાઈકલ} & આવર્તી વેવફોર્મનું એક પૂર્ણ પુનરાવર્તન શરૂઆતના બિંદુથી તે જ બિંદુ
સુધી \\
\textbf{ફોર્મ ફેક્ટર} & AC વેવફોર્મના RMS મૂલ્યનો સરેરાશ મૂલ્ય સાથેનો ગુણોત્તર (સાઇન
વેવ માટે = 1.11) \\
\textbf{પીક ફેક્ટર} & AC વેવફોર્મના મહત્તમ મૂલ્યનો RMS મૂલ્ય સાથેનો ગુણોત્તર (સાઇન
વેવ માટે = 1.414) \\
\end{longtable}
}

\end{solutionbox}
\begin{mnemonicbox}
``CFP'' - Cycle Finishes a Pattern, Form Factor =
Vrms/Vavg, Peak Factor = Vmax/Vrms

\end{mnemonicbox}
\subsection*{પ્રશ્ન 2(બ) OR [4
માર્ક્સ]}\label{uxaaauxab0uxab6uxaa8-2uxaac-or-4-uxaaeuxab0uxa95uxab8}

\textbf{20Ω, 30Ω અને 50Ω રેસિસ્ટર સમાંતર રીતે જોડાયેલા છે અને તેમને 60V સપ્લાય
આપવામાં આવે છે. તો (1) દરેક રેસિસ્ટરમાંથી પસાર થતો પ્રવાહ (2) કુલ કરંટ (3) સમતુલ્ય
પ્રતિરોધ (4) દરેક રેસિસ્ટરમાં પાવર લોસ. શોધો.}

\begin{solutionbox}

\textbf{આકૃતિ:}

\begin{verbatim}
         ┌─[20Ω]─┐
         │       │
     +───┼───────┼───+
     │   │       │   │
    (60V) ├─[30Ω]─┤  │
     │   │       │   │
     │   └─[50Ω]─┘   │
     │               │
     +───────────────+
\end{verbatim}

\textbf{ઉકેલ:}

{\def\LTcaptype{none} % do not increment counter
\begin{longtable}[]{@{}lll@{}}
\toprule\noalign{}
પરિમાણ & ગણતરી & પરિણામ \\
\midrule\noalign{}
\endhead
\bottomrule\noalign{}
\endlastfoot
20Ω માં કરંટ & I_{1} = 60V/20Ω & 3A \\
30Ω માં કરંટ & I_{2} = 60V/30Ω & 2A \\
50Ω માં કરંટ & I_{3} = 60V/50Ω & 1.2A \\
કુલ કરંટ & I = 3A + 2A + 1.2A & 6.2A \\
સમતુલ્ય અવરોધ & 1/Req = 1/20 + 1/30 + 1/50 & 9.68Ω \\
20Ω માં પાવર & P_{1} = 60V \times 3A & 180W \\
30Ω માં પાવર & P_{2} = 60V \times 2A & 120W \\
50Ω માં પાવર & P_{3} = 60V \times 1.2A & 72W \\
\end{longtable}
}

\end{solutionbox}
\begin{mnemonicbox}
``VICTIM'' - Voltage Is Constant, Total current Is
the Measure (in parallel)

\end{mnemonicbox}
\subsection*{પ્રશ્ન 2(ક) OR [7
માર્ક્સ]}\label{uxaaauxab0uxab6uxaa8-2uxa95-or-7-uxaaeuxab0uxa95uxab8}

\textbf{વેવફોર્મ અને વેક્ટર ડાયાગ્રામ સાથે શુદ્ધ કેપેસિટરમાં A.C સમજાવો.}

\begin{solutionbox}

શુદ્ધ કેપેસિટીવ સર્કિટમાં AC સપ્લાય સાથે:

\textbf{મુખ્ય લાક્ષણિકતાઓ:}

\begin{itemize}
\tightlist
\item
  કરંટ વોલ્ટેજથી 90^\circ \textbf{આગળ} હોય છે
\item
  કેપેસિટીવ રિએક્ટન્સ Xc = 1/(2πfC)
\item
  માત્ર રિએક્ટિવ પાવર (એક્ટિવ પાવર નહીં)
\item
  પાવર ફેક્ટર = 0 (લેગિંગ)
\item
  સંપૂર્ણ ચક્ર દરમિયાન સરેરાશ પાવર = 0
\end{itemize}

\textbf{વેવફોર્મ:}

\begin{verbatim}
           Current
    │      ╭─╮     ╭─╮     ╭─╮     ╭─╮
    │     ╱   ╲   ╱   ╲   ╱   ╲   ╱   ╲
    │    ╱     ╲ ╱     ╲ ╱     ╲ ╱     ╲
────┼───╱───────V───────V───────V───────╲─
    │  ╱         ╲       ╲       ╲       ╲
    │ ╱           ╲       ╲       ╲       ╲
    │╱             ╲       ╲       ╲       ╲
    V               ╰─╮     ╰─╮     ╰─╮     ╰
    │                 │       │       │
    │                 │       │       │
    │                 │       │       │
    │                 V       V       V     
    │                ╱ ╲     ╱ ╲     ╱ ╲   Voltage
    │               ╱   ╲   ╱   ╲   ╱   ╲ 
    │              ╱     ╲ ╱     ╲ ╱     ╲
\end{verbatim}

\textbf{વેક્ટર ડાયાગ્રામ:}

\begin{verbatim}
         │ I (current)
         │
         │
         │
─────────┼────────
         │
         │
         │
         V V (voltage)
\end{verbatim}

\end{solutionbox}
\begin{mnemonicbox}
``CLEAR-90'' - Capacitive Load has Electrical Angle
Reaching 90^\circ (current leads voltage)

\end{mnemonicbox}
\subsection*{પ્રશ્ન 3(અ) [3
માર્ક્સ]}\label{uxaaauxab0uxab6uxaa8-3uxa85-3-uxaaeuxab0uxa95uxab8}

\textbf{અલ્તેનિતંગ વેવફોર્મ માટે આરએમએસ વેલ્યુ અને એવરેજ વેલ્યુની વ્યાખ્યા આપો તથા તેમની
ફોર્મ્યુલા લખો.}

\begin{solutionbox}

{\def\LTcaptype{none} % do not increment counter
\begin{longtable}[]{@{}
  >{\raggedright\arraybackslash}p{(\linewidth - 4\tabcolsep) * \real{0.2222}}
  >{\raggedright\arraybackslash}p{(\linewidth - 4\tabcolsep) * \real{0.4444}}
  >{\raggedright\arraybackslash}p{(\linewidth - 4\tabcolsep) * \real{0.3333}}@{}}
\toprule\noalign{}
\begin{minipage}[b]{\linewidth}\raggedright
શબ્દ
\end{minipage} & \begin{minipage}[b]{\linewidth}\raggedright
વ્યાખ્યા
\end{minipage} & \begin{minipage}[b]{\linewidth}\raggedright
ફોર્મ્યુલા
\end{minipage} \\
\midrule\noalign{}
\endhead
\bottomrule\noalign{}
\endlastfoot
\textbf{RMS વેલ્યુ} & રૂટ મીન સ્ક્વેર વેલ્યુ - સમાન હીટિંગ ઈફેક્ટ આપતું DC મૂલ્ય & Vrms =
0.707 \times Vmax (સાઇન વેવ માટે) \\
\textbf{એવરેજ વેલ્યુ} & અર્ધા ચક્ર દરમિયાન તમામ ઇન્સ્ટન્ટેનિયસ મૂલ્યોનું સરેરાશ મૂલ્ય &
Vavg = 0.637 \times Vmax (સાઇન વેવ માટે) \\
\end{longtable}
}

\end{solutionbox}
\begin{mnemonicbox}
``RAM'' - RMS Averages the Mean square (RMS =
0.707\timesVmax, AVG = 0.637\timesVmax)

\end{mnemonicbox}
\subsection*{પ્રશ્ન 3(બ) [4
માર્ક્સ]}\label{uxaaauxab0uxab6uxaa8-3uxaac-4-uxaaeuxab0uxa95uxab8}

\textbf{એ.સી.કરંટ i=25 sin(314t). તો (૧) આર.એમ.એસ કીમત (૨) એવરેજ વેલ્યુ (૩)
આવૃત્તિ (૪) ટાઈમ પીરીયડ}

\begin{solutionbox}

\textbf{આપેલ સમીકરણ:} i = 25 sin(314t)

{\def\LTcaptype{none} % do not increment counter
\begin{longtable}[]{@{}lll@{}}
\toprule\noalign{}
પરિમાણ & ગણતરી & પરિણામ \\
\midrule\noalign{}
\endhead
\bottomrule\noalign{}
\endlastfoot
મહત્તમ મૂલ્ય & Imax = 25 A & 25 A \\
RMS મૂલ્ય & Irms = Imax/\sqrt2 = 25/1.414 & 17.68 A \\
સરેરાશ મૂલ્ય & Iavg = 2Imax/π = 2\times25/3.14 & 15.92 A \\
કોણીય આવૃત્તિ & ω = 314 rad/s & 314 rad/s \\
આવૃત્તિ &

f = ω/2π = 314/6.28 & 50 Hz \\

સમય અવધિ &

T = 1/f = 1/50 & 0.02 s \\

\end{longtable}
}

\end{solutionbox}
\begin{mnemonicbox}
``SMART'' - Sine's Maximum divided by root 2 equals
RMS Then 2/π for Average

\end{mnemonicbox}
\subsection*{પ્રશ્ન 3(ક) [7
માર્ક્સ]}\label{uxaaauxab0uxab6uxaa8-3uxa95-7-uxaaeuxab0uxa95uxab8}

\textbf{અવરોધોનું સ્ટાર જોડાણ સમજાઓ અને સ્ટાર જોડાણમાં વોલ્ટેજ અને કરંત વચ્ચેના સંબંધ
નું સમીકરણ તારવો.}

\begin{solutionbox}

\textbf{સ્ટાર (Y) જોડાણ:}

\begin{center}
\textbf{Mermaid Diagram (Code)}
\begin{verbatim}
{Shaded}
{Highlighting}[]
graph TD
    N((N)) {-{-}{-} R1[R1] {-}{-}{-} L1((L1))}
    N {-{-}{-} R2[R2] {-}{-}{-} L2((L2))}
    N {-{-}{-} R3[R3] {-}{-}{-} L3((L3))}
    N((Neutral))
{Highlighting}
{Shaded}
\end{verbatim}
\end{center}

\textbf{સ્ટાર જોડાણની લાક્ષણિકતાઓ:}

\begin{itemize}
\tightlist
\item
  ત્રણ અવરોધો સામાન્ય બિંદુ (ન્યૂટ્રલ) પર જોડાયેલા છે
\item
  લાઈન વોલ્ટેજ (VL) = \sqrt3 \times ફેઝ વોલ્ટેજ (Vph)
\item
  લાઈન કરંટ (IL) = ફેઝ કરંટ (Iph)
\item
  સંતુલિત લોડ માટે: IL = Iph
\item
  કુલ પાવર = 3 \times ફેઝ પાવર
\end{itemize}

\textbf{ગાણિતિક સંબંધ:}

\begin{itemize}
\tightlist
\item
  ફેઝ વોલ્ટેજ: Vph = VL/\sqrt3
\item
  ફેઝ કરંટ: Iph = IL
\item
  સંતુલિત અવરોધી લોડ માટે: Iph = Vph/R
\item
  તેથી: IL = VL/(\sqrt3\timesR)
\end{itemize}

\end{solutionbox}
\begin{mnemonicbox}
``SLIP-3'' - Star Line current Is Phase current,
Line voltage is Phase voltage times root-3

\end{mnemonicbox}
\subsection*{પ્રશ્ન 3(અ) OR [3
માર્ક્સ]}\label{uxaaauxab0uxab6uxaa8-3uxa85-or-3-uxaaeuxab0uxa95uxab8}

\textbf{અલ્તેનિતંગ E.M.F. કેવી રીતે ઉત્પન્ન થાય છે તે સમજાઓ.}

\begin{solutionbox}

\textbf{અલ્ટરનેટિંગ EMF ઉત્પાદન:}

\begin{center}
\textbf{Mermaid Diagram (Code)}
\begin{verbatim}
{Shaded}
{Highlighting}[]
graph LR
    subgraph "Rotating Coil in Magnetic Field"
    N[N] {-{-}{-} M((Magnet)) {-}{-}{-} S[S]}
    end
    M {-{-}{-} R[Rotating Coil]}
    R {-{-}{-} EMF[EMF Output]}
{Highlighting}
{Shaded}
\end{verbatim}
\end{center}

\textbf{પ્રક્રિયા:}

\begin{itemize}
\tightlist
\item
  કોઇલ એકસમાન ચુંબકીય ક્ષેત્રમાં ફરે છે
\item
  ફેરફારના ખૂણા સાથે ફ્લક્સ લિંકેજ બદલાય છે
\item
  ફ્લક્સના પરિવર્તનનો દર EMF પ્રેરિત કરે છે
\item
  EMF સાઇન પેટર્ન અનુસરે છે: e = Emax sin(ωt)
\item
  આવૃત્તિ રોટેશન સ્પીડ પર આધારિત છે
\end{itemize}

\end{solutionbox}
\begin{mnemonicbox}
``FRAME'' - Flux Rotation Alternates Magnetic EMF

\end{mnemonicbox}
\subsection*{પ્રશ્ન 3(બ) OR [4
માર્ક્સ]}\label{uxaaauxab0uxab6uxaa8-3uxaac-or-4-uxaaeuxab0uxa95uxab8}

\textbf{અલ્ટરનેતિંગ EMF= e=100 sin2π50t. તો (૧) EMF ની મેક્સિમમ વેલ્યુ (૨) આવૃત્તિ
(૩) ટાઈમ પીરીયડ (૪) એગ્યુંલર આવૃત્તિ શોધો.}

\begin{solutionbox}

\textbf{આપેલ સમીકરણ:} e = 100 sin2π50t

{\def\LTcaptype{none} % do not increment counter
\begin{longtable}[]{@{}lll@{}}
\toprule\noalign{}
પરિમાણ & ગણતરી & પરિણામ \\
\midrule\noalign{}
\endhead
\bottomrule\noalign{}
\endlastfoot
મહત્તમ EMF & Emax = 100 V & 100 V \\
કોણીય આવૃત્તિ &

ω = 2π50 = 314 rad/s & 314 rad/s \\

આવૃત્તિ & f = 50 Hz (સીધા સમીકરણમાંથી) & 50 Hz \\
સમય અવધિ &

T = 1/f = 1/50 & 0.02 s \\

\end{longtable}
}

\end{solutionbox}
\begin{mnemonicbox}
``FAST'' - Frequency And period are reciprocals,
Sin's Top value is maximum

\end{mnemonicbox}
\subsection*{પ્રશ્ન 3(ક) OR [7
માર્ક્સ]}\label{uxaaauxab0uxab6uxaa8-3uxa95-or-7-uxaaeuxab0uxa95uxab8}

\textbf{અવરોધોનું ડેલ્ટા જોડાણ સમજાઓ અને ડેલ્ટા જોડાણમાં વોલ્ટેજ અને કરંત વચ્ચેના સંબંધ
નું સમીકરણ તારવો.}

\begin{solutionbox}

\textbf{ડેલ્ટા (Δ) જોડાણ:}

\begin{center}
\textbf{Mermaid Diagram (Code)}
\begin{verbatim}
{Shaded}
{Highlighting}[]
graph LR
    L1((L1)) {-{-}{-} R1[R1] {-}{-}{-} L2((L2))}
    L2 {-{-}{-} R2[R2] {-}{-}{-} L3((L3))}
    L3 {-{-}{-} R3[R3] {-}{-}{-} L1}
{Highlighting}
{Shaded}
\end{verbatim}
\end{center}

\textbf{ડેલ્ટા જોડાણની લાક્ષણિકતાઓ:}

\begin{itemize}
\tightlist
\item
  ત્રણ અવરોધો બંધ લૂપમાં જોડાયેલા છે
\item
  લાઈન વોલ્ટેજ (VL) = ફેઝ વોલ્ટેજ (Vph)
\item
  લાઈન કરંટ (IL) = \sqrt3 \times ફેઝ કરંટ (Iph)
\item
  સંતુલિત લોડ માટે: Vph = VL
\item
  કુલ પાવર = 3 \times ફેઝ પાવર
\end{itemize}

\textbf{ગાણિતિક સંબંધ:}

\begin{itemize}
\tightlist
\item
  ફેઝ વોલ્ટેજ: Vph = VL
\item
  ફેઝ કરંટ: Iph = Vph/R
\item
  લાઈન કરંટ: IL = \sqrt3 \times Iph
\item
  તેથી: IL = \sqrt3 \times VL/R
\end{itemize}

\end{solutionbox}
\begin{mnemonicbox}
``DELVIr3'' - Delta Equal Line Voltage, Its line
current equals phase current times root-3

\end{mnemonicbox}
\subsection*{પ્રશ્ન 4(અ) [3
માર્ક્સ]}\label{uxaaauxab0uxab6uxaa8-4uxa85-3-uxaaeuxab0uxa95uxab8}

\textbf{વ્યાખ્યા આપો (૧) એમ.એમ.એફ (૨) રીલક્તંસ (૩) ફ્લક્સ}

\begin{solutionbox}

{\def\LTcaptype{none} % do not increment counter
\begin{longtable}[]{@{}
  >{\raggedright\arraybackslash}p{(\linewidth - 2\tabcolsep) * \real{0.3333}}
  >{\raggedright\arraybackslash}p{(\linewidth - 2\tabcolsep) * \real{0.6667}}@{}}
\toprule\noalign{}
\begin{minipage}[b]{\linewidth}\raggedright
શબ્દ
\end{minipage} & \begin{minipage}[b]{\linewidth}\raggedright
વ્યાખ્યા
\end{minipage} \\
\midrule\noalign{}
\endhead
\bottomrule\noalign{}
\endlastfoot
\textbf{એમ.એમ.એફ. (મેગ્નેટોમોટિવ ફોર્સ)} & ચુંબકીય સર્કિટમાં ચુંબકીય ફ્લક્સ ઉત્પન્ન
કરતું બળ, એમ્પિયર-ટર્ન્સ (AT)માં માપવામાં આવે છે \\
\textbf{રિલક્ટન્સ} & ચુંબકીય અવરોધનું સમકક્ષ, ચુંબકીય ફ્લક્સનો વિરોધ, AT/Wb માં
માપવામાં આવે છે \\
\textbf{ફ્લક્સ} & કોઈ સપાટીમાંથી પસાર થતું કુલ ચુંબકીય ક્ષેત્ર, વેબર (Wb)માં માપવામાં
આવે છે \\
\end{longtable}
}

\end{solutionbox}
\begin{mnemonicbox}
``MFR'' - MMF Flows against Reluctance like current
flows against resistance

\end{mnemonicbox}
\subsection*{પ્રશ્ન 4(બ) [4
માર્ક્સ]}\label{uxaaauxab0uxab6uxaa8-4uxaac-4-uxaaeuxab0uxa95uxab8}

\textbf{એ.સી. સર્કિટ માં એપેરંટ, એક્ટીવ તથા રીએક્ટીવ પાવર સમજાઓ}

\begin{solutionbox}

{\def\LTcaptype{none} % do not increment counter
\begin{longtable}[]{@{}lll@{}}
\toprule\noalign{}
પાવર પ્રકાર & પ્રતીક અને એકમ & વ્યાખ્યા \\
\midrule\noalign{}
\endhead
\bottomrule\noalign{}
\endlastfoot
\textbf{એપેરંટ પાવર} & S (VA) & એક્ટિવ અને રિએક્ટિવ પાવરનો વેક્ટર સરવાળો \\
\textbf{એક્ટિવ પાવર} & P (W) & લોડ દ્વારા વપરાયેલો વાસ્તવિક કાર્ય-ઉત્પાદક
પાવર \\
\textbf{રિએક્ટિવ પાવર} & Q (VAR) & સ્ત્રોત અને લોડ વચ્ચે આંદોલિત થતો પાવર \\
\end{longtable}
}

\textbf{પાવર ત્રિકોણ:}

\begin{verbatim}
          \^{ Q (Reactive Power)}
          │
          │
          │
          │           S (Apparent Power)
          │         /
          │        /
          │       /
          │      /
          │     /
          │    θ
          │   /
          └──/───────────{}
             P (Active Power)
\end{verbatim}

\textbf{સંબંધો:}

\begin{itemize}
\tightlist
\item
  S = \sqrt(P^{2} + Q^{2})
\item
  P = S \times cos θ
\item
  Q = S \times sin θ
\item
પાવર ફેક્ટર = cos

θ = P/S

\end{itemize}

\end{solutionbox}
\begin{mnemonicbox}
``SPARQ'' - S is Power Apparent, Real is P, Q is
reactive

\end{mnemonicbox}
\subsection*{પ્રશ્ન 4(ક) [7
માર્ક્સ]}\label{uxaaauxab0uxab6uxaa8-4uxa95-7-uxaaeuxab0uxa95uxab8}

\textbf{ઇલેક્ટ્રિક સર્કિટ તથા મેગનેટિક સર્કિટની સરખામણી કરો.}

\begin{solutionbox}

{\def\LTcaptype{none} % do not increment counter
\begin{longtable}[]{@{}lll@{}}
\toprule\noalign{}
પરિમાણ & ઇલેક્ટ્રિક સર્કિટ & મેગ્નેટિક સર્કિટ \\
\midrule\noalign{}
\endhead
\bottomrule\noalign{}
\endlastfoot
\textbf{બળ} & EMF (V) & MMF (AT) \\
\textbf{વિરોધ} & રેઝિસ્ટન્સ (Ω) & રિલક્ટન્સ (AT/Wb) \\
\textbf{પ્રવાહ} & કરંટ (A) & ફ્લક્સ (Wb) \\
\textbf{ઓહ્મનો નિયમ} & V = IR & MMF = Φ \times S \\
\textbf{માધ્યમ} & કન્ડક્ટર & ફેરોમેગ્નેટિક મટીરિયલ \\
\textbf{ઊર્જા} & ઇલેક્ટ્રિક ફીલ્ડમાં સંગ્રહિત & મેગ્નેટિક ફીલ્ડમાં સંગ્રહિત \\
\textbf{લીકેજ} & નગણ્ય & નોંધપાત્ર \\
\textbf{પાથ} & કન્ડક્ટર્સ & સામાન્ય રીતે બંધ લૂપ \\
\textbf{મટીરિયલ પ્રોપર્ટી} & કન્ડક્ટિવિટી & પર્મિએબિલિટી \\
\textbf{કરંટ ફ્લો} & ઇલેક્ટ્રોન ફ્લો & કોઈ પાર્ટિકલ ફ્લો નહીં \\
\end{longtable}
}

\end{solutionbox}
\begin{mnemonicbox}
``VIRO-MSΦS'' - Voltage Is to Resistance as MMF is
to Reluctance, Our φ flows Similar

\end{mnemonicbox}
\subsection*{પ્રશ્ન 4(અ) OR [3
માર્ક્સ]}\label{uxaaauxab0uxab6uxaa8-4uxa85-or-3-uxaaeuxab0uxa95uxab8}

\textbf{ફ્લેમિંગના ડાબા હાથના નિયમ નું વિધાન લખી સમજાઓ.}

\begin{solutionbox}

\textbf{ફ્લેમિંગનો ડાબા હાથનો નિયમ:} ચુંબકીય ક્ષેત્રમાં મૂકેલા કરંટ વહન કરતા વાહક
દ્વારા અનુભવાતા બળની દિશા શોધવા માટે વપરાય છે.

\begin{center}
\textbf{Mermaid Diagram (Code)}
\begin{verbatim}
{Shaded}
{Highlighting}[]
graph TD
    subgraph "Fleming{s Left Hand Rule"}
    T[Thumb: Force] {-{-}{-} F[Forefinger: Field] {-}{-}{-} M[Middle finger: Current]}
    end
{Highlighting}
{Shaded}
\end{verbatim}
\end{center}

\textbf{ઉપયોગ:}

\begin{itemize}
\tightlist
\item
  અંગૂઠો \rightarrow બળની દિશા (F)
\item
  તર્જની \rightarrow ચુંબકીય ક્ષેત્રની દિશા (B)
\item
  મધ્યમા \rightarrow કરંટની દિશા (I)
\item
  આંગળીઓ એકબીજાથી લંબ હોય ત્યારે જ કામ કરે છે
\end{itemize}

\end{solutionbox}
\begin{mnemonicbox}
``FBI-Left'' - Force, B-field, and I-current
directions are shown by the Left hand

\end{mnemonicbox}
\subsection*{પ્રશ્ન 4(બ) OR [4
માર્ક્સ]}\label{uxaaauxab0uxab6uxaa8-4uxaac-or-4-uxaaeuxab0uxa95uxab8}

\textbf{પાવર ત્રિકોણ દોરો અને તેના દરેક ભાગ સમજાઓ.}

\begin{solutionbox}

\textbf{પાવર ત્રિકોણ:}

\begin{center}
\textbf{Mermaid Diagram (Code)}
\begin{verbatim}
{Shaded}
{Highlighting}[]
graph LR
    O {-{-}{-} P[Active Power P]}
    O {-{-}{-} S[Hypotenuse: Apparent Power S]}
    P {-{-}{-} Q[Reactive Power Q]}
    P {-.{-} A[Power Factor Angle φ]}
{Highlighting}
{Shaded}
\end{verbatim}
\end{center}

\textbf{ઘટકો:}

{\def\LTcaptype{none} % do not increment counter
\begin{longtable}[]{@{}llll@{}}
\toprule\noalign{}
ઘટક & પ્રતીક & એકમ & અર્થ \\
\midrule\noalign{}
\endhead
\bottomrule\noalign{}
\endlastfoot
\textbf{એક્ટિવ પાવર} & P & વોટ (W) & ઉપયોગી કાર્ય કરતો વાસ્તવિક પાવર \\
\textbf{રિએક્ટિવ પાવર} & Q & VAR & સ્ત્રોત અને લોડ વચ્ચે આંદોલિત પાવર \\
\textbf{એપેરંટ પાવર} & S & VA & P અને Q નો વેક્ટર સરવાળો \\
\textbf{પાવર ફેક્ટર} & cos φ & - & એક્ટિવથી એપેરંટ પાવરનો ગુણોત્તર (P/S) \\
\end{longtable}
}

\textbf{સંબંધો:}

\begin{itemize}
\tightlist
\item
  S^{2} = P^{2} + Q^{2}
\item
  P = S \times cos φ
\item
  Q = S \times sin φ
\end{itemize}

\end{solutionbox}
\begin{mnemonicbox}
``SPQR'' - S is Pythagoras of P and Q, Ratio of P/S
is power factor

\end{mnemonicbox}
\subsection*{પ્રશ્ન 4(ક) OR [7
માર્ક્સ]}\label{uxaaauxab0uxab6uxaa8-4uxa95-or-7-uxaaeuxab0uxa95uxab8}

\textbf{સ્ટેટિકલી અને ડાઈનેમીકલી ઉત્પન્ન થતા ઈ.એમ.એફ.ની સરખામણી કરો.}

\begin{solutionbox}

{\def\LTcaptype{none} % do not increment counter
\begin{longtable}[]{@{}
  >{\raggedright\arraybackslash}p{(\linewidth - 4\tabcolsep) * \real{0.1803}}
  >{\raggedright\arraybackslash}p{(\linewidth - 4\tabcolsep) * \real{0.3934}}
  >{\raggedright\arraybackslash}p{(\linewidth - 4\tabcolsep) * \real{0.4262}}@{}}
\toprule\noalign{}
\begin{minipage}[b]{\linewidth}\raggedright
પરિમાણ
\end{minipage} & \begin{minipage}[b]{\linewidth}\raggedright
સ્ટેટિકલી ઇન્ડ્યુસ્ડ EMF
\end{minipage} & \begin{minipage}[b]{\linewidth}\raggedright
ડાયનેમિકલી ઇન્ડ્યુસ્ડ EMF
\end{minipage} \\
\midrule\noalign{}
\endhead
\bottomrule\noalign{}
\endlastfoot
\textbf{વ્યાખ્યા} & પ્રાથમિક કોઇલમાં કરંટના પરિવર્તનને કારણે પ્રેરિત EMF & વાહક અને
ચુંબકીય ક્ષેત્ર વચ્ચે સાપેક્ષ ગતિને કારણે પ્રેરિત EMF \\
\textbf{મેકેનિઝમ} & લિંકેજ ફ્લક્સમાં પરિવર્તન & ચુંબકીય ફ્લક્સનું કટિંગ \\
\textbf{મૂવમેન્ટ} & ભૌતિક હલનચલનની જરૂર નથી & સાપેક્ષ ગતિની જરૂર છે \\
\textbf{ઉદાહરણો} & ટ્રાન્સફોર્મર, ઇન્ડક્ટર & જનરેટર, મોટર \\
\textbf{ફેરાડેનો નિયમ} & e = -N(dΦ/dt) & e = Blv \\
\textbf{એપ્લિકેશન} & ગતિ વિના પાવર ટ્રાન્સફર & ગતિ દ્વારા પાવર જનરેશન \\
\textbf{એનર્જી કન્વર્ઝન} & ઇલેક્ટ્રિકલથી મેગ્નેટિક અને પાછું & મિકેનિકલથી ઇલેક્ટ્રિકલ
અથવા ઉલટું \\
\end{longtable}
}

\end{solutionbox}
\begin{mnemonicbox}
``STIM-DMOV'' - STatically Induced needs Magnetic
flux change, Dynamically needs MOVement

\end{mnemonicbox}
\subsection*{પ્રશ્ન 5(અ) [3
માર્ક્સ]}\label{uxaaauxab0uxab6uxaa8-5uxa85-3-uxaaeuxab0uxa95uxab8}

\textbf{વ્યાખ્યા આપો.(૧) સોલાર સેલ (૨) સોલર પેનલ (૩) સોલાર એરે}

\begin{solutionbox}

{\def\LTcaptype{none} % do not increment counter
\begin{longtable}[]{@{}
  >{\raggedright\arraybackslash}p{(\linewidth - 2\tabcolsep) * \real{0.3333}}
  >{\raggedright\arraybackslash}p{(\linewidth - 2\tabcolsep) * \real{0.6667}}@{}}
\toprule\noalign{}
\begin{minipage}[b]{\linewidth}\raggedright
શબ્દ
\end{minipage} & \begin{minipage}[b]{\linewidth}\raggedright
વ્યાખ્યા
\end{minipage} \\
\midrule\noalign{}
\endhead
\bottomrule\noalign{}
\endlastfoot
\textbf{સોલાર સેલ} & મૂળભૂત ફોટોવોલ્ટાઇક એકમ જે સૂર્યપ્રકાશને સીધો જ વીજળીમાં
રૂપાંતરિત કરે છે \\
\textbf{સોલર પેનલ} & સોલાર સેલનો સમૂહ જે એક ફ્રેમમાં શ્રેણી/સમાંતર જોડાયેલા હોય
છે \\
\textbf{સોલાર એરે} & એકસાથે જોડાયેલા અનેક સોલર પેનલો જે મોટી વીજળી-ઉત્પાદક એકમ
બનાવે છે \\
\end{longtable}
}

\end{solutionbox}
\begin{mnemonicbox}
``CPA'' - Cell Produces electricity, Panel Arrays
cells, Array is collection of panels

\end{mnemonicbox}
\subsection*{પ્રશ્ન 5(બ) [4
માર્ક્સ]}\label{uxaaauxab0uxab6uxaa8-5uxaac-4-uxaaeuxab0uxa95uxab8}

\textbf{HAWT અને VAWT વચ્ચે નો તફાવત લખો.}

\begin{solutionbox}

{\def\LTcaptype{none} % do not increment counter
\begin{longtable}[]{@{}
  >{\raggedright\arraybackslash}p{(\linewidth - 4\tabcolsep) * \real{0.1310}}
  >{\raggedright\arraybackslash}p{(\linewidth - 4\tabcolsep) * \real{0.4405}}
  >{\raggedright\arraybackslash}p{(\linewidth - 4\tabcolsep) * \real{0.4286}}@{}}
\toprule\noalign{}
\begin{minipage}[b]{\linewidth}\raggedright
પરિમાણ
\end{minipage} & \begin{minipage}[b]{\linewidth}\raggedright
હોરિઝોન્ટલ એક્સિસ વિન્ડ ટર્બાઇન (HAWT)
\end{minipage} & \begin{minipage}[b]{\linewidth}\raggedright
વર્ટિકલ એક્સિસ વિન્ડ ટર્બાઇન (VAWT)
\end{minipage} \\
\midrule\noalign{}
\endhead
\bottomrule\noalign{}
\endlastfoot
\textbf{અક્ષનું ઓરિએન્ટેશન} & જમીનની સમાંતર & જમીનને લંબ \\
\textbf{કાર્યક્ષમતા} & ઉચ્ચ (35-45\%) & નીચી (15-30\%) \\
\textbf{પવનની દિશા} & પવનની સામે ફેસ કરવાની જરૂર & કોઈપણ દિશાના પવન સાથે કામ
કરે છે \\
\textbf{જનરેટર સ્થાન} & ટાવરના ટોચ પર & જમીનના સ્તર પર મૂકી શકાય છે \\
\textbf{જગ્યાની જરૂરિયાત} & વધારે & ઓછી \\
\textbf{અવાજ} & વધારે & ઓછો \\
\textbf{ઉદાહરણો} & પ્રોપેલર-પ્રકાર, વ્યાપારિક ધોરણે વ્યાપકપણે વપરાય છે & ડેરિઅસ,
સેવોનિયસ ડિઝાઇન \\
\end{longtable}
}

\end{solutionbox}
\begin{mnemonicbox}
``HAVE'' - Horizontal Aligns with wind, Vertical
Enjoys omnidirectional wind

\end{mnemonicbox}
\subsection*{પ્રશ્ન 5(ક) [7
માર્ક્સ]}\label{uxaaauxab0uxab6uxaa8-5uxa95-7-uxaaeuxab0uxa95uxab8}

\textbf{સોલાર પાવર પ્લાન્ટ નો બ્લોક ડાયાગ્રામ દોરી સમજાઓ.}

\begin{solutionbox}

\textbf{સોલાર પાવર સિસ્ટમ બ્લોક ડાયાગ્રામ:}

\begin{verbatim}
flowchart LR
    S[Solar Panel] {-{-} C[Charge Controller]}
    C {-{-} B[Battery Bank]}
    B {-{-} I[Inverter]}
    I {-{-} L[AC Load]}
    B {-{-} D[DC Load]}
\end{verbatim}

\textbf{ઘટકો:}

\begin{enumerate}
\tightlist
\item
  \textbf{સોલાર પેનલ}: સૂર્યપ્રકાશનું DC વીજળીમાં રૂપાંતરણ
\item
  \textbf{ચાર્જ કંટ્રોલર}: બેટરી ચાર્જિંગનું નિયમન, ઓવરચાર્જિંગ રોકે
\item
  \textbf{બેટરી બેંક}: સૂર્યપ્રકાશ ન હોય ત્યારે ઉપયોગ માટે ઊર્જા સંગ્રહ
\item
  \textbf{ઇન્વર્ટર}: ઘરેલું ઉપકરણો માટે DC થી AC પાવરમાં રૂપાંતરણ
\item
  \textbf{લોડ્સ}: AC લોડ્સ (ઉપકરણો) અને DC લોડ્સ (LED લાઇટ્સ, વગેરે)
\end{enumerate}

\textbf{વૈકલ્પિક ઘટકો:}

\begin{itemize}
\tightlist
\item
  \textbf{મોનિટરિંગ સિસ્ટમ}: પાવર ઉત્પાદન/વપરાશ ટ્રેક કરે છે
\item
  \textbf{ગ્રિડ કનેક્શન}: વધારાની વીજળી વેચવાની મંજૂરી આપે છે
\end{itemize}

\end{solutionbox}
\begin{mnemonicbox}
``SCBIL'' - Solar Collects, Battery Inverts for
Loads

\end{mnemonicbox}
\subsection*{પ્રશ્ન 5(અ) OR [3
માર્ક્સ]}\label{uxaaauxab0uxab6uxaa8-5uxa85-or-3-uxaaeuxab0uxa95uxab8}

\textbf{આપણા ગ્રહ માટે ગ્રીન એનર્જીની જરૂરિયાત સમજાઓ.}

\begin{solutionbox}

\textbf{ગ્રીન એનર્જીની જરૂરિયાત:}

\begin{enumerate}
\tightlist
\item
  \textbf{ટકાઉપણું}: ફોસિલ ફ્યુઅલ્સની જેમ જ નહીં, પુનઃપ્રાપ્ય સ્ત્રોતો ખલાસ થતા નથી
\item
  \textbf{પ્રદૂષણ ઘટાડો}: ફોસિલ ફ્યુઅલ્સના બળવાથી હવા અને પાણીના પ્રદૂષણને ઘટાડે છે
\item
  \textbf{જળવાયુ પરિવર્તન}: ગ્લોબલ વોર્મિંગ પેદા કરતા ગ્રીનહાઉસ ગેસ ઉત્સર્જન ઘટાડે
  છે
\item
  \textbf{ઊર્જા સુરક્ષા}: આયાત કરેલા ફ્યુઅલ્સ પર નિર્ભરતા ઘટાડે છે
\item
  \textbf{આર્થિક લાભ}: નોકરીઓ સર્જે છે અને પ્રદૂષણ સંબંધિત આરોગ્ય ખર્ચ ઘટાડે છે
\end{enumerate}

\end{solutionbox}
\begin{mnemonicbox}
``SPECS'' - Sustainable, Pollution-free, Economic,
Climate-friendly, Secure

\end{mnemonicbox}
\subsection*{પ્રશ્ન 5(બ) OR [4
માર્ક્સ]}\label{uxaaauxab0uxab6uxaa8-5uxaac-or-4-uxaaeuxab0uxa95uxab8}

\textbf{ગ્રીન એનર્જીનું વર્ગીકરણ કરો અને કોઈ પણ એક સમજાઓ.}

\begin{solutionbox}

\textbf{ગ્રીન એનર્જી સ્ત્રોતોનું વર્ગીકરણ:}

\begin{verbatim}
mindmap
  root((Green Energy))
    Solar
    Wind
    Hydro
    Biomass
    Geothermal
    Tidal
\end{verbatim}

\textbf{સોલાર એનર્જી વિસ્તૃત રીતે:}

\begin{itemize}
\tightlist
\item
  \textbf{કાર્ય સિદ્ધાંત}: ફોટોવોલ્ટાઇક ઇફેક્ટ સૂર્યપ્રકાશને વીજળીમાં રૂપાંતરિત કરે છે
\item
  \textbf{ઘટકો}: સોલાર સેલ, પેનલ, ઇન્વર્ટર, બેટરી
\item
  \textbf{ઉપયોગો}: રહેણાંક પાવર, ઔદ્યોગિક ઉપયોગ, પરિવહન
\item
  \textbf{ફાયદા}: કોઈ પ્રદૂષણ નહીં, પુષ્કળ સ્ત્રોત, ઓછી જાળવણી
\item
  \textbf{મર્યાદાઓ}: હવામાન પર આધારિત, સ્ટોરેજની જરૂર, પ્રારંભિક ખર્ચ
\end{itemize}

\end{solutionbox}
\begin{mnemonicbox}
``SWHBGT'' - Sun Wind Hydro Biomass Geothermal Tidal
are green energy types

\end{mnemonicbox}
\subsection*{પ્રશ્ન 5(ક) OR [7
માર્ક્સ]}\label{uxaaauxab0uxab6uxaa8-5uxa95-or-7-uxaaeuxab0uxa95uxab8}

\textbf{વિન્ડ પાવર સીસ્ટમ નું ઓપરેશન બ્લોક ડાયાગ્રામ સાથે સમજાઓ.}

\begin{solutionbox}

\textbf{વિન્ડ પાવર સિસ્ટમ બ્લોક ડાયાગ્રામ:}

\begin{verbatim}
flowchart LR
    W[Wind Turbine] {-{-} G[Generator]}
    G {-{-} C[Controller]}
    C {-{-} B[Battery Storage]}
    C {-{-} I[Inverter]}
    I {-{-} L[Load]}
    C {-{-} GR[Grid Connection]}
\end{verbatim}

\textbf{ઓપરેશન:}

\begin{enumerate}
\tightlist
\item
  \textbf{વિન્ડ ટર્બાઇન}: પવનની ગતિજ ઊર્જાને યાંત્રિક ઊર્જામાં રૂપાંતરિત કરે છે
\item
  \textbf{જનરેટર}: યાંત્રિક રોટેશનને વીજ ઊર્જામાં રૂપાંતરિત કરે છે
\item
  \textbf{કંટ્રોલર}: પાવર આઉટપુટનું નિયમન કરે છે અને ઉચ્ચ પવનોથી રક્ષણ કરે છે
\item
  \textbf{બેટરી}: વધારાની ઊર્જા સંગ્રહિત કરે છે (ઓફ-ગ્રિડ સિસ્ટમ માટે)
\item
  \textbf{ઇન્વર્ટર}: વપરાશ માટે DC થી AC માં રૂપાંતરણ કરે છે
\item
  \textbf{ગ્રિડ કનેક્શન}: વધારાના પાવરને ગ્રિડમાં ફીડ કરે છે અથવા જરૂર પડે ત્યારે ખેંચે
  છે
\end{enumerate}

\textbf{વિન્ડ ટર્બાઇનના પ્રકારો:}

\begin{itemize}
\tightlist
\item
  હોરિઝોન્ટલ એક્સિસ (HAWT): મુખ્ય વ્યાપારિક પ્રકાર
\item
  વર્ટિકલ એક્સિસ (VAWT): શહેરી સેટિંગ્સ માટે વધુ સારું
\end{itemize}

\textbf{વિન્ડ સ્પીડ જરૂરિયાતો:}

\begin{itemize}
\tightlist
\item
  કટ-ઇન સ્પીડ: 3-5 m/s
\item
  રેટેડ આઉટપુટ: 12-15 m/s
\item
  કટ-આઉટ સ્પીડ: 25 m/s (સુરક્ષા માટે)
\end{itemize}

\end{solutionbox}
\begin{mnemonicbox}
``WGCBIL'' - Wind Generates, Controller Balances,
Inverter Loads

\end{mnemonicbox}

\end{document}
