\documentclass{article}

% content/resources/templates/preamble.tex
\usepackage[margin=0.6in]{geometry}
\author{Milav Dabgar}
\usepackage{amsmath,amssymb,amsthm}
\usepackage{booktabs}
\usepackage{multirow}
\usepackage{xcolor}
\usepackage{tcolorbox}
\tcbuselibrary{breakable,skins}
\usepackage[colorlinks=true,linkcolor=blue]{hyperref}
\usepackage{titlesec}
\usepackage{enumitem}
\usepackage{tikz}
\usepackage{pgfplots}
\usepackage{circuitikz}
\usepackage[version=4]{mhchem}
\usepackage{longtable}
\usepackage{array}
\usepackage{float}
\usepackage{caption}
\usepackage{listings}

\lstset{
  basicstyle=\small\ttfamily,
  breaklines=true,
  breakatwhitespace=false,
  postbreak=\mbox{\textcolor{red}{$\hookrightarrow$}\space},
  float=false,
  numbers=left,
  numberstyle=\tiny\color{gray},
  numbersep=10pt,
  xleftmargin=2em,
  keywordstyle=\color{blue},
  commentstyle=\color{green!60!black},
  stringstyle=\color{purple},
  backgroundcolor=\color{gray!5},
  showstringspaces=false,
  tabsize=2,
  captionpos=b,
  keepspaces=true,
  columns=flexible
}

\pgfplotsset{compat=1.18}
\usetikzlibrary{shapes,arrows,positioning,calc,patterns,decorations.pathmorphing,decorations.markings,arrows.meta}

% Color scheme
\definecolor{headcolor}{RGB}{0,102,204}
\definecolor{keycolor}{RGB}{220,20,60}
\definecolor{solutioncolor}{RGB}{34,139,34}
\definecolor{mnemoniccolor}{RGB}{148,0,211}
\definecolor{codecolor}{RGB}{0,0,100}

% Spacing
\setlength{\parskip}{3pt}
\setlist[itemize]{nosep}
\setlist[enumerate]{nosep}

% Title formatting
\titleformat{\section}{\Large\bfseries\color{headcolor}}{\thesection}{1em}{}
\titleformat{\subsection}{\large\bfseries\color{headcolor}}{\thesubsection}{1em}{}

% Pandoc tightlist compatibility
\providecommand{\tightlist}{%
  \setlength{\itemsep}{0pt}\setlength{\parskip}{0pt}}

% Pandoc longtable compatibility
\newcounter{none}
\def\thenone{}


% content/resources/templates/gujarati-boxes.tex
\usepackage{fontspec}
\usepackage{polyglossia}

% Set Gujarati as main language (document is primarily in Gujarati)
% Note: gloss-gujarati.ldf doesn't exist in polyglossia, but it will use hyphenation patterns
\setdefaultlanguage{gujarati}
\setotherlanguage{english}

% Configure Gujarati font properly
% Use Language=Default to prevent polyglossia from trying to add language-specific features
% that don't exist for Gujarati, which causes "empty feature" warnings
\newfontfamily\gujaratifont[Script=Gujarati,AutoFakeBold=2.5,AutoFakeSlant=0.3]{Noto Sans Gujarati}
\setmainfont[Script=Gujarati,AutoFakeBold=2.5,AutoFakeSlant=0.3]{Noto Sans Gujarati}
% Use Noto Sans Gujarati for monospace to support Gujarati in text
\setmonofont[Scale=0.9]{Noto Sans Gujarati}

% Configure English to use the same font
\newfontfamily\englishfont[Script=Gujarati,AutoFakeBold=2.5,AutoFakeSlant=0.3]{Noto Sans Gujarati}

% Translations for polyglossia
\gappto\captionsgujarati{
  \renewcommand{\tablename}{કોષ્ટક}
  \renewcommand{\figurename}{આકૃતિ}
}

% Helper for TikZ nodes to ensure Gujarati font
\newcommand{\gu}[1]{{\gujaratifont #1}}

% Custom environments
\newtcolorbox{solutionbox}{
    breakable,
    enhanced,
    colback=solutioncolor!5!white,
    colframe=solutioncolor!75!black,
    fonttitle=\bfseries,
    title=જવાબ
}

\newtcolorbox{solutionboxnobreak}{
 colback=solutioncolor!5!white,
 colframe=solutioncolor!75!black,
 fonttitle=\bfseries,
 title=જવાબ
}

\newtcolorbox{keyformula}{
 breakable,
 enhanced,
 colback=keycolor!5!white,
 colframe=keycolor!75!black,
 fonttitle=\bfseries,
 title=રાસાયણિક સમીકરણ/સૂત્ર
}

\newtcolorbox{mnemonicbox}{
 breakable,
 enhanced,
 colback=mnemoniccolor!5!white,
 colframe=mnemoniccolor!75!black,
 fonttitle=\bfseries,
 title=મેમરી ટ્રીક
}


% Custom commands for GTU solutions
% This file defines semantic commands for consistent formatting

% Question command with automatic formatting
\newcommand{\question}[2]{%
  \section*{Question #1}%
  \textbf{#2}%
}

% OR question variant
\newcommand{\questionor}[2]{%
  \section*{Question #1 OR}%
  \textbf{#2}%
}

% Proper table environment with caption
\newenvironment{answertable}[1]{%
  \begin{table}[htbp]
  \centering
  \caption{#1}
}{%
  \end{table}
}

% Proper figure environment for diagrams
\newenvironment{answerdiagram}[1]{%
  \begin{figure}[htbp]
  \centering
  \caption{#1}
}{%
  \end{figure}
}

% Semantic markup for key terms
\newcommand{\keyword}[1]{\textbf{#1}}
\newcommand{\code}[1]{\texttt{#1}}
\newcommand{\classname}[1]{\texttt{#1}}
\newcommand{\methodname}[1]{\texttt{#1}}

% Proper quotation marks
\newcommand{\mnemonic}[1]{``#1''}


\title{ફંડામેન્ટલ્સ ઓફ ઇલેક્ટ્રિકલ એન્જિનિયરિંગ (4311101) - સમર 2024 સોલ્યુશન}
\date{June 15, 2024}

\begin{document}
\maketitle

\questionmarks{1(a)}{3}{EMF, ઇલેક્ટ્રિક કરંટ અને પાવરની વ્યાખ્યા લખો. તથા તેઓના એકમ પણ લખો.}

\begin{solutionbox}
\textbf{જવાબ}:

\begin{center}
\captionof{table}{વ્યાખ્યા અને એકમ}
\begin{tabulary}{\linewidth}{|L|L|L|}
\hline
\textbf{શબ્દ} & \textbf{વ્યાખ્યા} & \textbf{એકમ} \\ \hline
\textbf{EMF (ઇલેક્ટ્રોમોટિવ ફોર્સ)} & એકમ ચાર્જ દીઠ સ્ત્રોત દ્વારા પૂરી પાડવામાં આવતી ઊર્જા & Volt (V) \\ \hline
\textbf{ઇલેક્ટ્રિક કરંટ} & ઇલેક્ટ્રિક ચાર્જના પ્રવાહનો દર & Ampere (A) \\ \hline
\textbf{પાવર} & જે દરે ઇલેક્ટ્રિકલ ઊર્જાનું સ્થાનાંતર થાય છે & Watt (W) \\ \hline
\end{tabulary}
\end{center}
\end{solutionbox}

\begin{mnemonicbox}
\mnemonic{EVA: EMF વોલ્ટમાં, કરંટ એમ્પિયરમાં, પાવર વોટમાં}
\end{mnemonicbox}

\questionmarks{1(b)}{4}{અનુક્રમે ૧૦૦૦ $\Omega$, ૨૦૦૦ $\Omega$ અને ૩૦૦૦ $\Omega$ નો રેઝિસ્ટન્સ ધરાવતા ત્રણ રેઝિસ્ટરને સિરીઝમાં જોડવામાં આવેલ છે. આ સિરીઝ જોડાણનો સમકક્ષ રેઝિસ્ટન્સ શોધો. હવે આ જ ત્રણ રેઝિસ્ટન્સને પેરેલલમાં જોડવામાં આવેલ છે. આ પેરેલલ જોડાણનો સમકક્ષ રેઝિસ્ટન્સ શોધો.}

\begin{solutionbox}
\textbf{જવાબ}:

\textbf{સિરીઝ જોડાણ માટે:}
\begin{align*}
R_{eq} &= R_1 + R_2 + R_3 \\
R_{eq} &= 1000 \Omega + 2000 \Omega + 3000 \Omega \\
R_{eq} &= 6000 \Omega
\end{align*}

\textbf{પેરેલલ જોડાણ માટે:}
\begin{align*}
\frac{1}{R_{eq}} &= \frac{1}{R_1} + \frac{1}{R_2} + \frac{1}{R_3} \\
\frac{1}{R_{eq}} &= \frac{1}{1000} + \frac{1}{2000} + \frac{1}{3000} \\
\frac{1}{R_{eq}} &= 0.001 + 0.0005 + 0.00033 \\
\frac{1}{R_{eq}} &= 0.00183 \\
R_{eq} &= 545.45 \Omega
\end{align*}

\begin{answerdiagram}{Resistor Connections}
\begin{center}
\begin{circuitikz}[american, scale=0.8, transform shape]
    % Series
    \draw (0,0) node[left]{In} to[R, l=1k$\Omega$] (2,0) to[R, l=2k$\Omega$] (4,0) to[R, l=3k$\Omega$] (6,0) node[right]{Out};
    \node at (3,-1) {Series Connection};

    % Parallel
    \begin{scope}[yshift=-3cm]
    \draw (0,1.5) -- (0,0) -- (6,0) -- (6,1.5);
    \draw (0,1.5) -- (6,1.5);
    \draw (1.5,1.5) to[R, l=1k$\Omega$] (1.5,0);
    \draw (3,1.5) to[R, l=2k$\Omega$] (3,0);
    \draw (4.5,1.5) to[R, l=3k$\Omega$] (4.5,0);
    \node at (0,0.75) [left] {In};
    \node at (6,0.75) [right] {Out};
    \node at (3,-1) {Parallel Connection};
    \end{scope}
\end{circuitikz}
\end{center}
\end{answerdiagram}
\end{solutionbox}

\begin{mnemonicbox}
\mnemonic{Series Sum, Parallel Product/Sum: સિરીઝમાં સીધા જ સરવાળો, પેરેલલમાં વ્યસ્ત સરવાળો}
\end{mnemonicbox}

\questionmarks{1(c)}{7}{રેઝિસ્ટર, કેપેસિટર અને ઇન્ડક્ટરની વ્યાખ્યા લખો. તેઓના સિમ્બોલ દોરો અને તેઓના એકમ લખો. તથા આ દરેક ડિવાઇસનો ઇલેક્ટ્રિક સર્કિટમાં શું ઉપયોગ છે તે લખો.}

\begin{solutionbox}
\textbf{જવાબ}:

\begin{center}
\captionof{table}{સર્કિટ ઘટકો}
\begin{tabulary}{\linewidth}{|L|L|C|L|L|}
\hline
\textbf{ઘટક} & \textbf{વ્યાખ્યા} & \textbf{સિમ્બોલ} & \textbf{એકમ} & \textbf{સર્કિટમાં ઉપયોગ} \\ \hline
\textbf{રેઝિસ્ટર} & એવું ઘટક જે ઇલેક્ટ્રિક કરંટના પ્રવાહનો વિરોધ કરે છે & \begin{circuitikz}[baseline, american, scale=0.5] \draw (0,0.2) to[R] (1,0.2); \end{circuitikz} & Ohm ($\Omega$) & કરંટને મર્યાદિત કરે છે, વોલ્ટેજ વિભાજન કરે છે, ગરમી ઉત્પન્ન કરે છે \\ \hline
\textbf{કેપેસિટર} & એવું ઘટક જે ઇલેક્ટ્રિક ચાર્જ સંગ્રહિત કરે છે & \begin{circuitikz}[baseline, american, scale=0.5] \draw (0,0.2) to[C] (1,0.2); \end{circuitikz} & Farad (F) & DC બ્લોક કરે છે, AC પસાર કરે છે, ઊર્જા સંગ્રહ, ફિલ્ટરિંગ \\ \hline
\textbf{ઇન્ડક્ટર} & એવું ઘટક જે ચુંબકીય ક્ષેત્રમાં ઊર્જા સંગ્રહિત કરે છે & \begin{circuitikz}[baseline, american, scale=0.5] \draw (0,0.2) to[L] (1,0.2); \end{circuitikz} & Henry (H) & AC બ્લોક કરે છે, DC પસાર કરે છે, ઊર્જા સંગ્રહ, ફિલ્ટરિંગ \\ \hline
\end{tabulary}
\end{center}
\end{solutionbox}

\begin{mnemonicbox}
\mnemonic{RCI: રેઝિસ્ટર કરંટ નિયંત્રિત કરે છે, કેપેસિટર ચાર્જ સંગ્રહે છે, ઇન્ડક્ટર ચુંબકીય ઊર્જા સંગ્રહે છે}
\end{mnemonicbox}

\questionmarks{1(c) OR}{7}{ઓહમનો નિયમ તથા ઓહમના નિયમનું સમીકરણ સર્કિટ ડાયાગ્રામની મદદથી લખો. ઓહમના નિયમના ઉપયોગો લખો. તથા ઓહમના નિયમની મર્યાદા લખો.}

\begin{solutionbox}
\textbf{જવાબ}:

\keyword{ઓહમનો નિયમ:} કોઈ વાહક માંથી પસાર થતો કરંટ, તેના છેડા પરના વોલ્ટેજના સીધા પ્રમાણમાં અને તેના અવરોધના વ્યસ્ત પ્રમાણમાં હોય છે.

\keyword{સમીકરણ:} $V = I \times R$

\begin{answerdiagram}{Ohm's Law Circuit}
\begin{circuitikz}[american]
    \draw (0,0) to[V, v=$V$] (0,2) -- (2,2)
          to[R, l=$R$, i=$I$] (2,0) -- (0,0);
\end{circuitikz}
\end{answerdiagram}

\keyword{ઓહમના નિયમના ઉપયોગો:}
\begin{itemize}
    \item સર્કિટમાં કરંટ, વોલ્ટેજ, અથવા અવરોધની ગણતરી કરવા
    \item ઇલેક્ટ્રિકલ અને ઇલેક્ટ્રોનિક સર્કિટની ડિઝાઇન કરવા
    \item પાવરની ગણતરી કરવા ($P = V \times I = I^2 \times R = V^2/R$)
    \item વોલ્ટેજ ડિવાઇડર અને કરંટ ડિવાઇડરનો ઉપયોગ કરીને સર્કિટનું વિશ્લેષણ
\end{itemize}

\keyword{ઓહમના નિયમની મર્યાદા:}
\begin{itemize}
    \item નોન-લિનિયર ઉપકરણો (ડાયોડ, ટ્રાન્ઝિસ્ટર) માટે લાગુ પડતો નથી
    \item ઉચ્ચ ફ્રિક્વન્સી AC સર્કિટ માટે માન્ય નથી
    \item બિન-ધાતુ વાહકો માટે લાગુ પડતો નથી
    \item પરિવર્તનશીલ પરિસ્થિતિઓમાં લાગુ પડતો નથી
\end{itemize}
\end{solutionbox}

\begin{mnemonicbox}
\mnemonic{VIR: વોલ્ટેજ = કરંટ ગુણ્યા અવરોધ}
\end{mnemonicbox}

\questionmarks{2(a)}{3}{જરૂરી ડાયાગ્રામ અને સમીકરણની મદદથી ઓલ્ટરનેટિંગ EMF કઈ રીતે ઉત્પન્ન કરવામાં આવે છે તે સમજાવો.}

\begin{solutionbox}
\textbf{જવાબ}:

ઓલ્ટરનેટિંગ EMF ત્યારે ઉત્પન્ન થાય છે જ્યારે વાહક ચુંબકીય ક્ષેત્રમાં ફરે છે.

\keyword{સમીકરણ:} $e = E_0 \sin(\omega t) = E_0 \sin(2\pi ft)$

જ્યાં:
\begin{itemize}
    \item $e$ = તત્કાલિક EMF
    \item $E_0$ = મહત્તમ EMF
    \item $\omega$ = કોણીય વેગ ($2\pi f$)
    \item $f$ = આવૃત્તિ
    \item $t$ = સમય
\end{itemize}

\begin{answerdiagram}{AC Generation Principle}
\begin{tikzpicture}
    % Magnetic Poles
    \draw[fill=red!20] (-2.5,-1.5) rectangle (-1.5,1.5) node[midway]{N};
    \draw[fill=blue!20] (1.5,-1.5) rectangle (2.5,1.5) node[midway]{S};
    
    % Field Lines
    \draw[dashed, ->] (-1.5,1) -- (1.5,1);
    \draw[dashed, ->] (-1.5,0) -- (1.5,0);
    \draw[dashed, ->] (-1.5,-1) -- (1.5,-1);
    
    % Coil
    \draw[thick] (-0.8,-0.5) rectangle (0.8,0.5);
    \draw[->] (0,0.8) arc (90:45:0.8) node[midway, above right] {$\omega$};
    \node at (0,-1) {Rotating Coil};
    
    % Connections
    \draw (0.8,0) -- (1,0) -- (1,-2) node[below] {To Slip Rings};
    \draw (-0.8,0) -- (-1,0) -- (-1,-2);
\end{tikzpicture}
\end{answerdiagram}
\end{solutionbox}

\begin{mnemonicbox}
\mnemonic{RCBS: ચુંબકીય ક્ષેત્રમાં કોઇલનું ફરવું સાઇનસોઇડલ EMF ઉત્પન્ન કરે છે}
\end{mnemonicbox}

\questionmarks{2(b)}{4}{જરૂરી સર્કિટ ડાયાગ્રામ અને સમીકરણની મદદથી શુદ્ધ કેપેસિટર સાથે AC વૉલ્ટેજની વર્તણૂક સમજાવો.}

\begin{solutionbox}
\textbf{જવાબ}:

\keyword{શુદ્ધ કેપેસિટર સાથે AC ની વર્તણૂક:}
\begin{itemize}
    \item શુદ્ધ કેપેસિટરમાં કરંટ વોલ્ટેજથી 90$^\circ$ આગળ હોય છે
    \item કેપેસિટિવ રિએક્ટન્સ ($X_c$) = $1/(2\pi fC)$
    \item જેમ ફ્રિક્વન્સી વધે છે, તેમ રિએક્ટન્સ ઘટે છે
    \item ચાર્જિંગ દરમિયાન ઇલેક્ટ્રિક ફીલ્ડમાં ઊર્જા સંગ્રહે છે
\end{itemize}

\begin{answerdiagram}{Capacitor Circuit and Waveform}
\begin{circuitikz}[american, scale=0.8]
    \draw (0,0) to[sinusoidal voltage source, l=AC] (0,2) -- (2,2) to[C, l=C] (2,0) -- (0,0);
\end{circuitikz}
\quad
\begin{tikzpicture}[scale=0.8]
    \begin{axis}[
        width=6cm, height=4cm,
        axis lines=middle,
        xtick={0, 1.57, 3.14},
        xticklabels={0, $\pi/2$, $\pi$},
        ytick=\empty,
        xlabel=$\omega t$,
        legend style={at={(0.5,-0.3)}, anchor=north, legend columns=-1, draw=none}
    ]
    \addplot[blue, thick, domain=0:4, samples=100] {sin(deg(x))};
    \addlegendentry{$V$}
    \addplot[red, dashed, thick, domain=0:4, samples=100] {cos(deg(x))};
    \addlegendentry{$I$}
    \end{axis}
\end{tikzpicture}
\end{answerdiagram}

\keyword{સમીકરણ:} $I = C \times \frac{dV}{dt}$
\end{solutionbox}

\begin{mnemonicbox}
\mnemonic{CIVIC: કેપેસિટરમાં કરંટ વોલ્ટેજથી 90 આગળ હોય છે}
\end{mnemonicbox}

\questionmarks{2(c)}{7}{એક AC વૉલ્ટેજને 300 Sin (628t) V વડે દર્શાવવામાં આવેલ છે. આ વૉલ્ટેજ માટે (i) એમ્પલીટ્યુડ (ii) આવૃત્તિ (ફ્રિક્વન્સી) (iii) ટાઈમ પિરિયડ (iv) એવરેજ વેલ્યૂ (v) RMS વેલ્યૂ (vi) ફોર્મ ફેક્ટર અને (vii) પીક ફેક્ટર ની વેલ્યૂ શોધો.}

\begin{solutionbox}
\textbf{જવાબ}:

આપેલ છે: $v = 300 \sin(628t)$ V

\begin{center}
\captionof{table}{ગણતરી કરેલ પરિમાણો}
\begin{tabulary}{\linewidth}{|L|L|L|L|}
\hline
\textbf{પરિમાણ} & \textbf{સૂત્ર} & \textbf{ગણતરી} & \textbf{પરિણામ} \\ \hline
\textbf{એમ્પલીટ્યુડ} & $V_m$ & 300 V & 300 V \\ \hline
\textbf{કોણીય આવૃત્તિ} & $\omega$ & 628 rad/s & 628 rad/s \\ \hline
\textbf{આવૃત્તિ} & $f = \omega/2\pi$ & $628/6.28$ & 100 Hz \\ \hline
\textbf{ટાઈમ પિરિયડ} & $T = 1/f$ & $1/100$ & 0.01 s \\ \hline
\textbf{એવરેજ વેલ્યૂ} & $V_{avg} = 2V_m/\pi$ & $2 \times 300 / 3.14$ & 191 V \\ \hline
\textbf{RMS વેલ્યૂ} & $V_{rms} = V_m/\sqrt{2}$ & $300/1.414$ & 212.16 V \\ \hline
\textbf{ફોર્મ ફેક્ટર} & $FF = V_{rms}/V_{avg}$ & $212.16/191$ & 1.11 \\ \hline
\textbf{પીક ફેક્ટર} & $PF = V_m/V_{rms}$ & $300/212.16$ & 1.414 \\ \hline
\end{tabulary}
\end{center}
\end{solutionbox}

\begin{mnemonicbox}
\mnemonic{FART FAFP: ફ્રિક્વન્સી, કોણીય, RMS, ટાઈમ પિરિયડ, ફોર્મ ફેક્ટર, એવરેજ, પીક ફેક્ટર}
\end{mnemonicbox}

\questionmarks{2(a) OR}{3}{3-ફેઝ ઓલ્ટરનેટિંગ EMF કઈ રીતે ઉત્પન્ન કરવામાં આવે છે તે સમજાવો.}

\begin{solutionbox}
\textbf{જવાબ}:

3-ફેઝ ઓલ્ટરનેટિંગ EMF ચુંબકીય ક્ષેત્રમાં 120$^\circ$ અંતરે મૂકેલી ત્રણ અલગ કોઇલનો ઉપયોગ કરીને ઉત્પન્ન થાય છે.

\keyword{મુખ્ય મુદ્દાઓ:}
\begin{itemize}
    \item ત્રણ સમાન કોઇલ 120$^\circ$ અંતરે મૂકવામાં આવે છે
    \item દરેક કોઇલ સાઇનુસોઇડલ EMF ઉત્પન્ન કરે છે
    \item ફેઝને R, Y, અને B (અથવા U, V, W) તરીકે લેબલ કરવામાં આવે છે
    \item કોઈપણ બે ફેઝ વચ્ચેનો ફેઝ તફાવત 120$^\circ$ છે
\end{itemize}

\begin{answerdiagram}{3-Phase Generation}
\begin{tikzpicture}
    % Stator
    \draw (0,0) circle (2cm);
    \node at (0,2.2) {$R$};
    \node at (0,-2.2) {$R'$};
    \node at (1.9,1.1) {$B'$};
    \node at (-1.9,-1.1) {$B$};
    \node at (1.9,-1.1) {$Y$};
    \node at (-1.9,1.1) {$Y'$};
    
    % Rotor
    \draw[fill=gray!20] (0,0) circle (0.5cm);
    \draw[->, thick] (0,0) -- (1.5,0) node[right] {N};
    \draw[->, thick] (0,0) -- (-1.5,0) node[left] {S};
    \draw[->] (0.5,0.5) arc (45:135:0.5) node[midway, above] {$\omega$};
\end{tikzpicture}
\end{answerdiagram}
\end{solutionbox}

\begin{mnemonicbox}
\mnemonic{THREE: ત્રણ કોઇલ 120° અંતરે ફરતી EMF ઉત્પન્ન કરે છે}
\end{mnemonicbox}

\questionmarks{2(b) OR}{4}{જરૂરી સર્કિટ ડાયાગ્રામ અને સમીકરણની મદદથી શુદ્ધ ઇન્ડક્ટર સાથે AC વૉલ્ટેજની વર્તણૂક સમજાવો.}

\begin{solutionbox}
\textbf{જવાબ}:

\keyword{શુદ્ધ ઇન્ડક્ટર સાથે AC ની વર્તણૂક:}
\begin{itemize}
    \item શુદ્ધ ઇન્ડક્ટરમાં કરંટ વોલ્ટેજથી 90$^\circ$ પાછળ હોય છે
    \item ઇન્ડક્ટિવ રિએક્ટન્સ ($X_L$) = $2\pi fL$
    \item જેમ ફ્રિક્વન્સી વધે છે, તેમ રિએક્ટન્સ વધે છે
    \item ચુંબકીય ક્ષેત્રમાં ઊર્જા સંગ્રહે છે
\end{itemize}

\begin{answerdiagram}{Inductor Circuit and Waveform}
\begin{circuitikz}[american, scale=0.8]
    \draw (0,0) to[sinusoidal voltage source, l=AC] (0,2) -- (2,2) to[L, l=L] (2,0) -- (0,0);
\end{circuitikz}
\quad
\begin{tikzpicture}[scale=0.8]
    \begin{axis}[
        width=6cm, height=4cm,
        axis lines=middle,
        xtick={0, 1.57, 3.14},
        xticklabels={0, $\pi/2$, $\pi$},
        ytick=\empty,
        xlabel=$\omega t$,
        legend style={at={(0.5,-0.3)}, anchor=north, legend columns=-1, draw=none}
    ]
    \addplot[blue, thick, domain=0:4, samples=100] {sin(deg(x))};
    \addlegendentry{$V$}
    \addplot[red, dashed, thick, domain=0:4, samples=100] {sin(deg(x)-90)};
    \addlegendentry{$I$}
    \end{axis}
\end{tikzpicture}
\end{answerdiagram}

\keyword{સમીકરણ:} $V = L \times \frac{dI}{dt}$
\end{solutionbox}

\begin{mnemonicbox}
\mnemonic{VLIC: ઇન્ડક્ટરમાં વોલ્ટેજ કરંટથી 90 આગળ હોય છે}
\end{mnemonicbox}

\questionmarks{2(c) OR}{7}{3-ફેઝ AC માટે ફેઝ વૉલ્ટેજ, લાઇન વૉલ્ટેજ, ફેઝ કરંટ અને લાઇન કરંટની વ્યાખ્યા લખો. (i) સ્ટાર (Y) કનેક્શન માટે જો ફેઝ વૉલ્ટેજની વેલ્યૂ 100V હોય તો લાઇન વૉલ્ટેજની વેલ્યૂ શોધો. તથા સ્ટાર (Y) કનેક્શન માટે જો ફેઝ કરંટની વેલ્યૂ 5A હોય તો લાઇન કરંટની વેલ્યૂ શોધો (ii) ડેલ્ટા ($\Delta$) કનેક્શન માટે જો ફેઝ વૉલ્ટેજની વેલ્યૂ 100V હોય તો લાઇન વૉલ્ટેજની વેલ્યૂ શોધો. તથા ડેલ્ટા ($\Delta$) કનેક્શન માટે જો ફેઝ કરંટની વેલ્યૂ 5A હોય તો લાઇન કરંટની વેલ્યૂ શોધો.}

\begin{solutionbox}
\textbf{જવાબ}:

\begin{center}
\captionof{table}{3-ફેઝ વ્યાખ્યાઓ}
\begin{tabulary}{\linewidth}{|L|L|}
\hline
\textbf{શબ્દ} & \textbf{વ્યાખ્યા} \\ \hline
\textbf{ફેઝ વૉલ્ટેજ} & સિંગલ ફેઝ ઘટક પરનો વૉલ્ટેજ \\ \hline
\textbf{લાઇન વૉલ્ટેજ} & કોઈપણ બે લાઇન વચ્ચેનો વૉલ્ટેજ \\ \hline
\textbf{ફેઝ કરંટ} & ફેઝ ઘટકમાંથી વહેતો કરંટ \\ \hline
\textbf{લાઇન કરંટ} & લાઇનમાંથી વહેતો કરંટ \\ \hline
\end{tabulary}
\end{center}

\textbf{સ્ટાર (Y) કનેક્શન:}
\begin{itemize}
    \item લાઇન વૉલ્ટેજ = $\sqrt{3} \times$ ફેઝ વૉલ્ટેજ = $\sqrt{3} \times 100 = 173.2$ V
    \item લાઇન કરંટ = ફેઝ કરંટ = 5 A
\end{itemize}

\textbf{ડેલ્ટા ($\Delta$) કનેક્શન:}
\begin{itemize}
    \item લાઇન વૉલ્ટેજ = ફેઝ વૉલ્ટેજ = 100 V
    \item લાઇન કરંટ = $\sqrt{3} \times$ ફેઝ કરંટ = $\sqrt{3} \times 5 = 8.66$ A
\end{itemize}

\begin{answerdiagram}{Star and Delta Connections}
\begin{circuitikz}[scale=0.7, transform shape]
    % Star
    \draw (0,0) node[anchor=north]{N} to[R, l=R] (0,2) node[anchor=south]{R};
    \draw (0,0) to[R, l=Y] (-1.73,-1) node[anchor=north]{Y};
    \draw (0,0) to[R, l=B] (1.73,-1) node[anchor=north]{B};
    \node at (0,-2.5) {Star Connection};

    % Delta
    \begin{scope}[xshift=5cm, yshift=-1cm]
    \draw (0,0) to[R, l=B] (4,0);
    \draw (0,0) to[R, l=R] (2,3.46);
    \draw (2,3.46) to[R, l=Y] (4,0);
    \node at (2,-1.5) {Delta Connection};
    \end{scope}
\end{circuitikz}
\end{answerdiagram}
\end{solutionbox}

\begin{mnemonicbox}
\mnemonic{SLIP: સ્ટાર કનેક્શનમાં: લાઇન વૉલ્ટેજ = root3 ફેઝ, ડેલ્ટામાં: ફેઝ = લાઇન}
\end{mnemonicbox}

\questionmarks{3(a)}{3}{જરૂરી ડાયાગ્રામ અને સમીકરણની મદદથી ફેરાડેના ઇલેક્ટ્રોમેગ્નેટિક ઇન્ડકશનના નિયમોને લખો અને સમજાવો.}

\begin{solutionbox}
\textbf{જવાબ}:

\keyword{ફેરાડેના નિયમો:}
\begin{enumerate}
    \item \keyword{પ્રથમ નિયમ:} જ્યારે વાહક ચુંબકીય ફ્લક્સને કાપે છે, ત્યારે EMF ઇન્ડ્યુસ થાય છે.
    \item \keyword{બીજો નિયમ:} ઇન્ડ્યુસ થયેલા EMF નો પરિમાણ ચુંબકીય ફ્લક્સના પરિવર્તનના દર સાથે પ્રમાણમાં હોય છે.
\end{enumerate}

\keyword{સમીકરણ:} $e = -N \frac{d\Phi}{dt}$

\begin{answerdiagram}{Faraday's Experiment}
\begin{tikzpicture}
    % Coil
    \foreach \x in {0,0.5,1,1.5,2}
        \draw[thick] (\x,0) ellipse (0.2 and 0.5);
    \draw[thick] (0,0.5) -- (-1,0.5) -- (-1, -1.5) -- (3,-1.5) -- (3,0.5) -- (2,0.5);
    
    % Galvanometer
    \draw[fill=white] (1,-1.5) circle(0.4);
    \node at (1,-1.5) {G};
    \draw[->] (1,-1.5) -- (1.2,-1.3);

    % Magnet
    \draw[fill=red!20] (-2.5,0) rectangle (-1.5,1) node[midway]{N};
    \draw[fill=blue!20] (-3.5,0) rectangle (-2.5,1) node[midway]{S};
    \draw[->, thick] (-1.4, 0.5) -- (-0.5, 0.5) node[midway, above] {$v$};
\end{tikzpicture}
\end{answerdiagram}
\end{solutionbox}

\begin{mnemonicbox}
\mnemonic{FIRE: ફ્લક્સમાં પરિવર્તન EMF ઇન્ડ્યુસ કરે છે}
\end{mnemonicbox}

\questionmarks{3(b)}{4}{ઓલ્ટરનેટિંગ ક્વોન્ટિટી માટે એમ્પલિટ્યુડ, ફ્રિક્વન્સી (આવૃત્તિ), ટાઈમ પિરિયડ અને RMS વેલ્યૂની વ્યાખ્યા લખો.}

\begin{solutionbox}
\textbf{જવાબ}:

\begin{center}
\captionof{table}{AC પરિમાણો}
\begin{tabulary}{\linewidth}{|L|L|L|}
\hline
\textbf{પરિમાણ} & \textbf{વ્યાખ્યા} & \textbf{સૂત્ર} \\ \hline
\textbf{એમ્પલિટ્યુડ} & ઓલ્ટરનેટિંગ ક્વોન્ટિટીનું મહત્તમ મૂલ્ય & $V_m$ \\ \hline
\textbf{ફ્રિક્વન્સી} & એક સેકન્ડમાં પૂર્ણ થતા ચક્રોની સંખ્યા & $f = 1/T$ \\ \hline
\textbf{ટાઈમ પિરિયડ} & એક ચક્ર પૂર્ણ કરવા માટે લાગતો સમય & $T = 1/f$ \\ \hline
\textbf{RMS મૂલ્ય} & અસરકારક મૂલ્ય, સમાન હીટિંગ ઉત્પન્ન કરતા DC ના બરાબર & $V_{rms} = 0.707V_m$ \\ \hline
\end{tabulary}
\end{center}

\begin{answerdiagram}{Waveform Parameters}
\begin{tikzpicture}
    \begin{axis}[
        width=8cm, height=4cm,
        axis lines=middle,
        xtick={0, 6.28},
        xticklabels={0, $2\pi$},
        ytick={1},
        yticklabels={$V_m$},
        xlabel=$\omega t$,
        ylabel=$V$
    ]
    \addplot[blue, thick, domain=0:6.5, samples=100] {sin(deg(x))};
    \draw[<->] (axis cs:0,-1.2) -- (axis cs:6.28,-1.2) node[midway, below] {Time Period $T$};
    \draw[dashed] (axis cs:1.57,0) -- (axis cs:1.57,1);
    \node at (axis cs:1.57,1.2) {Amplitude};
    \end{axis}
\end{tikzpicture}
\end{answerdiagram}
\end{solutionbox}

\begin{mnemonicbox}
\mnemonic{AFTR: એમ્પલિટ્યુડ મહત્તમ છે, ફ્રિક્વન્સી દર સેકન્ડે ચક્રો, ટાઈમ પિરિયડ 1/f છે, RMS અસરકારક છે}
\end{mnemonicbox}

\questionmarks{3(c)}{7}{સેલ્ફ ઇન્ડકટન્સ અને મ્યુચ્યુઅલ ઇન્ડકટન્સ સમજાવો. (i) જો કોઈલને 2 A કરંટ આપવાથી તેમાં 5 $\mu$Wb-turns જેટલું મેગ્નેટિક ફલ્સ કોઇલમાં ઇનડયૂસ થતું હોય તો કોઇલનું સેલ્ફ ઇન્ડકટન્સ શોધો (ii) કોઇલનું સેલ્ફ ઇન્ડકટન્સ શોધો જો આપેલ કોઇલના ભૌતિક પરિમાણો નીચે પ્રમાણે આપેલ હોય: કોઇલના ટર્નસ 10, કોઇલના મટિરિયલની રિલેટિવ પરમીએબીલીટી 3, કોઇલની લંબાઈ 5 cm અને કોઇલનો ક્રોસ સેક્શનલ એરિયા 2 cm$^2$ હોય.}

\begin{solutionbox}
\textbf{જવાબ}:

\keyword{સેલ્ફ ઇન્ડકટન્સ:} કોઇલનો એવો ગુણધર્મ જે તેમાંથી પસાર થતા કરંટમાં પરિવર્તનનો વિરોધ પોતાનામાં EMF ઉત્પન્ન કરીને કરે છે.

\keyword{મ્યુચ્યુઅલ ઇન્ડકટન્સ:} એક કોઇલનો એવો ગુણધર્મ જેનાથી તેમાંથી પસાર થતા કરંટમાં પરિવર્તનને કારણે બીજી કોઇલમાં EMF ઉત્પન્ન થાય છે.

\textbf{ભાગ (i):}
\begin{align*}
L &= \frac{\text{Flux Linkage}}{\text{Current}} \\
L &= \frac{5 \mu\text{Wb-turns}}{2 \text{A}} = 2.5 \mu\text{H}
\end{align*}

\textbf{ભાગ (ii):}
\begin{align*}
L &= \frac{\mu_0 \mu_r N^2 A}{l} \\
L &= \frac{4\pi \times 10^{-7} \times 3 \times 10^2 \times 2 \times 10^{-4}}{5 \times 10^{-2}} \\
L &= 15.07 \mu\text{H}
\end{align*}

\begin{answerdiagram}{Self vs Mutual Inductance}
\begin{tikzpicture}
    % Self
    \draw[thick] (0,0) ellipse (0.2 and 0.5);
    \draw[thick] (0.5,0) ellipse (0.2 and 0.5);
    \node at (0.25, -1) {Self Inductance};
    \draw[->] (-0.5,0) -- (0,0) node[midway, above] {$I$};

    % Mutual
    \begin{scope}[xshift=4cm]
    \draw[thick] (0,0) ellipse (0.2 and 0.5);
    \draw[thick] (2,0) ellipse (0.2 and 0.5);
    \draw[->, wave] (0.5,0.2) -- (1.5,0.2) node[midway, above] {$\Phi$};
    \node at (1, -1) {Mutual Inductance};
    \end{scope}
\end{tikzpicture}
\end{answerdiagram}
\end{solutionbox}

\begin{mnemonicbox}
\mnemonic{SLIM: સેલ્ફ ઇન્ડકટન્સ પોતાના ફ્લક્સથી, ઇન્ડકશન બે કોઇલ વચ્ચે મ્યુચ્યુઅલ}
\end{mnemonicbox}

\questionmarks{3(a) OR}{3}{ડાયનેમિકલી ઇનડયૂસડ ઈએમએફની વ્યાખ્યા લખો. જરૂરી ડાયાગ્રામ અને સમીકરણની મદદથી ડાયનેમિકલી ઇનડયૂસડ ઈએમએફને સમજાવો.}

\begin{solutionbox}
\textbf{જવાબ}:

\keyword{ડાયનેમિકલી ઇનડયૂસડ EMF:} વાહક અને ચુંબકીય ક્ષેત્ર વચ્ચેના સાપેક્ષ ગતિને કારણે વાહકમાં ઉત્પન્ન થતું EMF.

\keyword{સમીકરણ:} $e = B l v \sin\theta$
જ્યાં $B$ ચુંબકીય ફ્લક્સ ઘનતા, $l$ લંબાઈ, $v$ વેગ.

\begin{answerdiagram}{Dynamic EMF}
\begin{tikzpicture}
    \draw[fill=red!20] (0,0) rectangle (1,2) node[midway]{N};
    \draw[fill=blue!20] (4,0) rectangle (5,2) node[midway]{S};
    \draw[dashed, ->] (1,1) -- (4,1) node[midway, above] {$B$};
    
    \draw[ultra thick] (2.5, 0.5) -- (2.5, 1.5) node[midway, left] {Conductor};
    \draw[->, thick] (2.5, 1.5) -- (2.5, 2.2) node[right] {$v$};
\end{tikzpicture}
\end{answerdiagram}
\end{solutionbox}

\begin{mnemonicbox}
\mnemonic{MOVE: ચુંબકીય ક્ષેત્રમાં વાહકની ગતિ વોલ્ટેજ ઉત્પન્ન કરે છે}
\end{mnemonicbox}

\questionmarks{3(b) OR}{4}{ઓલ્ટરનેટિંગ ક્વોન્ટિટી માટે સાઇકલ, ફોર્મ ફેક્ટર અને પીક ફેક્ટરની વ્યાખ્યા લખો. તથા સાઈનુંસોઈડલ ક્વોન્ટિટી માટે ફોર્મ ફેક્ટર અને પીક ફેક્ટરની વેલ્યૂ લખો.}

\begin{solutionbox}
\textbf{જવાબ}:

\begin{center}
\captionof{table}{AC પરિમાણો}
\begin{tabulary}{\linewidth}{|L|L|L|}
\hline
\textbf{શબ્દ} & \textbf{વ્યાખ્યા} & \textbf{મૂલ્ય} \\ \hline
\textbf{સાઇકલ} & ઓલ્ટરનેટિંગ ક્વોન્ટિટીનું એક સંપૂર્ણ આંદોલન & - \\ \hline
\textbf{ફોર્મ ફેક્ટર} & RMS મૂલ્ય અને સરેરાશ મૂલ્યનો ગુણોત્તર ($V_{rms}/V_{avg}$) & 1.11 \\ \hline
\textbf{પીક ફેક્ટર} & મહત્તમ મૂલ્ય અને RMS મૂલ્યનો ગુણોત્તર ($V_m/V_{rms}$) & 1.414 \\ \hline
\end{tabulary}
\end{center}
\end{solutionbox}

\begin{mnemonicbox}
\mnemonic{CFP: સાઇકલ એક આંદોલન, ફોર્મ ફેક્ટર 1.11, પીક ફેક્ટર 1.414}
\end{mnemonicbox}

\questionmarks{3(c) OR}{7}{લેન્ઝનો નિયમ લખો અને સમજાવો. જનરેટર માટે ફ્લેમિંગનો જમણા હાથનો નિયમ લખો અને સમજાવો. જો 4 $\mu$H સેલ્ફ ઇન્ડકટન્સ ધરાવતા ઇન્ડક્ટરમાંથી 3 A કરંટ પસાર થતો હોય તો તે ઇન્ડક્ટરમાં સંગ્રહ થયેલ ઉર્જા શોધો.}

\begin{solutionbox}
\textbf{જવાબ}:

\keyword{લેન્ઝનો નિયમ:} ઇન્ડ્યુસ થયેલા EMF ની દિશા એવી હોય છે કે તે ચુંબકીય ફ્લક્સમાં થતા પરિવર્તનનો વિરોધ કરે છે.

\keyword{ફ્લેમિંગનો જમણા હાથનો નિયમ:}
\begin{itemize}
    \item \textbf{અંગૂઠો:} વાહકની ગતિની દિશા
    \item \textbf{પ્રથમ આંગળી:} ચુંબકીય ક્ષેત્રની દિશા
    \item \textbf{મધ્યમા આંગળી:} ઇન્ડ્યુસ થયેલા કરંટની દિશા
\end{itemize}

\keyword{ઊર્જાની ગણતરી:}
\begin{align*}
E &= \frac{1}{2} L I^2 \\
E &= \frac{1}{2} \times 4 \times 10^{-6} \times 3^2 \\
E &= 18 \times 10^{-6} \text{ J} = 18 \mu\text{J}
\end{align*}

\begin{answerdiagram}{Fleming's Right Hand Rule}
\begin{tikzpicture}
    \draw[->, ultra thick, blue] (0,0) -- (0,2) node[above] {Thumb: Motion};
    \draw[->, ultra thick, red] (0,0) -- (2,0) node[right] {Index: Field};
    \draw[->, ultra thick, green!60!black] (0,0) -- (0,0,-2) node[left] {Middle: Current};
    \node at (0,0) [circle, fill=black, inner sep=2pt] {};
\end{tikzpicture}
\end{answerdiagram}
\end{solutionbox}

\begin{mnemonicbox}
\mnemonic{LOF: લેન્ઝનો નિયમ ફ્લક્સ પરિવર્તનનો વિરોધ કરે છે, ફ્લેમિંગનો નિયમ જનરેટર માટે}
\end{mnemonicbox}

\questionmarks{4(a)}{3}{PV સેલની વ્યાખ્યા લખો. PV સેલનું કાર્ય સમજાવો.}

\begin{solutionbox}
\textbf{જવાબ}:

\keyword{PV સેલ:} ફોટોવોલ્ટેઇક સેલ એક અર્ધવાહક ઉપકરણ છે જે પ્રકાશ ઊર્જાને સીધી જ વિદ્યુત ઊર્જામાં રૂપાંતરિત કરે છે.

\keyword{કાર્ય:}
\begin{itemize}
    \item સૂર્યપ્રકાશમાંથી ફોટોન્સ શોષે છે
    \item ઇલેક્ટ્રોન-હોલ જોડી બનાવે છે
    \item p-n જંક્શન પર પોટેન્શિયલ તફાવત ઉત્પન્ન કરે છે
    \item વિદ્યુત ઊર્જામાં રૂપાંતરિત કરે છે
\end{itemize}

\begin{answerdiagram}{PV Cell Operation}
\begin{tikzpicture}
    \draw[fill=yellow!20] (0,1) rectangle (4,2) node[midway] {N-type};
    \draw[fill=green!20] (0,0) rectangle (4,1) node[midway] {P-type};
    \draw[->, decorate, decoration={snake, amplitude=1mm, segment length=2mm, post length=1mm}, thick, orange] (1,3) -- (1,2);
    \draw[->, decorate, decoration={snake, amplitude=1mm, segment length=2mm, post length=1mm}, thick, orange] (2,3) -- (2,2);
    \draw[->, decorate, decoration={snake, amplitude=1mm, segment length=2mm, post length=1mm}, thick, orange] (3,3) -- (3,2);
    \node at (2,3.5) {Sunlight};
    \draw (0,1.5) -- (-1,1.5) -- (-1,-0.5) to[R, l=Load] (5,-0.5) -- (5,0.5) -- (4,0.5);
\end{tikzpicture}
\end{answerdiagram}
\end{solutionbox}

\begin{mnemonicbox}
\mnemonic{PASE: PV સેલ સૂર્યપ્રકાશ શોષે છે અને વીજળી ઉત્પન્ન કરે છે}
\end{mnemonicbox}

\questionmarks{4(b)}{4}{ગ્રીન એનર્જીનું વર્ગીકરણ સમજાવો.}

\begin{solutionbox}
\textbf{જવાબ}:

\begin{center}
\captionof{table}{ગ્રીન એનર્જી સ્ત્રોત}
\begin{tabulary}{\linewidth}{|L|L|}
\hline
\textbf{પ્રકાર} & \textbf{સ્ત્રોત} \\ \hline
\textbf{સૌર ઊર્જા (Solar)} & સૂર્ય (PV, Thermal) \\ \hline
\textbf{પવન ઊર્જા (Wind)} & વાયુ પ્રવાહ (Turbines) \\ \hline
\textbf{જળ ઊર્જા (Hydro)} & વહેતું પાણી (Dams, Tidal) \\ \hline
\textbf{બાયોમાસ (Biomass)} & જૈવિક પદાર્થ (Biofuels) \\ \hline
\textbf{ભૂતાપીય (Geothermal)} & પૃથ્વીની ગરમી \\ \hline
\end{tabulary}
\end{center}

\begin{answerdiagram}{Green Energy Classification}
\begin{tikzpicture}[
  level 1/.style={sibling distance=3cm},
  level 2/.style={sibling distance=1.5cm}
]
\node [gtu root] {Green Energy}
    child { node [gtu child] {Solar} }
    child { node [gtu child] {Wind} }
    child { node [gtu child] {Hydro} }
    child { node [gtu child] {Biomass} }
    child { node [gtu child] {Geothermal} };
\end{tikzpicture}
\end{answerdiagram}
\end{solutionbox}

\begin{mnemonicbox}
\mnemonic{SWHBG: સૂર્ય, વાયુ, હાઇડ્રો, બાયોમાસ, ભૂતાપીય ઊર્જા સ્ત્રોત}
\end{mnemonicbox}

\questionmarks{4(c)}{7}{સોલર પાવર સિસ્ટમનો બ્લોક ડાયગ્રામ દોરો અને સમજાવો.}

\begin{solutionbox}
\textbf{જવાબ}:

\keyword{ઘટકો:}
\begin{itemize}
    \item \textbf{સોલર પેનલ:} સૂર્યપ્રકાશને DC વીજળીમાં રૂપાંતરિત કરે છે
    \item \textbf{ચાર્જ કંટ્રોલર:} બેટરી ચાર્જિંગનું નિયમન કરે છે
    \item \textbf{બેટરી બેંક:} વીજળી સંગ્રહિત કરે છે
    \item \textbf{ઇન્વર્ટર:} DC ને AC માં રૂપાંતરિત કરે છે
    \item \textbf{ડિસ્ટ્રિબ્યુશન પેનલ:} વીજળી વિતરિત કરે છે
\end{itemize}

\begin{answerdiagram}{Solar Power System Block Diagram}
\begin{tikzpicture}[node distance=1.5cm, auto]
    \node [gtu block] (S) {Solar Panel};
    \node [gtu block, right=1cm of S] (C) {Charge\\Controller};
    \node [gtu block, below=1cm of C] (B) {Battery};
    \node [gtu block, right=1cm of C] (I) {Inverter};
    \node [gtu block, right=1cm of I] (L) {AC Load};
    
    \path [gtu arrow] (S) -- (C);
    \path [gtu arrow] (C) -- (I);
    \path [gtu arrow] (I) -- (L);
    \path [gtu arrow] (C) edge[bend left] (B);
    \path [gtu arrow] (B) edge[bend left] (C);
\end{tikzpicture}
\end{answerdiagram}
\end{solutionbox}

\begin{mnemonicbox}
\mnemonic{SCBIL: સોલર પેનલ, ચાર્જ કંટ્રોલર, બેટરીઝ, ઇન્વર્ટર, લોડ}
\end{mnemonicbox}

\questionmarks{4(a) OR}{3}{ગ્રીન એનર્જી, કન્વેન્શનલ એનર્જી અને રિન્યુએબલ એનર્જીની વ્યાખ્યા લખો.}

\begin{solutionbox}
\textbf{જવાબ}:

\begin{center}
\captionof{table}{ઊર્જા વ્યાખ્યાઓ}
\begin{tabulary}{\linewidth}{|L|L|}
\hline
\textbf{શબ્દ} & \textbf{વ્યાખ્યા} \\ \hline
\textbf{ગ્રીન એનર્જી} & કુદરતી રીતે પુનઃપ્રાપ્ત થતા સ્ત્રોતો, પર્યાવરણ પર ન્યૂનતમ પ્રભાવ \\ \hline
\textbf{કન્વેન્શનલ એનર્જી} & પરંપરાગત ફોસિલ ફ્યુઅલ સ્ત્રોતો (કોલસો, તેલ) \\ \hline
\textbf{રિન્યુએબલ એનર્જી} & એવા સ્ત્રોતો જે કુદરતી રીતે પુનઃપૂર્તિ થાય છે \\ \hline
\end{tabulary}
\end{center}
\end{solutionbox}

\begin{mnemonicbox}
\mnemonic{GCR: ગ્રીન સ્વચ્છ છે, કન્વેન્શનલ કાર્બન છોડે છે, રિન્યુએબલ પુનઃપૂર્ણ થાય છે}
\end{mnemonicbox}

\questionmarks{4(b) OR}{4}{ગ્રીન એનર્જીની ઉપયોગિતા સમજાવો.}

\begin{solutionbox}
\textbf{જવાબ}:

\keyword{ગ્રીન એનર્જીની આવશ્યકતા:}
\begin{itemize}
    \item \textbf{પર્યાવરણ સંરક્ષણ:} પ્રદૂષણ ઘટાડે છે
    \item \textbf{સંસાધન સંરક્ષણ:} ફોસિલ ફ્યુઅલ બચાવે છે
    \item \textbf{ઊર્જા સુરક્ષા:} આયાત ઘટાડે છે
    \item \textbf{આર્થિક લાભ:} નોકરીઓ બનાવે છે
    \item \textbf{ટકાઉ વિકાસ:} ભવિષ્યની પેઢીઓ માટે
\end{itemize}
\end{solutionbox}

\begin{mnemonicbox}
\mnemonic{ERESS: પર્યાવરણ, સંસાધનો, ઊર્જા સુરક્ષા, બચત, ટકાઉપણું}
\end{mnemonicbox}

\questionmarks{4(c) OR}{7}{વિન્ડ પાવર સિસ્ટમનો બ્લોક ડાયાગ્રામ ટર્બાઈનના પ્રકાર સહિત દોરો અને સમજાવો.}

\begin{solutionbox}
\textbf{જવાબ}:

\keyword{ઘટકો:}
\begin{itemize}
    \item \textbf{વિન્ડ ટર્બાઈન:} પવન ઊર્જાને યાંત્રિક ઊર્જામાં રૂપાંતરિત કરે છે
    \item \textbf{ગિયરબોક્સ:} ફરવાની ગતિ વધારે છે
    \item \textbf{જનરેટર:} વિદ્યુત ઊર્જા ઉત્પન્ન કરે છે
    \item \textbf{કંટ્રોલર:} સિસ્ટમનું નિયંત્રણ કરે છે
    \item \textbf{ટ્રાન્સફોર્મર:} વોલ્ટેજ વધારે છે
\end{itemize}

\keyword{પ્રકાર:} HAWT (હોરિઝોન્ટલ એક્સિસ) અને VAWT (વર્ટિકલ એક્સિસ).

\begin{answerdiagram}{Wind Power System}
\begin{tikzpicture}[node distance=1.2cm, auto]
    \node [gtu block] (W) {Wind};
    \node [gtu block, right=0.5cm of W] (T) {Turbine};
    \node [gtu block, right=0.5cm of T] (G) {Gearbox};
    \node [gtu block, right=0.5cm of G] (Gen) {Generator};
    \node [gtu block, right=0.5cm of Gen] (C) {Controller};
    \node [gtu block, below=0.8cm of C] (Tr) {Transformer};
    \node [gtu block, left=0.5cm of Tr] (Grid) {Grid};
    
    \path [gtu arrow] (W) -- (T);
    \path [gtu arrow] (T) -- (G);
    \path [gtu arrow] (G) -- (Gen);
    \path [gtu arrow] (Gen) -- (C);
    \path [gtu arrow] (C) -- (Tr);
    \path [gtu arrow] (Tr) -- (Grid);
\end{tikzpicture}
\end{answerdiagram}
\end{solutionbox}

\begin{mnemonicbox}
\mnemonic{WGGTC: વિન્ડ ટર્બાઈન, ગિયરબોક્સ, જનરેટર, ટ્રાન્સફોર્મર, કંટ્રોલર}
\end{mnemonicbox}

\questionmarks{5(a)}{3}{અવરોધના રેઝિસ્ટન્સને અસર કરતાં પરિબળો સમજાવો.}

\begin{solutionbox}
\textbf{જવાબ}:

\keyword{રેઝિસ્ટન્સને અસર કરતા પરિબળો ($R = \rho l/A$):}
\begin{itemize}
    \item \textbf{લંબાઈ ($l$):} સીધા પ્રમાણમાં ($R \propto l$)
    \item \textbf{ક્ષેત્રફળ ($A$):} વ્યસ્ત પ્રમાણમાં ($R \propto 1/A$)
    \item \textbf{તાપમાન:} ધાતુઓમાં વધવાથી રેઝિસ્ટન્સ વધે છે
    \item \textbf{મટીરિયલ ($\rho$):} અવરોધકતા પર આધાર રાખે છે
\end{itemize}
\end{solutionbox}

\begin{mnemonicbox}
\mnemonic{TLAM: તાપમાન, લંબાઈ, ક્ષેત્રફળ, મટીરિયલ}
\end{mnemonicbox}

\questionmarks{5(b)}{4}{પાવર ત્રિકોણની મદદથી એક્ટિવ પાવર, રીએક્ટિવ પાવર, અપેરેન્ટ પાવર અને પાવર ફેક્ટરની વ્યાખ્યા લખો. તથા તેઓના એકમ લખો.}

\begin{solutionbox}
\textbf{જવાબ}:

\begin{center}
\captionof{table}{પાવર વ્યાખ્યાઓ}
\begin{tabulary}{\linewidth}{|L|L|L|}
\hline
\textbf{શબ્દ} & \textbf{સૂત્ર} & \textbf{એકમ} \\ \hline
\textbf{એક્ટિવ પાવર (P)} & $P = VI \cos \phi$ & Watt (W) \\ \hline
\textbf{રીએક્ટિવ પાવર (Q)} & $Q = VI \sin \phi$ & VAR \\ \hline
\textbf{અપેરેન્ટ પાવર (S)} & $S = VI$ & VA \\ \hline
\textbf{પાવર ફેક્ટર} & $\cos \phi = P/S$ & - \\ \hline
\end{tabulary}
\end{center}

\begin{answerdiagram}{Power Triangle}
\begin{tikzpicture}
    \draw[thick] (0,0) -- (4,0) node[midway, below] {$P$ (Watts)};
    \draw[thick] (4,0) -- (4,3) node[midway, right] {$Q$ (VAR)};
    \draw[thick] (0,0) -- (4,3) node[midway, above left] {$S$ (VA)};
    \draw (0.6,0) arc (0:36.87:0.6) node[midway, right] {$\phi$};
\end{tikzpicture}
\end{answerdiagram}
\end{solutionbox}

\begin{mnemonicbox}
\mnemonic{ARSP: એક્ટિવ, રીએક્ટિવ, અપેરેન્ટ, પાવર ફેક્ટર}
\end{mnemonicbox}

\questionmarks{5(c)}{7}{કિર્ચોફનો વૉલ્ટેજનો નિયમ અને કિર્ચોફનો કરંટનો નિયમ લખો અને સર્કિટ ડાયાગ્રામની મદદથી સમજાવો.}

\begin{solutionbox}
\textbf{જવાબ}:

\keyword{KVL:} બંધ લૂપમાં વૉલ્ટેજનો બેજિક સરવાળો શૂન્ય હોય છે ($\sum V = 0$).

\keyword{KCL:} જંક્શન પર કરંટનો બેજિક સરવાળો શૂન્ય હોય છે ($\sum I = 0$).

\begin{answerdiagram}{KVL and KCL}
\begin{circuitikz}[american, scale=0.8]
    % KVL
    \draw (0,0) to[V, l=$V_s$] (0,2) to[R, l=$R_1$] (2,2) to[R, l=$R_2$] (2,0) -- (0,0);
    \node at (1,-0.5) {KVL: $V_s - IR_1 - IR_2 = 0$};
    
    % KCL
    \begin{scope}[xshift=4cm]
    \node[circle, fill, inner sep=1.5pt] (N) at (0,1) {};
    \draw[<-] (N) -- (-1,1) node[left] {$I_1$};
    \draw[->] (N) -- (1,2) node[right] {$I_2$};
    \draw[->] (N) -- (1,0) node[right] {$I_3$};
    \node at (0,-0.5) {KCL: $I_1 = I_2 + I_3$};
    \end{scope}
\end{circuitikz}
\end{answerdiagram}
\end{solutionbox}

\begin{mnemonicbox}
\mnemonic{VCL: વૉલ્ટેજ ક્લોઝ્ડ લૂપ, કરંટ નોડ સરવાળો}
\end{mnemonicbox}

\questionmarks{5(a) OR}{3}{ઈએમએફ અને પોટેન્શિયલ ડિફરન્સ વચ્ચેનો તફાવત લખો તથા સેલ અને બેટરી વચ્ચેનો તફાવત લખો.}

\begin{solutionbox}
\textbf{જવાબ}:

\begin{center}
\captionof{table}{તફાવત}
\begin{tabulary}{\linewidth}{|L|L|}
\hline
\textbf{EMF} & \textbf{પોટેન્શિયલ ડિફરન્સ} \\ \hline
એકમ ચાર્જ દીઠ પૂરી પાડતી ઊર્જા & વપરાતી ઊર્જા \\ \hline
ખુલ્લી સર્કિટમાં હોય છે & બંધ સર્કિટમાં હોય છે \\ \hline
કરંટનું કારણ છે & કરંટનું પરિણામ છે \\ \hline
\end{tabulary}
\end{center}

\textbf{સેલ vs બેટરી:} સેલ એકલ એકમ છે; બેટરી સેલનો સમૂહ છે.
\end{solutionbox}

\begin{mnemonicbox}
\mnemonic{ESOP: EMF સ્ત્રોત, પોટેન્શિયલ ઓપરેટિંગ}
\end{mnemonicbox}

\questionmarks{5(b) OR}{4}{શુદ્ધ અવરોધ, શુદ્ધ કેપેસિટર અને શુદ્ધ ઇન્ડક્ટર માટે AC વૉલ્ટેજ અને AC કરંટ વચ્ચેનો સંબંધ લખો. શુદ્ધ અવરોધ, શુદ્ધ કેપેસિટર અને શુદ્ધ ઇન્ડક્ટર માટે AC વૉલ્ટેજ અને AC કરંટનો વેક્ટર ડાયાગ્રામ દોરો. તથા શુદ્ધ અવરોધ, શુદ્ધ કેપેસિટર અને શુદ્ધ ઇન્ડક્ટર માટે પાવર ફેક્ટરની વેલ્યૂ લખો.}

\begin{solutionbox}
\textbf{જવાબ}:

\begin{center}
\captionof{table}{સરખામણી}
\begin{tabulary}{\linewidth}{|L|L|L|L|}
\hline
\textbf{ઘટક} & \textbf{સંબંધ} & \textbf{ફેઝ} & \textbf{PF} \\ \hline
\textbf{રેઝિસ્ટર} & $V=IR$ & એકસરખા (0$^\circ$) & 1 \\ \hline
\textbf{ઇન્ડક્ટર} & $V=L(dI/dt)$ & $I$ પાછળ 90$^\circ$ & 0 (lag) \\ \hline
\textbf{કેપેસિટર} & $I=C(dV/dt)$ & $I$ આગળ 90$^\circ$ & 0 (lead) \\ \hline
\end{tabulary}
\end{center}

\begin{answerdiagram}{Vector Diagrams}
\begin{tikzpicture}[scale=0.8]
    % Resistor
    \draw[->, thick, blue] (0,0) -- (2,0) node[right] {$V$};
    \draw[->, thick, red, dashed] (0,-0.2) -- (1.5,-0.2) node[right] {$I$};
    \node at (1,-1) {Resistor};
    
    % Inductor
    \begin{scope}[xshift=3cm]
    \draw[->, thick, blue] (0,0) -- (0,2) node[above] {$V$};
    \draw[->, thick, red, dashed] (0,0) -- (2,0) node[right] {$I$};
    \node at (1,-1) {Inductor};
    \end{scope}

    % Capacitor
    \begin{scope}[xshift=6cm]
    \draw[->, thick, blue] (0,0) -- (2,0) node[right] {$V$};
    \draw[->, thick, red, dashed] (0,0) -- (0,2) node[above] {$I$};
    \node at (1,-1) {Capacitor};
    \end{scope}
\end{tikzpicture}
\end{answerdiagram}
\end{solutionbox}

\begin{mnemonicbox}
\mnemonic{RCI: રેઝિસ્ટર સાથે, ઇન્ડક્ટર પાછળ, કેપેસિટર આગળ}
\end{mnemonicbox}

\questionmarks{5(c) OR}{7}{મટિરિયલ માટે ટેમ્પરેચર કોએફિસિયન્ટની વ્યાખ્યા લખો અને તેનો એકમ લખો. વાહક ઉપર તાપમાનની અસર ટેમ્પરેચર કોએફિસિયન્ટની મદદથી સમજાવો.}

\begin{solutionbox}
\textbf{જવાબ}:

\keyword{ટેમ્પરેચર કોએફિસિયન્ટ ($\alpha$):} તાપમાનમાં એક ડિગ્રી પરિવર્તન દીઠ રેઝિસ્ટન્સમાં થતો આંશિક ફેરફાર.
\keyword{એકમ:} પ્રતિ ડિગ્રી સેલ્સિયસ ($^\circ$C$^{-1}$).

\keyword{વાહકો પર અસર:}
\begin{itemize}
    \item તાપમાન વધવાથી રેઝિસ્ટન્સ વધે છે (ધન $\alpha$)
    \item $R_2 = R_1 [1 + \alpha (T_2 - T_1)]$
\end{itemize}

\keyword{અર્ધવાહકો પર અસર:} રેઝિસ્ટન્સ ઘટે છે (ઋણ $\alpha$).

\begin{answerdiagram}{Temperature Effect}
\begin{tikzpicture}
    \begin{axis}[
        width=6cm, height=4cm,
        axis lines=left,
        xlabel=Temperature ($T$),
        ylabel=Resistance ($R$),
        xtick=\empty, ytick=\empty
    ]
    \addplot[blue, thick, domain=0:10] {1 + 0.1*x} node[right] {Conductors (+$\alpha$)};
    \addplot[red, dashed, thick, domain=0:10] {2 - 0.1*x} node[right] {Semiconductors (-$\alpha$)};
    \end{axis}
\end{tikzpicture}
\end{answerdiagram}
\end{solutionbox}

\begin{mnemonicbox}
\mnemonic{TRIP: તાપમાન રેઝિસ્ટન્સ વધારે છે (વાહકો માટે)}
\end{mnemonicbox}

\end{document}
