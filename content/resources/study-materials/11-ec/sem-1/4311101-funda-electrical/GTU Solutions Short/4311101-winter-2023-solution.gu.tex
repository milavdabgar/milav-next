\documentclass{article}

% content/resources/templates/preamble.tex
\usepackage[margin=0.6in]{geometry}
\author{Milav Dabgar}
\usepackage{amsmath,amssymb,amsthm}
\usepackage{booktabs}
\usepackage{multirow}
\usepackage{xcolor}
\usepackage{tcolorbox}
\tcbuselibrary{breakable,skins}
\usepackage[colorlinks=true,linkcolor=blue]{hyperref}
\usepackage{titlesec}
\usepackage{enumitem}
\usepackage{tikz}
\usepackage{pgfplots}
\usepackage{circuitikz}
\usepackage[version=4]{mhchem}
\usepackage{longtable}
\usepackage{array}
\usepackage{float}
\usepackage{caption}
\usepackage{listings}

\lstset{
  basicstyle=\small\ttfamily,
  breaklines=true,
  breakatwhitespace=false,
  postbreak=\mbox{\textcolor{red}{$\hookrightarrow$}\space},
  float=false,
  numbers=left,
  numberstyle=\tiny\color{gray},
  numbersep=10pt,
  xleftmargin=2em,
  keywordstyle=\color{blue},
  commentstyle=\color{green!60!black},
  stringstyle=\color{purple},
  backgroundcolor=\color{gray!5},
  showstringspaces=false,
  tabsize=2,
  captionpos=b,
  keepspaces=true,
  columns=flexible
}

\pgfplotsset{compat=1.18}
\usetikzlibrary{shapes,arrows,positioning,calc,patterns,decorations.pathmorphing,decorations.markings,arrows.meta}

% Color scheme
\definecolor{headcolor}{RGB}{0,102,204}
\definecolor{keycolor}{RGB}{220,20,60}
\definecolor{solutioncolor}{RGB}{34,139,34}
\definecolor{mnemoniccolor}{RGB}{148,0,211}
\definecolor{codecolor}{RGB}{0,0,100}

% Spacing
\setlength{\parskip}{3pt}
\setlist[itemize]{nosep}
\setlist[enumerate]{nosep}

% Title formatting
\titleformat{\section}{\Large\bfseries\color{headcolor}}{\thesection}{1em}{}
\titleformat{\subsection}{\large\bfseries\color{headcolor}}{\thesubsection}{1em}{}

% Pandoc tightlist compatibility
\providecommand{\tightlist}{%
  \setlength{\itemsep}{0pt}\setlength{\parskip}{0pt}}

% Pandoc longtable compatibility
\newcounter{none}
\def\thenone{}


% content/resources/templates/gujarati-boxes.tex
\usepackage{fontspec}
\usepackage{polyglossia}

% Set Gujarati as main language (document is primarily in Gujarati)
% Note: gloss-gujarati.ldf doesn't exist in polyglossia, but it will use hyphenation patterns
\setdefaultlanguage{gujarati}
\setotherlanguage{english}

% Configure Gujarati font properly
% Use Language=Default to prevent polyglossia from trying to add language-specific features
% that don't exist for Gujarati, which causes "empty feature" warnings
\newfontfamily\gujaratifont[Script=Gujarati,AutoFakeBold=2.5,AutoFakeSlant=0.3]{Noto Sans Gujarati}
\setmainfont[Script=Gujarati,AutoFakeBold=2.5,AutoFakeSlant=0.3]{Noto Sans Gujarati}
% Use Noto Sans Gujarati for monospace to support Gujarati in text
\setmonofont[Scale=0.9]{Noto Sans Gujarati}

% Configure English to use the same font
\newfontfamily\englishfont[Script=Gujarati,AutoFakeBold=2.5,AutoFakeSlant=0.3]{Noto Sans Gujarati}

% Translations for polyglossia
\gappto\captionsgujarati{
  \renewcommand{\tablename}{કોષ્ટક}
  \renewcommand{\figurename}{આકૃતિ}
}

% Helper for TikZ nodes to ensure Gujarati font
\newcommand{\gu}[1]{{\gujaratifont #1}}

% Custom environments
\newtcolorbox{solutionbox}{
    breakable,
    enhanced,
    colback=solutioncolor!5!white,
    colframe=solutioncolor!75!black,
    fonttitle=\bfseries,
    title=જવાબ
}

\newtcolorbox{solutionboxnobreak}{
 colback=solutioncolor!5!white,
 colframe=solutioncolor!75!black,
 fonttitle=\bfseries,
 title=જવાબ
}

\newtcolorbox{keyformula}{
 breakable,
 enhanced,
 colback=keycolor!5!white,
 colframe=keycolor!75!black,
 fonttitle=\bfseries,
 title=રાસાયણિક સમીકરણ/સૂત્ર
}

\newtcolorbox{mnemonicbox}{
 breakable,
 enhanced,
 colback=mnemoniccolor!5!white,
 colframe=mnemoniccolor!75!black,
 fonttitle=\bfseries,
 title=મેમરી ટ્રીક
}


% Custom commands for GTU solutions
% This file defines semantic commands for consistent formatting

% Question command with automatic formatting
\newcommand{\question}[2]{%
  \section*{Question #1}%
  \textbf{#2}%
}

% OR question variant
\newcommand{\questionor}[2]{%
  \section*{Question #1 OR}%
  \textbf{#2}%
}

% Proper table environment with caption
\newenvironment{answertable}[1]{%
  \begin{table}[htbp]
  \centering
  \caption{#1}
}{%
  \end{table}
}

% Proper figure environment for diagrams
\newenvironment{answerdiagram}[1]{%
  \begin{figure}[htbp]
  \centering
  \caption{#1}
}{%
  \end{figure}
}

% Semantic markup for key terms
\newcommand{\keyword}[1]{\textbf{#1}}
\newcommand{\code}[1]{\texttt{#1}}
\newcommand{\classname}[1]{\texttt{#1}}
\newcommand{\methodname}[1]{\texttt{#1}}

% Proper quotation marks
\newcommand{\mnemonic}[1]{``#1''}


\title{ફંડામેન્ટલ્સ ઓફ ઇલેક્ટ્રિકલ એન્જિનિયરિંગ (4311101) - વિન્ટર 2023 સોલ્યુશન}
\date{January 19, 2023}

\begin{document}
\maketitle

\questionmarks{1(a)}{3}{પાવર અને એનર્જી વ્યાખ્યાયિત કરો.}

\begin{solutionbox}
\textbf{જવાબ}:

\begin{itemize}
    \item \keyword{પાવર (Power)}: કાર્ય કરવાનો દર અથવા એકમ સમય દીઠ ઊર્જાનો વપરાશ. વોટ્સ (W)માં માપવામાં આવે છે.
    \item \keyword{એનર્જી (Energy)}: કાર્ય કરવાની ક્ષમતા અથવા કરેલ કાર્ય. જૂલ (J) અથવા વોટ-કલાક (Wh)માં માપવામાં આવે છે.
\end{itemize}

\begin{center}
\captionof{table}{પાવર vs એનર્જી}
\begin{tabulary}{\linewidth}{|L|L|L|L|}
\hline
\textbf{પેરામીટર} & \textbf{વ્યાખ્યા} & \textbf{ફોર્મ્યુલા} & \textbf{એકમ} \\ \hline
\textbf{પાવર} & ઊર્જા ટ્રાન્સફરનો દર & $P = W/t$ & Watt (W) \\ \hline
\textbf{એનર્જી} & કાર્ય કરવાની ક્ષમતા & $E = P \times t$ & Joule (J) or Watt-hour (Wh) \\ \hline
\end{tabulary}
\end{center}
\end{solutionbox}

\begin{mnemonicbox}
\mnemonic{Power Performs, Energy Endures}
\end{mnemonicbox}

\questionmarks{1(b)}{4}{વિદ્યુત્પ્રવાહ અને વિદ્યુત પોટેંશિયલ વ્યાખ્યાયિત કરો.}

\begin{solutionbox}
\textbf{જવાબ}:

\begin{itemize}
    \item \keyword{વિદ્યુત્પ્રવાહ (Current)}: એકમ સમય દીઠ વહેતો વિદ્યુત ચાર્જ. એમ્પિયર (A)માં માપવામાં આવે છે.
    \item \keyword{વિદ્યુત પોટેંશિયલ (Electrical Potential)}: એક બિંદુથી બીજા બિંદુ પર ચાર્જ ખસેડવા માટે એકમ ચાર્જ દીઠ કરવામાં આવતું કાર્ય. વોલ્ટ (V)માં માપવામાં આવે છે.
\end{itemize}

\begin{answerdiagram}{Current and Potential}
\begin{tikzpicture}[node distance=2cm, auto]
    \node [gtu block] (A) {Electron Flow};
    \node [gtu block, right=of A] (B) {Current};
    \node [gtu block, below=of A] (C) {Potential Energy};
    \node [gtu block, right=of C] (D) {Voltage};

    \draw [gtu arrow] (A) -- node {Rate of Flow} (B);
    \draw [gtu arrow] (C) -- node {Per Unit Charge} (D);
\end{tikzpicture}
\end{answerdiagram}
\end{solutionbox}

\begin{mnemonicbox}
\mnemonic{Current Charges, Potential Pushes}
\end{mnemonicbox}

\questionmarks{1(c)}{7}{ઉદાહરણો સાથે કેસીએલ અને કેવીએલ સમજાવો.}

\begin{solutionbox}
\textbf{જવાબ}:

\keyword{કિરચોફનો કરંટ નિયમ (KCL):}
\begin{itemize}
    \item નોડમાં પ્રવેશતા કરંટનો સરવાળો તેમાંથી બહાર નીકળતા કરંટના સરવાળા સમાન હોય છે.
    \item ઉદાહરણ: નોડ X પર, $i_1 + i_2 = i_3$
\end{itemize}

\keyword{કિરચોફનો વોલ્ટેજ નિયમ (KVL):}
\begin{itemize}
    \item કોઈપણ બંધ લૂપમાં વોલ્ટેજ ડ્રોપ્સનો સરવાળો શૂન્ય છે.
    \item ઉદાહરણ: $V_1 - V(R_1) - V(R_2) = 0$
\end{itemize}

\begin{answerdiagram}{KCL Circuit Example}
\begin{circuitikz}[american, scale=0.8]
    % Simple KCL/KVL illustration - reuse
    \draw (0,0) to[V, l=$V_1$] (0,3) to[R, l=$R_1$, i=$i_1$] (3,3)
          to[R, l=$R_2$, i=$i_3$] (3,0) -- (0,0);
    \draw (3,3) -- (5,3) to[R, l=$R_3$, i=$i_2$] (5,0) -- (3,0);
    \node at (3,3.3) {Node X};
\end{circuitikz}
\end{answerdiagram}
\end{solutionbox}

\begin{mnemonicbox}
\mnemonic{Currents Come-Leave, Voltages Voyage-Loop}
\end{mnemonicbox}

\questionmarks{1(c) OR}{7}{રેસિસ્ટર્સ માટે વિવિધ પ્રકારનાં જોડાણો સમજાવો.}

\begin{solutionbox}
\textbf{જવાબ}:

\begin{center}
\captionof{table}{શ્રેણી vs સમાંતર જોડાણ}
\begin{tabulary}{\linewidth}{|L|L|L|}
\hline
\textbf{પેરામીટર} & \textbf{શ્રેણી જોડાણ} & \textbf{સમાંતર જોડાણ} \\ \hline
\textbf{કુલ અવરોધ} & $R_{eq} = R_1 + R_2 + R_3 + \dots$ & $1/R_{eq} = 1/R_1 + 1/R_2 + 1/R_3 + \dots$ \\ \hline
\textbf{કરંટ} & બધા અવરોધો માટે સમાન & દરેક માર્ગમાં વહેંચાય છે \\ \hline
\textbf{વોલ્ટેજ} & અવરોધો વચ્ચે વહેંચાય છે & બધા અવરોધો માટે સમાન \\ \hline
\textbf{ઉપયોગ} & વોલ્ટેજ ડિવાઇડર & કરંટ વહેંચણી \\ \hline
\end{tabulary}
\end{center}

\begin{answerdiagram}{Resistor Connections}
\begin{center}
\begin{circuitikz}[american, scale=0.8, transform shape]
    % Series
    \draw (0,0) node[left]{A} to[R, l=$R_1$] (1.5,0) to[R, l=$R_2$] (3,0) to[R, l=$R_3$] (4.5,0) node[right]{B};
    \node at (2.25,-0.8) {Series Connection};

    % Parallel
    \begin{scope}[yshift=-2.5cm]
    \draw (0,1) -- (0,0) -- (4.5,0) -- (4.5,1);
    \draw (0,1) -- (4.5,1);
    \draw (1,1) to[R, l=$R_1$] (1,0);
    \draw (2.25,1) to[R, l=$R_2$] (2.25,0);
    \draw (3.5,1) to[R, l=$R_3$] (3.5,0);
    \node at (0,0.5) [left] {A};
    \node at (4.5,0.5) [right] {B};
    \node at (2.25,-0.8) {Parallel Connection};
    \end{scope}
\end{circuitikz}
\end{center}
\end{answerdiagram}
\end{solutionbox}

\begin{mnemonicbox}
\mnemonic{Series Sum, Parallel Parts}
\end{mnemonicbox}

\questionmarks{2(a)}{3}{અવરોધ અને અવરોધકતાને વ્યાખ્યાયિત કરો. તેમના એકમો પણ જણાવો.}

\begin{solutionbox}
\textbf{જવાબ}:

\begin{itemize}
    \item \keyword{અવરોધ (Resistance)}: કરંટ પ્રવાહમાં અડચણ, ઓહ્મ ($\Omega$)માં માપવામાં આવે છે.
    \[ R = \frac{V}{I} \]
    \item \keyword{અવરોધકતા (Resistivity)}: પદાર્થની એક ગુણધર્મ જે એકમ દિમેન્શન દીઠ અવરોધ દર્શાવે છે, ઓહ્મ-મીટર ($\Omega\cdot m$)માં માપવામાં આવે છે.
    \[ \rho = \frac{R \cdot A}{L} \]
\end{itemize}
\end{solutionbox}

\begin{mnemonicbox}
\mnemonic{Resistance Restricts, Resistivity Relates to material}
\end{mnemonicbox}

\questionmarks{2(b)}{4}{વિદ્યુત કોષને વ્યાખ્યાયિત કરો અને વિવિધ પ્રકારના વિદ્યુત કોષના નામ લખો.}

\begin{solutionbox}
\textbf{જવાબ}:

\keyword{વિદ્યુત કોષ (Cell)}: એક ઉપકરણ જે રાસાયણિક ઊર્જાને વિદ્યુત ઊર્જામાં રૂપાંતરિત કરીને વોલ્ટેજ ઉત્પન્ન કરે છે.

\textbf{વિદ્યુત કોષના પ્રકારો:}
\begin{enumerate}
    \item \keyword{પ્રાથમિક કોષ (Primary)}: ડ્રાય સેલ, આલ્કલાઇન સેલ, મર્ક્યુરી સેલ
    \item \keyword{દ્વિતીય કોષ (Secondary)}: લેડ-એસિડ, નિકલ-કેડમિયમ, લિથિયમ-આયન
\end{enumerate}

\begin{answerdiagram}{Analysis of a Battery Cell}
\begin{circuitikz}[american, scale=1.0]
    \draw (0,0) to[battery1, l=Battery] (2,0);
    \node at (1, -0.5) {Symbol for Cell/Battery};
\end{circuitikz}
\end{answerdiagram}
\end{solutionbox}

\begin{mnemonicbox}
\mnemonic{Primary Produces once, Secondary Serves repeatedly}
\end{mnemonicbox}

\questionmarks{2(c)}{7}{ઉપરોક્ત સર્કિટના કુલ સમકક્ષ અવરોધની ગણતરી કરો જેમા R1=5$\Omega$, R2=3$\Omega$, R3=4$\Omega$, R4=1$\Omega$, R5=2$\Omega$ લો.}

\begin{solutionbox}
\textbf{જવાબ}:

\textbf{પગલાવાર ઉકેલ:}
\begin{enumerate}
    \item $R_2$ અને $R_3$ શ્રેણીમાં છે:
    \[ R_{23} = R_2 + R_3 = 3\Omega + 4\Omega = 7\Omega \]
    \item $R_{23}$ અને $R_4$ સમાંતરમાં છે:
    \[ \frac{1}{R_{234}} = \frac{1}{R_{23}} + \frac{1}{R_4} = \frac{1}{7} + \frac{1}{1} = \frac{8}{7} \]
    \[ R_{234} = \frac{7}{8} = 0.875\Omega \]
    \item $R_1$, $R_{234}$, અને $R_5$ શ્રેણીમાં છે:
    \[ R_{eq} = R_1 + R_{234} + R_5 = 5\Omega + 0.875\Omega + 2\Omega = 7.875\Omega \]
\end{enumerate}

\textbf{આથી, સમકક્ષ અવરોધ = 7.875$\Omega$}

\begin{answerdiagram}{Circuit Diagram}
\begin{circuitikz}[american, scale=0.8]
    \draw (0,0) to[R, l=$R_1$] (2,0) -- (2,1) to[R, l=$R_2$] (4,1) to[R, l=$R_3$] (6,1) -- (6,0) to[R, l=$R_5$] (8,0);
    \draw (2,0) -- (2,-1) to[R, l=$R_4$] (6,-1) -- (6,0);
\end{circuitikz}
\end{answerdiagram}
\end{solutionbox}

\begin{mnemonicbox}
\mnemonic{Series-Sum, Parallel-Product over Sum}
\end{mnemonicbox}

\questionmarks{2(a) OR}{3}{જો 100 વોટનો બલ્બ 30 દિવસ માટે દરરોજ 10 કલાક ચલાવે તો એનર્જીની કિંમત શોધો. એનર્જી નો દર રૂપિયા 5/એકમ છે.}

\begin{solutionbox}
\textbf{જવાબ}:

\begin{center}
\captionof{table}{એનર્જી ગણતરી}
\begin{tabulary}{\linewidth}{|L|L|L|}
\hline
\textbf{પેરામીટર} & \textbf{મૂલ્ય} & \textbf{ગણતરી} \\ \hline
\textbf{પાવર} & $100\text{W} = 0.1\text{kW}$ & આપેલ છે \\ \hline
\textbf{ઓપરેટિંગ કલાકો} & $10 \text{ કલાક/દિવસ} \times 30 \text{ દિવસ} = 300 \text{ કલાક}$ & આપેલ છે \\ \hline
\textbf{વપરાયેલ એનર્જી} & $0.1\text{kW} \times 300\text{h} = 30\text{kWh} = 30 \text{ એકમ}$ & $E = P \times t$ \\ \hline
\textbf{દર} & રૂ. 5/એકમ & આપેલ છે \\ \hline
\textbf{કુલ કિંમત} & $30 \text{ એકમ} \times 5 \text{ રૂ./એકમ} = \text{રૂ. } 150$ & કિંમત = એકમો $\times$ દર \\ \hline
\end{tabulary}
\end{center}

\textbf{આથી, એનર્જીની કિંમત = રૂ. 150}
\end{solutionbox}

\begin{mnemonicbox}
\mnemonic{Energy x Rate = Electric bill fate}
\end{mnemonicbox}

\questionmarks{2(b) OR}{4}{ઓહમનો નિયમ લખો અને કોઈપણ સર્કિટમાં કરંટની ગણતરી કરવા માટે ઓહ્મના નિયમ નો ઉપયોગ સમજાવો.}

\begin{solutionbox}
\textbf{જવાબ}:

\keyword{ઓહમનો નિયમ (Ohm's Law):} વાહકમાંથી વહેતો કરંટ વોલ્ટેજના સીધા પ્રમાણમાં અને અવરોધના વ્યસ્ત પ્રમાણમાં હોય છે.

\keyword{ફોર્મ્યુલા:}
\[ V = I \times R \quad \text{અથવા} \quad I = \frac{V}{R} \quad \text{અથવા} \quad R = \frac{V}{I} \]

\keyword{ઉપયોગ:} સર્કિટમાં કરંટ શોધવા માટે, ઘટક પરના વોલ્ટેજને તેના અવરોધ વડે ભાગો ($I = V/R$).

\begin{answerdiagram}{Ohm's Law Triangle}
\begin{tikzpicture}
    \draw (0,0) -- (4,0) -- (2,3.46) -- cycle;
    \draw (1,1.73) -- (3,1.73);
    \draw (2,1.73) -- (2,0);
    \node at (2,2.5) {\Large V};
    \node at (1,0.8) {\Large I};
    \node at (3,0.8) {\Large R};
\end{tikzpicture}
\end{answerdiagram}
\end{solutionbox}

\begin{mnemonicbox}
\mnemonic{Volts Invite current, Resistance Restricts}
\end{mnemonicbox}

\questionmarks{2(c) OR}{7}{સાબિત કરો કે સંપૂર્ણ કેપેસિટીવ સર્કિટમાં કરંટ વોલ્ટેજ થી 90$^{\circ}$ આગળ હોઇ છે, અને સંપૂર્ણ રીતે ઇંડક્ટીવ સર્કિટમાં કરંટ વોલ્ટેજ થી 90$^{\circ}$ પાછળ હોઇ છે.}

\begin{solutionbox}
\textbf{જવાબ}:

\textbf{કેપેસિટીવ સર્કિટ માટે:}
\begin{itemize}
    \item વોલ્ટેજ સમીકરણ: $v = V_m \sin(\omega t)$
    \item કરંટ: $i = C \frac{dv}{dt} = \omega C V_m \cos(\omega t) = I_m \sin(\omega t + 90^\circ)$
    \item \textbf{પરિણામ:} કરંટ વોલ્ટેજથી 90$^\circ$ આગળ હોય છે
\end{itemize}

\textbf{ઇંડક્ટીવ સર્કિટ માટે:}
\begin{itemize}
    \item વોલ્ટેજ સમીકરણ: $v = L \frac{di}{dt}$
    \item કરંટ: $i = -\frac{V_m}{\omega L} \cos(\omega t) = I_m \sin(\omega t - 90^\circ)$
    \item \textbf{પરિણામ:} કરંટ વોલ્ટેજથી 90$^\circ$ પાછળ હોય છે
\end{itemize}

\begin{answerdiagram}{Phase Relationships}
\begin{tikzpicture}[scale=0.8]
    % Capacitor
    \begin{scope}[xshift=0cm]
        \draw[->] (0,0) -- (2,0) node[right] {$V$};
        \draw[->] (0,0) -- (0,2) node[above] {$I$};
        \node at (1,-1) {Capacitor (Lead)};
    \end{scope}
    
    % Inductor
    \begin{scope}[xshift=4cm]
        \draw[->] (0,0) -- (2,0) node[right] {$V$};
        \draw[->] (0,0) -- (0,-2) node[below] {$I$};
        \node at (1,-1) {Inductor (Lag)};
    \end{scope}
\end{tikzpicture}
\end{answerdiagram}
\end{solutionbox}

\begin{mnemonicbox}
\mnemonic{ELI the ICE man - In EL (inductor), I lags E; in ICE (capacitor), I leads E}
\end{mnemonicbox}

% Question 3
\questionmarks{3(a)}{3}{સાયકલ, ફોર્મ ફેક્ટર અને એમ્પ્લિટ્યુડને વ્યાખ્યાયિત કરો.}

\begin{solutionbox}
\textbf{જવાબ}:

\begin{itemize}
    \item \keyword{સાયકલ (Cycle)}: વેવફોર્મનું એક સંપૂર્ણ પુનરાવર્તન.
    \item \keyword{ફોર્મ ફેક્ટર (Form Factor)}: RMS મૂલ્યનો સરેરાશ મૂલ્ય સાથેનો ગુણોત્તર. સાઇન વેવ માટે = 1.11.
    \item \keyword{એમ્પ્લિટ્યુડ (Amplitude)}: વેવફોર્મનું તેના સરેરાશ સ્થાનથી મહત્તમ વિચલન.
\end{itemize}

\begin{answerdiagram}{Waveform Definitions}
\begin{tikzpicture}
    \begin{axis}[
        width=8cm, height=4cm,
        axis lines=middle,
        xtick={0, 6.28},
        xticklabels={0, $2\pi$},
        ytick={1},
        yticklabels={$V_m$},
        xlabel=$\omega t$,
        ylabel=$V$,
        ymin=-1.5, ymax=1.5
    ]
    \addplot[blue, thick, domain=0:6.5, samples=100] {sin(deg(x))};
    \draw[<->] (axis cs:0,-1.2) -- (axis cs:6.28,-1.2) node[midway, below] {Cycle};
    \draw[dashed] (axis cs:1.57,0) -- (axis cs:1.57,1);
    \node at (axis cs:1.57,1.2) {Amplitude};
    \end{axis}
\end{tikzpicture}
\end{answerdiagram}
\end{solutionbox}

\begin{mnemonicbox}
\mnemonic{Cycles Complete, Form Factors Find ratio, Amplitude Achieves maximum}
\end{mnemonicbox}

\questionmarks{3(b)}{4}{આરએમએસ અને સરેરાશ મૂલ્ય વ્યાખ્યાયિત કરો. સાઇન વેવફોર્મનું આરએમએસ અને સરેરાશ મૂલ્ય નુ સૂત્ર લખો.}

\begin{solutionbox}
\textbf{જવાબ}:

\begin{center}
\captionof{table}{RMS vs સરેરાશ મૂલ્ય}
\begin{tabulary}{\linewidth}{|L|L|L|}
\hline
\textbf{પેરામીટર} & \textbf{વ્યાખ્યા} & \textbf{સાઇન વેવ માટે ફોર્મ્યુલા} \\ \hline
\textbf{RMS મૂલ્ય} & વર્ગ કરેલા મૂલ્યોના સરેરાશનો વર્ગમૂળ & $V_{rms} = V_m/\sqrt{2} = 0.707 V_m$ \\ \hline
\textbf{સરેરાશ મૂલ્ય} & અર્ધ સાયકલ પર તમામ ક્ષણિક મૂલ્યોની સરેરાશ & $V_{avg} = 2V_m/\pi = 0.637 V_m$ \\ \hline
\end{tabulary}
\end{center}

\begin{itemize}
    \item \keyword{RMS (Root Mean Square)}: સમાન હીટિંગ અસર ઉત્પન્ન કરતું સમકક્ષ DC મૂલ્ય.
    \item \keyword{સરેરાશ મૂલ્ય (Average Value)}: અર્ધ સાયકલ પર તમામ ક્ષણિક મૂલ્યોની સરેરાશ.
\end{itemize}
\end{solutionbox}

\begin{mnemonicbox}
\mnemonic{RMS Relates to heating, Average Adds and divides}
\end{mnemonicbox}

\questionmarks{3(c)}{7}{એપરંટ પાવર, ટ્રુ પાવર અને રિયેક્ટીવ પાવર સમજાવો. તેમના માપનના એકમ જણાવો.}

\begin{solutionbox}
\textbf{જવાબ}:

\begin{center}
\captionof{table}{પાવરના પ્રકારો}
\begin{tabulary}{\linewidth}{|L|L|L|L|}
\hline
\textbf{પાવર પ્રકાર} & \textbf{વ્યાખ્યા} & \textbf{ફોર્મ્યુલા} & \textbf{એકમ} \\ \hline
\textbf{એપરંટ પાવર (S)} & કુલ પૂરો પાડેલો પાવર & $S = VI$ & VA (Volt-Ampere) \\ \hline
\textbf{ટ્રુ પાવર (P)} & ખરેખર વપરાયેલો પાવર & $P = VI \cos \phi$ & W (Watt) \\ \hline
\textbf{રિયેક્ટીવ પાવર (Q)} & સ્ત્રોત અને લોડ વચ્ચે આવતો-જતો પાવર & $Q = VI \sin \phi$ & VAR (Volt-Ampere Reactive) \\ \hline
\end{tabulary}
\end{center}

\keyword{પાવર ટ્રાયએંગલ:} $S^2 = P^2 + Q^2$

\begin{answerdiagram}{Power Triangle}
\begin{tikzpicture}
    \draw[thick] (0,0) -- (4,0) node[midway, below] {True Power $P$ (W)};
    \draw[thick] (4,0) -- (4,3) node[midway, right] {Reactive Power $Q$ (VAR)};
    \draw[thick] (0,0) -- (4,3) node[midway, above left] {Apparent Power $S$ (VA)};
    \draw (0.6,0) arc (0:36.87:0.6) node[midway, right] {$\phi$};
\end{tikzpicture}
\end{answerdiagram}
\end{solutionbox}

\begin{mnemonicbox}
\mnemonic{Active Performs work, Reactive Returns energy, Apparent Adds vectors}
\end{mnemonicbox}

% Question 3 OR
\questionmarks{3(a) OR}{3}{3-ફેઝ વોલ્ટેજના ગાણિતિક અભિવ્યક્તિઓ લખો.}

\begin{solutionbox}
\textbf{જવાબ}:

\textbf{થ્રી-ફેઝ વોલ્ટેજની અભિવ્યક્તિઓ:}

\begin{center}
\captionof{table}{3-ફેઝ વોલ્ટેજ}
\begin{tabulary}{\linewidth}{|L|L|}
\hline
\textbf{ફેઝ} & \textbf{અભિવ્યક્તિ} \\ \hline
\textbf{R-ફેઝ} & $V_R = V_m \sin(\omega t)$ \\ \hline
\textbf{Y-ફેઝ} & $V_Y = V_m \sin(\omega t - 120^\circ)$ \\ \hline
\textbf{B-ફેઝ} & $V_B = V_m \sin(\omega t - 240^\circ)$ \\ \hline
\end{tabulary}
\end{center}

જ્યાં $V_m$ મહત્તમ વોલ્ટેજ છે અને $\omega$ એન્ગ્યુલર ફ્રિક્વન્સી છે.
\end{solutionbox}

\begin{mnemonicbox}
\mnemonic{Red phase Reference, Yellow lags 120, Blue brings up 240}
\end{mnemonicbox}

\questionmarks{3(b) OR}{4}{ક્રેસ્ટ ફેક્ટર વ્યાખ્યાયિત કરો અને સાઇન વેવ માટે ક્રેસ્ટ ફેક્ટર ની કિમત લખો.}

\begin{solutionbox}
\textbf{જવાબ}:

\begin{itemize}
    \item \keyword{ક્રેસ્ટ ફેક્ટર (Crest Factor)}: વેવફોર્મના પીક મૂલ્યનો RMS મૂલ્ય સાથેનો ગુણોત્તર.
    \item \keyword{ફોર્મ્યુલા}: $\text{Crest Factor} = \frac{\text{Peak Value}}{\text{RMS Value}}$
    \item \keyword{સાઇન વેવ માટે}: $\text{Crest Factor} = \frac{1}{0.707} = 1.414$
\end{itemize}

\begin{answerdiagram}{Crest Factor Concept}
\begin{tikzpicture}
    \begin{axis}[
        width=8cm, height=4cm,
        axis lines=middle,
        xtick=\empty, ytick=\empty,
        ymin=0, ymax=1.5,
        xmin=0, xmax=6.5
    ]
    \addplot[blue, thick, domain=0:6.5, samples=100] {abs(sin(deg(x)))};
    \draw[dashed] (axis cs:0,1) -- (axis cs:6.5,1) node[right] {Peak};
    \draw[dashed] (axis cs:0,0.707) -- (axis cs:6.5,0.707) node[right] {RMS};
    \draw[<->] (axis cs:3.14, 0.707) -- (axis cs:3.14, 1) node[midway, right] {Crest Factor Ratio};
    \end{axis}
\end{tikzpicture}
\end{answerdiagram}
\end{solutionbox}

\begin{mnemonicbox}
\mnemonic{Crest Compares peak to RMS}
\end{mnemonicbox}

\questionmarks{3(c) OR}{7}{વિવિધ 3-ફેઝ વિદ્યુત જોડાણોનું વર્ણન કરો.}

\begin{solutionbox}
\textbf{જવાબ}:

\begin{center}
\captionof{table}{સ્ટાર vs ડેલ્ટા જોડાણ}
\begin{tabulary}{\linewidth}{|L|L|L|}
\hline
\textbf{પેરામીટર} & \textbf{સ્ટાર (Y) જોડાણ} & \textbf{ડેલ્ટા ($\Delta$) જોડાણ} \\ \hline
\textbf{લાઇન વોલ્ટેજ ($V_L$)} & $\sqrt{3} \times \text{ફેઝ વોલ્ટેજ}$ & ફેઝ વોલ્ટેજ જેટલું જ \\ \hline
\textbf{લાઇન કરંટ ($I_L$)} & ફેઝ કરંટ જેટલો જ & $\sqrt{3} \times \text{ફેઝ કરંટ}$ \\ \hline
\textbf{ન્યુટ્રલ વાયર} & હાજર & ગેરહાજર \\ \hline
\textbf{ઉપયોગ} & અસંતુલિત લોડ્સ, રહેણાંક & સંતુલિત લોડ્સ, ઔદ્યોગિક \\ \hline
\end{tabulary}
\end{center}

\begin{answerdiagram}{Star and Delta Connections}
\begin{circuitikz}[scale=0.7, transform shape]
    % Star
    \draw (0,0) node[anchor=north]{N} to[R, l=R] (0,2) node[anchor=south]{R};
    \draw (0,0) to[R, l=Y] (-1.73,-1) node[anchor=north]{Y};
    \draw (0,0) to[R, l=B] (1.73,-1) node[anchor=north]{B};
    \node at (0,-2.5) {Star Connection};

    % Delta
    \begin{scope}[xshift=5cm, yshift=-1cm]
    \draw (0,0) to[R, l=B] (4,0);
    \draw (0,0) to[R, l=R] (2,3.46);
    \draw (2,3.46) to[R, l=Y] (4,0);
    \node at (2,-1.5) {Delta Connection};
    \end{scope}
\end{circuitikz}
\end{answerdiagram}
\end{solutionbox}

\begin{mnemonicbox}
\mnemonic{Star Shows neutral, Delta Delivers higher current}
\end{mnemonicbox}

% Question 4
\questionmarks{4(a)}{3}{જો આરએમએસ મૂલ્ય 230V હોય તો સાઇનયુસાઇડલ વોલ્ટેજની પીક-ટુ-પીક કિંમતની ગણતરી કરો.}

\begin{solutionbox}
\textbf{જવાબ}:

\begin{center}
\captionof{table}{ગણતરીના પગલાં}
\begin{tabulary}{\linewidth}{|L|L|L|}
\hline
\textbf{પેરામીટર} & \textbf{ફોર્મ્યુલા} & \textbf{ગણતરી} \\ \hline
\textbf{RMS મૂલ્ય} & આપેલ છે & 230V \\ \hline
\textbf{પીક મૂલ્ય} & $V_m = \sqrt{2} \times V_{rms}$ & $V_m = \sqrt{2} \times 230 = 325.27\text{V}$ \\ \hline
\textbf{પીક-ટુ-પીક મૂલ્ય} & $V_{p-p} = 2 \times V_m$ & $V_{p-p} = 2 \times 325.27 = 650.54\text{V}$ \\ \hline
\end{tabulary}
\end{center}

\textbf{આથી, પીક-ટુ-પીક મૂલ્ય = 650.54V}
\end{solutionbox}

\begin{mnemonicbox}
\mnemonic{RMS to Peak - multiply by root2, Peak to Peak - double it}
\end{mnemonicbox}

\questionmarks{4(b)}{4}{આપેલા એસી પ્રવાહ i = 142.14sin628t માટે ફ્રીક્વંસી અને ટાઇમ પિરિયડ શોધો.}

\begin{solutionbox}
\textbf{જવાબ}:

\textbf{આપેલ સમીકરણ:} $i = 142.14 \sin(628t)$ implies $\omega = 628$ rad/s.

\begin{center}
\captionof{table}{ગણતરીના પગલાં}
\begin{tabulary}{\linewidth}{|L|L|L|}
\hline
\textbf{પેરામીટર} & \textbf{ફોર્મ્યુલા} & \textbf{ગણતરી} \\ \hline
\textbf{ફ્રીક્વંસી} & $f = \omega/(2\pi)$ & $f = 628/(2\pi) = 100 \text{ Hz}$ \\ \hline
\textbf{ટાઇમ પિરિયડ} & $T = 1/f$ & $T = 1/100 = 0.01 \text{ s} = 10 \text{ ms}$ \\ \hline
\end{tabulary}
\end{center}

\textbf{આથી, ફ્રીક્વંસી = 100 Hz અને ટાઇમ પિરિયડ = 0.01 s}
\end{solutionbox}

\begin{mnemonicbox}
\mnemonic{Frequency From omega divide 2pi, Time takes inverse}
\end{mnemonicbox}

\questionmarks{4(c)}{7}{ફ્લેમિંગના ડાબા હાથનો નિયમ અને જમણા હાથનો નિયમ સમજાવો.}

\begin{solutionbox}
\textbf{જવાબ}:

\begin{itemize}
    \item \keyword{ફ્લેમિંગનો ડાબા હાથનો નિયમ (મોટર):}
    \begin{itemize}
        \item ચુંબકીય ક્ષેત્રમાં વિદ્યુત પ્રવાહ વહનકર્તા પર લાગતા \textbf{બળ}ની દિશા નક્કી કરવા માટે વપરાય છે.
        \item અંગૂઠો: ગતિ (બળ)
        \item પ્રથમ આંગળી: ચુંબકીય ક્ષેત્ર
        \item મધ્ય આંગળી: વિદ્યુત પ્રવાહ
    \end{itemize}

    \item \keyword{ફ્લેમિંગનો જમણા હાથનો નિયમ (જનરેટર):}
    \begin{itemize}
        \item જ્યારે વાહક ચુંબકીય ક્ષેત્રમાં ગતિ કરે છે ત્યારે \textbf{પ્રેરિત વિદ્યુત પ્રવાહ}ની દિશા નક્કી કરવા માટે વપરાય છે.
        \item અંગૂઠો: વાહકની ગતિ
        \item પ્રથમ આંગળી: ચુંબકીય ક્ષેત્ર
        \item મધ્ય આંગળી: પ્રેરિત વિદ્યુત પ્રવાહ
    \end{itemize}
\end{itemize}

\begin{answerdiagram}{Fleming's Rules Hand Positions}
\begin{tikzpicture}
    % Left Hand
    \begin{scope}[xshift=0cm]
        \draw[->, Ultra Thick, blue] (0,0) -- (0,2) node[above] {Motion};
        \draw[->, Ultra Thick, red] (0,0) -- (2,0) node[right] {Field};
        \draw[->, Ultra Thick, green!60!black] (0,0) -- (0,0,-2) node[left] {Current};
        \node at (0,-1) {Left Hand (Motor)};
    \end{scope}
    
    % Right Hand
    \begin{scope}[xshift=5cm]
        \draw[->, Ultra Thick, blue] (0,0) -- (0,2) node[above] {Motion};
        \draw[->, Ultra Thick, red] (0,0) -- (2,0) node[right] {Field};
        \draw[->, Ultra Thick, purple] (0,0) -- (0,0,-2) node[left] {Induced Current};
        \node at (0,-1) {Right Hand (Generator)};
    \end{scope}
\end{tikzpicture}
\end{answerdiagram}
\end{solutionbox}

\begin{mnemonicbox}
\mnemonic{Left Lifts motors, Right Raises generators}
\end{mnemonicbox}

% Question 4 OR
\questionmarks{4(a) OR}{3}{0.6 ટેસ્લાના મેગ્નેટિક ફીલ્ડમાં 30 મીટર/સેકંડ ગતિ સાથે 1 મીટરની લંબાઈ નો વાહક ક્ષેત્ર સાથે 30$^{\circ}$ નો કોણ બનાવે છે. તેમાં ઉત્ત્પન્ન થતુ ડાયનેમીક ઇએમએફની ગણતરી કરો. (sin 30$^{\circ}$=0.5 નો ઉપયોગ કરો)}

\begin{solutionbox}
\textbf{જવાબ}:

\begin{center}
\captionof{table}{આપેલ પેરામીટર્સ}
\begin{tabulary}{\linewidth}{|L|L|}
\hline
\textbf{પેરામીટર} & \textbf{મૂલ્ય} \\ \hline
\textbf{લંબાઈ (l)} & 1 મીટર \\ \hline
\textbf{ગતિ (v)} & 30 m/s \\ \hline
\textbf{ચુંબકીય ક્ષેત્ર (B)} & 0.6 Tesla \\ \hline
\textbf{કોણ ($\theta$)} & $30^\circ$ \\ \hline
\end{tabulary}
\end{center}

\keyword{ફોર્મ્યુલા:} $E = Blv \sin \theta$

\textbf{ગણતરી:}
\[ E = 0.6 \times 1 \times 30 \times 0.5 = 9 \text{ volts} \]

\textbf{આથી, પ્રેરિત EMF = 9 volts}
\end{solutionbox}

\begin{mnemonicbox}
\mnemonic{EMF Emerges from Field, velocity and Length with angle}
\end{mnemonicbox}

\questionmarks{4(b) OR}{4}{લેન્ઝનો નિયમ લખો અને સમજાવો.}

\begin{solutionbox}
\textbf{જવાબ}:

\keyword{લેન્ઝનો નિયમ:} પ્રેરિત EMF અથવા વિદ્યુત પ્રવાહની દિશા હંમેશા એવી હોય છે કે તે તેને ઉત્પન્ન કરતા કારણનો વિરોધ કરે છે.

\keyword{ઉપયોગ:} જ્યારે ચુંબક કોઈલની નજીક આવે છે, ત્યારે પ્રેરિત વિદ્યુત પ્રવાહ એક ચુંબકીય ક્ષેત્ર બનાવે છે જે આવતા ચુંબકને પાછો ધક્કો મારે છે.

\begin{answerdiagram}{Lenz's Law}
\begin{tikzpicture}
    \draw[fill=gray!20] (0,0) circle (0.5 and 1.5);
    \draw (-0.5,1.5) to[L] (-0.5, -1.5); % Coil representation
    \draw[->, ultra thick] (-2,0) -- (-1,0); 
    \draw[fill=red!20] (-4,-0.5) rectangle (-2,0.5) node[midway] {N};
    \node at (-3, 0.8) {Magnet moving closer};
    \node at (2,0) {Induced N pole opposes motion};
\end{tikzpicture}
\end{answerdiagram}
\end{solutionbox}

\begin{mnemonicbox}
\mnemonic{Lenz Likes to Oppose}
\end{mnemonicbox}

\questionmarks{4(c) OR}{7}{સ્થિર અને ગતિશીલ રીતે પ્રેરિત ઇએમએફ સમજાવો.}

\begin{solutionbox}
\textbf{જવાબ}:

\begin{center}
\captionof{table}{સ્થિર vs ગતિશીલ પ્રેરિત EMF}
\begin{tabulary}{\linewidth}{|L|L|L|}
\hline
\textbf{પેરામીટર} & \textbf{સ્થિર પ્રેરિત EMF} & \textbf{ગતિશીલ પ્રેરિત EMF} \\ \hline
\textbf{વ્યાખ્યા} & કરંટ/ફ્લક્સમાં ફેરફાર થવાથી પ્રેરિત EMF & ચુંબકીય ક્ષેત્રમાં વાહકની ગતિથી પ્રેરિત EMF \\ \hline
\textbf{ભૌતિક ક્રિયા} & સ્થિર વાહક, બદલાતું ક્ષેત્ર & સ્થિર ક્ષેત્રમાં ગતિશીલ વાહક \\ \hline
\textbf{ઉદાહરણ} & ટ્રાન્સફોર્મર & જનરેટર \\ \hline
\textbf{ફોર્મ્યુલા} & $e = -N \frac{d\Phi}{dt}$ & $e = Blv \sin \theta$ \\ \hline
\end{tabulary}
\end{center}
\end{solutionbox}

\begin{mnemonicbox}
\mnemonic{Static Stays but flux Changes, Dynamic Drives through field}
\end{mnemonicbox}

% Question 5
\questionmarks{5(a)}{3}{પીવી સેલ સમજાવો.}

\begin{solutionbox}
\textbf{જવાબ}:

\begin{itemize}
    \item \keyword{PV સેલ}: ફોટોવોલ્ટિક અસરનો ઉપયોગ કરીને સૂર્યપ્રકાશને સીધા વીજળીમાં રૂપાંતરિત કરતું ઉપકરણ.
    \item \keyword{કાર્યપ્રણાલી}: સૂર્યપ્રકાશ અર્ધવાહક પદાર્થમાં ઇલેક્ટ્રોન્સને ઉત્તેજિત કરે છે, જેનાથી વોલ્ટેજ તફાવત ઉત્પન્ન થાય છે.
    \item \keyword{સામગ્રી}: સામાન્ય રીતે P-N જંક્શન સાથે સિલિકોનમાંથી બનાવવામાં આવે છે.
\end{itemize}

\begin{answerdiagram}{PV Cell Structure}
\begin{tikzpicture}
    \draw[fill=yellow!20] (0,1) rectangle (4,2) node[midway] {N-type};
    \draw[fill=green!20] (0,0) rectangle (4,1) node[midway] {P-type};
    \foreach \x in {1,2,3} \draw[->, decorate, decoration={snake, amplitude=1mm, segment length=2mm, post length=1mm}, thick, orange] (\x,3) -- (\x,2);
    \node at (2,3.5) {Sunlight};
    \draw (0,1.5) -- (-1,1.5) -- (-1,-0.5) to[R, l=Load] (5,-0.5) -- (5,0.5) -- (4,0.5);
\end{tikzpicture}
\end{answerdiagram}
\end{solutionbox}

\begin{mnemonicbox}
\mnemonic{Photons Visit, Current Created}
\end{mnemonicbox}

\questionmarks{5(b)}{4}{પીવી સોલર પેનલ અને એરેસ સમજાવો.}

\begin{solutionbox}
\textbf{જવાબ}:

\begin{center}
\captionof{table}{સોલર સિસ્ટમ હાયરાર્કી}
\begin{tabulary}{\linewidth}{|L|L|}
\hline
\textbf{ઘટક} & \textbf{વર્ણન} \\ \hline
\textbf{PV સેલ} & સૂર્યપ્રકાશને વીજળીમાં રૂપાંતરિત કરતું મૂળભૂત એકમ (0.5V - 0.6V) \\ \hline
\textbf{PV પેનલ} & શ્રેણી/સમાંતરમાં જોડાયેલા અનેક સેલ (સામાન્ય રીતે 12V, 24V) \\ \hline
\textbf{PV એરે} & જરૂરી વોલ્ટેજ/કરંટ મેળવવા માટે જોડાયેલા અનેક પેનલ \\ \hline
\end{tabulary}
\end{center}

\begin{answerdiagram}{Cell to Array Hierarchy}
\begin{tikzpicture}[node distance=1.5cm, auto]
    \node [gtu block] (Cell) {Solar Cell};
    \node [gtu block, right=of Cell] (Panel) {Solar Panel};
    \node [gtu block, right=of Panel] (Array) {Solar Array};
    \draw [gtu arrow] (Cell) -- node{Series Connection} (Panel);
    \draw [gtu arrow] (Panel) -- node{Matrix Connection} (Array);
\end{tikzpicture}
\end{answerdiagram}
\end{solutionbox}

\begin{mnemonicbox}
\mnemonic{Cells Combine into Panels, Panels Produce Arrays}
\end{mnemonicbox}

\questionmarks{5(c)}{7}{વિન્ડ પાવર સિસ્ટમનો બ્લોક ડાયાગ્રામ દોરો અને સમજાવો.}

\begin{solutionbox}
\textbf{જવાબ}:

\textbf{વિન્ડ પાવર સિસ્ટમના ઘટકો:}
\begin{enumerate}
    \item \keyword{વિન્ડ ટર્બાઇન}: પવનની ઊર્જાને યાંત્રિક ઊર્જામાં રૂપાંતરિત કરે છે
    \item \keyword{ગિયરબોક્સ}: જનરેટર માટે રોટેશનલ સ્પીડ વધારે છે
    \item \keyword{જનરેટર}: યાંત્રિક ઊર્જાને વિદ્યુત ઊર્જામાં રૂપાંતરિત કરે છે
    \item \keyword{પાવર ઇલેક્ટ્રોનિક્સ}: વિદ્યુત આઉટપુટને નિયંત્રિત અને નિયમિત કરે છે
    \item \keyword{ટ્રાન્સફોર્મર}: ટ્રાન્સમિશન/ડિસ્ટ્રિબ્યુશન માટે વોલ્ટેજ વધારે/ઘટાડે છે
    \item \keyword{કંટ્રોલ સિસ્ટમ}: સમગ્ર ઓપરેશનનું મોનિટરિંગ અને ઓપ્ટિમાઇઝેશન કરે છે
\end{enumerate}

\begin{answerdiagram}{Wind Power System Block Diagram}
\begin{tikzpicture}[node distance=1.2cm, auto]
    \node [gtu block] (W) {Wind};
    \node [gtu block, right=0.5cm of W] (T) {Turbine};
    \node [gtu block, right=0.5cm of T] (G) {Gearbox};
    \node [gtu block, right=0.5cm of G] (Gen) {Generator};
    \node [gtu block, right=0.5cm of Gen] (PE) {Power\\Electronics};
    \node [gtu block, below=0.8cm of PE] (Tr) {Transformer};
    \node [gtu block, left=0.5cm of Tr] (Grid) {Grid};
    
    \path [gtu arrow] (W) -- (T);
    \path [gtu arrow] (T) -- (G);
    \path [gtu arrow] (G) -- (Gen);
    \path [gtu arrow] (Gen) -- (PE);
    \path [gtu arrow] (PE) -- (Tr);
    \path [gtu arrow] (Tr) -- (Grid);
\end{tikzpicture}
\end{answerdiagram}
\end{solutionbox}

\begin{mnemonicbox}
\mnemonic{Wind Turns Gears, Generating Electrical Returns}
\end{mnemonicbox}

% Question 5 OR
\questionmarks{5(a) OR}{3}{ગ્રીન એનર્જી ના ફાયદા જણાવો.}

\begin{solutionbox}
\textbf{જવાબ}:

\begin{center}
\captionof{table}{ગ્રીન એનર્જીના ફાયદા}
\begin{tabulary}{\linewidth}{|L|L|}
\hline
\textbf{ફાયદા શ્રેણી} & \textbf{ઉદાહરણો} \\ \hline
\textbf{પર્યાવરણીય} & પ્રદૂષણ ઘટાડે છે, કાર્બન ફૂટપ્રિન્ટ ઘટાડે છે \\ \hline
\textbf{આર્થિક} & નોકરીઓ સર્જે છે, ઊર્જા પર આધારિતતા ઘટાડે છે \\ \hline
\textbf{આરોગ્ય} & હવાની ગુણવત્તા સુધારે છે, આરોગ્ય સમસ્યાઓ ઘટાડે છે \\ \hline
\textbf{ટકાઉપણું} & નવીનીકરણીય, અખૂટ સ્ત્રોત \\ \hline
\end{tabulary}
\end{center}
\end{solutionbox}

\begin{mnemonicbox}
\mnemonic{Clean Energy Creates Economic Salvation}
\end{mnemonicbox}

\questionmarks{5(b) OR}{4}{સોલર PV ના ઉપયોગો ટુંકમા સમજાવો.}

\begin{solutionbox}
\textbf{જવાબ}:

\textbf{સોલર PV ઉપયોગો:}
\begin{enumerate}
    \item \keyword{રહેણાંક}: રૂફટોપ સિસ્ટમ, સોલર વોટર હીટર
    \item \keyword{વ્યાપારી}: બિલ્ડિંગ ઇન્ટીગ્રેટેડ PV, સોલર પાર્કિંગ
    \item \keyword{ઔદ્યોગિક}: પ્રોસેસ હીટિંગ, પાવર જનરેશન
    \item \keyword{યુટિલિટી સ્કેલ}: સોલર ફાર્મ, ગ્રીડ સપોર્ટ
    \item \keyword{ઓફ-ગ્રિડ}: ગ્રામીણ વિદ્યુતીકરણ, રિમોટ એપ્લિકેશન્સ
\end{enumerate}

\begin{answerdiagram}{Solar PV Applications}
\begin{tikzpicture}[
  level 1/.style={sibling distance=2.5cm},
  level 2/.style={sibling distance=1.5cm}
]
\node [gtu root] {Solar PV}
    child { node [gtu child] {Residential} }
    child { node [gtu child] {Commercial} }
    child { node [gtu child] {Industrial} }
    child { node [gtu child] {Utility} }
    child { node [gtu child] {Off-grid} };
\end{tikzpicture}
\end{answerdiagram}
\end{solutionbox}

\begin{mnemonicbox}
\mnemonic{Residences, Commerce, Industry Utilize Solar}
\end{mnemonicbox}

\questionmarks{5(c) OR}{7}{ગ્રીન એનર્જી ના વિવિધ પ્રકારો સમજાવો.}

\begin{solutionbox}
\textbf{જવાબ}:

\begin{center}
\captionof{table}{ગ્રીન એનર્જીના પ્રકારો}
\begin{tabulary}{\linewidth}{|L|L|L|}
\hline
\textbf{પ્રકાર} & \textbf{સ્ત્રોત} & \textbf{ઉપયોગો} \\ \hline
\textbf{સોલર} & સૂર્ય & PV સિસ્ટમ, થર્મલ પ્લાન્ટ \\ \hline
\textbf{વિન્ડ} & હવાની ગતિ & વિન્ડ ટર્બાઇન, વિન્ડમિલ \\ \hline
\textbf{હાઇડ્રો} & વહેતા પાણી & ડેમ, રન-ઓફ-રિવર સિસ્ટમ \\ \hline
\textbf{બાયોમાસ} & જૈવિક પદાર્થ & દહન, બાયોગેસ ઉત્પાદન \\ \hline
\textbf{જીયોથર્મલ} & પૃથ્વીની ગરમી & ડાયરેક્ટ હીટિંગ, પાવર પ્લાન્ટ \\ \hline
\textbf{ટાઇડલ} & સમુદ્રના ભરતી-ઓટ & બેરેજ સિસ્ટમ, ટાઇડલ ટર્બાઇન \\ \hline
\end{tabulary}
\end{center}

\begin{answerdiagram}{Green Energy Sources Distribution}
\begin{tikzpicture}
    \pie[radius=2, text=pin, color={yellow!40, blue!20, cyan!30, green!30, orange!30, purple!30}]{
        30/Solar,
        25/Wind,
        20/Hydro,
        15/Biomass,
        7/Geothermal,
        3/Tidal
    }
\end{tikzpicture}
\end{answerdiagram}
\end{solutionbox}

\begin{mnemonicbox}
\mnemonic{Sun, Wind, Hydro, Biomass, Geothermal, Tidal}
\end{mnemonicbox}

\end{document}


