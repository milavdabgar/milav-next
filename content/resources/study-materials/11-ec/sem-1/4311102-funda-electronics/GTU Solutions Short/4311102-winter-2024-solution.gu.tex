\documentclass{article}

% content/resources/templates/preamble.tex
\usepackage[margin=0.6in]{geometry}
\author{Milav Dabgar}
\usepackage{amsmath,amssymb,amsthm}
\usepackage{booktabs}
\usepackage{multirow}
\usepackage{xcolor}
\usepackage{tcolorbox}
\tcbuselibrary{breakable,skins}
\usepackage[colorlinks=true,linkcolor=blue]{hyperref}
\usepackage{titlesec}
\usepackage{enumitem}
\usepackage{tikz}
\usepackage{pgfplots}
\usepackage{circuitikz}
\usepackage[version=4]{mhchem}
\usepackage{longtable}
\usepackage{array}
\usepackage{float}
\usepackage{caption}
\usepackage{listings}

\lstset{
  basicstyle=\small\ttfamily,
  breaklines=true,
  breakatwhitespace=false,
  postbreak=\mbox{\textcolor{red}{$\hookrightarrow$}\space},
  float=false,
  numbers=left,
  numberstyle=\tiny\color{gray},
  numbersep=10pt,
  xleftmargin=2em,
  keywordstyle=\color{blue},
  commentstyle=\color{green!60!black},
  stringstyle=\color{purple},
  backgroundcolor=\color{gray!5},
  showstringspaces=false,
  tabsize=2,
  captionpos=b,
  keepspaces=true,
  columns=flexible
}

\pgfplotsset{compat=1.18}
\usetikzlibrary{shapes,arrows,positioning,calc,patterns,decorations.pathmorphing,decorations.markings,arrows.meta}

% Color scheme
\definecolor{headcolor}{RGB}{0,102,204}
\definecolor{keycolor}{RGB}{220,20,60}
\definecolor{solutioncolor}{RGB}{34,139,34}
\definecolor{mnemoniccolor}{RGB}{148,0,211}
\definecolor{codecolor}{RGB}{0,0,100}

% Spacing
\setlength{\parskip}{3pt}
\setlist[itemize]{nosep}
\setlist[enumerate]{nosep}

% Title formatting
\titleformat{\section}{\Large\bfseries\color{headcolor}}{\thesection}{1em}{}
\titleformat{\subsection}{\large\bfseries\color{headcolor}}{\thesubsection}{1em}{}

% Pandoc tightlist compatibility
\providecommand{\tightlist}{%
  \setlength{\itemsep}{0pt}\setlength{\parskip}{0pt}}

% Pandoc longtable compatibility
\newcounter{none}
\def\thenone{}


% content/resources/templates/gujarati-boxes.tex
\usepackage{fontspec}
\usepackage{polyglossia}

% Set Gujarati as main language (document is primarily in Gujarati)
% Note: gloss-gujarati.ldf doesn't exist in polyglossia, but it will use hyphenation patterns
\setdefaultlanguage{gujarati}
\setotherlanguage{english}

% Configure Gujarati font properly
% Use Language=Default to prevent polyglossia from trying to add language-specific features
% that don't exist for Gujarati, which causes "empty feature" warnings
\newfontfamily\gujaratifont[Script=Gujarati,AutoFakeBold=2.5,AutoFakeSlant=0.3]{Noto Sans Gujarati}
\setmainfont[Script=Gujarati,AutoFakeBold=2.5,AutoFakeSlant=0.3]{Noto Sans Gujarati}
% Use Noto Sans Gujarati for monospace to support Gujarati in text
\setmonofont[Scale=0.9]{Noto Sans Gujarati}

% Configure English to use the same font
\newfontfamily\englishfont[Script=Gujarati,AutoFakeBold=2.5,AutoFakeSlant=0.3]{Noto Sans Gujarati}

% Translations for polyglossia
\gappto\captionsgujarati{
  \renewcommand{\tablename}{કોષ્ટક}
  \renewcommand{\figurename}{આકૃતિ}
}

% Helper for TikZ nodes to ensure Gujarati font
\newcommand{\gu}[1]{{\gujaratifont #1}}

% Custom environments
\newtcolorbox{solutionbox}{
    breakable,
    enhanced,
    colback=solutioncolor!5!white,
    colframe=solutioncolor!75!black,
    fonttitle=\bfseries,
    title=જવાબ
}

\newtcolorbox{solutionboxnobreak}{
 colback=solutioncolor!5!white,
 colframe=solutioncolor!75!black,
 fonttitle=\bfseries,
 title=જવાબ
}

\newtcolorbox{keyformula}{
 breakable,
 enhanced,
 colback=keycolor!5!white,
 colframe=keycolor!75!black,
 fonttitle=\bfseries,
 title=રાસાયણિક સમીકરણ/સૂત્ર
}

\newtcolorbox{mnemonicbox}{
 breakable,
 enhanced,
 colback=mnemoniccolor!5!white,
 colframe=mnemoniccolor!75!black,
 fonttitle=\bfseries,
 title=મેમરી ટ્રીક
}


% Custom commands for GTU solutions
% This file defines semantic commands for consistent formatting

% Question command with automatic formatting
\newcommand{\question}[2]{%
  \section*{Question #1}%
  \textbf{#2}%
}

% OR question variant
\newcommand{\questionor}[2]{%
  \section*{Question #1 OR}%
  \textbf{#2}%
}

% Proper table environment with caption
\newenvironment{answertable}[1]{%
  \begin{table}[htbp]
  \centering
  \caption{#1}
}{%
  \end{table}
}

% Proper figure environment for diagrams
\newenvironment{answerdiagram}[1]{%
  \begin{figure}[htbp]
  \centering
  \caption{#1}
}{%
  \end{figure}
}

% Semantic markup for key terms
\newcommand{\keyword}[1]{\textbf{#1}}
\newcommand{\code}[1]{\texttt{#1}}
\newcommand{\classname}[1]{\texttt{#1}}
\newcommand{\methodname}[1]{\texttt{#1}}

% Proper quotation marks
\newcommand{\mnemonic}[1]{``#1''}


\title{Fundamentals of Electronics (4311102) - Winter 2024 Solution}
\date{January 18, 2024}

\definecolor{lightblue}{RGB}{173,216,230}
\definecolor{lightgreen}{RGB}{144,238,144}
\definecolor{lightpink}{RGB}{255,182,193}
\definecolor{lightyellow}{RGB}{255,255,224}
\definecolor{lightcyan}{RGB}{224,255,255}
\definecolor{lightgray}{gray}{0.9}

\begin{document}
\maketitle

\questionmarks{1(a)}{3}{Give the difference between Passive components and Active components}

\begin{solutionbox}
\textbf{જવાબ}:

\begin{center}
\captionof{table}{Passive vs Active Components}
\begin{tabulary}{\linewidth}{|L|L|}
\hline
\textbf{Passive Components} & \textbf{Active Components} \\ \hline
ઓપરેટ થવા માટે બાહ્ય પાવર સોર્સની જરૂર નથી & ઓપરેટ થવા માટે બાહ્ય પાવર સોર્સની જરૂર પડે છે \\ \hline
સિગ્નલને એમ્પ્લીફાય કે પ્રોસેસ કરી શકતા નથી & સિગ્નલને એમ્પ્લીફાય, સ્વીચ કે પ્રોસેસ કરી શકે છે \\ \hline
ઉદાહરણ: Resistors, Capacitors, Inductors & ઉદાહરણ: Transistors, Diodes, ICs \\ \hline
અન્ય સિગ્નલ દ્વારા કરંટ ફ્લો નિયંત્રિત કરી શકતા નથી & અન્ય સિગ્નલનો ઉપયોગ કરીને કરંટ ફ્લો નિયંત્રિત કરી શકે છે \\ \hline
એનર્જી સ્ટોર કરે છે અથવા ડિસીપેટ (dissipate) કરે છે & એનર્જી જનરેટ કરે છે અથવા ગેઇન (gain) આપે છે \\ \hline
\end{tabulary}
\end{center}
\end{solutionbox}

\begin{mnemonicbox}
\mnemonic{PAPER-A: "Passive Are Power-free, Energy-storing/Resistive; Active Are Amplifying"}
\end{mnemonicbox}

\questionmarks{1(b)}{4}{Explain Working of Light dependent resistor with neat diagram.}

\begin{solutionbox}
\textbf{જવાબ}:

\begin{center}
\begin{tikzpicture}[node distance=2cm]
    \node [gtu block, fill=lightblue] (light) {Light};
    \node [gtu block, fill=lightgreen, right=of light] (ldr) {LDR};
    \node [gtu block, fill=lightpink, right=of ldr] (res) {Change in Resistance};
    
    \draw [gtu arrow] (light) -- (ldr);
    \draw [gtu arrow] (ldr) -- (res);
\end{tikzpicture}
\captionof{figure}{LDR Working Principle}
\end{center}

\textbf{LDR નું કાર્ય:}

\begin{itemize}
    \item \keyword{Construction}: LDR સેમિકન્ડક્ટર મટિરિયલ (સામાન્ય રીતે કેડમિયમ સલ્ફાઇડ) થી બનેલું છે જે અંધારામાં ઉચ્ચ રેઝિસ્ટન્સ ધરાવે છે
    \item \keyword{Photoconductivity}: જ્યારે સપાટી પર પ્રકાશ પડે છે, ત્યારે ફોટોન્સ ઇલેક્ટ્રોન્સને એનર્જી ટ્રાન્સફર કરે છે, જેનાથી ફ્રી ઇલેક્ટ્રોન-હોલ પેર સર્જાય છે
    \item \keyword{Resistance variation}: જેમ પ્રકાશની તીવ્રતા વધે છે તેમ રેઝિસ્ટન્સ નાટકીય રીતે ઘટે છે - અંધારામાં મેગાઓહ્મથી તેજસ્વી પ્રકાશમાં થોડા સો ઓહ્મ સુધી
    \item \keyword{Applications}: લાઇટ સેન્સિંગ સર્કિટ્સ, ઓટોમેટિક સ્ટ્રીટ લાઇટ્સ, કેમેરા એક્સપોઝર કંટ્રોલ
\end{itemize}
\end{solutionbox}

\begin{mnemonicbox}
\mnemonic{MILD: "More Illumination, Less resistance in Devices"}
\end{mnemonicbox}

\questionmarks{1(c)}{7}{Define Intrinsic and Extrinsic Semiconductor. Explain P type and N type semiconductors in detail.}

\begin{solutionbox}
\textbf{જવાબ}:

\begin{center}
\captionof{table}{Semiconductor Types}
\begin{tabulary}{\linewidth}{|L|L|}
\hline
\textbf{Semiconductor Type} & \textbf{વર્ણન} \\ \hline
\textbf{Intrinsic} & શુદ્ધ સેમિકન્ડક્ટર મટિરિયલ જેમાં કોઈ અશુદ્ધિ ઉમેરવામાં આવતી નથી \\ \hline
\textbf{Extrinsic} & સેમિકન્ડક્ટર જેમાં ડોપિંગ દ્વારા નિયંત્રિત અશુદ્ધિઓ ઉમેરવામાં આવે છે \\ \hline
\end{tabulary}
\end{center}

\textbf{P-type Semiconductor:}

\begin{itemize}
    \item \keyword{Doping}: શુદ્ધ સિલિકોનમાં ટ્રાયવેલન્ટ અશુદ્ધિઓ (બોરોન, ગેલિયમ, ઈન્ડિયમ) ઉમેરીને બનાવવામાં આવે છે
    \item \keyword{Hole creation}: દરેક અશુદ્ધિ પરમાણુ વેલેન્સ ઇલેક્ટ્રોન સ્વીકારીને એક હોલ બનાવે છે
    \item \keyword{Majority carriers}: હોલ્સ મેજોરિટી કેરિયર્સ છે
    \item \keyword{Minority carriers}: ઇલેક્ટ્રોન્સ માઈનોરિટી કેરિયર્સ છે
    \item \keyword{Electrical properties}: પોઝિટિવ ચાર્જ કેરિયર્સ વહન પર પ્રભુત્વ ધરાવે છે
\end{itemize}

\textbf{N-type Semiconductor:}

\begin{itemize}
    \item \keyword{Doping}: શુદ્ધ સિલિકોનમાં પેન્ટાવેલેન્ટ અશુદ્ધિઓ (ફોસ્ફરસ, આર્સેનિક, એન્ટિમોની) ઉમેરીને બનાવવામાં આવે છે
    \item \keyword{Electron creation}: દરેક અશુદ્ધિ પરમાણુ એક વધારાનો ઇલેક્ટ્રોન દાન કરે છે
    \item \keyword{Majority carriers}: ઇલેક્ટ્રોન્સ મેજોરિટી કેરિયર્સ છે
    \item \keyword{Minority carriers}: હોલ્સ માઈનોરિટી કેરિયર્સ છે
    \item \keyword{Electrical properties}: નેગેટિવ ચાર્જ કેરિયર્સ વહન પર પ્રભુત્વ ધરાવે છે
\end{itemize}

\textbf{Diagram:}

\begin{center}
\begin{tikzpicture}[scale=0.8, transform shape]
    % N-Type
    \node at (2.5, 4.5) {\Large \textbf{N-Type}};
    \foreach \x in {0, 1.5, 3}
        \foreach \y in {0, 1.5, 3}
            \node[draw, circle, inner sep=2pt] at (\x,\y) {Si};
            
    % Impurity
    \node[draw, circle, inner sep=2pt, fill=lightgreen] at (1.5,1.5) {P};
    
    % Bonds
    \draw (0,1.5) -- (1.5,1.5);
    \draw (3,1.5) -- (1.5,1.5);
    \draw (1.5,0) -- (1.5,1.5);
    \draw (1.5,3) -- (1.5,1.5);
    
    % Free Electron
    \node[circle, fill=black, inner sep=1.5pt, label=right:{$e^-$}] at (2.2, 2.2) {};
    \node at (1.5, -1) {Pentavalent Impurity (Donor)};

    % P-Type
    \begin{scope}[xshift=6cm]
    \node at (2.5, 4.5) {\Large \textbf{P-Type}};
    \foreach \x in {0, 1.5, 3}
        \foreach \y in {0, 1.5, 3}
            \node[draw, circle, inner sep=2pt] at (\x,\y) {Si};
            
    % Impurity
    \node[draw, circle, inner sep=2pt, fill=lightpink] at (1.5,1.5) {B};
    
    % Bonds
    \draw (0,1.5) -- (1.5,1.5);
    \draw (3,1.5) -- (1.5,1.5);
    \draw (1.5,0) -- (1.5,1.5);
    \draw (1.5,3) -- (1.5,1.5);
    
    % Hole
    \node[circle, draw, dashed, inner sep=2pt, label=right:{$h^+$}] at (2.2, 2.2) {};
    \node at (1.5, -1) {Trivalent Impurity (Acceptor)};
    \end{scope}
\end{tikzpicture}
\captionof{figure}{Semiconductor Doping}
\end{center}
\end{solutionbox}

\begin{mnemonicbox}
\mnemonic{PINE: "Positive Impurities make N-type Electrons, Pentavalent donors"}
\end{mnemonicbox}

\questionmarks{1(c) OR}{7}{What is filter circuit? Give type and necessity of Filter and Explain "PI" Filter circuit in brief.}

\begin{solutionbox}
\textbf{જવાબ}:

\textbf{Filter Circuit}: ઇલેક્ટ્રોનિક સર્કિટ જે સિગ્નલમાંથી અનિચ્છનીય ફ્રિકવન્સી કમ્પોનન્ટ્સને દૂર કરે છે, અને ઇચ્છિત ફ્રિકવન્સીઝને પસાર થવા દે છે.

\textbf{ફિલ્ટર્સની જરૂરિયાત}:

\begin{itemize}
    \item \keyword{Ripple reduction}: રેક્ટિફાયર આઉટપુટમાંથી AC રિપલ ઘટાડે છે
    \item \keyword{Clean DC}: સ્મૂધ DC આઉટપુટ વોલ્ટેજ પ્રદાન કરે છે
    \item \keyword{Component protection}: વોલ્ટેજ ફ્લક્ચ્યુએશનથી ડાઉનસ્ટ્રીમ કમ્પોનન્ટ્સનું રક્ષણ કરે છે
    \item \keyword{Efficiency}: એકંદર પાવર સપ્લાય કાર્યક્ષમતા સુધારે છે
\end{itemize}

\textbf{ફિલ્ટર્સના પ્રકારો}:

\begin{center}
\captionof{table}{Filter Types}
\begin{tabulary}{\linewidth}{|L|L|L|}
\hline
\textbf{Filter Type} & \textbf{Components} & \textbf{Application} \\ \hline
Shunt Capacitor & સમાંતરમાં સિંગલ કેપેસિટર & બેઝિક ફિલ્ટરિંગ \\ \hline
L-Type & ઇન્ડક્ટર અને કેપેસિટર & વધુ સારું ફિલ્ટરિંગ \\ \hline
π (Pi) Filter & બે કેપેસિટર અને એક ઇન્ડક્ટર & શ્રેષ્ઠ ફિલ્ટરિંગ \\ \hline
RC Filter & રેઝિસ્ટર અને કેપેસિટર & લો-પાવર એપ્લિકેશન્સ \\ \hline
\end{tabulary}
\end{center}

\textbf{Pi (π) Filter}:

\begin{center}
\begin{tikzpicture}
    \draw (0,0) to[short, o-] (1,0) to[C, l=$C_1$, *-*] (1,-2) to[short, -o] (0,-2);
    \draw (1,0) to[L, l=$L$] (4,0) to[C, l=$C_2$, *-*] (4,-2) -- (1,-2);
    \draw (4,0) to[short, -o] (5,0);
    \draw (4,-2) to[short, -o] (5,-2);
    
    \node[left] at (0, -1) {Input (Rectifier)};
    \node[right] at (5, -1) {Output (Load)};
\end{tikzpicture}
\captionof{figure}{Pi Filter Circuit}
\end{center}

\begin{itemize}
    \item \keyword{Working}: પ્રથમ કેપેસિટર ($C_1$) પ્રારંભિક રિપલ ઘટાડે છે, ઇન્ડક્ટર ($L$) AC કમ્પોનન્ટ્સ બ્લોક કરે છે, બીજું કેપેસિટર ($C_2$) બાકીના રિપલ્સને ફિલ્ટર કરે છે
    \item \keyword{Advantage}: સામાન્ય રીતે 0.5\% ની નીચે રિપલ ફેક્ટર સાથે શ્રેષ્ઠ ફિલ્ટરિંગ પ્રદાન કરે છે
    \item \keyword{Applications}: હાઈ-કરંટ પાવર સપ્લાયમાં વપરાય છે જ્યાં ક્લીન DC મહત્વપૂર્ણ છે
\end{itemize}
\end{solutionbox}

\begin{mnemonicbox}
\mnemonic{PIRO: "Pi filters Input Ripples Out effectively"}
\end{mnemonicbox}

\questionmarks{2(a)}{3}{Write down different types of capacitors and explain any two.}

\begin{solutionbox}
\textbf{જવાબ}:

\textbf{Capacitors ના પ્રકારો}:

\begin{itemize}
    \item Ceramic capacitors
    \item Electrolytic capacitors
    \item Tantalum capacitors
    \item Film capacitors
    \item Mica capacitors
    \item Variable capacitors
\end{itemize}

\textbf{Ceramic Capacitors}:

\begin{itemize}
    \item \keyword{Construction}: ડાઇઇલેક્ટ્રિક તરીકે સિરામિક મટિરિયલ મેટલ પ્લેટ્સ વચ્ચે
    \item \keyword{Capacity}: 1pF થી 1μF
    \item \keyword{Advantages}: ઓછી કિંમત, ઉચ્ચ સ્થિરતા, નોન-પોલરાઇઝ્ડ (non-polarized)
    \item \keyword{Applications}: હાઇ-ફ્રિકવન્સી ફિલ્ટરિંગ, કપલિંગ/ડીકપલિંગ
\end{itemize}

\textbf{Electrolytic Capacitors}:

\begin{itemize}
    \item \keyword{Construction}: એલ્યુમિનિયમ ફોઇલ સાથે ડાઇઇલેક્ટ્રિક તરીકે ઓક્સાઇડ લેયર
    \item \keyword{Capacity}: 1μF થી 10,000μF
    \item \keyword{Characteristics}: પોલરાઇઝ્ડ, વધારે લીકેજ કરંટ
    \item \keyword{Applications}: પાવર સપ્લાય ફિલ્ટરિંગ, ઓડિયો કપલિંગ
\end{itemize}
\end{solutionbox}

\begin{mnemonicbox}
\mnemonic{CAPEX: "Ceramics Are Precise, Electrolytics Expand capacity"}
\end{mnemonicbox}

\questionmarks{2(b)}{4}{Explain air core and toroidal inductor.}

\begin{solutionbox}
\textbf{જવાબ}:

\textbf{Air Core Inductor:}

\begin{center}
\begin{tikzpicture}
    \draw[thick, decoration={aspect=0.3, segment length=3mm, amplitude=3mm,coil}, decorate] (0,0) -- (3,0); 
    \node[below] at (1.5, -0.5) {Air Core (Soleneoid)};
\end{tikzpicture}
\captionof{figure}{Air Core Inductor}
\end{center}

\begin{itemize}
    \item \keyword{Construction}: નોન-મેગ્નેટિક મટિરિયલ (પ્લાસ્ટિક, હવા) ફરતે વાયર વીંટાળેલો હોય છે
    \item \keyword{Properties}: ઓછું ઇન્ડક્ટન્સ, મેગ્નેટિક કોર સેચ્યુરેશન નથી
    \item \keyword{Applications}: હાઇ-ફ્રિકવન્સી સર્કિટ્સ, RF એપ્લિકેશન્સ
    \item \keyword{Advantages}: કોર લોસ નથી, લીનિયર ઓપરેશન, સેચ્યુરેશન નથી
\end{itemize}

\textbf{Toroidal Inductor:}

\begin{center}
\begin{tikzpicture}
    \draw[thick] (0,0) circle (1cm);
    \draw[thick] (0,0) circle (1.5cm);
    \foreach \a in {0,20,...,340} {
        \draw[thick] ({1.25*cos(\a)}, {1.25*sin(\a)}) circle (0.25);
    }
    \node[right] at (2,0) {Wire Turns};
    \node at (0,0) {Core};
\end{tikzpicture}
\captionof{figure}{Toroidal Inductor}
\end{center}

\begin{itemize}
    \item \keyword{Construction}: રિંગ આકારના મેગ્નેટિક કોર ફરતે વાયર વીંટાળેલો હોય છે
    \item \keyword{Properties}: ઉચ્ચ ઇન્ડક્ટન્સ, સેલ્ફ-શિલ્ડિંગ મેગ્નેટિક ફિલ્ડ
    \item \keyword{Applications}: પાવર સપ્લાય, ફિલ્ટર્સ, ટ્રાન્સફોર્મર્સ
    \item \keyword{Advantages}: ઓછું ઇલેક્ટ્રોમેગ્નેટિક ઇન્ટરફિયરન્સ, કાર્યક્ષમ ફ્લક્સ કન્ટેનમેન્ટ
\end{itemize}
\end{solutionbox}

\begin{mnemonicbox}
\mnemonic{TACO: "Toroids Are Contained, Omnidirectional field reduction"}
\end{mnemonicbox}

\questionmarks{2(c)}{7}{Explain Half wave rectifier and Compare different rectifier circuits.}

\begin{solutionbox}
\textbf{જવાબ}:

\textbf{Half Wave Rectifier:}

\begin{center}
\begin{tikzpicture}
    \draw (0,0) node[transformer] (T) {};
    \draw (T.A1) node[left] {AC Mains};
    \draw (T.A2) node[left] {};
    \draw (T.B1) to[D] (3, 0) to[R, l=$R_L$] (3, -2.1) to[short] (T.B2);
    \node at (3, -2.5) {Ground};
    \draw (3, -2.1) node[ground]{};
\end{tikzpicture}
\captionof{figure}{Half Wave Rectifier}
\end{center}

\textbf{કાર્ય સિદ્ધાંત:}

\begin{itemize}
    \item પોઝિટિવ હાફ-સાયકલ દરમિયાન: ડાયોડ કન્ડક્ટ કરે છે, લોડમાંથી કરંટ વહે છે
    \item નેગેટિવ હાફ-સાયકલ દરમિયાન: ડાયોડ બ્લોક કરે છે, કોઈ કરંટ વહેતો નથી
    \item આઉટપુટમાં ઇનપુટ વેવફોર્મની માત્ર પોઝિટિવ હાફ-સાયકલ હોય છે
\end{itemize}

\textbf{Rectifiers ની સરખામણી:}

\begin{center}
\captionof{table}{Rectifier Comparison}
\begin{tabulary}{\linewidth}{|L|L|L|L|}
\hline
\textbf{Parameter} & \textbf{Half Wave} & \textbf{Full Wave (Center-Tap)} & \textbf{Bridge Rectifier} \\ \hline
Diodes required & 1 & 2 & 4 \\ \hline
Output frequency & $f_1 = f_{in}$ & $f_2 = 2 \times f_{in}$ & $f_2 = 2 \times f_{in}$ \\ \hline
Ripple factor & 1.21 & 0.48 & 0.48 \\ \hline
Efficiency & 40.6\% & 81.2\% & 81.2\% \\ \hline
PIV & $2V_m$ & $2V_m$ & $V_m$ \\ \hline
TUF & 0.287 & 0.693 & 0.812 \\ \hline
DC output & $V_m/\pi$ & $2V_m/\pi$ & $2V_m/\pi$ \\ \hline
\end{tabulary}
\end{center}
\end{solutionbox}

\begin{mnemonicbox}
\mnemonic{BRIEF: "Bridge Rectifiers Improve Efficiency Fundamentally"}
\end{mnemonicbox}

\questionmarks{2(a) OR}{3}{Write down different capacitor specifications and explain any two in detail.}

\begin{solutionbox}
\textbf{જવાબ}:

\textbf{Capacitor Specifications:}

\begin{itemize}
    \item Capacitance value
    \item Voltage rating
    \item Tolerance
    \item Temperature coefficient
    \item ESR (Equivalent Series Resistance)
    \item Leakage current
    \item Dielectric type
\end{itemize}

\textbf{Capacitance Value:}

\begin{itemize}
    \item \keyword{Definition}: પ્રતિ વોલ્ટ સ્ટોર કરેલો ઇલેક્ટ્રિક ચાર્જ જથ્થો
    \item \keyword{Units}: ફેરાડ (F) માં મપાય છે, સામાન્ય રીતે માઇક્રોફેરાડ્સ (μF), નેનોફેરાડ્સ (nF), અથવા પીકોફેરાડ્સ (pF)
    \item \keyword{Importance}: કપલિંગ, ફિલ્ટરિંગ, ટાઇમિંગ માટે એપ્લિકેશન યોગ્યતા નક્કી કરે છે
    \item \keyword{Marking}: કમ્પોનન્ટ પર સીધું પ્રિન્ટેડ અથવા કલર-કોડેડ
\end{itemize}

\textbf{Voltage Rating:}

\begin{itemize}
    \item \keyword{Definition}: બ્રેકડાઉન વિના એપ્લાય કરી શકાતો મહત્તમ વોલ્ટેજ
    \item \keyword{Specification}: વર્કિંગ વોલ્ટેજ (WVDC) અને સર્જ વોલ્ટેજ
    \item \keyword{Importance}: રેટિંગ વટાવવાથી ડાઇઇલેક્ટ્રિક બ્રેકડાઉન અને ફેલ્યોર થાય છે
    \item \keyword{Safety factor}: સામાન્ય રીતે સર્કિટ વોલ્ટેજ કરતા 50\% વધારે રેટિંગવાળા કેપેસિટર વાપરો
\end{itemize}
\end{solutionbox}

\begin{mnemonicbox}
\mnemonic{CAVERN: "Capacitance And Voltage Ensure Reliable Network"}
\end{mnemonicbox}

\questionmarks{2(b) OR}{4}{Explain classification of Resistor based on materials.}

\begin{solutionbox}
\textbf{જવાબ}:

\begin{center}
\captionof{table}{Resistor Classification}
\begin{tabulary}{\linewidth}{|L|L|L|L|}
\hline
\textbf{Resistor Type} & \textbf{Material} & \textbf{Properties} & \textbf{Applications} \\ \hline
\textbf{Carbon Composition} & Carbon particles + Ceramic binder & High temperature coefficient, noisy & General purpose, surge protection \\ \hline
\textbf{Carbon Film} & Carbon film on ceramic & Better stability than carbon composition & General purpose circuits \\ \hline
\textbf{Metal Film} & Nickel chromium film on ceramic & Low noise, stable, precise & Audio circuits, instrumentation \\ \hline
\textbf{Wire Wound} & Resistance wire around ceramic & High power, low temperature coefficient & Power supplies, high current applications \\ \hline
\textbf{Metal Oxide} & Metal oxide film on ceramic & Stable, high temperature tolerance & High stability applications, power supplies \\ \hline
\end{tabulary}
\end{center}

\textbf{Carbon Film Resistors ની લાક્ષણિકતાઓ:}

\begin{itemize}
    \item Temperature coefficient: -250 to 500 ppm/°C
    \item Tolerance: 5\% to 10\%
    \item Noise: મધ્યમ થી ઓછું
\end{itemize}

\textbf{Metal Film Resistors ની લાક્ષણિકતાઓ:}

\begin{itemize}
    \item Temperature coefficient: 50 to 100 ppm/°C
    \item Tolerance: 0.1\% to 2\%
    \item Noise: ખૂબ ઓછું
\end{itemize}
\end{solutionbox}

\begin{mnemonicbox}
\mnemonic{COMFORT: "Carbon Offers Moderate Films, Others Resist Temperature better"}
\end{mnemonicbox}

\questionmarks{2(c) OR}{7}{Explain full wave bridge and center tapped rectifier with diagram and waveform.}

\begin{solutionbox}
\textbf{જવાબ}:

\textbf{Full Wave Bridge Rectifier:}

\begin{center}
\begin{tikzpicture}
    \draw (0,0) node[transformer] (T) {};
    \draw (T.B1) -- (4, 1.5);
    \draw (T.B2) -- (4, -1.5);
    
    \draw (4, 1.5) to[D*, l=$D_1$] (6, 0);
    \draw (4, -1.5) to[D*, l=$D_2$] (6, 0);
    \draw (2, 0) to[D*, l=$D_4$] (4, 1.5);
    \draw (2, 0) to[D*, l=$D_3$] (4, -1.5);
    
    \draw (2,0) -- (2,-2) -- (5,-2) node[ground] {};
    \draw (6,0) -- (6,0) to[R, l=$R_L$] (6,-2) -- (5,-2);
\end{tikzpicture}
\captionof{figure}{Full Wave Bridge Rectifier}
\end{center}

\textbf{કાર્ય:}

\begin{itemize}
    \item \keyword{Positive half-cycle}: $D_1$ અને $D_3$ કન્ડક્ટ કરે છે, લોડમાંથી કરંટ વહે છે
    \item \keyword{Negative half-cycle}: $D_2$ અને $D_4$ કન્ડક્ટ કરે છે, કરંટ લોડમાંથી સમાન દિશામાં વહે છે
    \item \keyword{Output}: ઇનપુટની બંને હાફ-સાયકલ પોઝિટિવ આઉટપુટમાં રૂપાંતરિત થાય છે
\end{itemize}

\textbf{Center Tapped Full Wave Rectifier:}

\begin{center}
\begin{tikzpicture}
    \draw (0,0) node[transformer core] (T) {};
    \draw (T.B1) to[D*, l=$D_1$] (4,1) -- (4,0);
    \draw (T.B2) to[D*, l=$D_2$] (4,-1) -- (4,0);
    \draw (4,0) to[R, l=$R_L$] (6,0) -- (6,-2);
    \draw ($(T.B1)!0.5!(T.B2)$) -- (2,0) -- (2,-2) -- (6,-2);
    \draw (4,-2) node[ground] {};
\end{tikzpicture}
\captionof{figure}{Center Tapped Rectifier}
\end{center}

\textbf{કાર્ય:}

\begin{itemize}
    \item \keyword{Positive half-cycle}: $D_1$ કન્ડક્ટ કરે છે, $D_2$ બ્લોક કરે છે
    \item \keyword{Negative half-cycle}: $D_2$ કન્ડક્ટ કરે છે, $D_1$ બ્લોક કરે છે
    \item \keyword{Output}: ઇનપુટની બંને હાફ-સાયકલ પોઝિટિવ આઉટપુટમાં રૂપાંતરિત થાય છે
\end{itemize}

\textbf{Waveforms:}

\begin{center}
\begin{tikzpicture}
    \begin{scope}[xshift=0cm, yshift=2cm]
        \draw[->] (0,0) -- (4,0) node[right] {$t$};
        \draw[->] (0,-1) -- (0,1) node[above] {$V_{in}$};
        \draw[red, thick] plot[domain=0:3.5, samples=50] (\x, {0.8*sin(200*\x)});
        \node at (2, -1.5) {Input AC};
    \end{scope}
    
    \begin{scope}[xshift=5cm, yshift=2cm]
        \draw[->] (0,0) -- (4,0) node[right] {$t$};
        \draw[->] (0,0) -- (0,1) node[above] {$V_{out}$};
        \draw[blue, thick] plot[domain=0:3.5, samples=100] (\x, {abs(0.8*sin(200*\x))});
        \node at (2, -1.5) {Output DC};
    \end{scope}
\end{tikzpicture}
\captionof{figure}{Input and Output Waveforms}
\end{center}
\end{solutionbox}

\begin{mnemonicbox}
\mnemonic{FOUR-TWO: "FOUr diodes for Bridge, TWO diodes for Center-Tap"}
\end{mnemonicbox}

\questionmarks{3(a)}{3}{Explain the characteristic of Varactor diode.}

\begin{solutionbox}
\textbf{જવાબ}:

\textbf{Varactor Diode Characteristics:}

\begin{center}
\begin{tikzpicture}
    % Symbol
    \draw (0,2) to[varcap, l=Varactor] (2,2);
    
    % Graph
    \draw[->] (3,0) -- (6,0) node[right] {$V_R$};
    \draw[->] (3,0) -- (3,3) node[above] {$C$};
    \draw[thick, blue] (3.2, 2.8) .. controls (3.5, 0.5) .. (6, 0.2);
    \node at (5, 1) {$C \propto 1/\sqrt{V}$};
\end{tikzpicture}
\captionof{figure}{Varactor Diode C-V Curve}
\end{center}

\begin{itemize}
    \item \keyword{Operating principle}: જંકશન કેપેસિટન્સ રિવર્સ બાયસ વોલ્ટેજ સાથે બદલાય છે
    \item \keyword{C-V relationship}: જેમ રિવર્સ વોલ્ટેજ વધે છે તેમ કેપેસિટન્સ ઘટે છે
    \item \keyword{Tuning ratio}: સામાન્ય રીતે 4:1 થી 10:1 કેપેસિટન્સ ભિન્નતા
    \item \keyword{Applications}: વોલ્ટેજ-કંટ્રોલ્ડ ઓસિલેટર (VCO), FM મોડ્યુલેશન, ટ્યુનિંગ સર્કિટ્સ
\end{itemize}
\end{solutionbox}

\begin{mnemonicbox}
\mnemonic{VARA: "Voltage Adjusts Reverse-biased capacitance Automatically"}
\end{mnemonicbox}

\questionmarks{3(b)}{3}{State and explain Faraday's laws of electromagnetic induction.}

\begin{solutionbox}
\textbf{જવાબ}:

\textbf{Faraday's Laws of Electromagnetic Induction:}

\textbf{First Law:}
\begin{itemize}
    \item \keyword{Statement}: જ્યારે કોઈ કંડક્ટર મેગ્નેટિક ફ્લક્સને કાપે છે, ત્યારે કંડક્ટરમાં EMF પ્રેરિત થાય છે
    \item \keyword{Mathematical expression}: EMF $\propto$ મેગ્નેટિક ફ્લક્સના ફેરફારનો દર
    \item \keyword{Application}: જનરેટર, ટ્રાન્સફોર્મર, ઇન્ડક્ટરનો આધાર
\end{itemize}

\textbf{Second Law:}
\begin{itemize}
    \item \keyword{Statement}: પ્રેરિત EMF નું મૂલ્ય મેગ્નેટિક ફ્લક્સ લિંકેજના ફેરફારના દર જેટલું હોય છે
    \item \keyword{Mathematical expression}: $EMF = -N \times (d\Phi/dt)$
      \begin{itemize}
          \item જ્યાં: $N$ = આંટાઓની સંખ્યા, $d\Phi/dt$ = ફ્લક્સના ફેરફારનો દર
      \end{itemize}
    \item \keyword{Negative sign}: દિશા સૂચવે છે (Lenz's Law) - પ્રેરિત કરંટ ફેરફારનો વિરોધ કરે છે
\end{itemize}

\textbf{Diagram:}

\begin{center}
\begin{tikzpicture}
    \draw[thick, fill=gray!20] (0,0) rectangle (0.5, 2) node[midway, rotate=90] {Magnet};
    \node at (0.25, 2.2) {N};
    \node at (0.25, -0.2) {S};
    
    \draw[thick, ->] (1, 1) -- (2, 1) node[midway, above] {Motion};
    
    \draw[decoration={aspect=0.3, segment length=2mm, amplitude=3mm,coil}, decorate] (3,0) -- (6,0);
    \draw (3,0) -- (3,-1) -- node[midway, draw, circle, fill=white] {G} (6,-1) -- (6,0);
    
    \node at (4.5, 0.6) {Coil};
\end{tikzpicture}
\captionof{figure}{Electromagnetic Induction}
\end{center}
\end{solutionbox}

\begin{mnemonicbox}
\mnemonic{FACE: "Flux Alteration Creates Electricity"}
\end{mnemonicbox}

\questionmarks{3(c)}{7}{Compare different Transistor Configurations.}

\begin{solutionbox}
\textbf{જવાબ}:

\begin{center}
\captionof{table}{Transistor Configurations Comparison}
\begin{tabulary}{\linewidth}{|L|L|L|L|}
\hline
\textbf{Parameter} & \textbf{Common Emitter (CE)} & \textbf{Common Base (CB)} & \textbf{Common Collector (CC)} \\ \hline
\textbf{Input Terminal} & Base & Emitter & Base \\ \hline
\textbf{Output Terminal} & Collector & Collector & Emitter \\ \hline
\textbf{Common Terminal} & Emitter & Base & Collector \\ \hline
\textbf{Current Gain} & $\beta = I_C/I_B$ (20-500) & $\alpha = I_C/I_E$ (0.95-0.99) & $\gamma = I_E/I_B$ ($\beta+1$) \\ \hline
\textbf{Voltage Gain} & High (250-1000) & Medium (150-800) & Less than 1 \\ \hline
\textbf{Input Impedance} & મધ્યમ (1-2k$\Omega$) & ઓછું (30-150$\Omega$) & વધારે (50-500k$\Omega$) \\ \hline
\textbf{Output Impedance} & વધારે (30-50k$\Omega$) & ખૂબ વધારે (250k$\Omega$-1M$\Omega$) & ઓછું (50-100$\Omega$) \\ \hline
\textbf{Phase Shift} & 180° & 0° & 0° \\ \hline
\textbf{Applications} & Amplifiers, oscillators & RF amplifiers & Impedance matching, buffers \\ \hline
\end{tabulary}
\end{center}

\textbf{$\alpha$, $\beta$ અને $\gamma$ વચ્ચેનો સંબંધ:}
\begin{itemize}
    \item $\beta = \alpha/(1-\alpha)$
    \item $\alpha = \beta/(1+\beta)$
    \item $\gamma = \beta+1$
\end{itemize}
\end{solutionbox}

\begin{mnemonicbox}
\mnemonic{BEC: "Base input for Emitter output needs Collector as common terminal"}
\end{mnemonicbox}

\questionmarks{3(a) OR}{3}{What is forbidden energy gap? Draw the energy band diagram for insulator, conductor and semiconductor.}

\begin{solutionbox}
\textbf{જવાબ}:

\textbf{Forbidden Energy Gap:} સોલિડ મટિરિયલમાં એનર્જી રેન્જ જ્યાં કોઈ ઇલેક્ટ્રોન સ્ટેટ્સ અસ્તિત્વમાં નથી, જે વેલેન્સ બેન્ડને કન્ડક્શન બેન્ડથી અલગ કરે છે.

\textbf{Energy Band Diagrams:}

\begin{center}
\begin{tikzpicture}[scale=0.7]
    % Insulator
    \draw (0,0) rectangle (2,1) node[midway] {Valence};
    \draw (0,3) rectangle (2,4) node[midway] {Cond.};
    \node at (1, 2) {Gap $> 5eV$};
    \node[below] at (1, -0.5) {Insulator};
    
    % Semiconductor
    \draw (3,0) rectangle (5,1) node[midway] {Valence};
    \draw (3,2) rectangle (5,3) node[midway] {Cond.};
    \node at (4, 1.5) {$\approx 1eV$};
    \node[below] at (4, -0.5) {Semiconductor};
    
    % Conductor
    \draw (6,0) rectangle (8,1.5) node[midway, below] {Valence};
    \draw (6,1) rectangle (8,2.5) node[midway, above] {Cond.};
    \node at (7, 1.25) {Overlap};
    \node[below] at (7, -0.5) {Conductor};
\end{tikzpicture}
\captionof{figure}{Energy Band Diagrams}
\end{center}

\begin{itemize}
    \item \keyword{Insulator}: મોટો ફોર્બિડન ગેપ ($>5eV$) ઇલેક્ટ્રોન્સને કન્ડક્શન બેન્ડ સુધી પહોંચતા અટકાવે છે
    \item \keyword{Conductor}: ઓવરલેપિંગ બેન્ડ્સ મુક્ત ઇલેક્ટ્રોન હિલચાલને મંજૂરી આપે છે
    \item \keyword{Semiconductor}: નાનો ગેપ ($\approx 1eV$) રૂમ તાપમાને અથવા જ્યારે ઉત્તેજિત થાય ત્યારે કેટલાક ઇલેક્ટ્રોન્સને ક્રોસ કરવાની મંજૂરી આપે છે
\end{itemize}
\end{solutionbox}

\begin{mnemonicbox}
\mnemonic{IBCS: "Insulators Block, Conductors Share, Semiconductors have gap Between"}
\end{mnemonicbox}

\questionmarks{3(b) OR}{4}{Explain the function of Zener diode as a voltage regulator}

\begin{solutionbox}
\textbf{જવાબ}:

\begin{center}
\begin{tikzpicture}
    \draw (0,0) to[battery1, l=$V_{in}$] (0,3) to[R, l=$R_S$] (3,3) -- (5,3);
    \draw (3,3) to[zD, l=$V_Z$] (3,0);
    \draw (5,3) to[R, l=$R_L$] (5,0);
    \draw (0,0) -- (5,0);
    
    \node[right] at (5, 1.5) {$V_{out} = V_Z$};
\end{tikzpicture}
\captionof{figure}{Zener Voltage Regulator}
\end{center}

\textbf{કાર્ય સિદ્ધાંત:}

\begin{itemize}
    \item \keyword{Normal operation}: ઝેનર ડાયોડ રિવર્સ બાયસ્ડ હોય છે અને જ્યારે વોલ્ટેજ બ્રેકડાઉન વોલ્ટેજ સુધી પહોંચે ત્યારે કંડક્ટ કરે છે
    \item \keyword{Voltage regulation}: જ્યારે ઇનપુટ વોલ્ટેજ વધે છે, ત્યારે ઝેનર ડાયોડમાંથી વધુ કરંટ વહે છે, તેની ખાતરી કરે છે કે તેની આરપાર વોલ્ટેજ અચળ રહે
    \item \keyword{Load variation}: જ્યારે લોડ વધુ કરંટ ખેંચે છે, ત્યારે ઝેનરમાંથી ઓછો કરંટ વહે છે, વોલ્ટેજ સ્થિર રાખે છે
    \item \keyword{Series resistor}: કરંટને મર્યાદિત કરે છે અને વધારાનો વોલ્ટેજ ડ્રોપ કરે છે
\end{itemize}

\textbf{સર્કિટ બિહેવિયર:}
\begin{itemize}
    \item $V_{out} = V_z$ (Zener બ્રેકડાઉન વોલ્ટેજ)
    \item $I_z = (V_{in} - V_z)/R - I_L$
\end{itemize}
\end{solutionbox}

\begin{mnemonicbox}
\mnemonic{SERZ: "Series resistor Enables Regulation with Zener"}
\end{mnemonicbox}

\questionmarks{3(c) OR}{7}{Explain V-I char of P-N junction diode and give comparison between P-N junction diode and Zener diode.}

\begin{solutionbox}
\textbf{જવાબ}:

\textbf{P-N Junction Diode ની V-I Characteristics:}

\begin{center}
\begin{tikzpicture}
    \draw[->] (-3,0) -- (3,0) node[right] {$V$};
    \draw[->] (0,-3) -- (0,3) node[above] {$I$};
    
    % Forward
    \draw[thick, blue] (0,0) -- (0.7,0) .. controls (0.8,0.1) and (0.9,1) .. (1,2.5);
    \node at (1.5, 1) {Forward};
    
    % Reverse
    \draw[thick, red] (0,0) -- (-2,0) -- (-2,-2.5);
    \node[below] at (-2,0) {Breakdown};
    \node[left] at (-1, -1) {Reverse};
\end{tikzpicture}
\captionof{figure}{Diode V-I Characteristics}
\end{center}

\textbf{મુખ્ય મુદ્દાઓ:}
\begin{itemize}
    \item \keyword{Forward bias}: ની (knee) વોલ્ટેજ (સિલિકોન માટે $\approx 0.7V$) વટાવ્યા પછી સરળતાથી કંડક્ટ કરે છે
    \item \keyword{Reverse bias}: બ્રેકડાઉન વોલ્ટેજ સુધી ખૂબ ઓછો લીકેજ કરંટ
    \item \keyword{Breakdown region}: ઉચ્ચ રિવર્સ વોલ્ટેજ પર થાય છે, સામાન્ય ડાયોડ્સમાં નુકસાન પહોંચાડે છે
\end{itemize}

\textbf{P-N Junction Diode vs. Zener Diode:}

\begin{center}
\captionof{table}{Comparison P-N vs Zener}
\begin{tabulary}{\linewidth}{|L|L|L|}
\hline
\textbf{Parameter} & \textbf{P-N Junction Diode} & \textbf{Zener Diode} \\ \hline
\textbf{Symbol} & સ્ટાન્ડર્ડ ડાયોડ સિમ્બોલ & Z-સિમ્બોલ ડાયોડ \\ \hline
\textbf{Forward operation} & સરળતાથી કંડક્ટ કરે છે & સામાન્ય ડાયોડ જેવું જ \\ \hline
\textbf{Reverse breakdown} & ઉચ્ચ વોલ્ટેજ પર, નુકસાન કરે છે & નિયંત્રિત, નોન-ડિસ્ટ્રક્ટિવ \\ \hline
\textbf{Doping level} & મધ્યમ & હેવી ડોપ્ડ (Heavily doped) \\ \hline
\textbf{Operating region} & ફોરવર્ડ બાયસ્ડ & રિવર્સ બાયસ્ડ (બ્રેકડાઉન રીજન) \\ \hline
\textbf{Applications} & રેક્ટિફિકેશન, સ્વિચિંગ & વોલ્ટેજ રેગ્યુલેશન, રેફરન્સ \\ \hline
\textbf{Breakdown mechanism} & એવેલાન્ચ (Avalanche) & ઝેનર ઇફેક્ટ અને એવેલાન્ચ \\ \hline
\textbf{Temperature coefficient} & નેગેટિવ & પોઝિટિવ અથવા નેગેટિવ હોઈ શકે \\ \hline
\end{tabulary}
\end{center}
\end{solutionbox}

\begin{mnemonicbox}
\mnemonic{FORD: "Forward Operation for Rectifiers, Diodes; reverse operation for Zeners"}
\end{mnemonicbox}

\questionmarks{4(a)}{3}{Describe working principle of Photodiode.}

\begin{solutionbox}
\textbf{જવાબ}:

\textbf{Photodiode નો કાર્ય સિદ્ધાંત:}

\begin{center}
\begin{tikzpicture}[node distance=2cm]
    \node [gtu block, fill=lightyellow] (light) {Light};
    \node [gtu block, fill=lightpink, right=of light] (pn) {P-N Junction};
    \node [gtu block, fill=lightblue, right=of pn] (eh) {Electron-Hole Pairs};
    \node [gtu block, fill=lightgreen, right=of eh] (curr) {Photocurrent};
    
    \draw [gtu arrow] (light) -- (pn);
    \draw [gtu arrow] (pn) -- (eh);
    \draw [gtu arrow] (eh) -- (curr);
\end{tikzpicture}
\captionof{figure}{Photodiode Flow}
\end{center}

\begin{itemize}
    \item \keyword{Construction}: પારદર્શક વિન્ડો અથવા લેન્સ સાથે P-N જંકશન ડાયોડ
    \item \keyword{Operation}: લાઇટ ડિટેક્શન માટે રિવર્સ બાયસ્ડ ઓપરેશન
    \item \keyword{Photon absorption}: આવતા ફોટોન્સ ડિપ્લેશન રીજનમાં ઇલેક્ટ્રોન-હોલ પેર બનાવે છે
    \item \keyword{Current generation}: ઇલેક્ટ્રિક ફિલ્ડ કેરિયર્સને સંબંધિત ટર્મિનલ્સ તરફ ધકેલે છે, જેનાથી ફોટોકરંટ સર્જાય છે
    \item \keyword{Light sensitivity}: કરંટ પ્રકાશની તીવ્રતાના પ્રમાણમાં હોય છે
\end{itemize}
\end{solutionbox}

\begin{mnemonicbox}
\mnemonic{LIGER: "Light Induces Generation of Electrons in Reverse-bias"}
\end{mnemonicbox}

\questionmarks{4(b)}{4}{Explain the characteristic of Schottky barrier diode.}

\begin{solutionbox}
\textbf{જવાબ}:

\textbf{Schottky Barrier Diode Characteristics:}

\begin{center}
\begin{tikzpicture}
    \draw[->] (-2,0) -- (3,0) node[right] {$V$};
    \draw[->] (0,-2) -- (0,3) node[above] {$I$};
    
    \draw[thick, red] (0,0) -- (0.3,0) .. controls (0.4,0.1) and (0.5,1) .. (0.6,2.5);
    \node[left] at (0.6, 2) {Schottky};
    
    \draw[thick, blue, dashed] (0,0) -- (0.7,0) .. controls (0.8,0.1) and (0.9,1) .. (1,2.5);
    \node[right] at (1, 1.5) {PN Junction};
\end{tikzpicture}
\captionof{figure}{Schottky vs PN Junction}
\end{center}

\begin{itemize}
    \item \keyword{Low forward voltage drop}: સિલિકોન PN જંકશન માટે 0.7V ની સરખામણીમાં 0.2-0.3V
    \item \keyword{Fast switching}: માઇનોરિટી કેરિયર સ્ટોરેજ નથી, ઓછામાં ઓછો રિવર્સ રિકવરી ટાઇમ
    \item \keyword{Construction}: P-N જંકશનને બદલે મેટલ-સેમિકન્ડક્ટર જંકશન
    \item \keyword{No reverse recovery time}: મેજોરિટી કેરિયર ડિવાઇસ (સ્ટોર્ડ ચાર્જ નથી)
    \item \keyword{Applications}: હાઇ-ફ્રિકવન્સી એપ્લિકેશન્સ, પાવર સપ્લાયમાં રેક્ટિફાયર
\end{itemize}
\end{solutionbox}

\begin{mnemonicbox}
\mnemonic{FAST: "Forward voltage low, Allows Switching Timely"}
\end{mnemonicbox}

\questionmarks{4(c)}{7}{Explain working principle of PNP and NPN transistor.}

\begin{solutionbox}
\textbf{જવાબ}:

\textbf{NPN Transistor Structure and Working:}

\begin{center}
\begin{tikzpicture}
    \draw (0,0) rectangle (3,2);
    \draw (1,0) -- (1,2);
    \draw (2,0) -- (2,2);
    \node at (0.5, 1) {E (N)};
    \node at (1.5, 1) {B (P)};
    \node at (2.5, 1) {C (N)};
    
    \draw[->, thick] (0.5, -0.5) -- (0.5, 0); % Emitter connection
    \draw[->, thick] (1.5, -0.5) -- (1.5, 0); 
    \draw[->, thick] (2.5, -0.5) -- (2.5, 0);
    
    \node[below] at (0.5, -0.5) {Emitter};
    \node[below] at (1.5, -0.5) {Base};
    \node[below] at (2.5, -0.5) {Collector};
    
    \draw[->, dashed] (0.5, 0.5) -- (2.5, 0.5);
    \node[above] at (1.5, 2) {Electrons Flow};
\end{tikzpicture}
\captionof{figure}{NPN Structure}
\end{center}

\begin{itemize}
    \item \keyword{Biasing}: ઇમિટર-બેઝ જંકશન ફોરવર્ડ બાયસ્ડ, કલેક્ટર-બેઝ જંકશન રિવર્સ બાયસ્ડ
    \item \keyword{Current flow}: ઇલેક્ટ્રોન્સ ઇમિટરથી કલેક્ટરમાં પાતળા બેઝ વિસ્તાર દ્વારા વહે છે
    \item \keyword{Amplification principle}: નાનો બેઝ કરંટ મોટા કલેક્ટર કરંટને નિયંત્રિત કરે છે
    \item \keyword{Current relationship}: $I_E = I_B + I_C$
    \item \keyword{Majority carriers}: ઇલેક્ટ્રોન્સ
\end{itemize}

\textbf{PNP Transistor Structure and Working:}

\begin{itemize}
    \item \keyword{Biasing}: ઇમિટર-બેઝ જંકશન ફોરવર્ડ બાયસ્ડ, કલેક્ટર-બેઝ જંકશન રિવર્સ બાયસ્ડ
    \item \keyword{Current flow}: હોલ્સ ઇમિટરથી કલેક્ટરમાં પાતળા બેઝ વિસ્તાર દ્વારા વહે છે
    \item \keyword{Amplification principle}: નાનો બેઝ કરંટ મોટા કલેક્ટર કરંટને નિયંત્રિત કરે છે
    \item \keyword{Current relationship}: $I_E = I_B + I_C$
    \item \keyword{Majority carriers}: હોલ્સ
    \item \keyword{Current direction}: NPN ની વિરુદ્ધ (ઇમિટરથી કલેક્ટર સુધી કન્વેન્શનલ કરંટ)
\end{itemize}
\end{solutionbox}

\begin{mnemonicbox}
\mnemonic{NPNP: "Negative carriers in NPN, Positive carriers in PNP"}
\end{mnemonicbox}

\questionmarks{4(a) OR}{3}{Describe working principle of LED.}

\begin{solutionbox}
\textbf{જવાબ}:

\textbf{LED નો કાર્ય સિદ્ધાંત:}

\begin{center}
\begin{tikzpicture}[node distance=2cm]
    \node [gtu block, fill=lightblue] (bias) {Forward Bias};
    \node [gtu block, fill=lightpink, right=of bias] (recomb) {e-h Recombination};
    \node [gtu block, fill=lightyellow, right=of recomb] (energy) {Energy Release};
    \node [gtu block, fill=lightgreen, right=of energy] (light) {Light};
    
    \draw [gtu arrow] (bias) -- (recomb);
    \draw [gtu arrow] (recomb) -- (energy);
    \draw [gtu arrow] (energy) -- (light);
\end{tikzpicture}
\captionof{figure}{LED Principle}
\end{center}

\begin{itemize}
    \item \keyword{Construction}: ડાયરેક્ટ બેન્ડગેપ સેમિકન્ડક્ટર મટિરિયલ્સથી બનેલું P-N જંકશન
    \item \keyword{Forward biasing}: n-રીજનમાંથી ઇલેક્ટ્રોન્સ અને p-રીજનમાંથી હોલ્સ જંકશન પર રિકમ્બાઇન રિપ્લેસ થાય છે
    \item \keyword{Recombination}: ઇલેક્ટ્રોન્સ કન્ડક્શન બેન્ડમાંથી વેલેન્સ બેન્ડમાં આવે છે
    \item \keyword{Energy emission}: રિકમ્બિનેશન દરમિયાન મુક્ત થતી એનર્જી ફોટોન્સ (પ્રકાશ) ફેલાવે છે
\end{itemize}
\end{solutionbox}

\begin{mnemonicbox}
\mnemonic{REBEL: "Recombination of Electrons and holes By Energetic Light emission"}
\end{mnemonicbox}

\questionmarks{4(b) OR}{4}{Explain function of transistor as switch in cut off and application of saturation region.}

\begin{solutionbox}
\textbf{જવાબ}:

\textbf{Transistor as a Switch:}

\begin{center}
\begin{tikzpicture}
    \draw (0,0) node[npn] (Q) {};
    \draw (Q.E) node[ground] {};
    \draw (Q.C) to[R, l=$R_C$] (0,2) -- (0,2.5) node[above] {$V_{CC}$};
    \draw (Q.B) to[R, l=$R_1$] (-2,0) node[left] {Input};
    \draw (Q.C) -- (1,0) node[right] {Output};
\end{tikzpicture}
\captionof{figure}{Transistor Switch Circuit}
\end{center}

\textbf{Cut-off Region (Switch OFF):}
\begin{itemize}
    \item \keyword{Base voltage}: 0.7V થી નીચે (સિલિકોન માટે)
    \item \keyword{Base current}: લગભગ શૂન્ય
    \item \keyword{Collector current}: લગભગ શૂન્ય
    \item \keyword{Collector-emitter voltage}: સપ્લાય વોલ્ટેજ જેટલો
    \item \keyword{Applications}: લોજિક ગેટ્સ, ડિજિટલ સર્કિટ્સ, રિલે ડ્રાઇવર્સ
\end{itemize}

\textbf{Saturation Region (Switch ON):}
\begin{itemize}
    \item \keyword{Base voltage}: 0.7V થી ઘણું ઉપર
    \item \keyword{Base current}: લઘુત્તમ $V_{CE}$ સુનિશ્ચિત કરવા માટે પૂરતો
    \item \keyword{Collector current}: મહત્તમ (કલેક્ટર રેઝિસ્ટર દ્વારા મર્યાદિત)
    \item \keyword{Collector-emitter voltage}: ખૂબ ઓછો (0.2V - 0.3V)
    \item \keyword{Applications}: ડિજિટલ સ્વીચ, મોટર ડ્રાઇવર, LED ડ્રાઇવર
\end{itemize}
\end{solutionbox}

\begin{mnemonicbox}
\mnemonic{COSI: "Cutoff Opens Switch, Input saturates to close"}
\end{mnemonicbox}

\questionmarks{4(c) OR}{7}{Explain common emitter (CE) configuration of Transistor. Derive relation between α and β for transistor amplifier.}

\begin{solutionbox}
\textbf{જવાબ}:

\textbf{Common Emitter Configuration:}

\begin{center}
\begin{tikzpicture}
    \draw (0,0) node[npn] (Q) {};
    \node [right=0.5cm of Q] {CE};
    \draw (Q.E) node[ground] {};
    \draw (Q.B) to[short,-o] (-1,0) node[left] {Input};
    \draw (Q.C) to[short,-o] (1,0) node[right] {Output};
    \node[below] at (0,-1) {Emitter Grounded};
\end{tikzpicture}
\captionof{figure}{CE Configuration}
\end{center}

\textbf{લાક્ષણિકતાઓ:}
\begin{itemize}
    \item \keyword{Input/Output}: Base / Collector
    \item \keyword{Gains}: High Current ($\beta$), High Voltage
    \item \keyword{Impedance}: Medium Input, High Output
\end{itemize}

\textbf{$\alpha$ અને $\beta$ વચ્ચેનો સંબંધ:}

વ્યાખ્યા પ્રમાણે:
\begin{itemize}
    \item $\alpha = I_C/I_E$
    \item $\beta = I_C/I_B$
\end{itemize}

કિરચોફના કરંટ લૉ પરથી:
\[ I_E = I_B + I_C \]

$I_E$ વડે ભાગતા:
\[ 1 = \frac{I_B}{I_E} + \frac{I_C}{I_E} = \frac{I_B}{I_E} + \alpha \]
\[ \frac{I_B}{I_E} = 1 - \alpha \]

હવે,
\[ \beta = \frac{I_C}{I_B} = \frac{I_C/I_E}{I_B/I_E} = \frac{\alpha}{1-\alpha} \]
\end{solutionbox}

\begin{mnemonicbox}
\mnemonic{BEAR: "Beta Equals Alpha divided by (1-alpha) Relation"}
\end{mnemonicbox}

\questionmarks{5(a)}{3}{What do you mean by E-waste? What are the different methods of E-waste disposal?}

\begin{solutionbox}
\textbf{જવાબ}:

\textbf{E-waste (ઇલેક્ટ્રોનિક વેસ્ટ)}: ત્યજાયેલા ઇલેક્ટ્રોનિક ડિવાઇસ અને કમ્પોનન્ટ્સ જે તેમના જીવનકાળનાં અંતે પહોંચ્યા છે અથવા હવે ઉપયોગી નથી.

\textbf{ઇ-વેસ્ટ નિકાલની પદ્ધતિઓ:}

\begin{center}
\captionof{table}{Disposal Methods}
\begin{tabulary}{\linewidth}{|L|L|}
\hline
\textbf{Disposal Method} & \textbf{Description} \\ \hline
\textbf{Recycling} & મૂલ્યવાન સામગ્રી જેમ કે ધાતુઓ, પ્લાસ્ટિકને પુન:ઉપયોગ માટે અલગ કરવી \\ \hline
\textbf{Landfilling} & નિયુક્ત લેન્ડફિલ્સમાં નિકાલ (ભલામણ કરાતી નથી) \\ \hline
\textbf{Incineration} & ઉચ્ચ તાપમાને કચરાનું દહન (ઝેરી ઉત્સર્જન બનાવે છે) \\ \hline
\textbf{Reuse/Refurbishment} & વિસ્તારિત ઉપયોગ માટે રિપેરિંગ અને અપગ્રેડિંગ \\ \hline
\textbf{Extended Producer Responsibility} & ઉત્પાદકો પાછા લે અને નિકાલ સંભાળે છે \\ \hline
\end{tabulary}
\end{center}
\end{solutionbox}

\begin{mnemonicbox}
\mnemonic{RIPER: "Recycling Is Preferable to Environmentally-harmful Remedies"}
\end{mnemonicbox}

\questionmarks{5(b)}{4}{Explain methods of handling electronic waste with examples.}

\begin{solutionbox}
\textbf{જવાબ}:

\textbf{ઇલેક્ટ્રોનિક વેસ્ટ હેન્ડલિંગની પદ્ધતિઓ:}

\begin{center}
\begin{tikzpicture}[node distance=1.5cm]
    \node [gtu block] (coll) {Collection};
    \node [gtu block, right=of coll] (sort) {Sorting};
    \node [gtu block, right=of sort] (dism) {Dismantling};
    \node [gtu block, below=of dism] (recov) {Recovery};
    \node [gtu block, left=of recov] (disp) {Disposal};
    
    \draw [gtu arrow] (coll) -- (sort);
    \draw [gtu arrow] (sort) -- (dism);
    \draw [gtu arrow] (dism) -- (recov);
    \draw [gtu arrow] (recov) -- (disp);
\end{tikzpicture}
\captionof{figure}{E-waste Handling Flow}
\end{center}

\begin{itemize}
    \item \keyword{Collection and Segregation}: સમર્પિત ડબ્બાઓ, મિશ્રણ અટકાવે છે (દા.ત., ઇ-વેસ્ટ બિન્સ).
    \item \keyword{Dismantling and Resource Recovery}: PCBs માંથી સોનું/કોપર રિકવર કરવું.
    \item \keyword{Refurbishment and Reuse}: જૂના કમ્પ્યુટર્સની મરામત.
    \item \keyword{Proper Disposal}: જોખમી ભાગો માટે વિશેષ ટ્રીટમેન્ટ (મર્ક્યુરી).
\end{itemize}
\end{solutionbox}

\begin{mnemonicbox}
\mnemonic{CREED: "Collect, Recover, Extract, Extend, Dispose safely"}
\end{mnemonicbox}

\questionmarks{5(c)}{7}{What is ripple factor? Derive the equation of the ripple factor for rectifier.}

\begin{solutionbox}
\textbf{જવાબ}:

\textbf{Ripple Factor}: આઉટપુટમાં AC કમ્પોનન્ટના RMS મૂલ્ય અને DC કમ્પોનન્ટનો ગુણોત્તર ($\gamma = V_{AC}/V_{DC}$).

\textbf{હાફ વેવ રેક્ટિફાયર માટે તારવણી:}

ધારો કે $v = V_m\sin\omega t$.

\textbf{Step 1}: DC કમ્પોનન્ટ (એવરેજ વેલ્યુ) શોધો
\[ V_{DC} = \frac{1}{2\pi} \int_{0}^{\pi} V_m\sin\omega t \, d(\omega t) = \frac{V_m}{\pi} \]

\textbf{Step 2}: RMS વેલ્યુ શોધો
\[ V_{RMS} = \sqrt{\frac{1}{2\pi} \int_{0}^{\pi} V_m^2\sin^2\omega t \, d(\omega t)} = \frac{V_m}{2} \]

\textbf{Step 3}: AC કમ્પોનન્ટ શોધો
\[ V_{AC} = \sqrt{V_{RMS}^2 - V_{DC}^2} = \sqrt{\left(\frac{V_m}{2}\right)^2 - \left(\frac{V_m}{\pi}\right)^2} \]

\textbf{Step 4}: રિપલ ફેક્ટર ગણો
\[ \gamma = \frac{V_{AC}}{V_{DC}} = \frac{\sqrt{V_{RMS}^2 - V_{DC}^2}}{V_{DC}} = \sqrt{\left(\frac{V_{RMS}}{V_{DC}}\right)^2 - 1} \]
\[ \gamma = \sqrt{\left(\frac{V_m/2}{V_m/\pi}\right)^2 - 1} = \sqrt{\left(\frac{\pi}{2}\right)^2 - 1} = \sqrt{1.57^2 - 1} = 1.21 \]

ફુલ વેવ રેક્ટિફાયર માટે, $\gamma = 0.48$.
\end{solutionbox}

\begin{mnemonicbox}
\mnemonic{ROAD: "Ripple is Output's AC Divided by DC component"}
\end{mnemonicbox}

\questionmarks{5(a) OR}{3}{Which are the toxic substances present in e-waste?}

\begin{solutionbox}
\textbf{જવાબ}:

\begin{center}
\captionof{table}{Toxic Substances in E-waste}
\begin{tabulary}{\linewidth}{|L|L|L|}
\hline
\textbf{Toxic Substance} & \textbf{Source} & \textbf{Impact} \\ \hline
\textbf{Lead (Pb)} & Solder, CRT, બેટરીઓ & ન્યુરોલોજીકલ નુકસાન \\ \hline
\textbf{Mercury (Hg)} & સ્વિચ, બેકલાઇટ્સ & કિડનીને નુકસાન \\ \hline
\textbf{Cadmium (Cd)} & બેટરીઓ, PCBs & હાડકાના રોગો \\ \hline
\textbf{Flame Retardants} & પ્લાસ્ટિક & એન્ડોક્રાઇન ડિસ્રપ્શન \\ \hline
\textbf{Beryllium (Be)} & કનેક્ટર્સ & ફેફસાના રોગ \\ \hline
\end{tabulary}
\end{center}
\end{solutionbox}

\begin{mnemonicbox}
\mnemonic{LMBCHB: "Lead, Mercury, and Beryllium Cause Harmful Bodily effects"}
\end{mnemonicbox}

\questionmarks{5(b) OR}{4}{Write important parameters for selecting the right transistor for your application and explain any two.}

\begin{solutionbox}
\textbf{જવાબ}:

\textbf{પસંદગીના મહત્વપૂર્ણ પરિમાણો:}
\begin{itemize}
    \item મહત્તમ કલેક્ટર કરંટ ($I_C$)
    \item મહત્તમ કલેક્ટર-ઇમિટર વોલ્ટેજ ($V_{CEO}$)
    \item કરંટ ગેઇન ($h_{FE}$ or $\beta$)
    \item પાવર ડિસિપેશન ($P_{tot}$)
\end{itemize}

\textbf{મહત્તમ કલેક્ટર કરંટ ($I_C$):}
\begin{itemize}
    \item નુકસાન વિના કલેક્ટર મારફતે વહી શકે તેવો મહત્તમ કરંટ.
    \item એપ્લિકેશનની પીક જરૂરિયાત કરતાં વધુ હોવો જોઈએ.
\end{itemize}

\textbf{કરંટ ગેઇન ($\beta$):}
\begin{itemize}
    \item કલેક્ટર કરંટનો બેઝ કરંટ સાથેનો ગુણોત્તર.
    \item એમ્પલિફિકેશન ક્ષમતા નક્કી કરે છે; બેઝ ડ્રાઇવ ઘટાડવા માટે સ્વિચિંગ માટે ઉચ્ચ ગેઇન જરૂરી છે.
\end{itemize}
\end{solutionbox}

\begin{mnemonicbox}
\mnemonic{GIVE: "Gain and Ic are Very Essential parameters"}
\end{mnemonicbox}

\questionmarks{5(c) OR}{7}{What is rectifier efficiency? Find out efficiency of the full wave rectifier.}

\begin{solutionbox}
\textbf{જવાબ}:

\textbf{Rectifier Efficiency ($\eta$)}: DC આઉટપુટ પાવરનો AC ઇનપુટ પાવર સાથેનો ગુણોત્તર ($\eta = P_{DC}/P_{AC} \times 100\%$).

\textbf{ફુલ વેવ રેક્ટિફાયર માટે તારવણી:}

\textbf{Step 1}: DC આઉટપુટ પાવર ગણો
\[ V_{DC} = \frac{2V_m}{\pi}, \quad I_{DC} = \frac{V_{DC}}{R_L} \]
\[ P_{DC} = I_{DC}^2 R_L = \frac{4V_m^2}{\pi^2 R_L} \]

\textbf{Step 2}: AC ઇનપુટ પાવર ગણો
\[ V_{RMS} = \frac{V_m}{\sqrt{2}}, \quad I_{RMS} = \frac{V_{RMS}}{R_L} \]
\[ P_{AC} = I_{RMS}^2 R_L = \frac{V_m^2}{2R_L} \]

\textbf{Step 3}: કાર્યક્ષમતા ગણો
\[ \eta = \frac{P_{DC}}{P_{AC}} = \frac{4V_m^2 / (\pi^2 R_L)}{V_m^2 / (2R_L)} = \frac{8}{\pi^2} \]
\[ \eta = \frac{8}{9.87} \approx 0.812 = 81.2\% \]

\textbf{સરખામણી:}
\begin{itemize}
    \item Half Wave: 40.6\%
    \item Full Wave: 81.2\%
\end{itemize}
\end{solutionbox}

\begin{mnemonicbox}
\mnemonic{PIDE: "Power Input Determines Efficiency"}
\end{mnemonicbox}

\end{document}
