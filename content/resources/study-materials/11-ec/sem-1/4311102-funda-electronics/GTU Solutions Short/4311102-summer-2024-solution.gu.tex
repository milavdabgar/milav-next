\documentclass{article}

% content/resources/templates/preamble.tex
\usepackage[margin=0.6in]{geometry}
\author{Milav Dabgar}
\usepackage{amsmath,amssymb,amsthm}
\usepackage{booktabs}
\usepackage{multirow}
\usepackage{xcolor}
\usepackage{tcolorbox}
\tcbuselibrary{breakable,skins}
\usepackage[colorlinks=true,linkcolor=blue]{hyperref}
\usepackage{titlesec}
\usepackage{enumitem}
\usepackage{tikz}
\usepackage{pgfplots}
\usepackage{circuitikz}
\usepackage[version=4]{mhchem}
\usepackage{longtable}
\usepackage{array}
\usepackage{float}
\usepackage{caption}
\usepackage{listings}

\lstset{
  basicstyle=\small\ttfamily,
  breaklines=true,
  breakatwhitespace=false,
  postbreak=\mbox{\textcolor{red}{$\hookrightarrow$}\space},
  float=false,
  numbers=left,
  numberstyle=\tiny\color{gray},
  numbersep=10pt,
  xleftmargin=2em,
  keywordstyle=\color{blue},
  commentstyle=\color{green!60!black},
  stringstyle=\color{purple},
  backgroundcolor=\color{gray!5},
  showstringspaces=false,
  tabsize=2,
  captionpos=b,
  keepspaces=true,
  columns=flexible
}

\pgfplotsset{compat=1.18}
\usetikzlibrary{shapes,arrows,positioning,calc,patterns,decorations.pathmorphing,decorations.markings,arrows.meta}

% Color scheme
\definecolor{headcolor}{RGB}{0,102,204}
\definecolor{keycolor}{RGB}{220,20,60}
\definecolor{solutioncolor}{RGB}{34,139,34}
\definecolor{mnemoniccolor}{RGB}{148,0,211}
\definecolor{codecolor}{RGB}{0,0,100}

% Spacing
\setlength{\parskip}{3pt}
\setlist[itemize]{nosep}
\setlist[enumerate]{nosep}

% Title formatting
\titleformat{\section}{\Large\bfseries\color{headcolor}}{\thesection}{1em}{}
\titleformat{\subsection}{\large\bfseries\color{headcolor}}{\thesubsection}{1em}{}

% Pandoc tightlist compatibility
\providecommand{\tightlist}{%
  \setlength{\itemsep}{0pt}\setlength{\parskip}{0pt}}

% Pandoc longtable compatibility
\newcounter{none}
\def\thenone{}


% content/resources/templates/gujarati-boxes.tex
\usepackage{fontspec}
\usepackage{polyglossia}

% Set Gujarati as main language (document is primarily in Gujarati)
% Note: gloss-gujarati.ldf doesn't exist in polyglossia, but it will use hyphenation patterns
\setdefaultlanguage{gujarati}
\setotherlanguage{english}

% Configure Gujarati font properly
% Use Language=Default to prevent polyglossia from trying to add language-specific features
% that don't exist for Gujarati, which causes "empty feature" warnings
\newfontfamily\gujaratifont[Script=Gujarati,AutoFakeBold=2.5,AutoFakeSlant=0.3]{Noto Sans Gujarati}
\setmainfont[Script=Gujarati,AutoFakeBold=2.5,AutoFakeSlant=0.3]{Noto Sans Gujarati}
% Use Noto Sans Gujarati for monospace to support Gujarati in text
\setmonofont[Scale=0.9]{Noto Sans Gujarati}

% Configure English to use the same font
\newfontfamily\englishfont[Script=Gujarati,AutoFakeBold=2.5,AutoFakeSlant=0.3]{Noto Sans Gujarati}

% Translations for polyglossia
\gappto\captionsgujarati{
  \renewcommand{\tablename}{કોષ્ટક}
  \renewcommand{\figurename}{આકૃતિ}
}

% Helper for TikZ nodes to ensure Gujarati font
\newcommand{\gu}[1]{{\gujaratifont #1}}

% Custom environments
\newtcolorbox{solutionbox}{
    breakable,
    enhanced,
    colback=solutioncolor!5!white,
    colframe=solutioncolor!75!black,
    fonttitle=\bfseries,
    title=જવાબ
}

\newtcolorbox{solutionboxnobreak}{
 colback=solutioncolor!5!white,
 colframe=solutioncolor!75!black,
 fonttitle=\bfseries,
 title=જવાબ
}

\newtcolorbox{keyformula}{
 breakable,
 enhanced,
 colback=keycolor!5!white,
 colframe=keycolor!75!black,
 fonttitle=\bfseries,
 title=રાસાયણિક સમીકરણ/સૂત્ર
}

\newtcolorbox{mnemonicbox}{
 breakable,
 enhanced,
 colback=mnemoniccolor!5!white,
 colframe=mnemoniccolor!75!black,
 fonttitle=\bfseries,
 title=મેમરી ટ્રીક
}


% Custom commands for GTU solutions
% This file defines semantic commands for consistent formatting

% Question command with automatic formatting
\newcommand{\question}[2]{%
  \section*{Question #1}%
  \textbf{#2}%
}

% OR question variant
\newcommand{\questionor}[2]{%
  \section*{Question #1 OR}%
  \textbf{#2}%
}

% Proper table environment with caption
\newenvironment{answertable}[1]{%
  \begin{table}[htbp]
  \centering
  \caption{#1}
}{%
  \end{table}
}

% Proper figure environment for diagrams
\newenvironment{answerdiagram}[1]{%
  \begin{figure}[htbp]
  \centering
  \caption{#1}
}{%
  \end{figure}
}

% Semantic markup for key terms
\newcommand{\keyword}[1]{\textbf{#1}}
\newcommand{\code}[1]{\texttt{#1}}
\newcommand{\classname}[1]{\texttt{#1}}
\newcommand{\methodname}[1]{\texttt{#1}}

% Proper quotation marks
\newcommand{\mnemonic}[1]{``#1''}


\title{ઇલેક્ટ્રોનિક્સના મૂળભૂત સિદ્ધાંતો (4311102) - ઉનાળુ 2024 સોલ્યુશન}
\date{જૂન 21, 2024}


% \usetikzlibrary{circuits.ee.IEC} removed

\newcommand{\mysolutionbox}[3]{%
    \noindent\textbf{#1: #2}%
    \begin{solutionbox}
    #3
    \end{solutionbox}%
    \vspace{0.5em}
}

\begin{document}
\maketitle

\questionmarks{પ્રશ્ન 1}{14}{}
\textbf{દસમાંથી કોઈપણ સાત પ્રશ્નોના જવાબ આપો.}

\mysolutionbox{પ્રશ્ન 1(1)}{રેઝીસ્ટરની વ્યાખ્યા આપો અને તેનો એકમ જણાવો.}{
    રેઝીસ્ટર એ એક ઇલેક્ટ્રોનિક ઘટક છે જે વિદ્યુત પ્રવાહના પ્રવાહનો વિરોધ કરે છે. તેનો એકમ \keyword{Ohm} ($\Omega$) છે.

    \captionof{table}{રેઝીસ્ટરના ગુણધર્મો}
    \begin{tabulary}{\linewidth}{|L|L|}
    \hline
    \textbf{ગુણધર્મ} & \textbf{વર્ણન} \\
    \hline
    સિમ્બોલ & {\begin{tikzpicture}[baseline=-0.3em] \draw (0,0) to[R] (1,0); \end{tikzpicture}} \\
    \hline
    એકમ & ઓહમ ($\Omega$) \\
    \hline
    કાર્ય & પ્રવાહને મર્યાદિત કરે છે \\
    \hline
    \end{tabulary}

    \begin{mnemonicbox}
    \mnemonic{રેઝીસ્ટર્સ વિરોધ કરે પ્રવાહ (ROP)}
    \end{mnemonicbox}
}

\mysolutionbox{પ્રશ્ન 1(2)}{એક્ટીવ અને પેસીવ કમ્પોનન્ટના બે-બે ઉદાહરણ આપો.}{
    \captionof{table}{ઇલેક્ટ્રોનિક ઘટકોનું વર્ગીકરણ}
    \begin{tabulary}{\linewidth}{|L|L|}
    \hline
    \textbf{એક્ટીવ કમ્પોનન્ટ્સ} & \textbf{પેસીવ કમ્પોનન્ટ્સ} \\
    \hline
    1. ટ્રાન્ઝિસ્ટર & 1. રેઝીસ્ટર \\
    2. ડાયોડ & 2. કેપેસિટર \\
    \hline
    \end{tabulary}

    \begin{mnemonicbox}
    \mnemonic{TARD - Transistors And Resistors Differ}
    \end{mnemonicbox}
}

\mysolutionbox{પ્રશ્ન 1(3)}{કોઈપણ બે અર્ધવાહક ઉપકરણોના સિમ્બોલ દોરો.}{
    \begin{answerdiagram}{અર્ધવાહક ઉપકરણ સિમ્બોલ}
    \begin{tikzpicture}[, font=\sffamily]
        % Diode
        \draw (0,0) node[anchor=east] {Anode} to[diode] (2,0) node[anchor=west] {Cathode};
        \node at (1, -0.5) {PN Junction Diode};

        % NPN Transistor
        \begin{scope}[xshift=5cm, yshift=0cm]
            \draw (0,0) node[npn] (t1) {};
            \node[left] at (t1.B) {B};
            \node[above] at (t1.C) {C};
            \node[below] at (t1.E) {E};
            \node at (0, -1) {NPN Transistor};
        \end{scope}
    \end{tikzpicture}
    \end{answerdiagram}

    \begin{mnemonicbox}
    \mnemonic{ડાયોડ દિશા આપે, ટ્રાન્ઝિસ્ટર ટ્રાન્સફર કરે}
    \end{mnemonicbox}
}

\mysolutionbox{પ્રશ્ન 1(4)}{ઈન્ટ્રીસીક અને એક્સટ્રીસીક અર્ધવાહક વચ્ચેનો તફાવત લખો.}{
    \captionof{table}{ઈન્ટ્રીસીક વિરુદ્ધ એક્સટ્રીસીક અર્ધવાહક}
    \begin{tabulary}{\linewidth}{|L|L|}
    \hline
    \textbf{ઈન્ટ્રીસીક} & \textbf{એક્સટ્રીસીક} \\
    \hline
    અશુદ્ધિઓ વિનાના શુદ્ધ અર્ધવાહક & અશુદ્ધિઓ ઉમેરેલા અર્ધવાહક \\
    \hline
    હોલ્સ અને ઇલેક્ટ્રોન્સની સંખ્યા સમાન & હોલ્સ અને ઇલેક્ટ્રોન્સની સંખ્યા અસમાન \\
    \hline
    ઉદાહરણ: શુદ્ધ સિલિકોન, જર્મેનિયમ & ઉદાહરણ: ફોસ્ફરસ સાથે ડોપ કરેલ સિલિકોન \\
    \hline
    \end{tabulary}

    \begin{mnemonicbox}
    \mnemonic{શુદ્ધ ઈન, ડોપ્ડ એક્સ}
    \end{mnemonicbox}
}

\mysolutionbox{પ્રશ્ન 1(5)}{LED નું આખું નામ \_\_\_\_\_\_\_\_\_\_\_\_\_\_\_\_\_.}{
    LED નું આખું નામ \textbf{Light Emitting Diode} છે.

    \begin{answerdiagram}{LED સિમ્બોલ}
    \begin{tikzpicture}[]
        \draw (0,0) to[leDo] (2,0);
        \node at (1, -0.5) {Light Emitting Diode};
    \end{tikzpicture}
    \end{answerdiagram}

    \begin{mnemonicbox}
    \mnemonic{પ્રકાશ ઉત્સર્જિત ડાયોડ (LED)}
    \end{mnemonicbox}
}

\mysolutionbox{પ્રશ્ન 1(6)}{ફોટો ડાયોડના બે ઉપયોગો જણાવો.}{
    \captionof{table}{ફોટો-ડાયોડના ઉપયોગો}
    \begin{tabulary}{\linewidth}{|L|L|}
    \hline
    \textbf{ઉપયોગ} & \textbf{કેવી રીતે કામ કરે છે} \\
    \hline
    પ્રકાશ સેન્સર & પ્રકાશને વિદ્યુત પ્રવાહમાં રૂપાંતરિત કરે છે \\
    \hline
    ઓપ્ટિકલ કમ્યુનિકેશન & ફાઇબર ઓપ્ટિક્સમાં ઓપ્ટિકલ સિગ્નલ્સને શોધે છે \\
    \hline
    \end{tabulary}

    \begin{mnemonicbox}
    \mnemonic{પ્રકાશ સેન્સિંગ કમ્યુનિકેશન (LSC)}
    \end{mnemonicbox}
}

\mysolutionbox{પ્રશ્ન 1(7)}{ટ્રાન્ઝિસ્ટરના પ્રકારોની યાદી બનાવો અને તેમના પ્રતીકો દોરો.}{
    \textbf{ટ્રાન્ઝિસ્ટરના પ્રકારો:}
    \begin{enumerate}
        \item NPN ટ્રાન્ઝિસ્ટર
        \item PNP ટ્રાન્ઝિસ્ટર
    \end{enumerate}

    \begin{answerdiagram}{ટ્રાન્ઝિસ્ટર સિમ્બોલ}
    \begin{tikzpicture}[, font=\sffamily]
        % NPN
        \begin{scope}[xshift=0cm]
            \draw (0,0) node[npn, xscale=1.5, yscale=1.5] (npn) {};
            \node[left] at (npn.B) {B};
            \node[above] at (npn.C) {C};
            \node[below] at (npn.E) {E};
            \node at (0, -1.5) {NPN};
        \end{scope}

        % PNP
        \begin{scope}[xshift=4cm]
            \draw (0,0) node[pnp, xscale=1.5, yscale=1.5] (pnp) {};
            \node[left] at (pnp.B) {B};
            \node[above] at (pnp.C) {C};
            \node[below] at (pnp.E) {E};
            \node at (0, -1.5) {PNP};
        \end{scope}
    \end{tikzpicture}
    \end{answerdiagram}

    \begin{mnemonicbox}
    \mnemonic{Not Pointing iN, Pointing outP}
    \end{mnemonicbox}
}

\mysolutionbox{પ્રશ્ન 1(8)}{જર્મેનિયમ અને સિલિકોન ડાયોડના ફોરવર્ડ વોલ્ટેજ ડ્રોપનું મૂલ્ય આપો.}{
    \captionof{table}{ફોરવર્ડ વોલ્ટેજ ડ્રોપ મૂલ્યો}
    \begin{tabulary}{\linewidth}{|L|L|}
    \hline
    \textbf{ડાયોડનો પ્રકાર} & \textbf{ફોરવર્ડ વોલ્ટેજ ડ્રોપ} \\
    \hline
    જર્મેનિયમ & 0.3V \\
    \hline
    સિલિકોન & 0.7V \\
    \hline
    \end{tabulary}

    \begin{mnemonicbox}
    \mnemonic{જર્મેનિયમ ત્રણ, સિલિકોન સાત (0.3V, 0.7V)}
    \end{mnemonicbox}
}

\mysolutionbox{પ્રશ્ન 1(9)}{\_\_\_\_\_\_\_\_\_\_\_\_\_\_\_\_\_ ડાયોડનો ઉપયોગ લાઇટ ડિટેક્ટર તરીકે થઈ શકે છે.}{
    \textbf{ફોટોડાયોડ}નો ઉપયોગ લાઇટ ડિટેક્ટર તરીકે થઈ શકે છે.

    \begin{answerdiagram}{ફોટોડાયોડ સિમ્બોલ}
    \begin{tikzpicture}[]
        \draw (0,0) to[photodiode] (2,0);
        \node at (1, -0.5) {Photodiode};
    \end{tikzpicture}
    \end{answerdiagram}

    \begin{mnemonicbox}
    \mnemonic{ફોટો શોધે પ્રકાશ (PDL)}
    \end{mnemonicbox}
}

\mysolutionbox{પ્રશ્ન 1(10)}{કોઈલના Q-factor ની વ્યાખ્યા લખો.}{
    \keyword{Q-factor} (ક્વોલિટી ફેક્ટર) એ કોઈલના ઇન્ડક્ટિવ રિએક્ટન્સનો તેના રેઝિસ્ટન્સ સાથેનો ગુણોત્તર છે, જે સૂચવે છે કે તે કેટલી કાર્યક્ષમતાથી ઊર્જા સંગ્રહિત કરે છે.

    \captionof{table}{Q-Factor}
    \begin{tabulary}{\linewidth}{|L|L|}
    \hline
    \textbf{પેરામીટર} & \textbf{વર્ણન} \\
    \hline
    સૂત્ર & $Q = \frac{X_L}{R}$ \\
    \hline
    ઉચ્ચ Q & સારી ગુણવત્તા, ઓછો ઊર્જા વ્યય \\
    \hline
    નીચો Q & નબળી ગુણવત્તા, વધુ ઊર્જા વ્યય \\
    \hline
    \end{tabulary}

    \begin{mnemonicbox}
    \mnemonic{ગુણવત્તા બરાબર રિએક્ટન્સ વિભાજિત પ્રતિરોધ (QRR)}
    \end{mnemonicbox}
}

\questionmarks{પ્રશ્ન 2(અ)}{3}{}
\mysolutionbox{પ્રશ્ન 2(અ)}{રેઝીસ્ટરનો કલર કોડીંગ સમજાવો.}{
    રેઝીસ્ટર કલર કોડિંગ રંગીન પટ્ટીઓનો ઉપયોગ કરે છે જે પ્રતિરોધ મૂલ્ય અને ટોલરન્સ દર્શાવે છે.

    \captionof{table}{રેઝીસ્ટર કલર કોડ}
    \begin{tabulary}{\linewidth}{|L|L|L|}
    \hline
    \textbf{રંગ} & \textbf{અંક} & \textbf{ગુણાંક} \\
    \hline
    કાળો & 0 & $10^0$ \\
    \hline
    બ્રાઉન & 1 & $10^1$ \\
    \hline
    લાલ & 2 & $10^2$ \\
    \hline
    નારંગી & 3 & $10^3$ \\
    \hline
    પીળો & 4 & $10^4$ \\
    \hline
    લીલો & 5 & $10^5$ \\
    \hline
    વાદળી & 6 & $10^6$ \\
    \hline
    જાંબલી & 7 & $10^7$ \\
    \hline
    ગ્રે & 8 & $10^8$ \\
    \hline
    સફેદ & 9 & $10^9$ \\
    \hline
    \end{tabulary}

    \vspace{1em}
    \textbf{4-બેન્ડ રેઝિસ્ટર માટે:}
    \begin{itemize}
        \item પ્રથમ બેન્ડ: પ્રથમ અંક
        \item બીજી બેન્ડ: બીજો અંક
        \item ત્રીજી બેન્ડ: ગુણાંક
        \item ચોથી બેન્ડ: ટોલરન્સ
    \end{itemize}

    \begin{mnemonicbox}
    \mnemonic{Bad Boys Race Our Young Girls But Violet Generally Wins (રંગોના ક્રમમાં: કાળો, બ્રાઉન, લાલ, નારંગી, પીળો, લીલો, વાદળી, જાંબલી, ગ્રે, સફેદ)}
    \end{mnemonicbox}
}

\mysolutionbox{પ્રશ્ન 2(અ) અથવા}{લાઈટ ડિપેન્ડન્ટ રેઝીસ્ટર તેની લાક્ષણિકતાઓ સાથે સમજાવો.}{
    \keyword{LDR} એક રેઝિસ્ટર છે જેનો પ્રતિરોધ પ્રકાશની તીવ્રતા વધે ત્યારે ઘટે છે.

    \textbf{LDR ની લાક્ષણિકતાઓ:}
    \captionof{table}{LDR ગુણધર્મો}
    \begin{tabulary}{\linewidth}{|L|L|}
    \hline
    \textbf{પેરામીટર} & \textbf{વર્તન} \\
    \hline
    અંધારી સ્થિતિ & ઉચ્ચ પ્રતિરોધ ($M\Omega$) \\
    \hline
    પ્રકાશિત સ્થિતિ & નીચો પ્રતિરોધ ($k\Omega$) \\
    \hline
    પ્રતિસાદ સમય & થોડી મિલિસેકન્ડ \\
    \hline
    \end{tabulary}

    \begin{answerdiagram}{LDR લાક્ષણિકતાઓ}
    \begin{tikzpicture}
        \begin{axis}[
            gtu plot,
            xlabel={Light Intensity (Lux)},
            ylabel={Resistance ($\Omega$)},
            domain=0:100,
            xmin=0, xmax=100,
            ymin=0, ymax=100,
            xtick={0,50,100},
            ytick={0,50,100},
            axis lines=left,
        ]
        \addplot[thick, blue, smooth] coordinates {(5, 90) (20, 40) (50, 20) (90, 10)};
        \end{axis}
    \end{tikzpicture}
    \end{answerdiagram}

    \begin{mnemonicbox}
    \mnemonic{પ્રકાશ વધે, અવરોધ ઘટે (LVAG)}
    \end{mnemonicbox}
}

\questionmarks{પ્રશ્ન 2(બ)}{3}{}
\mysolutionbox{પ્રશ્ન 2(બ)}{કેપેસિટરનું વર્ગીકરણ વિગતવાર સમજાવો.}{
    કેપેસિટર્સને ડાયઇલેક્ટ્રિક મટીરિયલ અને બાંધકામના આધારે વર્ગીકૃત કરવામાં આવે છે.

    \begin{answerdiagram}{કેપેસિટર વર્ગીકરણ}
    \begin{tikzpicture}[edge from parent/.style={draw, -latex}, level distance=1.5cm, sibling distance=2.5cm]
        \node[gtu block] {Capacitors}
            child {node[gtu block, align=center] {Fixed\\Capacitors}
                child {node[gtu block] {Ceramic}}
                child {node[gtu block] {Electrolytic}}
                child {node[gtu block] {Polyester}}
            }
            child {node[gtu block, align=center] {Variable\\Capacitors}
                child {node[gtu block] {Air Gang}}
                child {node[gtu block] {Trimmer}}
            };
    \end{tikzpicture}
    \end{answerdiagram}

    \captionof{table}{કેપેસિટર વર્ગીકરણ}
    \begin{tabulary}{\linewidth}{|L|L|L|}
    \hline
    \textbf{પ્રકાર} & \textbf{ડાયઇલેક્ટ્રિક} & \textbf{ઉપયોગો} \\
    \hline
    સિરામિક & સિરામિક & ઉચ્ચ આવૃત્તિ \\
    \hline
    ઇલેક્ટ્રોલિટિક & એલ્યુમિનિયમ ઓક્સાઇડ & પાવર સપ્લાય \\
    \hline
    પોલિએસ્ટર & પ્લાસ્ટિક ફિલ્મ & સામાન્ય હેતુ \\
    \hline
    ટેન્ટલમ & ટેન્ટલમ ઓક્સાઇડ & નાના, ઉચ્ચ ક્ષમતા \\
    \hline
    \end{tabulary}

    \begin{mnemonicbox}
    \mnemonic{CEPT (Ceramic, Electrolytic, Polyester, Tantalum)}
    \end{mnemonicbox}
}

\mysolutionbox{પ્રશ્ન 2(બ) અથવા}{ઈન્ડક્ટરનું વર્ગીકરણ વિગતવાર સમજાવો.}{
    ઇન્ડક્ટર્સને કોર સામગ્રી અને બાંધકામના આધારે વર્ગીકૃત કરવામાં આવે છે.

    \begin{answerdiagram}{ઇન્ડક્ટર વર્ગીકરણ}
    \begin{tikzpicture}[edge from parent/.style={draw, -latex}, level distance=1.5cm, sibling distance=2.5cm]
        \node[gtu block] {Inductors}
            child {node[gtu block] {Air Core}}
            child {node[gtu block] {Iron Core}}
            child {node[gtu block] {Ferrite Core}}
            child {node[gtu block] {Toroidal}};
    \end{tikzpicture}
    \end{answerdiagram}

    \captionof{table}{ઇન્ડક્ટર વર્ગીકરણ}
    \begin{tabulary}{\linewidth}{|L|L|L|}
    \hline
    \textbf{પ્રકાર} & \textbf{કોર} & \textbf{લાક્ષણિકતાઓ} \\
    \hline
    એર કોર & હવા & ઓછો ઇન્ડક્ટન્સ, ઓછા નુકશાન \\
    \hline
    આયર્ન કોર & લોખંડ & ઉચ્ચ ઇન્ડક્ટન્સ, ઉચ્ચ નુકશાન \\
    \hline
    ફેરાઇટ કોર & ફેરાઇટ & મધ્યમ ઇન્ડક્ટન્સ, ઓછા નુકશાન \\
    \hline
    ટોરોઇડલ & રિંગ આકારનું & ઉચ્ચ કાર્યક્ષમતા, ઓછું EMI \\
    \hline
    \end{tabulary}

    \begin{mnemonicbox}
    \mnemonic{હવા લોખંડ ફેરાઇટ ટોરોઇડ (AIFT)}
    \end{mnemonicbox}
}

\questionmarks{પ્રશ્ન 2(ક)}{4}{}
\mysolutionbox{પ્રશ્ન 2(ક)}{ફેરાડેનો ઈલેક્ટ્રોમેગ્નેટીક ઈન્ડક્શનના નિયમો લખો તથા સમજાવો.}{
    ફેરાડેના નિયમો સમજાવે છે કે ઇલેક્ટ્રોમેગ્નેટિક ઇન્ડક્શન કેવી રીતે કામ કરે છે.

    \textbf{ફેરાડેનો પ્રથમ નિયમ:}
    જ્યારે વાહક સાથે જોડાયેલ ચુંબકીય ક્ષેત્ર બદલાય છે, ત્યારે વાહકમાં EMF પ્રેરિત થાય છે.

    \textbf{ફેરાડેનો બીજો નિયમ:}
    પ્રેરિત EMFનો પરિમાણ ચુંબકીય ફ્લક્સના પરિવર્તનના દરના સમપ્રમાણમાં હોય છે.

    \captionof{table}{ફેરાડેના નિયમોનો સારાંશ}
    \begin{tabulary}{\linewidth}{|L|L|L|}
    \hline
    \textbf{નિયમ} & \textbf{વિધાન} & \textbf{સૂત્ર} \\
    \hline
    પ્રથમ નિયમ & ચુંબકીય ક્ષેત્રમાં ફેરફારથી EMF પ્રેરિત થાય છે & - \\
    \hline
    બીજો નિયમ & EMF $\propto$ ફ્લક્સના પરિવર્તનનો દર & $E = -N \frac{d\Phi}{dt}$ \\
    \hline
    \end{tabulary}

    \begin{answerdiagram}{ફેરાડેનો નિયમ}
    \begin{tikzpicture}
        % Coil
        \draw[decorate, decoration={coil, amplitude=4mm, segment length=3mm, post length=3mm}] (0,0) -- (4,0);
        \draw (4,0) to[galvanometer] (4,-2) -- (0,-2) -- (0,0);
        
        % Magnet
        \draw[fill=red!60] (-3,0.5) rectangle (-1.5, -0.5);
        \node at (-2.6, 0) {N};
        \draw[fill=blue!60] (-1.5,0.5) rectangle (0, -0.5);
        \node at (-0.4, 0) {S};
        
        % Movement arrow
        \draw[->, ultra thick] (-1, 0.8) -- (1, 0.8) node[midway, above] {Motion};
    \end{tikzpicture}
    \end{answerdiagram}

    \begin{mnemonicbox}
    \mnemonic{ચુંબકીય ક્ષેત્ર બદલાય, વિદ્યુત પ્રવાહ પેદા થાય (CMFCEC)}
    \end{mnemonicbox}
}

\mysolutionbox{પ્રશ્ન 2(ક) અથવા}{કેપેસિટરના સ્પેસિફીકેશન લખો તથા કોઈ પણ બે વિગતવાર સમજાવો.}{
    \textbf{કેપેસિટરના સ્પેસિફિકેશન:}
    \begin{enumerate}
        \item કેપેસિટન્સ મૂલ્ય
        \item વોલ્ટેજ રેટિંગ
        \item ટોલરન્સ
        \item લીકેજ કરંટ
        \item તાપમાન ગુણાંક
    \end{enumerate}

    \textbf{વિગતવાર સમજૂતી:}
    
    \textbf{1. કેપેસિટન્સ મૂલ્ય:}
    દર વોલ્ટ પર કેપેસિટર કેટલો ચાર્જ સંગ્રહિત કરી શકે છે, જે ફેરડ (F)માં માપવામાં આવે છે.

    \textbf{2. વોલ્ટેજ રેટિંગ:}
    મહત્તમ વોલ્ટેજ જે કેપેસિટરને નુકસાન કર્યા વિના લાગુ કરી શકાય છે.

    \captionof{table}{કેપેસિટર સ્પેસિફિકેશન}
    \begin{tabulary}{\linewidth}{|L|L|L|}
    \hline
    \textbf{સ્પેસિફિકેશન} & \textbf{વર્ણન} & \textbf{સામાન્ય મૂલ્યો} \\
    \hline
    કેપેસિટન્સ & ચાર્જ સંગ્રહ ક્ષમતા & pF થી mF \\
    \hline
    વોલ્ટેજ રેટિંગ & મહત્તમ સુરક્ષિત વોલ્ટેજ & 16V, 25V, 50V \\
    \hline
    \end{tabulary}

    \begin{mnemonicbox}
    \mnemonic{કેપેસિટર્સ વૉલ્ટેજ ટોલરન્ટ ઓફ લો ટેમ્પરેચર (CVTLT)}
    \end{mnemonicbox}
}

\questionmarks{પ્રશ્ન 2(ડ)}{4}{}
\mysolutionbox{પ્રશ્ન 2(ડ)}{47$\Omega\pm$5\% મા​ટે કલર કોડ લખો.}{
    $47\Omega \pm 5\%$ રેઝિસ્ટર માટે, કલર બેન્ડ્સ આ છે:

    \captionof{table}{47$\Omega\pm$5\% માટે કલર બેન્ડ્સ}
    \begin{tabulary}{\linewidth}{|L|L|L|}
    \hline
    \textbf{બેન્ડ} & \textbf{રંગ} & \textbf{રજૂ કરે છે} \\
    \hline
    1લી બેન્ડ & પીળો & 4 \\
    \hline
    2જી બેન્ડ & જાંબલી & 7 \\
    \hline
    3જી બેન્ડ & કાળો & $\times 10^0$ \\
    \hline
    4થી બેન્ડ & સોનેરી & $\pm 5\%$ \\
    \hline
    \end{tabulary}

    \begin{answerdiagram}{રેઝિસ્ટર કલર કોડ: 47 $\pm$ 5\%}
    \begin{tikzpicture}[scale=0.8]
        \draw[fill=brown!20] (0,0) rectangle (6,1.5);
        \fill[yellow] (1,0) rectangle (1.4,1.5); % Yellow
        \fill[violet] (2,0) rectangle (2.4,1.5); % Violet
        \fill[black] (3,0) rectangle (3.4,1.5); % Black
        \fill[yellow!80!orange] (5,0) rectangle (5.4,1.5); % Gold
        
        \node[below] at (1.2,0) {Yellow (4)};
        \node[below] at (2.2,0) {Violet (7)};
        \node[below] at (3.2,0) {Black ($\times 1$)};
        \node[below] at (5.2,0) {Gold ($\pm 5\%$)};
    \end{tikzpicture}
    \end{answerdiagram}

    \begin{mnemonicbox}
    \mnemonic{Yellow Violets Bring Gold}
    \end{mnemonicbox}
}

\mysolutionbox{પ્રશ્ન 2(ડ) અથવા}{આપેલ કલર કોડ માટે રેઝીસ્ટરની કિંમત તથા ટોલરન્સ શોધો: Brown, Black, yellow.}{
    \captionof{table}{Brown, Black, Yellow નું અર્થઘટન}
    \begin{tabulary}{\linewidth}{|L|L|L|L|}
    \hline
    \textbf{બેન્ડ} & \textbf{રંગ} & \textbf{મૂલ્ય} & \textbf{અર્થ} \\
    \hline
    1લી & બ્રાઉન & 1 & પ્રથમ અંક \\
    \hline
    2જી & કાળો & 0 & બીજો અંક \\
    \hline
    3જી & પીળો & $10^4$ & ગુણાંક \\
    \hline
    \end{tabulary}

    \vspace{1em}
    \textbf{ગણતરી:}
    \begin{itemize}
        \item 1લો અંક: 1
        \item 2જો અંક: 0
        \item ગુણાંક: $10^4$
    \end{itemize}
    
    મૂલ્ય = $10 \times 10^4 = 100,000\Omega = 100 k\Omega$

    4થી બેન્ડનો અભાવ એટલે $\pm 20\%$ ટોલરન્સ.

    \begin{answerdiagram}{રેઝિસ્ટર: 100k$\Omega$}
    \begin{tikzpicture}[scale=0.8]
        \draw[fill=brown!20] (0,0) rectangle (6,1.5);
        \fill[brown] (1,0) rectangle (1.4,1.5); % Brown
        \fill[black] (2,0) rectangle (2.4,1.5); % Black
        \fill[yellow] (3,0) rectangle (3.4,1.5); % Yellow
        
        \node[below] at (1.2,0) {Brown (1)};
        \node[below] at (2.2,0) {Black (0)};
        \node[below] at (3.2,0) {Yellow ($10^4$)};
        \node[below] at (5.2,0) {No Band ($\pm 20\%$)};
    \end{tikzpicture}
    \end{answerdiagram}

    \begin{mnemonicbox}
    \mnemonic{બ્રાઉન બ્લેક યલો (BBY)}
    \end{mnemonicbox}
}

\questionmarks{પ્રશ્ન 3(અ)}{3}{}
\mysolutionbox{પ્રશ્ન 3(અ)}{ડોપિંગની વ્યાખ્યા લખો. ડોપિંગથી બનતા અર્ધવાહકોના નામ તથા ઉદાહરણ આપો.}{
    \keyword{ડોપિંગ} એ શુદ્ધ અર્ધવાહકમાં અશુદ્ધિઓ ઉમેરવાની પ્રક્રિયા છે જે તેના વિદ્યુત ગુણધર્મોને સંશોધિત કરે છે.

    \captionof{table}{ડોપ્ડ અર્ધવાહકો}
    \begin{tabulary}{\linewidth}{|L|L|L|L|}
    \hline
    \textbf{પ્રકાર} & \textbf{ઉમેરેલ ડોપન્ટ} & \textbf{ઉદાહરણ} & \textbf{મુખ્ય વાહકો} \\
    \hline
    P-type & ત્રિસંયોજક (બોરોન, ગેલિયમ) & બોરોન સાથે ડોપ કરેલ સિલિકોન & હોલ્સ \\
    \hline
    N-type & પંચસંયોજક (ફોસ્ફરસ, આર્સેનિક) & ફોસ્ફરસ સાથે ડોપ કરેલ સિલિકોન & ઇલેક્ટ્રોન્સ \\
    \hline
    \end{tabulary}

    \begin{answerdiagram}{ડોપિંગ પ્રક્રિયા}
    \begin{tikzpicture}[node distance=2.5cm, auto, >=latex]
        \node [gtu block] (pure) {Pure Semiconductor};
        \node [gtu block, below of=pure, xshift=-2.5cm] (ptype) {P-type};
        \node [gtu block, below of=pure, xshift=2.5cm] (ntype) {N-type};
        
        \path [gtu arrow] (pure) -- node [left, align=center] {Add Trivalent\\Impurity} (ptype);
        \path [gtu arrow] (pure) -- node [right, align=center] {Add Pentavalent\\Impurity} (ntype);
    \end{tikzpicture}
    \end{answerdiagram}

    \begin{mnemonicbox}
    \mnemonic{પોઝિટિવમાં પ્લસ હોલ્સ, નેગેટિવમાં નંબર ઇલેક્ટ્રોન્સ (PHNE)}
    \end{mnemonicbox}
}

\mysolutionbox{પ્રશ્ન 3(અ) અથવા}{વ્યાખ્યા લખો: રીપલ ફેક્ટર, પીક ઈનવર્સ વોલ્ટેજ, રેક્ટીફીકેશન એફીસીયન્સી.}{
    \captionof{table}{રેક્ટિફાયર પદો}
    \begin{tabulary}{\linewidth}{|L|L|L|}
    \hline
    \textbf{પદ} & \textbf{વ્યાખ્યા} & \textbf{સૂત્ર} \\
    \hline
    રીપલ ફેક્ટર & રેક્ટિફાઇડ આઉટપુટમાં AC ઘટકનું માપ & $r = \frac{V_{rms(AC)}}{V_{dc}}$ \\
    \hline
    પીક ઇન્વર્સ વોલ્ટેજ & મહત્તમ રિવર્સ વોલ્ટેજ જે ડાયોડ સહન કરી શકે છે & - \\
    \hline
    રેક્ટિફિકેશન એફિસિયન્સી & DC આઉટપુટ પાવરનો AC ઇનપુટ પાવર સાથેનો ગુણોત્તર & $\eta = \frac{P_{dc}}{P_{ac}} \times 100\%$ \\
    \hline
    \end{tabulary}

    \begin{mnemonicbox}
    \mnemonic{રિપલ્સ પીક એફિશિયન્ટલી (RPE)}
    \end{mnemonicbox}
}

\questionmarks{પ્રશ્ન 3(બ)}{3}{}
\mysolutionbox{પ્રશ્ન 3(બ)}{ક્રિસ્ટલ ડાયોડનું કાર્ય સમજાવો.}{
    \keyword{ક્રિસ્ટલ ડાયોડ} એ પોઇન્ટ-કોન્ટેક્ટ ડાયોડ છે જે અર્ધવાહક ક્રિસ્ટલ સાથે બનાવવામાં આવે છે.

    \textbf{બાંધકામ}: તે અર્ધવાહક ક્રિસ્ટલ (જર્મેનિયમ/સિલિકોન) અને તેની સામે દબાવવામાં આવતા પાતળા ટંગસ્ટન વાયર (કેટ્સી વ્હીસ્કર) ધરાવે છે.
    
    \textbf{કાર્ય}: તે ઉચ્ચ આવૃત્તિના રેડિયો સિગ્નલોનું રેક્ટિફિકેશન (ડીમોડ્યુલેશન) કરે છે.

    \begin{answerdiagram}{ક્રિસ્ટલ ડાયોડ બાંધકામ}
    \begin{tikzpicture}
        \draw[fill=gray!20] (0,0) rectangle (4,0.5); % Base
        \node at (2,0.25) {Metal Base};
        \draw[fill=blue!30] (1.5,0.5) rectangle (2.5,1.5); % Crystal
        \node at (2,1) {Ge Crystal};
        \draw[thick] (2,1.5) -- (2,2.5) -- (3,2.5) -- (3,0.5); % Whisker visualization (schematic)
        \draw[decorate, decoration={coil, aspect=0.3, segment length=1mm, amplitude=1mm}] (2,2.5) -- (2,1.5);
        \node[right] at (2,2) {Cat's Whisker};
        \draw (1,0.5) -- (1,2.5); % Case enclosure schematic
        \draw (3,0.5) -- (3,2.5);
        \draw (1,2.5) -- (3,2.5);
    \end{tikzpicture}
    \end{answerdiagram}

    \begin{mnemonicbox}
    \mnemonic{ક્રિસ્ટલ શોધે રેડિયો ફ્રીક્વન્સી (CDRF)}
    \end{mnemonicbox}
}

\mysolutionbox{પ્રશ્ન 3(બ) અથવા}{ફોટોડાયોડનું કાર્ય સમજાવો.}{
    \keyword{ફોટોડાયોડ} રિવર્સ બાયસમાં ઓપરેટ કરવામાં આવે ત્યારે પ્રકાશ ઊર્જાને વિદ્યુત પ્રવાહમાં રૂપાંતરિત કરે છે.

    \textbf{કાર્યপદ્ધતિ:}
    \begin{enumerate}
        \item પ્રકાશ PN જંક્શન પર પડે છે.
        \item ફોટોન્સ ઇલેક્ટ્રોન-હોલ જોડી ઉત્પન્ન કરે છે.
        \item રિવર્સ બાયસ ફિલ્ડ ચાર્જ કેરિયર્સને જંક્શન પાર ખેંચે છે, જેનાથી કરંટ વહે છે.
    \end{enumerate}

    \begin{answerdiagram}{ફોટોડાયોડ ઓપરેશન}
    \begin{tikzpicture}[]
        \draw (0,0) to[photodiode] (2,0);
        \draw[->, thick, orange] (0.5, 1) -- (1, 0.2);
        \draw[->, thick, orange] (1.0, 1) -- (1.3, 0.2);
        \node at (2.5, 0.5) {Light};
    \end{tikzpicture}
    \end{answerdiagram}

    \begin{mnemonicbox}
    \mnemonic{પ્રકાશ આવે, કરંટ જાય (LICO)}
    \end{mnemonicbox}
}

\questionmarks{પ્રશ્ન 3(ક)}{4}{}
\mysolutionbox{પ્રશ્ન 3(ક)}{સર્કિટ તથા વેવફોર્મ દોરી હાફ-વેવ રેક્ટીફાયર સમજાવો.}{
    \keyword{હાફ-વેવ રેક્ટિફાયર} AC ને પલ્સેટિંગ DCમાં રૂપાંતરિત કરે છે, માત્ર પોઝિટિવ હાફ સાયકલ દરમિયાન પ્રવાહને પસાર કરીને.

    \begin{answerdiagram}{હાફ-વેવ રેક્ટિફાયર સર્કિટ}
    \begin{tikzpicture}[]
        \draw (0,2) to[AC source, l={AC Input}] (0,0);
        \draw (0,2) to[diode] (3,2) to[resistor, l={Load}] (3,0) -- (0,0);
    \end{tikzpicture}
    \end{answerdiagram}

    \begin{answerdiagram}{હાફ-વેવ વેવફોર્મ્સ}
    \begin{tikzpicture}
        \begin{scope}[yshift=2cm]
            \draw[->] (0,0) -- (6.5,0) node[right] {$t$};
            \draw[->] (0,-1.2) -- (0,1.2) node[above] {$V_{in}$};
            \draw[thick, blue] plot[domain=0:6.28, samples=100] (\x, {sin(\x r)});
            \node at (3,1.5) {Input AC};
        \end{scope}
        
        \begin{scope}[yshift=0cm]
            \draw[->] (0,0) -- (6.5,0) node[right] {$t$};
            \draw[->] (0,-0.2) -- (0,1.2) node[above] {$V_{out}$};
            \draw[thick, red] plot[domain=0:3.14, samples=50] (\x, {sin(\x r)});
            \draw[thick, red] (3.14,0) -- (6.28,0);
            \node at (3,-0.5) {Output DC};
        \end{scope}
    \end{tikzpicture}
    \end{answerdiagram}

    \begin{mnemonicbox}
    \mnemonic{અર્ધ તરંગ અર્ધ પસાર (HWPH)}
    \end{mnemonicbox}
}

\mysolutionbox{પ્રશ્ન 3(ક) અથવા}{સર્કિટ તથા વેવફોર્મ દોરી ફુલ-વેવ રેક્ટીફાયર સમજાવો.}{
    \keyword{ફુલ-વેવ રેક્ટિફાયર} (બ્રિજ પ્રકાર) AC ઇનપુટના બંને અર્ધ ભાગોને DC માં રૂપાંતરિત કરે છે.

    \begin{answerdiagram}{બ્રિજ રેક્ટિફાયર સર્કિટ}
    \begin{tikzpicture}[]
        \draw (0,0) to[AC source, l={Input}] (0,3);
        \draw (2,1.5) to[diode] (3.5,3) to[diode] (5,1.5) to[diode] (3.5,0) to[diode] (2,1.5);
        \draw (0,3) -- (3.5,3);
        \draw (0,0) -- (3.5,0);
        \draw (2,1.5) -- (2,1.5) to[resistor, l={$R_L$}] (5,1.5); 
        % Simplified bridge drawing for clarity
    \end{tikzpicture}
    \end{answerdiagram}
    
    \begin{answerdiagram}{ફુલ-વેવ વેવફોર્મ્સ}
    \begin{tikzpicture}
        \begin{scope}[yshift=2cm]
            \draw[->] (0,0) -- (6.5,0) node[right] {$t$};
            \draw[->] (0,-1.2) -- (0,1.2) node[above] {$V_{in}$};
            \draw[thick, blue] plot[domain=0:6.28, samples=100] (\x, {sin(\x r)});
        \end{scope}
        
        \begin{scope}[yshift=0cm]
            \draw[->] (0,0) -- (6.5,0) node[right] {$t$};
            \draw[->] (0,-0.2) -- (0,1.2) node[above] {$V_{out}$};
            \draw[thick, red] plot[domain=0:3.14, samples=50] (\x, {sin(\x r)});
            \draw[thick, red] plot[domain=3.14:6.28, samples=50] (\x, {-sin(\x r)});
        \end{scope}
    \end{tikzpicture}
    \end{answerdiagram}

    \begin{mnemonicbox}
    \mnemonic{પૂર્ણ તરંગ પૂર્ણ ઉપયોગ (FWMFU)}
    \end{mnemonicbox}
}

\questionmarks{પ્રશ્ન 3(ડ)}{4}{}
\mysolutionbox{પ્રશ્ન 3(ડ)}{PN-જંક્શન ડાયોડના VI લાક્ષણિકતાઓ આકૃતિ દોરી સમજાવો.}{
    \begin{answerdiagram}{PN ડાયોડની VI લાક્ષણિકતાઓ}
    \begin{tikzpicture}
        \begin{axis}[
            gtu plot,
            width=8cm, height=6cm,
            xmin=-10, xmax=1.5,
            ymin=-2, ymax=10,
            axis lines=middle,
            xlabel={$V$ (Volts)},
            ylabel={$I$ (mA)},
            xtick={-10, -5, 0, 0.7},
            xticklabels={-10, -5, 0, 0.7V},
            ytick={0},
        ]
        % Forward
        \addplot[thick, blue, samples=100, domain=0:1] {exp(4*(x-0.7))};
        \node at (axis cs:0.8, 5) {Forward Bias};

        % Reverse
        \addplot[thick, red, samples=100, domain=-9:0] {-0.2};
        \addplot[thick, red] coordinates {(-9, -0.2) (-9, -2)};
        \node at (axis cs:-5, -1) {Reverse Leakage};
        \node at (axis cs:-9.5, -1) {Breakdown};
        \end{axis}
    \end{tikzpicture}
    \end{answerdiagram}

    \textbf{કોષ્ટક: લાક્ષણિકતાઓ}
    \begin{tabulary}{\linewidth}{|L|L|}
    \hline
    \textbf{પ્રદેશ} & \textbf{વર્તન} \\
    \hline
    ફોરવર્ડ બાયસ & 0.7V ($V_k$) પછી કરંટ એક્સપોનેન્શિયલી વધે છે \\
    \hline
    રિવર્સ બાયસ & નહિવત લીકેજ કરંટ \\
    \hline
    બ્રેકડાઉન & ઉચ્ચ રિવર્સ વોલ્ટેજ પર કરંટમાં તીવ્ર વધારો \\
    \hline
    \end{tabulary}

    \begin{mnemonicbox}
    \mnemonic{ફોરવર્ડ ફ્લો, રિવર્સ રેસ્ટ્રિક્ટ (FFRR)}
    \end{mnemonicbox}
}

\mysolutionbox{પ્રશ્ન 3(ડ) અથવા}{P-type અને N-type અર્ધવાહક વચ્ચેનો તફાવત લખો.}{
    \captionof{table}{P-type vs N-type}
    \begin{tabulary}{\linewidth}{|L|L|L|}
    \hline
    \textbf{ગુણધર્મ} & \textbf{P-type} & \textbf{N-type} \\
    \hline
    ડોપન્ટ & ત્રિસંયોજક (બોરોન) & પંચસંયોજક (ફોસ્ફરસ) \\
    \hline
    મુખ્ય વાહકો & હોલ્સ & ઇલેક્ટ્રોન્સ \\
    \hline
    ગૌણ વાહકો & ઇલેક્ટ્રોન્સ & હોલ્સ \\
    \hline
    \end{tabulary}

    \begin{mnemonicbox}
    \mnemonic{પોઝિટિવમાં પ્લસ હોલ્સ, નેગેટિવમાં નંબર ઇલેક્ટ્રોન્સ}
    \end{mnemonicbox}
}

\questionmarks{પ્રશ્ન 4(અ)}{3}{}
\mysolutionbox{પ્રશ્ન 4(અ)}{LED ની કાર્યપદ્ધતિ સમજાવો.}{
    \keyword{LED} (લાઇટ એમિટિંગ ડાયોડ) ફોરવર્ડ બાયસ થયેલ હોય ત્યારે ઇલેક્ટ્રોન-હોલ રિકોમ્બિનેશનને કારણે પ્રકાશ ઉત્સર્જિત કરે છે.

    \textbf{કાર્યপદ્ધતિનો સિદ્ધાંત:}
    જ્યારે ફોરવર્ડ બાયસ કરવામાં આવે છે, ત્યારે N-સાઇડથી ઇલેક્ટ્રોન્સ P-સાઇડ તરફ ગતિ કરે છે અને હોલ્સ સાથે રિકોમ્બાઇન થાય છે, જેના પરિણામે ફોટોન્સ (પ્રકાશ) તરીકે ઊર્જા છોડે છે.

    \begin{answerdiagram}{LED ઓપરેશન}
    \begin{tikzpicture}
        % PN Block
        \draw[fill=blue!10] (0,0) rectangle (2,2);
        \node at (1,1) {P-type};
        \draw[fill=green!10] (2,0) rectangle (4,2);
        \node at (3,1) {N-type};
        \draw[thick] (2,0) -- (2,2); % Junction

        % Recombination
        \draw[->, decorate, decoration={snake, amplitude=2mm, segment length=3mm, post length=2mm}, orange, thick] (2,1) -- (1,2.5);
        \draw[->, decorate, decoration={snake, amplitude=2mm, segment length=3mm, post length=2mm}, orange, thick] (2,1) -- (3,2.5);
        \node at (2, 2.8) {Light (Photons)};

        % Circuit
        \draw (0,1) -- (-1,1) -- (-1,-1) -- (5,-1) -- (5,1) -- (4,1);
        \draw[fill=white] (2,-1) circle (0.3);
        \node at (2,-1) {$+$ \hspace{1em} $-$};
    \end{tikzpicture}
    \end{answerdiagram}

    \begin{mnemonicbox}
    \mnemonic{ફોરવર્ડ કરંટ પ્રકાશ ઉત્સર્જિત કરે (FCEL)}
    \end{mnemonicbox}
}

\mysolutionbox{પ્રશ્ન 4(અ) અથવા}{LED ના ઉપયોગો જણાવો.}{
    \captionof{table}{LED ઉપયોગો}
    \begin{tabulary}{\linewidth}{|L|L|}
    \hline
    \textbf{ઉપયોગ} & \textbf{ફાયદો} \\
    \hline
    ડિસ્પ્લે ઇન્ડિકેટર્સ & ઓછો પાવર વપરાશ \\
    \hline
    ડિજિટલ ડિસ્પ્લે (7-સેગમેન્ટ) & વિવિધ રંગો ઉપલબ્ધ \\
    \hline
    લાઇટિંગ (બલ્બ) & ઊર્જા કાર્યક્ષમ \\
    \hline
    રિમોટ કંટ્રોલ & ઇન્ફ્રારેડ કમ્યુનિકેશન \\
    \hline
    ટ્રાફિક સિગ્નલ્સ & ઉચ્ચ દૃશ્યતા \\
    \hline
    \end{tabulary}
}

\questionmarks{પ્રશ્ન 4(બ)}{4}{}
\mysolutionbox{પ્રશ્ન 4(બ)}{"ઝેનર ડાયોડ વોલ્ટેજ રેગ્યુલેટર તરીકે" સમજાવો.}{
    \keyword{ઝેનર ડાયોડ} રિવર્સ બ્રેકડાઉન રીજીયનમાં ઓપરેટ કરવામાં આવે ત્યારે ઇનપુટ વોલ્ટેજની અસ્થિરતા છતાં સ્થિર આઉટપુટ વોલ્ટેજ જાળવે છે.

    \begin{answerdiagram}{ઝેનર વોલ્ટેજ રેગ્યુલેટર}
    \begin{tikzpicture}[]
        \draw (0,2) node[left] {$V_{in}$} to[resistor, l=$R_S$] (2,2) -- (4,2) -- (4,0) -- (0,0);
        \draw (2,2) to[zener diode, l=$V_Z$] (2,0);
        \draw (4,2) to[resistor, l=$R_L$] (4,0);
        \node[right] at (4,1) {$V_{out} = V_Z$};
    \end{tikzpicture}
    \end{answerdiagram}

    \textbf{કાર્ય:}
    \begin{itemize}
        \item સીરીઝ રેઝિસ્ટર $R_S$ કરંટ મર્યાદિત કરે છે.
        \item ઝેનર ડાયોડ તેની આરપાર વોલ્ટેજ ($V_Z$) સ્થિર રાખવા માટે ચલિત કરંટ વહન કરે છે.
    \end{itemize}

    \begin{mnemonicbox}
    \mnemonic{ઝેનર બ્રેક ટુ રેગ્યુલેટ (ZBR)}
    \end{mnemonicbox}
}

\mysolutionbox{પ્રશ્ન 4(બ) અથવા}{ઝેનર વોલ્ટેજ રેગ્યુલેટરની મર્યાદાઓ.}{
    \captionof{table}{મર્યાદાઓ}
    \begin{tabulary}{\linewidth}{|L|L|}
    \hline
    \textbf{મર્યાદા} & \textbf{અસર} \\
    \hline
    પાવર ડિસિપેશન & ઝેનર પાવર રેટિંગ દ્વારા મર્યાદિત \\
    \hline
    કરંટ ક્ષમતા & માત્ર નાના લોડ સંભાળી શકે છે \\
    \hline
    કાર્યક્ષમતા & $R_S$ માં પાવર લોસને કારણે ઓછી \\
    \hline
    \end{tabulary}
}

\questionmarks{Question 4(c)}{7}{}
\mysolutionbox{Question 4(c)}{Discuss the necessity of filter circuit in rectifier. List various types of filter circuits used in rectifier and explain any one with neat diagram.}{
    \textbf{Necessity:} Rectifier output is pulsating DC (contains AC ripples). Filter circuits remove these ripples to provide a steady DC voltage required by electronic circuits.

    \textbf{Types of Filters:}
    \begin{enumerate}
        \item Capacitor Filter (Shunt)
        \item Inductor Filter (Series)
        \item LC Filter (L-Section)
        \item $\pi$-Filter (C-L-C)
    \end{enumerate}

    \textbf{Capacitor Filter Explanation:}
    A capacitor is connected in parallel with the load.

    \begin{answerdiagram}{Capacitor Filter Circuit}
    \begin{tikzpicture}[]
        \draw (0,0) node[left] {Rectifier Out} -- (3,0);
        \draw (0,2) -- (1.5,2) to[capacitor, l=$C$] (1.5,0);
        \draw (1.5,2) -- (3,2) to[resistor, l=$R_L$] (3,0);
        \node[right] at (3,1) {$V_{out}$};
    \end{tikzpicture}
    \end{answerdiagram}

    \textbf{Operation:}
    \begin{itemize}
        \item During voltage peak, capacitor charges to $V_{peak}$.
        \item During voltage drop, capacitor discharges through load, maintaining voltage.
        \item Result: Reduced ripple, smoother DC.
    \end{itemize}

    \begin{answerdiagram}{Filter Waveform}
    \begin{tikzpicture}
        \draw[->] (0,0) -- (4,0) node[right] {$t$};
        \draw[->] (0,0) -- (0,2) node[above] {$V$};
        \draw[gray, dashed] plot[domain=0:4, samples=50] (\x, {abs(sin(2*\x r)) + 0.5});
        \draw[thick, blue] (0.4,1.5) -- (1.2,1.2) -- (1.9,1.5) -- (2.7,1.2) -- (3.5,1.5);
        \node at (2, 2.2) {Filtered Output};
    \end{tikzpicture}
    \end{answerdiagram}

    \begin{mnemonicbox}
    \mnemonic{Capacitors Hold Voltage During Drops}
    \end{mnemonicbox}
}

\questionmarks{પ્રશ્ન 5(અ)}{3}{}
\mysolutionbox{પ્રશ્ન 5(અ)}{ઈ-વેસ્ટની વ્યાખ્યા લખો. સામાન્ય ઈ-વેસ્ટ વસ્તુઓની યાદી બનાવો.}{
    \keyword{ઈ-વેસ્ટ} (Electronic Waste) એટલે ત્યજિત ઇલેક્ટ્રોનિક ઉપકરણો અને ઘટકો કે જે તેમના ઉપયોગી જીવનકાળના અંતે પહોંચ્યા છે.

    \captionof{table}{સામાન્ય ઈ-વેસ્ટ}
    \begin{tabulary}{\linewidth}{|L|L|}
    \hline
    \textbf{શ્રેણી} & \textbf{ઉદાહરણો} \\
    \hline
    કમ્પ્યુટિંગ & લેપટોપ, પીસી, ટેબ્લેટ \\
    \hline
    કમ્યુનિકેશન & મોબાઇલ ફોન, લેન્ડલાઇન \\
    \hline
    ઘરેલું ઉપકરણો & ટીવી, ફ્રિજ, વોશિંગ મશીન \\
    \hline
    ઘટકો & બેટરી, PCBs, કેબલ્સ \\
    \hline
    \end{tabulary}

    \begin{mnemonicbox}
    \mnemonic{કમ્પ્યુટર્સ, કમ્યુનિકેશન, કમ્પોનન્ટ્સ (CCC)}
    \end{mnemonicbox}
}

\mysolutionbox{પ્રશ્ન 5(બ)}{ઈ-વેસ્ટ મેનેજમેન્ટની વિવિધ વ્યૂહરચના જણાવો અને સમજાવો.}{
    \captionof{table}{મેનેજમેન્ટ વ્યૂહરચના}
    \begin{tabulary}{\linewidth}{|L|L|}
    \hline
    \textbf{વ્યૂહરચના} & \textbf{વર્ણન} \\
    \hline
    ઘટાડવું (Reduce) & ઓછું ખરીદવું, લાંબો સમય સાચવવું \\
    \hline
    ફરીથી ઉપયોગ (Reuse) & રિપેર, દાન, વેચાણ \\
    \hline
    રિસાયકલ (Recycle) & મૂલ્યવાન ધાતુઓ કાઢવી (Au, Ag, Cu) \\
    \hline
    નિકાલ (Disposal) & જોખમી પદાર્થોનો સુરક્ષિત નિકાલ \\
    \hline
    \end{tabulary}

    \begin{mnemonicbox}
    \mnemonic{3 R's: Reduce, Reuse, Recycle}
    \end{mnemonicbox}
}

\questionmarks{પ્રશ્ન 5(ક)}{4}{}
\mysolutionbox{પ્રશ્ન 5(ક)}{"ટ્રાનઝીસ્ટર સ્વીચ તરીકે" સમજાવો.}{
    ટ્રાન્ઝિસ્ટર \textbf{કટઓફ} (OFF) અને \textbf{સેચુરેશન} (ON) રીજીયનમાં ઓપરેટ કરીને સ્વિચ તરીકે વર્તે છે.

    \begin{answerdiagram}{ટ્રાન્ઝિસ્ટર સ્વીચ સર્કિટ}
    \begin{tikzpicture}[]
        \draw (0,0) node[npn] (t) {};
        \draw (t.E) -- (0,-1) node[ground] {};
        \draw (t.C) to[resistor, l=$R_C$] (0,2) node[above] {$+V_{CC}$};
        \draw (t.B) to[resistor, l=$R_B$] (-2,0) node[left] {$V_{in}$};
    \end{tikzpicture}
    \end{answerdiagram}

    \textbf{સ્થિતિઓ:}
    \begin{itemize}
        \item \textbf{OFF (ખુલ્લી સ્વિચ)}: $V_{in} = 0V$. બેઝ કરંટ $I_B = 0$, તેથી કલેક્ટર કરંટ $I_C = 0$. $V_{CE} = V_{CC}$.
        \item \textbf{ON (બંધ સ્વિચ)}: $V_{in} = High$. $I_B$ વહે છે, ટ્રાન્ઝિસ્ટર સેચુરેટ થાય છે. $V_{CE} \approx 0V$.
    \end{itemize}

    \begin{mnemonicbox}
    \mnemonic{નો બેઝ નો કરંટ (NBNC)}
    \end{mnemonicbox}
}

\questionmarks{પ્રશ્ન 5(ડ)}{4}{}
\mysolutionbox{પ્રશ્ન 5(ડ)}{ટ્રાંઝીસ્ટરના CE કંફીગરેશન માટે $\alpha$ તથા $\beta$ વચ્ચેનો સંબંધ તારવો.}{
    \textbf{વ્યાખ્યાઓ:}
    \begin{itemize}
        \item $\alpha = \frac{I_C}{I_E}$ (કોમન બેઝ ગેઇન)
        \item $\beta = \frac{I_C}{I_B}$ (કોમન એમિટર ગેઇન)
    \end{itemize}

    \textbf{તારણ:}
    આપણે જાણીએ છીએ કે એમિટર કરંટ એ બેઝ અને કલેક્ટર કરંટનો સરવાળો છે:
    \begin{equation}
        I_E = I_C + I_B
    \end{equation}
    
    સમીકરણ (1) ને $I_C$ વડે ભાગતા:
    $$ \frac{I_E}{I_C} = \frac{I_C}{I_C} + \frac{I_B}{I_C} $$
    $$ \frac{1}{\alpha} = 1 + \frac{1}{\beta} $$
    $$ \frac{1}{\alpha} = \frac{\beta + 1}{\beta} $$
    
    બંને બાજુ વ્યસ્ત કરતા:
    $$ \alpha = \frac{\beta}{1 + \beta} $$

    $\beta$ માટે ગોઠવતા:
    $$ \beta = \frac{\alpha}{1 - \alpha} $$

    \begin{mnemonicbox}
    \mnemonic{બીટા બરાબર આલ્ફા ડિવાઇડેડ બાય વન માઇનસ આલ્ફા}
    \end{mnemonicbox}
}

\end{document}
