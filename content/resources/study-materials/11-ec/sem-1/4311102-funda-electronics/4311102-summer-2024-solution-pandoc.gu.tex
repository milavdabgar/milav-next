\documentclass[10pt,a4paper]{article}

% content/resources/templates/preamble.tex
\usepackage[margin=0.6in]{geometry}
\author{Milav Dabgar}
\usepackage{amsmath,amssymb,amsthm}
\usepackage{booktabs}
\usepackage{multirow}
\usepackage{xcolor}
\usepackage{tcolorbox}
\tcbuselibrary{breakable,skins}
\usepackage[colorlinks=true,linkcolor=blue]{hyperref}
\usepackage{titlesec}
\usepackage{enumitem}
\usepackage{tikz}
\usepackage{pgfplots}
\usepackage{circuitikz}
\usepackage[version=4]{mhchem}
\usepackage{longtable}
\usepackage{array}
\usepackage{float}
\usepackage{caption}
\usepackage{listings}

\lstset{
  basicstyle=\small\ttfamily,
  breaklines=true,
  breakatwhitespace=false,
  postbreak=\mbox{\textcolor{red}{$\hookrightarrow$}\space},
  float=false,
  numbers=left,
  numberstyle=\tiny\color{gray},
  numbersep=10pt,
  xleftmargin=2em,
  keywordstyle=\color{blue},
  commentstyle=\color{green!60!black},
  stringstyle=\color{purple},
  backgroundcolor=\color{gray!5},
  showstringspaces=false,
  tabsize=2,
  captionpos=b,
  keepspaces=true,
  columns=flexible
}

\pgfplotsset{compat=1.18}
\usetikzlibrary{shapes,arrows,positioning,calc,patterns,decorations.pathmorphing,decorations.markings,arrows.meta}

% Color scheme
\definecolor{headcolor}{RGB}{0,102,204}
\definecolor{keycolor}{RGB}{220,20,60}
\definecolor{solutioncolor}{RGB}{34,139,34}
\definecolor{mnemoniccolor}{RGB}{148,0,211}
\definecolor{codecolor}{RGB}{0,0,100}

% Spacing
\setlength{\parskip}{3pt}
\setlist[itemize]{nosep}
\setlist[enumerate]{nosep}

% Title formatting
\titleformat{\section}{\Large\bfseries\color{headcolor}}{\thesection}{1em}{}
\titleformat{\subsection}{\large\bfseries\color{headcolor}}{\thesubsection}{1em}{}

% Pandoc tightlist compatibility
\providecommand{\tightlist}{%
  \setlength{\itemsep}{0pt}\setlength{\parskip}{0pt}}

% Pandoc longtable compatibility
\newcounter{none}
\def\thenone{}


% content/resources/templates/gujarati-boxes.tex
\usepackage{fontspec}
\usepackage{polyglossia}

% Set Gujarati as main language (document is primarily in Gujarati)
% Note: gloss-gujarati.ldf doesn't exist in polyglossia, but it will use hyphenation patterns
\setdefaultlanguage{gujarati}
\setotherlanguage{english}

% Configure Gujarati font properly
% Use Language=Default to prevent polyglossia from trying to add language-specific features
% that don't exist for Gujarati, which causes "empty feature" warnings
\newfontfamily\gujaratifont[Script=Gujarati,AutoFakeBold=2.5,AutoFakeSlant=0.3]{Noto Sans Gujarati}
\setmainfont[Script=Gujarati,AutoFakeBold=2.5,AutoFakeSlant=0.3]{Noto Sans Gujarati}
% Use Noto Sans Gujarati for monospace to support Gujarati in text
\setmonofont[Scale=0.9]{Noto Sans Gujarati}

% Configure English to use the same font
\newfontfamily\englishfont[Script=Gujarati,AutoFakeBold=2.5,AutoFakeSlant=0.3]{Noto Sans Gujarati}

% Translations for polyglossia
\gappto\captionsgujarati{
  \renewcommand{\tablename}{કોષ્ટક}
  \renewcommand{\figurename}{આકૃતિ}
}

% Helper for TikZ nodes to ensure Gujarati font
\newcommand{\gu}[1]{{\gujaratifont #1}}

% Custom environments
\newtcolorbox{solutionbox}{
    breakable,
    enhanced,
    colback=solutioncolor!5!white,
    colframe=solutioncolor!75!black,
    fonttitle=\bfseries,
    title=જવાબ
}

\newtcolorbox{solutionboxnobreak}{
 colback=solutioncolor!5!white,
 colframe=solutioncolor!75!black,
 fonttitle=\bfseries,
 title=જવાબ
}

\newtcolorbox{keyformula}{
 breakable,
 enhanced,
 colback=keycolor!5!white,
 colframe=keycolor!75!black,
 fonttitle=\bfseries,
 title=રાસાયણિક સમીકરણ/સૂત્ર
}

\newtcolorbox{mnemonicbox}{
 breakable,
 enhanced,
 colback=mnemoniccolor!5!white,
 colframe=mnemoniccolor!75!black,
 fonttitle=\bfseries,
 title=મેમરી ટ્રીક
}


\begin{document}

\begin{center}
{\Huge\bfseries\color{headcolor} Subject Name (Gujarati)}\\[5pt]
{\LARGE 4311102 -- Summer 2024}\\[3pt]
{\large Semester 1 Study Material}\\[3pt]
{\normalsize\textit{Detailed Solutions and Explanations}}
\end{center}

\vspace{10pt}

\subsection*{પ્રશ્ન 1 [14
ગુણ]}\label{uxaaauxab0uxab6uxaa8-1-14-uxa97uxaa3}

\textbf{દસમાંથી કોઈપણ સાત પ્રશ્નોના જવાબ આપો.}

\subsubsection{પ્રશ્ન 1(1) [2
ગુણ]}\label{uxaaauxab0uxab6uxaa8-11-2-uxa97uxaa3}

\textbf{રેઝીસ્ટરની વ્યાખ્યા આપો અને તેનો એકમ જણાવો.}

\begin{solutionbox}
રેઝીસ્ટર એ એક ઇલેક્ટ્રોનિક ઘટક છે જે વિદ્યુત પ્રવાહના પ્રવાહનો વિરોધ
કરે છે. તેનો એકમ ઓહમ (Ω) છે.


{\def\LTcaptype{none} % do not increment counter
\vspace{-5pt}
\captionof{table}{રેઝીસ્ટરના ગુણધર્મો}
\vspace{-10pt}
\begin{longtable}[]{@{}ll@{}}
\toprule\noalign{}
ગુણધર્મ & વર્ણન \\
\midrule\noalign{}
\endhead
\bottomrule\noalign{}
\endlastfoot
સિમ્બોલ & ⏅ \\
એકમ & ઓહમ (Ω) \\
કાર્ય & પ્રવાહને મર્યાદિત કરે છે \\
\end{longtable}
}

\textbf{નિયમ યાદ રાખવા માટે}: ``રેઝીસ્ટર્સ વિરોધ કરે પ્રવાહ'' (ROP)

\end{solutionbox}
\subsubsection{પ્રશ્ન 1(2) [2
ગુણ]}\label{uxaaauxab0uxab6uxaa8-12-2-uxa97uxaa3}

\textbf{એક્ટીવ અને પેસીવ કમ્પોનન્ટના બે-બે ઉદાહરણ આપો.}

\begin{solutionbox}


{\def\LTcaptype{none} % do not increment counter
\vspace{-5pt}
\captionof{table}{ઇલેક્ટ્રોનિક ઘટકોનું વર્ગીકરણ}
\vspace{-10pt}
\begin{longtable}[]{@{}ll@{}}
\toprule\noalign{}
એક્ટીવ કમ્પોનન્ટ્સ & પેસીવ કમ્પોનન્ટ્સ \\
\midrule\noalign{}
\endhead
\bottomrule\noalign{}
\endlastfoot
1. ટ્રાન્ઝિસ્ટર & 1. રેઝીસ્ટર \\
2. ડાયોડ & 2. કેપેસિટર \\
\end{longtable}
}

\textbf{નિયમ યાદ રાખવા માટે}: ``TARD'' - Transistors And Resistors Differ

\end{solutionbox}
\subsubsection{પ્રશ્ન 1(3) [2
ગુણ]}\label{uxaaauxab0uxab6uxaa8-13-2-uxa97uxaa3}

\textbf{કોઈપણ બે અર્ધવાહક ઉપકરણોના સિમ્બોલ દોરો.}

\begin{solutionbox}

\textbf{આકૃતિ:}

\begin{center}
\textbf{Mermaid Diagram (Code)}
\begin{verbatim}
{Shaded}
{Highlighting}[]
graph TD
    subgraph Diode
        A[Plus] {-{-}{} B["|{}|"] {-}{-}{} C[Minus]}
    end

    subgraph NPN\_Transistor
        D[C] {-{-}{} E}
        F[E] {-{-}{} E}
        G[B] {-{-}{} E}
    end
{Highlighting}
{Shaded}
\end{verbatim}
\end{center}

\textbf{નિયમ યાદ રાખવા માટે}: ``ડાયોડ દિશા આપે, ટ્રાન્ઝિસ્ટર ટ્રાન્સફર કરે''

\end{solutionbox}
\subsubsection{પ્રશ્ન 1(4) [2
ગુણ]}\label{uxaaauxab0uxab6uxaa8-14-2-uxa97uxaa3}

\textbf{ઈન્ટ્રીસીક અને એક્સટ્રીસીક અર્ધવાહક વચ્ચેનો તફાવત લખો.}

\begin{solutionbox}


{\def\LTcaptype{none} % do not increment counter
\vspace{-5pt}
\captionof{table}{ઈન્ટ્રીસીક વિરુદ્ધ એક્સટ્રીસીક અર્ધવાહક}
\vspace{-10pt}
\begin{longtable}[]{@{}ll@{}}
\toprule\noalign{}
ઈન્ટ્રીસીક & એક્સટ્રીસીક \\
\midrule\noalign{}
\endhead
\bottomrule\noalign{}
\endlastfoot
અશુદ્ધિઓ વિનાના શુદ્ધ અર્ધવાહક & અશુદ્ધિઓ ઉમેરેલા અર્ધવાહક \\
હોલ્સ અને ઇલેક્ટ્રોન્સની સંખ્યા સમાન & હોલ્સ અને ઇલેક્ટ્રોન્સની સંખ્યા અસમાન \\
ઉદાહરણ: શુદ્ધ સિલિકોન, જર્મેનિયમ & ઉદાહરણ: ફોસ્ફરસ સાથે ડોપ કરેલ સિલિકોન \\
\end{longtable}
}

\textbf{નિયમ યાદ રાખવા માટે}: ``શુદ્ધ ઈન, ડોપ્ડ એક્સ''

\end{solutionbox}
\subsubsection{પ્રશ્ન 1(5) [2
ગુણ]}\label{uxaaauxab0uxab6uxaa8-15-2-uxa97uxaa3}

\textbf{LED નું આખું નામ \_\_\_\_\_\_\_\_\_\_\_\_\_\_\_\_\_.}

\begin{solutionbox}
LED નું આખું નામ \textbf{Light Emitting Diode} છે.

\textbf{આકૃતિ:}

\begin{center}
\textbf{Mermaid Diagram (Code)}
\begin{verbatim}
{Shaded}
{Highlighting}[]
graph LR
    A[Light] {-{-}{} B[Emitting] {-}{-}{} C[Diode]}
    style A fill:\#f96,stroke:\#333
    style B fill:\#9cf,stroke:\#333
    style C fill:\#f9f,stroke:\#333
{Highlighting}
{Shaded}
\end{verbatim}
\end{center}

\textbf{નિયમ યાદ રાખવા માટે}: ``પ્રકાશ ઉત્સર્જિત ડાયોડ'' (LED)

\end{solutionbox}
\subsubsection{પ્રશ્ન 1(6) [2
ગુણ]}\label{uxaaauxab0uxab6uxaa8-16-2-uxa97uxaa3}

\textbf{ફોટો ડાયોડના બે ઉપયોગો જણાવો.}

\begin{solutionbox}


{\def\LTcaptype{none} % do not increment counter
\vspace{-5pt}
\captionof{table}{ફોટો-ડાયોડના ઉપયોગો}
\vspace{-10pt}
\begin{longtable}[]{@{}ll@{}}
\toprule\noalign{}
ઉપયોગ & કેવી રીતે કામ કરે છે \\
\midrule\noalign{}
\endhead
\bottomrule\noalign{}
\endlastfoot
પ્રકાશ સેન્સર & પ્રકાશને વિદ્યુત પ્રવાહમાં રૂપાંતરિત કરે છે \\
ઓપ્ટિકલ કમ્યુનિકેશન & ફાઇબર ઓપ્ટિક્સમાં ઓપ્ટિકલ સિગ્નલ્સને શોધે છે \\
\end{longtable}
}

\textbf{નિયમ યાદ રાખવા માટે}: ``પ્રકાશ સેન્સિંગ કમ્યુનિકેશન'' (LSC)

\end{solutionbox}
\subsubsection{પ્રશ્ન 1(7) [2
ગુણ]}\label{uxaaauxab0uxab6uxaa8-17-2-uxa97uxaa3}

\textbf{ટ્રાન્ઝિસ્ટરના પ્રકારોની યાદી બનાવો અને તેમના પ્રતીકો દોરો.}

\begin{solutionbox}

\textbf{ટ્રાન્ઝિસ્ટરના પ્રકારો:}

\begin{enumerate}
\tightlist
\item
  NPN ટ્રાન્ઝિસ્ટર
\item
  PNP ટ્રાન્ઝિસ્ટર
\end{enumerate}

\textbf{આકૃતિ:}

\begin{center}
\textbf{Mermaid Diagram (Code)}
\begin{verbatim}
{Shaded}
{Highlighting}[]
graph TD
    subgraph "NPN"
    A[C] {-{-}{-} B {-}{-}{-} C[E]}
    D[B] {-{-}{-} B}
    end
    subgraph "PNP"
    E[E] {-{-}{-} F {-}{-}{-} G[C]}
    H[B] {-{-}{-} F}
    end
{Highlighting}
{Shaded}
\end{verbatim}
\end{center}

\textbf{નિયમ યાદ રાખવા માટે}: ``Not Pointing iN, Pointing outP''

\end{solutionbox}
\subsubsection{પ્રશ્ન 1(8) [2
ગુણ]}\label{uxaaauxab0uxab6uxaa8-18-2-uxa97uxaa3}

\textbf{જર્મેનિયમ અને સિલિકોન ડાયોડના ફોરવર્ડ વોલ્ટેજ ડ્રોપનું મૂલ્ય આપો.}

\begin{solutionbox}


{\def\LTcaptype{none} % do not increment counter
\vspace{-5pt}
\captionof{table}{ફોરવર્ડ વોલ્ટેજ ડ્રોપ મૂલ્યો}
\vspace{-10pt}
\begin{longtable}[]{@{}ll@{}}
\toprule\noalign{}
ડાયોડનો પ્રકાર & ફોરવર્ડ વોલ્ટેજ ડ્રોપ \\
\midrule\noalign{}
\endhead
\bottomrule\noalign{}
\endlastfoot
જર્મેનિયમ & 0.3V \\
સિલિકોન & 0.7V \\
\end{longtable}
}

\textbf{નિયમ યાદ રાખવા માટે}: ``જર્મેનિયમ ત્રણ, સિલિકોન સાત'' (0.3V, 0.7V)

\end{solutionbox}
\subsubsection{પ્રશ્ન 1(9) [2
ગુણ]}\label{uxaaauxab0uxab6uxaa8-19-2-uxa97uxaa3}

\textbf{\_\_\_\_\_\_\_\_\_\_\_\_\_\_\_\_\_ ડાયોડનો ઉપયોગ લાઇટ ડિટેક્ટર
તરીકે થઈ શકે છે.}

\begin{solutionbox}
\textbf{ફોટોડાયોડ}નો ઉપયોગ લાઇટ ડિટેક્ટર તરીકે થઈ શકે છે.

\textbf{આકૃતિ:}

\begin{center}
\textbf{Mermaid Diagram (Code)}
\begin{verbatim}
{Shaded}
{Highlighting}[]
graph LR
    A[Light] {-{-}{}|detected by| B[Photodiode]}
    B {-{-}{}|generates| C[Current]}
    style A fill:\#ff9,stroke:\#333
    style B fill:\#9cf,stroke:\#333
    style C fill:\#f96,stroke:\#333
{Highlighting}
{Shaded}
\end{verbatim}
\end{center}

\textbf{નિયમ યાદ રાખવા માટે}: ``ફોટો શોધે પ્રકાશ'' (PDL)

\end{solutionbox}
\subsubsection{પ્રશ્ન 1(10) [2
ગુણ]}\label{uxaaauxab0uxab6uxaa8-110-2-uxa97uxaa3}

\textbf{કોઈલના Q-factor ની વ્યાખ્યા લખો.}

\begin{solutionbox}
Q-factor (ક્વોલિટી ફેક્ટર) એ કોઈલના ઇન્ડક્ટિવ રિએક્ટન્સનો તેના
રેઝિસ્ટન્સ સાથેનો ગુણોત્તર છે, જે સૂચવે છે કે તે કેટલી કાર્યક્ષમતાથી ઊર્જા સંગ્રહિત કરે છે.


{\def\LTcaptype{none} % do not increment counter
\vspace{-5pt}
\captionof{table}{Q-Factor}
\vspace{-10pt}
\begin{longtable}[]{@{}ll@{}}
\toprule\noalign{}
પેરામીટર & વર્ણન \\
\midrule\noalign{}
\endhead
\bottomrule\noalign{}
\endlastfoot
સૂત્ર & Q = XL/R \\
ઉચ્ચ Q & સારી ગુણવત્તા, ઓછો ઊર્જા વ્યય \\
નીચો Q & નબળી ગુણવત્તા, વધુ ઊર્જા વ્યય \\
\end{longtable}
}

\textbf{નિયમ યાદ રાખવા માટે}: ``ગુણવત્તા બરાબર રિએક્ટન્સ વિભાજિત પ્રતિરોધ''
(QRR)

\end{solutionbox}
\subsection*{પ્રશ્ન 2(અ) [3
ગુણ]}\label{uxaaauxab0uxab6uxaa8-2uxa85-3-uxa97uxaa3}

\textbf{રેઝીસ્ટરનો કલર કોડીંગ સમજાવો.}

\begin{solutionbox}

રેઝીસ્ટર કલર કોડિંગ રંગીન પટ્ટીઓનો ઉપયોગ કરે છે જે પ્રતિરોધ મૂલ્ય અને ટોલરન્સ દર્શાવે
છે.


{\def\LTcaptype{none} % do not increment counter
\vspace{-5pt}
\captionof{table}{રેઝીસ્ટર કલર કોડ}
\vspace{-10pt}
\begin{longtable}[]{@{}lll@{}}
\toprule\noalign{}
રંગ & અંક & ગુણાંક \\
\midrule\noalign{}
\endhead
\bottomrule\noalign{}
\endlastfoot
કાળો & 0 & 10^{0} \\
બ્રાઉન & 1 & 10^{1} \\
લાલ & 2 & 10^{2} \\
નારંગી & 3 & 10^{3} \\
પીળો & 4 & 10^{4} \\
\end{longtable}
}

4-બેન્ડ રેઝિસ્ટર માટે:

\begin{itemize}
\tightlist
\item
  પ્રથમ બેન્ડ: પ્રથમ અંક
\item
  બીજી બેન્ડ: બીજો અંક
\item
  ત્રીજી બેન્ડ: ગુણાંક
\item
  ચોથી બેન્ડ: ટોલરન્સ
\end{itemize}

\textbf{નિયમ યાદ રાખવા માટે}: ``Bad Boys Race Our Young Girls But Violet
Generally Wins'' (રંગોના ક્રમમાં: કાળો, બ્રાઉન, લાલ, નારંગી, પીળો, લીલો,
વાદળી, જાંબલી, ગ્રે, સફેદ)

\end{solutionbox}
\subsection*{પ્રશ્ન 2(અ) અથવા [3
ગુણ]}\label{uxaaauxab0uxab6uxaa8-2uxa85-uxa85uxaa5uxab5-3-uxa97uxaa3}

\textbf{લાઈટ ડિપેન્ડન્ટ રેઝીસ્ટર તેની લાક્ષણિકતાઓ સાથે સમજાવો.}

\begin{solutionbox}

LDR એક રેઝિસ્ટર છે જેનો પ્રતિરોધ પ્રકાશની તીવ્રતા વધે ત્યારે ઘટે છે.

\textbf{LDR ની લાક્ષણિકતાઓ:}


{\def\LTcaptype{none} % do not increment counter
\vspace{-5pt}
\captionof{table}{LDR ગુણધર્મો}
\vspace{-10pt}
\begin{longtable}[]{@{}ll@{}}
\toprule\noalign{}
પેરામીટર & વર્તન \\
\midrule\noalign{}
\endhead
\bottomrule\noalign{}
\endlastfoot
અંધારી સ્થિતિ & ઉચ્ચ પ્રતિરોધ (MΩ) \\
પ્રકાશિત સ્થિતિ & નીચો પ્રતિરોધ (kΩ) \\
પ્રતિસાદ સમય & થોડી મિલિસેકન્ડ \\
\end{longtable}
}

\textbf{આકૃતિ:}

\begin{center}
\textbf{Mermaid Diagram (Code)}
\begin{verbatim}
{Shaded}
{Highlighting}[]
graph TD
    A[Increase Light] {-{-}{}|Causes| B[Decrease Resistance]}
    C[Decrease Light] {-{-}{}|Causes| D[Increase Resistance]}
    style A fill:\#ff9,stroke:\#333
    style B fill:\#9cf,stroke:\#333
    style C fill:\#999,stroke:\#333
    style D fill:\#f96,stroke:\#333
{Highlighting}
{Shaded}
\end{verbatim}
\end{center}

\textbf{નિયમ યાદ રાખવા માટે}: ``પ્રકાશ વધે, અવરોધ ઘટે'' (LVAG)

\end{solutionbox}
\subsection*{પ્રશ્ન 2(બ) [3
ગુણ]}\label{uxaaauxab0uxab6uxaa8-2uxaac-3-uxa97uxaa3}

\textbf{કેપેસિટરનું વર્ગીકરણ વિગતવાર સમજાવો.}

\begin{solutionbox}

કેપેસિટર્સને ડાયઇલેક્ટ્રિક મટીરિયલ અને બાંધકામના આધારે વર્ગીકૃત કરવામાં આવે છે.


{\def\LTcaptype{none} % do not increment counter
\vspace{-5pt}
\captionof{table}{કેપેસિટર વર્ગીકરણ}
\vspace{-10pt}
\begin{longtable}[]{@{}lll@{}}
\toprule\noalign{}
પ્રકાર & ડાયઇલેક્ટ્રિક & ઉપયોગો \\
\midrule\noalign{}
\endhead
\bottomrule\noalign{}
\endlastfoot
સિરામિક & સિરામિક & ઉચ્ચ આવૃત્તિ \\
ઇલેક્ટ્રોલિટિક & એલ્યુમિનિયમ ઓક્સાઇડ & પાવર સપ્લાય \\
પોલિએસ્ટર & પ્લાસ્ટિક ફિલ્મ & સામાન્ય હેતુ \\
ટેન્ટલમ & ટેન્ટલમ ઓક્સાઇડ & નાના, ઉચ્ચ ક્ષમતા \\
\end{longtable}
}

\textbf{આકૃતિ:}

\begin{center}
\textbf{Mermaid Diagram (Code)}
\begin{verbatim}
{Shaded}
{Highlighting}[]
graph TD
    A[Capacitors] {-{-}{} B[Fixed]}
    A {-{-}{} C[Variable]}
    B {-{-}{} D[Ceramic]}
    B {-{-}{} E[Electrolytic]}
    B {-{-}{} F[Polyester/Film]}
    C {-{-}{} G[Air Gang]}
    C {-{-}{} H[Trimmer]}
    style A fill:\#f96,stroke:\#333
{Highlighting}
{Shaded}
\end{verbatim}
\end{center}

\textbf{નિયમ યાદ રાખવા માટે}: ``CEPT'' (Ceramic, Electrolytic, Polyester,
Tantalum)

\end{solutionbox}
\subsection*{પ્રશ્ન 2(બ) અથવા [3
ગુણ]}\label{uxaaauxab0uxab6uxaa8-2uxaac-uxa85uxaa5uxab5-3-uxa97uxaa3}

\textbf{ઈન્ડક્ટરનું વર્ગીકરણ વિગતવાર સમજાવો.}

\begin{solutionbox}

ઇન્ડક્ટર્સને કોર સામગ્રી અને બાંધકામના આધારે વર્ગીકૃત કરવામાં આવે છે.


{\def\LTcaptype{none} % do not increment counter
\vspace{-5pt}
\captionof{table}{ઇન્ડક્ટર વર્ગીકરણ}
\vspace{-10pt}
\begin{longtable}[]{@{}lll@{}}
\toprule\noalign{}
પ્રકાર & કોર & લાક્ષણિકતાઓ \\
\midrule\noalign{}
\endhead
\bottomrule\noalign{}
\endlastfoot
એર કોર & હવા & ઓછો ઇન્ડક્ટન્સ, ઓછા નુકશાન \\
આયર્ન કોર & લોખંડ & ઉચ્ચ ઇન્ડક્ટન્સ, ઉચ્ચ નુકશાન \\
ફેરાઇટ કોર & ફેરાઇટ & મધ્યમ ઇન્ડક્ટન્સ, ઓછા નુકશાન \\
ટોરોઇડલ & રિંગ આકારનું & ઉચ્ચ કાર્યક્ષમતા, ઓછું EMI \\
\end{longtable}
}

\textbf{આકૃતિ:}

\begin{center}
\textbf{Mermaid Diagram (Code)}
\begin{verbatim}
{Shaded}
{Highlighting}[]
graph TD
    A[Inductors] {-{-}{} B[Air Core]}
    A {-{-}{} C[Iron Core]}
    A {-{-}{} D[Ferrite Core]}
    A {-{-}{} E[Toroidal]}
    style A fill:\#9cf,stroke:\#333
{Highlighting}
{Shaded}
\end{verbatim}
\end{center}

\textbf{નિયમ યાદ રાખવા માટે}: ``હવા લોખંડ ફેરાઇટ ટોરોઇડ'' (AIFT)

\end{solutionbox}
\subsection*{પ્રશ્ન 2(ક) [4
ગુણ]}\label{uxaaauxab0uxab6uxaa8-2uxa95-4-uxa97uxaa3}

\textbf{ફેરાડેનો ઈલેક્ટ્રોમેગ્નેટીક ઈન્ડક્શનના નિયમો લખો તથા સમજાવો.}

\begin{solutionbox}

\textbf{ફેરાડેનો પ્રથમ નિયમ:} જ્યારે વાહક સાથે જોડાયેલ ચુંબકીય ક્ષેત્ર બદલાય છે, ત્યારે
વાહકમાં EMF પ્રેરિત થાય છે.

\textbf{ફેરાડેનો બીજો નિયમ:} પ્રેરિત EMFનો પરિમાણ ચુંબકીય ફ્લક્સના પરિવર્તનના
દરના સમપ્રમાણમાં હોય છે.


{\def\LTcaptype{none} % do not increment counter
\vspace{-5pt}
\captionof{table}{ફેરાડેના નિયમોનો સારાંશ}
\vspace{-10pt}
\begin{longtable}[]{@{}lll@{}}
\toprule\noalign{}
નિયમ & વિધાન & સૂત્ર \\
\midrule\noalign{}
\endhead
\bottomrule\noalign{}
\endlastfoot
પ્રથમ નિયમ & ચુંબકીય ક્ષેત્રમાં ફેરફારથી EMF પ્રેરિત થાય છે & - \\
બીજો નિયમ & EMF ∝ ફ્લક્સના પરિવર્તનનો દર & E = -N(dΦ/dt) \\
\end{longtable}
}

\textbf{આકૃતિ:}

\begin{center}
\textbf{Mermaid Diagram (Code)}
\begin{verbatim}
{Shaded}
{Highlighting}[]
graph LR
    A[Moving Magnet] {-{-}{}|Creates| B[Changing Magnetic Field]}
    B {-{-}{}|Induces| C[EMF in Conductor]}
    style A fill:\#f96,stroke:\#333
    style B fill:\#9cf,stroke:\#333
    style C fill:\#ff9,stroke:\#333
{Highlighting}
{Shaded}
\end{verbatim}
\end{center}

\textbf{નિયમ યાદ રાખવા માટે}: ``ચુંબકીય ક્ષેત્ર બદલાય, વિદ્યુત પ્રવાહ પેદા થાય''
(CMFCEC)

\end{solutionbox}
\subsection*{પ્રશ્ન 2(ક) અથવા [4
ગુણ]}\label{uxaaauxab0uxab6uxaa8-2uxa95-uxa85uxaa5uxab5-4-uxa97uxaa3}

\textbf{કેપેસિટરના સ્પેસિફીકેશન લખો તથા કોઈ પણ બે વિગતવાર સમજાવો.}

\begin{solutionbox}

\textbf{કેપેસિટરના સ્પેસિફિકેશન:}

\begin{enumerate}
\tightlist
\item
  કેપેસિટન્સ મૂલ્ય
\item
  વોલ્ટેજ રેટિંગ
\item
  ટોલરન્સ
\item
  લીકેજ કરંટ
\item
  તાપમાન ગુણાંક
\end{enumerate}

\textbf{વિગતવાર સમજૂતી:}

\textbf{કેપેસિટન્સ મૂલ્ય:} દર વોલ્ટ પર કેપેસિટર કેટલો ચાર્જ સંગ્રહિત કરી શકે છે, જે ફેરડ
(F)માં માપવામાં આવે છે.

\textbf{વોલ્ટેજ રેટિંગ:} મહત્તમ વોલ્ટેજ જે કેપેસિટરને નુકસાન કર્યા વિના લાગુ કરી શકાય
છે.


{\def\LTcaptype{none} % do not increment counter
\vspace{-5pt}
\captionof{table}{કેપેસિટર સ્પેસિફિકેશન}
\vspace{-10pt}
\begin{longtable}[]{@{}lll@{}}
\toprule\noalign{}
સ્પેસિફિકેશન & વર્ણન & સામાન્ય મૂલ્યો \\
\midrule\noalign{}
\endhead
\bottomrule\noalign{}
\endlastfoot
કેપેસિટન્સ & ચાર્જ સંગ્રહ ક્ષમતા & pF થી mF \\
વોલ્ટેજ રેટિંગ & મહત્તમ સુરક્ષિત વોલ્ટેજ & 16V, 25V, 50V, વગેરે \\
\end{longtable}
}

\textbf{આકૃતિ:}

\begin{center}
\textbf{Mermaid Diagram (Code)}
\begin{verbatim}
{Shaded}
{Highlighting}[]
graph TD
    A[Capacitor Specifications] {-{-}{} B[Capacitance Value]}
    A {-{-}{} C[Voltage Rating]}
    A {-{-}{} D[Tolerance]}
    A {-{-}{} E[Leakage Current]}
    A {-{-}{} F[Temperature Coefficient]}
    style A fill:\#9cf,stroke:\#333
{Highlighting}
{Shaded}
\end{verbatim}
\end{center}

\textbf{નિયમ યાદ રાખવા માટે}: ``કેપેસિટર્સ વૉલ્ટેજ ટોલરન્ટ ઓફ લો ટેમ્પરેચર''
(CVTLT)

\end{solutionbox}
\subsection*{પ્રશ્ન 2(ડ) [4
ગુણ]}\label{uxaaauxab0uxab6uxaa8-2uxaa1-4-uxa97uxaa3}

\textbf{47Ω\pm5\% મા\hspace{0pt}ટે કલર કોડ લખો.}

\begin{solutionbox}

47Ω\pm5\% રેઝિસ્ટર માટે, કલર બેન્ડ્સ આ છે:


{\def\LTcaptype{none} % do not increment counter
\vspace{-5pt}
\captionof{table}{47Ω\pm5\% માટે કલર બેન્ડ્સ}
\vspace{-10pt}
\begin{longtable}[]{@{}lll@{}}
\toprule\noalign{}
બેન્ડ & રંગ & રજૂ કરે છે \\
\midrule\noalign{}
\endhead
\bottomrule\noalign{}
\endlastfoot
1લી બેન્ડ & પીળો & 4 \\
2જી બેન્ડ & જાંબલી & 7 \\
3જી બેન્ડ & કાળો & \times10^{0} \\
4થી બેન્ડ & સોનેરી & \pm5\% \\
\end{longtable}
}

\textbf{આકૃતિ:}

\begin{center}
\textbf{Mermaid Diagram (Code)}
\begin{verbatim}
{Shaded}
{Highlighting}[]
graph LR
    A[Yellow] {-{-}{}|4| B[Violet] {-}{-}{}|7| C[Black] {-}{-}{}|10^{0}| D[Gold] {-}{-}{}|5\%| E[47Ω5\%]}
    style A fill:\#ff9,stroke:\#333
    style B fill:\#f0f,stroke:\#333
    style C fill:\#000,stroke:\#fff
    style D fill:\#fd0,stroke:\#333
    style E fill:\#fff,stroke:\#333
{Highlighting}
{Shaded}
\end{verbatim}
\end{center}

\textbf{નિયમ યાદ રાખવા માટે}: ``પીળો જાંબલી કાળો સોનેરી'' (રંગોનો ક્રમ)

\end{solutionbox}
\subsection*{પ્રશ્ન 2(ડ) અથવા [4
ગુણ]}\label{uxaaauxab0uxab6uxaa8-2uxaa1-uxa85uxaa5uxab5-4-uxa97uxaa3}

\textbf{આપેલ કલર કોડ માટે રેઝીસ્ટરની કિંમત તથા ટોલરન્સ શોધો: Brown, Black,
yellow.}

\begin{solutionbox}


{\def\LTcaptype{none} % do not increment counter
\vspace{-5pt}
\captionof{table}{Brown, Black, Yellow નું અર્થઘટન}
\vspace{-10pt}
\begin{longtable}[]{@{}llll@{}}
\toprule\noalign{}
બેન્ડ & રંગ & મૂલ્ય & અર્થ \\
\midrule\noalign{}
\endhead
\bottomrule\noalign{}
\endlastfoot
1લી & બ્રાઉન & 1 & પ્રથમ અંક \\
2જી & કાળો & 0 & બીજો અંક \\
3જી & પીળો & 10^{4} & ગુણાંક \\
\end{longtable}
}

ગણતરી: 1લો અંક: 1 2જો અંક: 0 ગુણાંક: 10^{4}

મૂલ્ય = 10 \times 10^{4} = 100,000Ω = 100kΩ

4થી બેન્ડનો અભાવ એટલે \pm20\% ટોલરન્સ

\textbf{આકૃતિ:}

\begin{center}
\textbf{Mermaid Diagram (Code)}
\begin{verbatim}
{Shaded}
{Highlighting}[]
graph LR
    A[Brown] {-{-}{}|1| B[Black] {-}{-}{}|0| C[Yellow] {-}{-}{}|10^{4}| D[100kΩ 20\%]}
    style A fill:\#a52a2a,stroke:\#333
    style B fill:\#000,stroke:\#fff
    style C fill:\#ff0,stroke:\#333
    style D fill:\#fff,stroke:\#333
{Highlighting}
{Shaded}
\end{verbatim}
\end{center}

\textbf{નિયમ યાદ રાખવા માટે}: ``બ્રાઉન બ્લેક યલો'' (BBY)

\end{solutionbox}
\subsection*{પ્રશ્ન 3(અ) [3
ગુણ]}\label{uxaaauxab0uxab6uxaa8-3uxa85-3-uxa97uxaa3}

\textbf{ડોપિંગની વ્યાખ્યા લખો. ડોપિંગથી બનતા અર્ધવાહકોના નામ તથા ઉદાહરણ આપો.}

\begin{solutionbox}

ડોપિંગ એ શુદ્ધ અર્ધવાહકમાં અશુદ્ધિઓ ઉમેરવાની પ્રક્રિયા છે જે તેના વિદ્યુત ગુણધર્મોને
સંશોધિત કરે છે.


{\def\LTcaptype{none} % do not increment counter
\vspace{-5pt}
\captionof{table}{ડોપ્ડ અર્ધવાહકો}
\vspace{-10pt}
\begin{longtable}[]{@{}
  >{\raggedright\arraybackslash}p{(\linewidth - 6\tabcolsep) * \real{0.1250}}
  >{\raggedright\arraybackslash}p{(\linewidth - 6\tabcolsep) * \real{0.2917}}
  >{\raggedright\arraybackslash}p{(\linewidth - 6\tabcolsep) * \real{0.1875}}
  >{\raggedright\arraybackslash}p{(\linewidth - 6\tabcolsep) * \real{0.3958}}@{}}
\toprule\noalign{}
\begin{minipage}[b]{\linewidth}\raggedright
પ્રકાર
\end{minipage} & \begin{minipage}[b]{\linewidth}\raggedright
ઉમેરેલ ડોપન્ટ
\end{minipage} & \begin{minipage}[b]{\linewidth}\raggedright
ઉદાહરણ
\end{minipage} & \begin{minipage}[b]{\linewidth}\raggedright
મુખ્ય વાહકો
\end{minipage} \\
\midrule\noalign{}
\endhead
\bottomrule\noalign{}
\endlastfoot
P-type & ત્રિસંયોજક (બોરોન, ગેલિયમ) & બોરોન સાથે ડોપ કરેલ સિલિકોન & હોલ્સ \\
N-type & પંચસંયોજક (ફોસ્ફરસ, આર્સેનિક) & ફોસ્ફરસ સાથે ડોપ કરેલ સિલિકોન &
ઇલેક્ટ્રોન્સ \\
\end{longtable}
}

\textbf{આકૃતિ:}

\begin{center}
\textbf{Mermaid Diagram (Code)}
\begin{verbatim}
{Shaded}
{Highlighting}[]
graph LR
    A[Pure Semiconductor] {-{-}{} B[Add Trivalent Impurity] {-}{-}{} C[P{-}type]}
    A {-{-}{} D[Add Pentavalent Impurity] {-}{-}{} E[N{-}type]}
    style A fill:\#9cf,stroke:\#333
    style C fill:\#f96,stroke:\#333
    style E fill:\#99f,stroke:\#333
{Highlighting}
{Shaded}
\end{verbatim}
\end{center}

\textbf{નિયમ યાદ રાખવા માટે}: ``પોઝિટિવમાં પ્લસ હોલ્સ, નેગેટિવમાં નંબર
ઇલેક્ટ્રોન્સ'' (PHNE)

\end{solutionbox}
\subsection*{પ્રશ્ન 3(અ) અથવા [3
ગુણ]}\label{uxaaauxab0uxab6uxaa8-3uxa85-uxa85uxaa5uxab5-3-uxa97uxaa3}

\textbf{વ્યાખ્યા લખો: રીપલ ફેક્ટર, પીક ઈનવર્સ વોલ્ટેજ, રેક્ટીફીકેશન એફીસીયન્સી.}

\begin{solutionbox}


{\def\LTcaptype{none} % do not increment counter
\vspace{-5pt}
\captionof{table}{રેક્ટિફાયર પદો}
\vspace{-10pt}
\begin{longtable}[]{@{}
  >{\raggedright\arraybackslash}p{(\linewidth - 4\tabcolsep) * \real{0.2222}}
  >{\raggedright\arraybackslash}p{(\linewidth - 4\tabcolsep) * \real{0.4444}}
  >{\raggedright\arraybackslash}p{(\linewidth - 4\tabcolsep) * \real{0.3333}}@{}}
\toprule\noalign{}
\begin{minipage}[b]{\linewidth}\raggedright
પદ
\end{minipage} & \begin{minipage}[b]{\linewidth}\raggedright
વ્યાખ્યા
\end{minipage} & \begin{minipage}[b]{\linewidth}\raggedright
સૂત્ર
\end{minipage} \\
\midrule\noalign{}
\endhead
\bottomrule\noalign{}
\endlastfoot
રિપલ ફેક્ટર & રેક્ટિફાઇડ આઉટપુટમાં AC ઘટકનું માપ & r = Vrms(AC)/Vdc \\
પીક ઇન્વર્સ વોલ્ટેજ & મહત્તમ રિવર્સ વોલ્ટેજ જે ડાયોડ સહન કરી શકે છે & - \\
રેક્ટિફિકેશન એફિસિયન્સી & DC આઉટપુટ પાવરનો AC ઇનપુટ પાવર સાથેનો ગુણોત્તર & η =
(Pdc/Pac) \times 100\% \\
\end{longtable}
}

\textbf{આકૃતિ:}

\begin{center}
\textbf{Mermaid Diagram (Code)}
\begin{verbatim}
{Shaded}
{Highlighting}[]
graph TD
    A[Rectifier Parameters] {-{-}{} B[Ripple Factor]}
    A {-{-}{} C[Peak Inverse Voltage]}
    A {-{-}{} D[Rectification Efficiency]}
    style A fill:\#9cf,stroke:\#333
{Highlighting}
{Shaded}
\end{verbatim}
\end{center}

\textbf{નિયમ યાદ રાખવા માટે}: ``રિપલ્સ પીક એફિશિયન્ટલી'' (RPE)

\end{solutionbox}
\subsection*{પ્રશ્ન 3(બ) [3
ગુણ]}\label{uxaaauxab0uxab6uxaa8-3uxaac-3-uxa97uxaa3}

\textbf{ક્રિસ્ટલ ડાયોડનું કાર્ય સમજાવો.}

\begin{solutionbox}

ક્રિસ્ટલ ડાયોડ એ પોઇન્ટ-કોન્ટેક્ટ ડાયોડ છે જે અર્ધવાહક ક્રિસ્ટલ સાથે બનાવવામાં આવે છે.


{\def\LTcaptype{none} % do not increment counter
\vspace{-5pt}
\captionof{table}{ક્રિસ્ટલ ડાયોડના ગુણધર્મો}
\vspace{-10pt}
\begin{longtable}[]{@{}ll@{}}
\toprule\noalign{}
ગુણધર્મ & વર્ણન \\
\midrule\noalign{}
\endhead
\bottomrule\noalign{}
\endlastfoot
બાંધકામ & અર્ધવાહક ક્રિસ્ટલ પર મેટલ પોઇન્ટ કોન્ટેક્ટ \\
કાર્ય & ઉચ્ચ આવૃત્તિના સિગ્નલનું રેક્ટિફિકેશન \\
ઉપયોગ & રેડિયો સિગ્નલ શોધ \\
\end{longtable}
}

\textbf{આકૃતિ:}

\begin{center}
\textbf{Mermaid Diagram (Code)}
\begin{verbatim}
{Shaded}
{Highlighting}[]
graph LR
    A[RF Signal] {-{-}{} B[Crystal Diode] {-}{-}{} C[Rectified Signal]}
    style A fill:\#9cf,stroke:\#333
    style B fill:\#f96,stroke:\#333
    style C fill:\#9f9,stroke:\#333
{Highlighting}
{Shaded}
\end{verbatim}
\end{center}

\textbf{નિયમ યાદ રાખવા માટે}: ``ક્રિસ્ટલ શોધે રેડિયો ફ્રીક્વન્સી'' (CDRF)

\end{solutionbox}
\subsection*{પ્રશ્ન 3(બ) અથવા [3
ગુણ]}\label{uxaaauxab0uxab6uxaa8-3uxaac-uxa85uxaa5uxab5-3-uxa97uxaa3}

\textbf{ફોટોડાયોડનું કાર્ય સમજાવો.}

\begin{solutionbox}

ફોટોડાયોડ રિવર્સ બાયસમાં ઓપરેટ કરવામાં આવે ત્યારે પ્રકાશ ઊર્જાને વિદ્યુત પ્રવાહમાં
રૂપાંતરિત કરે છે.


{\def\LTcaptype{none} % do not increment counter
\vspace{-5pt}
\captionof{table}{ફોટોડાયોડની લાક્ષણિકતાઓ}
\vspace{-10pt}
\begin{longtable}[]{@{}ll@{}}
\toprule\noalign{}
પેરામીટર & વર્તન \\
\midrule\noalign{}
\endhead
\bottomrule\noalign{}
\endlastfoot
પ્રકાશ સ્થિતિ & ઇલેક્ટ્રોન-હોલ જોડી ઉત્પન્ન કરે છે \\
રિવર્સ કરંટ & પ્રકાશની તીવ્રતા સાથે વધે છે \\
ગતિ & ઝડપી પ્રતિસાદ સમય \\
\end{longtable}
}

\textbf{આકૃતિ:}

\begin{center}
\textbf{Mermaid Diagram (Code)}
\begin{verbatim}
{Shaded}
{Highlighting}[]
graph LR
    A[Light] {-{-}{}|Strikes| B[PN Junction]}
    B {-{-}{}|Creates| C[Electron{-}Hole Pairs]}
    C {-{-}{}|Produces| D[Current Flow]}
    style A fill:\#ff9,stroke:\#333
    style D fill:\#9cf,stroke:\#333
{Highlighting}
{Shaded}
\end{verbatim}
\end{center}

\textbf{નિયમ યાદ રાખવા માટે}: ``પ્રકાશ આવે, કરંટ જાય'' (LICO)

\end{solutionbox}
\subsection*{પ્રશ્ન 3(ક) [4
ગુણ]}\label{uxaaauxab0uxab6uxaa8-3uxa95-4-uxa97uxaa3}

\textbf{સર્કિટ તથા વેવફોર્મ દોરી હાફ-વેવ રેક્ટીફાયર સમજાવો.}

\begin{solutionbox}

હાફ-વેવ રેક્ટિફાયર AC ને પલ્સેટિંગ DCમાં રૂપાંતરિત કરે છે, માત્ર પોઝિટિવ હાફ સાયકલ
દરમિયાન પ્રવાહને પસાર કરીને.

\textbf{સર્કિટ ડાયાગ્રામ:}

\begin{center}
\textbf{Mermaid Diagram (Code)}
\begin{verbatim}
{Shaded}
{Highlighting}[]
graph LR
    A[AC Input] {-{-}{-} B[Transformer] {-}{-}{-} C[Diode] {-}{-}{-} D[Load Resistor] {-}{-}{-} E[Ground]}
    E {-{-}{-} A}
    style A fill:\#9cf,stroke:\#333
    style D fill:\#f96,stroke:\#333
{Highlighting}
{Shaded}
\end{verbatim}
\end{center}

\textbf{વેવફોર્મ્સ:}

\begin{center}
\textbf{Mermaid Diagram (Code)}
\begin{verbatim}
{Shaded}
{Highlighting}[]
graph TD
    subgraph "Input AC"
    A[+Vp] {-{-}{-} B[(0)] {-}{-}{-} C[{-}Vp]}
    end
    subgraph "Output DC"
    D[+Vp] {-{-}{-} E[(0)] {-}{-}{-} F[(0)]}
    end
    style A fill:\#9cf,stroke:\#333
    style C fill:\#9cf,stroke:\#333
    style D fill:\#f96,stroke:\#333
{Highlighting}
{Shaded}
\end{verbatim}
\end{center}


{\def\LTcaptype{none} % do not increment counter
\vspace{-5pt}
\captionof{table}{હાફ-વેવ રેક્ટિફાયરની લાક્ષણિકતાઓ}
\vspace{-10pt}
\begin{longtable}[]{@{}ll@{}}
\toprule\noalign{}
પેરામીટર & મૂલ્ય \\
\midrule\noalign{}
\endhead
\bottomrule\noalign{}
\endlastfoot
રિપલ ફેક્ટર & 1.21 \\
કાર્યક્ષમતા & 40.6\% \\
આઉટપુટ ફ્રીક્વન્સી & ઇનપુટ જેવી જ \\
\end{longtable}
}

\textbf{નિયમ યાદ રાખવા માટે}: ``અર્ધ તરંગ અર્ધ પસાર'' (HWPH)

\end{solutionbox}
\subsection*{પ્રશ્ન 3(ક) અથવા [4
ગુણ]}\label{uxaaauxab0uxab6uxaa8-3uxa95-uxa85uxaa5uxab5-4-uxa97uxaa3}

\textbf{સર્કિટ તથા વેવફોર્મ દોરી ફુલ-વેવ રેક્ટીફાયર સમજાવો.}

\begin{solutionbox}

ફુલ-વેવ રેક્ટિફાયર AC ઇનપુટના બંને અર્ધ ભાગોને પલ્સેટિંગ DC આઉટપુટમાં રૂપાંતરિત કરે છે.

\textbf{સર્કિટ ડાયાગ્રામ (બ્રિજ પ્રકાર):}

\begin{center}
\textbf{Mermaid Diagram (Code)}
\begin{verbatim}
{Shaded}
{Highlighting}[]
graph LR
    A[AC Input] {-{-}{-} B[D1]}
    A {-{-}{-} C[D3]}
    B {-{-}{-} D[D2] {-}{-}{-} E[+Output]}
    C {-{-}{-} F[D4] {-}{-}{-} G[{-}Output]}
    E {-{-}{-} H[Load] {-}{-}{-} G}
    style A fill:\#9cf,stroke:\#333
    style H fill:\#f96,stroke:\#333
{Highlighting}
{Shaded}
\end{verbatim}
\end{center}

\textbf{વેવફોર્મ્સ:}

\begin{center}
\textbf{Mermaid Diagram (Code)}
\begin{verbatim}
{Shaded}
{Highlighting}[]
graph TD
    subgraph "Input AC"
    A[+Vp] {-{-}{-} B[(0)] {-}{-}{-} C[{-}Vp] {-}{-}{-} B}
    end
    subgraph "Output DC"
    D[+Vp] {-{-}{-} E[(0)] {-}{-}{-} D}
    end
    style A fill:\#9cf,stroke:\#333
    style C fill:\#9cf,stroke:\#333
    style D fill:\#f96,stroke:\#333
{Highlighting}
{Shaded}
\end{verbatim}
\end{center}


{\def\LTcaptype{none} % do not increment counter
\vspace{-5pt}
\captionof{table}{ફુલ-વેવ રેક્ટિફાયરની લાક્ષણિકતાઓ}
\vspace{-10pt}
\begin{longtable}[]{@{}ll@{}}
\toprule\noalign{}
પેરામીટર & મૂલ્ય \\
\midrule\noalign{}
\endhead
\bottomrule\noalign{}
\endlastfoot
રિપલ ફેક્ટર & 0.48 \\
કાર્યક્ષમતા & 81.2\% \\
આઉટપુટ ફ્રીક્વન્સી & ઇનપુટથી બમણી \\
\end{longtable}
}

\textbf{નિયમ યાદ રાખવા માટે}: ``પૂર્ણ તરંગ પૂર્ણ ઉપયોગ'' (FWMFU)

\end{solutionbox}
\subsection*{પ્રશ્ન 3(ડ) [4
ગુણ]}\label{uxaaauxab0uxab6uxaa8-3uxaa1-4-uxa97uxaa3}

\textbf{PN-જંક્શન ડાયોડના VI લાક્ષણિકતાઓ આકૃતિ દોરી સમજાવો.}

\begin{solutionbox}

\textbf{VI લાક્ષણિકતાઓ:}

\begin{center}
\textbf{Mermaid Diagram (Code)}
\begin{verbatim}
{Shaded}
{Highlighting}[]
graph TD
    subgraph "Forward Bias"
    A[Vf] {-{-}{} B[If]}
    end
    subgraph "Reverse Bias"
    C[Vr] {-{-}{} D[Ir]}
    E[Breakdown] {-{-}{} F[Reverse Current Increases]}
    end
    style A fill:\#9cf,stroke:\#333
    style C fill:\#f96,stroke:\#333
    style E fill:\#f00,stroke:\#333
{Highlighting}
{Shaded}
\end{verbatim}
\end{center}


{\def\LTcaptype{none} % do not increment counter
\vspace{-5pt}
\captionof{table}{PN જંક્શન ડાયોડની લાક્ષણિકતાઓ}
\vspace{-10pt}
\begin{longtable}[]{@{}ll@{}}
\toprule\noalign{}
પ્રદેશ & વર્તન \\
\midrule\noalign{}
\endhead
\bottomrule\noalign{}
\endlastfoot
ફોરવર્ડ બાયસ & 0.7V (Si) પછી કરંટ એક્સપોનેન્શિયલી વધે છે \\
રિવર્સ બાયસ & ખૂબ નાનો લીકેજ કરંટ વહે છે \\
બ્રેકડાઉન & ઉચ્ચ રિવર્સ વોલ્ટેજ પર થાય છે, કરંટ ઝડપથી વધે છે \\
\end{longtable}
}

\textbf{ફોરવર્ડ બાયસ}: P-સાઇડ પર પોઝિટિવ વોલ્ટેજ, થ્રેશોલ્ડ પછી કરંટ સરળતાથી વહે
છે. \textbf{રિવર્સ બાયસ}: N-સાઇડ પર પોઝિટિવ વોલ્ટેજ, માત્ર નાનો લીકેજ કરંટ વહે
છે.

\textbf{નિયમ યાદ રાખવા માટે}: ``ફોરવર્ડ ફ્લો, રિવર્સ રેસ્ટ્રિક્ટ'' (FFRR)

\end{solutionbox}
\subsection*{પ્રશ્ન 3(ડ) અથવા [4
ગુણ]}\label{uxaaauxab0uxab6uxaa8-3uxaa1-uxa85uxaa5uxab5-4-uxa97uxaa3}

\textbf{P-type અને N-type અર્ધવાહક વચ્ચેનો તફાવત લખો.}

\begin{solutionbox}


{\def\LTcaptype{none} % do not increment counter
\vspace{-5pt}
\captionof{table}{P-type vs N-type અર્ધવાહક}
\vspace{-10pt}
\begin{longtable}[]{@{}lll@{}}
\toprule\noalign{}
ગુણધર્મ & P-type & N-type \\
\midrule\noalign{}
\endhead
\bottomrule\noalign{}
\endlastfoot
ડોપન્ટ & ત્રિસંયોજક (બોરોન, ગેલિયમ) & પંચસંયોજક (ફોસ્ફરસ, આર્સેનિક) \\
મુખ્ય વાહકો & હોલ્સ & ઇલેક્ટ્રોન્સ \\
ગૌણ વાહકો & ઇલેક્ટ્રોન્સ & હોલ્સ \\
વિદ્યુત ચાર્જ & સાપેક્ષ રીતે પોઝિટિવ & સાપેક્ષ રીતે નેગેટિવ \\
વાહકતા & N-type કરતાં ઓછી & P-type કરતાં વધારે \\
\end{longtable}
}

\textbf{આકૃતિ:}

\begin{center}
\textbf{Mermaid Diagram (Code)}
\begin{verbatim}
{Shaded}
{Highlighting}[]
graph LR
    subgraph "P{-type"}
    A[Silicon] {-{-}{-} B[Boron]}
    C[Holes] {-{-}{-} D["{}+"]}
    end
    subgraph "N{-type"}
    E[Silicon] {-{-}{-} F[Phosphorus]}
    G[Electrons] {-{-}{-} H["{}{-}"]}
    end
    style C fill:\#f96,stroke:\#333
    style G fill:\#9cf,stroke:\#333
{Highlighting}
{Shaded}
\end{verbatim}
\end{center}

\textbf{નિયમ યાદ રાખવા માટે}: ``પોઝિટિવમાં પ્લસ હોલ્સ, નેગેટિવમાં નંબર
ઇલેક્ટ્રોન્સ'' (PHNE)

\end{solutionbox}
\subsection*{પ્રશ્ન 4(અ) [3
ગુણ]}\label{uxaaauxab0uxab6uxaa8-4uxa85-3-uxa97uxaa3}

\textbf{LED ની કાર્યપદ્ધતિ સમજાવો.}

\begin{solutionbox}

LED (લાઇટ એમિટિંગ ડાયોડ) ફોરવર્ડ બાયસ થયેલ હોય ત્યારે ઇલેક્ટ્રોન-હોલ રિકોમ્બિનેશનને
કારણે પ્રકાશ ઉત્સર્જિત કરે છે.

\textbf{કાર્યપદ્ધતિનો સિદ્ધાંત:} જ્યારે ફોરવર્ડ બાયસ કરવામાં આવે છે, ત્યારે N-સાઇડથી
ઇલેક્ટ્રોન્સ P-સાઇડ તરફ ગતિ કરે છે અને હોલ્સ સાથે રિકોમ્બાઇન થાય છે, જેના પરિણામે
ફોટોન્સ (પ્રકાશ) તરીકે ઊર્જા છોડે છે.


{\def\LTcaptype{none} % do not increment counter
\vspace{-5pt}
\captionof{table}{LED ઓપરેશન}
\vspace{-10pt}
\begin{longtable}[]{@{}ll@{}}
\toprule\noalign{}
પ્રક્રિયા & પરિણામ \\
\midrule\noalign{}
\endhead
\bottomrule\noalign{}
\endlastfoot
ફોરવર્ડ બાયસ & કરંટ વહે છે \\
ઇલેક્ટ્રોન-હોલ રિકોમ્બિનેશન & ઊર્જા રિલીઝ \\
એનર્જી બેન્ડ ગેપ & રંગ નક્કી કરે છે \\
\end{longtable}
}

\textbf{આકૃતિ:}

\begin{center}
\textbf{Mermaid Diagram (Code)}
\begin{verbatim}
{Shaded}
{Highlighting}[]
graph LR
    A[Forward Bias] {-{-}{}|Causes| B[Current Flow]}
    B {-{-}{}|Creates| C[Electron{-}Hole Recombination]}
    C {-{-}{}|Releases| D[Photons or Light]}
    style A fill:\#9cf,stroke:\#333
    style D fill:\#ff9,stroke:\#333
{Highlighting}
{Shaded}
\end{verbatim}
\end{center}

\textbf{નિયમ યાદ રાખવા માટે}: ``ફોરવર્ડ કરંટ પ્રકાશ ઉત્સર્જિત કરે'' (FCEL)

\end{solutionbox}
\subsection*{પ્રશ્ન 4(અ) અથવા [3
ગુણ]}\label{uxaaauxab0uxab6uxaa8-4uxa85-uxa85uxaa5uxab5-3-uxa97uxaa3}

\textbf{LED ના ઉપયોગો જણાવો.}

\begin{solutionbox}


{\def\LTcaptype{none} % do not increment counter
\vspace{-5pt}
\captionof{table}{LED ઉપયોગો}
\vspace{-10pt}
\begin{longtable}[]{@{}ll@{}}
\toprule\noalign{}
ઉપયોગ & ફાયદો \\
\midrule\noalign{}
\endhead
\bottomrule\noalign{}
\endlastfoot
ડિસ્પ્લે ઇન્ડિકેટર્સ & ઓછો પાવર વપરાશ \\
ડિજિટલ ડિસ્પ્લે & વિવિધ રંગો ઉપલબ્ધ \\
લાઇટિંગ & ઊર્જા કાર્યક્ષમ \\
રિમોટ કંટ્રોલ & ઇન્ફ્રારેડ કમ્યુનિકેશન \\
ટ્રાફિક સિગ્નલ્સ & લાંબી લાઇફ, ઉચ્ચ દૃશ્યતા \\
\end{longtable}
}

\textbf{આકૃતિ:}

\begin{center}
\textbf{Mermaid Diagram (Code)}
\begin{verbatim}
{Shaded}
{Highlighting}[]
graph TD
    A[LED Applications] {-{-}{} B[Indicators]}
    A {-{-}{} C[Displays]}
    A {-{-}{} D[Lighting]}
    A {-{-}{} E[Communication]}
    A {-{-}{} F[Signals]}
    style A fill:\#9cf,stroke:\#333
{Highlighting}
{Shaded}
\end{verbatim}
\end{center}

\textbf{નિયમ યાદ રાખવા માટે}: ``ડિસ્પ્લે લાઇટ્સ ઇન ક્લેવર સિગ્નલ્સ'' (DLICS)

\end{solutionbox}
\subsection*{પ્રશ્ન 4(બ) [4
ગુણ]}\label{uxaaauxab0uxab6uxaa8-4uxaac-4-uxa97uxaa3}

\textbf{``ઝેનર ડાયોડ વોલ્ટેજ રેગ્યુલેટર તરીકે'' સમજાવો.}

\begin{solutionbox}

ઝેનર ડાયોડ રિવર્સ બ્રેકડાઉન રીજીયનમાં ઓપરેટ કરવામાં આવે ત્યારે ઇનપુટ વોલ્ટેજની
અસ્થિરતા છતાં સ્થિર આઉટપુટ વોલ્ટેજ જાળવે છે.

\textbf{સર્કિટ:}

\begin{center}
\textbf{Mermaid Diagram (Code)}
\begin{verbatim}
{Shaded}
{Highlighting}[]
graph LR
    A[Unregulated DC] {-{-}{-} B[Series Resistor] {-}{-}{-} C[Output]}
    C {-{-}{-} D[Zener Diode] {-}{-}{-} E[Ground]}
    C {-{-}{-} F[Load] {-}{-}{-} E}
    style A fill:\#9cf,stroke:\#333
    style C fill:\#9f9,stroke:\#333
    style D fill:\#f96,stroke:\#333
{Highlighting}
{Shaded}
\end{verbatim}
\end{center}

\textbf{કાર્ય:}

\begin{itemize}
\tightlist
\item
  સીરીઝ રેઝિસ્ટર કરંટ મર્યાદિત કરે છે
\item
  ઝેનર બ્રેકડાઉન રીજીયનમાં કાર્ય કરે છે
\item
  લોડ પર સ્થિર વોલ્ટેજ જાળવે છે
\end{itemize}


{\def\LTcaptype{none} % do not increment counter
\vspace{-5pt}
\captionof{table}{ઝેનર રેગ્યુલેટરની લાક્ષણિકતાઓ}
\vspace{-10pt}
\begin{longtable}[]{@{}ll@{}}
\toprule\noalign{}
પેરામીટર & વર્ણન \\
\midrule\noalign{}
\endhead
\bottomrule\noalign{}
\endlastfoot
વોલ્ટેજ રેગ્યુલેશન & ઇનપુટમાં ફેરફાર છતાં સ્થિર આઉટપુટ જાળવે છે \\
પાવર રેટિંગ & પાવર ડિસિપેશન સંભાળવું જોઈએ \\
તાપમાન સ્થિરતા & આઉટપુટ તાપમાન સાથે થોડું બદલાય છે \\
\end{longtable}
}

\textbf{નિયમ યાદ રાખવા માટે}: ``ઝેનર બ્રેક ટુ રેગ્યુલેટ'' (ZBR)

\end{solutionbox}
\subsection*{પ્રશ્ન 4(બ) અથવા [4
ગુણ]}\label{uxaaauxab0uxab6uxaa8-4uxaac-uxa85uxaa5uxab5-4-uxa97uxaa3}

\textbf{ઝેનર વોલ્ટેજ રેગ્યુલેટરની મર્યાદાઓ.}

\begin{solutionbox}


{\def\LTcaptype{none} % do not increment counter
\vspace{-5pt}
\captionof{table}{ઝેનર વોલ્ટેજ રેગ્યુલેટરની મર્યાદાઓ}
\vspace{-10pt}
\begin{longtable}[]{@{}ll@{}}
\toprule\noalign{}
મર્યાદા & અસર \\
\midrule\noalign{}
\endhead
\bottomrule\noalign{}
\endlastfoot
પાવર ડિસિપેશન & ઝેનર પાવર રેટિંગ દ્વારા મર્યાદિત \\
કરંટ ક્ષમતા & માત્ર નાના લોડ સંભાળી શકે છે \\
તાપમાન સંવેદનશીલતા & આઉટપુટ તાપમાન સાથે બદલાય છે \\
કાર્યક્ષમતા & સીરીઝ રેઝિસ્ટરમાં પાવર લોસને કારણે ખરાબ કાર્યક્ષમતા \\
નોઈઝ & ઇલેક્ટ્રિકલ નોઈઝ ઉત્પન્ન કરે છે \\
\end{longtable}
}

\textbf{આકૃતિ:}

\begin{center}
\textbf{Mermaid Diagram (Code)}
\begin{verbatim}
{Shaded}
{Highlighting}[]
graph TD
    A[Zener Limitations] {-{-}{} B[Power Limits]}
    A {-{-}{} C[Current Limits]}
    A {-{-}{} D[Temperature Effects]}
    A {-{-}{} E[Efficiency Issues]}
    A {-{-}{} F[Noise Generation]}
    style A fill:\#f96,stroke:\#333
{Highlighting}
{Shaded}
\end{verbatim}
\end{center}

\textbf{નિયમ યાદ રાખવા માટે}: ``પાવર કરંટ ટેમ્પરેચર એફિશિયન્સી નોઇઝ'' (PCTEN)

\end{solutionbox}
\subsection*{પ્રશ્ન 4(ક) [7
ગુણ]}\label{uxaaauxab0uxab6uxaa8-4uxa95-7-uxa97uxaa3}

\textbf{રેક્ટીફાયરમાં ફિલ્ટર સર્કિટની જરૂરીયાત વર્ણવો. રેક્ટીફાયરમાં ઉપયોગી વિવિધ
પ્રકારની ફિલ્ટર સર્કિટના નામ જણાવો તથા કોઈ એક ફિલ્ટર સર્કિટ દોરી વિગતવાર
સમજાવો.}

\begin{solutionbox}

\textbf{ફિલ્ટર સર્કિટની જરૂરીયાત:} રેક્ટિફાયર આઉટપુટમાં AC રિપલ હોય છે જે સ્મૂધ DC
માટે દૂર કરવી જરૂરી છે. ફિલ્ટર્સ આ રિપલ ઘટાડીને સ્થિર DC આઉટપુટ પૂરું પાડે છે.

\textbf{ફિલ્ટર સર્કિટના પ્રકારો:}

\begin{enumerate}
\tightlist
\item
  કેપેસિટર ફિલ્ટર (શન્ટ કેપેસિટર)
\item
  LC ફિલ્ટર
\item
  π-ફિલ્ટર (પાઇ-ફિલ્ટર)
\item
  RC ફિલ્ટર
\end{enumerate}

\textbf{કેપેસિટર ફિલ્ટરની સમજૂતી:}

\textbf{સર્કિટ ડાયાગ્રામ:}

\begin{center}
\textbf{Mermaid Diagram (Code)}
\begin{verbatim}
{Shaded}
{Highlighting}[]
graph LR
    A[Rectifier Output] {-{-}{-} B[+]}
    B {-{-}{-} C[Load]}
    B {-{-}{-} D[Capacitor]}
    C {-{-}{-} E[Ground]}
    D {-{-}{-} E}
    style A fill:\#9cf,stroke:\#333
    style C fill:\#f96,stroke:\#333
    style D fill:\#9f9,stroke:\#333
{Highlighting}
{Shaded}
\end{verbatim}
\end{center}

\textbf{કાર્ય:}

\begin{itemize}
\tightlist
\item
  કેપેસિટર વોલ્ટેજના પીક્સ દરમિયાન ચાર્જ થાય છે
\item
  વોલ્ટેજ ડ્રોપ્સ દરમિયાન ધીમે ધીમે ડિસ્ચાર્જ થાય છે
\item
  પીક્સ વચ્ચે આઉટપુટ વોલ્ટેજ જાળવે છે
\item
  રિપલ વોલ્ટેજ ઘટાડે છે
\end{itemize}


{\def\LTcaptype{none} % do not increment counter
\vspace{-5pt}
\captionof{table}{કેપેસિટર ફિલ્ટરની લાક્ષણિકતાઓ}
\vspace{-10pt}
\begin{longtable}[]{@{}ll@{}}
\toprule\noalign{}
પેરામીટર & અસર \\
\midrule\noalign{}
\endhead
\bottomrule\noalign{}
\endlastfoot
કેપેસિટન્સ મૂલ્ય & ઉચ્ચ મૂલ્ય ઓછી રિપલ આપે છે \\
રિપલ ઘટાડો & સામાન્ય રીતે 70-80\% ઘટાડે છે \\
લોડ કરંટ & ઉચ્ચ લોડ કરંટ વધુ રિપલ ઉત્પન્ન કરે છે \\
ફ્રીક્વન્સી & ઉચ્ચ ફ્રીક્વન્સી ફિલ્ટર કરવી સરળ છે \\
\end{longtable}
}

\textbf{વેવફોર્મ્સ:}

\begin{center}
\textbf{Mermaid Diagram (Code)}
\begin{verbatim}
{Shaded}
{Highlighting}[]
graph TD
    subgraph "Rectifier Output"
    A[Pulsating DC]
    end
    subgraph "Filter Output"
    B[Smoother DC]
    end
    style A fill:\#f96,stroke:\#333
    style B fill:\#9f9,stroke:\#333
{Highlighting}
{Shaded}
\end{verbatim}
\end{center}

\textbf{નિયમ યાદ રાખવા માટે}: ``કેપેસિટર્સ હોલ્ડ વોલ્ટેજ ડ્યુરિંગ ડ્રોપ્સ'' (CHVDD)

\end{solutionbox}
\subsection*{પ્રશ્ન 5(અ) [3
ગુણ]}\label{uxaaauxab0uxab6uxaa8-5uxa85-3-uxa97uxaa3}

\textbf{ઈ-વેસ્ટની વ્યાખ્યા લખો. સામાન્ય ઈ-વેસ્ટ વસ્તુઓની યાદી બનાવો.}

\begin{solutionbox}

ઈ-વેસ્ટ એટલે ત્યજિત ઇલેક્ટ્રોનિક ઉપકરણો અને ઘટકો કે જે તેમના ઉપયોગી જીવનકાળના અંતે
પહોંચ્યા છે.


{\def\LTcaptype{none} % do not increment counter
\vspace{-5pt}
\captionof{table}{સામાન્ય ઈ-વેસ્ટ વસ્તુઓ}
\vspace{-10pt}
\begin{longtable}[]{@{}ll@{}}
\toprule\noalign{}
શ્રેણી & ઉદાહરણો \\
\midrule\noalign{}
\endhead
\bottomrule\noalign{}
\endlastfoot
કમ્પ્યુટિંગ ઉપકરણો & કમ્પ્યુટર્સ, લેપટોપ, ટેબ્લેટ \\
કમ્યુનિકેશન ઉપકરણો & મોબાઇલ ફોન, ટેલિફોન \\
ઘરેલું ઉપકરણો & ટીવી, રેફ્રિજરેટર, વોશિંગ મશીન \\
ઇલેક્ટ્રોનિક ઘટકો & સર્કિટ બોર્ડ, બેટરી, કેબલ્સ \\
ઓફિસ ઉપકરણો & પ્રિન્ટર, સ્કેનર, કોપિયર \\
\end{longtable}
}

\textbf{આકૃતિ:}

\begin{center}
\textbf{Mermaid Diagram (Code)}
\begin{verbatim}
{Shaded}
{Highlighting}[]
graph TD
    A[E{-waste] {-}{-}{} B[Computing]}
    A {-{-}{} C[Communication]}
    A {-{-}{} D[Home Appliances]}
    A {-{-}{} E[Components]}
    A {-{-}{} F[Office Equipment]}
    style A fill:\#f96,stroke:\#333
{Highlighting}
{Shaded}
\end{verbatim}
\end{center}

\textbf{નિયમ યાદ રાખવા માટે}: ``કમ્પ્યુટર્સ, કમ્યુનિકેશન, કમ્પોનન્ટ્સ, હોમ
એપ્લાયન્સિસ'' (CCCHA)

\end{solutionbox}
\subsection*{પ્રશ્ન 5(બ) [3
ગુણ]}\label{uxaaauxab0uxab6uxaa8-5uxaac-3-uxa97uxaa3}

\textbf{ઈ-વેસ્ટ મેનેજમેન્ટની વિવિધ વ્યૂહરચના જણાવો અને સમજાવો.}

\begin{solutionbox}


{\def\LTcaptype{none} % do not increment counter
\vspace{-5pt}
\captionof{table}{ઈ-વેસ્ટ મેનેજમેન્ટની વ્યૂહરચનાઓ}
\vspace{-10pt}
\begin{longtable}[]{@{}ll@{}}
\toprule\noalign{}
વ્યૂહરચના & વર્ણન \\
\midrule\noalign{}
\endhead
\bottomrule\noalign{}
\endlastfoot
ઘટાડવું & નવા ઇલેક્ટ્રોનિક્સની ખરીદી ઘટાડવી \\
ફરીથી ઉપયોગ & રિપેર અને રીપરપઝિંગ દ્વારા જીવનકાળ વધારવો \\
રિસાયકલ & મૂલ્યવાન સામગ્રી પુનઃપ્રાપ્ત કરવા માટે ઈ-વેસ્ટ પ્રોસેસ કરવો \\
જવાબદાર નિકાલ & અધિકૃત ઈ-વેસ્ટ સંગ્રહ કેન્દ્રોનો ઉપયોગ કરવો \\
વિસ્તૃત ઉત્પાદક જવાબદારી & ઉત્પાદકો જીવનકાળના અંત ઉત્પાદનો પાછા લે છે \\
\end{longtable}
}

\textbf{આકૃતિ:}

\begin{center}
\textbf{Mermaid Diagram (Code)}
\begin{verbatim}
{Shaded}
{Highlighting}[]
graph TD
    A[E{-waste Management] {-}{-}{} B[Reduce]}
    A {-{-}{} C[Reuse]}
    A {-{-}{} D[Recycle]}
    A {-{-}{} E[Responsible Disposal]}
    A {-{-}{} F[Extended Producer Responsibility]}
    style A fill:\#9cf,stroke:\#333
{Highlighting}
{Shaded}
\end{verbatim}
\end{center}

\textbf{નિયમ યાદ રાખવા માટે}: ``3R's અને 2 વધારાની કાર્યવાહી'' (3R2A)

\end{solutionbox}
\subsection*{પ્રશ્ન 5(ક) [4
ગુણ]}\label{uxaaauxab0uxab6uxaa8-5uxa95-4-uxa97uxaa3}

\textbf{``ટ્રાનઝીસ્ટર સ્વીચ તરીકે'' સમજાવો.}

\begin{solutionbox}

ટ્રાન્ઝિસ્ટર કટઓફ (OFF) અથવા સેચુરેશન (ON) રીજીયનમાં ઓપરેટ કરીને ઇલેક્ટ્રોનિક સ્વિચ
તરીકે કાર્ય કરી શકે છે.


{\def\LTcaptype{none} % do not increment counter
\vspace{-5pt}
\captionof{table}{ટ્રાન્ઝિસ્ટર સ્વિચ ઓપરેશન}
\vspace{-10pt}
\begin{longtable}[]{@{}lll@{}}
\toprule\noalign{}
સ્થિતિ & શરત & વર્તન \\
\midrule\noalign{}
\endhead
\bottomrule\noalign{}
\endlastfoot
OFF (કટઓફ) & બેઝ કરંટ = 0 & કોઈ કલેક્ટર કરંટ વહેતો નથી \\
ON (સેચુરેશન) & બેઝ કરંટ પૂરતો & મહત્તમ કલેક્ટર કરંટ વહે છે \\
\end{longtable}
}

\textbf{સર્કિટ ડાયાગ્રામ:}

\begin{center}
\textbf{Mermaid Diagram (Code)}
\begin{verbatim}
{Shaded}
{Highlighting}[]
graph LR
    A[+Vcc] {-{-}{-} B[Rc] {-}{-}{-} C[Collector]}
    C {-{-}{-} D[Emitter] {-}{-}{-} E[Ground]}
    F[Vin] {-{-}{-} G[Rb] {-}{-}{-} H[Base]}
    H {-{-}{-} D}
    style F fill:\#9cf,stroke:\#333
    style A fill:\#f96,stroke:\#333
{Highlighting}
{Shaded}
\end{verbatim}
\end{center}

\textbf{કાર્ય:}

\begin{itemize}
\tightlist
\item
  જ્યારે ઇનપુટ HIGH હોય: ટ્રાન્ઝિસ્ટર સેચુરેટ થાય છે, બંધ સ્વિચ જેવું વર્તન કરે છે
\item
  જ્યારે ઇનપુટ LOW હોય: ટ્રાન્ઝિસ્ટર કટ-ઓફ થાય છે, ખુલ્લા સ્વિચ જેવું વર્તન કરે છે
\end{itemize}

\textbf{નિયમ યાદ રાખવા માટે}: ``નો બેઝ નો કરંટ, એપ્લાય બેઝ કનેક્ટ સર્કિટ''
(NBNC-ABC)

\end{solutionbox}
\subsection*{પ્રશ્ન 5(ડ) [4
ગુણ]}\label{uxaaauxab0uxab6uxaa8-5uxaa1-4-uxa97uxaa3}

\textbf{ટ્રાંઝીસ્ટરના CE કંફીગરેશન માટે α તથા β વચ્ચેનો સંબંધ તારવો.}

\begin{solutionbox}

ટ્રાન્ઝિસ્ટરમાં, α (આલ્ફા) અને β (બીટા) કરંટ ગેઇન પેરામીટર્સ છે.

\textbf{વ્યાખ્યાઓ:}

\begin{itemize}
\tightlist
\item
  α = IC/IE (કોમન બેઝ કરંટ ગેઇન)
\item
  β = IC/IB (કોમન એમિટર કરંટ ગેઇન)
\end{itemize}

\textbf{તારણ:} IE = IC + IB થી, આપણે લખી શકીએ: α = IC/IE = IC/(IC + IB)

ન્યુમરેટર અને ડિનોમિનેટરને IBથી ભાગીએ:

α = (IC/IB)/[(IC/IB) + 1] = β/(β +

1)

તેથી: β = α/(1-α)


{\def\LTcaptype{none} % do not increment counter
\vspace{-5pt}
\captionof{table}{α અને β વચ્ચેનો સંબંધ}
\vspace{-10pt}
\begin{longtable}[]{@{}lll@{}}
\toprule\noalign{}
પેરામીટર & સૂત્ર & સામાન્ય રેન્જ \\
\midrule\noalign{}
\endhead
\bottomrule\noalign{}
\endlastfoot
α માંથી β & α = β/(β+1) & 0.9 થી 0.99 \\
β માંથી α & β = α/(1-α) & 50 થી 300 \\
\end{longtable}
}

\textbf{આકૃતિ:}

\begin{center}
\textbf{Mermaid Diagram (Code)}
\begin{verbatim}
{Shaded}
{Highlighting}[]
graph TD
    A[alpha = IC divided by IE] {-{-}{-} B[beta = IC divided by IB]}
    C[beta = alpha divided by 1 minus alpha] {-{-}{-} D[alpha = beta divided by beta plus 1]}

    style A fill:\#9cf,stroke:\#333
    style B fill:\#f96,stroke:\#333
{Highlighting}
{Shaded}
\end{verbatim}
\end{center}

\textbf{નિયમ યાદ રાખવા માટે}: ``બીટા બરાબર આલ્ફા ડિવાઇડેડ બાય વન માઇનસ
આલ્ફા'' (BAOA)

\end{solutionbox}

\end{document}
