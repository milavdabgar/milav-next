%% METADATA
%% subject-code: 4311102
%% subject-name: Fundamentals of Electronics
%% semester: 1
%% examination: Summer-2023
%% date: 31-07-2023
%% description: Solution guide for Fundamentals of Electronics Summer 2023 examination in Gujarati
%% tags: study-material, solutions, gtu, 4311102, electronics, gujarati
%% END METADATA

\documentclass{article}
% GTU Solutions - Gujarati Preamble
% Includes common preamble + Gujarati font setup

% Basic setup
\usepackage[margin=1in]{geometry}
\author{Milav Dabgar}

% Math and tables
\usepackage{amsmath,amssymb,amsthm}
\usepackage{booktabs}
\usepackage{tabularx}
\usepackage{graphicx}
\usepackage{float}  % Required for [H] float placement

% Code listings with syntax highlighting
\usepackage{xcolor}
\usepackage{listings}
\lstset{
  basicstyle=\small\ttfamily,
  breaklines=true,
  numbers=left,
  numberstyle=\tiny\color{gray},
  xleftmargin=2em,
  frame=single,
  showstringspaces=false,
  tabsize=2,
  keywordstyle=\color{blue},
  commentstyle=\color{green!60!black},
  stringstyle=\color{purple}
}

% Optional: TikZ for diagrams (remove if not needed)
\usepackage{tikz}
\usepackage{circuitikz}
\usetikzlibrary{shapes,arrows,positioning,calc}

% Header/footer with author and website
\usepackage{fancyhdr}
\usepackage{lastpage}

\pagestyle{fancy}
\fancyhf{}
\fancyhead[L]{\small\itshape\leftmark}
\fancyhead[R]{\small Milav Dabgar}
\fancyfoot[L]{\small\href{https://www.milav.in}{www.milav.in}}
\fancyfoot[R]{\small Page \thepage\ of \pageref{LastPage}}
\renewcommand{\headrulewidth}{0.4pt}
\renewcommand{\footrulewidth}{0.4pt}

% Hyperref (load before fontspec for Gujarati)
\usepackage[
  colorlinks=true,
  linkcolor=blue,
  urlcolor=blue,
  citecolor=blue,
  pdfauthor={Milav Dabgar},
  pdfsubject={GTU Exam Solutions},
  pdfkeywords={GTU, Java, Programming, Solutions, Gujarati},
  bookmarks=true
]{hyperref}

% Gujarati font setup
\usepackage{fontspec}
\usepackage{polyglossia}
\setdefaultlanguage{gujarati}
\setotherlanguage{english}
\newfontfamily\gujaratifont[Script=Gujarati,AutoFakeBold=2.5,AutoFakeSlant=0.3]{Noto Sans Gujarati}
\setmainfont[Script=Gujarati,AutoFakeBold=2.5,AutoFakeSlant=0.3]{Noto Sans Gujarati}
\setmonofont[Scale=0.9]{Noto Sans Gujarati}
\newfontfamily\englishfont[Script=Gujarati,AutoFakeBold=2.5,AutoFakeSlant=0.3]{Noto Sans Gujarati}
\gappto\captionsgujarati{
  \renewcommand{\tablename}{કોષ્ટક}
  \renewcommand{\figurename}{આકૃતિ}
}
\newcommand{\gu}[1]{{\gujaratifont #1}}


\title{Fundamentals of Electronics (4311102) - Summer 2023 Solution}
\date{જુલાઇ 31, 2023}

% PDF Metadata
\hypersetup{
  pdftitle={Fundamentals of Electronics (4311102) - Summer 2023 Solution (Gujarati)},
  pdfsubject={GTU Exam Solution - Summer-2023},
  pdfauthor={Milav Dabgar},
  pdfkeywords={study-material, solutions, gtu, 4311102, electronics, gujarati},
  pdfcreator={XeLaTeX}
}

\begin{document}
\maketitle

\setcounter{tocdepth}{5}
\tableofcontents
\newpage

% ========================================
% QUESTION 1(a): Define Active and Passive Components (3 marks)
% Demonstrates: Clear definitions with examples
% ========================================

\section{પ્રશ્ન 1}

\subsection{પ્રશ્ન 1(a) [3 ગુણ]}
\textbf{સક્રિય અને ક્રિનષ્ક્રિય ઘટકોને વ્યાખ્યાયિત કરો.}

\subsubsection{ઉકેલ}

ઇલેક્ટ્રોનિક કોમ્પોનન્ટ્સ એ ઇલેક્ટ્રોનિક સર્કિટ્સના મૂળભૂત બિલ્ડિંગ બ્લોક્સ છે અને તેમની એનર્જી હેન્ડલિંગ ક્ષમતાના આધારે વર્ગીકૃત કરવામાં આવે છે. \textbf{સક્રિય કોમ્પોનન્ટ્સ} એવા ઉપકરણો છે જે વીજળીના પ્રવાહને નિયંત્રિત કરી શકે છે અને બાહ્ય પાવર સ્રોતમાંથી એનર્જી ઉમેરીને ઇલેક્ટ્રિકલ સિગ્નલોને એમ્પ્લિફાય કરી શકે છે. તેમને કાર્ય કરવા માટે બાહ્ય પાવર સપ્લાય જરૂરી છે અને તેઓ એક કરતાં વધુ પાવર ગેઇન પ્રદાન કરી શકે છે. ઉદાહરણોમાં ટ્રાન્ઝિસ્ટર્સ, ડાયોડ્સ, ઇન્ટીગ્રેટેડ સર્કિટ્સ અને ઓપરેશનલ એમ્પ્લિફાયર્સનો સમાવેશ થાય છે.

તેનાથી વિપરીત, \textbf{ક્રિનષ્ક્રિય કોમ્પોનન્ટ્સ} એવા ઉપકરણો છે જે સિગ્નલોને એમ્પ્લિફાય કરી શકતા નથી અથવા સર્કિટમાં એનર્જી દાખલ કરી શકતા નથી. તેઓ ફક્ત સર્કિટમાં પહેલેથી હાજર એનર્જીને વાપરી, સંગ્રહિત અથવા મુક્ત કરી શકે છે. આ કોમ્પોનન્ટ્સને તેમના મૂળભૂત સંચાલન માટે બાહ્ય પાવર સ્રોતની જરૂર નથી અને તેઓ પાવર ગેઇન પ્રદાન કરી શકતા નથી. સામાન્ય ઉદાહરણોમાં રેઝિસ્ટર્સ, કેપેસિટર્સ, ઇન્ડક્ટર્સ અને ટ્રાન્સફોર્મર્સનો સમાવેશ થાય છે.

\paragraph{મુખ્ય તફાવતો:}
\begin{description}
    \item[સક્રિય કોમ્પોનન્ટ્સ:] બાહ્ય પાવર જરૂરી, સિગ્નલ્સને એમ્પ્લિફાય કરી શકે, પાવર ગેઇન \(> 1\) પ્રદાન કરે, ઉદાહરણો: BJT, FET, LED, SCR
    \item[ક્રિનષ્ક્રિય કોમ્પોનન્ટ્સ:] બાહ્ય પાવરની જરૂર નથી, એમ્પ્લિફાય કરી શકતા નથી, પાવર ગેઇન \(\leq 1\), ઉદાહરણો: R, L, C, transformers
\end{description}

\paragraph{મેમરી ટ્રીક:}
\emph{``ACTIVE પાવર ઉમેરે, PASSIVE ફક્ત પસાર કરે!''}

% ========================================
% QUESTION 1(b): Types of Capacitors Based on Materials (4 marks)
% Demonstrates: Classification with characteristics
% ========================================

\subsection{પ્રશ્ન 1(b) [4 ગુણ]}
\textbf{વપરાયેલ સામગ્રી પર આધારિત કેપેસિટરના પ્રકારો વર્ણવો.}

\subsubsection{ઉકેલ}

કેપેસિટર્સને તેમની પ્લેટ્સ વચ્ચે વપરાયેલ \textbf{ડાયલેક્ટ્રિક સામગ્રી} ના આધારે વર્ગીકૃત કરવામાં આવે છે, જે તેમની લાક્ષણિકતાઓ, એપ્લિકેશન્સ અને પર્ફોર્મન્સ નક્કી કરે છે. ડાયલેક્ટ્રિક સામગ્રી કેપેસિટન્સ વેલ્યુ, વોલ્ટેજ રેટિંગ, તાપમાન સ્થિરતા અને ફ્રિક્વન્સી રિસ્પોન્સને નોંધપાત્ર રીતે પ્રભાવિત કરે છે.

\paragraph{ડાયલેક્ટ્રિક સામગ્રી પર આધારિત પ્રકારો:}
\begin{description}
    \item[સિરામિક કેપેસિટર્સ:] ડાયલેક્ટ્રિક તરીકે સિરામિક સામગ્રીનો ઉપયોગ કરે છે. સારી સ્થિરતા સાથે નાના કદમાં ઉપલબ્ધ, સામાન્ય રીતે હાય-ફ્રિક્વન્સી એપ્લિકેશન્સમાં વપરાય છે. વેલ્યુઝ pF થી \(\mu F\) સુધીની શ્રેણીમાં.
    \item[ઇલેક્ટ્રોલિટિક કેપેસિટર્સ:] ડાયલેક્ટ્રિક તરીકે ઇલેક્ટ્રોલાઇટનો ઉપયોગ કરે છે. ઉચ્ચ કેપેસિટન્સ વેલ્યુઝ (1\,\(\mu F\) થી હજારો \(\mu F\)) સાથે પોલરાઇઝ્ડ કેપેસિટર્સ. પાવર સપ્લાય ફિલ્ટરિંગ અને કપલિંગ એપ્લિકેશન્સમાં વપરાય છે.
    \item[ફિલ્મ કેપેસિટર્સ:] ડાયલેક્ટ્રિક તરીકે પાતળી પ્લાસ્ટિક ફિલ્મ્સ (polyester, polypropylene, polystyrene) નો ઉપયોગ કરે છે. નોન-પોલરાઇઝ્ડ, ઉત્કૃષ્ટ સ્થિરતા, પ્રિસિઝન એપ્લિકેશન્સ અને ઓડિયો સર્કિટ્સમાં વપરાય છે.
    \item[પેપર કેપેસિટર્સ:] ડાયલેક્ટ્રિક તરીકે મીણયુક્ત કાગળ અથવા તેલયુક્ત કાગળનો ઉપયોગ કરે છે. જૂની તકનીક, ફિલ્મ કેપેસિટર્સ દ્વારા બદલાય છે. લો-ફ્રિક્વન્સી એપ્લિકેશન્સ માટે સારા.
    \item[માઇકા કેપેસિટર્સ:] ડાયલેક્ટ્રિક તરીકે માઇકા શીટ્સનો ઉપયોગ કરે છે. ઉત્કૃષ્ટ સ્થિરતા અને ઓછા નુકસાન, મોંઘા, RF અને હાય-ફ્રિક્વન્સી ટ્યુન્ડ સર્કિટ્સમાં વપરાય છે.
    \item[એર/વેક્યુમ કેપેસિટર્સ:] ડાયલેક્ટ્રિક તરીકે હવા અથવા વેક્યુમનો ઉપયોગ કરે છે. રેડિયો ટ્યુનિંગ સર્કિટ્સમાં ઉપયોગમાં લેવાતા વેરિએબલ કેપેસિટર્સ ખૂબ ઓછા નુકસાન સાથે.
\end{description}

\subparagraph{પસંદગી માપદંડ:}
પસંદગી જરૂરી કેપેસિટન્સ વેલ્યુ, વોલ્ટેજ રેટિંગ, ફ્રિક્વન્સી રેન્જ, કિંમત, કદની મર્યાદાઓ અને તાપમાન સ્થિરતાની જરૂરિયાતો પર આધારિત છે.

\paragraph{મેમરી ટ્રીક:}
\emph{``CEFPMA: Ceramic, Electrolytic, Film, Paper, Mica, Air -- સામગ્રી કેપેસિટર્સ બનાવે છે!''}

% ========================================
% QUESTION 1(c): Resistor Color Coding (7 marks)
% Demonstrates: Technique explanation with diagram and examples
% ========================================

\subsection{પ્રશ્ન 1(c) [7 ગુણ]}
\textbf{રેઝિસ્ટર કલર કોડિંગ ટેકનિક ઉદાહરણ સાથે સમજાવો.}

\subsubsection{ઉકેલ}

\textbf{રેઝિસ્ટર કલર કોડ} એ એક સ્ટાન્ડર્ડાઇઝ્ડ માર્કિંગ સિસ્ટમ છે જેનો ઉપયોગ રેઝિસ્ટર્સની રેઝિસ્ટન્સ વેલ્યુ અને ટોલરન્સ સૂચવવા માટે થાય છે. કારણ કે રેઝિસ્ટર્સ નાના કોમ્પોનન્ટ્સ છે, છપાયેલા નંબરોને બદલે કલર બેન્ડ્સનો ઉપયોગ થાય છે. આ સિસ્ટમ સંખ્યાત્મક વેલ્યુઝ રજૂ કરવા માટે રેઝિસ્ટર બોડીની આસપાસ પેઇન્ટ કરેલા રંગીન બેન્ડ્સનો ઉપયોગ કરે છે.

\paragraph{કલર કોડ સિસ્ટમ:}

સ્ટાન્ડર્ડ કલર કોડ દરેક રંગને સંખ્યાત્મક વેલ્યુ સોંપે છે:

\begin{table}[H]
\centering
\caption{રેઝિસ્ટર કલર કોડ વેલ્યુઝ}
\begin{tabularx}{\textwidth}{llll}
\toprule
\textbf{રંગ} & \textbf{ડિજિટ} & \textbf{મલ્ટિપ્લાયર} & \textbf{ટોલરન્સ} \\
\midrule
Black & 0 & \(\times 10^0\) & -- \\
Brown & 1 & \(\times 10^1\) & \(\pm 1\%\) \\
Red & 2 & \(\times 10^2\) & \(\pm 2\%\) \\
Orange & 3 & \(\times 10^3\) & -- \\
Yellow & 4 & \(\times 10^4\) & -- \\
Green & 5 & \(\times 10^5\) & \(\pm 0.5\%\) \\
Blue & 6 & \(\times 10^6\) & \(\pm 0.25\%\) \\
Violet & 7 & \(\times 10^7\) & \(\pm 0.1\%\) \\
Grey & 8 & \(\times 10^8\) & \(\pm 0.05\%\) \\
White & 9 & \(\times 10^9\) & -- \\
Gold & -- & \(\times 10^{-1}\) & \(\pm 5\%\) \\
Silver & -- & \(\times 10^{-2}\) & \(\pm 10\%\) \\
\bottomrule
\end{tabularx}
\end{table}

\paragraph{ચાર-બેન્ડ રેઝિસ્ટર રીડિંગ:}
\begin{enumerate}
    \item \textbf{બેન્ડ 1:} પ્રથમ નોંધપાત્ર અંક
    \item \textbf{બેન્ડ 2:} બીજો નોંધપાત્ર અંક
    \item \textbf{બેન્ડ 3:} મલ્ટિપ્લાયર (શૂન્યોની સંખ્યા)
    \item \textbf{બેન્ડ 4:} ટોલરન્સ
\end{enumerate}

\paragraph{ઉદાહરણ ગણતરી:}

કલર બેન્ડ્સ સાથેના રેઝિસ્ટરનો વિચાર કરો: \textbf{Brown, Black, Red, Gold}

\begin{figure}[H]
\centering
\begin{tikzpicture}[scale=1.2]
    % Resistor body
    \draw[thick, fill=gray!20] (0,0) rectangle (4,0.6);
    
    % Color bands
    \fill[brown!70!black] (0.5,0) rectangle (0.8,0.6);
    \fill[black] (1.2,0) rectangle (1.5,0.6);
    \fill[red] (1.9,0) rectangle (2.2,0.6);
    \fill[yellow!80!orange] (3.2,0) rectangle (3.5,0.6);
    
    % Labels
    \node[below, font=\small] at (0.65,-0.1) {Brown};
    \node[below, font=\small] at (1.35,-0.1) {Black};
    \node[below, font=\small] at (2.05,-0.1) {Red};
    \node[below, font=\small] at (3.35,-0.1) {Gold};
    
    \node[above, font=\small] at (0.65,0.7) {1};
    \node[above, font=\small] at (1.35,0.7) {0};
    \node[above, font=\small] at (2.05,0.7) {\(\times 100\)};
    \node[above, font=\small] at (3.35,0.7) {\(\pm 5\%\)};
    
    % Leads
    \draw[thick] (-0.5,0.3) -- (0,0.3);
    \draw[thick] (4,0.3) -- (4.5,0.3);
\end{tikzpicture}
\caption{Brown-Black-Red-Gold કલર બેન્ડ્સ સાથે રેઝિસ્ટર}
\end{figure}

\begin{itemize}
    \item બેન્ડ 1 (Brown): 1 (પ્રથમ અંક)
    \item બેન્ડ 2 (Black): 0 (બીજો અંક)
    \item બેન્ડ 3 (Red): \(\times 10^2 = \times 100\) (મલ્ટિપ્લાયર)
    \item બેન્ડ 4 (Gold): \(\pm 5\%\) (ટોલરન્સ)
\end{itemize}

\[
R = (10) \times 100 = 1000\,\Omega = 1\,k\Omega \pm  5\%
\]

\subparagraph{ટોલરન્સ રેન્જ:}
1\,k\(\Omega\) \(\pm\) 5\% માટે: લઘુત્તમ = \(1000 - 50 = 950\,\Omega\), મહત્તમ = \(1000 + 50 = 1050\,\Omega\)

\paragraph{પાંચ અને છ-બેન્ડ કોડ્સ:}
પ્રિસિઝન રેઝિસ્ટર્સ માટે, પાંચ અથવા છ બેન્ડ્સનો ઉપયોગ થાય છે જ્યાં પ્રથમ ત્રણ બેન્ડ્સ નોંધપાત્ર અંકો રજૂ કરે છે, ત્યારબાદ મલ્ટિપ્લાયર, ટોલરન્સ અને વૈકલ્પિક રીતે તાપમાન ગુણાંક આવે છે.

\paragraph{વાંચન દિશા:}
ટોલરન્સ બેન્ડ (સામાન્ય રીતે gold અથવા silver) અન્ય બેન્ડ્સથી થોડો અલગ છે અને વાંચતી વખતે જમણી બાજુ હોવો જોઈએ. વિરુદ્ધ છેડેથી વાંચવાનું શરૂ કરો.

\paragraph{મેમરી ટ્રીક:}
\emph{``BB ROY of Great Britain had a Very Good Wife: Black Brown Red Orange Yellow Green Blue Violet Grey White!''}

% ========================================
% QUESTION 1(c) OR: LDR-Light Dependent Resistor (7 marks)
% Demonstrates: Construction, working, characteristics, applications with diagram
% ========================================

\subsection{પ્રશ્ન 1(c) OR [7 ગુણ]}
\textbf{LDR નું બાંધકામ, કાર્યકારી લાક્ષણિકતાઓ અને એપ્લિકેશન સમજાવો.}

\subsubsection{ઉકેલ}

\textbf{Light Dependent Resistor (LDR)}, જેને photoresistor અથવા photocell પણ કહે છે, એક પેસિવ સેમિકન્ડક્ટર ઉપકરણ છે જેનું રેઝિસ્ટન્સ તેના પર પડતા પ્રકાશની તીવ્રતા સાથે વિપરીત બદલાય છે. તે વ્યાપક રીતે લાઇટ-સેન્સિંગ એપ્લિકેશન્સમાં વપરાય છે.

\paragraph{બાંધકામ:}

\begin{figure}[H]
\centering
\begin{tikzpicture}[scale=1.2]
    % LDR body
    \draw[thick, fill=gray!20] (0,0) rectangle (4,1.5);
    \node at (2,0.75) {Cadmium Sulfide (CdS)};
    
    % Serpentine track
    \draw[thick, red] (0.5,0.3) -- (0.5,1.2) -- (1.5,1.2) -- (1.5,0.3) -- (2.5,0.3) -- (2.5,1.2) -- (3.5,1.2) -- (3.5,0.3);
    
    % Terminals
    \draw[thick] (0,0.75) -- (0.5,0.75);
    \draw[thick] (4,0.75) -- (3.5,0.75);
    \fill (0,0.75) circle (0.08);
    \fill (4,0.75) circle (0.08);
    
    % Light rays
    \foreach \x in {0.8,1.5,2.2,2.9}
        \draw[->,yellow!70!orange, very thick] (\x,2.5) -- (\x,1.8);
    \node[above] at (2,2.5) {પ્રકાશ};
    
    % Labels
    \node[below] at (0,0) {ટર્મિનલ 1};
    \node[below] at (4,0) {ટર્મિનલ 2};
    \node[right] at (4,1.2) {ઝિગઝેગ ઇલેક્ટ્રોડ};
\end{tikzpicture}
\caption{LDR બાંધકામ}
\end{figure}

LDR માં સિરામિક સબસ્ટ્રેટ પર જમા કરાયેલ \textbf{સેમિકન્ડક્ટર સામગ્રી} (સામાન્ય રીતે Cadmium Sulfide - CdS અથવા Cadmium Selenide - CdSe) હોય છે. મહત્તમ સંપર્ક વિસ્તાર પ્રદાન કરવા માટે સેમિકન્ડક્ટર સપાટી પર ઝિગઝેગ પેટર્નમાં બે મેટલ ઇલેક્ટ્રોડ્સ મૂકવામાં આવે છે. સંવેદનશીલ સામગ્રી સુધી પ્રકાશ પહોંચી શકે તે માટે સંપૂર્ણ એસેમ્બલી પારદર્શક અથવા અર્ધ-પારદર્શક કેસિંગમાં સમાવિષ્ટ છે.

\paragraph{કાર્યનો સિદ્ધાંત:}

LDR નું સંચાલન \textbf{ફોટોકન્ડક્ટિવિટી} ઘટના પર આધારિત છે. અંધારામાં, સેમિકન્ડક્ટર સામગ્રીમાં ખૂબ ઊંચું રેઝિસ્ટન્સ હોય છે (અનેક megaohms હોઈ શકે છે) કારણ કે થોડા મુક્ત ચાર્જ વાહકો હોય છે. જ્યારે પ્રકાશ ફોટોન્સ સેમિકન્ડક્ટર સપાટી પર અથડાય છે, ત્યારે તેઓ ઇલેક્ટ્રોનને ઊર્જા આપે છે, જેના કારણે તેઓ અણુઓમાંથી મુક્ત થાય છે અને \emph{મુક્ત ચાર્જ વાહકો} બને છે. આ વાહકતા વધારે છે અને આમ રેઝિસ્ટન્સ ઘટે છે. પ્રકાશની તીવ્રતા સાથે રેઝિસ્ટન્સ ફેરફાર લગભગ લોગારિધમિક છે.

\paragraph{VI લાક્ષણિકતાઓ:}

VI લાક્ષણિકતા સ્થિર પ્રકાશ તીવ્રતા પર રેખીય સંબંધ (Ohmic વર્તન) દર્શાવે છે. વિવિધ પ્રકાશ સ્તરો વિવિધ ઢાળ ઉત્પન્ન કરે છે, ઊંચી પ્રકાશ તીવ્રતા વધુ ઢાળવાળા ઢાળ (નીચું રેઝિસ્ટન્સ) ઉત્પન્ન કરે છે. લાઇટ સ્થિતિમાં, LDR નીચું રેઝિસ્ટન્સ (થોડાક k\(\Omega\)) દર્શાવે છે, જેનાથી આપેલ વોલ્ટેજ માટે વધુ કરંટ વહે છે. ડાર્ક સ્થિતિમાં, રેઝિસ્ટન્સ ઊંચું હોય છે (M\(\Omega\) રેન્જમાં), જેના પરિણામે સમાન વોલ્ટેજ માટે ખૂબ ઓછો કરંટ વહે છે.

\paragraph{મુખ્ય સ્પષ્ટીકરણો:}
\begin{description}
    \item[ડાર્ક રેઝિસ્ટન્સ:] 1\,M\(\Omega\) થી 10\,M\(\Omega\) (સામાન્ય)
    \item[લાઇટ રેઝિસ્ટન્સ:] થોડાક સો ohms થી થોડાક k\(\Omega\)
    \item[રિસ્પોન્સ ટાઇમ:] પ્રમાણમાં ધીમી (milliseconds), photodiodes કરતાં ધીમી
    \item[સ્પેક્ટ્રલ રિસ્પોન્સ:] દૃશ્ય પ્રકાશ સ્પેક્ટ્રમમાં પીક સંવેદનશીલતા (CdS માટે 500-700\,nm)
\end{description}

\subparagraph{તાપમાન અસર:}
LDR રેઝિસ્ટન્સ તાપમાન સાથે પણ થોડું બદલાય છે, જે પ્રિસિઝન એપ્લિકેશન્સમાં ભૂલો રજૂ કરી શકે છે.

\paragraph{એપ્લિકેશન્સ:}
\begin{itemize}
    \item \textbf{ઓટોમેટિક સ્ટ્રીટ લાઇટ્સ:} આસપાસના પ્રકાશના આધારે ચાલુ/બંધ થાય છે
    \item \textbf{લાઇટ મીટર્સ:} કેમેરા અને ફોટોગ્રાફી સાધનોમાં
    \item \textbf{બર્ગલર એલાર્મ્સ:} બીમ વિક્ષેપ શોધે છે
    \item \textbf{સોલર ટ્રેકિંગ સિસ્ટમ્સ:} સોલર પેનલ અભિમુખતા ઑપ્ટિમાઇઝ કરે છે
    \item \textbf{ડિસ્પ્લે બ્રાઇટનેસ કંટ્રોલ:} સ્ક્રીન બ્રાઇટનેસ આપોઆપ ગોઠવો
\end{itemize}

\paragraph{મેમરી ટ્રીક:}
\emph{``LDR: Light Decreases Resistance -- વધુ પ્રકાશ, ઓછું રેઝિસ્ટન્સ!''}


% ========================================
% QUESTION 2(a): Classify Resistors Based on Materials (3 marks)
% ========================================

\section{પ્રશ્ન 2}

\subsection{પ્રશ્ન 2(a) [3 ગુણ]}
\textbf{સામગ્રીના આધારે રેઝિસ્ટરને વગીકૃત કરો.}

\subsubsection{ઉકેલ}

રેઝિસ્ટર્સને તેમના બાંધકામમાં વપરાયેલ \textbf{રેઝિસ્ટિવ સામગ્રી} ના આધારે વર્ગીકૃત કરવામાં આવે છે. સામગ્રી રેઝિસ્ટરની લાક્ષણિકતાઓ જેવી કે સ્થિરતા, ટોલરન્સ, તાપમાન ગુણાંક, પાવર રેટિંગ અને કિંમત નક્કી કરે છે.

\paragraph{સામગ્રી પર આધારિત વર્ગીકરણ:}
\begin{description}
    \item[કાર્બન કમ્પોઝિશન રેઝિસ્ટર્સ:] કાર્બન કણો અને ઇન્સ્યુલેટિંગ બાઇન્ડરના મિશ્રણમાંથી બનાવવામાં આવે છે. સસ્તા પરંતુ નબળી સ્થિરતા અને ઊંચો તાપમાન ગુણાંક. ટોલરન્સ સામાન્ય રીતે \(\pm 5\%\) થી \(\pm 20\%\).
    \item[કાર્બન ફિલ્મ રેઝિસ્ટર્સ:] સિરામિક સળિયા પર પાતળી કાર્બન ફિલ્મ જમા કરીને બનાવવામાં આવે છે. કાર્બન કમ્પોઝિશન કરતાં સારી સ્થિરતા, ઓછો અવાજ, ટોલરન્સ \(\pm 2\%\) થી \(\pm 5\%\).
    \item[મેટલ ફિલ્મ રેઝિસ્ટર્સ:] સિરામિક સબસ્ટ્રેટ પર પાતળી મેટલ (સામાન્ય રીતે nickel-chromium) ફિલ્મ જમા કરીને બનાવવામાં આવે છે. ઉત્કૃષ્ટ સ્થિરતા, ઓછો તાપમાન ગુણાંક, ચુસ્ત ટોલરન્સ (\(\pm 0.1\%\) થી \(\pm 2\%\)), પ્રિસિઝન સર્કિટ્સમાં વપરાય છે.
    \item[વાયર-વાઉન્ડ રેઝિસ્ટર્સ:] સિરામિક કોર પર રેઝિસ્ટન્સ વાયર (nichrome, manganin) વીંટાળીને બનાવવામાં આવે છે. ઊંચી પાવર હેન્ડલિંગ ક્ષમતા, ખૂબ સ્થિર, પાવર એપ્લિકેશન્સમાં વપરાય છે. ઊંચી ફ્રિક્વન્સીઓ પર inductive અસરો હોઈ શકે છે.
    \item[મેટલ ઓક્સાઇડ રેઝિસ્ટર્સ:] સિરામિક કોર પર મેટલ ઓક્સાઇડ ફિલ્મ જમા કરીને બનાવવામાં આવે છે. ઉત્કૃષ્ટ ઊંચા-તાપમાન સ્થિરતા, ઊંચી વિશ્વસનીયતા જરૂરી એપ્લિકેશન્સમાં વપરાય છે.
\end{description}

\paragraph{મેમરી ટ્રીક:}
\emph{``CCMWM: Carbon, Carbon-film, Metal-film, Wire-wound, Metal-oxide -- સામગ્રી રેઝિસ્ટર્સ બનાવે છે!''}

% ========================================
% QUESTION 2(b): Calculate Resistor Values (4 marks)
% ========================================

\subsection{પ્રશ્ન 2(b) [4 ગુણ]}
\textbf{આપેલ રંગ કોડ માટે રેઝિસ્ટરની કિંમતની ગણતરી કરો. (i) Brown, Black, Yellow, Golden (ii) Yellow, Violet, Red, Silver}

\subsubsection{ઉકેલ}

સ્ટાન્ડર્ડ રેઝિસ્ટર કલર કોડનો ઉપયોગ કરીને, અમે દરેક આપેલ રંગ સંયોજન માટે રેઝિસ્ટન્સ વેલ્યુની ગણતરી કરીએ છીએ.

\paragraph{(i) Brown, Black, Yellow, Golden:}

\begin{itemize}
    \item બેન્ડ 1 (Brown): 1 (પ્રથમ અંક)
    \item બેન્ડ 2 (Black): 0 (બીજો અંક)
    \item બેન્ડ 3 (Yellow): \(\times 10^4\) (મલ્ટિપ્લાયર)
    \item બેન્ડ 4 (Golden): \(\pm 5\%\) (ટોલરન્સ)
\end{itemize}

\[
R_1 = (10) \times 10^4 = 100{,}000\,\Omega = 100\,k\Omega \pm 5\%
\]

\subparagraph{ટોલરન્સ રેન્જ:}
લઘુત્તમ = \(100{,}000 - 5{,}000 = 95{,}000\,\Omega = 95\,k\Omega\)

મહત્તમ = \(100{,}000 + 5{,}000 = 105{,}000\,\Omega = 105\,k\Omega\)

\paragraph{(ii) Yellow, Violet, Red, Silver:}

\begin{itemize}
    \item બેન્ડ 1 (Yellow): 4 (પ્રથમ અંક)
    \item બેન્ડ 2 (Violet): 7 (બીજો અંક)
    \item બેન્ડ 3 (Red): \(\times 10^2\) (મલ્ટિપ્લાયર)
    \item બેન્ડ 4 (Silver): \(\pm 10\%\) (ટોલરન્સ)
\end{itemize}

\[
R_2 = (47) \times 10^2 = 4{,}700\,\Omega = 4.7\,k\Omega \pm 10\%
\]

\subparagraph{ટોલરન્સ રેન્જ:}
લઘુત્તમ = \(4{,}700 - 470 = 4{,}230\,\Omega = 4.23\,k\Omega\)

મહત્તમ = \(4{,}700 + 470 = 5{,}170\,\Omega = 5.17\,k\Omega\)

\paragraph{સારાંશ:}
\begin{description}
    \item[રેઝિસ્ટર 1:] \(100\,k\Omega \pm 5\%\) (રેન્જ: 95k\(\Omega\) થી 105k\(\Omega\))
    \item[રેઝિસ્ટર 2:] \(4.7\,k\Omega \pm 10\%\) (રેન્જ: 4.23k\(\Omega\) થી 5.17k\(\Omega\))
\end{description}

\paragraph{મેમરી ટ્રીક:}
\emph{``બેન્ડ 1-2 અંકો તરીકે વાંચો, બેન્ડ 3 શૂન્યો ઉમેરે, બેન્ડ 4 ટોલરન્સ દર્શાવે છે!''}

% ========================================
% QUESTION 2(c): Electrolytic Capacitor Construction (7 marks)
% ========================================

\subsection{પ્રશ્ન 2(c) [7 ગુણ]}
\textbf{ઇલેક્ટ્રોલીટીક કેપેસિટર્સનું બાંધકામ અને સંચાલન સમજાવો.}

\subsubsection{ઉકેલ}

\textbf{ઇલેક્ટ્રોલીટીક કેપેસિટર્સ} એ પોલરાઇઝ્ડ કેપેસિટર્સ છે જે એક પ્લેટ તરીકે ઇલેક્ટ્રોલાઇટ અને ડાયલેક્ટ્રિક તરીકે ઓક્સાઇડ લેયરનો ઉપયોગ કરે છે. તેઓ કોમ્પેક્ટ કદમાં ખૂબ ઊંચી કેપેસિટન્સ વેલ્યુઓ પ્રદાન કરે છે, જે તેમને પાવર સપ્લાય સર્કિટ્સમાં આવશ્યક બનાવે છે.

\paragraph{બાંધકામ:}

\begin{figure}[H]
\centering
\begin{tikzpicture}[scale=1.2]
    % Outer casing
    \draw[thick, fill=gray!30] (-0.3,-1.5) rectangle (2.3,1.5);
    
    % Anode foil (aluminum)
    \draw[thick, fill=orange!40] (0,-1.2) rectangle (0.3,1.2);
    \node[rotate=90, font=\small] at (0.15,0) {Al એનોડ};
    
    % Oxide layer (dielectric)
    \draw[thick, fill=blue!20] (0.3,-1.2) rectangle (0.45,1.2);
    \node[rotate=90, font=\tiny] at (0.375,0) {Al\(_2\)O\(_3\)};
    
    % Electrolyte
    \draw[thick, fill=green!30] (0.45,-1.2) rectangle (1.55,1.2);
    \node[rotate=90] at (1,0) {ઇલેક્ટ્રોલાઇટ};
    
    % Cathode foil
    \draw[thick, fill=orange!40] (1.55,-1.2) rectangle (1.85,1.2);
    \node[rotate=90, font=\small] at (1.7,0) {Al કેથોડ};
    
    % Terminals
    \draw[very thick, red] (0.15,1.5) -- (0.15,2) node[above] {+ (એનોડ)};
    \draw[very thick] (1.7,1.5) -- (1.7,2) node[above] {- (કેથોડ)};
    
    % Labels
    \node[below, font=\small] at (1,-1.5) {એલ્યુમિનિયમ કેન};
    \draw[->, thick] (2.5,0.8) -- (1.85,0.8) node[right, xshift=0.7cm] {સેપરેટર પેપર};
\end{tikzpicture}
\caption{ઇલેક્ટ્રોલીટીક કેપેસિટર ક્રોસ-સેક્શન}
\end{figure}

\paragraph{મુખ્ય ઘટકો:}
\begin{description}
    \item[એનોડ ફોઇલ:] એલ્યુમિનિયમ ફોઇલ કે જે સપાટીનો વિસ્તાર વધારવા માટે ઇલેક્ટ્રોકેમિકલ રીતે etched કરવામાં આવે છે. પોઝિટિવ ટર્મિનલ.
    \item[ડાયલેક્ટ્રિક લેયર:] એનોડ સપાટી પર એનોડાઇઝેશન દ્વારા બનેલી પાતળી એલ્યુમિનિયમ ઓક્સાઇડ (Al\(_2\)O\(_3\)) લેયર. જાડાઈ સામાન્ય રીતે 1-10\,nm, ખૂબ ઊંચી કેપેસિટન્સ પ્રદાન કરે છે.
    \item[ઇલેક્ટ્રોલાઇટ:] પ્રવાહી, જેલ અથવા ઘન ઇલેક્ટ્રોલાઇટ વાસ્તવિક કેથોડ (નેગેટિવ પ્લેટ) તરીકે સેવા આપે છે. આયોનિક વહન જાળવે છે અને ઓક્સાઇડ લેયરને સુધારે છે.
    \item[કેથોડ ફોઇલ:] ઇલેક્ટ્રોલાઇટના સંપર્કમાં બીજી એલ્યુમિનિયમ ફોઇલ, નેગેટિવ ટર્મિનલ સાથે જોડાયેલ.
    \item[સેપરેટર:] ઇલેક્ટ્રોલાઇટમાં ભીનું પોરસ પેપર, એનોડ અને કેથોડ વચ્ચે સીધો સંપર્ક અટકાવે છે.
    \item[કેસિંગ:] સેફ્ટી વેન્ટ સાથે સંપૂર્ણ એસેમ્બલીને ઘેરી લેતું એલ્યુમિનિયમ કેન.
\end{description}

\paragraph{સંચાલન સિદ્ધાંત:}

કેપેસિટન્સ ઓક્સાઇડ લેયરની જાડાઈ અને સપાટીના વિસ્તાર દ્વારા નક્કી થાય છે. જ્યારે સાચી પોલેરિટી સાથે DC વોલ્ટેજ લાગુ કરવામાં આવે છે, ત્યારે ઓક્સાઇડ લેયર ડાયલેક્ટ્રિક તરીકે કાર્ય કરે છે. ખૂબ પાતળી ઓક્સાઇડ લેયર (\(\sim\)1.4\,nm પ્રતિ volt) અને મોટો એનોડ સપાટી વિસ્તાર ઊંચી કેપેસિટન્સ વેલ્યુઝ (સામાન્ય રીતે 1\,\(\mu\)F થી 10{,}000\,\(\mu\)F અથવા વધુ) આપે છે.

\subparagraph{કેપેસિટન્સ ફોર્મ્યુલા:}
\[
C = \frac{\epsilon_0 \epsilon_r A}{d}
\]
જ્યાં Al\(_2\)O\(_3\) માટે \(\epsilon_r \approx 8-10\), etching ને લીધે \(A\) મોટું છે, \(d\) ખૂબ નાનું છે (ઓક્સાઇડ જાડાઈ).

\paragraph{મહત્વપૂર્ણ લાક્ષણિકતાઓ:}
\begin{itemize}
    \item \textbf{પોલેરિટી:} સાચી પોલેરિટી સાથે જોડવી આવશ્યક છે. રિવર્સ વોલ્ટેજ ઓક્સાઇડ લેયરને નુકસાન પહોંચાડે છે જેના કારણે નિષ્ફળતા અથવા વિસ્ફોટ થાય છે.
    \item \textbf{ઊંચી કેપેસિટન્સ:} નાના પેકેજમાં 1\,\(\mu\)F થી કેટલાક ફેરાડ્સ સુધીની વેલ્યુઝ.
    \item \textbf{વોલ્ટેજ રેટિંગ:} સામાન્ય રીતે 6.3V થી 450V. રેટેડ વોલ્ટેજ ક્યારેય ઓળંગશો નહીં.
    \item \textbf{ESR (Equivalent Series Resistance):} ripple current ક્ષમતા અને heating ને અસર કરે છે.
    \item \textbf{લીકેજ કરંટ:} ડાયલેક્ટ્રિકમાંથી નાનો DC કરંટ વહે છે, જે સામાન્ય છે પરંતુ ઉંમર અને તાપમાન સાથે વધે છે.
    \item \textbf{સેલ્ફ-હીલિંગ:} સ્ટોરેજ પછી ધીમે ધીમે વોલ્ટેજ લાગુ કરવામાં આવે તો ઓક્સાઇડ લેયરને અમુક અંશે સુધારી શકે છે.
\end{itemize}

\subparagraph{નિષ્ફળતા મોડ્સ:}
રિવર્સ પોલેરિટી, ઓવરવોલ્ટેજ, ઓવરહીટિંગ અથવા aging ના કારણે ઇલેક્ટ્રોલાઇટ બાષ્પીભવન, ઓક્સાઇડ breakdown અથવા દબાણ buildup થઈ શકે છે જે વેન્ટ એક્ટિવેશન અથવા catastrophic નિષ્ફળતા તરફ દોરી જાય છે.

\paragraph{એપ્લિકેશન્સ:}
\begin{enumerate}
    \item \textbf{પાવર સપ્લાય ફિલ્ટરિંગ:} DC પાવર સપ્લાયમાં રેક્ટિફાઇડ AC ને smoothing કરવું
    \item \textbf{કપલિંગ/ડિકપલિંગ:} DC ને બ્લોક કરતી વખતે AC સિગ્નલ્સને પાસ કરવા
    \item \textbf{એનર્જી સ્ટોરેજ:} કેમેરા ફ્લેશ, ઓડિયો એમ્પ્લિફાયર્સ, મોટર સ્ટાર્ટિંગ સર્કિટ્સમાં
    \item \textbf{ટાઇમિંગ સર્કિટ્સ:} જ્યાં મોટા time constants જરૂરી હોય
\end{enumerate}

\paragraph{મેમરી ટ્રીક:}
\emph{``ELECTROLYTIC: Etched anode, Large capacitance, Electrolyte cathode, Correct Polarity, Thin Oxide dielectric, Liquid inside, aloYed for high-C, Thin layer yields high capacitance, Ionically conducting, Can explode if reversed!''}


% ========================================
% QUESTION 3(a): Filter Circuit Importance (3 marks)
% Section starting from Question paper line 36
% ========================================

\section{પ્રશ્ન 3}

\subsection{પ્રશ્ન 3(a) [3 ગુણ]}
\textbf{રેક્ટિફાયરમાં ફિલ્ટર સર્કિટનું મહત્વ જણાવો.}

\subsubsection{ઉકેલ}

\textbf{ફિલ્ટર સર્કિટ} એ રેક્ટિફાયર સિસ્ટમ્સમાં એક આવશ્યક ઘટક છે જે રેક્ટિફાયરમાંથી pulsating DC આઉટપુટને ઇલેક્ટ્રોનિક સર્કિટ્સ માટે યોગ્ય steady DC વોલ્ટેજમાં smooth કરે છે. ફિલ્ટરિંગ વગર, rectified આઉટપુટમાં નોંધપાત્ર AC ripple કોમ્પોનન્ટ્સ હોય છે જે સંવેદનશીલ ઇલેક્ટ્રોનિક કોમ્પોનન્ટ્સને નુકસાન પહોંચાડી શકે છે.

\paragraph{ફિલ્ટર સર્કિટ્સનું મહત્વ:}
\begin{description}
    \item[રિપલ ઘટાડો:] pulsating DC માંથી AC કોમ્પોનન્ટ્સને દૂર કરે છે, smooth DC આઉટપુટ પ્રદાન કરે છે. ripple factor ને 1.21 (half-wave) અથવા 0.48 (full-wave) થી શૂન્ય નજીક ઘટાડે છે.
    \item[વોલ્ટેજ રેગ્યુલેશન:] લોડ કરંટ અથવા ઇનપુટ AC વોલ્ટેજમાં ફેરફારો હોવા છતાં પ્રમાણમાં સતત DC વોલ્ટેજ જાળવે છે.
    \item[સર્કિટ પ્રોટેક્શન:] ICs, ટ્રાન્ઝિસ્ટર્સ અને ઓપરેશનલ એમ્પ્લિફાયર્સ જેવા સંવેદનશીલ ઇલેક્ટ્રોનિક કોમ્પોનન્ટ્સને નુકસાન થતું અટકાવે છે જેને pure DC જરૂરી છે.
    \item[સુધારેલ કાર્યક્ષમતા:] AC ripple કોમ્પોનન્ટ્સમાં પાવર લોસ ઘટાડીને કાર્યક્ષમ પાવર ટ્રાન્સફર સક્ષમ કરે છે.
    \item[અવાજ ઘટાડો:] ripple વોલ્ટેજને કારણે ઓડિયો અને કમ્યુનિકેશન સાધનોમાં ઇલેક્ટ્રિકલ અવાજ અને hum ઘટાડે છે.
\end{description}

\paragraph{મેમરી ટ્રીક:}
\emph{``FILTER: Flatten ripples, Improve voltage stability, Less noise, Tame pulsations, Enable smooth DC, Regulate power!''}

% ========================================
% QUESTION 3(b): P-type vs N-type Semiconductors (4 marks)
% ========================================

\subsection{પ્રશ્ન 3(b) [4 ગુણ]}
\textbf{P પ્રકાર સેમિકન્ડક્ટર અને N પ્રકાર સેમિકન્ડક્ટર વચ્ચે તફાવત કરો.}

\subsubsection{ઉકેલ}

સેમિકન્ડક્ટર્સને intrinsic (શુદ્ધ) સેમિકન્ડક્ટર સામગ્રીમાં ઉમેરવામાં આવેલ doping impurity ના પ્રકાર પર આધારિત \textbf{P-type} અને \textbf{N-type} તરીકે વર્ગીકૃત કરવામાં આવે છે.

\begin{table}[H]
\centering
\caption{P-type બનામ N-type સેમિકન્ડક્ટર સરખામણી}
\begin{tabularx}{\textwidth}{lXX}
\toprule
\textbf{પેરામીટર} & \textbf{P-type સેમિકન્ડક્ટર} & \textbf{N-type સેમિકન્ડક્ટર} \\
\midrule
Doping & Trivalent impurity (Boron, Gallium, Indium) & Pentavalent impurity (Phosphorus, Arsenic, Antimony) \\
મુખ્ય વાહકો & Holes (પોઝિટિવ ચાર્જ વાહકો) & Electrons (નેગેટિવ ચાર્જ વાહકો) \\
ગૌણ વાહકો & Electrons & Holes \\
ચાર્જ & એકંદરે વિદ્યુત તટસ્થ & એકંદરે વિદ્યુત તટસ્થ \\
વાહકતા & hole concentration સાથે વધે છે & electron concentration સાથે વધે છે \\
એનર્જી લેવલ & Valence band નજીક acceptor energy level & Conduction band નજીક donor energy level \\
પ્રતીક & P & N \\
\bottomrule
\end{tabularx}
\end{table}

\paragraph{P-type રચના:}
જ્યારે trivalent impurity (3 valence electrons) ને silicon (4 valence electrons) માં ઉમેરવામાં આવે છે, ત્યારે તે covalent bond structure માં \emph{hole} અથવા vacancy બનાવે છે. આ holes પોઝિટિવ ચાર્જ વાહકો તરીકે કાર્ય કરે છે. impurity atoms ને \textbf{acceptors} કહેવામાં આવે છે કારણ કે તેઓ electrons સ્વીકારે છે.

\paragraph{N-type રચના:}
જ્યારે pentavalent impurity (5 valence electrons) ને silicon (4 valence electrons) માં ઉમેરવામાં આવે છે, ત્યારે વધારાનો electron મુક્ત ચાર્જ વાહક બને છે. આ મુક્ત electrons કરંટ conduct કરે છે. impurity atoms ને \textbf{donors} કહેવામાં આવે છે કારણ કે તેઓ electrons દાન કરે છે.

\subparagraph{મુખ્ય મુદ્દો:}
બંને વિદ્યુત તટસ્થ રહે છે કારણ કે એકંદર સ્ટ્રક્ચરમાં protons ની સંખ્યા electrons ની સંખ્યાની બરાબર છે.

\paragraph{મેમરી ટ્રીક:}
\emph{``P: Trivalent થી પોઝિટિવ holes; N: Pentavalent થી નેગેટિવ electrons!''}

% ========================================
% QUESTION 3(c): Bridge Rectifier with Waveforms (7 marks)
% ========================================

\subsection{પ્રશ્ન 3(c) [7 ગુણ]}
\textbf{વેવફોર્મ્સ સાથે બ્રિજ રેક્ટિફાયરનું કાર્ય સમજાવો.}

\subsubsection{ઉકેલ}

\textbf{બ્રિજ રેક્ટિફાયર} એ full-wave rectifier સર્કિટ છે જે AC વોલ્ટેજને pulsating DC માં કન્વર્ટ કરવા માટે બ્રિજ રૂપરેખાંકનમાં ચાર ડાયોડ્સનો ઉપયોગ કરે છે. તે ઇનપુટ AC waveform ના બંને half-cycles નો ઉપયોગ કરે છે, half-wave rectifiers કરતાં સારી કાર્યક્ષમતા પ્રદાન કરે છે.

\paragraph{સર્કિટ ડાયાગ્રામ:}

\begin{figure}[H]
\centering
\begin{circuitikz}[scale=1.3]
    % AC Source
    \draw (0,0) to[sV, l=\(V_{in}\)] (0,3);
    
    % Bridge diodes
    \draw (0,3) -- (2,3);
    \draw (2,3) to[D*, l=\(D_1\)] (4,3);
    \draw (4,3) -- (6,3);
    \draw (6,3) to[D*, l=\(D_2\)] (6,1.5);
    
    \draw (0,0) -- (2,0);
    \draw (2,0) to[D*, l=\(D_4\)] (4,0);
    \draw (4,0) -- (6,0);
    \draw (6,0) to[D*, l_=\(D_3\)] (6,1.5);
    
    % Load resistor
    \draw (6,1.5) to[R, l=\(R_L\)] (6,1.5);
    \draw (6,3) -- (7,3) node[above] {\(+\)};
    \draw (6,0) -- (7,0) node[below] {\(-\)};
    
    % Output voltage
    \draw[<->] (7.5,3) -- (7.5,0) node[midway, right] {\(V_{out}\)};
\end{circuitikz}
\caption{બ્રિજ રેક્ટિફાયર સર્કિટ}
\end{figure}

\paragraph{કાર્યકારી સિદ્ધાંત:}

બ્રિજ રેક્ટિફાયર બે half-cycles માં કાર્ય કરે છે:

\subparagraph{પોઝિટિવ હાફ-સાઇકલ:}
જ્યારે AC ઇનપુટ પોઝિટિવ હોય (ટોપ ટર્મિનલ પોઝિટિવ, બોટમ ટર્મિનલ નેગેટિવ), ડાયોડ્સ \(D_1\) અને \(D_3\) forward-biased થાય છે અને conduct કરે છે, જ્યારે \(D_2\) અને \(D_4\) reverse-biased થાય છે અને block કરે છે. કરંટ પાથ: AC સ્રોત \(\rightarrow\) \(D_1\) \(\rightarrow\) લોડ \(R_L\) \(\rightarrow\) \(D_3\) \(\rightarrow\) AC સ્રોત. આઉટપુટ વોલ્ટેજ લોડ પર દેખાય છે.

\subparagraph{નેગેટિવ હાફ-સાઇકલ:}
જ્યારે AC ઇનપુટ નેગેટિવ હોય (ટોપ ટર્મિનલ નેગેટિવ, બોટમ ટર્મિનલ પોઝિટિવ), ડાયોડ્સ \(D_2\) અને \(D_4\) forward-biased થાય છે અને conduct કરે છે, જ્યારે \(D_1\) અને \(D_3\) reverse-biased થાય છે અને block કરે છે. કરંટ પાથ: AC સ્રોત \(\rightarrow\) \(D_2\) \(\rightarrow\) લોડ \(R_L\) \(\rightarrow\) \(D_4\) \(\rightarrow\) AC સ્રોત. આઉટપુટ વોલ્ટેજ લોડ પર સમાન polarity સાથે દેખાય છે.

\paragraph{વેવફોર્મ્સ:}

\begin{figure}[H]
\centering
\begin{tikzpicture}[scale=1.0]
    % Input AC waveform
    \draw[->, thick] (0,0) -- (8,0) node[right] {\(t\)};
    \draw[->, thick] (0,-2) -- (0,2.5) node[above] {\(V_{in}\)};
    \draw[blue, very thick] plot[domain=0:7.85, samples=100] (\x, {1.5*sin(\x r)});
    \node[blue] at (4,2.2) {ઇનપુટ AC};
    
    % Output waveform
    \begin{scope}[yshift=-5.5cm]
        \draw[->, thick] (0,0) -- (8,0) node[right] {\(t\)};
        \draw[->, thick] (0,-0.3) -- (0,2.5) node[above] {\(V_{out}\)};
        \draw[red, very thick] plot[domain=0:7.85, samples=100] (\x, {1.5*abs(sin(\x r))});
        \node[red] at (4,2.2) {આઉટપુટ Pulsating DC};
    \end{scope}
\end{tikzpicture}
\caption{બ્રિજ રેક્ટિફાયર ઇનપુટ અને આઉટપુટ વેવફોર્મ્સ}
\end{figure}

\paragraph{મુખ્ય પેરામીટર્સ:}
\begin{description}
    \item[કાર્યક્ષમતા:] \(\eta = 81.2\%\) (થિયરેટિકલ મેક્સિમમ, half-wave કરતાં બમણી)
    \item[રિપલ ફેક્ટર:] \(r = 0.48\) (half-wave 1.21 કરતાં ખૂબ ઓછું)
    \item[પીક ઇનવર્સ વોલ્ટેજ (PIV):] \(PIV = V_m\) (દરેક ડાયોડ peak AC વોલ્ટેજ સહન કરી શકે)
    \item[DC આઉટપુટ:] \(V_{DC} = \frac{2V_m}{\pi} = 0.636 V_m\) (half-wave કરતાં બમણું)
    \item[ફ્રિક્વન્સી:] આઉટપુટ ripple ફ્રિક્વન્સી ઇનપુટ AC ફ્રિક્વન્સી કરતાં બમણી (50Hz ઇનપુટ માટે 100Hz)
\end{description}

\paragraph{ફાયદા:}
\begin{itemize}
    \item કોઈ center-tapped transformer જરૂરી નથી (ઓછો ખર્ચ)
    \item half-wave (40.6\%) ની સરખામણીએ ઊંચી કાર્યક્ષમતા (81.2\%)
    \item ઓછો ripple factor (સરળ ફિલ્ટરિંગ)
    \item સારું DC આઉટપુટ વોલ્ટેજ ઉપયોગ
    \item AC ના બંને half-cycles નો ઉપયોગ થાય છે
\end{itemize}

\paragraph{ગેરફાયદા:}
\begin{itemize}
    \item એકને બદલે ચાર ડાયોડ્સ જરૂરી
    \item બે ડાયોડ્સ એક સાથે conduct કરે છે, પરિણામે ઊંચો વોલ્ટેજ drop (લગભગ 1.4V)
    \item center-tapped full-wave rectifier કરતાં થોડું વધુ જટિલ સર્કિટ
\end{itemize}

\subparagraph{એપ્લિકેશન્સ:}
બ્રિજ rectifiers તેમની કાર્યક્ષમતા અને સરળતાને લીધે કમ્પ્યુટર્સ, મોબાઇલ ચાર્જર્સ, બેટરી ચાર્જર્સ અને ઔદ્યોગિક સાધનો માટે પાવર સપ્લાયમાં વ્યાપક રીતે વપરાય છે.

\paragraph{મેમરી ટ્રીક:}
\emph{``BRIDGE: Both cycles used, Rectifies with Improved efficiency, Diodes in Groups of 2, Gives smoother output, Economical (no center-tap)!''}


% ========================================
% QUESTION 4, 5, and 6 - Gujarati versions
% ========================================

\section{પ્રશ્ન 4}

\subsection{પ્રશ્ન 4(a) [3 ગુણ]}
\textbf{વ્યાખ્યાયિત કરો (1) PIV (2) રિપલ ફેક્ટર.}

\subsubsection{ઉકેલ}

\paragraph{(1) PIV - Peak Inverse Voltage:}

\textbf{Peak Inverse Voltage (PIV)} એ મેક્સિમમ reverse વોલ્ટેજ છે જે રેક્ટિફાયર સર્કિટમાં ડાયોડ non-conducting half-cycle દરમિયાન reverse-biased હોય ત્યારે સહન કરવાનું હોય છે. તે non-conducting ડાયોડ પર દેખાતા AC વોલ્ટેજની peak value દર્શાવે છે. ડાયોડ્સને breakdown અટકાવવા માટે PIV handle કરવા માટે rated હોવા જોઈએ. half-wave rectifier માટે PIV \(= V_m\), center-tap full-wave માટે PIV \(= 2V_m\), અને bridge rectifier માટે PIV \(= V_m\), જ્યાં \(V_m\) peak AC voltage છે.

\paragraph{(2) રિપલ ફેક્ટર:}

\textbf{રિપલ ફેક્ટર (r)} એ AC ને pure DC માં કન્વર્ટ કરવામાં રેક્ટિફાયર અને ફિલ્ટર સર્કિટની અસરકારકતાનું માપ છે છે. તેને આઉટપુટમાં DC component ના AC component ના RMS value ના રેશિયો તરીકે વ્યાખ્યાયિત કરવામાં આવે છે. ગાણિતિક રૂપે:

\[
r = \frac{V_{ac}(rms)}{V_{dc}} = \frac{\sqrt{V_{rms}^2 - V_{dc}^2}}{V_{dc}}
\]

ઓછો ripple factor સારા rectification અને filtering સૂચવે છે. આદર્શ DC માં \(r = 0\). Half-wave rectifier માં \(r = 1.21\), full-wave rectifier માં \(r = 0.48\).

\paragraph{મેમરી ટ્રીક:}
\emph{``PIV: Peak Inverse Voltage ડાયોડ્સે સહન કરવું જોઈએ; Ripple: જીવંત AC components દર્શાવતો રેશિયો!''}

\subsection{પ્રશ્ન 4(b) [4 ગુણ]}
\textbf{PN જંકશન ડાયોડની VI લાક્ષણિકતાઓ સમજાવો.}

\subsubsection{ઉકેલ}

PN junction diode ની \textbf{Voltage-Current (VI) લાક્ષણિકતા} ડાયોડ પર લાગુ કરવામાં આવેલ વોલ્ટેજ અને તેમાંથી વહેતા કરંટ વચ્ચેનો સંબંધ દર્શાવે છે.

\begin{figure}[H]
\centering
\begin{tikzpicture}[scale=1.1]
    % Axes
    \draw[->] (-3,0) -- (3,0) node[right] {\(V\) (વોલ્ટેજ)};
    \draw[->] (0,-2) -- (0,3) node[above] {\(I\) (કરંટ)};
    
    % Forward bias curve
    \draw[thick, blue, domain=0.6:2.4] plot (\x, {0.8*(\x-0.6)^2});
    
    % Reverse bias curve
    \draw[thick, red] (-2.5,-0.3) -- (-0.3,-0.3);
    
    % Breakdown
    \draw[thick, red] (-2.5,-0.3) -- (-2.5,-1.8);
    
    % Labels
    \node at (1.8,2.5) [blue] {Forward bias};
    \node at (-1.8,-0.6) [red] {Reverse bias};
    \node at (-2.2,-1.8) [red] {Breakdown};
    
    % Threshold voltage
    \draw[dashed] (0.7,0) -- (0.7,0.3);
    \node[below] at (0.7,-0.1) {\(V_{\gamma}\)};
    
    % Reverse saturation current
    \draw[dashed] (0,-0.3) -- (-0.5,-0.3);
    \node[left] at (-0.1,-0.3) {\(I_S\)};
\end{tikzpicture}
\caption{PN Junction Diode ની VI લાક્ષણિકતાઓ}
\end{figure}

\paragraph{Forward Bias પ્રદેશ:}
જ્યારે positive terminal P-side સાથે અને negative N-side સાથે જોડવામાં આવે છે, barrier potential ઘટે છે. જ્યાં સુધી વોલ્ટેજ threshold ને ઓળંગતું નથી (\(V_{\gamma} \approx 0.7V\) Si માટે, Ge માટે 0.3V) ત્યાં સુધી ખૂબ નાનો કરંટ વહે છે. Threshold પછી, કરંટ ઘાતાંકીય રીતે વધે છે: \(I = I_S(e^{V/\eta V_T} - 1)\).

\paragraph{Reverse Bias પ્રદેશ:}
જ્યારે positive terminal N-side સાથે અને negative P-side સાથે જોડવામાં આવે છે, barrier potential વધે છે. minority carriers ને લીધે ખૂબ નાનો reverse saturation કરંટ \(I_S\) (થોડા \(\mu A\)) વહે છે. reverse voltage થી સ્વતંત્ર લગભગ સતત રહે છે.

\paragraph{Breakdown પ્રદેશ:}
મોટા reverse voltage પર, avalanche અથવા Zener effect ને લીધે breakdown થાય છે. કરંટ ઝડપથી વધે છે. આ સામાન્ય ડાયોડ્સને નુકસાન પહોંચાડી શકે છે પરંતુ Zener diodes માં ઉપયોગ થાય છે.

\paragraph{મેમરી ટ્રીક:}
\emph{``VI Curve: Forward voltage threshold, પછી અનંત કરંટ વધારો; Reverse નાનો કરંટ આપે, ઊંચા reverse પર Breakdown!''}

\subsection{પ્રશ્ન 4(c) [7 ગુણ]}
\textbf{વેવફોર્મ્સ સાથે કેપેસિટર ઇનપુટ અને ચોક ઇનપુટ ફિલ્ટરની કામગીરી સમજાવો.}

\subsubsection{ઉકેલ}

ફિલ્ટર સર્કિટ્સ rectified આઉટપુટમાંથી ripple દૂર કરે છે. બે સામાન્ય પ્રકારો \textbf{કેપેસિટર ઇનપુટ ફિલ્ટર} અને \textbf{ચોક ઇનપુટ ફિલ્ટર} છે.

\paragraph{કેપેસિટર ઇનપુટ ફિલ્ટર:}

\subparagraph{સર્કિટ અને કાર્ય:}
મોટો કેપેસિટર લોડ સાથે parallel માં જોડાયેલ છે. પોઝિટિવ half-cycle દરમિયાન જ્યારે diode conduct કરે છે, કેપેસિટર peak voltage સુધી ચાર્જ થાય છે. જ્યારે ઇનપુટ ઘટે છે, ત્યારે diode reverse-biased બને છે અને કેપેસિટર લોડ દ્વારા discharge થાય છે, વોલ્ટેજ જાળવે છે. કેપેસિટર વારંવાર ચાર્જ અને discharge થાય છે, પ્રમાણમાં smooth DC પ્રદાન કરે છે.

\begin{figure}[H]
\centering
\begin{circuitikz}[scale=1.0]
    % Rectifier diode
    \draw (0,0) to[sV, l=\(V_{in}\)] (0,2);
    \draw (0,2) to[D, l=D] (3,2);
    
    % Capacitor
    \draw (3,2) to[short] (4,2);
    \draw (4,2) to[C, l=\(C\)] (4,0);
    
    % Load resistor
    \draw (4,2) to[short] (5.5,2);
    \draw (5.5,2) to[R, l=\(R_L\)] (5.5,0);
    
    % Ground
    \draw (0,0) -- (5.5,0);
    
    % Output voltage
    \draw[<->] (6,2) -- (6,0) node[midway, right] {\(V_{out}\)};
\end{circuitikz}
\caption{કેપેસિટર ઇનપુટ ફિલ્ટર સર્કિટ}
\end{figure}

\subparagraph{વેવફોર્મ:}
આઉટપુટ વોલ્ટેજ DC level પર નાની ripple riding ધરાવે છે. Ripple amplitude કેપેસિટન્સ અને લોડ resistance પર આધાર રાખે છે: \(V_{ripple} \approx \frac{I_{dc}}{fC}\) જ્યાં \(f\) ripple frequency છે.

\begin{figure}[H]
\centering
\begin{tikzpicture}[scale=0.9]
    % Input waveform
    \begin{scope}
        \draw[->, thick] (0,0) -- (6,0) node[right] {\(t\)};
        \draw[->,  thick] (0,0) -- (0,2) node[above] {\(V\)};
        \draw[blue, very thick] plot[domain=0:6, samples=100] (\x, {1.5*abs(sin(1.57*\x r))});
        \node[blue] at (3,2.2) {Rectified Input};
    \end{scope}
    
    % Output waveform with ripple
    \begin{scope}[yshift=-3.5cm]
        \draw[->, thick] (0,0) -- (6,0) node[right] {\(t\)};
        \draw[->, thick] (0,0) -- (0,2) node[above] {\(V\)};
        \draw[red, very thick] plot[domain=0:6, samples=200] (\x, {1.2 - 0.2*abs(sin(1.57*\x r))});
        \node[red] at (3,2.2) {Filtered Output};
        \draw[dashed] (0,1.2) -- (6,1.2) node[right, font=\small] {\(V_{DC}\)};
    \end{scope}
\end{tikzpicture}
\caption{કેપેસિટર ફિલ્ટર વેવફોર્મ્સ}
\end{figure}

\subparagraph{લાક્ષણિકતાઓ:}
લાઇટ લોડ્સ માટે સારું વોલ્ટેજ regulation, હેવી લોડ્સ માટે નબળું. ચાર્જિંગ દરમિયાન ઊંચો peak diode કરંટ. લો કરંટ applications માં વપરાય છે. Ripple factor \(\approx \frac{1}{4\sqrt{3}fCR_L}\).

\paragraph{ચોક ઇનપુટ ફિલ્ટર:}

\subparagraph{સર્કિટ અને કાર્ય:}
Inductor (choke) rectifier આઉટપુટ સાથે series માં જોડાયેલ છે, ત્યારબાદ લોડ સાથે parallel માં કેપેસિટર. Inductor તેની property \(V_L = L\frac{di}{dt}\) ને લીધે કરંટમાં અચાનક ફેરફારોનો વિરોધ કરે છે. તે કરંટ variations smooth કરે છે. કેપેસિટર પછી બાકી રહેલા ripple voltage ને ફિલ્ટર કરે છે.

\begin{figure}[H]
\centering
\begin{circuitikz}[scale=1.0]
    % Rectifier input
    \draw (0,0) to[sV, l=\(V_{in}\)] (0,2);
    \draw (0,2) to[D, l=D] (2,2);
    
    % Choke/Inductor
    \draw (2,2) to[L, l=\(L\)] (4,2);
    
    % Capacitor
    \draw (4,2) to[short] (5, 2);
    \draw (5,2) to[C, l=\(C\)] (5,0);
    
    % Load resistor
    \draw (5,2) to[short] (6.5,2);
    \draw (6.5,2) to[R, l=\(R_L\)] (6.5,0);
    
    % Ground
    \draw (0,0) -- (6.5,0);
    
    % Output voltage
    \draw[<->] (7,2) -- (7,0) node[midway, right] {\(V_{out}\)};
\end{circuitikz}
\caption{ચોક (LC) ઇનપુટ ફિલ્ટર સર્કિટ}
\end{figure}

\subparagraph{વેવફોર્મ:}
આઉટપુટ વોલ્ટેજ ન્યૂનતમ ripple સાથે ખૂબ smooth છે. L-C સંયોજન ઉત્કૃષ્ટ ફિલ્ટરિંગ પ્રદાન કરે છે.

\subparagraph{લાક્ષણિકતાઓ:}
બદલાતા લોડ્સ હેઠળ સારું વોલ્ટેજ regulation. ઓછો peak diode કરંટ. high current applications માં વપરાય છે. મોટા, ભારે, મોંઘા inductor જરૂરી છે. Ripple factor \(\approx \frac{R_L}{3\sqrt{2}\omega L}\) L filter માટે.

\paragraph{સરખામણી:}
\begin{description}
    \item[કેપેસિટર ફિલ્ટર:] સરળ, સસ્તું, લો કરંટ માટે સારું. હાય કરંટ્સ પર નબળું regulation. ઊંચો peak diode કરંટ.
    \item[ચોક ફિલ્ટર:] સારું regulation, high current માટે સારું. વધુ મોંઘું, ભારે. ઓછો peak diode કરંટ.
\end{description}

\paragraph{મેમરી ટ્રીક:}
\emph{``CAP filter: ઝડપથી ચાર્જ થાય, લાઇટ લોડ્સ માટે કાર્ય કરે, Peak current high; CHOKE filter: કરંટ smooth કરે, Heavy-duty, ફેરફારોનો વિરોધ કરે, regulation રાખે, મોંઘું પણ અસરકારક!''}

\section{પ્રશ્ન 5}

\subsection{પ્રશ્ન 5(a) [3 ગુણ]}
\textbf{ઝેનર ડાયોડનું કાર્ય અને મહત્વ જણાવો.}

\subsubsection{ઉકેલ}

\textbf{ઝેનર ડાયોડ} એ વિશેષ-હેતુ ડાયોડ છે જે reverse breakdown region માં સુરક્ષિત અને વિશ્વસનીય રીતે કાર્ય કરવા માટે designed કરવામાં આવ્યો છે. તેનું પ્રાથમિક કાર્ય \textbf{વોલ્ટેજ regulation} છે.

\paragraph{કાર્ય:}
જ્યારે forward-biased હોય, ત્યારે Zener diode સામાન્ય diode જેવું વર્તે છે. જ્યારે તેના \textbf{Zener breakdown voltage} (\(V_Z\)) પછી reverse-biased હોય, ત્યારે તે કરંટમાં ફેરફારો હોવા છતાં તેના ટર્મિનલ્સ પર લગભગ સતત વોલ્ટેજ જાળવે છે. આ property તેને voltage regulation માટે આદર્શ બનાવે છે.

\paragraph{મહત્વ:}
\begin{description}
    \item[વોલ્ટેજ રેગ્યુલેશન:] ઇનપુટ વોલ્ટેજ અથવા લોડ કરંટ variations હોવા છતાં સતત આઉટપુટ વોલ્ટેજ જાળવે છે. પાવર supplies માં આવશ્યક.
    \item[વોલ્ટેજ રેફરન્સ:] measurement circuits, ADCs અને precision applications માટે સ્થિર reference voltage પ્રદાન કરે છે.
    \item[Over-voltage Protection:] વધારાનો વોલ્ટેજ clip કરીને સંવેદનશીલ સર્કિટ્સને સુરક્ષિત કરે છે.
    \item[Wave Shaping:] waveform shapes સુધારવા માટે clipper અને clamper circuits માં વપરાય છે.
    \item[Meter Protection:] over-voltage નુકસાનથી analog meters ને સુરક્ષિત કરે છે.
\end{description}

\paragraph{મેમરી ટ્રીક:}
\emph{``ZENER: વોલ્ટેજમાં Zero variation, Regulation માટે excellent, Reverse bias જોઈએ, Essential reference, વિશ્વસનીય protection!''}

\subsection{પ્રશ્ન 5(b) [4 ગુણ]}
\textbf{Light emitting diode (LED) ને તેની લાક્ષણિકતા સાથે વર્ણવો.}

\subsubsection{ઉકેલ}

\textbf{Light Emitting Diode (LED)} એ PN junction diode છે જે forward-biased હોય ત્યારે પ્રકાશ emit કરે છે. તે electroluminescence દ્વારા ઇલેક્ટ્રિકલ energy ને સીધી light energy માં કન્વર્ટ કરે છે.

\paragraph{કાર્યકારી સિદ્ધાંત:}
જ્યારે LED દ્વારા forward current વહે છે, ત્યારે N-region માંથી electrons P-region માં holes સાથે junction પર recombine થાય છે. Recombination દરમિયાન, photons (light) ના સ્વરૂપમાં energy મુક્ત થાય છે. Emitted light નો રંગ semiconductor material અને energy band gap પર આધાર રાખે છે.

\paragraph{સામગ્રી અને રંગો:}
\begin{description}
    \item[Red:] Gallium Arsenide Phosphide (GaAsP), \(E_g \approx 1.8\ eV\)
    \item[Green:] Gallium Phosphide (GaP), \(E_g \approx 2.2\ eV\)
    \item[Blue:] Gallium Nitride (GaN), \(E_g \approx 2.9\ eV\)
    \item[White:] yellow phosphor coating સાથે blue LED અથવા RGB સંયોજન
\end{description}

\paragraph{VI લાક્ષણિકતાઓ:}
સામાન્ય diode જેવી પરંતુ ઊંચા forward voltage drop સાથે (\(V_f \approx 1.8-3.5V\) રંગ પર આધાર રાખીને). Series resistor વાપરીને કરંટ મર્યાદિત કરવો જોઈએ. LED reverse bias માં પ્રકાશ emit કરતું નથી. સામાન્ય operating current: 10-20\,mA.

\paragraph{ફાયદા:}
ઓછું power consumption, લાંબું જીવન (50{,}000+ કલાકો), ઝડપી switching, કોમ્પેક્ટ size, વિવિધ રંગોમાં ઉપલબ્ધ, warm-up time નથી, robust, પર્યાવરણને અનુકૂળ (mercury નથી).

\paragraph{એપ્લિકેશન્સ:}
Indicator lights, displays (seven-segment, dot matrix), backlighting, traffic signals, automotive lighting, general illumination, optical communication.

\paragraph{મેમરી ટ્રીક:}
\emph{``LED: electron-hole recombination દરમિયાન મુક્ત થતી Energy થી Light!''}

\subsection{પ્રશ્ન 5(c) [7 ગુણ]}
\textbf{વોલ્ટેજ regulator તરીકે Zener diode નું કાર્ય સમજાવો.}

\subsubsection{ઉકેલ}

\textbf{Zener diode voltage regulator} ઇનપુટ વોલ્ટેજ અથવા લોડ current માં ફેરફારો હોવા છતાં સતત આઉટપુટ વોલ્ટેજ જાળવે છે. તે reverse breakdown region માં કાર્ય કરે છે જ્યાં વોલ્ટેજ લગભગ સતત રહે છે.

\paragraph{બેસિક Zener Regulator સર્કિટ:}

\begin{figure}[H]
\centering
\begin{circuitikz}[scale=1.2]
    % Input voltage
    \draw (0,0) to[V, l=\(V_{in}\)] (0,3);
    
    % Series resistor
    \draw (0,3) to[R, l=\(R_S\)] (3,3);
    
    % Zener diode
    \draw (3,3) to[zzD, l=\(V_Z\)] (3,0);
    
    % Load resistor
    \draw (3,3) -- (5,3);
    \draw (5,3) to[R, l=\(R_L\)] (5,0);
    
    % Ground
    \draw (0,0) -- (5,0);
    
    % Output voltage
    \draw[<->] (5.5,3) -- (5.5,0) node[midway, right] {\(V_{out} = V_Z\)};
\end{circuitikz}
\caption{Zener Diode Voltage Regulator}
\end{figure}

\paragraph{કાર્યકારી સિદ્ધાંત:}

Series resistor \(R_S\) અને Zener diode voltage divider તરીકે કાર્ય કરે છે. જ્યારે \(V_{in} > V_Z\), Zener breakdown region માં કાર્ય કરે છે, \(V_{out} = V_Z\) જાળવે છે. Series resistor \(R_S\) વધારાનો વોલ્ટેજ drop કરે છે:

\[
V_{R_S} = V_{in} - V_Z
\]

\(R_S\) દ્વારા કરંટ:
\[
I_S = \frac{V_{in} - V_Z}{R_S} = I_Z + I_L
\]

જ્યાં \(I_Z\) Zener current છે અને \(I_L\) load current છે.

\paragraph{Line Regulation:}
જ્યારે \(V_{in}\) વધે છે, વધુ કરંટ \(R_S\) દ્વારા વહે છે. Zener સતત \(V_Z\) જાળવવા માટે વધારાનો current conduct કરે છે. આઉટપુટ સ્થિર રહે છે.

\paragraph{Load Regulation:}
જ્યારે load current \(I_L\) વધે છે, \(R_S\) દ્વારા કુલ current લગભગ સતત રાખવા માટે Zener current \(I_Z\) પ્રમાણસર ઘટે છે. આઉટપુટ વોલ્ટેજ \(V_Z\) પર રહે છે જ્યાં સુધી \(I_Z\) લઘુત્તમ holding current ઉપર રહે.

\paragraph{ડિઝાઇન વિચારણાઓ:}
\begin{itemize}
    \item Zener voltage \(V_Z\) ને desired output voltage ની બરાબર પસંદ કરો
    \item \(V_{in(min)} > V_Z + 2V\) (minimum overhead)
    \item \(R_S = \frac{V_{in} - V_Z}{I_Z + I_L}\)
    \item Zener power: \(P_Z = V_Z \times I_{Z(max)}\)
    \item યોગ્ય regulation માટે \(I_{Z(min)} < I_Z < I_{Z(max)}\) ખાતરી કરો
\end{itemize}

\paragraph{મર્યાદાઓ:}
મર્યાદિત current capability, નબળી efficiency, આઉટપુટ adjustable નથી, heat generate કરે છે, ripple rejection મર્યાદિત.

\paragraph{એપ્લિકેશન્સ:}
Low power voltage regulation, reference voltage sources, over-voltage protection, transistors માટે bias voltage, meter protection circuits.

\paragraph{મેમરી ટ્રીક:}
\emph{``REGULATOR: Reverse breakdown region, Rs દ્વારા વધારાનો voltage drop, સતત output generate કરે, stabilization માટે ઉપયોગ, લોડ variations handle કરે, આપોઆપ current adjustment, પાવર supplies માં સામાન્ય application, આઉટપુટ Vz ની બરાબર, low power માટે વિશ્વસનીય!''}

\section{પ્રશ્ન 6}

\subsection{પ્રશ્ન 6(a) [3 ગુણ]}
\textbf{ટ્રાન્ઝિસ્ટરની ટૂંકમાં ચર્ચા કરો.}

\subsubsection{ઉકેલ}

\textbf{ટ્રાન્ઝિસ્ટર} એ ત્રણ-ટર્મિનલ સક્રિય સેમિકન્ડક્ટર ઉપકરણ છે જે electronic signals ને amplify અથવા switch કરી શકે છે. તે આધુનિક electronic circuits નું મૂળભૂત building block છે.

\paragraph{પ્રકારો:}
\begin{description}
    \item[BJT (Bipolar Junction Transistor):] electrons અને holes બંનેનો ઉપયોગ કરે છે. બે પ્રકાર: NPN અને PNP. ત્રણ regions: Emitter, Base, Collector.
    \item[FET (Field Effect Transistor):] current control કરવા માટે electric field નો ઉપયોગ કરે છે. પ્રકારોમાં JFET અને MOSFET સામેલ છે. ત્રણ terminals: Source, Gate, Drain.
\end{description}

\paragraph{કાર્યો:}
\begin{description}
    \item[એમ્પ્લિફિકેશન:] નાનો input signal મોટા output signal ને control કરે છે. Power, voltage અથવા current amplification.
    \item[Switching:] Electronic switch તરીકે કાર્ય કરે છે - ON (saturation) અથવા OFF (cutoff). Digital circuits, power control માં વપરાય છે.
\end{description}

\paragraph{BJT ઓપરેટિંગ પ્રદેશો:}
Cutoff (બંને junctions reverse-biased, transistor OFF), Active (EB forward, CB reverse, amplification), Saturation (બંને junctions forward, transistor સંપૂર્ણ રીતે ON).

\paragraph{મેમરી ટ્રીક:}
\emph{``TRANSISTOR: ત્રણ terminals, Amplifies અથવા switches, બાયસિંગ જરૂરી, Silicon-based, દરેક જગ્યાએ Integrated, સેમિકન્ડક્ટર device, બે junction device, ઇનપુટ દ્વારા આઉટપુટ control, ઇલેક્ટ્રોનિક્સમાં ક્રાંતિ!''}

\subsection{પ્રશ્ન 6(b) [4 ગુણ]}
\textbf{ટ્રાન્ઝિસ્ટર એમ્પ્લિફાયર માટે \(\alpha\) અને \(\beta\) વચ્ચેનો સંબંધ મેળવો.}

\subsubsection{ઉકેલ}

ટ્રાન્ઝિસ્ટરમાં, \textbf{\(\alpha\)} (alpha) અને \textbf{\(\beta\)} (beta) current gain parameters છે જે emitter, base અને collector currents ને સંબંધિત કરે છે.

\paragraph{\(\alpha\) (Common Base Current Gain) ની વ્યાખ્યા:}
\[
\alpha = \frac{I_C}{I_E}
\]
Emitter current ના collector current નો રેશિયો. સામાન્ય રીતે \(\alpha \approx 0.95\) થી 0.99.

\paragraph{\(\beta\) (Common Emitter Current Gain) ની વ્યાખ્યા:}
\[
\beta = \frac{I_C}{I_B}
\]
Base current ના collector current નો રેશિયો. સામાન્ય રીતે \(\beta \approx 50\) થી 300.

\paragraph{વ્યુત્પત્તિ:}
ટ્રાન્ઝિસ્ટર node પર Kirchhoff's Current Law લાગુ કરવું:
\[
I_E = I_B + I_C
\]

\(\alpha\) ની વ્યાખ્યામાંથી:
\[
I_C = \alpha I_E = \alpha(I_B + I_C)
\]

Expanding:
\[
I_C = \alpha I_B + \alpha I_C
\]

Rearranging:
\[
I_C - \alpha I_C = \alpha I_B
\]

\[
I_C(1 - \alpha) = \alpha I_B
\]

\[
\frac{I_C}{I_B} = \frac{\alpha}{1 - \alpha}
\]

કારણ કે \(\beta = \frac{I_C}{I_B}\):

\[
\boxed{\beta = \frac{\alpha}{1 - \alpha}}
\]

તેવી જ રીતે, \(\alpha\) માટે solve કરવું:
\[
\beta(1 - \alpha) = \alpha
\]

\[
\beta - \beta\alpha = \alpha
\]

\[
\beta = \alpha + \beta\alpha = \alpha(1 + \beta)
\]

\[
\boxed{\alpha = \frac{\beta}{1 + \beta}}
\]

\paragraph{સંખ્યાત્મક ઉદાહરણ:}
જો \(\alpha = 0.98\):
\[\beta = \frac{0.98}{1-0.98} = \frac{0.98}{0.02} = 49\]

જો \(\beta = 100\):
\[\alpha = \frac{100}{1+100} = \frac{100}{101} = 0.99\]

\paragraph{મેમરી ટ્રીક:}
\emph{``one-minus-alpha પર Alpha Beta આપે છે; one-plus-beta પર Beta Alpha આપે છે!''}

\subsection{પ્રશ્ન 6(c) [7 ગુણ]}
\textbf{NPN અને PNP ટ્રાન્ઝિસ્ટરનું બાંધકામ વિગતવાર સમજાવો.}

\subsubsection{ઉકેલ}

\textbf{Bipolar Junction Transistors (BJTs)} ત્રણ-layer, બે-junction semiconductor devices છે જે બે રૂપરેખાંકનોમાં ઉપલબ્ધ છે: \textbf{NPN} અને \textbf{PNP}.

\paragraph{NPN Transistor બાંધકામ:}

\begin{figure}[H]
\centering
\begin{tikzpicture}[scale=1.3]
    % Layers
    \draw[fill=blue!30] (0,0) rectangle (1,3);
    \node at (0.5,1.5) [rotate=90] {N-type};
    \node[below] at (0.5,0) {Emitter (E)};
    
    \draw[fill=red!30] (1,0.8) rectangle (1.5,2.2);
    \node at (1.25,1.5) [rotate=90, font=\small] {P};
    \node[above] at (1.25,2.5) {Base (B)};
    
    \draw[fill=blue!30] (1.5,0) rectangle (2.5,3);
    \node at (2,1.5) [rotate=90] {N-type};
    \node[below] at (2,0) {Collector (C)};
    
    % Junctions
    \draw[thick, red] (1,0.8) -- (1,2.2);
    \node[left, font=\small] at (1,2.5) {EB Junction};
    \draw[thick, red] (1.5,0.8) -- (1.5,2.2);
    \node[right, font=\small] at (1.5,2.5) {CB Junction};
    
    % Terminals
    \draw[thick] (0.5,3) -- (0.5,3.5) node[above] {E};
    \draw[thick] (1.25,2.2) -- (1.25,3.5) node[above] {B};
    \draw[thick] (2,3) -- (2,3.5) node[above] {C};
\end{tikzpicture}
\caption{NPN Transistor સ્ટ્રક્ચર}
\end{figure}

\subparagraph{NPN સ્ટ્રક્ચર:}
\begin{description}
    \item[Emitter (N-type):] મુખ્ય ચાર્જ વાહકો (electrons) emit કરતો ભારે doped પ્રદેશ. મધ્યમ size, ઊંચી conductivity.
    \item[Base (P-type):] Emitter અને collector વચ્ચે ખૂબ પાતળો (\(\sim 1\ \mu m\)) અને હળવા doped પ્રદેશ. ટ્રાન્ઝિસ્ટર action માટે નિર્ણાયક.
    \item[Collector (N-type):] મધ્યમ રીતે doped, સૌથી મોટો પ્રદેશ. Base દ્વારા emitter માંથી carriers collect કરે છે.
\end{description}

\paragraph{PNP Transistor બાંધકામ:}

PNP transistor માં વિરુદ્ધ doping છે: P-type emitter, N-type base, P-type collector. સ્ટ્રક્ચર NPN નું mirror image છે.

\paragraph{મુખ્ય બાંધકામ લાક્ષણિકતાઓ:}

\subparagraph{Base પ્રદેશ:}
Recombination વગર મોટાભાગના carriers diffuse થવા દેવા માટે અત્યંત પાતળી. સામાન્ય જાડાઈ 1-10\,\(\mu m\). ઓછી recombination ખાતરી કરવા માટે હળવી doping.

\subparagraph{Emitter પ્રદેશ:}
મહત્તમ carriers inject કરવા માટે ભારે doped. Doping concentration \(\approx 10^{19}\) atoms/cm\(^3\).

\subparagraph{Collector પ્રદેશ:}
મધ્યમ doping, heat dissipate કરવા માટે emitter કરતાં મોટો area. Doping concentration \(\approx 10^{15}\) atoms/cm\(^3\).

\paragraph{Manufacturing Process:}
Silicon wafer પર diffusion, ion implantation, epitaxial growth જેવી techniques નો ઉપયોગ કરે છે. આધુનિક transistors photolithography નો ઉપયોગ કરીને integrated circuits ના ભાગ તરીકે fabricated કરવામાં આવે છે.

\paragraph{તફાવત NPN બનામ PNP:}
\begin{description}
    \item[NPN:] Electrons મુખ્ય વાહકો છે. કરંટ collector થી emitter વહે છે. ઊંચી electron mobility ને કારણે ઝડપી switching.
    \item[PNP:] Holes મુખ્ય વાહકો છે. કરંટ emitter થી collector વહે છે. સામાન્ય રીતે NPN કરતાં ધીમું.
\end{description}

\paragraph{પ્રતીક સંમેલન:}
Emitter પરનું arrow conventional current direction દર્શાવે છે. NPN: arrow બહાર તરફ points (N થી P). PNP: arrow અંદર તરફ points (P થી N).

\paragraph{મેમરી ટ્રીક:}
\emph{``NPN: Not Pointing iN; PNP: Points iN Purposely!''}

\end{document}
