\documentclass[10pt,a4paper]{article}

% content/resources/templates/preamble.tex
\usepackage[margin=0.6in]{geometry}
\author{Milav Dabgar}
\usepackage{amsmath,amssymb,amsthm}
\usepackage{booktabs}
\usepackage{multirow}
\usepackage{xcolor}
\usepackage{tcolorbox}
\tcbuselibrary{breakable,skins}
\usepackage[colorlinks=true,linkcolor=blue]{hyperref}
\usepackage{titlesec}
\usepackage{enumitem}
\usepackage{tikz}
\usepackage{pgfplots}
\usepackage{circuitikz}
\usepackage[version=4]{mhchem}
\usepackage{longtable}
\usepackage{array}
\usepackage{float}
\usepackage{caption}
\usepackage{listings}

\lstset{
  basicstyle=\small\ttfamily,
  breaklines=true,
  breakatwhitespace=false,
  postbreak=\mbox{\textcolor{red}{$\hookrightarrow$}\space},
  float=false,
  numbers=left,
  numberstyle=\tiny\color{gray},
  numbersep=10pt,
  xleftmargin=2em,
  keywordstyle=\color{blue},
  commentstyle=\color{green!60!black},
  stringstyle=\color{purple},
  backgroundcolor=\color{gray!5},
  showstringspaces=false,
  tabsize=2,
  captionpos=b,
  keepspaces=true,
  columns=flexible
}

\pgfplotsset{compat=1.18}
\usetikzlibrary{shapes,arrows,positioning,calc,patterns,decorations.pathmorphing,decorations.markings,arrows.meta}

% Color scheme
\definecolor{headcolor}{RGB}{0,102,204}
\definecolor{keycolor}{RGB}{220,20,60}
\definecolor{solutioncolor}{RGB}{34,139,34}
\definecolor{mnemoniccolor}{RGB}{148,0,211}
\definecolor{codecolor}{RGB}{0,0,100}

% Spacing
\setlength{\parskip}{3pt}
\setlist[itemize]{nosep}
\setlist[enumerate]{nosep}

% Title formatting
\titleformat{\section}{\Large\bfseries\color{headcolor}}{\thesection}{1em}{}
\titleformat{\subsection}{\large\bfseries\color{headcolor}}{\thesubsection}{1em}{}

% Pandoc tightlist compatibility
\providecommand{\tightlist}{%
  \setlength{\itemsep}{0pt}\setlength{\parskip}{0pt}}

% Pandoc longtable compatibility
\newcounter{none}
\def\thenone{}


% content/resources/templates/english-boxes.tex
% This file is currently empty - it exists to maintain consistency with the import structure.
% Add custom environments here if needed in the future.


\begin{document}

\begin{center}
{\Huge\bfseries\color{headcolor} Subject Name Solutions}\\[5pt]
{\LARGE 4311102 -- Winter 2023}\\[3pt]
{\large Semester 1 Study Material}\\[3pt]
{\normalsize\textit{Detailed Solutions and Explanations}}
\end{center}

\vspace{10pt}

\subsection*{Question 1(a) [3 marks]}\label{q1a}

\textbf{Define Forward and reverse bias of diode.}

\begin{solutionbox}

\textbf{Forward Bias of Diode}:

\begin{itemize}
\tightlist
\item
  \textbf{Connection Method}: P-type connected to positive terminal and
  N-type connected to negative terminal of battery
\item
  \textbf{Barrier Width}: Barrier width decreases
\item
  \textbf{Resistance}: Low resistance (typically 100-1000Ω)
\item
  \textbf{Current Flow}: Allows current to flow easily through the diode
\end{itemize}

\textbf{Reverse Bias of Diode}:

\begin{itemize}
\tightlist
\item
  \textbf{Connection Method}: P-type connected to negative terminal and
  N-type connected to positive terminal
\item
  \textbf{Barrier Width}: Barrier width increases
\item
  \textbf{Resistance}: Very high resistance (typically several MΩ)
\item
  \textbf{Current Flow}: Blocks current flow (only small leakage current
  flows)
\end{itemize}

\textbf{Diagram}:

\begin{center}
\textbf{Mermaid Diagram (Code)}
\begin{verbatim}
{Shaded}
{Highlighting}[]
graph TD
    A[Forward Bias] {-{-}{} B[P connected to +ve{}br /{}N connected to {-}ve]}
    A {-{-}{} C[Current flows easily]}
    D[Reverse Bias] {-{-}{} E[P connected to {-}ve{}br /{}N connected to +ve]}
    D {-{-}{} F[Current blocked]}
{Highlighting}
{Shaded}
\end{verbatim}
\end{center}

\end{solutionbox}
\begin{mnemonicbox}
``PFNR'' - ``Positive to P Forward, Negative to P
Reverse''

\end{mnemonicbox}
\subsection*{Question 1(b) [4 marks]}\label{q1b}

\textbf{Explain construction and working of LDR.}

\begin{solutionbox}

\textbf{Construction of LDR}:

\begin{itemize}
\tightlist
\item
  \textbf{Material}: Made of semiconductor material (Cadmium Sulfide)
\item
  \textbf{Pattern}: Zigzag pattern of photosensitive material on ceramic
  base
\item
  \textbf{Electrodes}: Metal electrodes at both ends
\item
  \textbf{Package}: Encapsulated in transparent plastic or glass case
\end{itemize}

\textbf{Working Principle}:

\begin{itemize}
\tightlist
\item
  \textbf{Photoconductivity}: Based on photoconductivity principle
\item
  \textbf{Dark Resistance}: High resistance (MΩ range) in dark
  conditions
\item
  \textbf{Light Exposure}: When exposed to light, photons release
  electrons
\item
  \textbf{Resistance Drop}: Resistance decreases (kΩ range) in bright
  light
\end{itemize}

\textbf{Diagram}:

\begin{verbatim}
 +{-{-}{-}{-}{-}{-}+}
 |      |    Zigzag pattern of
 | +{-/{-}+ {-} semiconductor material}
 | |    |
 | +{-/{-}+}
 |      |
 +{-{-}{-}{-}{-}{-}+}
  |    |
  |    |
  L    D {{-} Leads}
\end{verbatim}

\end{solutionbox}
\begin{mnemonicbox}
``MILD'' - ``More Illumination, Less
Dark-resistance''

\end{mnemonicbox}
\subsection*{Question 1(c) [7 marks]}\label{q1c}

\textbf{Explain the color band coding method of Resistor. Write color
band of 47kΩ \pm5\% resistance.}

\begin{solutionbox}

\textbf{Color Band Coding Method}:

{\def\LTcaptype{none} % do not increment counter
\begin{longtable}[]{@{}llll@{}}
\toprule\noalign{}
Color & Value & Multiplier & Tolerance \\
\midrule\noalign{}
\endhead
\bottomrule\noalign{}
\endlastfoot
Black & 0 & 10^{0} & - \\
Brown & 1 & 10^{1} & \pm1\% \\
Red & 2 & 10^{2} & \pm2\% \\
Orange & 3 & 10^{3} & - \\
Yellow & 4 & 10^{4} & - \\
Green & 5 & 10^{5} & \pm0.5\% \\
Blue & 6 & 10^{6} & \pm0.25\% \\
Violet & 7 & 10^{7} & \pm0.1\% \\
Grey & 8 & 10^{8} & \pm0.05\% \\
White & 9 & 10^{9} & - \\
Gold & - & 10^{-}^{1} & \pm5\% \\
Silver & - & 10^{-}^{2} & \pm10\% \\
Colorless & - & - & \pm20\% \\
\end{longtable}
}

\textbf{4-Band Resistor Color Code}:

\begin{itemize}
\tightlist
\item
  \textbf{First Band}: First significant digit
\item
  \textbf{Second Band}: Second significant digit
\item
  \textbf{Third Band}: Multiplier
\item
  \textbf{Fourth Band}: Tolerance
\end{itemize}

\textbf{For 47kΩ \pm5\%}:

\begin{itemize}
\tightlist
\item
  First digit: 4 = Yellow
\item
  Second digit: 7 = Violet
\item
  Multiplier: 10^{3} = Orange (for kΩ)
\item
  Tolerance: \pm5\% = Gold
\end{itemize}

\textbf{Color bands for 47kΩ \pm5\%}: Yellow-Violet-Orange-Gold

\textbf{Diagram}:

\begin{verbatim}
+{-{-}{-}+{-}{-}{-}+{-}{-}{-}+{-}{-}{-}+{-}{-}{-}{-}{-}{-}{-}{-}{-}{-}{-}{-}{-}+}
|   |   |   |   |             |
| Y | V | O | G |             |
|   |   |   |   |             |
+{-{-}{-}+{-}{-}{-}+{-}{-}{-}+{-}{-}{-}+{-}{-}{-}{-}{-}{-}{-}{-}{-}{-}{-}{-}{-}+}
  |   |   |   |
  |   |   |   +{-{-} Gold (5\%)}
  |   |   +{-{-}{-}{-}{-}{-} Orange (10^{3})}
  |   +{-{-}{-}{-}{-}{-}{-}{-}{-}{-} Violet (7)}
  +{-{-}{-}{-}{-}{-}{-}{-}{-}{-}{-}{-}{-}{-} Yellow (4)}
\end{verbatim}

\end{solutionbox}
\begin{mnemonicbox}
``BAND'' - ``Beginning digits, Amplify with
Multiplier, Note tolerance with last band, Decode carefully''

\end{mnemonicbox}
\subsection*{Question 1(c) [7 marks]
(OR)}\label{q1c}

\textbf{Explain Aluminum Electrolytic wet type capacitor.}

\begin{solutionbox}

\textbf{Aluminum Electrolytic Wet Type Capacitor}:

\textbf{Construction}:

\begin{itemize}
\tightlist
\item
  \textbf{Plates}: Two aluminum foils (anode and cathode)
\item
  \textbf{Dielectric}: Aluminum oxide layer on anode foil
\item
  \textbf{Electrolyte}: Liquid electrolyte (boric acid, sodium borate,
  etc.)
\item
  \textbf{Separator}: Paper separator soaked in electrolyte
\item
  \textbf{Enclosure}: Aluminum can with rubber seal
\end{itemize}

\textbf{Working Principle}:

\begin{itemize}
\tightlist
\item
  \textbf{Oxide Layer}: Thin aluminum oxide layer acts as dielectric
\item
  \textbf{Electrolyte}: Acts as cathode connection to second plate
\item
  \textbf{Polarization}: Has defined polarity (+ and -) terminals
\end{itemize}

\textbf{Characteristics}:

\begin{itemize}
\tightlist
\item
  \textbf{Capacitance Range}: 1μF to 47,000μF
\item
  \textbf{Voltage Rating}: 6.3V to 450V
\item
  \textbf{Polarity}: Polarized (must connect correctly)
\item
  \textbf{Leakage Current}: Higher than other capacitor types
\item
  \textbf{ESR}: Higher equivalent series resistance
\end{itemize}

\textbf{Diagram}:

\begin{center}
\textbf{Mermaid Diagram (Code)}
\begin{verbatim}
{Shaded}
{Highlighting}[]
graph TD
    A[Aluminum Electrolytic Capacitor] {-{-}{} B[Aluminum Can]}
    A {-{-}{} C[Anode Foil]}
    A {-{-}{} D[Cathode Foil]}
    A {-{-}{} E[Electrolyte]}
    A {-{-}{} F[Separator]}
    A {-{-}{} G[Aluminum Oxide Layer]}
    A {-{-}{} H[Terminal Posts]}
{Highlighting}
{Shaded}
\end{verbatim}
\end{center}

\end{solutionbox}
\begin{mnemonicbox}
``POLE'' - ``Polarized, Oxide layer, Liquid
electrolyte, Enormous capacitance''

\end{mnemonicbox}
\subsection*{Question 2(a) [3 marks]}\label{q2a}

\textbf{Draw the symbol of Schottkey diode, LED and Photo-diode.}

\begin{solutionbox}

\textbf{Symbols}:

\begin{verbatim}
Schottky Diode      LED                 Photo{-diode}
   +{-{-}{-}{-}|{-}{-}{-}+      +{-}{-}{-}||{-}{-}{-}+         +{-}{-}{-}||{-}{-}{-}+}
   |         |      |    |    |         |    ↓    |
   |         |      |   / {   |         |   /    |}
   +{-{-}{-}{-}{-}{-}{-}{-}{-}+      |  /     |         |     /  |}
                    | /     { |         |    /   |}
                    +{-{-}{-}{-}{-}{-}{-}{-}{-}+         +{-}{-}{-}{-}{-}{-}{-}{-}{-}+}
\end{verbatim}

\textbf{Key Features}:

\begin{itemize}
\tightlist
\item
  \textbf{Schottky Diode}: Standard diode symbol with curved bar
  (represents metal-semiconductor junction)
\item
  \textbf{LED}: Standard diode symbol with two arrows pointing away
  (represents light emission)
\item
  \textbf{Photo-diode}: Standard diode symbol with two arrows pointing
  toward diode (represents light detection)
\end{itemize}

\end{solutionbox}
\begin{mnemonicbox}
``SLP'' - ``Schottky has curve, LED emits,
Photo-diode absorbs''

\end{mnemonicbox}
\subsection*{Question 2(b) [4 marks]}\label{q2b}

\textbf{Define Active and Passive Components with example.}

\begin{solutionbox}

\textbf{Passive Components}:

{\def\LTcaptype{none} % do not increment counter
\begin{longtable}[]{@{}lll@{}}
\toprule\noalign{}
Characteristic & Description & Examples \\
\midrule\noalign{}
\endhead
\bottomrule\noalign{}
\endlastfoot
Power & Cannot generate power & Resistors, Capacitors, Inductors \\
Signal & Cannot amplify signals & Transformers, Diodes \\
Control & No control over current flow & Connectors, Switches \\
Energy & Store or dissipate energy & Fuses, Filters \\
\end{longtable}
}

\textbf{Active Components}:

{\def\LTcaptype{none} % do not increment counter
\begin{longtable}[]{@{}
  >{\raggedright\arraybackslash}p{(\linewidth - 4\tabcolsep) * \real{0.4211}}
  >{\raggedright\arraybackslash}p{(\linewidth - 4\tabcolsep) * \real{0.3158}}
  >{\raggedright\arraybackslash}p{(\linewidth - 4\tabcolsep) * \real{0.2632}}@{}}
\toprule\noalign{}
\begin{minipage}[b]{\linewidth}\raggedright
Characteristic
\end{minipage} & \begin{minipage}[b]{\linewidth}\raggedright
Description
\end{minipage} & \begin{minipage}[b]{\linewidth}\raggedright
Examples
\end{minipage} \\
\midrule\noalign{}
\endhead
\bottomrule\noalign{}
\endlastfoot
Power & Can generate power & Transistors, ICs \\
Signal & Can amplify signals & Op-amps, Amplifiers \\
Control & Control current flow & SCRs, MOSFETs \\
Dependency & Require external power & Voltage regulators,
Microcontrollers \\
\end{longtable}
}

\textbf{Diagram}:

\begin{verbatim}
graph TB
    A[Electronic Components] {-{-} B[Active Components]}
    A {-{-} C[Passive Components]}
    B {-{-} D[Transistors]}
    B {-{-} E[ICs]}
    B {-{-} F[Amplifiers]}
    C {-{-} G[Resistors]}
    C {-{-} H[Capacitors]}
    C {-{-} I[Inductors]}
\end{verbatim}

\end{solutionbox}
\begin{mnemonicbox}
``PASS-ACT'' - ``Passive stores or dissipates, Active
controls or amplifies''

\end{mnemonicbox}
\subsection*{Question 2(c) [7 marks]}\label{q2c}

\textbf{Explain working of full wave bridge rectifier.}

\begin{solutionbox}

\textbf{Full Wave Bridge Rectifier}:

\textbf{Circuit Construction}:

\begin{itemize}
\tightlist
\item
  \textbf{Diodes}: Four diodes arranged in bridge configuration
\item
  \textbf{Input}: AC supply from transformer secondary
\item
  \textbf{Output}: Pulsating DC across load resistor with filter
  capacitor
\end{itemize}

\textbf{Working Principle}:

\begin{itemize}
\tightlist
\item
  \textbf{Positive Half Cycle}: D1 and D3 conduct, D2 and D4 block
\item
  \textbf{Negative Half Cycle}: D2 and D4 conduct, D1 and D3 block
\item
  \textbf{Current Flow}: Always flows through load in same direction
\end{itemize}

\textbf{Performance Parameters}:

\begin{itemize}
\tightlist
\item
  \textbf{Ripple Frequency}: 2\times input frequency (100 Hz for 50 Hz input)
\item
  \textbf{Efficiency}: 81.2\%
\item
  \textbf{PIV}: V_{0}(max) per diode
\item
  \textbf{TUF}: 0.812 (Transformer Utilization Factor)
\end{itemize}

\textbf{Diagram}:

\begin{center}
\textbf{Mermaid Diagram (Code)}
\begin{verbatim}
{Shaded}
{Highlighting}[]
graph LR
    A[AC Input] {-{-}{} B[Bridge Rectifier]}
    B {-{-}{} C[D1]}
    B {-{-}{} D[D2]}
    B {-{-}{} E[D3]}
    B {-{-}{} F[D4]}
    C {-{-}{} G[Load]}
    D {-{-}{} G}
    E {-{-}{} G}
    F {-{-}{} G}
    G {-{-}{} H[Pulsating DC Output]}
    H {-{-}{} I[Filter Capacitor]}
    I {-{-}{} J[Smooth DC Output]}
{Highlighting}
{Shaded}
\end{verbatim}
\end{center}

\end{solutionbox}
\begin{mnemonicbox}
``BRIDGE'' - ``Better Rectification with Improved
Diode Geometry Efficiency''

\end{mnemonicbox}
\subsection*{Question 2(a) [3 marks]
(OR)}\label{q2a}

\textbf{Explain construction and working of LED.}

\begin{solutionbox}

\textbf{Construction of LED}:

\begin{itemize}
\tightlist
\item
  \textbf{Material}: Semiconductor (GaAs, GaP, AlGaInP, etc.)
\item
  \textbf{Junction}: P-N junction with heavily doped semiconductors
\item
  \textbf{Package}: Encased in transparent or colored epoxy lens
\item
  \textbf{Cathode}: Identified by flat side on package or shorter lead
\end{itemize}

\textbf{Working Principle}:

\begin{itemize}
\tightlist
\item
  \textbf{Forward Bias}: Applied to P-N junction
\item
  \textbf{Recombination}: Electrons and holes recombine at junction
\item
  \textbf{Energy Release}: Energy released as photons (light)
\item
  \textbf{Wavelength}: Determined by band gap of semiconductor material
\end{itemize}

\textbf{Diagram}:

\begin{verbatim}
        +{-{-}{-}{-}{-}{-}{-}+}
        |       |
        |   \^{   |}
        |  / {  | {-} Epoxy lens}
        | /   { |}
        |/     {|}
    {-{-}{-}{-}+{-}{-}{-}{-}{-}{-}{-}+{-}{-}{-}{-}}
    |       |       |
    |       |       |
    |       |       |
  Anode   Chip   Cathode
\end{verbatim}

\end{solutionbox}
\begin{mnemonicbox}
``LEDS'' - ``Light Emits During electron-hole
recombination in Semiconductor''

\end{mnemonicbox}
\subsection*{Question 2(b) [4 marks]
(OR)}\label{q2b}

\textbf{Explain composition type resistors.}

\begin{solutionbox}

\textbf{Composition Resistors}:

\textbf{Construction}:

\begin{itemize}
\tightlist
\item
  \textbf{Core Material}: Carbon particles mixed with insulating
  material (clay/ceramic)
\item
  \textbf{Binding}: Resin binder forms solid cylindrical shape
\item
  \textbf{Terminals}: Metal caps with leads attached to ends
\item
  \textbf{Protection}: Coated with insulating paint or plastic
\end{itemize}

\textbf{Characteristics}:

\begin{itemize}
\tightlist
\item
  \textbf{Resistance Range}: 1Ω to 22MΩ
\item
  \textbf{Power Rating}: 1/8W to 2W
\item
  \textbf{Tolerance}: \pm5\% to \pm20\%
\item
  \textbf{Temperature Coefficient}: -500 to +500 ppm/^\circC
\end{itemize}

\textbf{Advantages \& Limitations}:

\begin{itemize}
\tightlist
\item
  \textbf{Cost}: Low cost
\item
  \textbf{Noise}: Higher noise level
\item
  \textbf{Stability}: Less stable with temperature
\item
  \textbf{Applications}: General purpose, non-critical applications
\end{itemize}

\textbf{Diagram}:

\begin{verbatim}
    +{-{-}{-}{-}{-}{-}{-}{-}{-}{-}{-}{-}{-}{-}{-}{-}{-}{-}{-}{-}{-}+}
    |                     |
    |  +{-{-}{-}{-}{-}{-}{-}{-}{-}{-}{-}{-}{-}{-}{-}+  |}
    |  | Carbon        |  | {{-} Insulating}
    |  | Composition   |  |    coating
    |  +{-{-}{-}{-}{-}{-}{-}{-}{-}{-}{-}{-}{-}{-}{-}+  |}
    |                     |
    +{-{-}{-}{-}{-}{-}{-}{-}{-}{-}{-}{-}{-}{-}{-}{-}{-}{-}{-}{-}{-}+}
    |         |
    |         |
Lead         Lead
\end{verbatim}

\end{solutionbox}
\begin{mnemonicbox}
``CCRI'' - ``Carbon Composition Resistors are
Inexpensive''

\end{mnemonicbox}
\subsection*{Question 2(c) [7 marks]
(OR)}\label{q2c}

\textbf{Explain working of full wave rectifier with two diodes.}

\begin{solutionbox}

\textbf{Full Wave Rectifier with Two Diodes (Center-tap)}:

\textbf{Circuit Construction}:

\begin{itemize}
\tightlist
\item
  \textbf{Transformer}: Center-tapped transformer secondary
\item
  \textbf{Diodes}: Two diodes connected to opposite ends of secondary
\item
  \textbf{Output}: Taken between center tap and diode junction
\end{itemize}

\textbf{Working Principle}:

\begin{itemize}
\tightlist
\item
  \textbf{Positive Half Cycle}: Upper half of secondary positive, D1
  conducts, D2 blocks
\item
  \textbf{Negative Half Cycle}: Lower half of secondary positive, D2
  conducts, D1 blocks
\item
  \textbf{Current Flow}: Always flows through load in same direction
\end{itemize}

\textbf{Performance Parameters}:

\begin{itemize}
\tightlist
\item
  \textbf{Ripple Frequency}: 2\times input frequency (100 Hz for 50 Hz input)
\item
  \textbf{Efficiency}: 81.2\%
\item
  \textbf{PIV}: 2V_{0}(max) per diode (twice the center-tap rectifier)
\item
  \textbf{TUF}: 0.693 (Transformer Utilization Factor)
\end{itemize}

\textbf{Diagram}:

\begin{center}
\textbf{Mermaid Diagram (Code)}
\begin{verbatim}
{Shaded}
{Highlighting}[]
graph LR
    A[AC Input] {-{-}{} B[Center{-}Tapped Transformer]}
    B {-{-}{}|Upper Half| C[D1]}
    B {-{-}{}|Lower Half| D[D2]}
    B {-{-}{}|Center Tap| E[Ground]}
    C {-{-}{} F[Load]}
    D {-{-}{} F}
    F {-{-}{} E}
    F {-{-}{} G[Pulsating DC Output]}
    G {-{-}{} H[Filter]}
    H {-{-}{} I[Smooth DC Output]}
{Highlighting}
{Shaded}
\end{verbatim}
\end{center}

\end{solutionbox}
\begin{mnemonicbox}
``CTFWR'' - ``Center Tap Facilitates Whole-cycle
Rectification''

\end{mnemonicbox}
\subsection*{Question 3(a) [3 marks]}\label{q3a}

\textbf{Explain working of schhotkey diode.}

\begin{solutionbox}

\textbf{Working of Schottky Diode}:

\begin{itemize}
\tightlist
\item
  \textbf{Junction Type}: Metal-Semiconductor (M-S) junction instead of
  P-N
\item
  \textbf{Charge Carriers}: Majority carrier device (electrons in
  N-type)
\item
  \textbf{Barrier}: Schottky barrier formed at metal-semiconductor
  interface
\item
  \textbf{Forward Voltage}: Lower forward voltage drop (0.2-0.4V vs 0.7V
  for Si diode)
\end{itemize}

\textbf{Key Characteristics}:

\begin{itemize}
\tightlist
\item
  \textbf{Switching Speed}: Very fast switching (no minority carrier
  storage)
\item
  \textbf{Applications}: High-frequency circuits, power supplies
\item
  \textbf{Recovery Time}: Negligible reverse recovery time
\end{itemize}

\textbf{Diagram}:

\begin{verbatim}
Metal    |    N{-type}
         |
      +{-{-}+{-}{-}+}
      |     |
      | M{-S |  {-} Schottky Barrier}
      |     |
      +{-{-}{-}{-}{-}+}
\end{verbatim}

\end{solutionbox}
\begin{mnemonicbox}
``SFAM'' - ``Schottky's Fast And Metal-based''

\end{mnemonicbox}
\subsection*{Question 3(b) [4 marks]}\label{q3b}

\textbf{Explain N type semiconductor.}

\begin{solutionbox}

\textbf{N-type Semiconductor}:

\textbf{Formation}:

\begin{itemize}
\tightlist
\item
  \textbf{Base Material}: Intrinsic semiconductor (Silicon or Germanium)
\item
  \textbf{Doping Element}: Pentavalent impurity (P, As, Sb)
\item
  \textbf{Doping Process}: Thermal diffusion or ion implantation
\item
  \textbf{Concentration}: Typically 1 part impurity to 10^{8} parts silicon
\end{itemize}

\textbf{Characteristics}:

\begin{itemize}
\tightlist
\item
  \textbf{Majority Carriers}: Electrons (negative charge carriers)
\item
  \textbf{Minority Carriers}: Holes
\item
  \textbf{Conductivity}: Higher than intrinsic semiconductor
\item
  \textbf{Fermi Level}: Closer to conduction band
\end{itemize}

\textbf{Diagram}:

\begin{center}
\textbf{Mermaid Diagram (Code)}
\begin{verbatim}
{Shaded}
{Highlighting}[]
graph TD
    A[N{-type Semiconductor] {-}{-}{} B[Silicon Atom]}
    A {-{-}{} C[Pentavalent Impurity Atom]}
    C {-{-}{} D[Extra Free Electron]}
    A {-{-}{} E[Majority Carriers: Electrons]}
    A {-{-}{} F[Minority Carriers: Holes]}
{Highlighting}
{Shaded}
\end{verbatim}
\end{center}

\end{solutionbox}
\begin{mnemonicbox}
``PENT'' - ``Pentavalent Element makes N-Type with
free electrons''

\end{mnemonicbox}
\subsection*{Question 3(c) [7 marks]}\label{q3c}

\textbf{Explain construction and working of PN Junction Diode.}

\begin{solutionbox}

\textbf{Construction of PN Junction Diode}:

\begin{itemize}
\tightlist
\item
  \textbf{Materials}: P-type and N-type semiconductor regions
\item
  \textbf{Junction}: Formed by diffusion or epitaxial growth
\item
  \textbf{Depletion Region}: Forms at junction interface
\item
  \textbf{Contacts}: Metal contacts attached to both regions
\item
  \textbf{Packaging}: Sealed in glass, plastic, or metal case
\end{itemize}

\textbf{Working Principle}:

\begin{itemize}
\tightlist
\item
  \textbf{Depletion Region}: Forms due to diffusion of carriers
\item
  \textbf{Barrier Potential}: Created across junction (0.7V for Si, 0.3V
  for Ge)
\item
  \textbf{Forward Bias}: Current flows when forward voltage
  \textgreater{} barrier potential
\item
  \textbf{Reverse Bias}: Only small leakage current flows until
  breakdown
\end{itemize}

\textbf{Diagram}:

\begin{verbatim}
    +{-{-}{-}{-}{-}{-}{-}+{-}{-}{-}{-}{-}{-}{-}+}
    |       |       |
    |   P   |   N   |
    |       |       |
    +{-{-}{-}{-}{-}{-}{-}+{-}{-}{-}{-}{-}{-}{-}+}
        |       |
      Anode  Cathode

    Depletion region at junction
\end{verbatim}

\end{solutionbox}
\begin{mnemonicbox}
``BIRD'' - ``Barrier forms at Interface, Rectifies
Direct current''

\end{mnemonicbox}
\subsection*{Question 3(a) [3 marks]
(OR)}\label{q3a}

\textbf{Explain working of photo-diode.}

\begin{solutionbox}

\textbf{Working of Photo-diode}:

\begin{itemize}
\tightlist
\item
  \textbf{Operation Mode}: Reverse biased P-N junction
\item
  \textbf{Light Absorption}: Photons create electron-hole pairs in
  depletion region
\item
  \textbf{Carrier Generation}: Light energy \textgreater{} band gap
  energy creates free carriers
\item
  \textbf{Current Flow}: Photocurrent proportional to light intensity
\end{itemize}

\textbf{Key Characteristics}:

\begin{itemize}
\tightlist
\item
  \textbf{Sensitivity}: Depends on semiconductor material and wavelength
\item
  \textbf{Response Time}: Very fast (ns range)
\item
  \textbf{Operating Modes}: Photovoltaic mode or photoconductive mode
\item
  \textbf{Applications}: Light sensors, optical communication
\end{itemize}

\textbf{Diagram}:

\begin{verbatim}
       Light
         ↓
    +{-{-}{-}{-}+{-}{-}{-}{-}+}
    |         |
 {-{-}{-}+         +{-}{-}{-}}
    |    PN   |
    | Junction|
    |         |
 {-{-}{-}+         +{-}{-}{-}}
    |         |
    +{-{-}{-}{-}{-}{-}{-}{-}{-}+}
\end{verbatim}

\end{solutionbox}
\begin{mnemonicbox}
``PLIP'' - ``Photons Lead to Increased Photocurrent''

\end{mnemonicbox}
\subsection*{Question 3(b) [4 marks]
(OR)}\label{q3b}

\textbf{Explain P type Semiconductor.}

\begin{solutionbox}

\textbf{P-type Semiconductor}:

\textbf{Formation}:

\begin{itemize}
\tightlist
\item
  \textbf{Base Material}: Intrinsic semiconductor (Silicon or Germanium)
\item
  \textbf{Doping Element}: Trivalent impurity (B, Al, Ga)
\item
  \textbf{Doping Process}: Thermal diffusion or ion implantation
\item
  \textbf{Concentration}: Typically 1 part impurity to 10^{8} parts silicon
\end{itemize}

\textbf{Characteristics}:

\begin{itemize}
\tightlist
\item
  \textbf{Majority Carriers}: Holes (positive charge carriers)
\item
  \textbf{Minority Carriers}: Electrons
\item
  \textbf{Conductivity}: Higher than intrinsic semiconductor
\item
  \textbf{Fermi Level}: Closer to valence band
\end{itemize}

\textbf{Diagram}:

\begin{center}
\textbf{Mermaid Diagram (Code)}
\begin{verbatim}
{Shaded}
{Highlighting}[]
graph TD
    A[P{-type Semiconductor] {-}{-}{} B[Silicon Atom]}
    A {-{-}{} C[Trivalent Impurity Atom]}
    C {-{-}{} D[Hole Formation]}
    A {-{-}{} E[Majority Carriers: Holes]}
    A {-{-}{} F[Minority Carriers: Electrons]}
{Highlighting}
{Shaded}
\end{verbatim}
\end{center}

\end{solutionbox}
\begin{mnemonicbox}
``TRIP'' - ``TRIvalent impurity Produces holes in
P-type''

\end{mnemonicbox}
\subsection*{Question 3(c) [7 marks]
(OR)}\label{q3c}

\textbf{Compare half wave and full wave rectifier.}

\begin{solutionbox}

\textbf{Comparison between Half Wave and Full Wave Rectifier}:

{\def\LTcaptype{none} % do not increment counter
\begin{longtable}[]{@{}
  >{\raggedright\arraybackslash}p{(\linewidth - 4\tabcolsep) * \real{0.2075}}
  >{\raggedright\arraybackslash}p{(\linewidth - 4\tabcolsep) * \real{0.3962}}
  >{\raggedright\arraybackslash}p{(\linewidth - 4\tabcolsep) * \real{0.3962}}@{}}
\toprule\noalign{}
\begin{minipage}[b]{\linewidth}\raggedright
Parameter
\end{minipage} & \begin{minipage}[b]{\linewidth}\raggedright
Half Wave Rectifier
\end{minipage} & \begin{minipage}[b]{\linewidth}\raggedright
Full Wave Rectifier
\end{minipage} \\
\midrule\noalign{}
\endhead
\bottomrule\noalign{}
\endlastfoot
\textbf{Circuit Complexity} & Simple, uses 1 diode & Complex, uses 2 or
4 diodes \\
\textbf{Output Waveform} & Pulsating DC for half cycle & Pulsating DC
for full cycle \\
\textbf{Efficiency} & 40.6\% & 81.2\% \\
\textbf{Ripple Factor} & 1.21 & 0.48 \\
\textbf{Ripple Frequency} & Same as input (50 Hz) & Twice the input (100
Hz) \\
\textbf{PIV of Diode} & Vm & 2Vm (center-tap), Vm (bridge) \\
\textbf{TUF} & 0.287 & 0.693 (center-tap), 0.812 (bridge) \\
\textbf{DC Output Voltage} & 0.318Vm & 0.636Vm \\
\textbf{Form Factor} & 1.57 & 1.11 \\
\textbf{Applications} & Low power applications & Power supplies, battery
chargers \\
\end{longtable}
}

\textbf{Diagram}:

\begin{center}
\textbf{Mermaid Diagram (Code)}
\begin{verbatim}
{Shaded}
{Highlighting}[]
graph TD
    A[Rectifiers] {-{-}{} B[Half Wave]}
    A {-{-}{} C[Full Wave]}
    C {-{-}{} D[Center{-}Tapped]}
    C {-{-}{} E[Bridge]}
    B {-{-}{} F[Uses 1 diode]}
    B {-{-}{} G[Lower efficiency]}
    D {-{-}{} H[Uses 2 diodes]}
    E {-{-}{} I[Uses 4 diodes]}
    C {-{-}{} J[Higher efficiency]}
{Highlighting}
{Shaded}
\end{verbatim}
\end{center}

\end{solutionbox}
\begin{mnemonicbox}
``HERO'' - ``Half wave: Efficiency Reduced, One-half
cycle only''

\end{mnemonicbox}
\subsection*{Question 4(a) [3 marks]}\label{q4a}

\textbf{Draw the symbol and construction of PNP and NPN transistor with
proper notation.}

\begin{solutionbox}

\textbf{Transistor Symbols and Construction}:

\begin{verbatim}
NPN Symbol         PNP Symbol
    C                  C
    |                  |
    |                  |
    —                  —
   /                  /
  |                  |
  |{                 |}
  | {                |}
  |  {               |/}
  | /                |
  |/                 |
    —                  —
    |                  |
    |                  |
    B                  B
    |                  |
    |                  |
    —                  —
    |                  |
    |                  |
    E                  E
\end{verbatim}

\textbf{Construction}:

\begin{verbatim}
NPN Construction           PNP Construction
    +{-{-}{-}{-}{-}{-}{-}+                 +{-}{-}{-}{-}{-}{-}{-}+}
    |   N   |                 |   P   | {{-} Collector}
    +{-{-}{-}{-}{-}{-}{-}+                 +{-}{-}{-}{-}{-}{-}{-}+}
    |   P   |                 |   N   | {{-} Base}
    +{-{-}{-}{-}{-}{-}{-}+                 +{-}{-}{-}{-}{-}{-}{-}+}
    |   N   |                 |   P   | {{-} Emitter}
    +{-{-}{-}{-}{-}{-}{-}+                 +{-}{-}{-}{-}{-}{-}{-}+}
\end{verbatim}

\end{solutionbox}
\begin{mnemonicbox}
``NIN-PIP'' - ``N-P-N layers for NPN, P-N-P layers
for PNP''

\end{mnemonicbox}
\subsection*{Question 4(b) [4 marks]}\label{q4b}

\textbf{Explain working of Transistor amplifier.}

\begin{solutionbox}

\textbf{Working of Transistor Amplifier}:

\textbf{Circuit Configuration}:

\begin{itemize}
\tightlist
\item
  \textbf{Common Emitter}: Most commonly used
\item
  \textbf{Biasing}: Proper DC bias provided to operate in active region
\item
  \textbf{Coupling}: Input/output coupling through capacitors
\item
  \textbf{Load}: Collector resistor as load
\end{itemize}

\textbf{Working Principle}:

\begin{itemize}
\tightlist
\item
  \textbf{Input Signal}: Applied to base-emitter junction
\item
  \textbf{Base Current}: Small base current controls larger collector
  current
\item
  \textbf{Amplification}: Small input voltage variations cause larger
  output voltage variations
\item
  \textbf{Phase Shift}: 180^\circ phase shift between input and output
\end{itemize}

\textbf{Key Parameters}:

\begin{itemize}
\tightlist
\item
  \textbf{Voltage Gain}: Av = Vout/Vin
\item
  \textbf{Current Gain}: β = Ic/Ib
\item
  \textbf{Input Impedance}: Typically 1-2kΩ in CE configuration
\end{itemize}

\textbf{Diagram}:

\begin{center}
\textbf{Mermaid Diagram (Code)}
\begin{verbatim}
{Shaded}
{Highlighting}[]
graph LR
    A[Input Signal] {-{-}{} B[Base Current]}
    B {-{-}{} C[Controls Collector Current]}
    C {-{-}{} D[Voltage Drop Across R{}sub{}C{}/sub{}]}
    D {-{-}{} E[Amplified Output Signal]}
{Highlighting}
{Shaded}
\end{verbatim}
\end{center}

\end{solutionbox}
\begin{mnemonicbox}
``ABCD'' - ``Amplification through Base Controlled
collector Current Dynamics''

\end{mnemonicbox}
\subsection*{Question 4(c) [7 marks]}\label{q4c}

\textbf{Explain working of Zener diode.}

\begin{solutionbox}

\textbf{Working of Zener Diode}:

\textbf{Basic Structure}:

\begin{itemize}
\tightlist
\item
  \textbf{Junction}: Heavily doped P-N junction
\item
  \textbf{Construction}: Similar to normal diode but optimized for
  breakdown
\item
  \textbf{Breakdown}: Designed to operate in reverse breakdown region
\end{itemize}

\textbf{Working Principle}:

\begin{itemize}
\tightlist
\item
  \textbf{Forward Bias}: Acts like normal diode
\item
  \textbf{Reverse Bias}:

  \begin{itemize}
  \tightlist
  \item
    Below breakdown: Small leakage current
  \item
    At breakdown: Sharp increase in current at Zener voltage
  \item
    Beyond breakdown: Maintains constant voltage
  \end{itemize}
\end{itemize}

\textbf{Breakdown Mechanisms}:

\begin{itemize}
\tightlist
\item
  \textbf{Zener Effect}: Dominant below 5V (direct tunneling)
\item
  \textbf{Avalanche Effect}: Dominant above 5V (impact ionization)
\end{itemize}

\textbf{Applications}:

\begin{itemize}
\tightlist
\item
  \textbf{Voltage Regulation}: Maintains constant output voltage
\item
  \textbf{Reference Voltage}: Precise voltage reference
\item
  \textbf{Overvoltage Protection}: Protects sensitive components
\end{itemize}

\textbf{Diagram}:

\begin{verbatim}
    I
    \^{}
    |               /
    |              /
    |             /
    |            /
    |           /
    +{-{-}{-}{-}{-}{-}{-}{-}{-}{-}+{-}{-}{-}{-}{-}{-} V}
    |         /|
    |        / |
    |       /  |
    |      /   |
    |  Reverse | Forward
    |  Breakdown
\end{verbatim}

\end{solutionbox}
\begin{mnemonicbox}
``ZEBRA'' - ``Zener Effect Breaks at Regulated
Avalanche voltage''

\end{mnemonicbox}
\subsection*{Question 4(a) [3 marks]
(OR)}\label{q4a}

\textbf{Explain transistor as a switch.}

\begin{solutionbox}

\textbf{Transistor as a Switch}:

\textbf{Operating Regions}:

\begin{itemize}
\tightlist
\item
  \textbf{Cutoff Region}: Transistor OFF (IB = 0, IC \approx 0)
\item
  \textbf{Saturation Region}: Transistor ON (IB \textgreater{} IC/β, VCE
  \approx 0.2V)
\end{itemize}

\textbf{Switching Operation}:

\begin{itemize}
\tightlist
\item
  \textbf{OFF State}: No base current, high VCE, acts as open switch
\item
  \textbf{ON State}: Sufficient base current, low VCE, acts as closed
  switch
\end{itemize}

\textbf{Switching Characteristics}:

\begin{itemize}
\tightlist
\item
  \textbf{Turn-ON Time}: Time to go from cutoff to saturation
\item
  \textbf{Turn-OFF Time}: Time to go from saturation to cutoff
\end{itemize}

\textbf{Diagram}:

\begin{center}
\textbf{Mermaid Diagram (Code)}
\begin{verbatim}
{Shaded}
{Highlighting}[]
graph TD
    A[Transistor Switch] {-{-}{} B[OFF State: Cutoff Region]}
    A {-{-}{} C[ON State: Saturation Region]}
    B {-{-}{} D[I{}sub{}B{}/sub{} = 0, I{}sub{}C{}/sub{}  0]}
    B {-{-}{} E[High V{}sub{}CE{}/sub{}  V{}sub{}CC{}/sub{}]}
    C {-{-}{} F[I{}sub{}B{}/sub{} {} I{}sub{}C{}/sub{}/β]}
    C {-{-}{} G[Low V{}sub{}CE{}/sub{}  0.2V]}
{Highlighting}
{Shaded}
\end{verbatim}
\end{center}

\end{solutionbox}
\begin{mnemonicbox}
``COST'' - ``Cutoff Off, Saturation Turns-on''

\end{mnemonicbox}
\subsection*{Question 4(b) [4 marks]
(OR)}\label{q4b}

\textbf{Draw and Explain characteristics of CE amplifier.}

\begin{solutionbox}

\textbf{CE Amplifier Characteristics}:

\textbf{Input Characteristics}:

\begin{itemize}
\tightlist
\item
  \textbf{Plot}: IB vs VBE at constant VCE
\item
  \textbf{Behavior}: Resembles forward-biased diode curve
\item
  \textbf{Knee Voltage}: Approximately 0.7V for silicon
\item
  \textbf{Input Resistance}: Slope of curve (ΔVBE/ΔIB)
\end{itemize}

\textbf{Output Characteristics}:

\begin{itemize}
\tightlist
\item
  \textbf{Plot}: IC vs VCE at constant IB
\item
  \textbf{Regions}:

  \begin{itemize}
  \tightlist
  \item
    Saturation (VCE \textless{} 0.2V)
  \item
    Active (VCE \textgreater{} 0.2V)
  \item
    Cutoff (IB = 0)
  \end{itemize}
\item
  \textbf{Early Effect}: Slight increase in IC with increasing VCE
\end{itemize}

\textbf{Diagram}:

\begin{verbatim}
   I\_C |           I\_B3
       |         ,{-{-}{-}{-}{-}{-}}
       |        /
       |       /
       |      /  I\_B2
       |     ,{-{-}{-}{-}{-}{-}}
       |    /
       |   /
       |  /  I\_B1
       | ,{-{-}{-}{-}{-}{-}}
       |/
       +{-{-}{-}{-}{-}{-}{-}{-}{-}{-}{-}{-}{-} V\_CE}
       |
   
   I\_B |
       |        /
       |       /
       |      /
       |     /
       |    /
       |   /
       |  /
       | /
       |/
       +{-{-}{-}{-}{-}{-}{-}{-}{-}{-}{-}{-}{-} V\_BE}
       |   0.7V
\end{verbatim}

\end{solutionbox}
\begin{mnemonicbox}
``IAOC'' - ``Input curves At Origin, Output curves
show Current gain''

\end{mnemonicbox}
\subsection*{Question 4(c) [7 marks]
(OR)}\label{q4c}

\textbf{Explain working of Varactor diode.}

\begin{solutionbox}

\textbf{Working of Varactor Diode}:

\textbf{Basic Structure}:

\begin{itemize}
\tightlist
\item
  \textbf{Junction}: Special P-N junction diode
\item
  \textbf{Operation}: Always operated in reverse bias
\item
  \textbf{Property}: Junction capacitance varies with reverse voltage
\end{itemize}

\textbf{Working Principle}:

\begin{itemize}
\tightlist
\item
  \textbf{Depletion Layer}: Widens with increasing reverse voltage
\item
  \textbf{Capacitance Effect}: Depletion region acts as dielectric
  between P and N regions
\item
  \textbf{Capacitance Formula}: C ∝ 1/\sqrtVR
\item
  \textbf{Tuning Range}: Typically 4:1 to 10:1 capacitance
\end{itemize}

\textbf{Applications}:

\begin{itemize}
\tightlist
\item
  \textbf{Voltage-Controlled Capacitor}: In electronic tuning circuits
\item
  \textbf{Frequency Modulation}: In voltage-controlled oscillators
  (VCOs)
\item
  \textbf{Automatic Frequency Control}: In receivers
\item
  \textbf{Parametric Amplification}: In microwave circuits
\end{itemize}

\textbf{Diagram}:

\begin{center}
\textbf{Mermaid Diagram (Code)}
\begin{verbatim}
{Shaded}
{Highlighting}[]
graph LR
    A[Varactor Diode] {-{-}{} B[Reverse Bias Operation]}
    B {-{-}{} C[Depletion Region Width]}
    C {-{-}{} D[Junction Capacitance]}
    D {-{-}{} E[Changes with Applied Voltage]}
    E {-{-}{} F[Electronic Tuning]}
{Highlighting}
{Shaded}
\end{verbatim}
\end{center}

\end{solutionbox}
\begin{mnemonicbox}
``VCAP'' - ``Voltage Controls cAPacitance''

\end{mnemonicbox}
\subsection*{Question 5(a) [3 marks]}\label{q5a}

\textbf{Define Active, Saturation and Cut-off region for transistor
amplifier.}

\begin{solutionbox}

\textbf{Transistor Regions of Operation}:

{\def\LTcaptype{none} % do not increment counter
\begin{longtable}[]{@{}
  >{\raggedright\arraybackslash}p{(\linewidth - 6\tabcolsep) * \real{0.1538}}
  >{\raggedright\arraybackslash}p{(\linewidth - 6\tabcolsep) * \real{0.2308}}
  >{\raggedright\arraybackslash}p{(\linewidth - 6\tabcolsep) * \real{0.3654}}
  >{\raggedright\arraybackslash}p{(\linewidth - 6\tabcolsep) * \real{0.2500}}@{}}
\toprule\noalign{}
\begin{minipage}[b]{\linewidth}\raggedright
Region
\end{minipage} & \begin{minipage}[b]{\linewidth}\raggedright
Definition
\end{minipage} & \begin{minipage}[b]{\linewidth}\raggedright
Biasing Condition
\end{minipage} & \begin{minipage}[b]{\linewidth}\raggedright
Application
\end{minipage} \\
\midrule\noalign{}
\endhead
\bottomrule\noalign{}
\endlastfoot
\textbf{Active Region} & Both junctions are properly biased (BE forward,
BC reverse) & IB \textgreater{} 0, VCE \textgreater{} VCE(sat) &
Amplification \\
\textbf{Saturation Region} & Both junctions forward biased & IB
\textgreater{} IC/β, VCE \approx 0.2V & Switching (ON state) \\
\textbf{Cut-off Region} & Both junctions reverse biased & IB = 0, IC \approx
0, VCE \approx VCC & Switching (OFF state) \\
\end{longtable}
}

\textbf{Diagram}:

\begin{verbatim}
   I\_C |
       |         Active
       |         Region
       |        /|
       |       / |
       |      /  |
       |     /   |
       |    /    |
       |   /     |
       |  /      |
       | /       |
       |/        |
       +{-{-}{-}{-}{-}{-}{-}{-}{-}+{-}{-}{-}{-}{-}{-} V\_CE}
       |Saturation|Cut{-off}
\end{verbatim}

\end{solutionbox}
\begin{mnemonicbox}
``ASC'' - ``Active for Signals, Saturation \& Cutoff
for switches''

\end{mnemonicbox}
\subsection*{Question 5(b) [4 marks]}\label{q5b}

\textbf{If the value of IC = 10mA and IB = 100μA then find the value of
current gains α and β.}

\begin{solutionbox}

\textbf{Given}:

\begin{itemize}
\tightlist
\item
  Collector current (IC) = 10 mA
\item
  Base current (IB) = 100 μA = 0.1 mA
\end{itemize}

\textbf{Calculate β (Common Emitter Current Gain)}:

\begin{itemize}
\tightlist
\item
  β = IC / IB
\item
  β = 10 mA / 0.1 mA
\item
  β = 100
\end{itemize}

\textbf{Calculate α (Common Base Current Gain)}:

\begin{itemize}
\tightlist
\item
  IE = IC + IB = 10 mA + 0.1 mA = 10.1 mA
\item
  α = IC / IE
\item
  α = 10 mA / 10.1 mA
\item
  α = 0.990 or 0.99
\end{itemize}

\textbf{Relation between α and β}:

\begin{itemize}
\tightlist
\item
  α = β / (β + 1)
\item
  α = 100 / (100 + 1) = 100 / 101 = 0.990
\item
  β = α / (1 - α)
\item
  β = 0.99 / (1 - 0.99) = 0.99 / 0.01 = 99 \approx 100
\end{itemize}

\end{solutionbox}
\begin{mnemonicbox}
``ABC'' - ``Alpha equals Beta divided by (Beta plus
one) for Current gains''

\end{mnemonicbox}
\subsection*{Question 5(c) [7 marks]}\label{q5c}

\textbf{Discuss Strategies of electronic waste management in the small
electronics Industries.}

\begin{solutionbox}

\textbf{E-Waste Management Strategies for Small Electronics Industries}:

{\def\LTcaptype{none} % do not increment counter
\begin{longtable}[]{@{}
  >{\raggedright\arraybackslash}p{(\linewidth - 4\tabcolsep) * \real{0.2564}}
  >{\raggedright\arraybackslash}p{(\linewidth - 4\tabcolsep) * \real{0.3333}}
  >{\raggedright\arraybackslash}p{(\linewidth - 4\tabcolsep) * \real{0.4103}}@{}}
\toprule\noalign{}
\begin{minipage}[b]{\linewidth}\raggedright
Strategy
\end{minipage} & \begin{minipage}[b]{\linewidth}\raggedright
Description
\end{minipage} & \begin{minipage}[b]{\linewidth}\raggedright
Implementation
\end{minipage} \\
\midrule\noalign{}
\endhead
\bottomrule\noalign{}
\endlastfoot
\textbf{Segregation} & Separate e-waste from general waste & Dedicated
collection bins for different components \\
\textbf{Reduce} & Minimize waste generation & Efficient design, extended
product life, repair services \\
\textbf{Reuse} & Use components again & Refurbish, repurpose working
parts \\
\textbf{Recycle} & Process for material recovery & Partner with
authorized recyclers, follow guidelines \\
\textbf{Training} & Educate employees & Regular workshops on proper
handling procedures \\
\end{longtable}
}

\textbf{Key Implementation Steps}:

\begin{itemize}
\tightlist
\item
  \textbf{Inventory Management}: Track electronic components throughout
  lifecycle
\item
  \textbf{Authorized Partnerships}: Work only with certified e-waste
  handlers
\item
  \textbf{Documentation}: Maintain records of waste disposal for
  compliance
\item
  \textbf{Green Design}: Design products for easy disassembly and
  recycling
\end{itemize}

\textbf{Regulatory Compliance}:

\begin{itemize}
\tightlist
\item
  \textbf{Registration}: Register with pollution control board
\item
  \textbf{Authorization}: Obtain necessary permits
\item
  \textbf{Annual Returns}: Submit regular compliance reports
\end{itemize}

\textbf{Diagram}:

\begin{center}
\textbf{Mermaid Diagram (Code)}
\begin{verbatim}
{Shaded}
{Highlighting}[]
graph TD
    A[E{-Waste Management] {-}{-}{} B[Collection \& Segregation]}
    A {-{-}{} C[Storage]}
    A {-{-}{} D[Transportation]}
    A {-{-}{} E[Processing]}
    B {-{-}{} F[Separate bins for different components]}
    C {-{-}{} G[Safe storage in designated areas]}
    D {-{-}{} H[Authorized carriers only]}
    E {-{-}{} I[Authorized recyclers]}
    E {-{-}{} J[Material recovery]}
    E {-{-}{} K[Safe disposal of residues]}
{Highlighting}
{Shaded}
\end{verbatim}
\end{center}

\end{solutionbox}
\begin{mnemonicbox}
``SRRTA'' - ``Segregate, Reduce, Reuse, Train,
Authorize''

\end{mnemonicbox}
\subsection*{Question 5(a) [3 marks]
(OR)}\label{q5a}

\textbf{Draw CB, CE and CC transistor configuration circuits.}

\begin{solutionbox}

\textbf{Transistor Configuration Circuits}:

\begin{verbatim}
Common Base (CB)         Common Emitter (CE)        Common Collector (CC)
                                                     (Emitter Follower)
    +{-{-}{-}+                     +{-}{-}{-}+                      +{-}{-}{-}+}
    |   |                     |   |                      |   |
    | RC|                     | RC|                      |   |
    |   |                     |   |                      |   |
    +{-{-}{-}+                     +{-}{-}{-}+                      +{-}{-}{-}+}
      |                         |                          |
      |                         |                          |
      +{-{-}{-}{-}{-}+             +{-}{-}{-}{-}{-}+                    +{-}{-}{-}{-}{-}+}
      |     |             |     |                    |     |
 Cout +     +{-{-}{-}+    Cout +     +{-}{-}{-}+          +{-}{-}{-}{-}+     +{-}{-}{-} Vout}
      |     |             |     |              |     |     |
      +{-{-}{-}{-}{-}+             +{-}{-}{-}{-}{-}+              |     +{-}{-}{-}{-}{-}+}
        |                   |                  |       |
        |                   |                  |       |
    +{-{-}{-}+                   |                  |     +{-}{-}{-}+}
    |   |                   |                  |     |   |
    | RE|               +{-{-}{-}+{-}{-}{-}+              |     | RE|}
    |   |               |       |              |     |   |
    +{-{-}{-}+               |       |              |     +{-}{-}{-}+}
      |                 +{-{-}{-}{-}{-}{-}{-}+              |       |}
      |                     |                  |       |
     GND                   GND                 +{-{-}{-}{-}{-}{-}{-}+}
                                                   |
 Input to Emitter      Input to Base           Input to Base
 Output from Collector Output from Collector   Output from Emitter
\end{verbatim}

\textbf{Key Characteristics}:

\begin{itemize}
\tightlist
\item
  \textbf{CB}: High stability, low input impedance, high output
  impedance
\item
  \textbf{CE}: Medium stability, medium input impedance, medium output
  impedance
\item
  \textbf{CC}: Low stability, high input impedance, low output impedance
\end{itemize}

\end{solutionbox}
\begin{mnemonicbox}
``EBC'' - ``Emitter input for CB, Base input for
CE/CC, Collector output for CB/CE''

\end{mnemonicbox}
\subsection*{Question 5(b) [4 marks]
(OR)}\label{q5b}

\textbf{Derive relation between current gains α and β.}

\begin{solutionbox}

\textbf{Relation Between Current Gains α and β}:

\textbf{Given definitions}:

\begin{itemize}
\tightlist
\item
  α = IC/IE (Common Base current gain)
\item
  β = IC/IB (Common Emitter current gain)
\end{itemize}

\textbf{Step 1}: Use current relation in transistor

\begin{itemize}
\tightlist
\item
  IE = IC + IB
\end{itemize}

\textbf{Step 2}: Express α in terms of β

\begin{itemize}
\tightlist
\item
  α = IC/IE
\item
  α = IC/(IC + IB)
\end{itemize}

\textbf{Step 3}: Substitute IB = IC/β

\begin{itemize}
\tightlist
\item
  α = IC/(IC + IC/β)
\item
  α = IC/(IC(1 + 1/β))
\item
  α = IC/(IC(β + 1)/β)
\item
  α = β/(β + 1)
\end{itemize}

\textbf{Step 4}: Express β in terms of α

\begin{itemize}
\tightlist
\item
  β = α/(1 - α)
\end{itemize}

\textbf{Diagram}:

\begin{verbatim}
      I\_C
     ↗   ↘
    /     {}
   /       {}
  I\_B       I\_E

  α = I\_C/I\_E
  β = I\_C/I\_B
  I\_E = I\_C + I\_B
\end{verbatim}

\end{solutionbox}
\begin{mnemonicbox}
``ABR'' - ``Alpha = Beta divided by (Beta plus one)
Reciprocally''

\end{mnemonicbox}
\subsection*{Question 5(c) [7 marks]
(OR)}\label{q5c}

\textbf{Define E-Waste and Explain disposal of electronic waste.}

\begin{solutionbox}

\textbf{E-Waste Definition}: Electronic waste (e-waste) refers to
discarded electrical or electronic devices that have reached end-of-life
or become obsolete, including computers, televisions, mobile phones,
printers, and other electronic equipment containing hazardous components
like lead, mercury, cadmium, PCBs, and brominated flame retardants.

\textbf{Disposal Methods of E-Waste}:

{\def\LTcaptype{none} % do not increment counter
\begin{longtable}[]{@{}
  >{\raggedright\arraybackslash}p{(\linewidth - 4\tabcolsep) * \real{0.1905}}
  >{\raggedright\arraybackslash}p{(\linewidth - 4\tabcolsep) * \real{0.3095}}
  >{\raggedright\arraybackslash}p{(\linewidth - 4\tabcolsep) * \real{0.5000}}@{}}
\toprule\noalign{}
\begin{minipage}[b]{\linewidth}\raggedright
Method
\end{minipage} & \begin{minipage}[b]{\linewidth}\raggedright
Description
\end{minipage} & \begin{minipage}[b]{\linewidth}\raggedright
Environmental Impact
\end{minipage} \\
\midrule\noalign{}
\endhead
\bottomrule\noalign{}
\endlastfoot
\textbf{Collection \& Segregation} & Gathering and separating by type &
Reduces contamination \\
\textbf{Dismantling} & Manual disassembly of components & Enables
targeted recycling \\
\textbf{Material Recovery} & Extracting valuable materials & Conserves
natural resources \\
\textbf{Refurbishment} & Repairing for reuse & Extends product
lifecycle \\
\textbf{Authorized Recycling} & Processing by certified facilities &
Ensures proper handling \\
\end{longtable}
}

\textbf{Disposal Process Flow}:

\begin{itemize}
\tightlist
\item
  \textbf{Initial Assessment}: Determine if device can be
  repaired/reused
\item
  \textbf{Data Sanitization}: Secure erasure of personal/business data
\item
  \textbf{Disassembly}: Separation into component categories
\item
  \textbf{Resource Recovery}: Extraction of valuable materials
\item
  \textbf{Hazardous Waste}: Special handling of toxic components
\end{itemize}

\textbf{Diagram}:

\begin{center}
\textbf{Mermaid Diagram (Code)}
\begin{verbatim}
{Shaded}
{Highlighting}[]
graph TD
    A[E{-Waste Disposal] {-}{-}{} B[Collection]}
    B {-{-}{} C[Sorting \& Segregation]}
    C {-{-}{} D[Recycling]}
    C {-{-}{} E[Recovery]}
    C {-{-}{} F[Safe Disposal]}
    D {-{-}{} G[Disassembly]}
    G {-{-}{} H[Material Sorting]}
    H {-{-}{} I[Crushing \& Shredding]}
    I {-{-}{} J[Material Separation]}
    J {-{-}{} K[Refinement]}
    K {-{-}{} L[New Products]}
    F {-{-}{} M[Landfill for Inert Material]}
    F {-{-}{} N[Incineration with Pollution Control]}
{Highlighting}
{Shaded}
\end{verbatim}
\end{center}

\end{solutionbox}
\begin{mnemonicbox}
``CRESD'' - ``Collect, Recycle, Extract, Separate,
Dispose''

\end{mnemonicbox}

\end{document}
