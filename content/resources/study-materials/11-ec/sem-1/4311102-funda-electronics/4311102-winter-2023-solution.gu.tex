\documentclass{article}
% Adjust the relative path to point to the latex-templates directory

% content/resources/templates/preamble.tex
\usepackage[margin=0.6in]{geometry}
\author{Milav Dabgar}
\usepackage{amsmath,amssymb,amsthm}
\usepackage{booktabs}
\usepackage{multirow}
\usepackage{xcolor}
\usepackage{tcolorbox}
\tcbuselibrary{breakable,skins}
\usepackage[colorlinks=true,linkcolor=blue]{hyperref}
\usepackage{titlesec}
\usepackage{enumitem}
\usepackage{tikz}
\usepackage{pgfplots}
\usepackage{circuitikz}
\usepackage[version=4]{mhchem}
\usepackage{longtable}
\usepackage{array}
\usepackage{float}
\usepackage{caption}
\usepackage{listings}

\lstset{
  basicstyle=\small\ttfamily,
  breaklines=true,
  breakatwhitespace=false,
  postbreak=\mbox{\textcolor{red}{$\hookrightarrow$}\space},
  float=false,
  numbers=left,
  numberstyle=\tiny\color{gray},
  numbersep=10pt,
  xleftmargin=2em,
  keywordstyle=\color{blue},
  commentstyle=\color{green!60!black},
  stringstyle=\color{purple},
  backgroundcolor=\color{gray!5},
  showstringspaces=false,
  tabsize=2,
  captionpos=b,
  keepspaces=true,
  columns=flexible
}

\pgfplotsset{compat=1.18}
\usetikzlibrary{shapes,arrows,positioning,calc,patterns,decorations.pathmorphing,decorations.markings,arrows.meta}

% Color scheme
\definecolor{headcolor}{RGB}{0,102,204}
\definecolor{keycolor}{RGB}{220,20,60}
\definecolor{solutioncolor}{RGB}{34,139,34}
\definecolor{mnemoniccolor}{RGB}{148,0,211}
\definecolor{codecolor}{RGB}{0,0,100}

% Spacing
\setlength{\parskip}{3pt}
\setlist[itemize]{nosep}
\setlist[enumerate]{nosep}

% Title formatting
\titleformat{\section}{\Large\bfseries\color{headcolor}}{\thesection}{1em}{}
\titleformat{\subsection}{\large\bfseries\color{headcolor}}{\thesubsection}{1em}{}

% Pandoc tightlist compatibility
\providecommand{\tightlist}{%
  \setlength{\itemsep}{0pt}\setlength{\parskip}{0pt}}

% Pandoc longtable compatibility
\newcounter{none}
\def\thenone{}


% content/resources/templates/gujarati-boxes.tex
\usepackage{fontspec}
\usepackage{polyglossia}

% Set Gujarati as main language (document is primarily in Gujarati)
% Note: gloss-gujarati.ldf doesn't exist in polyglossia, but it will use hyphenation patterns
\setdefaultlanguage{gujarati}
\setotherlanguage{english}

% Configure Gujarati font properly
% Use Language=Default to prevent polyglossia from trying to add language-specific features
% that don't exist for Gujarati, which causes "empty feature" warnings
\newfontfamily\gujaratifont[Script=Gujarati,AutoFakeBold=2.5,AutoFakeSlant=0.3]{Noto Sans Gujarati}
\setmainfont[Script=Gujarati,AutoFakeBold=2.5,AutoFakeSlant=0.3]{Noto Sans Gujarati}
% Use Noto Sans Gujarati for monospace to support Gujarati in text
\setmonofont[Scale=0.9]{Noto Sans Gujarati}

% Configure English to use the same font
\newfontfamily\englishfont[Script=Gujarati,AutoFakeBold=2.5,AutoFakeSlant=0.3]{Noto Sans Gujarati}

% Translations for polyglossia
\gappto\captionsgujarati{
  \renewcommand{\tablename}{કોષ્ટક}
  \renewcommand{\figurename}{આકૃતિ}
}

% Helper for TikZ nodes to ensure Gujarati font
\newcommand{\gu}[1]{{\gujaratifont #1}}

% Custom environments
\newtcolorbox{solutionbox}{
    breakable,
    enhanced,
    colback=solutioncolor!5!white,
    colframe=solutioncolor!75!black,
    fonttitle=\bfseries,
    title=જવાબ
}

\newtcolorbox{solutionboxnobreak}{
 colback=solutioncolor!5!white,
 colframe=solutioncolor!75!black,
 fonttitle=\bfseries,
 title=જવાબ
}

\newtcolorbox{keyformula}{
 breakable,
 enhanced,
 colback=keycolor!5!white,
 colframe=keycolor!75!black,
 fonttitle=\bfseries,
 title=રાસાયણિક સમીકરણ/સૂત્ર
}

\newtcolorbox{mnemonicbox}{
 breakable,
 enhanced,
 colback=mnemoniccolor!5!white,
 colframe=mnemoniccolor!75!black,
 fonttitle=\bfseries,
 title=મેમરી ટ્રીક
}


% Custom commands for GTU solutions
% This file defines semantic commands for consistent formatting

% Question command with automatic formatting
\newcommand{\question}[2]{%
  \section*{Question #1}%
  \textbf{#2}%
}

% OR question variant
\newcommand{\questionor}[2]{%
  \section*{Question #1 OR}%
  \textbf{#2}%
}

% Proper table environment with caption
\newenvironment{answertable}[1]{%
  \begin{table}[htbp]
  \centering
  \caption{#1}
}{%
  \end{table}
}

% Proper figure environment for diagrams
\newenvironment{answerdiagram}[1]{%
  \begin{figure}[htbp]
  \centering
  \caption{#1}
}{%
  \end{figure}
}

% Semantic markup for key terms
\newcommand{\keyword}[1]{\textbf{#1}}
\newcommand{\code}[1]{\texttt{#1}}
\newcommand{\classname}[1]{\texttt{#1}}
\newcommand{\methodname}[1]{\texttt{#1}}

% Proper quotation marks
\newcommand{\mnemonic}[1]{``#1''}


\title{ડાયોડના ફોરવડડ અને રિવર્સ બાયસ (4311102) - વિન્ટર 2023 સોલ્યુશન}
\date{જાન્યુઆરી 24, 2023}

\begin{document}
\maketitle

\questionmarks{1(અ)}{3}{ડાયોડના ફોરવડડ અને રિવર્સ બાયસને વ્યાખ્યાયિત કરો.}

\begin{solutionbox}
\textbf{જવાબ}:

\textbf{ડાયોડનો ફોરવડડ બાયસ}:

\begin{itemize}
    \item \keyword{જોડાણની પદ્ધતિ}: P-ટાઈપ બેટરીના પોઝિટિવ ટર્મિનલ સાથે અને N-ટાઈપ નેગેટિવ ટર્મિનલ સાથે જોડાયેલા હોય છે.
    \item \keyword{અવરોધ પહોળાઈ}: અવરોધની પહોળાઈ (Barrier width) ઘટે છે.
    \item \keyword{અવરોધ}: ઓછો અવરોધ (આશરે 100-1000 $\Omega$).
    \item \keyword{કરંટ પ્રવાહ}: ડાયોડ દ્વારા સરળતાથી કરંટ પસાર થવા દે છે.
\end{itemize}

\textbf{ડાયોડનો રિવર્સ બાયસ}:

\begin{itemize}
    \item \keyword{જોડાણની પદ્ધતિ}: P-ટાઈપ નેગેટિવ ટર્મિનલ સાથે અને N-ટાઈપ પોઝિટિવ ટર્મિનલ સાથે જોડાયેલા હોય છે.
    \item \keyword{અવરોધ પહોળાઈ}: અવરોધની પહોળાઈ (Barrier width) વધે છે.
    \item \keyword{અવરોધ}: ખૂબ ઊંચો અવરોધ (આશરે કેટલાક M$\Omega$).
    \item \keyword{કરંટ પ્રવાહ}: કરંટ પ્રવાહને અટકાવે છે (માત્ર નાનો લીકેજ કરંટ પસાર થાય છે).
\end{itemize}

\textbf{ડાયાગ્રામ}:

\begin{center}
\begin{tikzpicture}[node distance=2.5cm, auto]
    \node [gtu state] (Fwd) {ફોરવર્ડ બાયસ};
    \node [gtu block, right=1cm of Fwd, text width=4cm] (FwdDesc) {P પોઝિટિવ સાથે, N નેગેટિવ સાથે\\કરંટ સરળતાથી પસાર થાય};
    \node [gtu block, right=1cm of FwdDesc] (FwdRes) {લો રેઝિસ્ટન્સ};

    \node [gtu state, below=1.5cm of Fwd] (Rev) {રિવર્સ બાયસ};
    \node [gtu block, right=1cm of Rev, text width=4cm] (RevDesc) {P નેગેટિવ સાથે, N પોઝિટિવ સાથે\\કરંટ બ્લોક થાય};
    \node [gtu block, right=1cm of RevDesc] (RevRes) {હાઈ રેઝિસ્ટન્સ};

    \path [gtu arrow] (Fwd) -- (FwdDesc);
    \path [gtu arrow] (FwdDesc) -- (FwdRes);
    \path [gtu arrow] (Rev) -- (RevDesc);
    \path [gtu arrow] (RevDesc) -- (RevRes);
\end{tikzpicture}
\captionof{figure}{ફોરવર્ડ અને રિવર્સ બાયસ}
\end{center}
\end{solutionbox}

\begin{mnemonicbox}
\mnemonic{PFNR: "Positive to P Forward, Negative to P Reverse"}
\end{mnemonicbox}

\questionmarks{1(બ)}{4}{LDRનું બંધારણ અને કાર્ય સમજાવો.}

\begin{solutionbox}
\textbf{જવાબ}:

\textbf{LDRનું બંધારણ}:

\begin{itemize}
    \item \keyword{સામગ્રી}: સેમિકંડક્ટર સામગ્રી (કેડમિયમ સલ્ફાઇડ - Cadmium Sulfide)થી બનેલું હોય છે.
    \item \keyword{પેટર્ન}: સિરામિક બેઝ પર ફોટોસેન્સિટિવ સામગ્રીનું ઝિગઝેગ પેટર્ન હોય છે.
    \item \keyword{ઇલેક્ટ્રોડ્સ}: બંને છેડે મેટલ ઇલેક્ટ્રોડ્સ હોય છે.
    \item \keyword{પેકેજિંગ}: પારદર્શક પ્લાસ્ટિક અથવા ગ્લાસ કેસમાં એન્કેપ્સ્યુલેટેડ હોય છે.
\end{itemize}

\textbf{કાર્યપ્રણાલી}:

\begin{itemize}
    \item \keyword{સિદ્ધાંત}: ફોટોકન્ડક્ટિવિટી (Photoconductivity) સિદ્ધાંત પર આધારિત છે.
    \item \keyword{અંધકારમાં અવરોધ}: અંધકારની સ્થિતિમાં ઉચ્ચ અવરોધ (M$\Omega$ રેન્જ) હોય છે.
    \item \keyword{પ્રકાશ સંપર્ક}: જ્યારે પ્રકાશના સંપર્કમાં આવે છે, ત્યારે ફોટોન્સ ઇલેક્ટ્રોન્સને મુક્ત કરે છે.
    \item \keyword{અવરોધમાં ઘટાડો}: તેજ પ્રકાશમાં અવરોધ ઘટે છે (k$\Omega$ રેન્જ).
\end{itemize}

\textbf{ડાયાગ્રામ}:

\begin{center}
\begin{tikzpicture}
    % Ceramic Base
    \draw[fill=white, thick] (0,0) circle (1.5cm);
    
    % Electrodes (Simplified Zigzag)
    \draw[line width=1mm, color=gray] (-0.5, 1) -- (-0.5, -1);
    \draw[line width=1mm, color=gray] (0.5, 1) -- (0.5, -1);
    \draw[line width=0.5mm] (-0.5, 0.8) -- (0.5, 0.6) -- (-0.5, 0.4) -- (0.5, 0.2) -- (-0.5, 0) -- (0.5, -0.2) -- (-0.5, -0.4) -- (0.5, -0.6) -- (-0.5, -0.8);
    
    % Leads
    \draw[thick] (-0.5, -1) -- (-0.5, -2);
    \draw[thick] (0.5, -1) -- (0.5, -2);
    \node at (0, -2.3) {લીડ્સ};
    
    % Arrows for light
     \draw[->, thick, decorate, decoration={snake,amplitude=.4mm,segment length=2mm,post length=1mm}] (-2, 2) -- (-0.5, 0.5);
     \draw[->, thick, decorate, decoration={snake,amplitude=.4mm,segment length=2mm,post length=1mm}] (-1.5, 2) -- (0, 0.5);
     \node at (-1.8, 2.2) {પ્રકાશ (Light)};

    % Label
    \node at (0, 1.8) {LDR બંધારણ};
\end{tikzpicture}
\captionof{figure}{LDR બંધારણ}
\end{center}
\end{solutionbox}

\begin{mnemonicbox}
\mnemonic{MILD: "More Illumination, Less Dark-resistance"}
\end{mnemonicbox}

\questionmarks{1(ક)}{7}{રેસિસ્ટરની કલર બેન્ડ કોડિંગ પદ્ધતિ સમજાવો. 47k$\Omega$ $\pm$5\% રેસિસ્ટરની કલર બેન્ડ લખો.}

\begin{solutionbox}
\textbf{જવાબ}:

\textbf{કલર બેન્ડ કોડિંગ પદ્ધતિ}:

\begin{center}
\captionof{table}{રેસિસ્ટર કલર કોડ}
\begin{tabulary}{\linewidth}{|L|L|L|L|}
\hline
\textbf{રંગ} & \textbf{મૂલ્ય} & \textbf{ગુણાંક (Multiplier)} & \textbf{ટોલરન્સ} \\ \hline
કાળો (Black) & 0 & $10^0$ & - \\ \hline
બ્રાઉન (Brown) & 1 & $10^1$ & $\pm1\%$ \\ \hline
લાલ (Red) & 2 & $10^2$ & $\pm2\%$ \\ \hline
નારંગી (Orange) & 3 & $10^3$ & - \\ \hline
પીળો (Yellow) & 4 & $10^4$ & - \\ \hline
લીલો (Green) & 5 & $10^5$ & $\pm0.5\%$ \\ \hline
બ્લુ (Blue) & 6 & $10^6$ & $\pm0.25\%$ \\ \hline
વાયોલેટ (Violet) & 7 & $10^7$ & $\pm0.1\%$ \\ \hline
ગ્રે (Grey) & 8 & $10^8$ & $\pm0.05\%$ \\ \hline
સફેદ (White) & 9 & $10^9$ & - \\ \hline
ગોલ્ડ (Gold) & - & $10^{-1}$ & $\pm5\%$ \\ \hline
સિલ્વર (Silver) & - & $10^{-2}$ & $\pm10\%$ \\ \hline
રંગવિહીન (Colorless) & - & - & $\pm20\%$ \\ \hline
\end{tabulary}
\end{center}

\textbf{4-બેન્ડ રેસિસ્ટર કલર કોડ}:
\begin{itemize}
    \item \keyword{પ્રથમ બેન્ડ}: પ્રથમ અર્થપૂર્ણ અંક (First significant digit).
    \item \keyword{બીજી બેન્ડ}: બીજો અર્થપૂર્ણ અંક (Second significant digit).
    \item \keyword{ત્રીજી બેન્ડ}: ગુણાંક (Multiplier).
    \item \keyword{ચોથી બેન્ડ}: ટોલરન્સ (Tolerance).
\end{itemize}

\textbf{47k$\Omega$ $\pm$5\% માટે}:
\begin{itemize}
    \item પ્રથમ અંક: 4 = પીળો (Yellow)
    \item બીજો અંક: 7 = વાયોલેટ (Violet)
    \item ગુણાંક: $10^3$ = નારંગી (Orange) (for k$\Omega$)
    \item ટોલરન્સ: $\pm5\%$ = ગોલ્ડ (Gold)
\end{itemize}

\textbf{47k$\Omega$ $\pm$5\% માટે કલર બેન્ડ}: પીળો-વાયોલેટ-નારંગી-ગોલ્ડ (Yellow-Violet-Orange-Gold)

\textbf{ડાયાગ્રામ}:

\begin{center}
\begin{tikzpicture}
    % Resistor Body
    \draw[thick, fill=gray!20] (0,0) rectangle (6,1.5);
    \draw[thick] (-1, 0.75) -- (0, 0.75);
    \draw[thick] (6, 0.75) -- (7, 0.75);
    
    % Bands
    \draw[fill=yellow] (1,0) rectangle (1.5,1.5);
    \node[below, rotate=45] at (1.25,0) {Yellow (4)};
    
    \draw[fill=violet] (2,0) rectangle (2.5,1.5);
    \node[below, rotate=45] at (2.25,0) {Violet (7)};
    
    \draw[fill=orange] (3,0) rectangle (3.5,1.5);
    \node[below, rotate=45] at (3.25,0) {Orange ($10^3$)};
    
    \draw[fill=yellow!80!black] (5,0) rectangle (5.5,1.5); % Goldish
    \node[below, rotate=45] at (5.25,0) {Gold ($\pm5\%$)};
\end{tikzpicture}
\captionof{figure}{રેસિસ્ટર કલર બેન્ડ્સ}
\end{center}
\end{solutionbox}

\begin{mnemonicbox}
\mnemonic{BAND: "Beginning digits, Amplify with Multiplier, Note tolerance with last band, Decode carefully"}
\end{mnemonicbox}

\questionmarks{1(ક OR)}{7}{એલ્યુમિનિયમ ઇલેક્ટ્રોલિટીક વેટ ટાઇપ કેપેસિટર સમજાવો.}

\begin{solutionbox}
\textbf{જવાબ}:

\textbf{એલ્યુમિનિયમ ઇલેક્ટ્રોલિટીક વેટ ટાઇપ કેપેસિટર}:

\textbf{બંધારણ}:
\begin{itemize}
    \item \keyword{પ્લેટ્સ}: બે એલ્યુમિનિયમ ફોઇલ્સ (એનોડ અને કેથોડ).
    \item \keyword{ડાયલેક્ટ્રિક}: એનોડ ફોઇલ પર એલ્યુમિનિયમ ઓક્સાઇડ લેયર.
    \item \keyword{ઇલેક્ટ્રોલાઇટ}: લિક્વિડ ઇલેક્ટ્રોલાઇટ (બોરિક એસિડ, સોડિયમ બોરેટ વગેરે).
    \item \keyword{સેપરેટર}: ઇલેક્ટ્રોલાઇટમાં પલાળેલ પેપર સેપરેટર.
    \item \keyword{એન્ક્લોઝર}: રબર સીલ સાથેનું એલ્યુમિનિયમ કેન.
\end{itemize}

\textbf{કાર્યપ્રણાલી}:
\begin{itemize}
    \item \keyword{ઓક્સાઇડ લેયર}: પાતળી એલ્યુમિનિયમ ઓક્સાઇડ લેયર ડાયલેક્ટ્રિક તરીકે કામ કરે છે.
    \item \keyword{ઇલેક્ટ્રોલાઇટ}: બીજી પ્લેટ સાથે કેથોડ કનેક્શન તરીકે કાર્ય કરે છે.
    \item \keyword{પોલરાઇઝેશન}: નિર્ધારિત ધ્રુવીયતા (+ અને -) ટર્મિનલ્સ ધરાવે છે.
\end{itemize}

\textbf{લાક્ષણિકતાઓ}:
\begin{itemize}
    \item \keyword{કેપેસીટન્સ રેન્જ}: 1$\mu$F થી 47,000$\mu$F
    \item \keyword{વોલ્ટેજ રેટીંગ}: 6.3V થી 450V
    \item \keyword{પોલરાઇઝેશન}: પોલરાઇઝ્ડ (સાચું કનેક્શન જરૂરી છે)
\end{itemize}

\textbf{ડાયાગ્રામ}:

\begin{center}
\begin{tikzpicture}
    % Capacitor Can
    \draw[thick] (0,0) rectangle (4,4);
    \node at (2, 4.3) {Aluminum Can};
    
    % Layers (Rolled representation)
    \draw[thick] (0.5, 0.5) rectangle (3.5, 3.5);
    \draw[fill=gray!20] (0.5, 0.5) rectangle (1, 3.5); \node[rotate=90] at (0.75, 2) {Anode (+)};
    \draw[fill=blue!10] (1, 0.5) rectangle (1.5, 3.5); \node[rotate=90] at (1.25, 2) {Separator};
    \draw[fill=gray!40] (1.5, 0.5) rectangle (2, 3.5); \node[rotate=90] at (1.75, 2) {Cathode (-)};
    \draw[fill=blue!10] (2, 0.5) rectangle (2.5, 3.5); \node[rotate=90] at (2.25, 2) {Electrolyte};
    
    % Terminals
    \draw[thick] (0.75, 3.5) -- (0.75, 4.5);
    \node[above] at (0.75, 4.5) {+};
    \draw[thick] (1.75, 3.5) -- (1.75, 4.5);
    \node[above] at (1.75, 4.5) {-};

    \node[align=center] at (3, 2) {Layered\\Structure};
\end{tikzpicture}
\captionof{figure}{એલ્યુમિનિયમ ઇલેક્ટ્રોલિટીક કેપેસિટર}
\end{center}
\end{solutionbox}

\begin{mnemonicbox}
\mnemonic{POLE: "Polarized, Oxide layer, Liquid electrolyte, Enormous capacitance"}
\end{mnemonicbox}

\questionmarks{2(અ)}{3}{શોટકી ડાયોડ, LED અને ફોટો-ડાયોડના સંજ્ઞા દોરો.}

\begin{solutionbox}
\textbf{જવાબ}:

\textbf{સંજ્ઞાઓ}:

\begin{center}
\begin{tikzpicture}
    % Schottky
    \draw (0,0) to[Schottky diode, l=Schottky] (2,0);
    
    % LED
    \draw (3,0) to[led, l=LED] (5,0);
    
    % Photo-diode
    \draw (6,0) to[photodiode, l=Photo-diode] (8,0);
\end{tikzpicture}
\captionof{figure}{ડાયોડ સંજ્ઞાઓ}
\end{center}

\textbf{મુખ્ય લક્ષણો}:
\begin{itemize}
    \item \keyword{શોટકી ડાયોડ (Schottky Diode)}: સ્ટાન્ડર્ડ ડાયોડ સંજ્ઞા સાથે વક્ર બાર (જે મેટલ-સેમિકંડક્ટર જંક્શન દર્શાવે છે).
    \item \keyword{LED}: સ્ટાન્ડર્ડ ડાયોડ સંજ્ઞા સાથે બહાર તરફ પોઈન્ટ કરતા બે તીર (જે પ્રકાશ ઉત્સર્જન દર્શાવે છે).
    \item \keyword{ફોટો-ડાયોડ (Photo-diode)}: સ્ટાન્ડર્ડ ડાયોડ સંજ્ઞા સાથે ડાયોડ તરફ પોઈન્ટ કરતા બે તીર (જે પ્રકાશ શોષણ દર્શાવે છે).
\end{itemize}
\end{solutionbox}

\begin{mnemonicbox}
\mnemonic{SLP: "Schottky has curve, LED emits, Photo-diode absorbs"}
\end{mnemonicbox}

\questionmarks{2(બ)}{4}{ઉદાહરણ સાથે એક્ટિવ અને પેસીવ કમ્પોનન્ટને વ્યાખ્યાયિત કરો.}

\begin{solutionbox}
\textbf{જવાબ}:

\textbf{પેસીવ કમ્પોનન્ટ્સ (Passive Components)}:
\begin{center}
\captionof{table}{પેસીવ કમ્પોનન્ટ્સ}
\begin{tabulary}{\linewidth}{|L|L|L|}
\hline
\textbf{લાક્ષણિકતા} & \textbf{વર્ણન} & \textbf{ઉદાહરણો} \\ \hline
પાવર & પાવર જનરેટ કરી શકતા નથી & રેસિસ્ટર્સ, કેપેસિટર્સ, ઇન્ડક્ટર્સ \\ \hline
સિગ્નલ & સિગ્નલને એમ્પલિફાય કરી શકતા નથી & ટ્રાન્સફોર્મર્સ, ડાયોડ્સ \\ \hline
શેર (Control) & કરંટ પ્રવાહ પર કોઈ નિયંત્રણ નથી & કનેક્ટર્સ, સ્વિચેસ \\ \hline
ઊર્જા & ઊર્જા સંગ્રહ અથવા વપરાશ કરે છે & ફ્યુઝ, ફિલ્ટર્સ \\ \hline
\end{tabulary}
\end{center}

\textbf{એક્ટિવ કમ્પોનન્ટ્સ (Active Components)}:
\begin{center}
\captionof{table}{એક્ટિવ કમ્પોનન્ટ્સ}
\begin{tabulary}{\linewidth}{|L|L|L|}
\hline
\textbf{લાક્ષણિકતા} & \textbf{વર્ણન} & \textbf{ઉદાહરણો} \\ \hline
પાવર & પાવર જનરેટ કરી શકે છે & ટ્રાન્ઝિસ્ટર્સ, ICs \\ \hline
સિગ્નલ & સિગ્નલને એમ્પલિફાય કરી શકે છે & એમ્પલિફાયર્સ, Op-amps \\ \hline
નિયંત્રણ & કરંટ પ્રવાહને નિયંત્રિત કરે છે & SCRs, MOSFETs \\ \hline
નિર્ભરતા & બાહ્ય પાવરની જરૂર પડે છે & વોલ્ટેજ રેગ્યુલેટર્સ \\ \hline
\end{tabulary}
\end{center}

\textbf{ડાયાગ્રામ}:

\begin{center}
\begin{tikzpicture}[node distance=1.5cm]
    \node [gtu block] (Main) {ઇલેક્ટ્રોનિક કમ્પોનન્ટ્સ};
    \node [gtu block, below left=1.0cm of Main] (Active) {એક્ટિવ કમ્પોનન્ટ્સ\\(નિયંત્રણ/એમ્પલિફિકેશન)};
    \node [gtu block, below right=1.0cm of Main] (Passive) {પેસીવ કમ્પોનન્ટ્સ\\(સંગ્રહ/વપરાશ)};
    
    \node [gtu state, below=0.5cm of Active] (ActEx) {Transistor, IC, SCR};
    \node [gtu state, below=0.5cm of Passive] (PasEx) {Resistor, Capacitor, Inductor};
    
    \path [gtu arrow] (Main) -- (Active);
    \path [gtu arrow] (Main) -- (Passive);
    \path [gtu arrow] (Active) -- (ActEx);
    \path [gtu arrow] (Passive) -- (PasEx);
\end{tikzpicture}
\captionof{figure}{કમ્પોનન્ટ્સનું વર્ગીકરણ}
\end{center}
\end{solutionbox}

\begin{mnemonicbox}
\mnemonic{PASS-ACT: "Passive stores or dissipates, Active controls or amplifies"}
\end{mnemonicbox}

\questionmarks{2(ક)}{7}{ફુલ વેવ બ્રિજ રેક્ટિફાયરની કાર્યપદ્ધતી સમજાવો.}

\begin{solutionbox}
\textbf{જવાબ}:

\textbf{ફુલ વેવ બ્રિજ રેક્ટિફાયર}:

\textbf{સર્કિટ બંધારણ}:
\begin{itemize}
    \item \keyword{ડાયોડ્સ}: બ્રિજ કોન્ફિગરેશનમાં ગોઠવાયેલા ચાર ડાયોડ્સ (D1-D4).
    \item \keyword{ઇનપુટ}: ટ્રાન્સફોર્મર સેકન્ડરીથી AC સપ્લાય.
    \item \keyword{આઉટપુટ}: ફિલ્ટર કેપેસિટર સાથે લોડ રેસિસ્ટર પર પલ્સેટિંગ DC.
\end{itemize}

\textbf{કાર્યપ્રણાલી}:
\begin{itemize}
    \item \keyword{પોઝિટિવ હાફ સાયકલ}: D1 અને D3 કન્ડક્ટ કરે છે, D2 અને D4 બ્લોક કરે છે. લોડ દ્વારા પ્રવાહ વહે છે.
    \item \keyword{નેગેટિવ હાફ સાયકલ}: D2 અને D4 કન્ડક્ટ કરે છે, D1 અને D3 બ્લોક કરે છે. લોડ દ્વારા સમાન દિશામાં પ્રવાહ વહે છે.
\end{itemize}

\textbf{પેરામીટર}:
\begin{itemize}
    \item \keyword{રિપલ ફ્રિક્વન્સી}: $2 \times$ ઇનપુટ ફ્રિક્વન્સી.
    \item \keyword{કાર્યક્ષમતા}: 81.2\%.
    \item \keyword{PIV}: $V_m$.
\end{itemize}

\textbf{ડાયાગ્રામ}:

\begin{center}
\begin{tikzpicture}
    % Circuit
    % Transformer
    \draw (0,0) node[transformer core] (T) {};
    
    \node[draw, minimum width=2cm, minimum height=2cm, rotate=45] (Bridge) at (4,0) {};
    \node at (4,0.5) {D1}; \node at (5,0) {D2};
    \node at (4,-0.5) {D3}; \node at (3,0) {D4};
    
    % Connecitons
    \draw (T.A1) -- (3,0); % Left point
    \draw (T.A2) to[jump crossing] (4.5,-1) -- (5,0); % Right point
    
    % DC Out
    \draw (4, 1.414) -- (4, 2) -- (7, 2); % Top point (DC+)
    \draw (4, -1.414) -- (4, -2) -- (7, -2); % Bottom point (DC-)
    
    % Load
    \draw (7, 2) to[R, l=$R_L$] (7, -2);
    
    % Filter Cap
    \draw (6, 2) to[C, l=C] (6, -2);
    
    % Waveforms
    \draw[->] (8, 1) -- (9, 1);
    \draw (8, 1.5) sin (8.25, 2) cos (8.5, 1.5) sin (8.75, 1) cos (9, 1.5); % Output (DC)
    \node[right] at (9,1.5) {આઉટપુટ};

\end{tikzpicture}
\captionof{figure}{ફુલ વેવ બ્રિજ રેક્ટિફાયર}
\end{center}
\end{solutionbox}

\begin{mnemonicbox}
\mnemonic{BRIDGE: "Better Rectification with Improved Diode Geometry Efficiency"}
\end{mnemonicbox}

\questionmarks{2(અ OR)}{3}{LED નું બંધારણ અને કાર્ય સમજાવો.}

\begin{solutionbox}
\textbf{જવાબ}:

\textbf{LED (Light Emitting Diode)}:

\textbf{બંધારણ}:
\begin{itemize}
    \item \keyword{સામગ્રી}: સેમિકંડક્ટર (GaAs, GaP) P-N જંક્શન.
    \item \keyword{પેકેજ}: પારદર્શક એપોક્સી લેન્સ.
    \item \keyword{લીડ્સ}: એનોડ (લાંબો) અને કેથોડ (ટૂંકો).
\end{itemize}

\textbf{કાર્યપ્રણાલી}:
\begin{itemize}
    \item \keyword{બાયસ}: ફોરવર્ડ બાયસ.
    \item \keyword{સિદ્ધાંત}: ઇલેક્ટ્રોન-હોલ રીકોમ્બિનેશન દરમિયાન ફોટોન (પ્રકાશ) સ્વરૂપે ઊર્જા મુક્ત થાય છે.
\end{itemize}

\textbf{ડાયાગ્રામ}:

\begin{center}
\begin{tikzpicture}
    \draw[thick] (0,0) arc (180:0:1);
    \draw[thick] (0,0) -- (0,-1) -- (2,-1) -- (2,0);
    \node at (1, 0.5) {Epoxy Lens};
    
    \draw[thick] (0.5, -1) -- (0.5, -2); \node[below] at (0.5, -2) {Anode (+)};
    \draw[thick] (1.5, -1) -- (1.5, -2); \node[below] at (1.5, -2) {Cathode (-)};
    
    % Light
    \foreach \angle in {45, 60, 120, 135}
        \draw[->, decorate, decoration={snake}] (1, 0.5) -- ++(\angle:1);
\end{tikzpicture}
\captionof{figure}{LED બંધારણ}
\end{center}
\end{solutionbox}

\begin{mnemonicbox}
\mnemonic{LEDS: "Light Emits During electron-hole recombination in Semiconductor"}
\end{mnemonicbox}

\questionmarks{2(બ OR)}{4}{કોમ્પોસીશન ટાઈપ રિસિસ્ટર સમજાવો.}

\begin{solutionbox}
\textbf{જવાબ}:

\textbf{કોમ્પોસીશન રિસિસ્ટર્સ (Composition Resistors)}:

\textbf{બંધારણ}:
\begin{itemize}
    \item \keyword{કોર સામગ્રી}: ઇન્સ્યુલેટિંગ સામગ્રી (માટી/સિરામિક) સાથે મિશ્ર કરેલા કાર્બન કણો.
    \item \keyword{બાઈન્ડર}: રેઝિન બાઈન્ડર જે નળાકાર આકાર આપે છે.
    \item \keyword{રક્ષણ}: ઇન્સ્યુલેટિંગ પેઇન્ટ અથવા પ્લાસ્ટિકનું કોટિંગ.
\end{itemize}

\textbf{લાક્ષણિકતાઓ}:
\begin{itemize}
    \item \keyword{કિંમત}: ઓછી કિંમત.
    \item \keyword{અવાજ}: ઉચ્ચ અવાજ (Noise).
    \item \keyword{સ્થિરતા}: ઓછી સ્થિરતા.
\end{itemize}

\textbf{ડાયાગ્રામ}:

\begin{center}
\begin{tikzpicture}
    \draw[fill=gray!30] (0,0) rectangle (4,1.5);
    \draw[fill=black!80] (0.5, 0.2) rectangle (3.5, 1.3);
    \node[white] at (2, 0.75) {Carbon Composition};
    
    \draw[thick] (-1, 0.75) -- (0, 0.75);
    \draw[thick] (4, 0.75) -- (5, 0.75);
    \node[below] at (-0.5, 0.75) {Lead};
    \node[below] at (4.5, 0.75) {Lead};
\end{tikzpicture}
\captionof{figure}{કાર્બન કોમ્પોસીશન રિસિસ્ટર}
\end{center}
\end{solutionbox}

\begin{mnemonicbox}
\mnemonic{CCRI: "Carbon Composition Resistors are Inexpensive"}
\end{mnemonicbox}

\questionmarks{2(ક OR)}{7}{બે ડાયોડ - ફુલ વેવ રેક્ટિફાયરની કાર્યપદ્ધતી સમજાવો.}

\begin{solutionbox}
\textbf{જવાબ}:

\textbf{બે ડાયોડ ફુલ વેવ રેક્ટિફાયર (સેન્ટર-ટેપ)}:

\textbf{સર્કિટ બંધારણ}:
\begin{itemize}
    \item \keyword{ટ્રાન્સફોર્મર}: સેન્ટર-ટેપ સેકન્ડરી ટ્રાન્સફોર્મર.
    \item \keyword{ડાયોડ્સ}: બે ડાયોડ્સ (D1, D2).
    \item \keyword{આઉટપુટ}: સેન્ટર ટેપ અને કેથોડ જંક્શન વચ્ચે.
\end{itemize}

\textbf{કાર્યપ્રણાલી}:
\begin{itemize}
    \item \keyword{પોઝિટિવ હાફ સાયકલ}: D1 કન્ડક્ટ કરે છે, D2 બ્લોક કરે છે.
    \item \keyword{નેગેટિવ હાફ સાયકલ}: D2 કન્ડક્ટ કરે છે, D1 બ્લોક કરે છે.
    \item \keyword{પરિણામ}: લોડમાં હંમેશા એક જ દિશામાં કરંટ વહે છે.
\end{itemize}

\textbf{ડાયાગ્રામ}:

\begin{center}
\begin{tikzpicture}[scale=0.9]
    \draw (0,0) node[transformer core] (T) {};
    \draw (T.A1) -- ++(1,0) to[diode, l=D1] ++(2,0) -- ++(0,-1) coordinate (A);
    \draw (T.A2) -- ++(1,0) to[diode, l=D2] ++(2,0) -- ++(0,1) coordinate (B);
    \draw (A) -- (B);
    \draw (A) -- ++(1,0) to[R, l=$R_L$] ++(0,-2) coordinate (G);
    \coordinate (CT) at ($(T.A1)!0.5!(T.A2)$); 
    \draw (CT) -- ++(2,0) -- ++(0,0) |- (G);
    \node[ground] at (G) {};
\end{tikzpicture}
\captionof{figure}{સેન્ટર-ટેપ ફુલ વેવ રેક્ટિફાયર}
\end{center}
\end{solutionbox}

\begin{mnemonicbox}
\mnemonic{CTFWR: "Center Tap Facilitates Whole-cycle Rectification"}
\end{mnemonicbox}

\questionmarks{3(અ)}{3}{શોટકી ડાયોડની કાર્યપદ્ધતી સમજાવો.}

\begin{solutionbox}
\textbf{જવાબ}:

\textbf{શોટકી ડાયોડ}:

\begin{itemize}
    \item \keyword{જંક્શન}: મેટલ-સેમિકંડક્ટર (Metal-Semiconductor) જંક્શન.
    \item \keyword{કેરિયર્સ}: મેજોરિટી કેરિયર ડિવાઇસ (ઇલેક્ટ્રોન્સ).
    \item \keyword{ફોરવર્ડ વોલ્ટેજ}: ખૂબ ઓછું (0.2-0.4V).
    \item \keyword{સ્વિચિંગ}: ખૂબ ઝડપી સ્વિચિંગ સ્પીડ (Fast switching).
    \item \keyword{ઉપયોગ}: હાઈ-ફ્રિક્વન્સી એપ્લિકેશન્સ.
\end{itemize}

\textbf{ડાયાગ્રામ}:

\begin{center}
\begin{tikzpicture}
    \draw[thick] (0,0) rectangle (1.5,2);
    \node at (0.75, 1) {Metal};
    \draw[thick] (1.5,0) rectangle (3,2);
    \node at (2.25, 1) {N-Type};
    
    \draw[dashed] (1.5, 0) -- (1.5, 2);
    \node[above] at (1.5, 2) {Junction};
\end{tikzpicture}
\captionof{figure}{શોટકી ડાયોડ સ્ટ્રક્ચર}
\end{center}
\end{solutionbox}

\begin{mnemonicbox}
\mnemonic{SFAM: "Schottky's Fast And Metal-based"}
\end{mnemonicbox}

\questionmarks{3(બ)}{4}{N ટાઈપ સેમિકંડક્ટર સમજાવો.}

\begin{solutionbox}
\textbf{જવાબ}:

\textbf{N-ટાઈપ સેમિકંડક્ટર}:
\begin{itemize}
    \item \keyword{ડોપિંગ}: શુદ્ધ સિલિકોનમાં પેન્ટાવેલન્ટ (Pentavalent) અશુદ્ધિ (જેમ કે ફોસ્ફરસ, આર્સેનિક) ઉમેરવામાં આવે છે.
    \item \keyword{પરિણામ}: દરેક અશુદ્ધિ પરમાણુ એક વધારાનો ઇલેક્ટ્રોન આપે છે.
    \item \keyword{કેરિયર્સ}: ઇલેક્ટ્રોન્સ (મેજોરિટી), હોલ્સ (માઇનોરિટી).
    \item \keyword{ચાર્જ}: એકંદરે તટસ્થ (Neutral) હોય છે.
\end{itemize}

\textbf{ડાયાગ્રામ}:

\begin{center}
\begin{tikzpicture}
    \node[draw, circle] (Si1) at (0,0) {Si};
    \node[draw, circle] (Si2) at (2,0) {Si};
    \node[draw, circle, fill=gray!20] (P) at (1,1.5) {P};
    
    \draw (Si1) -- (P); \draw (Si2) -- (P);
    
    \node[circle, fill=red, inner sep=1pt] (e) at (1.5, 2) {};
    \node[right] at (e) {મુક્ત ઇલેક્ટ્રોન};
    
    \node[below] at (1, -0.5) {N-Type લેટીસ};
\end{tikzpicture}
\captionof{figure}{N-Type ડોપિંગ}
\end{center}
\end{solutionbox}

\begin{mnemonicbox}
\mnemonic{PENT: "Pentavalent Element makes N-Type with free electrons"}
\end{mnemonicbox}

\questionmarks{3(ક)}{7}{PN જંક્શન ડાયોડનું બંધારણ અને કાર્ય સમજાવો.}

\begin{solutionbox}
\textbf{જવાબ}:

\textbf{બંધારણ}:
\begin{itemize}
    \item P-ટાઈપ અને N-ટાઈપ સેમિકંડક્ટરના સંયોજનથી બને છે.
    \item જંક્શન પર ડિપ્લેશન લેયર (Depletion Layer) રચાય છે.
\end{itemize}

\textbf{કાર્યપ્રણાલી}:
\begin{itemize}
    \item \keyword{ફોરવર્ડ બાયસ}: વોલ્ટેજ $>$ બેરિયર પોટેન્શિયલ (0.7V for Si). ડિપ્લેશન રીજન સાંકડો થાય છે અને કરંટ વહે છે.
    \item \keyword{રિવર્સ બાયસ}: ડિપ્લેશન રીજન પહોળો થાય છે. કરંટ બ્લોક થાય છે (લીકેજ સિવાય).
\end{itemize}

\textbf{ડાયાગ્રામ}:

\begin{center}
\begin{tikzpicture}
    \draw[thick] (0,0) rectangle (4,2);
    \draw[thick] (2,0) -- (2,2);
    \node at (1,1) {P-Type};
    \node at (3,1) {N-Type};
    \draw[pattern=north east lines] (1.8,0) rectangle (2.2,2);
    \node[above] at (2,2) {Depletion Region};
    
    \draw (0,1) -- (-1,1) node[left] {Anode};
    \draw (4,1) -- (5,1) node[right] {Cathode};
\end{tikzpicture}
\captionof{figure}{PN જંક્શન}
\end{center}
\end{solutionbox}

\begin{mnemonicbox}
\mnemonic{BIRD: "Barrier forms at Interface, Rectifies Direct current"}
\end{mnemonicbox}

\questionmarks{3(અ OR)}{3}{ફોટો ડાયોડની કાર્યપદ્ધતી સમજાવો.}

\begin{solutionbox}
\textbf{જવાબ}:

\textbf{કાર્યપદ્ધતી}:

\begin{itemize}
    \item \keyword{ઓપરેશન}: હંમેશા રિવર્સ બાયસ (Reverse Bias) સ્થિતિમાં કાર્ય કરે છે.
    \item \keyword{ડાર્ક કરંટ}: જ્યારે પ્રકાશ ન હોય ત્યારે ખૂબ ઓછો પ્રવાહ (Dark Current) વહે છે.
    \item \keyword{પ્રકાશ આપાત}: જ્યારે જંક્શન પર પ્રકાશ પડે છે, ત્યારે કોવેલેન્ટ બોન્ડ તૂટે છે.
    \item \keyword{કેરિયર જનરેશન}: ઇલેક્ટ્રોન-હોલ જોડી ઉત્પન્ન થાય છે.
    \item \keyword{ફોટોકરંટ}: રિવર્સ કરંટ પ્રકાશની તીવ્રતાના સમપ્રમાણમાં વધે છે.
\end{itemize}

\textbf{ડાયાગ્રામ}:

\begin{center}
\begin{tikzpicture}
    \draw[thick] (0,0) rectangle (4,2);
    
    \draw[thick] (2,0) -- (2,2);
    \node at (1,1) {P-Type};
    \node at (3,1) {N-Type};
    \draw[pattern=dots] (1.5,0) rectangle (2.5,2);
    
    % Light
    \draw[->, thick, decorate, decoration={snake}] (1, 3) -- (2, 2);
    \draw[->, thick, decorate, decoration={snake}] (2, 3) -- (2.5, 2);
    \node at (1.5, 3.2) {પ્રકાશ ($h\nu$)};
    
    % Bias
    \draw (0,1) -- (-1,1) -- (-1, -1) -- (5, -1) -- (5, 1) -- (4,1);
    \draw (-0.5, -1) to[battery1, l=$V_R$] (4.5, -1); % Reverse Bias
    \draw (1.5, -1) to[ammeter, l=$\mu A$] (3.5, -1);
\end{tikzpicture}
\captionof{figure}{ફોટો ડાયોડ ઓપરેશન}
\end{center}
\end{solutionbox}

\begin{mnemonicbox}
\mnemonic{DARK: "Dark current exists, Absorbs photons, Reverse bias, K-urrent increases"}
\end{mnemonicbox}

\questionmarks{3(બ OR)}{4}{P ટાઈપ સેમિકંડક્ટર સમજાવો.}

\begin{solutionbox}
\textbf{જવાબ}:

\textbf{P-ટાઈપ સેમિકંડક્ટર}:

\begin{itemize}
    \item \keyword{ડોપિંગ}: શુદ્ધ સેમિકંડક્ટરમાં ટ્રાઇવેલન્ટ (Trivalent) અશુદ્ધિ (જેમ કે બોરોન, એલ્યુમિનિયમ, ગેલિયમ) ઉમેરવામાં આવે છે.
    \item \keyword{હોલ્સ}: અશુદ્ધિ પરમાણુ હોલ્સ (Holes) ઉત્પન્ન કરે છે.
    \item \keyword{કેરિયર્સ}: હોલ્સ (મેજોરિટી), ઇલેક્ટ્રોન્સ (માઇનોરિટી).
    \item \keyword{ચાર્જ}: એકંદરે તટસ્થ (Neutral) હોય છે.
\end{itemize}

\textbf{ડાયાગ્રામ}:

\begin{center}
\begin{tikzpicture}
    \node[draw, circle] (Si1) at (0,0) {Si};
    \node[draw, circle] (Si2) at (2,0) {Si};
    \node[draw, circle, fill=gray!20] (B) at (1,1.5) {B};
    
    \draw (Si1) -- (B); \draw (Si2) -- (B);
    
    \node[circle, draw, dashed, inner sep=2pt] (h) at (1.2, 1.8) {};
    \node[right] at (h) {હોલ ($h^+$)};
    
    \node[below] at (1, -0.5) {P-Type લેટીસ};
\end{tikzpicture}
\captionof{figure}{P-Type ડોપિંગ}
\end{center}
\end{solutionbox}

\begin{mnemonicbox}
\mnemonic{TRIP: "Trivalent Impurity produces Positive holes"}
\end{mnemonicbox}

\questionmarks{3(ક OR)}{7}{હાફ વેવ અને ફુલ વેવ રેક્ટિફાયરની સરખામણી કરો.}

\begin{solutionbox}
\textbf{જવાબ}:

\begin{center}
\captionof{table}{રેક્ટિફાયર સરખામણી}
\begin{tabulary}{\linewidth}{|L|L|L|L|}
\hline
\textbf{પેરામીટર} & \textbf{હાફ વેવ} & \textbf{સેન્ટર ટેપ} & \textbf{બ્રિજ} \\ \hline
ડાયોડ સંખ્યા & 1 & 2 & 4 \\ \hline
મહત્તમ કાર્યક્ષમતા & 40.6\% & 81.2\% & 81.2\% \\ \hline
રિપલ ફેક્ટર & 1.21 & 0.48 & 0.48 \\ \hline
રિપલ ફ્રિક્વન્સી & $f_{in}$ & $2 f_{in}$ & $2 f_{in}$ \\ \hline
PIV રેટિંગ & $V_m$ & $2 V_m$ & $V_m$ \\ \hline
TUF & 0.287 & 0.693 & 0.812 \\ \hline
આઉટપુટ વોલ્ટેજ & $V_m/\pi$ & $2V_m/\pi$ & $2V_m/\pi$ \\ \hline
ટ્રાન્સફોર્મર & સાદું & સેન્ટર ટેપ & સાદું \\ \hline
\end{tabulary}
\end{center}
\end{solutionbox}

\questionmarks{4(અ)}{3}{PNP અને NPN ટ્રાન્ઝિસ્ટરની સંજ્ઞા અને બંધારણ યોગ્ય નામ નિદેશ સાથે દોરો.}

\begin{solutionbox}
\textbf{જવાબ}:

\textbf{બંધારણ અને સંજ્ઞાઓ}:

\begin{center}
\begin{tikzpicture}
    % NPN Symbol
    \draw (0,2) node[npn, l=NPN] (Q1) {};
    \node[below] at (0, 1) {NPN સંજ્ઞા};
    
    % PNP Symbol
    \draw (4,2) node[pnp, l=PNP] (Q2) {};
    \node[below] at (4, 1) {PNP સંજ્ઞા};
    
    % NPN Construction
    \draw (0,-1) rectangle (3,-3);
    \draw (1,-1) -- (1,-3);
    \draw (2,-1) -- (2,-3);
    \node at (0.5,-2) {E (N)};
    \node at (1.5,-2) {B (P)};
    \node at (2.5,-2) {C (N)};
    \node[below] at (1.5, -3) {NPN માળખું};
    
    % PNP Construction
    \draw (4,-1) rectangle (7,-3);
    \draw (5,-1) -- (5,-3);
    \draw (6,-1) -- (6,-3);
    \node at (4.5,-2) {E (P)};
    \node at (5.5,-2) {B (N)};
    \node at (6.5,-2) {C (P)};
    \node[below] at (5.5, -3) {PNP માળખું};
\end{tikzpicture}
\captionof{figure}{ટ્રાન્ઝિસ્ટર સંજ્ઞા અને બંધારણ}
\end{center}
\end{solutionbox}

\begin{mnemonicbox}
\mnemonic{P-POINT: "PNP Points In, NPN Points Out"}
\end{mnemonicbox}

\questionmarks{4(બ)}{4}{ટ્રાન્ઝિસ્ટર એમ્પ્લીફાયરની કાર્યપદ્ધતી સમજાવો.}

\begin{solutionbox}
\textbf{જવાબ}:

\textbf{કાર્યપદ્ધતી}:

\begin{itemize}
    \item \keyword{બાયસિંગ}: એમિટર-બેઝ જંક્શન ફોરવર્ડ બાયસ, કલેક્ટર-બેઝ જંક્શન રિવર્સ બાયસ (એક્ટિવ રીજન).
    \item \keyword{ઇનપુટ}: બેઝ-એમિટર વચ્ચે નિર્બળ AC સિગ્નલ આપવામાં આવે છે.
    \item \keyword{નિયંત્રણ}: બેઝ કરંટ ($I_B$) માં નાના ફેરફારો કલેક્ટર કરંટ ($I_C$) માં મોટા ફેરફારો કરે છે.
    \item \keyword{ગેઇન}: કરંટ ગેઇન $\beta$ ઊંચો હોય છે.
    \item \keyword{આઉટપુટ}: લોડ રેસિસ્ટર પર એમ્પ્લીફાઇડ વોલ્ટેજ મળે છે.
\end{itemize}

\textbf{ડાયાગ્રામ}:

\begin{center}
\begin{tikzpicture}
    \draw (0,0) node[npn] (Q) {};
    \draw (Q.E) -- (0,-1) node[ground] {};
    
    \draw (Q.B) -- (-1,0); 
    \draw (-1,0) to[sV, l=Input] (-1, -1) node[ground] {};
    \draw (-1, 0) to[R, l=$R_B$] (-3, 0) -- (-3, 2) -- (0, 2); 
    
    \draw (Q.C) to[R, l=$R_L$] (0, 2.5);
    \draw (0, 2.5) -- (2, 2.5) node[right] {$+V_{CC}$};
    
    \draw (Q.C) -- (1, 0) to[C, l=$C_{out}$] (2,0) -- (2.5,0) node[right] {Output};
    
    % Waveforms
    \draw (-2, -1.5) sin (-1.75, -1.25) cos (-1.5, -1.5);
    \node at (-2, -1.8) {ઇનપુટ};
    
    \draw (2, -1.5) sin (2.25, -0.5) cos (2.5, -1.5) sin (2.75, -2.5) cos (3, -1.5);
    \node at (2.5, -1.8) {આઉટપુટ};
\end{tikzpicture}
\captionof{figure}{CE એમ્પ્લીફાયર}
\end{center}
\end{solutionbox}

\questionmarks{4(ક)}{7}{ઝેનર ડાયોડની કાર્યપદ્ધતી સમજાવો.}

\begin{solutionbox}
\textbf{જવાબ}:

\textbf{ઝેનર ડાયોડ}:

\begin{itemize}
    \item \keyword{ઓપરેશન}: રિવર્સ બ્રેકડાઉન રીજનમાં કાર્ય કરવા માટે રચાયેલ છે.
    \item \keyword{ફોરવર્ડ બાયસ}: સામાન્ય ડાયોડની જેમ કાર્ય કરે છે.
    \item \keyword{રિવર્સ બાયસ}: અમુક વોલ્ટેજ ($V_Z$) સુધી કરંટ બ્લોક કરે છે.
    \item \keyword{બ્રેકડાઉન}: $V_Z$ પર, કરંટમાં તીવ્ર વધારો થાય છે (Zener Effect).
    \item \keyword{વોલ્ટેજ રેગ્યુલેશન}: કરંટમાં મોટા ફેરફાર છતાં વોલ્ટેજ અચળ રહે છે.
    \item \keyword{ઉપયોગ}: વોલ્ટેજ રેગ્યુલેટર તરીકે.
\end{itemize}

\textbf{કેરેક્ટરીસ્ટીક્સ}:

\begin{center}
\begin{tikzpicture}
    \draw[->] (-3,0) -- (3,0) node[right] {$V$};
    \draw[->] (0,-3) -- (0,3) node[above] {$I$};
    
    % Forward
    \draw[thick, blue] (0,0) -- (0.7,0) .. controls (0.8,0.1) and (0.9,1) .. (1,2.5);
    
    % Reverse
    \draw[thick, blue] (0,0) -- (-2,0) -- (-2,-2.5);
    \node[below] at (-2,0) {$V_Z$};
    \node[left] at (-2, -1) {બ્રેકડાઉન};
\end{tikzpicture}
\captionof{figure}{ઝેનર ડાયોડ ગ્રાફ}
\end{center}
\end{solutionbox}

\begin{mnemonicbox}
\mnemonic{ZAP: "Zener Always Provides constant voltage"}
\end{mnemonicbox}

\questionmarks{4(અ OR)}{3}{ટ્રાન્ઝિસ્ટરને સ્વીચ તરીકે સમજાવો.}

\begin{solutionbox}
\textbf{જવાબ}:

\textbf{ટ્રાન્ઝિસ્ટર સ્વીચ}:

\begin{itemize}
    \item \keyword{OFF (કટઓફ)}:
        \begin{itemize}
            \item બેઝ કરંટ $I_B=0$.
            \item કટ-ઓફ રીજન.
            \item ઓપન સ્વિચ તરીકે વર્તે છે.
        \end{itemize}
    \item \keyword{ON (સેચ્યુરેશન)}:
        \begin{itemize}
            \item પૂરતો બેઝ કરંટ આપવામાં આવે છે.
            \item સેચ્યુરેશન રીજન.
            \item ક્લોઝ્ડ સ્વિચ તરીકે વર્તે છે ($V_{CE} \approx 0.2V$).
        \end{itemize}
\end{itemize}

\textbf{ડાયાગ્રામ}:

\begin{center}
\begin{tikzpicture}
    \draw (0,0) node[npn] (Q) {};
    \draw (Q.E) node[ground] {};
    \draw (Q.C) to[lamp] (0, 2) node[above] {$V_{CC}$};
    \draw (Q.B) -- (-1,0); 
    \draw (-1, 0) -- (-1, -1) node[below] {Input 0V/5V};
\end{tikzpicture}
\captionof{figure}{ટ્રાન્ઝિસ્ટર સ્વીચ}
\end{center}
\end{solutionbox}

\begin{mnemonicbox}
\mnemonic{CO-SI: "Cut-off is Open, Saturation is Closed"}
\end{mnemonicbox}

\questionmarks{4(બ OR)}{4}{CE એમ્પ્લીફાયરની કેરેક્ટરીસ્ટીક્સ દોરો અને સમજાવો.}

\begin{solutionbox}
\textbf{જવાબ}:

\textbf{લાક્ષણિકતાઓ}:

\begin{enumerate}
    \item \keyword{ઇનપુટ લાક્ષણિકતા}: $V_{BE}$ વિરુદ્ધ $I_B$. ફોરવર્ડ બાયસ ડાયોડ જેવી હોય છે.
    \item \keyword{આઉટપુટ લાક્ષણિકતા}: $V_{CE}$ વિરુદ્ધ $I_C$ (અચળ $I_B$ પર).
        \begin{itemize}
            \item \keyword{કટ-ઓફ}: $I_B=0$, ટ્રાન્ઝિસ્ટર OFF.
            \item \keyword{એક્ટિવ}: ટ્રાન્ઝિસ્ટર એમ્પ્લીફાયર તરીકે કામ કરે છે.
            \item \keyword{સેચ્યુરેશન}: ટ્રાન્ઝિસ્ટર સંપૂર્ણ ON હોય છે.
        \end{itemize}
\end{enumerate}

\textbf{ડાયાગ્રામ}:

\begin{center}
\begin{tikzpicture}[scale=0.8]
    % Output Char
    \draw[->] (0,0) -- (5,0) node[right] {$V_{CE}$};
    \draw[->] (0,0) -- (0,5) node[above] {$I_C$};
    
    \foreach \y/\Ib in {1/10, 2/20, 3/30, 4/40} {
        \draw[thick, blue] (0,0) -- (0.5, \y) -- (5, \y);
        \node[right] at (5, \y) {$I_B=\Ib\mu A$};
    }
    
    \draw[dashed] (0.5, 0) -- (0.5, 5);
    \node[rotate=90] at (0.25, 2.5) {સેચ્યુરેશન};
    
    \node at (2.5, 2.5) {એક્ટિવ રીજન};
    
    \draw[thick, red] (0,0) -- (5,0.1);
    \node[below] at (2.5, 0) {કટ-ઓફ};
\end{tikzpicture}
\captionof{figure}{CE આઉટપુટ કેરેક્ટરીસ્ટીક્સ}
\end{center}
\end{solutionbox}

\questionmarks{4(ક OR)}{7}{વેરેક્ટર ડાયોડની કાર્યપદ્ધતી સમજાવો.}

\begin{solutionbox}
\textbf{જવાબ}:

\textbf{કાર્યપદ્ધતી}:

\begin{itemize}
    \item \keyword{કાર્ય}: વોલ્ટેજ દ્વારા નિયંત્રિત વેરિયેબલ કેપેસિટર (Variable Capacitor) તરીકે વર્તે છે.
    \item \keyword{બાયસ}: હંમેશા રિવર્સ બાયસમાં.
    \item \keyword{સિદ્ધાંત}: રિવર્સ વોલ્ટેજ વધારતા ડિપ્લેશન લેયરની પહોળાઈ વધે છે, જેથી કેપેસીટન્સ ઘટે છે ($C \propto 1/\sqrt{V}$).
    \item \keyword{ઉપયોગ}: ટ્યુનિંગ સર્કિટ્સ (Radio/TV), VCOs.
\end{itemize}

\textbf{ડાયાગ્રામ}:

\begin{center}
\begin{tikzpicture}
    % Symbol
    \draw (0,0) to[varcap, l=Varactor] (2,0);
    
    % Graph
    \draw[->] (3,0) -- (6,0) node[right] {$V_R$};
    \draw[->] (3,0) -- (3,3) node[above] {$C$};
    \draw[thick, blue] (3.2, 2.8) .. controls (3.5, 0.5) .. (6, 0.2);
    \node at (5, 1) {$C \downarrow V \uparrow$};
\end{tikzpicture}
\captionof{figure}{વેરેક્ટર ડાયોડ}
\end{center}
\end{solutionbox}

\begin{mnemonicbox}
\mnemonic{VARY: "Voltage Adjusts Reverse-bias Yielding capacitance"}
\end{mnemonicbox}

\questionmarks{5(અ)}{3}{ટ્રાન્ઝિસ્ટર એમ્પ્લીફાયર માટે એક્ટિવ, સેચ્યુરેશન અને કટ-ઓફ રીજીયનની વ્યાખ્યા આપો.}

\begin{solutionbox}
\textbf{જવાબ}:

\begin{itemize}
    \item \keyword{એક્ટિવ (Active)}: બેઝ-એમિટર જંક્શન ફોરવર્ડ, કલેક્ટર-બેઝ રિવર્સ. (એમ્પ્લીફાયર તરીકે વપરાય છે).
    \item \keyword{સેચ્યુરેશન (Saturation)}: બંને જંક્શન ફોરવર્ડ બાયસ. (ON સ્વિચ તરીકે).
    \item \keyword{કટ-ઓફ (Cut-off)}: બંને જંક્શન રિવર્સ બાયસ. (OFF સ્વિચ તરીકે).
\end{itemize}
\end{solutionbox}

\questionmarks{5(બ)}{4}{જો Ic = 10mA અને Ib = 100$\mu$A તો કરંટ ગેઈન $\alpha$, અને $\beta$ ની કીમત શોધો.}

\begin{solutionbox}
\textbf{જવાબ}:

\textbf{આપેલ}:
$I_C = 10 mA$, $I_B = 100 \mu A = 0.1 mA$.

\textbf{ગણતરી}:
\begin{align*}
\beta &= \frac{I_C}{I_B} = \frac{10}{0.1} = 100 \\
I_E &= I_C + I_B = 10 + 0.1 = 10.1 mA \\
\alpha &= \frac{I_C}{I_E} = \frac{10}{10.1} \approx 0.99
\end{align*}

\textbf{પરિણામ}: $\alpha = 0.99, \beta = 100$.
\end{solutionbox}

\questionmarks{5(ક)}{7}{નાના ઈલેક્ટ્રોનિક્સ ઉદ્યોગોમાં ઈલેક્ટ્રોનિક વેસ્ટ મેનેજમેન્ટની વ્યૂહરચનાઓની ચર્ચા કરો.}

\begin{solutionbox}
\textbf{જવાબ}:

\textbf{વ્યૂહરચનાઓ (Strategies)}:

\begin{enumerate}
    \item \keyword{ઇન્વેન્ટરી મેનેજમેન્ટ}: સાધનોનું આયુષ્ય અને જરૂરિયાતનું યોગ્ય આયોજન.
    \item \keyword{ઘટાડો (Reduce)}: બિનજરૂરી ખરીદી ટાળવી. મોડ્યુલર અપગ્રેડ્સ પસંદ કરવા.
    \item \keyword{પુનઃઉપયોગ (Reuse)}: જૂના સાધનોનો અન્ય કાર્યો માટે ફરીથી ઉપયોગ કરવો.
    \item \keyword{રિસાયકલ (Recycle)}: અધિકૃત રિસાયકલર્સ સાથે ભાગીદારી કરવી.
    \item \keyword{અલગીકરણ (Segregation)}: ઈ-વેસ્ટ માટે અલગ ડબ્બા રાખવા.
    \item \keyword{કર્મચારી તાલીમ}: યોગ્ય નિકાલ માટે કર્મચારીઓને જાગૃત કરવા.
\end{enumerate}

\textbf{ડાયાગ્રામ}:

\begin{center}
\begin{tikzpicture}[node distance=2cm]
    \node [gtu block] (Gen) {ઉત્પાદન};
    \node [gtu block, right=of Gen] (Col) {એકત્રીકરણ};
    \node [gtu block, right=of Col] (Seg) {અલગીકરણ};
    \node [gtu block, below=of Seg] (Rec) {રિસાયક્લિંગ};
    \node [gtu block, left=of Rec] (Ref) {પુનઃઉપયોગ};
    \node [gtu block, left=of Ref] (Dis) {નિકાલ};

    \path [gtu arrow] (Gen) -- (Col);
    \path [gtu arrow] (Col) -- (Seg);
    \path [gtu arrow] (Seg) -- (Rec);
    \path [gtu arrow] (Seg) -- (Ref);
    \path [gtu arrow] (Rec) -- (Dis);
\end{tikzpicture}
\captionof{figure}{ઇ-વેસ્ટ ફ્લો}
\end{center}
\end{solutionbox}

\begin{mnemonicbox}
\mnemonic{3R: "Reduce, Reuse, Recycle"}
\end{mnemonicbox}

\questionmarks{5(અ OR)}{3}{CB, CE અને CC ટ્રાન્ઝિસ્ટરની સરકીટ રૂપરેખાંકન દોરો.}

\begin{solutionbox}
\textbf{જવાબ}:

\begin{center}
\begin{tikzpicture}[scale=0.8]
    % CB
    \draw (0,0) node[npn, l=CB, rotate=270] (Q1) {};
    \node[below] at (0,-1) {Common Base};
    \draw (Q1.B) node[ground] {};
    
    % CE
    \draw (4,0) node[npn, l=CE] (Q2) {};
    \node[below] at (4,-1) {Common Emitter};
    \draw (Q2.E) node[ground] {};
    
    % CC
    \draw (8,0) node[npn, l=CC] (Q3) {};
    \node[below] at (8,-1) {Common Collector};
    \draw (Q3.C) node[ground] {}; 
\end{tikzpicture}
\captionof{figure}{ટ્રાન્ઝિસ્ટર રૂપરેખાંકનો}
\end{center}
\end{solutionbox}

\questionmarks{5(બ OR)}{4}{કરંટ ગેઈન $\alpha$ અને $\beta$ વચ્ચેનો સંબંધ મેળવો.}

\begin{solutionbox}
\textbf{જવાબ}:

\textbf{તારવણી}:

1. ટ્રાન્ઝિસ્ટર કરંટ સમીકરણ:
   \[ I_E = I_C + I_B \]

2. $I_C$ વડે ભાગતા:
   \[ \frac{I_E}{I_C} = 1 + \frac{I_B}{I_C} \implies \frac{1}{\alpha} = 1 + \frac{1}{\beta} \]

3. $\alpha$ માટે ઉકેલતા:
   \[ \alpha = \frac{\beta}{1+\beta} \]

4. $\beta$ માટે ઉકેલતા:
   \[ \frac{1}{\beta} = \frac{1}{\alpha} - 1 = \frac{1-\alpha}{\alpha} \implies \beta = \frac{\alpha}{1-\alpha} \]
\end{solutionbox}

\questionmarks{5(ક OR)}{7}{ઈ-વેસ્ટની વ્યાખ્યા કરો અને ઈલેક્ટ્રોનિક કચરાનો નિકાલ સમજાવો.}

\begin{solutionbox}
\textbf{જવાબ}:

\textbf{ઈ-વેસ્ટ (E-Waste)}: બિનઉપયોગી અથવા નકામા થઈ ગયેલા ઇલેક્ટ્રોનિક ઉપકરણો (જેમ કે કોમ્પ્યુટર, મોબાઈલ, પ્રિન્ટર).

\textbf{નિકાલ પદ્ધતિઓ}:

\begin{enumerate}
    \item \keyword{રિસાયક્લિંગ (Recycling)}: સૌથી શ્રેષ્ઠ પદ્ધતિ. કિંમતી ધાતુઓની પુનઃપ્રાપ્તિ અને પ્લાસ્ટિકનો પુનઃઉપયોગ.
    \item \keyword{ઇન્સિનરેશન (Incineration)}: નિયંત્રિત તાપમાને સળગાવવું. કચરાનું પ્રમાણ ઘટાડે છે પરંતુ વાયુ પ્રદૂષણ કરી શકે છે.
    \item \keyword{લેન્ડફિલ (Landfilling)}: જમીનમાં દાટવું. સૌથી ઓછી પસંદગીની પદ્ધતિ કારણ કે ઝેરી તત્વો જમીનમાં ઉતરે છે.
    \item \keyword{પુનઃઉપયોગ (Reuse)}: સમારકામ કરીને ફરી વાપરવું.
    \item \keyword{એસિડ બાથ}: ધાતુઓ મેળવવા એસિડનો ઉપયોગ (ખતરનાક પદ્ધતિ).
\end{enumerate}

\textbf{ડાયાગ્રામ}:

\begin{center}
\begin{tikzpicture}
    \draw (0,0) -- (4,0) -- (2,3.5) -- cycle;
    \draw (0.5, 0.8) -- (3.5, 0.8);
    \draw (1.1, 1.8) -- (2.9, 1.8);
    \draw (1.6, 2.7) -- (2.4, 2.7);
    
    \node at (2, 0.4) {નિકાલ (Disposal)};
    \node at (2, 1.3) {રિસાયકલ};
    \node at (2, 2.2) {પુનઃઉપયોગ};
    \node at (2, 3) {ઘટાડો};
\end{tikzpicture}
\captionof{figure}{વેસ્ટ મેનેજમેન્ટ પિરામિડ}
\end{center}
\end{solutionbox}

\end{document}

