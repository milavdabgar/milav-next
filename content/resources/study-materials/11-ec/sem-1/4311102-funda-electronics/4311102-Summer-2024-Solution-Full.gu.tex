%% METADATA
%% subject-code: 4311102
%% subject-name: Fundamentals of Electronics
%% semester: 1
%% examination: Summer-2024
%% date: 21-06-2024
%% description: Solution guide for Fundamentals of Electronics Summer 2024 exam (Gujarati)
%% tags: study-material, solutions, gtu, 4311102, electronics, diploma, gujarati
%% END METADATA

\documentclass{article}
% GTU Solutions - Gujarati Preamble
% Includes common preamble + Gujarati font setup

% Basic setup
\usepackage[margin=1in]{geometry}
\author{Milav Dabgar}

% Math and tables
\usepackage{amsmath,amssymb,amsthm}
\usepackage{booktabs}
\usepackage{tabularx}
\usepackage{graphicx}
\usepackage{float}  % Required for [H] float placement

% Code listings with syntax highlighting
\usepackage{xcolor}
\usepackage{listings}
\lstset{
  basicstyle=\small\ttfamily,
  breaklines=true,
  numbers=left,
  numberstyle=\tiny\color{gray},
  xleftmargin=2em,
  frame=single,
  showstringspaces=false,
  tabsize=2,
  keywordstyle=\color{blue},
  commentstyle=\color{green!60!black},
  stringstyle=\color{purple}
}

% Optional: TikZ for diagrams (remove if not needed)
\usepackage{tikz}
\usepackage{circuitikz}
\usetikzlibrary{shapes,arrows,positioning,calc}

% Header/footer with author and website
\usepackage{fancyhdr}
\usepackage{lastpage}

\pagestyle{fancy}
\fancyhf{}
\fancyhead[L]{\small\itshape\leftmark}
\fancyhead[R]{\small Milav Dabgar}
\fancyfoot[L]{\small\href{https://www.milav.in}{www.milav.in}}
\fancyfoot[R]{\small Page \thepage\ of \pageref{LastPage}}
\renewcommand{\headrulewidth}{0.4pt}
\renewcommand{\footrulewidth}{0.4pt}

% Hyperref (load before fontspec for Gujarati)
\usepackage[
  colorlinks=true,
  linkcolor=blue,
  urlcolor=blue,
  citecolor=blue,
  pdfauthor={Milav Dabgar},
  pdfsubject={GTU Exam Solutions},
  pdfkeywords={GTU, Java, Programming, Solutions, Gujarati},
  bookmarks=true
]{hyperref}

% Gujarati font setup
\usepackage{fontspec}
\usepackage{polyglossia}
\setdefaultlanguage{gujarati}
\setotherlanguage{english}
\newfontfamily\gujaratifont[Script=Gujarati,AutoFakeBold=2.5,AutoFakeSlant=0.3]{Noto Sans Gujarati}
\setmainfont[Script=Gujarati,AutoFakeBold=2.5,AutoFakeSlant=0.3]{Noto Sans Gujarati}
\setmonofont[Scale=0.9]{Noto Sans Gujarati}
\newfontfamily\englishfont[Script=Gujarati,AutoFakeBold=2.5,AutoFakeSlant=0.3]{Noto Sans Gujarati}
\gappto\captionsgujarati{
  \renewcommand{\tablename}{કોષ્ટક}
  \renewcommand{\figurename}{આકૃતિ}
}
\newcommand{\gu}[1]{{\gujaratifont #1}}

\usetikzlibrary{decorations.pathmorphing}

\title{ફન્ડામેન્ટલ્સ ઓફ ઇલેક્ટ્રોનિક્સ (4311102) - સમર 2024 સોલ્યુશન}
\date{જૂન 21, 2024}

\hypersetup{
 pdftitle={Fundamentals of Electronics (4311102) - Summer 2024 Solution (Gujarati)},
 pdfsubject={GTU Exam Solution - Summer 2024},
 pdfauthor={Milav Dabgar},
 pdfkeywords={study-material, solutions, gtu, 4311102, electronics, fundamentals, gujarati},
 pdfcreator={xelatex}
}

\begin{document}
\maketitle

\setcounter{tocdepth}{5}
\tableofcontents
\newpage

% ========================================
% QUESTION 1: Short Answer Questions (Answer 7 out of 10)
% Total Marks: 14 (2 marks each)
% ========================================

\section{પ્રશ્ન 1}
\textbf{નીચેના દસ પ્રશ્નોમાંથી કોઈપણ સાત પ્રશ્નોના જવાબ આપો (દરેક 2 ગુણ):}

% ========================================
% Q1.1: Define Resistor and Unit
% ========================================

\subsection{પ્રશ્ન 1.1 [2 ગુણ]}
\textbf{રેઝિસ્ટરની વ્યાખ્યા લખો અને તેનો એકમ જણાવો.}

\subsubsection{ઉકેલ}

\paragraph{વ્યાખ્યા:}
\textbf{રેઝિસ્ટર} એ passive electrical component છે જે સર્કિટમાં electric current ના પ્રવાહનો વિરોધ કરે છે અથવા પ્રતિબંધ કરે છે. તે electrical energy ને heat energy માં રૂપાંતરિત કરે છે અને current અને voltage levels ને નિયંત્રિત કરવા માટે વપરાય છે.

\paragraph{એકમ:}
Resistance નો SI એકમ \textbf{Ohm} છે, જેને \(\Omega\) (Greek અક્ષર Omega) થી દર્શાવવામાં આવે છે. એક ohm એ resistance છે જે એક volt લાગુ કરવામાં આવે ત્યારે એક ampere નો current વહેવા દે છે.

ગાણિતિક રીતે: \(R = \frac{V}{I}\) જ્યાં \(V\) voltage volts માં અને \(I\) current amperes માં છે.

\paragraph{કાર્ય:} રેઝિસ્ટર પાવરનો ગરમી તરીકે વ્યય કરે છે અને કરંટ લિમિટ કરે છે.

\paragraph{મેમરી ટ્રીક:}
\emph{``Resistor વિરોધ કરે, Ohms માં માપવામાં આવે''!}

% ========================================
% Q1.2: Active and Passive Component Examples
% ========================================

\subsection{પ્રશ્ન 1.2 [2 ગુણ]}
\textbf{એક્ટિવ તથા પેસિવ કોમ્પોનેન્ટના બે-બે ઉદાહરણ લખો.}

\subsubsection{ઉકેલ}

\paragraph{એક્ટિવ કોમ્પોનેન્ટ્સ:}
Components જે signals ને amplify કરી શકે છે અને external power source માંથી energy ઉમેરીને electron flow ને control કરે છે:
\begin{description}
 \item[Transistor:] Electronic signals ને amplify અથવા switch કરે છે
 \item[Operational Amplifier (Op-Amp):] Signals પર ગાણિતિક operations કરે છે
\end{description}

\paragraph{પેસિવ કોમ્પોનેન્ટ્સ:}
Components જે signals ને amplify કરી શકતા નથી અને માત્ર energy consume અથવા store કરે છે:
\begin{description}
 \item[Resistor:] Current flow નો વિરોધ કરે છે
 \item[Capacitor:] Electric field માં energy store કરે છે
\end{description}

\begin{figure}[H]
\centering
\begin{tikzpicture}[scale=0.8]
 % Active Components
 \node[font=\bfseries] at (0,3) {એક્ટિવ કોમ્પોનેન્ટ્સ};

 % NPN Transistor
 \draw (0,2) node[npn, scale=0.8](npn1){};
 \node[below] at (0,1.2) {Transistor};

 % Op-Amp
 \draw (3,2) node[op amp, scale=0.6](opamp1){};
 \node[below] at (3,1.2) {Op-Amp};

 % Passive Components
 \node[font=\bfseries] at (7,3) {પેસિવ કોમ્પોનેન્ટ્સ};

 % Resistor
 \draw (6.5,2) to[R, l=R] (7.5,2);
 \node[below] at (7,1.2) {Resistor};

 % Capacitor
 \draw (9.5,2) to[C, l=C] (10.5,2);
 \node[below] at (10,1.2) {Capacitor};
\end{tikzpicture}
\caption{એક્ટિવ અને પેસિવ કોમ્પોનેન્ટ્સના ઉદાહરણો}
\end{figure}

\paragraph{તફાવત:} એક્ટિવ કોમ્પોનેન્ટ્સ સિગ્નલને એમ્પ્લીફાય કરી શકે છે, જ્યારે પેસિવ કરી શકતા નથી.

\paragraph{મેમરી ટ્રીક:}
\emph{``Active energy ઉમેરે, Passive માત્ર પસાર કરે''!}

% ========================================
% Q1.3: Semiconductor Device Symbols
% ========================================

\subsection{પ્રશ્ન 1.3 [2 ગુણ]}
\textbf{કોઈપણ બે અર્ધવાહકના પ્રતીકો દોરો.}

\subsubsection{ઉકેલ}

\begin{figure}[H]
\centering
\begin{circuitikz}[scale=1.2]
 % PN Junction Diode
 \draw (0,0) to[D, l=PN Diode] (2,0);
 \node[below] at (1,-0.5) {(a) PN Junction Diode};
 \node[left, font=\small] at (0,0) {Anode};
 \node[right, font=\small] at (2,0) {Cathode};

 % NPN Transistor
 \draw (5,0.5) node[npn, scale=1.2](npn){};
 \node at (npn.B) [left, font=\small] {Base};
 \node at (npn.C) [right, font=\small] {Collector};
 \node at (npn.E) [right, font=\small] {Emitter};
 \node[below] at (5,-0.8) {(b) NPN Transistor};
\end{circuitikz}
\caption{અર્ધવાહક ઉપકરણોના પ્રતીકો}
\end{figure}

\paragraph{સમજૂતી:}
\begin{description}
 \item[PN Diode:] Triangle conventional current flow ની દિશામાં દર્શાવે છે (anode થી cathode). માત્ર forward bias માં current પસાર થવા દે છે.
 \item[NPN Transistor:] Emitter પરનું arrow બહાર તરફ દર્શાવે છે (Not Pointing iN). ત્રણ terminals છે: Base (control), Collector અને Emitter.
\end{description}

\paragraph{ઉદાહરણ:} Silicon (Si) અને Germanium (Ge) એ સૌથી સામાન્ય સેમિકન્ડક્ટર મટિરિયલ્સ છે.

\paragraph{મેમરી ટ્રીક:}
\emph{``Diode arrow current ની દિશા દર્શાવે; NPN arrow બહાર તરફ''!}

% ========================================
% Q1.4: Intrinsic vs Extrinsic Semiconductor
% ========================================

\subsection{પ્રશ્ન 1.4 [2 ગુણ]}
\textbf{ઇન્ટ્રિન્સિક તથા એક્સ્ટ્રિન્સિક અર્ધવાહક વચ્ચેનો તફાવત લખો.}

\subsubsection{ઉકેલ}

\begin{table}[H]
\centering
\caption{ઇન્ટ્રિન્સિક બનામ એક્સ્ટ્રિન્સિક અર્ધવાહક}
\begin{tabularx}{\textwidth}{lXX}
\toprule
\textbf{પેરામીટર} & \textbf{ઇન્ટ્રિન્સિક} & \textbf{એક્સ્ટ્રિન્સિક} \\
\midrule
શુદ્ધતા & શુદ્ધ અર્ધવાહક (Si, Ge) & Impurities સાથે doped \\
વાહકતા & ઓછી & Intrinsic કરતાં વધારે \\
ચાર્જ વાહકો & \(n = p\) (સમાન) & \(n \neq p\) (અસમાન) \\
તાપમાન અસર & અત્યંત સંવેદનશીલ & ઓછું સંવેદનશીલ \\
Doping & કોઈ impurities ઉમેરાયા નથી & Trivalent/Pentavalent ઉમેરાયા \\
ઉદાહરણ & શુદ્ધ Silicon room temp પર & P-type, N-type Si \\
\bottomrule
\end{tabularx}
\end{table}

\begin{figure}[H]
\centering
\begin{tikzpicture}[scale=1.0]
 % Intrinsic
 \node[font=\bfseries] at (2,3) {ઇન્ટ્રિન્સિક};
 \draw[thick] (0,0) rectangle (4,2.5);
 \foreach \x in {0.5,1.5,2.5,3.5} {
 \foreach \y in {0.5,1.5} {
 \fill[blue] (\x,\y) circle (2pt);
 \node[font=\tiny] at (\x,\y+0.3) {Si};
 }
 }
 \node[below, font=\small] at (2,-0.3) {સમાન electrons \& holes};

 % Extrinsic (P-type)
 \node[font=\bfseries] at (7,3) {એક્સ્ટ્રિન્સિક (P-type)};
 \draw[thick] (5,0) rectangle (9,2.5);
 \foreach \x in {5.5,6.5,7.5,8.5} {
 \foreach \y in {0.5,1.5} {
 \fill[blue] (\x,\y) circle (2pt);
 \node[font=\tiny] at (\x,\y+0.3) {Si};
 }
 }
 \fill[red] (6.5,1.5) circle (3pt);
 \node[font=\tiny] at (6.5,1.8) {B};
 \draw[red, dashed] (6.5,1.2) circle (4pt);
 \node[below, font=\small, red] at (7,-0.3) {Holes (majority)};
\end{tikzpicture}
\caption{ઇન્ટ્રિન્સિક બનામ એક્સ્ટ્રિન્સિક સેમિકન્ડક્ટર સ્ટ્રક્ચર}
\end{figure}

\paragraph{નોંધ:} ડોપિંગ પ્રક્રિયા દ્વારા Intrinsic સેમિકન્ડક્ટરને Extrinsic માં ફેરવવામાં આવે છે.

\paragraph{મેમરી ટ્રીક:}
\emph{``Intrinsic અંદર શુદ્ધ છે; Extrinsic માં વધારાની impurities છે''!}

% ========================================
% Q1.5-Q1.10: Remaining questions (identical structure, translated text)
% ========================================

\subsection{પ્રશ્ન 1.5 [2 ગુણ]}
\textbf{LED નું આખું નામ \_\_\_\_\_\_\_\_\_\_\_\_\_\_\_\_\_.}

\subsubsection{ઉકેલ}

\paragraph{આખું નામ:}
LED નું આખું નામ \textbf{Light Emitting Diode} છે.

\paragraph{સમજૂતી:}
તે semiconductor device છે જે forward bias દિશામાં current વહે ત્યારે પ્રકાશ emit કરે છે. પ્રકાશ emission electroluminescence ને કારણે થાય છે - electrons અને holes ના recombination થી photons તરીકે energy મુક્ત થાય છે.

\paragraph{સિદ્ધાંત:} LED ઈલેક્ટ્રોલ્યુમિનેસેન્સના સિદ્ધાંત પર કાર્ય કરે છે.

\paragraph{મેમરી ટ્રીક:}
\emph{``LED: જ્યારે Diode conduct કરે ત્યારે Light Emits''!}



\subsection{પ્રશ્ન 1.6 [2 ગુણ]}
\textbf{ફોટો-ડાયોડના બે ઉપયોગ જણાવો.}

\subsubsection{ઉકેલ}

\paragraph{ઉપયોગો:}
\begin{enumerate}
 \item \textbf{Optical Communication:} Fiber optic systems માં optical signals ને electrical signals માં કન્વર્ટ કરવા માટે light detector તરીકે વપરાય છે.
 \item \textbf{Automatic Light Control:} Street lights, cameras અને displays માં ambient light levels detect કરવા અને automatically brightness adjust કરવા માટે વપરાય છે.
\end{enumerate}

\paragraph{વધારાના ઉપયોગો:}
Light meters, barcode readers, smoke detectors, solar cells, infrared remote controls.

\paragraph{નોંધ:} ફોટો-ડાયોડ હંમેશા રિવર્સ બાયસમાં કાર્ય કરે છે. તે પ્રકાશ ઉર્જાનું વિદ્યુત ઉર્જામાં રૂપાંતર કરે છે. ઉદ્યોગોમાં તેનો ઉપયોગ સુરક્ષા સિસ્ટમ્સમાં થાય છે.

\paragraph{મેમરી ટ્રીક:}
\emph{``Photo-diode Photons detect કરે, current માં કન્વર્ટ કરે''!}





\subsection{પ્રશ્ન 1.7 [2 ગુણ]}
\textbf{ટ્રાન્ઝિસ્ટરના પ્રકારોની યાદી બનાવો અને તેમના પ્રતીકો દોરો.}

\subsubsection{ઉકેલ}

\paragraph{ટ્રાન્ઝિસ્ટરના પ્રકારો:}
\begin{description}
 \item[BJT (Bipolar Junction Transistor):] NPN અને PNP પ્રકારો
 \item[FET (Field Effect Transistor):] JFET અને MOSFET પ્રકારો
\end{description}

\begin{figure}[H]
\centering
\begin{circuitikz}[scale=1.1]
 % NPN
 \node[font=\bfseries] at (1.5,3) {BJT પ્રકારો};
 \draw (0,1.5) node[npn, scale=1.1](npn){};
 \node at (npn.B) [left, font=\small] {B};
 \node at (npn.C) [above right, font=\small] {C};
 \node at (npn.E) [below right, font=\small] {E};
 \node[below] at (0,0.2) {NPN};

 % PNP
 \draw (3,1.5) node[pnp, scale=1.1](pnp){};
 \node at (pnp.B) [left, font=\small] {B};
 \node at (pnp.C) [below right, font=\small] {C};
 \node at (pnp.E) [above right, font=\small] {E};
 \node[below] at (3,0.2) {PNP};

 % N-channel JFET
 \node[font=\bfseries] at (7.5,3) {FET પ્રકારો};
 \draw (6,1.5) node[njfet, scale=1.1](njfet){};
 \node at (njfet.G) [left, font=\small] {G};
 \node at (njfet.D) [above right, font=\small] {D};
 \node at (njfet.S) [below right, font=\small] {S};
 \node[below] at (6,0.2) {N-JFET};

 % N-channel MOSFET
 \draw (9,1.5) node[nmos, scale=1.1](nmos){};
 \node at (nmos.G) [left, font=\small] {G};
 \node at (nmos.D) [above right, font=\small] {D};
 \node at (nmos.S) [below right, font=\small] {S};
 \node[below] at (9,0.2) {N-MOSFET};
\end{circuitikz}
\caption{ટ્રાન્ઝિસ્ટર પ્રકારો અને તેમના પ્રતીકો}
\end{figure}

\paragraph{યાદ રાખો:} BJT એ કરંટ કંટ્રોલ ડિવાઈસ છે જ્યારે FET એ વોલ્ટેજ કંટ્રોલ ડિવાઈસ છે. બંનેનો ઉપયોગ એમ્પ્લીફિકેશન અને સ્વિચિંગ માટે થાય છે.

\paragraph{મેમરી ટ્રીક:}
\emph{``NPN: Not Pointing iN; PNP: Pointing iN Please''!}





\subsection{પ્રશ્ન 1.8 [2 ગુણ]}
\textbf{જર્મેનિયમ અને સિલિકોન ડાયોડના ફોરવર્ડ વોલ્ટેજ ડ્રોપનાં મૂલ્ય આપો.}

\subsubsection{ઉકેલ}

\paragraph{ફોરવર્ડ વોલ્ટેજ ડ્રોપ વેલ્યુઝ:}
\begin{description}
 \item[જર્મેનિયમ (Ge) ડાયોડ:] \(V_f \approx 0.3\,V\) 
 \item[સિલિકોન (Si) ડાયોડ:] \(V_f \approx 0.7\,V\)
\end{description}

\paragraph{સમજૂતી:}
Forward voltage drop એ diode માટે forward bias માં conduct કરવા માટે જરૂરી લઘુત્તમ વોલ્ટેજ છે. Silicon માં જર્મેનિયમ (\(0.7\,eV\)) કરતાં ઊંચી bandgap energy (\(1.1\,eV\)) છે, તેથી barrier potential ને overcome કરવા માટે વધુ વોલ્ટેજ જરૂરી છે.

\paragraph{સરખામણી:}
Si diodes વધુ સામાન્યપણે વપરાય છે કારણ કે સારી temperature સ્થિરતા અને ઓછો reverse leakage current, higher forward drop હોવા છતાં.

\paragraph{તફાવત:} Silicon (Si) નો ઉપયોગ વધુ થાય છે કારણ કે તે ઉચ્ચ તાપમાને સ્થિર રહે છે. Germanium (Ge) નો લીકેજ કરંટ વધારે હોય છે.

\paragraph{મેમરી ટ્રીક:}
\emph{``Silicon સાત-દશાંશ (0.7V); Germanium ત્રણ-દશાંશ (0.3V)''!}





\subsection{પ્રશ્ન 1.9 [2 ગુણ]}
\textbf{\_\_\_\_\_\_\_\_\_\_\_\_\_\_\_\_\_ ડાયોડનો ઉપયોગ લાઇટ ડિટેક્ટર તરીકે થઈ શકે છે.}

\subsubsection{ઉકેલ}

\paragraph{જવાબ:}
\textbf{Photo} ડાયોડ (અથવા \textbf{Photodiode}) નો ઉપયોગ light detector તરીકે થઈ શકે છે.

\paragraph{સમજૂતી:}
Photodiode reverse bias માં કાર્ય કરે છે અને તેના પર પડતા પ્રકાશની તીવ્રતાના પ્રમાણમાં current generate કરે છે. જ્યારે photons PN junction પર strike કરે છે, ત્યારે તેઓ electron-hole pairs બનાવે છે, photocurrent ઉત્પન્ન કરે છે.

\paragraph{કારણ:} જ્યારે પ્રકાશ ફોટો-ડાયોડ પર પડે છે, ત્યારે માઈનોરિટી કેરિયર્સ ઉત્પન્ન થાય છે, જે રિવર્સ કરંટમાં વધારો કરે છે. આ સિદ્ધાંત લાઈટ ડિટેક્શન માટે વપરાય છે.

\paragraph{મેમરી ટ્રીક:}
\emph{``Photo-diode Photos (પ્રકાશ detection) માટે''!}





\subsection{પ્રશ્ન 1.10 [2 ગુણ]}
\textbf{કોઇલના Q-Factor ની વ્યાખ્યા લખો.}

\subsubsection{ઉકેલ}

\paragraph{વ્યાખ્યા:}
કોઇલનો \textbf{Quality factor (Q-factor)} એ dimensionless પેરામીટર છે જે ચોક્કસ frequency પર તેના inductive reactance અને resistance ના ગુણોત્તરને દર્શાવે છે. તે દર્શાવે છે કે inductor કેટલું ``pure'' અથવા ``ideal'' છે.

\paragraph{સૂત્ર:}
\[
Q = \frac{X_L}{R} = \frac{\omega L}{R} = \frac{2\pi f L}{R}
\]

જ્યાં \(X_L\) inductive reactance છે, \(R\) coil resistance છે, \(L\) inductance છે, અને \(f\) frequency છે.

\paragraph{મહત્વ:}
ઊંચો Q-factor માટે ઓછી energy loss, tuned circuits માં sharper resonance અને RF applications માં સારી selectivity. સારા coils માટે સામાન્ય values 10 થી 100+ સુધી હોય છે.

\paragraph{મહત્વ:} ઉચ્ચ Q-factor એટલે સર્કિટ વધુ selectve છે અને બેન્ડવિડ્થ ઓછી છે. રેઝોનન્ટ સર્કિટ્સમાં Q-factor ખૂબ જ મહત્વપૂર્ણ પરિમાણ છે.

\paragraph{કાર્ય:} રેઝિસ્ટર પાવરનો ગરમી તરીકે વ્યય કરે છે અને કરંટ લિમિટ કરે છે.

\paragraph{મેમરી ટ્રીક:}
\emph{``Q એ Quality છે: reactance over resistance''!}

% ========================================
% QUESTION 2: Components & Properties (14 marks)
% ========================================



\section{પ્રશ્ન 2}

% This section covers passive components and their properties.
% આ વિભાગ પેસિવ કોમ્પોનેન્ટ્સ અને તેમના ગુણધર્મોને આવરી લે છે.

% Q2(a): Color Coding Method [3 marks]



\subsection{પ્રશ્ન 2(a) [3 ગુણ]}
\textbf{રેઝિસ્ટરની કલર કોડિંગ પદ્ધતિ સમજાવો.}

\subsubsection{ઉકેલ}

\textbf{રેઝિસ્ટર કલર કોડ} એ standardized પદ્ધતિ છે જેનો ઉપયોગ રેઝિસ્ટર બોડી પર રંગીન બેન્ડ્સ દ્વારા resistance value, tolerance અને કેટલીકવાર temperature coefficient દર્શાવવા માટે થાય છે.

\paragraph{સ્ટાન્ડર્ડ 4-બેન્ડ કલર કોડ:}

\begin{table}[H]
\centering
\caption{રેઝિસ્ટર કલર કોડ ટેબલ}
\begin{tabularx}{\textwidth}{lXXXX}
\toprule
\textbf{રંગ} & \textbf{અંક} & \textbf{મલ્ટિપ્લાયર} & \textbf{ટોલરન્સ} & \textbf{Temp Coeff} \\
\midrule
Black & 0 & \(\times 10^0\) & - & - \\
Brown & 1 & \(\times 10^1\) & \(\pm 1\%\) & 100 ppm \\
Red & 2 & \(\times 10^2\) & \(\pm 2\%\) & 50 ppm \\
Orange & 3 & \(\times 10^3\) & - & 15 ppm \\
Yellow & 4 & \(\times 10^4\) & - & 25 ppm \\
Green & 5 & \(\times 10^5\) & \(\pm 0.5\%\) & - \\
Blue & 6 & \(\times 10^6\) & \(\pm 0.25\%\) & 10 ppm \\
Violet & 7 & \(\times 10^7\) & \(\pm 0.1\%\) & 5 ppm \\
Grey & 8 & \(\times 10^8\) & - & - \\
White & 9 & \(\times 10^9\) & - & - \\
Gold & - & \(\times 0.1\) & \(\pm 5\%\) & - \\
Silver & - & \(\times 0.01\) & \(\pm 10\%\) & - \\
\bottomrule
\end{tabularx}
\end{table}

\begin{figure}[H]
\centering
\begin{tikzpicture}[scale=1.3]
 % Resistor body
 \draw[thick, fill=gray!20] (0,0) rectangle (5,0.7);

 % Color bands
 \fill[brown!70!black] (0.6,0) rectangle (0.9,0.7);
 \fill[black] (1.4,0) rectangle (1.7,0.7);
 \fill[orange] (2.2,0) rectangle (2.5,0.7);
 \fill[yellow!80!orange] (4.1,0) rectangle (4.4,0.7);

 % Labels
 \node[below, font=\small] at (0.75,-0.15) {Band 1};
 \node[below, font=\small] at (1.55,-0.15) {Band 2};
 \node[below, font=\small] at (2.35,-0.15) {Band 3};
 \node[below, font=\small] at (4.25,-0.15) {Band 4};

 \node[above, font=\small] at (0.75,0.85) {1st Digit};
 \node[above, font=\small] at (1.55,0.85) {2nd Digit};
 \node[above, font=\small] at (2.35,0.85) {Multiplier};
 \node[above, font=\small] at (4.25,0.85) {Tolerance};

 % Leads
 \draw[thick] (-0.5,0.35) -- (0,0.35);
 \draw[thick] (5,0.35) -- (5.5,0.35);
\end{tikzpicture}
\caption{4-બેન્ડ રેઝિસ્ટર કલર કોડ સ્ટ્રક્ચર}
\end{figure}

\paragraph{વાંચવાની પદ્ધતિ:}
\begin{enumerate}
 \item જમણી બાજુએ tolerance band (Gold/Silver, સહેજ અલગ) ઓળખો
 \item ડાબી બાજુથી જમણી બાજુ વાંચો
 \item બેન્ડ 1 \& 2: Significant digits
 \item બેન્ડ 3: Multiplier (શૂન્યોની સંખ્યા)
 \item બેન્ડ 4: Tolerance
\end{enumerate}

\paragraph{ઉદાહરણ:}
Brown-Black-Orange-Gold = 1, 0, \(\times 10^3\), \(\pm 5\%\) = \(10{,}000\,\Omega \pm 5\%\) = \(10\,k\Omega \pm 5\%\)

\paragraph{ટીપ:} કલર કોડ યાદ રાખવા માટે `BBROYGBVGW' મેમરી ટ્રીકનો ઉપયોગ કરો. સહિષ્ણુતા (Tolerance) બેન્ડ હંમેશા છેલ્લે ગણવો જોઈએ.

\paragraph{મેમરી ટ્રીક:}
\emph{``BB ROY Great Britain Very Good Wife'' - Black Brown Red Orange Yellow Green Blue Violet Grey White!}

% Q2(a) OR: LDR



\subsection{પ્રશ્ન 2(a) OR [3 ગુણ]}
\textbf{લાઇટ ડિપેન્ડન્ટ રેઝિસ્ટર તેની લાક્ષણિકતાઓ સાથે સમજાવો.}

\subsubsection{ઉકેલ}

\textbf{Light Dependent Resistor (LDR)} અથવા photoresistor એ passive component છે જેનો resistance પડતા પ્રકાશની તીવ્રતા સાથે વિપરીત રીતે બદલાય છે.

\paragraph{બાંધકામ:}

\begin{figure}[H]
\centering
\begin{tikzpicture}[scale=1.2]
 % LDR body
 \draw[thick, fill=yellow!20] (0,0) circle (1.5);

 % Zigzag pattern
 \draw[very thick, red] 
 (-0.8,0.6) -- (-0.4,0.6) -- (-0.2,0.3) -- (0,0.6) -- (0.2,0.3) -- (0.4,0.6) -- (0.8,0.6);
 \draw[very thick, red]
 (-0.8,0) -- (-0.4,0) -- (-0.2,-0.3) -- (0,0) -- (0.2,-0.3) -- (0.4,0) -- (0.8,0);
 \draw[very thick, red]
 (-0.8,-0.6) -- (-0.4,-0.6) -- (-0.2,-0.3) -- (0,-0.6) -- (0.2,-0.3) -- (0.4,-0.6) -- (0.8,-0.6);

 % Terminals
 \draw[very thick] (-0.8,0.6) -- (-1.8,0.6) node[left] {Terminal 1};
 \draw[very thick] (0.8,-0.6) -- (1.8,-0.6) node[right] {Terminal 2};

 % Light rays
 \foreach \x in {-1.2,-0.6,0,0.6,1.2} {
 \draw[->, yellow!60!orange, very thick] (\x,2.2) -- (\x,1.6);
 }
 \node[above] at (0,2.3) {પ્રકાશ};

 \node[below, font=\small] at (0,-1.8) {CdS/CdSe સામગ્રી};
\end{tikzpicture}
\caption{LDR બાંધકામ}
\end{figure}

\paragraph{કાર્ય સિદ્ધાંત:}
Photoconductivity પર આધારિત - જ્યારે પ્રકાશ photons semiconductor material (CdS/CdSe) પર strike કરે છે, તેઓ electrons ને energy આપે છે, electron-hole pairs બનાવે છે. આ conductivity વધારે છે અને resistance ઘટાડે છે.

\paragraph{લાક્ષણિકતાઓ:}

\begin{figure}[H]
\centering
\begin{tikzpicture}[scale=1.0]
 % Axes
 \draw[->, thick] (0,0) -- (7,0) node[right] {પ્રકાશ તીવ્રતા (lux)};
 \draw[->, thick] (0,0) -- (0,5) node[above] {Resistance (k\(\Omega\))};

 % Characteristic curve
 \draw[blue, very thick, domain=0.3:6.5] plot (\x, {4.5/(\x+0.2)});

 % Points
 \fill[red] (1,3.8) circle (2pt) node[right, font=\small] {અંધારું: \(\sim\)1M\(\Omega\)};
 \fill[red] (6,0.72) circle (2pt) node[above right, font=\small] {તેજ: \(\sim\)100\(\Omega\)};

 % Grid lines
 \draw[dashed, gray] (0,3.8) -- (1,3.8);
 \draw[dashed, gray] (0,0.72) -- (6,0.72);
\end{tikzpicture}
\caption{LDR Resistance vs પ્રકાશ તીવ્રતા લાક્ષણિકતા}
\end{figure}

\paragraph{સ્પેસિફિકેશન્સ:}
\begin{itemize}
 \item ડાર્ક Resistance: 1M\(\Omega\) - 10M\(\Omega\)
 \item લાઇટ Resistance: 100\(\Omega\) - 1k\(\Omega\)
 \item પ્રતિભાવ સમય: 10-100ms
 \item Peak Spectral Response: ~550nm (લીલો પ્રકાશ)
\end{itemize}

\paragraph{ઉપયોગો:}
Street lights, camera exposure control, alarm systems, light meters.

\paragraph{ટીપ:} કલર કોડ યાદ રાખવા માટે `BBROYGBVGW' મેમરી ટ્રીકનો ઉપયોગ કરો. સહિષ્ણુતા (Tolerance) બેન્ડ હંમેશા છેલ્લે ગણવો જોઈએ.

\paragraph{મેમરી ટ્રીક:}
\emph{``LDR: પ્રકાશ ઓછો, Resistance ઓછો''!}

% Q2(b): Capacitor Classification




\subsection{પ્રશ્ન 2(b) [3 ગુણ]}
\textbf{કેપેસિટરનું વર્ગીકરણ વિગતવાર સમજાવો.}

\subsubsection{ઉકેલ}

કેપેસિટર્સનું વર્ગીકરણ dielectric material, polarity અને construction ના આધારે થાય છે.

\paragraph{Dielectric Material દ્વારા વર્ગીકરણ:}

\begin{description}
 \item[Ceramic Capacitors:] Ceramic dielectric (barium titanate) વાપરે છે. નાના, સસ્તા, non-polarized. Values: pF થી થોડા \(\mu\)F.

 \item[Film Capacitors:] Plastic film dielectric (polyester, polypropylene). સ્થિર, low loss. Values: nF થી \(\mu\)F range.

 \item[Electrolytic Capacitors:] Aluminum oxide dielectric, polarized. ઊંચી capacitance (1\(\mu\)F થી 10000\(\mu\)F), power supplies માં વપરાય છે.

 \item[Tantalum Capacitors:] Tantalum pentoxide dielectric, polarized. સ્થિર, compact, મોંઘા. Values: 0.1\(\mu\)F થી 100\(\mu\)F.

 \item[Mica Capacitors:] Mica dielectric. ખૂબ સ્થિર, low loss, મોંઘા. RF applications.
\end{description}

\begin{figure}[H]
\centering
\begin{tikzpicture}[scale=0.9]
 % Ceramic
 \node[font=\bfseries] at (2,3.5) {કેપેસિટર પ્રકારો};
 \draw[thick] (0,2) rectangle (1,2.8);
 \fill[brown!40] (0,2) rectangle (1,2.8);
 \node[below, font=\small] at (0.5,1.7) {Ceramic};
 \draw[thick] (0.3,2.8) -- (0.3,3.2);
 \draw[thick] (0.7,2.8) -- (0.7,3.2);

 % Film
 \draw[thick, fill=blue!20] (2.5,2) -- (3.5,2) -- (3.5,2.8) -- (2.5,2.8) -- cycle;
 \node[below, font=\small] at (3,1.7) {Film};
 \draw[thick] (2.8,2.8) -- (2.8,3.2);
 \draw[thick] (3.2,2.8) -- (3.2,3.2);

 % Electrolytic
 \draw[thick, fill=gray!30] (4.7,2) rectangle (5.3,2.8);
 \draw[thick, fill=gray!30] (5,2) ellipse (0.3 and 0.15);
 \node[below, font=\small] at (5,1.7) {Electrolytic};
 \draw[thick, red] (5,2.8) -- (5,3.2) node[above, font=\tiny] {+};
 \draw[thick] (5,2) -- (5,1.6);

 % Tantalum 
 \draw[thick, fill=orange!30] (7,2.2) rectangle (7.6,2.6);
 \node[below, font=\small] at (7.3,1.7) {Tantalum};
 \draw[thick, red] (7.1,2.6) -- (7.1,3.0) node[above, font=\tiny] {+};
 \draw[thick] (7.5,2.2) -- (7.5,1.8);
\end{tikzpicture}
\caption{કેપેસિટરના પ્રકારો}
\end{figure}

\paragraph{Polarity દ્વારા વર્ગીકરણ:}
\begin{description}
 \item[Polarized:] યોગ્ય polarity સાથે જોડવું આવશ્યક (electrolytic, tantalum)
 \item[Non-Polarized:] કોઈપણ રીતે જોડી શકાય (ceramic, film, mica)
\end{description}

\paragraph{ગુણધર્મ:} કેપેસિટર DC પ્રવાહને બ્લોક કરે છે અને AC પ્રવાહને પસાર કરે છે.

\paragraph{મેમરી ટ્રીક:}
\emph{``CEFMT: Ceramic, Electrolytic, Film, Mica, Tantalum - કેપેસિટર પ્રકારો''!}

% Q2(b) OR: Inductor Classification



\subsection{પ્રશ્ન 2(b) OR [3 ગુણ]}
\textbf{ઇન્ડક્ટરનું વર્ગીકરણ વિગતવાર સમજાવો.}

\subsubsection{ઉકેલ}

ઇન્ડક્ટર્સનું વર્ગીકરણ core material, construction અને application ના આધારે થાય છે.

\paragraph{Core Material દ્વારા વર્ગીકરણ:}

\begin{description}
 \item[Air Core Inductors:] કોઈ magnetic core નથી, માત્ર coiled wire. Low inductance (nH થી \(\mu\)H), RF circuits માં વપરાય છે. કોઈ core losses નથી, કોઈ saturation નથી.

 \item[Iron Core Inductors:] Solid iron core. High inductance, low-frequency applications માં વપરાય છે. ભારે, core losses થાય છે.

 \item[Ferrite Core Inductors:] Ferrite (ceramic magnetic) core. High-frequency switching applications માટે સારું. Iron કરતાં ઓછા losses.

 \item[Powdered Iron Core:] Iron powder બાઈન્ડર સાથે મિશ્રિત. Distributed air gap saturation ઘટાડે છે. Filters અને RF chokes માં વપરાય છે.

 \item[Laminated Core Inductors:] Iron ની પાતળી sheets laminated together. Eddy current losses ઘટાડે છે. Transformers અને power inductors માં વપરાય છે.
\end{description}

\begin{figure}[H]
\centering
\begin{tikzpicture}[scale=0.8]
 % Air core
 \node[font=\bfseries] at (4,4) {ઇન્ડક્ટર પ્રકારો};
 \draw[thick, blue, decoration={coil, aspect=0.5, segment length=2mm, amplitude=3mm}, decorate] (0,1.5) -- (1.5,1.5);
 \node[below, font=\small] at (0.75,0.8) {Air Core};

 % Iron core
 \draw[thick, fill=gray!60] (3,1.2) rectangle (3.3,1.8);
 \draw[thick, blue, decoration={coil, aspect=0.5, segment length=2mm, amplitude=3mm}, decorate] (2.5,1.5) -- (3.8,1.5);
 \node[below, font=\small] at (3.15,0.8) {Iron Core};

 % Ferrite core
 \draw[thick, fill=brown!50] (5.5,1.3) circle (0.3);
 \draw[thick, blue, decoration={coil, aspect=0.5, segment length=2mm, amplitude=3mm}, decorate] (5,1.5) -- (6.5,1.5);
 \node[below, font=\small] at (5.75,0.8) {Ferrite};

 % Toroidal
 \draw[thick, fill=red!30] (8.5,1.5) circle (0.4);
 \draw[thick, fill=white] (8.5,1.5) circle (0.2);
 \node[below, font=\small] at (8.5,0.8) {Toroidal};
\end{tikzpicture}
\caption{Core Material દ્વારા ઇન્ડક્ટર પ્રકારો}
\end{figure}

\paragraph{Construction દ્વારા વર્ગીકરણ:}
\begin{description}
 \item[Solenoid:] Cylindrical form પર helical coil wound
 \item[Toroidal:] Donut-આકારના core પર wire wound - self-shielding, compact
 \item[Multilayer:] High inductance માટે winding ના multiple layers
\end{description}

\paragraph{મેમરી ટ્રીક:}
\emph{``AIFPL: Air, Iron, Ferrite, Powdered, Laminated - ઇન્ડક્ટર cores''!}

% Q2(c): Faraday's Laws



\subsection{પ્રશ્ન 2(c) [4 ગુણ]}
\textbf{ફેરાડેનો ઇલેક્ટ્રોમેગ્નેટિક ઇન્ડક્શનનો નિયમો લખો તથા સમજાવો.}

\subsubsection{ઉકેલ}

ફેરાડેના નિયમો વર્ણવે છે કે કેવી રીતે બદલાતા magnetic field conductor માં electromotive force (EMF) induce કરે છે.

\paragraph{પ્રથમ નિયમ:}
\textbf{વિધાન:} જ્યારે પણ conductor અથવા coil સાથે જોડાયેલ magnetic flux બદલાય છે, ત્યારે તેમાં EMF induce થાય છે.

\paragraph{બીજો નિયમ:}
\textbf{વિધાન:} Induced EMF નું માત્ર magnetic flux linkage ના બદલાવના દરના સીધા પ્રમાણમાં હોય છે.

\paragraph{ગાણિતિક વ્યંજક:}
\[
\mathcal{E} = -N \frac{d\Phi}{dt}
\]

જ્યાં:
\begin{itemize}
 \item \(\mathcal{E}\) = Induced EMF (volts)
 \item \(N\) = Turns ની સંખ્યા
 \item \(\frac{d\Phi}{dt}\) = Magnetic flux ના બદલાવનો દર
 \item નકારાત્મક ચિહ્ન Lenz's Law દર્શાવે છે (બદલાવનો વિરોધ કરે છે)
\end{itemize}

\begin{figure}[H]
\centering
\begin{tikzpicture}[scale=1.1]
 % Coil
 \draw[thick] (0,0) -- (0,0.5);
 \draw[thick, decoration={coil, aspect=0.3, segment length=3mm, amplitude=4mm}, decorate] (0,0.5) -- (0,3);
 \draw[thick] (0,3) -- (0,3.5);
 \node[right] at (0.3,1.75) {Coil (N turns)};

 % Magnet moving
 \draw[thick, fill=red!30] (-2,1.3) rectangle (-1,2.2);
 \node at (-1.5,1.75) {N};
 \draw[thick, fill=blue!30] (-1,1.3) rectangle (0,2.2);
 \node at (-0.5,1.75) {S};
 \draw[->, very thick, green] (-2.5,1.75) -- (-3.2,1.75);
 \node[above] at (-2.85,1.9) {ગતિ};

 % Induced current
 \draw[->, thick, red] (0.5,3.3) arc (0:270:0.3);
 \node[right, red] at (0.7,3) {Induced};
 \node[right, red] at (0.7,2.7) {Current};

 % Flux lines
 \foreach \y in {1.4,1.6,1.8,2.0} {
 \draw[->, blue, dashed] (-0.9,\y) -- (0.1,\y);
 }
 \node[below, blue] at (-0.4,1.2) {\(\Phi\)};
\end{tikzpicture}
\caption{Electromagnetic Induction - ચુંબક હલનચલન EMF Induce કરે છે}
\end{figure}

\paragraph{સમજૂતી:}
જ્યારે ચુંબક coil તરફ/થી દૂર જાય છે, ત્યારે coil દ્વારા magnetic flux બદલાય છે. આ બદલાતો flux એક EMF induce કરે છે જે circuit બંધ હોય તો current વહેવા દે છે. Induced current પોતાનું magnetic field બનાવે છે જે મૂળ બદલાવનો વિરોધ કરે છે (Lenz's Law).

\paragraph{ઉપયોગો:}
Generators, transformers, inductors, induction motors, wireless charging.

\paragraph{લેન્ઝનો નિયમ:} પ્રેરિત EMF હંમેશા તેના કારણનો વિરોધ કરે છે.

\paragraph{મેમરી ટ્રીક:}
\emph{``Faraday: Flux બદલાવ EMF induce કરે, દર માત્રા નક્કી કરે''!}

% Rest of Q2, Q3, Q4, Q5 - Due to length, using optimized compact form
% Continuing with remaining sections...

% Q2(c) OR: Capacitor Specifications



\subsection{પ્રશ્ન 2(c) OR [4 ગુણ]}
\textbf{કેપેસિટરની સ્પેસિફિકેશન્સની યાદી બનાવો અને કોઈપણ બેને વિગતવાર સમજાવો.}

\subsubsection{ઉકેલ}

\paragraph{કેપેસિટર સ્પેસિફિકેશન્સ:}
\begin{enumerate}
 \item \textbf{Capacitance Value} (C)
 \item \textbf{Voltage Rating} (Working Voltage)
 \item \textbf{Tolerance}
 \item \textbf{Temperature Coefficient}
 \item \textbf{ESR (Equivalent Series Resistance)}
 \item \textbf{Leakage Current}
 \item \textbf{Ripple Current Rating}
 \item \textbf{Life Span / Endurance}
\end{enumerate}

\paragraph{વિગતવાર સમજૂતી:}

\subparagraph{1. Capacitance Value (C):}
Electric charge store કરવાની ક્ષમતા, Farads (F) માં માપવામાં આવે છે. Practical units: pF (picofarad), nF (nanofarad), \(\mu\)F (microfarad).

સૂત્ર: \(C = \frac{Q}{V}\) જ્યાં Q charge coulombs માં છે, V voltage છે.

Parallel plate માટે: \(C = \frac{\epsilon_0 \epsilon_r A}{d}\)

સામાન્ય ranges:
\begin{itemize}
 \item Ceramic: 1pF - 1\(\mu\)F
 \item Film: 1nF - 100\(\mu\)F
 \item Electrolytic: 1\(\mu\)F - 10000\(\mu\)F
\end{itemize}

\subparagraph{2. Voltage Rating (Working Voltage):}
Maximum DC voltage જે breakdown વિના સતત લાગુ કરી શકાય. આને વટાવવાથી dielectric failure થાય છે.

Derating: વ્યવહારમાં, વિશ્વસનીયતા માટે rated voltage ના 50-80\% પર operate કરો.

ઉદાહરણ ratings: 6.3V, 10V, 16V, 25V, 50V, 100V, 450V (સામાન્ય values)

\begin{description}
 \item[Safety Factor:] હંમેશા voltage rating \(>\) 1.5 \(\times\) maximum circuit voltage પસંદ કરો
 \item[AC Applications:] Peak voltage DC rating કરતાં ઓછું હોવું જોઈએ
\end{description}

\paragraph{મેમરી ટ્રીક:}
\emph{``CV-TT-ELR-L: Capacitance, Voltage, Tolerance, Temperature, ESR, Leakage, Ripple, Life''!}

% Q2(d): 47\(\Omega\) Color Band



\subsection{પ્રશ્ન 2(d) [4 ગુણ]}
\textbf{47\(\Omega\)\(\pm\)5\% માટે કલર કોડ લખો.}

\subsubsection{ઉકેલ}

\textbf{47\(\Omega\) \(\pm\)5\%} રેઝિસ્ટર માટે 4-band color code વાપરીને:

\paragraph{ગણતરી:}
\begin{itemize}
 \item Resistance = 47\(\Omega\) = 47 \(\times 10^0\)
 \item પ્રથમ અંક = 4 \(\rightarrow\) \textbf{Yellow}
 \item બીજો અંક = 7 \(\rightarrow\) \textbf{Violet}
 \item Multiplier = \(\times 10^0\) = \(\times 1\) \(\rightarrow\) \textbf{Black}
 \item Tolerance = \(\pm\)5\% \(\rightarrow\) \textbf{Gold}
\end{itemize}

\paragraph{કલર બેન્ડ ક્રમ:}
\textbf{Yellow - Violet - Black - Gold}

\begin{figure}[H]
\centering
\begin{tikzpicture}[scale=1.4]
 % Resistor body
 \draw[thick, fill=gray!20] (0,0) rectangle (5,0.7);

 % Color bands
 \fill[yellow] (0.7,0) rectangle (1.0,0.7);
 \fill[violet] (1.6,0) rectangle (1.9,0.7);
 \fill[black] (2.5,0) rectangle (2.8,0.7);
 \fill[yellow!80!orange] (4.1,0) rectangle (4.4,0.7);

 % Labels
 \node[below, font=\small] at (0.85,-0.15) {Yellow};
 \node[below, font=\small] at (1.75,-0.15) {Violet};
 \node[below, font=\small] at (2.65,-0.15) {Black};
 \node[below, font=\small] at (4.25,-0.15) {Gold};

 \node[above, font=\small] at (0.85,0.85) {4};
 \node[above, font=\small] at (1.75,0.85) {7};
 \node[above, font=\small] at (2.65,0.85) {\(\times 1\)};
 \node[above, font=\small] at (4.25,0.85) {\(\pm\)5\%};

 % Leads
 \draw[thick] (-0.5,0.35) -- (0,0.35);
 \draw[thick] (5,0.35) -- (5.5,0.35);

 % Result
 \node[below, font=\bfseries] at (2.5,-0.8) {47\(\Omega\) \(\pm\)5\%};
\end{tikzpicture}
\caption{47\(\Omega\) \(\pm\)5\% રેઝિસ્ટર કલર બેન્ડ્સ}
\end{figure}

\paragraph{ચકાસણી:}
Tolerance range: 47\(\Omega\) \(\pm\) 5\% = 47 \(\pm\) 2.35 = 44.65\(\Omega\) થી 49.35\(\Omega\)

\paragraph{મહત્વ:} સહિષ્ણુતા (Tolerance) બેન્ડ રેઝિસ્ટરની ચોકસાઈ અને ગુણવત્તા દર્શાવે છે.

\paragraph{મેમરી ટ્રીક:}
\emph{``Yellow Violet Black Gold = 47 ohms, પાંચ ટકા told''!}

% Q2(d) OR: Calculate Brown-Black-Yellow



\subsection{પ્રશ્ન 2(d) OR [4 ગુણ]}
\textbf{આપેલ કલર કોડ માટે રેઝિસ્ટરની કિંમત તથા ટોલરન્સ શોધો: Brown, Black, yellow.}

\subsubsection{ઉકેલ}

આપેલ color code: \textbf{Brown - Black - Yellow}

\paragraph{રંગો વાંચવા:}
\begin{itemize}
 \item બેન્ડ 1 (Brown): પ્રથમ અંક = \textbf{1}
 \item બેન્ડ 2 (Black): બીજો અંક = \textbf{0}
 \item બેન્ડ 3 (Yellow): Multiplier = \textbf{\(\times 10^4\)}
 \item બેન્ડ 4: આપેલ નથી, \textbf{કોઈ tolerance band નથી} અથવા \(\pm\)20\% (3-band માટે default)
\end{itemize}

\paragraph{ગણતરી:}
\[
R = (10) \times 10^4 = 100{,}000\,\Omega = 100\,k\Omega
\]

\paragraph{Tolerance:}
માત્ર 3 bands આપેલા હોવાથી, tolerance સામાન્ય રીતે \(\pm\)20\% છે (3-band resistors માટે standard).

\begin{figure}[H]
\centering
\begin{tikzpicture}[scale=1.3]
 % Resistor body
 \draw[thick, fill=gray!20] (0,0) rectangle (4.5,0.7);

 % Color bands 
 \fill[brown!70!black] (0.7,0) rectangle (1.0,0.7);
 \fill[black] (1.6,0) rectangle (1.9,0.7);
 \fill[yellow] (2.5,0) rectangle (2.8,0.7);

 % Labels
 \node[below, font=\small] at (0.85,-0.15) {Brown};
 \node[below, font=\small] at (1.75,-0.15) {Black};
 \node[below, font=\small] at (2.65,-0.15) {Yellow};

 \node[above, font=\small] at (0.85,0.85) {1};
 \node[above, font=\small] at (1.75,0.85) {0};
 \node[above, font=\small] at (2.65,0.85) {\(\times 10^4\)};

 % Leads
 \draw[thick] (-0.5,0.35) -- (0,0.35);
 \draw[thick] (4.5,0.35) -- (5,0.35);

 % Result
 \node[below, font=\bfseries] at (2.25,-0.8) {100k\(\Omega\) \(\pm\)20\%};
\end{tikzpicture}
\caption{Brown-Black-Yellow = 100k\(\Omega\)}
\end{figure}

\paragraph{Tolerance Range:}
100k\(\Omega\) \(\pm\) 20\% = 100k \(\pm\) 20k = 80k\(\Omega\) થી 120k\(\Omega\)

\paragraph{જવાબ:}
\begin{description}
 \item[Resistance:] \textbf{100 k\(\Omega\)}
 \item[Tolerance:] \textbf{\(\pm\)20\%} (80k\(\Omega\) - 120k\(\Omega\))
\end{description}

\paragraph{મેમરી ટ્રીક:}
\emph{``Brown-Black-Yellow: 1-0 સાથે 10000 fellow = 100k''!}

% ========================================
% QUESTION 3: Semiconductors & Rectifiers (14 marks)
% ========================================


\section{પ્રશ્ન 3}

% This section covers semiconductors, diodes and rectifiers.
% આ વિભાગ સેમિકન્ડક્ટર્સ, ડાયોડ્સ અને રેક્ટિફાયર્સને આવરી લે છે.

% Q3(a): Define Doping



\subsection{પ્રશ્ન 3(a) [3 ગુણ]}
\textbf{ડોપિંગની વ્યાખ્યા લખો. ડોપિંગથી બનતા અર્ધવાહકોના નામ તથા ઉદાહરણ આપો.}

\subsubsection{ઉકેલ}

\paragraph{વ્યાખ્યા:}
\textbf{ડોપિંગ} એ ક્રિયા છે જેમાં charge carriers ની સંખ્યા વધારીને તેના electrical properties modify કરવા માટે શુદ્ધ (intrinsic) semiconductor માં જાણીજોઈને impurity atoms ઉમેરવામાં આવે છે.

\paragraph{ડોપિંગથી બનતા અર્ધવાહકો:}

\begin{description}
 \item[P-type Semiconductor:] \textbf{Trivalent} (Group III) impurities સાથે doping કરીને બનાવેલ.

 \textbf{ઉદાહરણ:} શુદ્ધ Silicon માં Boron (B), Gallium (Ga), અથવા Indium (In) doped.

 પરિણામ: ``Holes'' (positive charge carriers) majority carriers તરીકે બનાવે છે.

 \item[N-type Semiconductor:] \textbf{Pentavalent} (Group V) impurities સાથે doping કરીને બનાવેલ.

 \textbf{ઉદાહરણ:} શુદ્ધ Silicon માં Phosphorus (P), Arsenic (As), અથવા Antimony (Sb) doped.

 પરિણામ: વધારાના electrons majority carriers બને છે.
\end{description}

\begin{figure}[H]
\centering
\begin{tikzpicture}[scale=1.1]
 % P-type doping
 \node[font=\bfseries] at (2.5,4) {P-type ડોપિંગ};
 \draw[thick] (0,0.5) rectangle (5,3.5);

 % Si atoms
 \foreach \x/\y in {0.8/1, 1.6/1, 2.4/1, 3.2/1, 4/1, 0.8/2, 1.6/2, 2.4/2, 3.2/2, 4/2, 0.8/3, 1.6/3, 3.2/3, 4/3} {
 \fill[blue!60] (\x,\y) circle (3pt);
 \node[font=\tiny] at (\x,\y+0.25) {Si};
 }

 % Boron impurity
 \fill[red!70] (2.4,3) circle (4pt);
 \node[font=\tiny, white] at (2.4,3) {B};

 % Hole
 \draw[red, very thick, dashed] (2.4,2.5) circle (5pt);
 \node[below, red, font=\small] at (2.4,2.3) {hole};

 \node[below, font=\small] at (2.5,0.2) {Boron (3 electrons) hole બનાવે છે};

 % N-type doping
 \node[font=\bfseries] at (9,4) {N-type ડોપિંગ};
 \draw[thick] (6.5,0.5) rectangle (11.5,3.5);

 % Si atoms
 \foreach \x/\y in {7.3/1, 8.1/1, 8.9/1, 9.7/1, 10.5/1, 7.3/2, 8.1/2, 8.9/2, 9.7/2, 10.5/2, 7.3/3, 8.1/3, 9.7/3, 10.5/3} {
 \fill[blue!60] (\x,\y) circle (3pt);
 \node[font=\tiny] at (\x,\y+0.25) {Si};
 }

 % Phosphorus impurity
 \fill[green!60!black] (8.9,3) circle (4pt);
 \node[font=\tiny, white] at (8.9,3) {P};

 % Extra electron
 \fill[green!70] (8.9,2.5) circle (2.5pt);
 \node[below, green!60!black, font=\small] at (8.9,2.3) {e\(^-\)};

 \node[below, font=\small] at (9,0.2) {Phosphorus (5 electrons) electron donate કરે છે};
\end{tikzpicture}
\caption{P-type અને N-type ડોપિંગ પ્રક્રિયા}
\end{figure}

\paragraph{પ્રક્રિયા:} પેન્ટાવેલેન્ટ અશુદ્ધિ ઈલેક્ટ્રોન ઉમેરે છે (N-type), જ્યારે ટ્રાઈવેલેન્ટ હોલ્સ ઉમેરે છે (P-type).

\paragraph{મેમરી ટ્રીક:}
\emph{``P-type: Positive holes Trivalent થી; N-type: Negative electrons Pentavalent થી''!}

% Q3(a) OR: Define RPV, PIV, Efficiency



\subsection{પ્રશ્ન 3(a) OR [3 ગુણ]}
\textbf{વ્યાખ્યા લખો: રિપલ ફેક્ટર, પીક ઇનવર્સ વોલ્ટેજ, રેક્ટિફિકેશન એફિશિયન્સી.}

\subsubsection{ઉકેલ}

\paragraph{1. Ripple Factor (r):}
Rectifier output માં AC component ના RMS value અને DC component ના ગુણોત્તર.
\[
r = \frac{V_{ac(rms)}}{V_{dc}}
\]
ઓછો ripple factor સારું filtering દર્શાવે છે. Values: Half-wave = 1.21, Full-wave = 0.48.

\paragraph{2. Peak Inverse Voltage (PIV):}
Non-conducting સમયે diode પર maximum reverse voltage. Diode breakdown વગર આને સહન કરવું જોઈએ. Values: Half-wave = \(V_m\), Full-wave center-tap = \(2V_m\), Bridge = \(V_m\).

\paragraph{3. Rectification Efficiency (\(\eta\)):}
AC input power અને DC output power નો ગુણોત્તર.
\[
\eta = \frac{P_{dc}}{P_{ac}} \times 100\%
\]
Values: Half-wave = 40.6\%, Full-wave = 81.2\%.

\paragraph{મહત્વ:} Audio circuits માટે નીચો ripple factor જરૂરી છે. Diode damage અટકાવવા માટે PIV rating peak voltage કરતાં વધારે હોવું જોઈએ. Efficiency portable devices માં battery life નક્કી કરે છે.

\paragraph{મેમરી ટ્રીક:}
\emph{``RPE: Ripple બાકી AC દર્શાવે, PIV diode protect કરે, Efficiency conversion માપે''!}

% Q3(b): Crystal Diode Working

\subsection{પ્રશ્ન 3(b) [3 ગુણ]}
\textbf{ક્રિસ્ટલ ડાયોડનું કાર્ય સમજાવો.}

\subsubsection{ઉકેલ}

\textbf{Crystal diode} (PN junction diode) માત્ર એક દિશામાં current વહેવા દે છે.

\paragraph{બાંધકામ:}
P-type અને N-type semiconductors જોડાઈને PN junction બનાવે છે જેમાં depletion region હોય છે.

\paragraph{કાર્ય:}

\begin{description}
 \item[Forward Bias:] P-side positive સાથે, N-side negative સાથે જોડાય છે. Barrier potential ઘટે છે, current વહે છે જ્યારે \(V > V_{\gamma}\) (Si માટે 0.7V).

 \item[Reverse Bias:] P-side negative સાથે, N-side positive સાથે જોડાય છે. Barrier વધે છે, માત્ર નાનો leakage current વહે છે.
\end{description}

\paragraph{ઉપયોગો:}
Rectification, clipping, clamping, voltage regulation.

\paragraph{Depletion Region:} લાગુ કરેલ voltage સાથે depletion region ની પહોળાઈ બદલાય છે. તે reverse bias માં પહોળી અને forward bias માં સાંકડી થાય છે, current flow નિયંત્રિત કરે છે.

\paragraph{બાયસિંગ:} ફોરવર્ડ બાયસમાં ડેપ્લેશન લેયર ઘટે છે, રિવર્સ બાયસમાં વધે છે.

\paragraph{મેમરી ટ્રીક:}
\emph{``Crystal Clear: Forward વહે, Reverse block કરે''!}

% Q3(b) OR: Photodiode Working



\subsection{પ્રશ્ન 3(b) OR [3 ગુણ]}
\textbf{ફોટોડાયોડનું કાર્ય સમજાવો.}

\subsubsection{ઉકેલ}

\textbf{Photodiode} પ્રકાશ energy ને electrical current માં કન્વર્ટ કરે છે, reverse bias માં કાર્ય કરે છે.

\paragraph{કાર્ય સિદ્ધાંત:}
જ્યારે photons PN junction પર strike કરે છે, તેઓ electron-hole pairs બનાવે છે. Reverse bias માં, આ carriers electric field દ્વારા junction પાર swept થાય છે, photocurrent ઉત્પન્ન કરે છે જે પ્રકાશ તીવ્રતાના પ્રમાણમાં હોય છે.

\paragraph{ઉપયોગો:}
Optical communication, light detection, solar cells, barcode readers.

\paragraph{મુખ્ય લાક્ષણિકતા:} Photodiodes હંમેશા reverse bias માં કાર્યરત હોય છે કારણ કે પ્રકાશને કારણે minority carrier current માં થતો ફેરફાર forward current ફેરફારો કરતા વધુ નોંધપાત્ર અને માપવા માટે સરળ છે.

\paragraph{મેમરી ટ્રીક:}
\emph{``Photodiode: Photons current બનાવે''!}

% Q3(c): Half-Wave Rectifier

\subsection{પ્રશ્ન 3(c) [4 ગુણ]}
\textbf{સર્કિટ તથા વેવફોર્મ દોરી half-wave rectifier સમજાવો.}

\subsubsection{ઉકેલ}

\textbf{Half-wave rectifier} એક diode વાપરીને AC ને pulsating DC માં કન્વર્ટ કરે છે, input ના માત્ર એક half-cycle નો ઉપયોગ કરે છે.

\paragraph{સર્કિટ ડાયાગ્રામ:}

\begin{figure}[H]
\centering
\begin{circuitikz}[scale=1.1]
 % Transformer
 \draw (0,0) to[sV, l=\(V_{in}\)] (0,2);

 % Diode
 \draw (0,2) to[D, l=D] (3,2);

 % Load
 \draw (3,2) to[R, l=\(R_L\)] (3,0);

 % Ground
 \draw (0,0) -- (3,0);

 % Output
 \draw[<->] (3.5,2) -- (3.5,0) node[midway, right] {\(V_{out}\)};
\end{circuitikz}
\caption{Half-Wave Rectifier સર્કિટ}
\end{figure}

\paragraph{કાર્ય:}
\textbf{Positive half-cycle:} Diode forward-biased, conduct કરે છે. Output input ને follow કરે છે.
\textbf{Negative half-cycle:} Diode reverse-biased, block કરે છે. Output શૂન્ય છે.

\paragraph{વેવફોર્મ્સ:}

\begin{figure}[H]
\centering
\begin{tikzpicture}[scale=0.95]
 % Input waveform
 \draw[->, thick] (0,0) -- (8,0) node[right] {\(t\)};
 \draw[->, thick] (0,-2) -- (0,2.5) node[above] {\(V_{in}\)};
 \draw[blue, very thick, domain=0:7.85, samples=100] plot (\x, {1.5*sin(\x r)});
 \node[blue] at (4,2.2) {AC Input};

 % Output waveform
 \begin{scope}[yshift=-5cm]
 \draw[->, thick] (0,0) -- (8,0) node[right] {\(t\)};
 \draw[->, thick] (0,-0.3) -- (0,2.5) node[above] {\(V_{out}\)};
 % Only positive half cycles
 \draw[red, very thick, domain=0:3.14, samples=50] plot (\x, {1.5*sin(\x r)});
 \draw[red, very thick] (3.14,0) -- (6.28,0);
 \draw[red, very thick, domain=6.28:7.85, samples=30] plot (\x, {1.5*sin(\x r)});
 \node[red] at (4,2.2) {Pulsating DC Output};
 \draw[dashed] (0,0.95) -- (8,0.95) node[right, font=\small] {\(V_{DC}\)};
 \end{scope}
\end{tikzpicture}
\caption{Half-Wave Rectifier વેવફોર્મ્સ}
\end{figure}

\paragraph{સ્પેસિફિકેશન્સ:}
\begin{itemize}
 \item DC Output: \(V_{DC} = \frac{V_m}{\pi} = 0.318 V_m\)
 \item Efficiency: \(\eta = 40.6\%\)
 \item Ripple Factor: \(r = 1.21\)
 \item PIV: \(V_m\)
\end{itemize}

\paragraph{ફિલ્ટર:} રેક્ટિફાયરના આઉટપુટને સ્મૂધ DC બનાવવા માટે ફિલ્ટર સર્કિટની જરૂર પડે છે.

\paragraph{મેમરી ટ્રીક:}
\emph{``Half-wave અડધો cycle વાપરે''!}

% Q3(c) OR: Full-Wave Rectifier



\subsection{પ્રશ્ન 3(c) OR [4 ગુણ]}
\textbf{સર્કિટ તથા વેવફોર્મ દોરી full-wave rectifier સમજાવો.}

\subsubsection{ઉકેલ}

\textbf{Full-wave rectifier} બંને half-cycles વાપરીને AC ને pulsating DC માં કન્વર્ટ કરે છે, center-tapped transformer અને બે diodes સાથે.

\paragraph{સર્કિટ ડાયાગ્રામ:}

\begin{figure}[H]
\centering
\begin{circuitikz}[scale=1.0]
 % Transformer with center tap
 \draw (0,0) to[sV] (0,3);
 \draw (0,3) -- (1,3) -- (1,3.5) node[above] {A};
 \draw (0,1.5) -- (1,1.5) node[right] {CT};
 \draw (0,0) -- (1,0) -- (1,-0.5) node[below] {B};

 % Diodes
 \draw (1,3.5) to[D, l=\(D_1\)] (3.5,3.5);
 \draw (1,-0.5) to[D, l=\(D_2\)] (3.5,-0.5);

 % Load
 \draw (3.5,3.5) -- (3.5,2.5);
 \draw (3.5,2.5) to[R, l=\(R_L\)] (3.5,0.5);
 \draw (3.5,0.5) -- (3.5,-0.5);

 % Center tap to ground
 \draw (1,1.5) -- (3.5,1.5);
 \draw (2.25,1.5) node[ground]{};

 % Output
 \draw[<->] (4,2.5) -- (4,0.5) node[midway, right] {\(V_{out}\)};
\end{circuitikz}
\caption{Full-Wave Rectifier (Center-Tap) સર્કિટ}
\end{figure}

\paragraph{કાર્ય:}
\textbf{Positive half:} A positive, B negative. \(D_1\) conducts, \(D_2\) blocks.
\textbf{Negative half:} B positive, A negative. \(D_2\) conducts, \(D_1\) blocks.
બંને halves load દ્વારા સમાન દિશામાં output ઉત્પન્ન કરે છે.

\paragraph{વેવફોર્મ્સ:}

\begin{figure}[H]
\centering
\begin{tikzpicture}[scale=0.95]
 % Input
 \draw[->, thick] (0,0) -- (8,0) node[right] {\(t\)};
 \draw[->, thick] (0,-2) -- (0,2.5) node[above] {\(V\)};
 \draw[blue, very thick, domain=0:7.85, samples=100] plot (\x, {1.5*sin(\x r)});
 \node[blue] at (4,2.2) {AC Input};

 % Output
 \begin{scope}[yshift=-5cm]
 \draw[->, thick] (0,0) -- (8,0) node[right] {\(t\)};
 \draw[->, thick] (0,-0.3) -- (0,2.5) node[above] {\(V_{out}\)};
 \draw[red, very thick, domain=0:7.85, samples=100] plot (\x, {1.5*abs(sin(\x r))});
 \node[red] at (4,2.2) {Pulsating DC};
 \draw[dashed] (0,0.95) -- (8,0.95) node[right, font=\small] {\(V_{DC}\)};
 \end{scope}
\end{tikzpicture}
\caption{Full-Wave Rectifier વેવફોર્મ્સ}
\end{figure}

\paragraph{સ્પેસિફિકેશન્સ:}
\begin{itemize}
 \item DC Output: \(V_{DC} = \frac{2V_m}{\pi} = 0.636 V_m\)
 \item Efficiency: \(\eta = 81.2\%\)
 \item Ripple Factor: \(r = 0.48\)
 \item PIV: \(2V_m\)
 \item Ripple Frequency: \(2f\) (input frequency કરતાં બમણી)
\end{itemize}

\paragraph{મેમરી ટ્રીક:}
\emph{``Full-wave સંપૂર્ણ cycle વાપરે, બમણી efficiency''!}

% Q3(d): VI Characteristics

\subsection{પ્રશ્ન 3(d) [4 ગુણ]}
\textbf{PN junction diode ની VI લાક્ષણિકતાઓ આલેખ દોરી સમજાવો.}

\subsubsection{ઉકેલ}

\textbf{Voltage-Current (VI) characteristic} PN diode માં voltage અને current વચ્ચેનો સંબંધ દર્શાવે છે.

\begin{figure}[H]
\centering
\begin{tikzpicture}[scale=1.1]
 % Axes
 \draw[->, very thick] (-4,0) -- (4,0) node[right] {\(V\) (Voltage)};
 \draw[->, very thick] (0,-2.5) -- (0,3.5) node[above] {\(I\) (Current)};

 % Forward characteristic
 \draw[blue, very thick, domain=0.6:3.5, samples=50] plot (\x, {3*(\x-0.5)^2});

 % Reverse characteristic
 \draw[red, very thick] (-3.5,-0.4) -- (-0.1,-0.4);

 % Breakdown
 \draw[red, very thick] (-3.5,-0.4) -- (-3.5,-2.3);

 % Labels
 \node[blue] at (2.5,3.2) {Forward Bias};
 \node[red] at (-2.5,-0.7) {Reverse Bias};
 \node[red] at (-3,-2.3) {Breakdown};

 % Key points
 \draw[dashed] (0.7,0) -- (0.7,0.5);
 \node[below] at (0.7,-0.2) {\(V_{\gamma}\)};
 \node[below, font=\tiny] at (0.7,-0.5) {0.7V (Si)};

 \draw[dashed] (0,-0.4) -- (-1,-0.4);
 \node[left, font=\small] at (-0.2,-0.4) {\(-I_S\)};

 \draw[dashed] (-3.5,0) -- (-3.5,-0.4);
 \node[above, font=\small] at (-3.5,0.2) {\(V_{BR}\)};
\end{tikzpicture}
\caption{PN Junction Diode ની VI લાક્ષણિકતાઓ}
\end{figure}

\paragraph{પ્રદેશો:}

\begin{description}
 \item[Forward Bias (\(V > 0\)):] \(V_{\gamma}\) સુધી નાનો current (knee voltage). Threshold પછી, current exponentially વધે છે.

 \item[Reverse Bias (\(V < 0\)):] નાનો reverse saturation current \(I_S\) (થોડા \(\mu\)A) minority carriers ને લીધે. લગભગ સ્થિર.

 \item[Breakdown (\(V < V_{BR}\)):] મોટા reverse voltage પર, avalanche/Zener breakdown થાય છે. Current ઝડપથી વધે છે.
\end{description}

\paragraph{Knee Voltage:} Si ડાયોડ માટે 0.7V અને Ge ડાયોડ માટે 0.3V હોય છે.

\paragraph{મેમરી ટ્રીક:}
\emph{``Forward threshold પછી વહે; Reverse breakdown સિવાય અટકાવે''!}

% Q3(d) OR: P-type vs N-type



\subsection{પ્રશ્ન 3(d) OR [4 ગુણ]}
\textbf{P-type અને N-type semiconductor વચ્ચેનો તફાવત લખો.}

\subsubsection{ઉકેલ}

\begin{table}[H]
\centering
\caption{P-type બનામ N-type Semiconductor}
\begin{tabularx}{\textwidth}{lXX}
\toprule
\textbf{પેરામીટર} & \textbf{P-type} & \textbf{N-type} \\
\midrule
Doping & Trivalent (B, Ga, In) & Pentavalent (P, As, Sb) \\
Majority Carriers & Holes (\(h^+\)) & Electrons (\(e^-\)) \\
Minority Carriers & Electrons & Holes \\
Donor/Acceptor & Acceptor atoms & Donor atoms \\
વાહકતા & Hole conduction & Electron conduction \\
Fermi Level & Valence band નજીક & Conduction band નજીક \\
\bottomrule
\end{tabularx}
\end{table}

\paragraph{Energy Levels:} N-type માં, donor energy level conduction band (Ge માટે 0.01eV) ની ખૂબ નજીક છે. P-type માં, acceptor energy level valence band ની નજીક છે. આના માટે ionization માટે ઓછી energy જરૂરી છે.

\paragraph{વાહકતા:} N-type માં, વાહકતા મુખ્યત્વે electrons ને કારણે છે જે holes કરતા વધારે ગતિશીલતા ધરાવે છે. તેથી N-type devices સામાન્ય રીતે P-type કરતા ઝડપી હોય છે.

\paragraph{મેમરી ટ્રીક:}
\emph{``P માટે Positive holes; N માટે Negative electrons''!}

% ========================================
% QUESTION 4 \u0026 5: Remaining content (condensed for completion)
% ========================================

\section{પ્રશ્ન 4}

% This section covers special diodes and voltage regulation.
% આ વિભાગ ખાસ ડાયોડ્સ અને વોલ્ટેજ રેગ્યુલેશનને આવરી લે છે.



\subsection{પ્રશ્ન 4(a) [3 ગુણ]}
\textbf{LED ની કાર્યપદ્ધતિ સમજાવો.}

\subsubsection{ઉકેલ}

\textbf{Light Emitting Diode (LED)} \textbf{electroluminescence} દ્વારા electrical energy ને light માં કન્વર્ટ કરે છે.

\paragraph{સિદ્ધાંત:}
જ્યારે forward current વહે છે, ત્યારે N-region માંથી electrons P-region માં holes સાથે junction પર recombine થાય છે. Energy photons (light) તરીકે મુક્ત થાય છે. તરંગલંબાઇ (રંગ) semiconductor band gap energy પર આધાર રાખે છે.

\paragraph{Energy સંબંધ:}
\[
E = h\nu = \frac{hc}{\lambda} = E_g
\]
જ્યાં \(E_g\) band gap energy છે, રંગ નક્કી કરે છે.

\paragraph{સામગ્રી \u0026 રંગો:}
GaAs (Red), GaP (Green), GaN (Blue), InGaN (White).

\paragraph{સામગ્રી પસંદગી:} Silicon અને Germanium નો ઉપયોગ LEDs માટે થતો નથી કારણ કે તેઓ heat તરીકે ઉર્જા મુક્ત કરે છે. Gallium Arsenide/Phosphide direct bandgap materials છે જે light emission માટે વપરાય છે.

\paragraph{વોલ્ટેજ:} લાલ LED માટે ફોરવર્ડ વોલ્ટેજ સામાન્ય રીતે 1.8V હોય છે.

\paragraph{મેમરી ટ્રીક:}
\emph{``LED: Recombination દરમિયાન Energy Drop થી Light''!}



\subsection{પ્રશ્ન 4(a) OR [3 ગુણ]}
\textbf{LED ના ઉપયોગો જણાવો.}

\subsubsection{ઉકેલ}

\paragraph{ઉપયોગો:}
\begin{enumerate}
 \item Display panels (seven-segment, dot-matrix)
 \item Indicator lights (power, status)
 \item Traffic signals
 \item Automotive lighting
 \item General illumination
 \item Backlighting (LCD screens)
 \item Optical communication
\end{enumerate}

\paragraph{ફાયદા:} Incandescent bulbs ની સરખામણીમાં LEDs ખૂબ ઓછો power વાપરે છે, લાંબુ life span (>50,000 કલાક), ઝડપી switching speed (ns), અને ઉચ્ચ mechanical ruggedness ધરાવે છે.

\paragraph{કાર્યક્ષમતા:} LEDs આશરે 80-90% ઉર્જાને પ્રકાશમાં રૂપાંતરિત કરે છે, જ્યારે incandescent bulbs માત્ર 10-20% કરે છે, બાકીની ગરમી તરીકે વેડફાય છે. આ LEDs ને અત્યંત energy-efficient બનાવે છે.

\paragraph{ઔદ્યોગિક ઉપયોગ:} Automated industries માં, LEDs નો ઉપયોગ sensor systems, quality control scanners અને machine vision lighting માં તેમની reliability અને consistent spectral output ને કારણે થાય છે.

\paragraph{મેમરી ટ્રીક:}
\emph{``LED દરેક જગ્યાએ: Displays, Indicators, Traffic, Automotive''!}



\subsection{પ્રશ્ન 4(b) [4 ગુણ]}
\textbf{ઝેનર ડાયોડ voltage regulator તરીકે સમજાવો.}

\subsubsection{ઉકેલ}

\textbf{Zener diode} reverse breakdown region માં કાર્ય કરીને સ્થિર output voltage જાળવે છે.

\paragraph{સર્કિટ:}

\begin{figure}[H]
\centering
\begin{circuitikz}[scale=1.2]
 % Input
 \draw (0,0) to[V, l=\(V_{in}\)] (0,3);

 % Series resistor
 \draw (0,3) to[R, l=\(R_S\)] (3,3);

 % Zener diode
 \draw (3,3) to[zzD, l=\(V_Z\)] (3,0);

 % Load
 \draw (3,3) -- (5,3);
 \draw (5,3) to[R, l=\(R_L\)] (5,0);

 % Ground
 \draw (0,0) -- (5,0);

 % Output
 \draw[<->] (5.5,3) -- (5.5,0) node[midway, right] {\(V_{out}=V_Z\)};
\end{circuitikz}
\caption{Zener Voltage Regulator}
\end{figure}

\paragraph{કાર્ય:}
\(R_S\) વધારાનો voltage drops કરે છે. Zener \(V_{out} = V_Z\) જાળવે છે. \(V_{in}\) અથવા \(I_L\) માં ફેરફારો \(I_Z\) vary કરીને absorb થાય છે.

\[
I_S = I_Z + I_L = \frac{V_{in} - V_Z}{R_S}
\]

\paragraph{Design Condition:} યોગ્ય regulation માટે, Zener diode breakdown region માં રહેવો જોઈએ. Zener current (\(I_Z\)) તમામ load conditions હેઠળ \(I_{Z(\text{min})}\) અને \(I_{Z(\text{max})}\) વચ્ચે હોવો જોઈએ.

\paragraph{Regulation Factor:} Regulation ની ગુણવત્તા line regulation (input change) અને load regulation (load change) દ્વારા માપવામાં આવે છે. Zener ઓછા ખર્ચની applications માટે સ્વીકાર્ય regulation પૂરું પાડે છે.

\paragraph{ડોપિંગ:} ઝેનર ડાયોડમાં સામાન્ય ડાયોડ કરતાં ડોપિંગ લેવલ ઘણું વધારે હોય છે.

\paragraph{મેમરી ટ્રીક:}
\emph{``Zener વધારાને Zaps કરે, સ્થિર voltage જાળવે''!}



\subsection{પ્રશ્ન 4(b) OR [4 ગુણ]}
\textbf{Zener voltage regulator ની મર્યાદાઓ આપો.}

\subsubsection{ઉકેલ}

\paragraph{મર્યાદાઓ:}
\begin{enumerate}
 \item મર્યાદિત current capacity (થોડા mA થી થોડા A)
 \item નબળી efficiency (power heat તરીકે વિખેરાય છે)
 \item Output voltage adjustable નથી (Zener થી fixed)
 \item મર્યાદિત ripple rejection
 \item High power applications માટે યોગ્ય નથી
 \item Minimum load current જરૂરી છે
\end{enumerate}

\paragraph{તાપમાન નિર્ભરતા:} તાપમાન સાથે Zener voltage બદલાય છે. Zener breakdown negative temperature coefficient ધરાવે છે, જ્યારે Avalanche breakdown positive coefficient ધરાવે છે. આ precision circuits માં stability ને અસર કરી શકે છે.

\paragraph{ ઘોંઘાટ (Noise):} Avalanche breakdown region માં Zener diodes નોંધપાત્ર ઘોંઘાટ ઉત્પન્ન કરે છે. આ ઘોંઘાટ sensitive signal processing circuits માં દખલ કરી શકે છે, જેના માટે વધારાના bypass capacitors ની જરૂર પડે છે.

\paragraph{સ્થિરતા:} ખૂબ જ high precision voltage references માટે, standard Zener diodes ને ઘણીવાર bandgap reference circuits દ્વારા બદલવામાં આવે છે જે વધુ સારી temperature stability અને ઓછો noise performance આપે છે.

\paragraph{મેમરી ટ્રીક:}
\emph{``Zener મર્યાદાઓ: Low current, Fixed voltage, Heat loss''!}



\subsection{પ્રશ્ન 4(c) [7 ગુણ]}
\textbf{Rectifier માં filter circuit ની જરૂરિયાત વણવો. Rectifier માં વિવિધ પ્રકારની filter circuits ના નામ આપો અને કોઈ એક સર્કિટ સમજાવો.}

\subsubsection{ઉકેલ}

\paragraph{Filter Circuit ની જરૂરિયાત:}

Rectifier output સાથે pulsating DC છે જેમાં નોંધપાત્ર AC ripple હોય છે. Filters આને steady DC માં smooth કરે છે:
\begin{itemize}
 \item Sensitive electronic components protect કરવા
 \item Ripple factor ઘટાડવા
 \item Voltage regulation સુધારવા
 \item Electrical noise minimize કરવા
\end{itemize}

\paragraph{Filter Circuits ના પ્રકારો:}
\begin{enumerate}
 \item Capacitor filter (C)
 \item Inductor filter (L)
 \item LC filter
 \item CLC filter (\(\pi\)-filter)
 \item RC filter
\end{enumerate}

\paragraph{સમજૂતી: Capacitor Filter}

\subparagraph{સર્કિટ:}

\begin{figure}[H]
\centering
\begin{circuitikz}[scale=1.0]
 % Rectifier
 \draw (0,0) to[sV] (0,2);
 \draw (0,2) to[D] (2,2);

 % Capacitor
 \draw (2,2) to[short] (3,2);
 \draw (3,2) to[C, l=\(C\)] (3,0);

 % Load
 \draw (3,2) to[short] (4.5,2);
 \draw (4.5,2) to[R, l=\(R_L\)] (4.5,0);

 % Ground
 \draw (0,0) -- (4.5,0);

 % Output
 \draw[<->] (5,2) -- (5,0) node[midway, right] {\(V_{out}\)};
\end{circuitikz}
\caption{Capacitor Filter સર્કિટ}
\end{figure}

\subparagraph{કાર્ય:}
Capacitor conduction દરમિયાન peak પર ચાર્જ થાય છે, diode OFF હોય ત્યારે load દ્વારા discharge થાય છે. નાની ripple સાથે relatively smooth DC પ્રદાન કરે છે.

\subparagraph{વેવફોર્મ:}

\begin{figure}[H]
\centering
\begin{tikzpicture}[scale=0.9]
 \draw[->, thick] (0,0) -- (7,0) node[right] {\(t\)};
 \draw[->, thick] (0,0) -- (0,3) node[above] {\(V\)};

 % Ripple waveform
 \draw[red, very thick, domain=0:6.5, samples=200] plot (\x, {2 - 0.3*abs(sin(1.57*\x r))});
 \node[red] at (3.5,2.7) {Filtered Output};

 \draw[dashed, blue] (0,2) -- (7,2) node[right, font=\small] {\(V_{DC}\)};
\end{tikzpicture}
\caption{Capacitor Filter Output વેવફોર્મ}
\end{figure}

\subparagraph{સ્પેસિફિકેશન્સ:}
Ripple: \(V_{ripple} \approx \frac{I_{DC}}{fC}\). Low current loads માટે સારું.

\paragraph{સરખામણી:} L-filter heavy loads માટે સારું છે. C-filter light loads માટે સારું છે. Pi-filter શ્રેષ્ઠ smoothing પ્રદાન કરે છે પરંતુ bulky છે. પસંદગી load અને cost પર આધાર રાખે છે.

\paragraph{વધારાની માહિતી:} ફિલ્ટર સર્કિટ્સ વગર, ઈલેક્ટ્રોનિક ઉપકરણોમાં ઘોંઘાટ (hum) આવે છે. કેપેસીટર અને ઈન્ડક્ટરના સંયોજનથી શ્રેષ્ઠ ફિલ્ટરિંગ મળે છે.

\paragraph{મેમરી ટ્રીક:}
\emph{``Filter pulsating DC ને smooth DC માં Flattens કરે''!}






\section{પ્રશ્ન 5}

% This section covers environmental issues and transistor applications.
% આ વિભાગ પર્યાવરણીય સમસ્યાઓ અને ટ્રાન્ઝિસ્ટર એપ્લિકેશન્સને આવરી લે છે.



\subsection{પ્રશ્ન 5(a) [3 ગુણ]}
\textbf{ઈ-વેસ્ટની વ્યાખ્યા લખો. સામાન્ય ઈ-વેસ્ટ વસ્તુઓની યાદી બનાવો.}

\subsubsection{ઉકેલ}

\paragraph{વ્યાખ્યા:}
\textbf{E-waste (Electronic Waste)} એ discarded electrical અથવા electronic devices અને તેમના components ને સંદર્ભિત કરે છે. તેમાં obsolete, broken અથવા end-of-life electronic products નો સમાવેશ થાય છે.

\paragraph{સામાન્ય E-waste વસ્તુઓ:}
\begin{itemize}
 \item Computers, laptops, tablets
 \item Mobile phones, chargers
 \item TVs, monitors
 \item Printers, scanners
 \item Batteries (lithium, lead-acid)
 \item Circuit boards, cables
 \item ઘરેલું ઉપકરણો (refrigerators, washing machines)
\end{itemize}

\paragraph{પર્યાવરણીય અસર:} E-waste માં સામાન્ય રીતે Lead (Pb), Cadmium (Cd), Mercury (Hg) જેવા જોખમી પદાર્થો હોય છે જે dump કરવામાં આવે તો જમીન અને પાણીને દૂષિત કરે છે.

\paragraph{સ્ત્રોતો:} મુખ્ય સ્ત્રોતોમાં IT equipment (servers, PCs), consumer electronics (TVs, cameras), મોટા ઘરેલું ઉપકરણો અને રમકડાંનો સમાવેશ થાય છે. ઝડપી technology upgrades કચરાનું પ્રમાણ વધારે છે.

\paragraph{ચેતવણી:} E-waste ને સામાન્ય કચરા સાથે ફેંકવો જોઈએ નહીં. તેમાં રહેલા ઝેરી પદાર્થો રિસાયકલ કરવા જરૂરી છે.
\paragraph{મેમરી ટ્રીક:}
\emph{``E-waste: વાપર્યા પછી Electronics discarded''!}




\subsection{પ્રશ્ન 5(b) [3 ગુણ]}
\textbf{E-waste management ની વિવિધ વ્યૂહરચનાઓ જણાવો અને સમજાવો.}

\subsubsection{ઉકેલ}

\paragraph{E-waste Management વ્યૂહરચનાઓ:}

\begin{description}
 \item[Reduce:] Repair અને maintenance દ્વારા product lifespan extend કરો. Durable products ખરીદો.

 \item[Reuse:] Working devices donate અથવા sell કરો. Components repurpose કરો.

 \item[Recycle:] યોગ્ય recycling સુવિધાઓ દ્વારા valuable materials (gold, copper, rare metals) extract કરો.

 \item[Proper Disposal:] Authorized e-waste collection centers વાપરો. Landfills માં dump ન કરો.

 \item[Awareness:] Hazards અને proper disposal methods વિશે public ને શિક્ષિત કરો.
\end{description}

\paragraph{કાયદો:} સરકારો EPR (Extended Producer Responsibility) કાયદાઓ અમલમાં મૂકે છે, ઉત્પાદકોને electronic products ના સમગ્ર lifecycle માટે જવાબદાર બનાવે છે, eco-friendly design ને પ્રોત્સાહન આપે છે.

\paragraph{ફાયદો:} E-waste રિસાયક્લિંગથી સોનું, ચાંદી અને તાંબુ જેવી કિંમતી ધાતુઓ પુનઃપ્રાપ્ત કરી શકાય છે.
\paragraph{મેમરી ટ્રીક:}
\emph{``3R+: Reduce, Reuse, Recycle, Right disposal''!}




\subsection{પ્રશ્ન 5(c) [4 ગુણ]}
\textbf{ટ્રાન્ઝિસ્ટર સ્વીચ તરીકે સમજાવો.}

\subsubsection{ઉકેલ}

Transistor બે states સાથે electronic switch તરીકે કાર્ય કરે છે: ON (saturation) અને OFF (cutoff).

\paragraph{સર્કિટ:}

\begin{figure}[H]
\centering
\begin{circuitikz}[scale=1.1]
 % Supply
 \draw (0,4) to[V, l=\(V_{CC}\)] (0,0);

 % Load resistor
 \draw (0,4) to[R, l=\(R_C\)] (3,4);

 % Transistor
 \draw (3,2) node[npn](npn1){};
 \draw (npn1.C) -- (3,4);
 \draw (npn1.E) -- (3,0);

 % Base resistor
 \draw (0,2) to[R, l=\(R_B\)] (npn1.B);
 \draw (0,2) to[short] (-1,2) node[left] {\(V_{in}\)};

 % Ground
 \draw (0,0) -- (3,0);

 % Output
 \draw (3,4) to[short] (4,4) node[right] {\(V_{out}\)};
\end{circuitikz}
\caption{Transistor as Switch}
\end{figure}

\paragraph{States:}

\begin{description}
 \item[OFF (Cutoff):] \(V_{in} = 0V\), base current \(I_B = 0\), \(I_C \approx 0\), \(V_{out} = V_{CC}\) (Switch OPEN)

 \item[ON (Saturation):] \(V_{in} = High\), \(I_B\) પૂરતો, \(I_C = I_{C(sat)}\), \(V_{out} \approx 0V\) (Switch CLOSED)
\end{description}

\paragraph{ઉપયોગો:}
Digital circuits, relay drivers, LED drivers, motor control.

\paragraph{Ideal vs Real:} આદર્શ રીતે, ON હોય ત્યારે switch નો resistance શૂન્ય અને OFF હોય ત્યારે અનંત હોય છે. Transistors ON હોય ત્યારે નાનો saturation voltage (\(V_{CE(sat)} \approx 0.2V\)) ધરાવે છે.

\paragraph{સ્થિતિ:} સેચ્યુરેશન રીજીયનમાં ટ્રાન્ઝિસ્ટર ON સ્વિચ તરીકે અને કટ-ઓફમાં OFF સ્વિચ તરીકે વર્તે છે.
\paragraph{મેમરી ટ્રીક:}
\emph{``Transistor Switch: Base Collector current control કરે''!}




\subsection{પ્રશ્ન 5(d) [4 ગુણ]}
\textbf{ટ્રાન્ઝિસ્ટરના CE configuration માટે \(\alpha\) તથા \(\beta\) વચ્ચેનો સંબંધ તારવો.}

\subsubsection{ઉકેલ}

Transistor માટે Common Emitter configuration માં:

\paragraph{વ્યાખ્યાઓ:}
\[
\alpha = \frac{I_C}{I_E} \quad (\text{Common Base gain})
\]
\[
\beta = \frac{I_C}{I_B} \quad (\text{Common Emitter gain})
\]

\paragraph{ડેરિવેશન:}

Kirchhoff's Current Law થી:
\[
I_E = I_B + I_C
\]

\(\alpha\) ની વ્યાખ્યા થી:
\[
I_C = \alpha I_E = \alpha(I_B + I_C)
\]

વિસ્તરણ:
\[
I_C = \alpha I_B + \alpha I_C
\]

પુનઃગોઠવણી:

\[
I_C(1 - \alpha) = \alpha I_B
\]

\[
\frac{I_C}{I_B} = \frac{\alpha}{1-\alpha}
\]

\(\beta = \frac{I_C}{I_B}\) હોવાથી:

\[
\boxed{\beta = \frac{\alpha}{1-\alpha}}
\]

સમાન રીતે, \(\alpha\) માટે ઉકેલવા:
\[
\boxed{\alpha = \frac{\beta}{1+\beta}}
\]

\paragraph{ઉદાહરણ:}
જો \(\alpha = 0.98\): \(\beta = \frac{0.98}{1-0.98} = 49\)
જો \(\beta = 100\): \(\alpha = \frac{100}{1+100} = 0.99\)

\paragraph{મહત્વ:} \(\alpha\) નું મૂલ્ય હંમેશા 1 (0.95 થી 0.99) કરતા થોડું ઓછું હોય છે કારણ કે \(I_C < I_E\). \(\beta\) નું મૂલ્ય 1 કરતા ઘણું વધારે (સામાન્ય રીતે 50-300) હોય છે, જે CE configuration ની ઉચ્ચ current gain ક્ષમતા દર્શાવે છે.

\paragraph{ઉપયોગ:} ઉચ્ચ beta transistors ને amplifiers તરીકે કાર્ય કરવા દે છે. નાનો base current ફેરફાર મોટા collector current ફેરફારનું કારણ બને છે, જે input signal ને amplify કરે છે.

\paragraph{યાદ રાખો:} હંમેશા \(\beta > \alpha\) હોય છે. \(\alpha\) નું મૂલ્ય 1 ની નજીક હોય છે, જ્યારે \(\beta\) નું મૂલ્ય 20 થી 500 સુધી હોઈ શકે છે.
\paragraph{મેમરી ટ્રીક:}
\emph{``Alpha over one-minus-alpha equals Beta''!}

\end{document}



