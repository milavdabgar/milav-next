\documentclass[10pt,a4paper]{article}

% content/resources/templates/preamble.tex
\usepackage[margin=0.6in]{geometry}
\author{Milav Dabgar}
\usepackage{amsmath,amssymb,amsthm}
\usepackage{booktabs}
\usepackage{multirow}
\usepackage{xcolor}
\usepackage{tcolorbox}
\tcbuselibrary{breakable,skins}
\usepackage[colorlinks=true,linkcolor=blue]{hyperref}
\usepackage{titlesec}
\usepackage{enumitem}
\usepackage{tikz}
\usepackage{pgfplots}
\usepackage{circuitikz}
\usepackage[version=4]{mhchem}
\usepackage{longtable}
\usepackage{array}
\usepackage{float}
\usepackage{caption}
\usepackage{listings}

\lstset{
  basicstyle=\small\ttfamily,
  breaklines=true,
  breakatwhitespace=false,
  postbreak=\mbox{\textcolor{red}{$\hookrightarrow$}\space},
  float=false,
  numbers=left,
  numberstyle=\tiny\color{gray},
  numbersep=10pt,
  xleftmargin=2em,
  keywordstyle=\color{blue},
  commentstyle=\color{green!60!black},
  stringstyle=\color{purple},
  backgroundcolor=\color{gray!5},
  showstringspaces=false,
  tabsize=2,
  captionpos=b,
  keepspaces=true,
  columns=flexible
}

\pgfplotsset{compat=1.18}
\usetikzlibrary{shapes,arrows,positioning,calc,patterns,decorations.pathmorphing,decorations.markings,arrows.meta}

% Color scheme
\definecolor{headcolor}{RGB}{0,102,204}
\definecolor{keycolor}{RGB}{220,20,60}
\definecolor{solutioncolor}{RGB}{34,139,34}
\definecolor{mnemoniccolor}{RGB}{148,0,211}
\definecolor{codecolor}{RGB}{0,0,100}

% Spacing
\setlength{\parskip}{3pt}
\setlist[itemize]{nosep}
\setlist[enumerate]{nosep}

% Title formatting
\titleformat{\section}{\Large\bfseries\color{headcolor}}{\thesection}{1em}{}
\titleformat{\subsection}{\large\bfseries\color{headcolor}}{\thesubsection}{1em}{}

% Pandoc tightlist compatibility
\providecommand{\tightlist}{%
  \setlength{\itemsep}{0pt}\setlength{\parskip}{0pt}}

% Pandoc longtable compatibility
\newcounter{none}
\def\thenone{}


% content/resources/templates/english-boxes.tex
% This file is currently empty - it exists to maintain consistency with the import structure.
% Add custom environments here if needed in the future.


\begin{document}

\begin{center}
{\Huge\bfseries\color{headcolor} Subject Name Solutions}\\[5pt]
{\LARGE 4311102 -- Summer 2024}\\[3pt]
{\large Semester 1 Study Material}\\[3pt]
{\normalsize\textit{Detailed Solutions and Explanations}}
\end{center}

\vspace{10pt}

\subsection*{Question 1 [14 marks]}\label{question-1-14-marks}

\begin{solutionbox}

\end{solutionbox}
\subsubsection{Question 1(1) [2 marks]}\label{question-11-2-marks}

\textbf{Define resistor and give its unit.}

\begin{solutionbox}
A resistor is an electronic component that opposes the
flow of electric current. Its unit is Ohm (Ω).


{\def\LTcaptype{none} % do not increment counter
\vspace{-5pt}
\captionof{table}{Resistor Properties}
\vspace{-10pt}
\begin{longtable}[]{@{}ll@{}}
\toprule\noalign{}
Property & Description \\
\midrule\noalign{}
\endhead
\bottomrule\noalign{}
\endlastfoot
Symbol & ⏅ \\
Unit & Ohm (Ω) \\
Function & Limits current flow \\
\end{longtable}
}

\end{solutionbox}
\begin{mnemonicbox}
``Resistors Oppose Current'' (ROC)

\end{mnemonicbox}
\subsubsection{Question 1(2) [2 marks]}\label{question-12-2-marks}

\textbf{Give two examples of active and passive components each.}

\begin{solutionbox}


{\def\LTcaptype{none} % do not increment counter
\vspace{-5pt}
\captionof{table}{Electronic Components Classification}
\vspace{-10pt}
\begin{longtable}[]{@{}ll@{}}
\toprule\noalign{}
Active Components & Passive Components \\
\midrule\noalign{}
\endhead
\bottomrule\noalign{}
\endlastfoot
1. Transistors & 1. Resistors \\
2. Diodes & 2. Capacitors \\
\end{longtable}
}

\end{solutionbox}
\begin{mnemonicbox}
``TARD'' - Transistors And Resistors Differ

\end{mnemonicbox}
\subsubsection{Question 1(3) [2 marks]}\label{question-13-2-marks}

\textbf{Draw symbols of any two semiconductor devices.}

\begin{solutionbox}

\textbf{Diagram:}

\begin{center}
\textbf{Mermaid Diagram (Code)}
\begin{verbatim}
{Shaded}
{Highlighting}[]
graph TD
    subgraph Diode
        A[Plus] {-{-}{} B["|{}|"] {-}{-}{} C[Minus]}
    end

    subgraph NPN\_Transistor
        D[C] {-{-}{} E}
        F[E] {-{-}{} E}
        G[B] {-{-}{} E}
    end
{Highlighting}
{Shaded}
\end{verbatim}
\end{center}

\end{solutionbox}
\begin{mnemonicbox}
``Diodes Direct, Transistors Transfer''

\end{mnemonicbox}
\subsubsection{Question 1(4) [2 marks]}\label{question-14-2-marks}

\textbf{Differentiate between intrinsic and extrinsic semiconductor.}

\begin{solutionbox}


{\def\LTcaptype{none} % do not increment counter
\vspace{-5pt}
\captionof{table}{Intrinsic vs Extrinsic Semiconductors}
\vspace{-10pt}
\begin{longtable}[]{@{}
  >{\raggedright\arraybackslash}p{(\linewidth - 2\tabcolsep) * \real{0.5000}}
  >{\raggedright\arraybackslash}p{(\linewidth - 2\tabcolsep) * \real{0.5000}}@{}}
\toprule\noalign{}
\begin{minipage}[b]{\linewidth}\raggedright
Intrinsic
\end{minipage} & \begin{minipage}[b]{\linewidth}\raggedright
Extrinsic
\end{minipage} \\
\midrule\noalign{}
\endhead
\bottomrule\noalign{}
\endlastfoot
Pure semiconductor without impurities & Semiconductor with added
impurities \\
Equal number of holes and electrons & Unequal holes and electrons \\
Examples: Pure Silicon, Germanium & Examples: Silicon doped with
Phosphorus \\
\end{longtable}
}

\end{solutionbox}
\begin{mnemonicbox}
``Pure In, Doped Ex''

\end{mnemonicbox}
\subsubsection{Question 1(5) [2 marks]}\label{question-15-2-marks}

\textbf{LED stands for \_\_\_\_\_\_\_\_\_\_\_\_\_\_\_\_\_.}

\begin{solutionbox}
LED stands for \textbf{Light Emitting Diode}.

\textbf{Diagram:}

\begin{center}
\textbf{Mermaid Diagram (Code)}
\begin{verbatim}
{Shaded}
{Highlighting}[]
graph LR
    A[Light] {-{-}{} B[Emitting] {-}{-}{} C[Diode]}
    style A fill:\#f96,stroke:\#333
    style B fill:\#9cf,stroke:\#333
    style C fill:\#f9f,stroke:\#333
{Highlighting}
{Shaded}
\end{verbatim}
\end{center}

\end{solutionbox}
\begin{mnemonicbox}
``Light Emitters Dazzle''

\end{mnemonicbox}
\subsubsection{Question 1(6) [2 marks]}\label{question-16-2-marks}

\textbf{State any two applications of Photo-diode.}

\begin{solutionbox}


{\def\LTcaptype{none} % do not increment counter
\vspace{-5pt}
\captionof{table}{Photo-diode Applications}
\vspace{-10pt}
\begin{longtable}[]{@{}ll@{}}
\toprule\noalign{}
Application & How it works \\
\midrule\noalign{}
\endhead
\bottomrule\noalign{}
\endlastfoot
Light sensors & Converts light to electrical current \\
Optical communication & Detects optical signals in fiber optics \\
\end{longtable}
}

\end{solutionbox}
\begin{mnemonicbox}
``Light Sensing Communication'' (LSC)

\end{mnemonicbox}
\subsubsection{Question 1(7) [2 marks]}\label{question-17-2-marks}

\textbf{List the types of transistor and draw their symbols.}

\begin{solutionbox}

\textbf{Types of Transistors:}

\begin{enumerate}
\tightlist
\item
  NPN Transistor
\item
  PNP Transistor
\end{enumerate}

\textbf{Diagram:}

\begin{center}
\textbf{Mermaid Diagram (Code)}
\begin{verbatim}
{Shaded}
{Highlighting}[]
graph TD
    subgraph "NPN"
    A[C] {-{-}{-} B {-}{-}{-} C[E]}
    D[B] {-{-}{-} B}
    end
    subgraph "PNP"
    E[E] {-{-}{-} F {-}{-}{-} G[C]}
    H[B] {-{-}{-} F}
    end
{Highlighting}
{Shaded}
\end{verbatim}
\end{center}

\end{solutionbox}
\begin{mnemonicbox}
``Not Pointing iN, Pointing outP''

\end{mnemonicbox}
\subsubsection{Question 1(8) [2 marks]}\label{question-18-2-marks}

\textbf{Give the value of forward voltage drop of Germanium and Silicon
diode.}

\begin{solutionbox}


{\def\LTcaptype{none} % do not increment counter
\vspace{-5pt}
\captionof{table}{Forward Voltage Drop Values}
\vspace{-10pt}
\begin{longtable}[]{@{}ll@{}}
\toprule\noalign{}
Diode Type & Forward Voltage Drop \\
\midrule\noalign{}
\endhead
\bottomrule\noalign{}
\endlastfoot
Germanium & 0.3V \\
Silicon & 0.7V \\
\end{longtable}
}

\end{solutionbox}
\begin{mnemonicbox}
``Germanium's Three, Silicon's Seven'' (0.3V, 0.7V)

\end{mnemonicbox}
\subsubsection{Question 1(9) [2 marks]}\label{question-19-2-marks}

\textbf{The \_\_\_\_\_\_\_\_\_\_\_\_\_\_\_\_\_ diode can be used as a
light detector.}

\begin{solutionbox}
The \textbf{Photodiode} can be used as a light
detector.

\textbf{Diagram:}

\begin{center}
\textbf{Mermaid Diagram (Code)}
\begin{verbatim}
{Shaded}
{Highlighting}[]
graph LR
    A[Light] {-{-}{}|detected by| B[Photodiode]}
    B {-{-}{}|generates| C[Current]}
    style A fill:\#ff9,stroke:\#333
    style B fill:\#9cf,stroke:\#333
    style C fill:\#f96,stroke:\#333
{Highlighting}
{Shaded}
\end{verbatim}
\end{center}

\end{solutionbox}
\begin{mnemonicbox}
``Photo Detects Light'' (PDL)

\end{mnemonicbox}
\subsubsection{Question 1(10) [2 marks]}\label{question-110-2-marks}

\textbf{Define Q-factor of a coil.}

\begin{solutionbox}
Q-factor (Quality factor) of a coil is the ratio of its
inductive reactance to its resistance, indicating how efficiently it
stores energy.


{\def\LTcaptype{none} % do not increment counter
\vspace{-5pt}
\captionof{table}{Q-Factor}
\vspace{-10pt}
\begin{longtable}[]{@{}ll@{}}
\toprule\noalign{}
Parameter & Description \\
\midrule\noalign{}
\endhead
\bottomrule\noalign{}
\endlastfoot
Formula & Q = XL/R \\
Higher Q & Better quality, less energy loss \\
Lower Q & Poor quality, more energy loss \\
\end{longtable}
}

\end{solutionbox}
\begin{mnemonicbox}
``Quality equals Reactance over Resistance'' (QRR)

\end{mnemonicbox}
\subsection*{Question 2(a) [3 marks]}\label{q2a}

\textbf{Explain colour coding method of resistor.}

\begin{solutionbox}

Resistor color coding uses colored bands to indicate resistance value
and tolerance.


{\def\LTcaptype{none} % do not increment counter
\vspace{-5pt}
\captionof{table}{Resistor Color Code}
\vspace{-10pt}
\begin{longtable}[]{@{}lll@{}}
\toprule\noalign{}
Color & Digit & Multiplier \\
\midrule\noalign{}
\endhead
\bottomrule\noalign{}
\endlastfoot
Black & 0 & 10^{0} \\
Brown & 1 & 10^{1} \\
Red & 2 & 10^{2} \\
Orange & 3 & 10^{3} \\
Yellow & 4 & 10^{4} \\
\end{longtable}
}

For a 4-band resistor:

\begin{itemize}
\tightlist
\item
  First band: First digit
\item
  Second band: Second digit
\item
  Third band: Multiplier
\item
  Fourth band: Tolerance
\end{itemize}

\end{solutionbox}
\begin{mnemonicbox}
``Bad Boys Race Our Young Girls But Violet Generally
Wins'' (Colors in order: Black, Brown, Red, Orange, Yellow, Green, Blue,
Violet, Grey, White)

\end{mnemonicbox}
\subsection*{Question 2(a) OR [3
marks]}\label{q2a}

\textbf{Explain Light Dependent Resistor with its characteristics.}

\begin{solutionbox}

LDR is a resistor whose resistance decreases when light intensity
increases.

\textbf{Characteristics of LDR:}


{\def\LTcaptype{none} % do not increment counter
\vspace{-5pt}
\captionof{table}{LDR Properties}
\vspace{-10pt}
\begin{longtable}[]{@{}ll@{}}
\toprule\noalign{}
Parameter & Behavior \\
\midrule\noalign{}
\endhead
\bottomrule\noalign{}
\endlastfoot
Dark condition & High resistance (MΩ) \\
Bright condition & Low resistance (kΩ) \\
Response time & Few milliseconds \\
\end{longtable}
}

\textbf{Diagram:}

\begin{center}
\textbf{Mermaid Diagram (Code)}
\begin{verbatim}
{Shaded}
{Highlighting}[]
graph TD
    A[Increase Light] {-{-}{}|Causes| B[Decrease Resistance]}
    C[Decrease Light] {-{-}{}|Causes| D[Increase Resistance]}
    style A fill:\#ff9,stroke:\#333
    style B fill:\#9cf,stroke:\#333
    style C fill:\#999,stroke:\#333
    style D fill:\#f96,stroke:\#333
{Highlighting}
{Shaded}
\end{verbatim}
\end{center}

\end{solutionbox}
\begin{mnemonicbox}
``Light Up, Resistance Down'' (LURD)

\end{mnemonicbox}
\subsection*{Question 2(b) [3 marks]}\label{q2b}

\textbf{Explain classification of capacitors in detail.}

\begin{solutionbox}

Capacitors are classified based on dielectric material and construction.


{\def\LTcaptype{none} % do not increment counter
\vspace{-5pt}
\captionof{table}{Capacitor Classifications}
\vspace{-10pt}
\begin{longtable}[]{@{}lll@{}}
\toprule\noalign{}
Type & Dielectric & Applications \\
\midrule\noalign{}
\endhead
\bottomrule\noalign{}
\endlastfoot
Ceramic & Ceramic & High frequency \\
Electrolytic & Aluminum oxide & Power supplies \\
Polyester & Plastic film & General purpose \\
Tantalum & Tantalum oxide & Small, high capacity \\
\end{longtable}
}

\textbf{Diagram:}

\begin{center}
\textbf{Mermaid Diagram (Code)}
\begin{verbatim}
{Shaded}
{Highlighting}[]
graph TD
    A[Capacitors] {-{-}{} B[Fixed]}
    A {-{-}{} C[Variable]}
    B {-{-}{} D[Ceramic]}
    B {-{-}{} E[Electrolytic]}
    B {-{-}{} F[Polyester/Film]}
    C {-{-}{} G[Air Gang]}
    C {-{-}{} H[Trimmer]}
    style A fill:\#f96,stroke:\#333
{Highlighting}
{Shaded}
\end{verbatim}
\end{center}

\end{solutionbox}
\begin{mnemonicbox}
``CEPT'' (Ceramic, Electrolytic, Polyester, Tantalum)

\end{mnemonicbox}
\subsection*{Question 2(b) OR [3
marks]}\label{q2b}

\textbf{Explain classification of inductor in detail.}

\begin{solutionbox}

Inductors are classified based on core material and construction.


{\def\LTcaptype{none} % do not increment counter
\vspace{-5pt}
\captionof{table}{Inductor Classifications}
\vspace{-10pt}
\begin{longtable}[]{@{}lll@{}}
\toprule\noalign{}
Type & Core & Characteristics \\
\midrule\noalign{}
\endhead
\bottomrule\noalign{}
\endlastfoot
Air core & Air & Low inductance, low losses \\
Iron core & Iron & High inductance, high losses \\
Ferrite core & Ferrite & Medium inductance, low losses \\
Toroidal & Ring shaped & High efficiency, low EMI \\
\end{longtable}
}

\textbf{Diagram:}

\begin{center}
\textbf{Mermaid Diagram (Code)}
\begin{verbatim}
{Shaded}
{Highlighting}[]
graph TD
    A[Inductors] {-{-}{} B[Air Core]}
    A {-{-}{} C[Iron Core]}
    A {-{-}{} D[Ferrite Core]}
    A {-{-}{} E[Toroidal]}
    style A fill:\#9cf,stroke:\#333
{Highlighting}
{Shaded}
\end{verbatim}
\end{center}

\end{solutionbox}
\begin{mnemonicbox}
``Air Iron Ferrite Toroid'' (AIFT)

\end{mnemonicbox}
\subsection*{Question 2(c) [4 marks]}\label{q2c}

\textbf{State and explain Faraday's laws of Electromagnetic Induction.}

\begin{solutionbox}

Faraday's laws explain how electromagnetic induction works.

\textbf{Faraday's First Law:} When a magnetic field linked with a
conductor changes, an EMF is induced in the conductor.

\textbf{Faraday's Second Law:} The magnitude of induced EMF is
proportional to the rate of change of magnetic flux.


{\def\LTcaptype{none} % do not increment counter
\vspace{-5pt}
\captionof{table}{Faraday's Laws Summary}
\vspace{-10pt}
\begin{longtable}[]{@{}lll@{}}
\toprule\noalign{}
Law & Statement & Formula \\
\midrule\noalign{}
\endhead
\bottomrule\noalign{}
\endlastfoot
First Law & Change in magnetic field induces EMF & - \\
Second Law & EMF ∝ rate of change of flux & E = -N(dΦ/dt) \\
\end{longtable}
}

\textbf{Diagram:}

\begin{center}
\textbf{Mermaid Diagram (Code)}
\begin{verbatim}
{Shaded}
{Highlighting}[]
graph LR
    A[Moving Magnet] {-{-}{}|Creates| B[Changing Magnetic Field]}
    B {-{-}{}|Induces| C[EMF in Conductor]}
    style A fill:\#f96,stroke:\#333
    style B fill:\#9cf,stroke:\#333
    style C fill:\#ff9,stroke:\#333
{Highlighting}
{Shaded}
\end{verbatim}
\end{center}

\end{solutionbox}
\begin{mnemonicbox}
``Change Magnetic Field, Create Electric Current''
(CMFCEC)

\end{mnemonicbox}
\subsection*{Question 2(c) OR [4
marks]}\label{q2c}

\textbf{Enlist specifications of capacitors and explain two in detail.}

\begin{solutionbox}

\textbf{Specifications of Capacitors:}

\begin{enumerate}
\tightlist
\item
  Capacitance value
\item
  Voltage rating
\item
  Tolerance
\item
  Leakage current
\item
  Temperature coefficient
\end{enumerate}

\textbf{Detailed Explanation:}

\textbf{Capacitance Value:} The amount of charge a capacitor can store
per volt, measured in Farads (F).

\textbf{Voltage Rating:} The maximum voltage that can be applied without
damaging the capacitor.


{\def\LTcaptype{none} % do not increment counter
\vspace{-5pt}
\captionof{table}{Capacitor Specifications}
\vspace{-10pt}
\begin{longtable}[]{@{}lll@{}}
\toprule\noalign{}
Specification & Description & Typical Values \\
\midrule\noalign{}
\endhead
\bottomrule\noalign{}
\endlastfoot
Capacitance & Charge storage capacity & pF to mF \\
Voltage Rating & Maximum safe voltage & 16V, 25V, 50V, etc. \\
\end{longtable}
}

\textbf{Diagram:}

\begin{center}
\textbf{Mermaid Diagram (Code)}
\begin{verbatim}
{Shaded}
{Highlighting}[]
graph TD
    A[Capacitor Specifications] {-{-}{} B[Capacitance Value]}
    A {-{-}{} C[Voltage Rating]}
    A {-{-}{} D[Tolerance]}
    A {-{-}{} E[Leakage Current]}
    A {-{-}{} F[Temperature Coefficient]}
    style A fill:\#9cf,stroke:\#333
{Highlighting}
{Shaded}
\end{verbatim}
\end{center}

\end{solutionbox}
\begin{mnemonicbox}
``Capacitors Very Tolerant of Low Temperatures''
(CVTLT)

\end{mnemonicbox}
\subsection*{Question 2(d) [4 marks]}\label{q2d}

\textbf{Write colour band of 47Ω\pm5\% resistance.}

\begin{solutionbox}

For 47Ω\pm5\% resistor, the color bands are:


{\def\LTcaptype{none} % do not increment counter
\vspace{-5pt}
\captionof{table}{Color Bands for 47Ω\pm5\%}
\vspace{-10pt}
\begin{longtable}[]{@{}lll@{}}
\toprule\noalign{}
Band & Color & Represents \\
\midrule\noalign{}
\endhead
\bottomrule\noalign{}
\endlastfoot
1st band & Yellow & 4 \\
2nd band & Violet & 7 \\
3rd band & Black & \times10^{0} \\
4th band & Gold & \pm5\% \\
\end{longtable}
}

\textbf{Diagram:}

\begin{center}
\textbf{Mermaid Diagram (Code)}
\begin{verbatim}
{Shaded}
{Highlighting}[]
graph LR
    A[Yellow] {-{-}{}|4| B[Violet] {-}{-}{}|7| C[Black] {-}{-}{}|10^{0}| D[Gold] {-}{-}{}|5\%| E[47Ω5\%]}
    style A fill:\#ff9,stroke:\#333
    style B fill:\#f0f,stroke:\#333
    style C fill:\#000,stroke:\#fff
    style D fill:\#fd0,stroke:\#333
    style E fill:\#fff,stroke:\#333
{Highlighting}
{Shaded}
\end{verbatim}
\end{center}

\end{solutionbox}
\begin{mnemonicbox}
``Yellow Violets Bring Gold'' (The colors of the
bands)

\end{mnemonicbox}
\subsection*{Question 2(d) OR [4
marks]}\label{q2d}

\textbf{Calculate value of resistor and tolerance for a given colour
code: Brown, Black, yellow.}

\begin{solutionbox}


{\def\LTcaptype{none} % do not increment counter
\vspace{-5pt}
\captionof{table}{Interpretation of Brown, Black, Yellow}
\vspace{-10pt}
\begin{longtable}[]{@{}llll@{}}
\toprule\noalign{}
Band & Color & Value & Meaning \\
\midrule\noalign{}
\endhead
\bottomrule\noalign{}
\endlastfoot
1st & Brown & 1 & First digit \\
2nd & Black & 0 & Second digit \\
3rd & Yellow & 10^{4} & Multiplier \\
\end{longtable}
}

Calculation: 1st digit: 1 2nd digit: 0 Multiplier: 10^{4}

Value = 10 \times 10^{4} = 100,000Ω = 100kΩ

No 4th band means \pm20\% tolerance

\textbf{Diagram:}

\begin{center}
\textbf{Mermaid Diagram (Code)}
\begin{verbatim}
{Shaded}
{Highlighting}[]
graph LR
    A[Brown] {-{-}{}|1| B[Black] {-}{-}{}|0| C[Yellow] {-}{-}{}|10^{4}| D[100kΩ 20\%]}
    style A fill:\#a52a2a,stroke:\#333
    style B fill:\#000,stroke:\#fff
    style C fill:\#ff0,stroke:\#333
    style D fill:\#fff,stroke:\#333
{Highlighting}
{Shaded}
\end{verbatim}
\end{center}

\end{solutionbox}
\begin{mnemonicbox}
``Big Black Yield'' (Brown-Black-Yellow)

\end{mnemonicbox}
\subsection*{Question 3(a) [3 marks]}\label{q3a}

\textbf{Define doping. Give the name of semiconductor materials
fabricated by doping with an example of each.}

\begin{solutionbox}

Doping is the process of adding impurities to a pure semiconductor to
modify its electrical properties.


{\def\LTcaptype{none} % do not increment counter
\vspace{-5pt}
\captionof{table}{Doped Semiconductors}
\vspace{-10pt}
\begin{longtable}[]{@{}
  >{\raggedright\arraybackslash}p{(\linewidth - 6\tabcolsep) * \real{0.1250}}
  >{\raggedright\arraybackslash}p{(\linewidth - 6\tabcolsep) * \real{0.2917}}
  >{\raggedright\arraybackslash}p{(\linewidth - 6\tabcolsep) * \real{0.1875}}
  >{\raggedright\arraybackslash}p{(\linewidth - 6\tabcolsep) * \real{0.3958}}@{}}
\toprule\noalign{}
\begin{minipage}[b]{\linewidth}\raggedright
Type
\end{minipage} & \begin{minipage}[b]{\linewidth}\raggedright
Dopant Added
\end{minipage} & \begin{minipage}[b]{\linewidth}\raggedright
Example
\end{minipage} & \begin{minipage}[b]{\linewidth}\raggedright
Majority Carriers
\end{minipage} \\
\midrule\noalign{}
\endhead
\bottomrule\noalign{}
\endlastfoot
P-type & Trivalent (Boron, Gallium) & Silicon doped with Boron &
Holes \\
N-type & Pentavalent (Phosphorus, Arsenic) & Silicon doped with
Phosphorus & Electrons \\
\end{longtable}
}

\textbf{Diagram:}

\begin{center}
\textbf{Mermaid Diagram (Code)}
\begin{verbatim}
{Shaded}
{Highlighting}[]
graph LR
    A[Pure Semiconductor] {-{-}{} B[Add Trivalent Impurity] {-}{-}{} C[P{-}type]}
    A {-{-}{} D[Add Pentavalent Impurity] {-}{-}{} E[N{-}type]}
    style A fill:\#9cf,stroke:\#333
    style C fill:\#f96,stroke:\#333
    style E fill:\#99f,stroke:\#333
{Highlighting}
{Shaded}
\end{verbatim}
\end{center}

\end{solutionbox}
\begin{mnemonicbox}
``Positive has Plus Holes, Negative has Numerous
Electrons'' (PHNE)

\end{mnemonicbox}
\subsection*{Question 3(a) OR [3
marks]}\label{q3a}

\textbf{Define Ripple factor, Peak Inverse Voltage (PIV), Rectification
efficiency.}

\begin{solutionbox}


{\def\LTcaptype{none} % do not increment counter
\vspace{-5pt}
\captionof{table}{Rectifier Terms}
\vspace{-10pt}
\begin{longtable}[]{@{}
  >{\raggedright\arraybackslash}p{(\linewidth - 4\tabcolsep) * \real{0.2222}}
  >{\raggedright\arraybackslash}p{(\linewidth - 4\tabcolsep) * \real{0.4444}}
  >{\raggedright\arraybackslash}p{(\linewidth - 4\tabcolsep) * \real{0.3333}}@{}}
\toprule\noalign{}
\begin{minipage}[b]{\linewidth}\raggedright
Term
\end{minipage} & \begin{minipage}[b]{\linewidth}\raggedright
Definition
\end{minipage} & \begin{minipage}[b]{\linewidth}\raggedright
Formula
\end{minipage} \\
\midrule\noalign{}
\endhead
\bottomrule\noalign{}
\endlastfoot
Ripple Factor & Measure of AC component in rectified output & r =
Vrms(AC)/Vdc \\
Peak Inverse Voltage & Maximum reverse voltage a diode can withstand &
- \\
Rectification Efficiency & Ratio of DC output power to AC input power &
η = (Pdc/Pac) \times 100\% \\
\end{longtable}
}

\textbf{Diagram:}

\begin{center}
\textbf{Mermaid Diagram (Code)}
\begin{verbatim}
{Shaded}
{Highlighting}[]
graph TD
    A[Rectifier Parameters] {-{-}{} B[Ripple Factor]}
    A {-{-}{} C[Peak Inverse Voltage]}
    A {-{-}{} D[Rectification Efficiency]}
    style A fill:\#9cf,stroke:\#333
{Highlighting}
{Shaded}
\end{verbatim}
\end{center}

\end{solutionbox}
\begin{mnemonicbox}
``Ripples Peak Efficiently'' (RPE)

\end{mnemonicbox}
\subsection*{Question 3(b) [3 marks]}\label{q3b}

\textbf{Explain working of Crystal diode.}

\begin{solutionbox}

Crystal diode is a point-contact diode made with a semiconductor
crystal.


{\def\LTcaptype{none} % do not increment counter
\vspace{-5pt}
\captionof{table}{Crystal Diode Properties}
\vspace{-10pt}
\begin{longtable}[]{@{}ll@{}}
\toprule\noalign{}
Property & Description \\
\midrule\noalign{}
\endhead
\bottomrule\noalign{}
\endlastfoot
Construction & Metal point contact on semiconductor crystal \\
Function & Rectification of high frequency signals \\
Application & Radio signal detection \\
\end{longtable}
}

\textbf{Diagram:}

\begin{center}
\textbf{Mermaid Diagram (Code)}
\begin{verbatim}
{Shaded}
{Highlighting}[]
graph LR
    A[RF Signal] {-{-}{} B[Crystal Diode] {-}{-}{} C[Rectified Signal]}
    style A fill:\#9cf,stroke:\#333
    style B fill:\#f96,stroke:\#333
    style C fill:\#9f9,stroke:\#333
{Highlighting}
{Shaded}
\end{verbatim}
\end{center}

\end{solutionbox}
\begin{mnemonicbox}
``Crystal Detects Radio Frequencies'' (CDRF)

\end{mnemonicbox}
\subsection*{Question 3(b) OR [3
marks]}\label{q3b}

\textbf{Explain working of photodiode.}

\begin{solutionbox}

Photodiode converts light energy into electrical current when operated
in reverse bias.


{\def\LTcaptype{none} % do not increment counter
\vspace{-5pt}
\captionof{table}{Photodiode Characteristics}
\vspace{-10pt}
\begin{longtable}[]{@{}ll@{}}
\toprule\noalign{}
Parameter & Behavior \\
\midrule\noalign{}
\endhead
\bottomrule\noalign{}
\endlastfoot
Light condition & Generates electron-hole pairs \\
Reverse current & Increases with light intensity \\
Speed & Fast response time \\
\end{longtable}
}

\textbf{Diagram:}

\begin{center}
\textbf{Mermaid Diagram (Code)}
\begin{verbatim}
{Shaded}
{Highlighting}[]
graph LR
    A[Light] {-{-}{}|Strikes| B[PN Junction]}
    B {-{-}{}|Creates| C[Electron{-}Hole Pairs]}
    C {-{-}{}|Produces| D[Current Flow]}
    style A fill:\#ff9,stroke:\#333
    style D fill:\#9cf,stroke:\#333
{Highlighting}
{Shaded}
\end{verbatim}
\end{center}

\end{solutionbox}
\begin{mnemonicbox}
``Light In, Current Out'' (LICO)

\end{mnemonicbox}
\subsection*{Question 3(c) [4 marks]}\label{q3c}

\textbf{Explain half-wave rectifier with circuit diagram and waveforms.}

\begin{solutionbox}

Half-wave rectifier converts AC to pulsating DC by allowing current flow
only during positive half cycles.

\textbf{Circuit Diagram:}

\begin{center}
\textbf{Mermaid Diagram (Code)}
\begin{verbatim}
{Shaded}
{Highlighting}[]
graph LR
    A[AC Input] {-{-}{-} B[Transformer] {-}{-}{-} C[Diode] {-}{-}{-} D[Load Resistor] {-}{-}{-} E[Ground]}
    E {-{-}{-} A}
    style A fill:\#9cf,stroke:\#333
    style D fill:\#f96,stroke:\#333
{Highlighting}
{Shaded}
\end{verbatim}
\end{center}

\textbf{Waveforms:}

\begin{center}
\textbf{Mermaid Diagram (Code)}
\begin{verbatim}
{Shaded}
{Highlighting}[]
graph TD
    subgraph "Input AC"
    A[+Vp] {-{-}{-} B[(0)] {-}{-}{-} C[{-}Vp]}
    end
    subgraph "Output DC"
    D[+Vp] {-{-}{-} E[(0)] {-}{-}{-} F[(0)]}
    end
    style A fill:\#9cf,stroke:\#333
    style C fill:\#9cf,stroke:\#333
    style D fill:\#f96,stroke:\#333
{Highlighting}
{Shaded}
\end{verbatim}
\end{center}


{\def\LTcaptype{none} % do not increment counter
\vspace{-5pt}
\captionof{table}{Half-Wave Rectifier Properties}
\vspace{-10pt}
\begin{longtable}[]{@{}ll@{}}
\toprule\noalign{}
Parameter & Value \\
\midrule\noalign{}
\endhead
\bottomrule\noalign{}
\endlastfoot
Ripple Factor & 1.21 \\
Efficiency & 40.6\% \\
Output Frequency & Same as input \\
\end{longtable}
}

\end{solutionbox}
\begin{mnemonicbox}
``Half Wave Passes Half'' (HWPH)

\end{mnemonicbox}
\subsection*{Question 3(c) OR [4
marks]}\label{q3c}

\textbf{Explain full-wave rectifier with circuit diagram and waveforms.}

\begin{solutionbox}

Full-wave rectifier converts both halves of AC input to pulsating DC
output.

\textbf{Circuit Diagram (Bridge type):}

\begin{center}
\textbf{Mermaid Diagram (Code)}
\begin{verbatim}
{Shaded}
{Highlighting}[]
graph LR
    A[AC Input] {-{-}{-} B[D1]}
    A {-{-}{-} C[D3]}
    B {-{-}{-} D[D2] {-}{-}{-} E[+Output]}
    C {-{-}{-} F[D4] {-}{-}{-} G[{-}Output]}
    E {-{-}{-} H[Load] {-}{-}{-} G}
    style A fill:\#9cf,stroke:\#333
    style H fill:\#f96,stroke:\#333
{Highlighting}
{Shaded}
\end{verbatim}
\end{center}

\textbf{Waveforms:}

\begin{center}
\textbf{Mermaid Diagram (Code)}
\begin{verbatim}
{Shaded}
{Highlighting}[]
graph TD
    subgraph "Input AC"
    A[+Vp] {-{-}{-} B[(0)] {-}{-}{-} C[{-}Vp] {-}{-}{-} B}
    end
    subgraph "Output DC"
    D[+Vp] {-{-}{-} E[(0)] {-}{-}{-} D}
    end
    style A fill:\#9cf,stroke:\#333
    style C fill:\#9cf,stroke:\#333
    style D fill:\#f96,stroke:\#333
{Highlighting}
{Shaded}
\end{verbatim}
\end{center}


{\def\LTcaptype{none} % do not increment counter
\vspace{-5pt}
\captionof{table}{Full-Wave Rectifier Properties}
\vspace{-10pt}
\begin{longtable}[]{@{}ll@{}}
\toprule\noalign{}
Parameter & Value \\
\midrule\noalign{}
\endhead
\bottomrule\noalign{}
\endlastfoot
Ripple Factor & 0.48 \\
Efficiency & 81.2\% \\
Output Frequency & Twice the input \\
\end{longtable}
}

\end{solutionbox}
\begin{mnemonicbox}
``Full Wave Makes Full Use'' (FWMFU)

\end{mnemonicbox}
\subsection*{Question 3(d) [4 marks]}\label{q3d}

\textbf{Draw and explain VI characteristics of PN junction diode.}

\begin{solutionbox}

\textbf{VI Characteristics:}

\begin{center}
\textbf{Mermaid Diagram (Code)}
\begin{verbatim}
{Shaded}
{Highlighting}[]
graph TD
    subgraph "Forward Bias"
    A[Vf] {-{-}{} B[If]}
    end
    subgraph "Reverse Bias"
    C[Vr] {-{-}{} D[Ir]}
    E[Breakdown] {-{-}{} F[Reverse Current Increases]}
    end
    style A fill:\#9cf,stroke:\#333
    style C fill:\#f96,stroke:\#333
    style E fill:\#f00,stroke:\#333
{Highlighting}
{Shaded}
\end{verbatim}
\end{center}


{\def\LTcaptype{none} % do not increment counter
\vspace{-5pt}
\captionof{table}{PN Junction Diode Characteristics}
\vspace{-10pt}
\begin{longtable}[]{@{}
  >{\raggedright\arraybackslash}p{(\linewidth - 2\tabcolsep) * \real{0.4444}}
  >{\raggedright\arraybackslash}p{(\linewidth - 2\tabcolsep) * \real{0.5556}}@{}}
\toprule\noalign{}
\begin{minipage}[b]{\linewidth}\raggedright
Region
\end{minipage} & \begin{minipage}[b]{\linewidth}\raggedright
Behavior
\end{minipage} \\
\midrule\noalign{}
\endhead
\bottomrule\noalign{}
\endlastfoot
Forward Bias & Current increases exponentially after 0.7V (Si) \\
Reverse Bias & Very small leakage current flows \\
Breakdown & Occurs at high reverse voltage, current increases rapidly \\
\end{longtable}
}

\textbf{Forward Bias}: Positive voltage to P-side, current flows easily
after threshold. \textbf{Reverse Bias}: Positive voltage to N-side, only
small leakage current flows.

\end{solutionbox}
\begin{mnemonicbox}
``Forward Flows, Reverse Restricts'' (FFRR)

\end{mnemonicbox}
\subsection*{Question 3(d) OR [4
marks]}\label{q3d}

\textbf{Write difference between P-type and N-type semiconductor.}

\begin{solutionbox}


{\def\LTcaptype{none} % do not increment counter
\vspace{-5pt}
\captionof{table}{P-type vs N-type Semiconductor}
\vspace{-10pt}
\begin{longtable}[]{@{}
  >{\raggedright\arraybackslash}p{(\linewidth - 4\tabcolsep) * \real{0.3846}}
  >{\raggedright\arraybackslash}p{(\linewidth - 4\tabcolsep) * \real{0.3077}}
  >{\raggedright\arraybackslash}p{(\linewidth - 4\tabcolsep) * \real{0.3077}}@{}}
\toprule\noalign{}
\begin{minipage}[b]{\linewidth}\raggedright
Property
\end{minipage} & \begin{minipage}[b]{\linewidth}\raggedright
P-type
\end{minipage} & \begin{minipage}[b]{\linewidth}\raggedright
N-type
\end{minipage} \\
\midrule\noalign{}
\endhead
\bottomrule\noalign{}
\endlastfoot
Dopant & Trivalent (Boron, Gallium) & Pentavalent (Phosphorus,
Arsenic) \\
Majority Carriers & Holes & Electrons \\
Minority Carriers & Electrons & Holes \\
Electrical Charge & Relatively Positive & Relatively Negative \\
Conductivity & Lower than N-type & Higher than P-type \\
\end{longtable}
}

\textbf{Diagram:}

\begin{center}
\textbf{Mermaid Diagram (Code)}
\begin{verbatim}
{Shaded}
{Highlighting}[]
graph LR
    subgraph "P{-type"}
    A[Silicon] {-{-}{-} B[Boron]}
    C[Holes] {-{-}{-} D[+]}
    end
    subgraph "N{-type"}
    E[Silicon] {-{-}{-} F[Phosphorus]}
    G[Electrons] {-{-}{-} H[{-}]}
    end
    style C fill:\#f96,stroke:\#333
    style G fill:\#9cf,stroke:\#333
{Highlighting}
{Shaded}
\end{verbatim}
\end{center}

\end{solutionbox}
\begin{mnemonicbox}
``Positive has Plus Holes, Negative has Numerous
Electrons'' (PHNE)

\end{mnemonicbox}
\subsection*{Question 4(a) [3 marks]}\label{q4a}

\textbf{Explain the principle of operation of LED.}

\begin{solutionbox}

LED (Light Emitting Diode) emits light when forward biased due to
electron-hole recombination.

\textbf{Principle of Operation:} When forward biased, electrons from
N-side move to P-side and recombine with holes, releasing energy as
photons (light).


{\def\LTcaptype{none} % do not increment counter
\vspace{-5pt}
\captionof{table}{LED Operation}
\vspace{-10pt}
\begin{longtable}[]{@{}ll@{}}
\toprule\noalign{}
Process & Result \\
\midrule\noalign{}
\endhead
\bottomrule\noalign{}
\endlastfoot
Forward bias & Current flows \\
Electron-hole recombination & Energy release \\
Energy band gap & Determines color \\
\end{longtable}
}

\textbf{Diagram:}

\begin{center}
\textbf{Mermaid Diagram (Code)}
\begin{verbatim}
{Shaded}
{Highlighting}[]
graph LR
    A[Forward Bias] {-{-}{}|Causes| B[Current Flow]}
    B {-{-}{}|Creates| C[Electron{-}Hole Recombination]}
    C {-{-}{}|Releases| D[Photons or Light]}
{Highlighting}
{Shaded}
\end{verbatim}
\end{center}

\end{solutionbox}
\begin{mnemonicbox}
``Forward Current Emits Light'' (FCEL)

\end{mnemonicbox}
\subsection*{Question 4(a) OR [3
marks]}\label{q4a}

\textbf{State applications of LED.}

\begin{solutionbox}


{\def\LTcaptype{none} % do not increment counter
\vspace{-5pt}
\captionof{table}{LED Applications}
\vspace{-10pt}
\begin{longtable}[]{@{}ll@{}}
\toprule\noalign{}
Application & Advantage \\
\midrule\noalign{}
\endhead
\bottomrule\noalign{}
\endlastfoot
Display indicators & Low power consumption \\
Digital displays & Varied colors available \\
Lighting & Energy efficient \\
Remote controls & Infrared communication \\
Traffic signals & Long life, high visibility \\
\end{longtable}
}

\textbf{Diagram:}

\begin{center}
\textbf{Mermaid Diagram (Code)}
\begin{verbatim}
{Shaded}
{Highlighting}[]
graph TD
    A[LED Applications] {-{-}{} B[Indicators]}
    A {-{-}{} C[Displays]}
    A {-{-}{} D[Lighting]}
    A {-{-}{} E[Communication]}
    A {-{-}{} F[Signals]}
    style A fill:\#9cf,stroke:\#333
{Highlighting}
{Shaded}
\end{verbatim}
\end{center}

\end{solutionbox}
\begin{mnemonicbox}
``Display Lights In Clever Signals'' (DLICS)

\end{mnemonicbox}
\subsection*{Question 4(b) [4 marks]}\label{q4b}

\textbf{Explain Zener diode as voltage regulator.}

\begin{solutionbox}

Zener diode maintains constant output voltage despite input voltage
fluctuations when operated in reverse breakdown region.

\textbf{Circuit:}

\begin{center}
\textbf{Mermaid Diagram (Code)}
\begin{verbatim}
{Shaded}
{Highlighting}[]
graph LR
    A[Unregulated DC] {-{-}{-} B[Series Resistor] {-}{-}{-} C[Output]}
    C {-{-}{-} D[Zener Diode] {-}{-}{-} E[Ground]}
    C {-{-}{-} F[Load] {-}{-}{-} E}
    style A fill:\#9cf,stroke:\#333
    style C fill:\#9f9,stroke:\#333
    style D fill:\#f96,stroke:\#333
{Highlighting}
{Shaded}
\end{verbatim}
\end{center}

\textbf{Working:}

\begin{itemize}
\tightlist
\item
  Series resistor limits current
\item
  Zener operates in breakdown region
\item
  Maintains constant voltage across load
\end{itemize}


{\def\LTcaptype{none} % do not increment counter
\vspace{-5pt}
\captionof{table}{Zener Regulator Characteristics}
\vspace{-10pt}
\begin{longtable}[]{@{}ll@{}}
\toprule\noalign{}
Parameter & Description \\
\midrule\noalign{}
\endhead
\bottomrule\noalign{}
\endlastfoot
Voltage regulation & Maintains constant output despite input changes \\
Power rating & Must handle power dissipation \\
Temperature stability & Output varies slightly with temperature \\
\end{longtable}
}

\end{solutionbox}
\begin{mnemonicbox}
``Zeners Break to Regulate'' (ZBR)

\end{mnemonicbox}
\subsection*{Question 4(b) OR [4
marks]}\label{q4b}

\textbf{Give limitations of zener voltage regulator.}

\begin{solutionbox}


{\def\LTcaptype{none} % do not increment counter
\vspace{-5pt}
\captionof{table}{Limitations of Zener Voltage Regulator}
\vspace{-10pt}
\begin{longtable}[]{@{}ll@{}}
\toprule\noalign{}
Limitation & Effect \\
\midrule\noalign{}
\endhead
\bottomrule\noalign{}
\endlastfoot
Power Dissipation & Limited by zener power rating \\
Current Capacity & Can handle only small loads \\
Temperature Sensitivity & Output varies with temperature \\
Efficiency & Poor efficiency due to power loss in series resistor \\
Noise & Generates electrical noise \\
\end{longtable}
}

\textbf{Diagram:}

\begin{center}
\textbf{Mermaid Diagram (Code)}
\begin{verbatim}
{Shaded}
{Highlighting}[]
graph TD
    A[Zener Limitations] {-{-}{} B[Power Limits]}
    A {-{-}{} C[Current Limits]}
    A {-{-}{} D[Temperature Effects]}
    A {-{-}{} E[Efficiency Issues]}
    A {-{-}{} F[Noise Generation]}
    style A fill:\#f96,stroke:\#333
{Highlighting}
{Shaded}
\end{verbatim}
\end{center}

\end{solutionbox}
\begin{mnemonicbox}
``Power Current Temperature Efficiency Noise''
(PCTEN)

\end{mnemonicbox}
\subsection*{Question 4(c) [7 marks]}\label{q4c}

\textbf{Discuss the necessity of filter circuit in rectifier. List
various types of filter circuits used in rectifier and explain any one
with neat diagram.}

\begin{solutionbox}

\textbf{Necessity of Filter Circuit:} Rectifier output contains AC
ripple that must be removed for smoother DC. Filters reduce these
ripples to provide steady DC output.

\textbf{Types of Filter Circuits:}

\begin{enumerate}
\tightlist
\item
  Capacitor filter (Shunt capacitor)
\item
  LC filter
\item
  π-filter (Pi-filter)
\item
  RC filter
\end{enumerate}

\textbf{Explanation of Capacitor Filter:}

\textbf{Circuit Diagram:}

\begin{center}
\textbf{Mermaid Diagram (Code)}
\begin{verbatim}
{Shaded}
{Highlighting}[]
graph LR
    A[Rectifier Output] {-{-}{-} B[+]}
    B {-{-}{-} C[Load]}
    B {-{-}{-} D[Capacitor]}
    C {-{-}{-} E[Ground]}
    D {-{-}{-} E}
    style A fill:\#9cf,stroke:\#333
    style C fill:\#f96,stroke:\#333
    style D fill:\#9f9,stroke:\#333
{Highlighting}
{Shaded}
\end{verbatim}
\end{center}

\textbf{Working:}

\begin{itemize}
\tightlist
\item
  Capacitor charges during voltage peaks
\item
  Discharges slowly during voltage drops
\item
  Maintains output voltage between peaks
\item
  Reduces ripple voltage
\end{itemize}


{\def\LTcaptype{none} % do not increment counter
\vspace{-5pt}
\captionof{table}{Capacitor Filter Characteristics}
\vspace{-10pt}
\begin{longtable}[]{@{}ll@{}}
\toprule\noalign{}
Parameter & Effect \\
\midrule\noalign{}
\endhead
\bottomrule\noalign{}
\endlastfoot
Capacitance value & Higher value gives less ripple \\
Ripple reduction & Typically reduces by 70-80\% \\
Load current & Higher load current causes more ripple \\
Frequency & Higher frequency is easier to filter \\
\end{longtable}
}

\textbf{Waveforms:}

\begin{center}
\textbf{Mermaid Diagram (Code)}
\begin{verbatim}
{Shaded}
{Highlighting}[]
graph TD
    subgraph "Rectifier Output"
    A[Pulsating DC]
    end
    subgraph "Filter Output"
    B[Smoother DC]
    end
    style A fill:\#f96,stroke:\#333
    style B fill:\#9f9,stroke:\#333
{Highlighting}
{Shaded}
\end{verbatim}
\end{center}

\end{solutionbox}
\begin{mnemonicbox}
``Capacitors Hold Voltage During Drops'' (CHVDD)

\end{mnemonicbox}
\subsection*{Question 5(a) [3 marks]}\label{q5a}

\textbf{Define e-waste. List common e-waste items.}

\begin{solutionbox}

E-waste refers to discarded electronic devices and components that have
reached the end of their useful life.


{\def\LTcaptype{none} % do not increment counter
\vspace{-5pt}
\captionof{table}{Common E-waste Items}
\vspace{-10pt}
\begin{longtable}[]{@{}ll@{}}
\toprule\noalign{}
Category & Examples \\
\midrule\noalign{}
\endhead
\bottomrule\noalign{}
\endlastfoot
Computing devices & Computers, laptops, tablets \\
Communication devices & Mobile phones, telephones \\
Home appliances & TVs, refrigerators, washing machines \\
Electronic components & Circuit boards, batteries, cables \\
Office equipment & Printers, scanners, copiers \\
\end{longtable}
}

\textbf{Diagram:}

\begin{center}
\textbf{Mermaid Diagram (Code)}
\begin{verbatim}
{Shaded}
{Highlighting}[]
graph TD
    A[E{-waste] {-}{-}{} B[Computing]}
    A {-{-}{} C[Communication]}
    A {-{-}{} D[Home Appliances]}
    A {-{-}{} E[Components]}
    A {-{-}{} F[Office Equipment]}
    style A fill:\#f96,stroke:\#333
{Highlighting}
{Shaded}
\end{verbatim}
\end{center}

\end{solutionbox}
\begin{mnemonicbox}
``Computers, Communication, Components, Home
Appliances'' (CCCHA)

\end{mnemonicbox}
\subsection*{Question 5(b) [3 marks]}\label{q5b}

\textbf{State and explain various strategies of e-waste management.}

\begin{solutionbox}


{\def\LTcaptype{none} % do not increment counter
\vspace{-5pt}
\captionof{table}{E-waste Management Strategies}
\vspace{-10pt}
\begin{longtable}[]{@{}
  >{\raggedright\arraybackslash}p{(\linewidth - 2\tabcolsep) * \real{0.4348}}
  >{\raggedright\arraybackslash}p{(\linewidth - 2\tabcolsep) * \real{0.5652}}@{}}
\toprule\noalign{}
\begin{minipage}[b]{\linewidth}\raggedright
Strategy
\end{minipage} & \begin{minipage}[b]{\linewidth}\raggedright
Description
\end{minipage} \\
\midrule\noalign{}
\endhead
\bottomrule\noalign{}
\endlastfoot
Reduce & Minimize purchase of new electronics \\
Reuse & Extend life through repair and repurposing \\
Recycle & Process e-waste to recover valuable materials \\
Responsible disposal & Use authorized e-waste collection centers \\
Extended producer responsibility & Manufacturers take back end-of-life
products \\
\end{longtable}
}

\textbf{Diagram:}

\begin{center}
\textbf{Mermaid Diagram (Code)}
\begin{verbatim}
{Shaded}
{Highlighting}[]
graph TD
    A[E{-waste Management] {-}{-}{} B[Reduce]}
    A {-{-}{} C[Reuse]}
    A {-{-}{} D[Recycle]}
    A {-{-}{} E[Responsible Disposal]}
    A {-{-}{} F[Extended Producer Responsibility]}
    style A fill:\#9cf,stroke:\#333
{Highlighting}
{Shaded}
\end{verbatim}
\end{center}

\end{solutionbox}
\begin{mnemonicbox}
``3R's

\end{mnemonicbox}
\subsection*{Question 5(c) [4 marks]}\label{q5c}

\textbf{Explain transistor as switch.}

\begin{solutionbox}

Transistor can function as an electronic switch by operating in either
cutoff (OFF) or saturation (ON) region.


{\def\LTcaptype{none} % do not increment counter
\vspace{-5pt}
\captionof{table}{Transistor Switch Operation}
\vspace{-10pt}
\begin{longtable}[]{@{}
  >{\raggedright\arraybackslash}p{(\linewidth - 4\tabcolsep) * \real{0.2500}}
  >{\raggedright\arraybackslash}p{(\linewidth - 4\tabcolsep) * \real{0.3929}}
  >{\raggedright\arraybackslash}p{(\linewidth - 4\tabcolsep) * \real{0.3571}}@{}}
\toprule\noalign{}
\begin{minipage}[b]{\linewidth}\raggedright
State
\end{minipage} & \begin{minipage}[b]{\linewidth}\raggedright
Condition
\end{minipage} & \begin{minipage}[b]{\linewidth}\raggedright
Behavior
\end{minipage} \\
\midrule\noalign{}
\endhead
\bottomrule\noalign{}
\endlastfoot
OFF (Cutoff) & Base current = 0 & No collector current flows \\
ON (Saturation) & Base current sufficient & Maximum collector current
flows \\
\end{longtable}
}

\textbf{Circuit Diagram:}

\begin{center}
\textbf{Mermaid Diagram (Code)}
\begin{verbatim}
{Shaded}
{Highlighting}[]
graph LR
    A[+Vcc] {-{-}{-} B[Rc] {-}{-}{-} C[Collector]}
    C {-{-}{-} D[Emitter] {-}{-}{-} E[Ground]}
    F[Vin] {-{-}{-} G[Rb] {-}{-}{-} H[Base]}
    H {-{-}{-} D}
    style F fill:\#9cf,stroke:\#333
    style A fill:\#f96,stroke:\#333
{Highlighting}
{Shaded}
\end{verbatim}
\end{center}

\textbf{Working:}

\begin{itemize}
\tightlist
\item
  When input is HIGH: Transistor saturates, acts like closed switch
\item
  When input is LOW: Transistor cuts off, acts like open switch
\end{itemize}

\end{solutionbox}
\begin{mnemonicbox}
``No Base No Current, Apply Base Connect Circuit''
(NBNC-ABC)

\end{mnemonicbox}
\subsection*{Question 5(d) [4 marks]}\label{q5d}

\textbf{Derive relation between α and β for CE configuration of
transistor.}

\begin{solutionbox}

In transistors, α (alpha) and β (beta) are current gain parameters.

\textbf{Definitions:}

\begin{itemize}
\tightlist
\item
  α = IC/IE (Common Base current gain)
\item
  β = IC/IB (Common Emitter current gain)
\end{itemize}

\textbf{Derivation:} Since IE = IC + IB, we can write: α = IC/IE =
IC/(IC + IB)

Dividing numerator and denominator by IB: α = (IC/IB)/[(IC/IB) + 1]
= β/(β + 1)

Therefore: β = α/(1-α)


{\def\LTcaptype{none} % do not increment counter
\vspace{-5pt}
\captionof{table}{Relationship between α and β}
\vspace{-10pt}
\begin{longtable}[]{@{}lll@{}}
\toprule\noalign{}
Parameter & Formula & Typical Range \\
\midrule\noalign{}
\endhead
\bottomrule\noalign{}
\endlastfoot
α from β & α = β/(β+1) & 0.9 to 0.99 \\
β from α & β = α/(1-α) & 50 to 300 \\
\end{longtable}
}

\textbf{Diagram:}

\begin{center}
\textbf{Mermaid Diagram (Code)}
\begin{verbatim}
{Shaded}
{Highlighting}[]
graph TD
    A[alpha = IC divided by IE] {-{-}{-} B[beta = IC divided by IB]}
    C[beta = alpha divided by 1 minus alpha] {-{-}{-} D[alpha = beta divided by beta plus 1]}

    style A fill:\#9cf,stroke:\#333
    style B fill:\#f96,stroke:\#333
{Highlighting}
{Shaded}
\end{verbatim}
\end{center}

\end{solutionbox}
\begin{mnemonicbox}
``Beta equals Alpha divided by One minus Alpha''
(BAOA)

\end{mnemonicbox}

\end{document}
