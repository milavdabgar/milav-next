\documentclass[10pt,a4paper]{article}

% content/resources/templates/preamble.tex
\usepackage[margin=0.6in]{geometry}
\author{Milav Dabgar}
\usepackage{amsmath,amssymb,amsthm}
\usepackage{booktabs}
\usepackage{multirow}
\usepackage{xcolor}
\usepackage{tcolorbox}
\tcbuselibrary{breakable,skins}
\usepackage[colorlinks=true,linkcolor=blue]{hyperref}
\usepackage{titlesec}
\usepackage{enumitem}
\usepackage{tikz}
\usepackage{pgfplots}
\usepackage{circuitikz}
\usepackage[version=4]{mhchem}
\usepackage{longtable}
\usepackage{array}
\usepackage{float}
\usepackage{caption}
\usepackage{listings}

\lstset{
  basicstyle=\small\ttfamily,
  breaklines=true,
  breakatwhitespace=false,
  postbreak=\mbox{\textcolor{red}{$\hookrightarrow$}\space},
  float=false,
  numbers=left,
  numberstyle=\tiny\color{gray},
  numbersep=10pt,
  xleftmargin=2em,
  keywordstyle=\color{blue},
  commentstyle=\color{green!60!black},
  stringstyle=\color{purple},
  backgroundcolor=\color{gray!5},
  showstringspaces=false,
  tabsize=2,
  captionpos=b,
  keepspaces=true,
  columns=flexible
}

\pgfplotsset{compat=1.18}
\usetikzlibrary{shapes,arrows,positioning,calc,patterns,decorations.pathmorphing,decorations.markings,arrows.meta}

% Color scheme
\definecolor{headcolor}{RGB}{0,102,204}
\definecolor{keycolor}{RGB}{220,20,60}
\definecolor{solutioncolor}{RGB}{34,139,34}
\definecolor{mnemoniccolor}{RGB}{148,0,211}
\definecolor{codecolor}{RGB}{0,0,100}

% Spacing
\setlength{\parskip}{3pt}
\setlist[itemize]{nosep}
\setlist[enumerate]{nosep}

% Title formatting
\titleformat{\section}{\Large\bfseries\color{headcolor}}{\thesection}{1em}{}
\titleformat{\subsection}{\large\bfseries\color{headcolor}}{\thesubsection}{1em}{}

% Pandoc tightlist compatibility
\providecommand{\tightlist}{%
  \setlength{\itemsep}{0pt}\setlength{\parskip}{0pt}}

% Pandoc longtable compatibility
\newcounter{none}
\def\thenone{}


% content/resources/templates/gujarati-boxes.tex
\usepackage{fontspec}
\usepackage{polyglossia}

% Set Gujarati as main language (document is primarily in Gujarati)
% Note: gloss-gujarati.ldf doesn't exist in polyglossia, but it will use hyphenation patterns
\setdefaultlanguage{gujarati}
\setotherlanguage{english}

% Configure Gujarati font properly
% Use Language=Default to prevent polyglossia from trying to add language-specific features
% that don't exist for Gujarati, which causes "empty feature" warnings
\newfontfamily\gujaratifont[Script=Gujarati,AutoFakeBold=2.5,AutoFakeSlant=0.3]{Noto Sans Gujarati}
\setmainfont[Script=Gujarati,AutoFakeBold=2.5,AutoFakeSlant=0.3]{Noto Sans Gujarati}
% Use Noto Sans Gujarati for monospace to support Gujarati in text
\setmonofont[Scale=0.9]{Noto Sans Gujarati}

% Configure English to use the same font
\newfontfamily\englishfont[Script=Gujarati,AutoFakeBold=2.5,AutoFakeSlant=0.3]{Noto Sans Gujarati}

% Translations for polyglossia
\gappto\captionsgujarati{
  \renewcommand{\tablename}{કોષ્ટક}
  \renewcommand{\figurename}{આકૃતિ}
}

% Helper for TikZ nodes to ensure Gujarati font
\newcommand{\gu}[1]{{\gujaratifont #1}}

% Custom environments
\newtcolorbox{solutionbox}{
    breakable,
    enhanced,
    colback=solutioncolor!5!white,
    colframe=solutioncolor!75!black,
    fonttitle=\bfseries,
    title=જવાબ
}

\newtcolorbox{solutionboxnobreak}{
 colback=solutioncolor!5!white,
 colframe=solutioncolor!75!black,
 fonttitle=\bfseries,
 title=જવાબ
}

\newtcolorbox{keyformula}{
 breakable,
 enhanced,
 colback=keycolor!5!white,
 colframe=keycolor!75!black,
 fonttitle=\bfseries,
 title=રાસાયણિક સમીકરણ/સૂત્ર
}

\newtcolorbox{mnemonicbox}{
 breakable,
 enhanced,
 colback=mnemoniccolor!5!white,
 colframe=mnemoniccolor!75!black,
 fonttitle=\bfseries,
 title=મેમરી ટ્રીક
}


\begin{document}

\begin{center}
{\Huge\bfseries\color{headcolor} Subject Name (Gujarati)}\\[5pt]
{\LARGE 4311102 -- Summer 2023}\\[3pt]
{\large Semester 1 Study Material}\\[3pt]
{\normalsize\textit{Detailed Solutions and Explanations}}
\end{center}

\vspace{10pt}

\subsection*{પ્રશ્ન 1(a) [3
ગુણ]}\label{q1a}

\textbf{સક્રિય અને નિષ્ક્રિય ઘટકોને વ્યાખ્યાયિત કરો.}

\begin{solutionbox}

{\def\LTcaptype{none} % do not increment counter
\begin{longtable}[]{@{}
  >{\raggedright\arraybackslash}p{(\linewidth - 2\tabcolsep) * \real{0.4865}}
  >{\raggedright\arraybackslash}p{(\linewidth - 2\tabcolsep) * \real{0.5135}}@{}}
\toprule\noalign{}
\begin{minipage}[b]{\linewidth}\raggedright
સક્રિય ઘટકો
\end{minipage} & \begin{minipage}[b]{\linewidth}\raggedright
નિષ્ક્રિય ઘટકો
\end{minipage} \\
\midrule\noalign{}
\endhead
\bottomrule\noalign{}
\endlastfoot
• કામ કરવા માટે બાહ્ય પાવર સ્ત્રોતની જરૂર પડે છે & • બાહ્ય પાવર સ્ત્રોતની જરૂર પડતી
નથી \\
• ઇલેક્ટ્રિકલ સિગ્નલને મોટા કરી શકે છે અને પ્રોસેસ કરી શકે છે & • સિગ્નલને મોટા કરી
શકતા નથી અથવા પ્રોસેસ કરી શકતા નથી \\
• ઉદાહરણ: ટ્રાન્ઝિસ્ટર, ડાયોડ, ICs & • ઉદાહરણ: રેસિસ્ટર, કેપેસિટર, ઇન્ડક્ટર \\
\end{longtable}
}

\end{solutionbox}
\begin{mnemonicbox}
``APE'' - Active needs Power to Enhance signals

\end{mnemonicbox}
\subsection*{પ્રશ્ન 1(b) [4
ગુણ]}\label{q1b}

\textbf{વપરાયેલ સામગ્રી પર આધારિત કેપેસિટરના પ્રકારો વર્ણવો.}

\begin{solutionbox}

\textbf{ટેબલ: સામગ્રી આધારિત કેપેસિટરના પ્રકારો}

{\def\LTcaptype{none} % do not increment counter
\begin{longtable}[]{@{}
  >{\raggedright\arraybackslash}p{(\linewidth - 4\tabcolsep) * \real{0.2692}}
  >{\raggedright\arraybackslash}p{(\linewidth - 4\tabcolsep) * \real{0.3077}}
  >{\raggedright\arraybackslash}p{(\linewidth - 4\tabcolsep) * \real{0.4231}}@{}}
\toprule\noalign{}
\begin{minipage}[b]{\linewidth}\raggedright
મટીરિયલ ટાઇપ
\end{minipage} & \begin{minipage}[b]{\linewidth}\raggedright
કેપેસિટર પ્રકાર
\end{minipage} & \begin{minipage}[b]{\linewidth}\raggedright
સામાન્ય ઉપયોગો
\end{minipage} \\
\midrule\noalign{}
\endhead
\bottomrule\noalign{}
\endlastfoot
સેરામિક & સેરામિક ડિસ્ક, મલ્ટિલેયર & બાયપાસ, કપલિંગ, હાઈ ફ્રીક્વન્સી \\
પ્લાસ્ટિક ફિલ્મ & પોલિએસ્ટર, પોલિપ્રોપિલીન, ટેફ્લોન & ટાઈમિંગ, ફિલ્ટરિંગ,
પ્રીસિઝન \\
ઇલેક્ટ્રોલિટિક & એલ્યુમિનિયમ, ટેન્ટાલમ & પાવર સપ્લાય, DC બ્લોકિંગ, હાઈ કેપેસિટન્સ \\
પેપર & પેપર ડાયલેક્ટ્રિક & જૂના ઉપકરણોમાં, હવે સામાન્ય નથી \\
માઈકા & સિલ્વર્ડ માઈકા & હાઈ પ્રીસિઝન RF સર્કિટ્સ \\
ગ્લાસ & ગ્લાસ ડાયલેક્ટ્રિક & હાઈ વોલ્ટેજ એપ્લિકેશન \\
\end{longtable}
}

\end{solutionbox}
\begin{mnemonicbox}
``CEPPMG'' - Ceramic Electrolytic Paper Plastic Mica
Glass

\end{mnemonicbox}
\subsection*{પ્રશ્ન 1(c) [7
ગુણ]}\label{q1c}

\textbf{રેસિસ્ટર કલર કોડિંગ ટેકનિક ઉદાહરણ સાથે સમજાવો.}

\begin{solutionbox}

રેસિસ્ટર કલર કોડ રેસિસ્ટન્સ મૂલ્ય, ટોલરન્સ અને વિશ્વસનીયતા દર્શાવવા માટે રંગીન બેન્ડનો
ઉપયોગ કરે છે.

\textbf{ટેબલ: સ્ટાન્ડર્ડ રેસિસ્ટર કલર કોડ}

{\def\LTcaptype{none} % do not increment counter
\begin{longtable}[]{@{}llll@{}}
\toprule\noalign{}
રંગ & અંક મૂલ્ય & મલ્ટિપ્લાયર & ટોલરન્સ \\
\midrule\noalign{}
\endhead
\bottomrule\noalign{}
\endlastfoot
કાળો & 0 & \times10^{0} (1) & - \\
બ્રાઉન & 1 & \times10^{1} (10) & \pm1\% \\
લાલ & 2 & \times10^{2} (100) & \pm2\% \\
નારંગી & 3 & \times10^{3} (1,000) & - \\
પીળો & 4 & \times10^{4} (10,000) & - \\
લીલો & 5 & \times10^{5} (100,000) & \pm0.5\% \\
વાદળી & 6 & \times10^{6} (1,000,000) & \pm0.25\% \\
વાયોલેટ & 7 & \times10^{7} (10,000,000) & \pm0.1\% \\
ગ્રે & 8 & \times10^{8} (100,000,000) & \pm0.05\% \\
સફેદ & 9 & \times10^{9} (1,000,000,000) & - \\
સોનેરી & - & \times0.1 (0.1) & \pm5\% \\
ચાંદી & - & \times0.01 (0.01) & \pm10\% \\
\end{longtable}
}

\textbf{ઉદાહરણ 1:} લાલ-વાયોલેટ-નારંગી-સોનેરી

\begin{itemize}
\tightlist
\item
  1લી બેન્ડ (લાલ) = 2
\item
  2જી બેન્ડ (વાયોલેટ) = 7
\item
  3જી બેન્ડ (નારંગી) = \times1,000
\item
  4થી બેન્ડ (સોનેરી) = \pm5\% ટોલરન્સ
\item
  મૂલ્ય: 27 \times 1,000 = 27,000Ω = 27kΩ \pm5\%
\end{itemize}

\textbf{ઉદાહરણ 2:} બ્રાઉન-બ્લેક-યલો-સિલ્વર

\begin{itemize}
\tightlist
\item
  1લી બેન્ડ (બ્રાઉન) = 1
\item
  2જી બેન્ડ (બ્લેક) = 0
\item
  3જી બેન્ડ (યલો) = \times10,000
\item
  4થી બેન્ડ (સિલ્વર) = \pm10\% ટોલરન્સ
\item
  મૂલ્ય: 10 \times 10,000 = 100,000Ω = 100kΩ \pm10\%
\end{itemize}

\begin{verbatim}
flowchart LR
    A[1st Band{br /First Digit] {-}{-} B[2nd Bandbr /Second Digit]}
    B {-{-} C[3rd Bandbr /Multiplier]}
    C {-{-} D[4th Bandbr /Tolerance]}
    style A fill:\#f96,stroke:\#333
    style B fill:\#69f,stroke:\#333
    style C fill:\#f90,stroke:\#333
    style D fill:\#fc0,stroke:\#333
\end{verbatim}

\end{solutionbox}
\begin{mnemonicbox}
``BBROY Great Britain Very Good Wife'' - કલર 0-9 માટે
(Black Brown Red Orange Yellow Green Blue Violet Gray White)

\end{mnemonicbox}
\subsection*{પ્રશ્ન 1(c) OR [7
ગુણ]}\label{q1c}

\textbf{LDR નું બાંધકામ, કાર્યકારી લાક્ષણિકતાઓ અને એપ્લિકેશન સમજાવો.}

\begin{solutionbox}

\textbf{લાઈટ ડિપેન્ડન્ટ રેસિસ્ટર (LDR)}

{\def\LTcaptype{none} % do not increment counter
\begin{longtable}[]{@{}
  >{\raggedright\arraybackslash}p{(\linewidth - 2\tabcolsep) * \real{0.3810}}
  >{\raggedright\arraybackslash}p{(\linewidth - 2\tabcolsep) * \real{0.6190}}@{}}
\toprule\noalign{}
\begin{minipage}[b]{\linewidth}\raggedright
પાસું
\end{minipage} & \begin{minipage}[b]{\linewidth}\raggedright
વર્ણન
\end{minipage} \\
\midrule\noalign{}
\endhead
\bottomrule\noalign{}
\endlastfoot
\textbf{બાંધકામ} & • સેમિકન્ડક્ટર મટીરિયલ (કેડમિયમ સલ્ફાઈડ) ઝિગઝેગ પેટર્નમાં
ડિપોઝિટ• પ્રકાશને પસાર થવા દેવા માટે પારદર્શક કેસમાં પેકેજિંગ• સેમિકન્ડક્ટર સાથે બે
ટર્મિનલ જોડાયેલા \\
\textbf{કાર્ય સિદ્ધાંત} & • જ્યારે પ્રકાશની તીવ્રતા વધે છે ત્યારે પ્રતિરોધ ઘટે છે•
ફોટોન્સ સેમિકન્ડક્ટર સામગ્રીમાં ઇલેક્ટ્રોન્સ મુક્ત કરે છે• વધુ પ્રકાશ = વધુ મુક્ત ઇલેક્ટ્રોન્સ
= ઓછો પ્રતિરોધ \\
\textbf{લાક્ષણિકતાઓ} & • અંધકારમાં ઉચ્ચ પ્રતિરોધ (MΩ રેન્જ)• તેજ પ્રકાશમાં ઓછો
પ્રતિરોધ (100-5000Ω)• પ્રકાશની તીવ્રતા પ્રત્યે નોન-લીનિયર પ્રતિક્રિયા• ધીમી
પ્રતિક્રિયા સમય (દસ મિલિસેકન્ડ) \\
\textbf{ઉપયોગો} & • ઓટોમેટિક સ્ટ્રીટ લાઈટ્સ• કેમેરામાં લાઈટ મીટર• ચોર એલાર્મ
સિસ્ટમ• ડિસ્પ્લેમાં ઓટોમેટિક બ્રાઈટનેસ કંટ્રોલ \\
\end{longtable}
}

\begin{center}
\textbf{Mermaid Diagram (Code)}
\begin{verbatim}
{Shaded}
{Highlighting}[]
graph LR
    A[More Light] {-{-}{}|Releases electrons| B[More Free Electrons]}
    B {-{-}{} C[Lower Resistance]}
    D[Less Light] {-{-}{}|Fewer electrons released| E[Fewer Free Electrons]}
    E {-{-}{} F[Higher Resistance]}
{Highlighting}
{Shaded}
\end{verbatim}
\end{center}

\end{solutionbox}
\begin{mnemonicbox}
``MOLD'' - More light On, Less resistance Down

\end{mnemonicbox}
\subsection*{પ્રશ્ન 2(a) [3
ગુણ]}\label{q2a}

\textbf{સામગ્રીના આધારે રેસિસ્ટરને વર્ગીકૃત કરો.}

\begin{solutionbox}

\textbf{ટેબલ: સામગ્રી આધારિત રેસિસ્ટર વર્ગીકરણ}

{\def\LTcaptype{none} % do not increment counter
\begin{longtable}[]{@{}
  >{\raggedright\arraybackslash}p{(\linewidth - 4\tabcolsep) * \real{0.3415}}
  >{\raggedright\arraybackslash}p{(\linewidth - 4\tabcolsep) * \real{0.4146}}
  >{\raggedright\arraybackslash}p{(\linewidth - 4\tabcolsep) * \real{0.2439}}@{}}
\toprule\noalign{}
\begin{minipage}[b]{\linewidth}\raggedright
મટીરિયલ ટાઈપ
\end{minipage} & \begin{minipage}[b]{\linewidth}\raggedright
લાક્ષણિકતાઓ
\end{minipage} & \begin{minipage}[b]{\linewidth}\raggedright
ઉદાહરણો
\end{minipage} \\
\midrule\noalign{}
\endhead
\bottomrule\noalign{}
\endlastfoot
કાર્બન કોમ્પોઝિશન & ઓછી કિંમત, નોઈઝી, નબળી ટોલરન્સ & સામાન્ય હેતુના રેસિસ્ટર \\
કાર્બન ફિલ્મ & કાર્બન કોમ્પોઝિશન કરતાં વધુ સારી સ્થિરતા & ઓડિયો ઉપકરણો, સામાન્ય
સર્કિટ \\
મેટલ ફિલ્મ & ઉત્તમ સ્થિરતા, ઓછો નોઈઝ & પ્રિસિઝન સર્કિટ, ઈન્સ્ટ્રુમેન્ટેશન \\
મેટલ ઓક્સાઈડ & ઉચ્ચ સ્થિરતા, ગરમી પ્રતિરોધક & પાવર સપ્લાય, હાઈ વોલ્ટેજ સર્કિટ \\
વાયર વાઉન્ડ & ઉચ્ચ પાવર રેટિંગ, ઇન્ડક્ટિવ & પાવર સર્કિટ, હીટિંગ એલિમેન્ટ \\
થિક \& થિન ફિલ્મ & નાના કદ, સારી સ્થિરતા & સરફેસ માઉન્ટ ઍપ્લિકેશન \\
\end{longtable}
}

\end{solutionbox}
\begin{mnemonicbox}
``CMMWTF'' - Carbon Makes Much Wire To Form
resistors

\end{mnemonicbox}
\subsection*{પ્રશ્ન 2(b) [4
ગુણ]}\label{q2b}

\textbf{આપેલ રંગ કોડ માટે રેસિસ્ટરની કિંમતની ગણતરી કરો. - (i) બ્રાઉન, બ્લેક, યલો,
ગોલ્ડન (ii) યલો, વાયોલેટ, રેડ, સિલ્વર}

\begin{solutionbox}

\textbf{ભાગ (i): બ્રાઉન, બ્લેક, યલો, ગોલ્ડન}

\begin{itemize}
\tightlist
\item
  1લી બેન્ડ (બ્રાઉન) = 1
\item
  2જી બેન્ડ (બ્લેક) = 0
\item
  3જી બેન્ડ (યલો) = \times10,000
\item
  4થી બેન્ડ (ગોલ્ડન) = \pm5\% ટોલરન્સ
\end{itemize}

\textbf{ગણતરી:} મૂલ્ય = 10 \times 10,000 = 100,000Ω = 100kΩ \pm5\%

\textbf{ભાગ (ii): યલો, વાયોલેટ, રેડ, સિલ્વર}

\begin{itemize}
\tightlist
\item
  1લી બેન્ડ (યલો) = 4
\item
  2જી બેન્ડ (વાયોલેટ) = 7
\item
  3જી બેન્ડ (રેડ) = \times100
\item
  4થી બેન્ડ (સિલ્વર) = \pm10\% ટોલરન્સ
\end{itemize}

\textbf{ગણતરી:} મૂલ્ય = 47 \times 100 = 4,700Ω = 4.7kΩ \pm10\%

\end{solutionbox}
\begin{mnemonicbox}
``BBROY Great Britain Very Good Wife'' રંગ અનુક્રમ 0-9
માટે

\end{mnemonicbox}
\subsection*{પ્રશ્ન 2(c) [7
ગુણ]}\label{q2c}

\textbf{ઇલેક્ટ્રોલિટીક કેપેસિટર્સનું બાંધકામ અને સંચાલન સમજાવો.}

\begin{solutionbox}

\textbf{ઇલેક્ટ્રોલિટિક કેપેસિટર બાંધકામ અને કાર્યપ્રણાલી}

{\def\LTcaptype{none} % do not increment counter
\begin{longtable}[]{@{}ll@{}}
\toprule\noalign{}
ઘટક & વર્ણન \\
\midrule\noalign{}
\endhead
\bottomrule\noalign{}
\endlastfoot
\textbf{એનોડ} & ઓક્સાઇડ લેયર (ડાયલેક્ટ્રિક) સાથે એલ્યુમિનિયમ અથવા ટેન્ટાલમ ફોઇલ \\
\textbf{કેથોડ} & ઇલેક્ટ્રોલાઇટ (લિક્વિડ, પેસ્ટ અથવા સોલિડ) અને મેટલ ફોઇલ \\
\textbf{સેપરેટર} & ઇલેક્ટ્રોલાઇટમાં ભીંજવેલું પેપર \\
\textbf{કેસિંગ} & ઇન્સ્યુલેટિંગ સ્લીવ સાથે એલ્યુમિનિયમ કેન \\
\textbf{ટર્મિનલ} & પોઝિટિવ (+) અને નેગેટિવ (-) લીડ્સ \\
\end{longtable}
}

\textbf{કાર્યપ્રણાલી:}

\begin{enumerate}
\tightlist
\item
  એનોડ પર ઓક્સાઇડ લેયર અત્યંત પાતળા ડાયલેક્ટ્રિક તરીકે કાર્ય કરે છે
\item
  મોટા સરફેસ એરિયા અને પાતળા ડાયલેક્ટ્રિકથી ઉચ્ચ કેપેસિટન્સ બને છે
\item
  જ્યારે DC વોલ્ટેજ (યોગ્ય પોલારિટી સાથે) સાથે જોડાય છે, ત્યારે ચાર્જ એકત્રિત થાય છે
\item
  પોઝિટિવ પ્લેટ (+) નેગેટિવ ચાર્જને આકર્ષે છે; નેગેટિવ પ્લેટ (-) પોઝિટિવ ચાર્જને આકર્ષે છે
\end{enumerate}

\begin{center}
\textbf{Mermaid Diagram (Code)}
\begin{verbatim}
{Shaded}
{Highlighting}[]
graph LR
    A[Aluminum Foil{br /{}Anode] {-}{-}{} B[Oxide Layer{}br /{}Dielectric]}
    B {-{-}{} C[Electrolyte{}br /{}Cathode]}
    C {-{-}{} D[Aluminum Foil{}br /{}Terminal Connection]}
    style A fill:\#fc9,stroke:\#333
    style B fill:\#9cf,stroke:\#333
    style C fill:\#cfc,stroke:\#333
    style D fill:\#fc9,stroke:\#333
{Highlighting}
{Shaded}
\end{verbatim}
\end{center}

\textbf{મુખ્ય લાક્ષણિકતાઓ:}

\begin{itemize}
\tightlist
\item
  \textbf{પોલારિટી}: યોગ્ય રીતે જોડાવું જરૂરી (+/-)
\item
  \textbf{ઉચ્ચ કેપેસિટન્સ}: 1μF થી હજારો μF
\item
  \textbf{વોલ્ટેજ મર્યાદાઓ}: વધારે થવાથી બ્રેકડાઉન
\item
  \textbf{લીકેજ કરંટ}: અન્ય કેપેસિટર પ્રકારો કરતાં વધારે
\end{itemize}

\end{solutionbox}
\begin{mnemonicbox}
``PAVE'' - Polarized Aluminum with Very high
capacitance and Electrolyte

\end{mnemonicbox}
\subsection*{પ્રશ્ન 2(a) OR [3
ગુણ]}\label{q2a}

\textbf{રેક્ટિફાયરમાં ફિલ્ટર સર્કિટનું મહત્વ જણાવો.}

\begin{solutionbox}

\textbf{રેક્ટિફાયરમાં ફિલ્ટર સર્કિટનું મહત્વ}

{\def\LTcaptype{none} % do not increment counter
\begin{longtable}[]{@{}ll@{}}
\toprule\noalign{}
કાર્ય & વર્ણન \\
\midrule\noalign{}
\endhead
\bottomrule\noalign{}
\endlastfoot
\textbf{સ્મૂધિંગ} & રિપલ્સને ઘટાડીને પલ્સેટિંગ DCને સ્મૂધ DCમાં રૂપાંતરિત કરે છે \\
\textbf{વોલ્ટેજ સ્ટેબિલાઇઝેશન} & ઇનપુટ ફ્લક્ચુએશન છતાં સ્થિર આઉટપુટ વોલ્ટેજ જાળવે છે \\
\textbf{રિપલ રિડક્શન} & DC આઉટપુટમાં અનિચ્છનીય AC ઘટકોને ઘટાડે છે \\
\textbf{લોડ પ્રોટેક્શન} & વોલ્ટેજ વેરિએશનથી ઇલેક્ટ્રોનિક ઉપકરણોને સુરક્ષિત રાખે છે \\
\end{longtable}
}

\end{solutionbox}
\begin{mnemonicbox}
``SVRL'' - Smoothens Voltage by Reducing ripples for
Load

\end{mnemonicbox}
\subsection*{પ્રશ્ન 2(b) OR [4
ગુણ]}\label{q2b}

\textbf{P પ્રકાર સેમિકન્ડક્ટર અને N પ્રકાર સેમિકન્ડક્ટર વચ્ચે તફાવત કરો.}

\begin{solutionbox}

\textbf{ટેબલ: P-type vs N-type સેમિકન્ડક્ટર}

{\def\LTcaptype{none} % do not increment counter
\begin{longtable}[]{@{}
  >{\raggedright\arraybackslash}p{(\linewidth - 4\tabcolsep) * \real{0.2667}}
  >{\raggedright\arraybackslash}p{(\linewidth - 4\tabcolsep) * \real{0.3667}}
  >{\raggedright\arraybackslash}p{(\linewidth - 4\tabcolsep) * \real{0.3667}}@{}}
\toprule\noalign{}
\begin{minipage}[b]{\linewidth}\raggedright
લાક્ષણિકતા
\end{minipage} & \begin{minipage}[b]{\linewidth}\raggedright
P-type સેમિકન્ડક્ટર
\end{minipage} & \begin{minipage}[b]{\linewidth}\raggedright
N-type સેમિકન્ડક્ટર
\end{minipage} \\
\midrule\noalign{}
\endhead
\bottomrule\noalign{}
\endlastfoot
\textbf{ડોપન્ટ વપરાશ} & ત્રિસંયોજક તત્વો (B, Al, Ga) & પંચસંયોજક તત્વો (P, As,
Sb) \\
\textbf{મુખ્ય વાહકો} & હોલ્સ (પોઝિટિવ ચાર્જ વાહકો) & ઇલેક્ટ્રોન્સ (નેગેટિવ ચાર્જ
વાહકો) \\
\textbf{ગૌણ વાહકો} & ઇલેક્ટ્રોન્સ & હોલ્સ \\
\textbf{વીજવાહકતા} & હોલ્સની ગતિને કારણે & ઇલેક્ટ્રોન્સની ગતિને કારણે \\
\textbf{ઊર્જા સ્તર} & વેલેન્સ બેન્ડ નજીક એક્સેપ્ટર એટમ & કન્ડક્શન બેન્ડ નજીક ડોનર
એટમ \\
\textbf{ઇલેક્ટ્રિકલ ચાર્જ} & સમગ્ર ન્યૂટ્રલ, પરંતુ ઇલેક્ટ્રોન્સ સ્વીકારે છે & સમગ્ર ન્યૂટ્રલ,
પરંતુ ઇલેક્ટ્રોન્સ દાન કરે છે \\
\end{longtable}
}

\end{solutionbox}
\begin{mnemonicbox}
``HELP-NED'' - Holes Exist in Large quantities in
P-type, Negative Electrons Dominate N-type

\end{mnemonicbox}
\subsection*{પ્રશ્ન 2(c) OR [7
ગુણ]}\label{q2c}

\textbf{વેવફોર્મ્સ સાથે બ્રિજ રેક્ટિફાયરનું કાર્ય સમજાવો.}

\begin{solutionbox}

\textbf{બ્રિજ રેક્ટિફાયર કાર્ય સિદ્ધાંત}

{\def\LTcaptype{none} % do not increment counter
\begin{longtable}[]{@{}ll@{}}
\toprule\noalign{}
ઘટક & કાર્ય \\
\midrule\noalign{}
\endhead
\bottomrule\noalign{}
\endlastfoot
\textbf{ડાયોડ્સ (D1-D4)} & બ્રિજ કોન્ફિગરેશનમાં ગોઠવાયેલ ચાર ડાયોડ \\
\textbf{ઇનપુટ} & ટ્રાન્સફોર્મર સેકન્ડરીથી AC વોલ્ટેજ \\
\textbf{આઉટપુટ} & લોડ રેસિસ્ટર પર પલ્સેટિંગ DC વોલ્ટેજ \\
\textbf{ઓપરેશન} & AC સાયકલના બંને અર્ધભાગને સમાન ધ્રુવતામાં રૂપાંતરિત કરે છે \\
\end{longtable}
}

\textbf{પોઝિટિવ હાફ સાયકલમાં કાર્ય:}

\begin{itemize}
\tightlist
\item
  ડાયોડ D1 અને D3 કન્ડક્ટ કરે છે
\item
  ડાયોડ D2 અને D4 રિવર્સ બાયસ્ડ (ઓફ) હોય છે
\item
  કરંટ ફ્લો: AC+ \rightarrow D1 \rightarrow લોડ \rightarrow D3 \rightarrow AC-
\end{itemize}

\textbf{નેગેટિવ હાફ સાયકલમાં કાર્ય:}

\begin{itemize}
\tightlist
\item
  ડાયોડ D2 અને D4 કન્ડક્ટ કરે છે
\item
  ડાયોડ D1 અને D3 રિવર્સ બાયસ્ડ (ઓફ) હોય છે
\item
  કરંટ ફ્લો: AC- \rightarrow D2 \rightarrow લોડ \rightarrow D4 \rightarrow AC+
\end{itemize}

\begin{center}
\textbf{Mermaid Diagram (Code)}
\begin{verbatim}
{Shaded}
{Highlighting}[]
graph LR
    AC[AC Input] {-{-}{} D1[D1]}
    AC {-{-}{} D3[D3]}
    D1 {-{-}{} Load[Load]}
    D3 {-{-}{} Load}
    Load {-{-}{} D2[D2]}
    Load {-{-}{} D4[D4]}
    D2 {-{-}{} AC}
    D4 {-{-}{} AC}
    style AC fill:\#fcf,stroke:\#333
    style Load fill:\#cfc,stroke:\#333
    style D1 fill:\#9cf,stroke:\#333
    style D2 fill:\#9cf,stroke:\#333
    style D3 fill:\#9cf,stroke:\#333
    style D4 fill:\#9cf,stroke:\#333
{Highlighting}
{Shaded}
\end{verbatim}
\end{center}

\textbf{વેવફોર્મ્સ:}

\begin{verbatim}
AC Input:      \_/{      \_/      \_/      }
              /    {   /       /       }
    0 \_\_\_\_\_\_/      {\_/      \_/      \_\_}
             {    /       /       /}
              {\_\_/     \_\_/     \_\_/}


DC Output:     \_       \_       \_      
              / {     /      /      }
    0 \_\_\_\_\_\_/   {\_\_\_/   \_\_\_/   \_\_\_\_\_}
\end{verbatim}

\textbf{ફાયદાઓ:}

\begin{itemize}
\tightlist
\item
  AC ઇનપુટના બંને અર્ધ સાયકલનો ઉપયોગ કરે છે
\item
  હાફ-વેવની તુલનામાં ઉચ્ચ આઉટપુટ વોલ્ટેજ અને કાર્યક્ષમતા
\item
  સેન્ટર-ટેપ્ડ ટ્રાન્સફોર્મરની જરૂર નથી
\end{itemize}

\end{solutionbox}
\begin{mnemonicbox}
``FBRO'' - Four diodes, Both cycles, Rectified
Output

\end{mnemonicbox}
\subsection*{પ્રશ્ન 3(a) [3
ગુણ]}\label{q3a}

\textbf{વ્યાખ્યાયિત કરો (1) PIV (2) રિપલ ફેક્ટર.}

\begin{solutionbox}

{\def\LTcaptype{none} % do not increment counter
\begin{longtable}[]{@{}
  >{\raggedright\arraybackslash}p{(\linewidth - 2\tabcolsep) * \real{0.3333}}
  >{\raggedright\arraybackslash}p{(\linewidth - 2\tabcolsep) * \real{0.6667}}@{}}
\toprule\noalign{}
\begin{minipage}[b]{\linewidth}\raggedright
શબ્દ
\end{minipage} & \begin{minipage}[b]{\linewidth}\raggedright
વ્યાખ્યા
\end{minipage} \\
\midrule\noalign{}
\endhead
\bottomrule\noalign{}
\endlastfoot
\textbf{PIV (પીક ઇન્વર્સ વોલ્ટેજ)} & • રિવર્સ બાયસ સ્થિતિમાં ડાયોડ સહન કરી શકે તે
મહત્તમ વોલ્ટેજ• ડાયોડ બ્રેકડાઉન અટકાવવા માટે મહત્વની રેટિંગ• સર્કિટમાં મહત્તમ રિવર્સ
વોલ્ટેજ કરતાં ઉચ્ચ હોવું આવશ્યક \\
\textbf{રિપલ ફેક્ટર (r)} & • રેક્ટિફાયર ફિલ્ટરની અસરકારકતાનું માપ• આઉટપુટમાં AC
ઘટકના RMS મૂલ્યથી DC ઘટકના અનુપાત• ઓછો રિપલ ફેક્ટર વધુ સારી ફિલ્ટરિંગ સૂચવે છે \\
\end{longtable}
}

\textbf{ફોર્મ્યુલા:} રિપલ ફેક્ટર (r) = V_{(}ᵣ_{m}_{s}_{)}_{a}._{k} / V_{(}ᵈᶜ_{)}

\end{solutionbox}
\begin{mnemonicbox}
``PIR'' - Peak Inverse voltage Restricts, Ripple
indicates Rectification quality

\end{mnemonicbox}
\subsection*{પ્રશ્ન 3(b) [4
ગુણ]}\label{q3b}

\textbf{PN જંક્શન ડાયોડની VI લાક્ષણિકતાઓ સમજાવો.}

\begin{solutionbox}

\textbf{PN જંક્શન ડાયોડની V-I લાક્ષણિકતાઓ}

{\def\LTcaptype{none} % do not increment counter
\begin{longtable}[]{@{}
  >{\raggedright\arraybackslash}p{(\linewidth - 4\tabcolsep) * \real{0.2286}}
  >{\raggedright\arraybackslash}p{(\linewidth - 4\tabcolsep) * \real{0.2857}}
  >{\raggedright\arraybackslash}p{(\linewidth - 4\tabcolsep) * \real{0.4857}}@{}}
\toprule\noalign{}
\begin{minipage}[b]{\linewidth}\raggedright
ક્ષેત્ર
\end{minipage} & \begin{minipage}[b]{\linewidth}\raggedright
વર્તન
\end{minipage} & \begin{minipage}[b]{\linewidth}\raggedright
લાક્ષણિકતાઓ
\end{minipage} \\
\midrule\noalign{}
\endhead
\bottomrule\noalign{}
\endlastfoot
\textbf{ફોરવર્ડ બાયસ} & સરળતાથી કરંટ વહન કરે છે & • થ્રેશોલ્ડ પછી કરંટમાં
એક્સપોનેન્શિયલ વધારો• થ્રેશોલ્ડ વોલ્ટેજ: સિલિકોન માટે \textasciitilde0.7V,
જર્મેનિયમ માટે \textasciitilde0.3V \\
\textbf{રિવર્સ બાયસ} & કરંટને અવરોધે છે & • ખૂબ નાનો લીકેજ કરંટ (μA)• રિવર્સ
બ્રેકડાઉન વોલ્ટેજ પર બ્રેકડાઉન \\
\end{longtable}
}

\begin{verbatim}
         Current (I)
             ↑
             |              /
             |             /
             |            /
             |           /
             |          /
             |         /
             |        /
             |       /
    {-{-}{-}{-}{-}{-}{-}{-}{-}|{-}{-}{-}{-}{-}{-}/{-}{-}{-}{-}{-}{-}{-} Voltage (V)}
             |    0.7V
             |/
     \_\_\_\_\_\_\_\_|\_\_\_\_\_\_\_\_\_\_\_\_\_\_\_\_\_\_\_\_\_\_\_\_
             |
             | Small leakage current
             |
             |        Breakdown
             |           ↓
             |           |
             |           |
\end{verbatim}

\textbf{મુખ્ય પોઇન્ટ્સ:}

\begin{itemize}
\tightlist
\item
  \textbf{ફોરવર્ડ થ્રેશોલ્ડ}: Si માટે \textasciitilde0.7V, Ge માટે
  \textasciitilde0.3V
\item
  \textbf{ફોરવર્ડ રિજન}: ઉચ્ચ કન્ડક્ટિવિટી
\item
  \textbf{રિવર્સ રિજન}: ખૂબ ઉચ્ચ પ્રતિરોધ
\item
  \textbf{બ્રેકડાઉન રિજન}: રિવર્સ કરંટમાં અચાનક વધારો
\end{itemize}

\end{solutionbox}
\begin{mnemonicbox}
``FBRL'' - Forward Bias Resists Little, reverse
blocks lots

\end{mnemonicbox}
\subsection*{પ્રશ્ન 3(c) [7
ગુણ]}\label{q3c}

\textbf{તરંગ સ્વરૂપો સાથે કેપેસિટર ઇનપુટ અને ચોક ઇનપુટ ફિલ્ટરની કામગીરી સમજાવો.}

\begin{solutionbox}

\textbf{1. કેપેસિટર ઇનપુટ ફિલ્ટર}

{\def\LTcaptype{none} % do not increment counter
\begin{longtable}[]{@{}
  >{\raggedright\arraybackslash}p{(\linewidth - 2\tabcolsep) * \real{0.5238}}
  >{\raggedright\arraybackslash}p{(\linewidth - 2\tabcolsep) * \real{0.4762}}@{}}
\toprule\noalign{}
\begin{minipage}[b]{\linewidth}\raggedright
ઘટક
\end{minipage} & \begin{minipage}[b]{\linewidth}\raggedright
કાર્ય
\end{minipage} \\
\midrule\noalign{}
\endhead
\bottomrule\noalign{}
\endlastfoot
\textbf{કેપેસિટર} & લોડ રેસિસ્ટન્સ સાથે પેરેલલમાં જોડાયેલ \\
\textbf{કાર્ય સિદ્ધાંત} & • વોલ્ટેજના શિખર દરમિયાન ચાર્જ થાય છે• વોલ્ટેજના ડિપ
દરમિયાન ડિસ્ચાર્જ થાય છે• ચાર્જના ભંડાર તરીકે કાર્ય કરે છે \\
\textbf{વેવફોર્મ્સ} & • રિપલ નોંધપાત્ર રીતે ઘટાડે છે• આઉટપુટમાં થોડો ડિસ્ચાર્જ સ્લોપ
હોય છે \\
\end{longtable}
}

\textbf{ફાયદાઓ:}

\begin{itemize}
\tightlist
\item
  ઉચ્ચ DC આઉટપુટ વોલ્ટેજ
\item
  સરળ અને આર્થિક
\item
  સારું રિપલ રિડક્શન
\end{itemize}

\textbf{મર્યાદાઓ:}

\begin{itemize}
\tightlist
\item
  નબળું વોલ્ટેજ રેગ્યુલેશન
\item
  ઉચ્ચ પીક ડાયોડ કરંટ
\item
  ઓછા કરંટ એપ્લિકેશન માટે યોગ્ય
\end{itemize}

\textbf{2. ચોક ઇનપુટ ફિલ્ટર}

{\def\LTcaptype{none} % do not increment counter
\begin{longtable}[]{@{}
  >{\raggedright\arraybackslash}p{(\linewidth - 2\tabcolsep) * \real{0.5238}}
  >{\raggedright\arraybackslash}p{(\linewidth - 2\tabcolsep) * \real{0.4762}}@{}}
\toprule\noalign{}
\begin{minipage}[b]{\linewidth}\raggedright
ઘટક
\end{minipage} & \begin{minipage}[b]{\linewidth}\raggedright
કાર્ય
\end{minipage} \\
\midrule\noalign{}
\endhead
\bottomrule\noalign{}
\endlastfoot
\textbf{ઇન્ડક્ટર (ચોક)} & લોડ સાથે શ્રેણીમાં જોડાયેલ \\
\textbf{કેપેસિટર} & લોડ સાથે પેરેલલમાં જોડાયેલ \\
\textbf{કાર્ય સિદ્ધાંત} & • ઇન્ડક્ટર કરંટ પરિવર્તનનો વિરોધ કરે છે• કેપેસિટર બાકીના
રિપલને સ્મૂધ કરે છે \\
\textbf{વેવફોર્મ્સ} & • વધુ સતત કરંટ• ઓછું પરંતુ વધુ સ્થિર આઉટપુટ વોલ્ટેજ \\
\end{longtable}
}

\textbf{ફાયદાઓ:}

\begin{itemize}
\tightlist
\item
  વધુ સારું વોલ્ટેજ રેગ્યુલેશન
\item
  ઓછા પીક ડાયોડ કરંટ
\item
  ઉચ્ચ કરંટ એપ્લિકેશન માટે યોગ્ય
\end{itemize}

\textbf{મર્યાદાઓ:}

\begin{itemize}
\tightlist
\item
  ઓછું DC આઉટપુટ વોલ્ટેજ
\item
  વધુ ખર્ચાળ
\item
  કેપેસિટર ફિલ્ટર કરતાં વધુ મોટું
\end{itemize}

\begin{center}
\textbf{Mermaid Diagram (Code)}
\begin{verbatim}
{Shaded}
{Highlighting}[]
graph LR
    A[Rectifier Output] {-{-}{} B[Capacitor/Choke Input]}
    B {-{-}{} C[Filtered Output]}
    style A fill:\#f96,stroke:\#333
    style B fill:\#69f,stroke:\#333
    style C fill:\#6f9,stroke:\#333
{Highlighting}
{Shaded}
\end{verbatim}
\end{center}

\textbf{વેવફોર્મ તુલના:}

\begin{verbatim}
Rectifier output:     \_\_      \_\_      \_\_
                     /  {    /      /  }
                    /    {  /      /    }
           0 \_\_\_\_\_\_/      {/      /      \_\_\_\_}

Capacitor filter:    \_\_\_     \_\_\_     \_\_\_
                     {              }
                      {              }
           0 \_\_\_\_\_\_\_\_\_{\_\_\_\_\_\_\_\_\_\_\_\_\_\_\_\_\_\_}

Choke filter:         \_\_\_\_\_\_\_\_\_\_ \_\_\_\_\_\_\_\_\_\_
                     /          /
                    /          /
           0 \_\_\_\_\_\_/          /\_\_\_\_\_\_\_\_\_\_\_\_
\end{verbatim}

\end{solutionbox}
\begin{mnemonicbox}
``VOICE'' - Voltage Output Is Constant with Either
filter, but choke gives better regulation

\end{mnemonicbox}
\subsection*{પ્રશ્ન 3(a) OR [3
ગુણ]}\label{q3a}

\textbf{ઝેનર ડાયોડનું કાર્ય અને મહત્વ જણાવો.}

\begin{solutionbox}

\textbf{ઝેનર ડાયોડનું કાર્ય અને મહત્વ}

{\def\LTcaptype{none} % do not increment counter
\begin{longtable}[]{@{}ll@{}}
\toprule\noalign{}
કાર્ય & વર્ણન \\
\midrule\noalign{}
\endhead
\bottomrule\noalign{}
\endlastfoot
\textbf{વોલ્ટેજ રેગ્યુલેશન} & ઇનપુટ વેરિએશન છતાં સ્થિર આઉટપુટ વોલ્ટેજ જાળવે છે \\
\textbf{વોલ્ટેજ રેફરન્સ} & સર્કિટમાં ચોક્કસ રેફરન્સ વોલ્ટેજ પ્રદાન કરે છે \\
\textbf{વોલ્ટેજ પ્રોટેક્શન} & વોલ્ટેજ સ્પાઇક્સથી સર્કિટને નુકસાન થતું અટકાવે છે \\
\textbf{વોલ્ટેજ લિમિટિંગ} & સિગ્નલ વોલ્ટેજને પૂર્વનિર્ધારિત સ્તરે ક્લિપ કરે છે \\
\textbf{વેવફોર્મ ક્લિપિંગ} & વોલ્ટેજ સ્તરને મર્યાદિત કરીને વેવફોર્મ્સને આકાર આપે છે \\
\end{longtable}
}

\end{solutionbox}
\begin{mnemonicbox}
``VPRVW'' - Voltage Protection, Regulation, and
Voltage Waveform control

\end{mnemonicbox}
\subsection*{પ્રશ્ન 3(b) OR [4
ગુણ]}\label{q3b}

\textbf{પ્રકાશ ઉત્સર્જક ડાયોડ (LED) ને તેની લાક્ષણિકતા સાથે વર્ણવો.}

\begin{solutionbox}

\textbf{લાઈટ એમિટિંગ ડાયોડ (LED) લાક્ષણિકતાઓ}

{\def\LTcaptype{none} % do not increment counter
\begin{longtable}[]{@{}
  >{\raggedright\arraybackslash}p{(\linewidth - 2\tabcolsep) * \real{0.5517}}
  >{\raggedright\arraybackslash}p{(\linewidth - 2\tabcolsep) * \real{0.4483}}@{}}
\toprule\noalign{}
\begin{minipage}[b]{\linewidth}\raggedright
લાક્ષણિકતા
\end{minipage} & \begin{minipage}[b]{\linewidth}\raggedright
વર્ણન
\end{minipage} \\
\midrule\noalign{}
\endhead
\bottomrule\noalign{}
\endlastfoot
\textbf{બાંધકામ} & • ડાયરેક્ટ બેન્ડગેપ સેમિકન્ડક્ટરથી બનેલું P-N જંક્શન• સામાન્ય
મટીરિયલ: GaAs, GaP, AlGaInP, InGaN \\
\textbf{કાર્ય સિદ્ધાંત} & • ઇલેક્ટ્રોલ્યુમિનિસન્સ: ઇલેક્ટ્રોન્સ હોલ્સ સાથે રિકોમ્બાઇન
થાય છે• ઊર્જા ફોટોન્સ (પ્રકાશ) તરીકે મુક્ત થાય છે \\
\textbf{ફોરવર્ડ વોલ્ટેજ} & • લાલ: 1.8-2.1V• લીલો: 2.0-3.0V• વાદળી/સફેદ:
3.0-3.5V \\
\textbf{ઉપલબ્ધ રંગો} & • સેમિકન્ડક્ટર મટીરિયલ પર આધારિત• લાલ, લીલો, પીળો,
વાદળી, સફેદ, IR, UV \\
\textbf{I-V લાક્ષણિકતાઓ} & • થ્રેશોલ્ડથી ઉપર ફોરવર્ડ બાયસ પર કન્ડક્ટ કરે છે•
કરંટ-મર્યાદિત રેસિસ્ટરની જરૂર પડે છે• 5V ઉપરના રિવર્સ બાયસથી નુકસાન થાય છે \\
\textbf{ઉપયોગો} & • ઇન્ડિકેટર્સ, ડિસ્પ્લે, લાઇટિંગ, ઓપ્ટોકપલર્સ \\
\end{longtable}
}

\begin{center}
\textbf{Mermaid Diagram (Code)}
\begin{verbatim}
{Shaded}
{Highlighting}[]
graph LR
    A[Voltage Applied] {-{-}{}|Forward Bias| B[Electron{-}Hole Recombination]}
    B {-{-}{} C[Energy Released]}
    C {-{-}{} D[Light Emission]}
    style A fill:\#f96,stroke:\#333
    style B fill:\#69f,stroke:\#333
    style C fill:\#fc9,stroke:\#333
    style D fill:\#6f9,stroke:\#333
{Highlighting}
{Shaded}
\end{verbatim}
\end{center}

\end{solutionbox}
\begin{mnemonicbox}
``CRAVE'' - Current Regulated And Voltage Emits
light

\end{mnemonicbox}
\subsection*{પ્રશ્ન 3(c) OR [7
ગુણ]}\label{q3c}

\textbf{કેપેસિટર ઇનપુટ અને ચોક ઇનપુટ ફિલ્ટરનું કાર્ય સમજાવો.}

\begin{solutionbox}

\textbf{કેપેસિટર ઇનપુટ ફિલ્ટર:}

{\def\LTcaptype{none} % do not increment counter
\begin{longtable}[]{@{}
  >{\raggedright\arraybackslash}p{(\linewidth - 2\tabcolsep) * \real{0.5238}}
  >{\raggedright\arraybackslash}p{(\linewidth - 2\tabcolsep) * \real{0.4762}}@{}}
\toprule\noalign{}
\begin{minipage}[b]{\linewidth}\raggedright
ઘટક
\end{minipage} & \begin{minipage}[b]{\linewidth}\raggedright
કાર્ય
\end{minipage} \\
\midrule\noalign{}
\endhead
\bottomrule\noalign{}
\endlastfoot
\textbf{સર્કિટ સ્ટ્રક્ચર} & લોડ સાથે પેરેલલમાં જોડાયેલ કેપેસિટર \\
\textbf{ઓપરેશન} & • કેપેસિટર પીક વોલ્ટેજ સુધી ચાર્જ થાય છે• જ્યારે વોલ્ટેજ ઘટે છે ત્યારે
લોડ દ્વારા ધીમે ધીમે ડિસ્ચાર્જ થાય છે• ચાર્જના ભંડાર તરીકે કાર્ય કરે છે \\
\textbf{કામગીરી} & • સારું રિપલ રિડક્શન• ઉચ્ચ આઉટપુટ વોલ્ટેજ• વેરિંગ લોડ હેઠળ નબળું
રેગ્યુલેશન \\
\end{longtable}
}

\textbf{સર્કિટ ડાયાગ્રામ:}

\begin{verbatim}
    +{-{-}{-}{-}{-}{-}||{-}{-}{-}{-}{-}{-}+}
    |      D1       |
AC  |               | Load
In  |               | RL    +
    +{-{-}{-}{-}{-}{-}||{-}{-}{-}{-}{-}{-}+{-}{-}{-}{-}{-}||{-}{-}{-}+}
    |      D2       |     C    |
    +{-{-}{-}{-}{-}{-}{-}{-}{-}{-}{-}{-}{-}{-}{-}+{-}{-}{-}{-}{-}{-}{-}{-}{-}{-}+}
\end{verbatim}

\textbf{ચોક ઇનપુટ ફિલ્ટર:}

{\def\LTcaptype{none} % do not increment counter
\begin{longtable}[]{@{}
  >{\raggedright\arraybackslash}p{(\linewidth - 2\tabcolsep) * \real{0.5238}}
  >{\raggedright\arraybackslash}p{(\linewidth - 2\tabcolsep) * \real{0.4762}}@{}}
\toprule\noalign{}
\begin{minipage}[b]{\linewidth}\raggedright
ઘટક
\end{minipage} & \begin{minipage}[b]{\linewidth}\raggedright
કાર્ય
\end{minipage} \\
\midrule\noalign{}
\endhead
\bottomrule\noalign{}
\endlastfoot
\textbf{સર્કિટ સ્ટ્રક્ચર} & શ્રેણીમાં ઇન્ડક્ટર (ચોક), પેરેલલમાં કેપેસિટર \\
\textbf{ઓપરેશન} & • ઇન્ડક્ટર કરંટમાં ફેરફારનો વિરોધ કરે છે• કરંટ પ્રવાહને સ્મૂધ કરે છે•
કેપેસિટર વધુ વોલ્ટેજ રિપલ્સને ફિલ્ટર કરે છે \\
\textbf{કામગીરી} & • વધુ સારું વોલ્ટેજ રેગ્યુલેશન• ઓછું આઉટપુટ વોલ્ટેજ• ઉચ્ચ-કરંટ
એપ્લિકેશન માટે સારું \\
\end{longtable}
}

\textbf{સર્કિટ ડાયાગ્રામ:}

\begin{verbatim}
    +{-{-}{-}{-}{-}{-}||{-}{-}{-}{-}{-}{-}+}
    |      D1       |
AC  |               +{-{-}{-}{-}LLLLL{-}{-}{-}{-}+}
In  |                     L       |
    +{-{-}{-}{-}{-}{-}||{-}{-}{-}{-}{-}{-}+          RL +}
    |      D2       |     C       |
    +{-{-}{-}{-}{-}{-}{-}{-}{-}{-}{-}{-}{-}{-}{-}+{-}{-}{-}{-}||{-}{-}{-}{-}{-}{-}{-}+}
\end{verbatim}

\textbf{તુલના:}

{\def\LTcaptype{none} % do not increment counter
\begin{longtable}[]{@{}lll@{}}
\toprule\noalign{}
પેરામીટર & કેપેસિટર ઇનપુટ & ચોક ઇનપુટ \\
\midrule\noalign{}
\endhead
\bottomrule\noalign{}
\endlastfoot
\textbf{આઉટપુટ વોલ્ટેજ} & ઉચ્ચ (\approx1.4Vm) & નીચું (\approx0.9Vm) \\
\textbf{રિપલ ફેક્ટર} & ઉચ્ચ & નીચો \\
\textbf{વોલ્ટેજ રેગ્યુલેશન} & નબળું & સારું \\
\textbf{ડાયોડ કરંટ} & ઉચ્ચ પીક કરંટ & નીચા પીક કરંટ \\
\textbf{કિંમત \& કદ} & ઓછી, નાનું & ઉચ્ચ, મોટું \\
\textbf{ઉપયોગો} & ઓછા કરંટની જરૂરિયાત & ઉચ્ચ કરંટની જરૂરિયાત \\
\end{longtable}
}

\end{solutionbox}
\begin{mnemonicbox}
``CHEER'' - Capacitor Holds Energy, inductor Ensures
Regulated current

\end{mnemonicbox}
\subsection*{પ્રશ્ન 4(a) [3
ગુણ]}\label{q4a}

\textbf{PN જંક્શન ડાયોડની લાક્ષણિકતાઓની ચર્ચા કરો.}

\begin{solutionbox}

\textbf{PN જંક્શન ડાયોડની લાક્ષણિકતાઓ}

{\def\LTcaptype{none} % do not increment counter
\begin{longtable}[]{@{}
  >{\raggedright\arraybackslash}p{(\linewidth - 2\tabcolsep) * \real{0.5517}}
  >{\raggedright\arraybackslash}p{(\linewidth - 2\tabcolsep) * \real{0.4483}}@{}}
\toprule\noalign{}
\begin{minipage}[b]{\linewidth}\raggedright
લાક્ષણિકતા
\end{minipage} & \begin{minipage}[b]{\linewidth}\raggedright
વર્ણન
\end{minipage} \\
\midrule\noalign{}
\endhead
\bottomrule\noalign{}
\endlastfoot
\textbf{ફોરવર્ડ બાયસ} & • થ્રેશોલ્ડથી વધુ વોલ્ટેજ (Si માટે 0.7V, Ge માટે 0.3V) પર
કન્ડક્ટ કરે છે• વોલ્ટેજ સાથે કરંટ એક્સપોનેન્શિયલી વધે છે• ઓછા રેઝિસ્ટન્સની સ્થિતિ \\
\textbf{રિવર્સ બાયસ} & • કરંટ પ્રવાહને અવરોધે છે• નાનો લીકેજ કરંટ (μA)• ઉચ્ચ
રેઝિસ્ટન્સની સ્થિતિ \\
\textbf{બ્રેકડાઉન} & • ચોક્કસ રિવર્સ વોલ્ટેજ પર થાય છે• કરંટ ઝડપથી વધે છે• જો કરંટ
મર્યાદિત ન હોય તો ડાયોડને નુકસાન થઈ શકે છે \\
\textbf{તાપમાનની અસરો} & • તાપમાન સાથે ફોરવર્ડ વોલ્ટેજ ઘટે છે• દર 10^\circC પર
રિવર્સ લીકેજ કરંટ બમણો થાય છે \\
\textbf{કેપેસિટન્સ} & • જંક્શન કેપેસિટન્સ લાગુ વોલ્ટેજ સાથે બદલાય છે• ફોરવર્ડ બાયસમાં
વધુ \\
\end{longtable}
}

\end{solutionbox}
\begin{mnemonicbox}
``FRBCT'' - Forward conducts, Reverse blocks,
Breakdown destroys, Capacitance changes, Temperature affects

\end{mnemonicbox}
\subsection*{પ્રશ્ન 4(b) [4
ગુણ]}\label{q4b}

\textbf{પી-એન જંક્શન ડાયોડ અને ઝેનર ડાયોડ વચ્ચે સરખામણી કરો.}

\begin{solutionbox}

\textbf{ટેબલ: P-N જંક્શન ડાયોડ vs.~ઝેનર ડાયોડ}

{\def\LTcaptype{none} % do not increment counter
\begin{longtable}[]{@{}
  >{\raggedright\arraybackslash}p{(\linewidth - 4\tabcolsep) * \real{0.2558}}
  >{\raggedright\arraybackslash}p{(\linewidth - 4\tabcolsep) * \real{0.4419}}
  >{\raggedright\arraybackslash}p{(\linewidth - 4\tabcolsep) * \real{0.3023}}@{}}
\toprule\noalign{}
\begin{minipage}[b]{\linewidth}\raggedright
પેરામીટર
\end{minipage} & \begin{minipage}[b]{\linewidth}\raggedright
P-N જંક્શન ડાયોડ
\end{minipage} & \begin{minipage}[b]{\linewidth}\raggedright
ઝેનર ડાયોડ
\end{minipage} \\
\midrule\noalign{}
\endhead
\bottomrule\noalign{}
\endlastfoot
\textbf{સિમ્બોલ} & ▶〈 & ▶〈▶ \\
\textbf{ફોરવર્ડ ઓપરેશન} & 0.7V ઉપર કન્ડક્ટ કરે છે & 0.7V ઉપર કન્ડક્ટ કરે છે
(સમાન) \\
\textbf{રિવર્સ ઓપરેશન} & બ્રેકડાઉન સુધી કરંટને અવરોધે છે & નિયંત્રિત બ્રેકડાઉનમાં કાર્ય
કરવા માટે ડિઝાઇન કરેલ છે \\
\textbf{બ્રેકડાઉન વોલ્ટેજ} & ઉચ્ચ, ચોક્કસ રીતે નિર્દિષ્ટ નથી & ઓછું, ચોક્કસ રીતે
નિર્દિષ્ટ (2-200V) \\
\textbf{રિવર્સ બ્રેકડાઉન} & જો મર્યાદિત ન હોય તો વિનાશક & બિન-વિનાશક, કાર્ય
માટે ઉપયોગમાં લેવાય છે \\
\textbf{ઉપયોગો} & રેક્ટિફિકેશન, સ્વિચિંગ & વોલ્ટેજ રેગ્યુલેશન, પ્રોટેક્શન \\
\textbf{ડોપિંગ લેવલ} & સામાન્ય ડોપિંગ & બ્રેકડાઉન નિયંત્રિત કરવા માટે ભારે
ડોપિંગ \\
\end{longtable}
}

\end{solutionbox}
\begin{mnemonicbox}
``FORBAR'' - Forward Operation is Regular, Breakdown
Application is the Real difference

\end{mnemonicbox}
\subsection*{પ્રશ્ન 4(c) [7
ગુણ]}\label{q4c}

\textbf{વોલ્ટેજ રેગ્યુલેટર તરીકે ઝેનર ડાયોડનું કાર્ય સમજાવો.}

\begin{solutionbox}

\textbf{ઝેનર ડાયોડ અઝ વોલ્ટેજ રેગ્યુલેટર}

{\def\LTcaptype{none} % do not increment counter
\begin{longtable}[]{@{}
  >{\raggedright\arraybackslash}p{(\linewidth - 2\tabcolsep) * \real{0.5238}}
  >{\raggedright\arraybackslash}p{(\linewidth - 2\tabcolsep) * \real{0.4762}}@{}}
\toprule\noalign{}
\begin{minipage}[b]{\linewidth}\raggedright
ઘટક
\end{minipage} & \begin{minipage}[b]{\linewidth}\raggedright
કાર્ય
\end{minipage} \\
\midrule\noalign{}
\endhead
\bottomrule\noalign{}
\endlastfoot
\textbf{ઝેનર ડાયોડ} & બ્રેકડાઉન ક્ષેત્રમાં કોન્સ્ટન્ટ વોલ્ટેજ જાળવે છે \\
\textbf{સીરીઝ રેસિસ્ટર (Rs)} & કરંટને મર્યાદિત કરે છે અને વધારાના વોલ્ટેજને ડ્રોપ કરે
છે \\
\textbf{લોડ રેસિસ્ટર (RL)} & પાવર આપવામાં આવેલ સર્કિટનું પ્રતિનિધિત્વ કરે છે \\
\end{longtable}
}

\textbf{કાર્ય સિદ્ધાંત:}

\begin{enumerate}
\tightlist
\item
  ઝેનર ડાયોડ રિવર્સ બાયસમાં જોડાયેલ છે
\item
  જ્યારે ઇનપુટ વોલ્ટેજ ઝેનર વોલ્ટેજથી વધે છે, ત્યારે ડાયોડ કન્ડક્ટ કરે છે
\item
  વધારાનું વોલ્ટેજ સીરીઝ રેસિસ્ટર પર ડ્રોપ થાય છે
\item
  આઉટપુટ વોલ્ટેજ ઝેનર વોલ્ટેજ પર સ્થિર રહે છે
\end{enumerate}

\begin{center}
\textbf{Mermaid Diagram (Code)}
\begin{verbatim}
{Shaded}
{Highlighting}[]
graph LR
    A[Input Voltage] {-{-}{} B[Series Resistor]}
    B {-{-}{} C[Output Voltage]}
    C {-{-}{} D[Load]}
    C {-{-}{} E[Zener Diode]}
    E {-{-}{} F[Ground]}
    style A fill:\#f96,stroke:\#333
    style B fill:\#69f,stroke:\#333
    style C fill:\#6f9,stroke:\#333
    style D fill:\#fc9,stroke:\#333
    style E fill:\#f9f,stroke:\#333
{Highlighting}
{Shaded}
\end{verbatim}
\end{center}

\textbf{સર્કિટ ડાયાગ્રામ:}

\begin{verbatim}
     +{-{-}{-}{-}[Rs]{-}{-}{-}{-}{-}+{-}{-}{-}{-}{-}+}
     |             |     |
Vin  |             +    RL   Vout = Vz
     |             |     |
     +{-{-}{-}{-}{-}{-}{-}{-}||{-}{-}+{-}{-}{-}{-}{-}+}
              Zener
\end{verbatim}

\textbf{રેગ્યુલેશન કેસિસ:}

{\def\LTcaptype{none} % do not increment counter
\begin{longtable}[]{@{}
  >{\raggedright\arraybackslash}p{(\linewidth - 2\tabcolsep) * \real{0.5238}}
  >{\raggedright\arraybackslash}p{(\linewidth - 2\tabcolsep) * \real{0.4762}}@{}}
\toprule\noalign{}
\begin{minipage}[b]{\linewidth}\raggedright
સ્થિતિ
\end{minipage} & \begin{minipage}[b]{\linewidth}\raggedright
પ્રતિક્રિયા
\end{minipage} \\
\midrule\noalign{}
\endhead
\bottomrule\noalign{}
\endlastfoot
\textbf{ઇનપુટ વોલ્ટેજ વધે છે} & • ઝેનર દ્વારા વધુ કરંટ• Rs પર વધુ વોલ્ટેજ ડ્રોપ•
આઉટપુટ Vz પર રહે છે \\
\textbf{ઇનપુટ વોલ્ટેજ ઘટે છે} & • ઝેનર દ્વારા ઓછો કરંટ• Rs પર ઓછો વોલ્ટેજ ડ્રોપ•
આઉટપુટ Vz પર રહે છે (લઘુત્તમ ઓપરેટિંગ વોલ્ટેજ સુધી) \\
\textbf{લોડ કરંટ વધે છે} & • ઝેનર દ્વારા ઓછો કરંટ• લઘુત્તમ ઝેનર કરંટ સુધી આઉટપુટ
વોલ્ટેજ સ્થિર \\
\textbf{લોડ કરંટ ઘટે છે} & • ઝેનર દ્વારા વધુ કરંટ• આઉટપુટ વોલ્ટેજ સ્થિર રહે છે \\
\end{longtable}
}

\textbf{મર્યાદાઓ:}

\begin{itemize}
\tightlist
\item
  ઝેનર અને Rs માં પાવર ડિસિપેશન
\item
  લઘુત્તમ ઇનપુટ વોલ્ટેજની આવશ્યકતા (Vin \textgreater{} Vz + Rs પર વોલ્ટેજ ડ્રોપ)
\item
  મર્યાદિત કરંટ ક્ષમતા
\end{itemize}

\end{solutionbox}
\begin{mnemonicbox}
``VISOR'' - Voltage In Stays Out Regulated

\end{mnemonicbox}
\subsection*{પ્રશ્ન 4(a) OR [3
ગુણ]}\label{q4a}

\textbf{ટ્રાન્ઝિસ્ટરની ટૂંકમાં ચર્ચા કરો.}

\begin{solutionbox}

\textbf{ટ્રાન્ઝિસ્ટર ઓવરવ્યુ}

{\def\LTcaptype{none} % do not increment counter
\begin{longtable}[]{@{}
  >{\raggedright\arraybackslash}p{(\linewidth - 2\tabcolsep) * \real{0.3810}}
  >{\raggedright\arraybackslash}p{(\linewidth - 2\tabcolsep) * \real{0.6190}}@{}}
\toprule\noalign{}
\begin{minipage}[b]{\linewidth}\raggedright
પાસું
\end{minipage} & \begin{minipage}[b]{\linewidth}\raggedright
વર્ણન
\end{minipage} \\
\midrule\noalign{}
\endhead
\bottomrule\noalign{}
\endlastfoot
\textbf{વ્યાખ્યા} & • ઇલેક્ટ્રિકલ સિગ્નલને એમ્પ્લિફાય/સ્વિચ કરતું સેમિકન્ડક્ટર ડિવાઇસ•
ત્રણ-ટર્મિનલ ડિવાઇસ: એમિટર, બેઝ, કલેક્ટર \\
\textbf{પ્રકારો} & • બાયપોલર જંક્શન ટ્રાન્ઝિસ્ટર (BJT): NPN, PNP• ફીલ્ડ ઇફેક્ટ
ટ્રાન્ઝિસ્ટર (FET): JFET, MOSFET \\
\textbf{કાર્ય સિદ્ધાંત} & • કરંટ/વોલ્ટેજ નિયંત્રિત ડિવાઇસ• નાના બેઝ કરંટ મોટા
કલેક્ટર કરંટને નિયંત્રિત કરે છે (BJT)• ગેટ વોલ્ટેજ ચેનલ કન્ડક્ટિવિટી નિયંત્રિત કરે છે
(FET) \\
\textbf{ઉપયોગો} & • એમ્પ્લિફિકેશન: ઓડિયો, RF, પાવર• સ્વિચિંગ: ડિજિટલ સર્કિટ•
ઓસિલેટર્સ અને સિગ્નલ જનરેશન \\
\textbf{મહત્વ} & • આધુનિક ઇલેક્ટ્રોનિક્સનો પાયો• ઇલેક્ટ્રોનિક ડિવાઇસના
મિનિએચરાઇઝેશનને શક્ય બનાવ્યું \\
\end{longtable}
}

\end{solutionbox}
\begin{mnemonicbox}
``TAWAI'' - Transistors Amplify, Work As switches,
and are Integral to electronics

\end{mnemonicbox}
\subsection*{પ્રશ્ન 4(b) OR [4
ગુણ]}\label{q4b}

\textbf{ટ્રાન્ઝિસ્ટર એમ્પલીફાયર માટે α અને β વચ્ચેનો સંબંધ મેળવો.}

\begin{solutionbox}

\textbf{α અને β વચ્ચેનો સંબંધ}

{\def\LTcaptype{none} % do not increment counter
\begin{longtable}[]{@{}
  >{\raggedright\arraybackslash}p{(\linewidth - 4\tabcolsep) * \real{0.3438}}
  >{\raggedright\arraybackslash}p{(\linewidth - 4\tabcolsep) * \real{0.3750}}
  >{\raggedright\arraybackslash}p{(\linewidth - 4\tabcolsep) * \real{0.2812}}@{}}
\toprule\noalign{}
\begin{minipage}[b]{\linewidth}\raggedright
પેરામીટર
\end{minipage} & \begin{minipage}[b]{\linewidth}\raggedright
વ્યાખ્યા
\end{minipage} & \begin{minipage}[b]{\linewidth}\raggedright
ફોર્મ્યુલા
\end{minipage} \\
\midrule\noalign{}
\endhead
\bottomrule\noalign{}
\endlastfoot
\textbf{α (આલ્ફા)} & • કોમન બેઝ (CB) કરંટ ગેઇન• કલેક્ટર કરંટ અને એમિટર કરંટનો
ગુણોત્તર & α = IC/IE \\
\textbf{β (બીટા)} & • કોમન એમિટર (CE) કરંટ ગેઇન• કલેક્ટર કરંટ અને બેઝ કરંટનો
ગુણોત્તર & β = IC/IB \\
\end{longtable}
}

\textbf{ડેરિવેશન સ્ટેપ્સ:}

\begin{enumerate}
\item
  આપણે જાણીએ છીએ કે એમિટર કરંટ બેઝ અને કલેક્ટર કરંટનો સરવાળો છે: IE = IB + IC
\item
  આલ્ફા વ્યાખ્યા: α = IC/IE
\item
  બીટા વ્યાખ્યા: β = IC/IB
\item
  સ્ટેપ 1થી, આપણે લખી શકીએ: IB = IE - IC
\item
  બીટા વ્યાખ્યામાં સબ્સ્ટિટ્યુશન: β = IC/(IE - IC)
\item
આલ્ફા વ્યાખ્યાનો ઉપયોગ કરીને, IC = α \times IE:

β = (α \times IE)/(IE - α \times IE)

\item
  સરળીકરણ: β = α/(1 - α)
\item
  તેનાથી વિપરીત, આપણે α ને β ના સંદર્ભમાં પણ વ્યક્ત કરી શકીએ: α = β/(β + 1)
\end{enumerate}

\textbf{સંબંધ ટેબલ:}

{\def\LTcaptype{none} % do not increment counter
\begin{longtable}[]{@{}ll@{}}
\toprule\noalign{}
α (આલ્ફા) & β (બીટા) \\
\midrule\noalign{}
\endhead
\bottomrule\noalign{}
\endlastfoot
0.9 & 9 \\
0.95 & 19 \\
0.98 & 49 \\
0.99 & 99 \\
0.995 & 199 \\
\end{longtable}
}

\end{solutionbox}
\begin{mnemonicbox}
``ABR'' - Alpha and Beta are Related by α = β/(β+1)
or β = α/(1-α)

\end{mnemonicbox}
\subsection*{પ્રશ્ન 4(c) OR [7
ગુણ]}\label{q4c}

\textbf{NPN અને PNP ટ્રાન્ઝિસ્ટરનું બાંધકામ વિગતવાર સમજાવો.}

\begin{solutionbox}

\textbf{NPN અને PNP ટ્રાન્ઝિસ્ટરનું બાંધકામ}

{\def\LTcaptype{none} % do not increment counter
\begin{longtable}[]{@{}
  >{\raggedright\arraybackslash}p{(\linewidth - 4\tabcolsep) * \real{0.2558}}
  >{\raggedright\arraybackslash}p{(\linewidth - 4\tabcolsep) * \real{0.3721}}
  >{\raggedright\arraybackslash}p{(\linewidth - 4\tabcolsep) * \real{0.3721}}@{}}
\toprule\noalign{}
\begin{minipage}[b]{\linewidth}\raggedright
પેરામીટર
\end{minipage} & \begin{minipage}[b]{\linewidth}\raggedright
NPN ટ્રાન્ઝિસ્ટર
\end{minipage} & \begin{minipage}[b]{\linewidth}\raggedright
PNP ટ્રાન્ઝિસ્ટર
\end{minipage} \\
\midrule\noalign{}
\endhead
\bottomrule\noalign{}
\endlastfoot
\textbf{સ્ટ્રક્ચર} & • N-પ્રકાર (એમિટર)• P-પ્રકાર (બેઝ)• N-પ્રકાર (કલેક્ટર) & •
P-પ્રકાર (એમિટર)• N-પ્રકાર (બેઝ)• P-પ્રકાર (કલેક્ટર) \\
\textbf{સિમ્બોલ} &
\pandocbounded{\includegraphics[keepaspectratio,alt={NPN Symbol}]{બાહર તરફ એમિટર એરો સાથેનો ત્રિકોણ}}
&
\pandocbounded{\includegraphics[keepaspectratio,alt={PNP Symbol}]{અંદર તરફ એમિટર એરો સાથેનો ત્રિકોણ}} \\
\textbf{મટીરિયલ} & • સિલિકોન અથવા જર્મેનિયમ• એમિટર: ભારે ડોપ્ડ N-પ્રકાર• બેઝ:
હળવા ડોપ્ડ P-પ્રકાર• કલેક્ટર: મધ્યમ ડોપ્ડ N-પ્રકાર & • સિલિકોન અથવા જર્મેનિયમ•
એમિટર: ભારે ડોપ્ડ P-પ્રકાર• બેઝ: હળવા ડોપ્ડ N-પ્રકાર• કલેક્ટર: મધ્યમ ડોપ્ડ
P-પ્રકાર \\
\textbf{જાડાઈ} & • બેઝ: ખૂબ જ પાતળી (1-10 μm)• કલેક્ટર: સૌથી જાડી ક્ષેત્ર & •
બેઝ: ખૂબ જ પાતળી (1-10 μm)• કલેક્ટર: સૌથી જાડી ક્ષેત્ર \\
\textbf{ડોપિંગ લેવલ} & • એમિટર: સૌથી ઊંચું• બેઝ: સૌથી નીચું• કલેક્ટર: મધ્યમ & •
એમિટર: સૌથી ઊંચું• બેઝ: સૌથી નીચું• કલેક્ટર: મધ્યમ \\
\end{longtable}
}

\textbf{NPN ટ્રાન્ઝિસ્ટર બાંધકામ:}

\begin{verbatim}
    Emitter (N)   Base (P)   Collector (N)
       |            |            |
       v            v            v
    +{-{-}{-}{-}{-}{-}+     +{-}{-}{-}+     +{-}{-}{-}{-}{-}{-}{-}{-}{-}{-}+}
    |  N+  |     | P |     |    N     |
    +{-{-}{-}{-}{-}{-}+     +{-}{-}{-}+     +{-}{-}{-}{-}{-}{-}{-}{-}{-}{-}+}
       |           |           |
       |           |           |
       E           B           C
\end{verbatim}

\textbf{PNP ટ્રાન્ઝિસ્ટર બાંધકામ:}

\begin{verbatim}
    Emitter (P)   Base (N)   Collector (P)
       |            |            |
       v            v            v
    +{-{-}{-}{-}{-}{-}+     +{-}{-}{-}+     +{-}{-}{-}{-}{-}{-}{-}{-}{-}{-}+}
    |  P+  |     | N |     |    P     |
    +{-{-}{-}{-}{-}{-}+     +{-}{-}{-}+     +{-}{-}{-}{-}{-}{-}{-}{-}{-}{-}+}
       |           |           |
       |           |           |
       E           B           C
\end{verbatim}

\textbf{મેન્યુફેક્ચરિંગ પ્રોસેસ:}

\begin{enumerate}
\tightlist
\item
  સેમિકન્ડક્ટર સબસ્ટ્રેટ (N અથવા P પ્રકાર) થી શરૂ કરો
\item
  એપિટેક્ષિયલ ગ્રોથ દ્વારા લેયર્સ બનાવો
\item
  ડિફ્યુઝન અથવા આયન ઇમ્પ્લાન્ટેશન દ્વારા જંક્શન બનાવો
\item
  ટર્મિનલ્સ માટે મેટલ કોન્ટેક્ટ્સ ઉમેરો
\item
  પ્રોટેક્ટિવ કેસમાં પેકેજિંગ કરો
\end{enumerate}

\begin{center}
\textbf{Mermaid Diagram (Code)}
\begin{verbatim}
{Shaded}
{Highlighting}[]
graph LR
    A[Silicon Wafer] {-{-}{} B[Epitaxial Layer Growth]}
    B {-{-}{} C[Diffusion of Dopants]}
    C {-{-}{} D[Oxide Insulation]}
    D {-{-}{} E[Metallization]}
    E {-{-}{} F[Packaging]}
    style A fill:\#fc9,stroke:\#333
    style B fill:\#69f,stroke:\#333
    style C fill:\#f9f,stroke:\#333
    style D fill:\#cfc,stroke:\#333
    style E fill:\#f96,stroke:\#333
    style F fill:\#9cf,stroke:\#333
{Highlighting}
{Shaded}
\end{verbatim}
\end{center}

\end{solutionbox}
\begin{mnemonicbox}
``ENB-CPM'' - Emitter has N in NPN, Collector is
Proportionally Medium-doped

\end{mnemonicbox}
\subsection*{પ્રશ્ન 5(a) [3
ગુણ]}\label{q5a}

\textbf{ટૂંકમાં ઈ-વેસ્ટ સમજાવો.}

\begin{solutionbox}

\textbf{ઇલેક્ટ્રોનિક વેસ્ટ (ઈ-વેસ્ટ)}

{\def\LTcaptype{none} % do not increment counter
\begin{longtable}[]{@{}
  >{\raggedright\arraybackslash}p{(\linewidth - 2\tabcolsep) * \real{0.3810}}
  >{\raggedright\arraybackslash}p{(\linewidth - 2\tabcolsep) * \real{0.6190}}@{}}
\toprule\noalign{}
\begin{minipage}[b]{\linewidth}\raggedright
પાસું
\end{minipage} & \begin{minipage}[b]{\linewidth}\raggedright
વર્ણન
\end{minipage} \\
\midrule\noalign{}
\endhead
\bottomrule\noalign{}
\endlastfoot
\textbf{વ્યાખ્યા} & • ફેંકી દીધેલા ઇલેક્ટ્રોનિક ઉપકરણો અને સાધનો• મૂલ્યવાન સામગ્રી
અને જોખમી પદાર્થો બંને ધરાવે છે \\
\textbf{સ્ત્રોતો} & • કોમ્પ્યુટર, ફોન, ટીવી, ઉપકરણો• સર્કિટ બોર્ડ, બેટરી, ડિસ્પ્લે•
ઓફિસ ઉપકરણો, મેડિકલ ડિવાઇસ \\
\textbf{ચિંતાઓ} & • ઝેરી પદાર્થો (લેડ, મર્ક્યુરી, કેડમિયમ) ધરાવે છે• અયોગ્ય રીતે
ડિસ્પોઝ કરવાથી પર્યાવરણ પ્રદૂષણ• માનવ અને વન્યજીવન માટે આરોગ્ય જોખમો \\
\textbf{મહત્વ} & • વિશ્વમાં સૌથી ઝડપથી વધતો કચરાનો પ્રવાહ• સંસાધન પુનઃપ્રાપ્તિની
ક્ષમતા (સોનું, ચાંદી, તાંબું)• વિશિષ્ટ હેન્ડલિંગની જરૂર \\
\end{longtable}
}

\end{solutionbox}
\begin{mnemonicbox}
``TECH'' - Toxic Electronics Create Hazards when
improperly disposed

\end{mnemonicbox}
\subsection*{પ્રશ્ન 5(b) [4
ગુણ]}\label{q5b}

\textbf{આકૃતિ સાથે NPN ટ્રાન્ઝિસ્ટરની કામગીરી સમજાવો.}

\begin{solutionbox}

\textbf{NPN ટ્રાન્ઝિસ્ટર ઓપરેશન}

\textbf{સિમ્બોલ અને બેસિક ઓપરેશન:}

\begin{verbatim}
      Collector (C)
          |
          |
          v
    +{-{-}{-}{-}{-}+{-}{-}{-}{-}{-}+}
    |     |     |
    |    / {    |}
Base|{-{-}{-}|   |{-}{-}{-}| Collector}
(B) |    { /    |}
    |     |     |
    +{-{-}{-}{-}{-}+{-}{-}{-}{-}{-}+}
          |
          |
          v
       Emitter (E)
\end{verbatim}

\textbf{બેસિક ઓપરેટિંગ પ્રિન્સિપલ:}

\begin{itemize}
\tightlist
\item
  બેઝ-એમિટર જંક્શન ફોરવર્ડ બાયસ્ડ છે
\item
  બેઝ-કલેક્ટર જંક્શન રિવર્સ બાયસ્ડ છે
\item
  નાનો બેઝ કરંટ મોટા કલેક્ટર કરંટને નિયંત્રિત કરે છે
\end{itemize}

{\def\LTcaptype{none} % do not increment counter
\begin{longtable}[]{@{}
  >{\raggedright\arraybackslash}p{(\linewidth - 4\tabcolsep) * \real{0.3333}}
  >{\raggedright\arraybackslash}p{(\linewidth - 4\tabcolsep) * \real{0.3958}}
  >{\raggedright\arraybackslash}p{(\linewidth - 4\tabcolsep) * \real{0.2708}}@{}}
\toprule\noalign{}
\begin{minipage}[b]{\linewidth}\raggedright
ઓપરેટિંગ મોડ
\end{minipage} & \begin{minipage}[b]{\linewidth}\raggedright
બાયસિંગ સ્થિતિઓ
\end{minipage} & \begin{minipage}[b]{\linewidth}\raggedright
વર્ણન
\end{minipage} \\
\midrule\noalign{}
\endhead
\bottomrule\noalign{}
\endlastfoot
\textbf{એક્ટિવ મોડ} & • B-E: ફોરવર્ડ બાયસ્ડ• B-C: રિવર્સ બાયસ્ડ & • સામાન્ય
એમ્પ્લિફિકેશન મોડ• IC = β \times IB \\
\textbf{કટઓફ મોડ} & • B-E: રિવર્સ બાયસ્ડ• B-C: રિવર્સ બાયસ્ડ & • ટ્રાન્ઝિસ્ટર
OFF• કોઈ કલેક્ટર કરંટ નહીં \\
\textbf{સેચુરેશન મોડ} & • B-E: ફોરવર્ડ બાયસ્ડ• B-C: ફોરવર્ડ બાયસ્ડ & •
ટ્રાન્ઝિસ્ટર પૂરો ON• મહત્તમ કલેક્ટર કરંટ \\
\end{longtable}
}

\begin{center}
\textbf{Mermaid Diagram (Code)}
\begin{verbatim}
{Shaded}
{Highlighting}[]
graph LR
    A[Base Current Injected] {-{-}{} B[Electrons from Emitter Enter Base]}
    B {-{-}{} C[Most Electrons Reach Collector]}
    C {-{-}{} D[Small Change in Base Current Controls Larger Collector Current]}
    style A fill:\#f96,stroke:\#333
    style B fill:\#69f,stroke:\#333
    style C fill:\#f9f,stroke:\#333
    style D fill:\#cfc,stroke:\#333
{Highlighting}
{Shaded}
\end{verbatim}
\end{center}

\textbf{NPN ટ્રાન્ઝિસ્ટરમાં કરંટ ફ્લો:}

\begin{itemize}
\tightlist
\item
  ઇલેક્ટ્રોન્સ એમિટરથી કલેક્ટર તરફ વહે છે
\item
  નાનો બેઝ કરંટ મોટા કલેક્ટર કરંટને નિયંત્રિત કરે છે
\item
  એમ્પ્લિફિકેશન ફેક્ટર (β) = IC/IB
\end{itemize}

\end{solutionbox}
\begin{mnemonicbox}
``BECAN'' - Base current Enables
Collector-to-emitter current Amplification in NPN

\end{mnemonicbox}
\subsection*{પ્રશ્ન 5(c) [7
ગુણ]}\label{q5c}

\textbf{ઇનપુટ અને આઉટપુટ લાક્ષણિકતાઓ સાથે ટ્રાન્ઝિસ્ટરનું કોમન એમિટર (CE) રૂપરેખાંકન
સમજાવો.}

\begin{solutionbox}

\textbf{કોમન એમિટર (CE) કોન્ફિગરેશન}

{\def\LTcaptype{none} % do not increment counter
\begin{longtable}[]{@{}
  >{\raggedright\arraybackslash}p{(\linewidth - 2\tabcolsep) * \real{0.4583}}
  >{\raggedright\arraybackslash}p{(\linewidth - 2\tabcolsep) * \real{0.5417}}@{}}
\toprule\noalign{}
\begin{minipage}[b]{\linewidth}\raggedright
ઘટક
\end{minipage} & \begin{minipage}[b]{\linewidth}\raggedright
વર્ણન
\end{minipage} \\
\midrule\noalign{}
\endhead
\bottomrule\noalign{}
\endlastfoot
\textbf{સર્કિટ કોન્ફિગરેશન} & • એમિટર ઇનપુટ અને આઉટપુટ બંને માટે કોમન છે• બેઝ અને
એમિટર વચ્ચે ઇનપુટ• કલેક્ટર અને એમિટર વચ્ચે આઉટપુટ \\
\textbf{ઇનપુટ પેરામીટર્સ} & • બેઝ કરંટ (IB)• બેઝ-એમિટર વોલ્ટેજ (VBE) \\
\textbf{આઉટપુટ પેરામીટર્સ} & • કલેક્ટર કરંટ (IC)• કલેક્ટર-એમિટર વોલ્ટેજ (VCE) \\
\end{longtable}
}

\textbf{સર્કિટ ડાયાગ્રામ:}

\begin{verbatim}
                 +Vcc
                   |
                   |
                  RL
                   |
                   |
    +{-{-}{-}{-}{-}+    +{-}{-}{-}o{-}{-}{-} Vout}
    |     |    |   |
Vin o{-{-}{-}{-}{-}o{-}{-}{-}{-}|B  C}
    |     |    |   |
    RB    |    |   |
    |     |    |E  |
    |     |    |   |
    +{-{-}{-}{-}{-}+{-}{-}{-}{-}+{-}{-}{-}o{-}{-}{-} GND}
                   |
                  RE
                   |
                   +
\end{verbatim}

\textbf{ઇનપુટ લાક્ષણિકતાઓ:}

\begin{itemize}
\tightlist
\item
  વિવિધ VCE મૂલ્યો માટે IB vs VBE પ્લોટ કરે છે
\item
  ફોરવર્ડ-બાયસ્ડ ડાયોડ લાક્ષણિકતા જેવું દેખાય છે
\item
  સિલિકોન ટ્રાન્ઝિસ્ટર માટે થ્રેશોલ્ડ વોલ્ટેજ \textasciitilde0.7V
\end{itemize}

\begin{verbatim}
    IB (μA)
      ↑
      |                /
      |               /
      |              /
      |             /
      |            /
      |           /
      |          /
      |         /
    {-{-}|{-}{-}{-}{-}{-}{-}{-}{-}/{-}{-}{-}{-}{-}{-}{-}{-}{-}{-}{-}{-}{-}{-}{-} VBE (V)}
      |     0.7V
\end{verbatim}

\textbf{આઉટપુટ લાક્ષણિકતાઓ:}

\begin{itemize}
\tightlist
\item
  વિવિધ IB મૂલ્યો માટે IC vs VCE પ્લોટ કરે છે
\item
  ત્રણ ક્ષેત્રો બતાવે છે: એક્ટિવ, સેચુરેશન, કટઓફ
\end{itemize}

\begin{verbatim}
    IC (mA)
      ↑
      |                 \_\_\_\_\_\_\_\_ IB = 50μA
      |                /
      |               /\_\_\_\_\_\_\_\_ IB = 40μA
      |              /
      |             /\_\_\_\_\_\_\_\_\_ IB = 30μA
      |            /
      |           /\_\_\_\_\_\_\_\_\_\_ IB = 20μA
      |          /
      |         /\_\_\_\_\_\_\_\_\_\_\_\_ IB = 10μA
      |        /
      |       /
    {-{-}|{-}{-}{-}{-}{-}{-}/{-}{-}{-}{-}{-}{-}{-}{-}{-}{-}{-}{-}{-}{-}{-}{-}{-}{-}{-} VCE (V)}
      |  Saturation│  Active
      |  Region    │  Region
\end{verbatim}

\textbf{લાક્ષણિકતાઓ:}

\begin{itemize}
\tightlist
\item
  કરંટ ગેઇન (β) = IC/IB (સામાન્ય રીતે 50-200)
\item
  ઇનપુટ રેઝિસ્ટન્સ: 1-2 kΩ
\item
  આઉટપુટ રેઝિસ્ટન્સ: 40-50 kΩ
\item
  ફેઝ શિફ્ટ: ઇનપુટ અને આઉટપુટ વચ્ચે 180^\circ
\end{itemize}

\end{solutionbox}
\begin{mnemonicbox}
``CASIO'' - Common emitter Amplifies Signals with
Inverted Output

\end{mnemonicbox}
\subsection*{પ્રશ્ન 5(a) OR [3
ગુણ]}\label{q5a}

\textbf{ઈ-કચરાના પ્રકારો જણાવો.}

\begin{solutionbox}

\textbf{ઇલેક્ટ્રોનિક વેસ્ટ (ઈ-વેસ્ટ) ના પ્રકારો}

{\def\LTcaptype{none} % do not increment counter
\begin{longtable}[]{@{}
  >{\raggedright\arraybackslash}p{(\linewidth - 2\tabcolsep) * \real{0.5000}}
  >{\raggedright\arraybackslash}p{(\linewidth - 2\tabcolsep) * \real{0.5000}}@{}}
\toprule\noalign{}
\begin{minipage}[b]{\linewidth}\raggedright
કેટેગરી
\end{minipage} & \begin{minipage}[b]{\linewidth}\raggedright
ઉદાહરણો
\end{minipage} \\
\midrule\noalign{}
\endhead
\bottomrule\noalign{}
\endlastfoot
\textbf{IT \& ટેલિકોમ્યુનિકેશન} & • કોમ્પ્યુટર, લેપટોપ, પ્રિન્ટર• મોબાઇલ ફોન,
ટેબ્લેટ• સર્વર, નેટવર્કિંગ ઇક્વિપમેન્ટ \\
\textbf{કન્ઝ્યુમર ઇલેક્ટ્રોનિક્સ} & • ટીવી, મોનિટર, ઓડિયો ઇક્વિપમેન્ટ• DVD/બ્લુ-રે
પ્લેયર• કેમેરા, વિડિયો રેકોર્ડર \\
\textbf{હોમ એપ્લાયન્સિસ} & • રેફ્રિજરેટર, વોશિંગ મશીન• માઇક્રોવેવ ઓવન, એર
કન્ડિશનર• નાના રસોડાના ઉપકરણો \\
\textbf{લાઇટિંગ ઇક્વિપમેન્ટ} & • ફ્લોરસન્ટ લેમ્પ, LED લાઇટ્સ• હાઈ-ઇન્ટેન્સિટી
ડિસ્ચાર્જ લેમ્પ \\
\textbf{ઇલેક્ટ્રિકલ \& ઇલેક્ટ્રોનિક ટૂલ્સ} & • ડ્રિલ, સૉ, સોલ્ડરિંગ ઇક્વિપમેન્ટ• લૉન
મોવર, ગાર્ડનિંગ ટૂલ્સ \\
\textbf{મેડિકલ ડિવાઇસિસ} & • ડાયગ્નોસ્ટિક ઇક્વિપમેન્ટ• ટ્રીટમેન્ટ ઇક્વિપમેન્ટ• લેબ
ઇક્વિપમેન્ટ \\
\textbf{મોનિટરિંગ ઇન્સ્ટ્રુમેન્ટ્સ} & • સ્મોક ડિટેક્ટર• થર્મોસ્ટેટ• કંટ્રોલ પેનલ \\
\textbf{ઇલેક્ટ્રોનિક કોમ્પોનન્ટ્સ} & • સર્કિટ બોર્ડ• બેટરી• કેબલ અને વાયર \\
\end{longtable}
}

\end{solutionbox}
\begin{mnemonicbox}
``CLIMATE'' - Computing, Lighting, Industrial,
Medical, Appliances, Telecommunications, Electronic components

\end{mnemonicbox}
\subsection*{પ્રશ્ન 5(b) OR [4
ગુણ]}\label{q5b}

\textbf{ઇલેક્ટ્રોનિક્સ વેસ્ટની વિવિધ શ્રેણીઓનું વર્ણન કરો.}

\begin{solutionbox}

\textbf{ઇલેક્ટ્રોનિક વેસ્ટની શ્રેણીઓ}

{\def\LTcaptype{none} % do not increment counter
\begin{longtable}[]{@{}
  >{\raggedright\arraybackslash}p{(\linewidth - 4\tabcolsep) * \real{0.3030}}
  >{\raggedright\arraybackslash}p{(\linewidth - 4\tabcolsep) * \real{0.3939}}
  >{\raggedright\arraybackslash}p{(\linewidth - 4\tabcolsep) * \real{0.3030}}@{}}
\toprule\noalign{}
\begin{minipage}[b]{\linewidth}\raggedright
શ્રેણી
\end{minipage} & \begin{minipage}[b]{\linewidth}\raggedright
વર્ણન
\end{minipage} & \begin{minipage}[b]{\linewidth}\raggedright
ઉદાહરણો
\end{minipage} \\
\midrule\noalign{}
\endhead
\bottomrule\noalign{}
\endlastfoot
\textbf{મોટા ઘરેલુ ઉપકરણો} & • ભારે આઇટમ ઉચ્ચ ધાતુ સામગ્રી સાથે• અક્સર રેફ્રિજરન્ટ
ધરાવે છે & • રેફ્રિજરેટર, ફ્રીઝર• વોશિંગ મશીન• એર કન્ડિશનર \\
\textbf{નાના ઘરેલુ ઉપકરણો} & • પોર્ટેબલ ઘરેલુ ડિવાઇસ• મિશ્ર સામગ્રી કમ્પોઝિશન & •
વેક્યુમ ક્લીનર• ટોસ્ટર, કોફી મશીન• ઇલેક્ટ્રિક પંખા \\
\textbf{IT \& ટેલિકોમ ઇક્વિપમેન્ટ} & • ડેટા પ્રોસેસિંગ/કોમ્યુનિકેશન ડિવાઇસ• ઉચ્ચ
કિંમતી ધાતુ સામગ્રી & • કોમ્પ્યુટર, લેપટોપ• પ્રિન્ટર, કોપિંગ ઇક્વિપમેન્ટ• મોબાઇલ ફોન,
ટેલિકોમ ઇક્વિપમેન્ટ \\
\textbf{કન્ઝ્યુમર ઇક્વિપમેન્ટ} & • મનોરંજન/મીડિયા ડિવાઇસ• અક્સર ડિસ્પ્લે સ્ક્રીન સાથે
& • ટીવી, મોનિટર• ઓડિયો/વિડિયો ઇક્વિપમેન્ટ• મ્યુઝિકલ ઇન્સ્ટ્રુમેન્ટ \\
\textbf{લાઇટિંગ ઇક્વિપમેન્ટ} & • મર્ક્યુરી અને અન્ય ધાતુઓ ધરાવે છે• વિશેષ હેન્ડલિંગની
જરૂર & • ફ્લોરસન્ટ લેમ્પ• હાઈ-ઇન્ટેન્સિટી ડિસ્ચાર્જ લેમ્પ• LED લાઇટિંગ \\
\textbf{ઇલેક્ટ્રિકલ \& ઇલેક્ટ્રોનિક ટૂલ્સ} & • પોર્ટેબલ અથવા ફિક્સ્ડ પાવર ટૂલ્સ• ઊંચી
મોટર સામગ્રી & • ડ્રિલ, સૉ• સિલાઈ મશીન• બાંધકામ ઉપકરણો \\
\textbf{ટોય્સ \& સ્પોર્ટ્સ ઇક્વિપમેન્ટ} & • ઇલેક્ટ્રોનિક રમતો અને મનોરંજન આઇટમ• મિશ્ર
પ્લાસ્ટિક અને ઇલેક્ટ્રોનિક ઘટકો & • વિડિયો ગેમ કન્સોલ• ઇલેક્ટ્રિક ટ્રેન/રેસિંગ સેટ•
ઇલેક્ટ્રોનિક્સ સાથે એક્સરસાઇઝ ઇક્વિપમેન્ટ \\
\textbf{મેડિકલ ડિવાઇસિસ} & • વિશિષ્ટ હેલ્થકેર ઇક્વિપમેન્ટ• અક્સર મૂલ્યવાન અને જોખમી
સામગ્રી ધરાવે છે & • ડાયગ્નોસ્ટિક ઇક્વિપમેન્ટ• રેડિએશન થેરાપી ઇક્વિપમેન્ટ• લેબોરેટરી
ઇક્વિપમેન્ટ \\
\end{longtable}
}

\begin{verbatim}
pie
    title "Typical E{-Waste Composition by Category"}
    "IT \& Telecom" : 25
    "Large Appliances" : 29
    "Small Appliances" : 14
    "Consumer Electronics" : 17
    "Lighting" : 5
    "Other Categories" : 10
\end{verbatim}

\end{solutionbox}
\begin{mnemonicbox}
``LIMCEST'' - Large appliances, IT equipment,
Medical devices, Consumer electronics, Electronic tools, Small
appliances, Telecom equipment

\end{mnemonicbox}
\subsection*{પ્રશ્ન 5(c) OR [7
ગુણ]}\label{q5c}

\textbf{ટ્રાન્ઝિસ્ટરને કટઓફ અને સંતૃત્તિ પ્રદેશમાં સ્વિચ તરીકે સમજાવો.}

\begin{solutionbox}

\textbf{ટ્રાન્ઝિસ્ટર એઝ એ સ્વિચ}

{\def\LTcaptype{none} % do not increment counter
\begin{longtable}[]{@{}
  >{\raggedright\arraybackslash}p{(\linewidth - 6\tabcolsep) * \real{0.1818}}
  >{\raggedright\arraybackslash}p{(\linewidth - 6\tabcolsep) * \real{0.1591}}
  >{\raggedright\arraybackslash}p{(\linewidth - 6\tabcolsep) * \real{0.2727}}
  >{\raggedright\arraybackslash}p{(\linewidth - 6\tabcolsep) * \real{0.3864}}@{}}
\toprule\noalign{}
\begin{minipage}[b]{\linewidth}\raggedright
પ્રદેશ
\end{minipage} & \begin{minipage}[b]{\linewidth}\raggedright
સ્થિતિ
\end{minipage} & \begin{minipage}[b]{\linewidth}\raggedright
સ્થિતિઓ
\end{minipage} & \begin{minipage}[b]{\linewidth}\raggedright
લાક્ષણિકતાઓ
\end{minipage} \\
\midrule\noalign{}
\endhead
\bottomrule\noalign{}
\endlastfoot
\textbf{કટઓફ પ્રદેશ} & OFF & • VBE \textless{} 0.7V• IB \approx 0 & • IC \approx 0•
VCE \approx VCC• ઉચ્ચ ઇમ્પીડન્સ \\
\textbf{સેચુરેશન પ્રદેશ} & ON & • VBE \textgreater{} 0.7V• IB \textgreater{}
IC/β & • IC \approx IC(sat)• VCE \approx 0.2V• ઓછો ઇમ્પીડન્સ \\
\end{longtable}
}

\textbf{સર્કિટ ડાયાગ્રામ:}

\begin{verbatim}
                  +Vcc
                    |
                    |
                    R
                    |
                    |
                    C
           +{-{-}{-}{-}{-}{-}{-}{-}+{-}{-}{-}{-}{-}{-}{-}{-}+}
           |                 |
Input o{-{-}{-}{-}+{-}{-}{-}{-}[RB]{-}{-}{-}{-}+B   |}
           |        |E  |    |
           |        |   |    |
           +{-{-}{-}{-}{-}{-}{-}{-}+{-}{-}{-}+{-}{-}{-}{-}o Output}
                    |
                    |
                   GND
\end{verbatim}

\textbf{કટઓફ ઓપરેશન (OFF સ્ટેટ):}

\begin{itemize}
\tightlist
\item
  ઇનપુટ વોલ્ટેજ 0.7V કરતાં નીચે છે (સામાન્ય રીતે 0V)
\item
  બેઝ-એમિટર જંક્શન ફોરવર્ડ બાયસ્ડ નથી
\item
  કોઈ બેઝ કરંટ વહેતો નથી (IB \approx 0)
\item
  કોઈ કલેક્ટર કરંટ વહેતો નથી (IC \approx 0)
\item
  કલેક્ટર-એમિટર વોલ્ટેજ લગભગ VCC છે
\item
  ટ્રાન્ઝિસ્ટર ઓપન સ્વિચ તરીકે કાર્ય કરે છે
\end{itemize}

\textbf{સેચુરેશન ઓપરેશન (ON સ્ટેટ):}

\begin{itemize}
\tightlist
\item
  ઇનપુટ વોલ્ટેજ 0.7V થી ઉપર છે
\item
  બેઝ-એમિટર જંક્શન ફોરવર્ડ બાયસ્ડ છે
\item
  પૂરતો બેઝ કરંટ વહે છે (IB \textgreater{} IC/β)
\item
  કલેક્ટર કરંટ મહત્તમ સ્તરે પહોંચે છે (IC(sat))
\item
  કલેક્ટર-એમિટર વોલ્ટેજ લઘુત્તમ થઈ જાય છે (VCE(sat) \approx 0.2V)
\item
  ટ્રાન્ઝિસ્ટર ક્લોઝ્ડ સ્વિચ તરીકે કાર્ય કરે છે
\end{itemize}

\begin{center}
\textbf{Mermaid Diagram (Code)}
\begin{verbatim}
{Shaded}
{Highlighting}[]
graph LR
    A[Input Signal] {-{-}{} B\{Voltage Level?\}}
    B {-{-}{}|V {} 0.7V| C[Cutoff Region{}br /{}Switch OFF]}
    B {-{-}{}|V {} 0.7V| D[Saturation Region{}br /{}Switch ON]}
    C {-{-}{} E[High V\_CE{}br /{}No Current]}
    D {-{-}{} F[Low V\_CE{}br /{}Maximum Current]}
    style A fill:\#f96,stroke:\#333
    style B fill:\#69f,stroke:\#333
    style C fill:\#f9f,stroke:\#333
    style D fill:\#cfc,stroke:\#333
    style E fill:\#9cf,stroke:\#333
    style F fill:\#fc9,stroke:\#333
{Highlighting}
{Shaded}
\end{verbatim}
\end{center}

\textbf{ઉપયોગો:}

\begin{itemize}
\tightlist
\item
  ડિજિટલ લોજિક સર્કિટ
\item
  રિલે અને મોટર ડ્રાઇવર
\item
  LED અને લેમ્પ કંટ્રોલ
\item
  પાવર કન્વર્ટર
\item
  સિગ્નલ કન્ડિશનિંગ
\end{itemize}

\textbf{મુખ્ય ડિઝાઇન વિચારણાઓ:}

\begin{itemize}
\tightlist
\item
  બેઝ રેસિસ્ટર (RB) બેઝ કરંટને મર્યાદિત કરે છે
\item
  કલેક્ટર રેસિસ્ટર (RC) કલેક્ટર કરંટને મર્યાદિત કરે છે
\item
  વિશ્વસનીય સ્વિચિંગ માટે સેચુરેશનમાં IB \textgreater{} IC/β હોવું જરૂરી છે
\item
  ઝડપી સ્વિચિંગ માટે ચાર્જ સ્ટોરેજ ઇફેક્ટ્સનું ધ્યાન રાખવું જરૂરી છે
\end{itemize}

\end{solutionbox}
\begin{mnemonicbox}
``COSVL'' - Cutoff means Off State with Vce Large,
saturation means low Vce

\end{mnemonicbox}

\end{document}
