\documentclass[10pt,a4paper]{article}

% content/resources/templates/preamble.tex
\usepackage[margin=0.6in]{geometry}
\author{Milav Dabgar}
\usepackage{amsmath,amssymb,amsthm}
\usepackage{booktabs}
\usepackage{multirow}
\usepackage{xcolor}
\usepackage{tcolorbox}
\tcbuselibrary{breakable,skins}
\usepackage[colorlinks=true,linkcolor=blue]{hyperref}
\usepackage{titlesec}
\usepackage{enumitem}
\usepackage{tikz}
\usepackage{pgfplots}
\usepackage{circuitikz}
\usepackage[version=4]{mhchem}
\usepackage{longtable}
\usepackage{array}
\usepackage{float}
\usepackage{caption}
\usepackage{listings}

\lstset{
  basicstyle=\small\ttfamily,
  breaklines=true,
  breakatwhitespace=false,
  postbreak=\mbox{\textcolor{red}{$\hookrightarrow$}\space},
  float=false,
  numbers=left,
  numberstyle=\tiny\color{gray},
  numbersep=10pt,
  xleftmargin=2em,
  keywordstyle=\color{blue},
  commentstyle=\color{green!60!black},
  stringstyle=\color{purple},
  backgroundcolor=\color{gray!5},
  showstringspaces=false,
  tabsize=2,
  captionpos=b,
  keepspaces=true,
  columns=flexible
}

\pgfplotsset{compat=1.18}
\usetikzlibrary{shapes,arrows,positioning,calc,patterns,decorations.pathmorphing,decorations.markings,arrows.meta}

% Color scheme
\definecolor{headcolor}{RGB}{0,102,204}
\definecolor{keycolor}{RGB}{220,20,60}
\definecolor{solutioncolor}{RGB}{34,139,34}
\definecolor{mnemoniccolor}{RGB}{148,0,211}
\definecolor{codecolor}{RGB}{0,0,100}

% Spacing
\setlength{\parskip}{3pt}
\setlist[itemize]{nosep}
\setlist[enumerate]{nosep}

% Title formatting
\titleformat{\section}{\Large\bfseries\color{headcolor}}{\thesection}{1em}{}
\titleformat{\subsection}{\large\bfseries\color{headcolor}}{\thesubsection}{1em}{}

% Pandoc tightlist compatibility
\providecommand{\tightlist}{%
  \setlength{\itemsep}{0pt}\setlength{\parskip}{0pt}}

% Pandoc longtable compatibility
\newcounter{none}
\def\thenone{}


% content/resources/templates/english-boxes.tex
% This file is currently empty - it exists to maintain consistency with the import structure.
% Add custom environments here if needed in the future.


\begin{document}

\begin{center}
{\Huge\bfseries\color{headcolor} Subject Name Solutions}\\[5pt]
{\LARGE 4311102 -- Summer 2023}\\[3pt]
{\large Semester 1 Study Material}\\[3pt]
{\normalsize\textit{Detailed Solutions and Explanations}}
\end{center}

\vspace{10pt}

\subsection*{Question 1(a) [3 marks]}\label{q1a}

\textbf{Define Active and Passive components.}

\begin{solutionbox}

{\def\LTcaptype{none} % do not increment counter
\begin{longtable}[]{@{}
  >{\raggedright\arraybackslash}p{(\linewidth - 2\tabcolsep) * \real{0.4865}}
  >{\raggedright\arraybackslash}p{(\linewidth - 2\tabcolsep) * \real{0.5135}}@{}}
\toprule\noalign{}
\begin{minipage}[b]{\linewidth}\raggedright
Active Components
\end{minipage} & \begin{minipage}[b]{\linewidth}\raggedright
Passive Components
\end{minipage} \\
\midrule\noalign{}
\endhead
\bottomrule\noalign{}
\endlastfoot
• Require external power source to operate & • Do not need external
power source \\
• Can amplify and process electrical signals & • Cannot amplify or
process signals \\
• Examples: transistors, diodes, ICs & • Examples: resistors,
capacitors, inductors \\
\end{longtable}
}

\end{solutionbox}
\begin{mnemonicbox}
``APE'' - Active needs Power to Enhance signals

\end{mnemonicbox}
\subsection*{Question 1(b) [4 marks]}\label{q1b}

\textbf{State types of capacitors based on materials used.}

\begin{solutionbox}


{\def\LTcaptype{none} % do not increment counter
\vspace{-5pt}
\captionof{table}{Types of Capacitors Based on Materials}
\vspace{-10pt}
\begin{longtable}[]{@{}
  >{\raggedright\arraybackslash}p{(\linewidth - 4\tabcolsep) * \real{0.2692}}
  >{\raggedright\arraybackslash}p{(\linewidth - 4\tabcolsep) * \real{0.3077}}
  >{\raggedright\arraybackslash}p{(\linewidth - 4\tabcolsep) * \real{0.4231}}@{}}
\toprule\noalign{}
\begin{minipage}[b]{\linewidth}\raggedright
Material Type
\end{minipage} & \begin{minipage}[b]{\linewidth}\raggedright
Capacitor Type
\end{minipage} & \begin{minipage}[b]{\linewidth}\raggedright
Typical Applications
\end{minipage} \\
\midrule\noalign{}
\endhead
\bottomrule\noalign{}
\endlastfoot
Ceramic & Ceramic disc, multilayer & Bypass, coupling, high frequency \\
Plastic Film & Polyester, Polypropylene, Teflon & Timing, filtering,
precision \\
Electrolytic & Aluminum, Tantalum & Power supply, DC blocking, high
capacitance \\
Paper & Paper dielectric & Old equipment, not common now \\
Mica & Silvered mica & High precision RF circuits \\
Glass & Glass dielectric & High voltage applications \\
\end{longtable}
}

\end{solutionbox}
\begin{mnemonicbox}
``CEPPMG'' - Ceramic Electrolytic Paper Plastic Mica
Glass

\end{mnemonicbox}
\subsection*{Question 1(c) [7 marks]}\label{q1c}

\textbf{Explain resistor color coding technique with example.}

\begin{solutionbox}

The resistor color code uses colored bands to indicate resistance value,
tolerance, and reliability.


{\def\LTcaptype{none} % do not increment counter
\vspace{-5pt}
\captionof{table}{Standard Resistor Color Code}
\vspace{-10pt}
\begin{longtable}[]{@{}llll@{}}
\toprule\noalign{}
Color & Digit Value & Multiplier & Tolerance \\
\midrule\noalign{}
\endhead
\bottomrule\noalign{}
\endlastfoot
Black & 0 & \times10^{0} (1) & - \\
Brown & 1 & \times10^{1} (10) & \pm1\% \\
Red & 2 & \times10^{2} (100) & \pm2\% \\
Orange & 3 & \times10^{3} (1,000) & - \\
Yellow & 4 & \times10^{4} (10,000) & - \\
Green & 5 & \times10^{5} (100,000) & \pm0.5\% \\
Blue & 6 & \times10^{6} (1,000,000) & \pm0.25\% \\
Violet & 7 & \times10^{7} (10,000,000) & \pm0.1\% \\
Grey & 8 & \times10^{8} (100,000,000) & \pm0.05\% \\
White & 9 & \times10^{9} (1,000,000,000) & - \\
Gold & - & \times0.1 (0.1) & \pm5\% \\
Silver & - & \times0.01 (0.01) & \pm10\% \\
\end{longtable}
}

\textbf{Example 1:} Red-Violet-Orange-Gold

\begin{itemize}
\tightlist
\item
  1st band (Red) = 2
\item
  2nd band (Violet) = 7
\item
  3rd band (Orange) = \times1,000
\item
  4th band (Gold) = \pm5\% tolerance
\item
  Value: 27 \times 1,000 = 27,000Ω = 27kΩ \pm5\%
\end{itemize}

\textbf{Example 2:} Brown-Black-Yellow-Silver

\begin{itemize}
\tightlist
\item
  1st band (Brown) = 1
\item
  2nd band (Black) = 0
\item
  3rd band (Yellow) = \times10,000
\item
  4th band (Silver) = \pm10\% tolerance
\item
  Value: 10 \times 10,000 = 100,000Ω = 100kΩ \pm10\%
\end{itemize}

\begin{verbatim}
flowchart LR
    A[1st Band{br /First Digit] {-}{-} B[2nd Bandbr /Second Digit]}
    B {-{-} C[3rd Bandbr /Multiplier]}
    C {-{-} D[4th Bandbr /Tolerance]}
    style A fill:\#f96,stroke:\#333
    style B fill:\#69f,stroke:\#333
    style C fill:\#f90,stroke:\#333
    style D fill:\#fc0,stroke:\#333
\end{verbatim}

\end{solutionbox}
\begin{mnemonicbox}
``BBROY Great Britain Very Good Wife'' for colors 0-9
(Black Brown Red Orange Yellow Green Blue Violet Gray White)

\end{mnemonicbox}
\subsection*{Question 1(c) OR [7
marks]}\label{q1c}

\textbf{Explain construction, working Characteristic and application of
LDR.}

\begin{solutionbox}

\textbf{Light Dependent Resistor (LDR)}

{\def\LTcaptype{none} % do not increment counter
\begin{longtable}[]{@{}
  >{\raggedright\arraybackslash}p{(\linewidth - 2\tabcolsep) * \real{0.3810}}
  >{\raggedright\arraybackslash}p{(\linewidth - 2\tabcolsep) * \real{0.6190}}@{}}
\toprule\noalign{}
\begin{minipage}[b]{\linewidth}\raggedright
Aspect
\end{minipage} & \begin{minipage}[b]{\linewidth}\raggedright
Description
\end{minipage} \\
\midrule\noalign{}
\endhead
\bottomrule\noalign{}
\endlastfoot
\textbf{Construction} & • Semiconductor material (cadmium sulfide)
deposited in zigzag pattern• Packaged in transparent case to allow light
exposure• Two terminals connected to the semiconductor \\
\textbf{Working Principle} & • Resistance decreases when light intensity
increases• Photons release electrons in semiconductor material• More
light = more free electrons = lower resistance \\
\textbf{Characteristics} & • High resistance in darkness (MΩ range)• Low
resistance in bright light (100-5000Ω)• Non-linear response to light
intensity• Slow response time (tens of milliseconds) \\
\textbf{Applications} & • Automatic street lights• Light meters in
cameras• Burglar alarm systems• Automatic brightness control in
displays \\
\end{longtable}
}

\begin{center}
\textbf{Mermaid Diagram (Code)}
\begin{verbatim}
{Shaded}
{Highlighting}[]
graph LR
    A[More Light] {-{-}{}|Releases electrons| B[More Free Electrons]}
    B {-{-}{} C[Lower Resistance]}
    D[Less Light] {-{-}{}|Fewer electrons released| E[Fewer Free Electrons]}
    E {-{-}{} F[Higher Resistance]}
{Highlighting}
{Shaded}
\end{verbatim}
\end{center}

\end{solutionbox}
\begin{mnemonicbox}
``MOLD'' - More light On, Less resistance Down

\end{mnemonicbox}
\subsection*{Question 2(a) [3 marks]}\label{q2a}

\textbf{Classify Resistors based on materials.}

\begin{solutionbox}


{\def\LTcaptype{none} % do not increment counter
\vspace{-5pt}
\captionof{table}{Resistor Classification Based on Materials}
\vspace{-10pt}
\begin{longtable}[]{@{}
  >{\raggedright\arraybackslash}p{(\linewidth - 4\tabcolsep) * \real{0.3415}}
  >{\raggedright\arraybackslash}p{(\linewidth - 4\tabcolsep) * \real{0.4146}}
  >{\raggedright\arraybackslash}p{(\linewidth - 4\tabcolsep) * \real{0.2439}}@{}}
\toprule\noalign{}
\begin{minipage}[b]{\linewidth}\raggedright
Material Type
\end{minipage} & \begin{minipage}[b]{\linewidth}\raggedright
Characteristics
\end{minipage} & \begin{minipage}[b]{\linewidth}\raggedright
Examples
\end{minipage} \\
\midrule\noalign{}
\endhead
\bottomrule\noalign{}
\endlastfoot
Carbon Composition & Low cost, noisy, poor tolerance & General purpose
resistors \\
Carbon Film & Better stability than carbon composition & Audio
equipment, general circuits \\
Metal Film & Excellent stability, low noise & Precision circuits,
instrumentation \\
Metal Oxide & High stability, heat resistant & Power supplies,
high-voltage circuits \\
Wire Wound & High power rating, inductive & Power circuits, heating
elements \\
Thick \& Thin Film & Small size, good stability & Surface mount
applications \\
\end{longtable}
}

\end{solutionbox}
\begin{mnemonicbox}
``CMMWTF'' - Carbon Makes Much Wire To Form resistors

\end{mnemonicbox}
\subsection*{Question 2(b) [4 marks]}\label{q2b}

\textbf{Calculate value of resistor for a given color code. -- (i)
Brown, Black, Yellow, Golden (ii) Yellow, Violet, Red, Silver}

\begin{solutionbox}

\textbf{Part (i): Brown, Black, Yellow, Golden}

\begin{itemize}
\tightlist
\item
  1st Band (Brown) = 1
\item
  2nd Band (Black) = 0
\item
  3rd Band (Yellow) = \times10,000
\item
  4th Band (Golden) = \pm5\% tolerance
\end{itemize}

\textbf{Calculation:} Value = 10 \times 10,000 = 100,000Ω = 100kΩ \pm5\%

\textbf{Part (ii): Yellow, Violet, Red, Silver}

\begin{itemize}
\tightlist
\item
  1st Band (Yellow) = 4
\item
  2nd Band (Violet) = 7
\item
  3rd Band (Red) = \times100
\item
  4th Band (Silver) = \pm10\% tolerance
\end{itemize}

\textbf{Calculation:} Value = 47 \times 100 = 4,700Ω = 4.7kΩ \pm10\%

\end{solutionbox}
\begin{mnemonicbox}
``BBROY Great Britain Very Good Wife'' for the color
sequence 0-9

\end{mnemonicbox}
\subsection*{Question 2(c) [7 marks]}\label{q2c}

\textbf{Illustrate construction and operation of Electrolytic
capacitors.}

\begin{solutionbox}

\textbf{Electrolytic Capacitor Construction and Operation}

{\def\LTcaptype{none} % do not increment counter
\begin{longtable}[]{@{}ll@{}}
\toprule\noalign{}
Component & Description \\
\midrule\noalign{}
\endhead
\bottomrule\noalign{}
\endlastfoot
\textbf{Anode} & Aluminum or tantalum foil with oxide layer
(dielectric) \\
\textbf{Cathode} & Electrolyte (liquid, paste or solid) and metal
foil \\
\textbf{Separator} & Paper soaked in electrolyte \\
\textbf{Casing} & Aluminum can with insulating sleeve \\
\textbf{Terminals} & Positive (+) and negative (-) leads \\
\end{longtable}
}

\textbf{Operation Principle:}

\begin{enumerate}
\tightlist
\item
  The oxide layer on the anode acts as an extremely thin dielectric
\item
  The large surface area and thin dielectric create high capacitance
\item
  When connected to DC voltage (with correct polarity), charges
  accumulate
\item
  Positive plate (+) attracts negative charges; negative plate (-)
  attracts positive charges
\end{enumerate}

\begin{center}
\textbf{Mermaid Diagram (Code)}
\begin{verbatim}
{Shaded}
{Highlighting}[]
graph LR
    A[Aluminum Foil{br /{}Anode] {-}{-}{} B[Oxide Layer{}br /{}Dielectric]}
    B {-{-}{} C[Electrolyte{}br /{}Cathode]}
    C {-{-}{} D[Aluminum Foil{}br /{}Terminal Connection]}
    style A fill:\#fc9,stroke:\#333
    style B fill:\#9cf,stroke:\#333
    style C fill:\#cfc,stroke:\#333
    style D fill:\#fc9,stroke:\#333
{Highlighting}
{Shaded}
\end{verbatim}
\end{center}

\textbf{Key Characteristics:}

\begin{itemize}
\tightlist
\item
  \textbf{Polarity}: Must be connected correctly (+/-)
\item
  \textbf{High capacitance}: 1μF to thousands of μF
\item
  \textbf{Voltage limitations}: Breakdown if exceeded
\item
  \textbf{Leakage current}: Higher than other capacitor types
\end{itemize}

\end{solutionbox}
\begin{mnemonicbox}
``PAVE'' - Polarized Aluminum with Very high
capacitance and Electrolyte

\end{mnemonicbox}
\subsection*{Question 2(a) OR [3
marks]}\label{q2a}

\textbf{State the importance of filter circuit in rectifier.}

\begin{solutionbox}

\textbf{Importance of Filter Circuit in Rectifier}

{\def\LTcaptype{none} % do not increment counter
\begin{longtable}[]{@{}
  >{\raggedright\arraybackslash}p{(\linewidth - 2\tabcolsep) * \real{0.4348}}
  >{\raggedright\arraybackslash}p{(\linewidth - 2\tabcolsep) * \real{0.5652}}@{}}
\toprule\noalign{}
\begin{minipage}[b]{\linewidth}\raggedright
Function
\end{minipage} & \begin{minipage}[b]{\linewidth}\raggedright
Description
\end{minipage} \\
\midrule\noalign{}
\endhead
\bottomrule\noalign{}
\endlastfoot
\textbf{Smoothing} & Converts pulsating DC to smooth DC by reducing
ripples \\
\textbf{Voltage Stabilization} & Maintains steady output voltage despite
input fluctuations \\
\textbf{Ripple Reduction} & Decreases unwanted AC components in DC
output \\
\textbf{Load Protection} & Protects electronic devices from voltage
variations \\
\end{longtable}
}

\end{solutionbox}
\begin{mnemonicbox}
``SVRL'' - Smoothens Voltage by Reducing ripples for
Load

\end{mnemonicbox}
\subsection*{Question 2(b) OR [4
marks]}\label{q2b}

\textbf{Differentiate between P type semiconductor and N type
semiconductor.}

\begin{solutionbox}


{\def\LTcaptype{none} % do not increment counter
\vspace{-5pt}
\captionof{table}{P-type vs N-type Semiconductor}
\vspace{-10pt}
\begin{longtable}[]{@{}
  >{\raggedright\arraybackslash}p{(\linewidth - 4\tabcolsep) * \real{0.2667}}
  >{\raggedright\arraybackslash}p{(\linewidth - 4\tabcolsep) * \real{0.3667}}
  >{\raggedright\arraybackslash}p{(\linewidth - 4\tabcolsep) * \real{0.3667}}@{}}
\toprule\noalign{}
\begin{minipage}[b]{\linewidth}\raggedright
Characteristic
\end{minipage} & \begin{minipage}[b]{\linewidth}\raggedright
P-type Semiconductor
\end{minipage} & \begin{minipage}[b]{\linewidth}\raggedright
N-type Semiconductor
\end{minipage} \\
\midrule\noalign{}
\endhead
\bottomrule\noalign{}
\endlastfoot
\textbf{Dopant used} & Trivalent elements (B, Al, Ga) & Pentavalent
elements (P, As, Sb) \\
\textbf{Majority carriers} & Holes (positive charge carriers) &
Electrons (negative charge carriers) \\
\textbf{Minority carriers} & Electrons & Holes \\
\textbf{Conductivity} & Due to movement of holes & Due to movement of
electrons \\
\textbf{Energy level} & Acceptor atoms near valence band & Donor atoms
near conduction band \\
\textbf{Electrical charge} & Overall neutral, but accepts electrons &
Overall neutral, but donates electrons \\
\end{longtable}
}

\end{solutionbox}
\begin{mnemonicbox}
``HELP-NED'' - Holes Exist in Large quantities in
P-type, Negative Electrons Dominate N-type

\end{mnemonicbox}
\subsection*{Question 2(c) OR [7
marks]}\label{q2c}

\textbf{Illustrate working of Bridge Rectifier with waveforms.}

\begin{solutionbox}

\textbf{Bridge Rectifier Working Principle}

{\def\LTcaptype{none} % do not increment counter
\begin{longtable}[]{@{}ll@{}}
\toprule\noalign{}
Component & Function \\
\midrule\noalign{}
\endhead
\bottomrule\noalign{}
\endlastfoot
\textbf{Diodes (D1-D4)} & Four diodes arranged in bridge
configuration \\
\textbf{Input} & AC voltage from transformer secondary \\
\textbf{Output} & Pulsating DC voltage across load resistor \\
\textbf{Operation} & Converts both halves of AC cycle to same
polarity \\
\end{longtable}
}

\textbf{Working in Positive Half Cycle:}

\begin{itemize}
\tightlist
\item
  Diodes D1 and D3 conduct
\item
  Diodes D2 and D4 are reverse biased (off)
\item
  Current flows: AC+ \rightarrow D1 \rightarrow Load \rightarrow D3 \rightarrow AC-
\end{itemize}

\textbf{Working in Negative Half Cycle:}

\begin{itemize}
\tightlist
\item
  Diodes D2 and D4 conduct
\item
  Diodes D1 and D3 are reverse biased (off)
\item
  Current flows: AC- \rightarrow D2 \rightarrow Load \rightarrow D4 \rightarrow AC+
\end{itemize}

\begin{center}
\textbf{Mermaid Diagram (Code)}
\begin{verbatim}
{Shaded}
{Highlighting}[]
graph LR
    AC[AC Input] {-{-}{} D1[D1]}
    AC {-{-}{} D3[D3]}
    D1 {-{-}{} Load[Load]}
    D3 {-{-}{} Load}
    Load {-{-}{} D2[D2]}
    Load {-{-}{} D4[D4]}
    D2 {-{-}{} AC}
    D4 {-{-}{} AC}
    style AC fill:\#fcf,stroke:\#333
    style Load fill:\#cfc,stroke:\#333
    style D1 fill:\#9cf,stroke:\#333
    style D2 fill:\#9cf,stroke:\#333
    style D3 fill:\#9cf,stroke:\#333
    style D4 fill:\#9cf,stroke:\#333
{Highlighting}
{Shaded}
\end{verbatim}
\end{center}

\textbf{Waveforms:}

\begin{verbatim}
AC Input:      \_/{      \_/      \_/      }
              /    {   /       /       }
    0 \_\_\_\_\_\_/      {\_/      \_/      \_\_}
             {    /       /       /}
              {\_\_/     \_\_/     \_\_/}


DC Output:     \_       \_       \_      
              / {     /      /      }
    0 \_\_\_\_\_\_/   {\_\_\_/   \_\_\_/   \_\_\_\_\_}
\end{verbatim}

\textbf{Advantages:}

\begin{itemize}
\tightlist
\item
  Utilizes both half cycles of AC input
\item
  Higher output voltage and efficiency compared to half-wave
\item
  No center-tapped transformer required
\end{itemize}

\end{solutionbox}
\begin{mnemonicbox}
``FBRO'' - Four diodes, Both cycles, Rectified Output

\end{mnemonicbox}
\subsection*{Question 3(a) [3 marks]}\label{q3a}

\textbf{Define (1) PIV (2) Ripple Factor.}

\begin{solutionbox}

{\def\LTcaptype{none} % do not increment counter
\begin{longtable}[]{@{}
  >{\raggedright\arraybackslash}p{(\linewidth - 2\tabcolsep) * \real{0.3333}}
  >{\raggedright\arraybackslash}p{(\linewidth - 2\tabcolsep) * \real{0.6667}}@{}}
\toprule\noalign{}
\begin{minipage}[b]{\linewidth}\raggedright
Term
\end{minipage} & \begin{minipage}[b]{\linewidth}\raggedright
Definition
\end{minipage} \\
\midrule\noalign{}
\endhead
\bottomrule\noalign{}
\endlastfoot
\textbf{PIV (Peak Inverse Voltage)} & • Maximum voltage a diode can
withstand in reverse bias condition• Important rating to prevent diode
breakdown• Must be higher than maximum reverse voltage in circuit \\
\textbf{Ripple Factor (r)} & • Measure of effectiveness of a rectifier
filter• Ratio of RMS value of AC component to DC component in output•
Lower ripple factor indicates better filtering \\
\end{longtable}
}

\textbf{Formula:} Ripple Factor (r) = V_{(}ᵣ_{m}_{s}_{)}_{a}._{k} / V_{(}ᵈᶜ_{)}

\end{solutionbox}
\begin{mnemonicbox}
``PIR'' - Peak Inverse voltage Restricts, Ripple
indicates Rectification quality

\end{mnemonicbox}
\subsection*{Question 3(b) [4 marks]}\label{q3b}

\textbf{Illustrate VI characteristics of PN junction diode.}

\begin{solutionbox}

\textbf{V-I Characteristics of PN Junction Diode}

{\def\LTcaptype{none} % do not increment counter
\begin{longtable}[]{@{}
  >{\raggedright\arraybackslash}p{(\linewidth - 4\tabcolsep) * \real{0.2286}}
  >{\raggedright\arraybackslash}p{(\linewidth - 4\tabcolsep) * \real{0.2857}}
  >{\raggedright\arraybackslash}p{(\linewidth - 4\tabcolsep) * \real{0.4857}}@{}}
\toprule\noalign{}
\begin{minipage}[b]{\linewidth}\raggedright
Region
\end{minipage} & \begin{minipage}[b]{\linewidth}\raggedright
Behavior
\end{minipage} & \begin{minipage}[b]{\linewidth}\raggedright
Characteristics
\end{minipage} \\
\midrule\noalign{}
\endhead
\bottomrule\noalign{}
\endlastfoot
\textbf{Forward Bias} & Conducts current easily & • Exponential increase
in current after threshold• Threshold voltage: \textasciitilde0.7V for
silicon, \textasciitilde0.3V for germanium \\
\textbf{Reverse Bias} & Blocks current & • Very small leakage current
(μA)• Breakdown at reverse breakdown voltage \\
\end{longtable}
}

\begin{verbatim}
         Current (I)
             ↑
             |              /
             |             /
             |            /
             |           /
             |          /
    {-{-}{-}{-}{-}{-}{-}{-}{-}|{-}{-}{-}{-}{-}{-}{-}{-}/{-}{-}{-}{-}{-} Voltage (V)}
             |    0.7V
             |/
     \_\_\_\_\_\_\_\_|\_\_\_\_\_\_\_\_\_\_\_\_\_\_\_\_\_\_\_\_\_\_\_\_
             |
             | Small leakage current
             |
             |        Breakdown
             |           ↓
             |           |
             |           |
\end{verbatim}

\textbf{Key Points:}

\begin{itemize}
\tightlist
\item
  \textbf{Forward threshold}: \textasciitilde0.7V for Si,
  \textasciitilde0.3V for Ge
\item
  \textbf{Forward region}: High conductivity
\item
  \textbf{Reverse region}: Very high resistance
\item
  \textbf{Breakdown region}: Sudden increase in reverse current
\end{itemize}

\end{solutionbox}
\begin{mnemonicbox}
``FBRL'' - Forward Bias Resists Little, reverse
blocks lots

\end{mnemonicbox}
\subsection*{Question 3(c) [7 marks]}\label{q3c}

\textbf{Explain the working of capacitor input and choke input filter
with waveforms.}

\begin{solutionbox}

\textbf{1. Capacitor Input Filter}

{\def\LTcaptype{none} % do not increment counter
\begin{longtable}[]{@{}
  >{\raggedright\arraybackslash}p{(\linewidth - 2\tabcolsep) * \real{0.5238}}
  >{\raggedright\arraybackslash}p{(\linewidth - 2\tabcolsep) * \real{0.4762}}@{}}
\toprule\noalign{}
\begin{minipage}[b]{\linewidth}\raggedright
Component
\end{minipage} & \begin{minipage}[b]{\linewidth}\raggedright
Function
\end{minipage} \\
\midrule\noalign{}
\endhead
\bottomrule\noalign{}
\endlastfoot
\textbf{Capacitor} & Connected in parallel with load resistance \\
\textbf{Working Principle} & • Charges during voltage peaks• Discharges
during voltage dips• Acts as charge reservoir \\
\textbf{Waveforms} & • Reduces ripple significantly• Output has slight
discharge slope \\
\end{longtable}
}

\textbf{Advantages:}

\begin{itemize}
\tightlist
\item
  Higher DC output voltage
\item
  Simple and economical
\item
  Good ripple reduction
\end{itemize}

\textbf{Limitations:}

\begin{itemize}
\tightlist
\item
  Poor voltage regulation
\item
  High peak diode currents
\item
  Suitable for low current applications
\end{itemize}

\textbf{2. Choke Input Filter}

{\def\LTcaptype{none} % do not increment counter
\begin{longtable}[]{@{}
  >{\raggedright\arraybackslash}p{(\linewidth - 2\tabcolsep) * \real{0.5238}}
  >{\raggedright\arraybackslash}p{(\linewidth - 2\tabcolsep) * \real{0.4762}}@{}}
\toprule\noalign{}
\begin{minipage}[b]{\linewidth}\raggedright
Component
\end{minipage} & \begin{minipage}[b]{\linewidth}\raggedright
Function
\end{minipage} \\
\midrule\noalign{}
\endhead
\bottomrule\noalign{}
\endlastfoot
\textbf{Inductor (Choke)} & Connected in series with load \\
\textbf{Capacitor} & Connected in parallel with load \\
\textbf{Working Principle} & • Inductor opposes current changes•
Capacitor smooths remaining ripples \\
\textbf{Waveforms} & • More constant current• Lower but more stable
output voltage \\
\end{longtable}
}

\textbf{Advantages:}

\begin{itemize}
\tightlist
\item
  Better voltage regulation
\item
  Lower peak diode currents
\item
  Suitable for high current applications
\end{itemize}

\textbf{Limitations:}

\begin{itemize}
\tightlist
\item
  Lower DC output voltage
\item
  More expensive
\item
  Bulkier than capacitor filter
\end{itemize}

\begin{center}
\textbf{Mermaid Diagram (Code)}
\begin{verbatim}
{Shaded}
{Highlighting}[]
graph LR
    A[Rectifier Output] {-{-}{} B[Capacitor/Choke Input]}
    B {-{-}{} C[Filtered Output]}
    style A fill:\#f96,stroke:\#333
    style B fill:\#69f,stroke:\#333
    style C fill:\#6f9,stroke:\#333
{Highlighting}
{Shaded}
\end{verbatim}
\end{center}

\textbf{Waveform Comparison:}

\begin{verbatim}
Rectifier output:     \_\_      \_\_      \_\_
                     /  {    /      /  }
                    /    {  /      /    }
           0 \_\_\_\_\_\_/      {/      /      \_\_\_\_}

Capacitor filter:    \_\_\_     \_\_\_     \_\_\_
                     {              }
                      {              }
           0 \_\_\_\_\_\_\_\_\_\_{\_\_\_\_\_\_\_\_\_\_\_\_\_\_\_\_\_\_}

Choke filter:         \_\_\_\_\_\_\_\_\_\_ \_\_\_\_\_\_\_\_\_\_
                     /          /
                    /          /
           0 \_\_\_\_\_\_/          /\_\_\_\_\_\_\_\_\_\_\_\_
\end{verbatim}

\end{solutionbox}
\begin{mnemonicbox}
``VOICE'' - Voltage Output Is Constant with Either
filter, but choke gives better regulation

\end{mnemonicbox}
\subsection*{Question 3(a) OR [3
marks]}\label{q3a}

\textbf{State the function and importance of Zener diode.}

\begin{solutionbox}

\textbf{Function and Importance of Zener Diode}

{\def\LTcaptype{none} % do not increment counter
\begin{longtable}[]{@{}
  >{\raggedright\arraybackslash}p{(\linewidth - 2\tabcolsep) * \real{0.4348}}
  >{\raggedright\arraybackslash}p{(\linewidth - 2\tabcolsep) * \real{0.5652}}@{}}
\toprule\noalign{}
\begin{minipage}[b]{\linewidth}\raggedright
Function
\end{minipage} & \begin{minipage}[b]{\linewidth}\raggedright
Description
\end{minipage} \\
\midrule\noalign{}
\endhead
\bottomrule\noalign{}
\endlastfoot
\textbf{Voltage Regulation} & Maintains constant output voltage despite
input variations \\
\textbf{Voltage Reference} & Provides precise reference voltage in
circuits \\
\textbf{Voltage Protection} & Prevents voltage spikes from damaging
circuits \\
\textbf{Voltage Limiting} & Clips signal voltages to predetermined
levels \\
\textbf{Waveform Clipping} & Shapes waveforms by limiting voltage
levels \\
\end{longtable}
}

\end{solutionbox}
\begin{mnemonicbox}
``VPRVW'' - Voltage Protection, Regulation, and
Voltage Waveform control

\end{mnemonicbox}
\subsection*{Question 3(b) OR [4
marks]}\label{q3b}

\textbf{Describe Light emitting diode (LED) with its characteristic.}

\begin{solutionbox}

\textbf{Light Emitting Diode (LED) Characteristics}

{\def\LTcaptype{none} % do not increment counter
\begin{longtable}[]{@{}
  >{\raggedright\arraybackslash}p{(\linewidth - 2\tabcolsep) * \real{0.5517}}
  >{\raggedright\arraybackslash}p{(\linewidth - 2\tabcolsep) * \real{0.4483}}@{}}
\toprule\noalign{}
\begin{minipage}[b]{\linewidth}\raggedright
Characteristic
\end{minipage} & \begin{minipage}[b]{\linewidth}\raggedright
Description
\end{minipage} \\
\midrule\noalign{}
\endhead
\bottomrule\noalign{}
\endlastfoot
\textbf{Construction} & • P-N junction made from direct bandgap
semiconductors• Common materials: GaAs, GaP, AlGaInP, InGaN \\
\textbf{Working Principle} & • Electroluminescence: electrons recombine
with holes• Energy released as photons (light) \\
\textbf{Forward Voltage} & • Red: 1.8-2.1V• Green: 2.0-3.0V• Blue/White:
3.0-3.5V \\
\textbf{Colors Available} & • Depends on semiconductor material• Red,
green, yellow, blue, white, IR, UV \\
\textbf{I-V Characteristics} & • Conducts when forward biased above
threshold• Requires current-limiting resistor• Damaged by reverse bias
above 5V \\
\textbf{Applications} & • Indicators, displays, lighting,
optocouplers \\
\end{longtable}
}

\begin{center}
\textbf{Mermaid Diagram (Code)}
\begin{verbatim}
{Shaded}
{Highlighting}[]
graph LR
    A[Voltage Applied] {-{-}{}|Forward Bias| B[Electron{-}Hole Recombination]}
    B {-{-}{} C[Energy Released]}
    C {-{-}{} D[Light Emission]}
    style A fill:\#f96,stroke:\#333
    style B fill:\#69f,stroke:\#333
    style C fill:\#fc9,stroke:\#333
    style D fill:\#6f9,stroke:\#333
{Highlighting}
{Shaded}
\end{verbatim}
\end{center}

\end{solutionbox}
\begin{mnemonicbox}
``CRAVE'' - Current Regulated And Voltage Emits light

\end{mnemonicbox}
\subsection*{Question 3(c) OR [7
marks]}\label{q3c}

\textbf{Illustrate the working of capacitor input and choke input
filter.}

\begin{solutionbox}

\textbf{Capacitor Input Filter:}

{\def\LTcaptype{none} % do not increment counter
\begin{longtable}[]{@{}
  >{\raggedright\arraybackslash}p{(\linewidth - 2\tabcolsep) * \real{0.5238}}
  >{\raggedright\arraybackslash}p{(\linewidth - 2\tabcolsep) * \real{0.4762}}@{}}
\toprule\noalign{}
\begin{minipage}[b]{\linewidth}\raggedright
Component
\end{minipage} & \begin{minipage}[b]{\linewidth}\raggedright
Function
\end{minipage} \\
\midrule\noalign{}
\endhead
\bottomrule\noalign{}
\endlastfoot
\textbf{Circuit Structure} & Capacitor connected in parallel with
load \\
\textbf{Operation} & • Capacitor charges to peak voltage• Discharges
slowly through load when voltage drops• Acts as reservoir of charge \\
\textbf{Performance} & • Good ripple reduction• Higher output voltage•
Poor regulation under varying loads \\
\end{longtable}
}

\textbf{Circuit Diagram:}

\begin{verbatim}
    +{-{-}{-}{-}{-}{-}||{-}{-}{-}{-}{-}{-}+}
    |      D1       |
AC  |               | Load
In  |               | RL    +
    +{-{-}{-}{-}{-}{-}||{-}{-}{-}{-}{-}{-}+{-}{-}{-}{-}{-}||{-}{-}{-}+}
    |      D2       |     C    |
    +{-{-}{-}{-}{-}{-}{-}{-}{-}{-}{-}{-}{-}{-}{-}+{-}{-}{-}{-}{-}{-}{-}{-}{-}{-}+}
\end{verbatim}

\textbf{Choke Input Filter:}

{\def\LTcaptype{none} % do not increment counter
\begin{longtable}[]{@{}
  >{\raggedright\arraybackslash}p{(\linewidth - 2\tabcolsep) * \real{0.5238}}
  >{\raggedright\arraybackslash}p{(\linewidth - 2\tabcolsep) * \real{0.4762}}@{}}
\toprule\noalign{}
\begin{minipage}[b]{\linewidth}\raggedright
Component
\end{minipage} & \begin{minipage}[b]{\linewidth}\raggedright
Function
\end{minipage} \\
\midrule\noalign{}
\endhead
\bottomrule\noalign{}
\endlastfoot
\textbf{Circuit Structure} & Inductor (choke) in series, capacitor in
parallel \\
\textbf{Operation} & • Inductor opposes change in current• Smooths
current flow• Capacitor further filters voltage ripples \\
\textbf{Performance} & • Better voltage regulation• Lower output
voltage• Good for high-current applications \\
\end{longtable}
}

\textbf{Circuit Diagram:}

\begin{verbatim}
    +{-{-}{-}{-}{-}{-}||{-}{-}{-}{-}{-}{-}+}
    |      D1       |
AC  |               +{-{-}{-}{-}LLLLL{-}{-}{-}{-}+}
In  |                     L       |
    +{-{-}{-}{-}{-}{-}||{-}{-}{-}{-}{-}{-}+          RL +}
    |      D2       |     C       |
    +{-{-}{-}{-}{-}{-}{-}{-}{-}{-}{-}{-}{-}{-}{-}+{-}{-}{-}{-}||{-}{-}{-}{-}{-}{-}{-}+}
\end{verbatim}

\textbf{Comparison:}

{\def\LTcaptype{none} % do not increment counter
\begin{longtable}[]{@{}lll@{}}
\toprule\noalign{}
Parameter & Capacitor Input & Choke Input \\
\midrule\noalign{}
\endhead
\bottomrule\noalign{}
\endlastfoot
\textbf{Output Voltage} & Higher (\approx1.4Vm) & Lower (\approx0.9Vm) \\
\textbf{Ripple Factor} & Higher & Lower \\
\textbf{Voltage Regulation} & Poor & Good \\
\textbf{Diode Current} & High peak currents & Lower peak currents \\
\textbf{Cost \& Size} & Lower, smaller & Higher, larger \\
\textbf{Applications} & Low current needs & High current needs \\
\end{longtable}
}

\end{solutionbox}
\begin{mnemonicbox}
``CHEER'' - Capacitor Holds Energy, inductor Ensures
Regulated current

\end{mnemonicbox}
\subsection*{Question 4(a) [3 marks]}\label{q4a}

\textbf{Discuss characteristics of PN junction diode.}

\begin{solutionbox}

\textbf{Characteristics of PN Junction Diode}

{\def\LTcaptype{none} % do not increment counter
\begin{longtable}[]{@{}
  >{\raggedright\arraybackslash}p{(\linewidth - 2\tabcolsep) * \real{0.5517}}
  >{\raggedright\arraybackslash}p{(\linewidth - 2\tabcolsep) * \real{0.4483}}@{}}
\toprule\noalign{}
\begin{minipage}[b]{\linewidth}\raggedright
Characteristic
\end{minipage} & \begin{minipage}[b]{\linewidth}\raggedright
Description
\end{minipage} \\
\midrule\noalign{}
\endhead
\bottomrule\noalign{}
\endlastfoot
\textbf{Forward Bias} & • Conducts when voltage \textgreater{} threshold
(0.7V for Si, 0.3V for Ge)• Current increases exponentially with
voltage• Low resistance state \\
\textbf{Reverse Bias} & • Blocks current flow• Small leakage current
(μA)• High resistance state \\
\textbf{Breakdown} & • Occurs at specific reverse voltage• Current
increases rapidly• Can damage diode if current not limited \\
\textbf{Temperature Effects} & • Forward voltage decreases with
temperature• Reverse leakage current doubles every 10^\circC \\
\textbf{Capacitance} & • Junction capacitance varies with applied
voltage• Higher in forward bias \\
\end{longtable}
}

\end{solutionbox}
\begin{mnemonicbox}
``FRBCT'' - Forward conducts, Reverse blocks,
Breakdown destroys, Capacitance changes, Temperature affects

\end{mnemonicbox}
\subsection*{Question 4(b) [4 marks]}\label{q4b}

\textbf{Compare between P-N junction diode and Zener diode.}

\begin{solutionbox}


{\def\LTcaptype{none} % do not increment counter
\vspace{-5pt}
\captionof{table}{P-N Junction Diode vs.~Zener Diode}
\vspace{-10pt}
\begin{longtable}[]{@{}
  >{\raggedright\arraybackslash}p{(\linewidth - 4\tabcolsep) * \real{0.2558}}
  >{\raggedright\arraybackslash}p{(\linewidth - 4\tabcolsep) * \real{0.4419}}
  >{\raggedright\arraybackslash}p{(\linewidth - 4\tabcolsep) * \real{0.3023}}@{}}
\toprule\noalign{}
\begin{minipage}[b]{\linewidth}\raggedright
Parameter
\end{minipage} & \begin{minipage}[b]{\linewidth}\raggedright
P-N Junction Diode
\end{minipage} & \begin{minipage}[b]{\linewidth}\raggedright
Zener Diode
\end{minipage} \\
\midrule\noalign{}
\endhead
\bottomrule\noalign{}
\endlastfoot
\textbf{Symbol} & ▶〈 & ▶〈▶ \\
\textbf{Forward Operation} & Conducts above 0.7V & Conducts above 0.7V
(similar) \\
\textbf{Reverse Operation} & Blocks current until breakdown & Designed
to operate in controlled breakdown \\
\textbf{Breakdown Voltage} & Higher, not specified precisely & Lower,
precisely specified (2-200V) \\
\textbf{Reverse Breakdown} & Destructive if not limited &
Non-destructive, used for operation \\
\textbf{Applications} & Rectification, switching & Voltage regulation,
protection \\
\textbf{Doping Level} & Normal doping & Heavily doped to control
breakdown \\
\end{longtable}
}

\end{solutionbox}
\begin{mnemonicbox}
``FORBAR'' - Forward Operation is Regular, Breakdown
Application is the Real difference

\end{mnemonicbox}
\subsection*{Question 4(c) [7 marks]}\label{q4c}

\textbf{Illustrate the function of Zener diode as a voltage regulator.}

\begin{solutionbox}

\textbf{Zener Diode as Voltage Regulator}

{\def\LTcaptype{none} % do not increment counter
\begin{longtable}[]{@{}ll@{}}
\toprule\noalign{}
Component & Function \\
\midrule\noalign{}
\endhead
\bottomrule\noalign{}
\endlastfoot
\textbf{Zener Diode} & Maintains constant voltage in breakdown region \\
\textbf{Series Resistor (Rs)} & Limits current and drops excess
voltage \\
\textbf{Load Resistor (RL)} & Represents the circuit being powered \\
\end{longtable}
}

\textbf{Working Principle:}

\begin{enumerate}
\tightlist
\item
  Zener diode is connected in reverse bias
\item
  When input voltage rises above Zener voltage, diode conducts
\item
  Excess voltage is dropped across series resistor
\item
  Output voltage remains constant at Zener voltage
\end{enumerate}

\begin{center}
\textbf{Mermaid Diagram (Code)}
\begin{verbatim}
{Shaded}
{Highlighting}[]
graph LR
    A[Input Voltage] {-{-}{} B[Series Resistor]}
    B {-{-}{} C[Output Voltage]}
    C {-{-}{} D[Load]}
    C {-{-}{} E[Zener Diode]}
    E {-{-}{} F[Ground]}
    style A fill:\#f96,stroke:\#333
    style B fill:\#69f,stroke:\#333
    style C fill:\#6f9,stroke:\#333
    style D fill:\#fc9,stroke:\#333
    style E fill:\#f9f,stroke:\#333
{Highlighting}
{Shaded}
\end{verbatim}
\end{center}

\textbf{Circuit Diagram:}

\begin{verbatim}
     +{-{-}{-}{-}[Rs]{-}{-}{-}{-}{-}+{-}{-}{-}{-}+}
     |             |    |
Vin  |             +   RL   Vout = Vz
     |             |    |
     +{-{-}{-}{-}{-}{-}{-}{-}||{-}{-}+{-}{-}{-}{-}+}
              Zener
\end{verbatim}

\textbf{Regulation Cases:}

{\def\LTcaptype{none} % do not increment counter
\begin{longtable}[]{@{}
  >{\raggedright\arraybackslash}p{(\linewidth - 2\tabcolsep) * \real{0.5238}}
  >{\raggedright\arraybackslash}p{(\linewidth - 2\tabcolsep) * \real{0.4762}}@{}}
\toprule\noalign{}
\begin{minipage}[b]{\linewidth}\raggedright
Condition
\end{minipage} & \begin{minipage}[b]{\linewidth}\raggedright
Response
\end{minipage} \\
\midrule\noalign{}
\endhead
\bottomrule\noalign{}
\endlastfoot
\textbf{Input Voltage Increases} & • More current through Zener• More
voltage dropped across Rs• Output remains at Vz \\
\textbf{Input Voltage Decreases} & • Less current through Zener• Less
voltage dropped across Rs• Output remains at Vz (until minimum operating
voltage) \\
\textbf{Load Current Increases} & • Less current through Zener• Output
voltage stable until minimum Zener current \\
\textbf{Load Current Decreases} & • More current through Zener• Output
voltage remains stable \\
\end{longtable}
}

\textbf{Limitations:}

\begin{itemize}
\tightlist
\item
  Power dissipation in Zener and Rs
\item
  Minimum input voltage requirement (Vin \textgreater{} Vz + Voltage
  drop across Rs)
\item
  Limited current capability
\end{itemize}

\end{solutionbox}
\begin{mnemonicbox}
``VISOR'' - Voltage In Stays Out Regulated

\end{mnemonicbox}
\subsection*{Question 4(a) OR [3
marks]}\label{q4a}

\textbf{Discuss transistor in brief.}

\begin{solutionbox}

\textbf{Transistor Overview}

{\def\LTcaptype{none} % do not increment counter
\begin{longtable}[]{@{}
  >{\raggedright\arraybackslash}p{(\linewidth - 2\tabcolsep) * \real{0.3810}}
  >{\raggedright\arraybackslash}p{(\linewidth - 2\tabcolsep) * \real{0.6190}}@{}}
\toprule\noalign{}
\begin{minipage}[b]{\linewidth}\raggedright
Aspect
\end{minipage} & \begin{minipage}[b]{\linewidth}\raggedright
Description
\end{minipage} \\
\midrule\noalign{}
\endhead
\bottomrule\noalign{}
\endlastfoot
\textbf{Definition} & • Semiconductor device that amplifies/switches
electrical signals• Three-terminal device: emitter, base, collector \\
\textbf{Types} & • Bipolar Junction Transistor (BJT): NPN, PNP• Field
Effect Transistor (FET): JFET, MOSFET \\
\textbf{Working Principle} & • Current/voltage controlled device• Small
base current controls larger collector current (BJT)• Gate voltage
controls channel conductivity (FET) \\
\textbf{Applications} & • Amplification: audio, RF, power• Switching:
digital circuits• Oscillators and signal generation \\
\textbf{Importance} & • Foundation of modern electronics• Enabled
miniaturization of electronic devices \\
\end{longtable}
}

\end{solutionbox}
\begin{mnemonicbox}
``TAWAI'' - Transistors Amplify, Work As switches,
and are Integral to electronics

\end{mnemonicbox}
\subsection*{Question 4(b) OR [4
marks]}\label{q4b}

\textbf{Derive relation between α and β for transistor amplifier.}

\begin{solutionbox}

\textbf{Relation Between α and β}

{\def\LTcaptype{none} % do not increment counter
\begin{longtable}[]{@{}
  >{\raggedright\arraybackslash}p{(\linewidth - 4\tabcolsep) * \real{0.3438}}
  >{\raggedright\arraybackslash}p{(\linewidth - 4\tabcolsep) * \real{0.3750}}
  >{\raggedright\arraybackslash}p{(\linewidth - 4\tabcolsep) * \real{0.2812}}@{}}
\toprule\noalign{}
\begin{minipage}[b]{\linewidth}\raggedright
Parameter
\end{minipage} & \begin{minipage}[b]{\linewidth}\raggedright
Definition
\end{minipage} & \begin{minipage}[b]{\linewidth}\raggedright
Formula
\end{minipage} \\
\midrule\noalign{}
\endhead
\bottomrule\noalign{}
\endlastfoot
\textbf{α (Alpha)} & • Common Base (CB) current gain• Ratio of collector
current to emitter current & α = IC/IE \\
\textbf{β (Beta)} & • Common Emitter (CE) current gain• Ratio of
collector current to base current & β = IC/IB \\
\end{longtable}
}

\textbf{Derivation Steps:}

\begin{enumerate}
\item
  We know that emitter current is the sum of base and collector
  currents: IE = IB + IC
\item
  Alpha definition: α = IC/IE
\item
  Beta definition: β = IC/IB
\item
  From step 1, we can write: IB = IE - IC
\item
  Substituting into beta definition: β = IC/(IE - IC)
\item
Using alpha definition, IC = α \times IE:

β = (α \times IE)/(IE - α \times IE)

\item
  Simplifying: β = α/(1 - α)
\item
  Conversely, we can also express α in terms of β: α = β/(β + 1)
\end{enumerate}

\textbf{Relationship Table:}

{\def\LTcaptype{none} % do not increment counter
\begin{longtable}[]{@{}ll@{}}
\toprule\noalign{}
α (Alpha) & β (Beta) \\
\midrule\noalign{}
\endhead
\bottomrule\noalign{}
\endlastfoot
0.9 & 9 \\
0.95 & 19 \\
0.98 & 49 \\
0.99 & 99 \\
0.995 & 199 \\
\end{longtable}
}

\end{solutionbox}
\begin{mnemonicbox}
``ABR'' - Alpha and Beta are Related by α = β/(β+1)
or β = α/(1-α)

\end{mnemonicbox}
\subsection*{Question 4(c) OR [7
marks]}\label{q4c}

\textbf{Explain in detail the construction of NPN and PNP transistor.}

\begin{solutionbox}

\textbf{Construction of NPN and PNP Transistors}

{\def\LTcaptype{none} % do not increment counter
\begin{longtable}[]{@{}
  >{\raggedright\arraybackslash}p{(\linewidth - 4\tabcolsep) * \real{0.2558}}
  >{\raggedright\arraybackslash}p{(\linewidth - 4\tabcolsep) * \real{0.3721}}
  >{\raggedright\arraybackslash}p{(\linewidth - 4\tabcolsep) * \real{0.3721}}@{}}
\toprule\noalign{}
\begin{minipage}[b]{\linewidth}\raggedright
Parameter
\end{minipage} & \begin{minipage}[b]{\linewidth}\raggedright
NPN Transistor
\end{minipage} & \begin{minipage}[b]{\linewidth}\raggedright
PNP Transistor
\end{minipage} \\
\midrule\noalign{}
\endhead
\bottomrule\noalign{}
\endlastfoot
\textbf{Structure} & • N-type (Emitter)• P-type (Base)• N-type
(Collector) & • P-type (Emitter)• N-type (Base)• P-type (Collector) \\
\textbf{Symbol} &
\pandocbounded{\includegraphics[keepaspectratio,alt={NPN Symbol}]{Triangle with emitter arrow pointing out}}
&
\pandocbounded{\includegraphics[keepaspectratio,alt={PNP Symbol}]{Triangle with emitter arrow pointing in}} \\
\textbf{Materials} & • Silicon or Germanium• Emitter: Heavily doped
N-type• Base: Lightly doped P-type• Collector: Moderately doped N-type &
• Silicon or Germanium• Emitter: Heavily doped P-type• Base: Lightly
doped N-type• Collector: Moderately doped P-type \\
\textbf{Thickness} & • Base: Very thin (1-10 μm)• Collector: Thickest
region & • Base: Very thin (1-10 μm)• Collector: Thickest region \\
\textbf{Doping Level} & • Emitter: Highest• Base: Lowest• Collector:
Medium & • Emitter: Highest• Base: Lowest• Collector: Medium \\
\end{longtable}
}

\textbf{NPN Transistor Construction:}

\begin{verbatim}
    Emitter (N)   Base (P)   Collector (N)
       |            |            |
       v            v            v
    +{-{-}{-}{-}{-}{-}+     +{-}{-}{-}+     +{-}{-}{-}{-}{-}{-}{-}{-}{-}{-}+}
    |  N+  |     | P |     |    N     |
    +{-{-}{-}{-}{-}{-}+     +{-}{-}{-}+     +{-}{-}{-}{-}{-}{-}{-}{-}{-}{-}+}
       |           |           |
       |           |           |
       E           B           C
\end{verbatim}

\textbf{PNP Transistor Construction:}

\begin{verbatim}
    Emitter (P)   Base (N)   Collector (P)
       |            |            |
       v            v            v
    +{-{-}{-}{-}{-}{-}+     +{-}{-}{-}+     +{-}{-}{-}{-}{-}{-}{-}{-}{-}{-}+}
    |  P+  |     | N |     |    P     |
    +{-{-}{-}{-}{-}{-}+     +{-}{-}{-}+     +{-}{-}{-}{-}{-}{-}{-}{-}{-}{-}+}
       |           |           |
       |           |           |
       E           B           C
\end{verbatim}

\textbf{Manufacturing Process:}

\begin{enumerate}
\tightlist
\item
  Start with semiconductor substrate (N or P type)
\item
  Create layers through epitaxial growth
\item
  Form junctions through diffusion or ion implantation
\item
  Add metal contacts for terminals
\item
  Package in protective case
\end{enumerate}

\begin{center}
\textbf{Mermaid Diagram (Code)}
\begin{verbatim}
{Shaded}
{Highlighting}[]
graph LR
    A[Silicon Wafer] {-{-}{} B[Epitaxial Layer Growth]}
    B {-{-}{} C[Diffusion of Dopants]}
    C {-{-}{} D[Oxide Insulation]}
    D {-{-}{} E[Metallization]}
    E {-{-}{} F[Packaging]}
    style A fill:\#fc9,stroke:\#333
    style B fill:\#69f,stroke:\#333
    style C fill:\#f9f,stroke:\#333
    style D fill:\#cfc,stroke:\#333
    style E fill:\#f96,stroke:\#333
    style F fill:\#9cf,stroke:\#333
{Highlighting}
{Shaded}
\end{verbatim}
\end{center}

\end{solutionbox}
\begin{mnemonicbox}
``ENB-CPM'' - Emitter has N in NPN, Collector is
Proportionally Medium-doped

\end{mnemonicbox}
\subsection*{Question 5(a) [3 marks]}\label{q5a}

\textbf{Explain e-waste in brief.}

\begin{solutionbox}

\textbf{Electronic Waste (E-Waste)}

{\def\LTcaptype{none} % do not increment counter
\begin{longtable}[]{@{}
  >{\raggedright\arraybackslash}p{(\linewidth - 2\tabcolsep) * \real{0.3810}}
  >{\raggedright\arraybackslash}p{(\linewidth - 2\tabcolsep) * \real{0.6190}}@{}}
\toprule\noalign{}
\begin{minipage}[b]{\linewidth}\raggedright
Aspect
\end{minipage} & \begin{minipage}[b]{\linewidth}\raggedright
Description
\end{minipage} \\
\midrule\noalign{}
\endhead
\bottomrule\noalign{}
\endlastfoot
\textbf{Definition} & • Discarded electronic devices and equipment•
Contains both valuable materials and hazardous substances \\
\textbf{Sources} & • Computers, phones, TVs, appliances• Circuit boards,
batteries, displays• Office equipment, medical devices \\
\textbf{Concerns} & • Contains toxic materials (lead, mercury, cadmium)•
Environmental contamination if improperly disposed• Health risks to
humans and wildlife \\
\textbf{Importance} & • Fastest growing waste stream globally• Resource
recovery potential (gold, silver, copper)• Requires specialized
handling \\
\end{longtable}
}

\end{solutionbox}
\begin{mnemonicbox}
``TECH'' - Toxic Electronics Create Hazards when
improperly disposed

\end{mnemonicbox}
\subsection*{Question 5(b) [4 marks]}\label{q5b}

\textbf{Illustrate operation of NPN transistor with figure.}

\begin{solutionbox}

\textbf{NPN Transistor Operation}

\textbf{Symbol and Basic Operation:}

\begin{verbatim}
       Collector (C)
           |
           |
           v
     +{-{-}{-}{-}{-}+{-}{-}{-}{-}{-}+}
     |     |     |
     |    / {    |}
Base |{-{-}{-}|   |{-}{-}{-}| Collector}
(B)  |    { /    |}
     |     |     |
     +{-{-}{-}{-}{-}+{-}{-}{-}{-}{-}+}
           |
           |
           v
        Emitter (E)
\end{verbatim}

\textbf{Basic Operating Principle:}

\begin{itemize}
\tightlist
\item
  Base-Emitter junction is forward biased
\item
  Base-Collector junction is reverse biased
\item
  Small base current controls larger collector current
\end{itemize}

{\def\LTcaptype{none} % do not increment counter
\begin{longtable}[]{@{}
  >{\raggedright\arraybackslash}p{(\linewidth - 4\tabcolsep) * \real{0.3333}}
  >{\raggedright\arraybackslash}p{(\linewidth - 4\tabcolsep) * \real{0.3958}}
  >{\raggedright\arraybackslash}p{(\linewidth - 4\tabcolsep) * \real{0.2708}}@{}}
\toprule\noalign{}
\begin{minipage}[b]{\linewidth}\raggedright
Operating Mode
\end{minipage} & \begin{minipage}[b]{\linewidth}\raggedright
Biasing Conditions
\end{minipage} & \begin{minipage}[b]{\linewidth}\raggedright
Description
\end{minipage} \\
\midrule\noalign{}
\endhead
\bottomrule\noalign{}
\endlastfoot
\textbf{Active Mode} & • B-E: Forward biased• B-C: Reverse biased & •
Normal amplification mode• IC = β \times IB \\
\textbf{Cutoff Mode} & • B-E: Reverse biased• B-C: Reverse biased & •
Transistor OFF• No collector current \\
\textbf{Saturation Mode} & • B-E: Forward biased• B-C: Forward biased &
• Transistor fully ON• Maximum collector current \\
\end{longtable}
}

\begin{center}
\textbf{Mermaid Diagram (Code)}
\begin{verbatim}
{Shaded}
{Highlighting}[]
graph LR
    A[Base Current Injected] {-{-}{} B[Electrons from Emitter Enter Base]}
    B {-{-}{} C[Most Electrons Reach Collector]}
    C {-{-}{} D[Small Change in Base Current Controls Larger Collector Current]}
    style A fill:\#f96,stroke:\#333
    style B fill:\#69f,stroke:\#333
    style C fill:\#f9f,stroke:\#333
    style D fill:\#cfc,stroke:\#333
{Highlighting}
{Shaded}
\end{verbatim}
\end{center}

\textbf{Current Flow in NPN Transistor:}

\begin{itemize}
\tightlist
\item
  Electrons flow from emitter to collector
\item
  Small base current controls larger collector current
\item
  Amplification factor (β) = IC/IB
\end{itemize}

\end{solutionbox}
\begin{mnemonicbox}
``BECAN'' - Base current Enables Collector-to-emitter
current Amplification in NPN

\end{mnemonicbox}
\subsection*{Question 5(c) [7 marks]}\label{q5c}

\textbf{Illustrate common emitter (CE) configuration of Transistor with
input and output characteristics.}

\begin{solutionbox}

\textbf{Common Emitter (CE) Configuration}

{\def\LTcaptype{none} % do not increment counter
\begin{longtable}[]{@{}
  >{\raggedright\arraybackslash}p{(\linewidth - 2\tabcolsep) * \real{0.4583}}
  >{\raggedright\arraybackslash}p{(\linewidth - 2\tabcolsep) * \real{0.5417}}@{}}
\toprule\noalign{}
\begin{minipage}[b]{\linewidth}\raggedright
Component
\end{minipage} & \begin{minipage}[b]{\linewidth}\raggedright
Description
\end{minipage} \\
\midrule\noalign{}
\endhead
\bottomrule\noalign{}
\endlastfoot
\textbf{Circuit Configuration} & • Emitter is common to both input and
output• Input between base and emitter• Output between collector and
emitter \\
\textbf{Input Parameters} & • Base current (IB)• Base-emitter voltage
(VBE) \\
\textbf{Output Parameters} & • Collector current (IC)• Collector-emitter
voltage (VCE) \\
\end{longtable}
}

\textbf{Circuit Diagram:}

\begin{verbatim}
                 +Vcc
                   |
                   |
                  RL
                   |
                   |
    +{-{-}{-}{-}{-}+    +{-}{-}{-}o{-}{-}{-} Vout}
    |     |    |   |
Vin o{-{-}{-}{-}{-}o{-}{-}{-}{-}|B  C}
    |     |    |   |
    RB    |    |   |
    |     |    |E  |
    |     |    |   |
    +{-{-}{-}{-}{-}+{-}{-}{-}{-}+{-}{-}{-}o{-}{-}{-} GND}
                   |
                  RE
                   |
                   +
\end{verbatim}

\textbf{Input Characteristics:}

\begin{itemize}
\tightlist
\item
  Plots IB vs VBE for different VCE values
\item
  Resembles forward-biased diode characteristic
\item
  Threshold voltage \textasciitilde0.7V for silicon transistors
\end{itemize}

\begin{verbatim}
    IB (μA)
      ↑
      |                /
      |               /
      |              /
      |             /
      |            /
      |           /
      |          /
      |         /
    {-{-}|{-}{-}{-}{-}{-}{-}{-}{-}/{-}{-}{-}{-}{-}{-}{-}{-}{-}{-}{-}{-}{-}{-}{-} VBE (V)}
      |     0.7V
\end{verbatim}

\textbf{Output Characteristics:}

\begin{itemize}
\tightlist
\item
  Plots IC vs VCE for different IB values
\item
  Shows three regions: Active, Saturation, Cutoff
\end{itemize}

\begin{verbatim}
    IC (mA)
      ↑
      |                 \_\_\_\_\_\_\_\_ IB = 50μA
      |                /
      |               /\_\_\_\_\_\_\_\_ IB = 40μA
      |              /
      |             /\_\_\_\_\_\_\_\_\_ IB = 30μA
      |            /
      |           /\_\_\_\_\_\_\_\_\_\_ IB = 20μA
      |          /
      |         /\_\_\_\_\_\_\_\_\_\_\_\_ IB = 10μA
      |        /
      |       /
    {-{-}|{-}{-}{-}{-}{-}{-}/{-}{-}{-}{-}{-}{-}{-}{-}{-}{-}{-}{-}{-}{-}{-}{-}{-}{-}{-} VCE (V)}
      |  Saturation│  Active
      |  Region    │  Region
\end{verbatim}

\textbf{Characteristics:}

\begin{itemize}
\tightlist
\item
  Current gain (β) = IC/IB (typically 50-200)
\item
  Input resistance: 1-2 kΩ
\item
  Output resistance: 40-50 kΩ
\item
  Phase shift: 180^\circ between input and output
\end{itemize}

\end{solutionbox}
\begin{mnemonicbox}
``CASIO'' - Common emitter Amplifies Signals with
Inverted Output

\end{mnemonicbox}
\subsection*{Question 5(a) OR [3
marks]}\label{q5a}

\textbf{State types of e-waste.}

\begin{solutionbox}

\textbf{Types of Electronic Waste (E-Waste)}

{\def\LTcaptype{none} % do not increment counter
\begin{longtable}[]{@{}
  >{\raggedright\arraybackslash}p{(\linewidth - 2\tabcolsep) * \real{0.5000}}
  >{\raggedright\arraybackslash}p{(\linewidth - 2\tabcolsep) * \real{0.5000}}@{}}
\toprule\noalign{}
\begin{minipage}[b]{\linewidth}\raggedright
Category
\end{minipage} & \begin{minipage}[b]{\linewidth}\raggedright
Examples
\end{minipage} \\
\midrule\noalign{}
\endhead
\bottomrule\noalign{}
\endlastfoot
\textbf{IT \& Telecommunications} & • Computers, laptops, printers•
Mobile phones, tablets• Servers, networking equipment \\
\textbf{Consumer Electronics} & • TVs, monitors, audio equipment•
DVD/Blu-ray players• Cameras, video recorders \\
\textbf{Home Appliances} & • Refrigerators, washing machines• Microwave
ovens, air conditioners• Small kitchen appliances \\
\textbf{Lighting Equipment} & • Fluorescent lamps, LED lights•
High-intensity discharge lamps \\
\textbf{Electrical \& Electronic Tools} & • Drills, saws, soldering
equipment• Lawn mowers, gardening tools \\
\textbf{Medical Devices} & • Diagnostic equipment• Treatment equipment•
Lab equipment \\
\textbf{Monitoring Instruments} & • Smoke detectors• Thermostats•
Control panels \\
\textbf{Electronic Components} & • Circuit boards• Batteries• Cables and
wires \\
\end{longtable}
}

\end{solutionbox}
\begin{mnemonicbox}
``CLIMATE'' - Computing, Lighting, Industrial,
Medical, Appliances, Telecommunications, Electronic components

\end{mnemonicbox}
\subsection*{Question 5(b) OR [4
marks]}\label{q5b}

\textbf{Illustrate different categories of Electronics waste.}

\begin{solutionbox}

\textbf{Categories of Electronic Waste}

{\def\LTcaptype{none} % do not increment counter
\begin{longtable}[]{@{}
  >{\raggedright\arraybackslash}p{(\linewidth - 4\tabcolsep) * \real{0.3030}}
  >{\raggedright\arraybackslash}p{(\linewidth - 4\tabcolsep) * \real{0.3939}}
  >{\raggedright\arraybackslash}p{(\linewidth - 4\tabcolsep) * \real{0.3030}}@{}}
\toprule\noalign{}
\begin{minipage}[b]{\linewidth}\raggedright
Category
\end{minipage} & \begin{minipage}[b]{\linewidth}\raggedright
Description
\end{minipage} & \begin{minipage}[b]{\linewidth}\raggedright
Examples
\end{minipage} \\
\midrule\noalign{}
\endhead
\bottomrule\noalign{}
\endlastfoot
\textbf{Large Household Appliances} & • Bulky items with high metal
content• Often contain refrigerants & • Refrigerators, freezers• Washing
machines• Air conditioners \\
\textbf{Small Household Appliances} & • Portable household devices•
Mixed material composition & • Vacuum cleaners• Toasters, coffee
machines• Electric fans \\
\textbf{IT \& Telecom Equipment} & • Data processing/communication
devices• High precious metal content & • Computers, laptops• Printers,
copying equipment• Mobile phones, telecom equipment \\
\textbf{Consumer Equipment} & • Entertainment/media devices• Often with
display screens & • TVs, monitors• Audio/video equipment• Musical
instruments \\
\textbf{Lighting Equipment} & • Contains mercury and other metals•
Special handling required & • Fluorescent lamps• High-intensity
discharge lamps• LED lighting \\
\textbf{Electrical \& Electronic Tools} & • Portable or fixed power
tools• High motor content & • Drills, saws• Sewing machines•
Construction equipment \\
\textbf{Toys \& Sports Equipment} & • Electronic games and recreational
items• Mixed plastic and electronic components & • Video game consoles•
Electric trains/racing sets• Exercise equipment with electronics \\
\textbf{Medical Devices} & • Specialized healthcare equipment• Often
contains valuable and hazardous materials & • Diagnostic equipment•
Radiation therapy equipment• Laboratory equipment \\
\end{longtable}
}

\begin{verbatim}
pie
    title "Typical E{-Waste Composition by Category"}
    "IT \& Telecom" : 25
    "Large Appliances" : 29
    "Small Appliances" : 14
    "Consumer Electronics" : 17
    "Lighting" : 5
    "Other Categories" : 10
\end{verbatim}

\end{solutionbox}
\begin{mnemonicbox}
``LIMCEST'' - Large appliances, IT equipment, Medical
devices, Consumer electronics, Electronic tools, Small appliances,
Telecom equipment

\end{mnemonicbox}
\subsection*{Question 5(c) OR [7
marks]}\label{q5c}

\textbf{Explain transistor as a switch in cutoff and saturation region.}

\begin{solutionbox}

\textbf{Transistor as a Switch}

{\def\LTcaptype{none} % do not increment counter
\begin{longtable}[]{@{}
  >{\raggedright\arraybackslash}p{(\linewidth - 6\tabcolsep) * \real{0.1818}}
  >{\raggedright\arraybackslash}p{(\linewidth - 6\tabcolsep) * \real{0.1591}}
  >{\raggedright\arraybackslash}p{(\linewidth - 6\tabcolsep) * \real{0.2727}}
  >{\raggedright\arraybackslash}p{(\linewidth - 6\tabcolsep) * \real{0.3864}}@{}}
\toprule\noalign{}
\begin{minipage}[b]{\linewidth}\raggedright
Region
\end{minipage} & \begin{minipage}[b]{\linewidth}\raggedright
State
\end{minipage} & \begin{minipage}[b]{\linewidth}\raggedright
Conditions
\end{minipage} & \begin{minipage}[b]{\linewidth}\raggedright
Characteristics
\end{minipage} \\
\midrule\noalign{}
\endhead
\bottomrule\noalign{}
\endlastfoot
\textbf{Cutoff Region} & OFF & • VBE \textless{} 0.7V• IB \approx 0 & • IC \approx
0• VCE \approx VCC• High impedance \\
\textbf{Saturation Region} & ON & • VBE \textgreater{} 0.7V• IB
\textgreater{} IC/β & • IC \approx IC(sat)• VCE \approx 0.2V• Low impedance \\
\end{longtable}
}

\textbf{Circuit Diagram:}

\begin{verbatim}
                  +Vcc
                    |
                    |
                    R
                    |
                    |
                    C
           +{-{-}{-}{-}{-}{-}{-}{-}+{-}{-}{-}{-}{-}{-}{-}{-}+}
           |                 |
Input o{-{-}{-}{-}+{-}{-}{-}{-}[RB]{-}{-}{-}{-}+B   |}
           |        |E  |    |
           |        |   |    |
           +{-{-}{-}{-}{-}{-}{-}{-}+{-}{-}{-}+{-}{-}{-}{-}o Output}
                    |
                    |
                   GND
\end{verbatim}

\textbf{Cutoff Operation (OFF State):}

\begin{itemize}
\tightlist
\item
  Input voltage is below 0.7V (typically 0V)
\item
  Base-emitter junction is not forward biased
\item
  No base current flows (IB \approx 0)
\item
  No collector current flows (IC \approx 0)
\item
  Collector-emitter voltage is approximately VCC
\item
  Transistor acts as an open switch
\end{itemize}

\textbf{Saturation Operation (ON State):}

\begin{itemize}
\tightlist
\item
  Input voltage is above 0.7V
\item
  Base-emitter junction is forward biased
\item
  Sufficient base current flows (IB \textgreater{} IC/β)
\item
  Collector current reaches maximum (IC(sat))
\item
  Collector-emitter voltage drops to minimum (VCE(sat) \approx 0.2V)
\item
  Transistor acts as a closed switch
\end{itemize}

\begin{center}
\textbf{Mermaid Diagram (Code)}
\begin{verbatim}
{Shaded}
{Highlighting}[]
graph LR
    A[Input Signal] {-{-}{} B\{Voltage Level?\}}
    B {-{-}{}|V {} 0.7V| C[Cutoff Region{}br /{}Switch OFF]}
    B {-{-}{}|V {} 0.7V| D[Saturation Region{}br /{}Switch ON]}
    C {-{-}{} E[High V\_CE{}br /{}No Current]}
    D {-{-}{} F[Low V\_CE{}br /{}Maximum Current]}
    style A fill:\#f96,stroke:\#333
    style B fill:\#69f,stroke:\#333
    style C fill:\#f9f,stroke:\#333
    style D fill:\#cfc,stroke:\#333
    style E fill:\#9cf,stroke:\#333
    style F fill:\#fc9,stroke:\#333
{Highlighting}
{Shaded}
\end{verbatim}
\end{center}

\textbf{Applications:}

\begin{itemize}
\tightlist
\item
  Digital logic circuits
\item
  Relay and motor drivers
\item
  LED and lamp control
\item
  Power converters
\item
  Signal conditioning
\end{itemize}

\textbf{Key Design Considerations:}

\begin{itemize}
\tightlist
\item
  Base resistor (RB) limits base current
\item
  Collector resistor (RC) limits collector current
\item
  Saturation requires IB \textgreater{} IC/β for reliable switching
\item
  Fast switching requires consideration of charge storage effects
\end{itemize}

\end{solutionbox}
\begin{mnemonicbox}
``COSVL'' - Cutoff means Off State with Vce Large,
saturation means low Vce

\end{mnemonicbox}

\end{document}
