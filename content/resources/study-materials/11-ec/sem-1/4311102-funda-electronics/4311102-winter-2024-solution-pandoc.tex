\documentclass[10pt,a4paper]{article}

% content/resources/templates/preamble.tex
\usepackage[margin=0.6in]{geometry}
\author{Milav Dabgar}
\usepackage{amsmath,amssymb,amsthm}
\usepackage{booktabs}
\usepackage{multirow}
\usepackage{xcolor}
\usepackage{tcolorbox}
\tcbuselibrary{breakable,skins}
\usepackage[colorlinks=true,linkcolor=blue]{hyperref}
\usepackage{titlesec}
\usepackage{enumitem}
\usepackage{tikz}
\usepackage{pgfplots}
\usepackage{circuitikz}
\usepackage[version=4]{mhchem}
\usepackage{longtable}
\usepackage{array}
\usepackage{float}
\usepackage{caption}
\usepackage{listings}

\lstset{
  basicstyle=\small\ttfamily,
  breaklines=true,
  breakatwhitespace=false,
  postbreak=\mbox{\textcolor{red}{$\hookrightarrow$}\space},
  float=false,
  numbers=left,
  numberstyle=\tiny\color{gray},
  numbersep=10pt,
  xleftmargin=2em,
  keywordstyle=\color{blue},
  commentstyle=\color{green!60!black},
  stringstyle=\color{purple},
  backgroundcolor=\color{gray!5},
  showstringspaces=false,
  tabsize=2,
  captionpos=b,
  keepspaces=true,
  columns=flexible
}

\pgfplotsset{compat=1.18}
\usetikzlibrary{shapes,arrows,positioning,calc,patterns,decorations.pathmorphing,decorations.markings,arrows.meta}

% Color scheme
\definecolor{headcolor}{RGB}{0,102,204}
\definecolor{keycolor}{RGB}{220,20,60}
\definecolor{solutioncolor}{RGB}{34,139,34}
\definecolor{mnemoniccolor}{RGB}{148,0,211}
\definecolor{codecolor}{RGB}{0,0,100}

% Spacing
\setlength{\parskip}{3pt}
\setlist[itemize]{nosep}
\setlist[enumerate]{nosep}

% Title formatting
\titleformat{\section}{\Large\bfseries\color{headcolor}}{\thesection}{1em}{}
\titleformat{\subsection}{\large\bfseries\color{headcolor}}{\thesubsection}{1em}{}

% Pandoc tightlist compatibility
\providecommand{\tightlist}{%
  \setlength{\itemsep}{0pt}\setlength{\parskip}{0pt}}

% Pandoc longtable compatibility
\newcounter{none}
\def\thenone{}


% content/resources/templates/english-boxes.tex
% This file is currently empty - it exists to maintain consistency with the import structure.
% Add custom environments here if needed in the future.


\begin{document}

\begin{center}
{\Huge\bfseries\color{headcolor} Subject Name Solutions}\\[5pt]
{\LARGE 4311102 -- Winter 2024}\\[3pt]
{\large Semester 1 Study Material}\\[3pt]
{\normalsize\textit{Detailed Solutions and Explanations}}
\end{center}

\vspace{10pt}

\subsection*{Question 1(a) [3 marks]}\label{q1a}

\textbf{Give the difference between Passive components and Active
components}

\begin{solutionbox}

{\def\LTcaptype{none} % do not increment counter
\begin{longtable}[]{@{}
  >{\raggedright\arraybackslash}p{(\linewidth - 2\tabcolsep) * \real{0.5000}}
  >{\raggedright\arraybackslash}p{(\linewidth - 2\tabcolsep) * \real{0.5000}}@{}}
\toprule\noalign{}
\begin{minipage}[b]{\linewidth}\raggedright
\textbf{Passive Components}
\end{minipage} & \begin{minipage}[b]{\linewidth}\raggedright
\textbf{Active Components}
\end{minipage} \\
\midrule\noalign{}
\endhead
\bottomrule\noalign{}
\endlastfoot
Do not require external power source & Require external power source to
operate \\
Cannot amplify or process signals & Can amplify, switch or process
signals \\
Examples: Resistors, Capacitors, Inductors & Examples: Transistors,
Diodes, ICs \\
Cannot control current flow by another signal & Can control current flow
using another signal \\
Store or dissipate energy & Generate energy or provide gain \\
\end{longtable}
}

\end{solutionbox}
\begin{mnemonicbox}
``PAPER-A'' - Passive Are Power-free,
Energy-storing/Resistive; Active Are Amplifying

\end{mnemonicbox}
\subsection*{Question 1(b) [4 marks]}\label{q1b}

\textbf{Explain Working of Light dependent resistor with neat diagram.}

\begin{solutionbox}

\begin{center}
\textbf{Mermaid Diagram (Code)}
\begin{verbatim}
{Shaded}
{Highlighting}[]
graph LR
    A[Light] {-{-}{} B[LDR]}
    B {-{-}{} C[Change in Resistance]}
    style A fill:\#lightblue
    style B fill:\#lightgreen
    style C fill:\#lightpink
{Highlighting}
{Shaded}
\end{verbatim}
\end{center}

\textbf{Working of LDR:}

\begin{itemize}
\tightlist
\item
  \textbf{Construction}: LDR consists of a semiconductor material
  (typically cadmium sulfide) with high resistance in darkness
\item
  \textbf{Photoconductivity}: When light falls on the surface, photons
  transfer energy to electrons, creating free electron-hole pairs
\item
  \textbf{Resistance variation}: Resistance decreases dramatically as
  light intensity increases - from megaohms in darkness to few hundred
  ohms in bright light
\item
  \textbf{Applications}: Used in light sensing circuits, automatic
  street lights, camera exposure control
\end{itemize}

\end{solutionbox}
\begin{mnemonicbox}
``MILD'' - More Illumination, Less resistance in
Devices

\end{mnemonicbox}
\subsection*{Question 1(c) [7 marks]}\label{q1c}

\textbf{Define Intrinsic and Extrinsic Semiconductor. Explain P type and
N type semiconductors in detail.}

\begin{solutionbox}

{\def\LTcaptype{none} % do not increment counter
\begin{longtable}[]{@{}
  >{\raggedright\arraybackslash}p{(\linewidth - 2\tabcolsep) * \real{0.5000}}
  >{\raggedright\arraybackslash}p{(\linewidth - 2\tabcolsep) * \real{0.5000}}@{}}
\toprule\noalign{}
\begin{minipage}[b]{\linewidth}\raggedright
\textbf{Semiconductor Type}
\end{minipage} & \begin{minipage}[b]{\linewidth}\raggedright
\textbf{Description}
\end{minipage} \\
\midrule\noalign{}
\endhead
\bottomrule\noalign{}
\endlastfoot
\textbf{Intrinsic} & Pure semiconductor material with no impurities
added \\
\textbf{Extrinsic} & Semiconductor with controlled impurities added
through doping \\
\end{longtable}
}

\textbf{P-type Semiconductor:}

\begin{itemize}
\tightlist
\item
  \textbf{Doping}: Created by adding trivalent impurities (boron,
  gallium, indium) to pure silicon
\item
  \textbf{Hole creation}: Each impurity atom creates a hole by accepting
  valence electrons
\item
  \textbf{Majority carriers}: Holes are majority carriers
\item
  \textbf{Minority carriers}: Electrons are minority carriers
\item
  \textbf{Electrical properties}: Positive charge carriers dominate
  conduction
\end{itemize}

\textbf{N-type Semiconductor:}

\begin{itemize}
\tightlist
\item
  \textbf{Doping}: Created by adding pentavalent impurities (phosphorus,
  arsenic, antimony) to pure silicon
\item
  \textbf{Electron creation}: Each impurity atom donates an extra
  electron
\item
  \textbf{Majority carriers}: Electrons are majority carriers
\item
  \textbf{Minority carriers}: Holes are minority carriers
\item
  \textbf{Electrical properties}: Negative charge carriers dominate
  conduction
\end{itemize}

\textbf{Diagram:}

\begin{verbatim}
+{-{-}{-}{-}{-}{-}{-}{-}{-}{-}{-}{-}{-}{-}{-}{-}+   +{-}{-}{-}{-}{-}{-}{-}{-}{-}{-}{-}{-}{-}{-}{-}{-}+}
| N{-type         |   | P{-}type         |}
|                |   |                |
| Si Si Si Si Si |   | Si Si Si Si Si |
|                |   |                |
| Si Si P  Si Si |   | Si Si B  Si Si |
|      |         |   |      |         |
| Si Si|Si Si Si |   | Si Si|Si Si Si |
|      v         |   |      v         |
| Si Si e{- Si Si |   | Si Si h+ Si Si |}
|                |   |                |
| Si Si Si Si Si |   | Si Si Si Si Si |
+{-{-}{-}{-}{-}{-}{-}{-}{-}{-}{-}{-}{-}{-}{-}{-}+   +{-}{-}{-}{-}{-}{-}{-}{-}{-}{-}{-}{-}{-}{-}{-}{-}+}
  Extra electron       Extra hole
\end{verbatim}

\end{solutionbox}
\begin{mnemonicbox}
``PINE'' - Positive Impurities make N-type Electrons,
Pentavalent donors

\end{mnemonicbox}
\subsection*{Question 1(c) OR [7
marks]}\label{q1c}

\textbf{What is filter circuit? Give type and necessity of Filter and
Explain ``PI'' Filter circuit in brief.}

\begin{solutionbox}

\textbf{Filter Circuit}: Electronic circuit that removes unwanted
frequency components from a signal, allowing desired frequencies to pass
through.

\textbf{Necessity of Filters}:

\begin{itemize}
\tightlist
\item
  \textbf{Ripple reduction}: Reduces AC ripple from rectifier output
\item
  \textbf{Clean DC}: Provides smoother DC output voltage
\item
  \textbf{Component protection}: Protects downstream components from
  voltage fluctuations
\item
  \textbf{Efficiency}: Improves overall power supply efficiency
\end{itemize}

\textbf{Types of Filters}:

{\def\LTcaptype{none} % do not increment counter
\begin{longtable}[]{@{}lll@{}}
\toprule\noalign{}
\textbf{Filter Type} & \textbf{Components} & \textbf{Application} \\
\midrule\noalign{}
\endhead
\bottomrule\noalign{}
\endlastfoot
Shunt Capacitor & Single capacitor in parallel & Basic filtering \\
L-Type & Inductor and capacitor & Better filtering \\
π (Pi) Filter & Two capacitors and one inductor & Superior filtering \\
RC Filter & Resistor and capacitor & Low-power applications \\
\end{longtable}
}

\textbf{Pi (π) Filter}:

\begin{center}
\textbf{Mermaid Diagram (Code)}
\begin{verbatim}
{Shaded}
{Highlighting}[]
graph LR
    A[Input] {-{-}{} B[Capacitor C1]}
    B {-{-}{} C[Inductor L]}
    C {-{-}{} D[Capacitor C2]}
    D {-{-}{} E[Output]}
    style A fill:\#lightblue
    style B fill:\#lightgreen
    style C fill:\#lightpink
    style D fill:\#lightgreen
    style E fill:\#lightblue
{Highlighting}
{Shaded}
\end{verbatim}
\end{center}

\begin{itemize}
\tightlist
\item
  \textbf{Working}: First capacitor (C1) reduces initial ripple,
  inductor (L) blocks AC components, second capacitor (C2) filters
  remaining ripples
\item
  \textbf{Advantage}: Provides superior filtering with ripple factor
  typically below 0.5\%
\item
  \textbf{Applications}: Used in high-current power supplies where clean
  DC is critical
\end{itemize}

\end{solutionbox}
\begin{mnemonicbox}
``PIRO'' - Pi filters Input Ripples Out effectively

\end{mnemonicbox}
\subsection*{Question 2(a) [3 marks]}\label{q2a}

\textbf{Write down different types of capacitors and explain any two.}

\begin{solutionbox}

\textbf{Types of Capacitors}:

\begin{itemize}
\tightlist
\item
  Ceramic capacitors
\item
  Electrolytic capacitors
\item
  Tantalum capacitors
\item
  Film capacitors
\item
  Mica capacitors
\item
  Variable capacitors
\end{itemize}

\textbf{Ceramic Capacitors}:

\begin{itemize}
\tightlist
\item
  \textbf{Construction}: Made from ceramic material as dielectric
  between metal plates
\item
  \textbf{Capacity}: 1pF to 1μF
\item
  \textbf{Advantages}: Low cost, high stability, non-polarized
\item
  \textbf{Applications}: High-frequency filtering, coupling/decoupling
\end{itemize}

\textbf{Electrolytic Capacitors}:

\begin{itemize}
\tightlist
\item
  \textbf{Construction}: Aluminum foil with oxide layer as dielectric
\item
  \textbf{Capacity}: 1μF to 10,000μF
\item
  \textbf{Characteristics}: Polarized, higher leakage current
\item
  \textbf{Applications}: Power supply filtering, audio coupling
\end{itemize}

\end{solutionbox}
\begin{mnemonicbox}
``CAPEX'' - Ceramics Are Precise, Electrolytics
Expand capacity

\end{mnemonicbox}
\subsection*{Question 2(b) [4 marks]}\label{q2b}

\textbf{Explain air core and toroidal inductor.}

\begin{solutionbox}

\textbf{Air Core Inductor:}

\begin{verbatim}
   +{-{-}{-}{-}{-}{-}{-}{-}{-}{-}+}
   |    Air   |
   |          |
 +{-|{-}{-}{-}{-}{-}{-}{-}{-}{-}{-}|{-}{-}+}
 | |          |  |
 | |          |  |
 | |          |  |
 | |          |  |
 | |          |  |
 +{-|{-}{-}{-}{-}{-}{-}{-}{-}{-}{-}|{-}{-}+}
   |          |
   +{-{-}{-}{-}{-}{-}{-}{-}{-}{-}+}
   Wire windings
\end{verbatim}

\begin{itemize}
\tightlist
\item
  \textbf{Construction}: Wire coiled around non-magnetic material
  (plastic, air)
\item
  \textbf{Properties}: Lower inductance, no magnetic core saturation
\item
  \textbf{Applications}: High-frequency circuits, RF applications
\item
  \textbf{Advantages}: No core losses, linear operation, no saturation
\end{itemize}

\textbf{Toroidal Inductor:}

\begin{verbatim}
      +{-{-}{-}{-}{-}{-}{-}+}
     /         {}
    /           {}
   /     Air     {}
  |       +       |
  |      / {      |}
  |     /   {     |}
  |    +{-{-}{-}{-}{-}+    |}
   {   |     |   /}
    {  |     |  /}
     { |     | /}
      ++{-{-}{-}{-}{-}++}
     Wire windings
      around core
\end{verbatim}

\begin{itemize}
\tightlist
\item
  \textbf{Construction}: Wire wound around a ring-shaped magnetic core
\item
  \textbf{Properties}: Higher inductance, self-shielding magnetic field
\item
  \textbf{Applications}: Power supplies, filters, transformers
\item
  \textbf{Advantages}: Low electromagnetic interference, efficient flux
  containment
\end{itemize}

\end{solutionbox}
\begin{mnemonicbox}
``TACO'' - Toroids Are Contained, Omnidirectional
field reduction

\end{mnemonicbox}
\subsection*{Question 2(c) [7 marks]}\label{q2c}

\textbf{Explain Half wave rectifier and Compare different rectifier
circuits.}

\begin{solutionbox}

\textbf{Half Wave Rectifier:}

\begin{center}
\textbf{Mermaid Diagram (Code)}
\begin{verbatim}
{Shaded}
{Highlighting}[]
graph LR
    A[AC Input] {-{-}{} B[Transformer]}
    B {-{-}{} C[Diode]}
    C {-{-}{} D[Load]}
    C {-{-}{} E[Ground]}
    style A fill:\#lightblue
    style B fill:\#lightpink
    style C fill:\#lightyellow
    style D fill:\#lightgreen
    style E fill:\#lightgray
{Highlighting}
{Shaded}
\end{verbatim}
\end{center}

\textbf{Working Principle:}

\begin{itemize}
\tightlist
\item
  During positive half-cycle: Diode conducts, current flows through load
\item
  During negative half-cycle: Diode blocks, no current flows
\item
  Output contains only positive half-cycles of input waveform
\end{itemize}

\textbf{Comparison of Rectifiers:}

{\def\LTcaptype{none} % do not increment counter
\begin{longtable}[]{@{}
  >{\raggedright\arraybackslash}p{(\linewidth - 6\tabcolsep) * \real{0.2500}}
  >{\raggedright\arraybackslash}p{(\linewidth - 6\tabcolsep) * \real{0.2500}}
  >{\raggedright\arraybackslash}p{(\linewidth - 6\tabcolsep) * \real{0.2500}}
  >{\raggedright\arraybackslash}p{(\linewidth - 6\tabcolsep) * \real{0.2500}}@{}}
\toprule\noalign{}
\begin{minipage}[b]{\linewidth}\raggedright
\textbf{Parameter}
\end{minipage} & \begin{minipage}[b]{\linewidth}\raggedright
\textbf{Half Wave}
\end{minipage} & \begin{minipage}[b]{\linewidth}\raggedright
\textbf{Full Wave (Center-Tap)}
\end{minipage} & \begin{minipage}[b]{\linewidth}\raggedright
\textbf{Bridge Rectifier}
\end{minipage} \\
\midrule\noalign{}
\endhead
\bottomrule\noalign{}
\endlastfoot
Diodes required & 1 & 2 & 4 \\
Output frequency & f_{1} = fin & f_{2} = 2\timesfin & f_{2} = 2\timesfin \\
Ripple factor & 1.21 & 0.48 & 0.48 \\
Efficiency & 40.6\% & 81.2\% & 81.2\% \\
PIV & 2Vm & 2Vm & Vm \\
TUF & 0.287 & 0.693 & 0.812 \\
DC output & Vm/π & 2Vm/π & 2Vm/π \\
\end{longtable}
}

\end{solutionbox}
\begin{mnemonicbox}
``BRIEF'' - Bridge Rectifiers Improve Efficiency
Fundamentally

\end{mnemonicbox}
\subsection*{Question 2(a) OR [3
marks]}\label{q2a}

\textbf{Write down different capacitor specifications and explain any
two in detail.}

\begin{solutionbox}

\textbf{Capacitor Specifications:}

\begin{itemize}
\tightlist
\item
  Capacitance value
\item
  Voltage rating
\item
  Tolerance
\item
  Temperature coefficient
\item
  ESR (Equivalent Series Resistance)
\item
  Leakage current
\item
  Dielectric type
\end{itemize}

\textbf{Capacitance Value:}

\begin{itemize}
\tightlist
\item
  \textbf{Definition}: Amount of electric charge stored per volt
\item
  \textbf{Units}: Measured in farads (F), typically microfarads (μF),
  nanofarads (nF), or picofarads (pF)
\item
  \textbf{Importance}: Determines application suitability for coupling,
  filtering, timing
\item
  \textbf{Marking}: Directly printed or color-coded on component
\end{itemize}

\textbf{Voltage Rating:}

\begin{itemize}
\tightlist
\item
  \textbf{Definition}: Maximum voltage that can be applied without
  breakdown
\item
  \textbf{Specification}: Working voltage (WVDC) and surge voltage
\item
  \textbf{Importance}: Exceeding rating causes dielectric breakdown and
  failure
\item
  \textbf{Safety factor}: Typically use capacitors rated 50\% higher
  than circuit voltage
\end{itemize}

\end{solutionbox}
\begin{mnemonicbox}
``CAVERN'' - Capacitance And Voltage Ensure Reliable
Network

\end{mnemonicbox}
\subsection*{Question 2(b) OR [4
marks]}\label{q2b}

\textbf{Explain classification of Resistor based on materials.}

\begin{solutionbox}

{\def\LTcaptype{none} % do not increment counter
\begin{longtable}[]{@{}
  >{\raggedright\arraybackslash}p{(\linewidth - 6\tabcolsep) * \real{0.2500}}
  >{\raggedright\arraybackslash}p{(\linewidth - 6\tabcolsep) * \real{0.2500}}
  >{\raggedright\arraybackslash}p{(\linewidth - 6\tabcolsep) * \real{0.2500}}
  >{\raggedright\arraybackslash}p{(\linewidth - 6\tabcolsep) * \real{0.2500}}@{}}
\toprule\noalign{}
\begin{minipage}[b]{\linewidth}\raggedright
\textbf{Resistor Type}
\end{minipage} & \begin{minipage}[b]{\linewidth}\raggedright
\textbf{Material}
\end{minipage} & \begin{minipage}[b]{\linewidth}\raggedright
\textbf{Properties}
\end{minipage} & \begin{minipage}[b]{\linewidth}\raggedright
\textbf{Applications}
\end{minipage} \\
\midrule\noalign{}
\endhead
\bottomrule\noalign{}
\endlastfoot
\textbf{Carbon Composition} & Carbon particles + Ceramic binder & High
temperature coefficient, noisy & General purpose, surge protection \\
\textbf{Carbon Film} & Carbon film on ceramic & Better stability than
carbon composition & General purpose circuits \\
\textbf{Metal Film} & Nickel chromium film on ceramic & Low noise,
stable, precise & Audio circuits, instrumentation \\
\textbf{Wire Wound} & Resistance wire around ceramic & High power, low
temperature coefficient & Power supplies, high current applications \\
\textbf{Metal Oxide} & Metal oxide film on ceramic & Stable, high
temperature tolerance & High stability applications, power supplies \\
\end{longtable}
}

\textbf{Characteristics of Carbon Film Resistors:}

\begin{itemize}
\tightlist
\item
  Temperature coefficient: -250 to 500 ppm/^\circC
\item
  Tolerance: 5\% to 10\%
\item
  Noise: Moderate to low
\end{itemize}

\textbf{Characteristics of Metal Film Resistors:}

\begin{itemize}
\tightlist
\item
  Temperature coefficient: 50 to 100 ppm/^\circC
\item
  Tolerance: 0.1\% to 2\%
\item
  Noise: Very low
\end{itemize}

\end{solutionbox}
\begin{mnemonicbox}
``COMFORT'' - Carbon Offers Moderate Films, Others
Resist Temperature better

\end{mnemonicbox}
\subsection*{Question 2(c) OR [7
marks]}\label{q2c}

\textbf{Explain full wave bridge and center tapped rectifier with
diagram and waveform.}

\begin{solutionbox}

\textbf{Full Wave Bridge Rectifier:}

\begin{center}
\textbf{Mermaid Diagram (Code)}
\begin{verbatim}
{Shaded}
{Highlighting}[]
graph LR
    A[AC Input] {-{-}{} B[Transformer]}
    B {-{-}{} C[Bridge{}br /{}Rectifier]}
    C {-{-}{} D[D1]}
    C {-{-}{} E[D2]}
    C {-{-}{} F[D3]}
    C {-{-}{} G[D4]}
    D \& E \& F \& G {-{-}{} H[Load]}
    H {-{-}{} I[Ground]}
    style A fill:\#lightblue
    style B fill:\#lightpink
    style C fill:\#lightyellow
    style H fill:\#lightgreen
    style I fill:\#lightgray
{Highlighting}
{Shaded}
\end{verbatim}
\end{center}

\textbf{Working:}

\begin{itemize}
\tightlist
\item
  \textbf{Positive half-cycle}: D1 and D3 conduct, current flows through
  load
\item
  \textbf{Negative half-cycle}: D2 and D4 conduct, current still flows
  through load in same direction
\item
  \textbf{Output}: Both half-cycles of input converted to positive
  output
\end{itemize}

\textbf{Center Tapped Full Wave Rectifier:}

\begin{center}
\textbf{Mermaid Diagram (Code)}
\begin{verbatim}
{Shaded}
{Highlighting}[]
graph LR
    A[AC Input] {-{-}{} B[Center{-}Tapped{}br /{}Transformer]}
    B {-{-}{}|Upper Half| C[D1]}
    B {-{-}{}|Lower Half| D[D2]}
    C \& D {-{-}{} E[Load]}
    E {-{-}{} F[Ground]}
    F {-{-}{} B}
    style A fill:\#lightblue
    style B fill:\#lightpink
    style C fill:\#lightyellow
    style D fill:\#lightyellow
    style E fill:\#lightgreen
    style F fill:\#lightgray
{Highlighting}
{Shaded}
\end{verbatim}
\end{center}

\textbf{Working:}

\begin{itemize}
\tightlist
\item
  \textbf{Positive half-cycle}: D1 conducts, D2 blocks
\item
  \textbf{Negative half-cycle}: D2 conducts, D1 blocks
\item
  \textbf{Output}: Both half-cycles of input converted to positive
  output
\end{itemize}

\textbf{Waveforms:}

\begin{verbatim}
Input:  ∿∿∿∿∿∿∿∿∿∿∿∿∿∿∿
         |
         v
Bridge: 
Rectifier
         |
         v
Output: 
(with filter)
\end{verbatim}

\end{solutionbox}
\begin{mnemonicbox}
``FOUR-TWO'' - FOUr diodes for Bridge, TWO diodes for
Center-Tap

\end{mnemonicbox}
\subsection*{Question 3(a) [3 marks]}\label{q3a}

\textbf{Explain the characteristic of Varactor diode.}

\begin{solutionbox}

\textbf{Varactor Diode Characteristics:}

\begin{center}
\textbf{Mermaid Diagram (Code)}
\begin{verbatim}
{Shaded}
{Highlighting}[]
graph LR
    A[Reverse Bias{br /{}Voltage] {-}{-}{} B[Depletion{}br /{}Layer Width]}
    B {-{-}{} C[Junction{}br /{}Capacitance]}
    C {-{-}{} D[Frequency{}br /{}Tuning]}
    style A fill:\#lightblue
    style B fill:\#lightpink
    style C fill:\#lightgreen
    style D fill:\#lightyellow
{Highlighting}
{Shaded}
\end{verbatim}
\end{center}

\begin{itemize}
\tightlist
\item
  \textbf{Operating principle}: Junction capacitance varies with reverse
  bias voltage
\item
  \textbf{C-V relationship}: Capacitance decreases as reverse voltage
  increases
\item
  \textbf{Tuning ratio}: Typically 4:1 to 10:1 capacitance variation
\item
  \textbf{Applications}: Voltage-controlled oscillators, FM modulation,
  tuning circuits
\end{itemize}

\end{solutionbox}
\begin{mnemonicbox}
``VARA'' - Voltage Adjusts Reverse-biased capacitance
Automatically

\end{mnemonicbox}
\subsection*{Question 3(b) [3 marks]}\label{q3b}

\textbf{State and explain Faraday's laws of electromagnetic induction.}

\begin{solutionbox}

\textbf{Faraday's Laws of Electromagnetic Induction:}

\textbf{First Law:}

\begin{itemize}
\tightlist
\item
  \textbf{Statement}: Whenever a conductor cuts magnetic flux, an EMF is
  induced in the conductor
\item
  \textbf{Mathematical expression}: EMF ∝ Rate of change of magnetic
  flux
\item
  \textbf{Application}: Basis for generators, transformers, inductors
\end{itemize}

\textbf{Second Law:}

\begin{itemize}
\tightlist
\item
  \textbf{Statement}: The magnitude of induced EMF equals the rate of
  change of magnetic flux linkage
\item
  \textbf{Mathematical expression}: EMF = -N \times (dΦ/dt)

  \begin{itemize}
  \tightlist
  \item
Where:

N = number of turns, dΦ/dt = rate of change of flux

  \end{itemize}
\item
  \textbf{Negative sign}: Indicates direction (Lenz's Law) - induced
  current opposes the change
\end{itemize}

\textbf{Diagram:}

\begin{verbatim}
    N     S       
    |     |       
    v     v       
  +{-{-}{-}+ +{-}{-}{-}+     }
  |   | |   |     
  |   | |   |     
  +{-{-}{-}+ +{-}{-}{-}+     }
    \^{     \^{}       }
    |     |       
    |     |       
  +{-{-}{-}{-}{-}{-}{-}{-}{-}{-}+    }
  |   Coil   |{-{-}{-}{-}{-} Induced EMF}
  +{-{-}{-}{-}{-}{-}{-}{-}{-}{-}+    }
\end{verbatim}

\end{solutionbox}
\begin{mnemonicbox}
``FACE'' - Flux Alteration Creates Electricity

\end{mnemonicbox}
\subsection*{Question 3(c) [7 marks]}\label{q3c}

\textbf{Compare different Transistor Configurations.}

\begin{solutionbox}

{\def\LTcaptype{none} % do not increment counter
\begin{longtable}[]{@{}
  >{\raggedright\arraybackslash}p{(\linewidth - 6\tabcolsep) * \real{0.2500}}
  >{\raggedright\arraybackslash}p{(\linewidth - 6\tabcolsep) * \real{0.2500}}
  >{\raggedright\arraybackslash}p{(\linewidth - 6\tabcolsep) * \real{0.2500}}
  >{\raggedright\arraybackslash}p{(\linewidth - 6\tabcolsep) * \real{0.2500}}@{}}
\toprule\noalign{}
\begin{minipage}[b]{\linewidth}\raggedright
\textbf{Parameter}
\end{minipage} & \begin{minipage}[b]{\linewidth}\raggedright
\textbf{Common Emitter (CE)}
\end{minipage} & \begin{minipage}[b]{\linewidth}\raggedright
\textbf{Common Base (CB)}
\end{minipage} & \begin{minipage}[b]{\linewidth}\raggedright
\textbf{Common Collector (CC)}
\end{minipage} \\
\midrule\noalign{}
\endhead
\bottomrule\noalign{}
\endlastfoot
\textbf{Input Terminal} & Base & Emitter & Base \\
\textbf{Output Terminal} & Collector & Collector & Emitter \\
\textbf{Common Terminal} & Emitter & Base & Collector \\
\textbf{Current Gain (α, β, γ)} & β = IC/IB (20-500) & α = IC/IE
(0.95-0.99) & γ = IE/IB (β+1) \\
\textbf{Voltage Gain} & High (250-1000) & Medium (150-800) & Less than
1 \\
\textbf{Input Impedance} & Medium (1-2kΩ) & Low (30-150Ω) & High
(50-500kΩ) \\
\textbf{Output Impedance} & High (30-50kΩ) & Very high (250kΩ-1MΩ) & Low
(50-100Ω) \\
\textbf{Phase Shift} & 180^\circ & 0^\circ & 0^\circ \\
\textbf{Applications} & Amplifiers, oscillators & RF amplifiers,
high-frequency circuits & Impedance matching, buffers \\
\end{longtable}
}

\textbf{Relationship between α, β and γ:}

\begin{itemize}
\tightlist
\item
  β = α/(1-α)
\item
  α = β/(1+β)
\item
  γ = β+1
\end{itemize}

\end{solutionbox}
\begin{mnemonicbox}
``BEC'' - Base input for Emitter output needs
Collector as common terminal

\end{mnemonicbox}
\subsection*{Question 3(a) OR [3
marks]}\label{q3a}

\textbf{What is forbidden energy gap? Draw the energy band diagram for
insulator, conductor and semiconductor.}

\begin{solutionbox}

\textbf{Forbidden Energy Gap:} Energy range in a solid where no electron
states exist, separating the valence band from the conduction band.

\textbf{Energy Band Diagrams:}

\begin{verbatim}
+{-{-}{-}{-}{-}{-}{-}{-}{-}{-}{-}{-}{-}{-}{-}+  +{-}{-}{-}{-}{-}{-}{-}{-}{-}{-}{-}{-}{-}{-}{-}+  +{-}{-}{-}{-}{-}{-}{-}{-}{-}{-}{-}{-}{-}{-}{-}+}
|///////////////|  |///////////////|  |///////////////|
|/// Conduction |  |/// Conduction |  |/// Conduction |
|///////////////|  |///////////////|  |///////////////|
+{-{-}{-}{-}{-}{-}{-}{-}{-}{-}{-}{-}{-}{-}{-}+  +{-}{-}{-}{-}{-}{-}{-}{-}{-}{-}{-}{-}{-}{-}{-}+  +{-}{-}{-}{-}{-}{-}{-}{-}{-}{-}{-}{-}{-}{-}{-}+}
|               |  |///////////////|  |      |        |
|               |  |///////////////|  |      | Small  |
| Large         |  | Overlap       |  |      | Gap    |
| Forbidden     |  |///////////////|  |      |        |
| Gap ({5eV)    |  |///////////////|  |      | (1eV) |}
|               |  |///////////////|  |      |        |
+{-{-}{-}{-}{-}{-}{-}{-}{-}{-}{-}{-}{-}{-}{-}+  +{-}{-}{-}{-}{-}{-}{-}{-}{-}{-}{-}{-}{-}{-}{-}+  +{-}{-}{-}{-}{-}{-}{-}{-}{-}{-}{-}{-}{-}{-}{-}+}
|///////////////|  |///////////////|  |///////////////|
|/// Valence    |  |/// Valence    |  |/// Valence    |
|///////////////|  |///////////////|  |///////////////|
+{-{-}{-}{-}{-}{-}{-}{-}{-}{-}{-}{-}{-}{-}{-}+  +{-}{-}{-}{-}{-}{-}{-}{-}{-}{-}{-}{-}{-}{-}{-}+  +{-}{-}{-}{-}{-}{-}{-}{-}{-}{-}{-}{-}{-}{-}{-}+}
    Insulator          Conductor        Semiconductor
\end{verbatim}

\begin{itemize}
\tightlist
\item
  \textbf{Insulator}: Large forbidden gap (\textgreater5eV) prevents
  electrons from reaching conduction band
\item
  \textbf{Conductor}: Overlapping bands allow free electron movement
\item
  \textbf{Semiconductor}: Small gap (\textasciitilde1eV) allows some
  electrons to cross at room temperature or when excited
\end{itemize}

\end{solutionbox}
\begin{mnemonicbox}
``IBCS'' - Insulators Block, Conductors Share,
Semiconductors have gap Between

\end{mnemonicbox}
\subsection*{Question 3(b) OR [4
marks]}\label{q3b}

\textbf{Explain the function of Zener diode as a voltage regulator}

\begin{solutionbox}

\begin{center}
\textbf{Mermaid Diagram (Code)}
\begin{verbatim}
{Shaded}
{Highlighting}[]
graph LR
    A[Unregulated{br /{}DC Input] {-}{-}{} B[Series{}br /{}Resistor]}
    B {-{-}{} C[Load]}
    B {-{-}{} D[Zener{}br /{}Diode]}
    D {-{-}{} E[Ground]}
    style A fill:\#lightblue
    style B fill:\#lightpink
    style C fill:\#lightgreen
    style D fill:\#lightyellow
    style E fill:\#lightgray
{Highlighting}
{Shaded}
\end{verbatim}
\end{center}

\textbf{Working Principle:}

\begin{itemize}
\tightlist
\item
  \textbf{Normal operation}: Zener diode is reverse biased and conducts
  when voltage reaches breakdown voltage
\item
  \textbf{Voltage regulation}: When input voltage rises, more current
  flows through Zener diode, maintaining constant voltage across it
\item
  \textbf{Load variation}: When load draws more current, less current
  flows through Zener, keeping voltage stable
\item
  \textbf{Series resistor}: Limits current and drops excess voltage
\end{itemize}

\textbf{Circuit behavior:}

\begin{itemize}
\tightlist
\item
  \textbf{Vout = Vz} (Zener breakdown voltage)
\item
  \textbf{Iz = (Vin - Vz)/R - IL}
\end{itemize}

\end{solutionbox}
\begin{mnemonicbox}
``SERZ'' - Series resistor Enables Regulation with
Zener

\end{mnemonicbox}
\subsection*{Question 3(c) OR [7
marks]}\label{q3c}

\textbf{Explain V-I char of P-N junction diode and give comparison
between P-N junction diode and Zener diode.}

\begin{solutionbox}

\textbf{V-I Characteristics of P-N Junction Diode:}

\begin{verbatim}
                 I
                 \^{}
                 |              /
                 |             /
                 |            /
                 |           /
Forward current  |          /
                 |         /
                 |        /
                 |       /
                 |  Knee/
                 |     /
        {-V {-}{-}{-}{-}{-}|{-}{-}{-}{-}+{-}{-}{-} +V}
                 |   /|
                 |    |
                 |    |
Reverse current  |    |
                 |    |    Breakdown
                 |    |       |
                 |    |       v
                 |    |       ⌄\_\_\_\_\_
\end{verbatim}

\textbf{Key Points:}

\begin{itemize}
\tightlist
\item
  \textbf{Forward bias}: Conducts easily after exceeding knee voltage
  (\textasciitilde0.7V for silicon)
\item
  \textbf{Reverse bias}: Very small leakage current until breakdown
  voltage
\item
  \textbf{Breakdown region}: Occurs at high reverse voltage, causes
  damage in normal diodes
\end{itemize}

\textbf{P-N Junction Diode vs.~Zener Diode:}

{\def\LTcaptype{none} % do not increment counter
\begin{longtable}[]{@{}
  >{\raggedright\arraybackslash}p{(\linewidth - 4\tabcolsep) * \real{0.3333}}
  >{\raggedright\arraybackslash}p{(\linewidth - 4\tabcolsep) * \real{0.3333}}
  >{\raggedright\arraybackslash}p{(\linewidth - 4\tabcolsep) * \real{0.3333}}@{}}
\toprule\noalign{}
\begin{minipage}[b]{\linewidth}\raggedright
\textbf{Parameter}
\end{minipage} & \begin{minipage}[b]{\linewidth}\raggedright
\textbf{P-N Junction Diode}
\end{minipage} & \begin{minipage}[b]{\linewidth}\raggedright
\textbf{Zener Diode}
\end{minipage} \\
\midrule\noalign{}
\endhead
\bottomrule\noalign{}
\endlastfoot
\textbf{Symbol} & ▷\textbar--- & ▷\textbar---◁ \\
\textbf{Forward operation} & Conducts easily & Same as normal diode \\
\textbf{Reverse breakdown} & At high voltage, causes damage &
Controlled, non-destructive \\
\textbf{Doping level} & Moderate & Heavily doped \\
\textbf{Operating region} & Forward biased & Reverse biased (breakdown
region) \\
\textbf{Applications} & Rectification, switching & Voltage regulation,
reference \\
\textbf{Breakdown mechanism} & Avalanche & Zener effect and avalanche \\
\textbf{Temperature coefficient} & Negative & Can be positive or
negative \\
\end{longtable}
}

\end{solutionbox}
\begin{mnemonicbox}
``FORD'' - Forward Operation for Rectifiers, Diodes;
reverse operation for Zeners

\end{mnemonicbox}
\subsection*{Question 4(a) [3 marks]}\label{q4a}

\textbf{Describe working principle of Photodiode.}

\begin{solutionbox}

\textbf{Working Principle of Photodiode:}

\begin{center}
\textbf{Mermaid Diagram (Code)}
\begin{verbatim}
{Shaded}
{Highlighting}[]
graph LR
    A[Light] {-{-}{} B[P{-}N Junction]}
    B {-{-}{} C[Electron{-}Hole{}br /{}Pairs]}
    C {-{-}{} D[Photocurrent]}
    style A fill:\#lightyellow
    style B fill:\#lightpink
    style C fill:\#lightblue
    style D fill:\#lightgreen
{Highlighting}
{Shaded}
\end{verbatim}
\end{center}

\begin{itemize}
\tightlist
\item
  \textbf{Construction}: P-N junction diode with transparent window or
  lens
\item
  \textbf{Operation}: Reverse biased operation for light detection
\item
  \textbf{Photon absorption}: Incoming photons create electron-hole
  pairs in depletion region
\item
  \textbf{Current generation}: Electric field sweeps carriers to
  respective terminals, creating photocurrent
\item
  \textbf{Light sensitivity}: Current proportional to light intensity
\end{itemize}

\end{solutionbox}
\begin{mnemonicbox}
``LIGER'' - Light Induces Generation of Electrons in
Reverse-bias

\end{mnemonicbox}
\subsection*{Question 4(b) [4 marks]}\label{q4b}

\textbf{Explain the characteristic of Schottky barrier diode.}

\begin{solutionbox}

\textbf{Schottky Barrier Diode Characteristics:}

\begin{verbatim}
                 I
                 \^{}
                 |              /
                 |             / Schottky
                 |            /
                 |           /   PN Junction
Forward current  |          / ,/
                 |         / /
                 |        / /
                 |       / /
                 |      / /
                 |     //
        {-V {-}{-}{-}{-}{-}|{-}{-}{-}{-}|{-}{-}{-} +V}
                 |    |
                 |    |
                 |    |
Reverse current  |    |
                 |    |
\end{verbatim}

\begin{itemize}
\tightlist
\item
  \textbf{Low forward voltage drop}: 0.2-0.3V compared to 0.7V for
  silicon PN junction
\item
  \textbf{Fast switching}: No minority carrier storage, minimal reverse
  recovery time
\item
  \textbf{Construction}: Metal-semiconductor junction instead of P-N
  junction
\item
  \textbf{No reverse recovery time}: Majority carrier device (no stored
  charge)
\item
  \textbf{Applications}: High-frequency applications, rectifiers in
  power supplies
\end{itemize}

\end{solutionbox}
\begin{mnemonicbox}
``FAST'' - Forward voltage low, Allows Switching
Timely

\end{mnemonicbox}
\subsection*{Question 4(c) [7 marks]}\label{q4c}

\textbf{Explain working principle of PNP and NPN transistor.}

\begin{solutionbox}

\textbf{NPN Transistor Structure and Working:}

\begin{verbatim}
    +{-{-}{-}{-}{-}{-}{-}+     +{-}{-}{-}{-}{-}{-}{-}+     +{-}{-}{-}{-}{-}{-}{-}+}
    |       |     |       |     |       |
    |   N   |     |   P   |     |   N   |
    |       |     |       |     |       |
    +{-{-}{-}{-}{-}{-}{-}+     +{-}{-}{-}{-}{-}{-}{-}+     +{-}{-}{-}{-}{-}{-}{-}+}
    Emitter        Base        Collector
        |            |            |
        |            |            |
        v            v            v
    Electron      Hole        Electron
     source    controller     collector
\end{verbatim}

\begin{itemize}
\tightlist
\item
  \textbf{Biasing}: Emitter-base junction forward biased, collector-base
  junction reverse biased
\item
  \textbf{Current flow}: Electrons from emitter to collector through
  thin base region
\item
  \textbf{Amplification principle}: Small base current controls larger
  collector current
\item
  \textbf{Current relationship}: IE = IB + IC
\item
  \textbf{Majority carriers}: Electrons
\end{itemize}

\textbf{PNP Transistor Structure and Working:}

\begin{verbatim}
    +{-{-}{-}{-}{-}{-}{-}+     +{-}{-}{-}{-}{-}{-}{-}+     +{-}{-}{-}{-}{-}{-}{-}+}
    |       |     |       |     |       |
    |   P   |     |   N   |     |   P   |
    |       |     |       |     |       |
    +{-{-}{-}{-}{-}{-}{-}+     +{-}{-}{-}{-}{-}{-}{-}+     +{-}{-}{-}{-}{-}{-}{-}+}
    Emitter        Base        Collector
        |            |            |
        |            |            |
        v            v            v
     Hole         Electron        Hole
     source     controller     collector
\end{verbatim}

\begin{itemize}
\tightlist
\item
  \textbf{Biasing}: Emitter-base junction forward biased, collector-base
  junction reverse biased
\item
  \textbf{Current flow}: Holes from emitter to collector through thin
  base region
\item
  \textbf{Amplification principle}: Small base current controls larger
  collector current
\item
  \textbf{Current relationship}: IE = IB + IC
\item
  \textbf{Majority carriers}: Holes
\item
  \textbf{Current direction}: Opposite to NPN (conventional current from
  emitter to collector)
\end{itemize}

\end{solutionbox}
\begin{mnemonicbox}
``NPNP'' - Negative carriers in NPN, Positive
carriers in PNP

\end{mnemonicbox}
\subsection*{Question 4(a) OR [3
marks]}\label{q4a}

\textbf{Describe working principle of LED.}

\begin{solutionbox}

\textbf{Working Principle of LED:}

\begin{center}
\textbf{Mermaid Diagram (Code)}
\begin{verbatim}
{Shaded}
{Highlighting}[]
graph LR
    A[Forward Bias] {-{-}{} B[Electron{-}Hole{}br /{}Recombination]}
    B {-{-}{} C[Energy Release{}br /{}as Photons]}
    C {-{-}{} D[Light Emission]}
    style A fill:\#lightblue
    style B fill:\#lightpink
    style C fill:\#lightyellow
    style D fill:\#lightgreen
{Highlighting}
{Shaded}
\end{verbatim}
\end{center}

\begin{itemize}
\tightlist
\item
  \textbf{Construction}: P-N junction made from direct bandgap
  semiconductor materials
\item
  \textbf{Forward biasing}: Electrons from n-region and holes from
  p-region recombine at junction
\item
  \textbf{Recombination}: Electrons fall from conduction band to valence
  band
\item
  \textbf{Energy emission}: Energy released during recombination emits
  photons (light)
\item
  \textbf{Color determination}: Bandgap energy determines wavelength
  (color) of emitted light
\end{itemize}

\end{solutionbox}
\begin{mnemonicbox}
``REBEL'' - Recombination of Electrons and holes By
Energetic Light emission

\end{mnemonicbox}
\subsection*{Question 4(b) OR [4
marks]}\label{q4b}

\textbf{Explain function of transistor as switch in cut off and
application of saturation region.}

\begin{solutionbox}

\textbf{Transistor as a Switch:}

\begin{verbatim}
        Input                        Output
          |                            |
          |                            |
          v                            v
    +{-{-}{-}{-}{-}{-}{-}{-}{-}{-}+                 +{-}{-}{-}{-}{-}{-}{-}{-}{-}{-}+}
    |    R1    |                 |    RC    |
    +{-{-}{-}{-}{-}{-}{-}{-}{-}{-}+                 +{-}{-}{-}{-}{-}{-}{-}{-}{-}{-}+}
          |                            |
          |                            |
          |     +{-{-}{-}{-}{-}{-}{-}{-}{-}{-}{-}{-}{-}{-}{-}+      |}
          +{-{-}{-}{-}|     B     C   |{-}{-}{-}{-}{-}{-}+}
                |       |       |
                |       |       |
                |     E         |
                +{-{-}{-}{-}{-}{-}{-}{-}{-}{-}{-}{-}{-}{-}{-}+}
                     |
                     |
                 Ground
\end{verbatim}

\textbf{Cut-off Region (Switch OFF):}

\begin{itemize}
\tightlist
\item
  \textbf{Base voltage}: Below 0.7V (for silicon)
\item
  \textbf{Base current}: Approximately zero
\item
  \textbf{Collector current}: Approximately zero
\item
  \textbf{Collector-emitter voltage}: Equal to supply voltage
\item
  \textbf{Applications}: Logic gates, digital circuits, relay drivers
\end{itemize}

\textbf{Saturation Region (Switch ON):}

\begin{itemize}
\tightlist
\item
  \textbf{Base voltage}: Well above 0.7V
\item
  \textbf{Base current}: Sufficient to ensure minimum VCE
\item
  \textbf{Collector current}: Maximum (limited by collector resistor)
\item
  \textbf{Collector-emitter voltage}: Very low (0.2V - 0.3V)
\item
  \textbf{Applications}: Digital switches, motor drivers, LED drivers
\end{itemize}

\end{solutionbox}
\begin{mnemonicbox}
``COSI'' - Cutoff Opens Switch, Input saturates to
close

\end{mnemonicbox}
\subsection*{Question 4(c) OR [7
marks]}\label{q4c}

\textbf{Explain common emitter (CE) configuration of Transistor. Derive
relation between α and β for transistor amplifier.}

\begin{solutionbox}

\textbf{Common Emitter Configuration:}

\begin{verbatim}
graph TB
    A[Input Signal] {-{-} B[Base]}
    C[Output Signal] {-{-} D[Collector]}
    E[Ground] {-{-} F[Emitter]}
    style A fill:\#lightblue
    style B fill:\#lightpink
    style C fill:\#lightgreen
    style D fill:\#lightyellow
    style E fill:\#lightgray
    style F fill:\#lightcyan
\end{verbatim}

\textbf{Characteristics of Common Emitter Configuration:}

\begin{itemize}
\tightlist
\item
  \textbf{Input terminal}: Base
\item
  \textbf{Output terminal}: Collector
\item
  \textbf{Common terminal}: Emitter (grounded)
\item
  \textbf{Current gain (β)}: High (20-500)
\item
  \textbf{Voltage gain}: High (250-1000)
\item
  \textbf{Input impedance}: Medium (1-2kΩ)
\item
  \textbf{Output impedance}: High (30-50kΩ)
\item
  \textbf{Phase shift}: 180^\circ (output inverted from input)
\end{itemize}

\textbf{Relationship between α and β:}

By definition:

\begin{itemize}
\tightlist
\item
  α = IC/IE (Common Base current gain)
\item
  β = IC/IB (Common Emitter current gain)
\end{itemize}

From Kirchhoff's Current Law:

\begin{itemize}
\tightlist
\item
  IE = IB + IC
\end{itemize}

Dividing both sides by IE:

\begin{itemize}
\tightlist
\item
  1 = IB/IE + IC/IE
\item
  1 = IB/IE + α
\end{itemize}

Therefore:

\begin{itemize}
\tightlist
\item
  IB/IE = 1 - α
\end{itemize}

Now,

β = IC/IB = (IC/IE)/(IB/IE) = α/(1-α)


And conversely:

\begin{itemize}
\tightlist
\item
  α = β/(1+β)
\end{itemize}

\end{solutionbox}
\begin{mnemonicbox}
``BEAR'' - Beta Equals Alpha divided by (1-alpha)
Relation

\end{mnemonicbox}
\subsection*{Question 5(a) [3 marks]}\label{q5a}

\textbf{What do you mean by E-waste? What are the different methods of
E-waste disposal?}

\begin{solutionbox}

\textbf{E-waste (Electronic Waste)}: Discarded electronic devices and
components that have reached end of life or are no longer useful.

\textbf{Methods of E-waste Disposal:}

{\def\LTcaptype{none} % do not increment counter
\begin{longtable}[]{@{}
  >{\raggedright\arraybackslash}p{(\linewidth - 2\tabcolsep) * \real{0.5000}}
  >{\raggedright\arraybackslash}p{(\linewidth - 2\tabcolsep) * \real{0.5000}}@{}}
\toprule\noalign{}
\begin{minipage}[b]{\linewidth}\raggedright
\textbf{Disposal Method}
\end{minipage} & \begin{minipage}[b]{\linewidth}\raggedright
\textbf{Description}
\end{minipage} \\
\midrule\noalign{}
\endhead
\bottomrule\noalign{}
\endlastfoot
\textbf{Recycling} & Separating valuable materials like metals, plastics
for reuse \\
\textbf{Landfilling} & Disposing in designated landfills (not
recommended) \\
\textbf{Incineration} & Burning waste at high temperatures (creates
toxic emissions) \\
\textbf{Reuse/Refurbishment} & Repairing and upgrading for extended
use \\
\textbf{Extended Producer Responsibility} & Manufacturers take back and
handle disposal \\
\end{longtable}
}

\end{solutionbox}
\begin{mnemonicbox}
``RIPER'' - Recycling Is Preferable to
Environmentally-harmful Remedies

\end{mnemonicbox}
\subsection*{Question 5(b) [4 marks]}\label{q5b}

\textbf{Explain methods of handling electronic waste with examples.}

\begin{solutionbox}

\textbf{Methods of Handling Electronic Waste:}

\begin{center}
\textbf{Mermaid Diagram (Code)}
\begin{verbatim}
{Shaded}
{Highlighting}[]
graph LR
    A[E{-waste{}br /{}Collection] {-}{-}{} B[Sorting]}
    B {-{-}{} C[Dismantling]}
    C {-{-}{} D[Material{}br /{}Recovery]}
    D {-{-}{} E[Safe{}br /{}Disposal]}
    style A fill:\#lightblue
    style B fill:\#lightpink
    style C fill:\#lightyellow
    style D fill:\#lightgreen
    style E fill:\#lightgray
{Highlighting}
{Shaded}
\end{verbatim}
\end{center}

\textbf{Collection and Segregation:}

\begin{itemize}
\tightlist
\item
  \textbf{Example}: Dedicated e-waste bins in public places, e-waste
  collection drives
\item
  \textbf{Benefit}: Prevents mixing with general waste, enables proper
  processing
\end{itemize}

\textbf{Dismantling and Resource Recovery:}

\begin{itemize}
\tightlist
\item
  \textbf{Example}: Recovering gold, silver, copper from circuit boards
  and connectors
\item
  \textbf{Benefit}: Recovers valuable metals, reduces mining demands
\end{itemize}

\textbf{Refurbishment and Reuse:}

\begin{itemize}
\tightlist
\item
  \textbf{Example}: Repairing old computers for educational institutions
\item
  \textbf{Benefit}: Extends product lifecycle, reduces waste generation
\end{itemize}

\textbf{Proper Disposal of Hazardous Components:}

\begin{itemize}
\tightlist
\item
  \textbf{Example}: Specialized treatment for mercury-containing
  components
\item
  \textbf{Benefit}: Prevents toxic substances from entering environment
\end{itemize}

\end{solutionbox}
\begin{mnemonicbox}
``CREED'' - Collect, Recover, Extract, Extend,
Dispose safely

\end{mnemonicbox}
\subsection*{Question 5(c) [7 marks]}\label{q5c}

\textbf{What is ripple factor? Derive the equation of the ripple factor
for rectifier.}

\begin{solutionbox}

\textbf{Ripple Factor}: Measure of effectiveness of a rectifier's
filtering - the ratio of AC component (ripple) to DC component in the
output.

\textbf{Definition}:

\begin{itemize}
\tightlist
\item
  Ripple factor (γ) = RMS value of AC component / DC value
\item
  Lower ripple factor indicates better filtering
\end{itemize}

\textbf{Derivation for Half Wave Rectifier:}

Let's assume sinusoidal input: v = Vmsinωt

For half wave rectifier:

\begin{itemize}
\tightlist
\item
  Output is v = Vmsinωt for 0 \leq ωt \leq π
\item
  Output is v = 0 for π \leq ωt \leq 2π
\end{itemize}

\textbf{Step 1}: Find DC component (average value)

\begin{itemize}
\tightlist
\item
  VDC = (1/2π) \int02π v(ωt) d(ωt)
\item
  VDC = (1/2π) \int0π Vmsinωt d(ωt)
\item
  VDC = Vm/π
\end{itemize}

\textbf{Step 2}: Find RMS value

\begin{itemize}
\tightlist
\item
  VRMS = \sqrt[(1/2π) \int02π v^{2}(ωt) d(ωt)]
\item
  VRMS = \sqrt[(1/2π) \int0π Vm^{2}sin^{2}ωt d(ωt)]
\item
  VRMS = Vm/2
\end{itemize}

\textbf{Step 3}: Find AC component

\begin{itemize}
\tightlist
\item
  VAC^{2} = VRMS^{2} - VDC^{2}
\item
  VAC^{2} = (Vm/2)^{2} - (Vm/π)^{2}
\item
  VAC^{2} = Vm^{2}(1/4 - 1/π^{2})
\end{itemize}

\textbf{Step 4}: Calculate ripple factor

\begin{itemize}
\tightlist
\item
  γ = VAC/VDC
\item
  γ = \sqrt(Vm^{2}(1/4 - 1/π^{2}))/(Vm/π)
\item
  γ = π\sqrt(1/4 - 1/π^{2})
\item
  γ = 1.21 (for half wave rectifier)
\end{itemize}

\textbf{For Full Wave Rectifier}: Following similar steps leads to γ =
0.48

\end{solutionbox}
\begin{mnemonicbox}
``ROAD'' - Ripple is Output's AC Divided by DC
component

\end{mnemonicbox}
\subsection*{Question 5(a) OR [3
marks]}\label{q5a}

\textbf{Which are the toxic substances present in e-waste?}

\begin{solutionbox}

\textbf{Toxic Substances in E-waste:}

{\def\LTcaptype{none} % do not increment counter
\begin{longtable}[]{@{}
  >{\raggedright\arraybackslash}p{(\linewidth - 4\tabcolsep) * \real{0.3333}}
  >{\raggedright\arraybackslash}p{(\linewidth - 4\tabcolsep) * \real{0.3333}}
  >{\raggedright\arraybackslash}p{(\linewidth - 4\tabcolsep) * \real{0.3333}}@{}}
\toprule\noalign{}
\begin{minipage}[b]{\linewidth}\raggedright
\textbf{Toxic Substance}
\end{minipage} & \begin{minipage}[b]{\linewidth}\raggedright
\textbf{Source in Electronics}
\end{minipage} & \begin{minipage}[b]{\linewidth}\raggedright
\textbf{Health/Environmental Impact}
\end{minipage} \\
\midrule\noalign{}
\endhead
\bottomrule\noalign{}
\endlastfoot
\textbf{Lead (Pb)} & Solder, CRT monitors, batteries & Neurological
damage, developmental issues \\
\textbf{Mercury (Hg)} & Switches, backlights, batteries & Neurological
and kidney damage \\
\textbf{Cadmium (Cd)} & Rechargeable batteries, circuit boards & Kidney
damage, bone disease \\
\textbf{Brominated Flame Retardants} & Plastic casings, circuit boards &
Endocrine disruption, bioaccumulation \\
\textbf{Hexavalent Chromium} & Corrosion protection in metal parts &
Allergic reactions, DNA damage \\
\textbf{Beryllium (Be)} & Connectors, springs & Lung disease, skin
disorders \\
\end{longtable}
}

\end{solutionbox}
\begin{mnemonicbox}
``LMBCHB'' - Lead, Mercury, and Beryllium Cause
Harmful Bodily effects

\end{mnemonicbox}
\subsection*{Question 5(b) OR [4
marks]}\label{q5b}

\textbf{Write important parameters for selecting the right transistor
for your application and explain any two.}

\begin{solutionbox}

\textbf{Important Transistor Selection Parameters:}

\begin{itemize}
\tightlist
\item
  Maximum collector current (IC)
\item
  Maximum collector-emitter voltage (VCEO)
\item
  Maximum collector-base voltage (VCBO)
\item
  Current gain (hFE or β)
\item
  Frequency response (fT)
\item
  Power dissipation (Ptot)
\item
  Package type (TO-3, SMT, etc.)
\item
  Temperature range
\end{itemize}

\textbf{Maximum Collector Current (IC):}

\begin{itemize}
\tightlist
\item
  \textbf{Definition}: Maximum current that can flow through collector
  without damage
\item
  \textbf{Importance}: Must exceed application's peak current
  requirements with safety margin
\item
  \textbf{Typical values}: 100mA to 100A depending on transistor type
\item
  \textbf{Application consideration}: Select 50\% higher rating than
  maximum required current
\end{itemize}

\textbf{Current Gain (hFE or β):}

\begin{itemize}
\tightlist
\item
  \textbf{Definition}: Ratio of collector current to base current
\item
  \textbf{Importance}: Determines amplification capability and required
  base drive
\item
  \textbf{Typical values}: 20-500 for general-purpose transistors
\item
  \textbf{Application consideration}: For switching, high gain reduces
  base current requirement; for amplifiers, consistent gain across
  operating range is important
\end{itemize}

\end{solutionbox}
\begin{mnemonicbox}
``GIVE'' - Gain and Ic are Very Essential parameters

\end{mnemonicbox}
\subsection*{Question 5(c) OR [7
marks]}\label{q5c}

\textbf{What is rectifier efficiency? Find out efficiency of the full
wave rectifier.}

\begin{solutionbox}

\textbf{Rectifier Efficiency}: The ratio of DC output power to the AC
input power, expressed as a percentage.

\textbf{Definition}:

\begin{itemize}
\tightlist
\item
  Efficiency (η) = (PDC/PAC) \times 100\%
\item
  Higher efficiency means better conversion of AC to DC power
\end{itemize}

\textbf{Derivation for Full Wave Rectifier:}

\textbf{Step 1}: Calculate DC output power

\begin{itemize}
\tightlist
\item
  IDC = VDC/RL
\item
  PDC = IDC^{2} \times RL = VDC^{2}/RL
\item
  For full wave, VDC = 2Vm/π
\item
  PDC = (2Vm/π)^{2}/RL = 4Vm^{2}/(π^{2}RL)
\end{itemize}

\textbf{Step 2}: Calculate AC input power

\begin{itemize}
\tightlist
\item
  IRMS = VRMS/RL
\item
  PAC = IRMS^{2} \times RL = VRMS^{2}/RL
\item
  For sine wave, VRMS = Vm/\sqrt2
\item
  PAC = (Vm/\sqrt2)^{2}/RL = Vm^{2}/(2RL)
\end{itemize}

\textbf{Step 3}: Calculate efficiency

\begin{itemize}
\tightlist
\item
  η = (PDC/PAC) \times 100\%
\item
  η = [4Vm^{2}/(π^{2}RL)] / [Vm^{2}/(2RL)] \times 100\%
\item
  η = [4/(π^{2})] \times 2 \times 100\%
\item
  η = 8/(π^{2}) \times 100\%
\item
  η = 8/9.87 \times 100\%
\item
  η = 81.2\%
\end{itemize}

\textbf{Full Wave Rectifier Efficiency} = 81.2\%

\textbf{For comparison:}

\begin{itemize}
\tightlist
\item
  Half Wave Rectifier Efficiency = 40.6\%
\item
  Bridge Rectifier Efficiency = 81.2\%
\end{itemize}

\end{solutionbox}
\begin{mnemonicbox}
``PIDE'' - Power Input Determines Efficiency

\end{mnemonicbox}

\end{document}
