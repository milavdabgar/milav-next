\documentclass{article}

% content/resources/templates/preamble.tex
\usepackage[margin=0.6in]{geometry}
\author{Milav Dabgar}
\usepackage{amsmath,amssymb,amsthm}
\usepackage{booktabs}
\usepackage{multirow}
\usepackage{xcolor}
\usepackage{tcolorbox}
\tcbuselibrary{breakable,skins}
\usepackage[colorlinks=true,linkcolor=blue]{hyperref}
\usepackage{titlesec}
\usepackage{enumitem}
\usepackage{tikz}
\usepackage{pgfplots}
\usepackage{circuitikz}
\usepackage[version=4]{mhchem}
\usepackage{longtable}
\usepackage{array}
\usepackage{float}
\usepackage{caption}
\usepackage{listings}

\lstset{
  basicstyle=\small\ttfamily,
  breaklines=true,
  breakatwhitespace=false,
  postbreak=\mbox{\textcolor{red}{$\hookrightarrow$}\space},
  float=false,
  numbers=left,
  numberstyle=\tiny\color{gray},
  numbersep=10pt,
  xleftmargin=2em,
  keywordstyle=\color{blue},
  commentstyle=\color{green!60!black},
  stringstyle=\color{purple},
  backgroundcolor=\color{gray!5},
  showstringspaces=false,
  tabsize=2,
  captionpos=b,
  keepspaces=true,
  columns=flexible
}

\pgfplotsset{compat=1.18}
\usetikzlibrary{shapes,arrows,positioning,calc,patterns,decorations.pathmorphing,decorations.markings,arrows.meta}

% Color scheme
\definecolor{headcolor}{RGB}{0,102,204}
\definecolor{keycolor}{RGB}{220,20,60}
\definecolor{solutioncolor}{RGB}{34,139,34}
\definecolor{mnemoniccolor}{RGB}{148,0,211}
\definecolor{codecolor}{RGB}{0,0,100}

% Spacing
\setlength{\parskip}{3pt}
\setlist[itemize]{nosep}
\setlist[enumerate]{nosep}

% Title formatting
\titleformat{\section}{\Large\bfseries\color{headcolor}}{\thesection}{1em}{}
\titleformat{\subsection}{\large\bfseries\color{headcolor}}{\thesubsection}{1em}{}

% Pandoc tightlist compatibility
\providecommand{\tightlist}{%
  \setlength{\itemsep}{0pt}\setlength{\parskip}{0pt}}

% Pandoc longtable compatibility
\newcounter{none}
\def\thenone{}


% content/resources/templates/gujarati-boxes.tex
\usepackage{fontspec}
\usepackage{polyglossia}

% Set Gujarati as main language (document is primarily in Gujarati)
% Note: gloss-gujarati.ldf doesn't exist in polyglossia, but it will use hyphenation patterns
\setdefaultlanguage{gujarati}
\setotherlanguage{english}

% Configure Gujarati font properly
% Use Language=Default to prevent polyglossia from trying to add language-specific features
% that don't exist for Gujarati, which causes "empty feature" warnings
\newfontfamily\gujaratifont[Script=Gujarati,AutoFakeBold=2.5,AutoFakeSlant=0.3]{Noto Sans Gujarati}
\setmainfont[Script=Gujarati,AutoFakeBold=2.5,AutoFakeSlant=0.3]{Noto Sans Gujarati}
% Use Noto Sans Gujarati for monospace to support Gujarati in text
\setmonofont[Scale=0.9]{Noto Sans Gujarati}

% Configure English to use the same font
\newfontfamily\englishfont[Script=Gujarati,AutoFakeBold=2.5,AutoFakeSlant=0.3]{Noto Sans Gujarati}

% Translations for polyglossia
\gappto\captionsgujarati{
  \renewcommand{\tablename}{કોષ્ટક}
  \renewcommand{\figurename}{આકૃતિ}
}

% Helper for TikZ nodes to ensure Gujarati font
\newcommand{\gu}[1]{{\gujaratifont #1}}

% Custom environments
\newtcolorbox{solutionbox}{
    breakable,
    enhanced,
    colback=solutioncolor!5!white,
    colframe=solutioncolor!75!black,
    fonttitle=\bfseries,
    title=જવાબ
}

\newtcolorbox{solutionboxnobreak}{
 colback=solutioncolor!5!white,
 colframe=solutioncolor!75!black,
 fonttitle=\bfseries,
 title=જવાબ
}

\newtcolorbox{keyformula}{
 breakable,
 enhanced,
 colback=keycolor!5!white,
 colframe=keycolor!75!black,
 fonttitle=\bfseries,
 title=રાસાયણિક સમીકરણ/સૂત્ર
}

\newtcolorbox{mnemonicbox}{
 breakable,
 enhanced,
 colback=mnemoniccolor!5!white,
 colframe=mnemoniccolor!75!black,
 fonttitle=\bfseries,
 title=મેમરી ટ્રીક
}


% Custom commands for GTU solutions
% This file defines semantic commands for consistent formatting

% Question command with automatic formatting
\newcommand{\question}[2]{%
  \section*{Question #1}%
  \textbf{#2}%
}

% OR question variant
\newcommand{\questionor}[2]{%
  \section*{Question #1 OR}%
  \textbf{#2}%
}

% Proper table environment with caption
\newenvironment{answertable}[1]{%
  \begin{table}[htbp]
  \centering
  \caption{#1}
}{%
  \end{table}
}

% Proper figure environment for diagrams
\newenvironment{answerdiagram}[1]{%
  \begin{figure}[htbp]
  \centering
  \caption{#1}
}{%
  \end{figure}
}

% Semantic markup for key terms
\newcommand{\keyword}[1]{\textbf{#1}}
\newcommand{\code}[1]{\texttt{#1}}
\newcommand{\classname}[1]{\texttt{#1}}
\newcommand{\methodname}[1]{\texttt{#1}}

% Proper quotation marks
\newcommand{\mnemonic}[1]{``#1''}


\title{Fundamentals of Electronics (4311102) - Summer 2023 Solution}
\date{July 31, 2023}

\begin{document}
\maketitle

% Question 1
\questionmarks{1(a)}{3}{સક્રિય અને નિષ્ક્રિય ઘટકોને વ્યાખ્યાયિત કરો.}

\begin{solutionbox}
\textbf{જવાબ}:

\begin{center}
\captionof{table}{સક્રિય વિ. નિષ્ક્રિય ઘટકો}
\begin{tabulary}{\linewidth}{|L|L|}
\hline
\textbf{સક્રિય ઘટકો} & \textbf{નિષ્ક્રિય ઘટકો} \\ \hline
કામ કરવા માટે બાહ્ય પાવર સ્ત્રોતની જરૂર પડે છે. & બાહ્ય પાવર સ્ત્રોતની જરૂર પડતી નથી. \\ \hline
ઇલેક્ટ્રિકલ સિગ્નલને એમ્પ્લિફાય અને પ્રોસેસ કરી શકે છે. & સિગ્નલને એમ્પ્લિફાય અથવા પ્રોસેસ કરી શકતા નથી. \\ \hline
\textbf{ઉદાહરણ}: ટ્રાન્ઝિસ્ટર, ડાયોડ, ICs. & \textbf{ઉદાહરણ}: રેસિસ્ટર, કેપેસિટર, ઇન્ડક્ટર. \\ \hline
\end{tabulary}
\end{center}
\end{solutionbox}

\begin{mnemonicbox}
\mnemonic{APE: Active needs Power to Enhance signals}
\end{mnemonicbox}

\questionmarks{1(b)}{4}{વપરાયેલ સામગ્રી પર આધારિત કેપેસિટરના પ્રકારો વર્ણવો.}

\begin{solutionbox}
\textbf{જવાબ}:

\begin{center}
\captionof{table}{સામગ્રી આધારિત કેપેસિટરના પ્રકારો}
\begin{tabulary}{\linewidth}{|L|L|L|}
\hline
\textbf{મટીરિયલ ટાઇપ} & \textbf{કેપેસિટર પ્રકાર} & \textbf{સામાન્ય ઉપયોગો} \\ \hline
\textbf{સેરામિક} & સેરામિક ડિસ્ક, મલ્ટિલેયર & બાયપાસ, કપલિંગ, હાઈ ફ્રીક્વન્સી \\ \hline
\textbf{પ્લાસ્ટિક ફિલ્મ} & પોલિએસ્ટર, પોલિપ્રોપિલીન, ટેફ્લોન & ટાઈમિંગ, ફિલ્ટરિંગ, પ્રીસિઝન \\ \hline
\textbf{ઇલેક્ટ્રોલિટિક} & એલ્યુમિનિયમ, ટેન્ટાલમ & પાવર સપ્લાય, DC બ્લોકિંગ, હાઈ કેપેસિટન્સ \\ \hline
\textbf{પેપર} & પેપર ડાયલેક્ટ્રિક & જૂના ઉપકરણોમાં, હવે સામાન્ય નથી \\ \hline
\textbf{માઈકા} & સિલ્વર્ડ માઈકા & હાઈ પ્રીસિઝન RF સર્કિટ્સ \\ \hline
\textbf{ગ્લાસ} & ગ્લાસ ડાયલેક્ટ્રિક & હાઈ વોલ્ટેજ એપ્લિકેશન \\ \hline
\end{tabulary}
\end{center}
\end{solutionbox}

\begin{mnemonicbox}
\mnemonic{CEPPMG: Ceramic Electrolytic Paper Plastic Mica Glass}
\end{mnemonicbox}

\questionmarks{1(c)}{7}{રેસિસ્ટર કલર કોડિંગ ટેકનિક ઉદાહરણ સાથે સમજાવો.}

\begin{solutionbox}
\textbf{જવાબ}:

રેસિસ્ટર કલર કોડ રેસિસ્ટન્સ મૂલ્ય, ટોલરન્સ અને વિશ્વસનીયતા દર્શાવવા માટે રંગીન બેન્ડનો ઉપયોગ કરે છે.

\begin{center}
\captionof{table}{સ્ટાન્ડર્ડ રેસિસ્ટર કલર કોડ}
\begin{tabulary}{\linewidth}{|L|C|L|L|}
\hline
\textbf{રંગ} & \textbf{અંક} & \textbf{મલ્ટિપ્લાયર} & \textbf{ટોલરન્સ} \\ \hline
કાળો (Black) & 0 & $\times 10^0$ (1) & - \\ \hline
બ્રાઉન (Brown) & 1 & $\times 10^1$ (10) & $\pm 1\%$ \\ \hline
લાલ (Red) & 2 & $\times 10^2$ (100) & $\pm 2\%$ \\ \hline
નારંગી (Orange) & 3 & $\times 10^3$ (1k) & - \\ \hline
પીળો (Yellow) & 4 & $\times 10^4$ (10k) & - \\ \hline
લીલો (Green) & 5 & $\times 10^5$ (100k) & $\pm 0.5\%$ \\ \hline
વાદળી (Blue) & 6 & $\times 10^6$ (1M) & $\pm 0.25\%$ \\ \hline
વાયોલેટ (Violet) & 7 & $\times 10^7$ (10M) & $\pm 0.1\%$ \\ \hline
ગ્રે (Grey) & 8 & $\times 10^8$ & $\pm 0.05\%$ \\ \hline
સફેદ (White) & 9 & $\times 10^9$ & - \\ \hline
સોનેરી (Gold) & - & $\times 0.1$ & $\pm 5\%$ \\ \hline
ચાંદી (Silver) & - & $\times 0.01$ & $\pm 10\%$ \\ \hline
\end{tabulary}
\end{center}

\begin{answerdiagram}{Resistor Color Bands}
\begin{tikzpicture}
    \draw[thick, fill=orange!20] (0,0) rectangle (6,2);
    \draw[thick] (-1,1) -- (0,1);
    \draw[thick] (6,1) -- (7,1);
    
    % Bands
    \draw[fill=red] (1,0) rectangle (1.5,2) node[midway, below=1cm] {Band 1 (Digit 1)};
    \draw[fill=violet] (2,0) rectangle (2.5,2) node[midway, below=2.5cm] {Band 2 (Digit 2)};
    \draw[fill=orange] (3,0) rectangle (3.5,2) node[midway, below=1cm] {Band 3 (Multiplier)};
    \draw[fill=yellow!80!black] (5,0) rectangle (5.5,2) node[midway, below=2.5cm] {Band 4 (Tolerance)};
    
    % Example
    \node at (3, 3) {Example 1: Red-Violet-Orange-Gold};
    \node at (3, 2.5) {$27 \times 10^3 \Omega \pm 5\% = 27k\Omega$};
\end{tikzpicture}
\end{answerdiagram}

\textbf{ઉદાહરણ 1:} લાલ-વાયોલેટ-નારંગી-સોનેરી
\begin{itemize}
    \item 1લી (લાલ) = 2, 2જી (વાયોલેટ) = 7, 3જી (નારંગી) = $\times 1k$, 4થી (સોનેરી) = $\pm 5\%$
    \item મૂલ્ય: $27k\Omega \pm 5\%$
\end{itemize}

\textbf{ઉદાહરણ 2:} બ્રાઉન-બ્લેક-યલો-સિલ્વર
\begin{itemize}
    \item 1લી (બ્રાઉન) = 1, 2જી (બ્લેક) = 0, 3જી (યલો) = $\times 10k$, 4થી (સિલ્વર) = $\pm 10\%$
    \item મૂલ્ય: $100k\Omega \pm 10\%$
\end{itemize}
\end{solutionbox}

\begin{mnemonicbox}
\mnemonic{BBROY: BBROY Great Britain Very Good Wife (Black Brown Red Orange Yellow Green Blue Violet Gray White)}
\end{mnemonicbox}

% Question 1 OR
\questionmarks{1(c) OR}{7}{LDR નું બાંધકામ, કાર્યકારી લાક્ષણિકતાઓ અને એપ્લિકેશન સમજાવો.}

\begin{solutionbox}
\textbf{જવાબ}:

\textbf{લાઈટ ડિપેન્ડન્ટ રેસિસ્ટર (LDR)}

\begin{center}
\captionof{table}{LDR વિગતો}
\begin{tabulary}{\linewidth}{|L|L|}
\hline
\textbf{પાસું} & \textbf{વર્ણન} \\ \hline
\textbf{બાંધકામ} & સેમિકન્ડક્ટર મટીરિયલ (કેડમિયમ સલ્ફાઈડ) સિરામિક સબસ્ટ્રેટ પર ઝિગઝેગ પેટર્નમાં. પારદર્શક કેસમાં પેકેજિંગ. \\ \hline
\textbf{કાર્ય સિદ્ધાંત} & ફોટોકન્ડક્ટિવિટી: જ્યારે પ્રકાશ સામગ્રી પર પડે છે, ત્યારે ફોટોન્સ ઇલેક્ટ્રોન-હોલ જોડીઓ મુક્ત કરે છે, વાહકતા વધારે છે અને અવરોધ ઘટાડે છે. \\ \hline
\textbf{લાક્ષણિકતાઓ} & અંધકારમાં ઉચ્ચ પ્રતિરોધ (M$\Omega$). પ્રકાશમાં ઓછો પ્રતિરોધ (100-5000$\Omega$). વ્યસ્ત નોન-લીનિયર સંબંધ. ધીમો પ્રતિભાવ સમય. \\ \hline
\textbf{ઉપયોગો} & ઓટોમેટિક સ્ટ્રીટ લાઈટ્સ, કેમેરા લાઈટ મીટર, ચોર એલાર્મ, ડિસ્પ્લે બ્રાઈટનેસ કંટ્રોલ. \\ \hline
\end{tabulary}
\end{center}

\begin{answerdiagram}{LDR Characteristics and Symbol}
\begin{tikzpicture}
    \begin{scope}[xshift=0cm]
        \draw[thick] (0,0) rectangle (2,2);
        \draw (0.2, 0.5) -- (0.5, 0.5) -- (0.5, 1.5) -- (0.8, 1.5) -- (0.8, 0.5) -- (1.1, 0.5) -- (1.1, 1.5) -- (1.4, 1.5);
        \node at (1, -0.5) {Construction Pattern};
    \end{scope}

    \begin{scope}[xshift=4cm]
        \draw[thick, ->] (0,2) -- (0,0) -- (3,0);
        \node at (1.5, -0.5) {Light Intensity};
        \node at (-0.5, 1) {Resistance};
        \draw[blue, thick] (0.2, 1.8) to[out=-80, in=170] (2.8, 0.2);
    \end{scope}
    
    \begin{scope}[xshift=8cm, yshift=1cm]
         \draw (0,0) to[R, l=LDR] (2,0);
         \draw[->, thick] (0.5, 1) -- (1, 0.5);
         \draw[->, thick] (1, 1) -- (1.5, 0.5);
         \node at (1, -1) {Symbol};
    \end{scope}
\end{tikzpicture}
\end{answerdiagram}
\end{solutionbox}

\begin{mnemonicbox}
\mnemonic{MOLD: More light On, Less resistance Down}
\end{mnemonicbox}

% Question 2
\questionmarks{2(a)}{3}{સામગ્રીના આધારે રેસિસ્ટરને વર્ગીકૃત કરો.}

\begin{solutionbox}
\textbf{જવાબ}:

\begin{center}
\captionof{table}{રેસિસ્ટર વર્ગીકરણ}
\begin{tabulary}{\linewidth}{|L|L|L|}
\hline
\textbf{મટીરિયલ ટાઈપ} & \textbf{લાક્ષણિકતાઓ} & \textbf{ઉદાહરણો} \\ \hline
\textbf{કાર્બન કોમ્પોઝિશન} & ઓછી કિંમત, નોઈઝી, નબળી ટોલરન્સ. & સામાન્ય હેતુ. \\ \hline
\textbf{કાર્બન ફિલ્મ} & કોમ્પોઝિશન કરતાં વધુ સારી સ્થિરતા. & ઓડિયો, સામાન્ય સર્કિટ. \\ \hline
\textbf{મેટલ ફિલ્મ} & ઉત્તમ સ્થિરતા, ઓછો નોઈઝ. & પ્રિસિઝન સર્કિટ. \\ \hline
\textbf{મેટલ ઓક્સાઈડ} & ઉચ્ચ સ્થિરતા, ગરમી પ્રતિરોધક. & પાવર સપ્લાય. \\ \hline
\textbf{વાયર વાઉન્ડ} & ઉચ્ચ પાવર, ઇન્ડક્ટિવ. & હીટિંગ એલિમેન્ટ. \\ \hline
\textbf{થિક/થિન ફિલ્મ} & નાના કદ (SMD). & સરફેસ માઉન્ટ. \\ \hline
\end{tabulary}
\end{center}
\end{solutionbox}

\begin{mnemonicbox}
\mnemonic{CMMWTF: Carbon Makes Much Wire To Form resistors}
\end{mnemonicbox}

\questionmarks{2(b)}{4}{આપેલ કલર કોડ માટે રેસિસ્ટરની કિંમત ગણો. – (i) બ્રાઉન, બ્લેક, યલો, ગોલ્ડન (ii) યલો, વાયોલેટ, રેડ, સિલ્વર}

\begin{solutionbox}
\textbf{જવાબ}:

\textbf{ભાગ (i): બ્રાઉન, બ્લેક, યલો, ગોલ્ડન}
\begin{itemize}
    \item બ્રાઉન (1), બ્લેક (0), યલો ($\times 10^4$), ગોલ્ડન ($\pm 5\%$)
    \item $10 \times 10,000 = 100,000\Omega = 100k\Omega \pm 5\%$
\end{itemize}

\textbf{ભાગ (ii): યલો, વાયોલેટ, રેડ, સિલ્વર}
\begin{itemize}
    \item યલો (4), વાયોલેટ (7), રેડ ($\times 10^2$), સિલ્વર ($\pm 10\%$)
    \item $47 \times 100 = 4,700\Omega = 4.7k\Omega \pm 10\%$
\end{itemize}
\end{solutionbox}

\questionmarks{2(c)}{7}{ઇલેક્ટ્રોલિટીક કેપેસિટર્સનું બાંધકામ અને સંચાલન સમજાવો.}

\begin{solutionbox}
\textbf{જવાબ}:

\begin{center}
\captionof{table}{ઇલેક્ટ્રોલિટિક કેપેસિટર}
\begin{tabulary}{\linewidth}{|L|L|}
\hline
\textbf{ઘટક} & \textbf{વર્ણન} \\ \hline
\textbf{એનોડ (Anode)} & ઓક્સાઇડ લેયર (ડાયલેક્ટ્રિક) સાથે એલ્યુમિનિયમ ફોઇલ. \\ \hline
\textbf{કેથોડ (Cathode)} & ઇલેક્ટ્રોલાઇટ (પ્રવાહી/પેસ્ટ) અને મેટલ ફોઇલ. \\ \hline
\textbf{સેપરેટર} & ઇલેક્ટ્રોલાઇટમાં પલાળેલું પેપર. \\ \hline
\textbf{કામગીરી} & ઓક્સાઇડ લેયર અત્યંત પાતળા હોવાને કારણે ઉચ્ચ કેપેસિટન્સ ($C \propto A/d$) આપે છે. પોલરાઇઝ્ડ (સાચી +/- જોડાણ જરૂરી). \\ \hline
\end{tabulary}
\end{center}

\begin{answerdiagram}{Electrolytic Capacitor Construction}
\begin{tikzpicture}
    % Capacitor Can (Cylinder)
    \draw[fill=gray!20] (0,2) ellipse (1 and 0.3);
    \draw[fill=gray!20] (0,0) ellipse (1 and 0.3);
    \draw[fill=gray!20] (-1,0) rectangle (1,2);
    \draw[fill=gray!40] (0,2) ellipse (1 and 0.3); % Top cap
    \draw (-1,0) -- (-1,2);
    \draw (1,0) -- (1,2);
    
    % Internal layers roll
    \begin{scope}[xshift=3cm, yshift=1cm]
        \draw[thick] (0,0) -- (4,0);
        \draw[thick, fill=blue!20] (0,0.2) rectangle (4,0.4) node[midway] {Cathode (Foil + Electrolyte)};
        \draw[thick, fill=white] (0,0.4) rectangle (4,0.6) node[midway] {Separator (Paper)};
        \draw[thick, fill=red!20] (0,0.6) rectangle (4,0.8) node[midway] {Anode (Oxide Layer)};
        \draw[->] (2,0.8) -- (2,1.5) node[above] {Rolled into Cylinder};
    \end{scope}
\end{tikzpicture}
\end{answerdiagram}
\end{solutionbox}

\begin{mnemonicbox}
\mnemonic{PAVE: Polarized Aluminum with Very high capacitance and Electrolyte}
\end{mnemonicbox}

% Question 2 OR
\questionmarks{2(a) OR}{3}{રેક્ટિફાયરમાં ફિલ્ટર સર્કિટનું મહત્વ જણાવો.}

\begin{solutionbox}
\textbf{જવાબ}:

\begin{itemize}
    \item \keyword{સ્મૂધિંગ (Smoothing)}: રેક્ટિફાયરના પલ્સેટિંગ DC ને સ્થિર DC માં ફેરેવે છે.
    \item \keyword{રિપલ રિડક્શન (Ripple Reduction)}: અનિચ્છનીય AC ઘટકો (રિપલ્સ) દૂર કરે છે.
    \item \keyword{વોલ્ટેજ સ્ટેબિલાઇઝેશન}: સરેરાશ આઉટપુટ વોલ્ટેજ જાળવી રાખે છે.
    \item \keyword{ડિવાઇસ પ્રોટેક્શન}: સંવેદનશીલ ઇલેક્ટ્રોનિક ઘટકોને નુકસાનથી બચાવે છે.
\end{itemize}
\end{solutionbox}

\begin{mnemonicbox}
\mnemonic{SVRL: Smoothens Voltage by Reducing ripples for Load}
\end{mnemonicbox}

\questionmarks{2(b) OR}{4}{P પ્રકાર સેમિકન્ડક્ટર અને N પ્રકાર સેમિકન્ડક્ટર વચ્ચે તફાવત કરો.}

\begin{solutionbox}
\textbf{જવાબ}:

\begin{center}
\captionof{table}{P-type vs N-type}
\begin{tabulary}{\linewidth}{|L|L|L|}
\hline
\textbf{વિશેષતા} & \textbf{P-type} & \textbf{N-type} \\ \hline
\textbf{ડોપન્ટ} & ત્રિસંયોજક (B, Al, Ga) & પંચસંયોજક (P, As, Sb) \\ \hline
\textbf{મુખ્ય વાહકો} & હોલ્સ (+) & ઇલેક્ટ્રોન્સ (-) \\ \hline
\textbf{ગૌણ વાહકો} & ઇલેક્ટ્રોન્સ (-) & હોલ્સ (+) \\ \hline
\textbf{ઊર્જા સ્તર} & વેલેન્સ બેન્ડ નજીક એક્સેપ્ટર લેવલ & કન્ડક્શન બેન્ડ નજીક ડોનર લેવલ \\ \hline
\end{tabulary}
\end{center}
\end{solutionbox}

\begin{mnemonicbox}
\mnemonic{HELP-NED: Holes Exist Large in P, Negative Electrons Dominate N}
\end{mnemonicbox}

\questionmarks{2(c) OR}{7}{વેવફોર્મ્સ સાથે બ્રિજ રેક્ટિફાયરનું કાર્ય સમજાવો.}

\begin{solutionbox}
\textbf{જવાબ}:

\textbf{કાર્ય}:
\begin{itemize}
    \item \keyword{પોઝિટિવ હાફ}: D1, D3 કન્ડક્ટ કરે છે. લોડ દ્વારા કરંટ વહે છે.
    \item \keyword{નેગેટિવ હાફ}: D2, D4 કન્ડક્ટ કરે છે. લોડ દ્વારા કરંટ સમાન દિશામાં વહે છે.
    \item \keyword{પરિણામ}: સેન્ટર-ટેપ ટ્રાન્સફોર્મર વગર ફુલ વેવ રેક્ટિફિકેશન.
\end{itemize}

\begin{answerdiagram}{Bridge Rectifier Circuit and Waveforms}
\begin{tikzpicture}
    % Circuit
    \begin{scope}[scale=0.7]
        \coordinate (L) at (2, 2);
        \coordinate (R) at (6, 2);
        \coordinate (T) at (4, 4);
        \coordinate (B) at (4, 0);
        
        % Diodes (D1: L->T, D3: R->T, D4: B->L, D2: B->R)
        \draw (L) to[D*, l=D1] (T);
        \draw (R) to[D*, l=D3] (T);
        \draw (B) to[D*, l=D4] (L);
        \draw (B) to[D*, l=D2] (R);
        
        % AC Source
        \draw (0,1) to[sV, l=AC] (0,3);
        \draw (0,3) -- (L);
        \draw (0,1) -- (0,0) -- (6,0) -- (R);
        
        % Load
        \draw (T) -- (4,5) -- (8,5) to[R, l=$R_L$] (8,-1) -- (4,-1) -- (B);
        
        % Nodes
        \draw (L) node[circ]{}; \draw (R) node[circ]{};
        \draw (T) node[circ]{}; \draw (B) node[circ]{};
    \end{scope}

    % Waveforms
    \begin{scope}[xshift=6cm, yshift=0cm]
        \draw[->] (0,1) -- (4,1) node[right] {$t$};
        \draw[->] (0,0) -- (0,2.5) node[above] {$V_{out}$};
        \draw[blue, thick, smooth, samples=100, domain=0:3.5] plot (\x, {1 + abs(sin(\x*180))});
        \node at (2, -0.5) {Full Wave Output};
    \end{scope}
\end{tikzpicture}
\end{answerdiagram}
\end{solutionbox}

\begin{mnemonicbox}
\mnemonic{FBRO: Four diodes, Both cycles, Rectified Output}
\end{mnemonicbox}

% Question 3
\questionmarks{3(a)}{3}{વ્યાખ્યાયિત કરો (1) PIV (2) રિપલ ફેક્ટર.}

\begin{solutionbox}
\textbf{જવાબ}:

\begin{center}
\captionof{table}{PIV અને રિપલ ફેક્ટર}
\begin{tabulary}{\linewidth}{|L|L|}
\hline
\textbf{શબ્દ} & \textbf{વ્યાખ્યા} \\ \hline
\textbf{PIV (પીક ઇન્વર્સ વોલ્ટેજ)} & રિવર્સ બાયસ સ્થિતિમાં ડાયોડ સહન કરી શકે તે મહત્તમ વોલ્ટેજ. ડાયોડ બ્રેકડાઉન અટકાવવા માટે મહત્વની રેટિંગ. \\ \hline
\textbf{રિપલ ફેક્ટર (r)} & રેક્ટિફાયર ફિલ્ટરની અસરકારકતાનું માપ. આઉટપુટમાં AC ઘટકના RMS મૂલ્યથી DC ઘટકના અનુપાત. ઓછો રિપલ ફેક્ટર વધુ સારી ફિલ્ટરિંગ સૂચવે છે. \\ \hline
\end{tabulary}
\end{center}

\textbf{ફોર્મ્યુલા}: $r = \frac{V_{rms(ac)}}{V_{dc}}$
\end{solutionbox}

\begin{mnemonicbox}
\mnemonic{PIR: Peak Inverse voltage Restricts, Ripple indicates Rectification quality}
\end{mnemonicbox}

\questionmarks{3(b)}{4}{PN જંક્શન ડાયોડની VI લાક્ષણિકતાઓ સમજાવો.}

\begin{solutionbox}
\textbf{જવાબ}:

\begin{center}
\captionof{table}{PN જંક્શન લાક્ષણિકતાઓ}
\begin{tabulary}{\linewidth}{|L|L|}
\hline
\textbf{ક્ષેત્ર} & \textbf{વર્તન} \\ \hline
\textbf{ફોરવર્ડ બાયસ} & સરળતાથી કરંટ વહન કરે છે (થ્રેશોલ્ડ 0.7V Si પછી). કરંટમાં એક્સપોનેન્શિયલ વધારો. \\ \hline
\textbf{રિવર્સ બાયસ} & કરંટને અવરોધે છે. ખૂબ નાનો લીકેજ કરંટ ($\mu$A). ઉચ્ચ રિવર્સ વોલ્ટેજ પર બ્રેકડાઉન. \\ \hline
\end{tabulary}
\end{center}

\begin{answerdiagram}{VI Characteristics of PN Diode}
\begin{tikzpicture}
    \begin{axis}[
        axis lines=middle,
        xlabel={$V_D$ (V)}, ylabel={$I_D$ (mA)},
        xmin=-10, xmax=2,
        ymin=-5, ymax=10,
        xtick={0.7}, xticklabels={0.7},
        ytick=\empty,
        width=10cm, height=6cm,
        grid=major,
        every axis x label/.style={at={(current axis.right of origin)},anchor=west},
        every axis y label/.style={at={(current axis.above origin)},anchor=south},
    ]
        % Forward
        \addplot[blue, thick, domain=0:1.5, samples=100] {0.01*(exp(4*x)-1)};
        \node at (axis cs: 1, 4) {Forward Bias};

        % Reverse
        \addplot[red, thick, domain=-8:0] {-0.1}; % Leakage
        \addplot[red, thick, domain=-10:-8] {-50*(x+8)-0.1}; % Breakdown
        \node at (axis cs: -5, -1) {Reverse Leakage};
        \node at (axis cs: -9, -4) {Breakdown};
    \end{axis}
\end{tikzpicture}
\end{answerdiagram}
\end{solutionbox}

\begin{mnemonicbox}
\mnemonic{FBRL: Forward Bias Resists Little, reverse blocks lots}
\end{mnemonicbox}

\questionmarks{3(c)}{7}{તરંગ સ્વરૂપો સાથે કેપેસિટર ઇનપુટ અને ચોક ઇનપુટ ફિલ્ટરની કામગીરી સમજાવો.}

\begin{solutionbox}
\textbf{જવાબ}:

\textbf{1. કેપેસિટર ઇનપુટ ફિલ્ટર}
\begin{itemize}
    \item કેપેસિટર લોડ રેસિસ્ટન્સ સાથે પેરેલલમાં જોડાયેલ છે.
    \item વોલ્ટેજના શિખર દરમિયાન ચાર્જ થાય છે, ડિપ દરમિયાન ધીમેથી ડિસ્ચાર્જ થાય છે.
    \item ઉચ્ચ DC વોલ્ટેજ, પરંતુ નબળું રેગ્યુલેશન.
\end{itemize}

\textbf{2. ચોક ઇનપુટ ફિલ્ટર}
\begin{itemize}
    \item ઇન્ડક્ટર (ચોક) શ્રેણીમાં અને કેપેસિટર પેરેલલમાં.
    \item ઇન્ડક્ટર કરંટ પરિવર્તનનો વિરોધ કરે છે, પ્રવાહને સ્મૂધ કરે છે.
    \item વધુ સારું રેગ્યુલેશન, ઓછું DC વોલ્ટેજ.
\end{itemize}

\begin{answerdiagram}{Filter Circuits and Waveforms}
\begin{tikzpicture}
    % Capacitor Filter
    \begin{scope}[xshift=0cm]
        \draw (0,0) to[sV, l=AC] (0,2) -- (1,2) to[D] (2,2) -- (4,2);
        \draw (2,2) to[C, l=C] (2,0);
        \draw (4,2) to[R, l=$R_L$] (4,0) -- (0,0);
        \node at (2, -0.5) {Capacitor Input Filter};
        
        % Waveform
        \draw[thick, ->] (0,-2.5) -- (3,-2.5) node[right] {t};
        \draw[thick, ->] (0,-2.5) -- (0,-1);
        \draw[blue] (0,-2.5) sin (0.25,-1) cos (0.5,-2.5) sin (0.75,-1) cos (1,-2.5);
        \draw[red, thick] (0.25,-1) -- (0.75,-1.5) -- (1.25,-1);
        \node at (1.5, -2) {Ripple};
    \end{scope}

    % Choke Filter
    \begin{scope}[xshift=6cm]
        \draw (0,0) to[sV, l=AC] (0,2) -- (1,2) to[D] (2,2) to[L, l=L] (4,2) -- (5,2);
        \draw (4,2) to[C, l=C] (4,0);
        \draw (5,2) to[R, l=$R_L$] (5,0) -- (0,0);
        \node at (2.5, -0.5) {Choke Input Filter};
        
        % Waveform
        \draw[thick, ->] (0,-2.5) -- (3,-2.5) node[right] {t};
        \draw[dashed] (0,-1.5) -- (3,-1.5);
        \draw[red, thick, smooth] plot[domain=0:3, samples=50] (\x, {-1.5 + 0.2*sin(deg(\x*5))});
        \node at (1.5, -2) {Smoother Output};
    \end{scope}
\end{tikzpicture}
\end{answerdiagram}
\end{solutionbox}

\begin{mnemonicbox}
\mnemonic{VOICE: Voltage Output Is Constant with Either filter, but choke gives better regulation}
\end{mnemonicbox}

% Question 3 OR
\questionmarks{3(a) OR}{3}{ઝેનર ડાયોડનું કાર્ય અને મહત્વ જણાવો.}

\begin{solutionbox}
\textbf{જવાબ}:

\begin{center}
\captionof{table}{ઝેનર ડાયોડ કાર્યો}
\begin{tabulary}{\linewidth}{|L|L|}
\hline
\textbf{કાર્ય} & \textbf{વર્ણન} \\ \hline
\textbf{વોલ્ટેજ રેગ્યુલેશન} & સ્થિર આઉટપુટ વોલ્ટેજ જાળવે છે. \\ \hline
\textbf{વોલ્ટેજ રેફરન્સ} & ચોક્કસ રેફરન્સ વોલ્ટેજ પ્રદાન કરે છે. \\ \hline
\textbf{પ્રોટેક્શન} & વોલ્ટેજ સ્પાઇક્સથી નુકસાન અટકાવે છે. \\ \hline
\textbf{ઉપયોગ} & બ્રેકડાઉન ક્ષેત્રમાં કાર્ય કરે છે. \\ \hline
\end{tabulary}
\end{center}
\end{solutionbox}

\begin{mnemonicbox}
\mnemonic{VPRVW: Voltage Protection, Regulation, and Voltage Waveform control}
\end{mnemonicbox}

\questionmarks{3(b) OR}{4}{પ્રકાશ ઉત્સર્જક ડાયોડ (LED) ને તેની લાક્ષણિકતા સાથે વર્ણવો.}

\begin{solutionbox}
\textbf{જવાબ}:

\begin{center}
\captionof{table}{LED લાક્ષણિકતાઓ}
\begin{tabulary}{\linewidth}{|L|L|}
\hline
\textbf{પાસું} & \textbf{વર્ણન} \\ \hline
\textbf{સિદ્ધાંત} & ઇલેક્ટ્રોલ્યુમિનિસન્સ. ઇલેક્ટ્રોન-હોલ રિકોમ્બિનેશનથી પ્રકાશ ઉત્સર્જન. \\ \hline
\textbf{મટીરિયલ} & ડાયરેક્ટ બેન્ડગેપ સેમિકન્ડક્ટર (GaAs, GaP). \\ \hline
\textbf{વોલ્ટેજ} & લાલ: ~2V, વાદળી/સફેદ: ~3V. \\ \hline
\textbf{ઓપરેશન} & માત્ર ફોરવર્ડ બાયસમાં. રિવર્સ બાયસ (>5V) થી નુકસાન. \\ \hline
\end{tabulary}
\end{center}

\begin{answerdiagram}{LED Working}
\begin{tikzpicture}
    \draw (0,0) to[D*, l=LED, fill=red] (2,0);
    \draw[->, thick, decorate, decoration={snake}] (1,0.5) -- (1.5, 1);
    \draw[->, thick, decorate, decoration={snake}] (1.2,0.5) -- (1.7, 1);
    \node at (1, -0.5) {Emits Light};
\end{tikzpicture}
\end{answerdiagram}
\end{solutionbox}

\begin{mnemonicbox}
\mnemonic{CRAVE: Current Regulated And Voltage Emits light}
\end{mnemonicbox}

\questionmarks{3(c) OR}{7}{કેપેસિટર ઇનપુટ અને ચોક ઇનપુટ ફિલ્ટરનું કાર્ય સમજાવો.}

\begin{solutionbox}
\textbf{જવાબ}:

\textit{(વિગતવાર વેવફોર્મ્સ અને ડાયાગ્રામ માટે પ્રશ્ન 3(c) જુઓ. આ વિભાગ ઘટક વિશ્લેષણ પ્રદાન કરે છે.)}

\begin{center}
\captionof{table}{કેપેસિટર vs ચોક ફિલ્ટર}
\begin{tabulary}{\linewidth}{|L|L|L|}
\hline
\textbf{પેરામીટર} & \textbf{કેપેસિટર ઇનપુટ} & \textbf{ચોક ઇનપુટ} \\ \hline
\textbf{ઘટકો} & પેરેલલ કેપેસિટર. & ચોક (શ્રેણી) + કેપ (પેરેલલ). \\ \hline
\textbf{આઉટપુટ V} & ઉચ્ચ ($\approx V_m$). & નીચું ($\approx 0.9 V_m$). \\ \hline
\textbf{રેગ્યુલેશન} & નબળું (લોડ સાથે V ઘટે છે). & સારું (L ફેરફારનો વિરોધ કરે છે). \\ \hline
\textbf{ડાયોડ કરંટ} & ઉચ્ચ પીક સર્જ. & સતત, નીચા પીક. \\ \hline
\textbf{કિંમત/કદ} & ઓછી કિંમત, નાનું. & ભારે, મોટું, ખર્ચાળ. \\ \hline
\end{tabulary}
\end{center}
\end{solutionbox}

\begin{mnemonicbox}
\mnemonic{CHEER: Capacitor Holds Energy, inductor Ensures Regulated current}
\end{mnemonicbox}

% Question 4
\questionmarks{4(a)}{3}{PN જંક્શન ડાયોડની લાક્ષણિકતાઓની ચર્ચા કરો.}

\begin{solutionbox}
\textbf{જવાબ}:

\begin{itemize}
    \item \keyword{ફોરવર્ડ બાયસ}: ઓછો પ્રતિરોધ, ની વોલ્ટેજ પછી કરંટ વહે છે.
    \item \keyword{રિવર્સ બાયસ}: ઉચ્ચ પ્રતિરોધ, માત્ર લીકેજ કરંટ.
    \item \keyword{બ્રેકડાઉન}: ઝેનર/એવરેન્ચ વોલ્ટેજ પર કરંટમાં ઝડપી વધારો.
    \item \keyword{તાપમાન અસર}: ગરમી સાથે $V_f$ ઘટે છે, દર $10^\circ$C પર $I_r$ બમણો થાય છે.
\end{itemize}
\end{solutionbox}

\begin{mnemonicbox}
\mnemonic{FRBCT: Forward conducts, Reverse blocks, Breakdown destroys}
\end{mnemonicbox}

\questionmarks{4(b)}{4}{પી-એન જંક્શન ડાયોડ અને ઝેનર ડાયોડ વચ્ચે સરખામણી કરો.}

\begin{solutionbox}
\textbf{જવાબ}:

\begin{center}
\captionof{table}{સામાન્ય ડાયોડ vs ઝેનર ડાયોડ}
\begin{tabulary}{\linewidth}{|L|L|L|}
\hline
\textbf{વિશેષતા} & \textbf{PN ડાયોડ} & \textbf{ઝેનર ડાયોડ} \\ \hline
\textbf{સિમ્બોલ} & સામાન્ય એરો & 'Z' છેડા સાથે એરો \\ \hline
\textbf{ડોપિંગ} & મધ્યમ & ભારે \\ \hline
\textbf{બ્રેકડાઉન} & વિનાશક & બિન-વિનાશક (કાર્યકારી ક્ષેત્ર) \\ \hline
\textbf{મુખ્ય ઉપયોગ} & રેક્ટિફિકેશન & વોલ્ટેજ રેગ્યુલેશન \\ \hline
\end{tabulary}
\end{center}
\end{solutionbox}

\begin{mnemonicbox}
\mnemonic{FORBAR: Forward Operation is Regular, Breakdown Application is Real difference}
\end{mnemonicbox}

\questionmarks{4(c)}{7}{વોલ્ટેજ રેગ્યુલેટર તરીકે ઝેનર ડાયોડનું કાર્ય સમજાવો.}

\begin{solutionbox}
\textbf{જવાબ}:

\textbf{સર્કિટ ઓપરેશન}:
\begin{itemize}
    \item ઝેનર ડાયોડ \keyword{રિવર્સ બાયસ} માં જોડાયેલ છે.
    \item જ્યારે $V_{in} > V_z$, ઝેનર કન્ડક્ટ કરે છે અને $V_{out} = V_z$ જાળવી રાખે છે.
    \item સીરીઝ રેસિસ્ટર $R_s$ વધારાના વોલ્ટેજ ($V_{in} - V_z$) ને ડ્રોપ કરે છે.
    \item લોડ કરંટ અથવા ઇનપુટ વોલ્ટેજમાં ફેરફાર ઝેનર કરંટ બદલીને સરભર કરવામાં આવે છે.
\end{itemize}

\begin{answerdiagram}{Zener Regulator}
\begin{tikzpicture}
    \draw (0,0) to[V, l=$V_{in}$] (0,3) -- (2,3) to[R, l=$R_s$] (4,3) -- (6,3);
    \draw (4,3) to[zD, l=$V_z$, i=$I_z$] (4,0);
    \draw (6,3) to[R, l=$R_L$, v=$V_{out}$] (6,0) -- (0,0);
    \draw (4,0) node[ground]{};
    
    % Annotations
    \node[right] at (6.5, 2) {$V_{out} = V_z$};
    \node[right] at (6.5, 1) {(Steady)};
\end{tikzpicture}
\end{answerdiagram}
\end{solutionbox}

\begin{mnemonicbox}
\mnemonic{VISOR: Voltage In Stays Out Regulated}
\end{mnemonicbox}

% Question 4 OR
\questionmarks{4(a) OR}{3}{ટ્રાન્ઝિસ્ટરની ટૂંકમાં ચર્ચા કરો.}

\begin{solutionbox}
\textbf{જવાબ}:

\begin{itemize}
    \item \keyword{વ્યાખ્યા}: 3-ટર્મિનલ સેમિકન્ડક્ટર ડિવાઇસ (એમિટર, બેઝ, કલેક્ટર).
    \item \keyword{પ્રકારો}: BJT (NPN, PNP), FET (JFET, MOSFET).
    \item \keyword{કાર્ય}: નબળા સિગ્નલને એમ્પ્લિફાય કરે છે, સ્વિચ તરીકે કાર્ય કરે છે.
    \item \keyword{નિયંત્રણ}: કરંટ કંટ્રોલ્ડ (BJT) અથવા વોલ્ટેજ કંટ્રોલ્ડ (FET).
\end{itemize}
\end{solutionbox}

\begin{mnemonicbox}
\mnemonic{TAWAI: Transistors Amplify, Work As switches, and are Integral}
\end{mnemonicbox}

\questionmarks{4(b) OR}{4}{ટ્રાન્ઝિસ્ટર એમ્પલીફાયર માટે $\alpha$ અને $\beta$ વચ્ચેનો સંબંધ મેળવો.}

\begin{solutionbox}
\textbf{જવાબ}:

\textbf{વ્યાખ્યાઓ}:
\begin{itemize}
    \item $\alpha = \frac{I_C}{I_E}$ (કોમન બેઝ કરંટ ગેઇન)
    \item $\beta = \frac{I_C}{I_B}$ (કોમન એમિટર કરંટ ગેઇન)
\end{itemize}

\textbf{ડેરિવેશન}:
\begin{enumerate}
    \item મૂળભૂત સમીકરણ: $I_E = I_B + I_C$
    \item $I_C$ વડે ભાગો: $\frac{I_E}{I_C} = \frac{I_B}{I_C} + 1$
    \item વ્યાખ્યાઓ મૂકો: $\frac{1}{\alpha} = \frac{1}{\beta} + 1$
    \item પુનર્રચના: $\frac{1}{\alpha} = \frac{1 + \beta}{\beta}$
    \item તેથી: \keyword{$\alpha = \frac{\beta}{1 + \beta}$}
    \item $\beta$ માટે ઉકેલો: \keyword{$\beta = \frac{\alpha}{1 - \alpha}$}
\end{enumerate}

\textbf{ઉદાહરણ}: જો $\alpha = 0.99$, $\beta = \frac{0.99}{1-0.99} = 99$.
\end{solutionbox}

\begin{mnemonicbox}
\mnemonic{ABR: Alpha and Beta are Related}
\end{mnemonicbox}

\questionmarks{4(c) OR}{7}{NPN અને PNP ટ્રાન્ઝિસ્ટરનું બાંધકામ વિગતવાર સમજાવો.}

\begin{solutionbox}
\textbf{જવાબ}:

\begin{center}
\captionof{table}{NPN vs PNP બાંધકામ}
\begin{tabulary}{\linewidth}{|L|L|L|}
\hline
\textbf{પાસું} & \textbf{NPN} & \textbf{PNP} \\ \hline
\textbf{લેયર્સ} & N-P-N & P-N-P \\ \hline
\textbf{મુખ્ય વાહકો} & ઇલેક્ટ્રોન્સ & હોલ્સ \\ \hline
\textbf{ડોપિંગ} & એમિટર (ભારે), બેઝ (હળવા), કલેક્ટર (મધ્યમ) & સમાન \\ \hline
\textbf{પહોળાઈ} & રિકોમ્બિનેશન ઘટાડવા બેઝ ખૂબ પાતળો ($<10\mu m$) & સમાન \\ \hline
\end{tabulary}
\end{center}

\begin{answerdiagram}{Transistor Construction}
\begin{tikzpicture}
    % NPN
    \begin{scope}[xshift=0cm]
        \draw[thick, fill=blue!10] (0,0) rectangle (1.5, 2) node[midway] {E (N)};
        \draw[thick, fill=red!10] (1.5,0) rectangle (2.5, 2) node[midway] {B (P)};
        \draw[thick, fill=blue!10] (2.5,0) rectangle (5, 2) node[midway] {C (N)};
        \draw[->] (0.75, 2) -- (0.75, 2.5) node[above] {Emitter};
        \draw[->] (2, 2) -- (2, 2.5) node[above] {Base};
        \draw[->] (3.75, 2) -- (3.75, 2.5) node[above] {Collector};
        \node at (2.5, -0.5) {NPN Construction};
    \end{scope}

    % PNP
    \begin{scope}[xshift=6cm]
        \draw[thick, fill=red!10] (0,0) rectangle (1.5, 2) node[midway] {E (P)};
        \draw[thick, fill=blue!10] (1.5,0) rectangle (2.5, 2) node[midway] {B (N)};
        \draw[thick, fill=red!10] (2.5,0) rectangle (5, 2) node[midway] {C (P)};
        \draw[->] (0.75, 2) -- (0.75, 2.5) node[above] {Emitter};
        \draw[->] (2, 2) -- (2, 2.5) node[above] {Base};
        \draw[->] (3.75, 2) -- (3.75, 2.5) node[above] {Collector};
        \node at (2.5, -0.5) {PNP Construction};
    \end{scope}
\end{tikzpicture}
\end{answerdiagram}
\end{solutionbox}

\begin{mnemonicbox}
\mnemonic{ENB-CPM: Emitter has N in NPN, Collector is Proportionally Medium-doped}
\end{mnemonicbox}

% Question 5
\questionmarks{5(a)}{3}{ટૂંકમાં ઈ-વેસ્ટ સમજાવો.}

\begin{solutionbox}
\textbf{જવાબ}:

\textbf{ઈ-વેસ્ટ (ઇલેક્ટ્રોનિક વેસ્ટ)}: ફેંકી દીધેલા ઇલેક્ટ્રોનિક ઉપકરણો.

\begin{itemize}
    \item \keyword{જોખમો}: ઝેરી લેડ, મર્ક્યુરી, કેડમિયમ ધરાવે છે.
    \item \keyword{મૂલ્ય}: પુનઃપ્રાપ્ત કરી શકાય તેવું સોનું, ચાંદી, તાંબું ધરાવે છે.
    \item \keyword{અસર}: જો લેન્ડફિલમાં જાય તો પર્યાવરણીય પ્રદૂષણ.
    \item \keyword{જરૂરિયાત}: યોગ્ય રિસાયક્લિંગ અને નિકાલ વ્યવસ્થાપન.
\end{itemize}
\end{solutionbox}

\begin{mnemonicbox}
\mnemonic{TECH: Toxic Electronics Create Hazards}
\end{mnemonicbox}

\questionmarks{5(b)}{4}{આકૃતિ સાથે NPN ટ્રાન્ઝિસ્ટરની કામગીરી સમજાવો.}

\begin{solutionbox}
\textbf{જવાબ}:

\textbf{કાર્ય સિદ્ધાંત}:
\begin{itemize}
    \item \keyword{ફોરવર્ડ બાયસ્ડ} બેઝ-એમિટર: એમિટરથી બેઝમાં ઇલેક્ટ્રોન્સ ઇન્જેક્ટ થાય છે.
    \item \keyword{રિવર્સ બાયસ્ડ} બેઝ-કલેક્ટર: ઇલેક્ટ્રોન્સ બેઝથી કલેક્ટરમાં સ્વીપ થાય છે.
    \item નાનો બેઝ કરંટ ($I_B$) મોટા કલેક્ટર કરંટ ($I_C$) ને નિયંત્રિત કરે છે.
    \item સમીકરણ: $I_E = I_B + I_C$.
\end{itemize}

\begin{answerdiagram}{NPN Operation}
\begin{tikzpicture}
    % Block
    \draw[thick, fill=blue!10] (0,0) rectangle (1,2) node[midway] {N};
    \draw[thick, fill=red!10] (1,0) rectangle (2,2) node[midway] {P};
    \draw[thick, fill=blue!10] (2,0) rectangle (4,2) node[midway] {N};
    
    % Biasing batteries
    \draw (0,1) -- (-1,1) -- (-1, -1) to[battery1, l=$V_{EE}$] (1.5, -1) -- (1.5, 0);
    \draw (4,1) -- (5,1) -- (5, -1) to[battery1, l=$V_{CC}$] (1.5, -1);
    
    % Electron flow
    \draw[->, dashed, thick] (0.5, 1.5) -- (3.5, 1.5) node[midway, above] {Electrons};
    
    \node at (0.5, 2.2) {Emitter};
    \node at (1.5, 2.2) {Base};
    \node at (3, 2.2) {Collector};
\end{tikzpicture}
\end{answerdiagram}
\end{solutionbox}

\begin{mnemonicbox}
\mnemonic{BECAN: Base current Enables Collector Amplification in NPN}
\end{mnemonicbox}

\questionmarks{5(c)}{7}{ઇનપુટ અને આઉટપુટ લાક્ષણિકતાઓ સાથે ટ્રાન્ઝિસ્ટરનું કોમન એમિટર (CE) રૂપરેખાંકન સમજાવો.}

\begin{solutionbox}
\textbf{જવાબ}:

\textbf{CE કોન્ફિગરેશન}: એમિટર ગ્રાઉન્ડેડ (કોમન) છે. બેઝ પર ઇનપુટ, કલેક્ટર પર આઉટપુટ. ઉચ્ચ ગેઇન.

\begin{answerdiagram}{CE Circuit and Characteristics}
\begin{tikzpicture}
    % Circuit
    \begin{scope}[scale=0.8]
        \draw (0,0) node[npn](T){};
        \draw (T.E) node[ground]{};
        \draw (T.B) to[R, l=$R_B$] (-2,0) to[V, l=$V_{BB}$] (-2,-2) node[ground]{};
        \draw (T.C) to[R, l=$R_C$] (2,0.77) to[V, l=$V_{CC}$] (2,-2) node[ground]{};
        \node at (0, -2.5) {CE Circuit};
    \end{scope}

    % Output Characteristics
    \begin{scope}[xshift=4cm, yshift=-1.5cm, scale=0.6]
        \draw[->] (0,0) -- (6,0) node[right] {$V_{CE}$};
        \draw[->] (0,0) -- (0,5) node[above] {$I_C$};
        
        \draw[thick] (0,0) -- (0.5, 4) -- (5.5, 4.2) node[right] {$I_B=40\mu A$};
        \draw[thick] (0,0) -- (0.5, 3) -- (5.5, 3.2) node[right] {$I_B=30\mu A$};
        \draw[thick] (0,0) -- (0.5, 2) -- (5.5, 2.2) node[right] {$I_B=20\mu A$};
        \draw[thick] (0,0) -- (0.5, 1) -- (5.5, 1.2) node[right] {$I_B=10\mu A$};
        
        \node at (3, -1) {Output Characteristics};
    \end{scope}
\end{tikzpicture}
\end{answerdiagram}

\begin{itemize}
    \item \keyword{ઇનપુટ ચાર}: $I_B$ vs $V_{BE}$. ડાયોડ જેવું.
    \item \keyword{આઉટપુટ ચાર}: $I_C$ vs $V_{CE}$. સેચુરેશન, એક્ટિવ, કટઓફ ક્ષેત્રો.
\end{itemize}
\end{solutionbox}

\begin{mnemonicbox}
\mnemonic{CASIO: Common emitter Amplifies Signals with Inverted Output}
\end{mnemonicbox}

% Question 5 OR
\questionmarks{5(a) OR}{3}{ઈ-કચરાના પ્રકારો જણાવો.}

\begin{solutionbox}
\textbf{જવાબ}:

\begin{itemize}
    \item \keyword{IT \& ટેલિકોમ}: કોમ્પ્યુટર, ફોન, પ્રિન્ટર.
    \item \keyword{કન્ઝ્યુમર}: ટીવી, ઓડિયો સેટ, કેમેરા.
    \item \keyword{એપ્લાયન્સિસ}: ફ્રિજ, વોશિંગ મશીન.
    \item \keyword{લાઇટિંગ}: બલ્બ, LEDs.
    \item \keyword{મેડિકલ}: સ્કેનર, મોનિટર.
\end{itemize}
\end{solutionbox}

\begin{mnemonicbox}
\mnemonic{CLIMATE: Computing, Lighting, Industrial, Medical, Appliances, Telecom, Electronic components}
\end{mnemonicbox}

\questionmarks{5(b) OR}{4}{ઇલેક્ટ્રોનિક્સ વેસ્ટની વિવિધ શ્રેણીઓનું વર્ણન કરો.}

\begin{solutionbox}
\textbf{જવાબ}:

\begin{center}
\captionof{table}{E-Waste શ્રેણીઓ}
\begin{tabulary}{\linewidth}{|L|L|}
\hline
\textbf{શ્રેણી} & \textbf{ઉદાહરણો} \\ \hline
મોટા ઉપકરણો & વોશિંગ મશીન, AC \\ \hline
નાના ઉપકરણો & ટોસ્ટર, ઇસ્ત્રી \\ \hline
IT ઇક્વિપમેન્ટ & PC, લેપટોપ, મોબાઈલ \\ \hline
કન્ઝ્યુમર ઇલેક્ટ્રોનિક્સ & TV, સ્ટીરિયો \\ \hline
લાઇટિંગ & ટ્યુબલાઇટ \\ \hline
ટૂલ્સ & ડ્રિલ, આરી \\ \hline
\end{tabulary}
\end{center}

\begin{answerdiagram}{E-Waste Composition}
\begin{tikzpicture}
    \pie[text=legend, radius=2, color={blue!20, red!20, green!20, orange!20, yellow!20}]{
        30/Large Appliances,
        25/IT \& Telecom,
        15/Consumer,
        15/Small Appliances,
        15/Others
    }
\end{tikzpicture}
\end{answerdiagram}
\end{solutionbox}

\begin{mnemonicbox}
\mnemonic{LIMCEST: Large, IT, Medical, Consumer, Electronic tools, Small, Telecom}
\end{mnemonicbox}

\questionmarks{5(c) OR}{7}{ટ્રાન્ઝિસ્ટરને કટઓફ અને સંતૃત્તિ પ્રદેશમાં સ્વિચ તરીકે સમજાવો.}

\begin{solutionbox}
\textbf{જવાબ}:

\textbf{ટ્રાન્ઝિસ્ટર સ્વિચ સ્થિતિઓ}:
\begin{center}
\begin{tabulary}{\linewidth}{|L|L|L|}
\hline
\textbf{સ્થિતિ} & \textbf{પ્રદેશ} & \textbf{સ્થિતિઓ} \\ \hline
\textbf{OFF (ઓપન)} & \keyword{કટઓફ} & $V_{in} < 0.7V$, $I_B=0$, $I_C=0$, $V_{CE}=V_{CC}$. \\ \hline
\textbf{ON (ક્લોઝ્ડ)} & \keyword{સેચુરેશન} & $V_{in} > 0.7V$, $I_B$ max, $I_C$ max, $V_{CE} \approx 0.2V$. \\ \hline
\end{tabulary}
\end{center}

\begin{answerdiagram}{Transistor Switching}
\begin{tikzpicture}
    % Circuit
    \begin{scope}
        \draw (0,0) node[npn](T){};
        \draw (T.E) node[ground]{};
        \draw (T.B) to[R, l=$R_B$] (-2,0) to[sqV, l=Pulse] (-2,-2) node[ground]{};
        \draw (T.C) to[R, l=$R_C$] (2,1) to[short, -o] (2,1.5) node[right]{$V_{CC}$};
        \draw (T.C) to[short, -o] (1,1) node[right]{$V_{out}$};
        
        \node at (0, -2.5) {Switch Circuit};
    \end{scope}

    % Graph
    \begin{scope}[xshift=4cm, yshift=-1cm]
        \draw[->] (0,0) -- (3,0) node[right]{$V_{in}$};
        \draw[->] (0,0) -- (0,3) node[above]{$I_C$};
        \draw[thick] (0,0) -- (1,0) -- (2,2) -- (3,2);
        \node at (0.5, 0.5) {Cutoff};
        \node at (2.5, 2.5) {Saturation};
    \end{scope}
\end{tikzpicture}
\end{answerdiagram}
\end{solutionbox}

\begin{mnemonicbox}
\mnemonic{COSVL: Cutoff means Off State with Vce Large}
\end{mnemonicbox}

\end{document}
