\documentclass[10pt,a4paper]{article}

% content/resources/templates/preamble.tex
\usepackage[margin=0.6in]{geometry}
\author{Milav Dabgar}
\usepackage{amsmath,amssymb,amsthm}
\usepackage{booktabs}
\usepackage{multirow}
\usepackage{xcolor}
\usepackage{tcolorbox}
\tcbuselibrary{breakable,skins}
\usepackage[colorlinks=true,linkcolor=blue]{hyperref}
\usepackage{titlesec}
\usepackage{enumitem}
\usepackage{tikz}
\usepackage{pgfplots}
\usepackage{circuitikz}
\usepackage[version=4]{mhchem}
\usepackage{longtable}
\usepackage{array}
\usepackage{float}
\usepackage{caption}
\usepackage{listings}

\lstset{
  basicstyle=\small\ttfamily,
  breaklines=true,
  breakatwhitespace=false,
  postbreak=\mbox{\textcolor{red}{$\hookrightarrow$}\space},
  float=false,
  numbers=left,
  numberstyle=\tiny\color{gray},
  numbersep=10pt,
  xleftmargin=2em,
  keywordstyle=\color{blue},
  commentstyle=\color{green!60!black},
  stringstyle=\color{purple},
  backgroundcolor=\color{gray!5},
  showstringspaces=false,
  tabsize=2,
  captionpos=b,
  keepspaces=true,
  columns=flexible
}

\pgfplotsset{compat=1.18}
\usetikzlibrary{shapes,arrows,positioning,calc,patterns,decorations.pathmorphing,decorations.markings,arrows.meta}

% Color scheme
\definecolor{headcolor}{RGB}{0,102,204}
\definecolor{keycolor}{RGB}{220,20,60}
\definecolor{solutioncolor}{RGB}{34,139,34}
\definecolor{mnemoniccolor}{RGB}{148,0,211}
\definecolor{codecolor}{RGB}{0,0,100}

% Spacing
\setlength{\parskip}{3pt}
\setlist[itemize]{nosep}
\setlist[enumerate]{nosep}

% Title formatting
\titleformat{\section}{\Large\bfseries\color{headcolor}}{\thesection}{1em}{}
\titleformat{\subsection}{\large\bfseries\color{headcolor}}{\thesubsection}{1em}{}

% Pandoc tightlist compatibility
\providecommand{\tightlist}{%
  \setlength{\itemsep}{0pt}\setlength{\parskip}{0pt}}

% Pandoc longtable compatibility
\newcounter{none}
\def\thenone{}


% content/resources/templates/gujarati-boxes.tex
\usepackage{fontspec}
\usepackage{polyglossia}

% Set Gujarati as main language (document is primarily in Gujarati)
% Note: gloss-gujarati.ldf doesn't exist in polyglossia, but it will use hyphenation patterns
\setdefaultlanguage{gujarati}
\setotherlanguage{english}

% Configure Gujarati font properly
% Use Language=Default to prevent polyglossia from trying to add language-specific features
% that don't exist for Gujarati, which causes "empty feature" warnings
\newfontfamily\gujaratifont[Script=Gujarati,AutoFakeBold=2.5,AutoFakeSlant=0.3]{Noto Sans Gujarati}
\setmainfont[Script=Gujarati,AutoFakeBold=2.5,AutoFakeSlant=0.3]{Noto Sans Gujarati}
% Use Noto Sans Gujarati for monospace to support Gujarati in text
\setmonofont[Scale=0.9]{Noto Sans Gujarati}

% Configure English to use the same font
\newfontfamily\englishfont[Script=Gujarati,AutoFakeBold=2.5,AutoFakeSlant=0.3]{Noto Sans Gujarati}

% Translations for polyglossia
\gappto\captionsgujarati{
  \renewcommand{\tablename}{કોષ્ટક}
  \renewcommand{\figurename}{આકૃતિ}
}

% Helper for TikZ nodes to ensure Gujarati font
\newcommand{\gu}[1]{{\gujaratifont #1}}

% Custom environments
\newtcolorbox{solutionbox}{
    breakable,
    enhanced,
    colback=solutioncolor!5!white,
    colframe=solutioncolor!75!black,
    fonttitle=\bfseries,
    title=જવાબ
}

\newtcolorbox{solutionboxnobreak}{
 colback=solutioncolor!5!white,
 colframe=solutioncolor!75!black,
 fonttitle=\bfseries,
 title=જવાબ
}

\newtcolorbox{keyformula}{
 breakable,
 enhanced,
 colback=keycolor!5!white,
 colframe=keycolor!75!black,
 fonttitle=\bfseries,
 title=રાસાયણિક સમીકરણ/સૂત્ર
}

\newtcolorbox{mnemonicbox}{
 breakable,
 enhanced,
 colback=mnemoniccolor!5!white,
 colframe=mnemoniccolor!75!black,
 fonttitle=\bfseries,
 title=મેમરી ટ્રીક
}


\begin{document}

\begin{center}
{\Huge\bfseries\color{headcolor} Subject Name (Gujarati)}\\[5pt]
{\LARGE 4311102 -- Winter 2024}\\[3pt]
{\large Semester 1 Study Material}\\[3pt]
{\normalsize\textit{Detailed Solutions and Explanations}}
\end{center}

\vspace{10pt}

\subsection*{પ્રશ્ન 1(અ) [3
ગુણ]}\label{uxaaauxab0uxab6uxaa8-1uxa85-3-uxa97uxaa3}

\textbf{એક્ટિવ અને પેસિવ કમ્પોનન્ટ વચ્ચેનો તફાવત આપો.}

\begin{solutionbox}

{\def\LTcaptype{none} % do not increment counter
\begin{longtable}[]{@{}
  >{\raggedright\arraybackslash}p{(\linewidth - 2\tabcolsep) * \real{0.5000}}
  >{\raggedright\arraybackslash}p{(\linewidth - 2\tabcolsep) * \real{0.5000}}@{}}
\toprule\noalign{}
\begin{minipage}[b]{\linewidth}\raggedright
\textbf{પેસિવ કમ્પોનન્ટ}
\end{minipage} & \begin{minipage}[b]{\linewidth}\raggedright
\textbf{એક્ટિવ કમ્પોનન્ટ}
\end{minipage} \\
\midrule\noalign{}
\endhead
\bottomrule\noalign{}
\endlastfoot
બાહ્ય પાવર સ્ત્રોતની જરૂર પડતી નથી & કાર્ય કરવા માટે બાહ્ય પાવર સ્ત્રોતની જરૂર પડે
છે \\
સિગ્નલને એમ્પલિફાઈ કે પ્રોસેસ કરી શકતા નથી & સિગ્નલને એમ્પલિફાઈ, સ્વિચ કે પ્રોસેસ કરી
શકે છે \\
ઉદાહરણ: રેઝિસ્ટર, કેપેસિટર, ઇન્ડક્ટર & ઉદાહરણ: ટ્રાન્ઝિસ્ટર, ડાયોડ, ICs \\
બીજા સિગ્નલ દ્વારા કરંટ ફ્લો કંટ્રોલ કરી શકતા નથી & બીજા સિગ્નલનો ઉપયોગ કરીને
કરંટ ફ્લો કંટ્રોલ કરી શકે છે \\
ઊર્જાનો સંગ્રહ કે વ્યય કરે છે & ઊર્જા ઉત્પન્ન કરે છે અથવા ગેઈન પ્રદાન કરે છે \\
\end{longtable}
}

\end{solutionbox}
\begin{mnemonicbox}
``PAPER-A'' - Passive Are Power-free,
Energy-storing/Resistive; Active Are Amplifying

\end{mnemonicbox}
\subsection*{પ્રશ્ન 1(બ) [4
ગુણ]}\label{uxaaauxab0uxab6uxaa8-1uxaac-4-uxa97uxaa3}

\textbf{આકૃતિ સહિત Light dependent resistor ની કામગીરી સમજાવો.}

\begin{solutionbox}

\begin{center}
\textbf{Mermaid Diagram (Code)}
\begin{verbatim}
{Shaded}
{Highlighting}[]
graph LR
    A[પ્રકાશ] {-{-}{} B[LDR]}
    B {-{-}{} C[રેઝિસ્ટન્સમાં ફેરફાર]}
    style A fill:\#lightblue
    style B fill:\#lightgreen
    style C fill:\#lightpink
{Highlighting}
{Shaded}
\end{verbatim}
\end{center}

\textbf{LDR ની કાર્યપદ્ધતિ:}

\begin{itemize}
\tightlist
\item
  \textbf{રચના}: LDR અંધારામાં ઉચ્ચ રેઝિસ્ટન્સ ધરાવતા સેમિકન્ડક્ટર મટેરિયલ (સામાન્ય
  રીતે કેડમિયમ સલ્ફાઇડ) થી બનેલું હોય છે
\item
  \textbf{ફોટોકન્ડક્ટિવિટી}: જ્યારે સપાટી પર પ્રકાશ પડે છે, ત્યારે ફોટોન
  ઇલેક્ટ્રોન્સને ઊર્જા આપે છે, જેનાથી ફ્રી ઇલેક્ટ્રોન-હોલ જોડી બને છે
\item
  \textbf{રેઝિસ્ટન્સમાં ફેરફાર}: પ્રકાશની તીવ્રતા વધતાં રેઝિસ્ટન્સ નાટકીય રીતે ઘટે છે
  - અંધારામાં મેગાઓમ્સથી પ્રકાશમાં ફક્ત થોડાસો ઓમ્સ સુધી
\item
  \textbf{ઉપયોગો}: લાઇટ સેન્સિંગ સર્કિટ, ઓટોમેટિક સ્ટ્રીટ લાઇટ્સ, કેમેરા એક્સપોઝર
  કંટ્રોલમાં વપરાય છે
\end{itemize}

\end{solutionbox}
\begin{mnemonicbox}
``MILD'' - More Illumination, Less resistance in
Devices

\end{mnemonicbox}
\subsection*{પ્રશ્ન 1(ક) [7
ગુણ]}\label{uxaaauxab0uxab6uxaa8-1uxa95-7-uxa97uxaa3}

\textbf{Intrinsic અને Extrinsic સેમિકન્ડક્ટર વ્યાખ્યાયિત કરો. P અને N પ્રકારના
સેમીકન્ડક્ટરને સવિસ્તર સમજાવો.}

\begin{solutionbox}

{\def\LTcaptype{none} % do not increment counter
\begin{longtable}[]{@{}ll@{}}
\toprule\noalign{}
\textbf{સેમિકન્ડક્ટર પ્રકાર} & \textbf{વર્ણન} \\
\midrule\noalign{}
\endhead
\bottomrule\noalign{}
\endlastfoot
\textbf{Intrinsic} & શુદ્ધ સેમિકન્ડક્ટર મટેરિયલ જેમાં કોઈ અશુદ્ધિઓ ઉમેરવામાં આવતી
નથી \\
\textbf{Extrinsic} & ડોપિંગ દ્વારા નિયંત્રિત અશુદ્ધિઓ ઉમેરાયેલા સેમિકન્ડક્ટર \\
\end{longtable}
}

\textbf{P-પ્રકારના સેમિકન્ડક્ટર:}

\begin{itemize}
\tightlist
\item
  \textbf{ડોપિંગ}: શુદ્ધ સિલિકોનમાં ત્રિ-સંયોજી અશુદ્ધિઓ (બોરોન, ગેલિયમ, ઇન્ડિયમ)
  ઉમેરીને બનાવવામાં આવે છે
\item
  \textbf{હોલ ક્રિએશન}: દરેક અશુદ્ધિ અણુ વેલેન્સ ઇલેક્ટ્રોન સ્વીકારીને એક હોલ બનાવે છે
\item
  \textbf{મેજોરિટી કેરિયર્સ}: હોલ મેજોરિટી કેરિયર છે
\item
  \textbf{માઈનોરિટી કેરિયર્સ}: ઇલેક્ટ્રોન્સ માઈનોરિટી કેરિયર છે
\item
  \textbf{ઇલેક્ટ્રિકલ પ્રોપર્ટીઝ}: પોઝિટિવ ચાર્જ કેરિયર્સ કન્ડક્શનમાં મુખ્ય ભાગ ભજવે છે
\end{itemize}

\textbf{N-પ્રકારના સેમિકન્ડક્ટર:}

\begin{itemize}
\tightlist
\item
  \textbf{ડોપિંગ}: શુદ્ધ સિલિકોનમાં પંચ-સંયોજી અશુદ્ધિઓ (ફોસ્ફરસ, આર્સેનિક,
  એન્ટિમની) ઉમેરીને બનાવવામાં આવે છે
\item
  \textbf{ઇલેક્ટ્રોન ક્રિએશન}: દરેક અશુદ્ધિ અણુ એક વધારાનો ઇલેક્ટ્રોન આપે છે
\item
  \textbf{મેજોરિટી કેરિયર્સ}: ઇલેક્ટ્રોન મેજોરિટી કેરિયર છે
\item
  \textbf{માઈનોરિટી કેરિયર્સ}: હોલ માઈનોરિટી કેરિયર છે
\item
  \textbf{ઇલેક્ટ્રિકલ પ્રોપર્ટીઝ}: નેગેટિવ ચાર્જ કેરિયર્સ કન્ડક્શનમાં મુખ્ય ભાગ ભજવે છે
\end{itemize}

\textbf{આકૃતિ:}

\begin{verbatim}
+{-{-}{-}{-}{-}{-}{-}{-}{-}{-}{-}{-}{-}{-}{-}{-}+   +{-}{-}{-}{-}{-}{-}{-}{-}{-}{-}{-}{-}{-}{-}{-}{-}+}
| N{-type         |   | P{-}type         |}
|                |   |                |
| Si Si Si Si Si |   | Si Si Si Si Si |
|                |   |                |
| Si Si P  Si Si |   | Si Si B  Si Si |
|      |         |   |      |         |
| Si Si|Si Si Si |   | Si Si|Si Si Si |
|      v         |   |      v         |
| Si Si e{- Si Si |   | Si Si h+ Si Si |}
|                |   |                |
| Si Si Si Si Si |   | Si Si Si Si Si |
+{-{-}{-}{-}{-}{-}{-}{-}{-}{-}{-}{-}{-}{-}{-}{-}+   +{-}{-}{-}{-}{-}{-}{-}{-}{-}{-}{-}{-}{-}{-}{-}{-}+}
  Extra electron       Extra hole
\end{verbatim}

\end{solutionbox}
\begin{mnemonicbox}
``PINE'' - Positive Impurities make N-type
Electrons, Pentavalent donors

\end{mnemonicbox}
\subsection*{પ્રશ્ન 1(ક) OR [7
ગુણ]}\label{uxaaauxab0uxab6uxaa8-1uxa95-or-7-uxa97uxaa3}

\textbf{ફિલ્ટર સર્કિટ એટલે શું? તેના પ્રકાર અને જરૂરિયાત જણાવો અને ``પાઇ'' ફિલ્ટર
સર્કિટને ટૂંકમાં સમજાવો.}

\begin{solutionbox}

\textbf{ફિલ્ટર સર્કિટ}: ઇલેક્ટ્રોનિક સર્કિટ જે સિગ્નલમાંથી અવાંછિત ફ્રિક્વન્સી
કમ્પોનન્ટ્સને દૂર કરે છે, અને ઇચ્છિત ફ્રિક્વન્સીને પસાર થવા દે છે.

\textbf{ફિલ્ટરની જરૂરિયાત}:

\begin{itemize}
\tightlist
\item
  \textbf{રિપલ ઘટાડવા}: રેક્ટિફાયર આઉટપુટમાંથી AC રિપલ ઘટાડે છે
\item
  \textbf{ક્લિન DC}: વધુ સારી રીતે સ્મૂધ DC આઉટપુટ વોલ્ટેજ પ્રદાન કરે છે
\item
  \textbf{કમ્પોનન્ટ સુરક્ષા}: ડાઉનસ્ટ્રીમ કમ્પોનન્ટ્સને વોલ્ટેજ ફ્લક્ચ્યુએશનથી બચાવે છે
\item
  \textbf{કાર્યક્ષમતા}: સમગ્ર પાવર સપ્લાયની કાર્યક્ષમતા સુધારે છે
\end{itemize}

\textbf{ફિલ્ટરના પ્રકાર}:

{\def\LTcaptype{none} % do not increment counter
\begin{longtable}[]{@{}lll@{}}
\toprule\noalign{}
\textbf{ફિલ્ટરનો પ્રકાર} & \textbf{કમ્પોનન્ટ્સ} & \textbf{ઉપયોગ} \\
\midrule\noalign{}
\endhead
\bottomrule\noalign{}
\endlastfoot
શન્ટ કેપેસિટર & પેરેલલમાં એક કેપેસિટર & બેઝિક ફિલ્ટરિંગ \\
L-ટાઇપ & ઇન્ડક્ટર અને કેપેસિટર & બેટર ફિલ્ટરિંગ \\
π (પાઇ) ફિલ્ટર & બે કેપેસિટર અને એક ઇન્ડક્ટર & સુપિરિયર ફિલ્ટરિંગ \\
RC ફિલ્ટર & રેઝિસ્ટર અને કેપેસિટર & લો-પાવર એપ્લિકેશન \\
\end{longtable}
}

\textbf{પાઇ (π) ફિલ્ટર}:

\begin{center}
\textbf{Mermaid Diagram (Code)}
\begin{verbatim}
{Shaded}
{Highlighting}[]
graph LR
    A[ઇનપુટ] {-{-}{} B[કેપેસિટર C1]}
    B {-{-}{} C[ઇન્ડક્ટર L]}
    C {-{-}{} D[કેપેસિટર C2]}
    D {-{-}{} E[આઉટપુટ]}
    style A fill:\#lightblue
    style B fill:\#lightgreen
    style C fill:\#lightpink
    style D fill:\#lightgreen
    style E fill:\#lightblue
{Highlighting}
{Shaded}
\end{verbatim}
\end{center}

\begin{itemize}
\tightlist
\item
  \textbf{કાર્યપદ્ધતિ}: પ્રથમ કેપેસિટર (C1) પ્રારંભિક રિપલ ઘટાડે છે, ઇન્ડક્ટર (L)
  AC કમ્પોનન્ટને અવરોધે છે, બીજો કેપેસિટર (C2) બાકીના રિપલ્સને ફિલ્ટર કરે છે
\item
  \textbf{ફાયદો}: સાધારણ રીતે 0.5\% થી નીચેના રિપલ ફેક્ટર સાથે સુપિરિયર
  ફિલ્ટરિંગ પ્રદાન કરે છે
\item
  \textbf{ઉપયોગો}: હાઇ-કરંટ પાવર સપ્લાયમાં વપરાય છે જ્યાં ક્લિન DC જરૂરી હોય
\end{itemize}

\end{solutionbox}
\begin{mnemonicbox}
``PIRO'' - Pi filters Input Ripples Out effectively

\end{mnemonicbox}
\subsection*{પ્રશ્ન 2(અ) [3
ગુણ]}\label{uxaaauxab0uxab6uxaa8-2uxa85-3-uxa97uxaa3}

\textbf{વિવિધ પ્રકારના કેપેસિટર લખો અને કોઈ પણ બે સમજાવો.}

\begin{solutionbox}

\textbf{કેપેસિટરના પ્રકાર}:

\begin{itemize}
\tightlist
\item
  સિરામિક કેપેસિટર
\item
  ઇલેક્ટ્રોલિટિક કેપેસિટર
\item
  ટેન્ટાલમ કેપેસિટર
\item
  ફિલ્મ કેપેસિટર
\item
  માઇકા કેપેસિટર
\item
  વેરિએબલ કેપેસિટર
\end{itemize}

\textbf{સિરામિક કેપેસિટર}:

\begin{itemize}
\tightlist
\item
  \textbf{રચના}: ધાતુની પ્લેટો વચ્ચે ડાઇઇલેક્ટ્રિક તરીકે સિરામિક મટેરિયલથી બનેલા
\item
  \textbf{કેપેસિટી}: 1pF થી 1μF
\item
  \textbf{ફાયદા}: ઓછી કિંમત, ઉચ્ચ સ્થિરતા, નોન-પોલરાઈઝ્ડ
\item
  \textbf{ઉપયોગો}: હાઇ-ફ્રિક્વન્સી ફિલ્ટરિંગ
\end{itemize}

\textbf{ઇલેક્ટ્રોલિટિક કેપેસિટર}:

\begin{itemize}
\tightlist
\item
  \textbf{રચના}: એલ્યુમિનિયમ ફોઇલ સાથે ડાઇઇલેક્ટ્રિક તરીકે ઓક્સાઇડ લેયર
\item
  \textbf{કેપેસિટી}: 1μF થી 10,000μF
\item
  \textbf{લાક્ષણિકતાઓ}: પોલરાઈઝ્ડ, ઉચ્ચ લીકેજ કરંટ
\item
  \textbf{ઉપયોગો}: પાવર સપ્લાય ફિલ્ટરિંગ, ઓડિયો કપલિંગ
\end{itemize}

\end{solutionbox}
\begin{mnemonicbox}
``CAPEX'' - Ceramics Are Precise, Electrolytics
Expand capacity

\end{mnemonicbox}
\subsection*{પ્રશ્ન 2(બ) [4
ગુણ]}\label{uxaaauxab0uxab6uxaa8-2uxaac-4-uxa97uxaa3}

\textbf{એર કોર અને ટોરોઇડલ ઇન્ડક્ટર સમજાવો.}

\begin{solutionbox}

\textbf{એર કોર ઇન્ડક્ટર:}

\begin{verbatim}
   +{-{-}{-}{-}{-}{-}{-}{-}{-}{-}+}
   |    Air   |
   |          |
 +{-|{-}{-}{-}{-}{-}{-}{-}{-}{-}{-}|{-}{-}+}
 | |          |  |
 | |          |  |
 | |          |  |
 | |          |  |
 | |          |  |
 +{-|{-}{-}{-}{-}{-}{-}{-}{-}{-}{-}|{-}{-}+}
   |          |
   +{-{-}{-}{-}{-}{-}{-}{-}{-}{-}+}
   Wire windings
\end{verbatim}

\begin{itemize}
\tightlist
\item
  \textbf{રચના}: નોન-મેગ્નેટિક મટેરિયલ (પ્લાસ્ટિક, એર) પર વાયર કોઇલ કરીને
  બનાવવામાં આવે છે
\item
  \textbf{ગુણધર્મો}: ઓછી ઇન્ડક્ટન્સ, મેગ્નેટિક કોર સેચ્યુરેશન નથી
\item
  \textbf{ઉપયોગો}: હાઇ-ફ્રિક્વન્સી સર્કિટ, RF એપ્લિકેશન
\item
  \textbf{ફાયદા}: કોર લોસેસ નથી, લિનિયર ઓપરેશન, સેચ્યુરેશન નથી
\end{itemize}

\textbf{ટોરોઇડલ ઇન્ડક્ટર:}

\begin{verbatim}
      +{-{-}{-}{-}{-}{-}{-}+}
     /         {}
    /           {}
   /     Air     {}
  |       +       |
  |      / {      |}
  |     /   {     |}
  |    +{-{-}{-}{-}{-}+    |}
   {   |     |   /}
    {  |     |  /}
     { |     | /}
      ++{-{-}{-}{-}{-}++}
     Wire windings
      around core
\end{verbatim}

\begin{itemize}
\tightlist
\item
  \textbf{રચના}: રિંગ-આકારના મેગ્નેટિક કોર પર વાયર વીંટાળીને બનાવવામાં આવે છે
\item
  \textbf{ગુણધર્મો}: ઉચ્ચ ઇન્ડક્ટન્સ, સેલ્ફ-શીલ્ડિંગ મેગ્નેટિક ફિલ્ડ
\item
  \textbf{ઉપયોગો}: પાવર સપ્લાય, ફિલ્ટર, ટ્રાન્સફોર્મર
\item
  \textbf{ફાયદા}: ઓછી ઇલેક્ટ્રોમેગ્નેટિક ઇન્ટરફેરન્સ, કાર્યક્ષમ ફ્લક્સ કન્ટેઇનમેન્ટ
\end{itemize}

\end{solutionbox}
\begin{mnemonicbox}
``TACO'' - Toroids Are Contained, Omnidirectional
field reduction

\end{mnemonicbox}
\subsection*{પ્રશ્ન 2(ક) [7
ગુણ]}\label{uxaaauxab0uxab6uxaa8-2uxa95-7-uxa97uxaa3}

\textbf{હાફ વેવ રેક્ટિફાયર સમજાવો અને જુદા જુદા રેક્ટિફાયર સરખાવો.}

\begin{solutionbox}

\textbf{હાફ વેવ રેક્ટિફાયર:}

\begin{center}
\textbf{Mermaid Diagram (Code)}
\begin{verbatim}
{Shaded}
{Highlighting}[]
graph LR
    A[AC ઇનપુટ] {-{-}{} B[ટ્રાન્સફોર્મર]}
    B {-{-}{} C[ડાયોડ]}
    C {-{-}{} D[લોડ]}
    C {-{-}{} E[ગ્રાઉન્ડ]}
    style A fill:\#lightblue
    style B fill:\#lightpink
    style C fill:\#lightyellow
    style D fill:\#lightgreen
    style E fill:\#lightgray
{Highlighting}
{Shaded}
\end{verbatim}
\end{center}

\textbf{કાર્યસિદ્ધાંત:}

\begin{itemize}
\tightlist
\item
  પોઝિટિવ હાફ-સાયકલ દરમિયાન: ડાયોડ કન્ડક્ટ કરે છે, કરંટ લોડ દ્વારા વહે છે
\item
  નેગેટિવ હાફ-સાયકલ દરમિયાન: ડાયોડ બ્લોક કરે છે, કરંટ વહેતો નથી
\item
  આઉટપુટમાં ફક્ત ઇનપુટ વેવફોર્મના પોઝિટિવ હાફ-સાયકલ હોય છે
\end{itemize}

\textbf{રેક્ટિફાયરની સરખામણી:}

{\def\LTcaptype{none} % do not increment counter
\begin{longtable}[]{@{}
  >{\raggedright\arraybackslash}p{(\linewidth - 6\tabcolsep) * \real{0.2500}}
  >{\raggedright\arraybackslash}p{(\linewidth - 6\tabcolsep) * \real{0.2500}}
  >{\raggedright\arraybackslash}p{(\linewidth - 6\tabcolsep) * \real{0.2500}}
  >{\raggedright\arraybackslash}p{(\linewidth - 6\tabcolsep) * \real{0.2500}}@{}}
\toprule\noalign{}
\begin{minipage}[b]{\linewidth}\raggedright
\textbf{પેરામીટર}
\end{minipage} & \begin{minipage}[b]{\linewidth}\raggedright
\textbf{હાફ વેવ}
\end{minipage} & \begin{minipage}[b]{\linewidth}\raggedright
\textbf{ફુલ વેવ (સેન્ટર-ટેપ)}
\end{minipage} & \begin{minipage}[b]{\linewidth}\raggedright
\textbf{બ્રિજ રેક્ટિફાયર}
\end{minipage} \\
\midrule\noalign{}
\endhead
\bottomrule\noalign{}
\endlastfoot
જરૂરી ડાયોડ & 1 & 2 & 4 \\
આઉટપુટ ફ્રિક્વન્સી & f_{1} = fin & f_{2} = 2\timesfin & f_{2} = 2\timesfin \\
રિપલ ફેક્ટર & 1.21 & 0.48 & 0.48 \\
કાર્યક્ષમતા & 40.6\% & 81.2\% & 81.2\% \\
PIV & 2Vm & 2Vm & Vm \\
TUF & 0.287 & 0.693 & 0.812 \\
DC આઉટપુટ & Vm/π & 2Vm/π & 2Vm/π \\
\end{longtable}
}

\end{solutionbox}
\begin{mnemonicbox}
``BRIEF'' - Bridge Rectifiers Improve Efficiency
Fundamentally

\end{mnemonicbox}
\subsection*{પ્રશ્ન 2(અ) OR [3
ગુણ]}\label{uxaaauxab0uxab6uxaa8-2uxa85-or-3-uxa97uxaa3}

\textbf{વિવિધ કેપેસિટર સ્પષ્ટીકરણો લખો અને કોઈ પણ બે વિગતવાર સમજાવો.}

\begin{solutionbox}

\textbf{કેપેસિટર સ્પષ્ટીકરણો:}

\begin{itemize}
\tightlist
\item
  કેપેસિટન્સ વેલ્યુ
\item
  વોલ્ટેજ રેટિંગ
\item
  ટોલરન્સ
\item
  તાપમાન ગુણાંક
\item
  ESR (ઇક્વિવેલન્ટ સિરીઝ રેઝિસ્ટન્સ)
\item
  લીકેજ કરંટ
\item
  ડાઇઇલેક્ટ્રિક પ્રકાર
\end{itemize}

\textbf{કેપેસિટન્સ વેલ્યુ:}

\begin{itemize}
\tightlist
\item
  \textbf{વ્યાખ્યા}: દર વોલ્ટે સંગ્રહિત ઇલેક્ટ્રિક ચાર્જની માત્રા
\item
  \textbf{એકમો}: ફેરડ (F)માં માપવામાં આવે છે, સામાન્ય રીતે માઇક્રોફેરડ (μF),
  નેનોફેરડ (nF), અથવા પિકોફેરડ (pF)
\item
  \textbf{મહત્વ}: કપલિંગ, ફિલ્ટરિંગ, ટાઇમિંગ માટે એપ્લિકેશન યોગ્યતા નક્કી કરે છે
\item
  \textbf{માર્કિંગ}: સીધી રીતે પ્રિન્ટ કરેલી અથવા કમ્પોનન્ટ પર કલર-કોડેડ
\end{itemize}

\textbf{વોલ્ટેજ રેટિંગ:}

\begin{itemize}
\tightlist
\item
  \textbf{વ્યાખ્યા}: બ્રેકડાઉન વગર લાગુ કરી શકાય તેવું મહત્તમ વોલ્ટેજ
\item
  \textbf{સ્પેસિફિકેશન}: વર્કિંગ વોલ્ટેજ (WVDC) અને સર્જ વોલ્ટેજ
\item
  \textbf{મહત્વ}: રેટિંગથી વધારે જવાથી ડાઇઇલેક્ટ્રિક બ્રેકડાઉન અને નિષ્ફળતા થાય છે
\item
  \textbf{સેફ્ટી ફેક્ટર}: સામાન્ય રીતે સર્કિટ વોલ્ટેજથી 50\% વધુ રેટિંગવાળા કેપેસિટર
  વાપરવા જોઈએ
\end{itemize}

\end{solutionbox}
\begin{mnemonicbox}
``CAVERN'' - Capacitance And Voltage Ensure Reliable
Network

\end{mnemonicbox}
\subsection*{પ્રશ્ન 2(બ) OR [4
ગુણ]}\label{uxaaauxab0uxab6uxaa8-2uxaac-or-4-uxa97uxaa3}

\textbf{સામગ્રીના આધારે રેઝિસ્ટરનું વર્ગીકરણ સમજાવો.}

\begin{solutionbox}

{\def\LTcaptype{none} % do not increment counter
\begin{longtable}[]{@{}
  >{\raggedright\arraybackslash}p{(\linewidth - 6\tabcolsep) * \real{0.2500}}
  >{\raggedright\arraybackslash}p{(\linewidth - 6\tabcolsep) * \real{0.2500}}
  >{\raggedright\arraybackslash}p{(\linewidth - 6\tabcolsep) * \real{0.2500}}
  >{\raggedright\arraybackslash}p{(\linewidth - 6\tabcolsep) * \real{0.2500}}@{}}
\toprule\noalign{}
\begin{minipage}[b]{\linewidth}\raggedright
\textbf{રેઝિસ્ટર પ્રકાર}
\end{minipage} & \begin{minipage}[b]{\linewidth}\raggedright
\textbf{સામગ્રી}
\end{minipage} & \begin{minipage}[b]{\linewidth}\raggedright
\textbf{ગુણધર્મો}
\end{minipage} & \begin{minipage}[b]{\linewidth}\raggedright
\textbf{ઉપયોગો}
\end{minipage} \\
\midrule\noalign{}
\endhead
\bottomrule\noalign{}
\endlastfoot
\textbf{કાર્બન કમ્પોઝિશન} & કાર્બન પાર્ટિકલ્સ + સિરેમિક બાઇન્ડર & ઉચ્ચ તાપમાન
ગુણાંક, નોઈઝી & સામાન્ય ઉપયોગ, સર્જ પ્રોટેક્શન \\
\textbf{કાર્બન ફિલ્મ} & સિરેમિક પર કાર્બન ફિલ્મ & કાર્બન કમ્પોઝિશન કરતાં વધુ
સ્થિરતા & સામાન્ય ઉપયોગ સર્કિટ \\
\textbf{મેટલ ફિલ્મ} & સિરેમિક પર નિકલ ક્રોમિયમ ફિલ્મ & ઓછો નોઇઝ, સ્થિર, ચોક્કસ
& ઓડિયો સર્કિટ, ઇન્સ્ટ્રુમેન્ટેશન \\
\textbf{વાયર વાઉન્ડ} & સિરેમિક આસપાસ રેઝિસ્ટન્સ વાયર & હાઈ પાવર, લો તાપમાન
ગુણાંક & પાવર સપ્લાય, હાઈ કરંટ એપ્લિકેશન \\
\textbf{મેટલ ઓક્સાઇડ} & સિરેમિક પર મેટલ ઓક્સાઇડ ફિલ્મ & સ્ટેબલ, હાઈ તાપમાન
ટોલરન્સ & હાઈ સ્ટેબિલિટી એપ્લિકેશન, પાવર સપ્લાય \\
\end{longtable}
}

\textbf{કાર્બન ફિલ્મ રેઝિસ્ટરની લાક્ષણિકતાઓ:}

\begin{itemize}
\tightlist
\item
  તાપમાન ગુણાંક: -250 થી 500 ppm/^\circC
\item
  ટોલરન્સ: 5\% થી 10\%
\item
  નોઇઝ: મધ્યમથી ઓછો
\end{itemize}

\textbf{મેટલ ફિલ્મ રેઝિસ્ટરની લાક્ષણિકતાઓ:}

\begin{itemize}
\tightlist
\item
  તાપમાન ગુણાંક: 50 થી 100 ppm/^\circC
\item
  ટોલરન્સ: 0.1\% થી 2\%
\item
  નોઇઝ: ખૂબ જ ઓછો
\end{itemize}

\end{solutionbox}
\begin{mnemonicbox}
``COMFORT'' - Carbon Offers Moderate Films, Others
Resist Temperature better

\end{mnemonicbox}
\subsection*{પ્રશ્ન 2(ક) OR [7
ગુણ]}\label{uxaaauxab0uxab6uxaa8-2uxa95-or-7-uxa97uxaa3}

\textbf{ફુલ વેવ બ્રિજ અને સેન્ટર ટેપ્ડ રેક્ટિફાયર આકૃતિ સાથે સમજાવો.}

\begin{solutionbox}

\textbf{ફુલ વેવ બ્રિજ રેક્ટિફાયર:}

\begin{center}
\textbf{Mermaid Diagram (Code)}
\begin{verbatim}
{Shaded}
{Highlighting}[]
graph LR
    A[AC ઇનપુટ] {-{-}{} B[ટ્રાન્સફોર્મર]}
    B {-{-}{} C[બ્રિજ{}br /{}રેક્ટિફાયર]}
    C {-{-}{} D[D1]}
    C {-{-}{} E[D2]}
    C {-{-}{} F[D3]}
    C {-{-}{} G[D4]}
    D \& E \& F \& G {-{-}{} H[લોડ]}
    H {-{-}{} I[ગ્રાઉન્ડ]}
    style A fill:\#lightblue
    style B fill:\#lightpink
    style C fill:\#lightyellow
    style H fill:\#lightgreen
    style I fill:\#lightgray
{Highlighting}
{Shaded}
\end{verbatim}
\end{center}

\textbf{કાર્યપદ્ધતિ:}

\begin{itemize}
\tightlist
\item
  \textbf{પોઝિટિવ હાફ-સાયકલ}: D1 અને D3 કન્ડક્ટ કરે છે, કરંટ લોડ મારફતે વહે છે
\item
  \textbf{નેગેટિવ હાફ-સાયકલ}: D2 અને D4 કન્ડક્ટ કરે છે, કરંટ હજુ પણ એ જ દિશામાં
  લોડ મારફતે વહે છે
\item
  \textbf{આઉટપુટ}: ઇનપુટના બંને હાફ-સાયકલ પોઝિટિવ આઉટપુટમાં રૂપાંતરિત થાય છે
\end{itemize}

\textbf{સેન્ટર ટેપ્ડ ફુલ વેવ રેક્ટિફાયર:}

\begin{center}
\textbf{Mermaid Diagram (Code)}
\begin{verbatim}
{Shaded}
{Highlighting}[]
graph LR
    A[AC ઇનપુટ] {-{-}{} B[સેન્ટર{-}ટેપ્ડ{}br /{}ટ્રાન્સફોર્મર]}
    B {-{-}{}|ઉપર અર્ધો| C[D1]}
    B {-{-}{}|નીચે અર્ધો| D[D2]}
    C \& D {-{-}{} E[લોડ]}
    E {-{-}{} F[ગ્રાઉન્ડ]}
    F {-{-}{} B}
    style A fill:\#lightblue
    style B fill:\#lightpink
    style C fill:\#lightyellow
    style D fill:\#lightyellow
    style E fill:\#lightgreen
    style F fill:\#lightgray
{Highlighting}
{Shaded}
\end{verbatim}
\end{center}

\textbf{કાર્યપદ્ધતિ:}

\begin{itemize}
\tightlist
\item
  \textbf{પોઝિટિવ હાફ-સાયકલ}: D1 કન્ડક્ટ કરે છે, D2 બ્લોક કરે છે
\item
  \textbf{નેગેટિવ હાફ-સાયકલ}: D2 કન્ડક્ટ કરે છે, D1 બ્લોક કરે છે
\item
  \textbf{આઉટપુટ}: ઇનપુટના બંને હાફ-સાયકલ પોઝિટિવ આઉટપુટમાં રૂપાંતરિત થાય છે
\end{itemize}

\textbf{વેવફોર્મ:}

\begin{verbatim}
Input:  ∿∿∿∿∿∿∿∿∿∿∿∿∿∿∿
         |
         v
Bridge: 
Rectifier
         |
         v
Output: 
(with filter)
\end{verbatim}

\end{solutionbox}
\begin{mnemonicbox}
``FOUR-TWO'' - FOUr diodes for Bridge, TWO diodes
for Center-Tap

\end{mnemonicbox}
\subsection*{પ્રશ્ન 3(અ) [3
ગુણ]}\label{uxaaauxab0uxab6uxaa8-3uxa85-3-uxa97uxaa3}

\textbf{વેરેક્ટર ડાયોડની લાક્ષણિકતા સમજાવો.}

\begin{solutionbox}

\textbf{વેરેક્ટર ડાયોડની લાક્ષણિકતાઓ:}

\begin{center}
\textbf{Mermaid Diagram (Code)}
\begin{verbatim}
{Shaded}
{Highlighting}[]
graph LR
    A[રિવર્સ બાયસ{br /{}વોલ્ટેજ] {-}{-}{} B[ડિપ્લિશન{}br /{}લેયર વિડ્થ]}
    B {-{-}{} C[જંક્શન{}br /{}કેપેસિટન્સ]}
    C {-{-}{} D[ફ્રિક્વન્સી{}br /{}ટ્યુનિંગ]}
    style A fill:\#lightblue
    style B fill:\#lightpink
    style C fill:\#lightgreen
    style D fill:\#lightyellow
{Highlighting}
{Shaded}
\end{verbatim}
\end{center}

\begin{itemize}
\tightlist
\item
  \textbf{ઓપરેટિંગ સિદ્ધાંત}: જંક્શન કેપેસિટન્સ રિવર્સ બાયસ વોલ્ટેજ સાથે બદલાય છે
\item
  \textbf{C-V સંબંધ}: રિવર્સ વોલ્ટેજ વધતાં કેપેસિટન્સ ઘટે છે
\item
  \textbf{ટ્યુનિંગ રેશિયો}: સામાન્ય રીતે 4:1 થી 10:1 કેપેસિટન્સ વેરિએશન
\item
  \textbf{ઉપયોગો}: વોલ્ટેજ-કંટ્રોલ્ડ ઓસિલેટર, FM મોડ્યુલેશન, ટ્યુનિંગ સર્કિટ
\end{itemize}

\end{solutionbox}
\begin{mnemonicbox}
``VARA'' - Voltage Adjusts Reverse-biased
capacitance Automatically

\end{mnemonicbox}
\subsection*{પ્રશ્ન 3(બ) [4
ગુણ]}\label{uxaaauxab0uxab6uxaa8-3uxaac-4-uxa97uxaa3}

\textbf{ઇલેક્ટ્રોમેગ્નેટિક ઇન્ડક્શનના ફેરાડેના નિયમો જણાવો અને સમજાવો.}

\begin{solutionbox}

\textbf{ફેરાડેના ઇલેક્ટ્રોમેગ્નેટિક ઇન્ડક્શનના નિયમો:}

\textbf{પ્રથમ નિયમ:}

\begin{itemize}
\tightlist
\item
  \textbf{સ્ટેટમેન્ટ}: જ્યારે પણ કન્ડક્ટર મેગ્નેટિક ફ્લક્સને કાપે છે, ત્યારે કન્ડક્ટરમાં EMF
  ઇન્ડ્યુસ થાય છે
\item
  \textbf{ગણિતીય અભિવ્યક્તિ}: EMF ∝ મેગ્નેટિક ફ્લક્સના પરિવર્તનનો દર
\item
  \textbf{ઉપયોગ}: જનરેટર, ટ્રાન્સફોર્મર, ઇન્ડક્ટરનો આધાર
\end{itemize}

\textbf{બીજો નિયમ:}

\begin{itemize}
\tightlist
\item
  \textbf{સ્ટેટમેન્ટ}: ઇન્ડ્યુસ્ડ EMFનું પરિમાણ મેગ્નેટિક ફ્લક્સ લિંકેજના પરિવર્તનના દર
  સાથે સમાન છે
\item
  \textbf{ગણિતીય અભિવ્યક્તિ}: EMF = -N \times (dΦ/dt)

  \begin{itemize}
  \tightlist
  \item
જ્યાં:

N = લપેટાઓની સંખ્યા, dΦ/dt = ફ્લક્સના પરિવર્તનનો દર

  \end{itemize}
\item
  \textbf{નેગેટિવ ચિહ્ન}: દિશા દર્શાવે છે (લેન્ઝનો નિયમ) - ઇન્ડ્યુસ્ડ કરંટ પરિવર્તનનો
  વિરોધ કરે છે
\end{itemize}

\textbf{આકૃતિ:}

\begin{verbatim}
    N     S       
    |     |       
    v     v       
  +{-{-}{-}+ +{-}{-}{-}+     }
  |   | |   |     
  |   | |   |     
  +{-{-}{-}+ +{-}{-}{-}+     }
    \^{     \^{}       }
    |     |       
    |     |       
  +{-{-}{-}{-}{-}{-}{-}{-}{-}{-}+    }
  |   Coil   |{-{-}{-}{-}{-} Induced EMF}
  +{-{-}{-}{-}{-}{-}{-}{-}{-}{-}+    }
\end{verbatim}

\end{solutionbox}
\begin{mnemonicbox}
``FACE'' - Flux Alteration Creates Electricity

\end{mnemonicbox}
\subsection*{પ્રશ્ન 3(ક) [7
ગુણ]}\label{uxaaauxab0uxab6uxaa8-3uxa95-7-uxa97uxaa3}

\textbf{વિવિધ ટ્રાન્ઝિસ્ટર રૂપરેખાંકનોની તુલના કરો.}

\begin{solutionbox}

{\def\LTcaptype{none} % do not increment counter
\begin{longtable}[]{@{}
  >{\raggedright\arraybackslash}p{(\linewidth - 6\tabcolsep) * \real{0.2500}}
  >{\raggedright\arraybackslash}p{(\linewidth - 6\tabcolsep) * \real{0.2500}}
  >{\raggedright\arraybackslash}p{(\linewidth - 6\tabcolsep) * \real{0.2500}}
  >{\raggedright\arraybackslash}p{(\linewidth - 6\tabcolsep) * \real{0.2500}}@{}}
\toprule\noalign{}
\begin{minipage}[b]{\linewidth}\raggedright
\textbf{પેરામીટર}
\end{minipage} & \begin{minipage}[b]{\linewidth}\raggedright
\textbf{કોમન ઇમિટર (CE)}
\end{minipage} & \begin{minipage}[b]{\linewidth}\raggedright
\textbf{કોમન બેઝ (CB)}
\end{minipage} & \begin{minipage}[b]{\linewidth}\raggedright
\textbf{કોમન કલેક્ટર (CC)}
\end{minipage} \\
\midrule\noalign{}
\endhead
\bottomrule\noalign{}
\endlastfoot
\textbf{ઇનપુટ ટર્મિનલ} & બેઝ & ઇમિટર & બેઝ \\
\textbf{આઉટપુટ ટર્મિનલ} & કલેક્ટર & કલેક્ટર & ઇમિટર \\
\textbf{કોમન ટર્મિનલ} & ઇમિટર & બેઝ & કલેક્ટર \\
\textbf{કરંટ ગેઇન (α, β, γ)} & β = IC/IB (20-500) & α = IC/IE (0.95-0.99)
& γ = IE/IB (β+1) \\
\textbf{વોલ્ટેજ ગેઇન} & હાઈ (250-1000) & મધ્યમ (150-800) & 1 થી ઓછું \\
\textbf{ઇનપુટ ઇમ્પિડન્સ} & મધ્યમ (1-2kΩ) & લો (30-150Ω) & હાઈ (50-500kΩ) \\
\textbf{આઉટપુટ ઇમ્પિડન્સ} & હાઈ (30-50kΩ) & વેરી હાઈ (250kΩ-1MΩ) & લો
(50-100Ω) \\
\textbf{ફેઝ શિફ્ટ} & 180^\circ & 0^\circ & 0^\circ \\
\textbf{ઉપયોગો} & એમ્પલિફાયર, ઓસિલેટર & RF એમ્પલિફાયર, હાઈ-ફ્રિક્વન્સી સર્કિટ &
ઇમ્પિડન્સ મેચિંગ, બફર \\
\end{longtable}
}

\textbf{α, β અને γ વચ્ચેનો સંબંધ:}

\begin{itemize}
\tightlist
\item
  β = α/(1-α)
\item
  α = β/(1+β)
\item
  γ = β+1
\end{itemize}

\end{solutionbox}
\begin{mnemonicbox}
``BEC'' - Base input for Emitter output needs
Collector as common terminal

\end{mnemonicbox}
\subsection*{પ્રશ્ન 3(અ) OR [3
ગુણ]}\label{uxaaauxab0uxab6uxaa8-3uxa85-or-3-uxa97uxaa3}

\textbf{ફોરબિડન એનર્જી ગેપ શું છે? અવાહક, વાહક અને સેમીકન્ડક્ટર માટે એનર્જી બેન્ડ
ડાયાગ્રામ દોરો.}

\begin{solutionbox}

\textbf{ફોરબિડન એનર્જી ગેપ:} ઘન પદાર્થમાં એનર્જીની શ્રેણી જ્યાં કોઈ ઇલેક્ટ્રોન સ્ટેટ
અસ્તિત્વમાં નથી, વેલેન્સ બેન્ડને કન્ડક્શન બેન્ડથી અલગ કરે છે.

\textbf{એનર્જી બેન્ડ ડાયાગ્રામ:}

\begin{verbatim}
+{-{-}{-}{-}{-}{-}{-}{-}{-}{-}{-}{-}{-}{-}{-}+  +{-}{-}{-}{-}{-}{-}{-}{-}{-}{-}{-}{-}{-}{-}{-}+  +{-}{-}{-}{-}{-}{-}{-}{-}{-}{-}{-}{-}{-}{-}{-}+}
|///////////////|  |///////////////|  |///////////////|
|/// Conduction |  |/// Conduction |  |/// Conduction |
|///////////////|  |///////////////|  |///////////////|
+{-{-}{-}{-}{-}{-}{-}{-}{-}{-}{-}{-}{-}{-}{-}+  +{-}{-}{-}{-}{-}{-}{-}{-}{-}{-}{-}{-}{-}{-}{-}+  +{-}{-}{-}{-}{-}{-}{-}{-}{-}{-}{-}{-}{-}{-}{-}+}
|               |  |///////////////|  |      |        |
|               |  |///////////////|  |      | Small  |
| Large         |  | Overlap       |  |      | Gap    |
| Forbidden     |  |///////////////|  |      |        |
| Gap ({5eV)    |  |///////////////|  |      | (1eV) |}
|               |  |///////////////|  |      |        |
+{-{-}{-}{-}{-}{-}{-}{-}{-}{-}{-}{-}{-}{-}{-}+  +{-}{-}{-}{-}{-}{-}{-}{-}{-}{-}{-}{-}{-}{-}{-}+  +{-}{-}{-}{-}{-}{-}{-}{-}{-}{-}{-}{-}{-}{-}{-}+}
|///////////////|  |///////////////|  |///////////////|
|/// Valence    |  |/// Valence    |  |/// Valence    |
|///////////////|  |///////////////|  |///////////////|
+{-{-}{-}{-}{-}{-}{-}{-}{-}{-}{-}{-}{-}{-}{-}+  +{-}{-}{-}{-}{-}{-}{-}{-}{-}{-}{-}{-}{-}{-}{-}+  +{-}{-}{-}{-}{-}{-}{-}{-}{-}{-}{-}{-}{-}{-}{-}+}
    Insulator          Conductor        Semiconductor
\end{verbatim}

\begin{itemize}
\tightlist
\item
  \textbf{અવાહક (ઇન્સુલેટર)}: મોટો ફોરબિડન ગેપ (\textgreater5eV) ઇલેક્ટ્રોન્સને
  કન્ડક્શન બેન્ડ સુધી પહોંચતા અટકાવે છે
\item
  \textbf{વાહક (કન્ડક્ટર)}: ઓવરલેપિંગ બેન્ડ મુક્ત ઇલેક્ટ્રોન મૂવમેન્ટની મંજૂરી આપે છે
\item
  \textbf{સેમિકન્ડક્ટર}: નાનો ગેપ (\textasciitilde1eV) થોડા ઇલેક્ટ્રોન્સને રૂમ
  ટેમ્પરેચર પર અથવા ઉત્તેજિત થયા પછી ક્રોસ કરવાની મંજૂરી આપે છે
\end{itemize}

\end{solutionbox}
\begin{mnemonicbox}
``IBCS'' - Insulators Block, Conductors Share,
Semiconductors have gap Between

\end{mnemonicbox}
\subsection*{પ્રશ્ન 3(બ) OR [4
ગુણ]}\label{uxaaauxab0uxab6uxaa8-3uxaac-or-4-uxa97uxaa3}

\textbf{ઝેનર વોલ્ટેજ રેગ્યુલેટર સર્કિટની કામગીરીનું વર્ણન કરો.}

\begin{solutionbox}

\begin{center}
\textbf{Mermaid Diagram (Code)}
\begin{verbatim}
{Shaded}
{Highlighting}[]
graph LR
    A[અનરેગ્યુલેટેડ{br /{}DC ઇનપુટ] {-}{-}{} B[સિરીઝ{}br /{}રેઝિસ્ટર]}
    B {-{-}{} C[લોડ]}
    B {-{-}{} D[ઝેનર{}br /{}ડાયોડ]}
    D {-{-}{} E[ગ્રાઉન્ડ]}
    style A fill:\#lightblue
    style B fill:\#lightpink
    style C fill:\#lightgreen
    style D fill:\#lightyellow
    style E fill:\#lightgray
{Highlighting}
{Shaded}
\end{verbatim}
\end{center}

\textbf{કાર્યસિદ્ધાંત:}

\begin{itemize}
\tightlist
\item
  \textbf{સામાન્ય ઓપરેશન}: ઝેનર ડાયોડ રિવર્સ બાયસ્ડ છે અને જ્યારે વોલ્ટેજ બ્રેકડાઉન
  વોલ્ટેજ સુધી પહોંચે ત્યારે કન્ડક્ટ કરે છે
\item
  \textbf{વોલ્ટેજ રેગ્યુલેશન}: જ્યારે ઇનપુટ વોલ્ટેજ વધે છે, ત્યારે ઝેનર ડાયોડ મારફતે વધુ
  કરંટ વહે છે, જેનાથી તેના પર સ્થિર વોલ્ટેજ જળવાઈ રહે છે
\item
  \textbf{લોડ વેરિએશન}: જ્યારે લોડ વધુ કરંટ લે છે, ત્યારે ઝેનર મારફતે ઓછો કરંટ વહે છે,
  જેનાથી વોલ્ટેજ સ્થિર રહે છે
\item
  \textbf{સિરીઝ રેઝિસ્ટર}: કરંટને મર્યાદિત કરે છે અને વધારાના વોલ્ટેજને ડ્રોપ કરે છે
\end{itemize}

\textbf{સર્કિટ બિહેવિયર:}

\begin{itemize}
\tightlist
\item
  \textbf{Vout = Vz} (ઝેનર બ્રેકડાઉન વોલ્ટેજ)
\item
  \textbf{Iz = (Vin - Vz)/R - IL}
\end{itemize}

\end{solutionbox}
\begin{mnemonicbox}
``SERZ'' - Series resistor Enables Regulation with
Zener

\end{mnemonicbox}
\subsection*{પ્રશ્ન 3(ક) OR [7
ગુણ]}\label{uxaaauxab0uxab6uxaa8-3uxa95-or-7-uxa97uxaa3}

\textbf{P-N જંક્શન ડાયોડની V-I લાક્ષણિકતા સમજાવો અને P-N જંક્શન ડાયોડ અને ઝેનર
ડાયોડ વચ્ચે સરખામણી આપો.}

\begin{solutionbox}

\textbf{P-N જંક્શન ડાયોડની V-I લાક્ષણિકતા:}

\begin{verbatim}
                 I
                 \^{}
                 |              /
                 |             /
                 |            /
                 |           /
Forward current  |          /
                 |         /
                 |        /
                 |       /
                 |  Knee/
                 |     /
        {-V {-}{-}{-}{-}{-}|{-}{-}{-}{-}+{-}{-}{-} +V}
                 |   /|
                 |    |
                 |    |
Reverse current  |    |
                 |    |    Breakdown
                 |    |       |
                 |    |       v
                 |    |       ⌄\_\_\_\_\_
\end{verbatim}

\textbf{મુખ્ય પોઇન્ટ્સ:}

\begin{itemize}
\tightlist
\item
  \textbf{ફોરવર્ડ બાયસ}: ની વોલ્ટેજ (\textasciitilde0.7V સિલિકોન માટે) પછી
  સરળતાથી કન્ડક્ટ કરે છે
\item
  \textbf{રિવર્સ બાયસ}: બ્રેકડાઉન વોલ્ટેજ સુધી ખૂબ જ ઓછો લીકેજ કરંટ
\item
  \textbf{બ્રેકડાઉન રીજન}: ઉચ્ચ રિવર્સ વોલ્ટેજ પર થાય છે, સામાન્ય ડાયોડમાં નુકસાન
  કરે છે
\end{itemize}

\textbf{P-N જંક્શન ડાયોડ અને ઝેનર ડાયોડ વચ્ચેની સરખામણી:}

{\def\LTcaptype{none} % do not increment counter
\begin{longtable}[]{@{}
  >{\raggedright\arraybackslash}p{(\linewidth - 4\tabcolsep) * \real{0.3333}}
  >{\raggedright\arraybackslash}p{(\linewidth - 4\tabcolsep) * \real{0.3333}}
  >{\raggedright\arraybackslash}p{(\linewidth - 4\tabcolsep) * \real{0.3333}}@{}}
\toprule\noalign{}
\begin{minipage}[b]{\linewidth}\raggedright
\textbf{પેરામીટર}
\end{minipage} & \begin{minipage}[b]{\linewidth}\raggedright
\textbf{P-N જંક્શન ડાયોડ}
\end{minipage} & \begin{minipage}[b]{\linewidth}\raggedright
\textbf{ઝેનર ડાયોડ}
\end{minipage} \\
\midrule\noalign{}
\endhead
\bottomrule\noalign{}
\endlastfoot
\textbf{સિમ્બોલ} & ▷\textbar--- & ▷\textbar---◁ \\
\textbf{ફોરવર્ડ ઓપરેશન} & સરળતાથી કન્ડક્ટ કરે છે & સામાન્ય ડાયોડ જેવું જ \\
\textbf{રિવર્સ બ્રેકડાઉન} & ઉચ્ચ વોલ્ટેજ પર, નુકસાન કરે છે & નિયંત્રિત,
નોન-ડિસ્ટ્રક્ટિવ \\
\textbf{ડોપિંગ લેવલ} & મધ્યમ & ભારે ડોપિંગ \\
\textbf{ઓપરેટિંગ રીજન} & ફોરવર્ડ બાયસ્ડ & રિવર્સ બાયસ્ડ (બ્રેકડાઉન રીજન) \\
\textbf{ઉપયોગો} & રેક્ટિફિકેશન, સ્વિચિંગ & વોલ્ટેજ રેગ્યુલેશન, રેફરન્સ \\
\textbf{બ્રેકડાઉન મેકેનિઝમ} & એવલાન્ચ & ઝેનર ઇફેક્ટ અને એવલાન્ચ \\
\textbf{તાપમાન ગુણાંક} & નેગેટિવ & પોઝિટિવ અથવા નેગેટિવ હોઈ શકે છે \\
\end{longtable}
}

\end{solutionbox}
\begin{mnemonicbox}
``FORD'' - Forward Operation for Rectifiers, Diodes;
reverse operation for Zeners

\end{mnemonicbox}
\subsection*{પ્રશ્ન 4(અ) [3
ગુણ]}\label{uxaaauxab0uxab6uxaa8-4uxa85-3-uxa97uxaa3}

\textbf{ફોટો ડાયોડના કાર્ય સિદ્ધાંતનું વર્ણન કરો.}

\begin{solutionbox}

\textbf{ફોટોડાયોડના કાર્યસિદ્ધાંત:}

\begin{center}
\textbf{Mermaid Diagram (Code)}
\begin{verbatim}
{Shaded}
{Highlighting}[]
graph LR
    A[પ્રકાશ] {-{-}{} B[P{-}N જંક્શન]}
    B {-{-}{} C[ઇલેક્ટ્રોન{-}હોલ{}br /{}જોડીઓ]}
    C {-{-}{} D[ફોટોકરંટ]}
    style A fill:\#lightyellow
    style B fill:\#lightpink
    style C fill:\#lightblue
    style D fill:\#lightgreen
{Highlighting}
{Shaded}
\end{verbatim}
\end{center}

\begin{itemize}
\tightlist
\item
  \textbf{રચના}: પારદર્શક વિન્ડો અથવા લેન્સ સાથેનો P-N જંક્શન ડાયોડ
\item
  \textbf{ઓપરેશન}: પ્રકાશ ડિટેક્શન માટે રિવર્સ બાયસ્ડ ઓપરેશન
\item
  \textbf{ફોટોન એબ્સોર્પશન}: આવતા ફોટોન્સ ડિપ્લિશન રીજનમાં ઇલેક્ટ્રોન-હોલ જોડીઓ
  બનાવે છે
\item
  \textbf{કરંટ જનરેશન}: ઇલેક્ટ્રિક ફિલ્ડ કેરિયર્સને તેમના સંબંધિત ટર્મિનલ તરફ મોકલે છે,
  જેનાથી ફોટોકરંટ બને છે
\item
  \textbf{લાઇટ સેન્સિટિવિટી}: કરંટ પ્રકાશની તીવ્રતાના પ્રમાણમાં હોય છે
\end{itemize}

\end{solutionbox}
\begin{mnemonicbox}
``LIGER'' - Light Induces Generation of Electrons in
Reverse-bias

\end{mnemonicbox}
\subsection*{પ્રશ્ન 4(બ) [4
ગુણ]}\label{uxaaauxab0uxab6uxaa8-4uxaac-4-uxa97uxaa3}

\textbf{શોટકી બેરિયર ડાયોડની લાક્ષણિકતા સમજાવો.}

\begin{solutionbox}

\textbf{શોટકી બેરિયર ડાયોડની લાક્ષણિકતાઓ:}

\begin{verbatim}
                 I
                 \^{}
                 |              /
                 |             / Schottky
                 |            /
                 |           /   PN Junction
Forward current  |          / ,/
                 |         / /
                 |        / /
                 |       / /
                 |      / /
                 |     //
        {-V {-}{-}{-}{-}{-}|{-}{-}{-}{-}|{-}{-}{-} +V}
                 |    |
                 |    |
                 |    |
Reverse current  |    |
                 |    |
\end{verbatim}

\begin{itemize}
\tightlist
\item
  \textbf{ઓછો ફોરવર્ડ વોલ્ટેજ ડ્રોપ}: સિલિકોન PN જંક્શનના 0.7V ની તુલનામાં
  0.2-0.3V
\item
  \textbf{ફાસ્ટ સ્વિચિંગ}: કોઈ માઈનોરિટી કેરિયર સ્ટોરેજ નહીં, મિનિમલ રિવર્સ
  રિકવરી ટાઇમ
\item
  \textbf{રચના}: P-N જંક્શનને બદલે મેટલ-સેમિકન્ડક્ટર જંક્શન
\item
  \textbf{કોઈ રિવર્સ રિકવરી ટાઇમ નહીં}: મેજોરિટી કેરિયર ડિવાઇસ (કોઈ સ્ટોર્ડ
  ચાર્જ નહીં)
\item
  \textbf{ઉપયોગો}: હાઈ-ફ્રિક્વન્સી એપ્લિકેશન, પાવર સપ્લાયમાં રેક્ટિફાયર
\end{itemize}

\end{solutionbox}
\begin{mnemonicbox}
``FAST'' - Forward voltage low, Allows Switching
Timely

\end{mnemonicbox}
\subsection*{પ્રશ્ન 4(ક) [7
ગુણ]}\label{uxaaauxab0uxab6uxaa8-4uxa95-7-uxa97uxaa3}

\textbf{PNP અને NPN ટ્રાન્ઝિસ્ટરના કાર્ય સિદ્ધાંતને સમજાવો.}

\begin{solutionbox}

\textbf{NPN ટ્રાન્ઝિસ્ટરની સ્ટ્રક્ચર અને કાર્યપદ્ધતિ:}

\begin{verbatim}
    +{-{-}{-}{-}{-}{-}{-}+     +{-}{-}{-}{-}{-}{-}{-}+     +{-}{-}{-}{-}{-}{-}{-}+}
    |       |     |       |     |       |
    |   N   |     |   P   |     |   N   |
    |       |     |       |     |       |
    +{-{-}{-}{-}{-}{-}{-}+     +{-}{-}{-}{-}{-}{-}{-}+     +{-}{-}{-}{-}{-}{-}{-}+}
    Emitter        Base        Collector
        |            |            |
        |            |            |
        v            v            v
    Electron      Hole        Electron
     source    controller     collector
\end{verbatim}

\begin{itemize}
\tightlist
\item
  \textbf{બાયસિંગ}: ઇમિટર-બેઝ જંક્શન ફોરવર્ડ બાયસ્ડ, કલેક્ટર-બેઝ જંક્શન રિવર્સ બાયસ્ડ
\item
  \textbf{કરંટ ફ્લો}: ઇલેક્ટ્રોન્સ પાતળા બેઝ રીજન મારફતે ઇમિટરથી કલેક્ટર તરફ
\item
  \textbf{એમ્પલિફિકેશન સિદ્ધાંત}: નાનો બેઝ કરંટ મોટા કલેક્ટર કરંટને નિયંત્રિત કરે છે
\item
  \textbf{કરંટ સંબંધ}: IE = IB + IC
\item
  \textbf{મેજોરિટી કેરિયર્સ}: ઇલેક્ટ્રોન્સ
\end{itemize}

\textbf{PNP ટ્રાન્ઝિસ્ટરની સ્ટ્રક્ચર અને કાર્યપદ્ધતિ:}

\begin{verbatim}
    +{-{-}{-}{-}{-}{-}{-}+     +{-}{-}{-}{-}{-}{-}{-}+     +{-}{-}{-}{-}{-}{-}{-}+}
    |       |     |       |     |       |
    |   P   |     |   N   |     |   P   |
    |       |     |       |     |       |
    +{-{-}{-}{-}{-}{-}{-}+     +{-}{-}{-}{-}{-}{-}{-}+     +{-}{-}{-}{-}{-}{-}{-}+}
    Emitter        Base        Collector
        |            |            |
        |            |            |
        v            v            v
     Hole         Electron        Hole
     source     controller     collector
\end{verbatim}

\begin{itemize}
\tightlist
\item
  \textbf{બાયસિંગ}: ઇમિટર-બેઝ જંક્શન ફોરવર્ડ બાયસ્ડ, કલેક્ટર-બેઝ જંક્શન રિવર્સ બાયસ્ડ
\item
  \textbf{કરંટ ફ્લો}: હોલ્સ પાતળા બેઝ રીજન મારફતે ઇમિટરથી કલેક્ટર તરફ
\item
  \textbf{એમ્પલિફિકેશન સિદ્ધાંત}: નાનો બેઝ કરંટ મોટા કલેક્ટર કરંટને નિયંત્રિત કરે છે
\item
  \textbf{કરંટ સંબંધ}: IE = IB + IC
\item
  \textbf{મેજોરિટી કેરિયર્સ}: હોલ્સ
\item
  \textbf{કરંટ દિશા}: NPN કરતાં વિપરીત (કન્વેન્શનલ કરંટ ઇમિટરથી કલેક્ટર તરફ)
\end{itemize}

\end{solutionbox}
\begin{mnemonicbox}
``NPNP'' - Negative carriers in NPN, Positive
carriers in PNP

\end{mnemonicbox}
\subsection*{પ્રશ્ન 4(અ) OR [3
ગુણ]}\label{uxaaauxab0uxab6uxaa8-4uxa85-or-3-uxa97uxaa3}

\textbf{LED ના કાર્ય સિદ્ધાંતનું વર્ણન કરો.}

\begin{solutionbox}

\textbf{LED (લાઇટ ઇમિટિંગ ડાયોડ)ના કાર્યસિદ્ધાંત:}

\begin{center}
\textbf{Mermaid Diagram (Code)}
\begin{verbatim}
{Shaded}
{Highlighting}[]
graph LR
    A[ફોરવર્ડ બાયસ] {-{-}{} B[ઇલેક્ટ્રોન{-}હોલ{}br /{}રિકોમ્બિનેશન]}
    B {-{-}{} C[ફોટોન તરીકે{}br /{}ઊર્જા મુક્ત થાય]}
    C {-{-}{} D[પ્રકાશ ઉત્સર્જન]}
    style A fill:\#lightblue
    style B fill:\#lightpink
    style C fill:\#lightyellow
    style D fill:\#lightgreen
{Highlighting}
{Shaded}
\end{verbatim}
\end{center}

\begin{itemize}
\tightlist
\item
  \textbf{રચના}: ડાયરેક્ટ બેન્ડગેપ સેમિકન્ડક્ટર મટેરિયલથી બનેલા P-N જંક્શન
\item
  \textbf{ફોરવર્ડ બાયસિંગ}: n-રીજનમાંથી ઇલેક્ટ્રોન્સ અને p-રીજનમાંથી હોલ્સ જંક્શન પર
  રિકોમ્બાઇન થાય છે
\item
  \textbf{રિકોમ્બિનેશન}: ઇલેક્ટ્રોન કન્ડક્શન બેન્ડમાંથી વેલેન્સ બેન્ડમાં પડે છે
\item
  \textbf{ઊર્જા ઉત્સર્જન}: રિકોમ્બિનેશન દરમિયાન છૂટી પડેલી ઊર્જા ફોટોન્સ (પ્રકાશ)
  ઉત્સર્જિત કરે છે
\item
  \textbf{કલર ડિટરમિનેશન}: બેન્ડગેપ ઊર્જા ઉત્સર્જિત પ્રકાશની તરંગલંબાઈ (રંગ) નક્કી
  કરે છે
\end{itemize}

\end{solutionbox}
\begin{mnemonicbox}
``REBEL'' - Recombination of Electrons and holes By
Energetic Light emission

\end{mnemonicbox}
\subsection*{પ્રશ્ન 4(બ) OR [4
ગુણ]}\label{uxaaauxab0uxab6uxaa8-4uxaac-or-4-uxa97uxaa3}

\textbf{કટ ઓફ અને સેચ્યુરેશન રીજીયનમાં ટ્રાન્ઝિસ્ટરનું સ્વિચ તરીકે એપ્લિકેશન કાર્ય
સમજાવો.}

\begin{solutionbox}

\textbf{ટ્રાન્ઝિસ્ટર એઝ અ સ્વિચ:}

\begin{verbatim}
        Input                        Output
          |                            |
          |                            |
          v                            v
    +{-{-}{-}{-}{-}{-}{-}{-}{-}{-}+                 +{-}{-}{-}{-}{-}{-}{-}{-}{-}{-}+}
    |    R1    |                 |    RC    |
    +{-{-}{-}{-}{-}{-}{-}{-}{-}{-}+                 +{-}{-}{-}{-}{-}{-}{-}{-}{-}{-}+}
          |                            |
          |                            |
          |     +{-{-}{-}{-}{-}{-}{-}{-}{-}{-}{-}{-}{-}{-}{-}+      |}
          +{-{-}{-}{-}|     B     C   |{-}{-}{-}{-}{-}{-}+}
                |       |       |
                |       |       |
                |     E         |
                +{-{-}{-}{-}{-}{-}{-}{-}{-}{-}{-}{-}{-}{-}{-}+}
                     |
                     |
                 Ground
\end{verbatim}

\textbf{કટ-ઓફ રીજન (સ્વિચ OFF):}

\begin{itemize}
\tightlist
\item
  \textbf{બેઝ વોલ્ટેજ}: 0.7V (સિલિકોન માટે) થી નીચે
\item
  \textbf{બેઝ કરંટ}: લગભગ શૂન્ય
\item
  \textbf{કલેક્ટર કરંટ}: લગભગ શૂન્ય
\item
  \textbf{કલેક્ટર-ઇમિટર વોલ્ટેજ}: સપ્લાય વોલ્ટેજના બરાબર
\item
  \textbf{ઉપયોગો}: લોજિક ગેટ્સ, ડિજિટલ સર્કિટ, રિલે ડ્રાઇવર
\end{itemize}

\textbf{સેચ્યુરેશન રીજન (સ્વિચ ON):}

\begin{itemize}
\tightlist
\item
  \textbf{બેઝ વોલ્ટેજ}: 0.7V કરતાં ઘણું ઊંચું
\item
  \textbf{બેઝ કરંટ}: લઘુત્તમ VCE સુનિશ્ચિત કરવા માટે પર્યાપ્ત
\item
  \textbf{કલેક્ટર કરંટ}: મહત્તમ (કલેક્ટર રેઝિસ્ટર દ્વારા મર્યાદિત)
\item
  \textbf{કલેક્ટર-ઇમિટર વોલ્ટેજ}: ખૂબ જ ઓછું (0.2V - 0.3V)
\item
  \textbf{ઉપયોગો}: ડિજિટલ સ્વીચ, મોટર ડ્રાઇવર, LED ડ્રાઇવર
\end{itemize}

\end{solutionbox}
\begin{mnemonicbox}
``COSI'' - Cutoff Opens Switch, Input saturates to
close

\end{mnemonicbox}
\subsection*{પ્રશ્ન 4(ક) OR [7
ગુણ]}\label{uxaaauxab0uxab6uxaa8-4uxa95-or-7-uxa97uxaa3}

\textbf{C-E ટ્રાન્ઝિસ્ટર એમ્પ્લિફાયર રચના ટૂંકમાં સમજાવો. ટ્રાન્ઝિસ્ટર એમ્પ્લીફાયર
માટે α અને β વચ્ચેનો સંબંધ મેળવો.}

\begin{solutionbox}

\textbf{કોમન ઇમિટર કોન્ફિગરેશન:}

\begin{verbatim}
graph TB
    A[ઇનપુટ સિગ્નલ] {-{-} B[બેઝ]}
    C[આઉટપુટ સિગ્નલ] {-{-} D[કલેક્ટર]}
    E[ગ્રાઉન્ડ] {-{-} F[ઇમિટર]}
    style A fill:\#lightblue
    style B fill:\#lightpink
    style C fill:\#lightgreen
    style D fill:\#lightyellow
    style E fill:\#lightgray
    style F fill:\#lightcyan
\end{verbatim}

\textbf{કોમન ઇમિટર કોન્ફિગરેશનની લાક્ષણિકતાઓ:}

\begin{itemize}
\tightlist
\item
  \textbf{ઇનપુટ ટર્મિનલ}: બેઝ
\item
  \textbf{આઉટપુટ ટર્મિનલ}: કલેક્ટર
\item
  \textbf{કોમન ટર્મિનલ}: ઇમિટર (ગ્રાઉન્ડેડ)
\item
  \textbf{કરંટ ગેઇન (β)}: હાઈ (20-500)
\item
  \textbf{વોલ્ટેજ ગેઇન}: હાઈ (250-1000)
\item
  \textbf{ઇનપુટ ઇમ્પિડન્સ}: મધ્યમ (1-2kΩ)
\item
  \textbf{આઉટપુટ ઇમ્પિડન્સ}: હાઈ (30-50kΩ)
\item
  \textbf{ફેઝ શિફ્ટ}: 180^\circ (આઉટપુટ ઇનપુટથી ઇન્વર્ટેડ)
\end{itemize}

\textbf{α અને β વચ્ચેનો સંબંધ:}

વ્યાખ્યા પ્રમાણે:

\begin{itemize}
\tightlist
\item
  α = IC/IE (કોમન બેઝ કરંટ ગેઇન)
\item
  β = IC/IB (કોમન ઇમિટર કરંટ ગેઇન)
\end{itemize}

કિરચોફના કરંટ લૉ પરથી:

\begin{itemize}
\tightlist
\item
  IE = IB + IC
\end{itemize}

બંને બાજુને IE વડે ભાગીએ:

\begin{itemize}
\tightlist
\item
  1 = IB/IE + IC/IE
\item
  1 = IB/IE + α
\end{itemize}

તેથી:

\begin{itemize}
\tightlist
\item
  IB/IE = 1 - α
\end{itemize}

હવે,

β = IC/IB = (IC/IE)/(IB/IE) = α/(1-α)


અને તેથી ઉલટું:

\begin{itemize}
\tightlist
\item
  α = β/(1+β)
\end{itemize}

\end{solutionbox}
\begin{mnemonicbox}
``BEAR'' - Beta Equals Alpha divided by (1-alpha)
Relation

\end{mnemonicbox}
\subsection*{પ્રશ્ન 5(અ) [3
ગુણ]}\label{uxaaauxab0uxab6uxaa8-5uxa85-3-uxa97uxaa3}

\textbf{ઇ-વેસ્ટનો અર્થ શું છે? ઇ-કચરાના નિકાલની વિવિધ પદ્ધતિઓ શું છે?}

\begin{solutionbox}

\textbf{ઇ-વેસ્ટ (ઇલેક્ટ્રોનિક વેસ્ટ)}: ત્યજાયેલા ઇલેક્ટ્રોનિક ડિવાઇસ અને કમ્પોનન્ટ્સ જે
તેમના જીવનકાળનાં અંતે પહોંચ્યા છે અથવા હવે ઉપયોગી નથી.

\textbf{ઇ-વેસ્ટ નિકાલની પદ્ધતિઓ:}

{\def\LTcaptype{none} % do not increment counter
\begin{longtable}[]{@{}
  >{\raggedright\arraybackslash}p{(\linewidth - 2\tabcolsep) * \real{0.5000}}
  >{\raggedright\arraybackslash}p{(\linewidth - 2\tabcolsep) * \real{0.5000}}@{}}
\toprule\noalign{}
\begin{minipage}[b]{\linewidth}\raggedright
\textbf{નિકાલ પદ્ધતિ}
\end{minipage} & \begin{minipage}[b]{\linewidth}\raggedright
\textbf{વર્ણન}
\end{minipage} \\
\midrule\noalign{}
\endhead
\bottomrule\noalign{}
\endlastfoot
\textbf{રિસાયક્લિંગ} & મૂલ્યવાન સામગ્રી જેમ કે ધાતુઓ, પ્લાસ્ટિકને પુન:ઉપયોગ માટે અલગ
કરવી \\
\textbf{લેન્ડફિલિંગ} & નિયુક્ત લેન્ડફિલ્સમાં નિકાલ (ભલામણ કરાતી નથી) \\
\textbf{ઇન્સિનરેશન} & ઉચ્ચ તાપમાને કચરાનું દહન (ઝેરી ઉત્સર્જન બનાવે છે) \\
\textbf{રિયુઝ/રિફર્બિશમેન્ટ} & વિસ્તારિત ઉપયોગ માટે રિપેરિંગ અને અપગ્રેડિંગ \\
\textbf{ઇક્સટેન્ડેડ પ્રોડ્યુસર રિસ્પોન્સિબિલિટી} & ઉત્પાદકો પાછા લે અને નિકાલ સંભાળે
છે \\
\end{longtable}
}

\end{solutionbox}
\begin{mnemonicbox}
``RIPER'' - Recycling Is Preferable to
Environmentally-harmful Remedies

\end{mnemonicbox}
\subsection*{પ્રશ્ન 5(બ) [4
ગુણ]}\label{uxaaauxab0uxab6uxaa8-5uxaac-4-uxa97uxaa3}

\textbf{ઉદાહરણો સાથે ઈલેક્ટ્રોનિક કચરાનું સંચાલન કરવાની પદ્ધતિઓ સમજાવો.}

\begin{solutionbox}

\textbf{ઇલેક્ટ્રોનિક વેસ્ટ હેન્ડલિંગની પદ્ધતિઓ:}

\begin{center}
\textbf{Mermaid Diagram (Code)}
\begin{verbatim}
{Shaded}
{Highlighting}[]
graph LR
    A[ઇ{-વેસ્ટ{}br /{}કલેક્શન] {-}{-}{} B[સોર્ટિંગ]}
    B {-{-}{} C[ડિસમેન્ટલિંગ]}
    C {-{-}{} D[મટેરિયલ{}br /{}રિકવરી]}
    D {-{-}{} E[સેફ{}br /{}ડિસ્પોઝલ]}
    style A fill:\#lightblue
    style B fill:\#lightpink
    style C fill:\#lightyellow
    style D fill:\#lightgreen
    style E fill:\#lightgray
{Highlighting}
{Shaded}
\end{verbatim}
\end{center}

\textbf{કલેક્શન અને સેગ્રિગેશન:}

\begin{itemize}
\tightlist
\item
  \textbf{ઉદાહરણ}: જાહેર સ્થળોએ સમર્પિત ઇ-વેસ્ટ બિન્સ, ઇ-વેસ્ટ કલેક્શન ડ્રાઇવ્સ
\item
  \textbf{લાભ}: સામાન્ય કચરા સાથે મિશ્રણ અટકાવે છે, યોગ્ય પ્રોસેસિંગ સક્ષમ કરે છે
\end{itemize}

\textbf{ડિસમેન્ટલિંગ અને રિસોર્સ રિકવરી:}

\begin{itemize}
\tightlist
\item
  \textbf{ઉદાહરણ}: સર્કિટ બોર્ડ અને કનેક્ટર્સમાંથી સોનું, ચાંદી, કોપર રિકવર કરવા
\item
  \textbf{લાભ}: મૂલ્યવાન ધાતુઓ પુન:પ્રાપ્ત કરે છે, માઇનિંગની માંગ ઘટાડે છે
\end{itemize}

\textbf{રિફર્બિશમેન્ટ અને રિયુઝ:}

\begin{itemize}
\tightlist
\item
  \textbf{ઉદાહરણ}: શૈક્ષણિક સંસ્થાઓ માટે જૂના કમ્પ્યુટર્સની મરામત
\item
  \textbf{લાભ}: પ્રોડક્ટ લાઇફસાયકલ વિસ્તૃત કરે છે, કચરા ઉત્પાદન ઘટાડે છે
\end{itemize}

\textbf{હાનિકારક કમ્પોનન્ટ્સનો યોગ્ય નિકાલ:}

\begin{itemize}
\tightlist
\item
  \textbf{ઉદાહરણ}: મર્ક્યુરી-ધરાવતા કમ્પોનન્ટ્સ માટે સ્પેશિયલાઇઝ્ડ ટ્રીટમેન્ટ
\item
  \textbf{લાભ}: ઝેરી પદાર્થોને પર્યાવરણમાં પ્રવેશતા અટકાવે છે
\end{itemize}

\end{solutionbox}
\begin{mnemonicbox}
``CREED'' - Collect, Recover, Extract, Extend,
Dispose safely

\end{mnemonicbox}
\subsection*{પ્રશ્ન 5(ક) [7
ગુણ]}\label{uxaaauxab0uxab6uxaa8-5uxa95-7-uxa97uxaa3}

\textbf{રિપલ ફેક્ટર શું છે? રેક્ટિફાયર માટે રિપલ ફેક્ટરનું સમીકરણ મેળવો.}

\begin{solutionbox}

\textbf{રિપલ ફેક્ટર}: રેક્ટિફાયરની ફિલ્ટરિંગની અસરકારકતાનું માપ - આઉટપુટમાં AC
કમ્પોનન્ટ (રિપલ)નો DC કમ્પોનન્ટ સાથેનો ગુણોત્તર.

\textbf{વ્યાખ્યા}:

\begin{itemize}
\tightlist
\item
  રિપલ ફેક્ટર (γ) = AC કમ્પોનન્ટની RMS વેલ્યુ / DC વેલ્યુ
\item
  ઓછો રિપલ ફેક્ટર વધુ સારા ફિલ્ટરિંગનો સંકેત આપે છે
\end{itemize}

\textbf{હાફ વેવ રેક્ટિફાયર માટે ડેરિવેશન:}

ચાલો ધારીએ કે સાઇન્યુસોઇડલ ઇનપુટ: v = Vmsinωt

હાફ વેવ રેક્ટિફાયર માટે:

\begin{itemize}
\tightlist
\item
  આઉટપુટ v = Vmsinωt જ્યારે 0 \leq ωt \leq π
\item
  આઉટપુટ v = 0 જ્યારે π \leq ωt \leq 2π
\end{itemize}

\textbf{સ્ટેપ 1}: DC કમ્પોનન્ટ (એવરેજ વેલ્યુ) શોધો

\begin{itemize}
\tightlist
\item
  VDC = (1/2π) \int02π v(ωt) d(ωt)
\item
  VDC = (1/2π) \int0π Vmsinωt d(ωt)
\item
  VDC = Vm/π
\end{itemize}

\textbf{સ્ટેપ 2}: RMS વેલ્યુ શોધો

\begin{itemize}
\tightlist
\item
  VRMS = \sqrt[(1/2π) \int02π v^{2}(ωt) d(ωt)]
\item
  VRMS = \sqrt[(1/2π) \int0π Vm^{2}sin^{2}ωt d(ωt)]
\item
  VRMS = Vm/2
\end{itemize}

\textbf{સ્ટેપ 3}: AC કમ્પોનન્ટ શોધો

\begin{itemize}
\tightlist
\item
  VAC^{2} = VRMS^{2} - VDC^{2}
\item
  VAC^{2} = (Vm/2)^{2} - (Vm/π)^{2}
\item
  VAC^{2} = Vm^{2}(1/4 - 1/π^{2})
\end{itemize}

\textbf{સ્ટેપ 4}: રિપલ ફેક્ટર ગણો

\begin{itemize}
\tightlist
\item
  γ = VAC/VDC
\item
  γ = \sqrt(Vm^{2}(1/4 - 1/π^{2}))/(Vm/π)
\item
  γ = π\sqrt(1/4 - 1/π^{2})
\item
  γ = 1.21 (હાફ વેવ રેક્ટિફાયર માટે)
\end{itemize}

\textbf{ફુલ વેવ રેક્ટિફાયર માટે}: સમાન પગલાં અનુસરીને γ = 0.48 મળે છે

\end{solutionbox}
\begin{mnemonicbox}
``ROAD'' - Ripple is Output's AC Divided by DC
component

\end{mnemonicbox}
\subsection*{પ્રશ્ન 5(અ) OR [3
ગુણ]}\label{uxaaauxab0uxab6uxaa8-5uxa85-or-3-uxa97uxaa3}

\textbf{ઈ-વેસ્ટમાં કયા ઝેરી પદાર્થો હોય છે?}

\begin{solutionbox}

\textbf{ઇ-વેસ્ટમાં ઝેરી પદાર્થો:}

{\def\LTcaptype{none} % do not increment counter
\begin{longtable}[]{@{}
  >{\raggedright\arraybackslash}p{(\linewidth - 4\tabcolsep) * \real{0.3333}}
  >{\raggedright\arraybackslash}p{(\linewidth - 4\tabcolsep) * \real{0.3333}}
  >{\raggedright\arraybackslash}p{(\linewidth - 4\tabcolsep) * \real{0.3333}}@{}}
\toprule\noalign{}
\begin{minipage}[b]{\linewidth}\raggedright
\textbf{ઝેરી પદાર્થ}
\end{minipage} & \begin{minipage}[b]{\linewidth}\raggedright
\textbf{ઇલેક્ટ્રોનિક્સમાં સ્ત્રોત}
\end{minipage} & \begin{minipage}[b]{\linewidth}\raggedright
\textbf{આરોગ્ય/પર્યાવરણીય અસર}
\end{minipage} \\
\midrule\noalign{}
\endhead
\bottomrule\noalign{}
\endlastfoot
\textbf{લેડ (Pb)} & સોલ્ડર, CRT મોનિટર, બેટરીઓ & ન્યુરોલોજીકલ નુકસાન,
વિકાસાત્મક સમસ્યાઓ \\
\textbf{મર્ક્યુરી (Hg)} & સ્વિચ, બેકલાઇટ્સ, બેટરીઓ & ન્યુરોલોજીકલ અને કિડનીને
નુકસાન \\
\textbf{કેડમિયમ (Cd)} & રિચાર્જેબલ બેટરીઓ, સર્કિટ બોર્ડ & કિડનીને નુકસાન,
હાડકાના રોગો \\
\textbf{બ્રોમિનેટેડ ફ્લેમ રિટાર્ડન્ટ્સ} & પ્લાસ્ટિક કેસિંગ, સર્કિટ બોર્ડ & એન્ડોક્રાઇન
ડિસ્રપ્શન, બાયોએક્યુમ્યુલેશન \\
\textbf{હેક્સાવેલેન્ટ ક્રોમિયમ} & મેટલ પાર્ટ્સમાં કોરોઝન પ્રોટેક્શન & એલર્જીક રિએક્શન,
DNA નુકસાન \\
\textbf{બેરિલિયમ (Be)} & કનેક્ટર્સ, સ્પ્રિંગ્સ & ફેફસાના રોગ, ત્વચાના વિકાર \\
\end{longtable}
}

\end{solutionbox}
\begin{mnemonicbox}
``LMBCHB'' - Lead, Mercury, and Beryllium Cause
Harmful Bodily effects

\end{mnemonicbox}
\subsection*{પ્રશ્ન 5(બ) OR [4
ગુણ]}\label{uxaaauxab0uxab6uxaa8-5uxaac-or-4-uxa97uxaa3}

\textbf{તમારી એપ્લિકેશન માટે યોગ્ય ટ્રાન્ઝિસ્ટર પસંદ કરવા માટેના મહત્વપૂર્ણ પરિમાણો
લખો અને કોઈપણ બે સમજાવો.}

\begin{solutionbox}

\textbf{મહત્વપૂર્ણ ટ્રાન્ઝિસ્ટર સિલેક્શન પેરામીટર્સ:}

\begin{itemize}
\tightlist
\item
  મહત્તમ કલેક્ટર કરંટ (IC)
\item
  મહત્તમ કલેક્ટર-ઇમિટર વોલ્ટેજ (VCEO)
\item
  મહત્તમ કલેક્ટર-બેઝ વોલ્ટેજ (VCBO)
\item
  કરંટ ગેઇન (hFE અથવા β)
\item
  ફ્રિક્વન્સી રિસ્પોન્સ (fT)
\item
  પાવર ડિસિપેશન (Ptot)
\item
  પેકેજ ટાઇપ (TO-3, SMT, વગેરે)
\item
  તાપમાન રેન્જ
\end{itemize}

\textbf{મહત્તમ કલેક્ટર કરંટ (IC):}

\begin{itemize}
\tightlist
\item
  \textbf{વ્યાખ્યા}: નુકસાન વિના કલેક્ટર મારફતે વહી શકે તેવો મહત્તમ કરંટ
\item
  \textbf{મહત્વ}: એપ્લિકેશનની પીક કરંટ જરૂરિયાતોને સેફ્ટી માર્જિન સાથે વટાવવો જોઈએ
\item
  \textbf{સામાન્ય વેલ્યુ}: ટ્રાન્ઝિસ્ટર પ્રકાર પર આધારિત 100mA થી 100A
\item
  \textbf{એપ્લિકેશન કન્સિડરેશન}: મહત્તમ જરૂરી કરંટ કરતાં 50\% વધુ રેટિંગ પસંદ કરવી
\end{itemize}

\textbf{કરંટ ગેઇન (hFE અથવા β):}

\begin{itemize}
\tightlist
\item
  \textbf{વ્યાખ્યા}: કલેક્ટર કરંટનો બેઝ કરંટ સાથેનો ગુણોત્તર
\item
  \textbf{મહત્વ}: એમ્પલિફિકેશન ક્ષમતા અને જરૂરી બેઝ ડ્રાઇવ નક્કી કરે છે
\item
  \textbf{સામાન્ય વેલ્યુ}: સામાન્ય-હેતુના ટ્રાન્ઝિસ્ટર માટે 20-500
\item
  \textbf{એપ્લિકેશન કન્સિડરેશન}: સ્વિચિંગ માટે, ઉચ્ચ ગેઇન બેઝ કરંટની જરૂરિયાત ઘટાડે
  છે; એમ્પલિફાયર માટે, ઓપરેટિંગ રેન્જમાં સુસંગત ગેઇન મહત્વપૂર્ણ છે
\end{itemize}

\end{solutionbox}
\begin{mnemonicbox}
``GIVE'' - Gain and Ic are Very Essential parameters

\end{mnemonicbox}
\subsection*{પ્રશ્ન 5(ક) OR [7
ગુણ]}\label{uxaaauxab0uxab6uxaa8-5uxa95-or-7-uxa97uxaa3}

\textbf{રેક્ટિફાયર કાર્યક્ષમતા શું છે? ફુલ વેવ રેક્ટિફાયરની કાર્યક્ષમતા શોધો.}

\begin{solutionbox}

\textbf{રેક્ટિફાયર કાર્યક્ષમતા}: DC આઉટપુટ પાવરનો AC ઇનપુટ પાવર સાથેનો ગુણોત્તર,
ટકાવારીમાં વ્યક્ત.

\textbf{વ્યાખ્યા}:

\begin{itemize}
\tightlist
\item
  કાર્યક્ષમતા (η) = (PDC/PAC) \times 100\%
\item
  ઉચ્ચ કાર્યક્ષમતા એટલે AC થી DC માં વધુ સારું રૂપાંતરણ
\end{itemize}

\textbf{ફુલ વેવ રેક્ટિફાયર માટે ડેરિવેશન:}

\textbf{સ્ટેપ 1}: DC આઉટપુટ પાવર ગણો

\begin{itemize}
\tightlist
\item
  IDC = VDC/RL
\item
  PDC = IDC^{2} \times RL = VDC^{2}/RL
\item
  ફુલ વેવ માટે, VDC = 2Vm/π
\item
  PDC = (2Vm/π)^{2}/RL = 4Vm^{2}/(π^{2}RL)
\end{itemize}

\textbf{સ્ટેપ 2}: AC ઇનપુટ પાવર ગણો

\begin{itemize}
\tightlist
\item
  IRMS = VRMS/RL
\item
  PAC = IRMS^{2} \times RL = VRMS^{2}/RL
\item
  સાઇન વેવ માટે, VRMS = Vm/\sqrt2
\item
  PAC = (Vm/\sqrt2)^{2}/RL = Vm^{2}/(2RL)
\end{itemize}

\textbf{સ્ટેપ 3}: કાર્યક્ષમતા ગણો

\begin{itemize}
\tightlist
\item
  η = (PDC/PAC) \times 100\%
\item
  η = [4Vm^{2}/(π^{2}RL)] / [Vm^{2}/(2RL)] \times 100\%
\item
  η = [4/(π^{2})] \times 2 \times 100\%
\item
  η = 8/(π^{2}) \times 100\%
\item
  η = 8/9.87 \times 100\%
\item
  η = 81.2\%
\end{itemize}

\textbf{ફુલ વેવ રેક્ટિફાયર કાર્યક્ષમતા} = 81.2\%

\textbf{તુલના માટે:}

\begin{itemize}
\tightlist
\item
  હાફ વેવ રેક્ટિફાયર કાર્યક્ષમતા = 40.6\%
\item
  બ્રિજ રેક્ટિફાયર કાર્યક્ષમતા = 81.2\%
\end{itemize}

\end{solutionbox}
\begin{mnemonicbox}
``PIDE'' - Power Input Determines Efficiency

\end{mnemonicbox}

\end{document}
