\documentclass[10pt,a4paper]{article}

% content/resources/templates/preamble.tex
\usepackage[margin=0.6in]{geometry}
\author{Milav Dabgar}
\usepackage{amsmath,amssymb,amsthm}
\usepackage{booktabs}
\usepackage{multirow}
\usepackage{xcolor}
\usepackage{tcolorbox}
\tcbuselibrary{breakable,skins}
\usepackage[colorlinks=true,linkcolor=blue]{hyperref}
\usepackage{titlesec}
\usepackage{enumitem}
\usepackage{tikz}
\usepackage{pgfplots}
\usepackage{circuitikz}
\usepackage[version=4]{mhchem}
\usepackage{longtable}
\usepackage{array}
\usepackage{float}
\usepackage{caption}
\usepackage{listings}

\lstset{
  basicstyle=\small\ttfamily,
  breaklines=true,
  breakatwhitespace=false,
  postbreak=\mbox{\textcolor{red}{$\hookrightarrow$}\space},
  float=false,
  numbers=left,
  numberstyle=\tiny\color{gray},
  numbersep=10pt,
  xleftmargin=2em,
  keywordstyle=\color{blue},
  commentstyle=\color{green!60!black},
  stringstyle=\color{purple},
  backgroundcolor=\color{gray!5},
  showstringspaces=false,
  tabsize=2,
  captionpos=b,
  keepspaces=true,
  columns=flexible
}

\pgfplotsset{compat=1.18}
\usetikzlibrary{shapes,arrows,positioning,calc,patterns,decorations.pathmorphing,decorations.markings,arrows.meta}

% Color scheme
\definecolor{headcolor}{RGB}{0,102,204}
\definecolor{keycolor}{RGB}{220,20,60}
\definecolor{solutioncolor}{RGB}{34,139,34}
\definecolor{mnemoniccolor}{RGB}{148,0,211}
\definecolor{codecolor}{RGB}{0,0,100}

% Spacing
\setlength{\parskip}{3pt}
\setlist[itemize]{nosep}
\setlist[enumerate]{nosep}

% Title formatting
\titleformat{\section}{\Large\bfseries\color{headcolor}}{\thesection}{1em}{}
\titleformat{\subsection}{\large\bfseries\color{headcolor}}{\thesubsection}{1em}{}

% Pandoc tightlist compatibility
\providecommand{\tightlist}{%
  \setlength{\itemsep}{0pt}\setlength{\parskip}{0pt}}

% Pandoc longtable compatibility
\newcounter{none}
\def\thenone{}


% content/resources/templates/english-boxes.tex
% This file is currently empty - it exists to maintain consistency with the import structure.
% Add custom environments here if needed in the future.


\begin{document}

\begin{center}
{\Huge\bfseries\color{headcolor} Subject Name Solutions}\\[5pt]
{\LARGE 4321103 -- Winter 2024}\\[3pt]
{\large Semester 1 Study Material}\\[3pt]
{\normalsize\textit{Detailed Solutions and Explanations}}
\end{center}

\vspace{10pt}

\subsection*{Question 1(a) [3 marks]}\label{q1a}

\textbf{Explain amplifier parameters Ai, Ri and Ro for CE
configuration.}

\begin{solutionbox}

Common Emitter (CE) amplifier parameters:


{\def\LTcaptype{none} % do not increment counter
\vspace{-5pt}
\captionof{table}{CE Amplifier Parameters}
\vspace{-10pt}
\begin{longtable}[]{@{}
  >{\raggedright\arraybackslash}p{(\linewidth - 4\tabcolsep) * \real{0.2750}}
  >{\raggedright\arraybackslash}p{(\linewidth - 4\tabcolsep) * \real{0.3000}}
  >{\raggedright\arraybackslash}p{(\linewidth - 4\tabcolsep) * \real{0.4250}}@{}}
\toprule\noalign{}
\begin{minipage}[b]{\linewidth}\raggedright
Parameter
\end{minipage} & \begin{minipage}[b]{\linewidth}\raggedright
Definition
\end{minipage} & \begin{minipage}[b]{\linewidth}\raggedright
CE Configuration
\end{minipage} \\
\midrule\noalign{}
\endhead
\bottomrule\noalign{}
\endlastfoot
\textbf{Current Gain (Ai)} & Ratio of output current to input current &
High (20-500) \\
\textbf{Input Resistance (Ri)} & Opposition to current flow at input &
Medium (1-2 kΩ) \\
\textbf{Output Resistance (Ro)} & Opposition to current flow at output &
High (40-50 kΩ) \\
\end{longtable}
}

\textbf{Diagram:}

\begin{center}
\textbf{Mermaid Diagram (Code)}
\begin{verbatim}
{Shaded}
{Highlighting}[]
graph LR
    I[Input Signal] {-{-}{} R[Ri: 1{-}2 kΩ] {-}{-}{} A[CE Amplifier] {-}{-}{} O[Output Signal]}
    A {-{-}{} RO[Ro: 40{-}50 kΩ]}
    A {-{-} "Ai: 20{-}500" {-}{-}{} O}
{Highlighting}
{Shaded}
\end{verbatim}
\end{center}

\end{solutionbox}
\begin{mnemonicbox}
``CAR'' - CE has Current gain high, Average input
resistance, and Robust output resistance.

\end{mnemonicbox}
\subsection*{Question 1(b) [4 marks]}\label{q1b}

\textbf{Write short-note on heat sink.}

\begin{solutionbox}

\textbf{Heat Sink: Device that absorbs and dissipates heat from
electronic components}

\textbf{Diagram:}

\begin{center}
\textbf{Mermaid Diagram (Code)}
\begin{verbatim}
{Shaded}
{Highlighting}[]
graph LR
    T[Transistor] {-{-}{} HS[Heat Sink]}
    HS {-{-} "Heat Dissipation" {-}{-}{} A[Ambient Air]}

    subgraph Heat Sink Structure
    direction LR
    F[Fins] {-{-}{-} B[Base]}
    end
{Highlighting}
{Shaded}
\end{verbatim}
\end{center}

\textbf{Types of Heat Sinks:}

\begin{itemize}
\tightlist
\item
  \textbf{Passive Heat Sinks}: Rely on natural convection
\item
  \textbf{Active Heat Sinks}: Use fans for forced air convection
\item
  \textbf{Liquid-cooled Heat Sinks}: Use liquid for better heat transfer
\end{itemize}

\textbf{Key Functions:}

\begin{itemize}
\tightlist
\item
  \textbf{Thermal Conduction}: Draws heat away from components
\item
  \textbf{Thermal Convection}: Transfers heat to surrounding air
\item
  \textbf{Surface Area}: Fins increase surface area for better cooling
\end{itemize}

\end{solutionbox}
\begin{mnemonicbox}
``CRAFT'' - Cooling through Radiation And Fins for
Transistors.

\end{mnemonicbox}
\subsection*{Question 1(c) [7 marks]}\label{q1c}

\textbf{Describe Thermal Runaway and Thermal Stability. How can overcome
thermal run away in transistor?}

\begin{solutionbox}

\textbf{Thermal Runaway:} Self-reinforcing process where increased
temperature causes more current flow, which further increases
temperature

\textbf{Thermal Stability:} Ability of a transistor circuit to maintain
stable operation despite temperature changes

\textbf{Diagram:}

\begin{center}
\textbf{Mermaid Diagram (Code)}
\begin{verbatim}
{Shaded}
{Highlighting}[]
graph LR
    A[Increased Temperature] {-{-}{} B[Increased Collector Current]}
    B {-{-}{} C[More Power Dissipation]}
    C {-{-}{} A}

    D[Thermal Stability Methods] {-{-}{} E[Break This Cycle]}
{Highlighting}
{Shaded}
\end{verbatim}
\end{center}

\textbf{Methods to Overcome Thermal Runaway:}

\begin{itemize}
\tightlist
\item
  \textbf{Heat Sink}: Absorbs and dissipates excess heat
\item
  \textbf{Negative Feedback}: Using emitter resistor for stabilization
\item
  \textbf{Bias Stabilization}: Voltage divider biasing circuit
\item
  \textbf{Temperature Compensation}: Using diodes or thermistors
\end{itemize}

\textbf{Key Points:}

\begin{itemize}
\tightlist
\item
  \textbf{IC = ICBO(1+β) + βIB}: Shows collector current dependence
\item
  \textbf{ICBO doubles}: For every 10^\circC temperature rise
\item
  \textbf{Stability Factor S}: Lower S means better stability
\end{itemize}

\end{solutionbox}
\begin{mnemonicbox}
``RENT'' - Reduce heat with sinks, Emitter resistors
stabilize, Negative feedback helps, Temperature compensation.

\end{mnemonicbox}
\subsection*{Question 1(c) OR [7
marks]}\label{q1c}

\textbf{Write down types of biasing methods. Explain the voltage divider
biasing method in details.}

\begin{solutionbox}

\textbf{Types of Biasing Methods:}

\begin{itemize}
\tightlist
\item
  Fixed Bias
\item
  Collector-to-Base Bias
\item
  Voltage Divider Bias
\item
  Emitter Bias
\item
  Collector Feedback Bias
\end{itemize}

\textbf{Voltage Divider Bias Circuit:}

\begin{verbatim}
    +Vcc
     |
     R1
     |
     +{-{-}{-}{-}+}
     |    |
     R2   |
     |    |
     +    |
     |    |
GND {-+{-}{-}+{-}+{-}{-}+{-} B}
         |    |
         |    C
         |    |
         +{-{-}{-}{-}+}
         |    E
         RE   |
         |    |
        GND  GND
\end{verbatim}

\textbf{Operation:}

\begin{itemize}
\tightlist
\item
  \textbf{R1 and R2}: Form voltage divider providing base voltage
\item
  \textbf{RE}: Provides stability and negative feedback
\item
  \textbf{Stable Bias Point}: Less affected by temperature and β
  variations
\end{itemize}

\textbf{Advantages:}

\begin{itemize}
\tightlist
\item
  \textbf{Excellent Stability}: Less affected by temperature variations
\item
  \textbf{Independent of β}: Bias point not greatly affected by
  transistor gain
\item
  \textbf{Widely Used}: Most common biasing method for amplifiers
\end{itemize}

\end{solutionbox}
\begin{mnemonicbox}
``DIVE'' - Divider biasing Is Very Effective for
stability.

\end{mnemonicbox}
\subsection*{Question 2(a) [3 marks]}\label{q2a}

\textbf{Explain Stability Factor with features.}

\begin{solutionbox}

\textbf{Stability Factor (S): Measure of how well a biasing circuit
maintains stable operation with temperature changes}

\textbf{Mathematical Definition:} S = ΔIC/ΔICBO (Change in collector
current / Change in reverse saturation current)


{\def\LTcaptype{none} % do not increment counter
\vspace{-5pt}
\captionof{table}{Stability Factors for Different Bias Circuits}
\vspace{-10pt}
\begin{longtable}[]{@{}lll@{}}
\toprule\noalign{}
Biasing Method & Stability Factor & Stability Level \\
\midrule\noalign{}
\endhead
\bottomrule\noalign{}
\endlastfoot
Fixed Bias & S = 1+β & Poor \\
Collector-to-Base & S = β/(1+β) & Better \\
Voltage Divider & S \approx 1 & Excellent \\
\end{longtable}
}

\textbf{Key Features:}

\begin{itemize}
\tightlist
\item
  \textbf{Lower S Value}: Indicates better stability (ideal S=1)
\item
  \textbf{Temperature Resistance}: Measures immunity to temperature
  changes
\item
  \textbf{Circuit Design Tool}: Helps compare biasing methods
\end{itemize}

\end{solutionbox}
\begin{mnemonicbox}
``SOS'' - Stability Of circuit Shows in its S-factor.

\end{mnemonicbox}
\subsection*{Question 2(b) [4 marks]}\label{q2b}

\textbf{Describe direct coupling technique of cascading.}

\begin{solutionbox}

\textbf{Direct Coupling: Connecting stages without coupling capacitors,
directly connecting collector of one stage to base of next}

\textbf{Diagram:}

\begin{verbatim}
      +Vcc                +Vcc
        |                   |
        |                   |
        Rc                  Rc
        |                   |
  +{-{-}{-}{-}{-}+                   +{-}{-}{-}{-}{-}+}
  |     |                   |     |
  |     C       B           |     C  Output
  |     |{-{-}{-}{-}{-}{-}{-}+           |     |{-}{-}{-}{-}{-}{-}{-}+}
  |     |       |           |     |
Input   |       |           |     |
  +{-{-}{-}{-}{-}|B      |           |     |}
        |       |           |     |
        |       E           |     E
        |       |           |     |
       GND     GND         GND   GND
       
       First Stage          Second Stage
\end{verbatim}

\textbf{Key Characteristics:}

\begin{itemize}
\tightlist
\item
  \textbf{No Coupling Components}: Direct electrical connection
\item
  \textbf{Full Frequency Response}: Good low-frequency performance
\item
  \textbf{DC Level Shifting}: Required between stages
\end{itemize}

\textbf{Applications:}

\begin{itemize}
\tightlist
\item
  \textbf{Operational Amplifiers}: Internal stages
\item
  \textbf{DC Amplifiers}: Where low-frequency response is critical
\end{itemize}

\end{solutionbox}
\begin{mnemonicbox}
``DIRECT'' - DC signals Immediately REach Connecting
Transistors.

\end{mnemonicbox}
\subsection*{Question 2(c) [7 marks]}\label{q2c}

\textbf{Explain frequency response of two stage RC coupled amplifier.}

\begin{solutionbox}

\textbf{RC Coupled Amplifier: Uses resistor-capacitor networks to couple
between amplification stages}

\textbf{Frequency Response Diagram:}

\begin{center}
\textbf{Mermaid Diagram (Code)}
\begin{verbatim}
{Shaded}
{Highlighting}[]
graph LR
    subgraph Frequency Response
    L[Low Frequency] {-{-}{-} M[Mid Frequency] {-}{-}{-} H[High Frequency]}
    end

    L {-{-} "20Hz{-}500Hz{}br /{}Gain rises" {-}{-}{} M}
    M {-{-} "500Hz{-}20kHz{}br /{}Flat gain" {-}{-}{} H}
    H {-{-} "{}20kHz{}br /{}Gain falls" {-}{-}{} D[Drop{-}off]}
{Highlighting}
{Shaded}
\end{verbatim}
\end{center}


{\def\LTcaptype{none} % do not increment counter
\vspace{-5pt}
\captionof{table}{Frequency Regions}
\vspace{-10pt}
\begin{longtable}[]{@{}
  >{\raggedright\arraybackslash}p{(\linewidth - 6\tabcolsep) * \real{0.1270}}
  >{\raggedright\arraybackslash}p{(\linewidth - 6\tabcolsep) * \real{0.2698}}
  >{\raggedright\arraybackslash}p{(\linewidth - 6\tabcolsep) * \real{0.2698}}
  >{\raggedright\arraybackslash}p{(\linewidth - 6\tabcolsep) * \real{0.3333}}@{}}
\toprule\noalign{}
\begin{minipage}[b]{\linewidth}\raggedright
Region
\end{minipage} & \begin{minipage}[b]{\linewidth}\raggedright
Frequency Range
\end{minipage} & \begin{minipage}[b]{\linewidth}\raggedright
Characteristics
\end{minipage} & \begin{minipage}[b]{\linewidth}\raggedright
Limiting Components
\end{minipage} \\
\midrule\noalign{}
\endhead
\bottomrule\noalign{}
\endlastfoot
\textbf{Low} & 20Hz-500Hz & Gain rises with frequency & Coupling
capacitors \\
\textbf{Mid} & 500Hz-20kHz & Constant gain (maximum) & None \\
\textbf{High} & \textgreater20kHz & Gain falls with frequency &
Transistor capacitance \\
\end{longtable}
}

\textbf{Two-Stage Effect:}

\begin{itemize}
\tightlist
\item
  \textbf{Bandwidth}: Narrower than single stage
\item
  \textbf{Gain}: Approximately square of single stage (A_{1} \times A_{2})
\item
  \textbf{Phase Shift}: Doubled at low and high frequencies
\end{itemize}

\end{solutionbox}
\begin{mnemonicbox}
``LMH'' - Low frequencies by coupling caps, Mid
frequencies flat, High frequencies by transistor caps.

\end{mnemonicbox}
\subsection*{Question 2(a) OR [3
marks]}\label{q2a}

\textbf{Briefly explain bandwidth and gain-bandwidth product of an
amplifier.}

\begin{solutionbox}

\textbf{Bandwidth (BW): Range of frequencies where amplifier gain is at
least 70.7\% of maximum gain}

\textbf{Gain-Bandwidth Product (GBP): Product of voltage gain and
bandwidth, constant for a given amplifier}

\textbf{Diagram:}

\begin{center}
\textbf{Mermaid Diagram (Code)}
\begin{verbatim}
{Shaded}
{Highlighting}[]
graph LR
    F[Frequency] {-{-}{} G[Gain]}

    subgraph Bandwidth
    FL[f_{1: Lower Cutoff] {-}{-}{-} FM[Maximum Gain Region] {-}{-}{-} FH[f_{2}: Upper Cutoff]}
    end
    
    FL {-{-} "0.707" {-}{-}{} G}
    FH {-{-} "0.707" {-}{-}{} G}
{Highlighting}
{Shaded}
\end{verbatim}
\end{center}

\textbf{Key Formulas:}

\begin{itemize}
\tightlist
\item
  \textbf{Bandwidth}: BW = f_{2} - f_{1}
\item
  \textbf{Gain-Bandwidth Product}: GBP = A_{0} \times BW (constant)
\end{itemize}

\end{solutionbox}
\begin{mnemonicbox}
``BAND'' - Bandwidth And gain Never Drop together
(one increases when other decreases).

\end{mnemonicbox}
\subsection*{Question 2(b) OR [4
marks]}\label{q2b}

\textbf{Explain effects of emitter bypass capacitor and coupling
capacitor on frequency response of an amplifier.}

\begin{solutionbox}

\textbf{Effects on Frequency Response:}


{\def\LTcaptype{none} % do not increment counter
\vspace{-5pt}
\captionof{table}{Capacitor Effects}
\vspace{-10pt}
\begin{longtable}[]{@{}
  >{\raggedright\arraybackslash}p{(\linewidth - 4\tabcolsep) * \real{0.2157}}
  >{\raggedright\arraybackslash}p{(\linewidth - 4\tabcolsep) * \real{0.1961}}
  >{\raggedright\arraybackslash}p{(\linewidth - 4\tabcolsep) * \real{0.5882}}@{}}
\toprule\noalign{}
\begin{minipage}[b]{\linewidth}\raggedright
Capacitor
\end{minipage} & \begin{minipage}[b]{\linewidth}\raggedright
Function
\end{minipage} & \begin{minipage}[b]{\linewidth}\raggedright
Effect on Frequency Response
\end{minipage} \\
\midrule\noalign{}
\endhead
\bottomrule\noalign{}
\endlastfoot
\textbf{Coupling Capacitor (Cc)} & Blocks DC, passes AC & Limits
low-frequency response \\
\textbf{Bypass Capacitor (Ce)} & Bypasses emitter resistor & Increases
gain at mid and high frequencies \\
\end{longtable}
}

\textbf{Diagram:}

\begin{verbatim}
    +Vcc
     |
     Rc
     |
     +{-{-}{-}{-}{-}{-}{-}+}
     |       |
 Cc  |       |
 ||{-{-}+       C}
 ||  |       |
Input  B     |
     |       |
     |       E
     |       |
     Re      |
     |       |
     +{-{-}||{-}{-}{-}+}
     |   Ce
    GND
\end{verbatim}

\textbf{Key Effects:}

\begin{itemize}
\tightlist
\item
  \textbf{Without Ce}: Lower gain, better stability, better
  low-frequency response
\item
  \textbf{Without Cc}: DC coupling, excellent low-frequency response
\item
  \textbf{Capacitor Values}: Determine cutoff frequencies (f_{1}, f_{2})
\end{itemize}

\end{solutionbox}
\begin{mnemonicbox}
``CELL'' - Coupling affects Extremely Low
frequencies, bypass affects Low to high.

\end{mnemonicbox}
\subsection*{Question 2(c) OR [7
marks]}\label{q2c}

\textbf{Compare transformer coupled amplifier and RC coupled amplifier}

\begin{solutionbox}

\textbf{Table: Comparison of Transformer Coupled vs RC Coupled
Amplifier}

{\def\LTcaptype{none} % do not increment counter
\begin{longtable}[]{@{}
  >{\raggedright\arraybackslash}p{(\linewidth - 4\tabcolsep) * \real{0.2143}}
  >{\raggedright\arraybackslash}p{(\linewidth - 4\tabcolsep) * \real{0.5000}}
  >{\raggedright\arraybackslash}p{(\linewidth - 4\tabcolsep) * \real{0.2857}}@{}}
\toprule\noalign{}
\begin{minipage}[b]{\linewidth}\raggedright
Feature
\end{minipage} & \begin{minipage}[b]{\linewidth}\raggedright
Transformer Coupled
\end{minipage} & \begin{minipage}[b]{\linewidth}\raggedright
RC Coupled
\end{minipage} \\
\midrule\noalign{}
\endhead
\bottomrule\noalign{}
\endlastfoot
\textbf{Coupling Element} & Transformer & Capacitor and Resistor \\
\textbf{Efficiency} & High (90\%) & Moderate (20-30\%) \\
\textbf{Size and Weight} & Bulky and Heavy & Compact and Light \\
\textbf{Cost} & Expensive & Inexpensive \\
\textbf{Frequency Response} & Poor (limited bandwidth) & Good (wide
bandwidth) \\
\textbf{Impedance Matching} & Excellent & Poor \\
\textbf{DC Isolation} & Complete & Only AC signals \\
\textbf{Distortion} & Higher & Lower \\
\end{longtable}
}

\textbf{Diagram:}

\begin{verbatim}
graph TB
    subgraph "RC Coupled"
    RC[Resistor{-Capacitor] {-}{-} RCF[Flat Responsebr /Wide Bandwidth]}
    end

    subgraph "Transformer Coupled"
    TC[Transformer] {-{-} TCF[Peaked Responsebr /Narrow Bandwidth]}
    end
\end{verbatim}

\textbf{Applications:}

\begin{itemize}
\tightlist
\item
  \textbf{RC Coupled}: Audio amplifiers, general-purpose amplifiers
\item
  \textbf{Transformer Coupled}: Power amplifiers, radio transmitters
\end{itemize}

\end{solutionbox}
\begin{mnemonicbox}
``TRIP'' - Transformers are Robust for Impedance
matching, Problematic for bandwidth.

\end{mnemonicbox}
\subsection*{Question 3(a) [3 marks]}\label{q3a}

\textbf{Describe the transistor used as a tuned amplifier.}

\begin{solutionbox}

\textbf{Tuned Amplifier: Amplifier that selectively amplifies signals
within a narrow frequency band}

\textbf{Diagram:}

\begin{verbatim}
    +Vcc
     |
     |
     +{-{-}{-}+}
     |   |
     L   |
     |   |
 Cin |   |
 ||{-{-}+{-}{-}{-}+}
 ||  |   |
Input   B |
     |   |
     |   C      Cout
     |   +{-{-}{-}{-}{-}{-}||{-}{-}{-}{-}+ Output}
     |   |             |
     |   E             |
     |   |             |
     |  Re             |
     |   |             |
    GND GND           GND
\end{verbatim}

\textbf{Key Components:}

\begin{itemize}
\tightlist
\item
  \textbf{LC Tank Circuit}: Determines resonant frequency
\item
  \textbf{Transistor}: Provides amplification
\item
  \textbf{Resonant Frequency}: f_{0} = 1/(2π\sqrtLC)
\end{itemize}

\textbf{Applications:}

\begin{itemize}
\tightlist
\item
  \textbf{Radio Receivers}: Selects desired frequency
\item
  \textbf{TV Tuners}: Channel selection
\item
  \textbf{RF Amplifiers}: Communication systems
\end{itemize}

\end{solutionbox}
\begin{mnemonicbox}
``TUNE'' - Transistors Using Narrowband Elements for
frequency selection.

\end{mnemonicbox}
\subsection*{Question 3(b) [4 marks]}\label{q3b}

\textbf{Explain in brief Direct coupled amplifier.}

\begin{solutionbox}

\textbf{Direct Coupled Amplifier: Multiple stage amplifier where stages
are connected directly without coupling capacitors or transformers}

\textbf{Diagram:}

\begin{center}
\textbf{Mermaid Diagram (Code)}
\begin{verbatim}
{Shaded}
{Highlighting}[]
graph LR
    I[Input] {-{-}{} T1[Transistor 1] {-}{-}{} T2[Transistor 2] {-}{-}{} O[Output]}
    T1 {-{-} "Direct Connection{}br /{}No Coupling Components" {-}{-}{} T2}
{Highlighting}
{Shaded}
\end{verbatim}
\end{center}

\textbf{Key Characteristics:}

\begin{itemize}
\tightlist
\item
  \textbf{DC Amplification}: Can amplify from DC to high frequencies
\item
  \textbf{No Coupling Elements}: Collector directly connected to next
  base
\item
  \textbf{Level Shifting}: Required between stages
\item
  \textbf{Thermal Drift}: Challenge due to direct DC coupling
\end{itemize}

\textbf{Applications:}

\begin{itemize}
\tightlist
\item
  \textbf{Operational Amplifiers}: Internal stages
\item
  \textbf{DC Amplifiers}: Laboratory instruments
\item
  \textbf{Sensing Circuits}: Temperature and pressure sensors
\end{itemize}

\end{solutionbox}
\begin{mnemonicbox}
``DCAP'' - Direct Coupled Amplifier Passes all
frequencies including DC.

\end{mnemonicbox}
\subsection*{Question 3(c) [7 marks]}\label{q3c}

\textbf{Describe the importance of h parameters in two port networks.
Draw h-parameters circuit for CE amplifier.}

\begin{solutionbox}

\textbf{h-parameters (hybrid parameters): Set of four parameters that
define behavior of two-port network}

\textbf{Importance:}

\begin{itemize}
\tightlist
\item
  \textbf{Complete Characterization}: Fully describes amplifier behavior
\item
  \textbf{Easy Measurement}: Can be measured under simple conditions
\item
  \textbf{Analysis Tool}: Simplifies circuit analysis
\item
  \textbf{Standardized Approach}: Universal method for comparing
  transistors
\end{itemize}

\textbf{h-parameter Equations:}

\begin{itemize}
\tightlist
\item
  V_{1} = h_{1}_{1}I_{1} + h_{1}_{2}V_{2}
\item
  I_{2} = h_{2}_{1}I_{1} + h_{2}_{2}V_{2}
\end{itemize}

\textbf{h-parameter Circuit for CE Amplifier:}

\begin{verbatim}
                     +
                     |
                    Ic
               +{-{-}{-}{-}{-}+{-}{-}{-}{-}{-}+}
               |     |     |
         +     |     |     |
        Ii     |    hoe    |
     +{-{-}{-}{-}+   |     |     |}
     |     |   |     |     |    +
  +  |    hie  |    hfe·Ii |   Vo
 Vi  |     |   |     |     |    {-}
  {-  |     |   |     |     |}
     +{-{-}+{-}{-}+   |     |     |}
        |      |     |     |
        +{{-}{-}{-}{-}{-}+     |     |}
        hre·Vo       |     |
               |     |     |
               +{-{-}{-}{-}{-}+{-}{-}{-}{-}{-}+}
                     |
                     +
\end{verbatim}


{\def\LTcaptype{none} % do not increment counter
\vspace{-5pt}
\captionof{table}{h-parameters for CE Configuration}
\vspace{-10pt}
\begin{longtable}[]{@{}
  >{\raggedright\arraybackslash}p{(\linewidth - 6\tabcolsep) * \real{0.2115}}
  >{\raggedright\arraybackslash}p{(\linewidth - 6\tabcolsep) * \real{0.1538}}
  >{\raggedright\arraybackslash}p{(\linewidth - 6\tabcolsep) * \real{0.2885}}
  >{\raggedright\arraybackslash}p{(\linewidth - 6\tabcolsep) * \real{0.3462}}@{}}
\toprule\noalign{}
\begin{minipage}[b]{\linewidth}\raggedright
Parameter
\end{minipage} & \begin{minipage}[b]{\linewidth}\raggedright
Symbol
\end{minipage} & \begin{minipage}[b]{\linewidth}\raggedright
Typical Value
\end{minipage} & \begin{minipage}[b]{\linewidth}\raggedright
Physical Meaning
\end{minipage} \\
\midrule\noalign{}
\endhead
\bottomrule\noalign{}
\endlastfoot
\textbf{Input impedance} & h_{1}_{1} (hie) & 1-2 kΩ & Input resistance with
output shorted \\
\textbf{Reverse voltage transfer} & h_{1}_{2} (hre) & 1-4 \times 10^{-}^{4} & Reverse
feedback ratio \\
\textbf{Forward current transfer} & h_{2}_{1} (hfe) & 20-500 & Current gain
(β) \\
\textbf{Output admittance} & h_{2}_{2} (hoe) & 20-50 μS & Output
conductance \\
\end{longtable}
}

\end{solutionbox}
\begin{mnemonicbox}
``HIRE'' - h-parameters Include Resistance and
current gain Effectively.

\end{mnemonicbox}
\subsection*{Question 3(a) OR [3
marks]}\label{q3a}

\textbf{Compare transformer coupled amplifier and direct coupled
amplifier.}

\begin{solutionbox}

\textbf{Table: Comparison between Transformer and Direct Coupled
Amplifiers}

{\def\LTcaptype{none} % do not increment counter
\begin{longtable}[]{@{}
  >{\raggedright\arraybackslash}p{(\linewidth - 4\tabcolsep) * \real{0.2000}}
  >{\raggedright\arraybackslash}p{(\linewidth - 4\tabcolsep) * \real{0.4667}}
  >{\raggedright\arraybackslash}p{(\linewidth - 4\tabcolsep) * \real{0.3333}}@{}}
\toprule\noalign{}
\begin{minipage}[b]{\linewidth}\raggedright
Feature
\end{minipage} & \begin{minipage}[b]{\linewidth}\raggedright
Transformer Coupled
\end{minipage} & \begin{minipage}[b]{\linewidth}\raggedright
Direct Coupled
\end{minipage} \\
\midrule\noalign{}
\endhead
\bottomrule\noalign{}
\endlastfoot
\textbf{Coupling Element} & Transformer & None (direct connection) \\
\textbf{Frequency Response} & Limited at low frequencies & Excellent (DC
to high freq) \\
\textbf{DC Isolation} & Complete & None \\
\textbf{Size} & Bulky & Compact \\
\textbf{Cost} & Higher & Lower \\
\textbf{DC Shift Problem} & No & Yes \\
\end{longtable}
}

\textbf{Diagram:}

\begin{center}
\textbf{Mermaid Diagram (Code)}
\begin{verbatim}
{Shaded}
{Highlighting}[]
graph TD
    subgraph "Transformer Coupled"
    T1[Transistor 1] {-{-}{-} TR[Transformer] {-}{-}{-} T2[Transistor 2]}
    end

    subgraph "Direct Coupled"
    D1[Transistor 1] {-{-} "Direct Connection" {-}{-}{} D2[Transistor 2]}
    end
{Highlighting}
{Shaded}
\end{verbatim}
\end{center}

\end{solutionbox}
\begin{mnemonicbox}
``TDC'' - Transformers provide DC isolation, Direct
provides Complete frequency range.

\end{mnemonicbox}
\subsection*{Question 3(b) OR [4
marks]}\label{q3b}

\textbf{Draw and Explain circuit diagram of common emitter amplifier.}

\begin{solutionbox}

\textbf{Common Emitter Amplifier: Configuration where emitter is common
to both input and output circuits}

\textbf{Circuit Diagram:}

\begin{verbatim}
                 +Vcc
                  |
                  |
                  Rc
                  |
                  +{-{-}{-}{-}{-}{-}{-}{-}+ Output}
                  |        |
             +{-{-}{-}{-}+        |}
             |    |        |
     Input   |    C        |
     +{-{-}{-}{-}{-}{-}{-}|B   |        |}
     |       |    |        |
     |       |    E        |
     |       |    |        |
     |       |    +        |
     |       |    |        |
     |       |   Re        |
     |       |    |        |
    GND     GND  GND      GND
\end{verbatim}

\textbf{Operation:}

\begin{itemize}
\tightlist
\item
  \textbf{Input}: Applied between base and emitter
\item
  \textbf{Output}: Taken from collector and emitter
\item
  \textbf{Phase Shift}: 180^\circ between input and output
\item
  \textbf{Gain}: High voltage and current gain
\end{itemize}

\textbf{Key Features:}

\begin{itemize}
\tightlist
\item
  \textbf{High Gain}: Typical voltage gain 300-1000
\item
  \textbf{Medium Input Impedance}: 1-2 kΩ
\item
  \textbf{High Output Impedance}: 40-50 kΩ
\item
  \textbf{Signal Inversion}: Output is inverted
\end{itemize}

\end{solutionbox}
\begin{mnemonicbox}
``CEA'' - Common Emitter Amplifies with signal
inversion.

\end{mnemonicbox}
\subsection*{Question 3(c) OR [7
marks]}\label{q3c}

\textbf{Draw Transistor Two Port Network and describe h-parameters for
it. Write down advantages of hybrid parameters.}

\begin{solutionbox}

\textbf{Transistor Two-Port Network:}

\begin{verbatim}
        I1             I2
        {-{-}            {-}{-}}
    +{-{-}{-}{-}{-}{-}{-}+      +{-}{-}{-}{-}{-}{-}{-}+}
    |       |      |       |
    |       |      |       |
  + |       |      |       | +
 V1 |  Two  |      |  Port | V2
  {- |       |      |       | {-}}
    |       |      |       |
    |       |      |       |
    +{-{-}{-}{-}{-}{-}{-}+      +{-}{-}{-}{-}{-}{-}{-}+}
\end{verbatim}

\textbf{h-parameter Equations:}

\begin{itemize}
\tightlist
\item
  V_{1} = h_{1}_{1}I_{1} + h_{1}_{2}V_{2}
\item
  I_{2} = h_{2}_{1}I_{1} + h_{2}_{2}V_{2}
\end{itemize}


{\def\LTcaptype{none} % do not increment counter
\vspace{-5pt}
\captionof{table}{h-parameters Description}
\vspace{-10pt}
\begin{longtable}[]{@{}
  >{\raggedright\arraybackslash}p{(\linewidth - 6\tabcolsep) * \real{0.2037}}
  >{\raggedright\arraybackslash}p{(\linewidth - 6\tabcolsep) * \real{0.1481}}
  >{\raggedright\arraybackslash}p{(\linewidth - 6\tabcolsep) * \real{0.2407}}
  >{\raggedright\arraybackslash}p{(\linewidth - 6\tabcolsep) * \real{0.4074}}@{}}
\toprule\noalign{}
\begin{minipage}[b]{\linewidth}\raggedright
Parameter
\end{minipage} & \begin{minipage}[b]{\linewidth}\raggedright
Symbol
\end{minipage} & \begin{minipage}[b]{\linewidth}\raggedright
Description
\end{minipage} & \begin{minipage}[b]{\linewidth}\raggedright
Measurement Condition
\end{minipage} \\
\midrule\noalign{}
\endhead
\bottomrule\noalign{}
\endlastfoot
\textbf{Input impedance} & h_{1}_{1} & Ratio of V_{1}/I_{1} & V_{2} = 0 (Output
shorted) \\
\textbf{Reverse voltage transfer} & h_{1}_{2} & Ratio of V_{1}/V_{2} & I_{1} = 0 (Input
open) \\
\textbf{Forward current transfer} & h_{2}_{1} & Ratio of I_{2}/I_{1} & V_{2} = 0
(Output shorted) \\
\textbf{Output admittance} & h_{2}_{2} & Ratio of I_{2}/V_{2} & I_{1} = 0 (Input
open) \\
\end{longtable}
}

\textbf{Advantages of Hybrid Parameters:}

\begin{itemize}
\tightlist
\item
  \textbf{Easy Measurement}: Simple conditions for each parameter
\item
  \textbf{Universality}: Works for all transistor configurations
\item
  \textbf{Complete Characterization}: Fully describes behavior
\item
  \textbf{Mathematical Simplicity}: Linear equations
\item
  \textbf{Standardized}: Industry standard for specification
\end{itemize}

\end{solutionbox}
\begin{mnemonicbox}
``HAEM'' - Hybrid parameters Are Easily Measured and
mathematically simple.

\end{mnemonicbox}
\subsection*{Question 4(a) [3 marks]}\label{q4a}

\textbf{Explain Darlington pair and its applications.}

\begin{solutionbox}

\textbf{Darlington Pair: Configuration of two transistors where emitter
of first is connected to base of second}

\textbf{Diagram:}

\begin{verbatim}
           +Vcc
            |
            |
            Rc
            |
            +{-{-}{-}{-}{-}{-} Output}
            |
            |
     +{-{-}{-}{-}{-}{-}+}
     |      |
     |      C2
     |      |
Input|      |
+{-{-}{-}{-}+B1    |}
     |      |
     |  E1  |
     |  |   |
     |  +B2 |
     |      |
     |      E2
     |      |
    GND    GND
\end{verbatim}

\textbf{Key Features:}

\begin{itemize}
\tightlist
\item
  \textbf{Very High Current Gain}: β_{1} \times β_{2} (typical 1000-30000)
\item
  \textbf{High Input Impedance}: β_{2} \times Rin_{1}
\item
  \textbf{Low Output Impedance}: Similar to single transistor
\end{itemize}

\textbf{Applications:}

\begin{itemize}
\tightlist
\item
  \textbf{Power Amplifiers}: Audio equipment
\item
  \textbf{Buffer Circuits}: High impedance to low impedance
\item
  \textbf{Motor Drivers}: Control high-current loads
\item
  \textbf{Touch Sensors}: High sensitivity applications
\end{itemize}

\end{solutionbox}
\begin{mnemonicbox}
``DISH'' - Darlington Integrates Stages for High
current gain.

\end{mnemonicbox}
\subsection*{Question 4(b) [4 marks]}\label{q4b}

\textbf{Describe the diode clamper circuit with necessary diagram.}

\begin{solutionbox}

\textbf{Clamper Circuit: Shifts the DC level of a waveform without
changing its shape}

\textbf{Diagram:}

\begin{verbatim}
           C1
Input +{-{-}{-}{-}||{-}{-}{-}{-}+{-}{-}{-}{-}+ Output}
                 |    |
                 |    |
                 R    D
                 |    |
                 |    |
                GND  GND
\end{verbatim}

\textbf{Operation:}

\begin{itemize}
\tightlist
\item
  \textbf{Positive Clamper}: Shifts waveform downward
\item
  \textbf{Negative Clamper}: Shifts waveform upward
\item
  \textbf{Capacitor}: Blocks DC, passes AC
\item
  \textbf{Diode}: Conducts during one half-cycle
\item
  \textbf{Resistor}: Discharge path for capacitor
\end{itemize}

\textbf{Time Constants:}

\begin{itemize}
\tightlist
\item
  \textbf{Charging}: Very small (diode forward resistance \times C)
\item
  \textbf{Discharging}: Large (R \times C) compared to signal period
\end{itemize}

\textbf{Applications:}

\begin{itemize}
\tightlist
\item
  \textbf{TV Signal Processing}: Restores DC component
\item
  \textbf{Pulse Circuits}: Level shifting
\item
  \textbf{Signal Processing}: DC restoration
\end{itemize}

\end{solutionbox}
\begin{mnemonicbox}
``CLAMP'' - Circuit Levels Are Modified Precisely.

\end{mnemonicbox}
\subsection*{Question 4(c) [7 marks]}\label{q4c}

\textbf{Explain the construction, working and applications of OLED.}

\begin{solutionbox}

\textbf{OLED (Organic Light Emitting Diode): Light-emitting device using
organic compounds}

\textbf{Construction:}

\begin{center}
\textbf{Mermaid Diagram (Code)}
\begin{verbatim}
{Shaded}
{Highlighting}[]
graph TD
    subgraph OLED Structure
    direction LR
    C[Cathode{br /{}Metal Layer] {-}{-}{-} E[Emissive Layer{}br /{}Organic Material] {-}{-}{-} H[Hole Transport Layer{}br /{}Organic Material] {-}{-}{-} A[Anode{}br /{}Transparent ITO] {-}{-}{-} S[Substrate{}br /{}Glass or Plastic]}
    end
{Highlighting}
{Shaded}
\end{verbatim}
\end{center}

\textbf{Working Principle:}

\begin{itemize}
\tightlist
\item
  \textbf{Electron Injection}: Cathode injects electrons
\item
  \textbf{Hole Injection}: Anode injects holes
\item
  \textbf{Recombination}: Electrons and holes combine in emissive layer
\item
  \textbf{Light Emission}: Energy released as photons
\item
  \textbf{Color Control}: Different organic materials emit different
  colors
\end{itemize}


{\def\LTcaptype{none} % do not increment counter
\vspace{-5pt}
\captionof{table}{OLED Types}
\vspace{-10pt}
\begin{longtable}[]{@{}lll@{}}
\toprule\noalign{}
Type & Structure & Key Feature \\
\midrule\noalign{}
\endhead
\bottomrule\noalign{}
\endlastfoot
\textbf{PMOLED} & Passive Matrix & Simpler design, lower cost \\
\textbf{AMOLED} & Active Matrix & Better refresh rates, higher
resolution \\
\textbf{TOLED} & Transparent & See-through when off or on \\
\textbf{FOLED} & Flexible & Can be bent or rolled \\
\end{longtable}
}

\textbf{Applications:}

\begin{itemize}
\tightlist
\item
  \textbf{Displays}: Smartphones, TVs, smartwatches
\item
  \textbf{Lighting}: Thin, efficient lighting panels
\item
  \textbf{Signage}: High-contrast digital signs
\item
  \textbf{Wearable Technology}: Flexible displays
\end{itemize}

\end{solutionbox}
\begin{mnemonicbox}
``OLED'' - Organic Layers Emit Directly when
electrically stimulated.

\end{mnemonicbox}
\subsection*{Question 4(a) OR [3
marks]}\label{q4a}

\textbf{Explain Short note on LDR.}

\begin{solutionbox}

\textbf{LDR (Light Dependent Resistor): Photoresistor whose resistance
decreases with increasing light intensity}

\textbf{Symbol and Structure:}

\begin{verbatim}
    ┌─────┐
    │     │
────┤ /{ ├────}
    │     │
    └─────┘
     Symbol
     
      Light
       ↓↓↓
    ┌───────┐
    │┌─────┐│
────┤│CdS  ││────
    │└─────┘│
    └───────┘
     Structure
\end{verbatim}

\textbf{Key Characteristics:}

\begin{itemize}
\tightlist
\item
  \textbf{Material}: Usually Cadmium Sulfide (CdS)
\item
  \textbf{Dark Resistance}: High (MΩ range)
\item
  \textbf{Light Resistance}: Low (kΩ range)
\item
  \textbf{Response Time}: Milliseconds to seconds
\end{itemize}

\textbf{Applications:}

\begin{itemize}
\tightlist
\item
  \textbf{Light Sensors}: Automatic lighting control
\item
  \textbf{Camera Exposure Control}: Light metering
\item
  \textbf{Street Light Control}: Dawn-to-dusk activation
\item
  \textbf{Alarm Systems}: Light beam detection
\end{itemize}

\end{solutionbox}
\begin{mnemonicbox}
``LORD'' - Light Oppositely Reduces the Device's
resistance.

\end{mnemonicbox}
\subsection*{Question 4(b) OR [4
marks]}\label{q4b}

\textbf{Describe the diode clipper circuit with necessary diagram.}

\begin{solutionbox}

\textbf{Clipper Circuit: Removes (clips) portion of input signal that
exceeds certain voltage level}

\textbf{Diagram (Positive Clipper):}

\begin{verbatim}
                R     D
Input +{-{-}{-}{-}{-}+{-}{-}{-}www{-}{-}{-}+{-}{-}{-}+{-}{-}+ Output}
            |         |   |
            |         |   |
            |         +   {-}
            |         |   |
            |         |   |
            |         V   |
            |         |   |
            +{-{-}{-}{-}{-}{-}{-}{-}{-}+{-}{-}{-}+}
\end{verbatim}

\textbf{Types of Clippers:}

\begin{itemize}
\tightlist
\item
  \textbf{Positive Clipper}: Removes positive peaks
\item
  \textbf{Negative Clipper}: Removes negative peaks
\item
  \textbf{Biased Clipper}: Clips at non-zero reference
\item
  \textbf{Combination Clipper}: Clips both peaks
\end{itemize}

\textbf{Operation:}

\begin{itemize}
\tightlist
\item
  \textbf{Diode ON}: When signal exceeds reference voltage
\item
  \textbf{Diode OFF}: When signal is below reference voltage
\item
  \textbf{Clipping Level}: Determined by reference voltage
\end{itemize}

\textbf{Applications:}

\begin{itemize}
\tightlist
\item
  \textbf{Wave Shaping}: Creating square waves
\item
  \textbf{Circuit Protection}: Voltage limiting
\item
  \textbf{Noise Removal}: Limiting impulse noise
\end{itemize}

\end{solutionbox}
\begin{mnemonicbox}
``CLIP'' - Circuit Limits Input Peaks using diodes.

\end{mnemonicbox}
\subsection*{Question 4(c) OR [7
marks]}\label{q4c}

\textbf{Explain Half Wave and Full wave Voltage Doubler.}

\begin{solutionbox}

\textbf{Voltage Doubler: Circuit that produces DC output voltage
approximately twice the peak input voltage}

\textbf{Half-Wave Voltage Doubler:}

\begin{verbatim}
               D1
            +{-{-}{-}{-}|{-}{-}{-}+}
            |         |
            |         |
AC Input    |         | C1    + 2Vpeak
    +{-{-}{-}{-}{-}{-}{-}+         +{-}{-}{-}{-}{-}{-}{-}+  Output}
    |       |         |       |
    |       +{-{-}{-}{-}|{-}{-}{-}+       |}
    |          D2    |        |
    |               C2        |
    |                |        |
    +{-{-}{-}{-}{-}{-}{-}{-}{-}{-}{-}{-}{-}{-}{-}{-}+{-}{-}{-}{-}{-}{-}{-}{-}+}
                     |
                    GND
\end{verbatim}

\textbf{Full-Wave Voltage Doubler:}

\begin{verbatim}
               D1
            +{-{-}{-}{-}|{-}{-}{-}+}
            |         |
            |         |
AC Input    |         | C1    + 2Vpeak
    +{-{-}{-}{-}{-}{-}{-}+         +{-}{-}{-}{-}{-}{-}{-}+  Output}
    |       |         |       |
    |       |         |       |
    |       |    C2   |       |
    |       |    |    |       |
    |       +{-{-}{-}{-}|{-}{-}{-}+       |}
    |          D2             |
    |                |        |
    +{-{-}{-}{-}{-}{-}{-}{-}{-}{-}{-}{-}{-}{-}{-}{-}+{-}{-}{-}{-}{-}{-}{-}{-}+}
                     |
                    GND
\end{verbatim}


{\def\LTcaptype{none} % do not increment counter
\vspace{-5pt}
\captionof{table}{Comparison}
\vspace{-10pt}
\begin{longtable}[]{@{}lll@{}}
\toprule\noalign{}
Feature & Half-Wave & Full-Wave \\
\midrule\noalign{}
\endhead
\bottomrule\noalign{}
\endlastfoot
\textbf{Ripple} & Higher & Lower \\
\textbf{Efficiency} & Lower & Higher \\
\textbf{Response Time} & Slower & Faster \\
\textbf{Components} & 2 diodes, 2 capacitors & 2 diodes, 2 capacitors \\
\textbf{Regulation} & Poor & Better \\
\end{longtable}
}

\textbf{Operation:}

\begin{itemize}
\tightlist
\item
  \textbf{Half-Wave}: Charges each capacitor on alternate half-cycles
\item
  \textbf{Full-Wave}: Charges both capacitors on every cycle
\item
  \textbf{Output}: Sum of voltages across both capacitors
\end{itemize}

\textbf{Applications:}

\begin{itemize}
\tightlist
\item
  \textbf{Power Supplies}: Low-current high-voltage needs
\item
  \textbf{Cascade Connection}: For voltage multiplication
\item
  \textbf{Electronic Flash}: Camera equipment
\item
  \textbf{CRT Displays}: High voltage generation
\end{itemize}

\end{solutionbox}
\begin{mnemonicbox}
``DOUBLE'' - Diodes Organize Unidirectional Boost,
Lifting Electricity to twice input.

\end{mnemonicbox}
\subsection*{Question 5(a) [3 marks]}\label{q5a}

\textbf{Draw circuit diagram for +5 v Power Supply using its IC}

\begin{solutionbox}

\textbf{+5V Power Supply Using 7805 Voltage Regulator IC (continued):}

\begin{verbatim}
   AC Input    Bridge     7805
    +{-{-}+       Rect.    +{-}{-}{-}{-}{-}+}
       |     +{-{-}{-}{-}{-}{-}+   |     |}
       +{-{-}{-}{-}{-}+      +{-}{-}{-}+ IN  |}
       |     |      |   |     |   +5V
       |     +{-{-}{-}{-}{-}{-}+   |     +{-}{-}{-}+{-}{-}{-} Output}
       +{-{-}+{-}{-}+          | OUT |   |}
          |             |     |   |
         GND            +{-{-}+{-}{-}+   |}
                           |      |
                          GND    GND
                           
                  C1 |    C2 |
                 === |    === |
                  |  |     |  |
                 GND       GND
\end{verbatim}

\textbf{Key Components:}

\begin{itemize}
\tightlist
\item
  \textbf{7805 IC}: Three-terminal fixed voltage regulator
\item
  \textbf{Input Capacitor (C1)}: Filters input ripple
\item
  \textbf{Output Capacitor (C2)}: Improves transient response
\item
  \textbf{Bridge Rectifier}: Converts AC to pulsating DC
\end{itemize}

\end{solutionbox}
\begin{mnemonicbox}
``FIVE'' - Fixed IC Voltage Efficiently provided.

\end{mnemonicbox}
\subsection*{Question 5(b) [4 marks]}\label{q5b}

\textbf{Discuss load regulation and line regulation in reference to
power supply.}

\begin{solutionbox}

\textbf{Load Regulation: Ability of power supply to maintain constant
output voltage despite load current changes}

\textbf{Diagram:}

\begin{center}
\textbf{Mermaid Diagram (Code)}
\begin{verbatim}
{Shaded}
{Highlighting}[]
graph LR
    A[Power Supply] {-{-}{} B["Line Regulation{}br /{}(Input Voltage Changes)"]}
    A {-{-}{} C["Load Regulation{}br /{}(Output Current Changes)"]}
    B {-{-}{} D["Constant Output{}br /{}Voltage"]}
    C {-{-}{} D}
{Highlighting}
{Shaded}
\end{verbatim}
\end{center}

\textbf{Formulas:}

\begin{itemize}
\tightlist
\item
  \textbf{Load Regulation}: (V_{1} - V_{2})/V_{2} \times 100\%

  \begin{itemize}
  \tightlist
  \item
    V_{1} = No-load voltage
  \item
    V_{2} = Full-load voltage
  \end{itemize}
\item
  \textbf{Line Regulation}: (V_{1} - V_{2})/V_{2} \times 100\%

  \begin{itemize}
  \tightlist
  \item
    V_{1} = Output voltage at maximum input
  \item
    V_{2} = Output voltage at minimum input
  \end{itemize}
\end{itemize}

\textbf{Key Points:}

\begin{itemize}
\tightlist
\item
  \textbf{Lower Percentage}: Better regulation
\item
  \textbf{Feedback Circuit}: Improves regulation performance
\item
  \textbf{IC Regulators}: Typically offer good regulation (0.01-0.1\%)
\end{itemize}

\end{solutionbox}
\begin{mnemonicbox}
``LINE LOAD'' - Line Is Normal-input Efficiency, LOAD
is Output Adjustment Defense.

\end{mnemonicbox}
\subsection*{Question 5(c) [7 marks]}\label{q5c}

\textbf{Explain adjustable voltage regulator using LM317 with circuit
diagram.}

\begin{solutionbox}

\textbf{LM317 Adjustable Voltage Regulator: Three-terminal device that
provides variable regulated output voltage}

\textbf{Circuit Diagram:}

\begin{verbatim}
               LM317
              +{-{-}{-}{-}{-}{-}+}
              |      |
   Input {-{-}{-}{-}{-}+ IN   |}
              |      |    R1
              | ADJ  +{-{-}{-}{-}www{-}{-}{-}{-}+}
              |      |           |
              | OUT  |           |
              +{-{-}+{-}{-}{-}+           |}
                 |               |
                 +{-{-}{-}{-}{-}{-}{-}www{-}{-}{-}{-}{-}+}
                 |       R2      |
                 |               |
                 +      C1       +{-{-}{-}{-}{-}{-} Output}
                 |      ===      |
                 |       |       |
                GND     GND     GND
\end{verbatim}

\textbf{Operation:}

\begin{itemize}
\tightlist
\item
  \textbf{Reference Voltage}: 1.25V between OUT and ADJ terminals
\item
  \textbf{Output Voltage}: VOUT = 1.25V \times (1 + R2/R1)
\item
  \textbf{Adjustment Range}: 1.25V to 37V
\item
  \textbf{Maximum Current}: 1.5A (with proper heat sink)
\end{itemize}

\textbf{Component Selection:}

\begin{itemize}
\tightlist
\item
  \textbf{R1}: Typically 240Ω
\item
  \textbf{R2}: Variable resistor to adjust output
\item
  \textbf{C1}: Output capacitor for stability (1-10μF)
\end{itemize}

\textbf{Key Features:}

\begin{itemize}
\tightlist
\item
  \textbf{Current Limiting}: Built-in protection
\item
  \textbf{Thermal Shutdown}: Protection against overheating
\item
  \textbf{Safe Area Protection}: For output transistors
\item
  \textbf{Ripple Rejection}: 80dB typically
\end{itemize}

\end{solutionbox}
\begin{mnemonicbox}
``VARY'' - Voltage Adjustable Regulator Yields custom
outputs.

\end{mnemonicbox}
\subsection*{Question 5(a) OR [3
marks]}\label{q5a}

\textbf{Draw circuit diagram for -15 v Power Supply using its IC}

\begin{solutionbox}

\textbf{-15V Power Supply Using 7915 Voltage Regulator IC:}

\begin{verbatim}
   AC Input    Bridge     7915
    +{-{-}+       Rect.    +{-}{-}{-}{-}{-}+}
       |     +{-{-}{-}{-}{-}{-}+   |     |}
       +{-{-}{-}{-}{-}+      +{-}{-}{-}+ IN  |}
       |     |      |   |     |   {-15V}
       |     +{-{-}{-}{-}{-}{-}+   |     +{-}{-}{-}+{-}{-}{-} Output}
       +{-{-}+{-}{-}+          | OUT |   |}
          |             |     |   |
         GND            +{-{-}+{-}{-}+   |}
                           |      |
                          GND    GND
                           
                 C1 |    C2 |
                 === |    === |
                  |  |     |  |
                 GND       GND
\end{verbatim}

\textbf{Key Components:}

\begin{itemize}
\tightlist
\item
  \textbf{7915 IC}: Three-terminal negative voltage regulator
\item
  \textbf{Input Capacitor (C1)}: Filters input ripple
\item
  \textbf{Output Capacitor (C2)}: Improves transient response
\item
  \textbf{Bridge Rectifier}: Converts AC to pulsating DC
\end{itemize}

\end{solutionbox}
\begin{mnemonicbox}
``NINE'' - Negative IC Needs Efficient filtering.

\end{mnemonicbox}
\subsection*{Question 5(b) OR [4
marks]}\label{q5b}

\textbf{Explain working of UPS.}

\begin{solutionbox}

\textbf{UPS (Uninterruptible Power Supply): Device providing emergency
power when main power fails}

\textbf{Block Diagram:}

\begin{center}
\textbf{Mermaid Diagram (Code)}
\begin{verbatim}
{Shaded}
{Highlighting}[]
graph LR
    I[AC Input] {-{-}{} R[Rectifier]}
    R {-{-}{} C[Charger]}
    C {-{-}{} B[Battery]}
    B {-{-}{} Inv[Inverter]}
    I {-{-} "Normal Operation" {-}{-}{} S[Switch]}
    S {-{-}{} O[Output]}
    Inv {-{-} "During Power Failure" {-}{-}{} S}
{Highlighting}
{Shaded}
\end{verbatim}
\end{center}

\textbf{Types of UPS:}

\begin{itemize}
\tightlist
\item
  \textbf{Offline/Standby UPS}: Switches to battery when power fails
\item
  \textbf{Line-Interactive UPS}: Has voltage regulation
\item
  \textbf{Online/Double-Conversion UPS}: Always uses battery power
\end{itemize}

\textbf{Key Components:}

\begin{itemize}
\tightlist
\item
  \textbf{Rectifier}: Converts AC to DC
\item
  \textbf{Battery}: Stores energy
\item
  \textbf{Inverter}: Converts DC back to AC
\item
  \textbf{Control Circuit}: Monitors power and switches source
\end{itemize}

\textbf{Applications:}

\begin{itemize}
\tightlist
\item
  \textbf{Computers}: Prevents data loss
\item
  \textbf{Medical Equipment}: Critical operations
\item
  \textbf{Industrial Controls}: Prevents costly interruptions
\item
  \textbf{Telecommunications}: Maintains connections
\end{itemize}

\end{solutionbox}
\begin{mnemonicbox}
``UPBEAT'' - Uninterruptible Power Backup Ensures
Available Technology.

\end{mnemonicbox}
\subsection*{Question 5(c) OR [7
marks]}\label{q5c}

\textbf{Draw and explain SMPS block diagram with its advantages and
disadvantages.}

\begin{solutionbox}

\textbf{SMPS (Switch Mode Power Supply): Power supply that uses
switching regulation for efficiency}

\textbf{Block Diagram:}

\begin{center}
\textbf{Mermaid Diagram (Code)}
\begin{verbatim}
{Shaded}
{Highlighting}[]
graph LR
    AC[AC Input] {-{-}{} EMI[EMI Filter]}
    EMI {-{-}{} R[Rectifier \& Filter]}
    R {-{-}{} C[Chopper/Switching Circuit]}
    C {-{-}{} T[High Frequency Transformer]}
    T {-{-}{} O[Output Rectifier \& Filter]}
    O {-{-}{} Out[DC Output]}
    FB[Feedback \& Control] {-{-}{} C}
    O {-{-}{} FB}
{Highlighting}
{Shaded}
\end{verbatim}
\end{center}

\textbf{Operation:}

\begin{itemize}
\tightlist
\item
  \textbf{EMI Filter}: Reduces electromagnetic interference
\item
  \textbf{Rectifier}: Converts AC to unregulated DC
\item
  \textbf{Switching Circuit}: Chops DC at high frequency (20-100 kHz)
\item
  \textbf{Transformer}: Provides isolation and voltage conversion
\item
  \textbf{Output Stage}: Rectifies and filters to clean DC
\item
  \textbf{Feedback Loop}: Controls switching for regulation
\end{itemize}

\textbf{Advantages:}

\begin{itemize}
\tightlist
\item
  \textbf{High Efficiency}: 70-90\% (vs.~30-60\% for linear)
\item
  \textbf{Small Size}: Higher operating frequency means smaller
  components
\item
  \textbf{Light Weight}: Smaller transformer and heat sinks
\item
  \textbf{Wide Input Range}: Can operate on various input voltages
\item
  \textbf{Low Heat Generation}: Less power wasted as heat
\end{itemize}

\textbf{Disadvantages:}

\begin{itemize}
\tightlist
\item
  \textbf{Complex Design}: More sophisticated circuitry
\item
  \textbf{EMI Generation}: Switching creates interference
\item
  \textbf{Higher Cost}: For low-power applications
\item
  \textbf{Noise}: Higher output noise than linear supplies
\item
  \textbf{Slower Response}: To sudden load changes
\end{itemize}

\textbf{Applications:}

\begin{itemize}
\tightlist
\item
  \textbf{Computers}: Desktop and laptop power supplies
\item
  \textbf{TVs and Monitors}: Compact power source
\item
  \textbf{Mobile Chargers}: Small, efficient adapters
\item
  \textbf{Industrial Power}: High-efficiency needs
\end{itemize}

\end{solutionbox}
\begin{mnemonicbox}
``SWITCH'' - Smaller Weight, Improved Thermal
efficiency, Complex Hardware.

\end{mnemonicbox}

\end{document}
