\documentclass[10pt,a4paper]{article}

% content/resources/templates/preamble.tex
\usepackage[margin=0.6in]{geometry}
\author{Milav Dabgar}
\usepackage{amsmath,amssymb,amsthm}
\usepackage{booktabs}
\usepackage{multirow}
\usepackage{xcolor}
\usepackage{tcolorbox}
\tcbuselibrary{breakable,skins}
\usepackage[colorlinks=true,linkcolor=blue]{hyperref}
\usepackage{titlesec}
\usepackage{enumitem}
\usepackage{tikz}
\usepackage{pgfplots}
\usepackage{circuitikz}
\usepackage[version=4]{mhchem}
\usepackage{longtable}
\usepackage{array}
\usepackage{float}
\usepackage{caption}
\usepackage{listings}

\lstset{
  basicstyle=\small\ttfamily,
  breaklines=true,
  breakatwhitespace=false,
  postbreak=\mbox{\textcolor{red}{$\hookrightarrow$}\space},
  float=false,
  numbers=left,
  numberstyle=\tiny\color{gray},
  numbersep=10pt,
  xleftmargin=2em,
  keywordstyle=\color{blue},
  commentstyle=\color{green!60!black},
  stringstyle=\color{purple},
  backgroundcolor=\color{gray!5},
  showstringspaces=false,
  tabsize=2,
  captionpos=b,
  keepspaces=true,
  columns=flexible
}

\pgfplotsset{compat=1.18}
\usetikzlibrary{shapes,arrows,positioning,calc,patterns,decorations.pathmorphing,decorations.markings,arrows.meta}

% Color scheme
\definecolor{headcolor}{RGB}{0,102,204}
\definecolor{keycolor}{RGB}{220,20,60}
\definecolor{solutioncolor}{RGB}{34,139,34}
\definecolor{mnemoniccolor}{RGB}{148,0,211}
\definecolor{codecolor}{RGB}{0,0,100}

% Spacing
\setlength{\parskip}{3pt}
\setlist[itemize]{nosep}
\setlist[enumerate]{nosep}

% Title formatting
\titleformat{\section}{\Large\bfseries\color{headcolor}}{\thesection}{1em}{}
\titleformat{\subsection}{\large\bfseries\color{headcolor}}{\thesubsection}{1em}{}

% Pandoc tightlist compatibility
\providecommand{\tightlist}{%
  \setlength{\itemsep}{0pt}\setlength{\parskip}{0pt}}

% Pandoc longtable compatibility
\newcounter{none}
\def\thenone{}


% content/resources/templates/gujarati-boxes.tex
\usepackage{fontspec}
\usepackage{polyglossia}

% Set Gujarati as main language (document is primarily in Gujarati)
% Note: gloss-gujarati.ldf doesn't exist in polyglossia, but it will use hyphenation patterns
\setdefaultlanguage{gujarati}
\setotherlanguage{english}

% Configure Gujarati font properly
% Use Language=Default to prevent polyglossia from trying to add language-specific features
% that don't exist for Gujarati, which causes "empty feature" warnings
\newfontfamily\gujaratifont[Script=Gujarati,AutoFakeBold=2.5,AutoFakeSlant=0.3]{Noto Sans Gujarati}
\setmainfont[Script=Gujarati,AutoFakeBold=2.5,AutoFakeSlant=0.3]{Noto Sans Gujarati}
% Use Noto Sans Gujarati for monospace to support Gujarati in text
\setmonofont[Scale=0.9]{Noto Sans Gujarati}

% Configure English to use the same font
\newfontfamily\englishfont[Script=Gujarati,AutoFakeBold=2.5,AutoFakeSlant=0.3]{Noto Sans Gujarati}

% Translations for polyglossia
\gappto\captionsgujarati{
  \renewcommand{\tablename}{કોષ્ટક}
  \renewcommand{\figurename}{આકૃતિ}
}

% Helper for TikZ nodes to ensure Gujarati font
\newcommand{\gu}[1]{{\gujaratifont #1}}

% Custom environments
\newtcolorbox{solutionbox}{
    breakable,
    enhanced,
    colback=solutioncolor!5!white,
    colframe=solutioncolor!75!black,
    fonttitle=\bfseries,
    title=જવાબ
}

\newtcolorbox{solutionboxnobreak}{
 colback=solutioncolor!5!white,
 colframe=solutioncolor!75!black,
 fonttitle=\bfseries,
 title=જવાબ
}

\newtcolorbox{keyformula}{
 breakable,
 enhanced,
 colback=keycolor!5!white,
 colframe=keycolor!75!black,
 fonttitle=\bfseries,
 title=રાસાયણિક સમીકરણ/સૂત્ર
}

\newtcolorbox{mnemonicbox}{
 breakable,
 enhanced,
 colback=mnemoniccolor!5!white,
 colframe=mnemoniccolor!75!black,
 fonttitle=\bfseries,
 title=મેમરી ટ્રીક
}


\begin{document}

\begin{center}
{\Huge\bfseries\color{headcolor} Subject Name (Gujarati)}\\[5pt]
{\LARGE 4321103 -- Winter 2023}\\[3pt]
{\large Semester 1 Study Material}\\[3pt]
{\normalsize\textit{Detailed Solutions and Explanations}}
\end{center}

\vspace{10pt}

\subsection*{પ્રશ્ન 1(અ) [3
ગુણ]}\label{uxaaauxab0uxab6uxaa8-1uxa85-3-uxa97uxaa3}

\textbf{ટ્રાન્ઝિસ્ટર બાયસિંગ શું છે? તેની શું જરૂર છે?}

\begin{solutionbox}
ટ્રાન્ઝિસ્ટર બાયસિંગ એ AC સિગ્નલના યોગ્ય એમ્પ્લિફિકેશન માટે સ્થિર DC
ઓપરેટિંગ પોઈન્ટ (Q-પોઈન્ટ) સ્થાપિત કરવાની પ્રક્રિયા છે.


{\def\LTcaptype{none} % do not increment counter
\vspace{-5pt}
\captionof{table}{ટ્રાન્ઝિસ્ટર બાયસિંગની જરૂરિયાત}
\vspace{-10pt}
\begin{longtable}[]{@{}ll@{}}
\toprule\noalign{}
પાસું & મહત્વ \\
\midrule\noalign{}
\endhead
\bottomrule\noalign{}
\endlastfoot
સ્થિરતા & તાપમાન વધઘટ છતાં સ્થિર Q-પોઈન્ટ જાળવે છે \\
લિનિયરતા & વિકૃતિ-મુક્ત એમ્પ્લિફિકેશન માટે લિનિયર રીજનમાં કાર્ય સુનિશ્ચિત કરે છે \\
કાર્યક્ષમતા & સિગ્નલ ક્લિપિંગ અટકાવે છે અને સિગ્નલ સ્વિંગને મહત્તમ કરે છે \\
વિશ્વસનીયતા & થર્મલ રનઅવે ટાળે છે અને ટ્રાન્ઝિસ્ટરને સુરક્ષિત રાખે છે \\
\end{longtable}
}

\end{solutionbox}
\begin{mnemonicbox}
``SOLE ઓપરેશન'' (Stability, Operating point,
Linearity, Efficiency)

\end{mnemonicbox}
\subsection*{પ્રશ્ન 1(બ) [4
ગુણ]}\label{uxaaauxab0uxab6uxaa8-1uxaac-4-uxa97uxaa3}

\textbf{CE એમ્પ્લિફાયર માટે લોડ લાઇન સમજાવો}

\begin{solutionbox}
લોડ લાઇન એ ટ્રાન્ઝિસ્ટર સર્કિટના બધા સંભવિત ઓપરેટિંગ પોઈન્ટનું
ગ્રાફિકલ રેપ્રેઝન્ટેશન છે.

\textbf{આકૃતિ:}

\begin{center}
\textbf{Mermaid Diagram (Code)}
\begin{verbatim}
{Shaded}
{Highlighting}[]
graph LR
    A[DC Load Line] {-{-}{-} B[CE Amplifier]}
    B {-{-}{-} C[AC Load Line]}
    C {-{-}{-} D[Q{-}point]}

    style A fill:\#f9f,stroke:\#333,stroke{-width:1px}
    style B fill:\#bbf,stroke:\#333,stroke{-width:1px}
    style C fill:\#f9f,stroke:\#333,stroke{-width:1px}
    style D fill:\#bfb,stroke:\#333,stroke{-width:1px}
{Highlighting}
{Shaded}
\end{verbatim}
\end{center}

\begin{itemize}
\tightlist
\item
  \textbf{DC લોડ લાઇન}: સેચુરેશન પોઈન્ટ (Ic=Vcc/Rc, Vce=0) અને કટઓફ પોઈન્ટ
  (Ic=0, Vce=Vcc) વચ્ચે દોરાય છે
\item
  \textbf{AC લોડ લાઇન}: Q-પોઈન્ટમાંથી પસાર થાય છે, સ્લોપ = -1/rc (rc = AC
  કલેક્ટર રેસિસ્ટન્સ)
\item
  \textbf{Q-પોઈન્ટ}: ઓપરેટિંગ પોઈન્ટ જ્યાં DC બાયસિંગ કન્ડિશન્સ સ્થાપિત થાય છે
\end{itemize}

\end{solutionbox}
\begin{mnemonicbox}
``SCQ પોઈન્ટ્સ'' (Saturation, Cutoff, Q-point)

\end{mnemonicbox}
\subsection*{પ્રશ્ન 1(ક) [7
ગુણ]}\label{uxaaauxab0uxab6uxaa8-1uxa95-7-uxa97uxaa3}

\textbf{ટ્રાન્ઝિસ્ટરની વિવિધ બાયસિંગ પધ્ધતિની યાદી બનાવો અને તેમાથી કોઈપણ એક
સમજાવો.}

\begin{solutionbox}
ટ્રાન્ઝિસ્ટર માટેની વિવિધ બાયસિંગ પધ્ધતિઓ:


{\def\LTcaptype{none} % do not increment counter
\vspace{-5pt}
\captionof{table}{ટ્રાન્ઝિસ્ટર બાયસિંગ પધ્ધતિઓ}
\vspace{-10pt}
\begin{longtable}[]{@{}ll@{}}
\toprule\noalign{}
પધ્ધતિ & મુખ્ય લક્ષણ \\
\midrule\noalign{}
\endhead
\bottomrule\noalign{}
\endlastfoot
ફિક્સ્ડ બાયસ & બેઝ બાયસ માટે એક રેસિસ્ટર \\
કલેક્ટર-ટુ-બેઝ બાયસ & નેગેટિવ ફીડબેક દ્વારા સેલ્ફ-સ્ટેબિલાઈઝિંગ \\
વોલ્ટેજ ડિવાઈડર બાયસ & વોલ્ટેજ ડિવાઈડર નેટવર્ક દ્વારા સૌથી સ્થિર \\
એમિટર બાયસ & એમિટર રેસિસ્ટર સાથે ઉત્તમ સ્થિરતા \\
કોમ્બિનેશન બાયસ & ઓપ્ટિમલ સ્થિરતા માટે મલ્ટિપલ ફીડબેક પાથનો ઉપયોગ \\
\end{longtable}
}

\textbf{વોલ્ટેજ ડિવાઈડર બાયસ સમજૂતી:}

\textbf{આકૃતિ:}

\begin{verbatim}
     +Vcc
       |
       R1
       |
       +{-{-}{-}{-}{-}{-}+}
       |      |
       R2     Rc
       |      |
       |      C
  +{-{-}{-}{-}+      |}
  |   B  +{-{-}{-}{-}+{-}{-}{-}{-}+ Output}
  +{-{-}{-}{-}{-}|    |     |}
        |    |     |
  Input +{-{-}+ |     |}
        | C  E     |
        |    |     |
        +{-{-}{-}{-}+     |}
             |     |
             Re    |
             |     |
             +{-{-}{-}{-}{-}+}
             |
            GND
\end{verbatim}

\begin{itemize}
\tightlist
\item
  \textbf{ઓપરેશન}: R1 અને R2 બેઝ વોલ્ટેજ સેટ કરવા માટે વોલ્ટેજ ડિવાઈડર બનાવે છે
\item
  \textbf{સ્થિરતા}: સ્ટિફ વોલ્ટેજ ડિવાઈડરને કારણે ઉત્તમ થર્મલ સ્થિરતા
\item
  \textbf{કાર્યક્ષમતા}: β વેરિએશનથી સ્વતંત્ર હોવાથી સૌથી વધુ ઉપયોગમાં લેવાતી
  પધ્ધતિ
\item
  \textbf{ગણતરી}: બેઝ વોલ્ટેજ = Vcc \times R2/(R1+R2)
\end{itemize}

\end{solutionbox}
\begin{mnemonicbox}
``VISE ગ્રિપ'' (Voltage divider, Independent of β,
Stable, Efficient)

\end{mnemonicbox}
\subsection*{પ્રશ્ન 1(ક) અથવા [7
ગુણ]}\label{uxaaauxab0uxab6uxaa8-1uxa95-uxa85uxaa5uxab5-7-uxa97uxaa3}

\textbf{સર્કિટ ડાયગ્રામની મદદથી વોલ્ટેજ ડિવાઈડર બાયસિંગ પધ્ધતિ સમજાવો}

\begin{solutionbox}
વોલ્ટેજ ડિવાઈડર બાયસિંગ એ ટ્રાન્ઝિસ્ટરને બાયસ કરવાની સૌથી સ્થિર
પદ્ધતિ છે.

\textbf{આકૃતિ:}

\begin{verbatim}
     +Vcc
       |
       R1
       |
       +{-{-}{-}{-}{-}{-}+}
       |      |
       R2     Rc
       |      |
       |      C
  +{-{-}{-}{-}+      |}
  |   B  +{-{-}{-}{-}+{-}{-}{-}{-}+ Output}
  +{-{-}{-}{-}{-}|    |     |}
        |    |     |
  Input +{-{-}+ |     |}
        | C  E     |
        |    |     |
        +{-{-}{-}{-}+     |}
             |     |
             Re    |
             |     |
             +{-{-}{-}{-}{-}+}
             |
            GND
\end{verbatim}


{\def\LTcaptype{none} % do not increment counter
\vspace{-5pt}
\captionof{table}{વોલ્ટેજ ડિવાઈડર બાયસિંગની વિશેષતાઓ}
\vspace{-10pt}
\begin{longtable}[]{@{}ll@{}}
\toprule\noalign{}
કોમ્પોનન્ટ & કાર્ય \\
\midrule\noalign{}
\endhead
\bottomrule\noalign{}
\endlastfoot
R1, R2 & β થી સ્વતંત્ર સ્થિર બેઝ વોલ્ટેજ બનાવે છે \\
Rc & કલેક્ટર કરંટને મર્યાદિત કરે છે અને આઉટપુટ વોલ્ટેજ વિકસિત કરે છે \\
Re & નેગેટિવ ફીડબેક દ્વારા સ્થિરતા પ્રદાન કરે છે \\
બાયપાસ કેપેસિટર & ગેઇન વધારવા માટે Re ની આસપાસ AC સિગ્નલને બાયપાસ કરે છે \\
\end{longtable}
}

\begin{itemize}
\tightlist
\item
  \textbf{કાર્યરત સિદ્ધાંત}: R1 અને R2 બેઝ વોલ્ટેજ સેટ કરતા વોલ્ટેજ ડિવાઈડર બનાવે છે
\item
  \textbf{થર્મલ સ્થિરતા}: Re નેગેટિવ ફીડબેક માટે ઉત્તમ થર્મલ સ્થિરતા પ્રદાન કરે છે
\item
  \textbf{ફાયદો}: તાપમાન અને β માં ફેરફાર છતાં Q-પોઈન્ટ સ્થિર રહે છે
\end{itemize}

\end{solutionbox}
\begin{mnemonicbox}
``BEST બાયસ'' (Base voltage, Emitter stability,
Stiff divider, Temperature stable)

\end{mnemonicbox}
\subsection*{પ્રશ્ન 2(અ) [3
ગુણ]}\label{uxaaauxab0uxab6uxaa8-2uxa85-3-uxa97uxaa3}

\textbf{કાસ્કેડિંગ એમ્પ્લિફાયરની પદ્ધતિઓ લખો}

\begin{solutionbox}
કાસ્કેડિંગ એમ્પ્લિફાયરનો અર્થ એકંદર ગેઈન વધારવા માટે એકાધિક
એમ્પ્લિફાયર સ્ટેજને શ્રેણીમાં જોડવાનો છે.


{\def\LTcaptype{none} % do not increment counter
\vspace{-5pt}
\captionof{table}{કાસ્કેડિંગ એમ્પ્લિફાયરની પદ્ધતિઓ}
\vspace{-10pt}
\begin{longtable}[]{@{}ll@{}}
\toprule\noalign{}
પદ્ધતિ & મુખ્ય લક્ષણ \\
\midrule\noalign{}
\endhead
\bottomrule\noalign{}
\endlastfoot
RC કપલિંગ & ઇન્ટરસ્ટેજ કપલિંગ માટે કેપેસિટર અને રેસિસ્ટરનો ઉપયોગ \\
ટ્રાન્સફોર્મર કપલિંગ & ઇમ્પીડન્સ મેચિંગ અને આઇસોલેશન માટે ટ્રાન્સફોર્મરનો ઉપયોગ \\
ડાયરેક્ટ કપલિંગ & કોઈ કપલિંગ કોમ્પોનન્ટ નહીં, સ્ટેજ વચ્ચે સીધું કનેક્શન \\
LC કપલિંગ & હાઈ-ફ્રીક્વન્સી એપ્લિકેશન માટે ઇન્ડક્ટર-કેપેસિટરનો ઉપયોગ \\
\end{longtable}
}

\end{solutionbox}
\begin{mnemonicbox}
``RTDL કનેક્શન'' (RC, Transformer, Direct, LC)

\end{mnemonicbox}
\subsection*{પ્રશ્ન 2(બ) [4
ગુણ]}\label{uxaaauxab0uxab6uxaa8-2uxaac-4-uxa97uxaa3}

\textbf{CE અને CB એમ્પ્લિફાયરની સરખામણી કરો}

\begin{solutionbox}


{\def\LTcaptype{none} % do not increment counter
\vspace{-5pt}
\captionof{table}{CE અને CB એમ્પ્લિફાયરની સરખામણી}
\vspace{-10pt}
\begin{longtable}[]{@{}lll@{}}
\toprule\noalign{}
પેરામીટર & કોમન એમિટર (CE) & કોમન બેઝ (CB) \\
\midrule\noalign{}
\endhead
\bottomrule\noalign{}
\endlastfoot
ઇનપુટ ઇમ્પીડન્સ & મધ્યમ (\approx1kΩ) & નીચું (\approx50Ω) \\
આઉટપુટ ઇમ્પીડન્સ & ઊંચું (\approx50kΩ) & ખૂબ ઊંચું (\approx500kΩ) \\
વોલ્ટેજ ગેઇન & ઊંચું (\approx500) & ઊંચું (\approx500) \\
કરંટ ગેઇન & મધ્યમ (β) & 1 થી ઓછું (α) \\
ફેઝ શિફ્ટ & 180^\circ & 0^\circ \\
એપ્લિકેશન & વોલ્ટેજ એમ્પ્લિફિકેશન & હાઈ-ફ્રીક્વન્સી એમ્પ્લિફિકેશન \\
\end{longtable}
}

\end{solutionbox}
\begin{mnemonicbox}
``PIVOT તફાવતો'' (Phase shift, Impedance, Voltage
gain, Output impedance, Throughput)

\end{mnemonicbox}
\subsection*{પ્રશ્ન 2(ક) [7
ગુણ]}\label{uxaaauxab0uxab6uxaa8-2uxa95-7-uxa97uxaa3}

\textbf{RC કપલ્ડ એમ્પ્લિફાયરની સર્કિટ દોરો. આવૃત્તિ પ્રતિભાવ આપો અને સમજાવો}

\begin{solutionbox}
RC કપલ્ડ એમ્પ્લિફાયર ઇન્ટરસ્ટેજ કપલિંગ માટે રેસિસ્ટર-કેપેસિટર નેટવર્કનો
ઉપયોગ કરે છે.

\textbf{આકૃતિ:}

\begin{verbatim}
     +Vcc
       |
       +{-{-}{-}+{-}{-}{-}{-}{-}{-}{-}{-}+{-}{-}{-}+}
       |   |        |   |
       Rc1 |        Rc2 |
       |   |        |   |
       +{-{-}{-}+        +{-}{-}{-}+}
       |            |
       C            +{-{-}{-}+ Output}
       |            |   |
   +{-{-}{-}+{-}{-}{-}+    +{-}{-}{-}+{-}{-}{-}+}
   |   |   |    |   |   |
   | B |   |    | B |   |
   | | C   |    | | C   |
   +{-+ |   |    +{-}+ |   |}
   |   |   |    |   |   |
Input  |   |    |   |   |
   +{-{-}{-}+{-}{-}{-}+    +{-}{-}{-}+{-}{-}{-}+}
       |            |
       Re1          Re2
       |            |
       +{-{-}{-}{-}{-}{-}{-}{-}{-}{-}{-}{-}+}
       |
      GND
\end{verbatim}

\textbf{આવૃત્તિ પ્રતિભાવ:}

\begin{center}
\textbf{Mermaid Diagram (Code)}
\begin{verbatim}
{Shaded}
{Highlighting}[]
graph LR
    A[Low Frequency] {-{-}{-} B[Mid Frequency]}
    B {-{-}{-} C[High Frequency]}

    style A fill:\#f9f,stroke:\#333,stroke{-width:1px}
    style B fill:\#bbf,stroke:\#333,stroke{-width:1px}
    style C fill:\#f9f,stroke:\#333,stroke{-width:1px}
{Highlighting}
{Shaded}
\end{verbatim}
\end{center}

\begin{itemize}
\tightlist
\item
  \textbf{નીચી આવૃત્તિ વિસ્તાર}: કપલિંગ અને બાયપાસ કેપેસિટરને કારણે ગેઈન ઘટે છે
\item
  \textbf{મધ્ય આવૃત્તિ વિસ્તાર}: મહત્તમ ગેઈન સાથે ફ્લેટ પ્રતિસાદ
\item
  \textbf{ઊંચી આવૃત્તિ વિસ્તાર}: ટ્રાન્ઝિસ્ટરની આંતરિક કેપેસિટન્સને કારણે ગેઈન ઘટે છે
\item
  \textbf{બેન્ડવિડ્થ}: નીચા અને ઊંચા કટઓફ આવૃત્તિઓ દ્વારા નક્કી થાય છે
\end{itemize}

\end{solutionbox}
\begin{mnemonicbox}
``LMH વિસ્તારો'' (Low, Mid, High frequency regions)

\end{mnemonicbox}
\subsection*{પ્રશ્ન 2(અ) અથવા [3
ગુણ]}\label{uxaaauxab0uxab6uxaa8-2uxa85-uxa85uxaa5uxab5-3-uxa97uxaa3}

\textbf{એમ્પ્લિફાયરના ગેઇન, બેંડવિથ અને ગેઇન-બેંડવિથ ગુણાકારની વ્યાખ્યા લખો.}

\begin{solutionbox}


{\def\LTcaptype{none} % do not increment counter
\vspace{-5pt}
\captionof{table}{મુખ્ય એમ્પ્લિફાયર પેરામીટર્સ}
\vspace{-10pt}
\begin{longtable}[]{@{}
  >{\raggedright\arraybackslash}p{(\linewidth - 2\tabcolsep) * \real{0.4783}}
  >{\raggedright\arraybackslash}p{(\linewidth - 2\tabcolsep) * \real{0.5217}}@{}}
\toprule\noalign{}
\begin{minipage}[b]{\linewidth}\raggedright
પેરામીટર
\end{minipage} & \begin{minipage}[b]{\linewidth}\raggedright
વ્યાખ્યા
\end{minipage} \\
\midrule\noalign{}
\endhead
\bottomrule\noalign{}
\endlastfoot
ગેઇન (A) & આઉટપુટ સિગ્નલનો ઇનપુટ સિગ્નલ સાથેનો ગુણોત્તર (વોલ્ટેજ, કરંટ, અથવા
પાવર) \\
બેન્ડવિડ્થ (BW) & નીચા અને ઊંચા કટઓફ આવૃત્તિઓ વચ્ચેનો આવૃત્તિ રેન્જ (f_{2}-f_{1}) \\
ગેઇન-બેન્ડવિડ્થ ગુણાકાર (GBW) & ગેઇન અને બેન્ડવિડ્થનો ગુણાકાર, આપેલા એમ્પ્લિફાયર માટે
સ્થિર રહે છે \\
\end{longtable}
}

\end{solutionbox}
\begin{mnemonicbox}
``GBP સ્થિરાંકો'' (Gain, Bandwidth, Product constants)

\end{mnemonicbox}
\subsection*{પ્રશ્ન 2(બ) અથવા [4
ગુણ]}\label{uxaaauxab0uxab6uxaa8-2uxaac-uxa85uxaa5uxab5-4-uxa97uxaa3}

\textbf{સિંગલ સ્ટેજ એમ્પ્લિફાયરનો ફ્રિક્વન્સી રિસ્પોન્સ સમજાવો અને તેની કટઓફ
ફ્રિક્વન્સીઓ દર્શાવો.}

\begin{solutionbox}
ફ્રિક્વન્સી રિસ્પોન્સ સિંગલ સ્ટેજ એમ્પ્લિફાયરમાં આવૃત્તિ સાથે ગેઇનના
ફેરફાર દર્શાવે છે.

\textbf{આકૃતિ:}

\begin{center}
\textbf{Mermaid Diagram (Code)}
\begin{verbatim}
{Shaded}
{Highlighting}[]
graph TD
    A[Frequency Response] {-{-}{} B[Low f Region]}
    A {-{-}{} C[Mid f Region]}
    A {-{-}{} D[High f Region]}
    B {-{-}{} E[f_{1}: Lower Cutoff]}
    D {-{-}{} F[f_{2}: Upper Cutoff]}
    C {-{-}{} G[Maximum Gain]}

    style A fill:\#bbf,stroke:\#333,stroke{-width:1px}
    style B fill:\#f9f,stroke:\#333,stroke{-width:1px}
    style C fill:\#bfb,stroke:\#333,stroke{-width:1px}
    style D fill:\#f9f,stroke:\#333,stroke{-width:1px}
{Highlighting}
{Shaded}
\end{verbatim}
\end{center}

\begin{itemize}
\tightlist
\item
  \textbf{કટઓફ આવૃત્તિઓ}: જ્યાં ગેઇન મહત્તમ ગેઇનના 0.707 ગણા સુધી ઘટે છે તે બિંદુઓ
\item
  \textbf{નીચી કટઓફ આવૃત્તિ (f_{1})}: કપલિંગ અને બાયપાસ કેપેસિટર દ્વારા નિર્ધારિત
  થાય છે
\item
  \textbf{ઊંચી કટઓફ આવૃત્તિ (f_{2})}: ટ્રાન્ઝિસ્ટર જંક્શન કેપેસિટન્સ દ્વારા મર્યાદિત થાય
  છે
\item
  \textbf{બેન્ડવિડ્થ}: f_{1} અને f_{2} વચ્ચેનો આવૃત્તિ રેન્જ (BW = f_{2} - f_{1})
\end{itemize}

\end{solutionbox}
\begin{mnemonicbox}
``LUG પોઈન્ટ્સ'' (Lower cutoff, Upper cutoff, Gain
maximum)

\end{mnemonicbox}
\subsection*{પ્રશ્ન 2(ક) અથવા [7
ગુણ]}\label{uxaaauxab0uxab6uxaa8-2uxa95-uxa85uxaa5uxab5-7-uxa97uxaa3}

\textbf{સામાન્ય કલેક્ટર એમ્પ્લિફાયરની સર્કિટ ડાયગ્રામ દોરો અને સમજાવો}

\begin{solutionbox}
સામાન્ય કલેક્ટર (CC) એમ્પ્લિફાયરને એમિટર ફોલોઅર તરીકે પણ ઓળખવામાં
આવે છે.

\textbf{આકૃતિ:}

\begin{verbatim}
     +Vcc
       |
       Rc
       |
       +
       |
       |    C
       |    |
   +{-{-}{-}+    |}
   |   |    |
   | B |    |
Input{-+|    |}
   |   |    |
   |   |    |
   |   C    |
   |   |    |
   +{-{-}{-}+{-}{-}{-}{-}+{-}{-}+ Output}
       |    |
       Re   |
       |    |
       +{-{-}{-}{-}+}
       |
      GND
\end{verbatim}


{\def\LTcaptype{none} % do not increment counter
\vspace{-5pt}
\captionof{table}{સામાન્ય કલેક્ટર એમ્પ્લિફાયરની વિશેષતાઓ}
\vspace{-10pt}
\begin{longtable}[]{@{}ll@{}}
\toprule\noalign{}
પેરામીટર & લાક્ષણિકતા \\
\midrule\noalign{}
\endhead
\bottomrule\noalign{}
\endlastfoot
વોલ્ટેજ ગેઇન & લગભગ 1 (1 કરતાં ઓછો) \\
કરંટ ગેઇન & ઊંચો (β) \\
ઇનપુટ ઇમ્પીડન્સ & ખૂબ ઊંચી (\approx β \times Re) \\
આઉટપુટ ઇમ્પીડન્સ & ખૂબ નીચી (\approx 1/gm) \\
ફેઝ શિફ્ટ & 0^\circ (કોઈ ફેઝ ઇન્વર્ઝન નહીં) \\
એપ્લિકેશન & ઇમ્પીડન્સ મેચિંગ, બફર સ્ટેજ \\
\end{longtable}
}

\begin{itemize}
\tightlist
\item
  \textbf{કાર્યરત સિદ્ધાંત}: આઉટપુટ એમિટરથી લેવામાં આવે છે, કલેક્ટર ઇનપુટ અને આઉટપુટ
  માટે સામાન્ય છે
\item
  \textbf{મુખ્ય લક્ષણ}: વોલ્ટેજ ફોલોઅર જેમાં આઉટપુટ વોલ્ટેજ ઇનપુટ વોલ્ટેજને અનુસરે છે
\item
  \textbf{મુખ્ય ફાયદો}: ઊંચી ઇનપુટ ઇમ્પીડન્સ અને નીચી આઉટપુટ ઇમ્પીડન્સ
\end{itemize}

\end{solutionbox}
\begin{mnemonicbox}
``BIVOP લક્ષણો'' (Buffer, Impedance matching, Voltage
follower, One gain, Phase matched)

\end{mnemonicbox}
\subsection*{પ્રશ્ન 3(અ) [3
ગુણ]}\label{uxaaauxab0uxab6uxaa8-3uxa85-3-uxa97uxaa3}

\textbf{ટ્રાન્ઝિસ્ટર ટુ પોર્ટ નેટવર્ક દોરો અને તેના માટે h-પેરામીટરનું વર્ણન કરો.}

\begin{solutionbox}
ટ્રાન્ઝિસ્ટરને h-પેરામીટર્સ સાથે ટુ-પોર્ટ નેટવર્ક તરીકે રજૂ કરી શકાય
છે.

\textbf{આકૃતિ:}

\begin{verbatim}
     +{-{-}{-}{-}{-}{-}{-}{-}{-}{-}{-}{-}{-}{-}{-}{-}{-}{-}{-}{-}{-}+}
     |                     |
     |    Two{-Port         |}
Input|    Network     +{-{-}{-}{-}+{-}{-}{-} Output}
   +{|                |    |}
     |    Transistor  |    |
     |                v    |
     +{-{-}{-}{-}{-}{-}{-}{-}{-}{-}{-}{-}{-}{-}{-}{-}{-}{-}{-}{-}{-}+}
\end{verbatim}


{\def\LTcaptype{none} % do not increment counter
\vspace{-5pt}
\captionof{table}{h-પેરામીટર્સ}
\vspace{-10pt}
\begin{longtable}[]{@{}ll@{}}
\toprule\noalign{}
પેરામીટર & વર્ણન \\
\midrule\noalign{}
\endhead
\bottomrule\noalign{}
\endlastfoot
h_{1}_{1} (h\_i) & આઉટપુટ શોર્ટ-સર્કિટેડ હોય ત્યારે ઇનપુટ ઇમ્પીડન્સ \\
h_{1}_{2} (h\_r) & ઇનપુટ ઓપન-સર્કિટેડ હોય ત્યારે રિવર્સ વોલ્ટેજ ટ્રાન્સફર રેશિયો \\
h_{2}_{1} (h\_f) & આઉટપુટ શોર્ટ-સર્કિટેડ હોય ત્યારે ફોરવર્ડ કરંટ ટ્રાન્સફર રેશિયો \\
h_{2}_{2} (h\_o) & ઇનપુટ ઓપન-સર્કિટેડ હોય ત્યારે આઉટપુટ એડમિટન્સ \\
\end{longtable}
}

\end{solutionbox}
\begin{mnemonicbox}
``IRFO પેરામીટર્સ'' (Input impedance, Reverse
transfer, Forward transfer, Output admittance)

\end{mnemonicbox}
\subsection*{પ્રશ્ન 3(બ) [4
ગુણ]}\label{uxaaauxab0uxab6uxaa8-3uxaac-4-uxa97uxaa3}

\textbf{CE એમ્પ્લિફાયર માટે વોલ્ટેજ ગેઇન Av, કરંટ ગેઇન Ai અને પાવર ગેઇન Ap સમજાવો}

\begin{solutionbox}


{\def\LTcaptype{none} % do not increment counter
\vspace{-5pt}
\captionof{table}{CE એમ્પ્લિફાયર માટે ગેઇન એક્સપ્રેશન્સ}
\vspace{-10pt}
\begin{longtable}[]{@{}lll@{}}
\toprule\noalign{}
ગેઇન પ્રકાર & એક્સપ્રેશન & h-પેરામીટર્સ સાથે સંબંધ \\
\midrule\noalign{}
\endhead
\bottomrule\noalign{}
\endlastfoot
વોલ્ટેજ ગેઇન (Av) & V_{o}/Vᵢ & Av = -h\_fe \times R\_L / h\_ie \\
કરંટ ગેઇન (Ai) & I_{o}/Iᵢ & Ai = h\_fe / (1 + h\_oe \times R\_L) \\
પાવર ગેઇન (Ap) & P_{o}/Pᵢ & Ap = Av \times Ai = (વોલ્ટેજ ગેઇન \times કરંટ ગેઇન) \\
\end{longtable}
}

\begin{itemize}
\tightlist
\item
  \textbf{વોલ્ટેજ ગેઇન}: CE એમ્પ્લિફાયર માટે સામાન્ય રીતે 500-1000
\item
  \textbf{કરંટ ગેઇન}: ટ્રાન્ઝિસ્ટરના h\_fe (β) જેટલું
\item
  \textbf{પાવર ગેઇન}: વોલ્ટેજ ગેઇન અને કરંટ ગેઇનનો ગુણાકાર
\end{itemize}

\end{solutionbox}
\begin{mnemonicbox}
``VIP ગેઇન્સ'' (Voltage, Input-output current, Power)

\end{mnemonicbox}
\subsection*{પ્રશ્ન 3(ક) [7
ગુણ]}\label{uxaaauxab0uxab6uxaa8-3uxa95-7-uxa97uxaa3}

\textbf{ડાર્લિંગટન પેર, તેની વિશેષતાઓ અને ઉપયોગો સમજાવો}

\begin{solutionbox}
ડાર્લિંગટન પેરમાં બે ટ્રાન્ઝિસ્ટર હોય છે જે એક ઉચ્ચ-ગેઇન ટ્રાન્ઝિસ્ટર
તરીકે કાર્ય કરે છે.

\textbf{આકૃતિ:}

\begin{verbatim}
     +Vcc
       |
       Rc
       |
       +{-{-}{-}{-}{-}{-}{-}{-}+}
       |        |
       |        | Output
       |        |
       | +{-{-}{-}{-}{-}{-}+}
       | |
       | C
  +{-{-}{-}{-}+{-}+}
  |    | |
  |  +{-+ |}
  |  |   |
  |  | C |
  |  | | |
Input  | | |
  +{-{-}{-}{-}+{-}+ |}
       |   |
       +{-{-}{-}+}
       |
      GND
\end{verbatim}


{\def\LTcaptype{none} % do not increment counter
\vspace{-5pt}
\captionof{table}{ડાર્લિંગટન પેરની વિશેષતાઓ}
\vspace{-10pt}
\begin{longtable}[]{@{}ll@{}}
\toprule\noalign{}
વિશેષતા & વર્ણન \\
\midrule\noalign{}
\endhead
\bottomrule\noalign{}
\endlastfoot
કરંટ ગેઇન & ખૂબ ઊંચો (β_{1} \times β_{2}) \\
ઇનપુટ ઇમ્પીડન્સ & અત્યંત ઊંચી \\
વોલ્ટેજ ડ્રોપ & વધારે (\approx1.4V) બે B-E જંક્શનને કારણે \\
સ્વિચિંગ સ્પીડ & સિંગલ ટ્રાન્ઝિસ્ટર કરતાં ધીમી \\
થર્મલ સ્ટેબિલિટી & સિંગલ ટ્રાન્ઝિસ્ટર કરતાં નબળી \\
\end{longtable}
}

\begin{itemize}
\tightlist
\item
  \textbf{ઉપયોગો}: પાવર એમ્પ્લિફાયર, મોટર ડ્રાઈવર, ટચ સ્વિચ, સેન્સર
\item
  \textbf{ફાયદા}: ખૂબ ઊંચો કરંટ ગેઇન, ઊંચી ઇનપુટ ઇમ્પીડન્સ
\item
  \textbf{મર્યાદાઓ}: ઊંચો સેચુરેશન વોલ્ટેજ, ધીમું સ્વિચિંગ
\end{itemize}

\end{solutionbox}
\begin{mnemonicbox}
``CHIPS એપ્લિકેશન'' (Current amplification, High
impedance, Increased gain, Power handling, Slower switching)

\end{mnemonicbox}
\subsection*{પ્રશ્ન 3(અ) અથવા [3
ગુણ]}\label{uxaaauxab0uxab6uxaa8-3uxa85-uxa85uxaa5uxab5-3-uxa97uxaa3}

\textbf{LDR ના ઉપયોગની ચર્ચા કરો.}

\begin{solutionbox}
Light Dependent Resistor (LDR) એક ફોટોરેસિસ્ટર છે જેનો
રેસિસ્ટન્સ પ્રકાશની તીવ્રતા વધવાની સાથે ઘટે છે.


{\def\LTcaptype{none} % do not increment counter
\vspace{-5pt}
\captionof{table}{LDR ના ઉપયોગો}
\vspace{-10pt}
\begin{longtable}[]{@{}ll@{}}
\toprule\noalign{}
ઉપયોગ & કાર્ય સિદ્ધાંત \\
\midrule\noalign{}
\endhead
\bottomrule\noalign{}
\endlastfoot
ઓટોમેટિક સ્ટ્રીટ લાઈટ્સ & જ્યારે એમ્બિયન્ટ લાઈટ લેવલ ઘટે ત્યારે લાઈટ ચાલુ કરે છે \\
કેમેરા એક્સપોઝર કંટ્રોલ & પ્રકાશની તીવ્રતાના આધારે એપર્ચર/શટર એડજસ્ટ કરે છે \\
લાઈટ બીમ અલાર્મ & જ્યારે પ્રકાશનો બીમ અવરોધિત થાય ત્યારે અલાર્મ ટ્રિગર કરે છે \\
સોલર ટ્રેકર & સોલર પેનલને મહત્તમ સૂર્યપ્રકાશ તરફ ઓરિએન્ટ કરવામાં મદદ કરે છે \\
ઓટોમેટિક બ્રાઈટનેસ કંટ્રોલ & એમ્બિયન્ટ લાઈટના આધારે ડિસ્પ્લે બ્રાઈટનેસ એડજસ્ટ કરે છે \\
\end{longtable}
}

\end{solutionbox}
\begin{mnemonicbox}
``CASAL ઉપયોગો'' (Camera, Alarm, Street light,
Automatic control, Light measurement)

\end{mnemonicbox}
\subsection*{પ્રશ્ન 3(બ) અથવા [4
ગુણ]}\label{uxaaauxab0uxab6uxaa8-3uxaac-uxa85uxaa5uxab5-4-uxa97uxaa3}

\textbf{ક્લિપર અને ક્લેમ્પરની સરખામણી}

\begin{solutionbox}


{\def\LTcaptype{none} % do not increment counter
\vspace{-5pt}
\captionof{table}{ક્લિપર અને ક્લેમ્પર વચ્ચેની સરખામણી}
\vspace{-10pt}
\begin{longtable}[]{@{}lll@{}}
\toprule\noalign{}
પેરામીટર & ક્લિપર & ક્લેમ્પર \\
\midrule\noalign{}
\endhead
\bottomrule\noalign{}
\endlastfoot
કાર્ય & સિગ્નલની એમ્પ્લિટ્યુડ મર્યાદિત/ક્લિપ કરે છે & સિગ્નલનું DC લેવલ શિફ્ટ કરે છે \\
આઉટપુટ & થ્રેશોલ્ડથી બહારના ભાગો દૂર કરે છે & DC કોમ્પોનન્ટ ઉમેરે છે \\
કોમ્પોનન્ટ & ડાયોડ + રેસિસ્ટર & ડાયોડ + કેપેસિટર + રેસિસ્ટર \\
વેવ શેપ & વેવ શેપ બદલે છે & વેવ શેપ જાળવે છે \\
ઉપયોગો & નોઈઝ રિમૂવલ, વેવ શેપિંગ & TV સિગ્નલ પ્રોસેસિંગ, DC રિસ્ટોરેશન \\
\end{longtable}
}

\textbf{આકૃતિ:}

\begin{center}
\textbf{Mermaid Diagram (Code)}
\begin{verbatim}
{Shaded}
{Highlighting}[]
graph LR
    A[Input Signal] {-{-}{} B[Clipper]}
    A {-{-}{} C[Clamper]}
    B {-{-}{} D[Amplitude Limited]}
    C {-{-}{} E[DC Level Shifted]}

    style A fill:\#bbf,stroke:\#333,stroke{-width:1px}
    style B fill:\#f9f,stroke:\#333,stroke{-width:1px}
    style C fill:\#bfb,stroke:\#333,stroke{-width:1px}
    style D fill:\#f9f,stroke:\#333,stroke{-width:1px}
    style E fill:\#bfb,stroke:\#333,stroke{-width:1px}
{Highlighting}
{Shaded}
\end{verbatim}
\end{center}

\end{solutionbox}
\begin{mnemonicbox}
``CLIPS vs CLAMPS'' (Cut Levels In Peak Signal vs
Change Level And Maintain Peak Shape)

\end{mnemonicbox}
\subsection*{પ્રશ્ન 3(ક) અથવા [7
ગુણ]}\label{uxaaauxab0uxab6uxaa8-3uxa95-uxa85uxaa5uxab5-7-uxa97uxaa3}

\textbf{CE એમ્પ્લિફાયર માટે h-પેરામીટર સર્કિટનું વર્ણન કરો.}

\begin{solutionbox}
h-પેરામીટર્સ CE એમ્પ્લિફાયર પરફોર્મન્સ વિશ્લેષણની સરળ રીત પ્રદાન
કરે છે.

\textbf{આકૃતિ:}

\begin{verbatim}
     +{-{-}{-}{-}{-}{-}{-}{-}{-}{-}{-}{-}{-}{-}{-}{-}{-}{-}{-}{-}{-}+}
     |                     |
  Ii |    +{-{-}{-}{-}{-}{-}{-}{-}{-}{-}+     | Io}
   +{|    |          |     |+}
     |    |   h\_ie   |     |
  Vi |    +{-{-}{-}{-}{-}{-}{-}{-}{-}{-}+     | Vo}
   +{|    |          |     |+}
     |    | h\_re.Vi  |     |
     |    |          |     |
     |    |  h\_fe.Ii |{-{-}{-}{-}|}
     |    |          |     |
     |    |   h\_oe   |     |
     |    +{-{-}{-}{-}{-}{-}{-}{-}{-}{-}+     |}
     |                     |
     +{-{-}{-}{-}{-}{-}{-}{-}{-}{-}{-}{-}{-}{-}{-}{-}{-}{-}{-}{-}{-}+}
\end{verbatim}


{\def\LTcaptype{none} % do not increment counter
\vspace{-5pt}
\captionof{table}{CE કોન્ફિગરેશન માટે h-પેરામીટર્સ}
\vspace{-10pt}
\begin{longtable}[]{@{}llll@{}}
\toprule\noalign{}
પેરામીટર & સિમ્બોલ & ટિપિકલ વેલ્યુ & ફિઝિકલ સિગ્નિફિકન્સ \\
\midrule\noalign{}
\endhead
\bottomrule\noalign{}
\endlastfoot
ઇનપુટ ઇમ્પીડન્સ & h\_ie & 1-2 kΩ & બેઝ-એમિટર ઇનપુટ ઇમ્પીડન્સ \\
રિવર્સ વોલ્ટેજ રેશિયો & h\_re & 10^{-}^{4} & આઉટપુટથી ઇનપુટ તરફ ફીડબેક \\
ફોરવર્ડ કરંટ ગેઇન & h\_fe & 50-300 & કરંટ ગેઇન (β) \\
આઉટપુટ એડમિટન્સ & h\_oe & 10^{-}^{6} S & આઉટપુટ કન્ડક્ટન્સ \\
\end{longtable}
}

\begin{itemize}
\tightlist
\item
  \textbf{સર્કિટ એનાલિસિસ}: વોલ્ટેજ ગેઇન, કરંટ ગેઇન, ઇનપુટ/આઉટપુટ ઇમ્પીડન્સની
  ગણતરી માટે h-પેરામીટર્સનો ઉપયોગ
\item
  \textbf{ઇક્વિવેલન્ટ સર્કિટ}: h-પેરામીટર્સને ટુ-પોર્ટ નેટવર્ક રેપ્રેઝન્ટેશનમાં સંયોજિત કરે
  છે
\item
  \textbf{ફાયદો}: જટિલ ટ્રાન્ઝિસ્ટર વર્તનને લિનિયર પેરામીટર્સમાં સરળ બનાવે છે
\end{itemize}

\end{solutionbox}
\begin{mnemonicbox}
``FIRO પેરામીટર્સ'' (Forward gain, Input impedance,
Reverse feedback, Output admittance)

\end{mnemonicbox}
\subsection*{પ્રશ્ન 4(અ) [3
ગુણ]}\label{uxaaauxab0uxab6uxaa8-4uxa85-3-uxa97uxaa3}

\textbf{ડાર્લિંગટન જોડી પર ટૂંકી નોંધ લખો.}

\begin{solutionbox}
ડાર્લિંગટન જોડી બે ટ્રાન્ઝિસ્ટરને સંયોજિત કરીને સુપર-હાઈ ગેઇન
ટ્રાન્ઝિસ્ટર બનાવે છે.

\textbf{આકૃતિ:}

\begin{verbatim}
       +{-{-}{-}+}
       |   |
  +{-{-}{-}{-}+{-}{-}{-}+{-}{-}{-}{-}+}
  |    |        |
  |    +        |
  |  E/ {C      |}
  |   /B {      |}
Input + {-{-}+     |}
  |    |        |
  |    +        |
  |  E/ {C      |}
  |   /B {      |}
  |    |        |
  +{-{-}{-}{-}+{-}{-}{-}{-}{-}{-}{-}{-}+}
       |
     Output
\end{verbatim}

\begin{itemize}
\tightlist
\item
  \textbf{કોન્ફિગરેશન}: બે ટ્રાન્ઝિસ્ટર જેમાં પ્રથમ ટ્રાન્ઝિસ્ટરનો એમિટર બીજા
  ટ્રાન્ઝિસ્ટરના બેઝને ડ્રાઇવ કરે છે
\item
  \textbf{કુલ ગેઇન}: β_{1} \times β_{2} (વ્યક્તિગત ટ્રાન્ઝિસ્ટર ગેઇનનો ગુણાકાર)
\item
  \textbf{ઇનપુટ ઇમ્પીડન્સ}: અત્યંત ઊંચી (β_{2} \times R\_e1)
\end{itemize}

\end{solutionbox}
\begin{mnemonicbox}
``HIS ગુણધર્મો'' (High gain, Impedance boost, Sandwich
configuration)

\end{mnemonicbox}
\subsection*{પ્રશ્ન 4(બ) [4
ગુણ]}\label{uxaaauxab0uxab6uxaa8-4uxaac-4-uxa97uxaa3}

\textbf{ઝેનર ડાયોડને વોલ્ટેજ રેગ્યુલેટર તરીકે સમજાવો.}

\begin{solutionbox}
ઝેનર ડાયોડ રિવર્સ બ્રેકડાઉનમાં ઓપરેટ થાય ત્યારે સ્થિર વોલ્ટેજ રેફરન્સ
પ્રદાન કરે છે.

\textbf{આકૃતિ:}

\begin{verbatim}
       Rs
    +{-{-}www{-}{-}+}
    |        |
 Vi |      | | Vz    RL   Vo
 +{-{-}+      | +{-}{-}{-}+{-}{-}{-}www{-}{-}+}
    |      |/|   |        |
    |      |{|   |        |}
    |        |   |        |
    +{-{-}{-}{-}{-}{-}{-}{-}+{-}{-}{-}+{-}{-}{-}{-}{-}{-}{-}{-}+}
             |
            GND
\end{verbatim}


{\def\LTcaptype{none} % do not increment counter
\vspace{-5pt}
\captionof{table}{ઝેનર વોલ્ટેજ રેગ્યુલેટર}
\vspace{-10pt}
\begin{longtable}[]{@{}ll@{}}
\toprule\noalign{}
પેરામીટર & વર્ણન \\
\midrule\noalign{}
\endhead
\bottomrule\noalign{}
\endlastfoot
સિદ્ધાંત & રિવર્સ બ્રેકડાઉન રીજિયનમાં સ્થિર વોલ્ટેજ જાળવે છે \\
સીરીઝ રેસિસ્ટર (Rs) & કરંટ મર્યાદિત કરે છે અને વધારાનો વોલ્ટેજ ડ્રોપ કરે છે \\
લોડ રેસિસ્ટર (RL) & પાવર લેતા સર્કિટનું પ્રતિનિધિત્વ કરે છે \\
રેગ્યુલેશન & ઇનપુટ વોલ્ટેજની વધઘટ છતાં આઉટપુટ વોલ્ટેજ સ્થિર રાખે છે \\
\end{longtable}
}

\begin{itemize}
\tightlist
\item
  \textbf{કાર્યપદ્ધતિ}: ઝેનર બ્રેકડાઉન રીજિયનમાં કાર્ય કરે છે, સ્થિર વોલ્ટેજ જાળવે છે
\item
  \textbf{મર્યાદા}: પાવર ડિસિપેશન ક્ષમતા મહત્તમ કરંટને મર્યાદિત કરે છે
\end{itemize}

\end{solutionbox}
\begin{mnemonicbox}
``ZEBRA'' (Zener Effect Breakdown Regulates
Accurately)

\end{mnemonicbox}
\subsection*{પ્રશ્ન 4(ક) [7
ગુણ]}\label{uxaaauxab0uxab6uxaa8-4uxa95-7-uxa97uxaa3}

\textbf{ઓપ્ટોકપલર ને ફાયદા અને ગેરફાયદા સાથે સમજાવો.}

\begin{solutionbox}
ઓપ્ટોકપલર (ઓપ્ટોઆઇસોલેટર તરીકે પણ ઓળખાય છે) આઇસોલેટેડ સર્કિટ વચ્ચે
સિગ્નલ ટ્રાન્સફર કરવા માટે પ્રકાશનો ઉપયોગ કરે છે.

\textbf{આકૃતિ:}

\begin{verbatim}
   +{-{-}{-}{-}{-}{-}{-}{-}+{-}{-}{-}{-}{-}{-}{-}{-}{-}{-}{-}+}
   |        |           |
   |    +{-{-}{-}+{-}{-}{-}+       |}
   |    |       |       |
   |    |  LED  |       |
   |    |       |       |
   |    +{-{-}{-}+{-}{-}{-}+       |}
Input   |   |           | Output
   |    |   |     +{-{-}{-}{-}{-}+{-}{-}{-}{-}+}
   |    |   |     |     |    |
   |    |   +{-{-}{-}{-}|     |    |}
   |    |         |Photo|    |
   |    |         |sensor|   |
   |    |         |     |    |
   |    |         +{-{-}{-}{-}{-}+{-}{-}{-}{-}+}
   |    |           |        |
   +{-{-}{-}{-}+{-}{-}{-}{-}{-}{-}{-}{-}{-}{-}{-}+{-}{-}{-}{-}{-}{-}{-}{-}+}
\end{verbatim}


{\def\LTcaptype{none} % do not increment counter
\vspace{-5pt}
\captionof{table}{ઓપ્ટોકપલરના ફાયદા અને ગેરફાયદા}
\vspace{-10pt}
\begin{longtable}[]{@{}ll@{}}
\toprule\noalign{}
ફાયદા & ગેરફાયદા \\
\midrule\noalign{}
\endhead
\bottomrule\noalign{}
\endlastfoot
સંપૂર્ણ ઇલેક્ટ્રિકલ આઇસોલેશન & અપેક્ષાકૃત ધીમો રિસ્પોન્સ ટાઇમ \\
ઉચ્ચ નોઇઝ ઇમ્યુનિટી & મર્યાદિત બેન્ડવિડ્થ \\
ગ્રાઉન્ડ લૂપ્સ નથી & તાપમાન સંવેદનશીલ \\
ઉચ્ચ વોલ્ટેજ આઇસોલેશન & એજિંગ ઇફેક્ટ્સ \\
ટ્રાન્ઝિઅન્ટ્સ સામે સુરક્ષા & LED ડ્રાઇવ કરવા માટે કરંટની જરૂર પડે છે \\
\end{longtable}
}

\begin{itemize}
\tightlist
\item
  \textbf{કાર્યપદ્ધતિ}: ઇનપુટ સિગ્નલ LED ને ડ્રાઇવ કરે છે, જે પ્રકાશ ઉત્સર્જિત કરે છે
  અને ફોટોડિટેક્ટર દ્વારા શોધાય છે
\item
  \textbf{ઉપયોગો}: મેડિકલ ઇક્વિપમેન્ટ, ઇન્ડસ્ટ્રિયલ કંટ્રોલ, પાવર સપ્લાય, સિગ્નલ
  આઇસોલેશન
\item
  \textbf{પ્રકારો}: ફોટોરેસિસ્ટર, ફોટોડાયોડ, ફોટોટ્રાન્ઝિસ્ટર, ફોટો-SCR આધારિત
\end{itemize}

\end{solutionbox}
\begin{mnemonicbox}
``LIGHT ટ્રાન્સફર'' (Linked Isolated Galvanic-free
High-voltage Transfer)

\end{mnemonicbox}
\subsection*{પ્રશ્ન 4(અ) અથવા [3
ગુણ]}\label{uxaaauxab0uxab6uxaa8-4uxa85-uxa85uxaa5uxab5-3-uxa97uxaa3}

\textbf{હાફ વેવ વોલ્ટેજ ડબલર દોરો.}

\begin{solutionbox}
હાફ-વેવ વોલ્ટેજ ડબલર ડાયોડ અને કેપેસિટરનો ઉપયોગ કરીને ઇનપુટ પીક
વોલ્ટેજના લગભગ બમણા DC આઉટપુટ ઉત્પન્ન કરે છે.

\textbf{આકૃતિ:}

\begin{verbatim}
              D1
     +{-{-}{-}{-}{-}{-}{-}{-}+{-}{-}||{-}{-}{-}{-}{-}{-}{-}{-}{-}{-}+}
     |                        |
     |                        |
     |                        |
    \_|\_                      \_|\_
    { / D2                   {-}{-}{-} C2}
     |                        |
     |                        |
 Vin |        +{-{-}{-}{-}{-}{-}{-}{-}{-}{-}{-}{-}{-}{-}{-}+{-}{-}+ Vout(2Vin)}
     |        |
     |        |
     |       \_|\_
     |       {-{-}{-} C1}
     |        |
     +{-{-}{-}{-}{-}{-}{-}{-}+{-}{-}{-}{-}{-}{-}{-}{-}{-}{-}{-}{-}{-}{-}{-}{-}+}
\end{verbatim}

\begin{itemize}
\tightlist
\item
  \textbf{કોમ્પોનન્ટ્સ}: બે ડાયોડ અને બે કેપેસિટર
\item
  \textbf{આઉટપુટ}: ઇનપુટ પીક વોલ્ટેજના લગભગ બમણા
\end{itemize}

\end{solutionbox}
\begin{mnemonicbox}
``DC2'' (Doubles input using Capacitors and 2
Diodes)

\end{mnemonicbox}
\subsection*{પ્રશ્ન 4(બ) અથવા [4
ગુણ]}\label{uxaaauxab0uxab6uxaa8-4uxaac-uxa85uxaa5uxab5-4-uxa97uxaa3}

\textbf{OLED નું કાર્ય અને ઉપયોગો સમજાવો.}

\begin{solutionbox}
ઓર્ગેનિક લાઇટ એમિટિંગ ડાયોડ (OLED) ઓર્ગેનિક કોમ્પાઉન્ડનો ઉપયોગ
કરે છે જે તેમાંથી કરંટ પસાર થાય ત્યારે પ્રકાશ ઉત્સર્જિત કરે છે.

\textbf{આકૃતિ:}

\begin{center}
\textbf{Mermaid Diagram (Code)}
\begin{verbatim}
{Shaded}
{Highlighting}[]
graph TD
    A[OLED Structure] {-{-}{} B[Cathode]}
    A {-{-}{} C[Organic Layer]}
    A {-{-}{} D[Anode]}
    A {-{-}{} E[Substrate]}

    style A fill:\#bbf,stroke:\#333,stroke{-width:1px}
    style B fill:\#f9f,stroke:\#333,stroke{-width:1px}
    style C fill:\#bfb,stroke:\#333,stroke{-width:1px}
    style D fill:\#f9f,stroke:\#333,stroke{-width:1px}
    style E fill:\#bfb,stroke:\#333,stroke{-width:1px}
{Highlighting}
{Shaded}
\end{verbatim}
\end{center}


{\def\LTcaptype{none} % do not increment counter
\vspace{-5pt}
\captionof{table}{OLED કાર્ય અને ઉપયોગો}
\vspace{-10pt}
\begin{longtable}[]{@{}ll@{}}
\toprule\noalign{}
પાસું & વર્ણન \\
\midrule\noalign{}
\endhead
\bottomrule\noalign{}
\endlastfoot
કાર્યપદ્ધતિ & ઓર્ગેનિક લેયરમાં ઇલેક્ટ્રોન-હોલ રિકોમ્બિનેશન પ્રકાશ ઉત્પન્ન કરે છે \\
કાર્યક્ષમતા & ઉચ્ચ કાર્યક્ષમતા, ઓછા પાવરનો વપરાશ \\
વ્યૂઇંગ એન્ગલ & ઉત્તમ (લગભગ 180^\circ) \\
ઉપયોગો & સ્માર્ટફોન, ટીવી, વેરેબલ ડિવાઇસ, લાઇટિંગ \\
ફાયદા & પાતળી, ફ્લેક્સિબલ, વધુ સારું કોન્ટ્રાસ્ટ, ઝડપી રિસ્પોન્સ \\
\end{longtable}
}

\end{solutionbox}
\begin{mnemonicbox}
``VIEWS ટેકનોલોજી'' (Vibrant colors, Incredible
contrast, Excellent angle, Wide application, Self-emitting)

\end{mnemonicbox}
\subsection*{પ્રશ્ન 4(ક) અથવા [7
ગુણ]}\label{uxaaauxab0uxab6uxaa8-4uxa95-uxa85uxaa5uxab5-7-uxa97uxaa3}

\textbf{સોલર બેટરી ચાર્જર સર્કિટનું કાર્ય સમજાવો.}

\begin{solutionbox}
સોલર બેટરી ચાર્જર સૌર ઊર્જાને બેટરી ચાર્જ કરવા માટે ઇલેક્ટ્રિકલ
ઊર્જામાં રૂપાંતરિત કરે છે.

\textbf{આકૃતિ:}

\begin{verbatim}
   +{-{-}{-}{-}{-}{-}{-}{-}+        +{-}{-}{-}{-}{-}{-}{-}{-}{-}{-}+        +{-}{-}{-}{-}{-}{-}{-}{-}{-}+        +{-}{-}{-}{-}{-}{-}{-}{-}+}
   |        |        |          |        |         |        |        |
   | Solar  |        | Charge   |        | Voltage |        | Battery|
   | Panel  |{-{-}{-}{-}{-}{-}{-}|Controller|{-}{-}{-}{-}{-}{-}{-}|Regulator|{-}{-}{-}{-}{-}{-}{-}|        |}
   |        |        |          |        |         |        |        |
   +{-{-}{-}{-}{-}{-}{-}{-}+        +{-}{-}{-}{-}{-}{-}{-}{-}{-}{-}+        +{-}{-}{-}{-}{-}{-}{-}{-}{-}+        +{-}{-}{-}{-}{-}{-}{-}{-}+}
                         |                                      |
                         |                                      |
                         v                                      v
                     +{-{-}{-}{-}{-}{-}{-}{-}{-}+                             +{-}{-}{-}{-}{-}{-}{-}{-}+}
                     |Indicator|                             |  Load  |
                     | Circuit |                             |        |
                     +{-{-}{-}{-}{-}{-}{-}{-}{-}+                             +{-}{-}{-}{-}{-}{-}{-}{-}+}
\end{verbatim}


{\def\LTcaptype{none} % do not increment counter
\vspace{-5pt}
\captionof{table}{કોમ્પોનન્ટ્સ અને તેમના કાર્યો}
\vspace{-10pt}
\begin{longtable}[]{@{}ll@{}}
\toprule\noalign{}
કોમ્પોનન્ટ & કાર્ય \\
\midrule\noalign{}
\endhead
\bottomrule\noalign{}
\endlastfoot
સોલર પેનલ & સૂર્યપ્રકાશને DC ઇલેક્ટ્રિસિટીમાં રૂપાંતરિત કરે છે \\
ચાર્જ કંટ્રોલર & ઓવરચાર્જિંગ અને ડીપ ડિસ્ચાર્જ અટકાવે છે \\
વોલ્ટેજ રેગ્યુલેટર & યોગ્ય ચાર્જિંગ લેવલ પર વોલ્ટેજ સ્થિર કરે છે \\
બેટરી & ઇલેક્ટ્રિકલ ઊર્જા સંગ્રહિત કરે છે \\
ઇન્ડિકેટર સર્કિટ & ચાર્જિંગ સ્ટેટસ અને બેટરી લેવલ દર્શાવે છે \\
\end{longtable}
}

\begin{itemize}
\tightlist
\item
  \textbf{કાર્ય સિદ્ધાંત}: ફોટોવોલ્ટેઇક ઇફેક્ટ સૂર્યપ્રકાશને ઇલેક્ટ્રિસિટીમાં રૂપાંતરિત
  કરે છે
\item
  \textbf{રેગ્યુલેશન}: વોલ્ટેજ/કરંટ રેગ્યુલેશન દ્વારા ઓવરચાર્જિંગ અટકાવે છે
\item
  \textbf{સુરક્ષા}: રાત્રે બેટરી ડિસ્ચાર્જ થતી અટકાવવા માટે રિવર્સ કરંટ પ્રોટેક્શન
  સામેલ છે
\item
  \textbf{પ્રકારો}: PWM (પલ્સ વિડ્થ મોડ્યુલેશન) અને MPPT (મેક્સિમમ પાવર પોઇન્ટ
  ટ્રેકિંગ)
\end{itemize}

\end{solutionbox}
\begin{mnemonicbox}
``SCORE સિસ્ટમ'' (Solar Conversion, Overcharge
protection, Regulation, Energy storage)

\end{mnemonicbox}
\subsection*{પ્રશ્ન 5(અ) [3
ગુણ]}\label{uxaaauxab0uxab6uxaa8-5uxa85-3-uxa97uxaa3}

\textbf{રેગ્યુલેટેડ પાવર સપ્લાયનો બ્લોક ડાયાગ્રામ દોરો.}

\begin{solutionbox}
રેગ્યુલેટેડ પાવર સપ્લાય ઇનપુટ અથવા લોડમાં ફેરફાર છતાં સ્થિર DC આઉટપુટ
વોલ્ટેજ પ્રદાન કરે છે.

\textbf{આકૃતિ:}

\begin{center}
\textbf{Mermaid Diagram (Code)}
\begin{verbatim}
{Shaded}
{Highlighting}[]
graph LR
    A[Transformer] {-{-}{} B[Rectifier]}
    B {-{-}{} C[Filter]}
    C {-{-}{} D[Voltage Regulator]}
    D {-{-}{} E[Output]}

    style A fill:\#f9f,stroke:\#333,stroke{-width:1px}
    style B fill:\#bbf,stroke:\#333,stroke{-width:1px}
    style C fill:\#bfb,stroke:\#333,stroke{-width:1px}
    style D fill:\#f9f,stroke:\#333,stroke{-width:1px}
    style E fill:\#bbf,stroke:\#333,stroke{-width:1px}
{Highlighting}
{Shaded}
\end{verbatim}
\end{center}

\begin{itemize}
\tightlist
\item
  \textbf{કોમ્પોનન્ટ્સ}: ટ્રાન્સફોર્મર, રેક્ટિફાયર, ફિલ્ટર, વોલ્ટેજ રેગ્યુલેટર
\item
  \textbf{કાર્ય}: લોડ ચેન્જ છતાં AC ને સ્થિર DC માં રૂપાંતરિત કરે છે
\end{itemize}

\end{solutionbox}
\begin{mnemonicbox}
``TRFO બ્લોક્સ'' (Transformer, Rectifier, Filter,
Output regulator)

\end{mnemonicbox}
\subsection*{પ્રશ્ન 5(બ) [4
ગુણ]}\label{uxaaauxab0uxab6uxaa8-5uxaac-4-uxa97uxaa3}

\textbf{ટ્રાન્ઝિસ્ટર શંટ વોલ્ટેજ રેગ્યુલેટરનું વર્ણન કરો.}

\begin{solutionbox}
ટ્રાન્ઝિસ્ટર શંટ રેગ્યુલેટર લોડની સમાંતર ટ્રાન્ઝિસ્ટરમાંથી વધારાના
કરંટને ડાઇવર્ટ કરીને સ્થિર આઉટપુટ વોલ્ટેજ જાળવે છે.

\textbf{આકૃતિ:}

\begin{verbatim}
     +{-{-}{-}{-}{-}{-}{-}+}
     |       |
     |      \_|\_
     |      { / Zener}
     |       |
     |       |
     +{-{-}{-}+{-}{-}{-}+}
         |
         |   +{-{-}{-}{-}{-}{-}{-}{-}{-}{-}{-}+}
         +{-{-}{-}| Base      |}
             |           |
  Vin    Rs  | Transistor|  RL   Vout
  +{-{-}{-}+{-}{-}www{-}+           +{-}{-}{-}www{-}{-}{-}+}
             | Collector |         |
             |           |         |
             +{-{-}{-}{-}{-}{-}{-}{-}{-}{-}{-}+         |}
                 |                 |
                 +{-{-}{-}{-}{-}{-}{-}{-}{-}{-}{-}{-}{-}{-}{-}{-}{-}+}
                 |
                GND
\end{verbatim}


{\def\LTcaptype{none} % do not increment counter
\vspace{-5pt}
\captionof{table}{ટ્રાન્ઝિસ્ટર શંટ રેગ્યુલેટર}
\vspace{-10pt}
\begin{longtable}[]{@{}ll@{}}
\toprule\noalign{}
કોમ્પોનન્ટ & કાર્ય \\
\midrule\noalign{}
\endhead
\bottomrule\noalign{}
\endlastfoot
ઝેનર & રેફરન્સ વોલ્ટેજ પ્રદાન કરે છે \\
ટ્રાન્ઝિસ્ટર & વધારાના કરંટને શંટ કરે છે \\
સીરીઝ રેસિસ્ટર (Rs) & વધારાનો વોલ્ટેજ ડ્રોપ કરે છે \\
લોડ રેસિસ્ટર (RL) & પાવર લેતા સર્કિટનું પ્રતિનિધિત્વ કરે છે \\
\end{longtable}
}

\begin{itemize}
\tightlist
\item
  \textbf{કાર્યપદ્ધતિ}: જ્યારે આઉટપુટ વધવાનો પ્રયાસ કરે ત્યારે ટ્રાન્ઝિસ્ટર વધુ કન્ડક્ટ
  કરે છે
\item
  \textbf{ફાયદો}: સારા રેગ્યુલેશન સાથે સરળ સર્કિટ
\end{itemize}

\end{solutionbox}
\begin{mnemonicbox}
``ZEST સર્કિટ'' (Zener reference, Excess current,
Shunt transistor, Tension-free output)

\end{mnemonicbox}
\subsection*{પ્રશ્ન 5(ક) [7
ગુણ]}\label{uxaaauxab0uxab6uxaa8-5uxa95-7-uxa97uxaa3}

\textbf{SMPS બ્લોક ડાયાગ્રામ દોરો અને તેના ફાયદા ગેરફાયદા સાથે સમજાવો.}

\begin{solutionbox}
સ્વિચ્ડ મોડ પાવર સપ્લાય (SMPS) ઉચ્ચ કાર્યક્ષમતા માટે સ્વિચિંગ
રેગ્યુલેશનનો ઉપયોગ કરે છે.

\textbf{આકૃતિ:}

\begin{center}
\textbf{Mermaid Diagram (Code)}
\begin{verbatim}
{Shaded}
{Highlighting}[]
graph LR
    A[AC Input] {-{-}{} B[EMI Filter]}
    B {-{-}{} C[Rectifier \& Filter]}
    C {-{-}{} D[Switching Circuit]}
    D {-{-}{} E[Transformer]}
    E {-{-}{} F[Output Rectifier]}
    F {-{-}{} G[Output Filter]}
    G {-{-}{} H[DC Output]}
    I[Feedback \& Control] {-{-}{} D}
    H {-{-}{} I}

    style A fill:\#f9f,stroke:\#333,stroke{-width:1px}
    style D fill:\#bbf,stroke:\#333,stroke{-width:1px}
    style E fill:\#bfb,stroke:\#333,stroke{-width:1px}
    style H fill:\#f9f,stroke:\#333,stroke{-width:1px}
    style I fill:\#bbf,stroke:\#333,stroke{-width:1px}
{Highlighting}
{Shaded}
\end{verbatim}
\end{center}


{\def\LTcaptype{none} % do not increment counter
\vspace{-5pt}
\captionof{table}{SMPS ના ફાયદા અને ગેરફાયદા}
\vspace{-10pt}
\begin{longtable}[]{@{}ll@{}}
\toprule\noalign{}
ફાયદા & ગેરફાયદા \\
\midrule\noalign{}
\endhead
\bottomrule\noalign{}
\endlastfoot
ઉચ્ચ કાર્યક્ષમતા (80-95\%) & જટિલ સર્કિટ ડિઝાઇન \\
નાનું કદ અને હળવા વજન & ઉચ્ચ-આવૃત્તિ નોઇઝ ઉત્પન્ન કરે છે \\
વિશાળ ઇનપુટ વોલ્ટેજ રેન્જ & EMI/RFI ઇન્ટરફેરન્સ \\
સારું રેગ્યુલેશન & ઓછા પાવર માટે ઊંચી કિંમત \\
ઓછી ગરમી ઉત્પાદન & મુશ્કેલ ટ્રબલશૂટિંગ \\
\end{longtable}
}

\begin{itemize}
\tightlist
\item
  \textbf{કાર્ય સિદ્ધાંત}: ઉચ્ચ આવૃત્તિ પર પાવરને ઝડપથી ચાલુ/બંધ કરે છે
\item
  \textbf{કદ ઘટાડો}: ઊંચી સ્વિચિંગ આવૃત્તિ નાના ટ્રાન્સફોર્મરની મંજૂરી આપે છે
\item
  \textbf{ઉપયોગો}: કોમ્પ્યુટર, ટીવી, મોબાઇલ ચાર્જર, LED ડ્રાઇવર
\end{itemize}

\end{solutionbox}
\begin{mnemonicbox}
``SWEEP ફાયદા'' (Small size, Widerange input,
Efficient, Economical, Precise regulation)

\end{mnemonicbox}
\subsection*{પ્રશ્ન 5(અ) અથવા [3
ગુણ]}\label{uxaaauxab0uxab6uxaa8-5uxa85-uxa85uxaa5uxab5-3-uxa97uxaa3}

\textbf{ત્રણ ટર્મિનલ IC 7812 નો ઉપયોગ કરીને વોલ્ટેજ રેગ્યુલેટર દોરો.}

\begin{solutionbox}
ત્રણ ટર્મિનલ IC 7812 ફિક્સ્ડ +12V રેગ્યુલેટેડ આઉટપુટ વોલ્ટેજ પ્રદાન કરે
છે.

\textbf{આકૃતિ:}

\begin{verbatim}
        +{-{-}{-}{-}{-}{-}{-}+{-}{-}{-}{-}{-}{-}{-}+}
        |       |       |
   Vin  |       |       |  Vout
   +{-{-}{-}{-}|  IN   OUT  |{-}{-}{-}{-}+ (+12V)}
        |       |       |
   +{-{-}{-}{-}|  GND       |{-}{-}{-}{-}+}
   |    |       |       |
   |    +{-{-}{-}{-}{-}{-}{-}+{-}{-}{-}{-}{-}{-}{-}+}
   |        |
   |       \_|\_
   |       {-{-}{-} C1}
   |        |
   +{-{-}{-}{-}{-}{-}{-}{-}+}
           GND
\end{verbatim}

\begin{itemize}
\tightlist
\item
  \textbf{કોમ્પોનન્ટ્સ}: 7812 રેગ્યુલેટર IC અને ફિલ્ટર કેપેસિટર
\item
  \textbf{પિન કોન્ફિગરેશન}: ઇનપુટ, ગ્રાઉન્ડ, આઉટપુટ
\item
  \textbf{વિશેષતાઓ}: આંતરિક કરંટ લિમિટિંગ અને થર્મલ શટડાઉન
\end{itemize}

\end{solutionbox}
\begin{mnemonicbox}
``IGO પિન્સ'' (Input, Ground, Output)

\end{mnemonicbox}
\subsection*{પ્રશ્ન 5(બ) અથવા [4
ગુણ]}\label{uxaaauxab0uxab6uxaa8-5uxaac-uxa85uxaa5uxab5-4-uxa97uxaa3}

\textbf{ટ્રાન્ઝિસ્ટર સીરીઝ વોલ્ટેજ રેગ્યુલેટરનું વર્ણન કરો}

\begin{solutionbox}
ટ્રાન્ઝિસ્ટર સીરીઝ રેગ્યુલેટર સીરીઝ ટ્રાન્ઝિસ્ટરની કન્ડક્ટિવિટી બદલીને
આઉટપુટ વોલ્ટેજને નિયંત્રિત કરે છે.

\textbf{આકૃતિ:}

\begin{verbatim}
   +{-{-}{-}{-}{-}{-}{-}{-}{-}{-}+}
   |          |
   |         \_|\_
   |         { / Zener}
   |          |
   |          |
   +{-{-}{-}{-}{-}+{-}{-}{-}{-}+}
         |
         |   +{-{-}{-}{-}{-}{-}{-}{-}{-}{-}{-}+}
         +{-{-}{-}| Base      |}
             |           |
  Vin        | Transistor|     Vout
  +{-{-}{-}{-}{-}{-}{-}{-}{-}{-}| Collector |{-}{-}{-}{-}{-}{-}+}
             |           |      |
             | Emitter   |      |
             +{-{-}{-}{-}{-}{-}{-}{-}{-}{-}{-}+      |}
                 |              |
                 |             \_|\_
                 |             {-{-}{-} C}
                 |              |
                 +{-{-}{-}{-}{-}{-}{-}{-}{-}{-}{-}{-}{-}{-}+}
                             GND
\end{verbatim}


{\def\LTcaptype{none} % do not increment counter
\vspace{-5pt}
\captionof{table}{સીરીઝ વોલ્ટેજ રેગ્યુલેટરની વિશેષતાઓ}
\vspace{-10pt}
\begin{longtable}[]{@{}ll@{}}
\toprule\noalign{}
વિશેષતા & વર્ણન \\
\midrule\noalign{}
\endhead
\bottomrule\noalign{}
\endlastfoot
કંટ્રોલ એલિમેન્ટ & ટ્રાન્ઝિસ્ટર સીરીઝમાં વેરિએબલ રેસિસ્ટર તરીકે કાર્ય કરે છે \\
રેફરન્સ & ઝેનર ડાયોડ સ્થિર રેફરન્સ વોલ્ટેજ પ્રદાન કરે છે \\
રેગ્યુલેશન & ફીડબેક ટ્રાન્ઝિસ્ટર કન્ડક્ટિવિટી એડજસ્ટ કરે છે \\
કાર્યક્ષમતા & ઉચ્ચ કરંટ લોડ માટે શંટ રેગ્યુલેટર કરતાં વધુ સારી \\
\end{longtable}
}

\begin{itemize}
\tightlist
\item
  \textbf{કાર્ય સિદ્ધાંત}: સ્થિર આઉટપુટ જાળવવા માટે ટ્રાન્ઝિસ્ટર કન્ડક્ટિવિટી બદલાય
  છે
\item
  \textbf{ફાયદો}: ઉચ્ચ કરંટ માટે શંટ રેગ્યુલેટર કરતાં વધુ કાર્યક્ષમ
\end{itemize}

\end{solutionbox}
\begin{mnemonicbox}
``CERT સર્કિટ'' (Control transistor, Efficient
design, Reference voltage, Transistor in series)

\end{mnemonicbox}
\subsection*{પ્રશ્ન 5(ક) અથવા [7
ગુણ]}\label{uxaaauxab0uxab6uxaa8-5uxa95-uxa85uxaa5uxab5-7-uxa97uxaa3}

\textbf{UPS બ્લોક ડાયાગ્રામ દોરો અને તેના ફાયદા ગેરફાયદા સાથે સમજાવો.}

\begin{solutionbox}
અનઇન્ટરપ્ટિબલ પાવર સપ્લાય (UPS) મુખ્ય પાવર સપ્લાય ફેઇલ થાય ત્યારે
ઇમરજન્સી પાવર પ્રદાન કરે છે.

\textbf{આકૃતિ:}

\begin{center}
\textbf{Mermaid Diagram (Code)}
\begin{verbatim}
{Shaded}
{Highlighting}[]
graph LR
    A[AC Input] {-{-}{} B[Surge Protector]}
    B {-{-}{} C[Rectifier/Charger]}
    C {-{-}{} D[Battery]}
    C {-{-}{} E[Inverter]}
    D {-{-}{} E}
    E {-{-}{} F[Output Filter]}
    F {-{-}{} G[AC Output]}
    H[Control Circuit] {-{-}{} C}
    H {-{-}{} E}
    H {-{-}{} D}

    style A fill:\#f9f,stroke:\#333,stroke{-width:1px}
    style C fill:\#bbf,stroke:\#333,stroke{-width:1px}
    style D fill:\#bfb,stroke:\#333,stroke{-width:1px}
    style E fill:\#f9f,stroke:\#333,stroke{-width:1px}
    style H fill:\#bbf,stroke:\#333,stroke{-width:1px}
{Highlighting}
{Shaded}
\end{verbatim}
\end{center}


{\def\LTcaptype{none} % do not increment counter
\vspace{-5pt}
\captionof{table}{UPS ના ફાયદા અને ગેરફાયદા}
\vspace{-10pt}
\begin{longtable}[]{@{}ll@{}}
\toprule\noalign{}
ફાયદા & ગેરફાયદા \\
\midrule\noalign{}
\endhead
\bottomrule\noalign{}
\endlastfoot
બેકઅપ પાવર પ્રદાન કરે છે & મર્યાદિત બેકઅપ સમય \\
વોલ્ટેજ ફ્લક્ચુએશનથી બચાવે છે & નિયમિત બેટરી મેઇન્ટેનન્સ \\
સર્જ પ્રોટેક્શન & પ્રારંભિક ઊંચી કિંમત \\
સરળ પાવર ટ્રાન્ઝિશન & ઓપરેશન દરમિયાન ઘોંઘાટ \\
પાવર કન્ડિશનિંગ & સ્ટેન્ડબાયમાં ઓછી કાર્યક્ષમતા \\
\end{longtable}
}

\begin{itemize}
\tightlist
\item
  \textbf{પ્રકારો}: ઓફલાઇન/સ્ટેન્ડબાય, લાઇન-ઇન્ટરેક્ટિવ, ઓનલાઇન/ડબલ-કન્વર્ઝન
\item
  \textbf{ઉપયોગો}: કોમ્પ્યુટર, મેડિકલ ઇક્વિપમેન્ટ, ડેટા સેન્ટર, ટેલિકોમ્યુનિકેશન્સ
\item
  \textbf{કાર્યપદ્ધતિ}: સામાન્ય રીતે બેટરી ચાર્જ કરતી વખતે મુખ્ય પાવર પસાર કરે છે;
  પાવર જતા રહે ત્યારે બેટરી પાવર પર સ્વિચ કરે છે
\end{itemize}

\end{solutionbox}
\begin{mnemonicbox}
``POWER બેકઅપ'' (Protection from Outages, Waveform
conditioning, Emission-free, Reliability boost)

\end{mnemonicbox}

\end{document}
