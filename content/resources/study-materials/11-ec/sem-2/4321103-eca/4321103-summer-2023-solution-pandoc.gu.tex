\documentclass[10pt,a4paper]{article}

% content/resources/templates/preamble.tex
\usepackage[margin=0.6in]{geometry}
\author{Milav Dabgar}
\usepackage{amsmath,amssymb,amsthm}
\usepackage{booktabs}
\usepackage{multirow}
\usepackage{xcolor}
\usepackage{tcolorbox}
\tcbuselibrary{breakable,skins}
\usepackage[colorlinks=true,linkcolor=blue]{hyperref}
\usepackage{titlesec}
\usepackage{enumitem}
\usepackage{tikz}
\usepackage{pgfplots}
\usepackage{circuitikz}
\usepackage[version=4]{mhchem}
\usepackage{longtable}
\usepackage{array}
\usepackage{float}
\usepackage{caption}
\usepackage{listings}

\lstset{
  basicstyle=\small\ttfamily,
  breaklines=true,
  breakatwhitespace=false,
  postbreak=\mbox{\textcolor{red}{$\hookrightarrow$}\space},
  float=false,
  numbers=left,
  numberstyle=\tiny\color{gray},
  numbersep=10pt,
  xleftmargin=2em,
  keywordstyle=\color{blue},
  commentstyle=\color{green!60!black},
  stringstyle=\color{purple},
  backgroundcolor=\color{gray!5},
  showstringspaces=false,
  tabsize=2,
  captionpos=b,
  keepspaces=true,
  columns=flexible
}

\pgfplotsset{compat=1.18}
\usetikzlibrary{shapes,arrows,positioning,calc,patterns,decorations.pathmorphing,decorations.markings,arrows.meta}

% Color scheme
\definecolor{headcolor}{RGB}{0,102,204}
\definecolor{keycolor}{RGB}{220,20,60}
\definecolor{solutioncolor}{RGB}{34,139,34}
\definecolor{mnemoniccolor}{RGB}{148,0,211}
\definecolor{codecolor}{RGB}{0,0,100}

% Spacing
\setlength{\parskip}{3pt}
\setlist[itemize]{nosep}
\setlist[enumerate]{nosep}

% Title formatting
\titleformat{\section}{\Large\bfseries\color{headcolor}}{\thesection}{1em}{}
\titleformat{\subsection}{\large\bfseries\color{headcolor}}{\thesubsection}{1em}{}

% Pandoc tightlist compatibility
\providecommand{\tightlist}{%
  \setlength{\itemsep}{0pt}\setlength{\parskip}{0pt}}

% Pandoc longtable compatibility
\newcounter{none}
\def\thenone{}


% content/resources/templates/gujarati-boxes.tex
\usepackage{fontspec}
\usepackage{polyglossia}

% Set Gujarati as main language (document is primarily in Gujarati)
% Note: gloss-gujarati.ldf doesn't exist in polyglossia, but it will use hyphenation patterns
\setdefaultlanguage{gujarati}
\setotherlanguage{english}

% Configure Gujarati font properly
% Use Language=Default to prevent polyglossia from trying to add language-specific features
% that don't exist for Gujarati, which causes "empty feature" warnings
\newfontfamily\gujaratifont[Script=Gujarati,AutoFakeBold=2.5,AutoFakeSlant=0.3]{Noto Sans Gujarati}
\setmainfont[Script=Gujarati,AutoFakeBold=2.5,AutoFakeSlant=0.3]{Noto Sans Gujarati}
% Use Noto Sans Gujarati for monospace to support Gujarati in text
\setmonofont[Scale=0.9]{Noto Sans Gujarati}

% Configure English to use the same font
\newfontfamily\englishfont[Script=Gujarati,AutoFakeBold=2.5,AutoFakeSlant=0.3]{Noto Sans Gujarati}

% Translations for polyglossia
\gappto\captionsgujarati{
  \renewcommand{\tablename}{કોષ્ટક}
  \renewcommand{\figurename}{આકૃતિ}
}

% Helper for TikZ nodes to ensure Gujarati font
\newcommand{\gu}[1]{{\gujaratifont #1}}

% Custom environments
\newtcolorbox{solutionbox}{
    breakable,
    enhanced,
    colback=solutioncolor!5!white,
    colframe=solutioncolor!75!black,
    fonttitle=\bfseries,
    title=જવાબ
}

\newtcolorbox{solutionboxnobreak}{
 colback=solutioncolor!5!white,
 colframe=solutioncolor!75!black,
 fonttitle=\bfseries,
 title=જવાબ
}

\newtcolorbox{keyformula}{
 breakable,
 enhanced,
 colback=keycolor!5!white,
 colframe=keycolor!75!black,
 fonttitle=\bfseries,
 title=રાસાયણિક સમીકરણ/સૂત્ર
}

\newtcolorbox{mnemonicbox}{
 breakable,
 enhanced,
 colback=mnemoniccolor!5!white,
 colframe=mnemoniccolor!75!black,
 fonttitle=\bfseries,
 title=મેમરી ટ્રીક
}


\begin{document}

\begin{center}
{\Huge\bfseries\color{headcolor} Subject Name (Gujarati)}\\[5pt]
{\LARGE 4321103 -- Summer 2023}\\[3pt]
{\large Semester 1 Study Material}\\[3pt]
{\normalsize\textit{Detailed Solutions and Explanations}}
\end{center}

\vspace{10pt}

\subsection*{પ્રશ્ન 1(અ) [3
ગુણ]}\label{uxaaauxab0uxab6uxaa8-1uxa85-3-uxa97uxaa3}

\textbf{થર્મલ રનઅવે વિગતવાર સમજાવો.}

\begin{solutionbox}
થર્મલ રનઅવે એ BJT ટ્રાન્ઝિસ્ટરમાં થતી વિનાશક પ્રક્રિયા છે જેમાં
તાપમાનમાં વધારો સ્વ-પુનરાવર્તિત ચક્ર બનાવે છે જે ઉપકરણને નુકસાન પહોંચાડે છે.

\begin{verbatim}
flowchart LR
    A[તાપમાનમાં વધારો] {-{-} B[Ic માં વધારો]}
    B {-{-} C[પાવર વ્યય વધારો]}
    C {-{-} D[વધુ તાપમાન વધારો]}
    D {-{-} A}
\end{verbatim}

\begin{itemize}
\tightlist
\item
  \textbf{ગરમી ઉત્પાદન}: સામાન્ય કાર્ય દરમિયાન તાપમાન વધે છે
\item
  \textbf{લીકેજ કરંટ}: તાપમાન વધવાથી કલેક્ટર કરંટ Ic વધે છે
\item
  \textbf{પાવર વ્યય}: વધુ પાવર = તાપમાન વધુ વધે છે
\item
  \textbf{વિનાશક ચક્ર}: ટ્રાન્ઝિસ્ટર નાશ પામે ત્યાં સુધી સતત ચક્ર ચાલે છે
\end{itemize}

\end{solutionbox}
\begin{mnemonicbox}
``વધુ તાપમાન, વધુ કરંટ''

\end{mnemonicbox}
\subsection*{પ્રશ્ન 1(બ) [4
ગુણ]}\label{uxaaauxab0uxab6uxaa8-1uxaac-4-uxa97uxaa3}

\textbf{સરળ બ્લોક ડાયાગ્રામ સાથે એમ્પ્લીફાયર વ્યાખ્યાયિત કરો એમ્પ્લીફાયર પરિમાણો
લખો.}

\begin{solutionbox}
એમ્પ્લીફાયર એક ઇલેક્ટ્રોનિક ઉપકરણ છે જે ઇનપુટ સિગ્નલનો પાવર, વોલ્ટેજ
અથવા કરંટ વધારે છે.

\begin{verbatim}
flowchart LR
    A[ઇનપુટ સિગ્નલ] {-{-}|Vin| B[એમ્પ્લીફાયર]}
    B {-{-}|Vout| C[આઉટપુટ સિગ્નલ]}
    D[પાવર સપ્લાય] {-{-} B}
\end{verbatim}

{\def\LTcaptype{none} % do not increment counter
\begin{longtable}[]{@{}ll@{}}
\toprule\noalign{}
એમ્પ્લીફાયર પરિમાણ & વર્ણન \\
\midrule\noalign{}
\endhead
\bottomrule\noalign{}
\endlastfoot
\textbf{વોલ્ટેજ ગેઇન (Av)} & આઉટપુટ વોલ્ટેજનો ઇનપુટ વોલ્ટેજ સાથેનો ગુણોત્તર \\
\textbf{કરંટ ગેઇન (Ai)} & આઉટપુટ કરંટનો ઇનપુટ કરંટ સાથેનો ગુણોત્તર \\
\textbf{પાવર ગેઇન (Ap)} & વોલ્ટેજ ગેઇન અને કરંટ ગેઇનનો ગુણાકાર \\
\textbf{બેન્ડવિડ્થ} & એમ્પ્લીફાયર હેન્ડલ કરી શકે તેવી ફ્રીક્વન્સીની રેન્જ \\
\textbf{ઇનપુટ ઇમ્પીડન્સ} & ઇનપુટ સ્ત્રોત દ્વારા જોવામાં આવતો અવરોધ \\
\textbf{આઉટપુટ ઇમ્પીડન્સ} & એમ્પ્લીફાયરનો આંતરિક અવરોધ \\
\end{longtable}
}

\end{solutionbox}
\begin{mnemonicbox}
``VIPS-BIO'' (Voltage, Input impedance, Power,
Supply, Bandwidth, Impedance Output)

\end{mnemonicbox}
\subsection*{પ્રશ્ન 1(ક) [7
ગુણ]}\label{uxaaauxab0uxab6uxaa8-1uxa95-7-uxa97uxaa3}

\textbf{ટ્રાન્ઝિસ્ટરમાં બાયસિંગ વ્યાખ્યાયિત કરો? બાયસિંગ પદ્ધતિઓના પ્રકારો લખો.
વોલ્ટેજ વિભાજક બાયસિંગ પદ્ધતિને વિગતોમાં સમજાવો.}

\begin{solutionbox}
બાયસિંગ એ ટ્રાન્ઝિસ્ટર માટે DC વોલ્ટેજ આપીને સ્થિર ઓપરેટિંગ પોઈન્ટ
(Q-પોઈન્ટ) સ્થાપિત કરવાની પ્રક્રિયા છે.

{\def\LTcaptype{none} % do not increment counter
\begin{longtable}[]{@{}ll@{}}
\toprule\noalign{}
બાયસિંગ પદ્ધતિ & મુખ્ય લક્ષણો \\
\midrule\noalign{}
\endhead
\bottomrule\noalign{}
\endlastfoot
\textbf{ફિક્સ્ડ બાયસ} & સરળ, ઓછી સ્થિરતા \\
\textbf{કલેક્ટર ફીડબેક} & સ્વ-સમાયોજિત, વધુ સારી સ્થિરતા \\
\textbf{વોલ્ટેજ વિભાજક} & શ્રેષ્ઠ સ્થિરતા, વ્યાપકપણે વપરાતી \\
\textbf{એમિટર બાયસ} & સારી સ્થિરતા, નેગેટિવ ફીડબેક \\
\end{longtable}
}

\textbf{વોલ્ટેજ વિભાજક બાયસિંગ:}

\begin{verbatim}
flowchart LR
        VCC((+VCC)) {-{-}{-} R1[પ્રતિકાર R1]}
        R1 {-{-}{-} R2[પ્રતિકાર R2]}
        R2 {-{-}{-} GND((GND))}
        R2 {-{-}{-} B[બેઝ]}
        B {-{-}{-} C[કલેક્ટર]}
        B {-{-}{-} E[એમિટર]}
        C {-{-}{-} RC[પ્રતિકાર RC] {-}{-}{-} VCC}
        E {-{-}{-} RE[પ્રતિકાર RE] {-}{-}{-} GND}
\end{verbatim}

\begin{itemize}
\tightlist
\item
  \textbf{R1 \& R2}: બેઝને સ્થિર વોલ્ટેજ આપવા માટે વોલ્ટેજ વિભાજક બનાવે છે
\item
  \textbf{RE}: નેગેટિવ ફીડબેક દ્વારા સ્થિરીકરણ પ્રદાન કરે છે
\item
  \textbf{RC}: કલેક્ટર કરંટ અને વોલ્ટેજ ગેઇન નક્કી કરે છે
\item
  \textbf{સ્થિરતા}: તાપમાન ફેરફારો સામે શ્રેષ્ઠ સ્થિરતા
\end{itemize}

\end{solutionbox}
\begin{mnemonicbox}
``વિભાજીત વોલ્ટેજથી ટ્રાન્ઝિસ્ટર સારું વહન કરે''

\end{mnemonicbox}
\subsection*{પ્રશ્ન 1(ક) અથવા [7
ગુણ]}\label{uxaaauxab0uxab6uxaa8-1uxa95-uxa85uxaa5uxab5-7-uxa97uxaa3}

\textbf{હીટ સિંક સમજાવો.}

\begin{solutionbox}
હીટ સિંક એ પેસિવ હીટ એક્સચેન્જર છે જે ઇલેક્ટ્રોનિક ઉપકરણોમાંથી ગરમીને
આસપાસની હવામાં ટ્રાન્સફર કરે છે.

\begin{verbatim}
flowchart LR
    A[ગરમીનો સ્ત્રોત/ટ્રાન્ઝિસ્ટર] {-{-} B[ઇન્ટરફેસ મટિરિયલ]}
    B {-{-} C[હીટ સિંક બેઝ]}
    C {-{-} D[હીટ સિંક ફિન્સ]}
    D {-{-} E[એમ્બિયન્ટ એર]}
\end{verbatim}

{\def\LTcaptype{none} % do not increment counter
\begin{longtable}[]{@{}ll@{}}
\toprule\noalign{}
ભાગ & કાર્ય \\
\midrule\noalign{}
\endhead
\bottomrule\noalign{}
\endlastfoot
\textbf{બેઝ} & ડિવાઇસમાંથી ગરમી વહન કરે છે \\
\textbf{ફિન્સ} & ગરમી ફેલાવા માટે સરફેસ એરિયા વધારે છે \\
\textbf{થર્મલ ઇન્ટરફેસ મટિરિયલ} & ડિવાઇસ અને સિંક વચ્ચેનો સંપર્ક સુધારે છે \\
\textbf{પ્રકારો} & એક્સટ્રૂડેડ, બોન્ડેડ, ફોલ્ડેડ, ડાઇ-કાસ્ટ \\
\end{longtable}
}

\begin{itemize}
\tightlist
\item
  \textbf{થર્મલ રેઝિસ્ટન્સ}: ઓછું તે ગરમી ફેલાવા માટે વધુ સારું
\item
  \textbf{મટિરિયલ}: સામાન્ય રીતે એલ્યુમિનિયમ અથવા કોપર સારી કન્ડક્ટિવિટી માટે
\item
  \textbf{સરફેસ એરિયા}: વધુ ફિન્સ એટલે વધુ સારું કૂલિંગ
\item
  \textbf{એરફ્લો}: કુશળ ગરમી દૂર કરવા માટે મહત્વપૂર્ણ
\end{itemize}

\end{solutionbox}
\begin{mnemonicbox}
``હીટ સિંક ટ્રાન્ઝિસ્ટરને ઠંડુ રાખે''

\end{mnemonicbox}
\subsection*{પ્રશ્ન 2(અ) [3
ગુણ]}\label{uxaaauxab0uxab6uxaa8-2uxa85-3-uxa97uxaa3}

\textbf{D.C અને A.C. લોડ લાઇનોનું વર્ણન કરો.}

\begin{solutionbox}
લોડ લાઇન્સ ટ્રાન્ઝિસ્ટરનાં સંભવિત ઓપરેટિંગ પોઈન્ટ્સને તેના કેરેક્ટરિસ્ટિક
કર્વ પર ગ્રાફિકલી દર્શાવે છે.

\begin{verbatim}
                     Ic
                      ↑
                      |
                      |        DC લોડ લાઇન
                      |       ╱
         Q{-પોઈન્ટ       |      ╱}
                      |     *
                      |    ╱  AC લોડ લાઇન
                      |   ╱
                      |  ╱
                      | ╱
                      |╱
                      +{-{-}{-}{-}{-}{-}{-}{-}{-}{-}{-}{-}{-}{-}{-}{-} Vce}
                     0                Vcc
\end{verbatim}

\begin{itemize}
\tightlist
\item
  \textbf{DC લોડ લાઇન}: DC સ્થિતિઓ હેઠળ બધા શક્ય ઓપરેટિંગ પોઈન્ટ્સ બતાવે છે

  \begin{itemize}
  \tightlist
  \item
    \textbf{સમીકરણ}: Ic = (VCC - VCE)/RC
  \item
    \textbf{એન્ડપોઈન્ટ્સ}: (0, VCC/RC) અને (VCC, 0)
  \end{itemize}
\item
  \textbf{AC લોડ લાઇન}: AC સિગ્નલ હેન્ડલિંગ દરમિયાન ઓપરેટિંગ પોઈન્ટ્સ બતાવે છે

  \begin{itemize}
  \tightlist
  \item
    \textbf{વધુ તીક્ષ્ણ ઢાળ}: AC રેઝિસ્ટન્સ DC કરતાં ઓછો હોવાના કારણે
  \item
    \textbf{Q-પોઈન્ટ પર કેન્દ્રિત}: બાયસિંગ દ્વારા સ્થાપિત ઓપરેટિંગ પોઈન્ટ
  \end{itemize}
\end{itemize}

\end{solutionbox}
\begin{mnemonicbox}
``DC પૂર્ણ આલેખે, AC માર્ગ બદલે''

\end{mnemonicbox}
\subsection*{પ્રશ્ન 2(બ) [4
ગુણ]}\label{uxaaauxab0uxab6uxaa8-2uxaac-4-uxa97uxaa3}

\textbf{એમ્પ્લીફાયરની બેન્ડવિડ્થ અને ગેઇન-બેન્ડવિડ્થ ઉત્પાદનને સંક્ષિપ્તમાં સમજાવો.}

\begin{solutionbox}
બેન્ડવિડ્થ અને ગેઇન-બેન્ડવિડ્થ ઉત્પાદન એમ્પ્લીફાયર ફ્રીક્વન્સી પરફોર્મન્સ
માટેની મુખ્ય વિશેષતાઓ છે.

\begin{verbatim}
flowchart LR
    A[ઇનપુટ] {-{-} B[એમ્પ્લીફાયરbr /ગેઇન  બેન્ડવિડ્થ]}
    B {-{-} C[આઉટપુટ]}
\end{verbatim}

{\def\LTcaptype{none} % do not increment counter
\begin{longtable}[]{@{}ll@{}}
\toprule\noalign{}
પેરામીટર & વર્ણન \\
\midrule\noalign{}
\endhead
\bottomrule\noalign{}
\endlastfoot
\textbf{બેન્ડવિડ્થ} & ફ્રીક્વન્સી રેન્જ જ્યાં ગેઇન 3dB કરતાં ઓછો ઘટે છે \\
\textbf{લોઅર કટઓફ (f_{1})} & ફ્રીક્વન્સી જ્યાં નીચલા છેડે ગેઇન 3dB ઘટે છે \\
\textbf{અપર કટઓફ (f_{2})} & ફ્રીક્વન્સી જ્યાં ઉપલા છેડે ગેઇન 3dB ઘટે છે \\
\textbf{ગેઇન-બેન્ડવિડ્થ ઉત્પાદન} & ગેઇન અને બેન્ડવિડ્થનો ગુણાકાર, સ્થિર રહે છે \\
\end{longtable}
}

\begin{itemize}
\tightlist
\item
  \textbf{બેન્ડવિડ્થ ફોર્મ્યુલા}: BW = f_{2} - f_{1}
\item
  \textbf{ગેઇન-બેન્ડવિડ્થ}: ગેઇન બદલાય ત્યારે પણ સ્થિર રહે છે
\item
  \textbf{ટ્રેડ-ઓફ}: વધુ ગેઇન એટલે ઓછી બેન્ડવિડ્થ
\end{itemize}

\end{solutionbox}
\begin{mnemonicbox}
``સારી બેન્ડવિડ્થ શ્રેષ્ઠ ટ્રાન્સમિશન આપે''

\end{mnemonicbox}
\subsection*{પ્રશ્ન 2(ક) [7
ગુણ]}\label{uxaaauxab0uxab6uxaa8-2uxa95-7-uxa97uxaa3}

\textbf{બે તબક્કાના RC કપલ્ડ એમ્પ્લીફાયરનો આવર્તન પ્રતિભાવ સમજાવો.}

\begin{solutionbox}
બે-તબક્કાના RC કપલ્ડ એમ્પ્લીફાયરનો આવર્તન પ્રતિભાવ બતાવે છે કે ગેઇન
આવર્તન સાથે કેવી રીતે બદલાય છે.

\begin{verbatim}
flowchart LR
    A[ઇનપુટ] {-{-} B[પ્રથમbr /એમ્પ્લીફાયરbr /તબક્કો]}
    B {-{-}|RC કપલિંગ| C[બીજોbr /એમ્પ્લીફાયરbr /તબક્કો]}
    C {-{-} D[આઉટપુટ]}
\end{verbatim}

\begin{verbatim}
    ગેઇન(dB)
       ↑
       |    મધ્ય{-આવર્તન બેન્ડ}
       |    ┌───────────────┐
       |    │               │
       |    │               │
       |   ╱│               │╲
       |  ╱ │               │ ╲
       | ╱  │               │  ╲
       |╱   │               │   ╲
       +────┴───────────────┴──── આવર્તન(Hz)
          f_{1                 f_{2}}
     નીચલા આવર્તન       ઉચ્ચ આવર્તન
\end{verbatim}

\begin{itemize}
\tightlist
\item
  \textbf{નીચલા આવર્તન પ્રતિભાવ}: કપલિંગ કેપેસિટર્સ દ્વારા મર્યાદિત

  \begin{itemize}
  \tightlist
  \item
    \textbf{રોલ-ઓફ રેટ}: દરેક તબક્કા માટે -20 dB/decade
  \end{itemize}
\item
  \textbf{મધ્ય આવર્તન પ્રતિભાવ}: મહત્તમ અને સપાટ ગેઇન પ્રદેશ

  \begin{itemize}
  \tightlist
  \item
    \textbf{કુલ ગેઇન}: વ્યક્તિગત તબક્કાના ગેઇનનો ગુણાકાર
  \end{itemize}
\item
  \textbf{ઉચ્ચ આવર્તન પ્રતિભાવ}: ટ્રાન્ઝિસ્ટર કેપેસિટન્સ દ્વારા મર્યાદિત

  \begin{itemize}
  \tightlist
  \item
    \textbf{રોલ-ઓફ રેટ}: દરેક તબક્કા માટે -20 dB/decade
  \end{itemize}
\end{itemize}

\end{solutionbox}
\begin{mnemonicbox}
``નીચે કપલિંગ નબળું, ઉપર કેપેસિટન્સ રોકે''

\end{mnemonicbox}
\subsection*{પ્રશ્ન 2(અ) અથવા [3
ગુણ]}\label{uxaaauxab0uxab6uxaa8-2uxa85-uxa85uxaa5uxab5-3-uxa97uxaa3}

\textbf{ટ્રાન્ઝિસ્ટર બાયસિંગ માટે નિશ્ચિત બાયસ સર્કિટ સમજાવો.}

\begin{solutionbox}
ફિક્સ્ડ બાયસ એ ટ્રાન્ઝિસ્ટર માટેની સૌથી સરળ બાયસિંગ પદ્ધતિ છે, જેમાં
બેઝ સાથે જોડાયેલ એક રેઝિસ્ટરનો ઉપયોગ થાય છે.

\begin{verbatim}
        +Vcc
          |
          R
          |
          |   C
    {-{-}{-}{-}{-}{-}+{-}{-}{-}o}
    |     |
   Vin    |    RC
    |     |     |
    |     |    +Vcc
    |     |
    +{-{-}{-}{-}{-}+}
      બેઝ   કલેક્ટર
          |
          E
          |
         GND
\end{verbatim}

\begin{itemize}
\tightlist
\item
  \textbf{સર્કિટ તત્વો}: બેઝ રેઝિસ્ટર (RB) અને કલેક્ટર રેઝિસ્ટર (RC)
\item
  \textbf{બેઝ કરંટ}: IB = (VCC - VBE)/RB
\item
  \textbf{કલેક્ટર કરંટ}: IC = β \times IB
\item
  \textbf{નુકસાન}: ઓછી સ્થિરતા, તાપમાન ફેરફારોથી અસર પામે છે
\end{itemize}

\end{solutionbox}
\begin{mnemonicbox}
``ફિક્સ બાયસ, ફેસ બર્ડન'' (અસ્થિરતાનો)

\end{mnemonicbox}
\subsection*{પ્રશ્ન 2(બ) અથવા [4
ગુણ]}\label{uxaaauxab0uxab6uxaa8-2uxaac-uxa85uxaa5uxab5-4-uxa97uxaa3}

\textbf{સિંગલ સ્ટેજ એમ્પ્લીફાયરનો આવર્તન પ્રતિભાવ સમજાવો.}

\begin{solutionbox}
સિંગલ-સ્ટેજ એમ્પ્લીફાયરનો આવર્તન પ્રતિભાવ બતાવે છે કે ગેઇન વિભિન્ન
આવર્તનો પર કેવી રીતે બદલાય છે.

\begin{verbatim}
    ગેઇન(dB)
       ↑
       |    મધ્ય{-આવર્તન બેન્ડ}
       |    ┌───────────────┐
       |    │               │
       |   ╱│               │╲
       |  ╱ │               │ ╲
       | ╱  │               │  ╲
       |╱   │               │   ╲
       +────┴───────────────┴──── આવર્તન(Hz)
          f_{1                 f_{2}}
     નીચલા આવર્તન       ઉચ્ચ આવર્તન
\end{verbatim}

{\def\LTcaptype{none} % do not increment counter
\begin{longtable}[]{@{}ll@{}}
\toprule\noalign{}
આવર્તન રેન્જ & લક્ષણો \\
\midrule\noalign{}
\endhead
\bottomrule\noalign{}
\endlastfoot
\textbf{નીચલા આવર્તન પ્રદેશ} & કપલિંગ કેપેસિટર્સને કારણે ગેઇન ઘટે છે \\
\textbf{મધ્ય આવર્તન પ્રદેશ} & મહત્તમ અને સ્થિર ગેઇન \\
\textbf{ઉચ્ચ આવર્તન પ્રદેશ} & ટ્રાન્ઝિસ્ટર કેપેસિટન્સને કારણે ગેઇન ઘટે છે \\
\end{longtable}
}

\begin{itemize}
\tightlist
\item
  \textbf{નીચલી કટ-ઓફ આવર્તન}: કપલિંગ કેપેસિટર્સ દ્વારા નિર્ધારિત
\item
  \textbf{ઉપલી કટ-ઓફ આવર્તન}: આંતરિક ટ્રાન્ઝિસ્ટર કેપેસિટન્સથી મર્યાદિત
\item
  \textbf{બેન્ડવિડ્થ}: નીચલી અને ઉપલી કટ-ઓફ આવર્તનો વચ્ચેની રેન્જ
\end{itemize}

\end{solutionbox}
\begin{mnemonicbox}
``નીચું મધ્ય ઉંચું - કેપેસિટર અહીં મહત્વપૂર્ણ છે''

\end{mnemonicbox}
\subsection*{પ્રશ્ન 2(ક) અથવા [7
ગુણ]}\label{uxaaauxab0uxab6uxaa8-2uxa95-uxa85uxaa5uxab5-7-uxa97uxaa3}

\textbf{ટ્રાન્સફોર્મર કપલ્ડ એમ્પ્લીફાયર અને RC કપલ્ડ એમ્પ્લીફાયરની સરખામણી કરો}

\begin{solutionbox}

{\def\LTcaptype{none} % do not increment counter
\begin{longtable}[]{@{}lll@{}}
\toprule\noalign{}
પેરામીટર & RC કપલ્ડ એમ્પ્લીફાયર & ટ્રાન્સફોર્મર કપલ્ડ એમ્પ્લીફાયર \\
\midrule\noalign{}
\endhead
\bottomrule\noalign{}
\endlastfoot
\textbf{કપલિંગ તત્વ} & રેઝિસ્ટર અને કેપેસિટર & ટ્રાન્સફોર્મર \\
\textbf{આવર્તન પ્રતિભાવ} & વિશાળ બેન્ડવિડ્થ & મર્યાદિત બેન્ડવિડ્થ \\
\textbf{કાર્યક્ષમતા} & ઓછી (20-25\%) & ઉચ્ચ (50-60\%) \\
\textbf{કદ \& વજન} & નાનું અને હલકું વજન & મોટું અને ભારે \\
\textbf{કિંમત} & સસ્તી & મોંઘી \\
\textbf{ઇમ્પીડન્સ મેચિંગ} & નબળું મેચિંગ & ઉત્કૃષ્ટ મેચિંગ \\
\textbf{વિકૃતિ} & ઓછી વિકૃતિ & કોર સેચુરેશનને કારણે વધુ \\
\textbf{DC આઇસોલેશન} & સારું આઇસોલેશન & ઉત્કૃષ્ટ આઇસોલેશન \\
\textbf{એપ્લિકેશન્સ} & સામાન્ય હેતુ & ઓડિયો પાવર એમ્પ્લીફાયર \\
\end{longtable}
}

\begin{verbatim}
flowchart TB
    subgraph RC\_Coupling
        A1[ટ્રાન્ઝિસ્ટર 1] {-{-}|કપલિંગ કેપેસિટર| B1[ટ્રાન્ઝિસ્ટર 2]}
    end

    subgraph Transformer\_Coupling
        A2[ટ્રાન્ઝિસ્ટર 1] {-{-}|ટ્રાન્સફોર્મર| B2[ટ્રાન્ઝિસ્ટર 2]}
    end
\end{verbatim}

\end{solutionbox}
\begin{mnemonicbox}
``RC વિશાળતા લે, ટ્રાન્સફોર્મર પાવર લે''

\end{mnemonicbox}
\subsection*{પ્રશ્ન 3(અ) [3
ગુણ]}\label{uxaaauxab0uxab6uxaa8-3uxa85-3-uxa97uxaa3}

\textbf{ડાયરેક્ટ કપલ્ડ એમ્પ્લીફાયરને સંક્ષિપ્તમાં સમજાવો.}

\begin{solutionbox}
ડાયરેક્ટ-કપલ્ડ એમ્પ્લીફાયર તબક્કાઓને કપલિંગ કેપેસિટર્સ અથવા
ટ્રાન્સફોર્મર વિના જોડે છે, જે DC સિગ્નલ એમ્પ્લિફિકેશનની મંજૂરી આપે છે.

\begin{verbatim}
flowchart LR
    In[ઇનપુટ] {-{-} A[પ્રથમ તબક્કો]}
    A {-{-} સીધું જોડાણ {-}{-} B[બીજો તબક્કો]}
    B {-{-} Out[આઉટપુટ]}
\end{verbatim}

\begin{itemize}
\tightlist
\item
  \textbf{DC સિગ્નલ હેન્ડલિંગ}: ખૂબ નીચા આવર્તન અને DC એમ્પ્લિફાય કરી શકે છે
\item
  \textbf{કોઈ કપલિંગ તત્વો નહીં}: પ્રથમ તબક્કાનું આઉટપુટ સીધું આગલા તબક્કાના ઇનપુટને
  જોડે છે
\item
  \textbf{આવર્તન પ્રતિભાવ}: ઉત્કૃષ્ટ નીચલા આવર્તનનો પ્રતિભાવ
\item
  \textbf{નુકસાન}: થર્મલ ડ્રિફ્ટ, બાયસ સ્થિરતાના મુદ્દાઓ
\end{itemize}

\end{solutionbox}
\begin{mnemonicbox}
``સીધું જોડાયેલ, સંપૂર્ણ શૂન્ય આવર્તન સુધી''

\end{mnemonicbox}
\subsection*{પ્રશ્ન 3(બ) [4
ગુણ]}\label{uxaaauxab0uxab6uxaa8-3uxaac-4-uxa97uxaa3}

\textbf{એમ્પ્લીફાયરના ફ્રીક્વન્સી રિસ્પોન્સ પર એમિટર બાયપાસ કેપેસિટર અને કપલિંગ
કેપેસિટરની અસરો સમજાવો.}

\begin{solutionbox}

{\def\LTcaptype{none} % do not increment counter
\begin{longtable}[]{@{}
  >{\raggedright\arraybackslash}p{(\linewidth - 4\tabcolsep) * \real{0.2157}}
  >{\raggedright\arraybackslash}p{(\linewidth - 4\tabcolsep) * \real{0.1961}}
  >{\raggedright\arraybackslash}p{(\linewidth - 4\tabcolsep) * \real{0.5882}}@{}}
\toprule\noalign{}
\begin{minipage}[b]{\linewidth}\raggedright
કેપેસિટર
\end{minipage} & \begin{minipage}[b]{\linewidth}\raggedright
કાર્ય
\end{minipage} & \begin{minipage}[b]{\linewidth}\raggedright
આવર્તન પ્રતિભાવ પર અસર
\end{minipage} \\
\midrule\noalign{}
\endhead
\bottomrule\noalign{}
\endlastfoot
\textbf{એમિટર બાયપાસ કેપેસિટર} & RE આસપાસ AC બાયપાસ કરે છે & મધ્ય અને ઉચ્ચ
આવર્તનો પર ગેઇન વધારે છે \\
\textbf{કપલિંગ કેપેસિટર} & DC અવરોધે, AC પસાર કરે & નીચલી કટ-ઓફ આવર્તન નક્કી કરે
છે \\
\end{longtable}
}

\begin{verbatim}
flowchart TB
    subgraph "ગેઇન પર અસરો"
    A[કેપેસિટર વિના] {-{-}|"નીચો ગેઇન"| B[ફક્ત કપલિંગ સાથે]}
    B {-{-}|"મધ્યમ ગેઇન"| C[કપલિંગ + બાયપાસ સાથે]}
    C {-{-}|"ઉચ્ચ ગેઇન"| D[આદર્શ પ્રતિભાવ]}
    end
\end{verbatim}

\begin{itemize}
\tightlist
\item
  \textbf{એમિટર બાયપાસ કેપેસિટર}:

  \begin{itemize}
  \tightlist
  \item
    \textbf{વિના}: નેગેટિવ ફીડબેકને કારણે ઓછો ગેઇન
  \item
    \textbf{સાથે}: AC સિગ્નલ માટે RE બાયપાસ થવાથી ઉચ્ચ ગેઇન
  \end{itemize}
\item
  \textbf{કપલિંગ કેપેસિટર}:

  \begin{itemize}
  \tightlist
  \item
    \textbf{ખૂબ નાનું}: નબળો નીચલા-આવર્તન પ્રતિભાવ
  \item
    \textbf{મોટું મૂલ્ય}: વધુ સારો નીચલા-આવર્તન પ્રતિભાવ
  \end{itemize}
\end{itemize}

\end{solutionbox}
\begin{mnemonicbox}
``કપલિંગ નીચા નિયંત્રણ કરે, બાયપાસ બધાને વધારે''

\end{mnemonicbox}
\subsection*{પ્રશ્ન 3(ક) [7
ગુણ]}\label{uxaaauxab0uxab6uxaa8-3uxa95-7-uxa97uxaa3}

\textbf{ટ્રાન્ઝિસ્ટર ટુ પોર્ટ નેટવર્ક દોરો અને તેના માટે h-પેરામીટર્સનું વર્ણન કરો.
હાઇબ્રિડ પરિમાણોના ફાયદા લખો.}

\begin{solutionbox}
બે-પોર્ટ નેટવર્ક એ h-પેરામીટર્સ (હાઇબ્રિડ પેરામીટર્સ)નો ઉપયોગ કરીને
ટ્રાન્ઝિસ્ટર વર્તનનું વિશ્લેષણ કરવા માટેનું મોડેલ છે.

\begin{verbatim}
                i_{1                 i_{2}}
                                  
                |                  |
     +{-{-}{-}{-}{-}{-}{-}{-}{-}{-}+{-}{-}{-}{-}{-}{-}{-}{-}{-}{-}{-}{-}{-}{-}{-}{-}{-}{-}+{-}{-}{-}{-}{-}{-}{-}{-}{-}+}
     |          |                  |         |
     |          |                  |         |
 v_{1  |    +{-}{-}{-}{-}{-}+{-}{-}{-}{-}{-}{-}+    +{-}{-}{-}{-}{-}+{-}{-}{-}{-}{-}{-}+   |}
 ↓   |    |     |      |    |     |      |   |
     |    |  Two{-Port  |    |     ↓      |   |}
     +{-{-}{-}{-}+   Network  +{-}{-}{-}{-}+     v_{2}     +{-}{-}{-}+}
     |    |            |    |            |   |
     |    |            |    |            |   |
     |    +{-{-}{-}{-}{-}{-}{-}{-}{-}{-}{-}{-}+    +{-}{-}{-}{-}{-}{-}{-}{-}{-}{-}{-}{-}+   |}
     |                                       |
     +{-{-}{-}{-}{-}{-}{-}{-}{-}{-}{-}{-}{-}{-}{-}{-}{-}{-}{-}{-}{-}{-}{-}{-}{-}{-}{-}{-}{-}{-}{-}{-}{-}{-}{-}{-}{-}{-}{-}+}
\end{verbatim}

{\def\LTcaptype{none} % do not increment counter
\begin{longtable}[]{@{}
  >{\raggedright\arraybackslash}p{(\linewidth - 4\tabcolsep) * \real{0.3023}}
  >{\raggedright\arraybackslash}p{(\linewidth - 4\tabcolsep) * \real{0.2791}}
  >{\raggedright\arraybackslash}p{(\linewidth - 4\tabcolsep) * \real{0.4186}}@{}}
\toprule\noalign{}
\begin{minipage}[b]{\linewidth}\raggedright
H-પેરામીટર
\end{minipage} & \begin{minipage}[b]{\linewidth}\raggedright
વ્યાખ્યા
\end{minipage} & \begin{minipage}[b]{\linewidth}\raggedright
ભૌતિક અર્થ
\end{minipage} \\
\midrule\noalign{}
\endhead
\bottomrule\noalign{}
\endlastfoot
\textbf{h_{1}_{1} (hᵢ_{e})} & આઉટપુટ શોર્ટ-સર્કિટેડ સાથે ઇનપુટ ઇમ્પીડન્સ & બેઝ-એમિટર
રેઝિસ્ટન્સ \\
\textbf{h_{1}_{2} (hᵣ_{e})} & ઇનપુટ ઓપન-સર્કિટેડ સાથે રિવર્સ વોલ્ટેજ ગેઇન & આઉટપુટથી ઇનપુટ
તરફ ફીડબેક \\
\textbf{h_{2}_{1} (hf_{e})} & આઉટપુટ શોર્ટ-સર્કિટેડ સાથે ફોરવર્ડ કરંટ ગેઇન & કરંટ ગેઇન
(β) \\
\textbf{h_{2}_{2} (ho_{e})} & ઇનપુટ ઓપન-સર્કિટેડ સાથે આઉટપુટ એડમિટન્સ & આઉટપુટ
કન્ડકટન્સ \\
\end{longtable}
}

\textbf{H-પેરામીટર્સના ફાયદા:}

\begin{itemize}
\tightlist
\item
  \textbf{સરળતાથી માપી શકાય}: સરળ સર્કિટ્સ સાથે સીધા માપન
\item
  \textbf{મિશ્રિત એકમો}: વોલ્ટેજ અને કરંટના ગુણોત્તરનો ઉપયોગ કરે છે
\item
  \textbf{મોડેલ ચોકસાઈ}: ટ્રાન્ઝિસ્ટર વર્તનની નજીકની એપ્રોક્સિમેશન
\item
  \textbf{ગાણિતિક સરળતા}: વિશ્લેષણ માટે લીનિયર સમીકરણો
\end{itemize}

\end{solutionbox}
\begin{mnemonicbox}
``ઇનપુટ, રિવર્સ, ફોરવર્ડ, આઉટપુટ - IRFO પેરામીટર્સ''

\end{mnemonicbox}
\subsection*{પ્રશ્ન 3(અ) અથવા [3
ગુણ]}\label{uxaaauxab0uxab6uxaa8-3uxa85-uxa85uxaa5uxab5-3-uxa97uxaa3}

\textbf{એમ્પ્લીફાયરનો ફ્રીક્વન્સી રિસ્પોન્સ દોરો અને પ્રતિસાદ પર એમ્પ્લીફાયરની અપર
કટ-ઓફ ફ્રીક્વન્સી, લોઅર કટ-ઓફ ફ્રીક્વન્સી, બેન્ડવિડ્થ અને મિડ ફ્રીક્વન્સી ગેઇન સૂચવો.}

\begin{solutionbox}
ફ્રીક્વન્સી રિસ્પોન્સ ગ્રાફ એમ્પ્લીફાયર માટે આવર્તન સાથે ગેઇન કેવી રીતે
બદલાય છે તે બતાવે છે.

\begin{verbatim}
    Gain(dB)
       ↑
       |                Mid{-frequency gain}
       |    ┌─────────────────────────────┐
       |    │                             │
  0.707 {-+                             +{-}}
       |   /│                             │{}
       |  / │                             │ {}
       | /  │                             │  {}
       |/   │                             │   {}
       +────┴─────────────────────────────┴──── Frequency(log scale)
          f_{1                              f_{2}}
          │                               │
          │───────── Bandwidth ─────────│
          │                               │
     Lower cutoff                    Upper cutoff
     frequency                       frequency
\end{verbatim}

\begin{itemize}
\tightlist
\item
  \textbf{મિડ-ફ્રીક્વન્સી ગેઇન (Av)}: સપાટ ક્ષેત્રમાં મહત્તમ ગેઇન
\item
  \textbf{લોઅર કટ-ઓફ ફ્રીક્વન્સી (f_{1})}: આવર્તન જ્યાં ગેઇન 0.707\timesAv (-3dB) સુધી
  ઘટે છે
\item
  \textbf{અપર કટ-ઓફ ફ્રીક્વન્સી (f_{2})}: આવર્તન જ્યાં ગેઇન 0.707\timesAv (-3dB) સુધી ઘટે
  છે
\item
  \textbf{બેન્ડવિડ્થ}: અપર અને લોઅર કટ-ઓફ આવર્તનો વચ્ચેનો તફાવત (f_{2} - f_{1})
\end{itemize}

\end{solutionbox}
\begin{mnemonicbox}
``લોઅર બેન્ડવિડ્થ અપર એમ્પ્લીફાયર પ્રતિસાદ બનાવે''

\end{mnemonicbox}
\subsection*{પ્રશ્ન 3(બ) અથવા [4
ગુણ]}\label{uxaaauxab0uxab6uxaa8-3uxaac-uxa85uxaa5uxab5-4-uxa97uxaa3}

\textbf{ટ્યુન કરેલ એમ્પ્લીફાયર તરીકે ઉપયોગમાં લેવાતા ટ્રાન્ઝિસ્ટરનું વર્ણન કરો.}

\begin{solutionbox}
ટ્યુન્ડ એમ્પ્લીફાયર ચોક્કસ આવર્તનો પર સિગ્નલને પસંદગીપૂર્વક એમ્પ્લિફાય
કરવા માટે LC રેઝોનન્ટ સર્કિટનો ઉપયોગ કરે છે.

\begin{verbatim}
flowchart LR
    A[ઇનપુટ સિગ્નલ] {-{-} B[ટ્રાન્ઝિસ્ટર એમ્પ્લીફાયર]}
    B {-{-} C[LC ટ્યુન્ડ સર્કિટ]}
    C {-{-} D[આઉટપુટ સિગ્નલ]}
\end{verbatim}

{\def\LTcaptype{none} % do not increment counter
\begin{longtable}[]{@{}ll@{}}
\toprule\noalign{}
ઘટક & કાર્ય \\
\midrule\noalign{}
\endhead
\bottomrule\noalign{}
\endlastfoot
\textbf{LC ટેન્ક સર્કિટ} & ચોક્કસ આવર્તન પર રેઝોનેટ કરે છે \\
\textbf{ટ્રાન્ઝિસ્ટર} & એમ્પ્લિફિકેશન પ્રદાન કરે છે \\
\textbf{રેઝોનન્સ આવર્તન} & f = 1/(2π\sqrtLC) \\
\textbf{ક્વોલિટી ફેક્ટર (Q)} & બેન્ડવિડ્થ નક્કી કરે છે \\
\end{longtable}
}

\begin{itemize}
\tightlist
\item
  \textbf{ઉચ્ચ પસંદગી}: રેઝોનન્ટ આવર્તન પર સિગ્નલ એમ્પ્લિફાય કરે છે
\item
  \textbf{એપ્લિકેશન્સ}: RF રિસીવર્સ, ટ્રાન્સમિટર્સ, કમ્યુનિકેશન્સ
\item
  \textbf{પ્રકારો}: સિંગલ-ટ્યુન્ડ, ડબલ-ટ્યુન્ડ, સ્ટેગર-ટ્યુન્ડ
\item
  \textbf{બેન્ડવિડ્થ}: Q ફેક્ટરના વ્યસ્ત પ્રમાણમાં
\end{itemize}

\end{solutionbox}
\begin{mnemonicbox}
``ટ્યુનિંગ LC સિગ્નલ્સ ચોકસાઈથી પસંદ કરે''

\end{mnemonicbox}
\subsection*{પ્રશ્ન 3(ક) અથવા [7
ગુણ]}\label{uxaaauxab0uxab6uxaa8-3uxa95-uxa85uxaa5uxab5-7-uxa97uxaa3}

\textbf{બે પોર્ટ નેટવર્કમાં h પરિમાણોનું મહત્વ વર્ણવો. CE એમ્પ્લીફાયર માટે
h-પેરામીટર્સ સર્કિટ દોરો.}

\begin{solutionbox}
H-પેરામીટર્સ ટ્રાન્ઝિસ્ટર સર્કિટ્સને બે-પોર્ટ નેટવર્ક તરીકે વિશ્લેષણ
કરવા માટે સંપૂર્ણ ગાણિતિક મોડેલ પ્રદાન કરે છે.

\textbf{h-પેરામીટર્સનું મહત્વ:}

{\def\LTcaptype{none} % do not increment counter
\begin{longtable}[]{@{}ll@{}}
\toprule\noalign{}
પાસું & મહત્વ \\
\midrule\noalign{}
\endhead
\bottomrule\noalign{}
\endlastfoot
\textbf{સર્કિટ વિશ્લેષણ} & જટિલ સર્કિટ્સ માટે સરળીકૃત સમીકરણો \\
\textbf{ડિઝાઇન ગણતરીઓ} & ગેઇન, ઇનપુટ/આઉટપુટ ઇમ્પીડન્સની આગાહી \\
\textbf{મેન્યુફેક્ચરર સ્પેક્સ} & ટ્રાન્ઝિસ્ટર લક્ષણો નિર્દિષ્ટ કરવાની માનક રીત \\
\textbf{સ્થિરતા વિશ્લેષણ} & સ્થિરતા શરતો નક્કી કરો \\
\textbf{આવર્તન પર આધાર} & આવર્તનો પર વર્તણૂકનું મોડેલ \\
\end{longtable}
}

\textbf{CE એમ્પ્લીફાયર h-પેરામીટર સમતુલ્ય સર્કિટ:}

\begin{verbatim}
        +{-{-}{-}{-}{-}{-}{-}+      RC}
        |       |      ┌─┐
    ┌─┐ |       |   ┌──┘ └──┐
   IB─ |       |   |       |
        |       |   |       +{-{-}{-}o Vout}
Vin o───┤   hie  hre    |
        |       |   |       |
        |       |   |   hoe |
        |  hfe   |       |
        |       |   |       |
        +{-{-}{-}{-}{-}{-}{-}+   +{-}{-}{-}{-}{-}{-}{-}+}
            |           |
            └───────────┘
                 GND
\end{verbatim}

\begin{itemize}
\tightlist
\item
  \textbf{hie}: ઇનપુટ ઇમ્પીડન્સ (બેઝ-એમિટર રેઝિસ્ટન્સ)
\item
  \textbf{hre}: રિવર્સ વોલ્ટેજ ફીડબેક રેશિયો
\item
  \textbf{hfe}: ફોરવર્ડ કરંટ ગેઇન (β)
\item
  \textbf{hoe}: આઉટપુટ એડમિટન્સ
\end{itemize}

\end{solutionbox}
\begin{mnemonicbox}
``ઇનપુટ રેઝિસ્ટન્સ, ફીડબેક રેશિયો, ફોરવર્ડ ગેઇન, આઉટપુટ
કન્ડક્ટન્સ''

\end{mnemonicbox}
\subsection*{પ્રશ્ન 4(અ) [3
ગુણ]}\label{uxaaauxab0uxab6uxaa8-4uxa85-3-uxa97uxaa3}

\textbf{જરૂરી ડાયાગ્રામ સાથે ડાયોડ ક્લિપર સર્કિટનું વર્ણન કરો.}

\begin{solutionbox}
ક્લિપર સર્કિટ ઇનપુટ સિગ્નલના તે ભાગને મર્યાદિત કરે છે અથવા કાપી
નાખે છે જે ચોક્કસ વોલ્ટેજ લેવલથી વધી જાય છે.

\begin{verbatim}
flowchart LR
    A[ઇનપુટ સિગ્નલ] {-{-} B[ડાયોડ ક્લિપર]}
    B {-{-} C[આઉટપુટ સિગ્નલ]}
\end{verbatim}

\begin{verbatim}
    ઇનપુટ               આઉટપુટ
     o─────┬───────────────o
           |
           |    D1
           ├────▶|─────┐
           |          ─┴─
           R           V
           |           │
           └───────────┘
              ગ્રાઉન્ડ
\end{verbatim}

\begin{itemize}
\tightlist
\item
  \textbf{ઓપરેશન}: ડાયોડ વોલ્ટેજ થ્રેશોલ્ડથી વધી જાય ત્યારે કન્ડક્ટ કરે છે
\item
  \textbf{પ્રકારો}:

  \begin{itemize}
  \tightlist
  \item
    \textbf{પોઝિટિવ ક્લિપર}: પોઝિટિવ હાફ-સાયકલ્સ ક્લિપ કરે છે
  \item
    \textbf{નેગેટિવ ક્લિપર}: નેગેટિવ હાફ-સાયકલ્સ ક્લિપ કરે છે
  \item
    \textbf{બાયસ્ડ ક્લિપર}: શૂન્ય સિવાયના વોલ્ટેજ લેવલ પર ક્લિપ કરે છે
  \end{itemize}
\end{itemize}

\end{solutionbox}
\begin{mnemonicbox}
``નિશ્ચિત પોઈન્ટ પર ભાગોને કાપી નાખે''

\end{mnemonicbox}
\subsection*{પ્રશ્ન 4(બ) [4
ગુણ]}\label{uxaaauxab0uxab6uxaa8-4uxaac-4-uxa97uxaa3}

\textbf{LDR પર ટૂંકી નોંધ સમજાવો.}

\begin{solutionbox}
LDR (લાઇટ ડિપેન્ડન્ટ રેઝિસ્ટર) એ ફોટોરેઝિસ્ટર છે જેનો રેઝિસ્ટન્સ
પ્રકાશની તીવ્રતા વધવાથી ઘટે છે.

\begin{verbatim}
flowchart LR
    A[પ્રકાશ] {-{-} B[LDR]}
    B {-{-} C[રેઝિસ્ટન્સ બદલાય]}
\end{verbatim}

{\def\LTcaptype{none} % do not increment counter
\begin{longtable}[]{@{}ll@{}}
\toprule\noalign{}
ગુણધર્મ & વર્ણન \\
\midrule\noalign{}
\endhead
\bottomrule\noalign{}
\endlastfoot
\textbf{રચના} & કેડમિયમ સલ્ફાઇડ (CdS) અથવા કેડમિયમ સેલેનાઇડ (CdSe) \\
\textbf{રેઝિસ્ટન્સ રેન્જ} & 1MΩ (અંધકાર) થી થોડા KΩ (તેજ પ્રકાશ) \\
\textbf{પ્રતિસાદ સમય} & સામાન્ય રીતે 10-100ms \\
\textbf{સ્પેક્ટ્રલ પ્રતિસાદ} & દૃશ્યમાન સ્પેક્ટ્રમમાં શ્રેષ્ઠ સંવેદનશીલતા \\
\end{longtable}
}

\begin{itemize}
\tightlist
\item
  \textbf{પ્રકાશનું શોષણ}: મુક્ત વાહકો ઉત્પન્ન કરે છે
\item
  \textbf{રેઝિસ્ટન્સ}: પ્રકાશની તીવ્રતાના વ્યસ્ત પ્રમાણમાં
\item
  \textbf{એપ્લિકેશન્સ}: લાઇટ સેન્સર, ઓટોમેટિક લાઇટિંગ, કેમેરા એક્સપોઝર કંટ્રોલ
\item
  \textbf{સિમ્બોલ}: અંદર પોઇન્ટિંગ એરો સાથે વેરિએબલ રેઝિસ્ટર
\end{itemize}

\end{solutionbox}
\begin{mnemonicbox}
``પ્રકાશ રેઝિસ્ટન્સ ઘટાડે''

\end{mnemonicbox}
\subsection*{પ્રશ્ન 4(ક) [7
ગુણ]}\label{uxaaauxab0uxab6uxaa8-4uxa95-7-uxa97uxaa3}

\textbf{ડાર્લિંગ્ટન જોડી અને તેની એપ્લિકેશનો સમજાવો.}

\begin{solutionbox}
ડાર્લિંગ્ટન જોડીમાં બે ટ્રાન્ઝિસ્ટર એવી રીતે જોડાયેલા હોય છે કે પ્રથમ
દ્વારા એમ્પ્લિફાઇડ કરેલો કરંટ બીજા દ્વારા વધુ એમ્પ્લિફાય થાય છે.

\begin{verbatim}
flowchart LR
    A[ઇનપુટ સિગ્નલ] {-{-} B[ટ્રાન્ઝિસ્ટર 1]}
    B {-{-} C[ટ્રાન્ઝિસ્ટર 2]}
    C {-{-} D[આઉટપુટ સિગ્નલ]}
\end{verbatim}

\begin{verbatim}
             +Vcc
               │
               │
               R
               │
               │
    બેઝ o──────┴───┐
               |   |
               |   |  કલેક્ટર
               |   +{-{-}{-}{-}{-}{-}{-}o}
               |   |
               |   |
               └─┬─┘
                 │
                 └─┐
               |   |
               |   | 
               |   |
               └─┬─┘
                 │
                 │
                GND
\end{verbatim}

{\def\LTcaptype{none} % do not increment counter
\begin{longtable}[]{@{}ll@{}}
\toprule\noalign{}
લક્ષણ & વર્ણન \\
\midrule\noalign{}
\endhead
\bottomrule\noalign{}
\endlastfoot
\textbf{કરંટ ગેઇન} & β\_total = β_{1} \times β_{2} (ખૂબ ઊંચો) \\
\textbf{ઇનપુટ ઇમ્પીડન્સ} & ખૂબ ઊંચું (β_{2} \times R\_e1) \\
\textbf{આઉટપુટ ઇમ્પીડન્સ} & નીચું \\
\textbf{સ્વિચિંગ સ્પીડ} & સિંગલ ટ્રાન્ઝિસ્ટર કરતાં ધીમી \\
\end{longtable}
}

\textbf{એપ્લિકેશન્સ:}

\begin{itemize}
\tightlist
\item
  \textbf{પાવર એમ્પ્લીફાયર}: ઉચ્ચ કરંટ ગેઇન એપ્લિકેશન્સ
\item
  \textbf{ઓડિયો એમ્પ્લીફાયર}: ઉચ્ચ ઇનપુટ ઇમ્પીડન્સ સ્ટેજ
\item
  \textbf{બફર સર્કિટ્સ}: લોડિંગ ઇફેક્ટ્સ ઘટાડવા
\item
  \textbf{મોટર કંટ્રોલ}: ઉચ્ચ-કરંટ લોડ ચલાવવા
\item
  \textbf{ટચ સેન્સિટિવ સ્વિચ}: ઉચ્ચ ગેઇનને કારણે ઉચ્ચ સંવેદનશીલતા
\end{itemize}

\end{solutionbox}
\begin{mnemonicbox}
``બમણા ટ્રાન્ઝિસ્ટર ખૂબ વધારે એમ્પ્લિફાય કરે''

\end{mnemonicbox}
\subsection*{પ્રશ્ન 4(અ) અથવા [3
ગુણ]}\label{uxaaauxab0uxab6uxaa8-4uxa85-uxa85uxaa5uxab5-3-uxa97uxaa3}

\textbf{જરૂરી ડાયાગ્રામ સાથે ડાયોડ ક્લેમ્પર સર્કિટનું વર્ણન કરો.}

\begin{solutionbox}
ક્લેમ્પર સર્કિટ સમગ્ર વેવફોર્મને તેના આકારને બદલ્યા વિના DC ઘટક
ઉમેરીને ઉપર અથવા નીચે શિફ્ટ કરે છે.

\begin{verbatim}
flowchart LR
    A[ઇનપુટ સિગ્નલ] {-{-} B[ડાયોડ ક્લેમ્પર]}
    B {-{-} C[આઉટપુટ સિગ્નલbr /શિફ્ટેડ વેવફોર્મ]}
\end{verbatim}

\begin{verbatim}
    ઇનપુટ         D           આઉટપુટ
     o─────┬─────|◄──────┬─────o
           |            ─┴─
           C             │
           |             R
           └─────────────┘
              ગ્રાઉન્ડ
\end{verbatim}

\begin{itemize}
\tightlist
\item
  \textbf{ઓપરેશન}: કેપેસિટર એક હાફ-સાયકલ દરમિયાન ચાર્જ થાય છે, DC લેવલ જાળવે છે
\item
  \textbf{પ્રકારો}:

  \begin{itemize}
  \tightlist
  \item
    \textbf{પોઝિટિવ ક્લેમ્પર}: વેવફોર્મને ઉપર શિફ્ટ કરે છે
  \item
    \textbf{નેગેટિવ ક્લેમ્પર}: વેવફોર્મને નીચે શિફ્ટ કરે છે
  \item
    \textbf{બાયસ્ડ ક્લેમ્પર}: ચોક્કસ DC લેવલ પર શિફ્ટ કરે છે
  \end{itemize}
\end{itemize}

\end{solutionbox}
\begin{mnemonicbox}
``પીક્સને સતત નીચે જકડે''

\end{mnemonicbox}
\subsection*{પ્રશ્ન 4(બ) અથવા [4
ગુણ]}\label{uxaaauxab0uxab6uxaa8-4uxaac-uxa85uxaa5uxab5-4-uxa97uxaa3}

\textbf{OLED નું કાર્ય અને એપ્લિકેશન સમજાવો.}

\begin{solutionbox}
OLED (ઓર્ગેનિક લાઇટ એમિટિંગ ડાયોડ) એ ડિસ્પ્લે ટેકનોલોજી છે જે
ઓર્ગેનિક કમ્પાઉન્ડનો ઉપયોગ કરે છે જે ઇલેક્ટ્રિક કરંટ પસાર થવાથી પ્રકાશ ઉત્સર્જિત કરે છે.

\begin{verbatim}
flowchart LR
    A[ઇલેક્ટ્રિક કરંટ] {-{-} B[OLED લેયર]}
    B {-{-} C[પ્રકાશ ઉત્સર્જન]}
\end{verbatim}

{\def\LTcaptype{none} % do not increment counter
\begin{longtable}[]{@{}ll@{}}
\toprule\noalign{}
લેયર & કાર્ય \\
\midrule\noalign{}
\endhead
\bottomrule\noalign{}
\endlastfoot
\textbf{કેથોડ} & ઇલેક્ટ્રોન્સ ઇન્જેક્ટ કરે છે \\
\textbf{એમિસિવ લેયર} & ઓર્ગેનિક મટિરિયલ જે પ્રકાશ ઉત્સર્જિત કરે છે \\
\textbf{કન્ડક્ટિવ લેયર} & એનોડથી હોલ્સ વહન કરે છે \\
\textbf{એનોડ} & હોલ્સ ઇન્જેક્ટ કરે છે (સામાન્ય રીતે પારદર્શક) \\
\end{longtable}
}

\begin{itemize}
\tightlist
\item
  \textbf{કાર્ય સિદ્ધાંત}: ઇલેક્ટ્રોન-હોલ રિકોમ્બિનેશન ફોટોન્સ બનાવે છે
\item
  \textbf{સ્વ-પ્રકાશિત}: LCD ની વિપરીત બેકલાઇટની જરૂર નથી
\item
  \textbf{પ્રકારો}: PMOLED (પેસિવ મેટ્રિક્સ) અને AMOLED (એક્ટિવ મેટ્રિક્સ)
\item
  \textbf{ફાયદાઓ}: પાતળા, હલકા, વિશાળ દ્રશ્ય કોણ, વધુ સારો કોન્ટ્રાસ્ટ
\end{itemize}

\textbf{એપ્લિકેશન્સ:}

\begin{itemize}
\tightlist
\item
  સ્માર્ટફોન અને ટેબ્લેટ
\item
  ટેલિવિઝન સ્ક્રીન
\item
  ડિજિટલ કેમેરા ડિસ્પ્લે
\item
  વેરેબલ ડિવાઇસ
\item
  લાઇટિંગ પેનલ
\end{itemize}

\end{solutionbox}
\begin{mnemonicbox}
``ઓર્ગેનિક લેયર્સ ડાયોડ-પ્રકાશ ઉત્સર્જિત કરે''

\end{mnemonicbox}
\subsection*{પ્રશ્ન 4(ક) અથવા [7
ગુણ]}\label{uxaaauxab0uxab6uxaa8-4uxa95-uxa85uxaa5uxab5-7-uxa97uxaa3}

\textbf{રિલે ડ્રાઇવર તરીકે વપરાતા ટ્રાન્ઝિસ્ટરનું વર્ણન કરો.}

\begin{solutionbox}
રિલે ડ્રાઇવર એક ટ્રાન્ઝિસ્ટરનો ઉપયોગ કરીને રિલેને નિયંત્રિત કરે છે, જે
ઓછા-કરંટ કંટ્રોલ સિગ્નલને ઉચ્ચ-કરંટ લોડ સ્વિચ કરવાની મંજૂરી આપે છે.

\begin{verbatim}
flowchart LR
    A[કંટ્રોલ સિગ્નલ] {-{-} B[ટ્રાન્ઝિસ્ટર]}
    B {-{-} C[રિલે કોઇલ]}
    C {-{-} D[સ્વિચ્ડ લોડ]}
\end{verbatim}

\begin{verbatim}
    +Vcc
     │
     ┌┐
    ┌┘└┐ રિલે
    │  │ કોઇલ
    └┐┌┘
     ││
     ││    ફ્લાયબેક
     ││    ડાયોડ
     ││    ┌─┐
     └┴────┤{├─┐}
           └─┘ │
            ┌──┴─┐
            │    │
            │    │ ટ્રાન્ઝિસ્ટર
 ઇનપુટ ──────┤    │
            │    │
            └────┘
              │
             GND
\end{verbatim}

{\def\LTcaptype{none} % do not increment counter
\begin{longtable}[]{@{}ll@{}}
\toprule\noalign{}
ઘટક & કાર્ય \\
\midrule\noalign{}
\endhead
\bottomrule\noalign{}
\endlastfoot
\textbf{ટ્રાન્ઝિસ્ટર} & રિલે ચલાવવા માટે કંટ્રોલ સિગ્નલને એમ્પ્લિફાય કરે છે \\
\textbf{ફ્લાયબેક ડાયોડ} & બેક EMF થી ટ્રાન્ઝિસ્ટરને સુરક્ષિત કરે છે \\
\textbf{બેઝ રેઝિસ્ટર} & બેઝ કરંટ મર્યાદિત કરે છે \\
\textbf{રિલે કોઇલ} & ઇલેક્ટ્રોમેગ્નેટિક સ્વિચ \\
\end{longtable}
}

\textbf{એપ્લિકેશન્સ:}

\begin{itemize}
\tightlist
\item
  મોટર કંટ્રોલ સર્કિટ્સ
\item
  ઔદ્યોગિક ઓટોમેશન
\item
  ઓટોમોટિવ ઇલેક્ટ્રોનિક્સ
\item
  હોમ એપ્લાયન્સ કંટ્રોલ
\item
  પાવર ડિસ્ટ્રિબ્યુશન સિસ્ટમ
\end{itemize}

\end{solutionbox}
\begin{mnemonicbox}
``નાનું મોટા રિલે ચલાવે''

\end{mnemonicbox}
\subsection*{પ્રશ્ન 5(અ) [3
ગુણ]}\label{uxaaauxab0uxab6uxaa8-5uxa85-3-uxa97uxaa3}

\textbf{LM317 IC નો ઉપયોગ કરીને વેરિયેબલ પાવર સપ્લાયનો સર્કિટ ડાયાગ્રામ દોરો.}

\begin{solutionbox}
LM317 એક એડજસ્ટેબલ વોલ્ટેજ રેગ્યુલેટર છે જેનો ઉપયોગ વેરિયેબલ પાવર
સપ્લાય બનાવવા માટે થઈ શકે છે.

\begin{verbatim}
               LM317
    ઇનપુટ      ┌───┐
    o─────────┤IN │
              │   │
              │ADJ├─┬─────┬───o આઉટપુટ
              │   │ │     │
              └───┘ │     │
                │   │     │
                R1  R2    C2
                │   │     │
                └───┴─────┘
                    GND
\end{verbatim}

\begin{itemize}
\tightlist
\item
  \textbf{ઘટકો}:

  \begin{itemize}
  \tightlist
  \item
    \textbf{LM317}: એડજસ્ટેબલ વોલ્ટેજ રેગ્યુલેટર IC
  \item
    \textbf{R1}: ફિક્સ્ડ 240Ω રેઝિસ્ટર
  \item
    \textbf{R2}: વેરિયેબલ રેઝિસ્ટર (પોટેન્શિયોમીટર)
  \item
    \textbf{C1, C2}: ફિલ્ટર કેપેસિટર
  \end{itemize}
\item
  \textbf{આઉટપુટ વોલ્ટેજ}: VOUT = 1.25 \times (1 + R2/R1)
\end{itemize}

\end{solutionbox}
\begin{mnemonicbox}
``LM317 વોલ્ટેજ એડજસ્ટેબલ બનાવે''

\end{mnemonicbox}
\subsection*{પ્રશ્ન 5(બ) [4
ગુણ]}\label{uxaaauxab0uxab6uxaa8-5uxaac-4-uxa97uxaa3}

\textbf{યુપીએસની કામગીરી સમજાવો.}

\begin{solutionbox}
UPS (અનઇન્ટરપ્ટિબલ પાવર સપ્લાય) મુખ્ય પાવર ફેઇલ થાય ત્યારે
ઇમરજન્સી પાવર પ્રદાન કરે છે.

\begin{verbatim}
flowchart LR
    A[AC મેઇન્સ] {-{-} B[રેક્ટિફાયર]}
    B {-{-} C[બેટરી ચાર્જર]}
    C {-{-} D[બેટરી]}
    D {-{-} E[ઇન્વર્ટર]}
    E {-{-} F[આઉટપુટ લોડ]}
    A {-.બાયપાસ.{-} F}
\end{verbatim}

{\def\LTcaptype{none} % do not increment counter
\begin{longtable}[]{@{}ll@{}}
\toprule\noalign{}
UPS પ્રકાર & ઓપરેશન \\
\midrule\noalign{}
\endhead
\bottomrule\noalign{}
\endlastfoot
\textbf{ઓફલાઇન/સ્ટેન્ડબાય} & પાવર ફેઇલ થાય ત્યારે બેટરી પર સ્વિચ કરે છે \\
\textbf{લાઇન-ઇન્ટરેક્ટિવ} & વોલ્ટેજ રેગ્યુલેટ કરે છે અને બેટરી પર સ્વિચ કરે છે \\
\textbf{ઓનલાઇન/ડબલ-કન્વર્ઝન} & હંમેશા બેટરીથી પાવર આપે છે, સતત ચાર્જ થાય છે \\
\end{longtable}
}

\begin{itemize}
\tightlist
\item
  \textbf{મુખ્ય ઘટકો}: રેક્ટિફાયર, બેટરી, ઇન્વર્ટર, કંટ્રોલ સર્કિટ
\item
  \textbf{કાર્યો}:

  \begin{itemize}
  \tightlist
  \item
    પાવર કન્ડિશનિંગ
  \item
    વોલ્ટેજ રેગ્યુલેશન
  \item
    સર્જ પ્રોટેક્શન
  \item
    બેટરી બેકઅપ
  \end{itemize}
\end{itemize}

\end{solutionbox}
\begin{mnemonicbox}
``અવિરત પાવર બ્લેકઆઉટ દરમિયાન આપે''

\end{mnemonicbox}
\subsection*{પ્રશ્ન 5(ક) [7
ગુણ]}\label{uxaaauxab0uxab6uxaa8-5uxa95-7-uxa97uxaa3}

\textbf{SMPS બ્લોક ડાયાગ્રામ દોરો અને સમજાવો.}

\begin{solutionbox}
SMPS (સ્વિચ મોડ પાવર સપ્લાય) ઇલેક્ટ્રિકલ પાવરને કુશળતાથી રૂપાંતરિત
કરવા માટે સ્વિચિંગ રેગ્યુલેશનનો ઉપયોગ કરે છે.

\begin{verbatim}
flowchart LR
    A[AC ઇનપુટ] {-{-} B[EMI ફિલ્ટર]}
    B {-{-} C[રેક્ટિફાયર \& ફિલ્ટર]}
    C {-{-} D[હાઇ ફ્રીક્વન્સીbr /સ્વિચિંગ સર્કિટ]}
    D {-{-} E[ટ્રાન્સફોર્મર]}
    E {-{-} F[આઉટપુટ રેક્ટિફાયરbr /\& ફિલ્ટર]}
    F {-{-} G[DC આઉટપુટ]}
    H[ફીડબેક \& કંટ્રોલ] {-{-} D}
    F {-{-} H}
\end{verbatim}

{\def\LTcaptype{none} % do not increment counter
\begin{longtable}[]{@{}ll@{}}
\toprule\noalign{}
બ્લોક & કાર્ય \\
\midrule\noalign{}
\endhead
\bottomrule\noalign{}
\endlastfoot
\textbf{EMI ફિલ્ટર} & ઇલેક્ટ્રોમેગ્નેટિક ઇન્ટરફેરન્સ ઘટાડે છે \\
\textbf{રેક્ટિફાયર \& ફિલ્ટર} & AC ને DC માં રૂપાંતરિત કરે છે અને સ્મૂધ કરે છે \\
\textbf{સ્વિચિંગ સર્કિટ} & DC ને ઉચ્ચ આવર્તન પર ચોપ કરે છે \\
\textbf{ટ્રાન્સફોર્મર} & આઇસોલેશન અને વોલ્ટેજ રૂપાંતરણ પ્રદાન કરે છે \\
\textbf{આઉટપુટ રેક્ટિફાયર} & ઉચ્ચ-આવર્તન AC ને પાછું DC માં રૂપાંતરિત કરે છે \\
\textbf{ફીડબેક સર્કિટ} & આઉટપુટ વોલ્ટેજ નિયંત્રિત કરે છે \\
\end{longtable}
}

\begin{itemize}
\tightlist
\item
  \textbf{ફાયદા}: ઉચ્ચ કાર્યક્ષમતા (70-90\%), નાનું કદ, ઓછું વજન
\item
  \textbf{ઓપરેશન}: 20-200 kHz પર PWM (પલ્સ વિડ્થ મોડ્યુલેશન)નો ઉપયોગ કરે છે
\item
  \textbf{પ્રકારો}: ફોરવર્ડ, ફ્લાયબેક, પુશ-પુલ, હાફ બ્રિજ, ફુલ બ્રિજ
\item
  \textbf{એપ્લિકેશન્સ}: કમ્પ્યુટર્સ, ટીવી, મોબાઇલ ચાર્જર્સ, LED ડ્રાઇવર્સ
\end{itemize}

\end{solutionbox}
\begin{mnemonicbox}
``સ્વિચ પાવરને સ્થિર બનાવે''

\end{mnemonicbox}
\subsection*{પ્રશ્ન 5(અ) અથવા [3
ગુણ]}\label{uxaaauxab0uxab6uxaa8-5uxa85-uxa85uxaa5uxab5-3-uxa97uxaa3}

\textbf{IC નો ઉપયોગ કરીને +15 v પાવર સપ્લાય માટે સર્કિટ ડાયાગ્રામ દોરો અને
ટૂંકમાં સમજાવો}

\begin{solutionbox}
+15V પાવર સપ્લાય 7815 વોલ્ટેજ રેગ્યુલેટર IC નો ઉપયોગ કરીને બનાવી
શકાય છે.

\begin{verbatim}
    AC Input    Bridge     7815
      o         Rectifier   ┌───┐
    {        ┌───┐      │   │}
      o          │   ├──────┤IN │
                 │   │      │   │         +15V
                 │   │ C1   │OUT├─────────o
                 │   ├──┐   │   │    C2
                 └───┘  │   │   │    │
                        │   └───┘    │
                        │     │      │
                        └─────┴──────┘
                             GND
\end{verbatim}

\begin{itemize}
\tightlist
\item
  \textbf{ઘટકો}:

  \begin{itemize}
  \tightlist
  \item
    \textbf{7815}: ફિક્સ્ડ +15V વોલ્ટેજ રેગ્યુલેટર IC
  \item
    \textbf{બ્રિજ રેક્ટિફાયર}: AC ને પલ્સેટિંગ DC માં રૂપાંતરિત કરે છે
  \item
    \textbf{C1}: ઇનપુટ ફિલ્ટર કેપેસિટર (1000-2200µF)
  \item
    \textbf{C2}: આઉટપુટ ફિલ્ટર કેપેસિટર (10-100µF)
  \end{itemize}
\item
  \textbf{કાર્ય}: AC રેક્ટિફાય કરે છે, ફિલ્ટર કરે છે, પછી સ્થિર +15V DC માં રેગ્યુલેટ
  કરે છે
\end{itemize}

\end{solutionbox}
\begin{mnemonicbox}
``7815 Fixes Voltage To Fifteen''

\end{mnemonicbox}
\subsection*{પ્રશ્ન 5(બ) અથવા [4
ગુણ]}\label{uxaaauxab0uxab6uxaa8-5uxaac-uxa85uxaa5uxab5-4-uxa97uxaa3}

\textbf{સૌર બેટરી ચાર્જર સર્કિટનું કાર્ય સમજાવો.}

\begin{solutionbox}
સોલર બેટરી ચાર્જર સૂર્યપ્રકાશને ઇલેક્ટ્રિકલ એનર્જીમાં રૂપાંતરિત કરીને
બેટરીને ચાર્જ કરે છે.

\begin{verbatim}
flowchart LR
    A[સોલર પેનલ] {-{-} B[ચાર્જ કંટ્રોલર]}
    B {-{-} C[બેટરી]}
    C {-{-} D[લોડ]}
\end{verbatim}

{\def\LTcaptype{none} % do not increment counter
\begin{longtable}[]{@{}ll@{}}
\toprule\noalign{}
ઘટક & કાર્ય \\
\midrule\noalign{}
\endhead
\bottomrule\noalign{}
\endlastfoot
\textbf{સોલર પેનલ} & સૂર્યપ્રકાશને વીજળીમાં રૂપાંતરિત કરે છે \\
\textbf{બ્લોકિંગ ડાયોડ} & રાત્રે પેનલ મારફતે બેટરી ડિસ્ચાર્જ થતી અટકાવે છે \\
\textbf{ચાર્જ કંટ્રોલર} & ચાર્જિંગ વોલ્ટેજ અને કરંટને નિયંત્રિત કરે છે \\
\textbf{બેટરી} & ઇલેક્ટ્રિકલ એનર્જી સંગ્રહ કરે છે \\
\end{longtable}
}

\begin{itemize}
\tightlist
\item
  \textbf{ઓપરેટિંગ મોડ્સ}:

  \begin{itemize}
  \tightlist
  \item
    \textbf{બલ્ક ચાર્જિંગ}: \textasciitilde80\% ચાર્જ થાય ત્યાં સુધી મહત્તમ કરંટ
  \item
    \textbf{એબ્સોર્પશન}: સ્થિર વોલ્ટેજ, ઘટતો કરંટ
  \item
    \textbf{ફ્લોટ/ટ્રિકલ}: પૂર્ણ ચાર્જ જાળવે છે
  \end{itemize}
\item
  \textbf{સુરક્ષા ફીચર્સ}: ઓવરચાર્જ, ઓવર-ડિસ્ચાર્જ, રિવર્સ પોલારિટી
\end{itemize}

\end{solutionbox}
\begin{mnemonicbox}
``સૂર્ય બેટરી સુરક્ષિત ચાર્જ કરે''

\end{mnemonicbox}
\subsection*{પ્રશ્ન 5(ક) અથવા [7
ગુણ]}\label{uxaaauxab0uxab6uxaa8-5uxa95-uxa85uxaa5uxab5-7-uxa97uxaa3}

\textbf{લિનિયર રેગ્યુલેટેડ પાવર સપ્લાય સાથે સ્વિચ મોડ પાવર સપ્લાયની સરખામણી ચર્ચા
કરો.}

\begin{solutionbox}

{\def\LTcaptype{none} % do not increment counter
\begin{longtable}[]{@{}lll@{}}
\toprule\noalign{}
પેરામીટર & લિનિયર પાવર સપ્લાય & સ્વિચ મોડ પાવર સપ્લાય \\
\midrule\noalign{}
\endhead
\bottomrule\noalign{}
\endlastfoot
\textbf{ઓપરેટિંગ સિદ્ધાંત} & સતત વોલ્ટેજ રેગ્યુલેશન & ઉચ્ચ-આવર્તન સ્વિચિંગ \\
\textbf{કાર્યક્ષમતા} & નીચી (30-40\%) & ઉચ્ચ (70-90\%) \\
\textbf{કદ \& વજન} & મોટું અને ભારે & કોમ્પેક્ટ અને હલકું વજન \\
\textbf{ગરમી વિસર્જન} & ઉચ્ચ & નીચું \\
\textbf{આઉટપુટ નોઇઝ} & ખૂબ નીચું & ઉચ્ચ (સ્વિચિંગ નોઇઝ) \\
\textbf{પ્રતિસાદ સમય} & ઝડપી & ધીમું \\
\textbf{ઘટક સંખ્યા} & ઓછી & વધુ \\
\textbf{કિંમત} & ઓછી પાવર માટે ઓછી & ઉચ્ચ પાવર માટે ઓછી \\
\textbf{જટિલતા} & સરળ ડિઝાઇન & જટિલ ડિઝાઇન \\
\textbf{EMI} & નીચું & ઉચ્ચ (ફિલ્ટરિંગની જરૂર) \\
\end{longtable}
}

\begin{verbatim}
flowchart TB
    subgraph Linear
    A1[ટ્રાન્સફોર્મર] {-{-} B1[રેક્ટિફાયર]}
    B1 {-{-} C1[ફિલ્ટર]}
    C1 {-{-} D1[સિરીઝ પાસ એલિમેન્ટ]}
    D1 {-{-} E1[આઉટપુટ]}
    end

    subgraph SMPS
    A2[રેક્ટિફાયર] {-{-} B2[સ્વિચ]}
    B2 {-{-} C2[ટ્રાન્સફોર્મર]}
    C2 {-{-} D2[રેક્ટિફાયર \& ફિલ્ટર]}
    D2 {-{-} E2[આઉટપુટ]}
    F2[ફીડબેક] {-{-} B2}
    end
\end{verbatim}

\textbf{એપ્લિકેશન્સ:}

\begin{itemize}
\tightlist
\item
  \textbf{લિનિયર}: ઓડિયો ઇક્વિપમેન્ટ, લેબોરેટરી ઇન્સ્ટ્રુમેન્ટ્સ, સંવેદનશીલ સર્કિટ્સ
\item
  \textbf{SMPS}: કમ્પ્યુટર્સ, ટીવી, મોબાઇલ ચાર્જર્સ, ઔદ્યોગિક પાવર સપ્લાય
\end{itemize}

\end{solutionbox}
\begin{mnemonicbox}
``લિનિયર ઓછા નોઇઝને પસંદ કરે, સ્વિચિંગ કદ બચાવે''

\end{mnemonicbox}

\end{document}
