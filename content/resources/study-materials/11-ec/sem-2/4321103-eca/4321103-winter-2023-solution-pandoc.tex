\documentclass[10pt,a4paper]{article}

% content/resources/templates/preamble.tex
\usepackage[margin=0.6in]{geometry}
\author{Milav Dabgar}
\usepackage{amsmath,amssymb,amsthm}
\usepackage{booktabs}
\usepackage{multirow}
\usepackage{xcolor}
\usepackage{tcolorbox}
\tcbuselibrary{breakable,skins}
\usepackage[colorlinks=true,linkcolor=blue]{hyperref}
\usepackage{titlesec}
\usepackage{enumitem}
\usepackage{tikz}
\usepackage{pgfplots}
\usepackage{circuitikz}
\usepackage[version=4]{mhchem}
\usepackage{longtable}
\usepackage{array}
\usepackage{float}
\usepackage{caption}
\usepackage{listings}

\lstset{
  basicstyle=\small\ttfamily,
  breaklines=true,
  breakatwhitespace=false,
  postbreak=\mbox{\textcolor{red}{$\hookrightarrow$}\space},
  float=false,
  numbers=left,
  numberstyle=\tiny\color{gray},
  numbersep=10pt,
  xleftmargin=2em,
  keywordstyle=\color{blue},
  commentstyle=\color{green!60!black},
  stringstyle=\color{purple},
  backgroundcolor=\color{gray!5},
  showstringspaces=false,
  tabsize=2,
  captionpos=b,
  keepspaces=true,
  columns=flexible
}

\pgfplotsset{compat=1.18}
\usetikzlibrary{shapes,arrows,positioning,calc,patterns,decorations.pathmorphing,decorations.markings,arrows.meta}

% Color scheme
\definecolor{headcolor}{RGB}{0,102,204}
\definecolor{keycolor}{RGB}{220,20,60}
\definecolor{solutioncolor}{RGB}{34,139,34}
\definecolor{mnemoniccolor}{RGB}{148,0,211}
\definecolor{codecolor}{RGB}{0,0,100}

% Spacing
\setlength{\parskip}{3pt}
\setlist[itemize]{nosep}
\setlist[enumerate]{nosep}

% Title formatting
\titleformat{\section}{\Large\bfseries\color{headcolor}}{\thesection}{1em}{}
\titleformat{\subsection}{\large\bfseries\color{headcolor}}{\thesubsection}{1em}{}

% Pandoc tightlist compatibility
\providecommand{\tightlist}{%
  \setlength{\itemsep}{0pt}\setlength{\parskip}{0pt}}

% Pandoc longtable compatibility
\newcounter{none}
\def\thenone{}


% content/resources/templates/english-boxes.tex
% This file is currently empty - it exists to maintain consistency with the import structure.
% Add custom environments here if needed in the future.


\begin{document}

\begin{center}
{\Huge\bfseries\color{headcolor} Subject Name Solutions}\\[5pt]
{\LARGE 4321103 -- Winter 2023}\\[3pt]
{\large Semester 1 Study Material}\\[3pt]
{\normalsize\textit{Detailed Solutions and Explanations}}
\end{center}

\vspace{10pt}

\subsection*{Question 1(a) [3 marks]}\label{q1a}

\textbf{What is transistor biasing? What is its need?}

\begin{solutionbox}
Transistor biasing is the process of establishing a
stable DC operating point (Q-point) for proper amplification of AC
signals.


{\def\LTcaptype{none} % do not increment counter
\vspace{-5pt}
\captionof{table}{Need for Transistor Biasing}
\vspace{-10pt}
\begin{longtable}[]{@{}
  >{\raggedright\arraybackslash}p{(\linewidth - 2\tabcolsep) * \real{0.4000}}
  >{\raggedright\arraybackslash}p{(\linewidth - 2\tabcolsep) * \real{0.6000}}@{}}
\toprule\noalign{}
\begin{minipage}[b]{\linewidth}\raggedright
Aspect
\end{minipage} & \begin{minipage}[b]{\linewidth}\raggedright
Importance
\end{minipage} \\
\midrule\noalign{}
\endhead
\bottomrule\noalign{}
\endlastfoot
Stability & Maintains stable Q-point despite temperature variations \\
Linearity & Ensures operation in linear region for distortion-free
amplification \\
Efficiency & Prevents signal clipping and maximizes signal swing \\
Reliability & Avoids thermal runaway and protects the transistor \\
\end{longtable}
}

\end{solutionbox}
\begin{mnemonicbox}
``SOLE operation'' (Stability, Operating point,
Linearity, Efficiency)

\end{mnemonicbox}
\subsection*{Question 1(b) [4 marks]}\label{q1b}

\textbf{Explain load line for CE amplifier}

\begin{solutionbox}
Load line is a graphical representation of all possible
operating points of a transistor circuit.

\textbf{Diagram:}

\begin{center}
\textbf{Mermaid Diagram (Code)}
\begin{verbatim}
{Shaded}
{Highlighting}[]
graph LR
    A[DC Load Line] {-{-}{-} B[CE Amplifier]}
    B {-{-}{-} C[AC Load Line]}
    C {-{-}{-} D[Q{-}point]}

    style A fill:\#f9f,stroke:\#333,stroke{-width:1px}
    style B fill:\#bbf,stroke:\#333,stroke{-width:1px}
    style C fill:\#f9f,stroke:\#333,stroke{-width:1px}
    style D fill:\#bfb,stroke:\#333,stroke{-width:1px}
{Highlighting}
{Shaded}
\end{verbatim}
\end{center}

\begin{itemize}
\tightlist
\item
  \textbf{DC load line}: Drawn between saturation point (Ic=Vcc/Rc,
  Vce=0) and cutoff point (Ic=0, Vce=Vcc)
\item
  \textbf{AC load line}: Passes through Q-point with slope = -1/rc (rc =
  AC collector resistance)
\item
  \textbf{Q-point}: Operating point where DC biasing conditions are
  established
\end{itemize}

\end{solutionbox}
\begin{mnemonicbox}
``SCQ points'' (Saturation, Cutoff, Q-point)

\end{mnemonicbox}
\subsection*{Question 1(c) [7 marks]}\label{q1c}

\textbf{List various biasing method of transistor and explain any one of
them.}

\begin{solutionbox}
Various biasing methods for transistors include:


{\def\LTcaptype{none} % do not increment counter
\vspace{-5pt}
\captionof{table}{Transistor Biasing Methods}
\vspace{-10pt}
\begin{longtable}[]{@{}
  >{\raggedright\arraybackslash}p{(\linewidth - 2\tabcolsep) * \real{0.3810}}
  >{\raggedright\arraybackslash}p{(\linewidth - 2\tabcolsep) * \real{0.6190}}@{}}
\toprule\noalign{}
\begin{minipage}[b]{\linewidth}\raggedright
Method
\end{minipage} & \begin{minipage}[b]{\linewidth}\raggedright
Key Feature
\end{minipage} \\
\midrule\noalign{}
\endhead
\bottomrule\noalign{}
\endlastfoot
Fixed bias & Single resistor for base bias \\
Collector-to-base bias & Self-stabilizing due to negative feedback \\
Voltage divider bias & Most stable due to voltage divider network \\
Emitter bias & Provides excellent stability with emitter resistor \\
Combination bias & Uses multiple feedback paths for optimal stability \\
\end{longtable}
}

\textbf{Explanation of Voltage Divider Bias:}

\textbf{Diagram:}

\begin{verbatim}
     +Vcc
       |
       R1
       |
       +{-{-}{-}{-}{-}{-}+}
       |      |
       R2     Rc
       |      |
       |      C
  +{-{-}{-}{-}+      |}
  |   B  +{-{-}{-}{-}+{-}{-}{-}{-}+ Output}
  +{-{-}{-}{-}{-}|    |     |}
        |    |     |
  Input +{-{-}+ |     |}
        | C  E     |
        |    |     |
        +{-{-}{-}{-}+     |}
             |     |
             Re    |
             |     |
             +{-{-}{-}{-}{-}+}
             |
            GND
\end{verbatim}

\begin{itemize}
\tightlist
\item
  \textbf{Operation}: R1 and R2 form a voltage divider to set base
  voltage
\item
  \textbf{Stability}: Excellent thermal stability due to stiff voltage
  divider
\item
  \textbf{Efficiency}: Most widely used due to independence from β
  variations
\item
  \textbf{Calculation}: Base voltage = Vcc \times R2/(R1+R2)
\end{itemize}

\end{solutionbox}
\begin{mnemonicbox}
``VISE grip'' (Voltage divider, Independent of β,
Stable, Efficient)

\end{mnemonicbox}
\subsection*{Question 1(c) OR [7
marks]}\label{q1c}

\textbf{Explain voltage divider biasing method with help of circuit
diagram}

\begin{solutionbox}
Voltage divider biasing is the most stable method to
bias a transistor.

\textbf{Diagram:}

\begin{verbatim}
     +Vcc
       |
       R1
       |
       +{-{-}{-}{-}{-}{-}+}
       |      |
       R2     Rc
       |      |
       |      C
  +{-{-}{-}{-}+      |}
  |   B  +{-{-}{-}{-}+{-}{-}{-}{-}+ Output}
  +{-{-}{-}{-}{-}|    |     |}
        |    |     |
  Input +{-{-}+ |     |}
        | C  E     |
        |    |     |
        +{-{-}{-}{-}+     |}
             |     |
             Re    |
             |     |
             +{-{-}{-}{-}{-}+}
             |
            GND
\end{verbatim}


{\def\LTcaptype{none} % do not increment counter
\vspace{-5pt}
\captionof{table}{Features of Voltage Divider Biasing}
\vspace{-10pt}
\begin{longtable}[]{@{}ll@{}}
\toprule\noalign{}
Component & Function \\
\midrule\noalign{}
\endhead
\bottomrule\noalign{}
\endlastfoot
R1, R2 & Creates stable base voltage independent of β \\
Rc & Limits collector current and develops output voltage \\
Re & Provides stability via negative feedback \\
Bypass capacitor & Bypasses AC signal around Re to increase gain \\
\end{longtable}
}

\begin{itemize}
\tightlist
\item
  \textbf{Working principle}: R1 and R2 form a voltage divider that sets
  the base voltage
\item
  \textbf{Thermal stability}: Re provides negative feedback for
  excellent thermal stability
\item
  \textbf{Advantage}: Q-point remains stable despite variations in
  temperature and β
\end{itemize}

\end{solutionbox}
\begin{mnemonicbox}
``BEST bias'' (Base voltage, Emitter stability, Stiff
divider, Temperature stable)

\end{mnemonicbox}
\subsection*{Question 2(a) [3 marks]}\label{q2a}

\textbf{Write methods of cascading amplifiers}

\begin{solutionbox}
Cascading amplifiers means connecting multiple
amplifier stages in series to increase overall gain.


{\def\LTcaptype{none} % do not increment counter
\vspace{-5pt}
\captionof{table}{Methods of Cascading Amplifiers}
\vspace{-10pt}
\begin{longtable}[]{@{}
  >{\raggedright\arraybackslash}p{(\linewidth - 2\tabcolsep) * \real{0.3810}}
  >{\raggedright\arraybackslash}p{(\linewidth - 2\tabcolsep) * \real{0.6190}}@{}}
\toprule\noalign{}
\begin{minipage}[b]{\linewidth}\raggedright
Method
\end{minipage} & \begin{minipage}[b]{\linewidth}\raggedright
Key Feature
\end{minipage} \\
\midrule\noalign{}
\endhead
\bottomrule\noalign{}
\endlastfoot
RC Coupling & Uses capacitor and resistor for interstage coupling \\
Transformer Coupling & Uses transformer for impedance matching and
isolation \\
Direct Coupling & No coupling components, direct connection between
stages \\
LC Coupling & Uses inductor-capacitor for high-frequency applications \\
\end{longtable}
}

\end{solutionbox}
\begin{mnemonicbox}
``RTDL connection'' (RC, Transformer, Direct, LC)

\end{mnemonicbox}
\subsection*{Question 2(b) [4 marks]}\label{q2b}

\textbf{Compare CE and CB amplifiers}

\begin{solutionbox}


{\def\LTcaptype{none} % do not increment counter
\vspace{-5pt}
\captionof{table}{Comparison of CE and CB Amplifiers}
\vspace{-10pt}
\begin{longtable}[]{@{}lll@{}}
\toprule\noalign{}
Parameter & Common Emitter (CE) & Common Base (CB) \\
\midrule\noalign{}
\endhead
\bottomrule\noalign{}
\endlastfoot
Input Impedance & Medium (\approx1kΩ) & Low (\approx50Ω) \\
Output Impedance & High (\approx50kΩ) & Very high (\approx500kΩ) \\
Voltage Gain & High (\approx500) & High (\approx500) \\
Current Gain & Medium (β) & Less than 1 (α) \\
Phase Shift & 180^\circ & 0^\circ \\
Applications & Voltage amplification & High-frequency amplification \\
\end{longtable}
}

\end{solutionbox}
\begin{mnemonicbox}
``PIVOT differences'' (Phase shift, Impedance,
Voltage gain, Output impedance, Throughput)

\end{mnemonicbox}
\subsection*{Question 2(c) [7 marks]}\label{q2c}

\textbf{Draw the circuit of RC coupled amplifier. Give the frequency
response and explain}

\begin{solutionbox}
RC coupled amplifier uses resistor-capacitor network
for interstage coupling.

\textbf{Diagram:}

\begin{verbatim}
     +Vcc
       |
       +{-{-}{-}+{-}{-}{-}{-}{-}{-}{-}{-}+{-}{-}{-}+}
       |   |        |   |
       Rc1 |        Rc2 |
       |   |        |   |
       +{-{-}{-}+        +{-}{-}{-}+}
       |            |
       C            +{-{-}{-}+ Output}
       |            |   |
   +{-{-}{-}+{-}{-}{-}+    +{-}{-}{-}+{-}{-}{-}+}
   |   |   |    |   |   |
   | B |   |    | B |   |
   | | C   |    | | C   |
   +{-+ |   |    +{-}+ |   |}
   |   |   |    |   |   |
Input   |   |    |   |   |
   +{-{-}{-}+{-}{-}{-}+    +{-}{-}{-}+{-}{-}{-}+}
       |            |
       Re1          Re2
       |            |
       +{-{-}{-}{-}{-}{-}{-}{-}{-}{-}{-}{-}+}
       |
      GND
\end{verbatim}

\textbf{Frequency Response:}

\begin{center}
\textbf{Mermaid Diagram (Code)}
\begin{verbatim}
{Shaded}
{Highlighting}[]
graph LR
    A[Low Frequency] {-{-}{-} B[Mid Frequency]}
    B {-{-}{-} C[High Frequency]}

    style A fill:\#f9f,stroke:\#333,stroke{-width:1px}
    style B fill:\#bbf,stroke:\#333,stroke{-width:1px}
    style C fill:\#f9f,stroke:\#333,stroke{-width:1px}
{Highlighting}
{Shaded}
\end{verbatim}
\end{center}

\begin{itemize}
\tightlist
\item
  \textbf{Low frequency region}: Gain drops due to coupling and bypass
  capacitors
\item
  \textbf{Mid frequency region}: Flat response with maximum gain
\item
  \textbf{High frequency region}: Gain falls due to transistor internal
  capacitances
\item
  \textbf{Bandwidth}: Determined by the lower and upper cutoff
  frequencies
\end{itemize}

\end{solutionbox}
\begin{mnemonicbox}
``LMH regions'' (Low, Mid, High frequency regions)

\end{mnemonicbox}
\subsection*{Question 2(a) OR [3
marks]}\label{q2a}

\textbf{Write definition of gain, Bandwidth and Gain Bandwidth product
of an amplifier.}

\begin{solutionbox}


{\def\LTcaptype{none} % do not increment counter
\vspace{-5pt}
\captionof{table}{Key Amplifier Parameters}
\vspace{-10pt}
\begin{longtable}[]{@{}
  >{\raggedright\arraybackslash}p{(\linewidth - 2\tabcolsep) * \real{0.4783}}
  >{\raggedright\arraybackslash}p{(\linewidth - 2\tabcolsep) * \real{0.5217}}@{}}
\toprule\noalign{}
\begin{minipage}[b]{\linewidth}\raggedright
Parameter
\end{minipage} & \begin{minipage}[b]{\linewidth}\raggedright
Definition
\end{minipage} \\
\midrule\noalign{}
\endhead
\bottomrule\noalign{}
\endlastfoot
Gain (A) & Ratio of output signal to input signal (voltage, current, or
power) \\
Bandwidth (BW) & Frequency range between lower and upper cutoff
frequencies (f_{2}-f_{1}) \\
Gain-Bandwidth Product (GBW) & Product of gain and bandwidth, remains
constant for a given amplifier \\
\end{longtable}
}

\end{solutionbox}
\begin{mnemonicbox}
``GBP constants'' (Gain, Bandwidth, Product
constants)

\end{mnemonicbox}
\subsection*{Question 2(b) OR [4
marks]}\label{q2b}

\textbf{Explain frequency response of single stage amplifier and
indicate its cutoff frequencies.}

\begin{solutionbox}
Frequency response shows variation of gain with
frequency in a single stage amplifier.

\textbf{Diagram:}

\begin{center}
\textbf{Mermaid Diagram (Code)}
\begin{verbatim}
{Shaded}
{Highlighting}[]
graph TD
    A[Frequency Response] {-{-}{} B[Low f Region]}
    A {-{-}{} C[Mid f Region]}
    A {-{-}{} D[High f Region]}
    B {-{-}{} E[f_{1}: Lower Cutoff]}
    D {-{-}{} F[f_{2}: Upper Cutoff]}
    C {-{-}{} G[Maximum Gain]}

    style A fill:\#bbf,stroke:\#333,stroke{-width:1px}
    style B fill:\#f9f,stroke:\#333,stroke{-width:1px}
    style C fill:\#bfb,stroke:\#333,stroke{-width:1px}
    style D fill:\#f9f,stroke:\#333,stroke{-width:1px}
{Highlighting}
{Shaded}
\end{verbatim}
\end{center}

\begin{itemize}
\tightlist
\item
  \textbf{Cutoff frequencies}: Points where gain drops to 0.707 times
  maximum gain
\item
  \textbf{Lower cutoff frequency (f_{1})}: Determined by coupling and
  bypass capacitors
\item
  \textbf{Upper cutoff frequency (f_{2})}: Limited by transistor junction
  capacitances
\item
  \textbf{Bandwidth}: Frequency range between f_{1} and f_{2} (BW = f_{2} - f_{1})
\end{itemize}

\end{solutionbox}
\begin{mnemonicbox}
``LUG points'' (Lower cutoff, Upper cutoff, Gain
maximum)

\end{mnemonicbox}
\subsection*{Question 2(c) OR [7
marks]}\label{q2c}

\textbf{Draw and Explain circuit diagram of common collector amplifier}

\begin{solutionbox}
Common collector (CC) amplifier is also known as
emitter follower.

\textbf{Diagram:}

\begin{verbatim}
     +Vcc
       |
       Rc
       |
       +
       |
       |    C
       |    |
   +{-{-}{-}+    |}
   |   |    |
   | B |    |
Input{-+ |    |}
   |   |    |
   |   |    |
   |   C    |
   |   |    |
   +{-{-}{-}+{-}{-}{-}{-}+{-}{-}+ Output}
       |    |
       Re   |
       |    |
       +{-{-}{-}{-}+}
       |
      GND
\end{verbatim}


{\def\LTcaptype{none} % do not increment counter
\vspace{-5pt}
\captionof{table}{Features of Common Collector Amplifier}
\vspace{-10pt}
\begin{longtable}[]{@{}ll@{}}
\toprule\noalign{}
Parameter & Characteristic \\
\midrule\noalign{}
\endhead
\bottomrule\noalign{}
\endlastfoot
Voltage Gain & Approximately 1 (less than 1) \\
Current Gain & High (β) \\
Input Impedance & Very high (\approx β \times Re) \\
Output Impedance & Very low (\approx 1/gm) \\
Phase Shift & 0^\circ (no phase inversion) \\
Applications & Impedance matching, buffer stages \\
\end{longtable}
}

\begin{itemize}
\tightlist
\item
  \textbf{Working principle}: Output is taken from emitter, collector is
  common to input and output
\item
  \textbf{Key feature}: Voltage follower with output voltage following
  input voltage
\item
  \textbf{Main advantage}: High input impedance and low output impedance
\end{itemize}

\end{solutionbox}
\begin{mnemonicbox}
``BIVOP characters'' (Buffer, Impedance matching,
Voltage follower, One gain, Phase matched)

\end{mnemonicbox}
\subsection*{Question 3(a) [3 marks]}\label{q3a}

\textbf{Draw transistor two port network and describe h-parameters for
it.}

\begin{solutionbox}
Transistor can be represented as a two-port network
with h-parameters.

\textbf{Diagram:}

\begin{verbatim}
     +{-{-}{-}{-}{-}{-}{-}{-}{-}{-}{-}{-}{-}{-}{-}{-}{-}{-}{-}{-}{-}+}
     |                     |
     |    Two{-Port         |}
Input|    Network     +{-{-}{-}{-}+{-}{-}{-} Output}
   +{|                |    |}
     |    Transistor  |    |
     |                v    |
     +{-{-}{-}{-}{-}{-}{-}{-}{-}{-}{-}{-}{-}{-}{-}{-}{-}{-}{-}{-}{-}+}
\end{verbatim}


{\def\LTcaptype{none} % do not increment counter
\vspace{-5pt}
\captionof{table}{h-parameters}
\vspace{-10pt}
\begin{longtable}[]{@{}
  >{\raggedright\arraybackslash}p{(\linewidth - 2\tabcolsep) * \real{0.4783}}
  >{\raggedright\arraybackslash}p{(\linewidth - 2\tabcolsep) * \real{0.5217}}@{}}
\toprule\noalign{}
\begin{minipage}[b]{\linewidth}\raggedright
Parameter
\end{minipage} & \begin{minipage}[b]{\linewidth}\raggedright
Description
\end{minipage} \\
\midrule\noalign{}
\endhead
\bottomrule\noalign{}
\endlastfoot
h_{1}_{1} (h\_i) & Input impedance with output short-circuited \\
h_{1}_{2} (h\_r) & Reverse voltage transfer ratio with input open-circuited \\
h_{2}_{1} (h\_f) & Forward current transfer ratio with output
short-circuited \\
h_{2}_{2} (h\_o) & Output admittance with input open-circuited \\
\end{longtable}
}

\end{solutionbox}
\begin{mnemonicbox}
``IRFO parameters'' (Input impedance, Reverse
transfer, Forward transfer, Output admittance)

\end{mnemonicbox}
\subsection*{Question 3(b) [4 marks]}\label{q3b}

\textbf{Explain voltage gain Av, current gain Ai and Power gain Ap for
CE amplifier}

\begin{solutionbox}


{\def\LTcaptype{none} % do not increment counter
\vspace{-5pt}
\captionof{table}{Gain Expressions for CE Amplifier}
\vspace{-10pt}
\begin{longtable}[]{@{}
  >{\raggedright\arraybackslash}p{(\linewidth - 4\tabcolsep) * \real{0.2245}}
  >{\raggedright\arraybackslash}p{(\linewidth - 4\tabcolsep) * \real{0.2449}}
  >{\raggedright\arraybackslash}p{(\linewidth - 4\tabcolsep) * \real{0.5306}}@{}}
\toprule\noalign{}
\begin{minipage}[b]{\linewidth}\raggedright
Gain Type
\end{minipage} & \begin{minipage}[b]{\linewidth}\raggedright
Expression
\end{minipage} & \begin{minipage}[b]{\linewidth}\raggedright
Relation to h-parameters
\end{minipage} \\
\midrule\noalign{}
\endhead
\bottomrule\noalign{}
\endlastfoot
Voltage Gain (Av) & V_{o}/Vᵢ & Av = -h\_fe \times R\_L / h\_ie \\
Current Gain (Ai) & I_{o}/Iᵢ & Ai = h\_fe / (1 + h\_oe \times R\_L) \\
Power Gain (Ap) & P_{o}/Pᵢ & Ap = Av \times Ai = (voltage gain \times current
gain) \\
\end{longtable}
}

\begin{itemize}
\tightlist
\item
  \textbf{Voltage gain}: Typically 500-1000 for CE amplifier
\item
  \textbf{Current gain}: Approximately equal to h\_fe (β) of transistor
\item
  \textbf{Power gain}: Product of voltage gain and current gain
\end{itemize}

\end{solutionbox}
\begin{mnemonicbox}
``VIP gains'' (Voltage, Input-output current, Power)

\end{mnemonicbox}
\subsection*{Question 3(c) [7 marks]}\label{q3c}

\textbf{Explain Darlington pair, its features and applications}

\begin{solutionbox}
Darlington pair consists of two transistors connected
to act as a single high-gain transistor.

\textbf{Diagram:}

\begin{verbatim}
     +Vcc
       |
       Rc
       |
       +{-{-}{-}{-}{-}{-}{-}{-}+}
       |        |
       |        | Output
       |        |
       | +{-{-}{-}{-}{-}{-}+}
       | |
       | C
  +{-{-}{-}{-}+{-}+}
  |    | |
  |  +{-+ |}
  |  |   |
  |  | C |
  |  | | |
Input  | | |
  +{-{-}{-}{-}+{-}+ |}
       |   |
       +{-{-}{-}+}
       |
      GND
\end{verbatim}


{\def\LTcaptype{none} % do not increment counter
\vspace{-5pt}
\captionof{table}{Features of Darlington Pair}
\vspace{-10pt}
\begin{longtable}[]{@{}ll@{}}
\toprule\noalign{}
Feature & Description \\
\midrule\noalign{}
\endhead
\bottomrule\noalign{}
\endlastfoot
Current Gain & Very high (β_{1} \times β_{2}) \\
Input Impedance & Extremely high \\
Voltage Drop & Higher (\approx1.4V) due to two B-E junctions \\
Switching Speed & Slower than single transistor \\
Thermal Stability & Poorer than single transistor \\
\end{longtable}
}

\begin{itemize}
\tightlist
\item
  \textbf{Applications}: Power amplifiers, motor drivers, touch
  switches, sensors
\item
  \textbf{Advantages}: Very high current gain, high input impedance
\item
  \textbf{Limitations}: Higher saturation voltage, slower switching
\end{itemize}

\end{solutionbox}
\begin{mnemonicbox}
``CHIPS application'' (Current amplification, High
impedance, Increased gain, Power handling, Slower switching)

\end{mnemonicbox}
\subsection*{Question 3(a) OR [3
marks]}\label{q3a}

\textbf{Discuss applications of LDR.}

\begin{solutionbox}
Light Dependent Resistor (LDR) is a photoresistor whose
resistance decreases with increasing light intensity.


{\def\LTcaptype{none} % do not increment counter
\vspace{-5pt}
\captionof{table}{Applications of LDR}
\vspace{-10pt}
\begin{longtable}[]{@{}
  >{\raggedright\arraybackslash}p{(\linewidth - 2\tabcolsep) * \real{0.4062}}
  >{\raggedright\arraybackslash}p{(\linewidth - 2\tabcolsep) * \real{0.5938}}@{}}
\toprule\noalign{}
\begin{minipage}[b]{\linewidth}\raggedright
Application
\end{minipage} & \begin{minipage}[b]{\linewidth}\raggedright
Working Principle
\end{minipage} \\
\midrule\noalign{}
\endhead
\bottomrule\noalign{}
\endlastfoot
Automatic Street Lights & Turns on lights when ambient light level
falls \\
Camera Exposure Control & Adjusts aperture/shutter based on light
intensity \\
Light Beam Alarms & Triggers alarm when light beam is interrupted \\
Solar Trackers & Helps orient solar panels toward maximum sunlight \\
Automatic Brightness Control & Adjusts display brightness based on
ambient light \\
\end{longtable}
}

\end{solutionbox}
\begin{mnemonicbox}
``CASAL applications'' (Camera, Alarm, Street light,
Automatic control, Light measurement)

\end{mnemonicbox}
\subsection*{Question 3(b) OR [4
marks]}\label{q3b}

\textbf{Comparison of clipper and clamper}

\begin{solutionbox}


{\def\LTcaptype{none} % do not increment counter
\vspace{-5pt}
\captionof{table}{Comparison between Clipper and Clamper}
\vspace{-10pt}
\begin{longtable}[]{@{}
  >{\raggedright\arraybackslash}p{(\linewidth - 4\tabcolsep) * \real{0.3793}}
  >{\raggedright\arraybackslash}p{(\linewidth - 4\tabcolsep) * \real{0.3103}}
  >{\raggedright\arraybackslash}p{(\linewidth - 4\tabcolsep) * \real{0.3103}}@{}}
\toprule\noalign{}
\begin{minipage}[b]{\linewidth}\raggedright
Parameter
\end{minipage} & \begin{minipage}[b]{\linewidth}\raggedright
Clipper
\end{minipage} & \begin{minipage}[b]{\linewidth}\raggedright
Clamper
\end{minipage} \\
\midrule\noalign{}
\endhead
\bottomrule\noalign{}
\endlastfoot
Function & Limits/clips signal amplitude & Shifts DC level of signal \\
Output & Removes portions beyond threshold & Adds DC component \\
Components & Diode + Resistor & Diode + Capacitor + Resistor \\
Wave Shape & Changes wave shape & Preserves wave shape \\
Applications & Noise removal, wave shaping & TV signal processing, DC
restoration \\
\end{longtable}
}

\textbf{Diagram:}

\begin{center}
\textbf{Mermaid Diagram (Code)}
\begin{verbatim}
{Shaded}
{Highlighting}[]
graph LR
    A[Input Signal] {-{-}{} B[Clipper]}
    A {-{-}{} C[Clamper]}
    B {-{-}{} D[Amplitude Limited]}
    C {-{-}{} E[DC Level Shifted]}

    style A fill:\#bbf,stroke:\#333,stroke{-width:1px}
    style B fill:\#f9f,stroke:\#333,stroke{-width:1px}
    style C fill:\#bfb,stroke:\#333,stroke{-width:1px}
    style D fill:\#f9f,stroke:\#333,stroke{-width:1px}
    style E fill:\#bfb,stroke:\#333,stroke{-width:1px}
{Highlighting}
{Shaded}
\end{verbatim}
\end{center}

\end{solutionbox}
\begin{mnemonicbox}
``CLIPS vs CLAMPS'' (Cut Levels In Peak Signal vs
Change Level And Maintain Peak Shape)

\end{mnemonicbox}
\subsection*{Question 3(c) OR [7
marks]}\label{q3c}

\textbf{Describe h-parameters circuit for CE amplifier.}

\begin{solutionbox}
h-parameters provide a simple way to analyze CE
amplifier performance.

\textbf{Diagram:}

\begin{verbatim}
     +{-{-}{-}{-}{-}{-}{-}{-}{-}{-}{-}{-}{-}{-}{-}{-}{-}{-}{-}{-}{-}+}
     |                     |
  Ii |    +{-{-}{-}{-}{-}{-}{-}{-}{-}{-}+     | Io}
   +{|    |          |     |+}
     |    |   h\_ie   |     |
  Vi |    +{-{-}{-}{-}{-}{-}{-}{-}{-}{-}+     | Vo}
   +{|    |          |     |+}
     |    | h\_re.Vi  |     |
     |    |          |     |
     |    |  h\_fe.Ii |{-{-}{-}{-}|}
     |    |          |     |
     |    |   h\_oe   |     |
     |    +{-{-}{-}{-}{-}{-}{-}{-}{-}{-}+     |}
     |                     |
     +{-{-}{-}{-}{-}{-}{-}{-}{-}{-}{-}{-}{-}{-}{-}{-}{-}{-}{-}{-}{-}+}
\end{verbatim}


{\def\LTcaptype{none} % do not increment counter
\vspace{-5pt}
\captionof{table}{h-parameters for CE Configuration}
\vspace{-10pt}
\begin{longtable}[]{@{}llll@{}}
\toprule\noalign{}
Parameter & Symbol & Typical Value & Physical Significance \\
\midrule\noalign{}
\endhead
\bottomrule\noalign{}
\endlastfoot
Input impedance & h\_ie & 1-2 kΩ & Base-emitter input impedance \\
Reverse voltage ratio & h\_re & 10^{-}^{4} & Feedback from output to input \\
Forward current gain & h\_fe & 50-300 & Current gain (β) \\
Output admittance & h\_oe & 10^{-}^{6} S & Output conductance \\
\end{longtable}
}

\begin{itemize}
\tightlist
\item
  \textbf{Circuit analysis}: Uses h-parameters to calculate voltage
  gain, current gain, input/output impedance
\item
  \textbf{Equivalent circuit}: Combines h-parameters in a two-port
  network representation
\item
  \textbf{Advantage}: Simplifies complex transistor behavior into linear
  parameters
\end{itemize}

\end{solutionbox}
\begin{mnemonicbox}
``FIRO parameters'' (Forward gain, Input impedance,
Reverse feedback, Output admittance)

\end{mnemonicbox}
\subsection*{Question 4(a) [3 marks]}\label{q4a}

\textbf{Write short note on Darlington pair.}

\begin{solutionbox}
Darlington pair combines two transistors to create a
super-high gain transistor.

\textbf{Diagram:}

\begin{verbatim}
       +{-{-}{-}+}
       |   |
  +{-{-}{-}{-}+{-}{-}{-}+{-}{-}{-}{-}+}
  |    |        |
  |    +        |
  |  E/ {C      |}
  |   /B {      |}
Input + {-{-}+     |}
  |    |        |
  |    +        |
  |  E/ {C      |}
  |   /B {      |}
  |    |        |
  +{-{-}{-}{-}+{-}{-}{-}{-}{-}{-}{-}{-}+}
       |
     Output
\end{verbatim}

\begin{itemize}
\tightlist
\item
  \textbf{Configuration}: Two transistors where first transistor's
  emitter drives second transistor's base
\item
  \textbf{Total gain}: β_{1} \times β_{2} (product of individual transistor gains)
\item
  \textbf{Input impedance}: Extremely high (β_{2} \times R\_e1)
\end{itemize}

\end{solutionbox}
\begin{mnemonicbox}
``HIS properties'' (High gain, Impedance boost,
Sandwich configuration)

\end{mnemonicbox}
\subsection*{Question 4(b) [4 marks]}\label{q4b}

\textbf{Explain Zener diode as a voltage regulator.}

\begin{solutionbox}
Zener diode provides a constant voltage reference when
operated in reverse breakdown.

\textbf{Diagram:}

\begin{verbatim}
       Rs
    +{-{-}www{-}{-}+}
    |        |
 Vi |      | | Vz    RL   Vo
 +{-{-}+      | +{-}{-}{-}+{-}{-}{-}www{-}{-}+}
    |      |/|   |        |
    |      |{|   |        |}
    |        |   |        |
    +{-{-}{-}{-}{-}{-}{-}{-}+{-}{-}{-}+{-}{-}{-}{-}{-}{-}{-}{-}+}
             |
            GND
\end{verbatim}


{\def\LTcaptype{none} % do not increment counter
\vspace{-5pt}
\captionof{table}{Zener Voltage Regulator}
\vspace{-10pt}
\begin{longtable}[]{@{}
  >{\raggedright\arraybackslash}p{(\linewidth - 2\tabcolsep) * \real{0.4783}}
  >{\raggedright\arraybackslash}p{(\linewidth - 2\tabcolsep) * \real{0.5217}}@{}}
\toprule\noalign{}
\begin{minipage}[b]{\linewidth}\raggedright
Parameter
\end{minipage} & \begin{minipage}[b]{\linewidth}\raggedright
Description
\end{minipage} \\
\midrule\noalign{}
\endhead
\bottomrule\noalign{}
\endlastfoot
Principle & Maintains constant voltage in reverse breakdown region \\
Series Resistor (Rs) & Limits current and drops excess voltage \\
Load Resistor (RL) & Represents the circuit being powered \\
Regulation & Maintains constant output despite input voltage
fluctuations \\
\end{longtable}
}

\begin{itemize}
\tightlist
\item
  \textbf{Working}: Zener operates in breakdown region, maintaining
  fixed voltage
\item
  \textbf{Limitation}: Power dissipation capability limits maximum
  current
\end{itemize}

\end{solutionbox}
\begin{mnemonicbox}
``ZEBRA'' (Zener Effect Breakdown Regulates
Accurately)

\end{mnemonicbox}
\subsection*{Question 4(c) [7 marks]}\label{q4c}

\textbf{Explain Optocoupler with advantages and disadvantages.}

\begin{solutionbox}
Optocoupler (also called optoisolator) uses light to
transfer signals between isolated circuits.

\textbf{Diagram:}

\begin{verbatim}
   +{-{-}{-}{-}{-}{-}{-}{-}+{-}{-}{-}{-}{-}{-}{-}{-}{-}{-}{-}+}
   |        |           |
   |    +{-{-}{-}+{-}{-}{-}+       |}
   |    |       |       |
   |    |  LED  |       |
   |    |       |       |
   |    +{-{-}{-}+{-}{-}{-}+       |}
Input   |   |           | Output
   |    |   |     +{-{-}{-}{-}{-}+{-}{-}{-}{-}+}
   |    |   |     |     |    |
   |    |   +{-{-}{-}{-}|     |    |}
   |    |         |Photo|    |
   |    |         |sensor|   |
   |    |         |     |    |
   |    |         +{-{-}{-}{-}{-}+{-}{-}{-}{-}+}
   |    |           |        |
   +{-{-}{-}{-}+{-}{-}{-}{-}{-}{-}{-}{-}{-}{-}{-}+{-}{-}{-}{-}{-}{-}{-}{-}+}
\end{verbatim}


{\def\LTcaptype{none} % do not increment counter
\vspace{-5pt}
\captionof{table}{Advantages and Disadvantages of Optocoupler}
\vspace{-10pt}
\begin{longtable}[]{@{}ll@{}}
\toprule\noalign{}
Advantages & Disadvantages \\
\midrule\noalign{}
\endhead
\bottomrule\noalign{}
\endlastfoot
Complete electrical isolation & Relatively slow response time \\
High noise immunity & Limited bandwidth \\
No ground loops & Temperature sensitive \\
High voltage isolation & Aging effects \\
Protection against transients & Requires current to drive LED \\
\end{longtable}
}

\begin{itemize}
\tightlist
\item
  \textbf{Working}: Input signal drives LED, which emits light detected
  by photodetector
\item
  \textbf{Applications}: Medical equipment, industrial control, power
  supplies, signal isolation
\item
  \textbf{Types}: Photoresistor, photodiode, phototransistor, photo-SCR
  based
\end{itemize}

\end{solutionbox}
\begin{mnemonicbox}
``LIGHT transfer'' (Linked Isolated Galvanic-free
High-voltage Transfer)

\end{mnemonicbox}
\subsection*{Question 4(a) OR [3
marks]}\label{q4a}

\textbf{Draw Half Wave Voltage Doubler.}

\begin{solutionbox}
Half-wave voltage doubler uses diodes and capacitors to
produce DC output approximately twice the peak input voltage.

\textbf{Diagram:}

\begin{verbatim}
              D1
     +{-{-}{-}{-}{-}{-}{-}{-}+{-}{-}||{-}{-}{-}{-}{-}{-}{-}{-}{-}{-}+}
     |                        |
     |                        |
     |                        |
    \_|\_                      \_|\_
    { / D2                   {-}{-}{-} C2}
     |                        |
     |                        |
 Vin |        +{-{-}{-}{-}{-}{-}{-}{-}{-}{-}{-}{-}{-}{-}{-}+{-}{-}+ Vout(2Vin)}
     |        |
     |        |
     |       \_|\_
     |       {-{-}{-} C1}
     |        |
     +{-{-}{-}{-}{-}{-}{-}{-}+{-}{-}{-}{-}{-}{-}{-}{-}{-}{-}{-}{-}{-}{-}{-}{-}+}
\end{verbatim}

\begin{itemize}
\tightlist
\item
  \textbf{Components}: Two diodes and two capacitors
\item
  \textbf{Output}: Approximately twice the peak input voltage
\end{itemize}

\end{solutionbox}
\begin{mnemonicbox}
``DC2'' (Doubles input using Capacitors and 2 Diodes)

\end{mnemonicbox}
\subsection*{Question 4(b) OR [4
marks]}\label{q4b}

\textbf{Explain the working and applications of OLED.}

\begin{solutionbox}
Organic Light Emitting Diode (OLED) uses organic
compounds that emit light when current flows through them.

\textbf{Diagram:}

\begin{center}
\textbf{Mermaid Diagram (Code)}
\begin{verbatim}
{Shaded}
{Highlighting}[]
graph TD
    A[OLED Structure] {-{-}{} B[Cathode]}
    A {-{-}{} C[Organic Layer]}
    A {-{-}{} D[Anode]}
    A {-{-}{} E[Substrate]}

    style A fill:\#bbf,stroke:\#333,stroke{-width:1px}
    style B fill:\#f9f,stroke:\#333,stroke{-width:1px}
    style C fill:\#bfb,stroke:\#333,stroke{-width:1px}
    style D fill:\#f9f,stroke:\#333,stroke{-width:1px}
    style E fill:\#bfb,stroke:\#333,stroke{-width:1px}
{Highlighting}
{Shaded}
\end{verbatim}
\end{center}


{\def\LTcaptype{none} % do not increment counter
\vspace{-5pt}
\captionof{table}{Working and Applications of OLED}
\vspace{-10pt}
\begin{longtable}[]{@{}
  >{\raggedright\arraybackslash}p{(\linewidth - 2\tabcolsep) * \real{0.4000}}
  >{\raggedright\arraybackslash}p{(\linewidth - 2\tabcolsep) * \real{0.6000}}@{}}
\toprule\noalign{}
\begin{minipage}[b]{\linewidth}\raggedright
Aspect
\end{minipage} & \begin{minipage}[b]{\linewidth}\raggedright
Description
\end{minipage} \\
\midrule\noalign{}
\endhead
\bottomrule\noalign{}
\endlastfoot
Working & Electron-hole recombination in organic layer produces light \\
Efficiency & High efficiency, low power consumption \\
Viewing Angle & Excellent (nearly 180^\circ) \\
Applications & Smartphones, TVs, wearable devices, lighting \\
Advantages & Thin, flexible, better contrast, faster response \\
\end{longtable}
}

\end{solutionbox}
\begin{mnemonicbox}
``VIEWS technology'' (Vibrant colors, Incredible
contrast, Excellent angle, Wide application, Self-emitting)

\end{mnemonicbox}
\subsection*{Question 4(c) OR [7
marks]}\label{q4c}

\textbf{Explain working of solar battery charger circuits.}

\begin{solutionbox}
Solar battery charger converts solar energy to
electrical energy to charge batteries.

\textbf{Diagram:}

\begin{verbatim}
   +{-{-}{-}{-}{-}{-}{-}{-}+        +{-}{-}{-}{-}{-}{-}{-}{-}{-}{-}+        +{-}{-}{-}{-}{-}{-}{-}{-}{-}+        +{-}{-}{-}{-}{-}{-}{-}{-}+}
   |        |        |          |        |         |        |        |
   | Solar  |        | Charge   |        | Voltage |        | Battery|
   | Panel  |{-{-}{-}{-}{-}{-}{-}|Controller|{-}{-}{-}{-}{-}{-}{-}|Regulator|{-}{-}{-}{-}{-}{-}{-}|        |}
   |        |        |          |        |         |        |        |
   +{-{-}{-}{-}{-}{-}{-}{-}+        +{-}{-}{-}{-}{-}{-}{-}{-}{-}{-}+        +{-}{-}{-}{-}{-}{-}{-}{-}{-}+        +{-}{-}{-}{-}{-}{-}{-}{-}+}
                         |                                      |
                         |                                      |
                         v                                      v
                     +{-{-}{-}{-}{-}{-}{-}{-}{-}+                             +{-}{-}{-}{-}{-}{-}{-}{-}+}
                     |Indicator|                             |  Load  |
                     | Circuit |                             |        |
                     +{-{-}{-}{-}{-}{-}{-}{-}{-}+                             +{-}{-}{-}{-}{-}{-}{-}{-}+}
\end{verbatim}


{\def\LTcaptype{none} % do not increment counter
\vspace{-5pt}
\captionof{table}{Components and Their Functions}
\vspace{-10pt}
\begin{longtable}[]{@{}ll@{}}
\toprule\noalign{}
Component & Function \\
\midrule\noalign{}
\endhead
\bottomrule\noalign{}
\endlastfoot
Solar Panel & Converts sunlight to DC electricity \\
Charge Controller & Prevents overcharging and deep discharge \\
Voltage Regulator & Stabilizes voltage to appropriate charging level \\
Battery & Stores electrical energy \\
Indicator Circuit & Shows charging status and battery level \\
\end{longtable}
}

\begin{itemize}
\tightlist
\item
  \textbf{Working principle}: Photovoltaic effect converts sunlight to
  electricity
\item
  \textbf{Regulation}: Prevents overcharging using voltage/current
  regulation
\item
  \textbf{Protection}: Includes reverse current protection to prevent
  battery discharge at night
\item
  \textbf{Types}: PWM (Pulse Width Modulation) and MPPT (Maximum Power
  Point Tracking)
\end{itemize}

\end{solutionbox}
\begin{mnemonicbox}
``SCORE system'' (Solar Conversion, Overcharge
protection, Regulation, Energy storage)

\end{mnemonicbox}
\subsection*{Question 5(a) [3 marks]}\label{q5a}

\textbf{Draw a block diagram of regulated power supply.}

\begin{solutionbox}
Regulated power supply provides stable DC output
voltage despite variations in input or load.

\textbf{Diagram:}

\begin{center}
\textbf{Mermaid Diagram (Code)}
\begin{verbatim}
{Shaded}
{Highlighting}[]
graph LR
    A[Transformer] {-{-}{} B[Rectifier]}
    B {-{-}{} C[Filter]}
    C {-{-}{} D[Voltage Regulator]}
    D {-{-}{} E[Output]}

    style A fill:\#f9f,stroke:\#333,stroke{-width:1px}
    style B fill:\#bbf,stroke:\#333,stroke{-width:1px}
    style C fill:\#bfb,stroke:\#333,stroke{-width:1px}
    style D fill:\#f9f,stroke:\#333,stroke{-width:1px}
    style E fill:\#bbf,stroke:\#333,stroke{-width:1px}
{Highlighting}
{Shaded}
\end{verbatim}
\end{center}

\begin{itemize}
\tightlist
\item
  \textbf{Components}: Transformer, rectifier, filter, voltage regulator
\item
  \textbf{Function}: Converts AC to stable DC regardless of load changes
\end{itemize}

\end{solutionbox}
\begin{mnemonicbox}
``TRFO blocks'' (Transformer, Rectifier, Filter,
Output regulator)

\end{mnemonicbox}
\subsection*{Question 5(b) [4 marks]}\label{q5b}

\textbf{Describe Transistor shunt Voltage Regulator.}

\begin{solutionbox}
Transistor shunt regulator maintains constant output
voltage by diverting excess current through a transistor in parallel
with the load.

\textbf{Diagram:}

\begin{verbatim}
     +{-{-}{-}{-}{-}{-}{-}+}
     |       |
     |      \_|\_
     |      { / Zener}
     |       |
     |       |
     +{-{-}{-}+{-}{-}{-}+}
         |
         |   +{-{-}{-}{-}{-}{-}{-}{-}{-}{-}{-}+}
         +{-{-}{-}| Base      |}
             |           |
  Vin    Rs  | Transistor|  RL   Vout
  +{-{-}{-}+{-}{-}www{-}+           +{-}{-}{-}www{-}{-}{-}+}
             | Collector |         |
             |           |         |
             +{-{-}{-}{-}{-}{-}{-}{-}{-}{-}{-}+         |}
                 |                 |
                 +{-{-}{-}{-}{-}{-}{-}{-}{-}{-}{-}{-}{-}{-}{-}{-}{-}+}
                 |
                GND
\end{verbatim}


{\def\LTcaptype{none} % do not increment counter
\vspace{-5pt}
\captionof{table}{Transistor Shunt Regulator}
\vspace{-10pt}
\begin{longtable}[]{@{}ll@{}}
\toprule\noalign{}
Component & Function \\
\midrule\noalign{}
\endhead
\bottomrule\noalign{}
\endlastfoot
Zener & Provides reference voltage \\
Transistor & Shunts excess current \\
Series Resistor (Rs) & Drops excess voltage \\
Load Resistor (RL) & Represents circuit being powered \\
\end{longtable}
}

\begin{itemize}
\tightlist
\item
  \textbf{Working}: Transistor conducts more when output tries to
  increase
\item
  \textbf{Advantage}: Simple circuit with good regulation
\end{itemize}

\end{solutionbox}
\begin{mnemonicbox}
``ZEST circuit'' (Zener reference, Excess current,
Shunt transistor, Tension-free output)

\end{mnemonicbox}
\subsection*{Question 5(c) [7 marks]}\label{q5c}

\textbf{Draw and explain SMPS block diagram with its advantages and
disadvantages.}

\begin{solutionbox}
Switched Mode Power Supply (SMPS) uses switching
regulation for high efficiency.

\textbf{Diagram:}

\begin{center}
\textbf{Mermaid Diagram (Code)}
\begin{verbatim}
{Shaded}
{Highlighting}[]
graph LR
    A[AC Input] {-{-}{} B[EMI Filter]}
    B {-{-}{} C[Rectifier \& Filter]}
    C {-{-}{} D[Switching Circuit]}
    D {-{-}{} E[Transformer]}
    E {-{-}{} F[Output Rectifier]}
    F {-{-}{} G[Output Filter]}
    G {-{-}{} H[DC Output]}
    I[Feedback \& Control] {-{-}{} D}
    H {-{-}{} I}

    style A fill:\#f9f,stroke:\#333,stroke{-width:1px}
    style D fill:\#bbf,stroke:\#333,stroke{-width:1px}
    style E fill:\#bfb,stroke:\#333,stroke{-width:1px}
    style H fill:\#f9f,stroke:\#333,stroke{-width:1px}
    style I fill:\#bbf,stroke:\#333,stroke{-width:1px}
{Highlighting}
{Shaded}
\end{verbatim}
\end{center}


{\def\LTcaptype{none} % do not increment counter
\vspace{-5pt}
\captionof{table}{Advantages and Disadvantages of SMPS}
\vspace{-10pt}
\begin{longtable}[]{@{}ll@{}}
\toprule\noalign{}
Advantages & Disadvantages \\
\midrule\noalign{}
\endhead
\bottomrule\noalign{}
\endlastfoot
High efficiency (80-95\%) & Complex circuit design \\
Small size and lightweight & Generates high-frequency noise \\
Wide input voltage range & EMI/RFI interference \\
Good regulation & Higher cost for low power \\
Lower heat generation & Difficult troubleshooting \\
\end{longtable}
}

\begin{itemize}
\tightlist
\item
  \textbf{Working principle}: Rapidly switches power on/off at high
  frequency
\item
  \textbf{Size reduction}: Higher switching frequency allows smaller
  transformers
\item
  \textbf{Applications}: Computers, TVs, mobile chargers, LED drivers
\end{itemize}

\end{solutionbox}
\begin{mnemonicbox}
``SWEEP advantages'' (Small size, Widerange input,
Efficient, Economical, Precise regulation)

\end{mnemonicbox}
\subsection*{Question 5(a) OR [3
marks]}\label{q5a}

\textbf{Draw voltage regulator using three terminal IC 7812.}

\begin{solutionbox}
Three terminal IC 7812 provides fixed +12V regulated
output voltage.

\textbf{Diagram:}

\begin{verbatim}
        +{-{-}{-}{-}{-}{-}{-}+{-}{-}{-}{-}{-}{-}{-}+}
        |       |       |
   Vin  |       |       |  Vout
   +{-{-}{-}{-}|  IN   OUT  |{-}{-}{-}{-}+ (+12V)}
        |       |       |
   +{-{-}{-}{-}|  GND       |{-}{-}{-}{-}+}
   |    |       |       |
   |    +{-{-}{-}{-}{-}{-}{-}+{-}{-}{-}{-}{-}{-}{-}+}
   |        |
   |       \_|\_
   |       {-{-}{-} C1}
   |        |
   +{-{-}{-}{-}{-}{-}{-}{-}+}
           GND
\end{verbatim}

\begin{itemize}
\tightlist
\item
  \textbf{Components}: 7812 regulator IC and filter capacitors
\item
  \textbf{Pin configuration}: Input, Ground, Output
\item
  \textbf{Features}: Internal current limiting and thermal shutdown
\end{itemize}

\end{solutionbox}
\begin{mnemonicbox}
``IGO pins'' (Input, Ground, Output)

\end{mnemonicbox}
\subsection*{Question 5(b) OR [4
marks]}\label{q5b}

\textbf{Describe Transistor series Voltage Regulator}

\begin{solutionbox}
Transistor series regulator controls output voltage by
varying the conductivity of a series transistor.

\textbf{Diagram:}

\begin{verbatim}
   +{-{-}{-}{-}{-}{-}{-}{-}{-}{-}+}
   |          |
   |         \_|\_
   |         { / Zener}
   |          |
   |          |
   +{-{-}{-}{-}{-}+{-}{-}{-}{-}+}
         |
         |   +{-{-}{-}{-}{-}{-}{-}{-}{-}{-}{-}+}
         +{-{-}{-}| Base      |}
             |           |
  Vin        | Transistor|     Vout
  +{-{-}{-}{-}{-}{-}{-}{-}{-}{-}| Collector |{-}{-}{-}{-}{-}{-}+}
             |           |      |
             | Emitter   |      |
             +{-{-}{-}{-}{-}{-}{-}{-}{-}{-}{-}+      |}
                 |              |
                 |             \_|\_
                 |             {-{-}{-} C}
                 |              |
                 +{-{-}{-}{-}{-}{-}{-}{-}{-}{-}{-}{-}{-}{-}+}
                             GND
\end{verbatim}


{\def\LTcaptype{none} % do not increment counter
\vspace{-5pt}
\captionof{table}{Features of Series Voltage Regulator}
\vspace{-10pt}
\begin{longtable}[]{@{}ll@{}}
\toprule\noalign{}
Feature & Description \\
\midrule\noalign{}
\endhead
\bottomrule\noalign{}
\endlastfoot
Control Element & Transistor acts as variable resistor in series \\
Reference & Zener diode provides stable reference voltage \\
Regulation & Feedback adjusts transistor conductivity \\
Efficiency & Better than shunt regulator for high current loads \\
\end{longtable}
}

\begin{itemize}
\tightlist
\item
  \textbf{Working principle}: Transistor conductivity changes to
  maintain constant output
\item
  \textbf{Advantage}: More efficient than shunt regulators for higher
  currents
\end{itemize}

\end{solutionbox}
\begin{mnemonicbox}
``CERT circuit'' (Control transistor, Efficient
design, Reference voltage, Transistor in series)

\end{mnemonicbox}
\subsection*{Question 5(c) OR [7
marks]}\label{q5c}

\textbf{Draw and explain UPS block diagram with its advantages and
disadvantages.}

\begin{solutionbox}
Uninterruptible Power Supply (UPS) provides emergency
power when main supply fails.

\textbf{Diagram:}

\begin{center}
\textbf{Mermaid Diagram (Code)}
\begin{verbatim}
{Shaded}
{Highlighting}[]
graph LR
    A[AC Input] {-{-}{} B[Surge Protector]}
    B {-{-}{} C[Rectifier/Charger]}
    C {-{-}{} D[Battery]}
    C {-{-}{} E[Inverter]}
    D {-{-}{} E}
    E {-{-}{} F[Output Filter]}
    F {-{-}{} G[AC Output]}
    H[Control Circuit] {-{-}{} C}
    H {-{-}{} E}
    H {-{-}{} D}

    style A fill:\#f9f,stroke:\#333,stroke{-width:1px}
    style C fill:\#bbf,stroke:\#333,stroke{-width:1px}
    style D fill:\#bfb,stroke:\#333,stroke{-width:1px}
    style E fill:\#f9f,stroke:\#333,stroke{-width:1px}
    style H fill:\#bbf,stroke:\#333,stroke{-width:1px}
{Highlighting}
{Shaded}
\end{verbatim}
\end{center}


{\def\LTcaptype{none} % do not increment counter
\vspace{-5pt}
\captionof{table}{Advantages and Disadvantages of UPS}
\vspace{-10pt}
\begin{longtable}[]{@{}ll@{}}
\toprule\noalign{}
Advantages & Disadvantages \\
\midrule\noalign{}
\endhead
\bottomrule\noalign{}
\endlastfoot
Provides backup power & Limited backup time \\
Protects from voltage fluctuations & Regular battery maintenance \\
Surge protection & Initial high cost \\
Smooth power transition & Noise during operation \\
Power conditioning & Lower efficiency in standby \\
\end{longtable}
}

\begin{itemize}
\tightlist
\item
  \textbf{Types}: Offline/Standby, Line-interactive,
  Online/Double-conversion
\item
  \textbf{Applications}: Computers, medical equipment, data centers,
  telecommunications
\item
  \textbf{Working}: Normally passes main power while charging battery;
  switches to battery power during outage
\end{itemize}

\end{solutionbox}
\begin{mnemonicbox}
``POWER backup'' (Protection from Outages, Waveform
conditioning, Emission-free, Reliability boost)

\end{mnemonicbox}

\end{document}
