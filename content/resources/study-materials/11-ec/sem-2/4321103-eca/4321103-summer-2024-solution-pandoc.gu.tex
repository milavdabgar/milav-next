\documentclass[10pt,a4paper]{article}

% content/resources/templates/preamble.tex
\usepackage[margin=0.6in]{geometry}
\author{Milav Dabgar}
\usepackage{amsmath,amssymb,amsthm}
\usepackage{booktabs}
\usepackage{multirow}
\usepackage{xcolor}
\usepackage{tcolorbox}
\tcbuselibrary{breakable,skins}
\usepackage[colorlinks=true,linkcolor=blue]{hyperref}
\usepackage{titlesec}
\usepackage{enumitem}
\usepackage{tikz}
\usepackage{pgfplots}
\usepackage{circuitikz}
\usepackage[version=4]{mhchem}
\usepackage{longtable}
\usepackage{array}
\usepackage{float}
\usepackage{caption}
\usepackage{listings}

\lstset{
  basicstyle=\small\ttfamily,
  breaklines=true,
  breakatwhitespace=false,
  postbreak=\mbox{\textcolor{red}{$\hookrightarrow$}\space},
  float=false,
  numbers=left,
  numberstyle=\tiny\color{gray},
  numbersep=10pt,
  xleftmargin=2em,
  keywordstyle=\color{blue},
  commentstyle=\color{green!60!black},
  stringstyle=\color{purple},
  backgroundcolor=\color{gray!5},
  showstringspaces=false,
  tabsize=2,
  captionpos=b,
  keepspaces=true,
  columns=flexible
}

\pgfplotsset{compat=1.18}
\usetikzlibrary{shapes,arrows,positioning,calc,patterns,decorations.pathmorphing,decorations.markings,arrows.meta}

% Color scheme
\definecolor{headcolor}{RGB}{0,102,204}
\definecolor{keycolor}{RGB}{220,20,60}
\definecolor{solutioncolor}{RGB}{34,139,34}
\definecolor{mnemoniccolor}{RGB}{148,0,211}
\definecolor{codecolor}{RGB}{0,0,100}

% Spacing
\setlength{\parskip}{3pt}
\setlist[itemize]{nosep}
\setlist[enumerate]{nosep}

% Title formatting
\titleformat{\section}{\Large\bfseries\color{headcolor}}{\thesection}{1em}{}
\titleformat{\subsection}{\large\bfseries\color{headcolor}}{\thesubsection}{1em}{}

% Pandoc tightlist compatibility
\providecommand{\tightlist}{%
  \setlength{\itemsep}{0pt}\setlength{\parskip}{0pt}}

% Pandoc longtable compatibility
\newcounter{none}
\def\thenone{}


% content/resources/templates/gujarati-boxes.tex
\usepackage{fontspec}
\usepackage{polyglossia}

% Set Gujarati as main language (document is primarily in Gujarati)
% Note: gloss-gujarati.ldf doesn't exist in polyglossia, but it will use hyphenation patterns
\setdefaultlanguage{gujarati}
\setotherlanguage{english}

% Configure Gujarati font properly
% Use Language=Default to prevent polyglossia from trying to add language-specific features
% that don't exist for Gujarati, which causes "empty feature" warnings
\newfontfamily\gujaratifont[Script=Gujarati,AutoFakeBold=2.5,AutoFakeSlant=0.3]{Noto Sans Gujarati}
\setmainfont[Script=Gujarati,AutoFakeBold=2.5,AutoFakeSlant=0.3]{Noto Sans Gujarati}
% Use Noto Sans Gujarati for monospace to support Gujarati in text
\setmonofont[Scale=0.9]{Noto Sans Gujarati}

% Configure English to use the same font
\newfontfamily\englishfont[Script=Gujarati,AutoFakeBold=2.5,AutoFakeSlant=0.3]{Noto Sans Gujarati}

% Translations for polyglossia
\gappto\captionsgujarati{
  \renewcommand{\tablename}{કોષ્ટક}
  \renewcommand{\figurename}{આકૃતિ}
}

% Helper for TikZ nodes to ensure Gujarati font
\newcommand{\gu}[1]{{\gujaratifont #1}}

% Custom environments
\newtcolorbox{solutionbox}{
    breakable,
    enhanced,
    colback=solutioncolor!5!white,
    colframe=solutioncolor!75!black,
    fonttitle=\bfseries,
    title=જવાબ
}

\newtcolorbox{solutionboxnobreak}{
 colback=solutioncolor!5!white,
 colframe=solutioncolor!75!black,
 fonttitle=\bfseries,
 title=જવાબ
}

\newtcolorbox{keyformula}{
 breakable,
 enhanced,
 colback=keycolor!5!white,
 colframe=keycolor!75!black,
 fonttitle=\bfseries,
 title=રાસાયણિક સમીકરણ/સૂત્ર
}

\newtcolorbox{mnemonicbox}{
 breakable,
 enhanced,
 colback=mnemoniccolor!5!white,
 colframe=mnemoniccolor!75!black,
 fonttitle=\bfseries,
 title=મેમરી ટ્રીક
}


\begin{document}

\begin{center}
{\Huge\bfseries\color{headcolor} Subject Name (Gujarati)}\\[5pt]
{\LARGE 4321103 -- Summer 2024}\\[3pt]
{\large Semester 1 Study Material}\\[3pt]
{\normalsize\textit{Detailed Solutions and Explanations}}
\end{center}

\vspace{10pt}

\subsection*{પ્રશ્ન 1(અ) [3
માર્ક્સ]}\label{uxaaauxab0uxab6uxaa8-1uxa85-3-uxaaeuxab0uxa95uxab8}

\textbf{CE રૂપરેખાંકન માટે એમ્પ્લીફાયર પરિમાણો Ai, Ri અને Ro સમજાવો.}

\begin{solutionbox}
કોમન એમિટર (CE) રૂપરેખાંકનમાં, મુખ્ય પરિમાણો છે:

\textbf{આકૃતિ:}

\begin{verbatim}
   +Vcc
     |
     R
     |
     |C
B{-{-}{-}{-}|{-}{-}{-}{-}+{-}{-}{-}{-}Output}
     |    |
     |   RC
  RB |    |
     |    |
  {-{-}{-}|    |{-}{-}{-}}
 |   |    |   |
 |   |    |   |
 +{-{-}{-}+{-}{-}{-}{-}+{-}{-}{-}+}
     |
     |
    GND
\end{verbatim}

\begin{itemize}
\tightlist
\item
  \textbf{કરંટ ગેઇન (Ai)}: આઉટપુટ કરંટનો ઇનપુટ કરંટ સાથેનો ગુણોત્તર (Ic/Ib),
  સામાન્ય રીતે CE માં 50-200
\item
  \textbf{ઇનપુટ રેઝિસ્ટન્સ (Ri)}: બેઝ ટર્મિનલ પર ઇનપુટ કરંટનો વિરોધ, CE માં 1-2kΩ
\item
  \textbf{આઉટપુટ રેઝિસ્ટન્સ (Ro)}: કલેક્ટર ટર્મિનલ પર વિરોધ, સામાન્ય રીતે CE માં
  50kΩ
\end{itemize}

\end{solutionbox}
\begin{mnemonicbox}
``CIR પરિમાણો - કરંટ ગેઇન, ઇનપુટ રેઝિસ્ટન્સ, અને આઉટપુટ
રેઝિસ્ટન્સ એમ્પ્લીફાયરની કાર્યક્ષમતા નક્કી કરે છે''

\end{mnemonicbox}
\subsection*{પ્રશ્ન 1(બ) [4
માર્ક્સ]}\label{uxaaauxab0uxab6uxaa8-1uxaac-4-uxaaeuxab0uxa95uxab8}

\textbf{હીટ સિંક પર ટૂંકી નોંધ લખો.}

\begin{solutionbox}

\textbf{આકૃતિ:}

\begin{verbatim}
                        Fins
            |‾‾‾|‾‾‾|‾‾‾|‾‾‾|‾‾‾|‾‾‾|‾‾‾|‾‾‾|
            |   |   |   |   |   |   |   |   |
+{-{-}{-}{-}{-}{-}{-}{-}{-}{-}{-}+   |   |   |   |   |   |   |   |}
| Transistor|\_\_\_|\_\_\_|\_\_\_|\_\_\_|\_\_\_|\_\_\_|\_\_\_|\_\_\_|
+{-{-}{-}{-}{-}{-}{-}{-}{-}{-}{-}+}
      Base
\end{verbatim}

\begin{itemize}
\tightlist
\item
  \textbf{ઉદ્દેશ}: ઇલેક્ટ્રોનિક ઘટકોમાંથી થર્મલ નુકસાન રોકવા માટે વધારાની ગરમીનું
  વિસર્જન
\item
  \textbf{પ્રકારો}: પેસિવ હીટ સિંક (એલ્યુમિનિયમ/કોપર ફિન્સ) અને એક્ટિવ હીટ સિંક
  (ફેન સાથે)
\item
  \textbf{થર્મલ રેઝિસ્ટન્સ}: ઓછી થર્મલ રેઝિસ્ટન્સ (^\circC/W) વધુ સારી ગરમી વિસર્જન
  દર્શાવે છે
\item
  \textbf{સામગ્રી}: કોપર (શ્રેષ્ઠ વાહકતા), એલ્યુમિનિયમ (હલકું, કિફાયતી), કમ્પોઝિટ
\end{itemize}

\end{solutionbox}
\begin{mnemonicbox}
``HARD સિંક - ગરમીને Heat Away using Radiation and
Dissipation through metal sinks''

\end{mnemonicbox}
\subsection*{પ્રશ્ન 1(ક) [7
માર્ક્સ]}\label{uxaaauxab0uxab6uxaa8-1uxa95-7-uxaaeuxab0uxa95uxab8}

\textbf{થર્મલ રનઅવે અને થર્મલ સ્ટેબિલિટીનું વર્ણન કરો. ટ્રાન્ઝિસ્ટરમાં થર્મલ રનઅવે કેવી
રીતે દૂર કરી શકાય?}

\begin{solutionbox}

\textbf{આકૃતિ:}

\begin{verbatim}
flowchart LR
    A[ગરમી ઉત્પાદન] {-{-} B[તાપમાનમાં વધારો]}
    B {-{-} C[Ic માં વધારો]}
    C {-{-} D[વધુ પાવર વપરાશ]}
    D {-{-} A}
    E[થર્મલ સ્ટેબિલિટી પદ્ધતિઓ] {-{-} F[આ ચક્ર તોડો]}
\end{verbatim}

\textbf{થર્મલ રનઅવે:}

\begin{itemize}
\tightlist
\item
  \textbf{વ્યાખ્યા}: સ્વ-ત્વરિત પ્રક્રિયા જ્યાં ટ્રાન્ઝિસ્ટર ગરમ થાય છે, જેનાથી વધુ
  કરંટ પ્રવાહ અને વધુ ગરમી થાય છે
\item
  \textbf{કારણ}: તાપમાનમાં વધારો Ico (લીકેજ કરંટ)માં વધારો કરે છે જે Ic વધારે છે
\item
  \textbf{પરિણામ}: જો નિયંત્રણ ન કરવામાં આવે તો ટ્રાન્ઝિસ્ટરનો અંતિમ વિનાશ
\end{itemize}

\textbf{થર્મલ સ્ટેબિલિટી:}

\begin{itemize}
\tightlist
\item
  \textbf{વ્યાખ્યા}: તાપમાન પરિવર્તન છતાં સ્થિર ઓપરેટિંગ પોઇન્ટ જાળવવાની ક્ષમતા
\item
  \textbf{માપ}: સ્ટેબિલિટી ફેક્ટર (S) - ઓછા મૂલ્યો વધુ સારી સ્થિરતા દર્શાવે છે
\end{itemize}

\textbf{થર્મલ રનઅવે દૂર કરવાના ઉપાયો:}

\begin{itemize}
\tightlist
\item
  \textbf{હીટ સિંક્સ}: વધારાની ગરમી દૂર કરવા માટે જોડો
\item
  \textbf{એમિટર રેઝિસ્ટર}: નકારાત્મક ફીડબેક આપવા માટે અનબાયપાસ્ડ RE શામેલ કરો
\item
  \textbf{વોલ્ટેજ ડિવાઇડર બાયસ}: વધુ સારી સ્થિરતા માટે ફિક્સ્ડ બાયસ ને બદલે ઉપયોગ
  કરો
\item
  \textbf{થર્મલ કમ્પેન્સેશન}: બાયસ સર્કિટમાં તાપમાન-સંવેદનશીલ ઘટકો ઉમેરો
\end{itemize}

\end{solutionbox}
\begin{mnemonicbox}
``SHEER સુરક્ષા - ગરમી માટે સિંક્સ, એમિટર રેઝિસ્ટર્સ, બાહ્ય
કૂલિંગ, અને મજબૂત બાયસિંગ થર્મલ રનઅવે અટકાવે છે''

\end{mnemonicbox}
\subsection*{પ્રશ્ન 1(ક) OR [7
માર્ક્સ]}\label{uxaaauxab0uxab6uxaa8-1uxa95-or-7-uxaaeuxab0uxa95uxab8}

\textbf{બાયસિંગ પદ્ધતિઓના પ્રકારો લખો. વોલ્ટેજ વિભાજક બાયસિંગ પદ્ધતિને વિગતમાં
સમજાવો.}

\begin{solutionbox}

\textbf{બાયસિંગ પદ્ધતિઓના પ્રકારો:}


{\def\LTcaptype{none} % do not increment counter
\vspace{-5pt}
\captionof{table}{ટ્રાન્ઝિસ્ટર બાયસિંગ પદ્ધતિઓ}
\vspace{-10pt}
\begin{longtable}[]{@{}lll@{}}
\toprule\noalign{}
પદ્ધતિ & સ્થિરતા & જટિલતા \\
\midrule\noalign{}
\endhead
\bottomrule\noalign{}
\endlastfoot
ફિક્સ્ડ બાયસ & નબળી & સરળ \\
કલેક્ટર ફીડબેક & મધ્યમ & મધ્યમ \\
એમિટર બાયસ & સારી & મધ્યમ \\
વોલ્ટેજ ડિવાઇડર & ઉત્તમ & જટિલ \\
\end{longtable}
}

\textbf{વોલ્ટેજ ડિવાઇડર બાયસિંગ સર્કિટ:}

\begin{verbatim}
    +Vcc
      |
      R1
      |
      +{-{-}{-}{-}+}
      |    |
      R2   C1
      |    |
      +{-{-}{-}{-}+{-}{-}{-} Base}
      |
      |    +{-{-}{-}{-}+}
      |    |    |
      RE   RC   C2
      |    |    |
      +{-{-}{-}{-}+{-}{-}{-}{-}+{-}{-}{-} Output}
      |
     GND
\end{verbatim}

\textbf{વોલ્ટેજ ડિવાઇડર બાયસિંગ:}

\begin{itemize}
\tightlist
\item
  \textbf{સર્કિટ સ્ટ્રક્ચર}: બેઝ પર સ્થિર વોલ્ટેજ બનાવવા માટે શ્રેણીમાં બે રેઝિસ્ટર્સ
  (R1, R2) વાપરે છે
\item
  \textbf{ઓપરેટિંગ સિદ્ધાંત}: R2 પર વોલ્ટેજ બેઝ બાયસ સેટ કરે છે, β વેરિએશન છતાં સ્થિર
  રહે છે
\item
  \textbf{ફાયદા}: શ્રેષ્ઠ તાપમાન કોમ્પેન્સેશન સાથેની સૌથી સ્થિર બાયસિંગ તકનીક
\item
  \textbf{સૂત્ર}: બેઝ વોલ્ટેજ VB = Vcc \times (R2/(R1+R2))
\item
  \textbf{સ્થિરતા}: કલેક્ટર કરંટથી લગભગ સ્વતંત્ર બેઝ વોલ્ટેજ સાથે ઉચ્ચ સ્થિરતા ફેક્ટર
\end{itemize}

\end{solutionbox}
\begin{mnemonicbox}
``DIVE સ્થિરતા માટે - ડિવાઇડર તાપમાન અને β વેરિએશન માટે
ખૂબ અસરકારક છે''

\end{mnemonicbox}
\subsection*{પ્રશ્ન 2(અ) [3
માર્ક્સ]}\label{uxaaauxab0uxab6uxaa8-2uxa85-3-uxaaeuxab0uxa95uxab8}

\textbf{સ્ટેબિલિટી ફેક્ટર અને તેની વિશેષતાઓ સમજાવો.}

\begin{solutionbox}

\textbf{આકૃતિ:}

\begin{verbatim}
flowchart LR
    A[તાપમાન પરિવર્તન] {-{-} B\{સ્ટેબિલિટી ફેક્ટર S\}}
    B {-{-}|ઉચ્ચ S| C[અસ્થિર સર્કિટ]}
    B {-{-}|નીચું S| D[સ્થિર સર્કિટ]}
\end{verbatim}

\begin{itemize}
\tightlist
\item
  \textbf{વ્યાખ્યા}: સ્ટેબિલિટી ફેક્ટર (S) માપે છે કે લીકેજ કરંટથી કલેક્ટર કરંટ કેવી
  રીતે બદલાય છે
\item
  \textbf{સૂત્ર}: S = ΔIC/ΔICBO
\item
  \textbf{આદર્શ મૂલ્ય}: નીચું મૂલ્ય (S \approx 1) વધુ સારી સ્થિરતા દર્શાવે છે
\item
  \textbf{અસર કરતા પરિબળો}: બાયસિંગ સર્કિટ ડિઝાઇન, તાપમાન, અને ટ્રાન્ઝિસ્ટર
  પરિમાણો
\end{itemize}

\end{solutionbox}
\begin{mnemonicbox}
``LESS એટલે બેહતર - નીચા મૂલ્યો તાપમાન પરિવર્તન માટે સ્થિર
સિસ્ટમ સુનિશ્ચિત કરે છે''

\end{mnemonicbox}
\subsection*{પ્રશ્ન 2(બ) [4
માર્ક્સ]}\label{uxaaauxab0uxab6uxaa8-2uxaac-4-uxaaeuxab0uxa95uxab8}

\textbf{કાસ્કેડિંગની ડાયરેક્ટ કપલિંગ ટેકનિક વર્ણવો.}

\begin{solutionbox}

\textbf{આકૃતિ:}

\begin{verbatim}
     +Vcc
       |
       |
       RC1     RC2
       |       |
       +{-{-}{-}{-}{-}+ |}
       |     | |
    +{-{-}+     +{-}+{-}{-}+}
    |  |     |    |
Q1  |C |    C|    | Q2
    |  |     |    |
    +{-{-}+     +{-}+{-}{-}+}
    |  |     | |
    E  |     E |
    |  |     | |
    +{-{-}+     +{-}+}
       |       |
       RE1     RE2
       |       |
      GND     GND
\end{verbatim}

\begin{itemize}
\tightlist
\item
  \textbf{વ્યાખ્યા}: પ્રથમ તબક્કાના કલેક્ટરથી બીજા તબક્કાના બેઝ સાથે સીધો જોડાણ
\item
  \textbf{ફાયદા}: કપલિંગ ઘટકોની જરૂર નથી, ઉત્તમ નિમ્ન-આવર્તન પ્રતિસાદ
\item
  \textbf{ગેરફાયદા}: DC લેવલ્સ મેચ કરવા જોઈએ, થર્મલ ડ્રિફ્ટ સ્ટેજ દીઠ વધે છે
\item
  \textbf{ઉપયોગો}: DC એમ્પ્લીફાયર્સ, ઇન્ટિગ્રેટેડ સર્કિટ્સ, ઓપરેશનલ એમ્પ્લીફાયર્સ
\end{itemize}

\end{solutionbox}
\begin{mnemonicbox}
``DIAL DC માટે - કેપેસિટર વગર સીધા જોડાણ નિમ્ન આવર્તનોને
એમ્પ્લિફાય કરે છે''

\end{mnemonicbox}
\subsection*{પ્રશ્ન 2(ક) [7
માર્ક્સ]}\label{uxaaauxab0uxab6uxaa8-2uxa95-7-uxaaeuxab0uxa95uxab8}

\textbf{બે તબક્કાના RC કપલ્ડ એમ્પ્લીફાયરનો આવર્તન પ્રતિભાવ સમજાવો.}

\begin{solutionbox}

\textbf{ફ્રીક્વન્સી રિસ્પોન્સ કર્વ:}

\begin{verbatim}
    Gain (dB)
    \^{}
    |                 \_\_\_\_\_\_\_\_\_\_\_
    |                /           {}
    |               /             {}
    |              /               {}
    |             /                 {}
    |{-{-}{-}{-}{-}{-}{-}{-}{-}{-}{-}{-}/                   {-}{-}{-}{-}Frequency}
               f1                    f2
        Low frequency    Mid frequency    High frequency
          region           region          region
\end{verbatim}

\textbf{બે-તબક્કાનો RC કપલ્ડ એમ્પ્લીફાયર:}

\begin{itemize}
\tightlist
\item
  \textbf{સર્કિટ સ્ટ્રક્ચર}: કપલિંગ કેપેસિટર્સ દ્વારા જોડાયેલ બે ટ્રાન્ઝિસ્ટર
  એમ્પ્લીફાયર્સ
\item
  \textbf{નિમ્ન-આવર્તન પ્રતિસાદ (f \textless{} f1)}: કપલિંગ અને બાયપાસ
  કેપેસિટરની અસરોને કારણે ગેઇન ઘટે છે
\item
  \textbf{મધ્ય-આવર્તન પ્રતિસાદ (f1 \textless{} f \textless{} f2)}: મહત્તમ
  ગેઇન ક્ષેત્ર, સપાટ પ્રતિસાદ
\item
  \textbf{ઉચ્ચ-આવર્તન પ્રતિસાદ (f \textgreater{} f2)}: આંતરિક કેપેસિટન્સ અને
  મિલર ઇફેક્ટને કારણે ગેઇન ઘટે છે
\item
  \textbf{બેન્ડવિડ્થ}: નીચલા કટઓફ (f1) અને ઉપલા કટઓફ (f2) આવર્તન વચ્ચેની રેન્જ
\item
  \textbf{કુલ ગેઇન}: વ્યક્તિગત સ્ટેજ ગેઇનનો ગુણાકાર ઓછા કપલિંગ નુકસાન
\end{itemize}

\end{solutionbox}
\begin{mnemonicbox}
``LMH આવર્તન ક્ષેત્રો - નિમ્નમાં વધતો ગેઇન, મધ્યમાં સપાટ ગેઇન,
ઉચ્ચમાં ઘટતો ગેઇન''

\end{mnemonicbox}
\subsection*{પ્રશ્ન 2(અ) OR [3
માર્ક્સ]}\label{uxaaauxab0uxab6uxaa8-2uxa85-or-3-uxaaeuxab0uxa95uxab8}

\textbf{એમ્પ્લીફાયરની બેન્ડવિડ્થ અને ગેઇન-બેન્ડવિડ્થ ઉત્પાદનને સંક્ષિપ્તમાં સમજાવો.}

\begin{solutionbox}

\textbf{આકૃતિ:}

\begin{verbatim}
    Gain (dB)
    \^{}
    |     \_\_\_\_\_\_\_\_\_\_\_\_\_\_\_
    |    /|              {}
    |   / |               {}
    |  /  |                {}
    | /   |                 {}
    |/    |                  {}
    +{-{-}{-}{-}{-}|{-}{-}{-}{-}{-}{-}{-}{-}{-}{-}{-}{-}{-}{-}{-}{-}{-}{-}|{-}{-}{-}{-}{-} Frequency}
          f1                 f2
          |{{-}{-}{-}Bandwidth{-}{-}{-}{-}|}
\end{verbatim}

\begin{itemize}
\tightlist
\item
  \textbf{બેન્ડવિડ્થ}: નીચલા (f1) અને ઉપલા (f2) કટઓફ આવર્તનો વચ્ચેનો આવર્તન રેન્જ
  જ્યાં ગેઇન મહત્તમનો ઓછામાં ઓછો 70.7\% હોય છે
\item
  \textbf{સૂત્ર}: બેન્ડવિડ્થ = f2 - f1 (Hz માં માપવામાં આવે છે)
\item
  \textbf{ગેઇન-બેન્ડવિડ્થ ઉત્પાદન}: આપેલા એમ્પ્લીફાયર માટે ગેઇન ગુણાકાર બેન્ડવિડ્થનું
  અચળ મૂલ્ય
\item
  \textbf{મહત્વ}: એમ્પ્લીફાયર કાર્યક્ષમતાની મૂળભૂત મર્યાદાને દર્શાવે છે
\end{itemize}

\end{solutionbox}
\begin{mnemonicbox}
``BIG મૂલ્ય - બેન્ડવિડ્થ અને ગેઇન વચ્ચે વિપરીત સંબંધ અચળ છે''

\end{mnemonicbox}
\subsection*{પ્રશ્ન 2(બ) OR [4
માર્ક્સ]}\label{uxaaauxab0uxab6uxaa8-2uxaac-or-4-uxaaeuxab0uxa95uxab8}

\textbf{એમ્પ્લીફાયરના ફ્રીક્વન્સી રિસ્પોન્સ પર એમિટર બાયપાસ કેપેસિટર અને કપલિંગ
કેપેસિટરની અસરો સમજાવો.}

\begin{solutionbox}


{\def\LTcaptype{none} % do not increment counter
\vspace{-5pt}
\captionof{table}{કેપેસિટરની ફ્રીક્વન્સી રિસ્પોન્સ પર અસરો}
\vspace{-10pt}
\begin{longtable}[]{@{}llll@{}}
\toprule\noalign{}
Capacitor Type & Low Frequency & Mid Frequency & High Frequency \\
\midrule\noalign{}
\endhead
\bottomrule\noalign{}
\endlastfoot
Emitter Bypass & Affects gain & Full bypass & No effect \\
Coupling & Blocks signal & Full coupling & No effect \\
\end{longtable}
}

\textbf{કેપેસિટરની અસરો:}

\textbf{એમિટર બાયપાસ કેપેસિટર:}

\begin{itemize}
\tightlist
\item
  \textbf{હેતુ}: ગેઇન વધારવા માટે એમિટર રેઝિસ્ટરને બાયપાસ કરે છે
\item
  \textbf{નિમ્ન આવર્તન}: ઉચ્ચ ઇમ્પિડન્સ તરીકે કાર્ય કરે છે, ગેઇન ઘટાડે છે
\item
  \textbf{સૂત્ર}: Xc = 1/(2πfC) નિમ્ન આવર્તન પર વધે છે
\item
  \textbf{કટઓફ અસર}: RE સાથે નીચલી કટઓફ આવર્તન સેટ કરે છે
\end{itemize}

\textbf{કપલિંગ કેપેસિટર:}

\begin{itemize}
\tightlist
\item
  \textbf{હેતુ}: DC બ્લોક કરે છે, તબક્કાઓ વચ્ચે AC સિગ્નલની મંજૂરી આપે છે
\item
  \textbf{નિમ્ન આવર્તન}: ઉચ્ચ રિએક્ટન્સ સિગ્નલ ટ્રાન્સફર અવરોધે છે
\item
  \textbf{પ્રતિસાદ અસર}: મોટી કેપેસિટન્સ નિમ્ન-આવર્તન પ્રતિસાદ સુધારે છે
\item
  \textbf{ફેઝ શિફ્ટ}: નિમ્ન આવર્તનોએ ફેઝ શિફ્ટ ઉત્પન્ન કરે છે
\end{itemize}

\end{solutionbox}
\begin{mnemonicbox}
``CABLE અસર - કેપેસિટર્સ નિમ્ન આવર્તન પર અવરોધ તરીકે કાર્ય
કરે છે, ઉચ્ચ આવર્તન પર સુધારો કરે છે''

\end{mnemonicbox}
\subsection*{પ્રશ્ન 2(ક) OR [7
માર્ક્સ]}\label{uxaaauxab0uxab6uxaa8-2uxa95-or-7-uxaaeuxab0uxa95uxab8}

\textbf{ટ્રાન્સફોર્મર કપલ્ડ એમ્પ્લીફાયર અને RC કપલ્ડ એમ્પ્લીફાયરની સરખામણી કરો.}

\begin{solutionbox}


{\def\LTcaptype{none} % do not increment counter
\vspace{-5pt}
\captionof{table}{ટ્રાન્સફોર્મર કપલ્ડ અને RC કપલ્ડ એમ્પ્લીફાયર્સની સરખામણી}
\vspace{-10pt}
\begin{longtable}[]{@{}lll@{}}
\toprule\noalign{}
પરિમાણ & ટ્રાન્સફોર્મર કપલ્ડ & RC કપલ્ડ \\
\midrule\noalign{}
\endhead
\bottomrule\noalign{}
\endlastfoot
કપલિંગ ઘટક & ટ્રાન્સફોર્મર & કેપેસિટર અને રેઝિસ્ટર \\
કાર્યક્ષમતા & ઉચ્ચ (90\%) & નીચી (30-50\%) \\
આવર્તન પ્રતિસાદ & મર્યાદિત, છેડાઓ પર નબળો & વિશાળ, નિમ્ન આવર્તન પર વધુ સારો \\
કદ અને વજન & મોટું, ભારે & કોમ્પેક્ટ, હલકું \\
ખર્ચ & ઊંચો & નીચો \\
ઇમ્પિડન્સ મેચિંગ & ઉત્તમ & નબળું \\
વિકૃતિ & નીચી & ઊંચી \\
DC આઇસોલેશન & સંપૂર્ણ & સારું \\
\end{longtable}
}

\textbf{આકૃતિ સરખામણી:}

\begin{verbatim}
ટ્રાન્સફોર્મર કપલ્ડ             RC કપલ્ડ
    +Vcc                           +Vcc
      |                              |
      RC                             RC
      |                              |
      +-----|OOOO|-----+             +------||------+
      |     |OOOO|     |             |      CC      |
      C     |OOOO|     C             C              C
      |                |             |              |
      +                +             +              +
      |                |             |              |
     GND              GND           GND            GND
\end{verbatim}

\end{solutionbox}
\begin{mnemonicbox}
``TREE પરિબળો - ટ્રાન્સફોર્મર્સ મજબૂત કાર્યક્ષમતા અને ઉત્તમ
ઇમ્પિડન્સ મેચિંગ આપે છે, RC ખર્ચની બચત કરે છે''

\end{mnemonicbox}
\subsection*{પ્રશ્ન 3(અ) [3
માર્ક્સ]}\label{uxaaauxab0uxab6uxaa8-3uxa85-3-uxaaeuxab0uxa95uxab8}

\textbf{ટ્રાન્ઝિસ્ટર ટ્યુન કરેલ એમ્પ્લીફાયરનું વર્ણન કરો.}

\begin{solutionbox}

\textbf{સર્કિટ આકૃતિ:}

\begin{verbatim}
    +Vcc
      |
      |
      +{-{-}{-}{-}+}
      |    |
      L    C
      |    |
      +{-{-}{-}{-}+}
      |
      C1
      |
      +{-{-}{-}{-}+{-}{-}{-}Output}
      |    |
      Q    RC
      |    |
      +{-{-}{-}{-}+}
      |
     GND
\end{verbatim}

\begin{itemize}
\tightlist
\item
  \textbf{વ્યાખ્યા}: ચોક્કસ આવર્તન બેન્ડને એમ્પ્લિફાય કરવા માટે કલેક્ટરમાં LC ટેન્ક
  સર્કિટ સાથેનો એમ્પ્લીફાયર
\item
  \textbf{સિદ્ધાંત}: LC સર્કિટ fr = 1/(2π\sqrtLC) પર રેઝોનેટ થાય છે, રેઝોનન્સ પર
  મહત્તમ ગેઇન આપે છે
\item
  \textbf{બેન્ડવિડ્થ}: RC એમ્પ્લીફાયર્સ કરતાં સાંકડી, ટ્યુન્ડ સર્કિટના Q ફેક્ટર દ્વારા
  નિર્ધારિત
\item
  \textbf{ઉપયોગો}: RF એમ્પ્લીફાયર્સ, રેડિયો રિસીવર્સ, વાયરલેસ કોમ્યુનિકેશન સર્કિટ્સ
\end{itemize}

\end{solutionbox}
\begin{mnemonicbox}
``TRIP રેઝોનન્સ માટે - ટ્યુન્ડ રેઝોનન્ટ સર્કિટ્સ ચોક્કસ આવર્તનો
પર કાર્યક્ષમતા સુધારે છે''

\end{mnemonicbox}
\subsection*{પ્રશ્ન 3(બ) [4
માર્ક્સ]}\label{uxaaauxab0uxab6uxaa8-3uxaac-4-uxaaeuxab0uxa95uxab8}

\textbf{ડાયરેક્ટ કપલ્ડ એમ્પ્લીફાયર સંક્ષિપ્તમાં સમજાવો.}

\begin{solutionbox}

\textbf{સર્કિટ આકૃતિ:}

\begin{verbatim}
    +Vcc
      |
      RC2
      |
      +{-{-}{-}{-}{-}{-}+{-}{-}{-}Output}
      |      |
      C      RC1
      |      |
      +{-{-}{-}{-}{-}{-}+}
      |
      E
      |
     GND
\end{verbatim}

\begin{itemize}
\tightlist
\item
  \textbf{વ્યાખ્યા}: કપલિંગ ઘટકો વગર સીધા જોડાણવાળો મલ્ટી-સ્ટેજ એમ્પ્લીફાયર
\item
  \textbf{કાર્યપદ્ધતિ}: પ્રથમ તબક્કાનો કલેક્ટર બીજા તબક્કાના બેઝ સાથે સીધો જોડાય
  છે
\item
  \textbf{ફાયદા}: ઉત્તમ નિમ્ન-આવર્તન પ્રતિસાદ, ઓછા ઘટકો, કોમ્પેક્ટ ડિઝાઇન
\item
  \textbf{ગેરફાયદા}: DC બાયસ સમસ્યાઓ, થર્મલ સ્ટેબિલિટી સમસ્યાઓ, તબક્કા દીઠ
  મર્યાદિત ગેઇન
\end{itemize}

\end{solutionbox}
\begin{mnemonicbox}
``COLD ફાયદા - કોમ્પેક્ટ ડિઝાઇન, ઉત્તમ નિમ્ન-આવર્તન
પ્રતિસાદ, ઓછા ઘટકો, સીધું જોડાણ''

\end{mnemonicbox}
\subsection*{પ્રશ્ન 3(ક) [7
માર્ક્સ]}\label{uxaaauxab0uxab6uxaa8-3uxa95-7-uxaaeuxab0uxa95uxab8}

\textbf{બે પોર્ટ નેટવર્કમાં h પરિમાણોનું મહત્વ વર્ણવો. CE એમ્પ્લીફાયર માટે
h-પેરામીટર્સ સર્કિટ દોરો.}

\begin{solutionbox}

\textbf{CE માટે h-પેરામીટર સમકક્ષ સર્કિટ:}

\begin{verbatim}
                 RC
      +{-{-}{-}{-}{-}+    |}
      |     |    |
Input |    +++   | Output
 o{-{-}{-}{-}+{-}{-}{-}||{-}{-}{-}{-}o}
      |    +++   |
      |     |    |
      +{-{-}+{-}{-}+    |}
         |       |
        +++      |
        GND      |
\end{verbatim}

\textbf{h-પેરામીટર્સનું મહત્વ:}

\begin{itemize}
\tightlist
\item
  \textbf{સાર્વત્રિક ઉપયોગ}: બધા ટ્રાન્ઝિસ્ટર રૂપરેખાંકન (CE, CB, CC) માટે કામ કરે
  છે
\item
  \textbf{સરળ માપન}: પેરામીટર્સ સરળ સર્કિટ્સનો ઉપયોગ કરીને સીધા માપી શકાય છે
\item
  \textbf{સંપૂર્ણ લક્ષણો}: ચાર પેરામીટર્સ સાથે ટ્રાન્ઝિસ્ટર વર્તનનું સંપૂર્ણ વર્ણન કરે છે
\item
  \textbf{સર્કિટ એનાલિસિસ}: જટિલ ટ્રાન્ઝિસ્ટર સર્કિટ એનાલિસિસ સરળ બનાવે છે
\item
  \textbf{તાપમાન સ્વતંત્રતા}: સામાન્ય ઓપરેટિંગ તાપમાન પર પ્રમાણમાં સ્થિર
\end{itemize}

\textbf{CE માટે h-પેરામીટર્સ:}

\begin{itemize}
\tightlist
\item
  \textbf{h11 (hie)}: આઉટપુટ શોર્ટ-સર્કિટેડ સાથે ઇનપુટ ઇમ્પિડન્સ
\item
  \textbf{h12 (hre)}: રિવર્સ વોલ્ટેજ ટ્રાન્સફર રેશિયો
\item
  \textbf{h21 (hfe)}: ફોરવર્ડ કરંટ ગેઇન (β)
\item
  \textbf{h22 (hoe)}: ઇનપુટ ઓપન-સર્કિટેડ સાથે આઉટપુટ એડમિટન્સ
\end{itemize}

\end{solutionbox}
\begin{mnemonicbox}
``FINE પેરામીટર્સ - ચાર ઇન્ટરકનેક્ટેડ નેટવર્ક એલિમેન્ટ્સ
ટ્રાન્ઝિસ્ટરને સંપૂર્ણપણે વ્યાખ્યાયિત કરે છે''

\end{mnemonicbox}
\subsection*{પ્રશ્ન 3(અ) OR [3
માર્ક્સ]}\label{uxaaauxab0uxab6uxaa8-3uxa85-or-3-uxaaeuxab0uxa95uxab8}

\textbf{ટ્રાન્સફોર્મર કપલ્ડ એમ્પ્લીફાયર અને ડાયરેક્ટ કપલ્ડ એમ્પ્લીફાયરની સરખામણી
કરો.}

\begin{solutionbox}


{\def\LTcaptype{none} % do not increment counter
\vspace{-5pt}
\captionof{table}{ટ્રાન્સફોર્મર vs ડાયરેક્ટ કપલ્ડ એમ્પ્લીફાયર્સ}
\vspace{-10pt}
\begin{longtable}[]{@{}lll@{}}
\toprule\noalign{}
પરિમાણ & ટ્રાન્સફોર્મર કપલ્ડ & ડાયરેક્ટ કપલ્ડ \\
\midrule\noalign{}
\endhead
\bottomrule\noalign{}
\endlastfoot
DC આઇસોલેશન & સંપૂર્ણ & નથી \\
નિમ્ન આવર્તન પ્રતિસાદ & નબળો & ઉત્તમ \\
કદ & મોટું & કોમ્પેક્ટ \\
ઇમ્પિડન્સ મેચિંગ & ઉત્તમ & નબળું \\
વિકૃતિ & નીચી & ઊંચી હોઈ શકે \\
ખર્ચ & ઊંચો & નીચો \\
જટિલતા & મધ્યમ & સરળ \\
\end{longtable}
}

\end{solutionbox}
\begin{mnemonicbox}
``TIP પસંદગી માટે - ઇમ્પિડન્સ મેચિંગ અને પાવર ટ્રાન્સફર માટે
ટ્રાન્સફોર્મર, નિમ્ન આવર્તન માટે ડાયરેક્ટ''

\end{mnemonicbox}
\subsection*{પ્રશ્ન 3(બ) OR [4
માર્ક્સ]}\label{uxaaauxab0uxab6uxaa8-3uxaac-or-4-uxaaeuxab0uxa95uxab8}

\textbf{કોમન એમિટર એમ્પ્લીફાયરનું સર્કિટ ડાયાગ્રામ દોરો અને સમજાવો.}

\begin{solutionbox}

\textbf{CE એમ્પ્લીફાયર સર્કિટ:}

\begin{verbatim}
     +Vcc
       |
       RC
       |
       +{-{-}{-}{-}||{-}{-}{-}o Output}
       |    CC
       |
    +{-{-}+}
    |  |
    |  C
    |  |
 {-{-}{-}+{-}{-}+{-}{-}{-}}
 |  |  |  |
 |  |  |  |
 +{-{-}+{-}{-}+{-}{-}+}
    |
    RE
    |
   GND
\end{verbatim}

\begin{itemize}
\tightlist
\item
  \textbf{રૂપરેખાંકન}: બેઝ પર ઇનપુટ, કલેક્ટરથી આઉટપુટ, એમિટર બંનેમાં સામાન્ય છે
\item
  \textbf{લક્ષણો}: વોલ્ટેજ ગેઇન \textasciitilde50-500, કરંટ ગેઇન
  \textasciitilde50-200, ફેઝ શિફ્ટ 180^\circ
\item
  \textbf{ફાયદા}: ઉચ્ચ વોલ્ટેજ ગેઇન, મધ્યમ ઇનપુટ ઇમ્પિડન્સ, સારું વોલ્ટેજ એમ્પ્લિફિકેશન
\item
  \textbf{ઉપયોગો}: ઓડિયો એમ્પ્લીફાયર્સ, રેડિયો ફ્રીક્વન્સી એમ્પ્લીફાયર્સ, સ્વિચિંગ
  સર્કિટ્સ
\end{itemize}

\end{solutionbox}
\begin{mnemonicbox}
``GAIN લક્ષણો - ઉલટા આઉટપુટ અને નોંધપાત્ર કાર્યક્ષમતા સાથે
સારું એમ્પ્લિફિકેશન''

\end{mnemonicbox}
\subsection*{પ્રશ્ન 3(ક) OR [7
માર્ક્સ]}\label{uxaaauxab0uxab6uxaa8-3uxa95-or-7-uxaaeuxab0uxa95uxab8}

\textbf{ટ્રાન્ઝિસ્ટર ટુ પોર્ટ નેટવર્ક દોરો અને તેના માટે h-પેરામીટર્સનું વર્ણન કરો.
હાઇબ્રિડ પેરામીટર્સના ફાયદા લખો.}

\begin{solutionbox}

\textbf{ટુ-પોર્ટ નેટવર્ક આકૃતિ:}

\begin{verbatim}
       +{-{-}{-}{-}{-}{-}{-}{-}{-}{-}{-}{-}{-}+}
       |             |
 I1 {-{-}+             +{-}{-}{-} I2}
       |   Two{-Port  |}
 V1 {-{-}+   Network   +{-}{-}{-} V2}
       |             |
       +{-{-}{-}{-}{-}{-}{-}{-}{-}{-}{-}{-}{-}+}
\end{verbatim}

\textbf{h-પેરામીટર્સ સમીકરણો:}

\begin{itemize}
\tightlist
\item
  V1 = h11I1 + h12V2
\item
  I2 = h21I1 + h22V2
\end{itemize}

\textbf{h-પેરામીટર્સનું વર્ણન:}

\begin{itemize}
\tightlist
\item
  \textbf{h11}: આઉટપુટ શોર્ટ-સર્કિટેડ સાથે ઇનપુટ ઇમ્પિડન્સ (Ω)
\item
  \textbf{h12}: રિવર્સ વોલ્ટેજ ટ્રાન્સફર રેશિયો (અપરિમાણ)
\item
  \textbf{h21}: ફોરવર્ડ કરંટ ગેઇન (અપરિમાણ)
\item
  \textbf{h22}: ઇનપુટ ઓપન-સર્કિટેડ સાથે આઉટપુટ એડમિટન્સ (સીમેન્સ)
\end{itemize}

\textbf{હાઇબ્રિડ પેરામીટર્સના ફાયદા:}

\begin{itemize}
\tightlist
\item
  \textbf{સરળ માપન}: દરેક પેરામીટર વ્યક્તિગત રીતે માપી શકાય છે
\item
  \textbf{માનક સંજ્ઞા}: ઉદ્યોગ અને શૈક્ષણિક ક્ષેત્રમાં સાર્વત્રિક સ્વીકૃતિ
\item
  \textbf{સચોટ મોડેલ}: ટ્રાન્ઝિસ્ટર વર્તનનું સટીક મોડેલિંગ પ્રદાન કરે છે
\item
  \textbf{રૂપરેખાંકન લવચીકતા}: બધા ટ્રાન્ઝિસ્ટર રૂપરેખાંકન માટે લાગુ
\item
  \textbf{તાપમાન સ્થિરતા}: ઓપરેટિંગ તાપમાન શ્રેણી પર પ્રમાણમાં સ્થિર
\end{itemize}

\end{solutionbox}
\begin{mnemonicbox}
``SMART પેરામીટર્સ - સરળ માપન, સચોટ મોડેલિંગ, વિશ્વસનીય,
તાપમાન-સ્થિર''

\end{mnemonicbox}
\subsection*{પ્રશ્ન 4(અ) [3
માર્ક્સ]}\label{uxaaauxab0uxab6uxaa8-4uxa85-3-uxaaeuxab0uxa95uxab8}

\textbf{ડાર્લિંગ્ટન પેર અને તેની એપ્લિકેશનો સમજાવો.}

\begin{solutionbox}

\textbf{ડાર્લિંગ્ટન પેર સર્કિટ:}

\begin{verbatim}
    +{-{-}+}
    |  |
    |  C1    +{-{-}+}
 {-{-}{-}+{-}{-}+{-}{-}{-}{-}{-}|  |}
 |  |  |     |  C2   Output
 |  |  +{-{-}{-}{-}{-}|  |{-}{-}{-}{-}{-}o}
 |  B1 |     |  |
 o{-{-}{-}{-}{-}+{-}{-}{-}{-}{-}|  |}
    |     B2 |  |
    E1{-{-}{-}{-}{-}{-}{-}|  |}
             E2 |
              {-{-}|{-}{-}}
               GND
\end{verbatim}

\begin{itemize}
\tightlist
\item
  \textbf{વ્યાખ્યા}: બે ટ્રાન્ઝિસ્ટરનું રૂપરેખાંકન જ્યાં પ્રથમનો એમિટર બીજાના બેઝને
  ડ્રાઇવ કરે છે
\item
  \textbf{લક્ષણો}: ખૂબ ઉચ્ચ કરંટ ગેઇન (β1 \times β2), ઉચ્ચ ઇનપુટ ઇમ્પિડન્સ
\item
  \textbf{નુકસાન}: ઉચ્ચ સેચ્યુરેશન વોલ્ટેજ, ઓછી સ્વિચિંગ સ્પીડ
\item
  \textbf{ઉપયોગો}: પાવર એમ્પ્લીફાયર્સ, મોટર ડ્રાઇવર્સ, ટચ-સેન્સિટિવ સ્વિચ,
  ડાર્લિંગ્ટન ICs
\end{itemize}

\end{solutionbox}
\begin{mnemonicbox}
``HIGH ગેઇન - બે ટ્રાન્ઝિસ્ટર્સનો ઉપયોગ કરીને ખૂબ જ વધારેલો
ગેઇન''

\end{mnemonicbox}
\subsection*{પ્રશ્ન 4(બ) [4
માર્ક્સ]}\label{uxaaauxab0uxab6uxaa8-4uxaac-4-uxaaeuxab0uxa95uxab8}

\textbf{જરૂરી ડાયાગ્રામ સાથે ડાયોડ ક્લેમ્પર સર્કિટનું વર્ણન કરો.}

\begin{solutionbox}

\textbf{પોઝિટિવ ક્લેમ્પર સર્કિટ:}

\begin{verbatim}
               D
    Input o{-{-}{-}||{-}{-}{-}+{-}{-}{-}o Output}
              |     |
              C     R
              |     |
              +{-{-}{-}{-}{-}+}
              |
             GND
\end{verbatim}

\begin{itemize}
\tightlist
\item
  \textbf{વ્યાખ્યા}: DC ઘટક ઉમેરીને વેવફોર્મને ઉપર/નીચે શિફ્ટ કરતી સર્કિટ
\item
  \textbf{પ્રકારો}: પોઝિટિવ ક્લેમ્પર (ઉપર શિફ્ટ), નેગેટિવ ક્લેમ્પર (નીચે શિફ્ટ)
\item
  \textbf{કાર્યપદ્ધતિ}: કેપેસિટર પ્રથમ અર્ધ-ચક્ર દરમિયાન ચાર્જ થાય છે, પછી DC લેવલ
  જાળવે છે
\item
  \textbf{ઉપયોગો}: TV સિંક પલ્સ રિસ્ટોરેશન, પલ્સ મોડ્યુલેશન સર્કિટ્સ, વેવફોર્મ
  પ્રોસેસિંગ
\end{itemize}

\end{solutionbox}
\begin{mnemonicbox}
``CAPS અસર - કેપેસિટર અને ડાયોડ જોડી સિગ્નલને ચોક્કસ DC
લેવલથી શિફ્ટ કરે છે''

\end{mnemonicbox}
\subsection*{પ્રશ્ન 4(ક) [7
માર્ક્સ]}\label{uxaaauxab0uxab6uxaa8-4uxa95-7-uxaaeuxab0uxa95uxab8}

\textbf{OLED નું બાંધકામ, કાર્ય અને એપ્લિકેશન સમજાવો.}

\begin{solutionbox}

\textbf{OLED સ્ટ્રકચર:}

\begin{verbatim}
       +{-{-}{-}{-}{-}{-}{-}{-}{-}{-}{-}{-}{-}{-}{-}{-}+}
       | Cathode (Metal)|
       +{-{-}{-}{-}{-}{-}{-}{-}{-}{-}{-}{-}{-}{-}{-}{-}+}
       | Emissive Layer |
       +{-{-}{-}{-}{-}{-}{-}{-}{-}{-}{-}{-}{-}{-}{-}{-}+}
       |Conductive Layer|
       +{-{-}{-}{-}{-}{-}{-}{-}{-}{-}{-}{-}{-}{-}{-}{-}+}
       |   Anode (ITO)  |
       +{-{-}{-}{-}{-}{-}{-}{-}{-}{-}{-}{-}{-}{-}{-}{-}+}
       |   Substrate    |
       +{-{-}{-}{-}{-}{-}{-}{-}{-}{-}{-}{-}{-}{-}{-}{-}+}
\end{verbatim}

\textbf{OLED બાંધકામ:}

\begin{itemize}
\tightlist
\item
  \textbf{લેયર્સ}: સબસ્ટ્રેટ, એનોડ (ITO), કન્ડક્ટિવ લેયર, એમિસિવ લેયર, કેથોડ
\item
  \textbf{સામગ્રી}: ઇલેક્ટ્રોડ્સ વચ્ચે ઓર્ગેનિક સેમિકન્ડક્ટર સામગ્રી
\item
  \textbf{પ્રકારો}: PMOLED (પેસિવ મેટ્રિક્સ) અને AMOLED (એક્ટિવ મેટ્રિક્સ)
\end{itemize}

\textbf{કાર્યપદ્ધતિ:}

\begin{itemize}
\tightlist
\item
  \textbf{મિકેનિઝમ}: ઇલેક્ટ્રિક કરંટ ઇલેક્ટ્રોલ્યુમિનિસન્સ દ્વારા ઓર્ગેનિક સામગ્રીને
  પ્રકાશ ઉત્સર્જિત કરવા કારણ બને છે
\item
  \textbf{પ્રક્રિયા}: ઇલેક્ટ્રોન્સ અને હોલ્સ એમિસિવ લેયરમાં ફોટોન્સ ઉત્પન્ન કરવા માટે
  રીકોમ્બાઇન થાય છે
\item
  \textbf{કાર્યક્ષમતા}: બેકલાઇટ વગર સીધો પ્રકાશ ઉત્સર્જન, ઉચ્ચ કાર્યક્ષમતા
\end{itemize}

\textbf{ઉપયોગો:}

\begin{itemize}
\tightlist
\item
  \textbf{ડિસ્પ્લે}: સ્માર્ટફોન્સ, ટીવી, વેરેબલ્સ, ડિજિટલ કેમેરા
\item
  \textbf{લાઇટિંગ}: ફ્લેક્સિબલ અને પારદર્શક લાઇટિંગ પેનલ
\item
  \textbf{સાઇનેજ}: ઉચ્ચ-કોન્ટ્રાસ્ટ ડિજિટલ સાઇન અને બિલબોર્ડ
\end{itemize}

\end{solutionbox}
\begin{mnemonicbox}
``OLED ફાયદા - ઓર્ગેનિક સામગ્રી, હલકું ડિઝાઇન, કાર્યક્ષમ
ઓપરેશન, સીધું ઉત્સર્જન, સ્ટનિંગ કોન્ટ્રાસ્ટ''

\end{mnemonicbox}
\subsection*{પ્રશ્ન 4(અ) OR [3
માર્ક્સ]}\label{uxaaauxab0uxab6uxaa8-4uxa85-or-3-uxaaeuxab0uxa95uxab8}

\textbf{LDR પર ટૂંકી નોંધ સમજાવો.}

\begin{solutionbox}

\textbf{LDR સિમ્બોલ અને સ્ટ્રક્ચર:}

\begin{verbatim}
    Symbol              Structure
      ⌒   ⌒             +{-{-}{-}{-}{-}{-}{-}+}
     /     {            |///////|}
    +       +           |///////|
    |       |           +{-{-}{-}{-}{-}{-}{-}+}
    +       +
     {     /}
      ⌒   ⌒
\end{verbatim}

\begin{itemize}
\tightlist
\item
  \textbf{વ્યાખ્યા}: લાઇટ ડિપેન્ડન્ટ રેઝિસ્ટર, એક ફોટોરેઝિસ્ટર જેનો રેઝિસ્ટન્સ પ્રકાશ
  સાથે ઘટે છે
\item
  \textbf{સામગ્રી}: કેડમિયમ સલ્ફાઇડ (CdS) અથવા કેડમિયમ સેલેનાઇડ (CdSe)
\item
  \textbf{સિદ્ધાંત}: ફોટોકંડક્ટિવિટી - પ્રકાશ ઊર્જા ઇલેક્ટ્રોન્સ મુક્ત કરે છે, વાહકતા
  વધારે છે
\item
  \textbf{ઉપયોગો}: લાઇટ સેન્સર્સ, ઓટોમેટિક લાઇટિંગ કંટ્રોલ, કેમેરા એક્સપોઝર સિસ્ટમ
\end{itemize}

\end{solutionbox}
\begin{mnemonicbox}
``DARK રેઝિસ્ટન્સ વધારે છે - ઘટતો પ્રકાશ અને વધતો અંધકાર
રેઝિસ્ટન્સ ઊંચો રાખે છે''

\end{mnemonicbox}
\subsection*{પ્રશ્ન 4(બ) OR [4
માર્ક્સ]}\label{uxaaauxab0uxab6uxaa8-4uxaac-or-4-uxaaeuxab0uxa95uxab8}

\textbf{જરૂરી ડાયાગ્રામ સાથે ડાયોડ ક્લિપર સર્કિટનું વર્ણન કરો.}

\begin{solutionbox}

\textbf{પોઝિટિવ ક્લિપર સર્કિટ:}

\begin{verbatim}
              R
    Input o{-{-}{-}www{-}{-}{-}+{-}{-}{-}o Output}
                    |
                    |
                  \_\_|\_\_
                  {   /}
                   { /}
                    V
                    |
                   GND
\end{verbatim}

\begin{itemize}
\tightlist
\item
  \textbf{વ્યાખ્યા}: થ્રેશોલ્ડ ઉપર/નીચેના ઇનપુટ વેવફોર્મના ભાગોને મર્યાદિત (ક્લિપ)
  કરતી સર્કિટ
\item
  \textbf{પ્રકારો}: પોઝિટિવ ક્લિપર (પોઝિટિવ ક્લિપ), નેગેટિવ ક્લિપર (નેગેટિવ
  ક્લિપ), બાયસ્ડ ક્લિપર
\item
  \textbf{કાર્યપદ્ધતિ}: જ્યારે સિગ્નલ થ્રેશોલ્ડને વટાવે છે ત્યારે ડાયોડ કન્ડક્ટ કરે છે,
  આઉટપુટને મર્યાદિત કરે છે
\item
  \textbf{ઉપયોગો}: વેવફોર્મ શેપિંગ, પ્રોટેક્શન સર્કિટ્સ, સિગ્નલ કન્ડિશનિંગ
\end{itemize}

\end{solutionbox}
\begin{mnemonicbox}
``CLIP તરંગો - સર્કિટ ડાયોડ કન્ડક્શનનો ઉપયોગ કરીને ઇનપુટ
પીક્સને મર્યાદિત કરે છે''

\end{mnemonicbox}
\subsection*{પ્રશ્ન 4(ક) OR [7
માર્ક્સ]}\label{uxaaauxab0uxab6uxaa8-4uxa95-or-7-uxaaeuxab0uxa95uxab8}

\textbf{હાફ વેવ અને ફુલ વેવ વોલ્ટેજ ડબલર સમજાવો.}

\begin{solutionbox}

\textbf{હાફ-વેવ વોલ્ટેજ ડબલર:}

\begin{verbatim}
             D1
    AC o{-{-}{-}{-}{-}{-}||{-}{-}{-}{-}{-}{-}{-}+{-}{-}{-}{-}o Output}
    Input               |    (+2Vp)
               |        |
               C1       C2
               |        |
              GND      GND
\end{verbatim}

\textbf{ફુલ-વેવ વોલ્ટેજ ડબલર:}

\begin{verbatim}
             D1         
    AC o{-{-}{-}{-}{-}{-}||{-}{-}{-}{-}{-}{-}{-}+{-}{-}{-}{-}o Output}
    Input     |         |    (+2Vp)
              |         |
              C1        C2
              |         |
              |    D2   |
              +{-{-}{-}||{-}{-}{-}+}
              |
             GND
\end{verbatim}

\textbf{હાફ-વેવ વોલ્ટેજ ડબલર:}

\begin{itemize}
\tightlist
\item
  \textbf{ઓપરેશન}: નેગેટિવ હાફ સાયકલ દરમિયાન, C1 પીક વોલ્ટેજ સુધી ચાર્જ થાય છે;
  પોઝિટિવ સાયકલ દરમિયાન, આઉટપુટ 2Vp બને છે
\item
  \textbf{આઉટપુટ}: પીક વેલ્યુ ઇનપુટ પીકના બમણા સાથે પલ્સેટિંગ DC
\item
  \textbf{રિપલ}: ઉચ્ચ રિપલ સામગ્રી
\item
  \textbf{કાર્યક્ષમતા}: ફુલ-વેવ કરતાં નીચી
\end{itemize}

\textbf{ફુલ-વેવ વોલ્ટેજ ડબલર:}

\begin{itemize}
\tightlist
\item
  \textbf{ઓપરેશન}: બંને હાફ સાયકલ્સ આઉટપુટમાં યોગદાન આપે છે, દરેક કેપેસિટર વૈકલ્પિક
  સાયકલ્સ દરમિયાન ચાર્જ થાય છે
\item
  \textbf{આઉટપુટ}: પીક વેલ્યુ ઇનપુટ પીકના બમણા સાથે વધુ સ્મૂધ DC
\item
  \textbf{રિપલ}: ઓછી રિપલ સામગ્રી
\item
  \textbf{કાર્યક્ષમતા}: હાફ-વેવ કરતાં ઉચ્ચ
\end{itemize}

\textbf{ઉપયોગો:}

\begin{itemize}
\tightlist
\item
  \textbf{ઉચ્ચ વોલ્ટેજ જનરેશન}: CRT ડિસ્પ્લે, ફોટોમલ્ટિપ્લાયર્સ
\item
  \textbf{પાવર સપ્લાય}: ઓછા કરંટ, ઉચ્ચ વોલ્ટેજ એપ્લિકેશન્સ
\item
  \textbf{કેસ્કેડ કનેક્શન}: ડબલિંગ ઉપરાંત વોલ્ટેજ મલ્ટિપ્લિકેશન માટે
\end{itemize}

\end{solutionbox}
\begin{mnemonicbox}
``CHASE 2V - કેપેસિટર્સ 2\times વોલ્ટેજ ઉત્પન્ન કરવા માટે
ઓલ્ટરનેટિંગ સપ્લાય એનર્જી રાખે છે''

\end{mnemonicbox}
\subsection*{પ્રશ્ન 5(અ) [3
માર્ક્સ]}\label{uxaaauxab0uxab6uxaa8-5uxa85-3-uxaaeuxab0uxa95uxab8}

\textbf{IC નો ઉપયોગ કરીને +5v પાવર સપ્લાય માટે સર્કિટ ડાયાગ્રામ દોરો અને ટૂંકમાં
સમજાવો.}

\begin{solutionbox}

\textbf{7805 નો ઉપયોગ કરીને 5V પાવર સપ્લાય:}

\begin{verbatim}
        D1    D2
    o{-{-}{-}||{-}{-}{-}||{-}{-}{-}+{-}{-}{-}{-}{-}+{-}{-}{-}{-}{-}{-}{-}{-}o}
AC       |         |     |        +5V
Input    +{-{-}||{-}{-}+ |    7805      Output}
         |  D3   | |     |
         +{-{-}||{-}{-}+ |     |}
         |  D4     C1    C2
         |         |     |
    o{-{-}{-}{-}+{-}{-}{-}{-}{-}{-}{-}{-}{-}+{-}{-}{-}{-}{-}+{-}{-}{-}{-}{-}{-}{-}{-}o}
                  GND
\end{verbatim}

\begin{itemize}
\tightlist
\item
  \textbf{ઘટકો}: બ્રિજ રેક્ટિફાયર (D1-D4), ફિલ્ટર કેપેસિટર (C1), 7805 રેગ્યુલેટર,
  આઉટપુટ કેપેસિટર (C2)
\item
  \textbf{કાર્યપદ્ધતિ}: રેક્ટિફાયર દ્વારા AC ને DC માં રૂપાંતરિત, C1 દ્વારા ફિલ્ટર,
  7805 દ્વારા ચોક્કસ 5V માં નિયમિત
\item
  \textbf{વિશેષતાઓ}: શોર્ટ-સર્કિટ પ્રોટેક્શન, થર્મલ શટડાઉન, 1A સુધી કરંટ ક્ષમતા
\item
  \textbf{ઉપયોગો}: ડિજિટલ સર્કિટ્સ, માઇક્રોકન્ટ્રોલર્સ, ઇલેક્ટ્રોનિક્સ પ્રોજેક્ટ્સ
\end{itemize}

\end{solutionbox}
\begin{mnemonicbox}
``FIRM વોલ્ટેજ - ફિલ્ટર્ડ ઇનપુટ, 7805 દ્વારા રેગ્યુલેટેડ સ્થિર
વોલ્ટેજ બનાવે છે''

\end{mnemonicbox}
\subsection*{પ્રશ્ન 5(બ) [4
માર્ક્સ]}\label{uxaaauxab0uxab6uxaa8-5uxaac-4-uxaaeuxab0uxa95uxab8}

\textbf{પાવર સપ્લાયના સંદર્ભમાં લોડ રેગ્યુલેશન અને લાઇન રેગ્યુલેશનની ચર્ચા કરો.}

\begin{solutionbox}

\textbf{રેગ્યુલેશન પરફોર્મન્સ કર્વ્સ:}

\begin{verbatim}
    Vout           Vout
     \^{              \^{}}
     |              |
     |{-{-}            |{-}{-}}
     |  {           |  }
     |   { Load     |    Line}
     |    {         |    }
     +{-{-}{-}{-}{-}{-}{-}      +{-}{-}{-}{-}{-}{-}{-}}
           Iload           Vin
\end{verbatim}

\textbf{લોડ રેગ્યુલેશન:}

\begin{itemize}
\tightlist
\item
  \textbf{વ્યાખ્યા}: લોડ કરંટ પરિવર્તન છતાં સ્થિર આઉટપુટ વોલ્ટેજ જાળવવાની ક્ષમતા
\item
  \textbf{સૂત્ર}: \% લોડ રેગ્યુલેશન = ((VNL - VFL)/VFL) \times 100
\item
  \textbf{મહત્વ}: વિવિધ લોડ માંગ માટે સ્થિર વોલ્ટેજ સુનિશ્ચિત કરે છે
\item
  \textbf{આદર્શ મૂલ્ય}: 0\% (લોડ પરિવર્તન સાથે આઉટપુટ વોલ્ટેજમાં કોઈ ફેરફાર નહીં)
\end{itemize}

\textbf{લાઇન રેગ્યુલેશન:}

\begin{itemize}
\tightlist
\item
  \textbf{વ્યાખ્યા}: ઇનપુટ વોલ્ટેજમાં ફેરફાર છતાં સ્થિર આઉટપુટ જાળવવાની ક્ષમતા
\item
  \textbf{સૂત્ર}: \% લાઇન રેગ્યુલેશન = (ΔVout/ΔVin) \times 100
\item
  \textbf{મહત્વ}: મેઇન્સ વોલ્ટેજ ફ્લક્ચ્યુએશનથી સર્કિટ્સને બચાવે છે
\item
  \textbf{આદર્શ મૂલ્ય}: 0\% (ઇનપુટ પરિવર્તન સાથે આઉટપુટ વોલ્ટેજમાં કોઈ ફેરફાર નહીં)
\end{itemize}

\end{solutionbox}
\begin{mnemonicbox}
``LIVER સ્વાસ્થ્ય - ઇનપુટ વેરિએશન માટે લાઇન રેગ્યુલેશન, બાહ્ય
રેઝિસ્ટન્સ ફેરફારો માટે લોડ રેગ્યુલેશન''

\end{mnemonicbox}
\subsection*{પ્રશ્ન 5(ક) [7
માર્ક્સ]}\label{uxaaauxab0uxab6uxaa8-5uxa95-7-uxaaeuxab0uxa95uxab8}

\textbf{સર્કિટ ડાયાગ્રામ સાથે LM317 નો ઉપયોગ કરીને એડજસ્ટેબલ વોલ્ટેજ રેગ્યુલેટર
સમજાવો.}

\begin{solutionbox}

\textbf{LM317 એડજસ્ટેબલ રેગ્યુલેટર સર્કિટ:}

\begin{verbatim}
                   R1
     +Vin o{-{-}{-}+{-}{-}{-}{-}www{-}{-}{-}{-}+}
              |           |
              |    ADJ    |
              |  +{-{-}{-}{-}{-}+  |}
              +{-{-}| 317 |{-}{-}+{-}{-}o +Vout}
                 |     |     |
                 +{-{-}{-}{-}{-}+     |}
                             R2
                             |
                            GND
\end{verbatim}

\textbf{કાર્યપદ્ધતિનો સિદ્ધાંત:}

\begin{itemize}
\tightlist
\item
  \textbf{મૂળભૂત ઓપરેશન}: LM317 આઉટપુટ અને એડજસ્ટમેન્ટ પિન વચ્ચે 1.25V જાળવે છે
\item
  \textbf{આઉટપુટ વોલ્ટેજ}: Vout = 1.25V(1 + R2/R1) + IADJ(R2)
\item
  \textbf{સરળીકૃત સૂત્ર}: Vout \approx 1.25V(1 + R2/R1) (IADJ ખૂબ નાનો હોવાથી)
\item
  \textbf{એડજસ્ટમેન્ટ રેન્જ}: ઇનપુટ વોલ્ટેજના આધારે 1.25V થી 37V
\end{itemize}

\textbf{વિશેષતાઓ:}

\begin{itemize}
\tightlist
\item
  \textbf{કરંટ ક્ષમતા}: 1.5A સુધીનો આઉટપુટ કરંટ
\item
  \textbf{પ્રોટેક્શન}: આંતરિક થર્મલ ઓવરલોડ અને શોર્ટ સર્કિટ પ્રોટેક્શન
\item
  \textbf{ફાયદા}: સરળ ડિઝાઇન, ન્યુનતમ બાહ્ય ઘટકો, સ્થિર આઉટપુટ
\item
  \textbf{ઉપયોગો}: વેરિએબલ પાવર સપ્લાય, બેટરી ચાર્જર, કસ્ટમ વોલ્ટેજ રેગ્યુલેટર્સ
\end{itemize}

\end{solutionbox}
\begin{mnemonicbox}
``VAIR નિયંત્રણ - વેરિએબલ એડજસ્ટેબલ ઇન્ટિગ્રેટેડ રેગ્યુલેટર
વોલ્ટેજને ચોક્કસપણે નિયંત્રિત કરે છે''

\end{mnemonicbox}
\subsection*{પ્રશ્ન 5(અ) OR [3
માર્ક્સ]}\label{uxaaauxab0uxab6uxaa8-5uxa85-or-3-uxaaeuxab0uxa95uxab8}

\textbf{સૌર બેટરી ચાર્જર સર્કિટની કાર્યપદ્ધતિ સમજાવો.}

\begin{solutionbox}

\textbf{સૌર બેટરી ચાર્જર બ્લોક ડાયાગ્રામ:}

\begin{verbatim}
flowchart LR
    A[સોલર પેનલ] {-{-} B[ચાર્જ કંટ્રોલર]}
    B {-{-} C[બેટરી]}
    C {-{-} D[લોડ/આઉટપુટ]}
\end{verbatim}

\begin{itemize}
\tightlist
\item
  \textbf{ઘટકો}: સોલર પેનલ, ચાર્જ કંટ્રોલર, બેટરી, પ્રોટેક્શન સર્કિટ્સ
\item
  \textbf{કાર્યપદ્ધતિનો સિદ્ધાંત}: સોલર પેનલ DC જનરેટ કરે છે, કંટ્રોલર ચાર્જિંગ કરંટને
  નિયંત્રિત કરે છે
\item
  \textbf{ચાર્જ ફેઝ}: બલ્ક ચાર્જિંગ (સ્થિર કરંટ), એબ્સોર્પ્શન (સ્થિર વોલ્ટેજ), ફ્લોટ
  (જાળવણી)
\item
  \textbf{પ્રોટેક્શન વિશેષતાઓ}: ઓવરચાર્જ પ્રોટેક્શન, ડીપ ડિસ્ચાર્જ પ્રિવેન્શન, રિવર્સ
  પોલારિટી
\end{itemize}

\end{solutionbox}
\begin{mnemonicbox}
``SCBL સિસ્ટમ - સોલર પેનલ સૂર્યપ્રકાશને કન્વર્ટ કરે છે, બેટરી
સંગ્રહ કરે છે, લોડ વપરાશ કરે છે''

\end{mnemonicbox}
\subsection*{પ્રશ્ન 5(બ) OR [4
માર્ક્સ]}\label{uxaaauxab0uxab6uxaa8-5uxaac-or-4-uxaaeuxab0uxa95uxab8}

\textbf{UPS ની કાર્યપદ્ધતિ સમજાવો.}

\begin{solutionbox}

\textbf{UPS બ્લોક ડાયાગ્રામ:}

\begin{verbatim}
    +{-{-}{-}{-}{-}{-}+    +{-}{-}{-}{-}{-}{-}{-}+    +{-}{-}{-}{-}{-}{-}{-}{-}+}
    |      |    |       |    |        |
AC{-{-}+ Rect +{-}{-}{-}{-}+ Batt. +{-}{-}{-}{-}+ Invert +{-}{-}{-}AC}
    |      |    |       |    |        |
    +{-{-}{-}{-}{-}{-}+    +{-}{-}{-}{-}{-}{-}{-}+    +{-}{-}{-}{-}{-}{-}{-}{-}+}
       |                         |
       +{-{-}{-}{-}{-}{-}{-}{-}{-}+{-}{-}{-}{-}{-}{-}{-}{-}{-}{-}{-}{-}{-}{-}{-}+}
                 |
              Control
              Circuit
\end{verbatim}

\begin{itemize}
\tightlist
\item
  \textbf{વ્યાખ્યા}: અનઇન્ટરપ્ટિબલ પાવર સપ્લાય મુખ્ય સપ્લાય નિષ્ફળતા દરમિયાન બેકઅપ
  પાવર પ્રદાન કરે છે
\item
  \textbf{પ્રકારો}: ઓફલાઇન (સ્ટેન્ડબાય), લાઇન-ઇન્ટરેક્ટિવ, ઓનલાઇન (ડબલ કન્વર્ઝન)
\item
  \textbf{ઘટકો}: રેક્ટિફાયર, બેટરી, ઇન્વર્ટર, કંટ્રોલ સર્કિટ્રી, ટ્રાન્સફર સ્વિચ
\item
  \textbf{ઓપરેશન}: સામાન્ય રીતે ફિલ્ટર કરેલ મેઇન્સ પાવર પસાર કરે છે, આઉટેજ દરમિયાન
  બેટરી પર સ્વિચ કરે છે
\end{itemize}

\end{solutionbox}
\begin{mnemonicbox}
``PRIME પાવર - મેઇન્સ ઇલેક્ટ્રિસિટી સમસ્યાઓ દરમિયાન પાવર
અખંડિત રહે છે''

\end{mnemonicbox}
\subsection*{પ્રશ્ન 5(ક) OR [7
માર્ક્સ]}\label{uxaaauxab0uxab6uxaa8-5uxa95-or-7-uxaaeuxab0uxa95uxab8}

\textbf{SMPS બ્લોક ડાયાગ્રામ તેના ફાયદા અને ગેરફાયદા સાથે દોરો અને સમજાવો.}

\begin{solutionbox}

\textbf{SMPS બ્લોક ડાયાગ્રામ:}

\begin{verbatim}
flowchart LR
    A[AC ઇનપુટ] {-{-} B[EMI ફિલ્ટર]}
    B {-{-} C[રેક્ટિફાયર]}
    C {-{-} D[હાઈ{-}ફ્રિક્વન્સી સ્વિચ]}
    D {-{-} E[ટ્રાન્સફોર્મર]}
    E {-{-} F[આઉટપુટ રેક્ટિફાયર]}
    F {-{-} G[ફિલ્ટર]}
    G {-{-} H[DC આઉટપુટ]}
    I[ફીડબેક] {-{-} J[કંટ્રોલ સર્કિટ]}
    J {-{-} D}
\end{verbatim}

\textbf{કાર્યપદ્ધતિનો સિદ્ધાંત:}

\begin{itemize}
\tightlist
\item
  \textbf{ઇનપુટ સ્ટેજ}: AC રેક્ટિફાયર દ્વારા અનરેગ્યુલેટેડ DC માં રૂપાંતરિત
\item
  \textbf{સ્વિચિંગ સ્ટેજ}: હાઈ-ફ્રિક્વન્સી ટ્રાન્ઝિસ્ટર્સ DC ને પલ્સમાં કાપે છે
\item
  \textbf{ટ્રાન્સફોર્મર}: ઉચ્ચ આવર્તન પર આઇસોલેટ અને વોલ્ટેજ ટ્રાન્સફોર્મ કરે છે
\item
  \textbf{આઉટપુટ સ્ટેજ}: ક્લીન DC ઉત્પન્ન કરવા માટે રેક્ટિફાય અને ફિલ્ટર કરે છે
\item
  \textbf{ફીડબેક લૂપ}: આઉટપુટને મોનિટર કરે છે અને નિયમન માટે સ્વિચિંગ એડજસ્ટ કરે છે
\end{itemize}

\textbf{ફાયદા:}

\begin{itemize}
\tightlist
\item
  \textbf{કાર્યક્ષમતા}: લિનિયર સપ્લાય માટે 30-60\% ની સરખામણીએ 70-90\%
\item
  \textbf{કદ/વજન}: ઉચ્ચ-આવર્તન ઓપરેશનને કારણે નાના ટ્રાન્સફોર્મર
\item
  \textbf{હીટ જનરેશન}: ઓછો પાવર ડિસિપેશન, ઘટાડેલી કૂલિંગ જરૂરિયાતો
\item
  \textbf{વાઇડ ઇનપુટ રેન્જ}: વિશાળ ઇનપુટ વોલ્ટેજ વેરિએશન પર ઓપરેટ કરી શકે છે
\end{itemize}

\textbf{ગેરફાયદા:}

\begin{itemize}
\tightlist
\item
  \textbf{જટિલતા}: લિનિયર સપ્લાય કરતાં વધુ જટિલ ડિઝાઇન
\item
  \textbf{EMI/RFI}: ઇલેક્ટ્રોમેગ્નેટિક ઇન્ટરફેરન્સ ઉત્પન્ન કરે છે
\item
  \textbf{નોઇઝ}: સ્વિચિંગ ઓપરેશનને કારણે ઉચ્ચ આઉટપુટ નોઇઝ
\item
  \textbf{ખર્ચ}: ઓછી-પાવર એપ્લિકેશન્સ માટે વધુ ખર્ચાળ
\end{itemize}

\end{solutionbox}
\begin{mnemonicbox}
``FISH ફેક્ટર્સ - ફ્રીક્વન્સી સ્વિચિંગ, આઇસોલેશન, નાનું કદ, ઉચ્ચ
કાર્યક્ષમતા SMPS ના ફાયદા છે''

\end{mnemonicbox}
\subsection*{મુખ્ય કોન્સેપ્ટ્સનો
સારાંશ}\label{uxaaeuxa96uxaaf-uxa95uxaa8uxab8uxaaauxa9fuxab8uxaa8-uxab8uxab0uxab6}

\subsubsection{ટ્રાન્ઝિસ્ટર બાયસિંગ અને
સ્ટેબિલિટી}\label{uxa9fuxab0uxaa8uxa9duxab8uxa9fuxab0-uxaacuxaafuxab8uxa97-uxa85uxaa8-uxab8uxa9fuxaacuxab2uxa9f}

\begin{itemize}
\tightlist
\item
  \textbf{બાયસિંગ પદ્ધતિઓ}: ફિક્સ્ડ બાયસ, કલેક્ટર ફીડબેક, એમિટર બાયસ, વોલ્ટેજ
  ડિવાઇડર (સૌથી સ્થિર)
\item
  \textbf{થર્મલ સ્ટેબિલિટી}: થર્મલ રનઅવે અટકાવવા માટે એમિટર રેઝિસ્ટર્સ, વોલ્ટેજ
  ડિવાઇડર બાયસ, હીટ સિંક્સનો ઉપયોગ
\item
  \textbf{સ્ટેબિલિટી ફેક્ટર (S)}: નીચું મૂલ્ય તાપમાન પરિવર્તન સામે વધુ સારી સ્થિરતા
  દર્શાવે છે
\end{itemize}

\subsubsection{એમ્પ્લીફાયર
પેરામીટર્સ}\label{uxa8fuxaaeuxaaauxab2uxaabuxaafuxab0-uxaaauxab0uxaaeuxa9fuxab0uxab8}

\begin{itemize}
\tightlist
\item
  \textbf{CE એમ્પ્લીફાયર}: ઉચ્ચ વોલ્ટેજ ગેઇન (50-500), મધ્યમ ઇનપુટ ઇમ્પિડન્સ, 180^\circ
  ફેઝ શિફ્ટ
\item
  \textbf{h-પેરામીટર્સ}: h11 (ઇનપુટ ઇમ્પિડન્સ), h21 (કરંટ ગેઇન), h12 (રિવર્સ
  વોલ્ટેજ રેશિયો), h22 (આઉટપુટ એડમિટન્સ)
\item
  \textbf{ફ્રીક્વન્સી રિસ્પોન્સ}: નિમ્
\end{itemize}

\subsubsection{ફ્રીક્વન્સી
રિસ્પોન્સ}\label{uxaabuxab0uxa95uxab5uxaa8uxab8-uxab0uxab8uxaaauxaa8uxab8}

\begin{itemize}
\tightlist
\item
  \textbf{નિમ્ન આવર્તનો પર}: કપલિંગ કેપેસિટર્સની અસરોને કારણે ગેઇન ઘટે છે
\item
  \textbf{મધ્ય આવર્તનો પર}: મહત્તમ ગેઇન ક્ષેત્ર, સમતલ પ્રતિસાદ
\item
  \textbf{ઉચ્ચ આવર્તનો પર}: આંતરિક કેપેસિટન્સ અને મિલર ઇફેક્ટને કારણે ગેઇન ઘટે છે
\end{itemize}

\subsubsection{કપલિંગ
પદ્ધતિઓ}\label{uxa95uxaaauxab2uxa97-uxaaauxaa6uxaa7uxaa4uxa93}

\begin{itemize}
\tightlist
\item
  \textbf{RC કપલિંગ}: સરળ, ઓછી કિંમત, સારો આવર્તન પ્રતિસાદ (ખૂબ નિમ્ન આવર્તનો
  સિવાય)
\item
  \textbf{ટ્રાન્સફોર્મર કપલિંગ}: સારું ઇમ્પિડન્સ મેચિંગ, ઉત્તમ કાર્યક્ષમતા, મોટું અને
  ખર્ચાળ
\item
  \textbf{ડાયરેક્ટ કપલિંગ}: ઉત્તમ નિમ્ન-આવર્તન પ્રતિસાદ, DC બાયસ સમસ્યાઓ,
  ઇન્ટિગ્રેટેડ સર્કિટ્સમાં વપરાય છે
\end{itemize}

\subsubsection{પ્રેક્ટિકલ
એપ્લિકેશન્સ}\label{uxaaauxab0uxa95uxa9fuxa95uxab2-uxa8fuxaaauxab2uxa95uxab6uxaa8uxab8}

\begin{itemize}
\tightlist
\item
  \textbf{ક્લિપર \& ક્લેમ્પર}: વેવફોર્મ શેપિંગ, મર્યાદિત, લેવલ શિફ્ટિંગ સર્કિટ્સ
\item
  \textbf{વોલ્ટેજ મલ્ટિપ્લાયર્સ}: ઓછા AC ઇનપુટથી ઉચ્ચ DC વોલ્ટેજ જનરેટ કરે છે (ડબલર,
  ટ્રિપલર, વગેરે)
\item
  \textbf{ડાર્લિંગ્ટન પેર}: પાવર એપ્લિકેશન્સ માટે સુપર-હાઈ કરંટ ગેઇન કોન્ફિગરેશન
\item
  \textbf{OLED ડિસ્પ્લે}: ઉચ્ચ કોન્ટ્રાસ્ટ, ઊર્જા કાર્યક્ષમતા સાથે ઓર્ગેનિક
  લાઇટ-એમિટિંગ ડાયોડ
\end{itemize}

\subsubsection{પાવર સપ્લાય
સર્કિટ્સ}\label{uxaaauxab5uxab0-uxab8uxaaauxab2uxaaf-uxab8uxab0uxa95uxa9fuxab8}

\begin{itemize}
\tightlist
\item
  \textbf{વોલ્ટેજ રેગ્યુલેટર્સ}: 78xx સિરીઝ (પોઝિટિવ), 79xx સિરીઝ (નેગેટિવ),
  LM317 (એડજસ્ટેબલ)
\item
  \textbf{SMPS}: નાના કદ પરંતુ વધુ જટિલતા સાથે ઉચ્ચ-કાર્યક્ષમતા સ્વિચ-મોડ પાવર
  સપ્લાય
\item
  \textbf{UPS}: બેટરી-ઇન્વર્ટર સિસ્ટમનો ઉપયોગ કરીને આઉટેજ દરમિયાન બેકઅપ પાવર આપે
  છે
\item
  \textbf{સોલર ચાર્જર્સ}: ઓવરચાર્જ પ્રોટેક્શન સાથે બેટરી ચાર્જ કરવા માટે સૌર ઊર્જાને
  રૂપાંતરિત કરે છે
\end{itemize}

\subsection*{યાદ રાખવા માટે મહત્વપૂર્ણ
સૂત્રો}\label{uxaafuxaa6-uxab0uxa96uxab5-uxaaeuxa9f-uxaaeuxab9uxaa4uxab5uxaaauxab0uxaa3-uxab8uxaa4uxab0}

{\def\LTcaptype{none} % do not increment counter
\begin{longtable}[]{@{}lll@{}}
\toprule\noalign{}
પેરામીટર & સૂત્ર & વર્ણન \\
\midrule\noalign{}
\endhead
\bottomrule\noalign{}
\endlastfoot
વોલ્ટેજ ગેઇન (Av) & Vout/Vin & આઉટપુટથી ઇનપુટ વોલ્ટેજનો ગુણોત્તર \\
કરંટ ગેઇન (Ai) & Ic/Ib & કલેક્ટરથી બેઝ કરંટનો ગુણોત્તર \\
બેન્ડવિડ્થ & f2 - f1 & કટઓફ પોઇન્ટ્સ વચ્ચેની આવર્તન રેન્જ \\
લોડ રેગ્યુલેશન & ((VNL-VFL)/VFL)\times100\% & લોડ ચેન્જ સાથે વોલ્ટેજ સ્થિરતા \\
લાઇન રેગ્યુલેશન & (ΔVout/ΔVin)\times100\% & ઇનપુટ ચેન્જ સાથે વોલ્ટેજ સ્થિરતા \\
સ્ટેબિલિટી ફેક્ટર (S) & ΔIC/ΔICBO & લીકેજ સામે કલેક્ટર કરંટમાં ફેરફાર \\
LM317 આઉટપુટ & 1.25V(1+R2/R1) & એડજસ્ટેબલ રેગ્યુલેટર આઉટપુટ વોલ્ટેજ \\
રેઝોનન્ટ ફ્રીક્વન્સી & 1/(2π\sqrtLC) & ટ્યુન્ડ એમ્પ્લીફાયર રેઝોનન્સ પોઇન્ટ \\
\end{longtable}
}

\subsection*{ઇલેક્ટ્રોનિક સર્કિટ્સ માટે પરીક્ષા
ટિપ્સ}\label{uxa87uxab2uxa95uxa9fuxab0uxaa8uxa95-uxab8uxab0uxa95uxa9fuxab8-uxaaeuxa9f-uxaaauxab0uxa95uxab7-uxa9fuxaaauxab8}

\begin{enumerate}
\tightlist
\item
  \textbf{પહેલા બેઝિક્સ દોરો}: વિગતો ઉમેરતા પહેલા હંમેશા બેઝિક સર્કિટ ડાયાગ્રામથી
  શરૂઆત કરો
\item
  \textbf{ધ્રુવીયતાઓ યાદ રાખો}: વોલ્ટેજ ધ્રુવીયતા અને કરંટ દિશાઓ પર ધ્યાન આપો
\item
  \textbf{તુલના કોષ્ટકમાં કરો}: માહિતીને વ્યવસ્થિત કરવા માટે તુલના પ્રશ્નો માટે
  કોષ્ટકનો ઉપયોગ કરો
\item
  \textbf{પ્રેક્ટિકલ ઉપયોગો પર ધ્યાન કેન્દ્રિત કરો}: સૈદ્ધાંતિક ખ્યાલોને
  વાસ્તવિક-વિશ્વ એપ્લિકેશન્સ સાથે જોડો
\item
  \textbf{નંબરો જાણો}: ટિપિકલ મૂલ્યો (ગેઇન્સ, ઇમ્પિડન્સ, વોલ્ટેજ) યાદ રાખો
\item
  \textbf{નેમોનિક્સનો ઉપયોગ કરો}: જટિલ સંકલ્પનાઓ અને સૂત્રો માટે મેમરી એઇડ્સ બનાવો
\end{enumerate}

\subsection*{સામાન્ય ભૂલો
ટાળો}\label{uxab8uxaaeuxaa8uxaaf-uxaaduxab2-uxa9fuxab3}

\begin{enumerate}
\tightlist
\item
  \textbf{બાયસિંગ મિક્સ અપ}: વિવિધ બાયસિંગ પદ્ધતિઓ અને તેમના સ્ટેબિલિટી ફેક્ટર્સને
  ભ્રમિત ન કરો
\item
  \textbf{પેરામીટર કન્ફ્યુઝન}: h-પેરામીટર્સની વ્યાખ્યાઓ સ્પષ્ટ અને અલગ રાખો
\item
  \textbf{સાઇન એરર્સ}: કોમન એમિટર કોન્ફિગરેશનમાં ફેઝ ઇન્વર્ઝન્સ (180^\circ શિફ્ટ) યાદ
  રાખો
\item
  \textbf{રેગ્યુલેશન ફોર્મ્યુલા}: લોડ રેગ્યુલેશન અને લાઇન રેગ્યુલેશન સૂત્રો મિક્સ ન કરો
\item
  \textbf{ડાયાગ્રામ્સ ઓવરકોમ્પ્લિકેટિંગ}: સર્કિટ આરેખો સરળ અને મુખ્ય ઘટકો પર કેન્દ્રિત
  રાખો
\end{enumerate}

\subsection*{ક્વિક રેફરન્સ: કોમ્પોનન્ટ
સિમ્બોલ}\label{uxa95uxab5uxa95-uxab0uxaabuxab0uxaa8uxab8-uxa95uxaaeuxaaauxaa8uxaa8uxa9f-uxab8uxaaeuxaacuxab2}

\begin{verbatim}
Transistor (NPN)    Transistor (PNP)    Diode        LED
    C                   C                 A            A
    |                   |                 |            |
    |                   |                 +-|>|-+      +-|>|-+
    B---|               B---|             K            K  \/
    |                   |
    E                   E

Resistor     Capacitor    Inductor    Transformer
  --www--     --||--      --OOOO--    --OOOO--
                                       --OOOO--
\end{verbatim}


\end{document}
