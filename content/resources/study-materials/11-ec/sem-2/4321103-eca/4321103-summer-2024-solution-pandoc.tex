\documentclass[10pt,a4paper]{article}

% content/resources/templates/preamble.tex
\usepackage[margin=0.6in]{geometry}
\author{Milav Dabgar}
\usepackage{amsmath,amssymb,amsthm}
\usepackage{booktabs}
\usepackage{multirow}
\usepackage{xcolor}
\usepackage{tcolorbox}
\tcbuselibrary{breakable,skins}
\usepackage[colorlinks=true,linkcolor=blue]{hyperref}
\usepackage{titlesec}
\usepackage{enumitem}
\usepackage{tikz}
\usepackage{pgfplots}
\usepackage{circuitikz}
\usepackage[version=4]{mhchem}
\usepackage{longtable}
\usepackage{array}
\usepackage{float}
\usepackage{caption}
\usepackage{listings}

\lstset{
  basicstyle=\small\ttfamily,
  breaklines=true,
  breakatwhitespace=false,
  postbreak=\mbox{\textcolor{red}{$\hookrightarrow$}\space},
  float=false,
  numbers=left,
  numberstyle=\tiny\color{gray},
  numbersep=10pt,
  xleftmargin=2em,
  keywordstyle=\color{blue},
  commentstyle=\color{green!60!black},
  stringstyle=\color{purple},
  backgroundcolor=\color{gray!5},
  showstringspaces=false,
  tabsize=2,
  captionpos=b,
  keepspaces=true,
  columns=flexible
}

\pgfplotsset{compat=1.18}
\usetikzlibrary{shapes,arrows,positioning,calc,patterns,decorations.pathmorphing,decorations.markings,arrows.meta}

% Color scheme
\definecolor{headcolor}{RGB}{0,102,204}
\definecolor{keycolor}{RGB}{220,20,60}
\definecolor{solutioncolor}{RGB}{34,139,34}
\definecolor{mnemoniccolor}{RGB}{148,0,211}
\definecolor{codecolor}{RGB}{0,0,100}

% Spacing
\setlength{\parskip}{3pt}
\setlist[itemize]{nosep}
\setlist[enumerate]{nosep}

% Title formatting
\titleformat{\section}{\Large\bfseries\color{headcolor}}{\thesection}{1em}{}
\titleformat{\subsection}{\large\bfseries\color{headcolor}}{\thesubsection}{1em}{}

% Pandoc tightlist compatibility
\providecommand{\tightlist}{%
  \setlength{\itemsep}{0pt}\setlength{\parskip}{0pt}}

% Pandoc longtable compatibility
\newcounter{none}
\def\thenone{}


% content/resources/templates/english-boxes.tex
% This file is currently empty - it exists to maintain consistency with the import structure.
% Add custom environments here if needed in the future.


\begin{document}

\begin{center}
{\Huge\bfseries\color{headcolor} Subject Name Solutions}\\[5pt]
{\LARGE 4321103 -- Summer 2024}\\[3pt]
{\large Semester 1 Study Material}\\[3pt]
{\normalsize\textit{Detailed Solutions and Explanations}}
\end{center}

\vspace{10pt}

\subsection*{Question 1(a) [3 marks]}\label{q1a}

\textbf{Explain amplifier parameters Ai, Ri and Ro for CE
configuration.}

\begin{solutionbox}
In Common Emitter (CE) configuration, the key
parameters are:

\textbf{Diagram:}

\begin{verbatim}
   +Vcc
     |
     R
     |
     |C
B{-{-}{-}{-}|{-}{-}{-}{-}+{-}{-}{-}{-}Output}
     |    |
     |   RC
  RB |    |
     |    |
  {-{-}{-}|    |{-}{-}{-}}
 |   |    |   |
 |   |    |   |
 +{-{-}{-}+{-}{-}{-}{-}+{-}{-}{-}+}
     |
     |
    GND
\end{verbatim}

\begin{itemize}
\tightlist
\item
  \textbf{Current Gain (Ai)}: Ratio of output current to input current
  (Ic/Ib), typically 50-200 in CE
\item
  \textbf{Input Resistance (Ri)}: Opposition to input current at base
  terminal, ranges from 1-2kΩ in CE
\item
  \textbf{Output Resistance (Ro)}: Opposition at collector terminal,
  typically 50kΩ in CE
\end{itemize}

\end{solutionbox}
\begin{mnemonicbox}
``CIR parameters - Current gain, Input resistance,
and output Resistance determine amplifier performance''

\end{mnemonicbox}
\subsection*{Question 1(b) [4 marks]}\label{q1b}

\textbf{Write short-note on heat sink.}

\begin{solutionbox}

\textbf{Diagram:}

\begin{verbatim}
                        Fins
            |‾‾‾|‾‾‾|‾‾‾|‾‾‾|‾‾‾|‾‾‾|‾‾‾|‾‾‾|
            |   |   |   |   |   |   |   |   |
+{-{-}{-}{-}{-}{-}{-}{-}{-}{-}{-}+   |   |   |   |   |   |   |   |}
| Transistor|\_\_\_|\_\_\_|\_\_\_|\_\_\_|\_\_\_|\_\_\_|\_\_\_|\_\_\_|
+{-{-}{-}{-}{-}{-}{-}{-}{-}{-}{-}+}
      Base
\end{verbatim}

\begin{itemize}
\tightlist
\item
  \textbf{Purpose}: Dissipates excess heat from electronic components to
  prevent thermal damage
\item
  \textbf{Types}: Passive heat sinks (aluminum/copper fins) and active
  heat sinks (with fans)
\item
  \textbf{Thermal Resistance}: Lower thermal resistance (^\circC/W) indicates
  better heat dissipation
\item
  \textbf{Materials}: Copper (best conductivity), aluminum (lightweight,
  cost-effective), composite
\end{itemize}

\end{solutionbox}
\begin{mnemonicbox}
``HARD sinks - Heat Away using Radiation and
Dissipation through metal sinks''

\end{mnemonicbox}
\subsection*{Question 1(c) [7 marks]}\label{q1c}

\textbf{Describe Thermal Runaway and Thermal Stability. How can overcome
thermal run away in transistor?}

\begin{solutionbox}

\textbf{Diagram:}

\begin{verbatim}
flowchart LR
    A[Heat Generation] {-{-} B[Increased Temperature]}
    B {-{-} C[Increased Ic]}
    C {-{-} D[More Power Dissipation]}
    D {-{-} A}
    E[Thermal Stability Methods] {-{-} F[Break this cycle]}
\end{verbatim}

\textbf{Thermal Runaway:}

\begin{itemize}
\tightlist
\item
  \textbf{Definition}: Self-accelerating process where transistor heats
  up, causing more current flow and further heating
\item
  \textbf{Cause}: Increase in temperature increases Ico (leakage
  current) which increases Ic
\item
  \textbf{Result}: Eventual destruction of transistor if unchecked
\end{itemize}

\textbf{Thermal Stability:}

\begin{itemize}
\tightlist
\item
  \textbf{Definition}: Ability to maintain stable operating point
  despite temperature changes
\item
  \textbf{Measure}: Stability factor (S) - lower values indicate better
  stability
\end{itemize}

\textbf{Overcoming Thermal Runaway:}

\begin{itemize}
\tightlist
\item
  \textbf{Heat Sinks}: Attach to dissipate excess heat
\item
  \textbf{Emitter Resistor}: Include unbypassed RE to provide negative
  feedback
\item
  \textbf{Voltage Divider Bias}: Use instead of fixed bias for better
  stability
\item
  \textbf{Thermal Compensation}: Add temperature-sensitive components in
  the bias circuit
\end{itemize}

\end{solutionbox}
\begin{mnemonicbox}
``SHEER protection - Sinks for Heat, Emitter
resistors, External cooling, and Robust biasing prevent thermal
runaway''

\end{mnemonicbox}
\subsection*{Question 1(c) OR [7
marks]}\label{q1c}

\textbf{Write down types of biasing methods. Explain the voltage divider
biasing method in details.}

\begin{solutionbox}

\textbf{Types of Biasing Methods:}


{\def\LTcaptype{none} % do not increment counter
\vspace{-5pt}
\captionof{table}{Transistor Biasing Methods}
\vspace{-10pt}
\begin{longtable}[]{@{}lll@{}}
\toprule\noalign{}
Method & Stability & Complexity \\
\midrule\noalign{}
\endhead
\bottomrule\noalign{}
\endlastfoot
Fixed Bias & Poor & Simple \\
Collector Feedback & Medium & Medium \\
Emitter Bias & Good & Medium \\
Voltage Divider & Excellent & Complex \\
\end{longtable}
}

\textbf{Voltage Divider Biasing Circuit:}

\begin{verbatim}
    +Vcc
      |
      R1
      |
      +{-{-}{-}{-}+}
      |    |
      R2   C1
      |    |
      +{-{-}{-}{-}+{-}{-}{-} Base}
      |
      |    +{-{-}{-}{-}+}
      |    |    |
      RE   RC   C2
      |    |    |
      +{-{-}{-}{-}+{-}{-}{-}{-}+{-}{-}{-} Output}
      |
     GND
\end{verbatim}

\textbf{Voltage Divider Biasing:}

\begin{itemize}
\tightlist
\item
  \textbf{Circuit Structure}: Uses two resistors (R1, R2) in series to
  create stable voltage at base
\item
  \textbf{Operating Principle}: Voltage at R2 sets base bias, remains
  stable despite β variations
\item
  \textbf{Advantage}: Most stable biasing technique with excellent
  temperature compensation
\item
  \textbf{Formula}: Base voltage VB = Vcc \times (R2/(R1+R2))
\item
  \textbf{Stability}: High stability factor as base voltage is nearly
  independent of collector current
\end{itemize}

\end{solutionbox}
\begin{mnemonicbox}
``DIVE for stability - Divider Is Very Effective for
temperature and β variations''

\end{mnemonicbox}
\subsection*{Question 2(a) [3 marks]}\label{q2a}

\textbf{Explain Stability Factor with features.}

\begin{solutionbox}

\textbf{Diagram:}

\begin{verbatim}
flowchart LR
    A[Temperature Changes] {-{-} B\{Stability Factor S\}}
    B {-{-}|High S| C[Unstable Circuit]}
    B {-{-}|Low S| D[Stable Circuit]}
\end{verbatim}

\begin{itemize}
\tightlist
\item
  \textbf{Definition}: Stability factor (S) measures how collector
  current changes with leakage current
\item
  \textbf{Formula}: S = ΔIC/ΔICBO
\item
  \textbf{Ideal Value}: Lower value (S \approx 1) indicates better stability
\item
  \textbf{Factors Affecting}: Biasing circuit design, temperature, and
  transistor parameters
\end{itemize}

\end{solutionbox}
\begin{mnemonicbox}
``LESS is better - Lower values Ensure Stable System
for temperature changes''

\end{mnemonicbox}
\subsection*{Question 2(b) [4 marks]}\label{q2b}

\textbf{Describe direct coupling technique of cascading.}

\begin{solutionbox}

\textbf{Diagram:}

\begin{verbatim}
     +Vcc
       |
       |
       RC1     RC2
       |       |
       +{-{-}{-}{-}{-}+ |}
       |     | |
    +{-{-}+     +{-}+{-}{-}+}
    |  |     |    |
Q1  |C |    C|    | Q2
    |  |     |    |
    +{-{-}+     +{-}+{-}{-}+}
    |  |     | |
    E  |     E |
    |  |     | |
    +{-{-}+     +{-}+}
       |       |
       RE1     RE2
       |       |
      GND     GND
\end{verbatim}

\begin{itemize}
\tightlist
\item
  \textbf{Definition}: Direct connection between collector of first
  stage to base of second stage
\item
  \textbf{Advantages}: No coupling components needed, excellent
  low-frequency response
\item
  \textbf{Disadvantages}: DC levels must be matched, thermal drift
  compounds across stages
\item
  \textbf{Applications}: DC amplifiers, integrated circuits, operational
  amplifiers
\end{itemize}

\end{solutionbox}
\begin{mnemonicbox}
``DIAL for DC - Direct Interconnection Amplifies Low
frequencies without capacitors''

\end{mnemonicbox}
\subsection*{Question 2(c) [7 marks]}\label{q2c}

\textbf{Explain frequency response of two stages RC coupled amplifier.}

\begin{solutionbox}

\textbf{Frequency Response Curve:}

\begin{verbatim}
    Gain (dB)
    \^{}
    |                 \_\_\_\_\_\_\_\_\_\_\_
    |                /           {}
    |               /             {}
    |              /               {}
    |             /                 {}
    |{-{-}{-}{-}{-}{-}{-}{-}{-}{-}{-}{-}/                   {-}{-}{-}{-}Frequency}
               f1                    f2
        Low frequency    Mid frequency    High frequency
          region           region          region
\end{verbatim}

\textbf{Two-Stage RC Coupled Amplifier:}

\begin{itemize}
\tightlist
\item
  \textbf{Circuit Structure}: Two transistor amplifiers connected via
  coupling capacitors
\item
  \textbf{Low-Frequency Response (f \textless{} f1)}: Gain drops due to
  coupling and bypass capacitor effects
\item
  \textbf{Mid-Frequency Response (f1 \textless{} f \textless{} f2)}:
  Maximum gain region, flat response
\item
  \textbf{High-Frequency Response (f \textgreater{} f2)}: Gain drops due
  to internal capacitances and Miller effect
\item
  \textbf{Bandwidth}: Range between lower cutoff (f1) and upper cutoff
  (f2) frequencies
\item
  \textbf{Overall Gain}: Product of individual stage gains minus
  coupling losses
\end{itemize}

\end{solutionbox}
\begin{mnemonicbox}
``LMH frequency regions - Low has rising gain, Middle
has flat gain, High has falling gain''

\end{mnemonicbox}
\subsection*{Question 2(a) OR [3
marks]}\label{q2a}

\textbf{Briefly explain bandwidth and gain-bandwidth product of an
amplifier.}

\begin{solutionbox}

\textbf{Diagram:}

\begin{verbatim}
    Gain (dB)
    \^{}
    |     \_\_\_\_\_\_\_\_\_\_\_\_\_\_\_
    |    /|              {}
    |   / |               {}
    |  /  |                {}
    | /   |                 {}
    |/    |                  {}
    +{-{-}{-}{-}{-}|{-}{-}{-}{-}{-}{-}{-}{-}{-}{-}{-}{-}{-}{-}{-}{-}{-}{-}|{-}{-}{-}{-}{-} Frequency}
          f1                 f2
          |{{-}{-}{-}Bandwidth{-}{-}{-}{-}|}
\end{verbatim}

\begin{itemize}
\tightlist
\item
  \textbf{Bandwidth}: Frequency range between lower (f1) and upper (f2)
  cutoff frequencies where gain is at least 70.7\% of maximum
\item
  \textbf{Formula}: Bandwidth = f2 - f1 (measured in Hz)
\item
  \textbf{Gain-Bandwidth Product}: Constant value of gain multiplied by
  bandwidth for a given amplifier
\item
  \textbf{Significance}: Represents fundamental limitation of amplifier
  performance
\end{itemize}

\end{solutionbox}
\begin{mnemonicbox}
``BIG value - Bandwidth and gain Inverse relationship
is a Given constant''

\end{mnemonicbox}
\subsection*{Question 2(b) OR [4
marks]}\label{q2b}

\textbf{Explain effects of emitter bypass capacitor and coupling
capacitor on frequency response of an amplifier.}

\begin{solutionbox}


{\def\LTcaptype{none} % do not increment counter
\vspace{-5pt}
\captionof{table}{Capacitor Effects on Frequency Response}
\vspace{-10pt}
\begin{longtable}[]{@{}llll@{}}
\toprule\noalign{}
Capacitor Type & Low Frequency & Mid Frequency & High Frequency \\
\midrule\noalign{}
\endhead
\bottomrule\noalign{}
\endlastfoot
Emitter Bypass & Affects gain & Full bypass & No effect \\
Coupling & Blocks signal & Full coupling & No effect \\
\end{longtable}
}

\textbf{Effects of Capacitors:}

\textbf{Emitter Bypass Capacitor:}

\begin{itemize}
\tightlist
\item
  \textbf{Purpose}: Bypasses emitter resistor to increase gain
\item
  \textbf{Low Frequency}: Acts as high impedance, reduces gain
\item
  \textbf{Formula}: Xc = 1/(2πfC) increases at low frequencies
\item
  \textbf{Cutoff Effect}: Sets lower cutoff frequency with RE
\end{itemize}

\textbf{Coupling Capacitor:}

\begin{itemize}
\tightlist
\item
  \textbf{Purpose}: Blocks DC, allows AC signal between stages
\item
  \textbf{Low Frequency}: High reactance blocks signal transfer
\item
  \textbf{Response Impact}: Larger capacitance improves low-frequency
  response
\item
  \textbf{Phase Shift}: Creates phase shift at low frequencies
\end{itemize}

\end{solutionbox}
\begin{mnemonicbox}
``CABLE effect - Capacitors Act as Barriers at Low
frequencies, improving at higher frequencies''

\end{mnemonicbox}
\subsection*{Question 2(c) OR [7
marks]}\label{q2c}

\textbf{Compare transformer coupled amplifier and RC coupled amplifier.}

\begin{solutionbox}

\textbf{Table: Comparison of Transformer Coupled vs RC Coupled
Amplifiers}

{\def\LTcaptype{none} % do not increment counter
\begin{longtable}[]{@{}
  >{\raggedright\arraybackslash}p{(\linewidth - 4\tabcolsep) * \real{0.2500}}
  >{\raggedright\arraybackslash}p{(\linewidth - 4\tabcolsep) * \real{0.4773}}
  >{\raggedright\arraybackslash}p{(\linewidth - 4\tabcolsep) * \real{0.2727}}@{}}
\toprule\noalign{}
\begin{minipage}[b]{\linewidth}\raggedright
Parameter
\end{minipage} & \begin{minipage}[b]{\linewidth}\raggedright
Transformer Coupled
\end{minipage} & \begin{minipage}[b]{\linewidth}\raggedright
RC Coupled
\end{minipage} \\
\midrule\noalign{}
\endhead
\bottomrule\noalign{}
\endlastfoot
Coupling Element & Transformer & Capacitor \& Resistor \\
Efficiency & Higher (90\%) & Lower (30-50\%) \\
Frequency Response & Limited, poor at extremes & Wide, better at low
freq \\
Size \& Weight & Bulky, heavy & Compact, lightweight \\
Cost & Higher & Lower \\
Impedance Matching & Excellent & Poor \\
Distortion & Lower & Higher \\
DC Isolation & Complete & Good \\
\end{longtable}
}

\textbf{Diagram Comparison:}

\begin{verbatim}
Transformer Coupled             RC Coupled
    +Vcc                           +Vcc
      |                              |
      RC                             RC
      |                              |
      +-----|OOOO|-----+             +------||------+
      |     |OOOO|     |             |      CC      |
      C     |OOOO|     C             C              C
      |                |             |              |
      +                +             +              +
      |                |             |              |
     GND              GND           GND            GND
\end{verbatim}

\end{solutionbox}
\begin{mnemonicbox}
``TREE factors - Transformers provide Robust
Efficiency and Excellent impedance matching, RC provides Cost savings''

\end{mnemonicbox}
\subsection*{Question 3(a) [3 marks]}\label{q3a}

\textbf{Describe the transistorized tuned amplifier.}

\begin{solutionbox}

\textbf{Circuit Diagram:}

\begin{verbatim}
    +Vcc
      |
      |
      +{-{-}{-}{-}+}
      |    |
      L    C
      |    |
      +{-{-}{-}{-}+}
      |
      C1
      |
      +{-{-}{-}{-}+{-}{-}{-}Output}
      |    |
      Q    RC
      |    |
      +{-{-}{-}{-}+}
      |
     GND
\end{verbatim}

\begin{itemize}
\tightlist
\item
  \textbf{Definition}: Amplifier with LC tank circuit in collector to
  amplify specific frequency band
\item
  \textbf{Principle}: LC circuit resonates at fr = 1/(2π\sqrtLC), providing
  maximum gain at resonance
\item
  \textbf{Bandwidth}: Narrower than RC amplifiers, determined by Q
  factor of the tuned circuit
\item
  \textbf{Applications}: RF amplifiers, radio receivers, wireless
  communication circuits
\end{itemize}

\end{solutionbox}
\begin{mnemonicbox}
``TRIP to resonance - Tuned Resonant circuits Improve
Performance at specific frequencies''

\end{mnemonicbox}
\subsection*{Question 3(b) [4 marks]}\label{q3b}

\textbf{Explain in brief Direct coupled amplifier.}

\begin{solutionbox}

\textbf{Circuit Diagram:}

\begin{verbatim}
    +Vcc
      |
      RC2
      |
      +{-{-}{-}{-}{-}{-}+{-}{-}{-}Output}
      |      |
      C      RC1
      |      |
      +{-{-}{-}{-}{-}{-}+}
      |
      E
      |
     GND
\end{verbatim}

\begin{itemize}
\tightlist
\item
  \textbf{Definition}: Multi-stage amplifier where stages connect
  directly without coupling components
\item
  \textbf{Working}: Collector of first stage directly connects to base
  of next stage
\item
  \textbf{Advantages}: Excellent low-frequency response, fewer
  components, compact design
\item
  \textbf{Disadvantages}: DC bias problems, thermal stability issues,
  limited gain per stage
\end{itemize}

\end{solutionbox}
\begin{mnemonicbox}
``COLD advantages - Compact design, Outstanding
low-frequency response, Less components, Direct connection''

\end{mnemonicbox}
\subsection*{Question 3(c) [7 marks]}\label{q3c}

\textbf{Describe the importance of h parameters in two port network.
Draw h-parameters circuit for CE amplifier.}

\begin{solutionbox}

\textbf{h-parameter Equivalent Circuit for CE:}

\begin{verbatim}
                 RC
      +{-{-}{-}{-}{-}+    |}
      |     |    |
Input |    +++   | Output
 o{-{-}{-}{-}+{-}{-}{-}||{-}{-}{-}{-}o}
      |    +++   |
      |     |    |
      +{-{-}+{-}{-}+    |}
         |       |
        +++      |
        GND      |
\end{verbatim}

\textbf{Importance of h-parameters:}

\begin{itemize}
\tightlist
\item
  \textbf{Universal Application}: Works for all transistor
  configurations (CE, CB, CC)
\item
  \textbf{Easy Measurement}: Parameters can be directly measured using
  simple circuits
\item
  \textbf{Complete Characterization}: Fully describes transistor
  behavior with four parameters
\item
  \textbf{Circuit Analysis}: Simplifies complex transistor circuit
  analysis
\item
  \textbf{Temperature Independence}: Relatively stable over normal
  operating temperatures
\end{itemize}

\textbf{h-parameters for CE:}

\begin{itemize}
\tightlist
\item
  \textbf{h11 (hie)}: Input impedance with output short-circuited
\item
  \textbf{h12 (hre)}: Reverse voltage transfer ratio
\item
  \textbf{h21 (hfe)}: Forward current gain (β)
\item
  \textbf{h22 (hoe)}: Output admittance with input open-circuited
\end{itemize}

\end{solutionbox}
\begin{mnemonicbox}
``FINE parameters - Four Interconnected Network
Elements define transistor completely''

\end{mnemonicbox}
\subsection*{Question 3(a) OR [3
marks]}\label{q3a}

\textbf{Compare transformer coupled amplifier and direct coupled
amplifier.}

\begin{solutionbox}


{\def\LTcaptype{none} % do not increment counter
\vspace{-5pt}
\captionof{table}{Transformer vs Direct Coupled Amplifiers}
\vspace{-10pt}
\begin{longtable}[]{@{}lll@{}}
\toprule\noalign{}
Parameter & Transformer Coupled & Direct Coupled \\
\midrule\noalign{}
\endhead
\bottomrule\noalign{}
\endlastfoot
DC Isolation & Complete & None \\
Low Freq Response & Poor & Excellent \\
Size & Bulky & Compact \\
Impedance Matching & Excellent & Poor \\
Distortion & Low & Can be high \\
Cost & High & Low \\
Complexity & Medium & Simple \\
\end{longtable}
}

\end{solutionbox}
\begin{mnemonicbox}
``TIP for selection - Transformer for Impedance
matching and Power transfer, Direct for low frequencies''

\end{mnemonicbox}
\subsection*{Question 3(b) OR [4
marks]}\label{q3b}

\textbf{Draw and Explain circuit diagram of common emitter amplifier.}

\begin{solutionbox}

\textbf{CE Amplifier Circuit:}

\begin{verbatim}
     +Vcc
       |
       RC
       |
       +{-{-}{-}{-}||{-}{-}{-}o Output}
       |    CC
       |
    +{-{-}+}
    |  |
    |  C
    |  |
 {-{-}{-}+{-}{-}+{-}{-}{-}}
 |  |  |  |
 |  |  |  |
 +{-{-}+{-}{-}+{-}{-}+}
    |
    RE
    |
   GND
\end{verbatim}

\begin{itemize}
\tightlist
\item
  \textbf{Configuration}: Input at base, output from collector, emitter
  is common to both
\item
  \textbf{Characteristics}: Voltage gain \textasciitilde50-500, current
  gain \textasciitilde50-200, phase shift 180^\circ
\item
  \textbf{Advantages}: High voltage gain, medium input impedance, good
  voltage amplification
\item
  \textbf{Applications}: Audio amplifiers, radio frequency amplifiers,
  switching circuits
\end{itemize}

\end{solutionbox}
\begin{mnemonicbox}
``GAIN characteristics - Good Amplification with
Inverted output and Notable efficiency''

\end{mnemonicbox}
\subsection*{Question 3(c) OR [7
marks]}\label{q3c}

\textbf{Draw Transistor Two Port Network and describe h-parameters for
it. Write down advantages of hybrid parameters.}

\begin{solutionbox}

\textbf{Two-Port Network Diagram:}

\begin{verbatim}
       +{-{-}{-}{-}{-}{-}{-}{-}{-}{-}{-}{-}{-}+}
       |             |
 I1 {-{-}+             +{-}{-}{-} I2}
       |   Two{-Port  |}
 V1 {-{-}+   Network   +{-}{-}{-} V2}
       |             |
       +{-{-}{-}{-}{-}{-}{-}{-}{-}{-}{-}{-}{-}+}
\end{verbatim}

\textbf{h-parameters Equations:}

\begin{itemize}
\tightlist
\item
  V1 = h11I1 + h12V2
\item
  I2 = h21I1 + h22V2
\end{itemize}

\textbf{h-parameters Description:}

\begin{itemize}
\tightlist
\item
  \textbf{h11}: Input impedance (Ω) with output short-circuited
\item
  \textbf{h12}: Reverse voltage transfer ratio (dimensionless)
\item
  \textbf{h21}: Forward current gain (dimensionless)
\item
  \textbf{h22}: Output admittance (Siemens) with input open-circuited
\end{itemize}

\textbf{Advantages of Hybrid Parameters:}

\begin{itemize}
\tightlist
\item
  \textbf{Easy Measurement}: Each parameter can be measured individually
\item
  \textbf{Standard Notation}: Universal acceptance in industry and
  academics
\item
  \textbf{Accurate Model}: Provides precise modeling of transistor
  behavior
\item
  \textbf{Configuration Flexibility}: Applicable to all transistor
  configurations
\item
  \textbf{Temperature Stability}: Relatively stable over operating
  temperature range
\end{itemize}

\end{solutionbox}
\begin{mnemonicbox}
``SMART parameters - Simple Measurement, Accurate
modeling, Reliable, Temperature-stable''

\end{mnemonicbox}
\subsection*{Question 4(a) [3 marks]}\label{q4a}

\textbf{Explain Darlington pair and its applications.}

\begin{solutionbox}

\textbf{Darlington Pair Circuit:}

\begin{verbatim}
    +{-{-}+}
    |  |
    |  C1    +{-{-}+}
 {-{-}{-}+{-}{-}+{-}{-}{-}{-}{-}|  |}
 |  |  |     |  C2   Output
 |  |  +{-{-}{-}{-}{-}|  |{-}{-}{-}{-}{-}o}
 |  B1 |     |  |
 o{-{-}{-}{-}{-}+{-}{-}{-}{-}{-}|  |}
    |     B2 |  |
    E1{-{-}{-}{-}{-}{-}{-}|  |}
             E2 |
              {-{-}|{-}{-}}
               GND
\end{verbatim}

\begin{itemize}
\tightlist
\item
  \textbf{Definition}: Configuration of two transistors where emitter of
  first drives base of second
\item
  \textbf{Characteristics}: Very high current gain (β1 \times β2), high input
  impedance
\item
  \textbf{Drawbacks}: Higher saturation voltage, reduced switching speed
\item
  \textbf{Applications}: Power amplifiers, motor drivers,
  touch-sensitive switches, Darlington ICs
\end{itemize}

\end{solutionbox}
\begin{mnemonicbox}
``HIGH gain - Hugely Increased Gain from Harnessing
two transistors''

\end{mnemonicbox}
\subsection*{Question 4(b) [4 marks]}\label{q4b}

\textbf{Describe the diode clamper circuit with necessary diagram.}

\begin{solutionbox}

\textbf{Positive Clamper Circuit:}

\begin{verbatim}
               D
    Input o{-{-}{-}||{-}{-}{-}+{-}{-}{-}o Output}
              |     |
              C     R
              |     |
              +{-{-}{-}{-}{-}+}
              |
             GND
\end{verbatim}

\begin{itemize}
\tightlist
\item
  \textbf{Definition}: Circuit that shifts waveform up/down by adding DC
  component
\item
  \textbf{Types}: Positive clamper (shifts up), negative clamper (shifts
  down)
\item
  \textbf{Working Principle}: Capacitor charges during first half-cycle,
  then maintains DC level
\item
  \textbf{Applications}: TV sync pulse restoration, pulse modulation
  circuits, waveform processing
\end{itemize}

\end{solutionbox}
\begin{mnemonicbox}
``CAPS effect - Capacitor And diode Pair Shifts
signal by exact DC level''

\end{mnemonicbox}
\subsection*{Question 4(c) [7 marks]}\label{q4c}

\textbf{Explain the construction, working and applications of OLED.}

\begin{solutionbox}

\textbf{OLED Structure:}

\begin{verbatim}
       +{-{-}{-}{-}{-}{-}{-}{-}{-}{-}{-}{-}{-}{-}{-}{-}+}
       | Cathode (Metal)|
       +{-{-}{-}{-}{-}{-}{-}{-}{-}{-}{-}{-}{-}{-}{-}{-}+}
       | Emissive Layer |
       +{-{-}{-}{-}{-}{-}{-}{-}{-}{-}{-}{-}{-}{-}{-}{-}+}
       |Conductive Layer|
       +{-{-}{-}{-}{-}{-}{-}{-}{-}{-}{-}{-}{-}{-}{-}{-}+}
       |   Anode (ITO)  |
       +{-{-}{-}{-}{-}{-}{-}{-}{-}{-}{-}{-}{-}{-}{-}{-}+}
       |   Substrate    |
       +{-{-}{-}{-}{-}{-}{-}{-}{-}{-}{-}{-}{-}{-}{-}{-}+}
\end{verbatim}

\textbf{OLED Construction:}

\begin{itemize}
\tightlist
\item
  \textbf{Layers}: Substrate, anode (ITO), conductive layer, emissive
  layer, cathode
\item
  \textbf{Materials}: Organic semiconductor materials between electrodes
\item
  \textbf{Types}: PMOLED (passive matrix) and AMOLED (active matrix)
\end{itemize}

\textbf{Working Principle:}

\begin{itemize}
\tightlist
\item
  \textbf{Mechanism}: Electric current causes organic material to emit
  light via electroluminescence
\item
  \textbf{Process}: Electrons and holes recombine in emissive layer to
  produce photons
\item
  \textbf{Efficiency}: Direct light emission without backlight, high
  efficiency
\end{itemize}

\textbf{Applications:}

\begin{itemize}
\tightlist
\item
  \textbf{Displays}: Smartphones, TVs, wearables, digital cameras
\item
  \textbf{Lighting}: Flexible and transparent lighting panels
\item
  \textbf{Signage}: High-contrast digital signs and billboards
\end{itemize}

\end{solutionbox}
\begin{mnemonicbox}
``OLED benefits - Organic materials, Lightweight
design, Efficient operation, Direct emission, Stunning contrast''

\end{mnemonicbox}
\subsection*{Question 4(a) OR [3
marks]}\label{q4a}

\textbf{Explain Short note on LDR.}

\begin{solutionbox}

\textbf{LDR Symbol and Structure:}

\begin{verbatim}
    Symbol              Structure
      ⌒   ⌒             +{-{-}{-}{-}{-}{-}{-}+}
     /     {            |///////|}
    +       +           |///////|
    |       |           +{-{-}{-}{-}{-}{-}{-}+}
    +       +
     {     /}
      ⌒   ⌒
\end{verbatim}

\begin{itemize}
\tightlist
\item
  \textbf{Definition}: Light Dependent Resistor, a photoresistor whose
  resistance decreases with light
\item
  \textbf{Material}: Cadmium sulfide (CdS) or cadmium selenide (CdSe)
\item
  \textbf{Principle}: Photoconductivity - light energy releases
  electrons, increasing conductivity
\item
  \textbf{Applications}: Light sensors, automatic lighting controls,
  camera exposure systems
\end{itemize}

\end{solutionbox}
\begin{mnemonicbox}
``DARK increases resistance - Decreasing light And
Rising darkness Keep resistance high''

\end{mnemonicbox}
\subsection*{Question 4(b) OR [4
marks]}\label{q4b}

\textbf{Describe the diode clipper circuit with necessary diagram.}

\begin{solutionbox}

\textbf{Positive Clipper Circuit:}

\begin{verbatim}
              R
    Input o{-{-}{-}www{-}{-}{-}+{-}{-}{-}o Output}
                    |
                    |
                  \_\_|\_\_
                  {   /}
                   { /}
                    V
                    |
                   GND
\end{verbatim}

\begin{itemize}
\tightlist
\item
  \textbf{Definition}: Circuit that limits (clips) portions of input
  waveform above/below threshold
\item
  \textbf{Types}: Positive clipper (clips positive), negative clipper
  (clips negative), biased clipper
\item
  \textbf{Working Principle}: Diode conducts when signal exceeds
  threshold, limiting output
\item
  \textbf{Applications}: Waveform shaping, protection circuits, signal
  conditioning
\end{itemize}

\end{solutionbox}
\begin{mnemonicbox}
``CLIP waves - Circuit Limits Input Peaks by using
diode conduction''

\end{mnemonicbox}
\subsection*{Question 4(c) OR [7
marks]}\label{q4c}

\textbf{Explain Half Wave and Full wave Voltage Doubler.}

\begin{solutionbox}

\textbf{Half-Wave Voltage Doubler:}

\begin{verbatim}
             D1
    AC o{-{-}{-}{-}{-}{-}||{-}{-}{-}{-}{-}{-}{-}+{-}{-}{-}{-}o Output}
    Input               |    (+2Vp)
               |        |
               C1       C2
               |        |
              GND      GND
\end{verbatim}

\textbf{Full-Wave Voltage Doubler:}

\begin{verbatim}
             D1         
    AC o{-{-}{-}{-}{-}{-}||{-}{-}{-}{-}{-}{-}{-}+{-}{-}{-}{-}o Output}
    Input     |         |    (+2Vp)
              |         |
              C1        C2
              |         |
              |    D2   |
              +{-{-}{-}||{-}{-}{-}+}
              |
             GND
\end{verbatim}

\textbf{Half-Wave Voltage Doubler:}

\begin{itemize}
\tightlist
\item
  \textbf{Operation}: During negative half cycle, C1 charges to peak
  voltage; during positive cycle, output becomes 2Vp
\item
  \textbf{Output}: Pulsating DC with peak value twice input peak
\item
  \textbf{Ripple}: Higher ripple content
\item
  \textbf{Efficiency}: Lower than full-wave
\end{itemize}

\textbf{Full-Wave Voltage Doubler:}

\begin{itemize}
\tightlist
\item
  \textbf{Operation}: Both half cycles contribute to output, with each
  capacitor charging during alternate cycles
\item
  \textbf{Output}: Smoother DC with peak value twice input peak
\item
  \textbf{Ripple}: Lower ripple content
\item
  \textbf{Efficiency}: Higher than half-wave
\end{itemize}

\textbf{Applications:}

\begin{itemize}
\tightlist
\item
  \textbf{High voltage generation}: CRT displays, photomultipliers
\item
  \textbf{Power supplies}: Low current, high voltage applications
\item
  \textbf{Cascade connection}: For voltage multiplication beyond
  doubling
\end{itemize}

\end{solutionbox}
\begin{mnemonicbox}
``CHASE 2V - Capacitors Hold Alternating Supply
Energy to produce 2\times Voltage''

\end{mnemonicbox}
\subsection*{Question 5(a) [3 marks]}\label{q5a}

\textbf{Draw circuit diagram for +5v Power Supply using its IC and
explain in brief.}

\begin{solutionbox}

\textbf{5V Power Supply using 7805:}

\begin{verbatim}
        D1    D2
    o{-{-}{-}||{-}{-}{-}||{-}{-}{-}+{-}{-}{-}{-}{-}+{-}{-}{-}{-}{-}{-}{-}{-}o}
AC       |         |     |        +5V
Input    +{-{-}||{-}{-}+ |    7805      Output}
         |  D3   | |     |
         +{-{-}||{-}{-}+ |     |}
         |  D4     C1    C2
         |         |     |
    o{-{-}{-}{-}+{-}{-}{-}{-}{-}{-}{-}{-}{-}+{-}{-}{-}{-}{-}+{-}{-}{-}{-}{-}{-}{-}{-}o}
                  GND
\end{verbatim}

\begin{itemize}
\tightlist
\item
  \textbf{Components}: Bridge rectifier (D1-D4), filter capacitor (C1),
  7805 regulator, output capacitor (C2)
\item
  \textbf{Working}: AC converted to DC by rectifier, filtered by C1,
  regulated to exact 5V by 7805
\item
  \textbf{Features}: Short-circuit protection, thermal shutdown, up to
  1A current capability
\item
  \textbf{Applications}: Digital circuits, microcontrollers, electronics
  projects
\end{itemize}

\end{solutionbox}
\begin{mnemonicbox}
``FIRM voltage - Filtered Input, Regulated by 7805
Makes stable voltage''

\end{mnemonicbox}
\subsection*{Question 5(b) [4 marks]}\label{q5b}

\textbf{Discuss load regulation and line regulation in reference to
power supply.}

\begin{solutionbox}

\textbf{Regulation Performance Curves:}

\begin{verbatim}
    Vout           Vout
     \^{              \^{}}
     |              |
     |{-{-}            |{-}{-}}
     |  {           |  }
     |   { Load     |    Line}
     |    {         |    }
     +{-{-}{-}{-}{-}{-}{-}      +{-}{-}{-}{-}{-}{-}{-}}
           Iload           Vin
\end{verbatim}

\textbf{Load Regulation:}

\begin{itemize}
\tightlist
\item
  \textbf{Definition}: Ability to maintain constant output voltage
  despite load current changes
\item
  \textbf{Formula}: \% Load Regulation = ((VNL - VFL)/VFL) \times 100
\item
  \textbf{Importance}: Ensures stable voltage for varying load demands
\item
  \textbf{Ideal Value}: 0\% (no change in output voltage with load
  changes)
\end{itemize}

\textbf{Line Regulation:}

\begin{itemize}
\tightlist
\item
  \textbf{Definition}: Ability to maintain constant output despite input
  voltage variations
\item
  \textbf{Formula}: \% Line Regulation = (ΔVout/ΔVin) \times 100
\item
  \textbf{Importance}: Protects circuits from mains voltage fluctuations
\item
  \textbf{Ideal Value}: 0\% (no change in output voltage with input
  changes)
\end{itemize}

\end{solutionbox}
\begin{mnemonicbox}
``LIVER health - Line regulation for Input
Variations, load regulation for External Resistance changes''

\end{mnemonicbox}
\subsection*{Question 5(c) [7 marks]}\label{q5c}

\textbf{Explain adjustable voltage regulator using LM317 with circuit
diagram.}

\begin{solutionbox}

\textbf{LM317 Adjustable Regulator Circuit:}

\begin{verbatim}
                   R1
     +Vin o{-{-}{-}+{-}{-}{-}{-}www{-}{-}{-}{-}+}
              |           |
              |    ADJ    |
              |  +{-{-}{-}{-}{-}+  |}
              +{-{-}| 317 |{-}{-}+{-}{-}o +Vout}
                 |     |     |
                 +{-{-}{-}{-}{-}+     |}
                             R2
                             |
                            GND
\end{verbatim}

\textbf{Working Principle:}

\begin{itemize}
\tightlist
\item
  \textbf{Basic Operation}: LM317 maintains 1.25V between output and
  adjustment pin
\item
  \textbf{Output Voltage}: Vout = 1.25V(1 + R2/R1) + IADJ(R2)
\item
  \textbf{Simplified Formula}: Vout \approx 1.25V(1 + R2/R1) (since IADJ is
  very small)
\item
  \textbf{Adjustment Range}: 1.25V to 37V depending on input voltage
\end{itemize}

\textbf{Features:}

\begin{itemize}
\tightlist
\item
  \textbf{Current Capability}: Up to 1.5A output current
\item
  \textbf{Protection}: Internal thermal overload and short circuit
  protection
\item
  \textbf{Advantages}: Simple design, minimal external components,
  stable output
\item
  \textbf{Applications}: Variable power supplies, battery chargers,
  custom voltage regulators
\end{itemize}

\end{solutionbox}
\begin{mnemonicbox}
``VAIR control - Variable Adjustable Integrated
Regulator controls voltage precisely''

\end{mnemonicbox}
\subsection*{Question 5(a) OR [3
marks]}\label{q5a}

\textbf{Explain working of solar battery charger circuits.}

\begin{solutionbox}

\textbf{Solar Battery Charger Block Diagram:}

\begin{verbatim}
flowchart LR
    A[Solar Panel] {-{-} B[Charge Controller]}
    B {-{-} C[Battery]}
    C {-{-} D[Load/Output]}
\end{verbatim}

\begin{itemize}
\tightlist
\item
  \textbf{Components}: Solar panel, charge controller, battery,
  protection circuits
\item
  \textbf{Working Principle}: Solar panel generates DC, controller
  regulates charging current
\item
  \textbf{Charge Phases}: Bulk charging (constant current), absorption
  (constant voltage), float (maintenance)
\item
  \textbf{Protection Features}: Overcharge protection, deep discharge
  prevention, reverse polarity
\end{itemize}

\end{solutionbox}
\begin{mnemonicbox}
``SCBL system - Solar panel Converts sunlight,
Battery stores, Load consumes''

\end{mnemonicbox}
\subsection*{Question 5(b) OR [4
marks]}\label{q5b}

\textbf{Explain working of UPS.}

\begin{solutionbox}

\textbf{UPS Block Diagram:}

\begin{verbatim}
    +{-{-}{-}{-}{-}{-}+    +{-}{-}{-}{-}{-}{-}{-}+    +{-}{-}{-}{-}{-}{-}{-}{-}+}
    |      |    |       |    |        |
AC{-{-}+ Rect +{-}{-}{-}{-}+ Batt. +{-}{-}{-}{-}+ Invert +{-}{-}{-}AC}
    |      |    |       |    |        |
    +{-{-}{-}{-}{-}{-}+    +{-}{-}{-}{-}{-}{-}{-}+    +{-}{-}{-}{-}{-}{-}{-}{-}+}
       |                         |
       +{-{-}{-}{-}{-}{-}{-}{-}{-}+{-}{-}{-}{-}{-}{-}{-}{-}{-}{-}{-}{-}{-}{-}{-}+}
                 |
              Control
              Circuit
\end{verbatim}

\begin{itemize}
\tightlist
\item
  \textbf{Definition}: Uninterruptible Power Supply provides backup
  power during main supply failure
\item
  \textbf{Types}: Offline (standby), Line-interactive, Online (double
  conversion)
\item
  \textbf{Components}: Rectifier, battery, inverter, control circuitry,
  transfer switch
\item
  \textbf{Operation}: Normally passes filtered mains power, switches to
  battery during outage
\end{itemize}

\end{solutionbox}
\begin{mnemonicbox}
``PRIME power - Power Remains Intact during Mains
Electricity problems''

\end{mnemonicbox}
\subsection*{Question 5(c) OR [7
marks]}\label{q5c}

\textbf{Draw and explain SMPS block diagram with its advantages and
disadvantages.}

\begin{solutionbox}

\textbf{SMPS Block Diagram:}

\begin{verbatim}
flowchart LR
    A[AC Input] {-{-} B[EMI Filter]}
    B {-{-} C[Rectifier]}
    C {-{-} D[High{-}Freq Switch]}
    D {-{-} E[Transformer]}
    E {-{-} F[Output Rectifier]}
    F {-{-} G[Filter]}
    G {-{-} H[DC Output]}
    I[Feedback] {-{-} J[Control Circuit]}
    J {-{-} D}
\end{verbatim}

\textbf{Working Principle:}

\begin{itemize}
\tightlist
\item
  \textbf{Input Stage}: AC converted to unregulated DC by rectifier
\item
  \textbf{Switching Stage}: High-frequency transistors chop DC into
  pulses
\item
  \textbf{Transformer}: Isolates and transforms voltage at high
  frequency
\item
  \textbf{Output Stage}: Rectifies and filters to produce clean DC
\item
  \textbf{Feedback Loop}: Monitors output and adjusts switching for
  regulation
\end{itemize}

\textbf{Advantages:}

\begin{itemize}
\tightlist
\item
  \textbf{Efficiency}: 70-90\% compared to 30-60\% for linear supplies
\item
  \textbf{Size/Weight}: Smaller transformers due to high-frequency
  operation
\item
  \textbf{Heat Generation}: Less power dissipation, reduced cooling
  requirements
\item
  \textbf{Wide Input Range}: Can operate over wide input voltage
  variations
\end{itemize}

\textbf{Disadvantages:}

\begin{itemize}
\tightlist
\item
  \textbf{Complexity}: More complex design than linear supplies
\item
  \textbf{EMI/RFI}: Generates electromagnetic interference
\item
  \textbf{Noise}: Higher output noise due to switching operation
\item
  \textbf{Cost}: More expensive for low-power applications
\end{itemize}

\end{solutionbox}
\begin{mnemonicbox}
``FISH factors - Frequency switching, Isolation,
Small size, High efficiency are SMPS benefits''

\end{mnemonicbox}
\subsection*{Summary of Key Concepts}\label{summary-of-key-concepts}

\subsubsection{Transistor Biasing and
Stability}\label{transistor-biasing-and-stability}

\begin{itemize}
\tightlist
\item
  \textbf{Biasing Methods}: Fixed bias, Collector feedback, Emitter
  bias, Voltage divider (most stable)
\item
  \textbf{Thermal Stability}: Use emitter resistors, voltage divider
  bias, heat sinks to prevent thermal runaway
\item
  \textbf{Stability Factor (S)}: Lower value indicates better stability
  against temperature changes
\end{itemize}

\subsubsection{Amplifier Parameters}\label{amplifier-parameters}

\begin{itemize}
\tightlist
\item
  \textbf{CE Amplifier}: High voltage gain (50-500), medium input
  impedance, 180^\circ phase shift
\item
  \textbf{h-parameters}: h11 (input impedance), h21 (current gain), h12
  (reverse voltage ratio), h22 (output admittance)
\item
  \textbf{Frequency Response}: Affected by coupling capacitors at low
  frequencies, internal capacitances at high frequencies
\end{itemize}

\subsubsection{Coupling Methods}\label{coupling-methods}

\begin{itemize}
\tightlist
\item
  \textbf{RC Coupling}: Simple, low cost, good frequency response
  (except very low frequencies)
\item
  \textbf{Transformer Coupling}: Good impedance matching, excellent
  efficiency, bulky and expensive
\item
  \textbf{Direct Coupling}: Excellent low-frequency response, DC bias
  issues, used in integrated circuits
\end{itemize}

\subsubsection{Practical Applications}\label{practical-applications}

\begin{itemize}
\tightlist
\item
  \textbf{Clippers \& Clampers}: Waveform shaping, limiting, level
  shifting circuits
\item
  \textbf{Voltage Multipliers}: Generate higher DC voltages from lower
  AC inputs (doubler, tripler, etc.)
\item
  \textbf{Darlington Pair}: Super-high current gain configuration for
  power applications
\item
  \textbf{OLED Displays}: Organic light-emitting diodes with high
  contrast, energy efficiency
\end{itemize}

\subsubsection{Power Supply Circuits}\label{power-supply-circuits}

\begin{itemize}
\tightlist
\item
  \textbf{Voltage Regulators}: 78xx series (positive), 79xx series
  (negative), LM317 (adjustable)
\item
  \textbf{SMPS}: High-efficiency switch-mode power supplies with smaller
  size but greater complexity
\item
  \textbf{UPS}: Provides backup power during outages using
  battery-inverter systems
\item
  \textbf{Solar Chargers}: Convert solar energy to charge batteries with
  overcharge protection
\end{itemize}

\subsection*{Important Formulas to
Remember}\label{important-formulas-to-remember}

{\def\LTcaptype{none} % do not increment counter
\begin{longtable}[]{@{}
  >{\raggedright\arraybackslash}p{(\linewidth - 4\tabcolsep) * \real{0.3333}}
  >{\raggedright\arraybackslash}p{(\linewidth - 4\tabcolsep) * \real{0.2727}}
  >{\raggedright\arraybackslash}p{(\linewidth - 4\tabcolsep) * \real{0.3939}}@{}}
\toprule\noalign{}
\begin{minipage}[b]{\linewidth}\raggedright
Parameter
\end{minipage} & \begin{minipage}[b]{\linewidth}\raggedright
Formula
\end{minipage} & \begin{minipage}[b]{\linewidth}\raggedright
Description
\end{minipage} \\
\midrule\noalign{}
\endhead
\bottomrule\noalign{}
\endlastfoot
Voltage Gain (Av) & Vout/Vin & Ratio of output to input voltage \\
Current Gain (Ai) & Ic/Ib & Ratio of collector to base current \\
Bandwidth & f2 - f1 & Frequency range between cutoff points \\
Load Regulation & ((VNL-VFL)/VFL)\times100\% & Voltage stability with load
change \\
Line Regulation & (ΔVout/ΔVin)\times100\% & Voltage stability with input
change \\
Stability Factor (S) & ΔIC/ΔICBO & Change in collector current vs
leakage \\
LM317 Output & 1.25V(1+R2/R1) & Adjustable regulator output voltage \\
Resonant Frequency & 1/(2π\sqrtLC) & Tuned amplifier resonance point \\
\end{longtable}
}

\subsection*{Exam Tips for Electronic
Circuits}\label{exam-tips-for-electronic-circuits}

\begin{enumerate}
\tightlist
\item
  \textbf{Draw the Basics First}: Always begin with the basic circuit
  diagram before adding details
\item
  \textbf{Remember Polarities}: Pay attention to voltage polarities and
  current directions
\item
  \textbf{Compare in Tables}: Use tables for comparison questions to
  organize information
\item
  \textbf{Focus on Practical Uses}: Connect theoretical concepts to
  real-world applications
\item
  \textbf{Know the Numbers}: Memorize typical values (gains, impedances,
  voltages)
\item
  \textbf{Use Mnemonics}: Create memory aids for complex concepts and
  formulas
\end{enumerate}

\subsection*{Common Mistakes to Avoid}\label{common-mistakes-to-avoid}

\begin{enumerate}
\tightlist
\item
  \textbf{Mixing Up Biasing}: Don't confuse the different biasing
  methods and their stability factors
\item
  \textbf{Parameter Confusion}: Keep h-parameters definitions clear and
  distinct
\item
  \textbf{Sign Errors}: Remember phase inversions (180^\circ shift) in common
  emitter configurations
\item
  \textbf{Regulation Formulas}: Don't mix up load regulation and line
  regulation formulas
\item
  \textbf{Overcomplicating Diagrams}: Keep circuit diagrams simple and
  focused on key components
\end{enumerate}

\subsection*{Quick Reference: Component
Symbols}\label{quick-reference-component-symbols}

\begin{verbatim}
Transistor (NPN)    Transistor (PNP)    Diode        LED
    C                   C                 A            A
    |                   |                 |            |
    |                   |                 +-|>|-+      +-|>|-+
    B---|               B---|             K            K  \/
    |                   |
    E                   E

Resistor     Capacitor    Inductor    Transformer
  --www--     --||--      --OOOO--    --OOOO--
                                       --OOOO--
\end{verbatim}


\end{document}
