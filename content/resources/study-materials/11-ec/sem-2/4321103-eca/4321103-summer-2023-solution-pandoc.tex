\documentclass[10pt,a4paper]{article}

% content/resources/templates/preamble.tex
\usepackage[margin=0.6in]{geometry}
\author{Milav Dabgar}
\usepackage{amsmath,amssymb,amsthm}
\usepackage{booktabs}
\usepackage{multirow}
\usepackage{xcolor}
\usepackage{tcolorbox}
\tcbuselibrary{breakable,skins}
\usepackage[colorlinks=true,linkcolor=blue]{hyperref}
\usepackage{titlesec}
\usepackage{enumitem}
\usepackage{tikz}
\usepackage{pgfplots}
\usepackage{circuitikz}
\usepackage[version=4]{mhchem}
\usepackage{longtable}
\usepackage{array}
\usepackage{float}
\usepackage{caption}
\usepackage{listings}

\lstset{
  basicstyle=\small\ttfamily,
  breaklines=true,
  breakatwhitespace=false,
  postbreak=\mbox{\textcolor{red}{$\hookrightarrow$}\space},
  float=false,
  numbers=left,
  numberstyle=\tiny\color{gray},
  numbersep=10pt,
  xleftmargin=2em,
  keywordstyle=\color{blue},
  commentstyle=\color{green!60!black},
  stringstyle=\color{purple},
  backgroundcolor=\color{gray!5},
  showstringspaces=false,
  tabsize=2,
  captionpos=b,
  keepspaces=true,
  columns=flexible
}

\pgfplotsset{compat=1.18}
\usetikzlibrary{shapes,arrows,positioning,calc,patterns,decorations.pathmorphing,decorations.markings,arrows.meta}

% Color scheme
\definecolor{headcolor}{RGB}{0,102,204}
\definecolor{keycolor}{RGB}{220,20,60}
\definecolor{solutioncolor}{RGB}{34,139,34}
\definecolor{mnemoniccolor}{RGB}{148,0,211}
\definecolor{codecolor}{RGB}{0,0,100}

% Spacing
\setlength{\parskip}{3pt}
\setlist[itemize]{nosep}
\setlist[enumerate]{nosep}

% Title formatting
\titleformat{\section}{\Large\bfseries\color{headcolor}}{\thesection}{1em}{}
\titleformat{\subsection}{\large\bfseries\color{headcolor}}{\thesubsection}{1em}{}

% Pandoc tightlist compatibility
\providecommand{\tightlist}{%
  \setlength{\itemsep}{0pt}\setlength{\parskip}{0pt}}

% Pandoc longtable compatibility
\newcounter{none}
\def\thenone{}


% content/resources/templates/english-boxes.tex
% This file is currently empty - it exists to maintain consistency with the import structure.
% Add custom environments here if needed in the future.


\begin{document}

\begin{center}
{\Huge\bfseries\color{headcolor} Subject Name Solutions}\\[5pt]
{\LARGE 4321103 -- Summer 2023}\\[3pt]
{\large Semester 1 Study Material}\\[3pt]
{\normalsize\textit{Detailed Solutions and Explanations}}
\end{center}

\vspace{10pt}

\subsection*{Question 1(a) [3 marks]}\label{q1a}

\textbf{Explain thermal runaway in details.}

\begin{solutionbox}
Thermal runaway is a destructive mechanism in BJT
transistors where increased temperature creates a self-reinforcing cycle
leading to device failure.

\begin{verbatim}
flowchart LR
    A[Increase in Temperature] {-{-} B[Increase in Ic]}
    B {-{-} C[Increase in Power Dissipation]}
    C {-{-} D[Further Increase in Temperature]}
    D {-{-} A}
\end{verbatim}

\begin{itemize}
\tightlist
\item
  \textbf{Heat Generation}: Temperature rises from normal operation
\item
  \textbf{Leakage Current}: Collector current Ic increases with
  temperature
\item
  \textbf{Power Dissipation}: More power = Temperature rises further
\item
  \textbf{Destructive Cycle}: Continuous cycle until transistor destroys
  itself
\end{itemize}

\end{solutionbox}
\begin{mnemonicbox}
``The Higher Temperature, The Higher Current''

\end{mnemonicbox}
\subsection*{Question 1(b) [4 marks]}\label{q1b}

\textbf{Define amplifier with simple block diagram write down amplifier
parameters.}

\begin{solutionbox}
An amplifier is an electronic device that increases the
power, voltage or current of an input signal.

\begin{verbatim}
flowchart LR
    A[Input Signal] {-{-}|Vin| B[AMPLIFIER]}
    B {-{-}|Vout| C[Output Signal]}
    D[Power Supply] {-{-} B}
\end{verbatim}

{\def\LTcaptype{none} % do not increment counter
\begin{longtable}[]{@{}ll@{}}
\toprule\noalign{}
Amplifier Parameter & Description \\
\midrule\noalign{}
\endhead
\bottomrule\noalign{}
\endlastfoot
\textbf{Voltage Gain (Av)} & Ratio of output voltage to input voltage \\
\textbf{Current Gain (Ai)} & Ratio of output current to input current \\
\textbf{Power Gain (Ap)} & Product of voltage gain and current gain \\
\textbf{Bandwidth} & Range of frequencies amplifier can handle \\
\textbf{Input Impedance} & Resistance seen by the input source \\
\textbf{Output Impedance} & Internal resistance of amplifier \\
\end{longtable}
}

\end{solutionbox}
\begin{mnemonicbox}
``VIPS-BIO'' (Voltage, Input impedance, Power,
Supply, Bandwidth, Impedance Output)

\end{mnemonicbox}
\subsection*{Question 1(c) [7 marks]}\label{q1c}

\textbf{Define Biasing in transistor? Write down types of biasing
methods. Explain the voltage divider biasing method in details.}

\begin{solutionbox}
Biasing is the process of establishing a stable
operating point (Q-point) for a transistor by applying DC voltages.

{\def\LTcaptype{none} % do not increment counter
\begin{longtable}[]{@{}ll@{}}
\toprule\noalign{}
Biasing Method & Key Features \\
\midrule\noalign{}
\endhead
\bottomrule\noalign{}
\endlastfoot
\textbf{Fixed Bias} & Simple, poor stability \\
\textbf{Collector Feedback} & Self-adjusting, better stability \\
\textbf{Voltage Divider} & Best stability, widely used \\
\textbf{Emitter Bias} & Good stability, negative feedback \\
\end{longtable}
}

\textbf{Voltage Divider Biasing:}

\begin{verbatim}
flowchart LR
    VCC((+VCC)) {-{-}{-} R1}
    R1 {-{-}{-} R2 \& B((Base))}
    R2 {-{-}{-} GND}
    B {-{-}{-} C((Collector))}
    B {-{-}{-} E((Emitter))}
    C {-{-}{-} RC {-}{-}{-} VCC}
    E {-{-}{-} RE {-}{-}{-} GND}
\end{verbatim}

\begin{itemize}
\tightlist
\item
  \textbf{R1 \& R2}: Form voltage divider to provide stable base voltage
\item
  \textbf{RE}: Provides stabilization through negative feedback
\item
  \textbf{RC}: Determines collector current and voltage gain
\item
  \textbf{Stability}: Best stability against temperature variations
\end{itemize}

\end{solutionbox}
\begin{mnemonicbox}
``Divide Voltage Before Transistor Conducts''

\end{mnemonicbox}
\subsection*{Question 1(c) OR [7
marks]}\label{q1c}

\textbf{Explain Heat sink.}

\begin{solutionbox}
A heat sink is a passive heat exchanger that transfers
heat from electronic devices to the surrounding air.

\begin{verbatim}
flowchart LR
    A[Heat Source/Transistor] {-{-} B[Interface Material]}
    B {-{-} C[Heat Sink Base]}
    C {-{-} D[Heat Sink Fins]}
    D {-{-} E[Ambient Air]}
\end{verbatim}

{\def\LTcaptype{none} % do not increment counter
\begin{longtable}[]{@{}
  >{\raggedright\arraybackslash}p{(\linewidth - 2\tabcolsep) * \real{0.5238}}
  >{\raggedright\arraybackslash}p{(\linewidth - 2\tabcolsep) * \real{0.4762}}@{}}
\toprule\noalign{}
\begin{minipage}[b]{\linewidth}\raggedright
Component
\end{minipage} & \begin{minipage}[b]{\linewidth}\raggedright
Function
\end{minipage} \\
\midrule\noalign{}
\endhead
\bottomrule\noalign{}
\endlastfoot
\textbf{Base} & Conducts heat from device \\
\textbf{Fins} & Increases surface area for heat dissipation \\
\textbf{Thermal Interface Material} & Improves contact between device
and sink \\
\textbf{Types} & Extruded, Bonded, Folded, Die-cast \\
\end{longtable}
}

\begin{itemize}
\tightlist
\item
  \textbf{Thermal Resistance}: Lower is better for heat dissipation
\item
  \textbf{Material}: Usually aluminum or copper for good conductivity
\item
  \textbf{Surface Area}: More fins means better cooling
\item
  \textbf{Air Flow}: Critical for efficient heat removal
\end{itemize}

\end{solutionbox}
\begin{mnemonicbox}
``Heat Sinks Keep Transistors Running''

\end{mnemonicbox}
\subsection*{Question 2(a) [3 marks]}\label{q2a}

\textbf{Describe the D.C. and A.C. Load Lines.}

\begin{solutionbox}
Load lines graphically represent possible operating
points of a transistor on its characteristic curves.

\begin{verbatim}
                     Ic
                      ↑
                      |
                      |        DC Load Line
                      |       ╱
         Q{-point      |      ╱}
                      |     *
                      |    ╱  AC Load Line
                      |   ╱
                      |  ╱
                      | ╱
                      |╱
                      +{-{-}{-}{-}{-}{-}{-}{-}{-}{-}{-}{-}{-}{-}{-}{-} Vce}
                     0                Vcc
\end{verbatim}

\begin{itemize}
\tightlist
\item
  \textbf{DC Load Line}: Shows all possible operating points under DC
  conditions

  \begin{itemize}
  \tightlist
  \item
    \textbf{Equation}: Ic = (VCC - VCE)/RC
  \item
    \textbf{Endpoints}: (0, VCC/RC) and (VCC, 0)
  \end{itemize}
\item
  \textbf{AC Load Line}: Shows operating points during AC signal
  handling

  \begin{itemize}
  \tightlist
  \item
    \textbf{Steeper Slope}: Due to AC resistance being less than DC
  \item
    \textbf{Centered at Q-point}: The operating point established by
    biasing
  \end{itemize}
\end{itemize}

\end{solutionbox}
\begin{mnemonicbox}
``DC Draws Completely, AC Alters Course''

\end{mnemonicbox}
\subsection*{Question 2(b) [4 marks]}\label{q2b}

\textbf{Briefly explain bandwidth and gain-bandwidth product of an
amplifier.}

\begin{solutionbox}
Bandwidth and gain-bandwidth product are key
specifications for amplifier frequency performance.

\begin{verbatim}
flowchart LR
    A[Input] {-{-} B[Amplifierbr /Gain  Bandwidth]}
    B {-{-} C[Output]}
\end{verbatim}

{\def\LTcaptype{none} % do not increment counter
\begin{longtable}[]{@{}
  >{\raggedright\arraybackslash}p{(\linewidth - 2\tabcolsep) * \real{0.4583}}
  >{\raggedright\arraybackslash}p{(\linewidth - 2\tabcolsep) * \real{0.5417}}@{}}
\toprule\noalign{}
\begin{minipage}[b]{\linewidth}\raggedright
Parameter
\end{minipage} & \begin{minipage}[b]{\linewidth}\raggedright
Description
\end{minipage} \\
\midrule\noalign{}
\endhead
\bottomrule\noalign{}
\endlastfoot
\textbf{Bandwidth} & Frequency range where gain drops by less than
3dB \\
\textbf{Lower Cutoff (f_{1})} & Frequency where gain drops by 3dB at low
end \\
\textbf{Upper Cutoff (f_{2})} & Frequency where gain drops by 3dB at high
end \\
\textbf{Gain-Bandwidth Product} & Product of gain and bandwidth, remains
constant \\
\end{longtable}
}

\begin{itemize}
\tightlist
\item
  \textbf{Bandwidth Formula}: BW = f_{2} - f_{1}
\item
  \textbf{Gain-Bandwidth}: Remains constant when gain changes
\item
  \textbf{Trade-off}: Higher gain means lower bandwidth
\end{itemize}

\end{solutionbox}
\begin{mnemonicbox}
``Better Bandwidth Gets Perfect Transmission''

\end{mnemonicbox}
\subsection*{Question 2(c) [7 marks]}\label{q2c}

\textbf{Explain frequency response of two stage RC coupled amplifier.}

\begin{solutionbox}
The frequency response of a two-stage RC coupled
amplifier shows how gain varies with frequency.

\begin{verbatim}
flowchart LR
    A[Input] {-{-} B[Firstbr /Amplifierbr /Stage]}
    B {-{-}|RC Coupling| C[Secondbr /Amplifierbr /Stage]}
    C {-{-} D[Output]}
\end{verbatim}

\begin{verbatim}
    Gain(dB)
       ↑
       |    Mid{-frequency band}
       |    ┌───────────────┐
       |    │               │
       |    │               │
       |   ╱│               │╲
       |  ╱ │               │ ╲
       | ╱  │               │  ╲
       |╱   │               │   ╲
       +────┴───────────────┴──── Frequency(Hz)
          f_{1                 f_{2}}
       Low freq.         High freq.
\end{verbatim}

\begin{itemize}
\tightlist
\item
  \textbf{Low Frequency Response}: Limited by coupling capacitors

  \begin{itemize}
  \tightlist
  \item
    \textbf{Roll-off Rate}: -20 dB/decade for each stage
  \end{itemize}
\item
  \textbf{Mid Frequency Response}: Maximum and flat gain region

  \begin{itemize}
  \tightlist
  \item
    \textbf{Total Gain}: Product of individual stage gains
  \end{itemize}
\item
  \textbf{High Frequency Response}: Limited by transistor capacitances

  \begin{itemize}
  \tightlist
  \item
    \textbf{Roll-off Rate}: -20 dB/decade for each stage
  \end{itemize}
\end{itemize}

\end{solutionbox}
\begin{mnemonicbox}
``Low Couples Weakly, High Capacitance Blocks''

\end{mnemonicbox}
\subsection*{Question 2(a) OR [3
marks]}\label{q2a}

\textbf{Explain fixed bias circuit for transistor biasing.}

\begin{solutionbox}
Fixed bias is the simplest biasing method for
transistors, using a single resistor connected to the base.

\begin{verbatim}
        +Vcc
          |
          R
          |
          |   C
    {-{-}{-}{-}{-}{-}+{-}{-}{-}o}
    |     |
   Vin    |    RC
    |     |     |
    |     |    +Vcc
    |     |
    +{-{-}{-}{-}{-}+}
      Base  Collector
          |
          E
          |
         GND
\end{verbatim}

\begin{itemize}
\tightlist
\item
  \textbf{Circuit Elements}: Base resistor (RB) and collector resistor
  (RC)
\item
  \textbf{Base Current}: IB = (VCC - VBE)/RB
\item
  \textbf{Collector Current}: IC = β \times IB
\item
  \textbf{Drawbacks}: Poor stability, affected by temperature changes
\end{itemize}

\end{solutionbox}
\begin{mnemonicbox}
``Fix Bias, Face Burden'' (of instability)

\end{mnemonicbox}
\subsection*{Question 2(b) OR [4
marks]}\label{q2b}

\textbf{Explain frequency response of single stage amplifier.}

\begin{solutionbox}
The frequency response of a single-stage amplifier
shows gain variation across different frequencies.

\begin{verbatim}
    Gain(dB)
       ↑
       |    Mid{-frequency band}
       |    ┌───────────────┐
       |    │               │
       |   ╱│               │╲
       |  ╱ │               │ ╲
       | ╱  │               │  ╲
       |╱   │               │   ╲
       +────┴───────────────┴──── Frequency(Hz)
          f_{1                 f_{2}}
       Low freq.         High freq.
\end{verbatim}

{\def\LTcaptype{none} % do not increment counter
\begin{longtable}[]{@{}
  >{\raggedright\arraybackslash}p{(\linewidth - 2\tabcolsep) * \real{0.5000}}
  >{\raggedright\arraybackslash}p{(\linewidth - 2\tabcolsep) * \real{0.5000}}@{}}
\toprule\noalign{}
\begin{minipage}[b]{\linewidth}\raggedright
Frequency Range
\end{minipage} & \begin{minipage}[b]{\linewidth}\raggedright
Characteristics
\end{minipage} \\
\midrule\noalign{}
\endhead
\bottomrule\noalign{}
\endlastfoot
\textbf{Low frequency region} & Gain drops due to coupling capacitors \\
\textbf{Mid frequency region} & Maximum and constant gain \\
\textbf{High frequency region} & Gain decreases due to transistor
capacitances \\
\end{longtable}
}

\begin{itemize}
\tightlist
\item
  \textbf{Lower cutoff frequency}: Determined by coupling capacitors
\item
  \textbf{Upper cutoff frequency}: Limited by internal transistor
  capacitances
\item
  \textbf{Bandwidth}: Range between lower and upper cutoff frequencies
\end{itemize}

\end{solutionbox}
\begin{mnemonicbox}
``Low Middle High - Capacitors Matter Here''

\end{mnemonicbox}
\subsection*{Question 2(c) OR [7
marks]}\label{q2c}

\textbf{Compare transformer coupled amplifier and RC coupled amplifier}

\begin{solutionbox}

{\def\LTcaptype{none} % do not increment counter
\begin{longtable}[]{@{}lll@{}}
\toprule\noalign{}
Parameter & RC Coupled Amplifier & Transformer Coupled Amplifier \\
\midrule\noalign{}
\endhead
\bottomrule\noalign{}
\endlastfoot
\textbf{Coupling Element} & Resistor and capacitor & Transformer \\
\textbf{Frequency Response} & Wide bandwidth & Limited bandwidth \\
\textbf{Efficiency} & Lower (20-25\%) & Higher (50-60\%) \\
\textbf{Size \& Weight} & Small and lightweight & Bulky and heavy \\
\textbf{Cost} & Inexpensive & Expensive \\
\textbf{Impedance Matching} & Poor matching & Excellent matching \\
\textbf{Distortion} & Low distortion & Higher due to core saturation \\
\textbf{DC Isolation} & Good isolation & Excellent isolation \\
\textbf{Applications} & General purpose & Audio power amplifiers \\
\end{longtable}
}

\begin{verbatim}
flowchart TB
    subgraph RC
    A1[Transistor 1] {-{-}|Coupling Capacitor| B1[Transistor 2]}
    end
    subgraph Transformer
    A2[Transistor 1] {-{-}|Transformer| B2[Transistor 2]}
    end
\end{verbatim}

\end{solutionbox}
\begin{mnemonicbox}
``RC Takes Breadth, Transformer Takes Power''

\end{mnemonicbox}
\subsection*{Question 3(a) [3 marks]}\label{q3a}

\textbf{Explain in brief Direct coupled amplifier.}

\begin{solutionbox}
A direct-coupled amplifier connects stages without
coupling capacitors or transformers, allowing DC signal amplification.

\begin{verbatim}
flowchart LR
    In[Input] {-{-} A[First Stage]}
    A {-{-} Direct Connection {-}{-} B[Second Stage]}
    B {-{-} Out[Output]}
\end{verbatim}

\begin{itemize}
\tightlist
\item
  \textbf{DC Signal Handling}: Can amplify very low frequencies and DC
\item
  \textbf{No Coupling Elements}: Output of first stage directly connects
  to input of next
\item
  \textbf{Frequency Response}: Excellent low-frequency response
\item
  \textbf{Drawbacks}: Thermal drift, bias stability issues
\end{itemize}

\end{solutionbox}
\begin{mnemonicbox}
``Directly Connected, Down to Complete zero
frequency''

\end{mnemonicbox}
\subsection*{Question 3(b) [4 marks]}\label{q3b}

\textbf{Explain effects of emitter bypass capacitor and coupling
capacitor on frequency response of an amplifier.}

\begin{solutionbox}

{\def\LTcaptype{none} % do not increment counter
\begin{longtable}[]{@{}
  >{\raggedright\arraybackslash}p{(\linewidth - 4\tabcolsep) * \real{0.2157}}
  >{\raggedright\arraybackslash}p{(\linewidth - 4\tabcolsep) * \real{0.1961}}
  >{\raggedright\arraybackslash}p{(\linewidth - 4\tabcolsep) * \real{0.5882}}@{}}
\toprule\noalign{}
\begin{minipage}[b]{\linewidth}\raggedright
Capacitor
\end{minipage} & \begin{minipage}[b]{\linewidth}\raggedright
Function
\end{minipage} & \begin{minipage}[b]{\linewidth}\raggedright
Effect on Frequency Response
\end{minipage} \\
\midrule\noalign{}
\endhead
\bottomrule\noalign{}
\endlastfoot
\textbf{Emitter Bypass Capacitor} & Bypasses AC around RE & Increases
gain at mid and high frequencies \\
\textbf{Coupling Capacitor} & Blocks DC, passes AC & Determines lower
cutoff frequency \\
\end{longtable}
}

\begin{verbatim}
flowchart TB
    subgraph "Effects on Gain"
    A[Without Capacitors] {-{-}|"Low Gain"| B[With Coupling Only]}
    B {-{-}|"Medium Gain"| C[With Coupling + Bypass]}
    C {-{-}|"High Gain"| D[Ideal Response]}
    end
\end{verbatim}

\begin{itemize}
\tightlist
\item
  \textbf{Emitter Bypass Capacitor}:

  \begin{itemize}
  \tightlist
  \item
    \textbf{Without}: Lower gain due to negative feedback
  \item
    \textbf{With}: Higher gain as RE is bypassed for AC signals
  \end{itemize}
\item
  \textbf{Coupling Capacitor}:

  \begin{itemize}
  \tightlist
  \item
    \textbf{Too Small}: Poor low-frequency response
  \item
    \textbf{Larger Value}: Better low-frequency response
  \end{itemize}
\end{itemize}

\end{solutionbox}
\begin{mnemonicbox}
``Coupling Controls Lows, Bypass Boosts All''

\end{mnemonicbox}
\subsection*{Question 3(c) [7 marks]}\label{q3c}

\textbf{Draw Transistor Two Port Network and describe h-parameters for
it. Write down advantages of hybrid parameters.}

\begin{solutionbox}
A two-port network is a model to analyze transistor
behavior using h-parameters (hybrid parameters).

\begin{verbatim}
                i_{1                 i_{2}}
                                  
                |                  |
     +{-{-}{-}{-}{-}{-}{-}{-}{-}{-}+{-}{-}{-}{-}{-}{-}{-}{-}{-}{-}{-}{-}{-}{-}{-}{-}{-}{-}+{-}{-}{-}{-}{-}{-}{-}{-}{-}+}
     |          |                  |         |
     |          |                  |         |
 v_{1  |    +{-}{-}{-}{-}{-}+{-}{-}{-}{-}{-}{-}+    +{-}{-}{-}{-}{-}+{-}{-}{-}{-}{-}{-}+   |}
 ↓   |    |     |      |    |     |      |   |
     |    |  Two{-Port  |    |     ↓      |   |}
     +{-{-}{-}{-}+   Network  +{-}{-}{-}{-}+     v_{2}     +{-}{-}{-}+}
     |    |            |    |            |   |
     |    |            |    |            |   |
     |    +{-{-}{-}{-}{-}{-}{-}{-}{-}{-}{-}{-}+    +{-}{-}{-}{-}{-}{-}{-}{-}{-}{-}{-}{-}+   |}
     |                                       |
     +{-{-}{-}{-}{-}{-}{-}{-}{-}{-}{-}{-}{-}{-}{-}{-}{-}{-}{-}{-}{-}{-}{-}{-}{-}{-}{-}{-}{-}{-}{-}{-}{-}{-}{-}{-}{-}{-}{-}+}
\end{verbatim}

{\def\LTcaptype{none} % do not increment counter
\begin{longtable}[]{@{}
  >{\raggedright\arraybackslash}p{(\linewidth - 4\tabcolsep) * \real{0.3023}}
  >{\raggedright\arraybackslash}p{(\linewidth - 4\tabcolsep) * \real{0.2791}}
  >{\raggedright\arraybackslash}p{(\linewidth - 4\tabcolsep) * \real{0.4186}}@{}}
\toprule\noalign{}
\begin{minipage}[b]{\linewidth}\raggedright
H-Parameter
\end{minipage} & \begin{minipage}[b]{\linewidth}\raggedright
Definition
\end{minipage} & \begin{minipage}[b]{\linewidth}\raggedright
Physical Meaning
\end{minipage} \\
\midrule\noalign{}
\endhead
\bottomrule\noalign{}
\endlastfoot
\textbf{h_{1}_{1} (hᵢ_{e})} & Input impedance with output short-circuited &
Base-emitter resistance \\
\textbf{h_{1}_{2} (hᵣ_{e})} & Reverse voltage gain with input open-circuited &
Feedback from output to input \\
\textbf{h_{2}_{1} (hf_{e})} & Forward current gain with output short-circuited &
Current gain (β) \\
\textbf{h_{2}_{2} (ho_{e})} & Output admittance with input open-circuited &
Output conductance \\
\end{longtable}
}

\textbf{Advantages of H-Parameters:}

\begin{itemize}
\tightlist
\item
  \textbf{Easily Measured}: Direct measurement with simple circuits
\item
  \textbf{Mixed Units}: Uses ratios of voltage and current
\item
  \textbf{Model Accuracy}: Close approximation to transistor behavior
\item
  \textbf{Mathematical Simplicity}: Linear equations for analysis
\end{itemize}

\end{solutionbox}
\begin{mnemonicbox}
``Input, Reverse, Forward, Output - IRFO Parameters''

\end{mnemonicbox}
\subsection*{Question 3(a) OR [3
marks]}\label{q3a}

\textbf{Draw frequency response of an amplifier and indicate upper
cut-off frequency, lower cut-off frequency, bandwidth, and mid frequency
gain of the amplifier on the response.}

\begin{solutionbox}
The frequency response graph shows how gain varies with
frequency for an amplifier.

\begin{verbatim}
    Gain(dB)
       ↑
       |                Mid{-frequency gain}
       |    ┌─────────────────────────────┐
       |    │                             │
  0.707 {-+                             +{-}}
       |   /│                             │{}
       |  / │                             │ {}
       | /  │                             │  {}
       |/   │                             │   {}
       +────┴─────────────────────────────┴──── Frequency(log scale)
          f_{1                              f_{2}}
          │                               │
          │───────── Bandwidth ─────────│
          │                               │
     Lower cutoff                    Upper cutoff
     frequency                       frequency
\end{verbatim}

\begin{itemize}
\tightlist
\item
  \textbf{Mid-frequency Gain (Av)}: Maximum gain in the flat region
\item
  \textbf{Lower Cutoff Frequency (f_{1})}: Frequency where gain drops to
  0.707\timesAv (-3dB)
\item
  \textbf{Upper Cutoff Frequency (f_{2})}: Frequency where gain drops to
  0.707\timesAv (-3dB)
\item
  \textbf{Bandwidth}: The difference between upper and lower cutoff
  frequencies (f_{2} - f_{1})
\end{itemize}

\end{solutionbox}
\begin{mnemonicbox}
``Lower Bandwidth Upper Makes Amplifier Response''

\end{mnemonicbox}
\subsection*{Question 3(b) OR [4
marks]}\label{q3b}

\textbf{Describe the transistor used as a tuned amplifier.}

\begin{solutionbox}
A tuned amplifier uses LC resonant circuits to amplify
signals selectively at specific frequencies.

\begin{verbatim}
flowchart LR
    A[Input Signal] {-{-} B[Transistor Amplifier]}
    B {-{-} C[LC Tuned Circuit]}
    C {-{-} D[Output Signal]}
\end{verbatim}

{\def\LTcaptype{none} % do not increment counter
\begin{longtable}[]{@{}ll@{}}
\toprule\noalign{}
Component & Function \\
\midrule\noalign{}
\endhead
\bottomrule\noalign{}
\endlastfoot
\textbf{LC Tank Circuit} & Resonates at specific frequency \\
\textbf{Transistor} & Provides amplification \\
\textbf{Resonance Frequency} & f = 1/(2π\sqrtLC) \\
\textbf{Quality Factor (Q)} & Determines bandwidth \\
\end{longtable}
}

\begin{itemize}
\tightlist
\item
  \textbf{High Selectivity}: Amplifies signals at resonant frequency
\item
  \textbf{Applications}: RF receivers, transmitters, communications
\item
  \textbf{Types}: Single-tuned, double-tuned, stagger-tuned
\item
  \textbf{Bandwidth}: Inversely proportional to Q factor
\end{itemize}

\end{solutionbox}
\begin{mnemonicbox}
``Tuning LC Selects Signals Precisely''

\end{mnemonicbox}
\subsection*{Question 3(c) OR [7
marks]}\label{q3c}

\textbf{Describe the importance of h parameters in two port network.
Draw h-parameters circuit for CE amplifier.}

\begin{solutionbox}
H-parameters provide a complete mathematical model for
analyzing transistor circuits as two-port networks.

\textbf{Importance of h-parameters:}

{\def\LTcaptype{none} % do not increment counter
\begin{longtable}[]{@{}
  >{\raggedright\arraybackslash}p{(\linewidth - 2\tabcolsep) * \real{0.4000}}
  >{\raggedright\arraybackslash}p{(\linewidth - 2\tabcolsep) * \real{0.6000}}@{}}
\toprule\noalign{}
\begin{minipage}[b]{\linewidth}\raggedright
Aspect
\end{minipage} & \begin{minipage}[b]{\linewidth}\raggedright
Importance
\end{minipage} \\
\midrule\noalign{}
\endhead
\bottomrule\noalign{}
\endlastfoot
\textbf{Circuit Analysis} & Simplified equations for complex circuits \\
\textbf{Design Calculations} & Predict gain, input/output impedance \\
\textbf{Manufacturer Specs} & Standard way to specify transistor
characteristics \\
\textbf{Stability Analysis} & Determine stability conditions \\
\textbf{Frequency Dependence} & Model behavior across frequencies \\
\end{longtable}
}

\textbf{CE Amplifier h-parameter equivalent circuit:}

\begin{verbatim}
        +{-{-}{-}{-}{-}{-}{-}+      RC}
        |       |      ┌─┐
    ┌─┐ |       |   ┌──┘ └──┐
   IB─ |       |   |       |
        |       |   |       +{-{-}{-}o Vout}
Vin o───┤   hie  hre    |
        |       |   |       |
        |       |   |   hoe |
        |  hfe   |       |
        |       |   |       |
        +{-{-}{-}{-}{-}{-}{-}+   +{-}{-}{-}{-}{-}{-}{-}+}
            |           |
            └───────────┘
                 GND
\end{verbatim}

\begin{itemize}
\tightlist
\item
  \textbf{hie}: Input impedance (base-emitter resistance)
\item
  \textbf{hre}: Reverse voltage feedback ratio
\item
  \textbf{hfe}: Forward current gain (β)
\item
  \textbf{hoe}: Output admittance
\end{itemize}

\end{solutionbox}
\begin{mnemonicbox}
``Input Resistance, Feedback Ratio, Forward gain,
Output conductance''

\end{mnemonicbox}
\subsection*{Question 4(a) [3 marks]}\label{q4a}

\textbf{Describe the diode clipper circuit with necessary diagram.}

\begin{solutionbox}
A clipper circuit limits or clips off a portion of the
input signal that exceeds a certain voltage level.

\begin{verbatim}
flowchart LR
    A[Input Signal] {-{-} B[Diode Clipper]}
    B {-{-} C[Output Signal]}
\end{verbatim}

\begin{verbatim}
    Input                Output
     o─────┬───────────────o
           |
           |    D1
           ├────▶|─────┐
           |          ─┴─
           R           V
           |           │
           └───────────┘
              Ground
\end{verbatim}

\begin{itemize}
\tightlist
\item
  \textbf{Operation}: Diode conducts when voltage exceeds threshold
\item
  \textbf{Types}:

  \begin{itemize}
  \tightlist
  \item
    \textbf{Positive Clipper}: Clips positive half-cycles
  \item
    \textbf{Negative Clipper}: Clips negative half-cycles
  \item
    \textbf{Biased Clipper}: Clips at voltage level other than zero
  \end{itemize}
\end{itemize}

\end{solutionbox}
\begin{mnemonicbox}
``Clip Portions Passing Preset Points''

\end{mnemonicbox}
\subsection*{Question 4(b) [4 marks]}\label{q4b}

\textbf{Explain Short note on LDR.}

\begin{solutionbox}
LDR (Light Dependent Resistor) is a photoresistor whose
resistance decreases with increasing light intensity.

\begin{verbatim}
flowchart LR
    A[Light] {-{-} B[LDR]}
    B {-{-} C[Resistance Changes]}
\end{verbatim}

{\def\LTcaptype{none} % do not increment counter
\begin{longtable}[]{@{}ll@{}}
\toprule\noalign{}
Property & Description \\
\midrule\noalign{}
\endhead
\bottomrule\noalign{}
\endlastfoot
\textbf{Composition} & Cadmium sulfide (CdS) or cadmium selenide
(CdSe) \\
\textbf{Resistance Range} & 1MΩ (dark) to few KΩ (bright light) \\
\textbf{Response Time} & Typically 10-100ms \\
\textbf{Spectral Response} & Peak sensitivity in visible spectrum \\
\end{longtable}
}

\begin{itemize}
\tightlist
\item
  \textbf{Light Absorption}: Generates free carriers
\item
  \textbf{Resistance}: Inversely proportional to light intensity
\item
  \textbf{Applications}: Light sensors, automatic lighting, camera
  exposure control
\item
  \textbf{Symbol}: Variable resistor with arrow pointing inward
\end{itemize}

\end{solutionbox}
\begin{mnemonicbox}
``Light Decreases Resistance''

\end{mnemonicbox}
\subsection*{Question 4(c) [7 marks]}\label{q4c}

\textbf{Explain Darlington pair and its applications.}

\begin{solutionbox}
A Darlington pair consists of two transistors connected
so that the current amplified by the first is further amplified by the
second.

\begin{verbatim}
flowchart LR
    A[Input Signal] {-{-} B[Transistor 1]}
    B {-{-} C[Transistor 2]}
    C {-{-} D[Output Signal]}
\end{verbatim}

\begin{verbatim}
             +Vcc
               │
               │
               R
               │
               │
    Base o─────┴───┐
               |   |
               |   |  Collector
               |   +{-{-}{-}{-}{-}{-}{-}o}
               |   |
               |   |
               └─┬─┘
                 │
                 └─┐
               |   |
               |   | 
               |   |
               └─┬─┘
                 │
                 │
                GND
\end{verbatim}

{\def\LTcaptype{none} % do not increment counter
\begin{longtable}[]{@{}ll@{}}
\toprule\noalign{}
Characteristic & Description \\
\midrule\noalign{}
\endhead
\bottomrule\noalign{}
\endlastfoot
\textbf{Current Gain} & β\_total = β_{1} \times β_{2} (very high) \\
\textbf{Input Impedance} & Very high (β_{2} \times R\_e1) \\
\textbf{Output Impedance} & Low \\
\textbf{Switching Speed} & Slower than single transistor \\
\end{longtable}
}

\textbf{Applications:}

\begin{itemize}
\tightlist
\item
  \textbf{Power Amplifiers}: High current gain applications
\item
  \textbf{Audio Amplifiers}: High input impedance stages
\item
  \textbf{Buffer Circuits}: Minimizing loading effects
\item
  \textbf{Motor Control}: Driving high-current loads
\item
  \textbf{Touch Sensitive Switches}: High sensitivity due to high gain
\end{itemize}

\end{solutionbox}
\begin{mnemonicbox}
``Double Transistors Amplify Really Greatly''

\end{mnemonicbox}
\subsection*{Question 4(a) OR [3
marks]}\label{q4a}

\textbf{Describe the diode clamper circuit with necessary diagram.}

\begin{solutionbox}
A clamper circuit shifts the entire waveform up or down
by adding a DC component without changing its shape.

\begin{verbatim}
flowchart LR
    A[Input Signal] {-{-} B[Diode Clamper]}
    B {-{-} C[Output Signalbr /Shifted Waveform]}
\end{verbatim}

\begin{verbatim}
    Input         D           Output
     o─────┬─────|◄──────┬─────o
           |            ─┴─
           C             │
           |             R
           └─────────────┘
              Ground
\end{verbatim}

\begin{itemize}
\tightlist
\item
  \textbf{Operation}: Capacitor charges during one half-cycle,
  maintaining DC level
\item
  \textbf{Types}:

  \begin{itemize}
  \tightlist
  \item
    \textbf{Positive Clamper}: Shifts waveform upward
  \item
    \textbf{Negative Clamper}: Shifts waveform downward
  \item
    \textbf{Biased Clamper}: Shifts to specific DC level
  \end{itemize}
\end{itemize}

\end{solutionbox}
\begin{mnemonicbox}
``Clamps Peaks Down Consistently''

\end{mnemonicbox}
\subsection*{Question 4(b) OR [4
marks]}\label{q4b}

\textbf{Explain the working and applications of OLED.}

\begin{solutionbox}
OLED (Organic Light Emitting Diode) is a display
technology using organic compounds that emit light when electric current
passes through.

\begin{verbatim}
flowchart LR
    A[Electric Current] {-{-} B[OLED Layer]}
    B {-{-} C[Light Emission]}
\end{verbatim}

{\def\LTcaptype{none} % do not increment counter
\begin{longtable}[]{@{}ll@{}}
\toprule\noalign{}
Layer & Function \\
\midrule\noalign{}
\endhead
\bottomrule\noalign{}
\endlastfoot
\textbf{Cathode} & Injects electrons \\
\textbf{Emissive Layer} & Organic material that emits light \\
\textbf{Conductive Layer} & Conducts holes from anode \\
\textbf{Anode} & Injects holes (usually transparent) \\
\end{longtable}
}

\begin{itemize}
\tightlist
\item
  \textbf{Working Principle}: Electron-hole recombination creates
  photons
\item
  \textbf{Self-illuminating}: No backlight required unlike LCD
\item
  \textbf{Types}: PMOLED (Passive Matrix) and AMOLED (Active Matrix)
\item
  \textbf{Advantages}: Thinner, lighter, wider viewing angles, better
  contrast
\end{itemize}

\textbf{Applications:}

\begin{itemize}
\tightlist
\item
  Smartphones and tablets
\item
  Television screens
\item
  Digital camera displays
\item
  Wearable devices
\item
  Lighting panels
\end{itemize}

\end{solutionbox}
\begin{mnemonicbox}
``Organic Layers Emit Diode-light''

\end{mnemonicbox}
\subsection*{Question 4(c) OR [7
marks]}\label{q4c}

\textbf{Describe the transistor used as a relay driver.}

\begin{solutionbox}
A relay driver uses a transistor to control a relay,
allowing a low-current control signal to switch a high-current load.

\begin{verbatim}
flowchart LR
    A[Control Signal] {-{-} B[Transistor]}
    B {-{-} C[Relay Coil]}
    C {-{-} D[Switched Load]}
\end{verbatim}

\begin{verbatim}
    +Vcc
     │
     ┌┐
    ┌┘└┐ Relay
    │  │ Coil
    └┐┌┘
     ││
     ││    Flyback
     ││    Diode
     ││    ┌─┐
     └┴────┤{├─┐}
           └─┘ │
            ┌──┴─┐
            │    │
            │    │ Transistor
 Input ─────┤    │
            │    │
            └────┘
              │
             GND
\end{verbatim}

{\def\LTcaptype{none} % do not increment counter
\begin{longtable}[]{@{}ll@{}}
\toprule\noalign{}
Component & Function \\
\midrule\noalign{}
\endhead
\bottomrule\noalign{}
\endlastfoot
\textbf{Transistor} & Amplifies control signal to drive relay \\
\textbf{Flyback Diode} & Protects transistor from back EMF \\
\textbf{Base Resistor} & Limits base current \\
\textbf{Relay Coil} & Electromagnetic switch \\
\end{longtable}
}

\textbf{Applications:}

\begin{itemize}
\tightlist
\item
  Motor control circuits
\item
  Industrial automation
\item
  Automotive electronics
\item
  Home appliance control
\item
  Power distribution systems
\end{itemize}

\end{solutionbox}
\begin{mnemonicbox}
``Tiny Regulates Driving Relays''

\end{mnemonicbox}
\subsection*{Question 5(a) [3 marks]}\label{q5a}

\textbf{Draw circuit diagram of a variable power supply using LM317 IC.}

\begin{solutionbox}
LM317 is an adjustable voltage regulator that can be
used to create a variable power supply.

\begin{verbatim}
              LM317
    Input     ┌───┐
    o─────────┤IN │
              │   │
              │ADJ├─┬─────┬───o Output
              │   │ │     │
              └───┘ │     │
                │   │     │
                R1  R2    C2
                │   │     │
                └───┴─────┘
                    GND
\end{verbatim}

\begin{itemize}
\tightlist
\item
  \textbf{Components}:

  \begin{itemize}
  \tightlist
  \item
    \textbf{LM317}: Adjustable voltage regulator IC
  \item
    \textbf{R1}: Fixed 240Ω resistor
  \item
    \textbf{R2}: Variable resistor (potentiometer)
  \item
    \textbf{C1, C2}: Filter capacitors
  \end{itemize}
\item
  \textbf{Output Voltage}: VOUT = 1.25 \times (1 + R2/R1)
\end{itemize}

\end{solutionbox}
\begin{mnemonicbox}
``LM317 Makes Voltage Adjustable''

\end{mnemonicbox}
\subsection*{Question 5(b) [4 marks]}\label{q5b}

\textbf{Explain working of UPS.}

\begin{solutionbox}
UPS (Uninterruptible Power Supply) provides emergency
power when main power fails.

\begin{verbatim}
flowchart LR
    A[AC Mains] {-{-} B[Rectifier]}
    B {-{-} C[Battery Charger]}
    C {-{-} D[Battery]}
    D {-{-} E[Inverter]}
    E {-{-} F[Output Load]}
    A {-.Bypass.{-} F}
\end{verbatim}

{\def\LTcaptype{none} % do not increment counter
\begin{longtable}[]{@{}
  >{\raggedright\arraybackslash}p{(\linewidth - 2\tabcolsep) * \real{0.4762}}
  >{\raggedright\arraybackslash}p{(\linewidth - 2\tabcolsep) * \real{0.5238}}@{}}
\toprule\noalign{}
\begin{minipage}[b]{\linewidth}\raggedright
UPS Type
\end{minipage} & \begin{minipage}[b]{\linewidth}\raggedright
Operation
\end{minipage} \\
\midrule\noalign{}
\endhead
\bottomrule\noalign{}
\endlastfoot
\textbf{Offline/Standby} & Switches to battery when power fails \\
\textbf{Line-Interactive} & Regulates voltage and switches to battery \\
\textbf{Online/Double-Conversion} & Always powers from battery,
continuously charged \\
\end{longtable}
}

\begin{itemize}
\tightlist
\item
  \textbf{Main Components}: Rectifier, battery, inverter, control
  circuit
\item
  \textbf{Functions}:

  \begin{itemize}
  \tightlist
  \item
    Power conditioning
  \item
    Voltage regulation
  \item
    Surge protection
  \item
    Battery backup
  \end{itemize}
\end{itemize}

\end{solutionbox}
\begin{mnemonicbox}
``Uninterrupted Power Supplied During Blackouts''

\end{mnemonicbox}
\subsection*{Question 5(c) [7 marks]}\label{q5c}

\textbf{Draw and explain SMPS block diagram.}

\begin{solutionbox}
SMPS (Switch Mode Power Supply) uses switching
regulation to convert electrical power efficiently.

\begin{verbatim}
flowchart LR
    A[AC Input] {-{-} B[EMI Filter]}
    B {-{-} C[Rectifier \& Filter]}
    C {-{-} D[High Frequencybr /Switching Circuit]}
    D {-{-} E[Transformer]}
    E {-{-} F[Output Rectifierbr /\& Filter]}
    F {-{-} G[DC Output]}
    H[Feedback \& Control] {-{-} D}
    F {-{-} H}
\end{verbatim}

{\def\LTcaptype{none} % do not increment counter
\begin{longtable}[]{@{}ll@{}}
\toprule\noalign{}
Block & Function \\
\midrule\noalign{}
\endhead
\bottomrule\noalign{}
\endlastfoot
\textbf{EMI Filter} & Reduces electromagnetic interference \\
\textbf{Rectifier \& Filter} & Converts AC to DC and smooths it \\
\textbf{Switching Circuit} & Chops DC at high frequency \\
\textbf{Transformer} & Provides isolation and voltage conversion \\
\textbf{Output Rectifier} & Converts high-frequency AC back to DC \\
\textbf{Feedback Circuit} & Regulates output voltage \\
\end{longtable}
}

\begin{itemize}
\tightlist
\item
  \textbf{Advantages}: High efficiency (70-90\%), smaller size, lower
  weight
\item
  \textbf{Operation}: Uses PWM (Pulse Width Modulation) at 20-200 kHz
\item
  \textbf{Types}: Forward, Flyback, Push-pull, Half bridge, Full bridge
\item
  \textbf{Applications}: Computers, TVs, mobile chargers, LED drivers
\end{itemize}

\end{solutionbox}
\begin{mnemonicbox}
``Switch Makes Power Stable''

\end{mnemonicbox}
\subsection*{Question 5(a) OR [3
marks]}\label{q5a}

\textbf{Draw circuit diagram for +15 v Power Supply using its IC and
explain in brief}

\begin{solutionbox}
A +15V power supply can be built using the 7815 voltage
regulator IC.

\begin{verbatim}
    AC Input    Bridge     7815
      o         Rectifier   ┌───┐
    {        ┌───┐      │   │}
      o          │   ├──────┤IN │
                 │   │      │   │         +15V
                 │   │ C1   │OUT├─────────o
                 │   ├──┐   │   │    C2
                 └───┘  │   │   │    │
                        │   └───┘    │
                        │     │      │
                        └─────┴──────┘
                             GND
\end{verbatim}

\begin{itemize}
\tightlist
\item
  \textbf{Components}:

  \begin{itemize}
  \tightlist
  \item
    \textbf{7815}: Fixed +15V voltage regulator IC
  \item
    \textbf{Bridge Rectifier}: Converts AC to pulsating DC
  \item
    \textbf{C1}: Input filter capacitor (1000-2200µF)
  \item
    \textbf{C2}: Output filter capacitor (10-100µF)
  \end{itemize}
\item
  \textbf{Working}: Rectifies AC, filters it, then regulates to stable
  +15V DC
\end{itemize}

\end{solutionbox}
\begin{mnemonicbox}
``7815 Fixes Voltage To Fifteen''

\end{mnemonicbox}
\subsection*{Question 5(b) OR [4
marks]}\label{q5b}

\textbf{Explain working of solar battery charger circuits.}

\begin{solutionbox}
Solar battery chargers convert sunlight into electrical
energy to charge batteries.

\begin{verbatim}
flowchart LR
    A[Solar Panel] {-{-} B[Charge Controller]}
    B {-{-} C[Battery]}
    C {-{-} D[Load]}
\end{verbatim}

{\def\LTcaptype{none} % do not increment counter
\begin{longtable}[]{@{}
  >{\raggedright\arraybackslash}p{(\linewidth - 2\tabcolsep) * \real{0.5238}}
  >{\raggedright\arraybackslash}p{(\linewidth - 2\tabcolsep) * \real{0.4762}}@{}}
\toprule\noalign{}
\begin{minipage}[b]{\linewidth}\raggedright
Component
\end{minipage} & \begin{minipage}[b]{\linewidth}\raggedright
Function
\end{minipage} \\
\midrule\noalign{}
\endhead
\bottomrule\noalign{}
\endlastfoot
\textbf{Solar Panel} & Converts sunlight to electricity \\
\textbf{Blocking Diode} & Prevents battery discharge through panel at
night \\
\textbf{Charge Controller} & Regulates charging voltage and current \\
\textbf{Battery} & Stores electrical energy \\
\end{longtable}
}

\begin{itemize}
\tightlist
\item
  \textbf{Operating Modes}:

  \begin{itemize}
  \tightlist
  \item
    \textbf{Bulk Charging}: Maximum current until \textasciitilde80\%
    charged
  \item
    \textbf{Absorption}: Constant voltage, decreasing current
  \item
    \textbf{Float/Trickle}: Maintains full charge
  \end{itemize}
\item
  \textbf{Protection Features}: Overcharge, over-discharge, reverse
  polarity
\end{itemize}

\end{solutionbox}
\begin{mnemonicbox}
``Sun Charges Batteries Safely''

\end{mnemonicbox}
\subsection*{Question 5(c) OR [7
marks]}\label{q5c}

\textbf{Discuss comparison of linear regulated power supply with switch
mode power supply.}

\begin{solutionbox}

{\def\LTcaptype{none} % do not increment counter
\begin{longtable}[]{@{}
  >{\raggedright\arraybackslash}p{(\linewidth - 4\tabcolsep) * \real{0.1897}}
  >{\raggedright\arraybackslash}p{(\linewidth - 4\tabcolsep) * \real{0.3621}}
  >{\raggedright\arraybackslash}p{(\linewidth - 4\tabcolsep) * \real{0.4483}}@{}}
\toprule\noalign{}
\begin{minipage}[b]{\linewidth}\raggedright
Parameter
\end{minipage} & \begin{minipage}[b]{\linewidth}\raggedright
Linear Power Supply
\end{minipage} & \begin{minipage}[b]{\linewidth}\raggedright
Switch Mode Power Supply
\end{minipage} \\
\midrule\noalign{}
\endhead
\bottomrule\noalign{}
\endlastfoot
\textbf{Operating Principle} & Continuous voltage regulation &
High-frequency switching \\
\textbf{Efficiency} & Low (30-40\%) & High (70-90\%) \\
\textbf{Size \& Weight} & Large and heavy & Compact and lightweight \\
\textbf{Heat Dissipation} & High & Low \\
\textbf{Output Noise} & Very low & Higher (switching noise) \\
\textbf{Response Time} & Fast & Slower \\
\textbf{Component Count} & Lower & Higher \\
\textbf{Cost} & Less for low power & Less for high power \\
\textbf{Complexity} & Simple design & Complex design \\
\textbf{EMI} & Low & Higher (requires filtering) \\
\end{longtable}
}

\begin{verbatim}
flowchart TB
    subgraph Linear
    A1[Transformer] {-{-} B1[Rectifier]}
    B1 {-{-} C1[Filter]}
    C1 {-{-} D1[Series Pass Element]}
    D1 {-{-} E1[Output]}
    end

    subgraph SMPS
    A2[Rectifier] {-{-} B2[Switch]}
    B2 {-{-} C2[Transformer]}
    C2 {-{-} D2[Rectifier \& Filter]}
    D2 {-{-} E2[Output]}
    F2[Feedback] {-{-} B2}
    end
\end{verbatim}

\textbf{Applications:}

\begin{itemize}
\tightlist
\item
  \textbf{Linear}: Audio equipment, laboratory instruments, sensitive
  circuits
\item
  \textbf{SMPS}: Computers, TVs, mobile chargers, industrial power
  supplies
\end{itemize}

\end{solutionbox}
\begin{mnemonicbox}
``Linear Loves Low noise, Switching Saves Size''

\end{mnemonicbox}

\end{document}
