\documentclass[10pt,a4paper]{article}

% content/resources/templates/preamble.tex
\usepackage[margin=0.6in]{geometry}
\author{Milav Dabgar}
\usepackage{amsmath,amssymb,amsthm}
\usepackage{booktabs}
\usepackage{multirow}
\usepackage{xcolor}
\usepackage{tcolorbox}
\tcbuselibrary{breakable,skins}
\usepackage[colorlinks=true,linkcolor=blue]{hyperref}
\usepackage{titlesec}
\usepackage{enumitem}
\usepackage{tikz}
\usepackage{pgfplots}
\usepackage{circuitikz}
\usepackage[version=4]{mhchem}
\usepackage{longtable}
\usepackage{array}
\usepackage{float}
\usepackage{caption}
\usepackage{listings}

\lstset{
  basicstyle=\small\ttfamily,
  breaklines=true,
  breakatwhitespace=false,
  postbreak=\mbox{\textcolor{red}{$\hookrightarrow$}\space},
  float=false,
  numbers=left,
  numberstyle=\tiny\color{gray},
  numbersep=10pt,
  xleftmargin=2em,
  keywordstyle=\color{blue},
  commentstyle=\color{green!60!black},
  stringstyle=\color{purple},
  backgroundcolor=\color{gray!5},
  showstringspaces=false,
  tabsize=2,
  captionpos=b,
  keepspaces=true,
  columns=flexible
}

\pgfplotsset{compat=1.18}
\usetikzlibrary{shapes,arrows,positioning,calc,patterns,decorations.pathmorphing,decorations.markings,arrows.meta}

% Color scheme
\definecolor{headcolor}{RGB}{0,102,204}
\definecolor{keycolor}{RGB}{220,20,60}
\definecolor{solutioncolor}{RGB}{34,139,34}
\definecolor{mnemoniccolor}{RGB}{148,0,211}
\definecolor{codecolor}{RGB}{0,0,100}

% Spacing
\setlength{\parskip}{3pt}
\setlist[itemize]{nosep}
\setlist[enumerate]{nosep}

% Title formatting
\titleformat{\section}{\Large\bfseries\color{headcolor}}{\thesection}{1em}{}
\titleformat{\subsection}{\large\bfseries\color{headcolor}}{\thesubsection}{1em}{}

% Pandoc tightlist compatibility
\providecommand{\tightlist}{%
  \setlength{\itemsep}{0pt}\setlength{\parskip}{0pt}}

% Pandoc longtable compatibility
\newcounter{none}
\def\thenone{}


% content/resources/templates/gujarati-boxes.tex
\usepackage{fontspec}
\usepackage{polyglossia}

% Set Gujarati as main language (document is primarily in Gujarati)
% Note: gloss-gujarati.ldf doesn't exist in polyglossia, but it will use hyphenation patterns
\setdefaultlanguage{gujarati}
\setotherlanguage{english}

% Configure Gujarati font properly
% Use Language=Default to prevent polyglossia from trying to add language-specific features
% that don't exist for Gujarati, which causes "empty feature" warnings
\newfontfamily\gujaratifont[Script=Gujarati,AutoFakeBold=2.5,AutoFakeSlant=0.3]{Noto Sans Gujarati}
\setmainfont[Script=Gujarati,AutoFakeBold=2.5,AutoFakeSlant=0.3]{Noto Sans Gujarati}
% Use Noto Sans Gujarati for monospace to support Gujarati in text
\setmonofont[Scale=0.9]{Noto Sans Gujarati}

% Configure English to use the same font
\newfontfamily\englishfont[Script=Gujarati,AutoFakeBold=2.5,AutoFakeSlant=0.3]{Noto Sans Gujarati}

% Translations for polyglossia
\gappto\captionsgujarati{
  \renewcommand{\tablename}{કોષ્ટક}
  \renewcommand{\figurename}{આકૃતિ}
}

% Helper for TikZ nodes to ensure Gujarati font
\newcommand{\gu}[1]{{\gujaratifont #1}}

% Custom environments
\newtcolorbox{solutionbox}{
    breakable,
    enhanced,
    colback=solutioncolor!5!white,
    colframe=solutioncolor!75!black,
    fonttitle=\bfseries,
    title=જવાબ
}

\newtcolorbox{solutionboxnobreak}{
 colback=solutioncolor!5!white,
 colframe=solutioncolor!75!black,
 fonttitle=\bfseries,
 title=જવાબ
}

\newtcolorbox{keyformula}{
 breakable,
 enhanced,
 colback=keycolor!5!white,
 colframe=keycolor!75!black,
 fonttitle=\bfseries,
 title=રાસાયણિક સમીકરણ/સૂત્ર
}

\newtcolorbox{mnemonicbox}{
 breakable,
 enhanced,
 colback=mnemoniccolor!5!white,
 colframe=mnemoniccolor!75!black,
 fonttitle=\bfseries,
 title=મેમરી ટ્રીક
}


\begin{document}

\begin{center}
{\Huge\bfseries\color{headcolor} Subject Name (Gujarati)}\\[5pt]
{\LARGE 4321103 -- Winter 2024}\\[3pt]
{\large Semester 1 Study Material}\\[3pt]
{\normalsize\textit{Detailed Solutions and Explanations}}
\end{center}

\vspace{10pt}

\subsection*{પ્રશ્ન 1(અ) [3
ગુણ]}\label{uxaaauxab0uxab6uxaa8-1uxa85-3-uxa97uxaa3}

\textbf{CE રૂપરેખાંકન માટે એમ્પલીફાયર પરિમાણો Ai, Ri અને Ro સમજાવો.}

\begin{solutionbox}

Common Emitter (CE) એમ્પલીફાયર પરિમાણો:


{\def\LTcaptype{none} % do not increment counter
\vspace{-5pt}
\captionof{table}{CE એમ્પલીફાયર પરિમાણો}
\vspace{-10pt}
\begin{longtable}[]{@{}lll@{}}
\toprule\noalign{}
પરિમાણ & વ્યાખ્યા & CE રૂપરેખાંકન \\
\midrule\noalign{}
\endhead
\bottomrule\noalign{}
\endlastfoot
\textbf{કરંટ ગેઇન (Ai)} & આઉટપુટ કરંટનો ઇનપુટ કરંટ સાથેનો ગુણોત્તર & ઊંચો
(20-500) \\
\textbf{ઇનપુટ રેઝિસ્ટન્સ (Ri)} & ઇનપુટ પર કરંટ પ્રવાહનો વિરોધ & મધ્યમ (1-2
kΩ) \\
\textbf{આઉટપુટ રેઝિસ્ટન્સ (Ro)} & આઉટપુટ પર કરંટ પ્રવાહનો વિરોધ & ઊંચો (40-50
kΩ) \\
\end{longtable}
}

\textbf{આકૃતિ:}

\begin{center}
\textbf{Mermaid Diagram (Code)}
\begin{verbatim}
{Shaded}
{Highlighting}[]
graph LR
    I[Input Signal] {-{-}{} R[Ri: 1{-}2 kΩ] {-}{-}{} A[CE Amplifier] {-}{-}{} O[Output Signal]}
    A {-{-}{} RO[Ro: 40{-}50 kΩ]}
    A {-{-} "Ai: 20{-}500" {-}{-}{} O}
{Highlighting}
{Shaded}
\end{verbatim}
\end{center}

\textbf{યાદવાક્ય:} ``CAR'' - CE માં Current gain ઊંચો, Average input
resistance, અને Robust output resistance.

\end{solutionbox}
\subsection*{પ્રશ્ન 1(બ) [4
ગુણ]}\label{uxaaauxab0uxab6uxaa8-1uxaac-4-uxa97uxaa3}

\textbf{હીટ સિંક પર ટૂંકી નોંધ લખો.}

\begin{solutionbox}

\textbf{હીટ સિંક: એવું ઉપકરણ જે ઇલેક્ટ્રોનિક ઘટકોમાંથી ગરમી શોષે છે અને વિખેરે છે}

\textbf{આકૃતિ:}

\begin{center}
\textbf{Mermaid Diagram (Code)}
\begin{verbatim}
{Shaded}
{Highlighting}[]
graph LR
    T[Transistor] {-{-}{} HS[Heat Sink]}
    HS {-{-} "Heat Dissipation" {-}{-}{} A[Ambient Air]}

    subgraph Heat Sink Structure
    direction LR
    F[Fins] {-{-}{-} B[Base]}
    end
{Highlighting}
{Shaded}
\end{verbatim}
\end{center}

\textbf{હીટ સિંકના પ્રકારો:}

\begin{itemize}
\tightlist
\item
  \textbf{પેસિવ હીટ સિંક}: કુદરતી convection પર આધાર રાખે છે
\item
  \textbf{એક્ટિવ હીટ સિંક}: ફોર્સ્ડ એર convection માટે ફેન વાપરે છે
\item
  \textbf{લિક્વિડ-કૂલ્ડ હીટ સિંક}: વધુ સારા heat transfer માટે પ્રવાહી વાપરે છે
\end{itemize}

\textbf{મુખ્ય કાર્યો:}

\begin{itemize}
\tightlist
\item
  \textbf{થર્મલ કન્ડક્શન}: ઘટકોમાંથી ગરમી દૂર ખેંચે છે
\item
  \textbf{થર્મલ કન્વેક્શન}: ગરમી આસપાસની હવામાં ટ્રાન્સફર કરે છે
\item
  \textbf{સરફેસ એરિયા}: પાંખો વધુ સારા કૂલિંગ માટે સપાટી ક્ષેત્રફળ વધારે છે
\end{itemize}

\textbf{યાદવાક્ય:} ``CRAFT'' - Cooling through Radiation And Fins for
Transistors.

\end{solutionbox}
\subsection*{પ્રશ્ન 1(ક) [7
ગુણ]}\label{uxaaauxab0uxab6uxaa8-1uxa95-7-uxa97uxaa3}

\textbf{થર્મલ રનઅવે અને થર્મલ સ્ટેબિલિટીનું વર્ણન કરો. ટ્રાન્ઝિસ્ટરમાં થર્મલ રન અવે કેવી
રીતે દૂર કરી શકાય?}

\begin{solutionbox}

\textbf{થર્મલ રનઅવે: સ્વ-મજબૂત કરતી પ્રક્રિયા જ્યાં વધતા તાપમાનને કારણે વધુ કરંટ
પ્રવાહ થાય છે, જે આગળ તાપમાન વધારે છે}

\textbf{થર્મલ સ્ટેબિલિટી: તાપમાન ફેરફારો હોવા છતાં ટ્રાન્ઝિસ્ટર સર્કિટની સ્થિર
કામગીરી જાળવવાની ક્ષમતા}

\textbf{આકૃતિ:}

\begin{center}
\textbf{Mermaid Diagram (Code)}
\begin{verbatim}
{Shaded}
{Highlighting}[]
graph TD
    A[Increased Temperature] {-{-}{} B[Increased Collector Current]}
    B {-{-}{} C[More Power Dissipation]}
    C {-{-}{} A}

    D[Thermal Stability Methods] {-{-}{} E[Break This Cycle]}
{Highlighting}
{Shaded}
\end{verbatim}
\end{center}

\textbf{થર્મલ રનઅવે દૂર કરવાની પદ્ધતિઓ:}

\begin{itemize}
\tightlist
\item
  \textbf{હીટ સિંક}: વધારાની ગરમીને શોષે અને વિખેરે છે
\item
  \textbf{નેગેટિવ ફીડબેક}: સ્થિરતા માટે એમિટર રેઝિસ્ટર વાપરવો
\item
  \textbf{બાયસ સ્ટેબિલાઇઝેશન}: વોલ્ટેજ ડિવાઇડર બાયસિંગ સર્કિટ
\item
  \textbf{તાપમાન ક્ષતિપૂર્તિ}: ડાયોડ અથવા થર્મિસ્ટર્સનો ઉપયોગ કરવો
\end{itemize}

\textbf{મુખ્ય મુદ્દાઓ:}

\begin{itemize}
\tightlist
\item
  \textbf{IC = ICBO(1+β) + βIB}: કલેક્ટર કરંટ પરાધીનતા દર્શાવે છે
\item
  \textbf{ICBO બમણો થાય છે}: દર 10^\circC તાપમાન વધારા માટે
\item
  \textbf{સ્ટેબિલિટી ફેક્ટર S}: ઓછું S એટલે વધુ સારી સ્થિરતા
\end{itemize}

\textbf{યાદવાક્ય:} ``RENT'' - Reduce heat with sinks, Emitter resistors
stabilize, Negative feedback helps, Temperature compensation.

\end{solutionbox}
\subsection*{પ્રશ્ન 1(ક) OR [7
ગુણ]}\label{uxaaauxab0uxab6uxaa8-1uxa95-or-7-uxa97uxaa3}

\textbf{બાયસિંગ પદ્ધતિઓના પ્રકારો લખો. વોલ્ટેજ વિભાજક બાયસિંગ પદ્ધતિને વિગતોમાં
સમજાવો.}

\begin{solutionbox}

\textbf{બાયસિંગ પદ્ધતિઓના પ્રકારો:}

\begin{itemize}
\tightlist
\item
  ફિક્સ્ડ બાયસ
\item
  કલેક્ટર-ટુ-બેઝ બાયસ
\item
  વોલ્ટેજ ડિવાઇડર બાયસ
\item
  એમિટર બાયસ
\item
  કલેક્ટર ફીડબેક બાયસ
\end{itemize}

\textbf{વોલ્ટેજ ડિવાઇડર બાયસ સર્કિટ:}

\begin{verbatim}
    +Vcc
     |
     R1
     |
     +{-{-}{-}{-}+}
     |    |
     R2   |
     |    |
     +    |
     |    |
GND {-+{-}{-}+{-}+{-}{-}+{-} B}
         |    |
         |    C
         |    |
         +{-{-}{-}{-}+}
         |    E
         RE   |
         |    |
        GND  GND
\end{verbatim}

\textbf{કાર્યપ્રણાલી:}

\begin{itemize}
\tightlist
\item
  \textbf{R1 અને R2}: બેઝ વોલ્ટેજ પ્રદાન કરતા વોલ્ટેજ ડિવાઇડર બનાવે છે
\item
  \textbf{RE}: સ્થિરતા અને નેગેટિવ ફીડબેક પ્રદાન કરે છે
\item
  \textbf{સ્ટેબલ બાયસ પોઇન્ટ}: તાપમાન અને β ફેરફારોથી ઓછો પ્રભાવિત
\end{itemize}

\textbf{ફાયદાઓ:}

\begin{itemize}
\tightlist
\item
  \textbf{ઉત્તમ સ્થિરતા}: તાપમાન ફેરફારોથી ઓછો પ્રભાવિત
\item
  \textbf{β થી સ્વતંત્ર}: બાયસ પોઇન્ટ ટ્રાન્ઝિસ્ટર ગેઇનથી ખૂબ પ્રભાવિત નથી
\item
  \textbf{વ્યાપકપણે ઉપયોગમાં}: એમ્પ્લીફાયર માટે સૌથી સામાન્ય બાયસિંગ પદ્ધતિ
\end{itemize}

\textbf{યાદવાક્ય:} ``DIVE'' - Divider biasing Is Very Effective for
stability.

\end{solutionbox}
\subsection*{પ્રશ્ન 2(અ) [3
ગુણ]}\label{uxaaauxab0uxab6uxaa8-2uxa85-3-uxa97uxaa3}

\textbf{સ્ટેબિલિટી પરિબળનું લક્ષણો સમજાવો.}

\begin{solutionbox}

\textbf{સ્ટેબિલિટી ફેક્ટર (S): બાયસિંગ સર્કિટ તાપમાન ફેરફારો સાથે સ્થિર કામગીરી
કેટલી સારી રીતે જાળવે છે તેનું માપ}

\textbf{ગાણિતિક વ્યાખ્યા:} S = ΔIC/ΔICBO (કલેક્ટર કરંટમાં ફેરફાર / રિવર્સ સેચ્યુરેશન
કરંટમાં ફેરફાર)


{\def\LTcaptype{none} % do not increment counter
\vspace{-5pt}
\captionof{table}{વિવિધ બાયસ સર્કિટ્સ માટે સ્ટેબિલિટી ફેક્ટર્સ}
\vspace{-10pt}
\begin{longtable}[]{@{}lll@{}}
\toprule\noalign{}
બાયસિંગ મેથડ & સ્ટેબિલિટી ફેક્ટર & સ્ટેબિલિટી લેવલ \\
\midrule\noalign{}
\endhead
\bottomrule\noalign{}
\endlastfoot
ફિક્સ્ડ બાયસ & S = 1+β & ખરાબ \\
કલેક્ટર-ટુ-બેઝ & S = β/(1+β) & બેહતર \\
વોલ્ટેજ ડિવાઇડર & S \approx 1 & ઉત્તમ \\
\end{longtable}
}

\textbf{મુખ્ય લક્ષણો:}

\begin{itemize}
\tightlist
\item
  \textbf{ઓછો S મૂલ્ય}: વધુ સારી સ્થિરતા દર્શાવે છે (આદર્શ S=1)
\item
  \textbf{તાપમાન પ્રતિરોધ}: તાપમાન ફેરફારોથી રક્ષણની માત્રા માપે છે
\item
  \textbf{સર્કિટ ડિઝાઇન ટૂલ}: બાયસિંગ પદ્ધતિઓની તુલના કરવામાં મદદ કરે છે
\end{itemize}

\textbf{યાદવાક્ય:} ``SOS'' - Stability Of circuit Shows in its S-factor.

\end{solutionbox}
\subsection*{પ્રશ્ન 2(બ) [4
ગુણ]}\label{uxaaauxab0uxab6uxaa8-2uxaac-4-uxa97uxaa3}

\textbf{કાસ્કેડીંગની ડાયરેક્ટ કપ્લીંગ ટેકનિકનું વર્ણન કરો.}

\begin{solutionbox}

\textbf{ડાયરેક્ટ કપ્લીંગ: કપલિંગ કેપેસિટર્સ વિના સ્ટેજ જોડવું, એક સ્ટેજના કલેક્ટરને સીધો
આગલા સ્ટેજના બેઝ સાથે જોડવો}

\textbf{આકૃતિ:}

\begin{verbatim}
      +Vcc                +Vcc
        |                   |
        |                   |
        Rc                  Rc
        |                   |
  +{-{-}{-}{-}{-}+                   +{-}{-}{-}{-}{-}+}
  |     |                   |     |
  |     C       B           |     C  Output
  |     |{-{-}{-}{-}{-}{-}{-}+           |     |{-}{-}{-}{-}{-}{-}{-}+}
  |     |       |           |     |
Input   |       |           |     |
  +{-{-}{-}{-}{-}|B      |           |     |}
        |       |           |     |
        |       E           |     E
        |       |           |     |
       GND     GND         GND   GND
       
       First Stage          Second Stage
\end{verbatim}

\textbf{મુખ્ય લક્ષણો:}

\begin{itemize}
\tightlist
\item
  \textbf{કોઈ કપલિંગ ઘટકો નહીં}: સીધો ઇલેક્ટ્રિકલ કનેક્શન
\item
  \textbf{પૂર્ણ ફ્રીક્વન્સી રિસ્પોન્સ}: સારી લો-ફ્રીક્વન્સી પરફોર્મન્સ
\item
  \textbf{DC લેવલ શિફ્ટિંગ}: સ્ટેજ વચ્ચે જરૂરી છે
\end{itemize}

\textbf{એપ્લિકેશન્સ:}

\begin{itemize}
\tightlist
\item
  \textbf{ઓપરેશનલ એમ્પ્લીફાયર્સ}: આંતરિક સ્ટેજ
\item
  \textbf{DC એમ્પ્લીફાયર્સ}: જ્યાં લો-ફ્રીક્વન્સી રિસ્પોન્સ મહત્વપૂર્ણ છે
\end{itemize}

\textbf{યાદવાક્ય:} ``DIRECT'' - DC signals Immediately REach Connecting
Transistors.

\end{solutionbox}
\subsection*{પ્રશ્ન 2(ક) [7
ગુણ]}\label{uxaaauxab0uxab6uxaa8-2uxa95-7-uxa97uxaa3}

\textbf{બે તબક્કાનાં આર સી કપલ્ડ એમ્પલીફાયરનો આવર્તન પ્રતિભાવ સમજાવો.}

\begin{solutionbox}

\textbf{RC કપલ્ડ એમ્પ્લીફાયર: એમ્પલીફિકેશન સ્ટેજ વચ્ચે કપલિંગ માટે રેસિસ્ટર-કેપેસિટર
નેટવર્ક વાપરે છે}

\textbf{ફ્રીક્વન્સી રિસ્પોન્સ આકૃતિ:}

\begin{center}
\textbf{Mermaid Diagram (Code)}
\begin{verbatim}
{Shaded}
{Highlighting}[]
graph LR
    subgraph Frequency Response
    L[Low Frequency] {-{-}{-} M[Mid Frequency] {-}{-}{-} H[High Frequency]}
    end

    L {-{-} "20Hz{-}500Hz{}br /{}Gain rises" {-}{-}{} M}
    M {-{-} "500Hz{-}20kHz{}br /{}Flat gain" {-}{-}{} H}
    H {-{-} "{}20kHz{}br /{}Gain falls" {-}{-}{} D[Drop{-}off]}
{Highlighting}
{Shaded}
\end{verbatim}
\end{center}


{\def\LTcaptype{none} % do not increment counter
\vspace{-5pt}
\captionof{table}{ફ્રીક્વન્સી રીજન}
\vspace{-10pt}
\begin{longtable}[]{@{}llll@{}}
\toprule\noalign{}
રીજન & ફ્રીક્વન્સી રેન્જ & લક્ષણો & મર્યાદિત ઘટકો \\
\midrule\noalign{}
\endhead
\bottomrule\noalign{}
\endlastfoot
\textbf{લો} & 20Hz-500Hz & ફ્રીક્વન્સી સાથે ગેઇન વધે છે & કપલિંગ કેપેસિટર્સ \\
\textbf{મિડ} & 500Hz-20kHz & સ્થિર ગેઇન (મહત્તમ) & કોઈ નહીં \\
\textbf{હાઇ} & \textgreater20kHz & ફ્રીક્વન્સી સાથે ગેઇન ઘટે છે & ટ્રાન્ઝિસ્ટર
કેપેસિટન્સ \\
\end{longtable}
}

\textbf{બે-સ્ટેજની અસર:}

\begin{itemize}
\tightlist
\item
  \textbf{બેન્ડવિડ્થ}: સિંગલ સ્ટેજ કરતાં સાંકડી
\item
  \textbf{ગેઇન}: સિંગલ સ્ટેજના લગભગ વર્ગ જેટલો (A_{1} \times A_{2})
\item
  \textbf{ફેઝ શિફ્ટ}: લો અને હાઇ ફ્રીક્વન્સી પર બમણી
\end{itemize}

\textbf{યાદવાક્ય:} ``LMH'' - Low frequencies by coupling caps, Mid
frequencies flat, High frequencies by transistor caps.

\end{solutionbox}
\subsection*{પ્રશ્ન 2(અ) OR [3
ગુણ]}\label{uxaaauxab0uxab6uxaa8-2uxa85-or-3-uxa97uxaa3}

\textbf{એમ્પ્લીફાયરની બેન્ડવિડ્થ અને ગેઇન-બેન્ડવિડ્થ ઉત્પાદનને સંક્ષિપ્તમાં સમજાવો.}

\begin{solutionbox}

\textbf{બેન્ડવિડ્થ (BW): ફ્રીક્વન્સીઓની રેન્જ જ્યાં એમ્પ્લીફાયર ગેઇન મહત્તમ ગેઇનના
ઓછામાં ઓછા 70.7\% છે}

\textbf{ગેઇન-બેન્ડવિડ્થ પ્રોડક્ટ (GBP): વોલ્ટેજ ગેઇન અને બેન્ડવિડ્થનો ગુણાકાર, આપેલા
એમ્પલીફાયર માટે સ્થિર}

\textbf{આકૃતિ:}

\begin{center}
\textbf{Mermaid Diagram (Code)}
\begin{verbatim}
{Shaded}
{Highlighting}[]
graph LR
    F[Frequency] {-{-}{} G[Gain]}

    subgraph Bandwidth
    FL[f_{1: Lower Cutoff] {-}{-}{-} FM[Maximum Gain Region] {-}{-}{-} FH[f_{2}: Upper Cutoff]}
    end
    
    FL {-{-} "0.707" {-}{-}{} G}
    FH {-{-} "0.707" {-}{-}{} G}
{Highlighting}
{Shaded}
\end{verbatim}
\end{center}

\textbf{મુખ્ય સૂત્રો:}

\begin{itemize}
\tightlist
\item
  \textbf{બેન્ડવિડ્થ}: BW = f_{2} - f_{1}
\item
  \textbf{ગેઇન-બેન્ડવિડ્થ પ્રોડક્ટ}: GBP = A_{0} \times BW (સ્થિર)
\end{itemize}

\textbf{યાદવાક્ય:} ``BAND'' - Bandwidth And gain Never Drop together (એક
વધે ત્યારે બીજો ઘટે).

\end{solutionbox}
\subsection*{પ્રશ્ન 2(બ) OR [4
ગુણ]}\label{uxaaauxab0uxab6uxaa8-2uxaac-or-4-uxa97uxaa3}

\textbf{એમ્પલીફાયરના ફ્રીક્વન્સી રિસ્પોન્સ પર એમિટર બાયપાસ કેપેસિટર અને કપલિંગ
કેપેસિટરની અસરો સમજાવો.}

\begin{solutionbox}

\textbf{ફ્રીક્વન્સી રિસ્પોન્સ પર અસરો:}


{\def\LTcaptype{none} % do not increment counter
\vspace{-5pt}
\captionof{table}{કેપેસિટર અસરો}
\vspace{-10pt}
\begin{longtable}[]{@{}
  >{\raggedright\arraybackslash}p{(\linewidth - 4\tabcolsep) * \real{0.2157}}
  >{\raggedright\arraybackslash}p{(\linewidth - 4\tabcolsep) * \real{0.1961}}
  >{\raggedright\arraybackslash}p{(\linewidth - 4\tabcolsep) * \real{0.5882}}@{}}
\toprule\noalign{}
\begin{minipage}[b]{\linewidth}\raggedright
કેપેસિટર
\end{minipage} & \begin{minipage}[b]{\linewidth}\raggedright
કાર્ય
\end{minipage} & \begin{minipage}[b]{\linewidth}\raggedright
ફ્રીક્વન્સી રિસ્પોન્સ પર અસર
\end{minipage} \\
\midrule\noalign{}
\endhead
\bottomrule\noalign{}
\endlastfoot
\textbf{કપલિંગ કેપેસિટર (Cc)} & DC બ્લોક કરે, AC પસાર કરે & લો-ફ્રીક્વન્સી
રિસ્પોન્સ મર્યાદિત કરે \\
\textbf{બાયપાસ કેપેસિટર (Ce)} & એમિટર રેઝિસ્ટરને બાયપાસ કરે & મિડ અને હાઇ
ફ્રીક્વન્સી પર ગેઇન વધારે \\
\end{longtable}
}

\textbf{આકૃતિ:}

\begin{verbatim}
    +Vcc
     |
     Rc
     |
     +{-{-}{-}{-}{-}{-}{-}+}
     |       |
 Cc  |       |
 ||{-{-}+       C}
 ||  |       |
Input  B     |
     |       |
     |       E
     |       |
     Re      |
     |       |
     +{-{-}||{-}{-}{-}+}
     |   Ce
    GND
\end{verbatim}

\textbf{મુખ્ય અસરો:}

\begin{itemize}
\tightlist
\item
  \textbf{Ce વગર}: ઓછો ગેઇન, વધુ સારી સ્થિરતા, વધુ સારો લો-ફ્રીક્વન્સી રિસ્પોન્સ
\item
  \textbf{Cc વગર}: DC કપલિંગ, ઉત્તમ લો-ફ્રીક્વન્સી રિસ્પોન્સ
\item
  \textbf{કેપેસિટર મૂલ્યો}: કટઓફ ફ્રીક્વન્સીઓ (f_{1}, f_{2}) નક્કી કરે છે
\end{itemize}

\textbf{યાદવાક્ય:} ``CELL'' - Coupling affects Extremely Low frequencies,
bypass affects Low to high.

\end{solutionbox}
\subsection*{પ્રશ્ન 2(ક) OR [7
ગુણ]}\label{uxaaauxab0uxab6uxaa8-2uxa95-or-7-uxa97uxaa3}

\textbf{ટ્રાન્સફોર્મર કપલ્ડ એમ્પલીફાયર અને આરસી કપલ્ડ એમ્પલીફાયરની સરખામણી કરો}

\begin{solutionbox}


{\def\LTcaptype{none} % do not increment counter
\vspace{-5pt}
\captionof{table}{ટ્રાન્સફોર્મર કપલ્ડ vs RC કપલ્ડ એમ્પલીફાયરની સરખામણી}
\vspace{-10pt}
\begin{longtable}[]{@{}lll@{}}
\toprule\noalign{}
લક્ષણ & ટ્રાન્સફોર્મર કપલ્ડ & RC કપલ્ડ \\
\midrule\noalign{}
\endhead
\bottomrule\noalign{}
\endlastfoot
\textbf{કપલિંગ ઘટક} & ટ્રાન્સફોર્મર & કેપેસિટર અને રેઝિસ્ટર \\
\textbf{કાર્યક્ષમતા} & ઊંચી (90\%) & મધ્યમ (20-30\%) \\
\textbf{કદ અને વજન} & મોટું અને ભારે & કોમ્પેક્ટ અને હલકું \\
\textbf{ખર્ચ} & મોંઘું & સસ્તું \\
\textbf{ફ્રીક્વન્સી રિસ્પોન્સ} & ખરાબ (મર્યાદિત બેન્ડવિડ્થ) & સારો (વિશાળ
બેન્ડવિડ્થ) \\
\textbf{ઇમ્પીડન્સ મેચિંગ} & ઉત્તમ & ખરાબ \\
\textbf{DC આઇસોલેશન} & સંપૂર્ણ & માત્ર AC સિગ્નલ્સ \\
\textbf{ડિસ્ટોર્શન} & ઊંચું & નીચું \\
\end{longtable}
}

\textbf{આકૃતિ:}

\begin{verbatim}
graph TB
    subgraph "RC Coupled"
    RC[Resistor{-Capacitor] {-}{-} RCF[Flat Responsebr /Wide Bandwidth]}
    end

    subgraph "Transformer Coupled"
    TC[Transformer] {-{-} TCF[Peaked Responsebr /Narrow Bandwidth]}
    end
\end{verbatim}

\textbf{એપ્લિકેશન્સ:}

\begin{itemize}
\tightlist
\item
  \textbf{RC કપલ્ડ}: ઓડિયો એમ્પલીફાયર્સ, જનરલ-પર્પઝ એમ્પલીફાયર્સ
\item
  \textbf{ટ્રાન્સફોર્મર કપલ્ડ}: પાવર એમ્પલીફાયર્સ, રેડિયો ટ્રાન્સમિટર્સ
\end{itemize}

\textbf{યાદવાક્ય:} ``TRIP'' - Transformers are Robust for Impedance
matching, Problematic for bandwidth.

\end{solutionbox}
\subsection*{પ્રશ્ન 3(અ) [3
ગુણ]}\label{uxaaauxab0uxab6uxaa8-3uxa85-3-uxa97uxaa3}

\textbf{ટ્યુન કરેલ એમ્પલીફાયર તરીકે ઉપયોગમાં લેવાતા ટ્રાન્ઝિસ્ટરનું વર્ણન કરો.}

\begin{solutionbox}

\textbf{ટ્યુન્ડ એમ્પલીફાયર: એમ્પલીફાયર જે સાંકડા ફ્રીક્વન્સી બેન્ડમાં સિગ્નલ્સને
પસંદગીપૂર્વક એમ્પલિફાય કરે છે}

\textbf{આકૃતિ:}

\begin{verbatim}
    +Vcc
     |
     |
     +{-{-}{-}+}
     |   |
     L   |
     |   |
 Cin |   |
 ||{-{-}+{-}{-}{-}+}
 ||  |   |
Input  B |
     |   |
     |   C      Cout
     |   +{-{-}{-}{-}{-}{-}||{-}{-}{-}{-}+ Output}
     |   |             |
     |   E             |
     |   |             |
     |  Re             |
     |   |             |
    GND GND           GND
\end{verbatim}

\textbf{મુખ્ય ઘટકો:}

\begin{itemize}
\tightlist
\item
  \textbf{LC ટેંક સર્કિટ}: રેઝોનન્ટ ફ્રીક્વન્સી નક્કી કરે છે
\item
  \textbf{ટ્રાન્ઝિસ્ટર}: એમ્પલીફિકેશન પૂરું પાડે છે
\item
  \textbf{રેઝોનન્ટ ફ્રીક્વન્સી}: f_{0} = 1/(2π\sqrtLC)
\end{itemize}

\textbf{એપ્લિકેશન્સ:}

\begin{itemize}
\tightlist
\item
  \textbf{રેડિયો રિસીવર્સ}: ઇચ્છિત ફ્રીક્વન્સી પસંદ કરે છે
\item
  \textbf{TV ટ્યુનર્સ}: ચેનલ પસંદગી
\item
  \textbf{RF એમ્પલીફાયર્સ}: કમ્યુનિકેશન સિસ્ટમ્સ
\end{itemize}

\textbf{યાદવાક્ય:} ``TUNE'' - Transistors Using Narrowband Elements for
frequency selection.

\end{solutionbox}
\subsection*{પ્રશ્ન 3(બ) [4
ગુણ]}\label{uxaaauxab0uxab6uxaa8-3uxaac-4-uxa97uxaa3}

\textbf{ડાયરેક્ટ કપલ્ડ એમ્પલીફાયરને સંક્ષિપ્તમાં સમજાવો.}

\begin{solutionbox}

\textbf{ડાયરેક્ટ કપલ્ડ એમ્પલીફાયર: મલ્ટિપલ સ્ટેજ એમ્પલીફાયર જ્યાં કપલિંગ કેપેસિટર્સ
અથવા ટ્રાન્સફોર્મર્સ વગર સ્ટેજ સીધા જોડાયેલા છે}

\textbf{આકૃતિ:}

\begin{center}
\textbf{Mermaid Diagram (Code)}
\begin{verbatim}
{Shaded}
{Highlighting}[]
graph LR
    I[Input] {-{-}{} T1[Transistor 1] {-}{-}{} T2[Transistor 2] {-}{-}{} O[Output]}
    T1 {-{-} "Direct Connection{}br /{}No Coupling Components" {-}{-}{} T2}
{Highlighting}
{Shaded}
\end{verbatim}
\end{center}

\textbf{મુખ્ય લક્ષણો:}

\begin{itemize}
\tightlist
\item
  \textbf{DC એમ્પલીફિકેશન}: DC થી ઊંચી ફ્રીક્વન્સી સુધી એમ્પલિફાય કરી શકે છે
\item
  \textbf{કોઈ કપલિંગ ઘટકો નહીં}: કલેક્ટર આગલા બેઝ સાથે સીધો જોડાયેલો
\item
  \textbf{લેવલ શિફ્ટિંગ}: સ્ટેજ વચ્ચે જરૂરી છે
\item
  \textbf{થર્મલ ડ્રિફ્ટ}: સીધા DC કપલિંગને કારણે પડકાર
\end{itemize}

\textbf{એપ્લિકેશન્સ:}

\begin{itemize}
\tightlist
\item
  \textbf{ઓપરેશનલ એમ્પલીફાયર્સ}: આંતરિક સ્ટેજ
\item
  \textbf{DC એમ્પલીફાયર્સ}: લેબોરેટરી ઇન્સ્ટ્રુમેન્ટ્સ
\item
  \textbf{સેન્સિંગ સર્કિટ્સ}: તાપમાન અને દબાણ સેન્સર્સ
\end{itemize}

\textbf{યાદવાક્ય:} ``DCAP'' - Direct Coupled Amplifier Passes all
frequencies including DC.

\end{solutionbox}
\subsection*{પ્રશ્ન 3(ક) [7
ગુણ]}\label{uxaaauxab0uxab6uxaa8-3uxa95-7-uxa97uxaa3}

\textbf{બે પોર્ટ નેટવર્કમાં h પરિમાણોનું મહત્વ વર્ણવો. CE એમ્પલીફાયર માટે
h-પેરામીટર્સ સર્કિટ દોરો.}

\begin{solutionbox}

\textbf{h-પેરામીટર્સ (હાઇબ્રિડ પેરામીટર્સ): ચાર પેરામીટર્સનો સેટ જે બે-પોર્ટ નેટવર્કનું
વર્તન વ્યાખ્યાયિત કરે છે}

\textbf{મહત્વ:}

\begin{itemize}
\tightlist
\item
  \textbf{સંપૂર્ણ ચરિત્રીકરણ}: એમ્પલીફાયર વર્તનને સંપૂર્ણ રીતે વર્ણવે છે
\item
  \textbf{સરળ માપન}: સરળ સ્થિતિઓ હેઠળ માપી શકાય છે
\item
  \textbf{વિશ્લેષણ ટૂલ}: સર્કિટ વિશ્લેષણને સરળ બનાવે છે
\item
  \textbf{માનકીકૃત અભિગમ}: ટ્રાન્ઝિસ્ટર્સની તુલના માટે સાર્વત્રિક પદ્ધતિ
\end{itemize}

\textbf{h-પેરામીટર સમીકરણો:}

\begin{itemize}
\tightlist
\item
  V_{1} = h_{1}_{1}I_{1} + h_{1}_{2}V_{2}
\item
  I_{2} = h_{2}_{1}I_{1} + h_{2}_{2}V_{2}
\end{itemize}

\textbf{CE એમ્પલીફાયર માટે h-પેરામીટર સર્કિટ:}

\begin{verbatim}
                     +
                     |
                    Ic
               +{-{-}{-}{-}{-}+{-}{-}{-}{-}{-}+}
               |     |     |
         +     |     |     |
        Ii     |    hoe    |
     +{-{-}{-}{-}+   |     |     |}
     |     |   |     |     |    +
  +  |    hie  |    hfe·Ii |   Vo
 Vi  |     |   |     |     |    {-}
  {-  |     |   |     |     |}
     +{-{-}+{-}{-}+   |     |     |}
        |      |     |     |
        +{{-}{-}{-}{-}{-}+     |     |}
        hre·Vo       |     |
               |     |     |
               +{-{-}{-}{-}{-}+{-}{-}{-}{-}{-}+}
                     |
                     +
\end{verbatim}


{\def\LTcaptype{none} % do not increment counter
\vspace{-5pt}
\captionof{table}{CE કોન્ફિગરેશન માટે h-પેરામીટર્સ}
\vspace{-10pt}
\begin{longtable}[]{@{}llll@{}}
\toprule\noalign{}
પેરામીટર & સિમ્બોલ & સામાન્ય મૂલ્ય & ભૌતિક અર્થ \\
\midrule\noalign{}
\endhead
\bottomrule\noalign{}
\endlastfoot
\textbf{ઇનપુટ ઇમ્પીડન્સ} & h_{1}_{1} (hie) & 1-2 kΩ & આઉટપુટ શોર્ટ સાથે ઇનપુટ
રેઝિસ્ટન્સ \\
\textbf{રિવર્સ વોલ્ટેજ ટ્રાન્સફર} & h_{1}_{2} (hre) & 1-4 \times 10^{-}^{4} & રિવર્સ ફીડબેક
રેશિયો \\
\textbf{ફોરવર્ડ કરંટ ટ્રાન્સફર} & h_{2}_{1} (hfe) & 20-500 & કરંટ ગેઇન (β) \\
\textbf{આઉટપુટ એડમિટન્સ} & h_{2}_{2} (hoe) & 20-50 μS & આઉટપુટ કન્ડક્ટન્સ \\
\end{longtable}
}

\textbf{યાદવાક્ય:} ``HIRE'' - h-parameters Include Resistance and current
gain Effectively.

\end{solutionbox}
\subsection*{પ્રશ્ન 3(અ) OR [3
ગુણ]}\label{uxaaauxab0uxab6uxaa8-3uxa85-or-3-uxa97uxaa3}

\textbf{ટ્રાન્સફોર્મર કપલ્ડ એમ્પલીફાયર અને ડાયરેક્ટ કપલ્ડ એમ્પલીફાયરની સરખામણી
કરો.}

\begin{solutionbox}


{\def\LTcaptype{none} % do not increment counter
\vspace{-5pt}
\captionof{table}{ટ્રાન્સફોર્મર અને ડાયરેક્ટ કપલ્ડ એમ્પલીફાયર વચ્ચે સરખામણી}
\vspace{-10pt}
\begin{longtable}[]{@{}
  >{\raggedright\arraybackslash}p{(\linewidth - 4\tabcolsep) * \real{0.2000}}
  >{\raggedright\arraybackslash}p{(\linewidth - 4\tabcolsep) * \real{0.4667}}
  >{\raggedright\arraybackslash}p{(\linewidth - 4\tabcolsep) * \real{0.3333}}@{}}
\toprule\noalign{}
\begin{minipage}[b]{\linewidth}\raggedright
લક્ષણ
\end{minipage} & \begin{minipage}[b]{\linewidth}\raggedright
ટ્રાન્સફોર્મર કપલ્ડ
\end{minipage} & \begin{minipage}[b]{\linewidth}\raggedright
ડાયરેક્ટ કપલ્ડ
\end{minipage} \\
\midrule\noalign{}
\endhead
\bottomrule\noalign{}
\endlastfoot
\textbf{કપલિંગ ઘટક} & ટ્રાન્સફોર્મર & કોઈ નહીં (સીધું કનેક્શન) \\
\textbf{ફ્રીક્વન્સી રિસ્પોન્સ} & લો ફ્રીક્વન્સી પર મર્યાદિત & ઉત્તમ (DC થી ઊંચી
ફ્રીક્વન્સી) \\
\textbf{DC આઇસોલેશન} & સંપૂર્ણ & કોઈ નહીં \\
\textbf{કદ} & મોટું & કોમ્પેક્ટ \\
\textbf{ખર્ચ} & ઊંચો & નિમ્ન \\
\textbf{DC શિફ્ટ સમસ્યા} & ના & હા \\
\end{longtable}
}

\textbf{આકૃતિ:}

\begin{center}
\textbf{Mermaid Diagram (Code)}
\begin{verbatim}
{Shaded}
{Highlighting}[]
graph TD
    subgraph "Transformer Coupled"
    T1[Transistor 1] {-{-}{-} TR[Transformer] {-}{-}{-} T2[Transistor 2]}
    end

    subgraph "Direct Coupled"
    D1[Transistor 1] {-{-} "Direct Connection" {-}{-}{} D2[Transistor 2]}
    end
{Highlighting}
{Shaded}
\end{verbatim}
\end{center}

\textbf{યાદવાક્ય:} ``TDC'' - Transformers provide DC isolation, Direct
provides Complete frequency range.

\end{solutionbox}
\subsection*{પ્રશ્ન 3(બ) OR [4
ગુણ]}\label{uxaaauxab0uxab6uxaa8-3uxaac-or-4-uxa97uxaa3}

\textbf{કોમન એમિટર એમ્પલીફાયરનું સર્કિટ ડાયાગ્રામ દોરો અને સમજાવો.}

\begin{solutionbox}

\textbf{કોમન એમિટર એમ્પલીફાયર: એવી કોન્ફિગરેશન જ્યાં એમિટર ઇનપુટ અને આઉટપુટ બંને
સર્કિટ્સ માટે કોમન છે}

\textbf{સર્કિટ ડાયાગ્રામ:}

\begin{verbatim}
                 +Vcc
                  |
                  |
                  Rc
                  |
                  +{-{-}{-}{-}{-}{-}{-}{-}+ Output}
                  |        |
             +{-{-}{-}{-}+        |}
             |    |        |
     Input   |    C        |
     +{-{-}{-}{-}{-}{-}{-}|B   |        |}
     |       |    |        |
     |       |    E        |
     |       |    |        |
     |       |    +        |
     |       |    |        |
     |       |   Re        |
     |       |    |        |
    GND     GND  GND      GND
\end{verbatim}

\textbf{કાર્યપ્રણાલી:}

\begin{itemize}
\tightlist
\item
  \textbf{ઇનપુટ}: બેઝ અને એમિટર વચ્ચે લાગુ કરવામાં આવે છે
\item
  \textbf{આઉટપુટ}: કલેક્ટર અને એમિટરથી લેવામાં આવે છે
\item
  \textbf{ફેઝ શિફ્ટ}: ઇનપુટ અને આઉટપુટ વચ્ચે 180^\circ
\item
  \textbf{ગેઇન}: ઊંચો વોલ્ટેજ અને કરંટ ગેઇન
\end{itemize}

\textbf{મુખ્ય લક્ષણો:}

\begin{itemize}
\tightlist
\item
  \textbf{ઊંચો ગેઇન}: સામાન્ય વોલ્ટેજ ગેઇન 300-1000
\item
  \textbf{મધ્યમ ઇનપુટ ઇમ્પીડન્સ}: 1-2 kΩ
\item
  \textbf{ઊંચો આઉટપુટ ઇમ્પીડન્સ}: 40-50 kΩ
\item
  \textbf{સિગ્નલ ઇન્વર્ઝન}: આઉટપુટ ઇન્વર્ટેડ છે
\end{itemize}

\textbf{યાદવાક્ય:} ``CEA'' - Common Emitter Amplifies with signal
inversion.

\end{solutionbox}
\subsection*{પ્રશ્ન 3(ક) OR [7
ગુણ]}\label{uxaaauxab0uxab6uxaa8-3uxa95-or-7-uxa97uxaa3}

\textbf{ટ્રાન્ઝિસ્ટર ટુ પોર્ટ નેટવર્ક દોરો અને તેના માટે h-પેરામીટર્સનું વર્ણન કરો.
હાઇબ્રિડ પરિમાણોના ફાયદા લખો.}

\begin{solutionbox}

\textbf{ટ્રાન્ઝિસ્ટર ટુ-પોર્ટ નેટવર્ક:}

\begin{verbatim}
        I1             I2
        {-{-}            {-}{-}}
    +{-{-}{-}{-}{-}{-}{-}+      +{-}{-}{-}{-}{-}{-}{-}+}
    |       |      |       |
    |       |      |       |
  + |       |      |       | +
 V1 |  Two  |      |  Port | V2
  {- |       |      |       | {-}}
    |       |      |       |
    |       |      |       |
    +{-{-}{-}{-}{-}{-}{-}+      +{-}{-}{-}{-}{-}{-}{-}+}
\end{verbatim}

\textbf{h-પેરામીટર સમીકરણો:}

\begin{itemize}
\tightlist
\item
  V_{1} = h_{1}_{1}I_{1} + h_{1}_{2}V_{2}
\item
  I_{2} = h_{2}_{1}I_{1} + h_{2}_{2}V_{2}
\end{itemize}


{\def\LTcaptype{none} % do not increment counter
\vspace{-5pt}
\captionof{table}{h-પેરામીટર્સ વર્ણન}
\vspace{-10pt}
\begin{longtable}[]{@{}llll@{}}
\toprule\noalign{}
પેરામીટર & સિમ્બોલ & વર્ણન & માપન સ્થિતિ \\
\midrule\noalign{}
\endhead
\bottomrule\noalign{}
\endlastfoot
\textbf{ઇનપુટ ઇમ્પીડન્સ} & h_{1}_{1} & V_{1}/I_{1} નો ગુણોત્તર & V_{2} = 0 (આઉટપુટ શોર્ટ) \\
\textbf{રિવર્સ વોલ્ટેજ ટ્રાન્સફર} & h_{1}_{2} & V_{1}/V_{2} નો ગુણોત્તર & I_{1} = 0 (ઇનપુટ
ઓપન) \\
\textbf{ફોરવર્ડ કરંટ ટ્રાન્સફર} & h_{2}_{1} & I_{2}/I_{1} નો ગુણોત્તર & V_{2} = 0 (આઉટપુટ
શોર્ટ) \\
\textbf{આઉટપુટ એડમિટન્સ} & h_{2}_{2} & I_{2}/V_{2} નો ગુણોત્તર & I_{1} = 0 (ઇનપુટ ઓપન) \\
\end{longtable}
}

\textbf{હાઇબ્રિડ પેરામીટર્સના ફાયદા:}

\begin{itemize}
\tightlist
\item
  \textbf{સરળ માપન}: દરેક પેરામીટર માટે સરળ શરતો
\item
  \textbf{સાર્વત્રિકતા}: બધા ટ્રાન્ઝિસ્ટર કોન્ફિગરેશન માટે કામ કરે છે
\item
  \textbf{સંપૂર્ણ ચરિત્રીકરણ}: વર્તનનું સંપૂર્ણ વર્ણન કરે છે
\item
  \textbf{ગાણિતિક સરળતા}: લીનિયર સમીકરણો
\item
  \textbf{માનકીકૃત}: સ્પેસિફિકેશન માટે ઉદ્યોગ માનક
\end{itemize}

\textbf{યાદવાક્ય:} ``HAEM'' - Hybrid parameters Are Easily Measured and
mathematically simple.

\end{solutionbox}
\subsection*{પ્રશ્ન 4(અ) [3
ગુણ]}\label{uxaaauxab0uxab6uxaa8-4uxa85-3-uxa97uxaa3}

\textbf{ડાર્લિંગ્ટન જોડી અને તેની એપ્લિકેશનો સમજાવો.}

\begin{solutionbox}

\textbf{ડાર્લિંગ્ટન પેર: બે ટ્રાન્ઝિસ્ટર્સની કોન્ફિગરેશન જ્યાં પહેલાનો એમિટર બીજાના બેઝ
સાથે જોડાયેલો છે}

\textbf{આકૃતિ:}

\begin{verbatim}
           +Vcc
            |
            |
            Rc
            |
            +{-{-}{-}{-}{-}{-} Output}
            |
            |
     +{-{-}{-}{-}{-}{-}+}
     |      |
     |      C2
     |      |
Input|      |
+{-{-}{-}{-}+B1    |}
     |      |
     |  E1  |
     |  |   |
     |  +B2 |
     |      |
     |      E2
     |      |
    GND    GND
\end{verbatim}

\textbf{મુખ્ય લક્ષણો:}

\begin{itemize}
\tightlist
\item
  \textbf{ખૂબ ઊંચો કરંટ ગેઇન}: β_{1} \times β_{2} (સામાન્ય 1000-30000)
\item
  \textbf{ઊંચો ઇનપુટ ઇમ્પીડન્સ}: β_{2} \times Rin_{1}
\item
  \textbf{નિમ્ન આઉટપુટ ઇમ્પીડન્સ}: સિંગલ ટ્રાન્ઝિસ્ટર જેવું
\end{itemize}

\textbf{એપ્લિકેશન્સ:}

\begin{itemize}
\tightlist
\item
  \textbf{પાવર એમ્પલીફાયર્સ}: ઓડિયો ઇક્વિપમેન્ટ
\item
  \textbf{બફર સર્કિટ્સ}: ઊંચા ઇમ્પીડન્સથી નિમ્ન ઇમ્પીડન્સ
\item
  \textbf{મોટર ડ્રાઇવર્સ}: ઊંચા-કરંટ લોડ્સ કંટ્રોલ
\item
  \textbf{ટચ સેન્સર્સ}: ઊંચી સંવેદનશીલતા એપ્લિકેશન્સ
\end{itemize}

\textbf{યાદવાક્ય:} ``DISH'' - Darlington Integrates Stages for High
current gain.

\end{solutionbox}
\subsection*{પ્રશ્ન 4(બ) [4
ગુણ]}\label{uxaaauxab0uxab6uxaa8-4uxaac-4-uxa97uxaa3}

\textbf{જરૂરી ડાયાગ્રામ સાથે ડાયોડ ક્લેમ્પર સર્કિટનું વર્ણન કરો.}

\begin{solutionbox}

\textbf{ક્લેમ્પર સર્કિટ: વેવફોર્મના આકારને બદલ્યા વગર તેના DC લેવલને શિફ્ટ કરે છે}

\textbf{આકૃતિ:}

\begin{verbatim}
           C1
Input +{-{-}{-}{-}||{-}{-}{-}{-}+{-}{-}{-}{-}+ Output}
                 |    |
                 |    |
                 R    D
                 |    |
                 |    |
                GND  GND
\end{verbatim}

\textbf{કાર્યપ્રણાલી:}

\begin{itemize}
\tightlist
\item
  \textbf{પોઝિટિવ ક્લેમ્પર}: વેવફોર્મને નીચે શિફ્ટ કરે છે
\item
  \textbf{નેગેટિવ ક્લેમ્પર}: વેવફોર્મને ઉપર શિફ્ટ કરે છે
\item
  \textbf{કેપેસિટર}: DC બ્લોક કરે, AC પસાર કરે
\item
  \textbf{ડાયોડ}: એક હાફ-સાયકલ દરમિયાન કન્ડક્ટ કરે છે
\item
  \textbf{રેઝિસ્ટર}: કેપેસિટર માટે ડિસ્ચાર્જ પાથ
\end{itemize}

\textbf{ટાઇમ કોન્સ્ટન્ટ્સ:}

\begin{itemize}
\tightlist
\item
  \textbf{ચાર્જિંગ}: ખૂબ નાનું (ડાયોડ ફોરવર્ડ રેઝિસ્ટન્સ \times C)
\item
  \textbf{ડિસ્ચાર્જિંગ}: સિગ્નલ પીરિયડની સરખામણીમાં મોટું (R \times C)
\end{itemize}

\textbf{એપ્લિકેશન્સ:}

\begin{itemize}
\tightlist
\item
  \textbf{TV સિગ્નલ પ્રોસેસિંગ}: DC ઘટક પુનઃસ્થાપિત કરે છે
\item
  \textbf{પલ્સ સર્કિટ્સ}: લેવલ શિફ્ટિંગ
\item
  \textbf{સિગ્નલ પ્રોસેસિંગ}: DC પુનઃસ્થાપના
\end{itemize}

\textbf{યાદવાક્ય:} ``CLAMP'' - Circuit Levels Are Modified Precisely.

\end{solutionbox}
\subsection*{પ્રશ્ન 4(ક) [7
ગુણ]}\label{uxaaauxab0uxab6uxaa8-4uxa95-7-uxa97uxaa3}

\textbf{OLED નાં બાંધકામ, કાર્ય અને એપ્લિકેશન સમજાવો.}

\begin{solutionbox}

\textbf{OLED (ઓર્ગેનિક લાઇટ એમિટિંગ ડાયોડ): ઓર્ગેનિક કંપાઉન્ડ્સનો ઉપયોગ કરતું
પ્રકાશ-ઉત્સર્જક ઉપકરણ}

\textbf{બાંધકામ:}

\begin{center}
\textbf{Mermaid Diagram (Code)}
\begin{verbatim}
{Shaded}
{Highlighting}[]
graph TD
    subgraph OLED Structure
    direction LR
    C[Cathode{br /{}Metal Layer] {-}{-}{-} E[Emissive Layer{}br /{}Organic Material] {-}{-}{-} H[Hole Transport Layer{}br /{}Organic Material] {-}{-}{-} A[Anode{}br /{}Transparent ITO] {-}{-}{-} S[Substrate{}br /{}Glass or Plastic]}
    end
{Highlighting}
{Shaded}
\end{verbatim}
\end{center}

\textbf{કાર્ય સિદ્ધાંત:}

\begin{itemize}
\tightlist
\item
  \textbf{ઇલેક્ટ્રોન ઇન્જેક્શન}: કેથોડ ઇલેક્ટ્રોન્સ ઇન્જેક્ટ કરે છે
\item
  \textbf{હોલ ઇન્જેક્શન}: એનોડ હોલ્સ ઇન્જેક્ટ કરે છે
\item
  \textbf{રીકોમ્બિનેશન}: ઇલેક્ટ્રોન્સ અને હોલ્સ એમિસિવ લેયરમાં જોડાય છે
\item
  \textbf{પ્રકાશ ઉત્સર્જન}: ઊર્જા ફોટોન્સ તરીકે મુક્ત થાય છે
\item
  \textbf{રંગ નિયંત્રણ}: વિભિન્ન ઓર્ગેનિક સામગ્રી વિભિન્ન રંગો ઉત્સર્જિત કરે છે
\end{itemize}


{\def\LTcaptype{none} % do not increment counter
\vspace{-5pt}
\captionof{table}{OLED પ્રકારો}
\vspace{-10pt}
\begin{longtable}[]{@{}lll@{}}
\toprule\noalign{}
પ્રકાર & માળખું & મુખ્ય લક્ષણ \\
\midrule\noalign{}
\endhead
\bottomrule\noalign{}
\endlastfoot
\textbf{PMOLED} & પેસિવ મેટ્રિક્સ & સરળ ડિઝાઇન, ઓછી કિંમત \\
\textbf{AMOLED} & એક્ટિવ મેટ્રિક્સ & વધુ સારા રિફ્રેશ રેટ્સ, ઊંચી રેઝોલ્યુશન \\
\textbf{TOLED} & ટ્રાન્સપેરન્ટ & બંધ અથવા ચાલુ હોય ત્યારે પારદર્શક \\
\textbf{FOLED} & ફ્લેક્સિબલ & વાળી શકાય કે રોલ કરી શકાય \\
\end{longtable}
}

\textbf{એપ્લિકેશન્સ:}

\begin{itemize}
\tightlist
\item
  \textbf{ડિસ્પ્લે}: સ્માર્ટફોન્સ, ટીવી, સ્માર્ટવોચ
\item
  \textbf{લાઇટિંગ}: પાતળા, કાર્યક્ષમ લાઇટિંગ પેનલ્સ
\item
  \textbf{સાઇનેજ}: ઊંચા-કોન્ટ્રાસ્ટ ડિજિટલ સાઇન્સ
\item
  \textbf{વેરેબલ ટેક્નોલોજી}: ફ્લેક્સિબલ ડિસ્પ્લે
\end{itemize}

\textbf{યાદવાક્ય:} ``OLED'' - Organic Layers Emit Directly when
electrically stimulated.

\end{solutionbox}
\subsection*{પ્રશ્ન 4(અ) OR [3
ગુણ]}\label{uxaaauxab0uxab6uxaa8-4uxa85-or-3-uxa97uxaa3}

\textbf{LDR પર ટૂંકી નોંધ સમજાવો.}

\begin{solutionbox}

\textbf{LDR (લાઇટ ડિપેન્ડન્ટ રેઝિસ્ટર): ફોટોરેઝિસ્ટર જેનો રેઝિસ્ટન્સ વધતી પ્રકાશ
તીવ્રતા સાથે ઘટે છે}

\textbf{સિમ્બોલ અને માળખું:}

\begin{verbatim}
    ┌─────┐
    │     │
────┤ /{ ├────}
    │     │
    └─────┘
     Symbol
     
      Light
       ↓↓↓
    ┌───────┐
    │┌─────┐│
────┤│CdS  ││────
    │└─────┘│
    └───────┘
     Structure
\end{verbatim}

\textbf{મુખ્ય લક્ષણો:}

\begin{itemize}
\tightlist
\item
  \textbf{સામગ્રી}: સામાન્ય રીતે કેડમિયમ સલ્ફાઇડ (CdS)
\item
  \textbf{અંધકાર રેઝિસ્ટન્સ}: ઊંચો (MΩ રેન્જ)
\item
  \textbf{પ્રકાશ રેઝિસ્ટન્સ}: નિમ્ન (kΩ રેન્જ)
\item
  \textbf{રિસ્પોન્સ ટાઇમ}: મિલિસેકન્ડથી સેકન્ડ્સ
\end{itemize}

\textbf{એપ્લિકેશન્સ:}

\begin{itemize}
\tightlist
\item
  \textbf{લાઇટ સેન્સર્સ}: ઓટોમેટિક લાઇટિંગ કંટ્રોલ
\item
  \textbf{કેમેરા એક્સપોઝર કંટ્રોલ}: લાઇટ મીટરિંગ
\item
  \textbf{સ્ટ્રીટ લાઇટ કંટ્રોલ}: સૂર્યોદય-થી-સૂર્યાસ્ત સક્રિયતા
\item
  \textbf{અલાર્મ સિસ્ટમ્સ}: લાઇટ બીમ ડિટેક્શન
\end{itemize}

\textbf{યાદવાક્ય:} ``LORD'' - Light Oppositely Reduces the Device's
resistance.

\end{solutionbox}
\subsection*{પ્રશ્ન 4(બ) OR [4
ગુણ]}\label{uxaaauxab0uxab6uxaa8-4uxaac-or-4-uxa97uxaa3}

\textbf{જરૂરી ડાયાગ્રામ સાથે ડાયોડ ક્લિપર સર્કિટનું વર્ણન કરો.}

\begin{solutionbox}

\textbf{ક્લિપર સર્કિટ: ઇનપુટ સિગ્નલનો એવો ભાગ દૂર કરે છે (ક્લિપ) જે ચોક્કસ વોલ્ટેજ
લેવલથી વધી જાય}

\textbf{આકૃતિ (પોઝિટિવ ક્લિપર):}

\begin{verbatim}
                R     D
Input +{-{-}{-}{-}{-}+{-}{-}{-}www{-}{-}{-}+{-}{-}{-}+{-}{-}+ Output}
            |         |   |
            |         |   |
            |         +   {-}
            |         |   |
            |         |   |
            |         V   |
            |         |   |
            +{-{-}{-}{-}{-}{-}{-}{-}{-}+{-}{-}{-}+}
\end{verbatim}

\textbf{ક્લિપર્સના પ્રકારો:}

\begin{itemize}
\tightlist
\item
  \textbf{પોઝિટિવ ક્લિપર}: પોઝિટિવ પીક્સ દૂર કરે છે
\item
  \textbf{નેગેટિવ ક્લિપર}: નેગેટિવ પીક્સ દૂર કરે છે
\item
  \textbf{બાયસ્ડ ક્લિપર}: નોન-ઝીરો રેફરન્સ પર ક્લિપ કરે છે
\item
  \textbf{કોમ્બિનેશન ક્લિપર}: બંને પીક્સ ક્લિપ કરે છે
\end{itemize}

\textbf{કાર્યપ્રણાલી:}

\begin{itemize}
\tightlist
\item
  \textbf{ડાયોડ ON}: જ્યારે સિગ્નલ રેફરન્સ વોલ્ટેજથી વધે છે
\item
  \textbf{ડાયોડ OFF}: જ્યારે સિગ્નલ રેફરન્સ વોલ્ટેજથી નીચે છે
\item
  \textbf{ક્લિપિંગ લેવલ}: રેફરન્સ વોલ્ટેજ દ્વારા નિર્ધારિત
\end{itemize}

\textbf{એપ્લિકેશન્સ:}

\begin{itemize}
\tightlist
\item
  \textbf{વેવ શેપિંગ}: સ્ક્વેર વેવ્સ બનાવવા
\item
  \textbf{સર્કિટ પ્રોટેક્શન}: વોલ્ટેજ લિમિટિંગ
\item
  \textbf{નોઇઝ રિમૂવલ}: ઇમ્પલ્સ નોઇઝ મર્યાદિત કરવું
\end{itemize}

\textbf{યાદવાક્ય:} ``CLIP'' - Circuit Limits Input Peaks using diodes.

\end{solutionbox}
\subsection*{પ્રશ્ન 4(ક) OR [7
ગુણ]}\label{uxaaauxab0uxab6uxaa8-4uxa95-or-7-uxa97uxaa3}

\textbf{હાફ વેવ અને ફુલ વેવ વોલ્ટેજ ડબલર સમજાવો.}

\begin{solutionbox}

\textbf{વોલ્ટેજ ડબલર: સર્કિટ જે DC આઉટપુટ વોલ્ટેજ આશરે ઇનપુટ વોલ્ટેજના પીક કરતાં બમણું
ઉત્પન્ન કરે છે}

\textbf{હાફ-વેવ વોલ્ટેજ ડબલર:}

\begin{verbatim}
               D1
            +{-{-}{-}{-}|{-}{-}{-}+}
            |         |
            |         |
AC Input    |         | C1    + 2Vpeak
    +{-{-}{-}{-}{-}{-}{-}+         +{-}{-}{-}{-}{-}{-}{-}+  Output}
    |       |         |       |
    |       +{-{-}{-}{-}|{-}{-}{-}+       |}
    |          D2    |        |
    |               C2        |
    |                |        |
    +{-{-}{-}{-}{-}{-}{-}{-}{-}{-}{-}{-}{-}{-}{-}{-}+{-}{-}{-}{-}{-}{-}{-}{-}+}
                     |
                    GND
\end{verbatim}

\textbf{ફુલ-વેવ વોલ્ટેજ ડબલર:}

\begin{verbatim}
               D1
            +{-{-}{-}{-}|{-}{-}{-}+}
            |         |
            |         |
AC Input    |         | C1    + 2Vpeak
    +{-{-}{-}{-}{-}{-}{-}+         +{-}{-}{-}{-}{-}{-}{-}+  Output}
    |       |         |       |
    |       |         |       |
    |       |    C2   |       |
    |       |    |    |       |
    |       +{-{-}{-}{-}|{-}{-}{-}+       |}
    |          D2             |
    |                |        |
    +{-{-}{-}{-}{-}{-}{-}{-}{-}{-}{-}{-}{-}{-}{-}{-}+{-}{-}{-}{-}{-}{-}{-}{-}+}
                     |
                    GND
\end{verbatim}


{\def\LTcaptype{none} % do not increment counter
\vspace{-5pt}
\captionof{table}{સરખામણી}
\vspace{-10pt}
\begin{longtable}[]{@{}lll@{}}
\toprule\noalign{}
લક્ષણ & હાફ-વેવ & ફુલ-વેવ \\
\midrule\noalign{}
\endhead
\bottomrule\noalign{}
\endlastfoot
\textbf{રિપલ} & ઊંચો & નિમ્ન \\
\textbf{કાર્યક્ષમતા} & નિમ્ન & ઊંચી \\
\textbf{રિસ્પોન્સ ટાઇમ} & ધીમો & ઝડપી \\
\textbf{ઘટકો} & 2 ડાયોડ, 2 કેપેસિટર્સ & 2 ડાયોડ, 2 કેપેસિટર્સ \\
\textbf{રેગ્યુલેશન} & ખરાબ & વધુ સારું \\
\end{longtable}
}

\textbf{કાર્યપ્રણાલી:}

\begin{itemize}
\tightlist
\item
  \textbf{હાફ-વેવ}: દરેક કેપેસિટરને વૈકલ્પિક હાફ-સાયકલ પર ચાર્જ કરે છે
\item
  \textbf{ફુલ-વેવ}: દરેક સાયકલ પર બંને કેપેસિટર્સ ચાર્જ કરે છે
\item
  \textbf{આઉટપુટ}: બંને કેપેસિટર્સ પરના વોલ્ટેજનો સરવાળો
\end{itemize}

\textbf{એપ્લિકેશન્સ:}

\begin{itemize}
\tightlist
\item
  \textbf{પાવર સપ્લાય}: ઓછા-કરંટ ઊંચા-વોલ્ટેજ જરૂરિયાતો
\item
  \textbf{કેસ્કેડ કનેક્શન}: વોલ્ટેજ મલ્ટિપ્લિકેશન માટે
\item
  \textbf{ઇલેક્ટ્રોનિક ફ્લેશ}: કેમેરા ઇક્વિપમેન્ટ
\item
  \textbf{CRT ડિસ્પ્લે}: ઊંચા વોલ્ટેજ જનરેશન
\end{itemize}

\textbf{યાદવાક્ય:} ``DOUBLE'' - Diodes Organize Unidirectional Boost,
Lifting Electricity to twice input.

\end{solutionbox}
\subsection*{પ્રશ્ન 5(અ) [3
ગુણ]}\label{uxaaauxab0uxab6uxaa8-5uxa85-3-uxa97uxaa3}

\textbf{IC નો ઉપયોગ કરીને +5 v પાવર સપ્લાય માટે સર્કિટ ડાયાગ્રામ દોરો.}

\begin{solutionbox}

\textbf{7805 વોલ્ટેજ રેગ્યુલેટર IC વાપરીને +5V પાવર સપ્લાય:}

\begin{verbatim}
   AC Input    Bridge     7805
    +{-{-}+       Rect.    +{-}{-}{-}{-}{-}+}
       |     +{-{-}{-}{-}{-}{-}+   |     |}
       +{-{-}{-}{-}{-}+      +{-}{-}{-}+ IN  |}
       |     |      |   |     |   +5V
       |     +{-{-}{-}{-}{-}{-}+   |     +{-}{-}{-}+{-}{-}{-} Output}
       +{-{-}+{-}{-}+          | OUT |   |}
          |             |     |   |
         GND            +{-{-}+{-}{-}+   |}
                           |      |
                          GND    GND
                           
                  C1 |     C2 |
                 === |    === |
                  |  |     |  |
                 GND       GND
\end{verbatim}

\textbf{મુખ્ય ઘટકો:}

\begin{itemize}
\tightlist
\item
  \textbf{7805 IC}: થ્રી-ટર્મિનલ ફિક્સ્ડ વોલ્ટેજ રેગ્યુલેટર
\item
  \textbf{ઇનપુટ કેપેસિટર (C1)}: ઇનપુટ રિપલ ફિલ્ટર કરે છે
\item
  \textbf{આઉટપુટ કેપેસિટર (C2)}: ટ્રાન્ઝિયન્ટ રિસ્પોન્સ સુધારે છે
\item
  \textbf{બ્રિજ રેક્ટિફાયર}: AC ને પલ્સેટિંગ DC માં રૂપાંતર કરે છે
\end{itemize}

\textbf{યાદવાક્ય:} ``FIVE'' - Fixed IC Voltage Efficiently provided.

\end{solutionbox}
\subsection*{પ્રશ્ન 5(બ) [4
ગુણ]}\label{uxaaauxab0uxab6uxaa8-5uxaac-4-uxa97uxaa3}

\textbf{પાવર સપ્લાયના સંદર્ભમાં લોડ રેગ્યુલેશન અને લાઇન રેગ્યુલેશનની ચર્ચા કરો.}

\begin{solutionbox}

\textbf{લોડ રેગ્યુલેશન: લોડ કરંટ ફેરફારો હોવા છતાં પાવર સપ્લાયની સ્થિર આઉટપુટ
વોલ્ટેજ જાળવવાની ક્ષમતા}

\textbf{લાઇન રેગ્યુલેશન: ઇનપુટ વોલ્ટેજ ફેરફારો હોવા છતાં પાવર સપ્લાયની સ્થિર આઉટપુટ
વોલ્ટેજ જાળવવાની ક્ષમતા}

\textbf{આકૃતિ:}

\begin{center}
\textbf{Mermaid Diagram (Code)}
\begin{verbatim}
{Shaded}
{Highlighting}[]
graph LR
    A[Power Supply] {-{-}{} B["Line Regulation{}br /{}(Input Voltage Changes)"]}
    A {-{-}{} C["Load Regulation{}br /{}(Output Current Changes)"]}
    B {-{-}{} D["Constant Output{}br /{}Voltage"]}
    C {-{-}{} D}
{Highlighting}
{Shaded}
\end{verbatim}
\end{center}

\textbf{સૂત્રો:}

\begin{itemize}
\tightlist
\item
  \textbf{લોડ રેગ્યુલેશન}: (V_{1} - V_{2})/V_{2} \times 100\%

  \begin{itemize}
  \tightlist
  \item
    V_{1} = નો-લોડ વોલ્ટેજ
  \item
    V_{2} = ફુલ-લોડ વોલ્ટેજ
  \end{itemize}
\item
  \textbf{લાઇન રેગ્યુલેશન}: (V_{1} - V_{2})/V_{2} \times 100\%

  \begin{itemize}
  \tightlist
  \item
    V_{1} = મહત્તમ ઇનપુટ પર આઉટપુટ વોલ્ટેજ
  \item
    V_{2} = લઘુત્તમ ઇનપુટ પર આઉટપુટ વોલ્ટેજ
  \end{itemize}
\end{itemize}

\textbf{મુખ્ય મુદ્દાઓ:}

\begin{itemize}
\tightlist
\item
  \textbf{નિમ્ન ટકાવારી}: વધુ સારી રેગ્યુલેશન
\item
  \textbf{ફીડબેક સર્કિટ}: રેગ્યુલેશન પરફોર્મન્સ સુધારે છે
\item
  \textbf{IC રેગ્યુલેટર્સ}: સામાન્ય રીતે સારી રેગ્યુલેશન ઓફર કરે છે (0.01-0.1\%)
\end{itemize}

\textbf{યાદવાક્ય:} ``LINE LOAD'' - Line Is Normal-input Efficiency, LOAD
is Output Adjustment Defense.

\end{solutionbox}
\subsection*{પ્રશ્ન 5(ક) [7
ગુણ]}\label{uxaaauxab0uxab6uxaa8-5uxa95-7-uxa97uxaa3}

\textbf{સર્કિટ ડાયાગ્રામ સાથે LM317 નો ઉપયોગ કરીને એડજસ્ટેબલ વોલ્ટેજ રેગ્યુલેટર
સમજાવો.}

\begin{solutionbox}

\textbf{LM317 એડજસ્ટેબલ વોલ્ટેજ રેગ્યુલેટર: થ્રી-ટર્મિનલ ડિવાઇસ જે ચલ રેગ્યુલેટેડ આઉટપુટ
વોલ્ટેજ પ્રદાન કરે છે}

\textbf{સર્કિટ ડાયાગ્રામ:}

\begin{verbatim}
               LM317
              +{-{-}{-}{-}{-}{-}+}
              |      |
   Input {-{-}{-}{-}{-}+ IN   |}
              |      |    R1
              | ADJ  +{-{-}{-}{-}www{-}{-}{-}{-}+}
              |      |           |
              | OUT  |           |
              +{-{-}+{-}{-}{-}+           |}
                 |               |
                 +{-{-}{-}{-}{-}{-}{-}www{-}{-}{-}{-}{-}+}
                 |       R2      |
                 |               |
                 +      C1       +{-{-}{-}{-}{-}{-} Output}
                 |      ===      |
                 |       |       |
                GND     GND     GND
\end{verbatim}

\textbf{કાર્યપ્રણાલી:}

\begin{itemize}
\tightlist
\item
  \textbf{રેફરન્સ વોલ્ટેજ}: OUT અને ADJ ટર્મિનલ્સ વચ્ચે 1.25V
\item
  \textbf{આઉટપુટ વોલ્ટેજ}: VOUT = 1.25V \times (1 + R2/R1)
\item
  \textbf{એડજસ્ટમેન્ટ રેન્જ}: 1.25V થી 37V
\item
  \textbf{મહત્તમ કરંટ}: 1.5A (યોગ્ય હીટ સિંક સાથે)
\end{itemize}

\textbf{ઘટક પસંદગી:}

\begin{itemize}
\tightlist
\item
  \textbf{R1}: સામાન્ય રીતે 240Ω
\item
  \textbf{R2}: આઉટપુટ એડજસ્ટ કરવા માટે વેરિયેબલ રેઝિસ્ટર
\item
  \textbf{C1}: સ્થિરતા માટે આઉટપુટ કેપેસિટર (1-10μF)
\end{itemize}

\textbf{મુખ્ય લક્ષણો:}

\begin{itemize}
\tightlist
\item
  \textbf{કરંટ લિમિટિંગ}: બિલ્ટ-ઇન પ્રોટેક્શન
\item
  \textbf{થર્મલ શટડાઉન}: અતિશય ગરમી સામે રક્ષણ
\item
  \textbf{સેફ એરિયા પ્રોટેક્શન}: આઉટપુટ ટ્રાન્ઝિસ્ટર્સ માટે
\item
  \textbf{રિપલ રિજેક્શન}: સામાન્ય રીતે 80dB
\end{itemize}

\textbf{યાદવાક્ય:} ``VARY'' - Voltage Adjustable Regulator Yields custom
outputs.

\end{solutionbox}
\subsection*{પ્રશ્ન 5(અ) OR [3
ગુણ]}\label{uxaaauxab0uxab6uxaa8-5uxa85-or-3-uxa97uxaa3}

\textbf{IC નો ઉપયોગ કરીને -15 v પાવર સપ્લાય માટે સર્કિટ ડાયાગ્રામ દોરો.}

\begin{solutionbox}

\textbf{7915 વોલ્ટેજ રેગ્યુલેટર IC વાપરીને -15V પાવર સપ્લાય:}

\begin{verbatim}
   AC Input    Bridge     7915
    +{-{-}+       Rect.    +{-}{-}{-}{-}{-}+}
       |     +{-{-}{-}{-}{-}{-}+   |     |}
       +{-{-}{-}{-}{-}+      +{-}{-}{-}+ IN  |}
       |     |      |   |     |   {-15V}
       |     +{-{-}{-}{-}{-}{-}+   |     +{-}{-}{-}+{-}{-}{-} Output}
       +{-{-}+{-}{-}+          | OUT |   |}
          |             |     |   |
         GND            +{-{-}+{-}{-}+   |}
                           |      |
                          GND    GND
                           
                  C1 |     C2 |
                 === |    === |
                  |  |     |  |
                 GND       GND
\end{verbatim}

\textbf{મુખ્ય ઘટકો:}

\begin{itemize}
\tightlist
\item
  \textbf{7915 IC}: થ્રી-ટર્મિનલ નેગેટિવ વોલ્ટેજ રેગ્યુલેટર
\item
  \textbf{ઇનપુટ કેપેસિટર (C1)}: ઇનપુટ રિપલ ફિલ્ટર કરે છે
\item
  \textbf{આઉટપુટ કેપેસિટર (C2)}: ટ્રાન્ઝિયન્ટ રિસ્પોન્સ સુધારે છે
\item
  \textbf{બ્રિજ રેક્ટિફાયર}: AC ને પલ્સેટિંગ DC માં રૂપાંતર કરે છે
\end{itemize}

\textbf{યાદવાક્ય:} ``NINE'' - Negative IC Needs Efficient filtering.

\end{solutionbox}
\subsection*{પ્રશ્ન 5(બ) OR [4
ગુણ]}\label{uxaaauxab0uxab6uxaa8-5uxaac-or-4-uxa97uxaa3}

\textbf{યુપીએસની કામગીરી સમજાવો.}

\begin{solutionbox}

\textbf{UPS (અનઇન્ટરપ્ટિબલ પાવર સપ્લાય): ડિવાઇસ જે મુખ્ય પાવર ફેઇલ થાય ત્યારે
ઇમરજન્સી પાવર પ્રદાન કરે છે}

\textbf{બ્લોક ડાયાગ્રામ:}

\begin{center}
\textbf{Mermaid Diagram (Code)}
\begin{verbatim}
{Shaded}
{Highlighting}[]
graph LR
    I[AC Input] {-{-}{} R[Rectifier]}
    R {-{-}{} C[Charger]}
    C {-{-}{} B[Battery]}
    B {-{-}{} Inv[Inverter]}
    I {-{-} "Normal Operation" {-}{-}{} S[Switch]}
    S {-{-}{} O[Output]}
    Inv {-{-} "During Power Failure" {-}{-}{} S}
{Highlighting}
{Shaded}
\end{verbatim}
\end{center}

\textbf{UPS ના પ્રકારો:}

\begin{itemize}
\tightlist
\item
  \textbf{ઓફલાઇન/સ્ટેન્ડબાય UPS}: પાવર ફેઇલ થાય ત્યારે બેટરી પર સ્વિચ કરે છે
\item
  \textbf{લાઇન-ઇન્ટરએક્ટિવ UPS}: વોલ્ટેજ રેગ્યુલેશન ધરાવે છે
\item
  \textbf{ઓનલાઇન/ડબલ-કન્વર્ઝન UPS}: હંમેશા બેટરી પાવર વાપરે છે
\end{itemize}

\textbf{મુખ્ય ઘટકો:}

\begin{itemize}
\tightlist
\item
  \textbf{રેક્ટિફાયર}: AC ને DC માં રૂપાંતર કરે છે
\item
  \textbf{બેટરી}: ઊર્જા સંગ્રહ કરે છે
\item
  \textbf{ઇન્વર્ટર}: DC ને પાછું AC માં રૂપાંતર કરે છે
\item
  \textbf{કંટ્રોલ સર્કિટ}: પાવર મોનિટર કરે છે અને સ્ત્રોત સ્વિચ કરે છે
\end{itemize}

\textbf{એપ્લિકેશન્સ:}

\begin{itemize}
\tightlist
\item
  \textbf{કમ્પ્યુટર્સ}: ડેટા નુકસાન અટકાવે છે
\item
  \textbf{મેડિકલ ઇક્વિપમેન્ટ}: ક્રિટિકલ ઓપરેશન્સ
\item
  \textbf{ઇન્ડસ્ટ્રિયલ કંટ્રોલ્સ}: ખર્ચાળ અવરોધ અટકાવે છે
\item
  \textbf{ટેલિકોમ્યુનિકેશન્સ}: કનેક્શન્સ જાળવે છે
\end{itemize}

\textbf{યાદવાક્ય:} ``UPBEAT'' - Uninterruptible Power Backup Ensures
Available Technology.

\end{solutionbox}
\subsection*{પ્રશ્ન 5(ક) OR [7
ગુણ]}\label{uxaaauxab0uxab6uxaa8-5uxa95-or-7-uxa97uxaa3}

\textbf{SMPS બ્લોક ડાયાગ્રામ તેના ફાયદા અને ગેરફાયદા સાથે દોરો અને સમજાવો.}

\begin{solutionbox}

\textbf{SMPS (સ્વિચ મોડ પાવર સપ્લાય): કાર્યક્ષમતા માટે સ્વિચિંગ રેગ્યુલેશનનો ઉપયોગ
કરતો પાવર સપ્લાય}

\textbf{બ્લોક ડાયાગ્રામ:}

\begin{center}
\textbf{Mermaid Diagram (Code)}
\begin{verbatim}
{Shaded}
{Highlighting}[]
graph LR
    AC[AC Input] {-{-}{} EMI[EMI Filter]}
    EMI {-{-}{} R[Rectifier \& Filter]}
    R {-{-}{} C[Chopper/Switching Circuit]}
    C {-{-}{} T[High Frequency Transformer]}
    T {-{-}{} O[Output Rectifier \& Filter]}
    O {-{-}{} Out[DC Output]}
    FB[Feedback \& Control] {-{-}{} C}
    O {-{-}{} FB}
{Highlighting}
{Shaded}
\end{verbatim}
\end{center}

\textbf{કાર્યપ્રણાલી:}

\begin{itemize}
\tightlist
\item
  \textbf{EMI ફિલ્ટર}: ઇલેક્ટ્રોમેગ્નેટિક ઇન્ટરફેરન્સ ઘટાડે છે
\item
  \textbf{રેક્ટિફાયર}: AC ને અનરેગ્યુલેટેડ DC માં રૂપાંતર કરે છે
\item
  \textbf{સ્વિચિંગ સર્કિટ}: DC ને ઊંચી ફ્રીક્વન્સી પર ચોપ કરે છે (20-100 kHz)
\item
  \textbf{ટ્રાન્સફોર્મર}: આઇસોલેશન અને વોલ્ટેજ રૂપાંતર પ્રદાન કરે છે
\item
  \textbf{આઉટપુટ સ્ટેજ}: ક્લીન DC માટે રેક્ટિફાય અને ફિલ્ટર કરે છે
\item
  \textbf{ફીડબેક લૂપ}: રેગ્યુલેશન માટે સ્વિચિંગ નિયંત્રિત કરે છે
\end{itemize}

\textbf{ફાયદા:}

\begin{itemize}
\tightlist
\item
  \textbf{ઊંચી કાર્યક્ષમતા}: 70-90\% (vs.~30-60\% લિનિયર સપ્લાય)
\item
  \textbf{નાનું કદ}: ઊંચી ઓપરેટિંગ ફ્રીક્વન્સીને કારણે નાના ઘટકો
\item
  \textbf{હલકું વજન}: નાના ટ્રાન્સફોર્મર અને હીટ સિંક્સ
\item
  \textbf{વિશાળ ઇનપુટ રેન્જ}: વિવિધ ઇનપુટ વોલ્ટેજ પર કામ કરી શકે છે
\item
  \textbf{ઓછી ગરમી ઉત્પાદન}: ઓછી ઊર્જા ગરમી તરીકે બરબાદ થાય છે
\end{itemize}

\textbf{ગેરફાયદા:}

\begin{itemize}
\tightlist
\item
  \textbf{જટિલ ડિઝાઇન}: વધુ સુધારેલ સર્કિટરી
\item
  \textbf{EMI જનરેશન}: સ્વિચિંગ ઇન્ટરફેરન્સ પેદા કરે છે
\item
  \textbf{ઊંચી કિંમત}: લો-પાવર એપ્લિકેશન્સ માટે
\item
  \textbf{નોઇઝ}: લિનિયર સપ્લાય કરતાં ઊંચો આઉટપુટ નોઇઝ
\item
  \textbf{ધીમો રિસ્પોન્સ}: અચાનક લોડ ફેરફારો સામે
\end{itemize}

\textbf{એપ્લિકેશન્સ:}

\begin{itemize}
\tightlist
\item
  \textbf{કમ્પ્યુટર્સ}: ડેસ્કટોપ અને લેપટોપ પાવર સપ્લાય
\item
  \textbf{ટીવી અને મોનિટર્સ}: કોમ્પેક્ટ પાવર સ્ત્રોત
\item
  \textbf{મોબાઇલ ચાર્જર્સ}: નાના, કાર્યક્ષમ એડેપ્ટર્સ
\item
  \textbf{ઇન્ડસ્ટ્રિયલ પાવર}: ઊંચી-કાર્યક્ષમતા જરૂરિયાતો
\end{itemize}

\textbf{યાદવાક્ય:} ``SWITCH'' - Smaller Weight, Improved Thermal
efficiency, Complex Hardware.

\end{solutionbox}

\end{document}
