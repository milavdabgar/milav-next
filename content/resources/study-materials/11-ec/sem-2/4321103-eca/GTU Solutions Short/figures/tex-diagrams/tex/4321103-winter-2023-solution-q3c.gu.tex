% Tags: 4321103, ડાર્લિંગટન, તેની, ઉપયોગો, જવાબ, વિશેષતાઓ
\begin{circuitikz}
    \draw (0,0) node[npn] (Q1) {Q1};
    \draw (2,0) node[npn] (Q2) {Q2};
    \draw (Q1.C) -- (Q2.C) -- ++(0,0.5) node[vcc]{C};
    \draw (Q1.E) -- (Q2.B);
    \draw (Q1.B) -- ++(-0.5,0) node[left]{B};
    \draw (Q2.E) -- ++(0,-0.5) node[below]{E};
\end{circuitikz}
