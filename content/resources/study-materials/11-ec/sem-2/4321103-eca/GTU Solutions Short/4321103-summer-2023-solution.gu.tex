\documentclass{article}

% content/resources/templates/preamble.tex
\usepackage[margin=0.6in]{geometry}
\author{Milav Dabgar}
\usepackage{amsmath,amssymb,amsthm}
\usepackage{booktabs}
\usepackage{multirow}
\usepackage{xcolor}
\usepackage{tcolorbox}
\tcbuselibrary{breakable,skins}
\usepackage[colorlinks=true,linkcolor=blue]{hyperref}
\usepackage{titlesec}
\usepackage{enumitem}
\usepackage{tikz}
\usepackage{pgfplots}
\usepackage{circuitikz}
\usepackage[version=4]{mhchem}
\usepackage{longtable}
\usepackage{array}
\usepackage{float}
\usepackage{caption}
\usepackage{listings}

\lstset{
  basicstyle=\small\ttfamily,
  breaklines=true,
  breakatwhitespace=false,
  postbreak=\mbox{\textcolor{red}{$\hookrightarrow$}\space},
  float=false,
  numbers=left,
  numberstyle=\tiny\color{gray},
  numbersep=10pt,
  xleftmargin=2em,
  keywordstyle=\color{blue},
  commentstyle=\color{green!60!black},
  stringstyle=\color{purple},
  backgroundcolor=\color{gray!5},
  showstringspaces=false,
  tabsize=2,
  captionpos=b,
  keepspaces=true,
  columns=flexible
}

\pgfplotsset{compat=1.18}
\usetikzlibrary{shapes,arrows,positioning,calc,patterns,decorations.pathmorphing,decorations.markings,arrows.meta}

% Color scheme
\definecolor{headcolor}{RGB}{0,102,204}
\definecolor{keycolor}{RGB}{220,20,60}
\definecolor{solutioncolor}{RGB}{34,139,34}
\definecolor{mnemoniccolor}{RGB}{148,0,211}
\definecolor{codecolor}{RGB}{0,0,100}

% Spacing
\setlength{\parskip}{3pt}
\setlist[itemize]{nosep}
\setlist[enumerate]{nosep}

% Title formatting
\titleformat{\section}{\Large\bfseries\color{headcolor}}{\thesection}{1em}{}
\titleformat{\subsection}{\large\bfseries\color{headcolor}}{\thesubsection}{1em}{}

% Pandoc tightlist compatibility
\providecommand{\tightlist}{%
  \setlength{\itemsep}{0pt}\setlength{\parskip}{0pt}}

% Pandoc longtable compatibility
\newcounter{none}
\def\thenone{}


% content/resources/templates/gujarati-boxes.tex
\usepackage{fontspec}
\usepackage{polyglossia}

% Set Gujarati as main language (document is primarily in Gujarati)
% Note: gloss-gujarati.ldf doesn't exist in polyglossia, but it will use hyphenation patterns
\setdefaultlanguage{gujarati}
\setotherlanguage{english}

% Configure Gujarati font properly
% Use Language=Default to prevent polyglossia from trying to add language-specific features
% that don't exist for Gujarati, which causes "empty feature" warnings
\newfontfamily\gujaratifont[Script=Gujarati,AutoFakeBold=2.5,AutoFakeSlant=0.3]{Noto Sans Gujarati}
\setmainfont[Script=Gujarati,AutoFakeBold=2.5,AutoFakeSlant=0.3]{Noto Sans Gujarati}
% Use Noto Sans Gujarati for monospace to support Gujarati in text
\setmonofont[Scale=0.9]{Noto Sans Gujarati}

% Configure English to use the same font
\newfontfamily\englishfont[Script=Gujarati,AutoFakeBold=2.5,AutoFakeSlant=0.3]{Noto Sans Gujarati}

% Translations for polyglossia
\gappto\captionsgujarati{
  \renewcommand{\tablename}{કોષ્ટક}
  \renewcommand{\figurename}{આકૃતિ}
}

% Helper for TikZ nodes to ensure Gujarati font
\newcommand{\gu}[1]{{\gujaratifont #1}}

% Custom environments
\newtcolorbox{solutionbox}{
    breakable,
    enhanced,
    colback=solutioncolor!5!white,
    colframe=solutioncolor!75!black,
    fonttitle=\bfseries,
    title=જવાબ
}

\newtcolorbox{solutionboxnobreak}{
 colback=solutioncolor!5!white,
 colframe=solutioncolor!75!black,
 fonttitle=\bfseries,
 title=જવાબ
}

\newtcolorbox{keyformula}{
 breakable,
 enhanced,
 colback=keycolor!5!white,
 colframe=keycolor!75!black,
 fonttitle=\bfseries,
 title=રાસાયણિક સમીકરણ/સૂત્ર
}

\newtcolorbox{mnemonicbox}{
 breakable,
 enhanced,
 colback=mnemoniccolor!5!white,
 colframe=mnemoniccolor!75!black,
 fonttitle=\bfseries,
 title=મેમરી ટ્રીક
}


% Custom commands for GTU solutions
% This file defines semantic commands for consistent formatting

% Question command with automatic formatting
\newcommand{\question}[2]{%
  \section*{Question #1}%
  \textbf{#2}%
}

% OR question variant
\newcommand{\questionor}[2]{%
  \section*{Question #1 OR}%
  \textbf{#2}%
}

% Proper table environment with caption
\newenvironment{answertable}[1]{%
  \begin{table}[htbp]
  \centering
  \caption{#1}
}{%
  \end{table}
}

% Proper figure environment for diagrams
\newenvironment{answerdiagram}[1]{%
  \begin{figure}[htbp]
  \centering
  \caption{#1}
}{%
  \end{figure}
}

% Semantic markup for key terms
\newcommand{\keyword}[1]{\textbf{#1}}
\newcommand{\code}[1]{\texttt{#1}}
\newcommand{\classname}[1]{\texttt{#1}}
\newcommand{\methodname}[1]{\texttt{#1}}

% Proper quotation marks
\newcommand{\mnemonic}[1]{``#1''}


\title{ઇલેક્ટ્રોનિક સર્કિટ્સ એન્ડ એપ્લિકેશન્સ (4321103) - ઉનાળુ 2023 સોલ્યુશન}
\date{August 09, 2023}

\begin{document}
\maketitle

\questionmarks{1}{a}{3}
\textbf{થર્મલ રનઅવે વિગતવાર સમજાવો.}

\begin{solutionbox}
\textbf{થર્મલ રનઅવે:}
થર્મલ રનઅવે એ BJT ટ્રાન્ઝિસ્ટરમાં થતી વિનાશક પ્રક્રિયા છે જેમાં તાપમાનમાં વધારો સ્વ-પુનરાવર્તિત ચક્ર બનાવે છે જે ઉપકરણને નુકસાન પહોંચાડે છે.

\begin{center}
\includefigure{figures/tex-diagrams/pdf/4321103-summer-2023-solution-q1a.gu.pdf}
\end{center}

\begin{itemize}
    \item \textbf{ગરમી ઉત્પાદન}: સામાન્ય કાર્ય દરમિયાન તાપમાન વધે છે.
    \item \textbf{લીકેજ કરંટ}: તાપમાન વધવાથી કલેક્ટર કરંટ Ic વધે છે.
    \item \textbf{પાવર વ્યય}: વધુ પાવર = તાપમાન વધુ વધે છે.
    \item \textbf{વિનાશક ચક્ર}: ટ્રાન્ઝિસ્ટર નાશ પામે ત્યાં સુધી સતત ચક્ર ચાલે છે.
\end{itemize}

\mnemonicbox{વધુ તાપમાન, વધુ કરંટ}
\end{solutionbox}

\questionmarks{1}{b}{4}
\textbf{સરળ બ્લોક ડાયાગ્રામ સાથે એમ્પ્લીફાયર વ્યાખ્યાયિત કરો એમ્પ્લીફાયર પરિમાણો લખો.}

\begin{solutionbox}
\textbf{એમ્પ્લીફાયર:}
એમ્પ્લીફાયર એક ઇલેક્ટ્રોનિક ઉપકરણ છે જે ઇનપુટ સિગ્નલનો પાવર, વોલ્ટેજ અથવા કરંટ વધારે છે.

\begin{center}
\includefigure{figures/tex-diagrams/pdf/4321103-summer-2023-solution-q1b.gu.pdf}
\end{center}

\begin{center}
\captionof{table}{એમ્પ્લીફાયર પરિમાણ}
\begin{tabular}{|l|l|}
\hline
\textbf{એમ્પ્લીફાયર પરિમાણ} & \textbf{વર્ણન} \\ \hline
વોલ્ટેજ ગેઇન (Av) & આઉટપુટ વોલ્ટેજનો ઇનપુટ વોલ્ટેજ સાથેનો ગુણોત્તર \\ \hline
કરંટ ગેઇન (Ai) & આઉટપુટ કરંટનો ઇનપુટ કરંટ સાથેનો ગુણોત્તર \\ \hline
પાવર ગેઇન (Ap) & વોલ્ટેજ ગેઇન અને કરંટ ગેઇનનો ગુણાકાર \\ \hline
બેન્ડવિડ્થ & એમ્પ્લીફાયર હેન્ડલ કરી શકે તેવી ફ્રીક્વન્સીની રેન્જ \\ \hline
ઇનપુટ ઇમ્પીડન્સ & ઇનપુટ સ્ત્રોત દ્વારા જોવામાં આવતો અવરોધ \\ \hline
આઉટપુટ ઇમ્પીડન્સ & એમ્પ્લીફાયરનો આંતરિક અવરોધ \\ \hline
\end{tabular}
\end{center}

\mnemonicbox{VIPS-BIO (Voltage, Input impedance, Power, Supply, Bandwidth, Impedance Output)}
\end{solutionbox}

\questionmarks{1}{c}{7}
\textbf{ટ્રાન્ઝિસ્ટરમાં બાયસિંગ વ્યાખ્યાયિત કરો? બાયસિંગ પદ્ધતિઓના પ્રકારો લખો. વોલ્ટેજ વિભાજક બાયસિંગ પદ્ધતિને વિગતોમાં સમજાવો.}

\begin{solutionbox}
\textbf{બાયસિંગ:}
બાયસિંગ એ ટ્રાન્ઝિસ્ટર માટે DC વોલ્ટેજ આપીને સ્થિર ઓપરેટિંગ પોઈન્ટ (Q-પોઈન્ટ) સ્થાપિત કરવાની પ્રક્રિયા છે.

\textbf{બાયસિંગ પદ્ધતિઓના પ્રકારો:}
\begin{itemize}
    \item ફિક્સ્ડ બાયસ (સરળ, ઓછી સ્થિરતા)
    \item કલેક્ટર ફીડબેક (સ્વ-સમાયોજિત, વધુ સારી સ્થિરતા)
    \item વોલ્ટેજ વિભાજક (શ્રેષ્ઠ સ્થિરતા, વ્યાપકપણે વપરાતી)
    \item એમિટર બાયસ (સારી સ્થિરતા, નેગેટિવ ફીડબેક)
\end{itemize}

\textbf{વોલ્ટેજ વિભાજક બાયસિંગ:}

\begin{center}
\includefigure{figures/tex-diagrams/pdf/4321103-summer-2023-solution-q1c.gu.pdf}
\end{center}

\begin{itemize}
    \item \textbf{R1 \& R2}: બેઝને સ્થિર વોલ્ટેજ આપવા માટે વોલ્ટેજ વિભાજક બનાવે છે.
    \item \textbf{RE}: નેગેટિવ ફીડબેક દ્વારા સ્થિરીકરણ પ્રદાન કરે છે.
    \item \textbf{RC}: કલેક્ટર કરંટ અને વોલ્ટેજ ગેઇન નક્કી કરે છે.
    \item \textbf{સ્થિરતા}: તાપમાન ફેરફારો સામે શ્રેષ્ઠ સ્થિરતા.
\end{itemize}

\mnemonicbox{વિભાજીત વોલ્ટેજથી ટ્રાન્ઝિસ્ટર સારું વહન કરે}
\end{solutionbox}

\questionmarks{1}{c}{7}
\textbf{હીટ સિંક સમજાવો.}

\begin{solutionbox}
\textbf{હીટ સિંક:}
હીટ સિંક એ પેસિવ હીટ એક્સચેન્જર છે જે ઇલેક્ટ્રોનિક ઉપકરણોમાંથી ગરમીને આસપાસની હવામાં ટ્રાન્સફર કરે છે.

\begin{center}
\includefigure{figures/tex-diagrams/pdf/4321103-summer-2023-solution-q1c-2.gu.pdf}
\end{center}

\begin{center}
\captionof{table}{હીટ સિંક ભાગો}
\begin{tabular}{|l|l|}
\hline
\textbf{ભાગ} & \textbf{કાર્ય} \\ \hline
બેઝ & ડિવાઇસમાંથી ગરમી વહન કરે છે \\ \hline
ફિન્સ & ગરમી ફેલાવા માટે સરફેસ એરિયા વધારે છે \\ \hline
થર્મલ ઇન્ટરફેસ મટિરિયલ & ડિવાઇસ અને સિંક વચ્ચેનો સંપર્ક સુધારે છે \\ \hline
પ્રકારો & એક્સટ્રૂડેડ, બોન્ડેડ, ફોલ્ડેડ, ડાઇ-કાસ્ટ \\ \hline
\end{tabular}
\end{center}

\begin{itemize}
    \item \textbf{થર્મલ રેઝિસ્ટન્સ}: ઓછું તે ગરમી ફેલાવા માટે વધુ સારું.
    \item \textbf{મટિરિયલ}: સામાન્ય રીતે એલ્યુમિનિયમ અથવા કોપર સારી કન્ડક્ટિવિટી માટે.
    \item \textbf{સરફેસ એરિયા}: વધુ ફિન્સ એટલે વધુ સારું કૂલિંગ.
    \item \textbf{એરફ્લો}: કુશળ ગરમી દૂર કરવા માટે મહત્વપૂર્ણ.
\end{itemize}

\mnemonicbox{હીટ સિંક ટ્રાન્ઝિસ્ટરને ઠંડુ રાખે}
\end{solutionbox}

\questionmarks{2}{a}{3}
\textbf{D.C અને A.C. લોડ લાઇનોનું વર્ણન કરો.}

\begin{solutionbox}
\textbf{લોડ લાઇન્સ:}
લોડ લાઇન્સ ટ્રાન્ઝિસ્ટરનાં સંભવિત ઓપરેટિંગ પોઈન્ટ્સને તેના કેરેક્ટરિસ્ટિક કર્વ પર ગ્રાફિકલી દર્શાવે છે.

\begin{center}
\includefigure{figures/tex-diagrams/pdf/4321103-summer-2023-solution-q2a.gu.pdf}
\end{center}

\begin{itemize}
    \item \textbf{DC લોડ લાઇન}: DC સ્થિતિઓ હેઠળ બધા શક્ય ઓપરેટિંગ પોઈન્ટ્સ બતાવે છે.
    \begin{itemize}
        \item સમીકરણ: Ic = (VCC - VCE)/RC
        \item એન્ડપોઈન્ટ્સ: (0, VCC/RC) અને (VCC, 0)
    \end{itemize}
    \item \textbf{AC લોડ લાઇન}: AC સિગ્નલ હેન્ડલિંગ દરમિયાન ઓપરેટિંગ પોઈન્ટ્સ બતાવે છે.
    \begin{itemize}
        \item વધુ તીક્ષ્ણ ઢાળ: AC રેઝિસ્ટન્સ DC કરતાં ઓછો હોવાના કારણે.
        \item Q-પોઈન્ટ પર કેન્દ્રિત: બાયસિંગ દ્વારા સ્થાપિત ઓપરેટિંગ પોઈન્ટ.
    \end{itemize}
\end{itemize}

\mnemonicbox{DC પૂર્ણ આલેખે, AC માર્ગ બદલે}
\end{solutionbox}

\questionmarks{2}{b}{4}
\textbf{એમ્પ્લીફાયરની બેન્ડવિડ્થ અને ગેઇન-બેન્ડવિડ્થ ઉત્પાદનને સંક્ષિપ્તમાં સમજાવો.}

\begin{solutionbox}
\textbf{બેન્ડવિડ્થ અને ગેઇન-બેન્ડવિડ્થ ઉત્પાદન:}
એમ્પ્લીફાયર ફ્રીક્વન્સી પરફોર્મન્સ માટેની મુખ્ય વિશેષતાઓ.

\begin{center}
\includefigure{figures/tex-diagrams/pdf/4321103-summer-2023-solution-q2b.gu.pdf}
\end{center}

\begin{center}
\captionof{table}{ફ્રીક્વન્સી પરિમાણો}
\begin{tabular}{|l|l|}
\hline
\textbf{પેરામીટર} & \textbf{વર્ણન} \\ \hline
બેન્ડવિડ્થ & ફ્રીક્વન્સી રેન્જ જ્યાં ગેઇન 3dB કરતાં ઓછો ઘટે છે \\ \hline
લોઅર કટઓફ ($f_1$) & ફ્રીક્વન્સી જ્યાં નીચલા છેડે ગેઇન 3dB ઘટે છે \\ \hline
અપર કટઓફ ($f_2$) & ફ્રીક્વન્સી જ્યાં ઉપલા છેડે ગેઇન 3dB ઘટે છે \\ \hline
ગેઇન-બેન્ડવિડ્થ ઉત્પાદન & ગેઇન અને બેન્ડવિડ્થનો ગુણાકાર, સ્થિર રહે છે \\ \hline
\end{tabular}
\end{center}

\begin{itemize}
    \item \textbf{બેન્ડવિડ્થ ફોર્મ્યુલા}: $BW = f_2 - f_1$
    \item \textbf{ગેઇન-બેન્ડવિડ્થ}: ગેઇન બદલાય ત્યારે પણ સ્થિર રહે છે.
    \item \textbf{ટ્રેડ-ઓફ}: વધુ ગેઇન એટલે ઓછી બેન્ડવિડ્થ.
\end{itemize}

\mnemonicbox{સારી બેન્ડવિડ્થ શ્રેષ્ઠ ટ્રાન્સમિશન આપે}
\end{solutionbox}

\questionmarks{2}{c}{7}
\textbf{બે તબક્કાના RC કપલ્ડ એમ્પ્લીફાયરનો આવર્તન પ્રતિભાવ સમજાવો.}

\begin{solutionbox}
\textbf{બે-તબક્કાના RC કપલ્ડ એમ્પ્લીફાયર:}

\begin{center}
\includefigure{figures/tex-diagrams/pdf/4321103-summer-2023-solution-q2c.gu.pdf}
\end{center}

\begin{center}
\includefigure{figures/tex-diagrams/pdf/4321103-summer-2023-solution-q2c-2.gu.pdf}
\end{center}

\begin{itemize}
    \item \textbf{નીચલા આવર્તન પ્રતિભાવ}: કપલિંગ કેપેસિટર્સ દ્વારા મર્યાદિત.
    \item \textbf{મધ્ય આવર્તન પ્રતિભાવ}: મહત્તમ અને સપાટ ગેઇન.
    \item \textbf{ઉચ્ચ આવર્તન પ્રતિભાવ}: ટ્રાન્ઝિસ્ટર કેપેસિટન્સ દ્વારા મર્યાદિત.
\end{itemize}

\mnemonicbox{નીચે કપલિંગ નબળું, ઉપર કેપેસિટન્સ રોકે}
\end{solutionbox}

\questionmarks{2}{a}{3}
\textbf{ટ્રાન્ઝિસ્ટર બાયસિંગ માટે નિશ્ચિત બાયસ સર્કિટ સમજાવો.}

\begin{solutionbox}
\textbf{નિશ્ચિત બાયસ (Fixed Bias):}
બેઝ સાથે જોડાયેલ એક રેઝિસ્ટરનો ઉપયોગ થાય છે.

\begin{center}
\includefigure{figures/tex-diagrams/pdf/4321103-summer-2023-solution-q2a-2.gu.pdf}
\end{center}

\begin{itemize}
    \item \textbf{વિશ્લેષણ}:
    \begin{itemize}
        \item બેઝ કરંટ: $I_B = (V_{CC} - V_{BE}) / R_B$
        \item કલેક્ટર કરંટ: $I_C = \beta \times I_B$
    \end{itemize}
    \item \textbf{નુકસાન}: ઓછી સ્થિરતા, તાપમાન ફેરફારોથી અસર પામે છે.
\end{itemize}

\mnemonicbox{ફિક્સ બાયસ, ફેસ બર્ડન (અસ્થિરતાનો)}
\end{solutionbox}

\questionmarks{2}{b}{4}
\textbf{સિંગલ સ્ટેજ એમ્પ્લીફાયરનો આવર્તન પ્રતિભાવ સમજાવો.}

\begin{solutionbox}
\textbf{આવર્તન પ્રતિભાવ:}

\begin{center}
\includefigure{figures/tex-diagrams/pdf/4321103-summer-2023-solution-q2b-2.gu.pdf}
\end{center}

\begin{center}
\captionof{table}{પ્રદેશો}
\begin{tabular}{|l|l|}
\hline
\textbf{પ્રદેશ} & \textbf{લક્ષણો} \\ \hline
નીચલા આવર્તન & કપલિંગ કેપેસિટર્સને કારણે ગેઇન ઘટે છે \\ \hline
મધ્ય આવર્તન & મહત્તમ અને સ્થિર ગેઇન \\ \hline
ઉચ્ચ આવર્તન & ટ્રાન્ઝિસ્ટર કેપેસિટન્સને કારણે ગેઇન ઘટે છે \\ \hline
\end{tabular}
\end{center}

\begin{itemize}
    \item \textbf{કટ-ઓફ આવર્તન}: જ્યાં ગેઇન 3dB ઘટે છે.
    \item \textbf{બેન્ડવિડ્થ}: $BW = f_2 - f_1$.
\end{itemize}

\mnemonicbox{નીચું મધ્ય ઉંચું - કેપેસિટર અહીં મહત્વપૂર્ણ છે}
\end{solutionbox}

\questionmarks{2}{c}{7}
\textbf{ટ્રાન્સફોર્મર કપલ્ડ એમ્પ્લીફાયર અને RC કપલ્ડ એમ્પ્લીફાયરની સરખામણી કરો}

\begin{solutionbox}
\begin{center}
\captionof{table}{સરખામણી}
\begin{tabular}{|l|p{4cm}|p{4cm}|}
\hline
\textbf{પેરામીટર} & \textbf{RC કપલ્ડ} & \textbf{ટ્રાન્સફોર્મર કપલ્ડ} \\ \hline
કપલિંગ તત્વ & રેઝિસ્ટર અને કેપેસિટર & ટ્રાન્સફોર્મર \\ \hline
આવર્તન પ્રતિભાવ & વિશાળ બેન્ડવિડ્થ & મર્યાદિત બેન્ડવિડ્થ \\ \hline
કાર્યક્ષમતા & ઓછી (20-25\%) & ઉચ્ચ (50-60\%) \\ \hline
કદ \& વજન & નાનું, હલકું & મોટું, ભારે \\ \hline
કિંમત & સસ્તી & મોંઘી \\ \hline
ઇમ્પીડન્સ મેચિંગ & નબળું & ઉત્કૃષ્ટ \\ \hline
એપ્લિકેશન્સ & વોલ્ટેજ એમ્પ્લિફિકેશન & પાવર એમ્પ્લિફિકેશન \\ \hline
\end{tabular}
\end{center}

\begin{center}
\includefigure{figures/tex-diagrams/pdf/4321103-summer-2023-solution-q2c-3.gu.pdf}
\end{center}

\mnemonicbox{RC વિશાળતા લે, ટ્રાન્સફોર્મર પાવર લે}
\end{solutionbox}

\questionmarks{3}{a}{3}
\textbf{ડાયરેક્ટ કપલ્ડ એમ્પ્લીફાયરને સંક્ષિપ્તમાં સમજાવો.}

\begin{solutionbox}
\textbf{ડાયરેક્ટ કપલ્ડ એમ્પ્લીફાયર:}
તબક્કાઓને કપલિંગ કેપેસિટર્સ અથવા ટ્રાન્સફોર્મર વિના જોડે છે.

\begin{center}
\includefigure{figures/tex-diagrams/pdf/4321103-summer-2023-solution-q3a.gu.pdf}
\end{center}

\begin{itemize}
    \item \textbf{DC સિગ્નલ હેન્ડલિંગ}: ખૂબ નીચા આવર્તન અને DC એમ્પ્લિફાય કરી શકે છે.
    \item \textbf{કોઈ કપલિંગ તત્વો નહીં}: પ્રથમ તબક્કાનું આઉટપુટ સીધું આગલા તબક્કાના ઇનપુટને જોડે છે.
    \item \textbf{નુકસાન}: થર્મલ ડ્રિફ્ટ, બાયસ સ્થિરતાના મુદ્દાઓ.
\end{itemize}

\mnemonicbox{સીધું જોડાયેલ, સંપૂર્ણ શૂન્ય આવર્તન સુધી}
\end{solutionbox}

\questionmarks{3}{b}{4}
\textbf{એમ્પ્લીફાયરના ફ્રીક્વન્સી રિસ્પોન્સ પર એમિટર બાયપાસ કેપેસિટર અને કપલિંગ કેપેસિટરની અસરો સમજાવો.}

\begin{solutionbox}
\textbf{કેપેસિટરની અસરો:}

\begin{center}
\captionof{table}{અસરો}
\begin{tabular}{|l|l|p{5cm}|}
\hline
\textbf{ઘટક} & \textbf{કાર્ય} & \textbf{આવર્તન પ્રતિભાવ પર અસર} \\ \hline
એમિટર બાયપાસ કેપેસિટર & RE આસપાસ AC બાયપાસ કરે છે & મધ્ય અને ઉચ્ચ આવર્તનો પર ગેઇન વધારે છે. \\ \hline
કપલિંગ કેપેસિટર & DC અવરોધે, AC પસાર કરે & નીચલી કટ-ઓફ આવર્તન નક્કી કરે છે. \\ \hline
\end{tabular}
\end{center}

\begin{center}
\includefigure{figures/tex-diagrams/pdf/4321103-summer-2023-solution-q3b.gu.pdf}
\end{center}

\mnemonicbox{કપલિંગ નીચા નિયંત્રણ કરે, બાયપાસ બધાને વધારે}
\end{solutionbox}

\questionmarks{3}{c}{7}
\textbf{ટ્રાન્ઝિસ્ટર ટુ પોર્ટ નેટવર્ક દોરો અને તેના માટે h-પેરામીટર્સનું વર્ણન કરો. હાઇબ્રિડ પરિમાણોના ફાયદા લખો.}

\begin{solutionbox}
\textbf{બે-પોર્ટ નેટવર્ક મોડેલ:}

\begin{center}
\includefigure{figures/tex-diagrams/pdf/4321103-summer-2023-solution-q3c.gu.pdf}
\end{center}

\textbf{H-પેરામીટર્સ:}
\begin{enumerate}
    \item \textbf{$h_{11}$ ($h_i$)}: ઇનપુટ ઇમ્પીડન્સ.
    \item \textbf{$h_{12}$ ($h_r$)}: રિવર્સ વોલ્ટેજ ગેઇન.
    \item \textbf{$h_{21}$ ($h_f$)}: ફોરવર્ડ કરંટ ગેઇન.
    \item \textbf{$h_{22}$ ($h_o$)}: આઉટપુટ એડમિટન્સ.
\end{enumerate}

\textbf{ફાયદા:}
\begin{itemize}
    \item સરળતાથી માપી શકાય.
    \item મોડેલ ચોકસાઈ.
    \item મિશ્રિત એકમો.
\end{itemize}

\mnemonicbox{ઇનપુટ, રિવર્સ, ફોરવર્ડ, આઉટપુટ - IRFO પેરામીટર્સ}
\end{solutionbox}

\questionmarks{3}{a}{3}
\textbf{એમ્પ્લીફાયરનો ફ્રીક્વન્સી રિસ્પોન્સ દોરો અને પ્રતિસાદ પર એમ્પ્લીફાયરની અપર કટ-ઓફ ફ્રીક્વન્સી, લોઅર કટ-ઓફ ફ્રીક્વન્સી, બેન્ડવિડ્થ અને મિડ ફ્રીક્વન્સી ગેઇન સૂચવો.}

\begin{solutionbox}
\textbf{ફ્રીક્વન્સી રિસ્પોન્સ:}

\begin{center}
\includefigure{figures/tex-diagrams/pdf/4321103-summer-2023-solution-q3a-2.gu.pdf}
\end{center}

\mnemonicbox{લોઅર બેન્ડવિડ્થ અપર એમ્પ્લીફાયર પ્રતિસાદ બનાવે}
\end{solutionbox}

\questionmarks{3}{b}{4}
\textbf{ટ્યુન કરેલ એમ્પ્લીફાયર તરીકે ઉપયોગમાં લેવાતા ટ્રાન્ઝિસ્ટરનું વર્ણન કરો.}

\begin{solutionbox}
\textbf{ટ્યુન્ડ એમ્પ્લીફાયર:}
ચોક્કસ આવર્તનો પર સિગ્નલને પસંદગીપૂર્વક એમ્પ્લિફાય કરવા માટે LC રેઝોનન્ટ સર્કિટનો ઉપયોગ કરે છે.

\begin{center}
\includefigure{figures/tex-diagrams/pdf/4321103-summer-2023-solution-q3b-2.gu.pdf}
\end{center}

\begin{itemize}
    \item \textbf{રેઝોનન્સ આવર્તન}: $f = \frac{1}{2\pi\sqrt{LC}}$.
    \item \textbf{એપ્લિકેશન્સ}: RF રિસીવર્સ, કોમ્યુનિકેશન.
\end{itemize}

\mnemonicbox{ટ્યુનિંગ LC સિગ્નલ્સ ચોકસાઈથી પસંદ કરે}
\end{solutionbox}

\questionmarks{3}{c}{7}
\textbf{બે પોર્ટ નેટવર્કમાં h પરિમાણોનું મહત્વ વર્ણવો. CE એમ્પ્લીફાયર માટે h-પેરામીટર્સ સર્કિટ દોરો.}

\begin{solutionbox}
\textbf{મહત્વ:} સર્કિટ વિશ્લેષણ, ડિઝાઇન ગણતરીઓ, મેન્યુફેક્ચરર સ્પેક્સ.

\textbf{CE એમ્પ્લીફાયર h-પેરામીટર સર્કિટ:}

\begin{center}
\includefigure{figures/tex-diagrams/pdf/4321103-summer-2023-solution-q3c-2.gu.pdf}
\end{center}

\mnemonicbox{ઇનપુટ રેઝિસ્ટન્સ, ફીડબેક રેશિયો, ફોરવર્ડ ગેઇન, આઉટપુટ કન્ડક્ટન્સ}
\end{solutionbox}

\questionmarks{4}{a}{3}
\textbf{જરૂરી ડાયાગ્રામ સાથે ડાયોડ ક્લિપર સર્કિટનું વર્ણન કરો.}

\begin{solutionbox}
\textbf{ડાયોડ ક્લિપર:}
ક્લિપર સર્કિટ ઇનપુટ સિગ્નલના તે ભાગને મર્યાદિત કરે છે અથવા કાપી નાખે છે.

\begin{center}
\includefigure{figures/tex-diagrams/pdf/4321103-summer-2023-solution-q4a.gu.pdf}
\end{center}

\mnemonicbox{નિશ્ચિત પોઈન્ટ પર ભાગોને કાપી નાખે}
\end{solutionbox}

\questionmarks{4}{b}{4}
\textbf{LDR પર ટૂંકી નોંધ સમજાવો.}

\begin{solutionbox}
\textbf{LDR (લાઇટ ડિપેન્ડન્ટ રેઝિસ્ટર):}
પ્રકાશની તીવ્રતા વધવાથી રેઝિસ્ટન્સ ઘટે છે.

\begin{center}
\includefigure{figures/tex-diagrams/pdf/4321103-summer-2023-solution-q4b.gu.pdf}
\end{center}

\begin{itemize}
    \item \textbf{રચના}: કેડમિયમ સલ્ફાઇડ (CdS).
    \item \textbf{એપ્લિકેશન્સ}: લાઇટ સેન્સર, ઓટોમેટિક લાઇટિંગ.
\end{itemize}

\mnemonicbox{પ્રકાશ રેઝિસ્ટન્સ ઘટાડે}
\end{solutionbox}

\questionmarks{4}{c}{7}
\textbf{ડાર્લિંગ્ટન જોડી અને તેની એપ્લિકેશનો સમજાવો.}

\begin{solutionbox}
\textbf{ડાર્લિંગ્ટન જોડી:}
બે ટ્રાન્ઝિસ્ટર એવી રીતે જોડાયેલા હોય છે કે કરંટ ગેઇન ખૂબ ઊંચો થાય.

\begin{center}
\includefigure{figures/tex-diagrams/pdf/4321103-summer-2023-solution-q4c.gu.pdf}
\end{center}

\textbf{લક્ષણો:}
\begin{itemize}
    \item ખૂબ ઊંચો કરંટ ગેઇન ($\beta \approx \beta_1 \times \beta_2$).
    \item ખૂબ ઊંચું ઇનપુટ ઇમ્પીડન્સ.
\end{itemize}

\textbf{એપ્લિકેશન્સ}: પાવર એમ્પ્લીફાયર, રિલે ડ્રાઇવર.

\mnemonicbox{બમણા ટ્રાન્ઝિસ્ટર ખૂબ વધારે એમ્પ્લિફાય કરે}
\end{solutionbox}

\questionmarks{4}{a}{3}
\textbf{જરૂરી ડાયાગ્રામ સાથે ડાયોડ ક્લેમ્પર સર્કિટનું વર્ણન કરો.}

\begin{solutionbox}
\textbf{ડાયોડ ક્લેમ્પર:}
વેવફોર્મને તેના આકારને બદલ્યા વિના DC ઘટક ઉમેરીને ઉપર અથવા નીચે શિફ્ટ કરે છે.

\begin{center}
\includefigure{figures/tex-diagrams/pdf/4321103-summer-2023-solution-q4a-2.gu.pdf}
\end{center}

\mnemonicbox{પીક્સને સતત નીચે જકડે}
\end{solutionbox}

\questionmarks{4}{b}{4}
\textbf{OLED નું કાર્ય અને એપ્લિકેશન સમજાવો.}

\begin{solutionbox}
\textbf{OLED (ઓર્ગેનિક LED):}

\begin{center}
\includefigure{figures/tex-diagrams/pdf/4321103-summer-2023-solution-q4b-2.gu.pdf}
\end{center}

\begin{itemize}
    \item \textbf{ફાયદાઓ}: સ્વ-પ્રકાશિત, પાતળા, હલકા.
    \item \textbf{એપ્લિકેશન્સ}: સ્માર્ટફોન, ટીવી.
\end{itemize}

\mnemonicbox{ઓર્ગેનિક લેયર્સ ડાયોડ-પ્રકાશ ઉત્સર્જિત કરે}
\end{solutionbox}

\questionmarks{4}{c}{7}
\textbf{રિલે ડ્રાઇવર તરીકે વપરાતા ટ્રાન્ઝિસ્ટરનું વર્ણન કરો.}

\begin{solutionbox}
\textbf{રિલે ડ્રાઇવર:}
ટ્રાન્ઝિસ્ટર રિલેને નિયંત્રિત કરવા માટે સ્વિચ તરીકે કાર્ય કરે છે.

\begin{center}
\includefigure{figures/tex-diagrams/pdf/4321103-summer-2023-solution-q4c-2.gu.pdf}
\end{center}

\begin{itemize}
    \item \textbf{ફ્લાયબેક ડાયોડ}: બેક EMF થી ટ્રાન્ઝિસ્ટરને સુરક્ષિત કરે છે.
\end{itemize}

\mnemonicbox{નાનું મોટા રિલે ચલાવે}
\end{solutionbox}

\questionmarks{5}{a}{3}
\textbf{LM317 IC નો ઉપયોગ કરીને વેરિયેબલ પાવર સપ્લાયનો સર્કિટ ડાયાગ્રામ દોરો.}

\begin{solutionbox}
\textbf{LM317 વેરિયેબલ સપ્લાય:}

\begin{center}
\includefigure{figures/tex-diagrams/pdf/4321103-summer-2023-solution-q5a.gu.pdf}
\end{center}

ફોર્મ્યુલા: $V_{out} = 1.25(1 + \frac{R_2}{R_1})$.

\mnemonicbox{LM317 વોલ્ટેજ એડજસ્ટેબલ બનાવે}
\end{solutionbox}

\questionmarks{5}{b}{4}
\textbf{યુપીએસની કામગીરી સમજાવો.}

\begin{solutionbox}
\textbf{UPS (અનઇન્ટરપ્ટિબલ પાવર સપ્લાય):}
મુખ્ય પાવર ફેઇલ થાય ત્યારે ઇમરજન્સી પાવર આપે છે.

\begin{center}
\includefigure{figures/tex-diagrams/pdf/4321103-summer-2023-solution-q5b.gu.pdf}
\end{center}

\mnemonicbox{અવિરત પાવર બ્લેકઆઉટ દરમિયાન આપે}
\end{solutionbox}

\questionmarks{5}{c}{7}
\textbf{SMPS બ્લોક ડાયાગ્રામ દોરો અને સમજાવો.}

\begin{solutionbox}
\textbf{SMPS (સ્વિચ મોડ પાવર સપ્લાય):}
ઉચ્ચ કાર્યક્ષમતા માટે સ્વિચિંગ રેગ્યુલેશનનો ઉપયોગ કરે છે.

\begin{center}
\includefigure{figures/tex-diagrams/pdf/4321103-summer-2023-solution-q5c.gu.pdf}
\end{center}

\mnemonicbox{સ્વિચ પાવરને સ્થિર બનાવે}
\end{solutionbox}

\questionmarks{5}{a}{3}
\textbf{IC નો ઉપયોગ કરીને +15 v પાવર સપ્લાય માટે સર્કિટ ડાયાગ્રામ દોરો અને ટૂંકમાં સમજાવો}

\begin{solutionbox}
\textbf{+15V પાવર સપ્લાય (7815 IC):}

\begin{center}
\includefigure{figures/tex-diagrams/pdf/4321103-summer-2023-solution-q5a-2.gu.pdf}
\end{center}

\mnemonicbox{7815 Fixes Voltage To Fifteen}
\end{solutionbox}

\questionmarks{5}{b}{4}
\textbf{સૌર બેટરી ચાર્જર સર્કિટનું કાર્ય સમજાવો.}

\begin{solutionbox}
\textbf{સૌર બેટરી ચાર્જર:}

\begin{center}
\includefigure{figures/tex-diagrams/pdf/4321103-summer-2023-solution-q5b-2.gu.pdf}
\end{center}

\mnemonicbox{સૂર્ય બેટરી સુરક્ષિત ચાર્જ કરે}
\end{solutionbox}

\questionmarks{5}{c}{7}
\textbf{લિનિયર રેગ્યુલેટેડ પાવર સપ્લાય સાથે સ્વિચ મોડ પાવર સપ્લાયની સરખામણી ચર્ચા કરો.}

\begin{solutionbox}
\textbf{સરખામણી:}

\begin{center}
\captionof{table}{લિનિયર vs SMPS}
\begin{tabular}{|l|l|l|}
\hline
\textbf{પેરામીટર} & \textbf{લિનિયર PS} & \textbf{SMPS} \\ \hline
કાર્યક્ષમતા & નીચી (30-40\%) & ઉચ્ચ (70-90\%) \\ \hline
કદ/વજન & મોટું/ભારે & કોમ્પેક્ટ/હલકું \\ \hline
નોઇઝ & નીચું & ઉચ્ચ (સ્વિચિંગ નોઇઝ) \\ \hline
જટિલતા & સરળ & જટિલ \\ \hline
\end{tabular}
\end{center}

\begin{center}
\includefigure{figures/tex-diagrams/pdf/4321103-summer-2023-solution-q5c-2.gu.pdf}
\end{center}

\mnemonicbox{લિનિયર ઓછા નોઇઝને પસંદ કરે, સ્વિચિંગ કદ બચાવે}
\end{solutionbox}

\end{document}

