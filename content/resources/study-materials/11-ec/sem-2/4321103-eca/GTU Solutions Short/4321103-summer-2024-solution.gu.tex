\documentclass{article}

% content/resources/templates/preamble.tex
\usepackage[margin=0.6in]{geometry}
\author{Milav Dabgar}
\usepackage{amsmath,amssymb,amsthm}
\usepackage{booktabs}
\usepackage{multirow}
\usepackage{xcolor}
\usepackage{tcolorbox}
\tcbuselibrary{breakable,skins}
\usepackage[colorlinks=true,linkcolor=blue]{hyperref}
\usepackage{titlesec}
\usepackage{enumitem}
\usepackage{tikz}
\usepackage{pgfplots}
\usepackage{circuitikz}
\usepackage[version=4]{mhchem}
\usepackage{longtable}
\usepackage{array}
\usepackage{float}
\usepackage{caption}
\usepackage{listings}

\lstset{
  basicstyle=\small\ttfamily,
  breaklines=true,
  breakatwhitespace=false,
  postbreak=\mbox{\textcolor{red}{$\hookrightarrow$}\space},
  float=false,
  numbers=left,
  numberstyle=\tiny\color{gray},
  numbersep=10pt,
  xleftmargin=2em,
  keywordstyle=\color{blue},
  commentstyle=\color{green!60!black},
  stringstyle=\color{purple},
  backgroundcolor=\color{gray!5},
  showstringspaces=false,
  tabsize=2,
  captionpos=b,
  keepspaces=true,
  columns=flexible
}

\pgfplotsset{compat=1.18}
\usetikzlibrary{shapes,arrows,positioning,calc,patterns,decorations.pathmorphing,decorations.markings,arrows.meta}

% Color scheme
\definecolor{headcolor}{RGB}{0,102,204}
\definecolor{keycolor}{RGB}{220,20,60}
\definecolor{solutioncolor}{RGB}{34,139,34}
\definecolor{mnemoniccolor}{RGB}{148,0,211}
\definecolor{codecolor}{RGB}{0,0,100}

% Spacing
\setlength{\parskip}{3pt}
\setlist[itemize]{nosep}
\setlist[enumerate]{nosep}

% Title formatting
\titleformat{\section}{\Large\bfseries\color{headcolor}}{\thesection}{1em}{}
\titleformat{\subsection}{\large\bfseries\color{headcolor}}{\thesubsection}{1em}{}

% Pandoc tightlist compatibility
\providecommand{\tightlist}{%
  \setlength{\itemsep}{0pt}\setlength{\parskip}{0pt}}

% Pandoc longtable compatibility
\newcounter{none}
\def\thenone{}


% content/resources/templates/gujarati-boxes.tex
\usepackage{fontspec}
\usepackage{polyglossia}

% Set Gujarati as main language (document is primarily in Gujarati)
% Note: gloss-gujarati.ldf doesn't exist in polyglossia, but it will use hyphenation patterns
\setdefaultlanguage{gujarati}
\setotherlanguage{english}

% Configure Gujarati font properly
% Use Language=Default to prevent polyglossia from trying to add language-specific features
% that don't exist for Gujarati, which causes "empty feature" warnings
\newfontfamily\gujaratifont[Script=Gujarati,AutoFakeBold=2.5,AutoFakeSlant=0.3]{Noto Sans Gujarati}
\setmainfont[Script=Gujarati,AutoFakeBold=2.5,AutoFakeSlant=0.3]{Noto Sans Gujarati}
% Use Noto Sans Gujarati for monospace to support Gujarati in text
\setmonofont[Scale=0.9]{Noto Sans Gujarati}

% Configure English to use the same font
\newfontfamily\englishfont[Script=Gujarati,AutoFakeBold=2.5,AutoFakeSlant=0.3]{Noto Sans Gujarati}

% Translations for polyglossia
\gappto\captionsgujarati{
  \renewcommand{\tablename}{કોષ્ટક}
  \renewcommand{\figurename}{આકૃતિ}
}

% Helper for TikZ nodes to ensure Gujarati font
\newcommand{\gu}[1]{{\gujaratifont #1}}

% Custom environments
\newtcolorbox{solutionbox}{
    breakable,
    enhanced,
    colback=solutioncolor!5!white,
    colframe=solutioncolor!75!black,
    fonttitle=\bfseries,
    title=જવાબ
}

\newtcolorbox{solutionboxnobreak}{
 colback=solutioncolor!5!white,
 colframe=solutioncolor!75!black,
 fonttitle=\bfseries,
 title=જવાબ
}

\newtcolorbox{keyformula}{
 breakable,
 enhanced,
 colback=keycolor!5!white,
 colframe=keycolor!75!black,
 fonttitle=\bfseries,
 title=રાસાયણિક સમીકરણ/સૂત્ર
}

\newtcolorbox{mnemonicbox}{
 breakable,
 enhanced,
 colback=mnemoniccolor!5!white,
 colframe=mnemoniccolor!75!black,
 fonttitle=\bfseries,
 title=મેમરી ટ્રીક
}


% Custom commands for GTU solutions
% This file defines semantic commands for consistent formatting

% Question command with automatic formatting
\newcommand{\question}[2]{%
  \section*{Question #1}%
  \textbf{#2}%
}

% OR question variant
\newcommand{\questionor}[2]{%
  \section*{Question #1 OR}%
  \textbf{#2}%
}

% Proper table environment with caption
\newenvironment{answertable}[1]{%
  \begin{table}[htbp]
  \centering
  \caption{#1}
}{%
  \end{table}
}

% Proper figure environment for diagrams
\newenvironment{answerdiagram}[1]{%
  \begin{figure}[htbp]
  \centering
  \caption{#1}
}{%
  \end{figure}
}

% Semantic markup for key terms
\newcommand{\keyword}[1]{\textbf{#1}}
\newcommand{\code}[1]{\texttt{#1}}
\newcommand{\classname}[1]{\texttt{#1}}
\newcommand{\methodname}[1]{\texttt{#1}}

% Proper quotation marks
\newcommand{\mnemonic}[1]{``#1''}


\title{ઇલેક્ટ્રોનિક સર્કિટ્સ એન્ડ એપ્લિકેશન્સ (4321103) - ઉનાળુ 2024 સોલ્યુશન}
\date{June 18, 2024}

\begin{document}
\maketitle

\questionmarks{1}{a}{3}
\textbf{CE રૂપરેખાંકન માટે એમ્પ્લીફાયર પરિમાણો Ai, Ri અને Ro સમજાવો.}

\begin{solutionbox}
    કોમન એમિટર (CE) રૂપરેખાંકનમાં, મુખ્ય પરિમાણો છે:

    \begin{center}
    \begin{circuitikz}[american]
        \draw (0,0) node[npn](Q){};
        \draw (Q.E) -- (0,-1) node[ground]{};
        \draw (Q.C) to[short, -o] (2,1) node[right]{આઉટપુટ};
        \draw (Q.C) to[R, l=$R_C$] (0,3) node[vcc]{$+V_{CC}$};
        \draw (Q.B) to[R, l=$R_B$, -o] (-3,0) node[left]{ઇનપુટ};
    \end{circuitikz}
    \end{center}

    \begin{itemize}
        \item \textbf{કરંટ ગેઇન ($A_i$)}: આઉટપુટ કરંટનો ઇનપુટ કરંટ સાથેનો ગુણોત્તર ($I_c/I_b$). સામાન્ય રીતે CE માં 50-200.
        \item \textbf{ઇનપુટ રેઝિસ્ટન્સ ($R_i$)}: બેઝ ટર્મિનલ પર ઇનપુટ કરંટનો વિરોધ. CE માં 1-2 k$\Omega$.
        \item \textbf{આઉટપુટ રેઝિસ્ટન્સ ($R_o$)}: કલેક્ટર ટર્મિનલ પર વિરોધ. સામાન્ય રીતે CE માં 50 k$\Omega$.
    \end{itemize}

    \mnemonicbox{"CIR પરિમાણો - કરંટ ગેઇન, ઇનપુટ રેઝિસ્ટન્સ, અને આઉટપુટ રેઝિસ્ટન્સ એમ્પ્લીફાયરની કાર્યક્ષમતા નક્કી કરે છે"}
\end{solutionbox}

\questionmarks{1}{b}{4}
\textbf{હીટ સિંક પર ટૂંકી નોંધ લખો.}

\begin{solutionbox}
    \begin{center}
    \begin{tikzpicture}
        \draw[fill=gray!30] (0,0) rectangle (4,1);
        \node at (2,0.5) {ટ્રાન્ઝિસ્ટર};
        \foreach \x in {0.2, 0.6, ..., 3.8} {
            \draw[fill=gray!50] (\x,1) rectangle (\x+0.2,2);
        }
        \node at (2,2.3) {ફિન્સ};
        \draw[->] (2,2.1) -- (2,1.8);
    \end{tikzpicture}
    \end{center}

    \begin{itemize}
        \item \textbf{ઉદ્દેશ}: ઇલેક્ટ્રોનિક ઘટકોમાંથી (જેમ કે પાવર ટ્રાન્ઝિસ્ટર) થર્મલ નુકસાન રોકવા માટે વધારાની ગરમીનું વિસર્જન કરે છે.
        \item \textbf{પ્રકારો}:
        \begin{itemize}
            \item \textbf{પેસિવ}: એલ્યુમિનિયમ અથવા કોપર ફિન્સ જે કુદરતી સંવહન પર આધાર રાખે છે.
            \item \textbf{એક્ટિવ}: બળજબરીથી સંવહન માટે પંખા અથવા પ્રવાહી ઠંડકનો ઉપયોગ કરે છે.
        \end{itemize}
        \item \textbf{થર્મલ રેઝિસ્ટન્સ}: ઓછી થર્મલ રેઝિસ્ટન્સ ($\theta$, $^\circ$C/W માં મપાય છે) વધુ સારી ગરમી વિસર્જન ક્ષમતા દર્શાવે છે.
        \item \textbf{સામગ્રી}: કોપર (શ્રેષ્ઠ વાહકતા) અથવા એલ્યુમિનિયમ (હલકું, કિફાયતી).
    \end{itemize}

    \mnemonicbox{"HARD સિંક - ગરમીને \textbf{H}eat \textbf{A}way using \textbf{R}adiation and \textbf{D}issipation through metal sinks"}
\end{solutionbox}

\questionmarks{1}{c}{7}
\textbf{થર્મલ રનઅવે અને થર્મલ સ્ટેબિલિટીનું વર્ણન કરો. ટ્રાન્ઝિસ્ટરમાં થર્મલ રનઅવે કેવી રીતે દૂર કરી શકાય?}

\begin{solutionbox}
    \begin{center}
    \begin{tikzpicture}[node distance=1.5cm, auto,
        gtu block/.style={rectangle, draw, fill=blue!10, text width=2.5cm, text centered, rounded corners, minimum height=1.5em},
        line/.style={draw, -latex', thick}]
        \node [gtu block] (heat) {ગરમી ઉત્પાદન};
        \node [gtu block, right=of heat] (temp) {તાપમાનમાં વધારો};
        \node [gtu block, below=of temp] (ic) {$I_c$ માં વધારો};
        \node [gtu block, left=of ic] (power) {વધુ પાવર વપરાશ};
        
        \path [line] (heat) -- (temp);
        \path [line] (temp) -- (ic);
        \path [line] (ic) -- (power);
        \path [line] (power) -- (heat);
        
        \node [gtu block, below=of power, fill=green!10] (method) {સ્ટેબિલિટી પદ્ધતિઓ};
        \node [gtu block, right=of method, fill=green!10] (break) {ચક્ર તોડો};
        \path [line] (method) -- (break);
        \draw [line, dashed] (break.north) -- (ic.south);
    \end{tikzpicture}
    \end{center}

    \textbf{થર્મલ રનઅવે:}
    \begin{itemize}
        \item \textbf{વ્યાખ્યા}: સ્વ-ત્વરિત વિનાશક પ્રક્રિયા જ્યાં ટ્રાન્ઝિસ્ટર ગરમ થાય છે, જેનાથી વધુ કરંટ પ્રવાહ અને વધુ ગરમી થાય છે.
        \item \textbf{ચેઇન રિએક્શન}: તાપમાનમાં વધારો $\rightarrow$ લીકેજ કરંટ ($I_{CO}$) વધે છે $\rightarrow$ કલેક્ટર કરંટ ($I_C$) વધે છે $\rightarrow$ પાવર ડિસિપેશન ($P_D$) વધે છે $\rightarrow$ વધુ તાપમાન વધારો.
        \item \textbf{પરિણામ}: જો નિયંત્રણ ન કરવામાં આવે તો ટ્રાન્ઝિસ્ટરનો ભૌતિક વિનાશ થાય છે.
    \end{itemize}

    \textbf{થર્મલ સ્ટેબિલિટી:}
    \begin{itemize}
        \item \textbf{વ્યાખ્યા}: તાપમાન પરિવર્તન છતાં સ્થિર ઓપરેટિંગ પોઇન્ટ ($Q$-પોઇન્ટ) જાળવવાની ક્ષમતા.
        \item \textbf{માપ}: સ્ટેબિલિટી ફેક્ટર ($S$) દ્વારા માપવામાં આવે છે. ઓછા મૂલ્યો વધુ સારી સ્થિરતા દર્શાવે છે.
    \end{itemize}

    \textbf{થર્મલ રનઅવે દૂર કરવાના ઉપાયો:}
    \begin{enumerate}
        \item \textbf{હીટ સિંક્સ}: વધારાની ગરમી દૂર કરવા માટે સપાટીનો વિસ્તાર વધારવો.
        \item \textbf{એમિટર રેઝિસ્ટર}: નેગેટિવ ફીડબેક આપવા માટે અનબાયપાસ્ડ એમિટર રેઝિસ્ટર ($R_E$) વાપરો.
        \item \textbf{બાયસિંગ પદ્ધતિઓ}: ફિક્સ્ડ બાયસને બદલે વોલ્ટેજ ડિવાઇડર બાયસ વાપરો જે વધુ સ્થિર છે.
        \item \textbf{થર્મલ કમ્પેન્સેશન}: બાયસ સર્કિટમાં તાપમાન-સંવેદનશીલ ઘટકો (થર્મિસ્ટર, ડાયોડ) વાપરો.
    \end{enumerate}

    \mnemonicbox{"SHEER સુરક્ષા - ગરમી માટે \textbf{S}inks, \textbf{E}mitter resistors, \textbf{E}xternal cooling, અને \textbf{R}obust biasing થર્મલ રનઅવે અટકાવે છે"}
\end{solutionbox}

\questionmarks{1}{c}{7}
\textbf{બાયસિંગ પદ્ધતિઓના પ્રકારો લખો. વોલ્ટેજ વિભાજક બાયસિંગ પદ્ધતિને વિગતમાં સમજાવો.}

\begin{solutionbox}
    \textbf{બાયસિંગ પદ્ધતિઓના પ્રકારો:}
    \begin{center}
    \begin{tabulary}{\linewidth}{|L|L|L|}
        \hline
        \textbf{પદ્ધતિ} & \textbf{સ્થિરતા} & \textbf{જટિલતા} \\
        \hline
        ફિક્સ્ડ બાયસ & નબળી & સરળ \\
        \hline
        કલેક્ટર ફીડબેક & મધ્યમ & મધ્યમ \\
        \hline
        એમિટર બાયસ & સારી & મધ્યમ \\
        \hline
        વોલ્ટેજ વિભાજક & ઉત્તમ & જટિલ \\
        \hline
    \end{tabulary}
    \end{center}

    \textbf{વોલ્ટેજ વિભાજક બાયસિંગ:}

    \begin{center}
    \begin{circuitikz}[american]
        \draw (0,0) node[npn](Q){};
        \draw (Q.E) to[R, l=$R_E$] (0,-2) node[ground]{};
        \draw (Q.C) to[R, l=$R_C$] (0,2) -- (0,3) node[vcc]{$+V_{CC}$};
        \draw (Q.C) to[short, -o] (2,0) node[right]{આઉટપુટ};
        \draw (Q.B) -- (-1.5,0);
        \draw (-1.5,0) to[R, l=$R_2$] (-1.5,-2) node[ground]{};
        \draw (-1.5,0) to[R, l=$R_1$] (-1.5,3) node[vcc]{$+V_{CC}$};
        \draw (-1.5,0) to[C, l=$C_{in}$, -o] (-3.5,0) node[left]{ઇનપુટ};
        \draw (2,0) to[C, l=$C_{out}$] (1,0); 
    \end{circuitikz}
    \captionof{figure}{વોલ્ટેજ વિભાજક બાયસ સર્કિટ}
    \end{center}

    \begin{itemize}
        \item \textbf{સર્કિટ સ્ટ્રક્ચર}: બેઝ પર સ્થિર વોલ્ટેજ બનાવવા માટે $V_{CC}$ સાથે શ્રેણીમાં બે રેઝિસ્ટર્સ ($R_1, R_2$) વાપરે છે.
        \item \textbf{ઓપરેટિંગ સિદ્ધાંત}: $R_2$ પર મળતો વોલ્ટેજ ($V_B$) એમિટર જંકશનને ફોરવર્ડ બાયસ કરે છે.
        \item \textbf{બેઝ વોલ્ટેજ}: $V_B = V_{CC} \times \frac{R_2}{R_1 + R_2}$
        \item \textbf{સ્થિરતા}: આ સૌથી વધુ વપરાતી પદ્ધતિ છે કારણ કે ઓપરેટિંગ પોઇન્ટ ટ્રાન્ઝિસ્ટરના $\beta$ થી લગભગ સ્વતંત્ર છે.
            \begin{itemize}
                \item જો તાપમાનથી $I_C$ વધે, તો $I_E$ વધે છે ($I_E \approx I_C$).
                \item $R_E$ પર વોલ્ટેજ ડ્રોપ ($V_E = I_E R_E$) વધે છે.
                \item $V_B$ અચળ હોવાથી, $V_{BE} = V_B - V_E$ ઘટે છે.
                \item $V_{BE}$ ઘટવાથી $I_B$ ઘટે છે, જે $I_C$ ઘટાડે છે અને સર્કિટને સ્થિર કરે છે.
            \end{itemize}
        \item \textbf{ફાયદો}: ઉચ્ચ સ્થિરતા ફેક્ટર ($S \approx 1$).
    \end{itemize}

    \mnemonicbox{"DIVE સ્થિરતા માટે - \textbf{D}ivider \textbf{I}s \textbf{V}ery \textbf{E}ffective તાપમાન અને $\beta$ વેરિએશન માટે"}
\end{solutionbox}

\questionmarks{2}{a}{3}
\textbf{સ્ટેબિલિટી ફેક્ટર અને તેની વિશેષતાઓ સમજાવો.}

\begin{solutionbox}
    \begin{center}
    \begin{tikzpicture}[node distance=2cm, auto,
        gtu state/.style={circle, draw, fill=orange!10, text width=1.7cm, text centered, minimum size=1.5cm},
        line/.style={draw, -latex', thick}]
        
        \node [gtu state] (temp) {તાપમાન $\Delta T$};
        \node [rectangle, draw, right=of temp, text width=2cm, align=center] (sf) {સ્ટેબિલિટી ફેક્ટર $S$};
        \node [rectangle, draw, above right=of sf, fill=red!10] (unstable) {ઉચ્ચ $S$ અસ્થિર};
        \node [rectangle, draw, below right=of sf, fill=green!10] (stable) {નીચું $S$ સ્થિર};
        
        \path [line] (temp) -- (sf);
        \path [line] (sf) -- (unstable);
        \path [line] (sf) -- (stable);
    \end{tikzpicture}
    \end{center}

    \begin{itemize}
        \item \textbf{વ્યાખ્યા}: સ્ટેબિલિટી ફેક્ટર ($S$) દર્શાવે છે કે કલેક્ટર કરંટ ($I_C$) લીકેજ કરંટ ($I_{CO}$) સાથે કેટલો બદલાય છે (જ્યારે $\beta$ અને $V_{BE}$ અચળ હોય).
        \item \textbf{સૂત્ર}: $S = \frac{dI_C}{dI_{CO}}$
        \item \textbf{આદર્શ મૂલ્ય}: આદર્શ રીતે $S=1$. $S$ નું મૂલ્ય જેટલું નીચું, તેટલી સારી થર્મલ સ્થિરતા.
        \item \textbf{મહત્વ}: તે દર્શાવે છે કે Q-પોઇન્ટ તાપમાનના ફેરફારો માટે કેટલો સંવેદનશીલ છે.
    \end{itemize}

    \mnemonicbox{"LESS એટલે બેહતર - \textbf{L}ower values \textbf{E}nsure \textbf{S}table \textbf{S}ystem તાપમાન પરિવર્તન સામે"}
\end{solutionbox}

\questionmarks{2}{b}{4}
\textbf{કાસ્કેડિંગની ડાયરેક્ટ કપલિંગ ટેકનિક વર્ણવો.}

\begin{solutionbox}
    \begin{center}
    \begin{circuitikz}[american, scale=0.8]
        \draw (0,0) node[npn](Q1){$Q_1$};
        \draw (4,1) node[npn](Q2){$Q_2$};
        
        % Q1 biased
        \draw (Q1.E) to[R, l=$R_{E1}$] (0,-2) node[ground]{};
        \draw (Q1.C) to[R, l=$R_{C1}$] (0,3) node[vcc]{$+V_{CC}$};
        \draw (Q1.B) to[short, -o] (-1,0) node[left]{ઇનપુટ};
        
        % Direct coupling
        \draw (Q1.C) -- (Q2.B);
        
        % Q2 biased
        \draw (Q2.E) to[R, l=$R_{E2}$] (4,-1) node[ground]{};
        \draw (Q2.C) to[R, l=$R_{C2}$] (4,3) node[vcc]{$+V_{CC}$};
        \draw (Q2.C) to[short, -o] (5,1) node[right]{આઉટપુટ};
    \end{circuitikz}
    \end{center}

    \begin{itemize}
        \item \textbf{વ્યાખ્યા}: ડાયરેક્ટ કપલિંગમાં, પ્રથમ તબક્કાનું આઉટપુટ કોઈ પણ કપલિંગ કેપેસિટર કે ટ્રાન્સફોર્મર વિના સીધું આગલા તબક્કાના ઇનપુટ સાથે જોડાય છે.
        \item \textbf{કાર્યપદ્ધતિ}: પ્રથમ તબક્કાનો DC કલેક્ટર વોલ્ટેજ બીજા તબક્કા માટે DC બેઝ બાયસ પૂરો પાડે છે.
        \item \textbf{ફાયદા}:
            \begin{itemize}
                \item ખૂબ જ નીચલા આવર્તનો (DC સહિત) એમ્પ્લિફાય કરી શકે છે.
                \item સરળ સર્કિટ, ઓછા ઘટકો.
                \item IC (Integrated Circuits) માં બનાવવું સરળ છે.
            \end{itemize}
        \item \textbf{ગેરફાયદા}:
            \begin{itemize}
                \item થર્મલ ડ્રિફ્ટ (Q-પોઇન્ટમાં ફેરફાર) સ્ટેજ-દર-સ્ટેજ એમ્પ્લિફાય થાય છે.
                \item DC લેવલ મેચિંગ માટે સાવચેત ડિઝાઇનની જરૂર છે.
            \end{itemize}
    \end{itemize}

    \mnemonicbox{"DIAL DC માટે - \textbf{D}irect \textbf{I}nterconnection \textbf{A}mplifies \textbf{L}ow frequencies કેપેસિટર વગર"}
\end{solutionbox}

\questionmarks{2}{c}{7}
\textbf{બે તબક્કાના RC કપલ્ડ એમ્પ્લીફાયરનો આવર્તન પ્રતિભાવ સમજાવો.}

\begin{solutionbox}
    \begin{center}
    \begin{tikzpicture}
        \begin{semilogxaxis}[
            xlabel={આવર્તન ($f$) Hz},
            ylabel={વોલ્ટેજ ગેઇન ($A_v$) dB},
            xmin=10, xmax=100000,
            ymin=0, ymax=60,
            width=0.8\linewidth,
            height=6cm,
            grid=major
        ]
        \addplot[thick, smooth, blue] coordinates {
            (10, 10) (50, 47) (100, 50) (1000, 50) (10000, 50) (20000, 47) (100000, 20)
        };
        \draw[dashed, red] (axis cs:50,0) -- (axis cs:50,47) node[above left, black]{$f_1$};
        \draw[dashed, red] (axis cs:20000,0) -- (axis cs:20000,47) node[above right, black]{$f_2$};
        \draw[<->] (axis cs:50, 30) -- (axis cs:20000, 30) node[midway, above]{બેન્ડવિડ્થ};
        \node at (axis cs: 1000, 52) {અચળ મહત્તમ ગેઇન};
        \end{semilogxaxis}
    \end{tikzpicture}
    \end{center}

    \textbf{આવર્તન પ્રતિભાવ વિશ્લેષણ:}
    \begin{enumerate}
        \item \textbf{નિમ્ન આવર્તન ક્ષેત્ર ($f < f_1$)}:
            \begin{itemize}
                \item કપલિંગ કેપેસિટર્સ ($C_C$) નો રિએક્ટન્સ ઊંચો હોય છે.
                \item $C_C$ પર વોલ્ટેજ ડ્રોપ થાય છે, સિગ્નલ ઘટે છે.
                \item એમિટર બાયપાસ કેપેસિટર ($C_E$) પણ ગેઇન ઘટાડે છે.
            \end{itemize}
        \item \textbf{મધ્ય આવર્તન ક્ષેત્ર ($f_1 < f < f_2$)}:
            \begin{itemize}
                \item કેપેસિટર્સ શોર્ટ સર્કિટ તરીકે વર્તે છે.
                \item ગેઇન અચળ અને મહત્તમ રહે છે.
            \end{itemize}
        \item \textbf{ઉચ્ચ આવર્તન ક્ષેત્ર ($f > f_2$)}:
            \begin{itemize}
                \item આંતરિક ટ્રાન્ઝિસ્ટર કેપેસિટન્સ નો રિએક્ટન્સ ઘટે છે.
                \item આ સિગ્નલને ગ્રાઉન્ડ કરે છે, ગેઇન ઘટાડે છે.
            \end{itemize}
        \item \textbf{બેન્ડવિડ્થ}: $f_1$ અને $f_2$ વચ્ચેનો વિસ્તાર જ્યાં ગેઇન મહત્તમના 70.7\% (-3dB) હોય છે.
    \end{enumerate}

    \mnemonicbox{"LMH આવર્તન ક્ષેત્રો - \textbf{L}ow માં વધતો ગેઇન, \textbf{M}iddle માં સપાટ ગેઇન, \textbf{H}igh માં ઘટતો ગેઇન"}
\end{solutionbox}

\questionmarks{2}{a}{3}
\textbf{એમ્પ્લીફાયરની બેન્ડવિડ્થ અને ગેઇન-બેન્ડવિડ્થ ઉત્પાદનને સંક્ષિપ્તમાં સમજાવો.}

\begin{solutionbox}
    \textbf{બેન્ડવિડ્થ (BW):}
    \begin{itemize}
        \item આ તે ફ્રીક્વન્સી રેન્જ છે જ્યાં એમ્પ્લીફાયર સંતોષકારક ગેઇન પ્રદાન કરે છે (સામાન્ય રીતે ગેઇન $\ge 70.7\%$ મહત્તમ).
        \item સૂત્ર: $BW = f_2 - f_1$, જ્યાં $f_2$ અપર કટ-ઓફ અને $f_1$ લોઅર કટ-ઓફ ફ્રીક્વન્સી છે.
    \end{itemize}

    \textbf{ગેઇન-બેન્ડવિડ્થ ઉત્પાદન (GBW):}
    \begin{itemize}
        \item આપેલા એમ્પ્લીફાયર માટે, વોલ્ટેજ ગેઇન ($A_v$) અને બેન્ડવિડ્થ ($BW$) નો ગુણાકાર અચળ હોય છે.
        \item $GBW = A_v \times BW = \text{અચળ}$.
        \item \textbf{મહત્વ}: ગેઇન વધારવાથી (દા.ત. કાસ્કેડિંગ દ્વારા) બેન્ડવિડ્થ ઘટે છે, અને ઊલટું.
    \end{itemize}

    \mnemonicbox{"BIG મૂલ્ય - \textbf{B}andwidth અને ગેઇન વચ્ચે \textbf{I}nverse સંબંધ એક \textbf{G}iven અચળ છે"}
\end{solutionbox}

\questionmarks{2}{b}{4}
\textbf{એમ્પ્લીફાયરના ફ્રીક્વન્સી રિસ્પોન્સ પર એમિટર બાયપાસ કેપેસિટર અને કપલિંગ કેપેસિટરની અસરો સમજાવો.}

\begin{solutionbox}
    \begin{center}
    \begin{tabulary}{\linewidth}{|L|L|L|L|}
        \hline
        \textbf{કેપેસિટર} & \textbf{નિમ્ન આવર્તન} & \textbf{મધ્ય આવર્તન} & \textbf{ઉચ્ચ આવર્તન} \\
        \hline
        એમિટર બાયપાસ ($C_E$) & ઉચ્ચ રિએક્ટન્સ, ગેઇન ઘટાડે (ફીડબેક સક્રિય) & શોર્ટ સર્કિટ, મહત્તમ ગેઇન (ફીડબેક બાયપાસ) & શોર્ટ સર્કિટ, કોઈ અસર નહીં \\
        \hline
        કપલિંગ ($C_C$) & ઉચ્ચ રિએક્ટન્સ, સિગ્નલ બ્લોક/ઘટાડે & શોર્ટ સર્કિટ, પૂર્ણ સિગ્નલ & શોર્ટ સર્કિટ, કોઈ અસર નહીં \\
        \hline
    \end{tabulary}
    \end{center}

    \begin{itemize}
        \item \textbf{કપલિંગ કેપેસિટર ($C_C$)}:
            \begin{itemize}
                \item તબક્કાઓ વચ્ચે બાયસ ક્રિયાપ્રતિક્રિયા અટકાવવા DC બ્લોક કરે છે.
                \item નિમ્ન આવર્તન પર, તેનો ઉચ્ચ રિએક્ટન્સ સિગ્નલ નુકશાન કરે છે ($f_1$ નક્કી કરે છે).
            \end{itemize}
        \item \textbf{એમિટર બાયપાસ કેપેસિટર ($C_E$)}:
            \begin{itemize}
                \item $R_E$ ની સમાંતરે જોડાયેલ છે, AC સિગ્નલોને ગ્રાઉન્ડ પર બાયપાસ કરે છે.
                \item નેગેટિવ ફીડબેક અટકાવીને વોલ્ટેજ ગેઇન વધારે છે.
            \end{itemize}
    \end{itemize}

    \mnemonicbox{"CABLE અસર - \textbf{C}apacitors \textbf{A}ct as \textbf{B}arriers \textbf{L}ow ફ્રીક્વન્સી પર"}
\end{solutionbox}

\questionmarks{2}{c}{7}
\textbf{ટ્રાન્સફોર્મર કપલ્ડ એમ્પ્લીફાયર અને RC કપલ્ડ એમ્પ્લીફાયરની સરખામણી કરો.}

\begin{solutionbox}
    \begin{center}
    \begin{tabulary}{\linewidth}{|L|L|L|}
        \hline
        \textbf{પરિમાણ} & \textbf{ટ્રાન્સફોર્મર કપલ્ડ} & \textbf{RC કપલ્ડ} \\
        \hline
        કપલિંગ સાધન & ટ્રાન્સફોર્મર & કેપેસિટર અને રેઝિસ્ટર \\
        \hline
        ઇમ્પીડન્સ મેચિંગ & ઉત્કૃષ્ટ (ટર્ન્સ રેશિયો દ્વારા) & નબળું \\
        \hline
        ફ્રીક્વન્સી રિસ્પોન્સ & નબળો (મર્યાદિત બેન્ડવિડ્થ) & ઉત્કૃષ્ટ (વિશાળ અને સપાટ) \\
        \hline
        કાર્યક્ષમતા & ઉચ્ચ (કલેક્ટર રેઝિસ્ટરમાં પાવર વ્યય નથી) & ઓછી (કલેક્ટર રેઝિસ્ટરમાં પાવર વ્યય) \\
        \hline
        કદ અને વજન & મોટું અને ભારે & કોમ્પેક્ટ અને હલકું \\
        \hline
        કિંમત & મોંઘું & સસ્તું \\
        \hline
    \end{tabulary}
    \end{center}

    \begin{center}
    \begin{minipage}{0.45\linewidth}
        \centering
        \textbf{ટ્રાન્સફોર્મર કપલ્ડ}
        \begin{circuitikz}[american, scale=0.6]
             \draw (0,0) node[npn](Q){};
             \draw (Q.E) node[ground]{};
             \draw (Q.C) to[L] (0,2); % Primary
             \draw (0.5,2) to[L] (0.5,0); % Secondary
             \draw (0.25, 0) -- (0.25, 2); % Core
             \draw (0,2) -- (0,2.5) node[vcc]{};
        \end{circuitikz}
    \end{minipage}
    \begin{minipage}{0.45\linewidth}
        \centering
        \textbf{RC કપલ્ડ}
        \begin{circuitikz}[american, scale=0.6]
             \draw (0,0) node[npn](Q){};
             \draw (Q.E) node[ground]{};
             \draw (Q.C) to[R] (0,2.5) node[vcc]{};
             \draw (Q.C) to[C] (2,1);
        \end{circuitikz}
    \end{minipage}
    \end{center}

    \mnemonicbox{"TREE પરિબળો - \textbf{TR}ansformers \textbf{E}fficiency અને \textbf{E}xcellent ઇમ્પીડન્સ મેચિંગ આપે છે, RC ખર્ચ બચાવે છે"}
\end{solutionbox}

\questionmarks{3}{a}{3}
\textbf{ટ્રાન્ઝિસ્ટર ટ્યુન કરેલ એમ્પ્લીફાયરનું વર્ણન કરો.}

\begin{solutionbox}
    \begin{center}
    \begin{circuitikz}[american]
        \draw (0,0) node[npn](Q){};
        \draw (Q.E) node[ground]{};
        \draw (Q.C) -- (0,1.5);
        % Tank Circuit
        \draw (-1,1.5) -- (1,1.5);
        \draw (-1,1.5) to[L, l=$L$] (-1,3.5);
        \draw (1,1.5) to[C, l=$C$] (1,3.5);
        \draw (-1,3.5) -- (1,3.5);
        \draw (0,3.5) -- (0,4) node[vcc]{$+V_{CC}$};
        
        \draw (Q.B) to[short, -o] (-2,0) node[left]{ઇનપુટ};
        \draw (Q.C) to[C, -o] (2,0) node[right]{આઉટપુટ};
        
        % Biasing resistors (simplified)
        \draw (-1.5,0) to[R, l=$R_B$] (-1.5,3) node[vcc]{};
    \end{circuitikz}
    \end{center}

    \begin{itemize}
        \item \textbf{વ્યાખ્યા}: એક એમ્પ્લીફાયર જે ચોક્કસ સાંકડી આવર્તન બેન્ડને એમ્પ્લિફાય કરવા માટે કલેક્ટર લોડ તરીકે સમાંતર LC સર્કિટ (ટેન્ક સર્કિટ) વાપરે છે.
        \item \textbf{રેઝોનન્સ}: LC સર્કિટ $f_r = \frac{1}{2\pi\sqrt{LC}}$ પર રેઝોનેટ થાય છે.
        \item \textbf{ગેઇન}: રેઝોનન્સ પર, ટેન્ક સર્કિટનો ઇમ્પીડન્સ મહત્તમ હોય છે, તેથી વોલ્ટેજ ગેઇન મહત્તમ મળે છે.
        \item \textbf{ઉપયોગો}: રેડિયો ફ્રીક્વન્સી (RF) અને ઇન્ટરમીડિયેટ ફ્રીક્વન્સી (IF) સ્ટેજમાં.
    \end{itemize}

    \mnemonicbox{"TRIP રેઝોનન્સ માટે - \textbf{T}uned \textbf{R}esonant circuits \textbf{I}mprove \textbf{P}erformance ચોક્કસ આવર્તનો પર"}
\end{solutionbox}

\questionmarks{3}{b}{4}
\textbf{ડાયરેક્ટ કપલ્ડ એમ્પ્લીફાયર સંક્ષિપ્તમાં સમજાવો.}

\begin{solutionbox}
    \begin{center}
    \begin{circuitikz}[american, scale=0.8]
        \draw (0,0) node[npn](Q1){$Q_1$};
        \draw (3,1) node[npn](Q2){$Q_2$};
        \draw (Q1.C) -- (Q2.B); % Direct coupling
        \draw (Q1.E) node[ground]{};
        \draw (Q2.E) to[R] (3,-1) node[ground]{};
        \draw (Q1.C) to[R] (0,3) node[vcc]{$+V_{CC}$};
        \draw (Q2.C) to[R] (3,3) node[vcc]{};
        \draw (Q2.C) to[short, -o] (4,1) node[right]{આઉટપુટ};
    \end{circuitikz}
    \end{center}

    \begin{itemize}
        \item \textbf{વ્યાખ્યા}: એક મલ્ટી-સ્ટેજ એમ્પ્લીફાયર જ્યાં એક તબક્કાનું આઉટપુટ બીજા તબક્કાના ઇનપુટ સાથે રિએક્ટિવ ઘટકો વિના સીધું જોડાયેલું હોય છે.
        \item \textbf{લક્ષણો}:
            \begin{itemize}
                \item \textbf{નિમ્ન આવર્તન પ્રતિસાદ}: ઉત્તમ, DC (0 Hz) સુધી એમ્પ્લિફાય કરી શકે છે.
                \item \textbf{સરળતા}: ઓછા ઘટકોની જરૂર પડે છે (મોટા કેપેસિટર્સ નથી).
                \item \textbf{સમસ્યાઓ}: \keyword{DC ડ્રિફ્ટ} (થર્મલ અસ્થિરતાથી ઓપરેટિંગ પોઇન્ટ શિફ્ટ થાય છે) થાય છે.
            \end{itemize}
        \item \textbf{ઉપયોગ}: લિનિયર ICs, ઓપરેશનલ એમ્પ્લીફાયર્સ.
    \end{itemize}
    \mnemonicbox{"COLD ફાયદા - \textbf{C}ompact design, \textbf{O}utstanding low-frequency response, \textbf{L}ess components, \textbf{D}irect connection"}
\end{solutionbox}

\questionmarks{3}{c}{7}
\textbf{બે પોર્ટ નેટવર્કમાં h પરિમાણોનું મહત્વ વર્ણવો. CE એમ્પ્લીફાયર માટે h-પેરામીટર્સ સર્કિટ દોરો.}

\begin{solutionbox}
    \textbf{ટ્રાન્ઝિસ્ટર h-પેરામીટર મોડેલ (CE રૂપરેખાંકન):}
    \begin{center}
    \begin{circuitikz}[american, scale=0.9]
        % Input loop
        \draw (0,0) to[short, o-] (1,0) to[R, l=$h_{ie}$] (3,0) to[cV, l=$h_{re}V_{ce}$] (5,0) to[short, -o] (5,-2);
        \draw (0,-2) to[short, o-o] (5,-2);
        \node at (0, -1) {$V_{in}$};
        \node at (2.5, -2.5) {એમિટર (E)};
        \node at (0, 0.5) {બેઝ (B)};
        
        % Output loop
        \draw (7,0) to[cI, l=$h_{fe}I_b$] (7,-2); % Current source downward
        \draw (7,0) -- (9,0) to[R, l=$\frac{1}{h_{oe}}$] (9,-2) -- (7,-2);
        \draw (9,0) to[short, -o] (11,0); 
        \draw (9,-2) to[short, -o] (11,-2);
        \node at (11, -1) {$V_{ce}$};
        \node at (11, 0.5) {કલેક્ટર (C)};
        \draw (5,-2) -- (7,-2); % Common ground line
    \end{circuitikz}
    \captionof{figure}{CE માટે h-પેરામીટર સમકક્ષ સર્કિટ}
    \end{center}

    \textbf{h-પેરામીટર્સનું મહત્વ:}
    \begin{itemize}
        \item \textbf{હાઇબ્રિડ પ્રકૃતિ}: મિશ્રિત એકમો ($\Omega$, સીમેન્સ, અપરિમાણ) નો ઉપયોગ કરે છે.
        \item \textbf{સરળ માપન}: ઓપન-સર્કિટ અને શોર્ટ-સર્કિટ સ્થિતિઓ પર આધારિત છે જે માપવા સરળ છે.
        \item \textbf{ચોકસાઈ}: નિમ્ન આવર્તનો પર નાના-સિગ્નલ વિશ્લેષણ માટે સચોટ પરિણામો આપે છે.
        \item \textbf{માનકકરણ}: ઉત્પાદકો આ પરિમાણોનો ઉપયોગ કરીને ટ્રાન્ઝિસ્ટરની લાક્ષણિકતાઓ આપે છે.
    \end{itemize}

    \textbf{CE પરિમાણો:}
    \begin{itemize}
        \item $h_{ie}$ ($h_{11}$): ઇનપુટ ઇમ્પીડન્સ.
        \item $h_{re}$ ($h_{12}$): રિવર્સ વોલ્ટેજ રેશિયો.
        \item $h_{fe}$ ($h_{21}$): ફોરવર્ડ કરંટ ગેઇન.
        \item $h_{oe}$ ($h_{22}$): આઉટપુટ એડમિટન્સ.
    \end{itemize}

    \mnemonicbox{"FINE પેરામીટર્સ - \textbf{F}our \textbf{I}nterconnected \textbf{N}etwork \textbf{E}lements ટ્રાન્ઝિસ્ટરને સંપૂર્ણપણે વ્યાખ્યાયિત કરે છે"}
\end{solutionbox}

\questionmarks{3}{a}{3}
\textbf{ટ્રાન્સફોર્મર કપલ્ડ એમ્પ્લીફાયર અને ડાયરેક્ટ કપલ્ડ એમ્પ્લીફાયરની સરખામણી કરો.}

\begin{solutionbox}
    \begin{center}
    \begin{tabulary}{\linewidth}{|L|L|L|}
        \hline
        \textbf{પરિમાણ} & \textbf{ટ્રાન્સફોર્મર કપલ્ડ} & \textbf{ડાયરેક્ટ કપલ્ડ} \\
        \hline
        DC આઇસોલેશન & સંપૂર્ણ & નથી \\
        \hline
        આવર્તન પ્રતિસાદ & નબળો & ઉત્તમ (DC સુધી) \\
        \hline
        કદ/વજન & મોટું/ભારે & કોમ્પેક્ટ/હલકું \\
        \hline
        ઇમ્પીડન્સ મેચિંગ & ઉત્તમ & નબળું \\
        \hline
        ખર્ચ & ઊંચો & નીચો \\
        \hline
        જટિલતા & મધ્યમ & સરળ \\
        \hline
    \end{tabulary}
    \end{center}
    \mnemonicbox{"TIP પસંદગી માટે - \textbf{T}ransformer for \textbf{I}mpedance matching and \textbf{P}ower transfer, Direct for low frequencies"}
\end{solutionbox}

\questionmarks{3}{b}{4}
\textbf{કોમન એમિટર એમ્પ્લીફાયરનું સર્કિટ ડાયાગ્રામ દોરો અને સમજાવો.}

\begin{solutionbox}
    \begin{center}
    \begin{circuitikz}[american]
        \draw (0,0) node[npn](Q){};
        \draw (Q.E) to[R, l=$R_E$] (0,-2) node[ground]{};
        % Bypass Cap
        \draw (0,-0.5) -- (1.5,-0.5) to[C, l=$C_E$] (1.5,-2) -- (0,-2);
        
        \draw (Q.C) to[R, l=$R_C$] (0,2) -- (0,3) node[vcc]{$+V_{CC}$};
        \draw (Q.C) to[C, l=$C_{out}$, -o] (2,0) node[right]{આઉટપુટ};
        
        % Voltage Divider Check
        \draw (Q.B) -- (-1.5,0);
        \draw (-1.5,0) to[R, l=$R_2$] (-1.5,-2) node[ground]{};
        \draw (-1.5,0) to[R, l=$R_1$] (-1.5,3) node[vcc]{};
        
        \draw (-1.5,0) to[C, l=$C_{in}$, -o] (-3.5,0) node[left]{ઇનપુટ};
    \end{circuitikz}
    \end{center}

    \begin{itemize}
        \item \textbf{સર્કિટ}: વોલ્ટેજ ડિવાઇડર બાયસ ($R_1, R_2$) વાપરે છે. કેપેસિટર્સ $C_{in}$ અને $C_{out}$ DC ને રોકે છે. $C_E$ બાયપાસ કરે છે.
        \item \textbf{ઓપરેશન}: બેઝ પર નાનો AC ઇનપુટ બેઝ કરંટમાં ફેરફાર કરે છે, જે કલેક્ટર પર $\beta$ ગણો એમ્પ્લિફાય થાય છે.
        \item \textbf{ફેઝ શિફ્ટ}: આઉટપુટ ઇનપુટથી \keyword{180$^{\circ}$ ફેઝ શિફ્ટ} પર હોય છે.
        \item \textbf{લક્ષણો}: ઉચ્ચ વોલ્ટેજ ગેઇન, મધ્યમ ઇનપુટ ઇમ્પીડન્સ.
    \end{itemize}
    \mnemonicbox{"GAIN લક્ષણો - \textbf{G}ood \textbf{A}mplification with \textbf{I}nverted output and \textbf{N}otable efficiency"}
\end{solutionbox}

\questionmarks{3}{c}{7}
\textbf{ટ્રાન્ઝિસ્ટર ટુ પોર્ટ નેટવર્ક દોરો અને તેના માટે h-પેરામીટર્સનું વર્ણન કરો. હાઇબ્રિડ પેરામીટર્સના ફાયદા લખો.}

\begin{solutionbox}
    \textbf{ટુ-પોર્ટ નેટવર્ક:}
    \begin{center}
    \begin{tikzpicture}[
        box/.style={draw, rectangle, minimum width=3cm, minimum height=2cm, fill=blue!5},
        arrow/.style={->, thick}
    ]
        \node (net) [box] {ટુ-પોર્ટ નેટવર્ક};
        
        % Input Port
        \draw [arrow] (-3, 0.5) -- node[above] {$I_1$} (net.west |- 0, 0.5);
        \node at (-3.5, 0.5) {+};
        \node at (-3.5, -0.5) {--};
        \node at (-3.5, 0) {$V_1$};
        \draw (-3, -0.5) -- (net.west |- 0, -0.5);
        
        % Output Port
        \draw [arrow] (net.east |- 0, 0.5) -- node[above] {$I_2$} (3, 0.5);
        \node at (3.5, 0.5) {+};
        \node at (3.5, -0.5) {--};
        \node at (3.5, 0) {$V_2$};
        \draw (net.east |- 0, -0.5) -- (3, -0.5);
    \end{tikzpicture}
    \end{center}

    \textbf{h-પેરામીટર સમીકરણો:}
    \[ V_1 = h_{11}I_1 + h_{12}V_2 \]
    \[ I_2 = h_{21}I_1 + h_{22}V_2 \]

    \textbf{ફાયદા:}
    \begin{itemize}
        \item \textbf{સરળ માપન}: દરેક પેરામીટર વ્યક્તિગત રીતે માપી શકાય છે.
        \item \textbf{સચોટ મોડેલ}: ટ્રાન્ઝિસ્ટર વર્તનનું સટીક મોડેલિંગ પ્રદાન કરે છે.
        \item \textbf{સાર્વત્રિક}: બધા ટ્રાન્ઝિસ્ટર રૂપરેખાંકન (BJT, FET) માટે લાગુ.
        \item \textbf{મિશ્રિત એકમો}: જટિલ ક્રિયાપ્રતિક્રિયાઓને લવચીક રીતે વર્ણવે છે.
    \end{itemize}
    \mnemonicbox{"SMART પેરામીટર્સ - \textbf{S}imple \textbf{M}easurement, \textbf{A}ccurate modeling, \textbf{R}eliable, \textbf{T}emperature-stable"}
\end{solutionbox}

\questionmarks{4}{a}{3}
\textbf{ડાર્લિંગ્ટન જોડી અને તેની એપ્લિકેશનો સમજાવો.}

\begin{solutionbox}
    \begin{center}
    \begin{circuitikz}[american]
        \draw (0,0) node[npn](Q1){$Q_1$};
        \draw (2,-0.5) node[npn](Q2){$Q_2$};
        
        \draw (Q1.C) -- (Q2.C); % Collectors connected
        \draw (Q1.C) -- (1,2) node[above]{$C$};
        
        \draw (Q1.E) -- (Q2.B); % Emitter to Base
        
        \draw (Q1.B) -- (-1,0) node[left]{$B$};
        \draw (Q2.E) -- (2,-1.5) node[below]{$E$};
    \end{circuitikz}
    \end{center}

    \begin{itemize}
        \item \textbf{વ્યાખ્યા}: બે ટ્રાન્ઝિસ્ટર એવી રીતે જોડાયેલા હોય છે કે પ્રથમનો એમિટર કરંટ બીજાના બેઝ કરંટ તરીકે કાર્ય કરે છે.
        \item \textbf{કરંટ ગેઇન}: કુલ ગેઇન વ્યક્તિગત ગેઇનનો ગુણાકાર છે ($\beta_{total} \approx \beta_1 \times \beta_2$). ખૂબ જ ઊંચો.
        \item \textbf{ઇનપુટ ઇમ્પીડન્સ}: ખૂબ જ ઊંચું.
        \item \textbf{એપ્લિકેશન્સ}: હાઇ કરંટ ડ્રાઇવર્સ (રિલે, મોટર્સ), ટચ સ્વિચ.
    \end{itemize}
    \mnemonicbox{"HIGH ગેઇન - \textbf{H}ugely \textbf{I}ncreased \textbf{G}ain from \textbf{H}arnessing two transistors"}
\end{solutionbox}

\questionmarks{4}{b}{4}
\textbf{જરૂરી ડાયાગ્રામ સાથે ડાયોડ ક્લેમ્પર સર્કિટનું વર્ણન કરો.}

\begin{solutionbox}
    \textbf{પોઝિટિવ ક્લેમ્પર:}
    \begin{center}
    \begin{circuitikz}[american]
        \draw (0,0) node[left]{ઇનપુટ} to[C, l=$C$] (2,0) -- (4,0) node[right]{આઉટપુટ};
        \draw (3,0) to[R, l=$R$] (3,-2) node[ground]{};
        \draw (2,-2) node[ground]{} to[D*, l=$D$] (2,0); % Points Up
    \end{circuitikz}
    \end{center}

    \begin{itemize}
        \item \textbf{કાર્ય}: વેવફોર્મને તેના આકારને બદલ્યા વિના DC ઘટક ઉમેરીને શિફ્ટ કરે છે (DC રિસ્ટોરર).
        \item \textbf{પોઝિટિવ ક્લેમ્પર}: સિગ્નલને ઉપર શિફ્ટ કરે છે જેથી નેગેટિવ પીક્સ શૂન્ય પર રહે.
        \item \textbf{કાર્યપદ્ધતિ}:
            \begin{itemize}
                \item નેગેટિવ અર્ધ-ચક્ર દરમિયાન, ડાયોડ કન્ડક્ટ કરે છે, કેપેસિટર ચાર્જ થાય છે.
                \item આઉટપુટ વોલ્ટેજ $V_o = V_{in} + V_C$.
            \end{itemize}
    \end{itemize}
    \mnemonicbox{"CAPS અસર - \textbf{C}apacitor \textbf{A}nd diode \textbf{P}air \textbf{S}hifts signal ચોક્કસ DC લેવલથી"}
\end{solutionbox}

\questionmarks{4}{c}{7}
\textbf{OLED નું બાંધકામ, કાર્ય અને એપ્લિકેશન સમજાવો.}

\begin{solutionbox}
    \textbf{OLED બાંધકામ:}
    \begin{center}
    \begin{tikzpicture}
        \draw[thick] (0,0) rectangle (5,3);
        
        \draw[fill=gray!20] (0,0) rectangle (5,0.5); \node at (2.5, 0.25) {સબસ્ટ્રેટ};
        \draw[fill=yellow!20] (0,0.5) rectangle (5,1.0); \node at (2.5, 0.75) {એનોડ (ITO)};
        \draw[fill=blue!10] (0,1.0) rectangle (5,1.5); \node at (2.5, 1.25) {કન્ડક્ટિવ લેયર};
        \draw[fill=red!10] (0,1.5) rectangle (5,2.0); \node at (2.5, 1.75) {એમિસિવ લેયર};
        \draw[fill=gray!40] (0,2.0) rectangle (5,2.5); \node at (2.5, 2.25) {કેથોડ (મેટલ)};
        
        \node[above] at (2.5,3) {OLED નું માળખું};
        \draw[->] (2.5, 2.7) -- (2.5, 2.5);
    \end{tikzpicture}
    \end{center}

    \begin{itemize}
        \item \textbf{બાંધકામ}: બે ઇલેક્ટ્રોડ્સ (એનોડ અને કેથોડ) વચ્ચે સેન્ડવિચ થયેલ ઓર્ગેનિક સેમિકન્ડક્ટર સ્તરો.
        \item \textbf{સ્તરો}:
            \begin{itemize}
                \item \textbf{એમિસિવ લેયર}: ઓર્ગેનિક અણુઓ જે પ્રકાશ ઉત્સર્જિત કરે છે.
                \item \textbf{કન્ડક્ટિવ લેયર}: એનોડમાંથી હોલ્સનું વહન કરે છે.
            \end{itemize}
        \item \textbf{કાર્ય}:
            \begin{itemize}
                \item વોલ્ટેજ લાગુ પડે ત્યારે, ઇલેક્ટ્રોન અને હોલ્સ એમિસિવ લેયરમાં રિકોમ્બાઇન થાય છે અને ફોટોન્સ (પ્રકાશ) મુક્ત કરે છે.
            \end{itemize}
        \item \textbf{એપ્લિકેશન્સ}: પ્રીમિયમ સ્માર્ટફોન, ટીવી, ફ્લેક્સિબલ ડિસ્પ્લે.
    \end{itemize}
    \mnemonicbox{"OLED ફાયદા - \textbf{O}rganic materials, \textbf{L}ightweight design, \textbf{E}fficient operation, \textbf{D}irect emission"}
\end{solutionbox}

\questionmarks{4}{a}{3}
\textbf{LDR પર ટૂંકી નોંધ સમજાવો.}

\begin{solutionbox}
    \begin{center}
    \begin{tikzpicture}
        % Symbol
        \draw[thick] (0,0) rectangle (1,2);
        \draw (0.5,0) -- (0.5,-0.5);
        \draw (0.5,2) -- (0.5,2.5);
        \draw[->, thick, zigzag] (-0.5, 1.5) -- (0.2, 1);
        \draw[->, thick, zigzag] (-0.5, 1) -- (0.2, 0.5);
        \node[below] at (0.5, -0.5) {સિમ્બોલ};
        
        % Structure
        \draw[thick] (3,0) circle (1);
        \draw[decorate, decoration={snake}] (2.5,0) -- (3.5,0); % Zigzag track
        \node[below] at (3, -1.2) {રચના (ટોપ વ્યૂ)};
    \end{tikzpicture}
    \end{center}

    \begin{itemize}
        \item \textbf{પૂરું નામ}: લાઇટ ડિપેન્ડન્ટ રેઝિસ્ટર.
        \item \textbf{સિદ્ધાંત}: ફોટોકંડક્ટિવિટી. પ્રકાશની તીવ્રતા વધે તેમ અવરોધ ઘટે છે.
        \item \textbf{સામગ્રી}: કેડમિયમ સલ્ફાઇડ (CdS).
        \item \textbf{કાર્ય}: અંધારામાં અવરોધ ખૂબ ઊંચો (M$\Omega$) હોય છે. પ્રકાશમાં તે ઘટીને થોડા 100 $\Omega$ થઈ જાય છે.
        \item \textbf{ઉપયોગ}: ઓટોમેટિક સ્ટ્રીટ લાઈટ્સ, કેમેરા.
    \end{itemize}
    \mnemonicbox{"DARK રેઝિસ્ટન્સ વધારે - \textbf{D}ecreasing light \textbf{A}nd \textbf{R}ising darkness \textbf{K}eep resistance high"}
\end{solutionbox}

\questionmarks{4}{b}{4}
\textbf{જરૂરી ડાયાગ્રામ સાથે ડાયોડ ક્લિપર સર્કિટનું વર્ણન કરો.}

\begin{solutionbox}
    \textbf{પોઝિટિવ ક્લિપર:}
    \begin{center}
    \begin{circuitikz}[american]
        \draw (0,0) node[left]{ઇનપુટ} to[D, l=$D$, invert] (2,0) -- (4,0) node[right]{આઉટપુટ};
        \draw (3,0) to[R, l=$R_L$] (3,-2) node[ground]{};
        \draw (0,0) to[open, v=$V_{in}$] (0,-2);
        \draw (0,-2) node[ground]{};
    \end{circuitikz}
    \end{center}

    \begin{itemize}
        \item \textbf{વ્યાખ્યા}: એક સર્કિટ જે ઇનપુટ સિગ્નલ વેવફોર્મના ભાગને કાપી નાખે છે (ક્લિપ કરે છે).
        \item \textbf{પોઝિટિવ ક્લિપર}: પોઝિટિવ અર્ધ ચક્રને દૂર કરે છે.
        \item \textbf{કાર્ય}:
            \begin{itemize}
                \item પોઝિટિવ હાફ સાયકલ દરમિયાન, ડાયોડ રિવર્સ બાયસ્ડ (ઓપન) હોય છે. આઉટપુટશૂન્ય.
                \item નેગેટિવ હાફ સાયકલ દરમિયાન, ડાયોડ ફોરવર્ડ બાયસ્ડ (શોર્ટ) હોય છે. આઉટપુટ ઇનપુટ જેટલું.
            \end{itemize}
    \end{itemize}
    \mnemonicbox{"CLIP તરંગો - \textbf{C}ircuit \textbf{L}imits \textbf{I}nput \textbf{P}eaks ડાયોડ કન્ડક્શન દ્વારા"}
\end{solutionbox}

\questionmarks{4}{c}{7}
\textbf{હાફ વેવ અને ફુલ વેવ વોલ્ટેજ ડબલર સમજાવો.}

\begin{solutionbox}
    \textbf{હાફ-વેવ વોલ્ટેજ ડબલર:}
    \begin{center}
    \begin{circuitikz}[american, scale=0.8]
        \draw (0,0) node[left]{AC} to[C, l=$C_1$] (2,0);
        \draw (2,0) to[D, l=$D_1$, invert] (2,-2) node[ground]{}; % D1 conducts on -ve cycle to charge C1
        \draw (2,0) to[D, l=$D_2$] (4,0) -- (5,0) node[right]{આઉટપુટ};
        \draw (4,0) to[C, l=$C_2$] (4,-2) node[ground]{};
        \draw (0,-2) node[ground]{};
    \end{circuitikz}
    \end{center}

    \begin{itemize}
        \item \textbf{ઓપરેશન}:
            \begin{itemize}
                \item નેગેટિવ સાયકલ: $D_1$ કન્ડક્ટ કરે છે, $C_1$ ચાર્જ થાય છે.
                \item પોઝિટિવ સાયકલ: $D_2$ કન્ડક્ટ કરે છે. $C_2$ વોલ્ટેજ $2V_m$ સુધી ચાર્જ થાય છે.
            \end{itemize}
        \item \textbf{આઉટપુટ}: DC વોલ્ટેજ $\approx 2V_m$.
    \end{itemize}

    \textbf{ફુલ-વેવ વોલ્ટેજ ડબલર:}
    \begin{center}
    \begin{circuitikz}[american, scale=0.8]
        \draw (0,0) node[left]{AC} -- (2,0);
        \draw (2,0) to[D, l=$D_1$] (4,0); % Top branch
        \draw (2,0) to[D, l=$D_2$, invert] (4,-2); % Bottom branch
        \draw (4,0) to[C, l=$C_1$] (4,-1) -- (2,-1) to[short] (0,-1) node[left]{AC Ret};
        \draw (4,-1) to[C, l=$C_2$] (4,-2);
        \draw (4,0) -- (6,0) node[right]{+ Output};
        \draw (4,-2) -- (6,-2) node[right]{- Output};
    \end{circuitikz}
    \end{center}

    \begin{itemize}
        \item \textbf{ઓપરેશન}: એક સાયકલ દરમિયાન એક કેપેસિટર અને બીજી સાયકલ દરમિયાન બીજું કેપેસિટર ચાર્જ થાય છે.
        \item \textbf{ફાયદો}: સ્મૂધ આઉટપુટ (ઓછી રિપલ).
    \end{itemize}
    \mnemonicbox{"CHASE 2V - \textbf{C}apacitors \textbf{H}old \textbf{A}lternating \textbf{S}upply \textbf{E}nergy 2$\times$ વોલ્ટેજ માટે"}
\end{solutionbox}

\questionmarks{5}{a}{3}
\textbf{IC નો ઉપયોગ કરીને +5v પાવર સપ્લાય માટે સર્કિટ ડાયાગ્રામ દોરો અને ટૂંકમાં સમજાવો.}

\begin{solutionbox}
    \begin{center}
    \begin{circuitikz}[american, scale=0.8]
        % Transformer
        \draw (0,0) node[transformer core](T){};
        \node[left] at (T.A1) {AC In};
        
        \draw (T.B1) to[short] (2,1) to[D] (3,2); % Top Left D
        \draw (2,1) to[D, invert] (3,0); % Bot Left D
        \draw (T.B2) to[short] (4,1) to[D] (3,2); % Top Right D
        \draw (4,1) to[D, invert] (3,0); % Bot Right D (Invert means cathode at start?)
        
        % Filter Cap
        \draw (3,2) -- (5,2) to[C, l=$C_1$] (5,0) -- (3,0);
        
        % Regulator 7805
        \draw (6,2) node[draw, rectangle, minimum width=1.5cm, minimum height=1cm, anchor=west] (IC) {7805};
        \draw (5,2) -- (IC.west);
        \draw (5,0) -- (6,0) -- (IC.south); % GND pin
        \draw (IC.east) -- (9,2) node[right]{+5V};
        
        % Output Cap
        \draw (8.5,2) to[C, l=$C_2$] (8.5,0);
        \draw (6,0) -- (9,0) node[right]{GND};
    \end{circuitikz}
    \end{center}

    \begin{itemize}
        \item \textbf{ઘટકો}: ટ્રાન્સફોર્મર, રેક્ટિફાયર, ફિલ્ટર ($C_1$), IC 7805 રેગ્યુલેટર.
        \item \textbf{કાર્ય}: 7805 IC વધારાના વોલ્ટેજને ગરમી તરીકે વિખેરીને સતત 5V જાળવી રાખે છે.
    \end{itemize}
    \mnemonicbox{"FIRM વોલ્ટેજ - \textbf{F}iltered \textbf{I}nput, \textbf{R}egulated by 7805 \textbf{M}akes stable voltage"}
\end{solutionbox}

\questionmarks{5}{b}{4}
\textbf{પાવર સપ્લાયના સંદર્ભમાં લોડ રેગ્યુલેશન અને લાઇન રેગ્યુલેશનની ચર્ચા કરો.}

\begin{solutionbox}
    \begin{center}
    \begin{tikzpicture}
        \begin{axis}[
            width=0.45\linewidth, height=4cm,
            xlabel={લોડ કરંટ ($I_L$)}, ylabel={$V_{out}$},
            title={લોડ રેગ્યુલેશન},
            ymin=0, ymax=6, xmin=0, xmax=10,
            xtick=\empty, ytick=\empty,
            axis lines=left
        ]
        \draw[thick, blue] (axis cs:0,5) -- (axis cs:6,5);
        \draw[thick, blue, dashed] (axis cs:6,5) -- (axis cs:10,4.5) node[midway, below]{Droop};
        \end{axis}
    \end{tikzpicture}
    \hspace{0.5cm}
    \begin{tikzpicture}
        \begin{axis}[
            width=0.45\linewidth, height=4cm,
            xlabel={ઇનપુટ વોલ્ટેજ ($V_{in}$)}, ylabel={$V_{out}$},
            title={લાઇન રેગ્યુલેશન},
            ymin=0, ymax=6, xmin=0, xmax=10,
            xtick=\empty, ytick=\empty,
            axis lines=left
        ]
        \draw[thick, red] (axis cs:2,0) -- (axis cs:2,5);
        \draw[thick, red] (axis cs:2,5) -- (axis cs:10,5);
        \end{axis}
    \end{tikzpicture}
    \end{center}

    \begin{itemize}
        \item \textbf{લોડ રેગ્યુલેશન}: લોડ કરંટમાં ફેરફાર હોવા છતાં આઉટપુટ વોલ્ટેજ સ્થિર રાખવાની ક્ષમતા.
        \item \textbf{લાઇન રેગ્યુલેશન}: ઇનપુટ (મેઇન્સ) વોલ્ટેજમાં ફેરફાર હોવા છતાં આઉટપુટ વોલ્ટેજ સ્થિર રાખવાની ક્ષમતા.
    \end{itemize}
    \mnemonicbox{"LIVER સ્વાસ્થ્ય - \textbf{LI}ne regulation ઇનપુટ ફેરફારો માટે, load regulation બાહ્ય લોડ ફેરફારો માટે"}
\end{solutionbox}

\questionmarks{5}{c}{7}
\textbf{સર્કિટ ડાયાગ્રામ સાથે LM317 નો ઉપયોગ કરીને એડજસ્ટેબલ વોલ્ટેજ રેગ્યુલેટર સમજાવો.}

\begin{solutionbox}
    \begin{center}
    \begin{circuitikz}[american]
        \draw (0,0) node[draw, rectangle, minimum width=2cm, minimum height=1cm] (IC) {LM317};
        \node at (IC.west) [left] {In};
        \node at (IC.east) [right] {Out};
        \node at (IC.south) [below] {Adj};
        
        \draw (-2,0) to[short, o-] (IC.west);
        \node at (-2,0) [left] {$V_{in}$};
        
        \draw (IC.east) -- (2,0) to[short, -o] (4,0) node[right]{$V_{out}$};
        
        % Resistors
        \draw (2,0) to[R, l=$R_1$] (2,-2);
        \draw (IC.south) -- (0,-2) -- (2,-2) -- (2,-3);
        \draw (2,-3) to[vR, l=$R_2$] (2,-5) node[ground]{};
        
        \draw (4,0) to[C, l=$C_{out}$] (4,-2) -- (4,-5) node[ground]{};
        \draw (-1,0) to[C, l=$C_{in}$] (-1,-2) -- (-1,-5) node[ground]{};
    \end{circuitikz}
    \end{center}

    \begin{itemize}
        \item \textbf{વર્ણન}: LM317 એડજસ્ટેબલ પોઝિટિવ વોલ્ટેજ રેગ્યુલેટર છે (1.25V થી 37V).
        \item \textbf{કાર્ય}: તે આઉટપુટ અને એડજસ્ટમેન્ટ ટર્મિનલ વચ્ચે 1.25V સંદર્ભ વોલ્ટેજ જાળવે છે.
        \item \textbf{સૂત્ર}: $V_{out} = 1.25V \left( 1 + \frac{R_2}{R_1} \right)$.
        \item \textbf{ઉપયોગો}: વેરિએબલ બેન્ચ પાવર સપ્લાય.
    \end{itemize}
    \mnemonicbox{"VAIR નિયંત્રણ - \textbf{V}ariable \textbf{A}djustable \textbf{I}ntegrated \textbf{R}egulator વોલ્ટેજ નિયંત્રિત કરે છે"}
\end{solutionbox}

\questionmarks{5}{a}{3}
\textbf{સૌર બેટરી ચાર્જર સર્કિટની કાર્યપદ્ધતિ સમજાવો.}

\begin{solutionbox}
    \begin{center}
    \begin{tikzpicture}[node distance=2cm, auto,
        block/.style={rectangle, draw, fill=blue!10, text width=2cm, text centered, rounded corners, minimum height=3em},
        line/.style={draw, -latex', thick}]
        \node [block, fill=yellow!20] (solar) {સોલર પેનલ};
        \node [block, right=of solar] (ctrl) {ચાર્જ કંટ્રોલર};
        \node [block, right=of ctrl, fill=green!10] (batt) {બેટરી};
        \node [block, right=of batt] (load) {લોડ};
        
        \path [line] (solar) -- (ctrl);
        \path [line] (ctrl) -- (batt);
        \path [line] (batt) -- (load);
    \end{tikzpicture}
    \end{center}

    \begin{itemize}
        \item \textbf{કાર્ય}: સૌર ઊર્જાને ઇલેક્ટ્રિકલ ઊર્જામાં રૂપાંતરિત કરે છે.
        \item \textbf{ચાર્જ કંટ્રોલર}: વોલ્ટેજ અને કરંટનું નિયમન કરે છે. ઓવરચાર્જિંગ અટકાવે છે.
    \end{itemize}
    \mnemonicbox{"SCBL સિસ્ટમ - \textbf{S}olar panel \textbf{C}onverts sunlight, \textbf{B}attery stores, \textbf{L}oad consumes"}
\end{solutionbox}

\questionmarks{5}{b}{4}
\textbf{UPS ની કાર્યપદ્ધતિ સમજાવો.}

\begin{solutionbox}
    \begin{center}
    \begin{tikzpicture}[node distance=2.5cm, auto,
        block/.style={rectangle, draw, text width=2cm, text centered, minimum height=2em},
        line/.style={draw, -latex', thick}]
        
        \node (ac) {AC મેઇન્સ};
        \node [block, right=of ac] (rect) {રેક્ટિફાયર/\\ચાર્જર};
        \node [block, right=of rect] (batt) {બેટરી};
        \node [block, right=of batt] (inv) {ઇન્વર્ટર};
        \node [right=of inv] (load) {લોડ};
        
        \node [block, above=of batt, text width=4cm] (switch) {ટ્રાન્સફર સ્વિચ};
        
        \path [line] (ac) -- (rect);
        \path [line] (rect) -- (batt);
        \path [line] (batt) -- (inv);
        \path [line] (inv) -- (load);
        
        \draw [line, dashed] (ac) |- (switch);
        \draw [line, dashed] (switch) -| (load);
        
    \end{tikzpicture}
    \end{center}

    \begin{itemize}
        \item \textbf{વ્યાખ્યા}: મુખ્ય પાવર નિષ્ફળ જાય ત્યારે બેકઅપ પાવર આપે છે.
        \item \textbf{બેકઅપ મોડ}: ઇન્વર્ટર બેટરીના DC ને AC માં ફેરવી લોડને પાવર આપે છે.
        \item \textbf{ટ્રાન્સફર સ્વિચ}: મેઇન્સ અને ઇન્વર્ટર વચ્ચે સ્વિચ કરે છે.
    \end{itemize}
    \mnemonicbox{"PRIME પાવર - \textbf{P}ower \textbf{R}emains \textbf{I}ntact \textbf{M}ains \textbf{E}lectricity સમસ્યાઓ દરમિયાન"}
\end{solutionbox}

\questionmarks{5}{c}{7}
\textbf{SMPS બ્લોક ડાયાગ્રામ તેના ફાયદા અને ગેરફાયદા સાથે દોરો અને સમજાવો.}

\begin{solutionbox}
    \begin{center}
    \begin{tikzpicture}[node distance=1.5cm, auto,
        block/.style={rectangle, draw, fill=orange!5, text width=1.5cm, text centered, font=\footnotesize, minimum height=3em},
        line/.style={draw, -latex', thick}]
        
        \node (in) {AC In};
        \node [block, right=0.5cm of in] (rect1) {રેક્ટિફાયર};
        \node [block, right=of rect1] (switch) {HF સ્વિચ};
        \node [block, right=of switch] (trans) {ટ્રાન્સફોર્મર};
        \node [block, below of=trans] (rect2) {આઉટપુટ રેક્ટિફાયર};
        \node [block, left=of rect2] (filter) {ફિલ્ટર};
        \node [left=0.5cm of filter] (out) {DC Out};
        
        \node [block, left=of filter] (ctrl) {PWM/Control};
        
        \path [line] (in) -- (rect1);
        \path [line] (rect1) -- (switch);
        \path [line] (switch) -- (trans);
        \path [line] (trans) -- (rect2);
        \path [line] (rect2) -- (filter);
        \path [line] (filter) -- (out);
        
        \draw [line] (filter.south) |- ++(0,-0.5) -| (ctrl.south); % Feedback path
        \path [line] (ctrl) -| (switch);
    \end{tikzpicture}
    \end{center}
    
    \textbf{કાર્યપદ્ધતિ:}
    \begin{enumerate}
        \item \textbf{રેક્ટિફિકેશન}: AC ને DC માં રૂપાંતરિત કરે છે.
        \item \textbf{સ્વિચિંગ}: હાઈ-ફ્રિક્વન્સી ટ્રાન્ઝિસ્ટર DC ને પલ્સમાં કાપે છે.
        \item \textbf{આઉટપુટ}: સ્મૂધ DC મેળવવા માટે ફરીથી રેક્ટિફાય અને ફિલ્ટર કરવામાં આવે છે.
        \item \textbf{રેગ્યુલેશન}: PWM ફીડબેક દ્વારા વોલ્ટેજ જાળવવામાં આવે છે.
    \end{enumerate}

    \textbf{ફાયદા:}
    \begin{itemize}
        \item \textbf{ઉચ્ચ કાર્યક્ષમતા}: 70-90\%.
        \item \textbf{કોમ્પેક્ટ}: હલકું અને નાનું કદ.
    \end{itemize}

    \textbf{ગેરફાયદા:}
    \begin{itemize}
        \item \textbf{નોઇઝ}: સ્વિચિંગ નોઇઝ જનરેટ કરે છે.
        \item \textbf{જટિલતા}: ડિઝાઇન જટિલ છે.
    \end{itemize}

    \mnemonicbox{"FISH ફેક્ટર્સ - \textbf{F}requency switching, \textbf{I}solation, \textbf{S}mall size, \textbf{H}igh efficiency SMPS ના ફાયદા છે"}
\end{solutionbox}

\end{document}
