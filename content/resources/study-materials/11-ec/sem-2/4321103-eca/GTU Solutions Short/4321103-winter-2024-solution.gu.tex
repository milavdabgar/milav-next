\documentclass{article}

% content/resources/templates/preamble.tex
\usepackage[margin=0.6in]{geometry}
\author{Milav Dabgar}
\usepackage{amsmath,amssymb,amsthm}
\usepackage{booktabs}
\usepackage{multirow}
\usepackage{xcolor}
\usepackage{tcolorbox}
\tcbuselibrary{breakable,skins}
\usepackage[colorlinks=true,linkcolor=blue]{hyperref}
\usepackage{titlesec}
\usepackage{enumitem}
\usepackage{tikz}
\usepackage{pgfplots}
\usepackage{circuitikz}
\usepackage[version=4]{mhchem}
\usepackage{longtable}
\usepackage{array}
\usepackage{float}
\usepackage{caption}
\usepackage{listings}

\lstset{
  basicstyle=\small\ttfamily,
  breaklines=true,
  breakatwhitespace=false,
  postbreak=\mbox{\textcolor{red}{$\hookrightarrow$}\space},
  float=false,
  numbers=left,
  numberstyle=\tiny\color{gray},
  numbersep=10pt,
  xleftmargin=2em,
  keywordstyle=\color{blue},
  commentstyle=\color{green!60!black},
  stringstyle=\color{purple},
  backgroundcolor=\color{gray!5},
  showstringspaces=false,
  tabsize=2,
  captionpos=b,
  keepspaces=true,
  columns=flexible
}

\pgfplotsset{compat=1.18}
\usetikzlibrary{shapes,arrows,positioning,calc,patterns,decorations.pathmorphing,decorations.markings,arrows.meta}

% Color scheme
\definecolor{headcolor}{RGB}{0,102,204}
\definecolor{keycolor}{RGB}{220,20,60}
\definecolor{solutioncolor}{RGB}{34,139,34}
\definecolor{mnemoniccolor}{RGB}{148,0,211}
\definecolor{codecolor}{RGB}{0,0,100}

% Spacing
\setlength{\parskip}{3pt}
\setlist[itemize]{nosep}
\setlist[enumerate]{nosep}

% Title formatting
\titleformat{\section}{\Large\bfseries\color{headcolor}}{\thesection}{1em}{}
\titleformat{\subsection}{\large\bfseries\color{headcolor}}{\thesubsection}{1em}{}

% Pandoc tightlist compatibility
\providecommand{\tightlist}{%
  \setlength{\itemsep}{0pt}\setlength{\parskip}{0pt}}

% Pandoc longtable compatibility
\newcounter{none}
\def\thenone{}


% content/resources/templates/gujarati-boxes.tex
\usepackage{fontspec}
\usepackage{polyglossia}

% Set Gujarati as main language (document is primarily in Gujarati)
% Note: gloss-gujarati.ldf doesn't exist in polyglossia, but it will use hyphenation patterns
\setdefaultlanguage{gujarati}
\setotherlanguage{english}

% Configure Gujarati font properly
% Use Language=Default to prevent polyglossia from trying to add language-specific features
% that don't exist for Gujarati, which causes "empty feature" warnings
\newfontfamily\gujaratifont[Script=Gujarati,AutoFakeBold=2.5,AutoFakeSlant=0.3]{Noto Sans Gujarati}
\setmainfont[Script=Gujarati,AutoFakeBold=2.5,AutoFakeSlant=0.3]{Noto Sans Gujarati}
% Use Noto Sans Gujarati for monospace to support Gujarati in text
\setmonofont[Scale=0.9]{Noto Sans Gujarati}

% Configure English to use the same font
\newfontfamily\englishfont[Script=Gujarati,AutoFakeBold=2.5,AutoFakeSlant=0.3]{Noto Sans Gujarati}

% Translations for polyglossia
\gappto\captionsgujarati{
  \renewcommand{\tablename}{કોષ્ટક}
  \renewcommand{\figurename}{આકૃતિ}
}

% Helper for TikZ nodes to ensure Gujarati font
\newcommand{\gu}[1]{{\gujaratifont #1}}

% Custom environments
\newtcolorbox{solutionbox}{
    breakable,
    enhanced,
    colback=solutioncolor!5!white,
    colframe=solutioncolor!75!black,
    fonttitle=\bfseries,
    title=જવાબ
}

\newtcolorbox{solutionboxnobreak}{
 colback=solutioncolor!5!white,
 colframe=solutioncolor!75!black,
 fonttitle=\bfseries,
 title=જવાબ
}

\newtcolorbox{keyformula}{
 breakable,
 enhanced,
 colback=keycolor!5!white,
 colframe=keycolor!75!black,
 fonttitle=\bfseries,
 title=રાસાયણિક સમીકરણ/સૂત્ર
}

\newtcolorbox{mnemonicbox}{
 breakable,
 enhanced,
 colback=mnemoniccolor!5!white,
 colframe=mnemoniccolor!75!black,
 fonttitle=\bfseries,
 title=મેમરી ટ્રીક
}


% Custom commands for GTU solutions
% This file defines semantic commands for consistent formatting

% Question command with automatic formatting
\newcommand{\question}[2]{%
  \section*{Question #1}%
  \textbf{#2}%
}

% OR question variant
\newcommand{\questionor}[2]{%
  \section*{Question #1 OR}%
  \textbf{#2}%
}

% Proper table environment with caption
\newenvironment{answertable}[1]{%
  \begin{table}[htbp]
  \centering
  \caption{#1}
}{%
  \end{table}
}

% Proper figure environment for diagrams
\newenvironment{answerdiagram}[1]{%
  \begin{figure}[htbp]
  \centering
  \caption{#1}
}{%
  \end{figure}
}

% Semantic markup for key terms
\newcommand{\keyword}[1]{\textbf{#1}}
\newcommand{\code}[1]{\texttt{#1}}
\newcommand{\classname}[1]{\texttt{#1}}
\newcommand{\methodname}[1]{\texttt{#1}}

% Proper quotation marks
\newcommand{\mnemonic}[1]{``#1''}


\title{ઇલેક્ટ્રોનિક સર્કિટ્સ એન્ડ એપ્લિકેશન્સ (4321103) - વિન્ટર 2024 સોલ્યુશન}
\date{January 13, 2024}

\begin{document}
\maketitle

\questionmarks{1}{a}{3}
\textbf{CE રૂપરેખાંકન માટે એમ્પલીફાયર પરિમાણો Ai, Ri અને Ro સમજાવો.}

\begin{solutionbox}
\textbf{CE એમ્પલીફાયર પરિમાણો:}

\begin{center}
\captionof{table}{CE એમ્પલીફાયર પરિમાણો}
\begin{tabular}{|l|l|l|}
\hline
\textbf{પરિમાણ} & \textbf{વ્યાખ્યા} & \textbf{મૂલ્ય} \\ \hline
કરંટ ગેઇન ($A_i$) & આઉટપુટ કરંટનો ઇનપુટ કરંટ સાથેનો ગુણોત્તર & ઊંચો (20-500) \\ \hline
ઇનપુટ રેઝિસ્ટન્સ ($R_i$) & ઇનપુટ પર કરંટ પ્રવાહનો વિરોધ & મધ્યમ (1-2 k$\Omega$) \\ \hline
આઉટપુટ રેઝિસ્ટન્સ ($R_o$) & આઉટપુટ પર કરંટ પ્રવાહનો વિરોધ & ઊંચો (40-50 k$\Omega$) \\ \hline
\end{tabular}
\end{center}

\begin{center}
\includefigure{figures/tex-diagrams/pdf/4321103-winter-2024-solution-q1a.gu.pdf}
\end{center}
\end{solutionbox}
\begin{mnemonicbox}
CAR - CE has Current gain high, Average input resistance, and Robust output resistance.
\end{mnemonicbox}

\questionmarks{1}{b}{4}
\textbf{હીટ સિંક પર ટૂંકી નોંધ લખો.}

\begin{solutionbox}
\textbf{હીટ સિંક:}
એવું ઉપકરણ જે ઇલેક્ટ્રોનિક ઘટકોમાંથી ગરમી શોષે છે અને વિખેરે છે.

\begin{center}
\includefigure{figures/tex-diagrams/pdf/4321103-winter-2024-solution-q1b.gu.pdf}
\end{center}

\textbf{હીટ સિંકના પ્રકારો:}
\begin{itemize}
    \item \textbf{પેસિવ હીટ સિંક}: કુદરતી convection પર આધાર રાખે છે.
    \item \textbf{એક્ટિવ હીટ સિંક}: ફોર્સ્ડ એર convection માટે ફેન વાપરે છે.
    \item \textbf{લિક્વિડ-કૂલ્ડ હીટ સિંક}: વધુ સારા heat transfer માટે પ્રવાહી વાપરે છે.
\end{itemize}

\textbf{મુખ્ય કાર્યો:}
\begin{itemize}
    \item \textbf{થર્મલ કન્ડક્શન}: ઘટકોમાંથી ગરમી દૂર ખેંચે છે.
    \item \textbf{થર્મલ કન્વેક્શન}: ગરમી આસપાસની હવામાં ટ્રાન્સફર કરે છે.
    \item \textbf{સરફેસ એરિયા}: પાંખો વધુ સારા કૂલિંગ માટે સપાટી ક્ષેત્રફળ વધારે છે.
\end{itemize}
\end{solutionbox}
\begin{mnemonicbox}
CRAFT - Cooling through Radiation And Fins for Transistors.
\end{mnemonicbox}

\questionmarks{1}{c}{7}
\textbf{થર્મલ રનઅવે અને થર્મલ સ્ટેબિલિટીનું વર્ણન કરો. ટ્રાન્ઝિસ્ટરમાં થર્મલ રન અવે કેવી રીતે દૂર કરી શકાય?}

\begin{solutionbox}
\textbf{થર્મલ રનઅવે:}
સ્વ-મજબૂત કરતી પ્રક્રિયા જ્યાં વધતા તાપમાનને કારણે વધુ કરંટ પ્રવાહ થાય છે, જે આગળ તાપમાન વધારે છે.

\textbf{થર્મલ સ્ટેબિલિટી:}
તાપમાન ફેરફારો હોવા છતાં ટ્રાન્ઝિસ્ટર સર્કિટની સ્થિર કામગીરી જાળવવાની ક્ષમતા.

\begin{center}
\includefigure{figures/tex-diagrams/pdf/4321103-winter-2024-solution-q1c.gu.pdf}
\end{center}

\textbf{થર્મલ રનઅવે દૂર કરવાની પદ્ધતિઓ:}
\begin{itemize}
    \item \textbf{હીટ સિંક}: વધારાની ગરમીને શોષે અને વિખેરે છે.
    \item \textbf{નેગેટિવ ફીડબેક}: સ્થિરતા માટે એમિટર રેઝિસ્ટર વાપરવો.
    \item \textbf{બાયસ સ્ટેબિલાઇઝેશન}: વોલ્ટેજ ડિવાઇડર બાયસિંગ સર્કિટ.
    \item \textbf{તાપમાન ક્ષતિપૂર્તિ}: ડાયોડ અથવા થર્મિસ્ટર્સનો ઉપયોગ કરવો.
\end{itemize}

\textbf{મુખ્ય મુદ્દાઓ:}
\begin{itemize}
    \item $I_C = I_{CBO}(1+\beta) + \beta I_B$.
    \item $I_{CBO}$ બમણો થાય છે: દર 10$^\circ$C તાપમાન વધારા માટે.
    \item \textbf{સ્ટેબિલિટી ફેક્ટર S}: ઓછું S એટલે વધુ સારી સ્થિરતા.
\end{itemize}
\end{solutionbox}
\begin{mnemonicbox}
RENT - Reduce heat with sinks, Emitter resistors stabilize, Negative feedback helps, Temperature compensation.
\end{mnemonicbox}

\questionmarks{1}{c}{7}
\textbf{બાયસિંગ પદ્ધતિઓના પ્રકારો લખો. વોલ્ટેજ વિભાજક બાયસિંગ પદ્ધતિને વિગતોમાં સમજાવો.}

\begin{solutionbox}
\textbf{બાયસિંગ પદ્ધતિઓના પ્રકારો:}
\begin{itemize}
    \item ફિક્સ્ડ બાયસ
    \item કલેક્ટર-ટુ-બેઝ બાયસ
    \item વોલ્ટેજ ડિવાઇડર બાયસ
    \item એમિટર બાયસ
    \item કલેક્ટર ફીડબેક બાયસ
\end{itemize}

\textbf{વોલ્ટેજ ડિવાઇડર બાયસ સર્કિટ:}

\begin{center}
\includefigure{figures/tex-diagrams/pdf/4321103-winter-2024-solution-q1c-2.gu.pdf}
\end{center}

\textbf{કાર્યપ્રણાલી:}
\begin{itemize}
    \item \textbf{R1 અને R2}: બેઝ વોલ્ટેજ પ્રદાન કરતા વોલ્ટેજ ડિવાઇડર બનાવે છે.
    \item \textbf{RE}: સ્થિરતા અને નેગેટિવ ફીડબેક પ્રદાન કરે છે.
    \item \textbf{સ્ટેબલ બાયસ પોઇન્ટ}: તાપમાન અને $\beta$ ફેરફારોથી ઓછો પ્રભાવિત.
\end{itemize}

\textbf{ફાયદાઓ:}
\begin{itemize}
    \item \textbf{ઉત્તમ સ્થિરતા}: તાપમાન ફેરફારોથી ઓછો પ્રભાવિત.
    \item \textbf{$\beta$ થી સ્વતંત્ર}: બાયસ પોઇન્ટ ટ્રાન્ઝિસ્ટર ગેઇનથી ખૂબ પ્રભાવિત નથી.
    \item \textbf{વ્યાપકપણે ઉપયોગમાં}: એમ્પ્લીફાયર માટે સૌથી સામાન્ય બાયસિંગ પદ્ધતિ.
\end{itemize}
\end{solutionbox}
\begin{mnemonicbox}
DIVE - Divider biasing Is Very Effective for stability.
\end{mnemonicbox}

\questionmarks{2}{a}{3}
\textbf{સ્ટેબિલિટી પરિબળનું લક્ષણો સમજાવો.}

\begin{solutionbox}
\textbf{સ્ટેબિલિટી ફેક્ટર (S):}
બાયસિંગ સર્કિટ તાપમાન ફેરફારો સાથે સ્થિર કામગીરી કેટલી સારી રીતે જાળવે છે તેનું માપ.
ગાણિતિક વ્યાખ્યા: $S = \frac{\Delta I_C}{\Delta I_{CBO}}$

\begin{center}
\captionof{table}{વિવિધ બાયસ સર્કિટ્સ માટે સ્ટેબિલિટી ફેક્ટર્સ}
\begin{tabular}{|l|l|l|}
\hline
\textbf{બાયસિંગ મેથડ} & \textbf{સ્ટેબિલિટી ફેક્ટર} & \textbf{લેવલ} \\ \hline
ફિક્સ્ડ બાયસ & $S = 1+\beta$ & ખરાબ \\ \hline
કલેક્ટર-ટુ-બેઝ & $S = \frac{\beta}{1+\beta}$ & બેહતર \\ \hline
વોલ્ટેજ ડિવાઇડર & $S \approx 1$ & ઉત્તમ \\ \hline
\end{tabular}
\end{center}

\textbf{મુખ્ય લક્ષણો:}
\begin{itemize}
    \item \textbf{ઓછો S મૂલ્ય}: વધુ સારી સ્થિરતા દર્શાવે છે (આદર્શ $S=1$).
    \item \textbf{તાપમાન પ્રતિરોધ}: તાપમાન ફેરફારોથી રક્ષણની માત્રા માપે છે.
\end{itemize}
\end{solutionbox}
\begin{mnemonicbox}
SOS - Stability Of circuit Shows in its S-factor.
\end{mnemonicbox}

\questionmarks{2}{b}{4}
\textbf{કાસ્કેડીંગની ડાયરેક્ટ કપ્લીંગ ટેકનિકનું વર્ણન કરો.}

\begin{solutionbox}
\textbf{ડાયરેક્ટ કપ્લીંગ:}
કપલિંગ કેપેસિટર્સ વિના સ્ટેજ જોડવું, એક સ્ટેજના કલેક્ટરને સીધો આગલા સ્ટેજના બેઝ સાથે જોડવો.

\begin{center}
\includefigure{figures/tex-diagrams/pdf/4321103-winter-2024-solution-q2b.gu.pdf}
\end{center}

\textbf{મુખ્ય લક્ષણો:}
\begin{itemize}
    \item \textbf{કોઈ કપલિંગ ઘટકો નહીં}: સીધો ઇલેક્ટ્રિકલ કનેક્શન.
    \item \textbf{પૂર્ણ ફ્રીક્વન્સી રિસ્પોન્સ}: સારી લો-ફ્રીક્વન્સી પરફોર્મન્સ.
    \item \textbf{DC લેવલ શિફ્ટિંગ}: સ્ટેજ વચ્ચે જરૂરી છે.
\end{itemize}
\end{solutionbox}
\begin{mnemonicbox}
DIRECT - DC signals Immediately REach Connecting Transistors.
\end{mnemonicbox}

\questionmarks{2}{c}{7}
\textbf{બે તબક્કાનાં આર સી કપલ્ડ એમ્પલીફાયરનો આવર્તન પ્રતિભાવ સમજાવો.}

\begin{solutionbox}
\textbf{RC કપલ્ડ એમ્પ્લીફાયર ફ્રીક્વન્સી રિસ્પોન્સ:}

\begin{center}
\includefigure{figures/tex-diagrams/pdf/4321103-winter-2024-solution-q2c.gu.pdf}
\end{center}

\begin{center}
\captionof{table}{ફ્રીક્વન્સી રીજન}
\begin{tabular}{|l|l|l|}
\hline
\textbf{રીજન} & \textbf{ફ્રીક્વન્સી રેન્જ} & \textbf{કારણ} \\ \hline
લો & 20Hz-500Hz & કપલિંગ કેપેસિટર્સ \\ \hline
મિડ & 500Hz-20kHz & કોઈ નહીં (મહત્તમ ગેઇન) \\ \hline
હાઇ & >20kHz & ટ્રાન્ઝિસ્ટર કેપેસિટન્સ \\ \hline
\end{tabular}
\end{center}

\textbf{બે-સ્ટેજની અસર:}
\begin{itemize}
    \item \textbf{બેન્ડવિડ્થ}: સિંગલ સ્ટેજ કરતાં સાંકડી.
    \item \textbf{ગેઇન}: સિંગલ સ્ટેજના લગભગ વર્ગ જેટલો ($A_1 \times A_2$).
\end{itemize}
\end{solutionbox}
\begin{mnemonicbox}
LMH - Low frequencies by coupling caps, Mid frequencies flat, High frequencies by transistor caps.
\end{mnemonicbox}

\questionmarks{2}{a}{3}
\textbf{એમ્પ્લીફાયરની બેન્ડવિડ્થ અને ગેઇન-બેન્ડવિડ્થ ઉત્પાદનને સંક્ષિપ્તમાં સમજાવો.}

\begin{solutionbox}
\textbf{બેન્ડવિડ્થ (BW):}
ફ્રીક્વન્સીઓની રેન્જ જ્યાં એમ્પ્લીફાયર ગેઇન મહત્તમ ગેઇનના ઓછામાં ઓછા 70.7\% છે.
$BW = f_2 - f_1$

\textbf{ગેઇન-બેન્ડવિડ્થ પ્રોડક્ટ (GBP):}
વોલ્ટેજ ગેઇન અને બેન્ડવિડ્થનો ગુણાકાર, આપેલા એમ્પલીફાયર માટે સ્થિર છે.

\begin{center}
\includefigure{figures/tex-diagrams/pdf/4321103-winter-2024-solution-q2a.gu.pdf}
\end{center}
\end{solutionbox}
\begin{mnemonicbox}
BAND - Bandwidth And gain Never Drop together.
\end{mnemonicbox}

\questionmarks{2}{b}{4}
\textbf{એમ્પલીફાયરના ફ્રીક્વન્સી રિસ્પોન્સ પર એમિટર બાયપાસ કેપેસિટર અને કપલિંગ કેપેસિટરની અસરો સમજાવો.}

\begin{solutionbox}
\textbf{કેપેસિટર અસરો:}

\begin{center}
\captionof{table}{કેપેસિટર અસરો}
\begin{tabular}{|l|p{4cm}|p{7cm}|}
\hline
\textbf{કેપેસિટર} & \textbf{કાર્ય} & \textbf{ફ્રીક્વન્સી રિસ્પોન્સ પર અસર} \\ \hline
કપલિંગ ($C_C$) & DC બ્લોક કરે, AC પસાર કરે & લો-ફ્રીક્વન્સી રિસ્પોન્સ મર્યાદિત કરે. \\ \hline
બાયપાસ ($C_E$) & એમિટર રેઝિસ્ટરને બાયપાસ કરે & મિડ અને હાઇ ફ્રીક્વન્સી પર ગેઇન વધારે. \\ \hline
\end{tabular}
\end{center}
\end{solutionbox}
\begin{mnemonicbox}
CELL - Coupling affects Extremely Low frequencies, bypass affects Low to high.
\end{mnemonicbox}

\questionmarks{2}{c}{7}
\textbf{ટ્રાન્સફોર્મર કપલ્ડ એમ્પલીફાયર અને આરસી કપલ્ડ એમ્પલીફાયરની સરખામણી કરો}

\begin{solutionbox}
\textbf{સરખામણી:}

\begin{center}
\captionof{table}{ટ્રાન્સફોર્મર કપલ્ડ vs RC કપલ્ડ}
\begin{tabular}{|l|l|l|}
\hline
\textbf{લક્ષણ} & \textbf{ટ્રાન્સફોર્મર કપલ્ડ} & \textbf{RC કપલ્ડ} \\ \hline
કપલિંગ ઘટક & ટ્રાન્સફોર્મર & કેપેસિટર અને રેઝિસ્ટર \\ \hline
કાર્યક્ષમતા & ઊંચી (90\%) & મધ્યમ (20-30\%) \\ \hline
કદ અને વજન & મોટું અને ભારે & કોમ્પેક્ટ અને હલકું \\ \hline
ખર્ચ & મોંઘું & સસ્તું \\ \hline
ફ્રીક્વન્સી રિસ્પોન્સ & ખરાબ (મર્યાદિત) & સારો (વિશાળ) \\ \hline
ઇમ્પીડન્સ મેચિંગ & ઉત્તમ & ખરાબ \\ \hline
\end{tabular}
\end{center}
\end{solutionbox}
\begin{mnemonicbox}
TRIP - Transformers are Robust for Impedance matching, Problematic for bandwidth.
\end{mnemonicbox}

\questionmarks{3}{a}{3}
\textbf{ટ્યુન કરેલ એમ્પલીફાયર તરીકે ઉપયોગમાં લેવાતા ટ્રાન્ઝિસ્ટરનું વર્ણન કરો.}

\begin{solutionbox}
\textbf{ટ્યુન્ડ એમ્પલીફાયર:}
એમ્પલીફાયર જે સાંકડા ફ્રીક્વન્સી બેન્ડમાં સિગ્નલ્સને પસંદગીપૂર્વક એમ્પલિફાય કરે છે. કલેક્ટર લોડ તરીકે LC ટેંક સર્કિટ વાપરે છે.

\begin{center}
\includefigure{figures/tex-diagrams/pdf/4321103-winter-2024-solution-q3a.gu.pdf}
\end{center}

\textbf{મુખ્ય ઘટકો:}
\begin{itemize}
    \item \textbf{LC ટેંક સર્કિટ}: રેઝોનન્ટ ફ્રીક્વન્સી નક્કી કરે છે $f_0 = \frac{1}{2\pi\sqrt{LC}}$.
    \item \textbf{ટ્રાન્ઝિસ્ટર}: એમ્પલીફિકેશન પૂરું પાડે છે.
\end{itemize}

\textbf{એપ્લિકેશન્સ}: રેડિયો રિસીવર્સ, TV ટ્યુનર્સ.
\end{solutionbox}
\begin{mnemonicbox}
TUNE - Transistors Using Narrowband Elements for frequency selection.
\end{mnemonicbox}

\questionmarks{3}{b}{4}
\textbf{ડાયરેક્ટ કપલ્ડ એમ્પલીફાયરને સંક્ષિપ્તમાં સમજાવો.}

\begin{solutionbox}
\textbf{ડાયરેક્ટ કપલ્ડ એમ્પલીફાયર:}
(આકૃતિ માટે Q2(b) જુઓ)
મલ્ટિપલ સ્ટેજ એમ્પલીફાયર જ્યાં કપલિંગ કેપેસિટર્સ અથવા ટ્રાન્સફોર્મર્સ વગર સ્ટેજ સીધા જોડાયેલા છે.

\textbf{મુખ્ય લક્ષણો:}
\begin{itemize}
    \item \textbf{DC એમ્પલીફિકેશન}: DC થી ઊંચી ફ્રીક્વન્સી સુધી એમ્પલિફાય કરી શકે છે.
    \item \textbf{સરળ રચના}: ઓછા ઘટકો, ઓછો ખર્ચ.
    \item \textbf{થર્મલ ડ્રિફ્ટ}: તાપમાન સાથે Q-point બદલાય છે જે મુખ્ય ગેરફાયદો છે.
\end{itemize}
\end{solutionbox}
\begin{mnemonicbox}
DCAP - Direct Coupled Amplifier Passes all frequencies including DC.
\end{mnemonicbox}

\questionmarks{3}{c}{7}
\textbf{બે પોર્ટ નેટવર્કમાં h પરિમાણોનું મહત્વ વર્ણવો. CE એમ્પલીફાયર માટે h-પેરામીટર્સ સર્કિટ દોરો.}

\begin{solutionbox}
\textbf{h-પેરામીટર્સનું મહત્વ:}
\begin{itemize}
    \item \textbf{સંપૂર્ણ રિપ્રેઝન્ટેશન}: એમ્પલીફાયર વર્તનને સંપૂર્ણ રીતે વર્ણવે છે.
    \item \textbf{સરળ માપન}: શોર્ટ અને ઓપન સર્કિટ કન્ડિશન્સમાં સરળતાથી માપી શકાય છે.
    \item \textbf{સ્ટાન્ડર્ડ}: ટ્રાન્ઝિસ્ટર ડેટાશીટમાં h-પેરામીટર્સ આપવામાં આવે છે.
\end{itemize}

\textbf{CE એમ્પલીફાયર માટે h-પેરામીટર સર્કિટ:}

\begin{center}
\includefigure{figures/tex-diagrams/pdf/4321103-winter-2024-solution-q3c.gu.pdf}
\end{center}

\textbf{પેરામીટર્સ:}
\begin{enumerate}
    \item $h_{ie}$: ઇનપુટ ઇમ્પીડન્સ.
    \item $h_{re}$: રિવર્સ વોલ્ટેજ રેશિયો.
    \item $h_{fe}$: ફોરવર્ડ કરંટ ગેઇન ($\beta$).
    \item $h_{oe}$: આઉટપુટ એડમિટન્સ.
\end{enumerate}
\end{solutionbox}
\begin{mnemonicbox}
HIRE - h-parameters Include Resistance and current gain Effectively.
\end{mnemonicbox}

\questionmarks{3}{a}{3}
\textbf{ટ્રાન્સફોર્મર કપલ્ડ એમ્પલીફાયર અને ડાયરેક્ટ કપલ્ડ એમ્પલીફાયરની સરખામણી કરો.}

\begin{solutionbox}
\textbf{સરખામણી:}

\begin{center}
\captionof{table}{સરખામણી}
\begin{tabular}{|l|l|l|}
\hline
\textbf{લક્ષણ} & \textbf{ટ્રાન્સફોર્મર કપલ્ડ} & \textbf{ડાયરેક્ટ કપલ્ડ} \\ \hline
ફ્રીક્વન્સી રિસ્પોન્સ & લો ફ્રીક્વન્સી પર મર્યાદિત & ઉત્તમ (DC થી ઊંચી) \\ \hline
ખર્ચ & ઊંચો & નિમ્ન \\ \hline
કદ & મોટું & કોમ્પેક્ટ \\ \hline
ઇમ્પીડન્સ મેચિંગ & ઉત્તમ & ખરાબ \\ \hline
DC આઇસોલેશન & હા & ના \\ \hline
\end{tabular}
\end{center}
\end{solutionbox}
\begin{mnemonicbox}
TDC - Transformers provide DC isolation, Direct provides Complete frequency range.
\end{mnemonicbox}

\questionmarks{3}{b}{4}
\textbf{કોમન એમિટર એમ્પલીફાયરનું સર્કિટ ડાયાગ્રામ દોરો અને સમજાવો.}

\begin{solutionbox}
\textbf{કોમન એમિટર (CE) એમ્પલીફાયર:}

\begin{center}
\includefigure{figures/tex-diagrams/pdf/4321103-winter-2024-solution-q3b.gu.pdf}
\end{center}

\textbf{સમજૂતી:}
\begin{itemize}
    \item \textbf{ઇનપુટ}: બેઝ-એમિટર વચ્ચે.
    \item \textbf{આઉટપુટ}: કલેક્ટર-એમિટર વચ્ચે.
    \item \textbf{ફેઝ શિફ્ટ}: 180$^\circ$ (આઉટપુટ ઇન્વર્ટેડ છે).
    \item \textbf{ગેઇન}: ઊંચો વોલ્ટેજ અને કરંટ ગેઇન.
\end{itemize}
\end{solutionbox}
\begin{mnemonicbox}
CEA - Common Emitter Amplifies with signal inversion.
\end{mnemonicbox}

\questionmarks{3}{c}{7}
\textbf{ટ્રાન્ઝિસ્ટર ટુ પોર્ટ નેટવર્ક દોરો અને તેના માટે h-પેરામીટર્સનું વર્ણન કરો. હાઇબ્રિડ પરિમાણોના ફાયદા લખો.}

\begin{solutionbox}
\textbf{ટુ-પોર્ટ નેટવર્ક:} (જુઓ Q3(c) ઉપર).

\begin{center}
\includefigure{figures/tex-diagrams/pdf/4321103-winter-2024-solution-q3c-2.gu.pdf}
\end{center}

\textbf{હાઇબ્રિડ પેરામીટર્સના ફાયદા:}
\begin{itemize}
    \item માપવામાં સરળ છે.
    \item ઓડિયો ફ્રીક્વન્સી પર રિયલ નંબર્સ છે.
    \item સચોટ સર્કિટ એનાલિસિસ માટે ઉપયોગી.
\end{itemize}
\end{solutionbox}
\begin{mnemonicbox}
HAEM - Hybrid parameters Are Easily Measured and mathematically simple.
\end{mnemonicbox}

\questionmarks{4}{a}{3}
\textbf{ડાર્લિંગ્ટન જોડી અને તેની એપ્લિકેશનો સમજાવો.}

\begin{solutionbox}
\textbf{ડાર્લિંગ્ટન પેર:}
બે ટ્રાન્ઝિસ્ટર્સની કોન્ફિગરેશન જ્યાં પહેલાનો એમિટર બીજાના બેઝ સાથે જોડાયેલો છે.

\begin{center}
\includefigure{figures/tex-diagrams/pdf/4321103-winter-2024-solution-q4a.gu.pdf}
\end{center}

\textbf{લક્ષણો:}
\begin{itemize}
    \item ખૂબ ઊંચો કરંટ ગેઇન ($\beta \approx \beta_1 \beta_2$).
    \item ઊંચો ઇનપુટ ઇમ્પીડન્સ.
\end{itemize}
\textbf{એપ્લિકેશન્સ}: પાવર એમ્પલીફાયર્સ, મોટર ડ્રાઇવર્સ, ટચ સેન્સર્સ.
\end{solutionbox}
\begin{mnemonicbox}
DISH - Darlington Integrates Stages for High current gain.
\end{mnemonicbox}

\questionmarks{4}{b}{4}
\textbf{જરૂરી ડાયાગ્રામ સાથે ડાયોડ ક્લેમ્પર સર્કિટનું વર્ણન કરો.}

\begin{solutionbox}
\textbf{ડાયોડ ક્લેમ્પર:}
વેવફોર્મના આકારને બદલ્યા વગર તેના DC લેવલને શિફ્ટ કરે છે.

\begin{center}
\includefigure{figures/tex-diagrams/pdf/4321103-winter-2024-solution-q4b.gu.pdf}
\end{center}

\textbf{કાર્ય}: કેપેસિટર પીક વોલ્ટેજ પર ચાર્જ થાય છે અને બેટરી તરીકે વર્તે છે.
\textbf{ઉપયોગ}: TV રીસીવર્સ (DC પુનઃસ્થાપના).
\end{solutionbox}
\begin{mnemonicbox}
CLAMP - Circuit Levels Are Modified Precisely.
\end{mnemonicbox}

\questionmarks{4}{c}{7}
\textbf{OLED નાં બાંધકામ, કાર્ય અને એપ્લિકેશન સમજાવો.}

\begin{solutionbox}
\textbf{OLED (ઓર્ગેનિક લાઇટ એમિટિંગ ડાયોડ):}

\begin{center}
\includefigure{figures/tex-diagrams/pdf/4321103-winter-2024-solution-q4c.gu.pdf}
\end{center}

\textbf{કાર્ય સિદ્ધાંત}:
\begin{itemize}
    \item કેથોડ અને એનોડમાંથી ચાર્જ કેરિયર્સ ઇન્જેક્ટ થાય છે.
    \item એમિસિવ લેયરમાં રિકોમ્બિનેશન થાય છે.
    \item લાઈટ ઉત્સર્જન થાય છે (Electroluminescence).
\end{itemize}

\textbf{એપ્લિકેશન્સ}: વળાંકવાળી સ્ક્રીન, ફ્લેક્સિબલ ડિસ્પ્લે, સ્માર્ટફોન્સ.
\end{solutionbox}
\begin{mnemonicbox}
OLED - Organic Layers Emit Directly.
\end{mnemonicbox}

\questionmarks{4}{a}{3}
\textbf{LDR પર ટૂંકી નોંધ સમજાવો.}

\begin{solutionbox}
\textbf{LDR (લાઇટ ડિપેન્ડન્ટ રેઝિસ્ટર):}
ફોટોરેઝિસ્ટર (CdS) જેનો રેઝિસ્ટન્સ પ્રકાશ પડવાથી ઘટે છે.

\begin{center}
\includefigure{figures/tex-diagrams/pdf/4321103-winter-2024-solution-q4a-2.gu.pdf}
\end{center}

\textbf{સિદ્ધાંત}: પ્રકાશ $\rightarrow$ વધુ ચાર્જ કેરિયર્સ $\rightarrow$ ઓછો રેઝિસ્ટન્સ (Dark $M\Omega$, Light $k\Omega$).
\textbf{ઉપયોગ}: સ્ટ્રીટ લાઈટ્સ, કેમેરા.
\end{solutionbox}
\begin{mnemonicbox}
LORD - Light Oppositely Reduces the Device's resistance.
\end{mnemonicbox}

\questionmarks{4}{b}{4}
\textbf{જરૂરી ડાયાગ્રામ સાથે ડાયોડ ક્લેમ્પર સર્કિટનું વર્ણન કરો.}

\begin{solutionbox}
\textbf{ડાયોડ ક્લિપર સર્કિટ:}
સિગ્નલના અમુક ભાગને દૂર કરે છે.

\begin{center}
\includefigure{figures/tex-diagrams/pdf/4321103-winter-2024-solution-q4b-2.gu.pdf}
\end{center}

\textbf{પ્રકારો}: પોઝિટિવ ક્લિપર, નેગેટિવ ક્લિપર.
\textbf{ઉપયોગ}: વેવ શેપિંગ, પ્રોટેક્શન.
\end{solutionbox}
\begin{mnemonicbox}
CLIP - Circuit Limits Input Peaks.
\end{mnemonicbox}

\questionmarks{4}{c}{7}
\textbf{હાફ વેવ અને ફુલ વેવ વોલ્ટેજ ડબલર સમજાવો.}

\begin{solutionbox}
\textbf{વોલ્ટેજ ડબલર:}
DC આઉટપુટ વોલ્ટેજ પીક ઇનપુટ વોલ્ટેજ કરતાં બમણો ($2V_m$) ઉત્પન્ન કરે છે.

\textbf{હાફ-વેવ ડબલર:}
\begin{center}
\includefigure{figures/tex-diagrams/pdf/4321103-winter-2024-solution-q4c-2.gu.pdf}
\end{center}

\textbf{ફુલ-વેવ ડબલર:}
\begin{center}
\includefigure{figures/tex-diagrams/pdf/4321103-winter-2024-solution-q4c-3.gu.pdf}
\end{center}

\textbf{Explanation}: Capacitors charge in alternate cycles and their voltages sum up.
\end{solutionbox}
\begin{mnemonicbox}
DOUBLE - Diodes Organize Unidirectional Boost.
\end{mnemonicbox}

\questionmarks{5}{a}{3}
\textbf{IC નો ઉપયોગ કરીને +5 v પાવર સપ્લાય માટે સર્કિટ ડાયાગ્રામ દોરો}

\begin{solutionbox}
\textbf{+5V પાવર સપ્લાય (7805):}

\begin{center}
\includefigure{figures/tex-diagrams/pdf/4321103-winter-2024-solution-q5a.gu.pdf}
\end{center}
\end{solutionbox}
\begin{mnemonicbox}
FIVE - Fixed IC Voltage Efficiently provided.
\end{mnemonicbox}

\questionmarks{5}{b}{4}
\textbf{પાવર સપ્લાયના સંદર્ભમાં લોડ રેગ્યુલેશન અને લાઇન રેગ્યુલેશનની ચર્ચા કરો.}

\begin{solutionbox}
\textbf{રેગ્યુલેશન:} આઉટપુટ વોલ્ટેજ અચળ રાખવાની ક્ષમતા.

\begin{enumerate}
    \item \textbf{લાઇન રેગ્યુલેશન}: ઇનપુટ વોલ્ટેજ ($V_{in}$) બદલાય ત્યારે આઉટપુટ અચળ રહેવું.
    \item \textbf{લોડ રેગ્યુલેશન}: લોડ કરંટ ($I_L$) બદલાય ત્યારે આઉટપુટ અચળ રહેવું.
\end{enumerate}
\end{solutionbox}
\begin{mnemonicbox}
LINE LOAD - Line Is Normal-input Efficiency, LOAD is Output Adjustment Defense.
\end{mnemonicbox}

\questionmarks{5}{c}{7}
\textbf{સર્કિટ ડાયાગ્રામ સાથે LM317 નો ઉપયોગ કરીને એડજસ્ટેબલ વોલ્ટેજ રેગ્યુલેટર સમજાવો.}

\begin{solutionbox}
\textbf{LM317 એડજસ્ટેબલ રેગ્યુલેટર:}

\begin{center}
\includefigure{figures/tex-diagrams/pdf/4321103-winter-2024-solution-q5c.gu.pdf}
\end{center}

\textbf{સૂત્ર}: $V_{out} = 1.25(1 + \frac{R_2}{R_1})$.
\end{solutionbox}
\begin{mnemonicbox}
VARY - Voltage Adjustable Regulator Yields custom outputs.
\end{mnemonicbox}

\questionmarks{5}{a}{3}
\textbf{IC નો ઉપયોગ કરીને -15 v પાવર સપ્લાય માટે સર્કિટ ડાયાગ્રામ દોરો}

\begin{solutionbox}
\textbf{-15V પાવર સપ્લાય (7915):}

\begin{center}
\includefigure{figures/tex-diagrams/pdf/4321103-winter-2024-solution-q5a-2.gu.pdf}
\end{center}
\textit{નોંધ: 79xx સીરીઝ નેગેટિવ વોલ્ટેજ માટે છે.}
\end{solutionbox}
\begin{mnemonicbox}
NINE - Negative IC Needs Efficient filtering.
\end{mnemonicbox}

\questionmarks{5}{b}{4}
\textbf{UPS નું કાર્ય સમજાવો.}

\begin{solutionbox}
\textbf{UPS (અનઇન્ટરપ્ટિબલ પાવર સપ્લાય):}
ઇમરજન્સી પાવર પૂરો પાડે છે.

\begin{center}
\includefigure{figures/tex-diagrams/pdf/4321103-winter-2024-solution-q5b.gu.pdf}
\end{center}
\end{solutionbox}
\begin{mnemonicbox}
UPBEAT - Uninterruptible Power Backup.
\end{mnemonicbox}

\questionmarks{5}{c}{7}
\textbf{SMPS બ્લોક ડાયાગ્રામ દોરો અને તેના ફાયદા અને ગેરફાયદા સાથે સમજાવો.}

\begin{solutionbox}
\textbf{SMPS (સ્વિચ મોડ પાવર સપ્લાય):}

\begin{center}
\includefigure{figures/tex-diagrams/pdf/4321103-winter-2024-solution-q5c-2.gu.pdf}
\end{center}

\textbf{ફાયદા}: ઊંચી કાર્યક્ષમતા (>80\%), નાનું કદ, ઓછું વજન.
\textbf{ગેરફાયદા}: વધુ નોઇઝ (EMI), જટિલ સર્કિટ.
\end{solutionbox}
\begin{mnemonicbox}
SWITCH - Smaller Weight, Improved Thermal efficiency.
\end{mnemonicbox}

\end{document}
