% Tags: 4321102, code, convertor, gray, binary, logic
\begin{tikzpicture}[
    node distance=1.5cm,
    auto,
    block/.style={rectangle, draw, minimum width=1.5cm, minimum height=1cm, align=center}
]
    % Inputs
    \node (G3) at (0, 3) {G3};
    \node (G2) at (0, 2) {G2};
    \node (G1) at (0, 1) {G1};
    \node (G0) at (0, 0) {G0};
    
    % Outputs
    \node (B3) at (5, 3) {B3};
    \node (B2) at (5, 2) {B2};
    \node (B1) at (5, 1) {B1};
    \node (B0) at (5, 0) {B0};
    
    % XOR Gates
    \node[xor gate US, draw, logic gate inputs=nn] (xor1) at (3, 2) {};
    \node[xor gate US, draw, logic gate inputs=nn] (xor2) at (3, 1) {};
    \node[xor gate US, draw, logic gate inputs=nn] (xor3) at (3, 0) {};
    
    % Connections
    \draw (G3) -- (B3);
    
    \draw (G3) -- ++(1,0) -- ++(0,-0.5) -- ++(1.5,-0.5) |- (xor1.input 1); % Tricky wiring for G to B
    % Wait, rule: B3=G3. B2 = B3 xor G2. B1 = B2 xor G1.
    % So output B3 feeds into xor for B2.
    
    \draw (B3) ++(-1,0) -- ++(0,-0.5) |- (xor1.input 1); % From B3 line
    \draw (G2) -- (xor1.input 2);
    \draw (xor1.output) -- (B2);
    
    \draw (xor1.output) ++(0.5,0) -- ++(0,-0.5) |- (xor2.input 1);
    \draw (G1) -- (xor2.input 2);
    \draw (xor2.output) -- (B1);
    
    \draw (xor2.output) ++(0.5,0) -- ++(0,-0.5) |- (xor3.input 1);
    \draw (G0) -- (xor3.input 2);
    \draw (xor3.output) -- (B0);

\end{tikzpicture}
