% Tags: 4321102, code, gray, binary, converter, 4-bit
\begin{circuitikz}
    % Inputs G3..G0
    \foreach \i in {3,2,1,0} {
        \draw (0, \i*2) node (G\i) {G\textsubscript{\i}};
    }
    
    % XOR gates diagonal chain
    \draw (3, 5) node[xor port, rotate=-90] (xor3) {}; 
    \draw (3, 3) node[xor port, rotate=-90] (xor2) {};
    \draw (3, 1) node[xor port, rotate=-90] (xor1) {};
    
    % B3 = G3 directly
    \draw (G3) -- (5, 6) node[right] {B\textsubscript{3}};
    
    % Connections
    % Diagram structure: B3 comes from G3. B2 = B3 XOR G2.
    % This requires feedback from B output to next XOR input.
    % Simplified drawing for schematic layout:
    \draw (2,6) -- (xor3.in 1); % B3 feed
    \draw (G2) -- (xor3.in 2);
    \draw (xor3.out) -- (5, 4) node[right] {B\textsubscript{2}};
    
    \draw (2,4) -- (xor2.in 1); % B2 feed
    \draw (G1) -- (xor2.in 2);
    \draw (xor2.out) -- (5, 2) node[right] {B\textsubscript{1}};
    
    \draw (2,2) -- (xor1.in 1); % B1 feed
    \draw (G0) -- (xor1.in 2);
    \draw (xor1.out) -- (5, 0) node[right] {B\textsubscript{0}};
\end{circuitikz}
