\documentclass{article}

% content/resources/templates/preamble.tex
\usepackage[margin=0.6in]{geometry}
\author{Milav Dabgar}
\usepackage{amsmath,amssymb,amsthm}
\usepackage{booktabs}
\usepackage{multirow}
\usepackage{xcolor}
\usepackage{tcolorbox}
\tcbuselibrary{breakable,skins}
\usepackage[colorlinks=true,linkcolor=blue]{hyperref}
\usepackage{titlesec}
\usepackage{enumitem}
\usepackage{tikz}
\usepackage{pgfplots}
\usepackage{circuitikz}
\usepackage[version=4]{mhchem}
\usepackage{longtable}
\usepackage{array}
\usepackage{float}
\usepackage{caption}
\usepackage{listings}

\lstset{
  basicstyle=\small\ttfamily,
  breaklines=true,
  breakatwhitespace=false,
  postbreak=\mbox{\textcolor{red}{$\hookrightarrow$}\space},
  float=false,
  numbers=left,
  numberstyle=\tiny\color{gray},
  numbersep=10pt,
  xleftmargin=2em,
  keywordstyle=\color{blue},
  commentstyle=\color{green!60!black},
  stringstyle=\color{purple},
  backgroundcolor=\color{gray!5},
  showstringspaces=false,
  tabsize=2,
  captionpos=b,
  keepspaces=true,
  columns=flexible
}

\pgfplotsset{compat=1.18}
\usetikzlibrary{shapes,arrows,positioning,calc,patterns,decorations.pathmorphing,decorations.markings,arrows.meta}

% Color scheme
\definecolor{headcolor}{RGB}{0,102,204}
\definecolor{keycolor}{RGB}{220,20,60}
\definecolor{solutioncolor}{RGB}{34,139,34}
\definecolor{mnemoniccolor}{RGB}{148,0,211}
\definecolor{codecolor}{RGB}{0,0,100}

% Spacing
\setlength{\parskip}{3pt}
\setlist[itemize]{nosep}
\setlist[enumerate]{nosep}

% Title formatting
\titleformat{\section}{\Large\bfseries\color{headcolor}}{\thesection}{1em}{}
\titleformat{\subsection}{\large\bfseries\color{headcolor}}{\thesubsection}{1em}{}

% Pandoc tightlist compatibility
\providecommand{\tightlist}{%
  \setlength{\itemsep}{0pt}\setlength{\parskip}{0pt}}

% Pandoc longtable compatibility
\newcounter{none}
\def\thenone{}


% content/resources/templates/gujarati-boxes.tex
\usepackage{fontspec}
\usepackage{polyglossia}

% Set Gujarati as main language (document is primarily in Gujarati)
% Note: gloss-gujarati.ldf doesn't exist in polyglossia, but it will use hyphenation patterns
\setdefaultlanguage{gujarati}
\setotherlanguage{english}

% Configure Gujarati font properly
% Use Language=Default to prevent polyglossia from trying to add language-specific features
% that don't exist for Gujarati, which causes "empty feature" warnings
\newfontfamily\gujaratifont[Script=Gujarati,AutoFakeBold=2.5,AutoFakeSlant=0.3]{Noto Sans Gujarati}
\setmainfont[Script=Gujarati,AutoFakeBold=2.5,AutoFakeSlant=0.3]{Noto Sans Gujarati}
% Use Noto Sans Gujarati for monospace to support Gujarati in text
\setmonofont[Scale=0.9]{Noto Sans Gujarati}

% Configure English to use the same font
\newfontfamily\englishfont[Script=Gujarati,AutoFakeBold=2.5,AutoFakeSlant=0.3]{Noto Sans Gujarati}

% Translations for polyglossia
\gappto\captionsgujarati{
  \renewcommand{\tablename}{કોષ્ટક}
  \renewcommand{\figurename}{આકૃતિ}
}

% Helper for TikZ nodes to ensure Gujarati font
\newcommand{\gu}[1]{{\gujaratifont #1}}

% Custom environments
\newtcolorbox{solutionbox}{
    breakable,
    enhanced,
    colback=solutioncolor!5!white,
    colframe=solutioncolor!75!black,
    fonttitle=\bfseries,
    title=જવાબ
}

\newtcolorbox{solutionboxnobreak}{
 colback=solutioncolor!5!white,
 colframe=solutioncolor!75!black,
 fonttitle=\bfseries,
 title=જવાબ
}

\newtcolorbox{keyformula}{
 breakable,
 enhanced,
 colback=keycolor!5!white,
 colframe=keycolor!75!black,
 fonttitle=\bfseries,
 title=રાસાયણિક સમીકરણ/સૂત્ર
}

\newtcolorbox{mnemonicbox}{
 breakable,
 enhanced,
 colback=mnemoniccolor!5!white,
 colframe=mnemoniccolor!75!black,
 fonttitle=\bfseries,
 title=મેમરી ટ્રીક
}


% Custom commands for GTU solutions
% This file defines semantic commands for consistent formatting

% Question command with automatic formatting
\newcommand{\question}[2]{%
  \section*{Question #1}%
  \textbf{#2}%
}

% OR question variant
\newcommand{\questionor}[2]{%
  \section*{Question #1 OR}%
  \textbf{#2}%
}

% Proper table environment with caption
\newenvironment{answertable}[1]{%
  \begin{table}[htbp]
  \centering
  \caption{#1}
}{%
  \end{table}
}

% Proper figure environment for diagrams
\newenvironment{answerdiagram}[1]{%
  \begin{figure}[htbp]
  \centering
  \caption{#1}
}{%
  \end{figure}
}

% Semantic markup for key terms
\newcommand{\keyword}[1]{\textbf{#1}}
\newcommand{\code}[1]{\texttt{#1}}
\newcommand{\classname}[1]{\texttt{#1}}
\newcommand{\methodname}[1]{\texttt{#1}}

% Proper quotation marks
\newcommand{\mnemonic}[1]{``#1''}


\title{Digital Electronics (4321102) - Summer 2023 Solution (Gujarati)}
\date{August 07, 2023}

\begin{document}
\maketitle

\questionmarks{1(a)}{3}{બુલિયન એલ્જીબ્રા માટેના ડે-મોર્ગનના નિયમ સમજાવો}


\begin{solutionbox}
ડે-મોર્ગનના નિયમમાં બે કાયદા છે જે AND, OR અને NOT ક્રિયાઓ વચ્ચેના સંબંધને દર્શાવે છે:

\textbf{કાયદો 1}: સરવાળાના પૂરકની કિંમત પૂરકના ગુણાકાર બરાબર હોય છે
\[ \overline{A + B} = \overline{A} \cdot \overline{B} \]

\textbf{કાયદો 2}: ગુણાકારના પૂરકની કિંમત પૂરકના સરવાળા બરાબર હોય છે
\[ \overline{A \cdot B} = \overline{A} + \overline{B} \]

\captionof{table}{ડે-મોર્ગનના નિયમની ચકાસણી}
\begin{center}
\begin{tabulary}{\linewidth}{|C|C|C|C|C|C|C|}
\hline
A & B & A+B & $\overline{A+B}$ & $\overline{A}$ & $\overline{B}$ & $\overline{A}\cdot\overline{B}$ \\
\hline
0 & 0 & 0 & 1 & 1 & 1 & 1 \\
\hline
0 & 1 & 1 & 0 & 1 & 0 & 0 \\
\hline
1 & 0 & 1 & 0 & 0 & 1 & 0 \\
\hline
1 & 1 & 1 & 0 & 0 & 0 & 0 \\
\hline
\end{tabulary}
\end{center}
\end{solutionbox}
\mnemonicbox{"OR પર NOT થાય AND, AND પર NOT થાય OR"}

\questionmarks{1(બ)}{4}{નીચેના ડેસિમલ નંબરને બાયનરી અને ઓકટલ નંબરમાં ફેરવો (i) 215 (ii) 59}


\begin{solutionbox}
\textbf{બાયનરી રૂપાંતર:}

\textbf{215 માટે:}
\begin{itemize}
\item 2 વડે ભાગ કરો: $215/2 = 107$ શેષ 1
\item $107/2 = 53$ શેષ 1
\item $53/2 = 26$ શેષ 1
\item $26/2 = 13$ શેષ 0
\item $13/2 = 6$ શેષ 1
\item $6/2 = 3$ શેષ 0
\item $3/2 = 1$ શેષ 1
\item $1/2 = 0$ શેષ 1
\item તેથી, $(215)_{10} = (11010111)_2$
\end{itemize}

\textbf{59 માટે:}
\begin{itemize}
\item 2 વડે ભાગ કરો: $59/2 = 29$ શેષ 1
\item $29/2 = 14$ શેષ 1
\item $14/2 = 7$ શેષ 0
\item $7/2 = 3$ શેષ 1
\item $3/2 = 1$ શેષ 1
\item $1/2 = 0$ શેષ 1
\item તેથી, $(59)_{10} = (111011)_2$
\end{itemize}

\textbf{ઓકટલ રૂપાંતર:}

\textbf{215 માટે:}
\begin{itemize}
\item 8 વડે ભાગ કરો: $215/8 = 26$ શેષ 7
\item $26/8 = 3$ શેષ 2
\item $3/8 = 0$ શેષ 3
\item તેથી, $(215)_{10} = (327)_8$
\end{itemize}

\textbf{59 માટે:}
\begin{itemize}
\item 8 વડે ભાગ કરો: $59/8 = 7$ શેષ 3
\item $7/8 = 0$ શેષ 7
\item તેથી, $(59)_{10} = (73)_8$
\end{itemize}

\captionof{table}{સંખ્યા રૂપાંતર સારાંશ}
\begin{center}
\begin{tabulary}{\linewidth}{|L|L|L|}
\hline
ડેસિમલ & બાયનરી & ઓકટલ \\
\hline
215 & 11010111 & 327 \\
\hline
59 & 111011 & 73 \\
\hline
\end{tabulary}
\end{center}
\end{solutionbox}
\mnemonicbox{"આધાર વડે ભાગો, શેષ નીચેથી ઉપર વાંચો"}

\questionmarks{1(ક)(I)}{2}{ડેસિમલ, બાયનરી, ઓકટલ અને હેક્ઝાડેસિમલ નંબર સિસ્ટમનો બેઝ લખો}


\begin{solutionbox}
\captionof{table}{સંખ્યા પદ્ધતિના આધાર}
\begin{center}
\begin{tabulary}{\linewidth}{|L|L|}
\hline
સંખ્યા પદ્ધતિ & આધાર \\
\hline
ડેસિમલ & 10 \\
\hline
બાયનરી & 2 \\
\hline
ઓકટલ & 8 \\
\hline
હેક્ઝાડેસિમલ & 16 \\
\hline
\end{tabulary}
\end{center}
\end{solutionbox}
\mnemonicbox{"ડે-બા-ઓ-હે: 10-2-8-16"}

\questionmarks{1(ક)(II)}{2}{$(147)_{10} = (\_\_\_\_\_\_\_\_\_\_\_\_)_2 = (\_\_\_\_\_\_\_\_\_\_\_\_\_\_)_{16}$}


\begin{solutionbox}
\textbf{ડેસિમલથી બાયનરી રૂપાંતર:}
\begin{itemize}
\item $147/2 = 73$ શેષ 1
\item $73/2 = 36$ શેષ 1
\item $36/2 = 18$ શેષ 0
\item $18/2 = 9$ શેષ 0
\item $9/2 = 4$ શેષ 1
\item $4/2 = 2$ શેષ 0
\item $2/2 = 1$ શેષ 0
\item $1/2 = 0$ શેષ 1
\item તેથી, $(147)_{10} = (10010011)_2$
\end{itemize}

\textbf{ડેસિમલથી હેક્ઝાડેસિમલ રૂપાંતર:}
\begin{itemize}
\item બાયનરી અંકોને 4ના સમૂહમાં વિભાજિત કરો: 1001 0011
\item દરેક સમૂહને હેક્સમાં રૂપાંતરિત કરો: $1001 = 9$, $0011 = 3$
\item તેથી, $(147)_{10} = (93)_{16}$
\end{itemize}

\captionof{table}{રૂપાંતર પરિણામ}
\begin{center}
\begin{tabulary}{\linewidth}{|L|L|L|}
\hline
ડેસિમલ & બાયનરી & હેક્ઝાડેસિમલ \\
\hline
147 & 10010011 & 93 \\
\hline
\end{tabulary}
\end{center}
\end{solutionbox}
\mnemonicbox{"હેક્સ માટે જમણેથી 4ના સમૂહમાં વિભાજિત કરો"}

\questionmarks{1(ક)(III)}{3}{નીચેના બાયનરી કોડનું ગ્રે કોડમાં રૂપાંતર કરો (i) 1011 (ii) 1110}


\begin{solutionbox}
\textbf{બાયનરીથી ગ્રે કોડ રૂપાંતર પ્રક્રિયા:}
\begin{enumerate}
\item ગ્રે કોડનો MSB (ડાબી બાજુનો બિટ) બાયનરી કોડના MSB જેવો જ હોય છે
\item ગ્રે કોડના અન્ય બિટ્સ બાયનરી કોડના આસપાસના બિટ્સને XOR કરીને મેળવવામાં આવે છે
\end{enumerate}

\textbf{1011 માટે:}
\begin{itemize}
\item ગ્રે કોડનો MSB = બાયનરી કોડનો MSB = 1
\item બીજો બિટ = $1 \oplus 0 = 1$
\item ત્રીજો બિટ = $0 \oplus 1 = 1$
\item ચોથો બિટ = $1 \oplus 1 = 0$
\item તેથી, $(1011)_2 = (1110)_{gray}$
\end{itemize}

\textbf{1110 માટે:}
\begin{itemize}
\item ગ્રે કોડનો MSB = બાયનરી કોડનો MSB = 1
\item બીજો બિટ = $1 \oplus 1 = 0$
\item ત્રીજો બિટ = $1 \oplus 1 = 0$
\item ચોથો બિટ = $1 \oplus 0 = 1$
\item તેથી, $(1110)_2 = (1001)_{gray}$
\end{itemize}

\captionof{table}{બાયનરીથી ગ્રે કોડ રૂપાંતર}
\begin{center}
\begin{tabulary}{\linewidth}{|L|L|L|}
\hline
બાયનરી & રૂપાંતર પદ્ધતિ & ગ્રે કોડ \\
\hline
1011 & 1, $1\oplus0=1$, $0\oplus1=1$, $1\oplus1=0$ & 1110 \\
\hline
1110 & 1, $1\oplus1=0$, $1\oplus1=0$, $1\oplus0=1$ & 1001 \\
\hline
\end{tabulary}
\end{center}
\end{solutionbox}
\mnemonicbox{"પહેલો રાખો, બાકીના XOR કરો"}

\questionmarks{1 [OR] (I)}{2}{BCD અને ASCII નું ફૂલફોર્મ લખો}


\begin{solutionbox}
\captionof{table}{BCD અને ASCII નું પૂર્ણ નામ}
\begin{center}
\begin{tabulary}{\linewidth}{|L|L|}
\hline
સંક્ષિપ્ત રૂપ & પૂર્ણ નામ \\
\hline
BCD & Binary Coded Decimal \\
\hline
ASCII & American Standard Code for Information Interchange \\
\hline
\end{tabulary}
\end{center}
\end{solutionbox}
\mnemonicbox{"બાયનરી કોડેડ ડેસિમલ, અમેરિકન સ્ટાન્ડર્ડ કોડ ફોર ઇન્ફોર્મેશન ઇન્ટરચેન્જ"}

\questionmarks{1 [OR] (II)}{2}{નીચેના બાયનરી નંબરના 1's અને 2's કોમ્પ્લિમેન્ટ શોધો (i) 1010 (ii) 1011}


\begin{solutionbox}
\textbf{1's કોમ્પ્લિમેન્ટ:} બધા બિટ્સ ઉલટાવો (0 ને 1 અને 1 ને 0 માં બદલો) \\
\textbf{2's કોમ્પ્લિમેન્ટ:} 1's કોમ્પ્લિમેન્ટ લો અને 1 ઉમેરો

\textbf{1010 માટે:}
\begin{itemize}
\item 1's કોમ્પ્લિમેન્ટ: 0101
\item 2's કોમ્પ્લિમેન્ટ: $0101 + 1 = 0110$
\end{itemize}

\textbf{1011 માટે:}
\begin{itemize}
\item 1's કોમ્પ્લિમેન્ટ: 0100
\item 2's કોમ્પ્લિમેન્ટ: $0100 + 1 = 0101$
\end{itemize}

\captionof{table}{કોમ્પ્લિમેન્ટ પરિણામો}
\begin{center}
\begin{tabulary}{\linewidth}{|L|L|L|}
\hline
બાયનરી & 1's કોમ્પ્લિમેન્ટ & 2's કોમ્પ્લિમેન્ટ \\
\hline
1010 & 0101 & 0110 \\
\hline
1011 & 0100 & 0101 \\
\hline
\end{tabulary}
\end{center}
\end{solutionbox}
\mnemonicbox{"1's માટે બધા બિટ ઉલટાવો, 2's માટે એક ઉમેરો"}

\questionmarks{1 [OR] (III)}{3}{2's કોમ્પ્લિમેન્ટ મેથડથી બાદબાકી કરો (i) $(110110)_2 - (101010)_2$}


\begin{solutionbox}
2's કોમ્પ્લિમેન્ટ પદ્ધતિથી બાદબાકી માટે:
\begin{enumerate}
\item બાદ થનાર સંખ્યાનો 2's કોમ્પ્લિમેન્ટ શોધો
\item તેને મૂળ સંખ્યામાં ઉમેરો
\item બિટ વિડ્થની બહારના કેરીને છોડી દો
\end{enumerate}

\textbf{બાદબાકી: $(110110)_2 - (101010)_2$}

\textbf{પગલું 1:} 101010 નો 2's કોમ્પ્લિમેન્ટ શોધો
\begin{itemize}
\item 101010 નો 1's કોમ્પ્લિમેન્ટ = 010101
\item 2's કોમ્પ્લિમેન્ટ = $010101 + 1 = 010110$
\end{itemize}

\textbf{પગલું 2:} 110110 + 010110 ઉમેરો

\begin{verbatim}
  1 1 1 1 1
  1 1 0 1 1 0
+ 0 1 0 1 1 0
--------------
  0 0 1 1 0 0
\end{verbatim}

\textbf{પગલું 3:} પરિણામ $001100 = (12)_{10}$

\captionof{table}{ બાદબાકી પ્રક્રિયા}
\begin{center}
\begin{tabulary}{\linewidth}{|C|L|L|}
\hline
પગલું & ક્રિયા & પરિણામ \\
\hline
1 & 101010 નો 2's કોમ્પ્લિમેન્ટ & 010110 \\
\hline
2 & 110110 + 010110 ઉમેરો & 001100 \\
\hline
3 & અંતિમ પરિણામ (ડેસિમલ) & 12 \\
\hline
\end{tabulary}
\end{center}
\end{solutionbox}
\mnemonicbox{"બાદનારનો કોમ્પ્લિમેન્ટ લો, ઉમેરો, કેરી ભૂલી જાઓ"}

\questionmarks{2(અ)}{3}{NAND ગેટનો જ ઉપયોગ કરી AND, OR અને NOT ગેટની લૉજિક સર્કિટ બનાવો}


\begin{solutionbox}
\textbf{AND ગેટ NAND ગેટથી:}
\begin{itemize}
\item AND ગેટ = NAND ગેટ પછી NOT ગેટ (NAND ગેટ)
\end{itemize}

\textbf{OR ગેટ NAND ગેટથી:}
\begin{itemize}
\item OR ગેટ = બંને ઇનપુટને NOT (NAND ગેટ) લાગુ કરો, પછી તે પરિણામોને NAND કરો
\end{itemize}

\textbf{NOT ગેટ NAND ગેટથી:}
\begin{itemize}
\item NOT ગેટ = NAND ગેટ જેમાં બંને ઇનપુટ જોડાયેલા હોય
\end{itemize}

\begin{figure}[H]
\centering
\begin{circuitikz}[scale=0.8, transform shape]
% NOT Gate
\draw (0,4) node[nand port] (nand1) {};
\draw (nand1.in 1) -- ++(-0.5,0) node[left] {A};
\draw (nand1.in 2) -- ++(-0.5,0); 
\draw (nand1.out) -- ++(0.5,0) node[right] {NOT A};
\draw (nand1.in 1) -- (nand1.in 2); 

% AND Gate
\draw (5,4) node[nand port] (nand2) {};
\draw (7,4) node[nand port] (nand3) {};
\draw (nand2.in 1) -- ++(-0.5,0) node[left] {B};
\draw (nand2.in 2) -- ++(-0.5,0) node[left] {C};
\draw (nand2.out) -- (nand3.in 1);
\draw (nand2.out) -- (nand3.in 2); 
\draw (nand3.out) -- ++(0.5,0) node[right] {B AND C};

% OR Gate
\draw (0,0) node[nand port] (nand4) {};
\draw (0,-2) node[nand port] (nand5) {};
\draw (3,-1) node[nand port] (nand6) {};
\draw (nand4.in 1) -- (nand4.in 2) -- ++(-0.5,0) node[left] {D};
\draw (nand5.in 1) -- (nand5.in 2) -- ++(-0.5,0) node[left] {E};
\draw (nand4.out) -- (nand6.in 1);
\draw (nand5.out) -- (nand6.in 2);
\draw (nand6.out) -- ++(0.5,0) node[right] {D OR E};

\node at (0,2.5) {\textbf{NOT Gate}};
\node at (6,2.5) {\textbf{AND Gate}};
\node at (1.5,-3) {\textbf{OR Gate}};
\end{circuitikz}
\caption{Universal Gates implementation}
\end{figure}
\end{solutionbox}
\mnemonicbox{"NOT માટે એક NAND, AND માટે બે NAND, OR માટે ત્રણ NAND"}

\questionmarks{2(બ)}{4}{નીચેના લૉજિક ગેટનો લૉજિક સિમ્બોલ, ટ્રુથ ટેબલ અને સમીકરણ લખો/દોરો (i) XOR ગેટ (ii) OR ગેટ}


\begin{solutionbox}
\textbf{XOR ગેટ:}

\textbf{લૉજિક સિમ્બોલ:}
\begin{center}
\begin{circuitikz}
\draw (0,0) node[xor port] (xor) {};
\draw (xor.in 1) -- ++(-0.5,0) node[left] {A};
\draw (xor.in 2) -- ++(-0.5,0) node[left] {B};
\draw (xor.out) -- ++(0.5,0) node[right] {Y};
\end{circuitikz}
\end{center}

\textbf{ટ્રુથ ટેબલ:}
\begin{center}
\begin{tabulary}{\linewidth}{|C|C|C|}
\hline
A & B & Y (A$\oplus$B) \\
\hline
0 & 0 & 0 \\
\hline
0 & 1 & 1 \\
\hline
1 & 0 & 1 \\
\hline
1 & 1 & 0 \\
\hline
\end{tabulary}
\end{center}

\textbf{બુલિયન સમીકરણ:} $Y = A \oplus B = A'B + AB'$

\textbf{OR ગેટ:}

\textbf{લૉજિક સિમ્બોલ:}
\begin{center}
\begin{circuitikz}
\draw (0,0) node[or port] (or) {};
\draw (or.in 1) -- ++(-0.5,0) node[left] {A};
\draw (or.in 2) -- ++(-0.5,0) node[left] {B};
\draw (or.out) -- ++(0.5,0) node[right] {Y};
\end{circuitikz}
\end{center}

\textbf{ટ્રુથ ટેબલ:}
\begin{center}
\begin{tabulary}{\linewidth}{|C|C|C|}
\hline
A & B & Y (A+B) \\
\hline
0 & 0 & 0 \\
\hline
0 & 1 & 1 \\
\hline
1 & 0 & 1 \\
\hline
1 & 1 & 1 \\
\hline
\end{tabulary}
\end{center}

\textbf{બુલિયન સમીકરણ:} $Y = A+B$
\end{solutionbox}
\mnemonicbox{"XOR: એક્સક્લુસિવ OR - એક અથવા બીજું પણ બંને નહીં; OR: એક અથવા બીજું અથવા બંને"}

\questionmarks{2(ક)(I)}{3}{બુલિયન સમીકરણ Y = A + B[AC + (B + C̅)D] ને algebraic મેથડથી સરળ બનાવો}


\begin{solutionbox}
\textbf{પગલાંવાર સરળીકરણ:}

\begin{align*}
Y &= A + B[AC + (B + \overline{C})D] \\
Y &= A + B[AC + BD + \overline{C}D] \\
Y &= A + BAC + BBD + B\overline{C}D \\
Y &= A + ABC + BD + B\overline{C}D & (\text{કારણ કે } BB = B) \\
Y &= A + AC + BD + B\overline{C}D & (\text{અવશોષણ: } A + AB = A) \\
Y &= A + BD + B\overline{C}D & (\text{અવશોષણ: } A + AC = A) \\
Y &= A + BD(1 + \overline{C}) \\
Y &= A + BD & (\text{કારણ કે } 1 + X = 1)
\end{align*}

\textbf{અંતિમ સમીકરણ:} $Y = A + BD$

\captionof{table}{સરળીકરણ પગલાં}
\begin{center}
\begin{tabulary}{\linewidth}{|C|L|L|}
\hline
પગલું & સમીકરણ & લાગુ પડેલ નિયમ \\
\hline
1 & $A + B[AC + (B + \overline{C})D]$ & મૂળ \\
\hline
2 & $A + B[AC + BD + \overline{C}D]$ & વિતરણ \\
\hline
3 & $A + BAC + BBD + B\overline{C}D$ & વિતરણ \\
\hline
4 & $A + ABC + BD + B\overline{C}D$ & આઇડેમ્પોટન્ટ ($BB = B$) \\
\hline
5 & $A + AC + BD + B\overline{C}D$ & અવશોષણ \\
\hline
6 & $A + BD + B\overline{C}D$ & અવશોષણ ($A+AC=A$) \\
\hline
7 & $A + BD(1 + \overline{C})$ & ફેક્ટરિંગ \\
\hline
8 & $A + BD$ & પૂરક નિયમ \\
\hline
\end{tabulary}
\end{center}
\end{solutionbox}
\mnemonicbox{"આઇડેમ્પોટન્સ, અવશોષણ, અને પૂરક પેટર્ન માટે હંમેશા તપાસો"}

\questionmarks{2(ક)(II)}{4}{બુલિયન સમીકરણ F(A,B,C) = Σm(0, 2, 3, 4, 5, 6) ને Karnaugh Map ની મદદથી સરળ બનાવો}


\begin{solutionbox}
\textbf{F(A,B,C) = Σm(0, 2, 3, 4, 5, 6) માટે K-map બનાવો:}

\begin{center}
\begin{tikzpicture}
\draw (0,0) grid (4,2);
\draw (0,2) -- (-0.5,2.5);
\node at (-0.7,2.2) {A};
\node at (-0.2,2.7) {BC};
\node at (0.5,2.3) {00};
\node at (1.5,2.3) {01};
\node at (2.5,2.3) {11};
\node at (3.5,2.3) {10};
\node at (-0.3,1.5) {0};
\node at (-0.3,0.5) {1};

% Cell values
\node at (0.5,1.5) {1}; % 0
\node at (1.5,1.5) {0}; % 1
\node at (2.5,1.5) {0}; % 3
\node at (3.5,1.5) {1}; % 2
\node at (0.5,0.5) {1}; % 4
\node at (1.5,0.5) {1}; % 5
\node at (2.5,0.5) {0}; % 7
\node at (3.5,0.5) {1}; % 6

% Groupings
\draw[red, rounded corners] (0.2,1.8) rectangle (0.8,0.2); % 0,4
\draw[blue, rounded corners] (3.2,1.8) rectangle (3.8,0.2); % 2,6
\draw[green, rounded corners] (0.2,0.2) rectangle (1.8,0.8); % 4,5
\end{tikzpicture}
\end{center}

\textbf{1 ની ગ્રુપિંગ કરો:}
\begin{itemize}
\item ગ્રુપ 1: m(0,4) - $B'C'$ સાથે સંબંધિત
\item ગ્રુપ 2: m(2,6) - $BC'$ સાથે સંબંધિત
\item ગ્રુપ 3: m(4,5) - $AB'$ સાથે સંબંધિત
\end{itemize}

\textbf{સરળ સમીકરણ:} $F(A,B,C) = B'C' + BC' + AB'$

\textbf{વધુ સરળ કરીએ:}
$F = C'(B'+B) + AB' = C' + AB'$
\end{solutionbox}
\mnemonicbox{"2ની પાવરમાં આસપાસના 1 ને ગ્રુપ કરો"}

\questionmarks{2 [OR] (અ)}{3}{NOR ગેટનો જ ઉપયોગ કરી AND, OR અને NOT ગેટની લૉજિક સર્કિટ બનાવો}


\begin{solutionbox}
\textbf{NOT ગેટ NOR ગેટથી:}
\begin{itemize}
\item NOT ગેટ = NOR ગેટ જેમાં બંને ઇનપુટ જોડાયેલા હોય
\end{itemize}

\textbf{AND ગેટ NOR ગેટથી:}
\begin{itemize}
\item AND ગેટ = બંને ઇનપુટને NOT (NOR ગેટ) લાગુ કરો, પછી તે પરિણામોને ફરીથી NOR કરો
\end{itemize}

\textbf{OR ગેટ NOR ગેટથી:}
\begin{itemize}
\item OR ગેટ = NOR ગેટ પછી NOT ગેટ (NOR ગેટ)
\end{itemize}

\begin{figure}[H]
\centering
\begin{circuitikz}[scale=0.8, transform shape]
% NOT Gate
\draw (0,4) node[nor port] (nor1) {};
\draw (nor1.in 1) -- ++(-0.5,0) node[left] {A};
\draw (nor1.in 2) -- ++(-0.5,0);
\draw (nor1.in 1) -- (nor1.in 2);
\draw (nor1.out) -- ++(0.5,0) node[right] {NOT A};

% AND Gate
\draw (0,0) node[nor port] (nor2) {};
\draw (0,-2) node[nor port] (nor3) {};
\draw (3,-1) node[nor port] (nor4) {};
\draw (nor2.in 1) -- (nor2.in 2) -- ++(-0.5,0) node[left] {B};
\draw (nor3.in 1) -- (nor3.in 2) -- ++(-0.5,0) node[left] {C};
\draw (nor2.out) -- (nor4.in 1);
\draw (nor3.out) -- (nor4.in 2);
\draw (nor4.out) -- ++(0.5,0) node[right] {B AND C};

% OR Gate
\draw (5,4) node[nor port] (nor5) {};
\draw (7,4) node[nor port] (nor6) {};
\draw (nor5.in 1) -- ++(-0.5,0) node[left] {D};
\draw (nor5.in 2) -- ++(-0.5,0) node[left] {E};
\draw (nor5.out) -- (nor6.in 1);
\draw (nor5.out) -- (nor6.in 2);
\draw (nor6.out) -- ++(0.5,0) node[right] {D OR E};

\node at (0,2.5) {\textbf{NOT Gate}};
\node at (1.5,-3) {\textbf{AND Gate}};
\node at (6,2.5) {\textbf{OR Gate}};
\end{circuitikz}
\caption{Universal Gates implementation (NOR)}
\end{figure}
\end{solutionbox}
\mnemonicbox{"NOT માટે એક NOR, દરેકને NOT કરીને NOR કરો AND માટે, બે વાર NOR કરો OR માટે"}

\questionmarks{2 [OR] (બ)}{4}{નીચેના લૉજિક ગેટનો લૉજિક સિમ્બોલ, ટ્રુથ ટેબલ અને સમીકરણ લખો/દોરો (i) NOR ગેટ (ii) AND ગેટ}


\begin{solutionbox}
\textbf{NOR ગેટ:}

\textbf{લૉજિક સિમ્બોલ:}
\begin{center}
\begin{circuitikz}
\draw (0,0) node[nor port] (nor) {};
\draw (nor.in 1) -- ++(-0.5,0) node[left] {A};
\draw (nor.in 2) -- ++(-0.5,0) node[left] {B};
\draw (nor.out) -- ++(0.5,0) node[right] {Y};
\end{circuitikz}
\end{center}

\textbf{ટ્રુથ ટેબલ:}
\begin{center}
\begin{tabulary}{\linewidth}{|C|C|C|}
\hline
A & B & Y (A+B)' \\
\hline
0 & 0 & 1 \\
\hline
0 & 1 & 0 \\
\hline
1 & 0 & 0 \\
\hline
1 & 1 & 0 \\
\hline
\end{tabulary}
\end{center}

\textbf{બુલિયન સમીકરણ:} $Y = (A+B)' = A'B'$

\textbf{AND ગેટ:}

\textbf{લૉજિક સિમ્બોલ:}
\begin{center}
\begin{circuitikz}
\draw (0,0) node[and port] (and) {};
\draw (and.in 1) -- ++(-0.5,0) node[left] {A};
\draw (and.in 2) -- ++(-0.5,0) node[left] {B};
\draw (and.out) -- ++(0.5,0) node[right] {Y};
\end{circuitikz}
\end{center}

\textbf{ટ્રુથ ટેબલ:}
\begin{center}
\begin{tabulary}{\linewidth}{|C|C|C|}
\hline
A & B & Y (A$\cdot$B) \\
\hline
0 & 0 & 0 \\
\hline
0 & 1 & 0 \\
\hline
1 & 0 & 0 \\
\hline
1 & 1 & 1 \\
\hline
\end{tabulary}
\end{center}

\textbf{બુલિયન સમીકરણ:} $Y = A \cdot B$
\end{solutionbox}
\mnemonicbox{"NOR: NOT OR - ન તો એક કે ન તો બીજું; AND: બંને 1 હોવા જ જોઈએ"}

\questionmarks{2 [OR] (ક)}{7}{ઉપરની લૉજિક સર્કિટ માટે બુલિયન સમીકરણ લખો. આ સમીકરણને સરળ બનાવો અને આ સરળ સમીકરણની લૉજિક સર્કિટ AND-OR-Invert મેથડથી દોરો}


\begin{solutionbox}
\textbf{પગલું 1:} સર્કિટમાંથી બુલિયન સમીકરણ લખો:
$Q = (A + B) \cdot (B + C \cdot ((B + C)'))$ \\
$Q = (A + B) \cdot (B + C \cdot (B' \cdot C'))$ \\
$Q = (A + B) \cdot (B + C \cdot B' \cdot C')$

\textbf{પગલું 2:} સમીકરણને સરળ બનાવો:
\begin{itemize}
\item નોંધ કરો કે $C \cdot C' = 0$
\item તેથી, $C \cdot B' \cdot C' = 0$
\item એટલે $Q = (A + B) \cdot (B + 0) = (A + B) \cdot B = A \cdot B + B \cdot B = A \cdot B + B = B + A \cdot B = B(1 + A) = B$
\end{itemize}

\textbf{પગલું 3:} અંતિમ સરળ સમીકરણ: $Q = B$

\textbf{પગલું 4:} AND-OR-Invert દ્વારા $Q = B$ નું અમલીકરણ:
\begin{itemize}
\item આ ફક્ત ઇનપુટ B થી આઉટપુટ Q સુધીનો એક તાર છે
\end{itemize}

\begin{center}
\begin{tikzpicture}
\draw (0,0) node[left] {B} -- (2,0) node[right] {Q};
\end{tikzpicture}
\end{center}

\captionof{table}{સરળીકરણ પગલાં}
\begin{center}
\begin{tabulary}{\linewidth}{|C|L|L|}
\hline
પગલું & સમીકરણ & સરળીકરણ \\
\hline
1 & $(A + B) \cdot (B + C \cdot ((B + C)'))$ & મૂળ સમીકરણ \\
\hline
2 & $(A + B) \cdot (B + C \cdot B' \cdot C')$ & ડી મોર્ગનનો નિયમ લાગુ કરવો \\
\hline
3 & $(A + B) \cdot (B + 0)$ & $C \cdot C' = 0$ \\
\hline
4 & $(A + B) \cdot B$ & સરળીકરણ \\
\hline
5 & $A \cdot B + B \cdot B$ & વિતરણ ગુણધર્મ \\
\hline
6 & $A \cdot B + B$ & આઇડેમ્પોટન્ટ ગુણધર્મ ($B \cdot B=B$) \\
\hline
7 & $B(1 + A)$ & ફેક્ટરિંગ \\
\hline
8 & $B$ & $1 + A = 1$ \\
\hline
\end{tabulary}
\end{center}
\end{solutionbox}
\mnemonicbox{"જ્યારે પૂરક ચલ ગુણાકાર કરે, તેઓ શૂન્ય થાય"}

\questionmarks{3(અ)}{3}{કોમ્બીનેશનલ સર્કિટની વ્યાખ્યા લખો. કોમ્બીનેશનલ સર્કિટના બે ઉદાહરણ લખો}


\begin{solutionbox}
\textbf{કોમ્બીનેશનલ સર્કિટ:} એક ડિજિટલ સર્કિટ જેનું આઉટપુટ માત્ર વર્તમાન ઇનપુટ મૂલ્યો પર આધારિત હોય છે અને અગાઉના ઇનપુટ અથવા સ્થિતિઓ પર નહીં. કોમ્બીનેશનલ સર્કિટમાં કોઈ મેમરી અથવા ફીડબેક હોતા નથી.

\textbf{મુખ્ય લક્ષણો:}
\begin{itemize}
\item આઉટપુટ ફક્ત વર્તમાન ઇનપુટ પર આધારિત હોય છે
\item કોઈ મેમરી એલિમેન્ટ નથી
\item કોઈ ફીડબેક પાથ નથી
\end{itemize}

\textbf{કોમ્બીનેશનલ સર્કિટના ઉદાહરણો:}
\begin{enumerate}
\item મલ્ટિપ્લેક્સર (MUX)
\item ડિકોડર
\item એડર/સબટ્રેક્ટર
\item એનકોડર
\item કમ્પેરેટર
\end{enumerate}

\captionof{table}{કોમ્બીનેશનલ vs સિક્વેન્શિયલ સર્કિટ}
\begin{center}
\begin{tabulary}{\linewidth}{|L|L|L|}
\hline
લક્ષણ & કોમ્બીનેશનલ સર્કિટ & સિક્વેન્શિયલ સર્કિટ \\
\hline
મેમરી & ના & હા \\
\hline
ફીડબેક & ના & સામાન્ય રીતે \\
\hline
આઉટપુટ આધારિત & માત્ર વર્તમાન ઇનપુટ & વર્તમાન અને અગાઉના ઇનપુટ \\
\hline
ઉદાહરણો & મલ્ટિપ્લેક્સર, એડર & ફ્લિપ-ફ્લોપ, કાઉન્ટર \\
\hline
\end{tabulary}
\end{center}
\end{solutionbox}
\mnemonicbox{"કોમ્બીનેશનલ સર્કિટ: વર્તમાન આવે, વર્તમાન જાય - કોઈ યાદ નહીં"}

\questionmarks{3(બ)}{4}{લૉજિક સર્કિટ અને ટ્રુથ ટેબલની મદદથી હાફ એડર સમજાવો}


\begin{solutionbox}
\textbf{હાફ એડર:} એક કોમ્બીનેશનલ સર્કિટ જે બે બાયનરી અંકો ઉમેરે છે અને સમ અને કેરી આઉટપુટ ઉત્પન્ન કરે છે.

\textbf{લૉજિક સર્કિટ:}
\begin{center}
\begin{circuitikz}
\draw (0,2) node[xor port] (xor) {};
\draw (0,0) node[and port] (and) {};
\draw (xor.in 1) -- ++(-1,0) node[left] (A) {A};
\draw (xor.in 2) -- ++(-1,0) node[left] (B) {B};
\draw (A) |- (and.in 1);
\draw (B) |- (and.in 2);
\draw (xor.out) -- ++(1,0) node[right] {Sum};
\draw (and.out) -- ++(1,0) node[right] {Carry};
\end{circuitikz}
\end{center}

\textbf{ટ્રુથ ટેબલ:}
\begin{center}
\begin{tabulary}{\linewidth}{|C|C|C|C|}
\hline
A & B & Sum & Carry \\
\hline
0 & 0 & 0 & 0 \\
\hline
0 & 1 & 1 & 0 \\
\hline
1 & 0 & 1 & 0 \\
\hline
1 & 1 & 0 & 1 \\
\hline
\end{tabulary}
\end{center}

\textbf{બુલિયન સમીકરણ:}
\begin{itemize}
\item Sum = $A \oplus B = A'B + AB'$
\item Carry = $A \cdot B$
\end{itemize}

\textbf{મર્યાદાઓ:}
\begin{itemize}
\item ત્રણ બાયનરી અંકો ઉમેરી શકતા નથી
\item અગાઉના તબક્કામાંથી કેરી ઇનપુટ સમાવી શકતા નથી
\end{itemize}
\end{solutionbox}
\mnemonicbox{"XOR સમને માટે, AND કેરીને માટે"}

\questionmarks{3(ક)(I)}{3}{મલ્ટિપ્લેક્સર ટૂંકમાં સમજાવો}


\begin{solutionbox}
\textbf{મલ્ટિપ્લેક્સર (MUX):} એક કોમ્બીનેશનલ સર્કિટ જે સિલેક્ટ લાઇન્સના આધારે અનેક ઇનપુટ સિગ્નલ્સમાંથી એકને પસંદ કરે છે અને તેને એક આઉટપુટ લાઇન પર મોકલે છે.

\textbf{મુખ્ય લક્ષણો:}
\begin{itemize}
\item ડિજિટલ સ્વિચ તરીકે કાર્ય કરે છે
\item $2^n$ ડેટા ઇનપુટ, $n$ સિલેક્ટ લાઇન, અને 1 આઉટપુટ ધરાવે છે
\item સિલેક્ટ લાઇન્સ નક્કી કરે છે કે કયું ઇનપુટ આઉટપુટથી જોડાયેલું છે
\end{itemize}

\textbf{સામાન્ય મલ્ટિપ્લેક્સર:}
\begin{itemize}
\item 2:1 MUX (1 સિલેક્ટ લાઇન)
\item 4:1 MUX (2 સિલેક્ટ લાઇન)
\item 8:1 MUX (3 સિલેક્ટ લાઇન)
\end{itemize}

\textbf{મૂળભૂત રચના:}
\begin{center}
\begin{tikzpicture}[node distance=1.5cm]
\node [gtu input] (i0) {I0};
\node [gtu input, below of=i0] (i1) {I1};
\node [below of=i1] (dots) {...};
\node [gtu input, below of=dots] (in) {$I_{2^n-1}$};

\node [trapezium, draw, shape border rotate=270, minimum height=3cm, minimum width=2cm, right of=dots, xshift=2cm] (mux) {MUX};

\draw [gtu arrow] (i0) -- (mux.north west);
\draw [gtu arrow] (i1) -- (mux.west |- i1);
\draw [gtu arrow] (in) -- (mux.south west);

\node [gtu input, below of=mux, yshift=-1cm] (sel) {Select Lines};
\draw [gtu arrow] (sel) -- (mux.south);

\node [gtu output, right of=mux, xshift=2cm] (out) {Output Y};
\draw [gtu arrow] (mux.east) -- (out);

\end{tikzpicture}
\end{center}

\textbf{ઉપયોગો:}
\begin{itemize}
\item ડેટા રાઉટિંગ
\item ડેટા પસંદગી
\item પેરેલલથી સીરિયલ રૂપાંતર
\item બુલિયન ફંક્શનનું અમલીકરણ
\end{itemize}
\end{solutionbox}
\mnemonicbox{"ઘણા ઇન, સિલેક્શન પસંદ કરે, એક આઉટ"}

\questionmarks{3(ક)(II)}{4}{8:1 મલ્ટિપ્લેક્સર ડિઝાઇન કરો. તેનું ટ્રુથ ટેબલ લખો અને લૉજિક સર્કિટ દોરો}


\begin{solutionbox}
\textbf{8:1 મલ્ટિપ્લેક્સર ડિઝાઇન:}
\begin{itemize}
\item 8 ડેટા ઇનપુટ ($I_0$ થી $I_7$)
\item 3 સિલેક્ટ લાઇન ($S_2, S_1, S_0$)
\item 1 આઉટપુટ (Y)
\end{itemize}

\textbf{ટ્રુથ ટેબલ:}
\begin{center}
\begin{tabulary}{\linewidth}{|C|C|C|C|}
\hline
$S_2$ & $S_1$ & $S_0$ & આઉટપુટ Y \\
\hline
0 & 0 & 0 & $I_0$ \\
\hline
0 & 0 & 1 & $I_1$ \\
\hline
0 & 1 & 0 & $I_2$ \\
\hline
0 & 1 & 1 & $I_3$ \\
\hline
1 & 0 & 0 & $I_4$ \\
\hline
1 & 0 & 1 & $I_5$ \\
\hline
1 & 1 & 0 & $I_6$ \\
\hline
1 & 1 & 1 & $I_7$ \\
\hline
\end{tabulary}
\end{center}

\textbf{બુલિયન સમીકરણ:}
$Y = S_2'S_1'S_0'I_0 + S_2'S_1'S_0I_1 + S_2'S_1S_0'I_2 + S_2'S_1S_0I_3 + S_2S_1'S_0'I_4 + S_2S_1'S_0I_5 + S_2S_1S_0'I_6 + S_2S_1S_0I_7$

\textbf{લૉજિક સર્કિટ:}
\begin{center}
\begin{circuitikz}[scale=0.7, transform shape]
% AND Gates
\foreach \i in {0,...,7} {
    \draw (0, \i*1.5) node[and port, number inputs=4] (and\i) {};
    \draw (and\i.in 4) -- ++(-0.5,0) node[left] {$I_{\i}$};
}
% OR Gate
\draw (4, 5.25) node[or port, number inputs=8] (or) {};

% Connections to OR
\foreach \i [evaluate=\i as \j using int(\i+1)] in {0,...,7} {
    \draw (and\i.out) -- (or.in \j);
}
\node at (6, 5.25) {Y};
\draw (or.out) -- (6,5.25);
\end{circuitikz}
\end{center}
\end{solutionbox}
\mnemonicbox{"આઠ ઇનપુટ, ત્રણ સિલેક્ટ, ડિકોડ કરો અને આઉટપુટ મેળવવા OR કરો"}

\questionmarks{3 [OR] (અ)}{3}{4-bit બાયનરી પેરેલલ એડરનો બ્લોક ડાયાગ્રામ દોરો}


\begin{solutionbox}
\textbf{4-bit બાયનરી પેરેલલ એડર:}
બે 4-bit બાયનરી નંબર ઉમેરતી અને 4-bit સરવાળો અને એક કેરી આઉટપુટ ઉત્પન્ન કરતી સર્કિટ.

\begin{center}
\begin{tikzpicture}[node distance=2.5cm, auto]
 \node [gtu block] (fa0) {FA0};
 \node [gtu block, left of=fa0] (fa1) {FA1};
 \node [gtu block, left of=fa1] (fa2) {FA2};
 \node [gtu block, left of=fa2] (fa3) {FA3};

 % Inputs A and B
 \foreach \i in {0,1,2,3} {
  \draw [gtu arrow] ([yshift=0.5cm]fa\i.north) -- (fa\i.north) node[midway, right] {$A_\i, B_\i$};
  \draw [gtu arrow] (fa\i.south) -- ([yshift=-0.5cm]fa\i.south) node[midway, right] {$S_\i$};
 }

 % Carries
 \draw [gtu arrow] (fa0.west) -- (fa1.east) node[midway, above] {$C_1$};
 \draw [gtu arrow] (fa1.west) -- (fa2.east) node[midway, above] {$C_2$};
 \draw [gtu arrow] (fa2.west) -- (fa3.east) node[midway, above] {$C_3$};

 \draw [gtu arrow] ([xshift=0.5cm]fa0.east) -- (fa0.east) node[midway, above] {$C_{in}=0$};
 \draw [gtu arrow] (fa3.west) -- ([xshift=-0.5cm]fa3.west) node[midway, above] {$C_{out}$};

\end{tikzpicture}
\end{center}

\textbf{ઘટકો:}
\begin{itemize}
\item ચાર ફુલ એડર (FA) કેસ્કેડમાં જોડાયેલા
\item દરેક FA સંબંધિત બિટ્સ અને અગાઉના તબક્કાની કેરી ઉમેરે છે
\item પ્રારંભિક કેરી-ઇન (Cin) સામાન્ય રીતે 0 હોય છે
\end{itemize}
\end{solutionbox}
\mnemonicbox{"ચાર FA જોડાયેલા, કેરીઓ વચ્ચેથી પસાર થાય છે"}

\questionmarks{3 [OR] (બ)}{4}{લૉજિક સર્કિટ અને ટ્રુથ ટેબલની મદદથી ફૂલ એડર સમજાવો}


\begin{solutionbox}
\textbf{ફૂલ એડર:} એક કોમ્બીનેશનલ સર્કિટ જે ત્રણ બાયનરી અંક (બે ઇનપુટ અને એક કેરી-ઇન) ઉમેરે છે અને સરવાળો અને કેરી આઉટપુટ ઉત્પન્ન કરે છે.

\textbf{લૉજિક સર્કિટ:}
\begin{center}
\begin{circuitikz}
\draw (0,2) node[xor port] (xor1) {};
\draw (3,1.5) node[xor port] (xor2) {};
\draw (xor1.out) -- (xor2.in 1);
\draw (xor1.in 1) -- ++(-0.5,0) node[left] {A};
\draw (xor1.in 2) -- ++(-0.5,0) node[left] {B};
\draw (xor2.in 2) -- ++(-0.5,0) node[left] {$C_{in}$};
\draw (xor2.out) -- ++(0.5,0) node[right] {Sum};

\draw (3,-0.5) node[and port] (and1) {};
\draw (3,-2) node[and port] (and2) {};
\draw (5,-1.25) node[or port] (or) {};
% Wires
\draw (and1.out) -- (or.in 1);
\draw (and2.out) -- (or.in 2);
\draw (or.out) -- ++(0.5,0) node[right] {$C_{out}$};
% Input connections
\node at (0,2) (A_node) {}; 
\end{circuitikz}
\end{center}

\textbf{ટ્રુથ ટેબલ:}
\begin{center}
\begin{tabulary}{\linewidth}{|C|C|C|C|C|}
\hline
A & B & $C_{in}$ & Sum & $C_{out}$ \\
\hline
0 & 0 & 0 & 0 & 0 \\
\hline
0 & 0 & 1 & 1 & 0 \\
\hline
0 & 1 & 0 & 1 & 0 \\
\hline
0 & 1 & 1 & 0 & 1 \\
\hline
1 & 0 & 0 & 1 & 0 \\
\hline
1 & 0 & 1 & 0 & 1 \\
\hline
1 & 1 & 0 & 0 & 1 \\
\hline
1 & 1 & 1 & 1 & 1 \\
\hline
\end{tabulary}
\end{center}

\textbf{બુલિયન સમીકરણ:}
\begin{itemize}
\item Sum = $A \oplus B \oplus C_{in}$
\item $C_{out} = AB + C_{in}(A \oplus B)$
\end{itemize}
\end{solutionbox}
\mnemonicbox{"ત્રણેય XOR કરો સમ માટે, ANDsને OR કરો કેરી માટે"}

\questionmarks{3 [OR] (ક) (I)}{3}{લૉજિક સર્કિટ અને ટ્રુથ ટેબલની મદદથી 4:1 મલ્ટિપ્લેક્સર સમજાવો}


\begin{solutionbox}
\textbf{4:1 મલ્ટિપ્લેક્સર:} એક ડિજિટલ સ્વિચ જે બે સિલેક્ટ લાઇન્સના આધારે ચાર ઇનપુટ લાઇન્સમાંથી એકને પસંદ કરે છે અને તેને આઉટપુટથી જોડે છે.

\textbf{લૉજિક સર્કિટ:}
\begin{center}
\begin{circuitikz}[scale=0.8]
\draw (2, 3) node[and port, number inputs=3] (and0) {};
\draw (2, 1) node[and port, number inputs=3] (and1) {};
\draw (2, -1) node[and port, number inputs=3] (and2) {};
\draw (2, -3) node[and port, number inputs=3] (and3) {};

\draw (5, 0) node[or port, number inputs=4] (or) {};

\draw (and0.out) -- (or.in 1);
\draw (and1.out) -- (or.in 2);
\draw (and2.out) -- (or.in 3);
\draw (and3.out) -- (or.in 4);
\draw (or.out) -- ++(0.5,0) node[right] {Y};

\draw (and0.in 1) -- ++(-0.5,0) node[left] {$I_0$};
\draw (and1.in 1) -- ++(-0.5,0) node[left] {$I_1$};
\draw (and2.in 1) -- ++(-0.5,0) node[left] {$I_2$};
\draw (and3.in 1) -- ++(-0.5,0) node[left] {$I_3$};
\end{circuitikz}
\end{center}

\textbf{ટ્રુથ ટેબલ:}
\begin{center}
\begin{tabulary}{\linewidth}{|C|C|C|}
\hline
$S_1$ & $S_0$ & આઉટપુટ Y \\
\hline
0 & 0 & $I_0$ \\
\hline
0 & 1 & $I_1$ \\
\hline
1 & 0 & $I_2$ \\
\hline
1 & 1 & $I_3$ \\
\hline
\end{tabulary}
\end{center}

\textbf{બુલિયન સમીકરણ:}
$Y = S_1'S_0'I_0 + S_1'S_0I_1 + S_1S_0'I_2 + S_1S_0I_3$
\end{solutionbox}
\mnemonicbox{"બે સિલેક્ટ લાઇન ચાર ઇનપુટમાંથી એક પસંદ કરે છે"}

\questionmarks{3 [OR] (ક) (II)}{4}{બે 4:1 મલ્ટિપ્લેક્સરનો ઉપયોગ કરીને 8:1 મલ્ટિપ્લેક્સર ડિઝાઇન કરો.}


\begin{solutionbox}
\textbf{ડિઝાઇન અભિગમ:} 8:1 MUX બનાવવા માટે બે 4:1 MUX અને એક 2:1 MUX વાપરો.

\begin{enumerate}
\item પ્રથમ 4:1 MUX ઇનપુટ $I_0-I_3$ સંભાળે છે, સિલેક્ટ લાઇન $S_0,S_1$નો ઉપયોગ કરીને
\item બીજો 4:1 MUX ઇનપુટ $I_4-I_7$ સંભાળે છે, સિલેક્ટ લાઇન $S_0,S_1$નો ઉપયોગ કરીને
\item 2:1 MUX બે 4:1 MUXના આઉટપુટ વચ્ચે $S_2$નો ઉપયોગ કરીને પસંદગી કરે છે
\end{enumerate}

\textbf{બ્લોક ડાયાગ્રામ:}
\begin{center}
\begin{tikzpicture}[node distance=2.5cm, auto]
    \node [gtu block, minimum height=2cm] (mux1) {4:1 MUX \\ (I0-I3)};
    \node [gtu block, minimum height=2cm, below of=mux1] (mux2) {4:1 MUX \\ (I4-I7)};
    \node [gtu block, right of=mux1, yshift=-1.25cm] (mux3) {2:1 MUX};

    \draw [gtu arrow] (mux1.east) -- (mux3.west |- mux1.east); 
    \draw [gtu arrow] (mux2.east) -- (mux3.west |- mux2.east); 

    \draw [gtu arrow] (mux3.east) -- ++(1,0) node[right] {Y};
    
    \node [below of=mux2, node distance=1.5cm] (s0s1) {S0, S1};
    \draw [gtu arrow] (s0s1) -- (mux2.south);
    \draw [gtu arrow] (s0s1) -- (mux1.south); 
    
    \node [below of=mux3] (s2) {S2};
    \draw [gtu arrow] (s2) -- (mux3.south);

\end{tikzpicture}
\end{center}

\textbf{ટ્રુથ ટેબલ:}
\begin{center}
\begin{tabulary}{\linewidth}{|C|C|C|C|}
\hline
$S_2$ & $S_1$ & $S_0$ & આઉટપુટ Y \\
\hline
0 & 0 & 0 & $I_0$ \\
\hline
0 & 0 & 1 & $I_1$ \\
\hline
0 & 1 & 0 & $I_2$ \\
\hline
0 & 1 & 1 & $I_3$ \\
\hline
1 & 0 & 0 & $I_4$ \\
\hline
1 & 0 & 1 & $I_5$ \\
\hline
1 & 1 & 0 & $I_6$ \\
\hline
1 & 1 & 1 & $I_7$ \\
\hline
\end{tabulary}
\end{center}
\end{solutionbox}
\mnemonicbox{"S0,S1 દરેક 4:1 MUXમાંથી પસંદ કરે છે, S2 તેમની વચ્ચે પસંદ કરે છે"}

\questionmarks{4(અ)}{3}{સિક્વન્સીયલ સર્કિટની વ્યાખ્યા લખો. તેના બે ઉદાહરણ લખો}


\begin{solutionbox}
\textbf{સિક્વન્સીયલ સર્કિટ:} એક ડિજિટલ સર્કિટ જેનું આઉટપુટ માત્ર વર્તમાન ઇનપુટ પર જ નહીં પણ ઇનપુટના ભૂતકાળના ક્રમ (ઇતિહાસ/અગાઉની સ્થિતિ) પર પણ આધારિત હોય છે.

\textbf{મુખ્ય લક્ષણો:}
\begin{itemize}
\item મેમરી એલિમેન્ટ્સ (ફ્લિપ-ફ્લોપ) ધરાવે છે
\item આઉટપુટ વર્તમાન ઇનપુટ અને અગાઉની સ્થિતિઓ બંને પર આધારિત છે
\item સામાન્ય રીતે ફીડબેક પાથ સમાવે છે
\item સિંક્રોનાઇઝેશન માટે ક્લોક સિગ્નલની જરૂર પડે છે (સિંક્રોનસ સર્કિટ માટે)
\end{itemize}

\textbf{સિક્વન્સીયલ સર્કિટના ઉદાહરણો:}
\begin{enumerate}
\item ફ્લિપ-ફ્લોપ (SR, JK, D, T)
\item રજિસ્ટર (શિફ્ટ રજિસ્ટર)
\item કાઉન્ટર (બાયનરી, ડેકેડ, રિંગ કાઉન્ટર)
\item સ્ટેટ મશીન
\item મેમરી યુનિટ
\end{enumerate}

\captionof{table}{સિક્વન્સીયલ vs કોમ્બીનેશનલ સર્કિટ}
\begin{center}
\begin{tabulary}{\linewidth}{|L|L|L|}
\hline
લક્ષણ & સિક્વન્સીયલ સર્કિટ & કોમ્બીનેશનલ સર્કિટ \\
\hline
મેમરી & હા & ના \\
\hline
ફીડબેક & સામાન્ય રીતે & ના \\
\hline
આઉટપુટ આધારિત & વર્તમાન અને અગાઉના ઇનપુટ & માત્ર વર્તમાન ઇનપુટ \\
\hline
ક્લોક જરૂરી & સામાન્ય રીતે & ના \\
\hline
ઉદાહરણો & ફ્લિપ-ફ્લોપ, કાઉન્ટર & મલ્ટિપ્લેક્સર, એડર \\
\hline
\end{tabulary}
\end{center}
\end{solutionbox}
\mnemonicbox{"સિક્વન્સીયલ ઇતિહાસ યાદ રાખે છે, કોમ્બીનેશનલ માત્ર વર્તમાન જાણે છે"}

\questionmarks{4(બ)}{4}{ડિકેડ કાઉન્ટર ડિઝાઇન કરો}


\begin{solutionbox}
\textbf{ડિકેડ કાઉન્ટર:} એક સિક્વન્સીયલ સર્કિટ જે 0 થી 9 (ડેસિમલ) સુધી ગણે છે અને પછી 0 પર રીસેટ થાય છે.

\textbf{JK ફ્લિપ-ફ્લોપનો ઉપયોગ કરી ડિઝાઇન:}
\begin{itemize}
\item 4 બિટ બાયનરી નંબર રજૂ કરવા માટે 4 JK ફ્લિપ-ફ્લોપ ($Q_3,Q_2,Q_1,Q_0$) જરૂરી છે
\item 0000 થી 1001 (0-9 ડેસિમલ) સુધી ગણે છે પછી રીસેટ થાય છે
\end{itemize}

\textbf{J-K ઇનપુટ સમીકરણ:}
\begin{itemize}
\item $J_0 = K_0 = 1$ (દરેક ક્લોક પર ટોગલ)
\item $J_1 = K_1 = Q_0 \cdot \overline{Q_3}$
\item $J_2 = K_2 = Q_1 \cdot Q_0$
\item $J_3 = K_3 = Q_2 \cdot Q_1 \cdot Q_0 + Q_3 \cdot Q_0$
\end{itemize}

\textbf{રીસેટ સ્થિતિ:} જ્યારે $Q_3 \cdot Q_1 = 1$ (સ્થિતિ 1010), બધા ફ્લિપ-ફ્લોપ રીસેટ કરો

\textbf{બ્લોક ડાયાગ્રામ (Asynchronous/Ripple):}
\begin{center}
\begin{tikzpicture}[node distance=2.5cm]
\node [gtu block] (ff0) {JK0 \\ (LSB)};
\node [gtu block, right of=ff0] (ff1) {JK1};
\node [gtu block, right of=ff1] (ff2) {JK2};
\node [gtu block, right of=ff2] (ff3) {JK3 \\ (MSB)};

\draw [gtu arrow] ([yshift=0.5cm]ff0.west) -- (ff0.west) node[midway, above] {CLK};
\draw [gtu arrow] (ff0.east) -- (ff1.west);
\draw [gtu arrow] (ff1.east) -- (ff2.west);
\draw [gtu arrow] (ff2.east) -- (ff3.west);

\node [gtu decision, below of=ff1, xshift=1.25cm] (nand) {NAND};
\draw [gtu arrow] (ff1.south) |- (nand.west); % Q1
\draw [gtu arrow] (ff3.south) |- (nand.east); % Q3
\draw [gtu arrow] (nand.south) -- ++(0,-0.5) -| (ff0.south) node[pos=0.5, below] {CLR};
\draw [gtu arrow] (nand.south) -- ++(0,-0.5) -| (ff1.south);
\draw [gtu arrow] (nand.south) -- ++(0,-0.5) -| (ff2.south);
\draw [gtu arrow] (nand.south) -- ++(0,-0.5) -| (ff3.south);

\end{tikzpicture}
\end{center}
\end{solutionbox}
\mnemonicbox{"BCD ગણો, 9 પછી રીસેટ"}

\questionmarks{4(ક)(I)}{3}{NOR ગેટની મદદથી S-R ફ્લિપ-ફ્લોપ સમજાવો. તેનો લૉજિક સિમ્બોલ દોરો અને ટ્રુથ ટેબલ લખો.}


\begin{solutionbox}
\textbf{NOR ગેટથી S-R ફ્લિપ-ફ્લોપ:} બે ક્રોસ-કપલ્ડ NOR ગેટમાંથી બનેલું એક મૂળભૂત ફ્લિપ-ફ્લોપ જે એક બિટની માહિતી સંગ્રહિત કરી શકે છે.

\textbf{લૉજિક સર્કિટ:}
\begin{center}
\begin{circuitikz}
\draw (0,2) node[nor port] (nor1) {};
\draw (0,0) node[nor port] (nor2) {};
\draw (nor1.in 1) -- ++(-0.5,0) node[left] {R};
\draw (nor2.in 2) -- ++(-0.5,0) node[left] {S};
% Cross coupling
\draw (nor1.out) -- ++(0.5,0) coordinate (Q) -- ++(0,-0.5) -- ($(nor2.in 1) + (-0.5,0.5)$) -- (nor2.in 1);
\draw (nor2.out) -- ++(0.5,0) coordinate (Qn) -- ++(0,0.5) -- ($(nor1.in 2) + (-0.5,-0.5)$) -- (nor1.in 2);

\draw (Q) -- ++(1,0) node[right] {Q};
\draw (Qn) -- ++(1,0) node[right] {Q'};
\end{circuitikz}
\end{center}

\textbf{લૉજિક સિમ્બોલ:}
\begin{center}
\begin{tikzpicture}
\node [gtu block, minimum height=2cm, minimum width=1.5cm] (sr) {SR \\ FF};
\draw [gtu arrow] ([yshift=0.5cm]sr.west) -- ++(-0.5,0) node[left] {S};
\draw [gtu arrow] ([yshift=-0.5cm]sr.west) -- ++(-0.5,0) node[left] {R};
\draw [gtu arrow] (sr.east) -- ++(0.5,0) node[right] {Q};
\draw [gtu arrow] ([yshift=-0.5cm]sr.east) -- ++(0.5,0) node[right] {Q'};
\end{tikzpicture}
\end{center}

\textbf{ટ્રુથ ટેબલ:}
\begin{center}
\begin{tabulary}{\linewidth}{|C|C|C|C|L|}
\hline
S & R & Q (આગામી) & Q' (આગામી) & ઓપરેશન \\
\hline
0 & 0 & Q (અગાઉની) & Q' (અગાઉની) & મેમરી (કોઈ ફેરફાર નહીં) \\
\hline
0 & 1 & 0 & 1 & રીસેટ \\
\hline
1 & 0 & 1 & 0 & સેટ \\
\hline
1 & 1 & 0 & 0 & અમાન્ય (ટાળો) \\
\hline
\end{tabulary}
\end{center}
\end{solutionbox}
\mnemonicbox{"S થી 1 સેટ થાય, R થી 0 રીસેટ થાય, બંને એકસાથે અમાન્ય સ્થિતિ આપે"}

\questionmarks{4(ક)(II)}{4}{NAND ગેટની મદદથી S-R ફ્લિપ-ફ્લોપ સમજાવો. S-R ફ્લિપ-ફ્લોપની મર્યાદા લખો}


\begin{solutionbox}
\textbf{NAND ગેટથી S-R ફ્લિપ-ફ્લોપ:} બે ક્રોસ-કપલ્ડ NAND ગેટમાંથી બનેલું એક મૂળભૂત ફ્લિપ-ફ્લોપ.

\textbf{લૉજિક સર્કિટ:}
\begin{center}
\begin{circuitikz}
\draw (0,2) node[nand port] (nand1) {};
\draw (0,0) node[nand port] (nand2) {};
\draw (nand1.in 1) -- ++(-0.5,0) node[left] {S'};
\draw (nand2.in 2) -- ++(-0.5,0) node[left] {R'};
% Cross coupling
\draw (nand1.out) -- ++(0.5,0) coordinate (Q) -- ++(0,-0.5) -- ($(nand2.in 1) + (-0.5,0.5)$) -- (nand2.in 1);
\draw (nand2.out) -- ++(0.5,0) coordinate (Qn) -- ++(0,0.5) -- ($(nand1.in 2) + (-0.5,-0.5)$) -- (nand1.in 2);

\draw (Q) -- ++(1,0) node[right] {Q};
\draw (Qn) -- ++(1,0) node[right] {Q'};
\end{circuitikz}
\end{center}

\textbf{SR ફ્લિપ-ફ્લોપની મર્યાદાઓ:}
\begin{enumerate}
\item \textbf{અમાન્ય સ્થિતિ:} જ્યારે S=1, R=1 (NOR માટે) અથવા S=0, R=0 (NAND માટે), આઉટપુટ અનિશ્ચિત રહે છે
\item \textbf{રેસ કન્ડિશન:} જ્યારે ઇનપુટ એકસાથે બદલાય છે, ત્યારે અંતિમ સ્થિતિ અનિશ્ચિત હોઈ શકે છે
\item \textbf{ક્લોકિંગ મેકેનિઝમ નથી:} અન્ય ડિજિટલ ઘટકો સાથે સિંક્રોનાઇઝ થઈ શકતું નથી
\item \textbf{એજ-ટ્રિગર્ડ નથી:} ટૂંકા પલ્સને વિશ્વસનીય રીતે પ્રતિક્રિયા આપી શકતું નથી
\item \textbf{અનિચ્છનીય ટોગલિંગ:} નોઇઝ કે ગ્લિચને પ્રતિક્રિયા આપી શકે છે
\end{enumerate}

\captionof{table}{NAND vs NOR SR ફ્લિપ-ફ્લોપ}
\begin{center}
\begin{tabulary}{\linewidth}{|L|L|L|}
\hline
લક્ષણ & NAND SR ફ્લિપ-ફ્લોપ & NOR SR ફ્લિપ-ફ્લોપ \\
\hline
સક્રિય ઇનપુટ & લો (0) & હાઇ (1) \\
\hline
નિષ્ક્રિય ઇનપુટ & હાઇ (1) & લો (0) \\
\hline
અમાન્ય સ્થિતિ & S=0, R=0 & S=1, R=1 \\
\hline
\end{tabulary}
\end{center}
\end{solutionbox}
\mnemonicbox{"NAND: ઇનપુટ એક્ટિવ-લો, NOR: ઇનપુટ એક્ટિવ-હાઇ; બંનેમાં એક અમાન્ય સ્થિતિ છે"}

\questionmarks{4 [OR] (અ)}{3}{ફ્લિપ-ફ્લોપની વ્યાખ્યા લખો. ફ્લિપ-ફ્લોપના પ્રકાર લખો}


\begin{solutionbox}
\textbf{ફ્લિપ-ફ્લોપ:} એક મૂળભૂત સિક્વન્સીયલ ડિજિટલ સર્કિટ જે એક બિટની માહિતી સંગ્રહિત કરી શકે છે અને બે સ્થાયી સ્થિતિઓ (0 અથવા 1) ધરાવે છે. તે ડિજિટલ સિસ્ટમમાં મૂળભૂત મેમરી એલિમેન્ટ તરીકે કામ કરે છે.

\textbf{મુખ્ય લક્ષણો:}
\begin{itemize}
\item બાયસ્ટેબલ મલ્ટિવાયબ્રેટર (બે સ્થાયી સ્થિતિઓ)
\item જ્યાં સુધી બદલવાનો નિર્દેશ ન અપાય ત્યાં સુધી પોતાની સ્થિતિ અનિશ્ચિત સમય સુધી જાળવી રાખી શકે છે
\item રજિસ્ટર, કાઉન્ટર અને મેમરી સર્કિટ માટે મૂળભૂત બિલ્ડિંગ બ્લોક બને છે
\end{itemize}

\textbf{ફ્લિપ-ફ્લોપના પ્રકાર:}
\begin{center}
\begin{tabulary}{\linewidth}{|L|L|}
\hline
ફ્લિપ-ફ્લોપ પ્રકાર & વર્ણન \\
\hline
SR (સેટ-રીસેટ) & સૌથી મૂળભૂત ફ્લિપ-ફ્લોપ જેમાં સેટ અને રીસેટ ઇનપુટ હોય છે \\
\hline
JK & SR ફ્લિપ-ફ્લોપની સુધારેલી આવૃત્તિ જે અમાન્ય સ્થિતિ દૂર કરે છે \\
\hline
D (ડેટા) & ઇનપુટ D પરનો મૂલ્ય સંગ્રહિત કરે છે, ડેટા સ્ટોરેજ માટે વપરાય છે \\
\hline
T (ટોગલ) & ટ્રિગર થયે સ્થિતિ બદલે છે, કાઉન્ટર માટે ઉપયોગી \\
\hline
માસ્ટર-સ્લેવ & રેસ કન્ડિશન અટકાવતું બે-તબક્કાનું ફ્લિપ-ફ્લોપ \\
\hline
\end{tabulary}
\end{center}
\end{solutionbox}
\mnemonicbox{"એક સિંગલ સ્ટેટ સ્ટોરેજ: SR, JK, D, T"}

\questionmarks{4 [OR] (બ)}{4}{3-bit રિંગ કાઉન્ટર ડિઝાઇન કરો}


\begin{solutionbox}
\textbf{રિંગ કાઉન્ટર:} એક સર્ક્યુલર શિફ્ટ રજિસ્ટર જેમાં ફક્ત એક બિટ સેટ (1) હોય છે અને બાકી બધા રીસેટ (0) હોય છે. એકમાત્ર સેટ બિટ ક્લોક થતાં રજિસ્ટરમાં "ફરે" છે.

\textbf{D ફ્લિપ-ફ્લોપનો ઉપયોગ કરી ડિઝાઇન:}
\begin{itemize}
\item 3-bit કાઉન્ટર માટે 3 D ફ્લિપ-ફ્લોપ જરૂરી છે
\item પ્રારંભિક સ્થિતિ: 100, પછી 010, 001, અને પાછા 100 પર જાય છે
\end{itemize}

\textbf{બ્લોક ડાયાગ્રામ:}
\begin{center}
\begin{tikzpicture}[node distance=2.5cm]
\node [gtu block] (d0) {D0 \\ (MSB)};
\node [gtu block, right of=d0] (d1) {D1};
\node [gtu block, right of=d1] (d2) {D2 \\ (LSB)};

\draw [gtu arrow] ([yshift=0.5cm]d0.west) -- (d0.west) node[midway, above] {CLK};
\draw [gtu arrow] (d0.east) -- (d1.west) node[midway, above] {$Q_0$};
\draw [gtu arrow] (d1.east) -- (d2.west) node[midway, above] {$Q_1$};

% Feedback
\draw [gtu arrow] (d2.east) -- ++(0.5,0) -- ++(0,-1.5) -| (d0.west) node[pos=0.1, right] {$Q_2$};

\end{tikzpicture}
\end{center}
\end{solutionbox}
\mnemonicbox{"એક હોટ બિટ વર્તુળમાં ફરે છે"}

\questionmarks{4 [OR] (ક)(I)}{3}{લૉજિક સિમ્બોલ અને ટ્રુથ ટેબલની મદદથી J-K ફ્લિપ-ફ્લોપ સમજાવો}


\begin{solutionbox}
\textbf{J-K ફ્લિપ-ફ્લોપ:} SR ફ્લિપ-ફ્લોપની સુધારેલી આવૃત્તિ જે અમાન્ય સ્થિતિ દૂર કરે છે અને બધા ઇનપુટ સંયોજનોમાં સચોટ વર્તન દર્શાવે છે.

\textbf{લૉજિક સિમ્બોલ:}
\begin{center}
\begin{tikzpicture}
\node [gtu block, minimum height=2cm, minimum width=1.5cm] (jk) {JK \\ FF};
\draw [gtu arrow] ([yshift=0.5cm]jk.west) -- ++(-0.5,0) node[left] {J};
\draw [gtu arrow] ([yshift=-0.5cm]jk.west) -- ++(-0.5,0) node[left] {K};
\draw [gtu arrow] (jk.west) -- ++(-0.5,0) node[left] {CLK};
\draw [gtu arrow] (jk.east) -- ++(0.5,0) node[right] {Q};
\draw [gtu arrow] ([yshift=-0.5cm]jk.east) -- ++(0.5,0) node[right] {Q'};
\end{tikzpicture}
\end{center}

\textbf{ટ્રુથ ટેબલ:}
\begin{center}
\begin{tabulary}{\linewidth}{|C|C|C|L|}
\hline
J & K & Q (આગામી) & ઓપરેશન \\
\hline
0 & 0 & Q (અગાઉની) & કોઈ ફેરફાર નહીં \\
\hline
0 & 1 & 0 & રીસેટ \\
\hline
1 & 0 & 1 & સેટ \\
\hline
1 & 1 & Q' (અગાઉની) & ટોગલ \\
\hline
\end{tabulary}
\end{center}
\end{solutionbox}
\mnemonicbox{"J સેટ કરે, K રીસેટ કરે, બંને ટોગલ કરે, કોઈ નહીં યાદ રાખે"}

\questionmarks{4 [OR] (ક)(II)}{4}{J-K ફ્લિપ-ફ્લોપનો ઉપયોગ કરી D ફ્લિપ-ફ્લોપ અને T ફ્લિપ-ફ્લોપની લૉજિક સર્કિટ દોરો}


\begin{solutionbox}
\textbf{JK ફ્લિપ-ફ્લોપનો ઉપયોગ કરી D ફ્લિપ-ફ્લોપ:}
\begin{itemize}
\item D ઇનપુટને J સાથે જોડો
\item D' (NOT D)ને K સાથે જોડો
\end{itemize}

\textbf{લૉજિક સર્કિટ:}
\begin{center}
\begin{circuitikz}
\draw (2,1.5) node[gtu block, minimum width=2cm, minimum height=2cm] (jk) {JK FF};
\draw (0,2) node (D) {D};
\draw (0,1) node[not port, scale=0.7] (not) {};

\draw (D) -- (not.in);
\draw (D) -| ($(jk.west)+(0,0.5)$) -- ($(jk.west)+(0,0.5)$); % J
\draw (not.out) -| ($(jk.west)+(0,-0.5)$) -- ($(jk.west)+(0,-0.5)$); % K

\draw (jk.east) -- ++(1,0) node[right] {Q};
\end{circuitikz}
\end{center}

\textbf{JK ફ્લિપ-ફ્લોપનો ઉપયોગ કરી T ફ્લિપ-ફ્લોપ:}
\begin{itemize}
\item T ઇનપુટને J અને K બંને સાથે જોડો
\end{itemize}

\textbf{લૉજિક સર્કિટ:}
\begin{center}
\begin{circuitikz}
\draw (2,1.5) node[gtu block, minimum width=2cm, minimum height=2cm] (jk) {JK FF};
\draw (0,1.5) node (T) {T};

\draw (T) -- ++(0.5,0) coordinate (split);
\draw (split) |- ($(jk.west)+(0,0.5)$); % J
\draw (split) |- ($(jk.west)+(0,-0.5)$); % K

\draw (jk.east) -- ++(1,0) node[right] {Q};
\end{circuitikz}
\end{center}
\end{solutionbox}
\mnemonicbox{"D સીધું અનુસરે, T સાચું હોય ત્યારે ટોગલ થાય"}

\questionmarks{5(અ)}{3}{RAM અને ROMની સરખામણી કરો}


\begin{solutionbox}
\textbf{RAM (Random Access Memory) vs ROM (Read-Only Memory):}

\captionof{table}{RAM vs ROM સરખામણી}
\begin{center}
\begin{tabulary}{\linewidth}{|L|L|L|}
\hline
લક્ષણ & RAM & ROM \\
\hline
**પૂર્ણ નામ** & Random Access Memory & Read-Only Memory \\
\hline
**ડેટા નિભાવણી** & અસ્થાયી (પાવર બંધ થતાં ડેટા ગુમાવે) & સ્થાયી (પાવર વિના પણ ડેટા જળવાય) \\
\hline
**વાંચન/લેખન ક્ષમતા** & વાંચન અને લેખન બંને & મુખ્યત્વે માત્ર વાંચન \\
\hline
**ગતિ** & વધુ ઝડપી & ધીમી \\
\hline
**બિટ દીઠ ખર્ચ** & વધુ & ઓછો \\
\hline
**ઉપયોગો** & અસ્થાયી ડેટા સ્ટોરેજ & બૂટ સૂચનાઓ, ફર્મવેર \\
\hline
\end{tabulary}
\end{center}
\end{solutionbox}
\mnemonicbox{"RAM વાંચે અને સુધારે (પણ ભૂલી જાય), ROM શટડાઉન પર યાદ રાખે (પણ નિશ્ચિત)"}

\questionmarks{5(બ)}{4}{સિરિયલ ઇન સિરિયલ આઉટ શિફ્ટ રજીસ્ટર સમજાવો}


\begin{solutionbox}
\textbf{સિરિયલ ઇન સિરિયલ આઉટ (SISO) શિફ્ટ રજિસ્ટર:} એક સિક્વન્સીયલ સર્કિટ જે ઇનપુટ અને આઉટપુટ બંને પર ડેટાને એક સમયે એક બિટ શિફ્ટ કરે છે.

\textbf{કાર્યપદ્ધતિ:}
\begin{itemize}
\item ડેટા સિરિયલી એક બિટ એક વખતે દાખલ થાય છે
\item દરેક ક્લોક પલ્સ પર દરેક બિટ રજિસ્ટરમાંથી શિફ્ટ થાય છે
\item ડેટા સિરિયલી એક બિટ એક વખતે બહાર નીકળે છે
\end{itemize}

\textbf{બ્લોક ડાયાગ્રામ:}
\begin{center}
\begin{tikzpicture}[node distance=2.5cm]
\node [gtu block] (ff0) {D0};
\node [gtu block, right of=ff0] (ff1) {D1};
\node [gtu block, right of=ff1] (ff2) {D2};
\node [gtu block, right of=ff2] (ff3) {D3};

\draw [gtu arrow] ([yshift=0.5cm]ff0.west) -- (ff0.west) node[midway, above] {CLK};
\draw [gtu arrow] ([xshift=-1cm]ff0.west) -- (ff0.west) node[pos=0, left] {Serial In};

\draw [gtu arrow] (ff0.east) -- (ff1.west);
\draw [gtu arrow] (ff1.east) -- (ff2.west);
\draw [gtu arrow] (ff2.east) -- (ff3.west);
\draw [gtu arrow] (ff3.east) -- ++(1,0) node[right] {Serial Out};
\end{tikzpicture}
\end{center}
\end{solutionbox}
\mnemonicbox{"બિટ્સ લાઇનમાં પ્રવેશે, શ્રેણીમાં આગળ વધે, ક્રમમાં બહાર નીકળે"}

\questionmarks{5(ક)}{7}{લૉજિક ફેમિલિઝ પર ટૂંક નોંધ લખો}


\begin{solutionbox}
\textbf{લૉજિક ફેમિલિઝ:} સમાન ઇલેક્ટ્રિકલ લક્ષણો, ફેબ્રિકેશન ટેકનોલોજી અને લૉજિક અમલીકરણ સાથેના ડિજિટલ ઇન્ટિગ્રેટેડ સર્કિટના સમૂહો.

\textbf{મુખ્ય લૉજિક ફેમિલિઝ:}

\textbf{1. TTL (ટ્રાન્ઝિસ્ટર-ટ્રાન્ઝિસ્ટર લૉજિક):}
\begin{itemize}
\item બાયપોલર જંક્શન ટ્રાન્ઝિસ્ટર પર આધારિત
\item Supply voltage: 5V
\end{itemize}

\textbf{2. CMOS (કોમ્પ્લિમેન્ટરી મેટલ-ઓક્સાઇડ-સેમિકન્ડક્ટર):}
\begin{itemize}
\item MOSFETs (P-ટાઇપ અને N-ટાઇપ) પર આધારિત
\item ખૂબ ઓછો પાવર વપરાશ
\end{itemize}

\textbf{3. ECL (ઇમિટર-કપલ્ડ લૉજિક):}
\begin{itemize}
\item અત્યંત ઊંચી ઝડપ (સૌથી ઝડપી લૉજિક ફેમિલી)
\end{itemize}

\captionof{table}{સરખામણી કોષ્ટક}
\begin{center}
\begin{tabulary}{\linewidth}{|L|L|L|L|}
\hline
પેરામીટર & TTL & CMOS & ECL \\
\hline
ઝડપ & મધ્યમ & ઓછી થી ઊંચી & ખૂબ ઊંચી \\
\hline
પાવર વપરાશ & મધ્યમ & ખૂબ ઓછો & ઊંચો \\
\hline
નોઇઝ ઇમ્યુનિટી & ઊંચી & ખૂબ ઊંચી & ઓછી \\
\hline
ફેન-આઉટ & 10 & 50+ & 25 \\
\hline
\end{tabulary}
\end{center}
\end{solutionbox}
\mnemonicbox{"TTL ટ્રાન્ઝિસ્ટર ટેક્નોલોજી, CMOS કરંટ ઓછો વાપરે છે, ECL એક્સટ્રીમ ઝડપે કામ કરે છે"}

\questionmarks{5 [OR] (અ)}{3}{SRAM અને DRAMની સરખામણી કરો}


\begin{solutionbox}
\textbf{SRAM (સ્ટેટિક RAM) vs DRAM (ડાયનેમિક RAM):}

\captionof{table}{SRAM vs DRAM સરખામણી}
\begin{center}
\begin{tabulary}{\linewidth}{|L|L|L|}
\hline
લક્ષણ & SRAM & DRAM \\
\hline
**પૂર્ણ નામ** & Static Random Access Memory & Dynamic Random Access Memory \\
\hline
**સ્ટોરેજ એલિમેન્ટ** & ફ્લિપ-ફ્લોપ & કેપેસિટર \\
\hline
**રિફ્રેશિંગ** & જરૂરી નથી & સમયાંતરે જરૂરી (ms) \\
\hline
**ઝડપ** & વધુ ઝડપી & ધીમી \\
\hline
**ડેન્સિટી** & ઓછી (મોટો સેલ સાઇઝ) & ઊંચી (નાનો સેલ સાઇઝ) \\
\hline
**બિટ દીઠ ખર્ચ** & વધુ & ઓછો \\
\hline
**ઉપયોગો** & કેશ મેમરી & મુખ્ય મેમરી (RAM) \\
\hline
\end{tabulary}
\end{center}
\end{solutionbox}
\mnemonicbox{"સ્ટેટિક સ્થિર રહે છે છ ટ્રાન્ઝિસ્ટર સાથે, ડાયનેમિક ડ્રેઇન થાય અને નિયમિત રિફ્રેશ જોઈએ"}

\questionmarks{5 [OR] (બ)}{4}{8:3 એનકોડર સમજાવો}


\begin{solutionbox}
\textbf{8:3 એનકોડર:} એક કોમ્બીનેશનલ સર્કિટ જે 8 ઇનપુટ લાઇન્સને 3 આઉટપુટ લાઇન્સમાં રૂપાંતરિત કરે છે, મૂળભૂત રીતે સક્રિય ઇનપુટ લાઇનને તેની બાયનરી પોઝિશનમાં રૂપાંતરિત કરે છે.

\textbf{લૉજિક સર્કિટ:}
\begin{center}
\begin{circuitikz}[scale=0.8]
\draw (2, 3) node[or port, number inputs=4] (or0) {};
\draw (2, 0) node[or port, number inputs=4] (or1) {};
\draw (2, -3) node[or port, number inputs=4] (or2) {};

\draw (or0.out) -- ++(0.5,0) node[right] {$Y_0$};
\draw (or1.out) -- ++(0.5,0) node[right] {$Y_1$};
\draw (or2.out) -- ++(0.5,0) node[right] {$Y_2$};

% Inputs simplified
\node at (0,3) {Inputs $I_1, I_3, I_5, I_7$};
\node at (0,0) {Inputs $I_2, I_3, I_6, I_7$};
\node at (0,-3) {Inputs $I_4, I_5, I_6, I_7$};
\end{circuitikz}
\end{center}

\textbf{બુલિયન સમીકરણ:}
\begin{itemize}
\item $Y_0 = I_1 + I_3 + I_5 + I_7$
\item $Y_1 = I_2 + I_3 + I_6 + I_7$
\item $Y_2 = I_4 + I_5 + I_6 + I_7$
\end{itemize}
\end{solutionbox}
\mnemonicbox{"આઠ ઇનપુટ તેમના સ્થાન ત્રણ બિટમાં બને"}

\questionmarks{5 [OR] (ક)}{7}{લૉજિક ફેમિલિઝ માટે નીચેની વ્યાખ્યાઓ લખો (i) ફેન-ઇન (ii) ફેન-આઉટ (iii) નોઇસ માર્જિન (iv) પ્રોપેગેશન ડિલે (v) પાવર ડિસિપેશન}


\begin{solutionbox}
\textbf{લૉજિક ફેમિલિઝના મુખ્ય પેરામીટર:}

\begin{enumerate}
\item \textbf{ફેન-ઇન:} લૉજિક ગેટ સ્વીકારી શકે તેવા ઇનપુટની મહત્તમ સંખ્યા
\item \textbf{ફેન-આઉટ:} એક ગેટ આઉટપુટ દ્વારા વિશ્વસનીય રીતે ડ્રાઇવ થઈ શકતા સમાન ગેટની મહત્તમ સંખ્યા
\item \textbf{નોઇઝ માર્જિન:} અનિચ્છનીય ઇલેક્ટ્રિકલ નોઇઝ/સિગ્નલ સહન કરવાની ક્ષમતા
\item \textbf{પ્રોપેગેશન ડિલે:} ઇનપુટ ચેન્જ અને તેના તરત પછીના આઉટપુટ ચેન્જ વચ્ચેનો સમય વિલંબ
\item \textbf{પાવર ડિસિપેશન:} ગેટ દ્વારા વપરાતી પાવરની માત્રા
\end{enumerate}
\end{solutionbox}
\mnemonicbox{"પાંચ ફેક્ટર: ફેન-ઇન ઇનપુટ ગણે, ફેન-આઉટ ગેટ ચલાવે, નોઇઝ માર્જિન દખલ સામે લડે, પ્રોપેગેશન ડિલે ઝડપ માપે, પાવર ડિસિપેશન ગરમી ઉત્પન્ન કરે"}

\end{document}


