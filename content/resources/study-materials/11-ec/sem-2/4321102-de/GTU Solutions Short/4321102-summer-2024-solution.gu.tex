\documentclass{article}

% content/resources/templates/preamble.tex
\usepackage[margin=0.6in]{geometry}
\author{Milav Dabgar}
\usepackage{amsmath,amssymb,amsthm}
\usepackage{booktabs}
\usepackage{multirow}
\usepackage{xcolor}
\usepackage{tcolorbox}
\tcbuselibrary{breakable,skins}
\usepackage[colorlinks=true,linkcolor=blue]{hyperref}
\usepackage{titlesec}
\usepackage{enumitem}
\usepackage{tikz}
\usepackage{pgfplots}
\usepackage{circuitikz}
\usepackage[version=4]{mhchem}
\usepackage{longtable}
\usepackage{array}
\usepackage{float}
\usepackage{caption}
\usepackage{listings}

\lstset{
  basicstyle=\small\ttfamily,
  breaklines=true,
  breakatwhitespace=false,
  postbreak=\mbox{\textcolor{red}{$\hookrightarrow$}\space},
  float=false,
  numbers=left,
  numberstyle=\tiny\color{gray},
  numbersep=10pt,
  xleftmargin=2em,
  keywordstyle=\color{blue},
  commentstyle=\color{green!60!black},
  stringstyle=\color{purple},
  backgroundcolor=\color{gray!5},
  showstringspaces=false,
  tabsize=2,
  captionpos=b,
  keepspaces=true,
  columns=flexible
}

\pgfplotsset{compat=1.18}
\usetikzlibrary{shapes,arrows,positioning,calc,patterns,decorations.pathmorphing,decorations.markings,arrows.meta}

% Color scheme
\definecolor{headcolor}{RGB}{0,102,204}
\definecolor{keycolor}{RGB}{220,20,60}
\definecolor{solutioncolor}{RGB}{34,139,34}
\definecolor{mnemoniccolor}{RGB}{148,0,211}
\definecolor{codecolor}{RGB}{0,0,100}

% Spacing
\setlength{\parskip}{3pt}
\setlist[itemize]{nosep}
\setlist[enumerate]{nosep}

% Title formatting
\titleformat{\section}{\Large\bfseries\color{headcolor}}{\thesection}{1em}{}
\titleformat{\subsection}{\large\bfseries\color{headcolor}}{\thesubsection}{1em}{}

% Pandoc tightlist compatibility
\providecommand{\tightlist}{%
  \setlength{\itemsep}{0pt}\setlength{\parskip}{0pt}}

% Pandoc longtable compatibility
\newcounter{none}
\def\thenone{}


% content/resources/templates/gujarati-boxes.tex
\usepackage{fontspec}
\usepackage{polyglossia}

% Set Gujarati as main language (document is primarily in Gujarati)
% Note: gloss-gujarati.ldf doesn't exist in polyglossia, but it will use hyphenation patterns
\setdefaultlanguage{gujarati}
\setotherlanguage{english}

% Configure Gujarati font properly
% Use Language=Default to prevent polyglossia from trying to add language-specific features
% that don't exist for Gujarati, which causes "empty feature" warnings
\newfontfamily\gujaratifont[Script=Gujarati,AutoFakeBold=2.5,AutoFakeSlant=0.3]{Noto Sans Gujarati}
\setmainfont[Script=Gujarati,AutoFakeBold=2.5,AutoFakeSlant=0.3]{Noto Sans Gujarati}
% Use Noto Sans Gujarati for monospace to support Gujarati in text
\setmonofont[Scale=0.9]{Noto Sans Gujarati}

% Configure English to use the same font
\newfontfamily\englishfont[Script=Gujarati,AutoFakeBold=2.5,AutoFakeSlant=0.3]{Noto Sans Gujarati}

% Translations for polyglossia
\gappto\captionsgujarati{
  \renewcommand{\tablename}{કોષ્ટક}
  \renewcommand{\figurename}{આકૃતિ}
}

% Helper for TikZ nodes to ensure Gujarati font
\newcommand{\gu}[1]{{\gujaratifont #1}}

% Custom environments
\newtcolorbox{solutionbox}{
    breakable,
    enhanced,
    colback=solutioncolor!5!white,
    colframe=solutioncolor!75!black,
    fonttitle=\bfseries,
    title=જવાબ
}

\newtcolorbox{solutionboxnobreak}{
 colback=solutioncolor!5!white,
 colframe=solutioncolor!75!black,
 fonttitle=\bfseries,
 title=જવાબ
}

\newtcolorbox{keyformula}{
 breakable,
 enhanced,
 colback=keycolor!5!white,
 colframe=keycolor!75!black,
 fonttitle=\bfseries,
 title=રાસાયણિક સમીકરણ/સૂત્ર
}

\newtcolorbox{mnemonicbox}{
 breakable,
 enhanced,
 colback=mnemoniccolor!5!white,
 colframe=mnemoniccolor!75!black,
 fonttitle=\bfseries,
 title=મેમરી ટ્રીક
}


% Custom commands for GTU solutions
% This file defines semantic commands for consistent formatting

% Question command with automatic formatting
\newcommand{\question}[2]{%
  \section*{Question #1}%
  \textbf{#2}%
}

% OR question variant
\newcommand{\questionor}[2]{%
  \section*{Question #1 OR}%
  \textbf{#2}%
}

% Proper table environment with caption
\newenvironment{answertable}[1]{%
  \begin{table}[htbp]
  \centering
  \caption{#1}
}{%
  \end{table}
}

% Proper figure environment for diagrams
\newenvironment{answerdiagram}[1]{%
  \begin{figure}[htbp]
  \centering
  \caption{#1}
}{%
  \end{figure}
}

% Semantic markup for key terms
\newcommand{\keyword}[1]{\textbf{#1}}
\newcommand{\code}[1]{\texttt{#1}}
\newcommand{\classname}[1]{\texttt{#1}}
\newcommand{\methodname}[1]{\texttt{#1}}

% Proper quotation marks
\newcommand{\mnemonic}[1]{``#1''}

\usetikzlibrary{matrix}

\title{Digital Electronics (4321102) - Summer 2024 Solution}
\date{June 20, 2024}

\begin{document}
\maketitle

\questionmarks{1(a)}{3}{કન્વર્ટ કરો: (110101)$_2$ = ( \_\_\_ )$_{10}$ = ( \_\_\_ )$_8$ = ( \_\_\_ )$_{16}$}

\begin{solutionbox}
\textbf{સ્ટેપ-બાય-સ્ટેપ કન્વર્ઝન (110101)$_2$}:

\captionof{table}{બાઇનરી કન્વર્ઝન}
\begin{center}
\begin{tabulary}{\linewidth}{|C|C|C|C|}
\hline
બાઇનરી (110101)$_2$ & ડેસિમલ & ઑક્ટલ & હેક્ઝાડેસિમલ \\
\hline
$1\times2^5 + 1\times2^4 + 0\times2^3 + 1\times2^2 + 0\times2^1 + 1\times2^0$ & $32+16+0+4+0+1 = 53$ & $6\times8^1 + 5\times8^0 = 48+5 = 53$ & $3\times16^1 + 5\times16^0 = 48+5 = 35$ \\
\hline
(110101)$_2$ & (53)$_{10}$ & (65)$_8$ & (35)$_{16}$ \\
\hline
\end{tabulary}
\end{center}
\end{solutionbox}
\mnemonicbox{"બાઇનરી ડિજિટ આઉટ હિયર" (BDOH) બાઇનરી$\to$ડેસિમલ$\to$ઑક્ટલ$\to$હેક્ઝાડેસિમલ કન્વર્ઝન માટે.}

\questionmarks{1(b)}{4}{કરો: (i) (11101101)$_2$+(10101000)$_2$ (ii) (11011)$_2$*(1010)$_2$}

\begin{solutionbox}
\textbf{બાઇનરી સરવાળા અને ગુણાકાર માટે ટેબલ}:

\captionof{table}{બાઇનરી સરવાળો અને ગુણાકાર}
\begin{center}
\begin{tabular}{|p{0.45\linewidth}|p{0.45\linewidth}|}
\hline
(i) બાઇનરી સરવાળો & (ii) બાઇનરી ગુણાકાર \\
\hline
\begin{lstlisting}[basicstyle=\ttfamily]
  11101101
+ 10101000
----------
 110010101
\end{lstlisting} & 
\begin{lstlisting}[basicstyle=\ttfamily]
    11011
 ×   1010
---------
    00000
   11011
  00000
 11011
---------
 11101110
\end{lstlisting} \\
\hline
\end{tabular}
\end{center}

\textbf{ડેસિમલ વેરિફિકેશન}:
\begin{itemize}
\item (i) $(11101101)_2 = 237$, $(10101000)_2 = 168$, સરવાળો = 405 = $(110010101)_2$
\item (ii) $(11011)_2 = 27$, $(1010)_2 = 10$, ગુણાકાર = 270 = $(11101110)_2$
\end{itemize}
\end{solutionbox}
\mnemonicbox{સરવાળા માટે "કેરી અપ મેક્સ સમ" અને ગુણાકાર માટે "શિફ્ટ લેફ્ટ એડ પ્રોડક્ટ".}

\questionmarks{1(c)}{7}{(i) કન્વર્ટ કરો: (48)$_{10}$ = ( \_\_\_ )$_2$ = ( \_\_\_ )$_8$ = ( \_\_\_ )$_{16}$ \\ (ii) 2's Complement પદ્ધતિનો ઉપયોગ કરીને બાદબાકી કરો: (1110)$_2$ -- (1000)$_2$ \\ (iii) (1111101)$_2$ ને (101)$_2$ વડે વિભાજિત કરો.}

\begin{solutionbox}
\textbf{(i) કન્વર્ઝન ટેબલ}:

\captionof{table}{ડેસિમલ (48) કન્વર્ઝન}
\begin{center}
\begin{tabulary}{\linewidth}{|C|C|C|C|}
\hline
ડેસિમલ (48)$_{10}$ & બાઇનરી & ઑક્ટલ & હેક્ઝાડેસિમલ \\
\hline
$48\div2 = 24$ રેમ 0 & 110000 & 60 & 30 \\
\hline
$24\div2 = 12$ રેમ 0 & & & \\
\hline
$12\div2 = 6$ રેમ 0 & & & \\
\hline
$6\div2 = 3$ રેમ 0 & & & \\
\hline
$3\div2 = 1$ રેમ 1 & & & \\
\hline
$1\div2 = 0$ રેમ 1 & & & \\
\hline
(48)$_{10}$ & (110000)$_2$ & (60)$_8$ & (30)$_{16}$ \\
\hline
\end{tabulary}
\end{center}

\textbf{(ii) બાદબાકી ટેબલ}:

\captionof{table}{2's Complement બાદબાકી}
\begin{center}
\begin{tabulary}{\linewidth}{|L|L|}
\hline
2's Complement પદ્ધતિ & સ્ટેપ્સ \\
\hline
$(1110)_2 - (1000)_2$ & 1. (1000)$_2$ નો 2's complement શોધો \\
\hline
(1000)$_2$ નો 1's complement & (0111)$_2$ \\
\hline
2's complement & (0111)$_2 + 1 = (1000)_2$ \\
\hline
$(1110)_2 + (1000)_2$ & (10110)$_2$ \\
\hline
કેરી દૂર કરો & (0110)$_2$ \\
\hline
પરિણામ & (0110)$_2 = 6_{10}$ \\
\hline
\end{tabulary}
\end{center}

\textbf{(iii) ભાગાકાર}:

\begin{center}
\begin{lstlisting}[basicstyle=\ttfamily]
     11001
   -------
101)1111101
    101
    ---
    0101
     101
    ----
     0000
      000
     ----
       001
       000
       ---
         1
\end{lstlisting}
ભાગફળ = $(11)_2$, શેષ = $(1)_2$
\end{center}
\end{solutionbox}
\mnemonicbox{લાંબા ભાગાકાર પ્રક્રિયા માટે "ડિવિઝન ડ્રોપ્સ ડાઉન રિમેન્ડર્સ".}

\questionmarks{1(c OR)}{7}{કોડ્સ સમજાવો: ASCII, BCD, Gray}

\begin{solutionbox}
\textbf{સામાન્ય ડિજિટલ કોડ્સનું ટેબલ}:

\captionof{table}{સામાન્ય ડિજિટલ કોડ્સ}
\begin{center}
\begin{tabulary}{\linewidth}{|L|L|L|}
\hline
કોડ & વર્ણન & ઉદાહરણ \\
\hline
\textbf{ASCII (American Standard Code for Information Interchange)} & 128 કેરેક્ટર્સને રજૂ કરતો 7-બિટ કોડ જેમાં આલ્ફાબેટ્સ, નંબર્સ અને સ્પેશિયલ સિમ્બોલ્સ શામેલ છે & A = 65 (1000001)$_2$ \\
\hline
\textbf{BCD (Binary Coded Decimal)} & દરેક ડેસિમલ અંક (0-9) ને 4 બિટ્સનો ઉપયોગ કરીને રજૂ કરે છે & 42 = 0100 0010 \\
\hline
\textbf{Gray Code} & બાઇનરી કોડ જેમાં આસપાસના નંબરો માત્ર એક બિટથી અલગ પડે છે & (0,1,3,2) = (00,01,11,10) \\
\hline
\end{tabulary}
\end{center}

\textbf{ડાયાગ્રામ: ગ્રે કોડ જનરેશન}:

\begin{center}
\begin{tikzpicture}[node distance=2cm, auto]
    \node [gtu block] (bin) {બાઇનરી કોડ};
    \node [gtu block, right of=bin, xshift=2cm] (gray) {ગ્રે કોડ};
    
    \draw [gtu arrow] (bin) -- (gray);
    
    \node [align=center, below of=bin, yshift=-0.5cm] (ex_bin) {Binary: 0011};
    \node [align=center, below of=gray, yshift=-0.5cm] (ex_gray) {Gray: 0010};
    
    \draw [gtu arrow, dashed] (ex_bin) -- (ex_gray) node[midway, above] {XOR};
\end{tikzpicture}
\captionof{figure}{ગ્રે કોડ કન્સેપ્ટ}
\end{center}
\end{solutionbox}
\mnemonicbox{"ઓલવેઝ બાઇનરી જનરેટ્સ" - દરેક કોડનો પ્રથમ અક્ષર (ASCII, BCD, Gray).}

\questionmarks{2(a)}{3}{બુલિયન બીજગણિતનો ઉપયોગ કરીને સરળ બનાવો: Y = A B + A' B + A' B' + A B'}

\begin{solutionbox}
\textbf{સ્ટેપ-બાય-સ્ટેપ સરળીકરણ}:

\captionof{table}{બુલિયન સરળીકરણ}
\begin{center}
\begin{tabulary}{\linewidth}{|L|L|L|}
\hline
સ્ટેપ & એક્સપ્રેશન & બુલિયન નિયમ \\
\hline
$Y = A B + A' B + A' B' + A B'$ & પ્રારંભિક એક્સપ્રેશન & - \\
\hline
$Y = A(B + B') + A'(B + B')$ & ફેક્ટરિંગ & ડિસ્ટ્રિબ્યુટિવ લૉ \\
\hline
$Y = A(1) + A'(1)$ & કોમ્પ્લિમેન્ટ લૉ & $B + B' = 1$ \\
\hline
$Y = A + A'$ & સરળીકરણ & - \\
\hline
$Y = 1$ & કોમ્પ્લિમેન્ટ લૉ & $A + A' = 1$ \\
\hline
\end{tabulary}
\end{center}
\end{solutionbox}
\mnemonicbox{બુલિયન સરળીકરણ સ્ટેપ્સ માટે "ફેક્ટર, સિમ્પ્લિફાય, ફિનિશ".}

\questionmarks{2(b)}{4}{K-મેપનો ઉપયોગ કરીને નીચેના બુલિયન ફંક્શન ને સરળ બનાવો: f(A,B,C,D) = $\Sigma$m (0,3,4,6,8,11,12)}

\begin{solutionbox}
\textbf{K-મેપ સોલ્યુશન}:

\begin{center}
\begin{tikzpicture}
\matrix [matrix of nodes, draw, nodes in empty cells] (table) {
& 00 & 01 & 11 & 10 \\
00 & 1 & 0 & 0 & 1 \\
01 & 0 & 0 & 0 & 1 \\
11 & 0 & 1 & 0 & 0 \\
10 & 0 & 0 & 1 & 0 \\
};
\node[left=0.5cm of table-2-1] {CD 00};
\node[left=0.5cm of table-3-1] {01};
\node[left=0.5cm of table-4-1] {11};
\node[left=0.5cm of table-5-1] {10};
\node[above=0.5cm of table-1-2] {00};
\node[above=0.5cm of table-1-3] {01};
\node[above=0.5cm of table-1-4] {11};
\node[above=0.5cm of table-1-5] {10};
\node[above left=0.5cm of table-1-2] {AB};
\end{tikzpicture}
\end{center}

\textbf{ગ્રુપિંગ}:
\begin{itemize}
\item ગ્રુપ 1: m(0,8) = $A'C'D'$
\item ગ્રુપ 2: m(4,12) = $BD'$
\item ગ્રુપ 3: m(3,11) = $CD$
\item ગ્રુપ 4: m(6) = $A'B'CD'$
\end{itemize}

\textbf{સરળ કરેલ એક્સપ્રેશન}: $f(A,B,C,D) = A'C'D' + BD' + CD + A'B'CD'$
\end{solutionbox}
\mnemonicbox{K-મેપ ગ્રુપિંગ સ્ટ્રેટેજી માટે "ગ્રુપ પાવર્સ ઓફ ટુ".}

\questionmarks{2(c)}{7}{NOR ગેટને સ્વચ્છ આકૃતિઓ સાથે યુનિવર્સલ ગેટ તરીકે સમજાવો.}

\begin{solutionbox}
\textbf{NOR એઝ યુનિવર્સલ ગેટ}:

\captionof{table}{NOR ઇમ્પ્લિમેન્ટશન}
\begin{center}
\begin{tabulary}{\linewidth}{|L|L|L|}
\hline
ફંક્શન & NOR નો ઉપયોગ કરી ઇમ્પ્લિમેન્ટેશન & ટ્રુથ ટેબલ \\
\hline
\textbf{NOT ગેટ} & 
\begin{tikzpicture}[scale=0.6, transform shape, baseline]
    \node [nor port] (nor) {};
    \draw (nor.in 1) -- ++(-0.5,0) coordinate (A) node[left] {A};
    \draw (nor.in 2) -- ++(-0.5,0) coordinate (B);
    \draw (A) |- (B);
    \draw (nor.out) -- ++(0.5,0) node[right] {A'};
\end{tikzpicture} & 
\begin{tabular}{c|c}
A & A' \\ \hline 0 & 1 \\ 1 & 0
\end{tabular} \\
\hline
\textbf{AND ગેટ} & 
\begin{tikzpicture}[scale=0.5, transform shape, baseline]
    \node [nor port] (nor1) at (0,1) {};
    \node [nor port] (nor2) at (0,-1) {};
    \node [nor port] (nor3) at (3,0) {};
    \draw (nor1.in 1) -- ++(-0.2,0) coordinate (A); \draw (nor1.in 2) -- ++(-0.2,0) coordinate (B); \draw (A) -- (B) node[midway, left] {A};
    \draw (nor2.in 1) -- ++(-0.2,0) coordinate (C); \draw (nor2.in 2) -- ++(-0.2,0) coordinate (D); \draw (C) -- (D) node[midway, left] {B};
    \draw (nor1.out) -- (nor3.in 1);
    \draw (nor2.out) -- (nor3.in 2);
    \draw (nor3.out) -- ++(0.5,0) node[right] {AB};
\end{tikzpicture} & 
\begin{tabular}{cc|c}
A & B & AB \\ \hline 0 & 0 & 0 \\ 0 & 1 & 0 \\ 1 & 0 & 0 \\ 1 & 1 & 1
\end{tabular} \\
\hline
\textbf{OR ગેટ} & 
\begin{tikzpicture}[scale=0.5, transform shape, baseline]
    \node [nor port] (nor1) at (0,0) {};
    \node [nor port] (nor2) at (3,0) {};
    \draw (nor1.in 1) -- ++(-0.5,0) node[left] {A};
    \draw (nor1.in 2) -- ++(-0.5,0) node[left] {B};
    \draw (nor1.out) -- (nor2.in 1);
    \draw (nor2.in 2) -| ($(nor1.out)+(0.5,0)$);
    \draw (nor2.out) -- ++(0.5,0) node[right] {A+B};
\end{tikzpicture} & 
\begin{tabular}{cc|c}
A & B & A+B \\ \hline 0 & 0 & 0 \\ 0 & 1 & 1 \\ 1 & 0 & 1 \\ 1 & 1 & 1
\end{tabular} \\
\hline
\end{tabulary}
\end{center}
\end{solutionbox}
\mnemonicbox{NOR ગેટ ઇમ્પ્લિમેન્ટેશન માટે "NOT AND OR, NOR કરે મોર".}

\questionmarks{2(a OR)}{3}{બુલિયન સમીકરણ માટે લોજિક સર્કિટ દોરો: Y = (A + B') . (A' + B') . (B + C)}

\begin{solutionbox}
\textbf{લોજિક સર્કિટ ઇમ્પ્લિમેન્ટેશન}:

\begin{center}
\begin{circuitikz}[scale=0.8, transform shape]
    \node [or port] (or1) at (2, 3) {};
    \node [or port] (or2) at (2, 0) {};
    \node [or port] (or3) at (2, -3) {};
    \node [and port, number inputs=3] (and) at (6, 0) {};
    \draw (or1.in 1) -- ++(-1,0) node[left] {A};
    \draw (or1.in 2) -- ++(-1,0) node[left] {B'};
    \draw (or2.in 1) -- ++(-1,0) node[left] {A'};
    \draw (or2.in 2) -- ++(-1,0) node[left] {B'};
    \draw (or3.in 1) -- ++(-1,0) node[left] {B};
    \draw (or3.in 2) -- ++(-1,0) node[left] {C};
    \draw (or1.out) -- (and.in 1);
    \draw (or2.out) -- (and.in 2);
    \draw (or3.out) -- (and.in 3);
    \draw (and.out) -- ++(0.5,0) node[right] {Y};
\end{circuitikz}
\end{center}

\textbf{ટ્રુથ ટેબલ વેરિફિકેશન}:
\begin{itemize}
\item ટર્મ 1: $(A + B')$
\item ટર્મ 2: $(A' + B')$
\item ટર્મ 3: $(B + C)$
\item આઉટપુટ: Y = Term1 $\cdot$ Term2 $\cdot$ Term3
\end{itemize}
\end{solutionbox}
\mnemonicbox{જટિલ એક્સપ્રેશન માટે "દરેક ટર્મ અલગથી".}

\questionmarks{2(b OR)}{4}{ડી-મોર્ગન્સના પ્રમેય લખો અને તેને સાબિત કરો.}

\begin{solutionbox}
\textbf{ડી-મોર્ગન્સ પ્રમેય અને પ્રૂફ}:

\captionof{table}{ડી-મોર્ગન્સ પ્રમેય}
\begin{center}
\begin{tabulary}{\linewidth}{|L|L|L|}
\hline
પ્રમેય & સ્ટેટમેન્ટ & ટ્રુથ ટેબલ દ્વારા પ્રૂફ \\
\hline
\textbf{પ્રમેય 1} & $(A\cdot B)' = A' + B'$ & 
\begin{tabular}{cc|c|c|cc|c}
A & B & AB & (AB)' & A' & B' & A'+B' \\ \hline
0 & 0 & 0 & 1 & 1 & 1 & 1 \\
0 & 1 & 0 & 1 & 1 & 0 & 1 \\
1 & 0 & 0 & 1 & 0 & 1 & 1 \\
1 & 1 & 1 & 0 & 0 & 0 & 0
\end{tabular} \\
\hline
\textbf{પ્રમેય 2} & $(A+B)' = A'\cdot B'$ & 
\begin{tabular}{cc|c|c|cc|c}
A & B & A+B & (A+B)' & A' & B' & A'B' \\ \hline
0 & 0 & 0 & 1 & 1 & 1 & 1 \\
0 & 1 & 1 & 0 & 1 & 0 & 0 \\
1 & 0 & 1 & 0 & 0 & 1 & 0 \\
1 & 1 & 1 & 0 & 0 & 0 & 0
\end{tabular} \\
\hline
\end{tabulary}
\end{center}
\end{solutionbox}
\mnemonicbox{ડી-મોર્ગન્સ લૉ લાગુ કરવા માટે "બાર તોડો, ઓપરેશન બદલો, ઇનપુટ ઇન્વર્ટ કરો".}

\questionmarks{2(c OR)}{7}{સિમ્બોલ, ટ્રુથ ટેબલ અને સમીકરણની મદદથી તમામ લોજિક ગેટ્સ સમજાવો.}

\begin{solutionbox}
\textbf{લોજિક ગેટ્સ સમરી}:

\captionof{table}{લોજિક ગેટ્સ}
\begin{center}
\begin{tabulary}{\linewidth}{|L|L|L|L|L|}
\hline
ગેટ & સિમ્બોલ & ટ્રુથ ટેબલ & સમીકરણ & વર્ણન \\
\hline
\textbf{AND} & \begin{tikzpicture}[scale=0.5, baseline] \node[and port] {}; \end{tikzpicture} & \begin{tabular}{cc|c} 0&0&0\\0&1&0\\1&0&0\\1&1&1 \end{tabular} & $Y = A\cdot B$ & બધા ઇનપુટ્સ 1 હોય ત્યારે જ આઉટપુટ 1 \\
\hline
\textbf{OR} & \begin{tikzpicture}[scale=0.5, baseline] \node[or port] {}; \end{tikzpicture} & \begin{tabular}{cc|c} 0&0&0\\0&1&1\\1&0&1\\1&1&1 \end{tabular} & $Y = A+B$ & કોઈપણ ઇનપુટ 1 હોય ત્યારે આઉટપુટ 1 \\
\hline
\textbf{NOT} & \begin{tikzpicture}[scale=0.5, baseline] \node[not port] {}; \end{tikzpicture} & \begin{tabular}{c|c} 0&1\\1&0 \end{tabular} & $Y = A'$ & ઇનપુટને ઇન્વર્ટ કરે છે \\
\hline
\textbf{NAND} & \begin{tikzpicture}[scale=0.5, baseline] \node[nand port] {}; \end{tikzpicture} & \begin{tabular}{cc|c} 0&0&1\\0&1&1\\1&0&1\\1&1&0 \end{tabular} & $Y = (A\cdot B)'$ & AND પછી NOT \\
\hline
\textbf{NOR} & \begin{tikzpicture}[scale=0.5, baseline] \node[nor port] {}; \end{tikzpicture} & \begin{tabular}{cc|c} 0&0&1\\0&1&0\\1&0&0\\1&1&0 \end{tabular} & $Y = (A+B)'$ & OR પછી NOT \\
\hline
\textbf{XOR} & \begin{tikzpicture}[scale=0.5, baseline] \node[xor port] {}; \end{tikzpicture} & \begin{tabular}{cc|c} 0&0&0\\0&1&1\\1&0&1\\1&1&0 \end{tabular} & $Y = A\oplus B$ & ઇનપુટ્સ અલગ હોય ત્યારે આઉટપુટ 1 \\
\hline
\textbf{XNOR} & \begin{tikzpicture}[scale=0.5, baseline] \node[xnor port] {}; \end{tikzpicture} & \begin{tabular}{cc|c} 0&0&1\\0&1&0\\1&0&0\\1&1&1 \end{tabular} & $Y = (A\oplus B)'$ & ઇનપુટ્સ સમાન હોય ત્યારે આઉટપુટ 1 \\
\hline
\end{tabulary}
\end{center}
\end{solutionbox}
\mnemonicbox{"All Operations Need Necessary eXecution" (દરેક ગેટનો પહેલો અક્ષર - AND, OR, NOT, NAND, NOR, XOR).}

\questionmarks{3(a)}{3}{સંક્ષિપ્તમાં 4:2 એન્કોડર સમજાવો.}

\begin{solutionbox}
\textbf{4-to-2 એન્કોડર ઓવરવ્યુ}:

\captionof{table}{4:2 એન્કોડર}
\begin{center}
\begin{tabulary}{\linewidth}{|L|L|L|}
\hline
ફંક્શન & વર્ણન & ટ્રુથ ટેબલ \\
\hline
\textbf{4:2 એન્કોડર} & 4 ઇનપુટ લાઇન્સને 2 આઉટપુટ લાઇન્સમાં કન્વર્ટ કરે છે & \begin{tabular}{cccc|cc} $I_0$&$I_1$&$I_2$&$I_3$&$Y_1$&$Y_0$\\ \hline 1&0&0&0&0&0\\0&1&0&0&0&1\\0&0&1&0&1&0\\0&0&0&1&1&1 \end{tabular} \\
& એક સમયે માત્ર એક જ ઇનપુટ એક્ટિવ & \\
& ઇનપુટ પોઝિશન બાઇનરીમાં એન્કોડેડ & \\
\hline
\end{tabulary}
\end{center}

\textbf{ડાયાગ્રામ: 4:2 એન્કોડર}:
\begin{center}
\begin{tikzpicture}
    \node [gtu block, minimum height=2cm] (enc) {4:2 એન્કોડર};
    \draw [gtu arrow] ([yshift=0.75cm]enc.west) -- (enc.west |- {0,0.75}) node[left] {$I_0$};
    \draw [gtu arrow] ([yshift=0.25cm]enc.west) -- (enc.west |- {0,0.25}) node[left] {$I_1$};
    \draw [gtu arrow] ([yshift=-0.25cm]enc.west) -- (enc.west |- {0,-0.25}) node[left] {$I_2$};
    \draw [gtu arrow] ([yshift=-0.75cm]enc.west) -- (enc.west |- {0,-0.75}) node[left] {$I_3$};
    \draw [gtu arrow] ([yshift=0.25cm]enc.east) -- (enc.east |- {0,0.25}) node[right] {$Y_1$};
    \draw [gtu arrow] ([yshift=-0.25cm]enc.east) -- (enc.east |- {0,-0.25}) node[right] {$Y_0$};
\end{tikzpicture}
\end{center}
\end{solutionbox}
\mnemonicbox{એન્કોડર ફંક્શન માટે "ઇનપુટ પોઝિશન ક્રિએટ્સ આઉટપુટ".}

\questionmarks{3(b)}{4}{ફુલ એડર બ્લોક્સનો ઉપયોગ કરીને 4-બિટ પેરેલલ એડરને સમજાવો.}

\begin{solutionbox}
\textbf{4-બિટ પેરેલલ એડર}:

\captionof{table}{પેરેલલ એડર ઘટકો}
\begin{center}
\begin{tabulary}{\linewidth}{|L|L|}
\hline
કોમ્પોનન્ટ & ફંક્શન \\
\hline
\textbf{ફુલ એડર} & 3 બિટ્સ (A, B, Carry-in) ને એડ કરે છે અને Sum અને Carry-out આપે છે \\
\hline
\textbf{પેરેલલ એડર} & 4 ફુલ એડર્સને કેરી પ્રોપેગેશન સાથે જોડે છે \\
\hline
\end{tabulary}
\end{center}

\textbf{ડાયાગ્રામ: 4-બિટ પેરેલલ એડર}:
\begin{center}
\begin{tikzpicture}[node distance=2cm, auto]
    \node [gtu block] (fa0) {$FA_0$};
    \node [gtu block, left of=fa0] (fa1) {$FA_1$};
    \node [gtu block, left of=fa1] (fa2) {$FA_2$};
    \node [gtu block, left of=fa2] (fa3) {$FA_3$};
    
    \foreach \i in {0,1,2,3} {
        \draw [gtu arrow] ([yshift=0.5cm]fa\i.north) -- (fa\i.north) node[midway, right] {$A_\i B_\i$};
        \draw [gtu arrow] (fa\i.south) -- ([yshift=-0.5cm]fa\i.south) node[midway, right] {$S_\i$};
    }
    \draw [gtu arrow] (fa0.east) -- ++(0.5,0) node[right] {$C_{0}=0$};
    \draw [gtu arrow] (fa0.west) -- (fa1.east) node[midway, above] {$C_1$};
    \draw [gtu arrow] (fa1.west) -- (fa2.east) node[midway, above] {$C_2$};
    \draw [gtu arrow] (fa2.west) -- (fa3.east) node[midway, above] {$C_3$};
    \draw [gtu arrow] (fa3.west) -- ++(-0.5,0) node[left] {$C_4$};
\end{tikzpicture}
\end{center}
\end{solutionbox}
\mnemonicbox{પેરેલલ એડરમાં કેરી પ્રોપેગેશન માટે "કેરી ઓલવેઝ પાસેસ રાઇટ".}

\questionmarks{3(c)}{7}{ટ્રુથ ટેબલ, સમીકરણ અને સર્કિટ ડાયાગ્રામ સાથે 8:1 મલ્ટિપ્લેક્સરનું વર્ણન કરો.}

\begin{solutionbox}
\textbf{8:1 મલ્ટિપ્લેક્સર}:

\captionof{table}{8:1 MUX}
\begin{center}
\begin{tabulary}{\linewidth}{|L|L|L|}
\hline
કોમ્પોનન્ટ & વર્ણન & ફંક્શન \\
\hline
\textbf{8:1 MUX} & 8 ઇનપુટ્સ, 3 સિલેક્ટ લાઇન્સ, 1 આઉટપુટ વાળો ડેટા સિલેક્ટર & સિલેક્ટ લાઇન્સના આધારે 8 ઇનપુટ્સમાંથી એક પસંદ કરે છે \\
\hline
\end{tabulary}
\end{center}

\textbf{ટ્રુથ ટેબલ}:
\begin{center}
\begin{tabular}{|ccc|c|}
\hline
$S_2$ & $S_1$ & $S_0$ & Y \\
\hline
0 & 0 & 0 & $D_0$ \\
0 & 0 & 1 & $D_1$ \\
0 & 1 & 0 & $D_2$ \\
0 & 1 & 1 & $D_3$ \\
1 & 0 & 0 & $D_4$ \\
1 & 0 & 1 & $D_5$ \\
1 & 1 & 0 & $D_6$ \\
1 & 1 & 1 & $D_7$ \\
\hline
\end{tabular}
\end{center}

\textbf{બુલિયન સમીકરણ}:
\[ Y = S_2'S_1'S_0'D_0 + S_2'S_1'S_0D_1 + S_2'S_1S_0'D_2 + S_2'S_1S_0D_3 + S_2S_1'S_0'D_4 + S_2S_1'S_0D_5 + S_2S_1S_0'D_6 + S_2S_1S_0D_7 \]

\textbf{ડાયાગ્રામ: 8:1 MUX}:
\begin{center}
\begin{tikzpicture}
    \node [gtu block, minimum height=4cm, minimum width=2cm] (mux) {8:1 MUX};
    \foreach \i in {0,...,7} {
        \draw [gtu arrow] ([yshift=1.75cm-\i*0.5cm]mux.west) -- (mux.west |- {0,1.75-\i*0.5}) node[left] {$D_\i$};
    }
    \draw [gtu arrow] (mux.south) -- ++(0,-1) node[below] {$S_2 S_1 S_0$};
    \draw [gtu arrow] (mux.east) -- ++(1,0) node[right] {Y};
\end{tikzpicture}
\end{center}
\end{solutionbox}
\mnemonicbox{મલ્ટિપ્લેક્સર ઓપરેશન માટે "સિલેક્ટ ડિસાઇડ્સ ડેટા આઉટપુટ".}

\questionmarks{3(a OR)}{3}{હાફ સબટ્રેક્ટરની લોજિક સર્કિટ દોરો અને તેનું કાર્ય સમજાવો.}

\begin{solutionbox}
\textbf{હાફ સબટ્રેક્ટર}:

\captionof{table}{હાફ સબટ્રેક્ટર}
\begin{center}
\begin{tabulary}{\linewidth}{|L|L|L|}
\hline
ફંક્શન & વર્ણન & ટ્રુથ ટેબલ \\
\hline
\textbf{હાફ સબટ્રેક્ટર} & બે બિટ્સને બાદ કરે છે અને ડિફરન્સ અને બોરો આપે છે & \begin{tabular}{cc|cc} A&B&D&Bout\\ \hline 0&0&0&0\\0&1&1&1\\1&0&1&0\\1&1&0&0 \end{tabular} \\
\hline
\end{tabulary}
\end{center}

\textbf{લોજિક સર્કિટ}:
\begin{center}
\begin{circuitikz}
    \draw (0,2) node[xor port] (xor) {};
    \draw (0,0) node[and port] (and) {};
    \node[not port, scale=0.5] (not) at (-1.5, 0.2) {}; 
    \draw (xor.in 1) -- ++(-2,0) node[left] (A) {A};
    \draw (xor.in 2) -- ++(-2,0) node[left] (B) {B};
    \draw (A) |- (not.in);
    \draw (not.out) |- (and.in 1);
    \draw (B) |- (and.in 2);
    \draw (xor.out) -- ++(0.5,0) node[right] {D = A$\oplus$B};
    \draw (and.out) -- ++(0.5,0) node[right] {Bout = A'$\cdot$B};
\end{circuitikz}
\end{center}

\textbf{સમીકરણો}:
\begin{itemize}
\item ડિફરન્સ (D) = $A \oplus B$
\item બોરો આઉટ (Bout) = $A' \cdot B$
\end{itemize}
\end{solutionbox}
\mnemonicbox{હાફ સબટ્રેક્ટર ઓપરેશન માટે "ડિફરન્ટ બિટ્સ બોરો".}

\questionmarks{3(b OR)}{4}{ટ્રુથ ટેબલ અને સર્કિટ ડાયાગ્રામ સાથે 3:8 ડીકોડર સમજાવો.}

\begin{solutionbox}
\textbf{3:8 ડીકોડર}:

\captionof{table}{3:8 ડીકોડર}
\begin{center}
\begin{tabulary}{\linewidth}{|L|L|L|}
\hline
ફંક્શન & વર્ણન & ટ્રુથ ટેબલ (આંશિક) \\
\hline
\textbf{3:8 ડીકોડર} & 3-બિટ બાઇનરી ઇનપુટને 8 આઉટપુટ લાઇન્સમાં કન્વર્ટ કરે છે & \begin{tabular}{ccc|c} $A_2 A_1 A_0$ & $Y_0 ... Y_7$ \\ \hline 0 0 0 & 1 0... \\ 0 0 1 & 0 1... \\ ... & ... \\ 1 1 1 & ...0 1 \end{tabular} \\
& એક સમયે માત્ર એક જ આઉટપુટ એક્ટિવ & \\
\hline
\end{tabulary}
\end{center}

\textbf{સર્કિટ ડાયાગ્રામ}:
\begin{center}
\begin{tikzpicture}
    \node [gtu block, minimum height=4cm, minimum width=2.5cm] (dec) {3:8 ડીકોડર};
    \draw [gtu arrow] ([yshift=0.5cm]dec.west) -- (dec.west |- {0,0.5}) node[left] {$A_0$};
    \draw [gtu arrow] (dec.west) -- (dec.west) node[left] {$A_1$};
    \draw [gtu arrow] ([yshift=-0.5cm]dec.west) -- (dec.west |- {0,-0.5}) node[left] {$A_2$};
    \foreach \i in {0,...,7} {
        \draw [gtu arrow] ([yshift=1.75cm-\i*0.5cm]dec.east) -- (dec.east |- {0,1.75-\i*0.5}) node[right] {$Y_\i$};
    }
\end{tikzpicture}
\end{center}

\textbf{સમીકરણો}:
\begin{itemize}
\item $Y_0 = A_2' \cdot A_1' \cdot A_0'$
\item ...
\item $Y_7 = A_2 \cdot A_1 \cdot A_0$
\end{itemize}
\end{solutionbox}
\mnemonicbox{ડીકોડર ઓપરેશન માટે "બાઇનરી ઇનપુટ એક્ટિવેટ્સ આઉટપુટ".}

\questionmarks{3(c OR)}{7}{ટ્રુથ ટેબલ, સમીકરણ અને સર્કિટ ડાયાગ્રામ સાથે ગ્રે થી બાઈનરી કોડ કન્વર્ટર સમજાવો.}

\begin{solutionbox}
\textbf{ગ્રે ટુ બાઇનરી કન્વર્ટર}:

\captionof{table}{ગ્રે ટુ બાઇનરી}
\begin{center}
\begin{tabulary}{\linewidth}{|L|L|L|}
\hline
ફંક્શન & વર્ણન & ટેબલ \\
\hline
\textbf{ગ્રે ટુ બાઇનરી} & ગ્રે કોડને બાઇનરી કોડમાં કન્વર્ટ કરે છે & \begin{tabular}{c|c} ગ્રે & બાઇનરી \\ \hline 0000 & 0000 \\ 0001 & 0001 \\ 0011 & 0010 \\ 0010 & 0011 \\ ... & ... \end{tabular} \\
& બાઇનરીનો MSB ગ્રેના MSBને સમાન & \\
& દરેક બાઇનરી બિટ, હાલના ગ્રે બિટ અને અગાઉના બાઇનરી બિટનો XOR છે & \\
\hline
\end{tabulary}
\end{center}

\textbf{સર્કિટ ડાયાગ્રામ}:
\begin{center}
\begin{circuitikz}
    \draw (0, 3) node (g3) {$G_3$};
    \draw (0, 2) node (g2) {$G_2$};
    \draw (0, 1) node (g1) {$G_1$};
    \draw (0, 0) node (g0) {$G_0$};
    \draw (g3) -- ++(4,0) node[right] {$B_3$};
    \draw (2, 2) node[xor port, scale=0.8] (xor1) {};
    \draw (2, 1) node[xor port, scale=0.8] (xor2) {};
    \draw (2, 0) node[xor port, scale=0.8] (xor3) {};
    \draw (g3) -| (xor1.in 1);
    \draw (g2) -- (xor1.in 2);
    \draw (xor1.out) -- ++(2,0) node[right] {$B_2$};
    \draw (xor1.out) -- ++(0.5,0) |- (xor2.in 1);
    \draw (g1) -- (xor2.in 2);
    \draw (xor2.out) -- ++(1.5,0) node[right] {$B_1$};
    \draw (xor2.out) -- ++(0.5,0) |- (xor3.in 1);
    \draw (g0) -- (xor3.in 2);
    \draw (xor3.out) -- ++(1.5,0) node[right] {$B_0$};
\end{circuitikz}
\end{center}

\textbf{સમીકરણો}:
\begin{itemize}
\item $B_3 = G_3$
\item $B_2 = G_3 \oplus G_2$
\item $B_1 = B_2 \oplus G_1$
\item $B_0 = B_1 \oplus G_0$
\end{itemize}
\end{solutionbox}
\mnemonicbox{ગ્રે ટુ બાઇનરી કન્વર્ઝન માટે "MSB સ્ટેઝ, રેસ્ટ XOR".}

\questionmarks{4(a)}{3}{ટ્રુથ ટેબલ અને સર્કિટ ડાયાગ્રામ સાથે D ફ્લિપ ફ્લોપ સમજાવો.}

\begin{solutionbox}
\textbf{D ફ્લિપ-ફ્લોપ}:

\captionof{table}{D ફ્લિપ-ફ્લોપ}
\begin{center}
\begin{tabulary}{\linewidth}{|L|L|C|C|C|C|}
\hline
ફંક્શન & વર્ણન & CLK & D & Q & Q' \\
\hline
\textbf{D ફ્લિપ-ફ્લોપ} & ડેટા/ડિલે ફ્લિપ-ફ્લોપ & $\uparrow$ & 0 & 0 & 1 \\
\hline
& ક્લોક એજ પર Q, D ને ફોલો કરે છે & $\uparrow$ & 1 & 1 & 0 \\
\hline
\end{tabulary}
\end{center}

\textbf{સર્કિટ ડાયાગ્રામ}:
\begin{center}
\begin{circuitikz}
    \node [flipflop D, dot on notQ] (DFF) {D ફ્લિપ-ફ્લોપ};
    \draw (DFF.pin 1) -- ++(-1,0) node[left] {D};
    \draw (DFF.pin 3) -- ++(-1,0) node[left] {ક્લોક};
    \draw (DFF.pin 6) -- ++(1,0) node[right] {Q};
    \draw (DFF.pin 4) -- ++(1,0) node[right] {Q'};
\end{circuitikz}
\end{center}

\textbf{કેરેક્ટરિસ્ટિક સમીકરણ}: $Q(next) = D$
\end{solutionbox}
\mnemonicbox{D ફ્લિપ-ફ્લોપ ઓપરેશન માટે "ડેટા ડિલેઝ વન ક્લોક".}

\questionmarks{4(b)}{4}{માસ્ટર સ્લેવ JK ફ્લિપ ફ્લોપનું કાર્ય સમજાવો.}

\begin{solutionbox}
\textbf{માસ્ટર-સ્લેવ JK ફ્લિપ-ફ્લોપ}:

\captionof{table}{માસ્ટર-સ્લેવ ઓપરેશન}
\begin{center}
\begin{tabulary}{\linewidth}{|L|L|C|C|L|}
\hline
કોમ્પોનન્ટ & ઓપરેશન & J & K & Q(next) \\
\hline
\textbf{માસ્ટર} & CLK = 1 હોય ત્યારે ઇનપુટ્સને સેમ્પલ કરે છે & 0 & 0 & કોઈ ફેરફાર નહીં \\
\hline
\textbf{સ્લેવ} & CLK = 0 હોય ત્યારે માસ્ટર આઉટપુટને ટ્રાન્સફર કરે છે & 0 & 1 & 0 \\
\hline
& & 1 & 0 & 1 \\
\hline
& & 1 & 1 & ટોગલ \\
\hline
\end{tabulary}
\end{center}

\textbf{ડાયાગ્રામ: માસ્ટર-સ્લેવ JK}:
\begin{center}
\begin{tikzpicture}
    \node [gtu block] (master) {માસ્ટર JK};
    \node [gtu block, right of=master, xshift=2cm] (slave) {સ્લેવ JK};
    \node [not port, scale=0.5] (not) at (2,-1.5) {};
    \draw [gtu arrow] ([yshift=0.5cm]master.west) -- (master.west |- {0,0.5}) node[left] {J};
    \draw [gtu arrow] ([yshift=-0.5cm]master.west) -- (master.west |- {0,-0.5}) node[left] {K};
    \draw (master.south) -- ++(0,-0.5) node (clk) {};
    \draw (clk) -- ++(-1,0) node[left] {ક્લોક};
    \draw (clk) -| (not.in);
    \draw (not.out) |- (slave.south);
    \draw [gtu arrow] (master.east) -- (slave.west);
    \draw [gtu arrow] ([yshift=0.5cm]slave.east) -- (slave.east |- {0,0.5}) node[right] {Q};
    \draw [gtu arrow] ([yshift=-0.5cm]slave.east) -- (slave.east |- {0,-0.5}) node[right] {Q'};
\end{tikzpicture}
\end{center}

\textbf{કાર્યપદ્ધતિ}:
\begin{itemize}
\item \textbf{માસ્ટર સ્ટેજ}: ક્લોક હાઇ હોય ત્યારે ઇનપુટ કેપ્ચર કરે છે
\item \textbf{સ્લેવ સ્ટેજ}: ક્લોક લો હોય ત્યારે આઉટપુટ અપડેટ કરે છે
\item \textbf{રેસ કન્ડિશન અટકાવે છે}: ઇનપુટ કેપ્ચર અને આઉટપુટ અપડેટને અલગ કરીને
\end{itemize}
\end{solutionbox}
\mnemonicbox{માસ્ટર-સ્લેવ ઓપરેશન માટે "માસ્ટર સેમ્પલ્સ, સ્લેવ ટ્રાન્સફર્સ".}

\questionmarks{4(c)}{7}{બ્લોક ડાયાગ્રામની મદદથી શિફ્ટ રજિસ્ટર્સનું વર્ગીકરણ કરો અને તેમાંના કોઈપણ એકને વિગતવાર સમજાવો.}

\begin{solutionbox}
\textbf{શિફ્ટ રજિસ્ટર વર્ગીકરણ}:

\captionof{table}{શિફ્ટ રજિસ્ટર પ્રકારો}
\begin{center}
\begin{tabulary}{\linewidth}{|L|L|L|}
\hline
પ્રકાર & વર્ણન & ફંક્શન \\
\hline
\textbf{SISO} & સિરિયલ ઇન સિરિયલ આઉટ & ડેટા સિરિયલી, બિટ દર બિટ, એન્ટર થાય છે અને એક્ઝિટ થાય છે \\
\hline
\textbf{SIPO} & સિરિયલ ઇન પેરેલલ આઉટ & ડેટા સિરિયલી એન્ટર થાય છે, પેરેલલમાં એક્ઝિટ થાય છે \\
\hline
\textbf{PISO} & પેરેલલ ઇન સિરિયલ આઉટ & ડેટા પેરેલલમાં એન્ટર થાય છે, સિરિયલી એક્ઝિટ થાય છે \\
\hline
\textbf{PIPO} & પેરેલલ ઇન પેરેલલ આઉટ & ડેટા પેરેલલમાં એન્ટર થાય છે અને પેરેલલમાં એક્ઝિટ થાય છે \\
\hline
\end{tabulary}
\end{center}

\textbf{SIPO શિફ્ટ રજિસ્ટર વિગતવાર}:

\begin{center}
\begin{circuitikz}
    \foreach \i in {0,1,2,3} {
        \node [flipflop D, dot on notQ] (FF\i) at (\i*2.8, 0) {$FF_\i$};
        \draw (FF\i.pin 6) -- ++(0.5,0) node[right] {$Q_\i$};
    }
    \draw (FF0.pin 1) -- ++(-1,0) node[left] {ડેટા ઇન};
    \foreach \i [evaluate=\i as \j using int(\i+1)] in {0,1,2} {
        \draw (FF\i.pin 6) -- ++(0.5,0) -- ++(0,1.2) -- ++(1,0) |- (FF\j.pin 1);
    }
    \draw (FF0.pin 3) -- ++(0,-1) node (clk0) {};
    \draw (FF3.pin 3) -- ++(0,-1) node (clk3) {};
    \draw (clk0) -- (clk3);
    \draw (clk0) -- ++(-1,0) node[left] {ક્લોક};
\end{circuitikz}
\end{center}

\textbf{SIPO શિફ્ટ રજિસ્ટરનું કાર્ય}:
\begin{itemize}
\item \textbf{સિરિયલ ડેટા} ડેટા-ઇન પિન પર, પ્રતિ ક્લોક સાયકલ એક બિટ, પ્રવેશે છે
\item \textbf{દરેક ફ્લિપ-ફ્લોપ} ક્લોક પલ્સ પર તેની સામગ્રીને આગળના ફ્લિપ-ફ્લોપમાં પાસ કરે છે
\item \textbf{4 ક્લોક સાયકલ્સ પછી}, 4-બિટ ડેટા બધા ફ્લિપ-ફ્લોપ્સમાં સ્ટોર થાય છે
\item \textbf{પેરેલલ આઉટપુટ} Q0-Q3 પરથી એક સાથે ઉપલબ્ધ થાય છે
\end{itemize}

\textbf{SIPO માટે ટાઇમિંગ ડાયાગ્રામ}:
\begin{center}
\begin{tikzpicture}
    \foreach \bits [count=\i] in {1,0,0,0} {
        \node at (\i, 4) {Clk \i};
        \node at (\i, 3) {\bits};
    }
    \node at (0,3) {Din:};
    \node at (0,2) {Q0:}; \foreach \v in {1,0,0,0} \node at (1,2) {\v};
    \node at (0,1.5) {Q1:}; \foreach \v in {0,1,0,0} \node at (2,1.5) {\v};
    \node at (0,1) {Q2:}; \foreach \v in {0,0,1,0} \node at (3,1) {\v};
    \node at (0,0.5) {Q3:}; \foreach \v in {0,0,0,1} \node at (4,0.5) {\v};
\end{tikzpicture}
\end{center}
\end{solutionbox}
\mnemonicbox{SIPO ઓપરેશન માટે "સિરિયલ ઇનપુટ્સ પેરેલલ આઉટપુટ્સ".}

\questionmarks{4(a OR)}{3}{ટ્રુથ ટેબલ અને સર્કિટ ડાયાગ્રામ સાથે SR ફ્લિપ ફ્લોપ સમજાવો.}

\begin{solutionbox}
\textbf{SR ફ્લિપ-ફ્લોપ}:

\captionof{table}{SR ફ્લિપ-ફ્લોપ (સેટ-રિસેટ)}
\begin{center}
\begin{tabulary}{\linewidth}{|C|C|C|C|}
\hline
S & R & Q & Q' \\
\hline
0 & 0 & કોઈ ફેરફાર નહીં & કોઈ ફેરફાર નહીં \\
\hline
0 & 1 & 0 & 1 \\
\hline
1 & 0 & 1 & 0 \\
\hline
1 & 1 & અમાન્ય & અમાન્ય \\
\hline
\end{tabulary}
\end{center}

\textbf{સર્કિટ ડાયાગ્રામ}:
\begin{center}
\begin{circuitikz}
    \node[nor port] (nor1) at (0,2) {};
    \node[nor port] (nor2) at (0,0) {};
    \draw (nor1.in 1) -- ++(-1,0) node[left] {S};
    \draw (nor2.in 2) -- ++(-1,0) node[left] {R};
    \draw (nor1.out) -- ++(1,0) node[right] {Q};
    \draw (nor2.out) -- ++(1,0) node[right] {Q'};
    \draw (nor1.in 2) -- ++(-0.2,0) -- ++(0,-0.5) -- ++(1.5,0) |- (nor2.out);
    \draw (nor2.in 1) -- ++(-0.2,0) -- ++(0,0.5) -- ++(1.5,0) |- (nor1.out);
\end{circuitikz}
\end{center}
\end{solutionbox}
\mnemonicbox{SR ફ્લિપ-ફ્લોપ ઓપરેશન માટે "સેટ ટુ 1, રિસેટ ટુ 0".}

\questionmarks{4(b OR)}{4}{ટ્રુથ ટેબલ અને સર્કિટ ડાયાગ્રામ સાથે JK ફ્લિપ ફ્લોપ સમજાવો.}

\begin{solutionbox}
\textbf{JK ફ્લિપ-ફ્લોપ}:

\captionof{table}{JK ફ્લિપ-ફ્લોપ}
\begin{center}
\begin{tabulary}{\linewidth}{|L|C|C|L|}
\hline
વર્ણન & J & K & Q(next) \\
\hline
અમાન્ય કન્ડિશન હલ કરે છે & 0 & 0 & કોઈ ફેરફાર નહીં \\
\hline
& 0 & 1 & 0 \\
\hline
& 1 & 0 & 1 \\
\hline
& 1 & 1 & ટોગલ (Q') \\
\hline
\end{tabulary}
\end{center}

\textbf{સર્કિટ ડાયાગ્રામ}:
\begin{center}
\begin{circuitikz}
    \node[flipflop JK, dot on notQ] (jk) {JK ફ્લિપ-ફ્લોપ};
    \draw (jk.pin 1) -- ++(-0.5,0) node[left] {J};
    \draw (jk.pin 3) -- ++(-0.5,0) node[left] {CLK};
    \draw (jk.pin 2) -- ++(-0.5,0) node[left] {K};
    \draw (jk.pin 6) -- ++(0.5,0) node[right] {Q};
    \draw (jk.pin 4) -- ++(0.5,0) node[right] {Q'};
\end{circuitikz}
\end{center}

\textbf{કેરેક્ટરિસ્ટિક સમીકરણ}: $Q(next) = J\cdot Q' + K'\cdot Q$
\end{solutionbox}
\mnemonicbox{JK ફ્લિપ-ફ્લોપ સ્ટેટ્સ માટે "જમ્પ-કીપ-ટોગલ".}

\questionmarks{4(c OR)}{7}{ટ્રુથ ટેબલ અને સર્કિટ ડાયાગ્રામ સાથે 4-બિટ અસિંક્રોનસ અપ કાઉન્ટરનું વર્ણન કરો.}

\begin{solutionbox}
\textbf{4-બિટ અસિંક્રોનસ અપ કાઉન્ટર}:

\captionof{table}{અસિંક્રોનસ કાઉન્ટર}
\begin{center}
\begin{tabulary}{\linewidth}{|L|L|}
\hline
વર્ણન & કાઉન્ટ સિક્વન્સ \\
\hline
રિપલ કાઉન્ટર પણ કહેવાય છે & 0000 $\to$ 0001 $\to$ 0010 $\to$ 0011 \\
ક્લોક માત્ર પહેલા FF ને ડ્રાઇવ કરે છે & 0100 $\to$ 0101 $\to$ 0110 $\to$ 0111 \\
દરેક FF અગાઉના FF આઉટપુટ દ્વારા ટ્રિગર થાય છે & 1000 $\to$ 1001 $\to$ 1010 $\to$ 1011 \\
& 1100 $\to$ 1101 $\to$ 1110 $\to$ 1111 \\
\hline
\end{tabulary}
\end{center}

\textbf{સર્કિટ ડાયાગ્રામ}:
\begin{center}
\begin{circuitikz}
    \foreach \i in {0,1,2,3} {
        \node [flipflop JK, dot on notQ] (FF\i) at (\i*3.5, 0) {$FF_\i$};
        \draw (FF\i.pin 1) -- ++(-0.2,0) node[left, scale=0.6] {1}; % J=1
        \draw (FF\i.pin 2) -- ++(-0.2,0) node[left, scale=0.6] {1}; % K=1
        \draw (FF\i.pin 6) -- ++(0.5,0) node[right] {$Q_\i$};
    }
    \draw (FF0.pin 3) -- ++(-0.5,0) node[left] {ક્લોક};
    \draw (FF0.pin 6) -- ++(0.5,0) -- ++(0,1) -- ++(2.5,0) |- (FF1.pin 3);
    \draw (FF1.pin 6) -- ++(0.5,0) -- ++(0,1) -- ++(2.5,0) |- (FF2.pin 3);
    \draw (FF2.pin 6) -- ++(0.5,0) -- ++(0,1) -- ++(2.5,0) |- (FF3.pin 3);
\end{circuitikz}
\end{center}

\textbf{કાર્યપદ્ધતિ}:
\begin{itemize}
\item \textbf{પહેલો FF} દરેક ક્લોક પલ્સ પર ટોગલ થાય છે
\item \textbf{બીજો FF} જ્યારે પહેલો FF 1 થી 0 પર જાય છે ત્યારે ટોગલ થાય છે
\item \textbf{ત્રીજો FF} જ્યારે બીજો FF 1 થી 0 પર જાય છે ત્યારે ટોગલ થાય છે
\item \textbf{ચોથો FF} જ્યારે ત્રીજો FF 1 થી 0 પર જાય છે ત્યારે ટોગલ થાય છે
\end{itemize}
\end{solutionbox}
\mnemonicbox{અસિંક્રોનસ કાઉન્ટર ઓપરેશન માટે "રિપલ કેરીઝ પ્રોપેગેશન ડિલે".}

\questionmarks{5(a)}{3}{નીચેની લોજીક ફેમિલીઝની તુલના કરો: TTL, CMOS, ECL}

\begin{solutionbox}
\textbf{લોજિક ફેમિલીઝ કમ્પેરિઝન}:

\captionof{table}{લોજિક ફેમિલીઝ}
\begin{center}
\begin{tabulary}{\linewidth}{|L|L|L|L|}
\hline
પેરામીટર & TTL & CMOS & ECL \\
\hline
\textbf{ટેક્નોલોજી} & બાયપોલર ટ્રાન્ઝિસ્ટર્સ & MOSFETs & બાયપોલર ટ્રાન્ઝિસ્ટર્સ \\
\hline
\textbf{પાવર કન્ઝમ્પશન} & મધ્યમ & ખૂબ ઓછો & ઉચ્ચ \\
\hline
\textbf{સ્પીડ} & મધ્યમ & નીચી-મધ્યમ & ખૂબ ઉચ્ચ \\
\hline
\textbf{નોઇઝ ઇમ્યુનિટી} & મધ્યમ & ઉચ્ચ & નીચી \\
\hline
\textbf{ફેન-આઉટ} & 10 & 50+ & 25 \\
\hline
\textbf{સપ્લાય વોલ્ટેજ} & 5V & 3-15V & -5.2V \\
\hline
\end{tabulary}
\end{center}
\end{solutionbox}
\mnemonicbox{લોજિક ફેમિલીઝની તુલના માટે "ટેક્નોલોજી કન્ટ્રોલ્સ મેની ઇલેક્ટ્રિકલ કેરેક્ટરિસ્ટિક્સ".}

\questionmarks{5(b)}{4}{કોમ્બિનેશનલ અને સિક્વેન્શિયલ લોજિક સર્કિટ્સની સરખામણી કરો.}

\begin{solutionbox}
\textbf{કોમ્બિનેશનલ vs સિક્વેન્શિયલ સર્કિટ્સ}:

\captionof{table}{સરખામણી}
\begin{center}
\begin{tabulary}{\linewidth}{|L|L|L|}
\hline
પેરામીટર & કોમ્બિનેશનલ સર્કિટ્સ & સિક્વેન્શિયલ સર્કિટ્સ \\
\hline
\textbf{આઉટપુટ આધારિત છે} & માત્ર વર્તમાન ઇનપુટ્સ પર & વર્તમાન ઇનપુટ્સ અને અગાઉની સ્ટેટ પર \\
\hline
\textbf{મેમોરી} & કોઈ મેમોરી નથી & મેમોરી એલિમેન્ટ્સ ધરાવે છે \\
\hline
\textbf{ફીડબેક} & કોઈ ફીડબેક પાથ નથી & ફીડબેક પાથ્સ ધરાવે છે \\
\hline
\textbf{ઉદાહરણો} & એડર્સ, MUX, ડિકોડર્સ & ફ્લિપ-ફ્લોપ્સ, કાઉન્ટર્સ, રજિસ્ટર્સ \\
\hline
\textbf{ક્લોક} & ક્લોકની જરૂર નથી & ઘણી વાર ક્લોકની જરૂર પડે છે \\
\hline
\textbf{ડિઝાઇન એપ્રોચ} & ટ્રુથ ટેબલ્સ, K-મેપ્સ & સ્ટેટ ડાયાગ્રામ્સ, ટેબલ્સ \\
\hline
\end{tabulary}
\end{center}

\textbf{ડાયાગ્રામ: કમ્પેરિઝન}:
\begin{center}
\begin{tikzpicture}[node distance=2cm, auto]
    \node [gtu block] (comb) {કોમ્બિનેશનલ લોજિક};
    \draw [gtu arrow] (comb.west) -- ++(-1,0) node[left] {ઇનપુટ્સ};
    \draw [gtu arrow] (comb.east) -- ++(1,0) node[right] {આઉટપુટ્સ};
    \node [gtu block, right of=comb, xshift=5cm] (seq) {સિક્વેન્શિયલ લોજિક};
    \draw [gtu arrow] (seq.west) -- ++(-1,0) node[left] {ઇનપુટ્સ};
    \draw [gtu arrow] (seq.east) -- ++(1,0) node[right] {આઉટપુટ્સ};
    \node [gtu block, below of=seq] (mem) {મેમરી};
    \draw [gtu arrow] (seq.south) -- (mem.north);
    \draw [gtu arrow] (mem.east) -| ($(seq.east)+(0.5,0)$) -- (seq.east);
\end{tikzpicture}
\end{center}
\end{solutionbox}
\mnemonicbox{કોમ્બિનેશનલ અને સિક્વેન્શિયલ સર્કિટ્સ વચ્ચે તફાવત કરવા માટે "કરંટ ઓન્લી vs મેમોરી સ્ટેટ્સ".}

\questionmarks{5(c)}{7}{વ્યાખ્યાયિત કરો: ફેન ઇન, ફેન આઉટ, નોઇઝ માર્જિન, પ્રોપેગેશન ડિલે, પાવર ડિસીપેશન, ફિગર ઓફ મેરિટ, રેમ}

\begin{solutionbox}
\textbf{ડિજિટલ ઇલેક્ટ્રોનિક્સ કી ડેફિનિશન્સ}:

\captionof{table}{વ્યાખ્યાઓ}
\begin{center}
\begin{tabulary}{\linewidth}{|L|L|L|}
\hline
ટર્મ & વ્યાખ્યા & ટિપિકલ વેલ્યુઝ \\
\hline
\textbf{ફેન-ઇન} & લોજિક ગેટ જેટલા ઇનપુટ્સ હેન્ડલ કરી શકે તેની મહત્તમ સંખ્યા & TTL: 2-8, CMOS: 100+ \\
\hline
\textbf{ફેન-આઉટ} & સિંગલ આઉટપુટ દ્વારા જેટલા ગેટ ઇનપુટ્સ ડ્રાઇવ કરી શકાય તેની મહત્તમ સંખ્યા & TTL: 10, CMOS: 50 \\
\hline
\textbf{નોઇઝ માર્જિન} & એરર થાય તે પહેલાં ઉમેરી શકાય તેવો મહત્તમ નોઇઝ વોલ્ટેજ & TTL: 0.4V, CMOS: 1.5V \\
\hline
\textbf{પ્રોપેગેશન ડિલે} & ઇનપુટમાં બદલાવથી આઉટપુટમાં બદલાવ થવામાં લાગતો સમય & TTL: 10ns, CMOS: 20ns \\
\hline
\textbf{પાવર ડિસીપેશન} & ઓપરેશન દરમિયાન ગેટ દ્વારા વપરાતી શક્તિ & TTL: 10mW, CMOS: 0.1mW \\
\hline
\textbf{ફિગર ઓફ મેરિટ} & સ્પીડ અને પાવરનો ગુણાકાર (ઓછો વધુ સારો) & TTL: 100pJ, CMOS: 2pJ \\
\hline
\textbf{RAM} & રેન્ડમ એક્સેસ મેમોરી - ટેમ્પરરી સ્ટોરેજ ડિવાઇસ & પ્રકાર: SRAM, DRAM \\
\hline
\end{tabulary}
\end{center}

\textbf{ડાયાગ્રામ: ડિજિટલ પેરામીટર રિલેશનશિપ્સ}:
\begin{center}
\begin{tikzpicture}
    \node [gtu block] (delay) {નીચી પ્રોપેગેશન ડિલે};
    \node [gtu block, right of=delay, xshift=3cm] (speed) {સ્પીડ};
    \draw [gtu arrow] (delay) -- (speed) node[midway, above] {વધારે};
    
    \node [gtu block, below of=delay] (power) {નીચી પાવર ડિસીપેશન};
    \node [gtu block, below of=speed] (eff) {એફિશિયન્સી};
    \draw [gtu arrow] (power) -- (eff) node[midway, above] {વધારે};
    
    \node [gtu block, right of=speed, yshift=-1cm] (fom) {ફિગર ઓફ મેરિટ};
    \draw [gtu arrow] (speed) -- (fom);
    \draw [gtu arrow] (eff) -- (fom);
\end{tikzpicture}
\end{center}
\end{solutionbox}
\mnemonicbox{પેરામીટર ટર્મ્સ યાદ રાખવા માટે "ફાસ્ટ પાવર નીડ્સ પ્રોપર ફિગર રેટિંગ્સ".}

\questionmarks{5(a OR)}{3}{ડિજિટલ ICના ઇ-વેસ્ટ મેનેજમેન્ટના પગલાં અને જરૂરિયાતનું વર્ણન કરો.}

\begin{solutionbox}
\textbf{ડિજિટલ ICs માટે ઇ-વેસ્ટ મેનેજમેન્ટ}:

\captionof{table}{ઇ-વેસ્ટ મેનેજમેન્ટ}
\begin{center}
\begin{tabulary}{\linewidth}{|L|L|L|}
\hline
સ્ટેપ & વર્ણન & મહત્વ \\
\hline
\textbf{કલેક્શન} & ઇલેક્ટ્રોનિક વેસ્ટનું અલગ કલેક્શન & અયોગ્ય ડિસ્પોઝલને રોકે છે \\
\hline
\textbf{સેગ્રેગેશન} & ICsને અન્ય કોમ્પોનન્ટ્સથી અલગ કરવું & ટાર્ગેટેડ રિસાયક્લિંગ શક્ય બનાવે છે \\
\hline
\textbf{ડિસમેન્ટલિંગ} & હાનિકારક ભાગોને દૂર કરવા & પર્યાવરણીય નુકસાન ઘટાડે છે \\
\hline
\textbf{રિકવરી} & મૂલ્યવાન મટીરિયલ્સ (ગોલ્ડ, સિલિકોન) એક્સટ્રેક્ટ કરવા & સંસાધનો બચાવે છે \\
\hline
\textbf{સેફ ડિસ્પોઝલ} & નોન-રિસાયક્લેબલ પાર્ટ્સનો યોગ્ય નિકાલ & પ્રદૂષણ અટકાવે છે \\
\hline
\end{tabulary}
\end{center}

\textbf{ઇ-વેસ્ટ મેનેજમેન્ટની જરૂરિયાત}:
\begin{itemize}
\item \textbf{હાનિકારક મટીરિયલ્સ}: ICs લેડ, મર્ક્યુરી, કેડમિયમ ધરાવે છે
\item \textbf{રિસોર્સ કન્ઝર્વેશન}: કિંમતી ધાતુઓ અને દુર્લભ સામગ્રી પુનઃપ્રાપ્ત કરે છે
\item \textbf{પર્યાવરણ સંરક્ષણ}: જમીન અને પાણીના પ્રદૂષણને રોકે છે
\item \textbf{હેલ્થ સેફ્ટી}: ઝેરી પદાર્થોના સંપર્કને ઘટાડે છે
\end{itemize}
\end{solutionbox}
\mnemonicbox{ઇ-વેસ્ટ મેનેજમેન્ટ સ્ટેપ્સ માટે "કલેક્શન સ્ટાર્ટ્સ ડિસમેન્ટલિંગ રિકવરી સેફ્લી".}

\questionmarks{5(b OR)}{4}{સર્કિટ ડાયાગ્રામ સાથે રીંગ કાઉન્ટરનું કામ સમજાવો.}

\begin{solutionbox}
\textbf{રીંગ કાઉન્ટર}:

\captionof{table}{રીંગ કાઉન્ટર}
\begin{center}
\begin{tabulary}{\linewidth}{|L|L|}
\hline
વર્ણન & કાઉન્ટ સિક્વન્સ \\
\hline
સિંગલ 1 સાથે સર્ક્યુલર શિફ્ટ રજિસ્ટર & 1000 $\to$ 0100 $\to$ 0010 $\to$ 0001 $\to$ 1000 \\
કોઈપણ સમયે માત્ર એક જ ફ્લિપ-ફ્લોપ સેટ થયેલ હોય છે & \\
N સ્ટેટ્સ માટે N ફ્લિપ-ફ્લોપ્સ & \\
\hline
\end{tabulary}
\end{center}

\textbf{સર્કિટ ડાયાગ્રામ}:
\begin{center}
\begin{circuitikz}
    \foreach \i in {1,2,3,4} {
        \node [flipflop D, dot on notQ] (FF\i) at (\i*2.8, 0) {$FF_\i$};
        \draw (FF\i.pin 6) -- ++(0.5,0) node[right] {$Q_\i$};
    }
    \draw (FF4.pin 6) -- ++(0.5,0) -- ++(0,1.2) -- ++(-11,0) |- (FF1.pin 1);
    \draw (FF1.pin 6) -- (FF2.pin 1);
    \draw (FF2.pin 6) -- (FF3.pin 1);
    \draw (FF3.pin 6) -- (FF4.pin 1);
    \draw (FF1.pin 3) -- ++(0,-1) node (clk1) {};
    \draw (FF4.pin 3) -- ++(0,-1) node (clk4) {};
    \draw (clk1) -- (clk4);
    \draw (clk1) -- ++(-1,0) node[left] {ક્લોક};
    \draw (FF1.pin 4) -- ++(0,-0.5) node[below] {PR} node[circ] {}; % Preset approximation
\end{circuitikz}
\end{center}

\textbf{કાર્યપદ્ધતિ}:
\begin{itemize}
\item \textbf{ઇનિશિયલાઇઝેશન}: પહેલા FF ને 1 પર સેટ કરવામાં આવે છે, બાકીના 0 પર
\item \textbf{ઓપરેશન}: સિંગલ 1 બધા ફ્લિપ-ફ્લોપ્સમાં ફરે છે
\item \textbf{એપ્લિકેશન્સ}: સિક્વેન્સર્સ, કન્ટ્રોલર્સ, ટાઇમિંગ સર્કિટ્સ
\end{itemize}
\end{solutionbox}
\mnemonicbox{રીંગ કાઉન્ટર ઓપરેશન માટે "વન બિટ રોટેટ્સ ઓન્લી".}

\questionmarks{5(c OR)}{7}{વર્ગીકૃત કરો: (i) મેમોરીઝ (ii) વિવિધ લોજીક ફેમિલીઝ}

\begin{solutionbox}
\textbf{(i) મેમોરી વર્ગીકરણ}:

\captionof{table}{મેમોરી પ્રકારો}
\begin{center}
\begin{tabulary}{\linewidth}{|L|L|L|}
\hline
પ્રકાર & સબટાઇપ્સ & લક્ષણો \\
\hline
\textbf{RAM} & \textbf{SRAM} & સ્ટેટિક RAM, ફાસ્ટ, મોંઘી, ફ્લિપ-ફ્લોપ્સનો ઉપયોગ કરે છે, રિફ્રેશની જરૂર નથી \\
\hline
& \textbf{DRAM} & ડાયનેમિક RAM, સ્લોઅર, સસ્તી, કેપેસિટર્સનો ઉપયોગ કરે છે, પીરિયોડિક રિફ્રેશની જરૂર પડે છે \\
\hline
\textbf{ROM} & \textbf{PROM} & પ્રોગ્રામેબલ ROM, વન-ટાઇમ પ્રોગ્રામેબલ \\
\hline
& \textbf{EPROM} & ઇરેઝેબલ PROM, UV લાઇટ દ્વારા ઇરેઝેબલ, મલ્ટિપલ રીપ્રોગ્રામિંગ \\
\hline
& \textbf{EEPROM} & ઇલેક્ટ્રિકલી ઇરેઝેબલ PROM, ઇલેક્ટ્રિકલ ઇરેઝર, બાઇટ-લેવલ ઇરેઝર \\
\hline
& \textbf{ફ્લેશ} & EEPROM વેરિએન્ટ, બ્લોક-લેવલ ઇરેઝર, નોન-વોલેટાઇલ \\
\hline
\end{tabulary}
\end{center}

\textbf{(ii) લોજિક ફેમિલીઝ વર્ગીકરણ}:

\captionof{table}{લોજિક ફેમિલીઝ}
\begin{center}
\begin{tabulary}{\linewidth}{|L|L|L|}
\hline
ટેક્નોલોજી & ફેમિલીઝ & લક્ષણો \\
\hline
\textbf{બાયપોલર} & \textbf{TTL} & ટ્રાન્ઝિસ્ટર-ટ્રાન્ઝિસ્ટર લોજિક, મધ્યમ સ્પીડ, 5V ઓપરેશન \\
\hline
& \textbf{ECL} & એમિટર-કપલ્ડ લોજિક, ખૂબ હાઈ સ્પીડ, હાઈ પાવર કન્ઝમ્પશન \\
\hline
& \textbf{I$^2$L} & ઇન્ટિગ્રેટેડ ઇન્જેક્શન લોજિક, હાઈ ડેન્સિટી \\
\hline
\textbf{MOS} & \textbf{NMOS} & N-ચેનલ MOSFET, સિમ્પલર ફેબ્રિકેશન \\
\hline
& \textbf{PMOS} & P-ચેનલ MOSFET, લોઅર પરફોર્મન્સ \\
\hline
& \textbf{CMOS} & કોમ્પ્લિમેન્ટરી MOS, લો પાવર કન્ઝમ્પશન, હાઈ નોઇઝ ઇમ્યુનિટી \\
\hline
\textbf{હાઇબ્રિડ} & \textbf{BiCMOS} & બાયપોલર અને CMOSને કોમ્બાઇન કરે છે, લો પાવર સાથે હાઈ સ્પીડ \\
\hline
\end{tabulary}
\end{center}

\textbf{મેમોરી વર્ગીકરણ ડાયાગ્રામ}:
\begin{center}
\begin{tikzpicture}[node distance=1.5cm, auto]
    \node [gtu block] (mem) {સેમિકન્ડક્ટર મેમોરીઝ};
    \node [gtu block, below left of=mem, yshift=-1cm, xshift=-1cm] (ram) {RAM (વોલેટાઇલ)};
    \node [gtu block, below right of=mem, yshift=-1cm, xshift=1cm] (rom) {ROM (નોન-વોલેટાઇલ)};
    \draw [gtu arrow] (mem) -- (ram);
    \draw [gtu arrow] (mem) -- (rom);
    \node [below of=ram] (sram) {SRAM, DRAM};
    \node [below of=rom] (prom) {PROM, EPROM, EEPROM};
\end{tikzpicture}
\end{center}
\end{solutionbox}
\mnemonicbox{મેમોરી પ્રકારો માટે "રિમેમ્બર સિમ્પલ ડિવિઝન: પ્રોગ્રામેબલ ઇરેઝેબલ ઇલેક્ટ્રિકલ".}

\end{document}

