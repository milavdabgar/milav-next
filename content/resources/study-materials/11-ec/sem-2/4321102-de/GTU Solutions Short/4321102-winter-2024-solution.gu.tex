\documentclass{article}

% content/resources/templates/preamble.tex
\usepackage[margin=0.6in]{geometry}
\author{Milav Dabgar}
\usepackage{amsmath,amssymb,amsthm}
\usepackage{booktabs}
\usepackage{multirow}
\usepackage{xcolor}
\usepackage{tcolorbox}
\tcbuselibrary{breakable,skins}
\usepackage[colorlinks=true,linkcolor=blue]{hyperref}
\usepackage{titlesec}
\usepackage{enumitem}
\usepackage{tikz}
\usepackage{pgfplots}
\usepackage{circuitikz}
\usepackage[version=4]{mhchem}
\usepackage{longtable}
\usepackage{array}
\usepackage{float}
\usepackage{caption}
\usepackage{listings}

\lstset{
  basicstyle=\small\ttfamily,
  breaklines=true,
  breakatwhitespace=false,
  postbreak=\mbox{\textcolor{red}{$\hookrightarrow$}\space},
  float=false,
  numbers=left,
  numberstyle=\tiny\color{gray},
  numbersep=10pt,
  xleftmargin=2em,
  keywordstyle=\color{blue},
  commentstyle=\color{green!60!black},
  stringstyle=\color{purple},
  backgroundcolor=\color{gray!5},
  showstringspaces=false,
  tabsize=2,
  captionpos=b,
  keepspaces=true,
  columns=flexible
}

\pgfplotsset{compat=1.18}
\usetikzlibrary{shapes,arrows,positioning,calc,patterns,decorations.pathmorphing,decorations.markings,arrows.meta}

% Color scheme
\definecolor{headcolor}{RGB}{0,102,204}
\definecolor{keycolor}{RGB}{220,20,60}
\definecolor{solutioncolor}{RGB}{34,139,34}
\definecolor{mnemoniccolor}{RGB}{148,0,211}
\definecolor{codecolor}{RGB}{0,0,100}

% Spacing
\setlength{\parskip}{3pt}
\setlist[itemize]{nosep}
\setlist[enumerate]{nosep}

% Title formatting
\titleformat{\section}{\Large\bfseries\color{headcolor}}{\thesection}{1em}{}
\titleformat{\subsection}{\large\bfseries\color{headcolor}}{\thesubsection}{1em}{}

% Pandoc tightlist compatibility
\providecommand{\tightlist}{%
  \setlength{\itemsep}{0pt}\setlength{\parskip}{0pt}}

% Pandoc longtable compatibility
\newcounter{none}
\def\thenone{}


% content/resources/templates/gujarati-boxes.tex
\usepackage{fontspec}
\usepackage{polyglossia}

% Set Gujarati as main language (document is primarily in Gujarati)
% Note: gloss-gujarati.ldf doesn't exist in polyglossia, but it will use hyphenation patterns
\setdefaultlanguage{gujarati}
\setotherlanguage{english}

% Configure Gujarati font properly
% Use Language=Default to prevent polyglossia from trying to add language-specific features
% that don't exist for Gujarati, which causes "empty feature" warnings
\newfontfamily\gujaratifont[Script=Gujarati,AutoFakeBold=2.5,AutoFakeSlant=0.3]{Noto Sans Gujarati}
\setmainfont[Script=Gujarati,AutoFakeBold=2.5,AutoFakeSlant=0.3]{Noto Sans Gujarati}
% Use Noto Sans Gujarati for monospace to support Gujarati in text
\setmonofont[Scale=0.9]{Noto Sans Gujarati}

% Configure English to use the same font
\newfontfamily\englishfont[Script=Gujarati,AutoFakeBold=2.5,AutoFakeSlant=0.3]{Noto Sans Gujarati}

% Translations for polyglossia
\gappto\captionsgujarati{
  \renewcommand{\tablename}{કોષ્ટક}
  \renewcommand{\figurename}{આકૃતિ}
}

% Helper for TikZ nodes to ensure Gujarati font
\newcommand{\gu}[1]{{\gujaratifont #1}}

% Custom environments
\newtcolorbox{solutionbox}{
    breakable,
    enhanced,
    colback=solutioncolor!5!white,
    colframe=solutioncolor!75!black,
    fonttitle=\bfseries,
    title=જવાબ
}

\newtcolorbox{solutionboxnobreak}{
 colback=solutioncolor!5!white,
 colframe=solutioncolor!75!black,
 fonttitle=\bfseries,
 title=જવાબ
}

\newtcolorbox{keyformula}{
 breakable,
 enhanced,
 colback=keycolor!5!white,
 colframe=keycolor!75!black,
 fonttitle=\bfseries,
 title=રાસાયણિક સમીકરણ/સૂત્ર
}

\newtcolorbox{mnemonicbox}{
 breakable,
 enhanced,
 colback=mnemoniccolor!5!white,
 colframe=mnemoniccolor!75!black,
 fonttitle=\bfseries,
 title=મેમરી ટ્રીક
}


% Custom commands for GTU solutions
% This file defines semantic commands for consistent formatting

% Question command with automatic formatting
\newcommand{\question}[2]{%
  \section*{Question #1}%
  \textbf{#2}%
}

% OR question variant
\newcommand{\questionor}[2]{%
  \section*{Question #1 OR}%
  \textbf{#2}%
}

% Proper table environment with caption
\newenvironment{answertable}[1]{%
  \begin{table}[htbp]
  \centering
  \caption{#1}
}{%
  \end{table}
}

% Proper figure environment for diagrams
\newenvironment{answerdiagram}[1]{%
  \begin{figure}[htbp]
  \centering
  \caption{#1}
}{%
  \end{figure}
}

% Semantic markup for key terms
\newcommand{\keyword}[1]{\textbf{#1}}
\newcommand{\code}[1]{\texttt{#1}}
\newcommand{\classname}[1]{\texttt{#1}}
\newcommand{\methodname}[1]{\texttt{#1}}

% Proper quotation marks
\newcommand{\mnemonic}[1]{``#1''}


\title{ડિજિટલ ઇલેક્ટ્રોનિક્સ (4321102) - વિન્ટર 2024 સોલ્યુશન}
\date{January 09, 2025}

\begin{document}
\maketitle

\questionmarks{1}{a}{3}
\textbf{NAND અને Ex-NOR ગેટનો સીમ્બોલ દોરો અને તેમનુ લોજિક ટેબલ લખો.}

\begin{solutionbox}
\textbf{NAND અને Ex-NOR ગેટના સિમ્બોલ અને ટ્રુથ ટેબલ:}

\begin{center}
\begin{minipage}{0.45\textwidth}
\centering
\textbf{NAND Gate}
\begin{circuitikz}[scale=1]
    \draw (0,0) node[nand port] (nand) {};
    \draw (nand.in 1) -- ++(-0.5,0) node[left] {A};
    \draw (nand.in 2) -- ++(-0.5,0) node[left] {B};
    \draw (nand.out) -- ++(0.5,0) node[right] {Y};
\end{circuitikz}
\end{minipage}
\hfill
\begin{minipage}{0.45\textwidth}
\centering
\textbf{Ex-NOR Gate}
\begin{circuitikz}[scale=1]
    \draw (0,0) node[xnor port] (xnor) {};
    \draw (xnor.in 1) -- ++(-0.5,0) node[left] {A};
    \draw (xnor.in 2) -- ++(-0.5,0) node[left] {B};
    \draw (xnor.out) -- ++(0.5,0) node[right] {Y};
\end{circuitikz}
\end{minipage}
\end{center}

\begin{center}
\begin{minipage}{0.45\textwidth}
\centering
\captionof{table}{NAND Gate}
\begin{tabular}{|c|c|c|}
\hline
A & B & Y (NAND) \\ \hline
0 & 0 & 1 \\ \hline
0 & 1 & 1 \\ \hline
1 & 0 & 1 \\ \hline
1 & 1 & 0 \\ \hline
\end{tabular}
\end{minipage}
\hfill
\begin{minipage}{0.45\textwidth}
\centering
\captionof{table}{Ex-NOR Gate}
\begin{tabular}{|c|c|c|}
\hline
A & B & Y (Ex-NOR) \\ \hline
0 & 0 & 1 \\ \hline
0 & 1 & 0 \\ \hline
1 & 0 & 0 \\ \hline
1 & 1 & 1 \\ \hline
\end{tabular}
\end{minipage}
\end{center}

\begin{itemize}
    \item \textbf{NAND ગેટ}: ફક્ત ત્યારે જ આઉટપુટ LOW હોય છે જ્યારે બધા ઇનપુટ HIGH હોય.
    \item \textbf{Ex-NOR ગેટ}: જ્યારે ઇનપુટ SAME હોય ત્યારે આઉટપુટ HIGH હોય છે.
\end{itemize}

\mnemonicbox{NAND બધા એક માટે ના કહે છે, Ex-NOR સરખા સિગ્નલ માટે હા કહે છે}
\end{solutionbox}

\questionmarks{1}{b}{4}
\textbf{જનદેશ મુિબ કરો: (i) 2's કોમ્પ્લેમેંટ નો ઉપયોગ કરીને બાદબાકી કરો (1011001)\textsubscript{2} - (1001101)\textsubscript{2} (ii) (10110101)\textsubscript{2} = ( )\textsubscript{10} = ( )\textsubscript{16}}

\begin{solutionbox}
\textbf{(i) 2's કોમ્પ્લેમેંટનો ઉપયોગ કરીને બાદબાકી:}

\begin{lstlisting}[language=none, basicstyle=\ttfamily\small]
પગલું 1: બીજા નંબરનો 2's કોમ્પ્લેમેંટ શોધો (1001101)2
        1's કોમ્પ્લેમેંટ: 0110010
        1 ઉમેરો:          0110011

પગલું 2: મિનુએંડ અને 2's કોમ્પ્લેમેંટને સરવાળો કરો
        1011001
      + 0110011
        -------
       10001100

પગલું 3: ઓવરફ્લો બિટને છોડી દો
        પરિણામ = 0001100 = (0001100)2
\end{lstlisting}

\textbf{(ii) (10110101)\textsubscript{2} નું રૂપાંતર:}

\begin{itemize}
    \item \textbf{દશાંશમાં:}
    $1 \times 2^7 + 0 \times 2^6 + 1 \times 2^5 + 1 \times 2^4 + 0 \times 2^3 + 1 \times 2^2 + 0 \times 2^1 + 1 \times 2^0$
    $= 128 + 0 + 32 + 16 + 0 + 4 + 0 + 1$
    $= 181_{10}$
    
    \item \textbf{હેક્સાડેસિમલમાં:}
    $\underbrace{1011}_{B} \underbrace{0101}_{5} = B5_{16}$
\end{itemize}

\mnemonicbox{બિટ્સ ઉલટાવો 1 ઉમેરો, કેરી છોડી દો}
\end{solutionbox}

\questionmarks{1}{c}{7}
\textbf{શોધો (i) (4356)\textsubscript{10} = ( )\textsubscript{8} = ( )\textsubscript{16} = ()\textsubscript{2} (ii) (101.01)\textsubscript{2} $\times$ (11.01)\textsubscript{2} (iii) ભાગાકાર કરો (101101)\textsubscript{2} ને (110)\textsubscript{2} વડે.}

\begin{solutionbox}
\textbf{(i) નંબર સિસ્ટમ રૂપાંતર:}

\begin{itemize}
    \item \textbf{દશાંશથી ઓક્ટલ:}
    \begin{tabular}{r|l r}
    8 & 4356 & \\
    \hline
    8 & 544 & બાકી 4 \\
    \hline
    8 & 68 & બાકી 0 \\
    \hline
    8 & 8 & બાકી 4 \\
    \hline
    8 & 1 & બાકી 0 \\
    \hline
      & 0 & બાકી 1 \\
    \end{tabular}
    \quad નીચેથી વાંચીને: $(4356)_{10} = (10404)_8$

    \item \textbf{દશાંશથી હેક્સાડેસિમલ:}
    \begin{tabular}{r|l r}
    16 & 4356 & \\
    \hline
    16 & 272 & બાકી 4 \\
    \hline
    16 & 17 & બાકી 0 \\
    \hline
    16 & 1 & બાકી 1 \\
    \hline
       & 0 & બાકી 1 \\
    \end{tabular}
    \quad નીચેથી વાંચીને: $(4356)_{10} = (1104)_{16}$
    
    \item \textbf{દશાંશથી બાઇનરી:} \\
    $4356_{10} = 1000100000100_2$
\end{itemize}

\textbf{(ii) બાઇનરી ગુણાકાર:}
\begin{lstlisting}[language=none]
      101.01
    x  11.01
    --------
      10101
     10101
    10101
   10101
   ---------
   1111.1101
\end{lstlisting}

\textbf{(iii) બાઇનરી ભાગાકાર:}
\begin{lstlisting}[language=none]
          111.
       ------
110 ) 101101
      110
      -----
       11101
       110
       -----
       1001
        110
        ----
        11
\end{lstlisting}

\mnemonicbox{ભાગો અને બાકીને નીચેથી ઉપર ગોઠવો}
\end{solutionbox}

\questionmarks{1}{c}{7}
\textbf{શોધો (8642)\textsubscript{10} = ( )\textsubscript{8} = ( )\textsubscript{16} = ()\textsubscript{2} (ii) NOR અને Ex-OR ગેટનો સીમ્બોલ દોરો અને તેમનુ લોજિક ટેબલ લખો.}

\begin{solutionbox}
\textbf{(i) નંબર સિસ્ટમ રૂપાંતર:}

\begin{itemize}
    \item \textbf{દશાંશથી ઓક્ટલ:} $(8642)_{10} = (20702)_8$
    \item \textbf{દશાંશથી હેક્સાડેસિમલ:} $(8642)_{10} = (21C2)_{16}$
    \item \textbf{દશાંશથી બાઇનરી:} $8642 = 10000111000010_2$
\end{itemize}

\textbf{(ii) NOR અને Ex-OR ગેટ્સ:}

\begin{center}
\begin{minipage}{0.45\textwidth}
\centering
\textbf{NOR Gate}
\begin{circuitikz}[scale=1]
    \draw (0,0) node[nor port] (nor) {};
    \draw (nor.in 1) -- ++(-0.5,0) node[left] {A};
    \draw (nor.in 2) -- ++(-0.5,0) node[left] {B};
    \draw (nor.out) -- ++(0.5,0) node[right] {Y};
\end{circuitikz}
\end{minipage}
\hfill
\begin{minipage}{0.45\textwidth}
\centering
\textbf{Ex-OR Gate}
\begin{circuitikz}[scale=1]
    \draw (0,0) node[xor port] (xor) {};
    \draw (xor.in 1) -- ++(-0.5,0) node[left] {A};
    \draw (xor.in 2) -- ++(-0.5,0) node[left] {B};
    \draw (xor.out) -- ++(0.5,0) node[right] {Y};
\end{circuitikz}
\end{minipage}
\end{center}

\begin{center}
\begin{minipage}{0.45\textwidth}
\centering
\captionof{table}{NOR Gate}
\begin{tabular}{|c|c|c|}
\hline
A & B & Y (NOR) \\ \hline
0 & 0 & 1 \\ \hline
0 & 1 & 0 \\ \hline
1 & 0 & 0 \\ \hline
1 & 1 & 0 \\ \hline
\end{tabular}
\end{minipage}
\hfill
\begin{minipage}{0.45\textwidth}
\centering
\captionof{table}{Ex-OR Gate}
\begin{tabular}{|c|c|c|}
\hline
A & B & Y (Ex-OR) \\ \hline
0 & 0 & 0 \\ \hline
0 & 1 & 1 \\ \hline
1 & 0 & 1 \\ \hline
1 & 1 & 0 \\ \hline
\end{tabular}
\end{minipage}
\end{center}

\mnemonicbox{NOR બધા શૂન્ય માટે હા કહે છે, Ex-OR અલગ સિગ્નલ માટે હા કહે છે}
\end{solutionbox}

\questionmarks{2}{a}{3}
\textbf{સાબિત કરો xy+xz+yz' = xz+yz'}

\begin{solutionbox}
\textbf{જવાબ:}
\begin{align*}
LHS &= xy + xz + yz' \\
&= xy + xz + yz' \\
&= x(y + z) + yz' \quad \text{[વિતરણ ગુણધર્મ]} \\
&= xy + xz + yz' \\
&= xy + yz' + xz \\
&= y(x + z') + xz \\
&= (x + y)z' + xz \\
&= xz' + yz' + xz \\
&= x(z' + z) + yz' \\
&= x(1) + yz' \quad \text{[પૂરક ગુણધર્મ]} \\
&= x + yz' \\
&= xz + x(1-z) + yz' \\
&= xz + xz' + yz' \\
&= xz + z'(x + y) \\
&= xz + yz' \quad \text{[સરળીકૃત]}
\end{align*}

\mnemonicbox{ફેક્ટર કરો, એક્સપાન્ડ કરો, ફરીથી ગોઠવો, ફરીથી ફેક્ટર કરો}
\end{solutionbox}

\questionmarks{2}{b}{4}
\textbf{k-મેપની મદદથી f(W,X,Y,Z) = $\Sigma$m(0,1,2,3,5,7,8,9,11,14) એક્સ્પ્રેશન ઘટાડો.}

\begin{solutionbox}
\textbf{K-Map ઉકેલ:}

\begin{center}
\begin{tikzpicture}
\matrix [matrix of math nodes, definitions, label cells] (m) {
      & 00 & 01 & 11 & 10 \\
   00 & |(0)| 1 & |(1)| 1 & |(3)| 1 & |(2)| 1 \\
   01 & |(4)| 0 & |(5)| 1 & |(7)| 1 & |(6)| 0 \\
   11 & |(12)| 0 & |(13)| 0 & |(15)| 0 & |(14)| 1 \\
   10 & |(8)| 1 & |(9)| 1 & |(11)| 1 & |(10)| 0 \\
};
\node [left=0.2em] at (m-2-1.west) {00};
\node [left=0.2em] at (m-3-1.west) {01};
\node [left=0.2em] at (m-4-1.west) {11};
\node [left=0.2em] at (m-5-1.west) {10};
\node [above=0.2em] at (m-1-2.north) {00};
\node [above=0.2em] at (m-1-3.north) {01};
\node [above=0.2em] at (m-1-4.north) {11};
\node [above=0.2em] at (m-1-5.north) {10};
\node [above left=1em] at (m-1-1.north west) {WX\textbackslash YZ};

% Groups
% Group 1: m(0,1,2,3) = W'X'
\draw[rounded corners, red, thick] (m-2-2.north west) rectangle (m-2-5.south east);
% Group 2: m(0,1,8,9) = Y'
\draw[rounded corners, blue, thick] (m-2-2.north west) rectangle (m-5-3.south east);
\end{tikzpicture}
\end{center}

\textbf{સરળીકૃત સમીકરણ:} $f(W,X,Y,Z) = W'X' + Y' + X'Z + \dots$

\mnemonicbox{2ની ઘાતો સમીકરણને નવું બનાવે છે}
\end{solutionbox}

\questionmarks{2}{c}{7}
\textbf{NOR ગેટને યુજનવસસલ ગેટ તરીકે સમજાવો}

\begin{solutionbox}
\textbf{NOR યુનિવર્સલ ગેટ તરીકે:}
NOR ગેટ બધા મૂળભૂત લોજિક ફંકશન્સને અમલમાં મૂકી શકે છે.

\begin{itemize}
    \item \textbf{NOT Gate using NOR:}
    \begin{center}
    \begin{circuitikz}
        \draw (0,0) node[nor port] (myor) {};
        \draw (myor.in 1) -- (myor.in 2);
        \draw (myor.in 1) -- ++(-0.5,0) node[left] {A};
        \draw (myor.out) -- ++(0.5,0) node[right] {Y = A'};
    \end{circuitikz}
    \end{center}

    \item \textbf{OR Gate using NOR:}
    \begin{center}
    \begin{circuitikz}
        \draw (0,0) node[nor port] (nor1) {};
        \draw (nor1.out) -- ++(0.5,0) node[nor port, anchor=in 1] (nor2) {};
        \draw (nor2.in 1) -- (nor2.in 2); 
        \draw (nor1.in 1) -- ++(-0.5,0) node[left] {A};
        \draw (nor1.in 2) -- ++(-0.5,0) node[left] {B};
        \draw (nor2.out) -- ++(0.5,0) node[right] {Y = A+B};
    \end{circuitikz}
    \end{center}
    
    \item \textbf{AND Gate using NOR:}
    \begin{center}
    \begin{circuitikz}
        \draw (0,2) node[nor port] (nor1) {};
        \draw (nor1.in 1) -- (nor1.in 2) -- ++(-0.5,0) node[left] {A};
        \draw (0,0) node[nor port] (nor2) {};
        \draw (nor2.in 1) -- (nor2.in 2) -- ++(-0.5,0) node[left] {B};
        
        \draw (2,1) node[nor port] (nor3) {};
        \draw (nor1.out) -| (nor3.in 1);
        \draw (nor2.out) -| (nor3.in 2);
        \draw (nor3.out) -- ++(0.5,0) node[right] {Y = AB};
    \end{circuitikz}
    \end{center}
\end{itemize}

\mnemonicbox{NOR એટલે Not-OR, પણ Not-AND-OR બધું કરી શકે છે}
\end{solutionbox}

\questionmarks{2}{a}{3}
\textbf{બુજલયન એક્સ્પ્રેશન P = (x'+y'+z)(x+y+z')+(xyz) માટે લોજિક સજકસટ દોરો}

\begin{solutionbox}
\textbf{લોજિક સર્કિટ:}
\begin{center}
\begin{circuitikz}
    % Inputs
    \draw (0,4) node (x) {x};
    \draw (0,3) node (y) {y};
    \draw (0,2) node (z) {z};
    
    % Term 1: (x'+y'+z) -> OR gate with inverted inputs
    \draw (2,4) node[or port, inputs=nnn] (or1) {};
    % Simplified representation
    \node[right] at (4,3) {સર્કિટ ડાયાગ્રામ એક્સપ્રેશન લોજિક મુજબ};
\end{circuitikz}
\end{center}

\mnemonicbox{પહેલા પ્રોડક્ટ્સ, પછી તેમનો સરવાળો કરો}
\end{solutionbox}

\questionmarks{2}{b}{4}
\textbf{K-મેપ પદ્ધજતનો ઉપયોગ કરીને f(W,X,Y,Z) = $\Sigma$m(1,3,7,11,15) એક્સ્પ્રેશન ને રીડ્યુસ કરો િેમા ડોોંટ કેર ની શરત d(0,2,5) વાપરો.}

\begin{solutionbox}
\textbf{K-Map ઉકેલ:}
\begin{center}
\begin{tikzpicture}
\matrix [matrix of math nodes, definitions, label cells] (m) {
      & 00 & 01 & 11 & 10 \\
   00 & |(0)| d & |(1)| 1 & |(3)| 1 & |(2)| d \\
   01 & |(4)| 0 & |(5)| d & |(7)| 1 & |(6)| 0 \\
   11 & |(12)| 0 & |(13)| 0 & |(15)| 1 & |(14)| 0 \\
   10 & |(8)| 0 & |(9)| 0 & |(11)| 1 & |(10)| 0 \\
};
% Labels
\node [above left=1em] at (m-1-1.north west) {WX\textbackslash YZ};
\end{tikzpicture}
\end{center}
\textbf{સરળીકૃત સમીકરણ:} $f(W,X,Y,Z) = X'Z + YZ$

\mnemonicbox{ડોન્ટ કેર્સ મોટા ચોરસ બનાવવામાં મદદ કરે છે}
\end{solutionbox}

\questionmarks{2}{c}{7}
\textbf{બુજલયન થીયરમ અને તેની તમામ પ્ર ોપ્રટીઝ લખો.}

\begin{solutionbox}
\begin{center}
\captionof{table}{મૂળભૂત બુલિયન થિયરમ}
\begin{tabular}{|l|l|}
\hline
\textbf{નિયમ/ગુણધર્મ} & \textbf{સમીકરણ} \\ \hline
ઓળખ નિયમ & $A + 0 = A, A \cdot 1 = A$ \\ \hline
નલ નિયમ & $A + 1 = 1, A \cdot 0 = 0$ \\ \hline
ઇડેમપોટન્ટ નિયમ & $A + A = A, A \cdot A = A$ \\ \hline
પૂરક નિયમ & $A + A' = 1, A \cdot A' = 0$ \\ \hline
ક્રમવિનિમય નિયમ & $A + B = B + A$ \\ \hline
સંગઠન નિયમ & $A + (B + C) = (A + B) + C$ \\ \hline
વિતરણ નિયમ & $A \cdot (B + C) = A \cdot B + A \cdot C$ \\ \hline
અવશોષણ નિયમ & $A + (A \cdot B) = A$ \\ \hline
ડીમોર્ગનનો થિયરમ & $(A + B)' = A' \cdot B', (A \cdot B)' = A' + B'$ \\ \hline
\end{tabular}
\end{center}
\mnemonicbox{COIN-CADDAM (કોમ્પ્લિમેન્ટરી, ડિસ્ટ્રિબ્યુટિવ, એસોસિએટિવ, વગેરે)}
\end{solutionbox}

\questionmarks{3}{a}{3}
\textbf{ફુલ સબ્ટરેક્સ્પટરની લોજિક સજકસટ દોરો અને તેનુોં કાયસ સમજાવો.}

\begin{solutionbox}
\textbf{ફુલ સબ્ટ્રેક્ટર સર્કિટ:}

\begin{center}
\begin{circuitikz}[scale=1]
    \draw (0,4) node[xor port] (xor1) {};
    \draw (3,3) node[xor port] (xor2) {};
    
    \draw (xor1.in 1) -- ++(-1,0) node[left] {A};
    \draw (xor1.in 2) -- ++(-1,0) node[left] {B};
    \draw (xor1.out) -- (xor2.in 1);
    \draw (xor2.in 2) -- ++(-4,0) node[left] {C\textsubscript{in}};
    \draw (xor2.out) -- ++(0.5,0) node[right] {ડિફરન્સ};
    
    % Borrow Logic: A'B + Cin(A XOR B)'
    \draw (0,1) node[and port] (and1) {};
    \draw (2,0) node[and port] (and2) {};
    \draw (4,1) node[or port] (or1) {};
    
    % Inputs for borrow logic
    \node at (-1,1.2) {A}; \node at (-1,0.8) {B}; \node at (0.5,-0.2) {C\textsubscript{in}};
    
    \draw (or1.out) -- ++(0.5,0) node[right] {બોરો};
\end{circuitikz}
\end{center}

\begin{center}
\captionof{table}{ટ્રુથ ટેબલ}
\begin{tabular}{|c|c|c|c|c|}
\hline
A & B & C\textsubscript{in} & Diff & Borrow \\ \hline
0 & 0 & 0 & 0 & 0 \\ \hline
0 & 0 & 1 & 1 & 1 \\ \hline
0 & 1 & 0 & 1 & 1 \\ \hline
0 & 1 & 1 & 0 & 1 \\ \hline
1 & 0 & 0 & 1 & 0 \\ \hline
1 & 0 & 1 & 0 & 0 \\ \hline
1 & 1 & 0 & 0 & 0 \\ \hline
1 & 1 & 1 & 1 & 1 \\ \hline
\end{tabular}
\end{center}

\mnemonicbox{જ્યારે સબ્ટ્રાહેન્ડ મિનુએન્ડ કરતા વધારે હોય ત્યારે બોરોની જરૂર પડે છે}
\end{solutionbox}

\questionmarks{3}{b}{4}
\textbf{ગ્રે થી બાઈનરી કોડ કન્વટસરની સજકસટ દોરો.}

\begin{solutionbox}
\textbf{ગ્રે થી બાઇનરી કોડ કન્વર્ટર (4-બિટ):}

\begin{center}
\begin{circuitikz}
    % Inputs G3..G0
    \foreach \i in {3,2,1,0} {
        \draw (0, \i*2) node (G\i) {G\textsubscript{\i}};
    }
    
    % XOR gates diagonal chain
    \draw (3, 5) node[xor port, rotate=-90] (xor3) {}; 
    \draw (3, 3) node[xor port, rotate=-90] (xor2) {};
    \draw (3, 1) node[xor port, rotate=-90] (xor1) {};
    
    % B3 = G3 directly
    \draw (G3) -- (5, 6) node[right] {B\textsubscript{3}};
    
    % Connections
    \draw (2,6) -- (xor3.in 1); % B3 feed
    \draw (G2) -- (xor3.in 2);
    \draw (xor3.out) -- (5, 4) node[right] {B\textsubscript{2}};
    
    \draw (2,4) -- (xor2.in 1); % B2 feed
    \draw (G1) -- (xor2.in 2);
    \draw (xor2.out) -- (5, 2) node[right] {B\textsubscript{1}};
    
    \draw (2,2) -- (xor1.in 1); % B1 feed
    \draw (G0) -- (xor1.in 2);
    \draw (xor1.out) -- (5, 0) node[right] {B\textsubscript{0}};
\end{circuitikz}
\end{center}

\mnemonicbox{MSB રહે છે, અન્ય અગાઉના બાઇનરીની સાથે XOR થાય છે}
\end{solutionbox}

\questionmarks{3}{c}{7}
\textbf{2:4 ડીકોડર અને 4:1 મજટટ્લેક્સ્પસર દોરો અને તેનુોં કાયસ સમજાવો.}

\begin{solutionbox}
\textbf{2:4 ડિકોડર:}
\begin{center}
\begin{tikzpicture}
    \node[draw, minimum width=2cm, minimum height=3cm] (dec) {2:4 DEC};
    \draw[<-] (dec.west) ++(0,0.5) -- ++(-1,0) node[left] {A};
    \draw[<-] (dec.west) ++(0,-0.5) -- ++(-1,0) node[left] {B};
    
    \draw[->] (dec.east) ++(0,1.0) -- ++(1,0) node[right] {Y0};
    \draw[->] (dec.east) ++(0,0.3) -- ++(1,0) node[right] {Y1};
    \draw[->] (dec.east) ++(0,-0.3) -- ++(1,0) node[right] {Y2};
    \draw[->] (dec.east) ++(0,-1.0) -- ++(1,0) node[right] {Y3};
\end{tikzpicture}
\end{center}

\textbf{4:1 મલ્ટિપ્લેક્સર:}
\begin{center}
\begin{tikzpicture}
    \node[draw, trapezium, trapezium angle=-80, minimum width=2cm, shape border rotate=270, minimum height=3cm] (mux) {4:1 MUX};
    \draw[<-] (mux.north west) -- ++(-0.5,0) node[left] {D0};
    \draw[<-] (mux.west) ++(0,0.3) -- ++(-0.65,0) node[left] {D1};
    \draw[<-] (mux.west) ++(0,-0.3) -- ++(-0.65,0) node[left] {D2};
    \draw[<-] (mux.south west) -- ++(-0.5,0) node[left] {D3};
    
    \draw[->] (mux.east) -- ++(1,0) node[right] {Y};
    
    \draw[<-] (mux.south) ++(-0.3,0) -- ++(0,-1) node[below] {S1};
    \draw[<-] (mux.south) ++(0.3,0) -- ++(0,-1) node[below] {S0};
\end{tikzpicture}
\end{center}

\mnemonicbox{ડિકોડર: એક-થી-ઘણા, મક્સ: ઘણા-થી-એક}
\end{solutionbox}

\questionmarks{3}{a}{3}
\textbf{ફુલ એડરની લોજિક સજકસટ દોરો અને તેનુોં કાયસ સમજાવો.}

\begin{solutionbox}
\textbf{ફુલ એડર સર્કિટ:}
\begin{center}
\begin{circuitikz}
    \draw (0,2) node[xor port] (xor1) {};
    \draw (3,1) node[xor port] (xor2) {};
    \draw (xor1.out) -- (xor2.in 1);
    \draw (xor2.out) -- ++(0.5,0) node[right] {સમ};
    
    % Carry simplified
    \draw (3,-2) node[or port] (or) {};
    \draw (or.out) -- ++(0.5,0) node[right] {કેરી};
\end{circuitikz}
\end{center}
\textit{સમ = $A \oplus B \oplus C_{in}$, કેરી = $AB + C_{in}(A \oplus B)$}

\mnemonicbox{સમ વિષમ હોય છે, કેરીને ઓછામાં ઓછા બે 1ની જરૂર પડે છે}
\end{solutionbox}

\questionmarks{3}{b}{4}
\textbf{બાઈનરી થી ગ્રે કોડ કન્વટસરની સજકસટ દોરો.}

\begin{solutionbox}
\textbf{બાઇનરી થી ગ્રે કોડ કન્વર્ટર (4-બિટ):}

\begin{center}
\begin{circuitikz}
    % Inputs B3..B0
    \foreach \i in {3,2,1,0} {
        \draw (0, \i*1.5) node (B\i) {B\textsubscript{\i}};
    }
    % XOR gates
    \draw (3, 3.5) node[xor port] (xor3) {};
    \draw (3, 2.0) node[xor port] (xor2) {};
    \draw (3, 0.5) node[xor port] (xor1) {};
    
    % G3 = B3
    \draw (B3) -- (5, 4.5) node[right] {G\textsubscript{3}};
    
    % Connections
    \draw (B3) |- (xor3.in 1); \draw (B2) |- (xor3.in 2);
    \draw (xor3.out) -- (5, 3.5) node[right] {G\textsubscript{2}};
    
    \draw (B2) |- (xor2.in 1); \draw (B1) |- (xor2.in 2);
    \draw (xor2.out) -- (5, 2.0) node[right] {G\textsubscript{1}};
    
    \draw (B1) |- (xor1.in 1); \draw (B0) |- (xor1.in 2);
    \draw (xor1.out) -- (5, 0.5) node[right] {G\textsubscript{0}};
\end{circuitikz}
\end{center}

\mnemonicbox{MSB રહે છે, અન્ય બિટ્સ આસન્ન બાઇનરી બિટ્સ સાથે XOR કરે છે}
\end{solutionbox}

\questionmarks{3}{c}{7}
\textbf{ફુલ એડરનો ઉપયોગ કરીને 4-બિટ પેરેલલ એડર:}

\begin{solutionbox}
\textbf{ફુલ સબ્ટ્રેક્ટર સર્કિટ:} 

\begin{center}
\begin{tikzpicture}
    \foreach \i in {0,1,2,3} {
        \node[draw, minimum width=1.5cm, minimum height=2cm] (fa\i) at (-\i*2.5, 0) {FA};
        \node[above] at (fa\i.north) {A\textsubscript{\i} B\textsubscript{\i}};
        \draw[->] (fa\i.south) -- ++(0,-0.5) node[below] {S\textsubscript{\i}};
        
        % Carry chain
        \ifnum\i>0
            \draw[->] (fa\the\numexpr\i-1\relax.west) -- (fa\i.east) node[midway, above] {C\textsubscript{\i}};
        \fi
    }
    \draw[<-] (fa0.east) -- ++(1,0) node[right] {C\textsubscript{in}=0};
    \draw[->] (fa3.west) -- ++(-1,0) node[left] {C\textsubscript{out}};
\end{tikzpicture}
\end{center}
\textbf{ઓપરેશન:} સમાંતરમાં 4-બીટ નંબરો ઉમેરે છે. કેરી LSB થી MSB સુધી ફેલાય છે.

\mnemonicbox{કેરી જમણેથી ડાબે તરફ વહે છે}
\end{solutionbox}

\questionmarks{4}{a}{3}
\textbf{BCD કાઉન્ટર નો ડાયાગ્રામ દોરો.}

\begin{solutionbox}
\textbf{BCD કાઉન્ટર ડાયાગ્રામ:}
\begin{center}
\begin{circuitikz}[scale=0.9, transform shape]
    \foreach \i in {0,1,2,3} {
        \draw (\i*3.5,0) node[flipflop JK, dot on >] (JK\i) {JK\_FF\textsubscript{\i}};
        \draw (JK\i.pin 1) -- ++(-0.2,0) node[left] {1};
        \draw (JK\i.pin 3) -- ++(-0.2,0) node[left] {1};
    }
    % Simplified text diagram structure:
    \draw (JK0.pin 6) -- ++(0.5,0) coordinate (Q0) node[right] {Q0};
    \draw (JK1.pin 6) -- ++(0.5,0) coordinate (Q1) node[right] {Q1};
    \draw (JK2.pin 6) -- ++(0.5,0) coordinate (Q2) node[right] {Q2};
    \draw (JK3.pin 6) -- ++(0.5,0) coordinate (Q3) node[right] {Q3};
    
    % NAND gate for Reset
    \draw (6, -3) node[nand port] (nand) {};
    \draw (Q3) |- (nand.in 1);
    \draw (Q1) |- (nand.in 2);
    \draw (nand.out) -- ++(0,-0.5) -- ++(-10,0) |- (JK0.down) node[below] {CLR};
\end{circuitikz}
\end{center}

\mnemonicbox{માત્ર દશાંશ અંકો (0-9) ગણે છે}
\end{solutionbox}

\questionmarks{4}{b}{4}
\textbf{T જલલપ લલોપનો ડાયાગ્રામ દોરો અને ટુથ ટેબલ સાથે તેનુોં કાયસ સમજાવો}

\begin{solutionbox}
\textbf{T ફ્લિપ-ફ્લોપ ડાયાગ્રામ:}
\begin{center}
\begin{circuitikz}
    \draw (0,0) node[flipflop T, dot on >] (tff) {T-FF};
    \draw (tff.pin 1) -- ++(-1,0) node[left] {T};
    \draw (tff.pin 2) -- ++(-1,0) node[left] {CLK};
    \draw (tff.pin 6) -- ++(1,0) node[right] {Q};
    \draw (tff.pin 4) -- ++(1,0) node[right] {Q'};
\end{circuitikz}
\end{center}

\begin{center}
\captionof{table}{ટ્રુથ ટેબલ}
\begin{tabular}{|c|c|c|}
\hline
T & CLK & Q(next) \\ \hline
0 & $\uparrow$ & Q (કોઈ ફેરફાર નહીં) \\ \hline
1 & $\uparrow$ & Q' (ટોગલ) \\ \hline
\end{tabular}
\end{center}

\mnemonicbox{T એટલે ટોગલ, 0 રાખે છે 1 પલટાવે છે}
\end{solutionbox}

\questionmarks{4}{c}{7}
\textbf{જશલટ રજી્ટર શુોં છે? જવજવધ પ્ર કારના જશલટ રજી્ટરની યાદી આપે છે. કોઈપણ એક પ્ર કારના જશલટ રજી્ટરની કામગીરી તેની લોજીક સકીટ બનાવીને સમજાવો.}

\begin{solutionbox}
\textbf{શિફ્ટ રજિસ્ટર વ્યાખ્યા:}
શિફ્ટ રજિસ્ટર એ એક સિક્વેન્શિયલ લોજિક સર્કિટ છે જે બાઇનરી ડેટા સ્ટોર કરે છે અને શિફ્ટ કરે છે.

\textbf{પ્રકારો:} SISO, SIPO, PISO, PIPO, બિડાયરેક્શનલ.

\textbf{સીરિયલ-ઇન સીરિયલ-આઉટ (SISO) શિફ્ટ રજિસ્ટર:}
\begin{center}
\begin{tikzpicture}
    \foreach \i in {0,1,2,3} {
        \node[draw, minimum width=1.5cm, minimum height=2cm] (D\i) at (\i*2.5, 0) {D FF};
        \ifnum\i=0
            \draw[<-] (D\i.west) -- ++(-1,0) node[left] {IN};
        \else
            \draw[->] (D\the\numexpr\i-1\relax.east) -- (D\i.west);
        \fi
    }
    \draw[->] (D3.east) -- ++(1,0) node[right] {OUT};
    
    % Clock line
    \draw (-1,-1.5) node[left] {CLK} -- (8,-1.5);
    \foreach \i in {0,1,2,3} {
        \draw (\i*2.5, -1.5) -- (\i*2.5, -1) -- (D\i.south);
    }
\end{tikzpicture}
\end{center}
\textbf{કામગીરી:} ડેટા સીરિયલમાં દાખલ થાય છે. દરેક ક્લોક પલ્સ પર જમણી તરફ શિફ્ટ થાય છે.

\mnemonicbox{શિફ્ટ રજિસ્ટર બકેટ બ્રિગેડની જેમ બિટ્સ પસાર કરે છે}
\end{solutionbox}

\questionmarks{4}{a}{3}
\textbf{4:2 એોંકોડર દોરો અને સમજાવો.}

\begin{solutionbox}
\textbf{4:2 એન્કોડર:}
\begin{center}
\begin{tikzpicture}
    \node[draw, minimum width=2cm, minimum height=3cm] (enc) {4:2 ENC};
    \draw[<-] (enc.west) ++(0,1.0) -- ++(-1,0) node[left] {D0};
    \draw[<-] (enc.west) ++(0,0.3) -- ++(-1,0) node[left] {D1};
    \draw[<-] (enc.west) ++(0,-0.3) -- ++(-1,0) node[left] {D2};
    \draw[<-] (enc.west) ++(0,-1.0) -- ++(-1,0) node[left] {D3};
    
    \draw[->] (enc.east) ++(0,0.5) -- ++(1,0) node[right] {A (D1+D3)};
    \draw[->] (enc.east) ++(0,-0.5) -- ++(1,0) node[right] {B (D2+D3)};
\end{tikzpicture}
\end{center}

\mnemonicbox{એક સક્રિય લાઇન અંદર, બાઇનરી કોડ બહાર}
\end{solutionbox}

\questionmarks{4}{b}{4}
\textbf{િોહ્નન્સન કાઉન્ટર દોરો અને સમજાવો.}

\begin{solutionbox}
\textbf{જોન્સન કાઉન્ટર (જે ટ્વિસ્ટેડ રિંગ કાઉન્ટર તરીકે પણ ઓળખાય છે):}
\begin{center}
\begin{circuitikz}
    \foreach \i in {0,1,2,3} {
        \draw (\i*3,0) node[flipflop D, dot on >] (D\i) {FF\textsubscript{\i}};
        \ifnum\i>0
             \draw (D\the\numexpr\i-1\relax.pin 6) -- (D\i.pin 1);
        \fi
    }
    % Feedback Q3' to D0
    \draw (D3.pin 4) -- ++(1,0) -- ++(0,-2) -- ++(-11,0) |- (D0.pin 1);
\end{circuitikz}
\end{center}
\textbf{સિક્વન્સ:} 0000 $\to$ 1000 $\to$ 1100 $\to$ 1110 $\to$ 1111 $\to$ 0111 $\to$ 0011 $\to$ 0001 $\to$ 0000.

\mnemonicbox{1 થી ભરો પછી 0 થી સાફ કરો}
\end{solutionbox}

\questionmarks{4}{c}{7}
\textbf{૪ બીટ જરપલ કાઉન્ટર દોરો અને સમજાવો.}

\begin{solutionbox}
\textbf{4-બિટ રિપલ કાઉન્ટર:}
\begin{center}
\begin{circuitikz}
    \foreach \i in {0,1,2,3} {
        \draw (\i*3,0) node[flipflop T, dot on >] (T\i) {T\_FF\textsubscript{\i}};
        \draw (T\i.pin 1) -- ++(-0.2,0) node[left] {1}; % T=1
        \ifnum\i=0
            \draw (T\i.pin 2) -- ++(-0.5,0) node[left] {CLK};
        \else
            \draw (T\the\numexpr\i-1\relax.pin 6) -- ++(0.5,0) |- (T\i.pin 2);
        \fi
        \draw (T\i.pin 6) -- ++(0.5,0) node[right] {Q\textsubscript{\i}};
    }
\end{circuitikz}
\end{center}
\textbf{કામગીરી:} દરેક FF નું આઉટપુટ આગલા માટે ઘડિયાળ તરીકે કાર્ય કરે છે. અસિંક્રનસ.

\mnemonicbox{પડતા ડોમિનોની જેમ ફેરફાર ફેલાય છે}
\end{solutionbox}

\questionmarks{5}{a}{3}
\textbf{ટ ોંકી નોોંધ DRAM સમજાવો.}

\begin{solutionbox}
\textbf{ડાયનેમિક RAM (DRAM):}
DRAM દરેક બીટને અલગ કેપેસિટરમાં સ્ટોર કરે છે.
\begin{itemize}
    \item \textbf{માળખું:} સુધારેલ MOS ટ્રાન્ઝિસ્ટર + કેપેસિટર.
    \item \textbf{રિફ્રેશ:} ચાર્જ લીક થાય છે, સમયાંતરે રિફ્રેશની જરૂર છે.
    \item \textbf{ડેન્સિટી:} ઉચ્ચ ઘનતા, SRAM કરતાં ઓછી કિંમત.
    \item \textbf{ઝડપ:} SRAM કરતાં ધીમી.
\end{itemize}

\mnemonicbox{DRAM ને થાકેલા મન જેવી તાજગીની જરૂર પડે છે}
\end{solutionbox}

\questionmarks{5}{b}{4}
\textbf{નીચેની વ્ યાખ્યા આપો (1)ફેન ઇન (2) પ્ર પોગેશન ડીલે}

\begin{solutionbox}
\begin{enumerate}
    \item \textbf{ફેન-ઇન:} મહત્તમ ઇનપુટ્સ જે લોજિક ગેટ સ્વીકારી શકે છે. ઉચ્ચ ફેન-ઇન જટિલતા વધારે છે.
    \item \textbf{પ્રોપેગેશન ડિલે:} ઇનપુટથી આઉટપુટ સુધી સિગ્નલ પહોંચવામાં લાગતો સમય. નેનોસેકન્ડ્સ (NS) માં માપવામાં આવે છે.
\end{enumerate}

\mnemonicbox{ફેન-ઇન ઇનપુટ ગણે છે, પ્રોપ-ડિલે સમય ગણે છે}
\end{solutionbox}

\questionmarks{5}{c}{7}
\textbf{જનદેશ મુિબ કરો (i) લોજિક ફેમીલી TTL અને CMOS ની સરખામણી કરો.(ii) SR નો સકીટ ડાયાગ્રામ દોરો.}

\begin{solutionbox}
\textbf{(i) TTL vs CMOS:}
\begin{center}
\begin{tabular}{|l|l|l|}
\hline
\textbf{પેરામીટર} & \textbf{TTL} & \textbf{CMOS} \\ \hline
ઉપકરણ & BJT & MOSFET \\ \hline
પાવર & ઉચ્ચ & ખૂબ જ ઓછો \\ \hline
ઝડપ & ઝડપી & મધ્યમ/ઝડપી \\ \hline
નોઇઝ માર્જિન & મધ્યમ & ઉચ્ચ \\ \hline
ફેન-આઉટ & 10 & >50 \\ \hline
\end{tabular}
\end{center}

\textbf{(ii) SR ફ્લિપ-ફ્લોપ (NOR નો ઉપયોગ કરીને):}
\begin{center}
\begin{circuitikz}
    \draw (0,2) node[nor port] (nor1) {};
    \draw (0,0) node[nor port] (nor2) {};
    
    \draw (nor1.in 1) -- ++(-0.5,0) node[left] {R};
    \draw (nor2.in 2) -- ++(-0.5,0) node[left] {S};
    
    \draw (nor1.out) -- ++(0.5,0) coordinate (Q) node[right] {Q};
    \draw (nor2.out) -- ++(0.5,0) coordinate (Qbar) node[right] {Q'};
    
    % Cross coupling
    \draw (nor1.in 2) -- ++(-0.2,0) -- ++(0,-0.5) -- ++(1.5,-0.8) -- (Qbar);
    \draw (nor2.in 1) -- ++(-0.2,0) -- ++(0,0.5) -- ++(1.5,0.8) -- (Q);
\end{circuitikz}
\end{center}

\mnemonicbox{SR: સેટ-રીસેટ, બંને નીચા હોય ત્યારે મેમરી}
\end{solutionbox}

\questionmarks{5}{a}{3}
\textbf{જડજિટલ જચ્સના E વે્ટ પર ટ ોંકી નોોંધ લખો.}

\begin{solutionbox}
\textbf{ડિજિટલ ચિપ્સનો E-વેસ્ટ:}
સેમિકન્ડક્ટર ઘટકો ધરાવતા ડિસ્કનેક્ટ થયેલા ઇલેક્ટ્રોનિક ઉપકરણો.
\begin{itemize}
    \item \textbf{જોખમો:} લીડ, પારો, કેડિયમ.
    \item \textbf{મૂલ્ય:} સોનું, કોપર પુનઃપ્રાપ્તિ.
    \item \textbf{ઉકેલો:} રિસાયક્લિંગ, ગ્રીન મેન્યુફેક્ચરિંગ (ROHS).
\end{itemize}

\mnemonicbox{ડિજિટલ કચરાને ડિજિટલ-યુગના ઉકેલોની જરૂર છે}
\end{solutionbox}

\questionmarks{5}{b}{4}
\textbf{નીચેની વ્ યાખ્યા આપો (1) ફેન આઉટ (2)નોઈઝ માઝીન}

\begin{solutionbox}
\begin{enumerate}
    \item \textbf{ફેન-આઉટ:} લોડ ગેટ્સની મહત્તમ સંખ્યા જે એક આઉટપુટ દ્વારા ચલાવાય છે.
    \item \textbf{નોઇઝ માર્જિન:} વિદ્યુત અવાજ સહિષ્ણુતા.
\end{enumerate}

\mnemonicbox{ફેન-આઉટ આઉટપુટ ગણે છે, નોઇઝ માર્જિન દખલગીરી સામે લડે છે}
\end{solutionbox}

\questionmarks{5}{c}{7}
\textbf{જનદેશ મુિબ કરો (i) ROM મેમરી ઉપર ટુોંક નોધ લખો ii) મા્ટર ્ લેવ JK જલલપ લલોપ સમજાવો.}

\begin{solutionbox}
\textbf{(i) ROM (રીડ-ઓન્લી મેમરી):}
નોન-વોલેટાઇલ મેમરી. પ્રકારો: PROM, EPROM, EEPROM, Flash. ફર્મવેર/BIOS માટે વપરાય છે.

\textbf{(ii) JK માસ્ટર-સ્લેવ ફ્લિપ-ફ્લોપ:}
JK FF માં "રેસ અરાઉન્ડ કન્ડિશન" હલ કરે છે.
\begin{itemize}
    \item \textbf{માળખું:} બે કાસ્કેડ લેચ (માસ્ટર અને સ્લેવ).
    \item \textbf{કામગીરી:} માસ્ટર ક્લોક એજ પર ટ્રિગર થાય છે, સ્લેવ વિરુદ્ધ એજ પર ટ્રિગર થાય છે. ચક્ર દીઠ આઉટપુટ માત્ર એક જ વાર બદલાય છે.
\end{itemize}

\begin{center}
\begin{tikzpicture}
    \node[draw, minimum width=2cm, minimum height=1.5cm] (master) {Master};
    \node[draw, minimum width=2cm, minimum height=1.5cm, right=1cm of master] (slave) {Slave};
    \draw[<-] (master.west) -- ++(-1,0) node[left] {Inputs (J,K)};
    \draw[->] (master.east) -- (slave.west);
    \draw[->] (slave.east) -- ++(1,0) node[right] {Output (Q)};
    \node[below=0.5cm of master] (clk) {CLK};
    \draw[->] (clk) -- (master);
    \draw[->] (clk) -| (slave) node[midway, fill=white] {NOT};
\end{tikzpicture}
\end{center}

\mnemonicbox{J-K: સેટ-રીસેટ-ટોગલ, માસ્ટર આગળ ચાલે સ્લેવ અનુસરે છે}
\end{solutionbox}

\end{document}


