\documentclass[10pt,a4paper]{article}

% content/resources/templates/preamble.tex
\usepackage[margin=0.6in]{geometry}
\author{Milav Dabgar}
\usepackage{amsmath,amssymb,amsthm}
\usepackage{booktabs}
\usepackage{multirow}
\usepackage{xcolor}
\usepackage{tcolorbox}
\tcbuselibrary{breakable,skins}
\usepackage[colorlinks=true,linkcolor=blue]{hyperref}
\usepackage{titlesec}
\usepackage{enumitem}
\usepackage{tikz}
\usepackage{pgfplots}
\usepackage{circuitikz}
\usepackage[version=4]{mhchem}
\usepackage{longtable}
\usepackage{array}
\usepackage{float}
\usepackage{caption}
\usepackage{listings}

\lstset{
  basicstyle=\small\ttfamily,
  breaklines=true,
  breakatwhitespace=false,
  postbreak=\mbox{\textcolor{red}{$\hookrightarrow$}\space},
  float=false,
  numbers=left,
  numberstyle=\tiny\color{gray},
  numbersep=10pt,
  xleftmargin=2em,
  keywordstyle=\color{blue},
  commentstyle=\color{green!60!black},
  stringstyle=\color{purple},
  backgroundcolor=\color{gray!5},
  showstringspaces=false,
  tabsize=2,
  captionpos=b,
  keepspaces=true,
  columns=flexible
}

\pgfplotsset{compat=1.18}
\usetikzlibrary{shapes,arrows,positioning,calc,patterns,decorations.pathmorphing,decorations.markings,arrows.meta}

% Color scheme
\definecolor{headcolor}{RGB}{0,102,204}
\definecolor{keycolor}{RGB}{220,20,60}
\definecolor{solutioncolor}{RGB}{34,139,34}
\definecolor{mnemoniccolor}{RGB}{148,0,211}
\definecolor{codecolor}{RGB}{0,0,100}

% Spacing
\setlength{\parskip}{3pt}
\setlist[itemize]{nosep}
\setlist[enumerate]{nosep}

% Title formatting
\titleformat{\section}{\Large\bfseries\color{headcolor}}{\thesection}{1em}{}
\titleformat{\subsection}{\large\bfseries\color{headcolor}}{\thesubsection}{1em}{}

% Pandoc tightlist compatibility
\providecommand{\tightlist}{%
  \setlength{\itemsep}{0pt}\setlength{\parskip}{0pt}}

% Pandoc longtable compatibility
\newcounter{none}
\def\thenone{}


% content/resources/templates/english-boxes.tex
% This file is currently empty - it exists to maintain consistency with the import structure.
% Add custom environments here if needed in the future.


\begin{document}

\begin{center}
{\Huge\bfseries\color{headcolor} Subject Name Solutions}\\[5pt]
{\LARGE 4321102 -- Winter 2023}\\[3pt]
{\large Semester 1 Study Material}\\[3pt]
{\normalsize\textit{Detailed Solutions and Explanations}}
\end{center}

\vspace{10pt}

\subsection*{Question 1(a) [3 marks]}\label{q1a}

\textbf{(726)_{1}_{0} = (\_\_\_\_\_\_\_\_\_)_{2}}

\begin{solutionbox}


{\def\LTcaptype{none} % do not increment counter
\vspace{-5pt}
\captionof{table}{Decimal to Binary Conversion}
\vspace{-10pt}
\begin{longtable}[]{@{}lll@{}}
\toprule\noalign{}
Step & Calculation & Remainder \\
\midrule\noalign{}
\endhead
\bottomrule\noalign{}
\endlastfoot
1 & 726 \div 2 = 363 & 0 \\
2 & 363 \div 2 = 181 & 1 \\
3 & 181 \div 2 = 90 & 1 \\
4 & 90 \div 2 = 45 & 0 \\
5 & 45 \div 2 = 22 & 1 \\
6 & 22 \div 2 = 11 & 0 \\
7 & 11 \div 2 = 5 & 1 \\
8 & 5 \div 2 = 2 & 1 \\
9 & 2 \div 2 = 1 & 0 \\
10 & 1 \div 2 = 0 & 1 \\
\end{longtable}
}

Reading from bottom to top: (726)_{1}_{0} = (1011010110)_{2}

\end{solutionbox}
\begin{mnemonicbox}
``Divide By Two, Read Remainders Up''

\end{mnemonicbox}
\subsection*{Question 1(b) [4 marks]}\label{q1b}

\textbf{1) Convert binary number (10110101)_{2} into gray number.}

\textbf{2) Convert gray number (10110110)gray into binary number.}

\begin{solutionbox}

\textbf{Binary to Gray Conversion:}

\begin{verbatim}
Binary:   1 0 1 1 0 1 0 1
           ↓ ↓ ↓ ↓ ↓ ↓ ↓
XOR:      1\oplus0 0\oplus1 1\oplus1 1\oplus0 0\oplus1 1\oplus0 0\oplus1
           ↓   ↓   ↓   ↓   ↓   ↓   ↓
Gray:     1   1   0   1   1   1   1
\end{verbatim}

Therefore: (10110101)_{2} = (1101111)gray

\textbf{Gray to Binary Conversion:}

\begin{verbatim}
Gray:     1 0 1 1 0 1 1 0
           ↓
Binary:   1
          1\oplus0 = 1
          1\oplus1 = 0
          0\oplus1 = 1
          1\oplus0 = 1
          1\oplus1 = 0
          0\oplus1 = 1
          1\oplus0 = 1
\end{verbatim}

Therefore: (10110110)gray = (10110101)_{2}

\end{solutionbox}
\begin{mnemonicbox}
``First bit same, rest XOR with previous binary''

\end{mnemonicbox}
\subsection*{Question 1(c) [7 marks]}\label{q1c}

\textbf{Explain NAND as a universal gate.}

\begin{solutionbox}

\textbf{Diagram: NAND as Universal Gate}

\begin{center}
\textbf{Mermaid Diagram (Code)}
\begin{verbatim}
{Shaded}
{Highlighting}[]
graph TD
    subgraph "NOT Gate using NAND"
    A1(A){-{-}{}N1((NAND))}
    A1(A){-{-}{}N1}
    N1{-{-}{}Z1("A{}")}
    end

    subgraph "AND Gate using NAND"
    A2(A){-{-}{}N2((NAND))}
    B2(B){-{-}{}N2}
    N2{-{-}{}N3((NAND))}
    N2{-{-}{}N3}
    N3{-{-}{}Z2("A·B")}
    end
    
    subgraph "OR Gate using NAND"
    A3(A){-{-}{}N4((NAND))}
    A3(A){-{-}{}N4}
    B3(B){-{-}{}N5((NAND))}
    B3(B){-{-}{}N5}
    N4{-{-}{}N6((NAND))}
    N5{-{-}{}N6}
    N6{-{-}{}Z3("A+B")}
    end
{Highlighting}
{Shaded}
\end{verbatim}
\end{center}

\begin{itemize}
\tightlist
\item
  \textbf{Universal Property}: NAND gate can implement any Boolean
  function without needing any other type of gate
\item
  \textbf{NOT Implementation}: Connecting both inputs of NAND together
  creates NOT gate
\item
  \textbf{AND Implementation}: NAND followed by another NAND creates AND
  gate
\item
  \textbf{OR Implementation}: Two NAND gates with single inputs,
  followed by NAND creates OR gate
\end{itemize}


{\def\LTcaptype{none} % do not increment counter
\vspace{-5pt}
\captionof{table}{NAND Gate Implementations}
\vspace{-10pt}
\begin{longtable}[]{@{}ll@{}}
\toprule\noalign{}
Logic Function & NAND Implementation \\
\midrule\noalign{}
\endhead
\bottomrule\noalign{}
\endlastfoot
NOT(A) & NAND(A,A) \\
AND(A,B) & NAND(NAND(A,B),NAND(A,B)) \\
OR(A,B) & NAND(NAND(A,A),NAND(B,B)) \\
\end{longtable}
}

\end{solutionbox}
\begin{mnemonicbox}
``NAND can STAND as All gates''

\end{mnemonicbox}
\subsection*{Question 1(c) OR [7
marks]}\label{q1c}

\textbf{Explain NOR as a universal gate.}

\begin{solutionbox}

\textbf{Diagram: NOR as Universal Gate}

\begin{center}
\textbf{Mermaid Diagram (Code)}
\begin{verbatim}
{Shaded}
{Highlighting}[]
graph TD
    subgraph "NOT Gate using NOR"
    A1(A){-{-}{}N1((NOR))}
    A1(A){-{-}{}N1}
    N1{-{-}{}Z1("A{}")}
    end

    subgraph "OR Gate using NOR"
    A2(A){-{-}{}N2((NOR))}
    B2(B){-{-}{}N2}
    N2{-{-}{}N3((NOR))}
    N2{-{-}{}N3}
    N3{-{-}{}Z2("A+B")}
    end
    
    subgraph "AND Gate using NOR"
    A3(A){-{-}{}N4((NOR))}
    A3(A){-{-}{}N4}
    B3(B){-{-}{}N5((NOR))}
    B3(B){-{-}{}N5}
    N4{-{-}{}N6((NOR))}
    N5{-{-}{}N6}
    N6{-{-}{}Z3("A·B")}
    end
{Highlighting}
{Shaded}
\end{verbatim}
\end{center}

\begin{itemize}
\tightlist
\item
  \textbf{Universal Property}: NOR gate can implement any Boolean
  function without needing any other type of gate
\item
  \textbf{NOT Implementation}: Connecting both inputs of NOR together
  creates NOT gate
\item
  \textbf{OR Implementation}: NOR followed by another NOR creates OR
  gate
\item
  \textbf{AND Implementation}: Two NOR gates with single inputs,
  followed by NOR creates AND gate
\end{itemize}


{\def\LTcaptype{none} % do not increment counter
\vspace{-5pt}
\captionof{table}{NOR Gate Implementations}
\vspace{-10pt}
\begin{longtable}[]{@{}ll@{}}
\toprule\noalign{}
Logic Function & NOR Implementation \\
\midrule\noalign{}
\endhead
\bottomrule\noalign{}
\endlastfoot
NOT(A) & NOR(A,A) \\
OR(A,B) & NOR(NOR(A,B),NOR(A,B)) \\
AND(A,B) & NOR(NOR(A,A),NOR(B,B)) \\
\end{longtable}
}

\end{solutionbox}
\begin{mnemonicbox}
``NOR can form ALL logic cores''

\end{mnemonicbox}
\subsection*{Question 2(a) [3 marks]}\label{q2a}

\textbf{(11011011)_{2} X (110)_{2} = (\_\_\_\_\_\_\_\_\_)_{2}}

\begin{solutionbox}


\textbf{Table: Binary Multiplication}
\begin{verbatim}
    1 1 0 1 1 0 1 1
  \times         1 1 0
  ---------------
    1 1 0 1 1 0 1 1  (\times 0)
  1 1 0 1 1 0 1 1    (\times 1)
1 1 0 1 1 0 1 1      (\times 1)
-----------------
1 0 0 0 0 0 0 0 1 1 0
\end{verbatim}

Therefore: (11011011)_{2} \times (110)_{2} = (10000001110)_{2}

\end{solutionbox}
\begin{mnemonicbox}
``Multiply each bit, shift left, add all rows''

\end{mnemonicbox}
\subsection*{Question 2(b) [4 marks]}\label{q2b}

\textbf{Prove DeMorgan's theorem.}

\begin{solutionbox}


{\def\LTcaptype{none} % do not increment counter
\vspace{-5pt}
\captionof{table}{DeMorgan's Theorem Proof}
\vspace{-10pt}
\begin{longtable}[]{@{}lllllll@{}}
\toprule\noalign{}
A & B & A' & B' & A+B & (A+B)' & A'·B' \\
\midrule\noalign{}
\endhead
\bottomrule\noalign{}
\endlastfoot
0 & 0 & 1 & 1 & 0 & 1 & 1 \\
0 & 1 & 1 & 0 & 1 & 0 & 0 \\
1 & 0 & 0 & 1 & 1 & 0 & 0 \\
1 & 1 & 0 & 0 & 1 & 0 & 0 \\
\end{longtable}
}

DeMorgan's Theorems: 1. (A+B)' = A'·B' 2. (A·B)' = A'+B'

Truth table proves that (A+B)' = A'·B' since both columns match.

\end{solutionbox}
\begin{mnemonicbox}
``Break the line, change the sign''

\end{mnemonicbox}
\subsection*{Question 2(c) [7 marks]}\label{q2c}

\textbf{Explain full adder using logic circuit, Boolean equation and
truth table.}

\begin{solutionbox}

\textbf{Diagram: Full Adder Circuit}

\begin{center}
\textbf{Mermaid Diagram (Code)}
\begin{verbatim}
{Shaded}
{Highlighting}[]
graph LR
    A(A){-{-}{}XOR1(XOR)}
    B(B){-{-}{}XOR1}
    XOR1{-{-}{}XOR2(XOR)}
    Cin(Cin){-{-}{}XOR2}
    XOR2{-{-}{}Sum(Sum)}

    A{-{-}{}AND1(AND)}
    B{-{-}{}AND1}
    AND1{-{-}{}OR1(OR)}
    
    A{-{-}{}AND2(AND)}
    Cin{-{-}{}AND2}
    AND2{-{-}{}OR1}
    
    B{-{-}{}AND3(AND)}
    Cin{-{-}{}AND3}
    AND3{-{-}{}OR1}
    
    OR1{-{-}{}Cout(Cout)}
{Highlighting}
{Shaded}
\end{verbatim}
\end{center}


{\def\LTcaptype{none} % do not increment counter
\vspace{-5pt}
\captionof{table}{Full Adder Truth Table}
\vspace{-10pt}
\begin{longtable}[]{@{}lllll@{}}
\toprule\noalign{}
A & B & Cin & Sum & Cout \\
\midrule\noalign{}
\endhead
\bottomrule\noalign{}
\endlastfoot
0 & 0 & 0 & 0 & 0 \\
0 & 0 & 1 & 1 & 0 \\
0 & 1 & 0 & 1 & 0 \\
0 & 1 & 1 & 0 & 1 \\
1 & 0 & 0 & 1 & 0 \\
1 & 0 & 1 & 0 & 1 \\
1 & 1 & 0 & 0 & 1 \\
1 & 1 & 1 & 1 & 1 \\
\end{longtable}
}

\begin{itemize}
\tightlist
\item
  \textbf{Boolean Equations}:

  \begin{itemize}
  \tightlist
  \item
    Sum = A \oplus B \oplus Cin
  \item
    Cout = (A·B) + (B·Cin) + (A·Cin)
  \end{itemize}
\end{itemize}

\end{solutionbox}
\begin{mnemonicbox}
``Sum needs XOR three, Carry needs AND then OR''

\end{mnemonicbox}
\subsection*{Question 2(a) OR [3
marks]}\label{q2a}

\textbf{Divide (11010010)_{2} with (101)_{2} = (\_\_\_\_\_\_\_\_\_)_{2}}

\begin{solutionbox}


\textbf{Table: Binary Division}
\begin{verbatim}
            1 0 1 0 1 1
         ____________
101 ) 1 1 0 1 0 0 1 0
      1 0 1
      -----
        1 1 0
        1 0 1
        -----
          0 1 0
            0 0
          -----
            1 0 0
            1 0 1
            -----
              1 1 0
              1 0 1
              -----
                0 1 0
                  0 0
                -----
                  1 0
                   0
                 ----
                   0
\end{verbatim}

Therefore: (11010010)_{2} \div (101)_{2} = (101011)_{2} with remainder (0)_{2}

\end{solutionbox}
\begin{mnemonicbox}
``Divide like decimal, but use binary subtraction''

\end{mnemonicbox}
\subsection*{Question 2(b) OR [4
marks]}\label{q2b}

\textbf{Simplify the Boolean expression Y = A'B+AB'+A'B'+AB}

\begin{solutionbox}


{\def\LTcaptype{none} % do not increment counter
\vspace{-5pt}
\captionof{table}{Boolean Simplification}
\vspace{-10pt}
\begin{longtable}[]{@{}lll@{}}
\toprule\noalign{}
Step & Expression & Rule Applied \\
\midrule\noalign{}
\endhead
\bottomrule\noalign{}
\endlastfoot
1 & Y = A'B+AB'+A'B'+AB & Original \\
2 & Y = A'(B+B')+A(B'+B) & Factoring \\
3 &

Y = A'(1)+A(1) & B+B' = 1 \\

4 & Y = A'+A & Simplifying \\
5 &

Y = 1 & A'+A = 1 \\

\end{longtable}
}

Therefore: Y = 1 (Always TRUE)

\end{solutionbox}
\begin{mnemonicbox}
``Factor first, apply identities, combine like
terms''

\end{mnemonicbox}
\subsection*{Question 2(c) OR [7
marks]}\label{q2c}

\textbf{Explain full subtractor using logic circuit, boolean equation
and truth table.}

\begin{solutionbox}

\textbf{Diagram: Full Subtractor Circuit}

\begin{center}
\textbf{Mermaid Diagram (Code)}
\begin{verbatim}
{Shaded}
{Highlighting}[]
graph LR
    A(A){-{-}{}XOR1(XOR)}
    B(B){-{-}{}XOR1}
    XOR1{-{-}{}XOR2(XOR)}
    Bin(Bin){-{-}{}XOR2}
    XOR2{-{-}{}D(Difference)}

    A(A){-{-}{}NOT1(NOT)}
    NOT1{-{-}{}AND1(AND)}
    B{-{-}{}AND1}
    AND1{-{-}{}OR1(OR)}
    
    XOR1{-{-}{}NOT2(NOT)}
    NOT2{-{-}{}AND2(AND)}
    Bin{-{-}{}AND2}
    AND2{-{-}{}OR1}
    
    B{-{-}{}AND3(AND)}
    Bin{-{-}{}AND3}
    AND3{-{-}{}OR1}
    
    OR1{-{-}{}Bout(Borrow Out)}
{Highlighting}
{Shaded}
\end{verbatim}
\end{center}


{\def\LTcaptype{none} % do not increment counter
\vspace{-5pt}
\captionof{table}{Full Subtractor Truth Table}
\vspace{-10pt}
\begin{longtable}[]{@{}lllll@{}}
\toprule\noalign{}
A & B & Bin & Difference & Bout \\
\midrule\noalign{}
\endhead
\bottomrule\noalign{}
\endlastfoot
0 & 0 & 0 & 0 & 0 \\
0 & 0 & 1 & 1 & 1 \\
0 & 1 & 0 & 1 & 1 \\
0 & 1 & 1 & 0 & 1 \\
1 & 0 & 0 & 1 & 0 \\
1 & 0 & 1 & 0 & 0 \\
1 & 1 & 0 & 0 & 0 \\
1 & 1 & 1 & 1 & 1 \\
\end{longtable}
}

\begin{itemize}
\tightlist
\item
  \textbf{Boolean Equations}:

  \begin{itemize}
  \tightlist
  \item
    Difference = A \oplus B \oplus Bin
  \item
    Bout = (A'·B) + (A'·Bin) + (B·Bin)
  \end{itemize}
\end{itemize}

\end{solutionbox}
\begin{mnemonicbox}
``Difference uses triple XOR, Borrow when input is
greater''

\end{mnemonicbox}
\subsection*{Question 3(a) [3 marks]}\label{q3a}

\textbf{Using 2's complement subtract (1011001)_{2} from (1101101)_{2}.}

\begin{solutionbox}


{\def\LTcaptype{none} % do not increment counter
\vspace{-5pt}
\captionof{table}{2's Complement Subtraction}
\vspace{-10pt}
\begin{longtable}[]{@{}lll@{}}
\toprule\noalign{}
Step & Operation & Result \\
\midrule\noalign{}
\endhead
\bottomrule\noalign{}
\endlastfoot
1 & Number to subtract: & 1011001 \\
2 & 1's complement: & 0100110 \\
3 & 2's complement: & 0100111 \\
4 & (1101101) + (0100111) = & 10010100 \\
5 & Discard carry: & 0010100 \\
\end{longtable}
}

Therefore: (1101101)_{2} - (1011001)_{2} = (0010100)_{2} = (20)_{1}_{0}

\end{solutionbox}
\begin{mnemonicbox}
``Flip bits, add one, then add numbers''

\end{mnemonicbox}
\subsection*{Question 3(b) [4 marks]}\label{q3b}

\textbf{Simplify the Boolean equation using Karnaugh map (K' map)
method: F(A,B,C,D) = Σm(0,1,2,6,7,8,12,15)}

\begin{solutionbox}


\textbf{Table: Karnaugh Map}
\begin{verbatim}
      CD      
AB    00  01  11  10
00    1   1   0   1
01    0   0   1   1
11    0   0   1   0
10    1   0   0   0
\end{verbatim}

\textbf{Diagram: K-map Grouping}

\begin{verbatim}
+{-{-}{-}{-}{-}+{-}{-}{-}{-}{-}+{-}{-}{-}{-}{-}+{-}{-}{-}{-}{-}+}
|  1  |  1  |  0  |  1  |
|  A  |  A  |     |  A  |
+{-{-}{-}{-}{-}+{-}{-}{-}{-}{-}+{-}{-}{-}{-}{-}+{-}{-}{-}{-}{-}+}
|  0  |  0  |  1  |  1  |
|     |     |  B  |  B  |
+{-{-}{-}{-}{-}+{-}{-}{-}{-}{-}+{-}{-}{-}{-}{-}+{-}{-}{-}{-}{-}+}
|  0  |  0  |  1  |  0  |
|     |     |  B  |     |
+{-{-}{-}{-}{-}+{-}{-}{-}{-}{-}+{-}{-}{-}{-}{-}+{-}{-}{-}{-}{-}+}
|  1  |  0  |  0  |  0  |
|  C  |     |     |     |
+{-{-}{-}{-}{-}+{-}{-}{-}{-}{-}+{-}{-}{-}{-}{-}+{-}{-}{-}{-}{-}+}
\end{verbatim}

Group A: A'B'C' (4 cells) Group B: BCD (3 cells) Group C: A'B'CD' (1
cell)

Simplified expression: F(A,B,C,D) = A'B'C' + BCD + A'B'CD'

\end{solutionbox}
\begin{mnemonicbox}
``Find largest groups of 2^{n}, use minimal terms''

\end{mnemonicbox}
\subsection*{Question 3(c) [7 marks]}\label{q3c}

\textbf{Explain 3 to 8 decoder using logic circuit and truth table.}

\begin{solutionbox}

\textbf{Diagram: 3-to-8 Decoder}

\begin{center}
\textbf{Mermaid Diagram (Code)}
\begin{verbatim}
{Shaded}
{Highlighting}[]
graph TD
    A(A){-{-}{}NOT1(NOT)}
    B(B){-{-}{}NOT2(NOT)}
    C(C){-{-}{}NOT3(NOT)}

    NOT1{-{-}{}AND0(AND)}
    NOT2{-{-}{}AND0}
    NOT3{-{-}{}AND0}
    AND0{-{-}{}D0(D0)}
    
    NOT1{-{-}{}AND1(AND)}
    NOT2{-{-}{}AND1}
    C{-{-}{}AND1}
    AND1{-{-}{}D1(D1)}
    
    NOT1{-{-}{}AND2(AND)}
    B{-{-}{}AND2}
    NOT3{-{-}{}AND2}
    AND2{-{-}{}D2(D2)}
    
    NOT1{-{-}{}AND3(AND)}
    B{-{-}{}AND3}
    C{-{-}{}AND3}
    AND3{-{-}{}D3(D3)}
    
    A{-{-}{}AND4(AND)}
    NOT2{-{-}{}AND4}
    NOT3{-{-}{}AND4}
    AND4{-{-}{}D4(D4)}
    
    A{-{-}{}AND5(AND)}
    NOT2{-{-}{}AND5}
    C{-{-}{}AND5}
    AND5{-{-}{}D5(D5)}
    
    A{-{-}{}AND6(AND)}
    B{-{-}{}AND6}
    NOT3{-{-}{}AND6}
    AND6{-{-}{}D6(D6)}
    
    A{-{-}{}AND7(AND)}
    B{-{-}{}AND7}
    C{-{-}{}AND7}
    AND7{-{-}{}D7(D7)}
{Highlighting}
{Shaded}
\end{verbatim}
\end{center}


{\def\LTcaptype{none} % do not increment counter
\vspace{-5pt}
\captionof{table}{3-to-8 Decoder Truth Table}
\vspace{-10pt}
\begin{longtable}[]{@{}llllllllll@{}}
\toprule\noalign{}
Inputs & & Outputs & & & & & & & \\
\midrule\noalign{}
\endhead
\bottomrule\noalign{}
\endlastfoot
A & B & C & D0 & D1 & D2 & D3 & D4 & D5 & D6 \\
0 & 0 & 0 & 1 & 0 & 0 & 0 & 0 & 0 & 0 \\
0 & 0 & 1 & 0 & 1 & 0 & 0 & 0 & 0 & 0 \\
0 & 1 & 0 & 0 & 0 & 1 & 0 & 0 & 0 & 0 \\
0 & 1 & 1 & 0 & 0 & 0 & 1 & 0 & 0 & 0 \\
1 & 0 & 0 & 0 & 0 & 0 & 0 & 1 & 0 & 0 \\
1 & 0 & 1 & 0 & 0 & 0 & 0 & 0 & 1 & 0 \\
1 & 1 & 0 & 0 & 0 & 0 & 0 & 0 & 0 & 1 \\
1 & 1 & 1 & 0 & 0 & 0 & 0 & 0 & 0 & 0 \\
\end{longtable}
}

\begin{itemize}
\tightlist
\item
  \textbf{Function}: Activates one of 8 output lines based on 3-bit
  binary input
\item
  \textbf{Applications}: Memory addressing, data routing, instruction
  decoding
\item
  \textbf{Boolean Equations}: D0 = A'·B'·C', D1 = A'·B'·C, etc.
\end{itemize}

\end{solutionbox}
\begin{mnemonicbox}
``One hot output at binary address''

\end{mnemonicbox}
\subsection*{Question 3(a) OR [3
marks]}\label{q3a}

\textbf{Do as directed. 1) (101011010111)_{2} = (\_\_\_\_\_\_\_\_\_\_\_)_{8}}

\begin{solutionbox}


\textbf{Table: Binary to Octal Conversion}
\begin{verbatim}
Binary:    1 | 010 | 110 | 101 | 11
           ↓    ↓     ↓     ↓    ↓
Octal:     1    2     6     5    3
\end{verbatim}

Therefore: (101011010111)_{2} = (12653)_{8}

\end{solutionbox}
\begin{mnemonicbox}
``Group by threes, right to left''

\end{mnemonicbox}
\subsection*{Question 3(b) OR [4
marks]}\label{q3b}

\textbf{Simplify the Boolean equation using Karnaugh map (K' map)
method: F(A,B,C,D) = Σm(1,3,5,7,8,9,10,11)}

\begin{solutionbox}


\textbf{Table: Karnaugh Map}
\begin{verbatim}
      CD      
AB    00  01  11  10
00    0   1   1   0
01    0   1   1   0
11    0   0   0   0
10    1   1   1   1
\end{verbatim}

\textbf{Diagram: K-map Grouping}

\begin{verbatim}
+{-{-}{-}{-}{-}+{-}{-}{-}{-}{-}+{-}{-}{-}{-}{-}+{-}{-}{-}{-}{-}+}
|  0  |  1  |  1  |  0  |
|     |  A  |  A  |     |
+{-{-}{-}{-}{-}+{-}{-}{-}{-}{-}+{-}{-}{-}{-}{-}+{-}{-}{-}{-}{-}+}
|  0  |  1  |  1  |  0  |
|     |  A  |  A  |     |
+{-{-}{-}{-}{-}+{-}{-}{-}{-}{-}+{-}{-}{-}{-}{-}+{-}{-}{-}{-}{-}+}
|  0  |  0  |  0  |  0  |
|     |     |     |     |
+{-{-}{-}{-}{-}+{-}{-}{-}{-}{-}+{-}{-}{-}{-}{-}+{-}{-}{-}{-}{-}+}
|  1  |  1  |  1  |  1  |
|  B  |  B  |  B  |  B  |
+{-{-}{-}{-}{-}+{-}{-}{-}{-}{-}+{-}{-}{-}{-}{-}+{-}{-}{-}{-}{-}+}
\end{verbatim}

Group A: A'CD (4 cells) Group B: AB' (4 cells)

Simplified expression: F(A,B,C,D) = A'CD + AB'

\end{solutionbox}
\begin{mnemonicbox}
``Group powers of 2, minimize variables''

\end{mnemonicbox}
\subsection*{Question 3(c) OR [7
marks]}\label{q3c}

\textbf{Explain 8 to 1 multiplexer using logic circuit and truth table.}

\begin{solutionbox}

\textbf{Diagram: 8-to-1 Multiplexer}

\begin{center}
\textbf{Mermaid Diagram (Code)}
\begin{verbatim}
{Shaded}
{Highlighting}[]
graph TD
    D0(D0){-{-}{}AND0(AND)}
    D1(D1){-{-}{}AND1(AND)}
    D2(D2){-{-}{}AND2(AND)}
    D3(D3){-{-}{}AND3(AND)}
    D4(D4){-{-}{}AND4(AND)}
    D5(D5){-{-}{}AND5(AND)}
    D6(D6){-{-}{}AND6(AND)}
    D7(D7){-{-}{}AND7(AND)}

    S0(S0){-{-}{}NOT0(NOT)}
    S1(S1){-{-}{}NOT1(NOT)}
    S2(S2){-{-}{}NOT2(NOT)}
    
    NOT0{-{-}{}AND0}
    NOT1{-{-}{}AND0}
    NOT2{-{-}{}AND0}
    
    S0{-{-}{}AND1}
    NOT1{-{-}{}AND1}
    NOT2{-{-}{}AND1}
    
    NOT0{-{-}{}AND2}
    S1{-{-}{}AND2}
    NOT2{-{-}{}AND2}
    
    S0{-{-}{}AND3}
    S1{-{-}{}AND3}
    NOT2{-{-}{}AND3}
    
    NOT0{-{-}{}AND4}
    NOT1{-{-}{}AND4}
    S2{-{-}{}AND4}
    
    S0{-{-}{}AND5}
    NOT1{-{-}{}AND5}
    S2{-{-}{}AND5}
    
    NOT0{-{-}{}AND6}
    S1{-{-}{}AND6}
    S2{-{-}{}AND6}
    
    S0{-{-}{}AND7}
    S1{-{-}{}AND7}
    S2{-{-}{}AND7}
    
    AND0{-{-}{}OR1(OR)}
    AND1{-{-}{}OR1}
    AND2{-{-}{}OR1}
    AND3{-{-}{}OR1}
    AND4{-{-}{}OR1}
    AND5{-{-}{}OR1}
    AND6{-{-}{}OR1}
    AND7{-{-}{}OR1}
    
    OR1{-{-}{}Y(Output Y)}
{Highlighting}
{Shaded}
\end{verbatim}
\end{center}


{\def\LTcaptype{none} % do not increment counter
\vspace{-5pt}
\captionof{table}{8-to-1 Multiplexer Truth Table}
\vspace{-10pt}
\begin{longtable}[]{@{}llll@{}}
\toprule\noalign{}
Select Lines & & & Output \\
\midrule\noalign{}
\endhead
\bottomrule\noalign{}
\endlastfoot
S2 & S1 & S0 & Y \\
0 & 0 & 0 & D0 \\
0 & 0 & 1 & D1 \\
0 & 1 & 0 & D2 \\
0 & 1 & 1 & D3 \\
1 & 0 & 0 & D4 \\
1 & 0 & 1 & D5 \\
1 & 1 & 0 & D6 \\
1 & 1 & 1 & D7 \\
\end{longtable}
}

\begin{itemize}
\tightlist
\item
  \textbf{Function}: Selects one of 8 input data lines and routes it to
  output
\item
  \textbf{Applications}: Data routing, function generation,
  parallel-to-serial conversion
\item
  \textbf{Boolean Equation}: Y = S2'·S1'·S0'·D0 + S2'·S1'·S0·D1 +
  \ldots{} + S2·S1·S0·D7
\end{itemize}

\end{solutionbox}
\begin{mnemonicbox}
``Select bits route one input to output''

\end{mnemonicbox}
\subsection*{Question 4(a) [3 marks]}\label{q4a}

\textbf{Draw the logic circuit for binary to gray convertor.}

\begin{solutionbox}

\textbf{Diagram: Binary to Gray Code Converter}

\begin{center}
\textbf{Mermaid Diagram (Code)}
\begin{verbatim}
{Shaded}
{Highlighting}[]
graph TD
    B3(B3){-{-}{}G3(G3)}
    B3{-{-}{}XOR1(XOR)}
    B2(B2){-{-}{}XOR1}
    XOR1{-{-}{}G2(G2)}
    B2{-{-}{}XOR2(XOR)}
    B1(B1){-{-}{}XOR2}
    XOR2{-{-}{}G1(G1)}
    B1{-{-}{}XOR3(XOR)}
    B0(B0){-{-}{}XOR3}
    XOR3{-{-}{}G0(G0)}
{Highlighting}
{Shaded}
\end{verbatim}
\end{center}

\begin{itemize}
\tightlist
\item
  \textbf{Binary Inputs}: B3, B2, B1, B0 (most to least significant
  bits)
\item
  \textbf{Gray Outputs}: G3, G2, G1, G0 (most to least significant bits)
\item
  \textbf{Conversion Rule}: G3 = B3, G2 = B3 \oplus B2, G1 = B2 \oplus B1, G0 = B1
  \oplus B0
\end{itemize}

\end{solutionbox}
\begin{mnemonicbox}
``First bit same, rest XOR neighbors''

\end{mnemonicbox}
\subsection*{Question 4(b) [4 marks]}\label{q4b}

\textbf{Explain working of Serial in Serial out shift register.}

\begin{solutionbox}

\textbf{Diagram: Serial-In Serial-Out Shift Register}

\begin{center}
\textbf{Mermaid Diagram (Code)}
\begin{verbatim}
{Shaded}
{Highlighting}[]
graph LR
    Din(Data In){-{-}{}FF0(FF0)}
    CLK(Clock){-{-}{}FF0}
    CLK{-{-}{}FF1(FF1)}
    CLK{-{-}{}FF2(FF2)}
    CLK{-{-}{}FF3(FF3)}
    FF0{-{-}{}FF1}
    FF1{-{-}{}FF2}
    FF2{-{-}{}FF3}
    FF3{-{-}{}Dout(Data Out)}
{Highlighting}
{Shaded}
\end{verbatim}
\end{center}


{\def\LTcaptype{none} % do not increment counter
\vspace{-5pt}
\captionof{table}{Serial-In Serial-Out Operation}
\vspace{-10pt}
\begin{longtable}[]{@{}llllll@{}}
\toprule\noalign{}
Clock Cycle & FF0 & FF1 & FF2 & FF3 & Data Out \\
\midrule\noalign{}
\endhead
\bottomrule\noalign{}
\endlastfoot
Initial & 0 & 0 & 0 & 0 & 0 \\
1 (Din=1) & 1 & 0 & 0 & 0 & 0 \\
2 (Din=0) & 0 & 1 & 0 & 0 & 0 \\
3 (Din=1) & 1 & 0 & 1 & 0 & 0 \\
4 (Din=1) & 1 & 1 & 0 & 1 & 1 \\
\end{longtable}
}

\begin{itemize}
\tightlist
\item
  \textbf{Operation}: Data bits enter serially at input, shift through
  all flip-flops, and exit serially
\item
  \textbf{Applications}: Data transmission, time delay, serial-to-serial
  conversion
\item
  \textbf{Features}: Simple design, requires fewer I/O pins but more
  clock cycles
\end{itemize}

\end{solutionbox}
\begin{mnemonicbox}
``One bit in, shift all, one bit out''

\end{mnemonicbox}
\subsection*{Question 4(c) [7 marks]}\label{q4c}

\textbf{Explain workings of D flip flop and JK flip flop using circuit
diagram and truth table.}

\begin{solutionbox}

\textbf{Diagram: D Flip-Flop}

\begin{center}
\textbf{Mermaid Diagram (Code)}
\begin{verbatim}
{Shaded}
{Highlighting}[]
graph LR
    D(D){-{-}{}DFF(D Flip{-}Flop)}
    CLK(Clock){-{-}{}DFF}
    DFF{-{-}{}Q(Q)}
    DFF{-{-}{}Q{}("Q{}")}
{Highlighting}
{Shaded}
\end{verbatim}
\end{center}


{\def\LTcaptype{none} % do not increment counter
\vspace{-5pt}
\captionof{table}{D Flip-Flop Truth Table}
\vspace{-10pt}
\begin{longtable}[]{@{}lll@{}}
\toprule\noalign{}
D & Clock & Q(next) \\
\midrule\noalign{}
\endhead
\bottomrule\noalign{}
\endlastfoot
0 & ↑ & 0 \\
1 & ↑ & 1 \\
\end{longtable}
}

\textbf{Diagram: JK Flip-Flop}

\begin{center}
\textbf{Mermaid Diagram (Code)}
\begin{verbatim}
{Shaded}
{Highlighting}[]
graph LR
    J(J){-{-}{}JKFF(JK Flip{-}Flop)}
    K(K){-{-}{}JKFF}
    CLK(Clock){-{-}{}JKFF}
    JKFF{-{-}{}Q(Q)}
    JKFF{-{-}{}Q{}("Q{}")}
{Highlighting}
{Shaded}
\end{verbatim}
\end{center}


{\def\LTcaptype{none} % do not increment counter
\vspace{-5pt}
\captionof{table}{JK Flip-Flop Truth Table}
\vspace{-10pt}
\begin{longtable}[]{@{}llll@{}}
\toprule\noalign{}
J & K & Clock & Q(next) \\
\midrule\noalign{}
\endhead
\bottomrule\noalign{}
\endlastfoot
0 & 0 & ↑ & Q(no change) \\
0 & 1 & ↑ & 0 \\
1 & 0 & ↑ & 1 \\
1 & 1 & ↑ & Q' (toggle) \\
\end{longtable}
}

\begin{itemize}
\tightlist
\item
  \textbf{D Flip-Flop}: Data (D) input is transferred to output Q at
  positive clock edge
\item
  \textbf{JK Flip-Flop}: More versatile with set (J), reset (K), hold
  and toggle capabilities
\item
  \textbf{Applications}: Storage elements, counters, registers,
  sequential circuits
\end{itemize}

\end{solutionbox}
\begin{mnemonicbox}
``D Does what D is, JK Juggles Keep-Toggle-Set''

\end{mnemonicbox}
\subsection*{Question 4(a) OR [3
marks]}\label{q4a}

\textbf{Draw the logic circuit for gray to binary convertor.}

\begin{solutionbox}

\textbf{Diagram: Gray to Binary Code Converter}

\begin{center}
\textbf{Mermaid Diagram (Code)}
\begin{verbatim}
{Shaded}
{Highlighting}[]
graph LR
    G3(G3){-{-}{}B3(B3)}
    G3{-{-}{}XOR1(XOR)}
    G2(G2){-{-}{}XOR1}
    XOR1{-{-}{}B2(B2)}
    XOR1{-{-}{}XOR2(XOR)}
    G1(G1){-{-}{}XOR2}
    XOR2{-{-}{}B1(B1)}
    XOR2{-{-}{}XOR3(XOR)}
    G0(G0){-{-}{}XOR3}
    XOR3{-{-}{}B0(B0)}
{Highlighting}
{Shaded}
\end{verbatim}
\end{center}

\begin{itemize}
\tightlist
\item
  \textbf{Gray Inputs}: G3, G2, G1, G0 (most to least significant bits)
\item
  \textbf{Binary Outputs}: B3, B2, B1, B0 (most to least significant
  bits)
\item
  \textbf{Conversion Rule}: B3 = G3, B2 = B3 \oplus G2, B1 = B2 \oplus G1, B0 = B1
  \oplus G0
\end{itemize}

\end{solutionbox}
\begin{mnemonicbox}
``First bit same, rest XOR with previous result''

\end{mnemonicbox}
\subsection*{Question 4(b) OR [4
marks]}\label{q4b}

\textbf{Explain working of Parallel in Parallel out shift register.}

\begin{solutionbox}

\textbf{Diagram: Parallel-In Parallel-Out Shift Register}

\begin{center}
\textbf{Mermaid Diagram (Code)}
\begin{verbatim}
{Shaded}
{Highlighting}[]
graph TD
    D0(D0){-{-}{}FF0(FF0)}
    D1(D1){-{-}{}FF1(FF1)}
    D2(D2){-{-}{}FF2(FF2)}
    D3(D3){-{-}{}FF3(FF3)}
    CLK(Clock){-{-}{}FF0}
    CLK{-{-}{}FF1}
    CLK{-{-}{}FF2}
    CLK{-{-}{}FF3}
    LOAD(Load){-{-}{}FF0}
    LOAD{-{-}{}FF1}
    LOAD{-{-}{}FF2}
    LOAD{-{-}{}FF3}
    FF0{-{-}{}Q0(Q0)}
    FF1{-{-}{}Q1(Q1)}
    FF2{-{-}{}Q2(Q2)}
    FF3{-{-}{}Q3(Q3)}
{Highlighting}
{Shaded}
\end{verbatim}
\end{center}


{\def\LTcaptype{none} % do not increment counter
\vspace{-5pt}
\captionof{table}{Parallel-In Parallel-Out Operation}
\vspace{-10pt}
\begin{longtable}[]{@{}llll@{}}
\toprule\noalign{}
LOAD & Clock & D0-D3 & Q0-Q3 (after clock) \\
\midrule\noalign{}
\endhead
\bottomrule\noalign{}
\endlastfoot
1 & ↑ & 1010 & 1010 \\
0 & ↑ & xxxx & 1010 (no change) \\
1 & ↑ & 0101 & 0101 \\
\end{longtable}
}

\begin{itemize}
\tightlist
\item
  \textbf{Operation}: Data loaded in parallel, all bits simultaneously
  transferred to outputs
\item
  \textbf{Applications}: Data storage, buffering, temporary holding
  registers
\item
  \textbf{Features}: Fastest register type, requires most I/O pins, no
  bit shifting
\end{itemize}

\end{solutionbox}
\begin{mnemonicbox}
``All in, all out, all at once''

\end{mnemonicbox}
\subsection*{Question 4(c) OR [7
marks]}\label{q4c}

\textbf{Explain workings of T flip flop and SR flip flop using circuit
diagram and truth table.}

\begin{solutionbox}

\textbf{Diagram: T Flip-Flop}

\begin{center}
\textbf{Mermaid Diagram (Code)}
\begin{verbatim}
{Shaded}
{Highlighting}[]
graph LR
    T(T){-{-}{}TFF(T Flip{-}Flop)}
    CLK(Clock){-{-}{}TFF}
    TFF{-{-}{}Q(Q)}
    TFF{-{-}{}Q{}("Q{}")}
{Highlighting}
{Shaded}
\end{verbatim}
\end{center}


{\def\LTcaptype{none} % do not increment counter
\vspace{-5pt}
\captionof{table}{T Flip-Flop Truth Table}
\vspace{-10pt}
\begin{longtable}[]{@{}lll@{}}
\toprule\noalign{}
T & Clock & Q(next) \\
\midrule\noalign{}
\endhead
\bottomrule\noalign{}
\endlastfoot
0 & ↑ & Q (no change) \\
1 & ↑ & Q' (toggle) \\
\end{longtable}
}

\textbf{Diagram: SR Flip-Flop}

\begin{center}
\textbf{Mermaid Diagram (Code)}
\begin{verbatim}
{Shaded}
{Highlighting}[]
graph LR
    S(S){-{-}{}SRFF(SR Flip{-}Flop)}
    R(R){-{-}{}SRFF}
    CLK(Clock){-{-}{}SRFF}
    SRFF{-{-}{}Q(Q)}
    SRFF{-{-}{}Q{}("Q{}")}
{Highlighting}
{Shaded}
\end{verbatim}
\end{center}


{\def\LTcaptype{none} % do not increment counter
\vspace{-5pt}
\captionof{table}{SR Flip-Flop Truth Table}
\vspace{-10pt}
\begin{longtable}[]{@{}llll@{}}
\toprule\noalign{}
S & R & Clock & Q(next) \\
\midrule\noalign{}
\endhead
\bottomrule\noalign{}
\endlastfoot
0 & 0 & ↑ & Q (no change) \\
0 & 1 & ↑ & 0 (reset) \\
1 & 0 & ↑ & 1 (set) \\
1 & 1 & ↑ & Invalid \\
\end{longtable}
}

\begin{itemize}
\tightlist
\item
  \textbf{T Flip-Flop}: Toggle flip-flop changes state when T=1,
  maintains state when T=0
\item
  \textbf{SR Flip-Flop}: Basic flip-flop with Set (S) and Reset (R)
  inputs
\item
  \textbf{Applications}: T flip-flops for counters and frequency
  dividers, SR for basic memory
\end{itemize}

\end{solutionbox}
\begin{mnemonicbox}
``T Toggles when True, SR Sets or Resets''

\end{mnemonicbox}
\subsection*{Question 5(a) [3 marks]}\label{q5a}

\textbf{Compare TTL, CMOS and ECL logic families.}

\begin{solutionbox}


{\def\LTcaptype{none} % do not increment counter
\vspace{-5pt}
\captionof{table}{Comparison of Logic Families}
\vspace{-10pt}
\begin{longtable}[]{@{}llll@{}}
\toprule\noalign{}
Parameter & TTL & CMOS & ECL \\
\midrule\noalign{}
\endhead
\bottomrule\noalign{}
\endlastfoot
Power Consumption & Medium & Very Low & High \\
Speed & Medium & Low-Medium & Very High \\
Noise Immunity & Medium & High & Low \\
Fan-out & 10 & \textgreater50 & 25 \\
Supply Voltage & +5V & +3V to +15V & -5.2V \\
Complexity & Medium & Low & High \\
\end{longtable}
}

\begin{itemize}
\tightlist
\item
  \textbf{TTL}: Transistor-Transistor Logic - Good balance of speed and
  power
\item
  \textbf{CMOS}: Complementary Metal-Oxide-Semiconductor - Low power,
  high density
\item
  \textbf{ECL}: Emitter-Coupled Logic - Highest speed, used in
  high-performance applications
\end{itemize}

\end{solutionbox}
\begin{mnemonicbox}
``TCE: TTL Compromises, CMOS Economizes, ECL Excels
in speed''

\end{mnemonicbox}
\subsection*{Question 5(b) [4 marks]}\label{q5b}

\textbf{Explain decade counter with the help of logic circuit diagram
and truth table.}

\begin{solutionbox}

\textbf{Diagram: Decade Counter (BCD Counter)}

\begin{center}
\textbf{Mermaid Diagram (Code)}
\begin{verbatim}
{Shaded}
{Highlighting}[]
graph LR
    CLK(Clock){-{-}{}JK0(JK FF0)}
    JK0{-{-}{}Q0(Q0)}
    JK0{-{-}{}Q0{}("Q0{}")}
    Q0{{-}{-}{}JK1(JK FF1)}
    JK1{-{-}{}Q1(Q1)}
    JK1{-{-}{}Q1{}("Q1{}")}
    Q1{{-}{-}{}JK2(JK FF2)}
    JK2{-{-}{}Q2(Q2)}
    JK2{-{-}{}Q2{}("Q2{}")}
    Q2{{-}{-}{}JK3(JK FF3)}
    JK3{-{-}{}Q3(Q3)}
    JK3{-{-}{}Q3{}("Q3{}")}
    Q3{-{-}{}NAND1(NAND)}
    Q1{-{-}{}NAND1}
    NAND1{-{-}{}CLEAR(Clear)}
    CLEAR{-{-}{}JK0}
    CLEAR{-{-}{}JK1}
    CLEAR{-{-}{}JK2}
    CLEAR{-{-}{}JK3}
{Highlighting}
{Shaded}
\end{verbatim}
\end{center}


{\def\LTcaptype{none} % do not increment counter
\vspace{-5pt}
\captionof{table}{Decade Counter States}
\vspace{-10pt}
\begin{longtable}[]{@{}lllll@{}}
\toprule\noalign{}
Count & Q3 & Q2 & Q1 & Q0 \\
\midrule\noalign{}
\endhead
\bottomrule\noalign{}
\endlastfoot
0 & 0 & 0 & 0 & 0 \\
1 & 0 & 0 & 0 & 1 \\
2 & 0 & 0 & 1 & 0 \\
3 & 0 & 0 & 1 & 1 \\
4 & 0 & 1 & 0 & 0 \\
5 & 0 & 1 & 0 & 1 \\
6 & 0 & 1 & 1 & 0 \\
7 & 0 & 1 & 1 & 1 \\
8 & 1 & 0 & 0 & 0 \\
9 & 1 & 0 & 0 & 1 \\
0 & 0 & 0 & 0 & 0 \\
\end{longtable}
}

\begin{itemize}
\tightlist
\item
  \textbf{Function}: Counts from 0 to 9 (decimal) and then resets to 0
\item
  \textbf{Applications}: Digital clocks, frequency dividers, BCD
  counters
\item
  \textbf{Features}: Auto-reset at count 10, synchronous with clock
\end{itemize}

\end{solutionbox}
\begin{mnemonicbox}
``Counts one Decade, resets after nine''

\end{mnemonicbox}
\subsection*{Question 5(c) [7 marks]}\label{q5c}

\textbf{Give Classification of Memories in detail.}

\begin{solutionbox}

\textbf{Diagram: Memory Classification}

\begin{center}
\textbf{Mermaid Diagram (Code)}
\begin{verbatim}
{Shaded}
{Highlighting}[]
graph TD
    M[Memory]{-{-}{}PM[Primary Memory]}
    M{-{-}{}SM[Secondary Memory]}

    PM{-{-}{}RAM[RAM]}
    PM{-{-}{}ROM[ROM]}
    
    RAM{-{-}{}SRAM[Static RAM]}
    RAM{-{-}{}DRAM[Dynamic RAM]}
    
    ROM{-{-}{}MROM[Mask ROM]}
    ROM{-{-}{}PROM[Programmable ROM]}
    ROM{-{-}{}EPROM[Erasable PROM]}
    ROM{-{-}{}EEPROM[Electrically EPROM]}
    
    EEPROM{-{-}{}FLASH[Flash Memory]}
    
    SM{-{-}{}HD[Hard Disk]}
    SM{-{-}{}OD[Optical Disk]}
    SM{-{-}{}USB[USB Drives]}
    SM{-{-}{}SD[SD Cards]}
{Highlighting}
{Shaded}
\end{verbatim}
\end{center}


{\def\LTcaptype{none} % do not increment counter
\vspace{-5pt}
\captionof{table}{Memory Types Comparison}
\vspace{-10pt}
\begin{longtable}[]{@{}
  >{\raggedright\arraybackslash}p{(\linewidth - 8\tabcolsep) * \real{0.2031}}
  >{\raggedright\arraybackslash}p{(\linewidth - 8\tabcolsep) * \real{0.1875}}
  >{\raggedright\arraybackslash}p{(\linewidth - 8\tabcolsep) * \real{0.1875}}
  >{\raggedright\arraybackslash}p{(\linewidth - 8\tabcolsep) * \real{0.2188}}
  >{\raggedright\arraybackslash}p{(\linewidth - 8\tabcolsep) * \real{0.2031}}@{}}
\toprule\noalign{}
\begin{minipage}[b]{\linewidth}\raggedright
Memory Type
\end{minipage} & \begin{minipage}[b]{\linewidth}\raggedright
Volatility
\end{minipage} & \begin{minipage}[b]{\linewidth}\raggedright
Read/Write
\end{minipage} & \begin{minipage}[b]{\linewidth}\raggedright
Access Speed
\end{minipage} & \begin{minipage}[b]{\linewidth}\raggedright
Typical Use
\end{minipage} \\
\midrule\noalign{}
\endhead
\bottomrule\noalign{}
\endlastfoot
SRAM & Volatile & R/W & Very Fast & Cache memory \\
DRAM & Volatile & R/W & Fast & Main memory \\
ROM & Non-volatile & Read-only & Medium & BIOS, firmware \\
PROM & Non-volatile & Write-once & Medium & Permanent programs \\
EPROM & Non-volatile & Erasable UV & Medium & Upgradable firmware \\
EEPROM & Non-volatile & Electrically erasable & Medium & Configuration
data \\
Flash & Non-volatile & Block erasable & Medium-Fast & Storage devices \\
\end{longtable}
}

\begin{itemize}
\tightlist
\item
  \textbf{RAM (Random Access Memory)}: Temporary, volatile working
  memory
\item
  \textbf{ROM (Read Only Memory)}: Permanent, non-volatile program
  storage
\item
  \textbf{Characteristics}: Access time, data retention, capacity, cost
  per bit
\end{itemize}

\end{solutionbox}
\begin{mnemonicbox}
``RAM Vanishes, ROM Remains''

\end{mnemonicbox}
\subsection*{Question 5(a) OR [3
marks]}\label{q5a}

\textbf{Define: Fan out, Fan in and Figure of merit.}

\begin{solutionbox}


{\def\LTcaptype{none} % do not increment counter
\vspace{-5pt}
\captionof{table}{Digital Logic Parameters}
\vspace{-10pt}
\begin{longtable}[]{@{}
  >{\raggedright\arraybackslash}p{(\linewidth - 4\tabcolsep) * \real{0.2821}}
  >{\raggedright\arraybackslash}p{(\linewidth - 4\tabcolsep) * \real{0.3077}}
  >{\raggedright\arraybackslash}p{(\linewidth - 4\tabcolsep) * \real{0.4103}}@{}}
\toprule\noalign{}
\begin{minipage}[b]{\linewidth}\raggedright
Parameter
\end{minipage} & \begin{minipage}[b]{\linewidth}\raggedright
Definition
\end{minipage} & \begin{minipage}[b]{\linewidth}\raggedright
Typical Values
\end{minipage} \\
\midrule\noalign{}
\endhead
\bottomrule\noalign{}
\endlastfoot
Fan-out & Number of standard loads a gate output can drive & TTL: 10,
CMOS: \textgreater50 \\
Fan-in & Number of inputs a logic gate can handle & TTL: 8, CMOS:
100+ \\
Figure of Merit & Speed-power product (propagation delay \times power
consumption) & Lower is better \\
\end{longtable}
}

\begin{itemize}
\tightlist
\item
  \textbf{Fan-out}: Maximum number of gate inputs that can be connected
  to a gate output
\item
  \textbf{Fan-in}: Maximum number of inputs available on a single logic
  gate
\item
  \textbf{Figure of Merit}: Quality factor for comparing different logic
  families
\end{itemize}

\end{solutionbox}
\begin{mnemonicbox}
``Out drives many, In accepts many, Merit measures
goodness''

\end{mnemonicbox}
\subsection*{Question 5(b) OR [4
marks]}\label{q5b}

\textbf{Explain asynchronous up counter with the help of logic circuit
diagram and truth table.}

\begin{solutionbox}

\textbf{Diagram: 4-bit Asynchronous Up Counter}

\begin{center}
\textbf{Mermaid Diagram (Code)}
\begin{verbatim}
{Shaded}
{Highlighting}[]
graph LR
    CLK(Clock){-{-}{}TFF0(T FF0)}
    TFF0{-{-}{}Q0(Q0)}
    TFF0{-{-}{}Q0{}("Q0{}")}
    Q0{-{-}{}TFF1(T FF1)}
    TFF1{-{-}{}Q1(Q1)}
    TFF1{-{-}{}Q1{}("Q1{}")}
    Q1{-{-}{}TFF2(T FF2)}
    TFF2{-{-}{}Q2(Q2)}
    TFF2{-{-}{}Q2{}("Q2{}")}
    Q2{-{-}{}TFF3(T FF3)}
    TFF3{-{-}{}Q3(Q3)}
    TFF3{-{-}{}Q3{}("Q3{}")}

    CLR(Clear){-{-}{}TFF0}
    CLR{-{-}{}TFF1}
    CLR{-{-}{}TFF2}
    CLR{-{-}{}TFF3}
{Highlighting}
{Shaded}
\end{verbatim}
\end{center}


{\def\LTcaptype{none} % do not increment counter
\vspace{-5pt}
\captionof{table}{4-bit Asynchronous Counter States}
\vspace{-10pt}
\begin{longtable}[]{@{}lllll@{}}
\toprule\noalign{}
Count & Q3 & Q2 & Q1 & Q0 \\
\midrule\noalign{}
\endhead
\bottomrule\noalign{}
\endlastfoot
0 & 0 & 0 & 0 & 0 \\
1 & 0 & 0 & 0 & 1 \\
2 & 0 & 0 & 1 & 0 \\
\ldots{} & .. & .. & .. & .. \\
14 & 1 & 1 & 1 & 0 \\
15 & 1 & 1 & 1 & 1 \\
\end{longtable}
}

\begin{itemize}
\tightlist
\item
  \textbf{Operation}: Each flip-flop triggers the next when
  transitioning from 1 to 0
\item
  \textbf{Features}: Simple design but suffers from propagation delay
  (ripple)
\item
  \textbf{Applications}: Frequency division, basic counting applications
\end{itemize}

\end{solutionbox}
\begin{mnemonicbox}
``Ripples Up, Each bit triggers Next''

\end{mnemonicbox}
\subsection*{Question 5(c) OR [7
marks]}\label{q5c}

\textbf{Describe steps and the need of E-waste Management of Digital
ICs.}

\begin{solutionbox}

\textbf{Diagram: E-waste Management Cycle}

\begin{center}
\textbf{Mermaid Diagram (Code)}
\begin{verbatim}
{Shaded}
{Highlighting}[]
graph LR
    C[Collection]{-{-}{}S[Sorting]}
    S{-{-}{}D[Disassembly]}
    D{-{-}{}R[Recycling]}
    R{-{-}{}M[Material Recovery]}
    M{-{-}{}N[New Products]}
    N{-{-}{}C}
{Highlighting}
{Shaded}
\end{verbatim}
\end{center}


{\def\LTcaptype{none} % do not increment counter
\vspace{-5pt}
\captionof{table}{E-waste Management Steps}
\vspace{-10pt}
\begin{longtable}[]{@{}lll@{}}
\toprule\noalign{}
Step & Description & Importance \\
\midrule\noalign{}
\endhead
\bottomrule\noalign{}
\endlastfoot
Collection & Gathering obsolete ICs & Prevents improper disposal \\
Sorting & Categorizing by type & Enables efficient processing \\
Disassembly & Separating components & Facilitates material recovery \\
Recycling & Processing materials & Reduces environmental impact \\
Material Recovery & Extracting valuable metals & Conserves resources \\
Safe Disposal & Handling toxic components & Prevents contamination \\
\end{longtable}
}

\begin{itemize}
\tightlist
\item
  \textbf{Need for E-waste Management}:

  \begin{itemize}
  \tightlist
  \item
    \textbf{Environmental Protection}: Prevents toxic substances from
    leaching into soil/water
  \item
    \textbf{Resource Conservation}: Recovers valuable metals like gold,
    silver, copper
  \item
    \textbf{Health Safety}: Reduces exposure to hazardous materials like
    lead, mercury
  \item
    \textbf{Legal Compliance}: Follows regulations regarding electronic
    waste
  \end{itemize}
\end{itemize}

\end{solutionbox}
\begin{mnemonicbox}
``Collect, Sort, Disassemble, Recycle, Recover,
Reuse''

\end{mnemonicbox}

\end{document}
