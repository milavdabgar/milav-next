\documentclass[10pt,a4paper]{article}

% content/resources/templates/preamble.tex
\usepackage[margin=0.6in]{geometry}
\author{Milav Dabgar}
\usepackage{amsmath,amssymb,amsthm}
\usepackage{booktabs}
\usepackage{multirow}
\usepackage{xcolor}
\usepackage{tcolorbox}
\tcbuselibrary{breakable,skins}
\usepackage[colorlinks=true,linkcolor=blue]{hyperref}
\usepackage{titlesec}
\usepackage{enumitem}
\usepackage{tikz}
\usepackage{pgfplots}
\usepackage{circuitikz}
\usepackage[version=4]{mhchem}
\usepackage{longtable}
\usepackage{array}
\usepackage{float}
\usepackage{caption}
\usepackage{listings}

\lstset{
  basicstyle=\small\ttfamily,
  breaklines=true,
  breakatwhitespace=false,
  postbreak=\mbox{\textcolor{red}{$\hookrightarrow$}\space},
  float=false,
  numbers=left,
  numberstyle=\tiny\color{gray},
  numbersep=10pt,
  xleftmargin=2em,
  keywordstyle=\color{blue},
  commentstyle=\color{green!60!black},
  stringstyle=\color{purple},
  backgroundcolor=\color{gray!5},
  showstringspaces=false,
  tabsize=2,
  captionpos=b,
  keepspaces=true,
  columns=flexible
}

\pgfplotsset{compat=1.18}
\usetikzlibrary{shapes,arrows,positioning,calc,patterns,decorations.pathmorphing,decorations.markings,arrows.meta}

% Color scheme
\definecolor{headcolor}{RGB}{0,102,204}
\definecolor{keycolor}{RGB}{220,20,60}
\definecolor{solutioncolor}{RGB}{34,139,34}
\definecolor{mnemoniccolor}{RGB}{148,0,211}
\definecolor{codecolor}{RGB}{0,0,100}

% Spacing
\setlength{\parskip}{3pt}
\setlist[itemize]{nosep}
\setlist[enumerate]{nosep}

% Title formatting
\titleformat{\section}{\Large\bfseries\color{headcolor}}{\thesection}{1em}{}
\titleformat{\subsection}{\large\bfseries\color{headcolor}}{\thesubsection}{1em}{}

% Pandoc tightlist compatibility
\providecommand{\tightlist}{%
  \setlength{\itemsep}{0pt}\setlength{\parskip}{0pt}}

% Pandoc longtable compatibility
\newcounter{none}
\def\thenone{}


% content/resources/templates/gujarati-boxes.tex
\usepackage{fontspec}
\usepackage{polyglossia}

% Set Gujarati as main language (document is primarily in Gujarati)
% Note: gloss-gujarati.ldf doesn't exist in polyglossia, but it will use hyphenation patterns
\setdefaultlanguage{gujarati}
\setotherlanguage{english}

% Configure Gujarati font properly
% Use Language=Default to prevent polyglossia from trying to add language-specific features
% that don't exist for Gujarati, which causes "empty feature" warnings
\newfontfamily\gujaratifont[Script=Gujarati,AutoFakeBold=2.5,AutoFakeSlant=0.3]{Noto Sans Gujarati}
\setmainfont[Script=Gujarati,AutoFakeBold=2.5,AutoFakeSlant=0.3]{Noto Sans Gujarati}
% Use Noto Sans Gujarati for monospace to support Gujarati in text
\setmonofont[Scale=0.9]{Noto Sans Gujarati}

% Configure English to use the same font
\newfontfamily\englishfont[Script=Gujarati,AutoFakeBold=2.5,AutoFakeSlant=0.3]{Noto Sans Gujarati}

% Translations for polyglossia
\gappto\captionsgujarati{
  \renewcommand{\tablename}{કોષ્ટક}
  \renewcommand{\figurename}{આકૃતિ}
}

% Helper for TikZ nodes to ensure Gujarati font
\newcommand{\gu}[1]{{\gujaratifont #1}}

% Custom environments
\newtcolorbox{solutionbox}{
    breakable,
    enhanced,
    colback=solutioncolor!5!white,
    colframe=solutioncolor!75!black,
    fonttitle=\bfseries,
    title=જવાબ
}

\newtcolorbox{solutionboxnobreak}{
 colback=solutioncolor!5!white,
 colframe=solutioncolor!75!black,
 fonttitle=\bfseries,
 title=જવાબ
}

\newtcolorbox{keyformula}{
 breakable,
 enhanced,
 colback=keycolor!5!white,
 colframe=keycolor!75!black,
 fonttitle=\bfseries,
 title=રાસાયણિક સમીકરણ/સૂત્ર
}

\newtcolorbox{mnemonicbox}{
 breakable,
 enhanced,
 colback=mnemoniccolor!5!white,
 colframe=mnemoniccolor!75!black,
 fonttitle=\bfseries,
 title=મેમરી ટ્રીક
}


\begin{document}

\begin{center}
{\Huge\bfseries\color{headcolor} Subject Name (Gujarati)}\\[5pt]
{\LARGE 4321102 -- Summer 2024}\\[3pt]
{\large Semester 1 Study Material}\\[3pt]
{\normalsize\textit{Detailed Solutions and Explanations}}
\end{center}

\vspace{10pt}

\subsection*{પ્રશ્ન 1(અ) [3
માર્ક્સ]}\label{uxaaauxab0uxab6uxaa8-1uxa85-3-uxaaeuxab0uxa95uxab8}

\textbf{કન્વર્ટ કરો: (110101)_{2} = ( \_\_\_ )_{1}_{0} = ( \_\_\_ )_{8} = ( \_\_\_
)_{1}_{6}}

\begin{solutionbox}

\textbf{સ્ટેપ-બાય-સ્ટેપ કન્વર્ઝન (110101)_{2}}:

{\def\LTcaptype{none} % do not increment counter
\begin{longtable}[]{@{}
  >{\raggedright\arraybackslash}p{(\linewidth - 6\tabcolsep) * \real{0.3830}}
  >{\raggedright\arraybackslash}p{(\linewidth - 6\tabcolsep) * \real{0.1915}}
  >{\raggedright\arraybackslash}p{(\linewidth - 6\tabcolsep) * \real{0.1489}}
  >{\raggedright\arraybackslash}p{(\linewidth - 6\tabcolsep) * \real{0.2766}}@{}}
\toprule\noalign{}
\begin{minipage}[b]{\linewidth}\raggedright
બાઇનરી (110101)_{2}
\end{minipage} & \begin{minipage}[b]{\linewidth}\raggedright
ડેસિમલ
\end{minipage} & \begin{minipage}[b]{\linewidth}\raggedright
ઑક્ટલ
\end{minipage} & \begin{minipage}[b]{\linewidth}\raggedright
હેક્ઝાડેસિમલ
\end{minipage} \\
\midrule\noalign{}
\endhead
\bottomrule\noalign{}
\endlastfoot
1\times2^{5} + 1\times2^{4} + 0\times2^{3} + 1\times2^{2} + 0\times2^{1} + 1\times2^{0} & 32+16+0+4+0+1 = 53 & 6\times8^{1} +
5\times8^{0} = 48+5 = 53 & 3\times16^{1} + 5\times16^{0} = 48+5 = 35 \\
(110101)_{2} & (53)_{1}_{0} & (65)_{8} & (35)_{1}_{6} \\
\end{longtable}
}

\end{solutionbox}
\begin{mnemonicbox}
``બાઇનરી ડિજિટ આઉટ હિયર'' (BDOH)
બાઇનરી\rightarrowડેસિમલ\rightarrowઑક્ટલ\rightarrowહેક્ઝાડેસિમલ કન્વર્ઝન માટે.

\end{mnemonicbox}
\subsection*{પ્રશ્ન 1(બ) [4
માર્ક્સ]}\label{uxaaauxab0uxab6uxaa8-1uxaac-4-uxaaeuxab0uxa95uxab8}

**કરો: (i) (11101101)_{2}+(10101000)_{2} (ii) (11011)_{2}*(1010)_{2}**

\begin{solutionbox}

\textbf{બાઇનરી સરવાળા અને ગુણાકાર માટે ટેબલ}:

{\def\LTcaptype{none} % do not increment counter
\begin{longtable}[]{@{}ll@{}}
\toprule\noalign{}
(i) બાઇનરી સરવાળો & (ii) બાઇનરી ગુણાકાર \\
\midrule\noalign{}
\endhead
\bottomrule\noalign{}
\endlastfoot
11101101 & 11011 \\
+ 10101000 & \times 1010 \\
---------- & ------- \\
110010101 & 00000 \\
& 11011 \\
& 00000 \\
& 11011 \\
& -------- \\
& 11101110 \\
\end{longtable}
}

\textbf{ડેસિમલ વેરિફિકેશન}:

\begin{itemize}
\tightlist
\item
  \begin{enumerate}
  \tightlist
  \item
    (11101101)_{2} = 237, (10101000)_{2} = 168, સરવાળો = 405 = (110010101)_{2}
  \end{enumerate}
\item
  \begin{enumerate}
  \tightlist
  \item
    (11011)_{2} = 27, (1010)_{2} = 10, ગુણાકાર = 270 = (11101110)_{2}
  \end{enumerate}
\end{itemize}

\end{solutionbox}
\begin{mnemonicbox}
સરવાળા માટે ``કેરી અપ મેક્સ સમ'' અને ગુણાકાર માટે ``શિફ્ટ
લેફ્ટ એડ પ્રોડક્ટ''.

\end{mnemonicbox}
\subsection*{પ્રશ્ન 1(ક) [7
માર્ક્સ]}\label{uxaaauxab0uxab6uxaa8-1uxa95-7-uxaaeuxab0uxa95uxab8}

\textbf{(i) કન્વર્ટ કરો: (48)_{1}_{0} = ( \_\_\_ )_{2} = ( \_\_\_ )_{8} = ( \_\_\_
)_{1}_{6}} \textbf{(ii) 2's Complement પદ્ધતિનો ઉપયોગ કરીને બાદબાકી કરો: (1110)_{2}
-- (1000)_{2}} \textbf{(iii) (1111101)_{2} ને (101)_{2} વડે વિભાજિત કરો.}

\begin{solutionbox}

\textbf{(i) કન્વર્ઝન ટેબલ}:

{\def\LTcaptype{none} % do not increment counter
\begin{longtable}[]{@{}llll@{}}
\toprule\noalign{}
ડેસિમલ (48)_{1}_{0} & બાઇનરી & ઑક્ટલ & હેક્ઝાડેસિમલ \\
\midrule\noalign{}
\endhead
\bottomrule\noalign{}
\endlastfoot
48\div2 = 24 રેમ 0 & 110000 & 60 & 30 \\
24\div2 = 12 રેમ 0 & & & \\
12\div2 = 6 રેમ 0 & & & \\
6\div2 = 3 રેમ 0 & & & \\
3\div2 = 1 રેમ 1 & & & \\
1\div2 = 0 રેમ 1 & & & \\
(48)_{1}_{0} & (110000)_{2} & (60)_{8} & (30)_{1}_{6} \\
\end{longtable}
}

\textbf{(ii) બાદબાકી ટેબલ}:

{\def\LTcaptype{none} % do not increment counter
\begin{longtable}[]{@{}ll@{}}
\toprule\noalign{}
2's Complement પદ્ધતિ & સ્ટેપ્સ \\
\midrule\noalign{}
\endhead
\bottomrule\noalign{}
\endlastfoot
(1110)_{2} -- (1000)_{2} & 1. (1000)_{2} નો 2's complement શોધો \\
(1000)_{2} નો 1's complement & (0111)_{2} \\
2's complement & (0111)_{2} + 1 = (1000)_{2} \\
(1110)_{2} + (1000)_{2} & (10110)_{2} \\
કેરી દૂર કરો & (0110)_{2} \\
પરિણામ & (0110)_{2} = 6_{1}_{0} \\
\end{longtable}
}

\textbf{(iii) ભાગાકાર}:

\begin{verbatim}
flowchart LR
    A["(1111101)_{2  (101)_{2}"] {-}{-} B["101)1111101(11 ભાગફળ}
                                    101
                                    {-{-}{-}{-}{-}}
                                    100
                                    000
                                    {-{-}{-}{-}{-}}
                                    1001
                                    101
                                    {-{-}{-}{-}{-}}
                                    001 શેષ"]
    B {-{-} C["ભાગફળ = (11)_{2}}
            શેષ = (1)_{2"]}
\end{verbatim}

\end{solutionbox}
\begin{mnemonicbox}
લાંબા ભાગાકાર પ્રક્રિયા માટે ``ડિવિઝન ડ્રોપ્સ ડાઉન
રિમેન્ડર્સ''.

\end{mnemonicbox}
\subsection*{પ્રશ્ન 1(ક) અથવા [7
માર્ક્સ]}\label{uxaaauxab0uxab6uxaa8-1uxa95-uxa85uxaa5uxab5-7-uxaaeuxab0uxa95uxab8}

\textbf{કોડ્સ સમજાવો: ASCII, BCD, Gray}

\begin{solutionbox}

\textbf{સામાન્ય ડિજિટલ કોડ્સનું ટેબલ}:

{\def\LTcaptype{none} % do not increment counter
\begin{longtable}[]{@{}
  >{\raggedright\arraybackslash}p{(\linewidth - 4\tabcolsep) * \real{0.2143}}
  >{\raggedright\arraybackslash}p{(\linewidth - 4\tabcolsep) * \real{0.4643}}
  >{\raggedright\arraybackslash}p{(\linewidth - 4\tabcolsep) * \real{0.3214}}@{}}
\toprule\noalign{}
\begin{minipage}[b]{\linewidth}\raggedright
કોડ
\end{minipage} & \begin{minipage}[b]{\linewidth}\raggedright
વર્ણન
\end{minipage} & \begin{minipage}[b]{\linewidth}\raggedright
ઉદાહરણ
\end{minipage} \\
\midrule\noalign{}
\endhead
\bottomrule\noalign{}
\endlastfoot
\textbf{ASCII (American Standard Code for Information Interchange)} &
128 કેરેક્ટર્સને રજૂ કરતો 7-બિટ કોડ જેમાં આલ્ફાબેટ્સ, નંબર્સ અને સ્પેશિયલ સિમ્બોલ્સ શામેલ છે
& A = 65 (1000001)_{2} \\
\textbf{BCD (Binary Coded Decimal)} & દરેક ડેસિમલ અંક (0-9) ને 4 બિટ્સનો
ઉપયોગ કરીને રજૂ કરે છે & 42 = 0100 0010 \\
\textbf{Gray Code} & બાઇનરી કોડ જેમાં આસપાસના નંબરો માત્ર એક બિટથી અલગ પડે છે
& (0,1,3,2) = (00,01,11,10) \\
\end{longtable}
}

\textbf{ડાયાગ્રામ: ગ્રે કોડ જનરેશન}:

\begin{verbatim}
flowchart LR
    A["બાઇનરી કોડ"] {-{-} B["ગ્રે કોડ"]}
    B {-{-} C["MSB રહે છે એક સમાન}
    દરેક બિટ અગાઉના સાથે XOR થાય છે"]
    D["બાઇનરી: 0011"] {-{-} E["ગ્રે: 0010"]}
\end{verbatim}

\end{solutionbox}
\begin{mnemonicbox}
``ઓલવેઝ બાઇનરી જનરેટ્સ'' - દરેક કોડનો પ્રથમ અક્ષર (ASCII,
BCD, Gray).

\end{mnemonicbox}
\subsection*{પ્રશ્ન 2(અ) [3
માર્ક્સ]}\label{uxaaauxab0uxab6uxaa8-2uxa85-3-uxaaeuxab0uxa95uxab8}

\textbf{બુલિયન બીજગણિતનો ઉપયોગ કરીને સરળ બનાવો: Y = A B + A' B + A' B' + A
B'}

\begin{solutionbox}

\textbf{સ્ટેપ-બાય-સ્ટેપ સરળીકરણ}:

{\def\LTcaptype{none} % do not increment counter
\begin{longtable}[]{@{}lll@{}}
\toprule\noalign{}
સ્ટેપ & એક્સપ્રેશન & બુલિયન નિયમ \\
\midrule\noalign{}
\endhead
\bottomrule\noalign{}
\endlastfoot
Y = A B + A' B + A' B' + A B' & પ્રારંભિક એક્સપ્રેશન & - \\
Y = A(B + B') + A'(B + B') & ફેક્ટરિંગ & ડિસ્ટ્રિબ્યુટિવ લૉ \\
Y = A(1) + A'(1) & કોમ્પ્લિમેન્ટ લૉ & B + B' = 1 \\
Y = A + A' & સરળીકરણ & - \\
Y = 1 & કોમ્પ્લિમેન્ટ લૉ & A + A' = 1 \\
\end{longtable}
}

\end{solutionbox}
\begin{mnemonicbox}
બુલિયન સરળીકરણ સ્ટેપ્સ માટે ``ફેક્ટર, સિમ્પ્લિફાય, ફિનિશ''.

\end{mnemonicbox}
\subsection*{પ્રશ્ન 2(બ) [4
માર્ક્સ]}\label{uxaaauxab0uxab6uxaa8-2uxaac-4-uxaaeuxab0uxa95uxab8}

\textbf{K-મેપનો ઉપયોગ કરીને નીચેના બુલિયન ફંક્શન ને સરળ બનાવો: f(A,B,C,D) = Σm
(0,3,4,6,8,11,12)}

\begin{solutionbox}

\textbf{K-મેપ સોલ્યુશન}:

\begin{verbatim}
    AB
CD  00 01 11 10
00  1  0  0  1
01  0  0  0  1  
11  0  1  0  0
10  0  0  1  0
\end{verbatim}

\textbf{ગ્રુપિંગ}:

\begin{itemize}
\tightlist
\item
  ગ્રુપ 1: m(0,8) = A'C'D'
\item
  ગ્રુપ 2: m(4,12) = BD'
\item
  ગ્રુપ 3: m(3,11) = CD
\item
  ગ્રુપ 4: m(6) = A'B'CD'
\end{itemize}

\textbf{સરળ કરેલ એક્સપ્રેશન}: f(A,B,C,D) = A'C'D' + BD' + CD + A'B'CD'

\end{solutionbox}
\begin{mnemonicbox}
K-મેપ ગ્રુપિંગ સ્ટ્રેટેજી માટે ``ગ્રુપ પાવર્સ ઓફ ટુ''.

\end{mnemonicbox}
\subsection*{પ્રશ્ન 2(ક) [7
માર્ક્સ]}\label{uxaaauxab0uxab6uxaa8-2uxa95-7-uxaaeuxab0uxa95uxab8}

\textbf{NOR ગેટને સ્વચ્છ આકૃતિઓ સાથે યુનિવર્સલ ગેટ તરીકે સમજાવો.}

\begin{solutionbox}

\textbf{NOR એઝ યુનિવર્સલ ગેટ}:

{\def\LTcaptype{none} % do not increment counter
\begin{longtable}[]{@{}lll@{}}
\toprule\noalign{}
ફંક્શન & NOR નો ઉપયોગ કરી ઇમ્પ્લિમેન્ટેશન & ટ્રુથ ટેબલ \\
\midrule\noalign{}
\endhead
\bottomrule\noalign{}
\endlastfoot
\textbf{NOT ગેટ} &
\pandocbounded{\includegraphics[keepaspectratio,alt={NOT using NOR}]{diagram1}}
& A \\
& & 0 \\
& & 1 \\
\textbf{AND ગેટ} &
\pandocbounded{\includegraphics[keepaspectratio,alt={AND using NOR}]{diagram2}}
& A B \\
& & 0 0 \\
& & 0 1 \\
& & 1 0 \\
& & 1 1 \\
\textbf{OR ગેટ} &
\pandocbounded{\includegraphics[keepaspectratio,alt={OR using NOR}]{diagram3}}
& A B \\
& & 0 0 \\
& & 0 1 \\
& & 1 0 \\
& & 1 1 \\
\end{longtable}
}

\textbf{ડાયાગ્રામ: NOR ઇમ્પ્લિમેન્ટેશન}:

\begin{verbatim}
flowchart TD
    A["NOT: A {-1{-} A"]}
    B["AND: A {-{-}1{-}{-}|}
               |    1{{-}{-} A•B}
         B {-{-}1{-}{-}|"]}
    C["OR: A {-{-}1{-}{-}|}
              |    |{{-}{-} A+B}
        B {-{-}1{-}{-}|"]}
\end{verbatim}

\end{solutionbox}
\begin{mnemonicbox}
NOR ગેટ ઇમ્પ્લિમેન્ટેશન માટે ``NOT AND OR, NOR કરે મોર''.

\end{mnemonicbox}
\subsection*{પ્રશ્ન 2(અ) અથવા [3
માર્ક્સ]}\label{uxaaauxab0uxab6uxaa8-2uxa85-uxa85uxaa5uxab5-3-uxaaeuxab0uxa95uxab8}

\textbf{બુલિયન સમીકરણ માટે લોજિક સર્કિટ દોરો: Y = (A + B') . (A' + B') . (B
+ C)}

\begin{solutionbox}

\textbf{લોજિક સર્કિટ ઇમ્પ્લિમેન્ટેશન}:

\begin{verbatim}
flowchart TD
    A["A"] {-{-} D["OR"]}
    B["B{"] {-}{-} D}
    D {-{-} G["AND"]}
    A1["A{"] {-}{-} E["OR"]}
    B1["B{"] {-}{-} E}
    E {-{-} G}
    B2["B"] {-{-} F["OR"]}
    C["C"] {-{-} F}
    F {-{-} G}
    G {-{-} Y["Y"]}
\end{verbatim}

\textbf{ટ્રુથ ટેબલ વેરિફિકેશન}:

\begin{itemize}
\tightlist
\item
  ટર્મ 1: (A + B')
\item
  ટર્મ 2: (A' + B')
\item
  ટર્મ 3: (B + C)
\item
  આઉટપુટ: Y = Term1 • Term2 • Term3
\end{itemize}

\end{solutionbox}
\begin{mnemonicbox}
જટિલ એક્સપ્રેશન માટે ``દરેક ટર્મ અલગથી''.

\end{mnemonicbox}
\subsection*{પ્રશ્ન 2(બ) અથવા [4
માર્ક્સ]}\label{uxaaauxab0uxab6uxaa8-2uxaac-uxa85uxaa5uxab5-4-uxaaeuxab0uxa95uxab8}

\textbf{ડી-મોર્ગન્સના પ્રમેય લખો અને તેને સાબિત કરો.}

\begin{solutionbox}

\textbf{ડી-મોર્ગન્સ પ્રમેય અને પ્રૂફ}:

{\def\LTcaptype{none} % do not increment counter
\begin{longtable}[]{@{}lll@{}}
\toprule\noalign{}
પ્રમેય & સ્ટેટમેન્ટ & ટ્રુથ ટેબલ દ્વારા પ્રૂફ \\
\midrule\noalign{}
\endhead
\bottomrule\noalign{}
\endlastfoot
\textbf{પ્રમેય 1} & (A•B)' = A' + B' & A B \\
& & 0 0 \\
& & 0 1 \\
& & 1 0 \\
& & 1 1 \\
\textbf{પ્રમેય 2} & (A+B)' = A'•B' & A B \\
& & 0 0 \\
& & 0 1 \\
& & 1 0 \\
& & 1 1 \\
\end{longtable}
}

\textbf{ડાયાગ્રામ: ડી-મોર્ગન્સ લૉ વિઝ્યુલાઇઝેશન}:

\begin{verbatim}
flowchart TB
    A["(A•B){ = A+B"] {-}{-} B["ઓપરેશન ઇન્વર્ટ કરો}
                                AND  OR
                                વેરિયેબલ્સ ઇન્વર્ટ કરો"]
    C["(A+B){ = A•B"] {-}{-} D["ઓપરેશન ઇન્વર્ટ કરો}
                                OR  AND
                                વેરિયેબલ્સ ઇન્વર્ટ કરો"]
\end{verbatim}

\end{solutionbox}
\begin{mnemonicbox}
ડી-મોર્ગન્સ લૉ લાગુ કરવા માટે ``બાર તોડો, ઓપરેશન બદલો,
ઇનપુટ ઇન્વર્ટ કરો''.

\end{mnemonicbox}
\subsection*{પ્રશ્ન 2(ક) અથવા [7
માર્ક્સ]}\label{uxaaauxab0uxab6uxaa8-2uxa95-uxa85uxaa5uxab5-7-uxaaeuxab0uxa95uxab8}

\textbf{સિમ્બોલ, ટ્રુથ ટેબલ અને સમીકરણની મદદથી તમામ લોજિક ગેટ્સ સમજાવો.}

\begin{solutionbox}

\textbf{લોજિક ગેટ્સ સમરી}:

{\def\LTcaptype{none} % do not increment counter
\begin{longtable}[]{@{}lllll@{}}
\toprule\noalign{}
ગેટ & સિમ્બોલ & ટ્રુથ ટેબલ & સમીકરણ & વર્ણન \\
\midrule\noalign{}
\endhead
\bottomrule\noalign{}
\endlastfoot
\textbf{AND} &
\pandocbounded{\includegraphics[keepaspectratio,alt={AND}]{diagram}} & A
B & Y & Y = A•B \\
& & 0 0 & 0 & \\
& & 0 1 & 0 & \\
& & 1 0 & 0 & \\
& & 1 1 & 1 & \\
\textbf{OR} &
\pandocbounded{\includegraphics[keepaspectratio,alt={OR}]{diagram}} & A
B & Y & Y = A+B \\
& & 0 0 & 0 & \\
& & 0 1 & 1 & \\
& & 1 0 & 1 & \\
& & 1 1 & 1 & \\
\textbf{NOT} &
\pandocbounded{\includegraphics[keepaspectratio,alt={NOT}]{diagram}} & A
& Y & Y = A' \\
& & 0 & 1 & \\
& & 1 & 0 & \\
\textbf{NAND} &
\pandocbounded{\includegraphics[keepaspectratio,alt={NAND}]{diagram}} &
A B & Y & Y = (A•B)' \\
& & 0 0 & 1 & \\
& & 0 1 & 1 & \\
& & 1 0 & 1 & \\
& & 1 1 & 0 & \\
\textbf{NOR} &
\pandocbounded{\includegraphics[keepaspectratio,alt={NOR}]{diagram}} & A
B & Y & Y = (A+B)' \\
& & 0 0 & 1 & \\
& & 0 1 & 0 & \\
& & 1 0 & 0 & \\
& & 1 1 & 0 & \\
\textbf{XOR} &
\pandocbounded{\includegraphics[keepaspectratio,alt={XOR}]{diagram}} & A
B & Y & Y = A\oplusB \\
& & 0 0 & 0 & \\
& & 0 1 & 1 & \\
& & 1 0 & 1 & \\
& & 1 1 & 0 & \\
\textbf{XNOR} &
\pandocbounded{\includegraphics[keepaspectratio,alt={XNOR}]{diagram}} &
A B & Y & Y = (A\oplusB)' \\
& & 0 0 & 1 & \\
& & 0 1 & 0 & \\
& & 1 0 & 0 & \\
& & 1 1 & 1 & \\
\end{longtable}
}

\end{solutionbox}
\begin{mnemonicbox}
``All Operations Need Necessary eXecution'' (દરેક
ગેટનો પહેલો અક્ષર - AND, OR, NOT, NAND, NOR, XOR).

\end{mnemonicbox}
\subsection*{પ્રશ્ન 3(અ) [3
માર્ક્સ]}\label{uxaaauxab0uxab6uxaa8-3uxa85-3-uxaaeuxab0uxa95uxab8}

\textbf{સંક્ષિપ્તમાં 4:2 એન્કોડર સમજાવો.}

\begin{solutionbox}

\textbf{4-to-2 એન્કોડર ઓવરવ્યુ}:

{\def\LTcaptype{none} % do not increment counter
\begin{longtable}[]{@{}
  >{\raggedright\arraybackslash}p{(\linewidth - 4\tabcolsep) * \real{0.2778}}
  >{\raggedright\arraybackslash}p{(\linewidth - 4\tabcolsep) * \real{0.3611}}
  >{\raggedright\arraybackslash}p{(\linewidth - 4\tabcolsep) * \real{0.3611}}@{}}
\toprule\noalign{}
\begin{minipage}[b]{\linewidth}\raggedright
ફંક્શન
\end{minipage} & \begin{minipage}[b]{\linewidth}\raggedright
વર્ણન
\end{minipage} & \begin{minipage}[b]{\linewidth}\raggedright
ટ્રુથ ટેબલ
\end{minipage} \\
\midrule\noalign{}
\endhead
\bottomrule\noalign{}
\endlastfoot
\textbf{4:2 એન્કોડર} & 4 ઇનપુટ લાઇન્સને 2 આઉટપુટ લાઇન્સમાં કન્વર્ટ કરે છે & I_{0} I_{1}
I_{2} I_{3} \\
& એક સમયે માત્ર એક જ ઇનપુટ એક્ટિવ & 1 0 0 0 \\
& ઇનપુટ પોઝિશન બાઇનરીમાં એન્કોડેડ & 0 1 0 0 \\
& & 0 0 1 0 \\
& & 0 0 0 1 \\
\end{longtable}
}

\textbf{ડાયાગ્રામ: 4:2 એન્કોડર}:

\begin{verbatim}
flowchart TD
    I0["I_{0"] {-}{-} E["4:2 એન્કોડર"]}
    I1["I_{1"] {-}{-} E}
    I2["I_{2"] {-}{-} E}
    I3["I_{3"] {-}{-} E}
    E {-{-} Y1["Y_{1}"]}
    E {-{-} Y0["Y_{0}"]}
\end{verbatim}

\end{solutionbox}
\begin{mnemonicbox}
એન્કોડર ફંક્શન માટે ``ઇનપુટ પોઝિશન ક્રિએટ્સ આઉટપુટ''.

\end{mnemonicbox}
\subsection*{પ્રશ્ન 3(બ) [4
માર્ક્સ]}\label{uxaaauxab0uxab6uxaa8-3uxaac-4-uxaaeuxab0uxa95uxab8}

\textbf{ફુલ એડર બ્લોક્સનો ઉપયોગ કરીને 4-બિટ પેરેલલ એડરને સમજાવો.}

\begin{solutionbox}

\textbf{4-બિટ પેરેલલ એડર}:

{\def\LTcaptype{none} % do not increment counter
\begin{longtable}[]{@{}
  >{\raggedright\arraybackslash}p{(\linewidth - 2\tabcolsep) * \real{0.5238}}
  >{\raggedright\arraybackslash}p{(\linewidth - 2\tabcolsep) * \real{0.4762}}@{}}
\toprule\noalign{}
\begin{minipage}[b]{\linewidth}\raggedright
કોમ્પોનન્ટ
\end{minipage} & \begin{minipage}[b]{\linewidth}\raggedright
ફંક્શન
\end{minipage} \\
\midrule\noalign{}
\endhead
\bottomrule\noalign{}
\endlastfoot
\textbf{ફુલ એડર} & 3 બિટ્સ (A, B, Carry-in) ને એડ કરે છે અને Sum અને Carry-out
આપે છે \\
\textbf{પેરેલલ એડર} & 4 ફુલ એડર્સને કેરી પ્રોપેગેશન સાથે જોડે છે \\
\end{longtable}
}

\textbf{ડાયાગ્રામ: 4-બિટ પેરેલલ એડર}:

\begin{verbatim}
flowchart LR
    A0["A_{0"] {-}{-} FA0["FA"]}
    B0["B_{0"] {-}{-} FA0}
    C0["C_{0=0"] {-}{-} FA0}
    FA0 {-{-} S0["S_{0}"]}
    FA0 {-{-} "C_{1}" {-}{-} FA1["FA"]}

    A1["A_{1"] {-}{-} FA1}
    B1["B_{1"] {-}{-} FA1}
    FA1 {-{-} S1["S_{1}"]}
    FA1 {-{-} "C_{2}" {-}{-} FA2["FA"]}
    
    A2["A_{2"] {-}{-} FA2}
    B2["B_{2"] {-}{-} FA2}
    FA2 {-{-} S2["S_{2}"] }
    FA2 {-{-} "C_{3}" {-}{-} FA3["FA"]}
    
    A3["A_{3"] {-}{-} FA3}
    B3["B_{3"] {-}{-} FA3}
    FA3 {-{-} S3["S_{3}"]}
    FA3 {-{-} C4["C_{4}"]}
\end{verbatim}

\end{solutionbox}
\begin{mnemonicbox}
પેરેલલ એડરમાં કેરી પ્રોપેગેશન માટે ``કેરી ઓલવેઝ પાસેસ રાઇટ''.

\end{mnemonicbox}
\subsection*{પ્રશ્ન 3(ક) [7
માર્ક્સ]}\label{uxaaauxab0uxab6uxaa8-3uxa95-7-uxaaeuxab0uxa95uxab8}

\textbf{ટ્રુથ ટેબલ, સમીકરણ અને સર્કિટ ડાયાગ્રામ સાથે 8:1 મલ્ટિપ્લેક્સરનું વર્ણન કરો.}

\begin{solutionbox}

\textbf{8:1 મલ્ટિપ્લેક્સર}:

{\def\LTcaptype{none} % do not increment counter
\begin{longtable}[]{@{}
  >{\raggedright\arraybackslash}p{(\linewidth - 4\tabcolsep) * \real{0.3235}}
  >{\raggedright\arraybackslash}p{(\linewidth - 4\tabcolsep) * \real{0.3824}}
  >{\raggedright\arraybackslash}p{(\linewidth - 4\tabcolsep) * \real{0.2941}}@{}}
\toprule\noalign{}
\begin{minipage}[b]{\linewidth}\raggedright
કોમ્પોનન્ટ
\end{minipage} & \begin{minipage}[b]{\linewidth}\raggedright
વર્ણન
\end{minipage} & \begin{minipage}[b]{\linewidth}\raggedright
ફંક્શન
\end{minipage} \\
\midrule\noalign{}
\endhead
\bottomrule\noalign{}
\endlastfoot
\textbf{8:1 MUX} & 8 ઇનપુટ્સ, 3 સિલેક્ટ લાઇન્સ, 1 આઉટપુટ વાળો ડેટા સિલેક્ટર &
સિલેક્ટ લાઇન્સના આધારે 8 ઇનપુટ્સમાંથી એક પસંદ કરે છે \\
\end{longtable}
}

\textbf{ટ્રુથ ટેબલ}:

{\def\LTcaptype{none} % do not increment counter
\begin{longtable}[]{@{}ll@{}}
\toprule\noalign{}
સિલેક્ટ લાઇન્સ & આઉટપુટ \\
\midrule\noalign{}
\endhead
\bottomrule\noalign{}
\endlastfoot
S_{2} S_{1} S_{0} & Y \\
0 0 0 & D_{0} \\
0 0 1 & D_{1} \\
0 1 0 & D_{2} \\
0 1 1 & D_{3} \\
1 0 0 & D_{4} \\
1 0 1 & D_{5} \\
1 1 0 & D_{6} \\
1 1 1 & D_{7} \\
\end{longtable}
}

\textbf{બુલિયન સમીકરણ}: Y = S_{2}'·S_{1}'·S_{0}'·D_{0} + S_{2}'·S_{1}'·S_{0}·D_{1} +
S_{2}'·S_{1}·S_{0}'·D_{2} + S_{2}'·S_{1}·S_{0}·D_{3} + S_{2}·S_{1}'·S_{0}'·D_{4} + S_{2}·S_{1}'·S_{0}·D_{5} +
S_{2}·S_{1}·S_{0}'·D_{6} + S_{2}·S_{1}·S_{0}·D_{7}

\textbf{ડાયાગ્રામ: 8:1 MUX}:

\begin{verbatim}
flowchart TD
    D0["D_{0"] {-}{-} MUX["8:1 MUX"]}
    D1["D_{1"] {-}{-} MUX}
    D2["D_{2"] {-}{-} MUX}
    D3["D_{3"] {-}{-} MUX}
    D4["D_{4"] {-}{-} MUX}
    D5["D_{5"] {-}{-} MUX}
    D6["D_{6"] {-}{-} MUX}
    D7["D_{7"] {-}{-} MUX}
    S0["S_{0"] {-}{-} MUX}
    S1["S_{1"] {-}{-} MUX}
    S2["S_{2"] {-}{-} MUX}
    MUX {-{-} Y["Y"]}
\end{verbatim}

\end{solutionbox}
\begin{mnemonicbox}
મલ્ટિપ્લેક્સર ઓપરેશન માટે ``સિલેક્ટ ડિસાઇડ્સ ડેટા આઉટપુટ''.

\end{mnemonicbox}
\subsection*{પ્રશ્ન 3(અ) અથવા [3
માર્ક્સ]}\label{uxaaauxab0uxab6uxaa8-3uxa85-uxa85uxaa5uxab5-3-uxaaeuxab0uxa95uxab8}

\textbf{હાફ સબટ્રેક્ટરની લોજિક સર્કિટ દોરો અને તેનું કાર્ય સમજાવો.}

\begin{solutionbox}

\textbf{હાફ સબટ્રેક્ટર}:

{\def\LTcaptype{none} % do not increment counter
\begin{longtable}[]{@{}
  >{\raggedright\arraybackslash}p{(\linewidth - 4\tabcolsep) * \real{0.2778}}
  >{\raggedright\arraybackslash}p{(\linewidth - 4\tabcolsep) * \real{0.3611}}
  >{\raggedright\arraybackslash}p{(\linewidth - 4\tabcolsep) * \real{0.3611}}@{}}
\toprule\noalign{}
\begin{minipage}[b]{\linewidth}\raggedright
ફંક્શન
\end{minipage} & \begin{minipage}[b]{\linewidth}\raggedright
વર્ણન
\end{minipage} & \begin{minipage}[b]{\linewidth}\raggedright
ટ્રુથ ટેબલ
\end{minipage} \\
\midrule\noalign{}
\endhead
\bottomrule\noalign{}
\endlastfoot
\textbf{હાફ સબટ્રેક્ટર} & બે બિટ્સને બાદ કરે છે અને ડિફરન્સ અને બોરો આપે છે & A B \\
& & 0 0 \\
& & 0 1 \\
& & 1 0 \\
& & 1 1 \\
\end{longtable}
}

\textbf{લોજિક સર્કિટ}:

\begin{verbatim}
flowchart TD
    A["A"] {-{-} XOR[""]}
    B["B"] {-{-} XOR}
    XOR {-{-} D["D = A"]}

    A1["A{"] {-}{-} AND["•"]}
    B1["B"] {-{-} AND}
    AND {-{-} Bout["Bout = A•B"]}
\end{verbatim}

\textbf{સમીકરણો}:

\begin{itemize}
\tightlist
\item
  ડિફરન્સ (D) = A \oplus B
\item
  બોરો આઉટ (Bout) = A' • B
\end{itemize}

\end{solutionbox}
\begin{mnemonicbox}
હાફ સબટ્રેક્ટર ઓપરેશન માટે ``ડિફરન્ટ બિટ્સ બોરો''.

\end{mnemonicbox}
\subsection*{પ્રશ્ન 3(બ) અથવા [4
માર્ક્સ]}\label{uxaaauxab0uxab6uxaa8-3uxaac-uxa85uxaa5uxab5-4-uxaaeuxab0uxa95uxab8}

\textbf{ટ્રુથ ટેબલ અને સર્કિટ ડાયાગ્રામ સાથે 3:8 ડીકોડર સમજાવો.}

\begin{solutionbox}

\textbf{3:8 ડીકોડર}:

{\def\LTcaptype{none} % do not increment counter
\begin{longtable}[]{@{}
  >{\raggedright\arraybackslash}p{(\linewidth - 4\tabcolsep) * \real{0.2222}}
  >{\raggedright\arraybackslash}p{(\linewidth - 4\tabcolsep) * \real{0.2889}}
  >{\raggedright\arraybackslash}p{(\linewidth - 4\tabcolsep) * \real{0.4889}}@{}}
\toprule\noalign{}
\begin{minipage}[b]{\linewidth}\raggedright
ફંક્શન
\end{minipage} & \begin{minipage}[b]{\linewidth}\raggedright
વર્ણન
\end{minipage} & \begin{minipage}[b]{\linewidth}\raggedright
ટ્રુથ ટેબલ (આંશિક)
\end{minipage} \\
\midrule\noalign{}
\endhead
\bottomrule\noalign{}
\endlastfoot
\textbf{3:8 ડીકોડર} & 3-બિટ બાઇનરી ઇનપુટને 8 આઉટપુટ લાઇન્સમાં કન્વર્ટ કરે છે & A_{2}
A_{1} A_{0} \\
& એક સમયે માત્ર એક જ આઉટપુટ એક્ટિવ & 0 0 0 \\
& & 0 0 1 \\
& & \ldots{} \\
& & 1 1 1 \\
\end{longtable}
}

\textbf{સર્કિટ ડાયાગ્રામ}:

\begin{verbatim}
flowchart TD
    A0["A_{0"] {-}{-} Dec["3:8 ડીકોડર"]}
    A1["A_{1"] {-}{-} Dec}
    A2["A_{2"] {-}{-} Dec}
    Dec {-{-} Y0["Y_{0}"]}
    Dec {-{-} Y1["Y_{1}"]}
    Dec {-{-} Y2["Y_{2}"]}
    Dec {-{-} Y3["Y_{3}"]}
    Dec {-{-} Y4["Y_{4}"]}
    Dec {-{-} Y5["Y_{5}"]}
    Dec {-{-} Y6["Y_{6}"]}
    Dec {-{-} Y7["Y_{7}"]}
\end{verbatim}

\textbf{સમીકરણો}:

\begin{itemize}
\tightlist
\item
  Y_{0} = A_{2}' • A_{1}' • A_{0}'
\item
  Y_{1} = A_{2}' • A_{1}' • A_{0}
\item
  \ldots{}
\item
  Y_{7} = A_{2} • A_{1} • A_{0}
\end{itemize}

\end{solutionbox}
\begin{mnemonicbox}
ડીકોડર ઓપરેશન માટે ``બાઇનરી ઇનપુટ એક્ટિવેટ્સ આઉટપુટ''.

\end{mnemonicbox}
\subsection*{પ્રશ્ન 3(ક) અથવા [7
માર્ક્સ]}\label{uxaaauxab0uxab6uxaa8-3uxa95-uxa85uxaa5uxab5-7-uxaaeuxab0uxa95uxab8}

\textbf{ટ્રુથ ટેબલ, સમીકરણ અને સર્કિટ ડાયાગ્રામ સાથે ગ્રે થી બાઈનરી કોડ કન્વર્ટર
સમજાવો.}

\begin{solutionbox}

\textbf{ગ્રે ટુ બાઇનરી કન્વર્ટર}:

{\def\LTcaptype{none} % do not increment counter
\begin{longtable}[]{@{}
  >{\raggedright\arraybackslash}p{(\linewidth - 4\tabcolsep) * \real{0.2222}}
  >{\raggedright\arraybackslash}p{(\linewidth - 4\tabcolsep) * \real{0.2889}}
  >{\raggedright\arraybackslash}p{(\linewidth - 4\tabcolsep) * \real{0.4889}}@{}}
\toprule\noalign{}
\begin{minipage}[b]{\linewidth}\raggedright
ફંક્શન
\end{minipage} & \begin{minipage}[b]{\linewidth}\raggedright
વર્ણન
\end{minipage} & \begin{minipage}[b]{\linewidth}\raggedright
ટેબલ: ગ્રે ટુ બાઇનરી
\end{minipage} \\
\midrule\noalign{}
\endhead
\bottomrule\noalign{}
\endlastfoot
\textbf{ગ્રે ટુ બાઇનરી} & ગ્રે કોડને બાઇનરી કોડમાં કન્વર્ટ કરે છે & ગ્રે \\
& બાઇનરીનો MSB ગ્રેના MSBને સમાન & 0000 \\
& દરેક બાઇનરી બિટ, હાલના ગ્રે બિટ અને અગાઉના બાઇનરી બિટનો XOR છે & 0001 \\
& & 0011 \\
& & 0010 \\
& & 0110 \\
& & \ldots{} \\
\end{longtable}
}

\textbf{સર્કિટ ડાયાગ્રામ}:

\begin{verbatim}
flowchart LR
    G3["G_{3"] {-}{-} B3["B_{3}"]}
    G3 {-{-} XOR1[""]}
    G2["G_{2"] {-}{-} XOR1}
    XOR1 {-{-} B2["B_{2}"]}
    XOR1 {-{-} XOR2[""]}
    G1["G_{1"] {-}{-} XOR2}
    XOR2 {-{-} B1["B_{1}"]}
    XOR2 {-{-} XOR3[""]}
    G0["G_{0"] {-}{-} XOR3}
    XOR3 {-{-} B0["B_{0}"]}
\end{verbatim}

\textbf{સમીકરણો}:

\begin{itemize}
\tightlist
\item
  B_{3} = G_{3}
\item
  B_{2} = G_{3} \oplus G_{2}
\item
  B_{1} = B_{2} \oplus G_{1}
\item
  B_{0} = B_{1} \oplus G_{0}
\end{itemize}

\end{solutionbox}
\begin{mnemonicbox}
ગ્રે ટુ બાઇનરી કન્વર્ઝન માટે ``MSB સ્ટેઝ, રેસ્ટ XOR''.

\end{mnemonicbox}
\subsection*{પ્રશ્ન 4(અ) [3
માર્ક્સ]}\label{uxaaauxab0uxab6uxaa8-4uxa85-3-uxaaeuxab0uxa95uxab8}

\textbf{ટ્રુથ ટેબલ અને સર્કિટ ડાયાગ્રામ સાથે D ફ્લિપ ફ્લોપ સમજાવો.}

\begin{solutionbox}

\textbf{D ફ્લિપ-ફ્લોપ}:

{\def\LTcaptype{none} % do not increment counter
\begin{longtable}[]{@{}lll@{}}
\toprule\noalign{}
ફંક્શન & વર્ણન & ટ્રુથ ટેબલ \\
\midrule\noalign{}
\endhead
\bottomrule\noalign{}
\endlastfoot
\textbf{D ફ્લિપ-ફ્લોપ} & ડેટા/ડિલે ફ્લિપ-ફ્લોપ & CLK \\
& ક્લોક એજ પર Q, D ને ફોલો કરે છે & ↑ \\
& & ↑ \\
\end{longtable}
}

\textbf{સર્કિટ ડાયાગ્રામ}:

\begin{verbatim}
flowchart LR
    D["D"] {-{-} FF["D ફ્લિપ{-}ફ્લોપ"]}
    CLK["ક્લોક"] {-{-} FF}
    FF {-{-} Q["Q"]}
    FF {-{-} Qnot["Q"]}
\end{verbatim}

\textbf{કેરેક્ટરિસ્ટિક સમીકરણ}:

\begin{itemize}
\tightlist
\item
  Q(next) = D
\end{itemize}

\end{solutionbox}
\begin{mnemonicbox}
D ફ્લિપ-ફ્લોપ ઓપરેશન માટે ``ડેટા ડિલેઝ વન ક્લોક''.

\end{mnemonicbox}
\subsection*{પ્રશ્ન 4(બ) [4
માર્ક્સ]}\label{uxaaauxab0uxab6uxaa8-4uxaac-4-uxaaeuxab0uxa95uxab8}

\textbf{માસ્ટર સ્લેવ JK ફ્લિપ ફ્લોપનું કાર્ય સમજાવો.}

\begin{solutionbox}

\textbf{માસ્ટર-સ્લેવ JK ફ્લિપ-ફ્લોપ}:

{\def\LTcaptype{none} % do not increment counter
\begin{longtable}[]{@{}lll@{}}
\toprule\noalign{}
કોમ્પોનન્ટ & ઓપરેશન & ટ્રુથ ટેબલ \\
\midrule\noalign{}
\endhead
\bottomrule\noalign{}
\endlastfoot
\textbf{માસ્ટર} & CLK = 1 હોય ત્યારે ઇનપુટ્સને સેમ્પલ કરે છે & J K \\
\textbf{સ્લેવ} & CLK = 0 હોય ત્યારે માસ્ટર આઉટપુટને ટ્રાન્સફર કરે છે & 0 0 \\
& & 0 1 \\
& & 1 0 \\
& & 1 1 \\
\end{longtable}
}

\textbf{ડાયાગ્રામ: માસ્ટર-સ્લેવ JK}:

\begin{verbatim}
flowchart LR
    J["J"] {-{-} Master["માસ્ટર JK"]}
    K["K"] {-{-} Master}
    CLK["ક્લોક"] {-{-} Master}
    CLK{ {-}{-} Slave["સ્લેવ JK"]}
    Master {-{-} Slave}
    Slave {-{-} Q["Q"]}
    Slave {-{-} Q["Q"]}
\end{verbatim}

\textbf{કાર્યપદ્ધતિ}:

\begin{itemize}
\tightlist
\item
  \textbf{માસ્ટર સ્ટેજ}: ક્લોક હાઇ હોય ત્યારે ઇનપુટ કેપ્ચર કરે છે
\item
  \textbf{સ્લેવ સ્ટેજ}: ક્લોક લો હોય ત્યારે આઉટપુટ અપડેટ કરે છે
\item
  \textbf{રેસ કન્ડિશન અટકાવે છે} ઇનપુટ કેપ્ચર અને આઉટપુટ અપડેટને અલગ કરીને
\end{itemize}

\end{solutionbox}
\begin{mnemonicbox}
માસ્ટર-સ્લેવ ઓપરેશન માટે ``માસ્ટર સેમ્પલ્સ, સ્લેવ ટ્રાન્સફર્સ''.

\end{mnemonicbox}
\subsection*{પ્રશ્ન 4(ક) [7
માર્ક્સ]}\label{uxaaauxab0uxab6uxaa8-4uxa95-7-uxaaeuxab0uxa95uxab8}

\textbf{બ્લોક ડાયાગ્રામની મદદથી શિફ્ટ રજિસ્ટર્સનું વર્ગીકરણ કરો અને તેમાંના કોઈપણ
એકને વિગતવાર સમજાવો.}

\begin{solutionbox}

\textbf{શિફ્ટ રજિસ્ટર વર્ગીકરણ}:

{\def\LTcaptype{none} % do not increment counter
\begin{longtable}[]{@{}
  >{\raggedright\arraybackslash}p{(\linewidth - 4\tabcolsep) * \real{0.2069}}
  >{\raggedright\arraybackslash}p{(\linewidth - 4\tabcolsep) * \real{0.4483}}
  >{\raggedright\arraybackslash}p{(\linewidth - 4\tabcolsep) * \real{0.3448}}@{}}
\toprule\noalign{}
\begin{minipage}[b]{\linewidth}\raggedright
પ્રકાર
\end{minipage} & \begin{minipage}[b]{\linewidth}\raggedright
વર્ણન
\end{minipage} & \begin{minipage}[b]{\linewidth}\raggedright
ફંક્શન
\end{minipage} \\
\midrule\noalign{}
\endhead
\bottomrule\noalign{}
\endlastfoot
\textbf{SISO} & સિરિયલ ઇન સિરિયલ આઉટ & ડેટા સિરિયલી, બિટ દર બિટ, એન્ટર થાય
છે અને એક્ઝિટ થાય છે \\
\textbf{SIPO} & સિરિયલ ઇન પેરેલલ આઉટ & ડેટા સિરિયલી એન્ટર થાય છે, પેરેલલમાં
એક્ઝિટ થાય છે \\
\textbf{PISO} & પેરેલલ ઇન સિરિયલ આઉટ & ડેટા પેરેલલમાં એન્ટર થાય છે, સિરિયલી
એક્ઝિટ થાય છે \\
\textbf{PIPO} & પેરેલલ ઇન પેરેલલ આઉટ & ડેટા પેરેલલમાં એન્ટર થાય છે અને પેરેલલમાં એક્ઝિટ
થાય છે \\
\end{longtable}
}

\textbf{SIPO શિફ્ટ રજિસ્ટર વિગતવાર}:

\begin{verbatim}
flowchart LR
    Din["ડેટા ઇન"] {-{-} FF1["FF_{1}"]}
    FF1 {-{-} FF2["FF_{2}"]}
    FF2 {-{-} FF3["FF_{3}"]}
    FF3 {-{-} FF4["FF_{4}"]}
    CLK["ક્લોક"] {-{-} FF1}
    CLK {-{-} FF2}
    CLK {-{-} FF3}
    CLK {-{-} FF4}
    FF1 {-{-} Q0["Q_{0}"]}
    FF2 {-{-} Q1["Q_{1}"]}
    FF3 {-{-} Q2["Q_{2}"]}
    FF4 {-{-} Q3["Q_{3}"]}
\end{verbatim}

\textbf{SIPO શિફ્ટ રજિસ્ટરનું કાર્ય}:

\begin{itemize}
\tightlist
\item
  \textbf{સિરિયલ ડેટા} ડેટા-ઇન પિન પર, પ્રતિ ક્લોક સાયકલ એક બિટ, પ્રવેશે છે
\item
  \textbf{દરેક ફ્લિપ-ફ્લોપ} ક્લોક પલ્સ પર તેની સામગ્રીને આગળના ફ્લિપ-ફ્લોપમાં પાસ
  કરે છે
\item
  \textbf{4 ક્લોક સાયકલ્સ પછી}, 4-બિટ ડેટા બધા ફ્લિપ-ફ્લોપ્સમાં સ્ટોર થાય છે
\item
  \textbf{પેરેલલ આઉટપુટ} Q0-Q3 પરથી એક સાથે ઉપલબ્ધ થાય છે
\end{itemize}

\textbf{SIPO માટે ટાઇમિંગ ડાયાગ્રામ}:

\begin{verbatim}
Clock   \_|‾|\_|‾|\_|‾|\_|‾|\_
Data    \_\_\_|‾‾‾|\_\_\_|‾‾‾|\_
Q0      \_\_\_|‾‾‾|\_\_\_|‾‾‾|\_
Q1      \_\_\_\_\_|‾‾‾|\_\_\_|‾‾
Q2      \_\_\_\_\_\_\_|‾‾‾|\_\_\_
Q3      \_\_\_\_\_\_\_\_\_|‾‾‾|\_
\end{verbatim}

\end{solutionbox}
\begin{mnemonicbox}
SIPO ઓપરેશન માટે ``સિરિયલ ઇનપુટ્સ પેરેલલ આઉટપુટ્સ''.

\end{mnemonicbox}
\subsection*{પ્રશ્ન 4(અ) અથવા [3
માર્ક્સ]}\label{uxaaauxab0uxab6uxaa8-4uxa85-uxa85uxaa5uxab5-3-uxaaeuxab0uxa95uxab8}

\textbf{ટ્રુથ ટેબલ અને સર્કિટ ડાયાગ્રામ સાથે SR ફ્લિપ ફ્લોપ સમજાવો.}

\begin{solutionbox}

\textbf{SR ફ્લિપ-ફ્લોપ}:

{\def\LTcaptype{none} % do not increment counter
\begin{longtable}[]{@{}lll@{}}
\toprule\noalign{}
ફંક્શન & વર્ણન & ટ્રુથ ટેબલ \\
\midrule\noalign{}
\endhead
\bottomrule\noalign{}
\endlastfoot
\textbf{SR ફ્લિપ-ફ્લોપ} & સેટ-રિસેટ ફ્લિપ-ફ્લોપ & S R \\
& બેઝિક મેમોરી એલિમેન્ટ & 0 0 \\
& & 0 1 \\
& & 1 0 \\
& & 1 1 \\
\end{longtable}
}

\textbf{સર્કિટ ડાયાગ્રામ}:

\begin{verbatim}
flowchart LR
    S["S"] {-{-} NOR1["1"]}
    QN["Q{"] {-}{-} NOR1}
    NOR1 {-{-} Q["Q"]}
    R["R"] {-{-} NOR2["1"]}
    Q {-{-} NOR2}
    NOR2 {-{-} QN}
\end{verbatim}

\end{solutionbox}
\begin{mnemonicbox}
SR ફ્લિપ-ફ્લોપ ઓપરેશન માટે ``સેટ ટુ 1, રિસેટ ટુ 0''.

\end{mnemonicbox}
\subsection*{પ્રશ્ન 4(બ) અથવા [4
માર્ક્સ]}\label{uxaaauxab0uxab6uxaa8-4uxaac-uxa85uxaa5uxab5-4-uxaaeuxab0uxa95uxab8}

\textbf{ટ્રુથ ટેબલ અને સર્કિટ ડાયાગ્રામ સાથે JK ફ્લિપ ફ્લોપ સમજાવો.}

\begin{solutionbox}

\textbf{JK ફ્લિપ-ફ્લોપ}:

{\def\LTcaptype{none} % do not increment counter
\begin{longtable}[]{@{}lll@{}}
\toprule\noalign{}
ફંક્શન & વર્ણન & ટ્રુથ ટેબલ \\
\midrule\noalign{}
\endhead
\bottomrule\noalign{}
\endlastfoot
\textbf{JK ફ્લિપ-ફ્લોપ} & ઇમ્પ્રુવ્ડ SR ફ્લિપ-ફ્લોપ & J K \\
& અમાન્ય કન્ડિશન હલ કરે છે & 0 0 \\
& & 0 1 \\
& & 1 0 \\
& & 1 1 \\
\end{longtable}
}

\textbf{સર્કિટ ડાયાગ્રામ}:

\begin{verbatim}
flowchart LR
    J["J"] {-{-} AND1["•"]}
    Qn["Q{"] {-}{-} AND1}
    AND1 {-{-} OR["1"]}
    K["K"] {-{-} AND2["•"]}
    Q["Q"] {-{-} AND2}
    AND2 {-{-} OR}
    OR {-{-} FF["D FF"]}
    CLK["ક્લોક"] {-{-} FF}
    FF {-{-} Q}
    FF {-{-} Qn}
\end{verbatim}

\textbf{કેરેક્ટરિસ્ટિક સમીકરણ}:

\begin{itemize}
\tightlist
\item
  Q(next) = J•Q' + K'•Q
\end{itemize}

\end{solutionbox}
\begin{mnemonicbox}
JK ફ્લિપ-ફ્લોપ સ્ટેટ્સ માટે ``જમ્પ-કીપ-ટોગલ'' (J=1

K=0: 1

પર જમ્પ,

J=0

K=0: સ્ટેટ જાળવવો,

J=1

K=1: ટોગલ).


\end{mnemonicbox}
\subsection*{પ્રશ્ન 4(ક) અથવા [7
માર્ક્સ]}\label{uxaaauxab0uxab6uxaa8-4uxa95-uxa85uxaa5uxab5-7-uxaaeuxab0uxa95uxab8}

\textbf{ટ્રુથ ટેબલ અને સર્કિટ ડાયાગ્રામ સાથે 4-બિટ અસિંક્રોનસ અપ કાઉન્ટરનું વર્ણન
કરો.}

\begin{solutionbox}

\textbf{4-બિટ અસિંક્રોનસ અપ કાઉન્ટર}:

{\def\LTcaptype{none} % do not increment counter
\begin{longtable}[]{@{}
  >{\raggedright\arraybackslash}p{(\linewidth - 4\tabcolsep) * \real{0.2564}}
  >{\raggedright\arraybackslash}p{(\linewidth - 4\tabcolsep) * \real{0.3333}}
  >{\raggedright\arraybackslash}p{(\linewidth - 4\tabcolsep) * \real{0.4103}}@{}}
\toprule\noalign{}
\begin{minipage}[b]{\linewidth}\raggedright
ફંક્શન
\end{minipage} & \begin{minipage}[b]{\linewidth}\raggedright
વર્ણન
\end{minipage} & \begin{minipage}[b]{\linewidth}\raggedright
કાઉન્ટ સિક્વન્સ
\end{minipage} \\
\midrule\noalign{}
\endhead
\bottomrule\noalign{}
\endlastfoot
\textbf{અસિંક્રોનસ કાઉન્ટર} & રિપલ કાઉન્ટર પણ કહેવાય છે & 0000 \rightarrow 0001 \rightarrow 0010 \rightarrow
0011 \\
& ક્લોક માત્ર પહેલા FF ને ડ્રાઇવ કરે છે & 0100 \rightarrow 0101 \rightarrow 0110 \rightarrow 0111 \\
& દરેક FF અગાઉના FF આઉટપુટ દ્વારા ટ્રિગર થાય છે & 1000 \rightarrow 1001 \rightarrow 1010 \rightarrow
1011 \\
& & 1100 \rightarrow 1101 \rightarrow 1110 \rightarrow 1111 \\
\end{longtable}
}

\textbf{સર્કિટ ડાયાગ્રામ}:

\begin{verbatim}
flowchart LR
    CLK["ક્લોક"] {-{-} JK1["JK FF_{0}"]}
    J1["J=1"] {-{-} JK1}
    K1["K=1"] {-{-} JK1}
    JK1 {-{-} Q0["Q_{0}"]}
    JK1 {-{-}"Q_{0}"{-}{-} JK2["JK FF_{1}"]}
    J2["J=1"] {-{-} JK2}
    K2["K=1"] {-{-} JK2}
    JK2 {-{-} Q1["Q_{1}"]}
    JK2 {-{-}"Q_{1}"{-}{-} JK3["JK FF_{2}"]}
    J3["J=1"] {-{-} JK3}
    K3["K=1"] {-{-} JK3}
    JK3 {-{-} Q2["Q_{2}"]}
    JK3 {-{-}"Q_{2}"{-}{-} JK4["JK FF_{3}"]}
    J4["J=1"] {-{-} JK4}
    K4["K=1"] {-{-} JK4}
    JK4 {-{-} Q3["Q_{3}"]}
\end{verbatim}

\textbf{કાર્યપદ્ધતિ}:

\begin{itemize}
\tightlist
\item
  \textbf{પહેલો FF} દરેક ક્લોક પલ્સ પર ટોગલ થાય છે
\item
  \textbf{બીજો FF} જ્યારે પહેલો FF 1 થી 0 પર જાય છે ત્યારે ટોગલ થાય છે
\item
  \textbf{ત્રીજો FF} જ્યારે બીજો FF 1 થી 0 પર જાય છે ત્યારે ટોગલ થાય છે
\item
  \textbf{ચોથો FF} જ્યારે ત્રીજો FF 1 થી 0 પર જાય છે ત્યારે ટોગલ થાય છે
\end{itemize}

\end{solutionbox}
\begin{mnemonicbox}
અસિંક્રોનસ કાઉન્ટર ઓપરેશન માટે ``રિપલ કેરીઝ પ્રોપેગેશન ડિલે''.

\end{mnemonicbox}
\subsection*{પ્રશ્ન 5(અ) [3
માર્ક્સ]}\label{uxaaauxab0uxab6uxaa8-5uxa85-3-uxaaeuxab0uxa95uxab8}

\textbf{નીચેની લોજીક ફેમિલીઝની તુલના કરો: TTL, CMOS, ECL}

\begin{solutionbox}

\textbf{લોજિક ફેમિલીઝ કમ્પેરિઝન}:

{\def\LTcaptype{none} % do not increment counter
\begin{longtable}[]{@{}llll@{}}
\toprule\noalign{}
પેરામીટર & TTL & CMOS & ECL \\
\midrule\noalign{}
\endhead
\bottomrule\noalign{}
\endlastfoot
\textbf{ટેક્નોલોજી} & બાયપોલર ટ્રાન્ઝિસ્ટર્સ & MOSFETs & બાયપોલર ટ્રાન્ઝિસ્ટર્સ \\
\textbf{પાવર કન્ઝમ્પશન} & મધ્યમ & ખૂબ ઓછો & ઉચ્ચ \\
\textbf{સ્પીડ} & મધ્યમ & નીચી-મધ્યમ & ખૂબ ઉચ્ચ \\
\textbf{નોઇઝ ઇમ્યુનિટી} & મધ્યમ & ઉચ્ચ & નીચી \\
\textbf{ફેન-આઉટ} & 10 & 50+ & 25 \\
\textbf{સપ્લાય વોલ્ટેજ} & 5V & 3-15V & -5.2V \\
\end{longtable}
}

\end{solutionbox}
\begin{mnemonicbox}
લોજિક ફેમિલીઝની તુલના માટે ``ટેક્નોલોજી કન્ટ્રોલ્સ મેની
ઇલેક્ટ્રિકલ કેરેક્ટરિસ્ટિક્સ''.

\end{mnemonicbox}
\subsection*{પ્રશ્ન 5(બ) [4
માર્ક્સ]}\label{uxaaauxab0uxab6uxaa8-5uxaac-4-uxaaeuxab0uxa95uxab8}

\textbf{કોમ્બિનેશનલ અને સિક્વેન્શિયલ લોજિક સર્કિટ્સની સરખામણી કરો.}

\begin{solutionbox}

\textbf{કોમ્બિનેશનલ vs સિક્વેન્શિયલ સર્કિટ્સ}:

{\def\LTcaptype{none} % do not increment counter
\begin{longtable}[]{@{}
  >{\raggedright\arraybackslash}p{(\linewidth - 4\tabcolsep) * \real{0.1964}}
  >{\raggedright\arraybackslash}p{(\linewidth - 4\tabcolsep) * \real{0.4286}}
  >{\raggedright\arraybackslash}p{(\linewidth - 4\tabcolsep) * \real{0.3750}}@{}}
\toprule\noalign{}
\begin{minipage}[b]{\linewidth}\raggedright
પેરામીટર
\end{minipage} & \begin{minipage}[b]{\linewidth}\raggedright
કોમ્બિનેશનલ સર્કિટ્સ
\end{minipage} & \begin{minipage}[b]{\linewidth}\raggedright
સિક્વેન્શિયલ સર્કિટ્સ
\end{minipage} \\
\midrule\noalign{}
\endhead
\bottomrule\noalign{}
\endlastfoot
\textbf{આઉટપુટ આધારિત છે} & માત્ર વર્તમાન ઇનપુટ્સ પર & વર્તમાન ઇનપુટ્સ અને અગાઉની
સ્ટેટ પર \\
\textbf{મેમોરી} & કોઈ મેમોરી નથી & મેમોરી એલિમેન્ટ્સ ધરાવે છે \\
\textbf{ફીડબેક} & કોઈ ફીડબેક પાથ નથી & ફીડબેક પાથ્સ ધરાવે છે \\
\textbf{ઉદાહરણો} & એડર્સ, MUX, ડિકોડર્સ & ફ્લિપ-ફ્લોપ્સ, કાઉન્ટર્સ, રજિસ્ટર્સ \\
\textbf{ક્લોક} & ક્લોકની જરૂર નથી & ઘણી વાર ક્લોકની જરૂર પડે છે \\
\textbf{ડિઝાઇન એપ્રોચ} & ટ્રુથ ટેબલ્સ, K-મેપ્સ & સ્ટેટ ડાયાગ્રામ્સ, ટેબલ્સ \\
\end{longtable}
}

\textbf{ડાયાગ્રામ: કમ્પેરિઝન}:

\begin{verbatim}
flowchart TB
    A["કોમ્બિનેશનલ લોજિક"] {-{-} B["આઉટપુટ્સ = f(વર્તમાન ઇનપુટ્સ)"]}
    C["સિક્વેન્શિયલ લોજિક"] {-{-} D["આઉટપુટ્સ = f(વર્તમાન ઇનપુટ્સ, અગાઉની સ્ટેટ)"]}
\end{verbatim}

\end{solutionbox}
\begin{mnemonicbox}
કોમ્બિનેશનલ અને સિક્વેન્શિયલ સર્કિટ્સ વચ્ચે તફાવત કરવા માટે
``કરંટ ઓન્લી vs મેમોરી સ્ટેટ્સ''.

\end{mnemonicbox}
\subsection*{પ્રશ્ન 5(ક) [7
માર્ક્સ]}\label{uxaaauxab0uxab6uxaa8-5uxa95-7-uxaaeuxab0uxa95uxab8}

\textbf{વ્યાખ્યાયિત કરો: ફેન ઇન, ફેન આઉટ, નોઇઝ માર્જિન, પ્રોપેગેશન ડિલે, પાવર
ડિસીપેશન, ફિગર ઓફ મેરિટ, રેમ}

\begin{solutionbox}

\textbf{ડિજિટલ ઇલેક્ટ્રોનિક્સ કી ડેફિનિશન્સ}:

{\def\LTcaptype{none} % do not increment counter
\begin{longtable}[]{@{}
  >{\raggedright\arraybackslash}p{(\linewidth - 4\tabcolsep) * \real{0.1765}}
  >{\raggedright\arraybackslash}p{(\linewidth - 4\tabcolsep) * \real{0.3529}}
  >{\raggedright\arraybackslash}p{(\linewidth - 4\tabcolsep) * \real{0.4706}}@{}}
\toprule\noalign{}
\begin{minipage}[b]{\linewidth}\raggedright
ટર્મ
\end{minipage} & \begin{minipage}[b]{\linewidth}\raggedright
વ્યાખ્યા
\end{minipage} & \begin{minipage}[b]{\linewidth}\raggedright
ટિપિકલ વેલ્યુઝ
\end{minipage} \\
\midrule\noalign{}
\endhead
\bottomrule\noalign{}
\endlastfoot
\textbf{ફેન-ઇન} & લોજિક ગેટ જેટલા ઇનપુટ્સ હેન્ડલ કરી શકે તેની મહત્તમ સંખ્યા & TTL:
2-8, CMOS: 100+ \\
\textbf{ફેન-આઉટ} & સિંગલ આઉટપુટ દ્વારા જેટલા ગેટ ઇનપુટ્સ ડ્રાઇવ કરી શકાય તેની
મહત્તમ સંખ્યા & TTL: 10, CMOS: 50 \\
\textbf{નોઇઝ માર્જિન} & એરર થાય તે પહેલાં ઉમેરી શકાય તેવો મહત્તમ નોઇઝ વોલ્ટેજ &
TTL: 0.4V, CMOS: 1.5V \\
\textbf{પ્રોપેગેશન ડિલે} & ઇનપુટમાં બદલાવથી આઉટપુટમાં બદલાવ થવામાં લાગતો સમય &
TTL: 10ns, CMOS: 20ns \\
\textbf{પાવર ડિસીપેશન} & ઓપરેશન દરમિયાન ગેટ દ્વારા વપરાતી શક્તિ & TTL: 10mW,
CMOS: 0.1mW \\
\textbf{ફિગર ઓફ મેરિટ} & સ્પીડ અને પાવરનો ગુણાકાર (ઓછો વધુ સારો) & TTL:
100pJ, CMOS: 2pJ \\
\textbf{RAM} & રેન્ડમ એક્સેસ મેમોરી - ટેમ્પરરી સ્ટોરેજ ડિવાઇસ & પ્રકાર: SRAM,
DRAM \\
\end{longtable}
}

\textbf{ડાયાગ્રામ: ડિજિટલ પેરામીટર રિલેશનશિપ્સ}:

\begin{verbatim}
flowchart LR
    A["નીચી પ્રોપેગેશન ડિલે"]{-{-}"વધારે"{-}{-}B["સ્પીડ"]}
    C["નીચી પાવર ડિસીપેશન"]{-{-}"વધારે"{-}{-}D["એફિશિયન્સી"]}
    B{-{-}"x"{-}{-}E["ફિગર ઓફ મેરિટ"]}
    D{-{-}"x"{-}{-}E}
\end{verbatim}

\end{solutionbox}
\begin{mnemonicbox}
પેરામીટર ટર્મ્સ યાદ રાખવા માટે ``ફાસ્ટ પાવર નીડ્સ પ્રોપર
ફિગર રેટિંગ્સ''.

\end{mnemonicbox}
\subsection*{પ્રશ્ન 5(અ) અથવા [3
માર્ક્સ]}\label{uxaaauxab0uxab6uxaa8-5uxa85-uxa85uxaa5uxab5-3-uxaaeuxab0uxa95uxab8}

\textbf{ડિજિટલ ICના ઇ-વેસ્ટ મેનેજમેન્ટના પગલાં અને જરૂરિયાતનું વર્ણન કરો.}

\begin{solutionbox}

\textbf{ડિજિટલ ICs માટે ઇ-વેસ્ટ મેનેજમેન્ટ}:

{\def\LTcaptype{none} % do not increment counter
\begin{longtable}[]{@{}
  >{\raggedright\arraybackslash}p{(\linewidth - 4\tabcolsep) * \real{0.1935}}
  >{\raggedright\arraybackslash}p{(\linewidth - 4\tabcolsep) * \real{0.4194}}
  >{\raggedright\arraybackslash}p{(\linewidth - 4\tabcolsep) * \real{0.3871}}@{}}
\toprule\noalign{}
\begin{minipage}[b]{\linewidth}\raggedright
સ્ટેપ
\end{minipage} & \begin{minipage}[b]{\linewidth}\raggedright
વર્ણન
\end{minipage} & \begin{minipage}[b]{\linewidth}\raggedright
મહત્વ
\end{minipage} \\
\midrule\noalign{}
\endhead
\bottomrule\noalign{}
\endlastfoot
\textbf{કલેક્શન} & ઇલેક્ટ્રોનિક વેસ્ટનું અલગ કલેક્શન & અયોગ્ય ડિસ્પોઝલને રોકે છે \\
\textbf{સેગ્રેગેશન} & ICsને અન્ય કોમ્પોનન્ટ્સથી અલગ કરવું & ટાર્ગેટેડ રિસાયક્લિંગ શક્ય
બનાવે છે \\
\textbf{ડિસમેન્ટલિંગ} & હાનિકારક ભાગોને દૂર કરવા & પર્યાવરણીય નુકસાન ઘટાડે છે \\
\textbf{રિકવરી} & મૂલ્યવાન મટીરિયલ્સ (ગોલ્ડ, સિલિકોન) એક્સટ્રેક્ટ કરવા & સંસાધનો
બચાવે છે \\
\textbf{સેફ ડિસ્પોઝલ} & નોન-રિસાયક્લેબલ પાર્ટ્સનો યોગ્ય નિકાલ & પ્રદૂષણ અટકાવે
છે \\
\end{longtable}
}

\textbf{ઇ-વેસ્ટ મેનેજમેન્ટની જરૂરિયાત}:

\begin{itemize}
\tightlist
\item
  \textbf{હાનિકારક મટીરિયલ્સ}: ICs લેડ, મર્ક્યુરી, કેડમિયમ ધરાવે છે
\item
  \textbf{રિસોર્સ કન્ઝર્વેશન}: કિંમતી ધાતુઓ અને દુર્લભ સામગ્રી પુનઃપ્રાપ્ત કરે છે
\item
  \textbf{પર્યાવરણ સંરક્ષણ}: જમીન અને પાણીના પ્રદૂષણને રોકે છે
\item
  \textbf{હેલ્થ સેફ્ટી}: ઝેરી પદાર્થોના સંપર્કને ઘટાડે છે
\end{itemize}

\end{solutionbox}
\begin{mnemonicbox}
ઇ-વેસ્ટ મેનેજમેન્ટ સ્ટેપ્સ માટે ``કલેક્શન સ્ટાર્ટ્સ ડિસમેન્ટલિંગ
રિકવરી સેફ્લી''.

\end{mnemonicbox}
\subsection*{પ્રશ્ન 5(બ) અથવા [4
માર્ક્સ]}\label{uxaaauxab0uxab6uxaa8-5uxaac-uxa85uxaa5uxab5-4-uxaaeuxab0uxa95uxab8}

\textbf{સર્કિટ ડાયાગ્રામ સાથે રીંગ કાઉન્ટરનું કામ સમજાવો.}

\begin{solutionbox}

\textbf{રીંગ કાઉન્ટર}:

{\def\LTcaptype{none} % do not increment counter
\begin{longtable}[]{@{}
  >{\raggedright\arraybackslash}p{(\linewidth - 4\tabcolsep) * \real{0.2564}}
  >{\raggedright\arraybackslash}p{(\linewidth - 4\tabcolsep) * \real{0.3333}}
  >{\raggedright\arraybackslash}p{(\linewidth - 4\tabcolsep) * \real{0.4103}}@{}}
\toprule\noalign{}
\begin{minipage}[b]{\linewidth}\raggedright
ફંક્શન
\end{minipage} & \begin{minipage}[b]{\linewidth}\raggedright
વર્ણન
\end{minipage} & \begin{minipage}[b]{\linewidth}\raggedright
કાઉન્ટ સિક્વન્સ
\end{minipage} \\
\midrule\noalign{}
\endhead
\bottomrule\noalign{}
\endlastfoot
\textbf{રીંગ કાઉન્ટર} & સિંગલ 1 સાથે સર્ક્યુલર શિફ્ટ રજિસ્ટર & 1000 \rightarrow 0100 \rightarrow 0010
\rightarrow 0001 \rightarrow 1000 \\
& કોઈપણ સમયે માત્ર એક જ ફ્લિપ-ફ્લોપ સેટ થયેલ હોય છે & \\
& N સ્ટેટ્સ માટે N ફ્લિપ-ફ્લોપ્સ & \\
\end{longtable}
}

\textbf{સર્કિટ ડાયાગ્રામ}:

\begin{verbatim}
flowchart LR
    CLK["ક્લોક"] {-{-} FF1["FF_{1}"]}
    CLK {-{-} FF2["FF_{2}"]}
    CLK {-{-} FF3["FF_{3}"]}
    CLK {-{-} FF4["FF_{4}"]}
    FF4 {-{-} FF1}
    FF1 {-{-} Q1["Q_{1}"]}
    FF1 {-{-} FF2}
    FF2 {-{-} Q2["Q_{2}"]}
    FF2 {-{-} FF3}
    FF3 {-{-} Q3["Q_{3}"]}
    FF3 {-{-} FF4}
    FF4 {-{-} Q4["Q_{4}"]}
    CLR["પ્રીસેટ"] {-{-} FF1}
    CLR {-{-} FF2}
    CLR {-{-} FF3}
    CLR {-{-} FF4}
\end{verbatim}

\textbf{કાર્યપદ્ધતિ}:

\begin{itemize}
\tightlist
\item
  \textbf{ઇનિશિયલાઇઝેશન}: પહેલા FF ને 1 પર સેટ કરવામાં આવે છે, બાકીના 0 પર
\item
  \textbf{ઓપરેશન}: સિંગલ 1 બધા ફ્લિપ-ફ્લોપ્સમાં ફરે છે
\item
  \textbf{એપ્લિકેશન્સ}: સિક્વેન્સર્સ, કન્ટ્રોલર્સ, ટાઇમિંગ સર્કિટ્સ
\end{itemize}

\end{solutionbox}
\begin{mnemonicbox}
રીંગ કાઉન્ટર ઓપરેશન માટે ``વન બિટ રોટેટ્સ ઓન્લી''.

\end{mnemonicbox}
\subsection*{પ્રશ્ન 5(ક) અથવા [7
માર્ક્સ]}\label{uxaaauxab0uxab6uxaa8-5uxa95-uxa85uxaa5uxab5-7-uxaaeuxab0uxa95uxab8}

\textbf{વર્ગીકૃત કરો: (i) મેમોરીઝ (ii) વિવિધ લોજીક ફેમિલીઝ}

\begin{solutionbox}

\textbf{(i) મેમોરી વર્ગીકરણ}:

{\def\LTcaptype{none} % do not increment counter
\begin{longtable}[]{@{}
  >{\raggedright\arraybackslash}p{(\linewidth - 4\tabcolsep) * \real{0.1875}}
  >{\raggedright\arraybackslash}p{(\linewidth - 4\tabcolsep) * \real{0.3125}}
  >{\raggedright\arraybackslash}p{(\linewidth - 4\tabcolsep) * \real{0.5000}}@{}}
\toprule\noalign{}
\begin{minipage}[b]{\linewidth}\raggedright
પ્રકાર
\end{minipage} & \begin{minipage}[b]{\linewidth}\raggedright
સબટાઇપ્સ
\end{minipage} & \begin{minipage}[b]{\linewidth}\raggedright
લક્ષણો
\end{minipage} \\
\midrule\noalign{}
\endhead
\bottomrule\noalign{}
\endlastfoot
\textbf{RAM} & \textbf{SRAM} & - સ્ટેટિક RAM- ફાસ્ટ, મોંઘી- ફ્લિપ-ફ્લોપ્સનો
ઉપયોગ કરે છે- રિફ્રેશની જરૂર નથી \\
& \textbf{DRAM} & - ડાયનેમિક RAM- સ્લોઅર, સસ્તી- કેપેસિટર્સનો ઉપયોગ કરે છે-
પીરિયોડિક રિફ્રેશની જરૂર પડે છે \\
\textbf{ROM} & \textbf{PROM} & - પ્રોગ્રામેબલ ROM- વન-ટાઇમ પ્રોગ્રામેબલ \\
& \textbf{EPROM} & - ઇરેઝેબલ PROM- UV લાઇટ દ્વારા ઇરેઝેબલ- મલ્ટિપલ
રીપ્રોગ્રામિંગ \\
& \textbf{EEPROM} & - ઇલેક્ટ્રિકલી ઇરેઝેબલ PROM- ઇલેક્ટ્રિકલ ઇરેઝર- બાઇટ-લેવલ
ઇરેઝર \\
& \textbf{ફ્લેશ} & - EEPROM વેરિએન્ટ- બ્લોક-લેવલ ઇરેઝર- નોન-વોલેટાઇલ \\
\end{longtable}
}

\textbf{(ii) લોજિક ફેમિલીઝ વર્ગીકરણ}:

{\def\LTcaptype{none} % do not increment counter
\begin{longtable}[]{@{}
  >{\raggedright\arraybackslash}p{(\linewidth - 4\tabcolsep) * \real{0.3158}}
  >{\raggedright\arraybackslash}p{(\linewidth - 4\tabcolsep) * \real{0.2632}}
  >{\raggedright\arraybackslash}p{(\linewidth - 4\tabcolsep) * \real{0.4211}}@{}}
\toprule\noalign{}
\begin{minipage}[b]{\linewidth}\raggedright
ટેક્નોલોજી
\end{minipage} & \begin{minipage}[b]{\linewidth}\raggedright
ફેમિલીઝ
\end{minipage} & \begin{minipage}[b]{\linewidth}\raggedright
લક્ષણો
\end{minipage} \\
\midrule\noalign{}
\endhead
\bottomrule\noalign{}
\endlastfoot
\textbf{બાયપોલર} & \textbf{TTL} & - ટ્રાન્ઝિસ્ટર-ટ્રાન્ઝિસ્ટર લોજિક- મધ્યમ
સ્પીડ- 5V ઓપરેશન \\
& \textbf{ECL} & - એમિટર-કપલ્ડ લોજિક- ખૂબ હાઈ સ્પીડ- હાઈ પાવર કન્ઝમ્પશન \\
& \textbf{I^{2}L} & - ઇન્ટિગ્રેટેડ ઇન્જેક્શન લોજિક- હાઈ ડેન્સિટી \\
\textbf{MOS} & \textbf{NMOS} & - N-ચેનલ MOSFET- સિમ્પલર ફેબ્રિકેશન \\
& \textbf{PMOS} & - P-ચેનલ MOSFET- લોઅર પરફોર્મન્સ \\
& \textbf{CMOS} & - કોમ્પ્લિમેન્ટરી MOS- લો પાવર કન્ઝમ્પશન- હાઈ નોઇઝ
ઇમ્યુનિટી \\
\textbf{હાઇબ્રિડ} & \textbf{BiCMOS} & - બાયપોલર અને CMOSને કોમ્બાઇન કરે છે- લો
પાવર સાથે હાઈ સ્પીડ \\
\end{longtable}
}

\textbf{મેમોરી વર્ગીકરણ ડાયાગ્રામ}:

\begin{verbatim}
flowchart TB
    MEM["સેમિકન્ડક્ટર મેમોરીઝ"]
    MEM {-{-} RAM["રેન્ડમ એક્સેસ મેમોરી (વોલેટાઇલ)"]}
    MEM {-{-} ROM["રીડ ઓન્લી મેમોરી (નોન{-}વોલેટાઇલ)"]}
    RAM {-{-} SRAM["SRAM (સ્ટેટિક)"]}
    RAM {-{-} DRAM["DRAM (ડાયનેમિક)"]}
    ROM {-{-} PROM["PROM (વન{-}ટાઇમ)"]}
    ROM {-{-} EPROM["EPROM (UV ઇરેઝેબલ)"]}
    ROM {-{-} EEPROM["EEPROM (ઇલેક્ટ્રિકલ ઇરેઝેબલ)"]}
\end{verbatim}

\end{solutionbox}
\begin{mnemonicbox}
મેમોરી પ્રકારો માટે ``રિમેમ્બર સિમ્પલ ડિવિઝન: પ્રોગ્રામેબલ
ઇરેઝેબલ ઇલેક્ટ્રિકલ'' (RAM-SRAM-DRAM, PROM-EPROM-EEPROM).

\end{mnemonicbox}

\end{document}
