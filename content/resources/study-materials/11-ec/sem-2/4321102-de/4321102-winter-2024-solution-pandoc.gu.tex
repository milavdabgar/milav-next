\documentclass[10pt,a4paper]{article}

% content/resources/templates/preamble.tex
\usepackage[margin=0.6in]{geometry}
\author{Milav Dabgar}
\usepackage{amsmath,amssymb,amsthm}
\usepackage{booktabs}
\usepackage{multirow}
\usepackage{xcolor}
\usepackage{tcolorbox}
\tcbuselibrary{breakable,skins}
\usepackage[colorlinks=true,linkcolor=blue]{hyperref}
\usepackage{titlesec}
\usepackage{enumitem}
\usepackage{tikz}
\usepackage{pgfplots}
\usepackage{circuitikz}
\usepackage[version=4]{mhchem}
\usepackage{longtable}
\usepackage{array}
\usepackage{float}
\usepackage{caption}
\usepackage{listings}

\lstset{
  basicstyle=\small\ttfamily,
  breaklines=true,
  breakatwhitespace=false,
  postbreak=\mbox{\textcolor{red}{$\hookrightarrow$}\space},
  float=false,
  numbers=left,
  numberstyle=\tiny\color{gray},
  numbersep=10pt,
  xleftmargin=2em,
  keywordstyle=\color{blue},
  commentstyle=\color{green!60!black},
  stringstyle=\color{purple},
  backgroundcolor=\color{gray!5},
  showstringspaces=false,
  tabsize=2,
  captionpos=b,
  keepspaces=true,
  columns=flexible
}

\pgfplotsset{compat=1.18}
\usetikzlibrary{shapes,arrows,positioning,calc,patterns,decorations.pathmorphing,decorations.markings,arrows.meta}

% Color scheme
\definecolor{headcolor}{RGB}{0,102,204}
\definecolor{keycolor}{RGB}{220,20,60}
\definecolor{solutioncolor}{RGB}{34,139,34}
\definecolor{mnemoniccolor}{RGB}{148,0,211}
\definecolor{codecolor}{RGB}{0,0,100}

% Spacing
\setlength{\parskip}{3pt}
\setlist[itemize]{nosep}
\setlist[enumerate]{nosep}

% Title formatting
\titleformat{\section}{\Large\bfseries\color{headcolor}}{\thesection}{1em}{}
\titleformat{\subsection}{\large\bfseries\color{headcolor}}{\thesubsection}{1em}{}

% Pandoc tightlist compatibility
\providecommand{\tightlist}{%
  \setlength{\itemsep}{0pt}\setlength{\parskip}{0pt}}

% Pandoc longtable compatibility
\newcounter{none}
\def\thenone{}


% content/resources/templates/gujarati-boxes.tex
\usepackage{fontspec}
\usepackage{polyglossia}

% Set Gujarati as main language (document is primarily in Gujarati)
% Note: gloss-gujarati.ldf doesn't exist in polyglossia, but it will use hyphenation patterns
\setdefaultlanguage{gujarati}
\setotherlanguage{english}

% Configure Gujarati font properly
% Use Language=Default to prevent polyglossia from trying to add language-specific features
% that don't exist for Gujarati, which causes "empty feature" warnings
\newfontfamily\gujaratifont[Script=Gujarati,AutoFakeBold=2.5,AutoFakeSlant=0.3]{Noto Sans Gujarati}
\setmainfont[Script=Gujarati,AutoFakeBold=2.5,AutoFakeSlant=0.3]{Noto Sans Gujarati}
% Use Noto Sans Gujarati for monospace to support Gujarati in text
\setmonofont[Scale=0.9]{Noto Sans Gujarati}

% Configure English to use the same font
\newfontfamily\englishfont[Script=Gujarati,AutoFakeBold=2.5,AutoFakeSlant=0.3]{Noto Sans Gujarati}

% Translations for polyglossia
\gappto\captionsgujarati{
  \renewcommand{\tablename}{કોષ્ટક}
  \renewcommand{\figurename}{આકૃતિ}
}

% Helper for TikZ nodes to ensure Gujarati font
\newcommand{\gu}[1]{{\gujaratifont #1}}

% Custom environments
\newtcolorbox{solutionbox}{
    breakable,
    enhanced,
    colback=solutioncolor!5!white,
    colframe=solutioncolor!75!black,
    fonttitle=\bfseries,
    title=જવાબ
}

\newtcolorbox{solutionboxnobreak}{
 colback=solutioncolor!5!white,
 colframe=solutioncolor!75!black,
 fonttitle=\bfseries,
 title=જવાબ
}

\newtcolorbox{keyformula}{
 breakable,
 enhanced,
 colback=keycolor!5!white,
 colframe=keycolor!75!black,
 fonttitle=\bfseries,
 title=રાસાયણિક સમીકરણ/સૂત્ર
}

\newtcolorbox{mnemonicbox}{
 breakable,
 enhanced,
 colback=mnemoniccolor!5!white,
 colframe=mnemoniccolor!75!black,
 fonttitle=\bfseries,
 title=મેમરી ટ્રીક
}


\begin{document}

\begin{center}
{\Huge\bfseries\color{headcolor} Subject Name (Gujarati)}\\[5pt]
{\LARGE 4321102 -- Winter 2024}\\[3pt]
{\large Semester 1 Study Material}\\[3pt]
{\normalsize\textit{Detailed Solutions and Explanations}}
\end{center}

\vspace{10pt}

\subsection*{પ્રશ્ન 1(અ) [3
માર્ક્સ]}\label{uxaaauxab0uxab6uxaa8-1uxa85-3-uxaaeuxab0uxa95uxab8}

\textbf{NAND અને Ex-NOR ગેટનો સીમ્બોલ દોરો અને તેમનુ લોજિક ટેબલ લખો.}

\begin{solutionbox}

\textbf{NAND અને Ex-NOR ગેટના સિમ્બોલ અને ટ્રુથ ટેબલ:}

\begin{verbatim}
         NAND Gate                 Ex{-NOR Gate}
         \_\_\_\_\_\_\_                          \_\_\_\_\_\_\_
A {-{-}{-}{-}{-}|       |                A {-}{-}{-}{-}{-}|       |}
        |   \&   |{-{-}Y                    |   =   |{-}{-}Y}
B {-{-}{-}{-}{-}|\_\_\_\_\_\_\_|                B {-}{-}{-}{-}{-}|\_\_\_\_\_\_\_|}
        bubble output                 bubble output
\end{verbatim}

{\def\LTcaptype{none} % do not increment counter
\begin{longtable}[]{@{}lll@{}}
\toprule\noalign{}
A & B & Y (NAND) \\
\midrule\noalign{}
\endhead
\bottomrule\noalign{}
\endlastfoot
0 & 0 & 1 \\
0 & 1 & 1 \\
1 & 0 & 1 \\
1 & 1 & 0 \\
\end{longtable}
}

{\def\LTcaptype{none} % do not increment counter
\begin{longtable}[]{@{}lll@{}}
\toprule\noalign{}
A & B & Y (Ex-NOR) \\
\midrule\noalign{}
\endhead
\bottomrule\noalign{}
\endlastfoot
0 & 0 & 1 \\
0 & 1 & 0 \\
1 & 0 & 0 \\
1 & 1 & 1 \\
\end{longtable}
}

\begin{itemize}
\tightlist
\item
  \textbf{NAND ગેટ}: ફક્ત ત્યારે જ આઉટપુટ LOW હોય છે જ્યારે બધા ઇનપુટ HIGH હોય
\item
  \textbf{Ex-NOR ગેટ}: જ્યારે ઇનપુટ SAME હોય ત્યારે આઉટપુટ HIGH હોય છે
\end{itemize}

\end{solutionbox}
\begin{mnemonicbox}
``NAND બધા એક માટે ના કહે છે, Ex-NOR સરખા સિગ્નલ માટે હા
કહે છે''

\end{mnemonicbox}
\subsection*{પ્રશ્ન 1(બ) [4
માર્ક્સ]}\label{uxaaauxab0uxab6uxaa8-1uxaac-4-uxaaeuxab0uxa95uxab8}

\textbf{જનદેશ મુિબ કરો: (i) 2's કોમ્પ્લેમેંટ નો ઉપયોગ કરીને બાદબાકી કરો
(1011001)_{2} - (1001101)_{2} (ii) (10110101)_{2} = ( )_{1}_{0} = ( )_{1}_{6}}

\begin{solutionbox}

\textbf{(i) 2's કોમ્પ્લેમેંટનો ઉપયોગ કરીને બાદબાકી:}

\begin{verbatim}
પગલું 1: બીજા નંબરનો 2's કોમ્પ્લેમેંટ શોધો (1001101)_{2}
        1's કોમ્પ્લેમેંટ: 0110010
        1 ઉમેરો:          0110011

પગલું 2: મિનુએંડ અને 2's કોમ્પ્લેમેંટને સરવાળો કરો
        1011001
      + 0110011
        -------
       10001100

પગલું 3: ઓવરફ્લો બિટને છોડી દો
        પરિણામ = 0001100 = (0001100)_{2}
\end{verbatim}

\textbf{(ii) (10110101)_{2} નું રૂપાંતર:}

\begin{verbatim}
દશાંશમાં:
1\times2^{7} + 0\times2^{6} + 1\times2^{5} + 1\times2^{4} + 0\times2^{3} + 1\times2^{2} + 0\times2^{1} + 1\times2^{0}
= 128 + 0 + 32 + 16 + 0 + 4 + 0 + 1
= 181_{1}_{0}

હેક્સાડેસિમલમાં:
1011 0101
 B    5
= B5_{1}_{6}
\end{verbatim}

\begin{itemize}
\tightlist
\item
  \textbf{2's કોમ્પ્લેમેંટ}: બિટ્સને ઉલટાવો અને 1 ઉમેરો
\item
  \textbf{બાઇનરી થી દશાંશ}: દરેક બિટને તેની પોઝિશન વેલ્યુ (2^{n}) થી ગુણો
\item
  \textbf{બાઇનરી થી હેક્સ}: બિટ્સને ચારના જૂથમાં વિભાજિત કરો, દરેક જૂથને રૂપાંતરિત
  કરો
\end{itemize}

\end{solutionbox}
\begin{mnemonicbox}
``બિટ્સ ઉલટાવો 1 ઉમેરો, કેરી છોડી દો''

\end{mnemonicbox}
\subsection*{પ્રશ્ન 1(ક) [7
માર્ક્સ]}\label{uxaaauxab0uxab6uxaa8-1uxa95-7-uxaaeuxab0uxa95uxab8}

\textbf{શોધો (i) (4356)_{1}_{0} = ( )_{8} = ( )_{1}_{6} = ()_{2} (ii) (101.01)_{2} \times (11.01)_{2}
(iii) ભાગાકાર કરો (101101)_{2} ને (110)_{2} વડે.}

\begin{solutionbox}

\textbf{(i) નંબર સિસ્ટમ રૂપાંતર:}

\begin{verbatim}
દશાંશથી ઓક્ટલ:
4356 \div 8 = 544 બાકી 4
544 \div 8 = 68  બાકી 0
68 \div 8 = 8    બાકી 4
8 \div 8 = 1     બાકી 0
1 \div 8 = 0     બાકી 1
નીચેથી વાંચીને: (4356)_{1}_{0} = (10404)_{8}

દશાંશથી હેક્સાડેસિમલ:
4356 \div 16 = 272 બાકી 4
272 \div 16 = 17  બાકી 0
17 \div 16 = 1    બાકી 1
1 \div 16 = 0     બાકી 1
નીચેથી વાંચીને: (4356)_{1}_{0} = (1104)_{1}_{6}

દશાંશથી બાઇનરી:
4356 = 1000100000100_{2}
\end{verbatim}

\textbf{(ii) બાઇનરી ગુણાકાર:}

\begin{verbatim}
      101.01
    \times 11.01
    -------
      10101
     10101
    10101
   10101
   ---------
   1111.1101
\end{verbatim}

\textbf{(iii) બાઇનરી ભાગાકાર:}

\begin{verbatim}
          111.
       ------
110 ) 101101
      110
      -----
       11101
       110
       -----
       1001
        110
        ----
        11
\end{verbatim}

\begin{itemize}
\tightlist
\item
  \textbf{દશાંશ થી ઓક્ટલ}: વારંવાર 8 થી ભાગો
\item
  \textbf{દશાંશ થી હેક્સ}: વારંવાર 16 થી ભાગો
\item
  \textbf{બાઇનરી ઓપરેશન્સ}: દશાંશની જેમ જ પ્રક્રિયા અનુસરો
\end{itemize}

\end{solutionbox}
\begin{mnemonicbox}
``ભાગો અને બાકીને નીચેથી ઉપર ગોઠવો''

\end{mnemonicbox}
\subsection*{પ્રશ્ન 1(ક-OR) [7
માર્ક્સ]}\label{uxaaauxab0uxab6uxaa8-1uxa95-or-7-uxaaeuxab0uxa95uxab8}

\textbf{શોધો (8642)_{1}_{0} = ( )_{8} = ( )_{1}_{6} = ()_{2} (ii) NOR અને Ex-OR ગેટનો સીમ્બોલ
દોરો અને તેમનુ લોજિક ટેબલ લખો.}

\begin{solutionbox}

\textbf{(i) નંબર સિસ્ટમ રૂપાંતર:}

\begin{verbatim}
દશાંશથી ઓક્ટલ:
8642 \div 8 = 1080 બાકી 2
1080 \div 8 = 135  બાકી 0
135 \div 8 = 16    બાકી 7
16 \div 8 = 2      બાકી 0
2 \div 8 = 0       બાકી 2
નીચેથી વાંચીને: (8642)_{1}_{0} = (20702)_{8}

દશાંશથી હેક્સાડેસિમલ:
8642 \div 16 = 540 બાકી 2
540 \div 16 = 33   બાકી 12(C)
33 \div 16 = 2     બાકી 1
2 \div 16 = 0      બાકી 2
નીચેથી વાંચીને: (8642)_{1}_{0} = (21C2)_{1}_{6}

દશાંશથી બાઇનરી:
8642 = 10000111000010_{2}
\end{verbatim}

\textbf{(ii) NOR અને Ex-OR ગેટ્સ:}

\begin{verbatim}
         NOR Gate                 Ex{-OR Gate}
         \_\_\_\_\_\_\_                          \_\_\_\_\_\_\_
A {-{-}{-}{-}{-}|       |                A {-}{-}{-}{-}{-}|       |}
        |  1   |{-{-}Y                    |   =   |{-}{-}Y}
B {-{-}{-}{-}{-}|\_\_\_\_\_\_\_|                B {-}{-}{-}{-}{-}|\_\_\_\_\_\_\_|}
         bubble output
\end{verbatim}

{\def\LTcaptype{none} % do not increment counter
\begin{longtable}[]{@{}lll@{}}
\toprule\noalign{}
A & B & Y (NOR) \\
\midrule\noalign{}
\endhead
\bottomrule\noalign{}
\endlastfoot
0 & 0 & 1 \\
0 & 1 & 0 \\
1 & 0 & 0 \\
1 & 1 & 0 \\
\end{longtable}
}

{\def\LTcaptype{none} % do not increment counter
\begin{longtable}[]{@{}lll@{}}
\toprule\noalign{}
A & B & Y (Ex-OR) \\
\midrule\noalign{}
\endhead
\bottomrule\noalign{}
\endlastfoot
0 & 0 & 0 \\
0 & 1 & 1 \\
1 & 0 & 1 \\
1 & 1 & 0 \\
\end{longtable}
}

\begin{itemize}
\tightlist
\item
  \textbf{NOR ગેટ}: ફક્ત ત્યારે જ આઉટપુટ HIGH હોય છે જ્યારે બધા ઇનપુટ LOW હોય
\item
  \textbf{Ex-OR ગેટ}: જ્યારે ઇનપુટ DIFFERENT હોય ત્યારે આઉટપુટ HIGH હોય છે
\end{itemize}

\end{solutionbox}
\begin{mnemonicbox}
``NOR બધા શૂન્ય માટે હા કહે છે, Ex-OR અલગ સિગ્નલ માટે હા કહે
છે''

\end{mnemonicbox}
\subsection*{પ્રશ્ન 2(અ) [3
માર્ક્સ]}\label{uxaaauxab0uxab6uxaa8-2uxa85-3-uxaaeuxab0uxa95uxab8}

\textbf{સાબિત કરો xy+xz+yz' = xz+yz'}

\begin{solutionbox}

\begin{verbatim}
ડાબી બાજુ: xy + xz + yz'
= xy + xz + yz'
= x(y + z) + yz'        [વિતરણ ગુણધર્મ]
= xy + xz + yz'         [વિસ્તાર]
= xy + yz' + xz         [પુનર્ગઠન]
= y(x + z') + xz        [વિતરણ ગુણધર્મ]
= xy + yz' + xz         [વિસ્તાર]
= (x + y)z' + xz        [પુનર્ગઠન]
= xz' + yz' + xz        [વિસ્તાર]
= x(z' + z) + yz'       [વિતરણ ગુણધર્મ]
= x(1) + yz'            [પૂરક ગુણધર્મ]
= x + yz'               [ઓળખ ગુણધર્મ]
= xz + x(1-z) + yz'     [x = xz + xz']
= xz + xz' + yz'        [વિસ્તાર]
= xz + z'(x + y)        [વિતરણ ગુણધર્મ]
= xz + z'x + z'y        [વિસ્તાર]
= xz + xz' + yz'        [પુનર્ગઠન]
= x(z + z') + yz'       [વિતરણ ગુણધર્મ]
= x(1) + yz'            [પૂરક ગુણધર્મ]
= x + yz'               [ઓળખ ગુણધર્મ]
= xz + yz'              [જમણી બાજુ સમાન]
\end{verbatim}

\begin{itemize}
\tightlist
\item
  \textbf{વિતરણ ગુણધર્મ}: x(y+z) = xy+xz
\item
  \textbf{પૂરક ગુણધર્મ}: z+z' = 1
\item
  \textbf{ઓળખ ગુણધર્મ}: x\times1 = x
\end{itemize}

\end{solutionbox}
\begin{mnemonicbox}
``ફેક્ટર કરો, એક્સપાન્ડ કરો, ફરીથી ગોઠવો, ફરીથી ફેક્ટર
કરો''

\end{mnemonicbox}
\subsection*{પ્રશ્ન 2(બ) [4
માર્ક્સ]}\label{uxaaauxab0uxab6uxaa8-2uxaac-4-uxaaeuxab0uxa95uxab8}

\textbf{k-મેપની મદદથી f(W,X,Y,Z) = \summ(0,1,2,3,5,7,8,9,11,14) એક્સ્પ્રેશન
ઘટાડો.}

\begin{solutionbox}

\textbf{f(W,X,Y,Z) = \summ(0,1,2,3,5,7,8,9,11,14) માટે K-Map:}

\begin{verbatim}
      YZ
WX   00  01  11  10
00    1   1   0   1
01    1   1   1   0
11    0   0   1   1
10    1   1   0   0
\end{verbatim}

\textbf{ગ્રુપિંગ:}

\begin{itemize}
\tightlist
\item
  ગ્રુપ 1: m(0,1,2,3) = W'X' (2\times2 ગ્રુપ)
\item
  ગ્રુપ 2: m(0,1,8,9) = Y' (2\times2 ગ્રુપ)
\item
  ગ્રુપ 3: m(2,3,11) = X'Z (2\times2 ગ્રુપ, વ્રેપિંગ સાથે)
\item
  ગ્રુપ 4: m(7,14) = XZ (જોડી)
\end{itemize}

\textbf{સરળીકૃત સમીકરણ:} f(W,X,Y,Z) = W'X' + Y' + X'Z + XZ

\begin{itemize}
\tightlist
\item
  \textbf{K-Map ટેકનિક}: બાજુના 1ને 2ની ઘાતમાં ગ્રુપ કરો
\item
  \textbf{દરેક ગ્રુપ}: સરળીકૃત સમીકરણમાં એક પદનું પ્રતિનિધિત્વ કરે છે
\item
  \textbf{મોટા ગ્રુપ}: વધુ સરળ સમીકરણનો અર્થ
\end{itemize}

\end{solutionbox}
\begin{mnemonicbox}
``2ની ઘાતો સમીકરણને નવું બનાવે છે''

\end{mnemonicbox}
\subsection*{પ્રશ્ન 2(ક) [7
માર્ક્સ]}\label{uxaaauxab0uxab6uxaa8-2uxa95-7-uxaaeuxab0uxa95uxab8}

\textbf{NOR ગેટને યુજનવસસલ ગેટ તરીકે સમજાવો}

\begin{solutionbox}

\textbf{NOR યુનિવર્સલ ગેટ તરીકે:}

NOR ગેટ બધા મૂળભૂત લોજિક ફંકશન્સને અમલમાં મૂકી શકે છે, જે તેને યુનિવર્સલ ગેટ બનાવે છે.

\textbf{NOR વડે મૂળભૂત ગેટ્સનું અમલીકરણ:}

{\def\LTcaptype{none} % do not increment counter
\begin{longtable}[]{@{}ll@{}}
\toprule\noalign{}
ગેટ & NOR સાથે અમલીકરણ \\
\midrule\noalign{}
\endhead
\bottomrule\noalign{}
\endlastfoot
NOT & A NOR A \\
OR & (A NOR B) NOR (A NOR B) \\
AND & (A NOR A) NOR (B NOR B) \\
\end{longtable}
}

\textbf{સર્કિટ ડાયાગ્રામ્સ:}

\begin{verbatim}
NOT Gate using NOR:
A--->|NOR|-->Y

OR Gate using NOR:
A--->|     |      |     |
     | NOR |------|     |
B--->|     |      | NOR |-->Y
                  |     |
                  |     |

AND Gate using NOR:
A--->|NOR|-------|     |
                 |     |
                 | NOR |-->Y
                 |     |
B--->|NOR|-------|     |
\end{verbatim}

\begin{itemize}
\tightlist
\item
  \textbf{યુનિવર્સલ ગેટ}: કોઈપણ બુલિયન ફંક્શન અમલમાં મૂકી શકે છે
\item
  \textbf{NOR ઓપરેશન}: NOT OR, આઉટપુટ હાઈ ફક્ત ત્યારે જ જ્યારે બધા ઇનપુટ લો હોય
\item
  \textbf{અમલીકરણ ખર્ચ}: જટિલ ફંક્શન્સ માટે બહુવિધ NOR ગેટ્સની જરૂર પડે છે
\end{itemize}

\end{solutionbox}
\begin{mnemonicbox}
``NOR એટલે Not-OR, પણ Not-AND-OR બધું કરી શકે છે''

\end{mnemonicbox}
\subsection*{પ્રશ્ન 2(અ-OR) [3
માર્ક્સ]}\label{uxaaauxab0uxab6uxaa8-2uxa85-or-3-uxaaeuxab0uxa95uxab8}

\textbf{બુજલયન એક્સ્પ્રેશન P = (x'+y'+z)(x+y+z')+(xyz) માટે લોજિક સજકસટ દોરો}

\begin{solutionbox}

\textbf{P = (x'+y'+z)(x+y+z')+(xyz) માટે લોજિક સર્કિટ:}

\begin{verbatim}
      |{-{-}{-}{-}{-}{-}|}
x{{-}{-}{-}|      |}
y{{-}{-}{-}| OR   |{-}{-}{-}{-}{-}|}
z{-{-}{-}{-}|      |     |     |{-}{-}{-}{-}{-}{-}|}
      |{-{-}{-}{-}{-}{-}|     |{-}{-}{-}{-}|      |}
                         | AND  |{-{-}{-}{-}{-}{-}|}
      |{-{-}{-}{-}{-}{-}|     |{-}{-}{-}{-}|      |      |}
x{-{-}{-}{-}|      |     |     |{-}{-}{-}{-}{-}{-}|      |     |{-}{-}{-}{-}{-}{-}|}
y{-{-}{-}{-}| OR   |{-}{-}{-}{-}{-}|                   |{-}{-}{-}{-}|      |}
z{{-}{-}{-}|      |                         |     | OR   |{-}{-}{-} P}
      |{-{-}{-}{-}{-}{-}|                         |     |      |}
                                       |     |{-{-}{-}{-}{-}{-}|}
      |{-{-}{-}{-}{-}{-}|                         |}
x{-{-}{-}{-}|      |                         |}
y{-{-}{-}{-}| AND  |{-}{-}{-}{-}{-}{-}{-}{-}{-}{-}{-}{-}{-}{-}{-}{-}{-}{-}{-}{-}{-}{-}{-}{-}{-}|}
z{-{-}{-}{-}|      |}
      |{-{-}{-}{-}{-}{-}|}
\end{verbatim}

\begin{itemize}
\tightlist
\item
  \textbf{પગલું 1}: દરેક પ્રોડક્ટ ટર્મનું અમલીકરણ કરો
\item
  \textbf{પગલું 2}: પછી OR ગેટ સાથે જોડો
\item
  \textbf{પગલું 3}: ઓપરેટર પ્રાથમિકતા અનુસરો
\end{itemize}

\end{solutionbox}
\begin{mnemonicbox}
``પહેલા પ્રોડક્ટ્સ, પછી તેમનો સરવાળો કરો''

\end{mnemonicbox}
\subsection*{પ્રશ્ન 2(બ-OR) [4
માર્ક્સ]}\label{uxaaauxab0uxab6uxaa8-2uxaac-or-4-uxaaeuxab0uxa95uxab8}

\textbf{K-મેપ પદ્ધજતનો ઉપયોગ કરીને f(W,X,Y,Z) = \summ(1,3,7,11,15) એક્સ્પ્રેશન ને
રીડ્યુસ કરો િેમા ડોોંટ કેર ની શરત d(0,2,5) વાપરો.}

\begin{solutionbox}

\textbf{ડોન્ટ કેર કન્ડિશન્સ સાથે K-Map:}

\begin{verbatim}
      YZ
WX   00  01  11  10
00    d   1   0   d
01    0   0   1   d
11    0   0   1   1
10    0   0   1   0
\end{verbatim}

\textbf{ગ્રુપિંગ:}

\begin{itemize}
\tightlist
\item
  ગ્રુપ 1: m(1,3,7,15) + d(0,2) = X'Z + YZ (જોડીઓ)
\item
  ગ્રુપ 2: m(7,15) + d(5) = WYZ (ચતુષ્ક)
\end{itemize}

\textbf{સરળીકૃત સમીકરણ:} f(W,X,Y,Z) = X'Z + YZ

\begin{itemize}
\tightlist
\item
  \textbf{ડોન્ટ કેર કન્ડિશન્સ}: સરળતા માટે 0 અથવા 1 તરીકે ગણી શકાય છે
\item
  \textbf{ઇષ્ટતમ ગ્રુપિંગ}: મોટા જૂથો બનાવવા માટે ડોન્ટ કેર્સનો ઉપયોગ કરો
\item
  \textbf{સરળીકરણનો ધ્યેય}: પદોની સંખ્યા ઘટાડવી
\end{itemize}

\end{solutionbox}
\begin{mnemonicbox}
``ડોન્ટ કેર્સ મોટા ચોરસ બનાવવામાં મદદ કરે છે''

\end{mnemonicbox}
\subsection*{પ્રશ્ન 2(ક-OR) [7
માર્ક્સ]}\label{uxaaauxab0uxab6uxaa8-2uxa95-or-7-uxaaeuxab0uxa95uxab8}

\textbf{બુજલયન થીયરમ અને તેની તમામ પ્ર ોપ્રટીઝ લખો.}

\begin{solutionbox}

\textbf{મૂળભૂત બુલિયન થિયરમ અને તેના ગુણધર્મો:}

{\def\LTcaptype{none} % do not increment counter
\begin{longtable}[]{@{}
  >{\raggedright\arraybackslash}p{(\linewidth - 2\tabcolsep) * \real{0.5385}}
  >{\raggedright\arraybackslash}p{(\linewidth - 2\tabcolsep) * \real{0.4615}}@{}}
\toprule\noalign{}
\begin{minipage}[b]{\linewidth}\raggedright
નિયમ/ગુણધર્મ
\end{minipage} & \begin{minipage}[b]{\linewidth}\raggedright
સમીકરણ
\end{minipage} \\
\midrule\noalign{}
\endhead
\bottomrule\noalign{}
\endlastfoot
\textbf{ઓળખ નિયમ} & A + 0 = A, A · 1 = A \\
\textbf{નલ નિયમ} & A + 1 = 1, A · 0 = 0 \\
\textbf{ઇડેમપોટન્ટ નિયમ} & A + A = A, A · A = A \\
\textbf{પૂરક નિયમ} & A + A' = 1, A · A' = 0 \\
\textbf{ક્રમવિનિમય નિયમ} & A + B = B + A, A · B = B · A \\
\textbf{સંગઠન નિયમ} & A + (B + C) = (A + B) + C, A · (B · C) = (A · B) ·
C \\
\textbf{વિતરણ નિયમ} & A · (B + C) = A · B + A · C, A + (B · C) = (A + B)
· (A + C) \\
\textbf{અવશોષણ નિયમ} & A + (A · B) = A, A · (A + B) = A \\
\textbf{ડીમોર્ગનનો થિયરમ} & (A + B)' = A' · B', (A · B)' = A' + B' \\
\textbf{ડબલ કોમ્પ્લિમેન્ટ} & (A')' = A \\
\textbf{કોન્સેન્સસ થિયરમ} & (A · B) + (A' · C) + (B · C) = (A · B) + (A' ·
C) \\
\end{longtable}
}

\begin{itemize}
\tightlist
\item
  \textbf{મૂળભૂત ઓપરેશન્સ}: AND (·), OR (+), NOT (')
\item
  \textbf{કી એપ્લિકેશન્સ}: સર્કિટ સરળીકરણ અને ડિઝાઇન
\item
  \textbf{થિયરમનું મહત્વ}: ગેટ કાઉન્ટ અને જટિલતા ઘટાડે છે
\end{itemize}

\end{solutionbox}
\begin{mnemonicbox}
``COIN-CADDAM'' (કોમ્પ્લિમેન્ટરી, ડિસ્ટ્રિબ્યુટિવ,
એસોસિએટિવ, વગેરે)

\end{mnemonicbox}
\subsection*{પ્રશ્ન 3(અ) [3
માર્ક્સ]}\label{uxaaauxab0uxab6uxaa8-3uxa85-3-uxaaeuxab0uxa95uxab8}

\textbf{ફુલ સબ્ટરેક્સ્પટરની લોજિક સજકસટ દોરો અને તેનુોં કાયસ સમજાવો.}

\begin{solutionbox}

\textbf{ફુલ સબ્ટ્રેક્ટર સર્કિટ:}

\begin{verbatim}
          |{-{-}{-}{-}{-}{-}|}
A{-{-}{-}{-}{-}{-}{-}{-}|      |}
          | XOR  |{-{-}{-}{-}{-}{-}{-}{-}{-}|}
B{-{-}{-}{-}{-}{-}{-}{-}|      |         |     |{-}{-}{-}{-}{-}{-}|}
          |{-{-}{-}{-}{-}{-}|         |{-}{-}{-}{-}|      |}
                                 | XOR  |{-{-}{-} Difference}
         |{-{-}{-}{-}{-}{-}|          |{-}{-}{-}{-}|      |}
C\_in{-{-}{-}{-}|      |          |     |{-}{-}{-}{-}{-}{-}|}
         |      |{-{-}{-}{-}{-}{-}{-}{-}{-}{-}|}
A{-{-}{-}{-}{-}{-}{-}| NAND |}
         |      |
         |{-{-}{-}{-}{-}{-}|                |{-}{-}{-}{-}{-}{-}|}
                                 |      |{-{-}{-} Borrow}
          |{-{-}{-}{-}{-}{-}|         |{-}{-}{-}{-}| OR   |}
B{-{-}{-}{-}{-}{-}{-}{-}|      |         |     |      |}
          | NAND |{-{-}{-}{-}{-}{-}{-}{-}{-}|     |{-}{-}{-}{-}{-}{-}|}
C\_in{-{-}{-}{-}{-}|      |}
          |{-{-}{-}{-}{-}{-}|}
\end{verbatim}

\textbf{ટ્રુથ ટેબલ:}

{\def\LTcaptype{none} % do not increment counter
\begin{longtable}[]{@{}lllll@{}}
\toprule\noalign{}
A & B & C\_in & Difference & Borrow \\
\midrule\noalign{}
\endhead
\bottomrule\noalign{}
\endlastfoot
0 & 0 & 0 & 0 & 0 \\
0 & 0 & 1 & 1 & 1 \\
0 & 1 & 0 & 1 & 1 \\
0 & 1 & 1 & 0 & 1 \\
1 & 0 & 0 & 1 & 0 \\
1 & 0 & 1 & 0 & 0 \\
1 & 1 & 0 & 0 & 0 \\
1 & 1 & 1 & 1 & 1 \\
\end{longtable}
}

\begin{itemize}
\tightlist
\item
  \textbf{ડિફરન્સ}: A \oplus B \oplus C\_in (બધા ઇનપુટનો XOR)
\item
  \textbf{બોરો}: C\_in·(A \oplus B) + B·A' (જરૂર પડે ત્યારે જનરેટ થાય છે)
\end{itemize}

\end{solutionbox}
\begin{mnemonicbox}
``જ્યારે સબ્ટ્રાહેન્ડ મિનુએન્ડ કરતા વધારે હોય ત્યારે બોરોની જરૂર
પડે છે''

\end{mnemonicbox}
\subsection*{પ્રશ્ન 3(બ) [4
માર્ક્સ]}\label{uxaaauxab0uxab6uxaa8-3uxaac-4-uxaaeuxab0uxa95uxab8}

\textbf{ગ્રે થી બાઈનરી કોડ કન્વટસરની સજકસટ દોરો.}

\begin{solutionbox}

\textbf{ગ્રે થી બાઇનરી કોડ કન્વર્ટર (4-બિટ):}

\begin{verbatim}
       G3          G2          G1          G0
       |           |           |           |
       |           |           |           |
       v           v           v           v
       |{-{-}{-}{-}{-}{-}|    |{-}{-}{-}{-}{-}{-}|    |{-}{-}{-}{-}{-}{-}|    |}
       |      |    |      |    |      |    |
G3{-{-}{-}{-}| XNOR |{-}{-}{-}| XNOR |{-}{-}{-}| XNOR |{-}{-}{-}|{-}{-} B0}
       |      |    |      |    |      |    |
       |{-{-}{-}{-}{-}{-}|    |{-}{-}{-}{-}{-}{-}|    |{-}{-}{-}{-}{-}{-}|    |}
         \^{           \^{}           \^{}         |}
         |           |           |         |
      B3 |        B2 |        B1 |        G0=B0
\end{verbatim}

\textbf{રૂપાંતરણ ટેબલ:}

{\def\LTcaptype{none} % do not increment counter
\begin{longtable}[]{@{}ll@{}}
\toprule\noalign{}
ગ્રે & બાઇનરી \\
\midrule\noalign{}
\endhead
\bottomrule\noalign{}
\endlastfoot
G3G2G1G0 & B3B2B1B0 \\
0000 & 0000 \\
0001 & 0001 \\
0011 & 0010 \\
0010 & 0011 \\
0110 & 0100 \\
\ldots{} & \ldots{} \\
\end{longtable}
}

\begin{itemize}
\tightlist
\item
  \textbf{રૂપાંતરણ સિદ્ધાંત}: B_{3} = G_{3}, B_{2} = B_{3} \oplus G_{2}, B_{1} = B_{2} \oplus G_{1}, B_{0} = B_{1} \oplus
  G_{0}
\item
  \textbf{મુખ્ય વિશેષતા}: દરેક બાઇનરી બિટ તમામ અગાઉના ગ્રે બિટ્સ પર આધાર રાખે છે
\item
  \textbf{અનુપ્રયોગ}: ડિજિટલ ટ્રાન્સમિશનમાં ભૂલ શોધન
\end{itemize}

\end{solutionbox}
\begin{mnemonicbox}
``MSB રહે છે, અન્ય અગાઉના બાઇનરીની સાથે XOR થાય છે''

\end{mnemonicbox}
\subsection*{પ્રશ્ન 3(ક) [7
માર્ક્સ]}\label{uxaaauxab0uxab6uxaa8-3uxa95-7-uxaaeuxab0uxa95uxab8}

\textbf{2:4 ડીકોડર અને 4:1 મજટટ્લેક્સ્પસર દોરો અને તેનુોં કાયસ સમજાવો.}

\begin{solutionbox}

\textbf{2:4 ડિકોડર:}

\begin{verbatim}
       |{-{-}{-}{-}{-}{-}|}
       |      |{-{-}{-}{-}{-} Y0 (AB)}
       |      |
A{-{-}{-}{-}{-}| 2:4  |{-}{-}{-}{-}{-} Y1 (AB)}
       |      |
B{-{-}{-}{-}{-}| DEC  |{-}{-}{-}{-}{-} Y2 (AB)}
       |      |
       |      |{-{-}{-}{-}{-} Y3 (AB)}
       |{-{-}{-}{-}{-}{-}|}
\end{verbatim}

\textbf{ટ્રુથ ટેબલ:}

{\def\LTcaptype{none} % do not increment counter
\begin{longtable}[]{@{}llllll@{}}
\toprule\noalign{}
A & B & Y0 & Y1 & Y2 & Y3 \\
\midrule\noalign{}
\endhead
\bottomrule\noalign{}
\endlastfoot
0 & 0 & 1 & 0 & 0 & 0 \\
0 & 1 & 0 & 1 & 0 & 0 \\
1 & 0 & 0 & 0 & 1 & 0 \\
1 & 1 & 0 & 0 & 0 & 1 \\
\end{longtable}
}

\textbf{4:1 મલ્ટિપ્લેક્સર:}

\begin{verbatim}
       |{-{-}{-}{-}{-}{-}|}
D0{-{-}{-}{-}|      |}
       |      |
D1{-{-}{-}{-}| 4:1  |}
       |      |{-{-}{-}{-}{-} Y}
D2{-{-}{-}{-}| MUX  |}
       |      |
D3{-{-}{-}{-}|      |}
       |{-{-}{-}{-}{-}{-}|}
         \^{  \^{}}
         |  |
        S0 S1
\end{verbatim}

\textbf{ટ્રુથ ટેબલ:}

{\def\LTcaptype{none} % do not increment counter
\begin{longtable}[]{@{}lll@{}}
\toprule\noalign{}
S1 & S0 & Y \\
\midrule\noalign{}
\endhead
\bottomrule\noalign{}
\endlastfoot
0 & 0 & D0 \\
0 & 1 & D1 \\
1 & 0 & D2 \\
1 & 1 & D3 \\
\end{longtable}
}

\begin{itemize}
\tightlist
\item
  \textbf{ડિકોડર}: બાઇનરી કોડને વન-હોટ આઉટપુટમાં રૂપાંતરિત કરે છે
\item
  \textbf{મલ્ટિપ્લેક્સર}: સિલેક્શન લાઇન્સના આધારે ઘણા ઇનપુટમાંથી એક પસંદ કરે છે
\item
  \textbf{અનુપ્રયોગો}: મેમરી એડ્રેસિંગ, ડેટા રાઉટિંગ
\end{itemize}

\end{solutionbox}
\begin{mnemonicbox}
``ડિકોડર: એક-થી-ઘણા, મક્સ: ઘણા-થી-એક''

\end{mnemonicbox}
\subsection*{પ્રશ્ન 3(અ-OR) [3
માર્ક્સ]}\label{uxaaauxab0uxab6uxaa8-3uxa85-or-3-uxaaeuxab0uxa95uxab8}

\textbf{ફુલ એડરની લોજિક સજકસટ દોરો અને તેનુોં કાયસ સમજાવો.}

\begin{solutionbox}

\textbf{ફુલ એડર સર્કિટ:}

\begin{verbatim}
          |{-{-}{-}{-}{-}{-}|}
A{-{-}{-}{-}{-}{-}{-}{-}|      |}
          | XOR  |{-{-}{-}{-}{-}{-}{-}{-}{-}|}
B{-{-}{-}{-}{-}{-}{-}{-}|      |         |     |{-}{-}{-}{-}{-}{-}|}
          |{-{-}{-}{-}{-}{-}|         |{-}{-}{-}{-}|      |}
                                 | XOR  |{-{-}{-} Sum}
          |{-{-}{-}{-}{-}{-}|         |{-}{-}{-}{-}|      |}
C\_in{-{-}{-}{-}{-}|      |         |     |{-}{-}{-}{-}{-}{-}|}
          |      |{-{-}{-}{-}{-}{-}{-}{-}{-}|}
          |      |
          |{-{-}{-}{-}{-}{-}|}
                          |{-{-}{-}{-}{-}{-}|}
          |{-{-}{-}{-}{-}{-}|        |      |}
A{-{-}{-}{-}{-}{-}{-}{-}|      |{-}{-}{-}{-}{-}{-}{-}|      |}
          | AND  |        |      |
B{-{-}{-}{-}{-}{-}{-}{-}|      |        |  OR  |{-}{-}{-} Carry}
          |{-{-}{-}{-}{-}{-}|        |      |}
                          |      |
          |{-{-}{-}{-}{-}{-}|        |      |}
C\_in{-{-}{-}{-}{-}|      |{-}{-}{-}{-}{-}{-}{-}|      |}
          | AND  |        |{-{-}{-}{-}{-}{-}|}
XOR{-{-}{-}{-}{-}{-}|      |}
          |{-{-}{-}{-}{-}{-}|}
\end{verbatim}

\textbf{ટ્રુથ ટેબલ:}

{\def\LTcaptype{none} % do not increment counter
\begin{longtable}[]{@{}lllll@{}}
\toprule\noalign{}
A & B & C\_in & Sum & Carry \\
\midrule\noalign{}
\endhead
\bottomrule\noalign{}
\endlastfoot
0 & 0 & 0 & 0 & 0 \\
0 & 0 & 1 & 1 & 0 \\
0 & 1 & 0 & 1 & 0 \\
0 & 1 & 1 & 0 & 1 \\
1 & 0 & 0 & 1 & 0 \\
1 & 0 & 1 & 0 & 1 \\
1 & 1 & 0 & 0 & 1 \\
1 & 1 & 1 & 1 & 1 \\
\end{longtable}
}

\begin{itemize}
\tightlist
\item
  \textbf{સમ}: A \oplus B \oplus C\_in (બધા ઇનપુટનો XOR)
\item
  \textbf{કેરી}: (A · B) + (C\_in · (A \oplus B)) (જરૂર પડે ત્યારે જનરેટ થાય છે)
\end{itemize}

\end{solutionbox}
\begin{mnemonicbox}
``સમ વિષમ હોય છે, કેરીને ઓછામાં ઓછા બે 1ની જરૂર પડે છે''

\end{mnemonicbox}
\subsection*{પ્રશ્ન 3(બ-OR) [4
માર્ક્સ]}\label{uxaaauxab0uxab6uxaa8-3uxaac-or-4-uxaaeuxab0uxa95uxab8}

\textbf{બાઈનરી થી ગ્રે કોડ કન્વટસરની સજકસટ દોરો.}

\begin{solutionbox}

\textbf{બાઇનરી થી ગ્રે કોડ કન્વર્ટર (4-બિટ):}

\begin{verbatim}
      B3          B2          B1          B0
       |           |           |           |
       |           |           |           |
       v           v           v           v
       |{-{-}{-}{-}{-}{-}|    |{-}{-}{-}{-}{-}{-}|    |{-}{-}{-}{-}{-}{-}|    |}
       |      |    |      |    |      |    |
B3{-{-}{-}{-}|      |{-}{-}{-}|      |{-}{-}{-}|      |{-}{-}{-}|{-}{-} G0}
       | XOR  |    | XOR  |    | XOR  |    |
       |{-{-}{-}{-}{-}{-}|    |{-}{-}{-}{-}{-}{-}|    |{-}{-}{-}{-}{-}{-}|    |}
         \^{           \^{}           \^{}         |}
         |           |           |         |
        B2          B1          B0         |
                                           |
       |                                   |
       v                                   v
       G3                                  G0
\end{verbatim}

\textbf{રૂપાંતરણ ટેબલ:}

{\def\LTcaptype{none} % do not increment counter
\begin{longtable}[]{@{}ll@{}}
\toprule\noalign{}
બાઇનરી & ગ્રે \\
\midrule\noalign{}
\endhead
\bottomrule\noalign{}
\endlastfoot
B3B2B1B0 & G3G2G1G0 \\
0000 & 0000 \\
0001 & 0001 \\
0010 & 0011 \\
0011 & 0010 \\
0100 & 0110 \\
\ldots{} & \ldots{} \\
\end{longtable}
}

\begin{itemize}
\tightlist
\item
  \textbf{રૂપાંતરણ સિદ્ધાંત}: G_{3} = B_{3}, G_{2} = B_{3} \oplus B_{2}, G_{1} = B_{2} \oplus B_{1}, G_{0} = B_{1} \oplus
  B_{0}
\item
  \textbf{મુખ્ય વિશેષતા}: આસન્ન કોડ વચ્ચે ફક્ત એક બિટ બદલાય છે
\item
  \textbf{અનુપ્રયોગ}: રોટરી એન્કોડર્સ, ભૂલ શોધન
\end{itemize}

\end{solutionbox}
\begin{mnemonicbox}
``MSB રહે છે, અન્ય બિટ્સ આસન્ન બાઇનરી બિટ્સ સાથે XOR કરે છે''

\end{mnemonicbox}
\subsection*{પ્રશ્ન 3(ક-OR) [7
માર્ક્સ]}\label{uxaaauxab0uxab6uxaa8-3uxa95-or-7-uxaaeuxab0uxa95uxab8}

\textbf{ફુલ ઍડરનો ઉપયોગ કરીને 4 બીટ પેરેલલ ઍડરનો લૉજિક ડાયાગ્રામ દોરો અને તેનુોં
કાયસ સમજાવો}

\begin{solutionbox}

\textbf{ફુલ એડરનો ઉપયોગ કરીને 4-બિટ પેરેલલ એડર:}

\begin{verbatim}
A3    B3      A2    B2      A1    B1      A0    B0
 |     |       |     |       |     |       |     |
 v     v       v     v       v     v       v     v
|---------|  |---------|  |---------|  |---------|
|         |  |         |  |         |  |         |
|   FA    |  |   FA    |  |   FA    |  |   FA    |
|         |  |         |  |         |  |         |
|---------|  |---------|  |---------|  |---------|
     |            |            |            |
     v            v            v            v
    S3           S2           S1           S0
     
     ^            ^            ^            ^
     |            |            |            |
C_out         C3           C2           C1      C_in=0
\end{verbatim}

\textbf{ઓપરેશન:}

\begin{enumerate}
\tightlist
\item
  દરેક ફુલ એડર (FA) તત્સ્થાની બિટ્સ (Ai, Bi) તેમજ અગાઉના સ્ટેજમાંથી કેરી ઉમેરે છે
\item
  સમ (Si) અને કેરી (Ci+1) આગળના સ્ટેજ માટે ઉત્પન્ન કરે છે
\item
  પ્રથમ FA નું C\_in 0 છે (અથવા 1 ઉમેરવા માટે 1 હોઈ શકે છે)
\item
  છેલ્લા FA નું C\_out ઓવરફ્લો સૂચવે છે
\end{enumerate}

\textbf{ઉદાહરણ સરવાળો: 1101 + 1011}

\begin{itemize}
\item
  A_{3}A_{2}A_{1}A_{0} = 1101
\item
  B_{3}B_{2}B_{1}B_{0} = 1011
\item
  C\_in = 0
\item
  S_{3}S_{2}S_{1}S_{0} = 1000
\item
  C\_out = 1 (ઓવરફ્લો સૂચવે છે, વાસ્તવિક પરિણામ 11000 છે)
\item
  \textbf{પેરેલલ એડર}: એક સાથે ઘણી બિટ્સ ઉમેરે છે
\item
  \textbf{કેરી પ્રોપેગેશન}: સ્પીડ માટે મુખ્ય મર્યાદિત પરિબળ
\item
  \textbf{એડર એપ્લિકેશન્સ}: ALU, એડ્રેસ ગણતરી
\end{itemize}

\end{solutionbox}
\begin{mnemonicbox}
``કેરી જમણેથી ડાબે તરફ વહે છે''

\end{mnemonicbox}
\subsection*{પ્રશ્ન 4(અ) [3
માર્ક્સ]}\label{uxaaauxab0uxab6uxaa8-4uxa85-3-uxaaeuxab0uxa95uxab8}

\textbf{BCD કાઉન્ટર નો ડાયાગ્રામ દોરો.}

\begin{solutionbox}

\textbf{BCD કાઉન્ટર ડાયાગ્રામ:}

\begin{verbatim}
       |{-{-}{-}{-}{-}{-}|    |{-}{-}{-}{-}{-}{-}|    |{-}{-}{-}{-}{-}{-}|    |{-}{-}{-}{-}{-}{-}|}
       |      |    |      |    |      |    |      |
CLK{-{-}{-}| JK{-}FF|{-}{-}{-}| JK{-}FF|{-}{-}{-}| JK{-}FF|{-}{-}{-}| JK{-}FF|}
       |  Q0  |    |  Q1  |    |  Q2  |    |  Q3  |
       |{-{-}{-}{-}{-}{-}|    |{-}{-}{-}{-}{-}{-}|    |{-}{-}{-}{-}{-}{-}|    |{-}{-}{-}{-}{-}{-}|}
          |           |           |           |
          v           v           v           v
         Q0          Q1          Q2          Q3
          |           |           |           |
          |{-{-}{-}{-}{-}{-}{-}{-}{-}{-}{-}|{-}{-}{-}{-}{-}{-}{-}{-}{-}{-}{-}|{-}{-}{-}{-}{-}{-}{-}{-}{-}{-}{-}|}
                            |
                            v
                        |{-{-}{-}{-}{-}{-}|}
                        | NAND |{-{-}{-}{-}+}
                        |{-{-}{-}{-}{-}{-}|    |}
                                    |
                                    v
                                  RESET
\end{verbatim}

\textbf{કાઉન્ટર સિક્વન્સ:}

{\def\LTcaptype{none} % do not increment counter
\begin{longtable}[]{@{}lllll@{}}
\toprule\noalign{}
કાઉન્ટ & Q3 & Q2 & Q1 & Q0 \\
\midrule\noalign{}
\endhead
\bottomrule\noalign{}
\endlastfoot
0 & 0 & 0 & 0 & 0 \\
1 & 0 & 0 & 0 & 1 \\
2 & 0 & 0 & 1 & 0 \\
3 & 0 & 0 & 1 & 1 \\
4 & 0 & 1 & 0 & 0 \\
5 & 0 & 1 & 0 & 1 \\
6 & 0 & 1 & 1 & 0 \\
7 & 0 & 1 & 1 & 1 \\
8 & 1 & 0 & 0 & 0 \\
9 & 1 & 0 & 0 & 1 \\
0 & 0 & 0 & 0 & 0 \\
\end{longtable}
}

\begin{itemize}
\tightlist
\item
  \textbf{BCD કાઉન્ટર}: 0 થી 9 સુધી ગણે છે, પછી રીસેટ થાય છે
\item
  \textbf{રીસેટ મેકેનિઝમ}: 10 (1010) ની ગણતરીને શોધે છે અને 0 પર રીસેટ કરે છે
\item
  \textbf{અનુપ્રયોગો}: ડિજિટલ ઘડિયાળ, ફ્રિક્વન્સી કાઉન્ટર્સ
\end{itemize}

\end{solutionbox}
\begin{mnemonicbox}
``માત્ર દશાંશ અંકો (0-9) ગણે છે''

\end{mnemonicbox}
\subsection*{પ્રશ્ન 4(બ) [4
માર્ક્સ]}\label{uxaaauxab0uxab6uxaa8-4uxaac-4-uxaaeuxab0uxa95uxab8}

\textbf{T જલલપ લલોપનો ડાયાગ્રામ દોરો અને ટુથ ટેબલ સાથે તેનુોં કાયસ સમજાવો}

\begin{solutionbox}

\textbf{T ફ્લિપ-ફ્લોપ ડાયાગ્રામ:}

\begin{verbatim}
       |{-{-}{-}{-}{-}{-}|}
       |      |
T{-{-}{-}{-}{-}|      |}
       |  T   |
CLK{-{-}{-}|  FF  |{-}{-}{-}{-}{-} Q}
       |      |
       |      |{-{-}{-}{-}{-} Q}
       |{-{-}{-}{-}{-}{-}|}
\end{verbatim}

\textbf{JK ફ્લિપ-ફ્લોપનો ઉપયોગ કરીને અમલીકરણ:}

\begin{verbatim}
       |{-{-}{-}{-}{-}{-}|}
       |      |
T{-{-}{-}{-}{-}| J    |}
       |      |
       | JK   |{-{-}{-}{-}{-} Q}
CLK{-{-}{-}|      |}
       | FF   |{-{-}{-}{-}{-} Q}
       |      |
T{-{-}{-}{-}{-}| K    |}
       |{-{-}{-}{-}{-}{-}|}
\end{verbatim}

\textbf{ટ્રુથ ટેબલ:}

{\def\LTcaptype{none} % do not increment counter
\begin{longtable}[]{@{}lll@{}}
\toprule\noalign{}
T & CLK & Q(next) \\
\midrule\noalign{}
\endhead
\bottomrule\noalign{}
\endlastfoot
0 & ↑ & Q \\
1 & ↑ & Q' \\
\end{longtable}
}

\begin{itemize}
\tightlist
\item
  \textbf{T=0}: આઉટપુટમાં કોઈ ફેરફાર નહીં (હોલ્ડ)
\item
  \textbf{T=1}: આઉટપુટ ટોગલ થાય છે (કોમ્પ્લિમેન્ટ)
\item
  \textbf{ટોગલ ઓપરેશન}: T=1 હોય ત્યારે દરેક ક્લોક પલ્સ પર સ્થિતિ બદલે છે
\end{itemize}

\end{solutionbox}
\begin{mnemonicbox}
``T એટલે ટોગલ, 0 રાખે છે 1 પલટાવે છે''

\end{mnemonicbox}
\subsection*{પ્રશ્ન 4(ક) [7
માર્ક્સ]}\label{uxaaauxab0uxab6uxaa8-4uxa95-7-uxaaeuxab0uxa95uxab8}

\textbf{જશલટ રજી્ટર શુોં છે? જવજવધ પ્ર કારના જશલટ રજી્ટરની યાદી આપે છે. કોઈપણ એક
પ્ર કારના જશલટ રજી્ટરની કામગીરી તેની લોજીક સકીટ બનાવીને સમજાવો.}

\begin{solutionbox}

\textbf{શિફ્ટ રજિસ્ટર વ્યાખ્યા:} શિફ્ટ રજિસ્ટર એ એક સિક્વેન્શિયલ લોજિક સર્કિટ છે જે
બાઇનરી ડેટા સ્ટોર કરે છે અને શિફ્ટ કરે છે. તેમાં એક શ્રેણીબદ્ધ ફ્લિપ-ફ્લોપ્સ હોય છે જ્યાં
એક ફ્લિપ-ફ્લોપનો આઉટપુટ પછીના ફ્લિપ-ફ્લોપનો ઇનપુટ બને છે.

\textbf{શિફ્ટ રજિસ્ટરના પ્રકારો:}

{\def\LTcaptype{none} % do not increment counter
\begin{longtable}[]{@{}ll@{}}
\toprule\noalign{}
પ્રકાર & વર્ણન \\
\midrule\noalign{}
\endhead
\bottomrule\noalign{}
\endlastfoot
SISO & સીરિયલ ઇનપુટ સીરિયલ આઉટપુટ \\
SIPO & સીરિયલ ઇનપુટ પેરેલલ આઉટપુટ \\
PISO & પેરેલલ ઇનપુટ સીરિયલ આઉટપુટ \\
PIPO & પેરેલલ ઇનપુટ પેરેલલ આઉટપુટ \\
બિડાયરેક્શનલ & કોઈપણ દિશામાં શિફ્ટ કરી શકે છે \\
રિંગ કાઉન્ટર & છેલ્લા સ્ટેજનો આઉટપુટ પ્રથમ સ્ટેજને ફીડ કરાય છે \\
જોન્સન કાઉન્ટર & છેલ્લા સ્ટેજનું કોમ્પ્લિમેન્ટ પ્રથમ સ્ટેજને ફીડ કરાય છે \\
\end{longtable}
}

\textbf{સીરિયલ-ઇન સીરિયલ-આઉટ (SISO) શિફ્ટ રજિસ્ટર:}

\begin{verbatim}
      |{-{-}{-}{-}{-}{-}|    |{-}{-}{-}{-}{-}{-}|    |{-}{-}{-}{-}{-}{-}|    |{-}{-}{-}{-}{-}{-}|}
      |      |    |      |    |      |    |      |
IN{-{-}{-}| D FF |{-}{-}{-}| D FF |{-}{-}{-}| D FF |{-}{-}{-}| D FF |{-}{-}{-} OUT}
      |      |    |      |    |      |    |      |
      |{-{-}{-}{-}{-}{-}|    |{-}{-}{-}{-}{-}{-}|    |{-}{-}{-}{-}{-}{-}|    |{-}{-}{-}{-}{-}{-}|}
         \^{          \^{}          \^{}          \^{}}
         |          |          |          |
         |{-{-}{-}{-}{-}{-}{-}{-}{-}{-}|{-}{-}{-}{-}{-}{-}{-}{-}{-}{-}|{-}{-}{-}{-}{-}{-}{-}{-}{-}{-}|}
                       CLK
\end{verbatim}

\textbf{ઓપરેશન:}

\begin{enumerate}
\tightlist
\item
  ડેટા સીરિયલમાં બિટ દર બિટ ઇનપુટ મારફતે દાખલ થાય છે
\item
  દરેક ક્લોક પલ્સ સાથે, ડેટા એક સ્થાન જમણી તરફ શિફ્ટ થાય છે
\item
  4 ક્લોક પલ્સ પછી, પ્રથમ ઇનપુટ બિટ આઉટપુટ પર દેખાય છે
\item
  ઉદાહરણ: ``1101'' ઇનપુટ માટે, સંપૂર્ણ ટ્રાન્સમિશન માટે 4 ક્લોક પલ્સની જરૂર પડે છે
\end{enumerate}

\begin{itemize}
\tightlist
\item
  \textbf{મુખ્ય ઉપયોગ}: સીરિયલ અને પેરેલલ ફોર્મેટ વચ્ચે ડેટા રૂપાંતરણ
\item
  \textbf{અનુપ્રયોગો}: કોમ્યુનિકેશન સિસ્ટમ્સ, ઉપકરણો વચ્ચે ડેટા ટ્રાન્સફર
\item
  \textbf{ફાયદાઓ}: સરળ ડિઝાઇન, ન્યૂનતમ ઇન્ટરકનેક્શન્સ
\end{itemize}

\end{solutionbox}
\begin{mnemonicbox}
``શિફ્ટ રજિસ્ટર બકેટ બ્રિગેડની જેમ બિટ્સ પસાર કરે છે''

\end{mnemonicbox}
\subsection*{પ્રશ્ન 4(અ-OR) [3
માર્ક્સ]}\label{uxaaauxab0uxab6uxaa8-4uxa85-or-3-uxaaeuxab0uxa95uxab8}

\textbf{4:2 એોંકોડર દોરો અને સમજાવો.}

\begin{solutionbox}

\textbf{4:2 એન્કોડર ડાયાગ્રામ:}

\begin{verbatim}
      |{-{-}{-}{-}{-}{-}|}
D0{-{-}{-}|      |}
      |      |{-{-}{-} A}
D1{-{-}{-}|      |}
      | 4:2  |
D2{-{-}{-}|      |}
      |      |{-{-}{-} B}
D3{-{-}{-}|      |}
      |{-{-}{-}{-}{-}{-}|}
\end{verbatim}

\textbf{ટ્રુથ ટેબલ:}

{\def\LTcaptype{none} % do not increment counter
\begin{longtable}[]{@{}llllll@{}}
\toprule\noalign{}
D3 & D2 & D1 & D0 & B & A \\
\midrule\noalign{}
\endhead
\bottomrule\noalign{}
\endlastfoot
0 & 0 & 0 & 1 & 0 & 0 \\
0 & 0 & 1 & 0 & 0 & 1 \\
0 & 1 & 0 & 0 & 1 & 0 \\
1 & 0 & 0 & 0 & 1 & 1 \\
\end{longtable}
}

\textbf{લોજિકલ એક્સપ્રેશન્સ:}

\begin{itemize}
\item
  A = D1 + D3
\item
  B = D2 + D3
\item
  \textbf{એન્કોડર ફંક્શન}: વન-હોટ ઇનપુટને બાઇનરી કોડમાં રૂપાંતરિત કરે છે
\item
  \textbf{પ્રાયોરિટી એન્કોડર્સ}: પ્રાયોરિટી દ્વારા ઘણા સક્રિય ઇનપુટ્સને હેન્ડલ કરે છે
\item
  \textbf{અનુપ્રયોગો}: કીબોર્ડ સ્કેનિંગ, ઇન્ટરપ્ટ હેન્ડલિંગ
\end{itemize}

\end{solutionbox}
\begin{mnemonicbox}
``એક સક્રિય લાઇન અંદર, બાઇનરી કોડ બહાર''

\end{mnemonicbox}
\subsection*{પ્રશ્ન 4(બ-OR) [4
માર્ક્સ]}\label{uxaaauxab0uxab6uxaa8-4uxaac-or-4-uxaaeuxab0uxa95uxab8}

\textbf{િોહ્નન્સન કાઉન્ટર દોરો અને સમજાવો.}

\begin{solutionbox}

\textbf{જોન્સન કાઉન્ટર (4-બિટ):}

\begin{verbatim}
    |{-{-}{-}{-}{-}{-}|    |{-}{-}{-}{-}{-}{-}|    |{-}{-}{-}{-}{-}{-}|    |{-}{-}{-}{-}{-}{-}|}
    |      |    |      |    |      |    |      |
    | D FF |{-{-}{-}{-}| D FF |{-}{-}{-}{-}| D FF |{-}{-}{-}{-}| D FF |}
    |      |    |      |    |      |    |      |
    |{-{-}{-}{-}{-}{-}|    |{-}{-}{-}{-}{-}{-}|    |{-}{-}{-}{-}{-}{-}|    |{-}{-}{-}{-}{-}{-}|}
       \^{          \^{}          \^{}          \^{}}
       |          |          |          |
       |{-{-}{-}{-}{-}{-}{-}{-}{-}{-}|{-}{-}{-}{-}{-}{-}{-}{-}{-}{-}|{-}{-}{-}{-}{-}{-}{-}{-}{-}{-}|}
                    CLK
               |
        Q3{    |}
         |     |
         |     v
         {-{-}{-}{-}{-}|}
\end{verbatim}

\textbf{કાઉન્ટર સિક્વન્સ:}

{\def\LTcaptype{none} % do not increment counter
\begin{longtable}[]{@{}lllll@{}}
\toprule\noalign{}
કાઉન્ટ & Q3 & Q2 & Q1 & Q0 \\
\midrule\noalign{}
\endhead
\bottomrule\noalign{}
\endlastfoot
0 & 0 & 0 & 0 & 0 \\
1 & 1 & 0 & 0 & 0 \\
2 & 1 & 1 & 0 & 0 \\
3 & 1 & 1 & 1 & 0 \\
4 & 1 & 1 & 1 & 1 \\
5 & 0 & 1 & 1 & 1 \\
6 & 0 & 0 & 1 & 1 \\
7 & 0 & 0 & 0 & 1 \\
0 & 0 & 0 & 0 & 0 \\
\end{longtable}
}

\begin{itemize}
\tightlist
\item
  \textbf{જોન્સન કાઉન્ટર}: ટ્વિસ્ટેડ રિંગ કાઉન્ટર તરીકે પણ ઓળખાય છે
\item
  \textbf{સિક્વન્સ લંબાઈ}: 2n સ્ટેટ્સ જ્યાં n ફ્લિપ-ફ્લોપ્સની સંખ્યા છે
\item
  \textbf{મુખ્ય વિશેષતા}: આસન્ન સ્ટેટ્સ વચ્ચે ફક્ત એક બિટ બદલાય છે
\end{itemize}

\end{solutionbox}
\begin{mnemonicbox}
``1 થી ભરો પછી 0 થી સાફ કરો''

\end{mnemonicbox}
\subsection*{પ્રશ્ન 4(ક-OR) [7
માર્ક્સ]}\label{uxaaauxab0uxab6uxaa8-4uxa95-or-7-uxaaeuxab0uxa95uxab8}

\textbf{૪ બીટ જરપલ કાઉન્ટર દોરો અને સમજાવો.}

\begin{solutionbox}

\textbf{4-બિટ રિપલ કાઉન્ટર:}

\begin{verbatim}
         CLK
          |
          v
       |{-{-}{-}{-}{-}{-}|    |{-}{-}{-}{-}{-}{-}|    |{-}{-}{-}{-}{-}{-}|    |{-}{-}{-}{-}{-}{-}|}
       |      |    |      |    |      |    |      |
{-{-}{-}{-}{-}{-}| T FF |    | T FF |    | T FF |    | T FF |}
       |      |    |      |    |      |    |      |
       |{-{-}{-}{-}{-}{-}|    |{-}{-}{-}{-}{-}{-}|    |{-}{-}{-}{-}{-}{-}|    |{-}{-}{-}{-}{-}{-}|}
          |           |           |           |
          |           |           |           |
         Q0 {-{-}{-}{-}{-}{-}{-}{-}{-}Q1 {-}{-}{-}{-}{-}{-}{-}{-}Q2 {-}{-}{-}{-}{-}{-}{-}{-}Q3}
       (LSB)                                 (MSB)
\end{verbatim}

\textbf{ટ્રુથ ટેબલ (કાઉન્ટિંગ સિક્વન્સ):}

{\def\LTcaptype{none} % do not increment counter
\begin{longtable}[]{@{}lllll@{}}
\toprule\noalign{}
કાઉન્ટ & Q3 & Q2 & Q1 & Q0 \\
\midrule\noalign{}
\endhead
\bottomrule\noalign{}
\endlastfoot
0 & 0 & 0 & 0 & 0 \\
1 & 0 & 0 & 0 & 1 \\
2 & 0 & 0 & 1 & 0 \\
3 & 0 & 0 & 1 & 1 \\
\ldots{} & .. & .. & .. & .. \\
14 & 1 & 1 & 1 & 0 \\
15 & 1 & 1 & 1 & 1 \\
0 & 0 & 0 & 0 & 0 \\
\end{longtable}
}

\textbf{કાર્ય સિદ્ધાંત:}

\begin{enumerate}
\tightlist
\item
  બધા T ઇનપુટ્સ લોજિક 1 સાથે જોડાયેલા છે (ટોગલ મોડ)
\item
  પ્રથમ ફ્લિપ-ફ્લોપ દરેક ક્લોક પલ્સ પર ટોગલ થાય છે
\item
  દરેક પછીનું ફ્લિપ-ફ્લોપ ત્યારે ટોગલ થાય છે જ્યારે અગાઉનું 1 થી 0 માં બદલાય છે
\item
  દરેક સ્ટેજ સાથે પ્રોપેગેશન ડિલે વધે છે
\end{enumerate}

\begin{itemize}
\tightlist
\item
  \textbf{અસિંક્રોનસ કાઉન્ટર}: ક્લોક ફક્ત પ્રથમ ફ્લિપ-ફ્લોપને ડ્રાઇવ કરે છે
\item
  \textbf{રિપલ ઇફેક્ટ}: ફેરફારો સ્ટેજમાંથી પસાર થાય છે
\item
  \textbf{ગેરલાભ}: સંચિત પ્રોપેગેશન ડિલેને કારણે ધીમું
\end{itemize}

\end{solutionbox}
\begin{mnemonicbox}
``પડતા ડોમિનોની જેમ ફેરફાર ફેલાય છે''

\end{mnemonicbox}
\subsection*{પ્રશ્ન 5(અ) [3
માર્ક્સ]}\label{uxaaauxab0uxab6uxaa8-5uxa85-3-uxaaeuxab0uxa95uxab8}

\textbf{ટ ોંકમાોં DRAM સમજાવો.}

\begin{solutionbox}

\textbf{ડાયનેમિક રેન્ડમ એક્સેસ મેમરી (DRAM):}

DRAM એક પ્રકારની સેમિકન્ડક્ટર મેમરી છે જે દરેક બિટને અલગ કેપેસિટરમાં સ્ટોર કરે છે.

\textbf{મુખ્ય વિશેષતાઓ:}

{\def\LTcaptype{none} % do not increment counter
\begin{longtable}[]{@{}ll@{}}
\toprule\noalign{}
વિશેષતા & વર્ણન \\
\midrule\noalign{}
\endhead
\bottomrule\noalign{}
\endlastfoot
સ્ટોરેજ એલિમેન્ટ & દરેક બિટ દીઠ સિંગલ કેપેસિટર + ટ્રાન્ઝિસ્ટર \\
ડેન્સિટી & ખૂબ ઊંચી (ચિપ દીઠ વધુ બિટ્સ) \\
સ્પીડ & મધ્યમ (SRAM કરતાં ધીમી) \\
રિફ્રેશ & સમયાંતરે જરૂરી (સામાન્ય રીતે દર થોડી મિલિસેકન્ડ) \\
પાવર વપરાશ & SRAM કરતાં ઓછો \\
કિંમત & SRAM કરતાં ઓછી ખર્ચાળ \\
\end{longtable}
}

\begin{itemize}
\tightlist
\item
  \textbf{ડાયનેમિક પ્રકૃતિ}: ચાર્જ સમય સાથે લીક થાય છે, રિફ્રેશની જરૂર પડે છે
\item
  \textbf{અનુપ્રયોગો}: કમ્પ્યુટરમાં મુખ્ય મેમરી
\item
  \textbf{ફાયદો}: ઉચ્ચ ડેન્સિટી, બિટ દીઠ ઓછી કિંમત
\end{itemize}

\end{solutionbox}
\begin{mnemonicbox}
``DRAM ને થાકેલા મન જેવી તાજગીની જરૂર પડે છે''

\end{mnemonicbox}
\subsection*{પ્રશ્ન 5(બ) [4
માર્ક્સ]}\label{uxaaauxab0uxab6uxaa8-5uxaac-4-uxaaeuxab0uxa95uxab8}

\textbf{નીચેની વ્ યાખ્યા આપો (1)ફેન ઇન (2) પ્ર પોગેશન ડીલે}

\begin{solutionbox}

\textbf{ફેન-ઇન:}

ફેન-ઇન એ લોજિક ગેટ સ્વીકારી શકે તેવા ઇનપુટની મહત્તમ સંખ્યા છે.

\textbf{ફેન-ઇનની વિશેષતાઓ:}

\begin{itemize}
\tightlist
\item
  ઇનપુટ લોડ ક્ષમતા માપે છે
\item
  સર્કિટ જટિલતા અને ડિઝાઇનને અસર કરે છે
\item
  ઉચ્ચ ફેન-ઇન ગેટની સંખ્યા ઘટાડે છે પરંતુ જટિલતા વધારે છે
\item
  વિવિધ લોજિક ફેમિલીઓની વિવિધ ફેન-ઇન મર્યાદાઓ છે
\end{itemize}

\textbf{ઉદાહરણ:} એક સ્ટાન્ડર્ડ TTL NAND ગેટમાં સામાન્ય રીતે 8 ઇનપુટનો ફેન-ઇન હોય
છે.

\textbf{પ્રોપેગેશન ડિલે:}

પ્રોપેગેશન ડિલે એ લોજિક ગેટના ઇનપુટથી આઉટપુટ સુધી સિગ્નલ પહોંચવામાં લાગતો સમય છે.

\textbf{પ્રોપેગેશન ડિલેની વિશેષતાઓ:}

\begin{itemize}
\tightlist
\item
  નેનોસેકન્ડ (ns)માં માપવામાં આવે છે
\item
  હાઇ-સ્પીડ સર્કિટ પરફોર્મન્સ માટે મહત્વપૂર્ણ
\item
  તાપમાન, લોડિંગ અને સપ્લાય વોલ્ટેજ સાથે બદલાય છે
\item
  રાઇઝિંગ અને ફોલિંગ ટ્રાન્ઝિશન માટે અલગ છે
\end{itemize}

\textbf{ઉદાહરણ:} એક સામાન્ય TTL ગેટમાં 10-20 ns પ્રોપેગેશન ડિલે હોય છે.

\begin{itemize}
\tightlist
\item
  \textbf{સર્કિટ પર અસર}: મહત્તમ ઓપરેટિંગ ફ્રિક્વન્સી મર્યાદિત કરે છે
\item
  \textbf{ગણતરી}: ઇનપુટ અને આઉટપુટ સિગ્નલના 50\% પોઇન્ટ વચ્ચેનો સમય
\end{itemize}

\end{solutionbox}
\begin{mnemonicbox}
``ફેન-ઇન ઇનપુટ ગણે છે, પ્રોપ-ડિલે સમય ગણે છે''

\end{mnemonicbox}
\subsection*{પ્રશ્ન 5(ક) [7
માર્ક્સ]}\label{uxaaauxab0uxab6uxaa8-5uxa95-7-uxaaeuxab0uxa95uxab8}

\textbf{જનદેશ મુિબ કરો (i) લોજિક ફેમીલી TTL અને CMOS ની સરખામણી કરો.(ii) SR
નો સકીટ ડાયાગ્રામ દોરો.}

\begin{solutionbox}

\textbf{(i) TTL અને CMOS લોજિક ફેમિલીની સરખામણી:}

{\def\LTcaptype{none} % do not increment counter
\begin{longtable}[]{@{}lll@{}}
\toprule\noalign{}
પેરામીટર & TTL & CMOS \\
\midrule\noalign{}
\endhead
\bottomrule\noalign{}
\endlastfoot
ટેકનોલોજી & બાયપોલર ટ્રાન્ઝિસ્ટર્સ & MOSFETs \\
સપ્લાય વોલ્ટેજ & 5V (ફિક્સ્ડ) & 3-15V (ફ્લેક્સિબલ) \\
પાવર વપરાશ & ઉચ્ચ & ખૂબ નીચો (સ્ટેટિક) \\
સ્પીડ & મધ્યમથી ઉચ્ચ & નીચેથી ખૂબ ઉચ્ચ \\
નોઇઝ માર્જિન & મધ્યમ & ઉચ્ચ \\
ફેન-આઉટ & 10-20 & \textgreater50 \\
પ્રોપેગેશન ડિલે & 5-10 ns & 10-100 ns (સ્ટાન્ડર્ડ) \\
ઇનપુટ ઇમ્પિડન્સ & 4-40 kΩ & ખૂબ ઉચ્ચ (10^{1}^{2} Ω) \\
આઉટપુટ ઇમ્પિડન્સ & 100-300 Ω & ચલ \\
સ્ટેટિક પ્રત્યે સંવેદનશીલતા & નીચી & ઉચ્ચ \\
\end{longtable}
}

\textbf{(ii) SR ફ્લિપ-ફ્લોપ સર્કિટ ડાયાગ્રામ:}

\begin{verbatim}
             |{-{-}{-}{-}{-}{-}|}
             |      |
Set {-{-}{-}{-}{-}{-}{-}{-}|      |{-}{-}{-}{-}{-} Q}
             | NOR  |
             |      |
             |{-{-}{-}{-}{-}{-}|}
                \^{}
                |
                |    |{-{-}{-}{-}{-}{-}|}
                |    |      |
                |{-{-}{-}{-}|      |{-}{-}{-}{-}{-} Q}
                     | NOR  |
                     |      |
Reset {-{-}{-}{-}{-}{-}{-}{-}{-}{-}{-}{-}{-}{-}|      |}
                     |{-{-}{-}{-}{-}{-}|}
\end{verbatim}

\textbf{ટ્રુથ ટેબલ:}

{\def\LTcaptype{none} % do not increment counter
\begin{longtable}[]{@{}lllll@{}}
\toprule\noalign{}
S & R & Q & Q' & રિમાર્ક્સ \\
\midrule\noalign{}
\endhead
\bottomrule\noalign{}
\endlastfoot
0 & 0 & Q & Q' & મેમરી (કોઈ ફેરફાર નહીં) \\
0 & 1 & 0 & 1 & રીસેટ \\
1 & 0 & 1 & 0 & સેટ \\
1 & 1 & 0 & 0 & અમાન્ય (ટાળવું) \\
\end{longtable}
}

\begin{itemize}
\tightlist
\item
  \textbf{SR ફ્લિપ-ફ્લોપ}: ડિજિટલ સર્કિટમાં મૂળભૂત મેમરી એલિમેન્ટ
\item
  \textbf{ઓપરેશન}: સેટ (S=1, R=0) Q=1 બનાવે છે; રીસેટ (S=0, R=1) Q=0 બનાવે છે
\item
  \textbf{મેમરી સ્ટેટ}: જ્યારે S=0, R=0, આઉટપુટ અપરિવર્તિત રહે છે
\end{itemize}

\end{solutionbox}
\begin{mnemonicbox}
``SR: સેટ-રીસેટ, બંને નીચા હોય ત્યારે મેમરી''

\end{mnemonicbox}
\subsection*{પ્રશ્ન 5(અ-OR) [3
માર્ક્સ]}\label{uxaaauxab0uxab6uxaa8-5uxa85-or-3-uxaaeuxab0uxa95uxab8}

\textbf{જડજિટલ જચ્સના E વે્ટ પર ટ ોંકી નોોંધ લખો.}

\begin{solutionbox}

\textbf{ડિજિટલ ચિપ્સનો E-વેસ્ટ:}

ડિજિટલ ચિપ્સનો E-વેસ્ટ એ ત્યજી દેવાયેલા ઇલેક્ટ્રોનિક ઉપકરણોનો ઉલ્લેખ કરે છે જેમાં
સેમિકન્ડક્ટર કોમ્પોનન્ટ્સ હોય છે જે ખાસ હેન્ડલિંગ અને નિકાલની જરૂર હોય છે.

\textbf{મુખ્ય ચિંતાઓ:}

{\def\LTcaptype{none} % do not increment counter
\begin{longtable}[]{@{}ll@{}}
\toprule\noalign{}
પાસું & વિગતો \\
\midrule\noalign{}
\endhead
\bottomrule\noalign{}
\endlastfoot
જોખમી સામગ્રી & લેડ, મર્ક્યુરી, કેડમિયમ, બ્રોમિનેટેડ ફ્લેમ રિટાર્ડન્ટ \\
પર્યાવરણીય અસર & યોગ્ય રીતે ફેંકવામાં ન આવે તો માટી અને પાણીનું પ્રદૂષણ \\
સંસાધન પુનઃપ્રાપ્તિ & કિંમતી ધાતુઓ ધરાવે છે (સોનું, ચાંદી, તાંબું) \\
જથ્થો & ટેકનોલોજિકલ પ્રગતિ સાથે ઝડપથી વધી રહ્યો છે \\
નિયમો & ઘણા દેશોમાં WEEE, RoHS દિશાનિર્દેશો દ્વારા સંચાલિત \\
\end{longtable}
}

\textbf{મેનેજમેન્ટ અભિગમો:}

\begin{itemize}
\item
  અધિકૃત ઇ-કચરા હેન્ડલર્સ દ્વારા રિસાયકલિંગ
\item
  કિંમતી ધાતુઓની પુનઃપ્રાપ્તિ
\item
  જોખમી ઘટકોનો સુરક્ષિત નિકાલ
\item
  વિસ્તારિત ઉત્પાદક જવાબદારી કાર્યક્રમો
\item
  \textbf{પડકારો}: અનૌપચારિક રિસાયકલિંગ આરોગ્ય જોખમો પેદા કરી રહ્યું છે
\item
  \textbf{ઉકેલો}: ડિસએસેમ્બલી માટે ડિઝાઇન, ગ્રીન મેન્યુફેક્ચરિંગ
\end{itemize}

\end{solutionbox}
\begin{mnemonicbox}
``ડિજિટલ કચરાને ડિજિટલ-યુગના ઉકેલોની જરૂર છે''

\end{mnemonicbox}
\subsection*{પ્રશ્ન 5(બ-OR) [4
માર્ક્સ]}\label{uxaaauxab0uxab6uxaa8-5uxaac-or-4-uxaaeuxab0uxa95uxab8}

\textbf{નીચેની વ્ યાખ્યા આપો (1) ફેન આઉટ (2)નોઈઝ માઝીન}

\begin{solutionbox}

\textbf{ફેન-આઉટ:}

ફેન-આઉટ એ એક લોજિક ગેટ આઉટપુટ દ્વારા ડ્રાઇવ કરી શકાતા ગેટ ઇનપુટની મહત્તમ સંખ્યા છે
જે યોગ્ય લોજિક લેવલ જાળવી રાખે છે.

\textbf{ફેન-આઉટની વિશેષતાઓ:}

\begin{itemize}
\tightlist
\item
  આઉટપુટ ડ્રાઇવ ક્ષમતા માપે છે
\item
  ડિઝાઇન ફ્લેક્સિબિલિટી અને કિંમતને અસર કરે છે
\item
  ઉચ્ચ ફેન-આઉટ સરળ વાયરિંગ માટે પરવાનગી આપે છે
\item
  કરંટ સોર્સિંગ/સિંકિંગ ક્ષમતા દ્વારા મર્યાદિત
\end{itemize}

\textbf{ઉદાહરણ:} એક સ્ટાન્ડર્ડ TTL ગેટમાં 10નો ફેન-આઉટ હોય છે, એટલે કે તે 10 સમાન
ગેટ્સને ડ્રાઇવ કરી શકે છે.

\textbf{નોઇઝ માર્જિન:}

નોઇઝ માર્જિન એ નોઇઝ વોલ્ટેજની માત્રા છે જે ઇનપુટ સિગ્નલમાં ઉમેરી શકાય છે જેથી સર્કિટ
આઉટપુટમાં અનિચ્છનીય ફેરફાર થવા ન પામે.

\textbf{નોઇઝ માર્જિનની વિશેષતાઓ:}

\begin{itemize}
\tightlist
\item
  વોલ્ટ્સમાં વ્યક્ત
\item
  ઇલેક્ટ્રિકલ નોઇઝ સામે સર્કિટ ઇમ્યુનિટી માપે છે
\item
  ઉચ્ચ નોઇઝ માર્જિનનો અર્થ વધુ વિશ્વસનીય ઓપરેશન
\item
  હાઇ અને લો લોજિક લેવલ માટે અલગ
\end{itemize}

\textbf{ઉદાહરણ:} TTLમાં લોજિક લો માટે આશરે 0.4V અને લોજિક હાઇ માટે 0.7V નોઇઝ
માર્જિન હોય છે.

\begin{itemize}
\tightlist
\item
  \textbf{ગણતરી}: ગેરંટેડ આઉટપુટ અને જરૂરી ઇનપુટ લેવલ વચ્ચેનો તફાવત
\item
  \textbf{મહત્વ}: ઇલેક્ટ્રિકલી નોઇઝી વાતાવરણમાં મહત્વપૂર્ણ
\end{itemize}

\end{solutionbox}
\begin{mnemonicbox}
``ફેન-આઉટ આઉટપુટ ગણે છે, નોઇઝ માર્જિન દખલગીરી સામે લડે છે''

\end{mnemonicbox}
\subsection*{પ્રશ્ન 5(ક-OR) [7
માર્ક્સ]}\label{uxaaauxab0uxab6uxaa8-5uxa95-or-7-uxaaeuxab0uxa95uxab8}

\textbf{જનદેશ મુિબ કરો (i) ROM મેમરી ઉપર ટુોંક નોધ લખો ii) મા્ટર ્ લેવ JK જલલપ
લલોપ સમજાવો.}

\begin{solutionbox}

\textbf{(i) ROM પર ટૂંક નોંધ:}

ROM (રીડ-ઓન્લી મેમરી) એક નોન-વોલેટાઇલ મેમરી છે જેનો ઉપયોગ કાયમી અથવા અર્ધ-કાયમી
ડેટા સ્ટોર કરવા માટે થાય છે.

\textbf{ROM ના પ્રકારો:}

{\def\LTcaptype{none} % do not increment counter
\begin{longtable}[]{@{}lll@{}}
\toprule\noalign{}
પ્રકાર & વિશેષતાઓ & પ્રોગ્રામિંગ \\
\midrule\noalign{}
\endhead
\bottomrule\noalign{}
\endlastfoot
માસ્ક ROM & ફેક્ટરી પ્રોગ્રામ્ડ & ઉત્પાદન દરમિયાન \\
PROM & એક-વાર પ્રોગ્રામેબલ & યુઝર દ્વારા ઇલેક્ટ્રિકલ ફ્યુઝિંગ \\
EPROM & UV લાઇટ સાથે ભૂંસી શકાય & ઇલેક્ટ્રિકલ પ્રોગ્રામિંગ \\
EEPROM & ઇલેક્ટ્રિકલી ભૂંસી શકાય & ઇલેક્ટ્રિકલ પ્રોગ્રામિંગ/ભૂંસવું \\
ફ્લેશ ROM & ઝડપી ઇલેક્ટ્રિકલ ભૂંસવું & બ્લોક-વાઇઝ ભૂંસવું/લખવું \\
\end{longtable}
}

\textbf{અનુપ્રયોગો:}

\begin{itemize}
\item
  ફર્મવેર અને BIOS સ્ટોરેજ
\item
  ફિક્સ્ડ ફંક્શન્સ માટે લુક-અપ ટેબલ્સ
\item
  પ્રોસેસરમાં માઇક્રોકોડ
\item
  કમ્પ્યુટરમાં બૂટ કોડ
\item
  \textbf{ડેટા રિટેન્શન}: પાવર વગર ડેટા જાળવી રાખે છે
\item
  \textbf{એક્સેસ ટાઇમ}: સામાન્ય રીતે 45-150 ns
\item
  \textbf{ડેન્સિટી}: ઉચ્ચ સ્ટોરેજ ક્ષમતા
\end{itemize}

\textbf{(ii) JK માસ્ટર-સ્લેવ ફ્લિપ-ફ્લોપ:}

\begin{verbatim}
         |{-{-}{-}{-}{-}{-}{-}{-}{-}{-}{-}|        |{-}{-}{-}{-}{-}{-}{-}{-}{-}{-}{-}|}
         |           |        |           |
J {-{-}{-}{-}{-}{-}|           |        |           |{-}{-}{-}{-}{-} Q}
         |  Master   |{-{-}{-}{-}{-}{-}{-}|   Slave   |}
K {-{-}{-}{-}{-}{-}|           |        |           |{-}{-}{-}{-}{-} Q}
         |           |        |           |
         |{-{-}{-}{-}{-}{-}{-}{-}{-}{-}{-}|        |{-}{-}{-}{-}{-}{-}{-}{-}{-}{-}{-}|}
               \^{                    \^{}}
               |                    |
         CLK {-{-}|                 INV|{-}{-} CLK}
\end{verbatim}

\textbf{ટ્રુથ ટેબલ:}

{\def\LTcaptype{none} % do not increment counter
\begin{longtable}[]{@{}llll@{}}
\toprule\noalign{}
J & K & Q(next) & ફંક્શન \\
\midrule\noalign{}
\endhead
\bottomrule\noalign{}
\endlastfoot
0 & 0 & Q & કોઈ ફેરફાર નહીં \\
0 & 1 & 0 & રીસેટ \\
1 & 0 & 1 & સેટ \\
1 & 1 & Q' & ટોગલ \\
\end{longtable}
}

\textbf{ઓપરેશન:}

\begin{enumerate}
\tightlist
\item
  \textbf{માસ્ટર સ્ટેજ}: જ્યારે CLK=1, માસ્ટર લેચ J અને K ઇનપુટ્સને સેમ્પલ કરે છે
\item
  \textbf{સ્લેવ સ્ટેજ}: જ્યારે CLK=0, સ્લેવ લેચ માસ્ટર આઉટપુટને સેમ્પલ કરે છે
\item
  \textbf{ટુ-ફેઝ ઓપરેશન}: રેસ કન્ડિશન અટકાવે છે (ફેરફારો ફક્ત ક્લોક એજ પર થાય છે)
\item
  \textbf{ફાયદો}: SR ફ્લિપ-ફ્લોપ કરતાં વધુ બહુમુખી (કોઈ અમાન્ય સ્થિતિ નથી)
\end{enumerate}

\begin{itemize}
\tightlist
\item
  \textbf{ટોગલ મોડ}: જ્યારે J=K=1, આઉટપુટ દરેક ક્લોક સાયકલમાં ટોગલ થાય છે
\item
  \textbf{અનુપ્રયોગો}: કાઉન્ટર્સ, શિફ્ટ રજિસ્ટર્સ, સિક્વેન્શિયલ સર્કિટ્સ
\end{itemize}

\end{solutionbox}
\begin{mnemonicbox}
``J-K: સેટ-રીસેટ-ટોગલ, માસ્ટર આગળ ચાલે સ્લેવ અનુસરે છે''

\end{mnemonicbox}

\end{document}
