\documentclass[10pt,a4paper]{article}

% content/resources/templates/preamble.tex
\usepackage[margin=0.6in]{geometry}
\author{Milav Dabgar}
\usepackage{amsmath,amssymb,amsthm}
\usepackage{booktabs}
\usepackage{multirow}
\usepackage{xcolor}
\usepackage{tcolorbox}
\tcbuselibrary{breakable,skins}
\usepackage[colorlinks=true,linkcolor=blue]{hyperref}
\usepackage{titlesec}
\usepackage{enumitem}
\usepackage{tikz}
\usepackage{pgfplots}
\usepackage{circuitikz}
\usepackage[version=4]{mhchem}
\usepackage{longtable}
\usepackage{array}
\usepackage{float}
\usepackage{caption}
\usepackage{listings}

\lstset{
  basicstyle=\small\ttfamily,
  breaklines=true,
  breakatwhitespace=false,
  postbreak=\mbox{\textcolor{red}{$\hookrightarrow$}\space},
  float=false,
  numbers=left,
  numberstyle=\tiny\color{gray},
  numbersep=10pt,
  xleftmargin=2em,
  keywordstyle=\color{blue},
  commentstyle=\color{green!60!black},
  stringstyle=\color{purple},
  backgroundcolor=\color{gray!5},
  showstringspaces=false,
  tabsize=2,
  captionpos=b,
  keepspaces=true,
  columns=flexible
}

\pgfplotsset{compat=1.18}
\usetikzlibrary{shapes,arrows,positioning,calc,patterns,decorations.pathmorphing,decorations.markings,arrows.meta}

% Color scheme
\definecolor{headcolor}{RGB}{0,102,204}
\definecolor{keycolor}{RGB}{220,20,60}
\definecolor{solutioncolor}{RGB}{34,139,34}
\definecolor{mnemoniccolor}{RGB}{148,0,211}
\definecolor{codecolor}{RGB}{0,0,100}

% Spacing
\setlength{\parskip}{3pt}
\setlist[itemize]{nosep}
\setlist[enumerate]{nosep}

% Title formatting
\titleformat{\section}{\Large\bfseries\color{headcolor}}{\thesection}{1em}{}
\titleformat{\subsection}{\large\bfseries\color{headcolor}}{\thesubsection}{1em}{}

% Pandoc tightlist compatibility
\providecommand{\tightlist}{%
  \setlength{\itemsep}{0pt}\setlength{\parskip}{0pt}}

% Pandoc longtable compatibility
\newcounter{none}
\def\thenone{}


% content/resources/templates/english-boxes.tex
% This file is currently empty - it exists to maintain consistency with the import structure.
% Add custom environments here if needed in the future.


\begin{document}

\begin{center}
{\Huge\bfseries\color{headcolor} Subject Name Solutions}\\[5pt]
{\LARGE 4321102 -- Summer 2024}\\[3pt]
{\large Semester 1 Study Material}\\[3pt]
{\normalsize\textit{Detailed Solutions and Explanations}}
\end{center}

\vspace{10pt}

\subsection*{Question 1(a) [3 marks]}\label{q1a}

\textbf{Convert: (110101)_{2} = ( \_\_\_ )_{1}_{0} = ( \_\_\_ )_{8} = ( \_\_\_ )_{1}_{6}}

\begin{solutionbox}

\textbf{Step-by-step conversion of (110101)_{2}}:

{\def\LTcaptype{none} % do not increment counter
\begin{longtable}[]{@{}
  >{\raggedright\arraybackslash}p{(\linewidth - 6\tabcolsep) * \real{0.3830}}
  >{\raggedright\arraybackslash}p{(\linewidth - 6\tabcolsep) * \real{0.1915}}
  >{\raggedright\arraybackslash}p{(\linewidth - 6\tabcolsep) * \real{0.1489}}
  >{\raggedright\arraybackslash}p{(\linewidth - 6\tabcolsep) * \real{0.2766}}@{}}
\toprule\noalign{}
\begin{minipage}[b]{\linewidth}\raggedright
Binary (110101)_{2}
\end{minipage} & \begin{minipage}[b]{\linewidth}\raggedright
Decimal
\end{minipage} & \begin{minipage}[b]{\linewidth}\raggedright
Octal
\end{minipage} & \begin{minipage}[b]{\linewidth}\raggedright
Hexadecimal
\end{minipage} \\
\midrule\noalign{}
\endhead
\bottomrule\noalign{}
\endlastfoot
1\times2^{5} + 1\times2^{4} + 0\times2^{3} + 1\times2^{2} + 0\times2^{1} + 1\times2^{0} & 32+16+0+4+0+1 = 53 & 6\times8^{1} +
5\times8^{0} = 48+5 = 53 & 3\times16^{1} + 5\times16^{0} = 48+5 = 35 \\
(110101)_{2} & (53)_{1}_{0} & (65)_{8} & (35)_{1}_{6} \\
\end{longtable}
}

\end{solutionbox}
\begin{mnemonicbox}
``Binary Digits Out Here'' (BDOH) for
Binary\rightarrowDecimal\rightarrowOctal\rightarrowHexadecimal conversion.

\end{mnemonicbox}
\subsection*{Question 1(b) [4 marks]}\label{q1b}

**Perform: (i) (11101101)_{2}+(10101000)_{2} (ii) (11011)_{2}*(1010)_{2}**

\begin{solutionbox}

\textbf{Table for binary addition and multiplication}:

{\def\LTcaptype{none} % do not increment counter
\begin{longtable}[]{@{}
  >{\raggedright\arraybackslash}p{(\linewidth - 2\tabcolsep) * \real{0.4167}}
  >{\raggedright\arraybackslash}p{(\linewidth - 2\tabcolsep) * \real{0.5833}}@{}}
\toprule\noalign{}
\begin{minipage}[b]{\linewidth}\raggedright
(i) Binary Addition
\end{minipage} & \begin{minipage}[b]{\linewidth}\raggedright
(ii) Binary Multiplication
\end{minipage} \\
\midrule\noalign{}
\endhead
\bottomrule\noalign{}
\endlastfoot
\texttt{11101101\ \ \ \ \ \textbar{}} 11011 & \\
+ 10101000 & \times 1010 \\
---------- & ------- \\
110010101\texttt{\textbar{}\ \ \ \ \ \ \ \ \ 00000\ \ \ \ \ \ \ \ \ \ \ \ \ \textbar{}\ \textbar{}\ \ \ \ \ \ \ \ \ \ \ \ \ \ \ \ \ \ \ \ \textbar{}\ \ \ \ \ \ \ \ 11011\ \ \ \ \ \ \ \ \ \ \ \ \ \ \textbar{}\ \textbar{}\ \ \ \ \ \ \ \ \ \ \ \ \ \ \ \ \ \ \ \ \textbar{}\ \ \ \ \ \ \ 00000\ \ \ \ \ \ \ \ \ \ \ \ \ \ \ \textbar{}\ \textbar{}\ \ \ \ \ \ \ \ \ \ \ \ \ \ \ \ \ \ \ \ \textbar{}\ \ \ \ \ \ 11011\ \ \ \ \ \ \ \ \ \ \ \ \ \ \ \ \textbar{}\ \textbar{}\ \ \ \ \ \ \ \ \ \ \ \ \ \ \ \ \ \ \ \ \textbar{}\ \ \ \ \ \ -\/-\/-\/-\/-\/-\/-\/-\ \ \ \ \ \ \ \ \ \ \ \ \ \textbar{}\ \textbar{}\ \ \ \ \ \ \ \ \ \ \ \ \ \ \ \ \ \ \ \ \textbar{}\ \ \ \ \ \ 11101110}
& \\
\end{longtable}
}

\textbf{Decimal verification}:

\begin{itemize}
\tightlist
\item
  \begin{enumerate}
  \tightlist
  \item
    (11101101)_{2} = 237, (10101000)_{2} = 168, Sum = 405 = (110010101)_{2}
  \end{enumerate}
\item
  \begin{enumerate}
  \tightlist
  \item
    (11011)_{2} = 27, (1010)_{2} = 10, Product = 270 = (11101110)_{2}
  \end{enumerate}
\end{itemize}

\end{solutionbox}
\begin{mnemonicbox}
``Carry Up Makes Sum'' for addition and ``Shift Left
Add Product'' for multiplication.

\end{mnemonicbox}
\subsection*{Question 1(c) [7 marks]}\label{q1c}

\textbf{(i) Convert: (48)_{1}_{0} = ( \_\_\_ )_{2} = ( \_\_\_ )_{8} = ( \_\_\_ )_{1}_{6}}
\textbf{(ii) Subtract using 2's Complement method: (1110)_{2} -- (1000)_{2}}
\textbf{(iii) Divide (1111101)_{2} with (101)_{2}}

\begin{solutionbox}

\textbf{(i) Conversion Table}:

{\def\LTcaptype{none} % do not increment counter
\begin{longtable}[]{@{}llll@{}}
\toprule\noalign{}
Decimal (48)_{1}_{0} & Binary & Octal & Hexadecimal \\
\midrule\noalign{}
\endhead
\bottomrule\noalign{}
\endlastfoot
48\div2 = 24 rem 0 & 110000 & 60 & 30 \\
24\div2 = 12 rem 0 & & & \\
12\div2 = 6 rem 0 & & & \\
6\div2 = 3 rem 0 & & & \\
3\div2 = 1 rem 1 & & & \\
1\div2 = 0 rem 1 & & & \\
(48)_{1}_{0} & (110000)_{2} & (60)_{8} & (30)_{1}_{6} \\
\end{longtable}
}

\textbf{(ii) Subtraction Table}:

{\def\LTcaptype{none} % do not increment counter
\begin{longtable}[]{@{}ll@{}}
\toprule\noalign{}
2's Complement Method & Steps \\
\midrule\noalign{}
\endhead
\bottomrule\noalign{}
\endlastfoot
(1110)_{2} -- (1000)_{2} & 1. Find 2's complement of (1000)_{2} \\
1's complement of (1000)_{2} & (0111)_{2} \\
2's complement & (0111)_{2} + 1 = (1000)_{2} \\
(1110)_{2} + (1000)_{2} & (10110)_{2} \\
Discard carry & (0110)_{2} \\
Result & (0110)_{2} = 6_{1}_{0} \\
\end{longtable}
}

\textbf{(iii) Division}:

\begin{verbatim}
flowchart LR
    A["(1111101)_{2  (101)_{2}"] {-}{-} B["101)1111101(11 quotient}
                                    101
                                    {-{-}{-}{-}{-}}
                                    100
                                    000
                                    {-{-}{-}{-}{-}}
                                    1001
                                    101
                                    {-{-}{-}{-}{-}}
                                    001 remainder"]
    B {-{-} C["Quotient = (11)_{2}}
            Remainder = (1)_{2"]}
\end{verbatim}

\end{solutionbox}
\begin{mnemonicbox}
``Division Drops Down Remainders'' for long division
process.

\end{mnemonicbox}
\subsection*{Question 1(c) OR [7
marks]}\label{q1c}

\textbf{Explain Codes: ASCII, BCD, Gray}

\begin{solutionbox}

\textbf{Table of Common Digital Codes}:

{\def\LTcaptype{none} % do not increment counter
\begin{longtable}[]{@{}
  >{\raggedright\arraybackslash}p{(\linewidth - 4\tabcolsep) * \real{0.2143}}
  >{\raggedright\arraybackslash}p{(\linewidth - 4\tabcolsep) * \real{0.4643}}
  >{\raggedright\arraybackslash}p{(\linewidth - 4\tabcolsep) * \real{0.3214}}@{}}
\toprule\noalign{}
\begin{minipage}[b]{\linewidth}\raggedright
Code
\end{minipage} & \begin{minipage}[b]{\linewidth}\raggedright
Description
\end{minipage} & \begin{minipage}[b]{\linewidth}\raggedright
Example
\end{minipage} \\
\midrule\noalign{}
\endhead
\bottomrule\noalign{}
\endlastfoot
\textbf{ASCII (American Standard Code for Information Interchange)} &
7-bit code representing 128 characters including alphabets, numbers, and
special symbols & A = 65 (1000001)_{2} \\
\textbf{BCD (Binary Coded Decimal)} & Represents each decimal digit
(0-9) using 4 bits & 42 = 0100 0010 \\
\textbf{Gray Code} & Binary code where adjacent numbers differ by only
one bit & (0,1,3,2) = (00,01,11,10) \\
\end{longtable}
}

\textbf{Diagram: Gray Code Generation}:

\begin{verbatim}
flowchart LR
    A["Binary Code"] {-{-} B["Gray Code"]}
    B {-{-} C["MSB remains same}
    Each bit XORed with previous"]
    D["Binary: 0011"] {-{-} E["Gray: 0010"]}
\end{verbatim}

\end{solutionbox}
\begin{mnemonicbox}
``Always Binary Generates'' - first letter of each
code (ASCII, BCD, Gray).

\end{mnemonicbox}
\subsection*{Question 2(a) [3 marks]}\label{q2a}

\textbf{Simplify using Boolean Algebra: Y = A B + A' B + A' B' + A B'}

\begin{solutionbox}

\textbf{Step-by-step simplification}:

{\def\LTcaptype{none} % do not increment counter
\begin{longtable}[]{@{}lll@{}}
\toprule\noalign{}
Step & Expression & Boolean Law \\
\midrule\noalign{}
\endhead
\bottomrule\noalign{}
\endlastfoot
Y = A B + A' B + A' B' + A B' & Initial expression & - \\
Y = A(B + B') + A'(B + B') & Factoring & Distributive law \\
Y = A(1) + A'(1) & Complement law & B + B' = 1 \\
Y = A + A' & Simplification & - \\
Y = 1 & Complement law & A + A' = 1 \\
\end{longtable}
}

\end{solutionbox}
\begin{mnemonicbox}
``Factor, Simplify, Finish'' for Boolean
simplification steps.

\end{mnemonicbox}
\subsection*{Question 2(b) [4 marks]}\label{q2b}

\textbf{Simplify the following Boolean function using K-map: f(A,B,C,D)
= Σm (0,3,4,6,8,11,12)}

\begin{solutionbox}

\textbf{K-map Solution}:

\begin{verbatim}
    AB
CD  00 01 11 10
00  1  0  0  1
01  0  0  0  1  
11  0  1  0  0
10  0  0  1  0
\end{verbatim}

\textbf{Grouping}:

\begin{itemize}
\tightlist
\item
  Group 1: m(0,8) = A'C'D'
\item
  Group 2: m(4,12) = BD'
\item
  Group 3: m(3,11) = CD
\item
  Group 4: m(6) = A'B'CD'
\end{itemize}

\textbf{Simplified expression}: f(A,B,C,D) = A'C'D' + BD' + CD + A'B'CD'

\end{solutionbox}
\begin{mnemonicbox}
``Group Powers Of Two'' for K-map grouping strategy.

\end{mnemonicbox}
\subsection*{Question 2(c) [7 marks]}\label{q2c}

\textbf{Explain NOR gate as a universal gate with neat diagrams.}

\begin{solutionbox}

\textbf{NOR as Universal Gate}:

{\def\LTcaptype{none} % do not increment counter
\begin{longtable}[]{@{}lll@{}}
\toprule\noalign{}
Function & Implementation using NOR & Truth Table \\
\midrule\noalign{}
\endhead
\bottomrule\noalign{}
\endlastfoot
\textbf{NOT Gate} &
\pandocbounded{\includegraphics[keepaspectratio,alt={NOT using NOR}]{diagram1}}
& A \\
& & 0 \\
& & 1 \\
\textbf{AND Gate} &
\pandocbounded{\includegraphics[keepaspectratio,alt={AND using NOR}]{diagram2}}
& A B \\
& & 0 0 \\
& & 0 1 \\
& & 1 0 \\
& & 1 1 \\
\textbf{OR Gate} &
\pandocbounded{\includegraphics[keepaspectratio,alt={OR using NOR}]{diagram3}}
& A B \\
& & 0 0 \\
& & 0 1 \\
& & 1 0 \\
& & 1 1 \\
\end{longtable}
}

\textbf{Diagram: NOR Implementation}:

\begin{verbatim}
flowchart TD
    A["NOT: A {-1{-} A"]}
    B["AND: A {-{-}1{-}{-}|}
               |    1{{-}{-} A•B}
         B {-{-}1{-}{-}|"]}
    C["OR: A {-{-}1{-}{-}|}
              |    |{{-}{-} A+B}
        B {-{-}1{-}{-}|"]}
\end{verbatim}

\end{solutionbox}
\begin{mnemonicbox}
``NOT AND OR, NOR does more'' for remembering NOR
gate implementations.

\end{mnemonicbox}
\subsection*{Question 2(a) OR [3
marks]}\label{q2a}

\textbf{Draw logic circuit for Boolean expression: Y = (A + B') . (A' +
B') . (B + C)}

\begin{solutionbox}

\textbf{Logic Circuit Implementation}:

\begin{verbatim}
flowchart TD
    A["A"] {-{-} D["OR"]}
    B["B{"] {-}{-} D}
    D {-{-} G["AND"]}
    A1["A{"] {-}{-} E["OR"]}
    B1["B{"] {-}{-} E}
    E {-{-} G}
    B2["B"] {-{-} F["OR"]}
    C["C"] {-{-} F}
    F {-{-} G}
    G {-{-} Y["Y"]}
\end{verbatim}

\textbf{Truth Table Verification}:

\begin{itemize}
\tightlist
\item
  Term 1: (A + B')
\item
  Term 2: (A' + B')
\item
  Term 3: (B + C)
\item
  Output: Y = Term1 • Term2 • Term3
\end{itemize}

\end{solutionbox}
\begin{mnemonicbox}
``Each Term Separately'' for breaking complex
expressions.

\end{mnemonicbox}
\subsection*{Question 2(b) OR [4
marks]}\label{q2b}

\textbf{State De-Morgan's theorems and prove it.}

\begin{solutionbox}

\textbf{De-Morgan's Theorems and Proof}:

{\def\LTcaptype{none} % do not increment counter
\begin{longtable}[]{@{}lll@{}}
\toprule\noalign{}
Theorem & Statement & Proof by Truth Table \\
\midrule\noalign{}
\endhead
\bottomrule\noalign{}
\endlastfoot
\textbf{Theorem 1} & (A•B)' = A' + B' & A B \\
& & 0 0 \\
& & 0 1 \\
& & 1 0 \\
& & 1 1 \\
\textbf{Theorem 2} & (A+B)' = A'•B' & A B \\
& & 0 0 \\
& & 0 1 \\
& & 1 0 \\
& & 1 1 \\
\end{longtable}
}

\textbf{Diagram: De-Morgan's Law Visualization}:

\begin{verbatim}
flowchart TB
    A["(A•B){ = A+B"] {-}{-} B["Invert Operation}
                                AND  OR
                                Invert Variables"]
    C["(A+B){ = A•B"] {-}{-} D["Invert Operation}
                                OR  AND
                                Invert Variables"]
\end{verbatim}

\end{solutionbox}
\begin{mnemonicbox}
``Break BAR, Change Operation, Invert Inputs'' for
applying De-Morgan's law.

\end{mnemonicbox}
\subsection*{Question 2(c) OR [7
marks]}\label{q2c}

\textbf{Explain all the Logic Gates with the help of Symbol, Truth table
and equation.}

\begin{solutionbox}

\textbf{Logic Gates Summary}:

{\def\LTcaptype{none} % do not increment counter
\begin{longtable}[]{@{}lllll@{}}
\toprule\noalign{}
Gate & Symbol & Truth Table & Equation & Description \\
\midrule\noalign{}
\endhead
\bottomrule\noalign{}
\endlastfoot
\textbf{AND} &
\pandocbounded{\includegraphics[keepaspectratio,alt={AND}]{diagram}} & A
B & Y & Y = A•B \\
& & 0 0 & 0 & \\
& & 0 1 & 0 & \\
& & 1 0 & 0 & \\
& & 1 1 & 1 & \\
\textbf{OR} &
\pandocbounded{\includegraphics[keepaspectratio,alt={OR}]{diagram}} & A
B & Y & Y = A+B \\
& & 0 0 & 0 & \\
& & 0 1 & 1 & \\
& & 1 0 & 1 & \\
& & 1 1 & 1 & \\
\textbf{NOT} &
\pandocbounded{\includegraphics[keepaspectratio,alt={NOT}]{diagram}} & A
& Y & Y = A' \\
& & 0 & 1 & \\
& & 1 & 0 & \\
\textbf{NAND} &
\pandocbounded{\includegraphics[keepaspectratio,alt={NAND}]{diagram}} &
A B & Y & Y = (A•B)' \\
& & 0 0 & 1 & \\
& & 0 1 & 1 & \\
& & 1 0 & 1 & \\
& & 1 1 & 0 & \\
\textbf{NOR} &
\pandocbounded{\includegraphics[keepaspectratio,alt={NOR}]{diagram}} & A
B & Y & Y = (A+B)' \\
& & 0 0 & 1 & \\
& & 0 1 & 0 & \\
& & 1 0 & 0 & \\
& & 1 1 & 0 & \\
\textbf{XOR} &
\pandocbounded{\includegraphics[keepaspectratio,alt={XOR}]{diagram}} & A
B & Y & Y = A\oplusB \\
& & 0 0 & 0 & \\
& & 0 1 & 1 & \\
& & 1 0 & 1 & \\
& & 1 1 & 0 & \\
\textbf{XNOR} &
\pandocbounded{\includegraphics[keepaspectratio,alt={XNOR}]{diagram}} &
A B & Y & Y = (A\oplusB)' \\
& & 0 0 & 1 & \\
& & 0 1 & 0 & \\
& & 1 0 & 0 & \\
& & 1 1 & 1 & \\
\end{longtable}
}

\end{solutionbox}
\begin{mnemonicbox}
``All Operations Need Necessary eXecution'' (first
letter of each gate - AND, OR, NOT, NAND, NOR, XOR).

\end{mnemonicbox}
\subsection*{Question 3(a) [3 marks]}\label{q3a}

\textbf{Briefly explain 4:2 Encoder.}

\begin{solutionbox}

\textbf{4-to-2 Encoder Overview}:

{\def\LTcaptype{none} % do not increment counter
\begin{longtable}[]{@{}
  >{\raggedright\arraybackslash}p{(\linewidth - 4\tabcolsep) * \real{0.2778}}
  >{\raggedright\arraybackslash}p{(\linewidth - 4\tabcolsep) * \real{0.3611}}
  >{\raggedright\arraybackslash}p{(\linewidth - 4\tabcolsep) * \real{0.3611}}@{}}
\toprule\noalign{}
\begin{minipage}[b]{\linewidth}\raggedright
Function
\end{minipage} & \begin{minipage}[b]{\linewidth}\raggedright
Description
\end{minipage} & \begin{minipage}[b]{\linewidth}\raggedright
Truth Table
\end{minipage} \\
\midrule\noalign{}
\endhead
\bottomrule\noalign{}
\endlastfoot
\textbf{4:2 Encoder} & Converts 4 input lines to 2 output lines & I_{0} I_{1}
I_{2} I_{3} \\
& Only one input active at a time & 1 0 0 0 \\
& Input position encoded in binary & 0 1 0 0 \\
& & 0 0 1 0 \\
& & 0 0 0 1 \\
\end{longtable}
}

\textbf{Diagram: 4:2 Encoder}:

\begin{verbatim}
flowchart TD
    I0["I_{0"] {-}{-} E["4:2 Encoder"]}
    I1["I_{1"] {-}{-} E}
    I2["I_{2"] {-}{-} E}
    I3["I_{3"] {-}{-} E}
    E {-{-} Y1["Y_{1}"]}
    E {-{-} Y0["Y_{0}"]}
\end{verbatim}

\end{solutionbox}
\begin{mnemonicbox}
``Input Position Creates Output'' for encoder
function.

\end{mnemonicbox}
\subsection*{Question 3(b) [4 marks]}\label{q3b}

\textbf{Explain 4-bit Parallel adder using full adder blocks.}

\begin{solutionbox}

\textbf{4-bit Parallel Adder}:

{\def\LTcaptype{none} % do not increment counter
\begin{longtable}[]{@{}
  >{\raggedright\arraybackslash}p{(\linewidth - 2\tabcolsep) * \real{0.5238}}
  >{\raggedright\arraybackslash}p{(\linewidth - 2\tabcolsep) * \real{0.4762}}@{}}
\toprule\noalign{}
\begin{minipage}[b]{\linewidth}\raggedright
Component
\end{minipage} & \begin{minipage}[b]{\linewidth}\raggedright
Function
\end{minipage} \\
\midrule\noalign{}
\endhead
\bottomrule\noalign{}
\endlastfoot
\textbf{Full Adder} & Adds 3 bits (A, B, Carry-in) producing Sum and
Carry-out \\
\textbf{Parallel Adder} & Connects 4 full adders with carry
propagation \\
\end{longtable}
}

\textbf{Diagram: 4-bit Parallel Adder}:

\begin{verbatim}
flowchart LR
    A0["A_{0"] {-}{-} FA0["FA"]}
    B0["B_{0"] {-}{-} FA0}
    C0["C_{0=0"] {-}{-} FA0}
    FA0 {-{-} S0["S_{0}"]}
    FA0 {-{-} "C_{1}" {-}{-} FA1["FA"]}

    A1["A_{1"] {-}{-} FA1}
    B1["B_{1"] {-}{-} FA1}
    FA1 {-{-} S1["S_{1}"]}
    FA1 {-{-} "C_{2}" {-}{-} FA2["FA"]}
    
    A2["A_{2"] {-}{-} FA2}
    B2["B_{2"] {-}{-} FA2}
    FA2 {-{-} S2["S_{2}"] }
    FA2 {-{-} "C_{3}" {-}{-} FA3["FA"]}
    
    A3["A_{3"] {-}{-} FA3}
    B3["B_{3"] {-}{-} FA3}
    FA3 {-{-} S3["S_{3}"]}
    FA3 {-{-} C4["C_{4}"]}
\end{verbatim}

\end{solutionbox}
\begin{mnemonicbox}
``Carry Always Passes Right'' for the carry
propagation in parallel adder.

\end{mnemonicbox}
\subsection*{Question 3(c) [7 marks]}\label{q3c}

\textbf{Describe 8:1 Multiplexer with truth table, equation and circuit
diagram.}

\begin{solutionbox}

\textbf{8:1 Multiplexer}:

{\def\LTcaptype{none} % do not increment counter
\begin{longtable}[]{@{}
  >{\raggedright\arraybackslash}p{(\linewidth - 4\tabcolsep) * \real{0.3235}}
  >{\raggedright\arraybackslash}p{(\linewidth - 4\tabcolsep) * \real{0.3824}}
  >{\raggedright\arraybackslash}p{(\linewidth - 4\tabcolsep) * \real{0.2941}}@{}}
\toprule\noalign{}
\begin{minipage}[b]{\linewidth}\raggedright
Component
\end{minipage} & \begin{minipage}[b]{\linewidth}\raggedright
Description
\end{minipage} & \begin{minipage}[b]{\linewidth}\raggedright
Function
\end{minipage} \\
\midrule\noalign{}
\endhead
\bottomrule\noalign{}
\endlastfoot
\textbf{8:1 MUX} & Data selector with 8 inputs, 3 select lines, 1 output
& Selects one of 8 inputs based on select lines \\
\end{longtable}
}

\textbf{Truth Table}:

{\def\LTcaptype{none} % do not increment counter
\begin{longtable}[]{@{}ll@{}}
\toprule\noalign{}
Select Lines & Output \\
\midrule\noalign{}
\endhead
\bottomrule\noalign{}
\endlastfoot
S_{2} S_{1} S_{0} & Y \\
0 0 0 & D_{0} \\
0 0 1 & D_{1} \\
0 1 0 & D_{2} \\
0 1 1 & D_{3} \\
1 0 0 & D_{4} \\
1 0 1 & D_{5} \\
1 1 0 & D_{6} \\
1 1 1 & D_{7} \\
\end{longtable}
}

\textbf{Boolean Equation}: Y = S_{2}'·S_{1}'·S_{0}'·D_{0} + S_{2}'·S_{1}'·S_{0}·D_{1} +
S_{2}'·S_{1}·S_{0}'·D_{2} + S_{2}'·S_{1}·S_{0}·D_{3} + S_{2}·S_{1}'·S_{0}'·D_{4} + S_{2}·S_{1}'·S_{0}·D_{5} +
S_{2}·S_{1}·S_{0}'·D_{6} + S_{2}·S_{1}·S_{0}·D_{7}

\textbf{Diagram: 8:1 MUX}:

\begin{verbatim}
flowchart TD
    D0["D_{0"] {-}{-} MUX["8:1 MUX"]}
    D1["D_{1"] {-}{-} MUX}
    D2["D_{2"] {-}{-} MUX}
    D3["D_{3"] {-}{-} MUX}
    D4["D_{4"] {-}{-} MUX}
    D5["D_{5"] {-}{-} MUX}
    D6["D_{6"] {-}{-} MUX}
    D7["D_{7"] {-}{-} MUX}
    S0["S_{0"] {-}{-} MUX}
    S1["S_{1"] {-}{-} MUX}
    S2["S_{2"] {-}{-} MUX}
    MUX {-{-} Y["Y"]}
\end{verbatim}

\end{solutionbox}
\begin{mnemonicbox}
``Select Decides Data Output'' for multiplexer
operation.

\end{mnemonicbox}
\subsection*{Question 3(a) OR [3
marks]}\label{q3a}

\textbf{Draw the logic circuit of half Subtractor and explain its
working.}

\begin{solutionbox}

\textbf{Half Subtractor}:

{\def\LTcaptype{none} % do not increment counter
\begin{longtable}[]{@{}
  >{\raggedright\arraybackslash}p{(\linewidth - 4\tabcolsep) * \real{0.2778}}
  >{\raggedright\arraybackslash}p{(\linewidth - 4\tabcolsep) * \real{0.3611}}
  >{\raggedright\arraybackslash}p{(\linewidth - 4\tabcolsep) * \real{0.3611}}@{}}
\toprule\noalign{}
\begin{minipage}[b]{\linewidth}\raggedright
Function
\end{minipage} & \begin{minipage}[b]{\linewidth}\raggedright
Description
\end{minipage} & \begin{minipage}[b]{\linewidth}\raggedright
Truth Table
\end{minipage} \\
\midrule\noalign{}
\endhead
\bottomrule\noalign{}
\endlastfoot
\textbf{Half Subtractor} & Subtracts two bits producing Difference and
Borrow & A B \\
& & 0 0 \\
& & 0 1 \\
& & 1 0 \\
& & 1 1 \\
\end{longtable}
}

\textbf{Logic Circuit}:

\begin{verbatim}
flowchart TD
    A["A"] {-{-} XOR[""]}
    B["B"] {-{-} XOR}
    XOR {-{-} D["D = A"]}

    A1["A{"] {-}{-} AND["•"]}
    B1["B"] {-{-} AND}
    AND {-{-} Bout["Bout = A•B"]}
\end{verbatim}

\textbf{Equations}:

\begin{itemize}
\tightlist
\item
  Difference (D) = A \oplus B
\item
  Borrow out (Bout) = A' • B
\end{itemize}

\end{solutionbox}
\begin{mnemonicbox}
``Different Bits Borrow'' for half subtractor
operation.

\end{mnemonicbox}
\subsection*{Question 3(b) OR [4
marks]}\label{q3b}

\textbf{Explain 3:8 Decoder with truth table and circuit diagram.}

\begin{solutionbox}

\textbf{3:8 Decoder}:

{\def\LTcaptype{none} % do not increment counter
\begin{longtable}[]{@{}
  >{\raggedright\arraybackslash}p{(\linewidth - 4\tabcolsep) * \real{0.2222}}
  >{\raggedright\arraybackslash}p{(\linewidth - 4\tabcolsep) * \real{0.2889}}
  >{\raggedright\arraybackslash}p{(\linewidth - 4\tabcolsep) * \real{0.4889}}@{}}
\toprule\noalign{}
\begin{minipage}[b]{\linewidth}\raggedright
Function
\end{minipage} & \begin{minipage}[b]{\linewidth}\raggedright
Description
\end{minipage} & \begin{minipage}[b]{\linewidth}\raggedright
Truth Table (Partial)
\end{minipage} \\
\midrule\noalign{}
\endhead
\bottomrule\noalign{}
\endlastfoot
\textbf{3:8 Decoder} & Converts 3-bit binary input to 8 output lines &
A_{2} A_{1} A_{0} \\
& Only one output active at a time & 0 0 0 \\
& & 0 0 1 \\
& & \ldots{} \\
& & 1 1 1 \\
\end{longtable}
}

\textbf{Circuit Diagram}:

\begin{verbatim}
flowchart TD
    A0["A_{0"] {-}{-} Dec["3:8 Decoder"]}
    A1["A_{1"] {-}{-} Dec}
    A2["A_{2"] {-}{-} Dec}
    Dec {-{-} Y0["Y_{0}"]}
    Dec {-{-} Y1["Y_{1}"]}
    Dec {-{-} Y2["Y_{2}"]}
    Dec {-{-} Y3["Y_{3}"]}
    Dec {-{-} Y4["Y_{4}"]}
    Dec {-{-} Y5["Y_{5}"]}
    Dec {-{-} Y6["Y_{6}"]}
    Dec {-{-} Y7["Y_{7}"]}
\end{verbatim}

\textbf{Equations}:

\begin{itemize}
\tightlist
\item
  Y_{0} = A_{2}' • A_{1}' • A_{0}'
\item
  Y_{1} = A_{2}' • A_{1}' • A_{0}
\item
  \ldots{}
\item
  Y_{7} = A_{2} • A_{1} • A_{0}
\end{itemize}

\end{solutionbox}
\begin{mnemonicbox}
``Binary Input Activates Output'' for decoder
operation.

\end{mnemonicbox}
\subsection*{Question 3(c) OR [7
marks]}\label{q3c}

\textbf{Explain Gray to Binary code converter with truth table, equation
and circuit diagram.}

\begin{solutionbox}

\textbf{Gray to Binary Converter}:

{\def\LTcaptype{none} % do not increment counter
\begin{longtable}[]{@{}
  >{\raggedright\arraybackslash}p{(\linewidth - 4\tabcolsep) * \real{0.2222}}
  >{\raggedright\arraybackslash}p{(\linewidth - 4\tabcolsep) * \real{0.2889}}
  >{\raggedright\arraybackslash}p{(\linewidth - 4\tabcolsep) * \real{0.4889}}@{}}
\toprule\noalign{}
\begin{minipage}[b]{\linewidth}\raggedright
Function
\end{minipage} & \begin{minipage}[b]{\linewidth}\raggedright
Description
\end{minipage} & \begin{minipage}[b]{\linewidth}\raggedright
Table: Gray to Binary
\end{minipage} \\
\midrule\noalign{}
\endhead
\bottomrule\noalign{}
\endlastfoot
\textbf{Gray to Binary} & Converts Gray code to Binary code & Gray \\
& MSB of binary equals MSB of gray & 0000 \\
& Each binary bit is XOR of current gray bit and previous binary bit &
0001 \\
& & 0011 \\
& & 0010 \\
& & 0110 \\
& & \ldots{} \\
\end{longtable}
}

\textbf{Circuit Diagram}:

\begin{verbatim}
flowchart LR
    G3["G_{3"] {-}{-} B3["B_{3}"]}
    G3 {-{-} XOR1[""]}
    G2["G_{2"] {-}{-} XOR1}
    XOR1 {-{-} B2["B_{2}"]}
    XOR1 {-{-} XOR2[""]}
    G1["G_{1"] {-}{-} XOR2}
    XOR2 {-{-} B1["B_{1}"]}
    XOR2 {-{-} XOR3[""]}
    G0["G_{0"] {-}{-} XOR3}
    XOR3 {-{-} B0["B_{0}"]}
\end{verbatim}

\textbf{Equations}:

\begin{itemize}
\tightlist
\item
  B_{3} = G_{3}
\item
  B_{2} = G_{3} \oplus G_{2}
\item
  B_{1} = B_{2} \oplus G_{1}
\item
  B_{0} = B_{1} \oplus G_{0}
\end{itemize}

\end{solutionbox}
\begin{mnemonicbox}
``MSB Stays, Rest XOR'' for Gray to Binary
conversion.

\end{mnemonicbox}
\subsection*{Question 4(a) [3 marks]}\label{q4a}

\textbf{Explain D flip flop with truth table and circuit diagram.}

\begin{solutionbox}

\textbf{D Flip-Flop}:

{\def\LTcaptype{none} % do not increment counter
\begin{longtable}[]{@{}lll@{}}
\toprule\noalign{}
Function & Description & Truth Table \\
\midrule\noalign{}
\endhead
\bottomrule\noalign{}
\endlastfoot
\textbf{D Flip-Flop} & Data/Delay flip-flop & CLK \\
& Q follows D at clock edge & ↑ \\
& & ↑ \\
\end{longtable}
}

\textbf{Circuit Diagram}:

\begin{verbatim}
flowchart LR
    D["D"] {-{-} FF["D Flip{-}Flop"]}
    CLK["Clock"] {-{-} FF}
    FF {-{-} Q["Q"]}
    FF {-{-} Qnot["Q"]}
\end{verbatim}

\textbf{Characteristic Equation}:

\begin{itemize}
\tightlist
\item
  Q(next) = D
\end{itemize}

\end{solutionbox}
\begin{mnemonicbox}
``Data Delays one clock'' for D flip-flop operation.

\end{mnemonicbox}
\subsection*{Question 4(b) [4 marks]}\label{q4b}

\textbf{Explain working of Master Slave JK flip flop.}

\begin{solutionbox}

\textbf{Master-Slave JK Flip-Flop}:

{\def\LTcaptype{none} % do not increment counter
\begin{longtable}[]{@{}lll@{}}
\toprule\noalign{}
Component & Operation & Truth Table \\
\midrule\noalign{}
\endhead
\bottomrule\noalign{}
\endlastfoot
\textbf{Master} & Samples inputs when CLK = 1 & J K \\
\textbf{Slave} & Transfers master output when CLK = 0 & 0 0 \\
& & 0 1 \\
& & 1 0 \\
& & 1 1 \\
\end{longtable}
}

\textbf{Diagram: Master-Slave JK}:

\begin{verbatim}
flowchart LR
    J["J"] {-{-} Master["Master JK"]}
    K["K"] {-{-} Master}
    CLK["Clock"] {-{-} Master}
    CLK{ {-}{-} Slave["Slave JK"]}
    Master {-{-} Slave}
    Slave {-{-} Q["Q"]}
    Slave {-{-} Q["Q"]}
\end{verbatim}

\textbf{Working}:

\begin{itemize}
\tightlist
\item
  \textbf{Master stage}: Captures input during clock high
\item
  \textbf{Slave stage}: Updates output during clock low
\item
  \textbf{Prevents race condition} by separating input capture and
  output update
\end{itemize}

\end{solutionbox}
\begin{mnemonicbox}
``Master Samples, Slave Transfers'' for master-slave
operation.

\end{mnemonicbox}
\subsection*{Question 4(c) [7 marks]}\label{q4c}

\textbf{Classify Shift Registers with the help of Block diagram and
Explain any one of them in detail.}

\begin{solutionbox}

\textbf{Shift Register Classification}:

{\def\LTcaptype{none} % do not increment counter
\begin{longtable}[]{@{}
  >{\raggedright\arraybackslash}p{(\linewidth - 4\tabcolsep) * \real{0.2069}}
  >{\raggedright\arraybackslash}p{(\linewidth - 4\tabcolsep) * \real{0.4483}}
  >{\raggedright\arraybackslash}p{(\linewidth - 4\tabcolsep) * \real{0.3448}}@{}}
\toprule\noalign{}
\begin{minipage}[b]{\linewidth}\raggedright
Type
\end{minipage} & \begin{minipage}[b]{\linewidth}\raggedright
Description
\end{minipage} & \begin{minipage}[b]{\linewidth}\raggedright
Function
\end{minipage} \\
\midrule\noalign{}
\endhead
\bottomrule\noalign{}
\endlastfoot
\textbf{SISO} & Serial In Serial Out & Data enters and exits serially,
bit by bit \\
\textbf{SIPO} & Serial In Parallel Out & Data enters serially, exits in
parallel \\
\textbf{PISO} & Parallel In Serial Out & Data enters in parallel, exits
serially \\
\textbf{PIPO} & Parallel In Parallel Out & Data enters and exits in
parallel \\
\end{longtable}
}

\textbf{SIPO Shift Register in Detail}:

\begin{verbatim}
flowchart LR
    Din["Data In"] {-{-} FF1["FF_{1}"]}
    FF1 {-{-} FF2["FF_{2}"]}
    FF2 {-{-} FF3["FF_{3}"]}
    FF3 {-{-} FF4["FF_{4}"]}
    CLK["Clock"] {-{-} FF1}
    CLK {-{-} FF2}
    CLK {-{-} FF3}
    CLK {-{-} FF4}
    FF1 {-{-} Q0["Q_{0}"]}
    FF2 {-{-} Q1["Q_{1}"]}
    FF3 {-{-} Q2["Q_{2}"]}
    FF4 {-{-} Q3["Q_{3}"]}
\end{verbatim}

\textbf{Working of SIPO Shift Register}:

\begin{itemize}
\tightlist
\item
  \textbf{Serial data} enters at Data-In pin, one bit per clock cycle
\item
  \textbf{Each flip-flop} passes its content to the next on clock pulse
\item
  \textbf{After 4 clock cycles}, 4-bit data is stored in all flip-flops
\item
  \textbf{Parallel output} available from Q0-Q3 simultaneously
\end{itemize}

\textbf{Timing Diagram for SIPO}:

\begin{verbatim}
Clock   \_|‾|\_|‾|\_|‾|\_|‾|\_
Data    \_\_\_|‾‾‾|\_\_\_|‾‾‾|\_
Q0      \_\_\_|‾‾‾|\_\_\_|‾‾‾|\_
Q1      \_\_\_\_\_|‾‾‾|\_\_\_|‾‾
Q2      \_\_\_\_\_\_\_|‾‾‾|\_\_\_
Q3      \_\_\_\_\_\_\_\_\_|‾‾‾|\_
\end{verbatim}

\end{solutionbox}
\begin{mnemonicbox}
``Serial Inputs Parallel Outputs'' for SIPO
operation.

\end{mnemonicbox}
\subsection*{Question 4(a) OR [3
marks]}\label{q4a}

\textbf{Explain SR flip flop with truth table and circuit diagram.}

\begin{solutionbox}

\textbf{SR Flip-Flop}:

{\def\LTcaptype{none} % do not increment counter
\begin{longtable}[]{@{}lll@{}}
\toprule\noalign{}
Function & Description & Truth Table \\
\midrule\noalign{}
\endhead
\bottomrule\noalign{}
\endlastfoot
\textbf{SR Flip-Flop} & Set-Reset flip-flop & S R \\
& Basic memory element & 0 0 \\
& & 0 1 \\
& & 1 0 \\
& & 1 1 \\
\end{longtable}
}

\textbf{Circuit Diagram}:

\begin{verbatim}
flowchart LR
    S["S"] {-{-} NOR1["1"]}
    QN["Q{"] {-}{-} NOR1}
    NOR1 {-{-} Q["Q"]}
    R["R"] {-{-} NOR2["1"]}
    Q {-{-} NOR2}
    NOR2 {-{-} QN}
\end{verbatim}

\end{solutionbox}
\begin{mnemonicbox}
``Set to 1, Reset to 0'' for SR flip-flop operation.

\end{mnemonicbox}
\subsection*{Question 4(b) OR [4
marks]}\label{q4b}

\textbf{Describe JK flip flop with truth table and circuit diagram.}

\begin{solutionbox}

\textbf{JK Flip-Flop}:

{\def\LTcaptype{none} % do not increment counter
\begin{longtable}[]{@{}lll@{}}
\toprule\noalign{}
Function & Description & Truth Table \\
\midrule\noalign{}
\endhead
\bottomrule\noalign{}
\endlastfoot
\textbf{JK Flip-Flop} & Improved SR flip-flop & J K \\
& Resolves invalid condition & 0 0 \\
& & 0 1 \\
& & 1 0 \\
& & 1 1 \\
\end{longtable}
}

\textbf{Circuit Diagram}:

\begin{verbatim}
flowchart LR
    J["J"] {-{-} AND1["•"]}
    Qn["Q{"] {-}{-} AND1}
    AND1 {-{-} OR["1"]}
    K["K"] {-{-} AND2["•"]}
    Q["Q"] {-{-} AND2}
    AND2 {-{-} OR}
    OR {-{-} FF["D FF"]}
    CLK["Clock"] {-{-} FF}
    FF {-{-} Q}
    FF {-{-} Qn}
\end{verbatim}

\textbf{Characteristic Equation}:

\begin{itemize}
\tightlist
\item
  Q(next) = J•Q' + K'•Q
\end{itemize}

\end{solutionbox}
\begin{mnemonicbox}
``Jump-Keep-Toggle'' for JK flip-flop states (J=1
K=0: Jump to 1,

J=0

K=0: Keep state,

J=1

K=1: Toggle).


\end{mnemonicbox}
\subsection*{Question 4(c) OR [7
marks]}\label{q4c}

\textbf{Describe 4-bit Asynchronous UP Counter with truth table and
circuit diagram.}

\begin{solutionbox}

\textbf{4-bit Asynchronous UP Counter}:

{\def\LTcaptype{none} % do not increment counter
\begin{longtable}[]{@{}
  >{\raggedright\arraybackslash}p{(\linewidth - 4\tabcolsep) * \real{0.2564}}
  >{\raggedright\arraybackslash}p{(\linewidth - 4\tabcolsep) * \real{0.3333}}
  >{\raggedright\arraybackslash}p{(\linewidth - 4\tabcolsep) * \real{0.4103}}@{}}
\toprule\noalign{}
\begin{minipage}[b]{\linewidth}\raggedright
Function
\end{minipage} & \begin{minipage}[b]{\linewidth}\raggedright
Description
\end{minipage} & \begin{minipage}[b]{\linewidth}\raggedright
Count Sequence
\end{minipage} \\
\midrule\noalign{}
\endhead
\bottomrule\noalign{}
\endlastfoot
\textbf{Asynchronous Counter} & Also called ripple counter & 0000 \rightarrow 0001
\rightarrow 0010 \rightarrow 0011 \\
& Clock drives only first FF & 0100 \rightarrow 0101 \rightarrow 0110 \rightarrow 0111 \\
& Each FF triggered by previous FF output & 1000 \rightarrow 1001 \rightarrow 1010 \rightarrow 1011 \\
& & 1100 \rightarrow 1101 \rightarrow 1110 \rightarrow 1111 \\
\end{longtable}
}

\textbf{Circuit Diagram}:

\begin{verbatim}
flowchart LR
    CLK["Clock"] {-{-} JK1["JK FF_{0}"]}
    J1["J=1"] {-{-} JK1}
    K1["K=1"] {-{-} JK1}
    JK1 {-{-} Q0["Q_{0}"]}
    JK1 {-{-}"Q_{0}"{-}{-} JK2["JK FF_{1}"]}
    J2["J=1"] {-{-} JK2}
    K2["K=1"] {-{-} JK2}
    JK2 {-{-} Q1["Q_{1}"]}
    JK2 {-{-}"Q_{1}"{-}{-} JK3["JK FF_{2}"]}
    J3["J=1"] {-{-} JK3}
    K3["K=1"] {-{-} JK3}
    JK3 {-{-} Q2["Q_{2}"]}
    JK3 {-{-}"Q_{2}"{-}{-} JK4["JK FF_{3}"]}
    J4["J=1"] {-{-} JK4}
    K4["K=1"] {-{-} JK4}
    JK4 {-{-} Q3["Q_{3}"]}
\end{verbatim}

\textbf{Working}:

\begin{itemize}
\tightlist
\item
  \textbf{First FF} toggles on every clock pulse
\item
  \textbf{Second FF} toggles when first FF goes from 1 to 0
\item
  \textbf{Third FF} toggles when second FF goes from 1 to 0
\item
  \textbf{Fourth FF} toggles when third FF goes from 1 to 0
\end{itemize}

\end{solutionbox}
\begin{mnemonicbox}
``Ripple Carries Propagation Delay'' for asynchronous
counter operation.

\end{mnemonicbox}
\subsection*{Question 5(a) [3 marks]}\label{q5a}

\textbf{Compare following logic families: TTL, CMOS, ECL}

\begin{solutionbox}

\textbf{Logic Families Comparison}:

{\def\LTcaptype{none} % do not increment counter
\begin{longtable}[]{@{}llll@{}}
\toprule\noalign{}
Parameter & TTL & CMOS & ECL \\
\midrule\noalign{}
\endhead
\bottomrule\noalign{}
\endlastfoot
\textbf{Technology} & Bipolar transistors & MOSFETs & Bipolar
transistors \\
\textbf{Power Consumption} & Medium & Very low & High \\
\textbf{Speed} & Medium & Low-Medium & Very high \\
\textbf{Noise Immunity} & Medium & High & Low \\
\textbf{Fan-out} & 10 & 50+ & 25 \\
\textbf{Supply Voltage} & 5V & 3-15V & -5.2V \\
\end{longtable}
}

\end{solutionbox}
\begin{mnemonicbox}
``Technology Controls Many Electrical
Characteristics'' for comparing logic families.

\end{mnemonicbox}
\subsection*{Question 5(b) [4 marks]}\label{q5b}

\textbf{Compare Combinational and Sequential Logic Circuits.}

\begin{solutionbox}

\textbf{Combinational vs Sequential Circuits}:

{\def\LTcaptype{none} % do not increment counter
\begin{longtable}[]{@{}
  >{\raggedright\arraybackslash}p{(\linewidth - 4\tabcolsep) * \real{0.1964}}
  >{\raggedright\arraybackslash}p{(\linewidth - 4\tabcolsep) * \real{0.4286}}
  >{\raggedright\arraybackslash}p{(\linewidth - 4\tabcolsep) * \real{0.3750}}@{}}
\toprule\noalign{}
\begin{minipage}[b]{\linewidth}\raggedright
Parameter
\end{minipage} & \begin{minipage}[b]{\linewidth}\raggedright
Combinational Circuits
\end{minipage} & \begin{minipage}[b]{\linewidth}\raggedright
Sequential Circuits
\end{minipage} \\
\midrule\noalign{}
\endhead
\bottomrule\noalign{}
\endlastfoot
\textbf{Output depends on} & Current inputs only & Current inputs and
previous state \\
\textbf{Memory} & No memory & Has memory elements \\
\textbf{Feedback} & No feedback paths & Contains feedback paths \\
\textbf{Examples} & Adders, MUX, Decoders & Flip-flops, Counters,
Registers \\
\textbf{Clock} & No clock required & Clock often required \\
\textbf{Design approach} & Truth tables, K-maps & State diagrams,
tables \\
\end{longtable}
}

\textbf{Diagram: Comparison}:

\begin{verbatim}
flowchart TB
    A["Combinational Logic"] {-{-} B["Outputs = f(Current Inputs)"]}
    C["Sequential Logic"] {-{-} D["Outputs = f(Current Inputs, Previous State)"]}
\end{verbatim}

\end{solutionbox}
\begin{mnemonicbox}
``Current Only vs Memory States'' for differentiating
combinational and sequential circuits.

\end{mnemonicbox}
\subsection*{Question 5(c) [7 marks]}\label{q5c}

\textbf{Define: Fan in, Fan out, Noise margin, Propagation delay, Power
dissipation, Figure of merit, RAM}

\begin{solutionbox}

\textbf{Digital Electronics Key Definitions}:

{\def\LTcaptype{none} % do not increment counter
\begin{longtable}[]{@{}
  >{\raggedright\arraybackslash}p{(\linewidth - 4\tabcolsep) * \real{0.1765}}
  >{\raggedright\arraybackslash}p{(\linewidth - 4\tabcolsep) * \real{0.3529}}
  >{\raggedright\arraybackslash}p{(\linewidth - 4\tabcolsep) * \real{0.4706}}@{}}
\toprule\noalign{}
\begin{minipage}[b]{\linewidth}\raggedright
Term
\end{minipage} & \begin{minipage}[b]{\linewidth}\raggedright
Definition
\end{minipage} & \begin{minipage}[b]{\linewidth}\raggedright
Typical Values
\end{minipage} \\
\midrule\noalign{}
\endhead
\bottomrule\noalign{}
\endlastfoot
\textbf{Fan-in} & Maximum number of inputs a logic gate can handle &
TTL: 2-8, CMOS: 100+ \\
\textbf{Fan-out} & Maximum number of gate inputs that can be driven by a
single output & TTL: 10, CMOS: 50 \\
\textbf{Noise margin} & Maximum noise voltage that can be added before
causing error & TTL: 0.4V, CMOS: 1.5V \\
\textbf{Propagation delay} & Time taken for change in input to cause
change in output & TTL: 10ns, CMOS: 20ns \\
\textbf{Power dissipation} & Power consumed by gate during operation &
TTL: 10mW, CMOS: 0.1mW \\
\textbf{Figure of merit} & Product of speed and power (lower is better)
& TTL: 100pJ, CMOS: 2pJ \\
\textbf{RAM} & Random Access Memory - temporary storage device & Types:
SRAM, DRAM \\
\end{longtable}
}

\textbf{Diagram: Digital Parameter Relationships}:

\begin{verbatim}
flowchart LR
    A["Lower Propagation Delay"]{-{-}"Increases"{-}{-}B["Speed"]}
    C["Lower Power Dissipation"]{-{-}"Increases"{-}{-}D["Efficiency"]}
    B{-{-}"x"{-}{-}E["Figure of Merit"]}
    D{-{-}"x"{-}{-}E}
\end{verbatim}

\end{solutionbox}
\begin{mnemonicbox}
``Fast Power Needs Proper Figure Ratings'' for
remembering the parameter terms.

\end{mnemonicbox}
\subsection*{Question 5(a) OR [3
marks]}\label{q5a}

\textbf{Describe steps and the need of E-waste management of Digital
ICs.}

\begin{solutionbox}

\textbf{E-waste Management for Digital ICs}:

{\def\LTcaptype{none} % do not increment counter
\begin{longtable}[]{@{}
  >{\raggedright\arraybackslash}p{(\linewidth - 4\tabcolsep) * \real{0.1935}}
  >{\raggedright\arraybackslash}p{(\linewidth - 4\tabcolsep) * \real{0.4194}}
  >{\raggedright\arraybackslash}p{(\linewidth - 4\tabcolsep) * \real{0.3871}}@{}}
\toprule\noalign{}
\begin{minipage}[b]{\linewidth}\raggedright
Step
\end{minipage} & \begin{minipage}[b]{\linewidth}\raggedright
Description
\end{minipage} & \begin{minipage}[b]{\linewidth}\raggedright
Importance
\end{minipage} \\
\midrule\noalign{}
\endhead
\bottomrule\noalign{}
\endlastfoot
\textbf{Collection} & Separate collection of electronic waste & Prevents
improper disposal \\
\textbf{Segregation} & Separating ICs from other components & Enables
targeted recycling \\
\textbf{Dismantling} & Removal of hazardous parts & Reduces
environmental harm \\
\textbf{Recovery} & Extracting valuable materials (gold, silicon) &
Conserves resources \\
\textbf{Safe disposal} & Proper disposal of non-recyclable parts &
Prevents pollution \\
\end{longtable}
}

\textbf{Need for E-waste Management}:

\begin{itemize}
\tightlist
\item
  \textbf{Hazardous Materials}: ICs contain lead, mercury, cadmium
\item
  \textbf{Resource Conservation}: Recovers precious metals and rare
  materials
\item
  \textbf{Environmental Protection}: Prevents soil and water
  contamination
\item
  \textbf{Health Safety}: Reduces exposure to toxic substances
\end{itemize}

\end{solutionbox}
\begin{mnemonicbox}
``Collection Starts Dismantling Recovery Safely'' for
e-waste management steps.

\end{mnemonicbox}
\subsection*{Question 5(b) OR [4
marks]}\label{q5b}

\textbf{Explain working of Ring Counter with circuit diagram.}

\begin{solutionbox}

\textbf{Ring Counter}:

{\def\LTcaptype{none} % do not increment counter
\begin{longtable}[]{@{}
  >{\raggedright\arraybackslash}p{(\linewidth - 4\tabcolsep) * \real{0.2564}}
  >{\raggedright\arraybackslash}p{(\linewidth - 4\tabcolsep) * \real{0.3333}}
  >{\raggedright\arraybackslash}p{(\linewidth - 4\tabcolsep) * \real{0.4103}}@{}}
\toprule\noalign{}
\begin{minipage}[b]{\linewidth}\raggedright
Function
\end{minipage} & \begin{minipage}[b]{\linewidth}\raggedright
Description
\end{minipage} & \begin{minipage}[b]{\linewidth}\raggedright
Count Sequence
\end{minipage} \\
\midrule\noalign{}
\endhead
\bottomrule\noalign{}
\endlastfoot
\textbf{Ring Counter} & Circular shift register with single 1 & 1000 \rightarrow
0100 \rightarrow 0010 \rightarrow 0001 \rightarrow 1000 \\
& Only one flip-flop is set at any time & \\
& N flip-flops for N states & \\
\end{longtable}
}

\textbf{Circuit Diagram}:

\begin{verbatim}
flowchart LR
    CLK["Clock"] {-{-} FF1["FF_{1}"]}
    CLK {-{-} FF2["FF_{2}"]}
    CLK {-{-} FF3["FF_{3}"]}
    CLK {-{-} FF4["FF_{4}"]}
    FF4 {-{-} FF1}
    FF1 {-{-} Q1["Q_{1}"]}
    FF1 {-{-} FF2}
    FF2 {-{-} Q2["Q_{2}"]}
    FF2 {-{-} FF3}
    FF3 {-{-} Q3["Q_{3}"]}
    FF3 {-{-} FF4}
    FF4 {-{-} Q4["Q_{4}"]}
    CLR["Preset"] {-{-} FF1}
    CLR {-{-} FF2}
    CLR {-{-} FF3}
    CLR {-{-} FF4}
\end{verbatim}

\textbf{Working}:

\begin{itemize}
\tightlist
\item
  \textbf{Initialization}: First FF set to 1, others to 0
\item
  \textbf{Operation}: Single 1 rotates through all flip-flops
\item
  \textbf{Applications}: Sequencers, controllers, timing circuits
\end{itemize}

\end{solutionbox}
\begin{mnemonicbox}
``One Bit Rotates Only'' for ring counter operation.

\end{mnemonicbox}
\subsection*{Question 5(c) OR [7
marks]}\label{q5c}

\textbf{Classify: (i) Memories (ii) Different Logic Families}

\begin{solutionbox}

\textbf{(i) Memory Classification}:

{\def\LTcaptype{none} % do not increment counter
\begin{longtable}[]{@{}
  >{\raggedright\arraybackslash}p{(\linewidth - 4\tabcolsep) * \real{0.1875}}
  >{\raggedright\arraybackslash}p{(\linewidth - 4\tabcolsep) * \real{0.3125}}
  >{\raggedright\arraybackslash}p{(\linewidth - 4\tabcolsep) * \real{0.5000}}@{}}
\toprule\noalign{}
\begin{minipage}[b]{\linewidth}\raggedright
Type
\end{minipage} & \begin{minipage}[b]{\linewidth}\raggedright
Subtypes
\end{minipage} & \begin{minipage}[b]{\linewidth}\raggedright
Characteristics
\end{minipage} \\
\midrule\noalign{}
\endhead
\bottomrule\noalign{}
\endlastfoot
\textbf{RAM} & \textbf{SRAM} & - Static RAM- Fast, expensive- Uses
flip-flops- No refresh needed \\
& \textbf{DRAM} & - Dynamic RAM- Slower, cheaper- Uses capacitors- Needs
periodic refresh \\
\textbf{ROM} & \textbf{PROM} & - Programmable ROM- One-time
programmable \\
& \textbf{EPROM} & - Erasable PROM- UV light erasable- Multiple
reprogramming \\
& \textbf{EEPROM} & - Electrically Erasable PROM- Electrical erasure-
Byte-level erasure \\
& \textbf{Flash} & - EEPROM variant- Block-level erasure-
Non-volatile \\
\end{longtable}
}

\textbf{(ii) Logic Families Classification}:

{\def\LTcaptype{none} % do not increment counter
\begin{longtable}[]{@{}
  >{\raggedright\arraybackslash}p{(\linewidth - 4\tabcolsep) * \real{0.3158}}
  >{\raggedright\arraybackslash}p{(\linewidth - 4\tabcolsep) * \real{0.2632}}
  >{\raggedright\arraybackslash}p{(\linewidth - 4\tabcolsep) * \real{0.4211}}@{}}
\toprule\noalign{}
\begin{minipage}[b]{\linewidth}\raggedright
Technology
\end{minipage} & \begin{minipage}[b]{\linewidth}\raggedright
Families
\end{minipage} & \begin{minipage}[b]{\linewidth}\raggedright
Characteristics
\end{minipage} \\
\midrule\noalign{}
\endhead
\bottomrule\noalign{}
\endlastfoot
\textbf{Bipolar} & \textbf{TTL} & - Transistor-Transistor Logic- Medium
speed- 5V operation \\
& \textbf{ECL} & - Emitter-Coupled Logic- Very high speed- High power
consumption \\
& \textbf{I^{2}L} & - Integrated Injection Logic- High density \\
\textbf{MOS} & \textbf{NMOS} & - N-channel MOSFET- Simpler
fabrication \\
& \textbf{PMOS} & - P-channel MOSFET- Lower performance \\
& \textbf{CMOS} & - Complementary MOS- Low power consumption- High noise
immunity \\
\textbf{Hybrid} & \textbf{BiCMOS} & - Combines Bipolar and CMOS- High
speed with low power \\
\end{longtable}
}

\textbf{Memory Classification Diagram}:

\begin{verbatim}
flowchart TB
    MEM["Semiconductor Memories"]
    MEM {-{-} RAM["Random Access Memory (Volatile)"]}
    MEM {-{-} ROM["Read Only Memory (Non{-}volatile)"]}
    RAM {-{-} SRAM["SRAM (Static)"]}
    RAM {-{-} DRAM["DRAM (Dynamic)"]}
    ROM {-{-} PROM["PROM (One{-}time)"]}
    ROM {-{-} EPROM["EPROM (UV Erasable)"]}
    ROM {-{-} EEPROM["EEPROM (Electrical Erasable)"]}
\end{verbatim}

\end{solutionbox}
\begin{mnemonicbox}
``Remember Simple Division: Programmable Erasable
Electrical'' for memory types (RAM-SRAM-DRAM, PROM-EPROM-EEPROM).

\end{mnemonicbox}

\end{document}
