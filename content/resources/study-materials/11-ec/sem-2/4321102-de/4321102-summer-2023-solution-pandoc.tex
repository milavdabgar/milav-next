\documentclass[10pt,a4paper]{article}

% content/resources/templates/preamble.tex
\usepackage[margin=0.6in]{geometry}
\author{Milav Dabgar}
\usepackage{amsmath,amssymb,amsthm}
\usepackage{booktabs}
\usepackage{multirow}
\usepackage{xcolor}
\usepackage{tcolorbox}
\tcbuselibrary{breakable,skins}
\usepackage[colorlinks=true,linkcolor=blue]{hyperref}
\usepackage{titlesec}
\usepackage{enumitem}
\usepackage{tikz}
\usepackage{pgfplots}
\usepackage{circuitikz}
\usepackage[version=4]{mhchem}
\usepackage{longtable}
\usepackage{array}
\usepackage{float}
\usepackage{caption}
\usepackage{listings}

\lstset{
  basicstyle=\small\ttfamily,
  breaklines=true,
  breakatwhitespace=false,
  postbreak=\mbox{\textcolor{red}{$\hookrightarrow$}\space},
  float=false,
  numbers=left,
  numberstyle=\tiny\color{gray},
  numbersep=10pt,
  xleftmargin=2em,
  keywordstyle=\color{blue},
  commentstyle=\color{green!60!black},
  stringstyle=\color{purple},
  backgroundcolor=\color{gray!5},
  showstringspaces=false,
  tabsize=2,
  captionpos=b,
  keepspaces=true,
  columns=flexible
}

\pgfplotsset{compat=1.18}
\usetikzlibrary{shapes,arrows,positioning,calc,patterns,decorations.pathmorphing,decorations.markings,arrows.meta}

% Color scheme
\definecolor{headcolor}{RGB}{0,102,204}
\definecolor{keycolor}{RGB}{220,20,60}
\definecolor{solutioncolor}{RGB}{34,139,34}
\definecolor{mnemoniccolor}{RGB}{148,0,211}
\definecolor{codecolor}{RGB}{0,0,100}

% Spacing
\setlength{\parskip}{3pt}
\setlist[itemize]{nosep}
\setlist[enumerate]{nosep}

% Title formatting
\titleformat{\section}{\Large\bfseries\color{headcolor}}{\thesection}{1em}{}
\titleformat{\subsection}{\large\bfseries\color{headcolor}}{\thesubsection}{1em}{}

% Pandoc tightlist compatibility
\providecommand{\tightlist}{%
  \setlength{\itemsep}{0pt}\setlength{\parskip}{0pt}}

% Pandoc longtable compatibility
\newcounter{none}
\def\thenone{}


% content/resources/templates/english-boxes.tex
% This file is currently empty - it exists to maintain consistency with the import structure.
% Add custom environments here if needed in the future.


\begin{document}

\begin{center}
{\Huge\bfseries\color{headcolor} Subject Name Solutions}\\[5pt]
{\LARGE 4321102 -- Summer 2023}\\[3pt]
{\large Semester 1 Study Material}\\[3pt]
{\normalsize\textit{Detailed Solutions and Explanations}}
\end{center}

\vspace{10pt}

\subsection*{Question 1(a) [3 marks]}\label{q1a}

\textbf{Explain De-Morgan's theorem for Boolean algebra}

\begin{solutionbox}
De-Morgan's theorem consists of two laws that show the
relationship between AND, OR, and NOT operations:

\textbf{Law 1}: The complement of a sum equals the product of
complements \(\overline{A + B} = \overline{A} \cdot \overline{B}\)

\textbf{Law 2}: The complement of a product equals the sum of
complements \(\overline{A \cdot B} = \overline{A} + \overline{B}\)


{\def\LTcaptype{none} % do not increment counter
\vspace{-5pt}
\captionof{table}{De-Morgan's Laws Verification}
\vspace{-10pt}
\begin{longtable}[]{@{}
  >{\raggedright\arraybackslash}p{(\linewidth - 12\tabcolsep) * \real{0.0333}}
  >{\raggedright\arraybackslash}p{(\linewidth - 12\tabcolsep) * \real{0.0333}}
  >{\raggedright\arraybackslash}p{(\linewidth - 12\tabcolsep) * \real{0.0556}}
  >{\raggedright\arraybackslash}p{(\linewidth - 12\tabcolsep) * \real{0.1889}}
  >{\raggedright\arraybackslash}p{(\linewidth - 12\tabcolsep) * \real{0.1778}}
  >{\raggedright\arraybackslash}p{(\linewidth - 12\tabcolsep) * \real{0.1778}}
  >{\raggedright\arraybackslash}p{(\linewidth - 12\tabcolsep) * \real{0.3333}}@{}}
\toprule\noalign{}
\begin{minipage}[b]{\linewidth}\raggedright
A
\end{minipage} & \begin{minipage}[b]{\linewidth}\raggedright
B
\end{minipage} & \begin{minipage}[b]{\linewidth}\raggedright
A+B
\end{minipage} & \begin{minipage}[b]{\linewidth}\raggedright
\(\overline{A+B}\)
\end{minipage} & \begin{minipage}[b]{\linewidth}\raggedright
\(\overline{A}\)
\end{minipage} & \begin{minipage}[b]{\linewidth}\raggedright
\(\overline{B}\)
\end{minipage} & \begin{minipage}[b]{\linewidth}\raggedright
\(\overline{A}\cdot\overline{B}\)
\end{minipage} \\
\midrule\noalign{}
\endhead
\bottomrule\noalign{}
\endlastfoot
0 & 0 & 0 & 1 & 1 & 1 & 1 \\
0 & 1 & 1 & 0 & 1 & 0 & 0 \\
1 & 0 & 1 & 0 & 0 & 1 & 0 \\
1 & 1 & 1 & 0 & 0 & 0 & 0 \\
\end{longtable}
}

\end{solutionbox}
\begin{mnemonicbox}
``NOT over OR becomes AND of NOTs, NOT over AND
becomes OR of NOTs''

\end{mnemonicbox}
\subsection*{Question 1(b) [4 marks]}\label{q1b}

\textbf{Convert following decimal number into binary and octal number
(i) 215 (ii) 59}

\begin{solutionbox}

\textbf{Binary Conversion:}

\textbf{For 215:}

\begin{itemize}
\tightlist
\item
  Divide by 2 repeatedly: 215/2 = 107 remainder 1
\item
  107/2 = 53 remainder 1
\item
  53/2 = 26 remainder 1
\item
  26/2 = 13 remainder 0
\item
  13/2 = 6 remainder 1
\item
  6/2 = 3 remainder 0
\item
  3/2 = 1 remainder 1
\item
  1/2 = 0 remainder 1
\item
  Therefore, (215)_{1}_{0} = (11010111)_{2}
\end{itemize}

\textbf{For 59:}

\begin{itemize}
\tightlist
\item
  Divide by 2 repeatedly: 59/2 = 29 remainder 1
\item
  29/2 = 14 remainder 1
\item
  14/2 = 7 remainder 0
\item
  7/2 = 3 remainder 1
\item
  3/2 = 1 remainder 1
\item
  1/2 = 0 remainder 1
\item
  Therefore, (59)_{1}_{0} = (111011)_{2}
\end{itemize}

\textbf{Octal Conversion:}

\textbf{For 215:}

\begin{itemize}
\tightlist
\item
  Divide by 8 repeatedly: 215/8 = 26 remainder 7
\item
  26/8 = 3 remainder 2
\item
  3/8 = 0 remainder 3
\item
  Therefore, (215)_{1}_{0} = (327)_{8}
\end{itemize}

\textbf{For 59:}

\begin{itemize}
\tightlist
\item
  Divide by 8 repeatedly: 59/8 = 7 remainder 3
\item
  7/8 = 0 remainder 7
\item
  Therefore, (59)_{1}_{0} = (73)_{8}
\end{itemize}


{\def\LTcaptype{none} % do not increment counter
\vspace{-5pt}
\captionof{table}{Number Conversion Summary}
\vspace{-10pt}
\begin{longtable}[]{@{}lll@{}}
\toprule\noalign{}
Decimal & Binary & Octal \\
\midrule\noalign{}
\endhead
\bottomrule\noalign{}
\endlastfoot
215 & 11010111 & 327 \\
59 & 111011 & 73 \\
\end{longtable}
}

\end{solutionbox}
\begin{mnemonicbox}
``Divide by base, read remainders bottom-up''

\end{mnemonicbox}
\subsection*{Question 1(c)(I) [2 marks]}\label{q1c}

\textbf{Write the base of decimal, binary, octal and hexadecimal number
system}

\begin{solutionbox}


{\def\LTcaptype{none} % do not increment counter
\vspace{-5pt}
\captionof{table}{Number System Bases}
\vspace{-10pt}
\begin{longtable}[]{@{}ll@{}}
\toprule\noalign{}
Number System & Base \\
\midrule\noalign{}
\endhead
\bottomrule\noalign{}
\endlastfoot
Decimal & 10 \\
Binary & 2 \\
Octal & 8 \\
Hexadecimal & 16 \\
\end{longtable}
}

\end{solutionbox}
\begin{mnemonicbox}
``D-B-O-H: 10-2-8-16''

\end{mnemonicbox}
\subsection*{Question 1(c)(II) [2
marks]}\label{q1c}

\textbf{(147)_{1}_{0} = (\_\_\_\_\_\_\_\_\_\_\_\_)_{2} =
(\_\_\_\_\_\_\_\_\_\_\_\_\_\_)_{1}_{6}}

\begin{solutionbox}

\textbf{Decimal to Binary conversion:}

\begin{itemize}
\tightlist
\item
  147/2 = 73 remainder 1
\item
  73/2 = 36 remainder 1
\item
  36/2 = 18 remainder 0
\item
  18/2 = 9 remainder 0
\item
  9/2 = 4 remainder 1
\item
  4/2 = 2 remainder 0
\item
  2/2 = 1 remainder 0
\item
  1/2 = 0 remainder 1
\item
  Therefore, (147)_{1}_{0} = (10010011)_{2}
\end{itemize}

\textbf{Decimal to Hexadecimal conversion:}

\begin{itemize}
\tightlist
\item
  Group binary digits in sets of 4: 1001 0011
\item
  Convert each group to hex: 1001 = 9, 0011 = 3
\item
  Therefore, (147)_{1}_{0} = (93)_{1}_{6}
\end{itemize}


{\def\LTcaptype{none} % do not increment counter
\vspace{-5pt}
\captionof{table}{Conversion Result}
\vspace{-10pt}
\begin{longtable}[]{@{}lll@{}}
\toprule\noalign{}
Decimal & Binary & Hexadecimal \\
\midrule\noalign{}
\endhead
\bottomrule\noalign{}
\endlastfoot
147 & 10010011 & 93 \\
\end{longtable}
}

\end{solutionbox}
\begin{mnemonicbox}
``Group by 4 from right for hex''

\end{mnemonicbox}
\subsection*{Question 1(c)(III) [3
marks]}\label{q1c}

\textbf{Convert following binary code into grey code (i) 1011 (ii) 1110}

\begin{solutionbox}

\textbf{Binary to Gray code conversion procedure:}

\begin{enumerate}
\tightlist
\item
  The MSB (leftmost bit) of the Gray code is the same as the MSB of the
  binary code
\item
  Other bits of the Gray code are obtained by XORing adjacent bits of
  the binary code
\end{enumerate}

\textbf{For 1011:}

\begin{itemize}
\tightlist
\item
  MSB of Gray = MSB of Binary = 1
\item
  Second bit = 1 XOR 0 = 1
\item
  Third bit = 0 XOR 1 = 1
\item
  Fourth bit = 1 XOR 1 = 0
\item
  Therefore, (1011)_{2} = (1110)ᵍᵣ_{a}ᵧ
\end{itemize}

\textbf{For 1110:}

\begin{itemize}
\tightlist
\item
  MSB of Gray = MSB of Binary = 1
\item
  Second bit = 1 XOR 1 = 0
\item
  Third bit = 1 XOR 1 = 0
\item
  Fourth bit = 1 XOR 0 = 1
\item
  Therefore, (1110)_{2} = (1001)ᵍᵣ_{a}ᵧ
\end{itemize}


{\def\LTcaptype{none} % do not increment counter
\vspace{-5pt}
\captionof{table}{Binary to Gray Code Conversion}
\vspace{-10pt}
\begin{longtable}[]{@{}lll@{}}
\toprule\noalign{}
Binary & Step-by-step & Gray Code \\
\midrule\noalign{}
\endhead
\bottomrule\noalign{}
\endlastfoot
1011 & 1, 1\oplus0=1, 0\oplus1=1, 1\oplus1=0 & 1110 \\
1110 & 1, 1\oplus1=0, 1\oplus1=0, 1\oplus0=1 & 1001 \\
\end{longtable}
}

\end{solutionbox}
\begin{mnemonicbox}
``Keep first, XOR the rest''

\end{mnemonicbox}
\subsection*{Question 1(c) [OR Question] (I) [2
marks]}\label{q1c}

\textbf{Write the full form of BCD and ASCII}

\begin{solutionbox}


{\def\LTcaptype{none} % do not increment counter
\vspace{-5pt}
\captionof{table}{Full Forms of BCD and ASCII}
\vspace{-10pt}
\begin{longtable}[]{@{}ll@{}}
\toprule\noalign{}
Abbreviation & Full Form \\
\midrule\noalign{}
\endhead
\bottomrule\noalign{}
\endlastfoot
BCD & Binary Coded Decimal \\
ASCII & American Standard Code for Information Interchange \\
\end{longtable}
}

\end{solutionbox}
\begin{mnemonicbox}
``Binary Codes Decimal values, American Standards
Code Information''

\end{mnemonicbox}
\subsection*{Question 1(c) [OR Question] (II) [2
marks]}\label{q1c}

\textbf{Write 1's and 2's complement of following binary numbers: (i)
1010 (ii) 1011}

\begin{solutionbox}

\textbf{1's Complement:} Invert all bits (change 0 to 1 and 1 to 0)
\textbf{2's Complement:} Take 1's complement and add 1

\textbf{For 1010:}

\begin{itemize}
\tightlist
\item
  1's complement: 0101
\item
  2's complement: 0101 + 1 = 0110
\end{itemize}

\textbf{For 1011:}

\begin{itemize}
\tightlist
\item
  1's complement: 0100
\item
  2's complement: 0100 + 1 = 0101
\end{itemize}


{\def\LTcaptype{none} % do not increment counter
\vspace{-5pt}
\captionof{table}{Complement Results}
\vspace{-10pt}
\begin{longtable}[]{@{}lll@{}}
\toprule\noalign{}
Binary & 1's Complement & 2's Complement \\
\midrule\noalign{}
\endhead
\bottomrule\noalign{}
\endlastfoot
1010 & 0101 & 0110 \\
1011 & 0100 & 0101 \\
\end{longtable}
}

\end{solutionbox}
\begin{mnemonicbox}
``Flip all bits for 1's, Add one more for 2's''

\end{mnemonicbox}
\subsection*{Question 1(c) [OR Question] (III) [3
marks]}\label{q1c}

\textbf{Perform subtraction using 2's complement method (i) (110110)_{2} --
(101010)_{2}}

\begin{solutionbox}

To subtract using 2's complement method:

\begin{enumerate}
\tightlist
\item
  Find 2's complement of subtrahend
\item
  Add it to the minuend
\item
  Discard any carry beyond the bit width
\end{enumerate}

\textbf{Subtraction: (110110)_{2} -- (101010)_{2}}

\textbf{Step 1:} Find 2's complement of 101010

\begin{itemize}
\tightlist
\item
  1's complement of 101010 = 010101
\item
  2's complement = 010101 + 1 = 010110
\end{itemize}

\textbf{Step 2:} Add 110110 + 010110

\begin{verbatim}
  1 1 1 1 1
  1 1 0 1 1 0
+ 0 1 0 1 1 0
--------------
  0 0 1 1 0 0
\end{verbatim}

\textbf{Step 3:} Result is 001100 = (12)_{1}_{0}


{\def\LTcaptype{none} % do not increment counter
\vspace{-5pt}
\captionof{table}{Subtraction Process}
\vspace{-10pt}
\begin{longtable}[]{@{}lll@{}}
\toprule\noalign{}
Step & Operation & Result \\
\midrule\noalign{}
\endhead
\bottomrule\noalign{}
\endlastfoot
1 & 2's complement of 101010 & 010110 \\
2 & Add 110110 + 010110 & 001100 \\
3 & Final result (decimal) & 12 \\
\end{longtable}
}

\end{solutionbox}
\begin{mnemonicbox}
``Complement the subtracted, add them up, carry goes
away''

\end{mnemonicbox}
\subsection*{Question 2(a) [3 marks]}\label{q2a}

\textbf{Draw logic circuit of AND, OR and NOT gate using NAND gate only}

\begin{solutionbox}

\textbf{AND gate using NAND gates:}

\begin{itemize}
\tightlist
\item
  AND gate = NAND gate followed by NOT gate (NAND gate)
\end{itemize}

\textbf{OR gate using NAND gates:}

\begin{itemize}
\tightlist
\item
  OR gate = Apply NOT (NAND gate) to both inputs, then NAND those
  results
\end{itemize}

\textbf{NOT gate using NAND gate:}

\begin{itemize}
\tightlist
\item
  NOT gate = NAND gate with both inputs tied together
\end{itemize}

\begin{center}
\textbf{Mermaid Diagram (Code)}
\begin{verbatim}
{Shaded}
{Highlighting}[]
graph TD
  subgraph "NOT Gate"
    A1[A] {-{-}{} NAND1([NAND])}
    A1 {-{-}{} NAND1}
    NAND1 {-{-}{} notA[NOT A]}
  end
  
  subgraph "AND Gate"
    B1[B] {-{-}{} NAND2([NAND])}
    C1[C] {-{-}{} NAND2}
    NAND2 {-{-}{} NAND3([NAND])}
    NAND2 {-{-}{} NAND3}
    NAND3 {-{-}{} andResult[B AND C]}
  end
  
  subgraph "OR Gate"
    D1[D] {-{-}{} NAND4([NAND])}
    D1 {-{-}{} NAND4}
    E1[E] {-{-}{} NAND5([NAND])}
    E1 {-{-}{} NAND5}
    NAND4 {-{-}{} NAND6([NAND])}
    NAND5 {-{-}{} NAND6}
    NAND6 {-{-}{} orResult[D OR E]}
  end
{Highlighting}
{Shaded}
\end{verbatim}
\end{center}

\end{solutionbox}
\begin{mnemonicbox}
``NAND alone for NOT, NAND twice for AND, NAND each
then NAND again for OR''

\end{mnemonicbox}
\subsection*{Question 2(b) [4 marks]}\label{q2b}

\textbf{Draw/Write logic symbol, truth table and equation of following
logic gates (i) XOR gate (ii) OR gate}

\begin{solutionbox}

\textbf{XOR Gate:}

\textbf{Logic Symbol:}

\begin{center}
\textbf{Mermaid Diagram (Code)}
\begin{verbatim}
{Shaded}
{Highlighting}[]
graph LR
    A[A] {-{-}{} XOR([])}
    B[B] {-{-}{} XOR}
    XOR {-{-}{} Y[Y]}
{Highlighting}
{Shaded}
\end{verbatim}
\end{center}

\textbf{Truth Table:}

{\def\LTcaptype{none} % do not increment counter
\begin{longtable}[]{@{}lll@{}}
\toprule\noalign{}
A & B & Y (A\oplusB) \\
\midrule\noalign{}
\endhead
\bottomrule\noalign{}
\endlastfoot
0 & 0 & 0 \\
0 & 1 & 1 \\
1 & 0 & 1 \\
1 & 1 & 0 \\
\end{longtable}
}

\textbf{Boolean Equation:} Y = A\oplusB = A'B + AB'

\textbf{OR Gate:}

\textbf{Logic Symbol:}

\begin{center}
\textbf{Mermaid Diagram (Code)}
\begin{verbatim}
{Shaded}
{Highlighting}[]
graph LR
    A[A] {-{-}{} OR([1])}
    B[B] {-{-}{} OR}
    OR {-{-}{} Y[Y]}
{Highlighting}
{Shaded}
\end{verbatim}
\end{center}

\textbf{Truth Table:}

{\def\LTcaptype{none} % do not increment counter
\begin{longtable}[]{@{}lll@{}}
\toprule\noalign{}
A & B & Y (A+B) \\
\midrule\noalign{}
\endhead
\bottomrule\noalign{}
\endlastfoot
0 & 0 & 0 \\
0 & 1 & 1 \\
1 & 0 & 1 \\
1 & 1 & 1 \\
\end{longtable}
}

\textbf{Boolean Equation:} Y = A+B

\end{solutionbox}
\begin{mnemonicbox}
``XOR: Exclusive OR - one or the other but not both;
OR: one or the other or both''

\end{mnemonicbox}
\subsection*{Question 2(c)(I) [3 marks]}\label{q2c}

\textbf{Simplify the Boolean expression using algebraic method Y = A +
B[AC + (B + C̅)D]}

\begin{solutionbox}

\textbf{Step-by-step simplification:}

Y = A + B[AC + (B + C̅)D]

Y = A + B[AC + BD + C̅D]

Y = A + BAC +

BBD + BC̅D

Y = A + BAC + BD + BC̅D (Since BB = B)


\textbf{Apply absorption law (X + XY = X):} Y = A + AC + BD + BC̅D (Since
A + BAC = A + AC)

Y = A + BD + BC̅D (Since A + AC = A)

Y = A + B(D + C̅D)

Y = A + BD + BC̅D

Y = A + BD(1 + C̅)

Y = A + BD (Since 1 + C̅ = 1)


\textbf{Final expression:} Y = A + BD


{\def\LTcaptype{none} % do not increment counter
\vspace{-5pt}
\captionof{table}{Simplification Steps}
\vspace{-10pt}
\begin{longtable}[]{@{}lll@{}}
\toprule\noalign{}
Step & Expression & Law Applied \\
\midrule\noalign{}
\endhead
\bottomrule\noalign{}
\endlastfoot
1 & A + B[AC + (B + C̅)D] & Initial \\
2 & A + B[AC + BD + C̅D] & Distributive \\
3 & A + BAC + BBD + BC̅D & Distributive \\
4 & A + BAC + BD + BC̅D & Idempotent (BB = B) \\
5 & A + AC + BD + BC̅D & Absorption \\
6 & A + BD + BC̅D & Absorption (A+AC=A) \\
7 & A + B(D + C̅D) & Factoring \\
8 & A + BD & Complementary law \\
\end{longtable}
}

\end{solutionbox}
\begin{mnemonicbox}
``Always look for idempotence, absorption, and
complement patterns''

\end{mnemonicbox}
\subsection*{Question 2(c)(II) [4
marks]}\label{q2c}

\textbf{Simplify the Boolean expression using Karnaugh Map F(A,B,C) =
Σm(0, 2, 3, 4, 5, 6)}

\begin{solutionbox}

\textbf{Create K-map for F(A,B,C) = Σm(0, 2, 3, 4, 5, 6):}

\textbf{K-map:}

\begin{verbatim}
    BC
   00 01 11 10
A 0| 1  0  0  1
  1| 1  1  0  1
\end{verbatim}

\textbf{Group the 1s:}

\begin{itemize}
\tightlist
\item
  Group 1: m(0,4) - corresponds to A'B'C'
\item
  Group 2: m(2,6) - corresponds to B'C
\item
  Group 3: m(4,5) - corresponds to AB'
\end{itemize}

\textbf{Simplified expression:} F(A,B,C) = B'C + A'B'C' + AB'

\textbf{Let's simplify further:} F(A,B,C) = B'C + B'C'(A' + A) F(A,B,C)
= B'C + B'C' F(A,B,C) = B'(C + C') F(A,B,C) = B'

\textbf{Final expression:} F(A,B,C) = B'

\textbf{Diagram: K-map grouping}

\begin{verbatim}
    BC
   00 01 11 10
A 0| 1  0  0  1
  1| 1  1  0  1
    ↑_____↑  ↑_____↑
       │        │
    Group 1   Group 2
    
    ↑______↑
    │  
 Group 3
\end{verbatim}

\end{solutionbox}
\begin{mnemonicbox}
``Group adjacent 1s in powers of 2''

\end{mnemonicbox}
\subsection*{Question 2 [OR Question] (a) [3
marks]}\label{question-2-or-question-a-3-marks}

\textbf{Draw logic circuit of AND, OR and NOT gate using NOR gate only}

\begin{solutionbox}

\textbf{NOT gate using NOR gate:}

\begin{itemize}
\tightlist
\item
  NOT gate = NOR gate with both inputs tied together
\end{itemize}

\textbf{AND gate using NOR gates:}

\begin{itemize}
\tightlist
\item
  AND gate = Apply NOT (NOR gate) to both inputs, then NOR those results
  again
\end{itemize}

\textbf{OR gate using NOR gates:}

\begin{itemize}
\tightlist
\item
  OR gate = NOR gate followed by NOT gate (NOR gate)
\end{itemize}

\begin{center}
\textbf{Mermaid Diagram (Code)}
\begin{verbatim}
{Shaded}
{Highlighting}[]
graph TD
  subgraph "NOT Gate"
    A1[A] {-{-}{} NOR1([NOR])}
    A1 {-{-}{} NOR1}
    NOR1 {-{-}{} notA[NOT A]}
  end
  
  subgraph "AND Gate"
    B1[B] {-{-}{} NOR2([NOR])}
    B1 {-{-}{} NOR2}
    C1[C] {-{-}{} NOR3([NOR])}
    C1 {-{-}{} NOR3}
    NOR2 {-{-}{} NOR4([NOR])}
    NOR3 {-{-}{} NOR4}
    NOR4 {-{-}{} andResult[B AND C]}
  end
  
  subgraph "OR Gate"
    D1[D] {-{-}{} NOR5([NOR])}
    E1[E] {-{-}{} NOR5}
    NOR5 {-{-}{} NOR6([NOR])}
    NOR5 {-{-}{} NOR6}
    NOR6 {-{-}{} orResult[D OR E]}
  end
{Highlighting}
{Shaded}
\end{verbatim}
\end{center}

\end{solutionbox}
\begin{mnemonicbox}
``NOR alone for NOT, NOT each then NOR for AND,
Double NOR for OR''

\end{mnemonicbox}
\subsection*{Question 2 [OR Question] (b) [4
marks]}\label{question-2-or-question-b-4-marks}

\textbf{Draw/Write logic symbol, truth table and equation of following
logic gates (i) NOR gate (ii) AND gate}

\begin{solutionbox}

\textbf{NOR Gate:}

\textbf{Logic Symbol:}

\begin{center}
\textbf{Mermaid Diagram (Code)}
\begin{verbatim}
{Shaded}
{Highlighting}[]
graph LR
    A[A] {-{-}{} NOR([1 with bubble])}
    B[B] {-{-}{} NOR}
    NOR {-{-}{} Y[Y]}
{Highlighting}
{Shaded}
\end{verbatim}
\end{center}

\textbf{Truth Table:}

{\def\LTcaptype{none} % do not increment counter
\begin{longtable}[]{@{}lll@{}}
\toprule\noalign{}
A & B & Y (A+B)' \\
\midrule\noalign{}
\endhead
\bottomrule\noalign{}
\endlastfoot
0 & 0 & 1 \\
0 & 1 & 0 \\
1 & 0 & 0 \\
1 & 1 & 0 \\
\end{longtable}
}

\textbf{Boolean Equation:} Y = (A+B)' = A'B'

\textbf{AND Gate:}

\textbf{Logic Symbol:}

\begin{center}
\textbf{Mermaid Diagram (Code)}
\begin{verbatim}
{Shaded}
{Highlighting}[]
graph LR
    A[A] {-{-}{} AND([\&])}
    B[B] {-{-}{} AND}
    AND {-{-}{} Y[Y]}
{Highlighting}
{Shaded}
\end{verbatim}
\end{center}

\textbf{Truth Table:}

{\def\LTcaptype{none} % do not increment counter
\begin{longtable}[]{@{}lll@{}}
\toprule\noalign{}
A & B & Y (A·B) \\
\midrule\noalign{}
\endhead
\bottomrule\noalign{}
\endlastfoot
0 & 0 & 0 \\
0 & 1 & 0 \\
1 & 0 & 0 \\
1 & 1 & 1 \\
\end{longtable}
}

\textbf{Boolean Equation:} Y = A·B

\end{solutionbox}
\begin{mnemonicbox}
``NOR: NOT OR - neither one nor the other; AND: both
must be 1''

\end{mnemonicbox}
\subsection*{Question 2 [OR Question] (c) [7
marks]}\label{question-2-or-question-c-7-marks}

\textbf{Write the Boolean expression of Q for above logic circuit.
Simplify the Boolean expression of Q and draw the logic circuit of
simplified circuit using AND-OR-Invert method}

\begin{solutionbox}

\textbf{Step 1:} Write Boolean expression from the circuit: Q = (A + B)
· (B + C · ((B + C)`))

Q = (A + B) · (B + C · (B' · C'))

Q = (A + B) ·

(B + C · B' · C')

\textbf{Step 2:} Simplify the expression:

\begin{itemize}
\tightlist
\item
  Note that C · C' = 0
\item
  Therefore, C · B' · C' = 0
\item
So

Q = (A + B) · (B + 0) = (A + B) ·

B = A·B + B·B = A·B +

B = B + A·B

  = B(1 + A) = B
\end{itemize}

\textbf{Step 3:} Final simplified expression: Q = B

\textbf{Step 4:} AND-OR-Invert implementation of Q = B:

\begin{itemize}
\tightlist
\item
  This is simply a wire from input B to output Q
\end{itemize}

\begin{center}
\textbf{Mermaid Diagram (Code)}
\begin{verbatim}
{Shaded}
{Highlighting}[]
graph LR
    B[B] {-{-}{} Q[Q]}
{Highlighting}
{Shaded}
\end{verbatim}
\end{center}


{\def\LTcaptype{none} % do not increment counter
\vspace{-5pt}
\captionof{table}{Simplification Steps}
\vspace{-10pt}
\begin{longtable}[]{@{}lll@{}}
\toprule\noalign{}
Step & Expression & Simplification \\
\midrule\noalign{}
\endhead
\bottomrule\noalign{}
\endlastfoot
1 & (A + B) · (B + C · ((B + C)')) & Original expression \\
2 & (A + B) · (B + C · B' · C') & Applying De Morgan's \\
3 & (A + B) · (B + 0) & C · C' = 0 \\
4 & (A + B) · B & Simplifying \\
5 & A·B + B·B & Distributive property \\
6 & A·B + B & Idempotent property (B·B=B) \\
7 & B(1 + A) & Factoring \\
8 & B & 1 + A = 1 \\
\end{longtable}
}

\end{solutionbox}
\begin{mnemonicbox}
``When complementary variables multiply, they zero
out''

\end{mnemonicbox}
\subsection*{Question 3(a) [3 marks]}\label{q3a}

\textbf{Define combinational circuit. Give two examples of combinational
circuits}

\begin{solutionbox}

\textbf{Combinational circuit:} A digital circuit whose output depends
only on the current input values and not on previous inputs or states.
In combinational circuits, there is no memory or feedback.

\textbf{Key characteristics:}

\begin{itemize}
\tightlist
\item
  Output depends only on current inputs
\item
  No memory elements
\item
  No feedback paths
\end{itemize}

\textbf{Examples of combinational circuits:}

\begin{enumerate}
\tightlist
\item
  Multiplexers (MUX)
\item
  Decoders
\item
  Adders/Subtractors
\item
  Encoders
\item
  Comparators
\end{enumerate}


{\def\LTcaptype{none} % do not increment counter
\vspace{-5pt}
\captionof{table}{Combinational vs Sequential Circuits}
\vspace{-10pt}
\begin{longtable}[]{@{}
  >{\raggedright\arraybackslash}p{(\linewidth - 4\tabcolsep) * \real{0.2807}}
  >{\raggedright\arraybackslash}p{(\linewidth - 4\tabcolsep) * \real{0.3860}}
  >{\raggedright\arraybackslash}p{(\linewidth - 4\tabcolsep) * \real{0.3333}}@{}}
\toprule\noalign{}
\begin{minipage}[b]{\linewidth}\raggedright
Characteristic
\end{minipage} & \begin{minipage}[b]{\linewidth}\raggedright
Combinational Circuit
\end{minipage} & \begin{minipage}[b]{\linewidth}\raggedright
Sequential Circuit
\end{minipage} \\
\midrule\noalign{}
\endhead
\bottomrule\noalign{}
\endlastfoot
Memory & No & Yes \\
Feedback & No & Usually \\
Output depends on & Current inputs only & Current and previous inputs \\
Examples & Multiplexers, Adders & Flip-flops, Counters \\
\end{longtable}
}

\end{solutionbox}
\begin{mnemonicbox}
``Combinational circuits: Current In, Current Out -
no memory about''

\end{mnemonicbox}
\subsection*{Question 3(b) [4 marks]}\label{q3b}

\textbf{Explain half adder using logic circuit and truth table}

\begin{solutionbox}

\textbf{Half Adder:} A combinational circuit that adds two binary digits
and produces sum and carry outputs.

\textbf{Logic Circuit:}

\begin{center}
\textbf{Mermaid Diagram (Code)}
\begin{verbatim}
{Shaded}
{Highlighting}[]
graph LR
    A[A] {-{-}{} XOR([])}
    B[B] {-{-}{} XOR}
    XOR {-{-}{} S[Sum]}
    A {-{-}{} AND([\&])}
    B {-{-}{} AND}
    AND {-{-}{} C[Carry]}
{Highlighting}
{Shaded}
\end{verbatim}
\end{center}

\textbf{Truth Table:}

{\def\LTcaptype{none} % do not increment counter
\begin{longtable}[]{@{}llll@{}}
\toprule\noalign{}
A & B & Sum & Carry \\
\midrule\noalign{}
\endhead
\bottomrule\noalign{}
\endlastfoot
0 & 0 & 0 & 0 \\
0 & 1 & 1 & 0 \\
1 & 0 & 1 & 0 \\
1 & 1 & 0 & 1 \\
\end{longtable}
}

\textbf{Boolean Equations:}

\begin{itemize}
\tightlist
\item
Sum = A \oplus

B = A'B + AB'

\item
  Carry = A · B
\end{itemize}

\textbf{Limitations:}

\begin{itemize}
\tightlist
\item
  Cannot add three binary digits
\item
  Cannot accommodate carry input from previous stage
\end{itemize}

\end{solutionbox}
\begin{mnemonicbox}
``XOR for Sum, AND for Carry''

\end{mnemonicbox}
\subsection*{Question 3(c)(I) [3 marks]}\label{q3c}

\textbf{Explain multiplexer in brief}

\begin{solutionbox}

\textbf{Multiplexer (MUX):} A combinational circuit that selects one of
several input signals and forwards it to a single output line based on
select lines.

\textbf{Key features:}

\begin{itemize}
\tightlist
\item
  Acts as a digital switch
\item
  Has 2^{n} data inputs, n select lines, and 1 output
\item
  Select lines determine which input is connected to output
\end{itemize}

\textbf{Common multiplexers:}

\begin{itemize}
\tightlist
\item
  2:1 MUX (1 select line)
\item
  4:1 MUX (2 select lines)
\item
  8:1 MUX (3 select lines)
\end{itemize}

\textbf{Basic structure:}

\begin{center}
\textbf{Mermaid Diagram (Code)}
\begin{verbatim}
{Shaded}
{Highlighting}[]
graph TD
    I0[I0] {-{-}{} MUX([MUX])}
    I1[I1] {-{-}{} MUX}
    In[...] {-{-}{} MUX}
    I2n{-1[I2\^{}n{-}1] {-}{-}{} MUX}
    S[Select Lines] {-{-}{} MUX}
    MUX {-{-}{} Y[Output Y]}
{Highlighting}
{Shaded}
\end{verbatim}
\end{center}

\textbf{Applications:}

\begin{itemize}
\tightlist
\item
  Data routing
\item
  Data selection
\item
  Parallel to serial conversion
\item
  Implementation of Boolean functions
\end{itemize}

\end{solutionbox}
\begin{mnemonicbox}
``Many In, Selection picks, One Out''

\end{mnemonicbox}
\subsection*{Question 3(c)(II) [4
marks]}\label{q3c}

\textbf{Design 8:1 multiplexer. Write its truth table and draw its logic
circuit}

\begin{solutionbox}

\textbf{8:1 Multiplexer Design:}

\begin{itemize}
\tightlist
\item
  8 data inputs (I_{0} to I_{7})
\item
  3 select lines (S_{2}, S_{1}, S_{0})
\item
  1 output (Y)
\end{itemize}

\textbf{Truth Table:}

{\def\LTcaptype{none} % do not increment counter
\begin{longtable}[]{@{}llll@{}}
\toprule\noalign{}
S_{2} & S_{1} & S_{0} & Output Y \\
\midrule\noalign{}
\endhead
\bottomrule\noalign{}
\endlastfoot
0 & 0 & 0 & I_{0} \\
0 & 0 & 1 & I_{1} \\
0 & 1 & 0 & I_{2} \\
0 & 1 & 1 & I_{3} \\
1 & 0 & 0 & I_{4} \\
1 & 0 & 1 & I_{5} \\
1 & 1 & 0 & I_{6} \\
1 & 1 & 1 & I_{7} \\
\end{longtable}
}

\textbf{Boolean Equation:} Y = S_{2}'·S_{1}'·S_{0}'·I_{0} + S_{2}'·S_{1}'·S_{0}·I_{1} +
S_{2}'·S_{1}·S_{0}'·I_{2} + S_{2}'·S_{1}·S_{0}·I_{3} + S_{2}·S_{1}'·S_{0}'·I_{4} + S_{2}·S_{1}'·S_{0}·I_{5} +
S_{2}·S_{1}·S_{0}'·I_{6} + S_{2}·S_{1}·S_{0}·I_{7}

\textbf{Logic Circuit:}

\begin{center}
\textbf{Mermaid Diagram (Code)}
\begin{verbatim}
{Shaded}
{Highlighting}[]
graph TD
    I0[I0] {-{-}{} AND0([\&])}
    I1[I1] {-{-}{} AND1([\&])}
    I2[I2] {-{-}{} AND2([\&])}
    I3[I3] {-{-}{} AND3([\&])}
    I4[I4] {-{-}{} AND4([\&])}
    I5[I5] {-{-}{} AND5([\&])}
    I6[I6] {-{-}{} AND6([\&])}
    I7[I7] {-{-}{} AND7([\&])}

    S0n["S0{"] {-}{-}{} AND0}
    S1n["S1{"] {-}{-}{} AND0}
    S2n["S2{"] {-}{-}{} AND0}
    
    S0["S0"] {-{-}{} AND1}
    S1n {-{-}{} AND1}
    S2n {-{-}{} AND1}
    
    S0n {-{-}{} AND2}
    S1["S1"] {-{-}{} AND2}
    S2n {-{-}{} AND2}
    
    S0 {-{-}{} AND3}
    S1 {-{-}{} AND3}
    S2n {-{-}{} AND3}
    
    S0n {-{-}{} AND4}
    S1n {-{-}{} AND4}
    S2["S2"] {-{-}{} AND4}
    
    S0 {-{-}{} AND5}
    S1n {-{-}{} AND5}
    S2 {-{-}{} AND5}
    
    S0n {-{-}{} AND6}
    S1 {-{-}{} AND6}
    S2 {-{-}{} AND6}
    
    S0 {-{-}{} AND7}
    S1 {-{-}{} AND7}
    S2 {-{-}{} AND7}
    
    AND0 {-{-}{} OR([1])}
    AND1 {-{-}{} OR}
    AND2 {-{-}{} OR}
    AND3 {-{-}{} OR}
    AND4 {-{-}{} OR}
    AND5 {-{-}{} OR}
    AND6 {-{-}{} OR}
    AND7 {-{-}{} OR}
    
    OR {-{-}{} Y[Y]}
{Highlighting}
{Shaded}
\end{verbatim}
\end{center}

\end{solutionbox}
\begin{mnemonicbox}
``Eight inputs, three selects, decode and OR to
output''

\end{mnemonicbox}
\subsection*{Question 3 [OR Question] (a) [3
marks]}\label{question-3-or-question-a-3-marks}

\textbf{Draw the block diagram of 4-bit binary parallel adder}

\begin{solutionbox}

\textbf{4-bit Binary Parallel Adder:} A circuit that adds two 4-bit
binary numbers and produces a 4-bit sum and a carry output.

\begin{center}
\textbf{Mermaid Diagram (Code)}
\begin{verbatim}
{Shaded}
{Highlighting}[]
graph LR
    A0[A0] {-{-}{} FA0[FA]}
    B0[B0] {-{-}{} FA0}
    Cin[Cin=0] {-{-}{} FA0}
    FA0 {-{-}{} S0[S0]}

    A1[A1] {-{-}{} FA1[FA]}
    B1[B1] {-{-}{} FA1}
    FA0 {-{-}C1{-}{-}{} FA1}
    FA1 {-{-}{} S1[S1]}
    
    A2[A2] {-{-}{} FA2[FA]}
    B2[B2] {-{-}{} FA2}
    FA1 {-{-}C2{-}{-}{} FA2}
    FA2 {-{-}{} S2[S2]}
    
    A3[A3] {-{-}{} FA3[FA]}
    B3[B3] {-{-}{} FA3}
    FA2 {-{-}C3{-}{-}{} FA3}
    FA3 {-{-}{} S3[S3]}
    
    FA3 {-{-}C4{-}{-}{} Cout[Cout]}
{Highlighting}
{Shaded}
\end{verbatim}
\end{center}

\textbf{Components:}

\begin{itemize}
\tightlist
\item
  Four full adders (FA) connected in cascade
\item
  Each FA adds corresponding bits and the carry from previous stage
\item
  Initial carry-in (Cin) is typically 0
\end{itemize}

\end{solutionbox}
\begin{mnemonicbox}
``Four FAs linked, carries ripple through''

\end{mnemonicbox}
\subsection*{Question 3 [OR Question] (b) [4
marks]}\label{question-3-or-question-b-4-marks}

\textbf{Explain full adder using logic circuit and truth table}

\begin{solutionbox}

\textbf{Full Adder:} A combinational circuit that adds three binary
digits (two inputs and a carry-in) and produces sum and carry outputs.

\textbf{Logic Circuit:}

\begin{center}
\textbf{Mermaid Diagram (Code)}
\begin{verbatim}
{Shaded}
{Highlighting}[]
graph LR
    A[A] {-{-}{} XOR1([])}
    B[B] {-{-}{} XOR1}
    XOR1 {-{-}{} XOR2([])}
    Cin[Cin] {-{-}{} XOR2}
    XOR2 {-{-}{} Sum[Sum]}

    A {-{-}{} AND1([\&])}
    B {-{-}{} AND1}
    XOR1 {-{-}{} AND2([\&])}
    Cin {-{-}{} AND2}
    AND1 {-{-}{} OR([1])}
    AND2 {-{-}{} OR}
    OR {-{-}{} Cout[Carry out]}
{Highlighting}
{Shaded}
\end{verbatim}
\end{center}

\textbf{Truth Table:}

{\def\LTcaptype{none} % do not increment counter
\begin{longtable}[]{@{}lllll@{}}
\toprule\noalign{}
A & B & Cin & Sum & Cout \\
\midrule\noalign{}
\endhead
\bottomrule\noalign{}
\endlastfoot
0 & 0 & 0 & 0 & 0 \\
0 & 0 & 1 & 1 & 0 \\
0 & 1 & 0 & 1 & 0 \\
0 & 1 & 1 & 0 & 1 \\
1 & 0 & 0 & 1 & 0 \\
1 & 0 & 1 & 0 & 1 \\
1 & 1 & 0 & 0 & 1 \\
1 & 1 & 1 & 1 & 1 \\
\end{longtable}
}

\textbf{Boolean Equations:}

\begin{itemize}
\tightlist
\item
  Sum = A \oplus B \oplus Cin
\item
  Cout = A·B + (A\oplusB)·Cin
\end{itemize}

\end{solutionbox}
\begin{mnemonicbox}
``XOR all three for Sum, OR the ANDs for Carry''

\end{mnemonicbox}
\subsection*{Question 3 [OR Question] (c) (I) [3
marks]}\label{question-3-or-question-c-i-3-marks}

\textbf{Explain 4:1 multiplexer using logic circuit and truth table}

\begin{solutionbox}

\textbf{4:1 Multiplexer:} A digital switch that selects one of four
input lines and connects it to the output based on two select lines.

\textbf{Logic Circuit:}

\begin{center}
\textbf{Mermaid Diagram (Code)}
\begin{verbatim}
{Shaded}
{Highlighting}[]
graph TD
    I0[I0] {-{-}{} AND0([\&])}
    I1[I1] {-{-}{} AND1([\&])}
    I2[I2] {-{-}{} AND2([\&])}
    I3[I3] {-{-}{} AND3([\&])}

    S0n["S0{"] {-}{-}{} AND0}
    S1n["S1{"] {-}{-}{} AND0}
    
    S0["S0"] {-{-}{} AND1}
    S1n {-{-}{} AND1}
    
    S0n {-{-}{} AND2}
    S1["S1"] {-{-}{} AND2}
    
    S0 {-{-}{} AND3}
    S1 {-{-}{} AND3}
    
    AND0 {-{-}{} OR([1])}
    AND1 {-{-}{} OR}
    AND2 {-{-}{} OR}
    AND3 {-{-}{} OR}
    
    OR {-{-}{} Y[Y]}
{Highlighting}
{Shaded}
\end{verbatim}
\end{center}

\textbf{Truth Table:}

{\def\LTcaptype{none} % do not increment counter
\begin{longtable}[]{@{}lll@{}}
\toprule\noalign{}
S1 & S0 & Output Y \\
\midrule\noalign{}
\endhead
\bottomrule\noalign{}
\endlastfoot
0 & 0 & I0 \\
0 & 1 & I1 \\
1 & 0 & I2 \\
1 & 1 & I3 \\
\end{longtable}
}

\textbf{Boolean Equation:} Y = S1'·S0'·I0 + S1'·S0·I1 + S1·S0'·I2 +
S1·S0·I3

\end{solutionbox}
\begin{mnemonicbox}
``Two select lines choose one of four inputs''

\end{mnemonicbox}
\subsection*{Question 3 [OR Question] (c) (II) [4
marks]}\label{question-3-or-question-c-ii-4-marks}

\textbf{Design 8:1 multiplexer using two 4:1 multiplexer.}

\begin{solutionbox}

\textbf{Design approach:} Use two 4:1 MUXes and one 2:1 MUX to create an
8:1 MUX.

\begin{enumerate}
\tightlist
\item
  First 4:1 MUX handles inputs I0-I3 using select lines S0,S1
\item
  Second 4:1 MUX handles inputs I4-I7 using select lines S0,S1
\item
  2:1 MUX selects between outputs of the two 4:1 MUXes using S2
\end{enumerate}

\textbf{Block Diagram:}

\begin{center}
\textbf{Mermaid Diagram (Code)}
\begin{verbatim}
{Shaded}
{Highlighting}[]
graph TD
    I0[I0] {-{-}{} MUX1([4:1 MUX])}
    I1[I1] {-{-}{} MUX1}
    I2[I2] {-{-}{} MUX1}
    I3[I3] {-{-}{} MUX1}

    I4[I4] {-{-}{} MUX2([4:1 MUX])}
    I5[I5] {-{-}{} MUX2}
    I6[I6] {-{-}{} MUX2}
    I7[I7] {-{-}{} MUX2}
    
    S0[S0] {-{-}{} MUX1}
    S1[S1] {-{-}{} MUX1}
    S0 {-{-}{} MUX2}
    S1 {-{-}{} MUX2}
    
    MUX1 {-{-}{} MUX3([2:1 MUX])}
    MUX2 {-{-}{} MUX3}
    S2[S2] {-{-}{} MUX3}
    
    MUX3 {-{-}{} Y[Y]}
{Highlighting}
{Shaded}
\end{verbatim}
\end{center}

\textbf{Truth Table:}

{\def\LTcaptype{none} % do not increment counter
\begin{longtable}[]{@{}llll@{}}
\toprule\noalign{}
S2 & S1 & S0 & Output Y \\
\midrule\noalign{}
\endhead
\bottomrule\noalign{}
\endlastfoot
0 & 0 & 0 & I0 \\
0 & 0 & 1 & I1 \\
0 & 1 & 0 & I2 \\
0 & 1 & 1 & I3 \\
1 & 0 & 0 & I4 \\
1 & 0 & 1 & I5 \\
1 & 1 & 0 & I6 \\
1 & 1 & 1 & I7 \\
\end{longtable}
}

\end{solutionbox}
\begin{mnemonicbox}
``S0,S1 select from each 4:1 MUX, S2 selects between
them''

\end{mnemonicbox}
\subsection*{Question 4(a) [3 marks]}\label{q4a}

\textbf{Define sequential circuit. Give two examples of it}

\begin{solutionbox}

\textbf{Sequential Circuit:} A digital circuit whose output depends not
only on the current inputs but also on the past sequence of inputs
(history/previous states).

\textbf{Key characteristics:}

\begin{itemize}
\tightlist
\item
  Contains memory elements (flip-flops)
\item
  Output depends on both current inputs and previous states
\item
  Usually incorporates feedback paths
\item
  Requires clock signals for synchronization (for synchronous circuits)
\end{itemize}

\textbf{Examples of sequential circuits:}

\begin{enumerate}
\tightlist
\item
  Flip-flops (SR, JK, D, T)
\item
  Registers (shift registers)
\item
  Counters (binary, decade, ring counters)
\item
  State machines
\item
  Memory units
\end{enumerate}


{\def\LTcaptype{none} % do not increment counter
\vspace{-5pt}
\captionof{table}{Sequential vs Combinational Circuits}
\vspace{-10pt}
\begin{longtable}[]{@{}lll@{}}
\toprule\noalign{}
Characteristic & Sequential Circuit & Combinational Circuit \\
\midrule\noalign{}
\endhead
\bottomrule\noalign{}
\endlastfoot
Memory & Yes & No \\
Feedback & Usually & No \\
Output depends on & Current \& previous inputs & Current inputs only \\
Clock required & Usually & No \\
Examples & Flip-flops, Counters & Multiplexers, Adders \\
\end{longtable}
}

\end{solutionbox}
\begin{mnemonicbox}
``Sequential remembers history, combinational only
knows now''

\end{mnemonicbox}
\subsection*{Question 4(b) [4 marks]}\label{q4b}

\textbf{Design decade counter}

\begin{solutionbox}

\textbf{Decade Counter:} A sequential circuit that counts from 0 to 9
(decimal) and then resets to 0.

\textbf{Design using JK flip-flops:}

\begin{itemize}
\tightlist
\item
  Requires 4 JK flip-flops (Q3,Q2,Q1,Q0) to represent 4-bit binary
  number
\item
  Counts from 0000 to 1001 (0-9 decimal) then resets
\end{itemize}

\textbf{State Table:}

{\def\LTcaptype{none} % do not increment counter
\begin{longtable}[]{@{}ll@{}}
\toprule\noalign{}
Current State & Next State \\
\midrule\noalign{}
\endhead
\bottomrule\noalign{}
\endlastfoot
0 (0000) & 1 (0001) \\
1 (0001) & 2 (0010) \\
2 (0010) & 3 (0011) \\
3 (0011) & 4 (0100) \\
4 (0100) & 5 (0101) \\
5 (0101) & 6 (0110) \\
6 (0110) & 7 (0111) \\
7 (0111) & 8 (1000) \\
8 (1000) & 9 (1001) \\
9 (1001) & 0 (0000) \\
\end{longtable}
}

\textbf{Block Diagram:}

\begin{center}
\textbf{Mermaid Diagram (Code)}
\begin{verbatim}
{Shaded}
{Highlighting}[]
graph LR
    CLK[Clock] {-{-}{} FF0[JK0]}
    CLK {-{-}{} FF1[JK1]}
    CLK {-{-}{} FF2[JK2]}
    CLK {-{-}{} FF3[JK3]}

    AND([\&]) {-{-}{} R[Reset]}
    Q1 {-{-}{} AND}
    Q3 {-{-}{} AND}
    R {-{-}{} FF0}
    R {-{-}{} FF1}
    R {-{-}{} FF2}
    R {-{-}{} FF3}
    
    FF0 {-{-}Q0{-}{-}{} Q0[Q0]}
    FF1 {-{-}Q1{-}{-}{} Q1[Q1]}
    FF2 {-{-}Q2{-}{-}{} Q2[Q2]}
    FF3 {-{-}Q3{-}{-}{} Q3[Q3]}
    
    FF0 {-{-}Q0{-}{-}{} FF1}
    FF1 {-{-}Q1{-}{-}{} FF2}
    FF2 {-{-}Q2{-}{-}{} FF3}
{Highlighting}
{Shaded}
\end{verbatim}
\end{center}

\textbf{J-K Input Equations:}

\begin{itemize}
\tightlist
\item
  J0 = K0 = 1 (toggle on every clock)
\item
  J1 = K1 = Q0
\item
  J2 = K2 = Q1·Q0
\item
  J3 = K3 = Q2·Q1·Q0
\end{itemize}

\textbf{Reset condition:} When Q3·Q1 = 1 (state 1010), reset all
flip-flops

\end{solutionbox}
\begin{mnemonicbox}
``Count BCD, reset after 9''

\end{mnemonicbox}
\subsection*{Question 4(c)(I) [3 marks]}\label{q4c}

\textbf{Explain S-R flip-flop using NOR gate. Draw its logic symbol and
write its truth table.}

\begin{solutionbox}

\textbf{S-R Flip-flop using NOR gates:} A basic flip-flop constructed
from two cross-coupled NOR gates that can store one bit of information.

\textbf{Logic Circuit:}

\begin{center}
\textbf{Mermaid Diagram (Code)}
\begin{verbatim}
{Shaded}
{Highlighting}[]
graph LR
    S[S] {-{-}{} NOR1([1])}
    NOR2 {-{-}Q{}{-}{-}{} NOR1}

    R[R] {-{-}{} NOR2([1])}
    NOR1 {-{-}Q{-}{-}{} NOR2}
    
    NOR1 {-{-}Q{-}{-}{} Q[Q]}
    NOR2 {-{-}Q{}{-}{-}{} Qn[Q{}]}
{Highlighting}
{Shaded}
\end{verbatim}
\end{center}

\textbf{Logic Symbol:}

\begin{center}
\textbf{Mermaid Diagram (Code)}
\begin{verbatim}
{Shaded}
{Highlighting}[]
graph LR
    S[S] {-{-}{} SR[SR]}
    R[R] {-{-}{} SR}
    SR {-{-}Q{-}{-}{} Q[Q]}
    SR {-{-}Q{}{-}{-}{} Qn[Q{}]}
{Highlighting}
{Shaded}
\end{verbatim}
\end{center}

\textbf{Truth Table:}

{\def\LTcaptype{none} % do not increment counter
\begin{longtable}[]{@{}lllll@{}}
\toprule\noalign{}
S & R & Q (next) & Q' (next) & Operation \\
\midrule\noalign{}
\endhead
\bottomrule\noalign{}
\endlastfoot
0 & 0 & Q (prev) & Q' (prev) & Memory (no change) \\
0 & 1 & 0 & 1 & Reset \\
1 & 0 & 1 & 0 & Set \\
1 & 1 & 0 & 0 & Invalid (avoid) \\
\end{longtable}
}

\end{solutionbox}
\begin{mnemonicbox}
``S sets to 1, R resets to 0, both active gives
invalid state''

\end{mnemonicbox}
\subsection*{Question 4(c)(II) [4
marks]}\label{q4c}

\textbf{Explain S-R flip-flop using NAND gate. Write the limitation of
SR flip flop}

\begin{solutionbox}

\textbf{S-R Flip-flop using NAND gates:} A basic flip-flop constructed
from two cross-coupled NAND gates.

\textbf{Logic Circuit:}

\begin{center}
\textbf{Mermaid Diagram (Code)}
\begin{verbatim}
{Shaded}
{Highlighting}[]
graph LR
    S[S] {-{-}{} NAND1([\&])}
    NAND2 {-{-}Q{-}{-}{} NAND1}

    R[R] {-{-}{} NAND2([\&])}
    NAND1 {-{-}Q{}{-}{-}{} NAND2}
    
    NAND1 {-{-}Q{}{-}{-}{} Qn[Q{}]}
    NAND2 {-{-}Q{-}{-}{} Q[Q]}
{Highlighting}
{Shaded}
\end{verbatim}
\end{center}

\textbf{Truth Table:}

{\def\LTcaptype{none} % do not increment counter
\begin{longtable}[]{@{}lllll@{}}
\toprule\noalign{}
S & R & Q (next) & Q' (next) & Operation \\
\midrule\noalign{}
\endhead
\bottomrule\noalign{}
\endlastfoot
1 & 1 & Q (prev) & Q' (prev) & Memory (no change) \\
1 & 0 & 1 & 0 & Set \\
0 & 1 & 0 & 1 & Reset \\
0 & 0 & 1 & 1 & Invalid (avoid) \\
\end{longtable}
}

\textbf{Limitations of SR Flip-flop:}

\begin{enumerate}
\tightlist
\item
  \textbf{Invalid state:} When both S=1, R=1 (for NOR) or S=0, R=0 (for
  NAND), the output is unpredictable
\item
  \textbf{Race condition:} When inputs change simultaneously, the final
  state can be unpredictable
\item
  \textbf{No clocking mechanism:} Cannot synchronize with other digital
  components
\item
  \textbf{Not edge-triggered:} Cannot respond to brief pulses reliably
\item
  \textbf{Unwanted toggling:} May respond to noise or glitches
\end{enumerate}


{\def\LTcaptype{none} % do not increment counter
\vspace{-5pt}
\captionof{table}{NAND vs NOR SR Flip-flop}
\vspace{-10pt}
\begin{longtable}[]{@{}lll@{}}
\toprule\noalign{}
Characteristic & NAND SR Flip-flop & NOR SR Flip-flop \\
\midrule\noalign{}
\endhead
\bottomrule\noalign{}
\endlastfoot
Active inputs & Low (0) & High (1) \\
Inactive inputs & High (1) & Low (0) \\
Invalid state &

S=0,

R=0 &

S=1,

R=1 \\

\end{longtable}
}

\end{solutionbox}
\begin{mnemonicbox}
``NAND: active-low inputs, NOR: active-high inputs;
both have an invalid state''

\end{mnemonicbox}
\subsection*{Question 4 [OR Question] (a) [3
marks]}\label{question-4-or-question-a-3-marks}

\textbf{Write the definition of flip-flop. List the types of flip-flops}

\begin{solutionbox}

\textbf{Flip-flop:} A basic sequential digital circuit that can store
one bit of information and has two stable states (0 or 1). It serves as
a basic memory element in digital systems.

\textbf{Key characteristics:}

\begin{itemize}
\tightlist
\item
  Bistable multivibrator (two stable states)
\item
  Can maintain its state indefinitely until directed to change
\item
  Forms the basic building block for registers, counters, and memory
  circuits
\item
  Can be triggered by clock signals (synchronous) or level changes
  (asynchronous)
\end{itemize}

\textbf{Types of Flip-flops:}

{\def\LTcaptype{none} % do not increment counter
\begin{longtable}[]{@{}ll@{}}
\toprule\noalign{}
Flip-flop Type & Description \\
\midrule\noalign{}
\endhead
\bottomrule\noalign{}
\endlastfoot
SR (Set-Reset) & The most basic flip-flop with set and reset inputs \\
JK & Improved version of SR that eliminates invalid state \\
D (Data) & Stores the value at input D, used for data storage \\
T (Toggle) & Changes state when triggered, useful for counters \\
Master-Slave & Two-stage flip-flop that prevents race conditions \\
\end{longtable}
}

\end{solutionbox}
\begin{mnemonicbox}
``Storing a Single Step: SR, JK, D, T''

\end{mnemonicbox}
\subsection*{Question 4 [OR Question] (b) [4
marks]}\label{question-4-or-question-b-4-marks}

\textbf{Design 3-bit ring counter}

\begin{solutionbox}

\textbf{Ring Counter:} A circular shift register where only one bit is
set (1) and all others are reset (0). The single set bit ``rotates''
around the register when clocked.

\textbf{Design using D flip-flops:}

\begin{itemize}
\tightlist
\item
  Requires 3 D flip-flops for 3-bit counter
\item
  Initial state: 100, then cycles through 010, 001, and back to 100
\end{itemize}

\textbf{State Table:}

{\def\LTcaptype{none} % do not increment counter
\begin{longtable}[]{@{}ll@{}}
\toprule\noalign{}
Current State & Next State \\
\midrule\noalign{}
\endhead
\bottomrule\noalign{}
\endlastfoot
100 & 010 \\
010 & 001 \\
001 & 100 \\
\end{longtable}
}

\textbf{Block Diagram:}

\begin{center}
\textbf{Mermaid Diagram (Code)}
\begin{verbatim}
{Shaded}
{Highlighting}[]
graph LR
    CLK[Clock] {-{-}{} FF0[D0]}
    CLK {-{-}{} FF1[D1]}
    CLK {-{-}{} FF2[D2]}

    FF0 {-{-}Q0{-}{-}{} Q0[Q0]}
    FF1 {-{-}Q1{-}{-}{} Q1[Q1]}
    FF2 {-{-}Q2{-}{-}{} Q2[Q2]}
    
    Q2 {-{-}{} FF0}
    Q0 {-{-}{} FF1}
    Q1 {-{-}{} FF2}
    
    PRESET[Preset/Clear] {-.{-}{} FF0}
    PRESET {-.{-}{} FF1}
    PRESET {-.{-}{} FF2}
{Highlighting}
{Shaded}
\end{verbatim}
\end{center}

\textbf{D Input Equations:}

\begin{itemize}
\tightlist
\item
  D0 = Q2
\item
  D1 = Q0
\item
  D2 = Q1
\end{itemize}

\textbf{Initial state setting:} Preset FF0 to 1, Clear FF1 and FF2 to 0

\end{solutionbox}
\begin{mnemonicbox}
``One hot bit travels in a circle''

\end{mnemonicbox}
\subsection*{Question 4 [OR Question] (c)(I) [3
marks]}\label{question-4-or-question-ci-3-marks}

\textbf{Explain J-K flip-flop using its logic symbol and truth table}

\begin{solutionbox}

\textbf{J-K Flip-flop:} An improved version of SR flip-flop that
eliminates the invalid state and provides predictable behavior in all
input combinations.

\textbf{Logic Symbol:}

\begin{center}
\textbf{Mermaid Diagram (Code)}
\begin{verbatim}
{Shaded}
{Highlighting}[]
graph LR
    J[J] {-{-}{} JK[JK]}
    K[K] {-{-}{} JK}
    CLK[Clock] {-{-}{} JK}
    JK {-{-}Q{-}{-}{} Q[Q]}
    JK {-{-}Q{}{-}{-}{} Qn[Q{}]}
{Highlighting}
{Shaded}
\end{verbatim}
\end{center}

\textbf{Truth Table:}

{\def\LTcaptype{none} % do not increment counter
\begin{longtable}[]{@{}llll@{}}
\toprule\noalign{}
J & K & Q (next) & Operation \\
\midrule\noalign{}
\endhead
\bottomrule\noalign{}
\endlastfoot
0 & 0 & Q (prev) & No change \\
0 & 1 & 0 & Reset \\
1 & 0 & 1 & Set \\
1 & 1 & Q' (prev) & Toggle \\
\end{longtable}
}

\textbf{Key features:}

\begin{itemize}
\tightlist
\item
When

J=K=1, the flip-flop toggles (changes to opposite state)

\item
  No invalid state like in SR flip-flop
\item
  Can perform all operations: Set, Reset, Hold, Toggle
\end{itemize}

\end{solutionbox}
\begin{mnemonicbox}
``J sets, K resets, Both toggle, None remember''

\end{mnemonicbox}
\subsection*{Question 4 [OR Question] (c)(II) [4
marks]}\label{question-4-or-question-cii-4-marks}

\textbf{Draw logic circuit of D flip-flop and T flip-flop using J-K
flip-flop}

\begin{solutionbox}

\textbf{D Flip-flop using JK Flip-flop:}

To convert JK to D flip-flop:

\begin{itemize}
\tightlist
\item
  Connect D input to J
\item
  Connect D' (NOT D) to K
\end{itemize}

\textbf{Logic Circuit:}

\begin{center}
\textbf{Mermaid Diagram (Code)}
\begin{verbatim}
{Shaded}
{Highlighting}[]
graph LR
    D[D] {-{-}{} J[J]}
    D {-{-}{} NOT[NOT]}
    NOT {-{-}{} K[K]}
    J {-{-}{} JK[JK Flip{-}flop]}
    K {-{-}{} JK}
    CLK[Clock] {-{-}{} JK}
    JK {-{-}Q{-}{-}{} Q[Q]}
    JK {-{-}Q{}{-}{-}{} Qn[Q{}]}
{Highlighting}
{Shaded}
\end{verbatim}
\end{center}

\textbf{T Flip-flop using JK Flip-flop:}

To convert JK to T flip-flop:

\begin{itemize}
\tightlist
\item
  Connect T input to both J and K
\end{itemize}

\textbf{Logic Circuit:}

\begin{center}
\textbf{Mermaid Diagram (Code)}
\begin{verbatim}
{Shaded}
{Highlighting}[]
graph LR
    T[T] {-{-}{} J[J]}
    T {-{-}{} K[K]}
    J {-{-}{} JK[JK Flip{-}flop]}
    K {-{-}{} JK}
    CLK[Clock] {-{-}{} JK}
    JK {-{-}Q{-}{-}{} Q[Q]}
    JK {-{-}Q{}{-}{-}{} Qn[Q{}]}
{Highlighting}
{Shaded}
\end{verbatim}
\end{center}

\textbf{Truth Tables:}

\textbf{D Flip-flop:}

{\def\LTcaptype{none} % do not increment counter
\begin{longtable}[]{@{}lll@{}}
\toprule\noalign{}
D & Q (next) & Operation \\
\midrule\noalign{}
\endhead
\bottomrule\noalign{}
\endlastfoot
0 & 0 & Reset \\
1 & 1 & Set \\
\end{longtable}
}

\textbf{T Flip-flop:}

{\def\LTcaptype{none} % do not increment counter
\begin{longtable}[]{@{}lll@{}}
\toprule\noalign{}
T & Q (next) & Operation \\
\midrule\noalign{}
\endhead
\bottomrule\noalign{}
\endlastfoot
0 & Q (prev) & No change \\
1 & Q' (prev) & Toggle \\
\end{longtable}
}

\end{solutionbox}
\begin{mnemonicbox}
``D directly follows, T toggles when true''

\end{mnemonicbox}
\subsection*{Question 5(a) [3 marks]}\label{q5a}

\textbf{Compare RAM and ROM}

\begin{solutionbox}

\textbf{RAM (Random Access Memory) vs ROM (Read-Only Memory):}


{\def\LTcaptype{none} % do not increment counter
\vspace{-5pt}
\captionof{table}{RAM vs ROM Comparison}
\vspace{-10pt}
\begin{longtable}[]{@{}
  >{\raggedright\arraybackslash}p{(\linewidth - 4\tabcolsep) * \real{0.6154}}
  >{\raggedright\arraybackslash}p{(\linewidth - 4\tabcolsep) * \real{0.1923}}
  >{\raggedright\arraybackslash}p{(\linewidth - 4\tabcolsep) * \real{0.1923}}@{}}
\toprule\noalign{}
\begin{minipage}[b]{\linewidth}\raggedright
Characteristic
\end{minipage} & \begin{minipage}[b]{\linewidth}\raggedright
RAM
\end{minipage} & \begin{minipage}[b]{\linewidth}\raggedright
ROM
\end{minipage} \\
\midrule\noalign{}
\endhead
\bottomrule\noalign{}
\endlastfoot
\textbf{Full form} & Random Access Memory & Read-Only Memory \\
\textbf{Data retention} & Volatile (loses data when power off) &
Non-volatile (retains data without power) \\
\textbf{Read/Write capability} & Both read and write operations &
Primarily read-only (except in PROM, EPROM, EEPROM) \\
\textbf{Speed} & Faster & Slower \\
\textbf{Cost per bit} & Higher & Lower \\
\textbf{Applications} & Temporary data storage, active program execution
& Boot instructions, firmware, permanent data \\
\textbf{Types} & SRAM, DRAM & Mask ROM, PROM, EPROM, EEPROM, Flash \\
\textbf{Cell complexity} & More complex & Simpler \\
\end{longtable}
}

\end{solutionbox}
\begin{mnemonicbox}
``RAM Reads And Modifies (but forgets), ROM Remembers
On shutdown (but fixed)''

\end{mnemonicbox}
\subsection*{Question 5(b) [4 marks]}\label{q5b}

\textbf{Explain Serial In Serial Out shift register}

\begin{solutionbox}

\textbf{Serial In Serial Out (SISO) Shift Register:} A sequential
circuit that shifts data one bit at a time both at input and output.

\textbf{Operation:}

\begin{itemize}
\tightlist
\item
  Data enters serially one bit at a time
\item
  Each bit shifts through the register on each clock pulse
\item
  Data exits serially one bit at a time
\item
  First-in, first-out operation
\end{itemize}

\textbf{Block Diagram:}

\begin{center}
\textbf{Mermaid Diagram (Code)}
\begin{verbatim}
{Shaded}
{Highlighting}[]
graph LR
    SI[Serial In] {-{-}{} FF0[D0]}
    CLK[Clock] {-{-}{} FF0}
    CLK {-{-}{} FF1}
    CLK {-{-}{} FF2}
    CLK {-{-}{} FF3}

    FF0 {-{-}Q0{-}{-}{} FF1[D1]}
    FF1 {-{-}Q1{-}{-}{} FF2[D2]}
    FF2 {-{-}Q2{-}{-}{} FF3[D3]}
    FF3 {-{-}Q3{-}{-}{} SO[Serial Out]}
{Highlighting}
{Shaded}
\end{verbatim}
\end{center}

\textbf{Timing Diagram for shifting ``1011'':}

\begin{verbatim}
CLK   _|‾|_|‾|_|‾|_|‾|_|‾|_
SI    __|‾|_|‾|‾|_________
Q0    ______|‾|_|‾|‾|_____
Q1    ________|‾|_|‾|‾|___
Q2    ____________|‾|_|‾|‾|
SO    ______________|‾|_|‾|
\end{verbatim}

\textbf{Applications:}

\begin{itemize}
\tightlist
\item
  Data transmission between digital systems
\item
  Serial-to-serial data conversion
\item
  Time delay circuits
\item
  Signal filtering
\end{itemize}

\end{solutionbox}
\begin{mnemonicbox}
``Bits enter line, march through chain, exit in
sequence''

\end{mnemonicbox}
\subsection*{Question 5(c) [7 marks]}\label{q5c}

\textbf{Write short note on logic families}

\begin{solutionbox}

\textbf{Logic Families:} Groups of digital integrated circuits with
similar electrical characteristics, fabrication technology, and logic
implementations.

\textbf{Major Logic Families:}

\begin{enumerate}
\tightlist
\item
  \textbf{TTL (Transistor-Transistor Logic):}

  \begin{itemize}
  \tightlist
  \item
    Based on bipolar junction transistors
  \item
    Standard series: 7400
  \item
    Supply voltage: 5V
  \item
    Moderate speed and power consumption
  \item
    High noise immunity
  \item
    Variants: Standard TTL, Low-power TTL (74L), Schottky TTL (74S),
    Advanced Schottky (74AS)
  \end{itemize}
\item
  \textbf{CMOS (Complementary Metal-Oxide-Semiconductor):}

  \begin{itemize}
  \tightlist
  \item
    Based on MOSFETs (P-type and N-type)
  \item
    Standard series: 4000, 74C00
  \item
    Wide supply voltage range (3-15V)
  \item
    Very low power consumption
  \item
    High noise immunity
  \item
    Susceptible to static electricity
  \item
    Advanced variants: HC, HCT, AC, ACT, AHC, AHCT series
  \end{itemize}
\item
  \textbf{ECL (Emitter-Coupled Logic):}

  \begin{itemize}
  \tightlist
  \item
    Based on differential amplifier with emitter-coupled transistors
  \item
    Extremely high speed (fastest logic family)
  \item
    High power consumption
  \item
    Low noise immunity
  \item
    Negative supply voltage
  \item
    Used in high-speed applications
  \end{itemize}
\end{enumerate}

\textbf{Key Parameters of Logic Families:}

{\def\LTcaptype{none} % do not increment counter
\begin{longtable}[]{@{}
  >{\raggedright\arraybackslash}p{(\linewidth - 2\tabcolsep) * \real{0.4583}}
  >{\raggedright\arraybackslash}p{(\linewidth - 2\tabcolsep) * \real{0.5417}}@{}}
\toprule\noalign{}
\begin{minipage}[b]{\linewidth}\raggedright
Parameter
\end{minipage} & \begin{minipage}[b]{\linewidth}\raggedright
Description
\end{minipage} \\
\midrule\noalign{}
\endhead
\bottomrule\noalign{}
\endlastfoot
\textbf{Fan-in} & Maximum number of inputs a gate can accept \\
\textbf{Fan-out} & Maximum number of gates that can be driven by one
output \\
\textbf{Noise margin} & Ability to tolerate electrical noise \\
\textbf{Propagation delay} & Time delay between input and output
transitions \\
\textbf{Power dissipation} & Power consumed by the gate \\
\textbf{Figure of merit} & Product of speed and power (lower is
better) \\
\end{longtable}
}

\textbf{Comparison Table:}

{\def\LTcaptype{none} % do not increment counter
\begin{longtable}[]{@{}llll@{}}
\toprule\noalign{}
Parameter & TTL & CMOS & ECL \\
\midrule\noalign{}
\endhead
\bottomrule\noalign{}
\endlastfoot
Speed & Medium & Low to High & Very High \\
Power consumption & Medium & Very Low & High \\
Noise immunity & High & Very High & Low \\
Fan-out & 10 & 50+ & 25 \\
Supply voltage & 5V & 3-15V & -5.2V \\
Input/Output levels & 0.8V/2.0V & 30\%/70\% of VDD & -1.75V/-0.9V \\
\end{longtable}
}

\end{solutionbox}
\begin{mnemonicbox}
``TTL Takes Transistors, CMOS Conserves More
Operational Supply, ECL Executes Calculations Lightning-fast''

\end{mnemonicbox}
\subsection*{Question 5 [OR Question] (a) [3
marks]}\label{question-5-or-question-a-3-marks}

\textbf{Compare SRAM and DRAM}

\begin{solutionbox}

\textbf{SRAM (Static RAM) vs DRAM (Dynamic RAM):}


{\def\LTcaptype{none} % do not increment counter
\vspace{-5pt}
\captionof{table}{SRAM vs DRAM Comparison}
\vspace{-10pt}
\begin{longtable}[]{@{}
  >{\raggedright\arraybackslash}p{(\linewidth - 4\tabcolsep) * \real{0.5714}}
  >{\raggedright\arraybackslash}p{(\linewidth - 4\tabcolsep) * \real{0.2143}}
  >{\raggedright\arraybackslash}p{(\linewidth - 4\tabcolsep) * \real{0.2143}}@{}}
\toprule\noalign{}
\begin{minipage}[b]{\linewidth}\raggedright
Characteristic
\end{minipage} & \begin{minipage}[b]{\linewidth}\raggedright
SRAM
\end{minipage} & \begin{minipage}[b]{\linewidth}\raggedright
DRAM
\end{minipage} \\
\midrule\noalign{}
\endhead
\bottomrule\noalign{}
\endlastfoot
\textbf{Full form} & Static Random Access Memory & Dynamic Random Access
Memory \\
\textbf{Cell structure} & 6 transistors (flip-flop) & 1 transistor + 1
capacitor \\
\textbf{Storage element} & Flip-flop & Capacitor \\
\textbf{Refreshing} & Not required & Required periodically (ms) \\
\textbf{Speed} & Faster (access time: 10-30ns) & Slower (access time:
60-100ns) \\
\textbf{Density} & Lower (larger cell size) & Higher (smaller cell
size) \\
\textbf{Cost per bit} & Higher & Lower \\
\textbf{Power consumption} & Higher & Lower \\
\textbf{Applications} & Cache memory, buffer & Main memory (RAM) \\
\textbf{Data retention} & As long as power is supplied & Few
milliseconds, needs refresh \\
\end{longtable}
}

\end{solutionbox}
\begin{mnemonicbox}
``Static Stays steady with Six Transistors, Dynamic
Drains and needs regular refreshing''

\end{mnemonicbox}
\subsection*{Question 5 [OR Question] (b) [4
marks]}\label{question-5-or-question-b-4-marks}

\textbf{Explain 8:3 encoder}

\begin{solutionbox}

\textbf{8:3 Encoder:} A combinational circuit that converts 8 input
lines to 3 output lines, essentially converting an active input line to
its binary position.

\textbf{Function:}

\begin{itemize}
\tightlist
\item
  Has 8 input lines (I_{0} to I_{7}) and 3 output lines (Y_{2}, Y_{1}, Y_{0})
\item
  Only one input is active at a time
\item
  Output is the binary code representing position of active input
\end{itemize}

\textbf{Logic Circuit:}

\begin{center}
\textbf{Mermaid Diagram (Code)}
\begin{verbatim}
{Shaded}
{Highlighting}[]
graph TD
    I1[I1] {-{-}{} OR0([1])}
    I3[I3] {-{-}{} OR0}
    I5[I5] {-{-}{} OR0}
    I7[I7] {-{-}{} OR0}
    OR0 {-{-}{} Y0[Y0]}

    I2[I2] {-{-}{} OR1([1])}
    I3[I3] {-{-}{} OR1}
    I6[I6] {-{-}{} OR1}
    I7[I7] {-{-}{} OR1}
    OR1 {-{-}{} Y1[Y1]}
    
    I4[I4] {-{-}{} OR2([1])}
    I5[I5] {-{-}{} OR2}
    I6[I6] {-{-}{} OR2}
    I7[I7] {-{-}{} OR2}
    OR2 {-{-}{} Y2[Y2]}
{Highlighting}
{Shaded}
\end{verbatim}
\end{center}

\textbf{Truth Table:}

{\def\LTcaptype{none} % do not increment counter
\begin{longtable}[]{@{}ll@{}}
\toprule\noalign{}
Inputs & Outputs \\
\midrule\noalign{}
\endhead
\bottomrule\noalign{}
\endlastfoot
I_{7} I_{6} I_{5} I_{4} I_{3} I_{2} I_{1} I_{0} & Y_{2} Y_{1} Y_{0} \\
0 0 0 0 0 0 0 1 & 0 0 0 \\
0 0 0 0 0 0 1 0 & 0 0 1 \\
0 0 0 0 0 1 0 0 & 0 1 0 \\
0 0 0 0 1 0 0 0 & 0 1 1 \\
0 0 0 1 0 0 0 0 & 1 0 0 \\
0 0 1 0 0 0 0 0 & 1 0 1 \\
0 1 0 0 0 0 0 0 & 1 1 0 \\
1 0 0 0 0 0 0 0 & 1 1 1 \\
\end{longtable}
}

\textbf{Boolean Equations:}

\begin{itemize}
\tightlist
\item
  Y_{0} = I_{1} + I_{3} + I_{5} + I_{7}
\item
  Y_{1} = I_{2} + I_{3} + I_{6} + I_{7}
\item
  Y_{2} = I_{4} + I_{5} + I_{6} + I_{7}
\end{itemize}

\textbf{Applications:}

\begin{itemize}
\tightlist
\item
  Priority encoders
\item
  Keyboard encoders
\item
  Address decoders
\item
  Data selectors
\end{itemize}

\end{solutionbox}
\begin{mnemonicbox}
``Eight Inputs become their position in Three bits''

\end{mnemonicbox}
\subsection*{Question 5 [OR Question] (c) [7
marks]}\label{question-5-or-question-c-7-marks}

\textbf{Define (i) Fan-in (ii) Fan-out (iii) Noise margin (iv)
Propagation delay (v) Power dissipation for logic families}

\begin{solutionbox}

\textbf{Key Parameters of Logic Families:}

\textbf{1. Fan-in:}

\begin{itemize}
\tightlist
\item
  \textbf{Definition:} Maximum number of inputs a logic gate can accept
\item
  \textbf{Importance:} Determines complexity of logic implementation
\item
  \textbf{Typical values:} 2-8 for most families
\item
  \textbf{Example:} AND gate with 4 inputs has fan-in of 4
\end{itemize}

\textbf{2. Fan-out:}

\begin{itemize}
\tightlist
\item
  \textbf{Definition:} Maximum number of similar gates that one gate
  output can drive reliably
\item
  \textbf{Importance:} Determines loading capability and system
  expandability
\item
  \textbf{Calculation:} Based on output current capacity and input
  current requirements
\item
  \textbf{Typical values:} TTL: 10, CMOS: 50+, ECL: 25
\end{itemize}

\textbf{3. Noise Margin:}

\begin{itemize}
\tightlist
\item
  \textbf{Definition:} Measure of circuit's ability to tolerate unwanted
  electrical noise/signals
\item
  \textbf{Importance:} Ensures reliable operation in noisy environments
\item
  \textbf{Calculation:} Difference between minimum high output voltage
  and maximum high input voltage
\item
  \textbf{Typical values:} TTL: 0.4V, CMOS: 1.5V-2.25V, ECL: 0.2V
\end{itemize}

\textbf{4. Propagation Delay:}

\begin{itemize}
\tightlist
\item
  \textbf{Definition:} Time delay between input change and corresponding
  output change
\item
  \textbf{Importance:} Determines maximum operating frequency and speed
\item
  \textbf{Measurement:} Time from 50\% of input transition to 50\% of
  output transition
\item
  \textbf{Typical values:} TTL: 10ns, CMOS: 5-100ns, ECL: 1-2ns
\end{itemize}

\textbf{5. Power Dissipation:}

\begin{itemize}
\tightlist
\item
  \textbf{Definition:} Amount of power consumed by a logic gate
\item
  \textbf{Importance:} Affects heat generation, power supply
  requirements, battery life
\item
  \textbf{Calculation:} Product of supply voltage and current drawn
\item
  \textbf{Typical values:} TTL: 10mW, CMOS: 0.1mW (static), ECL: 25mW
\end{itemize}


{\def\LTcaptype{none} % do not increment counter
\vspace{-5pt}
\captionof{table}{Logic Family Comparison}
\vspace{-10pt}
\begin{longtable}[]{@{}llll@{}}
\toprule\noalign{}
Parameter & TTL & CMOS & ECL \\
\midrule\noalign{}
\endhead
\bottomrule\noalign{}
\endlastfoot
Fan-in & 3-8 & 2-unlimited & 2-4 \\
Fan-out & 10 & 50+ & 25 \\
Noise margin & 0.4V & 1.5V-2.25V & 0.2V \\
Propagation delay & 10ns & 5-100ns & 1-2ns \\
Power dissipation & 10mW & 0.1mW (static) & 25mW \\
Supply voltage & 5V & 3-15V & -5.2V \\
Figure of merit & 100pJ & 10pJ & 50pJ \\
\end{longtable}
}

\textbf{Diagram: Noise Margin and Switching Thresholds}

\begin{verbatim}
Voltage
   ^
   |                   VOH
   |    ┌───────┐      │
   |    │       │      │      Logic High
   |    │       │      │
   |    │       │      V      VIH
   |    │       │      │
   |    │  NMH  │      │      Undefined
   |    │       │      │
   |    │       │      V      VIL
   |    │       │      │
   |    │  NML  │      │      Logic Low
   |    │       │      V      VOL
   |    └───────┘
   └─────────────────────────> Signal
\end{verbatim}

\textbf{Relationships:}

\begin{itemize}
\tightlist
\item
  NMH (Noise Margin High) = VOH(min) - VIH(min)
\item
  NML (Noise Margin Low) = VIL(max) - VOL(max)
\item
  Figure of Merit = Power \times Delay product (lower is better)
\end{itemize}

\end{solutionbox}
\begin{mnemonicbox}
``Five Factors: Fan-in counts inputs, Fan-out drives
gates, Noise margin fights interference, Propagation delay measures
speed, Power dissipation generates heat''

\end{mnemonicbox}

\end{document}
