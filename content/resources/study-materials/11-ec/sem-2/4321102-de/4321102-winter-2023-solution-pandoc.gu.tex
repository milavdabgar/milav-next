\documentclass[10pt,a4paper]{article}

% content/resources/templates/preamble.tex
\usepackage[margin=0.6in]{geometry}
\author{Milav Dabgar}
\usepackage{amsmath,amssymb,amsthm}
\usepackage{booktabs}
\usepackage{multirow}
\usepackage{xcolor}
\usepackage{tcolorbox}
\tcbuselibrary{breakable,skins}
\usepackage[colorlinks=true,linkcolor=blue]{hyperref}
\usepackage{titlesec}
\usepackage{enumitem}
\usepackage{tikz}
\usepackage{pgfplots}
\usepackage{circuitikz}
\usepackage[version=4]{mhchem}
\usepackage{longtable}
\usepackage{array}
\usepackage{float}
\usepackage{caption}
\usepackage{listings}

\lstset{
  basicstyle=\small\ttfamily,
  breaklines=true,
  breakatwhitespace=false,
  postbreak=\mbox{\textcolor{red}{$\hookrightarrow$}\space},
  float=false,
  numbers=left,
  numberstyle=\tiny\color{gray},
  numbersep=10pt,
  xleftmargin=2em,
  keywordstyle=\color{blue},
  commentstyle=\color{green!60!black},
  stringstyle=\color{purple},
  backgroundcolor=\color{gray!5},
  showstringspaces=false,
  tabsize=2,
  captionpos=b,
  keepspaces=true,
  columns=flexible
}

\pgfplotsset{compat=1.18}
\usetikzlibrary{shapes,arrows,positioning,calc,patterns,decorations.pathmorphing,decorations.markings,arrows.meta}

% Color scheme
\definecolor{headcolor}{RGB}{0,102,204}
\definecolor{keycolor}{RGB}{220,20,60}
\definecolor{solutioncolor}{RGB}{34,139,34}
\definecolor{mnemoniccolor}{RGB}{148,0,211}
\definecolor{codecolor}{RGB}{0,0,100}

% Spacing
\setlength{\parskip}{3pt}
\setlist[itemize]{nosep}
\setlist[enumerate]{nosep}

% Title formatting
\titleformat{\section}{\Large\bfseries\color{headcolor}}{\thesection}{1em}{}
\titleformat{\subsection}{\large\bfseries\color{headcolor}}{\thesubsection}{1em}{}

% Pandoc tightlist compatibility
\providecommand{\tightlist}{%
  \setlength{\itemsep}{0pt}\setlength{\parskip}{0pt}}

% Pandoc longtable compatibility
\newcounter{none}
\def\thenone{}


% content/resources/templates/gujarati-boxes.tex
\usepackage{fontspec}
\usepackage{polyglossia}

% Set Gujarati as main language (document is primarily in Gujarati)
% Note: gloss-gujarati.ldf doesn't exist in polyglossia, but it will use hyphenation patterns
\setdefaultlanguage{gujarati}
\setotherlanguage{english}

% Configure Gujarati font properly
% Use Language=Default to prevent polyglossia from trying to add language-specific features
% that don't exist for Gujarati, which causes "empty feature" warnings
\newfontfamily\gujaratifont[Script=Gujarati,AutoFakeBold=2.5,AutoFakeSlant=0.3]{Noto Sans Gujarati}
\setmainfont[Script=Gujarati,AutoFakeBold=2.5,AutoFakeSlant=0.3]{Noto Sans Gujarati}
% Use Noto Sans Gujarati for monospace to support Gujarati in text
\setmonofont[Scale=0.9]{Noto Sans Gujarati}

% Configure English to use the same font
\newfontfamily\englishfont[Script=Gujarati,AutoFakeBold=2.5,AutoFakeSlant=0.3]{Noto Sans Gujarati}

% Translations for polyglossia
\gappto\captionsgujarati{
  \renewcommand{\tablename}{કોષ્ટક}
  \renewcommand{\figurename}{આકૃતિ}
}

% Helper for TikZ nodes to ensure Gujarati font
\newcommand{\gu}[1]{{\gujaratifont #1}}

% Custom environments
\newtcolorbox{solutionbox}{
    breakable,
    enhanced,
    colback=solutioncolor!5!white,
    colframe=solutioncolor!75!black,
    fonttitle=\bfseries,
    title=જવાબ
}

\newtcolorbox{solutionboxnobreak}{
 colback=solutioncolor!5!white,
 colframe=solutioncolor!75!black,
 fonttitle=\bfseries,
 title=જવાબ
}

\newtcolorbox{keyformula}{
 breakable,
 enhanced,
 colback=keycolor!5!white,
 colframe=keycolor!75!black,
 fonttitle=\bfseries,
 title=રાસાયણિક સમીકરણ/સૂત્ર
}

\newtcolorbox{mnemonicbox}{
 breakable,
 enhanced,
 colback=mnemoniccolor!5!white,
 colframe=mnemoniccolor!75!black,
 fonttitle=\bfseries,
 title=મેમરી ટ્રીક
}


\begin{document}

\begin{center}
{\Huge\bfseries\color{headcolor} Subject Name (Gujarati)}\\[5pt]
{\LARGE 4321102 -- Winter 2023}\\[3pt]
{\large Semester 1 Study Material}\\[3pt]
{\normalsize\textit{Detailed Solutions and Explanations}}
\end{center}

\vspace{10pt}

\subsection*{પ્રશ્ન 1(અ) [3
ગુણ]}\label{uxaaauxab0uxab6uxaa8-1uxa85-3-uxa97uxaa3}

\textbf{(726)_{1}_{0} = (\_\_\_\_\_\_\_\_\_)_{2}}

\begin{solutionbox}


{\def\LTcaptype{none} % do not increment counter
\vspace{-5pt}
\captionof{table}{દશાંશમાંથી બાઈનરીમાં રૂપાંતર}
\vspace{-10pt}
\begin{longtable}[]{@{}lll@{}}
\toprule\noalign{}
સ્ટેપ & ગણતરી & શેષ \\
\midrule\noalign{}
\endhead
\bottomrule\noalign{}
\endlastfoot
1 & 726 \div 2 = 363 & 0 \\
2 & 363 \div 2 = 181 & 1 \\
3 & 181 \div 2 = 90 & 1 \\
4 & 90 \div 2 = 45 & 0 \\
5 & 45 \div 2 = 22 & 1 \\
6 & 22 \div 2 = 11 & 0 \\
7 & 11 \div 2 = 5 & 1 \\
8 & 5 \div 2 = 2 & 1 \\
9 & 2 \div 2 = 1 & 0 \\
10 & 1 \div 2 = 0 & 1 \\
\end{longtable}
}

નીચેથી ઉપર વાંચતા: (726)_{1}_{0} = (1011010110)_{2}

\end{solutionbox}
\begin{mnemonicbox}
``બે વડે ભાગો, શેષ ઉપરથી વાંચો''

\end{mnemonicbox}
\subsection*{પ્રશ્ન 1(બ) [4
ગુણ]}\label{uxaaauxab0uxab6uxaa8-1uxaac-4-uxa97uxaa3}

\textbf{1) નીચેના બાઈનરી નંબર (10110101)_{2} ને ગ્રે નંબરમાં કન્વર્ટ કરો.}

\textbf{2) નીચેના ગ્રે નંબર (10110110)gray ને બાઈનરી નંબરમાં કન્વર્ટ કરો.}

\begin{solutionbox}

\textbf{બાઈનરીથી ગ્રે કન્વર્ઝન:}

\begin{verbatim}
Binary:   1 0 1 1 0 1 0 1
           ↓ ↓ ↓ ↓ ↓ ↓ ↓
XOR:      1\oplus0 0\oplus1 1\oplus1 1\oplus0 0\oplus1 1\oplus0 0\oplus1
           ↓   ↓   ↓   ↓   ↓   ↓   ↓
Gray:     1   1   0   1   1   1   1
\end{verbatim}

તેથી: (10110101)_{2} = (1101111)gray

\textbf{ગ્રેથી બાઈનરી કન્વર્ઝન:}

\begin{verbatim}
Gray:     1 0 1 1 0 1 1 0
           ↓
Binary:   1
          1\oplus0 = 1
          1\oplus1 = 0
          0\oplus1 = 1
          1\oplus0 = 1
          1\oplus1 = 0
          0\oplus1 = 1
          1\oplus0 = 1
\end{verbatim}

તેથી: (10110110)gray = (10110101)_{2}

\end{solutionbox}
\begin{mnemonicbox}
``પ્રથમ બિટ સરખો, બાકી XOR અગાઉના બાઈનરી સાથે''

\end{mnemonicbox}
\subsection*{પ્રશ્ન 1(ક) [7
ગુણ]}\label{uxaaauxab0uxab6uxaa8-1uxa95-7-uxa97uxaa3}

\textbf{NAND ને યુનિવર્સલ ગેટ તરીકે સમજાવો.}

\begin{solutionbox}

\textbf{આકૃતિ: NAND યુનિવર્સલ ગેટ તરીકે}

\begin{center}
\textbf{Mermaid Diagram (Code)}
\begin{verbatim}
{Shaded}
{Highlighting}[]
graph TD
    subgraph "NAND વડે NOT ગેટ"
    A1(A){-{-}{}N1((NAND))}
    A1(A){-{-}{}N1}
    N1{-{-}{}Z1("A{}")}
    end

    subgraph "NAND વડે AND ગેટ"
    A2(A){-{-}{}N2((NAND))}
    B2(B){-{-}{}N2}
    N2{-{-}{}N3((NAND))}
    N2{-{-}{}N3}
    N3{-{-}{}Z2("A·B")}
    end
    
    subgraph "NAND વડે OR ગેટ"
    A3(A){-{-}{}N4((NAND))}
    A3(A){-{-}{}N4}
    B3(B){-{-}{}N5((NAND))}
    B3(B){-{-}{}N5}
    N4{-{-}{}N6((NAND))}
    N5{-{-}{}N6}
    N6{-{-}{}Z3("A+B")}
    end
{Highlighting}
{Shaded}
\end{verbatim}
\end{center}

\begin{itemize}
\tightlist
\item
  \textbf{યુનિવર્સલ ગુણધર્મ}: NAND ગેટ કોઈપણ બુલિયન ફંક્શન બીજા કોઈપણ ગેટની જરૂર
  વિના બનાવી શકે છે
\item
  \textbf{NOT ઇમ્પ્લિમેન્ટેશન}: NAND ગેટના બંને ઇનપુટ જોડવાથી NOT ગેટ બને છે
\item
  \textbf{AND ઇમ્પ્લિમેન્ટેશન}: NAND પછી બીજો NAND ગેટ જોડવાથી AND ગેટ બને છે
\item
  \textbf{OR ઇમ્પ્લિમેન્ટેશન}: બે NAND ગેટના સિંગલ ઇનપુટ્સ, પછી NAND જોડવાથી OR ગેટ
  બને છે
\end{itemize}


{\def\LTcaptype{none} % do not increment counter
\vspace{-5pt}
\captionof{table}{NAND ગેટ ઇમ્પ્લિમેન્ટેશન}
\vspace{-10pt}
\begin{longtable}[]{@{}ll@{}}
\toprule\noalign{}
લોજિક ફંક્શન & NAND ઇમ્પ્લિમેન્ટેશન \\
\midrule\noalign{}
\endhead
\bottomrule\noalign{}
\endlastfoot
NOT(A) & NAND(A,A) \\
AND(A,B) & NAND(NAND(A,B),NAND(A,B)) \\
OR(A,B) & NAND(NAND(A,A),NAND(B,B)) \\
\end{longtable}
}

\end{solutionbox}
\begin{mnemonicbox}
``NAND બધા ગેટ બનાવી શકે છે''

\end{mnemonicbox}
\subsection*{પ્રશ્ન 1(ક) OR [7
ગુણ]}\label{uxaaauxab0uxab6uxaa8-1uxa95-or-7-uxa97uxaa3}

\textbf{NOR ને યુનિવર્સલ ગેટ તરીકે સમજાવો.}

\begin{solutionbox}

\textbf{આકૃતિ: NOR યુનિવર્સલ ગેટ તરીકે}

\begin{center}
\textbf{Mermaid Diagram (Code)}
\begin{verbatim}
{Shaded}
{Highlighting}[]
graph TD
    subgraph "NOR વડે NOT ગેટ"
    A1(A){-{-}{}N1((NOR))}
    A1(A){-{-}{}N1}
    N1{-{-}{}Z1("A{}")}
    end

    subgraph "NOR વડે OR ગેટ"
    A2(A){-{-}{}N2((NOR))}
    B2(B){-{-}{}N2}
    N2{-{-}{}N3((NOR))}
    N2{-{-}{}N3}
    N3{-{-}{}Z2("A+B")}
    end
    
    subgraph "NOR વડે AND ગેટ"
    A3(A){-{-}{}N4((NOR))}
    A3(A){-{-}{}N4}
    B3(B){-{-}{}N5((NOR))}
    B3(B){-{-}{}N5}
    N4{-{-}{}N6((NOR))}
    N5{-{-}{}N6}
    N6{-{-}{}Z3("A·B")}
    end
{Highlighting}
{Shaded}
\end{verbatim}
\end{center}

\begin{itemize}
\tightlist
\item
  \textbf{યુનિવર્સલ ગુણધર્મ}: NOR ગેટ કોઈપણ બુલિયન ફંક્શન બીજા કોઈપણ ગેટની જરૂર
  વિના બનાવી શકે છે
\item
  \textbf{NOT ઇમ્પ્લિમેન્ટેશન}: NOR ગેટના બંને ઇનપુટ જોડવાથી NOT ગેટ બને છે
\item
  \textbf{OR ઇમ્પ્લિમેન્ટેશન}: NOR પછી બીજો NOR ગેટ જોડવાથી OR ગેટ બને છે
\item
  \textbf{AND ઇમ્પ્લિમેન્ટેશન}: બે NOR ગેટના સિંગલ ઇનપુટ્સ, પછી NOR જોડવાથી AND ગેટ
  બને છે
\end{itemize}


{\def\LTcaptype{none} % do not increment counter
\vspace{-5pt}
\captionof{table}{NOR ગેટ ઇમ્પ્લિમેન્ટેશન}
\vspace{-10pt}
\begin{longtable}[]{@{}ll@{}}
\toprule\noalign{}
લોજિક ફંક્શન & NOR ઇમ્પ્લિમેન્ટેશન \\
\midrule\noalign{}
\endhead
\bottomrule\noalign{}
\endlastfoot
NOT(A) & NOR(A,A) \\
OR(A,B) & NOR(NOR(A,B),NOR(A,B)) \\
AND(A,B) & NOR(NOR(A,A),NOR(B,B)) \\
\end{longtable}
}

\end{solutionbox}
\begin{mnemonicbox}
``NOR બધા લોજિક સર્કિટ બનાવી શકે છે''

\end{mnemonicbox}
\subsection*{પ્રશ્ન 2(અ) [3
ગુણ]}\label{uxaaauxab0uxab6uxaa8-2uxa85-3-uxa97uxaa3}

\textbf{(11011011)_{2} X (110)_{2} = (\_\_\_\_\_\_\_\_\_)_{2}}

\begin{solutionbox}


\textbf{Table: બાઈનરી ગુણાકાર}
\begin{verbatim}
    1 1 0 1 1 0 1 1
  \times         1 1 0
  ---------------
    1 1 0 1 1 0 1 1  (\times 0)
  1 1 0 1 1 0 1 1    (\times 1)
1 1 0 1 1 0 1 1      (\times 1)
-----------------
1 0 0 0 0 0 0 0 1 1 0
\end{verbatim}

તેથી: (11011011)_{2} \times (110)_{2} = (10000001110)_{2}

\end{solutionbox}
\begin{mnemonicbox}
``દરેક બિટ સાથે ગુણાકાર કરો, પંક્તિઓ ઉમેરો''

\end{mnemonicbox}
\subsection*{પ્રશ્ન 2(બ) [4
ગુણ]}\label{uxaaauxab0uxab6uxaa8-2uxaac-4-uxa97uxaa3}

\textbf{ડીમોર્ગનનો પ્રમેય સાબિત કરો.}

\begin{solutionbox}


{\def\LTcaptype{none} % do not increment counter
\vspace{-5pt}
\captionof{table}{ડીમોર્ગનના પ્રમેયની સાબિતી}
\vspace{-10pt}
\begin{longtable}[]{@{}lllllll@{}}
\toprule\noalign{}
A & B & A' & B' & A+B & (A+B)' & A'·B' \\
\midrule\noalign{}
\endhead
\bottomrule\noalign{}
\endlastfoot
0 & 0 & 1 & 1 & 0 & 1 & 1 \\
0 & 1 & 1 & 0 & 1 & 0 & 0 \\
1 & 0 & 0 & 1 & 1 & 0 & 0 \\
1 & 1 & 0 & 0 & 1 & 0 & 0 \\
\end{longtable}
}

ડીમોર્ગનના પ્રમેય: 1. (A+B)' = A'·B' 2. (A·B)' = A'+B'

ટ્રુથ ટેબલ સાબિત કરે છે કે (A+B)' = A'·B' કારણ કે બંને કોલમ મેચ થાય છે.

\end{solutionbox}
\begin{mnemonicbox}
``રેખાને તોડો, ચિહ્ન બદલો''

\end{mnemonicbox}
\subsection*{પ્રશ્ન 2(ક) [7
ગુણ]}\label{uxaaauxab0uxab6uxaa8-2uxa95-7-uxa97uxaa3}

\textbf{લોજિક સર્કિટ, બુલિયન સમીકરણ અને ટ્રુથ ટેબલનો ઉપયોગ કરીને ફુલ એડર
સમજાવો.}

\begin{solutionbox}

\textbf{આકૃતિ: ફુલ એડર સર્કિટ}

\begin{center}
\textbf{Mermaid Diagram (Code)}
\begin{verbatim}
{Shaded}
{Highlighting}[]
graph LR
    A(A){-{-}{}XOR1(XOR)}
    B(B){-{-}{}XOR1}
    XOR1{-{-}{}XOR2(XOR)}
    Cin(Cin){-{-}{}XOR2}
    XOR2{-{-}{}Sum(Sum)}

    A{-{-}{}AND1(AND)}
    B{-{-}{}AND1}
    AND1{-{-}{}OR1(OR)}
    
    A{-{-}{}AND2(AND)}
    Cin{-{-}{}AND2}
    AND2{-{-}{}OR1}
    
    B{-{-}{}AND3(AND)}
    Cin{-{-}{}AND3}
    AND3{-{-}{}OR1}
    
    OR1{-{-}{}Cout(Cout)}
{Highlighting}
{Shaded}
\end{verbatim}
\end{center}


{\def\LTcaptype{none} % do not increment counter
\vspace{-5pt}
\captionof{table}{ફુલ એડર ટ્રુથ ટેબલ}
\vspace{-10pt}
\begin{longtable}[]{@{}lllll@{}}
\toprule\noalign{}
A & B & Cin & Sum & Cout \\
\midrule\noalign{}
\endhead
\bottomrule\noalign{}
\endlastfoot
0 & 0 & 0 & 0 & 0 \\
0 & 0 & 1 & 1 & 0 \\
0 & 1 & 0 & 1 & 0 \\
0 & 1 & 1 & 0 & 1 \\
1 & 0 & 0 & 1 & 0 \\
1 & 0 & 1 & 0 & 1 \\
1 & 1 & 0 & 0 & 1 \\
1 & 1 & 1 & 1 & 1 \\
\end{longtable}
}

\begin{itemize}
\tightlist
\item
  \textbf{બુલિયન સમીકરણો}:

  \begin{itemize}
  \tightlist
  \item
    Sum = A \oplus B \oplus Cin
  \item
    Cout = (A·B) + (B·Cin) + (A·Cin)
  \end{itemize}
\end{itemize}

\end{solutionbox}
\begin{mnemonicbox}
``સરવાળા માટે ત્રણ XOR, કેરી માટે AND પછી OR''

\end{mnemonicbox}
\subsection*{પ્રશ્ન 2(અ) OR [3
ગુણ]}\label{uxaaauxab0uxab6uxaa8-2uxa85-or-3-uxa97uxaa3}

\textbf{(11010010)_{2} સાથે (101)_{2} નો ભાગાકાર = (\_\_\_\_\_\_\_\_\_)_{2}}

\begin{solutionbox}


\textbf{Table: બાઈનરી ભાગાકાર}
\begin{verbatim}
            1 0 1 0 1 1
         ____________
101 ) 1 1 0 1 0 0 1 0
      1 0 1
      -----
        1 1 0
        1 0 1
        -----
          0 1 0
            0 0
          -----
            1 0 0
            1 0 1
            -----
              1 1 0
              1 0 1
              -----
                0 1 0
                  0 0
                -----
                  1 0
                   0
                 ----
                   0
\end{verbatim}

તેથી: (11010010)_{2} \div (101)_{2} = (101011)_{2} બાકી (0)_{2}

\end{solutionbox}
\begin{mnemonicbox}
``દશાંશની જેમ ભાગો, પણ બાઈનરી બાદબાકી વાપરો''

\end{mnemonicbox}
\subsection*{પ્રશ્ન 2(બ) OR [4
ગુણ]}\label{uxaaauxab0uxab6uxaa8-2uxaac-or-4-uxa97uxaa3}

\textbf{બુલિયન અભિવ્યક્તિ Y = A'B+AB'+A'B'+AB ને સરળ બનાવો}

\begin{solutionbox}


{\def\LTcaptype{none} % do not increment counter
\vspace{-5pt}
\captionof{table}{બુલિયન સરલીકરણ}
\vspace{-10pt}
\begin{longtable}[]{@{}lll@{}}
\toprule\noalign{}
સ્ટેપ & અભિવ્યક્તિ & વપરાયેલ નિયમ \\
\midrule\noalign{}
\endhead
\bottomrule\noalign{}
\endlastfoot
1 & Y = A'B+AB'+A'B'+AB & મૂળ \\
2 & Y = A'(B+B')+A(B'+B) & ફેક્ટરિંગ \\
3 &

Y = A'(1)+A(1) & B+B' = 1 \\

4 & Y = A'+A & સરલીકરણ \\
5 &

Y = 1 & A'+A = 1 \\

\end{longtable}
}

તેથી: Y = 1 (હંમેશા TRUE)

\end{solutionbox}
\begin{mnemonicbox}
``પહેલા ફેક્ટર કરો, ઓળખો લાગુ કરો, સમાન પદો જોડો''

\end{mnemonicbox}
\subsection*{પ્રશ્ન 2(ક) OR [7
ગુણ]}\label{uxaaauxab0uxab6uxaa8-2uxa95-or-7-uxa97uxaa3}

\textbf{લોજિક સર્કિટ, બુલિયન સમીકરણ અને ટ્રુથ ટેબલનો ઉપયોગ કરીને ફુલ સબટ્રેક્ટર
સમજાવો.}

\begin{solutionbox}

\textbf{આકૃતિ: ફુલ સબટ્રેક્ટર સર્કિટ}

\begin{center}
\textbf{Mermaid Diagram (Code)}
\begin{verbatim}
{Shaded}
{Highlighting}[]
graph LR
    A(A){-{-}{}XOR1(XOR)}
    B(B){-{-}{}XOR1}
    XOR1{-{-}{}XOR2(XOR)}
    Bin(Bin){-{-}{}XOR2}
    XOR2{-{-}{}D(Difference)}

    A(A){-{-}{}NOT1(NOT)}
    NOT1{-{-}{}AND1(AND)}
    B{-{-}{}AND1}
    AND1{-{-}{}OR1(OR)}
    
    XOR1{-{-}{}NOT2(NOT)}
    NOT2{-{-}{}AND2(AND)}
    Bin{-{-}{}AND2}
    AND2{-{-}{}OR1}
    
    B{-{-}{}AND3(AND)}
    Bin{-{-}{}AND3}
    AND3{-{-}{}OR1}
    
    OR1{-{-}{}Bout(Borrow Out)}
{Highlighting}
{Shaded}
\end{verbatim}
\end{center}


{\def\LTcaptype{none} % do not increment counter
\vspace{-5pt}
\captionof{table}{ફુલ સબટ્રેક્ટર ટ્રુથ ટેબલ}
\vspace{-10pt}
\begin{longtable}[]{@{}lllll@{}}
\toprule\noalign{}
A & B & Bin & Difference & Bout \\
\midrule\noalign{}
\endhead
\bottomrule\noalign{}
\endlastfoot
0 & 0 & 0 & 0 & 0 \\
0 & 0 & 1 & 1 & 1 \\
0 & 1 & 0 & 1 & 1 \\
0 & 1 & 1 & 0 & 1 \\
1 & 0 & 0 & 1 & 0 \\
1 & 0 & 1 & 0 & 0 \\
1 & 1 & 0 & 0 & 0 \\
1 & 1 & 1 & 1 & 1 \\
\end{longtable}
}

\begin{itemize}
\tightlist
\item
  \textbf{બુલિયન સમીકરણો}:

  \begin{itemize}
  \tightlist
  \item
    Difference = A \oplus B \oplus Bin
  \item
    Bout = (A'·B) + (A'·Bin) + (B·Bin)
  \end{itemize}
\end{itemize}

\end{solutionbox}
\begin{mnemonicbox}
``તફાવત માટે ત્રિગણો XOR, ઇનપુટ મોટો હોય ત્યારે બોરો''

\end{mnemonicbox}
\subsection*{પ્રશ્ન 3(અ) [3
ગુણ]}\label{uxaaauxab0uxab6uxaa8-3uxa85-3-uxa97uxaa3}

\textbf{૨'s કોંપ્લીમેંટનો ઉપયોગ કરીને (1011001)_{2} ને (1101101)_{2} માંથી બાદ કરો.}

\begin{solutionbox}


{\def\LTcaptype{none} % do not increment counter
\vspace{-5pt}
\captionof{table}{2's કોંપ્લીમેંટ બાદબાકી}
\vspace{-10pt}
\begin{longtable}[]{@{}lll@{}}
\toprule\noalign{}
સ્ટેપ & ઓપરેશન & પરિણામ \\
\midrule\noalign{}
\endhead
\bottomrule\noalign{}
\endlastfoot
1 & બાદ કરવાની સંખ્યા: & 1011001 \\
2 & 1's કોંપ્લીમેંટ: & 0100110 \\
3 & 2's કોંપ્લીમેંટ: & 0100111 \\
4 & (1101101) + (0100111) = & 10010100 \\
5 & કેરી છોડી દો: & 0010100 \\
\end{longtable}
}

તેથી: (1101101)_{2} - (1011001)_{2} = (0010100)_{2} = (20)_{1}_{0}

\end{solutionbox}
\begin{mnemonicbox}
``બિટ્સ ફ્લિપ કરો, એક ઉમેરો, પછી સંખ્યાઓ ઉમેરો''

\end{mnemonicbox}
\subsection*{પ્રશ્ન 3(બ) [4
ગુણ]}\label{uxaaauxab0uxab6uxaa8-3uxaac-4-uxa97uxaa3}

\textbf{કનોફ મેપ (K' મેપ) પદ્ધતિનો ઉપયોગ કરીને બુલિયન સમીકરણને સરળ બનાવો:
F(A,B,C,D) = Σm(0,1,2,6,7,8,12,15)}

\begin{solutionbox}


\textbf{Table: કનોફ મેપ}
\begin{verbatim}
      CD      
AB    00  01  11  10
00    1   1   0   1
01    0   0   1   1
11    0   0   1   0
10    1   0   0   0
\end{verbatim}

\textbf{આકૃતિ: K-map ગ્રુપિંગ}

\begin{verbatim}
+{-{-}{-}{-}{-}+{-}{-}{-}{-}{-}+{-}{-}{-}{-}{-}+{-}{-}{-}{-}{-}+}
|  1  |  1  |  0  |  1  |
|  A  |  A  |     |  A  |
+{-{-}{-}{-}{-}+{-}{-}{-}{-}{-}+{-}{-}{-}{-}{-}+{-}{-}{-}{-}{-}+}
|  0  |  0  |  1  |  1  |
|     |     |  B  |  B  |
+{-{-}{-}{-}{-}+{-}{-}{-}{-}{-}+{-}{-}{-}{-}{-}+{-}{-}{-}{-}{-}+}
|  0  |  0  |  1  |  0  |
|     |     |  B  |     |
+{-{-}{-}{-}{-}+{-}{-}{-}{-}{-}+{-}{-}{-}{-}{-}+{-}{-}{-}{-}{-}+}
|  1  |  0  |  0  |  0  |
|  C  |     |     |     |
+{-{-}{-}{-}{-}+{-}{-}{-}{-}{-}+{-}{-}{-}{-}{-}+{-}{-}{-}{-}{-}+}
\end{verbatim}

ગ્રુપ A: A'B'C' (4 સેલ) ગ્રુપ B: BCD (3 સેલ) ગ્રુપ C: A'B'CD' (1 સેલ)

સરળ અભિવ્યક્તિ: F(A,B,C,D) = A'B'C' + BCD + A'B'CD'

\end{solutionbox}
\begin{mnemonicbox}
``2^{n} ના મોટામાં મોટા સમૂહો શોધો, લઘુત્તમ પદો વાપરો''

\end{mnemonicbox}
\subsection*{પ્રશ્ન 3(ક) [7
ગુણ]}\label{uxaaauxab0uxab6uxaa8-3uxa95-7-uxa97uxaa3}

\textbf{લોજિક સર્કિટ અને ટ્રુથ ટેબલનો ઉપયોગ કરીને 3 થી 8 ડીકોડર સમજાવો.}

\begin{solutionbox}

\textbf{આકૃતિ: 3-થી-8 ડીકોડર}

\begin{center}
\textbf{Mermaid Diagram (Code)}
\begin{verbatim}
{Shaded}
{Highlighting}[]
graph TD
    A(A){-{-}{}NOT1(NOT)}
    B(B){-{-}{}NOT2(NOT)}
    C(C){-{-}{}NOT3(NOT)}

    NOT1{-{-}{}AND0(AND)}
    NOT2{-{-}{}AND0}
    NOT3{-{-}{}AND0}
    AND0{-{-}{}D0(D0)}
    
    NOT1{-{-}{}AND1(AND)}
    NOT2{-{-}{}AND1}
    C{-{-}{}AND1}
    AND1{-{-}{}D1(D1)}
    
    NOT1{-{-}{}AND2(AND)}
    B{-{-}{}AND2}
    NOT3{-{-}{}AND2}
    AND2{-{-}{}D2(D2)}
    
    NOT1{-{-}{}AND3(AND)}
    B{-{-}{}AND3}
    C{-{-}{}AND3}
    AND3{-{-}{}D3(D3)}
    
    A{-{-}{}AND4(AND)}
    NOT2{-{-}{}AND4}
    NOT3{-{-}{}AND4}
    AND4{-{-}{}D4(D4)}
    
    A{-{-}{}AND5(AND)}
    NOT2{-{-}{}AND5}
    C{-{-}{}AND5}
    AND5{-{-}{}D5(D5)}
    
    A{-{-}{}AND6(AND)}
    B{-{-}{}AND6}
    NOT3{-{-}{}AND6}
    AND6{-{-}{}D6(D6)}
    
    A{-{-}{}AND7(AND)}
    B{-{-}{}AND7}
    C{-{-}{}AND7}
    AND7{-{-}{}D7(D7)}
{Highlighting}
{Shaded}
\end{verbatim}
\end{center}


{\def\LTcaptype{none} % do not increment counter
\vspace{-5pt}
\captionof{table}{3-થી-8 ડીકોડર ટ્રુથ ટેબલ}
\vspace{-10pt}
\begin{longtable}[]{@{}llllllllll@{}}
\toprule\noalign{}
ઇનપુટ્સ & & આઉટપુટ્સ & & & & & & & \\
\midrule\noalign{}
\endhead
\bottomrule\noalign{}
\endlastfoot
A & B & C & D0 & D1 & D2 & D3 & D4 & D5 & D6 \\
0 & 0 & 0 & 1 & 0 & 0 & 0 & 0 & 0 & 0 \\
0 & 0 & 1 & 0 & 1 & 0 & 0 & 0 & 0 & 0 \\
0 & 1 & 0 & 0 & 0 & 1 & 0 & 0 & 0 & 0 \\
0 & 1 & 1 & 0 & 0 & 0 & 1 & 0 & 0 & 0 \\
1 & 0 & 0 & 0 & 0 & 0 & 0 & 1 & 0 & 0 \\
1 & 0 & 1 & 0 & 0 & 0 & 0 & 0 & 1 & 0 \\
1 & 1 & 0 & 0 & 0 & 0 & 0 & 0 & 0 & 1 \\
1 & 1 & 1 & 0 & 0 & 0 & 0 & 0 & 0 & 0 \\
\end{longtable}
}

\begin{itemize}
\tightlist
\item
  \textbf{કાર્ય}: 3-બિટ બાઈનરી ઇનપુટના આધારે 8 આઉટપુટ લાઈનમાંથી એક સક્રિય કરે છે
\item
  \textbf{ઉપયોગો}: મેમરી એડ્રેસિંગ, ડેટા રાઉટિંગ, ઇન્સ્ટ્રક્શન ડિકોડિંગ
\item
  \textbf{બુલિયન સમીકરણો}: D0 = A'·B'·C', D1 = A'·B'·C, વગેરે.
\end{itemize}

\end{solutionbox}
\begin{mnemonicbox}
``બાઈનરી એડ્રેસ પર એક હોટ આઉટપુટ''

\end{mnemonicbox}
\subsection*{પ્રશ્ન 3(અ) OR [3
ગુણ]}\label{uxaaauxab0uxab6uxaa8-3uxa85-or-3-uxa97uxaa3}

\textbf{નિર્દેશ મુજબ કરો. 1) (101011010111)_{2} = (\_\_\_\_\_\_\_\_\_\_\_)_{8}}

\begin{solutionbox}


\textbf{Table: બાઈનરીથી ઑક્ટલ કન્વર્ઝન}
\begin{verbatim}
Binary:    1 | 010 | 110 | 101 | 11
           ↓    ↓     ↓     ↓    ↓
Octal:     1    2     6     5    3
\end{verbatim}

તેથી: (101011010111)_{2} = (12653)_{8}

\end{solutionbox}
\begin{mnemonicbox}
``જમણેથી ડાબે ત્રણના સમૂહમાં વિભાજિત કરો''

\end{mnemonicbox}
\subsection*{પ્રશ્ન 3(બ) OR [4
ગુણ]}\label{uxaaauxab0uxab6uxaa8-3uxaac-or-4-uxa97uxaa3}

\textbf{કનોફ મેપ (K' મેપ) પદ્ધતિનો ઉપયોગ કરીને બુલિયન સમીકરણને સરળ બનાવો:
F(A,B,C,D) = Σm(1,3,5,7,8,9,10,11)}

\begin{solutionbox}


\textbf{Table: કનોફ મેપ}
\begin{verbatim}
      CD      
AB    00  01  11  10
00    0   1   1   0
01    0   1   1   0
11    0   0   0   0
10    1   1   1   1
\end{verbatim}

\textbf{આકૃતિ: K-map ગ્રુપિંગ}

\begin{verbatim}
+{-{-}{-}{-}{-}+{-}{-}{-}{-}{-}+{-}{-}{-}{-}{-}+{-}{-}{-}{-}{-}+}
|  0  |  1  |  1  |  0  |
|     |  A  |  A  |     |
+{-{-}{-}{-}{-}+{-}{-}{-}{-}{-}+{-}{-}{-}{-}{-}+{-}{-}{-}{-}{-}+}
|  0  |  1  |  1  |  0  |
|     |  A  |  A  |     |
+{-{-}{-}{-}{-}+{-}{-}{-}{-}{-}+{-}{-}{-}{-}{-}+{-}{-}{-}{-}{-}+}
|  0  |  0  |  0  |  0  |
|     |     |     |     |
+{-{-}{-}{-}{-}+{-}{-}{-}{-}{-}+{-}{-}{-}{-}{-}+{-}{-}{-}{-}{-}+}
|  1  |  1  |  1  |  1  |
|  B  |  B  |  B  |  B  |
+{-{-}{-}{-}{-}+{-}{-}{-}{-}{-}+{-}{-}{-}{-}{-}+{-}{-}{-}{-}{-}+}
\end{verbatim}

ગ્રુપ A: A'CD (4 સેલ) ગ્રુપ B: AB' (4 સેલ)

સરળ અભિવ્યક્તિ: F(A,B,C,D) = A'CD + AB'

\end{solutionbox}
\begin{mnemonicbox}
``2ની ઘાતના સમૂહો બનાવો, ચલો ઘટાડો''

\end{mnemonicbox}
\subsection*{પ્રશ્ન 3(ક) OR [7
ગુણ]}\label{uxaaauxab0uxab6uxaa8-3uxa95-or-7-uxa97uxaa3}

\textbf{લોજિક સર્કિટ અને ટ્રુથ ટેબલનો ઉપયોગ કરીને 8 થી 1 મલ્ટિપ્લેક્સર સમજાવો.}

\begin{solutionbox}

\textbf{આકૃતિ: 8-થી-1 મલ્ટિપ્લેક્સર}

\begin{center}
\textbf{Mermaid Diagram (Code)}
\begin{verbatim}
{Shaded}
{Highlighting}[]
graph TD
    D0(D0){-{-}{}AND0(AND)}
    D1(D1){-{-}{}AND1(AND)}
    D2(D2){-{-}{}AND2(AND)}
    D3(D3){-{-}{}AND3(AND)}
    D4(D4){-{-}{}AND4(AND)}
    D5(D5){-{-}{}AND5(AND)}
    D6(D6){-{-}{}AND6(AND)}
    D7(D7){-{-}{}AND7(AND)}

    S0(S0){-{-}{}NOT0(NOT)}
    S1(S1){-{-}{}NOT1(NOT)}
    S2(S2){-{-}{}NOT2(NOT)}
    
    NOT0{-{-}{}AND0}
    NOT1{-{-}{}AND0}
    NOT2{-{-}{}AND0}
    
    S0{-{-}{}AND1}
    NOT1{-{-}{}AND1}
    NOT2{-{-}{}AND1}
    
    NOT0{-{-}{}AND2}
    S1{-{-}{}AND2}
    NOT2{-{-}{}AND2}
    
    S0{-{-}{}AND3}
    S1{-{-}{}AND3}
    NOT2{-{-}{}AND3}
    
    NOT0{-{-}{}AND4}
    NOT1{-{-}{}AND4}
    S2{-{-}{}AND4}
    
    S0{-{-}{}AND5}
    NOT1{-{-}{}AND5}
    S2{-{-}{}AND5}
    
    NOT0{-{-}{}AND6}
    S1{-{-}{}AND6}
    S2{-{-}{}AND6}
    
    S0{-{-}{}AND7}
    S1{-{-}{}AND7}
    S2{-{-}{}AND7}
    
    AND0{-{-}{}OR1(OR)}
    AND1{-{-}{}OR1}
    AND2{-{-}{}OR1}
    AND3{-{-}{}OR1}
    AND4{-{-}{}OR1}
    AND5{-{-}{}OR1}
    AND6{-{-}{}OR1}
    AND7{-{-}{}OR1}
    
    OR1{-{-}{}Y(આઉટપુટ Y)}
{Highlighting}
{Shaded}
\end{verbatim}
\end{center}


{\def\LTcaptype{none} % do not increment counter
\vspace{-5pt}
\captionof{table}{8-થી-1 મલ્ટિપ્લેક્સર ટ્રુથ ટેબલ}
\vspace{-10pt}
\begin{longtable}[]{@{}llll@{}}
\toprule\noalign{}
સિલેક્ટ લાઈન્સ & & & આઉટપુટ \\
\midrule\noalign{}
\endhead
\bottomrule\noalign{}
\endlastfoot
S2 & S1 & S0 & Y \\
0 & 0 & 0 & D0 \\
0 & 0 & 1 & D1 \\
0 & 1 & 0 & D2 \\
0 & 1 & 1 & D3 \\
1 & 0 & 0 & D4 \\
1 & 0 & 1 & D5 \\
1 & 1 & 0 & D6 \\
1 & 1 & 1 & D7 \\
\end{longtable}
}

\begin{itemize}
\tightlist
\item
  \textbf{કાર્ય}: 8 ઇનપુટ ડેટા લાઈન્સમાંથી એક પસંદ કરી આઉટપુટ પર રૂટ કરે છે
\item
  \textbf{ઉપયોગો}: ડેટા રૂટિંગ, ફંક્શન જનરેશન, પેરેલલ-ટુ-સીરિયલ કન્વર્ઝન
\item
  \textbf{બુલિયન સમીકરણ}: Y = S2'·S1'·S0'·D0 + S2'·S1'·S0·D1 + \ldots{} +
  S2·S1·S0·D7
\end{itemize}

\end{solutionbox}
\begin{mnemonicbox}
``સિલેક્ટ બિટ્સ એક ઇનપુટને આઉટપુટ પર મોકલે છે''

\end{mnemonicbox}
\subsection*{પ્રશ્ન 4(અ) [3
ગુણ]}\label{uxaaauxab0uxab6uxaa8-4uxa85-3-uxa97uxaa3}

\textbf{બાઈનરી થી ગ્રે કન્વર્ટર માટે લોજિક સર્કિટ દોરો.}

\begin{solutionbox}

\textbf{આકૃતિ: બાઈનરી થી ગ્રે કોડ કન્વર્ટર}

\begin{center}
\textbf{Mermaid Diagram (Code)}
\begin{verbatim}
{Shaded}
{Highlighting}[]
graph TD
    B3(B3){-{-}{}G3(G3)}
    B3{-{-}{}XOR1(XOR)}
    B2(B2){-{-}{}XOR1}
    XOR1{-{-}{}G2(G2)}
    B2{-{-}{}XOR2(XOR)}
    B1(B1){-{-}{}XOR2}
    XOR2{-{-}{}G1(G1)}
    B1{-{-}{}XOR3(XOR)}
    B0(B0){-{-}{}XOR3}
    XOR3{-{-}{}G0(G0)}
{Highlighting}
{Shaded}
\end{verbatim}
\end{center}

\begin{itemize}
\tightlist
\item
  \textbf{બાઈનરી ઇનપુટ્સ}: B3, B2, B1, B0 (સૌથી વધુથી ઓછા મહત્વના બિટ્સ)
\item
  \textbf{ગ્રે આઉટપુટ્સ}: G3, G2, G1, G0 (સૌથી વધુથી ઓછા મહત્વના બિટ્સ)
\item
  \textbf{કન્વર્ઝન નિયમ}: G3 = B3, G2 = B3 \oplus B2, G1 = B2 \oplus B1, G0 = B1 \oplus
  B0
\end{itemize}

\end{solutionbox}
\begin{mnemonicbox}
``પ્રથમ બિટ સરખી, બાકી પડોશીઓ સાથે XOR''

\end{mnemonicbox}
\subsection*{પ્રશ્ન 4(બ) [4
ગુણ]}\label{uxaaauxab0uxab6uxaa8-4uxaac-4-uxa97uxaa3}

\textbf{સીરિયલ ઇન સીરિયલ આઉટ શિફ્ટ રજિસ્ટરનું કામ સમજાવો}

\begin{solutionbox}

\textbf{આકૃતિ: સીરિયલ-ઇન સીરિયલ-આઉટ શિફ્ટ રજિસ્ટર}

\begin{center}
\textbf{Mermaid Diagram (Code)}
\begin{verbatim}
{Shaded}
{Highlighting}[]
graph LR
    Din(ડેટા ઇન){-{-}{}FF0(FF0)}
    CLK(ક્લોક){-{-}{}FF0}
    CLK{-{-}{}FF1(FF1)}
    CLK{-{-}{}FF2(FF2)}
    CLK{-{-}{}FF3(FF3)}
    FF0{-{-}{}FF1}
    FF1{-{-}{}FF2}
    FF2{-{-}{}FF3}
    FF3{-{-}{}Dout(ડેટા આઉટ)}
{Highlighting}
{Shaded}
\end{verbatim}
\end{center}


{\def\LTcaptype{none} % do not increment counter
\vspace{-5pt}
\captionof{table}{સીરિયલ-ઇન સીરિયલ-આઉટ ઓપરેશન}
\vspace{-10pt}
\begin{longtable}[]{@{}llllll@{}}
\toprule\noalign{}
ક્લોક સાયકલ & FF0 & FF1 & FF2 & FF3 & ડેટા આઉટ \\
\midrule\noalign{}
\endhead
\bottomrule\noalign{}
\endlastfoot
પ્રારંભિક & 0 & 0 & 0 & 0 & 0 \\
1 (Din=1) & 1 & 0 & 0 & 0 & 0 \\
2 (Din=0) & 0 & 1 & 0 & 0 & 0 \\
3 (Din=1) & 1 & 0 & 1 & 0 & 0 \\
4 (Din=1) & 1 & 1 & 0 & 1 & 1 \\
\end{longtable}
}

\begin{itemize}
\tightlist
\item
  \textbf{કાર્ય}: ડેટા બિટ્સ ઇનપુટ પર ક્રમશઃ દાખલ થાય છે, બધા ફ્લિપ-ફ્લોપ્સ દ્વારા
  શિફ્ટ થાય છે, અને ક્રમશઃ બહાર નીકળે છે
\item
  \textbf{ઉપયોગો}: ડેટા ટ્રાન્સમિશન, સમય વિલંબ, સીરિયલ-ટુ-સીરિયલ કન્વર્ઝન
\item
  \textbf{વિશેષતાઓ}: સરળ ડિઝાઇન, ઓછા I/O પિન્સ જરૂરી પણ વધુ ક્લોક સાયકલ્સ લાગે
\end{itemize}

\end{solutionbox}
\begin{mnemonicbox}
``એક બિટ અંદર, બધા શિફ્ટ, એક બિટ બહાર''

\end{mnemonicbox}
\subsection*{પ્રશ્ન 4(ક) [7
ગુણ]}\label{uxaaauxab0uxab6uxaa8-4uxa95-7-uxa97uxaa3}

\textbf{સર્કિટ ડાયાગ્રામ અને ટ્રુથ ટેબલનો ઉપયોગ કરીને D ફ્લિપ ફ્લોપ અને JK ફ્લિપ
ફ્લોપની કામગીરી સમજાવો.}

\begin{solutionbox}

\textbf{આકૃતિ: D ફ્લિપ-ફ્લોપ}

\begin{center}
\textbf{Mermaid Diagram (Code)}
\begin{verbatim}
{Shaded}
{Highlighting}[]
graph LR
    D(D){-{-}{}DFF(D ફ્લિપ{-}ફ્લોપ)}
    CLK(ક્લોક){-{-}{}DFF}
    DFF{-{-}{}Q(Q)}
    DFF{-{-}{}Q{}("Q{}")}
{Highlighting}
{Shaded}
\end{verbatim}
\end{center}


{\def\LTcaptype{none} % do not increment counter
\vspace{-5pt}
\captionof{table}{D ફ્લિપ-ફ્લોપ ટ્રુથ ટેબલ}
\vspace{-10pt}
\begin{longtable}[]{@{}lll@{}}
\toprule\noalign{}
D & ક્લોક & Q(આગામી) \\
\midrule\noalign{}
\endhead
\bottomrule\noalign{}
\endlastfoot
0 & ↑ & 0 \\
1 & ↑ & 1 \\
\end{longtable}
}

\textbf{આકૃતિ: JK ફ્લિપ-ફ્લોપ}

\begin{center}
\textbf{Mermaid Diagram (Code)}
\begin{verbatim}
{Shaded}
{Highlighting}[]
graph LR
    J(J){-{-}{}JKFF(JK ફ્લિપ{-}ફ્લોપ)}
    K(K){-{-}{}JKFF}
    CLK(ક્લોક){-{-}{}JKFF}
    JKFF{-{-}{}Q(Q)}
    JKFF{-{-}{}Q{}("Q{}")}
{Highlighting}
{Shaded}
\end{verbatim}
\end{center}


{\def\LTcaptype{none} % do not increment counter
\vspace{-5pt}
\captionof{table}{JK ફ્લિપ-ફ્લોપ ટ્રુથ ટેબલ}
\vspace{-10pt}
\begin{longtable}[]{@{}llll@{}}
\toprule\noalign{}
J & K & ક્લોક & Q(આગામી) \\
\midrule\noalign{}
\endhead
\bottomrule\noalign{}
\endlastfoot
0 & 0 & ↑ & Q(કોઈ ફેરફાર નહીં) \\
0 & 1 & ↑ & 0 \\
1 & 0 & ↑ & 1 \\
1 & 1 & ↑ & Q' (ટોગલ) \\
\end{longtable}
}

\begin{itemize}
\tightlist
\item
  \textbf{D ફ્લિપ-ફ્લોપ}: ડેટા (D) ઇનપુટ ક્લોકના પોઝિટિવ એજ પર આઉટપુટ Q પર
  ટ્રાન્સફર થાય છે
\item
  \textbf{JK ફ્લિપ-ફ્લોપ}: વધુ બહુમુખી, સેટ (J), રીસેટ (K), હોલ્ડ અને ટોગલ ક્ષમતાઓ
  સાથે
\item
  \textbf{ઉપયોગો}: સ્ટોરેજ તત્વો, કાઉન્ટર્સ, રજિસ્ટર્સ, સિક્વેન્શિયલ સર્કિટ્સ
\end{itemize}

\end{solutionbox}
\begin{mnemonicbox}
``D માં જે હોય તે Q માં જાય, JK ક્રમશઃ સેટ, રીસેટ, હોલ્ડ,
ટોગલ કરે''

\end{mnemonicbox}
\subsection*{પ્રશ્ન 4(અ) OR [3
ગુણ]}\label{uxaaauxab0uxab6uxaa8-4uxa85-or-3-uxa97uxaa3}

\textbf{ગ્રે થી બાઈનરી કન્વર્ટર માટે લોજિક સર્કિટ દોરો.}

\begin{solutionbox}

\textbf{આકૃતિ: ગ્રે થી બાઈનરી કોડ કન્વર્ટર}

\begin{center}
\textbf{Mermaid Diagram (Code)}
\begin{verbatim}
{Shaded}
{Highlighting}[]
graph LR
    G3(G3){-{-}{}B3(B3)}
    G3{-{-}{}XOR1(XOR)}
    G2(G2){-{-}{}XOR1}
    XOR1{-{-}{}B2(B2)}
    XOR1{-{-}{}XOR2(XOR)}
    G1(G1){-{-}{}XOR2}
    XOR2{-{-}{}B1(B1)}
    XOR2{-{-}{}XOR3(XOR)}
    G0(G0){-{-}{}XOR3}
    XOR3{-{-}{}B0(B0)}
{Highlighting}
{Shaded}
\end{verbatim}
\end{center}

\begin{itemize}
\tightlist
\item
  \textbf{ગ્રે ઇનપુટ્સ}: G3, G2, G1, G0 (સૌથી વધુથી ઓછા મહત્વના બિટ્સ)
\item
  \textbf{બાઈનરી આઉટપુટ્સ}: B3, B2, B1, B0 (સૌથી વધુથી ઓછા મહત્વના બિટ્સ)
\item
  \textbf{કન્વર્ઝન નિયમ}: B3 = G3, B2 = B3 \oplus G2, B1 = B2 \oplus G1, B0 = B1 \oplus
  G0
\end{itemize}

\end{solutionbox}
\begin{mnemonicbox}
``પ્રથમ બિટ સરખી, બાકી અગાઉના પરિણામ સાથે XOR''

\end{mnemonicbox}
\subsection*{પ્રશ્ન 4(બ) OR [4
ગુણ]}\label{uxaaauxab0uxab6uxaa8-4uxaac-or-4-uxa97uxaa3}

\textbf{પેરેલલ ઇન પેરેલલ આઉટ શિફ્ટ રજિસ્ટરનું કામ સમજાવો}

\begin{solutionbox}

\textbf{આકૃતિ: પેરેલલ-ઇન પેરેલલ-આઉટ શિફ્ટ રજિસ્ટર}

\begin{center}
\textbf{Mermaid Diagram (Code)}
\begin{verbatim}
{Shaded}
{Highlighting}[]
graph LR
    D0(D0){-{-}{}FF0(FF0)}
    D1(D1){-{-}{}FF1(FF1)}
    D2(D2){-{-}{}FF2(FF2)}
    D3(D3){-{-}{}FF3(FF3)}
    CLK(ક્લોક){-{-}{}FF0}
    CLK{-{-}{}FF1}
    CLK{-{-}{}FF2}
    CLK{-{-}{}FF3}
    LOAD(લોડ){-{-}{}FF0}
    LOAD{-{-}{}FF1}
    LOAD{-{-}{}FF2}
    LOAD{-{-}{}FF3}
    FF0{-{-}{}Q0(Q0)}
    FF1{-{-}{}Q1(Q1)}
    FF2{-{-}{}Q2(Q2)}
    FF3{-{-}{}Q3(Q3)}
{Highlighting}
{Shaded}
\end{verbatim}
\end{center}


{\def\LTcaptype{none} % do not increment counter
\vspace{-5pt}
\captionof{table}{પેરેલલ-ઇન પેરેલલ-આઉટ ઓપરેશન}
\vspace{-10pt}
\begin{longtable}[]{@{}llll@{}}
\toprule\noalign{}
લોડ & ક્લોક & D0-D3 & Q0-Q3 (ક્લોક પછી) \\
\midrule\noalign{}
\endhead
\bottomrule\noalign{}
\endlastfoot
1 & ↑ & 1010 & 1010 \\
0 & ↑ & xxxx & 1010 (કોઈ ફેરફાર નહીં) \\
1 & ↑ & 0101 & 0101 \\
\end{longtable}
}

\begin{itemize}
\tightlist
\item
  \textbf{કાર્ય}: ડેટા સમાંતરમાં લોડ થાય છે, બધા બિટ્સ એક સાથે આઉટપુટ પર ટ્રાન્સફર
  થાય છે
\item
  \textbf{ઉપયોગો}: ડેટા સ્ટોરેજ, બફરિંગ, કામચલાઉ હોલ્ડિંગ રજિસ્ટર્સ
\item
  \textbf{વિશેષતાઓ}: સૌથી ઝડપી રજિસ્ટર પ્રકાર, સૌથી વધુ I/O પિન્સ જરૂરી, બિટ
  શિફ્ટિંગ નથી
\end{itemize}

\end{solutionbox}
\begin{mnemonicbox}
``બધું અંદર, બધું બહાર, બધું એક સાથે''

\end{mnemonicbox}
\subsection*{પ્રશ્ન 4(ક) OR [7
ગુણ]}\label{uxaaauxab0uxab6uxaa8-4uxa95-or-7-uxa97uxaa3}

\textbf{સર્કિટ ડાયાગ્રામ અને ટ્રુથ ટેબલનો ઉપયોગ કરીને T ફ્લિપ ફ્લોપ અને SR ફ્લિપ
ફ્લોપની કામગીરી સમજાવો.}

\begin{solutionbox}

\textbf{આકૃતિ: T ફ્લિપ-ફ્લોપ}

\begin{center}
\textbf{Mermaid Diagram (Code)}
\begin{verbatim}
{Shaded}
{Highlighting}[]
graph LR
    T(T){-{-}{}TFF(T ફ્લિપ{-}ફ્લોપ)}
    CLK(ક્લોક){-{-}{}TFF}
    TFF{-{-}{}Q(Q)}
    TFF{-{-}{}Q{}("Q{}")}
{Highlighting}
{Shaded}
\end{verbatim}
\end{center}


{\def\LTcaptype{none} % do not increment counter
\vspace{-5pt}
\captionof{table}{T ફ્લિપ-ફ્લોપ ટ્રુથ ટેબલ}
\vspace{-10pt}
\begin{longtable}[]{@{}lll@{}}
\toprule\noalign{}
T & ક્લોક & Q(આગામી) \\
\midrule\noalign{}
\endhead
\bottomrule\noalign{}
\endlastfoot
0 & ↑ & Q (કોઈ ફેરફાર નહીં) \\
1 & ↑ & Q' (ટોગલ) \\
\end{longtable}
}

\textbf{આકૃતિ: SR ફ્લિપ-ફ્લોપ}

\begin{center}
\textbf{Mermaid Diagram (Code)}
\begin{verbatim}
{Shaded}
{Highlighting}[]
graph LR
    S(S){-{-}{}SRFF(SR ફ્લિપ{-}ફ્લોપ)}
    R(R){-{-}{}SRFF}
    CLK(ક્લોક){-{-}{}SRFF}
    SRFF{-{-}{}Q(Q)}
    SRFF{-{-}{}Q{}("Q{}")}
{Highlighting}
{Shaded}
\end{verbatim}
\end{center}


{\def\LTcaptype{none} % do not increment counter
\vspace{-5pt}
\captionof{table}{SR ફ્લિપ-ફ્લોપ ટ્રુથ ટેબલ}
\vspace{-10pt}
\begin{longtable}[]{@{}llll@{}}
\toprule\noalign{}
S & R & ક્લોક & Q(આગામી) \\
\midrule\noalign{}
\endhead
\bottomrule\noalign{}
\endlastfoot
0 & 0 & ↑ & Q (કોઈ ફેરફાર નહીં) \\
0 & 1 & ↑ & 0 (રીસેટ) \\
1 & 0 & ↑ & 1 (સેટ) \\
1 & 1 & ↑ & અમાન્ય \\
\end{longtable}
}

\begin{itemize}
\tightlist
\item
  \textbf{T ફ્લિપ-ફ્લોપ}: ટોગલ ફ્લિપ-ફ્લોપ જ્યારે T=1 હોય ત્યારે સ્થિતિ બદલે છે,
  જ્યારે T=0 હોય ત્યારે સ્થિતિ જાળવે છે
\item
  \textbf{SR ફ્લિપ-ફ્લોપ}: સેટ (S) અને રીસેટ (R) ઇનપુટ્સ સાથેનો મૂળભૂત ફ્લિપ-ફ્લોપ
\item
  \textbf{ઉપયોગો}: T ફ્લિપ-ફ્લોપ કાઉન્ટર્સ અને ફ્રિક્વન્સી ડિવાઇડર્સ માટે, SR મૂળભૂત
  મેમરી માટે
\end{itemize}

\end{solutionbox}
\begin{mnemonicbox}
``T ટ્રુ હોય ત્યારે ટોગલ કરે, SR સેટ અથવા રીસેટ કરે''

\end{mnemonicbox}
\subsection*{પ્રશ્ન 5(અ) [3
ગુણ]}\label{uxaaauxab0uxab6uxaa8-5uxa85-3-uxa97uxaa3}

\textbf{TTL, CMOS અને ECL લોજિક ફેમિલીની સરખામણી કરો.}

\begin{solutionbox}


{\def\LTcaptype{none} % do not increment counter
\vspace{-5pt}
\captionof{table}{લોજિક ફેમિલીઓની સરખામણી}
\vspace{-10pt}
\begin{longtable}[]{@{}llll@{}}
\toprule\noalign{}
પેરામીટર & TTL & CMOS & ECL \\
\midrule\noalign{}
\endhead
\bottomrule\noalign{}
\endlastfoot
પાવર વપરાશ & મધ્યમ & ખૂબ ઓછો & ઉચ્ચ \\
સ્પીડ & મધ્યમ & ઓછી-મધ્યમ & ખૂબ ઉચ્ચ \\
નોઇઝ ઇમ્યુનિટી & મધ્યમ & ઉચ્ચ & ઓછી \\
ફેન-આઉટ & 10 & \textgreater50 & 25 \\
સપ્લાય વોલ્ટેજ & +5V & +3V થી +15V & -5.2V \\
જટિલતા & મધ્યમ & ઓછી & ઉચ્ચ \\
\end{longtable}
}

\begin{itemize}
\tightlist
\item
  \textbf{TTL}: ટ્રાન્ઝિસ્ટર-ટ્રાન્ઝિસ્ટર લોજિક - સ્પીડ અને પાવરનું સારું સંતુલન
\item
  \textbf{CMOS}: કોમ્પ્લિમેન્ટરી મેટલ-ઑક્સાઇડ-સેમિકન્ડક્ટર - ઓછો પાવર, ઉચ્ચ ઘનતા
\item
  \textbf{ECL}: એમિટર-કપલ્ડ લોજિક - સૌથી વધુ સ્પીડ, ઉચ્ચ-પરફોર્મન્સ એપ્લિકેશન્સમાં
  વપરાય છે
\end{itemize}

\end{solutionbox}
\begin{mnemonicbox}
``TTL સમાધાન, CMOS કરકસર, ECL સ્પીડમાં શ્રેષ્ઠ''

\end{mnemonicbox}
\subsection*{પ્રશ્ન 5(બ) [4
ગુણ]}\label{uxaaauxab0uxab6uxaa8-5uxaac-4-uxa97uxaa3}

\textbf{લોજિક સર્કિટ ડાયાગ્રામ અને ટ્રુથ ટેબલની મદદથી દાયકા કાઉન્ટર સમજાવો.}

\begin{solutionbox}

\textbf{આકૃતિ: દાયકા કાઉન્ટર (BCD કાઉન્ટર)}

\begin{center}
\textbf{Mermaid Diagram (Code)}
\begin{verbatim}
{Shaded}
{Highlighting}[]
graph LR
    CLK(ક્લોક){-{-}{}JK0(JK FF0)}
    JK0{-{-}{}Q0(Q0)}
    JK0{-{-}{}Q0{}("Q0{}")}
    Q0{{-}{-}{}JK1(JK FF1)}
    JK1{-{-}{}Q1(Q1)}
    JK1{-{-}{}Q1{}("Q1{}")}
    Q1{{-}{-}{}JK2(JK FF2)}
    JK2{-{-}{}Q2(Q2)}
    JK2{-{-}{}Q2{}("Q2{}")}
    Q2{{-}{-}{}JK3(JK FF3)}
    JK3{-{-}{}Q3(Q3)}
    JK3{-{-}{}Q3{}("Q3{}")}
    Q3{-{-}{}NAND1(NAND)}
    Q1{-{-}{}NAND1}
    NAND1{-{-}{}CLEAR(ક્લિયર)}
    CLEAR{-{-}{}JK0}
    CLEAR{-{-}{}JK1}
    CLEAR{-{-}{}JK2}
    CLEAR{-{-}{}JK3}
{Highlighting}
{Shaded}
\end{verbatim}
\end{center}


{\def\LTcaptype{none} % do not increment counter
\vspace{-5pt}
\captionof{table}{દાયકા કાઉન્ટર સ્ટેટ્સ}
\vspace{-10pt}
\begin{longtable}[]{@{}lllll@{}}
\toprule\noalign{}
ગણતરી & Q3 & Q2 & Q1 & Q0 \\
\midrule\noalign{}
\endhead
\bottomrule\noalign{}
\endlastfoot
0 & 0 & 0 & 0 & 0 \\
1 & 0 & 0 & 0 & 1 \\
2 & 0 & 0 & 1 & 0 \\
3 & 0 & 0 & 1 & 1 \\
4 & 0 & 1 & 0 & 0 \\
5 & 0 & 1 & 0 & 1 \\
6 & 0 & 1 & 1 & 0 \\
7 & 0 & 1 & 1 & 1 \\
8 & 1 & 0 & 0 & 0 \\
9 & 1 & 0 & 0 & 1 \\
0 & 0 & 0 & 0 & 0 \\
\end{longtable}
}

\begin{itemize}
\tightlist
\item
  \textbf{કાર્ય}: 0 થી 9 (દશાંશ) સુધી ગણે છે અને પછી 0 પર રિસેટ થાય છે
\item
  \textbf{ઉપયોગો}: ડિજિટલ ઘડિયાળો, ફ્રિક્વન્સી ડિવાઇડર્સ, BCD કાઉન્ટર્સ
\item
  \textbf{વિશેષતાઓ}: 10ની ગણતરી પર ઑટો-રિસેટ, ક્લોક સાથે સિંક્રોનસ
\end{itemize}

\end{solutionbox}
\begin{mnemonicbox}
``એક દાયકો ગણે, નવ પછી રીસેટ''

\end{mnemonicbox}
\subsection*{પ્રશ્ન 5(ક) [7
ગુણ]}\label{uxaaauxab0uxab6uxaa8-5uxa95-7-uxa97uxaa3}

\textbf{મેમરીનું વિગતવાર વર્ગીકરણ આપો.}

\begin{solutionbox}

\textbf{આકૃતિ: મેમરી વર્ગીકરણ}

\begin{center}
\textbf{Mermaid Diagram (Code)}
\begin{verbatim}
{Shaded}
{Highlighting}[]
graph TD
    M[મેમરી]{-{-}{}PM[પ્રાઇમરી મેમરી]}
    M{-{-}{}SM[સેકન્ડરી મેમરી]}

    PM{-{-}{}RAM[RAM]}
    PM{-{-}{}ROM[ROM]}
    
    RAM{-{-}{}SRAM[સ્ટેટિક RAM]}
    RAM{-{-}{}DRAM[ડાયનેમિક RAM]}
    
    ROM{-{-}{}MROM[માસ્ક ROM]}
    ROM{-{-}{}PROM[પ્રોગ્રામેબલ ROM]}
    ROM{-{-}{}EPROM[ઇરેસેબલ PROM]}
    ROM{-{-}{}EEPROM[ઇલેક્ટ્રિકલી EPROM]}
    
    EEPROM{-{-}{}FLASH[ફ્લેશ મેમરી]}
    
    SM{-{-}{}HD[હાર્ડ ડિસ્ક]}
    SM{-{-}{}OD[ઑપ્ટિકલ ડિસ્ક]}
    SM{-{-}{}USB[USB ડ્રાઇવ]}
    SM{-{-}{}SD[SD કાર્ડ]}
{Highlighting}
{Shaded}
\end{verbatim}
\end{center}


{\def\LTcaptype{none} % do not increment counter
\vspace{-5pt}
\captionof{table}{મેમરી પ્રકારોની સરખામણી}
\vspace{-10pt}
\begin{longtable}[]{@{}lllll@{}}
\toprule\noalign{}
મેમરી પ્રકાર & વોલેટિલિટી & રીડ/રાઇટ & એક્સેસ સ્પીડ & સામાન્ય ઉપયોગ \\
\midrule\noalign{}
\endhead
\bottomrule\noalign{}
\endlastfoot
SRAM & વોલેટાઇલ & R/W & ખૂબ ઝડપી & કેશ મેમરી \\
DRAM & વોલેટાઇલ & R/W & ઝડપી & મુખ્ય મેમરી \\
ROM & નોન-વોલેટાઇલ & માત્ર વાંચન & મધ્યમ & BIOS, ફર્મવેર \\
PROM & નોન-વોલેટાઇલ & એકવાર લખાણ & મધ્યમ & કાયમી પ્રોગ્રામ્સ \\
EPROM & નોન-વોલેટાઇલ & UV દ્વારા ભૂંસી શકાય & મધ્યમ & અપગ્રેડેબલ ફર્મવેર \\
EEPROM & નોન-વોલેટાઇલ & ઇલેક્ટ્રિકલી ભૂંસી શકાય & મધ્યમ & કોન્ફિગરેશન ડેટા \\
ફ્લેશ & નોન-વોલેટાઇલ & બ્લોક ભૂંસી શકાય & મધ્યમ-ઝડપી & સ્ટોરેજ ડિવાઇસ \\
\end{longtable}
}

\begin{itemize}
\tightlist
\item
  \textbf{RAM (રેન્ડમ એક્સેસ મેમરી)}: અસ્થાયી, વોલેટાઇલ વર્કિંગ મેમરી
\item
  \textbf{ROM (રીડ ઓન્લી મેમરી)}: કાયમી, નોન-વોલેટાઇલ પ્રોગ્રામ સ્ટોરેજ
\item
  \textbf{વિશેષતાઓ}: એક્સેસ ટાઇમ, ડેટા રિટેન્શન, ક્ષમતા, બિટ દીઠ કિંમત
\end{itemize}

\end{solutionbox}
\begin{mnemonicbox}
``RAM અદૃશ્ય થાય, ROM રહી જાય''

\end{mnemonicbox}
\subsection*{પ્રશ્ન 5(અ) OR [3
ગુણ]}\label{uxaaauxab0uxab6uxaa8-5uxa85-or-3-uxa97uxaa3}

\textbf{વ્યાખ્યાયિત કરો: ફેન આઉટ, ફેન ઇન અને ફિગર ઓફ મેરિટ.}

\begin{solutionbox}


{\def\LTcaptype{none} % do not increment counter
\vspace{-5pt}
\captionof{table}{ડિજિટલ લોજિક પેરામીટર્સ}
\vspace{-10pt}
\begin{longtable}[]{@{}
  >{\raggedright\arraybackslash}p{(\linewidth - 4\tabcolsep) * \real{0.2821}}
  >{\raggedright\arraybackslash}p{(\linewidth - 4\tabcolsep) * \real{0.3077}}
  >{\raggedright\arraybackslash}p{(\linewidth - 4\tabcolsep) * \real{0.4103}}@{}}
\toprule\noalign{}
\begin{minipage}[b]{\linewidth}\raggedright
પેરામીટર
\end{minipage} & \begin{minipage}[b]{\linewidth}\raggedright
વ્યાખ્યા
\end{minipage} & \begin{minipage}[b]{\linewidth}\raggedright
સામાન્ય મૂલ્યો
\end{minipage} \\
\midrule\noalign{}
\endhead
\bottomrule\noalign{}
\endlastfoot
ફેન-આઉટ & એક ગેટ આઉટપુટ ડ્રાઇવ કરી શકે તેવા સ્ટાન્ડર્ડ લોડ્સની સંખ્યા & TTL: 10,
CMOS: \textgreater50 \\
ફેન-ઇન & એક લોજિક ગેટ સંભાળી શકે તેવા ઇનપુટ્સની સંખ્યા & TTL: 8, CMOS: 100+ \\
ફિગર ઓફ મેરિટ & સ્પીડ-પાવર પ્રોડક્ટ (પ્રોપેગેશન ડિલે \times પાવર કન્ઝમ્પશન) & ઓછું હોય તે
સારું \\
\end{longtable}
}

\begin{itemize}
\tightlist
\item
  \textbf{ફેન-આઉટ}: એક ગેટ આઉટપુટથી જોડી શકાય તેવા ગેટ ઇનપુટ્સની મહત્તમ સંખ્યા
\item
  \textbf{ફેન-ઇન}: એક જ લોજિક ગેટ પર ઉપલબ્ધ ઇનપુટ્સની મહત્તમ સંખ્યા
\item
  \textbf{ફિગર ઓફ મેરિટ}: વિવિધ લોજિક ફેમિલીઓની તુલના માટેનો ગુણવત્તા ફેક્ટર
\end{itemize}

\end{solutionbox}
\begin{mnemonicbox}
``આઉટ ઘણાને ચલાવે, ઇન ઘણા સ્વીકારે, મેરિટ સારપ માપે''

\end{mnemonicbox}
\subsection*{પ્રશ્ન 5(બ) OR [4
ગુણ]}\label{uxaaauxab0uxab6uxaa8-5uxaac-or-4-uxa97uxaa3}

\textbf{લોજિક સર્કિટ ડાયાગ્રામ અને ટ્રુથ ટેબલની મદદથી અસિંક્રોનસ અપ કાઉન્ટર
સમજાવો.}

\begin{solutionbox}

\textbf{આકૃતિ: 4-બિટ અસિંક્રોનસ અપ કાઉન્ટર}

\begin{center}
\textbf{Mermaid Diagram (Code)}
\begin{verbatim}
{Shaded}
{Highlighting}[]
graph LR
    CLK(ક્લોક){-{-}{}TFF0(T FF0)}
    TFF0{-{-}{}Q0(Q0)}
    TFF0{-{-}{}Q0{}("Q0{}")}
    Q0{-{-}{}TFF1(T FF1)}
    TFF1{-{-}{}Q1(Q1)}
    TFF1{-{-}{}Q1{}("Q1{}")}
    Q1{-{-}{}TFF2(T FF2)}
    TFF2{-{-}{}Q2(Q2)}
    TFF2{-{-}{}Q2{}("Q2{}")}
    Q2{-{-}{}TFF3(T FF3)}
    TFF3{-{-}{}Q3(Q3)}
    TFF3{-{-}{}Q3{}("Q3{}")}

    CLR(ક્લિયર){-{-}{}TFF0}
    CLR{-{-}{}TFF1}
    CLR{-{-}{}TFF2}
    CLR{-{-}{}TFF3}
{Highlighting}
{Shaded}
\end{verbatim}
\end{center}


{\def\LTcaptype{none} % do not increment counter
\vspace{-5pt}
\captionof{table}{4-બિટ અસિંક્રોનસ કાઉન્ટર સ્ટેટ્સ}
\vspace{-10pt}
\begin{longtable}[]{@{}lllll@{}}
\toprule\noalign{}
ગણતરી & Q3 & Q2 & Q1 & Q0 \\
\midrule\noalign{}
\endhead
\bottomrule\noalign{}
\endlastfoot
0 & 0 & 0 & 0 & 0 \\
1 & 0 & 0 & 0 & 1 \\
2 & 0 & 0 & 1 & 0 \\
\ldots{} & .. & .. & .. & .. \\
14 & 1 & 1 & 1 & 0 \\
15 & 1 & 1 & 1 & 1 \\
\end{longtable}
}

\begin{itemize}
\tightlist
\item
  \textbf{કાર્ય}: દરેક ફ્લિપ-ફ્લોપ 1 થી 0 પર ટ્રાન્ઝિશન થતાં આગલાને ટ્રિગર કરે છે
\item
  \textbf{વિશેષતાઓ}: સરળ ડિઝાઇન પરંતુ પ્રોપેગેશન ડિલે (રિપલ)ની સમસ્યા
\item
  \textbf{ઉપયોગો}: ફ્રિક્વન્સી ડિવિઝન, બેઝિક કાઉન્ટિંગ એપ્લિકેશન્સ
\end{itemize}

\end{solutionbox}
\begin{mnemonicbox}
``ઉપર તરફ લહેરો, દરેક બિટ આગલાને ટ્રિગર કરે''

\end{mnemonicbox}
\subsection*{પ્રશ્ન 5(ક) OR [7
ગુણ]}\label{uxaaauxab0uxab6uxaa8-5uxa95-or-7-uxa97uxaa3}

\textbf{ડિજિટલ IC ના ઇ-વેસ્ટ મેનેજમેન્ટના પગલાં અને જરૂરિયાતનું વર્ણન કરો.}

\begin{solutionbox}

\textbf{આકૃતિ: ઇ-વેસ્ટ મેનેજમેન્ટ સાયકલ}

\begin{center}
\textbf{Mermaid Diagram (Code)}
\begin{verbatim}
{Shaded}
{Highlighting}[]
graph LR
    C[એકત્રીકરણ]{-{-}{}S[વર્ગીકરણ]}
    S{-{-}{}D[ડિસ્એસેમ્બલી]}
    D{-{-}{}R[રિસાયકલિંગ]}
    R{-{-}{}M[મટિરિયલ રિકવરી]}
    M{-{-}{}N[નવા ઉત્પાદનો]}
    N{-{-}{}C}
{Highlighting}
{Shaded}
\end{verbatim}
\end{center}


{\def\LTcaptype{none} % do not increment counter
\vspace{-5pt}
\captionof{table}{ઇ-વેસ્ટ મેનેજમેન્ટના પગલાં}
\vspace{-10pt}
\begin{longtable}[]{@{}lll@{}}
\toprule\noalign{}
પગલું & વર્ણન & મહત્વ \\
\midrule\noalign{}
\endhead
\bottomrule\noalign{}
\endlastfoot
એકત્રીકરણ & જૂના IC એકત્રિત કરવા & ખોટા નિકાલ રોકે છે \\
વર્ગીકરણ & પ્રકાર અનુસાર વર્ગીકરણ & કાર્યક્ષમ પ્રક્રિયા માટે \\
ડિસ્એસેમ્બલી & ઘટકોને અલગ કરવા & મટિરિયલ રિકવરી સરળ બનાવે છે \\
રિસાયકલિંગ & મટિરિયલ્સ પ્રોસેસિંગ & પર્યાવરણ પ્રભાવ ઘટાડે છે \\
મટિરિયલ રિકવરી & મૂલ્યવાન ધાતુઓ મેળવવી & સંસાધનો સંરક્ષિત કરે છે \\
સુરક્ષિત નિકાલ & વિષાક્ત ઘટકોનું સંચાલન & પ્રદૂષણ અટકાવે છે \\
\end{longtable}
}

\begin{itemize}
\tightlist
\item
  \textbf{ઇ-વેસ્ટ મેનેજમેન્ટની જરૂરિયાત}:

  \begin{itemize}
  \tightlist
  \item
    \textbf{પર્યાવરણ રક્ષણ}: વિષાક્ત પદાર્થોને જમીન/પાણીમાં મિશ્રિત થતા રોકે છે
  \item
    \textbf{સંસાધન સંરક્ષણ}: સોનું, ચાંદી, તાંબુ જેવી મૂલ્યવાન ધાતુઓ પુનઃપ્રાપ્ત કરે છે
  \item
    \textbf{આરોગ્ય સુરક્ષા}: લેડ, પારા જેવા જોખમી પદાર્થોના સંપર્કને ઘટાડે છે
  \item
    \textbf{કાયદાકીય અનુપાલન}: ઇલેક્ટ્રોનિક કચરા અંગેના નિયમોનું પાલન કરે છે
  \end{itemize}
\end{itemize}

\end{solutionbox}
\begin{mnemonicbox}
``એકત્રિત કરો, વર્ગીકૃત કરો, છૂટા પાડો, રિસાયકલ કરો,
પુનઃપ્રાપ્ત કરો, ફરીથી વાપરો''

\end{mnemonicbox}

\end{document}
