\documentclass[10pt,a4paper]{article}

% content/resources/templates/preamble.tex
\usepackage[margin=0.6in]{geometry}
\author{Milav Dabgar}
\usepackage{amsmath,amssymb,amsthm}
\usepackage{booktabs}
\usepackage{multirow}
\usepackage{xcolor}
\usepackage{tcolorbox}
\tcbuselibrary{breakable,skins}
\usepackage[colorlinks=true,linkcolor=blue]{hyperref}
\usepackage{titlesec}
\usepackage{enumitem}
\usepackage{tikz}
\usepackage{pgfplots}
\usepackage{circuitikz}
\usepackage[version=4]{mhchem}
\usepackage{longtable}
\usepackage{array}
\usepackage{float}
\usepackage{caption}
\usepackage{listings}

\lstset{
  basicstyle=\small\ttfamily,
  breaklines=true,
  breakatwhitespace=false,
  postbreak=\mbox{\textcolor{red}{$\hookrightarrow$}\space},
  float=false,
  numbers=left,
  numberstyle=\tiny\color{gray},
  numbersep=10pt,
  xleftmargin=2em,
  keywordstyle=\color{blue},
  commentstyle=\color{green!60!black},
  stringstyle=\color{purple},
  backgroundcolor=\color{gray!5},
  showstringspaces=false,
  tabsize=2,
  captionpos=b,
  keepspaces=true,
  columns=flexible
}

\pgfplotsset{compat=1.18}
\usetikzlibrary{shapes,arrows,positioning,calc,patterns,decorations.pathmorphing,decorations.markings,arrows.meta}

% Color scheme
\definecolor{headcolor}{RGB}{0,102,204}
\definecolor{keycolor}{RGB}{220,20,60}
\definecolor{solutioncolor}{RGB}{34,139,34}
\definecolor{mnemoniccolor}{RGB}{148,0,211}
\definecolor{codecolor}{RGB}{0,0,100}

% Spacing
\setlength{\parskip}{3pt}
\setlist[itemize]{nosep}
\setlist[enumerate]{nosep}

% Title formatting
\titleformat{\section}{\Large\bfseries\color{headcolor}}{\thesection}{1em}{}
\titleformat{\subsection}{\large\bfseries\color{headcolor}}{\thesubsection}{1em}{}

% Pandoc tightlist compatibility
\providecommand{\tightlist}{%
  \setlength{\itemsep}{0pt}\setlength{\parskip}{0pt}}

% Pandoc longtable compatibility
\newcounter{none}
\def\thenone{}


% content/resources/templates/english-boxes.tex
% This file is currently empty - it exists to maintain consistency with the import structure.
% Add custom environments here if needed in the future.


\begin{document}

\begin{center}
{\Huge\bfseries\color{headcolor} Subject Name Solutions}\\[5pt]
{\LARGE 4321102 -- Winter 2024}\\[3pt]
{\large Semester 1 Study Material}\\[3pt]
{\normalsize\textit{Detailed Solutions and Explanations}}
\end{center}

\vspace{10pt}

\subsection*{Question 1(a) [3 marks]}\label{q1a}

\textbf{Draw symbols and write Logic table of NAND and Ex-NOR gate}

\begin{solutionbox}

\textbf{NAND and Ex-NOR Gate Symbols and Truth Tables:}

\begin{verbatim}
         NAND Gate                 Ex{-NOR Gate}
         \_\_\_\_\_\_\_                          \_\_\_\_\_\_\_
A {-{-}{-}{-}{-}|       |                A {-}{-}{-}{-}{-}|       |}
        |   \&   |{-{-}Y                    |   =   |{-}{-}Y}
B {-{-}{-}{-}{-}|\_\_\_\_\_\_\_|                B {-}{-}{-}{-}{-}|\_\_\_\_\_\_\_|}
        bubble output                 bubble output
\end{verbatim}

{\def\LTcaptype{none} % do not increment counter
\begin{longtable}[]{@{}lll@{}}
\toprule\noalign{}
A & B & Y (NAND) \\
\midrule\noalign{}
\endhead
\bottomrule\noalign{}
\endlastfoot
0 & 0 & 1 \\
0 & 1 & 1 \\
1 & 0 & 1 \\
1 & 1 & 0 \\
\end{longtable}
}

{\def\LTcaptype{none} % do not increment counter
\begin{longtable}[]{@{}lll@{}}
\toprule\noalign{}
A & B & Y (Ex-NOR) \\
\midrule\noalign{}
\endhead
\bottomrule\noalign{}
\endlastfoot
0 & 0 & 1 \\
0 & 1 & 0 \\
1 & 0 & 0 \\
1 & 1 & 1 \\
\end{longtable}
}

\begin{itemize}
\tightlist
\item
  \textbf{NAND gate}: Output is LOW only when all inputs are HIGH
\item
  \textbf{Ex-NOR gate}: Output is HIGH when inputs are SAME
\end{itemize}

\end{solutionbox}
\begin{mnemonicbox}
``NAND says NO to ALL ones, Ex-NOR says YES to SAME
signals''

\end{mnemonicbox}
\subsection*{Question 1(b) [4 marks]}\label{q1b}

\textbf{Do as Directed: (i) (1011001)_{2} - (1001101)_{2} Using 2's Complement
(ii) (10110101)_{2} = ( )_{1}_{0} = ( )_{1}_{6}}

\begin{solutionbox}

\textbf{(i) Subtraction using 2's complement:}

\begin{verbatim}
Step 1: Find 2's complement of subtrahend (1001101)_{2}
        1's complement: 0110010
        Add 1:          0110011

Step 2: Add minuend and 2's complement
        1011001
      + 0110011
        -------
       10001100

Step 3: Discard overflow bit
        Result = 0001100 = (0001100)_{2}
\end{verbatim}

\textbf{(ii) Conversion of (10110101)_{2}:}

\begin{verbatim}
To Decimal:
1\times2^{7} + 0\times2^{6} + 1\times2^{5} + 1\times2^{4} + 0\times2^{3} + 1\times2^{2} + 0\times2^{1} + 1\times2^{0}
= 128 + 0 + 32 + 16 + 0 + 4 + 0 + 1
= 181_{1}_{0}

To Hexadecimal:
1011 0101
 B    5
= B5_{1}_{6}
\end{verbatim}

\begin{itemize}
\tightlist
\item
  \textbf{2's complement}: Invert bits and add 1
\item
  \textbf{Binary to decimal}: Multiply each bit by its position value
  (2^{n})
\item
  \textbf{Binary to hex}: Group bits in fours, convert each group
\end{itemize}

\end{solutionbox}
\begin{mnemonicbox}
``Flip bits Add 1, Drop the carry''

\end{mnemonicbox}
\subsection*{Question 1(c) [7 marks]}\label{q1c}

\textbf{Find (i) (4356)_{1}_{0} = ( )_{8} = ( )_{1}_{6} = ()_{2} (ii) (101.01)_{2} \times (11.01)_{2}
(iii) Divide (101101)_{2} with (110)_{2}}

\begin{solutionbox}

\textbf{(i) Number system conversion:}

\begin{verbatim}
Decimal to Octal:
4356 \div 8 = 544 remainder 4
544 \div 8 = 68  remainder 0
68 \div 8 = 8    remainder 4
8 \div 8 = 1     remainder 0
1 \div 8 = 0     remainder 1
Reading from bottom: (4356)_{1}_{0} = (10404)_{8}

Decimal to Hexadecimal:
4356 \div 16 = 272 remainder 4
272 \div 16 = 17  remainder 0
17 \div 16 = 1    remainder 1
1 \div 16 = 0     remainder 1
Reading from bottom: (4356)_{1}_{0} = (1104)_{1}_{6}

Decimal to Binary:
4356 = 1000100000100_{2}
\end{verbatim}

\textbf{(ii) Binary multiplication:}

\begin{verbatim}
      101.01
    \times 11.01
    -------
      10101
     10101
    10101
   10101
   ---------
   1111.1101
\end{verbatim}

\textbf{(iii) Binary division:}

\begin{verbatim}
          111.
       ------
110 ) 101101
      110
      -----
       11101
       110
       -----
       1001
        110
        ----
        11
\end{verbatim}

\begin{itemize}
\tightlist
\item
  \textbf{Decimal to Octal}: Divide repeatedly by 8
\item
  \textbf{Decimal to Hex}: Divide repeatedly by 16
\item
  \textbf{Binary operations}: Follow same process as decimal
\end{itemize}

\end{solutionbox}
\begin{mnemonicbox}
``Divide and stack up the remainders from bottom to
top''

\end{mnemonicbox}
\subsection*{Question 1(c-OR) [7
marks]}\label{question-1c-or-7-marks}

\textbf{Find (i) (8642)_{1}_{0} = ( )_{8} = ( )_{1}_{6} = ()_{2} (ii) Draw symbols and
write Logic table of NOR and Ex-OR gate}

\begin{solutionbox}

\textbf{(i) Number system conversion:}

\begin{verbatim}
Decimal to Octal:
8642 \div 8 = 1080 remainder 2
1080 \div 8 = 135  remainder 0
135 \div 8 = 16    remainder 7
16 \div 8 = 2      remainder 0
2 \div 8 = 0       remainder 2
Reading from bottom: (8642)_{1}_{0} = (20702)_{8}

Decimal to Hexadecimal:
8642 \div 16 = 540 remainder 2
540 \div 16 = 33   remainder 12(C)
33 \div 16 = 2     remainder 1
2 \div 16 = 0      remainder 2
Reading from bottom: (8642)_{1}_{0} = (21C2)_{1}_{6}

Decimal to Binary:
8642 = 10000111000010_{2}
\end{verbatim}

\textbf{(ii) NOR and Ex-OR Gates:}

\begin{verbatim}
         NOR Gate                 Ex{-OR Gate}
         \_\_\_\_\_\_\_                          \_\_\_\_\_\_\_
A {-{-}{-}{-}{-}|       |                A {-}{-}{-}{-}{-}|       |}
        |  1   |{-{-}Y                    |   =   |{-}{-}Y}
B {-{-}{-}{-}{-}|\_\_\_\_\_\_\_|                B {-}{-}{-}{-}{-}|\_\_\_\_\_\_\_|}
         bubble output
\end{verbatim}

{\def\LTcaptype{none} % do not increment counter
\begin{longtable}[]{@{}lll@{}}
\toprule\noalign{}
A & B & Y (NOR) \\
\midrule\noalign{}
\endhead
\bottomrule\noalign{}
\endlastfoot
0 & 0 & 1 \\
0 & 1 & 0 \\
1 & 0 & 0 \\
1 & 1 & 0 \\
\end{longtable}
}

{\def\LTcaptype{none} % do not increment counter
\begin{longtable}[]{@{}lll@{}}
\toprule\noalign{}
A & B & Y (Ex-OR) \\
\midrule\noalign{}
\endhead
\bottomrule\noalign{}
\endlastfoot
0 & 0 & 0 \\
0 & 1 & 1 \\
1 & 0 & 1 \\
1 & 1 & 0 \\
\end{longtable}
}

\begin{itemize}
\tightlist
\item
  \textbf{NOR gate}: Output is HIGH only when ALL inputs are LOW
\item
  \textbf{Ex-OR gate}: Output is HIGH when inputs are DIFFERENT
\end{itemize}

\end{solutionbox}
\begin{mnemonicbox}
``NOR says YES to ALL zeros, Ex-OR says YES to
DIFFERENT signals''

\end{mnemonicbox}
\subsection*{Question 2(a) [3 marks]}\label{q2a}

\textbf{Prove xy+xz+yz' = xz+yz'}

\begin{solutionbox}

\begin{verbatim}
Left side: xy + xz + yz'
= xy + xz + yz'
= x(y + z) + yz'        [Distributive property]
= xy + xz + yz'         [Expand]
= xy + yz' + xz         [Rearranging]
= y(x + z') + xz        [Distributive property]
= xy + yz' + xz         [Expand]
= (x + y)z' + xz        [Rearranging]
= xz' + yz' + xz        [Expand]
= x(z' + z) + yz'       [Distributive property]
= x(1) + yz'            [Complement property]
= x + yz'               [Identity property]
= xz + x(1-z) + yz'     [x = xz + xz']
= xz + xz' + yz'        [Expand]
= xz + z'(x + y)        [Distributive property]
= xz + z'x + z'y        [Expand]
= xz + xz' + yz'        [Rearranging]
= x(z + z') + yz'       [Distributive property]
= x(1) + yz'            [Complement property]
= x + yz'               [Identity property]
= xz + yz'              [Same as right side]
\end{verbatim}

\begin{itemize}
\tightlist
\item
  \textbf{Distributive property}: x(y+z) = xy+xz
\item
  \textbf{Complement property}: z+z' = 1
\item
  \textbf{Identity property}: x\times1 = x
\end{itemize}

\end{solutionbox}
\begin{mnemonicbox}
``Factor, Expand, Rearrange, Factor again''

\end{mnemonicbox}
\subsection*{Question 2(b) [4 marks]}\label{q2b}

\textbf{Reduce Expression f(W,X,Y,Z) = \summ(0,1,2,3,5,7,8,9,11,14) using
K-Map method.}

\begin{solutionbox}

\textbf{K-Map for f(W,X,Y,Z) = \summ(0,1,2,3,5,7,8,9,11,14):}

\begin{verbatim}
      YZ
WX   00  01  11  10
00    1   1   0   1
01    1   1   1   0
11    0   0   1   1
10    1   1   0   0
\end{verbatim}

\textbf{Grouping:}

\begin{itemize}
\tightlist
\item
  Group 1: m(0,1,2,3) = W'X' (2\times2 group)
\item
  Group 2: m(0,1,8,9) = Y' (2\times2 group)
\item
  Group 3: m(2,3,11) = X'Z (2\times2 group with wrapping)
\item
  Group 4: m(7,14) = XZ (pair)
\end{itemize}

\textbf{Simplified expression:} f(W,X,Y,Z) = W'X' + Y' + X'Z + XZ

\begin{itemize}
\tightlist
\item
  \textbf{K-Map technique}: Group adjacent 1's in powers of 2
\item
  \textbf{Each group}: Represents one term in simplified expression
\item
  \textbf{Larger groups}: Mean simpler expressions
\end{itemize}

\end{solutionbox}
\begin{mnemonicbox}
``Powers of 2 make expressions new''

\end{mnemonicbox}
\subsection*{Question 2(c) [7 marks]}\label{q2c}

\textbf{Explain NOR gate as Universal Gate}

\begin{solutionbox}

\textbf{NOR as Universal Gate:}

NOR gate can implement all basic logic functions, making it a universal
gate.

\textbf{Implementation of basic gates using NOR:}

{\def\LTcaptype{none} % do not increment counter
\begin{longtable}[]{@{}ll@{}}
\toprule\noalign{}
Gate & Implementation with NOR \\
\midrule\noalign{}
\endhead
\bottomrule\noalign{}
\endlastfoot
NOT & A NOR A \\
OR & (A NOR B) NOR (A NOR B) \\
AND & (A NOR A) NOR (B NOR B) \\
\end{longtable}
}

\textbf{Circuit diagrams:}

\begin{verbatim}
NOT Gate using NOR:
A--->|NOR|-->Y

OR Gate using NOR:
A--->|     |      |     |
     | NOR |------|     |
B--->|     |      | NOR |-->Y
                  |     |
                  |     |

AND Gate using NOR:
A--->|NOR|-------|     |
                 |     |
                 | NOR |-->Y
                 |     |
B--->|NOR|-------|     |
\end{verbatim}

\begin{itemize}
\tightlist
\item
  \textbf{Universal gate}: Can implement any Boolean function
\item
  \textbf{NOR operation}: NOT OR, output high only when all inputs low
\item
  \textbf{Implementation cost}: Requires multiple NOR gates for complex
  functions
\end{itemize}

\end{solutionbox}
\begin{mnemonicbox}
``NOR stands for Not-OR, but can do Not-AND-OR all''

\end{mnemonicbox}
\subsection*{Question 2(a-OR) [3
marks]}\label{question-2a-or-3-marks}

\textbf{Draw Logic circuit for Boolean expression P =
(x'+y'+z)(x+y+z')+(xyz)}

\begin{solutionbox}

\textbf{Logic Circuit for P = (x'+y'+z)(x+y+z')+(xyz):}

\begin{verbatim}
      |{-{-}{-}{-}{-}{-}|}
x{{-}{-}{-}|      |}
y{{-}{-}{-}| OR   |{-}{-}{-}{-}{-}|}
z{-{-}{-}{-}|      |     |     |{-}{-}{-}{-}{-}{-}|}
      |{-{-}{-}{-}{-}{-}|     |{-}{-}{-}{-}|      |}
                         | AND  |{-{-}{-}{-}{-}{-}|}
      |{-{-}{-}{-}{-}{-}|     |{-}{-}{-}{-}|      |      |}
x{-{-}{-}{-}|      |     |     |{-}{-}{-}{-}{-}{-}|      |     |{-}{-}{-}{-}{-}{-}|}
y{-{-}{-}{-}| OR   |{-}{-}{-}{-}{-}|                   |{-}{-}{-}{-}|      |}
z{{-}{-}{-}|      |                         |     | OR   |{-}{-}{-} P}
      |{-{-}{-}{-}{-}{-}|                         |     |      |}
                                       |     |{-{-}{-}{-}{-}{-}|}
      |{-{-}{-}{-}{-}{-}|                         |}
x{-{-}{-}{-}|      |                         |}
y{-{-}{-}{-}| AND  |{-}{-}{-}{-}{-}{-}{-}{-}{-}{-}{-}{-}{-}{-}{-}{-}{-}{-}{-}{-}{-}{-}{-}{-}{-}|}
z{-{-}{-}{-}|      |}
      |{-{-}{-}{-}{-}{-}|}
\end{verbatim}

\begin{itemize}
\tightlist
\item
  \textbf{Step 1}: Implement each product term
\item
  \textbf{Step 2}: Then combine with OR gate
\item
  \textbf{Step 3}: Follow operator precedence
\end{itemize}

\end{solutionbox}
\begin{mnemonicbox}
``Products first, then sum them up''

\end{mnemonicbox}
\subsection*{Question 2(b-OR) [4
marks]}\label{question-2b-or-4-marks}

\textbf{Reduce Expression using K-Map method f(W,X,Y,Z) =
\summ(1,3,7,11,15) and don't care condition are d(0,2,5)}

\begin{solutionbox}

\textbf{K-Map with don't care conditions:}

\begin{verbatim}
      YZ
WX   00  01  11  10
00    d   1   0   d
01    0   0   1   d
11    0   0   1   1
10    0   0   1   0
\end{verbatim}

\textbf{Grouping:}

\begin{itemize}
\tightlist
\item
  Group 1: m(1,3,7,15) + d(0,2) = X'Z + YZ (pairs)
\item
  Group 2: m(7,15) + d(5) = WYZ (quad)
\end{itemize}

\textbf{Simplified expression:} f(W,X,Y,Z) = X'Z + YZ

\begin{itemize}
\tightlist
\item
  \textbf{Don't care conditions}: Can be treated as 0 or 1 based on
  convenience
\item
  \textbf{Optimal grouping}: Use don't cares to form larger groups
\item
  \textbf{Simplification goal}: Minimize number of terms
\end{itemize}

\end{solutionbox}
\begin{mnemonicbox}
``Don't cares help make bigger squares''

\end{mnemonicbox}
\subsection*{Question 2(c-OR) [7
marks]}\label{question-2c-or-7-marks}

\textbf{Write Basic Boolean Theorem and its all Properties.}

\begin{solutionbox}

\textbf{Basic Boolean Theorems and Properties:}

{\def\LTcaptype{none} % do not increment counter
\begin{longtable}[]{@{}
  >{\raggedright\arraybackslash}p{(\linewidth - 2\tabcolsep) * \real{0.5385}}
  >{\raggedright\arraybackslash}p{(\linewidth - 2\tabcolsep) * \real{0.4615}}@{}}
\toprule\noalign{}
\begin{minipage}[b]{\linewidth}\raggedright
Law/Property
\end{minipage} & \begin{minipage}[b]{\linewidth}\raggedright
Expression
\end{minipage} \\
\midrule\noalign{}
\endhead
\bottomrule\noalign{}
\endlastfoot
\textbf{Identity Law} & A + 0 = A, A · 1 = A \\
\textbf{Null Law} & A + 1 = 1, A · 0 = 0 \\
\textbf{Idempotent Law} & A + A = A, A · A = A \\
\textbf{Complementary Law} & A + A' = 1, A · A' = 0 \\
\textbf{Commutative Law} & A + B = B + A, A · B = B · A \\
\textbf{Associative Law} & A + (B + C) = (A + B) + C, A · (B · C) = (A ·
B) · C \\
\textbf{Distributive Law} & A · (B + C) = A · B + A · C, A + (B · C) =
(A + B) · (A + C) \\
\textbf{Absorption Law} & A + (A · B) = A, A · (A + B) = A \\
\textbf{DeMorgan's Theorem} & (A + B)' = A' · B', (A · B)' = A' + B' \\
\textbf{Double Complement} & (A')' = A \\
\textbf{Consensus Theorem} & (A · B) + (A' · C) + (B · C) = (A · B) +
(A' · C) \\
\end{longtable}
}

\begin{itemize}
\tightlist
\item
  \textbf{Basic operations}: AND (·), OR (+), NOT (')
\item
  \textbf{Key applications}: Circuit simplification and design
\item
  \textbf{Theorem importance}: Reduces gate count and complexity
\end{itemize}

\end{solutionbox}
\begin{mnemonicbox}
``COIN-CADDAM'' (Complementary, Distributive,
Associative, etc.)

\end{mnemonicbox}
\subsection*{Question 3(a) [3 marks]}\label{q3a}

\textbf{Draw the Logic circuit of Full substractor and explain its
working.}

\begin{solutionbox}

\textbf{Full Subtractor Circuit:}

\begin{verbatim}
          |{-{-}{-}{-}{-}{-}|}
A{-{-}{-}{-}{-}{-}{-}{-}|      |}
          | XOR  |{-{-}{-}{-}{-}{-}{-}{-}{-}|}
B{-{-}{-}{-}{-}{-}{-}{-}|      |         |     |{-}{-}{-}{-}{-}{-}|}
          |{-{-}{-}{-}{-}{-}|         |{-}{-}{-}{-}|      |}
                                 | XOR  |{-{-}{-} Difference}
         |{-{-}{-}{-}{-}{-}|          |{-}{-}{-}{-}|      |}
C\_in{-{-}{-}{-}|      |          |     |{-}{-}{-}{-}{-}{-}|}
         |      |{-{-}{-}{-}{-}{-}{-}{-}{-}{-}|}
A{-{-}{-}{-}{-}{-}{-}| NAND |}
         |      |
         |{-{-}{-}{-}{-}{-}|                |{-}{-}{-}{-}{-}{-}|}
                                 |      |{-{-}{-} Borrow}
          |{-{-}{-}{-}{-}{-}|         |{-}{-}{-}{-}| OR   |}
B{-{-}{-}{-}{-}{-}{-}{-}|      |         |     |      |}
          | NAND |{-{-}{-}{-}{-}{-}{-}{-}{-}|     |{-}{-}{-}{-}{-}{-}|}
C\_in{-{-}{-}{-}{-}|      |}
          |{-{-}{-}{-}{-}{-}|}
\end{verbatim}

\textbf{Truth Table:}

{\def\LTcaptype{none} % do not increment counter
\begin{longtable}[]{@{}lllll@{}}
\toprule\noalign{}
A & B & C\_in & Difference & Borrow \\
\midrule\noalign{}
\endhead
\bottomrule\noalign{}
\endlastfoot
0 & 0 & 0 & 0 & 0 \\
0 & 0 & 1 & 1 & 1 \\
0 & 1 & 0 & 1 & 1 \\
0 & 1 & 1 & 0 & 1 \\
1 & 0 & 0 & 1 & 0 \\
1 & 0 & 1 & 0 & 0 \\
1 & 1 & 0 & 0 & 0 \\
1 & 1 & 1 & 1 & 1 \\
\end{longtable}
}

\begin{itemize}
\tightlist
\item
  \textbf{Difference}: A \oplus B \oplus C\_in (XOR of all inputs)
\item
  \textbf{Borrow}: C\_in·(A \oplus B) + B·A' (generated when needed)
\end{itemize}

\end{solutionbox}
\begin{mnemonicbox}
``Borrow is needed when subtrahend exceeds minuend''

\end{mnemonicbox}
\subsection*{Question 3(b) [4 marks]}\label{q3b}

\textbf{Draw the circuit of Gray to binary code converter.}

\begin{solutionbox}

\textbf{Gray to Binary Code Converter (4-bit):}

\begin{verbatim}
       G3          G2          G1          G0
       |           |           |           |
       |           |           |           |
       v           v           v           v
       |{-{-}{-}{-}{-}{-}|    |{-}{-}{-}{-}{-}{-}|    |{-}{-}{-}{-}{-}{-}|    |}
       |      |    |      |    |      |    |
G3{-{-}{-}{-}| XNOR |{-}{-}{-}| XNOR |{-}{-}{-}| XNOR |{-}{-}{-}|{-}{-} B0}
       |      |    |      |    |      |    |
       |{-{-}{-}{-}{-}{-}|    |{-}{-}{-}{-}{-}{-}|    |{-}{-}{-}{-}{-}{-}|    |}
          \^{           \^{}           \^{}         |}
          |           |           |         |
       B3 |        B2 |        B1 |        G0=B0
\end{verbatim}

\textbf{Conversion Table:}

{\def\LTcaptype{none} % do not increment counter
\begin{longtable}[]{@{}ll@{}}
\toprule\noalign{}
Gray & Binary \\
\midrule\noalign{}
\endhead
\bottomrule\noalign{}
\endlastfoot
G3G2G1G0 & B3B2B1B0 \\
0000 & 0000 \\
0001 & 0001 \\
0011 & 0010 \\
0010 & 0011 \\
0110 & 0100 \\
\ldots{} & \ldots{} \\
\end{longtable}
}

\begin{itemize}
\tightlist
\item
  \textbf{Conversion principle}: B_{3} = G_{3}, B_{2} = B_{3} \oplus G_{2}, B_{1} = B_{2} \oplus G_{1}, B_{0}
  = B_{1} \oplus G_{0}
\item
  \textbf{Key feature}: Each binary bit depends on all previous gray
  bits
\item
  \textbf{Application}: Error detection in digital transmission
\end{itemize}

\end{solutionbox}
\begin{mnemonicbox}
``MSB stays, others XOR with previous binary''

\end{mnemonicbox}
\subsection*{Question 3(c) [7 marks]}\label{q3c}

\textbf{Draw and explain 2:4 Decoder and 4:1 Multiplexer and explain its
working.}

\begin{solutionbox}

\textbf{2:4 Decoder:}

\begin{verbatim}
       |{-{-}{-}{-}{-}{-}|}
       |      |{-{-}{-}{-}{-} Y0 (AB)}
       |      |
A{-{-}{-}{-}{-}| 2:4  |{-}{-}{-}{-}{-} Y1 (AB)}
       |      |
B{-{-}{-}{-}{-}| DEC  |{-}{-}{-}{-}{-} Y2 (AB)}
       |      |
       |      |{-{-}{-}{-}{-} Y3 (AB)}
       |{-{-}{-}{-}{-}{-}|}
\end{verbatim}

\textbf{Truth Table:}

{\def\LTcaptype{none} % do not increment counter
\begin{longtable}[]{@{}llllll@{}}
\toprule\noalign{}
A & B & Y0 & Y1 & Y2 & Y3 \\
\midrule\noalign{}
\endhead
\bottomrule\noalign{}
\endlastfoot
0 & 0 & 1 & 0 & 0 & 0 \\
0 & 1 & 0 & 1 & 0 & 0 \\
1 & 0 & 0 & 0 & 1 & 0 \\
1 & 1 & 0 & 0 & 0 & 1 \\
\end{longtable}
}

\textbf{4:1 Multiplexer:}

\begin{verbatim}
       |{-{-}{-}{-}{-}{-}|}
D0{-{-}{-}{-}|      |}
       |      |
D1{-{-}{-}{-}| 4:1  |}
       |      |{-{-}{-}{-}{-} Y}
D2{-{-}{-}{-}| MUX  |}
       |      |
D3{-{-}{-}{-}|      |}
       |{-{-}{-}{-}{-}{-}|}
         \^{  \^{}}
         |  |
        S0 S1
\end{verbatim}

\textbf{Truth Table:}

{\def\LTcaptype{none} % do not increment counter
\begin{longtable}[]{@{}lll@{}}
\toprule\noalign{}
S1 & S0 & Y \\
\midrule\noalign{}
\endhead
\bottomrule\noalign{}
\endlastfoot
0 & 0 & D0 \\
0 & 1 & D1 \\
1 & 0 & D2 \\
1 & 1 & D3 \\
\end{longtable}
}

\begin{itemize}
\tightlist
\item
  \textbf{Decoder}: Converts binary code to one-hot output
\item
  \textbf{Multiplexer}: Selects one of many inputs based on selection
  lines
\item
  \textbf{Applications}: Memory addressing, data routing
\end{itemize}

\end{solutionbox}
\begin{mnemonicbox}
``Decoder: One-to-Many, Mux: Many-to-One''

\end{mnemonicbox}
\subsection*{Question 3(a-OR) [3
marks]}\label{question-3a-or-3-marks}

\textbf{Draw the Logic circuit of full adder and explain its working.}

\begin{solutionbox}

\textbf{Full Adder Circuit:}

\begin{verbatim}
          |{-{-}{-}{-}{-}{-}|}
A{-{-}{-}{-}{-}{-}{-}{-}|      |}
          | XOR  |{-{-}{-}{-}{-}{-}{-}{-}{-}|}
B{-{-}{-}{-}{-}{-}{-}{-}|      |         |     |{-}{-}{-}{-}{-}{-}|}
          |{-{-}{-}{-}{-}{-}|         |{-}{-}{-}{-}|      |}
                                 | XOR  |{-{-}{-} Sum}
          |{-{-}{-}{-}{-}{-}|         |{-}{-}{-}{-}|      |}
C\_in{-{-}{-}{-}{-}|      |         |     |{-}{-}{-}{-}{-}{-}|}
          |      |{-{-}{-}{-}{-}{-}{-}{-}{-}|}
          |      |
          |{-{-}{-}{-}{-}{-}|}
                          |{-{-}{-}{-}{-}{-}|}
          |{-{-}{-}{-}{-}{-}|        |      |}
A{-{-}{-}{-}{-}{-}{-}{-}|      |{-}{-}{-}{-}{-}{-}{-}|      |}
          | AND  |        |      |
B{-{-}{-}{-}{-}{-}{-}{-}|      |        |  OR  |{-}{-}{-} Carry}
          |{-{-}{-}{-}{-}{-}|        |      |}
                          |      |
          |{-{-}{-}{-}{-}{-}|        |      |}
C\_in{-{-}{-}{-}{-}|      |{-}{-}{-}{-}{-}{-}{-}|      |}
          | AND  |        |{-{-}{-}{-}{-}{-}|}
XOR{-{-}{-}{-}{-}{-}|      |}
          |{-{-}{-}{-}{-}{-}|}
\end{verbatim}

\textbf{Truth Table:}

{\def\LTcaptype{none} % do not increment counter
\begin{longtable}[]{@{}lllll@{}}
\toprule\noalign{}
A & B & C\_in & Sum & Carry \\
\midrule\noalign{}
\endhead
\bottomrule\noalign{}
\endlastfoot
0 & 0 & 0 & 0 & 0 \\
0 & 0 & 1 & 1 & 0 \\
0 & 1 & 0 & 1 & 0 \\
0 & 1 & 1 & 0 & 1 \\
1 & 0 & 0 & 1 & 0 \\
1 & 0 & 1 & 0 & 1 \\
1 & 1 & 0 & 0 & 1 \\
1 & 1 & 1 & 1 & 1 \\
\end{longtable}
}

\begin{itemize}
\tightlist
\item
  \textbf{Sum}: A \oplus B \oplus C\_in (XOR of all inputs)
\item
  \textbf{Carry}: (A · B) + (C\_in · (A \oplus B)) (generated when needed)
\end{itemize}

\end{solutionbox}
\begin{mnemonicbox}
``Sum is odd, Carry needs at least two 1's''

\end{mnemonicbox}
\subsection*{Question 3(b-OR) [4
marks]}\label{question-3b-or-4-marks}

\textbf{Draw the circuit of Binary to gray code converter.}

\begin{solutionbox}

\textbf{Binary to Gray Code Converter (4-bit):}

\begin{verbatim}
      B3          B2          B1          B0
       |           |           |           |
       |           |           |           |
       v           v           v           v
       |{-{-}{-}{-}{-}{-}|    |{-}{-}{-}{-}{-}{-}|    |{-}{-}{-}{-}{-}{-}|    |}
       |      |    |      |    |      |    |
B3{-{-}{-}{-}|      |{-}{-}{-}|      |{-}{-}{-}|      |{-}{-}{-}|{-}{-} G0}
       | XOR  |    | XOR  |    | XOR  |    |
       |{-{-}{-}{-}{-}{-}|    |{-}{-}{-}{-}{-}{-}|    |{-}{-}{-}{-}{-}{-}|    |}
         \^{           \^{}           \^{}         |}
         |           |           |         |
        B2          B1          B0         |
                                           |
         |                                 |
         v                                 v
         G3                                G0
\end{verbatim}

\textbf{Conversion Table:}

{\def\LTcaptype{none} % do not increment counter
\begin{longtable}[]{@{}ll@{}}
\toprule\noalign{}
Binary & Gray \\
\midrule\noalign{}
\endhead
\bottomrule\noalign{}
\endlastfoot
B3B2B1B0 & G3G2G1G0 \\
0000 & 0000 \\
0001 & 0001 \\
0010 & 0011 \\
0011 & 0010 \\
0100 & 0110 \\
\ldots{} & \ldots{} \\
\end{longtable}
}

\begin{itemize}
\tightlist
\item
  \textbf{Conversion principle}: G_{3} = B_{3}, G_{2} = B_{3} \oplus B_{2}, G_{1} = B_{2} \oplus B_{1}, G_{0}
  = B_{1} \oplus B_{0}
\item
  \textbf{Key feature}: Only one bit changes between adjacent codes
\item
  \textbf{Application}: Rotary encoders, error detection
\end{itemize}

\end{solutionbox}
\begin{mnemonicbox}
``MSB stays, others XOR adjacent binary bits''

\end{mnemonicbox}
\subsection*{Question 3(c-OR) [7
marks]}\label{question-3c-or-7-marks}

\textbf{Draw the logic diagram of 4 bit parallel adder using full adder
and explain its working.}

\begin{solutionbox}

\textbf{4-bit Parallel Adder using Full Adders:}

\begin{verbatim}
A3    B3      A2    B2      A1    B1      A0    B0
 |     |       |     |       |     |       |     |
 v     v       v     v       v     v       v     v
|---------|  |---------|  |---------|  |---------|
|         |  |         |  |         |  |         |
|   FA    |  |   FA    |  |   FA    |  |   FA    |
|         |  |         |  |         |  |         |
|---------|  |---------|  |---------|  |---------|
     |            |            |            |
     v            v            v            v
    S3           S2           S1           S0
     
     ^            ^            ^            ^
     |            |            |            |
C_out         C3           C2           C1      C_in=0
\end{verbatim}

\textbf{Operation:}

\begin{enumerate}
\tightlist
\item
  Each Full Adder (FA) adds corresponding bits (Ai, Bi) plus carry from
  previous stage
\item
  Produces Sum (Si) and Carry (Ci+1) for next stage
\item
  C\_in of first FA is 0 (or can be 1 for adding 1)
\item
  C\_out of last FA indicates overflow
\end{enumerate}

\textbf{Example Addition: 1101 + 1011}

\begin{itemize}
\item
  A_{3}A_{2}A_{1}A_{0} = 1101
\item
  B_{3}B_{2}B_{1}B_{0} = 1011
\item
  C\_in = 0
\item
  S_{3}S_{2}S_{1}S_{0} = 1000
\item
  C\_out = 1 (indicating overflow, actual result is 11000)
\item
  \textbf{Parallel adder}: Adds multiple bits simultaneously
\item
  \textbf{Carry propagation}: Key limiting factor for speed
\item
  \textbf{Adder applications}: ALU, address calculation
\end{itemize}

\end{solutionbox}
\begin{mnemonicbox}
``Carries cascade from right to left''

\end{mnemonicbox}
\subsection*{Question 4(a) [3 marks]}\label{q4a}

\textbf{Draw the Diagram of BCD counter}

\begin{solutionbox}

\textbf{BCD Counter Diagram:}

\begin{verbatim}
       |{-{-}{-}{-}{-}{-}|    |{-}{-}{-}{-}{-}{-}|    |{-}{-}{-}{-}{-}{-}|    |{-}{-}{-}{-}{-}{-}|}
       |      |    |      |    |      |    |      |
CLK{-{-}{-}| JK{-}FF|{-}{-}{-}| JK{-}FF|{-}{-}{-}| JK{-}FF|{-}{-}{-}| JK{-}FF|}
       |  Q0  |    |  Q1  |    |  Q2  |    |  Q3  |
       |{-{-}{-}{-}{-}{-}|    |{-}{-}{-}{-}{-}{-}|    |{-}{-}{-}{-}{-}{-}|    |{-}{-}{-}{-}{-}{-}|}
          |           |           |           |
          v           v           v           v
         Q0          Q1          Q2          Q3
          |           |           |           |
          |{-{-}{-}{-}{-}{-}{-}{-}{-}{-}{-}|{-}{-}{-}{-}{-}{-}{-}{-}{-}{-}{-}|{-}{-}{-}{-}{-}{-}{-}{-}{-}{-}{-}|}
                            |
                            v
                        |{-{-}{-}{-}{-}{-}|}
                        | NAND |{-{-}{-}{-}+}
                        |{-{-}{-}{-}{-}{-}|    |}
                                    |
                                    v
                                  RESET
\end{verbatim}

\textbf{Counter Sequence:}

{\def\LTcaptype{none} % do not increment counter
\begin{longtable}[]{@{}lllll@{}}
\toprule\noalign{}
Count & Q3 & Q2 & Q1 & Q0 \\
\midrule\noalign{}
\endhead
\bottomrule\noalign{}
\endlastfoot
0 & 0 & 0 & 0 & 0 \\
1 & 0 & 0 & 0 & 1 \\
2 & 0 & 0 & 1 & 0 \\
3 & 0 & 0 & 1 & 1 \\
4 & 0 & 1 & 0 & 0 \\
5 & 0 & 1 & 0 & 1 \\
6 & 0 & 1 & 1 & 0 \\
7 & 0 & 1 & 1 & 1 \\
8 & 1 & 0 & 0 & 0 \\
9 & 1 & 0 & 0 & 1 \\
0 & 0 & 0 & 0 & 0 \\
\end{longtable}
}

\begin{itemize}
\tightlist
\item
  \textbf{BCD counter}: Counts from 0 to 9, then resets
\item
  \textbf{Reset mechanism}: Detects count of 10 (1010) and resets to 0
\item
  \textbf{Applications}: Digital clocks, frequency counters
\end{itemize}

\end{solutionbox}
\begin{mnemonicbox}
``Counts Decimal Digits Only (0-9)''

\end{mnemonicbox}
\subsection*{Question 4(b) [4 marks]}\label{q4b}

\textbf{Draw T flip flop diagram and explain its working with truth
table}

\begin{solutionbox}

\textbf{T Flip-Flop Diagram:}

\begin{verbatim}
       |{-{-}{-}{-}{-}{-}|}
       |      |
T{-{-}{-}{-}{-}|      |}
       |  T   |
CLK{-{-}{-}|  FF  |{-}{-}{-}{-}{-} Q}
       |      |
       |      |{-{-}{-}{-}{-} Q}
       |{-{-}{-}{-}{-}{-}|}
\end{verbatim}

\textbf{Implementation using JK Flip-Flop:}

\begin{verbatim}
       |{-{-}{-}{-}{-}{-}|}
       |      |
T{-{-}{-}{-}{-}| J    |}
       |      |
       | JK   |{-{-}{-}{-}{-} Q}
CLK{-{-}{-}|      |}
       | FF   |{-{-}{-}{-}{-} Q}
       |      |
T{-{-}{-}{-}{-}| K    |}
       |{-{-}{-}{-}{-}{-}|}
\end{verbatim}

\textbf{Truth Table:}

{\def\LTcaptype{none} % do not increment counter
\begin{longtable}[]{@{}lll@{}}
\toprule\noalign{}
T & CLK & Q(next) \\
\midrule\noalign{}
\endhead
\bottomrule\noalign{}
\endlastfoot
0 & ↑ & Q \\
1 & ↑ & Q' \\
\end{longtable}
}

\begin{itemize}
\tightlist
\item
  \textbf{T=0}: No change in output (Hold)
\item
  \textbf{T=1}: Output toggles (Complement)
\item
  \textbf{Toggle operation}: Changes state on each clock pulse when T=1
\end{itemize}

\end{solutionbox}
\begin{mnemonicbox}
``T for Toggle, 0 holds 1 flips''

\end{mnemonicbox}
\subsection*{Question 4(c) [7 marks]}\label{q4c}

\textbf{What is shift register? Lists different types of shift register.
Explain working of any one type shift register with its logic circuit.}

\begin{solutionbox}

\textbf{Shift Register Definition:} A shift register is a sequential
logic circuit that stores and shifts binary data. It consists of a
series of flip-flops where the output of one flip-flop becomes input to
the next.

\textbf{Types of Shift Registers:}

{\def\LTcaptype{none} % do not increment counter
\begin{longtable}[]{@{}ll@{}}
\toprule\noalign{}
Type & Description \\
\midrule\noalign{}
\endhead
\bottomrule\noalign{}
\endlastfoot
SISO & Serial Input Serial Output \\
SIPO & Serial Input Parallel Output \\
PISO & Parallel Input Serial Output \\
PIPO & Parallel Input Parallel Output \\
Bidirectional & Can shift in either direction \\
Ring Counter & Output of last stage fed to first stage \\
Johnson Counter & Complement of last stage fed to first stage \\
\end{longtable}
}

\textbf{Serial-In Serial-Out (SISO) Shift Register:}

\begin{verbatim}
      |{-{-}{-}{-}{-}{-}|    |{-}{-}{-}{-}{-}{-}|    |{-}{-}{-}{-}{-}{-}|    |{-}{-}{-}{-}{-}{-}|}
      |      |    |      |    |      |    |      |
IN{-{-}{-}| D FF |{-}{-}{-}| D FF |{-}{-}{-}| D FF |{-}{-}{-}| D FF |{-}{-}{-} OUT}
      |      |    |      |    |      |    |      |
      |{-{-}{-}{-}{-}{-}|    |{-}{-}{-}{-}{-}{-}|    |{-}{-}{-}{-}{-}{-}|    |{-}{-}{-}{-}{-}{-}|}
         \^{          \^{}          \^{}          \^{}}
         |          |          |          |
         |{-{-}{-}{-}{-}{-}{-}{-}{-}{-}|{-}{-}{-}{-}{-}{-}{-}{-}{-}{-}|{-}{-}{-}{-}{-}{-}{-}{-}{-}{-}|}
                       CLK
\end{verbatim}

\textbf{Operation:}

\begin{enumerate}
\tightlist
\item
  Data enters serially bit by bit through the input
\item
  With each clock pulse, data shifts one position to the right
\item
  After 4 clock pulses, the first input bit appears at the output
\item
  Example: For input ``1101'', need 4 clock pulses for complete
  transmission
\end{enumerate}

\begin{itemize}
\tightlist
\item
  \textbf{Primary use}: Data conversion between serial and parallel
  formats
\item
  \textbf{Applications}: Communication systems, data transfer between
  devices
\item
  \textbf{Advantages}: Simple design, minimal interconnections
\end{itemize}

\end{solutionbox}
\begin{mnemonicbox}
``Shift registers pass bits like a bucket brigade''

\end{mnemonicbox}
\subsection*{Question 4(a-OR) [3
marks]}\label{question-4a-or-3-marks}

\textbf{Draw and Explain 4:2 Encoder.}

\begin{solutionbox}

\textbf{4:2 Encoder Diagram:}

\begin{verbatim}
      |{-{-}{-}{-}{-}{-}|}
D0{-{-}{-}|      |}
      |      |{-{-}{-} A}
D1{-{-}{-}|      |}
      | 4:2  |
D2{-{-}{-}|      |}
      |      |{-{-}{-} B}
D3{-{-}{-}|      |}
      |{-{-}{-}{-}{-}{-}|}
\end{verbatim}

\textbf{Truth Table:}

{\def\LTcaptype{none} % do not increment counter
\begin{longtable}[]{@{}llllll@{}}
\toprule\noalign{}
D3 & D2 & D1 & D0 & B & A \\
\midrule\noalign{}
\endhead
\bottomrule\noalign{}
\endlastfoot
0 & 0 & 0 & 1 & 0 & 0 \\
0 & 0 & 1 & 0 & 0 & 1 \\
0 & 1 & 0 & 0 & 1 & 0 \\
1 & 0 & 0 & 0 & 1 & 1 \\
\end{longtable}
}

\textbf{Logical Expressions:}

\begin{itemize}
\item
  A = D1 + D3
\item
  B = D2 + D3
\item
  \textbf{Encoder function}: Converts one-hot input to binary code
\item
  \textbf{Priority encoders}: Handle multiple active inputs by priority
\item
  \textbf{Applications}: Keyboard scanning, interrupt handling
\end{itemize}

\end{solutionbox}
\begin{mnemonicbox}
``One active line IN, binary code OUT''

\end{mnemonicbox}
\subsection*{Question 4(b-OR) [4
marks]}\label{question-4b-or-4-marks}

\textbf{Draw and explain Johnson counter.}

\begin{solutionbox}

\textbf{Johnson Counter (4-bit):}

\begin{verbatim}
    |{-{-}{-}{-}{-}{-}|    |{-}{-}{-}{-}{-}{-}|    |{-}{-}{-}{-}{-}{-}|    |{-}{-}{-}{-}{-}{-}|}
    |      |    |      |    |      |    |      |
    | D FF |{-{-}{-}{-}| D FF |{-}{-}{-}{-}| D FF |{-}{-}{-}{-}| D FF |}
    |      |    |      |    |      |    |      |
    |{-{-}{-}{-}{-}{-}|    |{-}{-}{-}{-}{-}{-}|    |{-}{-}{-}{-}{-}{-}|    |{-}{-}{-}{-}{-}{-}|}
       \^{          \^{}          \^{}          \^{}}
       |          |          |          |
       |{-{-}{-}{-}{-}{-}{-}{-}{-}{-}|{-}{-}{-}{-}{-}{-}{-}{-}{-}{-}|{-}{-}{-}{-}{-}{-}{-}{-}{-}{-}|}
                    CLK
               |
        Q3{    |}
         |     |
         |     v
         {-{-}{-}{-}{-}|}
\end{verbatim}

\textbf{Counter Sequence:}

{\def\LTcaptype{none} % do not increment counter
\begin{longtable}[]{@{}lllll@{}}
\toprule\noalign{}
Count & Q3 & Q2 & Q1 & Q0 \\
\midrule\noalign{}
\endhead
\bottomrule\noalign{}
\endlastfoot
0 & 0 & 0 & 0 & 0 \\
1 & 1 & 0 & 0 & 0 \\
2 & 1 & 1 & 0 & 0 \\
3 & 1 & 1 & 1 & 0 \\
4 & 1 & 1 & 1 & 1 \\
5 & 0 & 1 & 1 & 1 \\
6 & 0 & 0 & 1 & 1 \\
7 & 0 & 0 & 0 & 1 \\
0 & 0 & 0 & 0 & 0 \\
\end{longtable}
}

\begin{itemize}
\tightlist
\item
  \textbf{Johnson counter}: Also called twisted ring counter
\item
  \textbf{Sequence length}: 2n states where n is number of flip-flops
\item
  \textbf{Key feature}: Only one bit changes between adjacent states
\end{itemize}

\end{solutionbox}
\begin{mnemonicbox}
``Fill with 1's then clear with 0's''

\end{mnemonicbox}
\subsection*{Question 4(c-OR) [7
marks]}\label{question-4c-or-7-marks}

\textbf{Draw and explain 4 bit Ripple counter.}

\begin{solutionbox}

\textbf{4-bit Ripple Counter:}

\begin{verbatim}
         CLK
          |
          v
       |{-{-}{-}{-}{-}{-}|    |{-}{-}{-}{-}{-}{-}|    |{-}{-}{-}{-}{-}{-}|    |{-}{-}{-}{-}{-}{-}|}
       |      |    |      |    |      |    |      |
{-{-}{-}{-}{-}{-}| T FF |    | T FF |    | T FF |    | T FF |}
       |      |    |      |    |      |    |      |
       |{-{-}{-}{-}{-}{-}|    |{-}{-}{-}{-}{-}{-}|    |{-}{-}{-}{-}{-}{-}|    |{-}{-}{-}{-}{-}{-}|}
          |           |           |           |
          |           |           |           |
         Q0 {-{-}{-}{-}{-}{-}{-}{-}{-}Q1 {-}{-}{-}{-}{-}{-}{-}{-}Q2 {-}{-}{-}{-}{-}{-}{-}{-}Q3}
       (LSB)                                 (MSB)
\end{verbatim}

\textbf{Truth Table (Counting Sequence):}

{\def\LTcaptype{none} % do not increment counter
\begin{longtable}[]{@{}lllll@{}}
\toprule\noalign{}
Count & Q3 & Q2 & Q1 & Q0 \\
\midrule\noalign{}
\endhead
\bottomrule\noalign{}
\endlastfoot
0 & 0 & 0 & 0 & 0 \\
1 & 0 & 0 & 0 & 1 \\
2 & 0 & 0 & 1 & 0 \\
3 & 0 & 0 & 1 & 1 \\
\ldots{} & .. & .. & .. & .. \\
14 & 1 & 1 & 1 & 0 \\
15 & 1 & 1 & 1 & 1 \\
0 & 0 & 0 & 0 & 0 \\
\end{longtable}
}

\textbf{Working Principle:}

\begin{enumerate}
\tightlist
\item
  All T inputs are connected to logic 1 (toggle mode)
\item
  First flip-flop toggles on every clock pulse
\item
  Each subsequent flip-flop toggles when the previous one changes from 1
  to 0
\item
  Propagation delay increases with each stage
\end{enumerate}

\begin{itemize}
\tightlist
\item
  \textbf{Asynchronous counter}: Clock drives only first flip-flop
\item
  \textbf{Ripple effect}: Changes propagate through stages
\item
  \textbf{Disadvantage}: Slower due to cumulative propagation delays
\end{itemize}

\end{solutionbox}
\begin{mnemonicbox}
``Change ripples through like falling dominoes''

\end{mnemonicbox}
\subsection*{Question 5(a) [3 marks]}\label{q5a}

\textbf{Explain DRAM in short.}

\begin{solutionbox}

\textbf{Dynamic Random Access Memory (DRAM):}

DRAM is a type of semiconductor memory that stores each bit in a
separate capacitor.

\textbf{Key Features:}

{\def\LTcaptype{none} % do not increment counter
\begin{longtable}[]{@{}ll@{}}
\toprule\noalign{}
Feature & Description \\
\midrule\noalign{}
\endhead
\bottomrule\noalign{}
\endlastfoot
Storage element & Single capacitor + transistor per bit \\
Density & Very high (more bits per chip) \\
Speed & Moderate (slower than SRAM) \\
Refresh & Required periodically (typically every few ms) \\
Power consumption & Lower than SRAM \\
Cost & Less expensive than SRAM \\
\end{longtable}
}

\begin{itemize}
\tightlist
\item
  \textbf{Dynamic nature}: Charge leaks over time, requiring refresh
\item
  \textbf{Applications}: Main memory in computers
\item
  \textbf{Advantage}: High density, low cost per bit
\end{itemize}

\end{solutionbox}
\begin{mnemonicbox}
``DRAM needs refreshing like a tired mind''

\end{mnemonicbox}
\subsection*{Question 5(b) [4 marks]}\label{q5b}

\textbf{Define the following (1) Fan in (2) Propagation Delay}

\begin{solutionbox}

\textbf{Fan-in:}

Fan-in refers to the maximum number of inputs that a logic gate can
accept.

\textbf{Characteristics of Fan-in:}

\begin{itemize}
\tightlist
\item
  Measures input load capability
\item
  Affects circuit complexity and design
\item
  Higher fan-in reduces gate count but increases complexity
\item
  Different logic families have different fan-in limits
\end{itemize}

\textbf{Example:} A standard TTL NAND gate typically has a fan-in of 8
inputs.

\textbf{Propagation Delay:}

Propagation delay is the time taken for a signal to travel from input to
output of a logic gate.

\textbf{Characteristics of Propagation Delay:}

\begin{itemize}
\tightlist
\item
  Measured in nanoseconds (ns)
\item
  Critical for high-speed circuit performance
\item
  Varies with temperature, loading, and supply voltage
\item
  Different for rising and falling transitions
\end{itemize}

\textbf{Example:} A typical TTL gate has a propagation delay of 10-20
ns.

\begin{itemize}
\tightlist
\item
  \textbf{Impact on circuits}: Limits maximum operating frequency
\item
  \textbf{Calculation}: Time between 50\% points of input and output
  signals
\end{itemize}

\end{solutionbox}
\begin{mnemonicbox}
``Fan-in counts inputs, Prop-delay counts time''

\end{mnemonicbox}
\subsection*{Question 5(c) [7 marks]}\label{q5c}

\textbf{Do as Directed (i) Compare Logic families TTL and CMOS (ii) Draw
Circuit Diagram of SR flip flop.}

\begin{solutionbox}

\textbf{(i) Comparison of TTL and CMOS Logic Families:}

{\def\LTcaptype{none} % do not increment counter
\begin{longtable}[]{@{}lll@{}}
\toprule\noalign{}
Parameter & TTL & CMOS \\
\midrule\noalign{}
\endhead
\bottomrule\noalign{}
\endlastfoot
Technology & Bipolar transistors & MOSFETs \\
Supply voltage & 5V (fixed) & 3-15V (flexible) \\
Power consumption & Higher & Very low (static) \\
Speed & Medium to high & Low to very high \\
Noise margin & Moderate & High \\
Fan-out & 10-20 & \textgreater50 \\
Propagation delay & 5-10 ns & 10-100 ns (standard) \\
Input impedance & 4-40 kΩ & Very high (10^{1}^{2} Ω) \\
Output impedance & 100-300 Ω & Variable \\
Susceptibility to static & Low & High \\
\end{longtable}
}

\textbf{(ii) SR Flip-Flop Circuit Diagram:}

\begin{verbatim}
             |{-{-}{-}{-}{-}{-}|}
             |      |
Set {-{-}{-}{-}{-}{-}{-}{-}|      |{-}{-}{-}{-}{-} Q}
             | NOR  |
             |      |
             |{-{-}{-}{-}{-}{-}|}
                \^{}
                |
                |    |{-{-}{-}{-}{-}{-}|}
                |    |      |
                |{-{-}{-}{-}|      |{-}{-}{-}{-}{-} Q}
                     | NOR  |
                     |      |
Reset {-{-}{-}{-}{-}{-}{-}{-}{-}{-}{-}{-}{-}{-}|      |}
                     |{-{-}{-}{-}{-}{-}|}
\end{verbatim}

\textbf{Truth Table:}

{\def\LTcaptype{none} % do not increment counter
\begin{longtable}[]{@{}lllll@{}}
\toprule\noalign{}
S & R & Q & Q' & Remarks \\
\midrule\noalign{}
\endhead
\bottomrule\noalign{}
\endlastfoot
0 & 0 & Q & Q' & Memory (no change) \\
0 & 1 & 0 & 1 & Reset \\
1 & 0 & 1 & 0 & Set \\
1 & 1 & 0 & 0 & Invalid (avoid) \\
\end{longtable}
}

\begin{itemize}
\tightlist
\item
  \textbf{SR flip-flop}: Basic memory element in digital circuits
\item
  \textbf{Operation}: Set (S=1, R=0) makes Q=1; Reset (S=0, R=1) makes
  Q=0
\item
  \textbf{Memory state}: When S=0, R=0, output remains unchanged
\end{itemize}

\end{solutionbox}
\begin{mnemonicbox}
``SR: Set-Reset, memory when both low''

\end{mnemonicbox}
\subsection*{Question 5(a-OR) [3
marks]}\label{question-5a-or-3-marks}

\textbf{Write short note on E Waste of Digital Chips.}

\begin{solutionbox}

\textbf{E-Waste of Digital Chips:}

E-waste from digital chips refers to discarded electronic devices
containing semiconductor components that require special handling and
disposal.

\textbf{Key Concerns:}

{\def\LTcaptype{none} % do not increment counter
\begin{longtable}[]{@{}
  >{\raggedright\arraybackslash}p{(\linewidth - 2\tabcolsep) * \real{0.4706}}
  >{\raggedright\arraybackslash}p{(\linewidth - 2\tabcolsep) * \real{0.5294}}@{}}
\toprule\noalign{}
\begin{minipage}[b]{\linewidth}\raggedright
Aspect
\end{minipage} & \begin{minipage}[b]{\linewidth}\raggedright
Details
\end{minipage} \\
\midrule\noalign{}
\endhead
\bottomrule\noalign{}
\endlastfoot
Hazardous materials & Lead, mercury, cadmium, brominated flame
retardants \\
Environmental impact & Soil and water contamination if improperly
disposed \\
Resource recovery & Contains valuable metals (gold, silver, copper) \\
Volume & Growing rapidly with technological advancement \\
Regulations & Governed by WEEE, RoHS directives in many countries \\
\end{longtable}
}

\textbf{Management Approaches:}

\begin{itemize}
\item
  Recycling through authorized e-waste handlers
\item
  Recovery of precious metals
\item
  Safe disposal of toxic components
\item
  Extended producer responsibility programs
\item
  \textbf{Challenges}: Informal recycling causing health hazards
\item
  \textbf{Solutions}: Design for disassembly, green manufacturing
\end{itemize}

\end{solutionbox}
\begin{mnemonicbox}
``Digital waste needs digital-age solutions''

\end{mnemonicbox}
\subsection*{Question 5(b-OR) [4
marks]}\label{question-5b-or-4-marks}

\textbf{Define the following (1) Fan out (2) Noise margin}

\begin{solutionbox}

\textbf{Fan-out:}

Fan-out is the maximum number of gate inputs that can be driven by a
single logic gate output while maintaining proper logic levels.

\textbf{Characteristics of Fan-out:}

\begin{itemize}
\tightlist
\item
  Measures output drive capability
\item
  Affects design flexibility and cost
\item
  Higher fan-out allows simpler wiring
\item
  Limited by current sourcing/sinking capacity
\end{itemize}

\textbf{Example:} A standard TTL gate has a fan-out of 10, meaning it
can drive 10 similar gates.

\textbf{Noise Margin:}

Noise margin is the amount of noise voltage that can be added to an
input signal without causing an undesired change in the circuit output.

\textbf{Characteristics of Noise Margin:}

\begin{itemize}
\tightlist
\item
  Expressed in volts
\item
  Measures circuit immunity to electrical noise
\item
  Higher noise margin means more reliable operation
\item
  Different for high and low logic levels
\end{itemize}

\textbf{Example:} TTL has noise margins of approximately 0.4V for logic
low and 0.7V for logic high.

\begin{itemize}
\tightlist
\item
  \textbf{Calculation}: Difference between guaranteed output and
  required input levels
\item
  \textbf{Importance}: Critical in electrically noisy environments
\end{itemize}

\end{solutionbox}
\begin{mnemonicbox}
``Fan-out counts outputs, Noise margin fights
interference''

\end{mnemonicbox}
\subsection*{Question 5(c-OR) [7
marks]}\label{question-5c-or-7-marks}

\textbf{Do as Directed (i) Write short note on ROM (ii) Explain JK
master slave flipflop.}

\begin{solutionbox}

\textbf{(i) Short Note on ROM:}

ROM (Read-Only Memory) is a non-volatile memory used to store permanent
or semi-permanent data.

\textbf{Types of ROM:}

{\def\LTcaptype{none} % do not increment counter
\begin{longtable}[]{@{}lll@{}}
\toprule\noalign{}
Type & Characteristics & Programming \\
\midrule\noalign{}
\endhead
\bottomrule\noalign{}
\endlastfoot
Mask ROM & Factory programmed & During manufacturing \\
PROM & One-time programmable & Electrical fusing by user \\
EPROM & Erasable with UV light & Electrical programming \\
EEPROM & Electrically erasable & Electrical programming/erasing \\
Flash ROM & Fast electrical erase & Block-wise erase/write \\
\end{longtable}
}

\textbf{Applications:}

\begin{itemize}
\item
  Firmware and BIOS storage
\item
  Look-up tables for fixed functions
\item
  Microcode in processors
\item
  Boot code in computers
\item
  \textbf{Data retention}: Maintains data without power
\item
  \textbf{Access time}: Typically 45-150 ns
\item
  \textbf{Density}: High storage capacity
\end{itemize}

\textbf{(ii) JK Master-Slave Flip-Flop:}

\begin{verbatim}
         |{-{-}{-}{-}{-}{-}{-}{-}{-}{-}{-}|        |{-}{-}{-}{-}{-}{-}{-}{-}{-}{-}{-}|}
         |           |        |           |
J {-{-}{-}{-}{-}{-}|           |        |           |{-}{-}{-}{-}{-} Q}
         |  Master   |{-{-}{-}{-}{-}{-}{-}|   Slave   |}
K {-{-}{-}{-}{-}{-}|           |        |           |{-}{-}{-}{-}{-} Q}
         |           |        |           |
         |{-{-}{-}{-}{-}{-}{-}{-}{-}{-}{-}|        |{-}{-}{-}{-}{-}{-}{-}{-}{-}{-}{-}|}
               \^{                    \^{}}
               |                    |
         CLK {-{-}|                 INV|{-}{-} CLK}
\end{verbatim}

\textbf{Truth Table:}

{\def\LTcaptype{none} % do not increment counter
\begin{longtable}[]{@{}llll@{}}
\toprule\noalign{}
J & K & Q(next) & Function \\
\midrule\noalign{}
\endhead
\bottomrule\noalign{}
\endlastfoot
0 & 0 & Q & No change \\
0 & 1 & 0 & Reset \\
1 & 0 & 1 & Set \\
1 & 1 & Q' & Toggle \\
\end{longtable}
}

\textbf{Operation:}

\begin{enumerate}
\tightlist
\item
  \textbf{Master stage}: When CLK=1, master latch samples J and K inputs
\item
  \textbf{Slave stage}: When CLK=0, slave latch samples master output
\item
  \textbf{Two-phase operation}: Prevents race condition (changes occur
  only on clock edge)
\item
  \textbf{Advantage}: More versatile than SR flip-flop (no invalid
  state)
\end{enumerate}

\begin{itemize}
\tightlist
\item
  \textbf{Toggle mode}: When J=K=1, output toggles on each clock cycle
\item
  \textbf{Applications}: Counters, shift registers, sequential circuits
\end{itemize}

\end{solutionbox}
\begin{mnemonicbox}
``J-K: Set-Reset-Toggle, Master leads Slave follows''

\end{mnemonicbox}

\end{document}
