\documentclass[10pt,a4paper]{article}

% content/resources/templates/preamble.tex
\usepackage[margin=0.6in]{geometry}
\author{Milav Dabgar}
\usepackage{amsmath,amssymb,amsthm}
\usepackage{booktabs}
\usepackage{multirow}
\usepackage{xcolor}
\usepackage{tcolorbox}
\tcbuselibrary{breakable,skins}
\usepackage[colorlinks=true,linkcolor=blue]{hyperref}
\usepackage{titlesec}
\usepackage{enumitem}
\usepackage{tikz}
\usepackage{pgfplots}
\usepackage{circuitikz}
\usepackage[version=4]{mhchem}
\usepackage{longtable}
\usepackage{array}
\usepackage{float}
\usepackage{caption}
\usepackage{listings}

\lstset{
  basicstyle=\small\ttfamily,
  breaklines=true,
  breakatwhitespace=false,
  postbreak=\mbox{\textcolor{red}{$\hookrightarrow$}\space},
  float=false,
  numbers=left,
  numberstyle=\tiny\color{gray},
  numbersep=10pt,
  xleftmargin=2em,
  keywordstyle=\color{blue},
  commentstyle=\color{green!60!black},
  stringstyle=\color{purple},
  backgroundcolor=\color{gray!5},
  showstringspaces=false,
  tabsize=2,
  captionpos=b,
  keepspaces=true,
  columns=flexible
}

\pgfplotsset{compat=1.18}
\usetikzlibrary{shapes,arrows,positioning,calc,patterns,decorations.pathmorphing,decorations.markings,arrows.meta}

% Color scheme
\definecolor{headcolor}{RGB}{0,102,204}
\definecolor{keycolor}{RGB}{220,20,60}
\definecolor{solutioncolor}{RGB}{34,139,34}
\definecolor{mnemoniccolor}{RGB}{148,0,211}
\definecolor{codecolor}{RGB}{0,0,100}

% Spacing
\setlength{\parskip}{3pt}
\setlist[itemize]{nosep}
\setlist[enumerate]{nosep}

% Title formatting
\titleformat{\section}{\Large\bfseries\color{headcolor}}{\thesection}{1em}{}
\titleformat{\subsection}{\large\bfseries\color{headcolor}}{\thesubsection}{1em}{}

% Pandoc tightlist compatibility
\providecommand{\tightlist}{%
  \setlength{\itemsep}{0pt}\setlength{\parskip}{0pt}}

% Pandoc longtable compatibility
\newcounter{none}
\def\thenone{}


% content/resources/templates/english-boxes.tex
% This file is currently empty - it exists to maintain consistency with the import structure.
% Add custom environments here if needed in the future.


\begin{document}

\begin{center}
{\Huge\bfseries\color{headcolor} Subject Name Solutions}\\[5pt]
{\LARGE 4331102 -- Summer 2025}\\[3pt]
{\large Semester 1 Study Material}\\[3pt]
{\normalsize\textit{Detailed Solutions and Explanations}}
\end{center}

\vspace{10pt}

\subsection*{Question 1(a) [3 marks]}\label{q1a}

\textbf{Define Accuracy, Precision, and Sensitivity.}

\begin{solutionbox}

\begin{itemize}
\tightlist
\item
  \textbf{Accuracy}: The closeness of a measured value to the actual or
  true value of a quantity.
\item
  \textbf{Precision}: The ability of an instrument to reproduce the same
  output reading when the same input is applied repeatedly under the
  same conditions.
\item
  \textbf{Sensitivity}: The ratio of change in output of an instrument
  to the change in input, indicating how much output changes for a small
  change in input.
\end{itemize}


{\def\LTcaptype{none} % do not increment counter
\vspace{-5pt}
\captionof{table}{Differences between Accuracy and Precision}
\vspace{-10pt}
\begin{longtable}[]{@{}lll@{}}
\toprule\noalign{}
Parameter & Accuracy & Precision \\
\midrule\noalign{}
\endhead
\bottomrule\noalign{}
\endlastfoot
Definition & Closeness to true value & Repeatability of measurement \\
Focus on & Correctness & Consistency \\
Representation & Bulls-eye center hits & Clustered hits \\
\end{longtable}
}

\end{solutionbox}
\begin{mnemonicbox}
``APS - Accuracy Pinpoints truth, Precision Shows
repeatability, Sensitivity Signals small changes''

\end{mnemonicbox}
\subsection*{Question 1(b) [4 marks]}\label{q1b}

\textbf{Describe the working and limitations of the Wheatstone bridge
with circuit diagram.}

\begin{solutionbox}

\textbf{Working}: The Wheatstone bridge measures unknown resistance by
balancing two legs of a bridge circuit.

\textbf{Circuit Diagram}:

\begin{center}
\textbf{Mermaid Diagram (Code)}
\begin{verbatim}
{Shaded}
{Highlighting}[]
graph LR
    A[Battery] {-{-}{} B[Point A]}
    A {-{-}{} C[Point C]}
    B {-{-}{} D[Point B]}
    B {-{-}{} E[Point D]}
    C {-{-}{} E}
    C {-{-}{} F[Point C]}
    D {-{-}{-} G[Galvanometer]}
    F {-{-}{-} G}
    B {-{-} R1 {-}{-}{-} D}
    D {-{-} R2 {-}{-}{-} C}
    B {-{-} R3 {-}{-}{-} F}
    F {-{-} Rx {-}{-}{-} C}
{Highlighting}
{Shaded}
\end{verbatim}
\end{center}

When bridge is balanced: R1/R2 = R3/Rx, so Rx = R3\times(R2/R1)

\textbf{Limitations}:

\begin{itemize}
\tightlist
\item
  \textbf{Limited range}: Not suitable for very low or very high
  resistances
\item
  \textbf{Temperature effects}: Resistance changes with temperature
\item
  \textbf{Battery errors}: Output voltage must remain stable
\item
  \textbf{Galvanometer sensitivity}: Limited by detector sensitivity
\end{itemize}

\end{solutionbox}
\begin{mnemonicbox}
``BALR - Balance is key, Adjust until null, Low/high
resistances problematic, Range is limited''

\end{mnemonicbox}
\subsection*{Question 1(c) [7 marks]}\label{q1c}

\textbf{Explain various transducers used for temperature measurement.
Explain the construction and working of the following in detail: (i)
Thermocouple (ii) Thermistor.}

\begin{solutionbox}

\textbf{Temperature Transducers Types}:

{\def\LTcaptype{none} % do not increment counter
\begin{longtable}[]{@{}
  >{\raggedright\arraybackslash}p{(\linewidth - 8\tabcolsep) * \real{0.1017}}
  >{\raggedright\arraybackslash}p{(\linewidth - 8\tabcolsep) * \real{0.3220}}
  >{\raggedright\arraybackslash}p{(\linewidth - 8\tabcolsep) * \real{0.1186}}
  >{\raggedright\arraybackslash}p{(\linewidth - 8\tabcolsep) * \real{0.2034}}
  >{\raggedright\arraybackslash}p{(\linewidth - 8\tabcolsep) * \real{0.2542}}@{}}
\toprule\noalign{}
\begin{minipage}[b]{\linewidth}\raggedright
Type
\end{minipage} & \begin{minipage}[b]{\linewidth}\raggedright
Working Principle
\end{minipage} & \begin{minipage}[b]{\linewidth}\raggedright
Range
\end{minipage} & \begin{minipage}[b]{\linewidth}\raggedright
Advantages
\end{minipage} & \begin{minipage}[b]{\linewidth}\raggedright
Disadvantages
\end{minipage} \\
\midrule\noalign{}
\endhead
\bottomrule\noalign{}
\endlastfoot
Thermocouple & Seebeck effect & -270^\circC to 2300^\circC & Wide range, robust &
Nonlinear, reference needed \\
Thermistor & Resistance change & -50^\circC to 300^\circC & High sensitivity &
Nonlinear, limited range \\
RTD & Resistance change & -200^\circC to 850^\circC & High accuracy, linear &
Expensive, self-heating \\
IC Sensors & Semiconductor & -55^\circC to 150^\circC & Linear output, easy
interface & Limited range \\
\end{longtable}
}

\textbf{(i) Thermocouple}:

\textbf{Construction}: Two dissimilar metal wires (like
copper-constantan or iron-constantan) joined at one end to form
measuring junction and other ends connected to measuring instrument.

\begin{center}
\textbf{Mermaid Diagram (Code)}
\begin{verbatim}
{Shaded}
{Highlighting}[]
graph LR
    A[Metal A] {-{-}{-} B[Measuring Junction]}
    C[Metal B] {-{-}{-} B}
    A {-{-}{-} D[Reference Junction]}
    C {-{-}{-} D}
    D {-{-}{-} E[Measuring Instrument]}
{Highlighting}
{Shaded}
\end{verbatim}
\end{center}

\textbf{Working}: When junctions are at different temperatures, a small
voltage proportional to temperature difference is generated (Seebeck
effect).

\textbf{Key Points}:

\begin{itemize}
\tightlist
\item
  \textbf{Seebeck effect}: Temperature difference creates voltage
\item
  \textbf{Cold junction compensation}: Required for accuracy
\item
  \textbf{Types}: J, K, T, E based on metal combinations
\end{itemize}

\textbf{(ii) Thermistor}:

\textbf{Construction}: A semiconductor material (metal oxides like
manganese, nickel, cobalt) shaped into a bead, disk, or rod with two
lead wires.

\begin{verbatim}
  Lead Wire        Lead Wire
      |               |
      v               v
    +{-{-}{-}{-}{-}{-}{-}{-}{-}{-}{-}{-}{-}{-}{-}{-}{-}+}
    |   Ceramic or    |
    | Semiconductor   |
    |     Body        |
    +{-{-}{-}{-}{-}{-}{-}{-}{-}{-}{-}{-}{-}{-}{-}{-}{-}+}
\end{verbatim}

\textbf{Working}: Resistance decreases as temperature increases (NTC
type) or increases with temperature (PTC type).

\textbf{Key Points}:

\begin{itemize}
\tightlist
\item
  \textbf{NTC (Negative Temperature Coefficient)}: Most common type
\item
  \textbf{High sensitivity}: Large resistance change for small
  temperature change
\item
  \textbf{Nonlinear response}: Requires linearization circuits
\item
  \textbf{Self-heating}: Current passing through it causes heating
\end{itemize}

\end{solutionbox}
\begin{mnemonicbox}
``TRIP - Thermocouples React to junction differences,
Thermistors Intensely change resistance, Point sensors at what you
measure''

\end{mnemonicbox}
\subsection*{Question 1(c) OR [7
marks]}\label{q1c}

\textbf{Explain the working principles of the following sensors:
Temperature sensor, Gas sensor, Humidity sensor and Proximity sensor.}

\begin{solutionbox}

\textbf{Comparison of Sensors}:

{\def\LTcaptype{none} % do not increment counter
\begin{longtable}[]{@{}
  >{\raggedright\arraybackslash}p{(\linewidth - 6\tabcolsep) * \real{0.2407}}
  >{\raggedright\arraybackslash}p{(\linewidth - 6\tabcolsep) * \real{0.3519}}
  >{\raggedright\arraybackslash}p{(\linewidth - 6\tabcolsep) * \real{0.1481}}
  >{\raggedright\arraybackslash}p{(\linewidth - 6\tabcolsep) * \real{0.2593}}@{}}
\toprule\noalign{}
\begin{minipage}[b]{\linewidth}\raggedright
Sensor Type
\end{minipage} & \begin{minipage}[b]{\linewidth}\raggedright
Working Principle
\end{minipage} & \begin{minipage}[b]{\linewidth}\raggedright
Output
\end{minipage} & \begin{minipage}[b]{\linewidth}\raggedright
Applications
\end{minipage} \\
\midrule\noalign{}
\endhead
\bottomrule\noalign{}
\endlastfoot
Temperature & Resistance/voltage change & Analog/Digital & HVAC, Medical
devices \\
Gas & Chemical reaction & Resistance change & Safety systems, Air
quality \\
Humidity & Capacitance/resistance change & Analog & Weather stations,
HVAC \\
Proximity & Electromagnetic field disruption & Digital & Automation,
Security \\
\end{longtable}
}

\textbf{1. Temperature Sensor (LM35)}:

\begin{itemize}
\tightlist
\item
  \textbf{Principle}: Semiconductor junction voltage varies with
  temperature
\item
  \textbf{Working}: Integrated circuit provides output voltage
  proportional to temperature (10mV/^\circC)
\item
  \textbf{Features}: Linear output, no external calibration needed
\end{itemize}

\textbf{2. Gas Sensor (MQ-2)}:

\begin{itemize}
\tightlist
\item
  \textbf{Principle}: Chemical reaction between gas and sensing material
\item
  \textbf{Working}: Gas molecules interact with metal oxide
  semiconductor, changing its resistance
\item
  \textbf{Detection}: When gas concentration exceeds threshold, output
  voltage changes
\end{itemize}

\begin{center}
\textbf{Mermaid Diagram (Code)}
\begin{verbatim}
{Shaded}
{Highlighting}[]
graph LR
    A[Gas molecules] {-{-}{} B[Sensing layer]}
    B {-{-}{} C[Resistance changes]}
    C {-{-}{} D[Voltage output changes]}
    D {-{-}{} E[Comparator circuit]}
    E {-{-}{} F[Alarm/Output signal]}
{Highlighting}
{Shaded}
\end{verbatim}
\end{center}

\textbf{3. Humidity Sensor (Hygrometer)}:

\begin{itemize}
\tightlist
\item
  \textbf{Principle}: Capacitance or resistance varies with moisture
  absorption
\item
  \textbf{Working}: Dielectric material absorbs moisture, changing
  electrical properties
\item
  \textbf{Types}: Capacitive (more accurate) and resistive (simpler)
\end{itemize}

\textbf{4. Proximity Sensor}:

\begin{itemize}
\tightlist
\item
  \textbf{Principle}: Detects objects without physical contact
\item
  \textbf{Working}: Emits electromagnetic field/beam; detects changes
  when object enters field
\item
  \textbf{Types}: Inductive (metals), capacitive (any material),
  ultrasonic (distance)
\end{itemize}

\end{solutionbox}
\begin{mnemonicbox}
``TGHP - Temperature Generates voltage, Gas Hits
semiconductors, Humidity Holds moisture, Proximity Perceives objects''

\end{mnemonicbox}
\subsection*{Question 2(a) [3 marks]}\label{q2a}

\textbf{List types of DVM and mention one advantage of each.}

\begin{solutionbox}

\textbf{Types of Digital Voltmeters (DVM)}:

{\def\LTcaptype{none} % do not increment counter
\begin{longtable}[]{@{}
  >{\raggedright\arraybackslash}p{(\linewidth - 4\tabcolsep) * \real{0.2500}}
  >{\raggedright\arraybackslash}p{(\linewidth - 4\tabcolsep) * \real{0.4750}}
  >{\raggedright\arraybackslash}p{(\linewidth - 4\tabcolsep) * \real{0.2750}}@{}}
\toprule\noalign{}
\begin{minipage}[b]{\linewidth}\raggedright
DVM Type
\end{minipage} & \begin{minipage}[b]{\linewidth}\raggedright
Working Principle
\end{minipage} & \begin{minipage}[b]{\linewidth}\raggedright
Advantage
\end{minipage} \\
\midrule\noalign{}
\endhead
\bottomrule\noalign{}
\endlastfoot
Ramp Type & Compares input with reference ramp & Simple design, low
cost \\
Integrating Type & Measures average over time & Good noise rejection \\
Successive Approximation & Binary search algorithm & Fast conversion
speed \\
Dual Slope & Integration with fixed time & Excellent noise rejection \\
\end{longtable}
}

\textbf{Key Points}:

\begin{itemize}
\tightlist
\item
  \textbf{Ramp type}: Simple but affected by noise
\item
  \textbf{Integrating type}: Reduces effect of periodic noise
\item
  \textbf{Successive approximation}: Quick readings, good for changing
  signals
\item
  \textbf{Dual slope}: Best accuracy, immune to most noise
\end{itemize}

\end{solutionbox}
\begin{mnemonicbox}
``RISD - Ramp Is Simple Design, Integrating Ignores
noise, Successive Secures speed, Dual Deals with interference''

\end{mnemonicbox}
\subsection*{Question 2(b) [4 marks]}\label{q2b}

\textbf{Draw and explain Maxwells's bridge.}

\begin{solutionbox}

\textbf{Maxwell's Bridge} is used to measure unknown inductance by
comparing it with a standard capacitance.

\textbf{Circuit Diagram}:

\begin{center}
\textbf{Mermaid Diagram (Code)}
\begin{verbatim}
{Shaded}
{Highlighting}[]
graph LR
    A[Supply] {-{-}{} B[Point B]}
    A {-{-}{} C[Point D]}
    B {-{-}{} D[Point A]}
    B {-{-}{} E[Point C]}
    C {-{-}{} E}
    C {-{-}{} F[Point D]}
    D {-{-}{-} G[Detector]}
    F {-{-}{-} G}
    B {-{-} R1 {-}{-}{-} D}
    D {-{-} R2 {-}{-}{-} C}
    B {-{-} R3 {-}{-}{-} F}
    F {-{-} L,R4 {-}{-}{-} C}
{Highlighting}
{Shaded}
\end{verbatim}
\end{center}

\textbf{Balance Equations}:

\begin{itemize}
\tightlist
\item
  Unknown inductance L = R2 \times R3 \times C
\item
  Resistance R4 = R1 \times (R3/R2)
\end{itemize}

\textbf{Working}:

\begin{itemize}
\tightlist
\item
  Bridge contains four arms with R1, R2, R3, and L,R4
\item
  When bridge is balanced, no current flows through detector
\item
  Values of L and R4 calculated using balance equations
\end{itemize}

\textbf{Advantages}:

\begin{itemize}
\tightlist
\item
  \textbf{High accuracy}: Good for medium value inductors
\item
  \textbf{Independent balance}: Resistance and inductance balanced
  separately
\end{itemize}

\end{solutionbox}
\begin{mnemonicbox}
``MILL - Maxwell's Inductance is Like L = R2R3C, when
the detector shows Lowered current''

\end{mnemonicbox}
\subsection*{Question 2(c) [7 marks]}\label{q2c}

\textbf{Draw the block diagram of a Successive Approximation type
Digital Voltmeter (DVM) and explain its working.}

\begin{solutionbox}

\textbf{Successive Approximation DVM} converts analog input to digital
output using binary search algorithm.

\textbf{Block Diagram}:

\begin{center}
\textbf{Mermaid Diagram (Code)}
\begin{verbatim}
{Shaded}
{Highlighting}[]
graph LR
    A[Analog Input] {-{-}{} B[Signal Conditioning]}
    B {-{-}{} C[Sample \& Hold]}
    C {-{-}{} D[Comparator]}
    E[Clock] {-{-}{} F[Successive Approximation Register]}
    F {-{-}{} G[D/A Converter]}
    G {-{-}{} D}
    D {-{-}{} F}
    F {-{-}{} H[Digital Display]}
    I[Reference Voltage] {-{-}{} G}
{Highlighting}
{Shaded}
\end{verbatim}
\end{center}

\textbf{Working}:

\begin{enumerate}
\tightlist
\item
  \textbf{Signal conditioning}: Scales input voltage to measurement
  range
\item
  \textbf{Sample \& Hold}: Captures instantaneous input value
\item
  \textbf{SAR (Successive Approximation Register)}: Performs binary
  search
\item
  \textbf{DAC (Digital-to-Analog Converter)}: Converts digital value to
  analog
\item
  \textbf{Comparator}: Compares input with DAC output
\item
  \textbf{Digital Display}: Shows final digital value
\end{enumerate}

\textbf{Example Conversion Process}:

\begin{itemize}
\tightlist
\item
  For 4-bit conversion of 9V (range 0-15V):

  \begin{itemize}
  \tightlist
  \item
    Try 8V (1000) \rightarrow Input \textgreater{} 8V \rightarrow Keep 1
  \item
    Try 12V (1100) \rightarrow Input \textless{} 12V \rightarrow Change to 0
  \item
    Try 10V (1010) \rightarrow Input \textless{} 10V \rightarrow Change to 0
  \item
    Try 9V (1001) \rightarrow Input = 9V \rightarrow Keep 1
  \item
    Result: 1001 (9V)
  \end{itemize}
\end{itemize}

\textbf{Advantages}:

\begin{itemize}
\tightlist
\item
  \textbf{Fast conversion}: Fixed conversion time regardless of input
\item
  \textbf{Good accuracy}: Suitable for most applications
\item
  \textbf{Medium complexity}: Balance of performance and cost
\end{itemize}

\end{solutionbox}
\begin{mnemonicbox}
``SHARP - Sample, Hold, Approximate, Register stores,
Present result''

\end{mnemonicbox}
\subsection*{Question 2(a) OR [3
marks]}\label{q2a}

\textbf{State and explain the working principle of PMMC instruments.}

\begin{solutionbox}

\textbf{PMMC (Permanent Magnet Moving Coil)} instruments operate based
on electromagnetic principles.

\textbf{Working Principle}: When current flows through a coil placed in
a magnetic field, a torque is produced causing the coil to rotate
proportionally to the current.

\textbf{Key Components}:

\begin{itemize}
\tightlist
\item
  \textbf{Permanent magnet}: Creates strong magnetic field
\item
  \textbf{Moving coil}: Wound on aluminum frame
\item
  \textbf{Control springs}: Provide restoring torque
\item
  \textbf{Pointer}: Indicates reading on scale
\end{itemize}

\textbf{Diagram}:

\begin{verbatim}
                  N
    Spring       | |      Spring
      ↓          | |        ↓
    +=================+
    |      |=====|    |
    |      | Coil|    |
    |      |=====|    |
    |               .{-|{-}.}
    |              /     {}
    |              |     |  Pointer
    |              {     /}
    |               {{-}|{-}}
    +=================+
                  | |
                  S
\end{verbatim}

\end{solutionbox}
\begin{mnemonicbox}
``PMMC - Permanent Magnet Makes Coil turn when
Current flows''

\end{mnemonicbox}
\subsection*{Question 2(b) OR [4
marks]}\label{q2b}

\textbf{Draw and explain Schering bridge.}

\begin{solutionbox}

\textbf{Schering Bridge} is used to measure capacitance and dissipation
factor of a capacitor.

\textbf{Circuit Diagram}:

\begin{center}
\textbf{Mermaid Diagram (Code)}
\begin{verbatim}
{Shaded}
{Highlighting}[]
graph LR
    A[AC Supply] {-{-}{} B[Point A]}
    A {-{-}{} C[Point C]}
    B {-{-}{} D[Point B]}
    B {-{-}{} E[Point D]}
    C {-{-}{} E}
    C {-{-}{} F[Point C]}
    D {-{-}{-} G[Detector]}
    F {-{-}{-} G}
    B {-{-} R1 {-}{-}{-} D}
    D {-{-} C2 {-}{-}{-} C}
    B {-{-} C4,R4 {-}{-}{-} F}
    F {-{-} Cx,Rx {-}{-}{-} C}
{Highlighting}
{Shaded}
\end{verbatim}
\end{center}

\textbf{Balance Equations}:

\begin{itemize}
\tightlist
\item
  Unknown capacitance Cx = C2 \times (R1/R4)
\item
  Unknown resistance Rx = R4 \times (C4/C2)
\item
Dissipation factor

D = ωCxRx = ωC4R4

\end{itemize}

\textbf{Working}:

\begin{itemize}
\tightlist
\item
  Contains four arms with R1, C2, Cx-Rx, and C4-R4
\item
  When bridge is balanced, no current flows through detector
\item
  Values of Cx and Rx calculated using balance equations
\end{itemize}

\textbf{Applications}:

\begin{itemize}
\tightlist
\item
  \textbf{Capacitor testing}: Measures capacitance and losses
\item
  \textbf{Insulation testing}: Evaluates dielectric properties
\end{itemize}

\end{solutionbox}
\begin{mnemonicbox}
``SCAN - Schering Capacitance And taN delta measured
together''

\end{mnemonicbox}
\subsection*{Question 2(c) OR [7
marks]}\label{q2c}

\textbf{Draw and explain Dual slope integrating type DVM.}

\begin{solutionbox}

\textbf{Dual Slope Integrating DVM} is a type of digital voltmeter that
converts analog input to digital form using integration method.

\textbf{Block Diagram}:

\begin{center}
\textbf{Mermaid Diagram (Code)}
\begin{verbatim}
{Shaded}
{Highlighting}[]
graph LR
    A[Analog Input] {-{-}{} B[Input Buffer]}
    B {-{-}{} C[Integrator]}
    D[Reference Voltage] {-{-}{} E[Polarity Switch]}
    E {-{-}{} C}
    C {-{-}{} F[Comparator]}
    G[Zero Reference] {-{-}{} F}
    F {-{-}{} H[Control Logic]}
    I[Clock] {-{-}{} H}
    H {-{-}{} E}
    H {-{-}{} J[Counter]}
    J {-{-}{} K[Digital Display]}
    H {-{-}{} J}
{Highlighting}
{Shaded}
\end{verbatim}
\end{center}

\textbf{Working Principle}:

\begin{enumerate}
\tightlist
\item
  \textbf{First phase} (Fixed time T1):

  \begin{itemize}
  \tightlist
  \item
    Input voltage integrated for fixed time T1
  \item
    Output of integrator = -(1/RC)\intV(in)dt
  \item
    Counter counts clock pulses
  \end{itemize}
\item
  \textbf{Second phase} (Variable time T2):

  \begin{itemize}
  \tightlist
  \item
    Reference voltage of opposite polarity applied
  \item
    Integrator output returns to zero
  \item
    Time T2 proportional to input voltage
  \item
    T2 = T1 \times (Vin/Vref)
  \end{itemize}
\end{enumerate}

\textbf{Advantages}:

\begin{itemize}
\tightlist
\item
  \textbf{Excellent noise rejection}: Especially power line frequency
  (50/60 Hz)
\item
  \textbf{High accuracy}: Depends only on reference voltage and clock
  stability
\item
  \textbf{Automatic zero correction}: Self-calibrating feature
\end{itemize}

\textbf{Key Points}:

\begin{itemize}
\tightlist
\item
  \textbf{Integration time}: Usually multiple of power line period (20ms
  or 16.67ms)
\item
  \textbf{Resolution}: Determined by clock frequency and counter
  capacity
\end{itemize}

\end{solutionbox}
\begin{mnemonicbox}
``FIRE - First Integrate input, then Integrate
Reference, until Equal to zero''

\end{mnemonicbox}
\subsection*{Question 3(a) [3 marks]}\label{q3a}

\textbf{What is the importance of delay line and trigger circuit in a
CRO?}

\begin{solutionbox}

\textbf{Delay Line Importance}:

\begin{itemize}
\tightlist
\item
  \textbf{Purpose}: Delays the signal to display events that trigger the
  sweep
\item
  \textbf{Function}: Allows viewing of leading edge of signal that
  caused trigger
\item
  \textbf{Implementation}: Artificial transmission line with LC network
  or microstrip
\end{itemize}

\textbf{Trigger Circuit Importance}:

\begin{itemize}
\tightlist
\item
  \textbf{Purpose}: Initiates sweep at specific point on input signal
\item
  \textbf{Function}: Ensures stable, stationary display of repetitive
  waveforms
\item
  \textbf{Controls}: Level, slope, source, and coupling
\end{itemize}


{\def\LTcaptype{none} % do not increment counter
\vspace{-5pt}
\captionof{table}{Delay Line vs Trigger Circuit}
\vspace{-10pt}
\begin{longtable}[]{@{}
  >{\raggedright\arraybackslash}p{(\linewidth - 4\tabcolsep) * \real{0.3793}}
  >{\raggedright\arraybackslash}p{(\linewidth - 4\tabcolsep) * \real{0.3103}}
  >{\raggedright\arraybackslash}p{(\linewidth - 4\tabcolsep) * \real{0.3103}}@{}}
\toprule\noalign{}
\begin{minipage}[b]{\linewidth}\raggedright
Component
\end{minipage} & \begin{minipage}[b]{\linewidth}\raggedright
Purpose
\end{minipage} & \begin{minipage}[b]{\linewidth}\raggedright
Benefit
\end{minipage} \\
\midrule\noalign{}
\endhead
\bottomrule\noalign{}
\endlastfoot
Delay Line & Delays signal path & Shows complete waveform including
trigger point \\
Trigger Circuit & Initiates sweep & Creates stable display with
synchronized timing \\
\end{longtable}
}

\end{solutionbox}
\begin{mnemonicbox}
``DT-SS - Delay To See Signal, Trigger Stops Screen
drift''

\end{mnemonicbox}
\subsection*{Question 3(b) [4 marks]}\label{q3b}

\textbf{Explain the internal structure and working of a Cathode Ray Tube
(CRT) with a neat diagram.}

\begin{solutionbox}

\textbf{Cathode Ray Tube (CRT)} is the heart of an oscilloscope that
converts electrical signals into visual display.

\textbf{Structure Diagram}:

\begin{verbatim}
       Electron Gun                Deflection System               Screen
      |{-{-}{-}{-}{-}{-}{-}{-}{-}{-}{-}{-}|            |{-}{-}{-}{-}{-}{-}{-}{-}{-}{-}{-}{-}{-}{-}{-}{-}|              |{-}{-}{-}{-}{-}{-}{-}|}
 +{-{-}{-}{-}|{-}{-}{-}{-}{-}{-}{-}{-}{-}{-}{-}{-}|{-}{-}{-}{-}{-}{-}{-}{-}{-}{-}{-}{-}|{-}{-}{-}{-}{-}{-}{-}{-}{-}{-}{-}{-}{-}{-}{-}{-}|{-}{-}{-}{-}{-}{-}{-}{-}{-}{-}{-}{-}{-}{-}|{-}{-}{-}{-}{-}{-}{-}|{-}{-}{-}{-}+}
 |    |            |            |                |              |       |    |
 |    V            V            V                V              V       |    |
 |  +{-{-}{-}{-}+      +{-}{-}{-}+        +{-}{-}{-}{-}+          +{-}{-}{-}{-}+          +{-}{-}{-}{-}+     |    |}
 |  |Cath|      |   |        |    |          |    |          |    |     |    |
 |  |ode |{-{-}{-}{-}{-}| F |{-}{-}{-}{-}{-}{-}{-}| VA |{-}{-}{-}{-}{-}{-}{-}{-}{-}| H  |{-}{-}{-}{-}{-}{-}{-}{-}{-}| P  |     |    |}
 |  |    |      |   |        |    |    |     |Def.|    |     |Scr.|     |    |
 |  +{-{-}{-}{-}+      +{-}{-}{-}+        +{-}{-}{-}{-}+    |     +{-}{-}{-}{-}+    |     +{-}{-}{-}{-}+     |    |}
 |                                     |               |                |    |
 |                                   +{-{-}{-}{-}+         +{-}{-}{-}{-}+              |    |}
 |                                   |    |         |    |              |    |
 |                                   | V  |         |Flu.|              |    |
 |                                   |Def.|{-{-}{-}{-}{-}{-}{-}{-}|Scr.|              |    |}
 |                                   |    |         |    |              |    |
 |                                   +{-{-}{-}{-}+         +{-}{-}{-}{-}+              |    |}
 |                                                                      |    |
 +{-{-}{-}{-}{-}{-}{-}{-}{-}{-}{-}{-}{-}{-}{-}{-}{-}{-}{-}{-}{-}{-}{-}{-}{-}{-}{-}{-}{-}{-}{-}{-}{-}{-}{-}{-}{-}{-}{-}{-}{-}{-}{-}{-}{-}{-}{-}{-}{-}{-}{-}{-}{-}{-}{-}{-}{-}{-}{-}{-}{-}{-}{-}{-}{-}{-}{-}{-}{-}{-}+    |}
      |                                                                      |
    Glass                                                               Vacuum
   Envelope
\end{verbatim}

\textbf{Key Components}:

\begin{enumerate}
\tightlist
\item
  \textbf{Electron Gun}:

  \begin{itemize}
  \tightlist
  \item
    \textbf{Cathode}: Heated filament that emits electrons
  \item
    \textbf{Control Grid}: Regulates electron beam intensity
  \item
    \textbf{Focusing Anodes}: Concentrate electrons into beam
  \item
    \textbf{Accelerating Anodes}: Increase electron velocity
  \end{itemize}
\item
  \textbf{Deflection System}:

  \begin{itemize}
  \tightlist
  \item
    \textbf{Horizontal Deflection Plates}: Control X-axis movement
  \item
    \textbf{Vertical Deflection Plates}: Control Y-axis movement
  \end{itemize}
\item
  \textbf{Screen}:

  \begin{itemize}
  \tightlist
  \item
    \textbf{Phosphor Coating}: Glows when struck by electrons
  \item
    \textbf{Glass Envelope}: Maintains vacuum and provides structure
  \end{itemize}
\end{enumerate}

\textbf{Working}:

\begin{itemize}
\tightlist
\item
  Heated cathode emits electrons
\item
  Control grid regulates beam intensity (brightness)
\item
  Focusing anodes form narrow beam
\item
  Accelerating anodes speed up electrons
\item
  Deflection plates bend beam horizontally and vertically
\item
  Electron beam strikes phosphor screen, creating visible spot
\end{itemize}

\end{solutionbox}
\begin{mnemonicbox}
``EFADS - Electrons Fly, Anodes Direct, Screen shows
signals''

\end{mnemonicbox}
\subsection*{Question 3(c) [7 marks]}\label{q3c}

\textbf{Explain the working of a Cathode Ray Oscilloscope (CRO) with the
help of a block diagram and describe the function of each block.}

\begin{solutionbox}

\textbf{Cathode Ray Oscilloscope (CRO)} is an electronic instrument used
to visualize and analyze electrical signals.

\textbf{Block Diagram}:

\begin{center}
\textbf{Mermaid Diagram (Code)}
\begin{verbatim}
{Shaded}
{Highlighting}[]
graph LR
    A[Vertical Input] {-{-}{} B[Vertical Attenuator]}
    B {-{-}{} C[Vertical Amplifier]}
    C {-{-}{} D[Delay Line]}
    D {-{-}{} E[Vertical Deflection Plates]}
    F[Trigger Circuit] {-{-}{} G[Time Base Generator]}
    G {-{-}{} H[Horizontal Amplifier]}
    H {-{-}{} I[Horizontal Deflection Plates]}
    J[External Trigger Input] {-{-}{} F}
    C {-{-}{} F}
    G {-{-}{} K[Blanking Circuit]}
    K {-{-}{} L[CRT]}
    E {-{-}{} L}
    I {-{-}{} L}
    M[Power Supply] {-{-}{} L}
    M {-{-}{} All}
{Highlighting}
{Shaded}
\end{verbatim}
\end{center}

\textbf{Functions of Each Block}:

{\def\LTcaptype{none} % do not increment counter
\begin{longtable}[]{@{}ll@{}}
\toprule\noalign{}
Block & Function \\
\midrule\noalign{}
\endhead
\bottomrule\noalign{}
\endlastfoot
Vertical Attenuator & Scales input signal to suitable level \\
Vertical Amplifier & Amplifies signal for deflection plates \\
Delay Line & Delays signal to see triggering event \\
Trigger Circuit & Initiates sweep at specific point \\
Time Base Generator & Creates sawtooth wave for horizontal sweep \\
Horizontal Amplifier & Amplifies sweep signal \\
Blanking Circuit & Cuts beam during retrace \\
CRT & Converts electrical signals to visual display \\
Power Supply & Provides various DC voltages \\
\end{longtable}
}

\textbf{Working Process}:

\begin{enumerate}
\tightlist
\item
  \textbf{Signal Input}: Connected to vertical attenuator
\item
  \textbf{Vertical Processing}: Signal scaled, amplified, delayed
\item
  \textbf{Triggering}: Trigger circuit starts time base at specific
  point
\item
  \textbf{Horizontal Sweep}: Time base creates horizontal movement
\item
  \textbf{Display}: Electron beam traces signal on screen
\item
  \textbf{Retrace}: Beam returns quickly (blanked) for next sweep
\end{enumerate}

\textbf{Controls}:

\begin{itemize}
\tightlist
\item
  \textbf{Vertical}: Volts/div, position, coupling
\item
  \textbf{Horizontal}: Time/div, position
\item
  \textbf{Trigger}: Level, slope, source, mode
\end{itemize}

\end{solutionbox}
\begin{mnemonicbox}
``VATH-CDS - Vertical Attenuates Then amplifies,
Horizontal Creates Deflection Sweep''

\end{mnemonicbox}
\subsection*{Question 3(a) OR [3
marks]}\label{q3a}

\textbf{Give the differences between Cathode Ray Oscilloscope (CRO) and
Digital Storage Oscilloscope (DSO).}

\begin{solutionbox}

\textbf{Comparison between CRO and DSO}:

{\def\LTcaptype{none} % do not increment counter
\begin{longtable}[]{@{}
  >{\raggedright\arraybackslash}p{(\linewidth - 4\tabcolsep) * \real{0.1410}}
  >{\raggedright\arraybackslash}p{(\linewidth - 4\tabcolsep) * \real{0.4103}}
  >{\raggedright\arraybackslash}p{(\linewidth - 4\tabcolsep) * \real{0.4487}}@{}}
\toprule\noalign{}
\begin{minipage}[b]{\linewidth}\raggedright
Parameter
\end{minipage} & \begin{minipage}[b]{\linewidth}\raggedright
Cathode Ray Oscilloscope (CRO)
\end{minipage} & \begin{minipage}[b]{\linewidth}\raggedright
Digital Storage Oscilloscope (DSO)
\end{minipage} \\
\midrule\noalign{}
\endhead
\bottomrule\noalign{}
\endlastfoot
Signal Processing & Analog & Digital (ADC conversion) \\
Storage Capability & None (real-time only) & Can store waveforms in
memory \\
Bandwidth & Limited by CRT technology & Higher bandwidth possible \\
Display & Phosphor screen & LCD/LED screen \\
Additional Features & Basic measurements & Advanced analysis, FFT, math
functions \\
\end{longtable}
}

\textbf{Key Differences}:

\begin{itemize}
\tightlist
\item
  \textbf{Waveform Storage}: DSO can save waveforms, CRO cannot
\item
  \textbf{Signal Processing}: DSO converts analog to digital, CRO is
  purely analog
\item
  \textbf{Pre-trigger Display}: DSO can show events before trigger
\item
  \textbf{Analysis Features}: DSO offers measurements, math functions,
  FFT
\end{itemize}

\end{solutionbox}
\begin{mnemonicbox}
``DSO-MAPS - Digital Storage Oscilloscope Measures,
Analyzes, Processes, Stores signals''

\end{mnemonicbox}
\subsection*{Question 3(b) OR [4
marks]}\label{q3b}

\textbf{Explain how frequency and phase angle can be determined with the
help of CRO.}

\begin{solutionbox}

\textbf{Frequency Measurement on CRO}:

\textbf{Method}:

\begin{enumerate}
\tightlist
\item
  Display signal on screen
\item
  Measure time period (T) using horizontal time/div setting
\item
  Calculate frequency: f = 1/T
\end{enumerate}

\textbf{Example Calculation}:

\begin{itemize}
\tightlist
\item
  If 3 cycles span 6 divisions at 0.5ms/div
\item
  Time for 3 cycles = 6 div \times 0.5ms/div = 3ms
\item
  Time for 1 cycle (T) = 3ms \div 3 = 1ms
\item
  Frequency (f) = 1/T = 1/1ms = 1kHz
\end{itemize}

\textbf{Phase Angle Measurement}:

\textbf{Method}:

\begin{enumerate}
\tightlist
\item
  Display both signals on dual channel
\item
  Measure time difference (Δt) between corresponding points
\item
  Measure time period (T) of complete cycle
\item
  Calculate phase difference: φ = (Δt/T) \times 360^\circ
\end{enumerate}

\textbf{Diagram}:

\begin{verbatim}
    Voltage
       \^{}
       |
       |    Signal 1      Signal 2
       |       /{           /}
       |      /  {         /  }
       |     /    {       /    }
       |{-{-}{-}{-}/{-}{-}{-}{-}{-}{-}{-}{-}{-}{-}{-}/{-}{-}{-}{-}{-}{-}{-}{-}{-}{-}{-} Time}
       |   /        {   /        }
       |  /          { /          }
       | /            V            {}
       |/                           {}
       +{-{-}{-}{-}{-}{-}{-}{-}{-}{-}{-}{-}{-}{-}{-}{-}{-}{-}{-}{-}{-}{-}{-}{-}{-}{-}{-}{-}{-}}
           |{{-}{-}Δt{-}{-}|}
           |{{-}{-}{-}{-}{-}{-}{-}T{-}{-}{-}{-}{-}{-}{-}{-}|}
\end{verbatim}

\textbf{Calculation}:

\begin{itemize}
\tightlist
\item
If Δt = 1 div at 0.2ms/div, and

T = 5 div at 0.2ms/div

\item
Δt = 0.2ms and

T = 1ms

\item
Phase difference:

φ = (0.2ms/1ms) \times 360^\circ = 72^\circ

\end{itemize}

\end{solutionbox}
\begin{mnemonicbox}
``FPL - Frequency = Period's Length reciprocal, Phase
= (Lag/Period) \times 360''

\end{mnemonicbox}
\subsection*{Question 3(c) OR [7
marks]}\label{q3c}

\textbf{Draw the block diagram of a Digital Storage Oscilloscope (DSO)
and explain the function of each block.}

\begin{solutionbox}

\textbf{Digital Storage Oscilloscope (DSO)} converts analog signals to
digital form for storage and analysis.

\textbf{Block Diagram}:

\begin{center}
\textbf{Mermaid Diagram (Code)}
\begin{verbatim}
{Shaded}
{Highlighting}[]
graph LR
    A[Analog Input] {-{-}{} B[Attenuator/Amplifier]}
    B {-{-}{} C[Anti{-}aliasing Filter]}
    C {-{-}{} D[Analog{-}to{-}Digital Converter]}
    D {-{-}{} E[Acquisition Memory]}
    E {-{-}{} F[Digital Signal Processor]}
    F {-{-}{} G[Display Memory]}
    G {-{-}{} H[Display Controller]}
    H {-{-}{} I[LCD Display]}
    J[Trigger System] {-{-}{} D}
    K[Microprocessor] {-{-}{} F}
    K {-{-}{} J}
    K {-{-}{} H}
    L[Control Panel] {-{-}{} K}
    M[Clock Generator] {-{-}{} D}
    M {-{-}{} K}
{Highlighting}
{Shaded}
\end{verbatim}
\end{center}

\textbf{Functions of Each Block}:

{\def\LTcaptype{none} % do not increment counter
\begin{longtable}[]{@{}
  >{\raggedright\arraybackslash}p{(\linewidth - 2\tabcolsep) * \real{0.4118}}
  >{\raggedright\arraybackslash}p{(\linewidth - 2\tabcolsep) * \real{0.5882}}@{}}
\toprule\noalign{}
\begin{minipage}[b]{\linewidth}\raggedright
Block
\end{minipage} & \begin{minipage}[b]{\linewidth}\raggedright
Function
\end{minipage} \\
\midrule\noalign{}
\endhead
\bottomrule\noalign{}
\endlastfoot
Attenuator/Amplifier & Conditions input signal to ADC range \\
Anti-aliasing Filter & Removes high frequencies to prevent aliasing \\
ADC & Converts analog signal to digital samples \\
Acquisition Memory & Stores digitized waveform data \\
Digital Signal Processor & Performs mathematical operations on
signals \\
Display Memory & Stores processed data for display \\
Display Controller & Controls screen update and format \\
Microprocessor & Controls overall operation and user interface \\
Trigger System & Determines when to start data acquisition \\
Clock Generator & Provides timing for sampling and processing \\
\end{longtable}
}

\textbf{Advantages of DSO}:

\begin{itemize}
\tightlist
\item
  \textbf{Single-shot capture}: Can capture transient events
\item
  \textbf{Pre-trigger viewing}: Shows signal before trigger point
\item
  \textbf{Waveform storage}: Saves signals for later analysis
\item
  \textbf{Advanced measurements}: Automated amplitude, timing, etc.
\item
  \textbf{Mathematical functions}: Addition, FFT, integration, etc.
\end{itemize}

\textbf{Working Process}:

\begin{enumerate}
\tightlist
\item
  Input signal conditioned by attenuator/amplifier
\item
  Signal filtered to prevent aliasing
\item
  ADC samples signal at regular intervals
\item
  Digital data stored in acquisition memory
\item
  Processor analyzes and prepares data for display
\item
  Display shows waveform and measurements
\end{enumerate}

\end{solutionbox}
\begin{mnemonicbox}
``AADPD - Attenuate Analog, Digitize, Process,
Display the signal''

\end{mnemonicbox}
\subsection*{Question 4(a) [3 marks]}\label{q4a}

\textbf{Give the classification of different types of transducers.}

\begin{solutionbox}

\textbf{Classification of Transducers}:

{\def\LTcaptype{none} % do not increment counter
\begin{longtable}[]{@{}
  >{\raggedright\arraybackslash}p{(\linewidth - 2\tabcolsep) * \real{0.7586}}
  >{\raggedright\arraybackslash}p{(\linewidth - 2\tabcolsep) * \real{0.2414}}@{}}
\toprule\noalign{}
\begin{minipage}[b]{\linewidth}\raggedright
Classification Basis
\end{minipage} & \begin{minipage}[b]{\linewidth}\raggedright
Types
\end{minipage} \\
\midrule\noalign{}
\endhead
\bottomrule\noalign{}
\endlastfoot
Principle of Operation & Mechanical, Electrical, Thermal, Optical,
Chemical \\
Input/Output Relationship & Primary, Secondary \\
Signal Generation & Active, Passive \\
Electrical Parameters & Resistive, Capacitive, Inductive \\
Transduction & Photoelectric, Electrochemical, Thermoelectric \\
\end{longtable}
}

\textbf{Primary Classification}:

\begin{enumerate}
\tightlist
\item
  \textbf{Based on Energy Conversion}:

  \begin{itemize}
  \tightlist
  \item
    \textbf{Active Transducers}: Generate electrical output without
    external power (e.g., thermocouple)
  \item
    \textbf{Passive Transducers}: Require external power (e.g.,
    thermistor)
  \end{itemize}
\item
  \textbf{Based on Principle of Operation}:

  \begin{itemize}
  \tightlist
  \item
    \textbf{Primary Transducers}: Convert physical change directly to
    electrical signal
  \item
    \textbf{Secondary Transducers}: Require intermediate conversion
  \end{itemize}
\end{enumerate}

\end{solutionbox}
\begin{mnemonicbox}
``APRCI - Active/Passive,
Resistive/Capacitive/Inductive are key categories''

\end{mnemonicbox}
\subsection*{Question 4(b) [4 marks]}\label{q4b}

\textbf{Explain the construction and working of a strain gauge.}

\begin{solutionbox}

\textbf{Strain Gauge} converts mechanical strain (deformation) into
electrical resistance change.

\textbf{Construction}:

\begin{itemize}
\tightlist
\item
  \textbf{Grid Pattern}: Thin foil or wire in zigzag pattern
\item
  \textbf{Backing Material}: Polyimide or epoxy carrier
\item
  \textbf{Lead Wires}: Connected to measurement circuit
\item
  \textbf{Adhesive}: Bonds gauge to test surface
\end{itemize}

\textbf{Diagram}:

\begin{verbatim}
   Lead Wire                Lead Wire
      |                        |
      v                        v
    +{-{-}{-}{-}{-}{-}{-}{-}{-}{-}{-}{-}{-}{-}{-}{-}{-}{-}{-}{-}{-}{-}{-}{-}{-}{-}{-}{-}+}
    |                            |  Backing
    |   +{-{-}{-}{-}{-}{-}{-}{-}{-}{-}{-}{-}{-}{-}{-}{-}{-}{-}{-}{-}+   |  Material}
    |   | /{//////// |   |}
    |   | {                / |   |}
    |   |  {    Grid     /   |   |}
    |   |   {   Pattern /    |   |}
    |   |    {/////      |   |}
    |   +{-{-}{-}{-}{-}{-}{-}{-}{-}{-}{-}{-}{-}{-}{-}{-}{-}{-}{-}{-}+   |}
    |                            |
    +{-{-}{-}{-}{-}{-}{-}{-}{-}{-}{-}{-}{-}{-}{-}{-}{-}{-}{-}{-}{-}{-}{-}{-}{-}{-}{-}{-}+}
\end{verbatim}

\textbf{Working Principle}:

\begin{itemize}
\tightlist
\item
  Based on piezoresistive effect
\item
  When object deforms, gauge deforms
\item
  Deformation changes resistance per formula:

  \begin{itemize}
  \tightlist
  \item
    ΔR/R = GF \times ε
  \item
Where GF = Gauge Factor,

ε = Strain

  \end{itemize}
\end{itemize}

\textbf{Measurement Circuit}:

\begin{itemize}
\tightlist
\item
  Usually connected in Wheatstone bridge
\item
  Converts small resistance change to voltage
\item
  Output voltage proportional to strain
\end{itemize}

\textbf{Applications}:

\begin{itemize}
\tightlist
\item
  Load cells, pressure sensors
\item
  Structural testing
\item
  Mechanical stress analysis
\end{itemize}

\end{solutionbox}
\begin{mnemonicbox}
``GRID - Gauge Resistance Increases with
Deformation''

\end{mnemonicbox}
\subsection*{Question 4(c) [7 marks]}\label{q4c}

\textbf{Explain the Linear Variable Differential Transducer (LVDT) with
its construction, working, advantages, and applications.}

\begin{solutionbox}

\textbf{Linear Variable Differential Transformer (LVDT)} is an
electromechanical sensor that converts linear displacement into
electrical signal.

\textbf{Construction}:

\begin{itemize}
\tightlist
\item
  \textbf{Primary Coil}: Central winding excited by AC source
\item
  \textbf{Secondary Coils}: Two identical coils on either side
\item
  \textbf{Core}: Ferromagnetic material that moves with displacement
\item
  \textbf{Housing}: Cylindrical shell with terminals
\end{itemize}

\textbf{Diagram}:

\begin{center}
\textbf{Mermaid Diagram (Code)}
\begin{verbatim}
{Shaded}
{Highlighting}[]
graph LR
    A[AC Source] {-{-}{} B[Primary Coil]}
    C[Core] {-{-}{-} B}
    B {-{-}{-} D[Secondary Coil 1]}
    B {-{-}{-} E[Secondary Coil 2]}
    D {-{-}{} F[Signal Conditioning]}
    E {-{-}{} F}
    F {-{-}{} G[Output]}
    H[Movement] {-{-}{} C}
{Highlighting}
{Shaded}
\end{verbatim}
\end{center}

\textbf{Working Principle}:

\begin{itemize}
\tightlist
\item
  AC voltage applied to primary coil
\item
  Magnetic flux couples to secondary coils
\item
  Core position determines coupling efficiency
\item
  Voltage differential between secondaries ∝ displacement
\item
  At null position (center), secondary voltages are equal and opposite
\end{itemize}

\textbf{Characteristic Curve}:

\begin{verbatim}
     Output
       \^{}
       |                Secondary Voltages
       |                     /
       |                    /
       |                   /
       |                  /
       |                 /
       |{-{-}{-}{-}{-}{-}{-}{-}{-}{-}{-}{-}{-}{-}{-}{-}/{-}{-}{-}{-}{-}{-}{-}{-}{-}{-}{-}{-}{-}{-}{-}{-}{-} Displacement}
       |               /
       |              /
       |             /
     {-{-}|{-}{-}{-}{-}{-}{-}{-}{-}{-}{-}{-}{-}/{-}{-}{-}{-}}
       |           /
  Null Position
\end{verbatim}

\textbf{Advantages}:

\begin{itemize}
\tightlist
\item
  \textbf{Frictionless operation}: No mechanical contact
\item
  \textbf{Infinite resolution}: Analog output
\item
  \textbf{High linearity}: Direct proportional output
\item
  \textbf{Ruggedness}: Resistant to shock and vibration
\item
  \textbf{Long life}: No wearing parts
\end{itemize}

\textbf{Applications}:

\begin{itemize}
\tightlist
\item
  \textbf{Industrial}: Automated machine tools, robotics
\item
  \textbf{Aerospace}: Flight control systems
\item
  \textbf{Civil Engineering}: Structural testing
\item
  \textbf{Metrology}: Precision measurement instruments
\end{itemize}

\end{solutionbox}
\begin{mnemonicbox}
``LVDT-MAPS - Linear Variable Differential
Transformer Measures Accurately Position by Secondary voltage
differences''

\end{mnemonicbox}
\subsection*{Question 4(a) OR [3
marks]}\label{q4a}

\textbf{State any three uses of PH sensors.}

\begin{solutionbox}

\textbf{Uses of PH Sensors}:

{\def\LTcaptype{none} % do not increment counter
\begin{longtable}[]{@{}
  >{\raggedright\arraybackslash}p{(\linewidth - 4\tabcolsep) * \real{0.3824}}
  >{\raggedright\arraybackslash}p{(\linewidth - 4\tabcolsep) * \real{0.2647}}
  >{\raggedright\arraybackslash}p{(\linewidth - 4\tabcolsep) * \real{0.3529}}@{}}
\toprule\noalign{}
\begin{minipage}[b]{\linewidth}\raggedright
Application
\end{minipage} & \begin{minipage}[b]{\linewidth}\raggedright
Purpose
\end{minipage} & \begin{minipage}[b]{\linewidth}\raggedright
Importance
\end{minipage} \\
\midrule\noalign{}
\endhead
\bottomrule\noalign{}
\endlastfoot
Water Treatment & Monitor and control water quality & Ensures safe
drinking water \\
Agriculture & Soil monitoring for optimal plant growth & Increases crop
yield \\
Medical Diagnostics & Measuring body fluid acidity & Critical for
patient health \\
\end{longtable}
}

\textbf{Additional Applications}:

\begin{itemize}
\tightlist
\item
  \textbf{Food Processing}: Quality control during production
\item
  \textbf{Aquaculture}: Maintaining optimal water conditions
\item
  \textbf{Chemical Manufacturing}: Process control
\end{itemize}

\end{solutionbox}
\begin{mnemonicbox}
``WAM - Water quality control, Agriculture soil
testing, Medical diagnostics are key PH sensor applications''

\end{mnemonicbox}
\subsection*{Question 4(b) OR [4
marks]}\label{q4b}

\textbf{Explain the construction and working of a capacitive
transducer.}

\begin{solutionbox}

\textbf{Capacitive Transducer} converts physical change into capacitance
variation which is measured electrically.

\textbf{Construction}:

\begin{itemize}
\tightlist
\item
  \textbf{Parallel Plates}: Two conductive plates
\item
  \textbf{Dielectric Medium}: Air, ceramic, or other material
\item
  \textbf{Housing}: Protective enclosure
\item
  \textbf{Terminals}: Electrical connections
\end{itemize}

\textbf{Diagram}:

\begin{verbatim}
    Terminal A             Terminal B
        |                     |
        v                     v
    +{-{-}{-}{-}{-}{-}{-}{-}{-}{-}+         +{-}{-}{-}{-}{-}{-}{-}{-}{-}{-}+}
    |          |         |          |
    |  Plate A |         |  Plate B |
    |          |{{-}{-}{-}{-}{-}{-}{-}|          |}
    |          |    d    |          |
    +{-{-}{-}{-}{-}{-}{-}{-}{-}{-}+         +{-}{-}{-}{-}{-}{-}{-}{-}{-}{-}+}
           |                 |
           |    Dielectric   |
           |     Material    |
           |                 |
    +{-{-}{-}{-}{-}{-}{-}{-}{-}{-}{-}{-}{-}{-}{-}{-}{-}{-}{-}{-}{-}{-}{-}{-}{-}{-}+}
    |        Housing           |
    +{-{-}{-}{-}{-}{-}{-}{-}{-}{-}{-}{-}{-}{-}{-}{-}{-}{-}{-}{-}{-}{-}{-}{-}{-}{-}+}
\end{verbatim}

\textbf{Working Principle}:

\begin{itemize}
\tightlist
\item
  Capacitance C = ε_{0}εᵣA/d

  \begin{itemize}
  \tightlist
  \item
    ε_{0} = Permittivity of free space
  \item
    εᵣ = Relative permittivity of dielectric
  \item
    A = Area of plates
  \item
    d = Distance between plates
  \end{itemize}
\end{itemize}

\textbf{Types of Variation}:

\begin{enumerate}
\tightlist
\item
  \textbf{Area variation}: Changing overlap of plates
\item
  \textbf{Distance variation}: Changing separation between plates
\item
  \textbf{Dielectric variation}: Changing dielectric material
\end{enumerate}

\textbf{Applications}:

\begin{itemize}
\tightlist
\item
  \textbf{Pressure sensors}: Diaphragm changes plate distance
\item
  \textbf{Level sensors}: Dielectric changes with fluid level
\item
  \textbf{Humidity sensors}: Dielectric changes with moisture
\item
  \textbf{Proximity sensors}: Distance changes with object presence
\end{itemize}

\end{solutionbox}
\begin{mnemonicbox}
``CAD - Capacitance changes with Area, Distance, or
Dielectric variations''

\end{mnemonicbox}
\subsection*{Question 4(c) OR [7
marks]}\label{q4c}

\textbf{Describe absolute optical encoder and its A, B, C waveform
outputs with proper illustration.}

\begin{solutionbox}

\textbf{Absolute Optical Encoder} directly measures angular position by
generating a unique digital code for each position.

\textbf{Construction}:

\begin{itemize}
\tightlist
\item
  \textbf{Code Disc}: Contains concentric tracks with transparent/opaque
  sectors
\item
  \textbf{Light Source}: LED array illuminating the disc
\item
  \textbf{Photo Detectors}: Sensors detecting light through disc
  patterns
\item
  \textbf{Signal Conditioning}: Converts photodetector signals to
  digital outputs
\end{itemize}

\textbf{Diagram}:

\begin{center}
\textbf{Mermaid Diagram (Code)}
\begin{verbatim}
{Shaded}
{Highlighting}[]
graph LR
    A[LED Light Source] {-{-}{} B[Code Disc]}
    B {-{-}{} C[Photodetectors]}
    C {-{-}{} D[Signal Conditioning Circuit]}
    D {-{-}{} E[Digital Output]}
    F[Rotating Shaft] {-{-}{} B}
{Highlighting}
{Shaded}
\end{verbatim}
\end{center}

\textbf{Code Disc Pattern}:

\begin{verbatim}
          Track C (Index)
            |
     {-{-}{-}{-}{-}{-}{-}|{-}{-}{-}{-}{-}{-}{-}{-}{-}{-}{-}{-}{-}{-}{-}{-}{-}}
            |
          {-{-}+{-}{-}  {-}{-}+{-}{-}  {-}{-}+{-}{-}}
           Track B
     {-{-}{-}{-}{-}{-}{-}|{-}{-}{-}{-}{-}{-}{-}{-}{-}{-}{-}{-}{-}{-}{-}{-}{-}}
            |
          {-++{-}{-}  {-}++{-}{-}  {-}++{-}{-}}
           Track A
     {-{-}{-}{-}{-}{-}{-}|{-}{-}{-}{-}{-}{-}{-}{-}{-}{-}{-}{-}{-}{-}{-}{-}{-}}
            |
          {-+++{-} {-}+++{-} {-}+++{-} {-}}
     {-{-}{-}{-}{-}{-}{-}|{-}{-}{-}{-}{-}{-}{-}{-}{-}{-}{-}{-}{-}{-}{-}{-}{-}}
            |
            V
          Rotation
\end{verbatim}

\textbf{Waveform Outputs}:

{\def\LTcaptype{none} % do not increment counter
\begin{longtable}[]{@{}lll@{}}
\toprule\noalign{}
Signal & Purpose & Characteristics \\
\midrule\noalign{}
\endhead
\bottomrule\noalign{}
\endlastfoot
A Signal & Position information & Square wave, 50\% duty cycle \\
B Signal & Direction information & 90^\circ phase shifted from A \\
C Signal & Reference/index & Single pulse per revolution \\
\end{longtable}
}

\textbf{Output Waveforms}:

\begin{verbatim}
    A Signal  \_\_\_\_\_|‾‾‾‾‾|\_\_\_\_\_|‾‾‾‾‾|\_\_\_\_\_|‾‾‾‾‾|\_\_\_\_\_|‾‾‾‾‾|\_\_\_\_\_
    
    B Signal  \_\_|‾‾‾‾‾|\_\_\_\_\_|‾‾‾‾‾|\_\_\_\_\_|‾‾‾‾‾|\_\_\_\_\_|‾‾‾‾‾|\_\_\_\_\_|‾‾
    
    C Signal  \_\_\_\_\_|‾|\_\_\_\_\_\_\_\_\_\_\_\_\_\_\_\_\_\_\_\_\_\_\_\_\_\_\_\_\_\_\_\_\_\_\_\_\_\_\_\_\_\_\_\_\_\_\_\_\_
              
              0^    90^   180^   270^   360^   450^   540^   630^   720^
\end{verbatim}

\textbf{Working principle}:

\begin{itemize}
\tightlist
\item
  A \& B output provides quadrature signals (90^\circ out of phase)
\item
  Direction determined by which signal leads:

  \begin{itemize}
  \tightlist
  \item
    If A leads B: Clockwise rotation
  \item
    If B leads A: Counter-clockwise rotation
  \end{itemize}
\item
  Position determined by counting pulses
\item
  C signal provides reference/home position
\end{itemize}

\textbf{Applications}:

\begin{itemize}
\tightlist
\item
  \textbf{CNC machines}: Precise position control
\item
  \textbf{Robotics}: Joint angle measurement
\item
  \textbf{Camera systems}: Lens positioning
\item
  \textbf{Industrial automation}: Motor control
\end{itemize}

\end{solutionbox}
\begin{mnemonicbox}
``ABC-PDP - Absolute encoder tracks A, B, C Provide
Direction, Position, and reference pulse''

\end{mnemonicbox}
\subsection*{Question 5(a) [3 marks]}\label{q5a}

\textbf{Describe the working principle of a basic frequency counter.}

\begin{solutionbox}

\textbf{Frequency Counter} measures frequency of an input signal by
counting events over a precise time interval.

\textbf{Working Principle}:

\begin{itemize}
\tightlist
\item
  Count number of cycles/pulses of input signal
\item
  Divide by the precise gate time
\item
  Display resulting frequency
\end{itemize}

\textbf{Basic Blocks}:

\begin{itemize}
\tightlist
\item
  \textbf{Input Conditioning}: Shapes signal to digital levels
\item
  \textbf{Gate Control}: Opens gate for precise time
\item
  \textbf{Counter}: Counts pulses during gate open time
\item
  \textbf{Time Base}: Generates precise gate timing
\item
  \textbf{Display}: Shows frequency value
\end{itemize}

\textbf{Simplified Diagram}:

\begin{center}
\textbf{Mermaid Diagram (Code)}
\begin{verbatim}
{Shaded}
{Highlighting}[]
graph LR
    A[Input Signal] {-{-}{} B[Input Conditioning]}
    B {-{-}{} C[AND Gate]}
    D[Time Base] {-{-}{} E[Gate Control]}
    E {-{-}{} C}
    C {-{-}{} F[Counter]}
    F {-{-}{} G[Display]}
{Highlighting}
{Shaded}
\end{verbatim}
\end{center}

\end{solutionbox}
\begin{mnemonicbox}
``CTPG - Count The Pulses, Gate the time''

\end{mnemonicbox}
\subsection*{Question 5(b) [4 marks]}\label{q5b}

\textbf{Draw the diagram of an energy meter and explain its working
principle.}

\begin{solutionbox}

\textbf{Electronic Energy Meter} measures electrical energy consumption
in kilowatt-hours (kWh).

\textbf{Block Diagram}:

\begin{center}
\textbf{Mermaid Diagram (Code)}
\begin{verbatim}
{Shaded}
{Highlighting}[]
graph LR
    A[Voltage Sensor] {-{-}{} C[Analog Multiplier]}
    B[Current Sensor] {-{-}{} C}
    C {-{-}{} D[Voltage{-}to{-}Frequency Converter]}
    D {-{-}{} E[Pulse Counter]}
    E {-{-}{} F[Microcontroller]}
    F {-{-}{} G[LCD Display]}
    H[Crystal Oscillator] {-{-}{} F}
    F {-{-}{} I[LED Indicator]}
    F {-{-}{} J[Communication Interface]}
{Highlighting}
{Shaded}
\end{verbatim}
\end{center}

\textbf{Working Principle}:

\begin{itemize}
\tightlist
\item
  Energy = Power \times Time
\item
  Power = Voltage \times Current
\item
  Voltage and current sensed separately
\item
  Multiplied to get instantaneous power
\item
  Integrated over time to get energy
\item
  Pulses generated proportional to energy
\item
  Each pulse represents fixed energy unit
\item
  Counter accumulates pulses
\item
  Display shows accumulated energy
\end{itemize}

\textbf{Features}:

\begin{itemize}
\tightlist
\item
  \textbf{Tamper detection}: Prevents electricity theft
\item
  \textbf{Multiple tariffs}: Different rates for different times
\item
  \textbf{Communication}: Remote reading capability
\end{itemize}

\end{solutionbox}
\begin{mnemonicbox}
``VCPI - Voltage and Current are multiplied, Pulses
Indicate energy used''

\end{mnemonicbox}
\subsection*{Question 5(c) [7 marks]}\label{q5c}

\textbf{Briefly explain the working principle and functions of a
function generator. Describe its front panel controls and explain how it
is used to test electronic circuits with suitable examples.}

\begin{solutionbox}

\textbf{Function Generator} is an electronic test instrument that
generates different waveforms with adjustable frequency and amplitude.

\textbf{Working Principle}:

\begin{itemize}
\tightlist
\item
  Generates base signal using oscillator circuit
\item
  Shapes waveform using wave-shaping circuits
\item
  Adjusts amplitude, frequency, and offset parameters
\item
  Outputs waveform through buffer amplifier
\end{itemize}

\textbf{Block Diagram}:

\begin{center}
\textbf{Mermaid Diagram (Code)}
\begin{verbatim}
{Shaded}
{Highlighting}[]
graph LR
    A[Oscillator] {-{-}{} B[Wave Shaper]}
    B {-{-}{} C[Output Amplifier]}
    D[Frequency Control] {-{-}{} A}
    E[Waveform Selector] {-{-}{} B}
    F[Amplitude Control] {-{-}{} C}
    G[DC Offset Control] {-{-}{} C}
    C {-{-}{} H[Output]}
    I[Modulation Input] {-{-}{} A}
{Highlighting}
{Shaded}
\end{verbatim}
\end{center}

\textbf{Front Panel Controls}:

{\def\LTcaptype{none} % do not increment counter
\begin{longtable}[]{@{}
  >{\raggedright\arraybackslash}p{(\linewidth - 4\tabcolsep) * \real{0.2647}}
  >{\raggedright\arraybackslash}p{(\linewidth - 4\tabcolsep) * \real{0.2941}}
  >{\raggedright\arraybackslash}p{(\linewidth - 4\tabcolsep) * \real{0.4412}}@{}}
\toprule\noalign{}
\begin{minipage}[b]{\linewidth}\raggedright
Control
\end{minipage} & \begin{minipage}[b]{\linewidth}\raggedright
Function
\end{minipage} & \begin{minipage}[b]{\linewidth}\raggedright
Typical Range
\end{minipage} \\
\midrule\noalign{}
\endhead
\bottomrule\noalign{}
\endlastfoot
Frequency & Sets signal frequency & 0.1 Hz - 20 MHz \\
Amplitude & Sets signal amplitude & 0 - 20 Vpp \\
DC Offset & Adds DC voltage & \pm10V \\
Waveform Select & Chooses waveform type & Sine, Triangle, Square,
Pulse \\
Duty Cycle & Adjusts pulse width & 10\% - 90\% \\
Modulation & AM/FM modulation & Internal/External \\
\end{longtable}
}

\textbf{Output Waveforms}:

\begin{verbatim}
    Sine      /{      /      /}
             /  {    /      /  }
    \_\_\_\_\_\_\_ /    {\_\_/    \_\_/    \_\_}
    
    Square   \_\_\_\_\_\_      \_\_\_\_\_\_
            |      |    |      |
    \_\_\_\_\_\_\_\_|      |\_\_\_\_|      |\_\_\_\_
    
    Triangle  /{      /      /}
             /  {    /      /  }
    \_\_\_\_\_\_\_\_/    {\_\_/    \_\_/    \_\_}
    
    Pulse     \_\_        \_\_        \_\_
             |  |      |  |      |  |
    \_\_\_\_\_\_\_\_\_|  |\_\_\_\_\_\_|  |\_\_\_\_\_\_|  |\_
\end{verbatim}

\textbf{Circuit Testing Applications}:

{\def\LTcaptype{none} % do not increment counter
\begin{longtable}[]{@{}lll@{}}
\toprule\noalign{}
Application & Waveform Used & Purpose \\
\midrule\noalign{}
\endhead
\bottomrule\noalign{}
\endlastfoot
Amplifier Testing & Sine wave & Gain, frequency response \\
Digital Circuit Testing & Square wave & Logic timing, thresholds \\
Filter Testing & Sine sweep & Cutoff frequency, response \\
Triggering Circuits & Pulse & Threshold testing \\
\end{longtable}
}

\textbf{Example: Testing Amplifier}

\begin{enumerate}
\tightlist
\item
  Connect function generator to amplifier input
\item
  Set sine wave of appropriate amplitude
\item
  Vary frequency to test frequency response
\item
  Monitor output on oscilloscope
\item
  Calculate gain = Output amplitude / Input amplitude
\end{enumerate}

\end{solutionbox}
\begin{mnemonicbox}
``FAWOD - Frequency, Amplitude, Waveform, Offset,
Duty cycle are key controls''

\end{mnemonicbox}
\subsection*{Question 5(a) OR [3
marks]}\label{q5a}

\textbf{Describe the working of a spectrum analyzer.}

\begin{solutionbox}

\textbf{Spectrum Analyzer} measures signal amplitude versus frequency,
showing frequency components of signals.

\textbf{Working Principle}:

\begin{itemize}
\tightlist
\item
  Converts time-domain signals to frequency-domain
\item
  Shows spectral components and their amplitudes
\item
  Uses superheterodyne receiver architecture
\item
  Sweeps local oscillator to analyze frequency range
\end{itemize}

\textbf{Block Diagram}:

\begin{center}
\textbf{Mermaid Diagram (Code)}
\begin{verbatim}
{Shaded}
{Highlighting}[]
graph LR
    A[Input Signal] {-{-}{} B[Attenuator/Amplifier]}
    B {-{-}{} C[Mixer]}
    D[Local Oscillator] {-{-}{} C}
    C {-{-}{} E[IF Filter]}
    E {-{-}{} F[Detector]}
    F {-{-}{} G[Display]}
    H[Sweep Generator] {-{-}{} D}
    H {-{-}{} G}
{Highlighting}
{Shaded}
\end{verbatim}
\end{center}

\textbf{Applications}:

\begin{itemize}
\tightlist
\item
  \textbf{Signal analysis}: Measuring harmonics, distortion
\item
  \textbf{EMI testing}: Finding interference sources
\item
  \textbf{Communications}: Channel analysis, modulation quality
\end{itemize}

\end{solutionbox}
\begin{mnemonicbox}
``SAME - Spectrum Analyzer Maps signal Energy across
frequencies''

\end{mnemonicbox}
\subsection*{Question 5(b) OR [4
marks]}\label{q5b}

\textbf{Draw a neat diagram of a clamp-on meter and explain its
working.}

\begin{solutionbox}

\textbf{Clamp-on Meter} (Current Clamp) is a non-contact device for
measuring AC/DC current.

\textbf{Construction Diagram}:

\begin{verbatim}
         Display
         .{-{-}{-}{-}{-}{-}.}
        /        {}
       /  120.5A  {    Function}
      /            {    Selector}
     |   O |{-{-}|     |  .{-}{-}{-}{-}.}
     |     |  |     |  |    |
     |     |  |     |{-{-}    |}
     |     |  |     |       |
     |     |  |     |       |
     |     {{-}{-}     |       |}
      {   Trigger   /       |}
       {           /        |}
        |         |        /
        |         |       /
        |         |      /
        {{-}{-}{-}{-}{-}{-}{-}{-}{-}´     /}
             |         /
           Clamp      /
             |       /
            /       /
           /       /
          /       /
         {{-}\_\_\_\_\_\_´}
         Test Leads
\end{verbatim}

\textbf{Working Principle}:

\begin{itemize}
\tightlist
\item
  Based on electromagnetic induction (Faraday's Law)
\item
  Current-carrying conductor creates magnetic field
\item
  Clamp's ferromagnetic core concentrates field
\item
  Secondary coil in the clamp induces proportional voltage
\item
  Circuit converts induced voltage to current reading
\end{itemize}

\textbf{Advantages}:

\begin{itemize}
\tightlist
\item
  \textbf{Non-contact}: No need to disconnect circuit
\item
  \textbf{Safety}: Isolation from high voltages
\item
  \textbf{Convenience}: Easy to use in confined spaces
\end{itemize}

\textbf{Applications}:

\begin{itemize}
\tightlist
\item
  \textbf{Electrical maintenance}: Motor current, load testing
\item
  \textbf{Power quality}: Measuring power factor, harmonics
\item
  \textbf{Troubleshooting}: Finding unbalanced loads
\end{itemize}

\end{solutionbox}
\begin{mnemonicbox}
``CLIP - Clamp measures current, Lets magnetic
Induction Produce voltage''

\end{mnemonicbox}
\subsection*{Question 5(c) OR [7
marks]}\label{q5c}

\textbf{Explain the working principle of a digital IC tester. Describe
its block diagram and explain how it is used to test the functionality
of digital ICs with a suitable example.}

\begin{solutionbox}

\textbf{Digital IC Tester} verifies functionality of digital integrated
circuits by applying test patterns and comparing responses.

\textbf{Working Principle}:

\begin{itemize}
\tightlist
\item
  Applies predefined test vectors to IC pins
\item
  Compares actual outputs with expected outputs
\item
  Identifies faulty ICs or incorrect functions
\item
  Tests multiple IC types using stored test patterns
\end{itemize}

\textbf{Block Diagram}:

\begin{center}
\textbf{Mermaid Diagram (Code)}
\begin{verbatim}
{Shaded}
{Highlighting}[]
graph LR
    A[Microcontroller] {-{-}{} B[ROM/Test Pattern Memory]}
    A {-{-}{} C[Input Pattern Generator]}
    C {-{-}{} D[ZIF Socket/IC Under Test]}
    D {-{-}{} E[Output Response Analyzer]}
    E {-{-}{} A}
    A {-{-}{} F[Display]}
    G[Keypad/Control Panel] {-{-}{} A}
    H[Power Supply] {-{-}{} D}
    H {-{-}{} A}
{Highlighting}
{Shaded}
\end{verbatim}
\end{center}

\textbf{Major Components}:

\begin{itemize}
\tightlist
\item
  \textbf{ZIF Socket}: Zero Insertion Force socket for easy IC placement
\item
  \textbf{Test Pattern Memory}: Stores test vectors for various ICs
\item
  \textbf{Output Response Analyzer}: Compares actual vs.~expected
  outputs
\item
  \textbf{Microcontroller}: Controls testing sequence and evaluation
\item
  \textbf{Display}: Shows test results and status
\end{itemize}

\textbf{Testing Method}:

{\def\LTcaptype{none} % do not increment counter
\begin{longtable}[]{@{}lll@{}}
\toprule\noalign{}
Step & Action & Purpose \\
\midrule\noalign{}
\endhead
\bottomrule\noalign{}
\endlastfoot
1 & Select IC type & Load correct test parameters \\
2 & Insert IC in ZIF socket & Prepare for testing \\
3 & Start test & Begin test sequence \\
4 & Apply test vectors & Exercise IC functions \\
5 & Compare responses & Identify errors \\
6 & Display results & Show pass/fail status \\
\end{longtable}
}

\textbf{Example: Testing 7400 NAND Gate IC}:

\begin{enumerate}
\tightlist
\item
  Select ``7400'' from IC list
\item
  Insert IC in ZIF socket
\item
  Tester applies all input combinations:

  \begin{itemize}
  \tightlist
  \item
    Input 1A=0, 1B=0 \rightarrow Expected output 1Y=1
  \item
    Input 1A=0, 1B=1 \rightarrow Expected output 1Y=1
  \item
    Input 1A=1, 1B=0 \rightarrow Expected output 1Y=1
  \item
    Input 1A=1, 1B=1 \rightarrow Expected output 1Y=0
  \end{itemize}
\item
  Repeat for all gates in package (7400 has 4 NAND gates)
\item
  Compare actual outputs to expected truth table
\item
  Display ``PASS'' if all tests succeed, or error code if failure
\end{enumerate}

\textbf{Features of Modern IC Testers}:

\begin{itemize}
\tightlist
\item
  \textbf{Auto-identification}: Detects unknown ICs
\item
  \textbf{Learning mode}: Creates test patterns for new ICs
\item
  \textbf{Functional testing}: Tests in-circuit operation
\item
  \textbf{Parameter testing}: Checks timing, voltage margins
\end{itemize}

\end{solutionbox}
\begin{mnemonicbox}
``TEST - Test patterns Exercise all States, Then
verify outputs''

\begin{verbatim}
  Lead Wire        Lead Wire
      |               |
      v               v
    +{-{-}{-}{-}{-}{-}{-}{-}{-}{-}{-}{-}{-}{-}{-}{-}{-}+}
    |   Ceramic or    |
    | Semiconductor   |
    |     Body        |
    +{-{-}{-}{-}{-}{-}{-}{-}{-}{-}{-}{-}{-}{-}{-}{-}{-}+}
\end{verbatim}

\textbf{Working}: Resistance decreases as temperature increases (NTC
type) or increases with temperature (PTC type).

\textbf{Key Points}:

\begin{itemize}
\tightlist
\item
  \textbf{NTC (Negative Temperature Coefficient)}: Most common type
\item
  \textbf{High sensitivity}: Large resistance change for small
  temperature change
\item
  \textbf{Nonlinear response}: Requires linearization circuits
\item
  \textbf{Self-heating}: Current passing through it causes heating
\end{itemize}

\end{mnemonicbox}
\begin{mnemonicbox}
``TRIP - Thermocouples React to junction differences,
Thermistors Intensely change resistance, Point sensors at what you
measure''

\end{mnemonicbox}

\end{document}
