\documentclass[10pt,a4paper]{article}

% content/resources/templates/preamble.tex
\usepackage[margin=0.6in]{geometry}
\author{Milav Dabgar}
\usepackage{amsmath,amssymb,amsthm}
\usepackage{booktabs}
\usepackage{multirow}
\usepackage{xcolor}
\usepackage{tcolorbox}
\tcbuselibrary{breakable,skins}
\usepackage[colorlinks=true,linkcolor=blue]{hyperref}
\usepackage{titlesec}
\usepackage{enumitem}
\usepackage{tikz}
\usepackage{pgfplots}
\usepackage{circuitikz}
\usepackage[version=4]{mhchem}
\usepackage{longtable}
\usepackage{array}
\usepackage{float}
\usepackage{caption}
\usepackage{listings}

\lstset{
  basicstyle=\small\ttfamily,
  breaklines=true,
  breakatwhitespace=false,
  postbreak=\mbox{\textcolor{red}{$\hookrightarrow$}\space},
  float=false,
  numbers=left,
  numberstyle=\tiny\color{gray},
  numbersep=10pt,
  xleftmargin=2em,
  keywordstyle=\color{blue},
  commentstyle=\color{green!60!black},
  stringstyle=\color{purple},
  backgroundcolor=\color{gray!5},
  showstringspaces=false,
  tabsize=2,
  captionpos=b,
  keepspaces=true,
  columns=flexible
}

\pgfplotsset{compat=1.18}
\usetikzlibrary{shapes,arrows,positioning,calc,patterns,decorations.pathmorphing,decorations.markings,arrows.meta}

% Color scheme
\definecolor{headcolor}{RGB}{0,102,204}
\definecolor{keycolor}{RGB}{220,20,60}
\definecolor{solutioncolor}{RGB}{34,139,34}
\definecolor{mnemoniccolor}{RGB}{148,0,211}
\definecolor{codecolor}{RGB}{0,0,100}

% Spacing
\setlength{\parskip}{3pt}
\setlist[itemize]{nosep}
\setlist[enumerate]{nosep}

% Title formatting
\titleformat{\section}{\Large\bfseries\color{headcolor}}{\thesection}{1em}{}
\titleformat{\subsection}{\large\bfseries\color{headcolor}}{\thesubsection}{1em}{}

% Pandoc tightlist compatibility
\providecommand{\tightlist}{%
  \setlength{\itemsep}{0pt}\setlength{\parskip}{0pt}}

% Pandoc longtable compatibility
\newcounter{none}
\def\thenone{}


% content/resources/templates/english-boxes.tex
% This file is currently empty - it exists to maintain consistency with the import structure.
% Add custom environments here if needed in the future.


\begin{document}

\begin{center}
{\Huge\bfseries\color{headcolor} Subject Name Solutions}\\[5pt]
{\LARGE 4331102 -- Winter 2022}\\[3pt]
{\large Semester 1 Study Material}\\[3pt]
{\normalsize\textit{Detailed Solutions and Explanations}}
\end{center}

\vspace{10pt}

\subsection*{Question 1(a) [3 marks]}\label{q1a}

\textbf{Draw and explain working of Basic Q-Meter.}

\begin{solutionbox}
Q-meter is an instrument used to measure the quality
factor (Q) of an inductor or capacitor.

\textbf{Diagram:}

\begin{center}
\textbf{Mermaid Diagram (Code)}
\begin{verbatim}
{Shaded}
{Highlighting}[]
graph LR
    A[Oscillator] {-{-}{} B[Amplifier]}
    B {-{-}{} C[Meter Circuit]}
    C {-{-}{} D[Voltage Indicator]}
    C {-{-}{} E[Unknown Component]}
    E {-{-}{} C}
{Highlighting}
{Shaded}
\end{verbatim}
\end{center}

\begin{itemize}
\tightlist
\item
  \textbf{Oscillator}: Generates variable frequency signal
\item
  \textbf{Amplifier}: Amplifies the signal to required level
\item
  \textbf{Resonance Circuit}: Contains the component under test
\item
  \textbf{Voltage Indicator}: Measures the voltage across component
\end{itemize}

\end{solutionbox}
\begin{mnemonicbox}
``OARV - Oscillate, Amplify, Resonate, View''

\end{mnemonicbox}
\subsection*{Question 1(b) [4 marks]}\label{q1b}

\textbf{Explain Spectrum Analyzer in brief.}

\begin{solutionbox}
A spectrum analyzer measures the magnitude of an input
signal versus frequency within the full frequency range of the
instrument.

\textbf{Diagram:}

\begin{center}
\textbf{Mermaid Diagram (Code)}
\begin{verbatim}
{Shaded}
{Highlighting}[]
graph LR
    A[Input Signal] {-{-}{} B[Mixer]}
    C[Local Oscillator] {-{-}{} B}
    B {-{-}{} D[IF Filter]}
    D {-{-}{} E[Detector]}
    E {-{-}{} F[Display]}
{Highlighting}
{Shaded}
\end{verbatim}
\end{center}

\begin{itemize}
\tightlist
\item
  \textbf{Input Signal Processing}: Signals enter through attenuator and
  filters
\item
  \textbf{Frequency Domain Conversion}: Converts time domain to
  frequency domain
\item
  \textbf{Display System}: Shows amplitude vs.~frequency plot
\item
  \textbf{Applications}: Signal analysis, distortion measurement, EMI
  testing
\end{itemize}

\end{solutionbox}
\begin{mnemonicbox}
``SAME-FD: Signal Analysis Measures Everything in
Frequency Domain''

\end{mnemonicbox}
\subsection*{Question 1(c) [7 marks]}\label{q1c}

\textbf{Explain Wheatstone bridge with circuit diagram. List its
advantages and disadvantages.}

\begin{solutionbox}
Wheatstone bridge is a circuit used to measure unknown
resistance with high accuracy.

\textbf{Diagram:}

\begin{center}
\textbf{Mermaid Diagram (Code)}
\begin{verbatim}
{Shaded}
{Highlighting}[]
graph TD
    A(({+)) {-}{-}{-} R1}
    A {-{-}{-} R3}
    R1 {-{-}{-} B((G))}
    R3 {-{-}{-} B}
    R1 {-{-}{-} R2}
    R3 {-{-}{-} Rx}
    R2 {-{-}{-} C((−))}
    Rx {-{-}{-} C}
{Highlighting}
{Shaded}
\end{verbatim}
\end{center}

Where:

\begin{itemize}
\tightlist
\item
  R1, R2, R3 are known resistances
\item
  Rx is unknown resistance
\item
  G is galvanometer
\end{itemize}

\textbf{Working Principle}:

\begin{itemize}
\tightlist
\item
  Bridge is balanced when R1/R2 = R3/Rx
\item
  At balance, no current flows through galvanometer
\item
  Unknown resistance Rx = R3(R2/R1)
\end{itemize}

{\def\LTcaptype{none} % do not increment counter
\begin{longtable}[]{@{}
  >{\raggedright\arraybackslash}p{(\linewidth - 2\tabcolsep) * \real{0.4444}}
  >{\raggedright\arraybackslash}p{(\linewidth - 2\tabcolsep) * \real{0.5556}}@{}}
\toprule\noalign{}
\begin{minipage}[b]{\linewidth}\raggedright
Advantages
\end{minipage} & \begin{minipage}[b]{\linewidth}\raggedright
Disadvantages
\end{minipage} \\
\midrule\noalign{}
\endhead
\bottomrule\noalign{}
\endlastfoot
High accuracy & Limited range \\
Good sensitivity & Temperature effects \\
Null type measurement & Requires balance adjustment \\
No need for calibrated meter & Not suitable for very low/high
resistances \\
\end{longtable}
}

\end{solutionbox}
\begin{mnemonicbox}
``BARN - Balance Achieved when Ratios are Null''

\end{mnemonicbox}
\subsection*{Question 1(c) OR [7
marks]}\label{q1c}

\textbf{Define Instrument and explain its characteristics.}

\begin{solutionbox}
An instrument is a device used for measuring,
displaying or recording physical quantities.

{\def\LTcaptype{none} % do not increment counter
\begin{longtable}[]{@{}ll@{}}
\toprule\noalign{}
Characteristics & Description \\
\midrule\noalign{}
\endhead
\bottomrule\noalign{}
\endlastfoot
\textbf{Accuracy} & Closeness of measurement to true value \\
\textbf{Precision} & Repeatability of measurements \\
\textbf{Resolution} & Smallest change that can be detected \\
\textbf{Sensitivity} & Ratio of output signal to input signal change \\
\textbf{Linearity} & Proportional relationship between input and
output \\
\textbf{Range} & Minimum to maximum measurable values \\
\textbf{Response time} & Time required to show true reading \\
\end{longtable}
}

\textbf{Diagram:}

\begin{center}
\textbf{Mermaid Diagram (Code)}
\begin{verbatim}
{Shaded}
{Highlighting}[]
graph LR
    A[Input] {-{-}{} B[Instrument]}
    B {-{-}{} C[Output Reading]}
    D[Error Sources] {-{-}{} B}
    E[Environmental Factors] {-{-}{} B}
{Highlighting}
{Shaded}
\end{verbatim}
\end{center}

\begin{itemize}
\tightlist
\item
  \textbf{Static Characteristics}: Properties that don't vary with time
\item
  \textbf{Dynamic Characteristics}: Properties that vary with time
\end{itemize}

\end{solutionbox}
\begin{mnemonicbox}
``APRS-LRR: Accuracy and Precision, Resolution and
Sensitivity, Linearity, Range, Response time''

\end{mnemonicbox}
\subsection*{Question 2(a) [3 marks]}\label{q2a}

\textbf{Draw the construction diagram of Energy meter.}

\begin{solutionbox}
Energy meter measures electrical energy consumption in
kilowatt-hours.

\textbf{Diagram:}

\begin{verbatim}
                   +{-{-}{-}{-}{-}{-}{-}+}
                   | Meter |
                   | Dial  |
                   +{-{-}{-}{-}{-}{-}{-}+}
                       |
                    +{-{-}{-}{-}{-}+}
                    |Brake|
                    |Disc |
                    +{-{-}{-}{-}{-}+}
                    /     {}
                   /       {}
           +{-{-}{-}{-}{-}{-}{-}+         +{-}{-}{-}{-}{-}{-}{-}+}
           |Current|         |Voltage|
           |Coil   |         |Coil   |
           +{-{-}{-}{-}{-}{-}{-}+         +{-}{-}{-}{-}{-}{-}{-}+}
           
\end{verbatim}

\begin{itemize}
\tightlist
\item
  \textbf{Rotating Aluminum Disc}: Moves proportional to power
\item
  \textbf{Current Coil}: Creates flux proportional to current
\item
  \textbf{Voltage Coil}: Creates flux proportional to voltage
\item
  \textbf{Permanent Magnet}: Provides braking torque
\end{itemize}

\end{solutionbox}
\begin{mnemonicbox}
``DVCP: Disc Velocity measures Consumed Power''

\end{mnemonicbox}
\subsection*{Question 2(b) [4 marks]}\label{q2b}

\textbf{Explain working of PMMC in short.}

\begin{solutionbox}
PMMC (Permanent Magnet Moving Coil) is a basic
mechanism used in various meters.

\textbf{Diagram:}

\begin{verbatim}
      +{-{-}{-}{-}{-}{-}{-}+}
      |       |
    S |  Coil | N
      |       |
      +{-{-}{-}{-}{-}{-}{-}+}
      |Spring |
      +{-{-}{-}{-}{-}{-}{-}+}
        Pointer
\end{verbatim}

{\def\LTcaptype{none} % do not increment counter
\begin{longtable}[]{@{}ll@{}}
\toprule\noalign{}
Component & Function \\
\midrule\noalign{}
\endhead
\bottomrule\noalign{}
\endlastfoot
Permanent Magnet & Creates strong magnetic field \\
Moving Coil & Carries current to be measured \\
Spring & Provides controlling torque \\
Pointer & Indicates reading on scale \\
\end{longtable}
}

\begin{itemize}
\tightlist
\item
  \textbf{Deflection Principle}: When current flows through coil, it
  produces torque proportional to current
\item
  \textbf{Advantages}: Linear scale, high accuracy, low power
  consumption
\end{itemize}

\end{solutionbox}
\begin{mnemonicbox}
``CODA: Current through cOil causes Deflection by
Attraction''

\end{mnemonicbox}
\subsection*{Question 2(c) [7 marks]}\label{q2c}

\textbf{1- A moving coil ammeter reading up to 1 ampere has a resistance
of 0.02 ohm. How this instrument could be adopted to read current up to
100 amperes?}

\textbf{2- A moving coil voltmeter reading up to 20 mV has a resistance
of 2 ohms. How this instrument can be adopted to read voltage up to 300
volts?}

\begin{solutionbox}

\textbf{1. Ammeter Range Extension:}

\textbf{Diagram:}

\begin{verbatim}
    I = 100A
    +{-{-}{-}{-}{-}{-}{-}{-}{-}{-}{-}+}
    |           |
    +{-{-}+     +{-}{-}+}
       |     |
       |     |
     +{-+{-}+ +{-}+{-}+}
     |Rm | |Rs |
     +{-+{-}+ +{-}+{-}+}
       |     |
       |     |
    +{-{-}+     +{-}{-}+}
    |           |
    +{-{-}{-}{-}{-}{-}{-}{-}{-}{-}{-}+}
\end{verbatim}

\begin{itemize}
\tightlist
\item
  \textbf{Shunt Resistance Calculation}: Rs = Rm \times Im/(I - Im)
\item
  \textbf{Given}: Rm = 0.02Ω, Im = 1A, I = 100A
\item
  \textbf{Solution}: Rs = 0.02 \times 1/(100 - 1) = 0.02/99 = 0.000202Ω
\end{itemize}

\textbf{2. Voltmeter Range Extension:}

\textbf{Diagram:}

\begin{verbatim}
    +{-{-}{-}{-}Rs{-}{-}{-}{-}+}
    |          |
    |    +{-{-}+  |}
    +{-{-}{-}{-}+Rm+{-}{-}+}
         +{-{-}+}
          V
\end{verbatim}

\begin{itemize}
\tightlist
\item
  \textbf{Series Resistance Calculation}: Rs = Rm \times (V/Vm - 1)
\item
  \textbf{Given}: Rm = 2Ω, Vm = 20mV, V = 300V
\item
  \textbf{Solution}: Rs = 2 \times (300/0.02 - 1) = 2 \times (15000 - 1) = 2 \times
  14999 = 29,998Ω
\end{itemize}

\end{solutionbox}
\begin{mnemonicbox}
``SHIP: Shunt Has Inverse Proportion for current;
Series for voltage''

\end{mnemonicbox}
\subsection*{Question 2(a) OR [3
marks]}\label{q2a}

\textbf{Explain working of electronic multimeter.}

\begin{solutionbox}
Electronic multimeter measures multiple electrical
parameters using electronic components.

\textbf{Diagram:}

\begin{center}
\textbf{Mermaid Diagram (Code)}
\begin{verbatim}
{Shaded}
{Highlighting}[]
graph LR
    A[Input Signal] {-{-}{} B[Range Selection]}
    B {-{-}{} C[Conversion Circuit]}
    C {-{-}{} D[Display System]}
{Highlighting}
{Shaded}
\end{verbatim}
\end{center}

\begin{itemize}
\tightlist
\item
  \textbf{Range Selection}: Selects appropriate measurement range
\item
  \textbf{Signal Conditioning}: Converts input to proportional voltage
\item
  \textbf{ADC}: Converts analog to digital for display
\item
  \textbf{Digital Display}: Shows measured value
\end{itemize}

\end{solutionbox}
\begin{mnemonicbox}
``RSAD: Range Select, Amplify, Digitize''

\end{mnemonicbox}
\subsection*{Question 2(b) OR [4
marks]}\label{q2b}

\textbf{Explain working of Moving Iron type instruments.}

\begin{solutionbox}
Moving Iron instruments measure AC/DC current and
voltage based on magnetic attraction/repulsion.

{\def\LTcaptype{none} % do not increment counter
\begin{longtable}[]{@{}ll@{}}
\toprule\noalign{}
Types & Working Principle \\
\midrule\noalign{}
\endhead
\bottomrule\noalign{}
\endlastfoot
Attraction Type & Iron piece is attracted toward electromagnet \\
Repulsion Type & Two iron pieces repel each other \\
\end{longtable}
}

\textbf{Diagram:}

\begin{verbatim}
    +{-{-}{-}{-}{-}{-}{-}{-}+}
    | Spring |
    +{-{-}{-}{-}+{-}{-}{-}+}
         |
    +{-{-}{-}{-}+{-}{-}{-}{-}+}
    |Iron Vane|{-{-}{-}{-}{-}{-}{-}{-}+ Pointer}
    +{-{-}{-}{-}+{-}{-}{-}{-}+}
         |
    +{-{-}{-}{-}+{-}{-}{-}+}
    |  Coil  |
    +{-{-}{-}{-}{-}{-}{-}{-}+}
\end{verbatim}

\begin{itemize}
\tightlist
\item
  \textbf{Operating Principle}: Current through coil creates magnetic
  field
\item
  \textbf{Scale}: Non-linear (crowded at lower end)
\item
  \textbf{Applications}: AC and DC measurements, ammeters, voltmeters
\end{itemize}

\end{solutionbox}
\begin{mnemonicbox}
``CADS: Current Activates, Deflection Shows''

\end{mnemonicbox}
\subsection*{Question 2(c) OR [7
marks]}\label{q2c}

\textbf{Draw the block diagram of Ramp type DVM. Illustrate process of
obtaining Multirange DC voltmeter with circuit diagram.}

\begin{solutionbox}
Ramp type DVM converts voltage to time interval using
ramp comparison.

\textbf{Diagram for Ramp Type DVM:}

\begin{center}
\textbf{Mermaid Diagram (Code)}
\begin{verbatim}
{Shaded}
{Highlighting}[]
graph LR
    A[Input Voltage] {-{-}{} B[Comparator]}
    C[Ramp Generator] {-{-}{} B}
    B {-{-}{} D[Gate Control]}
    E[Clock] {-{-}{} F[Counter]}
    D {-{-}{} F}
    F {-{-}{} G[Display]}
{Highlighting}
{Shaded}
\end{verbatim}
\end{center}

\begin{itemize}
\tightlist
\item
  \textbf{Working Principle}: Measures time taken for ramp to equal
  input voltage
\item
  \textbf{Comparator}: Compares input with ramp voltage
\item
  \textbf{Counter}: Counts clock pulses during comparison
\item
  \textbf{Display}: Shows digital reading
\end{itemize}

\textbf{Multirange DC Voltmeter Circuit:}

\begin{verbatim}
       +{-{-}R1{-}{-}+}
       |      |
    Input     +{-{-}R2{-}{-}+}
       |      |      |
    +{-{-}+      +{-}{-}R3{-}{-}+}
    |            |   |
    +{-{-}Switch{-}{-}{-}{-}+   |}
                     |
                   +{-+{-}+}
                   |DVM|
                   +{-+{-}+}
\end{verbatim}

\textbf{Range Switching Process:}

\begin{itemize}
\tightlist
\item
  Each resistor provides different voltage division ratio
\item
  Switch selects appropriate voltage divider network
\item
  Voltage divider reduces input to fit DVM range
\end{itemize}

\end{solutionbox}
\begin{mnemonicbox}
``CRCD: Compare Ramp, Count Duration''

\end{mnemonicbox}
\subsection*{Question 3(a) [3 marks]}\label{q3a}

\textbf{Describe features of Digital storage oscilloscope (DSO).}

\begin{solutionbox}
Digital Storage Oscilloscope converts analog signals to
digital for storage and analysis.

{\def\LTcaptype{none} % do not increment counter
\begin{longtable}[]{@{}ll@{}}
\toprule\noalign{}
Features & Description \\
\midrule\noalign{}
\endhead
\bottomrule\noalign{}
\endlastfoot
\textbf{Digital Storage} & Stores waveforms for later analysis \\
\textbf{Triggering} & Multiple trigger modes and sources \\
\textbf{Waveform Processing} & Math operations on waveforms \\
\textbf{FFT Analysis} & Frequency domain view of signals \\
\textbf{Multiple Channels} & Simultaneous viewing of signals \\
\textbf{USB/LAN Connectivity} & Data transfer capabilities \\
\end{longtable}
}

\begin{itemize}
\tightlist
\item
  \textbf{Sampling Rate}: Typically 1 GS/s or higher
\item
  \textbf{Memory Depth}: Determines maximum capture time
\end{itemize}

\end{solutionbox}
\begin{mnemonicbox}
``SACRED: Storage, Analysis, Connectivity,
Resolution, Extended functions, Digital processing''

\end{mnemonicbox}
\subsection*{Question 3(b) [4 marks]}\label{q3b}

\textbf{Explain frequency measurement method using Lissajous pattern.}

\begin{solutionbox}
Lissajous patterns are used to compare frequencies of
two signals.

\textbf{Diagram:}

\begin{verbatim}
    +{-{-}{-}{-}{-}{-}{-}+     +{-}{-}{-}{-}{-}{-}{-}+}
    |       |     |       |
    |   o   |     |   8   |
    |       |     |       |
    +{-{-}{-}{-}{-}{-}{-}+     +{-}{-}{-}{-}{-}{-}{-}+}
    1:1 ratio     2:1 ratio
    
    +{-{-}{-}{-}{-}{-}{-}+     +{-}{-}{-}{-}{-}{-}{-}+}
    |       |     |       |
    |      |     |   ⋮⋮⋮  |
    |       |     |       |
    +{-{-}{-}{-}{-}{-}{-}+     +{-}{-}{-}{-}{-}{-}{-}+}
    3:1 ratio     4:1 ratio
\end{verbatim}

\textbf{Method:}

\begin{enumerate}
\tightlist
\item
  Apply unknown frequency to X-input
\item
  Apply reference frequency to Y-input
\item
  Observe Lissajous pattern on screen
\item
  Count tangent points to determine ratio
\end{enumerate}

\textbf{Formula:} fx/fy = Ny/Nx

\begin{itemize}
\tightlist
\item
  Where Nx = horizontal tangent points
\item
  Ny = vertical tangent points
\end{itemize}

\end{solutionbox}
\begin{mnemonicbox}
``XTYN: X-Tangents to Y-tangents gives the Number
ratio''

\end{mnemonicbox}
\subsection*{Question 3(c) [7 marks]}\label{q3c}

\textbf{Explain CRO with help of Block diagram.}

\begin{solutionbox}
Cathode Ray Oscilloscope (CRO) is used to display and
analyze waveforms.

\textbf{Block Diagram:}

\begin{center}
\textbf{Mermaid Diagram (Code)}
\begin{verbatim}
{Shaded}
{Highlighting}[]
graph LR
    A[Vertical Input] {-{-}{} B[Vertical Attenuator]}
    B {-{-}{} C[Vertical Amplifier]}
    C {-{-}{} D[Vertical Deflection Plates]}
    E[Trigger Circuit] {-{-}{} F[Time Base Generator]}
    F {-{-}{} G[Horizontal Amplifier]}
    G {-{-}{} H[Horizontal Deflection Plates]}
    I[Power Supply] {-{-}{} J[CRT]}
    D {-{-}{} J}
    H {-{-}{} J}
{Highlighting}
{Shaded}
\end{verbatim}
\end{center}

{\def\LTcaptype{none} % do not increment counter
\begin{longtable}[]{@{}ll@{}}
\toprule\noalign{}
Block & Function \\
\midrule\noalign{}
\endhead
\bottomrule\noalign{}
\endlastfoot
Vertical Section & Processes input signal for Y-deflection \\
Horizontal Section & Generates sweep signal for X-deflection \\
Trigger Circuit & Synchronizes sweep with input signal \\
CRT & Displays the waveform pattern \\
Power Supply & Provides required voltages \\
\end{longtable}
}

\begin{itemize}
\tightlist
\item
  \textbf{Electron Gun}: Produces electron beam
\item
  \textbf{Deflection System}: Moves beam in X and Y directions
\item
  \textbf{Screen}: Phosphor coating converts electrons to visible light
\end{itemize}

\end{solutionbox}
\begin{mnemonicbox}
``VCTHP: Vertical input, Conditioned signal,
Triggered sweep, Horizontal deflection, Phosphor display''

\end{mnemonicbox}
\subsection*{Question 3(a) OR [3
marks]}\label{q3a}

\textbf{Explain different types of CRO probes.}

\begin{solutionbox}
CRO probes connect the circuit under test to the
oscilloscope input.

{\def\LTcaptype{none} % do not increment counter
\begin{longtable}[]{@{}
  >{\raggedright\arraybackslash}p{(\linewidth - 4\tabcolsep) * \real{0.2857}}
  >{\raggedright\arraybackslash}p{(\linewidth - 4\tabcolsep) * \real{0.3810}}
  >{\raggedright\arraybackslash}p{(\linewidth - 4\tabcolsep) * \real{0.3333}}@{}}
\toprule\noalign{}
\begin{minipage}[b]{\linewidth}\raggedright
Probe Type
\end{minipage} & \begin{minipage}[b]{\linewidth}\raggedright
Characteristics
\end{minipage} & \begin{minipage}[b]{\linewidth}\raggedright
Applications
\end{minipage} \\
\midrule\noalign{}
\endhead
\bottomrule\noalign{}
\endlastfoot
\textbf{Passive Probes} & Simple, economical, high impedance &
General-purpose measurements \\
\textbf{Active Probes} & Built-in amplifier, low loading &
High-frequency circuits \\
\textbf{Current Probes} & Measures current without circuit breaking &
Current waveform measurements \\
\textbf{Differential Probes} & Measures between two points & Floating
measurements \\
\end{longtable}
}

\textbf{Diagram:}

\begin{verbatim}
    +{-{-}{-}{-}{-}{-}{-}+      +{-}{-}{-}{-}{-}{-}{-}+}
    | Scope |{{-}{-}{-}{-}{-}| Probe |}
    +{-{-}{-}{-}{-}{-}{-}+      +{-}{-}{-}{-}{-}{-}{-}+}
                       |
                   +{-{-}{-}+{-}{-}{-}+}
                   |Circuit|
                   +{-{-}{-}{-}{-}{-}{-}+}
\end{verbatim}

\begin{itemize}
\tightlist
\item
  \textbf{Attenuation Ratio}: Typically 1:1 or 10:1
\item
  \textbf{Compensation}: Adjustable to match oscilloscope input
\end{itemize}

\end{solutionbox}
\begin{mnemonicbox}
``PACD: Passive, Active, Current, Differential''

\end{mnemonicbox}
\subsection*{Question 3(b) OR [4
marks]}\label{q3b}

\textbf{Draw internal structure of CRT. Explain in brief.}

\begin{solutionbox}
Cathode Ray Tube (CRT) is the display device in an
oscilloscope.

\textbf{Diagram:}

\begin{verbatim}
    Electron Gun                Deflection Plates         Screen
    +{-{-}{-}{-}{-}{-}{-}{-}{-}{-}{-}+                  +{-}{-}+     +{-}{-}{-}+            +{-}{-}{-}{-}{-}+}
    |           |                  |  |     |   |            |     |
    | C G A1 A2 |{-{-}{-}{-}{-}{-}{-}{-}{-}{-}{-}{-}{-}{-}{-}{-}{-}{-}|Y |{-}{-}{-}{-}{-}| X |{-}{-}{-}{-}{-}{-}{-}{-}{-}{-}{-}{-}| P   |}
    |           |                  |  |     |   |            |     |
    +{-{-}{-}{-}{-}{-}{-}{-}{-}{-}{-}+                  +{-}{-}+     +{-}{-}{-}+            +{-}{-}{-}{-}{-}+}
    
    C: Cathode, G: Grid, A1, A2: Anodes, Y,X: Deflection Plates, P: Phosphor
\end{verbatim}

{\def\LTcaptype{none} % do not increment counter
\begin{longtable}[]{@{}ll@{}}
\toprule\noalign{}
Component & Function \\
\midrule\noalign{}
\endhead
\bottomrule\noalign{}
\endlastfoot
Electron Gun & Produces electron beam \\
Control Grid & Regulates beam intensity \\
Focusing Anodes & Concentrates electron beam \\
Deflection Plates & Control beam position \\
Phosphor Screen & Converts electrons to light \\
\end{longtable}
}

\begin{itemize}
\tightlist
\item
  \textbf{Electron Beam}: High-velocity electrons emitted by cathode
\item
  \textbf{Focusing System}: Anodes form electron lens
\item
  \textbf{Deflection System}: X-Y plates move beam position
\item
  \textbf{Phosphor Screen}: Glows where beam hits
\end{itemize}

\end{solutionbox}
\begin{mnemonicbox}
``GAFDS: Gun Aims, Focusing Directs, Screen shows''

\end{mnemonicbox}
\subsection*{Question 3(c) OR [7
marks]}\label{q3c}

\textbf{Draw and explain block diagram of DSO in detail.}

\begin{solutionbox}
Digital Storage Oscilloscope (DSO) captures, stores and
analyzes signals in digital form.

\textbf{Block Diagram:}

\begin{center}
\textbf{Mermaid Diagram (Code)}
\begin{verbatim}
{Shaded}
{Highlighting}[]
graph LR
    A[Input] {-{-}{} B[Attenuator/Amplifier]}
    B {-{-}{} C[Anti{-}aliasing Filter]}
    C {-{-}{} D[ADC]}
    D {-{-}{} E[Memory]}
    E {-{-}{} F[Microprocessor]}
    F {-{-}{} G[Display]}
    H[Timebase] {-{-}{} F}
    I[Trigger] {-{-}{} F}
    J[Control Panel] {-{-}{} F}
{Highlighting}
{Shaded}
\end{verbatim}
\end{center}

{\def\LTcaptype{none} % do not increment counter
\begin{longtable}[]{@{}ll@{}}
\toprule\noalign{}
Block & Function \\
\midrule\noalign{}
\endhead
\bottomrule\noalign{}
\endlastfoot
Input Section & Signal conditioning and scaling \\
ADC & Converts analog to digital signals \\
Memory & Stores digitized waveform data \\
Microprocessor & Controls acquisition and processing \\
Display System & Shows waveforms and measurements \\
Trigger System & Determines when to start acquisition \\
\end{longtable}
}

\begin{itemize}
\tightlist
\item
  \textbf{Sampling Rate}: Number of samples per second
\item
  \textbf{Resolution}: Number of bits in ADC (typically 8-12 bits)
\item
  \textbf{Memory Depth}: Number of samples that can be stored
\item
  \textbf{Processing}: Waveform math, measurements, analysis
\end{itemize}

\end{solutionbox}
\begin{mnemonicbox}
``SAMPLE-D: Signal Acquisition, Memory Processing,
Locking trigger, Display''

\end{mnemonicbox}
\subsection*{Question 4(a) [3 marks]}\label{q4a}

\textbf{Give the comparison of NTC and PTC thermistor.}

\begin{solutionbox}

{\def\LTcaptype{none} % do not increment counter
\begin{longtable}[]{@{}
  >{\raggedright\arraybackslash}p{(\linewidth - 4\tabcolsep) * \real{0.2558}}
  >{\raggedright\arraybackslash}p{(\linewidth - 4\tabcolsep) * \real{0.3721}}
  >{\raggedright\arraybackslash}p{(\linewidth - 4\tabcolsep) * \real{0.3721}}@{}}
\toprule\noalign{}
\begin{minipage}[b]{\linewidth}\raggedright
Parameter
\end{minipage} & \begin{minipage}[b]{\linewidth}\raggedright
NTC Thermistor
\end{minipage} & \begin{minipage}[b]{\linewidth}\raggedright
PTC Thermistor
\end{minipage} \\
\midrule\noalign{}
\endhead
\bottomrule\noalign{}
\endlastfoot
\textbf{Resistance Change} & Decreases with temperature & Increases with
temperature \\
\textbf{Material} & Metal oxides (Mn, Ni, Co, Cu) & Barium titanate,
polymers \\
\textbf{Response} & Exponential decrease & Sharp increase above
threshold \\
\textbf{Applications} & Temperature measurement, compensation &
Overcurrent protection, heating \\
\textbf{Temperature Range} & -50^\circC to 300^\circC & 0^\circC to 200^\circC \\
\end{longtable}
}

\textbf{Diagram:}

\begin{verbatim}
    R |      /
      |     /
      |    /   PTC
      |   /
      |  /
      | /
      |/
      |{}
      | {}
      |  {}
      |   {    NTC}
      |    {}
      |     {}
      +{-{-}{-}{-}{-}{-}+{-}{-}{-}}
             T
\end{verbatim}

\end{solutionbox}
\begin{mnemonicbox}
``IN-DP: Increase Negative, Decrease Positive''

\end{mnemonicbox}
\subsection*{Question 4(b) [4 marks]}\label{q4b}

\textbf{Explain working principle and construction of Thermocouple.}

\begin{solutionbox}
Thermocouple is a temperature sensor that works on the
principle of Seebeck effect.

\textbf{Diagram:}

\begin{verbatim}
    Metal A   +{-{-}{-}{-}{-}{-}{-}{-}+}
    {-{-}{-}{-}{-}{-}{-}{-}{-}|        |}
              | V{-meter|}
    Metal B   |        |
    {-{-}{-}{-}{-}{-}{-}{-}{-}+{-}{-}{-}{-}{-}{-}{-}{-}+}
        |
        |
    +{-{-}{-}+{-}{-}{-}+}
    |Hot End|
    +{-{-}{-}{-}{-}{-}{-}+}
\end{verbatim}

\textbf{Construction:}

\begin{itemize}
\tightlist
\item
  Two dissimilar metals joined at one end (measuring junction)
\item
  Other ends connected to measuring circuit (reference junction)
\item
  Protective sheath for industrial applications
\end{itemize}

\textbf{Working Principle:}

\begin{itemize}
\tightlist
\item
  Temperature difference between junctions creates EMF
\item
  EMF is proportional to temperature difference
\item
  Output voltage typically in millivolts range
\item
  Different metal combinations for different ranges
\end{itemize}

\end{solutionbox}
\begin{mnemonicbox}
``STEM: Seebeck-effect Transforms temperature to EMF
in Metals''

\end{mnemonicbox}
\subsection*{Question 4(c) [7 marks]}\label{q4c}

\textbf{Explain Working of strain Gauge and Load cell. Give advantages
and disadvantages of RTD.}

\begin{solutionbox}

\textbf{Strain Gauge Working:}

\begin{itemize}
\tightlist
\item
  \textbf{Principle}: Resistance changes with mechanical deformation
\item
  \textbf{Construction}: Thin wire or foil grid mounted on backing
  material
\item
  \textbf{Operation}: When strained, resistance changes proportionally
\item
  \textbf{Gauge Factor}: Ratio of relative change in resistance to
  strain
\end{itemize}

\textbf{Diagram for Strain Gauge:}

\begin{verbatim}
    +{-{-}{-}{-}{-}{-}{-}{-}{-}{-}{-}{-}{-}{-}{-}{-}{-}{-}{-}{-}{-}{-}+}
    |  ┌─┐┌─┐┌─┐┌─┐┌─┐┌─┐  |
    |  │ ││ ││ ││ ││ ││ │  |
    |  └─┘└─┘└─┘└─┘└─┘└─┘  |
    +{-{-}{-}{-}{-}{-}{-}{-}{-}{-}{-}{-}{-}{-}{-}{-}{-}{-}{-}{-}{-}{-}+}
           Backing
\end{verbatim}

\textbf{Load Cell Working:}

\begin{itemize}
\tightlist
\item
  \textbf{Construction}: Strain gauges mounted on metal body (beam/ring)
\item
  \textbf{Operation}: Weight causes deformation measured by strain
  gauges
\item
  \textbf{Circuit}: Typically Wheatstone bridge configuration
\item
  \textbf{Output}: Usually few millivolts per volt of excitation
\end{itemize}

\textbf{Diagram for Load Cell:}

\begin{verbatim}
    +{-{-}{-}{-}{-}{-}{-}+    Force   +{-}{-}{-}{-}{-}{-}{-}+}
    |       |{-{-}{-}{-}{-}{-}{-}{-}{-}{-}{-}|       |}
    |Fixed  |            |Strain |
    |Support|            |Gauges |
    +{-{-}{-}{-}{-}{-}{-}+            +{-}{-}{-}{-}{-}{-}{-}+}
\end{verbatim}

\textbf{RTD (Resistance Temperature Detector):}

{\def\LTcaptype{none} % do not increment counter
\begin{longtable}[]{@{}ll@{}}
\toprule\noalign{}
Advantages & Disadvantages \\
\midrule\noalign{}
\endhead
\bottomrule\noalign{}
\endlastfoot
High accuracy & Expensive \\
Good stability & Requires excitation current \\
Wide temperature range & Self-heating effects \\
Linear response & Lower sensitivity than thermistor \\
Good repeatability & Slower response time \\
\end{longtable}
}

\end{solutionbox}
\begin{mnemonicbox}
``SPANNER: Strain Proportionally Alters Nominal
Nominal Electrical Resistance''

\end{mnemonicbox}
\subsection*{Question 4(a) OR [3
marks]}\label{q4a}

\textbf{Explain Humidity Sensor Hygrometer.}

\begin{solutionbox}
Humidity sensor hygrometer measures relative humidity
in air.

\textbf{Diagram:}

\begin{center}
\textbf{Mermaid Diagram (Code)}
\begin{verbatim}
{Shaded}
{Highlighting}[]
graph LR
    A[Humidity] {-{-}{} B[Sensing Element]}
    B {-{-}{} C[Signal Conditioning]}
    C {-{-}{} D[Display/Output]}
{Highlighting}
{Shaded}
\end{verbatim}
\end{center}

{\def\LTcaptype{none} % do not increment counter
\begin{longtable}[]{@{}ll@{}}
\toprule\noalign{}
Type & Sensing Principle \\
\midrule\noalign{}
\endhead
\bottomrule\noalign{}
\endlastfoot
Capacitive & Humidity changes dielectric constant \\
Resistive & Humidity changes resistance \\
Thermal & Humidity affects thermal conductivity \\
\end{longtable}
}

\begin{itemize}
\tightlist
\item
  \textbf{Relative Humidity}: Ratio of actual to maximum water vapor
\item
  \textbf{Measurement Range}: Typically 0-100\% RH
\item
  \textbf{Applications}: Weather stations, HVAC systems, industrial
  processes
\end{itemize}

\end{solutionbox}
\begin{mnemonicbox}
``CRT-H: Capacitance/Resistance/Thermal changes with
Humidity''

\end{mnemonicbox}
\subsection*{Question 4(b) OR [4
marks]}\label{q4b}

\textbf{Draw and explain Piezoelectric transducer.}

\begin{solutionbox}
Piezoelectric transducer converts mechanical stress to
electrical signals and vice versa.

\textbf{Diagram:}

\begin{verbatim}
    +{-{-}{-}{-}{-}{-}{-}{-}{-}{-}{-}{-}{-}{-}{-}+}
    |    Electrodes |
    |  +{-{-}{-}{-}{-}{-}{-}{-}{-}+  |}
    |  |         |  |
    |  | Crystal |  |
    |  |         |  |
    |  +{-{-}{-}{-}{-}{-}{-}{-}{-}+  |}
    |    Electrodes |
    +{-{-}{-}{-}{-}{-}{-}{-}{-}{-}{-}{-}{-}{-}{-}+}
       |         |
       + Output  +
\end{verbatim}

\textbf{Working Principle:}

\begin{itemize}
\tightlist
\item
  \textbf{Direct Effect}: Pressure produces electrical charge
\item
  \textbf{Inverse Effect}: Voltage produces mechanical deformation
\item
  \textbf{Materials}: Quartz, PZT, barium titanate
\end{itemize}

\textbf{Applications:}

\begin{itemize}
\tightlist
\item
  Pressure sensors
\item
  Accelerometers
\item
  Ultrasonic transducers
\item
  Vibration sensors
\end{itemize}

\end{solutionbox}
\begin{mnemonicbox}
``PEMS: Pressure Ensures Measurable Signal''

\end{mnemonicbox}
\subsection*{Question 4(c) OR [7
marks]}\label{q4c}

\textbf{Give the classification of transducers in detail.}

\begin{solutionbox}
Transducers convert one form of energy to another,
classified in several ways:

{\def\LTcaptype{none} % do not increment counter
\begin{longtable}[]{@{}
  >{\raggedright\arraybackslash}p{(\linewidth - 4\tabcolsep) * \real{0.4848}}
  >{\raggedright\arraybackslash}p{(\linewidth - 4\tabcolsep) * \real{0.2121}}
  >{\raggedright\arraybackslash}p{(\linewidth - 4\tabcolsep) * \real{0.3030}}@{}}
\toprule\noalign{}
\begin{minipage}[b]{\linewidth}\raggedright
Classification
\end{minipage} & \begin{minipage}[b]{\linewidth}\raggedright
Types
\end{minipage} & \begin{minipage}[b]{\linewidth}\raggedright
Examples
\end{minipage} \\
\midrule\noalign{}
\endhead
\bottomrule\noalign{}
\endlastfoot
\textbf{Based on Energy Conversion} & \textbf{Mechanical to Electrical}
& Strain gauge, LVDT \\
& \textbf{Thermal to Electrical} & Thermocouple, RTD \\
& \textbf{Optical to Electrical} & Photodiode, LDR \\
& \textbf{Chemical to Electrical} & pH sensor, gas sensor \\
\textbf{Based on Operating Principle} & \textbf{Resistive} & Strain
gauge, thermistor \\
& \textbf{Inductive} & LVDT, proximity sensor \\
& \textbf{Capacitive} & Humidity sensor, pressure sensor \\
& \textbf{Piezoelectric} & Accelerometer, force sensor \\
\textbf{Based on Application} & \textbf{Temperature} & Thermocouple,
RTD, thermistor \\
& \textbf{Pressure} & Diaphragm, strain gauge based \\
& \textbf{Flow} & Ultrasonic, turbine, venturi \\
& \textbf{Level} & Float, ultrasonic, capacitive \\
\end{longtable}
}

\textbf{Diagram:}

\begin{center}
\textbf{Mermaid Diagram (Code)}
\begin{verbatim}
{Shaded}
{Highlighting}[]
graph TD
    A[Transducers] {-{-}{} B[Active/Passive]}
    A {-{-}{} C[Primary/Secondary]}
    A {-{-}{} D[Analog/Digital]}
    B {-{-}{} B1[Active: Self{-}generating]}
    B {-{-}{} B2[Passive: External power]}
    C {-{-}{} C1[Primary: Direct conversion]}
    C {-{-}{} C2[Secondary: Multiple steps]}
    D {-{-}{} D1[Analog: Continuous output]}
    D {-{-}{} D2[Digital: Discrete output]}
{Highlighting}
{Shaded}
\end{verbatim}
\end{center}

\end{solutionbox}
\begin{mnemonicbox}
``APAD RICE: Active/Passive, Analog/Digital with
Resistive, Inductive, Capacitive, Electromagnetic''

\end{mnemonicbox}
\subsection*{Question 5(a) [3 marks]}\label{q5a}

\textbf{Write short note on various Capacitive transducer.}

\begin{solutionbox}
Capacitive transducers operate on the principle that
capacitance changes with physical parameters.

{\def\LTcaptype{none} % do not increment counter
\begin{longtable}[]{@{}
  >{\raggedright\arraybackslash}p{(\linewidth - 4\tabcolsep) * \real{0.1538}}
  >{\raggedright\arraybackslash}p{(\linewidth - 4\tabcolsep) * \real{0.4872}}
  >{\raggedright\arraybackslash}p{(\linewidth - 4\tabcolsep) * \real{0.3590}}@{}}
\toprule\noalign{}
\begin{minipage}[b]{\linewidth}\raggedright
Type
\end{minipage} & \begin{minipage}[b]{\linewidth}\raggedright
Working Principle
\end{minipage} & \begin{minipage}[b]{\linewidth}\raggedright
Applications
\end{minipage} \\
\midrule\noalign{}
\endhead
\bottomrule\noalign{}
\endlastfoot
\textbf{Displacement} & Gap between plates changes & Precision
measurement \\
\textbf{Pressure} & Diaphragm deflection changes gap & Pressure
sensors \\
\textbf{Level} & Dielectric changes with medium & Liquid level
measurement \\
\textbf{Humidity} & Dielectric changes with moisture & Humidity
sensors \\
\end{longtable}
}

\textbf{Diagram:}

\begin{verbatim}
    +{-{-}{-}{-}{-}{-}{-}{-}{-}{-}{-}{-}+}
    |   Fixed    |
    |   Plate    |
    +{-{-}{-}{-}{-}{-}{-}{-}{-}{-}{-}{-}+}
           \^{}
           | Gap (d)
           v
    +{-{-}{-}{-}{-}{-}{-}{-}{-}{-}{-}{-}+}
    |  Movable   |
    |   Plate    |
    +{-{-}{-}{-}{-}{-}{-}{-}{-}{-}{-}{-}+}
\end{verbatim}

\begin{itemize}
\tightlist
\item
  \textbf{Capacitance}: C = εA/d (ε: permittivity, A: area, d: distance)
\item
  \textbf{Advantages}: High sensitivity, no physical contact needed
\item
  \textbf{Limitations}: Affected by stray capacitance
\end{itemize}

\end{solutionbox}
\begin{mnemonicbox}
``PALD: Parameter Alters the Leading Dielectric''

\end{mnemonicbox}
\subsection*{Question 5(b) [4 marks]}\label{q5b}

\textbf{Explain LVDT Transducer.}

\begin{solutionbox}
LVDT (Linear Variable Differential Transformer)
measures linear displacement.

\textbf{Diagram:}

\begin{verbatim}
    Primary   Secondary 1   Secondary 2
      Coil        Coil         Coil
     +{-{-}{-}+        +{-}{-}{-}+       +{-}{-}{-}+}
     |   |        |   |       |   |
     |   |        |   |       |   |
     +{-{-}{-}+        +{-}{-}{-}+       +{-}{-}{-}+}
       |            |           |
       |            |           |
    +{-{-}+{-}{-}{-}{-}{-}{-}{-}{-}{-}{-}{-}{-}+{-}{-}{-}{-}{-}{-}{-}{-}{-}{-}{-}+{-}{-}+}
    |     Ferromagnetic Core       |
    +{-{-}{-}{-}{-}{-}{-}{-}{-}{-}{-}{-}{-}{-}{-}{-}{-}{-}{-}{-}{-}{-}{-}{-}{-}{-}{-}{-}{-}{-}+}
\end{verbatim}

\textbf{Working Principle:}

\begin{itemize}
\tightlist
\item
  Primary coil excited by AC voltage
\item
  Core position determines coupling to secondaries
\item
  Output voltage proportional to core displacement
\item
  Null position when core centered (output = 0)
\end{itemize}

\textbf{Characteristics:}

\begin{itemize}
\tightlist
\item
  \textbf{Range}: Typically \pm0.5mm to \pm25cm
\item
  \textbf{Linearity}: Excellent around null position
\item
  \textbf{Sensitivity}: High, typically mV/mm
\item
  \textbf{Resolution}: Nearly infinite (analog device)
\end{itemize}

\end{solutionbox}
\begin{mnemonicbox}
``MDVN: Movement Determines Voltage from Null''

\end{mnemonicbox}
\subsection*{Question 5(c) [7 marks]}\label{q5c}

\textbf{Draw and explain Harmonics Distortion Analyzer.}

\begin{solutionbox}
Harmonic Distortion Analyzer measures distortion in
audio and electronic signals.

\textbf{Block Diagram:}

\begin{center}
\textbf{Mermaid Diagram (Code)}
\begin{verbatim}
{Shaded}
{Highlighting}[]
graph LR
    A[Input Signal] {-{-}{} B[Attenuator]}
    B {-{-}{} C[Input Amplifier]}
    C {-{-}{} D[Fundamental Notch Filter]}
    D {-{-}{} E[Residual Amplifier]}
    E {-{-}{} F[RMS Detector]}
    F {-{-}{} G[Display]}
    C {-{-}{} H[Reference Level Detector]}
    H {-{-}{} G}
{Highlighting}
{Shaded}
\end{verbatim}
\end{center}

\textbf{Working Principle:}

\begin{enumerate}
\tightlist
\item
  Input signal is conditioned and amplified
\item
  Fundamental frequency is removed using notch filter
\item
  Remaining harmonic content is measured
\item
  Distortion calculated as ratio of harmonics to total signal
\end{enumerate}

\textbf{Characteristics:}

\begin{itemize}
\tightlist
\item
  \textbf{Measurement Range}: Typically 0.001\% to 100\%
\item
  \textbf{Frequency Range}: 20Hz to 100kHz
\item
  \textbf{Applications}: Audio equipment testing, power quality analysis
\item
  \textbf{Measurements}: THD (Total Harmonic Distortion), THD+N (THD
  plus Noise)
\end{itemize}

\textbf{Calculation}: THD = \sqrt(V_{2}^{2} + V_{3}^{2} + V_{4}^{2} + \ldots)/(V_{1} + V_{2} + V_{3} +
\ldots)

\begin{itemize}
\tightlist
\item
  Where V_{1} is fundamental, V_{2}, V_{3}, etc. are harmonics
\end{itemize}

\end{solutionbox}
\begin{mnemonicbox}
``FAIR-D: Filter And Isolate Residuals for
Distortion''

\end{mnemonicbox}
\subsection*{Question 5(a) OR [3
marks]}\label{q5a}

\textbf{Explain the working principle of Proximity sensors.}

\begin{solutionbox}
Proximity sensors detect objects without physical
contact.

{\def\LTcaptype{none} % do not increment counter
\begin{longtable}[]{@{}lll@{}}
\toprule\noalign{}
Type & Working Principle & Detection Range \\
\midrule\noalign{}
\endhead
\bottomrule\noalign{}
\endlastfoot
\textbf{Inductive} & Detects metal using electromagnetic field &
0.5-60mm \\
\textbf{Capacitive} & Detects any material by capacitance change &
3-60mm \\
\textbf{Ultrasonic} & Uses sound wave reflection & 1cm-10m \\
\textbf{Photoelectric} & Uses light beam interruption & Up to 50m \\
\end{longtable}
}

\textbf{Diagram:}

\begin{verbatim}
    +{-{-}{-}{-}{-}{-}{-}{-}+         +{-}{-}{-}{-}{-}{-}{-}{-}+}
    | Sensor |  Field  | Object |
    +{-{-}{-}{-}{-}{-}{-}{-}+ {-}{-}{-}{-}{-}{-} +{-}{-}{-}{-}{-}{-}{-}{-}+}
       |  \^{}
       |  |
    +{-{-}+{-}{-}+{-}{-}+}
    |Controller|
    +{-{-}{-}{-}{-}{-}{-}{-}{-}{-}+}
\end{verbatim}

\begin{itemize}
\tightlist
\item
  \textbf{Operating Modes}: Normally open or normally closed
\item
  \textbf{Output Types}: Digital (on/off) or analog (proportional)
\item
  \textbf{Applications}: Manufacturing, automation, security systems
\end{itemize}

\end{solutionbox}
\begin{mnemonicbox}
``CUPS: Capacitive, Ultrasonic, Photoelectric,
Sense''

\end{mnemonicbox}
\subsection*{Question 5(b) OR [4
marks]}\label{q5b}

\textbf{Explain absolute and incremental type of Optical encoder.}

\begin{solutionbox}
Optical encoders convert mechanical position to digital
signals using light detection.

{\def\LTcaptype{none} % do not increment counter
\begin{longtable}[]{@{}
  >{\raggedright\arraybackslash}p{(\linewidth - 4\tabcolsep) * \real{0.2200}}
  >{\raggedright\arraybackslash}p{(\linewidth - 4\tabcolsep) * \real{0.3600}}
  >{\raggedright\arraybackslash}p{(\linewidth - 4\tabcolsep) * \real{0.4200}}@{}}
\toprule\noalign{}
\begin{minipage}[b]{\linewidth}\raggedright
Parameter
\end{minipage} & \begin{minipage}[b]{\linewidth}\raggedright
Absolute Encoder
\end{minipage} & \begin{minipage}[b]{\linewidth}\raggedright
Incremental Encoder
\end{minipage} \\
\midrule\noalign{}
\endhead
\bottomrule\noalign{}
\endlastfoot
\textbf{Output Format} & Complete position code & Pulse train \\
\textbf{Resolution} & Fixed by number of tracks & Determined by disk
divisions \\
\textbf{Position Knowledge} & Maintained after power loss & Lost after
power loss \\
\textbf{Complexity} & Higher (multiple tracks) & Lower (single track) \\
\textbf{Cost} & Higher & Lower \\
\end{longtable}
}

\textbf{Diagram of Absolute Encoder:}

\begin{verbatim}
    +{-{-}{-}{-}{-}{-}{-}{-}{-}{-}{-}{-}{-}+}
    |  1 0 1 0 1  | {{-} Code Tracks}
    |  1 1 0 0 1  |
    |  0 0 1 1 1  |
    +{-{-}{-}{-}{-}{-}{-}{-}{-}{-}{-}{-}{-}+}
           |
    +{-{-}{-}{-}{-}{-}+{-}{-}{-}{-}{-}{-}{-}+}
    | Light Source |
    +{-{-}{-}{-}{-}{-}+{-}{-}{-}{-}{-}{-}{-}+}
           |
    +{-{-}{-}{-}{-}{-}+{-}{-}{-}{-}{-}{-}{-}+}
    |   Detectors  |
    +{-{-}{-}{-}{-}{-}{-}{-}{-}{-}{-}{-}{-}{-}+}
\end{verbatim}

\textbf{Diagram of Incremental Encoder:}

\begin{verbatim}
    +{-{-}{-}{-}{-}{-}{-}{-}{-}{-}{-}+}
    |           |
    |  //////   | {{-} Single Track with slots}
    |           |
    +{-{-}{-}{-}{-}{-}{-}{-}{-}{-}{-}+}
           |
    +{-{-}{-}{-}{-}{-}+{-}{-}{-}{-}{-}{-}{-}+}
    | Light Source |
    +{-{-}{-}{-}{-}{-}+{-}{-}{-}{-}{-}{-}{-}+}
           |
    +{-{-}{-}{-}{-}{-}+{-}{-}{-}{-}{-}{-}{-}+}
    |   Detectors  |
    +{-{-}{-}{-}{-}{-}{-}{-}{-}{-}{-}{-}{-}{-}+}
\end{verbatim}

\begin{itemize}
\tightlist
\item
  \textbf{A, B, Z Outputs}:

  \begin{itemize}
  \tightlist
  \item
    A and B outputs are 90^\circ out of phase for direction detection
  \item
    Z (index) pulse once per revolution for reference
  \end{itemize}
\end{itemize}

\end{solutionbox}
\begin{mnemonicbox}
``APIR-CD: Absolute Provides Immediate Reading,
Counter Determines incremental''

\end{mnemonicbox}
\subsection*{Question 5(c) OR [7
marks]}\label{q5c}

\textbf{Write short note on Digital IC Tester.}

\begin{solutionbox}
Digital IC Tester is used to verify functionality and
detect faults in digital integrated circuits.

\textbf{Block Diagram:}

\begin{center}
\textbf{Mermaid Diagram (Code)}
\begin{verbatim}
{Shaded}
{Highlighting}[]
graph LR
    A[Test Pattern Generator] {-{-}{} B[IC Socket]}
    C[IC Under Test] {-{-}{} B}
    B {-{-}{} D[Response Analyzer]}
    D {-{-}{} E[Result Display]}
    F[Microcontroller] {-{-}{} A}
    F {-{-}{} D}
    F {-{-}{} E}
    G[User Interface] {-{-}{} F}
    H[Power Supply] {-{-}{} B}
{Highlighting}
{Shaded}
\end{verbatim}
\end{center}

{\def\LTcaptype{none} % do not increment counter
\begin{longtable}[]{@{}ll@{}}
\toprule\noalign{}
Component & Function \\
\midrule\noalign{}
\endhead
\bottomrule\noalign{}
\endlastfoot
\textbf{Test Pattern Generator} & Creates input test signals \\
\textbf{IC Socket} & Holds the IC under test \\
\textbf{Response Analyzer} & Compares actual vs.~expected outputs \\
\textbf{Display} & Shows test results and IC status \\
\textbf{Microcontroller} & Controls test sequence \\
\end{longtable}
}

\textbf{Testing Methods:}

\begin{enumerate}
\tightlist
\item
  \textbf{Functional Testing}: Verifies logic functionality
\item
  \textbf{Parametric Testing}: Measures electrical parameters
\item
  \textbf{Fault Detection}: Identifies shorts, opens, stuck bits
\end{enumerate}

\textbf{Types of IC Testers:}

\begin{itemize}
\tightlist
\item
  \textbf{Universal Testers}: Test multiple IC families (TTL, CMOS)
\item
  \textbf{Dedicated Testers}: Designed for specific IC types
\item
  \textbf{In-Circuit Testers}: Test ICs while in the circuit
\end{itemize}

\textbf{Capabilities:}

\begin{itemize}
\tightlist
\item
  \textbf{IC Identification}: Recognizes unknown ICs
\item
  \textbf{Fault Diagnosis}: Identifies specific faults
\item
  \textbf{Auto Test}: Performs comprehensive testing sequence
\end{itemize}

\end{solutionbox}
\begin{mnemonicbox}
``GATES: Generate And Test Every Signal''

\end{mnemonicbox}
\subsection*{Question 5(c) (Additional) [7
marks]}\label{q5c}

\textbf{Below are the solved solutions for remaining questions present
in the question paper:}

\textbf{Explain working of electronic multimeter.}

\begin{solutionbox}
Electronic multimeter uses electronic components to
measure various electrical parameters.

\textbf{Block Diagram:}

\begin{center}
\textbf{Mermaid Diagram (Code)}
\begin{verbatim}
{Shaded}
{Highlighting}[]
graph LR
    A[Input] {-{-}{} B[Range Selection]}
    B {-{-}{} C[Signal Conditioning]}
    C {-{-}{} D[ADC]}
    D {-{-}{} E[Display]}
    F[Power Supply] {-{-}{} C}
    F {-{-}{} D}
    F {-{-}{} E}
{Highlighting}
{Shaded}
\end{verbatim}
\end{center}

{\def\LTcaptype{none} % do not increment counter
\begin{longtable}[]{@{}
  >{\raggedright\arraybackslash}p{(\linewidth - 4\tabcolsep) * \real{0.2564}}
  >{\raggedright\arraybackslash}p{(\linewidth - 4\tabcolsep) * \real{0.4872}}
  >{\raggedright\arraybackslash}p{(\linewidth - 4\tabcolsep) * \real{0.2564}}@{}}
\toprule\noalign{}
\begin{minipage}[b]{\linewidth}\raggedright
Function
\end{minipage} & \begin{minipage}[b]{\linewidth}\raggedright
Circuit Components
\end{minipage} & \begin{minipage}[b]{\linewidth}\raggedright
Features
\end{minipage} \\
\midrule\noalign{}
\endhead
\bottomrule\noalign{}
\endlastfoot
\textbf{Voltage Measurement} & Input attenuator, amplifier & High
impedance input \\
\textbf{Current Measurement} & Shunt resistor, amplifier & Low insertion
loss \\
\textbf{Resistance Measurement} & Constant current source & Auto-ranging
capability \\
\textbf{Display} & LCD or LED with drivers & Digital readout \\
\end{longtable}
}

\begin{itemize}
\tightlist
\item
  \textbf{Advantages}: High input impedance, auto-ranging, digital
  accuracy
\item
  \textbf{Applications}: Electronics troubleshooting, circuit testing,
  device calibration
\end{itemize}

\end{solutionbox}
\begin{mnemonicbox}
``MAAD: Measure, Amplify, Analyze, Display''

\textbf{Explain working of Moving Iron type instruments.}

\end{mnemonicbox}
\begin{solutionbox}
Moving Iron instruments operate based on magnetic force
between current-carrying coil and iron piece.

{\def\LTcaptype{none} % do not increment counter
\begin{longtable}[]{@{}
  >{\raggedright\arraybackslash}p{(\linewidth - 4\tabcolsep) * \real{0.1765}}
  >{\raggedright\arraybackslash}p{(\linewidth - 4\tabcolsep) * \real{0.3235}}
  >{\raggedright\arraybackslash}p{(\linewidth - 4\tabcolsep) * \real{0.5000}}@{}}
\toprule\noalign{}
\begin{minipage}[b]{\linewidth}\raggedright
Type
\end{minipage} & \begin{minipage}[b]{\linewidth}\raggedright
Operation
\end{minipage} & \begin{minipage}[b]{\linewidth}\raggedright
Characteristics
\end{minipage} \\
\midrule\noalign{}
\endhead
\bottomrule\noalign{}
\endlastfoot
\textbf{Attraction Type} & Iron piece attracted to coil & Simple
construction \\
\textbf{Repulsion Type} & Two iron pieces repel & Better accuracy \\
\end{longtable}
}

\textbf{Diagram:}

\begin{verbatim}
                  Pivot
                    |
    +{-{-}{-}{-}{-}+       +{-}+{-}+}
    |     |       | \^{ |}
    |Coil |       | | | Iron
    |     |       | | | Vane
    +{-{-}{-}{-}{-}+       +{-}+{-}+}
                    |
                    v
                  Pointer
\end{verbatim}

\textbf{Characteristics:}

\begin{itemize}
\tightlist
\item
  \textbf{Scale}: Non-linear, compressed at lower end
\item
  \textbf{Response}: Measures both AC and DC (responds to RMS value)
\item
  \textbf{Accuracy}: Lower than PMMC type
\item
  \textbf{Power Consumption}: Relatively high
\end{itemize}

\end{solutionbox}
\begin{mnemonicbox}
``AMIR: Attraction Moves Iron with Reluctance''

\textbf{Explain Humidity Sensor Hygrometer.}

\end{mnemonicbox}
\begin{solutionbox}
Humidity sensors measure the amount of water vapor in
air or other gases.

\textbf{Types of Humidity Sensors:}

{\def\LTcaptype{none} % do not increment counter
\begin{longtable}[]{@{}
  >{\raggedright\arraybackslash}p{(\linewidth - 4\tabcolsep) * \real{0.1429}}
  >{\raggedright\arraybackslash}p{(\linewidth - 4\tabcolsep) * \real{0.4524}}
  >{\raggedright\arraybackslash}p{(\linewidth - 4\tabcolsep) * \real{0.4048}}@{}}
\toprule\noalign{}
\begin{minipage}[b]{\linewidth}\raggedright
Type
\end{minipage} & \begin{minipage}[b]{\linewidth}\raggedright
Working Principle
\end{minipage} & \begin{minipage}[b]{\linewidth}\raggedright
Characteristics
\end{minipage} \\
\midrule\noalign{}
\endhead
\bottomrule\noalign{}
\endlastfoot
\textbf{Capacitive} & Humidity changes dielectric constant & Wide range,
good accuracy \\
\textbf{Resistive} & Humidity changes resistance & Simple,
cost-effective \\
\textbf{Thermal} & Humidity affects thermal conductivity & Good for high
temperatures \\
\end{longtable}
}

\textbf{Diagram:}

\begin{verbatim}
    +{-{-}{-}{-}{-}{-}{-}{-}{-}{-}+}
    | Humidity | 
    | Sensing  |{-{-}+}
    | Element  |  |
    +{-{-}{-}{-}{-}{-}{-}{-}{-}{-}+  |}
                  |
    +{-{-}{-}{-}{-}{-}{-}{-}{-}{-}+  |}
    | Signal   |{{-}+}
    | Circuit  |{-{-}+}
    +{-{-}{-}{-}{-}{-}{-}{-}{-}{-}+  |}
                  |
    +{-{-}{-}{-}{-}{-}{-}{-}{-}{-}+  |}
    | Display/ |{{-}+}
    | Output   |
    +{-{-}{-}{-}{-}{-}{-}{-}{-}{-}+}
\end{verbatim}

\textbf{Measurements:}

\begin{itemize}
\tightlist
\item
  \textbf{Relative Humidity (RH)}: Percentage of actual to maximum
  moisture
\item
  \textbf{Dew Point}: Temperature at which condensation occurs
\item
  \textbf{Absolute Humidity}: Mass of water vapor per volume
\end{itemize}

\textbf{Applications:}

\begin{itemize}
\tightlist
\item
  Weather stations
\item
  HVAC systems
\item
  Industrial process control
\item
  Medical equipment
\end{itemize}

\end{solutionbox}
\begin{mnemonicbox}
``CRAP-H: Capacitance or Resistance Alters with
Presence of Humidity''

\textbf{Draw and explain Piezoelectric transducer.}

\end{mnemonicbox}
\begin{solutionbox}
Piezoelectric transducers convert mechanical force to
electrical signal and vice versa.

\textbf{Diagram:}

\begin{verbatim}
           Force
             |
             v
    +{-{-}{-}{-}{-}{-}{-}{-}{-}{-}{-}{-}{-}{-}{-}{-}{-}{-}{-}{-}+}
    |      Metal         |
    |    Electrodes      |
    | +{-{-}{-}{-}{-}{-}{-}{-}{-}{-}{-}{-}{-}{-}{-}{-}+ |}
    | |                | |
    | | Piezoelectric  | |
    | |   Crystal      | |
    | |                | |
    | +{-{-}{-}{-}{-}{-}{-}{-}{-}{-}{-}{-}{-}{-}{-}{-}+ |}
    |      Metal         |
    |    Electrodes      |
    +{-{-}{-}{-}{-}{-}{-}{-}{-}{-}{-}{-}{-}{-}{-}{-}{-}{-}{-}{-}+}
          |      |
          +      {-}
       Electrical Output
\end{verbatim}

\textbf{Working Principle:}

\begin{itemize}
\tightlist
\item
  \textbf{Direct Effect}: Pressure generates electric charge
\item
  \textbf{Reverse Effect}: Electric field causes mechanical deformation
\item
  \textbf{Materials}: Quartz, PZT, barium titanate, lithium niobate
\end{itemize}

\textbf{Characteristics:}

\begin{itemize}
\tightlist
\item
  \textbf{High Frequency Response}: Up to MHz range
\item
  \textbf{High Output Impedance}: Requires charge amplifier
\item
  \textbf{Self-Generating}: No external power for sensing
\item
  \textbf{Dynamic Response}: Not suitable for static measurements
\end{itemize}

\textbf{Applications:}

\begin{itemize}
\tightlist
\item
  Accelerometers
\item
  Pressure sensors
\item
  Ultrasonic transducers
\item
  Microphones
\item
  Ignition systems
\end{itemize}

\end{solutionbox}
\begin{mnemonicbox}
``PEMS: Pressure Equals Measurable Signal''

\end{mnemonicbox}

\end{document}
