\documentclass{article}

% content/resources/templates/preamble.tex
\usepackage[margin=0.6in]{geometry}
\author{Milav Dabgar}
\usepackage{amsmath,amssymb,amsthm}
\usepackage{booktabs}
\usepackage{multirow}
\usepackage{xcolor}
\usepackage{tcolorbox}
\tcbuselibrary{breakable,skins}
\usepackage[colorlinks=true,linkcolor=blue]{hyperref}
\usepackage{titlesec}
\usepackage{enumitem}
\usepackage{tikz}
\usepackage{pgfplots}
\usepackage{circuitikz}
\usepackage[version=4]{mhchem}
\usepackage{longtable}
\usepackage{array}
\usepackage{float}
\usepackage{caption}
\usepackage{listings}

\lstset{
  basicstyle=\small\ttfamily,
  breaklines=true,
  breakatwhitespace=false,
  postbreak=\mbox{\textcolor{red}{$\hookrightarrow$}\space},
  float=false,
  numbers=left,
  numberstyle=\tiny\color{gray},
  numbersep=10pt,
  xleftmargin=2em,
  keywordstyle=\color{blue},
  commentstyle=\color{green!60!black},
  stringstyle=\color{purple},
  backgroundcolor=\color{gray!5},
  showstringspaces=false,
  tabsize=2,
  captionpos=b,
  keepspaces=true,
  columns=flexible
}

\pgfplotsset{compat=1.18}
\usetikzlibrary{shapes,arrows,positioning,calc,patterns,decorations.pathmorphing,decorations.markings,arrows.meta}

% Color scheme
\definecolor{headcolor}{RGB}{0,102,204}
\definecolor{keycolor}{RGB}{220,20,60}
\definecolor{solutioncolor}{RGB}{34,139,34}
\definecolor{mnemoniccolor}{RGB}{148,0,211}
\definecolor{codecolor}{RGB}{0,0,100}

% Spacing
\setlength{\parskip}{3pt}
\setlist[itemize]{nosep}
\setlist[enumerate]{nosep}

% Title formatting
\titleformat{\section}{\Large\bfseries\color{headcolor}}{\thesection}{1em}{}
\titleformat{\subsection}{\large\bfseries\color{headcolor}}{\thesubsection}{1em}{}

% Pandoc tightlist compatibility
\providecommand{\tightlist}{%
  \setlength{\itemsep}{0pt}\setlength{\parskip}{0pt}}

% Pandoc longtable compatibility
\newcounter{none}
\def\thenone{}


% content/resources/templates/gujarati-boxes.tex
\usepackage{fontspec}
\usepackage{polyglossia}

% Set Gujarati as main language (document is primarily in Gujarati)
% Note: gloss-gujarati.ldf doesn't exist in polyglossia, but it will use hyphenation patterns
\setdefaultlanguage{gujarati}
\setotherlanguage{english}

% Configure Gujarati font properly
% Use Language=Default to prevent polyglossia from trying to add language-specific features
% that don't exist for Gujarati, which causes "empty feature" warnings
\newfontfamily\gujaratifont[Script=Gujarati,AutoFakeBold=2.5,AutoFakeSlant=0.3]{Noto Sans Gujarati}
\setmainfont[Script=Gujarati,AutoFakeBold=2.5,AutoFakeSlant=0.3]{Noto Sans Gujarati}
% Use Noto Sans Gujarati for monospace to support Gujarati in text
\setmonofont[Scale=0.9]{Noto Sans Gujarati}

% Configure English to use the same font
\newfontfamily\englishfont[Script=Gujarati,AutoFakeBold=2.5,AutoFakeSlant=0.3]{Noto Sans Gujarati}

% Translations for polyglossia
\gappto\captionsgujarati{
  \renewcommand{\tablename}{કોષ્ટક}
  \renewcommand{\figurename}{આકૃતિ}
}

% Helper for TikZ nodes to ensure Gujarati font
\newcommand{\gu}[1]{{\gujaratifont #1}}

% Custom environments
\newtcolorbox{solutionbox}{
    breakable,
    enhanced,
    colback=solutioncolor!5!white,
    colframe=solutioncolor!75!black,
    fonttitle=\bfseries,
    title=જવાબ
}

\newtcolorbox{solutionboxnobreak}{
 colback=solutioncolor!5!white,
 colframe=solutioncolor!75!black,
 fonttitle=\bfseries,
 title=જવાબ
}

\newtcolorbox{keyformula}{
 breakable,
 enhanced,
 colback=keycolor!5!white,
 colframe=keycolor!75!black,
 fonttitle=\bfseries,
 title=રાસાયણિક સમીકરણ/સૂત્ર
}

\newtcolorbox{mnemonicbox}{
 breakable,
 enhanced,
 colback=mnemoniccolor!5!white,
 colframe=mnemoniccolor!75!black,
 fonttitle=\bfseries,
 title=મેમરી ટ્રીક
}


% Custom commands for GTU solutions
% This file defines semantic commands for consistent formatting

% Question command with automatic formatting
\newcommand{\question}[2]{%
  \section*{Question #1}%
  \textbf{#2}%
}

% OR question variant
\newcommand{\questionor}[2]{%
  \section*{Question #1 OR}%
  \textbf{#2}%
}

% Proper table environment with caption
\newenvironment{answertable}[1]{%
  \begin{table}[htbp]
  \centering
  \caption{#1}
}{%
  \end{table}
}

% Proper figure environment for diagrams
\newenvironment{answerdiagram}[1]{%
  \begin{figure}[htbp]
  \centering
  \caption{#1}
}{%
  \end{figure}
}

% Semantic markup for key terms
\newcommand{\keyword}[1]{\textbf{#1}}
\newcommand{\code}[1]{\texttt{#1}}
\newcommand{\classname}[1]{\texttt{#1}}
\newcommand{\methodname}[1]{\texttt{#1}}

% Proper quotation marks
\newcommand{\mnemonic}[1]{``#1''}


\title{ઇલેક્ટ્રોનિક મેઝરમેન્ટ્સ એન્ડ ઇન્સ્ટ્રુમેન્ટ્સ (4331102) - વિન્ટર 2022 સોલ્યુશન}
\date{ફેબ્રુઆરી 27, 2023}

\begin{document}
\maketitle

\questionmarks{1(a)}{3}{મૂળભૂત Q-મીટરની કામગીરી દોરો અને સમજાવો.}

\begin{solutionbox}
\textbf{Q-મીટર} એ સાધન છે જે ઇન્ડક્ટર અથવા કેપેસિટરના ક્વોલિટી ફેક્ટર (Q)ને માપે છે.

\textbf{બ્લોક ડાયાગ્રામ}:
\begin{center}
\begin{tikzpicture}[node distance=1.5cm, auto]
    \node [gtu block] (Osc) {ઓસિલેટર};
    \node [gtu block, right=of Osc] (Amp) {એમ્પ્લિફાયર};
    \node [gtu block, right=of Amp] (Res) {રેઝોનન્સ સર્કિટ};
    \node [gtu block, right=of Res] (Volt) {વોલ્ટેજ ઇન્ડિકેટર};
    \node [gtu block, below=of Res] (DUT) {DUT};
    
    \draw [gtu arrow] (Osc) -- (Amp);
    \draw [gtu arrow] (Amp) -- (Res);
    \draw [gtu arrow] (Res) -- (Volt);
    \draw [gtu arrow] (Res) -- (DUT);
    \draw [gtu arrow] (DUT) -- (Res);
\end{tikzpicture}
\captionof{figure}{મૂળભૂત Q-મીટર બ્લોક ડાયાગ્રામ}
\end{center}

\textbf{કારક્યો}:
\begin{itemize}
    \item \keyword{ઓસિલેટર}: ચલિત આવૃત્તિનું સિગ્નલ ઉત્પન્ન કરે છે.
    \item \keyword{એમ્પ્લિફાયર}: સિગ્નલને જરૂરી સ્તર સુધી વધારે છે.
    \item \keyword{રેઝોનન્સ સર્કિટ}: પરીક્ષણ હેઠળના ઘટકને ધરાવે છે.
    \item \keyword{વોલ્ટેજ ઇન્ડિકેટર}: ઘટક પર વોલ્ટેજ માપે છે.
\end{itemize}
\end{solutionbox}

\begin{mnemonicbox}
\mnemonic{OARV - Oscillate, Amplify, Resonate, View}
\end{mnemonicbox}

\questionmarks{1(b)}{4}{સ્પેક્ટ્રમ એનાલાઇઝર ટૂંકમાં સમજાવો.}

\begin{solutionbox}
\textbf{સ્પેક્ટ્રમ એનાલાઇઝર} એ સાધનની સંપૂર્ણ આવૃત્તિ શ્રેણીની અંદર ઇનપુટ સિગ્નલના મેગ્નિટ્યુડને આવૃત્તિની સામે માપે છે.

\textbf{બ્લોક ડાયાગ્રામ}:
\begin{center}
\begin{tikzpicture}[node distance=1.5cm, auto]
    \node [gtu block] (Mix) {મિક્સર};
    \node [gtu block, left=of Mix] (In) {ઇનપુટ};
    \node [gtu block, below=of Mix] (LO) {લોકલ ઓસિલેટર};
    \node [gtu block, right=of Mix] (IF) {IF ફિલ્ટર};
    \node [gtu block, right=of IF] (Det) {ડિટેક્ટર};
    \node [gtu block, right=of Det] (Disp) {ડિસ્પ્લે};
    
    \draw [gtu arrow] (In) -- (Mix);
    \draw [gtu arrow] (LO) -- (Mix);
    \draw [gtu arrow] (Mix) -- (IF);
    \draw [gtu arrow] (IF) -- (Det);
    \draw [gtu arrow] (Det) -- (Disp);
\end{tikzpicture}
\captionof{figure}{સ્પેક્ટ્રમ એનાલાઇઝર બ્લોક ડાયાગ્રામ}
\end{center}

\textbf{મુખ્ય પાસાઓ}:
\begin{itemize}
    \item \keyword{ઇનપુટ સિગ્નલ પ્રોસેસિંગ}: સિગ્નલ એટેન્યુએટર અને ફિલ્ટર દ્વારા પ્રવેશે છે.
    \item \keyword{ફ્રિક્વન્સી ડોમેન કન્વર્ઝન}: ટાઇમ ડોમેનને ફ્રિક્વન્સી ડોમેનમાં રૂપાંતરિત કરે છે.
    \item \keyword{ડિસ્પ્લે સિસ્ટમ}: એમ્પ્લિટ્યુડ વિરુદ્ધ આવૃત્તિ પ્લોટ બતાવે છે.
    \item \keyword{એપ્લિકેશન}: સિગ્નલ એનાલિસિસ, ડિસ્ટોર્શન મેઝરમેન્ટ, EMI ટેસ્ટિંગ.
\end{itemize}
\end{solutionbox}

\begin{mnemonicbox}
\mnemonic{SAME-FD: Signal Analysis Measures Everything in Frequency Domain}
\end{mnemonicbox}

\questionmarks{1(c)}{7}{સર્કિટ ડાયાગ્રામ વડે વ્હીટસ્ટોન બ્રિજ સમજાવો. તેના ફાયદા અને ગેરફાયદાની યાદી આપો.}

\begin{solutionbox}
\textbf{વ્હીટસ્ટોન બ્રિજ} એ અજ્ઞાત રેસિસ્ટન્સને ઉચ્ચ ચોકસાઈથી માપવા માટે વપરાય છે.

\textbf{સર્કિટ ડાયાગ્રામ}:
\begin{center}
\begin{circuitikz}[american, scale=0.8]
    \draw (0,3) node[left] {A} to[R, l=$R_1$] (3,5) node[above] {B} to[R, l=$R_3$] (6,3) node[right] {C};
    \draw (6,3) to[R, l=$R_x$] (3,1) node[below] {D} to[R, l=$R_2$] (0,3);
    \draw (3,5) to[rmeter, t=G] (3,1);
    \draw (0,3) -- (-0.5,3) -- (-0.5, 0.5) to[battery1] (6.5, 0.5) -- (6.5, 3) -- (6,3);
\end{circuitikz}
\captionof{figure}{વ્હીટસ્ટોન બ્રિજ}
\end{center}

જ્યાં:
\begin{itemize}
    \item $R_1, R_2, R_3$ એ જાણીતા રેસિસ્ટન્સ છે.
    \item $R_x$ અજ્ઞાત રેસિસ્ટન્સ છે.
    \item $G$ ગેલ્વેનોમીટર છે.
\end{itemize}

\textbf{કાર્ય સિદ્ધાંત}:
\begin{itemize}
    \item બ્રિજ સંતુલિત થાય છે જ્યારે B અને D પર પોટેન્શિયલ સમાન હોય અને G માંથી કોઈ પ્રવાહ વહેતો નથી.
    \item \keyword{સંતુલન શરત}: $\frac{R_1}{R_2} = \frac{R_3}{R_x}$
    \item \keyword{અજ્ઞાત રેસિસ્ટન્સ}: $R_x = R_3 (\frac{R_2}{R_1})$
\end{itemize}

\begin{center}
\captionof{table}{ફાયદા અને ગેરફાયદા}
\begin{tabulary}{\linewidth}{|L|L|}
\hline
\textbf{ફાયદા} & \textbf{ગેરફાયદા} \\ \hline
ઉચ્ચ ચોકસાઈ & મર્યાદિત શ્રેણી \\ \hline
સારી સંવેદનશીલતા & તાપમાન અસરો \\ \hline
નલ પ્રકારનું માપન & સંતુલન સમાયોજન જરૂરી \\ \hline
કેલિબ્રેટેડ મીટરની જરૂર નથી & ખૂબ ઓછા/ઉચ્ચ રેસિસ્ટન્સ માટે યોગ્ય નથી \\ \hline
\end{tabulary}
\end{center}
\end{solutionbox}

\begin{mnemonicbox}
\mnemonic{BARN - Balance Achieved when Ratios are Null}
\end{mnemonicbox}

\questionmarks{1(c) OR}{7}{સાધનને વ્યાખ્યાયિત કરો અને તેની લાક્ષણિકતાઓ સમજાવો.}

\begin{solutionbox}
\textbf{સાધન} એ એક ઉપકરણ છે જે ભૌતિક જથ્થાઓને માપવા, પ્રદર્શિત કરવા અથવા રેકોર્ડ કરવા માટે વપરાય છે.

\textbf{સાધન ડાયાગ્રામ}:
\begin{center}
\begin{tikzpicture}[node distance=1.5cm, auto]
    \node [gtu block] (Inst) {સાધન};
    \node [coordinate, left=of Inst] (In) {};
    \node [coordinate, right=of Inst] (Out) {};
    \node [gtu block, above=of Inst] (Env) {પર્યાવરણીય};
    \node [gtu block, below=of Inst] (Err) {ત્રુટિ};
    
    \draw [gtu arrow] (In) -- node[above]{ઇનપુટ} (Inst);
    \draw [gtu arrow] (Inst) -- node[above]{રીડિંગ} (Out);
    \draw [gtu arrow] (Env) -- (Inst);
    \draw [gtu arrow] (Err) -- (Inst);
\end{tikzpicture}
\captionof{figure}{સામાન્ય સાધન સિસ્ટમ}
\end{center}

\begin{center}
\captionof{table}{સાધનની લાક્ષણિકતાઓ}
\begin{tabulary}{\linewidth}{|L|L|}
\hline
\textbf{લાક્ષણિકતા} & \textbf{વર્ણન} \\ \hline
\keyword{ચોકસાઈ} & માપનની સાચા મૂલ્ય સાથેની નિકટતા \\ \hline
\keyword{પ્રિસિઝન} & માપણીની પુનરાવર્તિતા \\ \hline
\keyword{રિઝોલ્યુશન} & નાનામાં નાનો ફેરફાર જે શોધી શકાય છે \\ \hline
\keyword{સંવેદનશીલતા} & ઇનપુટ સિગ્નલ ફેરફારમાં આઉટપુટ સિગ્નલનો ગુણોત્તર \\ \hline
\keyword{લિનિયરતા} & ઇનપુટ અને આઉટપુટ વચ્ચે પ્રમાણસર સંબંધ \\ \hline
\keyword{રેન્જ} & લઘુત્તમથી મહત્તમ માપી શકાય તેવા મૂલ્યો \\ \hline
\keyword{પ્રતિસાદ સમય} & સાચું વાચન બતાવવા માટે જરૂરી સમય \\ \hline
\end{tabulary}
\end{center}
\end{solutionbox}

\begin{mnemonicbox}
\mnemonic{APRS-LRR: Accuracy and Precision, Resolution and Sensitivity, Linearity, Range, Response time}
\end{mnemonicbox}

\questionmarks{2(a)}{3}{એનર્જી મીટરનું બાંધકામ ડાયાગ્રામ દોરો.}

\begin{solutionbox}
\textbf{એનર્જી મીટર} કિલોવોટ-કલાક (kWh) માં વીજળી ઊર્જાનો વપરાશ માપે છે.

\textbf{આકૃતિ}:
\begin{center}
\begin{tikzpicture}
    % Disc
    \draw [fill=gray!20] (0,0) ellipse (2 and 0.5);
    \node at (0,0) {એલ્યુમિનિયમ ડિસ્ક};
    \draw [thick] (0,0) -- (0,3); % Spindle
    \draw (0,3) -- (1,3); \node at (1.5,3) {ડાયલ તરફ};
    
    % Magnets/Coils
    \draw (-1.5, 1) rectangle (-0.5, 2); \node at (-1, 1.5) {કરંટ કોઇલ};
    \draw (0.5, 1) rectangle (1.5, 2); \node at (1, 1.5) {વોલ્ટેજ કોઇલ};
    \draw (-0.5, -1) rectangle (0.5, -0.5); \node at (0, -0.75) {બ્રેક મેગ્નેટ};
    
    % Connections
    \draw [thick] (-2, -2) -- (-1, 1);
    \draw [thick] (2, -2) -- (1, 1);
\end{tikzpicture}
\captionof{figure}{ઇન્ડક્શન ટાઇપ એનર્જી મીટર}
\end{center}

\textbf{ઘટકો}:
\begin{itemize}
    \item \keyword{ફરતી એલ્યુમિનિયમ ડિસ્ક}: પાવરના પ્રમાણમાં ખસે છે.
    \item \keyword{કરંટ કોઇલ}: કરંટના પ્રમાણમાં ચુંબકીય પ્રવાહ બનાવે છે.
    \item \keyword{વોલ્ટેજ કોઇલ}: વોલ્ટેજના પ્રમાણમાં ચુંબકીય પ્રવાહ બનાવે છે.
    \item \keyword{કાયમી ચુંબક}: બ્રેકિંગ ટોર્ક પૂરો પાડે છે.
\end{itemize}
\end{solutionbox}

\begin{mnemonicbox}
\mnemonic{DVCP: Disc Velocity measures Consumed Power}
\end{mnemonicbox}

\questionmarks{2(b)}{4}{ટૂંકમાં PMMC ની કામગીરી સમજાવો.}

\begin{solutionbox}
\textbf{PMMC (પર્મેનન્ટ મેગ્નેટ મૂવિંગ કોઇલ)} એ વિવિધ DC મીટરોમાં વપરાતી મૂળભૂત પદ્ધતિ છે.

\textbf{બાંધકામ આકૃતિ}:
\begin{center}
\begin{tikzpicture}
    % Magnets
    \draw [fill=gray!30] (-2, -1) rectangle (-1, 1); \node at (-1.5, 0) {N};
    \draw [fill=gray!30] (1, -1) rectangle (2, 1); \node at (1.5, 0) {S};
    % Core
    \draw (0,0) circle (0.7); \node at (0,0) {કોર};
    % Coil
    \draw [thick, color=red] (-0.8, -0.8) rectangle (0.8, 0.8);
    % Pointer
    \draw [thick, ->] (0, 0.8) -- (1.5, 2);
    % Scale
    \draw (0.5, 2) arc (60:120:2);
    \foreach \x in {60, 75, 90, 105, 120} \draw (\x:2.1) -- (\x:2.2);
    % Spring
    \draw [decorate, decoration={coil, segment length=3pt, amplitude=2pt}] (0,0) -- (0, 0.8);
\end{tikzpicture}
\captionof{figure}{PMMC બાંધકામ}
\end{center}

\begin{center}
\captionof{table}{મુખ્ય ઘટકો}
\begin{tabulary}{\linewidth}{|L|L|}
\hline
\textbf{ઘટક} & \textbf{કાર્ય} \\ \hline
કાયમી ચુંબક & મજબૂત ચુંબકીય ક્ષેત્ર બનાવે છે \\ \hline
ફરતી કોઇલ & માપવાના કરંટને વહન કરે છે \\ \hline
સ્પ્રિંગ & નિયંત્રિત ટોર્ક પૂરો પાડે છે \\ \hline
પોઇન્ટર & સ્કેલ પર વાચન દર્શાવે છે \\ \hline
\end{tabulary}
\end{center}

\textbf{કામગીરી}: જ્યારે ચુંબકીય ક્ષેત્રમાં મૂકેલી કોઇલમાંથી પ્રવાહ વહે છે, ત્યારે તેના પર બળ લાગે છે, જે ડિફ્લેક્ટિંગ ટોર્ક $T_d \propto I$ ઉત્પન્ન કરે છે.
\end{solutionbox}

\begin{mnemonicbox}
\mnemonic{CODA: Current through cOil causes Deflection by Attraction}
\end{mnemonicbox}

\questionmarks{2(c)}{7}{1- 1 એમ્પીયર સુધીની મૂવિંગ કોઇલ એમીટર રીડિંગ 0.02 ઓહ્મનો પ્રતિકાર ધરાવે છે. 100 એમ્પીયર સુધીનો કરંટ વાંચવા માટે આ સાધન કેવી રીતે અપનાવી શકાય? 2- મૂવિંગ કોઇલ વોલ્ટમીટર 20 mV સુધીનું રીડિંગ 2 ઓહ્મનું પ્રતિકાર ધરાવે છે. 300 વોલ્ટ સુધીના વોલ્ટેજને વાંચવા માટે આ સાધનને કેવી રીતે અપનાવી શકાય?}

\begin{solutionbox}
\textbf{ભાગ 1: એમીટર રેન્જ એક્સટેન્શન}

એમીટર રેન્જ વધારવા માટે, \keyword{શન્ટ રેસિસ્ટન્સ ($R_sh$)} સમાંતરમાં જોડાયેલ છે.

\textbf{આકૃતિ}:
\begin{center}
\begin{circuitikz}[american]
    \draw (0,2) to[short, i=$I$] (1,2) -- (4,2);
    \draw (2,2) to[R, l=$R_{sh}$] (2,0);
    \draw (3,2) to[rmeter, l=$R_m$, t=$A$] (3,0);
    \draw (1,0) -- (4,0);
\end{circuitikz}
\captionof{figure}{શન્ટ સાથે એમીટર}
\end{center}

આપેલ: $I_m = 1A$, $R_m = 0.02\Omega$, $I = 100A$.
સૂત્ર: $R_{sh} = \frac{R_m \cdot I_m}{I - I_m}$
ગણતરી:
$$ R_{sh} = \frac{0.02 \times 1}{100 - 1} = \frac{0.02}{99} \approx 0.000202\Omega $$

\textbf{ભાગ 2: વોલ્ટમીટર રેન્જ એક્સટેન્શન}

વોલ્ટમીટર રેન્જ વધારવા માટે, \keyword{સીરીઝ મલ્ટિપ્લાયર રેસિસ્ટન્સ ($R_s$)} શ્રેણીમાં જોડાયેલ છે.

\textbf{આકૃતિ}:
\begin{center}
\begin{circuitikz}[american]
    \draw (0,0) to[R, l=$R_s$] (2,0) to[rmeter, l=$R_m$, t=$V$] (4,0);
    \draw (0,0) to[open, v=$V$] (4,0);
\end{circuitikz}
\captionof{figure}{મલ્ટિપ્લાયર સાથે વોલ્ટમીટર}
\end{center}

આપેલ: $V_m = 20mV = 0.02V$, $R_m = 2\Omega$, $V = 300V$.
સૂત્ર: $R_s = R_m (\frac{V}{V_m} - 1)$
ગણતરી:
$$ R_s = 2 \times (\frac{300}{0.02} - 1) = 2 \times (15000 - 1) = 2 \times 14999 = 29,998\Omega $$
\end{solutionbox}

\begin{mnemonicbox}
\mnemonic{SHIP: Shunt Has Inverse Proportion for current; Series for voltage}
\end{mnemonicbox}

\questionmarks{2(a) OR}{3}{ઇલેક્ટ્રોનિક મલ્ટિમીટરની કામગીરી સમજાવો.}

\begin{solutionbox}
\textbf{ઇલેક્ટ્રોનિક મલ્ટિમીટર} ઇલેક્ટ્રોનિક ઘટકોનો ઉપયોગ કરીને વોલ્ટેજ, કરંટ અને રેસિસ્ટન્સ માપે છે.

\textbf{બ્લોક ડાયાગ્રામ}:
\begin{center}
\begin{tikzpicture}[node distance=1.5cm, auto]
    \node [gtu block] (Range) {રેન્જ સિલેક્ટ};
    \node [gtu block, left=of Range] (Input) {ઇનપુટ};
    \node [gtu block, right=of Range] (Cond) {સિગ્નલ કન્ડિશનિંગ};
    \node [gtu block, right=of Cond] (ADC) {ADC};
    \node [gtu block, right=of ADC] (Disp) {ડિસ્પ્લે};
    
    \draw [gtu arrow] (Input) -- (Range);
    \draw [gtu arrow] (Range) -- (Cond);
    \draw [gtu arrow] (Cond) -- (ADC);
    \draw [gtu arrow] (ADC) -- (Disp);
\end{tikzpicture}
\captionof{figure}{ઇલેક્ટ્રોનિક મલ્ટિમીટર બ્લોક ડાયાગ્રામ}
\end{center}

\textbf{કામગીરી}:
\begin{itemize}
    \item \keyword{રેન્જ સિલેક્શન}: એટેન્યુએટર/એમ્પ્લિફાયર નેટવર્ક રેન્જ પસંદ કરે છે.
    \item \keyword{સિગ્નલ કન્ડિશનિંગ}: AC ને DC, કરંટ ને વોલ્ટેજ, રેસિસ્ટન્સ ને વોલ્ટેજ માં રૂપાંતરિત કરે છે.
    \item \keyword{ADC}: એનાલોગ ટુ ડિજિટલ કન્વર્ટર સિગ્નલને ડિજિટાઇઝ કરે છે.
    \item \keyword{ડિસ્પ્લે}: LCD/LED આંકડાકીય મૂલ્ય બતાવે છે.
\end{itemize}
\end{solutionbox}

\begin{mnemonicbox}
\mnemonic{RSAD: Range Select, Amplify, Digitize}
\end{mnemonicbox}

\questionmarks{2(b) OR}{4}{મૂવિંગ આયર્ન પ્રકારના સાધનોની કામગીરી સમજાવો.}

\begin{solutionbox}
\textbf{મૂવિંગ આયર્ન ઇન્સ્ટ્રુમેન્ટ્સ} ફિક્સ કોઇલ અને મૂવિંગ આયર્ન પીસ વચ્ચેના ચુંબકીય બળનો ઉપયોગ કરે છે.

\textbf{કાર્ય સિદ્ધાંત}:
\begin{itemize}
    \item ફિક્સ કોઇલમાંથી કરંટ વહે છે, જે ચુંબકીય ક્ષેત્ર બનાવે છે.
    \item ક્ષેત્રમાં મૂકેલ લોખંડનો ટુકડો ચુંબક બને છે.
    \item \keyword{એટ્રેક્શન ટાઇપ}: લોખંડનો બાર કોઇલમાં આકર્ષાય છે.
    \item \keyword{રીપલ્શન ટાઇપ}: સમાન ધ્રુવીયતા સાથે ચુંબકિત બે લોખંડના વેન એકબીજાને પ્રતિકર્ષિત કરે છે.
\end{itemize}

\textbf{આકૃતિ (રીપલ્શન ટાઇપ)}:
\begin{center}
\begin{tikzpicture}
    \draw (0,0) circle (1.5);
    \node at (0,1.8) {કોઇલ સેક્શન};
    
    % Fixed Vane
    \draw [thick, fill=gray] (-0.2, -0.5) rectangle (0.2, 0.5); \node at (0.5, 0) {ફિક્સ};
    % Moving Vane
    \draw [thick, fill=gray, rotate=30] (0.3, -0.5) rectangle (0.7, 0.5); 
    \draw [thick, ->] (0.5, 0.2) -- (1.5, 1.5); \node at (1.8, 1.5) {પોઇન્ટર};
    
\end{tikzpicture}
\captionof{figure}{મૂવિંગ આયર્ન મિકેનિઝમ}
\end{center}

\textbf{વિશેષતાઓ}: નોન-લીનિયર સ્કેલ, AC અને DC માટે વપરાય છે, મજબૂત.
\end{solutionbox}

\begin{mnemonicbox}
\mnemonic{CADS: Current Activates, Deflection Shows}
\end{mnemonicbox}

\questionmarks{2(c) OR}{7}{રેમ્પ પ્રકાર DVM નો બ્લોક ડાયાગ્રામ દોરો. સર્કિટ ડાયાગ્રામ સાથે મલ્ટિરેન્જ DC વોલ્ટમીટર મેળવવાની પ્રક્રિયાને સમજાવો.}

\begin{solutionbox}
\textbf{રેમ્પ પ્રકાર DVM} વોલ્ટેજને સમયમાં રૂપાંતરિત કરે છે ($V \propto t$).

\textbf{બ્લોક ડાયાગ્રામ}:
\begin{center}
\begin{tikzpicture}[node distance=1.5cm, auto]
    \node [gtu block] (Comp) {કમ્પેરેટર};
    \node [gtu block, left=of Comp] (In) {ઇનપુટ};
    \node [gtu block, below=of Comp] (Ramp) {રેમ્પ જનરેટર};
    \node [gtu block, right=of Comp] (Gate) {ગેટ};
    \node [gtu block, right=of Gate] (Count) {કાઉન્ટર};
    \node [gtu block, right=of Count] (Disp) {ડિસ્પ્લે};
    \node [gtu block, below=of Gate] (Osc) {ઓસિલેટર};
    
    \draw [gtu arrow] (In) -- (Comp);
    \draw [gtu arrow] (Ramp) -- (Comp);
    \draw [gtu arrow] (Comp) -- (Gate);
    \draw [gtu arrow] (Osc) -- (Gate);
    \draw [gtu arrow] (Gate) -- (Count);
    \draw [gtu arrow] (Count) -- (Disp);
\end{tikzpicture}
\captionof{figure}{રેમ્પ પ્રકાર DVM}
\end{center}

\textbf{મલ્ટિરેન્જ DC વોલ્ટમીટર સર્કિટ}:
વિવિધ વોલ્ટેજ રેન્જ માપવા માટે, વોલ્ટેજ ડિવાઇડર નેટવર્કનો ઉપયોગ થાય છે.

\begin{center}
\begin{circuitikz}[american]
    \draw (0,4) node[left]{$V_{in}$} to[short, o-] (1,4) -- (1,0); 
    \draw (1,4) to[R, l=$R_1$] (3,4);
    \draw (1,3) to[R, l=$R_2$] (3,3);
    \draw (1,2) to[R, l=$R_3$] (3,2);
    
    \draw (3,4) node[right]{રેન્જ 1};
    \draw (3,3) node[right]{રેન્જ 2};
    \draw (3,2) node[right]{રેન્જ 3};
    
    \draw (4,4) to[short] (4,2);
    \draw (4,3) to[short] (5,3) to[short] (5,1) -- (6,1);
    \node[draw] at (6.5,1) {DVM};
    \draw (0,0) node[left]{COM} to[short, o-] (6,0) -- (6,0.5);
    
    \node at (3.5, 5) {સિલેક્ટર સ્વિચ};
\end{circuitikz}
\captionof{figure}{મલ્ટિરેન્જ એટેન્યુએટર}
\end{center}
\end{solutionbox}

\begin{mnemonicbox}
\mnemonic{CRCD: Compare Ramp, Count Duration}
\end{mnemonicbox}

\questionmarks{3(a)}{3}{ડિજિટલ સ્ટોરેજ ઓસિલોસ્કોપ (DSO)ની વિશેષતાઓનું વર્ણન કરો.}

\begin{solutionbox}
\textbf{DSO ની વિશેષતાઓ}:
\begin{itemize}
    \item \keyword{ડિજિટલ સ્ટોરેજ}: વેવફોર્મ્સને અનંત સમય માટે મેમરીમાં સંગ્રહિત કરે છે.
    \item \keyword{પ્રી-ટ્રિગર વ્યૂઇંગ}: ટ્રિગર પોઇન્ટ પહેલાંની ઘટનાઓ જોઈ શકે છે.
    \item \keyword{ઓટોમેટિક માપન}: Vpp, Vrms, આવૃત્તિ આપમેળે ગણતરી કરે છે.
    \item \keyword{PC કનેક્ટિવિટી}: ડેટા લોગિંગ માટે USB/LAN.
    \item \keyword{એડવાન્સ્ડ ટ્રિગરિંગ}: પલ્સ વિડ્થ, વિડિયો, પેટર્ન ટ્રિગરિંગ.
\end{itemize}
\end{solutionbox}

\begin{mnemonicbox}
\mnemonic{SACRED: Storage, Analysis, Connectivity, Resolution, Extended functions, Digital processing}
\end{mnemonicbox}

\questionmarks{3(b)}{4}{લિસાજસ પેટર્નનો ઉપયોગ કરીને આવર્તન માપન પદ્ધતિ સમજાવો.}

\begin{solutionbox}
\textbf{લિસાજસ પેટર્ન્સ} જ્યારે CRO ના X અને Y પ્લેટ્સ પર બે સાઇન વેવ્સ લાગુ કરવામાં આવે છે ત્યારે રચાય છે.

\textbf{પદ્ધતિ}:
\begin{enumerate}
    \item અજ્ઞાત આવૃત્તિ ($f_y$) ને Y-ઇનપુટ સાથે જોડો.
    \item સ્ટાન્ડર્ડ વેરિએબલ ફ્રીક્વન્સી ($f_x$) ને X-ઇનપુટ સાથે જોડો.
    \item સ્થિર ક્લોઝ્ડ લૂપ પેટર્ન દેખાય ત્યાં સુધી $f_x$ સમાયોજિત કરો.
\end{enumerate}

\textbf{સૂત્ર}:
$$ \frac{f_y}{f_x} = \frac{\text{આડા સ્પર્શ બિંદુઓની સંખ્યા ($N_x$)}}{\text{ઊભા સ્પર્શ બિંદુઓની સંખ્યા ($N_y$)}} $$

\textbf{પેટર્ન્સ}:
\begin{center}
\begin{tikzpicture}
    % 1:1 Circle
    \draw (0,0) circle (0.8); \node at (0,-1.2) {1:1 (વર્તુળ)};
    % 2:1 Figure 8
    \draw [xshift=3cm, domain=0:360, samples=100] plot ({0.8*sin(\x)}, {0.8*sin(2*\x)}); \node at (3,-1.2) {2:1 (આકૃતિ 8)};
\end{tikzpicture}
\captionof{figure}{લિસાજસ પેટર્ન્સ}
\end{center}
\end{solutionbox}

\begin{mnemonicbox}
\mnemonic{XTYN: X-Tangents to Y-tangents gives the Number ratio}
\end{mnemonicbox}

\questionmarks{3(c)}{7}{બ્લોક ડાયાગ્રામની મદદથી CRO સમજાવો.}

\begin{solutionbox}
\textbf{કેથોડ રે ઓસિલોસ્કોપ (CRO)} સિગ્નલ વોલ્ટેજ વિરુદ્ધ સમય દર્શાવે છે.

\textbf{બ્લોક ડાયાગ્રામ}:
\begin{center}
\begin{tikzpicture}[node distance=1.5cm, auto]
    \node [gtu block] (VA) {વર્ટિકલ એમ્પ્લિફાયર};
    \node [gtu block, below=of VA] (DL) {ડીલે લાઇન};
    \node [coordinate, left=of VA] (In) {};
    \node [gtu block, right=of DL] (CRT) {CRT};
    \node [gtu block, below=of DL] (Trig) {ટ્રિગર};
    \node [gtu block, right=of Trig] (TB) {ટાઇમ બેઝ};
    \node [gtu block, right=of TB] (HA) {હોરિઝ એમ્પ્લિફાયર};
    \node [gtu block, below=of Trig] (PS) {પાવર સપ્લાય};
    
    \draw [gtu arrow] (In) -- node[above]{ઇનપુટ} (VA);
    \draw [gtu arrow] (VA) -- (DL);
    \draw [gtu arrow] (DL) -- (CRT);
    \draw [gtu arrow] (VA) |- (Trig);
    \draw [gtu arrow] (Trig) -- (TB);
    \draw [gtu arrow] (TB) -- (HA);
    \draw [gtu arrow] (HA) -| (CRT);
    \draw [gtu arrow] (PS) -| (CRT);
\end{tikzpicture}
\captionof{figure}{CRO બ્લોક ડાયાગ્રામ}
\end{center}

\textbf{મુખ્ય બ્લોક્સ}:
\begin{itemize}
    \item \keyword{વર્ટિકલ એમ્પ્લિફાયર}: ઇનપુટ સિગ્નલને મજબૂત બનાવે છે.
    \item \keyword{ડીલે લાઇન}: વર્ટિકલ સિગ્નલને વિલંબિત કરે છે જેથી હોરિઝોન્ટલ સ્વીપ શરૂ થઈ શકે.
    \item \keyword{ટાઇમ બેઝ}: હોરિઝોન્ટલ ડિફ્લેક્શન માટે સો-ટૂથ વેવ ઉત્પન્ન કરે છે.
    \item \keyword{ટ્રિગર}: સિગ્નલ સાથે સ્વીપને સિન્ક્રોનાઇઝ કરે છે.
    \item \keyword{CRT}: ઇલેક્ટ્રોન બીમ ટ્રેસ પ્રદર્શિત કરે છે.
\end{itemize}
\end{solutionbox}

\begin{mnemonicbox}
\mnemonic{VCTHP: Vertical input, Conditioned signal, Triggered sweep, Horizontal deflection, Phosphor display}
\end{mnemonicbox}

\questionmarks{3(a) OR}{3}{વિવિધ પ્રકારના CRO પ્રોબ સમજાવો.}

\begin{solutionbox}
\textbf{CRO પ્રોબ્સ} ટેસ્ટ સર્કિટને ઓસિલોસ્કોપ સાથે જોડે છે.

\begin{center}
\captionof{table}{પ્રોબના પ્રકાર}
\begin{tabulary}{\linewidth}{|L|L|}
\hline
\textbf{પ્રકાર} & \textbf{લાક્ષણિકતાઓ} \\ \hline
\textbf{પેસિવ પ્રોબ} & મજબૂત, સરળ, 1:1 અથવા 10:1 એટેન્યુએશન. ઉચ્ચ ઇનપુટ ઇમ્પિડન્સ. \\ \hline
\textbf{એક્ટિવ પ્રોબ} & બિલ્ટ-ઇન FET એમ્પ્લિફાયર. ઓછી કેપેસિટન્સ, ઉચ્ચ બેન્ડવિડ્થ. પાવર જરૂરી. \\ \hline
\textbf{કરંટ પ્રોબ} & ચુંબકીય ક્ષેત્ર (ક્લિપ-ઓન) દ્વારા કરંટ માપે છે. સર્કિટ તોડવાની જરૂર નથી. \\ \hline
\textbf{ડિફરેન્શિયલ પ્રોબ} & બે પોઇન્ટ વચ્ચેના વોલ્ટેજ તફાવતને માપે છે, કોમન મોડ નોઇઝને નકારે છે. \\ \hline
\end{tabulary}
\end{center}
\end{solutionbox}

\begin{mnemonicbox}
\mnemonic{PACD: Passive, Active, Current, Differential}
\end{mnemonicbox}

\questionmarks{3(b) OR}{4}{CRT ની આંતરિક રચના દોરો. ટૂંકમાં સમજાવો.}

\begin{solutionbox}
\textbf{CRT (કેથોડ રે ટ્યૂબ)} એક વેક્યુમ ટ્યૂબ છે જે દ્રશ્યમાન ડિસ્પ્લે ઉત્પન્ન કરે છે.

\textbf{રચના ડાયાગ્રામ}:
\begin{center}
\begin{tikzpicture}[xscale=0.8, yscale=0.8]
    % Glass envelope
    \draw [thick] (0,1) -- (2,1) -- (3,2) -- (7,2.5) -- (7,-2.5) -- (3,-2) -- (2,-1) -- (0,-1) -- cycle;
    \node at (7,0) [right] {સ્ક્રીન};
    
    % Gun components
    \draw (0.5,-0.2) rectangle (1,0.2); \node at (0.75,0.4) {K}; % Cathode
    \draw (1.2,-0.3) rectangle (1.4,0.3); \node at (1.3,0.5) {G}; % Grid
    \draw (1.6,-0.3) rectangle (1.9,0.3); \node at (1.75,0.5) {A}; % Anodes
    
    % Deflection Plates
    \draw (3.5, 0.5) -- (4.5, 0.5); \draw (3.5, -0.5) -- (4.5, -0.5); \node at (4, -0.8) {Y};
    \draw (5, 0.5) -- (5, 1.5); \draw (5.5, 0.5) -- (5.5, 1.5); \node at (5.25, 1.8) {X};
    
    % Beam
    \draw [dashed, red] (1,0) -- (7,0);
\end{tikzpicture}
\captionof{figure}{CRT રચના}
\end{center}

\textbf{ભાગો}:
\begin{itemize}
    \item \keyword{ઇલેક્ટ્રોન ગન}: K, G, A કેન્દ્રિત બીમ ઉત્પન્ન કરે છે.
    \item \keyword{ડિફ્લેક્શન પ્લેટ્સ}: Y-પ્લેટ્સ (વર્ટિકલ) અને X-પ્લેટ્સ (હોરિઝોન્ટલ) બીમને ખસેડે છે.
    \item \keyword{સ્ક્રીન}: ફોસ્ફર સાથે કોટેડ, ઇલેક્ટ્રોન અથડાતા પ્રકાશે છે.
\end{itemize}
\end{solutionbox}

\begin{mnemonicbox}
\mnemonic{GAFDS: Gun Aims, Focusing Directs, Screen shows}
\end{mnemonicbox}

\questionmarks{3(c) OR}{7}{DSO નો બ્લોક ડાયાગ્રામ વિગતવાર દોરો અને સમજાવો.}

\begin{solutionbox}
\textbf{ડિજિટલ સ્ટોરેજ ઓસિલોસ્કોપ (DSO)} વેવફોર્મ્સને ડિજિટાઇઝ અને સ્ટોર કરે છે.

\textbf{બ્લોક ડાયાગ્રામ}:
\begin{center}
\begin{tikzpicture}[node distance=1.5cm, auto]
    \node [gtu block] (ADC) {ADC};
    \node [gtu block, left=of ADC] (Cond) {સિગ્નલ કન્ડિશન};
    \node [coordinate, left=of Cond] (In) {};
    \node [gtu block, right=of ADC] (Mem) {મેમરી};
    \node [gtu block, right=of Mem] (Proc) {પ્રોસેસર};
    \node [gtu block, below=of Proc] (Disp) {ડિસ્પ્લે};
    \node [gtu block, below=of ADC] (Cont) {કંટ્રોલ/ક્લોક};
    
    \draw [gtu arrow] (In) -- (Cond);
    \draw [gtu arrow] (Cond) -- (ADC);
    \draw [gtu arrow] (ADC) -- (Mem);
    \draw [gtu arrow] (Mem) -- (Proc);
    \draw [gtu arrow] (Proc) -- (Disp);
    \draw [gtu arrow] (Cont) -- (ADC);
    \draw [gtu arrow] (Cont) -- (Mem);
\end{tikzpicture}
\captionof{figure}{DSO બ્લોક ડાયાગ્રામ}
\end{center}

\textbf{કામગીરી}:
\begin{enumerate}
    \item ઇનપુટ સિગ્નલ કન્ડિશન (એમ્પ્લિફાય/એટેન્યુએટ) થાય છે.
    \item \keyword{ADC} સિગ્નલને સેમ્પલ કરે છે અને બાઈનરી ડેટામાં રૂપાંતરિત કરે છે.
    \item ડેટા \keyword{ડિજિટલ મેમરી} માં સંગ્રહિત થાય છે.
    \item \keyword{માઇક્રોપ્રોસેસર} મેમરી વાંચે છે અને ડિસ્પ્લે માટે વેવફોર્મ પુનર્નિર્મિત કરે છે.
\end{enumerate}
\end{solutionbox}

\begin{mnemonicbox}
\mnemonic{SAMPLE-D: Signal Acquisition, Memory Processing, Locking trigger, Display}
\end{mnemonicbox}

\questionmarks{4(a)}{3}{NTC અને PTC થર્મિસ્ટરની સરખામણી આપો.}

\begin{solutionbox}
\begin{center}
\captionof{table}{NTC વિરુદ્ધ PTC થર્મિસ્ટર}
\begin{tabulary}{\linewidth}{|L|L|L|}
\hline
\textbf{પેરામીટર} & \textbf{NTC (નેગેટિવ ટેમ્પ કોએફિફિશિયન્ટ)} & \textbf{PTC (પોઝિટિવ ટેમ્પ કોએફિફિશિયન્ટ)} \\ \hline
રેસિસ્ટન્સ ફેરફાર & તાપમાન વધતા ઘટે છે & તાપમાન વધતા વધે છે \\ \hline
મટીરિયલ & સેમિકન્ડક્ટર ઓક્સાઇડ્સ (Mn, Ni) & બેરિયમ ટાઇટાનેટ \\ \hline
લિનિયરતા & નોન-લીનિયર (ઘાતાંકીય) & નોન-લીનિયર (તીવ્ર વધારો) \\ \hline
એપ્લિકેશન & તાપમાન માપન, કોમ્પેન્સેશન & ઓવરકરંટ પ્રોટેક્શન, હીટિંગ \\ \hline
\end{tabulary}
\end{center}
\end{solutionbox}

\begin{mnemonicbox}
\mnemonic{IN-DP: Increase Negative, Decrease Positive}
\end{mnemonicbox}

\questionmarks{4(b)}{4}{થર્મોકપલના કાર્યકારી સિદ્ધાંત અને બાંધકામ સમજાવો.}

\begin{solutionbox}
\textbf{થર્મોકપલ} સીબેક ઇફેક્ટના આધારે તાપમાન માપે છે.

\textbf{બાંધકામ}: એક છેડે જોડાયેલ બે અસમાન ધાતુના વાયર (હોટ જંક્શન). અન્ય છેડા (કોલ્ડ જંક્શન) મીટર પર જાય છે.

\textbf{આકૃતિ}:
\begin{center}
\begin{tikzpicture}
    \draw [thick, red] (0,0) -- (4,0); \node at (2, 0.2) {ધાતુ A};
    \draw [thick, blue] (0,0) -- (4, -1); \node at (2, -1.2) {ધાતુ B};
    
    \node [circle, fill, inner sep=2pt, label=left:હોટ જંક્શન] at (0,0) {};
    
    \draw [thick] (4,0) -- (4.5,0);
    \draw [thick] (4,-1) -- (4.5,-1);
    \node [draw] at (5, -0.5) {mV};
    
    \draw [dashed] (0,0) circle (0.5); \node at (0, -0.8) {$T_1$};
    \node at (4.5, -1.5) {રેફરન્સ જંક્શન $T_2$};
\end{tikzpicture}
\captionof{figure}{થર્મોકપલ}
\end{center}

\textbf{સિદ્ધાંત}: જ્યારે બે અલગ અલગ ધાતુઓ જોડાય છે અને જંક્શન વિવિધ તાપમાને હોય છે ($T_1 \neq T_2$), ત્યારે ઇલેક્ટ્રોમોટિવ ફોર્સ (EMF) ઉત્પન્ન થાય છે. $E = k(T_1 - T_2)$.
\end{solutionbox}

\begin{mnemonicbox}
\mnemonic{STEM: Seebeck-effect Transforms temperature to EMF in Metals}
\end{mnemonicbox}

\questionmarks{4(c)}{7}{સ્ટ્રેઇન ગેજ અને લોડ સેલની કામગીરી સમજાવો. RTD ના ફાયદા અને ગેરફાયદા આપો.}

\begin{solutionbox}
\textbf{સ્ટ્રેઇન ગેજ}: યાંત્રિક સ્ટ્રેઇન માપે છે.
\begin{itemize}
    \item \keyword{વેરિએબલ રેસિસ્ટન્સ ટ્રાન્સડ્યુસર}.
    \item બાંધકામ: કાગળ/બેકિંગ પર પાતળા વાયરની ગ્રીડ.
    \item કામગીરી: તણાવ $\rightarrow$ લંબાઈ $\uparrow$, ક્ષેત્રફળ $\downarrow$ $\rightarrow$ રેસિસ્ટન્સ $\uparrow$.
    \item ગેજ ફેક્ટર $G.F. = \frac{\Delta R / R}{\Delta L / L}$.
\end{itemize}

\textbf{લોડ સેલ}: ફોર્સ ટ્રાન્સડ્યુસર.
\begin{itemize}
    \item ધાતુના તત્વ (કોલમ/બીમ) સાથે જોડાયેલા સ્ટ્રેઇન ગેજ નો ઉપયોગ કરે છે.
    \item બળ વિકૃતિનું કારણ બને છે, જે વ્હીટસ્ટોન બ્રિજમાં સ્ટ્રેઇન ગેજ દ્વારા શોધાય છે.
\end{itemize}

\textbf{RTD (રેસિસ્ટન્સ ટેમ્પરેચર ડિટેક્ટર)}:
\begin{itemize}
    \item \textbf{ફાયદા}: ઉચ્ચ ચોકસાઈ, સ્થિર, લીનિયર.
    \item \textbf{ગેરફાયદા}: સેલ્ફ-હીટિંગ, થર્મોકપલ કરતાં ધીમો પ્રતિસાદ, બાહ્ય પાવર જરૂરી.
\end{itemize}
\end{solutionbox}

\begin{mnemonicbox}
\mnemonic{SPANNER: Strain Proportionally Alters Nominal Nominal Electrical Resistance}
\end{mnemonicbox}

\questionmarks{4(a) OR}{3}{ભેજ સેન્સર હાઇગ્રોમીટર સમજાવો.}

\begin{solutionbox}
\textbf{ભેજ સેન્સર (હાઇગ્રોમીટર)} સાપેક્ષ ભેજ (RH) માપે છે.

\textbf{પ્રકાર}:
\begin{itemize}
    \item \keyword{રેસિસ્ટિવ}: હાઇગ્રોસ્કોપિક પદાર્થ ભેજ શોષે છે $\rightarrow$ રેસિસ્ટન્સ ઘટે છે.
    \item \keyword{કેપેસિટિવ}: ભેજ શોષણ ડાઇલેક્ટ્રિક અચળાંક બદલે છે $\rightarrow$ કેપેસિટન્સ બદલાય છે.
\end{itemize}

\textbf{બ્લોક ડાયાગ્રામ}:
\begin{center}
\begin{tikzpicture}[node distance=1.5cm, auto]
    \node [gtu block] (Sens) {સેન્સિંગ એલિમેન્ટ};
    \node [gtu block, right=of Sens] (Cond) {સિગ્નલ કન્ડિશન};
    \node [gtu block, right=of Cond] (Disp) {ડિસ્પ્લે};
    \node [left=of Sens] (Air) {ભેજવાળી હવા};
    
    \draw [gtu arrow] (Air) -- (Sens);
    \draw [gtu arrow] (Sens) -- (Cond);
    \draw [gtu arrow] (Cond) -- (Disp);
\end{tikzpicture}
\captionof{figure}{ભેજ સેન્સર}
\end{center}
\end{solutionbox}

\begin{mnemonicbox}
\mnemonic{CRT-H: Capacitance/Resistance/Thermal changes with Humidity}
\end{mnemonicbox}

\questionmarks{4(b) OR}{4}{પીઝોઇલેક્ટ્રિક ટ્રાન્સડ્યુસર દોરો અને સમજાવો.}

\begin{solutionbox}
\textbf{પીઝોઇલેક્ટ્રિક ટ્રાન્સડ્યુસર} દબાણ/પ્રવેગકને વોલ્ટેજમાં રૂપાંતરિત કરે છે.

\textbf{આકૃતિ}:
\begin{center}
\begin{tikzpicture}
    \draw [fill=blue!10] (0,0) rectangle (3,1); \node at (1.5,0.5) {ક્રિસ્ટલ};
    \draw [thick] (0,1) -- (3,1); \node at (1.5,1.2) {ઇલેક્ટ્રોડ};
    \draw [thick] (0,0) -- (3,0); \node at (1.5,-0.2) {ઇલેક્ટ્રોડ};
    
    \draw [->, thick] (1.5, 2) -- (1.5, 1); \node at (1.5, 2.2) {બળ $F$};
    \draw (3,1) -- (4,1); \draw (3,0) -- (4,0);
    \node at (4.5, 0.5) {$V_{out}$};
\end{tikzpicture}
\captionof{figure}{પીઝોઇલેક્ટ્રિક ક્રિસ્ટલ}
\end{center}

\textbf{કામગીરી}:
\begin{itemize}
    \item \keyword{પીઝોઇલેક્ટ્રિક ઇફેક્ટ} પર આધારિત.
    \item જ્યારે ક્રિસ્ટલ (ક્વાર્ટ્ઝ, રોશેલ સોલ્ટ, PZT) પર દબાણ લાગુ કરવામાં આવે છે, ત્યારે સપાટી પર ચાર્જ એકઠા થાય છે.
    \item $V = Q/C = d \cdot F / C$.
    \item \textbf{એક્ટિવ ટ્રાન્સડ્યુસર} (સેલ્ફ-જનરેટિંગ).
\end{itemize}
\end{solutionbox}

\begin{mnemonicbox}
\mnemonic{PEMS: Pressure Ensures Measurable Signal}
\end{mnemonicbox}

\questionmarks{4(c) OR}{7}{ટ્રાન્સડ્યુસરનું વર્ગીકરણ વિગતવાર આપો.}

\begin{solutionbox}
\textbf{ટ્રાન્સડ્યુસર વર્ગીકરણ}:

\begin{enumerate}
    \item \textbf{ટ્રાન્સડક્શન સિદ્ધાંતના આધારે}:
    \begin{itemize}
        \item રેસિસ્ટિવ (પોટેન્શિયોમીટર, સ્ટ્રેઇન ગેજ)
        \item કેપેસિટિવ (ચલ અંતર, ડાઇલેક્ટ્રિક)
        \item ઇન્ડક્ટિવ (LVDT)
        \item પીઝોઇલેક્ટ્રિક
        \item ફોટોવોલ્ટેઇક/ફોટોકન્ડક્ટિવ
    \end{itemize}
    \item \textbf{એક્ટિવ વિરુદ્ધ પેસિવ}:
    \begin{itemize}
        \item \keyword{એક્ટિવ}: પોતાનું વોલ્ટેજ ઉત્પન્ન કરે છે (થર્મોકપલ, પીઝો).
        \item \keyword{પેસિવ}: બાહ્ય પાવરની જરૂર છે (RTD, સ્ટ્રેઇન ગેજ, LVDT).
    \end{itemize}
    \item \textbf{પ્રાઇમરી વિરુદ્ધ સેકન્ડરી}:
    \begin{itemize}
        \item \keyword{પ્રાઇમરી}: ભૌતિક ઘટના શોધો (બોર્ડન ટ્યૂબ).
        \item \keyword{સેકન્ડરી}: પ્રાઇમરી આઉટપુટને ઇલેક્ટ્રિકલમાં રૂપાંતરિત કરે છે (LVDT on Bourdon).
    \end{itemize}
    \item \textbf{એનાલોગ વિરુદ્ધ ડિજિટલ}:
    \begin{itemize}
        \item એનાલોગ: કન્ટિન્યુઅસ આઉટપુટ.
        \item ડિજિટલ: પલ્સ/બાઈનરી આઉટપુટ (એન્કોડર્સ).
    \end{itemize}
\end{enumerate}
\end{solutionbox}

\begin{mnemonicbox}
\mnemonic{APAD RICE: Active/Passive, Analog/Digital with Resistive, Inductive, Capacitive, Electromagnetic}
\end{mnemonicbox}

\questionmarks{5(a)}{3}{વિવિધ કેપેસિટિવ ટ્રાન્સડ્યુસર પર ટૂંક નોંધ લખો.}

\begin{solutionbox}
\textbf{કેપેસિટિવ ટ્રાન્સડ્યુસર્સ} $C = \frac{\epsilon A}{d}$ પર કામ કરે છે.

\textbf{પ્રકાર}:
\begin{enumerate}
    \item \textbf{ચલ વિભાજન ($d$)}: પ્લેટ ખસેડવાથી અંતર બદલાય છે. ડિસ્પ્લેસમેન્ટ, પ્રેશર માટે વપરાય છે.
    \item \textbf{ચલ ક્ષેત્રફળ ($A$)}: ફરતી પ્લેટ સ્થિર પ્લેટને ઓવરલેપ કરે છે. મોટા ડિસ્પ્લેસમેન્ટ માટે વપરાય છે.
    \item \textbf{ચલ ડાઇલેક્ટ્રિક ($\epsilon$)}: ડાઇલેક્ટ્રિક સામગ્રી પ્લેટો વચ્ચે ખસે છે. લિક્વિડ લેવલ માટે વપરાય છે.
\end{enumerate}
\end{solutionbox}

\begin{mnemonicbox}
\mnemonic{PALD: Parameter Alters the Leading Dielectric}
\end{mnemonicbox}

\questionmarks{5(b)}{4}{LVDT ટ્રાન્સડ્યુસર સમજાવો.}

\begin{solutionbox}
\textbf{LVDT (લીનિયર વેરિએબલ ડિફરેન્શિયલ ટ્રાન્સફોર્મર)} લીનિયર ડિસ્પ્લેસમેન્ટ માટે ઇન્ડક્ટિવ ટ્રાન્સડ્યુસર છે.

\textbf{બાંધકામ આકૃતિ}:
\begin{center}
\begin{circuitikz}[american, scale=0.8]
    % Core
    \draw [fill=gray!30] (1,3) rectangle (3,5);
    \node at (2,4) {કોર};
    \draw [->, thick] (2,5.2) -- (2,5.8) node[above] {Disp.};

    % Primary
    \draw (0,3.5) to[L, l=$P$] (0,4.5);
    \draw (-1,3.5) to[sV, l=$V_{in}$] (-1,4.5);
    \draw (-1,3.5) -- (0,3.5); \draw (-1,4.5) -- (0,4.5);

    % Secondaries
    \draw (4,5) to[L, l=$S_1$] (4,6);
    \draw (4,2) to[L, l=$S_2$] (4,3);
    
    % Series Opp
    \draw (4,3) -- (4,5);
    \draw (4,6) -- (5,6) node[right] {+};
    \draw (4,2) -- (5,2) node[right] {-};
    \node at (5.5, 4) {$V_{out}$};
\end{circuitikz}
\captionof{figure}{LVDT બાંધકામ}
\end{center}

\textbf{કામગીરી}:
\begin{itemize}
    \item પ્રાઇમરી AC દ્વારા સંચાલિત.
    \item સેકન્ડરીમાં પ્રેરિત વોલ્ટેજ કોર પોઝિશન પર આધારિત છે.
    \item સેકન્ડરી \keyword{સીરીઝ અપોઝિશન} માં જોડાયેલા છે: $V_{out} = V_{s1} - V_{s2}$.
    \item કેન્દ્ર પર (નલ): $V_{out} = 0$.
\end{itemize}
\end{solutionbox}

\begin{mnemonicbox}
\mnemonic{MDVN: Movement Determines Voltage from Null}
\end{mnemonicbox}

\questionmarks{5(c)}{7}{હાર્મોનિક્સ ડિસ્ટોર્શન એનાલાઇઝર દોરો અને સમજાવો.}

\begin{solutionbox}
\textbf{હાર્મોનિક ડિસ્ટોર્શન એનાલાઇઝર} ટોટલ હાર્મોનિક ડિસ્ટોર્શન (THD) માપે છે.

\textbf{બ્લોક ડાયાગ્રામ}:
\begin{center}
\begin{tikzpicture}[node distance=1.5cm, auto]
    \node [gtu block] (Imp) {ઇમ્પિડન્સ કન્વ.};
    \node [gtu block, left=of Imp] (In) {ઇનપુટ};
    \node [gtu block, right=of Imp] (Notch) {નોચ ફિલ્ટર};
    \node [gtu block, right=of Notch] (Amp) {એમ્પ્લિફાયર};
    \node [gtu block, right=of Amp] (Meter) {મીટર};
    
    \draw [gtu arrow] (In) -- (Imp);
    \draw [gtu arrow] (Imp) -- (Notch);
    \draw [gtu arrow] (Notch) -- (Amp);
    \draw [gtu arrow] (Amp) -- (Meter);
    
    % Switch bypass
    \draw [dashed] (Imp) -- ++(0,-1) -| (Amp); \node at (1.5, -1.2) {100\% સેટ (બાયપાસ)};
\end{tikzpicture}
\captionof{figure}{ફન્ડામેન્ટલ સપ્રેશન એનાલાઇઝર}
\end{center}

\textbf{કામગીરી}:
\begin{enumerate}
    \item \keyword{સેટ લેવલ}: ફિલ્ટર બાયપાસ. મીટર કુલ સિગ્નલ વાંચે છે (ફન્ડામેન્ટલ + હાર્મોનિક્સ). 100\% માર્ક કરવા ગેઇન સેટ કરો.
    \item \keyword{મેઝર}: ફિલ્ટર દાખલ. ફન્ડામેન્ટલ આવૃત્તિ દૂર કરવામાં આવે છે. મીટર બાકીના હાર્મોનિક્સ માપે છે.
    \item $THD = \frac{\sqrt{\sum V_n^2}}{V_1}$.
\end{enumerate}
\end{solutionbox}

\begin{mnemonicbox}
\mnemonic{FAIR-D: Filter And Isolate Residuals for Distortion}
\end{mnemonicbox}

\questionmarks{5(a) OR}{3}{પ્રોક્સિમિટી સેન્સરના કાર્યકારી સિદ્ધાંતને સમજાવો.}

\begin{solutionbox}
\textbf{પ્રોક્સિમિટી સેન્સર} સંપર્ક વિના ઓબ્જેક્ટ્સની હાજરી શોધે છે.

\textbf{પ્રકાર}:
\begin{itemize}
    \item \keyword{ઇન્ડક્ટિવ}: એડી કરંટ દ્વારા ધાતુના પદાર્થો શોધે છે.
    \item \keyword{કેપેસિટિવ}: ડાઇલેક્ટ્રિક ફેરફાર દ્વારા કોઈપણ પદાર્થ શોધે છે.
    \item \keyword{ઓપ્ટિકલ}: પ્રકાશ કિરણ અવરોધ/પ્રતિબિંબ દ્વારા શોધે છે.
\end{itemize}

\textbf{કામગીરી}: ઇલેક્ટ્રોમેગ્નેટિક અથવા ઇલેક્ટ્રોસ્ટેટિક ક્ષેત્ર ઉત્સર્જિત થાય છે. ક્ષેત્રમાં પ્રવેશતી વસ્તુ ક્ષેત્રના ગુણધર્મોને બદલે છે (ઓસિલેશન ભીનાશ અથવા કેપેસિટન્સ બદલવી), જે સ્વિચિંગ સર્કિટને ટ્રિગર કરે છે.
\end{solutionbox}

\begin{mnemonicbox}
\mnemonic{CUPS: Capacitive, Ultrasonic, Photoelectric, Sense}
\end{mnemonicbox}

\questionmarks{5(b) OR}{4}{એબ્સોલ્યુટ અને ઇન્ક્રીમેન્ટલ પ્રકારના ઓપ્ટિકલ એન્કોડર સમજાવો.}

\begin{solutionbox}
\textbf{ઓપ્ટિકલ એન્કોડર} કોણીય સ્થિતિ/ઝડપ માપે છે.

\textbf{એબ્સોલ્યુટ એન્કોડર}:
\begin{itemize}
    \item ડિસ્કમાં દરેક કોણ માટે અનન્ય બાઈનરી કોડ સાથે બહુવિધ ટ્રેક હોય છે.
    \item તરત જ સંપૂર્ણ સ્થિતિનું આઉટપુટ આપે છે.
    \item પાવર ગુમાવવા પર સ્થિતિ ગુમાવતું નથી.
\end{itemize}

\textbf{ઇન્ક્રીમેન્ટલ એન્કોડર}:
\begin{itemize}
    \item ડિસ્ક પર પરિઘ પર સ્લોટ હોય છે.
    \item જેમ તે ફરે છે તેમ પલ્સ આઉટપુટ આપે છે.
    \item ઝડપ અને સાપેક્ષ ફેરફાર માપે છે.
    \item પાવર ગુમાવવા પર સ્થિતિ ગુમાવે છે.
\end{itemize}

\textbf{આકૃતિ}:
\begin{center}
\begin{tikzpicture}
    % Disc
    \draw (0,0) circle (1.5);
    \foreach \x in {0, 45, ..., 360} \draw [thick] (0,0) -- (\x:1.5);
    \node at (0,2) {સ્લોટેડ ડિસ્ક};
    
    % LED/Sensor
    \node at (2,0) {LED $\rightarrow$ સેન્સર};
    \draw [dashed] (1.5, 0) -- (2.5, 0);
\end{tikzpicture}
\captionof{figure}{મૂળભૂત એન્કોડર સિદ્ધાંત}
\end{center}
\end{solutionbox}

\begin{mnemonicbox}
\mnemonic{APIR-CD: Absolute Provides Immediate Reading, Counter Determines incremental}
\end{mnemonicbox}

\questionmarks{5(c) OR}{7}{ડિજિટલ IC ટેસ્ટર પર ટૂંકી નોંધ લખો.}

\begin{solutionbox}
\textbf{ડિજિટલ IC ટેસ્ટર} લોજિક ગેટ્સ, ફ્લિપ-ફ્લોપ્સ વગેરેની કાર્યક્ષમતા તપાસે છે.

\textbf{બ્લોક ડાયાગ્રામ}:
\begin{center}
\begin{tikzpicture}[node distance=1.5cm, auto]
    \node [gtu block] (CPU) {CPU};
    \node [gtu block, below=of CPU] (Mem) {પેટર્ન ROM};
    \node [gtu block, right=of CPU] (Socket) {ZIF સોકેટ};
    \node [gtu block, right=of Socket] (Comp) {કમ્પેરેટર};
    \node [gtu block, left=of CPU] (Disp) {ડિસ્પ્લે/કીપેડ};
    
    \draw [gtu arrow] (CPU) -- (Socket);
    \draw [gtu arrow] (Mem) -- (CPU);
    \draw [gtu arrow] (Socket) -- (Comp);
    \draw [gtu arrow] (Comp) -- node[above]{પાસ/ફેલ} (CPU);
    \draw [gtu arrow] (Disp) -- (CPU);
\end{tikzpicture}
\captionof{figure}{IC ટેસ્ટર}
\end{center}

\textbf{ઓપરેશન}:
\begin{enumerate}
    \item યુઝર IC નંબર દાખલ કરે છે (દા.ત., 7400).
    \item CPU ROM માંથી ટ્રુથ ટેબલ મેળવે છે.
    \item CPU ZIF સોકેટ દ્વારા IC પિન પર ઇનપુટ લાગુ કરે છે.
    \item કમ્પેરેટર વાસ્તવિક આઉટપુટની અપેક્ષિત આઉટપુટ સાથે તુલના કરે છે.
    \item જો બધા મેચ થાય $\rightarrow$ PASS. અન્યથા $\rightarrow$ FAIL.
\end{enumerate}
\end{solutionbox}

\begin{mnemonicbox}
\mnemonic{GATES: Generate And Test Every Signal}
\end{mnemonicbox}

\end{document}
