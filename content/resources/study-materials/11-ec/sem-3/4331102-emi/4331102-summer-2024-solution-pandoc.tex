\documentclass[10pt,a4paper]{article}

% content/resources/templates/preamble.tex
\usepackage[margin=0.6in]{geometry}
\author{Milav Dabgar}
\usepackage{amsmath,amssymb,amsthm}
\usepackage{booktabs}
\usepackage{multirow}
\usepackage{xcolor}
\usepackage{tcolorbox}
\tcbuselibrary{breakable,skins}
\usepackage[colorlinks=true,linkcolor=blue]{hyperref}
\usepackage{titlesec}
\usepackage{enumitem}
\usepackage{tikz}
\usepackage{pgfplots}
\usepackage{circuitikz}
\usepackage[version=4]{mhchem}
\usepackage{longtable}
\usepackage{array}
\usepackage{float}
\usepackage{caption}
\usepackage{listings}

\lstset{
  basicstyle=\small\ttfamily,
  breaklines=true,
  breakatwhitespace=false,
  postbreak=\mbox{\textcolor{red}{$\hookrightarrow$}\space},
  float=false,
  numbers=left,
  numberstyle=\tiny\color{gray},
  numbersep=10pt,
  xleftmargin=2em,
  keywordstyle=\color{blue},
  commentstyle=\color{green!60!black},
  stringstyle=\color{purple},
  backgroundcolor=\color{gray!5},
  showstringspaces=false,
  tabsize=2,
  captionpos=b,
  keepspaces=true,
  columns=flexible
}

\pgfplotsset{compat=1.18}
\usetikzlibrary{shapes,arrows,positioning,calc,patterns,decorations.pathmorphing,decorations.markings,arrows.meta}

% Color scheme
\definecolor{headcolor}{RGB}{0,102,204}
\definecolor{keycolor}{RGB}{220,20,60}
\definecolor{solutioncolor}{RGB}{34,139,34}
\definecolor{mnemoniccolor}{RGB}{148,0,211}
\definecolor{codecolor}{RGB}{0,0,100}

% Spacing
\setlength{\parskip}{3pt}
\setlist[itemize]{nosep}
\setlist[enumerate]{nosep}

% Title formatting
\titleformat{\section}{\Large\bfseries\color{headcolor}}{\thesection}{1em}{}
\titleformat{\subsection}{\large\bfseries\color{headcolor}}{\thesubsection}{1em}{}

% Pandoc tightlist compatibility
\providecommand{\tightlist}{%
  \setlength{\itemsep}{0pt}\setlength{\parskip}{0pt}}

% Pandoc longtable compatibility
\newcounter{none}
\def\thenone{}


% content/resources/templates/english-boxes.tex
% This file is currently empty - it exists to maintain consistency with the import structure.
% Add custom environments here if needed in the future.


\begin{document}

\begin{center}
{\Huge\bfseries\color{headcolor} Subject Name Solutions}\\[5pt]
{\LARGE 4331102 -- Summer 2024}\\[3pt]
{\large Semester 1 Study Material}\\[3pt]
{\normalsize\textit{Detailed Solutions and Explanations}}
\end{center}

\vspace{10pt}

\subsection*{Question 1(a) [3 marks]}\label{q1a}

\textbf{Define following term: (1) Accuracy (2) precision (3)
Reproducibility}

\begin{solutionbox}

\begin{itemize}
\tightlist
\item
  \textbf{Accuracy}: Closeness of measured value to the true value of
  measured quantity
\item
  \textbf{Precision}: Ability of an instrument to reproduce the same
  output for repeated applications of same input under identical
  conditions
\item
  \textbf{Reproducibility}: Degree of agreement between results of
  measurements of same quantity when measured under changed conditions
  (different method, observer, or time)
\end{itemize}

\end{solutionbox}
\begin{mnemonicbox}
``APR: Accurate-to-truth, Precise-repeats,
Reproduce-under-change''

\end{mnemonicbox}
\subsection*{Question 1(b) [4 marks]}\label{q1b}

\textbf{Explain construction of RTD Transducer with necessary diagram in
detail. Also list application of it.}

\begin{solutionbox}

RTD (Resistance Temperature Detector) is a temperature sensor that
operates on the principle that electrical resistance of metals changes
with temperature.

\textbf{Diagram:}

\begin{center}
\textbf{Mermaid Diagram (Code)}
\begin{verbatim}
{Shaded}
{Highlighting}[]
graph LR
    A[Sensing Element] {-{-}{} B[Lead Wires]}
    B {-{-}{} C[Support]}
    C {-{-}{} D[Protective Sheath]}
    style A fill:\#f9f,stroke:\#333,stroke{-width:2px}
    style D fill:\#bbf,stroke:\#333,stroke{-width:2px}
{Highlighting}
{Shaded}
\end{verbatim}
\end{center}

\begin{itemize}
\tightlist
\item
  \textbf{Sensing Element}: Pure platinum, nickel, or copper wire wound
  around ceramic core
\item
  \textbf{Lead Wires}: Connect RTD to measuring circuit
\item
  \textbf{Support}: Provides mechanical stability to sensing element
\item
  \textbf{Protective Sheath}: Protects sensing element from external
  environment
\end{itemize}

\textbf{Applications of RTD:}

\begin{itemize}
\tightlist
\item
  Temperature measurement in process industries
\item
  Food processing temperature monitoring
\item
  HVAC systems
\item
  Medical equipment
\end{itemize}

\end{solutionbox}
\begin{mnemonicbox}
``RTD: Resistance Temperature Detector - Precise
Temperature Measurement''

\end{mnemonicbox}
\subsection*{Question 1(c) [7 marks]}\label{q1c}

\textbf{Explain working of Maxwell's Bridge with circuit diagram. List
its advantages, disadvantages and applications.}

\begin{solutionbox}

Maxwell's Bridge is used to measure unknown inductance in terms of known
capacitance and resistance.

\textbf{Circuit Diagram:}

\begin{center}
\textbf{Mermaid Diagram (Code)}
\begin{verbatim}
{Shaded}
{Highlighting}[]
graph LR
    A((R1)) {-{-}{-} B((R2))}
    B {-{-}{-} C((R3))}
    C {-{-}{-} D((R4))}
    D {-{-}{-} A}
    E[L1] {-{-}{-} A}
    E {-{-}{-} D}
    F[C4] {-{-}{-} B}
    F {-{-}{-} C}
    G[Supply] {-{-}{-} A}
    G {-{-}{-} C}
    H[Detector] {-{-}{-} B}
    H {-{-}{-} D}
    style E fill:\#f96,stroke:\#333,stroke{-width:2px}
    style F fill:\#9cf,stroke:\#333,stroke{-width:2px}
{Highlighting}
{Shaded}
\end{verbatim}
\end{center}

\textbf{Working:} At balance condition: L1 = C4 \times R2 \times R3

When the bridge is balanced, the detector shows zero current. The
unknown inductance L1 is calculated using above equation, where C4 is
known capacitance and R2, R3 are known resistances.

{\def\LTcaptype{none} % do not increment counter
\begin{longtable}[]{@{}ll@{}}
\toprule\noalign{}
Parameter & Value \\
\midrule\noalign{}
\endhead
\bottomrule\noalign{}
\endlastfoot
Balance Equation & L1 = C4 \times R2 \times R3 \\
Quality Factor &

Q = ωL1/R1 = ωC4R3 \\

\end{longtable}
}

\textbf{Advantages:}

\begin{itemize}
\tightlist
\item
  High accuracy for medium Q inductors
\item
  Balance equations are independent of frequency
\item
  Simple calculation for inductance
\end{itemize}

\textbf{Disadvantages:}

\begin{itemize}
\tightlist
\item
  Not suitable for low Q inductor measurement
\item
  Requires variable standard capacitor
\item
  Affected by stray capacitance
\end{itemize}

\textbf{Applications:}

\begin{itemize}
\tightlist
\item
  Measuring inductance in laboratories
\item
  Calibration of inductance standards
\item
  Testing of inductive components
\end{itemize}

\end{solutionbox}
\begin{mnemonicbox}
``Maxwell's Magic: Inductance equals Capacitance
times Resistance squared''

\end{mnemonicbox}
\subsection*{Question 1(c) OR [7
marks]}\label{q1c}

\textbf{Explain working of Wheatstone bridge with circuit diagram for
balance condition. List its advantages, disadvantages, and
applications.}

\begin{solutionbox}

Wheatstone bridge is used to measure unknown resistance by comparing it
with known resistance values.

\textbf{Circuit Diagram:}

\begin{center}
\textbf{Mermaid Diagram (Code)}
\begin{verbatim}
{Shaded}
{Highlighting}[]
graph LR
    A((P)) {-{-}{-} B((Q))}
    B {-{-}{-} C((S))}
    C {-{-}{-} D((R))}
    D {-{-}{-} A}
    E[Battery] {-{-}{-} A}
    E {-{-}{-} C}
    F[Galvanometer] {-{-}{-} B}
    F {-{-}{-} D}
    style D fill:\#f96,stroke:\#333,stroke{-width:2px}
    style F fill:\#9cf,stroke:\#333,stroke{-width:2px}
{Highlighting}
{Shaded}
\end{verbatim}
\end{center}

\textbf{Working:} At balance condition: P/Q = R/S or R = S \times (P/Q)

When bridge is balanced, galvanometer shows zero deflection. Unknown
resistance R is calculated using the ratio of other resistances.

{\def\LTcaptype{none} % do not increment counter
\begin{longtable}[]{@{}ll@{}}
\toprule\noalign{}
Component & Function \\
\midrule\noalign{}
\endhead
\bottomrule\noalign{}
\endlastfoot
P, Q, S & Known resistances \\
R & Unknown resistance \\
G & Galvanometer (detector) \\
E & DC voltage source \\
\end{longtable}
}

\textbf{Advantages:}

\begin{itemize}
\tightlist
\item
  High accuracy in resistance measurement
\item
  Simple construction and operation
\item
  Wide range of resistance measurement
\end{itemize}

\textbf{Disadvantages:}

\begin{itemize}
\tightlist
\item
  Cannot measure very low or very high resistances
\item
  Requires battery as power source
\item
  Temperature effects on resistors cause errors
\end{itemize}

\textbf{Applications:}

\begin{itemize}
\tightlist
\item
  Precise resistance measurement
\item
  Strain gauge measurements
\item
  Temperature sensing using RTDs
\item
  Transducer applications
\end{itemize}

\end{solutionbox}
\begin{mnemonicbox}
``When WheatStone Balances: Product of opposites are
equal (P\timesS = Q\timesR)''

\end{mnemonicbox}
\subsection*{Question 2(a) [3 marks]}\label{q2a}

\textbf{Compare moving iron and moving coil type instruments.}

\begin{solutionbox}

{\def\LTcaptype{none} % do not increment counter
\begin{longtable}[]{@{}lll@{}}
\toprule\noalign{}
Characteristic & Moving Iron Type & Moving Coil Type \\
\midrule\noalign{}
\endhead
\bottomrule\noalign{}
\endlastfoot
Principle & Magnetic attraction/repulsion & Electromagnetic force \\
Scale & Non-uniform & Uniform \\
Damping & Poor & Good \\
Accuracy & Less accurate (2-5\%) & High accuracy (0.1-2\%) \\
Frequency range & DC and AC & DC only (without rectifier) \\
Power consumption & High & Low \\
Cost & Less expensive & More expensive \\
\end{longtable}
}

\end{solutionbox}
\begin{mnemonicbox}
``IMAP-CAD: Iron-Magnetic-AC-Poor damping,
Coil-Accurate-DC-Damped well''

\end{mnemonicbox}
\subsection*{Question 2(b) [4 marks]}\label{q2b}

\textbf{Explain working and construction of successive approximation
type DVM with necessary diagram.}

\begin{solutionbox}

Successive Approximation type Digital Voltmeter (DVM) converts analog
voltage to digital value using binary search technique.

\textbf{Block Diagram:}

\begin{center}
\textbf{Mermaid Diagram (Code)}
\begin{verbatim}
{Shaded}
{Highlighting}[]
graph LR
    A[Input] {-{-}{} B[Sample \& Hold]}
    B {-{-}{} C[Comparator]}
    D[DAC] {-{-}{} C}
    C {-{-}{} E[SAR {-} Successive Approximation Register]}
    E {-{-}{} D}
    E {-{-}{} F[Display]}
    G[Clock] {-{-}{} E}
    style E fill:\#f96,stroke:\#333,stroke{-width:2px}
    style C fill:\#9cf,stroke:\#333,stroke{-width:2px}
{Highlighting}
{Shaded}
\end{verbatim}
\end{center}

\textbf{Working:}

\begin{enumerate}
\tightlist
\item
  Sample \& Hold circuit captures input voltage
\item
  SAR sets MSB to 1, other bits to 0
\item
  DAC converts digital word to analog voltage
\item
  Comparator compares DAC output with input voltage
\item
  If DAC output \textgreater{} input, bit is reset to 0; otherwise kept
  1
\item
  Process repeats for next bit until all bits are tested
\item
  Final digital word represents input voltage
\end{enumerate}

\textbf{Advantages:}

\begin{itemize}
\tightlist
\item
  Medium conversion speed (10-100 μs)
\item
  Good resolution and accuracy
\item
  Moderate cost
\end{itemize}

\end{solutionbox}
\begin{mnemonicbox}
``SAR DVM: Sample-And-Register by
Digital-Voltage-Matching''

\end{mnemonicbox}
\subsection*{Question 2(c) [7 marks]}\label{q2c}

\textbf{1- A moving coil ammeter reading up to 10 amperes has a
resistance of 0.02 ohm. How this instrument could be adopted to read
current up to 1000 amperes?} \textbf{2- A moving coil voltmeter reading
up to 200 mV has a resistance of 5 ohms. How this instrument can be
adopted to read voltage up to 300 volts?}

\begin{solutionbox}

\textbf{Part 1: Ammeter Range Extension}

To extend ammeter range from 10A to 1000A, a shunt resistor is connected
in parallel with the meter.

\textbf{Diagram:}

\begin{center}
\textbf{Mermaid Diagram (Code)}
\begin{verbatim}
{Shaded}
{Highlighting}[]
graph LR
    A[Current Input] {-{-}{} B\{Branch\}}
    B {-{-}{}|Shunt Path| C[Rsh]}
    B {-{-}{}|Meter Path| D[Meter]}
    C {-{-}{} E\{Join\}}
    D {-{-}{} E}
    E {-{-}{} F[Output]}
    style C fill:\#f96,stroke:\#333,stroke{-width:2px}
{Highlighting}
{Shaded}
\end{verbatim}
\end{center}

\textbf{Calculation:}

\begin{itemize}
\tightlist
\item
  Original meter resistance (Rm) = 0.02 Ω
\item
  Original full-scale current (Im) = 10 A
\item
  Desired full-scale current (I) = 1000 A
\item
  Current through shunt (Ish) = I - Im = 1000 - 10 = 990 A
\item
  Voltage across meter = Voltage across shunt
\item
  Im \times Rm = Ish \times Rsh
\item
  Rsh = (Im \times Rm) \div Ish = (10 \times 0.02) \div 990 = 0.0002 Ω
\end{itemize}

\textbf{Part 2: Voltmeter Range Extension}

To extend voltmeter range from 200mV to 300V, a multiplier resistor is
connected in series with the meter.

\textbf{Diagram:}

\begin{center}
\textbf{Mermaid Diagram (Code)}
\begin{verbatim}
{Shaded}
{Highlighting}[]
graph LR
    A[Voltage Input] {-{-}{} B[Rs]}
    B {-{-}{} C[Meter]}
    C {-{-}{} D[Output]}
    style B fill:\#f96,stroke:\#333,stroke{-width:2px}
{Highlighting}
{Shaded}
\end{verbatim}
\end{center}

\textbf{Calculation:}

\begin{itemize}
\tightlist
\item
  Original meter resistance (Rm) = 5 Ω
\item
  Original full-scale voltage (Vm) = 200 mV = 0.2 V
\item
  Desired full-scale voltage (V) = 300 V
\item
  Series resistance (Rs) = [(V \div Vm) - 1] \times Rm
\item
  Rs = [(300 \div 0.2) - 1] \times 5 = (1500 - 1) \times 5 = 1499 \times 5 = 7495 Ω
\end{itemize}

\end{solutionbox}
\begin{mnemonicbox}
``ShuntSeries: Shunt-for-Current,
Series-for-Voltage''

\end{mnemonicbox}
\subsection*{Question 2(a) OR [3
marks]}\label{q2a}

\textbf{Explain working and construction of Clamp on Meter with
necessary diagram.}

\begin{solutionbox}

Clamp on Meter (Current Clamp) measures current without breaking the
circuit by using electromagnetic induction.

\textbf{Diagram:}

\begin{center}
\textbf{Mermaid Diagram (Code)}
\begin{verbatim}
{Shaded}
{Highlighting}[]
graph LR
    A[Clamp Jaw] {-{-}{} B[Current Transformer]}
    B {-{-}{} C[Rectifier Circuit]}
    C {-{-}{} D[Measuring Circuit]}
    D {-{-}{} E[Display]}
    style A fill:\#f96,stroke:\#333,stroke{-width:2px}
    style B fill:\#9cf,stroke:\#333,stroke{-width:2px}
{Highlighting}
{Shaded}
\end{verbatim}
\end{center}

\textbf{Construction \& Working:}

\begin{itemize}
\tightlist
\item
  \textbf{Clamp Jaw}: Split core transformer that can be opened to
  encircle conductor
\item
  \textbf{Current Transformer}: Converts primary current to proportional
  secondary current
\item
  \textbf{Rectifier}: Converts AC to DC for measurement circuit
\item
  \textbf{Measuring Circuit}: Processes signal and calculates current
  value
\item
  \textbf{Display}: Shows measured current value
\end{itemize}

When a current-carrying conductor passes through the clamp jaw, it
induces current in the secondary winding proportional to primary
current, which is then measured.

\end{solutionbox}
\begin{mnemonicbox}
``CLAMP: Current-Loop Amplifies Magnetic
Proportionally''

\end{mnemonicbox}
\subsection*{Question 2(b) OR [4
marks]}\label{q2b}

\textbf{Explain working of PMMC instruments with necessary diagram.}

\begin{solutionbox}

PMMC (Permanent Magnet Moving Coil) instruments operate on the principle
of electromagnetic force on current-carrying conductor in magnetic
field.

\textbf{Diagram:}

\begin{center}
\textbf{Mermaid Diagram (Code)}
\begin{verbatim}
{Shaded}
{Highlighting}[]
graph LR
    A[Permanent Magnet] {-{-}{} B[Air Gap]}
    B {-{-}{} C[Moving Coil]}
    C {-{-}{} D[Pointer]}
    C {-{-}{} E[Spring]}
    C {-{-}{} F[Damping Mechanism]}
    style C fill:\#f96,stroke:\#333,stroke{-width:2px}
    style A fill:\#9cf,stroke:\#333,stroke{-width:2px}
{Highlighting}
{Shaded}
\end{verbatim}
\end{center}

\textbf{Working:}

\begin{enumerate}
\tightlist
\item
  Current flows through rectangular coil placed in magnetic field
\item
  Electromagnetic force produces torque proportional to current
\item
  Spring provides controlling torque
\item
  Pointer deflects proportionally to current
\item
  Damping system prevents oscillations
\end{enumerate}

\textbf{Components:}

\begin{itemize}
\tightlist
\item
  Permanent magnet creates strong magnetic field
\item
  Soft iron core concentrates magnetic flux
\item
  Moving coil carries current to be measured
\item
  Control springs provide restoring force
\item
  Damping system (air or eddy current) reduces oscillations
\end{itemize}

\end{solutionbox}
\begin{mnemonicbox}
``PMMC: Permanent Magnet Makes Current-proportional
movement''

\end{mnemonicbox}
\subsection*{Question 2(c) OR [7
marks]}\label{q2c}

\textbf{Draw the block diagram, working and construction of Integrating
type DVM with necessary diagram and waveform.}

\begin{solutionbox}

Integrating type DVM (Digital Voltmeter) converts analog voltage to
digital value by integrating the input over a fixed time.

\textbf{Block Diagram:}

\begin{center}
\textbf{Mermaid Diagram (Code)}
\begin{verbatim}
{Shaded}
{Highlighting}[]
graph LR
    A[Input Voltage] {-{-}{} B[Buffer Amplifier]}
    B {-{-}{} C[Integrator]}
    D[Clock] {-{-}{} E[Control Logic]}
    E {-{-}{} C}
    E {-{-}{} F[Counter]}
    C {-{-}{} G[Comparator]}
    G {-{-}{} E}
    F {-{-}{} H[Display]}
    I[Reference Voltage] {-{-}{} G}
    style C fill:\#f96,stroke:\#333,stroke{-width:2px}
    style G fill:\#9cf,stroke:\#333,stroke{-width:2px}
{Highlighting}
{Shaded}
\end{verbatim}
\end{center}

\textbf{Waveforms:}

\begin{verbatim}
    \^{}
    |    \_\_\_\_\_\_ Time T1 \_\_\_\_\_\_
 Vi |   /|                    |{}
    |  / |                    | {}
    | /  |                    |  {}
    |/   |                    |   {}
    +{-{-}{-}{-}+{-}{-}{-}{-}{-}{-}{-}{-}{-}{-}{-}{-}{-}{-}{-}{-}{-}{-}{-}{-}+{-}{-}{-}{-}+{-}{-}{-} t}
         |                    |
         | Integration period |
         |{{-}{-}{-}{-}{-}{-}{-}{-}{-}{-}{-}{-}{-}{-}{-}{-}{-}|}
\end{verbatim}

\textbf{Working:}

\begin{enumerate}
\tightlist
\item
  \textbf{Dual-Slope Method:}

  \begin{itemize}
  \tightlist
  \item
    Input voltage is integrated for fixed time T1
  \item
    Integrator is connected to negative reference voltage
  \item
    Time T2 required to return to zero is proportional to input voltage
  \item
    Digital display shows count proportional to T2
  \end{itemize}
\end{enumerate}

{\def\LTcaptype{none} % do not increment counter
\begin{longtable}[]{@{}ll@{}}
\toprule\noalign{}
Phase & Action \\
\midrule\noalign{}
\endhead
\bottomrule\noalign{}
\endlastfoot
Phase 1 & Integrate unknown voltage for fixed time T1 \\
Phase 2 & Integrate known reference voltage until zero \\
Phase 3 & Count clock pulses during phase 2 (T2) \\
\end{longtable}
}

\textbf{Advantages:}

\begin{itemize}
\tightlist
\item
  High noise rejection (especially 50/60 Hz)
\item
  Good accuracy
\item
  Automatic zero adjustment
\end{itemize}

\end{solutionbox}
\begin{mnemonicbox}
``Integrate-twice: Up with unknown, Down with
reference''

\end{mnemonicbox}
\subsection*{Question 3(a) [3 marks]}\label{q3a}

\textbf{In CRO What is the value of unknown DC voltage, if a straight
line below x-axis is obtained with a displacement of 4cm and volt/div
knob = 3V. calculate the unknown voltage Vdc.}

\begin{solutionbox}

\textbf{Calculation:} Displacement = 4 cm (below x-axis) Volt/div
setting = 3 V/div Direction = Below x-axis (negative voltage)

Vdc = -(Displacement \times Volt/div) Vdc = -(4 cm \times 3 V/div) Vdc = -12 V

Therefore, the unknown DC voltage is -12 V.

\end{solutionbox}
\begin{mnemonicbox}
``Voltage = Deflection \times Scale''

\end{mnemonicbox}
\subsection*{Question 3(b) [4 marks]}\label{q3b}

\textbf{Draw internal structure of CRT. Explain in short.}

\begin{solutionbox}

CRT (Cathode Ray Tube) is the display device used in analog
oscilloscopes.

\textbf{Diagram:}

\begin{center}
\textbf{Mermaid Diagram (Code)}
\begin{verbatim}
{Shaded}
{Highlighting}[]
graph LR
    A[Electron Gun] {-{-}{} B[Focusing System]}
    B {-{-}{} C[Deflection System]}
    C {-{-}{} D[Phosphor Screen]}
    E[Glass Envelope] {-{-}{} A}
    E {-{-}{} B}
    E {-{-}{} C}
    E {-{-}{} D}
    style A fill:\#f96,stroke:\#333,stroke{-width:2px}
    style D fill:\#9cf,stroke:\#333,stroke{-width:2px}
{Highlighting}
{Shaded}
\end{verbatim}
\end{center}

\textbf{Components:}

\begin{itemize}
\tightlist
\item
  \textbf{Electron Gun}: Consists of heater, cathode, control grid, and
  anodes; produces electron beam
\item
  \textbf{Focusing System}: Focuses electron beam into sharp point using
  electrostatic lenses
\item
  \textbf{Deflection System}: Deflects electron beam horizontally and
  vertically using deflection plates
\item
  \textbf{Phosphor Screen}: Converts electron energy to visible light
\item
  \textbf{Glass Envelope}: Vacuum-sealed container housing all
  components
\end{itemize}

\textbf{Working:}

\begin{enumerate}
\tightlist
\item
  Electron gun emits electrons
\item
  Focusing system narrows electron beam
\item
  Deflection plates move beam across screen
\item
  Beam strikes phosphor screen creating visible trace
\end{enumerate}

\end{solutionbox}
\begin{mnemonicbox}
``GFDS: Gun-Focus-Deflect-Screen''

\end{mnemonicbox}
\subsection*{Question 3(c) [7 marks]}\label{q3c}

\textbf{Explain Construction, Block diagram, working and advantage of
DSO with necessary diagram.}

\begin{solutionbox}

Digital Storage Oscilloscope (DSO) converts analog signals to digital
form and stores them for display and analysis.

\textbf{Block Diagram:}

\begin{center}
\textbf{Mermaid Diagram (Code)}
\begin{verbatim}
{Shaded}
{Highlighting}[]
graph LR
    A[Input] {-{-}{} B[Attenuator/Amplifier]}
    B {-{-}{} C[ADC]}
    C {-{-}{} D[Memory]}
    D {-{-}{} E[Microprocessor]}
    E {-{-}{} F[DAC]}
    F {-{-}{} G[Display]}
    E {-{-}{} H[Control Panel]}
    style C fill:\#f96,stroke:\#333,stroke{-width:2px}
    style D fill:\#9cf,stroke:\#333,stroke{-width:2px}
    style E fill:\#f9f,stroke:\#333,stroke{-width:2px}
{Highlighting}
{Shaded}
\end{verbatim}
\end{center}

\textbf{Construction and Working:}

\begin{enumerate}
\tightlist
\item
  \textbf{Input Stage}: Attenuator/amplifier conditions signal
\item
  \textbf{ADC}: Converts analog signal to digital at sampling rate
\item
  \textbf{Memory}: Stores digital samples
\item
  \textbf{Microprocessor}: Controls operation and processes data
\item
  \textbf{DAC}: Converts digital data back to analog for display
\item
  \textbf{Display}: Shows waveform
\end{enumerate}

\textbf{Advantages of DSO:}

\begin{itemize}
\tightlist
\item
  Signal storage capability for later analysis
\item
  Pre-trigger viewing of signal
\item
  Single-shot signal capture
\item
  Automatic measurements and calculations
\item
  Waveform processing (FFT, averaging, etc.)
\item
  Digital interfacing (USB, Ethernet)
\item
  Higher bandwidth and sampling rates
\end{itemize}

\end{solutionbox}
\begin{mnemonicbox}
``SAMPLE:
Store-Analyze-Measure-Process-Link-Examine''

\end{mnemonicbox}
\subsection*{Question 3(a) OR [3
marks]}\label{q3a}

\textbf{In CRO vertical displacement for peak is = 1cm and volt/div knob
= 10mV. Find peak value and RMS value of voltage.}

\begin{solutionbox}

\textbf{Calculation:} Vertical displacement (peak) = 1 cm Volt/div
setting = 10 mV/div

Peak value (Vp) = Displacement \times Volt/div Vp = 1 cm \times 10 mV/div = 10 mV

For sinusoidal waveform: RMS value (Vrms) = Vp \div \sqrt2 Vrms = 10 mV \div 1.414
= 7.07 mV

Therefore, peak value = 10 mV and RMS value = 7.07 mV.

\end{solutionbox}
\begin{mnemonicbox}
``Peak-to-RMS: Divide by root-2''

\end{mnemonicbox}
\subsection*{Question 3(b) OR [4
marks]}\label{q3b}

\textbf{Explain CRO Screen in detail.}

\begin{solutionbox}

CRO (Cathode Ray Oscilloscope) screen displays waveforms and provides
measurement references.

\textbf{Diagram:}

\begin{verbatim}
+{-{-}{-}{-}{-}{-}{-}{-}{-}{-}{-}{-}{-}{-}{-}{-}{-}{-}{-}{-}{-}{-}{-}{-}{-}{-}{-}{-}{-}{-}{-}+}
|                               |
|       GRATICULE LINES         |
|   +{-{-}{-}+{-}{-}{-}+{-}{-}{-}+{-}{-}{-}+{-}{-}{-}+{-}{-}{-}+   |}
|   |   |   |   |   |   |   |   |
| {-{-}+{-}{-}{-}+{-}{-}{-}+{-}{-}{-}+{-}{-}{-}+{-}{-}{-}+{-}{-}{-}+{-}{-} |}
|   |   |   |   |   |   |   |   |
|   +{-{-}{-}+{-}{-}{-}+{-}{-}{-}+{-}{-}{-}+{-}{-}{-}+{-}{-}{-}+   |}
|   |   |   |   |   |   |   |   |
| {-{-}+{-}{-}{-}+{-}{-}{-}+{-}{-}{-}+{-}{-}{-}+{-}{-}{-}+{-}{-}{-}+{-}{-} |}
|   |   |   |   |   |   |   |   |
|   +{-{-}{-}+{-}{-}{-}+{-}{-}{-}+{-}{-}{-}+{-}{-}{-}+{-}{-}{-}+   |}
|                               |
+{-{-}{-}{-}{-}{-}{-}{-}{-}{-}{-}{-}{-}{-}{-}{-}{-}{-}{-}{-}{-}{-}{-}{-}{-}{-}{-}{-}{-}{-}{-}+}
\end{verbatim}

\textbf{Components:}

\begin{itemize}
\tightlist
\item
  \textbf{Phosphor Coating}: Converts electron energy to visible light
\item
  \textbf{Graticule}: Grid pattern for measurements
\item
  \textbf{X-Axis}: Represents time (horizontal)
\item
  \textbf{Y-Axis}: Represents voltage (vertical)
\item
  \textbf{Center Point}: Reference for measurements (0,0)
\end{itemize}

\textbf{Screen Features:}

\begin{itemize}
\tightlist
\item
  \textbf{Divisions}: Typically 8\times10 divisions for measurement
\item
  \textbf{Intensity Control}: Adjusts brightness of display
\item
  \textbf{Focus Control}: Sharpens displayed trace
\item
  \textbf{Scale Illumination}: Illuminates graticule
\end{itemize}

\end{solutionbox}
\begin{mnemonicbox}
``PAXED:
Phosphor-Axes-X-time-Y-amplitude-Equal-Divisions''

\end{mnemonicbox}
\subsection*{Question 3(c) OR [7
marks]}\label{q3c}

\textbf{Explain Measurement of Voltage, Frequency, Time delay and Phase
angle using CRO with necessary diagram.}

\begin{solutionbox}

CRO (Cathode Ray Oscilloscope) can measure various electrical parameters
accurately.

\textbf{1. Voltage Measurement:}

\begin{verbatim}
    \^{}
    |
    |   /{      /}
    |  /  {    /  }
    | /    {  /    }
 {-{-}{-}+{-}{-}{-}{-}{-}{-}{-}/{-}{-}{-}{-}{-}{-}/{-}{-} t}
    |
    |
\end{verbatim}

\textbf{Method:}

\begin{itemize}
\tightlist
\item
  Set vertical position to center line
\item
  Count vertical divisions of waveform
\item
  Multiply by V/div setting
\item
  Amplitude = Vertical divisions \times V/div
\end{itemize}

\textbf{2. Frequency Measurement:}

\begin{verbatim}
    \^{}
    |
    |   /{      /      /}
    |  /  {    /      /  }
    | /    {  /      /    }
 {-{-}{-}+{-}{-}{-}{-}{-}{-}{-}/{-}{-}{-}{-}{-}{-}/{-}{-}{-}{-}{-}{-}/{-}{-} t}
    |        {{-}T{-}}
    |
\end{verbatim}

\textbf{Method:}

\begin{itemize}
\tightlist
\item
  Measure time period (T) between similar points
\item
  Frequency = 1/T
\item
  T = Horizontal divisions \times Time/div setting
\item
  Frequency = 1/(Horizontal divisions \times Time/div)
\end{itemize}

\textbf{3. Time Delay Measurement:}

\begin{verbatim}
    \^{}
    |      Signal 1    Signal 2
    |        /{          /}
    |       /  {        /  }
    |      /    {      /    }
 {-{-}{-}+{-}{-}{-}{-}{-}/{-}{-}{-}{-}{-}{-}{-}{-}{-}{-}/{-}{-}{-}{-}{-}{-}{-}{-}{-}{-} t}
    |    /        {  /        }
    |   /          {/          }
    |  /                        {}
    | /                          {}
    |{{-}{-}{-}{-}{-}Delay Time (Td){-}{-}{-}{-}{-}{-}|}
\end{verbatim}

\textbf{Method:}

\begin{itemize}
\tightlist
\item
  Trigger on first signal
\item
  Measure horizontal distance to second signal
\item
  Time delay = Horizontal divisions \times Time/div setting
\end{itemize}

\textbf{4. Phase Angle Measurement:}

\begin{verbatim}
    \^{}
    |      Signal 1    Signal 2
    |        /{          /}
    |       /  {        /  }
    |      /    {      /    }
 {-{-}{-}+{-}{-}{-}{-}{-}/{-}{-}{-}{-}{-}{-}{-}{-}{-}{-}/{-}{-}{-}{-}{-}{-}{-}{-}{-}{-} t}
    |    /        {  /        }
    |   /          {/          }
    |  /                        {}
    | /                          {}
    |{{-}{-}{-}{-}{-}{-}{-}{-}{-}{-}{-}T{-}{-}{-}{-}{-}{-}{-}{-}{-}{-}{-}{-}{-}{-}|}
    |{{-}{-}{-}{-}Td{-}{-}{-}{-}|}
\end{verbatim}

\textbf{Method:}

\begin{itemize}
\tightlist
\item
  Measure time period (T) of one complete cycle
\item
  Measure time delay (Td) between corresponding points
\item
  Phase angle = (Td/T) \times 360^\circ
\end{itemize}

\end{solutionbox}
\begin{mnemonicbox}
``VFTP: Vertical-Frequency-Time-Phase''

\end{mnemonicbox}
\subsection*{Question 4(a) [3 marks]}\label{q4a}

\textbf{Compare active and passive transducers.}

\begin{solutionbox}

{\def\LTcaptype{none} % do not increment counter
\begin{longtable}[]{@{}
  >{\raggedright\arraybackslash}p{(\linewidth - 4\tabcolsep) * \real{0.2857}}
  >{\raggedright\arraybackslash}p{(\linewidth - 4\tabcolsep) * \real{0.3393}}
  >{\raggedright\arraybackslash}p{(\linewidth - 4\tabcolsep) * \real{0.3750}}@{}}
\toprule\noalign{}
\begin{minipage}[b]{\linewidth}\raggedright
Characteristic
\end{minipage} & \begin{minipage}[b]{\linewidth}\raggedright
Active Transducers
\end{minipage} & \begin{minipage}[b]{\linewidth}\raggedright
Passive Transducers
\end{minipage} \\
\midrule\noalign{}
\endhead
\bottomrule\noalign{}
\endlastfoot
Power source & Self-generating (no external power) & Requires external
power \\
Output & Generates energy from input & Modifies external energy \\
Examples & Thermocouple, Photovoltaic cell & Strain gauge, RTD, LVDT \\
Sensitivity & Generally lower & Generally higher \\
Response time & Faster & Slower \\
Cost & Usually less expensive & Usually more expensive \\
Complexity & Simpler & More complex \\
\end{longtable}
}

\end{solutionbox}
\begin{mnemonicbox}
``APE-GSR: Active-Produces-Energy,
Gets-Signal-Requiring-power''

\end{mnemonicbox}
\subsection*{Question 4(b) [4 marks]}\label{q4b}

\textbf{Explain Working of strain Gauge with necessary diagram in
detail. Also list application of it.}

\begin{solutionbox}

Strain gauge converts mechanical deformation to electrical resistance
change.

\textbf{Diagram:}

\begin{verbatim}
    +{-{-}{-}{-}{-}{-}{-}{-}{-}{-}{-}{-}{-}{-}{-}{-}{-}{-}{-}{-}{-}{-}{-}{-}{-}{-}{-}{-}+}
    |                            |
    |  /{//////////    |}
    | /                      {   |}
    |/                        {  |}
    |{                        /  |}
    | {                      /   |}
    |  {///////////    |}
    |                            |
    +{-{-}{-}{-}{-}{-}{-}{-}{-}{-}{-}{-}{-}{-}{-}{-}{-}{-}{-}{-}{-}{-}{-}{-}{-}{-}{-}{-}+}
\end{verbatim}

\textbf{Working:}

\begin{enumerate}
\tightlist
\item
  When a conductor is stretched, its length increases and
  cross-sectional area decreases
\item
  This causes an increase in electrical resistance: ΔR/R = GF \times ε

  \begin{itemize}
  \tightlist
  \item
    Where ΔR/R is fractional change in resistance
  \item
    GF is gauge factor (sensitivity)
  \item
    ε is strain
  \end{itemize}
\end{enumerate}

\textbf{Types:}

\begin{itemize}
\tightlist
\item
  Metal foil strain gauges
\item
  Semiconductor strain gauges
\item
  Wire strain gauges
\end{itemize}

\textbf{Applications:}

\begin{itemize}
\tightlist
\item
  Load cells for weighing systems
\item
  Structural health monitoring
\item
  Pressure sensors
\item
  Torque measurement
\item
  Mechanical stress analysis
\end{itemize}

\end{solutionbox}
\begin{mnemonicbox}
``STRAIN:
Stretch-To-Resistance-Alteration-In-Narrow-conductor''

\end{mnemonicbox}
\subsection*{Question 4(c) [7 marks]}\label{q4c}

\textbf{Explain Gas Sensor MQ2 with necessary diagram in detail.}

\begin{solutionbox}

MQ2 is a semiconductor gas sensor that detects combustible gases, smoke,
and LPG.

\textbf{Diagram:}

\begin{center}
\textbf{Mermaid Diagram (Code)}
\begin{verbatim}
{Shaded}
{Highlighting}[]
graph LR
    A[Anti{-explosion Network] {-}{-}{} B[SnO2 Sensing Element]}
    B {-{-}{} C[Heater Coil]}
    D[Electrode] {-{-}{} B}
    E[Housing] {-{-}{} A}
    E {-{-}{} B}
    E {-{-}{} C}
    E {-{-}{} D}
    style B fill:\#f96,stroke:\#333,stroke{-width:2px}
    style C fill:\#9cf,stroke:\#333,stroke{-width:2px}
{Highlighting}
{Shaded}
\end{verbatim}
\end{center}

\textbf{Construction:}

\begin{itemize}
\tightlist
\item
  \textbf{Sensing Element}: Tin dioxide (SnO2) semiconductor
\item
  \textbf{Heater}: Maintains operating temperature (around 200-400^\circC)
\item
  \textbf{Electrodes}: Measure resistance changes
\item
  \textbf{Housing}: Protects components and allows gas flow
\end{itemize}

\textbf{Working Principle:}

\begin{enumerate}
\tightlist
\item
  In clean air, sensor has high resistance
\item
  When combustible gases present, surface reactions occur
\item
  Electrons are released, decreasing resistance
\item
  Resistance decreases proportionally to gas concentration
\end{enumerate}

\textbf{Circuit Connection:}

\begin{verbatim}
    Vcc +5V
      |
      |
    +{-+{-}+     +{-}{-}{-}{-}{-}{-}{-}+}
    |   |{-{-}{-}{-}{-}|       |}
    | R |     |  MQ2  |
    |   |{-{-}{-}{-}{-}|       |}
    +{-+{-}+     +{-}{-}{-}{-}{-}{-}{-}+}
      |           |
      |           |
    Vout         GND
\end{verbatim}

\textbf{Applications:}

\begin{itemize}
\tightlist
\item
  Domestic gas leakage detectors
\item
  Industrial combustible gas alarms
\item
  Portable gas detectors
\item
  Air quality monitoring
\item
  Fire alarms
\end{itemize}

\end{solutionbox}
\begin{mnemonicbox}
``MQ2: Measures Quick-leaks of 2+ gases (LPG,
Propane)''

\end{mnemonicbox}
\subsection*{Question 4(a) OR [3
marks]}\label{q4a}

\textbf{Compare primary and secondary transducers}

\begin{solutionbox}

{\def\LTcaptype{none} % do not increment counter
\begin{longtable}[]{@{}
  >{\raggedright\arraybackslash}p{(\linewidth - 4\tabcolsep) * \real{0.2712}}
  >{\raggedright\arraybackslash}p{(\linewidth - 4\tabcolsep) * \real{0.3559}}
  >{\raggedright\arraybackslash}p{(\linewidth - 4\tabcolsep) * \real{0.3729}}@{}}
\toprule\noalign{}
\begin{minipage}[b]{\linewidth}\raggedright
Characteristic
\end{minipage} & \begin{minipage}[b]{\linewidth}\raggedright
Primary Transducers
\end{minipage} & \begin{minipage}[b]{\linewidth}\raggedright
Secondary Transducers
\end{minipage} \\
\midrule\noalign{}
\endhead
\bottomrule\noalign{}
\endlastfoot
Definition & Directly convert physical quantity to electrical signal &
Convert output of primary transducer to usable form \\
Function & First stage of conversion & Second stage of conversion \\
Examples & Thermocouple, Photocell, Piezoelectric & Amplifiers, ADCs,
Signal conditioners \\
Input & Physical parameter & Output from primary transducer \\
Output & Electrical signal & Modified electrical signal \\
Location & At sensing point & May be remote from primary transducer \\
Accuracy & Affects overall system accuracy & Further processes already
converted signal \\
\end{longtable}
}

\end{solutionbox}
\begin{mnemonicbox}
``PS-FLIP: Primary-Senses,
Secondary-Further-Level-Improves-Processing''

\end{mnemonicbox}
\subsection*{Question 4(b) OR [4
marks]}\label{q4b}

\textbf{Explain Capacitive Transducer with necessary diagram in detail.
Also list application of it.}

\begin{solutionbox}

Capacitive transducer converts physical displacement into capacitance
change which is then converted to electrical signal.

\textbf{Diagram:}

\begin{center}
\textbf{Mermaid Diagram (Code)}
\begin{verbatim}
{Shaded}
{Highlighting}[]
graph LR
    A[Fixed Plate] {-{-}{-} B[Dielectric]}
    B {-{-}{-} C[Movable Plate]}
    D[Physical Parameter] {-{-}{-} C}
    E[Circuit] {-{-}{-} A}
    E {-{-}{-} C}
    style B fill:\#f96,stroke:\#333,stroke{-width:2px}
    style C fill:\#9cf,stroke:\#333,stroke{-width:2px}
{Highlighting}
{Shaded}
\end{verbatim}
\end{center}

\textbf{Working:} Capacitance C = ε_{0}εᵣA/d Where:

\begin{itemize}
\tightlist
\item
  ε_{0} = Permittivity of free space
\item
  εᵣ = Relative permittivity of dielectric
\item
  A = Area of plates
\item
  d = Distance between plates
\end{itemize}

Capacitance changes by:

\begin{enumerate}
\tightlist
\item
  Varying distance between plates
\item
  Varying overlap area of plates
\item
  Varying dielectric constant
\end{enumerate}

\textbf{Applications:}

\begin{itemize}
\tightlist
\item
  Pressure sensors
\item
  Displacement measurements
\item
  Level indicators
\item
  Humidity sensors
\item
  Thickness measurement
\item
  Touch screens
\end{itemize}

\end{solutionbox}
\begin{mnemonicbox}
``CAPACITIVE:
Change-Area-Plates-And-Change-In-Thickness-Impacts-Value-Electrically''

\end{mnemonicbox}
\subsection*{Question 4(c) OR [7
marks]}\label{q4c}

\textbf{Explain LVDT Transducer operation, construction with necessary
diagram in detail. Also list advantage, disadvantage and application of
LVDT.}

\begin{solutionbox}

LVDT (Linear Variable Differential Transformer) is an electromagnetic
transducer that converts linear displacement to electrical signal.

\textbf{Diagram:}

\begin{center}
\textbf{Mermaid Diagram (Code)}
\begin{verbatim}
{Shaded}
{Highlighting}[]
graph LR
    A[Primary Coil] {-{-}{-} B[Core]}
    C[Secondary Coil 1] {-{-}{-} B}
    D[Secondary Coil 2] {-{-}{-} B}
    E[AC Excitation] {-{-}{-} A}
    F[Output] {-{-}{-} C}
    F {-{-}{-} D}
    style B fill:\#f96,stroke:\#333,stroke{-width:2px}
    style A fill:\#9cf,stroke:\#333,stroke{-width:2px}
{Highlighting}
{Shaded}
\end{verbatim}
\end{center}

\textbf{Construction:}

\begin{itemize}
\tightlist
\item
  \textbf{Primary Coil}: Center coil excited by AC source
\item
  \textbf{Secondary Coils}: Two coils connected in series opposition
\item
  \textbf{Core}: Ferromagnetic material that moves with measured
  displacement
\item
  \textbf{Housing}: Protects the coil assembly
\end{itemize}

\textbf{Working:}

\begin{enumerate}
\tightlist
\item
  AC excitation applied to primary coil
\item
  At null position (center), equal voltages induced in secondary coils
\item
  Moving core changes magnetic coupling
\item
  Differential voltage proportional to displacement
\item
  Phase indicates direction of movement
\end{enumerate}

\textbf{Advantages:}

\begin{itemize}
\tightlist
\item
  Non-contact operation (frictionless)
\item
  High resolution and sensitivity
\item
  Infinite resolution
\item
  Good linearity
\item
  Robust construction
\item
  Long operational life
\end{itemize}

\textbf{Disadvantages:}

\begin{itemize}
\tightlist
\item
  Requires AC excitation source
\item
  Sensitive to external magnetic fields
\item
  Larger size compared to other transducers
\item
  Higher cost
\item
  Requires signal conditioning circuit
\end{itemize}

\textbf{Applications:}

\begin{itemize}
\tightlist
\item
  Machine tool positioning
\item
  Hydraulic/pneumatic cylinder position feedback
\item
  Robotics and automation
\item
  Aircraft control systems
\item
  Structural testing
\item
  Process control systems
\end{itemize}

\end{solutionbox}
\begin{mnemonicbox}
``LVDT: Linear-Variation-Detected-Through
electromagnetic induction''

\end{mnemonicbox}
\subsection*{Question 5(a) [3 marks]}\label{q5a}

\textbf{Explain working of Thermocouple sensor with necessary diagram in
detail.}

\begin{solutionbox}

Thermocouple is a temperature sensor based on the Seebeck effect, where
junction of two dissimilar metals generates voltage proportional to
temperature difference.

\textbf{Diagram:}

\begin{center}
\textbf{Mermaid Diagram (Code)}
\begin{verbatim}
{Shaded}
{Highlighting}[]
graph LR
    A[Metal A] {-{-}{-} B((Hot Junction))}
    B {-{-}{-} C[Metal B]}
    A {-{-}{-} D((Cold Junction))}
    C {-{-}{-} D}
    D {-{-}{-} E[Voltmeter]}
    style B fill:\#f96,stroke:\#333,stroke{-width:2px}
    style D fill:\#9cf,stroke:\#333,stroke{-width:2px}
{Highlighting}
{Shaded}
\end{verbatim}
\end{center}

\textbf{Working:}

\begin{enumerate}
\tightlist
\item
  Two dissimilar metals joined at two points (hot and cold junctions)
\item
  Temperature difference between junctions creates Seebeck voltage
\item
  EMF generated is proportional to temperature difference
\item
  Voltage measured is calibrated to temperature
\end{enumerate}

\textbf{Types:}

\begin{itemize}
\tightlist
\item
  Type K (Chromel-Alumel): General purpose, -200^\circC to 1260^\circC
\item
  Type J (Iron-Constantan): -40^\circC to 750^\circC
\item
  Type T (Copper-Constantan): -250^\circC to 350^\circC
\end{itemize}

\end{solutionbox}
\begin{mnemonicbox}
``THC: Temperature-produces Hot-junction Current''

\end{mnemonicbox}
\subsection*{Question 5(b) [4 marks]}\label{q5b}

\textbf{Explain working of Digital IC tester with necessary diagram in
detail.}

\begin{solutionbox}

Digital IC Tester is used to test functionality of digital integrated
circuits by applying test vectors and analyzing responses.

\textbf{Block Diagram:}

\begin{center}
\textbf{Mermaid Diagram (Code)}
\begin{verbatim}
{Shaded}
{Highlighting}[]
graph LR
    A[Power Supply] {-{-}{} B[Control Unit]}
    B {-{-}{} C[Test Vector Generator]}
    C {-{-}{} D[IC Under Test]}
    D {-{-}{} E[Response Analyzer]}
    E {-{-}{} F[Display Unit]}
    B {-{-}{} F}
    G[User Interface] {-{-}{} B}
    style C fill:\#f96,stroke:\#333,stroke{-width:2px}
    style E fill:\#9cf,stroke:\#333,stroke{-width:2px}
{Highlighting}
{Shaded}
\end{verbatim}
\end{center}

\textbf{Working:}

\begin{enumerate}
\tightlist
\item
  IC placed in test socket with proper orientation
\item
  Test mode selected (test, multiple test, or unknown IC)
\item
  Test vectors applied to IC pins
\item
  Output responses compared with expected results
\item
  Pass/Fail indication displayed
\end{enumerate}

\textbf{Features:}

\begin{itemize}
\tightlist
\item
  Tests various IC families (TTL, CMOS, HCMOS)
\item
  Auto-detection of unknown ICs
\item
  Tests for stuck-at faults, open circuits
\item
  Multiple test patterns for thorough verification
\end{itemize}

\end{solutionbox}
\begin{mnemonicbox}
``VECTOR:
Verify-Each-Circuit-Through-Output-Response''

\end{mnemonicbox}
\subsection*{Question 5(c) [7 marks]}\label{q5c}

\textbf{Explain working of function generator with necessary diagram in
detail.}

\begin{solutionbox}

Function generator produces different waveforms (sine, square, triangle)
with adjustable frequency and amplitude.

\textbf{Block Diagram:}

\begin{center}
\textbf{Mermaid Diagram (Code)}
\begin{verbatim}
{Shaded}
{Highlighting}[]
graph LR
    A[Oscillator] {-{-}{} B[Waveshaping Circuit]}
    B {-{-}{} C[Attenuator]}
    C {-{-}{} D[Output Amplifier]}
    D {-{-}{} E[Output]}
    F[Frequency Control] {-{-}{} A}
    G[Amplitude Control] {-{-}{} C}
    H[DC Offset Control] {-{-}{} D}
    I[Waveform Selector] {-{-}{} B}
    style A fill:\#f96,stroke:\#333,stroke{-width:2px}
    style B fill:\#9cf,stroke:\#333,stroke{-width:2px}
{Highlighting}
{Shaded}
\end{verbatim}
\end{center}

\textbf{Working:}

\begin{enumerate}
\tightlist
\item
  \textbf{Oscillator}: Generates basic waveform (typically triangle)
\item
  \textbf{Waveshaping Circuit}: Converts to sine, square, or triangle
  waveforms
\item
  \textbf{Attenuator}: Controls amplitude of signal
\item
  \textbf{Output Amplifier}: Provides low output impedance and DC offset
\item
  \textbf{Controls}: Adjust frequency, amplitude, DC offset, duty cycle
\end{enumerate}

\textbf{Waveform Generation:}

\begin{itemize}
\tightlist
\item
  Triangle wave: Basic output of oscillator circuit
\item
  Square wave: Generated by comparator from triangle wave
\item
  Sine wave: Generated by waveshaping from triangle wave
\end{itemize}

\textbf{Applications:}

\begin{itemize}
\tightlist
\item
  Testing electronic circuits
\item
  Signal source for experiments
\item
  Calibration of instruments
\item
  Educational demonstrations
\item
  Frequency response testing
\end{itemize}

\end{solutionbox}
\begin{mnemonicbox}
``FAST: Frequency-Amplitude-Signal-Type control''

\end{mnemonicbox}
\subsection*{Question 5(a) OR [3
marks]}\label{q5a}

\textbf{Explain working of PH sensor with necessary diagram in detail.}

\begin{solutionbox}

PH sensor measures hydrogen ion concentration in a solution, indicating
acidity or alkalinity.

\textbf{Diagram:}

\begin{center}
\textbf{Mermaid Diagram (Code)}
\begin{verbatim}
{Shaded}
{Highlighting}[]
graph LR
    A[Glass Electrode] {-{-}{-} B[Reference Electrode]}
    A {-{-}{-} C[pH Sensitive Bulb]}
    B {-{-}{-} D[Reference Solution]}
    A {-{-}{-} E[Voltage Measurement Circuit]}
    B {-{-}{-} E}
    E {-{-}{-} F[Display]}
    style C fill:\#f96,stroke:\#333,stroke{-width:2px}
    style D fill:\#9cf,stroke:\#333,stroke{-width:2px}
{Highlighting}
{Shaded}
\end{verbatim}
\end{center}

\textbf{Working:}

\begin{enumerate}
\tightlist
\item
  Glass electrode contains buffer solution with known pH
\item
  H^{+} ions in test solution interact with glass membrane
\item
  Potential difference develops proportional to pH difference
\item
  Reference electrode provides stable comparison voltage
\item
  Voltage difference = 59.16 mV per pH unit at 25^\circC
\end{enumerate}

\textbf{Components:}

\begin{itemize}
\tightlist
\item
  Glass electrode with pH-sensitive membrane
\item
  Reference electrode (often silver/silver chloride)
\item
  Temperature compensation circuit
\item
  Signal conditioning electronics
\end{itemize}

\end{solutionbox}
\begin{mnemonicbox}
``pH-MVH: Potential-of-Hydrogen Measured by Voltage
per Hydrogen-ion concentration''

\end{mnemonicbox}
\subsection*{Question 5(b) OR [4
marks]}\label{q5b}

\textbf{Describe working of Spectrum Analyzer with necessary diagram in
detail}

\begin{solutionbox}

Spectrum Analyzer displays signal amplitude vs.~frequency, showing
frequency components of signals.

\textbf{Block Diagram:}

\begin{center}
\textbf{Mermaid Diagram (Code)}
\begin{verbatim}
{Shaded}
{Highlighting}[]
graph LR
    A[Input Signal] {-{-}{} B[Attenuator/Amplifier]}
    B {-{-}{} C[Mixer]}
    D[Local Oscillator] {-{-}{} C}
    C {-{-}{} E[IF Filter]}
    E {-{-}{} F[Envelope Detector]}
    F {-{-}{} G[Display]}
    H[Sweep Generator] {-{-}{} D}
    H {-{-}{} G}
    style C fill:\#f96,stroke:\#333,stroke{-width:2px}
    style E fill:\#9cf,stroke:\#333,stroke{-width:2px}
{Highlighting}
{Shaded}
\end{verbatim}
\end{center}

\textbf{Working:}

\begin{enumerate}
\tightlist
\item
  \textbf{Input Stage}: Attenuates or amplifies signal to optimum level
\item
  \textbf{Mixer}: Combines input with local oscillator signal
\item
  \textbf{IF Filter}: Passes only desired frequency components
\item
  \textbf{Detector}: Measures amplitude of IF signal
\item
  \textbf{Display}: Shows amplitude vs.~frequency
\end{enumerate}

\textbf{Types:}

\begin{itemize}
\tightlist
\item
  Swept-tuned spectrum analyzer
\item
  FFT (Fast Fourier Transform) spectrum analyzer
\item
  Real-time spectrum analyzer
\end{itemize}

\textbf{Applications:}

\begin{itemize}
\tightlist
\item
  Signal purity measurement
\item
  EMI/EMC testing
\item
  Modulation analysis
\item
  Communication system testing
\end{itemize}

\end{solutionbox}
\begin{mnemonicbox}
``SAFE-D:
Signal-Amplitude-Frequency-Evaluation-Display''

\end{mnemonicbox}
\subsection*{Question 5(c) OR [7
marks]}\label{q5c}

\textbf{Explain working of basic frequency counter with necessary
diagram in detail}

\begin{solutionbox}

Frequency counter measures frequency of input signal by counting cycles
in a precise time interval.

\textbf{Block Diagram:}

\begin{center}
\textbf{Mermaid Diagram (Code)}
\begin{verbatim}
{Shaded}
{Highlighting}[]
graph LR
    A[Input Signal] {-{-}{} B[Input Conditioning]}
    B {-{-}{} C[Schmitt Trigger]}
    C {-{-}{} D[Gate]}
    E[Time Base] {-{-}{} F[Control Logic]}
    F {-{-}{} D}
    D {-{-}{} G[Counter]}
    G {-{-}{} H[Display]}
    F {-{-}{} G}
    style D fill:\#f96,stroke:\#333,stroke{-width:2px}
    style E fill:\#9cf,stroke:\#333,stroke{-width:2px}
    style G fill:\#f9f,stroke:\#333,stroke{-width:2px}
{Highlighting}
{Shaded}
\end{verbatim}
\end{center}

\textbf{Working:}

\begin{enumerate}
\tightlist
\item
  \textbf{Input Conditioning}: Amplifies and shapes input signal
\item
  \textbf{Schmitt Trigger}: Converts to square wave
\item
  \textbf{Time Base}: Crystal oscillator provides accurate reference
\item
  \textbf{Gate Control}: Opens gate for precise measurement interval
\item
  \textbf{Counter}: Counts input cycles during gate open time
\item
  \textbf{Display}: Shows counted frequency
\end{enumerate}

\textbf{Measurement Process:}

\begin{itemize}
\tightlist
\item
  Signal cycles are counted during precise gate time
\item
  Gate time determined by time base oscillator
\item
  Frequency = Count / Gate time
\end{itemize}

\textbf{Accuracy Factors:}

\begin{itemize}
\tightlist
\item
  Time base stability (crystal oscillator quality)
\item
  Gate time (longer time improves resolution)
\item
  Trigger error (\pm1 count uncertainty)
\item
  Input signal conditioning quality
\end{itemize}

\textbf{Applications:}

\begin{itemize}
\tightlist
\item
  Frequency measurement in laboratories
\item
  Radio transmitter calibration
\item
  Crystal oscillator testing
\item
  Digital system clock verification
\end{itemize}

\end{solutionbox}
\begin{mnemonicbox}
``COUNT: Cycles-Over-Unit-time-Numerically-Tallied''

\end{mnemonicbox}

\end{document}
