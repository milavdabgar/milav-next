\documentclass[10pt,a4paper]{article}

% content/resources/templates/preamble.tex
\usepackage[margin=0.6in]{geometry}
\author{Milav Dabgar}
\usepackage{amsmath,amssymb,amsthm}
\usepackage{booktabs}
\usepackage{multirow}
\usepackage{xcolor}
\usepackage{tcolorbox}
\tcbuselibrary{breakable,skins}
\usepackage[colorlinks=true,linkcolor=blue]{hyperref}
\usepackage{titlesec}
\usepackage{enumitem}
\usepackage{tikz}
\usepackage{pgfplots}
\usepackage{circuitikz}
\usepackage[version=4]{mhchem}
\usepackage{longtable}
\usepackage{array}
\usepackage{float}
\usepackage{caption}
\usepackage{listings}

\lstset{
  basicstyle=\small\ttfamily,
  breaklines=true,
  breakatwhitespace=false,
  postbreak=\mbox{\textcolor{red}{$\hookrightarrow$}\space},
  float=false,
  numbers=left,
  numberstyle=\tiny\color{gray},
  numbersep=10pt,
  xleftmargin=2em,
  keywordstyle=\color{blue},
  commentstyle=\color{green!60!black},
  stringstyle=\color{purple},
  backgroundcolor=\color{gray!5},
  showstringspaces=false,
  tabsize=2,
  captionpos=b,
  keepspaces=true,
  columns=flexible
}

\pgfplotsset{compat=1.18}
\usetikzlibrary{shapes,arrows,positioning,calc,patterns,decorations.pathmorphing,decorations.markings,arrows.meta}

% Color scheme
\definecolor{headcolor}{RGB}{0,102,204}
\definecolor{keycolor}{RGB}{220,20,60}
\definecolor{solutioncolor}{RGB}{34,139,34}
\definecolor{mnemoniccolor}{RGB}{148,0,211}
\definecolor{codecolor}{RGB}{0,0,100}

% Spacing
\setlength{\parskip}{3pt}
\setlist[itemize]{nosep}
\setlist[enumerate]{nosep}

% Title formatting
\titleformat{\section}{\Large\bfseries\color{headcolor}}{\thesection}{1em}{}
\titleformat{\subsection}{\large\bfseries\color{headcolor}}{\thesubsection}{1em}{}

% Pandoc tightlist compatibility
\providecommand{\tightlist}{%
  \setlength{\itemsep}{0pt}\setlength{\parskip}{0pt}}

% Pandoc longtable compatibility
\newcounter{none}
\def\thenone{}


% content/resources/templates/gujarati-boxes.tex
\usepackage{fontspec}
\usepackage{polyglossia}

% Set Gujarati as main language (document is primarily in Gujarati)
% Note: gloss-gujarati.ldf doesn't exist in polyglossia, but it will use hyphenation patterns
\setdefaultlanguage{gujarati}
\setotherlanguage{english}

% Configure Gujarati font properly
% Use Language=Default to prevent polyglossia from trying to add language-specific features
% that don't exist for Gujarati, which causes "empty feature" warnings
\newfontfamily\gujaratifont[Script=Gujarati,AutoFakeBold=2.5,AutoFakeSlant=0.3]{Noto Sans Gujarati}
\setmainfont[Script=Gujarati,AutoFakeBold=2.5,AutoFakeSlant=0.3]{Noto Sans Gujarati}
% Use Noto Sans Gujarati for monospace to support Gujarati in text
\setmonofont[Scale=0.9]{Noto Sans Gujarati}

% Configure English to use the same font
\newfontfamily\englishfont[Script=Gujarati,AutoFakeBold=2.5,AutoFakeSlant=0.3]{Noto Sans Gujarati}

% Translations for polyglossia
\gappto\captionsgujarati{
  \renewcommand{\tablename}{કોષ્ટક}
  \renewcommand{\figurename}{આકૃતિ}
}

% Helper for TikZ nodes to ensure Gujarati font
\newcommand{\gu}[1]{{\gujaratifont #1}}

% Custom environments
\newtcolorbox{solutionbox}{
    breakable,
    enhanced,
    colback=solutioncolor!5!white,
    colframe=solutioncolor!75!black,
    fonttitle=\bfseries,
    title=જવાબ
}

\newtcolorbox{solutionboxnobreak}{
 colback=solutioncolor!5!white,
 colframe=solutioncolor!75!black,
 fonttitle=\bfseries,
 title=જવાબ
}

\newtcolorbox{keyformula}{
 breakable,
 enhanced,
 colback=keycolor!5!white,
 colframe=keycolor!75!black,
 fonttitle=\bfseries,
 title=રાસાયણિક સમીકરણ/સૂત્ર
}

\newtcolorbox{mnemonicbox}{
 breakable,
 enhanced,
 colback=mnemoniccolor!5!white,
 colframe=mnemoniccolor!75!black,
 fonttitle=\bfseries,
 title=મેમરી ટ્રીક
}


\begin{document}

\begin{center}
{\Huge\bfseries\color{headcolor} Subject Name (Gujarati)}\\[5pt]
{\LARGE 4331102 -- Winter 2023}\\[3pt]
{\large Semester 1 Study Material}\\[3pt]
{\normalsize\textit{Detailed Solutions and Explanations}}
\end{center}

\vspace{10pt}

\subsection*{પ્રશ્ન 1(અ) [3
ગુણ]}\label{uxaaauxab0uxab6uxaa8-1uxa85-3-uxa97uxaa3}

\textbf{એક્યુરેસી, રીપ્રોડ્યુસીબિબિટી અને રિપીટેબિલિટી ની વ્યાખ્યા આપો.}

\begin{solutionbox}

{\def\LTcaptype{none} % do not increment counter
\begin{longtable}[]{@{}
  >{\raggedright\arraybackslash}p{(\linewidth - 2\tabcolsep) * \real{0.3333}}
  >{\raggedright\arraybackslash}p{(\linewidth - 2\tabcolsep) * \real{0.6667}}@{}}
\toprule\noalign{}
\begin{minipage}[b]{\linewidth}\raggedright
પદ
\end{minipage} & \begin{minipage}[b]{\linewidth}\raggedright
વ્યાખ્યા
\end{minipage} \\
\midrule\noalign{}
\endhead
\bottomrule\noalign{}
\endlastfoot
\textbf{એક્યુરેસી} & માપવામાં આવતા પરિમાણની વાસ્તવિક કિંમત સાથે માપેલી કિંમતની
નજીકતા \\
\textbf{રીપ્રોડ્યુસીબિલિટી} & અલગ-અલગ પરિસ્થિતિઓમાં (અલગ ઓપરેટર, સ્થાન, સમય) એક
જ ઇનપુટ માટે એકસમાન માપ આપવાની ઉપકરણની ક્ષમતા \\
\textbf{રિપીટેબિલિટી} & એક જ પરિસ્થિતિઓમાં વારંવાર માપ લેવામાં આવે ત્યારે એક જ
ઇનપુટ માટે એકસમાન માપ આપવાની ઉપકરણની ક્ષમતા \\
\end{longtable}
}

\end{solutionbox}
\begin{mnemonicbox}
``ARR - સચોટ પરિણામો વારંવાર''

\end{mnemonicbox}
\subsection*{પ્રશ્ન 1(બ) [4
ગુણ]}\label{uxaaauxab0uxab6uxaa8-1uxaac-4-uxa97uxaa3}

\textbf{વ્હીટસ્ટોન બ્રિજની આકૃતિ દોરી અને સમજાવો.}

\begin{solutionbox}

\textbf{આકૃતિ:}

\begin{center}
\textbf{Mermaid Diagram (Code)}
\begin{verbatim}
{Shaded}
{Highlighting}[]
graph LR
    A[Supply+] {-{-}{-} R1}
    A {-{-}{-} R3}
    R1 {-{-}{-} B[Output+]}
    R3 {-{-}{-} C[Output{-}]}
    B {-{-}{-} R2}
    C {-{-}{-} R4}
    R2 {-{-}{-} D[Supply{-}]}
    R4 {-{-}{-} D}
{Highlighting}
{Shaded}
\end{verbatim}
\end{center}

{\def\LTcaptype{none} % do not increment counter
\begin{longtable}[]{@{}
  >{\raggedright\arraybackslash}p{(\linewidth - 2\tabcolsep) * \real{0.4091}}
  >{\raggedright\arraybackslash}p{(\linewidth - 2\tabcolsep) * \real{0.5909}}@{}}
\toprule\noalign{}
\begin{minipage}[b]{\linewidth}\raggedright
લક્ષણ
\end{minipage} & \begin{minipage}[b]{\linewidth}\raggedright
વિગત
\end{minipage} \\
\midrule\noalign{}
\endhead
\bottomrule\noalign{}
\endlastfoot
\textbf{રચના} & હીરા આકારમાં જોડાયેલા ચાર અવરોધકો \\
\textbf{સંતુલન શરત} & R1/R2 = R3/R4 (જ્યારે આઉટપુટ વોલ્ટેજ શૂન્ય હોય) \\
\textbf{ઉપયોગ} & અજ્ઞાત અવરોધનું ચોક્કસ માપન \\
\textbf{કાર્યપદ્ધતિ} & એક બાજુમાં અજ્ઞાત અવરોધક મૂકવામાં આવે છે, બ્રિજ સંતુલિત થાય
ત્યાં સુધી બાકીના અવરોધકો સમાયોજિત કરવામાં આવે છે \\
\end{longtable}
}

\end{solutionbox}
\begin{mnemonicbox}
``WBMP - સંતુલિત થઈને ચોક્કસ માપો''

\end{mnemonicbox}
\subsection*{પ્રશ્ન 1(ક) [7
ગુણ]}\label{uxaaauxab0uxab6uxaa8-1uxa95-7-uxa97uxaa3}

\textbf{Q મીટરનો સિદ્ધાંત સમજાવો. અને સાથે સાથે પ્રેક્ટીકલ Q મીટરની આકૃતિ દોરી અને
સમજાવો.}

\begin{solutionbox}

\textbf{Q મીટરનો સિદ્ધાંત:}

Q-મીટર શ્રેણી અનુનાદના સિદ્ધાંત પર કાર્ય કરે છે, જ્યાં Q ફેક્ટર અનુનાદ સમયે લાગુ
વોલ્ટેજની તુલનામાં કેપેસિટર પરના વોલ્ટેજના ગુણોત્તર તરીકે માપવામાં આવે છે.

\textbf{પ્રેક્ટીકલ Q મીટરની આકૃતિ:}

\begin{center}
\textbf{Mermaid Diagram (Code)}
\begin{verbatim}
{Shaded}
{Highlighting}[]
graph LR
    A[RF Oscillator] {-{-}{} B[Work Coil]}
    B {-{-}{} C[Series Circuit]}
    C {-{-}{} D[Unknown Inductor L]}
    D {-{-}{} E[Variable Capacitor C]}
    E {-{-}{} F[VTVM]}
    F {-{-}{} G[Q{-}Scale]}
{Highlighting}
{Shaded}
\end{verbatim}
\end{center}

{\def\LTcaptype{none} % do not increment counter
\begin{longtable}[]{@{}ll@{}}
\toprule\noalign{}
ઘટક & કાર્ય \\
\midrule\noalign{}
\endhead
\bottomrule\noalign{}
\endlastfoot
\textbf{RF ઓસિલેટર} & ચલ આવૃત્તિ સિગ્નલ પૂરા પાડે છે \\
\textbf{વર્ક કોઇલ} & ટેસ્ટ સર્કિટમાં ઇન્ડક્ટિવલી સિગ્નલ જોડે છે \\
\textbf{અનુનાદ સર્કિટ} & ચલ કેપેસિટર C સાથે ટેસ્ટ ઇન્ડક્ટર L શ્રેણીમાં \\
\textbf{VTVM} & કેપેસિટર પરના વોલ્ટેજને માપે છે \\
\textbf{Q-સ્કેલ} & સીધો Q મૂલ્ય વાંચવા માટે અંશાંકિત \\
\end{longtable}
}

\begin{itemize}
\tightlist
\item
  \textbf{અનુનાદ સૂત્ર}: f = 1/(2π\sqrtLC)
\item
  \textbf{Q ગણતરી}: Q = Vc/Vs (કેપેસિટર પરનું વોલ્ટેજ / સ્રોત વોલ્ટેજ)
\end{itemize}

\end{solutionbox}
\begin{mnemonicbox}
``RIVQ - અનુનાદ મૂલ્યવાન ગુણવત્તા દર્શાવે છે''

\end{mnemonicbox}
\subsection*{પ્રશ્ન 1(ક OR) [7
ગુણ]}\label{uxaaauxab0uxab6uxaa8-1uxa95-or-7-uxa97uxaa3}

\textbf{મુવિંગ કોઈલ ટાઈપ ઇન્સ્ટ્રુમેન્ટની રચના દોરો અનેસમજાવો.}

\begin{solutionbox}

\textbf{આકૃતિ:}

\begin{verbatim}
            +{-{-}{-}{-}{-}{-}{-}{-}{-}+}
            |    N    |
            |         |
            |         |
   +{-{-}{-}{-}{-}{-}{-}{-}+{-}+     +{-}+{-}{-}{-}{-}{-}{-}{-}{-}+}
   |    |     |     |     |    |
   |    |     |     |     |    |
   |    |  S  |     |  S  |    |
   |    |     |     |     |    |
   +{-{-}{-}{-}+     |     |     +{-}{-}{-}{-}+}
        |     +{-{-}{-}{-}{-}+     |}
        |      Coil       |
        |                 |
        +{-{-}{-}{-}{-}{-}{-}{-}{-}{-}{-}{-}{-}{-}{-}{-}{-}+}
\end{verbatim}

{\def\LTcaptype{none} % do not increment counter
\begin{longtable}[]{@{}ll@{}}
\toprule\noalign{}
ઘટક & વિગત \\
\midrule\noalign{}
\endhead
\bottomrule\noalign{}
\endlastfoot
\textbf{કાયમી ચુંબક} & મજબૂત ચુંબકીય ક્ષેત્ર બનાવે છે \\
\textbf{મુવિંગ કોઇલ} & એલ્યુમિનિયમ ફ્રેમ પર વીંટળાયેલી હળવી કોઇલ \\
\textbf{સ્પ્રિંગ્સ} & નિયંત્રિત બળ પૂરું પાડે છે અને વીજળીક જોડાણો બનાવે છે \\
\textbf{પોઇન્ટર} & કોઇલ સાથે જોડાયેલ, અંશાંકિત સ્કેલ પર ગતિ કરે છે \\
\textbf{કોર} & ચુંબકીય પ્રવાહને કેન્દ્રિત કરવા માટે નરમ લોખંડનો નળાકાર કોર \\
\end{longtable}
}

\begin{itemize}
\tightlist
\item
  \textbf{કાર્ય સિદ્ધાંત}: વળાંક બળ = BIlN (B-ક્ષેત્ર તીવ્રતા, I-વીજપ્રવાહ,
  l-લંબાઈ, N-આંટા)
\item
  \textbf{નિયંત્રિત બળ}: વળાંક ખૂણા પ્રમાણે સ્પ્રિંગ્સ દ્વારા પ્રદાન કરાયેલ
\end{itemize}

\end{solutionbox}
\begin{mnemonicbox}
``MAPS-C: ચુંબક ક્રિયા કરે છે, પોઇન્ટર વીજપ્રવાહ બતાવે છે''

\end{mnemonicbox}
\subsection*{પ્રશ્ન 2(અ) [3
ગુણ]}\label{uxaaauxab0uxab6uxaa8-2uxa85-3-uxa97uxaa3}

\textbf{અલગ અલગ પ્રકારની એરરની યાદી બનાવો અને કોઈપણ બે સમજાવો.}

\begin{solutionbox}

{\def\LTcaptype{none} % do not increment counter
\begin{longtable}[]{@{}l@{}}
\toprule\noalign{}
એરર ના પ્રકાર \\
\midrule\noalign{}
\endhead
\bottomrule\noalign{}
\endlastfoot
\textbf{ગ્રોસ એરર (મોટી ભૂલો)} \\
\textbf{સિસ્ટેમેટિક એરર (પદ્ધતિસરની ભૂલો)} \\
\textbf{રેન્ડમ એરર (અનિયમિત ભૂલો)} \\
\textbf{પર્યાવરણીય એરર} \\
\textbf{લોડિંગ એરર} \\
\end{longtable}
}

\textbf{બે એરર ની સમજૂતી:}

\begin{enumerate}
\tightlist
\item
  \textbf{સિસ્ટેમેટિક એરર}:

  \begin{itemize}
  \tightlist
  \item
    વાસ્તવિક મૂલ્યથી સાતત્યપૂર્ણ અને અનુમાનિત વિચલન
  \item
    ઉપકરણ અંશાંકન, ડિઝાઇન, અથવા પદ્ધતિને કારણે થાય છે
  \end{itemize}
\item
  \textbf{રેન્ડમ એરર}:

  \begin{itemize}
  \tightlist
  \item
    માપનમાં અણધારી વિવિધતાઓ
  \item
    નોઇઝ, પર્યાવરણીય ફેરફારો, અથવા નિરીક્ષકની મર્યાદાઓને કારણે થાય છે
  \end{itemize}
\end{enumerate}

\end{solutionbox}
\begin{mnemonicbox}
``GSREL - સારી પદ્ધતિઓ ભૂલ સ્તર ઘટાડે છે''

\end{mnemonicbox}
\subsection*{પ્રશ્ન 2(બ) [4
ગુણ]}\label{uxaaauxab0uxab6uxaa8-2uxaac-4-uxa97uxaa3}

\textbf{મેક્સવેલ બ્રિજ દોરો અને સમજાવો.}

\begin{solutionbox}

\textbf{આકૃતિ:}

\begin{center}
\textbf{Mermaid Diagram (Code)}
\begin{verbatim}
{Shaded}
{Highlighting}[]
graph LR
    A[Supply] {-{-}{-} R1}
    A {-{-}{-} R3}
    R1 {-{-}{-} B[Detector]}
    R3 {-{-}{-} C[Detector]}
    B {-{-}{-} R2}
    C {-{-}{-} R4}
    B {-{-}{-} L["Unknown L"]}
    C {-{-}{-} C1["Capacitor C"]}
    R2 {-{-}{-} D[Ground]}
    R4 {-{-}{-} D}
    L {-{-}{-} D}
    C1 {-{-}{-} D}
{Highlighting}
{Shaded}
\end{verbatim}
\end{center}

{\def\LTcaptype{none} % do not increment counter
\begin{longtable}[]{@{}ll@{}}
\toprule\noalign{}
ઘટક & કાર્ય \\
\midrule\noalign{}
\endhead
\bottomrule\noalign{}
\endlastfoot
\textbf{R1, R2, R3, R4} & બ્રિજના બાહુઓમાં ચોકસાઈપૂર્ણ અવરોધકો \\
\textbf{અજ્ઞાત L} & માપવાના અવરોધ સાથેનો ઇન્ડક્ટર \\
\textbf{કેપેસિટર C} & સામેની બાજુમાં પ્રમાણભૂત કેપેસિટર \\
\textbf{ડિટેક્ટર} & નલ ડિટેક્ટર (ગેલ્વેનોમીટર) \\
\end{longtable}
}

\begin{itemize}
\tightlist
\item
  \textbf{સંતુલન સમીકરણ}: L = CR2R3
\item
  \textbf{અવરોધ સમીકરણ}: RL = R2R3/R4
\item
  \textbf{ઉપયોગ}: નોંધપાત્ર અવરોધ સાથેના ઇન્ડક્ટન્સનું માપન
\end{itemize}

\end{solutionbox}
\begin{mnemonicbox}
``MBLR - મેક્સવેલ બ્રિજ અવરોધને જોડે છે''

\end{mnemonicbox}
\subsection*{પ્રશ્ન 2(ક) [7
ગુણ]}\label{uxaaauxab0uxab6uxaa8-2uxa95-7-uxa97uxaa3}

\textbf{મુવિંગ આયર્ન ટાઈપ ઇન્સ્ટ્રુમેન્ટની રચના દોરો અનેસમજાવો.}

\begin{solutionbox}

\textbf{આકૃતિ:}

\begin{verbatim}
     +{-{-}{-}{-}{-}{-}{-}{-}{-}{-}{-}{-}{-}{-}{-}{-}{-}{-}{-}{-}{-}+}
     |                     |
     |     +{-{-}{-}{-}{-}{-}{-}{-}{-}{-}+    |}
     |     |          |    |
     |     |   Coil   |    |
     |     |          |    |
     |     +{-{-}{-}{-}{-}{-}{-}{-}{-}{-}+    |}
     |          ||         |
     |    +{-{-}{-}{-}{-}++{-}{-}{-}{-}{-}{-}+  |}
     |    |     ||      |  |
     |    |  Iron Vanes |  |
     |    |             |  |
     |    +{-{-}{-}{-}{-}{-}{-}{-}{-}{-}{-}{-}{-}+  |}
     |                     |
     +{-{-}{-}{-}{-}{-}{-}{-}{-}{-}{-}{-}{-}{-}{-}{-}{-}{-}{-}{-}{-}+}
\end{verbatim}

{\def\LTcaptype{none} % do not increment counter
\begin{longtable}[]{@{}
  >{\raggedright\arraybackslash}p{(\linewidth - 2\tabcolsep) * \real{0.4583}}
  >{\raggedright\arraybackslash}p{(\linewidth - 2\tabcolsep) * \real{0.5417}}@{}}
\toprule\noalign{}
\begin{minipage}[b]{\linewidth}\raggedright
ઘટક
\end{minipage} & \begin{minipage}[b]{\linewidth}\raggedright
વિગત
\end{minipage} \\
\midrule\noalign{}
\endhead
\bottomrule\noalign{}
\endlastfoot
\textbf{કોઇલ} & માપન કરવાના વીજપ્રવાહને વહન કરતી સ્થિર કોઇલ \\
\textbf{આયર્ન વેન્સ} & બે નરમ લોખંડના ટુકડા (એક સ્થિર, એક ગતિશીલ) \\
\textbf{પોઇન્ટર} & ગતિશીલ વેન સાથે જોડાયેલ \\
\textbf{કંટ્રોલ સ્પ્રિંગ} & અવરોધિત બળ પૂરું પાડે છે \\
\textbf{ડેમ્પિંગ મિકેનિઝમ} & હલકા એલ્યુમિનિયમ પિસ્ટનનો ઉપયોગ કરીને હવાના ઘર્ષણ
દ્વારા ડેમ્પિંગ \\
\end{longtable}
}

\begin{itemize}
\tightlist
\item
  \textbf{કાર્ય સિદ્ધાંત}: જ્યારે કોઇલમાંથી વીજપ્રવાહ પસાર થાય છે, ત્યારે બંને
  લોખંડના ટુકડા સમાન ધ્રુવતા સાથે ચુંબકિત થાય છે જેના કારણે વિકર્ષણ થાય છે
\item
  \textbf{ફાયદા}: AC અને DC બંને માટે યોગ્ય, મજબૂત બાંધકામ
\item
  \textbf{ગેરફાયદા}: બિન-સમાન સ્કેલ, PMMC કરતાં વધુ વીજ વપરાશ
\end{itemize}

\end{solutionbox}
\begin{mnemonicbox}
``IRAM - આયર્ન વિકર્ષણ ગતિ સક્રિય કરે છે''

\end{mnemonicbox}
\subsection*{પ્રશ્ન 2(અ OR) [3
ગુણ]}\label{uxaaauxab0uxab6uxaa8-2uxa85-or-3-uxa97uxaa3}

\textbf{બેસિક ડીસી વોલ્ટમીટર સમજાવો.}

\begin{solutionbox}

\textbf{આકૃતિ:}

\begin{verbatim}
  +{-{-}{-}{-}{-}{-}{-}+    +{-}{-}{-}{-}{-}{-}{-}{-}{-}+    +{-}{-}{-}{-}{-}{-}{-}{-}{-}{-}{-}+}
  | PMMC  |{-{-}{-}| Series  |{-}{-}{-}| Scale     |}
  | Meter |    | Resistor|    | Calibrated|
  +{-{-}{-}{-}{-}{-}{-}+    +{-}{-}{-}{-}{-}{-}{-}{-}{-}+    +{-}{-}{-}{-}{-}{-}{-}{-}{-}{-}{-}+}
\end{verbatim}

{\def\LTcaptype{none} % do not increment counter
\begin{longtable}[]{@{}ll@{}}
\toprule\noalign{}
ઘટક & કાર્ય \\
\midrule\noalign{}
\endhead
\bottomrule\noalign{}
\endlastfoot
\textbf{PMMC મૂવમેન્ટ} & મૂળભૂત વીજપ્રવાહ-સંવેદનશીલ મૂવમેન્ટ \\
\textbf{મલ્ટિપ્લાયર રેઝિસ્ટર} & ઉચ્ચ-મૂલ્યનો શ્રેણી અવરોધક \\
\textbf{સ્કેલ} & સીધા વોલ્ટેજ વાંચવા માટે અંશાંકિત \\
\end{longtable}
}

\begin{itemize}
\tightlist
\item
  \textbf{કાર્ય સિદ્ધાંત}: વોલ્ટમીટર શ્રેણી અવરોધક સાથેનું PMMC મીટર છે
\item
  \textbf{ગણતરી}: Rs = (V/Im) - Rm (Rs=શ્રેણી અવરોધક, V=વોલ્ટેજ, Im=પૂર્ણ સ્કેલ
  વીજપ્રવાહ, Rm=મીટર અવરોધ)
\end{itemize}

\end{solutionbox}
\begin{mnemonicbox}
``SVM - શ્રેણી વોલ્ટેજ માપન''

\end{mnemonicbox}
\subsection*{પ્રશ્ન 2(બ OR) [4
ગુણ]}\label{uxaaauxab0uxab6uxaa8-2uxaac-or-4-uxa97uxaa3}

\textbf{શેરિંગ બ્રિજ દોરો અને સમજાવો.}

\begin{solutionbox}

\textbf{આકૃતિ:}

\begin{center}
\textbf{Mermaid Diagram (Code)}
\begin{verbatim}
{Shaded}
{Highlighting}[]
graph LR
    A[AC Supply] {-{-}{-} C1["Unknown Capacitance"]}
    A {-{-}{-} R3}
    C1 {-{-}{-} B[Detector]}
    R3 {-{-}{-} C[Detector]}
    B {-{-}{-} R1}
    C {-{-}{-} C4["Standard C"]}
    R1 {-{-}{-} D[Ground]}
    C4 {-{-}{-} R4["Variable R"]}
    R4 {-{-}{-} D}
{Highlighting}
{Shaded}
\end{verbatim}
\end{center}

{\def\LTcaptype{none} % do not increment counter
\begin{longtable}[]{@{}ll@{}}
\toprule\noalign{}
ઘટક & કાર્ય \\
\midrule\noalign{}
\endhead
\bottomrule\noalign{}
\endlastfoot
\textbf{C1} & અજ્ઞાત કેપેસિટર (લોસ સાથે) \\
\textbf{R1} & C1 માં લોસનું પ્રતિનિધિત્વ કરતો અવરોધ \\
\textbf{R3, R4} & ચોકસાઈપૂર્ણ અવરોધકો \\
\textbf{C4} & પ્રમાણભૂત લોસ-ફ્રી કેપેસિટર \\
\textbf{ડિટેક્ટર} & નલ સૂચક \\
\end{longtable}
}

\begin{itemize}
\tightlist
\item
  \textbf{સંતુલન સમીકરણ}: C1 = C4(R3/R1)
\item
  \textbf{વિસર્જન ફેક્ટર}: D = ωC1R1 = ωC4R4
\item
  \textbf{ઉપયોગ}: કેપેસિટન્સ અને ડાયલેક્ટ્રિક લોસનું માપન
\end{itemize}

\end{solutionbox}
\begin{mnemonicbox}
``SCDR - શેરિંગ કેપેસિટન્સ અવરોધ નક્કી કરે છે''

\end{mnemonicbox}
\subsection*{પ્રશ્ન 2(ક OR) [7
ગુણ]}\label{uxaaauxab0uxab6uxaa8-2uxa95-or-7-uxa97uxaa3}

\textbf{ઇલેક્ટ્રોનિક મલ્ટીમીટર ઉપર ટૂંકનોંધ લખો.}

\begin{solutionbox}

\textbf{આકૃતિ:}

\begin{center}
\textbf{Mermaid Diagram (Code)}
\begin{verbatim}
{Shaded}
{Highlighting}[]
graph LR
    A[Input] {-{-}{} B[Attenuator/Range Selector]}
    B {-{-}{} C[Signal Converter]}
    C {-{-}{} D[Amplifier]}
    D {-{-}{} E[Rectifier/Detector]}
    E {-{-}{} F[Display]}
{Highlighting}
{Shaded}
\end{verbatim}
\end{center}

{\def\LTcaptype{none} % do not increment counter
\begin{longtable}[]{@{}
  >{\raggedright\arraybackslash}p{(\linewidth - 2\tabcolsep) * \real{0.4091}}
  >{\raggedright\arraybackslash}p{(\linewidth - 2\tabcolsep) * \real{0.5909}}@{}}
\toprule\noalign{}
\begin{minipage}[b]{\linewidth}\raggedright
લક્ષણ
\end{minipage} & \begin{minipage}[b]{\linewidth}\raggedright
વિગત
\end{minipage} \\
\midrule\noalign{}
\endhead
\bottomrule\noalign{}
\endlastfoot
\textbf{કાર્યો} & વોલ્ટેજ (AC/DC), વીજપ્રવાહ (AC/DC), અવરોધ, અને અન્ય પરિમાણોનું
માપન કરે છે \\
\textbf{સંવેદનશીલતા} & એનાલોગ મીટર કરતાં વધુ સંવેદનશીલતા (સામાન્ય રીતે 10MΩ ઇનપુટ
ઇમ્પીડન્સ) \\
\textbf{રેન્જ} & ઘણી પસંદ કરી શકાય તેવી માપન રેન્જ \\
\textbf{ચોકસાઈ} & ગુણવત્તા અને પરિમાણ પર આધારિત 0.1\% થી 3\% \\
\textbf{ડિસ્પ્લે} & ડિજિટલ રીડઆઉટ અથવા એનાલોગ પોઇન્ટર \\
\end{longtable}
}

\begin{itemize}
\tightlist
\item
  \textbf{પ્રકાર}: એનાલોગ ઇલેક્ટ્રોનિક મલ્ટીમીટર, ડિજિટલ મલ્ટીમીટર (DMM)
\item
  \textbf{ફાયદા}: ઉચ્ચ ઇનપુટ ઇમ્પીડન્સ, ન્યૂનતમ લોડિંગ અસર, ઘણા કાર્યો
\item
  \textbf{મુખ્ય સર્કિટ}: ઇનપુટ એટેન્યુએટર, સિગ્નલ કન્વર્ટર, એમ્પ્લિફાયર, રેક્ટિફાયર,
  ડિસ્પ્લે ડ્રાઇવર
\end{itemize}

\end{solutionbox}
\begin{mnemonicbox}
``VCAR-D: વોલ્ટેજ, વીજપ્રવાહ અને અવરોધ - પ્રદર્શિત''

\end{mnemonicbox}
\subsection*{પ્રશ્ન 3(અ) [3
ગુણ]}\label{uxaaauxab0uxab6uxaa8-3uxa85-3-uxa97uxaa3}

\textbf{CRO ના અલગ અલગ પ્રોબ્સ સમજાવો.}

\begin{solutionbox}

{\def\LTcaptype{none} % do not increment counter
\begin{longtable}[]{@{}ll@{}}
\toprule\noalign{}
પ્રોબના પ્રકાર & વિગત \\
\midrule\noalign{}
\endhead
\bottomrule\noalign{}
\endlastfoot
\textbf{પેસિવ પ્રોબ (1X)} & સીધા જોડાણ પ્રોબ, કોઈ ઘટાડો નહીં \\
\textbf{પેસિવ પ્રોબ (10X)} & સિગ્નલને 10 ગણો ઘટાડે છે, સર્કિટ લોડિંગ ઘટાડે છે \\
\textbf{એક્ટિવ પ્રોબ} & ઉચ્ચ ઇમ્પીડન્સ અને ઓછા કેપેસિટન્સ માટે એક્ટિવ ઘટકો ધરાવે
છે \\
\textbf{કરંટ પ્રોબ} & ચુંબકીય ક્ષેત્ર દ્વારા વીજપ્રવાહ માપે છે \\
\end{longtable}
}

\begin{itemize}
\tightlist
\item
  \textbf{પસંદગીના માપદંડ}: બેન્ડવિડ્થ, લોડિંગ ઇફેક્ટ, માપન રેન્જ
\item
  \textbf{કોમ્પેન્સેશન}: સચોટ વેવફોર્મ માટે 10X પ્રોબ્સને કોમ્પેન્સેશન એડજસ્ટમેન્ટની જરૂર
  પડે છે
\end{itemize}

\end{solutionbox}
\begin{mnemonicbox}
``PAC-S: પ્રોબ્સ સર્કિટ સેન્સિંગની મંજૂરી આપે છે''

\end{mnemonicbox}
\subsection*{પ્રશ્ન 3(બ) [4
ગુણ]}\label{uxaaauxab0uxab6uxaa8-3uxaac-4-uxa97uxaa3}

\textbf{ક્લેમ્પોન મીટરની રચના દોરો અને સમજાવો.}

\begin{solutionbox}

\textbf{આકૃતિ:}

\begin{verbatim}
       +{-{-}{-}{-}{-}{-}{-}{-}{-}{-}{-}{-}{-}{-}{-}+}
       |    Display    |
       +{-{-}{-}{-}{-}{-}{-}{-}{-}{-}{-}{-}{-}{-}{-}+}
       |               |
       |    Circuit    |
       |               |
    +{-{-}+               +{-}{-}+}
    |  |               |  |
    |  +{-{-}{-}{-}{-}{-}{-}{-}{-}{-}{-}{-}{-}{-}{-}+  |}
    |                     |
    |      +{-{-}{-}{-}{-}{-}{-}+      |}
    |      |       |      |
    +{-{-}{-}{-}{-}{-}+       +{-}{-}{-}{-}{-}{-}+}
           | Wire  |
           +{-{-}{-}{-}{-}{-}{-}+}
\end{verbatim}

{\def\LTcaptype{none} % do not increment counter
\begin{longtable}[]{@{}ll@{}}
\toprule\noalign{}
ઘટક & કાર્ય \\
\midrule\noalign{}
\endhead
\bottomrule\noalign{}
\endlastfoot
\textbf{સ્પ્લિટ કોર CT} & વાહક ચારે બાજુ ક્લેમ્પ કરતું ફેરાઇટ કોર \\
\textbf{કોઇલ વાઇન્ડિંગ} & પ્રેરિત વીજપ્રવાહ ઉત્પન્ન કરતી સેકન્ડરી વાઇન્ડિંગ \\
\textbf{સિગ્નલ સર્કિટરી} & વીજપ્રવાહને માપી શકાય તેવા સિગ્નલમાં રૂપાંતરિત કરે છે \\
\textbf{ડિસ્પ્લે યુનિટ} & એમ્પ્સમાં અંશાંકિત ડિજિટલ/એનાલોગ ડિસ્પ્લે \\
\textbf{ટ્રિગર મિકેનિઝમ} & વાહક આસપાસ કોર ખોલે/બંધ કરે છે \\
\end{longtable}
}

\begin{itemize}
\tightlist
\item
  \textbf{કાર્ય સિદ્ધાંત}: કરંટ ટ્રાન્સફોર્મર પર આધારિત, સર્કિટ તોડ્યા વિના
  વીજપ્રવાહ માપે છે
\item
  \textbf{ઉપયોગો}: લાઇવ વાહકોમાં AC વીજપ્રવાહને સુરક્ષિત રીતે માપવો
\end{itemize}

\end{solutionbox}
\begin{mnemonicbox}
``CAMP - ચુંબકીય સિદ્ધાંત દ્વારા વીજપ્રવાહનું વિશ્લેષણ''

\end{mnemonicbox}
\subsection*{પ્રશ્ન 3(ક) [7
ગુણ]}\label{uxaaauxab0uxab6uxaa8-3uxa95-7-uxa97uxaa3}

\textbf{સક્સેસિવ એપ્રોક્સિમેશન ટાઈપ DVM ઉપર ટૂંક નોંધ લખો.}

\begin{solutionbox}

\textbf{બ્લોક ડાયાગ્રામ:}

\begin{center}
\textbf{Mermaid Diagram (Code)}
\begin{verbatim}
{Shaded}
{Highlighting}[]
graph LR
    A[Input] {-{-}{} B[Sample \& Hold]}
    B {-{-}{} C[Comparator]}
    C {-{-}{} D[SAR Logic]}
    D {-{-}{} E[DAC]}
    E {-{-}{} C}
    D {-{-}{} F[Digital Display]}
{Highlighting}
{Shaded}
\end{verbatim}
\end{center}

{\def\LTcaptype{none} % do not increment counter
\begin{longtable}[]{@{}ll@{}}
\toprule\noalign{}
ઘટક & કાર્ય \\
\midrule\noalign{}
\endhead
\bottomrule\noalign{}
\endlastfoot
\textbf{સેમ્પલ એન્ડ હોલ્ડ} & ઇનપુટ વોલ્ટેજને પકડે અને જાળવે છે \\
\textbf{કમ્પેરેટર} & ઇનપુટને DAC આઉટપુટ સાથે સરખાવે છે \\
\textbf{સક્સેસિવ એપ્રોક્સિમેશન રજિસ્ટર} & બાઇનરી સર્ચ એલ્ગોરિધમને નિયંત્રિત કરે છે \\
\textbf{D/A કન્વર્ટર} & તુલના માટે એનાલોગ વોલ્ટેજ ઉત્પન્ન કરે છે \\
\textbf{ડિજિટલ ડિસ્પ્લે} & માપેલી કિંમત બતાવે છે \\
\end{longtable}
}

\begin{itemize}
\tightlist
\item
  \textbf{કાર્ય સિદ્ધાંત}: એનાલોગ ઇનપુટને મેળ ખાતી ડિજિટલ કિંમત શોધવા બાઇનરી
  સર્ચ એલ્ગોરિધમનો ઉપયોગ કરે છે
\item
  \textbf{રૂપાંતરનો સમય}: ઇનપુટના કદની પરવા કર્યા વિના નિશ્ચિત (8-16 બિટ માટે
  8-16 ક્લોક સાયકલ)
\item
  \textbf{ફાયદા}: મધ્યમ ગતિ, સારી રિઝોલ્યુશન, સાતત્યપૂર્ણ રૂપાંતરનો સમય
\item
  \textbf{ઉપયોગો}: સામાન્ય હેતુના માપન જ્યાં મધ્યમ ગતિ પૂરતી છે
\end{itemize}

\end{solutionbox}
\begin{mnemonicbox}
``SACD - સેમ્પલ, એપ્રોક્સિમેટ, કમ્પેર, ડિસ્પ્લે''

\end{mnemonicbox}
\subsection*{પ્રશ્ન 3(અ OR) [3
ગુણ]}\label{uxaaauxab0uxab6uxaa8-3uxa85-or-3-uxa97uxaa3}

\textbf{PH સેન્સર સમજાવો.}

\begin{solutionbox}

\textbf{આકૃતિ:}

\begin{verbatim}
    +{-{-}{-}{-}{-}{-}{-}{-}{-}{-}{-}{-}{-}{-}{-}{-}{-}+}
    | Glass Electrode |{-{-}{-}+}
    +{-{-}{-}{-}{-}{-}{-}{-}{-}{-}{-}{-}{-}{-}{-}{-}{-}+   |}
                          |
    +{-{-}{-}{-}{-}{-}{-}{-}{-}{-}{-}{-}{-}{-}{-}{-}{-}+   |}
    | Reference       |{-{-}{-}+{-}{-}{-}{-}{-} Output}
    | Electrode       |   |
    +{-{-}{-}{-}{-}{-}{-}{-}{-}{-}{-}{-}{-}{-}{-}{-}{-}+   |}
                          |
    +{-{-}{-}{-}{-}{-}{-}{-}{-}{-}{-}{-}{-}{-}{-}{-}{-}+   |}
    | Temperature     |{-{-}{-}+}
    | Compensation    |
    +{-{-}{-}{-}{-}{-}{-}{-}{-}{-}{-}{-}{-}{-}{-}{-}{-}+}
\end{verbatim}

{\def\LTcaptype{none} % do not increment counter
\begin{longtable}[]{@{}ll@{}}
\toprule\noalign{}
ઘટક & કાર્ય \\
\midrule\noalign{}
\endhead
\bottomrule\noalign{}
\endlastfoot
\textbf{ગ્લાસ ઇલેક્ટ્રોડ} & હાઇડ્રોજન આયન સાંદ્રતા પ્રત્યે સંવેદનશીલ \\
\textbf{રેફરન્સ ઇલેક્ટ્રોડ} & સ્થિર સંદર્ભ પોટેન્શિયલ પ્રદાન કરે છે \\
\textbf{તાપમાન સેન્સર} & તાપમાનની અસરો માટે વળતર આપે છે \\
\textbf{સિગ્નલ કન્ડિશનર} & મિલિવોલ્ટ સિગ્નલને એમ્પ્લિફાય અને પ્રોસેસ કરે છે \\
\end{longtable}
}

\begin{itemize}
\tightlist
\item
  \textbf{કાર્ય સિદ્ધાંત}: હાઇડ્રોજન આયન સાંદ્રતાના પ્રમાણમાં વોલ્ટેજ ઉત્પન્ન કરે છે
\item
  \textbf{આઉટપુટ}: 25^\circC પર દર pH એકમ દીઠ \textasciitilde59 mV
\item
  \textbf{રેન્જ}: 0-14 pH સ્કેલ (એસિડિક થી આલ્કલાઇન)
\end{itemize}

\end{solutionbox}
\begin{mnemonicbox}
``PHRV - pH વોલ્ટેજ સાથે સંબંધિત છે''

\end{mnemonicbox}
\subsection*{પ્રશ્ન 3(બ OR) [4
ગુણ]}\label{uxaaauxab0uxab6uxaa8-3uxaac-or-4-uxa97uxaa3}

\textbf{ઇલેક્ટ્રોનિક વોટ મીટરની રચના દોરો અને સમજાવો.}

\begin{solutionbox}

\textbf{બ્લોક ડાયાગ્રામ:}

\begin{center}
\textbf{Mermaid Diagram (Code)}
\begin{verbatim}
{Shaded}
{Highlighting}[]
graph LR
    A[Current Input] {-{-}{} B[Current Transformer]}
    C[Voltage Input] {-{-}{} D[Voltage Transformer]}
    B {-{-}{} E[Multiplier Circuit]}
    D {-{-}{} E}
    E {-{-}{} F[Integrator]}
    F {-{-}{} G[Digital Display]}
{Highlighting}
{Shaded}
\end{verbatim}
\end{center}

{\def\LTcaptype{none} % do not increment counter
\begin{longtable}[]{@{}ll@{}}
\toprule\noalign{}
ઘટક & કાર્ય \\
\midrule\noalign{}
\endhead
\bottomrule\noalign{}
\endlastfoot
\textbf{કરંટ સેન્સર} & CT અથવા શન્ટ દ્વારા લોડ કરંટ માપે છે \\
\textbf{વોલ્ટેજ સેન્સર} & પોટેન્શિયલ ડિવાઇડર દ્વારા વોલ્ટેજ માપે છે \\
\textbf{મલ્ટિપ્લાયર} & ક્ષણિક વોલ્ટેજ અને વીજપ્રવાહને ગુણાકાર કરે છે \\
\textbf{ઇન્ટિગ્રેટર} & સમય પર પાવરની સરેરાશ લે છે \\
\textbf{ડિસ્પ્લે} & વોટ્સમાં ડિજિટલ રીડઆઉટ \\
\end{longtable}
}

\begin{itemize}
\tightlist
\item
  \textbf{કાર્ય સિદ્ધાંત}: પાવર = V \times I \times cosθ (cosθ એ પાવર ફેક્ટર છે)
\item
  \textbf{ફાયદા}: ઉચ્ચ ચોકસાઈ, વિશાળ શ્રેણી, ડિજિટલ ડિસ્પ્લે
\item
  \textbf{પ્રકાર}: ટ્રુ RMS, એવરેજ સેન્સિંગ
\end{itemize}

\end{solutionbox}
\begin{mnemonicbox}
``VIMP - વોલ્ટેજ અને તીવ્રતા પાવર બનાવે છે''

\end{mnemonicbox}
\subsection*{પ્રશ્ન 3(ક OR) [7
ગુણ]}\label{uxaaauxab0uxab6uxaa8-3uxa95-or-7-uxa97uxaa3}

\textbf{ઇન્ટીગ્રેટિંગ ટાઈપ DVM ઉપર ટૂંક નોંધ લખો.}

\begin{solutionbox}

\textbf{બ્લોક ડાયાગ્રામ:}

\begin{center}
\textbf{Mermaid Diagram (Code)}
\begin{verbatim}
{Shaded}
{Highlighting}[]
graph LR
    A[Input] {-{-}{} B[Integrator]}
    B {-{-}{} C[Comparator]}
    D[Clock] {-{-}{} E[Counter \& Control]}
    C {-{-}{} E}
    E {-{-}{} F[Digital Display]}
{Highlighting}
{Shaded}
\end{verbatim}
\end{center}

{\def\LTcaptype{none} % do not increment counter
\begin{longtable}[]{@{}
  >{\raggedright\arraybackslash}p{(\linewidth - 2\tabcolsep) * \real{0.2400}}
  >{\raggedright\arraybackslash}p{(\linewidth - 2\tabcolsep) * \real{0.7600}}@{}}
\toprule\noalign{}
\begin{minipage}[b]{\linewidth}\raggedright
પ્રકાર
\end{minipage} & \begin{minipage}[b]{\linewidth}\raggedright
કાર્ય સિદ્ધાંત
\end{minipage} \\
\midrule\noalign{}
\endhead
\bottomrule\noalign{}
\endlastfoot
\textbf{ડ્યુઅલ-સ્લોપ} & નિશ્ચિત સમય માટે ઇનપુટને ઇન્ટિગ્રેટ કરે છે, પછી સંદર્ભ સાથે
ડિસ્ચાર્જ સમય માપે છે \\
\textbf{વોલ્ટેજ-ટુ-ફ્રિક્વન્સી} & વોલ્ટેજને આવૃત્તિમાં રૂપાંતરિત કરે છે, નિશ્ચિત સમય પર
પલ્સની ગણતરી કરે છે \\
\textbf{ચાર્જ-બેલેન્સ} & ઇનપુટ ચાર્જને સંદર્ભ ચાર્જ સાથે સંતુલિત કરે છે \\
\end{longtable}
}

\textbf{મુખ્ય લક્ષણો:}

\begin{itemize}
\tightlist
\item
  \textbf{નોઇઝ રિજેક્શન}: પાવર લાઇન નોઇઝ (50/60Hz) નું ઉત્કૃષ્ટ રિજેક્શન
\item
  \textbf{ચોકસાઈ}: સમય સરેરાશને કારણે ઉચ્ચ ચોકસાઈ
\item
  \textbf{રૂપાંતરની ગતિ}: સક્સેસિવ એપ્રોક્સિમેશન પ્રકાર કરતાં ધીમી
\item
  \textbf{રિઝોલ્યુશન}: સામાન્ય રીતે 4½ થી 6½ અંક
\end{itemize}

\textbf{ઉપયોગો}: ચોકસાઈપૂર્ણ માપ, ધોંધાટિયા વાતાવરણ, બેન્ચ ઇન્સ્ટ્રુમેન્ટ્સ

\end{solutionbox}
\begin{mnemonicbox}
``TINA - સમય ઇન્ટિગ્રેશન સરેરાશને શૂન્ય કરે છે''

\end{mnemonicbox}
\subsection*{પ્રશ્ન 4(અ) [3
ગુણ]}\label{uxaaauxab0uxab6uxaa8-4uxa85-3-uxa97uxaa3}

\textbf{ડિજિટલ સ્ટોરેજ ઓસીલોસ્કોપના ફાયદા અને ઉપયોગો લખો.}

\begin{solutionbox}

{\def\LTcaptype{none} % do not increment counter
\begin{longtable}[]{@{}ll@{}}
\toprule\noalign{}
ફાયદા & ઉપયોગો \\
\midrule\noalign{}
\endhead
\bottomrule\noalign{}
\endlastfoot
\textbf{પ્રી-ટ્રિગર વ્યુઇંગ} & ક્ષણિક ઘટનાઓને કેપ્ચર કરવી \\
\textbf{સિગ્નલ સ્ટોરેજ} & અનિયમિત ખામીઓનું વિશ્લેષણ \\
\textbf{વેવફોર્મ પ્રોસેસિંગ} & જટિલ સિગ્નલ વિશ્લેષણ \\
\textbf{ઉચ્ચ બેન્ડવિડ્થ} & ઉચ્ચ-ગતિ ડિજિટલ સર્કિટ ટેસ્ટિંગ \\
\textbf{મલ્ટિપલ ચેનલ ડિસ્પ્લે} & ઘણા સિગ્નલોની તુલના \\
\end{longtable}
}

\begin{itemize}
\tightlist
\item
  \textbf{મુખ્ય લાભ}: એક-વખતની ઘટનાઓને કેપ્ચર કરી શકે છે, પછીના વિશ્લેષણ માટે
  વેવફોર્મ સંગ્રહિત કરી શકે છે
\item
  \textbf{ડિજિટલ સુવિધાઓ}: ઓટોમેટેડ માપ, FFT વિશ્લેષણ, PC કનેક્ટિવિટી
\end{itemize}

\end{solutionbox}
\begin{mnemonicbox}
``SPADE - સંગ્રહ, પ્રોસેસિંગ, વિશ્લેષણ, ડિસ્પ્લે, ઘટનાઓ''

\end{mnemonicbox}
\subsection*{પ્રશ્ન 4(બ) [4
ગુણ]}\label{uxaaauxab0uxab6uxaa8-4uxaac-4-uxa97uxaa3}

\textbf{ઇલેક્ટ્રોનિક એનર્જી મીટર ઉપર ટૂંકનોંધ લખો.}

\begin{solutionbox}

\textbf{બ્લોક ડાયાગ્રામ:}

\begin{center}
\textbf{Mermaid Diagram (Code)}
\begin{verbatim}
{Shaded}
{Highlighting}[]
graph LR
    A[Voltage Sensor] {-{-}{} C[Multiplier]}
    B[Current Sensor] {-{-}{} C}
    C {-{-}{} D[Integrator]}
    D {-{-}{} E[Pulse Generator]}
    E {-{-}{} F[Counter]}
    F {-{-}{} G[Display]}
{Highlighting}
{Shaded}
\end{verbatim}
\end{center}

{\def\LTcaptype{none} % do not increment counter
\begin{longtable}[]{@{}ll@{}}
\toprule\noalign{}
ઘટક & કાર્ય \\
\midrule\noalign{}
\endhead
\bottomrule\noalign{}
\endlastfoot
\textbf{વોલ્ટેજ અને કરંટ સેન્સર} & લાઇન વોલ્ટેજ અને લોડ કરંટ માપે છે \\
\textbf{મલ્ટિપ્લાયર સર્કિટ} & ક્ષણિક પાવરની ગણતરી કરે છે \\
\textbf{ઇન્ટિગ્રેટર} & સમય પર પાવરને ઊર્જામાં રૂપાંતરિત કરે છે \\
\textbf{માઇક્રોકંટ્રોલર} & સિગ્નલ પ્રોસેસ કરે છે અને ડિસ્પ્લેને નિયંત્રિત કરે છે \\
\textbf{LCD ડિસ્પ્લે} & kWh માં ઊર્જા વપરાશ બતાવે છે \\
\end{longtable}
}

\begin{itemize}
\tightlist
\item
  \textbf{કાર્ય સિદ્ધાંત}: ઊર્જા = \intP.dt (સમય પર પાવરનું ઇન્ટિગ્રલ)
\item
  \textbf{ફાયદા}: કોઈ ગતિશીલ ભાગો નહીં, ઉચ્ચ ચોકસાઈ, છેડછાડ શોધન
\item
  \textbf{સુવિધાઓ}: મલ્ટિપલ ટેરિફ સપોર્ટ, બે-દિશા માપન, રિમોટ રીડિંગ
\end{itemize}

\end{solutionbox}
\begin{mnemonicbox}
``VICES - વોલ્ટેજ અને કરંટ ઊર્જા સરવાળો''

\end{mnemonicbox}
\subsection*{પ્રશ્ન 4(ક) [7
ગુણ]}\label{uxaaauxab0uxab6uxaa8-4uxa95-7-uxa97uxaa3}

\textbf{એનાલોગ C.R.O. નો બ્લોક ડાયાગ્રામ દોરો અને સમજાવો, અને દરેક બ્લોકનું
વર્કિંગ લખો.}

\begin{solutionbox}

\textbf{બ્લોક ડાયાગ્રામ:}

\begin{center}
\textbf{Mermaid Diagram (Code)}
\begin{verbatim}
{Shaded}
{Highlighting}[]
graph LR
    A[Vertical Input] {-{-}{} B[Vertical Attenuator]}
    B {-{-}{} C[Vertical Amplifier]}
    C {-{-}{} D[Vertical Deflection Plates]}
    E[Trigger Circuit] {-{-}{} F[Time Base Generator]}
    F {-{-}{} G[Horizontal Amplifier]}
    G {-{-}{} H[Horizontal Deflection Plates]}
    I[Cathode Ray Tube] {-{-}{} J[Screen]}
    D {-{-}{} I}
    H {-{-}{} I}
    K[Power Supply] {-{-}{} All}
{Highlighting}
{Shaded}
\end{verbatim}
\end{center}

{\def\LTcaptype{none} % do not increment counter
\begin{longtable}[]{@{}
  >{\raggedright\arraybackslash}p{(\linewidth - 2\tabcolsep) * \real{0.4118}}
  >{\raggedright\arraybackslash}p{(\linewidth - 2\tabcolsep) * \real{0.5882}}@{}}
\toprule\noalign{}
\begin{minipage}[b]{\linewidth}\raggedright
બ્લોક
\end{minipage} & \begin{minipage}[b]{\linewidth}\raggedright
કાર્ય
\end{minipage} \\
\midrule\noalign{}
\endhead
\bottomrule\noalign{}
\endlastfoot
\textbf{વર્ટિકલ સિસ્ટમ} & એમ્પ્લિટ્યુડ ડિસ્પ્લેને નિયંત્રિત કરે છે (સિગ્નલ અટેન્યુએશન,
એમ્પ્લિફિકેશન) \\
\textbf{હોરિઝોન્ટલ સિસ્ટમ} & ટાઇમ બેઝને નિયંત્રિત કરે છે (સ્વીપ જનરેશન) \\
\textbf{ટ્રિગર સિસ્ટમ} & ઇનપુટ સિગ્નલ સાથે હોરિઝોન્ટલ સ્વીપને સિંક્રનાઇઝ કરે છે \\
\textbf{CRT} & સિગ્નલને પ્રદર્શિત કરે છે (ઇલેક્ટ્રોન ગન, ડિફ્લેક્શન પ્લેટ્સ, ફોસ્ફર
સ્ક્રીન) \\
\textbf{પાવર સપ્લાય} & બધા સર્કિટને જરૂરી વોલ્ટેજ પ્રદાન કરે છે \\
\end{longtable}
}

\begin{itemize}
\tightlist
\item
  \textbf{વર્ટિકલ સિસ્ટમ}: ઇનપુટ સિગ્નલને પ્રોસેસ કરે છે, Y-એક્સિસ ડિફ્લેક્શનને
  નિયંત્રિત કરે છે
\item
  \textbf{હોરિઝોન્ટલ સિસ્ટમ}: X-એક્સિસ ડિફ્લેક્શનને નિયંત્રિત કરે છે (ટાઇમ બેઝ)
\item
  \textbf{ટ્રિગરિંગ}: એક જ બિંદુ પર સ્વીપ શરૂ કરીને વેવફોર્મ ડિસ્પ્લેને સ્થિર કરે છે
\item
  \textbf{CRT ડિસ્પ્લે}: ઇલેક્ટ્રિકલ સિગ્નલને દેખાતી ટ્રેસમાં રૂપાંતરિત કરે છે
\end{itemize}

\end{solutionbox}
\begin{mnemonicbox}
``VTHCP - વર્ટિકલ, ટાઇમ, હોરિઝોન્ટલ, CRT, પાવર''

\end{mnemonicbox}
\subsection*{પ્રશ્ન 4(અ OR) [3
ગુણ]}\label{uxaaauxab0uxab6uxaa8-4uxa85-or-3-uxa97uxaa3}

\textbf{પીજો ઈલેક્ટ્રીક ટ્રાન્સડ્યુસર દોરો અને સમજાવો.}

\begin{solutionbox}

\textbf{આકૃતિ:}

\begin{verbatim}
      Force
        ↓
    +{-{-}{-}{-}{-}{-}{-}{-}+}
    |        |
    | Quartz |{-{-}{-} Output Voltage}
    | Crystal|
    |        |
    +{-{-}{-}{-}{-}{-}{-}{-}+}
\end{verbatim}

{\def\LTcaptype{none} % do not increment counter
\begin{longtable}[]{@{}ll@{}}
\toprule\noalign{}
લક્ષણ & વિગત \\
\midrule\noalign{}
\endhead
\bottomrule\noalign{}
\endlastfoot
\textbf{સિદ્ધાંત} & યાંત્રિક રીતે દબાણ કરવામાં આવે ત્યારે વિદ્યુત ચાર્જ ઉત્પન્ન કરે
છે \\
\textbf{સામગ્રી} & ક્વાર્ટ્ઝ, રોશેલ સોલ્ટ, PZT સિરામિક્સ \\
\textbf{કાર્યપદ્ધતિ} & સીધી અસર: બળ \rightarrow વોલ્ટેજ, વિપરીત અસર: વોલ્ટેજ \rightarrow
વિસ્થાપન \\
\textbf{આઉટપુટ} & લાગુ કરેલા બળના પ્રમાણમાં ઉચ્ચ ઇમ્પીડન્સ વોલ્ટેજ \\
\end{longtable}
}

\begin{itemize}
\tightlist
\item
  \textbf{ઉપયોગો}: પ્રેશર સેન્સર, એક્સેલેરોમીટર, અલ્ટ્રાસોનિક ઉપકરણો
\item
  \textbf{ફાયદા}: ઉચ્ચ સંવેદનશીલતા, ઝડપી પ્રતિસાદ, વિશાળ આવૃત્તિ શ્રેણી
\item
  \textbf{મર્યાદાઓ}: ઉચ્ચ આઉટપુટ ઇમ્પીડન્સ, તાપમાન સંવેદનશીલ
\end{itemize}

\end{solutionbox}
\begin{mnemonicbox}
``PFVD - દબાણ વિસ્થાપન દ્વારા વોલ્ટેજ બનાવે છે''

\end{mnemonicbox}
\subsection*{પ્રશ્ન 4(બ OR) [4
ગુણ]}\label{uxaaauxab0uxab6uxaa8-4uxaac-or-4-uxa97uxaa3}

\textbf{CRO ની મદદથી ફ્રિકવન્સી મેઝરમેન્ટ માટેની આકૃતિ દોરો અને સમજાવો.}

\begin{solutionbox}

\textbf{પદ્ધતિ 1: લિસાજોસ પેટર્ન નો ઉપયોગ}

\begin{verbatim}
    +{-{-}{-}{-}{-}{-}{-}{-}{-}{-}{-}{-}{-}+}
    |             |
    |    o o o    |
    |   o     o   |
    |  o       o  |
    |   o     o   |
    |    o o o    |
    |             |
    +{-{-}{-}{-}{-}{-}{-}{-}{-}{-}{-}{-}{-}+}
\end{verbatim}

\textbf{પદ્ધતિ 2: ટાઇમ બેઝનો ઉપયોગ}

\begin{verbatim}
    +{-{-}{-}{-}{-}{-}{-}{-}{-}{-}{-}{-}{-}+}
    |        /{   |}
    |       /  {  |}
    |      /    { |}
    |     /      {|}
    |    /        |
    |   /         |
    +{-{-}{-}{-}{-}{-}{-}{-}{-}{-}{-}{-}{-}+}
\end{verbatim}

{\def\LTcaptype{none} % do not increment counter
\begin{longtable}[]{@{}
  >{\raggedright\arraybackslash}p{(\linewidth - 2\tabcolsep) * \real{0.3810}}
  >{\raggedright\arraybackslash}p{(\linewidth - 2\tabcolsep) * \real{0.6190}}@{}}
\toprule\noalign{}
\begin{minipage}[b]{\linewidth}\raggedright
પદ્ધતિ
\end{minipage} & \begin{minipage}[b]{\linewidth}\raggedright
ગણતરી
\end{minipage} \\
\midrule\noalign{}
\endhead
\bottomrule\noalign{}
\endlastfoot
\textbf{લિસાજોસ પેટર્ન} & Fx = Fy \times (Nx/Ny) \\
\textbf{સમય માપન} & f = 1/T (T એ ટાઇમ બેઝનો ઉપયોગ કરીને માપવામાં આવેલો
સમયગાળો છે) \\
\textbf{XY મોડ} & જાણીતા સંદર્ભ સાથે અજ્ઞાત આવૃત્તિની તુલના \\
\end{longtable}
}

\begin{itemize}
\tightlist
\item
  \textbf{ટાઇમ બેઝ પદ્ધતિ}: વેવફોર્મનો સમયગાળો માપો, આવૃત્તિની ગણતરી 1/T તરીકે
  કરો
\item
  \textbf{લિસાજોસ પદ્ધતિ}: સંદર્ભને X ઇનપુટ સાથે જોડો, અજ્ઞાતને Y ઇનપુટ સાથે જોડો
\item
  \textbf{ડિજિટલ CRO}: આંતરિક કાઉન્ટરનો ઉપયોગ કરીને સીધો આવૃત્તિ રીડઆઉટ
\end{itemize}

\end{solutionbox}
\begin{mnemonicbox}
``LTX - X-અક્ષ માટે લિસાજોસ અથવા સમય''

\end{mnemonicbox}
\subsection*{પ્રશ્ન 4(ક OR) [7
ગુણ]}\label{uxaaauxab0uxab6uxaa8-4uxa95-or-7-uxa97uxaa3}

\textbf{થર્મિસ્ટર અને થર્મોકપલ દોરો અને સમજાવો.}

\begin{solutionbox}

\textbf{થર્મિસ્ટર આકૃતિ:}

\begin{verbatim}
    +{-{-}{-}{-}{-}{-}{-}{-}{-}{-}{-}+}
    |           |
    | Thermistor|{-{-}{-}+}
    |           |   |
    +{-{-}{-}{-}{-}{-}{-}{-}{-}{-}{-}+   |}
                    |
    +{-{-}{-}{-}{-}{-}{-}{-}{-}{-}+    |}
    |          |    |
    | Resistor |{-{-}{-}{-}+{-}{-}{-}{-} Output}
    |          |
    +{-{-}{-}{-}{-}{-}{-}{-}{-}{-}+}
\end{verbatim}

\textbf{થર્મોકપલ આકૃતિ:}

\begin{verbatim}
     Metal A
    +{-{-}{-}{-}{-}{-}{-}{-}+}
              {}
               +{-{-}{-} Output}
              /
    +{-{-}{-}{-}{-}{-}{-}{-}+}
     Metal B
\end{verbatim}

{\def\LTcaptype{none} % do not increment counter
\begin{longtable}[]{@{}
  >{\raggedright\arraybackslash}p{(\linewidth - 4\tabcolsep) * \real{0.3000}}
  >{\raggedright\arraybackslash}p{(\linewidth - 4\tabcolsep) * \real{0.2750}}
  >{\raggedright\arraybackslash}p{(\linewidth - 4\tabcolsep) * \real{0.4250}}@{}}
\toprule\noalign{}
\begin{minipage}[b]{\linewidth}\raggedright
ટ્રાન્સડ્યુસર
\end{minipage} & \begin{minipage}[b]{\linewidth}\raggedright
સિદ્ધાંત
\end{minipage} & \begin{minipage}[b]{\linewidth}\raggedright
લક્ષણો
\end{minipage} \\
\midrule\noalign{}
\endhead
\bottomrule\noalign{}
\endlastfoot
\textbf{થર્મિસ્ટર} & તાપમાન સાથે અવરોધમાં ફેરફાર & ઉચ્ચ સંવેદનશીલતા, બિન-રેખીય,
મર્યાદિત શ્રેણી \\
\textbf{થર્મોકપલ} & અસમાન ધાતુઓના સંયોજનથી વોલ્ટેજ ઉત્પન્ન થાય છે & વિશાળ શ્રેણી,
રેખીય, ઓછી સંવેદનશીલતા \\
\end{longtable}
}

\textbf{થર્મિસ્ટર પ્રકાર:}

\begin{itemize}
\tightlist
\item
  \textbf{NTC}: નેગેટિવ તાપમાન ગુણાંક (તાપમાન વધવાથી અવરોધ ઘટે છે)
\item
  \textbf{PTC}: પોઝિટિવ તાપમાન ગુણાંક (તાપમાન વધવાથી અવરોધ વધે છે)
\end{itemize}

\textbf{થર્મોકપલ પ્રકાર:}

\begin{itemize}
\tightlist
\item
  \textbf{ટાઇપ K}: ક્રોમેલ-એલ્યુમેલ (-200^\circC થી 1350^\circC)
\item
  \textbf{ટાઇપ J}: આયર્ન-કોન્સ્ટન્ટન (-40^\circC થી 750^\circC)
\item
  \textbf{ટાઇપ T}: કોપર-કોન્સ્ટન્ટન (-200^\circC થી 350^\circC)
\end{itemize}

\end{solutionbox}
\begin{mnemonicbox}
``TRT/TVJ - તાપમાન અવરોધ/વોલ્ટેજ જંક્શન''

\end{mnemonicbox}
\subsection*{પ્રશ્ન 5(અ) [3
ગુણ]}\label{uxaaauxab0uxab6uxaa8-5uxa85-3-uxa97uxaa3}

\textbf{વેલોસિટી ટ્રાન્સડ્યુસર દોરો અને સમજાવો.}

\begin{solutionbox}

\textbf{આકૃતિ:}

\begin{verbatim}
    +{-{-}{-}{-}{-}{-}{-}{-}{-}{-}{-}{-}{-}{-}{-}{-}{-}{-}+}
    |                  |
    |  N     S    N    |
    |  |     |    |    |
    +{-{-}+{-}{-}{-}{-}{-}+{-}{-}{-}{-}+{-}{-}{-}{-}+}
       |     |    |
       |  Magnet  |
       |     |    |
    +{-{-}+{-}{-}{-}{-}{-}+{-}{-}{-}{-}+{-}{-}{-}{-}+}
    |                  |
    |      Coil        |{-{-}{-}{-} Output}
    |                  |
    +{-{-}{-}{-}{-}{-}{-}{-}{-}{-}{-}{-}{-}{-}{-}{-}{-}{-}+}
\end{verbatim}

{\def\LTcaptype{none} % do not increment counter
\begin{longtable}[]{@{}ll@{}}
\toprule\noalign{}
ઘટક & કાર્ય \\
\midrule\noalign{}
\endhead
\bottomrule\noalign{}
\endlastfoot
\textbf{કાયમી ચુંબક} & ચુંબકીય ક્ષેત્ર બનાવે છે \\
\textbf{મુવિંગ કોઇલ} & વેગના પ્રમાણમાં વોલ્ટેજ ઉત્પન્ન કરે છે \\
\textbf{હાઉસિંગ} & માળખાને અને ચુંબકીય સર્કિટને સમર્થન આપે છે \\
\textbf{આઉટપુટ સર્કિટ} & માપન માટે સિગ્નલને કન્ડિશન કરે છે \\
\end{longtable}
}

\begin{itemize}
\tightlist
\item
  \textbf{કાર્ય સિદ્ધાંત}: ફેરાડેના ઇલેક્ટ્રોમેગ્નેટિક ઇન્ડક્શનના નિયમ પર આધારિત
\item
  \textbf{આઉટપુટ}: વેગના પ્રમાણમાં વોલ્ટેજ (V = Blv)
\item
  \textbf{ઉપયોગો}: વાયબ્રેશન માપન, ભૂકંપીય મોનિટરિંગ, મોશન નિયંત્રણ
\end{itemize}

\end{solutionbox}
\begin{mnemonicbox}
``VMMF - વેગ ચુંબકીય પ્રવાહ બનાવે છે''

\end{mnemonicbox}
\subsection*{પ્રશ્ન 5(બ) [4
ગુણ]}\label{uxaaauxab0uxab6uxaa8-5uxaac-4-uxa97uxaa3}

\textbf{ટ્રાન્સડ્યુસર નું વર્ગીકરણ કરો અને સમજાવો.}

\begin{solutionbox}

{\def\LTcaptype{none} % do not increment counter
\begin{longtable}[]{@{}ll@{}}
\toprule\noalign{}
વર્ગીકરણ & પ્રકાર \\
\midrule\noalign{}
\endhead
\bottomrule\noalign{}
\endlastfoot
\textbf{ઊર્જા રૂપાંતરણ દ્વારા} & એક્ટિવ (સ્વ-જનરેટિંગ) vs.~પેસિવ (બાહ્ય પાવરની
જરૂર) \\
\textbf{માપન પદ્ધતિ દ્વારા} & પ્રાથમિક vs.~ગૌણ \\
\textbf{ભૌતિક સિદ્ધાંત દ્વારા} & રેઝિસ્ટિવ, કેપેસિટિવ, ઇન્ડક્ટિવ, ફોટોઇલેક્ટ્રિક,
વગેરે \\
\textbf{ઉપયોગ દ્વારા} & તાપમાન, દબાણ, પ્રવાહ, સ્તર, વગેરે \\
\end{longtable}
}

\textbf{સમજૂતી:}

{\def\LTcaptype{none} % do not increment counter
\begin{longtable}[]{@{}lll@{}}
\toprule\noalign{}
પ્રકાર & ઉદાહરણો & લક્ષણો \\
\midrule\noalign{}
\endhead
\bottomrule\noalign{}
\endlastfoot
\textbf{એક્ટિવ} & થર્મોકપલ, પિઝોઇલેક્ટ્રિક & બાહ્ય પાવર વિના આઉટપુટ ઉત્પન્ન કરે
છે \\
\textbf{પેસિવ} & RTD, સ્ટ્રેન ગેજ & બાહ્ય ઉત્તેજનાની જરૂર પડે છે \\
\textbf{રેઝિસ્ટિવ} & થર્મિસ્ટર, પોટેન્શિયોમીટર & ઇનપુટ સાથે અવરોધ બદલે છે \\
\textbf{કેપેસિટિવ} & પ્રેશર સેન્સર, પ્રોક્સિમિટી & ઇનપુટ સાથે કેપેસિટન્સ બદલે છે \\
\textbf{ઇન્ડક્ટિવ} & LVDT, પ્રોક્સિમિટી & ઇનપુટ સાથે ઇન્ડક્ટન્સ બદલે છે \\
\end{longtable}
}

\end{solutionbox}
\begin{mnemonicbox}
``APRCI - એક્ટિવ પેસિવ રેઝિસ્ટિવ કેપેસિટિવ ઇન્ડક્ટિવ''

\end{mnemonicbox}
\subsection*{પ્રશ્ન 5(ક) [7
ગુણ]}\label{uxaaauxab0uxab6uxaa8-5uxa95-7-uxa97uxaa3}

\textbf{LVDT ઉપર ટૂંકનોંધ લખો.}

\begin{solutionbox}

\textbf{આકૃતિ:}

\begin{center}
\textbf{Mermaid Diagram (Code)}
\begin{verbatim}
{Shaded}
{Highlighting}[]
graph LR
    A[Primary Coil] {-{-}{} B[Core]}
    B {-{-}{} C[Secondary Coil 1]}
    B {-{-}{} D[Secondary Coil 2]}
    E[AC Excitation] {-{-}{} A}
    C {-{-}{} F[Phase Sensitive Detector]}
    D {-{-}{} F}
    F {-{-}{} G[Output]}
{Highlighting}
{Shaded}
\end{verbatim}
\end{center}

{\def\LTcaptype{none} % do not increment counter
\begin{longtable}[]{@{}ll@{}}
\toprule\noalign{}
ઘટક & કાર્ય \\
\midrule\noalign{}
\endhead
\bottomrule\noalign{}
\endlastfoot
\textbf{પ્રાથમિક કોઇલ} & AC સોર્સ સાથે જોડાયેલ ઉત્તેજના કોઇલ \\
\textbf{સેકન્ડરી કોઇલ} & શ્રેણી વિરોધી જોડાણમાં બે સમાન કોઇલ \\
\textbf{ફેરોમેગ્નેટિક કોર} & પારસ્પરિક ઇન્ડક્ટન્સ બદલતો ગતિશીલ કોર \\
\textbf{સિગ્નલ કન્ડિશનર} & ડિફરેન્શિયલ આઉટપુટને વિસ્થાપન માપનમાં રૂપાંતરિત કરે છે \\
\end{longtable}
}

\textbf{કાર્ય સિદ્ધાંત:}

\begin{itemize}
\tightlist
\item
  શૂન્ય સ્થિતિએ: બંને સેકન્ડરીમાં સમાન વોલ્ટેજ પ્રેરિત થાય છે, નેટ આઉટપુટ શૂન્ય
\item
  કોર મૂવમેન્ટ: સેકન્ડરી વોલ્ટેજમાં અસંતુલન બનાવે છે
\item
  આઉટપુટ વોલ્ટેજ: વિસ્થાપનના પ્રમાણમાં, ફેઝ દિશા દર્શાવે છે
\end{itemize}

\textbf{લક્ષણો:}

\begin{itemize}
\tightlist
\item
  \textbf{રેન્જ}: સામાન્ય રીતે \pm0.5mm થી \pm25cm
\item
  \textbf{રેખિયતા}: નિર્ધારિત રેન્જમાં ઉત્કૃષ્ટ
\item
  \textbf{રિઝોલ્યુશન}: લગભગ અનંત (રીડઆઉટ સર્કિટ દ્વારા મર્યાદિત)
\item
  \textbf{ફાયદા}: ઘર્ષણ વિનાનું, મજબૂત, વિશ્વસનીય, ઉચ્ચ રિઝોલ્યુશન
\end{itemize}

\end{solutionbox}
\begin{mnemonicbox}
``CPSO: કોર પોઝિશન આઉટપુટ બદલે છે''

\end{mnemonicbox}
\subsection*{પ્રશ્ન 5(અ OR) [3
ગુણ]}\label{uxaaauxab0uxab6uxaa8-5uxa85-or-3-uxa97uxaa3}

\textbf{સાદા ફ્રિક્વન્સી કાઉન્ટરનો બ્લોક ડાયાગ્રામ દોરો અને સમજાવો.}

\begin{solutionbox}

\textbf{બ્લોક ડાયાગ્રામ:}

\begin{center}
\textbf{Mermaid Diagram (Code)}
\begin{verbatim}
{Shaded}
{Highlighting}[]
graph LR
    A[Input] {-{-}{} B[Input Conditioning]}
    B {-{-}{} C[Gate Control]}
    D[Time Base] {-{-}{} C}
    C {-{-}{} E[Counter]}
    E {-{-}{} F[Display]}
{Highlighting}
{Shaded}
\end{verbatim}
\end{center}

{\def\LTcaptype{none} % do not increment counter
\begin{longtable}[]{@{}ll@{}}
\toprule\noalign{}
બ્લોક & કાર્ય \\
\midrule\noalign{}
\endhead
\bottomrule\noalign{}
\endlastfoot
\textbf{ઇનપુટ કન્ડિશનિંગ} & સિગ્નલને પલ્સમાં રૂપાંતરિત કરે છે \\
\textbf{ગેટ કંટ્રોલ} & ટાઇમ બેઝના આધારે ગણતરી અવધિને નિયંત્રિત કરે છે \\
\textbf{ટાઇમ બેઝ} & ચોક્કસ સંદર્ભ સમય અંતરાલ પ્રદાન કરે છે \\
\textbf{કાઉન્ટર} & ગેટ અવધિ દરમિયાન ઇનપુટ પલ્સની ગણતરી કરે છે \\
\textbf{ડિસ્પ્લે} & ગણતરી પરિણામ (આવૃત્તિ) બતાવે છે \\
\end{longtable}
}

\begin{itemize}
\tightlist
\item
  \textbf{કાર્ય સિદ્ધાંત}: ચોક્કસ સમય અંતરાલ (સામાન્ય રીતે 1 સેકન્ડ) પર પલ્સની
  ગણતરી કરે છે
\item
  \textbf{આવૃત્તિ ગણતરી}: f = ગણતરી/સમય અંતરાલ
\item
  \textbf{રિઝોલ્યુશન}: ટાઇમ બેઝ ચોકસાઈ અને ગેટ સમય દ્વારા નિર્ધારિત
\end{itemize}

\end{solutionbox}
\begin{mnemonicbox}
``IGTCD - ઇનપુટ ગેટેડ ટાઇમ કાઉન્ટ્સ ડિસ્પ્લે''

\end{mnemonicbox}
\subsection*{પ્રશ્ન 5(બ OR) [4
ગુણ]}\label{uxaaauxab0uxab6uxaa8-5uxaac-or-4-uxa97uxaa3}

\textbf{કેપેસિટીવ ટ્રાન્સડ્યુસર દોરો અને સમજાવો.}

\begin{solutionbox}

\textbf{આકૃતિ:}

\begin{verbatim}
    +{-{-}{-}{-}{-}{-}{-}{-}{-}{-}{-}{-}{-}+}
    |    Fixed    |
    |   Plate 1   |
    +{-{-}{-}{-}{-}{-}{-}{-}{-}{-}{-}{-}{-}+}
           ↑
           d      ↓ Force
    +{-{-}{-}{-}{-}{-}{-}{-}{-}{-}{-}{-}{-}+}
    |   Movable   |
    |   Plate 2   |{-{-}{-}{-}{-} Output}
    +{-{-}{-}{-}{-}{-}{-}{-}{-}{-}{-}{-}{-}+}
\end{verbatim}

{\def\LTcaptype{none} % do not increment counter
\begin{longtable}[]{@{}
  >{\raggedright\arraybackslash}p{(\linewidth - 4\tabcolsep) * \real{0.3846}}
  >{\raggedright\arraybackslash}p{(\linewidth - 4\tabcolsep) * \real{0.2821}}
  >{\raggedright\arraybackslash}p{(\linewidth - 4\tabcolsep) * \real{0.3333}}@{}}
\toprule\noalign{}
\begin{minipage}[b]{\linewidth}\raggedright
કોન્ફિગરેશન
\end{minipage} & \begin{minipage}[b]{\linewidth}\raggedright
સિદ્ધાંત
\end{minipage} & \begin{minipage}[b]{\linewidth}\raggedright
ઉપયોગ
\end{minipage} \\
\midrule\noalign{}
\endhead
\bottomrule\noalign{}
\endlastfoot
\textbf{વેરિએબલ ગેપ} & C = ε_{0}εᵣA/d (અંતર સાથે વ્યસ્ત રીતે બદલાય છે) & દબાણ,
વિસ્થાપન \\
\textbf{વેરિએબલ એરિયા} & C = ε_{0}εᵣA/d (ઓવરલેપ એરિયા સાથે સીધો બદલાવ) & ખૂણીય
સ્થિતિ, સ્તર \\
\textbf{વેરિએબલ ડાયલેક્ટ્રિક} & C = ε_{0}εᵣA/d (ડાયલેક્ટ્રિક કોન્સ્ટન્ટ સાથે બદલાય છે)
& ભેજ, સામગ્રી વિશ્લેષણ \\
\end{longtable}
}

\textbf{કાર્ય સિદ્ધાંત:}

\begin{itemize}
\tightlist
\item
  ભૌતિક પરિમાણ સાથે કેપેસિટન્સ બદલાય છે
\item
  સિગ્નલ કન્ડિશનિંગ કેપેસિટન્સને વોલ્ટેજ/વીજપ્રવાહમાં રૂપાંતરિત કરે છે
\item
  ઉચ્ચ ઇમ્પીડન્સ આઉટપુટને યોગ્ય શીલ્ડિંગની જરૂર પડે છે
\end{itemize}

\textbf{ફાયદા}: ઉચ્ચ સંવેદનશીલતા, કોઈ ગતિશીલ સંપર્ક નહીં, ઓછું દળ

\end{solutionbox}
\begin{mnemonicbox}
``CGAD - કેપેસિટન્સ ગેપ એરિયા ડાયલેક્ટ્રિક''

\end{mnemonicbox}
\subsection*{પ્રશ્ન 5(ક OR) [7
ગુણ]}\label{uxaaauxab0uxab6uxaa8-5uxa95-or-7-uxa97uxaa3}

\textbf{ફંકશન જનરેટરનો બ્લોક ડાયાગ્રામ દોરો અને સમજાવો.}

\begin{solutionbox}

\textbf{બ્લોક ડાયાગ્રામ:}

\begin{center}
\textbf{Mermaid Diagram (Code)}
\begin{verbatim}
{Shaded}
{Highlighting}[]
graph LR
    A[Frequency Control] {-{-}{} B[Waveform Generator]}
    C[Mode Selector] {-{-}{} B}
    B {-{-}{} D[Amplifier \& Attenuator]}
    D {-{-}{} E[Output Buffer]}
    E {-{-}{} F[Output]}
    G[Sweep Circuit] {-{-}{} B}
    H[AM/FM Modulator] {-{-}{} D}
{Highlighting}
{Shaded}
\end{verbatim}
\end{center}

{\def\LTcaptype{none} % do not increment counter
\begin{longtable}[]{@{}
  >{\raggedright\arraybackslash}p{(\linewidth - 2\tabcolsep) * \real{0.4118}}
  >{\raggedright\arraybackslash}p{(\linewidth - 2\tabcolsep) * \real{0.5882}}@{}}
\toprule\noalign{}
\begin{minipage}[b]{\linewidth}\raggedright
બ્લોક
\end{minipage} & \begin{minipage}[b]{\linewidth}\raggedright
કાર્ય
\end{minipage} \\
\midrule\noalign{}
\endhead
\bottomrule\noalign{}
\endlastfoot
\textbf{ફ્રિક્વન્સી કંટ્રોલ} & ઓસિલેટરની આવૃત્તિ સેટ કરે છે (સામાન્ય રીતે 0.1Hz થી
20MHz) \\
\textbf{વેવફોર્મ જનરેટર} & મૂળભૂત વેવફોર્મ ઉત્પન્ન કરે છે (સાઇન, સ્ક્વેર, ટ્રાયએંગલ) \\
\textbf{મોડ સિલેક્ટર} & આઉટપુટ વેવફોર્મના પ્રકારની પસંદગી કરે છે \\
\textbf{એમ્પ્લિફાયર અને એટેન્યુએટર} & આઉટપુટ એમ્પ્લિટ્યુડને નિયંત્રિત કરે છે \\
\textbf{આઉટપુટ બફર} & ઓછી આઉટપુટ ઇમ્પીડન્સ પ્રદાન કરે છે \\
\textbf{સ્વીપ સર્કિટ} & રેન્જ પર આવૃત્તિને આપોઆપ બદલે છે \\
\textbf{AM/FM મોડ્યુલેટર} & મોડ્યુલેશન કાર્યો માટે સિગ્નલ બદલે છે \\
\end{longtable}
}

\textbf{કાર્ય સિદ્ધાંત:}

\begin{itemize}
\tightlist
\item
  RC ઓસિલેટર અથવા DDS નો ઉપયોગ કરીને સાઇન વેવ ઉત્પન્ન કરે છે
\item
  શેપ કન્વર્ટર્સ સાઇનને સ્ક્વેર અને ટ્રાયએંગલમાં રૂપાંતરિત કરે છે
\item
  આઉટપુટ એમ્પ્લિટ્યુડ એટેન્યુએટર સર્કિટ દ્વારા નિયંત્રિત
\item
  આધુનિક જનરેટર ડિજિટલ સિન્થેસિસ તકનીકોનો ઉપયોગ કરે છે
\end{itemize}

\textbf{ઉપયોગો}: સર્કિટ ટેસ્ટિંગ, સિગ્નલ ઇન્જેક્શન, ફિલ્ટર કેરેક્ટરાઇઝેશન

\end{solutionbox}
\begin{mnemonicbox}
``FWMASO - ફ્રિક્વન્સી વેવફોર્મ મોડ એમ્પ્લિટ્યુડ સ્વીપ આઉટપુટ''

\end{mnemonicbox}

\end{document}
