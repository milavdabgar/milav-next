\documentclass{article}

% content/resources/templates/preamble.tex
\usepackage[margin=0.6in]{geometry}
\author{Milav Dabgar}
\usepackage{amsmath,amssymb,amsthm}
\usepackage{booktabs}
\usepackage{multirow}
\usepackage{xcolor}
\usepackage{tcolorbox}
\tcbuselibrary{breakable,skins}
\usepackage[colorlinks=true,linkcolor=blue]{hyperref}
\usepackage{titlesec}
\usepackage{enumitem}
\usepackage{tikz}
\usepackage{pgfplots}
\usepackage{circuitikz}
\usepackage[version=4]{mhchem}
\usepackage{longtable}
\usepackage{array}
\usepackage{float}
\usepackage{caption}
\usepackage{listings}

\lstset{
  basicstyle=\small\ttfamily,
  breaklines=true,
  breakatwhitespace=false,
  postbreak=\mbox{\textcolor{red}{$\hookrightarrow$}\space},
  float=false,
  numbers=left,
  numberstyle=\tiny\color{gray},
  numbersep=10pt,
  xleftmargin=2em,
  keywordstyle=\color{blue},
  commentstyle=\color{green!60!black},
  stringstyle=\color{purple},
  backgroundcolor=\color{gray!5},
  showstringspaces=false,
  tabsize=2,
  captionpos=b,
  keepspaces=true,
  columns=flexible
}

\pgfplotsset{compat=1.18}
\usetikzlibrary{shapes,arrows,positioning,calc,patterns,decorations.pathmorphing,decorations.markings,arrows.meta}

% Color scheme
\definecolor{headcolor}{RGB}{0,102,204}
\definecolor{keycolor}{RGB}{220,20,60}
\definecolor{solutioncolor}{RGB}{34,139,34}
\definecolor{mnemoniccolor}{RGB}{148,0,211}
\definecolor{codecolor}{RGB}{0,0,100}

% Spacing
\setlength{\parskip}{3pt}
\setlist[itemize]{nosep}
\setlist[enumerate]{nosep}

% Title formatting
\titleformat{\section}{\Large\bfseries\color{headcolor}}{\thesection}{1em}{}
\titleformat{\subsection}{\large\bfseries\color{headcolor}}{\thesubsection}{1em}{}

% Pandoc tightlist compatibility
\providecommand{\tightlist}{%
  \setlength{\itemsep}{0pt}\setlength{\parskip}{0pt}}

% Pandoc longtable compatibility
\newcounter{none}
\def\thenone{}


% content/resources/templates/gujarati-boxes.tex
\usepackage{fontspec}
\usepackage{polyglossia}

% Set Gujarati as main language (document is primarily in Gujarati)
% Note: gloss-gujarati.ldf doesn't exist in polyglossia, but it will use hyphenation patterns
\setdefaultlanguage{gujarati}
\setotherlanguage{english}

% Configure Gujarati font properly
% Use Language=Default to prevent polyglossia from trying to add language-specific features
% that don't exist for Gujarati, which causes "empty feature" warnings
\newfontfamily\gujaratifont[Script=Gujarati,AutoFakeBold=2.5,AutoFakeSlant=0.3]{Noto Sans Gujarati}
\setmainfont[Script=Gujarati,AutoFakeBold=2.5,AutoFakeSlant=0.3]{Noto Sans Gujarati}
% Use Noto Sans Gujarati for monospace to support Gujarati in text
\setmonofont[Scale=0.9]{Noto Sans Gujarati}

% Configure English to use the same font
\newfontfamily\englishfont[Script=Gujarati,AutoFakeBold=2.5,AutoFakeSlant=0.3]{Noto Sans Gujarati}

% Translations for polyglossia
\gappto\captionsgujarati{
  \renewcommand{\tablename}{કોષ્ટક}
  \renewcommand{\figurename}{આકૃતિ}
}

% Helper for TikZ nodes to ensure Gujarati font
\newcommand{\gu}[1]{{\gujaratifont #1}}

% Custom environments
\newtcolorbox{solutionbox}{
    breakable,
    enhanced,
    colback=solutioncolor!5!white,
    colframe=solutioncolor!75!black,
    fonttitle=\bfseries,
    title=જવાબ
}

\newtcolorbox{solutionboxnobreak}{
 colback=solutioncolor!5!white,
 colframe=solutioncolor!75!black,
 fonttitle=\bfseries,
 title=જવાબ
}

\newtcolorbox{keyformula}{
 breakable,
 enhanced,
 colback=keycolor!5!white,
 colframe=keycolor!75!black,
 fonttitle=\bfseries,
 title=રાસાયણિક સમીકરણ/સૂત્ર
}

\newtcolorbox{mnemonicbox}{
 breakable,
 enhanced,
 colback=mnemoniccolor!5!white,
 colframe=mnemoniccolor!75!black,
 fonttitle=\bfseries,
 title=મેમરી ટ્રીક
}


% Custom commands for GTU solutions
% This file defines semantic commands for consistent formatting

% Question command with automatic formatting
\newcommand{\question}[2]{%
  \section*{Question #1}%
  \textbf{#2}%
}

% OR question variant
\newcommand{\questionor}[2]{%
  \section*{Question #1 OR}%
  \textbf{#2}%
}

% Proper table environment with caption
\newenvironment{answertable}[1]{%
  \begin{table}[htbp]
  \centering
  \caption{#1}
}{%
  \end{table}
}

% Proper figure environment for diagrams
\newenvironment{answerdiagram}[1]{%
  \begin{figure}[htbp]
  \centering
  \caption{#1}
}{%
  \end{figure}
}

% Semantic markup for key terms
\newcommand{\keyword}[1]{\textbf{#1}}
\newcommand{\code}[1]{\texttt{#1}}
\newcommand{\classname}[1]{\texttt{#1}}
\newcommand{\methodname}[1]{\texttt{#1}}

% Proper quotation marks
\newcommand{\mnemonic}[1]{``#1''}


\title{ઇલેક્ટ્રોનિક મેઝરમેન્ટ્સ એન્ડ ઇન્સ્ટ્રુમેન્ટ્સ (૪૩૩૧૧૦૨) - ગ્રીષ્મ ૨૦૨૪ સોલ્યુશન}
\date{જૂન ૧૦, ૨૦૨૪}

\begin{document}
\maketitle

\questionmarks{1(અ)}{3}{નીચેના શબ્દને વ્યાખ્યાયિત કરો: (1) Accuracy (2) precision (3) Reproducibility}

\begin{solutionbox}
\begin{itemize}
    \item \keyword{Accuracy}: માપવામાં આવેલા મૂલ્યની વાસ્તવિક મૂલ્યની નજીકતા.
    \item \keyword{Precision}: એક જ ઇનપુટને વારંવાર લાગુ કરવા પર સમાન આઉટપુટ પુનઃઉત્પન્ન કરવાની સાધનની ક્ષમતા.
    \item \keyword{Reproducibility}: બદલાયેલી પરિસ્થિતિઓ (અલગ પદ્ધતિ, નિરીક્ષક, અથવા સમય) હેઠળ માપવામાં આવે ત્યારે સમાન જથ્થાનાં માપનના પરિણામો વચ્ચે સંમતિની ડિગ્રી.
\end{itemize}
\end{solutionbox}

\begin{mnemonicbox}
\mnemonic{APR: ચોક્કસતા-સત્ય માટે, ચોકસાઈ-પુનરાવર્તન, પુન:ઉત્પાદન-ફેરફાર હેઠળ}
\end{mnemonicbox}

\questionmarks{1(બ)}{4}{RTD ટ્રાન્સડ્યુસરનું બાંધકામ જરૂરી આકૃતિ સાથે વિગતવાર સમજાવો. તેની એપ્લિકેશનની યાદી બનાવો.}

\begin{solutionbox}
RTD (Resistance Temperature Detector) એ તાપમાન સેન્સર છે જે ધાતુઓના ઇલેક્ટ્રિકલ રેસિસ્ટન્સ તાપમાન સાથે બદલાય છે તે સિદ્ધાંત પર કાર્ય કરે છે.

\textbf{આકૃતિ:}

\begin{center}
\begin{tikzpicture}[node distance=1.5cm, auto]
    \node [gtu block] (S) {Sensing Element};
    \node [gtu block, right=of S] (L) {Lead Wires};
    \node [gtu block, right=of L] (Sup) {Support};
    \node [gtu block, right=of Sup] (P) {Protective Sheath};
    
    \draw [gtu arrow] (S) -- (L);
    \draw [gtu arrow] (L) -- (Sup);
    \draw [gtu arrow] (Sup) -- (P);
\end{tikzpicture}
\captionof{figure}{RTD બાંધકામ બ્લોક ડાયાગ્રામ}
\end{center}

\begin{itemize}
    \item \textbf{સેન્સિંગ એલિમેન્ટ}: સિરામિક કોર પર વીંટળાયેલા શુદ્ધ પ્લેટિનમ, નિકલ, અથવા કોપર વાયર.
    \item \textbf{લીડ વાયર}: RTDને માપન સર્કિટ સાથે જોડે છે.
    \item \textbf{સપોર્ટ}: સેન્સિંગ એલિમેન્ટને યાંત્રિક સ્થિરતા પ્રદાન કરે છે.
    \item \textbf{પ્રોટેક્ટિવ શીથ}: સેન્સિંગ એલિમેન્ટને બાહ્ય વાતાવરણથી રક્ષણ આપે છે.
\end{itemize}

\textbf{RTDના ઉપયોગો:}
\begin{itemize}
    \item પ્રોસેસ ઉદ્યોગોમાં તાપમાન માપન.
    \item ફૂડ પ્રોસેસિંગ તાપમાન મોનિટરિંગ.
    \item HVAC સિસ્ટમ્સ.
    \item મેડિકલ ઉપકરણો.
\end{itemize}
\end{solutionbox}

\begin{mnemonicbox}
\mnemonic{RTD: Resistance Temperature Detector - ચોક્કસ તાપમાન માપન}
\end{mnemonicbox}

\questionmarks{1(ક)}{7}{સર્કિટ ડાયાગ્રામ સાથે મેક્સવેલના બ્રિજનું કાર્ય સમજાવો. તેના ફાયદા, ગેરફાયદા અને એપ્લિકેશનોની યાદી બનાવો.}

\begin{solutionbox}
મેક્સવેલ બ્રિજનો ઉપયોગ જાણીતા કેપેસિટન્સ અને રેસિસ્ટન્સની સંદર્ભમાં અજ્ઞાત ઇન્ડક્ટન્સ માપવા માટે થાય છે.

\textbf{સર્કિટ આકૃતિ:}

\begin{center}
\begin{circuitikz}[american, scale=0.8, transform shape]
    \draw (0,4) coordinate(left) to[short, *-] (0,4);
    \draw (6,4) coordinate(right) to[short, *-] (6,4);
    \draw (3,7) coordinate(top) to[short, *-] (3,7);
    \draw (3,1) coordinate(bottom) to[short, *-] (3,1);

    % Arm 1 (Top-Left): Lx series Rx
    \draw (left) to[R, l=$R_1$] (0.5, 6) to[L, l=$L_1$] (top);

    % Arm 2 (Top-Right): R2
    \draw (top) to[R, l=$R_2$] (right);

    % Arm 3 (Bottom-Right): R3
    \draw (right) to[R, l=$R_3$] (bottom);

    % Arm 4 (Bottom-Left): R4 || C4
    \draw (bottom) -- (2.5, 2.5) to[R, l=$R_4$] (0.5, 4.5) -- (left);
    \draw (bottom) -- (1, 2) to[C, l=$C_4$] (-0.5, 3.5) -- (left);

    % Detector
    \draw (top) to[rmeter, t=D] (bottom);

    % Source
    \draw (left) to[short] (-2,4) to[short] (-2,0) to[battery1, l=$E$] (8,0) to[short] (8,4) to[short] (right);
\end{circuitikz}
\captionof{figure}{મેક્સવેલનો ઇન્ડક્ટન્સ કેપેસિટન્સ બ્રિજ}
\end{center}

\textbf{કાર્યપ્રણાલી:}
સંતુલન શરત પર: $L_1 = C_4 \times R_2 \times R_3$

જ્યારે બ્રિજ સંતુલિત હોય, ત્યારે ડિટેક્ટર શૂન્ય કરંટ દર્શાવે છે. અજ્ઞાત ઇન્ડક્ટન્સ $L_1$ ઉપરોક્ત સમીકરણનો ઉપયોગ કરીને ગણવામાં આવે છે, જ્યાં $C_4$ જાણીતા કેપેસિટન્સ અને $R_2, R_3$ જાણીતા રેસિસ્ટન્સ છે.

\begin{center}
\captionof{table}{મેક્સવેલ બ્રિજ પરિમાણો}
\begin{tabulary}{\linewidth}{|L|L|}
\hline
\textbf{પરિમાણ} & \textbf{મૂલ્ય} \\ \hline
સંતુલન સમીકરણ & $L_1 = C_4 \times R_2 \times R_3$ \\ \hline
ક્વોલિટી ફેક્ટર & $Q = \omega L_1/R_1 = \omega C_4 R_3$ \\ \hline
\end{tabulary}
\end{center}

\textbf{ફાયદાઓ:}
\begin{itemize}
    \item મધ્યમ Q ઇન્ડક્ટર્સ માટે ઉચ્ચ ચોકસાઈ.
    \item સંતુલન સમીકરણો ફ્રીક્વન્સીથી સ્વતંત્ર છે.
    \item ઇન્ડક્ટન્સ માટે સરળ ગણતરી.
\end{itemize}

\textbf{ગેરફાયદાઓ:}
\begin{itemize}
    \item ઓછા Q ઇન્ડક્ટર માપન માટે યોગ્ય નથી.
    \item પરિવર્તનશીલ સ્ટાન્ડર્ડ કેપેસિટરની જરૂર પડે છે.
    \item સ્ટ્રે કેપેસિટન્સથી પ્રભાવિત થાય છે.
\end{itemize}

\textbf{એપ્લિકેશન્સ:}
\begin{itemize}
    \item પ્રયોગશાળાઓમાં ઇન્ડક્ટન્સ માપવા.
    \item ઇન્ડક્ટન્સ માનકોનું કેલિબ્રેશન.
    \item ઇન્ડક્ટિવ ઘટકોનું પરીક્ષણ.
\end{itemize}
\end{solutionbox}

\begin{mnemonicbox}
\mnemonic{મેક્સવેલની જાદુ: ઇન્ડક્ટન્સ = કેપેસિટન્સ × રેસિસ્ટન્સ વર્ગ}
\end{mnemonicbox}

\questionmarks{1(ક) OR}{7}{સંતુલન સ્થિતિ માટે સર્કિટ ડાયાગ્રામ સાથે વ્હીટસ્ટોન બ્રિજનું કાર્ય સમજાવો. તેના ફાયદા, ગેરફાયદા અને એપ્લિકેશનોની યાદી બનાવો.}

\begin{solutionbox}
વ્હીટસ્ટોન બ્રિજનો ઉપયોગ જાણીતા રેસિસ્ટન્સ મૂલ્યો સાથે તેની તુલના કરીને અજ્ઞાત રેસિસ્ટન્સ માપવા માટે થાય છે.

\textbf{સર્કિટ આકૃતિ:}

\begin{center}
\begin{circuitikz}[american, scale=0.8, transform shape]
    % Diamond structure
    \draw (0,3) coordinate(left) -- (3,6) coordinate(top) -- (6,3) coordinate(right) -- (3,0) coordinate(bottom) -- (0,3);
    
    % Resistors (clearing the path first)
    \draw (left) to[R, l=$P$, *-*] (top);
    \draw (top) to[R, l=$Q$, *-*] (right);
    \draw (right) to[R, l=$S$, *-*] (bottom);
    \draw (bottom) to[R, l=$R$, *-*] (left);
    
    % Galvanometer
    \draw (top) to[rmeter, t=G] (bottom);
    
    % Battery
    \draw (left) to[short] (-1,3) to[short] (-1,-1) to[battery1, l=$E$] (7,-1) to[short] (7,3) to[short] (right);
\end{circuitikz}
\captionof{figure}{વ્હીટસ્ટોન બ્રિજ સર્કિટ}
\end{center}

\textbf{કાર્યપ્રણાલી:}
સંતુલન સ્થિતિ પર: $P/Q = R/S$ અથવા $R = S \times (P/Q)$

જ્યારે બ્રિજ સંતુલિત હોય, ત્યારે ગેલ્વેનોમીટર શૂન્ય વિક્ષેપ બતાવે છે. અજ્ઞાત રેસિસ્ટન્સ $R$ અન્ય રેસિસ્ટન્સના ગુણોત્તરનો ઉપયોગ કરીને ગણવામાં આવે છે.

\begin{center}
\captionof{table}{વ્હીટસ્ટોન બ્રિજ ઘટકો}
\begin{tabulary}{\linewidth}{|L|L|}
\hline
\textbf{ઘટક} & \textbf{કાર્ય} \\ \hline
$P, Q, S$ & જાણીતા રેસિસ્ટન્સ \\ \hline
$R$ & અજ્ઞાત રેસિસ્ટન્સ \\ \hline
$G$ & ગેલ્વેનોમીટર (ડિટેક્ટર) \\ \hline
$E$ & DC વોલ્ટેજ સ્ત્રોત \\ \hline
\end{tabulary}
\end{center}

\textbf{ફાયદાઓ:}
\begin{itemize}
    \item રેસિસ્ટન્સ માપનમાં ઉચ્ચ ચોકસાઈ.
    \item સરળ બાંધકામ અને સંચાલન.
    \item રેસિસ્ટન્સ માપનની વિશાળ શ્રેણી.
\end{itemize}

\textbf{ગેરફાયદાઓ:}
\begin{itemize}
    \item ખૂબ ઓછા અથવા ખૂબ ઊંચા રેસિસ્ટન્સ માપી શકતા નથી.
    \item પાવર સોર્સ તરીકે બેટરીની જરૂર પડે છે.
    \item રેસિસ્ટર્સ પર તાપમાનની અસરો ભૂલો પેદા કરે છે.
\end{itemize}

\textbf{એપ્લિકેશન્સ:}
\begin{itemize}
    \item ચોક્સાઈપૂર્ણ રેસિસ્ટન્સ માપન.
    \item સ્ટ્રેન ગેજ માપન.
    \item RTDsનો ઉપયોગ કરીને તાપમાન સંવેદન.
    \item ટ્રાન્સડ્યુસર એપ્લિકેશન્સ.
\end{itemize}
\end{solutionbox}

\begin{mnemonicbox}
\mnemonic{જ્યારે વ્હીટસ્ટોન સંતુલિત થાય: વિરોધાભાસી પાસાઓનું ગુણનફળ સમાન હોય છે ($P\times S = Q\times R$)}
\end{mnemonicbox}

% Q2 Start
\questionmarks{2(અ)}{3}{મૂવિંગ આયર્ન અને મૂવિંગ કોઇલ પ્રકારના સાધનોની સરખામણી કરો.}

\begin{solutionbox}
\begin{center}
\captionof{table}{મૂવિંગ આયર્ન vs મૂવિંગ કોઇલ સાધનો}
\begin{tabulary}{\linewidth}{|L|L|L|}
\hline
\textbf{વિશેષતા} & \textbf{મૂવિંગ આયર્ન ટાઇપ} & \textbf{મૂવિંગ કોઇલ ટાઇપ} \\ \hline
સિદ્ધાંત & ચુંબકીય આકર્ષણ/અપકર્ષણ & ઇલેક્ટ્રોમેગ્નેટિક બળ \\ \hline
સ્કેલ & બિન-એકસરખી & એકસરખી \\ \hline
ડેમ્પિંગ & નબળી & સારી \\ \hline
ચોકસાઈ & ઓછી ચોકસાઈ (2-5\%) & ઉચ્ચ ચોકસાઈ (0.1-2\%) \\ \hline
આવૃત્તિ શ્રેણી & DC અને AC & DC ફક્ત (રેક્ટિફાયર વિના) \\ \hline
પાવર વપરાશ & ઉચ્ચ & નીચો \\ \hline
કિંમત & ઓછી ખર્ચાળ & વધુ ખર્ચાળ \\ \hline
\end{tabulary}
\end{center}
\end{solutionbox}

\begin{mnemonicbox}
\mnemonic{IMAP-CAD: આયર્ન-ચુંબકીય-AC-નબળી ડેમ્પિંગ, કોઇલ-ચોક્કસ-DC-સારી ડેમ્પિંગ}
\end{mnemonicbox}

\questionmarks{2(બ)}{4}{Successive approximation પ્રકાર DVM નું કાર્ય અને બાંધકામ જરૂરી ડાયાગ્રામ સાથે સમજાવો.}

\begin{solutionbox}
Successive Approximation પ્રકારનું Digital Voltmeter (DVM) દ્વિઅંકી શોધ તકનીકનો ઉપયોગ કરીને એનાલોગ વોલ્ટેજને ડિજિટલ મૂલ્યમાં રૂપાંતરિત કરે છે.

\textbf{બ્લોક ડાયાગ્રામ:}
\begin{center}
\begin{tikzpicture}[node distance=1.5cm, auto]
    \node [gtu block] (SH) {Sample \& Hold};
    \node [gtu block, right=of SH] (Comp) {Comparator};
    \node [gtu block, right=of Comp, align=center] (SAR) {SAR Logic \&\\Register};
    \node [gtu block, below=of SAR] (DAC) {DAC};
    \node [gtu block, right=of SAR] (Disp) {Display};
    \node [coordinate, left=of SH] (In) {};
    
    \draw [gtu arrow] (In) -- node[above]{Input} (SH);
    \draw [gtu arrow] (SH) -- (Comp);
    \draw [gtu arrow] (Comp) -- (SAR);
    \draw [gtu arrow] (SAR) -- (DAC);
    \draw [gtu arrow] (DAC) -| (Comp);
    \draw [gtu arrow] (SAR) -- (Disp);
    
    \node [above=0.5cm of SAR] (Clk) {Clock};
    \draw [gtu arrow] (Clk) -- (SAR);
\end{tikzpicture}
\captionof{figure}{Successive Approximation DVM}
\end{center}

\textbf{કાર્યપ્રણાલી:}
\begin{itemize}
    \item Sample \& Hold સર્કિટ ઇનપુટ વોલ્ટેજને પકડે છે.
    \item SAR MSBને 1, અન્ય બિટ્સને 0 પર સેટ કરે છે.
    \item DAC ડિજિટલ શબ્દને એનાલોગ વોલ્ટેજમાં રૂપાંતરિત કરે છે.
    \item કમ્પેરેટર DAC આઉટપુટની ઇનપુટ વોલ્ટેજ સાથે તુલના કરે છે.
    \item જો DAC આઉટપુટ $>$ ઇનપુટ, બિટ 0 પર રીસેટ થાય છે; અન્યથા 1 રાખે છે.
    \item બધા બિટ્સનું પરીક્ષણ થાય ત્યાં સુધી પ્રક્રિયા આગલા બિટ માટે પુનરાવર્તિત થાય છે.
    \item અંતિમ ડિજિટલ શબ્દ ઇનપુટ વોલ્ટેજનું પ્રતિનિધિત્વ કરે છે.
\end{itemize}

\textbf{ફાયદાઓ:}
\begin{itemize}
    \item મધ્યમ રૂપાંતર ગતિ (10-100 $\mu$s).
    \item સારા રિઝોલ્યુશન અને ચોકસાઈ.
    \item મધ્યમ કિંમત.
\end{itemize}
\end{solutionbox}

\begin{mnemonicbox}
\mnemonic{SAR DVM: Sample-And-Register દ્વારા Digital-Voltage-Matching}
\end{mnemonicbox}

\questionmarks{2(ક)}{7}{1- 10 એમ્પીયર સુધી રીડિંગ કરતી મૂવિંગ કોઇલ એમીટર 0.02 ઓહ્મનો પ્રતિકાર ધરાવે છે. 1000 એમ્પીયર સુધીનો વર્તમાન વાંચવા માટે આ સાધન કેવી રીતે અપનાવી શકાય? \\ 2- મૂવિંગ કોઇલ વોલ્ટમીટર 200 mV સુધીનું રીડિંગ 5 ઓહ્મનું પ્રતિકાર ધરાવે છે. 300 વોલ્ટ સુધીના વોલ્ટેજને વાંચવા માટે આ સાધનને કેવી રીતે અપનાવી શકાય?}

\begin{solutionbox}
\textbf{ભાગ 1: એમીટર રેન્જ એક્સટેન્શન}

એમીટરની રેન્જ 10A થી 1000A સુધી વધારવા માટે, મીટરની સમાંતર શંટ રેસિસ્ટર જોડવામાં આવે છે.

\textbf{આકૃતિ:}
\begin{center}
\begin{circuitikz}[american, scale=0.8, transform shape]
    \draw (0,2) to[short, i=$I$] (1,2) coordinate(split);
    \draw (split) -- (1,3) to[R, l=$R_{sh}$, i=$I_{sh}$] (4,3) -- (4,2) coordinate(join);
    \draw (split) -- (1,1) to[rmeter, l=$R_m$, t=A, i=$I_m$] (4,1) -- (join);
    \draw (join) to[short] (5,2);
\end{circuitikz}
\captionof{figure}{શંટ સાથે એમીટર}
\end{center}

\textbf{ગણતરી:}
\begin{itemize}
    \item મૂળ મીટર રેસિસ્ટન્સ ($R_m$) = 0.02 $\Omega$
    \item મૂળ પૂર્ણ-સ્કેલ કરંટ ($I_m$) = 10 A
    \item ઇચ્છિત પૂર્ણ-સ્કેલ કરંટ ($I$) = 1000 A
    \item શંટ દ્વારા કરંટ ($I_{sh}$) = $I - I_m = 1000 - 10 = 990$ A
    \item મીટર પરનું વોલ્ટેજ = શંટ પરનું વોલ્ટેજ
    \item $I_m \times R_m = I_{sh} \times R_{sh}$
    \item $R_{sh} = \frac{10 \times 0.02}{990} = 0.000202 \Omega$
\end{itemize}

\textbf{ભાગ 2: વોલ્ટમીટર રેન્જ એક્સટેન્શન}

વોલ્ટમીટરની રેન્જ 200mV થી 300V સુધી વધારવા માટે, મીટર સાથે શ્રેણીમાં મલ્ટિપ્લાયર રેસિસ્ટર જોડવામાં આવે છે.

\textbf{આકૃતિ:}
\begin{center}
\begin{circuitikz}[american, scale=0.8, transform shape]
    \draw (0,1) to[short, o-] (1,1) to[R, l=$R_s$] (3,1) to[rmeter, l=$R_m$, t=V] (5,1) to[short, -o] (6,1);
    \node at (0,1) [left] {+};
    \node at (6,1) [right] {-};
    \node at (3,0) {$V_{total}$};
\end{circuitikz}
\captionof{figure}{મલ્ટિપ્લાયર સાથે વોલ્ટમીટર}
\end{center}

\textbf{ગણતરી:}
\begin{itemize}
    \item મૂળ મીટર રેસિસ્ટન્સ ($R_m$) = 5 $\Omega$
    \item મૂળ પૂર્ણ-સ્કેલ વોલ્ટેજ ($V_m$) = 200 mV = 0.2 V
    \item ઇચ્છિત પૂર્ણ-સ્કેલ વોલ્ટેજ ($V$) = 300 V
    \item શ્રેણી રેસિસ્ટન્સ ($R_s$) = $R_m \times (\frac{V}{V_m} - 1)$
    \item $R_s = 5 \times (\frac{300}{0.2} - 1) = 5 \times 1499 = 7495 \Omega$
\end{itemize}
\end{solutionbox}

\begin{mnemonicbox}
\mnemonic{શંટ-શ્રેણી: શંટ-કરંટ-માટે, શ્રેણી-વોલ્ટેજ-માટે}
\end{mnemonicbox}

\questionmarks{2(અ) OR}{3}{ક્લેમ્પનું મીટર કાર્ય અને બાંધકામ જરૂરી ડાયાગ્રામ સાથે સમજાવો.}

\begin{solutionbox}
ક્લેમ્પ ઓન મીટર (કરંટ ક્લેમ્પ) ઇલેક્ટ્રોમેગ્નેટિક ઇન્ડક્શનનો ઉપયોગ કરીને સર્કિટને તોડ્યા વિના કરંટ માપે છે.

\textbf{આકૃતિ:}
\begin{center}
\begin{tikzpicture}[node distance=1.5cm, auto]
    \node [gtu block, align=center] (Clamp) {Clamp Jaw\\(Transformer Core)};
    \node [gtu block, below=of Clamp] (Rect) {Rectifier};
    \node [gtu block, below=of Rect] (ADC) {Measuring Ckt};
    \node [gtu block, below=of ADC] (Disp) {Display};
    \draw [thick] (Clamp.west) -- ++(-1,0) node[left] {Conductor};
    \draw [thick] (Clamp.east) -- ++(1,0) node[right] (current) {};
    
    \draw [gtu arrow] (Clamp) -- node[right]{Induced Current} (Rect);
    \draw [gtu arrow] (Rect) -- (ADC);
    \draw [gtu arrow] (ADC) -- (Disp);
\end{tikzpicture}
\captionof{figure}{ક્લેમ્પ મીટર બ્લોક ડાયાગ્રામ}
\end{center}

\textbf{બાંધકામ અને કાર્યપ્રણાલી:}
\begin{itemize}
    \item \textbf{ક્લેમ્પ જો}: સ્પ્લિટ કોર ટ્રાન્સફોર્મર જે વાહકને ફરતે રાખવા માટે ખોલી શકાય છે.
    \item \textbf{કરંટ ટ્રાન્સફોર્મર}: પ્રાથમિક કરંટને પ્રમાણસર ગૌણ કરંટમાં રૂપાંતરિત કરે છે.
    \item \textbf{રેક્ટિફાયર}: ACને માપન સર્કિટ માટે DCમાં રૂપાંતરિત કરે છે.
    \item \textbf{માપન સર્કિટ}: સિગ્નલ પર પ્રક્રિયા કરે છે અને કરંટ મૂલ્યની ગણતરી કરે છે.
    \item \textbf{ડિસ્પ્લે}: માપવામાં આવેલા કરંટ મૂલ્યને બતાવે છે.
\end{itemize}
જ્યારે કરંટ-વહન કરતો વાહક ક્લેમ્પ જો મારફતે પસાર થાય છે, ત્યારે તે ગૌણ વાઇન્ડિંગમાં પ્રાથમિક કરંટના પ્રમાણમાં કરંટ પ્રેરિત કરે છે, જેનું પછી માપન કરવામાં આવે છે.
\end{solutionbox}

\begin{mnemonicbox}
\mnemonic{CLAMP: Current-Loop Amplifies Magnetic Proportionally}
\end{mnemonicbox}

\questionmarks{2(બ) OR}{4}{PMMC સાધનોની કામગીરી જરૂરી ડાયાગ્રામ સાથે સમજાવો.}

\begin{solutionbox}
PMMC (પર્મેનન્ટ મેગ્નેટ મૂવિંગ કોઇલ) સાધનો ચુંબકીય ક્ષેત્રમાં કરંટ-વહન કરતા વાહક પર ઇલેક્ટ્રોમેગ્નેટિક બળના સિદ્ધાંત પર કાર્ય કરે છે.

\textbf{આકૃતિ:}
\begin{center}
\begin{tikzpicture}[scale=0.8]
    % Magnets
    \draw[fill=lightgray] (-2,2) rectangle (-1,-2);
    \node at (-1.5,0) {N};
    \draw[fill=lightgray] (1,2) rectangle (2,-2);
    \node at (1.5,0) {S};
    
    % Core
    \draw (0,0) circle (0.8);
    \node at (0,0) {Core};
    
    % Coil (simplified)
    \draw[thick] (-0.9, -0.2) rectangle (0.9, 0.2);
    
    % Pointer
    \draw[->, thick, red] (0,0.2) -- (0,2);
    
    % Scale
    \draw (20:2.5) arc (20:160:2.5);
    \foreach \ang in {30,60,90,120,150}
        \draw (\ang:2.4) -- (\ang:2.6);
        
    % Springs (curly lines)
    \draw[decorate, decoration={coil, aspect=0.3, segment length=1mm, amplitude=1mm}] (0,-0.2) -- (0,-1);
    \node at (0,-1.2) {Spring};

\end{tikzpicture}
\captionof{figure}{PMMC બાંધકામ}
\end{center}

\textbf{કાર્યપ્રણાલી:}
\begin{itemize}
    \item ચુંબકીય ક્ષેત્રમાં મૂકેલી લંબચોરસ કોઇલ મારફતે કરંટ વહે છે.
    \item ઇલેક્ટ્રોમેગ્નેટિક બળ કરંટના પ્રમાણમાં ટોર્ક પેદા કરે છે ($T_d \propto I$).
    \item સ્પ્રિંગ નિયંત્રિત ટોર્ક પ્રદાન કરે છે ($T_c \propto \theta$).
    \item પોઇન્ટર કરંટના પ્રમાણમાં વિક્ષેપિત થાય છે.
\end{itemize}

\textbf{ઘટકો:}
\begin{itemize}
    \item કાયમી ચુંબક મજબૂત ચુંબકીય ક્ષેત્ર બનાવે છે.
    \item સોફ્ટ આયર્ન કોર ચુંબકીય ફ્લક્સને કેન્દ્રિત કરે છે.
    \item મૂવિંગ કોઇલ માપવામાં આવતા કરંટને વહન કરે છે.
    \item કંટ્રોલ સ્પ્રિંગ્સ પુનઃપ્રાપ્તિ બળ પૂરું પાડે છે.
    \item ડેમ્પિંગ સિસ્ટમ (હવા અથવા એડી કરંટ) દોલનોને ઘટાડે છે.
\end{itemize}
\end{solutionbox}

\begin{mnemonicbox}
\mnemonic{PMMC: Permanent Magnet Makes Current-proportional movement}
\end{mnemonicbox}

\questionmarks{2(ક) OR}{7}{જરૂરી ડાયાગ્રામ અને વેવફોર્મ સાથે ઇન્ટિગ્રેટિંગ ટાઇપ DVM નું બ્લોક ડાયાગ્રામ, કામગીરી અને બાંધકામ દોરો.}

\begin{solutionbox}
ઇન્ટિગ્રેટિંગ ટાઇપ DVM (ડિજિટલ વોલ્ટમીટર) નિશ્ચિત સમય દરમિયાન ઇનપુટનું એકીકરણ કરીને એનાલોગ વોલ્ટેજને ડિજિટલ મૂલ્યમાં રૂપાંતરિત કરે છે.

\textbf{બ્લોક ડાયાગ્રામ:}
\begin{center}
\begin{tikzpicture}[node distance=1.5cm, auto]
    \node [gtu block] (Int) {Integrator};
    \node [gtu block, right=of Int] (Comp) {Comparator};
    \node [gtu block, right=of Comp] (Logic) {Control Logic};
    \node [gtu block, below=of Logic] (Counter) {Counter};
    \node [gtu block, below=of Counter] (Disp) {Display};
    
    \node [coordinate, left=of Int] (In) {};
    \node [above=0.5cm of In] {Input $V_{in}$};
    \node [below=0.5cm of In] {Ref $V_{ref}$};
    
    % Switch logic simplified
    \draw [gtu arrow] (In) -- (Int); % Switch implied
    \draw [gtu arrow] (Int) -- (Comp);
    \draw [gtu arrow] (Comp) -- (Logic);
    \draw [gtu arrow] (Logic) -- (Counter);
    \draw [gtu arrow] (Counter) -- (Disp);
    \draw [gtu arrow] (Logic) -| (Int); % Control switch
    
    \node [above=0.5cm of Logic] (Osc) {Oscillator};
    \draw [gtu arrow] (Osc) -- (Logic);
\end{tikzpicture}
\captionof{figure}{ઇન્ટિગ્રેટિંગ DVM (ડ્યુઅલ સ્લોપ)}
\end{center}

\textbf{વેવફોર્મ્સ:}
\begin{center}
\begin{tikzpicture}[scale=0.8]
    \draw[->] (0,0) -- (6,0) node[right] {$t$};
    \draw[->] (0,0) -- (0,4) node[above] {$V_{out}$};
    
    % T1 Integration
    \draw[thick] (0,0) -- (3,3) node[midway, above, sloped] {Slope $\propto V_{in}$};
    \draw[dashed] (3,0) -- (3,3);
    \node at (1.5, -0.5) {$T_1$ (Fixed)};
    
    % T2 De-integration
    \draw[thick] (3,3) -- (5,0) node[midway, above, sloped] {Slope $\propto V_{ref}$};
    \draw[dashed] (5,0) -- (5,0); % Point
    \node at (4, -0.5) {$T_2$ (Prop. to $V_{in}$)};
    
\end{tikzpicture}
\captionof{figure}{ડ્યુઅલ સ્લોપ વેવફોર્મ્સ}
\end{center}

\textbf{કાર્યપ્રણાલી:}
\begin{itemize}
    \item **ફેઝ 1**: નિશ્ચિત સમય $T_1$ માટે અજ્ઞાત વોલ્ટેજ ($V_{in}$) ને એકીકૃત કરો. કેપેસિટર ચાર્જ થાય છે.
    \item **ફેઝ 2**: આઉટપુટ શૂન્ય સુધી પહોંચે ત્યાં સુધી વિરુદ્ધ પોલેરિટીના જાણીતા સંદર્ભ વોલ્ટેજ ($V_{ref}$) ને એકીકૃત કરો. કેપેસિટર ડિસ્ચાર્જ થાય છે.
    \item **ફેઝ 3**: કાઉન્ટર ફેઝ 2 ($T_2$) દરમિયાન ક્લોક પલ્સની ગણતરી કરે છે.
    \item $V_{in} = V_{ref} \times (T_2 / T_1)$.
\end{itemize}

\textbf{ફાયદાઓ:}
\begin{itemize}
    \item ઉચ્ચ નોઇઝ રિજેક્શન.
    \item સારી ચોકસાઈ.
    \item ઓટોમેટિક ઝીરો એડજસ્ટમેન્ટ.
\end{itemize}
\end{solutionbox}

\begin{mnemonicbox}
\mnemonic{બે વાર એકીકૃત કરો: અજ્ઞાત સાથે ઉપર, સંદર્ભ સાથે નીચે}
\end{mnemonicbox}

% Q3 Start
\questionmarks{3(અ)}{3}{CRO માં અજાણ્યા ડીસી વોલ્ટેજનું મૂલ્ય શું છે, જો x-અક્ષની નીચે એક સીધી રેખા 4cm અને વોલ્ટ/ડીવ નોબ = 3V ના વિસ્થાપન સાથે મેળવવામાં આવે છે. અજ્ઞાત વોલ્ટેજ Vdc ની ગણતરી કરો.}

\begin{solutionbox}
\textbf{ગણતરી:}
\begin{itemize}
    \item વિસ્થાપન = 4 cm (x-અક્ષની નીચે)
    \item વોલ્ટ/ડીવ સેટિંગ = 3 V/ડીવ
    \item દિશા = x-અક્ષની નીચે (નકારાત્મક વોલ્ટેજ)
\end{itemize}

$$V_{dc} = -(\text{Displacement} \times \text{Volt/div})$$
$$V_{dc} = -(4 \text{ cm} \times 3 \text{ V/div})$$
$$V_{dc} = -12 \text{ V}$$

તેથી, અજ્ઞાત DC વોલ્ટેજ \textbf{-12 V} છે.
\end{solutionbox}

\begin{mnemonicbox}
\mnemonic{વોલ્ટેજ = વિક્ષેપણ × સ્કેલ}
\end{mnemonicbox}

\questionmarks{3(બ)}{4}{CRT ની આંતરિક રચના દોરો. ટૂંકમાં સમજાવો.}

\begin{solutionbox}
CRT (કેથોડ રે ટ્યુબ) એ એનાલોગ ઓસિલોસ્કોપમાં વપરાતું ડિસ્પ્લે ઉપકરણ છે.

\textbf{આકૃતિ:}
\begin{center}
\begin{tikzpicture}[scale=0.8]
    % Glass envelope (funnel shape)
    \draw (0,1) -- (2,1) -- (6,3) -- (6,-3) -- (2,-1) -- (0,-1) -- cycle;
    
    % Screen
    \draw[fill=cyan!20] (6,3) arc (90:-90:0.5 and 3) -- cycle;
    \node at (6.5,0) {Phosphor Screen};
    
    % Electron Gun components
    \draw[fill=gray] (0.2,0.2) rectangle (0.8,-0.2); \node at (0.5,-0.5) {K}; % Cathode
    \draw[fill=gray] (1.0,0.3) rectangle (1.2,-0.3); \node at (1.1,-0.6) {G}; % Grid
    \draw[fill=gray] (1.4,0.3) rectangle (1.8,-0.3); \node at (1.6,-0.6) {A}; % Anodes
    
    % Deflection Plates
    \draw (2.5,0.5) -- (3.5,0.5); \draw (2.5,-0.5) -- (3.5,-0.5); \node at (3,0.8) {Y-Plates};
    \draw (4.0, 0.5) -- (4.0, -0.5); \draw (4.2, 0.5) -- (4.2, -0.5); \node at (4.1,-0.8) {X-Plates};
    
    % Electron Beam
    \draw[dashed, red, thick] (0.8,0) -- (6,0);
\end{tikzpicture}
\captionof{figure}{CRT ની આંતરિક રચના}
\end{center}

\textbf{ઘટકો:}
\begin{itemize}
    \item \textbf{ઇલેક્ટ્રોન ગન}: હીટર, કેથોડ, કંટ્રોલ ગ્રિડ, અને એનોડ્સ સમાવે છે; ઇલેક્ટ્રોન બીમ ઉત્પન્ન કરે છે.
    \item \textbf{ફોકસિંગ સિસ્ટમ}: ઇલેક્ટ્રોસ્ટેટિક લેન્સનો ઉપયોગ કરીને ઇલેક્ટ્રોન બીમને તીક્ષ્ણ બિંદુમાં કેન્દ્રિત કરે છે.
    \item \textbf{ડિફ્લેક્શન સિસ્ટમ}: ડિફ્લેક્શન પ્લેટ્સનો ઉપયોગ કરીને ઇલેક્ટ્રોન બીમને આડી અને ઊભી રીતે વિક્ષેપિત કરે છે.
    \item \textbf{ફોસ્ફર સ્ક્રીન}: ઇલેક્ટ્રોન ઊર્જાને દૃશ્યમાન પ્રકાશમાં રૂપાંતરિત કરે છે.
    \item \textbf{ગ્લાસ એનવેલોપ}: તમામ ઘટકોને સમાવતું વેક્યુમ-સીલ કન્ટેનર.
\end{itemize}

\textbf{કાર્યપ્રણાલી:}
\begin{itemize}
    \item ઇલેક્ટ્રોન ગન ઇલેક્ટ્રોન્સ ઉત્સર્જિત કરે છે.
    \item ફોકસિંગ સિસ્ટમ ઇલેક્ટ્રોન બીમને સાંકડી બનાવે છે.
    \item ડિફ્લેક્શન પ્લેટ્સ બીમને સ્ક્રીન પર ફેરવે છે.
    \item બીમ ફોસ્ફર સ્ક્રીન પર અથડાય છે જેથી દૃશ્યમાન ટ્રેસ બને છે.
\end{itemize}
\end{solutionbox}

\begin{mnemonicbox}
\mnemonic{GFDS: ગન-ફોકસ-ડિફ્લેક્ટ-સ્ક્રીન}
\end{mnemonicbox}

\questionmarks{3(ક)}{7}{કન્સ્ટ્રક્શન, બ્લોક ડાયાગ્રામ, કામગીરી અને DSO ના ફાયદા જરૂરી ડાયાગ્રામ સાથે સમજાવો.}

\begin{solutionbox}
ડિજિટલ સ્ટોરેજ ઓસિલોસ્કોપ (DSO) એનાલોગ સિગ્નલને ડિજિટલ ફોર્મમાં રૂપાંતરિત કરે છે અને તેને ડિસ્પ્લે અને વિશ્લેષણ માટે સંગ્રહિત કરે છે.

\textbf{બ્લોક ડાયાગ્રામ:}
\begin{center}
\begin{tikzpicture}[node distance=1.5cm, auto]
    \node [gtu block] (Att) {Attenuator};
    \node [gtu block, right=of Att] (ADC) {ADC};
    \node [gtu block, right=of ADC] (Mem) {Memory};
    \node [gtu block, right=of Mem] (Micro) {Microprocessor};
    \node [gtu block, below=of Micro] (DAC) {DAC};
    \node [gtu block, left=of DAC] (Disp) {Display}; % Changed position for layout
    
    \node [coordinate, left=of Att] (In) {};
    \draw [gtu arrow] (In) -- node[above]{Input} (Att);
    
    \draw [gtu arrow] (Att) -- (ADC);
    \draw [gtu arrow] (ADC) -- (Mem);
    \draw [gtu arrow] (Mem) -- (Micro);
    \draw [gtu arrow] (Micro) -- (DAC);
    \draw [gtu arrow] (DAC) -- (Disp);
    
    \node [gtu block, above=of Micro] (CP) {Control Panel};
    \draw [gtu arrow] (CP) -- (Micro);
\end{tikzpicture}
\captionof{figure}{ડિજિટલ સ્ટોરેજ ઓસિલોસ્કોપ બ્લોક ડાયાગ્રામ}
\end{center}

\textbf{બાંધકામ અને કાર્યપ્રણાલી:}
\begin{itemize}
    \item \textbf{ઇનપુટ સ્ટેજ}: એટેન્યુએટર/એમ્પ્લિફાયર સિગ્નલને કન્ડિશન કરે છે.
    \item \textbf{ADC}: એનાલોગ સિગ્નલને સેમ્પલિંગ રેટ પર ડિજિટલમાં રૂપાંતરિત કરે છે.
    \item \textbf{મેમરી}: ડિજિટલ સેમ્પલ્સને સંગ્રહિત કરે છે.
    \item \textbf{માઇક્રોપ્રોસેસર}: ઓપરેશન નિયંત્રિત કરે છે અને ડેટા પર પ્રક્રિયા કરે છે.
    \item \textbf{DAC}: ડિસ્પ્લે માટે ડિજિટલ ડેટાને પાછો એનાલોગમાં રૂપાંતરિત કરે છે.
    \item \textbf{ડિસ્પ્લે}: વેવફોર્મ બતાવે છે.
\end{itemize}

\textbf{DSO ના ફાયદાઓ:}
\begin{itemize}
    \item પછીના વિશ્લેષણ માટે સિગ્નલ સ્ટોરેજ ક્ષમતા.
    \item પ્રી-ટ્રિગર સિગ્નલ જોવાની ક્ષમતા.
    \item સિંગલ-શોટ સિગ્નલ કેપ્ચર.
    \item ઓટોમેટિક માપન અને ગણતરીઓ.
    \item વેવફોર્મ પ્રોસેસિંગ (FFT, એવરેજિંગ, વગેરે).
    \item ડિજિટલ ઇન્ટરફેસિંગ (USB, ઇથરનેટ).
\end{itemize}
\end{solutionbox}

\begin{mnemonicbox}
\mnemonic{SAMPLE: સ્ટોર-એનાલાઇઝ-મેઝર-પ્રોસેસ-લિંક-એક્ઝામિન}
\end{mnemonicbox}

\questionmarks{3(અ) OR}{3}{CRO માં peak માટે વર્ટિકલ ડિસ્પ્લેસમેન્ટ = 1cm અને વોલ્ટ/div knob = 10mV છે. વોલ્ટેજનું ટોચનું મૂલ્ય અને RMS મૂલ્ય શોધો.}

\begin{solutionbox}
\textbf{ગણતરી:}
\begin{itemize}
    \item વર્ટિકલ ડિસ્પ્લેસમેન્ટ (પીક) = 1 cm
    \item વોલ્ટ/ડીવ સેટિંગ = 10 mV/ડીવ
\end{itemize}

પીક મૂલ્ય ($V_p$) = ડિસ્પ્લેસમેન્ટ $\times$ વોલ્ટ/ડીવ
$$V_p = 1 \text{ cm} \times 10 \text{ mV/div} = 10 \text{ mV}$$

સાઇનોસોઇડલ વેવફોર્મ માટે:
RMS મૂલ્ય ($V_{rms}$) = $V_p \div \sqrt{2}$
$$V_{rms} = 10 \text{ mV} \div 1.414 = 7.07 \text{ mV}$$

તેથી, \textbf{પીક મૂલ્ય = 10 mV} અને \textbf{RMS મૂલ્ય = 7.07 mV}.
\end{solutionbox}

\begin{mnemonicbox}
\mnemonic{પીક-થી-RMS: $\sqrt{2}$ થી ભાગો}
\end{mnemonicbox}

\questionmarks{3(બ) OR}{4}{CRO સ્ક્રીનને વિગતવાર સમજાવો.}

\begin{solutionbox}
CRO (કેથોડ રે ઓસિલોસ્કોપ) સ્ક્રીન વેવફોર્મ્સ પ્રદર્શિત કરે છે અને માપન સંદર્ભ પ્રદાન કરે છે.

\textbf{આકૃતિ:}
\begin{center}
\begin{tikzpicture}[scale=0.6]
    % Screen bezel
    \draw[thick, rounded corners] (-4,-3) rectangle (4,3);
    
    % Graticule
    \draw[step=1cm, gray, very thin] (-3.9,-2.9) grid (3.9,2.9);
    
    % Axes
    \draw[thick, ->] (-4,0) -- (4,0) node[right] {X (સમય)};
    \draw[thick, ->] (0,-3) -- (0,3) node[above] {Y (વોલ્ટેજ)};
    
    % Trace
    \draw[blue, thick, domain=-3.8:3.8, samples=50] plot (\x, {2*sin(deg(\x))});
    
    \node at (0,-3.5) {CRO સ્ક્રીન ગ્રેટિક્યુલ સાથે};
\end{tikzpicture}
\captionof{figure}{CRO સ્ક્રીન}
\end{center}

\textbf{ઘટકો:}
\begin{itemize}
    \item \textbf{ફોસ્ફર કોટિંગ}: ઇલેક્ટ્રોન ઊર્જાને દૃશ્યમાન પ્રકાશમાં રૂપાંતરિત કરે છે.
    \item \textbf{ગ્રેટિક્યુલ}: માપન માટે ગ્રિડ પેટર્ન (સામાન્ય રીતે $8 \times 10$ ડિવિઝન્સ).
    \item \textbf{X-અક્ષ}: સમય (આડો) દર્શાવે છે.
    \item \textbf{Y-અક્ષ}: વોલ્ટેજ (ઊભો) દર્શાવે છે.
    \item \textbf{સેન્ટર પોઇન્ટ}: માપન માટે સંદર્ભ (0,0).
\end{itemize}

\textbf{સ્ક્રીન વિશેષતાઓ:}
\begin{itemize}
    \item \textbf{ડિવિઝન્સ}: સામાન્ય રીતે માપન માટે 8×10 ડિવિઝન્સ.
    \item \textbf{ઇન્ટેન્સિટી કંટ્રોલ}: ડિસ્પ્લેની ચમક એડજસ્ટ કરે છે.
    \item \textbf{ફોકસ કંટ્રોલ}: ડિસ્પ્લે થયેલા ટ્રેસને તીક્ષ્ણ બનાવે છે.
    \item \textbf{સ્કેલ ઇલ્યુમિનેશન}: ગ્રેટિક્યુલને પ્રકાશિત કરે છે.
\end{itemize}
\end{solutionbox}

\begin{mnemonicbox}
\mnemonic{PAXED: ફોસ્ફર-અક્ષો-X-સમય-Y-એમ્પ્લિટ્યુડ-સમાન-ડિવિઝન્સ}
\end{mnemonicbox}

\questionmarks{3(ક) OR}{7}{CRO નો ઉપયોગ કરીને વોલ્ટેજ, ફ્રીક્વન્સી, સમય વિલંબ અને તબક્કા કોણનું(Phase angle) માપન જરૂરી ડાયાગ્રામ સાથે સમજાવો.}

\begin{solutionbox}
CRO (કેથોડ રે ઓસિલોસ્કોપ) વિવિધ ઇલેક્ટ્રિકલ પરિમાણોને ચોકસાઈથી માપી શકે છે.

\textbf{1. વોલ્ટેજ માપન:}
\begin{center}
\begin{tikzpicture}[scale=0.5]
    \draw[->] (0,0) -- (6,0) node[right] {$t$};
    \draw[->] (0,-2) -- (0,2) node[above] {$V$};
    \draw[thick, blue] plot[domain=0:6] (\x, {1.5*sin(deg(\x))});
    \draw[<->] (1.5,0) -- (1.5,1.5) node[midway, right] {$V_{peak}$};
    \draw[<->] (4.7,-1.5) -- (4.7,1.5) node[midway, right] {$V_{pp}$};
\end{tikzpicture}
\end{center}
\begin{itemize}
    \item વર્ટિકલ પોઝિશનને સેન્ટર લાઇન પર સેટ કરો.
    \item વેવફોર્મના વર્ટિકલ ડિવિઝન્સની ગણતરી કરો.
    \item V/div સેટિંગથી ગુણો.
    \item એમ્પ્લિટ્યુડ = વર્ટિકલ ડિવિઝન્સ $\times$ V/div.
\end{itemize}

\textbf{2. ફ્રીક્વન્સી માપન:}
\begin{center}
\begin{tikzpicture}[scale=0.5]
    \draw[->] (0,0) -- (6,0) node[right] {$t$};
    \draw[->] (0,-1.5) -- (0,1.5) node[above] {$V$};
    \draw[thick, blue] plot[domain=0:6] (\x, {sin(deg(2*\x))});
    \draw[<->] (1.57, 1.2) -- (4.71, 1.2) node[midway, above] {$T$};
\end{tikzpicture}
\end{center}
\begin{itemize}
    \item સમાન બિંદુઓ વચ્ચે સમય અવધિ ($T$) માપો.
    \item ફ્રીક્વન્સી $f = 1/T$.
    \item $T = \text{Horizontal divisions} \times \text{Time/div}$.
\end{itemize}

\textbf{3. સમય વિલંબ \& 4. ફેઝ એંગલ માપન:}
\begin{center}
\begin{tikzpicture}[scale=0.5]
    \draw[->] (0,0) -- (7,0) node[right] {$t$};
    \draw[->] (0,-1.5) -- (0,1.5) node[above] {$V$};
    
    % Signal 1
    \draw[thick, blue] plot[domain=0:6] (\x, {sin(deg(\x))});
    \node[blue] at (1, 1.2) {Sig 1};
    
    % Signal 2 (Shifted)
    \draw[thick, red] plot[domain=0:6] (\x, {sin(deg(\x - 1))});
    \node[red] at (3, 1.2) {Sig 2};
    
    % Drawing delay
    \draw[dashed] (3.14, 0) -- (3.14, -1.8);
    \draw[dashed] (4.14, 0) -- (4.14, -1.8);
    \draw[<->] (3.14, -1.5) -- (4.14, -1.5) node[midway, below] {$T_d$};
    
    % Period T
    \draw[<->] (0, 1.8) -- (6.28, 1.8) node[midway, above] {$T (360^\circ)$};
\end{tikzpicture}
\end{center}
\begin{itemize}
    \item \textbf{સમય વિલંબ}: બે સિગ્નલોના અનુરૂપ બિંદુઓ વચ્ચેનું ક્ષૈતિજ અંતર ($T_d$) માપો.
    \item \textbf{ફેઝ એંગલ}:
    \begin{itemize}
        \item એક સંપૂર્ણ સાયકલની સમય અવધિ ($T$) માપો.
        \item સમય વિલંબ ($T_d$) માપો.
        \item ફેઝ એંગલ $\phi = (T_d/T) \times 360^\circ$.
    \end{itemize}
\end{itemize}
\end{solutionbox}

\begin{mnemonicbox}
\mnemonic{VFTP: વર્ટિકલ-ફ્રીક્વન્સી-ટાઇમ-ફેઝ}
\end{mnemonicbox}

% End of Q3

% Q4 Start
\questionmarks{4(અ)}{3}{Active અને passive ટ્રાન્સડ્યુસરની સરખામણી કરો.}

\begin{solutionbox}
\begin{center}
\captionof{table}{Active vs Passive ટ્રાન્સડ્યુસર}
\begin{tabulary}{\linewidth}{|L|L|L|}
\hline
\textbf{વિશેષતા} & \textbf{Active ટ્રાન્સડ્યુસર} & \textbf{Passive ટ્રાન્સડ્યુસર} \\ \hline
પાવર સ્ત્રોત & સ્વ-જનરેટિંગ (બાહ્ય પાવરની જરૂર નથી) & બાહ્ય પાવરની જરૂર પડે છે \\ \hline
આઉટપુટ & ઇનપુટથી ઊર્જા ઉત્પન્ન કરે છે & બાહ્ય ઊર્જાને સંશોધિત કરે છે \\ \hline
ઉદાહરણો & થર્મોકપલ, ફોટોવોલ્ટેઇક સેલ & સ્ટ્રેન ગેજ, RTD, LVDT \\ \hline
સંવેદનશીલતા & સામાન્ય રીતે ઓછી & સામાન્ય રીતે ઉચ્ચ \\ \hline
પ્રતિક્રિયા સમય & ઝડપી & ધીમું \\ \hline
જટિલતા & સરળ & વધુ જટિલ \\ \hline
\end{tabulary}
\end{center}
\end{solutionbox}

\begin{mnemonicbox}
\mnemonic{APE-GSR: Active-Produces-Energy, Gets-Signal-Requiring-power}
\end{mnemonicbox}

\questionmarks{4(બ)}{4}{સ્ટ્રેઈન ગેજની કામગીરીને જરૂરી ડાયાગ્રામ સાથે વિગતવાર સમજાવો. તેની એપ્લિકેશનની યાદી પણ.}

\begin{solutionbox}
સ્ટ્રેઇન ગેજ યાંત્રિક વિરૂપણને ઇલેક્ટ્રિકલ રેસિસ્ટન્સ પરિવર્તનમાં રૂપાંતરિત કરે છે.

\textbf{આકૃતિ:}
\begin{center}
\begin{tikzpicture}
    % Backing
    \draw[fill=yellow!20, rounded corners] (-2,-1) rectangle (2,1);
    % Wire grid
    \draw[thick] (-1.5, -0.8) -- (-1.5, 0.8) -- (-1.3, 0.8) -- (-1.3, -0.8) -- (-1.1, -0.8) -- (-1.1, 0.8) -- (-0.9, 0.8) -- (-0.9, -0.8) -- (-0.7, -0.8) -- (-0.7, 0.8) -- (-0.5, 0.8) -- (-0.5, -0.8) -- (-0.3, -0.8) -- (-0.3, 0.8) -- (-0.1, 0.8) -- (-0.1, -0.8) -- (0.1, -0.8) -- (0.1, 0.8) -- (0.3, 0.8) -- (0.3, -0.8) -- (0.5, -0.8) -- (0.5, 0.8) -- (0.7, 0.8) -- (0.7, -0.8) -- (0.9, -0.8) -- (0.9, 0.8) -- (1.1, 0.8) -- (1.1, -0.8) -- (1.3, -0.8) -- (1.3, 0.8) -- (1.5, 0.8) -- (1.5, -0.8);
    % Terminals
    \draw[thick] (-1.5, -0.8) -- (-1.5, -1.2) node[below] {T1};
    \draw[thick] (1.5, -0.8) -- (1.5, -1.2) node[below] {T2};
    \node at (0, 0) {Wire Grid};
\end{tikzpicture}
\captionof{figure}{બોન્ડેડ મેટલ ફોઇલ સ્ટ્રેઇન ગેજ}
\end{center}

\textbf{કાર્યપ્રણાલી:}
\begin{itemize}
    \item જ્યારે વાહક ખેંચાય છે, ત્યારે તેની લંબાઈ વધે છે અને આડછેદ વિસ્તાર ઘટે છે.
    \item આના કારણે ઇલેક્ટ્રિકલ રેસિસ્ટન્સમાં વધારો થાય છે: $\Delta R/R = GF \times \varepsilon$
    \item જ્યાં $\Delta R/R$ રેસિસ્ટન્સમાં અંશ પરિવર્તન છે, $GF$ એ ગેજ ફેક્ટર છે, $\varepsilon$ એ સ્ટ્રેઇન છે.
\end{itemize}

\textbf{પ્રકારો:}
\begin{itemize}
    \item મેટલ ફોઇલ સ્ટ્રેઇન ગેજ
    \item સેમિકન્ડક્ટર સ્ટ્રેઇન ગેજ
    \item વાયર સ્ટ્રેઇન ગેજ
\end{itemize}

\textbf{એપ્લિકેશન્સ:}
\begin{itemize}
    \item વજન પ્રણાલી માટે લોડ સેલ.
    \item સ્ટ્રક્ચરલ હેલ્થ મોનિટરિંગ.
    \item પ્રેશર સેન્સર્સ.
    \item ટોર્ક માપન.
\end{itemize}
\end{solutionbox}

\begin{mnemonicbox}
\mnemonic{STRAIN: Stretch-To-Resistance-Alteration-In-Narrow-conductor}
\end{mnemonicbox}

\questionmarks{4(ક)}{7}{ગેસ સેન્સર MQ2 ને જરૂરી ડાયાગ્રામ સાથે વિગતવાર સમજાવો.}

\begin{solutionbox}
MQ2 એ સેમિકન્ડક્ટર ગેસ સેન્સર છે જે કોમ્બસ્ટિબલ ગેસ, ધુમાડો અને LPG શોધે છે.

\textbf{આકૃતિ:}
\begin{center}
\begin{tikzpicture}[node distance=1.5cm, auto]
    \node [gtu block] (Sens) {SnO2 Sensing Element};
    \node [gtu block, below=of Sens] (Heat) {Heater};
    \node [gtu block, left=of Sens] (Net) {Mesh/Network};
    \node [gtu block, right=of Sens] (Elect) {Electrodes};
    
    \draw [thick, dashed] (-3,-3) rectangle (4,1.5);
    \node at (0.5, 1.2) {Housing};
    
    \draw [gtu arrow] (Net) -- (Sens);
    \draw [gtu arrow] (Heat) -- (Sens);
    \draw [gtu arrow] (Sens) -- (Elect);
\end{tikzpicture}
\captionof{figure}{MQ2 બાંધકામ}
\end{center}

\textbf{સર્કિટ કનેક્શન:}
\begin{center}
\begin{circuitikz}[american, scale=0.8]
    \draw (0,0) to[short, o-*] (1,0) to[R, l=$R_{load}$] (1,2) to[short] (2,2) to[R, l=$R_{sensor}$] (2,4) to[short, -o] (0,4);
    \node at (0,4) [left] {$V_{cc} (+5V)$};
    \node at (0,0) [left] {GND};
    \draw (1,2) to[short, -o] (4,2) node[right] {$V_{out}$};
    
    % Heater
    \draw (3,0) to[short, o-] (4,0) to[R, l=$Heater$] (4,1) to[short, -o] (3,1); 
    \node at (3,1) [left] {$+5V$};
    \node at (3,0) [left] {GND};
\end{circuitikz}
\captionof{figure}{MQ2 સેન્સર સર્કિટ}
\end{center}

\textbf{કાર્યપ્રણાલી:}
\begin{itemize}
    \item સ્વચ્છ હવામાં, સેન્સરનો રેસિસ્ટન્સ ઊંચો હોય છે.
    \item જ્યારે કોમ્બસ્ટિબલ ગેસ હાજર હોય, ત્યારે સપાટી પ્રતિક્રિયાઓ થાય છે.
    \item ઇલેક્ટ્રોન્સ છોડવામાં આવે છે, જેના કારણે રેસિસ્ટન્સ ઘટે છે.
    \item રેસિસ્ટન્સ ગેસ કન્સન્ટ્રેશનના પ્રમાણમાં ઘટે છે (જેથી $V_{out}$ વધે છે).
\end{itemize}

\textbf{એપ્લિકેશન્સ:}
\begin{itemize}
    \item ઘરેલુ ગેસ લીકેજ ડિટેક્ટર્સ.
    \item ઔદ્યોગિક કોમ્બસ્ટિબલ ગેસ અલાર્મ.
    \item એર ક્વોલિટી મોનિટરિંગ.
\end{itemize}
\end{solutionbox}

\begin{mnemonicbox}
\mnemonic{MQ2: Measures Quick-leaks of 2+ gases (LPG, Propane)}
\end{mnemonicbox}

\questionmarks{4(અ) OR}{3}{પ્રાથમિક અને ગૌણ ટ્રાન્સડ્યુસરની સરખામણી કરો.}

\begin{solutionbox}
\begin{center}
\captionof{table}{પ્રાથમિક vs ગૌણ ટ્રાન્સડ્યુસર}
\begin{tabulary}{\linewidth}{|L|L|L|}
\hline
\textbf{વિશેષતા} & \textbf{પ્રાથમિક ટ્રાન્સડ્યુસર} & \textbf{ગૌણ ટ્રાન્સડ્યુસર} \\ \hline
વ્યાખ્યા & સીધા જ ભૌતિક જથ્થાને ઇલેક્ટ્રિકલ સિગ્નલમાં રૂપાંતરિત કરે છે & પ્રાથમિક ટ્રાન્સડ્યુસરના આઉટપુટને વાપરવા યોગ્ય સ્વરૂપમાં રૂપાંતરિત કરે છે \\ \hline
કાર્ય & રૂપાંતરણનો પ્રથમ તબક્કો & રૂપાંતરણનો બીજો તબક્કો \\ \hline
ઉદાહરણો & થર્મોકપલ, ફોટોસેલ, પીઝોઇલેક્ટ્રિક & એમ્પ્લિફાયર્સ, ADCs, સિગ્નલ કંડિશનર્સ \\ \hline
સ્થાન & સેન્સિંગ પોઇન્ટ પર & પ્રાથમિક ટ્રાન્સડ્યુસરથી દૂર હોઈ શકે છે \\ \hline
\end{tabulary}
\end{center}
\end{solutionbox}

\begin{mnemonicbox}
\mnemonic{PS-FLIP: Primary-Senses, Secondary-Further-Level-Improves-Processing}
\end{mnemonicbox}

\questionmarks{4(બ) OR}{4}{કેપેસિટિવ ટ્રાન્સડ્યુસરને જરૂરી ડાયાગ્રામ સાથે વિગતવાર સમજાવો. તેની એપ્લિકેશનની યાદી બનાવો.}

\begin{solutionbox}
કેપેસિટિવ ટ્રાન્સડ્યુસર ભૌતિક વિસ્થાપનને કેપેસિટન્સ પરિવર્તનમાં રૂપાંતરિત કરે છે જે પછી ઇલેક્ટ્રિકલ સિગ્નલમાં રૂપાંતરિત થાય છે.

\textbf{આકૃતિ:}
\begin{center}
\begin{tikzpicture}
    % Fixed plate
    \draw [thick, fill=gray!20] (0,2) rectangle (4,2.2);
    \node at (2,2.5) {Fixed Plate};
    
    % Movable plate
    \draw [thick, fill=gray!20] (0,0) rectangle (4,0.2);
    \node at (2,-0.3) {Movable Plate};
    
    % Connections
    \draw (0,2.1) -- (-1,2.1);
    \draw (0,0.1) -- (-1,0.1);
    
    % Displacement
    \draw [->, thick] (2, -0.5) -- (2, 0) node[midway, right] {Displacement ($d$)};
    
    % Dielectric
    \node at (2,1.1) {Dielectric ($\epsilon_r$)};
    \draw [dashed] (0,0.2) -- (0,2);
    \draw [dashed] (4,0.2) -- (4,2);
\end{tikzpicture}
\captionof{figure}{કેપેસિટિવ ટ્રાન્સડ્યુસર સિદ્ધાંત}
\end{center}

\textbf{કાર્યપ્રણાલી:}
$$C = \frac{\epsilon_0 \epsilon_r A}{d}$$
કેપેસિટન્સ આમાં ફેરફાર કરીને બદલાય છે:
\begin{itemize}
    \item પ્લેટ્સ વચ્ચેનું અંતર ($d$) બદલવું.
    \item પ્લેટ્સના ઓવરલેપ વિસ્તાર ($A$) માં ફેરફાર કરવો.
    \item ડાયઇલેક્ટ્રિક કોન્સ્ટન્ટ ($\epsilon_r$) માં ફેરફાર કરવો.
\end{itemize}

\textbf{એપ્લિકેશન્સ:}
\begin{itemize}
    \item પ્રેશર સેન્સર્સ.
    \item ડિસ્પ્લેસમેન્ટ માપન.
    \item લેવલ ઇન્ડિકેટર્સ.
    \item ટચ સ્ક્રીન.
\end{itemize}
\end{solutionbox}

\begin{mnemonicbox}
\mnemonic{CAPACITIVE: Change-Area-Plates-And-Change-In-Thickness-Impacts-Value-Electrically}
\end{mnemonicbox}

\questionmarks{4(ક) OR}{7}{LVDT ટ્રાન્સડ્યુસર ઓપરેશન, બાંધકામને જરૂરી આકૃતિ સાથે વિગતવાર સમજાવો. એલવીડીટીના લાભ, ગેરલાભ અને એપ્લિકેશનની પણ યાદી બનાવો.}

\begin{solutionbox}
LVDT (લિનિયર વેરિએબલ ડિફરન્શિયલ ટ્રાન્સફોર્મર) એ ઇલેક્ટ્રોમેગ્નેટિક ટ્રાન્સડ્યુસર છે જે લીનિયર ડિસ્પ્લેસમેન્ટને ઇલેક્ટ્રિકલ સિગ્નલમાં રૂપાંતરિત કરે છે.

\textbf{આકૃતિ:}
\begin{center}
\begin{circuitikz}[american, scale=0.8]
    % Core
    \draw [fill=gray!30] (1,3) rectangle (3,5);
    \node at (2,4) {Core};
    \draw [->, thick] (2,5.2) -- (2,5.8) node[above] {Motion};
    \draw [->, thick] (2,2.8) -- (2,2.2);

    % Primary
    \draw (0,3.5) to[L, l=$P$] (0,4.5);
    \draw (-1,3.5) to[sinusoidal voltage source, l=$V_{in}$] (-1,4.5);
    \draw (-1,3.5) -- (0,3.5); \draw (-1,4.5) -- (0,4.5);

    % Secondaries
    \draw (4,5) to[L, l=$S_1$] (4,6);
    \draw (4,2) to[L, l=$S_2$] (4,3);
    
    % Connection (Series Opposition)
    \draw (4,3) -- (4,5); % Connecting S1 bottom to S2 top? No, usually series opposition.
    % Let's draw output
    \draw (4,6) -- (5,6) node[right] {+};
    \draw (4,2) -- (5,2) node[right] {-};
    \node at (5.5, 4) {$V_{out} = V_{s1} - V_{s2}$};
\end{circuitikz}
\captionof{figure}{LVDT સ્કીમેટિક}
\end{center}

\textbf{બાંધકામ:}
\begin{itemize}
    \item \textbf{પ્રાઇમરી કોઇલ}: સેન્ટર કોઇલ જે AC સ્ત્રોત દ્વારા ઉત્તેજિત થાય છે.
    \item \textbf{સેકન્ડરી કોઇલ્સ}: સીરીઝ વિરોધમાં જોડાયેલી બે કોઇલ.
    \item \textbf{કોર}: ફેરોમેગ્નેટિક મટીરિયલ જે માપવામાં આવતા ડિસ્પ્લેસમેન્ટ સાથે ખસે છે.
\end{itemize}

\textbf{કાર્યપ્રણાલી:}
\begin{itemize}
    \item પ્રાઇમરી કોઇલને AC ઉત્તેજના આપવામાં આવે છે.
    \item નલ પોઝિશન (સેન્ટર) પર, સેકન્ડરી કોઇલ્સમાં સમાન વોલ્ટેજ પ્રેરિત થાય છે.
    \item કોરને ખસેડવાથી ચુંબકીય કપલિંગ બદલાય છે.
    \item ડિફરન્શિયલ વોલ્ટેજ ડિસ્પ્લેસમેન્ટના પ્રમાણમાં હોય છે.
\end{itemize}

\textbf{ફાયદાઓ:}
\begin{itemize}
    \item નોન-કોન્ટેક્ટ ઓપરેશન (ઘર્ષણ વિનાનું).
    \item ઉચ્ચ રિઝોલ્યુશન અને સંવેદનશીલતા.
    \item લાંબું ઓપરેશનલ જીવન.
\end{itemize}

\textbf{ગેરફાયદાઓ:}
\begin{itemize}
    \item AC ઉત્તેજના સ્ત્રોતની જરૂર પડે છે.
    \item બાહ્ય ચુંબકીય ક્ષેત્રો પ્રત્યે સંવેદનશીલ.
\end{itemize}

\textbf{એપ્લિકેશન્સ:}
\begin{itemize}
    \item મશીન ટૂલ પોઝિશનિંગ.
    \item રોબોટિક્સ અને ઓટોમેશન.
    \item એરક્રાફ્ટ કંટ્રોલ સિસ્ટમ્સ.
\end{itemize}
\end{solutionbox}

\begin{mnemonicbox}
\mnemonic{LVDT: Linear-Variation-Detected-Through electromagnetic induction}
\end{mnemonicbox}

% Q5 Start
\questionmarks{5(અ)}{3}{થર્મોકપલ સેન્સરનું કાર્ય જરૂરી ડાયાગ્રામ સાથે વિગતવાર સમજાવો.}

\begin{solutionbox}
થર્મોકપલ એ સીબેક ઇફેક્ટ પર આધારિત તાપમાન સેન્સર છે, જ્યાં બે અસમાન ધાતુઓના જંક્શન તાપમાનના તફાવતના પ્રમાણમાં વોલ્ટેજ ઉત્પન્ન કરે છે.

\textbf{આકૃતિ:}
\begin{center}
\begin{circuitikz}[american, scale=0.8]
    \draw (0,0) coordinate(ref) node[left] {Ref Junction ($T_c$)} -- (2,0) coordinate(join1) -- (2,2) coordinate(hot) node[right] {Hot Junction ($T_h$)};
    \draw (ref) -- (0,2) -- (join1); % This is weird, let's do standard
    
    \draw (0,0) to[short, -*] (3,0); % Wire A
    \draw (0,2) to[short, -*] (3,2); % Wire A
    \draw (3,0) -- (4,1) -- (3,2); % Wire B (Junction at 4,1)
    
    \node at (4.2,1) {Hot ($T_h$)};
    \node at (-0.2,1) {Cold ($T_c$)};
    
    \draw (0,2) to[rmeter, t=mV] (0,0);
    
    \node at (1.5, 2.2) {Metal A};
    \node at (1.5, -0.2) {Metal A};
    \node at (3.5, 1.5) {Metal B};
\end{circuitikz}
\captionof{figure}{થર્મોકપલ સર્કિટ}
\end{center}

\textbf{કાર્યપ્રણાલી:}
\begin{itemize}
    \item બે અસમાન ધાતુઓ બે બિંદુઓ (હોટ અને કોલ્ડ જંક્શન) પર જોડાયેલા છે.
    \item જંક્શન વચ્ચેના તાપમાનના તફાવતથી સીબેક વોલ્ટેજ ઉત્પન્ન થાય છે.
    \item ઉત્પન્ન થયેલ EMF તાપમાનના તફાવતના પ્રમાણમાં હોય છે.
\end{itemize}

\textbf{પ્રકારો:}
\begin{itemize}
    \item ટાઇપ K (ક્રોમેલ-એલુમેલ)
    \item ટાઇપ J (આયર્ન-કોન્સ્ટન્ટન)
    \item ટાઇપ T (કોપર-કોન્સ્ટન્ટન)
\end{itemize}
\end{solutionbox}

\begin{mnemonicbox}
\mnemonic{THC: Temperature-produces Hot-junction Current}
\end{mnemonicbox}

\questionmarks{5(બ)}{4}{ડીજીટલ આઈસી ટેસ્ટરનું કાર્ય જરૂરી ડાયાગ્રામ સાથે વિગતવાર સમજાવો.}

\begin{solutionbox}
ડિજિટલ IC ટેસ્ટર ટેસ્ટ વેક્ટર્સ લાગુ કરીને અને પ્રતિસાદોનું વિશ્લેષણ કરીને ડિજિટલ ઇન્ટિગ્રેટેડ સર્કિટની કાર્યક્ષમતાનું પરીક્ષણ કરવા માટે વપરાય છે.

\textbf{બ્લોક ડાયાગ્રામ:}
\begin{center}
\begin{tikzpicture}[node distance=1.5cm, auto]
    \node [gtu block] (Ctrl) {Control Unit};
    \node [gtu block, right=of Ctrl] (Gen) {Test Vector Gen};
    \node [gtu block, right=of Gen] (IC) {IC Under Test};
    \node [gtu block, below=of IC] (Resp) {Response Analyzer};
    \node [gtu block, left=of Resp] (Disp) {Display};
    \node [gtu block, left=of Disp] (UI) {User Interface};
    
    \draw [gtu arrow] (UI) -- (Ctrl);
    \draw [gtu arrow] (Ctrl) -- (Gen);
    \draw [gtu arrow] (Gen) -- (IC);
    \draw [gtu arrow] (IC) -- (Resp);
    \draw [gtu arrow] (Resp) -- (Disp);
    \draw [gtu arrow] (Ctrl) -- (Disp);
\end{tikzpicture}
\captionof{figure}{ડિજિટલ IC ટેસ્ટર}
\end{center}

\textbf{કાર્યપ્રણાલી:}
\begin{itemize}
    \item IC યોગ્ય ઓરિએન્ટેશન સાથે ટેસ્ટ સોકેટમાં મૂકવામાં આવે છે.
    \item ટેસ્ટ મોડ પસંદ કરવામાં આવે છે.
    \item ટેસ્ટ વેક્ટર્સ IC પિન્સ પર લાગુ થાય છે.
    \item આઉટપુટ રિસ્પોન્સની અપેક્ષિત પરિણામો સાથે તુલના કરવામાં આવે છે.
    \item પાસ/ફેલ સૂચન પ્રદર્શિત થાય છે.
\end{itemize}
\end{solutionbox}

\begin{mnemonicbox}
\mnemonic{VECTOR: Verify-Each-Circuit-Through-Output-Response}
\end{mnemonicbox}

\questionmarks{5(ક)}{7}{ફંક્શન જનરેટરનું કાર્ય જરૂરી ડાયાગ્રામ સાથે વિગતવાર સમજાવો.}

\begin{solutionbox}
ફંક્શન જનરેટર વિવિધ વેવફોર્મ્સ (સાઇન, સ્ક્વેર, ટ્રાયએંગલ) એડજસ્ટેબલ ફ્રીક્વન્સી અને એમ્પ્લિટ્યુડ સાથે ઉત્પન્ન કરે છે.

\textbf{બ્લોક ડાયાગ્રામ:}
\begin{center}
\begin{tikzpicture}[node distance=1.5cm, auto]
    \node [gtu block] (Freq) {Freq Control};
    \node [gtu block, right=of Freq] (Osc) {Oscillator};
    \node [gtu block, right=of Osc] (Shape) {Wave Shaper};
    \node [gtu block, below=of Shape] (Att) {Attenuator};
    \node [gtu block, right=of Att] (OutAmp) {Output Amp};
    \node [coordinate, right=of OutAmp] (Out) {};

    \draw [gtu arrow] (Freq) -- (Osc);
    \draw [gtu arrow] (Osc) -- node[above]{Triangle} (Shape);
    \draw [gtu arrow] (Shape) -- (Att);
    \draw [gtu arrow] (Att) -- (OutAmp);
    \draw [gtu arrow] (OutAmp) -- (Out) node[right]{Output};
    
    \node [gtu block, below=of Osc] (Type) {Function Switch};
    \draw [gtu arrow] (Type) -- (Shape);
\end{tikzpicture}
\captionof{figure}{ફંક્શન જનરેટર બ્લોક ડાયાગ્રામ}
\end{center}

\textbf{કાર્યપ્રણાલી:}
\begin{itemize}
    \item \textbf{ઓસિલેટર}: મૂળભૂત વેવફોર્મ (સામાન્ય રીતે ટ્રાયએંગલ) ઉત્પન્ન કરે છે.
    \item \textbf{વેવશેપિંગ સર્કિટ}: સાઇન, સ્ક્વેર, અથવા ટ્રાયએંગલ વેવફોર્મમાં રૂપાંતરિત કરે છે.
    \item \textbf{એટેન્યુએટર}: સિગ્નલની એમ્પ્લિટ્યુડ નિયંત્રિત કરે છે.
    \item \textbf{આઉટપુટ એમ્પ્લિફાયર}: ઓછા આઉટપુટ ઇમ્પીડન્સ અને DC ઓફસેટ પ્રદાન કરે છે.
\end{itemize}

\textbf{વેવફોર્મ જનરેશન:}
\begin{itemize}
    \item **ટ્રાયએંગલ વેવ**: ઓસિલેટર સર્કિટનો મૂળભૂત આઉટપુટ.
    \item **સ્ક્વેર વેવ**: કમ્પેરેટર દ્વારા ટ્રાયએંગલ વેવમાંથી ઉત્પન્ન થાય છે.
    \item **સાઇન વેવ**: વેવશેપિંગ દ્વારા ટ્રાયએંગલ વેવમાંથી ઉત્પન્ન થાય છે.
\end{itemize}
\end{solutionbox}

\begin{mnemonicbox}
\mnemonic{FAST: Frequency-Amplitude-Signal-Type control}
\end{mnemonicbox}

\questionmarks{5(અ) OR}{3}{PH સેન્સરનું કાર્ય જરૂરી ડાયાગ્રામ સાથે વિગતવાર સમજાવો.}

\begin{solutionbox}
pH સેન્સર દ્રાવણમાં હાઇડ્રોજન આયન કન્સન્ટ્રેશન માપે છે, જે એસિડિટી અથવા અલ્કલિનિટી દર્શાવે છે.

\textbf{આકૃતિ:}
\begin{center}
\begin{tikzpicture}[node distance=1.5cm, auto]
    \node [gtu block] (Glass) {Glass Electrode};
    \node [gtu block, below=of Glass] (Ref) {Ref Electrode};
    \node [gtu block, right=of Glass] (Meas) {Measurement Ckt};
    \node [gtu block, right=of Meas] (Disp) {Display};
    
    \coordinate (sol) at (-1, -1.5);
    \node at (sol) [left] {Solution};
    
    \draw [gtu arrow] (Glass) -- (Meas);
    \draw [gtu arrow] (Ref) -| (Meas);
    \draw [gtu arrow] (Meas) -- (Disp);
    
    \draw [dashed] (-2, -3) rectangle (0, 1); % Beaker
\end{tikzpicture}
\captionof{figure}{pH માપન સિસ્ટમ}
\end{center}

\textbf{કાર્યપ્રણાલી:}
\begin{itemize}
    \item ગ્લાસ ઇલેક્ટ્રોડમાં જાણીતા pH સાથે બફર સોલ્યુશન હોય છે.
    \item ટેસ્ટ સોલ્યુશનમાં $H^+$ આયન ગ્લાસ મેમ્બ્રેન સાથે ઇન્ટરેક્ટ કરે છે.
    \item pH તફાવતના પ્રમાણમાં પોટેન્શિયલ ડિફરન્સ વિકસે છે.
    \item રેફરન્સ ઇલેક્ટ્રોડ સ્થિર તુલના વોલ્ટેજ પ્રદાન કરે છે.
    \item વોલ્ટેજ તફાવત = $25^\circ C$ પર પ્રતિ pH એકમ 59.16 mV.
\end{itemize}
\end{solutionbox}

\begin{mnemonicbox}
\mnemonic{pH-MVH: Potential-of-Hydrogen Measured by Voltage per Hydrogen-ion concentration}
\end{mnemonicbox}

\questionmarks{5(બ) OR}{4}{Spectrum Analyzerનું કાર્ય જરૂરી ડાયાગ્રામ સાથે વિગતવાર સમજાવો.}

\begin{solutionbox}
સ્પેક્ટ્રમ એનાલાઇઝર સિગ્નલના ફ્રીક્વન્સી ઘટકો બતાવતું સિગ્નલ એમ્પ્લિટ્યુડ વિ. ફ્રીક્વન્સી પ્રદર્શિત કરે છે.

\textbf{બ્લોક ડાયાગ્રામ:}
\begin{center}
\begin{tikzpicture}[node distance=1.5cm, auto]
    \node [gtu block] (Mix) {Mixer};
    \node [gtu block, right=of Mix] (IF) {IF Filter};
    \node [gtu block, right=of IF] (Det) {Detector};
    \node [gtu block, right=of Det] (Disp) {CRT Display};
    
    \node [gtu block, below=of Mix] (LO) {Local Osc};
    \node [gtu block, below=of Det] (Sweep) {Sweep Gen};
    
    \node [coordinate, left=of Mix] (In) {};
    \draw [gtu arrow] (In) -- node[above]{RF In} (Mix);
    
    \draw [gtu arrow] (Mix) -- (IF);
    \draw [gtu arrow] (IF) -- (Det);
    \draw [gtu arrow] (Det) -- (Disp);
    \draw [gtu arrow] (LO) -- (Mix);
    \draw [gtu arrow] (Sweep) -- (LO);
    \draw [gtu arrow] (Sweep) -| (Disp);
\end{tikzpicture}
\captionof{figure}{સ્પેક્ટ્રમ એનાલાઇઝર}
\end{center}

\textbf{કાર્યપ્રણાલી:}
\begin{itemize}
    \item **ઇનપુટ સ્ટેજ**: ઓપ્ટિમમ લેવલ પર સિગ્નલને એટેન્યુએટ અથવા એમ્પ્લિફાય કરે છે.
    \item **મિક્સર**: ઇનપુટને લોકલ ઓસિલેટર સિગ્નલ સાથે જોડે છે.
    \item **IF ફિલ્ટર**: ફક્ત ઇચ્છિત ફ્રીક્વન્સી ઘટકોને પસાર કરે છે.
    \item **ડિટેક્ટર**: IF સિગ્નલની એમ્પ્લિટ્યુડ માપે છે.
    \item **ડિસ્પ્લે**: એમ્પ્લિટ્યુડ વિ. ફ્રીક્વન્સી બતાવે છે.
\end{itemize}
\end{solutionbox}

\begin{mnemonicbox}
\mnemonic{SAFE-D: Signal-Amplitude-Frequency-Evaluation-Display}
\end{mnemonicbox}

\questionmarks{5(ક) OR}{7}{મૂળભૂત ફ્રિકવન્સી કાઉન્ટરનું કાર્ય જરૂરી ડાયાગ્રામ સાથે વિગતવાર સમજાવો.}

\begin{solutionbox}
ફ્રીક્વન્સી કાઉન્ટર ચોક્કસ સમય અંતરાલમાં સાયકલ્સ ગણીને ઇનપુટ સિગ્નલની ફ્રીક્વન્સી માપે છે.

\textbf{બ્લોક ડાયાગ્રામ:}
\begin{center}
\begin{tikzpicture}[node distance=1.5cm, auto]
    \node [gtu block] (ST) {Schmitt Trigger};
    \node [gtu block, right=of ST] (Gate) {Main Gate};
    \node [gtu block, right=of Gate] (Count) {Counter};
    \node [gtu block, right=of Count] (Disp) {Display};
    
    \node [gtu block, below=of Gate] (TB) {Time Base};
    
    \coordinate [left=of ST] (In);
    \draw [gtu arrow] (In) -- node[above]{Input} (ST);
    \draw [gtu arrow] (ST) -- (Gate);
    \draw [gtu arrow] (Gate) -- (Count);
    \draw [gtu arrow] (Count) -- (Disp);
    \draw [gtu arrow] (TB) -- (Gate);
\end{tikzpicture}
\captionof{figure}{ફ્રીક્વન્સી કાઉન્ટર}
\end{center}

\textbf{કાર્યપ્રણાલી:}
\begin{itemize}
    \item \textbf{શ્મિટ ટ્રિગર}: ઇનપુટ સિગ્નલને સ્ક્વેર વેવમાં રૂપાંતરિત કરે છે.
    \item \textbf{ટાઇમ બેઝ}: ક્રિસ્ટલ ઓસિલેટર ચોક્કસ સંદર્ભ પ્રદાન કરે છે.
    \item \textbf{ગેટ કંટ્રોલ}: ચોક્કસ માપન અંતરાલ માટે ગેટ ખોલે છે.
    \item \textbf{કાઉન્ટર}: ગેટ ખુલ્લા સમય દરમિયાન ઇનપુટ સાયકલ્સ ગણે છે.
    \item \textbf{ડિસ્પ્લે}: ગણતરી કરેલી ફ્રીક્વન્સી બતાવે છે.
\end{itemize}

\textbf{માપન પ્રક્રિયા:}
ફ્રીક્વન્સી = ગણતરી / ગેટ સમય
\end{solutionbox}

\begin{mnemonicbox}
\mnemonic{COUNT: Cycles-Over-Unit-time-Numerically-Tallied}
\end{mnemonicbox}

\end{document}

