\documentclass{article}

% content/resources/templates/preamble.tex
\usepackage[margin=0.6in]{geometry}
\author{Milav Dabgar}
\usepackage{amsmath,amssymb,amsthm}
\usepackage{booktabs}
\usepackage{multirow}
\usepackage{xcolor}
\usepackage{tcolorbox}
\tcbuselibrary{breakable,skins}
\usepackage[colorlinks=true,linkcolor=blue]{hyperref}
\usepackage{titlesec}
\usepackage{enumitem}
\usepackage{tikz}
\usepackage{pgfplots}
\usepackage{circuitikz}
\usepackage[version=4]{mhchem}
\usepackage{longtable}
\usepackage{array}
\usepackage{float}
\usepackage{caption}
\usepackage{listings}

\lstset{
  basicstyle=\small\ttfamily,
  breaklines=true,
  breakatwhitespace=false,
  postbreak=\mbox{\textcolor{red}{$\hookrightarrow$}\space},
  float=false,
  numbers=left,
  numberstyle=\tiny\color{gray},
  numbersep=10pt,
  xleftmargin=2em,
  keywordstyle=\color{blue},
  commentstyle=\color{green!60!black},
  stringstyle=\color{purple},
  backgroundcolor=\color{gray!5},
  showstringspaces=false,
  tabsize=2,
  captionpos=b,
  keepspaces=true,
  columns=flexible
}

\pgfplotsset{compat=1.18}
\usetikzlibrary{shapes,arrows,positioning,calc,patterns,decorations.pathmorphing,decorations.markings,arrows.meta}

% Color scheme
\definecolor{headcolor}{RGB}{0,102,204}
\definecolor{keycolor}{RGB}{220,20,60}
\definecolor{solutioncolor}{RGB}{34,139,34}
\definecolor{mnemoniccolor}{RGB}{148,0,211}
\definecolor{codecolor}{RGB}{0,0,100}

% Spacing
\setlength{\parskip}{3pt}
\setlist[itemize]{nosep}
\setlist[enumerate]{nosep}

% Title formatting
\titleformat{\section}{\Large\bfseries\color{headcolor}}{\thesection}{1em}{}
\titleformat{\subsection}{\large\bfseries\color{headcolor}}{\thesubsection}{1em}{}

% Pandoc tightlist compatibility
\providecommand{\tightlist}{%
  \setlength{\itemsep}{0pt}\setlength{\parskip}{0pt}}

% Pandoc longtable compatibility
\newcounter{none}
\def\thenone{}


% content/resources/templates/gujarati-boxes.tex
\usepackage{fontspec}
\usepackage{polyglossia}

% Set Gujarati as main language (document is primarily in Gujarati)
% Note: gloss-gujarati.ldf doesn't exist in polyglossia, but it will use hyphenation patterns
\setdefaultlanguage{gujarati}
\setotherlanguage{english}

% Configure Gujarati font properly
% Use Language=Default to prevent polyglossia from trying to add language-specific features
% that don't exist for Gujarati, which causes "empty feature" warnings
\newfontfamily\gujaratifont[Script=Gujarati,AutoFakeBold=2.5,AutoFakeSlant=0.3]{Noto Sans Gujarati}
\setmainfont[Script=Gujarati,AutoFakeBold=2.5,AutoFakeSlant=0.3]{Noto Sans Gujarati}
% Use Noto Sans Gujarati for monospace to support Gujarati in text
\setmonofont[Scale=0.9]{Noto Sans Gujarati}

% Configure English to use the same font
\newfontfamily\englishfont[Script=Gujarati,AutoFakeBold=2.5,AutoFakeSlant=0.3]{Noto Sans Gujarati}

% Translations for polyglossia
\gappto\captionsgujarati{
  \renewcommand{\tablename}{કોષ્ટક}
  \renewcommand{\figurename}{આકૃતિ}
}

% Helper for TikZ nodes to ensure Gujarati font
\newcommand{\gu}[1]{{\gujaratifont #1}}

% Custom environments
\newtcolorbox{solutionbox}{
    breakable,
    enhanced,
    colback=solutioncolor!5!white,
    colframe=solutioncolor!75!black,
    fonttitle=\bfseries,
    title=જવાબ
}

\newtcolorbox{solutionboxnobreak}{
 colback=solutioncolor!5!white,
 colframe=solutioncolor!75!black,
 fonttitle=\bfseries,
 title=જવાબ
}

\newtcolorbox{keyformula}{
 breakable,
 enhanced,
 colback=keycolor!5!white,
 colframe=keycolor!75!black,
 fonttitle=\bfseries,
 title=રાસાયણિક સમીકરણ/સૂત્ર
}

\newtcolorbox{mnemonicbox}{
 breakable,
 enhanced,
 colback=mnemoniccolor!5!white,
 colframe=mnemoniccolor!75!black,
 fonttitle=\bfseries,
 title=મેમરી ટ્રીક
}


% Custom commands for GTU solutions
% This file defines semantic commands for consistent formatting

% Question command with automatic formatting
\newcommand{\question}[2]{%
  \section*{Question #1}%
  \textbf{#2}%
}

% OR question variant
\newcommand{\questionor}[2]{%
  \section*{Question #1 OR}%
  \textbf{#2}%
}

% Proper table environment with caption
\newenvironment{answertable}[1]{%
  \begin{table}[htbp]
  \centering
  \caption{#1}
}{%
  \end{table}
}

% Proper figure environment for diagrams
\newenvironment{answerdiagram}[1]{%
  \begin{figure}[htbp]
  \centering
  \caption{#1}
}{%
  \end{figure}
}

% Semantic markup for key terms
\newcommand{\keyword}[1]{\textbf{#1}}
\newcommand{\code}[1]{\texttt{#1}}
\newcommand{\classname}[1]{\texttt{#1}}
\newcommand{\methodname}[1]{\texttt{#1}}

% Proper quotation marks
\newcommand{\mnemonic}[1]{``#1''}


\title{ઇલેક્ટ્રોનિક મેઝરમેન્ટ્સ એન્ડ ઈન્સ્ટ્રુમેન્ટ્સ (4331102) - સમર 2025 સોલ્યુશન}
\date{May 13, 2025}

\begin{document}
\maketitle

% Q1 Start
\questionmarks{1(a)}{3}{Accuracy, Precision, અને Sensitivity ની વ્યાખ્યા આપો.}

\begin{solutionbox}
\begin{itemize}
    \item \keyword{Accuracy}: માપેલા મૂલ્યની વાસ્તવિક મૂલ્યની નજીકતા.
    \item \keyword{Precision}: એક જ ઈનપુટ વારંવાર આપવામાં આવે ત્યારે સાધનની એક સરખા આઉટપુટ રીડિંગ ફરીથી ઉત્પન્ન કરવાની ક્ષમતા.
    \item \keyword{Sensitivity}: સાધનના આઉટપુટમાં થતા ફેરફારનો ઈનપુટમાં થતા ફેરફાર સાથેનો ગુણોત્તર, જે દર્શાવે છે કે નાના ફેરફાર માટે આઉટપુટમાં કેટલો ફેરફાર થાય છે.
\end{itemize}

\begin{center}
\captionof{table}{Accuracy અને Precision વચ્ચેના તફાવત}
\begin{tabulary}{\linewidth}{|L|L|L|}
\hline
\textbf{પેરામીટર} & \textbf{Accuracy} & \textbf{Precision} \\ \hline
વ્યાખ્યા & સાચા મૂલ્યની નજીકતા & માપની પુનરાવર્તિતા \\ \hline
ફોકસ & સચોટતા & સુસંગતતા \\ \hline
પ્રતિનિધિત્વ & બુલ્સ-આઇના સેન્ટરના હિટ્સ & ક્લસ્ટર્ડ હિટ્સ \\ \hline
\end{tabulary}
\end{center}
\end{solutionbox}

\begin{mnemonicbox}
\mnemonic{APS - Accuracy સત્યતા દર્શાવે છે, Precision પુનરાવર્તિતા બતાવે છે, Sensitivity નાના ફેરફારો સંકેત આપે છે}
\end{mnemonicbox}

\questionmarks{1(b)}{4}{વ્હીટસ્ટોન બ્રિજના કાર્ય અને મર્યાદાઓ તેના સર્કિટ ડાયાગ્રામ દોરી સમજાવો.}

\begin{solutionbox}
\textbf{કાર્ય}: વ્હીટસ્ટોન બ્રિજ બ્રિજ સર્કિટની બે ભુજાઓને સંતુલિત કરીને અજ્ઞાત અવરોધ માપે છે.

\textbf{સર્કિટ ડાયાગ્રામ}:
\begin{center}
\begin{circuitikz}[american, scale=0.8]
    \draw (0,2) node[left] {B} to[R, l=$R_1$] (2,4) node[above] {A} to[R, l=$R_2$] (4,2) node[right] {D};
    \draw (4,2) to[R, l=$R_x$] (2,0) node[below] {C} to[R, l=$R_3$] (0,2);
    \draw (0,2) to[short] (-1,2) to[battery1, l=$E$] (-1,0) to[short] (2,0); % Supply
    \draw (2,4) to[rmeter, t=G] (2,0); % Galvanometer
\end{circuitikz}
\captionof{figure}{વ્હીટસ્ટોન બ્રિજ}
\end{center}

જ્યારે બ્રિજ સંતુલિત હોય છે: $\frac{R_1}{R_2} = \frac{R_3}{R_x}$, તેથી $R_x = R_3 \times (\frac{R_2}{R_1})$

\textbf{મર્યાદાઓ}:
\begin{itemize}
    \item \keyword{મર્યાદિત રેન્જ}: ખૂબ ઓછા કે ખૂબ વધારે અવરોધ માટે યોગ્ય નથી.
    \item \keyword{તાપમાન અસરો}: તાપમાન સાથે અવરોધ બદલાય છે.
    \item \keyword{બેટરી ભૂલો}: આઉટપુટ વોલ્ટેજ સ્થિર રહેવું જોઈએ.
    \item \keyword{ગેલ્વેનોમીટર સંવેદનશીલતા}: ડિટેક્ટરની સંવેદનશીલતાથી મર્યાદિત.
\end{itemize}
\end{solutionbox}

\begin{mnemonicbox}
\mnemonic{BALR - Balance મહત્વનું છે, Adjust શૂન્ય સુધી, Low/high અવરોધો સમસ્યારૂપ, Range મર્યાદિત છે}
\end{mnemonicbox}

\questionmarks{1(c)}{7}{તાપમાન માપવા માટે ઉપયોગમાં લેવાતા વિવિધ પ્રકારના ટ્રાન્સડ્યુસર સમજાવો. નીચેના માટે બાંધકામ અને કાર્ય વિગતવાર સમજાવો: (i) થર્મોકપલ (ii) થર્મિસ્ટર.}

\begin{solutionbox}
\textbf{તાપમાન ટ્રાન્સડ્યુસર પ્રકારો}:
\begin{center}
\captionof{table}{તાપમાન ટ્રાન્સડ્યુસર તુલના}
\begin{tabulary}{\linewidth}{|L|L|L|L|L|}
\hline
\textbf{પ્રકાર} & \textbf{કાર્ય સિદ્ધાંત} & \textbf{રેન્જ} & \textbf{ફાયદા} & \textbf{ગેરફાયદા} \\ \hline
થર્મોકપલ & સીબેક ઇફેક્ટ & $-270^\circ$C થી $2300^\circ$C & વિશાળ રેન્જ, મજબૂત & નોન-લિનિયર, સંદર્ભની જરૂર \\ \hline
થર્મિસ્ટર & અવરોધ પરિવર્તન & $-50^\circ$C થી $300^\circ$C & ઉચ્ચ સંવેદનશીલતા & નોન-લિનિયર, મર્યાદિત રેન્જ \\ \hline
RTD & અવરોધ પરિવર્તન & $-200^\circ$C થી $850^\circ$C & ઉચ્ચ ચોકસાઈ, લિનિયર & મોંઘું, સેલ્ફ-હીટિંગ \\ \hline
IC સેન્સર & સેમિકન્ડક્ટર & $-55^\circ$C થી $150^\circ$C & લિનિયર આઉટપુટ, સરળ & મર્યાદિત રેન્જ \\ \hline
\end{tabulary}
\end{center}

\textbf{(i) થર્મોકપલ}:
\textbf{બાંધકામ}: બે અલગ-અલગ ધાતુના તાર (જેમ કે કોપર-કોન્સ્ટંટન અથવા આયર્ન-કોન્સ્ટંટન) એક છેડે જોડાયેલા હોય છે જે માપન જંક્શન બનાવે છે અને બીજા છેડે માપન ઉપકરણ સાથે જોડાયેલા હોય છે.

\textbf{આકૃતિ}:
\begin{center}
\begin{circuitikz}[american, scale=0.8]
    \draw (0,0) coordinate(ref) node[left] {સંદર્ભ જંક્શન ($T_c$)} -- (2,0) coordinate(join1) -- (2,2) coordinate(hot) node[right] {માપન જંક્શન ($T_h$)};
    \draw (ref) -- (0,2) -- (join1); 
    
    \draw (0,0) to[short, -*] (3,0); % Wire A
    \draw (0,2) to[short, -*] (3,2); % Wire A
    \draw (3,0) -- (4,1) -- (3,2); % Wire B (Junction at 4,1)
    
    \node at (4.2,1) {Hot ($T_h$)};
    \node at (-0.2,1) {Cold ($T_c$)};
    
    \draw (0,2) to[rmeter, t=mV] (0,0);
    
    \node at (1.5, 2.2) {Metal A};
    \node at (3.5, 1.5) {Metal B};
\end{circuitikz}
\captionof{figure}{થર્મોકપલ}
\end{center}

\textbf{કાર્ય}: જ્યારે જંક્શનો અલગ-અલગ તાપમાને હોય છે, ત્યારે તાપમાન તફાવતના પ્રમાણમાં નાનું વોલ્ટેજ ઉત્પન્ન થાય છે (\keyword{સીબેક ઇફેક્ટ}).

\textbf{મુખ્ય બિંદુઓ}:
\begin{itemize}
    \item \keyword{સીબેક ઇફેક્ટ}: તાપમાન તફાવત વોલ્ટેજ ઉત્પન્ન કરે છે.
    \item \keyword{કોલ્ડ જંક્શન કોમ્પેન્સેશન}: ચોકસાઈ માટે જરૂરી.
    \item \keyword{પ્રકારો}: J, K, T, E ધાતુના સંયોજનના આધારે.
\end{itemize}

\textbf{(ii) થર્મિસ્ટર}:
\textbf{બાંધકામ}: અર્ધવાહક સામગ્રી (મેંગેનીઝ, નિકલ, કોબાલ્ટ જેવા ધાતુ ઓક્સાઇડ્સ) બીડ, ડિસ્ક અથવા રોડના આકારમાં બે લીડ વાયર સાથે બનાવવામાં આવે છે.

\textbf{આકૃતિ}:
\begin{center}
\begin{circuitikz}[american]
    \draw (0,0) to[thermistor, l=Thermistor] (2,0);
\end{circuitikz}
\captionof{figure}{થર્મિસ્ટર સિમ્બોલ}
\end{center}

\textbf{કાર્ય}: તાપમાન વધવાની સાથે અવરોધ ઘટે છે (NTC પ્રકાર) અથવા તાપમાન સાથે વધે છે (PTC પ્રકાર).

\textbf{મુખ્ય બિંદુઓ}:
\begin{itemize}
    \item \keyword{NTC (નેગેટિવ ટેમ્પરેચર કોઇફિશિયન્ટ)}: સૌથી સામાન્ય પ્રકાર.
    \item \keyword{ઉચ્ચ સંવેદનશીલતા}: નાના તાપમાન ફેરફાર માટે મોટો અવરોધ ફેરફાર.
    \item \keyword{નોન-લિનિયર રિસ્પોન્સ}: લિનિયરાઇઝેશન સર્કિટની જરૂર પડે છે.
    \item \keyword{સેલ્ફ-હીટિંગ}: તેમાંથી પસાર થતો પ્રવાહ ગરમી ઉત્પન્ન કરે છે.
\end{itemize}
\end{solutionbox}

\begin{mnemonicbox}
\mnemonic{TRIP - થર્મોકપલ જંક્શન તફાવતોને પ્રતિક્રિયા આપે છે, થર્મિસ્ટર અવરોધમાં તીવ્ર ફેરફાર કરે છે, સેન્સર જે માપવું છે તેના પર લક્ષ્ય કરો}
\end{mnemonicbox}

\questionmarks{1(c) OR}{7}{નીચેના sensor ના કાર્યસિદ્ધાંત સમજાવો: Temperature sensor, Gas sensor, Humidity sensor અને Proximity sensor.}

\begin{solutionbox}
\textbf{સેન્સરની તુલના}:
\begin{center}
\captionof{table}{સેન્સર તુલના}
\begin{tabulary}{\linewidth}{|L|L|L|L|}
\hline
\textbf{સેન્સરનો પ્રકાર} & \textbf{કાર્ય સિદ્ધાંત} & \textbf{આઉટપુટ} & \textbf{ઉપયોગો} \\ \hline
તાપમાન & અવરોધ/વોલ્ટેજ પરિવર્તન & એનાલોગ/ડિજિટલ & HVAC, મેડિકલ ડિવાઇસ \\ \hline
ગેસ & રાસાયણિક પ્રતિક્રિયા & અવરોધમાં ફેરફાર & સલામતી સિસ્ટમ, હવા ગુણવત્તા \\ \hline
ભેજ & કેપેસિટન્સ/અવરોધ ફેરફાર & એનાલોગ & વેધર સ્ટેશન, HVAC \\ \hline
પ્રોક્સિમિટી & ઇલેક્ટ્રોમેગ્નેટિક ફિલ્ડ ડિસરપ્શન & ડિજિટલ & ઓટોમેશન, સુરક્ષા \\ \hline
\end{tabulary}
\end{center}

\textbf{1. તાપમાન સેન્સર (LM35)}:
\begin{itemize}
    \item \keyword{સિદ્ધાંત}: સેમિકન્ડક્ટર જંક્શન વોલ્ટેજ તાપમાન સાથે બદલાય છે.
    \item \keyword{કાર્ય}: ઇન્ટિગ્રેટેડ સર્કિટ તાપમાનના પ્રમાણમાં આઉટપુટ વોલ્ટેજ આપે છે ($10mV/^\circ C$).
    \item \keyword{લક્ષણો}: લિનિયર આઉટપુટ, બાહ્ય કેલિબ્રેશનની જરૂર નથી.
\end{itemize}

\textbf{2. ગેસ સેન્સર (MQ-2)}:
\begin{itemize}
    \item \keyword{સિદ્ધાંત}: ગેસ અને સેન્સિંગ મટિરિયલ વચ્ચે રાસાયણિક પ્રતિક્રિયા.
    \item \keyword{કાર્ય}: ગેસ અણુઓ અર્ધવાહક ધાતુ ઓક્સાઇડ સાથે ક્રિયા કરે છે, જેનાથી તેનો અવરોધ બદલાય છે.
    \item \keyword{ડિટેક્શન}: જ્યારે ગેસનું સાંદ્રતા થ્રેશોલ્ડથી વધે છે, તો આઉટપુટ વોલ્ટેજ બદલાય છે.
\end{itemize}
\begin{center}
\begin{tikzpicture}[node distance=1.5cm, auto]
    \node [gtu block] (Gas) {ગેસ અણુઓ};
    \node [gtu block, right=of Gas] (Sense) {સેન્સિંગ લેયર};
    \node [gtu block, right=of Sense] (Res) {અવરોધ ફેરફાર};
    \node [gtu block, right=of Res] (Volt) {વોલ્ટેજ આઉટપુટ};
    \node [gtu block, below=of Volt] (Comp) {કોમ્પેરેટર};
    \node [gtu block, left=of Comp] (Alarm) {એલાર્મ};
    
    \draw [gtu arrow] (Gas) -- (Sense);
    \draw [gtu arrow] (Sense) -- (Res);
    \draw [gtu arrow] (Res) -- (Volt);
    \draw [gtu arrow] (Volt) -- (Comp);
    \draw [gtu arrow] (Comp) -- (Alarm);
\end{tikzpicture}
\captionof{figure}{ગેસ સેન્સર કાર્ય}
\end{center}

\textbf{3. ભેજ સેન્સર (હાઇગ્રોમીટર)}:
\begin{itemize}
    \item \keyword{સિદ્ધાંત}: ભેજ શોષણ સાથે કેપેસિટન્સ અથવા અવરોધમાં ફેરફાર.
    \item \keyword{કાર્ય}: ડાયલેકટ્રિક મટિરિયલ ભેજ શોષે છે, જેથી ઇલેક્ટ્રિકલ ગુણધર્મો બદલાય છે.
    \item \keyword{પ્રકારો}: કેપેસિટિવ (વધુ ચોક્કસ) અને રેઝિસ્ટિવ (સરળ).
\end{itemize}

\textbf{4. પ્રોક્સિમિટી સેન્સર}:
\begin{itemize}
    \item \keyword{સિદ્ધાંત}: ભૌતિક સંપર્ક વિના વસ્તુઓનું શોધન.
    \item \keyword{કાર્ય}: ઇલેક્ટ્રોમેગ્નેટિક ફિલ્ડ/બીમ ઉત્સર્જિત કરે છે; જ્યારે વસ્તુ ફિલ્ડમાં પ્રવેશે ત્યારે ફેરફારોનું શોધન.
    \item \keyword{પ્રકારો}: ઇન્ડક્ટિવ (ધાતુઓ), કેપેસિટિવ (કોઈપણ સામગ્રી), અલ્ટ્રાસોનિક (અંતર).
\end{itemize}
\end{solutionbox}

\begin{mnemonicbox}
\mnemonic{TGHP - તાપમાન વોલ્ટેજ પેદા કરે છે, ગેસ અર્ધવાહકો પર અસર કરે છે, ભેજ જાળવે છે, પ્રોક્સિમિટી વસ્તુઓને શોધે છે}
\end{mnemonicbox}

% Q2 Start
\questionmarks{2(a)}{3}{ડીવીએમ(DVM) ના પ્રકારો આપો અને દરેકના ફાયદા જણાવો.}

\begin{solutionbox}
\textbf{ડિજિટલ વોલ્ટમીટર (DVM) પ્રકારો}:
\begin{center}
\captionof{table}{DVM પ્રકારો}
\begin{tabulary}{\linewidth}{|L|L|L|}
\hline
\textbf{DVM પ્રકાર} & \textbf{કાર્ય સિદ્ધાંત} & \textbf{ફાયદા} \\ \hline
રેમ્પ ટાઇપ & ઇનપુટને રેફરન્સ રેમ્પ સાથે સરખાવે છે & સરળ ડિઝાઇન, ઓછી કિંમત \\ \hline
ઇન્ટિગ્રેટિંગ ટાઇપ & સમય દરમિયાન સરેરાશ માપે છે & સારો નોઈઝ રિજેક્શન \\ \hline
સક્સેસિવ એપ્રોક્સિમેશન & બાઇનરી સર્ચ એલ્ગોરિધમ & ઝડપી રૂપાંતરણ \\ \hline
ડ્યુઅલ સ્લોપ & ફિક્સ્ડ સમય સાથે ઇન્ટિગ્રેશન & ઉત્કૃષ્ટ નોઈઝ રિજેક્શન \\ \hline
\end{tabulary}
\end{center}

\textbf{મુખ્ય બિંદુઓ}:
\begin{itemize}
    \item \keyword{રેમ્પ ટાઇપ}: સરળ પરંતુ નોઈઝથી પ્રભાવિત.
    \item \keyword{ઇન્ટિગ્રેટિંગ ટાઇપ}: સામયિક નોઈઝની અસર ઘટાડે છે.
    \item \keyword{સક્સેસિવ એપ્રોક્સિમેશન}: ઝડપી વાંચન, બદલાતા સિગ્નલ માટે સારું.
    \item \keyword{ડ્યુઅલ સ્લોપ}: શ્રેષ્ઠ ચોકસાઈ, મોટાભાગના નોઈઝથી અસર રહિત.
\end{itemize}
\end{solutionbox}

\begin{mnemonicbox}
\mnemonic{RISD - રેમ્પ સરળ ડિઝાઇન છે, ઇન્ટિગ્રેટિંગ નોઈઝને અવગણે છે, સક્સેસિવ ઝડપ સુનિશ્ચિત કરે છે, ડ્યુઅલ હસ્તક્ષેપ સાથે વ્યવહાર કરે છે}
\end{mnemonicbox}

\questionmarks{2(b)}{4}{મેક્સવેલ બ્રીજ દોરો અને સમજાવો.}

\begin{solutionbox}
\textbf{મેક્સવેલ બ્રીજ} સ્ટાન્ડર્ડ કેપેસિટન્સ સાથે સરખામણી કરીને અજ્ઞાત ઇન્ડક્ટન્સને માપે છે.

\textbf{સર્કિટ ડાયાગ્રામ}:
\begin{center}
\begin{circuitikz}[american, scale=0.8]
    \draw (0,4) node[left] {a} to[R, l=$R_3$] (3,6) node[above] {b} to[R, l=$R_2$] (6,4) node[right] {c};
    \draw (6,4) to[L, l=$L_x$] (4.5, 2.8) to[R, l=$R_x$] (3,2) node[below] {d} to[C, l=$C_1$] (0,4);
    \draw (0,4) to[short] (0.5, 3.5) to[R, l=$R_1$] (2.5, 2.5) to[short] (3,2); % R1 || C1
    
    \draw (3,6) to[rmeter, t=D] (3,2); % Detector
    \draw (0,4) to[short] (-1,4) to[sV, l=$V_{AC}$] (-1,0) to[short] (7,0) to[short] (6,4);
\end{circuitikz}
\captionof{figure}{મેક્સવેલ બ્રિજ}
\end{center}

\textbf{બેલેન્સ ઇક્વેશન્સ}:
\begin{itemize}
    \item અજ્ઞાત ઇન્ડક્ટન્સ $L_x = R_2 R_3 C_1$
    \item અવરોધ $R_x = \frac{R_2 R_3}{R_1}$
\end{itemize}

\textbf{કાર્ય}:
\begin{itemize}
    \item બ્રિજમાં ચાર ભુજાઓ હોય છે.
    \item જ્યારે બ્રિજ સંતુલિત હોય છે, ત્યારે ડિટેક્ટરમાંથી પ્રવાહ વહેતો નથી.
    \item $L$ અને $R$ ના મૂલ્ય બેલેન્સ ઇક્વેશન્સ વડે ગણવામાં આવે છે.
\end{itemize}

\textbf{ફાયદાઓ}:
\begin{itemize}
    \item \keyword{ઉચ્ચ ચોકસાઈ}: મધ્યમ મૂલ્યના ઇન્ડક્ટર્સ માટે સારું (Q 1 થી 10 વચ્ચે).
    \item \keyword{સ્વતંત્ર બેલેન્સ}: અવરોધ અને ઇન્ડક્ટન્સ અલગથી સંતુલિત થાય છે.
\end{itemize}
\end{solutionbox}

\begin{mnemonicbox}
\mnemonic{MILL - મેક્સવેલ્સ ઇન્ડક્ટન્સ L = R2R3C જેવું છે, જ્યારે ડિટેક્ટર ઓછો પ્રવાહ બતાવે છે}
\end{mnemonicbox}

\questionmarks{2(c)}{7}{સક્સેસિવ એપ્રોક્સિમેશન પ્રકારના ડિજિટલ વોલ્ટમીટર (DVM)નો બ્લોક ડાયાગ્રામ દોરીને તેનું કાર્ય સમજાવો.}

\begin{solutionbox}
\textbf{સક્સેસિવ એપ્રોક્સિમેશન DVM} બાઇનરી સર્ચ એલ્ગોરિધમનો ઉપયોગ કરીને એનાલોગ ઇનપુટને ડિજિટલ આઉટપુટમાં રૂપાંતરિત કરે છે.

\textbf{બ્લોક ડાયાગ્રામ}:
\begin{center}
\begin{tikzpicture}[node distance=1.5cm, auto]
    \node [gtu block] (SAR) {SAR રજિસ્ટર};
    \node [gtu block, below=of SAR] (DAC) {D/A કન્વર્ટર};
    \node [gtu block, left=of DAC] (Comp) {કોમ્પેરેટર};
    \node [coordinate, left=of Comp] (In) {};
    \node [gtu block, right=of SAR] (Disp) {ડિજિટલ ડિસ્પ્લે};
    \node [gtu block, above=of SAR] (Clock) {ક્લોક};
    \node [gtu block, right=of DAC] (Ref) {રેફરન્સ વોલ્ટેજ};
    
    \draw [gtu arrow] (In) -- node[above]{Input} (Comp);
    \draw [gtu arrow] (Comp) |- (SAR);
    \draw [gtu arrow] (SAR) -- (DAC);
    \draw [gtu arrow] (DAC) -- (Comp);
    \draw [gtu arrow] (Clock) -- (SAR);
    \draw [gtu arrow] (SAR) -- (Disp);
    \draw [gtu arrow] (Ref) -- (DAC);
\end{tikzpicture}
\captionof{figure}{સક્સેસિવ એપ્રોક્સિમેશન DVM}
\end{center}

\textbf{કાર્ય}:
\begin{enumerate}
    \item \keyword{સિગ્નલ કન્ડિશનિંગ}: ઇનપુટ વોલ્ટેજને માપન રેન્જમાં સ્કેલ કરે છે.
    \item \keyword{સેમ્પલ \& હોલ્ડ}: ક્ષણિક ઇનપુટ મૂલ્યને પકડે છે.
    \item \keyword{SAR (સક્સેસિવ એપ્રોક્સિમેશન રજિસ્ટર)}: બાઇનરી સર્ચ કરે છે.
    \item \keyword{DAC}: ડિજિટલ મૂલ્યને એનાલોગમાં રૂપાંતરિત કરે છે.
    \item \keyword{કોમ્પેરેટર}: ઇનપુટને DAC આઉટપુટ સાથે સરખાવે છે.
    \item \keyword{ડિજિટલ ડિસ્પ્લે}: અંતિમ ડિજિટલ મૂલ્ય બતાવે છે.
\end{enumerate}

\textbf{ઉદાહરણ}: 9V ના 4-બિટ રૂપાંતરણ માટે (0-15V રેન્જ): 
8V (1000) < 9V (1 રાખો) $\rightarrow$ 12V (1100) > 9V (0 કરો) $\rightarrow$ 10V (1010) > 9V (0 કરો) $\rightarrow$ 9V (1001) = 9V (1 રાખો). પરિણામ: 1001.

\textbf{ફાયદાઓ}:
\begin{itemize}
    \item \keyword{ઝડપી રૂપાંતરણ}: ફિક્સ્ડ રૂપાંતરણ સમય.
    \item \keyword{સારી ચોકસાઈ}: મોટાભાગના ઉપયોગો માટે યોગ્ય.
\end{itemize}
\end{solutionbox}

\begin{mnemonicbox}
\mnemonic{SHARP - સેમ્પલ, હોલ્ડ, એપ્રોક્સિમેટ, રજિસ્ટર સંગ્રહ કરે છે, પરિણામ રજૂ કરે છે}
\end{mnemonicbox}

\questionmarks{2(a) OR}{3}{PMMC સાધનનો કાર્ય સિદ્ધાંત જણાવો અને તેના વિષે સમજાવો.}

\begin{solutionbox}
\textbf{PMMC (પર્મેનન્ટ મેગ્નેટ મૂવિંગ કોઇલ)} સાધનો ઇલેક્ટ્રોમેગ્નેટિક સિદ્ધાંતો પર આધારિત કાર્ય કરે છે.

\textbf{કાર્ય સિદ્ધાંત}: જ્યારે ચુંબકીય ક્ષેત્રમાં મૂકેલા કોઇલમાંથી પ્રવાહ વહે છે, ત્યારે એક ટોર્ક ઉત્પન્ન થાય છે જે પ્રવાહના પ્રમાણમાં કોઇલને ફેરવે છે ($T \propto I$).

\textbf{આકૃતિ}:
\begin{center}
\begin{tikzpicture}
    % Magnets
    \draw [fill=gray!30] (-2, -1) rectangle (-1, 1); \node at (-1.5, 0) {N};
    \draw [fill=gray!30] (1, -1) rectangle (2, 1); \node at (1.5, 0) {S};
    % Core
    \draw (0,0) circle (0.6); \node at (0,0) {કોર};
    % Coil
    \draw [thick, color=red] (-0.7, -0.7) rectangle (0.7, 0.7);
    % Pointer
    \draw [thick, ->] (0, 0.7) -- (1, 1.5);
    % Scale
    \draw (0.5, 1.5) arc (60:120:1);
    \foreach \x in {60, 75, 90, 105, 120} \draw (\x:1.6) -- (\x:1.7);
\end{tikzpicture}
\captionof{figure}{PMMC રચના}
\end{center}

\textbf{મુખ્ય ઘટકો}:
\begin{itemize}
    \item \keyword{કાયમી ચુંબક}: મજબૂત ચુંબકીય ક્ષેત્ર બનાવે છે.
    \item \keyword{મૂવિંગ કોઇલ}: પ્રવાહ વહન કરે છે, ટોર્ક ઉત્પન્ન કરે છે.
    \item \keyword{કંટ્રોલ સ્પ્રિંગ્સ}: પુનઃસ્થાપિત ટોર્ક પ્રદાન કરે છે.
    \item \keyword{પોઇન્ટર}: સ્કેલ પર વાંચન દર્શાવે છે.
\end{itemize}
\end{solutionbox}

\begin{mnemonicbox}
\mnemonic{PMMC - કાયમી ચુંબક પ્રવાહ પસાર થાય ત્યારે કોઇલ ફેરવે છે}
\end{mnemonicbox}

\questionmarks{2(b) OR}{4}{Schering બ્રીજ દોરો અને સમજાવો.}

\begin{solutionbox}
\textbf{Schering બ્રીજ} કેપેસિટરના કેપેસિટન્સ અને ડિસિપેશન ફેક્ટર માપવા માટે વપરાય છે.

\textbf{સર્કિટ ડાયાગ્રામ}:
\begin{center}
\begin{circuitikz}[american, scale=0.8]
    \draw (0,4) node[left] {a} to[C, l=$C_1$] (3,6) node[above] {b} to[C, l=$C_2$] (6,4) node[right] {c};
    \draw (6,4) to[R, l=$R_4$] (4.5, 2.8) to[C, l=$C_4$] (3,2) node[below] {d} to[R, l=$R_3$] (0,4);
    % Again, verifying structure. C1=Std, C2=Std?? English had:
    % Arm 1 (top-R): C2. Arm 2 (top-L): C1. Arm 3 (Bot-L): R3. Arm 4 (Bot-R): C4||R4.
    % Wait, looking at English tex above: 
    % a(L) -> C1 -> b(T) -> C2 -> c(R).
    % c(R) -> R4 -> d(B) -> R3 -> a(L). AND d-C4-? In English code I used:
    % (6,4) to[R, l=$R_4$] (4.5, 2.8) to[C, l=$C_4$] (3,2)
    % This is series R4-C4. Schering usually has Parallel R4-C4 in Arm 4.
    % English code: \draw (0,3) to[short] (0.5, 2.5) to[R, l=$R_4$] (2.5, 1.5) to[short] (3,1); % R4 || C4
    % In English tex I drew R4 || C4.
    
    % Let's use the same diagram code as English for consistency.
    \draw (0,3) node[left] {A} to[R, l=$R_1$] (3,5) node[above] {B} to[C, l=$C_2$] (6,3) node[right] {C};
    \draw (6,3) to[C, l=$C_x$] (4.5, 1.5) to[R, l=$R_x$] (3,1) node[below] {D} to[C, l=$C_4$] (0,3);
    \draw (0,3) to[short] (0.5, 2.5) to[R, l=$R_4$] (2.5, 1.5) to[short] (3,1); % R4 || C4
    
    \draw (3,5) to[rmeter, t=D] (3,1);
    \draw (0,3) to[short] (-1,3) to[sV, l=AC] (-1,0) to[short] (7,0) to[short] (6,3);
\end{circuitikz}
\captionof{figure}{Schering બ્રિજ}
\end{center}

\textbf{બેલેન્સ ઇક્વેશન્સ}:
\begin{itemize}
    \item અજ્ઞાત કેપેસિટન્સ $C_x = C_2(\frac{R_1}{R_4})$
    \item અજ્ઞાત અવરોધ $R_x = R_4(\frac{C_4}{C_2})$
    \item ડિસિપેશન ફેક્ટર $D = \omega C_x R_x = \omega C_4 R_4$
\end{itemize}

\textbf{ઉપયોગો}: કેપેસિટર પરીક્ષણ, ઇન્સુલેશન પરીક્ષણ.
\end{solutionbox}

\begin{mnemonicbox}
\mnemonic{SCAN - Schering કેપેસિટન્સ અને ટેન ડેલ્ટા એક સાથે માપે છે}
\end{mnemonicbox}

\questionmarks{2(c) OR}{7}{ડ્યુઅલ સ્લોપ ઇન્ટિગ્રેટિંગ પ્રકારના ડિજિટલ વોલ્ટમીટર (DVM) ની આકૃતિ દોરો અને સમજાવો.}

\begin{solutionbox}
\textbf{ડ્યુઅલ સ્લોપ ઇન્ટિગ્રેટિંગ DVM} ઇન્ટિગ્રેશન પદ્ધતિનો ઉપયોગ કરે છે.

\textbf{બ્લોક ડાયાગ્રામ}:
\begin{center}
\begin{tikzpicture}[node distance=1.5cm, auto]
    \node [gtu block] (Buff) {બફર્ડ ઇનપુટ};
    \node [gtu block, right=of Buff] (Int) {ઇન્ટિગ્રેટર};
    \node [gtu block, right=of Int] (Comp) {કોમ્પેરેટર};
    \node [gtu block, right=of Comp] (Logic) {કંટ્રોલ લોજિક};
    \node [gtu block, below=of Logic] (Count) {કાઉન્ટર};
    \node [gtu block, right=of Logic] (Disp) {ડિસ્પ્લે};
    \node [gtu block, below=of Int] (Switch) {સ્વિચ};
    \node [gtu block, left=of Switch] (Ref) {રેફરન્સ};
    
    \draw [gtu arrow] (Buff) -- (Int);
    \draw [gtu arrow] (Int) -- (Comp);
    \draw [gtu arrow] (Comp) -- (Logic);
    \draw [gtu arrow] (Logic) -- (Count);
    \draw [gtu arrow] (Count) -- (Disp);
    \draw [gtu arrow] (Logic) -- (Switch);
    \draw [gtu arrow] (Ref) -- (Switch);
    \draw [gtu arrow] (Switch) -- (Int);
\end{tikzpicture}
\captionof{figure}{ડ્યુઅલ સ્લોપ DVM}
\end{center}

\textbf{કાર્ય સિદ્ધાંત}:
\begin{itemize}
    \item \keyword{પ્રથમ તબક્કો (T1)}: ફિક્સ્ડ સમય માટે ઇનપુટ વોલ્ટેજ ઇન્ટિગ્રેટ થાય છે.
    $$V_{out} = -\frac{1}{RC} \int V_{in} dt$$
    \item \keyword{બીજો તબક્કો (T2)}: રેફરન્સ વોલ્ટેજ ઇન્ટિગ્રેટ થાય છે જ્યાં સુધી આઉટપુટ શૂન્ય ન થાય.
    $$T_2 = T_1 \times (\frac{V_{in}}{V_{ref}})$$
\end{itemize}

\textbf{ફાયદાઓ}: ઉત્કૃષ્ટ નોઈઝ રિજેક્શન, ઉચ્ચ ચોકસાઈ.
\end{solutionbox}

\begin{mnemonicbox}
\mnemonic{FIRE - પ્રથમ ઇનપુટ ઇન્ટિગ્રેટ કરો, પછી રેફરન્સ ઇન્ટિગ્રેટ કરો, જ્યાં સુધી શૂન્ય ન થાય}
\end{mnemonicbox}

% Q3 Start
\questionmarks{3(a)}{3}{CRO માં ડિલે લાઇન અને ટ્રિગર સર્કિટનું મહત્વ શું છે?}

\begin{solutionbox}
\textbf{ડિલે લાઇન મહત્વ}:
\begin{itemize}
    \item \keyword{હેતુ}: સ્વીપને ટ્રિગર કરતી ઘટનાઓને પ્રદર્શિત કરવા માટે સિગ્નલમાં વિલંબ કરે છે.
    \item \keyword{લાભ}: ટ્રિગરનું કારણ બનેલા સિગ્નલના અગ્ર કિનારાને જોવાની મંજૂરી આપે છે.
\end{itemize}

\textbf{ટ્રિગર સર્કિટ મહત્વ}:
\begin{itemize}
    \item \keyword{હેતુ}: ઇનપુટ સિગ્નલના ચોક્કસ બિંદુએ સ્વીપ શરૂ કરે છે.
    \item \keyword{લાભ}: પુનરાવર્તિત તરંગ માટે સ્થિર ડિસ્પ્લે સુનિશ્ચિત કરે છે.
\end{itemize}

\begin{center}
\captionof{table}{ડિલે લાઇન અને ટ્રિગર સર્કિટ}
\begin{tabulary}{\linewidth}{|L|L|L|}
\hline
\textbf{ઘટક} & \textbf{હેતુ} & \textbf{લાભ} \\ \hline
ડિલે લાઇન & સિગ્નલ પાથમાં વિલંબ & આખું વેવફોર્મ દેખાય છે \\ \hline
ટ્રિગર સર્કિટ & સ્વીપ શરૂ કરે છે & સ્થિર ડિસ્પ્લે આપે છે \\ \hline
\end{tabulary}
\end{center}
\end{solutionbox}

\begin{mnemonicbox}
\mnemonic{DT-SS - ડિલે ટુ સી સિગ્નલ, ટ્રિગર સ્ટોપ્સ સ્ક્રીન ડ્રિફ્ટ}
\end{mnemonicbox}

\questionmarks{3(b)}{4}{કેથોડ રે ટ્યુબ (CRT) ની આંતરિક રચના અને કાર્ય સ્વચ્છ આકૃતી સાથે સમજાવો.}

\begin{solutionbox}
\textbf{કેથોડ રે ટ્યુબ (CRT)} વિદ્યુત સિગ્નલોને દૃશ્ય પ્રદર્શનમાં રૂપાંતરિત કરે છે.

\textbf{રચના આકૃતિ}:
\begin{center}
\begin{tikzpicture}[xscale=0.8, yscale=0.8]
    % Glass envelope
    \draw [thick] (0,1) -- (2,1) -- (3,2) -- (7,2.5) -- (7,-2.5) -- (3,-2) -- (2,-1) -- (0,-1) -- cycle;
    \node at (7,0) [right] {સ્ક્રીન};
    
    % Gun components
    \draw (0.5,-0.2) rectangle (1,0.2); \node at (0.75,0.4) {K}; % Cathode
    \draw (1.2,-0.3) rectangle (1.4,0.3); \node at (1.3,0.5) {G}; % Grid
    \draw (1.6,-0.3) rectangle (1.9,0.3); \node at (1.75,0.5) {A1}; % Focusing
    \draw (2.1,-0.3) rectangle (2.4,0.3); \node at (2.25,0.5) {A2}; % Accelerating
    
    % Deflection Plates
    \draw (3.5, 0.5) -- (4.5, 0.5); \draw (3.5, -0.5) -- (4.5, -0.5); \node at (4, -0.8) {Y-પ્લેટ્સ};
    \draw (5, 0.5) -- (5, 1.5); \draw (5.5, 0.5) -- (5.5, 1.5); \node at (5.25, 1.8) {X-પ્લેટ્સ};
    
    % Beam
    \draw [dashed, red] (1,0) -- (7,0);
\end{tikzpicture}
\captionof{figure}{CRT રચના}
\end{center}

\textbf{કાર્ય}:
\begin{enumerate}
    \item \keyword{ઇલેક્ટ્રોન ગન}: ઇલેક્ટ્રોન બીમ જનરેટ કરે છે (કેથોડ), નિયંત્રિત કરે છે (ગ્રિડ), અને ફોકસ કરે છે (એનોડ્સ).
    \item \keyword{ડિફ્લેક્શન સિસ્ટમ}: બીમને ઊભી (Y) અને ક્ષૈતિજ (X) રીતે વાળે છે.
    \item \keyword{સ્ક્રીન}: ઇલેક્ટ્રોન અથડાવાથી ફોસ્ફર કોટિંગ ચમકે છે.
\end{enumerate}
\end{solutionbox}

\begin{mnemonicbox}
\mnemonic{EFADS - ઇલેક્ટ્રોન્સ ફ્લાય, એનોડ્સ ડાયરેક્ટ, સ્ક્રીન સિગ્નલ્સ બતાવે છે}
\end{mnemonicbox}

\questionmarks{3(c)}{7}{બ્લોક ડાયાગ્રામની મદદથી કેથોડ રે ઓસિલોસ્કોપ (CRO) નું કાર્ય સમજાવો અને દરેક બ્લોકના કાર્યનું વર્ણન કરો.}

\begin{solutionbox}
\textbf{બ્લોક ડાયાગ્રામ}:
\begin{center}
\begin{tikzpicture}[node distance=1.5cm, auto]
    \node [gtu block] (VA) {વર્ટિકલ Amp};
    \node [gtu block, below=of VA] (DL) {ડિલે લાઇન};
    \node [coordinate, left=of VA] (In) {};
    \node [gtu block, right=of DL] (CRT) {CRT};
    \node [gtu block, below=of DL] (Trig) {ટ્રિગર};
    \node [gtu block, right=of Trig] (TB) {ટાઇમ બેઝ};
    \node [gtu block, right=of TB] (HA) {હોરિઝોન્ટલ Amp};
    \node [gtu block, below=of Trig] (PS) {પાવર સપ્લાય};
    
    \draw [gtu arrow] (In) -- node[above]{Input} (VA);
    \draw [gtu arrow] (VA) -- (DL);
    \draw [gtu arrow] (DL) -- (CRT);
    \draw [gtu arrow] (VA) |- (Trig);
    \draw [gtu arrow] (Trig) -- (TB);
    \draw [gtu arrow] (TB) -- (HA);
    \draw [gtu arrow] (HA) -| (CRT);
    \draw [gtu arrow] (PS) -| (CRT);
\end{tikzpicture}
\captionof{figure}{CRO બ્લોક ડાયાગ્રામ}
\end{center}

\textbf{કાર્ય}:
\begin{itemize}
    \item \keyword{વર્ટિકલ એમ્પ્લિફાયર}: વર્ટિકલ ડિફ્લેક્શન માટે સિગ્નલ એમ્પ્લિફાય કરે છે.
    \item \keyword{ડિલે લાઇન}: સિગ્નલને સ્વીપ સાથે સિન્ક્રોનાઇઝ કરવા વિલંબ કરે છે.
    \item \keyword{ટ્રિગર સર્કિટ}: સ્વીપને સિગ્નલ ફ્રિકવન્સી સાથે સિંક કરે છે.
    \item \keyword{ટાઇમ બેઝ}: ક્ષૈતિજ સ્વીપ માટે સોટૂથ વેવ બનાવે છે.
    \item \keyword{હોરિઝોન્ટલ એમ્પ્લિફાયર}: સ્વીપ સિગ્નલ એમ્પ્લિફાય કરે છે.
    \item \keyword{CRT}: વેવફોર્મ પ્રદર્શિત કરે છે.
\end{itemize}
\end{solutionbox}

\begin{mnemonicbox}
\mnemonic{VATH-CDS - વર્ટિકલ એટેન્યુએટ્સ થેન એમ્પ્લિફાઇઝ, હોરિઝોન્ટલ ક્રિએટ્સ ડિફ્લેક્શન સ્વીપ}
\end{mnemonicbox}

\questionmarks{3(a) OR}{3}{કેથોડ રે ઓસિલોસ્કોપ (CRO) અને ડિજિટલ સ્ટોરેજ ઓસિલોસ્કોપ (DSO) વચ્ચેનો તફાવત આપો.}

\begin{solutionbox}
\begin{center}
\captionof{table}{CRO અને DSO તુલના}
\begin{tabulary}{\linewidth}{|L|L|L|}
\hline
\textbf{પેરામીટર} & \textbf{CRO} & \textbf{DSO} \\ \hline
સિગ્નલ પ્રોસેસિંગ & એનાલોગ & ડિજિટલ (ADC રૂપાંતરણ) \\ \hline
સ્ટોરેજ & કોઈ નહીં (રીયલ-ટાઇમ) & વેવફોર્મ સ્ટોર કરે છે \\ \hline
બેન્ડવિડ્થ & મર્યાદિત & ઉચ્ચ શક્ય \\ \hline
વિશ્લેષણ & મૂળભૂત & અદ્યતન (FFT, ઓટો મેઝર) \\ \hline
\end{tabulary}
\end{center}
\end{solutionbox}

\begin{mnemonicbox}
\mnemonic{DSO-MAPS - ડિજિટલ સ્ટોરેજ ઓસિલોસ્કોપ માપે, એનાલાઇઝ, પ્રોસેસ, સિગ્નલ્સ સંગ્રહે છે}
\end{mnemonicbox}

\questionmarks{3(b) OR}{4}{ફ્રીકવન્સી અને ફેઝ એંગલ CRO (Cathode Ray Oscilloscope)ની મદદથી કેવી રીતે નિર્ધારિત કરી શકાય છે તે સમજાવો.}

\begin{solutionbox}
\textbf{ફ્રીકવન્સી માપન}:
\begin{enumerate}
    \item ટાઇમ પીરિયડ $T$ માપો (1 સાયકલનું અંતર $\times$ Time/div).
    \item ફ્રીકવન્સી $f = 1/T$.
\end{enumerate}

\textbf{ફેઝ એંગલ માપન}:
\begin{enumerate}
    \item બે સિગ્નલ દર્શાવો.
    \item સમય તફાવત $\Delta t$ માપો.
    \item પીરિયડ $T$ માપો.
    \item ફેઝ $\phi = (\frac{\Delta t}{T}) \times 360^\circ$.
\end{enumerate}

\textbf{આકૃતિ}:
\begin{center}
\begin{tikzpicture}
    \draw [->] (0,0) -- (6,0) node[right] {સમય};
    \draw [->] (0,-2) -- (0,2) node[above] {વોલ્ટેજ};
    \draw [blue, thick] plot [domain=0:6, samples=100] (\x, {sin(100*\x)});
    \draw [red, thick, dashed] plot [domain=0:6, samples=100] (\x, {sin(100*\x - 60)});
    
    % Annotations
    \draw [|<->|] (0.9, 1.2) -- (4.5, 1.2) node[midway, above] {પીરિયડ T};
    \draw [|<->|] (1.8, 0) -- (2.4, 0) node[midway, below] {$\Delta t$};
\end{tikzpicture}
\captionof{figure}{ફેઝ શિફ્ટ માપન}
\end{center}
\end{solutionbox}

\begin{mnemonicbox}
\mnemonic{FPL - ફ્રીકવન્સી = પિરિયડની લંબાઈનો વ્યસ્ત, ફેઝ = (લેગ/પિરિયડ) × 360}
\end{mnemonicbox}

\questionmarks{3(c) OR}{7}{ડિજિટલ સ્ટોરેજ ઓસિલોસ્કોપ (DSO) નો બ્લોક ડાયાગ્રામ દોરો અને દરેક બ્લોકનું કાર્ય સમજાવો.}

\begin{solutionbox}
\textbf{ડિજિટલ સ્ટોરેજ ઓસિલોસ્કોપ (DSO)} એનાલોગ સિગ્નલને ડિજિટલ સ્વરૂપમાં ફેરવે છે.

\textbf{બ્લોક ડાયાગ્રામ}:
\begin{center}
\begin{tikzpicture}[node distance=1.5cm, auto]
    \node [gtu block] (ADC) {ADC};
    \node [gtu block, left=of ADC] (Amp) {Amp/Atten};
    \node [coordinate, left=of Amp] (In) {};
    \node [gtu block, right=of ADC] (Mem) {મેમરી};
    \node [gtu block, right=of Mem] (Proc) {પ્રોસેસર};
    \node [gtu block, below=of Proc] (Disp) {ડિસ્પ્લે};
    \node [gtu block, below=of ADC] (Clock) {ક્લોક/કંટ્રોલ};
    
    \draw [gtu arrow] (In) -- node[above]{Input} (Amp);
    \draw [gtu arrow] (Amp) -- (ADC);
    \draw [gtu arrow] (ADC) -- (Mem);
    \draw [gtu arrow] (Mem) -- (Proc);
    \draw [gtu arrow] (Proc) -- (Disp);
    \draw [gtu arrow] (Clock) -- (ADC);
    \draw [gtu arrow] (Clock) |- (Mem);
\end{tikzpicture}
\captionof{figure}{DSO બ્લોક ડાયાગ્રામ}
\end{center}

\textbf{કાર્ય}:
\begin{itemize}
    \item \keyword{ADC}: સેમ્પલિંગ અને ડિજિટાઇઝેશન.
    \item \keyword{મેમરી}: ડિજિટલ સેમ્પલ્સ સંગ્રહે છે.
    \item \keyword{પ્રોસેસર}: વેવફોર્મ બનાવે છે અને ગણતરી કરે છે.
    \item \keyword{ડિસ્પ્લે}: LCD પર સિગ્નલ બતાવે છે.
\end{itemize}

\textbf{ફાયદા}: સિંગલ-શોટ કેપ્ચર, પ્રી-ટ્રિગર વ્યૂ, ગાણિતિક વિશ્લેષણ.
\end{solutionbox}

\begin{mnemonicbox}
\mnemonic{AADPD - એટેન્યુએટ એનાલોગ, ડિજિટાઇઝ, પ્રોસેસ, ડિસ્પ્લે સિગ્નલ}
\end{mnemonicbox}

% Q4 Start
\questionmarks{4(a)}{3}{વિવિધ પ્રકારના ટ્રાન્સડ્યૂસરનું વર્ગીકરણ કરો.}

\begin{solutionbox}
\textbf{ટ્રાન્સડ્યૂસરનું વર્ગીકરણ}:
\begin{center}
\captionof{table}{ટ્રાન્સડ્યૂસર વર્ગીકરણ}
\begin{tabulary}{\linewidth}{|L|L|}
\hline
\textbf{વર્ગીકરણ આધાર} & \textbf{પ્રકારો} \\ \hline
ઓપરેશનનો સિદ્ધાંત & મિકેનિકલ, ઇલેક્ટ્રિકલ, થર્મલ, ઓપ્ટિકલ, કેમિકલ \\ \hline
ઇનપુટ/આઉટપુટ સંબંધ & પ્રાઇમરી, સેકન્ડરી \\ \hline
સિગ્નલ જનરેશન & એક્ટિવ, પેસિવ \\ \hline
ઇલેક્ટ્રિકલ પેરામીટર્સ & રેઝિસ્ટિવ, કેપેસિટિવ, ઇન્ડક્ટિવ \\ \hline
\end{tabulary}
\end{center}

\textbf{મુખ્ય વર્ગીકરણ}:
\begin{itemize}
    \item \keyword{એક્ટિવ ટ્રાન્સડ્યૂસર}: બાહ્ય પાવર વિના ઇલેક્ટ્રિકલ આઉટપુટ જનરેટ કરે છે (દા.ત., થર્મોકપલ).
    \item \keyword{પેસિવ ટ્રાન્સડ્યૂસર}: બાહ્ય પાવરની જરૂર પડે છે (દા.ત., થર્મિસ્ટર).
    \item \keyword{પ્રાઇમરી ટ્રાન્સડ્યૂસર}: ભૌતિક ફેરફારને સીધા ઇલેક્ટ્રિકલ સિગ્નલમાં રૂપાંતરિત કરે છે.
\end{itemize}
\end{solutionbox}

\begin{mnemonicbox}
\mnemonic{APRCI - એક્ટિવ/પેસિવ, રેઝિસ્ટિવ/કેપેસિટિવ/ઇન્ડક્ટિવ મુખ્ય કેટેગરી છે}
\end{mnemonicbox}

\questionmarks{4(b)}{4}{સ્ટ્રેઇન ગેજનું બંધારણ અને કાર્ય સમજાવો.}

\begin{solutionbox}
\textbf{સ્ટ્રેઇન ગેજ} યાંત્રિક સ્ટ્રેઇન (વિરૂપણ)ને વિદ્યુત અવરોધ પરિવર્તનમાં રૂપાંતરિત કરે છે.

\textbf{બંધારણ}:
\begin{itemize}
    \item \keyword{ગ્રીડ પેટર્ન}: ઝિગઝેગ પેટર્નમાં પાતળી ફોઇલ અથવા વાયર.
    \item \keyword{બેકિંગ મટીરિયલ}: પોલિમાઇડ અથવા એપોક્સી કેરિયર.
\end{itemize}

\textbf{આકૃતિ}:
\begin{center}
\begin{tikzpicture}
    % Backing
    \fill[yellow!20] (0,0) rectangle (4,2.5);
    \draw (0,0) rectangle (4,2.5);
    
    % Grid
    \draw [thick] (0.5, 0.5) -- (3.5, 0.5);
    \foreach \y in {0.5, 0.7, ..., 2.0}
        \draw [thick] (0.5, \y) -- (3.5, \y);
    \draw [thick] (0.5, 0.5) -- (0.5, 2.0); % Connectors left
    \draw [thick] (3.5, 0.5) -- (3.5, 2.0); % Connectors right
    
    % Zigzag approximation
    \draw [thick] (0.5, 0.5) -- (1, 0.5) -- (1, 2) -- (1.2, 2) -- (1.2, 0.5) -- (1.4, 0.5) -- (1.4, 2) -- (3, 2); % Simple Rep just lines
    % Better drawing of grid:
    \draw [thick] (1,1) -- (3,1); % Just symbolic
    \node at (2, 1.25) {વાયર ગ્રીડ};
    
    \draw [thick, ->] (0.5, 1.25) -- (-0.5, 1.25) node[left] {લીડ વાયર};
    \draw [thick, ->] (3.5, 1.25) -- (4.5, 1.25) node[right] {લીડ વાયર};
\end{tikzpicture}
\captionof{figure}{સ્ટ્રેઇન ગેજ}
\end{center}

\textbf{કાર્ય સિદ્ધાંત}:
\begin{itemize}
    \item પિઝોરેઝિસ્ટિવ ઇફેક્ટ પર આધારિત.
    \item વિરૂપણ સૂત્ર: $\frac{\Delta R}{R} = GF \times \epsilon$ (જ્યાં GF = ગેજ ફેક્ટર).
    \item વ્હીટસ્ટોન બ્રિજમાં જોડીને અવરોધ ફેરફાર માપવામાં આવે છે.
\end{itemize}
\end{solutionbox}

\begin{mnemonicbox}
\mnemonic{GRID - ગેજ રેઝિસ્ટન્સ ઇન્ક્રીઝ વિથ ડિફોર્મેશન}
\end{mnemonicbox}

\questionmarks{4(c)}{7}{લિનિયર વેરિએબલ ડિફરન્શિયલ ટ્રાન્સડ્યુસર (LVDT) ને તેના બંધારણ, કાર્યપદ્ધતિ, ફાયદા અને ઉપયોગો સાથે સમજાવો.}

\begin{solutionbox}
\textbf{LVDT} લિનિયર ડિસ્પ્લેસમેન્ટને ઇલેક્ટ્રિકલ સિગ્નલમાં રૂપાંતરિત કરે છે.

\textbf{આકૃતિ}:
\begin{center}
\begin{circuitikz}[american, scale=0.8]
    \draw (0,0) to[L, l=$P$] (0,4); % Primary
    \draw (3,4) to[L, l=$S_1$] (3,2); % Sec 1
    \draw (3,2) to[L, l=$S_2$] (3,0); % Sec 2
    
    \draw [thick] (1, 0.5) rectangle (1.5, 3.5); \node at (1.25, 2) {Core};
    \draw [->] (1.25, 3.6) -- (1.25, 4.2) node[above] {ગતિ};
    
    \draw (0,4) to[short] (-1,4) to[sV, l=AC] (-1,0) to[short] (0,0);
\end{circuitikz}
\captionof{figure}{LVDT બાંધકામ}
\end{center}

\textbf{કાર્ય સિદ્ધાંત}:
\begin{itemize}
    \item પ્રાઇમરી કોઇલને AC વોલ્ટેજ અપાય છે.
    \item કોરની સ્થિતિ મુજબ સેકન્ડરી કોઇલ્સ ($S_1, S_2$) માં વોલ્ટેજ પ્રેરિત થાય છે.
    \item આઉટપુટ $V_{out} = V_{S1} - V_{S2}$.
    \item નલ પોઝિશન પર $V_{out} = 0$.
\end{itemize}

\textbf{ફાયદાઓ}: ઘર્ષણ વિનાનું કાર્ય, અનંત રિઝોલ્યુશન, ઉચ્ચ લિનિયરિટી.

\textbf{ઉપયોગો}: LVDT નો ઉપયોગ ઔદ્યોગિક ઓટોમેશન, એરોસ્પેસ અને સિવિલ એન્જિનિયરિંગમાં થાય છે.
\end{solutionbox}

\begin{mnemonicbox}
\mnemonic{LVDT-MAPS - લિનિયર વેરિએબલ ડિફરન્શિયલ ટ્રાન્સફોર્મર સેકન્ડરી વોલ્ટેજ તફાવત દ્વારા પોઝિશન ચોકસાઇથી માપે છે}
\end{mnemonicbox}

\questionmarks{4(a) OR}{3}{પીએચ સેન્સરના ત્રણ ઉપયોગો જણાવો.}

\begin{solutionbox}
\textbf{PH સેન્સરના ઉપયોગો}:
\begin{center}
\captionof{table}{PH સેન્સર ઉપયોગો}
\begin{tabulary}{\linewidth}{|L|L|}
\hline
\textbf{ઉપયોગ} & \textbf{હેતુ} \\ \hline
વોટર ટ્રીટમેન્ટ & પાણીની શુદ્ધતા મોનિટર કરવા \\ \hline
કૃષિ & જમીનની એસિડિટી માપવા માટે \\ \hline
મેડિકલ & રક્ત pH માપન માટે \\ \hline
ફૂડ પ્રોસેસિંગ & ઉત્પાદન ગુણવત્તા માટે \\ \hline
\end{tabulary}
\end{center}
\end{solutionbox}

\begin{mnemonicbox}
\mnemonic{WAM - વોટર ક્વાલિટી કંટ્રોલ, એગ્રિકલ્ચર સોઇલ ટેસ્ટિંગ, મેડિકલ ડાયગ્નોસ્ટિક્સ મુખ્ય PH સેન્સર ઉપયોગો છે}
\end{mnemonicbox}

\questionmarks{4(b) OR}{4}{કેપેસિટિવ ટ્રાન્સડ્યૂસરનું બંધારણ અને કાર્ય સમજાવો.}

\begin{solutionbox}
\textbf{કેપેસિટિવ ટ્રાન્સડ્યૂસર}:

\textbf{આકૃતિ}:
\begin{center}
\begin{tikzpicture}
    \draw [thick] (0,2) -- (4,2) node[right] {Plate A};
    \draw [thick] (0,1) -- (4,1) node[right] {Plate B};
    \draw [<->] (2,1) -- (2,2) node[midway, right] {d};
    \node at (2, 1.5) {Dielectric ($\epsilon_r$)};
    \draw (0,2) -- (-0.5, 2); \draw (0,1) -- (-0.5,1);
\end{tikzpicture}
\captionof{figure}{કેપેસિટિવ ટ્રાન્સડ્યૂસર}
\end{center}

\textbf{કાર્ય સિદ્ધાંત}:
\begin{itemize}
    \item $C = \frac{\epsilon A}{d}$.
    \item \keyword{A}: પ્લેટ એરિયા બદલવાથી.
    \item \keyword{d}: પ્લેટ અંતર બદલવાથી.
    \item \keyword{$\epsilon$}: ડાઇલેક્ટ્રિક બદલવાથી.
\end{itemize}

\textbf{ઉપયોગો}: પ્રેશર, લેવલ, અને ડિસ્પ્લેસમેન્ટ માપન.
\end{solutionbox}

\begin{mnemonicbox}
\mnemonic{CAD - કેપેસિટન્સ એરિયા, ડિસ્ટન્સ, અથવા ડાઇલેક્ટ્રિક પરિવર્તન સાથે બદલાય છે}
\end{mnemonicbox}

\questionmarks{4(c) OR}{7}{એબ્સોલ્યુટ ઑપ્ટિકલ એન્કોડર શું છે? એના A, B અને C આઉટપુટ વેવફોર્મ વિશે સમજાવો અને યોગ્ય આકૃતિ આપો. તેની વિગતવાર સમજૂતી આપો.}

\begin{solutionbox}
\textbf{એબ્સોલ્યુટ ઑપ્ટિકલ એન્કોડર} ડિજિટલ કોડ દ્વારા આંતરિક પોઝિશન માપે છે. (નોંધ: પ્રશ્ન એબ્સોલ્યુટ વિશે છે, પણ A, B, C આઉટપુટ સામાન્ય રીતે ઇન્ક્રીમેન્ટલ એન્કોડરના હોય છે. અહીં આપણે ઇન્ક્રીમેન્ટલના આઉટપુટ વેવફોર્મ સમજાવીશું કેમ કે પ્રશ્ન A, B, C માંગે છે).

\textbf{આકૃતિ}:
\begin{center}
\begin{tikzpicture}
    \node at (0,3) {LED Source};
    \draw [fill=gray!20] (0,2) ellipse (2 and 0.5); \node at (2.5,2) {Code Disk};
    \node at (0,1) {Photodetectors};
    \draw [->] (0,2.8) -- (0,2.2);
    \draw [->] (0,1.8) -- (0,1.2);
\end{tikzpicture}
\captionof{figure}{એન્કોડર રચના}
\end{center}

\textbf{આઉટપુટ વેવફોર્મ્સ}:
\begin{center}
\begin{tikzpicture}
    % A Signal
    \draw (0,4) node[left] {A} -- (1,4) -- (1,5) -- (2,5) -- (2,4) -- (3,4) -- (3,5) -- (4,5);
    % B Signal (Assuming 90 deg shift)
    \draw (0,2.5) node[left] {B} -- (0.5,2.5) -- (0.5,3.5) -- (1.5,3.5) -- (1.5,2.5) -- (2.5,2.5) -- (2.5,3.5) -- (3.5,3.5);
    % C Signal (Index)
    \draw (0,1) node[left] {C} -- (3.5,1) -- (3.5,2) -- (3.7,2) -- (3.7,1) -- (4,1);
\end{tikzpicture}
\captionof{figure}{A, B, C વેવફોર્મ્સ}
\end{center}

\textbf{સમજૂતી}:
\begin{itemize}
    \item \keyword{A Signal}: પોઝિશન માહિતી (પલ્સ ગણતરી).
    \item \keyword{B Signal}: દિશા માહિતી (A સાથેનો ફેઝ તફાવત).
    \item \keyword{C Signal}: રેફરન્સ પલ્સ (પ્રતિ રિવોલ્યુશન એક વાર).
\end{itemize}
\end{solutionbox}

\begin{mnemonicbox}
\mnemonic{ABC-PDP - એબ્સોલ્યુટ એન્કોડર ટ્રેક્સ A, B, C દિશા, પોઝિશન, અને રેફરન્સ પલ્સ પ્રદાન કરે છે}
\end{mnemonicbox}

% Q5 Start
\questionmarks{5(a)}{3}{બેસિક ફ્રિકવન્સી કાઉન્ટરનો કાર્યસિદ્ધાંત સમજાવો.}

\begin{solutionbox}
\textbf{ફ્રિકવન્સી કાઉન્ટર} ચોક્કસ સમય અંતરાલ ઉપર ઘટનાઓ ગણીને ઇનપુટ સિગ્નલની ફ્રિકવન્સી માપે છે.

\textbf{બ્લોક ડાયાગ્રામ}:
\begin{center}
\begin{tikzpicture}[node distance=1.5cm, auto]
    \node [gtu block] (In) {ઇનપુટ};
    \node [gtu block, right=of In] (Gate) {AND ગેટ};
    \node [gtu block, right=of Gate] (Count) {કાઉન્ટર};
    \node [gtu block, right=of Count] (Disp) {ડિસ્પ્લે};
    \node [gtu block, below=of Gate] (Control) {ગેટ કંટ્રોલ};
    \node [gtu block, left=of Control] (Time) {ટાઇમ બેઝ};
    
    \draw [gtu arrow] (In) -- (Gate);
    \draw [gtu arrow] (Gate) -- (Count);
    \draw [gtu arrow] (Count) -- (Disp);
    \draw [gtu arrow] (Time) -- (Control);
    \draw [gtu arrow] (Control) -- (Gate);
\end{tikzpicture}
\captionof{figure}{ફ્રિકવન્સી કાઉન્ટર}
\end{center}

\textbf{કાર્ય સિદ્ધાંત}: ઈનપુટ પલ્સ ગણવામાં આવે છે જ્યારે ગેટ ખુલ્લો હોય છે. ફ્રિકવન્સી = ગણતરી / ગેટ સમય.
\end{solutionbox}

\begin{mnemonicbox}
\mnemonic{CTPG - કાઉન્ટ ધ પલ્સીસ, ગેટ ધ ટાઇમ}
\end{mnemonicbox}

\questionmarks{5(b)}{4}{એનર્જી મીટરનો ડાયાગ્રામ દોરો અને તેનો કાર્યસિદ્ધાંત સમજાવો.}

\begin{solutionbox}
\textbf{ઇલેક્ટ્રોનિક એનર્જી મીટર} kWh માં ઊર્જા વપરાશ માપે છે.

\textbf{બ્લોક ડાયાગ્રામ}:
\begin{center}
\begin{tikzpicture}[node distance=1.5cm, auto]
    \node [gtu block] (Mult) {મલ્ટિપ્લાયર};
    \node [gtu block, left=of Mult] (Sensors) {V \& I સેન્સર્સ};
    \node [gtu block, right=of Mult] (VFC) {V-to-F કન્વર્ટર};
    \node [gtu block, right=of VFC] (Count) {કાઉન્ટર};
    \node [gtu block, right=of Count] (Micro) {માઇક્રોકંટ્રોલર};
    \node [gtu block, below=of Micro] (Disp) {LCD};
    
    \draw [gtu arrow] (Sensors) -- (Mult);
    \draw [gtu arrow] (Mult) -- (VFC);
    \draw [gtu arrow] (VFC) -- (Count);
    \draw [gtu arrow] (Count) -- (Micro);
    \draw [gtu arrow] (Micro) -- (Disp);
\end{tikzpicture}
\captionof{figure}{એનર્જી મીટર}
\end{center}

\textbf{કાર્ય}: વોલ્ટેજ અને કરંટનો ગુણાકાર કરી પાવર મેળવવામાં આવે છે, જે સમય સાથે ઇન્ટિગ્રેટ થઈ ઊર્જા (kWh) આપે છે.
\end{solutionbox}

\begin{mnemonicbox}
\mnemonic{VCPI - વોલ્ટેજ અને કરંટ ગુણાકાર થાય છે, પલ્સ ઊર્જા વપરાશ દર્શાવે છે}
\end{mnemonicbox}

\questionmarks{5(c)}{7}{ફંક્શન જનરેટરનો કાર્યસિદ્ધાંત અને કાર્યવિધી સંક્ષિપ્તમાં સમજાવો. તેના ફ્રન્ટ પેનલ કંટ્રોલ્સનું વર્ણન કરો અને તે કેવી રીતે ઇલેક્ટ્રોનિક પરિપથોની તપાસ માટે ઉપયોગી છે તે ઉદાહરણ સાથે સમજાવો.}

\begin{solutionbox}
\textbf{ફંક્શન જનરેટર} વિવિધ વેવફોર્મ્સ (સાઇન, સ્ક્વેર, ટ્રાયેંગલ) ઉત્પન્ન કરે છે.

\textbf{બ્લોક ડાયાગ્રામ}:
\begin{center}
\begin{tikzpicture}[node distance=1.5cm, auto]
    \node [gtu block] (Osc) {ઓસિલેટર};
    \node [gtu block, right=of Osc] (Shaper) {વેવ શેપર};
    \node [gtu block, right=of Shaper] (Amp) {આઉટપુટ Amp};
    \node [coordinate, right=of Amp] (Out) {};
    \node [gtu block, below=of Osc] (Freq) {ફ્રિકવન્સી કંટ્રોલ};
    \node [gtu block, below=of Amp] (Gain) {એમ્પ્લિટ્યુડ કંટ્રોલ};
    
    \draw [gtu arrow] (Osc) -- (Shaper);
    \draw [gtu arrow] (Shaper) -- (Amp);
    \draw [gtu arrow] (Amp) -- (Out);
    \draw [gtu arrow] (Freq) -- (Osc);
    \draw [gtu arrow] (Gain) -- (Amp);
\end{tikzpicture}
\captionof{figure}{ફંક્શન જનરેટર}
\end{center}

\textbf{ફ્રન્ટ પેનલ કંટ્રોલ્સ}:
\begin{itemize}
    \item \keyword{ફ્રિકવન્સી}: 0.1 Hz - 20 MHz.
    \item \keyword{એમ્પ્લિટ્યુડ}: 0 - 20 Vpp.
    \item \keyword{DC ઓફસેટ}: DC લેવલ શિફ્ટ.
    \item \keyword{વેવફોર્મ}: સાઇન, સ્ક્વેર, ટ્રાયેંગલ.
\end{itemize}

\textbf{ઉપયોગ}: એમ્પ્લિફાયર ગેઇન ટેસ્ટિંગ માટે ઇનપુટ સિગ્નલ તરીકે વપરાય છે.
\end{solutionbox}

\begin{mnemonicbox}
\mnemonic{FAWOD - ફ્રિકવન્સી, એમ્પ્લિટ્યુડ, વેવફોર્મ, ઓફસેટ, ડ્યુટી સાયકલ મુખ્ય કંટ્રોલ્સ છે}
\end{mnemonicbox}

\questionmarks{5(a) OR}{3}{સ્પેક્ટ્રમ એનાલાઈઝરનું કાર્ય સમજાવો.}

\begin{solutionbox}
\textbf{સ્પેક્ટ્રમ એનાલાઇઝર} સિગ્નલને ફ્રિકવન્સી ડોમેનમાં દર્શાવે છે (Amplitude vs Frequency).

\textbf{બ્લોક ડાયાગ્રામ}:
\begin{center}
\begin{tikzpicture}[node distance=1.5cm, auto]
    \node [gtu block] (Mix) {મિક્સર};
    \node [gtu block, left=of Mix] (In) {Input};
    \node [gtu block, below=of Mix] (LO) {Local Osc};
    \node [gtu block, right=of Mix] (IF) {IF Filter};
    \node [gtu block, right=of IF] (Det) {Detector};
    \node [gtu block, right=of Det] (Disp) {Display};
    \node [gtu block, below=of LO] (Sweep) {Sweep Gen};
    
    \draw [gtu arrow] (In) -- (Mix);
    \draw [gtu arrow] (LO) -- (Mix);
    \draw [gtu arrow] (Mix) -- (IF);
    \draw [gtu arrow] (IF) -- (Det);
    \draw [gtu arrow] (Det) -- (Disp);
    \draw [gtu arrow] (Sweep) -- (LO);
    \draw [gtu arrow] (Sweep) -| (Disp);
\end{tikzpicture}
\captionof{figure}{સ્પેક્ટ્રમ એનાલાઇઝર}
\end{center}

\textbf{ઉપયોગો}: હાર્મોનિક્સ, ડિસ્ટોર્શન, અને EMI મેઝરમેન્ટ.
\end{solutionbox}

\begin{mnemonicbox}
\mnemonic{SAME - સ્પેક્ટ્રમ એનાલાઇઝર ફ્રિકવન્સી પર સિગ્નલ એનર્જી મેપ કરે છે}
\end{mnemonicbox}

\questionmarks{5(b) OR}{4}{ક્લેમ્પ ઓન મીટરનો ડાયાગ્રામ દોરો અને તેનું કાર્ય સમજાવો.}

\begin{solutionbox}
\textbf{ક્લેમ્પ-ઓન મીટર} નોન-કોન્ટેક્ટ કરંટ મેઝરમેન્ટ માટે વપરાય છે.

\textbf{આકૃતિ}:
\begin{center}
\begin{tikzpicture}
    % Body
    \draw [rounded corners] (0,0) rectangle (2,4);
    % Clamp
    \draw [thick] (0,4) arc (180:0:1);
    \draw [thick] (0.2,4) arc (180:0:0.8);
    % Display
    \draw (0.5, 2.5) rectangle (1.5, 3.5); \node at (1,3) {10.5 A};
    % Selector
    \draw (1, 1.5) circle (0.3);
\end{tikzpicture}
\captionof{figure}{ક્લેમ્પ મીટર}
\end{center}

\textbf{કાર્ય સિદ્ધાંત}:
\begin{itemize}
    \item ટ્રાન્સફોર્મર સિદ્ધાંત પર કાર્ય કરે છે.
    \item કંડક્ટર પ્રાઇમરી વાઇન્ડિંગ તરીકે અને ક્લેમ્પ સેકન્ડરી તરીકે વર્તે છે.
    \item પ્રેરિત કરંટ માપવામાં આવે છે.
\end{itemize}
\end{solutionbox}

\begin{mnemonicbox}
\mnemonic{CLIP - ક્લેમ્પ કરંટ માપે છે, મેગ્નેટિક ઇન્ડક્શન વોલ્ટેજ પેદા કરે છે}
\end{mnemonicbox}

\questionmarks{5(c) OR}{7}{ડિજિટલ IC ટેસ્ટરનું કાર્યસિદ્ધાંત સમજાવો. તેનો બ્લોક ડાયાગ્રામ સમજાવો અને તે ડિજિટલ IC ની કાર્યક્ષમતા કઈ રીતે ચકાસે છે તે ઉદાહરણ સાથે સમજાવો.}

\begin{solutionbox}
\textbf{ડિજિટલ IC ટેસ્ટર} IC ની કાર્યક્ષમતા ચકાસે છે.

\textbf{બ્લોક ડાયાગ્રામ}:
\begin{center}
\begin{tikzpicture}[node distance=1.5cm, auto]
    \node [gtu block] (Micro) {માઇક્રોકંટ્રોલર};
    \node [gtu block, right=of Micro] (Pattern) {પેટર્ન જનરેટર};
    \node [gtu block, right=of Pattern] (Socket) {ZIF સોકેટ (IC)};
    \node [gtu block, below=of Socket] (Analyze) {રિસ્પોન્સ એનાલાઇઝર};
    \node [gtu block, left=of Analyze] (Display) {ડિસ્પ્લે};
    \node [gtu block, above=of Micro] (Keypad) {કીપેડ};
    
    \draw [gtu arrow] (Micro) -- (Pattern);
    \draw [gtu arrow] (Pattern) -- (Socket);
    \draw [gtu arrow] (Socket) -- (Analyze);
    \draw [gtu arrow] (Analyze) -- (Micro);
    \draw [gtu arrow] (Micro) -- (Display);
    \draw [gtu arrow] (Keypad) -- (Micro);
\end{tikzpicture}
\captionof{figure}{IC ટેસ્ટર}
\end{center}

\textbf{કાર્ય સિદ્ધાંત}:
\begin{enumerate}
    \item IC પ્રકાર પસંદ કરો.
    \item ટેસ્ટર ઇનપુટ પેટર્ન લાગુ કરે છે.
    \item આઉટપુટની અપેક્ષિત પરિણામ સાથે સરખામણી કરે છે.
    \item PASS/FAIL દર્શાવે છે.
\end{enumerate}

\textbf{ઉદાહરણ (7400 NAND)}: બધા પિન કોમ્બિનેશન (00, 01, 10, 11) આપી ટ્રુથ ટેબલ ચકાસવામાં આવે છે.
\end{solutionbox}

\begin{mnemonicbox}
\mnemonic{TEST - ટેસ્ટ પેટર્ન બધી સ્ટેટ્સનો અભ્યાસ કરે છે, પછી આઉટપુટ ચકાસે છે}
\end{mnemonicbox}

\end{document}




