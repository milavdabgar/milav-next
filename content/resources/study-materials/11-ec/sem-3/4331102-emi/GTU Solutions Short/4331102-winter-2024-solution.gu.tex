\documentclass{article}

% content/resources/templates/preamble.tex
\usepackage[margin=0.6in]{geometry}
\author{Milav Dabgar}
\usepackage{amsmath,amssymb,amsthm}
\usepackage{booktabs}
\usepackage{multirow}
\usepackage{xcolor}
\usepackage{tcolorbox}
\tcbuselibrary{breakable,skins}
\usepackage[colorlinks=true,linkcolor=blue]{hyperref}
\usepackage{titlesec}
\usepackage{enumitem}
\usepackage{tikz}
\usepackage{pgfplots}
\usepackage{circuitikz}
\usepackage[version=4]{mhchem}
\usepackage{longtable}
\usepackage{array}
\usepackage{float}
\usepackage{caption}
\usepackage{listings}

\lstset{
  basicstyle=\small\ttfamily,
  breaklines=true,
  breakatwhitespace=false,
  postbreak=\mbox{\textcolor{red}{$\hookrightarrow$}\space},
  float=false,
  numbers=left,
  numberstyle=\tiny\color{gray},
  numbersep=10pt,
  xleftmargin=2em,
  keywordstyle=\color{blue},
  commentstyle=\color{green!60!black},
  stringstyle=\color{purple},
  backgroundcolor=\color{gray!5},
  showstringspaces=false,
  tabsize=2,
  captionpos=b,
  keepspaces=true,
  columns=flexible
}

\pgfplotsset{compat=1.18}
\usetikzlibrary{shapes,arrows,positioning,calc,patterns,decorations.pathmorphing,decorations.markings,arrows.meta}

% Color scheme
\definecolor{headcolor}{RGB}{0,102,204}
\definecolor{keycolor}{RGB}{220,20,60}
\definecolor{solutioncolor}{RGB}{34,139,34}
\definecolor{mnemoniccolor}{RGB}{148,0,211}
\definecolor{codecolor}{RGB}{0,0,100}

% Spacing
\setlength{\parskip}{3pt}
\setlist[itemize]{nosep}
\setlist[enumerate]{nosep}

% Title formatting
\titleformat{\section}{\Large\bfseries\color{headcolor}}{\thesection}{1em}{}
\titleformat{\subsection}{\large\bfseries\color{headcolor}}{\thesubsection}{1em}{}

% Pandoc tightlist compatibility
\providecommand{\tightlist}{%
  \setlength{\itemsep}{0pt}\setlength{\parskip}{0pt}}

% Pandoc longtable compatibility
\newcounter{none}
\def\thenone{}


% content/resources/templates/gujarati-boxes.tex
\usepackage{fontspec}
\usepackage{polyglossia}

% Set Gujarati as main language (document is primarily in Gujarati)
% Note: gloss-gujarati.ldf doesn't exist in polyglossia, but it will use hyphenation patterns
\setdefaultlanguage{gujarati}
\setotherlanguage{english}

% Configure Gujarati font properly
% Use Language=Default to prevent polyglossia from trying to add language-specific features
% that don't exist for Gujarati, which causes "empty feature" warnings
\newfontfamily\gujaratifont[Script=Gujarati,AutoFakeBold=2.5,AutoFakeSlant=0.3]{Noto Sans Gujarati}
\setmainfont[Script=Gujarati,AutoFakeBold=2.5,AutoFakeSlant=0.3]{Noto Sans Gujarati}
% Use Noto Sans Gujarati for monospace to support Gujarati in text
\setmonofont[Scale=0.9]{Noto Sans Gujarati}

% Configure English to use the same font
\newfontfamily\englishfont[Script=Gujarati,AutoFakeBold=2.5,AutoFakeSlant=0.3]{Noto Sans Gujarati}

% Translations for polyglossia
\gappto\captionsgujarati{
  \renewcommand{\tablename}{કોષ્ટક}
  \renewcommand{\figurename}{આકૃતિ}
}

% Helper for TikZ nodes to ensure Gujarati font
\newcommand{\gu}[1]{{\gujaratifont #1}}

% Custom environments
\newtcolorbox{solutionbox}{
    breakable,
    enhanced,
    colback=solutioncolor!5!white,
    colframe=solutioncolor!75!black,
    fonttitle=\bfseries,
    title=જવાબ
}

\newtcolorbox{solutionboxnobreak}{
 colback=solutioncolor!5!white,
 colframe=solutioncolor!75!black,
 fonttitle=\bfseries,
 title=જવાબ
}

\newtcolorbox{keyformula}{
 breakable,
 enhanced,
 colback=keycolor!5!white,
 colframe=keycolor!75!black,
 fonttitle=\bfseries,
 title=રાસાયણિક સમીકરણ/સૂત્ર
}

\newtcolorbox{mnemonicbox}{
 breakable,
 enhanced,
 colback=mnemoniccolor!5!white,
 colframe=mnemoniccolor!75!black,
 fonttitle=\bfseries,
 title=મેમરી ટ્રીક
}


% Custom commands for GTU solutions
% This file defines semantic commands for consistent formatting

% Question command with automatic formatting
\newcommand{\question}[2]{%
  \section*{Question #1}%
  \textbf{#2}%
}

% OR question variant
\newcommand{\questionor}[2]{%
  \section*{Question #1 OR}%
  \textbf{#2}%
}

% Proper table environment with caption
\newenvironment{answertable}[1]{%
  \begin{table}[htbp]
  \centering
  \caption{#1}
}{%
  \end{table}
}

% Proper figure environment for diagrams
\newenvironment{answerdiagram}[1]{%
  \begin{figure}[htbp]
  \centering
  \caption{#1}
}{%
  \end{figure}
}

% Semantic markup for key terms
\newcommand{\keyword}[1]{\textbf{#1}}
\newcommand{\code}[1]{\texttt{#1}}
\newcommand{\classname}[1]{\texttt{#1}}
\newcommand{\methodname}[1]{\texttt{#1}}

% Proper quotation marks
\newcommand{\mnemonic}[1]{``#1''}

\usepackage{fontspec}
\newfontfamily\gujaratifont{Noto Sans Gujarati}

\title{Electronic Measurements \& Instruments (4331102) - Winter 2024 Solution (Gujarati)}
\date{December 05, 2024}

\begin{document}
\maketitle

\questionmarks{1(a)}{3}{Define following term: (1) Accuracy (2) Resolution (3) Error}

\begin{solutionbox}
\begin{center}
\captionof{table}{વ્યાખ્યાઓ (Definitions)}
\begin{tabulary}{\linewidth}{|L|L|}
\hline
\textbf{Term} & \textbf{Definition} \\ \hline
\textbf{Accuracy (ચોકસાઈ)} & માપેલ મૂલ્ય સાચા મૂલ્યની કેટલી નજીક છે તે \\ \hline
\textbf{Resolution (રિઝોલ્યુશન)} & ઇનપુટમાં થતો સૌથી નાનો ફેરફાર જે સાધન દ્વારા શોધી શકાય છે \\ \hline
\textbf{Error (ત્રુટિ)} & માપેલ મૂલ્ય અને સાચા મૂલ્ય વચ્ચેનો તફાવત \\ \hline
\end{tabulary}
\end{center}
\end{solutionbox}

\begin{mnemonicbox}
\mnemonic{ARE precise: Accuracy shows Reality, Error shows deviation, Resolution shows detail.}
\end{mnemonicbox}

\questionmarks{1(b)}{4}{Explain construction of unbounded strain gauge transducer with necessary diagram in detail. Also list application of it.}

\begin{solutionbox}
અનબાઉન્ડેડ સ્ટ્રેન ગેજમાં બારીક વાયરને ગ્રીડ પેટર્નમાં બેકિંગ મટિરિયલ સાથે જોડેલો હોય છે.

\begin{center}
\begin{tikzpicture}
    % Frame
    \draw[fill=gray!10] (0,0) rectangle (6,4);
    \node at (3,3.5) {\textguj{બેકિંગ મટિરિયલ/ફ્રેમ} (Backing Material)};
    
    % Pins
    \foreach \x in {1, 5} \foreach \y in {1, 3} \fill (\x,\y) circle (2pt);
    \node[left] at (1,3) {\textguj{સ્થિર} (Fixed)}; \node[right] at (5,3) {\textguj{સ્થિર} (Fixed)};
    \node[left] at (1,1) {\textguj{હલનચલન કરી શકે તેવું} (Movable)}; \node[right] at (5,1) {\textguj{હલનચલન કરી શકે તેવું} (Movable)};
    
    % Wire
    \draw[thick, red, decorate, decoration={coil, amplitude=2pt, segment length=5pt}] (1,3) -- (5,3) -- (5,1) -- (1,1) -- (1,3);
    \node at (3,2) [align=center] {\textguj{બારીક વાયર ગ્રીડ} (Fine Wire Grid)};
    
    % Leads
    \draw[thick, blue] (1,3) -- (1,4) node[above] {\textguj{લીડ 1} (Lead 1)};
    \draw[thick, blue] (5,3) -- (5,4) node[above] {\textguj{લીડ 2} (Lead 2)};
\end{tikzpicture}
\captionof{figure}{Unbounded Strain Gauge}
\end{center}

\begin{itemize}
    \item \textbf{રચનાના તત્વો (Construction elements)}: બારીક પ્રતિરોધક વાયરને ઇન્સ્યુલેટીંગ બેઝ મટિરિયલ પર આગળ-પાછળ લૂપ કરવામાં આવે છે.
    \item \textbf{કાર્ય સિદ્ધાંત (Working principle)}: જ્યારે તાણ (strain) અનુભવે છે ત્યારે તેનો પ્રતિરોધ (resistance) બદલાય છે.
    \item \textbf{ઉપયોગો (Applications)}: વજન માપન, પ્રેશર સેન્સર્સ, ફોર્સ સેન્સર્સ, સ્ટ્રક્ચરલ હેલ્થ મોનિટરિંગ.
\end{itemize}
\end{solutionbox}

\begin{mnemonicbox}
\mnemonic{WIRE Flexes: Wire grids Indicate Resistance changes from External stress.}
\end{mnemonicbox}

\questionmarks{1(c)}{7}{Explain working of Schering Bridge with circuit diagram for balance condition. List its advantages, disadvantages and applications.}

\begin{solutionbox}
શેરિંગ બ્રિજ એ AC બ્રિજ છે જેનો ઉપયોગ અજાણ્યા કેપેસીટન્સ (unknown capacitance) અને તેના ડિસીપેશન ફેક્ટર (dissipation factor) ને માપવા માટે થાય છે.

\begin{center}
\begin{circuitikz}[american, scale=0.8]
    \draw (0,3) to[C, l=$C_x$] (3,5) node[above] {B} to[C, l=$C_2$] (6,3) node[right] {C};
    \draw (6,3) to[R, l=$R_2$] (3,1) node[below] {D};
    \draw (3,1) to[R, l=$R_1$] (0,3);
    \draw (3,5) to[rmeter, t=D] (3,1);
    \draw (0,3) -- (-1,3) to[sV, l=AC] (-1,1) -- (3,1);
\end{circuitikz}
\captionof{figure}{Schering Bridge}
\end{center}

\textbf{સંતુલન શરત (Balance condition):}

\begin{center}
\captionof{table}{Balance Condition}
\begin{tabulary}{\linewidth}{|L|L|}
\hline
\textbf{સમીકરણ (Equation)} & \textbf{વર્ણન (Description)} \\ \hline
$C_x = C_2(R_2/R_1)$ & કેપેસીટન્સ ગણતરી માટે \\ \hline
$D_x = R_2(C_2/C_x)$ & ડિસીપેશન ફેક્ટર માટે \\ \hline
\end{tabulary}
\end{center}

\textbf{ફાયદા (Advantages):}
\begin{itemize}
    \item ઉચ્ચ ચોકસાઈ (High accuracy)
    \item કેપેસીટન્સનું સીધું વાંચન
    \item વિશાળ માપન શ્રેણી (Wide measurement range)
\end{itemize}

\textbf{ગેરફાયદા (Disadvantages):}
\begin{itemize}
    \item સાવચેતીપૂર્ણ શિલ્ડિંગની જરૂર છે
    \item આવર્તન આધારિત ભૂલો (Frequency dependent errors)
    \item સંતુલિત કરવા માટે જટિલ
\end{itemize}

\textbf{ઉપયોગો (Applications):}
\begin{itemize}
    \item કેપેસિટર ટેસ્ટિંગ
    \item ઇન્સ્યુલેશન ટેસ્ટિંગ
    \item ડાઇલેક્ટ્રિક મટિરિયલ મૂલ્યાંકન
\end{itemize}
\end{solutionbox}

\begin{mnemonicbox}
\mnemonic{SCUBA dive: Schering Calculates Unknown capacitance By Advanced circuit Designs In Various Equipment.}
\end{mnemonicbox}

\questionmarks{1(c) OR}{7}{Explain working of Maxwell's bridge with circuit diagram for balance condition. List its advantages, disadvantages, and applications.}

\begin{solutionbox}
મેક્સવેલ બ્રિજ (Maxwell's bridge) નો ઉપયોગ જાણીતા કેપેસીટન્સના સંદર્ભમાં અજાણ્યા ઇન્ડક્ટન્સ (unknown inductance) ને માપવા માટે થાય છે.

\begin{center}
\begin{circuitikz}[american, scale=0.8]
    \draw (0,3) node[left] {A} to[R, l=$R_1$] (3,6) node[above] {B} to[R, l=$R_3$] (6,3) node[right] {C};
    \draw (6,3) to[R, l=$R_2$] (6,1.5) -- (3,0) node[below] {D};
    \draw (6,3) -- (6,4) to[C, l=$C_4$] (3,0);
    \draw (0,3) to[L, l=$L_x$] (1.5,1.5) to[R, l=$R_x$] (3,0);
    \draw (3,6) to[rmeter, t=D] (3,0);
    \draw (0,3) -- (-1,3) to[sV, l=AC] (-1,0) -- (3,0);
\end{circuitikz}
\captionof{figure}{Maxwell's Bridge}
\end{center}

\textbf{સંતુલન શરત (Balance condition):}

\begin{center}
\captionof{table}{Balance Condition}
\begin{tabulary}{\linewidth}{|L|L|}
\hline
\textbf{સમીકરણ (Equation)} & \textbf{વર્ણન (Description)} \\ \hline
$L_x = C_4 \cdot R_2 \cdot R_3$ & ઇન્ડક્ટન્સ ગણતરી માટે \\ \hline
$R_x = R_1 \cdot (R_3/R_2)$ & અવરોધ ગણતરી માટે \\ \hline
\end{tabulary}
\end{center}

\textbf{ફાયદા (Advantages):}
\begin{itemize}
    \item આવર્તનથી સ્વતંત્ર (Independent of frequency)
    \item મધ્યમ Q કોઇલ માટે ઉચ્ચ ચોકસાઈ
    \item સંતુલિત કરવા માટે સરળ
\end{itemize}

\textbf{ગેરફાયદા (Disadvantages):}
\begin{itemize}
    \item નીચા Q કોઇલ માટે યોગ્ય નથી
    \item પ્રમાણભૂત કેપેસિટરની જરૂર છે
    \item મર્યાદિત શ્રેણી (Limited range)
\end{itemize}

\textbf{ઉપયોગો (Applications):}
\begin{itemize}
    \item ઇન્ડક્ટર્સ માપવા
    \item ઓડિયો ફ્રીક્વન્સી માપન
    \item ટ્રાન્સફોર્મર ટેસ્ટિંગ
\end{itemize}
\end{solutionbox}

\begin{mnemonicbox}
\mnemonic{MAGIC bridge: Maxwell Analyses Great Inductors by Comparing bridge Elements.}
\end{mnemonicbox}

\questionmarks{2(a)}{3}{Explain working of electronic multimeter with necessary diagram.}

\begin{solutionbox}
ઇલેક્ટ્રોનિક મલ્ટિમીટર માપન માટે વિવિધ વિદ્યુત પરિમાણોને પ્રમાણસર DC વોલ્ટેજમાં રૂપાંતરિત કરે છે.

\begin{center}
\begin{tikzpicture}[node distance=1.5cm, auto, scale=0.8, every node/.style={transform shape}]
    \node [gtu block] (Input) {\textguj{ઇનપુટ પસંદગી} (Input Selection)};
    \node [gtu block, right=of Input] (Atten) {\textguj{એટેન્યુએટર} (Attenuator)};
    \node [gtu block, right=of Atten] (Conv) {\textguj{કન્વર્ટર સર્કિટ} (Converter)};
    \node [gtu block, below=of Conv] (Amp) {\textguj{એમ્પ્લીફાયર} (Amplifier)};
    \node [gtu block, left=of Amp] (ADC) {ADC};
    \node [gtu block, left=of ADC] (Disp) {\textguj{ડિસ્પ્લે} (Display)};

    \draw [gtu arrow] (Input) -- (Atten);
    \draw [gtu arrow] (Atten) -- (Conv);
    \draw [gtu arrow] (Conv) -- (Amp);
    \draw [gtu arrow] (Amp) -- (ADC);
    \draw [gtu arrow] (ADC) -- (Disp);
\end{tikzpicture}
\captionof{figure}{Electronic Multimeter Block Diagram}
\end{center}

\begin{itemize}
    \item \textbf{સર્કિટ તત્વો (Circuit elements)}: ઇનપુટ સિલેક્ટર $\to$ એટેન્યુએટર $\to$ કન્વર્ટર $\to$ એમ્પ્લીફાયર $\to$ ADC $\to$ ડિસ્પ્લે
    \item \textbf{માપન પ્રકારો (Measurement types)}: DC વોલ્ટેજ, AC વોલ્ટેજ, કરંટ, અવરોધ
    \item \textbf{પાવર સ્ત્રોત (Power source)}: પોર્ટેબિલિટી અને સલામતી માટે બેટરી સંચાલિત
\end{itemize}
\end{solutionbox}

\begin{mnemonicbox}
\mnemonic{SACRED device: Signal Attenuated, Converted And Rectified for Electronic Display.}
\end{mnemonicbox}

\questionmarks{2(b)}{4}{Differentiate between Digital Voltmeter over Analog Voltmeter.}

\begin{solutionbox}
\begin{center}
\captionof{table}{ડિજિટલ વોલ્ટમીટર vs એનાલોગ વોલ્ટમીટર}
\begin{tabulary}{\linewidth}{|L|L|L|}
\hline
\textbf{પરિમાણ (Parameter)} & \textbf{ડિજિટલ વોલ્ટમીટર (Digital Voltmeter)} & \textbf{એનાલોગ વોલ્ટમીટર (Analog Voltmeter)} \\ \hline
\textbf{ડિસ્પ્લે પ્રકાર} & ન્યુમેરિક LCD/LED ડિસ્પ્લે & સ્કેલ પર ફરતો નિર્દેશક (Pointer) \\ \hline
\textbf{ચોકસાઈ (Accuracy)} & ઉચ્ચ ($\pm 0.1\%$ લાક્ષણિક) & ઓછી ($\pm 2-5\%$ લાક્ષણિક) \\ \hline
\textbf{વાંચન ભૂલો} & પેરેલેક્સ એરર (parallax error) નથી & પેરેલેક્સ એરર થવાની સંભાવના છે \\ \hline
\textbf{રિઝોલ્યુશન} & ઉચ્ચ (3-6 અંકો પ્રદર્શિત કરી શકે છે) & સ્કેલ વિભાગો દ્વારા મર્યાદિત \\ \hline
\textbf{ઇનપુટ ઇમ્પિડન્સ} & ખૂબ ઊંચું ($>10M\Omega$) & ઓછું ($20-200k\Omega/V$) \\ \hline
\textbf{પ્રતિસાદ સમય} & ધીમો સેમ્પલિંગ દર & ત્વરિત પ્રતિસાદ \\ \hline
\end{tabulary}
\end{center}
\end{solutionbox}

\begin{mnemonicbox}
\mnemonic{PARIOS: Parallax-free, Accurate, Resolution high, Impedance high, Observation digital, Sampling rate.}
\end{mnemonicbox}

\questionmarks{2(c)}{7}{Describe construction diagram of Energy meter and explain in detail.}

\begin{solutionbox}
એનર્જી મીટર કિલોવોટ-કલાક (kWh) માં સમય જતાં વિદ્યુત ઊર્જાના વપરાશને માપે છે.

\begin{center}
\begin{tikzpicture}
    % Construction
    \draw[fill=gray!20] (0,3) rectangle (4,4); \node at (2,3.5) {\textguj{વોલ્ટેજ કોઇલ} (Shunt Magnet)};
    \draw[fill=blue!10] (1,2) circle (1.2); \node at (1,2) {\textguj{ડિસ્ક} (Disc)};
    \draw[fill=gray!20] (0,0) rectangle (4,1); \node at (2,0.5) {\textguj{કરંટ કોઇલ} (Series Magnet)};
    \draw[fill=black] (3,2) rectangle (3.5, 2.5); \node at (3.25, 2.6) {\textguj{બ્રેક} (Brake)};
    \node[draw] at (2, 4.5) {\textguj{ગણતરી મિકેનિઝમ} (Counter)};
    \draw [->] (2, 2) -- (2, 4.5);
\end{tikzpicture}
\captionof{figure}{Energy Meter Construction}
\end{center}

\textbf{ઘટકો (Components):}
\begin{itemize}
    \item \textbf{વોલ્ટેજ કોઇલ}: વોલ્ટેજના પ્રમાણમાં ફ્લક્સ બનાવે છે
    \item \textbf{કરંટ કોઇલ}: કરંટના પ્રમાણમાં ફ્લક્સ બનાવે છે
    \item \textbf{એલ્યુમિનિયમ ડિસ્ક}: એડી કરંટને કારણે ફરે છે
    \item \textbf{ગણતરી મિકેનિઝમ (Counting mechanism)}: ડિસ્ક રોટેશનની નોંધણી કરે છે
    \item \textbf{કાયમી ચુંબક}: ડિસ્કની ગતિને નિયંત્રિત કરવા માટે બ્રેક તરીકે કામ કરે છે
\end{itemize}

\textbf{કાર્ય સિદ્ધાંત}: ડિસ્ક પરિભ્રમણ ગતિ પાવર વપરાશ ($V \times I \times \cos\Phi$) ના પ્રમાણસર છે.
\end{solutionbox}

\begin{mnemonicbox}
\mnemonic{VADCR meter: Voltage And current Drive Counter through Rotations.}
\end{mnemonicbox}

\questionmarks{2(a) OR}{3}{Explain working of clamp on Ammeter with necessary diagram.}

\begin{solutionbox}
ક્લેમ્પ-ઓન એમીટર ઇલેક્ટ્રોમેગ્નેટિક ઇન્ડક્શનનો ઉપયોગ કરીને સર્કિટ તોડ્યા વિના કરંટને માપે છે.

\begin{center}
\begin{tikzpicture}
    \draw[thick, fill=gray!30] (0,0) circle (1.5); \draw[fill=white] (0,0) circle (1);
    \node at (0, 1.25) {\textguj{ક્લેમ્પ જડબા} (Clamp Jaws)};
    \draw[ultra thick, blue] (0, -2) -- (0, 2); \node[left] at (0, 0) {\textguj{વાયર} (Wire)};
    \draw (1.5, 0) -- (3, 0); \draw (3, -1) rectangle (5, 1); \node at (4, 0) {\textguj{મીટર} (Meter)};
\end{tikzpicture}
\captionof{figure}{Clamp-on Ammeter}
\end{center}

\begin{itemize}
    \item \textbf{રચના (Construction)}: સેન્સિંગ કોઇલ સાથે સ્પ્લિટ ફેરાઇટ કોર.
    \item \textbf{કાર્ય સિદ્ધાંત}: કરંટ વહન કરતો વાયર ચુંબકીય ક્ષેત્ર બનાવે છે $\to$ સેન્સિંગ કોઇલમાં વોલ્ટેજ પ્રેરિત કરે છે.
    \item \textbf{ફાયદા}: નોન-કોન્ટેક્ટ મેઝરમેન્ટ, ઝડપી, સલામત.
\end{itemize}
\end{solutionbox}

\begin{mnemonicbox}
\mnemonic{CICS: Clamping Induces Current Signal.}
\end{mnemonicbox}

\questionmarks{2(b) OR}{4}{Differentiate between PMMC type Meter over Moving iron type Meter.}

\begin{solutionbox}
\begin{center}
\captionof{table}{PMMC vs Moving Iron}
\begin{tabulary}{\linewidth}{|L|L|L|}
\hline
\textbf{પરિમાણ (Parameter)} & \textbf{PMMC Type Meter} & \textbf{Moving Iron Type Meter} \\ \hline
\textbf{ઓપરેટિંગ સિદ્ધાંત} & ચુંબકીય ક્ષેત્ર ક્રિયાપ્રતિક્રિયા & ચુંબકીય આકર્ષણ/અપાકર્ષણ \\ \hline
\textbf{કરંટ પ્રકાર} & ફક્ત DC & બંને AC અને DC \\ \hline
\textbf{સ્કેલ (Scale)} & સમાન (Uniform) & અસમાન (છેડા પર ગીચ) \\ \hline
\textbf{ચોકસાઈ} & ઉચ્ચ ($\pm 0.5\%$ લાક્ષણિક) & ઓછી ($\pm 1-5\%$ લાક્ષણિક) \\ \hline
\textbf{ડેમ્પિંગ} & એડી કરંટ ડેમ્પિંગ & એર ફ્રિક્શન ડેમ્પિંગ \\ \hline
\textbf{પાવર વપરાશ} & ઓછો & વધારે \\ \hline
\end{tabulary}
\end{center}
\end{solutionbox}

\begin{mnemonicbox}
\mnemonic{PMMC is DAUPHIN: DC only, Accurate, Uniform scale, Power efficient, High sensitivity, Independent of frequency, Needs polarity.}
\end{mnemonicbox}

\questionmarks{2(c) OR}{7}{Draw the block diagram and Explain working of Integrating type DVM with necessary diagram and waveform.}

\begin{solutionbox}
ઇન્ટિગ્રેટિંગ પ્રકાર DVM ઉચ્ચ ચોકસાઈ માપન માટે ઇન્ટિગ્રેશન દ્વારા ઇનપુટ વોલ્ટેજને સમયમાં રૂપાંતરિત કરે છે.

\begin{center}
\begin{tikzpicture}[node distance=1.5cm, auto, scale=0.8, every node/.style={transform shape}]
    \node [gtu block] (Buffer) {\textguj{ઇનપુટ બફર}};
    \node [gtu block, right=of Buffer] (Integrator) {\textguj{ઇન્ટિગ્રેટર}};
    \node [gtu block, right=of Integrator] (Comparator) {\textguj{કમ્પેરેટર}};
    \node [gtu block, below=of Comparator] (Logic) {\textguj{કંટ્રોલ લોજિક}};
    \node [gtu block, right=of Logic] (Counter) {\textguj{કાઉન્ટર}};
    \node [gtu block, right=of Counter] (Display) {\textguj{ડિસ્પ્લે}};
    \node [gtu block, below=of Logic] (Clock) {\textguj{ક્લોક}};

    \draw [gtu arrow] (Buffer) -- (Integrator);
    \draw [gtu arrow] (Integrator) -- (Comparator);
    \draw [gtu arrow] (Comparator) -- (Logic);
    \draw [gtu arrow] (Logic) -- (Counter);
    \draw [gtu arrow] (Counter) -- (Display);
    \draw [gtu arrow] (Clock) -- (Logic);
    \draw [gtu arrow] (Logic) -| (Integrator) node[midway, above] {Reset};
\end{tikzpicture}
\captionof{figure}{Integrating DVM Block Diagram}
\end{center}

\textbf{કાર્ય સિદ્ધાંત (Working principle):}
\begin{itemize}
    \item ઇનપુટ વોલ્ટેજ નિશ્ચિત સમયગાળા માટે સંકલિત (integrated) થાય છે.
    \item ઇન્ટિગ્રેટરનું આઉટપુટ ઇનપુટના પ્રમાણમાં વધે છે.
    \item વિરુદ્ધ પોલેરિટી સાથે સંદર્ભ વોલ્ટેજ (Reference voltage) ઇન્ટિગ્રેટરને ડિસ્ચાર્જ કરે છે.
    \item ડિસ્ચાર્જ માટે લેવામાં આવેલો સમય ક્લોક પલ્સ (clock pulses) ગણીને માપવામાં આવે છે.
    \item કાઉન્ટ એ ઇનપુટ વોલ્ટેજના પ્રમાણસર છે.
\end{itemize}
\end{solutionbox}

\begin{mnemonicbox}
\mnemonic{DIRT meter: Direct Integration Relates Time to measure voltage.}
\end{mnemonicbox}

\questionmarks{3(a)}{3}{Differentiate between CRO over DSO.}

\begin{solutionbox}
\begin{center}
\captionof{table}{CRO vs DSO}
\begin{tabulary}{\linewidth}{|L|L|L|}
\hline
\textbf{પરિમાણ} & \textbf{CRO (Analog Oscilloscope)} & \textbf{DSO (Digital Storage Oscilloscope)} \\ \hline
\textbf{સિગ્નલ પ્રોસેસિંગ} & સમગ્ર એનાલોગ & ADC કન્વર્ઝન પછી ડિજિટલ \\ \hline
\textbf{સ્ટોરેજ ક્ષમતા} & વેવફોર્મ્સ સ્ટોર કરી શકતું નથી & બહુવિધ વેવફોર્મ્સ સ્ટોર કરી શકે છે \\ \hline
\textbf{બેન્ડવિડ્થ} & સામાન્ય રીતે ઓછી & ઉચ્ચ (GHz થી વધી શકે છે) \\ \hline
\textbf{ટ્રિગરિંગ} & મૂળભૂત ટ્રિગર વિકલ્પો & અદ્યતન ટ્રિગર ક્ષમતાઓ \\ \hline
\textbf{વિશ્લેષણ સુવિધાઓ} & મર્યાદિત & વ્યાપક (FFT, માપન) \\ \hline
\end{tabulary}
\end{center}
\end{solutionbox}

\begin{mnemonicbox}
\mnemonic{PASSED: Processing-Analog/digital, Storage-none/yes, Signal-raw/processed, Easy-basic/advanced, Display-phosphor/digital.}
\end{mnemonicbox}

\questionmarks{3(b)}{4}{Explain CRO Screen.}

\begin{solutionbox}
CRO સ્ક્રીન વિદ્યુત સંકેતો પ્રદર્શિત કરે છે અને તેમાં ઘણા મહત્વપૂર્ણ ઘટકો હોય છે.

\begin{center}
\begin{tikzpicture}
    \draw[fill=black!80] (0,0) rectangle (6,4);
    \draw[green, thick] (0,2) -- (6,2);
    \draw[green, thick] (3,0) -- (3,4);
    \node[green] at (3,2) {+};
    \node[white] at (3, 4.5) {\textguj{ફોસ્ફર સ્ક્રીન} (Phosphor Screen)};
\end{tikzpicture}
\captionof{figure}{CRO Screen Graticule}
\end{center}

\textbf{ઘટકો (Components):}
\begin{itemize}
    \item \textbf{ફોસ્ફર કોટિંગ}: ઇલેક્ટ્રોન દ્વારા અથડાય ત્યારે પ્રકાશ ફેંકે છે.
    \item \textbf{ગ્રેટીક્યુલ (Graticule)}: માપન સંદર્ભ માટે ગ્રીડ લાઇન.
    \item \textbf{સ્કેલ (Scales)}: વોલ્ટેજ/સમય માટે કેલિબ્રેટેડ નિશાનો.
\end{itemize}
\end{solutionbox}

\begin{mnemonicbox}
\mnemonic{PGSCR: Phosphor Glows when Struck, Creating Representation.}
\end{mnemonicbox}

\questionmarks{3(c)}{7}{Explain Block diagram, working and advantage of CRO with necessary diagram.}

\begin{solutionbox}
CRO (Cathode Ray Oscilloscope) ઇલેક્ટ્રિકલ સિગ્નલોને વેવફોર્મ તરીકે વિઝ્યુઅલાઈઝ કરે છે.

\begin{center}
\begin{tikzpicture}[node distance=1.2cm, auto, scale=0.7, every node/.style={transform shape}]
    \node [gtu block] (Input) {Vertical Input};
    \node [gtu block, right=of Input] (Atten) {Vert. Attenuator};
    \node [gtu block, right=of Atten] (Amp) {Vert. Amp};
    \node [gtu block, right=of Amp] (PlatesV) {Vertical Plates};
    \node [gtu block, below=of Input] (Trig) {Trigger Circuit};
    \node [gtu block, right=of Trig] (TimeBase) {Time Base};
    \node [gtu block, right=of TimeBase] (HorAmp) {Horiz. Amp};
    \node [gtu block, right=of HorAmp] (PlatesH) {Horiz. Plates};
    \node [gtu block, below=of Trig] (Power) {Power Supply};
    \node [gtu block, right=of Power] (Gun) {Electron Gun};
    \node [circle, draw, minimum size=1cm, right=of PlatesV] (CRT) {CRT};

    \draw [gtu arrow] (Input) -- (Atten);
    \draw [gtu arrow] (Atten) -- (Amp);
    \draw [gtu arrow] (Amp) -- (PlatesV);
    \draw [gtu arrow] (Trig) -- (TimeBase);
    \draw [gtu arrow] (TimeBase) -- (HorAmp);
    \draw [gtu arrow] (HorAmp) -- (PlatesH);
    \draw [gtu arrow] (Power) -- (Gun);
    \draw [gtu arrow] (Gun) -| (CRT);
    \draw [gtu arrow] (PlatesV) -- (CRT);
    \draw [gtu arrow] (PlatesH) -- (CRT);
\end{tikzpicture}
\captionof{figure}{CRO Block Diagram}
\end{center}

\textbf{કાર્ય સિદ્ધાંત (Working principle):}
\begin{itemize}
    \item \textbf{ઇલેક્ટ્રોન ગન}: ઇલેક્ટ્રોન બીમ બનાવે છે.
    \item \textbf{વર્ટિકલ સિસ્ટમ}: ઇનપુટ સિગ્નલના પ્રમાણમાં Y-axis ડિફ્લેક્શનને નિયંત્રિત કરે છે.
    \item \textbf{હોરીઝોન્ટલ સિસ્ટમ}: બીમને સ્ક્રીન પર સતત દરે સ્વીપ કરે છે.
    \item \textbf{ટ્રિગર સર્કિટ}: ઇનપુટ સિગ્નલ સાથે હોરીઝોન્ટલ સ્વીપને સિંક્રનાઇઝ કરે છે.
    \item \textbf{CRT}: ફોસ્ફર સ્ક્રીન પર ઇલેક્ટ્રોન બીમની હિલચાલ દર્શાવે છે.
\end{itemize}

\textbf{ફાયદા (Advantages):}
\begin{itemize}
    \item રીઅલ-ટાઇમ સિગ્નલ ડિસ્પ્લે
    \item વાઈડ બેન્ડવિડ્થ (Wide bandwidth)
    \item ઉચ્ચ ઇનપુટ ઇમ્પિડન્સ
\end{itemize}
\end{solutionbox}

\begin{mnemonicbox}
\mnemonic{EARTH view: Electron beam Amplification Reveals Time-based Horizontal view.}
\end{mnemonicbox}


\questionmarks{3(a) OR}{3}{Apply Lissajous pattern for frequency measurement and Phase angle measurement.}

\begin{solutionbox}
જ્યારે CRO ના X અને Y ઇનપુટ્સ પર બે સાઈન વેવ્સ (sine waves) લાગુ કરવામાં આવે છે ત્યારે લિસાજેસ પેટર્ન (Lissajous patterns) બનાવવામાં આવે છે.

\begin{center}
\captionof{table}{Lissajous Measurements}
\begin{tabulary}{\linewidth}{|L|L|L|}
\hline
\textbf{પેટર્ન પ્રકાર (Pattern Type)} & \textbf{માપન સૂત્ર (Measurement Formula)} \\ \hline
\textbf{ફ્રીક્વન્સી માપન (Frequency Measurement)} & $f_x/f_y = n_y/n_x$ (Tangent ratio) \\ \hline
\textbf{ફેઝ એન્ગલ માપન (Phase Angle Measurement)} & $\sin(\phi) = A/B$ (Intercept/Max Height) \\ \hline
\end{tabulary}
\end{center}

\begin{center}
\begin{tikzpicture}
    % Frequency (2:1)
    \begin{scope}[shift={(0,0)}]
        \draw[thick, samples=100, domain=0:2*pi] plot ({sin(2*\x r)}, {sin(\x r)});
        \node at (0, -1.5) {Frequency Ratio (2:1)};
        \draw[red, dashed] (-1.2, 0.5) -- (1.2, 0.5); 
        \draw[blue, dashed] (0.5, -1.2) -- (0.5, 1.2);
    \end{scope}
    
    % Phase
    \begin{scope}[shift={(4,0)}]
        \draw[thick, rotate=45] (0,0) ellipse (0.7 and 1.2);
        \draw[->] (-1.5,0) -- (1.5,0);
        \draw[->] (0,-1.5) -- (0,1.5);
        \node at (0, -1.5) {Phase Shift};
        \draw[dashed] (0, 0.5) -- (1, 0.5) node[right] {A};
        \draw[dashed] (0, 1.2) -- (1, 1.2) node[right] {B};
    \end{scope}
\end{tikzpicture}
\captionof{figure}{Lissajous Patterns}
\end{center}

\begin{itemize}
    \item \textbf{ફ્રીક્વન્સી રેશિયો}: ઊભી સ્પર્શરેખા બિંદુઓ / આડી સ્પર્શરેખા બિંદુઓની ગણતરી કરો.
    \item \textbf{ફેઝ માપન}: $\sin(\phi) = A/B$ જ્યાં A એ શૂન્ય ક્રોસિંગ પર પેટર્નની ઊંચાઈ છે, B એ મહત્તમ ઊંચાઈ છે.
    \item \textbf{ઉપયોગો}: સિગ્નલ સરખામણી, ફ્રીક્વન્સી કેલિબ્રેશન.
\end{itemize}
\end{solutionbox}

\begin{mnemonicbox}
\mnemonic{LIPS patterns: Lissajous Indicates Phase and Sine frequency.}
\end{mnemonicbox}

\questionmarks{3(b) OR}{4}{Explain Graticules in CRO. Also Explain its types.}

\begin{solutionbox}
ગ્રેટીક્યુલ્સ (Graticules) એ CRO સ્ક્રીન પર સંદર્ભ ગ્રીડ છે જે વેવફોર્મ પેરામીટર્સના માપનમાં મદદ કરે છે.

\begin{center}
\begin{tikzpicture}
    \draw[step=1cm, gray!50, thin] (0,0) grid (6,4);
    \draw[thick] (0,0) rectangle (6,4);
    \draw[thick] (3,0) -- (3,4);
    \draw[thick] (0,2) -- (6,2);
    \foreach \x in {0,1,2,3,4,5,6} \draw (\x, 1.9) -- (\x, 2.1);
    \foreach \y in {0,1,2,3,4} \draw (2.9, \y) -- (3.1, \y);
\end{tikzpicture}
\captionof{figure}{CRO Graticule}
\end{center}

\textbf{ગ્રેટીક્યુલ્સના પ્રકાર (Types of graticules):}

\begin{center}
\captionof{table}{Graticule Types}
\begin{tabulary}{\linewidth}{|L|L|L|}
\hline
\textbf{પ્રકાર (Type)} & \textbf{વર્ણન (Description)} & \textbf{ઉપયોગ (Application)} \\ \hline
\textbf{આંતરિક ગ્રેટીક્યુલ (Internal)} & CRT ની અંદર કોતરવામાં આવે છે & પેરેલેક્સ એરર દૂર કરે છે \\ \hline
\textbf{બાહ્ય ગ્રેટીક્યુલ (External)} & અલગ પારદર્શક પ્લેટ & સરળ રિપ્લેસમેન્ટ \\ \hline
\textbf{ઇલેક્ટ્રોનિક ગ્રેટીક્યુલ} & ઇલેક્ટ્રોનિકલી જનરેટ થાય છે & ડિજિટલ ઓસિલોસ્કોપ \\ \hline
\textbf{ખાસ હેતુ (Special purpose)} & ચોક્કસ માપન માટે કસ્ટમ નિશાનો & વિશિષ્ટ પરીક્ષણ \\ \hline
\end{tabulary}
\end{center}
\end{solutionbox}

\begin{mnemonicbox}
\mnemonic{GRIT: Graticules Render Important Time-voltage measurements.}
\end{mnemonicbox}

\questionmarks{3(c) OR}{7}{Describe Block diagram, working and advantage of Digital storage oscilloscope (DSO).}

\begin{solutionbox}
ડિજિટલ સ્ટોરેજ ઓસિલોસ્કોપ (DSO) સ્ટોરેજ, પ્રોસેસિંગ અને ડિસ્પ્લે માટે સિગ્નલોને ડિજિટાઈઝ કરે છે.

\begin{center}
\begin{tikzpicture}[node distance=1.2cm, auto, scale=0.7, every node/.style={transform shape}]
    \node [gtu block] (Input) {Input Signal};
    \node [gtu block, right=of Input] (Atten) {Attenuator};
    \node [gtu block, right=of Atten] (ADC) {ADC};
    \node [gtu block, right=of ADC] (Mem) {Memory};
    \node [gtu block, right=of Mem] (Proc) {Processor};
    \node [gtu block, right=of Proc] (DAC) {DAC};
    \node [gtu block, right=of DAC] (Display) {Display};
    \node [gtu block, below=of Proc] (Control) {Control Panel};
    \node [gtu block, below=of ADC] (TimeBase) {Time Base};
    \node [gtu block, below=of Memory] (Trigger) {Trigger};

    \draw [gtu arrow] (Input) -- (Atten);
    \draw [gtu arrow] (Atten) -- (ADC);
    \draw [gtu arrow] (ADC) -- (Mem);
    \draw [gtu arrow] (Mem) -- (Proc);
    \draw [gtu arrow] (Proc) -- (DAC);
    \draw [gtu arrow] (DAC) -- (Display);
    \draw [gtu arrow] (Control) -- (Proc);
    \draw [gtu arrow] (TimeBase) -- (ADC);
    \draw [gtu arrow] (Trigger) -| (Proc);
\end{tikzpicture}
\captionof{figure}{DSO Block Diagram}
\end{center}

\textbf{કાર્ય સિદ્ધાંત (Working principle):}
\begin{itemize}
    \item \textbf{એક્વિઝિશન (Acquisition)}: ADC દ્વારા સિગ્નલ ઊંચા દરે સેમ્પલ કરવામાં આવે છે.
    \item \textbf{સ્ટોરેજ (Storage)}: મેમરીમાં સંગ્રહિત ડિજિટલ મૂલ્યો.
    \item \textbf{પ્રોસેસિંગ (Processing)}: ડિજિટલ સિગ્નલ પ્રોસેસિંગ વિશ્લેષણ વધારે છે.
    \item \textbf{ડિસ્પ્લે (Display)}: સ્ક્રીન પર પુનઃનિર્મિત સિગ્નલ બતાવવામાં આવે છે.
    \item \textbf{ટ્રિગરિંગ (Triggering)}: અદ્યતન ડિજિટલ ટ્રિગરિંગ વિકલ્પો.
\end{itemize}

\textbf{ફાયદા (Advantages):}
\begin{itemize}
    \item સિગ્નલ સ્ટોરેજ ક્ષમતા (Signal storage capability)
    \item પ્રી-ટ્રિગર જોવા (Pre-trigger viewing)
    \item વન-શોટ સિગ્નલ કેપ્ચર (One-shot signal capture)
    \item અદ્યતન માપન (Advanced measurements) (FFT, ગણતરી)
\end{itemize}
\end{solutionbox}

\begin{mnemonicbox}
\mnemonic{SAMPLE: Storage And Memory Preserves Long-term Events.}
\end{mnemonicbox}

\questionmarks{4(a)}{3}{Differentiate RTD and Thermistor.}

\begin{solutionbox}
\begin{center}
\captionof{table}{RTD vs Thermistor}
\begin{tabulary}{\linewidth}{|L|L|L|}
\hline
\textbf{પરિમાણ (Parameter)} & \textbf{RTD (Resistance Temperature Detector)} & \textbf{Thermistor} \\ \hline
\textbf{સામગ્રી (Material)} & પ્લેટિનમ, નિકલ, કોપર & મેટલ ઓક્સાઇડ, સેમિકન્ડક્ટર્સ \\ \hline
\textbf{R-T સંબંધ} & રેખીય (Linear), પોઝિટિવ કોએફિશિયન્ટ & બિન-રેખીય (Non-linear), સામાન્ય રીતે નેગેટિવ \\ \hline
\textbf{તાપમાન શ્રેણી} & -200°C થી +850°C & -50°C થી +300°C \\ \hline
\textbf{સંવેદનશીલતા} & ઓછી (0.00385 $\Omega/\Omega/^\circ$C લાક્ષણિક) & ઉચ્ચ (3-5\% પ્રતિ $^\circ$C લાક્ષણિક) \\ \hline
\textbf{ચોકસાઈ} & ઉચ્ચ & ઓછી \\ \hline
\textbf{પ્રતિસાદ સમય} & ધીમો & ઝડપી \\ \hline
\end{tabulary}
\end{center}
\end{solutionbox}

\begin{mnemonicbox}
\mnemonic{RTD is PLAINS: Platinum, Linear, Accurate, Industrial range, Narrow sensitivity, Stable.}
\end{mnemonicbox}

\questionmarks{4(b)}{4}{Explain Optical encoder with its output waveform.}

\begin{solutionbox}
ઓપ્ટિકલ એન્કોડર મિકેનિકલ ગતિને કોડેડ ડિસ્ક દ્વારા પ્રકાશના વિક્ષેપનો ઉપયોગ કરીને ડિજિટલ પલ્સમાં રૂપાંતરિત કરે છે.

\begin{center}
\begin{tikzpicture}
    % Block diagram style
    \node[draw, fill=yellow!20] (Light) at (0,3) {\textguj{પ્રકાશ સ્રોત} (Light)};
    \node[draw, cylinder, shape border rotate=90, aspect=0.2, minimum height=1cm, minimum width=2cm, fill=gray!20] (Disc) at (0,1.5) {\textguj{કોડ ડિસ્ક} (Disc)};
    \node[draw, fill=green!20] (Detector) at (0,0) {\textguj{ફોટોડિટેક્ટર} (Detector)};
    \draw[->, thick, wave] (Light) -- (Disc);
    \draw[->, thick, wave] (Disc) -- (Detector);
    \draw[->] (Detector) -- (0,-1) node[below] {Output Signal};
    \node at (2,1.5) {$\leftarrow$ Motion};
    
    % Waveforms
    \begin{scope}[shift={(4,0)}]
        \draw (0,3) node[left] {Ch A};
        \draw[thick] (0,2.5) -- (0.5,2.5) -- (0.5,3) -- (1,3) -- (1,2.5) -- (1.5,2.5) -- (1.5,3) -- (2,3);
        
        \draw (0,1.5) node[left] {Ch B};
        \draw[thick] (0,1.5) -- (0.25,1.5) -- (0.25,2) -- (0.75,2) -- (0.75,1.5) -- (1.25,1.5) -- (1.25,2) -- (1.75,2);
        
        \draw[<->] (0.25, 1) -- (0.5, 1) node[midway, below] {90$^\circ$};
    \end{scope}
\end{tikzpicture}
\captionof{figure}{Optical Encoder \& Waveforms}
\end{center}

\begin{itemize}
    \item \textbf{ઘટકો (Components)}: પ્રકાશ સ્રોત, કોડેડ ડિસ્ક, ફોટોડિટેક્ટર.
    \item \textbf{પ્રકારો (Types)}: ઇન્ક્રીમેન્ટલ (પલ્સ) અથવા એબ્સોલ્યુટ (યુનિક પોઝિશન કોડ).
    \item \textbf{ઉપયોગો (Applications)}: પોઝિશન માપન, ઝડપ શોધ, ગતિ નિયંત્રણ.
\end{itemize}
\end{solutionbox}

\begin{mnemonicbox}
\mnemonic{DROPS: Disc Rotation Outputs Pulse Signals.}
\end{mnemonicbox}

\questionmarks{4(c)}{7}{Describe Thermocouple with working principle, types and application.}

\begin{solutionbox}
થર્મોકોપલ (Thermocouple) એ તાપમાન સેન્સર છે જે સીબેક ઇફેક્ટ (Seebeck effect) પર કાર્ય કરે છે, જે તાપમાનના તફાવતના પ્રમાણમાં વોલ્ટેજ ઉત્પન્ન કરે છે.

\begin{center}
\begin{tikzpicture}
    \draw[thick, red] (0,0) -- (4,0);
    \draw[thick, blue] (0,0) -- (0,2) -- (4,2) -- (4,0);
    \fill[red] (0,0) circle (3pt) node[left] {Hot Junction};
    \fill[blue] (4,0) circle (3pt) node[right] {Cold Junction};
    \node at (2, 2.3) {Metal A};
    \node at (2, -0.3) {Metal B};
    \draw (4,0) -- (5,0) to[rmeter, t=V] (5,2) -- (4,2);
\end{tikzpicture}
\captionof{figure}{Thermocouple Principle}
\end{center}

\textbf{કાર્ય સિદ્ધાંત (Working principle):}
\begin{itemize}
    \item એક છેડે જોડાયેલી બે અસમાન ધાતુઓ (હોટ જંકશન)
    \item હોટ અને કોલ્ડ જંકશન વચ્ચેનો તાપમાન તફાવત વોલ્ટેજ પેદા કરે છે.
    \item વોલ્ટેજ તાપમાનના તફાવતના પ્રમાણમાં હોય છે (સીબેક ઇફેક્ટ).
\end{itemize}

\textbf{થર્મોકોપલના પ્રકારો (Types):}

\begin{center}
\captionof{table}{Thermocouple Types}
\begin{tabulary}{\linewidth}{|L|L|L|L|}
\hline
\textbf{પ્રકાર} & \textbf{સામગ્રી (Materials)} & \textbf{તાપમાન શ્રેણી} & \textbf{ઉપયોગ} \\ \hline
\textbf{K} & ક્રોમેલ-એલ્યુમેલ & -200°C થી +1350°C & સામાન્ય હેતુ \\ \hline
\textbf{J} & આયર્ન-કોન્સ્ટેન્ટન & -40°C થી +750°C & રિડ્યુસિંગ એટમોસ્ફિયર \\ \hline
\textbf{E} & ક્રોમેલ-કોન્સ્ટેન્ટન & -200°C થી +900°C & ક્રાયોજેનિક, ઉચ્ચ આઉટપુટ \\ \hline
\textbf{T} & કોપર-કોન્સ્ટેન્ટન & -250°C થી +350°C & નીચું તાપમાન, ખોરાક \\ \hline
\textbf{R/S} & પ્લેટિનમ-રોડિયમ & 0°C થી +1700°C & ઉચ્ચ તાપમાન, લેબ \\ \hline
\end{tabulary}
\end{center}

\textbf{ઉપયોગો (Applications):} ઔદ્યોગિક ભઠ્ઠીઓ, એન્જિન, રાસાયણિક પ્રક્રિયા, ફૂડ પ્રોસેસિંગ, સંશોધન.
\end{solutionbox}

\begin{mnemonicbox}
\mnemonic{SHOVE theory: Seebeck Hot-cold Output Voltage Equals Temperature.}
\end{mnemonicbox}

\questionmarks{4(a) OR}{3}{Differentiate active and passive transducers.}

\begin{solutionbox}
\begin{center}
\captionof{table}{Active vs Passive Transducers}
\begin{tabulary}{\linewidth}{|L|L|L|}
\hline
\textbf{પરિમાણ} & \textbf{એક્ટિવ ટ્રાન્સડ્યુસર્સ} & \textbf{પેસિવ ટ્રાન્સડ્યુસર્સ} \\ \hline
\textbf{ઊર્જા રૂપાંતરણ} & ભૌતિક જથ્થાને સીધા વિદ્યુત આઉટપુટમાં રૂપાંતરિત કરે છે & બાહ્ય પાવર સ્ત્રોતની જરૂર છે \\ \hline
\textbf{આઉટપુટ સિગ્નલ} & સ્વ-ઉત્પન્ન (Self-generating) & બાહ્ય ઊર્જાને મોડ્યુલેટ કરે છે \\ \hline
\textbf{ઉદાહરણો} & થર્મોકોપલ, પીઝોઇલેક્ટ્રિક, ફોટોવોલ્ટેઇક & RTD, સ્ટ્રેન ગેજ, LVDT \\ \hline
\textbf{સંવેદનશીલતા} & સામાન્ય રીતે ઓછી & સામાન્ય રીતે વધારે \\ \hline
\textbf{પાવર જરૂરિયાત} & બાહ્ય પાવરની જરૂર નથી & બાહ્ય પાવર જરૂરી છે \\ \hline
\end{tabulary}
\end{center}
\end{solutionbox}

\begin{mnemonicbox}
\mnemonic{SIMPLE difference: Self-powered Is Main Principle of Leading Energy transducers.}
\end{mnemonicbox}

\questionmarks{4(b) OR}{4}{Explain Capacitive Transducer with necessary diagram in detail. Also list application of it.}

\begin{solutionbox}
કેપેસીટીવ ટ્રાન્સડ્યુસર ભૌતિક વિસ્થાપનને કારણે કેપેસીટન્સમાં ફેરફારના સિદ્ધાંત પર કામ કરે છે.

\begin{center}
\begin{tikzpicture}
    \draw[thick] (0,0) -- (4,0); \node at (2,-0.3) {Fixed Plate};
    \draw[thick] (0,1) -- (4,1); \node at (2,1.3) {Movable Plate};
    \draw[<->] (4.5,0) -- (4.5,1) node[midway, right] {d};
    \draw[dashed] (1,0) rectangle (3,1); \node at (2,0.5) {Dielectric};
    \draw (0,0) -- (-1,0) to[C] (-1,1) -- (0,1);
\end{tikzpicture}
\captionof{figure}{Capacitive Transducer}
\end{center}

\textbf{કાર્ય સિદ્ધાંત (Working principle):}
\begin{itemize}
    \item કેપેસીટન્સ $C = \epsilon_0\epsilon_r A/d$
    \item વિસ્તાર (A), અંતર (d), અથવા ડાઇલેક્ટ્રિક કોન્સ્ટન્ટ ($\epsilon_r$) માં ફેરફાર સાથે બદલાય છે.
    \item ડિસ્પ્લેસમેન્ટ કેપેસીટન્સ બદલે છે.
\end{itemize}

\textbf{ઉપયોગો (Applications):} પ્રેશર માપન, લિક્વિડ લેવલ સેન્સિંગ, હ્યુમિડિટી સેન્સર્સ, એક્સિલરોમીટર્સ.
\end{solutionbox}

\begin{mnemonicbox}
\mnemonic{CADAP: Capacitance Alters with Distance, Area, or Permittivity.}
\end{mnemonicbox}

\questionmarks{4(c) OR}{7}{Explain LVDT Transducer operation, construction with necessary diagram in detail. Also list advantage, disadvantage and application of LVDT.}

\begin{solutionbox}
LVDT (Linear Variable Differential Transformer) રેખીય વિસ્થાપન (linear displacement) ને વિદ્યુત આઉટપુટમાં રૂપાંતરિત કરે છે.

\begin{center}
\begin{circuitikz}[american, scale=0.8]
    \draw (0,0) node[inputarrow] {} to[L, l=Pri] (0,2);
    \draw (2, -1) to[L, l=Sec1] (2,0.8);
    \draw (2, 1.2) to[L, l=Sec2] (2,3);
    \draw[thick] (0.5, 0.5) rectangle (1.5, 1.5); \node at (1,1) {Core};
    \draw[<->] (1, 1.6) -- (1, 2.5) node[above] {Displacement};
    \node at (0, -0.5) {AC Input};
    \node at (3, 1) {Output $V_o = V_{s1} - V_{s2}$};
\end{circuitikz}
\captionof{figure}{LVDT}
\end{center}

\textbf{રચના (Construction):} કેન્દ્રમાં પ્રાથમિક કોઇલ, બે ગૌણ કોઇલ, મૂવેબલ ફેરોમેગ્નેટિક કોર.

\textbf{કાર્ય (Operation):}
\begin{itemize}
    \item AC એક્સાઇટેશન પ્રાથમિક કોઇલને ઊર્જા આપે છે.
    \item કોર સ્થિતિ સેકન્ડરી સાથે મેગ્નેટિક કપલિંગ નક્કી કરે છે.
    \item વિસ્થાપન (displacement) ના પ્રમાણમાં ડિફરન્શિયલ વોલ્ટેજ આઉટપુટ.
\end{itemize}

\textbf{ફાયદા (Advantages):} નોન-કોન્ટેક્ટ, અનંત રિઝોલ્યુશન, ઉચ્ચ લીનિયરિટી, મજબૂત construction.

\textbf{ગેરફાયદા (Disadvantages):} AC ની જરૂર છે, કદમાં મોટું (Bulky), ચુંબકીય સંવેદનશીલતા.

\textbf{ઉપયોગો (Applications):} પ્રિસિઝન માપન, હાઇડ્રોલિક સિસ્ટમ્સ, એરક્રાફ્ટ કંટ્રોલ્સ.
\end{solutionbox}

\begin{mnemonicbox}
\mnemonic{CDPOS sensor: Core Displacement Produces Output Signal.}
\end{mnemonicbox}

\questionmarks{5(a)}{3}{Demonstrate working and principle of Semiconductor Temperature Sensor LM35.}

\begin{solutionbox}
LM35 એ IC તાપમાન સેન્સર છે જે સેલ્સિયસમાં તાપમાનના સપ્રમાણમાં વોલ્ટેજ આઉટપુટ આપે છે.

\begin{center}
\begin{tikzpicture}
    \draw (0,0) rectangle (2,3);
    \node at (1,1.5) {LM35};
    \draw (0.5,0) -- (0.5,-1) node[below] {VCC};
    \draw (1,0) -- (1,-1) node[below] {Output};
    \draw (1.5,0) -- (1.5,-1) node[below] {GND};
\end{tikzpicture}
\captionof{figure}{LM35 Pinout}
\end{center}

\textbf{કાર્ય સિદ્ધાંત (Working principle):}
\begin{itemize}
    \item બિલ્ટ-ઇન તાપમાન-સેન્સિંગ તત્વ સાથે સંકલિત સર્કિટ (Integrated circuit).
    \item લીનિયર આઉટપુટ વોલ્ટેજ: $+10mV/^\circ$C
    \item સેલ્સિયસમાં સીધું કેલિબ્રેટેડ.
    \item ઓપરેટિંગ રેન્જ: -55°C થી +150°C
\end{itemize}
\end{solutionbox}

\begin{mnemonicbox}
\mnemonic{TEN mV TRICK: Temperature Escalation Noted in milliVolts: Ten Rise Indicates Celsius Kelvin.}
\end{mnemonicbox}

\questionmarks{5(b)}{4}{Describe working of Harmonic distortion analyzer with necessary diagram.}

\begin{solutionbox}
હાર્મોનિક ડિસ્ટોર્શન વિશ્લેષક સિગ્નલ ગુણવત્તા નક્કી કરવા માટે હાર્મોનિક સામગ્રીને માપે છે.

\begin{center}
\begin{tikzpicture}[node distance=1.2cm, auto, scale=0.7, every node/.style={transform shape}]
    \node [gtu block] (Input) {Input};
    \node [gtu block, right=of Input] (Atten) {Attenuator};
    \node [gtu block, right=of Atten] (Notch) {Notch Filter};
    \node [gtu block, right=of Notch] (Amp) {Amplifier};
    \node [gtu block, right=of Amp] (RMS) {RMS Detector};
    \node [gtu block, right=of RMS] (Display) {Display};
    
    \draw [gtu arrow] (Input) -- (Atten);
    \draw [gtu arrow] (Atten) -- (Notch);
    \draw [gtu arrow] (Notch) -- (Amp);
    \draw [gtu arrow] (Amp) -- (RMS);
    \draw [gtu arrow] (RMS) -- (Display);
\end{tikzpicture}
\captionof{figure}{Harmonic Distortion Analyzer}
\end{center}

\textbf{કાર્ય સિદ્ધાંત (Working principle):}
\begin{itemize}
    \item ફંડામેન્ટલ ફ્રીક્વન્સી નોચ ફિલ્ટરનો ઉપયોગ કરીને ફિલ્ટર કરવામાં આવે છે.
    \item બાકીના હાર્મોનિક્સ માપવામાં આવે છે.
    \item THD = (VRMS of harmonics)/(VRMS of fundamental)
\end{itemize}
\end{solutionbox}

\begin{mnemonicbox}
\mnemonic{FRONT analysis: Filter Removes Original Note Totally for Analyzing Leftover Signals.}
\end{mnemonicbox}

\questionmarks{5(c)}{7}{Describe working of Spectrum Analyzer with necessary diagram in detail.}

\begin{solutionbox}
સ્પેક્ટ્રમ વિશ્લેષક (Spectrum Analyzer) સિગ્નલ એમ્પ્લીટ્યુડ વિરુદ્ધ ફ્રીક્વન્સી પ્રદર્શિત કરે છે.

\begin{center}
\begin{tikzpicture}[node distance=1.2cm, auto, scale=0.7, every node/.style={transform shape}]
    \node [gtu block] (Input) {RF Input};
    \node [gtu block, right=of Input] (Mixer) {Mixer};
    \node [gtu block, right=of Mixer] (IF) {IF Filter};
    \node [gtu block, right=of IF] (Det) {Detector};
    \node [gtu block, right=of Det] (Disp) {Display};
    \node [gtu block, below=of Mixer] (LO) {Local Osc};
    \node [gtu block, below=of Disp] (Sweep) {Sweep Gen};
    
    \draw [gtu arrow] (Input) -- (Mixer);
    \draw [gtu arrow] (Mixer) -- (IF);
    \draw [gtu arrow] (IF) -- (Det);
    \draw [gtu arrow] (Det) -- (Disp);
    \draw [gtu arrow] (LO) -- (Mixer);
    \draw [gtu arrow] (Sweep) -- (LO);
    \draw [gtu arrow] (Sweep) -| (Disp);
\end{tikzpicture}
\captionof{figure}{Spectrum Analyzer}
\end{center}

\textbf{કાર્ય સિદ્ધાંત (Working principle):}
\begin{itemize}
    \item \textbf{Superheterodyne}: LO સાથે મિશ્રિત ઇનપુટ.
    \item \textbf{Sweep}: LO આવર્તન શ્રેણીમાં સ્વીપ કરે છે.
    \item \textbf{Display}: ફ્રીક્વન્સી ડોમેન સ્પેક્ટ્રમ બતાવે છે.
\end{itemize}

\textbf{ઉપયોગો (Applications):} સિગ્નલ વિશ્લેષણ, EMI પરીક્ષણ, હાર્મોનિક વિશ્લેષણ.
\end{solutionbox}

\begin{mnemonicbox}
\mnemonic{SAFER view: Sweep Analyzes Frequencies for Examining RF.}
\end{mnemonicbox}

\questionmarks{5(a) OR}{3}{Explain analog transducer and digital transducer. Also explain primary transducer and secondary transducer.}

\begin{solutionbox}
\begin{center}
\captionof{table}{Transducer Types}
\begin{tabulary}{\linewidth}{|L|L|}
\hline
\textbf{પ્રકાર (Type)} & \textbf{વર્ણન (Description)} \\ \hline
\textbf{એનાલોગ (Analog)} & સતત (continuous) આઉટપુટ સિગ્નલ ઉત્પન્ન કરે છે \\ \hline
\textbf{ડિજિટલ (Digital)} & ડિસ્ક્રીટ/બાઈનરી આઉટપુટ સિગ્નલ ઉત્પન્ન કરે છે \\ \hline
\textbf{પ્રાથમિક (Primary)} & ભૌતિક જથ્થાને સીધા જ યાંત્રિક/વિદ્યુત સિગ્નલમાં ફેરવે છે \\ \hline
\textbf{ગૌણ (Secondary)} & પ્રાથમિક ટ્રાન્સડ્યુસરના આઉટપુટને બીજા સ્વરૂપમાં ફેરવે છે \\ \hline
\end{tabulary}
\end{center}
\end{solutionbox}

\begin{mnemonicbox}
\mnemonic{PADS: Primary And Digital/analog Secondary.}
\end{mnemonicbox}

\questionmarks{5(b) OR}{4}{Explain working of Digital IC tester with necessary diagram in detail.}

\begin{solutionbox}
ડિજિટલ IC ટેસ્ટર સંકલિત સર્કિટ (integrated circuits) ની કાર્યક્ષમતાને ચકાસે છે.

\begin{center}
\begin{tikzpicture}[node distance=1.5cm, auto, scale=0.7, every node/.style={transform shape}]
    \node [gtu block] (Micro) {Microcontroller};
    \node [gtu block, right=of Micro] (Pattern) {Pattern Gen};
    \node [gtu block, right=of Pattern] (Socket) {Test Socket};
    \node [gtu block, below=of Pattern] (Analyzer) {Result Analyzer};
    
    \draw [gtu arrow] (Micro) -- (Pattern);
    \draw [gtu arrow] (Pattern) -- (Socket);
    \draw [gtu arrow] (Socket) |- (Analyzer);
    \draw [gtu arrow] (Analyzer) -| (Micro);
\end{tikzpicture}
\captionof{figure}{Digital IC Tester}
\end{center}

\textbf{કાર્ય સિદ્ધાંત (Working principle):}
\begin{itemize}
    \item સોકેટમાં IC પર ટેસ્ટ પેટર્ન લાગુ કરે છે.
    \item અપેક્ષિત પરિણામો સાથે આઉટપુટની તુલના કરે છે.
    \item પાસ/ફેલ દર્શાવે છે.
\end{itemize}
\end{solutionbox}

\begin{mnemonicbox}
\mnemonic{TRIG test: Test, Run patterns, Identify faults, Generate report.}
\end{mnemonicbox}

\questionmarks{5(c) OR}{7}{Explain working of function generator with necessary diagram in detail.}

\begin{solutionbox}
ફંક્શન જનરેટર પરીક્ષણ માટે વિવિધ વેવફોર્મ્સ બનાવે છે.

\begin{center}
\begin{tikzpicture}[node distance=1.2cm, auto, scale=0.7, every node/.style={transform shape}]
    \node [gtu block] (Freq) {Freq Control};
    \node [gtu block, right=of Freq] (Osc) {Oscillator};
    \node [gtu block, right=of Osc] (Shaper) {Wave Shaper};
    \node [gtu block, right=of Shaper] (Amp) {Amp Control};
    \node [gtu block, right=of Amp] (Out) {Output Amp};
    \node [gtu block, right=of Out] (O) {Output};
    
    \draw [gtu arrow] (Freq) -- (Osc);
    \draw [gtu arrow] (Osc) -- (Shaper);
    \draw [gtu arrow] (Shaper) -- (Amp);
    \draw [gtu arrow] (Amp) -- (Out);
    \draw [gtu arrow] (Out) -- (O);
\end{tikzpicture}
\captionof{figure}{Function Generator}
\end{center}

\textbf{વેવફોર્મ્સ (Waveforms):} Sine, Square, Triangle, Ramp.

\textbf{ઉપયોગો (Applications):} એમ્પ્લીફાયરનું પરીક્ષણ, સંદર્ભ સંકેતો, શૈક્ષણિક ડેમો.
\end{solutionbox}

\begin{mnemonicbox}
\mnemonic{SWATOR: Sine Wave And Triangle OSCillator Renders signals.}
\end{mnemonicbox}

\end{document}
