\documentclass{article}

% content/resources/templates/preamble.tex
\usepackage[margin=0.6in]{geometry}
\author{Milav Dabgar}
\usepackage{amsmath,amssymb,amsthm}
\usepackage{booktabs}
\usepackage{multirow}
\usepackage{xcolor}
\usepackage{tcolorbox}
\tcbuselibrary{breakable,skins}
\usepackage[colorlinks=true,linkcolor=blue]{hyperref}
\usepackage{titlesec}
\usepackage{enumitem}
\usepackage{tikz}
\usepackage{pgfplots}
\usepackage{circuitikz}
\usepackage[version=4]{mhchem}
\usepackage{longtable}
\usepackage{array}
\usepackage{float}
\usepackage{caption}
\usepackage{listings}

\lstset{
  basicstyle=\small\ttfamily,
  breaklines=true,
  breakatwhitespace=false,
  postbreak=\mbox{\textcolor{red}{$\hookrightarrow$}\space},
  float=false,
  numbers=left,
  numberstyle=\tiny\color{gray},
  numbersep=10pt,
  xleftmargin=2em,
  keywordstyle=\color{blue},
  commentstyle=\color{green!60!black},
  stringstyle=\color{purple},
  backgroundcolor=\color{gray!5},
  showstringspaces=false,
  tabsize=2,
  captionpos=b,
  keepspaces=true,
  columns=flexible
}

\pgfplotsset{compat=1.18}
\usetikzlibrary{shapes,arrows,positioning,calc,patterns,decorations.pathmorphing,decorations.markings,arrows.meta}

% Color scheme
\definecolor{headcolor}{RGB}{0,102,204}
\definecolor{keycolor}{RGB}{220,20,60}
\definecolor{solutioncolor}{RGB}{34,139,34}
\definecolor{mnemoniccolor}{RGB}{148,0,211}
\definecolor{codecolor}{RGB}{0,0,100}

% Spacing
\setlength{\parskip}{3pt}
\setlist[itemize]{nosep}
\setlist[enumerate]{nosep}

% Title formatting
\titleformat{\section}{\Large\bfseries\color{headcolor}}{\thesection}{1em}{}
\titleformat{\subsection}{\large\bfseries\color{headcolor}}{\thesubsection}{1em}{}

% Pandoc tightlist compatibility
\providecommand{\tightlist}{%
  \setlength{\itemsep}{0pt}\setlength{\parskip}{0pt}}

% Pandoc longtable compatibility
\newcounter{none}
\def\thenone{}


% content/resources/templates/gujarati-boxes.tex
\usepackage{fontspec}
\usepackage{polyglossia}

% Set Gujarati as main language (document is primarily in Gujarati)
% Note: gloss-gujarati.ldf doesn't exist in polyglossia, but it will use hyphenation patterns
\setdefaultlanguage{gujarati}
\setotherlanguage{english}

% Configure Gujarati font properly
% Use Language=Default to prevent polyglossia from trying to add language-specific features
% that don't exist for Gujarati, which causes "empty feature" warnings
\newfontfamily\gujaratifont[Script=Gujarati,AutoFakeBold=2.5,AutoFakeSlant=0.3]{Noto Sans Gujarati}
\setmainfont[Script=Gujarati,AutoFakeBold=2.5,AutoFakeSlant=0.3]{Noto Sans Gujarati}
% Use Noto Sans Gujarati for monospace to support Gujarati in text
\setmonofont[Scale=0.9]{Noto Sans Gujarati}

% Configure English to use the same font
\newfontfamily\englishfont[Script=Gujarati,AutoFakeBold=2.5,AutoFakeSlant=0.3]{Noto Sans Gujarati}

% Translations for polyglossia
\gappto\captionsgujarati{
  \renewcommand{\tablename}{કોષ્ટક}
  \renewcommand{\figurename}{આકૃતિ}
}

% Helper for TikZ nodes to ensure Gujarati font
\newcommand{\gu}[1]{{\gujaratifont #1}}

% Custom environments
\newtcolorbox{solutionbox}{
    breakable,
    enhanced,
    colback=solutioncolor!5!white,
    colframe=solutioncolor!75!black,
    fonttitle=\bfseries,
    title=જવાબ
}

\newtcolorbox{solutionboxnobreak}{
 colback=solutioncolor!5!white,
 colframe=solutioncolor!75!black,
 fonttitle=\bfseries,
 title=જવાબ
}

\newtcolorbox{keyformula}{
 breakable,
 enhanced,
 colback=keycolor!5!white,
 colframe=keycolor!75!black,
 fonttitle=\bfseries,
 title=રાસાયણિક સમીકરણ/સૂત્ર
}

\newtcolorbox{mnemonicbox}{
 breakable,
 enhanced,
 colback=mnemoniccolor!5!white,
 colframe=mnemoniccolor!75!black,
 fonttitle=\bfseries,
 title=મેમરી ટ્રીક
}


% Custom commands for GTU solutions
% This file defines semantic commands for consistent formatting

% Question command with automatic formatting
\newcommand{\question}[2]{%
  \section*{Question #1}%
  \textbf{#2}%
}

% OR question variant
\newcommand{\questionor}[2]{%
  \section*{Question #1 OR}%
  \textbf{#2}%
}

% Proper table environment with caption
\newenvironment{answertable}[1]{%
  \begin{table}[htbp]
  \centering
  \caption{#1}
}{%
  \end{table}
}

% Proper figure environment for diagrams
\newenvironment{answerdiagram}[1]{%
  \begin{figure}[htbp]
  \centering
  \caption{#1}
}{%
  \end{figure}
}

% Semantic markup for key terms
\newcommand{\keyword}[1]{\textbf{#1}}
\newcommand{\code}[1]{\texttt{#1}}
\newcommand{\classname}[1]{\texttt{#1}}
\newcommand{\methodname}[1]{\texttt{#1}}

% Proper quotation marks
\newcommand{\mnemonic}[1]{``#1''}


\title{ઇલેક્ટ્રોનિક મેઝરમેન્ટ્સ એન્ડ ઇન્સ્ટ્રુમેન્ટ્સ (4331102) - વિન્ટર 2023 સોલ્યુશન}
\date{મે 20, 2024}

\begin{document}
\maketitle

\questionmarks{1(a)}{3}{એક્યુરેસી, રીપ્રોડ્યુસીબિબિટી અને રિપીટેબિલિટી ની વ્યાખ્યા આપો.}

\begin{solutionbox}
\begin{center}
\captionof{table}{વ્યાખ્યાઓ}
\begin{tabulary}{\linewidth}{|L|L|}
\hline
\textbf{પદ} & \textbf{વ્યાખ્યા} \\ \hline
\textbf{એક્યુરેસી} & માપવામાં આવતા જથ્થાના સાચા મૂલ્યની નજીકતા \\ \hline
\textbf{રીપ્રોડ્યુસીબિલિટી} & જુદી જુદી શરતો (વિવિધ ઓપરેટરો, સ્થાનો, સમય) હેઠળ માપવામાં આવે ત્યારે સમાન ઇનપુટ માટે સમાન માપ આપવાની સાધનની ક્ષમતા \\ \hline
\textbf{રિપીટેબિલિટી} & સમાન પરિસ્થિતિઓમાં વારંવાર માપવામાં આવે ત્યારે સમાન ઇનપુટ માટે સમાન માપ આપવાની સાધનની ક્ષમતા \\ \hline
\end{tabulary}
\end{center}
\end{solutionbox}

\begin{mnemonicbox}
\mnemonic{ARR - Accurate Results Repeatedly}
\end{mnemonicbox}

\questionmarks{1(b)}{4}{વ્હીટસ્ટોન બ્રિજની આકૃતિ દોરી અને સમજાવો.}

\begin{solutionbox}
\textbf{વ્હીટસ્ટોન બ્રિજ} અજ્ઞાત રેસિસ્ટન્સના ચોક્કસ માપન માટે વપરાય છે.

\textbf{સર્કિટ ડાયાગ્રામ}:
\begin{center}
\begin{circuitikz}[american, scale=0.8]
    \draw (0,3) node[left] {A} to[R, l=$R_1$] (3,5) node[above] {B} to[R, l=$R_3$] (6,3) node[right] {C};
    \draw (6,3) to[R, l=$R_4$] (3,1) node[below] {D} to[R, l=$R_2$] (0,3);
    \draw (3,5) to[rmeter, t=G] (3,1);
    \draw (0,3) -- (-0.5,3) -- (-0.5, 0.5) to[battery1] (6.5, 0.5) -- (6.5, 3) -- (6,3);
\end{circuitikz}
\captionof{figure}{વ્હીટસ્ટોન બ્રિજ}
\end{center}

\begin{center}
\captionof{table}{મુખ્ય લક્ષણો}
\begin{tabulary}{\linewidth}{|L|L|}
\hline
\textbf{લક્ષણ} & \textbf{વર્ણન} \\ \hline
\textbf{રચના} & હીરા આકારમાં જોડાયેલા ચાર રેસિસ્ટર \\ \hline
\textbf{સંતુલન શરત} & $R_1/R_2 = R_3/R_4$ (જ્યારે આઉટપુટ વોલ્ટેજ શૂન્ય હોય) \\ \hline
\textbf{ઉપયોગ} & અજ્ઞાત રેસિસ્ટન્સનું ચોક્કસ માપન \\ \hline
\textbf{કાર્યપદ્ધતિ} & એક હાથમાં અજ્ઞાત રેસિસ્ટર મૂકવામાં આવે છે, બ્રિજ સંતુલિત થાય ત્યાં સુધી બાકીના રેસિસ્ટર ગોઠવવામાં આવે છે \\ \hline
\end{tabulary}
\end{center}
\end{solutionbox}

\begin{mnemonicbox}
\mnemonic{WBMP - When Balanced, Measure Precisely}
\end{mnemonicbox}

\questionmarks{1(c)}{7}{Q મીટરનો સિદ્ધાંત સમજાવો. અને સાથે સાથે પ્રેક્ટીકલ Q મીટરની આકૃતિ દોરી અને સમજાવો.}

\begin{solutionbox}
\textbf{Q મીટરનો સિદ્ધાંત}:
Q-મીટર \keyword{સીરીઝ રેઝોનન્સ} ના સિદ્ધાંત પર કાર્ય કરે છે, જ્યાં Q ફેક્ટર રેઝોનન્સ પર કેપેસિટર વોલ્ટેજ અને લાગુ વોલ્ટેજના ગુણોત્તર તરીકે માપવામાં આવે છે.

\textbf{બ્લોક ડાયાગ્રામ}:
\begin{center}
\begin{tikzpicture}[node distance=1.5cm, auto]
    \node [gtu block] (Osc) {RF ઓસિલેટર};
    \node [gtu block, right=of Osc] (Coil) {વર્ક કોઇલ};
    \node [gtu block, right=of Coil] (Series) {સીરીઝ સર્કિટ};
    \node [gtu block, below=of Series] (L) {ઇન્ડક્ટર L};
    \node [gtu block, left=of L] (C) {વેરિએબલ C};
    \node [gtu block, left=of C] (VTVM) {VTVM};
    \node [gtu block, left=of VTVM] (Scale) {Q-સ્કેલ};

    \draw [gtu arrow] (Osc) -- (Coil);
    \draw [gtu arrow] (Coil) -- (Series);
    \draw [gtu arrow] (Series) -- (L);
    \draw [gtu arrow] (L) -- (C);
    \draw [gtu arrow] (C) -- (VTVM);
    \draw [gtu arrow] (VTVM) -- (Scale);
\end{tikzpicture}
\captionof{figure}{પ્રેક્ટીકલ Q મીટર}
\end{center}

\begin{center}
\captionof{table}{ઘટકો}
\begin{tabulary}{\linewidth}{|L|L|}
\hline
\textbf{ઘટક} & \textbf{કાર્ય} \\ \hline
\textbf{RF ઓસિલેટર} & વેરિએબલ ફ્રિક્વન્સી સિગ્નલ પ્રદાન કરે છે \\ \hline
\textbf{વર્ક કોઇલ} & ટેસ્ટ સર્કિટ સાથે સિગ્નલ જોડે છે \\ \hline
\textbf{રેઝોનન્સ સર્કિટ} & વેરિએબલ કેપેસિટર C સાથે સીરીઝમાં ટેસ્ટ ઇન્ડક્ટર L \\ \hline
\textbf{VTVM} & કેપેસિટર પર વોલ્ટેજ માપે છે \\ \hline
\textbf{Q-સ્કેલ} & સીધા Q મૂલ્ય વાંચવા માટે કેલિબ્રેટેડ થયેલ છે \\ \hline
\end{tabulary}
\end{center}

\begin{itemize}
    \item \textbf{રેઝોનન્સ ફોર્મ્યુલા}: $f = \frac{1}{2\pi\sqrt{LC}}$
    \item \textbf{Q ગણતરી}: $Q = \frac{V_c}{V_s}$ (કેપેસિટર વોલ્ટેજ / સોર્સ વોલ્ટેજ)
\end{itemize}
\end{solutionbox}

\begin{mnemonicbox}
\mnemonic{RIVQ - Resonance Indicates Valuable Quality}
\end{mnemonicbox}

\questionmarks{1(c) OR}{7}{મુવિંગ કોઈલ ટાઈપ ઇન્સ્ટ્રુમેન્ટની રચના દોરો અનેસમજાવો.}

\begin{solutionbox}
\textbf{રચના ડાયાગ્રામ}:
\begin{center}
\begin{tikzpicture}
    % Magnets
    \draw [fill=gray!30] (-2, -1) rectangle (-1, 1); \node at (-1.5, 0) {N};
    \draw [fill=gray!30] (1, -1) rectangle (2, 1); \node at (1.5, 0) {S};
    % Core
    \draw (0,0) circle (0.7); \node at (0,0) {કોર};
    % Coil
    \draw [thick, color=red] (-0.8, -0.8) rectangle (0.8, 0.8); \node at (0, -0.5) [below] {કોઇલ};
    % Pointer
    \draw [thick, ->] (0, 0.8) -- (1.5, 2); \node at (1.6, 2) [right] {પોઇન્ટર};
    % Scale
    \draw (0.5, 2) arc (60:120:2);
    \foreach \x in {60, 75, 90, 105, 120} \draw (\x:2.1) -- (\x:2.2);
    % Spring
    \draw [decorate, decoration={coil, segment length=3pt, amplitude=2pt}] (0,0) -- (0, 0.8); \node at (-0.2, 0.4) [left] {સ્પ્રિંગ};
\end{tikzpicture}
\captionof{figure}{PMMC રચના}
\end{center}

\begin{center}
\captionof{table}{રચના વિગતો}
\begin{tabulary}{\linewidth}{|L|L|}
\hline
\textbf{ઘટક} & \textbf{વર્ણન} \\ \hline
\textbf{કાયમી ચુંબક} & મજબૂત ચુંબકીય ક્ષેત્ર બનાવે છે \\ \hline
\textbf{મૂવિંગ કોઇલ} & એલ્યુમિનિયમ ફ્રેમ પર ઘાયેલ હળવી કોઇલ \\ \hline
\textbf{સ્પ્રિંગ્સ} & નિયંત્રિત ટોર્ક અને વિદ્યુત જોડાણો પ્રદાન કરે છે \\ \hline
\textbf{પોઇન્ટર} & કોઇલ સાથે જોડાયેલ છે, કેલિબ્રેટેડ સ્કેલ પર ફરે છે \\ \hline
\textbf{કોર} & ચુંબકીય પ્રવાહને કેન્દ્રિત કરવા માટે સોફ્ટ આયર્ન કોર \\ \hline
\end{tabulary}
\end{center}
\end{solutionbox}

\begin{mnemonicbox}
\mnemonic{MAPS-C: Magnet Acts, Pointer Shows Current}
\end{mnemonicbox}

\questionmarks{2(a)}{3}{અલગ અલગ પ્રકારની એરરની યાદી બનાવો અને કોઈપણ બે સમજાવો.}

\begin{solutionbox}
\begin{center}
\captionof{table}{એરરના પ્રકારો}
\begin{tabulary}{\linewidth}{|L|}
\hline
\textbf{ગ્રોસ એરર} (Gross Errors) \\ \hline
\textbf{સિસ્ટેમેટિક એરર} (Systematic Errors) \\ \hline
\textbf{રેન્ડમ એરર} (Random Errors) \\ \hline
\textbf{તંત્રાત્મક એરર} (Instrumental Errors) \\ \hline
\textbf{પર્યાવરણીય એરર} (Environmental Errors) \\ \hline
\end{tabulary}
\end{center}

\textbf{સમજૂતી}:
\begin{enumerate}
    \item \keyword{સિસ્ટેમેટિક એરર}: વાસ્તવિક મૂલ્યથી સુસંગત વિચલન. ઇન્સ્ટ્રુમેન્ટ કેલિબ્રેશન અથવા ડિઝાઇનમાં ખામીને કારણે થાય છે.
    \item \keyword{રેન્ડમ એરર}: માપનમાં અણધારી ભિન્નતા. ઘોંઘાટ અથવા પર્યાવરણીય વધઘટને કારણે થાય છે.
\end{enumerate}
\end{solutionbox}

\begin{mnemonicbox}
\mnemonic{GSREL - Good Systems Reduce Error Levels}
\end{mnemonicbox}

\questionmarks{2(b)}{4}{મેક્સવેલ બ્રિજ દોરો અને સમજાવો.}

\begin{solutionbox}
\textbf{મેક્સવેલ બ્રિજ} પ્રમાણભૂત કેપેસિટર સાથે સરખામણી કરીને ઇન્ડક્ટન્સ માપે છે.

\textbf{સર્કિટ ડાયાગ્રામ}:
\begin{center}
\begin{circuitikz}[american, scale=0.8]
    \draw (0,3) node[left] {A} to[R, l=$R_1$] (3,5) node[above] {B} to[R, l=$R_3$] (6,3) node[right] {C};
    \draw (0,0) coordinate (D) to[R, l=$R_2$] (0,3) coordinate (A) to[R, l=$R_1$] (3,3) coordinate (B);
    \draw (0,3) to[C, l=$C_1$,, color=blue] (3,3); 
    \draw (3,3) to[R, l=$R_3$] (3,0) coordinate (C);
    \draw (0,0) to[L, l=$L_x$] (1.5,0) to[R, l=$R_x$] (3,0);
    \draw (0,3) -- (-1,3) to[sV, l=AC] (-1,0) -- (0,0);
    \draw (0,1.5) to[rmeter, t=D] (3,1.5); 
\end{circuitikz} 
\captionof{figure}{મેક્સવેલ બ્રિજ}
\end{center}

\begin{center}
\captionof{table}{ઘટકો}
\begin{tabulary}{\linewidth}{|L|L|}
\hline
\textbf{ઘટક} & \textbf{કાર્ય} \\ \hline
$L_x$ & અજ્ઞાત ઇન્ડક્ટર \\ \hline
$C_1$ & સ્ટાન્ડર્ડ કેપેસિટર \\ \hline
$R_1, R_2, R_3$ & ચોકસાઈ રેઝિસ્ટર \\ \hline
\end{tabulary}
\end{center}

\begin{itemize}
    \item \textbf{સૂત્ર}: $L_x = R_2 R_3 C_1$.
    \item \textbf{ઉપયોગ}: મધ્યમ Q કોઇલ માપવા માટે.
\end{itemize}
\end{solutionbox}

\begin{mnemonicbox}
\mnemonic{MBLR - Maxwell Bridge Links Resistance}
\end{mnemonicbox}

\questionmarks{2(c)}{7}{મુવિંગ આયર્ન ટાઈપ ઇન્સ્ટ્રુમેન્ટની રચના દોરો અનેસમજાવો.}

\begin{solutionbox}
\textbf{રચના ડાયાગ્રામ}:
\begin{center}
\begin{tikzpicture}
    % Coil
    \draw (0,0) circle (1.5); 
    \node at (0, 1.8) {ફિક્સ કોઇલ};
    
    % Fixed Vane
    \draw [thick, fill=gray] (-0.3, -0.5) rectangle (0.3, 0.5); \node at (0.6, 0.5) {ફિક્સ વેન};
    
    % Moving Vane
    \draw [thick, fill=gray, rotate=30] (0.3, -0.5) rectangle (0.7, 0.5); 
    \draw [thick, ->] (0.5, 0.2) -- (1.5, 1.5); \node at (1.8, 1.5) {પોઇન્ટર};
    
    % Scale
    \draw (0, 1.5) arc (60:120:3);
\end{tikzpicture}
\captionof{figure}{મૂવિંગ આયર્ન (રીપલ્શન પ્રકાર)}
\end{center}

\begin{center}
\captionof{table}{રચના વિગતો}
\begin{tabulary}{\linewidth}{|L|L|}
\hline
\textbf{કોઇલ} & વર્તમાન તે માપે છે વહન કરે છે \\ \hline
\textbf{આયર્ન વેન્સ} & બે સોફ્ટ આયર્ન ટુકડાઓ (એક ફિક્સ, એક મૂવિંગ) \\ \hline
\textbf{પોઇન્ટર} & મૂવિંગ વેન સાથે જોડાયેલ \\ \hline
\end{tabulary}
\end{center}

\begin{itemize}
    \item \textbf{સિદ્ધાંત}: બંને લોખંડના ટુકડા સમાન ધ્રુવીયતા સાથે ચુંબકિત થાય છે, જેના કારણે અપાકર્ષણ થાય છે.
    \item \textbf{લાભ}: AC અને DC બંને માટે વપરાય છે.
\end{itemize}
\end{solutionbox}

\begin{mnemonicbox}
\mnemonic{IRAM - Iron Repulsion Activates Movement}
\end{mnemonicbox}

\questionmarks{2(a) OR}{3}{બેસિક ડીસી વોલ્ટમીટર સમજાવો.}

\begin{solutionbox}
\textbf{DC વોલ્ટમીટર} એ શ્રેણીમાં ઉચ્ચ રેઝિસ્ટન્સ સાથેનું PMMC મીટર છે.

\textbf{સર્કિટ}:
\begin{center}
\begin{circuitikz}[american]
    \draw (0,0) to[R, l=$R_s$ (મલ્ટિપ્લાયર)] (3,0) to[rmeter, l=$R_m$, t=PMMC] (5,0);
    \draw (0,0) to[open, v=$V_{in}$] (5,0);
\end{circuitikz}
\captionof{figure}{બેસિક DC વોલ્ટમીટર}
\end{center}

\begin{itemize}
    \item \textbf{મલ્ટિપ્લાયર}: વર્તમાન મર્યાદિત કરવા માટે ઉચ્ચ મૂલ્ય શ્રેણી રેઝિસ્ટર.
    \item \textbf{ગણતરી}: $R_s = \frac{V}{I_m} - R_m$.
\end{itemize}
\end{solutionbox}

\begin{mnemonicbox}
\mnemonic{SVM - Series Voltage Measurement}
\end{mnemonicbox}

\questionmarks{2(b) OR}{4}{શેરિંગ બ્રિજ દોરો અને સમજાવો.}

\begin{solutionbox}
\textbf{શેરિંગ બ્રિજ} કેપેસિટન્સ માપવા માટે વપરાય છે.

\textbf{સર્કિટ ડાયાગ્રામ}:
\begin{center}
\begin{circuitikz}[american, scale=0.8]
    \draw (0,3) to[C, l=$C_1$] (1.5, 3) to[R, l=$R_1$] (3,3) coordinate (B); 
    \draw (0,0) coordinate (D) to[C, l=$C_4$] (0,3) coordinate (A);
    \draw (3,3) to[R, l=$R_3$] (3,0) coordinate (C);
    \draw (0,0) to[C, l=$C_2$] (3,0); 
    \draw (0,0.5) to[R, l=$R_2$] (3,0.5);
    \draw (0,3) -- (-1,3) to[sV, l=AC] (-1,0) -- (0,0);
    \draw (B) to[rmeter, t=D] (D);
\end{circuitikz}
\captionof{figure}{શેરિંગ બ્રિજ}
\end{center}

\begin{itemize}
    \item \textbf{સંતુલન}: $C_1 = C_4 \frac{R_2}{R_3}$.
    \item \textbf{ઉપયોગ}: ડાઇલેક્ટ્રિક લોસ માપન.
\end{itemize}
\end{solutionbox}

\begin{mnemonicbox}
\mnemonic{SCDR - Schering Capacitance Determines Resistance}
\end{mnemonicbox}

\questionmarks{2(c) OR}{7}{ઇલેક્ટ્રોનિક મલ્ટીમીટર ઉપર ટૂંકનોંધ લખો.}

\begin{solutionbox}
\textbf{ઇલેક્ટ્રોનિક મલ્ટિમીટર} ઉચ્ચ ઇનપુટ ઇમ્પિડન્સ પ્રદાન કરવા માટે ઇલેક્ટ્રોનિક સર્કિટ (એમ્પ્લીફાયર) નો ઉપયોગ કરે છે.

\textbf{બ્લોક ડાયાગ્રામ}:
\begin{center}
\begin{tikzpicture}[node distance=1.5cm, auto]
    \node [gtu block] (Att) {એટેન્યુએટર};
    \node [gtu block, left=of Att] (In) {ઇનપુટ};
    \node [gtu block, right=of Att] (Conv) {કન્વર્ટર};
    \node [gtu block, right=of Conv] (Amp) {એમ્પ્લિફાયર};
    \node [gtu block, right=of Amp] (Rect) {રેક્ટિફાયર};
    \node [gtu block, right=of Rect] (Disp) {ડિસ્પ્લે};

    \draw [gtu arrow] (In) -- (Att);
    \draw [gtu arrow] (Att) -- (Conv);
    \draw [gtu arrow] (Conv) -- (Amp);
    \draw [gtu arrow] (Amp) -- (Rect);
    \draw [gtu arrow] (Rect) -- (Disp);
\end{tikzpicture}
\captionof{figure}{ઇલેક્ટ્રોનિક મલ્ટિમીટર}
\end{center}
\end{solutionbox}

\begin{mnemonicbox}
\mnemonic{VCAR-D: Voltage, Current And Resistance - Displayed}
\end{mnemonicbox}

\questionmarks{3(a)}{3}{CRO ના અલગ અલગ પ્રોબ્સ સમજાવો.}

\begin{solutionbox}
\begin{center}
\captionof{table}{પ્રોબના પ્રકારો}
\begin{tabulary}{\linewidth}{|L|L|}
\hline
\textbf{પ્રકાર} & \textbf{વર્ણન} \\ \hline
\textbf{પેસિવ પ્રોબ (1X)} & સીધો જોડાણ પ્રોબ \\ \hline
\textbf{પેસિવ પ્રોબ (10X)} & સિગ્નલને 10 ના પરિબળ દ્વારા ઘટાડે છે \\ \hline
\textbf{એક્ટિવ પ્રોબ} & સક્રિય ઘટકો (FETs) ધરાવે છે \\ \hline
\textbf{કરંટ પ્રોબ} & ચુંબકીય ક્ષેત્ર સંવેદના દ્વારા પ્રવાહ માપે છે \\ \hline
\end{tabulary}
\end{center}
\end{solutionbox}

\begin{mnemonicbox}
\mnemonic{PAC-S: Probes Allow Circuit Sensing}
\end{mnemonicbox}

\questionmarks{3(b)}{4}{ક્લેમ્પોન મીટરની રચના દોરો અને સમજાવો.}

\begin{solutionbox}
\textbf{ક્લેમ્પ મીટર} સર્કિટ તોડ્યા વિના AC પ્રવાહ માપે છે.

\textbf{બાંધકામ ડાયાગ્રામ}:
\begin{center}
\begin{tikzpicture}
    % Clamp
    \draw [thick, fill=gray!20] (0,2) arc (180:0:1) -- (2,0) arc (0:180:1) -- cycle;
    \node at (1,1.5) {કોર};
    % Body
    \draw (0, -3) rectangle (2, 0);
    \node at (1, -1.5) [draw, fill=white] {05.2 A}; \node at (1, -2.5) {ડિસ્પ્લે};
    % Wire
    \draw [line width=2pt] (-1, 1) -- (3, 1); \node at (2.5, 1.2) {વાહક};
\end{tikzpicture}
\captionof{figure}{ક્લેમ્પ મીટર}
\end{center}

\begin{itemize}
    \item \textbf{સિદ્ધાંત}: કરંટ ટ્રાન્સફોર્મર (CT).
    \item \textbf{ઉપયોગ}: લાઇવ વાયર કરંટ માપન.
\end{itemize}
\end{solutionbox}

\begin{mnemonicbox}
\mnemonic{CAMP - Current Analyzed by Magnetic Principle}
\end{mnemonicbox}


\questionmarks{3(c)}{7}{સક્સેસિવ એપ્રોક્સિમેશન ટાઈપ DVM ઉપર ટૂંક નોંધ લખો.}

\begin{solutionbox}
\textbf{SAR DVM} એનાલોગ વોલ્ટેજને ડિજિટાઇઝ કરવા માટે બાઈનરી સર્ચ અલ્ગોરિધમનો ઉપયોગ કરે છે.

\textbf{બ્લોક ડાયાગ્રામ}:
\begin{center}
\begin{tikzpicture}[node distance=1.5cm, auto]
    \node [gtu block] (Comp) {કમ્પેરેટર};
    \node [gtu block, left=of Comp] (SH) {સેમ્પલ \& હોલ્ડ};
    \node [gtu block, left=of SH] (In) {ઇનપુટ};
    \node [gtu block, right=of Comp] (SAR) {SAR લોજિક};
    \node [gtu block, below=of SAR] (DAC) {DAC};
    \node [gtu block, right=of SAR] (Disp) {ડિસ્પ્લે};

    \draw [gtu arrow] (In) -- (SH);
    \draw [gtu arrow] (SH) -- (Comp);
    \draw [gtu arrow] (Comp) -- (SAR);
    \draw [gtu arrow] (SAR) -- (DAC);
    \draw [gtu arrow] (DAC) -| (Comp);
    \draw [gtu arrow] (SAR) -- (Disp);
\end{tikzpicture}
\captionof{figure}{SAR DVM}
\end{center}

\begin{itemize}
    \item \textbf{રૂપાંતર સમય}: નિશ્ચિત ($n$ clock cycles).
    \item \textbf{લાભ}: મધ્યમ ગતિ, સતત રૂપાંતર સમય.
\end{itemize}
\end{solutionbox}

\begin{mnemonicbox}
\mnemonic{SACD - Sample, Approximate, Compare, Display}
\end{mnemonicbox}

\questionmarks{3(a) OR}{3}{PH સેન્સર સમજાવો.}

\begin{solutionbox}
\textbf{pH સેન્સર} દ્રાવણની એસિડિટી અથવા આલ્કલાઇનિટી માપે છે.

\textbf{આકૃતિ}:
\begin{center}
\begin{tikzpicture}
    % Beaker
    \draw (0,0) -- (0,3) -- (4,3) -- (4,0) -- cycle; \node at (2,0.5) {દ્રાવણ};
    
    % Electrodes
    \draw [thick] (1,3.5) -- (1,1); \draw [fill=white] (0.8,1) rectangle (1.2,1.5); \node at (1.5, 2) {ગ્લાસ ઇલેક્ટ્રોડ};
    \draw [thick] (3,3.5) -- (3,1); \draw [fill=white] (2.8,1) rectangle (3.2,1.5); \node at (3.5, 2) {રેફરન્સ ઇલેક્ટ્રોડ};
    
    % Meter
    \draw (1,3.5) -- (2,4) -- (3,3.5);
    \node [draw, circle] at (2,4) {mV};
\end{tikzpicture}
\captionof{figure}{pH માપન સિસ્ટમ}
\end{center}

\begin{itemize}
    \item \textbf{સિદ્ધાંત}: $H^+$ આયન સાંદ્રતાના પ્રમાણમાં વોલ્ટેજ ઉત્પન્ન કરે છે (~59mV/pH).
\end{itemize}
\end{solutionbox}

\begin{mnemonicbox}
\mnemonic{PHRV - PH Related to Voltage}
\end{mnemonicbox}

\questionmarks{3(b) OR}{4}{ઇલેક્ટ્રોનિક વોટ મીટરની રચના દોરો અને સમજાવો.}

\begin{solutionbox}
\textbf{ઇલેક્ટ્રોનિક વોટમીટર} પાવર માપે છે ($P = VI \cos \phi$).

\textbf{બ્લોક ડાયાગ્રામ}:
\begin{center}
\begin{tikzpicture}[node distance=1.5cm, auto]
    \node [gtu block] (Mult) {મલ્ટિપ્લાયર};
    \node [gtu block, left=of Mult, yshift=1cm] (VS) {વોલ્ટેજ સેન્સર};
    \node [gtu block, left=of Mult, yshift=-1cm] (CS) {કરંટ સેન્સર};
    \node [gtu block, right=of Mult] (Int) {ઈન્ટીગ્રેટર};
    \node [gtu block, right=of Int] (Disp) {ડિસ્પ્લે};

    \draw [gtu arrow] (VS) -- (Mult);
    \draw [gtu arrow] (CS) -- (Mult);
    \draw [gtu arrow] (Mult) -- (Int);
    \draw [gtu arrow] (Int) -- (Disp);
\end{tikzpicture}
\captionof{figure}{ઇલેક્ટ્રોનિક વોટમીટર}
\end{center}
\end{solutionbox}

\begin{mnemonicbox}
\mnemonic{VIMP - Voltage \& Intensity Make Power}
\end{mnemonicbox}

\questionmarks{3(c) OR}{7}{ઇન્ટીગ્રેટિંગ ટાઈપ DVM ઉપર ટૂંક નોંધ લખો.}

\begin{solutionbox}
\textbf{ઇન્ટીગ્રેટિંગ DVM} નિશ્ચિત સમયગાળા માટે ઇનપુટ વોલ્ટેજનું સરેરાશ મૂલ્ય માપે છે. (ઉદાહરણ: ડ્યુઅલ-સ્લોપ).

\textbf{બ્લોક ડાયાગ્રામ}:
\begin{center}
\begin{tikzpicture}[node distance=1.5cm, auto]
    \node [gtu block] (Int) {ઈન્ટીગ્રેટર};
    \node [gtu block, left=of Int] (Sw) {સ્વિચ};
    \node [gtu block, right=of Int] (Comp) {કમ્પેરેટર};
    \node [gtu block, right=of Comp] (Logic) {કંટ્રોલ લોજિક};
    \node [gtu block, below=of Logic] (Count) {કાઉન્ટર};
    \node [gtu block, right=of Logic] (Disp) {ડિસ્પ્લે};
    \node [gtu block, below=of Int] (Vref) {$V_{ref}$};
    
    \draw [gtu arrow] (Sw) -- (Int);
    \draw [gtu arrow] (Int) -- (Comp);
    \draw [gtu arrow] (Comp) -- (Logic);
    \draw [gtu arrow] (Logic) -- (Count);
    \draw [gtu arrow] (Count) -- (Disp);
    \draw [gtu arrow] (Logic) -| (Sw);
    \draw [gtu arrow] (Vref) -| (Sw);
\end{tikzpicture}
\captionof{figure}{ડ્યુઅલ સ્લોપ DVM}
\end{center}

\begin{itemize}
    \item \textbf{સિદ્ધાંત}: $V_{in}$ ને એકીકૃત કરો, પછી $V_{ref}$ દ્વારા ડિસ્ચાર્જ કરો.
    \item \textbf{લક્ષણો}: ઉત્કૃષ્ટ અવાજ અસ્વીકાર (noise rejection), ઉચ્ચ ચોકસાઈ.
\end{itemize}
\end{solutionbox}

\begin{mnemonicbox}
\mnemonic{TINA - Time Integration Nullifies Average}
\end{mnemonicbox}

\questionmarks{4(a)}{3}{ડિજિટલ સ્ટોરેજ ઓસીલોસ્કોપના ફાયદા અને ઉપયોગો લખો.}

\begin{solutionbox}
\begin{center}
\captionof{table}{ફાયદા અને ઉપયોગો}
\begin{tabulary}{\linewidth}{|L|L|}
\hline
\textbf{ફાયદા} & \textbf{ઉપયોગો} \\ \hline
પ્રી-ટ્રિગર વ્યુઇંગ & ક્ષણિક ઘટનાઓને પકડવી \\ \hline
અનંત સ્ટોરેજ & તૂટક તૂટક ખામીઓનું વિશ્લેષણ \\ \hline
વેવફોર્મ પ્રોસેસિંગ & જટિલ સિગ્નલ વિશ્લેષણ \\ \hline
હાર્ડ કોપી/PC ઈન્ટરફેસ & ડેટા લોગીંગ \\ \hline
\end{tabulary}
\end{center}
\end{solutionbox}

\begin{mnemonicbox}
\mnemonic{SPADE - Storage, Processing, Analysis, Display, Events}
\end{mnemonicbox}

\questionmarks{4(b)}{4}{ઇલેક્ટ્રોનિક એનર્જી મીટર ઉપર ટૂંકનોંધ લખો.}

\begin{solutionbox}
\textbf{બ્લોક ડાયાગ્રામ}:
\begin{center}
\begin{tikzpicture}[node distance=1.5cm, auto]
    \node [gtu block] (P) {પાવર CPU};
    \node [gtu block, left=of P] (Sens) {V \& I સેન્સર};
    \node [gtu block, right=of P] (Mic) {માઇક્રોકંટ્રોલર};
    \node [gtu block, right=of Mic] (Disp) {LCD};
    \node [gtu block, below=of P] (Pulse) {પલ્સ LED};

    \draw [gtu arrow] (Sens) -- (P);
    \draw [gtu arrow] (P) -- (Mic);
    \draw [gtu arrow] (Mic) -- (Disp);
    \draw [gtu arrow] (P) -- (Pulse);
\end{tikzpicture}
\captionof{figure}{એનર્જી મીટર}
\end{center}
\end{solutionbox}

\begin{mnemonicbox}
\mnemonic{VICES - Voltage \& Current Energy Summation}
\end{mnemonicbox}

\questionmarks{4(c)}{7}{એનાલોગ C.R.O. નો બ્લોક ડાયાગ્રામ દોરો અને સમજાવો, અને દરેક બ્લોકનું વર્કિંગ લખો.}

\begin{solutionbox}
\textbf{બ્લોક ડાયાગ્રામ}:
\begin{center}
\begin{tikzpicture}[node distance=1.5cm, auto]
    \node [gtu block] (V_Amp) {વર્ટિકલ Amp};
    \node [gtu block, left=of V_Amp] (In) {ઇનપુટ};
    \node [gtu block, right=of V_Amp] (Delay) {ડીલે લાઇન};
    \node [gtu block, below=of V_Amp] (Trig) {ટ્રિગર};
    \node [gtu block, right=of Trig] (TB) {ટાઇમ બેઝ};
    \node [gtu block, right=of TB] (H_Amp) {હોરીઝ. Amp};
    \node [gtu block, right=of Delay] (CRT) {CRT};
    \node [gtu block, below=of Trig] (PS) {પાવર સપ્લાય};

    \draw [gtu arrow] (In) -- (V_Amp);
    \draw [gtu arrow] (V_Amp) -- (Delay);
    \draw [gtu arrow] (Delay) -- node[above]{Y} (CRT);
    \draw [gtu arrow] (V_Amp) -- (Trig);
    \draw [gtu arrow] (Trig) -- (TB);
    \draw [gtu arrow] (TB) -- (H_Amp);
    \draw [gtu arrow] (H_Amp) -- node[right]{X} (CRT);
    \draw [gtu arrow] (PS) -| (CRT);
\end{tikzpicture}
\captionof{figure}{CRO બ્લોક ડાયાગ્રામ}
\end{center}

\begin{center}
\captionof{table}{બ્લોક કાર્યો}
\begin{tabulary}{\linewidth}{|L|L|}
\hline
\textbf{વર્ટિકલ એમ્પ્લીફાયર} & Y-ડિફ્લેક્શન માટે નબળા ઇનપુટ સિગ્નલને એમ્પ્લીફાય કરે છે \\ \hline
\textbf{ડીલે લાઇન} & સિગ્નલને Y-પ્લેટ્સમાં વિલંબિત કરે છે \\ \hline
\textbf{ટાઇમ બેઝ} & X-ડિફ્લેક્શન માટે સો-ટૂથ વેવ જનરેટ કરે છે \\ \hline
\textbf{CRT} & વેવફોર્મ દર્શાવે છે \\ \hline
\end{tabulary}
\end{center}
\end{solutionbox}

\begin{mnemonicbox}
\mnemonic{VTHCP - Vertical, Time, Horizontal, CRT, Power}
\end{mnemonicbox}

\questionmarks{4(a) OR}{3}{પીજો ઈલેક્ટ્રીક ટ્રાન્સડ્યુસર દોરો અને સમજાવો.}

\begin{solutionbox}
\textbf{પીઝોઇલેક્ટ્રિક ટ્રાન્સડ્યુસર} દબાણને વોલ્ટેજમાં રૂપાંતરિત કરે છે.

\textbf{આકૃતિ}:
\begin{center}
\begin{tikzpicture}
    % Crystal
    \draw [fill=blue!10] (0,0) rectangle (3,1); \node at (1.5,0.5) {ક્વાર્ટઝ ક્રિસ્ટલ};
    % Force
    \draw [->, thick] (1.5, 2) -- (1.5, 1); \node at (1.5, 2.2) {બળ $F$};
    % Output
    \draw (3,1) -- (4,1); \draw (3,0) -- (4,0);
    \node at (4.5, 0.5) {$V_{out}$};
\end{tikzpicture}
\captionof{figure}{પીઝોઇલેક્ટ્રિક ક્રિસ્ટલ}
\end{center}

\begin{itemize}
    \item \textbf{સિદ્ધાંત}: પીઝોઇલેક્ટ્રિક ઇફેક્ટ. તણાવ $\rightarrow$ ચાર્જ.
    \item \textbf{સામગ્રી}: ક્વાર્ટઝ, રોશેલ સોલ્ટ, PZT.
\end{itemize}
\end{solutionbox}

\begin{mnemonicbox}
\mnemonic{PFVD - Pressure Forms Voltage via Displacement}
\end{mnemonicbox}

\questionmarks{4(b) OR}{4}{CRO ની મદદથી ફ્રિકવન્સી મેઝરમેન્ટ માટેની આકૃતિ દોરો અને સમજાવો.}

\begin{solutionbox}
\textbf{પદ્ધતિ 1: ટાઇમ બેઝ (ડાયરેક્ટ)}
\begin{itemize}
    \item એક ચક્રનો સમયગાળો $T$ માપો. $f = 1/T$.
\end{itemize}

\textbf{પદ્ધતિ 2: લિસાજોસ ફિગર્સ (XY મોડ)}
\begin{center}
\begin{tikzpicture}
    \draw [domain=0:360, samples=100] plot ({sin(\x)}, {sin(2*\x)});
    \node at (0,-1.5) {પેટર્ન ($f_y : f_x = 2:1$)};
\end{tikzpicture}
\captionof{figure}{લિસાજોસ પેટર્ન}
\end{center}

\begin{itemize}
    \item $f_y/f_x = \text{આડા સ્પર્શકો} / \text{ઊભા સ્પર્શકો}$.
\end{itemize}
\end{solutionbox}

\begin{mnemonicbox}
\mnemonic{LTX - Lissajous or Time for X-axis}
\end{mnemonicbox}

\questionmarks{4(c) OR}{7}{થર્મિસ્ટર અને થર્મોકપલ દોરો અને સમજાવો.}

\begin{solutionbox}
\textbf{1. થર્મિસ્ટર}:
તાપમાન સાથે બદલાતો રેઝિસ્ટર.
\begin{itemize}
    \item \textbf{પ્રકાર}: NTC (નેગેટિવ કોએફિફિશિયન્ટ), PTC (પોઝિટિવ).
    \item \textbf{લક્ષણ}: ઉચ્ચ સંવેદનશીલ, બિન-રેખીય.
\end{itemize}

\textbf{2. થર્મોકપલ}:
સીબેક અસર પર આધારિત સક્રિય ટ્રાન્સડ્યુસર.
\begin{center}
\begin{tikzpicture}
    \draw [thick, red] (0,0) -- (4,0); \node at (2, 0.2) {મેટલ A};
    \draw [thick, blue] (0,0) -- (4, -1); \node at (2, -1.2) {મેટલ B};
    \node [circle, fill, inner sep=2pt, label=left:$T_1$ (Hot)] at (0,0) {};
    \node at (4.5, -0.5) {આઉટપુટ};
\end{tikzpicture}
\captionof{figure}{થર્મોકપલ}
\end{center}
\begin{itemize}
    \item \textbf{સિદ્ધાંત}: અસમાન ધાતુઓના જંકશન EMF ઉત્પન્ન કરે છે.
    \item \textbf{પ્રકાર}: J, K.
\end{itemize}
\end{solutionbox}

\begin{mnemonicbox}
\mnemonic{TRT/TVJ - Temperature Resistance/Voltage Junction}
\end{mnemonicbox}

\questionmarks{5(a)}{3}{વેલોસિટી ટ્રાન્સડ્યુસર દોરો અને સમજાવો.}

\begin{solutionbox}
\textbf{વેલોસિટી ટ્રાન્સડ્યુસર} (ઇલેક્ટ્રોમેગ્નેટિક).

\textbf{ડાયાગ્રામ}:
\begin{center}
\begin{tikzpicture}
    % Magnet
    \draw [fill=gray!30] (0,0) rectangle (1,3); \node at (0.5, 1.5) {ચુંબક};
    % Coil
    \draw [thick] (1.5, 1) rectangle (2.5, 2); \node at (2, 1.5) {કોઇલ};
    \draw [dashed] (1.5, 1.5) -- (3.5, 1.5) node[right]{ગતિ};
    % Terminals
    \draw (2.5, 1.8) -- (3, 1.8);
    \draw (2.5, 1.2) -- (3, 1.2); \node at (3.5, 1.5) {$V_{out}$};
\end{tikzpicture}
\captionof{figure}{વેલોસિટી ટ્રાન્સડ્યુસર}
\end{center}

\begin{itemize}
    \item \textbf{સિદ્ધાંત}: ફેરાડેનો નિયમ ($e = N \frac{d\phi}{dt}$).
    \item \textbf{આઉટપુટ}: વોલ્ટેજ વેગના પ્રમાણમાં છે.
\end{itemize}
\end{solutionbox}

\begin{mnemonicbox}
\mnemonic{VMMF - Velocity Makes Magnetic Flux}
\end{mnemonicbox}

\questionmarks{5(b)}{4}{ટ્રાન્સડ્યુસર નું વર્ગીકરણ કરો અને સમજાવો.}

\begin{solutionbox}
\begin{center}
\captionof{table}{વર્ગીકરણ}
\begin{tabulary}{\linewidth}{|L|L|}
\hline
\textbf{આધાર} & \textbf{પ્રકારો} \\ \hline
\textbf{પાવર સોર્સ} & \textbf{સક્રિય}: સ્વતઃ જનરેટિંગ (થર્મોકપલ). \textbf{નિષ્ક્રિય}: બાહ્ય શક્તિ જરૂરી (RTD). \\ \hline
\textbf{ટ્રાન્સડક્શન} & રેઝિસ્ટિવ, ઇન્ડક્ટિવ, કેપેસિટીવ, વગેરે. \\ \hline
\textbf{કાર્ય} & \textbf{પ્રાથમિક}: ઘટના શોધે છે. \textbf{ગૌણ}: વિદ્યુતમાં ફેરવે છે. \\ \hline
\textbf{આઉટપુટ} & એનાલોગ વિ ડિજિટલ. \\ \hline
\end{tabulary}
\end{center}
\end{solutionbox}

\begin{mnemonicbox}
\mnemonic{APRCI - Active Passive Resistive Capacitive Inductive}
\end{mnemonicbox}

\questionmarks{5(c)}{7}{LVDT ઉપર ટૂંકનોંધ લખો.}

\begin{solutionbox}
\textbf{LVDT} વિસ્થાપન માટે ઇન્ડક્ટિવ ટ્રાન્સડ્યુસર છે.

\textbf{ડાયાગ્રામ}:
\begin{center}
\begin{circuitikz}
    \draw (0,0) node[left]{Excitation} to[L, l=$P$] (0,2);
    \draw (2,2) to[L, l=$S_1$] (2,1) -- (2,0.5);
    \draw (2,-0.5) -- (2,-1) to[L, l=$S_2$] (2,-2);
    \draw (2,1) -- (3,1) node[right]{આઉટપુટ +};
    \draw (2,-2) -- (3,-2) node[right]{આઉટપુટ -};
    \draw (2,0.5) -- (2,-0.5); 
    \node at (1,0) {કોર};
\end{circuitikz}
\captionof{figure}{LVDT યોજનાકીય}
\end{center}

\begin{itemize}
    \item \textbf{રચના}: એક પ્રાથમિક વિન્ડિંગ, બે ગૌણ. મૂવિંગ સોફ્ટ આયર્ન કોર.
    \item \textbf{કાર્ય}: કોરની ગતિ ફ્લક્સ જોડાણને બદલે છે, ડિફરન્શિયલ આઉટપુટ બનાવે છે.
\end{itemize}
\end{solutionbox}

\begin{mnemonicbox}
\mnemonic{CPSO: Core Position Shifts Output}
\end{mnemonicbox}

\questionmarks{5(a) OR}{3}{સાદા ફ્રિક્વન્સી કાઉન્ટરનો બ્લોક ડાયાગ્રામ દોરો અને સમજાવો.}

\begin{solutionbox}
\textbf{ડિજિટલ ફ્રીક્વન્સી કાઉન્ટર} પલ્સની ગણતરી કરે છે.

\textbf{બ્લોક ડાયાગ્રામ}:
\begin{center}
\begin{tikzpicture}[node distance=1.5cm, auto]
    \node [gtu block] (Gate) {મેઈન ગેટ};
    \node [gtu block, left=of Gate] (In) {ઇનપુટ};
    \node [gtu block, below=of Gate] (TB) {ટાઇમ બેઝ};
    \node [gtu block, right=of Gate] (Count) {કાઉન્ટર};
    \node [gtu block, right=of Count] (Disp) {ડિસ્પ્લે};

    \draw [gtu arrow] (In) -- (Gate);
    \draw [gtu arrow] (TB) -- (Gate);
    \draw [gtu arrow] (Gate) -- (Count);
    \draw [gtu arrow] (Count) -- (Disp);
\end{tikzpicture}
\captionof{figure}{ફ્રીક્વન્સી કાઉન્ટર}
\end{center}
\end{solutionbox}

\begin{mnemonicbox}
\mnemonic{IGTCD - Input Gated Time Counts Display}
\end{mnemonicbox}

\questionmarks{5(b) OR}{4}{કેપેસિટીવ ટ્રાન્સડ્યુસર દોરો અને સમજાવો.}

\begin{solutionbox}
\textbf{કેપેસિટીવ ટ્રાન્સડ્યુસર} $C = \frac{\epsilon A}{d}$ પર કામ કરે છે.

\textbf{ડાયાગ્રામ}:
\begin{center}
\begin{tikzpicture}
    \draw [thick] (0,2) -- (3,2); \node at (1.5, 2.3) {સ્થિર પ્લેટ};
    \draw [thick] (0,1) -- (3,1); \node at (1.5, 0.7) {મૂવિંગ પ્લેટ};
    \draw [<->] (3.2, 1) -- (3.2, 2); \node at (3.5, 1.5) {$d$};
    \draw [->] (1.5, 0) -- (1.5, 1); \node at (1.5, -0.3) {બળ};
\end{tikzpicture}
\captionof{figure}{વેરિએબલ ગેપ કેપેસિટીવ ટ્રાન્સડ્યુસર}
\end{center}
\end{solutionbox}

\begin{mnemonicbox}
\mnemonic{CGAD - Capacitance Gap Area Dielectric}
\end{mnemonicbox}

\questionmarks{5(c) OR}{7}{ફંકશન જનરેટરનો બ્લોક ડાયાગ્રામ દોરો અને સમજાવો.}

\begin{solutionbox}
\textbf{ફંક્શન જનરેટર} સાઈન, સ્ક્વેર અને ત્રિકોણાકાર તરંગો ઉત્પન્ન કરે છે.

\textbf{બ્લોક ડાયાગ્રામ}:
\begin{center}
\begin{tikzpicture}[node distance=1.5cm, auto]
    \node [gtu block] (Osc) {ઓસિલેટર};
    \node [gtu block, right=of Osc] (Tri) {ટ્રાયએંગલ};
    \node [gtu block, right=of Tri] (Shape) {સાઇન શેપર};
    \node [gtu block, below=of Tri] (Comp) {કમ્પેરેટર};
    \node [gtu block, right=of Shape, xshift=1cm] (Amp) {આઉટપુટ Amp};
    
    \draw [gtu arrow] (Osc) -- (Tri);
    \draw [gtu arrow] (Tri) -- (Shape);
    \draw [gtu arrow] (Tri) -- (Comp);
    \draw [gtu arrow] (Comp) -| (Osc);
    
    \draw [dashed] (Tri) -- (Amp);
    \draw [dashed] (Shape) -- (Amp);
    \draw [dashed] (Comp.east) -- ++(0.5,0) |- (Amp);
    
\end{tikzpicture}
\captionof{figure}{ફંક્શન જનરેટર}
\end{center}
\end{solutionbox}

\begin{mnemonicbox}
\mnemonic{FWMASO - Frequency Waveform Mode Amplitude Sweep Output}
\end{mnemonicbox}

\end{document}
