\documentclass{article}

% content/resources/templates/preamble.tex
\usepackage[margin=0.6in]{geometry}
\author{Milav Dabgar}
\usepackage{amsmath,amssymb,amsthm}
\usepackage{booktabs}
\usepackage{multirow}
\usepackage{xcolor}
\usepackage{tcolorbox}
\tcbuselibrary{breakable,skins}
\usepackage[colorlinks=true,linkcolor=blue]{hyperref}
\usepackage{titlesec}
\usepackage{enumitem}
\usepackage{tikz}
\usepackage{pgfplots}
\usepackage{circuitikz}
\usepackage[version=4]{mhchem}
\usepackage{longtable}
\usepackage{array}
\usepackage{float}
\usepackage{caption}
\usepackage{listings}

\lstset{
  basicstyle=\small\ttfamily,
  breaklines=true,
  breakatwhitespace=false,
  postbreak=\mbox{\textcolor{red}{$\hookrightarrow$}\space},
  float=false,
  numbers=left,
  numberstyle=\tiny\color{gray},
  numbersep=10pt,
  xleftmargin=2em,
  keywordstyle=\color{blue},
  commentstyle=\color{green!60!black},
  stringstyle=\color{purple},
  backgroundcolor=\color{gray!5},
  showstringspaces=false,
  tabsize=2,
  captionpos=b,
  keepspaces=true,
  columns=flexible
}

\pgfplotsset{compat=1.18}
\usetikzlibrary{shapes,arrows,positioning,calc,patterns,decorations.pathmorphing,decorations.markings,arrows.meta}

% Color scheme
\definecolor{headcolor}{RGB}{0,102,204}
\definecolor{keycolor}{RGB}{220,20,60}
\definecolor{solutioncolor}{RGB}{34,139,34}
\definecolor{mnemoniccolor}{RGB}{148,0,211}
\definecolor{codecolor}{RGB}{0,0,100}

% Spacing
\setlength{\parskip}{3pt}
\setlist[itemize]{nosep}
\setlist[enumerate]{nosep}

% Title formatting
\titleformat{\section}{\Large\bfseries\color{headcolor}}{\thesection}{1em}{}
\titleformat{\subsection}{\large\bfseries\color{headcolor}}{\thesubsection}{1em}{}

% Pandoc tightlist compatibility
\providecommand{\tightlist}{%
  \setlength{\itemsep}{0pt}\setlength{\parskip}{0pt}}

% Pandoc longtable compatibility
\newcounter{none}
\def\thenone{}


% content/resources/templates/gujarati-boxes.tex
\usepackage{fontspec}
\usepackage{polyglossia}

% Set Gujarati as main language (document is primarily in Gujarati)
% Note: gloss-gujarati.ldf doesn't exist in polyglossia, but it will use hyphenation patterns
\setdefaultlanguage{gujarati}
\setotherlanguage{english}

% Configure Gujarati font properly
% Use Language=Default to prevent polyglossia from trying to add language-specific features
% that don't exist for Gujarati, which causes "empty feature" warnings
\newfontfamily\gujaratifont[Script=Gujarati,AutoFakeBold=2.5,AutoFakeSlant=0.3]{Noto Sans Gujarati}
\setmainfont[Script=Gujarati,AutoFakeBold=2.5,AutoFakeSlant=0.3]{Noto Sans Gujarati}
% Use Noto Sans Gujarati for monospace to support Gujarati in text
\setmonofont[Scale=0.9]{Noto Sans Gujarati}

% Configure English to use the same font
\newfontfamily\englishfont[Script=Gujarati,AutoFakeBold=2.5,AutoFakeSlant=0.3]{Noto Sans Gujarati}

% Translations for polyglossia
\gappto\captionsgujarati{
  \renewcommand{\tablename}{કોષ્ટક}
  \renewcommand{\figurename}{આકૃતિ}
}

% Helper for TikZ nodes to ensure Gujarati font
\newcommand{\gu}[1]{{\gujaratifont #1}}

% Custom environments
\newtcolorbox{solutionbox}{
    breakable,
    enhanced,
    colback=solutioncolor!5!white,
    colframe=solutioncolor!75!black,
    fonttitle=\bfseries,
    title=જવાબ
}

\newtcolorbox{solutionboxnobreak}{
 colback=solutioncolor!5!white,
 colframe=solutioncolor!75!black,
 fonttitle=\bfseries,
 title=જવાબ
}

\newtcolorbox{keyformula}{
 breakable,
 enhanced,
 colback=keycolor!5!white,
 colframe=keycolor!75!black,
 fonttitle=\bfseries,
 title=રાસાયણિક સમીકરણ/સૂત્ર
}

\newtcolorbox{mnemonicbox}{
 breakable,
 enhanced,
 colback=mnemoniccolor!5!white,
 colframe=mnemoniccolor!75!black,
 fonttitle=\bfseries,
 title=મેમરી ટ્રીક
}


% Custom commands for GTU solutions
% This file defines semantic commands for consistent formatting

% Question command with automatic formatting
\newcommand{\question}[2]{%
  \section*{Question #1}%
  \textbf{#2}%
}

% OR question variant
\newcommand{\questionor}[2]{%
  \section*{Question #1 OR}%
  \textbf{#2}%
}

% Proper table environment with caption
\newenvironment{answertable}[1]{%
  \begin{table}[htbp]
  \centering
  \caption{#1}
}{%
  \end{table}
}

% Proper figure environment for diagrams
\newenvironment{answerdiagram}[1]{%
  \begin{figure}[htbp]
  \centering
  \caption{#1}
}{%
  \end{figure}
}

% Semantic markup for key terms
\newcommand{\keyword}[1]{\textbf{#1}}
\newcommand{\code}[1]{\texttt{#1}}
\newcommand{\classname}[1]{\texttt{#1}}
\newcommand{\methodname}[1]{\texttt{#1}}

% Proper quotation marks
\newcommand{\mnemonic}[1]{``#1''}


\title{ઇલેક્ટ્રોનિક મેઝરમેન્ટ્સ એન્ડ ઇન્સ્ટ્રુમેન્ટ્સ (4331102) - સમર 2023 સોલ્યુશન}
\date{July 19, 2023}

\begin{document}
\maketitle

\questionmarks{1(a)}{3}{તમામ પ્રકારની સિસ્ટેમેટીક ભૂલને ઘટાડવા માટેના પગલાંઓનું વર્ણન કરો.}

\begin{solutionbox}
\textbf{સિસ્ટેમેટીક ભૂલ ઘટાડવાના પગલાં:}
\begin{center}
\captionof{table}{Systematic Errors ઘટાડવાના પગલાં}
\begin{tabulary}{\linewidth}{|L|L|}
\hline
\textbf{પગલું} & \textbf{વર્ણન} \\ \hline
\textbf{1. કેલિબ્રેશન} & પ્રમાણભૂત સંદર્ભ સાથે સાધનોનું સમયાંતરે કેલિબ્રેશન કરવું \\ \hline
\textbf{2. સુધારણા} & સુધારણા ફેક્ટર અથવા ઓફસેટ વેલ્યુ લાગુ કરવું \\ \hline
\textbf{3. નિયંત્રણ} & સ્થિર પર્યાવરણીય પરિસ્થિતિઓ (તાપમાન, ભેજ) જાળવવી \\ \hline
\textbf{4. તકનીક} & યોગ્ય માપન તકનીકો અને પ્રક્રિયાઓનો ઉપયોગ કરવો \\ \hline
\textbf{5. સાધન} & જરૂરી ચોકસાઈ સાથે યોગ્ય સાધનોની પસંદગી કરવી \\ \hline
\end{tabulary}
\end{center}
\end{solutionbox}

\begin{mnemonicbox}
\mnemonic{CCCTS: Calibrate, Correct, Control, Technique, Select}
\end{mnemonicbox}

\questionmarks{1(b)}{4}{વ્યાખ્યાયિત કરો: રીઝોલ્યુશન, પ્રિસિજન, સેન્સીટિવિટી અને એક્યુરસી.}

\begin{solutionbox}
\begin{center}
\captionof{table}{માપન લાક્ષણિકતાઓ વ્યાખ્યાઓ}
\begin{tabulary}{\linewidth}{|L|L|}
\hline
\textbf{શબ્દ} & \textbf{વ્યાખ્યા} \\ \hline
\textbf{રીઝોલ્યુશન} & સાધન દ્વારા શોધી શકાય તેવો ઇનપુટમાં સૌથી નાનો ફેરફાર \\ \hline
\textbf{પ્રિસિજન} & ન્યૂનતમ રેન્ડમ ભૂલ સાથે માપનની સુસંગતતા અથવા પુનરાવર્તનીયતા \\ \hline
\textbf{સેન્સીટિવિટી} & ઇનપુટના ફેરફાર માટે આઉટપુટમાં ફેરફારનું પ્રમાણ ($\Delta O/\Delta I$) \\ \hline
\textbf{એક્યુરસી} & માપેલા મૂલ્યનો સાચા અથવા સ્વીકૃત માનક મૂલ્ય સાથે નજીકપણું \\ \hline
\end{tabulary}
\end{center}

\begin{center}
\begin{circuitikz}[
    level 1/.style = {sibling distance=4cm},
    level 2/.style = {sibling distance=2.5cm},
    edge from parent/.style = {draw, -latex},
    every node/.style = {rectangle, draw, rounded corners, align=center, font=\small}
]
    \node {Measurement Quality}
        child { node {Resolution}
            child { node {Smallest Change} }
        }
        child { node {Precision}
            child { node {Repeatability} }
        }
        child { node {Sensitivity}
            child { node {Output/Input} }
        }
        child { node {Accuracy}
            child { node {Closeness to Truth} }
        };
\end{circuitikz}
\captionof{figure}{Measurement Characteristics}
\end{center}
\end{solutionbox}

\begin{mnemonicbox}
\mnemonic{RSPA: Resolve Signals Precisely and Accurately}
\end{mnemonicbox}

\questionmarks{1(c)}{7}{Q મીટરનો સિદ્ધાંત અને પ્રેક્ટીકલ Q મીટરની કામગીરી સમજાવો.}

\begin{solutionbox}
\textbf{સિદ્ધાંત:}
\begin{itemize}
    \item સીરીઝ રેઝોનન્સ પર આધારિત જ્યાં $Q = X_L/R$ અથવા $X_C/R$ રેઝોનન્સ સ્થિતિએ
    \item રેઝોનન્સ સ્થિતિએ વોલ્ટેજ મેગ્નિફિકેશન માપે છે
\end{itemize}

\textbf{પ્રેક્ટીકલ Q મીટરની કામગીરી:}
\begin{center}
\captionof{table}{પ્રાયોગિક Q મીટરના ઘટકો}
\begin{tabulary}{\linewidth}{|L|L|}
\hline
\textbf{ઘટક} & \textbf{કાર્ય} \\ \hline
\textbf{ઓસિલેટર} & વેરીએબલ ફ્રીકવન્સી સિગ્નલ (50kHz થી 50MHz) જનરેટ કરે છે \\ \hline
\textbf{વર્ક કોઇલ} & ટેસ્ટ હેઠળની ઇન્ડક્ટર (કેલિબ્રેટેડ કેપેસિટર સાથે સીરીઝમાં જોડાયેલ) \\ \hline
\textbf{કેપેસિટર} & રેઝોનન્સ ટ્યુનિંગ માટે વેરીએબલ કેલિબ્રેટેડ કેપેસિટર \\ \hline
\textbf{VTVM} & કેપેસિટર પર રેઝોનન્ટ વોલ્ટેજ માપે છે \\ \hline
\textbf{શન્ટ રેઝિસ્ટર} & સર્કિટમાં કરંટનું મોનિટરિંગ કરે છે \\ \hline
\end{tabulary}
\end{center}

\begin{center}
\begin{circuitikz}[auto]
    \draw (0,0) to [sV, l=RF Source] (0,2) -- (2,2) coordinate (A);
    \draw (A) to [L, l=$L_x$ (Work Coil)] (4,2) coordinate (B);
    \draw (B) to [vC, l=$C$ (Tuning)] (4,0) coordinate (C);
    \draw (C) -- (0,0);
    
    \draw (B) -- (6,2) coordinate (Vtop);
    \draw (C) -- (6,0) coordinate (Vbot);
    \draw (Vtop) to [voltmeter, l=VTVM (Q Reading)] (Vbot);
    
    \draw (0,0) to [R, l=$R_{sh}$] (2,0); % Series injection Rsh usually in osc path, simplified here
\end{circuitikz}
\captionof{figure}{Practical Q Meter}
\end{center}

\begin{itemize}
    \item \textbf{Q ફેક્ટર ગણતરી}: $Q = V_2/V_1$ જ્યાં $V_2$ કેપેસિટર પરનું વોલ્ટેજ અને $V_1$ એપ્લાઈડ વોલ્ટેજ છે
    \item \textbf{રેઝોનન્સ ઇન્ડિકેશન}: કેપેસિટર પર મહત્તમ વોલ્ટેજ રેઝોનન્સ દર્શાવે છે
\end{itemize}
\end{solutionbox}

\begin{mnemonicbox}
\mnemonic{VOCAL: Voltage ratio at resonance Oscillator Creates Amplification to measure coiL quality}
\end{mnemonicbox}

\questionmarks{1(c OR)}{7}{વ્હીટસ્ટોન બ્રિજ સમજાવો અને બેલેન્સ કંડીશન માટે સમીકરણ મેળવો. વ્હીટસ્ટોન બ્રિજની એપ્લિકેશન અને મર્યાદા લખો.}

\begin{solutionbox}
વ્હીટસ્ટોન બ્રિજ એ ઉચ્ચ સચોટતા સાથે અજ્ઞાત પ્રતિરોધ માપવા માટે વપરાતું નેટવર્ક છે.

\begin{center}
\begin{circuitikz}[auto]
    \draw (0,3) coordinate (A) -- (2,4) coordinate (D) -- (4,3) coordinate (C) -- (2,2) coordinate (B) -- (A);
    \draw (A) -- (-1,3); \draw (C) -- (5,3); % Terminals
    
    \node [above] at (D) {D}; \node [below] at (B) {B};
    \node [left] at (A) {A}; \node [right] at (C) {C};
    
    \path (A) -- (D) node[midway, above left] {$R_1$};
    \path (D) -- (C) node[midway, above right] {$R_x$ (Unknown)};
    \path (A) -- (B) node[midway, below left] {$R_2$};
    \path (B) -- (C) node[midway, below right] {$R_3$};
    
    \draw (D) to [ammeter, l=G] (B);
    
    \draw (-1,3) to [battery1, l=E] (-1,1) -| (2,1) -- (B); % Simplified source connection
    \draw (2,1) -- (2,0.5) node[ground]{}; % Optional ground
\end{circuitikz}
\captionof{figure}{Wheatstone Bridge}
\end{center}

\textbf{બેલેન્સ કંડીશન સમીકરણની તારણ:}
\begin{itemize}
    \item બેલેન્સ સ્થિતિએ, ગેલ્વેનોમીટરમાંથી કરંટ પસાર થતો નથી ($I_G = 0$)
    \item પોઇન્ટ D પરનું પોટેન્શિયલ = પોઇન્ટ B પરનું પોટેન્શિયલ
    \item $R_1$ પરનું વોલ્ટેજ = $R_2$ પરનું વોલ્ટેજ ($I_1 R_1 = I_2 R_2$)
    \item $R_x$ પરનું વોલ્ટેજ = $R_3$ પરનું વોલ્ટેજ ($I_1 R_x = I_2 R_3$)
\end{itemize}
સમીકરણોને ભાગતા:
\[ \frac{I_1 R_1}{I_1 R_x} = \frac{I_2 R_2}{I_2 R_3} \implies \frac{R_1}{R_x} = \frac{R_2}{R_3} \implies R_x = R_3 \left( \frac{R_1}{R_2} \right) \]

\textbf{એપ્લિકેશન અને મર્યાદા:}
\begin{center}
\captionof{table}{Wheatstone Bridge ના ઉપયોગો અને મર્યાદાઓ}
\begin{tabulary}{\linewidth}{|L|L|}
\hline
\textbf{એપ્લિકેશન} & \textbf{મર્યાદા} \\ \hline
પ્રિસીઝન રેઝિસ્ટન્સ મેઝરમેન્ટ & ખૂબ ઓછા રેઝિસ્ટન્સ ($<1\Omega$) માટે નબળી ચોકસાઈ \\ \hline
ટ્રાન્સડ્યુસર ઇન્ટરફેસ (Strain gauge, RTD) & ગેલ્વેનોમીટરની સેન્સિટિવિટી દ્વારા મર્યાદિત \\ \hline
તાપમાન સેન્સિંગ & સંપર્ક પ્રતિરોધ ચોકસાઈને અસર કરે છે \\ \hline
\end{tabulary}
\end{center}
\end{solutionbox}

\begin{mnemonicbox}
\mnemonic{BEAR: Balance Equation at Arms Ratio}
\end{mnemonicbox}

% ==================================================================
% QUESTION 2
% ==================================================================

\questionmarks{2(a)}{3}{મૂવિંગ આયર્ન અને મૂવિંગ કોઇલ પ્રકારના સાધનો વચ્ચે તફાવત કરો.}

\begin{solutionbox}
\begin{center}
\captionof{table}{તફાવત: Moving Iron vs PMMC ઇન્સ્ટ્રુમેન્ટ્સ}
\begin{tabulary}{\linewidth}{|L|L|L|}
\hline
\textbf{પેરામીટર} & \textbf{મૂવિંગ આયર્ન ઇન્સ્ટ્રુમેન્ટ} & \textbf{મૂવિંગ કોઇલ ઇન્સ્ટ્રુમેન્ટ} \\ \hline
\textbf{ઓપરેટિંગ પ્રિન્સિપલ} & મેગ્નેટિક એટ્રેક્શન અથવા રિપલ્શન & કરંટ-કેરીંગ કન્ડક્ટર પર ઇલેક્ટ્રોમેગ્નેટિક ફોર્સ \\ \hline
\textbf{સ્કેલ} & નોન-યુનિફોર્મ સ્કેલ & યુનિફોર્મ સ્કેલ \\ \hline
\textbf{ફ્રીકવન્સી રેન્જ} & AC અને DC બંને માટે કામ કરે છે & માત્ર DC (રેક્ટિફાઈ કર્યા સિવાય) \\ \hline
\textbf{ચોકસાઈ} & ઓછી (1-2.5\%) & વધારે (0.1-1\%) \\ \hline
\textbf{ડેમ્પિંગ} & એર ફ્રિક્શન ડેમ્પિંગ & એડી કરંટ ડેમ્પિંગ \\ \hline
\textbf{પાવર વપરાશ} & વધારે & ઓછી \\ \hline
\end{tabulary}
\end{center}
\end{solutionbox}

\begin{mnemonicbox}
\mnemonic{IRON-COIL: Iron uses Repulsion with Non-uniform scale; COIL uses Current with Organized, Improved, Linear scale}
\end{mnemonicbox}

\questionmarks{2(b)}{4}{ક્લેમ્પ ઓન એમીટરનું કન્સ્ટ્રક્શન દોરો અને વિગતવાર સમજાવો.}

\begin{solutionbox}
\textbf{ક્લેમ્પ-ઓન એમીટરનો કન્સ્ટ્રક્શન આકૃતિ:}
\begin{center}
\begin{circuitikz}[auto]
    \draw (0,0) [rounded corners=0.5cm] -- (0,3) -- (3,3) -- (3,0) -- cycle; % Body
    \draw (0.5,3) [rounded corners=1cm] -- (0.5,5) -- (2.5,5) -- (2.5,3); % Clamp jaws
    \draw (0.5,3) -- (2.5,3); % Jaw base
    \draw [thick] (1.5,4) circle (0.2); \node at (1.5,4) {$\bullet$}; \node [right] at (1.7,4) {Conductor};
    
    \draw (0.5,2) rectangle (2.5,2.8); \node at (1.5,2.4) {Display};
    \draw (0.5,0.5) rectangle (2.5,1.5); \node at (1.5,1) {Controls};
    
    \node [left] at (0,4) {Jaws};
    \draw [->] (-0.5,4) -- (0.5,4.5);
\end{circuitikz}
\captionof{figure}{Clamp-on Ammeter}
\end{center}

\textbf{ઘટકો અને કાર્ય:}
\begin{itemize}
    \item \textbf{કોર}: સ્પ્લિટ લેમિનેટેડ ફેરોમેગ્નેટિક કોર જે ખોલી/બંધ કરી શકાય છે
    \item \textbf{કોઇલ}: કોર પર વીંટાળેલા સેકન્ડરી વાઇન્ડીંગ
    \item \textbf{કન્ડક્ટર}: પ્રાઈમરી કન્ડક્ટર (માપવાના કરંટ) કોરમાંથી પસાર થાય છે
    \item \textbf{મેઝરમેન્ટ સર્કિટ}: ઇન્ડ્યુસ્ડ કરંટ પ્રોસેસ કરે છે અને રીડિંગ દર્શાવે છે
    \item \textbf{વર્કિંગ પ્રિન્સિપલ}: ટ્રાન્સફોર્મર પ્રિન્સિપલ પર આધારિત જ્યાં કન્ડક્ટર સિંગલ-ટર્ન પ્રાઈમરી વાઇન્ડિંગ તરીકે કામ કરે છે, જે કરંટના પ્રમાણમાં મેગ્નેટિક ફ્લક્સ બનાવે છે.
\end{itemize}
\end{solutionbox}

\begin{mnemonicbox}
\mnemonic{CLASP: Conductor-Loop Amperes Sensed by Primary-secondary relationship}
\end{mnemonicbox}

\questionmarks{2(c)}{7}{યોગ્ય ડાયાગ્રામ સાથે ઇન્ટીગ્રેટીંગ પ્રકારના DVMનું કાર્ય અને ફાયદાઓનું વર્ણન કરો.}

\begin{solutionbox}
ઇન્ટિગ્રેટિંગ-ટાઇપ ડિજિટલ વોલ્ટમીટર ડ્યુઅલ-સ્લોપ ઇન્ટિગ્રેશન વડે એનાલોગ વોલ્ટેજને ડિજિટલ વેલ્યુમાં રૂપાંતરિત કરે છે.

\textbf{બ્લોક ડાયાગ્રામ:}
\begin{center}
\begin{tikzpicture}[auto, node distance=2cm, every node/.style={draw, rectangle, minimum height=1cm, align=center}]
    \node (buf) {Input Buffer};
    \node [right of=buf, node distance=2.5cm] (int) {Integrator};
    \node [right of=int, node distance=2.5cm] (cmp) {Comparator};
    \node [right of=cmp, node distance=2.5cm] (cnt) {Counter};
    \node [right of=cnt, node distance=2cm] (disp) {Display};
    
    \node [below of=int] (log) {Control Logic};
    \node [left of=log, node distance=2.5cm] (ref) {Ref Voltage};
    
    \draw [->] (buf) -- (int);
    \draw [->] (int) -- (cmp);
    \draw [->] (cmp) -- (cnt);
    \draw [->] (cnt) -- (disp);
    
    \draw [->] (log) -| (ref);
    \draw [->] (ref) |- (int); % Switch conceptually
    \draw [->] (cmp) |- (log);
    \draw [->] (log) -| (cnt);
\end{tikzpicture}
\captionof{figure}{Integrating DVM}
\end{center}

\textbf{વર્કિંગ પ્રિન્સિપલ:}
\begin{center}
\captionof{table}{ઇન્ટીગ્રેટીંગ DVM ના તબક્કા}
\begin{tabulary}{\linewidth}{|L|L|}
\hline
\textbf{ફેઝ} & \textbf{વર્ણન} \\ \hline
\textbf{1. રન-અપ} & અજ્ઞાત ઇનપુટ વોલ્ટેજનું ફિક્સ્ડ સમય $T_1$ માટે ઇન્ટિગ્રેશન થાય છે \\ \hline
\textbf{2. રન-ડાઉન} & રેફરન્સ વોલ્ટેજ (વિપરીત પોલારિટી) નું આઉટપુટ શૂન્ય થાય ત્યાં સુધી ઇન્ટિગ્રેશન થાય છે \\ \hline
\textbf{3. મેઝરમેન્ટ} & રન-ડાઉનનો સમય $T_2$ ઇનપુટ વોલ્ટેજના પ્રમાણમાં હોય છે ($V_{in} = V_{ref} \frac{T_2}{T_1}$) \\ \hline
\textbf{4. ડિસ્પ્લે} & ગણતરી કરેલ ડિજિટલ વેલ્યુ પ્રદર્શિત થાય છે \\ \hline
\end{tabulary}
\end{center}

\textbf{ફાયદાઓ:}
\begin{itemize}
    \item \textbf{નોઇઝ રિજેક્શન}: પાવર લાઇન નોઇઝ માટે ઉત્તમ રિજેક્શન
    \item \textbf{ચોકસાઈ}: અત્યંત ચોકસાઈ (0.005\% થી 0.05\%)
    \item \textbf{સ્થિરતા}: ઘટક સહનશીલતાથી ઓછી અસર પામે છે
\end{itemize}
\end{solutionbox}

\begin{mnemonicbox}
\mnemonic{RISES: Ramp Integration Samples and Eliminates Spikes}
\end{mnemonicbox}

\questionmarks{2(a OR)}{3}{એનાલોગ વોલ્ટમીટર અને ડિજિટલ વોલ્ટમીટર વચ્ચે તફાવત કરો.}

\begin{solutionbox}
\begin{center}
\captionof{table}{તફાવત: ડિજિટલ vs એનાલોગ વોલ્ટમીટર}
\begin{tabulary}{\linewidth}{|L|L|L|}
\hline
\textbf{પેરામીટર} & \textbf{ડિજિટલ વોલ્ટમીટર} & \textbf{એનાલોગ વોલ્ટમીટર} \\ \hline
\textbf{ડિસ્પ્લે} & ન્યુમેરિક ડિસ્પ્લે (અંકો) & સ્કેલ પર પોઇન્ટર મૂવમેન્ટ \\ \hline
\textbf{રીડિંગ એરર} & કોઈ પેરેલેક્સ એરર નહીં & પેરેલેક્સ એરર ને આધિન \\ \hline
\textbf{રીઝોલ્યુશન} & ઉચ્ચ (ડિજિટ્સની સંખ્યા દ્વારા સીમિત) & સ્કેલ ડિવિઝન દ્વારા મર્યાદિત \\ \hline
\textbf{ચોકસાઈ} & વધુ સારી (સામાન્ય રીતે 0.05\% થી 0.5\%) & ઓછી (સામાન્ય રીતે 1\% થી 3\%) \\ \hline
\textbf{આઉટપુટ} & ઇન્ટરફેસિંગ માટે ડિજિટલ આઉટપુટ આપી શકે છે & સીધું ડિજિટલ આઉટપુટ નથી \\ \hline
\end{tabulary}
\end{center}
\end{solutionbox}

\begin{mnemonicbox}
\mnemonic{DAPPER: Digital Accuracy and Precise readings; Parallax Error in Reading analog}
\end{mnemonicbox}

\questionmarks{2(b OR)}{4}{મૂવિંગ આયર્ન ટાઇપ મીટરનું કન્સ્ટ્રક્શન ડાયાગ્રામ દોરો અને વિગતવાર સમજાવો.}

\begin{solutionbox}
\textbf{મૂવિંગ આયર્ન મીટરનો કન્સ્ટ્રક્શન ડાયાગ્રામ:}
\begin{center}
\begin{tikzpicture}
    \draw (0,0) ellipse (0.5 and 2); % Coil
    \node at (0,2.2) {Field Coil};
    
    \draw [thick, fill=gray!30] (-0.2,0) rectangle (0,1); % Fixed iron
    \node at (-1.5, 0.5) {Fixed Iron}; \draw [->] (-0.8,0.5) -- (-0.2,0.5);
    
    \draw [thick, rotate around={30:(0,0)}, fill=black] (0.2,0) rectangle (0.4,1); % Moving iron (approx)
     \node at (2, 0.5) {Moving Iron}; \draw [->] (1.2,0.5) -- (0.5,0.4);
    
    \draw [thin] (0,0) -- (0,3); % Spindle
    \draw [->, thick] (0,3) -- (2,4); % Pointer
    \draw (1,4) arc (60:120:2); % Scale
    
    \draw [decorate,decoration={coil,aspect=0.3,segment length=1mm,amplitude=1mm}] (-0.5,-0.5) -- (0.5,-0.5); % Spring
    \node at (0,-1) {Control Spring};
\end{tikzpicture}
\captionof{figure}{Repulsion Type MI Instrument}
\end{center}

\textbf{વર્કિંગ પ્રિન્સિપલ અને ઘટકો:}
\begin{itemize}
    \item \textbf{કોઇલ}: કરંટના પ્રમાણમાં મેગ્નેટિક ફિલ્ડ ઉત્પન્ન કરે છે
    \item \textbf{આયર્ન વેન્સ}: બે સોફ્ટ આયર્ન પીસ (એક ફિક્સ્ડ, એક હલનચલન કરી શકે તેવું)
    \item \textbf{મૂવમેન્ટ}: સમાન રીતે મેગ્નેટાઇઝ્ડ આયર્ન પીસ વચ્ચે મેગ્નેટિક રિપલ્શન
    \item \textbf{કંટ્રોલ}: સ્પ્રિંગ દ્વારા વિરોધી ટોર્ક પ્રદાન કરે છે
    \item \textbf{ડેમ્પિંગ}: એર ફ્રિક્શન ડેમ્પિંગ મેકેનિઝમ
\end{itemize}
\end{solutionbox}

\begin{mnemonicbox}
\mnemonic{MIRROR: Magnetic Interaction Requires Repulsion/attraction Of Related iron pieces}
\end{mnemonicbox}

\questionmarks{2(c OR)}{7}{એનર્જી મીટરના કન્સ્ટ્રક્શન ડાયાગ્રામનું વર્ણન કરો અને વિગતવાર સમજાવો.}

\begin{solutionbox}
ઇલેક્ટ્રોનિક એનર્જી મીટર કિલોવોટ-અવરમાં વીજળી ઊર્જાની ખપત માપે છે.

\textbf{કન્સ્ટ્રક્શન ડાયાગ્રામ:}
\begin{center}
\begin{tikzpicture}[node distance=1.5cm, auto, every node/.style={rectangle, draw, align=center, minimum height=1cm}]
    \node (in) {Input Lines\\(V, I)};
    \node [right of=in, node distance=3cm] (sens) {Voltage \& Current\\Sensors};
    \node [right of=sens, node distance=3cm] (cond) {Signal\\Conditioning};
    \node [below of=cond, node distance=2cm] (proc) {Microcontroller\\/ Processor};
    \node [left of=proc, node distance=3cm] (disp) {Display\\(kWh)};
    \node [right of=proc, node distance=3cm] (led) {Pulse LED};

    \draw [->] (in) -- (sens);
    \draw [->] (sens) -- (cond);
    \draw [->] (cond) -- (proc);
    \draw [->] (proc) -- (disp);
    \draw [->] (proc) -- (led);
\end{tikzpicture}
\captionof{figure}{Electronic Energy Meter}
\end{center}

\textbf{ઘટકો અને કાર્ય:}
\begin{center}
\captionof{table}{એનર્જી મીટરના ઘટકો}
\begin{tabulary}{\linewidth}{|L|L|}
\hline
\textbf{ઘટક} & \textbf{કાર્ય} \\ \hline
\textbf{વોલ્ટેજ સેન્સર} & વોલ્ટેજ માપવા માટે પોટેન્શિયલ ટ્રાન્સફોર્મર અથવા રેઝિસ્ટિવ ડિવાઇડર \\ \hline
\textbf{કરંટ સેન્સર} & કરંટ માપવા માટે કરંટ ટ્રાન્સફોર્મર અથવા શન્ટ રેઝિસ્ટર \\ \hline
\textbf{મલ્ટિપ્લાયર} & ઇન્સ્ટન્ટેનિયસ વોલ્ટેજ અને કરંટ વેલ્યુને ગુણાકાર કરે છે \\ \hline
\textbf{માઇક્રોકંટ્રોલર} & સિગ્નલ પ્રોસેસ કરે છે અને ઊર્જા વપરાશની ગણતરી કરે છે \\ \hline
\textbf{ડિસ્પ્લે} & kWh માં વપરાશ બતાવવા માટે LCD અથવા LED \\ \hline
\end{tabulary}
\end{center}

\textbf{વર્કિંગ પ્રિન્સિપલ:}
\begin{itemize}
    \item વોલ્ટેજ અને કરંટ સંબંધિત સેન્સર દ્વારા સેન્સ થાય છે
    \item સિગ્નલ્સનો ગુણાકાર ઇન્સ્ટન્ટેનિયસ પાવર મેળવવા માટે થાય છે
    \item ઊર્જાની ગણતરી માટે સમય પર પાવરનું ઇન્ટિગ્રેશન થાય છે
    \item ઊર્જા કિલોવોટ-અવર (kWh) તરીકે પ્રદર્શિત થાય છે
\end{itemize}
\end{solutionbox}

\begin{mnemonicbox}
\mnemonic{WATTAGE: Work And Time Tracked As Generated Electrical energy}
\end{mnemonicbox}


% ==================================================================
% QUESTION 3
% ==================================================================

\questionmarks{3(a)}{3}{ફ્રીકવંસી માપન અને ફેઝ એંગલ માપન માટે લિસાજસ પેટર્ન લાગુ કરો.}

\begin{solutionbox}
ઓસિલોસ્કોપ સ્ક્રીન પર લિસાજસ પેટર્ન ફ્રીકવન્સી રેશિયો અને ફેઝ ડિફરન્સ માપવામાં મદદ કરે છે.

\textbf{ફ્રીકવન્સી મેઝરમેન્ટ:}
\begin{itemize}
    \item $f_y / f_x = \frac{\text{Y-એક્સિસ પર ટેન્જન્ટ પોઇન્ટ્સ}}{\text{X-એક્સિસ પર ટેન્જન્ટ પોઇન્ટ્સ}}$
    \item સર્કલ/ઇલિપ્સ = 1:1 રેશિયો
    \item ફિગર 8 = 2:1 રેશિયો
\end{itemize}

\textbf{ફેઝ એંગલ મેઝરમેન્ટ (1:1 રેશિયો):}
\begin{itemize}
    \item પેટર્ન ઇલિપ્સ છે
    \item $\sin \phi = A/B$
    \item $A$: Y-એક્સિસ પર ઇન્ટરસેપ્ટ (સેન્ટરથી)
    \item $B$: Y-એક્સિસ પર મેક્સિમમ ડિફ્લેક્શન
    \item સર્કલ = $90^\circ$, લાઇન = $0^\circ$ અથવા $180^\circ$
\end{itemize}
\end{solutionbox}

\begin{mnemonicbox}
\mnemonic{LIPS: Lissajous Indicates Phase and Signal frequency}
\end{mnemonicbox}

\questionmarks{3(b)}{4}{CRO માં ગ્રેટીક્યુલ્સ અને તેના પ્રકારોના પણ સમજાવો.}

\begin{solutionbox}
ગ્રેટીક્યુલ્સ એ CRO સ્ક્રીન પર માપન માટેના રેફરન્સ માર્કિંગ્સ છે.

\textbf{પ્રકારો:}
\begin{center}
\captionof{table}{CRO Graticules ના પ્રકારો}
\begin{tabulary}{\linewidth}{|L|L|L|}
\hline
\textbf{ગ્રેટીક્યુલ પ્રકાર} & \textbf{વર્ણન} & \textbf{એપ્લિકેશન} \\ \hline
\textbf{ઇન્ટરનલ} & CRT ગ્લાસની અંદર માર્કિંગ્સ & પેરેલેક્સ એરર દૂર કરે છે \\ \hline
\textbf{એક્સટર્નલ} & સ્ક્રીન પર પ્લાસ્ટિક ઓવરલે & બદલી શકાય તેવું, સસ્તું \\ \hline
\textbf{ઇલેક્ટ્રોનિક} & ઇલેક્ટ્રોનિક રીતે જનરેટ થયેલું & ડિજિટલ સ્ટોરેજ ઓસિલોસ્કોપ્સ \\ \hline
\end{tabulary}
\end{center}

\begin{center}
\begin{tikzpicture}[scale=0.5]
    \draw [step=1cm, gray, thin] (0,0) grid (10,8);
    \draw [thick] (5,0) -- (5,8); % Y axis
    \draw [thick] (0,4) -- (10,4); % X axis
    \node at (5,-0.5) {Standard 8x10 div Graticule};
\end{tikzpicture}
\captionof{figure}{CRO Graticule}
\end{center}
\end{solutionbox}

\begin{mnemonicbox}
\mnemonic{GRID: Graticule References for Intensity and Distance}
\end{mnemonicbox}

\questionmarks{3(c)}{7}{ડિજિટલ સ્ટોરેજ ઓસિલોસ્કોપ (DSO) ના બાંધકામ, બ્લોક ડાયાગ્રામ, કાર્ય અને ફાયદાનું વર્ણન કરો.}

\begin{solutionbox}
\textbf{બ્લોક ડાયાગ્રામ:}
\begin{center}
\begin{tikzpicture}[node distance=1.5cm, auto, every node/.style={rectangle, draw, align=center, font=\footnotesize}]
    \node (in) {Input Attenuator};
    \node [right of=in, node distance=2.5cm] (amp) {Vert Amp};
    \node [right of=amp, node distance=2.5cm] (adc) {A/D Converter};
    \node [right of=adc, node distance=2.5cm] (mem) {Memory (RAM)};
    \node [below of=mem] (proc) {Microprocessor};
    \node [right of=mem, node distance=2.5cm] (dac) {D/A Converter};
    \node [below of=dac] (disp) {CRT / Display};
    
    \draw [->] (in) -- (amp);
    \draw [->] (amp) -- (adc);
    \draw [->] (adc) -- (mem);
    \draw [->] (mem) -- (dac);
    \draw [->] (dac) -- (disp);
    \draw [<->] (proc) -- (mem);
    \draw [->] (proc) -- (disp);
\end{tikzpicture}
\captionof{figure}{DSO Block Diagram}
\end{center}

\textbf{વર્કિંગ પ્રિન્સિપલ:}
\begin{enumerate}
    \item \textbf{સિગ્નલ એક્વિઝિશન}: એનાલોગ સિગ્નલ ઉચ્ચ ગતિએ સેમ્પલ કરવામાં આવે છે
    \item \textbf{A/D કન્વર્ઝન}: કન્ટિન્યુઅસ સિગ્નલ ડિસ્ક્રીટ ડિજિટલ વેલ્યુમાં કન્વર્ટ થાય છે
    \item \textbf{સ્ટોરેજ}: ડિજિટલ વેલ્યુ મેમરીમાં સ્ટોર થાય છે
    \item \textbf{પ્રોસેસિંગ}: માઇક્રોપ્રોસેસર સ્ટોર્ડ ડેટાનું એનાલિસિસ કરે છે
    \item \textbf{ડિસ્પ્લે}: ડેટા ડિસ્પ્લે માટે પાછો એનાલોગમાં કન્વર્ટ થાય છે અથવા સીધો LCD પર બતાવાય છે
\end{enumerate}

\textbf{DSOના ફાયદાઓ:}
\begin{center}
\captionof{table}{DSO ના ફાયદા}
\begin{tabulary}{\linewidth}{|L|L|}
\hline
\textbf{ફાયદો} & \textbf{વર્ણન} \\ \hline
\textbf{પ્રી-ટ્રિગર વ્યુઇંગ} & ટ્રિગર ઇવેન્ટ પહેલાનો સિગ્નલ જોઈ શકાય છે \\ \hline
\textbf{સિંગલ-શોટ કેપ્ચર} & ટ્રાન્ઝિઅન્ટ ઇવેન્ટ્સ કેપ્ચર કરી શકાય છે \\ \hline
\textbf{વેવફોર્મ સ્ટોરેજ} & પછીના એનાલિસિસ માટે વેવફોર્મ સેવ કરી શકાય છે \\ \hline
\textbf{સિગ્નલ પ્રોસેસિંગ} & સિગ્નલ્સ પર એડવાન્સ્ડ મેથેમેટિકલ ઓપરેશન્સ \\ \hline
\end{tabulary}
\end{center}
\end{solutionbox}

\begin{mnemonicbox}
\mnemonic{SAMPLE: Storage And Memory Processes Live Events}
\end{mnemonicbox}

\questionmarks{3(a OR)}{3}{CRO અને DSO વચ્ચે તફાવત કરો.}

\begin{solutionbox}
\begin{center}
\captionof{table}{તફાવત: CRO vs DSO}
\begin{tabulary}{\linewidth}{|L|L|L|}
\hline
\textbf{પેરામીટર} & \textbf{એનાલોગ CRO} & \textbf{ડિજિટલ સ્ટોરેજ ઓસિલોસ્કોપ} \\ \hline
\textbf{સિગ્નલ પ્રોસેસિંગ} & રીયલ-ટાઇમ એનાલોગ & ડિજિટાઇઝ્ડ અને સ્ટોર્ડ \\ \hline
\textbf{સ્ટોરેજ કેપેબિલિટી} & કોઈ નહીં (ફક્ત ફોસ્ફર પર્સિસ્ટન્સ) & મેમરીમાં વેવફોર્મ સ્ટોર કરી શકે છે \\ \hline
\textbf{બેન્ડવિડ્થ} & સામાન્ય રીતે સરખી કિંમત રેન્જમાં ઉચ્ચ & સેમ્પલિંગ રેટ દ્વારા મર્યાદિત \\ \hline
\textbf{પ્રી-ટ્રિગર વ્યુ} & શક્ય નથી & ઉપલબ્ધ છે \\ \hline
\textbf{સિંગલ-શોટ ઇવેન્ટ્સ} & કેપ્ચર કરવા મુશ્કેલ & સરળતાથી કેપ્ચર થાય છે \\ \hline
\end{tabulary}
\end{center}
\end{solutionbox}

\begin{mnemonicbox}
\mnemonic{ASPAD: Analog Shows Present; Digital Archives Data}
\end{mnemonicbox}

\questionmarks{3(b OR)}{4}{10:1 પ્રોબનું માળખું વિગતવાર સમજાવો.}

\begin{solutionbox}
10:1 પ્રોબ ઓસિલોસ્કોપની રેન્જ વધારવા માટે સિગ્નલ એમ્પ્લિટ્યુડને 10 ગણું ઘટાડે છે.

\textbf{માળખું:}
\begin{center}
\begin{circuitikz}[auto]
    \node [left] at (0,2) {Tip};
    \draw (0,2) to [R, l=$R_p (9M\Omega)$] (3,2);
    \draw (0.5,2) -- (0.5,3) to [C, l=$C_p (Comp)$] (2.5,3) -- (2.5,2); % tuning cap in parallel
    
    \draw (3,2) -- (5,2) node[right] {To Scope ($1M\Omega$)};
    \draw (3,2) to [C, l=$C_{cable}$] (3,0) node[ground]{};
    \node at (4,1) {Cable};
\end{circuitikz}
\captionof{figure}{10:1 Probe Circuit}
\end{center}

\textbf{ઘટકો:}
\begin{center}
\captionof{table}{પ્રોબના ઘટકો}
\begin{tabulary}{\linewidth}{|L|L|}
\hline
\textbf{ઘટક} & \textbf{વર્ણન} \\ \hline
\textbf{પ્રોબ ટિપ} & મેટલ કોન્ટેક્ટ પોઇન્ટ જે સર્કિટને સ્પર્શ કરે છે \\ \hline
\textbf{કૉમ્પેન્સેશન નેટવર્ક} & ફ્રીકવન્સી કૉમ્પેન્સેશન માટે RC સર્કિટ \\ \hline
\textbf{પ્રોબ બોડી} & ઘટકો માટે ઇન્સ્યુલેટેડ હાઉસિંગ \\ \hline
\textbf{કેબલ} & લો-કેપેસિટન્સ કોએક્સિયલ કેબલ \\ \hline
\end{tabulary}
\end{center}

\textbf{વર્કિંગ પ્રિન્સિપલ:}
\begin{itemize}
    \item ઓસિલોસ્કોપ ઇનપુટ સાથે વોલ્ટેજ ડિવાઇડર બનાવે છે (9MΩ પ્રોબ + 1MΩ સ્કોપ = 10:1 ડિવિઝન)
    \item કૉમ્પેન્સેટિંગ કેપેસિટર ફ્લેટ ફ્રીકવન્સી રિસપોન્સ સુનિશ્ચિત કરે છે
    \item સર્કિટ લોડિંગ ઇફેક્ટ ઘટાડે છે કારણ કે ઇફેક્ટિવ ઇનપુટ ઇમ્પિડન્સ વધે છે
\end{itemize}
\end{solutionbox}

\begin{mnemonicbox}
\mnemonic{TAPER: Ten-to-one Attenuation Preserves and Extends Range}
\end{mnemonicbox}

\questionmarks{3(c OR)}{7}{CROનું બ્લોક ડાયાગ્રામ, કાર્ય અને એપ્લિકેશનનું વર્ણન કરો.}

\begin{solutionbox}
\textbf{બ્લોક ડાયાગ્રામ:}
\begin{center}
\begin{tikzpicture}[auto, node distance=1.5cm, every node/.style={rectangle, draw, align=center, font=\scriptsize}]
    \node (crt) {CRT};
    \node [left of=crt, node distance=3cm] (va) {Vertical Amp};
    \node [below of=crt, node distance=2cm] (ha) {Horizontal Amp};
    \node [below of=va] (dl) {Delay Line};
    \node [left of=va, node distance=2.5cm] (in) {Input};
    \node [below of=dl] (trig) {Trigger};
    \node [left of=ha, node distance=2.5cm] (tb) {Time Base};
    \node [below of=trig] (ps) {Power Supply};
    
    \draw [->] (in) -- (va);
    \draw [->] (va) -- (dl);
    \draw [->] (dl) -- node[above, sloped] {Y-Plates} (crt);
    \draw [->] (va) -- (trig);
    \draw [->] (trig) -- (tb);
    \draw [->] (tb) -- (ha);
    \draw [->] (ha) -- node[right] {X-Plates} (crt);
    \draw [->] (ps) -| (crt);
    \draw [->] (ps) -| (va);
\end{tikzpicture}
\captionof{figure}{CRO Block Diagram}
\end{center}

\textbf{વર્કિંગ પ્રિન્સિપલ:}
\begin{itemize}
    \item \textbf{વર્ટિકલ સિસ્ટમ}: Y-ડિફ્લેક્શન માટે સિગ્નલ એમ્પ્લીફાય કરે છે
    \item \textbf{હોરિઝોન્ટલ સિસ્ટમ}: X-સ્વીપ માટે સો-ટૂથ વેવ (ટાઇમ બેઝ) જનરેટ કરે છે
    \item \textbf{ટ્રિગર}: સિગ્નલ સ્ટાર્ટ સાથે સ્વીપને સિંક્રનાઇઝ કરે છે
    \item \textbf{CRT}: ટ્રેસ પ્રદર્શિત કરે છે
\end{itemize}

\textbf{CROની એપ્લિકેશન:}
\begin{center}
\captionof{table}{CRO ના ઉપયોગો}
\begin{tabulary}{\linewidth}{|L|L|}
\hline
\textbf{એપ્લિકેશન} & \textbf{વર્ણન} \\ \hline
વેવફોર્મ એનાલિસિસ & સિગ્નલ શેપ અને લક્ષણો વિઝ્યુઅલાઇઝ કરવા \\ \hline
ફ્રીકવન્સી મેઝરમેન્ટ & ટાઇમ પીરિયડ માપી ફ્રીકવન્સી ગણવા \\ \hline
ફેઝ મેઝરમેન્ટ & સિગ્નલ્સ વચ્ચે ફેઝ રિલેશનશિપ સરખાવવા \\ \hline
વોલ્ટેજ મેઝરમેન્ટ & સિગ્નલ એમ્પ્લિટ્યુડ માપવા \\ \hline
\end{tabulary}
\end{center}
\end{solutionbox}

\begin{mnemonicbox}
\mnemonic{VIEW: Voltage Inspection and Electrical Waveform observation}
\end{mnemonicbox}


% ==================================================================
% QUESTION 4
% ==================================================================


\questionmarks{4(a)}{3}{ઓપ્ટિકલ એન્કોડરની કાર્ય પદ્ધતિ સમજાવો.}

\begin{solutionbox}
ઓપ્ટિકલ એન્કોડર પોઝિશન અથવા મોશનને ડિટેક્ટ કરવા માટે પ્રકાશનો ઉપયોગ કરે છે.

\textbf{કાર્ય પદ્ધતિ:}
\begin{itemize}
    \item એક સ્લોટેડ ડિસ્ક લાઇટ સોર્સ (LED) અને ફોટોડિટેક્ટરની વચ્ચે ફરે છે
    \item જ્યારે સ્લોટ પસાર થાય છે, લાઇટ ડિટેક્ટર સુધી પહોંચે છે (Logic 1)
    \item જ્યારે ઓપેક (અપારદર્શક) ભાગ આવે છે, લાઇટ બ્લોક થાય છે (Logic 0)
    \item પલ્સેસની સંખ્યા રોટેશન અથવા સ્પીડનું માપ આપે છે
    \item બે ડિટેક્ટર્સ (Quadrature) દિશા પણ નક્કી કરી શકે છે
\end{itemize}
\end{solutionbox}

\begin{mnemonicbox}
\mnemonic{LIGHT: Light Interrupted Generates High-speed Ticks}
\end{mnemonicbox}

\questionmarks{4(b)}{4}{સ્પીડ મેઝરમેન્ટ માટે ડિજિટલ ટેકોમીટર સમજાવો.}

\begin{solutionbox}
ડિજિટલ ટેકોમીટર ફરતી શાફ્ટની સ્પીડ (RPM) માપે છે.

\textbf{બ્લોક ડાયાગ્રામ:}
\begin{center}
\begin{tikzpicture}[auto, node distance=2cm, every node/.style={rectangle, draw, align=center, font=\footnotesize}]
    \node (shaft) {Rotating\\Shaft};
    \node [right of=shaft, node distance=2.5cm] (sens) {Optical/Magnetic\\Sensor};
    \node [right of=sens, node distance=2.5cm] (shaper) {Signal\\Shaper};
    \node [right of=shaper, node distance=2.5cm] (gate) {Gating\\Circuit};
    \node [right of=gate, node distance=2.5cm] (cnt) {Counter};
    \node [below of=cnt] (disp) {Display};
    \node [below of=gate] (time) {Time Base\\Generator};
    
    \draw [dashed] (shaft) -- (sens);
    \draw [->] (sens) -- (shaper);
    \draw [->] (shaper) -- (gate);
    \draw [->] (gate) -- (cnt);
    \draw [->] (cnt) -- (disp);
    \draw [->] (time) -- (gate);
    \draw [->] (time) -| (cnt); % Reset
\end{tikzpicture}
\captionof{figure}{Digital Tachometer}
\end{center}

\textbf{વર્કિંગ પ્રિન્સિપલ:}
\begin{enumerate}
    \item સેન્સર દરેક રોટેશન માટે પલ્સેસ જનરેટ કરે છે
    \item સિગ્નલ શેપર પલ્સેસને સ્ક્વેર વેવમાં કન્વર્ટ કરે છે
    \item ગેટિંગ સર્કિટ ફિક્સ્ડ સમય (દા.ત. 1 સેકન્ડ) માટે પલ્સેસને પસાર થવા દે છે
    \item કાઉન્ટર તે સમયગાળામાં પલ્સેસ ગણે છે
    \item કાઉન્ટ RPM (Revolutions Per Minute) માં સ્કેલ કરીને ડિસ્પ્લે થાય છે
\end{enumerate}
\end{solutionbox}

\begin{mnemonicbox}
\mnemonic{COUNT: Counter Operates Using Numbered Ticks}
\end{mnemonicbox}

\questionmarks{4(c)}{7}{સ્ટ્રેન ગેજ સેન્સર સમજાવો.}

\begin{solutionbox}
સ્ટ્રેન ગેજ એ પેસિવ ટ્રાન્સડ્યુસર છે જે ફોર્સ લાગુ પડતા મિકેનિકલ ઇલોન્ગેશન અથવા કમ્પ્રેશનને કારણે પોતાનો રેઝિસ્ટન્સ બદલે છે.

\textbf{સિદ્ધાંત (પિઝોરેઝિસ્ટિવ ઇફેક્ટ):}
જ્યારે કન્ડક્ટર સ્ટ્રેચ (ખેંચાય) થાય છે, ત્યારે તેની લંબાઈ વધે છે અને આડછેદનું ક્ષેત્રફળ ઘટે છે, જેથી રેઝિસ્ટન્સ વધે છે ($R = \rho L/A$).

\textbf{ગેજ ફેક્ટર (GF):}
સેન્સિટિવિટી ગેજ ફેક્ટર દ્વારા વ્યાખ્યાયિત થાય છે:
\[ GF = \frac{\Delta R / R}{\Delta L / L} = \frac{\Delta R / R}{\epsilon} \]
જ્યાં:
\begin{itemize}
    \item $\epsilon$ = સ્ટ્રેન ($\Delta L / L$)
    \item $\Delta R$ = રેઝિસ્ટન્સમાં ફેરફાર
    \item $R$ = મૂળ રેઝિસ્ટન્સ
\end{itemize}

\textbf{પ્રકારો:}
\begin{itemize}
    \item \textbf{વાયર વાઉન્ડ}: બેકિંગ મટીરીયલ પર પાતળા વાયર
    \item \textbf{ફોઇલ ટાઇપ}: ફોટો-ઈચિંગ ટેકનિકથી બનેલી મેટલ ફોઇલ (સૌથી સામાન્ય)
    \item \textbf{સેમિકન્ડક્ટર}: ઉચ્ચ ગેજ ફેક્ટર અને સેન્સિટિવિટી
\end{itemize}
\end{solutionbox}

\begin{mnemonicbox}
\mnemonic{STRETCH: Strain Tension Resistance Elongation To Calculate Height/force}
\end{mnemonicbox}

\questionmarks{4(a OR)}{3}{ઇન્ડક્ટિવ ટ્રાન્સડ્યુસરનો સિદ્ધાંત સમજાવો.}

\begin{solutionbox}
ઇન્ડક્ટિવ ટ્રાન્સડ્યુસર મેઝરમેન્ટ માટે સેલ્ફ-ઇન્ડક્ટન્સ અથવા મ્યુચ્યુઅલ ઇન્ડક્ટન્સમાં ફેરફારના સિદ્ધાંત પર કામ કરે છે.

\textbf{સિદ્ધાંત:}
ઇન્ડક્ટન્સ $L$ નીચે મુજબ છે:
\[ L = \frac{N^2 \mu A}{l} \]
જ્યાં:
\begin{itemize}
    \item $N$ = આંટાઓની સંખ્યા
    \item $\mu$ = કોરની પરમીએબિલિટી (પારગમ્યતા)
    \item $A$ = કોરનો આડછેદ વિસ્તાર
    \item $l$ = મેગ્નેટિક સર્કિટની લંબાઈ (એર ગેપ)
\end{itemize}
માપવામાં આવતી રાશિ (ડિસ્પ્લેસમેન્ટ) આમાંથી કોઈપણ પેરામીટર (ખાસ કરીને એર ગેપ $l$ અથવા પરમીએબિલિટી $\mu$) માં ફેરફાર કરે છે, જેથી $L$ બદલાય છે.
\end{solutionbox}

\begin{mnemonicbox}
\mnemonic{FLUX: Field Loop Uses X-change in inductance}
\end{mnemonicbox}

\begin{mnemonicbox}
\mnemonic{LOTUS: Linear Output Temperature Units from Semiconductor}
\end{mnemonicbox}

\questionmarks{4(b OR)}{4}{ઓપ્ટિકલ એન્કોડરના ઓપરેશનની ચર્ચા કરો.}

\begin{solutionbox}
SHAFT એન્કોડર્સ શાફ્ટની કોણીય સ્થિતિ (Angular Position) ને ડિજિટલ કોડમાં રૂપાંતરિત કરે છે.

\textbf{ઓપરેશન:}
\begin{center}
\begin{tikzpicture}[auto, node distance=1.5cm]
    \node [cylinder, shape border rotate=90, draw, minimum height=2cm, minimum width=0.5cm] (led) at (0,2) {LED};
    \draw [fill=gray!20] (-2,1) -- (2,1) -- (0,1.5) -- (-2,1); % Disk perspective
    \node at (0,1.2) {Slotted Disk};
    \node [rectangle, draw] (sens) at (0,0) {Phototransistor};
    
    \draw [->, red, dashed] (led) -- (sens);
    
    \node [right] at (2,1) {\textbf{Rotation}};
    \draw [->] (1.5,0.8) arc (-30:30:1);
\end{tikzpicture}
\captionof{figure}{Optical Encoder Principle}
\end{center}

\begin{itemize}
    \item \textbf{લાઇટ સોર્સ (LED)} સ્લોટેડ ડિસ્ક મારફતે પ્રકાશ પસાર કરે છે
    \item \textbf{ડિસ્ક} ફરે તેમ ફોટોડિટેક્ટર્સ લાઇટ પલ્સેસ પ્રાપ્ત કરે છે
    \item \textbf{બે આઉટપુટ ચેનલ્સ} (A અને B) 90° આઉટ ઓફ ફેઝ હોય છે
    \item \textbf{દિશાનું નિર્ધારણ} કયો ચેનલ લીડ કરે છે તેના પરથી થાય છે
    \item \textbf{રિઝોલ્યુશન} ડિસ્ક પરના સ્લોટ્સની સંખ્યા પર આધાર રાખે છે
\end{itemize}
\end{solutionbox}

\begin{mnemonicbox}
\mnemonic{PADS: Pulses from A and Determine Speed}
\end{mnemonicbox}

\questionmarks{4(c OR)}{7}{LVDT ની કામગીરીનું ફાયદા, ગેરફાયદા અને ઉપયોગ સાથે વર્ણન કરો.}

\begin{solutionbox}
LVDT (લિનિયર વેરિએબલ ડિફરેન્શિયલ ટ્રાન્સફોર્મર) એ લિનિયર ડિસ્પ્લેસમેન્ટને ઇલેક્ટ્રિકલ સિગ્નલમાં રૂપાંતરિત કરતું ઇલેક્ટ્રોમિકેનિકલ ટ્રાન્સડ્યુસર છે.

\textbf{કન્સ્ટ્રક્શન:}
\begin{center}
\begin{circuitikz}[auto]
    \draw [thick] (0,0) rectangle (4,1); % Case
    \draw [fill=gray] (1.5,0.2) rectangle (2.5,0.8); % Core
    \node at (2,0.5) {Core};
    \draw [<-] (1.5,0.5) -- (0.5,0.5) node[left] {Motion};
    \draw [->] (2.5,0.5) -- (3.5,0.5);
    
    \draw (0.5,1) -- (0.5,1.5) node[above] {Pri};
    \draw (3.5,1) -- (3.5,1.5) node[above] {Sec};
    \node at (1,1.2) {S1}; \node at (3,1.2) {S2};
    \node at (2,1.2) {P};
\end{circuitikz}
\captionof{figure}{LVDT Structure}
\end{center}

\textbf{ઓપરેશન:}
\begin{enumerate}
    \item પ્રાયમરી કોઇલમાં AC એક્સાઇટેશન આપવામાં આવે છે
    \item કોરની પોઝિશન સેકન્ડરી કોઇલ્સમાં મેગ્નેટિક ફ્લક્સ લિંકેજ નક્કી કરે છે
    \item નલ પોઝિશન પર, આઉટપુટ વોલ્ટેજ શૂન્ય હોય છે ($V_{s1} = V_{s2}$)
    \item ડિસ્પ્લેસમેન્ટ સાથે ડિફરેન્શિયલ આઉટપુટ ($V_{out} = V_{s1} - V_{s2}$) લીનિયરલી વધે છે
\end{enumerate}

\textbf{ફાયદા અને ગેરફાયદા:}
\begin{center}
\captionof{table}{LVDT ના ફાયદા અને ગેરફાયદા}
\begin{tabulary}{\linewidth}{|L|L|}
\hline
\textbf{ફાયદા} & \textbf{ગેરફાયદા} \\ \hline
ફ્રિક્શનલેસ (કોઈ ઘર્ષણ નહીં) & AC એક્સાઇટેશન જરૂરી \\ \hline
ઇનફિનિટ રિઝોલ્યુશન & તાપમાન સેન્સિટિવ \\ \hline
ઉચ્ચ સેન્સિટિવિટી અને મજબૂતાઈ & મર્યાદિત ડાયનેમિક રેન્જ \\ \hline
\end{tabulary}
\end{center}

\textbf{એપ્લિકેશન:} મશીન ટૂલ પોઝિશનિંગ, હાઇડ્રોલિક સિસ્ટમ્સ, એરક્રાફ્ટ કંટ્રોલ.
\end{solutionbox}

\begin{mnemonicbox}
\mnemonic{MOVE-AC: Magnetic Output Varies with Exact Armature Core position}
\end{mnemonicbox}


% ==================================================================
% QUESTION 5
% ==================================================================

\questionmarks{5(a)}{3}{કેપેસીટીવ ટ્રાન્સડ્યુસરનો ઉપયોગ કરીને દબાણ માપનની કામગીરીનું વર્ણન કરો.}

\begin{solutionbox}
કેપેસિટિવ પ્રેશર ટ્રાન્સડ્યુસર દબાણ માપવા માટે કેપેસિટન્સમાં ફેરફારનો ઉપયોગ કરે છે.

\textbf{વર્કિંગ પ્રિન્સિપલ:}
\begin{itemize}
    \item દબાણ ડાયાફ્રામને ડિફોર્મ કરે છે ($d$ બદલાય છે)
    \item $C = \epsilon A / d$ મુજબ કેપેસિટન્સ બદલાય છે
    \item કેપેસિટન્સમાં ફેરફાર ઇલેક્ટ્રિકલ સિગ્નલમાં રૂપાંતરિત થાય છે
\end{itemize}

\textbf{ડાયાગ્રામ:}
\begin{center}
\begin{tikzpicture}
    \draw [thick] (0,0) rectangle (4,2.5); % Housing
    \draw [fill=gray!20] (0.5, 2.5) rectangle (3.5, 2.7); \node at (2, 2.6) {Diaphragm};
    \draw [dashed] (0.5, 2.5) .. controls (2, 2.2) .. (3.5, 2.5); % Deflected
    \draw [thick] (1, 0.5) rectangle (3, 0.7); \node at (2, 0.3) {Fixed Plate};
    \draw [->] (2, 3.2) -- (2, 2.7); \node at (2, 3.4) {Pressure P};
    \node at (2, 1.5) {Air Gap ($d$)};
    \draw (3, 0.7) -- (4.2, 0.7); \draw (3.5, 2.7) -- (4.2, 2.7);
    \node at (4.5, 1.7) {To Circuit};
\end{tikzpicture}
\captionof{figure}{Capacitive Pressure Transducer}
\end{center}

\textbf{એપ્લિકેશન:} ઇન્ડસ્ટ્રિયલ પ્રોસેસ, લેવલ મેઝરમેન્ટ.
\end{solutionbox}

\begin{mnemonicbox}
\mnemonic{CAPS: Capacitance Alters as Pressure Shifts}
\end{mnemonicbox}

\questionmarks{5(b)}{4}{Define rise time, fall time, Pulse width and duty cycle.}

\begin{solutionbox}
\begin{center}
\begin{tikzpicture}[scale=0.8]
    \draw [->] (0,0) -- (6,0) node[right] {t};
    \draw [->] (0,0) -- (0,3) node[above] {V};
    
    \draw [thick] (1,0) -- (1.5,2.5) -- (4,2.5) -- (4.5,0);
    \draw [dashed] (0,0.25) -- (6,0.25) node[right] {10\%};
    \draw [dashed] (0,2.25) -- (6,2.25) node[right] {90\%};
    \draw [dashed] (0,1.25) -- (6,1.25) node[right] {50\%};
    
    \draw [<->] (1.05,0.25) -- (1.45,2.25) node[midway, left] {\scriptsize Rise};
    \draw [<->] (4.05,2.25) -- (4.45,0.25) node[midway, right] {\scriptsize Fall};
    \draw [<->] (1.25,1.25) -- (4.25,1.25) node[midway, above] {Pulse Width};
\end{tikzpicture}
\captionof{figure}{Pulse Characteristics}
\end{center}
\begin{itemize}
    \item \textbf{Rise Time}: 10\% to 90\% of max.
    \item \textbf{Fall Time}: 90\% to 10\% of max.
    \item \textbf{Pulse Width}: Width at 50\%.
    \item \textbf{Duty Cycle}: (Pulse Width / Total Period) $\times 100\%$.
\end{itemize}
\end{solutionbox}

\begin{mnemonicbox}
\mnemonic{RPFD: Rise Pulses, Fall Determines}
\end{mnemonicbox}

\questionmarks{5(c)}{7}{ફંક્શન જનરેટર બ્લોક ડાયાગ્રામની ચર્ચા કરો.}

\begin{solutionbox}
ફંક્શન જનરેટર સાઇન, સ્ક્વેર અને ટ્રાયએંગલ વેવફોર્મ્સ ઉત્પન્ન કરે છે.

\textbf{બ્લોક ડાયાગ્રામ:}
\begin{center}
\begin{tikzpicture}[auto, node distance=2cm, every node/.style={rectangle, draw, align=center, font=\scriptsize}]
    \node (freq) {Frequency\\Control};
    \node [right of=freq, node distance=2.5cm] (ccs) {Upper\\Current Source};
    \node [below of=ccs] (lcs) {Lower\\Current Source};
    \node [right of=ccs, node distance=2.5cm] (int) {Integrator};
    \node [right of=int, node distance=2.5cm] (comp) {Comparator};
    \node [below of=int] (shape) {Sine Shaper};
    \node [right of=comp, node distance=2.5cm] (out) {Output\\Amp};
    
    \draw [->] (freq) -- (ccs); \draw [->] (freq) -- (lcs);
    \draw [->] (ccs) -- (int); \draw [->] (lcs) -- (int);
    \draw [->] (int) -- node [above] {Triangle} (comp);
    \draw [->] (comp) -- node [above] {Square} (ccs); % Feedback
    \draw [->] (int) -- (shape);
    \draw [->] (shape) -| node [near start, below] {Sine} (out);
\end{tikzpicture}
\captionof{figure}{Function Generator Block Diagram}
\end{center}

\textbf{દરેક બ્લોકનું કાર્ય:}
\begin{center}
\captionof{table}{ફંક્શન જનરેટર બ્લોક્સ}
\begin{tabulary}{\linewidth}{|L|L|}
\hline
\textbf{બ્લોક} & \textbf{કાર્ય} \\ \hline
\textbf{ફ્રીકવન્સી કંટ્રોલ} & કરંટ સોર્સિસને કંટ્રોલ કરીને ફ્રીકવન્સી સેટ કરે છે \\ \hline
\textbf{કરંટ સોર્સિસ} & કેપેસિટર (ઇન્ટિગ્રેટર) ને ચાર્જ/ડિસ્ચાર્જ કરે છે \\ \hline
\textbf{ઇન્ટિગ્રેટર} & ટ્રાયએંગલ વેવ ઉત્પન્ન કરે છે \\ \hline
\textbf{કમ્પેરેટર} & ટ્રાયએંગલ વેવને સ્ક્વેર વેવમાં કન્વર્ટ કરે છે \\ \hline
\textbf{શેપિંગ સર્કિટ} & ટ્રાયએંગલ વેવને સાઇન વેવમાં કન્વર્ટ કરે છે \\ \hline
\textbf{આઉટપુટ એમ્પ} & સિગ્નલ લેવલ અને ઇમ્પિડન્સ મેચિંગ આપે છે \\ \hline
\end{tabulary}
\end{center}
\end{solutionbox}

\begin{mnemonicbox}
\mnemonic{FASTEST: Frequency Amplitude Shaping Together Ensures Signal Types}
\end{mnemonicbox}

\questionmarks{5(a OR)}{3}{સ્ટ્રેન ગેજની કામગીરી, બાંધકામની ચર્ચા યોગ્ય આકૃતિઓ સાથે કરો.}

\begin{solutionbox}
સ્ટ્રેન ગેજ મિકેનિકલ ડિફોર્મેશનને ઇલેક્ટ્રિકલ રેઝિસ્ટન્સ ચેન્જમાં રૂપાંતરિત કરે છે.

\textbf{કન્સ્ટ્રક્શન:}
\begin{center}
\begin{tikzpicture}
    \draw [fill=yellow!20] (0,0) rectangle (3,1.5); \node at (1.5,-0.3) {Backing};
    \draw [thick] (0.2, 0.75) -- (0.5, 0.75) -- (0.5, 1.2) -- (2.5, 1.2) -- (2.5, 0.3) -- (0.5, 0.3) -- (0.5, 0.75); % Loop representation
    \node at (1.5, 0.75) {Resistive Grid};
    \draw [thick] (0.2, 0.75) -- (-0.5, 0.75); % Lead
    \draw [thick] (0.5, 0.75) -- (-0.5, 0.5); % Just symbolic
    % Simplified grid
\end{tikzpicture}
\captionof{figure}{Strain Gauge}
\end{center}

\textbf{વર્કિંગ:}
\begin{itemize}
    \item પિઝોરેઝિસ્ટિવ ઇફેક્ટ પર આધારિત: મિકેનિકલ ડિફોર્મેશન સાથે રેઝિસ્ટન્સ બદલાય છે
    \item જ્યારે ઓબ્જેક્ટ સાથે બોન્ડેડ હોય, ત્યારે સ્ટ્રેન ગેજ તેની સાથે ડિફોર્મ થાય છે
    \item ટેન્શન -> રેઝિસ્ટન્સ વધે છે ($+ \Delta R$)
    \item કમ્પ્રેશન -> રેઝિસ્ટન્સ ઘટે છે ($- \Delta R$)
\end{itemize}

\textbf{રીલેશન:} $\Delta R/R = GF \times \epsilon$.
\end{solutionbox}

\begin{mnemonicbox}
\mnemonic{SERB: Strain Effects Resistance by Bonding}
\end{mnemonicbox}

\questionmarks{5(b OR)}{4}{ડિજિટલ IC ટેસ્ટરની કામગીરીનું વર્ણન યોગ્ય આકૃતિઓ સાથે કરો.}

\begin{solutionbox}
ડિજિટલ IC ટેસ્ટર ટેસ્ટ પેટર્ન્સ અપ્લાય કરીને ઇન્ટિગ્રેટેડ સર્કિટ્સની કાર્યક્ષમતા ચકાસે છે.

\textbf{બ્લોક ડાયાગ્રામ:}
\begin{center}
\begin{tikzpicture}[auto, node distance=2cm, every node/.style={rectangle, draw, align=center, font=\footnotesize}]
    \node (cpu) {CPU / Microcontroller};
    \node [right of=cpu, node distance=3cm] (sock) {IC Socket\\(ZIF)};
    \node [above of=cpu] (key) {Keypad};
    \node [left of=cpu] (disp) {Display};
    \node [below of=cpu] (mem) {Test Pattern\\Memory};
    
    \draw [<->] (cpu) -- (sock);
    \draw [->] (key) -- (cpu);
    \draw [->] (cpu) -- (disp);
    \draw [<->] (cpu) -- (mem);
\end{tikzpicture}
\captionof{figure}{IC Tester}
\end{center}

\textbf{વર્કિંગ:}
\begin{enumerate}
    \item IC ટેસ્ટ સોકેટમાં મૂકવામાં આવે છે
    \item CPU ટેસ્ટ પેટર્ન્સ જનરેટ કરે છે
    \item પેટર્ન્સ IC ઇનપુટ્સ પર લાગુ થાય છે
    \item આઉટપુટ રિસ્પોન્સ અપેક્ષિત વેલ્યુ સાથે સરખાવવામાં આવે છે
    \item પરિણામ (PASS/FAIL) ડિસ્પ્લે થાય છે
\end{enumerate}
\end{solutionbox}

\begin{mnemonicbox}
\mnemonic{PIPE: Pattern Input, Pin Examination}
\end{mnemonicbox}

\questionmarks{5(c OR)}{7}{Discuss working of Spectrum Analyzer with suitable diagrams.}

\begin{solutionbox}
Displays Amplitude vs Frequency.

\textbf{Block Diagram (Swept Superheterodyne):}
\begin{center}
\begin{tikzpicture}[node distance=1.5cm, auto, every node/.style={rectangle, draw, align=center, font=\scriptsize}]
    \node (in) {Input};
    \node [right of=in, node distance=2cm] (mix) {Mixer};
    \node [right of=mix, node distance=2cm] (if) {IF Filter};
    \node [right of=if, node distance=2cm] (det) {Detector};
    \node [right of=det, node distance=2cm] (vert) {Vertical};
    \node [below of=mix] (lo) {Local Osc\\(VCO)};
    \node [below of=det] (sweep) {Sweep Gen};
    \node [right of=sweep, node distance=2.5cm] (crt) {CRT};
    
    \draw [->] (in) -- (mix);
    \draw [->] (mix) -- (if);
    \draw [->] (if) -- (det);
    \draw [->] (det) -- (vert);
    \draw [->] (vert) -| (crt);
    \draw [->] (lo) -- (mix);
    \draw [->] (sweep) -- (lo);
    \draw [->] (sweep) -| (crt);
\end{tikzpicture}
\captionof{figure}{Spectrum Analyzer}
\end{center}

\textbf{Working:}
\begin{itemize}
    \item Sweep generator ramps LO frequency and drives X-axis.
    \item Mixer shifts input frequencies to IF.
    \item IF filter selects current frequency component.
    \item Detector recovers amplitude (Y-axis).
\end{itemize}
\end{solutionbox}

\begin{mnemonicbox}
\mnemonic{SHAFT: Sweep, Heterodyne, Analyze Frequency and Time}
\end{mnemonicbox}

\end{document}
