\documentclass[10pt,a4paper]{article}

% content/resources/templates/preamble.tex
\usepackage[margin=0.6in]{geometry}
\author{Milav Dabgar}
\usepackage{amsmath,amssymb,amsthm}
\usepackage{booktabs}
\usepackage{multirow}
\usepackage{xcolor}
\usepackage{tcolorbox}
\tcbuselibrary{breakable,skins}
\usepackage[colorlinks=true,linkcolor=blue]{hyperref}
\usepackage{titlesec}
\usepackage{enumitem}
\usepackage{tikz}
\usepackage{pgfplots}
\usepackage{circuitikz}
\usepackage[version=4]{mhchem}
\usepackage{longtable}
\usepackage{array}
\usepackage{float}
\usepackage{caption}
\usepackage{listings}

\lstset{
  basicstyle=\small\ttfamily,
  breaklines=true,
  breakatwhitespace=false,
  postbreak=\mbox{\textcolor{red}{$\hookrightarrow$}\space},
  float=false,
  numbers=left,
  numberstyle=\tiny\color{gray},
  numbersep=10pt,
  xleftmargin=2em,
  keywordstyle=\color{blue},
  commentstyle=\color{green!60!black},
  stringstyle=\color{purple},
  backgroundcolor=\color{gray!5},
  showstringspaces=false,
  tabsize=2,
  captionpos=b,
  keepspaces=true,
  columns=flexible
}

\pgfplotsset{compat=1.18}
\usetikzlibrary{shapes,arrows,positioning,calc,patterns,decorations.pathmorphing,decorations.markings,arrows.meta}

% Color scheme
\definecolor{headcolor}{RGB}{0,102,204}
\definecolor{keycolor}{RGB}{220,20,60}
\definecolor{solutioncolor}{RGB}{34,139,34}
\definecolor{mnemoniccolor}{RGB}{148,0,211}
\definecolor{codecolor}{RGB}{0,0,100}

% Spacing
\setlength{\parskip}{3pt}
\setlist[itemize]{nosep}
\setlist[enumerate]{nosep}

% Title formatting
\titleformat{\section}{\Large\bfseries\color{headcolor}}{\thesection}{1em}{}
\titleformat{\subsection}{\large\bfseries\color{headcolor}}{\thesubsection}{1em}{}

% Pandoc tightlist compatibility
\providecommand{\tightlist}{%
  \setlength{\itemsep}{0pt}\setlength{\parskip}{0pt}}

% Pandoc longtable compatibility
\newcounter{none}
\def\thenone{}


% content/resources/templates/gujarati-boxes.tex
\usepackage{fontspec}
\usepackage{polyglossia}

% Set Gujarati as main language (document is primarily in Gujarati)
% Note: gloss-gujarati.ldf doesn't exist in polyglossia, but it will use hyphenation patterns
\setdefaultlanguage{gujarati}
\setotherlanguage{english}

% Configure Gujarati font properly
% Use Language=Default to prevent polyglossia from trying to add language-specific features
% that don't exist for Gujarati, which causes "empty feature" warnings
\newfontfamily\gujaratifont[Script=Gujarati,AutoFakeBold=2.5,AutoFakeSlant=0.3]{Noto Sans Gujarati}
\setmainfont[Script=Gujarati,AutoFakeBold=2.5,AutoFakeSlant=0.3]{Noto Sans Gujarati}
% Use Noto Sans Gujarati for monospace to support Gujarati in text
\setmonofont[Scale=0.9]{Noto Sans Gujarati}

% Configure English to use the same font
\newfontfamily\englishfont[Script=Gujarati,AutoFakeBold=2.5,AutoFakeSlant=0.3]{Noto Sans Gujarati}

% Translations for polyglossia
\gappto\captionsgujarati{
  \renewcommand{\tablename}{કોષ્ટક}
  \renewcommand{\figurename}{આકૃતિ}
}

% Helper for TikZ nodes to ensure Gujarati font
\newcommand{\gu}[1]{{\gujaratifont #1}}

% Custom environments
\newtcolorbox{solutionbox}{
    breakable,
    enhanced,
    colback=solutioncolor!5!white,
    colframe=solutioncolor!75!black,
    fonttitle=\bfseries,
    title=જવાબ
}

\newtcolorbox{solutionboxnobreak}{
 colback=solutioncolor!5!white,
 colframe=solutioncolor!75!black,
 fonttitle=\bfseries,
 title=જવાબ
}

\newtcolorbox{keyformula}{
 breakable,
 enhanced,
 colback=keycolor!5!white,
 colframe=keycolor!75!black,
 fonttitle=\bfseries,
 title=રાસાયણિક સમીકરણ/સૂત્ર
}

\newtcolorbox{mnemonicbox}{
 breakable,
 enhanced,
 colback=mnemoniccolor!5!white,
 colframe=mnemoniccolor!75!black,
 fonttitle=\bfseries,
 title=મેમરી ટ્રીક
}


\begin{document}

\begin{center}
{\Huge\bfseries\color{headcolor} Subject Name (Gujarati)}\\[5pt]
{\LARGE 4331102 -- Summer 2025}\\[3pt]
{\large Semester 1 Study Material}\\[3pt]
{\normalsize\textit{Detailed Solutions and Explanations}}
\end{center}

\vspace{10pt}

\subsection*{પ્રશ્ન 1(અ) [3
ગુણ]}\label{uxaaauxab0uxab6uxaa8-1uxa85-3-uxa97uxaa3}

\textbf{Accuracy, Precision, અને Sensitivity ની વ્યાખ્યા આપો.}

\begin{solutionbox}

\begin{itemize}
\tightlist
\item
  \textbf{Accuracy}: માપેલા મૂલ્યની વાસ્તવિક મૂલ્યની નજીકતા.
\item
  \textbf{Precision}: એક જ ઈનપુટ વારંવાર આપવામાં આવે ત્યારે સાધનની એક સરખા
  આઉટપુટ રીડિંગ ફરીથી ઉત્પન્ન કરવાની ક્ષમતા.
\item
  \textbf{Sensitivity}: સાધનના આઉટપુટમાં થતા ફેરફારનો ઈનપુટમાં થતા ફેરફાર
  સાથેનો ગુણોત્તર, જે દર્શાવે છે કે નાના ફેરફાર માટે આઉટપુટમાં કેટલો ફેરફાર થાય છે.
\end{itemize}


{\def\LTcaptype{none} % do not increment counter
\vspace{-5pt}
\captionof{table}{Accuracy અને Precision વચ્ચેના તફાવત}
\vspace{-10pt}
\begin{longtable}[]{@{}lll@{}}
\toprule\noalign{}
પેરામીટર & Accuracy & Precision \\
\midrule\noalign{}
\endhead
\bottomrule\noalign{}
\endlastfoot
વ્યાખ્યા & સાચા મૂલ્યની નજીકતા & માપની પુનરાવર્તિતા \\
ફોકસ & સચોટતા & સુસંગતતા \\
પ્રતિનિધિત્વ & બુલ્સ-આઇના સેન્ટરના હિટ્સ & ક્લસ્ટર્ડ હિટ્સ \\
\end{longtable}
}

\end{solutionbox}
\begin{mnemonicbox}
``APS - Accuracy સત્યતા દર્શાવે છે, Precision પુનરાવર્તિતા
બતાવે છે, Sensitivity નાના ફેરફારો સંકેત આપે છે''

\end{mnemonicbox}
\subsection*{પ્રશ્ન 1(બ) [4
ગુણ]}\label{uxaaauxab0uxab6uxaa8-1uxaac-4-uxa97uxaa3}

\textbf{વ્હીટસ્ટોન બ્રિજના કાર્ય અને મર્યાદાઓ તેના સર્કિટ ડાયાગ્રામ દોરી સમજાવો.}

\begin{solutionbox}

\textbf{કાર્ય}: વ્હીટસ્ટોન બ્રિજ બ્રિજ સર્કિટની બે ભુજાઓને સંતુલિત કરીને અજ્ઞાત અવરોધ
માપે છે.

\textbf{સર્કિટ ડાયાગ્રામ}:

\begin{center}
\textbf{Mermaid Diagram (Code)}
\begin{verbatim}
{Shaded}
{Highlighting}[]
graph LR
    A[Battery] {-{-}{} B[Point A]}
    A {-{-}{} C[Point C]}
    B {-{-}{} D[Point B]}
    B {-{-}{} E[Point D]}
    C {-{-}{} E}
    C {-{-}{} F[Point C]}
    D {-{-}{-} G[Galvanometer]}
    F {-{-}{-} G}
    B {-{-} R1 {-}{-}{-} D}
    D {-{-} R2 {-}{-}{-} C}
    B {-{-} R3 {-}{-}{-} F}
    F {-{-} Rx {-}{-}{-} C}
{Highlighting}
{Shaded}
\end{verbatim}
\end{center}

જ્યારે બ્રિજ સંતુલિત હોય છે: R1/R2 = R3/Rx, તેથી Rx = R3\times(R2/R1)

\textbf{મર્યાદાઓ}:

\begin{itemize}
\tightlist
\item
  \textbf{મર્યાદિત રેન્જ}: ખૂબ ઓછા કે ખૂબ વધારે અવરોધ માટે યોગ્ય નથી
\item
  \textbf{તાપમાન અસરો}: તાપમાન સાથે અવરોધ બદલાય છે
\item
  \textbf{બેટરી ભૂલો}: આઉટપુટ વોલ્ટેજ સ્થિર રહેવું જોઈએ
\item
  \textbf{ગેલ્વેનોમીટર સંવેદનશીલતા}: ડિટેક્ટરની સંવેદનશીલતાથી મર્યાદિત
\end{itemize}

\end{solutionbox}
\begin{mnemonicbox}
``BALR - Balance મહત્વનું છે, Adjust શૂન્ય સુધી, Low/high
અવરોધો સમસ્યારૂપ, Range મર્યાદિત છે''

\end{mnemonicbox}
\subsection*{પ્રશ્ન 1(ક) [7
ગુણ]}\label{uxaaauxab0uxab6uxaa8-1uxa95-7-uxa97uxaa3}

\textbf{તાપમાન માપવા માટે ઉપયોગમાં લેવાતા વિવિધ પ્રકારના ટ્રાન્સડ્યુસર સમજાવો.
નીચેના માટે બાંધકામ અને કાર્ય વિગતવાર સમજાવો: (i) થર્મોકપલ (ii) થર્મિસ્ટર.}

\begin{solutionbox}

\textbf{તાપમાન ટ્રાન્સડ્યુસર પ્રકારો}:

{\def\LTcaptype{none} % do not increment counter
\begin{longtable}[]{@{}
  >{\raggedright\arraybackslash}p{(\linewidth - 8\tabcolsep) * \real{0.1017}}
  >{\raggedright\arraybackslash}p{(\linewidth - 8\tabcolsep) * \real{0.3220}}
  >{\raggedright\arraybackslash}p{(\linewidth - 8\tabcolsep) * \real{0.1186}}
  >{\raggedright\arraybackslash}p{(\linewidth - 8\tabcolsep) * \real{0.2034}}
  >{\raggedright\arraybackslash}p{(\linewidth - 8\tabcolsep) * \real{0.2542}}@{}}
\toprule\noalign{}
\begin{minipage}[b]{\linewidth}\raggedright
પ્રકાર
\end{minipage} & \begin{minipage}[b]{\linewidth}\raggedright
કાર્ય સિદ્ધાંત
\end{minipage} & \begin{minipage}[b]{\linewidth}\raggedright
રેન્જ
\end{minipage} & \begin{minipage}[b]{\linewidth}\raggedright
ફાયદા
\end{minipage} & \begin{minipage}[b]{\linewidth}\raggedright
ગેરફાયદા
\end{minipage} \\
\midrule\noalign{}
\endhead
\bottomrule\noalign{}
\endlastfoot
થર્મોકપલ & સીબેક ઇફેક્ટ & -270^\circC થી 2300^\circC & વિશાળ રેન્જ, મજબૂત & નોન-લિનિયર,
સંદર્ભની જરૂર \\
થર્મિસ્ટર & અવરોધ પરિવર્તન & -50^\circC થી 300^\circC & ઉચ્ચ સંવેદનશીલતા & નોન-લિનિયર,
મર્યાદિત રેન્જ \\
RTD & અવરોધ પરિવર્તન & -200^\circC થી 850^\circC & ઉચ્ચ ચોકસાઈ, લિનિયર & મોંઘું,
સેલ્ફ-હીટિંગ \\
IC સેન્સર & સેમિકન્ડક્ટર & -55^\circC થી 150^\circC & લિનિયર આઉટપુટ, સરળ & મર્યાદિત
રેન્જ \\
\end{longtable}
}

\textbf{(i) થર્મોકપલ}:

\textbf{બાંધકામ}: બે અલગ-અલગ ધાતુના તાર (જેમ કે કોપર-કોન્સ્ટંટન અથવા
આયર્ન-કોન્સ્ટંટન) એક છેડે જોડાયેલા હોય છે જે માપન જંક્શન બનાવે છે અને બીજા છેડે માપન
ઉપકરણ સાથે જોડાયેલા હોય છે.

\begin{center}
\textbf{Mermaid Diagram (Code)}
\begin{verbatim}
{Shaded}
{Highlighting}[]
graph LR
    A[ધાતુ A] {-{-}{-} B[માપન જંક્શન]}
    C[ધાતુ B] {-{-}{-} B}
    A {-{-}{-} D[સંદર્ભ જંક્શન]}
    C {-{-}{-} D}
    D {-{-}{-} E[માપન ઉપકરણ]}
{Highlighting}
{Shaded}
\end{verbatim}
\end{center}

\textbf{કાર્ય}: જ્યારે જંક્શનો અલગ-અલગ તાપમાને હોય છે, ત્યારે તાપમાન તફાવતના
પ્રમાણમાં નાનું વોલ્ટેજ ઉત્પન્ન થાય છે (સીબેક ઇફેક્ટ).

\textbf{મુખ્ય બિંદુઓ}:

\begin{itemize}
\tightlist
\item
  \textbf{સીબેક ઇફેક્ટ}: તાપમાન તફાવત વોલ્ટેજ ઉત્પન્ન કરે છે
\item
  \textbf{કોલ્ડ જંક્શન કોમ્પેન્સેશન}: ચોકસાઈ માટે જરૂરી
\item
  \textbf{પ્રકારો}: J, K, T, E ધાતુના સંયોજનના આધારે
\end{itemize}

\textbf{(ii) થર્મિસ્ટર}:

\textbf{બાંધકામ}: અર્ધવાહક સામગ્રી (મેંગેનીઝ, નિકલ, કોબાલ્ટ જેવા ધાતુ ઓક્સાઇડ્સ)
બીડ, ડિસ્ક અથવા રોડના આકારમાં બે લીડ વાયર સાથે બનાવવામાં આવે છે.

\begin{verbatim}
  Lead Wire        Lead Wire
      |               |
      v               v
    +{-{-}{-}{-}{-}{-}{-}{-}{-}{-}{-}{-}{-}{-}{-}{-}{-}+}
    |   Ceramic or    |
    | Semiconductor   |
    |     Body        |
    +{-{-}{-}{-}{-}{-}{-}{-}{-}{-}{-}{-}{-}{-}{-}{-}{-}+}
\end{verbatim}

\textbf{કાર્ય}: તાપમાન વધવાની સાથે અવરોધ ઘટે છે (NTC પ્રકાર) અથવા તાપમાન સાથે
વધે છે (PTC પ્રકાર).

\textbf{મુખ્ય બિંદુઓ}:

\begin{itemize}
\tightlist
\item
  \textbf{NTC (નેગેટિવ ટેમ્પરેચર કોઇફિશિયન્ટ)}: સૌથી સામાન્ય પ્રકાર
\item
  \textbf{ઉચ્ચ સંવેદનશીલતા}: નાના તાપમાન ફેરફાર માટે મોટો અવરોધ ફેરફાર
\item
  \textbf{નોન-લિનિયર રિસ્પોન્સ}: લિનિયરાઇઝેશન સર્કિટની જરૂર પડે છે
\item
  \textbf{સેલ્ફ-હીટિંગ}: તેમાંથી પસાર થતો પ્રવાહ ગરમી ઉત્પન્ન કરે છે
\end{itemize}

\end{solutionbox}
\begin{mnemonicbox}
``TRIP - થર્મોકપલ જંક્શન તફાવતોને પ્રતિક્રિયા આપે છે,
થર્મિસ્ટર અવરોધમાં તીવ્ર ફેરફાર કરે છે, સેન્સર જે માપવું છે તેના પર લક્ષ્ય કરો''

\end{mnemonicbox}
\subsection*{પ્રશ્ન 1(ક) OR [7
ગુણ]}\label{uxaaauxab0uxab6uxaa8-1uxa95-or-7-uxa97uxaa3}

\textbf{નીચેના sensor ના કાર્યસિદ્ધાંત સમજાવો: Temperature sensor, Gas
sensor, Humidity sensor અને Proximity sensor.}

\begin{solutionbox}

\textbf{સેન્સરની તુલના}:

{\def\LTcaptype{none} % do not increment counter
\begin{longtable}[]{@{}llll@{}}
\toprule\noalign{}
સેન્સરનો પ્રકાર & કાર્ય સિદ્ધાંત & આઉટપુટ & ઉપયોગો \\
\midrule\noalign{}
\endhead
\bottomrule\noalign{}
\endlastfoot
તાપમાન & અવરોધ/વોલ્ટેજ પરિવર્તન & એનાલોગ/ડિજિટલ & HVAC, મેડિકલ ડિવાઇસ \\
ગેસ & રાસાયણિક પ્રતિક્રિયા & અવરોધમાં ફેરફાર & સલામતી સિસ્ટમ, હવા ગુણવત્તા \\
ભેજ & કેપેસિટન્સ/અવરોધ ફેરફાર & એનાલોગ & વેધર સ્ટેશન, HVAC \\
પ્રોક્સિમિટી & ઇલેક્ટ્રોમેગ્નેટિક ફિલ્ડ ડિસરપ્શન & ડિજિટલ & ઓટોમેશન, સુરક્ષા \\
\end{longtable}
}

\textbf{1. તાપમાન સેન્સર (LM35)}:

\begin{itemize}
\tightlist
\item
  \textbf{સિદ્ધાંત}: સેમિકન્ડક્ટર જંક્શન વોલ્ટેજ તાપમાન સાથે બદલાય છે
\item
  \textbf{કાર્ય}: ઇન્ટિગ્રેટેડ સર્કિટ તાપમાનના પ્રમાણમાં આઉટપુટ વોલ્ટેજ આપે છે
  (10mV/^\circC)
\item
  \textbf{લક્ષણો}: લિનિયર આઉટપુટ, બાહ્ય કેલિબ્રેશનની જરૂર નથી
\end{itemize}

\textbf{2. ગેસ સેન્સર (MQ-2)}:

\begin{itemize}
\tightlist
\item
  \textbf{સિદ્ધાંત}: ગેસ અને સેન્સિંગ મટિરિયલ વચ્ચે રાસાયણિક પ્રતિક્રિયા
\item
  \textbf{કાર્ય}: ગેસ અણુઓ અર્ધવાહક ધાતુ ઓક્સાઇડ સાથે ક્રિયા કરે છે, જેનાથી તેનો
  અવરોધ બદલાય છે
\item
  \textbf{ડિટેક્શન}: જ્યારે ગેસનું સાંદ્રતા થ્રેશોલ્ડથી વધે છે, તો આઉટપુટ વોલ્ટેજ બદલાય છે
\end{itemize}

\begin{center}
\textbf{Mermaid Diagram (Code)}
\begin{verbatim}
{Shaded}
{Highlighting}[]
graph LR
    A[ગેસ અણુઓ] {-{-}{} B[સેન્સિંગ લેયર]}
    B {-{-}{} C[અવરોધમાં ફેરફાર]}
    C {-{-}{} D[વોલ્ટેજ આઉટપુટમાં ફેરફાર]}
    D {-{-}{} E[કોમ્પેરેટર સર્કિટ]}
    E {-{-}{} F[એલાર્મ/આઉટપુટ સિગ્નલ]}
{Highlighting}
{Shaded}
\end{verbatim}
\end{center}

\textbf{3. ભેજ સેન્સર (હાઇગ્રોમીટર)}:

\begin{itemize}
\tightlist
\item
  \textbf{સિદ્ધાંત}: ભેજ શોષણ સાથે કેપેસિટન્સ અથવા અવરોધમાં ફેરફાર
\item
  \textbf{કાર્ય}: ડાયલેકટ્રિક મટિરિયલ ભેજ શોષે છે, જેથી ઇલેક્ટ્રિકલ ગુણધર્મો બદલાય છે
\item
  \textbf{પ્રકારો}: કેપેસિટિવ (વધુ ચોક્કસ) અને રેઝિસ્ટિવ (સરળ)
\end{itemize}

\textbf{4. પ્રોક્સિમિટી સેન્સર}:

\begin{itemize}
\tightlist
\item
  \textbf{સિદ્ધાંત}: ભૌતિક સંપર્ક વિના વસ્તુઓનું શોધન
\item
  \textbf{કાર્ય}: ઇલેક્ટ્રોમેગ્નેટિક ફિલ્ડ/બીમ ઉત્સર્જિત કરે છે; જ્યારે વસ્તુ ફિલ્ડમાં
  પ્રવેશે ત્યારે ફેરફારોનું શોધન
\item
  \textbf{પ્રકારો}: ઇન્ડક્ટિવ (ધાતુઓ), કેપેસિટિવ (કોઈપણ સામગ્રી), અલ્ટ્રાસોનિક
  (અંતર)
\end{itemize}

\end{solutionbox}
\begin{mnemonicbox}
``TGHP - તાપમાન વોલ્ટેજ પેદા કરે છે, ગેસ અર્ધવાહકો પર અસર
કરે છે, ભેજ જાળવે છે, પ્રોક્સિમિટી વસ્તુઓને શોધે છે''

\end{mnemonicbox}
\subsection*{પ્રશ્ન 2(અ) [3
ગુણ]}\label{uxaaauxab0uxab6uxaa8-2uxa85-3-uxa97uxaa3}

\textbf{ડીવીએમ(DVM) ના પ્રકારો આપો અને દરેકના ફાયદા જણાવો.}

\begin{solutionbox}

\textbf{ડિજિટલ વોલ્ટમીટર (DVM) પ્રકારો}:

{\def\LTcaptype{none} % do not increment counter
\begin{longtable}[]{@{}lll@{}}
\toprule\noalign{}
DVM પ્રકાર & કાર્ય સિદ્ધાંત & ફાયદા \\
\midrule\noalign{}
\endhead
\bottomrule\noalign{}
\endlastfoot
રેમ્પ ટાઇપ & ઇનપુટને રેફરન્સ રેમ્પ સાથે સરખાવે છે & સરળ ડિઝાઇન, ઓછી કિંમત \\
ઇન્ટિગ્રેટિંગ ટાઇપ & સમય દરમિયાન સરેરાશ માપે છે & સારો નોઈઝ રિજેક્શન \\
સક્સેસિવ એપ્રોક્સિમેશન & બાઇનરી સર્ચ એલ્ગોરિધમ & ઝડપી રૂપાંતરણ \\
ડ્યુઅલ સ્લોપ & ફિક્સ્ડ સમય સાથે ઇન્ટિગ્રેશન & ઉત્કૃષ્ટ નોઈઝ રિજેક્શન \\
\end{longtable}
}

\textbf{મુખ્ય બિંદુઓ}:

\begin{itemize}
\tightlist
\item
  \textbf{રેમ્પ ટાઇપ}: સરળ પરંતુ નોઈઝથી પ્રભાવિત
\item
  \textbf{ઇન્ટિગ્રેટિંગ ટાઇપ}: સામયિક નોઈઝની અસર ઘટાડે છે
\item
  \textbf{સક્સેસિવ એપ્રોક્સિમેશન}: ઝડપી વાંચન, બદલાતા સિગ્નલ માટે સારું
\item
  \textbf{ડ્યુઅલ સ્લોપ}: શ્રેષ્ઠ ચોકસાઈ, મોટાભાગના નોઈઝથી અસર રહિત
\end{itemize}

\end{solutionbox}
\begin{mnemonicbox}
``RISD - રેમ્પ સરળ ડિઝાઇન છે, ઇન્ટિગ્રેટિંગ નોઈઝને અવગણે છે,
સક્સેસિવ ઝડપ સુનિશ્ચિત કરે છે, ડ્યુઅલ હસ્તક્ષેપ સાથે વ્યવહાર કરે છે''

\end{mnemonicbox}
\subsection*{પ્રશ્ન 2(બ) [4
ગુણ]}\label{uxaaauxab0uxab6uxaa8-2uxaac-4-uxa97uxaa3}

\textbf{મેક્સવેલ બ્રીજ દોરો અને સમજાવો.}

\begin{solutionbox}

\textbf{મેક્સવેલ બ્રીજ} સ્ટાન્ડર્ડ કેપેસિટન્સ સાથે સરખામણી કરીને અજ્ઞાત ઇન્ડક્ટન્સને માપે
છે.

\textbf{સર્કિટ ડાયાગ્રામ}:

\begin{center}
\textbf{Mermaid Diagram (Code)}
\begin{verbatim}
{Shaded}
{Highlighting}[]
graph LR
    A[સપ્લાય] {-{-}{} B[પોઇન્ટ B]}
    A {-{-}{} C[પોઇન્ટ D]}
    B {-{-}{} D[પોઇન્ટ A]}
    B {-{-}{} E[પોઇન્ટ C]}
    C {-{-}{} E}
    C {-{-}{} F[પોઇન્ટ D]}
    D {-{-}{-} G[ડિટેક્ટર]}
    F {-{-}{-} G}
    B {-{-} R1 {-}{-}{-} D}
    D {-{-} R2 {-}{-}{-} C}
    B {-{-} R3 {-}{-}{-} F}
    F {-{-} L,R4 {-}{-}{-} C}
{Highlighting}
{Shaded}
\end{verbatim}
\end{center}

\textbf{બેલેન્સ ઇક્વેશન્સ}:

\begin{itemize}
\tightlist
\item
  અજ્ઞાત ઇન્ડક્ટન્સ L = R2 \times R3 \times C
\item
  અવરોધ R4 = R1 \times (R3/R2)
\end{itemize}

\textbf{કાર્ય}:

\begin{itemize}
\tightlist
\item
  બ્રિજમાં R1, R2, R3, અને L,R4 સાથે ચાર ભુજાઓ હોય છે
\item
  જ્યારે બ્રિજ સંતુલિત હોય છે, ત્યારે ડિટેક્ટરમાંથી પ્રવાહ વહેતો નથી
\item
  L અને R4 ના મૂલ્ય બેલેન્સ ઇક્વેશન્સ વડે ગણવામાં આવે છે
\end{itemize}

\textbf{ફાયદાઓ}:

\begin{itemize}
\tightlist
\item
  \textbf{ઉચ્ચ ચોકસાઈ}: મધ્યમ મૂલ્યના ઇન્ડક્ટર્સ માટે સારું
\item
  \textbf{સ્વતંત્ર બેલેન્સ}: અવરોધ અને ઇન્ડક્ટન્સ અલગથી સંતુલિત થાય છે
\end{itemize}

\end{solutionbox}
\begin{mnemonicbox}
``MILL - મેક્સવેલ્સ ઇન્ડક્ટન્સ L = R2R3C જેવું છે, જ્યારે ડિટેક્ટર
ઓછો પ્રવાહ બતાવે છે''

\end{mnemonicbox}
\subsection*{પ્રશ્ન 2(ક) [7
ગુણ]}\label{uxaaauxab0uxab6uxaa8-2uxa95-7-uxa97uxaa3}

\textbf{સક્સેસિવ એપ્રોક્સિમેશન પ્રકારના ડિજિટલ વોલ્ટમીટર (DVM)નો બ્લોક ડાયાગ્રામ
દોરીને તેનું કાર્ય સમજાવો.}

\begin{solutionbox}

\textbf{સક્સેસિવ એપ્રોક્સિમેશન DVM} બાઇનરી સર્ચ એલ્ગોરિધમનો ઉપયોગ કરીને એનાલોગ
ઇનપુટને ડિજિટલ આઉટપુટમાં રૂપાંતરિત કરે છે.

\textbf{બ્લોક ડાયાગ્રામ}:

\begin{center}
\textbf{Mermaid Diagram (Code)}
\begin{verbatim}
{Shaded}
{Highlighting}[]
graph LR
    A[એનાલોગ ઇનપુટ] {-{-}{} B[સિગ્નલ કન્ડિશનિંગ]}
    B {-{-}{} C[સેમ્પલ \& હોલ્ડ]}
    C {-{-}{} D[કોમ્પેરેટર]}
    E[ક્લોક] {-{-}{} F[સક્સેસિવ એપ્રોક્સિમેશન રજિસ્ટર]}
    F {-{-}{} G[D/A કન્વર્ટર]}
    G {-{-}{} D}
    D {-{-}{} F}
    F {-{-}{} H[ડિજિટલ ડિસ્પ્લે]}
    I[રેફરન્સ વોલ્ટેજ] {-{-}{} G}
{Highlighting}
{Shaded}
\end{verbatim}
\end{center}

\textbf{કાર્ય}:

\begin{enumerate}
\tightlist
\item
  \textbf{સિગ્નલ કન્ડિશનિંગ}: ઇનપુટ વોલ્ટેજને માપન રેન્જમાં સ્કેલ કરે છે
\item
  \textbf{સેમ્પલ \& હોલ્ડ}: ક્ષણિક ઇનપુટ મૂલ્યને પકડે છે
\item
  \textbf{SAR (સક્સેસિવ એપ્રોક્સિમેશન રજિસ્ટર)}: બાઇનરી સર્ચ કરે છે
\item
  \textbf{DAC (ડિજિટલ-ટુ-એનાલોગ કન્વર્ટર)}: ડિજિટલ મૂલ્યને એનાલોગમાં રૂપાંતરિત કરે
  છે
\item
  \textbf{કોમ્પેરેટર}: ઇનપુટને DAC આઉટપુટ સાથે સરખાવે છે
\item
  \textbf{ડિજિટલ ડિસ્પ્લે}: અંતિમ ડિજિટલ મૂલ્ય બતાવે છે
\end{enumerate}

\textbf{રૂપાંતરણ પ્રક્રિયા ઉદાહરણ}:

\begin{itemize}
\tightlist
\item
  9V ના 4-બિટ રૂપાંતરણ માટે (0-15V રેન્જ):

  \begin{itemize}
  \tightlist
  \item
    8V (1000) પ્રયાસ કરો \rightarrow ઇનપુટ \textgreater{} 8V \rightarrow 1 રાખો
  \item
    12V (1100) પ્રયાસ કરો \rightarrow ઇનપુટ \textless{} 12V \rightarrow 0 માં બદલો
  \item
    10V (1010) પ્રયાસ કરો \rightarrow ઇનપુટ \textless{} 10V \rightarrow 0 માં બદલો
  \item
    9V (1001) પ્રયાસ કરો \rightarrow ઇનપુટ = 9V \rightarrow 1 રાખો
  \item
    પરિણામ: 1001 (9V)
  \end{itemize}
\end{itemize}

\textbf{ફાયદાઓ}:

\begin{itemize}
\tightlist
\item
  \textbf{ઝડપી રૂપાંતરણ}: ઇનપુટને ધ્યાનમાં લીધા વગર ફિક્સ્ડ રૂપાંતરણ સમય
\item
  \textbf{સારી ચોકસાઈ}: મોટાભાગના ઉપયોગો માટે યોગ્ય
\item
  \textbf{મધ્યમ જટિલતા}: પ્રદર્શન અને કિંમતનું સંતુલન
\end{itemize}

\end{solutionbox}
\begin{mnemonicbox}
``SHARP - સેમ્પલ, હોલ્ડ, એપ્રોક્સિમેટ, રજિસ્ટર સંગ્રહ કરે છે,
પરિણામ રજૂ કરે છે''

\end{mnemonicbox}
\subsection*{પ્રશ્ન 2(અ) OR [3
ગુણ]}\label{uxaaauxab0uxab6uxaa8-2uxa85-or-3-uxa97uxaa3}

\textbf{PMMC સાધનનો કાર્ય સિદ્ધાંત જણાવો અને તેના વિષે સમજાવો.}

\begin{solutionbox}

\textbf{PMMC (પર્મેનન્ટ મેગ્નેટ મૂવિંગ કોઇલ)} સાધનો ઇલેક્ટ્રોમેગ્નેટિક સિદ્ધાંતો પર
આધારિત કાર્ય કરે છે.

\textbf{કાર્ય સિદ્ધાંત}: જ્યારે ચુંબકીય ક્ષેત્રમાં મૂકેલા કોઇલમાંથી પ્રવાહ વહે છે, ત્યારે
એક ટોર્ક ઉત્પન્ન થાય છે જે પ્રવાહના પ્રમાણમાં કોઇલને ફેરવે છે.

\textbf{મુખ્ય ઘટકો}:

\begin{itemize}
\tightlist
\item
  \textbf{કાયમી ચુંબક}: મજબૂત ચુંબકીય ક્ષેત્ર બનાવે છે
\item
  \textbf{મૂવિંગ કોઇલ}: એલ્યુમિનિયમ ફ્રેમ પર વીંટળાયેલી
\item
  \textbf{કંટ્રોલ સ્પ્રિંગ્સ}: પુનઃસ્થાપિત ટોર્ક પ્રદાન કરે છે
\item
  \textbf{પોઇન્ટર}: સ્કેલ પર વાંચન દર્શાવે છે
\end{itemize}

\textbf{આકૃતિ}:

\begin{verbatim}
                  N
    Spring       | |      Spring
      ↓          | |        ↓
    +=================+
    |      |=====|    |
    |      | Coil|    |
    |      |=====|    |
    |               .{-|{-}.}
    |              /     {}
    |              |     |  Pointer
    |              {     /}
    |               {{-}|{-}}
    +=================+
                  | |
                  S
\end{verbatim}

\end{solutionbox}
\begin{mnemonicbox}
``PMMC - કાયમી ચુંબક પ્રવાહ પસાર થાય ત્યારે કોઇલ ફેરવે છે''

\end{mnemonicbox}
\subsection*{પ્રશ્ન 2(બ) OR [4
ગુણ]}\label{uxaaauxab0uxab6uxaa8-2uxaac-or-4-uxa97uxaa3}

\textbf{Schering બ્રીજ દોરો અને સમજાવો.}

\begin{solutionbox}

\textbf{Schering બ્રીજ} કેપેસિટરના કેપેસિટન્સ અને ડિસિપેશન ફેક્ટર માપવા માટે વપરાય
છે.

\textbf{સર્કિટ ડાયાગ્રામ}:

\begin{center}
\textbf{Mermaid Diagram (Code)}
\begin{verbatim}
{Shaded}
{Highlighting}[]
graph LR
    A[AC સપ્લાય] {-{-}{} B[પોઇન્ટ A]}
    A {-{-}{} C[પોઇન્ટ C]}
    B {-{-}{} D[પોઇન્ટ B]}
    B {-{-}{} E[પોઇન્ટ D]}
    C {-{-}{} E}
    C {-{-}{} F[પોઇન્ટ C]}
    D {-{-}{-} G[ડિટેક્ટર]}
    F {-{-}{-} G}
    B {-{-} R1 {-}{-}{-} D}
    D {-{-} C2 {-}{-}{-} C}
    B {-{-} C4,R4 {-}{-}{-} F}
    F {-{-} Cx,Rx {-}{-}{-} C}
{Highlighting}
{Shaded}
\end{verbatim}
\end{center}

\textbf{બેલેન્સ ઇક્વેશન્સ}:

\begin{itemize}
\tightlist
\item
  અજ્ઞાત કેપેસિટન્સ Cx = C2 \times (R1/R4)
\item
  અજ્ઞાત અવરોધ Rx = R4 \times (C4/C2)
\item
ડિસિપેશન ફેક્ટર

D = ωCxRx = ωC4R4

\end{itemize}

\textbf{કાર્ય}:

\begin{itemize}
\tightlist
\item
  ચાર ભુજાઓમાં R1, C2, Cx-Rx, અને C4-R4 હોય છે
\item
  જ્યારે બ્રિજ સંતુલિત હોય છે, ત્યારે ડિટેક્ટરમાંથી પ્રવાહ વહેતો નથી
\item
  Cx અને Rx ના મૂલ્ય બેલેન્સ ઇક્વેશન્સ વડે ગણવામાં આવે છે
\end{itemize}

\textbf{ઉપયોગો}:

\begin{itemize}
\tightlist
\item
  \textbf{કેપેસિટર પરીક્ષણ}: કેપેસિટન્સ અને નુકસાન માપે છે
\item
  \textbf{ઇન્સુલેશન પરીક્ષણ}: ડાયલેક્ટ્રિક ગુણધર્મોનું મૂલ્યાંકન કરે છે
\end{itemize}

\end{solutionbox}
\begin{mnemonicbox}
``SCAN - Schering કેપેસિટન્સ અને ટેન ડેલ્ટા એક સાથે માપે છે''

\end{mnemonicbox}
\subsection*{પ્રશ્ન 2(ક) OR [7
ગુણ]}\label{uxaaauxab0uxab6uxaa8-2uxa95-or-7-uxa97uxaa3}

\textbf{ડ્યુઅલ સ્લોપ ઇન્ટિગ્રેટિંગ પ્રકારના ડિજિટલ વોલ્ટમીટર (DVM) ની આકૃતિ દોરો
અને સમજાવો.}

\begin{solutionbox}

\textbf{ડ્યુઅલ સ્લોપ ઇન્ટિગ્રેટિંગ DVM} એક પ્રકારનું ડિજિટલ વોલ્ટમીટર છે જે ઇન્ટિગ્રેશન
પદ્ધતિનો ઉપયોગ કરીને એનાલોગ ઇનપુટને ડિજિટલ સ્વરૂપમાં રૂપાંતરિત કરે છે.

\textbf{બ્લોક ડાયાગ્રામ}:

\begin{center}
\textbf{Mermaid Diagram (Code)}
\begin{verbatim}
{Shaded}
{Highlighting}[]
graph LR
    A[એનાલોગ ઇનપુટ] {-{-}{} B[ઇનપુટ બફર]}
    B {-{-}{} C[ઇન્ટિગ્રેટર]}
    D[રેફરન્સ વોલ્ટેજ] {-{-}{} E[પોલારિટી સ્વિચ]}
    E {-{-}{} C}
    C {-{-}{} F[કોમ્પેરેટર]}
    G[ઝીરો રેફરન્સ] {-{-}{} F}
    F {-{-}{} H[કંટ્રોલ લોજિક]}
    I[ક્લોક] {-{-}{} H}
    H {-{-}{} E}
    H {-{-}{} J[કાઉન્ટર]}
    J {-{-}{} K[ડિજિટલ ડિસ્પ્લે]}
    H {-{-}{} J}
{Highlighting}
{Shaded}
\end{verbatim}
\end{center}

\textbf{કાર્ય સિદ્ધાંત}:

\begin{enumerate}
\tightlist
\item
  \textbf{પ્રથમ તબક્કો} (ફિક્સ્ડ સમય T1):

  \begin{itemize}
  \tightlist
  \item
    ઇનપુટ વોલ્ટેજ ફિક્સ્ડ સમય T1 માટે ઇન્ટિગ્રેટ થાય છે
  \item
    ઇન્ટિગ્રેટરનું આઉટપુટ = -(1/RC)\intV(in)dt
  \item
    કાઉન્ટર ક્લોક પલ્સ ગણે છે
  \end{itemize}
\item
  \textbf{બીજો તબક્કો} (પરિવર્તનશીલ સમય T2):

  \begin{itemize}
  \tightlist
  \item
    વિરુદ્ધ ધ્રુવતાનું રેફરન્સ વોલ્ટેજ લાગુ કરવામાં આવે છે
  \item
    ઇન્ટિગ્રેટર આઉટપુટ શૂન્ય પર પાછું ફરે છે
  \item
    સમય T2 ઇનપુટ વોલ્ટેજના પ્રમાણમાં હોય છે
  \item
    T2 = T1 \times (Vin/Vref)
  \end{itemize}
\end{enumerate}

\textbf{ફાયદાઓ}:

\begin{itemize}
\tightlist
\item
  \textbf{ઉત્કૃષ્ટ નોઈઝ રિજેક્શન}: ખાસ કરીને પાવર લાઇન ફ્રિક્વન્સી (50/60 Hz)
\item
  \textbf{ઉચ્ચ ચોકસાઈ}: માત્ર રેફરન્સ વોલ્ટેજ અને ક્લોક સ્થિરતા પર આધારિત
\item
  \textbf{ઓટોમેટિક ઝીરો સુધારણા}: સેલ્ફ-કેલિબ્રેટિંગ સુવિધા
\end{itemize}

\textbf{મુખ્ય બિંદુઓ}:

\begin{itemize}
\tightlist
\item
  \textbf{ઇન્ટિગ્રેશન સમય}: સામાન્ય રીતે પાવર લાઇન પીરિયડના ગુણાંક (20ms અથવા
  16.67ms)
\item
  \textbf{રિઝોલ્યુશન}: ક્લોક ફ્રિક્વન્સી અને કાઉન્ટર ક્ષમતા દ્વારા નક્કી થાય છે
\end{itemize}

\end{solutionbox}
\begin{mnemonicbox}
``FIRE - પ્રથમ ઇનપુટ ઇન્ટિગ્રેટ કરો, પછી રેફરન્સ ઇન્ટિગ્રેટ
કરો, જ્યાં સુધી શૂન્ય ન થાય''

\end{mnemonicbox}
\subsection*{પ્રશ્ન 3(અ) [3
ગુણ]}\label{uxaaauxab0uxab6uxaa8-3uxa85-3-uxa97uxaa3}

\textbf{CRO માં ડિલે લાઇન અને ટ્રિગર સર્કિટનું મહત્વ શું છે?}

\begin{solutionbox}

\textbf{ડિલે લાઇન મહત્વ}:

\begin{itemize}
\tightlist
\item
  \textbf{હેતુ}: સ્વીપને ટ્રિગર કરતી ઘટનાઓને પ્રદર્શિત કરવા માટે સિગ્નલમાં વિલંબ
\item
  \textbf{કાર્ય}: ટ્રિગરનું કારણ બનેલા સિગ્નલના અગ્ર કિનારાને જોવાની મંજૂરી આપે છે
\item
  \textbf{અમલીકરણ}: LC નેટવર્ક અથવા માઇક્રોસ્ટ્રિપ સાથે કૃત્રિમ ટ્રાન્સમિશન લાઇન
\end{itemize}

\textbf{ટ્રિગર સર્કિટ મહત્વ}:

\begin{itemize}
\tightlist
\item
  \textbf{હેતુ}: ઇનપુટ સિગ્નલના ચોક્કસ બિંદુએ સ્વીપ શરૂ કરે છે
\item
  \textbf{કાર્ય}: પુનરાવર્તિત તરંગ માટે સ્થિર, સ્થિર ડિસ્પ્લે સુનિશ્ચિત કરે છે
\item
  \textbf{નિયંત્રણો}: લેવલ, સ્લોપ, સોર્સ અને કપલિંગ
\end{itemize}


{\def\LTcaptype{none} % do not increment counter
\vspace{-5pt}
\captionof{table}{ડિલે લાઇન વિરુદ્ધ ટ્રિગર સર્કિટ}
\vspace{-10pt}
\begin{longtable}[]{@{}
  >{\raggedright\arraybackslash}p{(\linewidth - 4\tabcolsep) * \real{0.3793}}
  >{\raggedright\arraybackslash}p{(\linewidth - 4\tabcolsep) * \real{0.3103}}
  >{\raggedright\arraybackslash}p{(\linewidth - 4\tabcolsep) * \real{0.3103}}@{}}
\toprule\noalign{}
\begin{minipage}[b]{\linewidth}\raggedright
ઘટક
\end{minipage} & \begin{minipage}[b]{\linewidth}\raggedright
હેતુ
\end{minipage} & \begin{minipage}[b]{\linewidth}\raggedright
લાભ
\end{minipage} \\
\midrule\noalign{}
\endhead
\bottomrule\noalign{}
\endlastfoot
ડિલે લાઇન & સિગ્નલ પાથમાં વિલંબ & ટ્રિગર પોઇન્ટ સહિત સંપૂર્ણ તરંગ બતાવે છે \\
ટ્રિગર સર્કિટ & સ્વીપ શરૂ કરે છે & સિન્ક્રોનાઇઝ્ડ ટાઇમિંગ સાથે સ્થિર ડિસ્પ્લે બનાવે છે \\
\end{longtable}
}

\end{solutionbox}
\begin{mnemonicbox}
``DT-SS - ડિલે ટુ સી સિગ્નલ, ટ્રિગર સ્ટોપ્સ સ્ક્રીન ડ્રિફ્ટ''

\end{mnemonicbox}
\subsection*{પ્રશ્ન 3(બ) [4
ગુણ]}\label{uxaaauxab0uxab6uxaa8-3uxaac-4-uxa97uxaa3}

\textbf{કેથોડ રે ટ્યુબ (CRT) ની આંતરિક રચના અને કાર્ય સ્વચ્છ આકૃતી સાથે સમજાવો.}

\begin{solutionbox}

\textbf{કેથોડ રે ટ્યુબ (CRT)} ઓસિલોસ્કોપનું હૃદય છે જે વિદ્યુત સિગ્નલોને દૃશ્ય પ્રદર્શનમાં
રૂપાંતરિત કરે છે.

\textbf{રચના આકૃતિ}:

\begin{verbatim}
       Electron Gun                Deflection System               Screen
      |{-{-}{-}{-}{-}{-}{-}{-}{-}{-}{-}{-}|            |{-}{-}{-}{-}{-}{-}{-}{-}{-}{-}{-}{-}{-}{-}{-}{-}|              |{-}{-}{-}{-}{-}{-}{-}|}
 +{-{-}{-}{-}|{-}{-}{-}{-}{-}{-}{-}{-}{-}{-}{-}{-}|{-}{-}{-}{-}{-}{-}{-}{-}{-}{-}{-}{-}|{-}{-}{-}{-}{-}{-}{-}{-}{-}{-}{-}{-}{-}{-}{-}{-}|{-}{-}{-}{-}{-}{-}{-}{-}{-}{-}{-}{-}{-}{-}|{-}{-}{-}{-}{-}{-}{-}|{-}{-}{-}{-}+}
 |    |            |            |                |              |       |    |
 |    V            V            V                V              V       |    |
 |  +{-{-}{-}{-}+      +{-}{-}{-}+        +{-}{-}{-}{-}+          +{-}{-}{-}{-}+          +{-}{-}{-}{-}+     |    |}
 |  |Cath|      |   |        |    |          |    |          |    |     |    |
 |  |ode |{-{-}{-}{-}{-}| F |{-}{-}{-}{-}{-}{-}{-}| VA |{-}{-}{-}{-}{-}{-}{-}{-}{-}| H  |{-}{-}{-}{-}{-}{-}{-}{-}{-}| P  |     |    |}
 |  |    |      |   |        |    |    |     |Def.|    |     |Scr.|     |    |
 |  +{-{-}{-}{-}+      +{-}{-}{-}+        +{-}{-}{-}{-}+    |     +{-}{-}{-}{-}+    |     +{-}{-}{-}{-}+     |    |}
 |                                     |               |                |    |
 |                                   +{-{-}{-}{-}+         +{-}{-}{-}{-}+              |    |}
 |                                   |    |         |    |              |    |
 |                                   | V  |         |Flu.|              |    |
 |                                   |Def.|{-{-}{-}{-}{-}{-}{-}{-}|Scr.|              |    |}
 |                                   |    |         |    |              |    |
 |                                   +{-{-}{-}{-}+         +{-}{-}{-}{-}+              |    |}
 |                                                                      |    |
 +{-{-}{-}{-}{-}{-}{-}{-}{-}{-}{-}{-}{-}{-}{-}{-}{-}{-}{-}{-}{-}{-}{-}{-}{-}{-}{-}{-}{-}{-}{-}{-}{-}{-}{-}{-}{-}{-}{-}{-}{-}{-}{-}{-}{-}{-}{-}{-}{-}{-}{-}{-}{-}{-}{-}{-}{-}{-}{-}{-}{-}{-}{-}{-}{-}{-}{-}{-}{-}{-}+    |}
      |                                                                      |
    Glass                                                               Vacuum
   Envelope
\end{verbatim}

\textbf{મુખ્ય ઘટકો}:

\begin{enumerate}
\tightlist
\item
  \textbf{ઇલેક્ટ્રોન ગન}:

  \begin{itemize}
  \tightlist
  \item
    \textbf{કેથોડ}: ગરમ ફિલામેન્ટ જે ઇલેક્ટ્રોન્સ છોડે છે
  \item
    \textbf{કંટ્રોલ ગ્રિડ}: ઇલેક્ટ્રોન બીમની તીવ્રતા નિયંત્રિત કરે છે
  \item
    \textbf{ફોકસિંગ એનોડ્સ}: ઇલેક્ટ્રોન્સને બીમમાં કેન્દ્રિત કરે છે
  \item
    \textbf{એક્સિલરેટિંગ એનોડ્સ}: ઇલેક્ટ્રોન વેગ વધારે છે
  \end{itemize}
\item
  \textbf{ડિફ્લેક્શન સિસ્ટમ}:

  \begin{itemize}
  \tightlist
  \item
    \textbf{હોરિઝોન્ટલ ડિફ્લેક્શન પ્લેટ્સ}: X-અક્ષ હલનચલન નિયંત્રિત કરે છે
  \item
    \textbf{વર્ટિકલ ડિફ્લેક્શન પ્લેટ્સ}: Y-અક્ષ હલનચલન નિયંત્રિત કરે છે
  \end{itemize}
\item
  \textbf{સ્ક્રીન}:

  \begin{itemize}
  \tightlist
  \item
    \textbf{ફોસ્ફર કોટિંગ}: ઇલેક્ટ્રોન્સથી અથડાતાં ચમકે છે
  \item
    \textbf{ગ્લાસ એન્વેલોપ}: વેક્યુમ જાળવે છે અને સ્ટ્રક્ચર પ્રદાન કરે છે
  \end{itemize}
\end{enumerate}

\textbf{કાર્ય}:

\begin{itemize}
\tightlist
\item
  ગરમ કેથોડ ઇલેક્ટ્રોન્સ છોડે છે
\item
  કંટ્રોલ ગ્રિડ બીમ તીવ્રતા (બ્રાઇટનેસ) નિયંત્રિત કરે છે
\item
  ફોકસિંગ એનોડ્સ સાંકડો બીમ બનાવે છે
\item
  એક્સિલરેટિંગ એનોડ્સ ઇલેક્ટ્રોન્સને ઝડપી બનાવે છે
\item
  ડિફ્લેક્શન પ્લેટ્સ બીમને ક્ષૈતિજ અને ઊભી રીતે વાળે છે
\item
  ઇલેક્ટ્રોન બીમ ફોસ્ફર સ્ક્રીન પર અથડાય છે, જે દૃશ્યમાન સ્પોટ બનાવે છે
\end{itemize}

\end{solutionbox}
\begin{mnemonicbox}
``EFADS - ઇલેક્ટ્રોન્સ ફ્લાય, એનોડ્સ ડાયરેક્ટ, સ્ક્રીન સિગ્નલ્સ
બતાવે છે''

\end{mnemonicbox}
\subsection*{પ્રશ્ન 3(ક) [7
ગુણ]}\label{uxaaauxab0uxab6uxaa8-3uxa95-7-uxa97uxaa3}

\textbf{બ્લોક ડાયાગ્રામની મદદથી કેથોડ રે ઓસિલોસ્કોપ (CRO) નું કાર્ય સમજાવો અને
દરેક બ્લોકના કાર્યનું વર્ણન કરો.}

\begin{solutionbox}

\textbf{કેથોડ રે ઓસિલોસ્કોપ (CRO)} એક ઇલેક્ટ્રોનિક ઉપકરણ છે જે વિદ્યુત સિગ્નલને
દૃશ્યમાન કરવા અને વિશ્લેષણ કરવા માટે વપરાય છે.

\textbf{બ્લોક ડાયાગ્રામ}:

\begin{center}
\textbf{Mermaid Diagram (Code)}
\begin{verbatim}
{Shaded}
{Highlighting}[]
graph LR
    A[વર્ટિકલ ઇનપુટ] {-{-}{} B[વર્ટિકલ એટેન્યુએટર]}
    B {-{-}{} C[વર્ટિકલ એમ્પ્લિફાયર]}
    C {-{-}{} D[ડિલે લાઇન]}
    D {-{-}{} E[વર્ટિકલ ડિફ્લેક્શન પ્લેટ્સ]}
    F[ટ્રિગર સર્કિટ] {-{-}{} G[ટાઇમ બેઝ જનરેટર]}
    G {-{-}{} H[હોરિઝોન્ટલ એમ્પ્લિફાયર]}
    H {-{-}{} I[હોરિઝોન્ટલ ડિફ્લેક્શન પ્લેટ્સ]}
    J[એક્સટર્નલ ટ્રિગર ઇનપુટ] {-{-}{} F}
    C {-{-}{} F}
    G {-{-}{} K[બ્લેંકિંગ સર્કિટ]}
    K {-{-}{} L[CRT]}
    E {-{-}{} L}
    I {-{-}{} L}
    M[પાવર સપ્લાય] {-{-}{} L}
    M {-{-}{} All}
{Highlighting}
{Shaded}
\end{verbatim}
\end{center}

\textbf{દરેક બ્લોકનું કાર્ય}:

{\def\LTcaptype{none} % do not increment counter
\begin{longtable}[]{@{}ll@{}}
\toprule\noalign{}
બ્લોક & કાર્ય \\
\midrule\noalign{}
\endhead
\bottomrule\noalign{}
\endlastfoot
વર્ટિકલ એટેન્યુએટર & ઇનપુટ સિગ્નલને યોગ્ય સ્તરે સ્કેલ કરે છે \\
વર્ટિકલ એમ્પ્લિફાયર & ડિફ્લેક્શન પ્લેટ્સ માટે સિગ્નલને એમ્પ્લિફાય કરે છે \\
ડિલે લાઇન & ટ્રિગરિંગ ઘટના જોવા માટે સિગ્નલમાં વિલંબ કરે છે \\
ટ્રિગર સર્કિટ & ચોક્કસ બિંદુએ સ્વીપ શરૂ કરે છે \\
ટાઇમ બેઝ જનરેટર & ક્ષૈતિજ સ્વીપ માટે સોટૂથ વેવ બનાવે છે \\
હોરિઝોન્ટલ એમ્પ્લિફાયર & સ્વીપ સિગ્નલને એમ્પ્લિફાય કરે છે \\
બ્લેંકિંગ સર્કિટ & રીટ્રેસ દરમિયાન બીમ કાપે છે \\
CRT & વિદ્યુત સિગ્નલને દૃશ્ય પ્રદર્શનમાં રૂપાંતરિત કરે છે \\
પાવર સપ્લાય & વિવિધ DC વોલ્ટેજ પ્રદાન કરે છે \\
\end{longtable}
}

\textbf{કાર્ય પ્રક્રિયા}:

\begin{enumerate}
\tightlist
\item
  \textbf{સિગ્નલ ઇનપુટ}: વર્ટિકલ એટેન્યુએટર સાથે જોડાયેલ છે
\item
  \textbf{વર્ટિકલ પ્રોસેસિંગ}: સિગ્નલ સ્કેલ, એમ્પ્લિફાય, ડિલે થયેલ
\item
  \textbf{ટ્રિગરિંગ}: ટ્રિગર સર્કિટ ચોક્કસ બિંદુએ ટાઇમ બેઝ શરૂ કરે છે
\item
  \textbf{હોરિઝોન્ટલ સ્વીપ}: ટાઇમ બેઝ ક્ષૈતિજ હલનચલન બનાવે છે
\item
  \textbf{ડિસ્પ્લે}: ઇલેક્ટ્રોન બીમ સ્ક્રીન પર સિગ્નલ ટ્રેસ કરે છે
\item
  \textbf{રીટ્રેસ}: બીમ ઝડપથી પાછો ફરે છે (બ્લેંક) આગલા સ્વીપ માટે
\end{enumerate}

\textbf{નિયંત્રણો}:

\begin{itemize}
\tightlist
\item
  \textbf{વર્ટિકલ}: વોલ્ટ્સ/div, પોઝિશન, કપલિંગ
\item
  \textbf{હોરિઝોન્ટલ}: ટાઇમ/div, પોઝિશન
\item
  \textbf{ટ્રિગર}: લેવલ, સ્લોપ, સોર્સ, મોડ
\end{itemize}

\end{solutionbox}
\begin{mnemonicbox}
``VATH-CDS - વર્ટિકલ એટેન્યુએટ્સ થેન એમ્પ્લિફાઇઝ, હોરિઝોન્ટલ
ક્રિએટ્સ ડિફ્લેક્શન સ્વીપ''

\end{mnemonicbox}
\subsection*{પ્રશ્ન 3(અ) OR [3
ગુણ]}\label{uxaaauxab0uxab6uxaa8-3uxa85-or-3-uxa97uxaa3}

\textbf{કેથોડ રે ઓસિલોસ્કોપ (CRO) અને ડિજિટલ સ્ટોરેજ ઓસિલોસ્કોપ (DSO) વચ્ચેનો
તફાવત આપો.}

\begin{solutionbox}

\textbf{CRO અને DSO વચ્ચેની તુલના}:

{\def\LTcaptype{none} % do not increment counter
\begin{longtable}[]{@{}
  >{\raggedright\arraybackslash}p{(\linewidth - 4\tabcolsep) * \real{0.1410}}
  >{\raggedright\arraybackslash}p{(\linewidth - 4\tabcolsep) * \real{0.4103}}
  >{\raggedright\arraybackslash}p{(\linewidth - 4\tabcolsep) * \real{0.4487}}@{}}
\toprule\noalign{}
\begin{minipage}[b]{\linewidth}\raggedright
પેરામીટર
\end{minipage} & \begin{minipage}[b]{\linewidth}\raggedright
કેથોડ રે ઓસિલોસ્કોપ (CRO)
\end{minipage} & \begin{minipage}[b]{\linewidth}\raggedright
ડિજિટલ સ્ટોરેજ ઓસિલોસ્કોપ (DSO)
\end{minipage} \\
\midrule\noalign{}
\endhead
\bottomrule\noalign{}
\endlastfoot
સિગ્નલ પ્રોસેસિંગ & એનાલોગ & ડિજિટલ (ADC રૂપાંતરણ) \\
સ્ટોરેજ ક્ષમતા & કોઈ નહીં (માત્ર રીયલ-ટાઇમ) & મેમરીમાં તરંગો સંગ્રહ કરી શકે છે \\
બેન્ડવિડ્થ & CRT ટેકનોલોજી દ્વારા મર્યાદિત & ઉચ્ચ બેન્ડવિડ્થ શક્ય છે \\
ડિસ્પ્લે & ફોસ્ફર સ્ક્રીન & LCD/LED સ્ક્રીન \\
વધારાની સુવિધાઓ & મૂળભૂત માપન & અદ્યતન વિશ્લેષણ, FFT, મેથ ફંક્શન્સ \\
\end{longtable}
}

\textbf{મુખ્ય તફાવતો}:

\begin{itemize}
\tightlist
\item
  \textbf{વેવફોર્મ સ્ટોરેજ}: DSO તરંગો સાચવી શકે છે, CRO નહીં
\item
  \textbf{સિગ્નલ પ્રોસેસિંગ}: DSO એનાલોગને ડિજિટલમાં રૂપાંતરિત કરે છે, CRO સંપૂર્ણપણે
  એનાલોગ છે
\item
  \textbf{પ્રી-ટ્રિગર ડિસ્પ્લે}: DSO ટ્રિગર પહેલાંની ઘટનાઓ બતાવી શકે છે
\item
  \textbf{એનાલિસિસ ફીચર્સ}: DSO માપન, મેથ ફંક્શન્સ, FFT પ્રદાન કરે છે
\end{itemize}

\end{solutionbox}
\begin{mnemonicbox}
``DSO-MAPS - ડિજિટલ સ્ટોરેજ ઓસિલોસ્કોપ માપે, એનાલાઇઝ,
પ્રોસેસ, સિગ્નલ્સ સંગ્રહે છે''

\end{mnemonicbox}
\subsection*{પ્રશ્ન 3(બ) OR [4
ગુણ]}\label{uxaaauxab0uxab6uxaa8-3uxaac-or-4-uxa97uxaa3}

\textbf{ફ્રીકવન્સી અને ફેઝ એંગલ CRO (Cathode Ray Oscilloscope)ની મદદથી કેવી
રીતે નિર્ધારિત કરી શકાય છે તે સમજાવો.}

\begin{solutionbox}

\textbf{CRO પર ફ્રીકવન્સી માપન}:

\textbf{પદ્ધતિ}:

\begin{enumerate}
\tightlist
\item
  સિગ્નલને સ્ક્રીન પર દર્શાવો
\item
  હોરિઝોન્ટલ ટાઇમ/div સેટિંગનો ઉપયોગ કરીને સમય પીરિયડ (T) માપો
\item
  ફ્રીકવન્સી ગણો: f = 1/T
\end{enumerate}

\textbf{ગણતરી ઉદાહરણ}:

\begin{itemize}
\tightlist
\item
  જો 3 સાયકલ 6 ડિવિઝન પર 0.5ms/div પર ફેલાય છે
\item
  3 સાયકલનો સમય = 6 div \times 0.5ms/div = 3ms
\item
  1 સાયકલનો સમય (T) = 3ms \div 3 = 1ms
\item
  ફ્રીકવન્સી (f) = 1/T = 1/1ms = 1kHz
\end{itemize}

\textbf{ફેઝ એંગલ માપન}:

\textbf{પદ્ધતિ}:

\begin{enumerate}
\tightlist
\item
  ડ્યુઅલ ચેનલ પર બંને સિગ્નલ દર્શાવો
\item
  સંબંધિત બિંદુઓ વચ્ચેનો સમય તફાવત (Δt) માપો
\item
  સંપૂર્ણ સાયકલનો સમય પીરિયડ (T) માપો
\item
  ફેઝ તફાવત ગણો: φ = (Δt/T) \times 360^\circ
\end{enumerate}

\textbf{આકૃતિ}:

\begin{verbatim}
    Voltage
       \^{}
       |
       |    Signal 1      Signal 2
       |       /{           /}
       |      /  {         /  }
       |     /    {       /    }
       |{-{-}{-}{-}/{-}{-}{-}{-}{-}{-}{-}{-}{-}{-}{-}/{-}{-}{-}{-}{-}{-}{-}{-}{-}{-}{-} Time}
       |   /        {   /        }
       |  /          { /          }
       | /            V            {}
       |/                           {}
       +{-{-}{-}{-}{-}{-}{-}{-}{-}{-}{-}{-}{-}{-}{-}{-}{-}{-}{-}{-}{-}{-}{-}{-}{-}{-}{-}{-}{-}}
           |{{-}{-}Δt{-}{-}|}
           |{{-}{-}{-}{-}{-}{-}{-}T{-}{-}{-}{-}{-}{-}{-}{-}|}
\end{verbatim}

\textbf{ગણતરી}:

\begin{itemize}
\tightlist
\item
જો Δt = 1 div અને 0.2ms/div, અને

T = 5 div અને 0.2ms/div

\item
Δt = 0.2ms અને

T = 1ms

\item
ફેઝ તફાવત:

φ = (0.2ms/1ms) \times 360^\circ = 72^\circ

\end{itemize}

\end{solutionbox}
\begin{mnemonicbox}
``FPL - ફ્રીકવન્સી = પિરિયડની લંબાઈનો વ્યસ્ત, ફેઝ =
(લેગ/પિરિયડ) \times 360''

\end{mnemonicbox}
\subsection*{પ્રશ્ન 3(ક) OR [7
ગુણ]}\label{uxaaauxab0uxab6uxaa8-3uxa95-or-7-uxa97uxaa3}

\textbf{ડિજિટલ સ્ટોરેજ ઓસિલોસ્કોપ (DSO) નો બ્લોક ડાયાગ્રામ દોરો અને દરેક બ્લોકનું
કાર્ય સમજાવો.}

\begin{solutionbox}

\textbf{ડિજિટલ સ્ટોરેજ ઓસિલોસ્કોપ (DSO)} એનાલોગ સિગ્નલને સ્ટોરેજ અને વિશ્લેષણ માટે
ડિજિટલ સ્વરૂપમાં રૂપાંતરિત કરે છે.

\textbf{બ્લોક ડાયાગ્રામ}:

\begin{center}
\textbf{Mermaid Diagram (Code)}
\begin{verbatim}
{Shaded}
{Highlighting}[]
graph LR
    A[એનાલોગ ઇનપુટ] {-{-}{} B[એટેન્યુએટર/એમ્પ્લિફાયર]}
    B {-{-}{} C[એન્ટી{-}એલિયાસિંગ ફિલ્ટર]}
    C {-{-}{} D[એનાલોગ{-}ટુ{-}ડિજિટલ કન્વર્ટર]}
    D {-{-}{} E[એક્વિઝિશન મેમરી]}
    E {-{-}{} F[ડિજિટલ સિગ્નલ પ્રોસેસર]}
    F {-{-}{} G[ડિસ્પ્લે મેમરી]}
    G {-{-}{} H[ડિસ્પ્લે કંટ્રોલર]}
    H {-{-}{} I[LCD ડિસ્પ્લે]}
    J[ટ્રિગર સિસ્ટમ] {-{-}{} D}
    K[માઇક્રોપ્રોસેસર] {-{-}{} F}
    K {-{-}{} J}
    K {-{-}{} H}
    L[કંટ્રોલ પેનલ] {-{-}{} K}
    M[ક્લોક જનરેટર] {-{-}{} D}
    M {-{-}{} K}
{Highlighting}
{Shaded}
\end{verbatim}
\end{center}

\textbf{દરેક બ્લોકનું કાર્ય}:

{\def\LTcaptype{none} % do not increment counter
\begin{longtable}[]{@{}ll@{}}
\toprule\noalign{}
બ્લોક & કાર્ય \\
\midrule\noalign{}
\endhead
\bottomrule\noalign{}
\endlastfoot
એટેન્યુએટર/એમ્પ્લિફાયર & ઇનપુટ સિગ્નલને ADC રેન્જમાં કન્ડિશન કરે છે \\
એન્ટી-એલિયાસિંગ ફિલ્ટર & એલિયાસિંગને રોકવા ઉચ્ચ ફ્રીકવન્સીને દૂર કરે છે \\
ADC & એનાલોગ સિગ્નલને ડિજિટલ સેમ્પલ્સમાં રૂપાંતરિત કરે છે \\
એક્વિઝિશન મેમરી & ડિજિટાઇઝ્ડ વેવફોર્મ ડેટા સ્ટોર કરે છે \\
ડિજિટલ સિગ્નલ પ્રોસેસર & સિગ્નલ્સ પર ગાણિતિક ઓપરેશન કરે છે \\
ડિસ્પ્લે મેમરી & ડિસ્પ્લે માટે પ્રોસેસ કરેલ ડેટા સ્ટોર કરે છે \\
ડિસ્પ્લે કંટ્રોલર & સ્ક્રીન અપડેટ અને ફોર્મેટ નિયંત્રિત કરે છે \\
માઇક્રોપ્રોસેસર & સમગ્ર ઓપરેશન અને યુઝર ઇન્ટરફેસ નિયંત્રિત કરે છે \\
ટ્રિગર સિસ્ટમ & ડેટા એક્વિઝિશન ક્યારે શરૂ કરવું તે નક્કી કરે છે \\
ક્લોક જનરેટર & સેમ્પલિંગ અને પ્રોસેસિંગ માટે ટાઇમિંગ પ્રદાન કરે છે \\
\end{longtable}
}

\textbf{DSO ના ફાયદા}:

\begin{itemize}
\tightlist
\item
  \textbf{સિંગલ-શોટ કેપ્ચર}: ક્ષણિક ઘટનાઓ કેપ્ચર કરી શકે છે
\item
  \textbf{પ્રી-ટ્રિગર વ્યુઇંગ}: ટ્રિગર પોઇન્ટ પહેલાના સિગ્નલને બતાવે છે
\item
  \textbf{વેવફોર્મ સ્ટોરેજ}: પછીના વિશ્લેષણ માટે સિગ્નલ્સ સાચવે છે
\item
  \textbf{અદ્યતન માપન}: ઓટોમેટેડ એમ્પ્લિટ્યુડ, ટાઇમિંગ, વગેરે
\item
  \textbf{ગાણિતિક ફંક્શન્સ}: સરવાળા, FFT, ઇન્ટિગ્રેશન, વગેરે
\end{itemize}

\textbf{કાર્ય પ્રક્રિયા}:

\begin{enumerate}
\tightlist
\item
  એટેન્યુએટર/એમ્પ્લિફાયર દ્વારા ઇનપુટ સિગ્નલ કન્ડિશન થાય છે
\item
  એલિયાસિંગ રોકવા માટે સિગ્નલ ફિલ્ટર થાય છે
\item
  ADC નિયમિત અંતરાલે સિગ્નલનું સેમ્પલિંગ કરે છે
\item
  ડિજિટલ ડેટા એક્વિઝિશન મેમરીમાં સ્ટોર થાય છે
\item
  પ્રોસેસર ડેટાનું વિશ્લેષણ કરે છે અને ડિસ્પ્લે માટે તૈયાર કરે છે
\item
  ડિસ્પ્લે વેવફોર્મ અને માપન બતાવે છે
\end{enumerate}

\end{solutionbox}
\begin{mnemonicbox}
``AADPD - એટેન્યુએટ એનાલોગ, ડિજિટાઇઝ, પ્રોસેસ, ડિસ્પ્લે
સિગ્નલ''

\end{mnemonicbox}
\subsection*{પ્રશ્ન 4(અ) [3
ગુણ]}\label{uxaaauxab0uxab6uxaa8-4uxa85-3-uxa97uxaa3}

\textbf{વિવિધ પ્રકારના ટ્રાન્સડ્યૂસરનું વર્ગીકરણ કરો.}

\begin{solutionbox}

\textbf{ટ્રાન્સડ્યૂસરનું વર્ગીકરણ}:

{\def\LTcaptype{none} % do not increment counter
\begin{longtable}[]{@{}ll@{}}
\toprule\noalign{}
વર્ગીકરણ આધાર & પ્રકારો \\
\midrule\noalign{}
\endhead
\bottomrule\noalign{}
\endlastfoot
ઓપરેશનનો સિદ્ધાંત & મિકેનિકલ, ઇલેક્ટ્રિકલ, થર્મલ, ઓપ્ટિકલ, કેમિકલ \\
ઇનપુટ/આઉટપુટ સંબંધ & પ્રાઇમરી, સેકન્ડરી \\
સિગ્નલ જનરેશન & એક્ટિવ, પેસિવ \\
ઇલેક્ટ્રિકલ પેરામીટર્સ & રેઝિસ્ટિવ, કેપેસિટિવ, ઇન્ડક્ટિવ \\
ટ્રાન્સડક્શન & ફોટોઇલેક્ટ્રિક, ઇલેક્ટ્રોકેમિકલ, થર્મોઇલેક્ટ્રિક \\
\end{longtable}
}

\textbf{મુખ્ય વર્ગીકરણ}:

\begin{enumerate}
\tightlist
\item
  \textbf{ઊર્જા રૂપાંતરણ પર આધારિત}:

  \begin{itemize}
  \tightlist
  \item
    \textbf{એક્ટિવ ટ્રાન્સડ્યૂસર}: બાહ્ય પાવર વિના ઇલેક્ટ્રિકલ આઉટપુટ જનરેટ કરે છે
    (દા.ત., થર્મોકપલ)
  \item
    \textbf{પેસિવ ટ્રાન્સડ્યૂસર}: બાહ્ય પાવરની જરૂર પડે છે (દા.ત., થર્મિસ્ટર)
  \end{itemize}
\item
  \textbf{કાર્ય સિદ્ધાંત પર આધારિત}:

  \begin{itemize}
  \tightlist
  \item
    \textbf{પ્રાઇમરી ટ્રાન્સડ્યૂસર}: ભૌતિક ફેરફારને સીધા ઇલેક્ટ્રિકલ સિગ્નલમાં
    રૂપાંતરિત કરે છે
  \item
    \textbf{સેકન્ડરી ટ્રાન્સડ્યૂસર}: મધ્યવર્તી રૂપાંતરણની જરૂર પડે છે
  \end{itemize}
\end{enumerate}

\end{solutionbox}
\begin{mnemonicbox}
``APRCI - એક્ટિવ/પેસિવ, રેઝિસ્ટિવ/કેપેસિટિવ/ઇન્ડક્ટિવ મુખ્ય
કેટેગરી છે''

\end{mnemonicbox}
\subsection*{પ્રશ્ન 4(બ) [4
ગુણ]}\label{uxaaauxab0uxab6uxaa8-4uxaac-4-uxa97uxaa3}

\textbf{સ્ટ્રેઇન ગેજનું બંધારણ અને કાર્ય સમજાવો.}

\begin{solutionbox}

\textbf{સ્ટ્રેઇન ગેજ} યાંત્રિક સ્ટ્રેઇન (વિરૂપણ)ને વિદ્યુત અવરોધ પરિવર્તનમાં રૂપાંતરિત કરે
છે.

\textbf{બંધારણ}:

\begin{itemize}
\tightlist
\item
  \textbf{ગ્રીડ પેટર્ન}: ઝિગઝેગ પેટર્નમાં પાતળી ફોઇલ અથવા વાયર
\item
  \textbf{બેકિંગ મટીરિયલ}: પોલિમાઇડ અથવા એપોક્સી કેરિયર
\item
  \textbf{લીડ વાયર}: માપન સર્કિટ સાથે જોડાયેલ
\item
  \textbf{એડહેસિવ}: ગેજને ટેસ્ટ સરફેસ સાથે જોડે છે
\end{itemize}

\textbf{આકૃતિ}:

\begin{verbatim}
   Lead Wire                Lead Wire
      |                        |
      v                        v
    +{-{-}{-}{-}{-}{-}{-}{-}{-}{-}{-}{-}{-}{-}{-}{-}{-}{-}{-}{-}{-}{-}{-}{-}{-}{-}{-}{-}+}
    |                            |  Backing
    |   +{-{-}{-}{-}{-}{-}{-}{-}{-}{-}{-}{-}{-}{-}{-}{-}{-}{-}{-}{-}+   |  Material}
    |   | /{//////// |   |}
    |   | {                / |   |}
    |   |  {    Grid     /   |   |}
    |   |   {   Pattern /    |   |}
    |   |    {/////      |   |}
    |   +{-{-}{-}{-}{-}{-}{-}{-}{-}{-}{-}{-}{-}{-}{-}{-}{-}{-}{-}{-}+   |}
    |                            |
    +{-{-}{-}{-}{-}{-}{-}{-}{-}{-}{-}{-}{-}{-}{-}{-}{-}{-}{-}{-}{-}{-}{-}{-}{-}{-}{-}{-}+}
\end{verbatim}

\textbf{કાર્ય સિદ્ધાંત}:

\begin{itemize}
\tightlist
\item
  પિઝોરેઝિસ્ટિવ ઇફેક્ટ પર આધારિત
\item
  જ્યારે ઓબ્જેક્ટ વિરૂપિત થાય છે, ત્યારે ગેજ વિરૂપિત થાય છે
\item
  વિરૂપણ સૂત્ર અનુસાર અવરોધ બદલે છે:

  \begin{itemize}
  \tightlist
  \item
    ΔR/R = GF \times ε
  \item
જ્યાં GF = ગેજ ફેક્ટર,

ε = સ્ટ્રેઇન

  \end{itemize}
\end{itemize}

\textbf{માપન સર્કિટ}:

\begin{itemize}
\tightlist
\item
  સામાન્ય રીતે વ્હીટસ્ટોન બ્રિજમાં જોડાયેલ
\item
  નાના અવરોધ ફેરફારને વોલ્ટેજમાં રૂપાંતરિત કરે છે
\item
  આઉટપુટ વોલ્ટેજ સ્ટ્રેઇનના પ્રમાણમાં હોય છે
\end{itemize}

\textbf{ઉપયોગો}:

\begin{itemize}
\tightlist
\item
  લોડ સેલ, પ્રેશર સેન્સર
\item
  સ્ટ્રક્ચરલ ટેસ્ટિંગ
\item
  મિકેનિકલ સ્ટ્રેસ એનાલિસિસ
\end{itemize}

\end{solutionbox}
\begin{mnemonicbox}
``GRID - ગેજ રેઝિસ્ટન્સ ઇન્ક્રીઝ વિથ ડિફોર્મેશન''

\end{mnemonicbox}
\subsection*{પ્રશ્ન 4(ક) [7
ગુણ]}\label{uxaaauxab0uxab6uxaa8-4uxa95-7-uxa97uxaa3}

\textbf{લિનિયર વેરિએબલ ડિફરન્શિયલ ટ્રાન્સડ્યુસર (LVDT) ને તેના બંધારણ, કાર્યપદ્ધતિ,
ફાયદા અને ઉપયોગો સાથે સમજાવો.}

\begin{solutionbox}

\textbf{લિનિયર વેરિએબલ ડિફરન્શિયલ ટ્રાન્સફોર્મર (LVDT)} એક ઇલેક્ટ્રોમિકેનિકલ સેન્સર
છે જે લિનિયર ડિસ્પ્લેસમેન્ટને ઇલેક્ટ્રિકલ સિગ્નલમાં રૂપાંતરિત કરે છે.

\textbf{બંધારણ}:

\begin{itemize}
\tightlist
\item
  \textbf{પ્રાઇમરી કોઇલ}: કેન્દ્રીય વાઇન્ડિંગ AC સ્ત્રોતથી એક્સાઇટ થાય છે
\item
  \textbf{સેકન્ડરી કોઇલ્સ}: બંને બાજુએ બે સરખા કોઇલ્સ
\item
  \textbf{કોર}: ડિસ્પ્લેસમેન્ટ સાથે હલનચલન કરતી ફેરોમેગ્નેટિક સામગ્રી
\item
  \textbf{હાઉસિંગ}: ટર્મિનલ્સ સહિત સિલિન્ડ્રિકલ શેલ
\end{itemize}

\textbf{આકૃતિ}:

\begin{center}
\textbf{Mermaid Diagram (Code)}
\begin{verbatim}
{Shaded}
{Highlighting}[]
graph LR
    A[AC સ્ત્રોત] {-{-}{} B[પ્રાઇમરી કોઇલ]}
    C[કોર] {-{-}{-} B}
    B {-{-}{-} D[સેકન્ડરી કોઇલ 1]}
    B {-{-}{-} E[સેકન્ડરી કોઇલ 2]}
    D {-{-}{} F[સિગ્નલ કન્ડિશનિંગ]}
    E {-{-}{} F}
    F {-{-}{} G[આઉટપુટ]}
    H[મૂવમેન્ટ] {-{-}{} C}
{Highlighting}
{Shaded}
\end{verbatim}
\end{center}

\textbf{કાર્ય સિદ્ધાંત}:

\begin{itemize}
\tightlist
\item
  પ્રાઇમરી કોઇલને AC વોલ્ટેજ અપાય છે
\item
  ચુંબકીય ફ્લક્સ સેકન્ડરી કોઇલ્સમાં કપલ થાય છે
\item
  કોરની સ્થિતિ કપલિંગ કાર્યક્ષમતા નક્કી કરે છે
\item
  સેકન્ડરીઓ વચ્ચેનું વોલ્ટેજ તફાવત ∝ ડિસ્પ્લેસમેન્ટ
\item
  નલ પોઝિશન (સેન્ટર) પર, સેકન્ડરી વોલ્ટેજ સરખા અને વિરુદ્ધ હોય છે
\end{itemize}

\textbf{ચારિત્ર્યિક વક્ર}:

\begin{verbatim}
     Output
       \^{}
       |                Secondary Voltages
       |                     /
       |                    /
       |                   /
       |                  /
       |                 /
       |{-{-}{-}{-}{-}{-}{-}{-}{-}{-}{-}{-}{-}{-}{-}{-}/{-}{-}{-}{-}{-}{-}{-}{-}{-}{-}{-}{-}{-}{-}{-}{-}{-} Displacement}
       |               /
       |              /
       |             /
     {-{-}|{-}{-}{-}{-}{-}{-}{-}{-}{-}{-}{-}{-}/{-}{-}{-}{-}}
       |           /
  Null Position
\end{verbatim}

\textbf{ફાયદાઓ}:

\begin{itemize}
\tightlist
\item
  \textbf{ઘર્ષણ વિનાનું કાર્ય}: કોઈ યાંત્રિક સંપર્ક નહીં
\item
  \textbf{અનંત રિઝોલ્યુશન}: એનાલોગ આઉટપુટ
\item
  \textbf{ઉચ્ચ લિનિયરિટી}: સીધું પ્રમાણસર આઉટપુટ
\item
  \textbf{મજબૂતાઈ}: આઘાત અને કંપનને પ્રતિરોધક
\item
  \textbf{લાંબો જીવનકાળ}: ઘસાતા ભાગો નથી
\end{itemize}

\textbf{ઉપયોગો}:

\begin{itemize}
\tightlist
\item
  \textbf{ઔદ્યોગિક}: ઓટોમેટેડ મશીન ટૂલ્સ, રોબોટિક્સ
\item
  \textbf{એરોસ્પેસ}: ફ્લાઇટ કંટ્રોલ સિસ્ટમ્સ
\item
  \textbf{સિવિલ એન્જિનિયરિંગ}: સ્ટ્રક્ચરલ ટેસ્ટિંગ
\item
  \textbf{મેટ્રોલોજી}: પ્રિસિઝન મેઝરમેન્ટ ઇન્સ્ટ્રુમેન્ટ્સ
\end{itemize}

\end{solutionbox}
\begin{mnemonicbox}
``LVDT-MAPS - લિનિયર વેરિએબલ ડિફરન્શિયલ ટ્રાન્સફોર્મર
સેકન્ડરી વોલ્ટેજ તફાવત દ્વારા પોઝિશન ચોકસાઇથી માપે છે''

\end{mnemonicbox}
\subsection*{પ્રશ્ન 4(અ) OR [3
ગુણ]}\label{uxaaauxab0uxab6uxaa8-4uxa85-or-3-uxa97uxaa3}

\textbf{પીએચ સેન્સરના ત્રણ ઉપયોગો જણાવો.}

\begin{solutionbox}

\textbf{PH સેન્સરના ઉપયોગો}:

{\def\LTcaptype{none} % do not increment counter
\begin{longtable}[]{@{}
  >{\raggedright\arraybackslash}p{(\linewidth - 4\tabcolsep) * \real{0.3824}}
  >{\raggedright\arraybackslash}p{(\linewidth - 4\tabcolsep) * \real{0.2647}}
  >{\raggedright\arraybackslash}p{(\linewidth - 4\tabcolsep) * \real{0.3529}}@{}}
\toprule\noalign{}
\begin{minipage}[b]{\linewidth}\raggedright
ઉપયોગ
\end{minipage} & \begin{minipage}[b]{\linewidth}\raggedright
હેતુ
\end{minipage} & \begin{minipage}[b]{\linewidth}\raggedright
મહત્વ
\end{minipage} \\
\midrule\noalign{}
\endhead
\bottomrule\noalign{}
\endlastfoot
વોટર ટ્રીટમેન્ટ & પાણીની ગુણવત્તા મોનિટર અને નિયંત્રિત કરવા & સુરક્ષિત પીવાનું પાણી
સુનિશ્ચિત કરે છે \\
કૃષિ & શ્રેષ્ઠ વનસ્પતિ વૃદ્ધિ માટે જમીન મોનિટરિંગ & પાક ઉપજ વધારે છે \\
મેડિકલ ડાયગ્નોસ્ટિક્સ & શરીરના પ્રવાહની એસિડિટી માપન & દર્દીના સ્વાસ્થ્ય માટે
મહત્વપૂર્ણ \\
\end{longtable}
}

\textbf{વધારાના ઉપયોગો}:

\begin{itemize}
\tightlist
\item
  \textbf{ફૂડ પ્રોસેસિંગ}: ઉત્પાદન દરમિયાન ગુણવત્તા નિયંત્રણ
\item
  \textbf{એક્વાકલ્ચર}: પાણીની ઓપ્ટિમલ સ્થિતિ જાળવવી
\item
  \textbf{કેમિકલ મેન્યુફેક્ચરિંગ}: પ્રક્રિયા નિયંત્રણ
\end{itemize}

\end{solutionbox}
\begin{mnemonicbox}
``WAM - વોટર ક્વાલિટી કંટ્રોલ, એગ્રિકલ્ચર સોઇલ ટેસ્ટિંગ,
મેડિકલ ડાયગ્નોસ્ટિક્સ મુખ્ય PH સેન્સર ઉપયોગો છે''

\end{mnemonicbox}
\subsection*{પ્રશ્ન 4(બ) OR [4
ગુણ]}\label{uxaaauxab0uxab6uxaa8-4uxaac-or-4-uxa97uxaa3}

\textbf{કેપેસિટિવ ટ્રાન્સડ્યૂસરનું બંધારણ અને કાર્ય સમજાવો.}

\begin{solutionbox}

\textbf{કેપેસિટિવ ટ્રાન્સડ્યૂસર} ભૌતિક ફેરફારને કેપેસિટન્સ પરિવર્તનમાં રૂપાંતરિત કરે છે જે
વિદ્યુત રીતે માપવામાં આવે છે.

\textbf{બંધારણ}:

\begin{itemize}
\tightlist
\item
  \textbf{સમાંતર પ્લેટ્સ}: બે વાહક પ્લેટ્સ
\item
  \textbf{ડાઇલેક્ટ્રિક મિડિયમ}: હવા, સિરામિક, અથવા અન્ય સામગ્રી
\item
  \textbf{હાઉસિંગ}: સુરક્ષાત્મક આવરણ
\item
  \textbf{ટર્મિનલ્સ}: વિદ્યુત જોડાણો
\end{itemize}

\textbf{આકૃતિ}:

\begin{verbatim}
    Terminal A             Terminal B
        |                     |
        v                     v
    +{-{-}{-}{-}{-}{-}{-}{-}{-}{-}+         +{-}{-}{-}{-}{-}{-}{-}{-}{-}{-}+}
    |          |         |          |
    |  Plate A |         |  Plate B |
    |          |{{-}{-}{-}{-}{-}{-}{-}|          |}
    |          |    d    |          |
    +{-{-}{-}{-}{-}{-}{-}{-}{-}{-}+         +{-}{-}{-}{-}{-}{-}{-}{-}{-}{-}+}
           |                 |
           |    Dielectric   |
           |     Material    |
           |                 |
    +{-{-}{-}{-}{-}{-}{-}{-}{-}{-}{-}{-}{-}{-}{-}{-}{-}{-}{-}{-}{-}{-}{-}{-}{-}{-}+}
    |        Housing           |
    +{-{-}{-}{-}{-}{-}{-}{-}{-}{-}{-}{-}{-}{-}{-}{-}{-}{-}{-}{-}{-}{-}{-}{-}{-}{-}+}
\end{verbatim}

\textbf{કાર્ય સિદ્ધાંત}:

\begin{itemize}
\tightlist
\item
  કેપેસિટન્સ C = ε_{0}εᵣA/d

  \begin{itemize}
  \tightlist
  \item
    ε_{0} = મુક્ત અવકાશની પર્મિટિવિટી
  \item
    εᵣ = ડાઇલેક્ટ્રિકની સાપેક્ષ પર્મિટિવિટી
  \item
    A = પ્લેટ્સનું ક્ષેત્રફળ
  \item
    d = પ્લેટ્સ વચ્ચેનું અંતર
  \end{itemize}
\end{itemize}

\textbf{પરિવર્તનના પ્રકારો}:

\begin{enumerate}
\tightlist
\item
  \textbf{ક્ષેત્રફળ પરિવર્તન}: પ્લેટ્સનું ઓવરલેપ બદલવું
\item
  \textbf{અંતર પરિવર્તન}: પ્લેટ્સ વચ્ચેનું અંતર બદલવું
\item
  \textbf{ડાઇલેક્ટ્રિક પરિવર્તન}: ડાઇલેક્ટ્રિક સામગ્રી બદલવી
\end{enumerate}

\textbf{ઉપયોગો}:

\begin{itemize}
\tightlist
\item
  \textbf{પ્રેશર સેન્સર}: ડાયાફ્રામ પ્લેટ અંતર બદલે છે
\item
  \textbf{લેવલ સેન્સર}: પ્રવાહી સ્તર સાથે ડાઇલેક્ટ્રિક બદલાય છે
\item
  \textbf{હ્યુમિડિટી સેન્સર}: ભેજ સાથે ડાઇલેક્ટ્રિક બદલાય છે
\item
  \textbf{પ્રોક્સિમિટી સેન્સર}: ઓબ્જેક્ટની હાજરી સાથે અંતર બદલાય છે
\end{itemize}

\end{solutionbox}
\begin{mnemonicbox}
``CAD - કેપેસિટન્સ એરિયા, ડિસ્ટન્સ, અથવા ડાઇલેક્ટ્રિક
પરિવર્તન સાથે બદલાય છે''

\end{mnemonicbox}
\subsection*{પ્રશ્ન 4(ક) OR [7
ગુણ]}\label{uxaaauxab0uxab6uxaa8-4uxa95-or-7-uxa97uxaa3}

\textbf{એબ્સોલ્યુટ ઑપ્ટિકલ એન્કોડર શું છે? એના A, B અને C આઉટપુટ વેવફોર્મ વિશે સમજાવો
અને યોગ્ય આકૃતિ આપો. તેની વિગતવાર સમજૂતી આપો.}

\begin{solutionbox}

\textbf{એબ્સોલ્યુટ ઑપ્ટિકલ એન્કોડર} દરેક પોઝિશન માટે અનન્ય ડિજિટલ કોડ જનરેટ કરીને
સીધું એન્ગ્યુલર પોઝિશન માપે છે.

\textbf{બંધારણ}:

\begin{itemize}
\tightlist
\item
  \textbf{કોડ ડિસ્ક}: પારદર્શક/અપારદર્શક સેક્ટર સાથે કૉન્સેન્ટ્રિક ટ્રેક્સ ધરાવે છે
\item
  \textbf{લાઇટ સોર્સ}: ડિસ્કને પ્રકાશિત કરતા LED એરે
\item
  \textbf{ફોટો ડિટેક્ટર્સ}: ડિસ્ક પેટર્ન દ્વારા પ્રકાશને શોધતા સેન્સર્સ
\item
  \textbf{સિગ્નલ કન્ડિશનિંગ}: ફોટોડિટેક્ટર સિગ્નલ્સને ડિજિટલ આઉટપુટમાં રૂપાંતરિત કરે
  છે
\end{itemize}

\textbf{આકૃતિ}:

\begin{center}
\textbf{Mermaid Diagram (Code)}
\begin{verbatim}
{Shaded}
{Highlighting}[]
graph LR
    A[LED લાઇટ સોર્સ] {-{-}{} B[કોડ ડિસ્ક]}
    B {-{-}{} C[ફોટોડિટેક્ટર્સ]}
    C {-{-}{} D[સિગ્નલ કન્ડિશનિંગ સર્કિટ]}
    D {-{-}{} E[ડિજિટલ આઉટપુટ]}
    F[રોટેટિંગ શાફ્ટ] {-{-}{} B}
{Highlighting}
{Shaded}
\end{verbatim}
\end{center}

\textbf{કોડ ડિસ્ક પેટર્ન}:

\begin{verbatim}
          Track C (Index)
            |
     {-{-}{-}{-}{-}{-}{-}|{-}{-}{-}{-}{-}{-}{-}{-}{-}{-}{-}{-}{-}{-}{-}{-}{-}}
            |
          {-{-}+{-}{-}  {-}{-}+{-}{-}  {-}{-}+{-}{-}}
           Track B
     {-{-}{-}{-}{-}{-}{-}|{-}{-}{-}{-}{-}{-}{-}{-}{-}{-}{-}{-}{-}{-}{-}{-}{-}}
            |
          {-++{-}{-}  {-}++{-}{-}  {-}++{-}{-}}
           Track A
     {-{-}{-}{-}{-}{-}{-}|{-}{-}{-}{-}{-}{-}{-}{-}{-}{-}{-}{-}{-}{-}{-}{-}{-}}
            |
          {-+++{-} {-}+++{-} {-}+++{-} {-}}
     {-{-}{-}{-}{-}{-}{-}|{-}{-}{-}{-}{-}{-}{-}{-}{-}{-}{-}{-}{-}{-}{-}{-}{-}}
            |
            V
          Rotation
\end{verbatim}

\textbf{વેવફોર્મ આઉટપુટ્સ}:

{\def\LTcaptype{none} % do not increment counter
\begin{longtable}[]{@{}lll@{}}
\toprule\noalign{}
સિગ્નલ & હેતુ & ચારિત્ર્યિક લક્ષણો \\
\midrule\noalign{}
\endhead
\bottomrule\noalign{}
\endlastfoot
A સિગ્નલ & પોઝિશન માહિતી & સ્ક્વેર વેવ, 50\% ડ્યુટી સાયકલ \\
B સિગ્નલ & દિશા માહિતી & A થી 90^\circ ફેઝ શિફ્ટેડ \\
C સિગ્નલ & રેફરન્સ/ઇન્ડેક્સ & પ્રતિ રિવોલ્યુશન એક પલ્સ \\
\end{longtable}
}

\textbf{આઉટપુટ વેવફોર્મ્સ}:

\begin{verbatim}
    A Signal  \_\_\_\_\_|‾‾‾‾‾|\_\_\_\_\_|‾‾‾‾‾|\_\_\_\_\_|‾‾‾‾‾|\_\_\_\_\_|‾‾‾‾‾|\_\_\_\_\_
    
    B Signal  \_\_|‾‾‾‾‾|\_\_\_\_\_|‾‾‾‾‾|\_\_\_\_\_|‾‾‾‾‾|\_\_\_\_\_|‾‾‾‾‾|\_\_\_\_\_|‾‾
    
    C Signal  \_\_\_\_\_|‾|\_\_\_\_\_\_\_\_\_\_\_\_\_\_\_\_\_\_\_\_\_\_\_\_\_\_\_\_\_\_\_\_\_\_\_\_\_\_\_\_\_\_\_\_\_\_\_\_\_
              
              0^    90^   180^   270^   360^   450^   540^   630^   720^
\end{verbatim}

\textbf{કાર્ય સિદ્ધાંત}:

\begin{itemize}
\tightlist
\item
  A \& B આઉટપુટ ક્વોડ્રેચર સિગ્નલ્સ (90^\circ ફેઝ શિફ્ટ) પ્રદાન કરે છે
\item
  કયો સિગ્નલ આગળ છે તે દ્વારા દિશા નક્કી થાય છે:

  \begin{itemize}
  \tightlist
  \item
    જો A, B થી આગળ હોય: ક્લોકવાઇઝ રોટેશન
  \item
    જો B, A થી આગળ હોય: કાઉન્ટર-ક્લોકવાઇઝ રોટેશન
  \end{itemize}
\item
  પલ્સ ગણીને પોઝિશન નક્કી થાય છે
\item
  C સિગ્નલ રેફરન્સ/હોમ પોઝિશન પ્રદાન કરે છે
\end{itemize}

\textbf{ઉપયોગો}:

\begin{itemize}
\tightlist
\item
  \textbf{CNC મશીન}: ચોક્સાઈવાળું પોઝિશન કંટ્રોલ
\item
  \textbf{રોબોટિક્સ}: જોઇન્ટ એંગલ મેઝરમેન્ટ
\item
  \textbf{કેમેરા સિસ્ટમ્સ}: લેન્સ પોઝિશનિંગ
\item
  \textbf{ઔદ્યોગિક ઓટોમેશન}: મોટર કંટ્રોલ
\end{itemize}

\end{solutionbox}
\begin{mnemonicbox}
``ABC-PDP - એબ્સોલ્યુટ એન્કોડર ટ્રેક્સ A, B, C દિશા,
પોઝિશન, અને રેફરન્સ પલ્સ પ્રદાન કરે છે''

\end{mnemonicbox}
\subsection*{પ્રશ્ન 5(અ) [3
ગુણ]}\label{uxaaauxab0uxab6uxaa8-5uxa85-3-uxa97uxaa3}

\textbf{બેસિક ફ્રિકવન્સી કાઉન્ટરનો કાર્યસિદ્ધાંત સમજાવો.}

\begin{solutionbox}

\textbf{ફ્રિકવન્સી કાઉન્ટર} ચોક્કસ સમય અંતરાલ ઉપર ઘટનાઓ ગણીને ઇનપુટ સિગ્નલની
ફ્રિકવન્સી માપે છે.

\textbf{કાર્ય સિદ્ધાંત}:

\begin{itemize}
\tightlist
\item
  ઇનપુટ સિગ્નલના સાયકલ્સ/પલ્સની સંખ્યા ગણો
\item
  ચોક્કસ ગેટ સમયથી ભાગાકાર કરો
\item
  પરિણામી ફ્રિકવન્સી દર્શાવો
\end{itemize}

\textbf{મૂળભૂત બ્લોક્સ}:

\begin{itemize}
\tightlist
\item
  \textbf{ઇનપુટ કન્ડિશનિંગ}: સિગ્નલને ડિજિટલ લેવલમાં આકાર આપે છે
\item
  \textbf{ગેટ કંટ્રોલ}: ચોક્કસ સમય માટે ગેટ ખોલે છે
\item
  \textbf{કાઉન્ટર}: ગેટ ખુલ્લા સમય દરમિયાન પલ્સ ગણે છે
\item
  \textbf{ટાઇમ બેઝ}: ચોક્કસ ગેટ ટાઇમિંગ ઉત્પન્ન કરે છે
\item
  \textbf{ડિસ્પ્લે}: ફ્રિકવન્સી મૂલ્ય બતાવે છે
\end{itemize}

\textbf{સરળીકૃત આકૃતિ}:

\begin{center}
\textbf{Mermaid Diagram (Code)}
\begin{verbatim}
{Shaded}
{Highlighting}[]
graph LR
    A[ઇનપુટ સિગ્નલ] {-{-}{} B[ઇનપુટ કન્ડિશનિંગ]}
    B {-{-}{} C[AND ગેટ]}
    D[ટાઇમ બેઝ] {-{-}{} E[ગેટ કંટ્રોલ]}
    E {-{-}{} C}
    C {-{-}{} F[કાઉન્ટર]}
    F {-{-}{} G[ડિસ્પ્લે]}
{Highlighting}
{Shaded}
\end{verbatim}
\end{center}

\end{solutionbox}
\begin{mnemonicbox}
``CTPG - કાઉન્ટ ધ પલ્સીસ, ગેટ ધ ટાઇમ''

\end{mnemonicbox}
\subsection*{પ્રશ્ન 5(બ) [4
ગુણ]}\label{uxaaauxab0uxab6uxaa8-5uxaac-4-uxa97uxaa3}

\textbf{એનર્જી મીટરનો ડાયાગ્રામ દોરો અને તેનો કાર્યસિદ્ધાંત સમજાવો.}

\begin{solutionbox}

\textbf{ઇલેક્ટ્રોનિક એનર્જી મીટર} કિલોવોટ-અવર (kWh)માં વિદ્યુત ઊર્જા વપરાશ માપે
છે.

\textbf{બ્લોક ડાયાગ્રામ}:

\begin{center}
\textbf{Mermaid Diagram (Code)}
\begin{verbatim}
{Shaded}
{Highlighting}[]
graph LR
    A[વોલ્ટેજ સેન્સર] {-{-}{} C[એનાલોગ મલ્ટિપ્લાયર]}
    B[કરંટ સેન્સર] {-{-}{} C}
    C {-{-}{} D[વોલ્ટેજ{-}ટુ{-}ફ્રિકવન્સી કન્વર્ટર]}
    D {-{-}{} E[પલ્સ કાઉન્ટર]}
    E {-{-}{} F[માઇક્રોકંટ્રોલર]}
    F {-{-}{} G[LCD ડિસ્પ્લે]}
    H[ક્રિસ્ટલ ઓસિલેટર] {-{-}{} F}
    F {-{-}{} I[LED ઇન્ડિકેટર]}
    F {-{-}{} J[કોમ્યુનિકેશન ઇન્ટરફેસ]}
{Highlighting}
{Shaded}
\end{verbatim}
\end{center}

\textbf{કાર્ય સિદ્ધાંત}:

\begin{itemize}
\tightlist
\item
  ઊર્જા = પાવર \times સમય
\item
  પાવર = વોલ્ટેજ \times કરંટ
\item
  વોલ્ટેજ અને કરંટ અલગથી સેન્સ થાય છે
\item
  ક્ષણિક પાવર મેળવવા ગુણાકાર કરાય છે
\item
  ઊર્જા મેળવવા સમય પર ઇન્ટિગ્રેટ કરાય છે
\item
  ઊર્જાના પ્રમાણમાં પલ્સ ઉત્પન્ન થાય છે
\item
  દરેક પલ્સ ફિક્સ્ડ ઊર્જા યુનિટ દર્શાવે છે
\item
  કાઉન્ટર પલ્સ એકત્રિત કરે છે
\item
  ડિસ્પ્લે એકત્રિત ઊર્જા બતાવે છે
\end{itemize}

\textbf{લક્ષણો}:

\begin{itemize}
\tightlist
\item
  \textbf{ટેમ્પર ડિટેક્શન}: વિજળી ચોરી રોકે છે
\item
  \textbf{મલ્ટિપલ ટેરિફ}: વિવિધ સમય માટે અલગ દરો
\item
  \textbf{કોમ્યુનિકેશન}: રિમોટ રીડિંગ ક્ષમતા
\end{itemize}

\end{solutionbox}
\begin{mnemonicbox}
``VCPI - વોલ્ટેજ અને કરંટ ગુણાકાર થાય છે, પલ્સ ઊર્જા વપરાશ
દર્શાવે છે''

\end{mnemonicbox}
\subsection*{પ્રશ્ન 5(ક) [7
ગુણ]}\label{uxaaauxab0uxab6uxaa8-5uxa95-7-uxa97uxaa3}

\textbf{ફંક્શન જનરેટરનો કાર્યસિદ્ધાંત અને કાર્યવિધી સંક્ષિપ્તમાં સમજાવો. તેના ફ્રન્ટ
પેનલ કંટ્રોલ્સનું વર્ણન કરો અને તે કેવી રીતે ઇલેક્ટ્રોનિક પરિપથોની તપાસ માટે ઉપયોગી છે તે
ઉદાહરણ સાથે સમજાવો.}

\begin{solutionbox}

\textbf{ફંક્શન જનરેટર} એક ઇલેક્ટ્રોનિક ટેસ્ટ ઇન્સ્ટ્રુમેન્ટ છે જે એડજસ્ટેબલ ફ્રિકવન્સી અને
એમ્પ્લિટ્યુડ સાથે વિવિધ વેવફોર્મ્સ ઉત્પન્ન કરે છે.

\textbf{કાર્ય સિદ્ધાંત}:

\begin{itemize}
\tightlist
\item
  ઓસિલેટર સર્કિટનો ઉપયોગ કરીને બેઝ સિગ્નલ ઉત્પન્ન કરે છે
\item
  વેવ-શેપિંગ સર્કિટનો ઉપયોગ કરીને વેવફોર્મ આકાર આપે છે
\item
  એમ્પ્લિટ્યુડ, ફ્રિકવન્સી અને ઓફસેટ પેરામીટર્સ એડજસ્ટ કરે છે
\item
  બફર એમ્પ્લિફાયર મારફતે વેવફોર્મ આઉટપુટ કરે છે
\end{itemize}

\textbf{બ્લોક ડાયાગ્રામ}:

\begin{center}
\textbf{Mermaid Diagram (Code)}
\begin{verbatim}
{Shaded}
{Highlighting}[]
graph LR
    A[ઓસિલેટર] {-{-}{} B[વેવ શેપર]}
    B {-{-}{} C[આઉટપુટ એમ્પ્લિફાયર]}
    D[ફ્રિકવન્સી કંટ્રોલ] {-{-}{} A}
    E[વેવફોર્મ સિલેક્ટર] {-{-}{} B}
    F[એમ્પ્લિટ્યુડ કંટ્રોલ] {-{-}{} C}
    G[DC ઓફસેટ કંટ્રોલ] {-{-}{} C}
    C {-{-}{} H[આઉટપુટ]}
    I[મોડ્યુલેશન ઇનપુટ] {-{-}{} A}
{Highlighting}
{Shaded}
\end{verbatim}
\end{center}

\textbf{ફ્રન્ટ પેનલ કંટ્રોલ્સ}:

{\def\LTcaptype{none} % do not increment counter
\begin{longtable}[]{@{}lll@{}}
\toprule\noalign{}
કંટ્રોલ & કાર્ય & ટિપિકલ રેન્જ \\
\midrule\noalign{}
\endhead
\bottomrule\noalign{}
\endlastfoot
ફ્રિકવન્સી & સિગ્નલ ફ્રિકવન્સી સેટ કરે છે & 0.1 Hz - 20 MHz \\
એમ્પ્લિટ્યુડ & સિગ્નલ એમ્પ્લિટ્યુડ સેટ કરે છે & 0 - 20 Vpp \\
DC ઓફસેટ & DC વોલ્ટેજ ઉમેરે છે & \pm10V \\
વેવફોર્મ સિલેક્ટ & વેવફોર્મ પ્રકાર પસંદ કરે છે & સાઇન, ટ્રાયેંગલ, સ્ક્વેર, પલ્સ \\
ડ્યુટી સાયકલ & પલ્સ વિડ્થ એડજસ્ટ કરે છે & 10\% - 90\% \\
મોડ્યુલેશન & AM/FM મોડ્યુલેશન & ઇન્ટર્નલ/એક્સટર્નલ \\
\end{longtable}
}

\textbf{આઉટપુટ વેવફોર્મ્સ}:

\begin{verbatim}
    Sine      /{      /      /}
             /  {    /      /  }
    \_\_\_\_\_\_\_ /    {\_\_/    \_\_/    \_\_}
    
    Square   \_\_\_\_\_\_      \_\_\_\_\_\_
            |      |    |      |
    \_\_\_\_\_\_\_\_|      |\_\_\_\_|      |\_\_\_\_
    
    Triangle  /{      /      /}
             /  {    /      /  }
    \_\_\_\_\_\_\_\_/    {\_\_/    \_\_/    \_\_}
    
    Pulse     \_\_        \_\_        \_\_
             |  |      |  |      |  |
    \_\_\_\_\_\_\_\_\_|  |\_\_\_\_\_\_|  |\_\_\_\_\_\_|  |\_
\end{verbatim}

\textbf{સર્કિટ ટેસ્ટિંગ ઉપયોગો}:

{\def\LTcaptype{none} % do not increment counter
\begin{longtable}[]{@{}lll@{}}
\toprule\noalign{}
ઉપયોગ & વપરાતો વેવફોર્મ & હેતુ \\
\midrule\noalign{}
\endhead
\bottomrule\noalign{}
\endlastfoot
એમ્પ્લિફાયર ટેસ્ટિંગ & સાઇન વેવ & ગેઇન, ફ્રિકવન્સી રિસ્પોન્સ \\
ડિજિટલ સર્કિટ ટેસ્ટિંગ & સ્ક્વેર વેવ & લોજિક ટાઇમિંગ, થ્રેશોલ્ડ \\
ફિલ્ટર ટેસ્ટિંગ & સાઇન સ્વીપ & કટઓફ ફ્રિકવન્સી, રિસ્પોન્સ \\
ટ્રિગરિંગ સર્કિટ્સ & પલ્સ & થ્રેશોલ્ડ ટેસ્ટિંગ \\
\end{longtable}
}

\textbf{ઉદાહરણ: એમ્પ્લિફાયર ટેસ્ટિંગ}

\begin{enumerate}
\tightlist
\item
  ફંક્શન જનરેટરને એમ્પ્લિફાયર ઇનપુટ સાથે કનેક્ટ કરો
\item
  યોગ્ય એમ્પ્લિટ્યુડનો સાઇન વેવ સેટ કરો
\item
  ફ્રિકવન્સી રિસ્પોન્સ ટેસ્ટ કરવા ફ્રિકવન્સી બદલો
\item
  ઓસિલોસ્કોપ પર આઉટપુટ મોનિટર કરો
\item
  ગેઇન ગણો = આઉટપુટ એમ્પ્લિટ્યુડ / ઇનપુટ એમ્પ્લિટ્યુડ
\end{enumerate}

\end{solutionbox}
\begin{mnemonicbox}
``FAWOD - ફ્રિકવન્સી, એમ્પ્લિટ્યુડ, વેવફોર્મ, ઓફસેટ, ડ્યુટી
સાયકલ મુખ્ય કંટ્રોલ્સ છે''

\end{mnemonicbox}
\subsection*{પ્રશ્ન 5(અ) OR [3
ગુણ]}\label{uxaaauxab0uxab6uxaa8-5uxa85-or-3-uxa97uxaa3}

\textbf{સ્પેક્ટ્રમ એનાલાઈઝરનું કાર્ય સમજાવો.}

\begin{solutionbox}

\textbf{સ્પેક્ટ્રમ એનાલાઇઝર} સિગ્નલની ફ્રિકવન્સી વિરુદ્ધ એમ્પ્લિટ્યુડ માપે છે, સિગ્નલના
ફ્રિકવન્સી ઘટકો બતાવે છે.

\textbf{કાર્ય સિદ્ધાંત}:

\begin{itemize}
\tightlist
\item
  ટાઇમ-ડોમેન સિગ્નલને ફ્રિકવન્સી-ડોમેનમાં રૂપાંતરિત કરે છે
\item
  સ્પેક્ટ્રલ ઘટકો અને તેમની એમ્પ્લિટ્યુડ બતાવે છે
\item
  સુપરહેટેરોડાઇન રિસીવર આર્કિટેક્ચરનો ઉપયોગ કરે છે
\item
  ફ્રિકવન્સી રેન્જનું વિશ્લેષણ કરવા લોકલ ઓસિલેટર સ્વીપ કરે છે
\end{itemize}

\textbf{બ્લોક ડાયાગ્રામ}:

\begin{center}
\textbf{Mermaid Diagram (Code)}
\begin{verbatim}
{Shaded}
{Highlighting}[]
graph LR
    A[ઇનપુટ સિગ્નલ] {-{-}{} B[એટેન્યુએટર/એમ્પ્લિફાયર]}
    B {-{-}{} C[મિક્સર]}
    D[લોકલ ઓસિલેટર] {-{-}{} C}
    C {-{-}{} E[IF ફિલ્ટર]}
    E {-{-}{} F[ડિટેક્ટર]}
    F {-{-}{} G[ડિસ્પ્લે]}
    H[સ્વીપ જનરેટર] {-{-}{} D}
    H {-{-}{} G}
{Highlighting}
{Shaded}
\end{verbatim}
\end{center}

\textbf{ઉપયોગો}:

\begin{itemize}
\tightlist
\item
  \textbf{સિગ્નલ એનાલિસિસ}: હાર્મોનિક્સ, ડિસ્ટોર્શન માપન
\item
  \textbf{EMI ટેસ્ટિંગ}: ઇન્ટરફેરન્સ સ્ત્રોતો શોધવા
\item
  \textbf{કોમ્યુનિકેશન્સ}: ચેનલ એનાલિસિસ, મોડ્યુલેશન ક્વોલિટી
\end{itemize}

\end{solutionbox}
\begin{mnemonicbox}
``SAME - સ્પેક્ટ્રમ એનાલાઇઝર ફ્રિકવન્સી પર સિગ્નલ એનર્જી મેપ
કરે છે''

\end{mnemonicbox}
\subsection*{પ્રશ્ન 5(બ) OR [4
ગુણ]}\label{uxaaauxab0uxab6uxaa8-5uxaac-or-4-uxa97uxaa3}

\textbf{ક્લેમ્પ ઓન મીટરનો ડાયાગ્રામ દોરો અને તેનું કાર્ય સમજાવો.}

\begin{solutionbox}

\textbf{ક્લેમ્પ-ઓન મીટર} (કરંટ ક્લેમ્પ) AC/DC કરંટ માપવા માટેનું નોન-કોન્ટેક્ટ ડિવાઇસ
છે.

\textbf{બંધારણ આકૃતિ}:

\begin{verbatim}
         Display
         .{-{-}{-}{-}{-}{-}.}
        /        {}
       /  120.5A  {    Function}
      /            {    Selector}
     |   O |{-{-}|     |  .{-}{-}{-}{-}.}
     |     |  |     |  |    |
     |     |  |     |{-{-}    |}
     |     |  |     |       |
     |     |  |     |       |
     |     {{-}{-}     |       |}
      {   Trigger   /       |}
       {           /        |}
        |         |        /
        |         |       /
        |         |      /
        {{-}{-}{-}{-}{-}{-}{-}{-}{-}´     /}
             |         /
           Clamp      /
             |       /
            /       /
           /       /
          /       /
         {{-}\_\_\_\_\_\_´}
         Test Leads
\end{verbatim}

\textbf{કાર્ય સિદ્ધાંત}:

\begin{itemize}
\tightlist
\item
  ઇલેક્ટ્રોમેગ્નેટિક ઇન્ડક્શન (ફેરાડેના નિયમ) પર આધારિત
\item
  કરંટ-વહન કરતો વાહક ચુંબકીય ક્ષેત્ર ઉત્પન્ન કરે છે
\item
  ક્લેમ્પનો ફેરોમેગ્નેટિક કોર ફિલ્ડને કેન્દ્રિત કરે છે
\item
  ક્લેમ્પમાં સેકન્ડરી કોઇલ પ્રમાણસર વોલ્ટેજ પ્રેરિત કરે છે
\item
  સર્કિટ પ્રેરિત વોલ્ટેજને કરંટ રીડિંગમાં રૂપાંતરિત કરે છે
\end{itemize}

\textbf{ફાયદાઓ}:

\begin{itemize}
\tightlist
\item
  \textbf{નોન-કોન્ટેક્ટ}: સર્કિટ ડિસકનેક્ટ કરવાની જરૂર નથી
\item
  \textbf{સલામતી}: ઉચ્ચ વોલ્ટેજથી આઇસોલેશન
\item
  \textbf{સુવિધા}: સીમિત જગ્યામાં વાપરવામાં સરળ
\end{itemize}

\textbf{ઉપયોગો}:

\begin{itemize}
\tightlist
\item
  \textbf{ઇલેક્ટ્રિકલ મેઇન્ટેનન્સ}: મોટર કરંટ, લોડ ટેસ્ટિંગ
\item
  \textbf{પાવર ક્વોલિટી}: પાવર ફેક્ટર, હાર્મોનિક્સ માપન
\item
  \textbf{ટ્રબલશૂટિંગ}: અનબેલેન્સ્ડ લોડ શોધવા
\end{itemize}

\end{solutionbox}
\begin{mnemonicbox}
``CLIP - ક્લેમ્પ કરંટ માપે છે, મેગ્નેટિક ઇન્ડક્શન વોલ્ટેજ પેદા કરે
છે''

\end{mnemonicbox}
\subsection*{પ્રશ્ન 5(ક) OR [7
ગુણ]}\label{uxaaauxab0uxab6uxaa8-5uxa95-or-7-uxa97uxaa3}

\textbf{ડિજિટલ IC ટેસ્ટરનું કાર્યસિદ્ધાંત સમજાવો. તેનો બ્લોક ડાયાગ્રામ સમજાવો અને તે
ડિજિટલ IC ની કાર્યક્ષમતા કઈ રીતે ચકાસે છે તે ઉદાહરણ સાથે સમજાવો.}

\begin{solutionbox}

\textbf{ડિજિટલ IC ટેસ્ટર} ટેસ્ટ પેટર્ન લાગુ કરીને અને પ્રતિક્રિયાઓની સરખામણી કરીને
ડિજિટલ ઇન્ટિગ્રેટેડ સર્કિટની કાર્યક્ષમતા ચકાસે છે.

\textbf{કાર્યસિદ્ધાંત}:

\begin{itemize}
\tightlist
\item
  IC પીન્સને પૂર્વનિર્ધારિત ટેસ્ટ વેક્ટર્સ લાગુ કરે છે
\item
  વાસ્તવિક આઉટપુટની અપેક્ષિત આઉટપુટ સાથે સરખામણી કરે છે
\item
  ખામીયુક્ત IC અથવા ખોટા કાર્યોની ઓળખ કરે છે
\item
  સંગ્રહિત ટેસ્ટ પેટર્નનો ઉપયોગ કરીને બહુવિધ IC પ્રકારો ટેસ્ટ કરે છે
\end{itemize}

\textbf{બ્લોક ડાયાગ્રામ}:

\begin{center}
\textbf{Mermaid Diagram (Code)}
\begin{verbatim}
{Shaded}
{Highlighting}[]
graph LR
    A[માઇક્રોકંટ્રોલર] {-{-}{} B[ROM/ટેસ્ટ પેટર્ન મેમરી]}
    A {-{-}{} C[ઇનપુટ પેટર્ન જનરેટર]}
    C {-{-}{} D[ZIF સોકેટ/ટેસ્ટિંગ હેઠળની IC]}
    D {-{-}{} E[આઉટપુટ રિસ્પોન્સ એનાલાઇઝર]}
    E {-{-}{} A}
    A {-{-}{} F[ડિસ્પ્લે]}
    G[કીપેડ/કંટ્રોલ પેનલ] {-{-}{} A}
    H[પાવર સપ્લાય] {-{-}{} D}
    H {-{-}{} A}
{Highlighting}
{Shaded}
\end{verbatim}
\end{center}

\textbf{મુખ્ય ઘટકો}:

\begin{itemize}
\tightlist
\item
  \textbf{ZIF સોકેટ}: ઝીરો ઇન્સર્શન ફોર્સ સોકેટ સરળ IC પ્લેસમેન્ટ માટે
\item
  \textbf{ટેસ્ટ પેટર્ન મેમરી}: વિવિધ IC માટે ટેસ્ટ વેક્ટર્સ સંગ્રહે છે
\item
  \textbf{આઉટપુટ રિસ્પોન્સ એનાલાઇઝર}: વાસ્તવિક વિરુદ્ધ અપેક્ષિત આઉટપુટની સરખામણી
  કરે છે
\item
  \textbf{માઇક્રોકંટ્રોલર}: ટેસ્ટિંગ સિક્વન્સ અને મૂલ્યાંકન નિયંત્રિત કરે છે
\item
  \textbf{ડિસ્પ્લે}: ટેસ્ટ પરિણામો અને સ્થિતિ બતાવે છે
\end{itemize}

\textbf{ટેસ્ટિંગ પદ્ધતિ}:

{\def\LTcaptype{none} % do not increment counter
\begin{longtable}[]{@{}lll@{}}
\toprule\noalign{}
સ્ટેપ & ક્રિયા & હેતુ \\
\midrule\noalign{}
\endhead
\bottomrule\noalign{}
\endlastfoot
1 & IC પ્રકાર પસંદ કરો & સાચા ટેસ્ટ પેરામીટર્સ લોડ કરો \\
2 & ZIF સોકેટમાં IC ઇન્સર્ટ કરો & ટેસ્ટિંગ માટે તૈયાર કરો \\
3 & ટેસ્ટ શરૂ કરો & ટેસ્ટ સિક્વન્સ શરૂ કરો \\
4 & ટેસ્ટ વેક્ટર્સ લાગુ કરો & IC ફંક્શન્સનો અભ્યાસ કરો \\
5 & પ્રતિક્રિયાઓની સરખામણી કરો & ભૂલો ઓળખો \\
6 & પરિણામો દર્શાવો & પાસ/ફેલ સ્થિતિ બતાવો \\
\end{longtable}
}

\textbf{ઉદાહરણ: 7400 NAND ગેટ IC ટેસ્ટિંગ}:

\begin{enumerate}
\tightlist
\item
  IC લિસ્ટમાંથી ``7400'' પસંદ કરો
\item
  ZIF સોકેટમાં IC ઇન્સર્ટ કરો
\item
  ટેસ્ટર બધા ઇનપુટ કોમ્બિનેશન્સ લાગુ કરે છે:

  \begin{itemize}
  \tightlist
  \item
    ઇનપુટ 1A=0, 1B=0 \rightarrow અપેક્ષિત આઉટપુટ 1Y=1
  \item
    ઇનપુટ 1A=0, 1B=1 \rightarrow અપેક્ષિત આઉટપુટ 1Y=1
  \item
    ઇનપુટ 1A=1, 1B=0 \rightarrow અપેક્ષિત આઉટપુટ 1Y=1
  \item
    ઇનપુટ 1A=1, 1B=1 \rightarrow અપેક્ષિત આઉટપુટ 1Y=0
  \end{itemize}
\item
  પેકેજમાં બધા ગેટ્સ માટે પુનરાવર્તન કરો (7400માં 4 NAND ગેટ્સ છે)
\item
  વાસ્તવિક આઉટપુટની અપેક્ષિત ટ્રુથ ટેબલ સાથે સરખામણી કરો
\item
  જો બધા ટેસ્ટ સફળ થાય, તો ``PASS'' ડિસ્પ્લે કરો, અથવા નિષ્ફળતા હોય તો એરર કોડ
  ડિસ્પ્લે કરો
\end{enumerate}

\textbf{મોડર્ન IC ટેસ્ટર્સની વિશેષતાઓ}:

\begin{itemize}
\tightlist
\item
  \textbf{ઓટો-આઇડેન્ટિફિકેશન}: અજ્ઞાત IC શોધે છે
\item
  \textbf{લર્નિંગ મોડ}: નવા IC માટે ટેસ્ટ પેટર્ન બનાવે છે
\item
  \textbf{ફંક્શનલ ટેસ્ટિંગ}: ઇન-સર્કિટ ઓપરેશન ટેસ્ટ કરે છે
\item
  \textbf{પેરામીટર ટેસ્ટિંગ}: ટાઇમિંગ, વોલ્ટેજ માર્જિન ચેક કરે છે
\end{itemize}

\end{solutionbox}
\begin{mnemonicbox}
``TEST - ટેસ્ટ પેટર્ન બધી સ્ટેટ્સનો અભ્યાસ કરે છે, પછી આઉટપુટ
ચકાસે છે''

\end{mnemonicbox}

\end{document}
