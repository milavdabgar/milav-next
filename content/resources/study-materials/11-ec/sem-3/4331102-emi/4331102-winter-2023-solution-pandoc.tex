\documentclass[10pt,a4paper]{article}

% content/resources/templates/preamble.tex
\usepackage[margin=0.6in]{geometry}
\author{Milav Dabgar}
\usepackage{amsmath,amssymb,amsthm}
\usepackage{booktabs}
\usepackage{multirow}
\usepackage{xcolor}
\usepackage{tcolorbox}
\tcbuselibrary{breakable,skins}
\usepackage[colorlinks=true,linkcolor=blue]{hyperref}
\usepackage{titlesec}
\usepackage{enumitem}
\usepackage{tikz}
\usepackage{pgfplots}
\usepackage{circuitikz}
\usepackage[version=4]{mhchem}
\usepackage{longtable}
\usepackage{array}
\usepackage{float}
\usepackage{caption}
\usepackage{listings}

\lstset{
  basicstyle=\small\ttfamily,
  breaklines=true,
  breakatwhitespace=false,
  postbreak=\mbox{\textcolor{red}{$\hookrightarrow$}\space},
  float=false,
  numbers=left,
  numberstyle=\tiny\color{gray},
  numbersep=10pt,
  xleftmargin=2em,
  keywordstyle=\color{blue},
  commentstyle=\color{green!60!black},
  stringstyle=\color{purple},
  backgroundcolor=\color{gray!5},
  showstringspaces=false,
  tabsize=2,
  captionpos=b,
  keepspaces=true,
  columns=flexible
}

\pgfplotsset{compat=1.18}
\usetikzlibrary{shapes,arrows,positioning,calc,patterns,decorations.pathmorphing,decorations.markings,arrows.meta}

% Color scheme
\definecolor{headcolor}{RGB}{0,102,204}
\definecolor{keycolor}{RGB}{220,20,60}
\definecolor{solutioncolor}{RGB}{34,139,34}
\definecolor{mnemoniccolor}{RGB}{148,0,211}
\definecolor{codecolor}{RGB}{0,0,100}

% Spacing
\setlength{\parskip}{3pt}
\setlist[itemize]{nosep}
\setlist[enumerate]{nosep}

% Title formatting
\titleformat{\section}{\Large\bfseries\color{headcolor}}{\thesection}{1em}{}
\titleformat{\subsection}{\large\bfseries\color{headcolor}}{\thesubsection}{1em}{}

% Pandoc tightlist compatibility
\providecommand{\tightlist}{%
  \setlength{\itemsep}{0pt}\setlength{\parskip}{0pt}}

% Pandoc longtable compatibility
\newcounter{none}
\def\thenone{}


% content/resources/templates/english-boxes.tex
% This file is currently empty - it exists to maintain consistency with the import structure.
% Add custom environments here if needed in the future.


\begin{document}

\begin{center}
{\Huge\bfseries\color{headcolor} Subject Name Solutions}\\[5pt]
{\LARGE 4331102 -- Winter 2023}\\[3pt]
{\large Semester 1 Study Material}\\[3pt]
{\normalsize\textit{Detailed Solutions and Explanations}}
\end{center}

\vspace{10pt}

\subsection*{Question 1(a) [3 marks]}\label{q1a}

\textbf{Give Definition of Accuracy, Reproducibility and Repeatability.}

\begin{solutionbox}

{\def\LTcaptype{none} % do not increment counter
\begin{longtable}[]{@{}
  >{\raggedright\arraybackslash}p{(\linewidth - 2\tabcolsep) * \real{0.3333}}
  >{\raggedright\arraybackslash}p{(\linewidth - 2\tabcolsep) * \real{0.6667}}@{}}
\toprule\noalign{}
\begin{minipage}[b]{\linewidth}\raggedright
Term
\end{minipage} & \begin{minipage}[b]{\linewidth}\raggedright
Definition
\end{minipage} \\
\midrule\noalign{}
\endhead
\bottomrule\noalign{}
\endlastfoot
\textbf{Accuracy} & Closeness of measured value to the true or actual
value of the quantity being measured \\
\textbf{Reproducibility} & Ability of an instrument to give identical
measurements for the same input when measured under different conditions
(different operators, locations, times) \\
\textbf{Repeatability} & Ability of an instrument to give identical
measurements for the same input when measured repeatedly under the same
conditions \\
\end{longtable}
}

\end{solutionbox}
\begin{mnemonicbox}
``ARR - Accurate Results Repeatedly''

\end{mnemonicbox}
\subsection*{Question 1(b) [4 marks]}\label{q1b}

\textbf{Draw and Explain Wheatstone bridge.}

\begin{solutionbox}

\textbf{Diagram:}

\begin{center}
\textbf{Mermaid Diagram (Code)}
\begin{verbatim}
{Shaded}
{Highlighting}[]
graph LR
    A[Supply+] {-{-}{-} R1}
    A {-{-}{-} R3}
    R1 {-{-}{-} B[Output+]}
    R3 {-{-}{-} C[Output{-}]}
    B {-{-}{-} R2}
    C {-{-}{-} R4}
    R2 {-{-}{-} D[Supply{-}]}
    R4 {-{-}{-} D}
{Highlighting}
{Shaded}
\end{verbatim}
\end{center}

{\def\LTcaptype{none} % do not increment counter
\begin{longtable}[]{@{}
  >{\raggedright\arraybackslash}p{(\linewidth - 2\tabcolsep) * \real{0.4091}}
  >{\raggedright\arraybackslash}p{(\linewidth - 2\tabcolsep) * \real{0.5909}}@{}}
\toprule\noalign{}
\begin{minipage}[b]{\linewidth}\raggedright
Feature
\end{minipage} & \begin{minipage}[b]{\linewidth}\raggedright
Description
\end{minipage} \\
\midrule\noalign{}
\endhead
\bottomrule\noalign{}
\endlastfoot
\textbf{Configuration} & Four resistors connected in diamond pattern \\
\textbf{Balance Condition} & R1/R2 = R3/R4 (when output voltage is
zero) \\
\textbf{Application} & Precise measurement of unknown resistance \\
\textbf{Operation} & Unknown resistor placed in one arm, remaining
resistors adjusted until bridge is balanced \\
\end{longtable}
}

\end{solutionbox}
\begin{mnemonicbox}
``WBMP - When Balanced, Measure Precisely''

\end{mnemonicbox}
\subsection*{Question 1(c) [7 marks]}\label{q1c}

\textbf{Explain Principle of Q meter. Also draw and explain Practical Q
Meter.}

\begin{solutionbox}

\textbf{Principle of Q Meter:}

The Q-meter operates on the principle of series resonance, where Q
factor is measured as the ratio of voltage across the capacitor to the
applied voltage at resonance.

\textbf{Diagram of Practical Q Meter:}

\begin{center}
\textbf{Mermaid Diagram (Code)}
\begin{verbatim}
{Shaded}
{Highlighting}[]
graph LR
    A[RF Oscillator] {-{-}{} B[Work Coil]}
    B {-{-}{} C[Series Circuit]}
    C {-{-}{} D[Unknown Inductor L]}
    D {-{-}{} E[Variable Capacitor C]}
    E {-{-}{} F[VTVM]}
    F {-{-}{} G[Q{-}Scale]}
{Highlighting}
{Shaded}
\end{verbatim}
\end{center}

{\def\LTcaptype{none} % do not increment counter
\begin{longtable}[]{@{}
  >{\raggedright\arraybackslash}p{(\linewidth - 2\tabcolsep) * \real{0.5238}}
  >{\raggedright\arraybackslash}p{(\linewidth - 2\tabcolsep) * \real{0.4762}}@{}}
\toprule\noalign{}
\begin{minipage}[b]{\linewidth}\raggedright
Component
\end{minipage} & \begin{minipage}[b]{\linewidth}\raggedright
Function
\end{minipage} \\
\midrule\noalign{}
\endhead
\bottomrule\noalign{}
\endlastfoot
\textbf{RF Oscillator} & Provides variable frequency signals \\
\textbf{Work Coil} & Inductively couples signal to test circuit \\
\textbf{Resonant Circuit} & Test inductor L in series with variable
capacitor C \\
\textbf{VTVM} & Measures voltage across capacitor \\
\textbf{Q-Scale} & Calibrated to read Q value directly \\
\end{longtable}
}

\begin{itemize}
\tightlist
\item
  \textbf{Resonant Formula}: f = 1/(2π\sqrtLC)
\item
  \textbf{Q Calculation}: Q = Vc/Vs (voltage across capacitor / source
  voltage)
\end{itemize}

\end{solutionbox}
\begin{mnemonicbox}
``RIVQ - Resonance Indicates Valuable Quality''

\end{mnemonicbox}
\subsection*{Question 1(c OR) [7
marks]}\label{question-1c-or-7-marks}

\textbf{Draw and explain construction of Moving coil type instruments.}

\begin{solutionbox}

\textbf{Diagram:}

\begin{verbatim}
            +{-{-}{-}{-}{-}{-}{-}{-}{-}+}
            |    N    |
            |         |
            |         |
   +{-{-}{-}{-}{-}{-}{-}{-}+{-}+     +{-}+{-}{-}{-}{-}{-}{-}{-}{-}+}
   |    |     |     |     |    |
   |    |     |     |     |    |
   |    |  S  |     |  S  |    |
   |    |     |     |     |    |
   +{-{-}{-}{-}+     |     |     +{-}{-}{-}{-}+}
        |     +{-{-}{-}{-}{-}+     |}
        |      Coil       |
        |                 |
        +{-{-}{-}{-}{-}{-}{-}{-}{-}{-}{-}{-}{-}{-}{-}{-}{-}+}
\end{verbatim}

{\def\LTcaptype{none} % do not increment counter
\begin{longtable}[]{@{}ll@{}}
\toprule\noalign{}
Component & Description \\
\midrule\noalign{}
\endhead
\bottomrule\noalign{}
\endlastfoot
\textbf{Permanent Magnet} & Creates strong magnetic field \\
\textbf{Moving Coil} & Lightweight coil wound on aluminum frame \\
\textbf{Springs} & Provide controlling torque and electrical
connections \\
\textbf{Pointer} & Attached to coil, moves over calibrated scale \\
\textbf{Core} & Soft iron cylindrical core to concentrate magnetic
flux \\
\end{longtable}
}

\begin{itemize}
\tightlist
\item
  \textbf{Operating Principle}: Deflecting torque = BIlN (B-field
  strength, I-current, l-length, N-turns)
\item
  \textbf{Controlling Torque}: Provided by springs proportional to
  deflection angle
\end{itemize}

\end{solutionbox}
\begin{mnemonicbox}
``MAPS-C: Magnet Acts, Pointer Shows Current''

\end{mnemonicbox}
\subsection*{Question 2(a) [3 marks]}\label{q2a}

\textbf{List out different Types of errors. Explain any Two.}

\begin{solutionbox}

{\def\LTcaptype{none} % do not increment counter
\begin{longtable}[]{@{}l@{}}
\toprule\noalign{}
Types of Errors \\
\midrule\noalign{}
\endhead
\bottomrule\noalign{}
\endlastfoot
\textbf{Gross Errors} \\
\textbf{Systematic Errors} \\
\textbf{Random Errors} \\
\textbf{Environmental Errors} \\
\textbf{Loading Errors} \\
\end{longtable}
}

\textbf{Explanation of Two Errors:}

\begin{enumerate}
\tightlist
\item
  \textbf{Systematic Errors}:

  \begin{itemize}
  \tightlist
  \item
    Consistent and predictable deviations from actual value
  \item
    Caused by instrument calibration, design, or method
  \end{itemize}
\item
  \textbf{Random Errors}:

  \begin{itemize}
  \tightlist
  \item
    Unpredictable variations in measurements
  \item
    Caused by noise, environmental fluctuations, or observer limitations
  \end{itemize}
\end{enumerate}

\end{solutionbox}
\begin{mnemonicbox}
``GSREL - Good Systems Reduce Error Levels''

\end{mnemonicbox}
\subsection*{Question 2(b) [4 marks]}\label{q2b}

\textbf{Draw and Explain Maxwell's bridge.}

\begin{solutionbox}

\textbf{Diagram:}

\begin{center}
\textbf{Mermaid Diagram (Code)}
\begin{verbatim}
{Shaded}
{Highlighting}[]
graph LR
    A[Supply] {-{-}{-} R1}
    A {-{-}{-} R3}
    R1 {-{-}{-} B[Detector]}
    R3 {-{-}{-} C[Detector]}
    B {-{-}{-} R2}
    C {-{-}{-} R4}
    B {-{-}{-} L["Unknown L"]}
    C {-{-}{-} C1["Capacitor C"]}
    R2 {-{-}{-} D[Ground]}
    R4 {-{-}{-} D}
    L {-{-}{-} D}
    C1 {-{-}{-} D}
{Highlighting}
{Shaded}
\end{verbatim}
\end{center}

{\def\LTcaptype{none} % do not increment counter
\begin{longtable}[]{@{}ll@{}}
\toprule\noalign{}
Component & Function \\
\midrule\noalign{}
\endhead
\bottomrule\noalign{}
\endlastfoot
\textbf{R1, R2, R3, R4} & Precision resistors in bridge arms \\
\textbf{Unknown L} & Inductor with resistance to be measured \\
\textbf{Capacitor C} & Standard capacitor in opposite arm \\
\textbf{Detector} & Null detector (galvanometer) \\
\end{longtable}
}

\begin{itemize}
\tightlist
\item
  \textbf{Balance Equation}: L = CR2R3
\item
  \textbf{Resistance Equation}: RL = R2R3/R4
\item
  \textbf{Application}: Measures inductance with significant resistance
\end{itemize}

\end{solutionbox}
\begin{mnemonicbox}
``MBLR - Maxwell Bridge Links Resistance''

\end{mnemonicbox}
\subsection*{Question 2(c) [7 marks]}\label{q2c}

\textbf{Draw and explain construction of moving iron type instruments.}

\begin{solutionbox}

\textbf{Diagram:}

\begin{verbatim}
     +{-{-}{-}{-}{-}{-}{-}{-}{-}{-}{-}{-}{-}{-}{-}{-}{-}{-}{-}{-}{-}+}
     |                     |
     |     +{-{-}{-}{-}{-}{-}{-}{-}{-}{-}+    |}
     |     |          |    |
     |     |   Coil   |    |
     |     |          |    |
     |     +{-{-}{-}{-}{-}{-}{-}{-}{-}{-}+    |}
     |          ||         |
     |    +{-{-}{-}{-}{-}++{-}{-}{-}{-}{-}{-}+  |}
     |    |     ||      |  |
     |    |  Iron Vanes |  |
     |    |             |  |
     |    +{-{-}{-}{-}{-}{-}{-}{-}{-}{-}{-}{-}{-}+  |}
     |                     |
     +{-{-}{-}{-}{-}{-}{-}{-}{-}{-}{-}{-}{-}{-}{-}{-}{-}{-}{-}{-}{-}+}
\end{verbatim}

{\def\LTcaptype{none} % do not increment counter
\begin{longtable}[]{@{}
  >{\raggedright\arraybackslash}p{(\linewidth - 2\tabcolsep) * \real{0.4583}}
  >{\raggedright\arraybackslash}p{(\linewidth - 2\tabcolsep) * \real{0.5417}}@{}}
\toprule\noalign{}
\begin{minipage}[b]{\linewidth}\raggedright
Component
\end{minipage} & \begin{minipage}[b]{\linewidth}\raggedright
Description
\end{minipage} \\
\midrule\noalign{}
\endhead
\bottomrule\noalign{}
\endlastfoot
\textbf{Coil} & Fixed coil that carries measuring current \\
\textbf{Iron Vanes} & Two soft iron pieces (one fixed, one movable) \\
\textbf{Pointer} & Attached to movable vane \\
\textbf{Control Spring} & Provides restraining torque \\
\textbf{Damping Mechanism} & Air friction damping using light aluminum
piston \\
\end{longtable}
}

\begin{itemize}
\tightlist
\item
  \textbf{Working Principle}: When current flows through coil, both iron
  pieces get magnetized with same polarity causing repulsion
\item
  \textbf{Advantages}: Suitable for both AC and DC, robust construction
\item
  \textbf{Disadvantages}: Non-uniform scale, higher power consumption
  than PMMC
\end{itemize}

\end{solutionbox}
\begin{mnemonicbox}
``IRAM - Iron Repulsion Activates Movement''

\end{mnemonicbox}
\subsection*{Question 2(a OR) [3
marks]}\label{question-2a-or-3-marks}

\textbf{Explain basic DC voltmeter.}

\begin{solutionbox}

\textbf{Diagram:}

\begin{verbatim}
  +{-{-}{-}{-}{-}{-}{-}+    +{-}{-}{-}{-}{-}{-}{-}{-}{-}+    +{-}{-}{-}{-}{-}{-}{-}{-}{-}{-}{-}+}
  | PMMC  |{-{-}{-}| Series  |{-}{-}{-}| Scale     |}
  | Meter |    | Resistor|    | Calibrated|
  +{-{-}{-}{-}{-}{-}{-}+    +{-}{-}{-}{-}{-}{-}{-}{-}{-}+    +{-}{-}{-}{-}{-}{-}{-}{-}{-}{-}{-}+}
\end{verbatim}

{\def\LTcaptype{none} % do not increment counter
\begin{longtable}[]{@{}ll@{}}
\toprule\noalign{}
Component & Function \\
\midrule\noalign{}
\endhead
\bottomrule\noalign{}
\endlastfoot
\textbf{PMMC Movement} & Basic current-sensitive movement \\
\textbf{Multiplier Resistor} & High-value series resistor \\
\textbf{Scale} & Calibrated to read voltage directly \\
\end{longtable}
}

\begin{itemize}
\tightlist
\item
  \textbf{Working Principle}: Voltmeter is PMMC meter with series
  resistor
\item
  \textbf{Calculation}: Rs = (V/Im) - Rm (Rs=series resistor, V=voltage,
  Im=full scale current, Rm=meter resistance)
\end{itemize}

\end{solutionbox}
\begin{mnemonicbox}
``SVM - Series Voltage Measurement''

\end{mnemonicbox}
\subsection*{Question 2(b OR) [4
marks]}\label{question-2b-or-4-marks}

\textbf{Draw and Explain Schering bridge.}

\begin{solutionbox}

\textbf{Diagram:}

\begin{center}
\textbf{Mermaid Diagram (Code)}
\begin{verbatim}
{Shaded}
{Highlighting}[]
graph LR
    A[AC Supply] {-{-}{-} C1["Unknown Capacitance"]}
    A {-{-}{-} R3}
    C1 {-{-}{-} B[Detector]}
    R3 {-{-}{-} C[Detector]}
    B {-{-}{-} R1}
    C {-{-}{-} C4["Standard C"]}
    R1 {-{-}{-} D[Ground]}
    C4 {-{-}{-} R4["Variable R"]}
    R4 {-{-}{-} D}
{Highlighting}
{Shaded}
\end{verbatim}
\end{center}

{\def\LTcaptype{none} % do not increment counter
\begin{longtable}[]{@{}ll@{}}
\toprule\noalign{}
Component & Function \\
\midrule\noalign{}
\endhead
\bottomrule\noalign{}
\endlastfoot
\textbf{C1} & Unknown capacitor (with loss) \\
\textbf{R1} & Resistance representing loss in C1 \\
\textbf{R3, R4} & Precision resistors \\
\textbf{C4} & Standard loss-free capacitor \\
\textbf{Detector} & Null indicator \\
\end{longtable}
}

\begin{itemize}
\tightlist
\item
  \textbf{Balance Equations}: C1 = C4(R3/R1)
\item
  \textbf{Dissipation Factor}: D = ωC1R1 = ωC4R4
\item
  \textbf{Application}: Measurement of capacitance and dielectric loss
\end{itemize}

\end{solutionbox}
\begin{mnemonicbox}
``SCDR - Schering Capacitance Determines Resistance''

\end{mnemonicbox}
\subsection*{Question 2(c OR) [7
marks]}\label{question-2c-or-7-marks}

\textbf{Write shortnote on Electronic Multimeter.}

\begin{solutionbox}

\textbf{Diagram:}

\begin{center}
\textbf{Mermaid Diagram (Code)}
\begin{verbatim}
{Shaded}
{Highlighting}[]
graph LR
    A[Input] {-{-}{} B[Attenuator/Range Selector]}
    B {-{-}{} C[Signal Converter]}
    C {-{-}{} D[Amplifier]}
    D {-{-}{} E[Rectifier/Detector]}
    E {-{-}{} F[Display]}
{Highlighting}
{Shaded}
\end{verbatim}
\end{center}

{\def\LTcaptype{none} % do not increment counter
\begin{longtable}[]{@{}
  >{\raggedright\arraybackslash}p{(\linewidth - 2\tabcolsep) * \real{0.4091}}
  >{\raggedright\arraybackslash}p{(\linewidth - 2\tabcolsep) * \real{0.5909}}@{}}
\toprule\noalign{}
\begin{minipage}[b]{\linewidth}\raggedright
Feature
\end{minipage} & \begin{minipage}[b]{\linewidth}\raggedright
Description
\end{minipage} \\
\midrule\noalign{}
\endhead
\bottomrule\noalign{}
\endlastfoot
\textbf{Functions} & Measures voltage (AC/DC), current (AC/DC),
resistance, and other parameters \\
\textbf{Sensitivity} & Higher sensitivity than analog meters (10MΩ input
impedance typical) \\
\textbf{Ranges} & Multiple selectable measurement ranges \\
\textbf{Accuracy} & 0.1\% to 3\% depending on quality and parameter \\
\textbf{Display} & Digital readout or analog pointer \\
\end{longtable}
}

\begin{itemize}
\tightlist
\item
  \textbf{Types}: Analog electronic multimeter, Digital multimeter (DMM)
\item
  \textbf{Advantages}: High input impedance, minimal loading effect,
  multiple functions
\item
  \textbf{Key Circuit}: Input attenuator, signal converter, amplifier,
  rectifier, display driver
\end{itemize}

\end{solutionbox}
\begin{mnemonicbox}
``VCAR-D: Voltage, Current And Resistance -
Displayed''

\end{mnemonicbox}
\subsection*{Question 3(a) [3 marks]}\label{q3a}

\textbf{Explain Various probes for CRO.}

\begin{solutionbox}

{\def\LTcaptype{none} % do not increment counter
\begin{longtable}[]{@{}
  >{\raggedright\arraybackslash}p{(\linewidth - 2\tabcolsep) * \real{0.5185}}
  >{\raggedright\arraybackslash}p{(\linewidth - 2\tabcolsep) * \real{0.4815}}@{}}
\toprule\noalign{}
\begin{minipage}[b]{\linewidth}\raggedright
Type of Probe
\end{minipage} & \begin{minipage}[b]{\linewidth}\raggedright
Description
\end{minipage} \\
\midrule\noalign{}
\endhead
\bottomrule\noalign{}
\endlastfoot
\textbf{Passive Probe (1X)} & Direct connection probe with no
attenuation \\
\textbf{Passive Probe (10X)} & Attenuates signal by factor of 10,
reduces circuit loading \\
\textbf{Active Probe} & Contains active components for high impedance,
low capacitance \\
\textbf{Current Probe} & Measures current by sensing magnetic field \\
\end{longtable}
}

\begin{itemize}
\tightlist
\item
  \textbf{Selection Criteria}: Bandwidth, loading effect, measurement
  range
\item
  \textbf{Compensation}: 10X probes require compensation adjustment for
  accurate waveforms
\end{itemize}

\end{solutionbox}
\begin{mnemonicbox}
``PAC-S: Probes Allow Circuit Sensing''

\end{mnemonicbox}
\subsection*{Question 3(b) [4 marks]}\label{q3b}

\textbf{Draw and explain construction of Clamp on Meter.}

\begin{solutionbox}

\textbf{Diagram:}

\begin{verbatim}
       +{-{-}{-}{-}{-}{-}{-}{-}{-}{-}{-}{-}{-}{-}{-}+}
       |    Display    |
       +{-{-}{-}{-}{-}{-}{-}{-}{-}{-}{-}{-}{-}{-}{-}+}
       |               |
       |    Circuit    |
       |               |
    +{-{-}+               +{-}{-}+}
    |  |               |  |
    |  +{-{-}{-}{-}{-}{-}{-}{-}{-}{-}{-}{-}{-}{-}{-}+  |}
    |                     |
    |      +{-{-}{-}{-}{-}{-}{-}+      |}
    |      |       |      |
    +{-{-}{-}{-}{-}{-}+       +{-}{-}{-}{-}{-}{-}+}
           | Wire  |
           +{-{-}{-}{-}{-}{-}{-}+}
\end{verbatim}

{\def\LTcaptype{none} % do not increment counter
\begin{longtable}[]{@{}ll@{}}
\toprule\noalign{}
Component & Function \\
\midrule\noalign{}
\endhead
\bottomrule\noalign{}
\endlastfoot
\textbf{Split Core CT} & Ferrite core that clamps around conductor \\
\textbf{Coil Winding} & Secondary winding that generates induced
current \\
\textbf{Signal Circuitry} & Converts current to measurable signal \\
\textbf{Display Unit} & Digital/analog display calibrated in amps \\
\textbf{Trigger Mechanism} & Opens/closes core around conductor \\
\end{longtable}
}

\begin{itemize}
\tightlist
\item
  \textbf{Working Principle}: Based on current transformer, measures
  current without breaking circuit
\item
  \textbf{Applications}: Measuring AC current in live conductors safely
\end{itemize}

\end{solutionbox}
\begin{mnemonicbox}
``CAMP - Current Analyzed by Magnetic Principle''

\end{mnemonicbox}
\subsection*{Question 3(c) [7 marks]}\label{q3c}

\textbf{Write shortnote on successive approximation type DVM.}

\begin{solutionbox}

\textbf{Block Diagram:}

\begin{center}
\textbf{Mermaid Diagram (Code)}
\begin{verbatim}
{Shaded}
{Highlighting}[]
graph LR
    A[Input] {-{-}{} B[Sample \& Hold]}
    B {-{-}{} C[Comparator]}
    C {-{-}{} D[SAR Logic]}
    D {-{-}{} E[DAC]}
    E {-{-}{} C}
    D {-{-}{} F[Digital Display]}
{Highlighting}
{Shaded}
\end{verbatim}
\end{center}

{\def\LTcaptype{none} % do not increment counter
\begin{longtable}[]{@{}
  >{\raggedright\arraybackslash}p{(\linewidth - 2\tabcolsep) * \real{0.5238}}
  >{\raggedright\arraybackslash}p{(\linewidth - 2\tabcolsep) * \real{0.4762}}@{}}
\toprule\noalign{}
\begin{minipage}[b]{\linewidth}\raggedright
Component
\end{minipage} & \begin{minipage}[b]{\linewidth}\raggedright
Function
\end{minipage} \\
\midrule\noalign{}
\endhead
\bottomrule\noalign{}
\endlastfoot
\textbf{Sample \& Hold} & Captures and holds input voltage \\
\textbf{Comparator} & Compares input with DAC output \\
\textbf{Successive Approximation Register} & Controls binary search
algorithm \\
\textbf{D/A Converter} & Generates analog voltage for comparison \\
\textbf{Digital Display} & Shows measured value \\
\end{longtable}
}

\begin{itemize}
\tightlist
\item
  \textbf{Working Principle}: Uses binary search algorithm to find
  digital value matching analog input
\item
  \textbf{Conversion Time}: Fixed regardless of input magnitude (8-16
  clock cycles for 8-16 bit)
\item
  \textbf{Advantages}: Medium speed, good resolution, consistent
  conversion time
\item
  \textbf{Applications}: General purpose measurements where medium speed
  is sufficient
\end{itemize}

\end{solutionbox}
\begin{mnemonicbox}
``SACD - Sample, Approximate, Compare, Display''

\end{mnemonicbox}
\subsection*{Question 3(a OR) [3
marks]}\label{question-3a-or-3-marks}

\textbf{Explain PH Sensor.}

\begin{solutionbox}

\textbf{Diagram:}

\begin{verbatim}
    +{-{-}{-}{-}{-}{-}{-}{-}{-}{-}{-}{-}{-}{-}{-}{-}{-}+}
    | Glass Electrode |{-{-}{-}+}
    +{-{-}{-}{-}{-}{-}{-}{-}{-}{-}{-}{-}{-}{-}{-}{-}{-}+   |}
                          |
    +{-{-}{-}{-}{-}{-}{-}{-}{-}{-}{-}{-}{-}{-}{-}{-}{-}+   |}
    | Reference       |{-{-}{-}+{-}{-}{-}{-}{-} Output}
    | Electrode       |   |
    +{-{-}{-}{-}{-}{-}{-}{-}{-}{-}{-}{-}{-}{-}{-}{-}{-}+   |}
                          |
    +{-{-}{-}{-}{-}{-}{-}{-}{-}{-}{-}{-}{-}{-}{-}{-}{-}+   |}
    | Temperature     |{-{-}{-}+}
    | Compensation    |
    +{-{-}{-}{-}{-}{-}{-}{-}{-}{-}{-}{-}{-}{-}{-}{-}{-}+}
\end{verbatim}

{\def\LTcaptype{none} % do not increment counter
\begin{longtable}[]{@{}
  >{\raggedright\arraybackslash}p{(\linewidth - 2\tabcolsep) * \real{0.5238}}
  >{\raggedright\arraybackslash}p{(\linewidth - 2\tabcolsep) * \real{0.4762}}@{}}
\toprule\noalign{}
\begin{minipage}[b]{\linewidth}\raggedright
Component
\end{minipage} & \begin{minipage}[b]{\linewidth}\raggedright
Function
\end{minipage} \\
\midrule\noalign{}
\endhead
\bottomrule\noalign{}
\endlastfoot
\textbf{Glass Electrode} & Sensitive to hydrogen ion concentration \\
\textbf{Reference Electrode} & Provides stable reference potential \\
\textbf{Temperature Sensor} & Compensates for temperature effects \\
\textbf{Signal Conditioner} & Amplifies and processes the millivolt
signal \\
\end{longtable}
}

\begin{itemize}
\tightlist
\item
  \textbf{Working Principle}: Generates voltage proportional to hydrogen
  ion concentration
\item
  \textbf{Output}: \textasciitilde59 mV per pH unit at 25^\circC
\item
  \textbf{Range}: 0-14 pH scale (acidic to alkaline)
\end{itemize}

\end{solutionbox}
\begin{mnemonicbox}
``PHRV - PH Related to Voltage''

\end{mnemonicbox}
\subsection*{Question 3(b OR) [4
marks]}\label{question-3b-or-4-marks}

\textbf{Draw and explain construction of Electronic Watt Meter.}

\begin{solutionbox}

\textbf{Block Diagram:}

\begin{center}
\textbf{Mermaid Diagram (Code)}
\begin{verbatim}
{Shaded}
{Highlighting}[]
graph LR
    A[Current Input] {-{-}{} B[Current Transformer]}
    C[Voltage Input] {-{-}{} D[Voltage Transformer]}
    B {-{-}{} E[Multiplier Circuit]}
    D {-{-}{} E}
    E {-{-}{} F[Integrator]}
    F {-{-}{} G[Digital Display]}
{Highlighting}
{Shaded}
\end{verbatim}
\end{center}

{\def\LTcaptype{none} % do not increment counter
\begin{longtable}[]{@{}ll@{}}
\toprule\noalign{}
Component & Function \\
\midrule\noalign{}
\endhead
\bottomrule\noalign{}
\endlastfoot
\textbf{Current Sensor} & Measures load current via CT or shunt \\
\textbf{Voltage Sensor} & Measures voltage via potential divider \\
\textbf{Multiplier} & Multiplies instantaneous voltage and current \\
\textbf{Integrator} & Averages power over time \\
\textbf{Display} & Digital readout in watts \\
\end{longtable}
}

\begin{itemize}
\tightlist
\item
  \textbf{Working Principle}: Power = V \times I \times cosθ (cosθ is power
  factor)
\item
  \textbf{Advantages}: High accuracy, wide range, digital display
\item
  \textbf{Types}: True RMS, average sensing
\end{itemize}

\end{solutionbox}
\begin{mnemonicbox}
``VIMP - Voltage \& Intensity Make Power''

\end{mnemonicbox}
\subsection*{Question 3(c OR) [7
marks]}\label{question-3c-or-7-marks}

\textbf{Write shortnote on Integrating type DVM.}

\begin{solutionbox}

\textbf{Block Diagram:}

\begin{center}
\textbf{Mermaid Diagram (Code)}
\begin{verbatim}
{Shaded}
{Highlighting}[]
graph LR
    A[Input] {-{-}{} B[Integrator]}
    B {-{-}{} C[Comparator]}
    D[Clock] {-{-}{} E[Counter \& Control]}
    C {-{-}{} E}
    E {-{-}{} F[Digital Display]}
{Highlighting}
{Shaded}
\end{verbatim}
\end{center}

{\def\LTcaptype{none} % do not increment counter
\begin{longtable}[]{@{}
  >{\raggedright\arraybackslash}p{(\linewidth - 2\tabcolsep) * \real{0.2400}}
  >{\raggedright\arraybackslash}p{(\linewidth - 2\tabcolsep) * \real{0.7600}}@{}}
\toprule\noalign{}
\begin{minipage}[b]{\linewidth}\raggedright
Type
\end{minipage} & \begin{minipage}[b]{\linewidth}\raggedright
Working Principle
\end{minipage} \\
\midrule\noalign{}
\endhead
\bottomrule\noalign{}
\endlastfoot
\textbf{Dual-Slope} & Integrates input for fixed time, then measures
discharge time with reference \\
\textbf{Voltage-to-Frequency} & Converts voltage to frequency, counts
pulses over fixed time \\
\textbf{Charge-Balance} & Balances input charge with reference charge \\
\end{longtable}
}

\textbf{Key Features:}

\begin{itemize}
\tightlist
\item
  \textbf{Noise Rejection}: Excellent rejection of power line noise
  (50/60Hz)
\item
  \textbf{Accuracy}: High accuracy due to time averaging
\item
  \textbf{Conversion Speed}: Slower than successive approximation type
\item
  \textbf{Resolution}: Typically 4½ to 6½ digits
\end{itemize}

\textbf{Applications}: Precision measurements, noisy environments, bench
instruments

\end{solutionbox}
\begin{mnemonicbox}
``TINA - Time Integration Nullifies Average''

\end{mnemonicbox}
\subsection*{Question 4(a) [3 marks]}\label{q4a}

\textbf{Write advantages and applications of Digital storage
oscilloscope.}

\begin{solutionbox}

{\def\LTcaptype{none} % do not increment counter
\begin{longtable}[]{@{}ll@{}}
\toprule\noalign{}
Advantages & Applications \\
\midrule\noalign{}
\endhead
\bottomrule\noalign{}
\endlastfoot
\textbf{Pre-trigger Viewing} & Capturing transient events \\
\textbf{Signal Storage} & Analyzing intermittent faults \\
\textbf{Waveform Processing} & Complex signal analysis \\
\textbf{Higher Bandwidth} & High-speed digital circuit testing \\
\textbf{Multiple Channel Display} & Comparing multiple signals \\
\end{longtable}
}

\begin{itemize}
\tightlist
\item
  \textbf{Key Benefits}: Can capture one-time events, store waveforms
  for later analysis
\item
  \textbf{Digital Features}: Automated measurements, FFT analysis, PC
  connectivity
\end{itemize}

\end{solutionbox}
\begin{mnemonicbox}
``SPADE - Storage, Processing, Analysis, Display,
Events''

\end{mnemonicbox}
\subsection*{Question 4(b) [4 marks]}\label{q4b}

\textbf{Write shortnote on Electronic Energy Meter.}

\begin{solutionbox}

\textbf{Block Diagram:}

\begin{center}
\textbf{Mermaid Diagram (Code)}
\begin{verbatim}
{Shaded}
{Highlighting}[]
graph LR
    A[Voltage Sensor] {-{-}{} C[Multiplier]}
    B[Current Sensor] {-{-}{} C}
    C {-{-}{} D[Integrator]}
    D {-{-}{} E[Pulse Generator]}
    E {-{-}{} F[Counter]}
    F {-{-}{} G[Display]}
{Highlighting}
{Shaded}
\end{verbatim}
\end{center}

{\def\LTcaptype{none} % do not increment counter
\begin{longtable}[]{@{}
  >{\raggedright\arraybackslash}p{(\linewidth - 2\tabcolsep) * \real{0.5238}}
  >{\raggedright\arraybackslash}p{(\linewidth - 2\tabcolsep) * \real{0.4762}}@{}}
\toprule\noalign{}
\begin{minipage}[b]{\linewidth}\raggedright
Component
\end{minipage} & \begin{minipage}[b]{\linewidth}\raggedright
Function
\end{minipage} \\
\midrule\noalign{}
\endhead
\bottomrule\noalign{}
\endlastfoot
\textbf{Voltage \& Current Sensors} & Measure line voltage and load
current \\
\textbf{Multiplier Circuit} & Calculates instantaneous power \\
\textbf{Integrator} & Converts power to energy over time \\
\textbf{Microcontroller} & Processes signals and controls display \\
\textbf{LCD Display} & Shows energy consumption in kWh \\
\end{longtable}
}

\begin{itemize}
\tightlist
\item
  \textbf{Working Principle}: Energy = \intP.dt (integral of power over
  time)
\item
  \textbf{Advantages}: No moving parts, high accuracy, tamper detection
\item
  \textbf{Features}: Multiple tariff support, bi-directional
  measurement, remote reading
\end{itemize}

\end{solutionbox}
\begin{mnemonicbox}
``VICES - Voltage \& Current Energy Summation''

\end{mnemonicbox}
\subsection*{Question 4(c) [7 marks]}\label{q4c}

\textbf{Draw and explain Block diagram of Analog C.R.O. and working of
each block in brief.}

\begin{solutionbox}

\textbf{Block Diagram:}

\begin{center}
\textbf{Mermaid Diagram (Code)}
\begin{verbatim}
{Shaded}
{Highlighting}[]
graph LR
    A[Vertical Input] {-{-}{} B[Vertical Attenuator]}
    B {-{-}{} C[Vertical Amplifier]}
    C {-{-}{} D[Vertical Deflection Plates]}
    E[Trigger Circuit] {-{-}{} F[Time Base Generator]}
    F {-{-}{} G[Horizontal Amplifier]}
    G {-{-}{} H[Horizontal Deflection Plates]}
    I[Cathode Ray Tube] {-{-}{} J[Screen]}
    D {-{-}{} I}
    H {-{-}{} I}
    K[Power Supply] {-{-}{} All}
{Highlighting}
{Shaded}
\end{verbatim}
\end{center}

{\def\LTcaptype{none} % do not increment counter
\begin{longtable}[]{@{}
  >{\raggedright\arraybackslash}p{(\linewidth - 2\tabcolsep) * \real{0.4118}}
  >{\raggedright\arraybackslash}p{(\linewidth - 2\tabcolsep) * \real{0.5882}}@{}}
\toprule\noalign{}
\begin{minipage}[b]{\linewidth}\raggedright
Block
\end{minipage} & \begin{minipage}[b]{\linewidth}\raggedright
Function
\end{minipage} \\
\midrule\noalign{}
\endhead
\bottomrule\noalign{}
\endlastfoot
\textbf{Vertical System} & Controls amplitude display (signal
attenuation, amplification) \\
\textbf{Horizontal System} & Controls time base (sweep generation) \\
\textbf{Trigger System} & Synchronizes horizontal sweep with input
signal \\
\textbf{CRT} & Displays signal (electron gun, deflection plates,
phosphor screen) \\
\textbf{Power Supply} & Provides required voltages to all circuits \\
\end{longtable}
}

\begin{itemize}
\tightlist
\item
  \textbf{Vertical System}: Processes input signal, controls Y-axis
  deflection
\item
  \textbf{Horizontal System}: Controls X-axis deflection (time base)
\item
  \textbf{Triggering}: Stabilizes waveform display by starting sweep at
  same point
\item
  \textbf{CRT Display}: Converts electrical signals to visible trace
\end{itemize}

\end{solutionbox}
\begin{mnemonicbox}
``VTHCP - Vertical, Time, Horizontal, CRT, Power''

\end{mnemonicbox}
\subsection*{Question 4(a OR) [3
marks]}\label{question-4a-or-3-marks}

\textbf{Draw and explain PIEZO-ELECTRIC transducer.}

\begin{solutionbox}

\textbf{Diagram:}

\begin{verbatim}
      Force
        ↓
    +{-{-}{-}{-}{-}{-}{-}{-}+}
    |        |
    | Quartz |{-{-}{-} Output Voltage}
    | Crystal|
    |        |
    +{-{-}{-}{-}{-}{-}{-}{-}+}
\end{verbatim}

{\def\LTcaptype{none} % do not increment counter
\begin{longtable}[]{@{}
  >{\raggedright\arraybackslash}p{(\linewidth - 2\tabcolsep) * \real{0.4348}}
  >{\raggedright\arraybackslash}p{(\linewidth - 2\tabcolsep) * \real{0.5652}}@{}}
\toprule\noalign{}
\begin{minipage}[b]{\linewidth}\raggedright
Property
\end{minipage} & \begin{minipage}[b]{\linewidth}\raggedright
Description
\end{minipage} \\
\midrule\noalign{}
\endhead
\bottomrule\noalign{}
\endlastfoot
\textbf{Principle} & Generates electric charge when mechanically
stressed \\
\textbf{Materials} & Quartz, Rochelle salt, PZT ceramics \\
\textbf{Operation} & Direct effect: force \rightarrow voltage, Inverse effect:
voltage \rightarrow displacement \\
\textbf{Output} & High impedance voltage proportional to applied
force \\
\end{longtable}
}

\begin{itemize}
\tightlist
\item
  \textbf{Applications}: Pressure sensors, accelerometers, ultrasonic
  devices
\item
  \textbf{Advantages}: High sensitivity, fast response, wide frequency
  range
\item
  \textbf{Limitations}: High output impedance, temperature sensitive
\end{itemize}

\end{solutionbox}
\begin{mnemonicbox}
``PFVD - Pressure Forms Voltage via Displacement''

\end{mnemonicbox}
\subsection*{Question 4(b OR) [4
marks]}\label{question-4b-or-4-marks}

\textbf{Draw and explain Measurement of Frequency by using CRO.}

\begin{solutionbox}

\textbf{Method 1: Using Lissajous Patterns}

\begin{verbatim}
    +{-{-}{-}{-}{-}{-}{-}{-}{-}{-}{-}{-}{-}+}
    |             |
    |    o o o    |
    |   o     o   |
    |  o       o  |
    |   o     o   |
    |    o o o    |
    |             |
    +{-{-}{-}{-}{-}{-}{-}{-}{-}{-}{-}{-}{-}+}
\end{verbatim}

\textbf{Method 2: Using Time Base}

\begin{verbatim}
    +{-{-}{-}{-}{-}{-}{-}{-}{-}{-}{-}{-}{-}+}
    |        /{   |}
    |       /  {  |}
    |      /    { |}
    |     /      {|}
    |    /        |
    |   /         |
    +{-{-}{-}{-}{-}{-}{-}{-}{-}{-}{-}{-}{-}+}
\end{verbatim}

{\def\LTcaptype{none} % do not increment counter
\begin{longtable}[]{@{}
  >{\raggedright\arraybackslash}p{(\linewidth - 2\tabcolsep) * \real{0.3810}}
  >{\raggedright\arraybackslash}p{(\linewidth - 2\tabcolsep) * \real{0.6190}}@{}}
\toprule\noalign{}
\begin{minipage}[b]{\linewidth}\raggedright
Method
\end{minipage} & \begin{minipage}[b]{\linewidth}\raggedright
Calculation
\end{minipage} \\
\midrule\noalign{}
\endhead
\bottomrule\noalign{}
\endlastfoot
\textbf{Lissajous Pattern} & Fx = Fy \times (Nx/Ny) \\
\textbf{Time Measurement} & f = 1/T (T is period measured using time
base) \\
\textbf{XY Mode} & Comparing unknown frequency with known reference \\
\end{longtable}
}

\begin{itemize}
\tightlist
\item
  \textbf{Time Base Method}: Measure period of waveform, calculate
  frequency as 1/T
\item
  \textbf{Lissajous Method}: Connect reference to X input, unknown to Y
  input
\item
  \textbf{Digital CRO}: Direct frequency readout using internal counter
\end{itemize}

\end{solutionbox}
\begin{mnemonicbox}
``LTX - Lissajous or Time for X-axis''

\end{mnemonicbox}
\subsection*{Question 4(c OR) [7
marks]}\label{question-4c-or-7-marks}

\textbf{Draw and explain Thermistor and Thermocouple.}

\begin{solutionbox}

\textbf{Thermistor Diagram:}

\begin{verbatim}
    +{-{-}{-}{-}{-}{-}{-}{-}{-}{-}{-}+}
    |           |
    | Thermistor|{-{-}{-}+}
    |           |   |
    +{-{-}{-}{-}{-}{-}{-}{-}{-}{-}{-}+   |}
                    |
    +{-{-}{-}{-}{-}{-}{-}{-}{-}{-}+    |}
    |          |    |
    | Resistor |{-{-}{-}{-}+{-}{-}{-}{-} Output}
    |          |
    +{-{-}{-}{-}{-}{-}{-}{-}{-}{-}+}
\end{verbatim}

\textbf{Thermocouple Diagram:}

\begin{verbatim}
     Metal A
    +{-{-}{-}{-}{-}{-}{-}{-}+}
              {}
               +{-{-}{-} Output}
              /
    +{-{-}{-}{-}{-}{-}{-}{-}+}
     Metal B
\end{verbatim}

{\def\LTcaptype{none} % do not increment counter
\begin{longtable}[]{@{}
  >{\raggedright\arraybackslash}p{(\linewidth - 4\tabcolsep) * \real{0.3000}}
  >{\raggedright\arraybackslash}p{(\linewidth - 4\tabcolsep) * \real{0.2750}}
  >{\raggedright\arraybackslash}p{(\linewidth - 4\tabcolsep) * \real{0.4250}}@{}}
\toprule\noalign{}
\begin{minipage}[b]{\linewidth}\raggedright
Transducer
\end{minipage} & \begin{minipage}[b]{\linewidth}\raggedright
Principle
\end{minipage} & \begin{minipage}[b]{\linewidth}\raggedright
Characteristics
\end{minipage} \\
\midrule\noalign{}
\endhead
\bottomrule\noalign{}
\endlastfoot
\textbf{Thermistor} & Resistance changes with temperature & High
sensitivity, non-linear, limited range \\
\textbf{Thermocouple} & Junction of dissimilar metals generates voltage
& Wide range, linear, low sensitivity \\
\end{longtable}
}

\textbf{Thermistor Types:}

\begin{itemize}
\tightlist
\item
  \textbf{NTC}: Negative Temperature Coefficient (resistance decreases
  with temperature)
\item
  \textbf{PTC}: Positive Temperature Coefficient (resistance increases
  with temperature)
\end{itemize}

\textbf{Thermocouple Types:}

\begin{itemize}
\tightlist
\item
  \textbf{Type K}: Chromel-Alumel (-200^\circC to 1350^\circC)
\item
  \textbf{Type J}: Iron-Constantan (-40^\circC to 750^\circC)
\item
  \textbf{Type T}: Copper-Constantan (-200^\circC to 350^\circC)
\end{itemize}

\end{solutionbox}
\begin{mnemonicbox}
``TRT/TVJ - Temperature Resistance/Voltage Junction''

\end{mnemonicbox}
\subsection*{Question 5(a) [3 marks]}\label{q5a}

\textbf{Draw and Explain Velocity transducer.}

\begin{solutionbox}

\textbf{Diagram:}

\begin{verbatim}
    +{-{-}{-}{-}{-}{-}{-}{-}{-}{-}{-}{-}{-}{-}{-}{-}{-}{-}+}
    |                  |
    |  N     S    N    |
    |  |     |    |    |
    +{-{-}+{-}{-}{-}{-}{-}+{-}{-}{-}{-}+{-}{-}{-}{-}+}
       |     |    |
       |  Magnet  |
       |     |    |
    +{-{-}+{-}{-}{-}{-}{-}+{-}{-}{-}{-}+{-}{-}{-}{-}+}
    |                  |
    |      Coil        |{-{-}{-}{-} Output}
    |                  |
    +{-{-}{-}{-}{-}{-}{-}{-}{-}{-}{-}{-}{-}{-}{-}{-}{-}{-}+}
\end{verbatim}

{\def\LTcaptype{none} % do not increment counter
\begin{longtable}[]{@{}ll@{}}
\toprule\noalign{}
Component & Function \\
\midrule\noalign{}
\endhead
\bottomrule\noalign{}
\endlastfoot
\textbf{Permanent Magnet} & Creates magnetic field \\
\textbf{Moving Coil} & Generates voltage proportional to velocity \\
\textbf{Housing} & Supports structure and magnetic circuit \\
\textbf{Output Circuit} & Conditions signal for measurement \\
\end{longtable}
}

\begin{itemize}
\tightlist
\item
  \textbf{Working Principle}: Based on Faraday's law of electromagnetic
  induction
\item
  \textbf{Output}: Voltage proportional to velocity (V = Blv)
\item
  \textbf{Applications}: Vibration measurement, seismic monitoring,
  motion control
\end{itemize}

\end{solutionbox}
\begin{mnemonicbox}
``VMMF - Velocity Makes Magnetic Flux''

\end{mnemonicbox}
\subsection*{Question 5(b) [4 marks]}\label{q5b}

\textbf{Give Classification of transducers and explain it.}

\begin{solutionbox}

{\def\LTcaptype{none} % do not increment counter
\begin{longtable}[]{@{}
  >{\raggedright\arraybackslash}p{(\linewidth - 2\tabcolsep) * \real{0.6957}}
  >{\raggedright\arraybackslash}p{(\linewidth - 2\tabcolsep) * \real{0.3043}}@{}}
\toprule\noalign{}
\begin{minipage}[b]{\linewidth}\raggedright
Classification
\end{minipage} & \begin{minipage}[b]{\linewidth}\raggedright
Types
\end{minipage} \\
\midrule\noalign{}
\endhead
\bottomrule\noalign{}
\endlastfoot
\textbf{By Energy Conversion} & Active (self-generating) vs.~Passive
(requiring external power) \\
\textbf{By Measurement Method} & Primary vs.~Secondary \\
\textbf{By Physical Principle} & Resistive, Capacitive, Inductive,
Photoelectric, etc. \\
\textbf{By Application} & Temperature, Pressure, Flow, Level, etc. \\
\end{longtable}
}

\textbf{Explanation:}

{\def\LTcaptype{none} % do not increment counter
\begin{longtable}[]{@{}
  >{\raggedright\arraybackslash}p{(\linewidth - 4\tabcolsep) * \real{0.1818}}
  >{\raggedright\arraybackslash}p{(\linewidth - 4\tabcolsep) * \real{0.3030}}
  >{\raggedright\arraybackslash}p{(\linewidth - 4\tabcolsep) * \real{0.5152}}@{}}
\toprule\noalign{}
\begin{minipage}[b]{\linewidth}\raggedright
Type
\end{minipage} & \begin{minipage}[b]{\linewidth}\raggedright
Examples
\end{minipage} & \begin{minipage}[b]{\linewidth}\raggedright
Characteristics
\end{minipage} \\
\midrule\noalign{}
\endhead
\bottomrule\noalign{}
\endlastfoot
\textbf{Active} & Thermocouple, Piezoelectric & Generate output without
external power \\
\textbf{Passive} & RTD, Strain gauge & Require external excitation \\
\textbf{Resistive} & Thermistor, Potentiometer & Change resistance with
input \\
\textbf{Capacitive} & Pressure sensors, Proximity & Change capacitance
with input \\
\textbf{Inductive} & LVDT, Proximity & Change inductance with input \\
\end{longtable}
}

\end{solutionbox}
\begin{mnemonicbox}
``APRCI - Active Passive Resistive Capacitive
Inductive''

\end{mnemonicbox}
\subsection*{Question 5(c) [7 marks]}\label{q5c}

\textbf{Write shortnote on LVDT.}

\begin{solutionbox}

\textbf{Diagram:}

\begin{center}
\textbf{Mermaid Diagram (Code)}
\begin{verbatim}
{Shaded}
{Highlighting}[]
graph LR
    A[Primary Coil] {-{-}{} B[Core]}
    B {-{-}{} C[Secondary Coil 1]}
    B {-{-}{} D[Secondary Coil 2]}
    E[AC Excitation] {-{-}{} A}
    C {-{-}{} F[Phase Sensitive Detector]}
    D {-{-}{} F}
    F {-{-}{} G[Output]}
{Highlighting}
{Shaded}
\end{verbatim}
\end{center}

{\def\LTcaptype{none} % do not increment counter
\begin{longtable}[]{@{}
  >{\raggedright\arraybackslash}p{(\linewidth - 2\tabcolsep) * \real{0.5238}}
  >{\raggedright\arraybackslash}p{(\linewidth - 2\tabcolsep) * \real{0.4762}}@{}}
\toprule\noalign{}
\begin{minipage}[b]{\linewidth}\raggedright
Component
\end{minipage} & \begin{minipage}[b]{\linewidth}\raggedright
Function
\end{minipage} \\
\midrule\noalign{}
\endhead
\bottomrule\noalign{}
\endlastfoot
\textbf{Primary Coil} & Excitation coil connected to AC source \\
\textbf{Secondary Coils} & Two identical coils connected in series
opposition \\
\textbf{Ferromagnetic Core} & Movable core that varies mutual
inductance \\
\textbf{Signal Conditioner} & Converts differential output to
displacement measurement \\
\end{longtable}
}

\textbf{Working Principle:}

\begin{itemize}
\tightlist
\item
  At null position: Equal voltage induced in both secondaries, net
  output zero
\item
  Core movement: Creates imbalance in secondary voltages
\item
  Output voltage: Proportional to displacement, phase indicates
  direction
\end{itemize}

\textbf{Characteristics:}

\begin{itemize}
\tightlist
\item
  \textbf{Range}: Typically \pm0.5mm to \pm25cm
\item
  \textbf{Linearity}: Excellent within rated range
\item
  \textbf{Resolution}: Virtually infinite (limited by readout circuit)
\item
  \textbf{Advantages}: Frictionless, robust, reliable, high resolution
\end{itemize}

\end{solutionbox}
\begin{mnemonicbox}
``CPSO: Core Position Shifts Output''

\end{mnemonicbox}
\subsection*{Question 5(a OR) [3
marks]}\label{question-5a-or-3-marks}

\textbf{Draw and Explain block diagram of simple frequency Counter.}

\begin{solutionbox}

\textbf{Block Diagram:}

\begin{center}
\textbf{Mermaid Diagram (Code)}
\begin{verbatim}
{Shaded}
{Highlighting}[]
graph LR
    A[Input] {-{-}{} B[Input Conditioning]}
    B {-{-}{} C[Gate Control]}
    D[Time Base] {-{-}{} C}
    C {-{-}{} E[Counter]}
    E {-{-}{} F[Display]}
{Highlighting}
{Shaded}
\end{verbatim}
\end{center}

{\def\LTcaptype{none} % do not increment counter
\begin{longtable}[]{@{}ll@{}}
\toprule\noalign{}
Block & Function \\
\midrule\noalign{}
\endhead
\bottomrule\noalign{}
\endlastfoot
\textbf{Input Conditioning} & Amplifies, shapes input signal into
pulses \\
\textbf{Gate Control} & Controls counting period based on time base \\
\textbf{Time Base} & Provides accurate reference time interval \\
\textbf{Counter} & Counts input pulses during gate period \\
\textbf{Display} & Shows count result (frequency) \\
\end{longtable}
}

\begin{itemize}
\tightlist
\item
  \textbf{Working Principle}: Counts pulses over precise time interval
  (typically 1 second)
\item
  \textbf{Frequency Calculation}: f = counts/time interval
\item
  \textbf{Resolution}: Determined by time base accuracy and gate time
\end{itemize}

\end{solutionbox}
\begin{mnemonicbox}
``IGTCD - Input Gated Time Counts Display''

\end{mnemonicbox}
\subsection*{Question 5(b OR) [4
marks]}\label{question-5b-or-4-marks}

\textbf{Draw and Explain Capacitive Transducer.}

\begin{solutionbox}

\textbf{Diagram:}

\begin{verbatim}
    +{-{-}{-}{-}{-}{-}{-}{-}{-}{-}{-}{-}{-}+}
    |    Fixed    |
    |   Plate 1   |
    +{-{-}{-}{-}{-}{-}{-}{-}{-}{-}{-}{-}{-}+}
           ↑
           d      ↓ Force
    +{-{-}{-}{-}{-}{-}{-}{-}{-}{-}{-}{-}{-}+}
    |   Movable   |
    |   Plate 2   |{-{-}{-}{-}{-} Output}
    +{-{-}{-}{-}{-}{-}{-}{-}{-}{-}{-}{-}{-}+}
\end{verbatim}

{\def\LTcaptype{none} % do not increment counter
\begin{longtable}[]{@{}
  >{\raggedright\arraybackslash}p{(\linewidth - 4\tabcolsep) * \real{0.3846}}
  >{\raggedright\arraybackslash}p{(\linewidth - 4\tabcolsep) * \real{0.2821}}
  >{\raggedright\arraybackslash}p{(\linewidth - 4\tabcolsep) * \real{0.3333}}@{}}
\toprule\noalign{}
\begin{minipage}[b]{\linewidth}\raggedright
Configuration
\end{minipage} & \begin{minipage}[b]{\linewidth}\raggedright
Principle
\end{minipage} & \begin{minipage}[b]{\linewidth}\raggedright
Application
\end{minipage} \\
\midrule\noalign{}
\endhead
\bottomrule\noalign{}
\endlastfoot
\textbf{Variable Gap} & C = ε_{0}εᵣA/d (varies inversely with distance) &
Pressure, displacement \\
\textbf{Variable Area} & C = ε_{0}εᵣA/d (varies directly with overlap area)
& Angular position, level \\
\textbf{Variable Dielectric} & C = ε_{0}εᵣA/d (varies with dielectric
constant) & Humidity, material analysis \\
\end{longtable}
}

\textbf{Working Principle:}

\begin{itemize}
\tightlist
\item
  Capacitance changes with physical parameter
\item
  Signal conditioning converts capacitance to voltage/current
\item
  High impedance output requires proper shielding
\end{itemize}

\textbf{Advantages}: High sensitivity, no moving contacts, low mass

\end{solutionbox}
\begin{mnemonicbox}
``CGAD - Capacitance Gap Area Dielectric''

\end{mnemonicbox}
\subsection*{Question 5(c OR) [7
marks]}\label{question-5c-or-7-marks}

\textbf{Draw and Explain block diagram of Function generator.}

\begin{solutionbox}

\textbf{Block Diagram:}

\begin{center}
\textbf{Mermaid Diagram (Code)}
\begin{verbatim}
{Shaded}
{Highlighting}[]
graph LR
    A[Frequency Control] {-{-}{} B[Waveform Generator]}
    C[Mode Selector] {-{-}{} B}
    B {-{-}{} D[Amplifier \& Attenuator]}
    D {-{-}{} E[Output Buffer]}
    E {-{-}{} F[Output]}
    G[Sweep Circuit] {-{-}{} B}
    H[AM/FM Modulator] {-{-}{} D}
{Highlighting}
{Shaded}
\end{verbatim}
\end{center}

{\def\LTcaptype{none} % do not increment counter
\begin{longtable}[]{@{}
  >{\raggedright\arraybackslash}p{(\linewidth - 2\tabcolsep) * \real{0.4118}}
  >{\raggedright\arraybackslash}p{(\linewidth - 2\tabcolsep) * \real{0.5882}}@{}}
\toprule\noalign{}
\begin{minipage}[b]{\linewidth}\raggedright
Block
\end{minipage} & \begin{minipage}[b]{\linewidth}\raggedright
Function
\end{minipage} \\
\midrule\noalign{}
\endhead
\bottomrule\noalign{}
\endlastfoot
\textbf{Frequency Control} & Sets oscillator frequency (typically 0.1Hz
to 20MHz) \\
\textbf{Waveform Generator} & Produces basic waveforms (sine, square,
triangle) \\
\textbf{Mode Selector} & Selects output waveform type \\
\textbf{Amplifier \& Attenuator} & Controls output amplitude \\
\textbf{Output Buffer} & Provides low output impedance \\
\textbf{Sweep Circuit} & Automatically varies frequency over range \\
\textbf{AM/FM Modulator} & Modifies signal for modulation functions \\
\end{longtable}
}

\textbf{Working Principle:}

\begin{itemize}
\tightlist
\item
  Generates sine wave using RC oscillator or DDS
\item
  Shape converters transform sine into square and triangle
\item
  Output amplitude controlled by attenuator circuit
\item
  Modern generators use digital synthesis techniques
\end{itemize}

\textbf{Applications}: Circuit testing, signal injection, filter
characterization

\end{solutionbox}
\begin{mnemonicbox}
``FWMASO - Frequency Waveform Mode Amplitude Sweep
Output''

\end{mnemonicbox}

\end{document}
