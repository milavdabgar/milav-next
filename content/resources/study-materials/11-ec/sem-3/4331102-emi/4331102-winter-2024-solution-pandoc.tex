\documentclass[10pt,a4paper]{article}

% content/resources/templates/preamble.tex
\usepackage[margin=0.6in]{geometry}
\author{Milav Dabgar}
\usepackage{amsmath,amssymb,amsthm}
\usepackage{booktabs}
\usepackage{multirow}
\usepackage{xcolor}
\usepackage{tcolorbox}
\tcbuselibrary{breakable,skins}
\usepackage[colorlinks=true,linkcolor=blue]{hyperref}
\usepackage{titlesec}
\usepackage{enumitem}
\usepackage{tikz}
\usepackage{pgfplots}
\usepackage{circuitikz}
\usepackage[version=4]{mhchem}
\usepackage{longtable}
\usepackage{array}
\usepackage{float}
\usepackage{caption}
\usepackage{listings}

\lstset{
  basicstyle=\small\ttfamily,
  breaklines=true,
  breakatwhitespace=false,
  postbreak=\mbox{\textcolor{red}{$\hookrightarrow$}\space},
  float=false,
  numbers=left,
  numberstyle=\tiny\color{gray},
  numbersep=10pt,
  xleftmargin=2em,
  keywordstyle=\color{blue},
  commentstyle=\color{green!60!black},
  stringstyle=\color{purple},
  backgroundcolor=\color{gray!5},
  showstringspaces=false,
  tabsize=2,
  captionpos=b,
  keepspaces=true,
  columns=flexible
}

\pgfplotsset{compat=1.18}
\usetikzlibrary{shapes,arrows,positioning,calc,patterns,decorations.pathmorphing,decorations.markings,arrows.meta}

% Color scheme
\definecolor{headcolor}{RGB}{0,102,204}
\definecolor{keycolor}{RGB}{220,20,60}
\definecolor{solutioncolor}{RGB}{34,139,34}
\definecolor{mnemoniccolor}{RGB}{148,0,211}
\definecolor{codecolor}{RGB}{0,0,100}

% Spacing
\setlength{\parskip}{3pt}
\setlist[itemize]{nosep}
\setlist[enumerate]{nosep}

% Title formatting
\titleformat{\section}{\Large\bfseries\color{headcolor}}{\thesection}{1em}{}
\titleformat{\subsection}{\large\bfseries\color{headcolor}}{\thesubsection}{1em}{}

% Pandoc tightlist compatibility
\providecommand{\tightlist}{%
  \setlength{\itemsep}{0pt}\setlength{\parskip}{0pt}}

% Pandoc longtable compatibility
\newcounter{none}
\def\thenone{}


% content/resources/templates/english-boxes.tex
% This file is currently empty - it exists to maintain consistency with the import structure.
% Add custom environments here if needed in the future.


\begin{document}

\begin{center}
{\Huge\bfseries\color{headcolor} Subject Name Solutions}\\[5pt]
{\LARGE 4331102 -- Winter 2024}\\[3pt]
{\large Semester 1 Study Material}\\[3pt]
{\normalsize\textit{Detailed Solutions and Explanations}}
\end{center}

\vspace{10pt}

\subsection*{Question 1(a) [3 marks]}\label{q1a}

\textbf{Define following term: (1) Accuracy (2) Resolution (3) Error}

\begin{solutionbox}

{\def\LTcaptype{none} % do not increment counter
\begin{longtable}[]{@{}
  >{\raggedright\arraybackslash}p{(\linewidth - 2\tabcolsep) * \real{0.3333}}
  >{\raggedright\arraybackslash}p{(\linewidth - 2\tabcolsep) * \real{0.6667}}@{}}
\toprule\noalign{}
\begin{minipage}[b]{\linewidth}\raggedright
Term
\end{minipage} & \begin{minipage}[b]{\linewidth}\raggedright
Definition
\end{minipage} \\
\midrule\noalign{}
\endhead
\bottomrule\noalign{}
\endlastfoot
\textbf{Accuracy} & The closeness of a measurement to the true value \\
\textbf{Resolution} & The smallest change in input that can be detected
by an instrument \\
\textbf{Error} & The difference between measured value and true value \\
\end{longtable}
}

\end{solutionbox}
\begin{mnemonicbox}
``ARE precise: Accuracy shows Reality, Error shows
deviation, Resolution shows detail.''

\end{mnemonicbox}
\subsection*{Question 1(b) [4 marks]}\label{q1b}

\textbf{Explain construction of unbounded strain gauge transducer with
necessary diagram in detail. Also list application of it.}

\begin{solutionbox}

An unbounded strain gauge consists of a fine wire wound in a grid
pattern attached to a backing material.

\begin{center}
\textbf{Mermaid Diagram (Code)}
\begin{verbatim}
{Shaded}
{Highlighting}[]
graph LR
    A[Backing Material] {-{-}{-} B[Fine Wire Grid]}
    B {-{-}{-} C[Lead Wires]}
    C {-{-}{-} D[Electrical Connections]}
    style B fill:\#f9f,stroke:\#333,stroke{-width:2px}
{Highlighting}
{Shaded}
\end{verbatim}
\end{center}

\begin{itemize}
\tightlist
\item
  \textbf{Construction elements}: Fine resistance wire is looped back
  and forth on an insulating base material
\item
  \textbf{Working principle}: Changes resistance when subjected to
  strain
\item
  \textbf{Applications}: Weight measurement, pressure sensors, force
  sensors, structural health monitoring
\end{itemize}

\end{solutionbox}
\begin{mnemonicbox}
``WIRE Flexes: Wire grids Indicate Resistance changes
from External stress.''

\end{mnemonicbox}
\subsection*{Question 1(c) [7 marks]}\label{q1c}

\textbf{Explain working of Schering Bridge with circuit diagram for
balance condition. List its advantages, disadvantages and applications.}

\begin{solutionbox}

Schering Bridge is an AC bridge used to measure unknown capacitance and
its dissipation factor.

\begin{center}
\textbf{Mermaid Diagram (Code)}
\begin{verbatim}
{Shaded}
{Highlighting}[]
graph LR
    A[R1] {-{-}{-} B[R2]}
    B {-{-}{-} C[C2]}
    C {-{-}{-} D[Cx]}
    D {-{-}{-} A}
    E[AC Source] {-{-}{-} A}
    E {-{-}{-} C}
    F[Detector] {-{-}{-} B}
    F {-{-}{-} D}
    style Cx fill:\#f9f,stroke:\#333,stroke{-width:2px}
{Highlighting}
{Shaded}
\end{verbatim}
\end{center}

\textbf{Balance condition:}

{\def\LTcaptype{none} % do not increment counter
\begin{longtable}[]{@{}ll@{}}
\toprule\noalign{}
Equation & Description \\
\midrule\noalign{}
\endhead
\bottomrule\noalign{}
\endlastfoot
Cx = C2(R2/R1) & For capacitance calculation \\
Dx = R2(C2/Cx) & For dissipation factor \\
\end{longtable}
}

\textbf{Advantages:}

\begin{itemize}
\tightlist
\item
  High accuracy
\item
  Direct reading of capacitance
\item
  Wide measurement range
\end{itemize}

\textbf{Disadvantages:}

\begin{itemize}
\tightlist
\item
  Requires careful shielding
\item
  Frequency dependent errors
\item
  Complex to balance
\end{itemize}

\textbf{Applications:}

\begin{itemize}
\tightlist
\item
  Capacitor testing
\item
  Insulation testing
\item
  Dielectric material evaluation
\end{itemize}

\end{solutionbox}
\begin{mnemonicbox}
``SCUBA dive: Schering Calculates Unknown capacitance
By Advanced circuit Designs In Various Equipment.''

\end{mnemonicbox}
\subsection*{Question 1(c OR) [7
marks]}\label{question-1c-or-7-marks}

\textbf{Explain working of Maxwell's bridge with circuit diagram for
balance condition. List its advantages, disadvantages, and
applications.}

\begin{solutionbox}

Maxwell's bridge is used to measure unknown inductance in terms of known
capacitance.

\begin{center}
\textbf{Mermaid Diagram (Code)}
\begin{verbatim}
{Shaded}
{Highlighting}[]
graph LR
    A[R1] {-{-}{-} B[R2]}
    B {-{-}{-} C[R3]}
    C {-{-}{-} D[Lx + Rx]}
    D {-{-}{-} A}
    E[AC Source] {-{-}{-} A}
    E {-{-}{-} C}
    F[Detector] {-{-}{-} B}
    F {-{-}{-} D}
    G[C4] {-{-}{-} B}
    G {-{-}{-} C}
    style G fill:\#f9f,stroke:\#333,stroke{-width:2px}
    style D fill:\#bbf,stroke:\#333,stroke{-width:2px}
{Highlighting}
{Shaded}
\end{verbatim}
\end{center}

\textbf{Balance condition:}

{\def\LTcaptype{none} % do not increment counter
\begin{longtable}[]{@{}ll@{}}
\toprule\noalign{}
Equation & Description \\
\midrule\noalign{}
\endhead
\bottomrule\noalign{}
\endlastfoot
Lx = C4·R2·R3 & For inductance calculation \\
Rx = R1·(R3/R2) & For resistance calculation \\
\end{longtable}
}

\textbf{Advantages:}

\begin{itemize}
\tightlist
\item
  Independent of frequency
\item
  High accuracy for medium Q coils
\item
  Easy to balance
\end{itemize}

\textbf{Disadvantages:}

\begin{itemize}
\tightlist
\item
  Not suitable for low Q coils
\item
  Requires standard capacitor
\item
  Limited range
\end{itemize}

\textbf{Applications:}

\begin{itemize}
\tightlist
\item
  Measuring inductors
\item
  Audio frequency measurements
\item
  Transformer testing
\end{itemize}

\end{solutionbox}
\begin{mnemonicbox}
``MAGIC bridge: Maxwell Analyses Great Inductors by
Comparing bridge Elements.''

\end{mnemonicbox}
\subsection*{Question 2(a) [3 marks]}\label{q2a}

\textbf{Explain working of electronic multimeter with necessary
diagram.}

\begin{solutionbox}

Electronic multimeter converts various electrical parameters into
proportional DC voltage for measurement.

\begin{center}
\textbf{Mermaid Diagram (Code)}
\begin{verbatim}
{Shaded}
{Highlighting}[]
graph LR
    A[Input Selection] {-{-}{} B[Attenuator/Range Selector]}
    B {-{-}{} C[Converter Circuit]}
    C {-{-}{} D[Amplifier]}
    D {-{-}{} E[ADC]}
    E {-{-}{} F[Display]}
    style E fill:\#f9f,stroke:\#333,stroke{-width:2px}
{Highlighting}
{Shaded}
\end{verbatim}
\end{center}

\begin{itemize}
\tightlist
\item
  \textbf{Circuit elements}: Input selector \rightarrow Attenuator \rightarrow Converter \rightarrow
  Amplifier \rightarrow ADC \rightarrow Display
\item
  \textbf{Measurement types}: DC voltage, AC voltage, Current,
  Resistance
\item
  \textbf{Power source}: Battery powered for portability and safety
\end{itemize}

\end{solutionbox}
\begin{mnemonicbox}
``SACRED device: Signal Attenuated, Converted And
Rectified for Electronic Display.''

\end{mnemonicbox}
\subsection*{Question 2(b) [4 marks]}\label{q2b}

\textbf{Differentiate between Digital Voltmeter over Analog Voltmeter.}

\begin{solutionbox}

{\def\LTcaptype{none} % do not increment counter
\begin{longtable}[]{@{}
  >{\raggedright\arraybackslash}p{(\linewidth - 4\tabcolsep) * \real{0.2292}}
  >{\raggedright\arraybackslash}p{(\linewidth - 4\tabcolsep) * \real{0.3958}}
  >{\raggedright\arraybackslash}p{(\linewidth - 4\tabcolsep) * \real{0.3750}}@{}}
\toprule\noalign{}
\begin{minipage}[b]{\linewidth}\raggedright
Parameter
\end{minipage} & \begin{minipage}[b]{\linewidth}\raggedright
Digital Voltmeter
\end{minipage} & \begin{minipage}[b]{\linewidth}\raggedright
Analog Voltmeter
\end{minipage} \\
\midrule\noalign{}
\endhead
\bottomrule\noalign{}
\endlastfoot
\textbf{Display type} & Numeric LCD/LED display & Moving pointer on
scale \\
\textbf{Accuracy} & Higher (\pm0.1\% typical) & Lower (\pm2-5\% typical) \\
\textbf{Reading errors} & No parallax error & Prone to parallax error \\
\textbf{Resolution} & Higher (can display 3-6 digits) & Limited by scale
divisions \\
\textbf{Input impedance} & Very high (\textgreater10MΩ) & Lower
(20-200kΩ/V) \\
\textbf{Response time} & Slower sampling rate & Instant response \\
\end{longtable}
}

\end{solutionbox}
\begin{mnemonicbox}
``PARIOS: Parallax-free, Accurate, Resolution high,
Impedance high, Observation digital, Sampling rate.''

\end{mnemonicbox}
\subsection*{Question 2(c) [7 marks]}\label{q2c}

\textbf{Describe construction diagram of Energy meter and explain in
detail.}

\begin{solutionbox}

Energy meter measures electrical energy consumption over time in
kilowatt-hours (kWh).

\begin{center}
\textbf{Mermaid Diagram (Code)}
\begin{verbatim}
{Shaded}
{Highlighting}[]
graph LR
    A[Voltage Coil] {-{-}{} B[Current Coil]}
    B {-{-}{} C[Aluminum Disc]}
    C {-{-}{} D[Mechanical Counter]}
    E[Permanent Magnet] {-{-}{} C}
    F[Braking System] {-{-}{} C}
    G[Load Terminals] {-{-}{} B}
    style C fill:\#f9f,stroke:\#333,stroke{-width:2px}
{Highlighting}
{Shaded}
\end{verbatim}
\end{center}

\textbf{Components:}

\begin{itemize}
\tightlist
\item
  \textbf{Voltage coil}: Creates flux proportional to voltage
\item
  \textbf{Current coil}: Creates flux proportional to current
\item
  \textbf{Aluminum disc}: Rotates due to eddy currents
\item
  \textbf{Counting mechanism}: Registers disc rotations
\item
  \textbf{Permanent magnet}: Acts as brake to control disc speed
\item
  \textbf{Adjustment systems}: For calibration and accuracy
\end{itemize}

\textbf{Working principle}: Disc rotation speed is proportional to power
consumption (V\timesI\timescosΦ)

\end{solutionbox}
\begin{mnemonicbox}
``VADCR meter: Voltage And current Drive Counter
through Rotations.''

\end{mnemonicbox}
\subsection*{Question 2(a OR) [3
marks]}\label{question-2a-or-3-marks}

\textbf{Explain working of clamp on Ammeter with necessary diagram.}

\begin{solutionbox}

Clamp-on ammeter measures current without breaking the circuit by using
electromagnetic induction.

\begin{verbatim}
            ╭─────────╮
            │ Display │
            ╰─────────╯
                 │
     ╭───────────────────╮
     │ Signal Processing │
     ╰───────────────────╯
                 │
          ╭────────────╮
          │  CT Core   │───┐
          ╰────────────╯   │
              │    │       │
              │    │       │
              │    │       │
          ╭───┴────┴───╮   │
          │ Clamp Jaws │   │
          ╰────────────╯   │
                 │         │
             Current       │
             Carrying      │
              Wire         │
                           │
                          ─┴─
                          GND
\end{verbatim}

\begin{itemize}
\tightlist
\item
  \textbf{Construction}: Split ferrite core with sensing coil
\item
  \textbf{Working principle}: Current-carrying wire creates magnetic
  field \rightarrow induces voltage in sensing coil
\item
  \textbf{Advantages}: Non-contact measurement, quick, safe
\end{itemize}

\end{solutionbox}
\begin{mnemonicbox}
``CICS: Clamping Induces Current Signal.''

\end{mnemonicbox}
\subsection*{Question 2(b OR) [4
marks]}\label{question-2b-or-4-marks}

\textbf{Differentiate between PMMC type Meter over Moving iron type
Meter.}

\begin{solutionbox}

{\def\LTcaptype{none} % do not increment counter
\begin{longtable}[]{@{}
  >{\raggedright\arraybackslash}p{(\linewidth - 4\tabcolsep) * \real{0.2115}}
  >{\raggedright\arraybackslash}p{(\linewidth - 4\tabcolsep) * \real{0.3269}}
  >{\raggedright\arraybackslash}p{(\linewidth - 4\tabcolsep) * \real{0.4615}}@{}}
\toprule\noalign{}
\begin{minipage}[b]{\linewidth}\raggedright
Parameter
\end{minipage} & \begin{minipage}[b]{\linewidth}\raggedright
PMMC Type Meter
\end{minipage} & \begin{minipage}[b]{\linewidth}\raggedright
Moving Iron Type Meter
\end{minipage} \\
\midrule\noalign{}
\endhead
\bottomrule\noalign{}
\endlastfoot
\textbf{Operating principle} & Magnetic field interaction & Magnetic
attraction/repulsion \\
\textbf{Current type} & DC only & Both AC and DC \\
\textbf{Scale} & Uniform & Non-uniform (crowded at ends) \\
\textbf{Accuracy} & Higher (\pm0.5\% typical) & Lower (\pm1-5\% typical) \\
\textbf{Damping} & Eddy current damping & Air friction damping \\
\textbf{Power consumption} & Low & High \\
\textbf{Frequency errors} & Not applicable & Affected by frequency
changes \\
\end{longtable}
}

\end{solutionbox}
\begin{mnemonicbox}
``PMMC is DAUPHIN: DC only, Accurate, Uniform scale,
Power efficient, High sensitivity, Independent of frequency, Needs
polarity.''

\end{mnemonicbox}
\subsection*{Question 2(c OR) [7
marks]}\label{question-2c-or-7-marks}

\textbf{Draw the block diagram and Explain working of Integrating type
DVM with necessary diagram and waveform.}

\begin{solutionbox}

Integrating type DVM converts input voltage to time through integration
for high accuracy measurements.

\begin{center}
\textbf{Mermaid Diagram (Code)}
\begin{verbatim}
{Shaded}
{Highlighting}[]
graph LR
    A[Input Buffer] {-{-}{} B[Integrator]}
    B {-{-}{} C[Comparator]}
    C {-{-}{} D[Control Logic]}
    D {-{-}{} E[Clock]}
    D {-{-}{} F[Counter]}
    F {-{-}{} G[Display]}
    D {-{-}{}|Reset| B}
    style B fill:\#f9f,stroke:\#333,stroke{-width:2px}
{Highlighting}
{Shaded}
\end{verbatim}
\end{center}

\textbf{Working principle:}

\begin{itemize}
\tightlist
\item
  Input voltage is integrated for fixed time period
\item
  Integrator output ramps up proportionally to input
\item
  Reference voltage with opposite polarity discharges integrator
\item
  Time taken for discharge is measured by counting clock pulses
\item
  Count is proportional to input voltage
\end{itemize}

\textbf{Waveforms:}

\begin{verbatim}
Input      ────────────────────────────────
                                           
Integrator      /{                /       }
output         /  {              /        }
              /    {            /         }
             /      {          /          }
            /        {        /           }
           /          {      /            }
          /            {    /             }

Control    ────┐      ┌─────────┐      ┌──
signals        │      │         │      │  
               └──────┘         └──────┘  

Clock      ┌┐┌┐┌┐┌┐┌┐┌┐┌┐┌┐┌┐┌┐┌┐┌┐┌┐┌┐┌┐┌┐
pulses     └┘└┘└┘└┘└┘└┘└┘└┘└┘└┘└┘└┘└┘└┘└┘└┘
\end{verbatim}

\textbf{Advantages:}

\begin{itemize}
\tightlist
\item
  High noise rejection
\item
  Good accuracy
\item
  Excellent resolution
\item
  Rejects common-mode noise
\end{itemize}

\end{solutionbox}
\begin{mnemonicbox}
``DIRT meter: Direct Integration Relates Time to
measure voltage.''

\end{mnemonicbox}
\subsection*{Question 3(a) [3 marks]}\label{q3a}

\textbf{Differentiate between CRO over DSO.}

\begin{solutionbox}

{\def\LTcaptype{none} % do not increment counter
\begin{longtable}[]{@{}
  >{\raggedright\arraybackslash}p{(\linewidth - 4\tabcolsep) * \real{0.1507}}
  >{\raggedright\arraybackslash}p{(\linewidth - 4\tabcolsep) * \real{0.3699}}
  >{\raggedright\arraybackslash}p{(\linewidth - 4\tabcolsep) * \real{0.4795}}@{}}
\toprule\noalign{}
\begin{minipage}[b]{\linewidth}\raggedright
Parameter
\end{minipage} & \begin{minipage}[b]{\linewidth}\raggedright
CRO (Analog Oscilloscope)
\end{minipage} & \begin{minipage}[b]{\linewidth}\raggedright
DSO (Digital Storage Oscilloscope)
\end{minipage} \\
\midrule\noalign{}
\endhead
\bottomrule\noalign{}
\endlastfoot
\textbf{Signal processing} & Analog throughout & Digital after ADC
conversion \\
\textbf{Storage capability} & Cannot store waveforms & Can store
multiple waveforms \\
\textbf{Bandwidth} & Typically lower & Higher (can exceed GHz) \\
\textbf{Triggering} & Basic trigger options & Advanced trigger
capabilities \\
\textbf{Analysis features} & Limited & Extensive (FFT, measurements) \\
\textbf{Display persistence} & Phosphor persistence & Adjustable digital
persistence \\
\end{longtable}
}

\end{solutionbox}
\begin{mnemonicbox}
``PASSED: Processing-Analog/digital,
Storage-none/yes, Signal-raw/processed, Easy-basic/advanced,
Display-phosphor/digital.''

\end{mnemonicbox}
\subsection*{Question 3(b) [4 marks]}\label{q3b}

\textbf{Explain CRO Screen.}

\begin{solutionbox}

CRO screen displays electrical signals and consists of several important
elements.

\begin{verbatim}
┌───────────────────────────────────────┐
│                                       │
│             PHOSPHOR SCREEN           │
│                                       │
│     │         │         │         │   │
│  ───┼─────────┼─────────┼─────────┼── │
│     │         │         │         │   │
│     │         │         │         │   │
│  ───┼─────────┼─────────┼─────────┼── │
│     │         │         │         │   │
│     │         │         │         │   │
│  ───┼─────────┼─────────┼─────────┼── │
│     │         │         │         │   │
│     │         │         │         │   │
│  ───┼─────────┼─────────┼─────────┼── │
│     │         │         │         │   │
│                                       │
└───────────────────────────────────────┘
\end{verbatim}

\textbf{Components:}

\begin{itemize}
\tightlist
\item
  \textbf{Phosphor coating}: Emits light when struck by electrons
\item
  \textbf{Graticule}: Grid lines for measurement reference
\item
  \textbf{Scales}: Calibrated markings for voltage/time
\item
  \textbf{Center reference point}: (0,0) coordinate
\item
  \textbf{Intensity control}: Adjusts brightness of display
\end{itemize}

\end{solutionbox}
\begin{mnemonicbox}
``PGSCR: Phosphor Glows when Struck, Creating
Representation.''

\end{mnemonicbox}
\subsection*{Question 3(c) [7 marks]}\label{q3c}

\textbf{Explain Block diagram, working and advantage of CRO with
necessary diagram.}

\begin{solutionbox}

CRO (Cathode Ray Oscilloscope) visualizes electrical signals as
waveforms.

\begin{center}
\textbf{Mermaid Diagram (Code)}
\begin{verbatim}
{Shaded}
{Highlighting}[]
graph LR
    A[Vertical Input] {-{-}{} B[Vertical Attenuator]}
    B {-{-}{} C[Vertical Amplifier]}
    C {-{-}{} D[Vertical Deflection Plates]}
    E[Trigger Circuit] {-{-}{} F[Time Base Generator]}
    F {-{-}{} G[Horizontal Amplifier]}
    G {-{-}{} H[Horizontal Deflection Plates]}
    I[Power Supply] {-{-}{} J[Electron Gun]}
    J {-{-}{} K[CRT]}
    D {-{-}{} K}
    H {-{-}{} K}
    I {-{-}{} B}
    I {-{-}{} C}
    I {-{-}{} E}
    I {-{-}{} F}
    I {-{-}{} G}
    style K fill:\#f9f,stroke:\#333,stroke{-width:2px}
{Highlighting}
{Shaded}
\end{verbatim}
\end{center}

\textbf{Working principle:}

\begin{itemize}
\tightlist
\item
  \textbf{Electron gun}: Generates electron beam
\item
  \textbf{Vertical system}: Controls Y-axis deflection proportional to
  input signal
\item
  \textbf{Horizontal system}: Sweeps beam across screen at constant rate
\item
  \textbf{Trigger circuit}: Synchronizes horizontal sweep with input
  signal
\item
  \textbf{CRT}: Displays electron beam movement on phosphor screen
\end{itemize}

\textbf{Advantages:}

\begin{itemize}
\tightlist
\item
  Real-time signal display
\item
  Wide bandwidth
\item
  High input impedance
\item
  Versatile triggering options
\item
  Multiple signal analysis
\end{itemize}

\end{solutionbox}
\begin{mnemonicbox}
``EARTH view: Electron beam Amplification Reveals
Time-based Horizontal view.''

\end{mnemonicbox}
\subsection*{Question 3(a OR) [3
marks]}\label{question-3a-or-3-marks}

\textbf{Apply Lissajous pattern for frequency measurement and Phase
angle measurement.}

\begin{solutionbox}

Lissajous patterns are created when two sine waves are applied to X and
Y inputs of CRO.

{\def\LTcaptype{none} % do not increment counter
\begin{longtable}[]{@{}
  >{\raggedright\arraybackslash}p{(\linewidth - 4\tabcolsep) * \real{0.3182}}
  >{\raggedright\arraybackslash}p{(\linewidth - 4\tabcolsep) * \real{0.2045}}
  >{\raggedright\arraybackslash}p{(\linewidth - 4\tabcolsep) * \real{0.4773}}@{}}
\toprule\noalign{}
\begin{minipage}[b]{\linewidth}\raggedright
Pattern Type
\end{minipage} & \begin{minipage}[b]{\linewidth}\raggedright
Example
\end{minipage} & \begin{minipage}[b]{\linewidth}\raggedright
Measurement Formula
\end{minipage} \\
\midrule\noalign{}
\endhead
\bottomrule\noalign{}
\endlastfoot
\textbf{Frequency Measurement} &
\pandocbounded{\includegraphics[keepaspectratio,alt={Lissajous for frequency}]{https://example.com}}
& fx/fy = ny/nx \\
\textbf{Phase Angle Measurement} &
\pandocbounded{\includegraphics[keepaspectratio,alt={Lissajous for phase}]{https://example.com}}
& sin(φ) = A/B \\
\end{longtable}
}

\begin{verbatim}
    Frequency                 Phase
      B B                     B  
    A     A                A     A
    │     │                │  │  │
    └─────┘                └──┘──┘
      
    fx/fy = 2/1          sin(φ) = sin/sin
\end{verbatim}

\begin{itemize}
\tightlist
\item
  \textbf{Frequency ratio}: Count vertical tangent points / horizontal
  tangent points
\item
  \textbf{Phase measurement}: sin(φ) = sin/sinmax where sin is pattern
  height at zero crossing
\item
  \textbf{Applications}: Signal comparison, frequency calibration
\end{itemize}

\end{solutionbox}
\begin{mnemonicbox}
``LIPS patterns: Lissajous Indicates Phase and Sine
frequency.''

\end{mnemonicbox}
\subsection*{Question 3(b OR) [4
marks]}\label{question-3b-or-4-marks}

\textbf{Explain Graticules in CRO. Also Explain its types.}

\begin{solutionbox}

Graticules are reference grids on a CRO screen that help in measurement
of waveform parameters.

\begin{verbatim}
┌───────────────────────────────────┐
│                                   │
│  │    │    │    │    │    │    │  │
│──┼────┼────┼────┼────┼────┼────┼──│
│  │    │    │    │    │    │    │  │
│  │    │    │    │    │    │    │  │
│──┼────┼────┼────┼────┼────┼────┼──│
│  │    │    │    │    │    │    │  │
│  │    │    │    │    │    │    │  │
│──┼────┼────┼────┼────┼────┼────┼──│
│  │    │    │    │    │    │    │  │
│                                   │
└───────────────────────────────────┘
\end{verbatim}

\textbf{Types of graticules:}

{\def\LTcaptype{none} % do not increment counter
\begin{longtable}[]{@{}
  >{\raggedright\arraybackslash}p{(\linewidth - 4\tabcolsep) * \real{0.1875}}
  >{\raggedright\arraybackslash}p{(\linewidth - 4\tabcolsep) * \real{0.4062}}
  >{\raggedright\arraybackslash}p{(\linewidth - 4\tabcolsep) * \real{0.4062}}@{}}
\toprule\noalign{}
\begin{minipage}[b]{\linewidth}\raggedright
Type
\end{minipage} & \begin{minipage}[b]{\linewidth}\raggedright
Description
\end{minipage} & \begin{minipage}[b]{\linewidth}\raggedright
Application
\end{minipage} \\
\midrule\noalign{}
\endhead
\bottomrule\noalign{}
\endlastfoot
\textbf{Internal graticule} & Etched on inside of CRT & Eliminates
parallax error \\
\textbf{External graticule} & Separate transparent plate & Easy
replacement \\
\textbf{Electronic graticule} & Generated electronically & Digital
oscilloscopes \\
\textbf{Special purpose} & Custom markings for specific measurements &
Specialized testing \\
\end{longtable}
}

\end{solutionbox}
\begin{mnemonicbox}
``GRIT: Graticules Render Important Time-voltage
measurements.''

\end{mnemonicbox}
\subsection*{Question 3(c OR) [7
marks]}\label{question-3c-or-7-marks}

\textbf{Describe Block diagram, working and advantage of Digital storage
oscilloscope (DSO).}

\begin{solutionbox}

Digital Storage Oscilloscope (DSO) digitizes signals for storage,
processing, and display.

\begin{center}
\textbf{Mermaid Diagram (Code)}
\begin{verbatim}
{Shaded}
{Highlighting}[]
graph LR
    A[Input Signal] {-{-}{} B[Attenuator/Amplifier]}
    B {-{-}{} C[ADC]}
    C {-{-}{} D[Memory]}
    D {-{-}{} E[Processor]}
    E {-{-}{} F[DAC]}
    F {-{-}{} G[Display]}
    H[Time Base] {-{-}{} E}
    I[Trigger System] {-{-}{} E}
    J[Control Panel] {-{-}{} E}
    style D fill:\#f9f,stroke:\#333,stroke{-width:2px}
{Highlighting}
{Shaded}
\end{verbatim}
\end{center}

\textbf{Working principle:}

\begin{itemize}
\tightlist
\item
  \textbf{Acquisition}: Signal is sampled at high rate by ADC
\item
  \textbf{Storage}: Digital values stored in memory
\item
  \textbf{Processing}: Digital signal processing enhances analysis
\item
  \textbf{Display}: Reconstructed signal shown on screen
\item
  \textbf{Triggering}: Advanced digital triggering options
\end{itemize}

\textbf{Advantages:}

\begin{itemize}
\tightlist
\item
  Signal storage capability
\item
  Pre-trigger viewing
\item
  One-shot signal capture
\item
  Advanced measurements
\item
  Deep memory for long captures
\item
  Digital filtering and analysis
\item
  Network connectivity
\end{itemize}

\end{solutionbox}
\begin{mnemonicbox}
``SAMPLE: Storage And Memory Preserves Long-term
Events.''

\end{mnemonicbox}
\subsection*{Question 4(a) [3 marks]}\label{q4a}

\textbf{Differentiate RTD and Thermistor.}

\begin{solutionbox}

{\def\LTcaptype{none} % do not increment counter
\begin{longtable}[]{@{}
  >{\raggedright\arraybackslash}p{(\linewidth - 4\tabcolsep) * \real{0.1803}}
  >{\raggedright\arraybackslash}p{(\linewidth - 4\tabcolsep) * \real{0.6230}}
  >{\raggedright\arraybackslash}p{(\linewidth - 4\tabcolsep) * \real{0.1967}}@{}}
\toprule\noalign{}
\begin{minipage}[b]{\linewidth}\raggedright
Parameter
\end{minipage} & \begin{minipage}[b]{\linewidth}\raggedright
RTD (Resistance Temperature Detector)
\end{minipage} & \begin{minipage}[b]{\linewidth}\raggedright
Thermistor
\end{minipage} \\
\midrule\noalign{}
\endhead
\bottomrule\noalign{}
\endlastfoot
\textbf{Material} & Platinum, Nickel, Copper & Metal oxides,
semiconductors \\
\textbf{Resistance-temperature relation} & Linear, positive coefficient
& Non-linear, usually negative coefficient \\
\textbf{Temperature range} & -200^\circC to +850^\circC & -50^\circC to +300^\circC \\
\textbf{Sensitivity} & Lower (0.00385 Ω/Ω/^\circC typical) & Higher (3-5\%
per ^\circC typical) \\
\textbf{Accuracy} & Higher & Lower \\
\textbf{Response time} & Slower & Faster \\
\end{longtable}
}

\end{solutionbox}
\begin{mnemonicbox}
``RTD is PLAINS: Platinum, Linear, Accurate,
Industrial range, Narrow sensitivity, Stable.''

\end{mnemonicbox}
\subsection*{Question 4(b) [4 marks]}\label{q4b}

\textbf{Explain Optical encoder with its output waveform.}

\begin{solutionbox}

Optical encoder converts mechanical motion to digital pulses using light
interruption through a coded disc.

\begin{verbatim}
    ┌─────────────┐
    │  Light      │
    │  Source     │
    └─────┬───────┘
          │
          v
    ┌─────────────┐
    │  Code       │
    │  Disc       │◄────Motion
    └─────┬───────┘
          │
          v
    ┌─────────────┐
    │  Photo{-     │}
    │  detector   │
    └─────┬───────┘
          │
          v
     Output Signal
\end{verbatim}

\textbf{Output waveforms:}

\begin{verbatim}
Channel A: ┌──┐  ┌──┐  ┌──┐  ┌──┐
           │  │  │  │  │  │  │  │
           └──┘  └──┘  └──┘  └──┘

Channel B: ┌──┐  ┌──┐  ┌──┐  ┌──┐
         ┌─┘  └──┘  └──┘  └──┘  └─
         │
90^ phase│
   shift └─────────────────────────
\end{verbatim}

\begin{itemize}
\tightlist
\item
  \textbf{Components}: Light source, coded disc, photodetector
\item
  \textbf{Types}: Incremental (pulses) or absolute (unique position
  code)
\item
  \textbf{Applications}: Position measurement, speed detection, motion
  control
\end{itemize}

\end{solutionbox}
\begin{mnemonicbox}
``DROPS: Disc Rotation Outputs Pulse Signals.''

\end{mnemonicbox}
\subsection*{Question 4(c) [7 marks]}\label{q4c}

\textbf{Describe Thermocouple with working principle, types and
application.}

\begin{solutionbox}

Thermocouple is a temperature sensor that operates on the Seebeck
effect, generating voltage proportional to temperature difference.

\begin{center}
\textbf{Mermaid Diagram (Code)}
\begin{verbatim}
{Shaded}
{Highlighting}[]
graph LR
    A[Hot Junction] {-{-}{-} B[Metal A]}
    A {-{-}{-} C[Metal B]}
    B {-{-}{-} D[Cold Junction]}
    C {-{-}{-} D}
    D {-{-}{-} E[Measuring Instrument]}
    style A fill:\#f9f,stroke:\#333,stroke{-width:2px}
{Highlighting}
{Shaded}
\end{verbatim}
\end{center}

\textbf{Working principle:}

\begin{itemize}
\tightlist
\item
  Two dissimilar metals joined at one end (hot junction)
\item
  Temperature difference between hot and cold junctions generates
  voltage
\item
  Voltage is proportional to temperature difference
\end{itemize}

\textbf{Types of thermocouples:}

{\def\LTcaptype{none} % do not increment counter
\begin{longtable}[]{@{}
  >{\raggedright\arraybackslash}p{(\linewidth - 6\tabcolsep) * \real{0.1224}}
  >{\raggedright\arraybackslash}p{(\linewidth - 6\tabcolsep) * \real{0.2245}}
  >{\raggedright\arraybackslash}p{(\linewidth - 6\tabcolsep) * \real{0.3878}}
  >{\raggedright\arraybackslash}p{(\linewidth - 6\tabcolsep) * \real{0.2653}}@{}}
\toprule\noalign{}
\begin{minipage}[b]{\linewidth}\raggedright
Type
\end{minipage} & \begin{minipage}[b]{\linewidth}\raggedright
Materials
\end{minipage} & \begin{minipage}[b]{\linewidth}\raggedright
Temperature Range
\end{minipage} & \begin{minipage}[b]{\linewidth}\raggedright
Application
\end{minipage} \\
\midrule\noalign{}
\endhead
\bottomrule\noalign{}
\endlastfoot
\textbf{Type K} & Chromel-Alumel & -200^\circC to +1350^\circC & General purpose,
oxidizing atmosphere \\
\textbf{Type J} & Iron-Constantan & -40^\circC to +750^\circC & Reducing
atmosphere, vacuum \\
\textbf{Type E} & Chromel-Constantan & -200^\circC to +900^\circC & Cryogenic,
higher output \\
\textbf{Type T} & Copper-Constantan & -250^\circC to +350^\circC & Low
temperatures, food industry \\
\textbf{Type R/S} & Platinum-Rhodium & 0^\circC to +1700^\circC & High
temperature, laboratory \\
\end{longtable}
}

\textbf{Applications:} Industrial furnaces, engines, chemical
processing, food processing, research

\end{solutionbox}
\begin{mnemonicbox}
``SHOVE theory: Seebeck Hot-cold Output Voltage
Equals Temperature.''

\end{mnemonicbox}
\subsection*{Question 4(a OR) [3
marks]}\label{question-4a-or-3-marks}

\textbf{Differentiate active and passive transducers.}

\begin{solutionbox}

{\def\LTcaptype{none} % do not increment counter
\begin{longtable}[]{@{}
  >{\raggedright\arraybackslash}p{(\linewidth - 4\tabcolsep) * \real{0.2157}}
  >{\raggedright\arraybackslash}p{(\linewidth - 4\tabcolsep) * \real{0.3725}}
  >{\raggedright\arraybackslash}p{(\linewidth - 4\tabcolsep) * \real{0.4118}}@{}}
\toprule\noalign{}
\begin{minipage}[b]{\linewidth}\raggedright
Parameter
\end{minipage} & \begin{minipage}[b]{\linewidth}\raggedright
Active Transducers
\end{minipage} & \begin{minipage}[b]{\linewidth}\raggedright
Passive Transducers
\end{minipage} \\
\midrule\noalign{}
\endhead
\bottomrule\noalign{}
\endlastfoot
\textbf{Energy conversion} & Convert physical quantity directly to
electrical output & Require external power source \\
\textbf{Output signal} & Self-generating & Modulate external energy \\
\textbf{Examples} & Thermocouple, Piezoelectric, Photovoltaic & RTD,
Strain gauge, LVDT \\
\textbf{Sensitivity} & Generally lower & Generally higher \\
\textbf{Circuit complexity} & Simpler & More complex \\
\textbf{Power requirement} & No external power needed & External power
required \\
\end{longtable}
}

\end{solutionbox}
\begin{mnemonicbox}
``SIMPLE difference: Self-powered Is Main Principle
of Leading Energy transducers.''

\end{mnemonicbox}
\subsection*{Question 4(b OR) [4
marks]}\label{question-4b-or-4-marks}

\textbf{Explain Capacitive Transducer with necessary diagram in detail.
Also list application of it.}

\begin{solutionbox}

Capacitive transducer works on the principle of change in capacitance
due to physical displacement.

\begin{center}
\textbf{Mermaid Diagram (Code)}
\begin{verbatim}
{Shaded}
{Highlighting}[]
graph LR
    A[Fixed Plate] {-{-}{-} B[Dielectric]}
    B {-{-}{-} C[Movable Plate]}
    C {-{-}{-} D[Physical Parameter]}
    E[Capacitance Measuring Circuit] {-{-}{-} A}
    E {-{-}{-} C}
    style B fill:\#f9f,stroke:\#333,stroke{-width:2px}
{Highlighting}
{Shaded}
\end{verbatim}
\end{center}

\textbf{Working principle:}

\begin{itemize}
\tightlist
\item
  Capacitance C = ε_{0}εᵣA/d
\item
  Varies with change in: area (A), distance (d), or dielectric constant
  (εᵣ)
\item
  Displacement changes the capacitance
\item
  Measured using bridge circuit or oscillator
\end{itemize}

\textbf{Applications:}

\begin{itemize}
\tightlist
\item
  Pressure measurement
\item
  Liquid level sensing
\item
  Humidity sensors
\item
  Displacement measurement
\item
  Accelerometers
\end{itemize}

\end{solutionbox}
\begin{mnemonicbox}
``CADAP: Capacitance Alters with Distance, Area, or
Permittivity.''

\end{mnemonicbox}
\subsection*{Question 4(c OR) [7
marks]}\label{question-4c-or-7-marks}

\textbf{Explain LVDT Transducer operation, construction with necessary
diagram in detail. Also list advantage, disadvantage and application of
LVDT.}

\begin{solutionbox}

LVDT (Linear Variable Differential Transformer) is an electromechanical
transducer that converts linear displacement into electrical output.

\begin{center}
\textbf{Mermaid Diagram (Code)}
\begin{verbatim}
{Shaded}
{Highlighting}[]
graph LR
    A[Primary Coil] {-{-}{-} B[Secondary Coil 1]}
    A {-{-}{-} C[Secondary Coil 2]}
    D[AC Excitation] {-{-}{-} A}
    E[Ferromagnetic Core] {-{-}{-} F[Core Rod]}
    B {-{-}{-} G[Output Voltage]}
    C {-{-}{-} G}
    style E fill:\#f9f,stroke:\#333,stroke{-width:2px}
{Highlighting}
{Shaded}
\end{verbatim}
\end{center}

\textbf{Construction:}

\begin{itemize}
\tightlist
\item
  Primary coil in center
\item
  Two secondary coils wound symmetrically
\item
  Movable ferromagnetic core
\item
  Signal conditioning circuitry
\end{itemize}

\textbf{Operation:}

\begin{itemize}
\tightlist
\item
  AC excitation energizes primary coil
\item
  Core position determines magnetic coupling to secondaries
\item
  Differential voltage output proportional to displacement
\item
  Phase indicates direction of displacement
\end{itemize}

\textbf{Advantages:}

\begin{itemize}
\tightlist
\item
  Non-contact operation
\item
  Infinite resolution
\item
  High linearity
\item
  Robust construction
\item
  Long operational life
\item
  Immunity to harsh environments
\end{itemize}

\textbf{Disadvantages:}

\begin{itemize}
\tightlist
\item
  Requires AC excitation
\item
  Bulky compared to other sensors
\item
  Affected by external magnetic fields
\item
  Limited dynamic response
\end{itemize}

\textbf{Applications:}

\begin{itemize}
\tightlist
\item
  Precision measurement
\item
  Hydraulic systems
\item
  Aircraft controls
\item
  Power plant controls
\item
  Automated manufacturing
\end{itemize}

\end{solutionbox}
\begin{mnemonicbox}
``CDPOS sensor: Core Displacement Produces Output
Signal.''

\end{mnemonicbox}
\subsection*{Question 5(a) [3 marks]}\label{q5a}

\textbf{Demonstrate working and principle of Semiconductor Temperature
Sensor LM35.}

\begin{solutionbox}

LM35 is an integrated circuit temperature sensor that outputs voltage
linearly proportional to temperature in Celsius.

\begin{verbatim}
     ┌───┬───┬───┐
     │   │   │   │
     │ 1 │ 2 │ 3 │
     │   │   │   │
     └───┴───┴───┘
       │   │   │
       │   │   │
       │   │   └── GND
       │   └────── Output (10mV/^)
       └────────── VCC (+4V to +30V)
\end{verbatim}

\textbf{Working principle:}

\begin{itemize}
\tightlist
\item
  Integrated circuit with built-in temperature-sensing element
\item
  Linear output voltage: +10mV/^\circC
\item
  Calibrated directly in Celsius
\item
  Operating range: -55^\circC to +150^\circC
\end{itemize}

\textbf{Circuit:}

\begin{itemize}
\tightlist
\item
  Requires only power supply connection
\item
  Output directly readable with voltmeter
\item
  No external calibration needed
\end{itemize}

\end{solutionbox}
\begin{mnemonicbox}
``TEN mV TRICK: Temperature Escalation Noted in
milliVolts: Ten Rise Indicates Celsius Kelvin.''

\end{mnemonicbox}
\subsection*{Question 5(b) [4 marks]}\label{q5b}

\textbf{Describe working of Harmonic distortion analyzer with necessary
diagram.}

\begin{solutionbox}

Harmonic distortion analyzer measures the harmonic content in signals to
determine signal quality.

\begin{center}
\textbf{Mermaid Diagram (Code)}
\begin{verbatim}
{Shaded}
{Highlighting}[]
graph LR
    A[Input Signal] {-{-}{} B[Attenuator]}
    B {-{-}{} C[Notch Filter]}
    C {-{-}{} D[Amplifier]}
    D {-{-}{} E[RMS Detector]}
    A {-{-}{} F[Reference RMS]}
    E {-{-}{} G[Calculator]}
    F {-{-}{} G}
    G {-{-}{} H[Display]}
    style C fill:\#f9f,stroke:\#333,stroke{-width:2px}
{Highlighting}
{Shaded}
\end{verbatim}
\end{center}

\textbf{Working principle:}

\begin{itemize}
\tightlist
\item
  Fundamental frequency is filtered out using notch filter
\item
  Remaining harmonics are measured
\item
  THD = (VRMS of harmonics)/(VRMS of fundamental)
\item
  Expressed as percentage or dB
\end{itemize}

\textbf{Operation steps:}

\begin{enumerate}
\tightlist
\item
  Measure total signal RMS
\item
  Filter out fundamental
\item
  Measure remaining harmonics
\item
  Calculate THD ratio
\end{enumerate}

\end{solutionbox}
\begin{mnemonicbox}
``FRONT analysis: Filter Removes Original Note
Totally for Analyzing Leftover Signals.''

\end{mnemonicbox}
\subsection*{Question 5(c) [7 marks]}\label{q5c}

\textbf{Describe working of Spectrum Analyzer with necessary diagram in
detail.}

\begin{solutionbox}

Spectrum Analyzer displays signal amplitude versus frequency, showing
the spectral composition of signals.

\begin{center}
\textbf{Mermaid Diagram (Code)}
\begin{verbatim}
{Shaded}
{Highlighting}[]
graph LR
    A[RF Input] {-{-}{} B[Attenuator]}
    B {-{-}{} C[Mixer]}
    D[Local Oscillator] {-{-}{} C}
    C {-{-}{} E[IF Filter]}
    E {-{-}{} F[Detector]}
    F {-{-}{} G[Display]}
    H[Sweep Generator] {-{-}{} D}
    H {-{-}{} G}
    style E fill:\#f9f,stroke:\#333,stroke{-width:2px}
{Highlighting}
{Shaded}
\end{verbatim}
\end{center}

\textbf{Working principle:}

\begin{itemize}
\tightlist
\item
  \textbf{Superheterodyne principle}: Input signal mixed with local
  oscillator
\item
  \textbf{Sweep technique}: LO frequency swept across range of interest
\item
  \textbf{Resolution bandwidth}: Controlled by IF filter bandwidth
\item
  \textbf{Detection}: Converts IF signal to amplitude information
\item
  \textbf{Display}: Shows frequency domain representation
\end{itemize}

\textbf{Types:}

\begin{itemize}
\tightlist
\item
  Swept-tuned spectrum analyzer
\item
  FFT-based spectrum analyzer
\item
  Real-time spectrum analyzer
\end{itemize}

\textbf{Applications:}

\begin{itemize}
\tightlist
\item
  Signal analysis
\item
  EMI/EMC testing
\item
  Communication systems testing
\item
  Harmonic analysis
\item
  Modulation analysis
\end{itemize}

\end{solutionbox}
\begin{mnemonicbox}
``SAFER view: Sweep Analyzes Frequencies for
Examining RF.''

\end{mnemonicbox}
\subsection*{Question 5(a OR) [3
marks]}\label{question-5a-or-3-marks}

\textbf{Explain analog transducer and digital transducer. Also explain
primary transducer and secondary transducer.}

\begin{solutionbox}

{\def\LTcaptype{none} % do not increment counter
\begin{longtable}[]{@{}
  >{\raggedright\arraybackslash}p{(\linewidth - 2\tabcolsep) * \real{0.5517}}
  >{\raggedright\arraybackslash}p{(\linewidth - 2\tabcolsep) * \real{0.4483}}@{}}
\toprule\noalign{}
\begin{minipage}[b]{\linewidth}\raggedright
Transducer Type
\end{minipage} & \begin{minipage}[b]{\linewidth}\raggedright
Description
\end{minipage} \\
\midrule\noalign{}
\endhead
\bottomrule\noalign{}
\endlastfoot
\textbf{Analog Transducer} & Produces continuous output signal
proportional to input physical quantity \\
\textbf{Digital Transducer} & Produces discrete/binary output signal
that represents input quantity \\
\textbf{Primary Transducer} & Directly converts physical quantity into
electrical signal \\
\textbf{Secondary Transducer} & Converts output of primary transducer
into another form \\
\end{longtable}
}

\begin{verbatim}
Analog vs Digital Output:

Analog:   ────────────────
             /{      /}
            /  {    /  }
           /    {  /    }

Digital:  ┌───┐    ┌───┐
          │   │    │   │
          │   │    │   │
          └───┘    └───┘
\end{verbatim}

\end{solutionbox}
\begin{mnemonicbox}
``PADS: Primary And Digital/analog Secondary.''

\end{mnemonicbox}
\subsection*{Question 5(b OR) [4
marks]}\label{question-5b-or-4-marks}

\textbf{Explain working of Digital IC tester with necessary diagram in
detail.}

\begin{solutionbox}

Digital IC tester verifies functionality of integrated circuits by
applying test patterns and analyzing responses.

\begin{center}
\textbf{Mermaid Diagram (Code)}
\begin{verbatim}
{Shaded}
{Highlighting}[]
graph LR
    A[Microcontroller] {-{-}{} B[Test Pattern Generator]}
    A {-{-}{} C[Result Analyzer]}
    B {-{-}{} D[Test Socket]}
    D {-{-}{} C}
    A {-{-}{} E[Display/Interface]}
    F[Power Supply] {-{-}{} D}
    style D fill:\#f9f,stroke:\#333,stroke{-width:2px}
{Highlighting}
{Shaded}
\end{verbatim}
\end{center}

\textbf{Working principle:}

\begin{itemize}
\tightlist
\item
  IC inserted in ZIF (Zero Insertion Force) socket
\item
  Test parameters selected for IC type
\item
  Pattern generator applies specific input signals
\item
  Outputs compared with expected results
\item
  Pass/fail indication displayed
\end{itemize}

\textbf{Features:}

\begin{itemize}
\tightlist
\item
  Tests TTL, CMOS, memory ICs
\item
  Identifies unknown ICs
\item
  Detects open/short circuits
\item
  Function verification
\end{itemize}

\end{solutionbox}
\begin{mnemonicbox}
``TRIG test: Test, Run patterns, Identify faults,
Generate report.''

\end{mnemonicbox}
\subsection*{Question 5(c OR) [7
marks]}\label{question-5c-or-7-marks}

\textbf{Explain working of function generator with necessary diagram in
detail.}

\begin{solutionbox}

Function generator produces various waveforms at different frequencies
for testing electronic circuits.

\begin{center}
\textbf{Mermaid Diagram (Code)}
\begin{verbatim}
{Shaded}
{Highlighting}[]
graph LR
    A[Frequency Control] {-{-}{} B[Oscillator]}
    C[Waveform Selector] {-{-}{} D[Waveform Shaper]}
    B {-{-}{} D}
    D {-{-}{} E[Amplitude Control]}
    E {-{-}{} F[Output Amplifier]}
    F {-{-}{} G[Output]}
    H[DC Offset] {-{-}{} F}
    style B fill:\#f9f,stroke:\#333,stroke{-width:2px}
{Highlighting}
{Shaded}
\end{verbatim}
\end{center}

\textbf{Working principle:}

\begin{itemize}
\tightlist
\item
  \textbf{Oscillator}: Generates basic waveform (usually triangle)
\item
  \textbf{Waveform shaper}: Converts to sine, square, triangle, ramp
\item
  \textbf{Frequency control}: Sets oscillation rate
\item
  \textbf{Amplitude control}: Adjusts output voltage level
\item
  \textbf{DC offset}: Adds bias to output signal
\item
  \textbf{Output amplifier}: Provides low impedance output
\end{itemize}

\textbf{Output waveforms:}

\begin{verbatim}
Sine:       ────────────
               /{    /}
              /  {  /  }
             /    {/    }

Square:     ┌───┐  ┌───┐
            │   │  │   │
            │   │  │   │
            └───┘  └───┘

Triangle:   ────────────
                /{    /}
               /  {  /  }
              /    {/    }

Ramp:       ────────────
                /|   /|
               / |  / |
              /  | /  |
\end{verbatim}

\textbf{Applications:}

\begin{itemize}
\tightlist
\item
  Testing amplifiers
\item
  Filter characterization
\item
  Signal analysis
\item
  Educational demonstrations
\item
  Calibration reference
\end{itemize}

\end{solutionbox}
\begin{mnemonicbox}
``SWATOR: Sine Wave And Triangle OSCillator Renders
signals.''

\end{mnemonicbox}

\end{document}
