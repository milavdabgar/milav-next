\documentclass[10pt,a4paper]{article}

% content/resources/templates/preamble.tex
\usepackage[margin=0.6in]{geometry}
\author{Milav Dabgar}
\usepackage{amsmath,amssymb,amsthm}
\usepackage{booktabs}
\usepackage{multirow}
\usepackage{xcolor}
\usepackage{tcolorbox}
\tcbuselibrary{breakable,skins}
\usepackage[colorlinks=true,linkcolor=blue]{hyperref}
\usepackage{titlesec}
\usepackage{enumitem}
\usepackage{tikz}
\usepackage{pgfplots}
\usepackage{circuitikz}
\usepackage[version=4]{mhchem}
\usepackage{longtable}
\usepackage{array}
\usepackage{float}
\usepackage{caption}
\usepackage{listings}

\lstset{
  basicstyle=\small\ttfamily,
  breaklines=true,
  breakatwhitespace=false,
  postbreak=\mbox{\textcolor{red}{$\hookrightarrow$}\space},
  float=false,
  numbers=left,
  numberstyle=\tiny\color{gray},
  numbersep=10pt,
  xleftmargin=2em,
  keywordstyle=\color{blue},
  commentstyle=\color{green!60!black},
  stringstyle=\color{purple},
  backgroundcolor=\color{gray!5},
  showstringspaces=false,
  tabsize=2,
  captionpos=b,
  keepspaces=true,
  columns=flexible
}

\pgfplotsset{compat=1.18}
\usetikzlibrary{shapes,arrows,positioning,calc,patterns,decorations.pathmorphing,decorations.markings,arrows.meta}

% Color scheme
\definecolor{headcolor}{RGB}{0,102,204}
\definecolor{keycolor}{RGB}{220,20,60}
\definecolor{solutioncolor}{RGB}{34,139,34}
\definecolor{mnemoniccolor}{RGB}{148,0,211}
\definecolor{codecolor}{RGB}{0,0,100}

% Spacing
\setlength{\parskip}{3pt}
\setlist[itemize]{nosep}
\setlist[enumerate]{nosep}

% Title formatting
\titleformat{\section}{\Large\bfseries\color{headcolor}}{\thesection}{1em}{}
\titleformat{\subsection}{\large\bfseries\color{headcolor}}{\thesubsection}{1em}{}

% Pandoc tightlist compatibility
\providecommand{\tightlist}{%
  \setlength{\itemsep}{0pt}\setlength{\parskip}{0pt}}

% Pandoc longtable compatibility
\newcounter{none}
\def\thenone{}


% content/resources/templates/english-boxes.tex
% This file is currently empty - it exists to maintain consistency with the import structure.
% Add custom environments here if needed in the future.


\begin{document}

\begin{center}
{\Huge\bfseries\color{headcolor} Subject Name Solutions}\\[5pt]
{\LARGE 4331102 -- Summer 2023}\\[3pt]
{\large Semester 1 Study Material}\\[3pt]
{\normalsize\textit{Detailed Solutions and Explanations}}
\end{center}

\vspace{10pt}

\subsection*{Question 1(a) [3 marks]}\label{q1a}

\textbf{Illustrate steps to minimize that all type of systematic error.}

\begin{solutionbox}

Steps to minimize systematic errors:

{\def\LTcaptype{none} % do not increment counter
\begin{longtable}[]{@{}
  >{\raggedright\arraybackslash}p{(\linewidth - 2\tabcolsep) * \real{0.3158}}
  >{\raggedright\arraybackslash}p{(\linewidth - 2\tabcolsep) * \real{0.6842}}@{}}
\toprule\noalign{}
\begin{minipage}[b]{\linewidth}\raggedright
Step
\end{minipage} & \begin{minipage}[b]{\linewidth}\raggedright
Description
\end{minipage} \\
\midrule\noalign{}
\endhead
\bottomrule\noalign{}
\endlastfoot
1. Calibration & Periodically calibrate instruments against standard
references \\
2. Correction & Apply correction factors or offset values \\
3. Control & Maintain constant environmental conditions (temperature,
humidity) \\
4. Technique & Use proper measurement techniques and procedures \\
5. Equipment & Select appropriate instruments with required accuracy \\
\end{longtable}
}

\end{solutionbox}
\begin{mnemonicbox}
``CCCTS: Calibrate, Correct, Control, Technique,
Select''

\end{mnemonicbox}
\subsection*{Question 1(b) [4 marks]}\label{q1b}

\textbf{Define: Resolution, Precision, Sensitivity and Accuracy.}

\begin{solutionbox}

{\def\LTcaptype{none} % do not increment counter
\begin{longtable}[]{@{}
  >{\raggedright\arraybackslash}p{(\linewidth - 2\tabcolsep) * \real{0.3333}}
  >{\raggedright\arraybackslash}p{(\linewidth - 2\tabcolsep) * \real{0.6667}}@{}}
\toprule\noalign{}
\begin{minipage}[b]{\linewidth}\raggedright
Term
\end{minipage} & \begin{minipage}[b]{\linewidth}\raggedright
Definition
\end{minipage} \\
\midrule\noalign{}
\endhead
\bottomrule\noalign{}
\endlastfoot
\textbf{Resolution} & The smallest change in input that can be detected
by the instrument \\
\textbf{Precision} & Consistency or repeatability of measurements with
minimal random error \\
\textbf{Sensitivity} & The ratio of change in output to the change in
input (ΔO/ΔI) \\
\textbf{Accuracy} & Closeness of measured value to the true or accepted
standard value \\
\end{longtable}
}

\textbf{Diagram:}

\begin{center}
\textbf{Mermaid Diagram (Code)}
\begin{verbatim}
{Shaded}
{Highlighting}[]
graph TD
    A[Measurement Quality] {-{-}{} B[Resolution]}
    A {-{-}{} C[Precision]}
    A {-{-}{} D[Sensitivity]}
    A {-{-}{} E[Accuracy]}
    B {-{-}{} F[Smallest detectable change]}
    C {-{-}{} G[Repeatability]}
    D {-{-}{} H[Output/Input ratio]}
    E {-{-}{} I[Closeness to truth]}
{Highlighting}
{Shaded}
\end{verbatim}
\end{center}

\end{solutionbox}
\begin{mnemonicbox}
``RSPA: Resolve Signals Precisely and Accurately''

\end{mnemonicbox}
\subsection*{Question 1(c) [7 marks]}\label{q1c}

\textbf{Explain a principle of Q Meter and Working of practical Q
Meter.}

\begin{solutionbox}

Q Meter operates on the resonance principle to measure quality factor
(Q) of coils and capacitors.

\textbf{Principle:}

\begin{itemize}
\tightlist
\item
  Based on series resonance where Q = XL/R or XC/R at resonance
\item
  Measures voltage magnification at resonance condition
\end{itemize}

\textbf{Working of practical Q meter:}

{\def\LTcaptype{none} % do not increment counter
\begin{longtable}[]{@{}
  >{\raggedright\arraybackslash}p{(\linewidth - 2\tabcolsep) * \real{0.5238}}
  >{\raggedright\arraybackslash}p{(\linewidth - 2\tabcolsep) * \real{0.4762}}@{}}
\toprule\noalign{}
\begin{minipage}[b]{\linewidth}\raggedright
Component
\end{minipage} & \begin{minipage}[b]{\linewidth}\raggedright
Function
\end{minipage} \\
\midrule\noalign{}
\endhead
\bottomrule\noalign{}
\endlastfoot
Oscillator & Generates variable frequency signal (50kHz to 50MHz) \\
Work coil & Inductor under test (connected in series with calibrated
capacitor) \\
Capacitor & Variable calibrated capacitor for resonance tuning \\
VTVM & Measures resonant voltage across capacitor \\
Shunt resistor & Monitors current through the circuit \\
\end{longtable}
}

\textbf{Diagram:}

\begin{verbatim}
+{-{-}{-}{-}{-}{-}{-}{-}{-}{-}{-}{-}{-}{-}{-}+       +{-}{-}{-}{-}{-}{-}{-}{-}{-}{-}{-}{-}{-}{-}{-}+}
|  RF           |       |               |
|  OSCILLATOR   +{-{-}{-}{-}{-}{-}{-}  WORK COIL    |}
|               |       |   (Lx)        |
+{-{-}{-}{-}{-}{-}{-}{-}{-}{-}{-}{-}{-}{-}{-}+       +{-}{-}{-}{-}{-}{-}{-}+{-}{-}{-}{-}{-}{-}{-}+}
                                |
+{-{-}{-}{-}{-}{-}{-}{-}{-}{-}{-}{-}{-}{-}{-}+       +{-}{-}{-}{-}{-}{-}{-}v{-}{-}{-}{-}{-}{-}{-}+}
|               |       |               |
|    VTVM       {{-}{-}{-}{-}{-}{-}{-}+  CAPACITOR    |}
|  (Q READING)  |       |   (C)         |
+{-{-}{-}{-}{-}{-}{-}{-}{-}{-}{-}{-}{-}{-}{-}+       +{-}{-}{-}{-}{-}{-}{-}{-}{-}{-}{-}{-}{-}{-}{-}+}
\end{verbatim}

\begin{itemize}
\tightlist
\item
  \textbf{Q factor calculation}: Q = V_{2}/V_{1} where V_{2} is voltage across
  capacitor and V_{1} is the applied voltage
\item
  \textbf{Applications}: Testing RF components, coil quality measurement
\item
  \textbf{Resonance indication}: Maximum voltage across capacitor
  indicates resonance
\end{itemize}

\end{solutionbox}
\begin{mnemonicbox}
``VOCAL: Voltage ratio at resonance Oscillator
Creates Amplification to measure coiL quality''

\end{mnemonicbox}
\subsection*{Question 1(c OR) [7
marks]}\label{question-1c-or-7-marks}

\textbf{Explain Wheatstone bridge and derive equation for balanced
condition. State application and limitation of Wheatstone bridge.}

\begin{solutionbox}

Wheatstone bridge is a network used to measure unknown resistance with
high precision.

\textbf{Circuit diagram:}

\begin{verbatim}
       A
       o
       |
      +{-+}
      |R1|
      +{-+}
       |
R      o{-{-}{-}{-}{-}{-}{-}{-}{-}o D}
       |         |
      +{-+       +{-}+}
      |R2|      |Rx|
      +{-+       +{-}+}
       |         |
       o{-{-}{-}{-}{-}{-}{-}{-}{-}o}
       B         C
     
    G = Galvanometer
    Rx = Unknown resistance
\end{verbatim}

\textbf{Balanced condition equation derivation:}

\begin{itemize}
\tightlist
\item
  At balance, no current flows through galvanometer
\item
  Potential at point D = Potential at point B
\item
  Voltage across R_{1} = Voltage across Rx
\item
  Voltage across R_{2} = Voltage across R_{3}
\end{itemize}

Therefore:

\begin{itemize}
\tightlist
\item
  (R_{1}/R_{2}) = (Rx/R_{3})
\item
  Rx = R_{3}(R_{1}/R_{2})
\end{itemize}

\textbf{Applications:}

{\def\LTcaptype{none} % do not increment counter
\begin{longtable}[]{@{}
  >{\raggedright\arraybackslash}p{(\linewidth - 2\tabcolsep) * \real{0.5000}}
  >{\raggedright\arraybackslash}p{(\linewidth - 2\tabcolsep) * \real{0.5000}}@{}}
\toprule\noalign{}
\begin{minipage}[b]{\linewidth}\raggedright
Application
\end{minipage} & \begin{minipage}[b]{\linewidth}\raggedright
Description
\end{minipage} \\
\midrule\noalign{}
\endhead
\bottomrule\noalign{}
\endlastfoot
Precision resistance measurement & Accurate measurement of unknown
resistors \\
Temperature sensing & When used with RTD or thermistor \\
Strain measurement & With strain gauges for stress analysis \\
Transducer interface & Converting physical quantities to electrical
signals \\
\end{longtable}
}

\textbf{Limitations:}

{\def\LTcaptype{none} % do not increment counter
\begin{longtable}[]{@{}
  >{\raggedright\arraybackslash}p{(\linewidth - 2\tabcolsep) * \real{0.4800}}
  >{\raggedright\arraybackslash}p{(\linewidth - 2\tabcolsep) * \real{0.5200}}@{}}
\toprule\noalign{}
\begin{minipage}[b]{\linewidth}\raggedright
Limitation
\end{minipage} & \begin{minipage}[b]{\linewidth}\raggedright
Description
\end{minipage} \\
\midrule\noalign{}
\endhead
\bottomrule\noalign{}
\endlastfoot
Low resistance measurement & Poor accuracy for very low resistances
(\textless1Ω) \\
Sensitivity & Limited by galvanometer sensitivity \\
Range & Limited range of measurement (typically 1Ω to 100kΩ) \\
Contact resistance & Affects accuracy in low resistance measurements \\
\end{longtable}
}

\end{solutionbox}
\begin{mnemonicbox}
``BEAR: Balance Equation at Arms Ratio''

\end{mnemonicbox}
\subsection*{Question 2(a) [3 marks]}\label{q2a}

\textbf{Differentiate between moving iron and moving coil type
instruments.}

\begin{solutionbox}

{\def\LTcaptype{none} % do not increment counter
\begin{longtable}[]{@{}
  >{\raggedright\arraybackslash}p{(\linewidth - 4\tabcolsep) * \real{0.1864}}
  >{\raggedright\arraybackslash}p{(\linewidth - 4\tabcolsep) * \real{0.4068}}
  >{\raggedright\arraybackslash}p{(\linewidth - 4\tabcolsep) * \real{0.4068}}@{}}
\toprule\noalign{}
\begin{minipage}[b]{\linewidth}\raggedright
Parameter
\end{minipage} & \begin{minipage}[b]{\linewidth}\raggedright
Moving Iron Instrument
\end{minipage} & \begin{minipage}[b]{\linewidth}\raggedright
Moving Coil Instrument
\end{minipage} \\
\midrule\noalign{}
\endhead
\bottomrule\noalign{}
\endlastfoot
Operating principle & Magnetic attraction or repulsion & Electromagnetic
force on current-carrying conductor \\
Scales & Non-uniform scale & Uniform scale \\
Accuracy & Lower (1-2.5\%) & Higher (0.1-1\%) \\
Frequency range & Works for both AC and DC & Only DC (unless
rectified) \\
Damping & Air friction damping & Eddy current damping \\
Power consumption & Higher & Lower \\
\end{longtable}
}

\end{solutionbox}
\begin{mnemonicbox}
``IRON-COIL: Iron uses Repulsion with Non-uniform
scale; COIL uses Current with Organized, Improved, Linear scale''

\end{mnemonicbox}
\subsection*{Question 2(b) [4 marks]}\label{q2b}

\textbf{Draw the construction diagram of clamp on Ammeter and explain in
detail.}

\begin{solutionbox}

\textbf{Construction diagram of clamp-on ammeter:}

\begin{verbatim}
                  +{-{-}{-}{-}{-}{-}{-}{-}{-}{-}{-}{-}+}
     +{-{-}{-}{-}+       |   Display  |}
     |    |       +{-{-}{-}{-}{-}{-}{-}{-}{-}{-}{-}{-}+}
     |    |       +{-{-}{-}{-}{-}{-}{-}{-}{-}{-}{-}{-}+}
     | C  |       |    CT      |
     | L  |       |            |
     | A  +{-{-}{-}{-}{-}{-}{-}+    |       |}
     | M  +{-{-}{-}{-}{-}{-}{-}+    |       |}
     | P  |       | Circuit    |
     |    |       |            |
     |    |       +{-{-}{-}{-}{-}{-}{-}{-}{-}{-}{-}{-}+}
     +{-{-}{-}{-}+       +{-}{-}{-}{-}{-}{-}{-}{-}{-}{-}{-}{-}+}
                  |  Controls  |
                  +{-{-}{-}{-}{-}{-}{-}{-}{-}{-}{-}{-}+}
\end{verbatim}

\textbf{Components and working:}

\begin{itemize}
\tightlist
\item
  \textbf{Core}: Split laminated ferromagnetic core that can be
  opened/closed
\item
  \textbf{Coil}: Secondary winding wrapped around the core
\item
  \textbf{Conductor}: Primary conductor (current to be measured) passes
  through the core
\item
  \textbf{Measurement circuit}: Processes induced current and displays
  reading
\item
  \textbf{Spring mechanism}: For easy opening and closing of the jaw
\end{itemize}

\textbf{Working principle}: Based on transformer principle where
conductor acts as single-turn primary winding, creating magnetic flux
proportional to current.

\end{solutionbox}
\begin{mnemonicbox}
``CLASP: Conductor-Loop Amperes Sensed by
Primary-secondary relationship''

\end{mnemonicbox}
\subsection*{Question 2(c) [7 marks]}\label{q2c}

\textbf{Describe working and advantages of Integrating type DVM with
suitable diagram.}

\begin{solutionbox}

Integrating-type Digital Voltmeter converts analog voltage to digital
value using dual-slope integration.

\textbf{Block diagram:}

\begin{verbatim}
+{-{-}{-}{-}{-}{-}{-}{-}{-}{-}{-}{-}{-}+     +{-}{-}{-}{-}{-}{-}{-}{-}{-}{-}{-}{-}+     +{-}{-}{-}{-}{-}{-}{-}{-}{-}{-}{-}{-}{-}+     +{-}{-}{-}{-}{-}{-}{-}{-}{-}{-}+}
| Input       |     | Integrator |     | Comparator  |     | Counter  |
| Circuit     +{-{-}{-}{-}+            +{-}{-}{-}{-}+             +{-}{-}{-}{-}+          |}
| Buffer      |     |            |     |             |     |          |
+{-{-}{-}{-}{-}{-}{-}{-}{-}{-}{-}{-}{-}+     +{-}{-}{-}{-}{-}{-}{-}{-}{-}{-}{-}{-}+     +{-}{-}{-}{-}{-}{-}{-}{-}{-}{-}{-}{-}{-}+     +{-}{-}{-}{-}{-}{-}{-}{-}{-}{-}+}
       \^{                                      \^{}                 |}
       |                                      |                 v
+{-{-}{-}{-}{-}{-}{-}{-}{-}{-}{-}{-}{-}+                        +{-}{-}{-}{-}{-}{-}{-}{-}{-}{-}{-}{-}{-}+    +{-}{-}{-}{-}{-}{-}{-}{-}{-}{-}+}
| Reference   |                        | Control     |    | Display  |
| Voltage     |{{-}{-}{-}{-}{-}{-}{-}{-}{-}{-}{-}{-}{-}{-}{-}{-}{-}{-}{-}{-}{-}{-}{-}+ Logic       |{-}{-}{-}+          |}
| Source      |                        | \& Clock     |    |          |
+{-{-}{-}{-}{-}{-}{-}{-}{-}{-}{-}{-}{-}+                        +{-}{-}{-}{-}{-}{-}{-}{-}{-}{-}{-}{-}{-}+    +{-}{-}{-}{-}{-}{-}{-}{-}{-}{-}+}
\end{verbatim}

\textbf{Working principle:}

{\def\LTcaptype{none} % do not increment counter
\begin{longtable}[]{@{}
  >{\raggedright\arraybackslash}p{(\linewidth - 2\tabcolsep) * \real{0.3500}}
  >{\raggedright\arraybackslash}p{(\linewidth - 2\tabcolsep) * \real{0.6500}}@{}}
\toprule\noalign{}
\begin{minipage}[b]{\linewidth}\raggedright
Phase
\end{minipage} & \begin{minipage}[b]{\linewidth}\raggedright
Description
\end{minipage} \\
\midrule\noalign{}
\endhead
\bottomrule\noalign{}
\endlastfoot
1. Run-up & Unknown input voltage is integrated for fixed time T_{1} \\
2. Run-down & Reference voltage (opposite polarity) is integrated until
output returns to zero \\
3. Measurement & Time T_{2} of run-down is proportional to input voltage \\
4. Display & Digital value based on T_{2}/T_{1} \times Vref is displayed \\
\end{longtable}
}

\textbf{Advantages:}

\begin{itemize}
\tightlist
\item
  \textbf{Noise rejection}: Excellent rejection of power line noise
  (50/60Hz)
\item
  \textbf{Accuracy}: Highly accurate (0.005\% to 0.05\%)
\item
  \textbf{Resolution}: High resolution (up to 6½ digits)
\item
  \textbf{Stability}: Less affected by component tolerances
\item
  \textbf{Common mode rejection}: High CMRR
\end{itemize}

\end{solutionbox}
\begin{mnemonicbox}
``RISES: Ramp Integration Samples and Eliminates
Spikes''

\end{mnemonicbox}
\subsection*{Question 2(a OR) [3
marks]}\label{question-2a-or-3-marks}

\textbf{Differentiate between Digital Voltmeter over Analog Voltmeter.}

\begin{solutionbox}

{\def\LTcaptype{none} % do not increment counter
\begin{longtable}[]{@{}
  >{\raggedright\arraybackslash}p{(\linewidth - 4\tabcolsep) * \real{0.2292}}
  >{\raggedright\arraybackslash}p{(\linewidth - 4\tabcolsep) * \real{0.3958}}
  >{\raggedright\arraybackslash}p{(\linewidth - 4\tabcolsep) * \real{0.3750}}@{}}
\toprule\noalign{}
\begin{minipage}[b]{\linewidth}\raggedright
Parameter
\end{minipage} & \begin{minipage}[b]{\linewidth}\raggedright
Digital Voltmeter
\end{minipage} & \begin{minipage}[b]{\linewidth}\raggedright
Analog Voltmeter
\end{minipage} \\
\midrule\noalign{}
\endhead
\bottomrule\noalign{}
\endlastfoot
Display & Numeric display (digits) & Pointer movement on scale \\
Reading error & No parallax error & Subject to parallax error \\
Resolution & Higher (limited by number of digits) & Limited by scale
divisions \\
Accuracy & Better (typically 0.05\% to 0.5\%) & Lower (typically 1\% to
3\%) \\
Output & Can provide digital output for interfacing & No direct digital
output \\
Power requirement & Requires power supply & Can be passive (PMMC
type) \\
\end{longtable}
}

\end{solutionbox}
\begin{mnemonicbox}
``DAPPER: Digital Accuracy and Precise readings;
Parallax Error in Reading analog''

\end{mnemonicbox}
\subsection*{Question 2(b OR) [4
marks]}\label{question-2b-or-4-marks}

\textbf{Draw the construction diagram of Moving iron type Meter and
explain in detail.}

\begin{solutionbox}

\textbf{Construction diagram of moving iron meter:}

\begin{verbatim}
                 +{-{-}{-}{-}+ Pointer}
                /
         +{-{-}{-}{-}{-}+  }
         |     |
         | +{-+ | }
Scale    | |\^{| | Moving iron}
+{-{-}{-}{-}{-}{-}{-}{-}+ +{-}+ |}
|        |     |
|        |  \#  | Fixed iron
|        |  \#  |
|        +{-{-}{-}{-}{-}+}
|          | |
|          | | Coil
+{-{-}{-}{-}{-}{-}{-}{-}{-}{-}+ +{-}{-}{-}{-}{-}{-}+}
             Spring
\end{verbatim}

\textbf{Working principle and components:}

\begin{itemize}
\tightlist
\item
  \textbf{Coil}: Creates magnetic field proportional to current
\item
  \textbf{Iron vanes}: Two soft iron pieces (one fixed, one movable)
\item
  \textbf{Movement}: Magnetic repulsion between similarly magnetized
  iron pieces
\item
  \textbf{Control}: Spring provides opposing torque
\item
  \textbf{Damping}: Air friction damping mechanism
\item
  \textbf{Scale}: Non-uniform scale due to non-linear magnetic force
\end{itemize}

\textbf{Types:}

\begin{itemize}
\tightlist
\item
  Attraction type: Works on magnetic attraction principle
\item
  Repulsion type: Works on magnetic repulsion principle
\end{itemize}

\end{solutionbox}
\begin{mnemonicbox}
``MIRROR: Magnetic Interaction Requires
Repulsion/attraction Of Related iron pieces''

\end{mnemonicbox}
\subsection*{Question 2(c OR) [7
marks]}\label{question-2c-or-7-marks}

\textbf{Describe construction diagram of Energy meter and explain in
detail.}

\begin{solutionbox}

Electronic energy meter measures electrical energy consumption in
kilowatt-hours.

\textbf{Construction diagram:}

\begin{verbatim}
+{-{-}{-}{-}{-}{-}{-}{-}{-}{-}{-}{-}{-}{-}{-}{-}{-}{-}{-}{-}{-}{-}{-}{-}{-}{-}{-}+}
|        Display            |
|    +{-{-}{-}+{-}{-}{-}+{-}{-}{-}+{-}{-}{-}+{-}{-}{-}+  |}
|    | 0 | 0 | 0 | 0 | 0 |  |
|    +{-{-}{-}+{-}{-}{-}+{-}{-}{-}+{-}{-}{-}+{-}{-}{-}+  |}
|                           |
|  +{-{-}{-}{-}{-}{-}{-}{-}{-}{-}{-}{-}{-}{-}{-}{-}{-}{-}{-}{-}{-}+  |}
|  |    Microcontroller  |  |
|  +{-{-}{-}{-}{-}{-}{-}{-}{-}{-}+{-}{-}{-}{-}{-}{-}{-}{-}{-}{-}+  |}
|             \^{             |}
|             |             |
|  +{-{-}{-}{-}{-}{-}{-}{-}{-}{-}+{-}{-}{-}{-}{-}{-}{-}{-}{-}+   |}
|  | Signal Conditioning|   |
|  +{-{-}{-}{-}{-}{-}{-}{-}{-}{-}+{-}{-}{-}{-}{-}{-}{-}{-}{-}+   |}
|             \^{             |}
|             |             |
|  +{-{-}{-}{-}{-}{-}{-}{-}{-}{-}+{-}{-}{-}{-}{-}{-}{-}{-}{-}{-}+  |}
|  | Voltage \& Current   |  |
|  | Sensing Circuits    |  |
|  +{-{-}{-}{-}{-}{-}{-}{-}{-}{-}+{-}{-}{-}{-}{-}{-}{-}{-}{-}{-}+  |}
|             \^{             |}
|      Input Terminals      |
+{-{-}{-}{-}{-}{-}{-}{-}{-}{-}{-}{-}{-}{-}{-}{-}{-}{-}{-}{-}{-}{-}{-}{-}{-}{-}{-}+}
\end{verbatim}

\textbf{Components and working:}

{\def\LTcaptype{none} % do not increment counter
\begin{longtable}[]{@{}
  >{\raggedright\arraybackslash}p{(\linewidth - 2\tabcolsep) * \real{0.5238}}
  >{\raggedright\arraybackslash}p{(\linewidth - 2\tabcolsep) * \real{0.4762}}@{}}
\toprule\noalign{}
\begin{minipage}[b]{\linewidth}\raggedright
Component
\end{minipage} & \begin{minipage}[b]{\linewidth}\raggedright
Function
\end{minipage} \\
\midrule\noalign{}
\endhead
\bottomrule\noalign{}
\endlastfoot
Voltage sensor & Potential transformer or resistive divider to measure
voltage \\
Current sensor & Current transformer or shunt resistor to measure
current \\
Multiplier & Multiplies instantaneous voltage and current values \\
Integrator & Integrates power over time to calculate energy \\
Microcontroller & Processes signals and calculates energy consumption \\
Display & LCD or LED to show consumption in kWh \\
Pulse LED & Blinks at a rate proportional to power consumption \\
\end{longtable}
}

\textbf{Working principle:}

\begin{enumerate}
\tightlist
\item
  Voltage and current are sensed by respective sensors
\item
  Signals are multiplied to obtain instantaneous power
\item
  Power is integrated over time to calculate energy
\item
  Energy is displayed as kilowatt-hours (kWh)
\end{enumerate}

\end{solutionbox}
\begin{mnemonicbox}
``WATTAGE: Work And Time Tracked As Generated
Electrical energy''

\end{mnemonicbox}
\subsection*{Question 3(a) [3 marks]}\label{q3a}

\textbf{Apply Lissajous pattern for frequency measurement and Phase
angle measurement.}

\begin{solutionbox}

Lissajous patterns on oscilloscope screen help measure frequency ratio
and phase difference.

\textbf{Frequency measurement:}

\begin{itemize}
\tightlist
\item
  Apply reference signal to X-axis and unknown signal to Y-axis
\item
  Frequency ratio = Number of tangent points on Y-axis / Number of
  tangent points on X-axis
\item
  Frequency of unknown = Frequency of reference \times Frequency ratio
\end{itemize}

{\def\LTcaptype{none} % do not increment counter
\begin{longtable}[]{@{}ll@{}}
\toprule\noalign{}
Pattern & Frequency Ratio (Y:X) \\
\midrule\noalign{}
\endhead
\bottomrule\noalign{}
\endlastfoot
\pandocbounded{\includegraphics[keepaspectratio,alt={Circle}]{O}} &
1:1 \\
\pandocbounded{\includegraphics[keepaspectratio,alt={Figure-8}]{8}} &
2:1 \\
\pandocbounded{\includegraphics[keepaspectratio,alt={Complex}]{\infty}} &
n:m \\
\end{longtable}
}

\textbf{Phase angle measurement:}

\begin{itemize}
\tightlist
\item
  If both frequencies are equal, phase angle (φ) can be measured
\item
φ = sin^{-}^{1}(A/B) where

A = minor axis and

B = major axis of ellipse

\end{itemize}

\end{solutionbox}
\begin{mnemonicbox}
``LIPS: Lissajous Indicates Phase and Signal
frequency''

\end{mnemonicbox}
\subsection*{Question 3(b) [4 marks]}\label{q3b}

\textbf{Explain Graticules in CRO also Explain its types.}

\begin{solutionbox}

Graticules are reference markings on CRO screen for measurements.

{\def\LTcaptype{none} % do not increment counter
\begin{longtable}[]{@{}
  >{\raggedright\arraybackslash}p{(\linewidth - 4\tabcolsep) * \real{0.3810}}
  >{\raggedright\arraybackslash}p{(\linewidth - 4\tabcolsep) * \real{0.3095}}
  >{\raggedright\arraybackslash}p{(\linewidth - 4\tabcolsep) * \real{0.3095}}@{}}
\toprule\noalign{}
\begin{minipage}[b]{\linewidth}\raggedright
Graticule Type
\end{minipage} & \begin{minipage}[b]{\linewidth}\raggedright
Description
\end{minipage} & \begin{minipage}[b]{\linewidth}\raggedright
Application
\end{minipage} \\
\midrule\noalign{}
\endhead
\bottomrule\noalign{}
\endlastfoot
\textbf{Internal graticule} & Markings inside CRT glass & Eliminates
parallax error \\
\textbf{External graticule} & Plastic overlay on screen & Replaceable,
economical \\
\textbf{Electronic graticule} & Generated electronically & Digital
storage oscilloscopes \\
\end{longtable}
}

\textbf{Standard graticule features:}

\begin{itemize}
\tightlist
\item
  10 \times 8 divisions typically
\item
  Center lines darker for reference
\item
  Small hash marks for subdivisions
\item
  Percentage markings (rise time)
\end{itemize}

\textbf{Diagram:}

\begin{verbatim}
+{-{-}{-}{-}{-}{-}{-}{-}{-}{-}{-}{-}{-}{-}{-}{-}{-}{-}{-}{-}{-}{-}{-}{-}{-}{-}{-}{-}{-}{-}{-}{-}{-}{-}{-}{-}{-}+}
|                                     |
|                                     |
|     |           |           |       |
|{-{-}{-}{-}{-}+{-}{-}{-}{-}{-}{-}{-}{-}{-}{-}{-}+{-}{-}{-}{-}{-}{-}{-}{-}{-}{-}{-}+{-}{-}{-}{-}{-}{-}{-}|}
|     |           |           |       |
|     |           |           |       |
|     |           |           |       |
|{-{-}{-}{-}{-}+{-}{-}{-}{-}{-}{-}{-}{-}{-}{-}{-}+{-}{-}{-}{-}{-}{-}{-}{-}{-}{-}{-}+{-}{-}{-}{-}{-}{-}{-}|}
|     |           |           |       |
|     |           |           |       |
|                                     |
+{-{-}{-}{-}{-}{-}{-}{-}{-}{-}{-}{-}{-}{-}{-}{-}{-}{-}{-}{-}{-}{-}{-}{-}{-}{-}{-}{-}{-}{-}{-}{-}{-}{-}{-}{-}{-}+}
\end{verbatim}

\end{solutionbox}
\begin{mnemonicbox}
``GRID: Graticule References for Intensity and
Distance''

\end{mnemonicbox}
\subsection*{Question 3(c) [7 marks]}\label{q3c}

\textbf{Describe Construction, Block diagram, working and advantage of
Digital storage oscilloscope (DSO).}

\begin{solutionbox}

Digital Storage Oscilloscope (DSO) converts analog signals to digital
for storage and processing.

\textbf{Block diagram:}

\begin{verbatim}
+{-{-}{-}{-}{-}{-}{-}{-}{-}{-}+     +{-}{-}{-}{-}{-}{-}{-}+     +{-}{-}{-}{-}{-}{-}+     +{-}{-}{-}{-}{-}{-}{-}+     +{-}{-}{-}{-}{-}{-}{-}+}
| Vertical |     | ADC   |     | RAM  |     | DAC   |     | CRT/  |
| Amplifier+{-{-}{-}{-}+       +{-}{-}{-}{-}+      +{-}{-}{-}{-}+       +{-}{-}{-}{-}+ LCD   |}
+{-{-}{-}{-}{-}{-}{-}{-}{-}{-}+     +{-}{-}{-}{-}{-}{-}{-}+     +{-}{-}{-}{-}{-}{-}+     +{-}{-}{-}{-}{-}{-}{-}+     +{-}{-}{-}{-}{-}{-}{-}+}
      \^{                           \^{}                          \^{}}
      |                           |                          |
+{-{-}{-}{-}{-}{-}{-}{-}{-}{-}+                  +{-}{-}{-}{-}{-}{-}{-}+                  +{-}{-}{-}{-}{-}{-}{-}{-}+}
| Input    |                  | CPU   |                  | Display|
|Attenuator|{{-}{-}{-}{-}{-}{-}{-}{-}{-}{-}{-}{-}{-}{-}{-}{-}{-}+       +{-}{-}{-}{-}{-}{-}{-}{-}{-}{-}{-}{-}{-}{-}{-}{-}{-}+ Control|}
+{-{-}{-}{-}{-}{-}{-}{-}{-}{-}+                  +{-}{-}{-}{-}{-}{-}{-}+                  +{-}{-}{-}{-}{-}{-}{-}{-}+}
                                 \^{}
                                 |
                              +{-{-}{-}{-}{-}{-}{-}+}
                              | Timing|
                              |Circuit|
                              +{-{-}{-}{-}{-}{-}{-}+}
\end{verbatim}

\textbf{Working principle:}

\begin{enumerate}
\tightlist
\item
  \textbf{Signal acquisition}: Analog signal is sampled at high speed
\item
  \textbf{A/D conversion}: Continuous signal converted to discrete
  digital values
\item
  \textbf{Storage}: Digital values stored in memory
\item
  \textbf{Processing}: Microprocessor analyzes stored data
\item
  \textbf{Display}: Data converted back to analog for display or shown
  directly on LCD
\end{enumerate}

\textbf{Advantages of DSO:}

{\def\LTcaptype{none} % do not increment counter
\begin{longtable}[]{@{}ll@{}}
\toprule\noalign{}
Advantage & Description \\
\midrule\noalign{}
\endhead
\bottomrule\noalign{}
\endlastfoot
Pre-trigger viewing & Can see signal before trigger event \\
Single-shot capture & Can capture transient events \\
Waveform storage & Can save waveforms for later analysis \\
Signal processing & Advanced mathematical operations on signals \\
Automated measurements & Automatic parameter measurements \\
Digital interfaces & Can transfer data to computers \\
\end{longtable}
}

\end{solutionbox}
\begin{mnemonicbox}
``SAMPLE: Storage And Memory Processes Live Events''

\end{mnemonicbox}
\subsection*{Question 3(a OR) [3
marks]}\label{question-3a-or-3-marks}

\textbf{Differentiate between CRO and DSO.}

\begin{solutionbox}

{\def\LTcaptype{none} % do not increment counter
\begin{longtable}[]{@{}
  >{\raggedright\arraybackslash}p{(\linewidth - 4\tabcolsep) * \real{0.2075}}
  >{\raggedright\arraybackslash}p{(\linewidth - 4\tabcolsep) * \real{0.2264}}
  >{\raggedright\arraybackslash}p{(\linewidth - 4\tabcolsep) * \real{0.5660}}@{}}
\toprule\noalign{}
\begin{minipage}[b]{\linewidth}\raggedright
Parameter
\end{minipage} & \begin{minipage}[b]{\linewidth}\raggedright
Analog CRO
\end{minipage} & \begin{minipage}[b]{\linewidth}\raggedright
Digital Storage Oscilloscope
\end{minipage} \\
\midrule\noalign{}
\endhead
\bottomrule\noalign{}
\endlastfoot
Signal processing & Real-time analog & Digitized and stored \\
Storage capability & None (phosphor persistence only) & Can store
waveforms in memory \\
Bandwidth & Typically higher for same price range & Limited by sampling
rate \\
Pre-trigger view & Not possible & Available \\
Single-shot events & Difficult to capture & Easily captured \\
Signal analysis & Basic measurements only & Advanced mathematical
analysis \\
\end{longtable}
}

\end{solutionbox}
\begin{mnemonicbox}
``ASPAD: Analog Shows Present; Digital Archives
Data''

\end{mnemonicbox}
\subsection*{Question 3(b OR) [4
marks]}\label{question-3b-or-4-marks}

\textbf{Explain structure of 10:1 Probes in detail.}

\begin{solutionbox}

10:1 probe reduces signal amplitude by 10 times to extend oscilloscope
range.

\textbf{Structure:}

\begin{verbatim}
    Probe tip       Cable           Compensation
      \^{              \^{}                  \^{}}
+{-{-}{-}{-}{-}+{-}{-}{-}{-}{-}+   +{-}{-}{-}{-}+{-}{-}{-}{-}+   +{-}{-}{-}{-}{-}{-}{-}{-}+{-}{-}{-}{-}{-}{-}{-}{-}+}
|     |     |   |         |   |        |        |
+{-{-}+{-}{-}+{-}{-}{-}{-}{-}+{-}{-}{-}+{-}{-}{-}{-}{-}{-}{-}{-}{-}+{-}{-}{-}+{-}{-}{-}{-}{-}{-}{-}{-}+{-}{-}{-}{-}{-}{-}{-}{-}+}
   |                                    |
   |                                    |
   |  Rm=9MΩ                           |
   +{-{-}///{-}{-}+                        |}
   |          |                        |
   |      Cm=12pF                      |
   |         ||                        |
   |         ||                        |
   +{-{-}{-}{-}{-}{-}{-}{-}{-}++{-}{-}{-}{-}{-}{-}{-}{-}{-}{-}{-}{-}{-}{-}{-}{-}{-}{-}{-}{-}{-}{-}{-}{-}+}
              |
              v
            Ground
\end{verbatim}

\textbf{Components:}

{\def\LTcaptype{none} % do not increment counter
\begin{longtable}[]{@{}ll@{}}
\toprule\noalign{}
Component & Description \\
\midrule\noalign{}
\endhead
\bottomrule\noalign{}
\endlastfoot
Probe tip & Metal contact point that touches circuit \\
Ground clip & Reference connection to circuit ground \\
Compensation network & RC circuit for frequency compensation \\
Probe body & Insulated housing for components \\
Cable & Low-capacitance coaxial cable \\
Connector & BNC connector for oscilloscope input \\
\end{longtable}
}

\textbf{Working principle:}

\begin{itemize}
\tightlist
\item
  Forms voltage divider with oscilloscope input (9MΩ probe + 1MΩ scope =
  10:1 division)
\item
  Compensating capacitor ensures flat frequency response
\item
  Reduces circuit loading effect by increasing effective input impedance
\end{itemize}

\end{solutionbox}
\begin{mnemonicbox}
``TAPER: Ten-to-one Attenuation Preserves and Extends
Range''

\end{mnemonicbox}
\subsection*{Question 3(c OR) [7
marks]}\label{question-3c-or-7-marks}

\textbf{Describe Block diagram, working and application of CRO.}

\begin{solutionbox}

CRO (Cathode Ray Oscilloscope) displays and measures electrical signals.

\textbf{Block diagram:}

\begin{verbatim}
                +{-{-}{-}{-}{-}{-}{-}{-}{-}{-}{-}{-}{-}+}
                | Cathode Ray |
                |    Tube     |
                +{-{-}{-}{-}{-}{-}+{-}{-}{-}{-}{-}{-}+}
                       \^{}
         +{-{-}{-}{-}{-}{-}+      |      +{-}{-}{-}{-}{-}{-}+}
         |      |      |      |      |
+{-{-}{-}{-}{-}{-}{-}{-}{-}+     +{-}{-}{-}{-}{-}{-}+{-}{-}{-}{-}{-}{-}+      +{-}{-}{-}{-}{-}{-}{-}{-}{-}{-}+}
|Vertical |     |             |      |Horizontal|
|Amplifier|     |  Deflection |      |Amplifier |
+{-{-}{-}+{-}{-}{-}{-}{-}+     |   System    |      +{-}{-}{-}{-}+{-}{-}{-}{-}{-}+}
    \^{           |             |           \^{}}
    |           +{-{-}{-}{-}{-}{-}+{-}{-}{-}{-}{-}{-}+           |}
+{-{-}{-}+{-}{-}{-}{-}{-}{-}+           |             +{-}{-}{-}{-}+{-}{-}{-}{-}{-}+}
|Vertical  |           |             |Horizontal|
|Attenuator|           |             |Time Base |
+{-{-}{-}+{-}{-}{-}{-}{-}{-}+           |             +{-}{-}{-}{-}+{-}{-}{-}{-}{-}+}
    \^{                  |                  \^{}}
    |                  v                  |
+{-{-}{-}+{-}{-}{-}{-}+      +{-}{-}{-}{-}{-}{-}+{-}{-}{-}{-}{-}{-}+      +{-}{-}{-}{-}+{-}{-}{-}{-}+}
| Signal |      |   Power     |      | Trigger |
| Input  |      |   Supply    |      | Circuit |
+{-{-}{-}{-}{-}{-}{-}{-}+      +{-}{-}{-}{-}{-}{-}{-}{-}{-}{-}{-}{-}{-}+      +{-}{-}{-}{-}{-}{-}{-}{-}{-}+}
\end{verbatim}

\textbf{Working principle:}

\begin{enumerate}
\tightlist
\item
  \textbf{Electron beam generation}: CRT produces focused electron beam
\item
  \textbf{Vertical deflection}: Y-plates deflect beam proportional to
  input signal
\item
  \textbf{Horizontal deflection}: X-plates sweep beam across screen
\item
  \textbf{Triggering}: Synchronizes sweep with input signal
\item
  \textbf{Display}: Beam strikes phosphor screen creating visible trace
\end{enumerate}

\textbf{Applications of CRO:}

{\def\LTcaptype{none} % do not increment counter
\begin{longtable}[]{@{}ll@{}}
\toprule\noalign{}
Application & Description \\
\midrule\noalign{}
\endhead
\bottomrule\noalign{}
\endlastfoot
Waveform analysis & Visualize signal shape and characteristics \\
Frequency measurement & Measure time period and calculate frequency \\
Phase measurement & Compare phase relationship between signals \\
Voltage measurement & Measure signal amplitude \\
Component testing & Check behavior of electronic components \\
Transient analysis & Observe fast-changing events \\
\end{longtable}
}

\end{solutionbox}
\begin{mnemonicbox}
``VIEW: Voltage Inspection and Electrical Waveform
observation''

\end{mnemonicbox}
\subsection*{Question 4(a) [3 marks]}\label{q4a}

\textbf{Differentiate RTD and Thermistor.}

\begin{solutionbox}

{\def\LTcaptype{none} % do not increment counter
\begin{longtable}[]{@{}
  >{\raggedright\arraybackslash}p{(\linewidth - 4\tabcolsep) * \real{0.1803}}
  >{\raggedright\arraybackslash}p{(\linewidth - 4\tabcolsep) * \real{0.6230}}
  >{\raggedright\arraybackslash}p{(\linewidth - 4\tabcolsep) * \real{0.1967}}@{}}
\toprule\noalign{}
\begin{minipage}[b]{\linewidth}\raggedright
Parameter
\end{minipage} & \begin{minipage}[b]{\linewidth}\raggedright
RTD (Resistance Temperature Detector)
\end{minipage} & \begin{minipage}[b]{\linewidth}\raggedright
Thermistor
\end{minipage} \\
\midrule\noalign{}
\endhead
\bottomrule\noalign{}
\endlastfoot
Material & Pure metals (Pt, Ni, Cu) & Semiconductor materials \\
Resistance-temp relation & Linear (positive) & Highly non-linear
(usually negative) \\
Temperature range & -200^\circC to 850^\circC & -50^\circC to 300^\circC \\
Sensitivity & Lower (0.4\%/^\circC) & Higher (4\%/^\circC) \\
Accuracy & Higher & Lower \\
Cost & Higher & Lower \\
Response time & Slower & Faster \\
\end{longtable}
}

\end{solutionbox}
\begin{mnemonicbox}
``METAL-SEMI: Metal Elements Temperature-Linear
vs.~SEMIconductor Exponential Measurement Instrument''

\end{mnemonicbox}
\subsection*{Question 4(b) [4 marks]}\label{q4b}

\textbf{Give and explain two example of primary and Secondary
transducer.}

\begin{solutionbox}

{\def\LTcaptype{none} % do not increment counter
\begin{longtable}[]{@{}
  >{\raggedright\arraybackslash}p{(\linewidth - 4\tabcolsep) * \real{0.2069}}
  >{\raggedright\arraybackslash}p{(\linewidth - 4\tabcolsep) * \real{0.3448}}
  >{\raggedright\arraybackslash}p{(\linewidth - 4\tabcolsep) * \real{0.4483}}@{}}
\toprule\noalign{}
\begin{minipage}[b]{\linewidth}\raggedright
Type
\end{minipage} & \begin{minipage}[b]{\linewidth}\raggedright
Examples
\end{minipage} & \begin{minipage}[b]{\linewidth}\raggedright
Explanation
\end{minipage} \\
\midrule\noalign{}
\endhead
\bottomrule\noalign{}
\endlastfoot
\textbf{Primary Transducers} & & \\
1. Thermocouple & Directly converts temperature difference to voltage
using Seebeck effect & Two dissimilar metals generate voltage
proportional to temperature difference \\
2. Piezoelectric crystal & Directly converts mechanical force to
electrical charge & Quartz crystal develops charge proportional to
applied pressure \\
\textbf{Secondary Transducers} & & \\
1. Strain gauge & Requires intermediate conversion; change in dimension
alters resistance & Mechanical strain \rightarrow resistance change \rightarrow electrical
signal \\
2. LVDT & Requires intermediate conversion; displacement changes
magnetic coupling & Mechanical displacement \rightarrow magnetic coupling \rightarrow
electrical signal \\
\end{longtable}
}

\textbf{Diagram:}

\begin{center}
\textbf{Mermaid Diagram (Code)}
\begin{verbatim}
{Shaded}
{Highlighting}[]
graph TD
    A[Transducers] {-{-}{} B[Primary]}
    A {-{-}{} C[Secondary]}
    B {-{-}{} D[Direct conversion]}
    C {-{-}{} E[Uses intermediate steps]}
    D {-{-}{} F[Thermocouple: Temperature  Voltage]}
    D {-{-}{} G[Piezoelectric: Force  Charge]}
    E {-{-}{} H[Strain Gauge: Force  Resistance  Voltage]}
    E {-{-}{} I[LVDT: Displacement  Magnetic coupling  Voltage]}
{Highlighting}
{Shaded}
\end{verbatim}
\end{center}

\end{solutionbox}
\begin{mnemonicbox}
``PIDS: Primary Is Direct; Secondary is Stepwise''

\end{mnemonicbox}
\subsection*{Question 4(c) [7 marks]}\label{q4c}

\textbf{Describe Thermocouple with working principle, types and
application.}

\begin{solutionbox}

Thermocouple is a temperature sensor based on the Seebeck effect.

\textbf{Working principle:}

\begin{itemize}
\tightlist
\item
  When two dissimilar metals are joined, a voltage is generated
  proportional to temperature difference
\item
  Seebeck effect: Temperature gradient creates electromotive force
\end{itemize}

\textbf{Diagram:}

\begin{verbatim}
                   Hot Junction
                        V
  Metal A {-{-}{-}+        /         +{-}{-}{-} Metal A}
             |       /   {       |}
             +{-{-}{-}{-}{-}{-}+     +{-}{-}{-}{-}{-}{-}+}
             |      |     |      |
  Metal B {-{-}{-}+      +{-}{-}{-}{-}{-}+      +{-}{-}{-} Metal B}
                        \^{}
                   Cold Junction
                        |
                        v
                    Voltmeter
\end{verbatim}

\textbf{Types of thermocouples:}

{\def\LTcaptype{none} % do not increment counter
\begin{longtable}[]{@{}
  >{\raggedright\arraybackslash}p{(\linewidth - 6\tabcolsep) * \real{0.1224}}
  >{\raggedright\arraybackslash}p{(\linewidth - 6\tabcolsep) * \real{0.2245}}
  >{\raggedright\arraybackslash}p{(\linewidth - 6\tabcolsep) * \real{0.3878}}
  >{\raggedright\arraybackslash}p{(\linewidth - 6\tabcolsep) * \real{0.2653}}@{}}
\toprule\noalign{}
\begin{minipage}[b]{\linewidth}\raggedright
Type
\end{minipage} & \begin{minipage}[b]{\linewidth}\raggedright
Materials
\end{minipage} & \begin{minipage}[b]{\linewidth}\raggedright
Temperature Range
\end{minipage} & \begin{minipage}[b]{\linewidth}\raggedright
Application
\end{minipage} \\
\midrule\noalign{}
\endhead
\bottomrule\noalign{}
\endlastfoot
Type J & Iron-Constantan & -40^\circC to 750^\circC & General purpose, reducing
atmosphere \\
Type K & Chromel-Alumel & -200^\circC to 1350^\circC & Oxidizing atmosphere, high
temperatures \\
Type T & Copper-Constantan & -200^\circC to 350^\circC & Low temperature, food
industry \\
Type E & Chromel-Constantan & -200^\circC to 900^\circC & Highest sensitivity,
cryogenics \\
Type R/S & Platinum-Rhodium & 0^\circC to 1600^\circC & High temperature,
laboratory standards \\
\end{longtable}
}

\textbf{Applications:}

\begin{itemize}
\tightlist
\item
  Industrial temperature measurement
\item
  Furnace and kiln temperature control
\item
  Chemical processing
\item
  Food processing
\item
  Automotive engine sensors
\item
  Medical equipment
\end{itemize}

\end{solutionbox}
\begin{mnemonicbox}
``STEVE: Seebeck Thermoelectric Effect Verifies
Elevated temperatures''

\end{mnemonicbox}
\subsection*{Question 4(a OR) [3
marks]}\label{question-4a-or-3-marks}

\textbf{Demonstrate working and principle Semiconductor Temperature
Sensor LM35.}

\begin{solutionbox}

LM35 is a precision integrated-circuit temperature sensor that provides
output voltage proportional to temperature.

\textbf{Principle:}

\begin{itemize}
\tightlist
\item
  Based on the predictable change in base-emitter voltage (VBE) of a
  transistor with temperature
\item
  Output voltage linearly proportional to Celsius temperature (10mV/^\circC)
\end{itemize}

\textbf{Circuit diagram:}

\begin{verbatim}
    +{-{-}{-}{-}{-}{-}+}
    |      |
 +{-{-}+ Vs   |}
 |  |      |
 |  | LM35 +{-{-}{-}+ Vout (10mV/^)}
 |  |      |
 |  | GND  +{-{-}{-}+}
 |  |      |   |
 |  +{-{-}{-}{-}{-}{-}+   |}
 |             |
 +{-{-}{-}{-}{-}{-}{-}{-}{-}{-}{-}{-}{-}+}
       GND
\end{verbatim}

\textbf{Working characteristics:}

\begin{itemize}
\tightlist
\item
  Linear output: 10mV/^\circC (0.01V/^\circC) scale factor
\item
  Range: -55^\circC to +150^\circC
\item
  Accuracy: \pm0.5^\circC (typical)
\item
  Low self-heating: 0.08^\circC in still air
\item
  Low impedance output: 0.1Ω for 1mA load
\end{itemize}

\end{solutionbox}
\begin{mnemonicbox}
``LOTUS: Linear Output Temperature Units from
Semiconductor''

\end{mnemonicbox}
\subsection*{Question 4(b OR) [4
marks]}\label{question-4b-or-4-marks}

\textbf{Describe incremental type of Optical encoder with it's output
waveform.}

\begin{solutionbox}

Incremental optical encoder generates pulses as shaft rotates to measure
position, speed, and direction.

\textbf{Construction:}

\begin{verbatim}
           +{-{-}{-}{-}{-}{-}{-}{-}{-}+}
           | LED     |
           |    {    |}
           |     {   |}
           |      {  |}
           |       { |}
           |        {|}
+{-{-}{-}{-}{-}{-}{-}{-}{-}{-}+         |}
| Rotating |         |
| Disk     |         |
| with     |         |
| slots    |         |
+{-{-}{-}{-}{-}{-}{-}{-}{-}{-}+         |}
           |        /|
           |       / |
           |      /  |
           |     /   |
           | Photo   |
           | detector|
           +{-{-}{-}{-}{-}{-}{-}{-}{-}+}
\end{verbatim}

\textbf{Output waveform:}

\begin{verbatim}
Channel A: \_\_\_\_\_          \_\_\_\_\_          \_\_\_\_\_
                |\_\_\_\_\_\_\_\_|     |\_\_\_\_\_\_\_\_|     |\_\_\_\_\_\_\_\_

Channel B: \_\_\_      \_\_\_\_\_       \_\_\_\_\_       \_\_\_\_\_
             |\_\_\_\_\_|     |\_\_\_\_\_|     |\_\_\_\_\_|     |\_\_\_\_\_

          {{-}{-}{-}{-}{-}{-}{-}{-}{-}{-} One Rotation {-}{-}{-}{-}{-}{-}{-}{-}{-}{-}}
\end{verbatim}

\textbf{Working principle:}

\begin{itemize}
\tightlist
\item
  Light source (LED) shines through slotted disk
\item
  Photodetectors receive light pulses as disk rotates
\item
  Two output channels (A and B) are 90^\circ out of phase
\item
  Direction determined by which channel leads
\item
  Resolution depends on number of slots on disk
\end{itemize}

\end{solutionbox}
\begin{mnemonicbox}
``PADS: Pulses from A and Determine Speed''

\end{mnemonicbox}
\subsection*{Question 4(c OR) [7
marks]}\label{question-4c-or-7-marks}

\textbf{Describe construction, operation of LVDT with advantages,
disadvantages and application.}

\begin{solutionbox}

LVDT (Linear Variable Differential Transformer) is an electromechanical
transducer that converts linear displacement into electrical signal.

\textbf{Construction:}

\begin{verbatim}
                   Core
              +{-{-}{-}{-}+{-}{-}{-}{-}+}
              |    |    |
              v    v    v
      +{-{-}{-}{-}{-}{-}+++++++++++++{-}{-}{-}{-}{-}{-}+}
      |      |    |    |        |
+{-{-}{-}{-}{-}+{-}{-}{-}{-}{-}{-}+{-}{-}{-}{-}+{-}{-}{-}{-}+{-}{-}{-}{-}{-}{-}{-}{-}+{-}{-}{-}{-}{-}+}
|     |Primary|    |    |Secondary    |
|     |  Coil |    |    |   Coil      |
+{-{-}{-}{-}{-}+{-}{-}{-}{-}{-}{-}+{-}{-}{-}{-}+{-}{-}{-}{-}+{-}{-}{-}{-}{-}{-}{-}{-}+{-}{-}{-}{-}{-}+}
      |      |    |    |        |
      +{-{-}{-}{-}{-}{-}+++++++++++++{-}{-}{-}{-}{-}{-}+}
                   \^{}
                   |
              Moving Core
\end{verbatim}

\textbf{Operation:}

\begin{enumerate}
\tightlist
\item
  AC excitation applied to primary coil
\item
  Magnetic flux couples to secondary coils
\item
  Core position determines differential voltage output
\item
  Null position: Equal voltage in both secondaries
\item
  Movement: Voltage increases in one secondary, decreases in other
\end{enumerate}

\textbf{Advantages:}

{\def\LTcaptype{none} % do not increment counter
\begin{longtable}[]{@{}ll@{}}
\toprule\noalign{}
Advantage & Description \\
\midrule\noalign{}
\endhead
\bottomrule\noalign{}
\endlastfoot
Frictionless & No mechanical contact between core and coils \\
Infinite resolution & Analog output with no quantization \\
Robustness & Long operational life, high reliability \\
Null position stability & Highly stable reference position \\
High sensitivity & Small displacements can be measured \\
\end{longtable}
}

\textbf{Disadvantages:}

{\def\LTcaptype{none} % do not increment counter
\begin{longtable}[]{@{}ll@{}}
\toprule\noalign{}
Disadvantage & Description \\
\midrule\noalign{}
\endhead
\bottomrule\noalign{}
\endlastfoot
AC excitation required & Needs AC power source \\
Temperature sensitive & Output varies with temperature \\
Position limited & Measurement range is limited \\
Bulky & Larger size compared to other sensors \\
\end{longtable}
}

\textbf{Applications:}

\begin{itemize}
\tightlist
\item
  Machine tool positioning
\item
  Hydraulic and pneumatic systems
\item
  Aircraft and missile systems
\item
  Automated manufacturing
\item
  Structural testing
\end{itemize}

\end{solutionbox}
\begin{mnemonicbox}
``MOVE-AC: Magnetic Output Varies with Exact Armature
Core position''

\end{mnemonicbox}
\subsection*{Question 5(a) [3 marks]}\label{q5a}

\textbf{Describe working of Pressure measurement using Capacitive
transducer.}

\begin{solutionbox}

Capacitive pressure transducer uses changes in capacitance to measure
pressure.

\textbf{Working principle:}

\begin{itemize}
\tightlist
\item
  Pressure deforms diaphragm, changing distance between capacitor plates
\item
  Capacitance inversely proportional to distance (C = ε_{0}ε_{a}A/d)
\item
  Change in capacitance is measured and converted to pressure reading
\end{itemize}

\textbf{Diagram:}

\begin{verbatim}
     Pressure
        ↓
    +{-{-}{-}+{-}{-}{-}+}
    |       | Metal housing
+{-{-}{-}+{-}{-}{-}{-}{-}{-}{-}+{-}{-}{-}+}
|   |       |   |
|   |       |   |
|   |    | Diaphragm (movable plate)
|   |       |   |
+{-{-}{-}+{-}{-}{-}+{-}{-}{-}+{-}{-}{-}+}
|   |   |   |   |
|   |   |   |   | Air gap
|   |   |   |   |
+{-{-}{-}+{-}{-}{-}+{-}{-}{-}+{-}{-}{-}+ Fixed plate}
|               |
+{-{-}{-}{-}{-}{-}{-}{-}{-}{-}{-}{-}{-}{-}{-}+}
    Insulator
\end{verbatim}

\textbf{Application:} Industrial process monitoring, atmospheric
pressure measurement, liquid level sensing

\end{solutionbox}
\begin{mnemonicbox}
``CAPS: Capacitance Alters as Pressure Shifts''

\end{mnemonicbox}
\subsection*{Question 5(b) [4 marks]}\label{q5b}

\textbf{Define rise time, fall time, Pulse width and duty cycle.}

\begin{solutionbox}

{\def\LTcaptype{none} % do not increment counter
\begin{longtable}[]{@{}
  >{\raggedright\arraybackslash}p{(\linewidth - 2\tabcolsep) * \real{0.4783}}
  >{\raggedright\arraybackslash}p{(\linewidth - 2\tabcolsep) * \real{0.5217}}@{}}
\toprule\noalign{}
\begin{minipage}[b]{\linewidth}\raggedright
Parameter
\end{minipage} & \begin{minipage}[b]{\linewidth}\raggedright
Definition
\end{minipage} \\
\midrule\noalign{}
\endhead
\bottomrule\noalign{}
\endlastfoot
\textbf{Rise Time} & Time taken for pulse to rise from 10\% to 90\% of
its maximum amplitude \\
\textbf{Fall Time} & Time taken for pulse to fall from 90\% to 10\% of
its maximum amplitude \\
\textbf{Pulse Width} & Time interval between 50\% amplitude points on
rising and falling edges \\
\textbf{Duty Cycle} & Ratio of pulse width to total period, expressed as
percentage \\
\end{longtable}
}

\textbf{Diagram:}

\begin{verbatim}
     \^{ Amplitude}
     |
     |    +{-{-}{-}{-}{-}{-}{-}{-}{-}+}
     |    |         |
90\%  |{-{-}{-}{-}+         +{-}{-}{-}{-}}
     |   /|         |{}
     |  / |         | {}
     | /  |         |  {}
50\%  |/   |         |   {}
     +{-{-}{-}{-}+{-}{-}{-}{-}{-}{-}{-}{-}{-}+{-}{-}{-}{-}+{-}{-}{-} Time}
     |    |         |    |
10\%  |    |         |    |
     |    |{{-}Pulse{-}|    |}
     |    | Width   |    |
     |    |         |    |
     |{{-}{-}|         |{-}{-}|}
     Rise |         |Fall
     Time |         |Time
     |    |{{-}{-}{-}Period{-}{-}{-}|}
\end{verbatim}

\end{solutionbox}
\begin{mnemonicbox}
``RPFD: Rise Pulses, Fall Determines''

\end{mnemonicbox}
\subsection*{Question 5(c) [7 marks]}\label{q5c}

\textbf{Discuss Function generator block diagram.}

\begin{solutionbox}

Function generator produces various waveforms over a range of
frequencies.

\textbf{Block diagram:}

\begin{verbatim}
+{-{-}{-}{-}{-}{-}{-}{-}{-}{-}+     +{-}{-}{-}{-}{-}{-}{-}{-}{-}{-}+     +{-}{-}{-}{-}{-}{-}{-}{-}{-}{-}+     +{-}{-}{-}{-}{-}{-}{-}{-}{-}{-}+     +{-}{-}{-}{-}{-}{-}{-}{-}{-}{-}+}
| Frequency|     | Waveform |     | Amplitude|     | Output   |     | Output   |
| Control  +{-{-}{-}{-}+ Generator+{-}{-}{-}{-}+ Control  +{-}{-}{-}{-}+ Buffer   +{-}{-}{-}{-}+          |}
| Circuit  |     | (VCO)    |     | Circuit  |     | Amplifier|     |          |
+{-{-}{-}{-}{-}{-}{-}{-}{-}{-}+     +{-}{-}{-}{-}{-}{-}{-}{-}{-}{-}+     +{-}{-}{-}{-}{-}{-}{-}{-}{-}{-}+     +{-}{-}{-}{-}{-}{-}{-}{-}{-}{-}+     +{-}{-}{-}{-}{-}{-}{-}{-}{-}{-}+}
      \^{               |                                                  \^{}}
      |               v                                                  |
      |          +{-{-}{-}{-}{-}{-}{-}{-}{-}{-}+     +{-}{-}{-}{-}{-}{-}{-}{-}{-}{-}+                     +{-}{-}{-}{-}{-}{-}{-}{-}{-}{-}{-}+}
      |          | Waveshape|     | DC Offset|                     | Attenuator|
      |          | Circuit  |     | Circuit  |                     | Circuit   |
      |          +{-{-}{-}{-}{-}{-}{-}{-}{-}{-}+     +{-}{-}{-}{-}{-}{-}{-}{-}{-}{-}+                     +{-}{-}{-}{-}{-}{-}{-}{-}{-}{-}{-}+}
      |               |                |                                 \^{}
      |               v                v                                 |
      |          +{-{-}{-}{-}{-}{-}{-}{-}{-}{-}+     +{-}{-}{-}{-}{-}{-}{-}{-}{-}{-}+     +{-}{-}{-}{-}{-}{-}{-}{-}{-}{-}+    +{-}{-}{-}{-}{-}{-}{-}{-}{-}{-}{-}+}
      +{-{-}{-}{-}{-}{-}{-}{-}{-}{-}+ Sync     |     | Duty     |     | Trigger  |    | Protection|}
                 | Output   |     | Cycle    |     | Circuit  |    | Circuit   |
                 +{-{-}{-}{-}{-}{-}{-}{-}{-}{-}+     +{-}{-}{-}{-}{-}{-}{-}{-}{-}{-}+     +{-}{-}{-}{-}{-}{-}{-}{-}{-}{-}+    +{-}{-}{-}{-}{-}{-}{-}{-}{-}{-}{-}+}
\end{verbatim}

\textbf{Function and operation of each block:}

{\def\LTcaptype{none} % do not increment counter
\begin{longtable}[]{@{}
  >{\raggedright\arraybackslash}p{(\linewidth - 2\tabcolsep) * \real{0.4118}}
  >{\raggedright\arraybackslash}p{(\linewidth - 2\tabcolsep) * \real{0.5882}}@{}}
\toprule\noalign{}
\begin{minipage}[b]{\linewidth}\raggedright
Block
\end{minipage} & \begin{minipage}[b]{\linewidth}\raggedright
Function
\end{minipage} \\
\midrule\noalign{}
\endhead
\bottomrule\noalign{}
\endlastfoot
\textbf{Frequency Control} & Sets the operating frequency using variable
capacitor/resistor network \\
\textbf{Waveform Generator} & Voltage-controlled oscillator producing
basic waveform (usually triangle) \\
\textbf{Waveshape Circuit} & Converts triangle wave to sine/square waves
through shaping circuits \\
\textbf{Amplitude Control} & Adjusts output amplitude of the generated
waveform \\
\textbf{DC Offset} & Adds DC bias to shift the waveform up or down from
zero reference \\
\textbf{Output Buffer} & Provides low output impedance for proper
loading \\
\textbf{Attenuator} & Controls final output level with calibrated
steps \\
\textbf{Protection Circuit} & Protects output from short circuits or
overload \\
\end{longtable}
}

\textbf{Output waveforms:}

{\def\LTcaptype{none} % do not increment counter
\begin{longtable}[]{@{}ll@{}}
\toprule\noalign{}
Waveform & Generation Method \\
\midrule\noalign{}
\endhead
\bottomrule\noalign{}
\endlastfoot
Sine & Shaped from triangle wave using non-linear shaping circuit \\
Square & Derived from triangle wave using comparator \\
Triangle & Basic output from integrator circuit \\
Ramp & Modified triangle wave with different rise/fall times \\
Pulse & Square wave with variable duty cycle \\
\end{longtable}
}

\end{solutionbox}
\begin{mnemonicbox}
``FASTEST: Frequency Amplitude Shaping Together
Ensures Signal Types''

\end{mnemonicbox}
\subsection*{Question 5(a OR) [3
marks]}\label{question-5a-or-3-marks}

\textbf{Discuss Working, construction of strain gauge.}

\begin{solutionbox}

Strain gauge converts mechanical deformation to electrical resistance
change.

\textbf{Construction:}

\begin{verbatim}
            Terminals
               ||
    +{-{-}{-}{-}{-}{-}{-}{-}{-}{-}++{-}{-}{-}{-}{-}{-}{-}{-}{-}{-}+}
    |                      |
    | +{-{-}{-}{-}{-}{-}{-}{-}{-}{-}{-}{-}{-}{-}{-}{-}{-}+  |}
    | |  /{////// |  |}
    | | /{/////// |  | Resistive}
    | |/{/////// |  | Grid}
    | |{/ ////// |  |}
    | | {/////// |  |}
    | +{-{-}{-}{-}{-}{-}{-}{-}{-}{-}{-}{-}{-}{-}{-}{-}{-}+  |}
    |                      |
    +{-{-}{-}{-}{-}{-}{-}{-}{-}{-}{-}{-}{-}{-}{-}{-}{-}{-}{-}{-}{-}{-}+}
           Backing material
\end{verbatim}

\textbf{Working principle:}

\begin{itemize}
\tightlist
\item
  Based on piezoresistive effect: resistance changes with mechanical
  deformation
\item
  When bonded to object, strain gauge deforms along with it
\item
  Resistance increases with tension (elongation)
\item
  Resistance decreases with compression (shortening)
\item
  Resistance change is measured using bridge circuit
\end{itemize}

\textbf{Resistance change relation:}

\begin{itemize}
\tightlist
\item
  ΔR/R = GF \times ε
\item
Where: ΔR = resistance change,

R = initial resistance

\item
GF = gauge factor (sensitivity),

ε = strain

\end{itemize}

\textbf{Materials used:}

\begin{itemize}
\tightlist
\item
  Foil: Constantan, Karma, Nichrome alloys
\item
  Semiconductor: Silicon, Germanium for higher sensitivity
\end{itemize}

\end{solutionbox}
\begin{mnemonicbox}
``SERB: Strain Effects Resistance by Bonding''

\end{mnemonicbox}
\subsection*{Question 5(b OR) [4
marks]}\label{question-5b-or-4-marks}

\textbf{Describe working of Digital IC tester with suitable diagrams.}

\begin{solutionbox}

Digital IC tester verifies functionality of integrated circuits by
applying test patterns.

\textbf{Block diagram:}

\begin{verbatim}
    +{-{-}{-}{-}{-}{-}{-}{-}{-}{-}{-}{-}{-}+      +{-}{-}{-}{-}{-}{-}{-}{-}{-}{-}{-}{-}{-}+}
    | Keypad/     |      | Display     |
    | Interface   |      | LCD/LED     |
    +{-{-}{-}{-}{-}{-}+{-}{-}{-}{-}{-}{-}+      +{-}{-}{-}{-}{-}{-}+{-}{-}{-}{-}{-}{-}+}
           |                    \^{}
           v                    |
    +{-{-}{-}{-}{-}{-}+{-}{-}{-}{-}{-}{-}{-}{-}{-}{-}{-}{-}{-}{-}{-}{-}{-}{-}{-}{-}+{-}{-}{-}{-}{-}{-}+}
    |                                  |
    |         Microcontroller          |
    |                                  |
    +{-{-}{-}+{-}{-}{-}{-}{-}{-}{-}{-}{-}{-}{-}{-}+{-}{-}{-}{-}{-}{-}{-}{-}{-}{-}{-}{-}{-}+{-}{-}{-}+}
        |            |             |
        v            v             v
+{-{-}{-}{-}{-}{-}{-}+{-}{-}{-}{-}{-}{-}+ +{-}{-}{-}{-}+{-}{-}{-}{-}{-}{-}+ +{-}{-}{-}{-}+{-}{-}{-}{-}{-}{-}{-}+}
| Test Pattern | | IC Socket | | Result     |
| Generator    | | Interface | | Comparator |
+{-{-}{-}{-}{-}{-}{-}{-}{-}{-}{-}{-}{-}{-}+ +{-}{-}{-}{-}+{-}{-}{-}{-}{-}{-}+ +{-}{-}{-}{-}{-}{-}{-}{-}{-}{-}{-}{-}+}
                      |               \^{}
                      v               |
                +{-{-}{-}{-}{-}+{-}{-}{-}{-}{-}{-}{-}{-}{-}{-}{-}{-}{-}{-}{-}+{-}{-}{-}+}
                |                         |
                |     IC Under Test       |
                |                         |
                +{-{-}{-}{-}{-}{-}{-}{-}{-}{-}{-}{-}{-}{-}{-}{-}{-}{-}{-}{-}{-}{-}{-}{-}{-}+}
\end{verbatim}

\textbf{Working principle:}

\begin{enumerate}
\tightlist
\item
  IC is inserted into test socket
\item
  User selects IC type/number using keypad
\item
  Microcontroller loads appropriate test pattern
\item
  Test patterns applied to IC inputs
\item
  Output responses compared with expected values
\item
  Pass/Fail result displayed
\end{enumerate}

\textbf{Features of digital IC tester:}

\begin{itemize}
\tightlist
\item
  Tests TTL, CMOS, HCMOS logic families
\item
  Can identify unknown ICs by analyzing pin functions
\item
  Performs functional and parametric tests
\item
  Checks for static and dynamic characteristics
\end{itemize}

\end{solutionbox}
\begin{mnemonicbox}
``PIPE: Pattern Input, Pin Examination''

\end{mnemonicbox}
\subsection*{Question 5(c OR) [7
marks]}\label{question-5c-or-7-marks}

\textbf{Discuss working of Spectrum Analyzer with suitable diagrams.}

\begin{solutionbox}

Spectrum analyzer displays signal amplitude versus frequency, showing
frequency components.

\textbf{Block diagram:}

\begin{verbatim}
+{-{-}{-}{-}{-}{-}{-}{-}{-}{-}{-}+    +{-}{-}{-}{-}{-}{-}{-}{-}{-}{-}{-}{-}+    +{-}{-}{-}{-}{-}{-}{-}{-}{-}{-}+    +{-}{-}{-}{-}{-}{-}{-}{-}{-}{-}{-}+    +{-}{-}{-}{-}{-}{-}{-}{-}{-}{-}+}
| RF Input  |    | Attenuator |    | Mixer    |    | IF        |    | Detector |
| Circuit   +{-{-}{-}+ \& Filters  +{-}{-}{-}+ Circuit  +{-}{-}{-}+ Filter    +{-}{-}{-}+ Circuit  |}
+{-{-}{-}{-}{-}{-}{-}{-}{-}{-}{-}+    +{-}{-}{-}{-}{-}{-}{-}{-}{-}{-}{-}{-}+    +{-}{-}{-}{-}+{-}{-}{-}{-}{-}+    +{-}{-}{-}{-}{-}{-}{-}{-}{-}{-}{-}+    +{-}{-}{-}{-}{-}+{-}{-}{-}{-}+}
                                        \^{                                  |}
                                        |                                  v
                                   +{-{-}{-}{-}+{-}{-}{-}{-}{-}+                       +{-}{-}{-}{-}+{-}{-}{-}{-}{-}+}
                                   | Local    |                       | Video    |
                                   | Oscillator|                      | Filter   |
                                   +{-{-}{-}{-}+{-}{-}{-}{-}{-}+                       +{-}{-}{-}{-}{-}+{-}{-}{-}{-}+}
                                        \^{                                   |}
                                        |                                   v
+{-{-}{-}{-}{-}{-}{-}{-}{-}{-}{-}+    +{-}{-}{-}{-}{-}{-}{-}{-}{-}{-}{-}{-}+    +{-}{-}{-}{-}+{-}{-}{-}{-}{-}+                       +{-}{-}{-}{-}{-}+{-}{-}{-}{-}+}
| Control   |    | CPU \&      |    | Sweep    |    +{-{-}{-}{-}{-}{-}{-}{-}{-}{-}{-}{-}+     | Display  |}
| Panel     +{-{-}{-}+ Processor  +{-}{-}{-}+ Generator+{-}{-}{-}+ Horizontal |{-}{-}{-}{-}+ Circuit  |}
+{-{-}{-}{-}{-}{-}{-}{-}{-}{-}{-}+    +{-}{-}{-}{-}{-}{-}{-}{-}{-}{-}{-}{-}+    +{-}{-}{-}{-}{-}{-}{-}{-}{-}{-}+    | Deflection |     +{-}{-}{-}{-}{-}{-}{-}{-}{-}{-}+}
                                                   +{-{-}{-}{-}{-}{-}{-}{-}{-}{-}{-}{-}+}
\end{verbatim}

\textbf{Working principle:}

\begin{enumerate}
\tightlist
\item
  \textbf{Superheterodyne conversion}: Input signal mixed with local
  oscillator
\item
  \textbf{Frequency sweep}: Local oscillator sweeps across frequency
  range
\item
  \textbf{IF filtering}: Narrow bandpass filter selects frequency
  components
\item
  \textbf{Detection}: Amplitude of each frequency component is measured
\item
  \textbf{Display}: Amplitude vs.~frequency plot shown on screen
\end{enumerate}

\textbf{Types of spectrum analyzers:}

{\def\LTcaptype{none} % do not increment counter
\begin{longtable}[]{@{}
  >{\raggedright\arraybackslash}p{(\linewidth - 4\tabcolsep) * \real{0.2000}}
  >{\raggedright\arraybackslash}p{(\linewidth - 4\tabcolsep) * \real{0.3667}}
  >{\raggedright\arraybackslash}p{(\linewidth - 4\tabcolsep) * \real{0.4333}}@{}}
\toprule\noalign{}
\begin{minipage}[b]{\linewidth}\raggedright
Type
\end{minipage} & \begin{minipage}[b]{\linewidth}\raggedright
Principle
\end{minipage} & \begin{minipage}[b]{\linewidth}\raggedright
Application
\end{minipage} \\
\midrule\noalign{}
\endhead
\bottomrule\noalign{}
\endlastfoot
Swept-tuned & Superheterodyne with swept LO & RF and microwave
signals \\
FFT (Fast Fourier Transform) & Digital conversion and FFT algorithm &
Audio and low-frequency signals \\
Real-time & Combination of FFT with high-speed processing & Transient
and dynamic signals \\
\end{longtable}
}

\textbf{Applications:}

\begin{itemize}
\tightlist
\item
  EMI/EMC testing
\item
  Signal purity measurement
\item
  Harmonic distortion analysis
\item
  Communication system testing
\item
  Modulation analysis
\end{itemize}

\end{solutionbox}
\begin{mnemonicbox}
``SHAFT: Sweep, Heterodyne, Analyze Frequency and
Time''

\end{mnemonicbox}

\end{document}
