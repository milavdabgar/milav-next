\documentclass[10pt,a4paper]{article}

% content/resources/templates/preamble.tex
\usepackage[margin=0.6in]{geometry}
\author{Milav Dabgar}
\usepackage{amsmath,amssymb,amsthm}
\usepackage{booktabs}
\usepackage{multirow}
\usepackage{xcolor}
\usepackage{tcolorbox}
\tcbuselibrary{breakable,skins}
\usepackage[colorlinks=true,linkcolor=blue]{hyperref}
\usepackage{titlesec}
\usepackage{enumitem}
\usepackage{tikz}
\usepackage{pgfplots}
\usepackage{circuitikz}
\usepackage[version=4]{mhchem}
\usepackage{longtable}
\usepackage{array}
\usepackage{float}
\usepackage{caption}
\usepackage{listings}

\lstset{
  basicstyle=\small\ttfamily,
  breaklines=true,
  breakatwhitespace=false,
  postbreak=\mbox{\textcolor{red}{$\hookrightarrow$}\space},
  float=false,
  numbers=left,
  numberstyle=\tiny\color{gray},
  numbersep=10pt,
  xleftmargin=2em,
  keywordstyle=\color{blue},
  commentstyle=\color{green!60!black},
  stringstyle=\color{purple},
  backgroundcolor=\color{gray!5},
  showstringspaces=false,
  tabsize=2,
  captionpos=b,
  keepspaces=true,
  columns=flexible
}

\pgfplotsset{compat=1.18}
\usetikzlibrary{shapes,arrows,positioning,calc,patterns,decorations.pathmorphing,decorations.markings,arrows.meta}

% Color scheme
\definecolor{headcolor}{RGB}{0,102,204}
\definecolor{keycolor}{RGB}{220,20,60}
\definecolor{solutioncolor}{RGB}{34,139,34}
\definecolor{mnemoniccolor}{RGB}{148,0,211}
\definecolor{codecolor}{RGB}{0,0,100}

% Spacing
\setlength{\parskip}{3pt}
\setlist[itemize]{nosep}
\setlist[enumerate]{nosep}

% Title formatting
\titleformat{\section}{\Large\bfseries\color{headcolor}}{\thesection}{1em}{}
\titleformat{\subsection}{\large\bfseries\color{headcolor}}{\thesubsection}{1em}{}

% Pandoc tightlist compatibility
\providecommand{\tightlist}{%
  \setlength{\itemsep}{0pt}\setlength{\parskip}{0pt}}

% Pandoc longtable compatibility
\newcounter{none}
\def\thenone{}


% content/resources/templates/gujarati-boxes.tex
\usepackage{fontspec}
\usepackage{polyglossia}

% Set Gujarati as main language (document is primarily in Gujarati)
% Note: gloss-gujarati.ldf doesn't exist in polyglossia, but it will use hyphenation patterns
\setdefaultlanguage{gujarati}
\setotherlanguage{english}

% Configure Gujarati font properly
% Use Language=Default to prevent polyglossia from trying to add language-specific features
% that don't exist for Gujarati, which causes "empty feature" warnings
\newfontfamily\gujaratifont[Script=Gujarati,AutoFakeBold=2.5,AutoFakeSlant=0.3]{Noto Sans Gujarati}
\setmainfont[Script=Gujarati,AutoFakeBold=2.5,AutoFakeSlant=0.3]{Noto Sans Gujarati}
% Use Noto Sans Gujarati for monospace to support Gujarati in text
\setmonofont[Scale=0.9]{Noto Sans Gujarati}

% Configure English to use the same font
\newfontfamily\englishfont[Script=Gujarati,AutoFakeBold=2.5,AutoFakeSlant=0.3]{Noto Sans Gujarati}

% Translations for polyglossia
\gappto\captionsgujarati{
  \renewcommand{\tablename}{કોષ્ટક}
  \renewcommand{\figurename}{આકૃતિ}
}

% Helper for TikZ nodes to ensure Gujarati font
\newcommand{\gu}[1]{{\gujaratifont #1}}

% Custom environments
\newtcolorbox{solutionbox}{
    breakable,
    enhanced,
    colback=solutioncolor!5!white,
    colframe=solutioncolor!75!black,
    fonttitle=\bfseries,
    title=જવાબ
}

\newtcolorbox{solutionboxnobreak}{
 colback=solutioncolor!5!white,
 colframe=solutioncolor!75!black,
 fonttitle=\bfseries,
 title=જવાબ
}

\newtcolorbox{keyformula}{
 breakable,
 enhanced,
 colback=keycolor!5!white,
 colframe=keycolor!75!black,
 fonttitle=\bfseries,
 title=રાસાયણિક સમીકરણ/સૂત્ર
}

\newtcolorbox{mnemonicbox}{
 breakable,
 enhanced,
 colback=mnemoniccolor!5!white,
 colframe=mnemoniccolor!75!black,
 fonttitle=\bfseries,
 title=મેમરી ટ્રીક
}


\begin{document}

\begin{center}
{\Huge\bfseries\color{headcolor} Subject Name (Gujarati)}\\[5pt]
{\LARGE 4331102 -- Winter 2024}\\[3pt]
{\large Semester 1 Study Material}\\[3pt]
{\normalsize\textit{Detailed Solutions and Explanations}}
\end{center}

\vspace{10pt}

\subsection*{પ્રશ્ન 1(અ) [3
ગુણ]}\label{uxaaauxab0uxab6uxaa8-1uxa85-3-uxa97uxaa3}

\textbf{નીચેના શબ્દને વ્યાખ્યાયિત કરો: (1) Accuracy (2) Resolution (3) Error}

\begin{solutionbox}

{\def\LTcaptype{none} % do not increment counter
\begin{longtable}[]{@{}ll@{}}
\toprule\noalign{}
શબ્દ & વ્યાખ્યા \\
\midrule\noalign{}
\endhead
\bottomrule\noalign{}
\endlastfoot
\textbf{Accuracy} & માપન અને વાસ્તવિક મૂલ્ય વચ્ચેની નજીકતા \\
\textbf{Resolution} & નાનામાં નાના ફેરફાર કે જે એક ઉપકરણ દ્વારા માપી શકાય
છે \\
\textbf{Error} & માપેલા મૂલ્ય અને વાસ્તવિક મૂલ્ય વચ્ચેનો તફાવત \\
\end{longtable}
}

\end{solutionbox}
\begin{mnemonicbox}
``ARE સચોટ: Accuracy વાસ્તવિકતા દર્શાવે છે, Error વિચલન
બતાવે છે, Resolution વિગત દર્શાવે છે.''

\end{mnemonicbox}
\subsection*{પ્રશ્ન 1(બ) [4
ગુણ]}\label{uxaaauxab0uxab6uxaa8-1uxaac-4-uxa97uxaa3}

\textbf{અનબાઉન્ડેડ સ્ટ્રેઈન ગેજ ટ્રાન્સડ્યુસરનું બાંધકામ જરૂરી આકૃતિ સાથે વિગતવાર
સમજાવો. તેની એપ્લિકેશનની યાદી બનાવો.}

\begin{solutionbox}

અનબાઉન્ડેડ સ્ટ્રેઈન ગેજમાં પાતળા તારની ગ્રીડ પેટર્ન હોય છે જે એક બેકિંગ મટીરિયલ પર
લગાવેલી હોય છે.

\begin{center}
\textbf{Mermaid Diagram (Code)}
\begin{verbatim}
{Shaded}
{Highlighting}[]
graph LR
    A[બેકિંગ મટીરિયલ] {-{-}{-} B[પાતળા તારની ગ્રીડ]}
    B {-{-}{-} C[લીડ તાર]}
    C {-{-}{-} D[ઇલેક્ટ્રિકલ કનેક્શન]}
    style B fill:\#f9f,stroke:\#333,stroke{-width:2px}
{Highlighting}
{Shaded}
\end{verbatim}
\end{center}

\begin{itemize}
\tightlist
\item
  \textbf{બાંધકામના ઘટકો}: પાતળા રેસિસ્ટન્સ તારને ઇન્સ્યુલેટિંગ બેઝ મટીરિયલ પર
  આગળ-પાછળ લૂપ્સમાં ગોઠવેલ હોય છે
\item
  \textbf{કાર્યસિદ્ધાંત}: જ્યારે સ્ટ્રેઈન (તણાવ) લાગે ત્યારે પ્રતિરોધમાં ફેરફાર થાય છે
\item
  \textbf{એપ્લિકેશન}: વજન માપન, પ્રેશર સેન્સર, ફોર્સ સેન્સર, સ્ટ્રક્ચરલ હેલ્થ મોનિટરિંગ
\end{itemize}

\end{solutionbox}
\begin{mnemonicbox}
``WIRE Flexes: તાર ગ્રીડ બાહ્ય તણાવથી પ્રતિરોધ બદલાવ
દર્શાવે છે.''

\end{mnemonicbox}
\subsection*{પ્રશ્ન 1(ક) [7
ગુણ]}\label{uxaaauxab0uxab6uxaa8-1uxa95-7-uxa97uxaa3}

\textbf{સંતુલન સ્થિતિ માટે સર્કિટ ડાયાગ્રામ સાથે Schering બ્રિજનું કાર્ય સમજાવો.
તેના ફાયદા, ગેરફાયદા અને એપ્લિકેશનોની યાદી બનાવો.}

\begin{solutionbox}

Schering બ્રિજ એ AC બ્રિજ છે જે અજ્ઞાત કેપેસિટન્સ અને તેના ડિસિપેશન ફેક્ટરને માપવા માટે
વપરાય છે.

\begin{center}
\textbf{Mermaid Diagram (Code)}
\begin{verbatim}
{Shaded}
{Highlighting}[]
graph LR
    A[R1] {-{-}{-} B[R2]}
    B {-{-}{-} C[C2]}
    C {-{-}{-} D[Cx]}
    D {-{-}{-} A}
    E[AC સ્ત્રોત] {-{-}{-} A}
    E {-{-}{-} C}
    F[ડિટેક્ટર] {-{-}{-} B}
    F {-{-}{-} D}
    style Cx fill:\#f9f,stroke:\#333,stroke{-width:2px}
{Highlighting}
{Shaded}
\end{verbatim}
\end{center}

\textbf{સંતુલન શરત:}

{\def\LTcaptype{none} % do not increment counter
\begin{longtable}[]{@{}ll@{}}
\toprule\noalign{}
સમીકરણ & વર્ણન \\
\midrule\noalign{}
\endhead
\bottomrule\noalign{}
\endlastfoot
Cx = C2(R2/R1) & કેપેસિટન્સ ગણતરી માટે \\
Dx = R2(C2/Cx) & ડિસિપેશન ફેક્ટર માટે \\
\end{longtable}
}

\textbf{ફાયદા:}

\begin{itemize}
\tightlist
\item
  ઉચ્ચ ચોકસાઈ
\item
  કેપેસિટન્સનું સીધું રીડિંગ
\item
  વ્યાપક માપન શ્રેણી
\end{itemize}

\textbf{ગેરફાયદા:}

\begin{itemize}
\tightlist
\item
  સાવચેત શીલ્ડિંગની જરૂર પડે છે
\item
  આવૃત્તિ આધારિત ભૂલો
\item
  સંતુલન સાધવામાં જટિલ
\end{itemize}

\textbf{એપ્લિકેશન:}

\begin{itemize}
\tightlist
\item
  કેપેસિટર ટેસ્ટિંગ
\item
  ઇન્સ્યુલેશન ટેસ્ટિંગ
\item
  ડાઇલેક્ટ્રિક મટીરિયલ મૂલ્યાંકન
\end{itemize}

\end{solutionbox}
\begin{mnemonicbox}
``SCUBA ડાઇવ: Schering અજ્ઞાત કેપેસિટન્સને એડવાન્સ્ડ સર્કિટ
ડિઝાઇન દ્વારા વિવિધ ઉપકરણોમાં ગણે છે.''

\end{mnemonicbox}
\subsection*{પ્રશ્ન 1(ક OR) [7
ગુણ]}\label{uxaaauxab0uxab6uxaa8-1uxa95-or-7-uxa97uxaa3}

\textbf{સંતુલન સ્થિતિ માટે સર્કિટ ડાયાગ્રામ સાથે Maxwell's બ્રિજનું કાર્ય સમજાવો.
તેના ફાયદા, ગેરફાયદા અને એપ્લિકેશનોની યાદી બનાવો.}

\begin{solutionbox}

Maxwell's બ્રિજ અજ્ઞાત ઇન્ડક્ટન્સને જાણીતા કેપેસિટન્સના સંદર્ભમાં માપવા માટે વપરાય છે.

\begin{center}
\textbf{Mermaid Diagram (Code)}
\begin{verbatim}
{Shaded}
{Highlighting}[]
graph LR
    A[R1] {-{-}{-} B[R2]}
    B {-{-}{-} C[R3]}
    C {-{-}{-} D[Lx + Rx]}
    D {-{-}{-} A}
    E[AC સ્ત્રોત] {-{-}{-} A}
    E {-{-}{-} C}
    F[ડિટેક્ટર] {-{-}{-} B}
    F {-{-}{-} D}
    G[C4] {-{-}{-} B}
    G {-{-}{-} C}
    style G fill:\#f9f,stroke:\#333,stroke{-width:2px}
    style D fill:\#bbf,stroke:\#333,stroke{-width:2px}
{Highlighting}
{Shaded}
\end{verbatim}
\end{center}

\textbf{સંતુલન શરત:}

{\def\LTcaptype{none} % do not increment counter
\begin{longtable}[]{@{}ll@{}}
\toprule\noalign{}
સમીકરણ & વર્ણન \\
\midrule\noalign{}
\endhead
\bottomrule\noalign{}
\endlastfoot
Lx = C4·R2·R3 & ઇન્ડક્ટન્સ ગણતરી માટે \\
Rx = R1·(R3/R2) & રેસિસ્ટન્સ ગણતરી માટે \\
\end{longtable}
}

\textbf{ફાયદા:}

\begin{itemize}
\tightlist
\item
  આવૃત્તિથી સ્વતંત્ર
\item
  મધ્યમ Q કોઈલ્સ માટે ઉચ્ચ ચોકસાઈ
\item
  સંતુલન સાધવામાં સરળ
\end{itemize}

\textbf{ગેરફાયદા:}

\begin{itemize}
\tightlist
\item
  ઓછા Q કોઈલ્સ માટે યોગ્ય નથી
\item
  સ્ટાન્ડર્ડ કેપેસિટરની જરૂર પડે છે
\item
  મર્યાદિત રેન્જ
\end{itemize}

\textbf{એપ્લિકેશન:}

\begin{itemize}
\tightlist
\item
  ઇન્ડક્ટર્સનું માપન
\item
  ઓડિયો ફ્રિક્વન્સી માપન
\item
  ટ્રાન્સફોર્મર ટેસ્ટિંગ
\end{itemize}

\end{solutionbox}
\begin{mnemonicbox}
``MAGIC બ્રિજ: Maxwell મહાન ઇન્ડક્ટર્સનું બ્રિજ તત્વોની તુલના
દ્વારા વિશ્લેષણ કરે છે.''

\end{mnemonicbox}
\subsection*{પ્રશ્ન 2(અ) [3
ગુણ]}\label{uxaaauxab0uxab6uxaa8-2uxa85-3-uxa97uxaa3}

\textbf{જરૂરી ડાયાગ્રામ સાથે ઇલેક્ટ્રોનિક મલ્ટિમીટરની કામગીરી સમજાવો.}

\begin{solutionbox}

ઇલેક્ટ્રોનિક મલ્ટિમીટર વિવિધ ઇલેક્ટ્રિકલ પેરામીટર્સને સમપ્રમાણિત DC વોલ્ટેજમાં રૂપાંતરિત
કરે છે.

\begin{center}
\textbf{Mermaid Diagram (Code)}
\begin{verbatim}
{Shaded}
{Highlighting}[]
graph LR
    A[ઇનપુટ સિલેક્શન] {-{-}{} B[એટેન્યુએટર/રેન્જ સિલેક્ટર]}
    B {-{-}{} C[કન્વર્ટર સર્કિટ]}
    C {-{-}{} D[એમ્પ્લિફાયર]}
    D {-{-}{} E[ADC]}
    E {-{-}{} F[ડિસ્પ્લે]}
    style E fill:\#f9f,stroke:\#333,stroke{-width:2px}
{Highlighting}
{Shaded}
\end{verbatim}
\end{center}

\begin{itemize}
\tightlist
\item
  \textbf{સર્કિટ ઘટકો}: ઇનપુટ સિલેક્ટર \rightarrow એટેન્યુએટર \rightarrow કન્વર્ટર \rightarrow એમ્પ્લિફાયર \rightarrow ADC
  \rightarrow ડિસ્પ્લે
\item
  \textbf{માપન પ્રકારો}: DC વોલ્ટેજ, AC વોલ્ટેજ, કરંટ, રેસિસ્ટન્સ
\item
  \textbf{પાવર સ્ત્રોત}: પોર્ટેબિલિટી અને સુરક્ષા માટે બેટરી પાવર
\end{itemize}

\end{solutionbox}
\begin{mnemonicbox}
``SACRED ઉપકરણ: સિગ્નલ એટેન્યુએટ, કન્વર્ટ અને રેક્ટિફાઈ થઈને
ઇલેક્ટ્રોનિક ડિસ્પ્લે પર દર્શાવાય છે.''

\end{mnemonicbox}
\subsection*{પ્રશ્ન 2(બ) [4
ગુણ]}\label{uxaaauxab0uxab6uxaa8-2uxaac-4-uxa97uxaa3}

\textbf{એનાલોગ વોલ્ટમીટર અને ડિજિટલ વોલ્ટમીટર વચ્ચે તફાવત લખો.}

\begin{solutionbox}

{\def\LTcaptype{none} % do not increment counter
\begin{longtable}[]{@{}lll@{}}
\toprule\noalign{}
પેરામીટર & ડિજિટલ વોલ્ટમીટર & એનાલોગ વોલ્ટમીટર \\
\midrule\noalign{}
\endhead
\bottomrule\noalign{}
\endlastfoot
\textbf{ડિસ્પ્લે પ્રકાર} & ન્યુમેરિક LCD/LED ડિસ્પ્લે & સ્કેલ પર ફરતો પોઈન્ટર \\
\textbf{ચોકસાઈ} & ઉચ્ચ (\pm0.1\% સામાન્ય) & નિમ્ન (\pm2-5\% સામાન્ય) \\
\textbf{રીડિંગ ભૂલો} & પેરેલેક્સ ભૂલ નથી & પેરેલેક્સ ભૂલ થઈ શકે \\
\textbf{રિઝોલ્યૂશન} & ઉચ્ચ (3-6 અંકો દર્શાવી શકે) & સ્કેલ ડિવિઝન દ્વારા
મર્યાદિત \\
\textbf{ઇનપુટ ઇમ્પિડન્સ} & ખૂબ ઉચ્ચ (\textgreater10MΩ) & નિમ્ન (20-200kΩ/V) \\
\textbf{પ્રતિક્રિયા સમય} & ધીમો સેમ્પલિંગ રેટ & તાત્કાલિક પ્રતિસાદ \\
\end{longtable}
}

\end{solutionbox}
\begin{mnemonicbox}
``PARIOS: પેરેલેક્સ-ફ્રી, ચોકસાઈ ઉચ્ચ, રિઝોલ્યૂશન ઉચ્ચ,
ઇમ્પિડન્સ ઉચ્ચ, અવલોકન ડિજિટલ, સેમ્પલિંગ રેટ.''

\end{mnemonicbox}
\subsection*{પ્રશ્ન 2(ક) [7
ગુણ]}\label{uxaaauxab0uxab6uxaa8-2uxa95-7-uxa97uxaa3}

\textbf{એનર્જીમીટરના બાંધકામ ડાયાગ્રામનું વર્ણન લખો અને વિગતવાર સમજાવો.}

\begin{solutionbox}

એનર્જી મીટર સમય સાથે કિલોવોટ-અવર્સ (kWh) માં વીજળી ઊર્જાની ખપતને માપે છે.

\begin{center}
\textbf{Mermaid Diagram (Code)}
\begin{verbatim}
{Shaded}
{Highlighting}[]
graph LR
    A[વોલ્ટેજ કોઈલ] {-{-}{} B[કરંટ કોઈલ]}
    B {-{-}{} C[એલ્યુમિનિયમ ડિસ્ક]}
    C {-{-}{} D[મેકેનિકલ કાઉન્ટર]}
    E[પર્મેનેન્ટ મેગ્નેટ] {-{-}{} C}
    F[બ્રેકિંગ સિસ્ટમ] {-{-}{} C}
    G[લોડ ટર્મિનલ] {-{-}{} B}
    style C fill:\#f9f,stroke:\#333,stroke{-width:2px}
{Highlighting}
{Shaded}
\end{verbatim}
\end{center}

\textbf{ઘટકો:}

\begin{itemize}
\tightlist
\item
  \textbf{વોલ્ટેજ કોઈલ}: વોલ્ટેજના સમપ્રમાણમાં ફ્લક્સ ઉત્પન્ન કરે છે
\item
  \textbf{કરંટ કોઈલ}: કરંટના સમપ્રમાણમાં ફ્લક્સ ઉત્પન્ન કરે છે
\item
  \textbf{એલ્યુમિનિયમ ડિસ્ક}: એડી કરંટને કારણે ફરે છે
\item
  \textbf{કાઉન્ટિંગ મેકેનિઝમ}: ડિસ્કના ફરવાની ગણતરી કરે છે
\item
  \textbf{પર્મેનેન્ટ મેગ્નેટ}: ડિસ્કની ગતિ નિયંત્રિત કરવા બ્રેક તરીકે કાર્ય કરે છે
\item
  \textbf{એડજસ્ટમેન્ટ સિસ્ટમ}: કેલિબ્રેશન અને ચોકસાઈ માટે
\end{itemize}

\textbf{કાર્યસિદ્ધાંત}: ડિસ્કની ફરવાની ગતિ પાવર વપરાશ (V\timesI\timescosΦ) ના
સમપ્રમાણમાં હોય છે

\end{solutionbox}
\begin{mnemonicbox}
``VADCR મીટર: વોલ્ટેજ અને કરંટ ફરવા દ્વારા કાઉન્ટરને ચલાવે
છે.''

\end{mnemonicbox}
\subsection*{પ્રશ્ન 2(અ OR) [3
ગુણ]}\label{uxaaauxab0uxab6uxaa8-2uxa85-or-3-uxa97uxaa3}

\textbf{ક્લેમ્પ ઓન એમીટરનું કામ જરૂરી ડાયાગ્રામ સાથે સમજાવો.}

\begin{solutionbox}

ક્લેમ્પ-ઓન એમીટર ઇલેક્ટ્રોમેગ્નેટિક ઇન્ડક્શનનો ઉપયોગ કરીને સર્કિટને તોડ્યા વિના કરંટ માપે
છે.

\begin{verbatim}
            ╭─────────╮
            │ Display │
            ╰─────────╯
                 │
     ╭───────────────────╮
     │ Signal Processing │
     ╰───────────────────╯
                 │
          ╭────────────╮
          │  CT Core   │───┐
          ╰────────────╯   │
              │    │       │
              │    │       │
              │    │       │
          ╭───┴────┴───╮   │
          │ Clamp Jaws │   │
          ╰────────────╯   │
                 │         │
             Current       │
             Carrying      │
              Wire         │
                           │
                          ─┴─
                          GND
\end{verbatim}

\begin{itemize}
\tightlist
\item
  \textbf{બાંધકામ}: સેન્સિંગ કોઈલ સાથે સ્પ્લિટ ફેરાઈટ કોર
\item
  \textbf{કાર્યસિદ્ધાંત}: કરંટ-વાહક તાર ચુંબકીય ક્ષેત્ર ઉત્પન્ન કરે છે \rightarrow સેન્સિંગ કોઈલમાં
  વોલ્ટેજ પ્રેરિત કરે છે
\item
  \textbf{ફાયદા}: નોન-કોન્ટેક્ટ માપન, ઝડપી, સુરક્ષિત
\end{itemize}

\end{solutionbox}
\begin{mnemonicbox}
``CICS: ક્લેમ્પિંગ દ્વારા કરંટ સિગ્નલ પ્રેરિત થાય છે.''

\end{mnemonicbox}
\subsection*{પ્રશ્ન 2(બ OR) [4
ગુણ]}\label{uxaaauxab0uxab6uxaa8-2uxaac-or-4-uxa97uxaa3}

\textbf{PMMC પ્રકાર મીટર અને મૂવિંગ આયર્ન પ્રકાર મીટર વચ્ચે તફાવત લખો.}

\begin{solutionbox}

{\def\LTcaptype{none} % do not increment counter
\begin{longtable}[]{@{}lll@{}}
\toprule\noalign{}
પેરામીટર & PMMC પ્રકાર મીટર & મૂવિંગ આયર્ન પ્રકાર મીટર \\
\midrule\noalign{}
\endhead
\bottomrule\noalign{}
\endlastfoot
\textbf{કાર્યસિદ્ધાંત} & ચુંબકીય ક્ષેત્ર આંતરક્રિયા & ચુંબકીય આકર્ષણ/વિકર્ષણ \\
\textbf{કરંટ પ્રકાર} & માત્ર DC & AC અને DC બંને \\
\textbf{સ્કેલ} & સમાન & અસમાન (છેડે ગીચ) \\
\textbf{ચોકસાઈ} & ઉચ્ચ (\pm0.5\% સામાન્ય) & નિમ્ન (\pm1-5\% સામાન્ય) \\
\textbf{ડેમ્પિંગ} & એડી કરંટ ડેમ્પિંગ & હવા ઘર્ષણ ડેમ્પિંગ \\
\textbf{પાવર વપરાશ} & ઓછો & વધારે \\
\textbf{આવૃત્તિ ભૂલો} & લાગુ પડતું નથી & આવૃત્તિ પરિવર્તનથી અસર પામે છે \\
\end{longtable}
}

\end{solutionbox}
\begin{mnemonicbox}
``PMMC એ DAUPHIN છે: ફક્ત DC, ચોકસાઈ ઉચ્ચ, સમાન સ્કેલ,
પાવર કાર્યક્ષમ, ઉચ્ચ સંવેદનશીલતા, આવૃત્તિથી સ્વતંત્ર, ધ્રુવીયતા જરૂરી.''

\end{mnemonicbox}
\subsection*{પ્રશ્ન 2(ક OR) [7
ગુણ]}\label{uxaaauxab0uxab6uxaa8-2uxa95-or-7-uxa97uxaa3}

\textbf{જરૂરી ડાયાગ્રામ અને વેવફોર્મ સાથે ઇન્ટિગ્રેટિંગ ટાઇપ DVM નું બ્લોક ડાયાગ્રામ
અને કામગીરી સમજાવો.}

\begin{solutionbox}

ઇન્ટિગ્રેટિંગ ટાઇપ DVM ઉચ્ચ ચોકસાઈના માપન માટે ઇનપુટ વોલ્ટેજને ઇન્ટિગ્રેશન દ્વારા સમય
સાથે રૂપાંતરિત કરે છે.

\begin{center}
\textbf{Mermaid Diagram (Code)}
\begin{verbatim}
{Shaded}
{Highlighting}[]
graph LR
    A[ઇનપુટ બફર] {-{-}{} B[ઇન્ટિગ્રેટર]}
    B {-{-}{} C[કમ્પેરેટર]}
    C {-{-}{} D[કંટ્રોલ લોજિક]}
    D {-{-}{} E[ક્લોક]}
    D {-{-}{} F[કાઉન્ટર]}
    F {-{-}{} G[ડિસ્પ્લે]}
    D {-{-}{}|રીસેટ| B}
    style B fill:\#f9f,stroke:\#333,stroke{-width:2px}
{Highlighting}
{Shaded}
\end{verbatim}
\end{center}

\textbf{કાર્યસિદ્ધાંત:}

\begin{itemize}
\tightlist
\item
  ઇનપુટ વોલ્ટેજને નિશ્ચિત સમય માટે ઇન્ટિગ્રેટ કરવામાં આવે છે
\item
  ઇન્ટિગ્રેટર આઉટપુટ ઇનપુટના પ્રમાણમાં ઉપર તરફ જાય છે
\item
  વિરુદ્ધ ધ્રુવીયતા સાથે રેફરન્સ વોલ્ટેજ ઇન્ટિગ્રેટરને ડિસ્ચાર્જ કરે છે
\item
  ડિસ્ચાર્જ માટે લાગતો સમય ક્લોક પલ્સ દ્વારા માપવામાં આવે છે
\item
  કાઉન્ટ ઇનપુટ વોલ્ટેજના પ્રમાણમાં હોય છે
\end{itemize}

\textbf{વેવફોર્મ:}

\begin{verbatim}
Input      ────────────────────────────────
                                           
Integrator      /{                /       }
output         /  {              /        }
              /    {            /         }
             /      {          /          }
            /        {        /           }
           /          {      /            }
          /            {    /             }

Control    ────┐      ┌─────────┐      ┌──
signals        │      │         │      │  
               └──────┘         └──────┘  

Clock      ┌┐┌┐┌┐┌┐┌┐┌┐┌┐┌┐┌┐┌┐┌┐┌┐┌┐┌┐┌┐┌┐
pulses     └┘└┘└┘└┘└┘└┘└┘└┘└┘└┘└┘└┘└┘└┘└┘└┘
\end{verbatim}

\textbf{ફાયદા:}

\begin{itemize}
\tightlist
\item
  ઉચ્ચ નોઈઝ રિજેક્શન
\item
  સારી ચોકસાઈ
\item
  ઉત્તમ રિઝોલ્યુશન
\item
  કોમન-મોડ નોઈઝને નકારે છે
\end{itemize}

\end{solutionbox}
\begin{mnemonicbox}
``DIRT મીટર: ડાયરેક્ટ ઇન્ટિગ્રેશન વોલ્ટેજ માપવા માટે સમયને
સંબંધિત કરે છે.''

\end{mnemonicbox}
\subsection*{પ્રશ્ન 3(અ) [3
ગુણ]}\label{uxaaauxab0uxab6uxaa8-3uxa85-3-uxa97uxaa3}

\textbf{CRO અને DSO વચ્ચે તફાવત લખો.}

\begin{solutionbox}

{\def\LTcaptype{none} % do not increment counter
\begin{longtable}[]{@{}
  >{\raggedright\arraybackslash}p{(\linewidth - 4\tabcolsep) * \real{0.1507}}
  >{\raggedright\arraybackslash}p{(\linewidth - 4\tabcolsep) * \real{0.3699}}
  >{\raggedright\arraybackslash}p{(\linewidth - 4\tabcolsep) * \real{0.4795}}@{}}
\toprule\noalign{}
\begin{minipage}[b]{\linewidth}\raggedright
પેરામીટર
\end{minipage} & \begin{minipage}[b]{\linewidth}\raggedright
CRO (એનાલોગ ઓસિલોસ્કોપ)
\end{minipage} & \begin{minipage}[b]{\linewidth}\raggedright
DSO (ડિજિટલ સ્ટોરેજ ઓસિલોસ્કોપ)
\end{minipage} \\
\midrule\noalign{}
\endhead
\bottomrule\noalign{}
\endlastfoot
\textbf{સિગ્નલ પ્રોસેસિંગ} & સંપૂર્ણ એનાલોગ & ADC રૂપાંતર પછી ડિજિટલ \\
\textbf{સ્ટોરેજ ક્ષમતા} & વેવફોર્મ સંગ્રહ કરી શકતું નથી & અનેક વેવફોર્મ સંગ્રહ કરી શકે
છે \\
\textbf{બેન્ડવિડ્થ} & સામાન્ય રીતે ઓછી & ઉચ્ચ (GHz સુધી) \\
\textbf{ટ્રિગરિંગ} & મૂળભૂત ટ્રિગર વિકલ્પો & અદ્યતન ટ્રિગર ક્ષમતાઓ \\
\textbf{એનાલિસિસ ફીચર્સ} & મર્યાદિત & વિસ્તૃત (FFT, માપન) \\
\textbf{ડિસ્પ્લે પર્સિસ્ટન્સ} & ફોસ્ફર પર્સિસ્ટન્સ & એડજસ્ટેબલ ડિજિટલ પર્સિસ્ટન્સ \\
\end{longtable}
}

\end{solutionbox}
\begin{mnemonicbox}
``PASSED: પ્રોસેસિંગ-એનાલોગ/ડિજિટલ, સ્ટોરેજ-નહીં/હા,
સિગ્નલ-કાચો/પ્રોસેસ્ડ, સરળ-મૂળભૂત/અદ્યતન, ડિસ્પ્લે-ફોસ્ફર/ડિજિટલ.''

\end{mnemonicbox}
\subsection*{પ્રશ્ન 3(બ) [4
ગુણ]}\label{uxaaauxab0uxab6uxaa8-3uxaac-4-uxa97uxaa3}

\textbf{CRO સ્ક્રીન સમજાવો.}

\begin{solutionbox}

CRO સ્ક્રીન ઇલેક્ટ્રિકલ સિગ્નલ્સને દર્શાવે છે અને તેમાં કેટલાક મહત્વપૂર્ણ ઘટકો હોય છે.

\begin{verbatim}
┌───────────────────────────────────────┐
│                                       │
│             PHOSPHOR SCREEN           │
│                                       │
│     │         │         │         │   │
│  ───┼─────────┼─────────┼─────────┼── │
│     │         │         │         │   │
│     │         │         │         │   │
│  ───┼─────────┼─────────┼─────────┼── │
│     │         │         │         │   │
│     │         │         │         │   │
│  ───┼─────────┼─────────┼─────────┼── │
│     │         │         │         │   │
│     │         │         │         │   │
│  ───┼─────────┼─────────┼─────────┼── │
│     │         │         │         │   │
│                                       │
└───────────────────────────────────────┘
\end{verbatim}

\textbf{ઘટકો:}

\begin{itemize}
\tightlist
\item
  \textbf{ફોસ્ફર કોટિંગ}: ઇલેક્ટ્રોન દ્વારા અથડાવા પર પ્રકાશ આપે છે
\item
  \textbf{ગ્રેટિક્યુલ}: માપન સંદર્ભ માટે ગ્રિડ લાઈન્સ
\item
  \textbf{સ્કેલ}: વોલ્ટેજ/સમય માટે કેલિબ્રેટેડ માર્કિંગ્સ
\item
  \textbf{સેન્ટર રેફરન્સ પોઈન્ટ}: (0,0) કોઓર્ડિનેટ
\item
  \textbf{ઇન્ટેન્સિટી કંટ્રોલ}: ડિસ્પ્લેની બ્રાઇટનેસ એડજસ્ટ કરે છે
\end{itemize}

\end{solutionbox}
\begin{mnemonicbox}
``PGSCR: ફોસ્ફર ઇલેક્ટ્રોન અથડાવાથી પ્રકાશે છે, પ્રતિનિધિત્વ
બનાવે છે.''

\end{mnemonicbox}
\subsection*{પ્રશ્ન 3(ક) [7
ગુણ]}\label{uxaaauxab0uxab6uxaa8-3uxa95-7-uxa97uxaa3}

\textbf{CRO નો બ્લોક ડાયાગ્રામ, કામગીરી અને ફાયદા જરૂરી ડાયાગ્રામ સાથે
સમજાવો.}

\begin{solutionbox}

CRO (કેથોડ રે ઓસિલોસ્કોપ) ઇલેક્ટ્રિકલ સિગ્નલને વેવફોર્મ તરીકે વિઝ્યુઅલાઈઝ કરે છે.

\begin{center}
\textbf{Mermaid Diagram (Code)}
\begin{verbatim}
{Shaded}
{Highlighting}[]
graph LR
    A[વર્ટિકલ ઇનપુટ] {-{-}{} B[વર્ટિકલ એટેન્યુએટર]}
    B {-{-}{} C[વર્ટિકલ એમ્પ્લિફાયર]}
    C {-{-}{} D[વર્ટિકલ ડિફ્લેક્શન પ્લેટ્સ]}
    E[ટ્રિગર સર્કિટ] {-{-}{} F[ટાઇમ બેઝ જનરેટર]}
    F {-{-}{} G[હોરિઝોન્ટલ એમ્પ્લિફાયર]}
    G {-{-}{} H[હોરિઝોન્ટલ ડિફ્લેક્શન પ્લેટ્સ]}
    I[પાવર સપ્લાય] {-{-}{} J[ઇલેક્ટ્રોન ગન]}
    J {-{-}{} K[CRT]}
    D {-{-}{} K}
    H {-{-}{} K}
    I {-{-}{} B}
    I {-{-}{} C}
    I {-{-}{} E}
    I {-{-}{} F}
    I {-{-}{} G}
    style K fill:\#f9f,stroke:\#333,stroke{-width:2px}
{Highlighting}
{Shaded}
\end{verbatim}
\end{center}

\textbf{કાર્યસિદ્ધાંત:}

\begin{itemize}
\tightlist
\item
  \textbf{ઇલેક્ટ્રોન ગન}: ઇલેક્ટ્રોન બીમ ઉત્પન્ન કરે છે
\item
  \textbf{વર્ટિકલ સિસ્ટમ}: ઇનપુટ સિગ્નલના પ્રમાણમાં Y-એક્સિસ ડિફ્લેક્શન નિયંત્રિત કરે
  છે
\item
  \textbf{હોરિઝોન્ટલ સિસ્ટમ}: સ્થિર દરે સ્ક્રીન પર બીમને સ્વીપ કરે છે
\item
  \textbf{ટ્રિગર સર્કિટ}: ઇનપુટ સિગ્નલ સાથે હોરિઝોન્ટલ સ્વીપ સિન્ક્રોનાઈઝ કરે છે
\item
  \textbf{CRT}: ફોસ્ફર સ્ક્રીન પર ઇલેક્ટ્રોન બીમની ગતિ દર્શાવે છે
\end{itemize}

\textbf{ફાયદા:}

\begin{itemize}
\tightlist
\item
  રીઅલ-ટાઇમ સિગ્નલ ડિસ્પ્લે
\item
  વિશાળ બેન્ડવિડ્થ
\item
  ઉચ્ચ ઇનપુટ ઇમ્પિડન્સ
\item
  બહુવિધ ટ્રિગરિંગ વિકલ્પો
\item
  એકાધિક સિગ્નલ એનાલિસિસ
\end{itemize}

\end{solutionbox}
\begin{mnemonicbox}
``EARTH વ્યૂ: ઇલેક્ટ્રોન બીમ એમ્પ્લિફિકેશન સમય-આધારિત
હોરિઝોન્ટલ વ્યૂને પ્રગટ કરે છે.''

\end{mnemonicbox}
\subsection*{પ્રશ્ન 3(અ OR) [3
ગુણ]}\label{uxaaauxab0uxab6uxaa8-3uxa85-or-3-uxa97uxaa3}

\textbf{(frequency) આવર્તન માપન અને ફેઝ એંગલ માપન માટે લિસાજસ પેટર્ન લાગુ કરો.}

\begin{solutionbox}

લિસાજસ પેટર્ન ત્યારે બને છે જ્યારે બે સાઇન વેવ્સ CROના X અને Y ઇનપુટ પર લાગુ કરવામાં આવે
છે.

{\def\LTcaptype{none} % do not increment counter
\begin{longtable}[]{@{}
  >{\raggedright\arraybackslash}p{(\linewidth - 4\tabcolsep) * \real{0.4000}}
  >{\raggedright\arraybackslash}p{(\linewidth - 4\tabcolsep) * \real{0.2571}}
  >{\raggedright\arraybackslash}p{(\linewidth - 4\tabcolsep) * \real{0.3429}}@{}}
\toprule\noalign{}
\begin{minipage}[b]{\linewidth}\raggedright
પેટર્ન પ્રકાર
\end{minipage} & \begin{minipage}[b]{\linewidth}\raggedright
ઉદાહરણ
\end{minipage} & \begin{minipage}[b]{\linewidth}\raggedright
માપન સૂત્ર
\end{minipage} \\
\midrule\noalign{}
\endhead
\bottomrule\noalign{}
\endlastfoot
\textbf{આવૃત્તિ માપન} &
\pandocbounded{\includegraphics[keepaspectratio,alt={લિસાજસ ફોર ફ્રિક્વન્સી}]{https://example.com}}
& fx/fy = ny/nx \\
\textbf{ફેઝ એંગલ માપન} &
\pandocbounded{\includegraphics[keepaspectratio,alt={લિસાજસ ફોર ફેઝ}]{https://example.com}}
& sin(φ) = A/B \\
\end{longtable}
}

\begin{verbatim}
    Frequency                 Phase
      B B                     B  
    A     A                A     A
    │     │                │  │  │
    └─────┘                └──┘──┘
      
    fx/fy = 2/1          sin(φ) = sin/sin
\end{verbatim}

\begin{itemize}
\tightlist
\item
  \textbf{આવૃત્તિ ગુણોત્તર}: ઊભા ટેન્જન્ટ પોઈન્ટ્સ / આડા ટેન્જન્ટ પોઈન્ટ્સની ગણતરી
\item
  \textbf{ફેઝ માપન}: sin(φ) = sin/sinmax જ્યાં sin એ ઝીરો ક્રોસિંગ પર પેટર્નની
  ઊંચાઈ છે
\item
  \textbf{એપ્લિકેશન}: સિગ્નલ તુલના, આવૃત્તિ કેલિબ્રેશન
\end{itemize}

\end{solutionbox}
\begin{mnemonicbox}
``LIPS પેટર્ન: લિસાજસ ફેઝ અને સાઇન આવૃત્તિ દર્શાવે છે.''

\end{mnemonicbox}
\subsection*{પ્રશ્ન 3(બ OR) [4
ગુણ]}\label{uxaaauxab0uxab6uxaa8-3uxaac-or-4-uxa97uxaa3}

\textbf{CRO માં Graticules સમજાવો. તેના પ્રકારો પણ સમજાવો.}

\begin{solutionbox}

ગ્રેટિક્યુલ્સ એ CRO સ્ક્રીન પર રેફરન્સ ગ્રિડ છે જે વેવફોર્મ પેરામીટર્સના માપનમાં મદદ કરે
છે.

\begin{verbatim}
┌───────────────────────────────────┐
│                                   │
│  │    │    │    │    │    │    │  │
│──┼────┼────┼────┼────┼────┼────┼──│
│  │    │    │    │    │    │    │  │
│  │    │    │    │    │    │    │  │
│──┼────┼────┼────┼────┼────┼────┼──│
│  │    │    │    │    │    │    │  │
│  │    │    │    │    │    │    │  │
│──┼────┼────┼────┼────┼────┼────┼──│
│  │    │    │    │    │    │    │  │
│                                   │
└───────────────────────────────────┘
\end{verbatim}

\textbf{ગ્રેટિક્યુલ્સના પ્રકારો:}

{\def\LTcaptype{none} % do not increment counter
\begin{longtable}[]{@{}
  >{\raggedright\arraybackslash}p{(\linewidth - 4\tabcolsep) * \real{0.2692}}
  >{\raggedright\arraybackslash}p{(\linewidth - 4\tabcolsep) * \real{0.2692}}
  >{\raggedright\arraybackslash}p{(\linewidth - 4\tabcolsep) * \real{0.4615}}@{}}
\toprule\noalign{}
\begin{minipage}[b]{\linewidth}\raggedright
પ્રકાર
\end{minipage} & \begin{minipage}[b]{\linewidth}\raggedright
વર્ણન
\end{minipage} & \begin{minipage}[b]{\linewidth}\raggedright
એપ્લિકેશન
\end{minipage} \\
\midrule\noalign{}
\endhead
\bottomrule\noalign{}
\endlastfoot
\textbf{આંતરિક ગ્રેટિક્યુલ} & CRTની અંદર ખોદાયેલ & પેરેલેક્સ ભૂલને દૂર કરે છે \\
\textbf{બાહ્ય ગ્રેટિક્યુલ} & અલગ પારદર્શક પ્લેટ & સરળતાથી બદલી શકાય છે \\
\textbf{ઇલેક્ટ્રોનિક ગ્રેટિક્યુલ} & ઇલેક્ટ્રોનિક રીતે ઉત્પન્ન થાય છે & ડિજિટલ
ઓસિલોસ્કોપ્સ \\
\textbf{સ્પેશિયલ પરપઝ} & ચોક્કસ માપન માટે કસ્ટમ માર્કિંગ્સ & વિશિષ્ટ પરીક્ષણ \\
\end{longtable}
}

\end{solutionbox}
\begin{mnemonicbox}
``GRIT: ગ્રેટિક્યુલ્સ સમય-વોલ્ટેજ માપન માટે મહત્વપૂર્ણ રેન્ડરિંગ
કરે છે.''

\end{mnemonicbox}
\subsection*{પ્રશ્ન 3(ક OR) [7
ગુણ]}\label{uxaaauxab0uxab6uxaa8-3uxa95-or-7-uxa97uxaa3}

\textbf{ડિજિટલ સ્ટોરેજ ઓસિલોસ્કોપ (DSO)નો બ્લોક ડાયાગ્રામ, કામગીરી અને ફાયદા
જરૂરી ડાયાગ્રામ સાથે સમજાવો.}

\begin{solutionbox}

ડિજિટલ સ્ટોરેજ ઓસિલોસ્કોપ (DSO) સિગ્નલને સ્ટોરેજ, પ્રોસેસિંગ અને ડિસ્પ્લે માટે ડિજિટાઇઝ
કરે છે.

\begin{center}
\textbf{Mermaid Diagram (Code)}
\begin{verbatim}
{Shaded}
{Highlighting}[]
graph LR
    A[ઇનપુટ સિગ્નલ] {-{-}{} B[એટેન્યુએટર/એમ્પ્લિફાયર]}
    B {-{-}{} C[ADC]}
    C {-{-}{} D[મેમરી]}
    D {-{-}{} E[પ્રોસેસર]}
    E {-{-}{} F[DAC]}
    F {-{-}{} G[ડિસ્પ્લે]}
    H[ટાઇમ બેઝ] {-{-}{} E}
    I[ટ્રિગર સિસ્ટમ] {-{-}{} E}
    J[કંટ્રોલ પેનલ] {-{-}{} E}
    style D fill:\#f9f,stroke:\#333,stroke{-width:2px}
{Highlighting}
{Shaded}
\end{verbatim}
\end{center}

\textbf{કાર્યસિદ્ધાંત:}

\begin{itemize}
\tightlist
\item
  \textbf{એક્વિઝિશન}: ADC દ્વારા ઉચ્ચ દરે સિગ્નલ સેમ્પલ કરવામાં આવે છે
\item
  \textbf{સ્ટોરેજ}: ડિજિટલ વેલ્યૂ મેમરીમાં સંગ્રહિત કરવામાં આવે છે
\item
  \textbf{પ્રોસેસિંગ}: ડિજિટલ સિગ્નલ પ્રોસેસિંગ એનાલિસિસને વધારે છે
\item
  \textbf{ડિસ્પ્લે}: પુનર્નિર્મિત સિગ્નલ સ્ક્રીન પર દર્શાવવામાં આવે છે
\item
  \textbf{ટ્રિગરિંગ}: અદ્યતન ડિજિટલ ટ્રિગરિંગ વિકલ્પો
\end{itemize}

\textbf{ફાયદા:}

\begin{itemize}
\tightlist
\item
  સિગ્નલ સંગ્રહ ક્ષમતા
\item
  પ્રી-ટ્રિગર વ્યૂઇંગ
\item
  વન-શોટ સિગ્નલ કેપ્ચર
\item
  અદ્યતન માપન
\item
  લાંબા કેપ્ચર માટે ડીપ મેમરી
\item
  ડિજિટલ ફિલ્ટરિંગ અને એનાલિસિસ
\item
  નેટવર્ક કનેક્ટિવિટી
\end{itemize}

\end{solutionbox}
\begin{mnemonicbox}
``SAMPLE: સ્ટોરેજ અને મેમરી લાંબા સમયના ઇવેન્ટ્સને સાચવે છે.''

\end{mnemonicbox}
\subsection*{પ્રશ્ન 4(અ) [3
ગુણ]}\label{uxaaauxab0uxab6uxaa8-4uxa85-3-uxa97uxaa3}

\textbf{RTD અને થર્મિસ્ટરનો તફાવત લખો.}

\begin{solutionbox}

{\def\LTcaptype{none} % do not increment counter
\begin{longtable}[]{@{}
  >{\raggedright\arraybackslash}p{(\linewidth - 4\tabcolsep) * \real{0.1803}}
  >{\raggedright\arraybackslash}p{(\linewidth - 4\tabcolsep) * \real{0.6230}}
  >{\raggedright\arraybackslash}p{(\linewidth - 4\tabcolsep) * \real{0.1967}}@{}}
\toprule\noalign{}
\begin{minipage}[b]{\linewidth}\raggedright
પેરામીટર
\end{minipage} & \begin{minipage}[b]{\linewidth}\raggedright
RTD (રેસિસ્ટન્સ ટેમ્પરેચર ડિટેક્ટર)
\end{minipage} & \begin{minipage}[b]{\linewidth}\raggedright
થર્મિસ્ટર
\end{minipage} \\
\midrule\noalign{}
\endhead
\bottomrule\noalign{}
\endlastfoot
\textbf{મટીરિયલ} & પ્લેટિનમ, નિકલ, કોપર & મેટલ ઓક્સાઇડ્સ, સેમિકન્ડક્ટર્સ \\
\textbf{રેસિસ્ટન્સ-તાપમાન સંબંધ} & રેખીય, પોઝિટિવ કોએફિશિયન્ટ & નોન-લિનિયર,
સામાન્ય રીતે નેગેટિવ કોએફિશિયન્ટ \\
\textbf{તાપમાન શ્રેણી} & -200^\circC થી +850^\circC & -50^\circC થી +300^\circC \\
\textbf{સંવેદનશીલતા} & ઓછી (0.00385 Ω/Ω/^\circC સામાન્ય) & ઉચ્ચ (3-5\% પ્રતિ ^\circC
સામાન્ય) \\
\textbf{ચોકસાઈ} & ઉચ્ચ & નિમ્ન \\
\textbf{પ્રતિક્રિયા સમય} & ધીમો & ઝડપી \\
\end{longtable}
}

\end{solutionbox}
\begin{mnemonicbox}
``RTD એ PLAINS છે: પ્લેટિનમ, લિનિયર, ચોક્કસ, ઔદ્યોગિક
શ્રેણી, સાંકડી સંવેદનશીલતા, સ્થિર.''

\end{mnemonicbox}
\subsection*{પ્રશ્ન 4(બ) [4
ગુણ]}\label{uxaaauxab0uxab6uxaa8-4uxaac-4-uxa97uxaa3}

\textbf{ઓપ્ટિકલ એનકોડરનું તેના આઉટપુટ વેવફોર્મ સાથે સમજાવો.}

\begin{solutionbox}

ઓપ્ટિકલ એનકોડર, પ્રકાશનું કોડેડ ડિસ્ક મારફતે અવરોધન થવાથી યાંત્રિક ગતિને ડિજિટલ
પલ્સમાં રૂપાંતરિત કરે છે.

\begin{verbatim}
    ┌─────────────┐
    │  Light      │
    │  Source     │
    └─────┬───────┘
          │
          v
    ┌─────────────┐
    │  Code       │
    │  Disc       │◄────Motion
    └─────┬───────┘
          │
          v
    ┌─────────────┐
    │  Photo{-     │}
    │  detector   │
    └─────┬───────┘
          │
          v
     Output Signal
\end{verbatim}

\textbf{આઉટપુટ વેવફોર્મ:}

\begin{verbatim}
Channel A: ┌──┐  ┌──┐  ┌──┐  ┌──┐
           │  │  │  │  │  │  │  │
           └──┘  └──┘  └──┘  └──┘

Channel B: ┌──┐  ┌──┐  ┌──┐  ┌──┐
         ┌─┘  └──┘  └──┘  └──┘  └─
         │
90^ phase│
   shift └─────────────────────────
\end{verbatim}

\begin{itemize}
\tightlist
\item
  \textbf{ઘટકો}: પ્રકાશ સ્ત્રોત, કોડેડ ડિસ્ક, ફોટોડિટેક્ટર
\item
  \textbf{પ્રકારો}: ઇન્ક્રિમેન્ટલ (પલ્સ) અથવા એબ્સોલ્યુટ (યુનિક પોઝિશન કોડ)
\item
  \textbf{એપ્લિકેશન}: પોઝિશન માપન, સ્પીડ ડિટેક્શન, મોશન કંટ્રોલ
\end{itemize}

\end{solutionbox}
\begin{mnemonicbox}
``DROPS: ડિસ્ક રોટેશન પલ્સ સિગ્નલ આઉટપુટ કરે છે.''

\end{mnemonicbox}
\subsection*{પ્રશ્ન 4(ક) [7
ગુણ]}\label{uxaaauxab0uxab6uxaa8-4uxa95-7-uxa97uxaa3}

\textbf{થર્મોકપલનું કાર્યકારી સિદ્ધાંત, પ્રકારો અને એપ્લિકેશન સાથે વર્ણન કરો.}

\begin{solutionbox}

થર્મોકપલ એ તાપમાન સેન્સર છે જે સીબેક ઇફેક્ટ પર કાર્ય કરે છે અને તાપમાનના તફાવતના
પ્રમાણમાં વોલ્ટેજ ઉત્પન્ન કરે છે.

\begin{center}
\textbf{Mermaid Diagram (Code)}
\begin{verbatim}
{Shaded}
{Highlighting}[]
graph LR
    A[હોટ જંક્શન] {-{-}{-} B[મેટલ A]}
    A {-{-}{-} C[મેટલ B]}
    B {-{-}{-} D[કોલ્ડ જંક્શન]}
    C {-{-}{-} D}
    D {-{-}{-} E[માપન ઉપકરણ]}
    style A fill:\#f9f,stroke:\#333,stroke{-width:2px}
{Highlighting}
{Shaded}
\end{verbatim}
\end{center}

\textbf{કાર્યસિદ્ધાંત:}

\begin{itemize}
\tightlist
\item
  બે અલગ-અલગ મેટલ એક છેડે (હોટ જંક્શન) જોડાયેલા હોય છે
\item
  હોટ અને કોલ્ડ જંક્શન વચ્ચેના તાપમાનના તફાવતથી વોલ્ટેજ ઉત્પન્ન થાય છે
\item
  વોલ્ટેજ તાપમાન તફાવતના પ્રમાણમાં હોય છે
\end{itemize}

\textbf{થર્મોકપલના પ્રકારો:}

{\def\LTcaptype{none} % do not increment counter
\begin{longtable}[]{@{}
  >{\raggedright\arraybackslash}p{(\linewidth - 6\tabcolsep) * \real{0.1224}}
  >{\raggedright\arraybackslash}p{(\linewidth - 6\tabcolsep) * \real{0.2245}}
  >{\raggedright\arraybackslash}p{(\linewidth - 6\tabcolsep) * \real{0.3878}}
  >{\raggedright\arraybackslash}p{(\linewidth - 6\tabcolsep) * \real{0.2653}}@{}}
\toprule\noalign{}
\begin{minipage}[b]{\linewidth}\raggedright
પ્રકાર
\end{minipage} & \begin{minipage}[b]{\linewidth}\raggedright
મટીરિયલ
\end{minipage} & \begin{minipage}[b]{\linewidth}\raggedright
તાપમાન શ્રેણી
\end{minipage} & \begin{minipage}[b]{\linewidth}\raggedright
એપ્લિકેશન
\end{minipage} \\
\midrule\noalign{}
\endhead
\bottomrule\noalign{}
\endlastfoot
\textbf{ટાઇપ K} & ક્રોમેલ-એલુમેલ & -200^\circC થી +1350^\circC & જનરલ પરપઝ, ઓક્સિડાઇઝિંગ
એટમોસ્ફિયર \\
\textbf{ટાઇપ J} & આયર્ન-કોન્સ્ટન્ટન & -40^\circC થી +750^\circC & રિડ્યુસિંગ એટમોસ્ફિયર,
વેક્યુમ \\
\textbf{ટાઇપ E} & ક્રોમેલ-કોન્સ્ટન્ટન & -200^\circC થી +900^\circC & ક્રાયોજેનિક, ઉચ્ચ
આઉટપુટ \\
\textbf{ટાઇપ T} & કોપર-કોન્સ્ટન્ટન & -250^\circC થી +350^\circC & લો ટેમ્પરેચર, ફૂડ
ઇન્ડસ્ટ્રી \\
\textbf{ટાઇપ R/S} & પ્લેટિનમ-રોડિયમ & 0^\circC થી +1700^\circC & હાઇ ટેમ્પરેચર,
લેબોરેટરી \\
\end{longtable}
}

\textbf{એપ્લિકેશન:} ઔદ્યોગિક ફર્નેસ, એન્જિન, કેમિકલ પ્રોસેસિંગ, ફૂડ પ્રોસેસિંગ, રિસર્ચ

\end{solutionbox}
\begin{mnemonicbox}
``SHOVE સિદ્ધાંત: સીબેક હોટ-કોલ્ડ આઉટપુટ વોલ્ટેજ તાપમાનની
બરાબર.''

\end{mnemonicbox}
\subsection*{પ્રશ્ન 4(અ OR) [3
ગુણ]}\label{uxaaauxab0uxab6uxaa8-4uxa85-or-3-uxa97uxaa3}

\textbf{એક્ટીવ અને પેસિવ ટ્રાન્સડ્યુસરનો તફાવત લખો.}

\begin{solutionbox}

{\def\LTcaptype{none} % do not increment counter
\begin{longtable}[]{@{}
  >{\raggedright\arraybackslash}p{(\linewidth - 4\tabcolsep) * \real{0.2157}}
  >{\raggedright\arraybackslash}p{(\linewidth - 4\tabcolsep) * \real{0.3725}}
  >{\raggedright\arraybackslash}p{(\linewidth - 4\tabcolsep) * \real{0.4118}}@{}}
\toprule\noalign{}
\begin{minipage}[b]{\linewidth}\raggedright
પેરામીટર
\end{minipage} & \begin{minipage}[b]{\linewidth}\raggedright
એક્ટીવ ટ્રાન્સડ્યુસર
\end{minipage} & \begin{minipage}[b]{\linewidth}\raggedright
પેસિવ ટ્રાન્સડ્યુસર
\end{minipage} \\
\midrule\noalign{}
\endhead
\bottomrule\noalign{}
\endlastfoot
\textbf{ઊર્જા રૂપાંતરણ} & ભૌતિક જથ્થાને સીધા જ ઇલેક્ટ્રિકલ આઉટપુટમાં રૂપાંતરિત કરે છે
& બાહ્ય ઊર્જા સ્ત્રોતની જરૂર પડે છે \\
\textbf{આઉટપુટ સિગ્નલ} & સેલ્ફ-જનરેટિંગ & બાહ્ય ઊર્જાને મોડ્યુલેટ કરે છે \\
\textbf{ઉદાહરણો} & થર્મોકપલ, પિઝોઇલેક્ટ્રિક, ફોટોવોલ્ટેઇક & RTD, સ્ટ્રેઇન ગેજ,
LVDT \\
\textbf{સંવેદનશીલતા} & સામાન્ય રીતે ઓછી & સામાન્ય રીતે ઉચ્ચ \\
\textbf{સર્કિટ જટિલતા} & સરળ & વધુ જટિલ \\
\textbf{પાવર જરૂરિયાત} & બાહ્ય પાવરની જરૂર નથી & બાહ્ય પાવર જરૂરી \\
\end{longtable}
}

\end{solutionbox}
\begin{mnemonicbox}
``SIMPLE તફાવત: સેલ્ફ-પાવર્ડ આગળ પડતા ઊર્જા ટ્રાન્સડ્યુસરનો
મુખ્ય સિદ્ધાંત છે.''

\end{mnemonicbox}
\subsection*{પ્રશ્ન 4(બ OR) [4
ગુણ]}\label{uxaaauxab0uxab6uxaa8-4uxaac-or-4-uxa97uxaa3}

\textbf{કેપેસિટીવ ટ્રાન્સડ્યુસરને જરૂરી ડાયાગ્રામ સાથે વિગતવાર સમજાવો. તેની
એપ્લિકેશનની યાદી બનાવો.}

\begin{solutionbox}

કેપેસિટીવ ટ્રાન્સડ્યુસર ભૌતિક ડિસ્પ્લેસમેન્ટને કારણે કેપેસિટન્સમાં થતા ફેરફારના સિદ્ધાંત પર
કાર્ય કરે છે.

\begin{center}
\textbf{Mermaid Diagram (Code)}
\begin{verbatim}
{Shaded}
{Highlighting}[]
graph LR
    A[ફિક્સ્ડ પ્લેટ] {-{-}{-} B[ડાઇલેક્ટ્રિક]}
    B {-{-}{-} C[મૂવેબલ પ્લેટ]}
    C {-{-}{-} D[ભૌતિક પેરામીટર]}
    E[કેપેસિટન્સ માપન સર્કિટ] {-{-}{-} A}
    E {-{-}{-} C}
    style B fill:\#f9f,stroke:\#333,stroke{-width:2px}
{Highlighting}
{Shaded}
\end{verbatim}
\end{center}

\textbf{કાર્યસિદ્ધાંત:}

\begin{itemize}
\tightlist
\item
  કેપેસિટન્સ C = ε_{0}εᵣA/d
\item
  આમાં ફેરફાર થાય છે: ક્ષેત્રફળ (A), અંતર (d), અથવા ડાઇલેક્ટ્રિક સ્થિરાંક (εᵣ) માં
  ફેરફારથી
\item
  ડિસ્પ્લેસમેન્ટ કેપેસિટન્સને બદલે છે
\item
  બ્રિજ સર્કિટ અથવા ઓસિલેટર દ્વારા માપવામાં આવે છે
\end{itemize}

\textbf{એપ્લિકેશન:}

\begin{itemize}
\tightlist
\item
  પ્રેશર માપન
\item
  લિક્વિડ લેવલ સેન્સિંગ
\item
  હ્યુમિડિટી સેન્સર
\item
  ડિસ્પ્લેસમેન્ટ માપન
\item
  એક્સેલેરોમીટર
\end{itemize}

\end{solutionbox}
\begin{mnemonicbox}
``CADAP: કેપેસિટન્સ અંતર, ક્ષેત્રફળ, અથવા પર્મિટિવિટી સાથે
બદલાય છે.''

\end{mnemonicbox}
\subsection*{પ્રશ્ન 4(ક OR) [7
ગુણ]}\label{uxaaauxab0uxab6uxaa8-4uxa95-or-7-uxa97uxaa3}

\textbf{LVDT ટ્રાન્સડ્યુસર ઓપરેશન, બાંધકામને જરૂરી આકૃતિ સાથે વિગતવાર સમજાવો.
એલવીડીટીના લાભ, ગેરલાભ અને એપ્લિકેશનની પણ યાદી બનાવો.}

\begin{solutionbox}

LVDT (લિનિયર વેરિએબલ ડિફરેન્શિયલ ટ્રાન્સફોર્મર) એ ઇલેક્ટ્રોમેકેનિકલ ટ્રાન્સડ્યુસર છે જે
લિનિયર ડિસ્પ્લેસમેન્ટને ઇલેક્ટ્રિકલ આઉટપુટમાં રૂપાંતરિત કરે છે.

\begin{center}
\textbf{Mermaid Diagram (Code)}
\begin{verbatim}
{Shaded}
{Highlighting}[]
graph LR
    A[પ્રાઇમરી કોઈલ] {-{-}{-} B[સેકન્ડરી કોઈલ 1]}
    A {-{-}{-} C[સેકન્ડરી કોઈલ 2]}
    D[AC એક્સાઇટેશન] {-{-}{-} A}
    E[ફેરોમેગ્નેટિક કોર] {-{-}{-} F[કોર રોડ]}
    B {-{-}{-} G[આઉટપુટ વોલ્ટેજ]}
    C {-{-}{-} G}
    style E fill:\#f9f,stroke:\#333,stroke{-width:2px}
{Highlighting}
{Shaded}
\end{verbatim}
\end{center}

\textbf{બાંધકામ:}

\begin{itemize}
\tightlist
\item
  મધ્યમાં પ્રાઇમરી કોઈલ
\item
  સમમિત રીતે વીંટળાયેલી બે સેકન્ડરી કોઈલ
\item
  હલનચલન કરી શકે તેવો ફેરોમેગ્નેટિક કોર
\item
  સિગ્નલ કન્ડિશનિંગ સર્કિટરી
\end{itemize}

\textbf{ઓપરેશન:}

\begin{itemize}
\tightlist
\item
  AC એક્સાઇટેશન પ્રાઇમરી કોઈલને ઊર્જાવાન કરે છે
\item
  કોરની સ્થિતિ સેકન્ડરીમાં મેગ્નેટિક કપલિંગ નક્કી કરે છે
\item
  ડિસ્પ્લેસમેન્ટના પ્રમાણમાં ડિફરેન્શિયલ વોલ્ટેજ આઉટપુટ મળે છે
\item
  ફેઝ ડિસ્પ્લેસમેન્ટની દિશા દર્શાવે છે
\end{itemize}

\textbf{લાભ:}

\begin{itemize}
\tightlist
\item
  નોન-કોન્ટેક્ટ ઓપરેશન
\item
  અનંત રિઝોલ્યૂશન
\item
  ઉચ્ચ લિનિયરિટી
\item
  મજબૂત બાંધકામ
\item
  લાંબું ઓપરેશનલ જીવન
\item
  ખરાબ પરિસ્થિતિમાં પણ ઇમ્યુનિટી
\end{itemize}

\textbf{ગેરલાભ:}

\begin{itemize}
\tightlist
\item
  AC એક્સાઇટેશનની જરૂર પડે છે
\item
  અન્ય સેન્સર્સની તુલનામાં મોટું
\item
  બાહ્ય ચુંબકીય ક્ષેત્રોથી અસર પામે છે
\item
  મર્યાદિત ડાયનેમિક પ્રતિસાદ
\end{itemize}

\textbf{એપ્લિકેશન:}

\begin{itemize}
\tightlist
\item
  પ્રિસિઝન માપન
\item
  હાઇડ્રોલિક સિસ્ટમ
\item
  એરક્રાફ્ટ કંટ્રોલ
\item
  પાવર પ્લાન્ટ કંટ્રોલ
\item
  ઓટોમેટેડ મેન્યુફેક્ચરિંગ
\end{itemize}

\end{solutionbox}
\begin{mnemonicbox}
``CDPOS સેન્સર: કોર ડિસ્પ્લેસમેન્ટ આઉટપુટ સિગ્નલ ઉત્પન્ન કરે
છે.''

\end{mnemonicbox}
\subsection*{પ્રશ્ન 5(અ) [3
ગુણ]}\label{uxaaauxab0uxab6uxaa8-5uxa85-3-uxa97uxaa3}

\textbf{સેમિકન્ડક્ટર ટેમ્પરેચર સેન્સર LM35નો સિદ્ધાંત અને કાર્ય દર્શાવો.}

\begin{solutionbox}

LM35 એક ઇન્ટિગ્રેટેડ સર્કિટ ટેમ્પરેચર સેન્સર છે જે સેલ્સિયસમાં તાપમાનના પ્રમાણમાં રેખીય
વોલ્ટેજ આઉટપુટ આપે છે.

\begin{verbatim}
     ┌───┬───┬───┐
     │   │   │   │
     │ 1 │ 2 │ 3 │
     │   │   │   │
     └───┴───┴───┘
       │   │   │
       │   │   │
       │   │   └── GND
       │   └────── આઉટપુટ (10mV/^)
       └────────── VCC (+4V થી +30V)
\end{verbatim}

\textbf{કાર્યસિદ્ધાંત:}

\begin{itemize}
\tightlist
\item
  બિલ્ટ-ઇન તાપમાન-સેન્સિંગ એલિમેન્ટ સાથે ઇન્ટિગ્રેટેડ સર્કિટ
\item
  લિનિયર આઉટપુટ વોલ્ટેજ: +10mV/^\circC
\item
  સીધા સેલ્સિયસમાં કેલિબ્રેટેડ
\item
  ઓપરેટિંગ રેન્જ: -55^\circC થી +150^\circC
\end{itemize}

\textbf{સર્કિટ:}

\begin{itemize}
\tightlist
\item
  ફક્ત પાવર સપ્લાય કનેક્શનની જરૂર
\item
  આઉટપુટ વોલ્ટમીટર સાથે સીધું વાંચી શકાય
\item
  બાહ્ય કેલિબ્રેશનની જરૂર નથી
\end{itemize}

\end{solutionbox}
\begin{mnemonicbox}
``TEN mV TRICK: તાપમાન વધારો મિલિવોલ્ટ્સમાં નોંધાય છે: દસ
વધારો સેલ્સિયસ કેલ્વિન સૂચવે છે.''

\end{mnemonicbox}
\subsection*{પ્રશ્ન 5(બ) [4
ગુણ]}\label{uxaaauxab0uxab6uxaa8-5uxaac-4-uxa97uxaa3}

\textbf{હાર્મોનિક ડિસ્ટોરશન એનાલાઇઝરની કામગીરીનું વર્ણન જરૂરી આકૃતિ સાથે કરો.}

\begin{solutionbox}

હાર્મોનિક ડિસ્ટોરશન એનાલાઇઝર સિગ્નલ ક્વોલિટી નક્કી કરવા માટે સિગ્નલમાં હાર્મોનિક
કન્ટેન્ટનું માપન કરે છે.

\begin{center}
\textbf{Mermaid Diagram (Code)}
\begin{verbatim}
{Shaded}
{Highlighting}[]
graph LR
    A[ઇનપુટ સિગ્નલ] {-{-}{} B[એટેન્યુએટર]}
    B {-{-}{} C[નોચ ફિલ્ટર]}
    C {-{-}{} D[એમ્પ્લિફાયર]}
    D {-{-}{} E[RMS ડિટેક્ટર]}
    A {-{-}{} F[રેફરન્સ RMS]}
    E {-{-}{} G[કેલ્ક્યુલેટર]}
    F {-{-}{} G}
    G {-{-}{} H[ડિસ્પ્લે]}
    style C fill:\#f9f,stroke:\#333,stroke{-width:2px}
{Highlighting}
{Shaded}
\end{verbatim}
\end{center}

\textbf{કાર્યસિદ્ધાંત:}

\begin{itemize}
\tightlist
\item
  નોચ ફિલ્ટર દ્વારા મૂળભૂત આવૃત્તિ ફિલ્ટર કરવામાં આવે છે
\item
  બાકીના હાર્મોનિક્સ માપવામાં આવે છે
\item
  THD = (હાર્મોનિક્સનો VRMS)/(મૂળભૂત આવૃત્તિનો VRMS)
\item
  ટકાવારી અથવા dB માં વ્યક્ત કરવામાં આવે છે
\end{itemize}

\textbf{ઓપરેશન સ્ટેપ્સ:}

\begin{enumerate}
\tightlist
\item
  કુલ સિગ્નલ RMS માપો
\item
  મૂળભૂત આવૃત્તિ ફિલ્ટર કરો
\item
  બાકીના હાર્મોનિક્સ માપો
\item
  THD રેશિઓની ગણતરી કરો
\end{enumerate}

\end{solutionbox}
\begin{mnemonicbox}
``FRONT એનાલિસિસ: ફિલ્ટર મૂળ નોટને સંપૂર્ણપણે દૂર કરે છે
બાકીના સિગ્નલના એનાલિસિસ માટે.''

\end{mnemonicbox}
\subsection*{પ્રશ્ન 5(ક) [7
ગુણ]}\label{uxaaauxab0uxab6uxaa8-5uxa95-7-uxa97uxaa3}

\textbf{સ્પેક્ટ્રમ એનાલાયઝરનું કાર્ય જરૂરી ડાયાગ્રામ સાથે વિગતવાર વર્ણન કરો.}

\begin{solutionbox}

સ્પેક્ટ્રમ એનાલાઇઝર સિગ્નલના સ્પેક્ટ્રલ રચનાને દર્શાવતા આવૃત્તિ સામે સિગ્નલ એમ્પ્લિટ્યુડને
દર્શાવે છે.

\begin{center}
\textbf{Mermaid Diagram (Code)}
\begin{verbatim}
{Shaded}
{Highlighting}[]
graph LR
    A[RF ઇનપુટ] {-{-}{} B[એટેન્યુએટર]}
    B {-{-}{} C[મિક્સર]}
    D[લોકલ ઓસિલેટર] {-{-}{} C}
    C {-{-}{} E[IF ફિલ્ટર]}
    E {-{-}{} F[ડિટેક્ટર]}
    F {-{-}{} G[ડિસ્પ્લે]}
    H[સ્વીપ જનરેટર] {-{-}{} D}
    H {-{-}{} G}
    style E fill:\#f9f,stroke:\#333,stroke{-width:2px}
{Highlighting}
{Shaded}
\end{verbatim}
\end{center}

\textbf{કાર્યસિદ્ધાંત:}

\begin{itemize}
\tightlist
\item
  \textbf{સુપરહેટેરોડાઇન સિદ્ધાંત}: ઇનપુટ સિગ્નલ લોકલ ઓસિલેટર સાથે મિક્સ કરવામાં આવે
  છે
\item
  \textbf{સ્વીપ ટેકનિક}: LO આવૃત્તિ રસપ્રદ શ્રેણી પર સ્વીપ કરવામાં આવે છે
\item
  \textbf{રિઝોલ્યૂશન બેન્ડવિડ્થ}: IF ફિલ્ટર બેન્ડવિડ્થ દ્વારા નિયંત્રિત
\item
  \textbf{ડિટેક્શન}: IF સિગ્નલને એમ્પ્લિટ્યુડ માહિતીમાં રૂપાંતરિત કરે છે
\item
  \textbf{ડિસ્પ્લે}: ફ્રિક્વન્સી ડોમેઇન રજૂઆત બતાવે છે
\end{itemize}

\textbf{પ્રકારો:}

\begin{itemize}
\tightlist
\item
  સ્વેપ્ટ-ટ્યુન્ડ સ્પેક્ટ્રમ એનાલાઇઝર
\item
  FFT-આધારિત સ્પેક્ટ્રમ એનાલાઇઝર
\item
  રીયલ-ટાઇમ સ્પેક્ટ્રમ એનાલાઇઝર
\end{itemize}

\textbf{એપ્લિકેશન:}

\begin{itemize}
\tightlist
\item
  સિગ્નલ એનાલિસિસ
\item
  EMI/EMC ટેસ્ટિંગ
\item
  કોમ્યુનિકેશન સિસ્ટમ ટેસ્ટિંગ
\item
  હાર્મોનિક એનાલિસિસ
\item
  મોડ્યુલેશન એનાલિસિસ
\end{itemize}

\end{solutionbox}
\begin{mnemonicbox}
``SAFER વ્યૂ: સ્વીપ RF તપાસવા માટે આવૃત્તિઓનું એનાલિસિસ કરે
છે.''

\end{mnemonicbox}
\subsection*{પ્રશ્ન 5(અ OR) [3
ગુણ]}\label{uxaaauxab0uxab6uxaa8-5uxa85-or-3-uxa97uxaa3}

\textbf{એનાલોગ ટ્રાન્સડ્યુસર અને ડીજીટલ ટ્રાન્સડ્યુસર સમજાવો. પ્રાથમિક ટ્રાન્સડ્યુસર
અને સેકન્ડરી ટ્રાન્સડ્યુસર પણ સમજાવો.}

\begin{solutionbox}

{\def\LTcaptype{none} % do not increment counter
\begin{longtable}[]{@{}
  >{\raggedright\arraybackslash}p{(\linewidth - 2\tabcolsep) * \real{0.5517}}
  >{\raggedright\arraybackslash}p{(\linewidth - 2\tabcolsep) * \real{0.4483}}@{}}
\toprule\noalign{}
\begin{minipage}[b]{\linewidth}\raggedright
ટ્રાન્સડ્યુસર પ્રકાર
\end{minipage} & \begin{minipage}[b]{\linewidth}\raggedright
વર્ણન
\end{minipage} \\
\midrule\noalign{}
\endhead
\bottomrule\noalign{}
\endlastfoot
\textbf{એનાલોગ ટ્રાન્સડ્યુસર} & ઇનપુટ ભૌતિક જથ્થાના પ્રમાણમાં સતત આઉટપુટ સિગ્નલ
ઉત્પન્ન કરે છે \\
\textbf{ડિજિટલ ટ્રાન્સડ્યુસર} & ઇનપુટ જથ્થાનું પ્રતિનિધિત્વ કરતા ડિસ્ક્રીટ/બાઇનરી
આઉટપુટ સિગ્નલ ઉત્પન્ન કરે છે \\
\textbf{પ્રાથમિક ટ્રાન્સડ્યુસર} & ભૌતિક જથ્થાને સીધા જ ઇલેક્ટ્રિકલ સિગ્નલમાં રૂપાંતરિત
કરે છે \\
\textbf{સેકન્ડરી ટ્રાન્સડ્યુસર} & પ્રાથમિક ટ્રાન્સડ્યુસરના આઉટપુટને બીજા સ્વરૂપમાં
રૂપાંતરિત કરે છે \\
\end{longtable}
}

\begin{verbatim}
Analog vs Digital Output:

Analog:   ────────────────
             /{      /}
            /  {    /  }
           /    {  /    }

Digital:  ┌───┐    ┌───┐
          │   │    │   │
          │   │    │   │
          └───┘    └───┘
\end{verbatim}

\end{solutionbox}
\begin{mnemonicbox}
``PADS: પ્રાથમિક અને ડિજિટલ/એનાલોગ સેકન્ડરી.''

\end{mnemonicbox}
\subsection*{પ્રશ્ન 5(બ OR) [4
ગુણ]}\label{uxaaauxab0uxab6uxaa8-5uxaac-or-4-uxa97uxaa3}

\textbf{ડીજીટલ આઈસી ટેસ્ટરનું કાર્ય જરૂરી ડાયાગ્રામ સાથે વિગતવાર સમજાવો.}

\begin{solutionbox}

ડિજિટલ IC ટેસ્ટર ટેસ્ટ પેટર્ન લાગુ કરીને અને પ્રતિસાદનું વિશ્લેષણ કરીને ઇન્ટિગ્રેટેડ
સર્કિટની કાર્યક્ષમતા ચકાસે છે.

\begin{center}
\textbf{Mermaid Diagram (Code)}
\begin{verbatim}
{Shaded}
{Highlighting}[]
graph LR
    A[માઇક્રોકંટ્રોલર] {-{-}{} B[ટેસ્ટ પેટર્ન જનરેટર]}
    A {-{-}{} C[રિઝલ્ટ એનાલાઇઝર]}
    B {-{-}{} D[ટેસ્ટ સોકેટ]}
    D {-{-}{} C}
    A {-{-}{} E[ડિસ્પ્લે/ઇન્ટરફેસ]}
    F[પાવર સપ્લાય] {-{-}{} D}
    style D fill:\#f9f,stroke:\#333,stroke{-width:2px}
{Highlighting}
{Shaded}
\end{verbatim}
\end{center}

\textbf{કાર્યસિદ્ધાંત:}

\begin{itemize}
\tightlist
\item
  IC ને ZIF (ઝીરો ઇન્સર્શન ફોર્સ) સોકેટમાં દાખલ કરવામાં આવે છે
\item
  IC પ્રકાર માટે ટેસ્ટ પેરામીટર્સ પસંદ કરવામાં આવે છે
\item
  પેટર્ન જનરેટર ચોક્કસ ઇનપુટ સિગ્નલ લાગુ કરે છે
\item
  આઉટપુટની અપેક્ષિત પરિણામો સાથે તુલના કરવામાં આવે છે
\item
  પાસ/ફેલ સૂચના પ્રદર્શિત થાય છે
\end{itemize}

\textbf{ફીચર્સ:}

\begin{itemize}
\tightlist
\item
  TTL, CMOS, મેમરી ICs ટેસ્ટ કરે છે
\item
  અજ્ઞાત ICs ઓળખે છે
\item
  ઓપન/શોર્ટ સર્કિટ શોધે છે
\item
  ફંક્શન વેરિફિકેશન
\end{itemize}

\end{solutionbox}
\begin{mnemonicbox}
``TRIG ટેસ્ટ: ટેસ્ટ, પેટર્ન ચલાવો, ખામીઓ ઓળખો, રિપોર્ટ જનરેટ
કરો.''

\end{mnemonicbox}
\subsection*{પ્રશ્ન 5(ક OR) [7
ગુણ]}\label{uxaaauxab0uxab6uxaa8-5uxa95-or-7-uxa97uxaa3}

\textbf{ફંક્શન જનરેટરનું કાર્ય જરૂરી ડાયાગ્રામ સાથે વિગતવાર સમજાવો.}

\begin{solutionbox}

ફંક્શન જનરેટર ઇલેક્ટ્રોનિક સર્કિટના પરીક્ષણ માટે વિવિધ આવૃત્તિઓ પર વિવિધ વેવફોર્મ
ઉત્પન્ન કરે છે.

\begin{center}
\textbf{Mermaid Diagram (Code)}
\begin{verbatim}
{Shaded}
{Highlighting}[]
graph LR
    A[ફ્રિક્વન્સી કંટ્રોલ] {-{-}{} B[ઓસિલેટર]}
    C[વેવફોર્મ સિલેક્ટર] {-{-}{} D[વેવફોર્મ શેપર]}
    B {-{-}{} D}
    D {-{-}{} E[એમ્પ્લિટ્યુડ કંટ્રોલ]}
    E {-{-}{} F[આઉટપુટ એમ્પ્લિફાયર]}
    F {-{-}{} G[આઉટપુટ]}
    H[DC ઓફસેટ] {-{-}{} F}
    style B fill:\#f9f,stroke:\#333,stroke{-width:2px}
{Highlighting}
{Shaded}
\end{verbatim}
\end{center}

\textbf{કાર્યસિદ્ધાંત:}

\begin{itemize}
\tightlist
\item
  \textbf{ઓસિલેટર}: મૂળભૂત વેવફોર્મ (સામાન્ય રીતે ત્રિકોણાકાર) ઉત્પન્ન કરે છે
\item
  \textbf{વેવફોર્મ શેપર}: સાઇન, સ્ક્વેર, ત્રિકોણાકાર, રેમ્પમાં રૂપાંતરિત કરે છે
\item
  \textbf{ફ્રિક્વન્સી કંટ્રોલ}: ઓસિલેશનનો દર સેટ કરે છે
\item
  \textbf{એમ્પ્લિટ્યુડ કંટ્રોલ}: આઉટપુટ વોલ્ટેજ લેવલ એડજસ્ટ કરે છે
\item
  \textbf{DC ઓફસેટ}: આઉટપુટ સિગ્નલમાં બાયસ ઉમેરે છે
\item
  \textbf{આઉટપુટ એમ્પ્લિફાયર}: લો ઇમ્પિડન્સ આઉટપુટ પ્રદાન કરે છે
\end{itemize}

\textbf{આઉટપુટ વેવફોર્મ:}

\begin{verbatim}
Sine:       ────────────
               /{    /}
              /  {  /  }
             /    {/    }

Square:     ┌───┐  ┌───┐
            │   │  │   │
            │   │  │   │
            └───┘  └───┘

Triangle:   ────────────
                /{    /}
               /  {  /  }
              /    {/    }

Ramp:       ────────────
                /|   /|
               / |  / |
              /  | /  |
\end{verbatim}

\textbf{એપ્લિકેશન:}

\begin{itemize}
\tightlist
\item
  એમ્પ્લિફાયર ટેસ્ટિંગ
\item
  ફિલ્ટર કેરેક્ટરાઇઝેશન
\item
  સિગ્નલ એનાલિસિસ
\item
  શૈક્ષણિક પ્રદર્શન
\item
  કેલિબ્રેશન રેફરન્સ
\end{itemize}

\end{solutionbox}
\begin{mnemonicbox}
``SWATOR: સાઇન વેવ અને ત્રિકોણાકાર ઓસિલેટર સિગ્નલ્સ ઉત્પન્ન
કરે છે.''

\end{mnemonicbox}

\end{document}
