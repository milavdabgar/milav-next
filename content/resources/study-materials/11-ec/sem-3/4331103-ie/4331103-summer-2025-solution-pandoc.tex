\documentclass[10pt,a4paper]{article}

% content/resources/templates/preamble.tex
\usepackage[margin=0.6in]{geometry}
\author{Milav Dabgar}
\usepackage{amsmath,amssymb,amsthm}
\usepackage{booktabs}
\usepackage{multirow}
\usepackage{xcolor}
\usepackage{tcolorbox}
\tcbuselibrary{breakable,skins}
\usepackage[colorlinks=true,linkcolor=blue]{hyperref}
\usepackage{titlesec}
\usepackage{enumitem}
\usepackage{tikz}
\usepackage{pgfplots}
\usepackage{circuitikz}
\usepackage[version=4]{mhchem}
\usepackage{longtable}
\usepackage{array}
\usepackage{float}
\usepackage{caption}
\usepackage{listings}

\lstset{
  basicstyle=\small\ttfamily,
  breaklines=true,
  breakatwhitespace=false,
  postbreak=\mbox{\textcolor{red}{$\hookrightarrow$}\space},
  float=false,
  numbers=left,
  numberstyle=\tiny\color{gray},
  numbersep=10pt,
  xleftmargin=2em,
  keywordstyle=\color{blue},
  commentstyle=\color{green!60!black},
  stringstyle=\color{purple},
  backgroundcolor=\color{gray!5},
  showstringspaces=false,
  tabsize=2,
  captionpos=b,
  keepspaces=true,
  columns=flexible
}

\pgfplotsset{compat=1.18}
\usetikzlibrary{shapes,arrows,positioning,calc,patterns,decorations.pathmorphing,decorations.markings,arrows.meta}

% Color scheme
\definecolor{headcolor}{RGB}{0,102,204}
\definecolor{keycolor}{RGB}{220,20,60}
\definecolor{solutioncolor}{RGB}{34,139,34}
\definecolor{mnemoniccolor}{RGB}{148,0,211}
\definecolor{codecolor}{RGB}{0,0,100}

% Spacing
\setlength{\parskip}{3pt}
\setlist[itemize]{nosep}
\setlist[enumerate]{nosep}

% Title formatting
\titleformat{\section}{\Large\bfseries\color{headcolor}}{\thesection}{1em}{}
\titleformat{\subsection}{\large\bfseries\color{headcolor}}{\thesubsection}{1em}{}

% Pandoc tightlist compatibility
\providecommand{\tightlist}{%
  \setlength{\itemsep}{0pt}\setlength{\parskip}{0pt}}

% Pandoc longtable compatibility
\newcounter{none}
\def\thenone{}


% content/resources/templates/english-boxes.tex
% This file is currently empty - it exists to maintain consistency with the import structure.
% Add custom environments here if needed in the future.


\begin{document}

\begin{center}
{\Huge\bfseries\color{headcolor} Subject Name Solutions}\\[5pt]
{\LARGE 4331103 -- Summer 2025}\\[3pt]
{\large Semester 1 Study Material}\\[3pt]
{\normalsize\textit{Detailed Solutions and Explanations}}
\end{center}

\vspace{10pt}

\subsection*{Question 1(a) [3 marks]}\label{q1a}

\textbf{Draw characteristics of Opto-Isolators, Opto-TRIAC and
Opto-transistor.}

\begin{solutionbox}

\textbf{Characteristics of Opto-Electronic Devices:}

{\def\LTcaptype{none} % do not increment counter
\begin{longtable}[]{@{}
  >{\centering\arraybackslash}p{(\linewidth - 4\tabcolsep) * \real{0.3409}}
  >{\centering\arraybackslash}p{(\linewidth - 4\tabcolsep) * \real{0.2727}}
  >{\centering\arraybackslash}p{(\linewidth - 4\tabcolsep) * \real{0.3864}}@{}}
\toprule\noalign{}
\begin{minipage}[b]{\linewidth}\centering
Opto-Isolator
\end{minipage} & \begin{minipage}[b]{\linewidth}\centering
Opto-TRIAC
\end{minipage} & \begin{minipage}[b]{\linewidth}\centering
Opto-Transistor
\end{minipage} \\
\midrule\noalign{}
\endhead
\bottomrule\noalign{}
\endlastfoot
\pandocbounded{\includegraphics[keepaspectratio,alt={Opto-Isolator Characteristic}]{https://i.ibb.co/3p0M3Cj/opto-isolator.png}}
&
\pandocbounded{\includegraphics[keepaspectratio,alt={Opto-TRIAC Characteristic}]{https://i.ibb.co/bJ8wPdz/opto-triac.png}}
&
\pandocbounded{\includegraphics[keepaspectratio,alt={Opto-Transistor Characteristic}]{https://i.ibb.co/h2GYVsP/opto-transistor.png}} \\
Linear relationship between LED current and photodetector current &
Non-linear triggering response with threshold & Linear current transfer
characteristic \\
CTR (Current Transfer Ratio) is key parameter & Triggering occurs at
specific current threshold & Collector current depends on base
illumination \\
\end{longtable}
}

\begin{itemize}
\tightlist
\item
  \textbf{CTR (Current Transfer Ratio)}: Ratio of output current to
  input current
\item
  \textbf{Trigger Current}: Minimum current needed to activate the
  device
\item
  \textbf{Linearity}: How proportional the output is to the input light
\end{itemize}

\end{solutionbox}
\begin{mnemonicbox}
``LTL - Light Transfers Like current flows -- Linear
for isolators/transistors, Triggered for TRIACs''

\end{mnemonicbox}
\subsection*{Question 1(b) [4 marks]}\label{q1b}

\textbf{Describe working \& constructional features of IGBT.}

\begin{solutionbox}

\textbf{IGBT Structure and Operation:}

\begin{center}
\textbf{Mermaid Diagram (Code)}
\begin{verbatim}
{Shaded}
{Highlighting}[]
graph LR
    A[Gate] {-{-}{} B[Metal Oxide]}
    B {-{-}{} C[P+ Body]}
    C {-{-}{} D[N{-} Drift Region]}
    D {-{-}{} E[P+ Collector/Substrate]}
    F[Emitter] {-{-}{} C}
    E {-{-}{} G[Collector]}
    style A fill:\#91a6ff
    style B fill:\#ffeead
    style C fill:\#ff9e9e
    style D fill:\#d9ffb3
    style E fill:\#ff9e9e
    style F fill:\#91a6ff
    style G fill:\#91a6ff
{Highlighting}
{Shaded}
\end{verbatim}
\end{center}

{\def\LTcaptype{none} % do not increment counter
\begin{longtable}[]{@{}ll@{}}
\toprule\noalign{}
Feature & Description \\
\midrule\noalign{}
\endhead
\bottomrule\noalign{}
\endlastfoot
Structure & Combines MOSFET input with BJT output \\
Layers & Gate/Metal Oxide/P+ Body/N- Drift/P+ Collector \\
Advantages & High input impedance, low conduction loss \\
Switching & Faster than BJT, better power handling than MOSFET \\
\end{longtable}
}

\begin{itemize}
\tightlist
\item
  \textbf{Voltage Controlled}: Device is controlled by gate voltage like
  MOSFET
\item
  \textbf{Conductivity Modulation}: P+ collector injects holes into
  drift region
\item
  \textbf{Low On-State Voltage}: Conduction losses lower than MOSFET
\end{itemize}

\end{solutionbox}
\begin{mnemonicbox}
``IGBT MBC'' - ``Input from MOS, Body handles
current, Collector acts like BJT''

\end{mnemonicbox}
\subsection*{Question 1(c) [7 marks]}\label{q1c}

\textbf{Explain working of SCR using two-transistor analogy.}

\begin{solutionbox}

\textbf{SCR as Two-Transistor Model:}

\begin{center}
\textbf{Mermaid Diagram (Code)}
\begin{verbatim}
{Shaded}
{Highlighting}[]
graph LR
    A[Anode] {-{-}{} B[P1]}
    B {-{-}{} C[N1]}
    C {-{-}{} D[P2]}
    D {-{-}{} E[N2]}
    E {-{-}{} F[Cathode]}
    G[Gate] {-{-}{} D}
    
    subgraph PNP Transistor
    B
    C
    D
    end
    
    subgraph NPN Transistor
    C
    D
    E
    end
    
    style A fill:\#91a6ff
    style B fill:\#ff9e9e
    style C fill:\#d9ffb3
    style D fill:\#ff9e9e
    style E fill:\#d9ffb3
    style F fill:\#91a6ff
    style G fill:\#91a6ff
{Highlighting}
{Shaded}
\end{verbatim}
\end{center}

\textbf{Two-Transistor Explanation:}

{\def\LTcaptype{none} % do not increment counter
\begin{longtable}[]{@{}
  >{\raggedright\arraybackslash}p{(\linewidth - 4\tabcolsep) * \real{0.3235}}
  >{\raggedright\arraybackslash}p{(\linewidth - 4\tabcolsep) * \real{0.2941}}
  >{\raggedright\arraybackslash}p{(\linewidth - 4\tabcolsep) * \real{0.3824}}@{}}
\toprule\noalign{}
\begin{minipage}[b]{\linewidth}\raggedright
Component
\end{minipage} & \begin{minipage}[b]{\linewidth}\raggedright
Function
\end{minipage} & \begin{minipage}[b]{\linewidth}\raggedright
Connections
\end{minipage} \\
\midrule\noalign{}
\endhead
\bottomrule\noalign{}
\endlastfoot
PNP (T1) & Upper transistor & Emitter to Anode, Collector to N1, Base to
P2-N1 junction \\
NPN (T2) & Lower transistor & Emitter to Cathode, Collector to P1-N1
junction, Base to Gate \\
Feedback & Regenerative action & T1's collector current = T2's base
current \& vice versa \\
\end{longtable}
}

\begin{itemize}
\tightlist
\item
  \textbf{Latching Mechanism}: Once triggered, transistors keep each
  other ON
\item
  \textbf{Triggering}: Small gate current \rightarrow T2 turns ON \rightarrow T1 gets base
  current \rightarrow Both remain ON
\item
  \textbf{Holding Current}: Minimum current needed to maintain
  regenerative action
\item
  \textbf{Turn-OFF}: Anode current must fall below holding current
\end{itemize}

\end{solutionbox}
\begin{mnemonicbox}
``PPFF'' - ``Positive feedback Perpetuates Forward
conduction''

\end{mnemonicbox}
\subsection*{Question 1(c) OR [7
marks]}\label{q1c}

\textbf{Explain the working of Solid state relay using Opto-SCR.}

\begin{solutionbox}

\textbf{Solid State Relay with Opto-SCR:}

\begin{center}
\textbf{Mermaid Diagram (Code)}
\begin{verbatim}
{Shaded}
{Highlighting}[]
graph LR
    A[AC/DC Input] {-{-}{} B[LED]}
    B {-{-}{} C[Photo{-}SCR/Detector]}
    C {-{-}{} D[Main SCR/TRIAC]}
    D {-{-}{} E[Output Load]}
    F[Zero Crossing Circuit] {-{-}{} D}
    style A fill:\#b3e0ff
    style B fill:\#ffcccc
    style C fill:\#ffee99
    style D fill:\#ccffcc
    style E fill:\#dddddd
    style F fill:\#e6ccff
{Highlighting}
{Shaded}
\end{verbatim}
\end{center}

\textbf{Working Principle and Components:}

{\def\LTcaptype{none} % do not increment counter
\begin{longtable}[]{@{}
  >{\raggedright\arraybackslash}p{(\linewidth - 4\tabcolsep) * \real{0.2500}}
  >{\raggedright\arraybackslash}p{(\linewidth - 4\tabcolsep) * \real{0.3571}}
  >{\raggedright\arraybackslash}p{(\linewidth - 4\tabcolsep) * \real{0.3929}}@{}}
\toprule\noalign{}
\begin{minipage}[b]{\linewidth}\raggedright
Stage
\end{minipage} & \begin{minipage}[b]{\linewidth}\raggedright
Function
\end{minipage} & \begin{minipage}[b]{\linewidth}\raggedright
Advantage
\end{minipage} \\
\midrule\noalign{}
\endhead
\bottomrule\noalign{}
\endlastfoot
Input & Low voltage control signal activates LED & Isolation from high
power \\
Opto-Coupler & LED light triggers photo-sensitive SCR & Electrical
isolation \\
Driver Circuit & Photo-SCR activates main switching device &
Amplification of switching capacity \\
Output Stage & Main SCR/TRIAC controls high-power load & Handles load
current \\
Snubber & RC circuit protects from voltage spikes & Prevents false
triggering \\
\end{longtable}
}

\begin{itemize}
\tightlist
\item
  \textbf{Electrical Isolation}: Complete separation between control and
  power circuits (\textgreater1000V)
\item
  \textbf{Zero-Crossing}: Switching only at zero voltage reduces EMI/RFI
  noise
\item
  \textbf{Silent Operation}: No mechanical clicks unlike traditional
  relays
\item
  \textbf{Long Life}: No mechanical wear as in conventional relays
\end{itemize}

\end{solutionbox}
\begin{mnemonicbox}
``LIPO'' - ``Light In, Power Out - isolation
guaranteed''

\end{mnemonicbox}
\subsection*{Question 2(a) [3 marks]}\label{q2a}

\textbf{Explain the working of snubber circuit for SCR.}

\begin{solutionbox}

\textbf{Snubber Circuit for SCR:}

\begin{verbatim}
    +{-{-}{-}||{-}{-}{-}+}
    |   C1   |
    |        |
A{-{-}{-}+        +{-}{-}{-}R1{-}{-}{-}+}
|                     |
SCR                   |
|                     |
K{-{-}{-}{-}{-}{-}{-}{-}{-}{-}{-}{-}{-}{-}{-}{-}{-}{-}{-}{-}{-}+}
\end{verbatim}

{\def\LTcaptype{none} % do not increment counter
\begin{longtable}[]{@{}
  >{\raggedright\arraybackslash}p{(\linewidth - 4\tabcolsep) * \real{0.2619}}
  >{\raggedright\arraybackslash}p{(\linewidth - 4\tabcolsep) * \real{0.2143}}
  >{\raggedright\arraybackslash}p{(\linewidth - 4\tabcolsep) * \real{0.5238}}@{}}
\toprule\noalign{}
\begin{minipage}[b]{\linewidth}\raggedright
Component
\end{minipage} & \begin{minipage}[b]{\linewidth}\raggedright
Purpose
\end{minipage} & \begin{minipage}[b]{\linewidth}\raggedright
Sizing Consideration
\end{minipage} \\
\midrule\noalign{}
\endhead
\bottomrule\noalign{}
\endlastfoot
Capacitor (C1) & Limits dv/dt rate & Based on max dv/dt rating of SCR \\
Resistor (R1) & Limits discharge current & Based on capacitor value and
switching frequency \\
\end{longtable}
}

\begin{itemize}
\tightlist
\item
  \textbf{dv/dt Protection}: Prevents false triggering due to rapid
  voltage rise
\item
  \textbf{Turn-OFF Support}: Helps in commutation by providing alternate
  path
\item
  \textbf{Energy Absorption}: Absorbs energy from inductive loads during
  switching
\end{itemize}

\end{solutionbox}
\begin{mnemonicbox}
``CARD'' - ``Capacitor And Resistor Damp unwanted
triggering''

\end{mnemonicbox}
\subsection*{Question 2(b) [4 marks]}\label{q2b}

\textbf{Write the differences between forced commutation and natural
commutation.}

\begin{solutionbox}

\textbf{Comparison of Commutation Methods:}

{\def\LTcaptype{none} % do not increment counter
\begin{longtable}[]{@{}
  >{\raggedright\arraybackslash}p{(\linewidth - 4\tabcolsep) * \real{0.2157}}
  >{\raggedright\arraybackslash}p{(\linewidth - 4\tabcolsep) * \real{0.3725}}
  >{\raggedright\arraybackslash}p{(\linewidth - 4\tabcolsep) * \real{0.4118}}@{}}
\toprule\noalign{}
\begin{minipage}[b]{\linewidth}\raggedright
Parameter
\end{minipage} & \begin{minipage}[b]{\linewidth}\raggedright
Forced Commutation
\end{minipage} & \begin{minipage}[b]{\linewidth}\raggedright
Natural Commutation
\end{minipage} \\
\midrule\noalign{}
\endhead
\bottomrule\noalign{}
\endlastfoot
Definition & External circuit forces SCR to turn OFF & AC source
naturally reduces current to zero \\
Application & DC circuits primarily & AC circuits primarily \\
Components & Requires additional components (capacitors, inductors) & No
extra components needed \\
Complexity & More complex circuit design & Simpler circuit design \\
Energy & Extra energy needed for commutation & Uses existing source
energy \\
Control & Can be controlled precisely & Happens at fixed points of AC
cycle \\
Cost & Higher due to extra components & Lower cost implementation \\
\end{longtable}
}

\begin{itemize}
\tightlist
\item
  \textbf{Timing Control}: Forced commutation offers better timing
  control
\item
  \textbf{Circuit Size}: Natural commutation results in smaller circuit
  size
\item
  \textbf{Reliability}: Natural commutation has fewer components to fail
\end{itemize}

\end{solutionbox}
\begin{mnemonicbox}
``DANCE'' - ``DC needs Active commutation, Natural
for AC, Costs Extra for forced''

\end{mnemonicbox}
\subsection*{Question 2(c) [7 marks]}\label{q2c}

\textbf{Describe the working of UPS with the help of block diagram.}

\begin{solutionbox}

\textbf{UPS Block Diagram and Operation:}

\begin{center}
\textbf{Mermaid Diagram (Code)}
\begin{verbatim}
{Shaded}
{Highlighting}[]
graph LR
    A[AC Input] {-{-}{} B[Rectifier/Charger]}
    B {-{-}{} C[Battery Bank]}
    C {-{-}{} D[Inverter]}
    B {-{-}{} D}
    D {-{-}{} E[Output Filter]}
    E {-{-}{} F[AC Output]}
    G[Control Circuit] {-{-}{} B}
    G {-{-}{} D}
    H[Bypass Switch] {-{-}{} F}
    A {-{-}{} H}
    style A fill:\#b3e0ff
    style B fill:\#ffcccc
    style C fill:\#ffffb3
    style D fill:\#ccffcc
    style E fill:\#e6ccff
    style F fill:\#b3e0ff
    style G fill:\#ffee99
    style H fill:\#ffddbb
{Highlighting}
{Shaded}
\end{verbatim}
\end{center}

\textbf{UPS Operation Modes:}

{\def\LTcaptype{none} % do not increment counter
\begin{longtable}[]{@{}
  >{\raggedright\arraybackslash}p{(\linewidth - 4\tabcolsep) * \real{0.1935}}
  >{\raggedright\arraybackslash}p{(\linewidth - 4\tabcolsep) * \real{0.4194}}
  >{\raggedright\arraybackslash}p{(\linewidth - 4\tabcolsep) * \real{0.3871}}@{}}
\toprule\noalign{}
\begin{minipage}[b]{\linewidth}\raggedright
Mode
\end{minipage} & \begin{minipage}[b]{\linewidth}\raggedright
Description
\end{minipage} & \begin{minipage}[b]{\linewidth}\raggedright
Power Path
\end{minipage} \\
\midrule\noalign{}
\endhead
\bottomrule\noalign{}
\endlastfoot
Normal & AC source powers load via rectifier and inverter & AC Input \rightarrow
Rectifier \rightarrow Inverter \rightarrow Output \\
Battery & Battery powers load when AC fails & Battery \rightarrow Inverter \rightarrow
Output \\
Bypass & AC directly connects to load for maintenance & AC Input \rightarrow
Bypass Switch \rightarrow Output \\
Charging & Battery charges while in normal mode & Rectifier \rightarrow Battery \\
\end{longtable}
}

\begin{itemize}
\tightlist
\item
  \textbf{Online UPS}: Power always flows through rectifier/inverter
  (double conversion)
\item
  \textbf{Offline UPS}: Power flows directly to load, switches to
  battery when power fails
\item
  \textbf{Line-Interactive}: Similar to offline but with voltage
  regulation
\item
  \textbf{Backup Time}: Depends on battery capacity and load
  requirements
\end{itemize}

\end{solutionbox}
\begin{mnemonicbox}
``BRIC'' - ``Battery Ready when Input Cuts off''

\end{mnemonicbox}
\subsection*{Question 2(a) OR [3
marks]}\label{q2a}

\textbf{Explain pulse gate triggering method of SCR.}

\begin{solutionbox}

\textbf{Pulse Gate Triggering Method:}

\begin{verbatim}
      +{-{-}{-}{-}{-}+}
      |Pulse|
      |Gen. |
      +{-{-}+{-}{-}+}
         |
         v
A{-{-}{-}+{-}{-}{-}{-}{-}{-}{-}{-}+}
|   |        |
|   |  SCR   |
|   |        |
K{-{-}{-}+{-}{-}{-}{-}{-}{-}{-}{-}+}
\end{verbatim}

{\def\LTcaptype{none} % do not increment counter
\begin{longtable}[]{@{}lll@{}}
\toprule\noalign{}
Parameter & Specification & Advantage \\
\midrule\noalign{}
\endhead
\bottomrule\noalign{}
\endlastfoot
Pulse Width & 10-100 μs & Ensures proper turn-on \\
Amplitude & 1-3V above threshold & Reliable triggering \\
Rise Time & Fast (\textless1 μs) & Quick turn-on \\
Frequency & Single or train of pulses & Control over timing \\
\end{longtable}
}

\begin{itemize}
\tightlist
\item
  \textbf{Precise Control}: Exact timing of SCR turn-on
\item
  \textbf{Noise Immunity}: Less susceptible to false triggering
\item
  \textbf{Power Efficiency}: Low average gate power consumption
\item
  \textbf{Isolation}: Can be coupled through pulse transformer or
  opto-isolator
\end{itemize}

\end{solutionbox}
\begin{mnemonicbox}
``TRAP'' - ``Timed, Reliable, Amplitude-controlled
Pulses''

\end{mnemonicbox}
\subsection*{Question 2(b) OR [4
marks]}\label{q2b}

\textbf{List the commutation methods of SCR and explain any one in
detail.}

\begin{solutionbox}

\textbf{Commutation Methods of SCR:}

{\def\LTcaptype{none} % do not increment counter
\begin{longtable}[]{@{}lll@{}}
\toprule\noalign{}
Method & Circuit Type & Application \\
\midrule\noalign{}
\endhead
\bottomrule\noalign{}
\endlastfoot
Class A & Self-commutated by resonating LC & Low-power inverters \\
Class B & Self-commutated by AC source & AC power control \\
Class C & Complementary SCR commutation & DC choppers \\
Class D & External pulse commutation & DC/AC converters \\
Class E & External capacitor commutation & DC power control \\
Class F & Line commutation & AC line controlled rectifiers \\
\end{longtable}
}

\textbf{Detailed Explanation of Class E (Capacitor Commutation):}

\begin{center}
\textbf{Mermaid Diagram (Code)}
\begin{verbatim}
{Shaded}
{Highlighting}[]
graph LR
    A[DC Source] {-{-}{} B[SCR1]}
    B {-{-}{} C[Load]}
    C {-{-}{} D[Ground]}
    A {-{-}{} E[Commutating Capacitor]}
    E {-{-}{} F[Auxiliary SCR2]}
    F {-{-}{} D}
    style A fill:\#b3e0ff
    style B fill:\#ffcccc
    style C fill:\#ffffb3
    style D fill:\#ccffcc
    style E fill:\#e6ccff
    style F fill:\#ffcccc
{Highlighting}
{Shaded}
\end{verbatim}
\end{center}

\begin{itemize}
\tightlist
\item
  \textbf{Working Principle}: When SCR1 is ON and carrying load current,
  firing SCR2 connects pre-charged capacitor across SCR1, reverse
  biasing it
\item
  \textbf{Turn-OFF Time}: Determined by capacitor value and circuit
  resistance
\item
  \textbf{Applications}: DC choppers, power control circuits, inverters
\item
  \textbf{Advantages}: Simple circuit, reliable operation,
  cost-effective
\end{itemize}

\end{solutionbox}
\begin{mnemonicbox}
``CARE'' - ``Capacitor Applies Reverse voltage for
Extinction''

\end{mnemonicbox}
\subsection*{Question 2(c) OR [7
marks]}\label{q2c}

\textbf{Describe the working of SMPS with the help of block diagram.}

\begin{solutionbox}

\textbf{SMPS Block Diagram and Operation:}

\begin{center}
\textbf{Mermaid Diagram (Code)}
\begin{verbatim}
{Shaded}
{Highlighting}[]
graph LR
    A[AC Input] {-{-}{} B[EMI Filter]}
    B {-{-}{} C[Rectifier/PFC]}
    C {-{-}{} D[High Frequency Inverter]}
    D {-{-}{} E[HF Transformer]}
    E {-{-}{} F[Rectifier/Filter]}
    F {-{-}{} G[Output DC]}
    H[Feedback Control] {-{-}{} D}
    F {-{-}{} H}
    style A fill:\#b3e0ff
    style B fill:\#ffddbb
    style C fill:\#ffcccc
    style D fill:\#ccffcc
    style E fill:\#ffffb3
    style F fill:\#e6ccff
    style G fill:\#b3e0ff
    style H fill:\#ffee99
{Highlighting}
{Shaded}
\end{verbatim}
\end{center}

\textbf{SMPS Working Principles:}

{\def\LTcaptype{none} % do not increment counter
\begin{longtable}[]{@{}
  >{\raggedright\arraybackslash}p{(\linewidth - 4\tabcolsep) * \real{0.2121}}
  >{\raggedright\arraybackslash}p{(\linewidth - 4\tabcolsep) * \real{0.3030}}
  >{\raggedright\arraybackslash}p{(\linewidth - 4\tabcolsep) * \real{0.4848}}@{}}
\toprule\noalign{}
\begin{minipage}[b]{\linewidth}\raggedright
Block
\end{minipage} & \begin{minipage}[b]{\linewidth}\raggedright
Function
\end{minipage} & \begin{minipage}[b]{\linewidth}\raggedright
Key Components
\end{minipage} \\
\midrule\noalign{}
\endhead
\bottomrule\noalign{}
\endlastfoot
EMI Filter & Suppresses noise & Inductors, capacitors \\
Rectifier/PFC & Converts AC to DC, improves power factor & Diodes, boost
converter \\
HF Inverter & Creates high-frequency AC & Switching transistors
(MOSFET/IGBT) \\
HF Transformer & Isolates and transforms voltage & Ferrite core
transformer \\
Output Stage & Rectifies and filters to clean DC & Fast diodes, LC
filter \\
Feedback & Regulates output voltage & Opto-isolator, PWM controller \\
\end{longtable}
}

\begin{itemize}
\tightlist
\item
  \textbf{High Efficiency}: 70-95\% efficient compared to 50-60\% for
  linear power supplies
\item
  \textbf{Size Reduction}: High-frequency operation allows smaller
  transformers
\item
  \textbf{Regulation}: Feedback loop maintains stable output despite
  input/load changes
\item
  \textbf{Protection}: Built-in overcurrent, overvoltage, and thermal
  protection
\end{itemize}

\end{solutionbox}
\begin{mnemonicbox}
``RELIEF'' - ``Rectify, Energize at high frequency,
Isolate, Extract DC, Feedback''

\end{mnemonicbox}
\subsection*{Question 3(a) [3 marks]}\label{q3a}

\textbf{State the method to protect SCR against over voltage.}

\begin{solutionbox}

\textbf{SCR Overvoltage Protection Methods:}

{\def\LTcaptype{none} % do not increment counter
\begin{longtable}[]{@{}
  >{\raggedright\arraybackslash}p{(\linewidth - 4\tabcolsep) * \real{0.1600}}
  >{\raggedright\arraybackslash}p{(\linewidth - 4\tabcolsep) * \real{0.4800}}
  >{\raggedright\arraybackslash}p{(\linewidth - 4\tabcolsep) * \real{0.3600}}@{}}
\toprule\noalign{}
\begin{minipage}[b]{\linewidth}\raggedright
Method
\end{minipage} & \begin{minipage}[b]{\linewidth}\raggedright
Circuit Implementation
\end{minipage} & \begin{minipage}[b]{\linewidth}\raggedright
Protection Level
\end{minipage} \\
\midrule\noalign{}
\endhead
\bottomrule\noalign{}
\endlastfoot
Snubber Circuit & RC network across SCR & dv/dt protection \\
MOV (Metal Oxide Varistor) & Connected across SCR & Transient
suppression \\
Voltage Clamping & Zener diodes in series & Fixed voltage limiting \\
Crowbar Circuit & Sensing and shunting circuit & Complete shutdown \\
\end{longtable}
}

\begin{itemize}
\tightlist
\item
  \textbf{Voltage Rating}: Always use SCR with voltage rating 2-3 times
  normal operating voltage
\item
  \textbf{Rate-of-Rise}: Protect against fast transients with snubber
  circuits (dv/dt protection)
\item
  \textbf{Breakdown Voltage}: Never exceed reverse breakdown voltage of
  SCR junction
\item
  \textbf{Coordinated Protection}: Use multiple methods for critical
  applications
\end{itemize}

\end{solutionbox}
\begin{mnemonicbox}
``SCRAM'' - ``Snubber Circuits Reduce Abnormal
Maximum voltages''

\end{mnemonicbox}
\subsection*{Question 3(b) [4 marks]}\label{q3b}

\textbf{State any four advantages of polyphase rectifiers over
single-phase rectifiers.}

\begin{solutionbox}

\textbf{Advantages of Polyphase Rectifiers:}

{\def\LTcaptype{none} % do not increment counter
\begin{longtable}[]{@{}
  >{\raggedright\arraybackslash}p{(\linewidth - 4\tabcolsep) * \real{0.3438}}
  >{\raggedright\arraybackslash}p{(\linewidth - 4\tabcolsep) * \real{0.4062}}
  >{\raggedright\arraybackslash}p{(\linewidth - 4\tabcolsep) * \real{0.2500}}@{}}
\toprule\noalign{}
\begin{minipage}[b]{\linewidth}\raggedright
Advantage
\end{minipage} & \begin{minipage}[b]{\linewidth}\raggedright
Explanation
\end{minipage} & \begin{minipage}[b]{\linewidth}\raggedright
Impact
\end{minipage} \\
\midrule\noalign{}
\endhead
\bottomrule\noalign{}
\endlastfoot
Higher Power Handling & Distributes load across phases & Suitable for
high-power applications \\
Reduced Ripple & Overlapping phases reduce output ripple & Less
filtering required \\
Better Transformer Utilization & Higher transformer utilization factor
(0.955 vs 0.812) & More economical design \\
Improved Power Factor & Better line utilization & Reduced line losses \\
Lower Harmonic Content & Harmonics start at higher frequencies & Reduced
EMI issues \\
Higher Efficiency & Reduced losses due to better distribution & Lower
operating costs \\
\end{longtable}
}

\begin{itemize}
\tightlist
\item
  \textbf{Form Factor}: Lower form factor means better DC quality
\item
  \textbf{Ripple Frequency}: Higher ripple frequency is easier to filter
\item
  \textbf{Balanced Load}: Polyphase draws balanced current from supply
\item
  \textbf{Size Reduction}: Smaller filter components needed
\end{itemize}

\end{solutionbox}
\begin{mnemonicbox}
``HERBS'' - ``Higher efficiency, Even load, Reduced
ripple, Better PF, Smaller filters''

\end{mnemonicbox}
\subsection*{Question 3(c) [7 marks]}\label{q3c}

\textbf{Describe the working of solar Photovoltaic (PV) based power
generation with the help of block diagram.}

\begin{solutionbox}

\textbf{Solar PV Power Generation System:}

\begin{center}
\textbf{Mermaid Diagram (Code)}
\begin{verbatim}
{Shaded}
{Highlighting}[]
graph LR
    A[Solar PV Array] {-{-}{} B[Charge Controller]}
    B {-{-}{} C[Battery Bank]}
    C {-{-}{} D[Inverter]}
    D {-{-}{} E[AC Loads]}
    D {-{-}{} F[Grid Connection]}
    B {-{-}{} G[DC Loads]}
    H[Maximum Power Point Tracker] {-{-}{} B}
    A {-{-}{} H}
    style A fill:\#ffffb3
    style B fill:\#ffcccc
    style C fill:\#b3e0ff
    style D fill:\#ccffcc
    style E fill:\#e6ccff
    style F fill:\#ffddbb
    style G fill:\#e6ccff
    style H fill:\#ffee99
{Highlighting}
{Shaded}
\end{verbatim}
\end{center}

\textbf{System Components and Functions:}

{\def\LTcaptype{none} % do not increment counter
\begin{longtable}[]{@{}
  >{\raggedright\arraybackslash}p{(\linewidth - 4\tabcolsep) * \real{0.3143}}
  >{\raggedright\arraybackslash}p{(\linewidth - 4\tabcolsep) * \real{0.2857}}
  >{\raggedright\arraybackslash}p{(\linewidth - 4\tabcolsep) * \real{0.4000}}@{}}
\toprule\noalign{}
\begin{minipage}[b]{\linewidth}\raggedright
Component
\end{minipage} & \begin{minipage}[b]{\linewidth}\raggedright
Function
\end{minipage} & \begin{minipage}[b]{\linewidth}\raggedright
Key Features
\end{minipage} \\
\midrule\noalign{}
\endhead
\bottomrule\noalign{}
\endlastfoot
PV Array & Converts sunlight to DC electricity & Multiple
series/parallel connected panels \\
MPPT & Maximizes power extraction & Tracks optimal operating point \\
Charge Controller & Manages battery charging & Prevents
overcharging/deep discharge \\
Battery Bank & Energy storage & Deep cycle batteries for reliability \\
Inverter & Converts DC to AC & Pure sine wave for sensitive equipment \\
Distribution Panel & Routes power to loads & Includes protection
devices \\
\end{longtable}
}

\begin{itemize}
\tightlist
\item
  \textbf{Grid-Tied Systems}: Connected to utility grid, can sell excess
  power
\item
  \textbf{Off-Grid Systems}: Standalone systems with battery storage
\item
  \textbf{Hybrid Systems}: Can operate in both modes with battery backup
\item
  \textbf{Efficiency}: Typical system efficiency 15-20\% from sunlight
  to usable electricity
\end{itemize}

\end{solutionbox}
\begin{mnemonicbox}
``SIMPLE'' - ``Sun In, Maximum Power, Local Energy''

\end{mnemonicbox}
\subsection*{Question 3(a) OR [3
marks]}\label{q3a}

\textbf{State the method to protect SCR against over current.}

\begin{solutionbox}

\textbf{SCR Overcurrent Protection Methods:}

{\def\LTcaptype{none} % do not increment counter
\begin{longtable}[]{@{}
  >{\raggedright\arraybackslash}p{(\linewidth - 4\tabcolsep) * \real{0.2051}}
  >{\raggedright\arraybackslash}p{(\linewidth - 4\tabcolsep) * \real{0.4103}}
  >{\raggedright\arraybackslash}p{(\linewidth - 4\tabcolsep) * \real{0.3846}}@{}}
\toprule\noalign{}
\begin{minipage}[b]{\linewidth}\raggedright
Method
\end{minipage} & \begin{minipage}[b]{\linewidth}\raggedright
Implementation
\end{minipage} & \begin{minipage}[b]{\linewidth}\raggedright
Response Time
\end{minipage} \\
\midrule\noalign{}
\endhead
\bottomrule\noalign{}
\endlastfoot
Fuses & Fast-acting semiconductor fuses & Very fast (microseconds) \\
Circuit Breakers & Magnetic/thermal breakers & Medium (milliseconds) \\
Current Limiting Reactors & Series inductors & Instantaneous \\
Electronic Current Limiting & Sensing and control circuits & Fast
(microseconds) \\
\end{longtable}
}

\begin{itemize}
\tightlist
\item
  \textbf{Current Rating}: Always use SCR with current rating above
  maximum operating current
\item
  \textbf{di/dt Protection}: Limit rate of current rise to prevent
  junction damage
\item
  \textbf{Thermal Management}: Proper heatsinking to prevent thermal
  runaway
\item
  \textbf{Coordination}: Protection device must act before SCR is
  damaged
\end{itemize}

\end{solutionbox}
\begin{mnemonicbox}
``FIRE'' - ``Fuses Immediately Restrict Excessive
current''

\end{mnemonicbox}
\subsection*{Question 3(b) OR [4
marks]}\label{q3b}

\textbf{Explain basic principle of DC chopper.}

\begin{solutionbox}

\textbf{DC Chopper Basic Principle:}

\begin{center}
\textbf{Mermaid Diagram (Code)}
\begin{verbatim}
{Shaded}
{Highlighting}[]
graph LR
    A[DC Input] {-{-}{} B[Switching Device]}
    B {-{-}{} C[Filter]}
    C {-{-}{} D[DC Output]}
    E[Control Circuit] {-{-}{} B}
    style A fill:\#b3e0ff
    style B fill:\#ffcccc
    style C fill:\#ffffb3
    style D fill:\#ccffcc
    style E fill:\#ffee99
{Highlighting}
{Shaded}
\end{verbatim}
\end{center}

{\def\LTcaptype{none} % do not increment counter
\begin{longtable}[]{@{}
  >{\raggedright\arraybackslash}p{(\linewidth - 4\tabcolsep) * \real{0.3438}}
  >{\raggedright\arraybackslash}p{(\linewidth - 4\tabcolsep) * \real{0.4062}}
  >{\raggedright\arraybackslash}p{(\linewidth - 4\tabcolsep) * \real{0.2500}}@{}}
\toprule\noalign{}
\begin{minipage}[b]{\linewidth}\raggedright
Parameter
\end{minipage} & \begin{minipage}[b]{\linewidth}\raggedright
Description
\end{minipage} & \begin{minipage}[b]{\linewidth}\raggedright
Effect
\end{minipage} \\
\midrule\noalign{}
\endhead
\bottomrule\noalign{}
\endlastfoot
Duty Cycle (α) & Ratio of ON time to total period & Controls output
voltage \\
Switching Frequency & Number of ON/OFF cycles per second & Affects
ripple and filter size \\
Chopping Method & Step-up, Step-down, Buck-boost & Determines voltage
conversion \\
Control Strategy & PWM, Current mode, etc. & Affects system response \\
\end{longtable}
}

\begin{itemize}
\tightlist
\item
  \textbf{Basic Equation}: Vout = Vin \times Duty Cycle (for step-down
  chopper)
\item
  \textbf{Operating Principle}: Rapid switching controls average voltage
\item
  \textbf{Advantages}: High efficiency, precise control, compact size
\item
  \textbf{Applications}: DC motor drives, battery charging, DC voltage
  regulation
\end{itemize}

\end{solutionbox}
\begin{mnemonicbox}
``DISC'' - ``Duty cycle Influences Switching to
Control output''

\end{mnemonicbox}
\subsection*{Question 3(c) OR [7
marks]}\label{q3c}

\textbf{Draw the circuit diagram of 3-Φ Full Wave rectifier using diode
and explain it's working.}

\begin{solutionbox}

\textbf{3-Phase Full Wave Diode Rectifier (Bridge Configuration):}

\begin{verbatim}
    D1      D3      D5
    /{      /      /}
   /  {    /      /  }
  /    {  /      /    }
R{-{-}{-}{-}{-}+{-}{-}+{-}{-}{-}{-}{-}{-}+{-}{-}+{-}{-}{-}{-}+{-}{-}{-}{-}+}
      |           |         |
      |           |     +   |
S{-{-}{-}{-}{-}+{-}{-}+{-}{-}{-}{-}{-}{-}+{-}{-}+{-}{-}{-}{-}| Load |}
      |  |      |  |    |   |
      |  |      |  |    +   |
T{-{-}{-}{-}{-}+{-}{-}+{-}{-}{-}{-}{-}{-}+{-}{-}+{-}{-}{-}{-}+{-}{-}{-}{-}+}
       {/       /       /}
       D2       D4       D6
\end{verbatim}

\textbf{Working Principles:}

{\def\LTcaptype{none} % do not increment counter
\begin{longtable}[]{@{}lll@{}}
\toprule\noalign{}
Phase & Conduction Pattern & Output Characteristics \\
\midrule\noalign{}
\endhead
\bottomrule\noalign{}
\endlastfoot
0^\circ-60^\circ & D1 and D6 conduct & R and T phases connected to load \\
60^\circ-120^\circ & D1 and D2 conduct & R and S phases connected to load \\
120^\circ-180^\circ & D3 and D2 conduct & S and R phases connected to load \\
180^\circ-240^\circ & D3 and D4 conduct & S and T phases connected to load \\
240^\circ-300^\circ & D5 and D4 conduct & T and S phases connected to load \\
300^\circ-360^\circ & D5 and D6 conduct & T and R phases connected to load \\
\end{longtable}
}

\begin{itemize}
\tightlist
\item
  \textbf{Ripple Frequency}: 6 times the input frequency (300/360Hz for
  50/60Hz input)
\item
  \textbf{Ripple Factor}: Approximately 4.2\% (much lower than
  single-phase)
\item
  \textbf{Average Output Voltage}: Vdc = 1.35 \times Vrms (line voltage)
\item
  \textbf{Conduction Angle}: Each diode conducts for 120^\circ of cycle
\end{itemize}

\end{solutionbox}
\begin{mnemonicbox}
``PRESTO'' - ``Pairs of diodes Rectify Efficiently,
Six Times per cycle Output''

\end{mnemonicbox}
\subsection*{Question 4(a) [3 marks]}\label{q4a}

\textbf{Write the applications of Induction heating.}

\begin{solutionbox}

\textbf{Applications of Induction Heating:}

{\def\LTcaptype{none} % do not increment counter
\begin{longtable}[]{@{}
  >{\raggedright\arraybackslash}p{(\linewidth - 4\tabcolsep) * \real{0.3953}}
  >{\raggedright\arraybackslash}p{(\linewidth - 4\tabcolsep) * \real{0.3256}}
  >{\raggedright\arraybackslash}p{(\linewidth - 4\tabcolsep) * \real{0.2791}}@{}}
\toprule\noalign{}
\begin{minipage}[b]{\linewidth}\raggedright
Application Area
\end{minipage} & \begin{minipage}[b]{\linewidth}\raggedright
Specific Uses
\end{minipage} & \begin{minipage}[b]{\linewidth}\raggedright
Advantages
\end{minipage} \\
\midrule\noalign{}
\endhead
\bottomrule\noalign{}
\endlastfoot
Metal Heat Treatment & Hardening, annealing, tempering & Precise
control, localized heating \\
Melting & Foundry operations, precious metals & Clean, efficient
melting \\
Welding & Pipe welding, brazing, soldering & Concentrated heat, no
contact \\
Forging & Pre-heating billets, hot forming & Rapid heating, energy
efficient \\
Domestic & Induction cooktops & Safety, efficiency, control \\
Medical & Hyperthermia treatment & Controlled deep tissue heating \\
\end{longtable}
}

\begin{itemize}
\tightlist
\item
  \textbf{Industrial Advantages}: Fast heating, energy efficiency, clean
  process
\item
  \textbf{Control Benefits}: Precise temperature control, repeatable
  results
\item
  \textbf{Environmental Impact}: Reduced emissions compared to fossil
  fuel heating
\item
  \textbf{Metallurgical Quality}: Improved material properties in many
  applications
\end{itemize}

\end{solutionbox}
\begin{mnemonicbox}
``HAMMER'' - ``Hardening, Annealing, Melting,
Medical, Eddy-current cooking, Reshaping metals''

\end{mnemonicbox}
\subsection*{Question 4(b) [4 marks]}\label{q4b}

\textbf{Draw and explain the circuit of controlling AC load using TRIAC
and DIAC.}

\begin{solutionbox}

\textbf{AC Load Control with TRIAC and DIAC:}

\begin{verbatim}
      R1          C1
AC o{-{-}///{-}{-}+{-}{-}{-}||{-}{-}{-}+}
              |        |
              |  DIAC  |
              |   |    |
              |   v    |
              +{-{-}{-}+{-}{-}{-}{-}+}
              |        |
              | TRIAC  |
              |        |
AC o{-{-}{-}{-}{-}{-}{-}{-}{-}{-}+{-}{-}{-}{-}{-}{-}{-}{-}+{-}{-}{-}o LOAD}
\end{verbatim}

\textbf{Circuit Operation:}

{\def\LTcaptype{none} % do not increment counter
\begin{longtable}[]{@{}lll@{}}
\toprule\noalign{}
Component & Function & Effect on Circuit \\
\midrule\noalign{}
\endhead
\bottomrule\noalign{}
\endlastfoot
R1 & Variable resistor & Controls charging rate of C1 \\
C1 & Timing capacitor & Creates phase shift for triggering \\
DIAC & Bi-directional trigger & Provides sharp triggering pulse \\
TRIAC & Power control device & Controls current to load \\
RC Network & Phase-shift network & Determines firing angle \\
\end{longtable}
}

\begin{itemize}
\tightlist
\item
  \textbf{Phase Control}: Adjusting R1 changes phase angle at which DIAC
  triggers
\item
  \textbf{Power Control}: Varying firing angle controls average power to
  load
\item
  \textbf{Bi-directional Control}: Works on both half-cycles of AC input
\item
  \textbf{Applications}: Light dimmers, fan speed control, heater
  control
\end{itemize}

\end{solutionbox}
\begin{mnemonicbox}
``CRAFT'' - ``Capacitor and Resistor Adjust Firing
Time''

\end{mnemonicbox}
\subsection*{Question 4(c) [7 marks]}\label{q4c}

\textbf{Explain Spot Welding with Working and Applications.}

\begin{solutionbox}

\textbf{Spot Welding Process and Applications:}

\begin{center}
\textbf{Mermaid Diagram (Code)}
\begin{verbatim}
{Shaded}
{Highlighting}[]
graph LR
    A[Step 1: Material Positioning] {-{-}{} B[Step 2: Electrode Contact]}
    B {-{-}{} C[Step 3: Current Flow]}
    C {-{-}{} D[Step 4: Heat Generation]}
    D {-{-}{} E[Step 5: Weld Formation]}
    E {-{-}{} F[Step 6: Cooling]}
    style A fill:\#ffffb3
    style B fill:\#ffcccc
    style C fill:\#b3e0ff
    style D fill:\#e6ccff
    style E fill:\#ccffcc
    style F fill:\#ffddbb
{Highlighting}
{Shaded}
\end{verbatim}
\end{center}

\textbf{Spot Welding Working Principle:}

{\def\LTcaptype{none} % do not increment counter
\begin{longtable}[]{@{}
  >{\raggedright\arraybackslash}p{(\linewidth - 4\tabcolsep) * \real{0.2500}}
  >{\raggedright\arraybackslash}p{(\linewidth - 4\tabcolsep) * \real{0.3214}}
  >{\raggedright\arraybackslash}p{(\linewidth - 4\tabcolsep) * \real{0.4286}}@{}}
\toprule\noalign{}
\begin{minipage}[b]{\linewidth}\raggedright
Stage
\end{minipage} & \begin{minipage}[b]{\linewidth}\raggedright
Process
\end{minipage} & \begin{minipage}[b]{\linewidth}\raggedright
Parameters
\end{minipage} \\
\midrule\noalign{}
\endhead
\bottomrule\noalign{}
\endlastfoot
Setup & Material placed between electrodes & Sheet thickness, material
type \\
Contact & Electrodes apply pressure & 200-1000 pounds pressure \\
Current Flow & High current passes through workpiece & 1000-100,000
amperes \\
Heating & Resistance creates localized heating & Temperatures around
2500^\circF \\
Fusion & Material melts and forms nugget & 0.1-1 seconds duration \\
Cooling & Pressure maintained during cooling & Electrode cooling
important \\
\end{longtable}
}

\textbf{Applications of Spot Welding:}

\begin{itemize}
\tightlist
\item
  \textbf{Automotive}: Car body assembly, sheet metal joining
\item
  \textbf{Electronics}: Battery tabs, small component assembly
\item
  \textbf{Appliances}: Refrigerators, washing machines, dishwashers
\item
  \textbf{Aerospace}: Aircraft panel assembly, lightweight structures
\item
  \textbf{Medical}: Surgical instruments, implantable devices
\item
  \textbf{Consumer Products}: Metal furniture, containers, toys
\end{itemize}

\end{solutionbox}
\begin{mnemonicbox}
``PCAFRI'' - ``Position, Compress, Apply current,
Form nugget, Release after cooling, Inspect''

\end{mnemonicbox}
\subsection*{Question 4(a) OR [3
marks]}\label{q4a}

\textbf{Write the applications of Dielectric heating.}

\begin{solutionbox}

\textbf{Applications of Dielectric Heating:}

{\def\LTcaptype{none} % do not increment counter
\begin{longtable}[]{@{}
  >{\raggedright\arraybackslash}p{(\linewidth - 4\tabcolsep) * \real{0.2778}}
  >{\raggedright\arraybackslash}p{(\linewidth - 4\tabcolsep) * \real{0.3889}}
  >{\raggedright\arraybackslash}p{(\linewidth - 4\tabcolsep) * \real{0.3333}}@{}}
\toprule\noalign{}
\begin{minipage}[b]{\linewidth}\raggedright
Industry
\end{minipage} & \begin{minipage}[b]{\linewidth}\raggedright
Applications
\end{minipage} & \begin{minipage}[b]{\linewidth}\raggedright
Advantages
\end{minipage} \\
\midrule\noalign{}
\endhead
\bottomrule\noalign{}
\endlastfoot
Food Processing & Defrosting, cooking, pasteurization & Uniform heating,
speed \\
Wood Industry & Drying, glue curing, delamination & Reduced time,
improved quality \\
Textile & Drying yarns, fibers, finished goods & Energy efficiency,
speed \\
Plastics & Preheating, molding, welding & Uniform heating, no surface
damage \\
Pharmaceutical & Drying, sterilization & Controlled process, speed \\
Paper & Drying, glue setting & Uniform moisture removal \\
\end{longtable}
}

\begin{itemize}
\tightlist
\item
  \textbf{Process Benefits}: Volumetric heating (heats throughout, not
  just surface)
\item
  \textbf{Speed Advantage}: Significantly faster than conventional
  heating
\item
  \textbf{Quality Improvement}: More uniform heating, better product
  quality
\item
  \textbf{Energy Efficiency}: Direct energy transfer to material
\end{itemize}

\end{solutionbox}
\begin{mnemonicbox}
``FITPP'' - ``Food, Insulation drying, Textiles,
Plastics, Pharmaceutical products''

\end{mnemonicbox}
\subsection*{Question 4(b) OR [4
marks]}\label{q4b}

\textbf{Write short note on SCR Delay timer.}

\begin{solutionbox}

\textbf{SCR Delay Timer:}

\begin{center}
\textbf{Mermaid Diagram (Code)}
\begin{verbatim}
{Shaded}
{Highlighting}[]
graph LR
    A[Trigger Input] {-{-}{} B[RC Timing Circuit]}
    B {-{-}{} C[SCR]}
    C {-{-}{} D[Relay/Output Device]}
    E[Power Supply] {-{-}{} B}
    E {-{-}{} C}
    E {-{-}{} D}
    style A fill:\#b3e0ff
    style B fill:\#ffcccc
    style C fill:\#ffffb3
    style D fill:\#ccffcc
    style E fill:\#e6ccff
{Highlighting}
{Shaded}
\end{verbatim}
\end{center}

{\def\LTcaptype{none} % do not increment counter
\begin{longtable}[]{@{}lll@{}}
\toprule\noalign{}
Component & Function & Selection Criteria \\
\midrule\noalign{}
\endhead
\bottomrule\noalign{}
\endlastfoot
RC Network & Determines time delay & R\timesC gives approximate timing \\
SCR & Switching element & Current rating based on load \\
UJT/Trigger & Provides gate pulse & Reliable triggering circuit \\
Output Stage & Controls load & Relay or direct load connection \\
\end{longtable}
}

\begin{itemize}
\tightlist
\item
  \textbf{Timing Principle}: RC charging time determines delay period
\item
  \textbf{Accuracy}: Typically \pm5-10\% of set time
\item
  \textbf{Applications}: Industrial process control, sequence control,
  protection circuits
\item
  \textbf{Advantages}: Simple design, reliable operation, cost-effective
\end{itemize}

\end{solutionbox}
\begin{mnemonicbox}
``TIME'' - ``Timing Is Managed by Electronics''

\end{mnemonicbox}
\subsection*{Question 4(c) OR [7
marks]}\label{q4c}

\textbf{Explain the working of SCR as static switch. Write the
advantages of static switch.}

\begin{solutionbox}

\textbf{SCR as Static Switch:}

\begin{verbatim}
    +{-{-}{-}{-}{-}{-}{-}{-}{-}{-}{-}{-}{-}{-}{-}{-}{-}{-}+}
    |                  |
AC/DC o{-{-}{-}{-}{-}{-}+         |}
            SCR        LOAD
             |         |
Control o{-{-}{-}{-}|         |}
    |        |         |
    +{-{-}{-}{-}{-}{-}{-}{-}+{-}{-}{-}{-}{-}{-}{-}{-}{-}+}
\end{verbatim}

\textbf{Working Principles:}

{\def\LTcaptype{none} % do not increment counter
\begin{longtable}[]{@{}
  >{\raggedright\arraybackslash}p{(\linewidth - 4\tabcolsep) * \real{0.2000}}
  >{\raggedright\arraybackslash}p{(\linewidth - 4\tabcolsep) * \real{0.2333}}
  >{\raggedright\arraybackslash}p{(\linewidth - 4\tabcolsep) * \real{0.5667}}@{}}
\toprule\noalign{}
\begin{minipage}[b]{\linewidth}\raggedright
Mode
\end{minipage} & \begin{minipage}[b]{\linewidth}\raggedright
State
\end{minipage} & \begin{minipage}[b]{\linewidth}\raggedright
Characteristics
\end{minipage} \\
\midrule\noalign{}
\endhead
\bottomrule\noalign{}
\endlastfoot
OFF State & No gate signal & High impedance, minimal leakage \\
ON State & Gate triggered & Low impedance, high current flow \\
Turn-ON & Gate pulse applied & Fast transition (μs range) \\
Turn-OFF & Current falls below holding & Automatic in AC, needs
commutation in DC \\
\end{longtable}
}

\begin{itemize}
\tightlist
\item
  \textbf{DC Operation}: Requires commutation circuit for turn-off
\item
  \textbf{AC Operation}: Natural turn-off at zero crossing
\item
  \textbf{Control Methods}: Direct gate drive, pulse triggering,
  opto-isolation
\item
  \textbf{Protection}: Requires snubber circuits, current limiting
\end{itemize}

\textbf{Advantages of Static Switches:}

{\def\LTcaptype{none} % do not increment counter
\begin{longtable}[]{@{}
  >{\raggedright\arraybackslash}p{(\linewidth - 4\tabcolsep) * \real{0.2157}}
  >{\raggedright\arraybackslash}p{(\linewidth - 4\tabcolsep) * \real{0.2549}}
  >{\raggedright\arraybackslash}p{(\linewidth - 4\tabcolsep) * \real{0.5294}}@{}}
\toprule\noalign{}
\begin{minipage}[b]{\linewidth}\raggedright
Advantage
\end{minipage} & \begin{minipage}[b]{\linewidth}\raggedright
Description
\end{minipage} & \begin{minipage}[b]{\linewidth}\raggedright
Comparison with Mechanical
\end{minipage} \\
\midrule\noalign{}
\endhead
\bottomrule\noalign{}
\endlastfoot
No Moving Parts & No mechanical wear or tear & Longer lifetime (millions
of operations) \\
Silent Operation & No audible noise during switching & Important in
noise-sensitive applications \\
Fast Switching & Microsecond range switching & Much faster than
mechanical contacts \\
No Arcing & No contact bounce or arcing & Safer in hazardous
environments \\
Size \& Weight & Compact and lightweight & Significant space savings \\
EMI/RFI & Less electromagnetic interference & Better for sensitive
electronics \\
\end{longtable}
}

\begin{itemize}
\tightlist
\item
  \textbf{Reliability}: Higher MTBF (Mean Time Between Failures)
\item
  \textbf{Compatibility}: Works with electronic control systems
\item
  \textbf{Voltage Isolation}: Can incorporate opto-isolation
\item
  \textbf{Surge Handling}: Better transient protection with proper
  design
\end{itemize}

\end{solutionbox}
\begin{mnemonicbox}
``FANS'' - ``Fast switching, Arc-free operation, No
moving parts, Silent operation''

\end{mnemonicbox}
\subsection*{Question 5(a) [3 marks]}\label{q5a}

\textbf{What is DC Drive? Give Classification of DC Drives.}

\begin{solutionbox}

\textbf{DC Drive Definition and Classification:}

{\def\LTcaptype{none} % do not increment counter
\begin{longtable}[]{@{}
  >{\raggedright\arraybackslash}p{(\linewidth - 2\tabcolsep) * \real{0.3810}}
  >{\raggedright\arraybackslash}p{(\linewidth - 2\tabcolsep) * \real{0.6190}}@{}}
\toprule\noalign{}
\begin{minipage}[b]{\linewidth}\raggedright
Aspect
\end{minipage} & \begin{minipage}[b]{\linewidth}\raggedright
Description
\end{minipage} \\
\midrule\noalign{}
\endhead
\bottomrule\noalign{}
\endlastfoot
Definition & Electronic system that controls speed, torque, and
direction of DC motors \\
Basic Function & Controls armature voltage and/or field current to
regulate motor parameters \\
\end{longtable}
}

\textbf{Classification of DC Drives:}

{\def\LTcaptype{none} % do not increment counter
\begin{longtable}[]{@{}
  >{\raggedright\arraybackslash}p{(\linewidth - 4\tabcolsep) * \real{0.4667}}
  >{\raggedright\arraybackslash}p{(\linewidth - 4\tabcolsep) * \real{0.1556}}
  >{\raggedright\arraybackslash}p{(\linewidth - 4\tabcolsep) * \real{0.3778}}@{}}
\toprule\noalign{}
\begin{minipage}[b]{\linewidth}\raggedright
Classification Basis
\end{minipage} & \begin{minipage}[b]{\linewidth}\raggedright
Types
\end{minipage} & \begin{minipage}[b]{\linewidth}\raggedright
Characteristics
\end{minipage} \\
\midrule\noalign{}
\endhead
\bottomrule\noalign{}
\endlastfoot
Power Rating & Fractional, Integral, High Power & Based on horsepower
rating \\
Control Method & Open Loop, Closed Loop & Based on feedback mechanism \\
Quadrant Operation & Single, Two, Four Quadrant & Based on speed/torque
direction \\
Power Supply & Single-phase, Three-phase & Based on input power
configuration \\
Converter Type & Half-wave, Full-wave, Chopper & Based on power
conversion method \\
Application & General Purpose, Servo, Specialized & Based on intended
use \\
\end{longtable}
}

\begin{itemize}
\tightlist
\item
  \textbf{Power Range}: From fractional HP to several thousand HP
\item
  \textbf{Control Precision}: From basic to high-precision (0.01\%)
\item
  \textbf{Response Time}: From milliseconds to microseconds
\item
  \textbf{Protection}: Various built-in protection features
\end{itemize}

\end{solutionbox}
\begin{mnemonicbox}
``PQCAS'' - ``Power rating, Quadrants, Control type,
AC input phases, Switching method''

\end{mnemonicbox}
\subsection*{Question 5(b) [4 marks]}\label{q5b}

\textbf{Draw and explain the construction of variable reluctance type
Stepper motor.}

\begin{solutionbox}

\textbf{Variable Reluctance Stepper Motor Construction:}

\begin{verbatim}
    +{-{-}{-}{-}{-}{-}{-}{-}{-}{-}{-}{-}{-}{-}{-}{-}{-}+}
    |                 |
    |     Stator      |
    |    +{-{-}{-}{-}{-}{-}{-}+    |}
    |    |       |    |
    |    |Rotor  |    |
    |    |       |    |
    |    +{-{-}{-}{-}{-}{-}{-}+    |}
    |                 |
    +{-{-}{-}{-}{-}{-}{-}{-}{-}{-}{-}{-}{-}{-}{-}{-}{-}+}
\end{verbatim}

{\def\LTcaptype{none} % do not increment counter
\begin{longtable}[]{@{}
  >{\raggedright\arraybackslash}p{(\linewidth - 4\tabcolsep) * \real{0.3143}}
  >{\raggedright\arraybackslash}p{(\linewidth - 4\tabcolsep) * \real{0.4000}}
  >{\raggedright\arraybackslash}p{(\linewidth - 4\tabcolsep) * \real{0.2857}}@{}}
\toprule\noalign{}
\begin{minipage}[b]{\linewidth}\raggedright
Component
\end{minipage} & \begin{minipage}[b]{\linewidth}\raggedright
Construction
\end{minipage} & \begin{minipage}[b]{\linewidth}\raggedright
Function
\end{minipage} \\
\midrule\noalign{}
\endhead
\bottomrule\noalign{}
\endlastfoot
Stator & Laminated steel with multiple poles and windings & Creates
magnetic field when energized \\
Rotor & Soft iron with multiple teeth, NO permanent magnets & Aligns
with energized stator poles \\
Air Gap & Small space between rotor and stator & Affects step accuracy
and torque \\
Windings & Multiple phase windings on stator & Sequential energizing
creates rotation \\
\end{longtable}
}

\begin{itemize}
\tightlist
\item
  \textbf{Tooth Configuration}: Typically rotor teeth fewer than stator
  teeth
\item
  \textbf{Step Angle}: Determined by: Step angle = 360^\circ \div (Number of
  rotor teeth \times Number of phases)
\item
  \textbf{Construction Simplicity}: No permanent magnets or windings on
  rotor
\item
  \textbf{Operating Principle}: Magnetic reluctance path seeks to
  minimize when phases energized
\end{itemize}

\end{solutionbox}
\begin{mnemonicbox}
``STAR'' - ``Stator energizes, Teeth Align with
minimum Reluctance''

\end{mnemonicbox}
\subsection*{Question 5(c) [7 marks]}\label{q5c}

\textbf{Explain the working of VFD (Variable Frequency Drive).}

\begin{solutionbox}

\textbf{Variable Frequency Drive (VFD) Working:}

\begin{center}
\textbf{Mermaid Diagram (Code)}
\begin{verbatim}
{Shaded}
{Highlighting}[]
graph LR
    A[AC Input] {-{-}{} B[Rectifier]}
    B {-{-}{} C[DC Bus/Filter]}
    C {-{-}{} D[Inverter]}
    D {-{-}{} E[AC Motor]}
    F[Control System] {-{-}{} B}
    F {-{-}{} D}
    G[Operator Interface] {-{-}{} F}
    H[Feedback Sensors] {-{-}{} F}
    style A fill:\#b3e0ff
    style B fill:\#ffcccc
    style C fill:\#ffffb3
    style D fill:\#ccffcc
    style E fill:\#e6ccff
    style F fill:\#ffee99
    style G fill:\#ffddbb
    style H fill:\#d9ffb3
{Highlighting}
{Shaded}
\end{verbatim}
\end{center}

\textbf{VFD Components and Functions:}

{\def\LTcaptype{none} % do not increment counter
\begin{longtable}[]{@{}
  >{\raggedright\arraybackslash}p{(\linewidth - 4\tabcolsep) * \real{0.3548}}
  >{\raggedright\arraybackslash}p{(\linewidth - 4\tabcolsep) * \real{0.3226}}
  >{\raggedright\arraybackslash}p{(\linewidth - 4\tabcolsep) * \real{0.3226}}@{}}
\toprule\noalign{}
\begin{minipage}[b]{\linewidth}\raggedright
Component
\end{minipage} & \begin{minipage}[b]{\linewidth}\raggedright
Function
\end{minipage} & \begin{minipage}[b]{\linewidth}\raggedright
Features
\end{minipage} \\
\midrule\noalign{}
\endhead
\bottomrule\noalign{}
\endlastfoot
Rectifier & Converts AC to DC & 6-pulse or 12-pulse designs \\
DC Bus & Filters and stores energy & Capacitors and inductors \\
Inverter & Creates variable frequency AC & IGBT or MOSFET based \\
Control System & Manages overall operation & Microprocessor based \\
HMI & User interface & Display, keypad, communication \\
Protection & System protection & Current, voltage, temperature
sensors \\
\end{longtable}
}

\textbf{Working Principles:}

\begin{itemize}
\tightlist
\item
  \textbf{Speed Control Equation}: Motor Speed (RPM) = (Frequency \times 120)
  \div Number of poles
\item
  \textbf{Torque Control}: Maintaining V/F ratio controls torque output
\item
  \textbf{Soft Start}: Gradual frequency/voltage ramp-up reduces inrush
  current
\item
  \textbf{Braking Methods}: Regenerative, dynamic, or DC injection
  braking
\item
  \textbf{Energy Savings}: Significant energy savings at reduced speeds
\item
  \textbf{Advanced Features}: PID control, network communication,
  programmable functions
\end{itemize}

\end{solutionbox}
\begin{mnemonicbox}
``DRIVE'' - ``DC conversion, Regulation, Inverter
creates, Variable frequency, Efficient motor control''

\end{mnemonicbox}
\subsection*{Question 5(a) OR [3
marks]}\label{q5a}

\textbf{What are Hall effect sensors and what is their role in DC
motors?}

\begin{solutionbox}

\textbf{Hall Effect Sensors in DC Motors:}

{\def\LTcaptype{none} % do not increment counter
\begin{longtable}[]{@{}
  >{\raggedright\arraybackslash}p{(\linewidth - 2\tabcolsep) * \real{0.3810}}
  >{\raggedright\arraybackslash}p{(\linewidth - 2\tabcolsep) * \real{0.6190}}@{}}
\toprule\noalign{}
\begin{minipage}[b]{\linewidth}\raggedright
Aspect
\end{minipage} & \begin{minipage}[b]{\linewidth}\raggedright
Description
\end{minipage} \\
\midrule\noalign{}
\endhead
\bottomrule\noalign{}
\endlastfoot
Definition & Semiconductor-based sensors that detect magnetic fields \\
Principle & Voltage difference generated perpendicular to current flow
in magnetic field \\
Signal Output & Digital (ON/OFF) or analog (proportional to field
strength) \\
Size & Compact, can be integrated into motor housing \\
\end{longtable}
}

\textbf{Role in DC Motors:}

{\def\LTcaptype{none} % do not increment counter
\begin{longtable}[]{@{}
  >{\raggedright\arraybackslash}p{(\linewidth - 4\tabcolsep) * \real{0.3125}}
  >{\raggedright\arraybackslash}p{(\linewidth - 4\tabcolsep) * \real{0.4062}}
  >{\raggedright\arraybackslash}p{(\linewidth - 4\tabcolsep) * \real{0.2812}}@{}}
\toprule\noalign{}
\begin{minipage}[b]{\linewidth}\raggedright
Function
\end{minipage} & \begin{minipage}[b]{\linewidth}\raggedright
Application
\end{minipage} & \begin{minipage}[b]{\linewidth}\raggedright
Benefit
\end{minipage} \\
\midrule\noalign{}
\endhead
\bottomrule\noalign{}
\endlastfoot
Position Sensing & Rotor position detection & Precise commutation
timing \\
Speed Measurement & Pulse generation for RPM calculation & Accurate
speed feedback \\
Direction Detection & Phase sequence monitoring & Rotation direction
control \\
Current Sensing & Non-contact current measurement & Overload
protection \\
\end{longtable}
}

\begin{itemize}
\tightlist
\item
  \textbf{BLDC Motors}: Critical for electronic commutation (replacing
  mechanical commutator)
\item
  \textbf{Precision}: Higher accuracy than mechanical sensors
\item
  \textbf{Reliability}: No mechanical wear, longer service life
\item
  \textbf{Integration}: Can be integrated with drive electronics
\end{itemize}

\end{solutionbox}
\begin{mnemonicbox}
``MAPS'' - ``Measures position, Aids commutation,
Provides speed data, Senses magnetic fields''

\end{mnemonicbox}
\subsection*{Question 5(b) OR [4
marks]}\label{q5b}

\textbf{Explain working principle of stepper motor.}

\begin{solutionbox}

\textbf{Stepper Motor Working Principle:}

\begin{center}
\textbf{Mermaid Diagram (Code)}
\begin{verbatim}
{Shaded}
{Highlighting}[]
graph TD
    A[Step 1: Energize Phase A] {-{-}{} B[Rotor aligns with Phase A]}
    B {-{-}{} C[Step 2: Energize Phase B]}
    C {-{-}{} D[Rotor aligns with Phase B]}
    D {-{-}{} E[Step 3: Energize Phase C]}
    E {-{-}{} F[Rotor aligns with Phase C]}
    F {-{-}{} G[Step 4: Energize Phase D]}
    G {-{-}{} H[Rotor aligns with Phase D]}
    H {-{-}{} A}
    style A fill:\#ffffb3
    style B fill:\#ffcccc
    style C fill:\#b3e0ff
    style D fill:\#ccffcc
    style E fill:\#e6ccff
    style F fill:\#ffddbb
    style G fill:\#d9ffb3
    style H fill:\#ffee99
{Highlighting}
{Shaded}
\end{verbatim}
\end{center}

{\def\LTcaptype{none} % do not increment counter
\begin{longtable}[]{@{}
  >{\raggedright\arraybackslash}p{(\linewidth - 4\tabcolsep) * \real{0.3902}}
  >{\raggedright\arraybackslash}p{(\linewidth - 4\tabcolsep) * \real{0.3171}}
  >{\raggedright\arraybackslash}p{(\linewidth - 4\tabcolsep) * \real{0.2927}}@{}}
\toprule\noalign{}
\begin{minipage}[b]{\linewidth}\raggedright
Operating Mode
\end{minipage} & \begin{minipage}[b]{\linewidth}\raggedright
Description
\end{minipage} & \begin{minipage}[b]{\linewidth}\raggedright
Advantages
\end{minipage} \\
\midrule\noalign{}
\endhead
\bottomrule\noalign{}
\endlastfoot
Full Step & One phase energized at a time & Maximum torque \\
Half Step & Alternating one and two phases energized & Double
resolution, smoother \\
Microstepping & Proportional current in phases & Very smooth motion,
high resolution \\
Wave Drive & Sequential single phase energization & Lower power
consumption \\
\end{longtable}
}

\begin{itemize}
\tightlist
\item
  \textbf{Position Control}: Precise angular positioning without
  feedback
\item
  \textbf{Step Angle}: Common step angles are 1.8^\circ (200 steps/rev) or
  0.9^\circ (400 steps/rev)
\item
  \textbf{Holding Torque}: Maintains position when phases energized at
  standstill
\item
  \textbf{Open-Loop Control}: No position feedback normally required
\item
  \textbf{Speed-Torque}: Torque decreases as speed increases
\end{itemize}

\end{solutionbox}
\begin{mnemonicbox}
``STEPS'' - ``Sequential Triggering of
Electromagnetic Phases causes Stepping''

\end{mnemonicbox}
\subsection*{Question 5(c) OR [7
marks]}\label{q5c}

\textbf{Draw the block diagram of PLC and explain the function of each
block.}

\begin{solutionbox}

\textbf{PLC Block Diagram and Functions:}

\begin{center}
\textbf{Mermaid Diagram (Code)}
\begin{verbatim}
{Shaded}
{Highlighting}[]
graph TD
    A[Power Supply] {-{-}{} B[CPU/Processor]}
    C[Input Interface] {-{-}{} B}
    B {-{-}{} D[Output Interface]}
    B {-{-}{} E[Memory]}
    F[Programming Device] {-{-}{} B}
    G[Communication Interface] {-{-}{} B}
    style A fill:\#ffffb3
    style B fill:\#ffcccc
    style C fill:\#b3e0ff
    style D fill:\#ccffcc
    style E fill:\#e6ccff
    style F fill:\#ffddbb
    style G fill:\#d9ffb3
{Highlighting}
{Shaded}
\end{verbatim}
\end{center}

\textbf{Functions of Each Block:}

{\def\LTcaptype{none} % do not increment counter
\begin{longtable}[]{@{}
  >{\raggedright\arraybackslash}p{(\linewidth - 4\tabcolsep) * \real{0.2059}}
  >{\raggedright\arraybackslash}p{(\linewidth - 4\tabcolsep) * \real{0.2941}}
  >{\raggedright\arraybackslash}p{(\linewidth - 4\tabcolsep) * \real{0.5000}}@{}}
\toprule\noalign{}
\begin{minipage}[b]{\linewidth}\raggedright
Block
\end{minipage} & \begin{minipage}[b]{\linewidth}\raggedright
Function
\end{minipage} & \begin{minipage}[b]{\linewidth}\raggedright
Characteristics
\end{minipage} \\
\midrule\noalign{}
\endhead
\bottomrule\noalign{}
\endlastfoot
Power Supply & Converts main power to system voltages & Regulated,
protected, with isolation \\
CPU/Processor & Executes program, controls operations & Speed measured
in scan time (ms) \\
Input Interface & Connects to sensors and switches & Digital/analog,
isolation, filtering \\
Output Interface & Connects to actuators and indicators &
Relay/transistor/triac outputs \\
Memory & Stores program and data & Program, data, and system memory
areas \\
Programming Device & Used to develop and load programs & PC, handheld
programmer, software \\
Communication & Connects to networks/other devices & Industrial
protocols, remote I/O \\
\end{longtable}
}

\begin{itemize}
\tightlist
\item
  \textbf{Scan Cycle}: Sequential process of reading inputs, executing
  program, updating outputs
\item
  \textbf{Programming Languages}: Ladder Diagram (LD), Function Block
  Diagram (FBD), Structured Text (ST), Instruction List (IL), Sequential
  Function Chart (SFC)
\item
  \textbf{Modularity}: Expandable with additional I/O modules
\item
  \textbf{Robustness}: Designed for harsh industrial environments
\item
  \textbf{Reliability}: Typically MTBF \textgreater100,000 hours
\end{itemize}

\end{solutionbox}
\begin{mnemonicbox}
``PICO MPC'' - ``Power, Inputs, CPU, Outputs, Memory,
Programming interface, Communication''

\end{mnemonicbox}

\end{document}
