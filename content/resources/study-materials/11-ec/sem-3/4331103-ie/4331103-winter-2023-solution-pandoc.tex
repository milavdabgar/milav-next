\documentclass[10pt,a4paper]{article}

% content/resources/templates/preamble.tex
\usepackage[margin=0.6in]{geometry}
\author{Milav Dabgar}
\usepackage{amsmath,amssymb,amsthm}
\usepackage{booktabs}
\usepackage{multirow}
\usepackage{xcolor}
\usepackage{tcolorbox}
\tcbuselibrary{breakable,skins}
\usepackage[colorlinks=true,linkcolor=blue]{hyperref}
\usepackage{titlesec}
\usepackage{enumitem}
\usepackage{tikz}
\usepackage{pgfplots}
\usepackage{circuitikz}
\usepackage[version=4]{mhchem}
\usepackage{longtable}
\usepackage{array}
\usepackage{float}
\usepackage{caption}
\usepackage{listings}

\lstset{
  basicstyle=\small\ttfamily,
  breaklines=true,
  breakatwhitespace=false,
  postbreak=\mbox{\textcolor{red}{$\hookrightarrow$}\space},
  float=false,
  numbers=left,
  numberstyle=\tiny\color{gray},
  numbersep=10pt,
  xleftmargin=2em,
  keywordstyle=\color{blue},
  commentstyle=\color{green!60!black},
  stringstyle=\color{purple},
  backgroundcolor=\color{gray!5},
  showstringspaces=false,
  tabsize=2,
  captionpos=b,
  keepspaces=true,
  columns=flexible
}

\pgfplotsset{compat=1.18}
\usetikzlibrary{shapes,arrows,positioning,calc,patterns,decorations.pathmorphing,decorations.markings,arrows.meta}

% Color scheme
\definecolor{headcolor}{RGB}{0,102,204}
\definecolor{keycolor}{RGB}{220,20,60}
\definecolor{solutioncolor}{RGB}{34,139,34}
\definecolor{mnemoniccolor}{RGB}{148,0,211}
\definecolor{codecolor}{RGB}{0,0,100}

% Spacing
\setlength{\parskip}{3pt}
\setlist[itemize]{nosep}
\setlist[enumerate]{nosep}

% Title formatting
\titleformat{\section}{\Large\bfseries\color{headcolor}}{\thesection}{1em}{}
\titleformat{\subsection}{\large\bfseries\color{headcolor}}{\thesubsection}{1em}{}

% Pandoc tightlist compatibility
\providecommand{\tightlist}{%
  \setlength{\itemsep}{0pt}\setlength{\parskip}{0pt}}

% Pandoc longtable compatibility
\newcounter{none}
\def\thenone{}


% content/resources/templates/english-boxes.tex
% This file is currently empty - it exists to maintain consistency with the import structure.
% Add custom environments here if needed in the future.


\begin{document}

\begin{center}
{\Huge\bfseries\color{headcolor} Subject Name Solutions}\\[5pt]
{\LARGE 4331103 -- Winter 2023}\\[3pt]
{\large Semester 1 Study Material}\\[3pt]
{\normalsize\textit{Detailed Solutions and Explanations}}
\end{center}

\vspace{10pt}

\subsection*{Question 1(a) [3 marks]}\label{q1a}

\textbf{Draw symbol and construction of SCR. Also write down
applications of SCR.}

\begin{solutionbox}

\textbf{Symbol and Construction of SCR:}

\begin{verbatim}
    Anode (A)
       |
       v
       \_
      | |
      | |
  G {-|\_|  }
      | |
      |\_|
       |
       v
   Cathode (K)
\end{verbatim}

\textbf{Construction:}

\begin{center}
\textbf{Mermaid Diagram (Code)}
\begin{verbatim}
{Shaded}
{Highlighting}[]
graph LR
    A[Anode] {-{-}{-} P1[P{-}Layer]}
    P1 {-{-}{-} N1[N{-}Layer]}
    N1 {-{-}{-} P2[P{-}Layer]}
    P2 {-{-}{-} N2[N{-}Layer]}
    N2 {-{-}{-} K[Cathode]}
    G[Gate] {-{-}{-} P2}
{Highlighting}
{Shaded}
\end{verbatim}
\end{center}

\textbf{Applications of SCR:}

\begin{itemize}
\tightlist
\item
  \textbf{Power control}: AC/DC power regulators
\item
  \textbf{Motor drives}: Speed control of motors
\item
  \textbf{Lighting control}: Dimmer circuits
\item
  \textbf{Inverters}: DC to AC conversion
\end{itemize}

\end{solutionbox}
\begin{mnemonicbox}
``PALS'' - Power control, Appliance control, Lighting
systems, Speed regulators

\end{mnemonicbox}
\subsection*{Question 1(b) [4 marks]}\label{q1b}

\textbf{State full form of (i) SCS (ii) LASCR (iii) MCT (iv) PUT.}

\begin{solutionbox}

{\def\LTcaptype{none} % do not increment counter
\begin{longtable}[]{@{}ll@{}}
\toprule\noalign{}
Device & Full Form \\
\midrule\noalign{}
\endhead
\bottomrule\noalign{}
\endlastfoot
\textbf{SCS} & Silicon Controlled Switch \\
\textbf{LASCR} & Light Activated Silicon Controlled Rectifier \\
\textbf{MCT} & MOS Controlled Thyristor \\
\textbf{PUT} & Programmable Unijunction Transistor \\
\end{longtable}
}

\end{solutionbox}
\begin{mnemonicbox}
``SLaMP'' - Silicon controlled switch, Light
activated SCR, MOS controlled thyristor, Programmable UJT

\end{mnemonicbox}
\subsection*{Question 1(c) [7 marks]}\label{q1c}

\textbf{Draw and explain V-I characteristics of TRIAC. Also write down
applications of TRIAC.}

\begin{solutionbox}

\textbf{V-I Characteristics of TRIAC:}

\begin{center}
\textbf{Mermaid Diagram (Code)}
\begin{verbatim}
{Shaded}
{Highlighting}[]
graph LR
    subgraph "V{-I Characteristics"}
    style V{-I fill:\#f9f9f9,stroke:\#333,stroke{-}width:1px}

    MT2((MT2)) {-{-}{-} O[O]}
    O {-{-}{-} MT1((MT1))}
    
    V1[V] {-{-}{-} I1[I]}
    
    G[Gate Triggering]
    
    quad1[I quadrant] {-{-}{-} quad3[III quadrant]}
    breakover1[Breakover voltage +Vbo] {-{-}{-} breakover2[Breakover voltage {-}Vbo]}
    
    holding1[Holding current +Ih] {-{-}{-} holding2[Holding current {-}Ih]}
    end
{Highlighting}
{Shaded}
\end{verbatim}
\end{center}

\textbf{TRIAC V-I characteristics explanation:}

\begin{itemize}
\tightlist
\item
  \textbf{Bidirectional device}: Conducts in both directions
\item
  \textbf{Quadrant operation}: Works in 1st and 3rd quadrants
\item
  \textbf{Breakover voltage}: Starts conducting when voltage exceeds
  \pmVbo
\item
  \textbf{Holding current}: Minimum current to maintain conduction state
\item
  \textbf{Gate triggering}: Can be triggered with positive/negative gate
  voltage
\end{itemize}

\textbf{Applications of TRIAC:}

\begin{itemize}
\tightlist
\item
  \textbf{AC power control}: Lamp dimmers, heater controls
\item
  \textbf{Motor speed control}: AC motor regulators
\item
  \textbf{Fan regulators}: Domestic fan speed control
\item
  \textbf{Light dimmers}: Adjustable lighting systems
\end{itemize}

\end{solutionbox}
\begin{mnemonicbox}
``HALF'' - Heaters, AC controls, Lighting systems,
Fan regulators

\end{mnemonicbox}
\subsection*{Question 1(c) OR [7
marks]}\label{q1c}

\textbf{Explain construction and working of IGBT in detail.}

\begin{solutionbox}

\textbf{IGBT Construction and Working:}

\begin{center}
\textbf{Mermaid Diagram (Code)}
\begin{verbatim}
{Shaded}
{Highlighting}[]
graph LR
    G[Gate] {-{-}{-} E[Emitter]}
    E {-{-}{-} N+[N+ Layer]}
    N+ {-{-}{-} P[P Layer]}
    P {-{-}{-} N{-}[N{-} Drift Region]}
    N{- {-}{-}{-} N+B[N+ Buffer Layer]}
    N+B {-{-}{-} C[Collector]}
{Highlighting}
{Shaded}
\end{verbatim}
\end{center}

\textbf{Construction details:}

\begin{itemize}
\tightlist
\item
  \textbf{Three-terminal device}: Gate, Emitter, Collector
\item
  \textbf{Multilayer structure}: N+, P, N-, N+ buffer, P+ substrate
\item
  \textbf{Hybrid device}: Combines MOSFET input with BJT output
  characteristics
\end{itemize}

\textbf{Working principle:}

\begin{itemize}
\tightlist
\item
  \textbf{Gate control}: Positive voltage at gate forms inversion layer
  in P-region
\item
  \textbf{Channel formation}: Electrons flow from N+ emitter to N- drift
  region
\item
  \textbf{Conductivity modulation}: P-N- junction injects holes,
  lowering resistance
\item
  \textbf{Turn-off process}: Removing gate voltage stops electron flow
\end{itemize}

\textbf{Advantages of IGBT:}

\begin{itemize}
\tightlist
\item
  \textbf{High input impedance}: Easy voltage control
\item
  \textbf{Low conduction losses}: Efficient power handling
\item
  \textbf{Fast switching}: Good for high-frequency applications
\end{itemize}

\end{solutionbox}
\begin{mnemonicbox}
``GIVE'' - Gate controlled, Input high impedance,
Voltage driven, Efficient conduction

\end{mnemonicbox}
\subsection*{Question 2(a) [3 marks]}\label{q2a}

\textbf{Discuss relaxation oscillator circuit using UJT.}

\begin{solutionbox}

\textbf{UJT Relaxation Oscillator:}

\begin{center}
\textbf{Mermaid Diagram (Code)}
\begin{verbatim}
{Shaded}
{Highlighting}[]
graph LR
    VCC[VCC] {-{-}{-} R1[R1] {-}{-}{-} E[Emitter]}
    E {-{-}{-} C[Capacitor] {-}{-}{-} GND[GND]}
    E {-{-}{-} UJT[UJT]}
    UJT {-{-}{-} B1[Base 1] {-}{-}{-} R2[R2] {-}{-}{-} GND}
    UJT {-{-}{-} B2[Base 2] {-}{-}{-} R3[R3] {-}{-}{-} VCC}
    B1 {-{-}{-} Output[Output]}
{Highlighting}
{Shaded}
\end{verbatim}
\end{center}

\textbf{Working principle:}

\begin{itemize}
\tightlist
\item
  \textbf{Capacitor charging}: C charges through R1 until reaching UJT
  firing voltage
\item
  \textbf{UJT fires}: When emitter voltage reaches peak point voltage
\item
  \textbf{Discharge cycle}: Capacitor discharges through emitter-base1
  junction
\item
  \textbf{Oscillation}: Process repeats creating sawtooth waveform
\end{itemize}

\end{solutionbox}
\begin{mnemonicbox}
``CROP'' - Capacitor charges, Reaches threshold,
Oscillates, Produces sawtooth

\end{mnemonicbox}
\subsection*{Question 2(b) [4 marks]}\label{q2b}

\textbf{Discuss the triggering methods of SCR.}

\begin{solutionbox}

{\def\LTcaptype{none} % do not increment counter
\begin{longtable}[]{@{}
  >{\raggedright\arraybackslash}p{(\linewidth - 2\tabcolsep) * \real{0.5000}}
  >{\raggedright\arraybackslash}p{(\linewidth - 2\tabcolsep) * \real{0.5000}}@{}}
\toprule\noalign{}
\begin{minipage}[b]{\linewidth}\raggedright
Triggering Method
\end{minipage} & \begin{minipage}[b]{\linewidth}\raggedright
Working Principle
\end{minipage} \\
\midrule\noalign{}
\endhead
\bottomrule\noalign{}
\endlastfoot
\textbf{Gate Triggering} & Applying positive voltage between gate and
cathode \\
\textbf{Thermal Triggering} & Temperature increase reduces breakover
voltage \\
\textbf{Light Triggering} & Photons create electron-hole pairs in
LASCR \\
\textbf{dv/dt Triggering} & Rapid voltage rise across SCR causes
capacitive current \\
\textbf{Breakover Triggering} & Voltage exceeds breakover voltage
without gate signal \\
\end{longtable}
}

\textbf{Key points:}

\begin{itemize}
\tightlist
\item
  \textbf{Gate triggering}: Most common method
\item
  \textbf{Light triggering}: Used in opto-isolators
\item
  \textbf{dv/dt triggering}: Often undesirable, requiring snubber
  circuits
\end{itemize}

\end{solutionbox}
\begin{mnemonicbox}
``GLTDB'' - Gate, Light, Thermal, dv/dt, Breakover

\end{mnemonicbox}
\subsection*{Question 2(c) [7 marks]}\label{q2c}

\textbf{Explain class A type commutation method.}

\begin{solutionbox}

\textbf{Class A Commutation (Self-commutation by LC circuit):}

\begin{center}
\textbf{Mermaid Diagram (Code)}
\begin{verbatim}
{Shaded}
{Highlighting}[]
graph LR
    DC\_Source[DC Source] {-{-}{-} SCR[SCR] {-}{-}{-} Load[Load]}
    SCR {-{-}{-} L[Inductor] {-}{-}{-} C[Capacitor]}
    C {-{-}{-} SW[Switch] {-}{-}{-} DC\_Source}
{Highlighting}
{Shaded}
\end{verbatim}
\end{center}

\textbf{Working principle:}

\begin{itemize}
\tightlist
\item
  \textbf{Initial state}: SCR conducting, capacitor charged with
  polarity (+) on right
\item
  \textbf{Commutation start}: When switch SW closed
\item
  \textbf{Resonant circuit}: LC circuit forms resonant path
\item
  \textbf{Reverse current}: Capacitor discharge creates reverse current
  through SCR
\item
  \textbf{Turn-off}: SCR turns off when current falls below holding
  current
\item
  \textbf{Recharging}: Capacitor recharges with opposite polarity
\end{itemize}

\textbf{Applications:}

\begin{itemize}
\tightlist
\item
  \textbf{Inverter circuits}: DC to AC conversion
\item
  \textbf{Chopper circuits}: DC to DC conversion
\end{itemize}

\end{solutionbox}
\begin{mnemonicbox}
``SCCRRT'' - Switch closes, Capacitor discharges,
Current reverses, SCR turns off, Recharging begins, Turn-off complete

\end{mnemonicbox}
\subsection*{Question 2(a) OR [3
marks]}\label{q2a}

\textbf{State full form of GTO and draw the structure of GTO.}

\begin{solutionbox}

\textbf{Full form of GTO:} Gate Turn-Off Thyristor

\textbf{Structure of GTO:}

\begin{center}
\textbf{Mermaid Diagram (Code)}
\begin{verbatim}
{Shaded}
{Highlighting}[]
graph LR
    A[Anode] {-{-}{-} P1[P+ Anode Layer]}
    P1 {-{-}{-} N[N Base Layer]}
    N {-{-}{-} P2[P Base Layer]}
    P2 {-{-}{-} N2[N+ Cathode Layer]}
    N2 {-{-}{-} K[Cathode]}
    G[Gate] {-{-}{-} P2}
{Highlighting}
{Shaded}
\end{verbatim}
\end{center}

\end{solutionbox}
\begin{mnemonicbox}
``PANG'' - P-anode, And, N-base, Gate-controlled
thyristor

\end{mnemonicbox}
\subsection*{Question 2(b) OR [4
marks]}\label{q2b}

\textbf{Discuss the design and requirement of snubber circuit for SCR.}

\begin{solutionbox}

\textbf{Snubber Circuit for SCR:}

\begin{center}
\textbf{Mermaid Diagram (Code)}
\begin{verbatim}
{Shaded}
{Highlighting}[]
graph LR
    SCR[SCR] {-{-}{-} R[Resistor] {-}{-}{-} C[Capacitor]}
    C {-{-}{-} SCR}
{Highlighting}
{Shaded}
\end{verbatim}
\end{center}

\textbf{Design requirements:}

\begin{itemize}
\tightlist
\item
  \textbf{Resistor selection}: Limits discharge current of capacitor
\item
  \textbf{Capacitor selection}: Controls rate of voltage rise (dv/dt)
\item
  \textbf{RC time constant}: Determines response time
\end{itemize}

\textbf{Purpose of snubber circuit:}

\begin{itemize}
\tightlist
\item
  \textbf{dv/dt protection}: Prevents false triggering due to rapid
  voltage changes
\item
  \textbf{Voltage spike suppression}: Absorbs inductive voltage spikes
\item
  \textbf{Transient protection}: Protects SCR during switching
\end{itemize}

\end{solutionbox}
\begin{mnemonicbox}
``RAPE'' - Resistor And capacitor Protect against
Excessive voltage rise

\end{mnemonicbox}
\subsection*{Question 2(c) OR [7
marks]}\label{q2c}

\textbf{Explain class C type commutation method.}

\begin{solutionbox}

\textbf{Class C Commutation (Complementary commutation):}

\begin{center}
\textbf{Mermaid Diagram (Code)}
\begin{verbatim}
{Shaded}
{Highlighting}[]
graph LR
    DC\_Source[DC Source] {-{-}{-} SCR1[SCR1] {-}{-}{-} Load1[Load 1]}
    DC\_Source {-{-}{-} SCR2[SCR2] {-}{-}{-} Load2[Load 2]}
    SCR1 {-{-}{-} SCR2}
{Highlighting}
{Shaded}
\end{verbatim}
\end{center}

\textbf{Working principle:}

\begin{itemize}
\tightlist
\item
  \textbf{Initial state}: SCR1 conducting, SCR2 off
\item
  \textbf{Commutation start}: SCR2 is triggered
\item
  \textbf{Load transfer}: Current transfers from SCR1 to SCR2
\item
  \textbf{Voltage reversal}: Voltage across SCR1 becomes negative
\item
  \textbf{Turn-off}: SCR1 turns off as current falls below holding
  current
\item
  \textbf{Alternating operation}: SCR1 and SCR2 conduct alternatively
\end{itemize}

\textbf{Applications:}

\begin{itemize}
\tightlist
\item
  \textbf{Inverter circuits}: Used in bridge inverters
\item
  \textbf{Dual load systems}: Where alternate operation is required
\end{itemize}

\end{solutionbox}
\begin{mnemonicbox}
``TACTOR'' - Triggering Alternate SCRs Creates
Turn-Off and Reversal

\end{mnemonicbox}
\subsection*{Question 3(a) [3 marks]}\label{q3a}

\textbf{State the advantages of poly-phase rectifier over single phase
rectifier.}

\begin{solutionbox}

{\def\LTcaptype{none} % do not increment counter
\begin{longtable}[]{@{}
  >{\raggedright\arraybackslash}p{(\linewidth - 2\tabcolsep) * \real{0.4583}}
  >{\raggedright\arraybackslash}p{(\linewidth - 2\tabcolsep) * \real{0.5417}}@{}}
\toprule\noalign{}
\begin{minipage}[b]{\linewidth}\raggedright
Advantage
\end{minipage} & \begin{minipage}[b]{\linewidth}\raggedright
Description
\end{minipage} \\
\midrule\noalign{}
\endhead
\bottomrule\noalign{}
\endlastfoot
\textbf{Higher efficiency} & Lower power loss and better transformer
utilization \\
\textbf{Lower ripple factor} & Smoother DC output requiring smaller
filter components \\
\textbf{Higher power handling} & Can handle higher power levels than
single phase \\
\textbf{Better transformer utilization} & Higher transformer utilization
factor \\
\textbf{Lower harmonic content} & Reduced harmonic distortion in
output \\
\end{longtable}
}

\end{solutionbox}
\begin{mnemonicbox}
``HELPS'' - Higher efficiency, Even output, Lower
ripple, Power handling better, Smaller filter

\end{mnemonicbox}
\subsection*{Question 3(b) [4 marks]}\label{q3b}

\textbf{Draw and explain the circuit of single phase Half Wave
rectifier. Draw the waveforms.}

\begin{solutionbox}

\textbf{Single Phase Half Wave Rectifier:}

\begin{center}
\textbf{Mermaid Diagram (Code)}
\begin{verbatim}
{Shaded}
{Highlighting}[]
graph LR
    AC[AC Supply] {-{-}{-} D[Diode] {-}{-}{-} R[Load Resistor]}
    R {-{-}{-} AC}
{Highlighting}
{Shaded}
\end{verbatim}
\end{center}

\textbf{Waveform:}

\begin{verbatim}
    Voltage
      \^{}
      |     /{      /      /}
      |    /  {    /      /  }
      |{-{-}{-}/{-}{-}{-}{-}{-}{-}/{-}{-}{-}{-}{-}{-}/{-}{-}{-}{-}{-}{-}{-}{-} Time}
      |         {              }
      |          {              }
      |
   Input AC
   
    Voltage
      \^{}
      |     /{      /      /}
      |    /  {    /      /  }
      |{-{-}{-}/{-}{-}{-}{-}{-}{-}/{-}{-}{-}{-}{-}{-}/{-}{-}{-}{-}{-}{-}{-}{-} Time}
      |    
      |    
      |
   Output DC (Pulsating)
\end{verbatim}

\textbf{Working principle:}

\begin{itemize}
\tightlist
\item
  \textbf{Forward bias}: Diode conducts during positive half-cycle
\item
  \textbf{Reverse bias}: Diode blocks current during negative half-cycle
\item
  \textbf{Output}: Pulsating DC with high ripple factor
\item
  \textbf{Frequency}: Output frequency same as input frequency
\end{itemize}

\end{solutionbox}
\begin{mnemonicbox}
``PROF'' - Positive half conducts, Reverse half
blocks, Output is pulsating, Frequency unchanged

\end{mnemonicbox}
\subsection*{Question 3(c) [7 marks]}\label{q3c}

\textbf{List all types of Inverters. Out of that explain single phase
full bridge Inverter.}

\begin{solutionbox}

\textbf{Types of Inverters:}

\begin{enumerate}
\tightlist
\item
  Based on circuit: Series, Parallel, Bridge
\item
  Based on phases: Single-phase, Three-phase
\item
  Based on output: Square wave, Modified sine wave, Pure sine wave
\item
  Based on commutation: SCR-based, Transistor-based
\end{enumerate}

\textbf{Single Phase Full Bridge Inverter:}

\begin{center}
\textbf{Mermaid Diagram (Code)}
\begin{verbatim}
{Shaded}
{Highlighting}[]
graph LR
    DC[DC Source] {-{-}{-} S1[Switch S1] {-}{-}{-} S2[Switch S2] {-}{-}{-} DC}
    S1 {-{-}{-} Load[Load] {-}{-}{-} S3[Switch S3]}
    S2 {-{-}{-} Load}
    S3 {-{-}{-} S4[Switch S4] {-}{-}{-} DC}
{Highlighting}
{Shaded}
\end{verbatim}
\end{center}

\textbf{Working principle:}

\begin{itemize}
\tightlist
\item
  \textbf{First half-cycle}: S1 and S4 ON, S2 and S3 OFF
\item
  \textbf{Second half-cycle}: S2 and S3 ON, S1 and S4 OFF
\item
  \textbf{Output waveform}: AC square wave across load
\item
  \textbf{Control method}: Gate signals to switches with 180^\circ phase
  shift
\end{itemize}

\textbf{Advantages:}

\begin{itemize}
\tightlist
\item
  \textbf{Higher output power}: Twice the output of half bridge
\item
  \textbf{Better voltage utilization}: Full DC bus voltage across load
\item
  \textbf{Lower current rating}: Each switch carries only load current
\end{itemize}

\end{solutionbox}
\begin{mnemonicbox}
``SOAP'' - Switches Operate Alternately in Pairs

\end{mnemonicbox}
\subsection*{Question 3(a) OR [3
marks]}\label{q3a}

\textbf{Compare UPS and SMPS.}

\begin{solutionbox}

{\def\LTcaptype{none} % do not increment counter
\begin{longtable}[]{@{}
  >{\raggedright\arraybackslash}p{(\linewidth - 4\tabcolsep) * \real{0.1325}}
  >{\raggedright\arraybackslash}p{(\linewidth - 4\tabcolsep) * \real{0.4458}}
  >{\raggedright\arraybackslash}p{(\linewidth - 4\tabcolsep) * \real{0.4217}}@{}}
\toprule\noalign{}
\begin{minipage}[b]{\linewidth}\raggedright
Parameter
\end{minipage} & \begin{minipage}[b]{\linewidth}\raggedright
UPS (Uninterruptible Power Supply)
\end{minipage} & \begin{minipage}[b]{\linewidth}\raggedright
SMPS (Switched Mode Power Supply)
\end{minipage} \\
\midrule\noalign{}
\endhead
\bottomrule\noalign{}
\endlastfoot
\textbf{Primary function} & Provides backup power during outages &
Converts AC to regulated DC \\
\textbf{Battery backup} & Contains batteries for backup & No battery
backup \\
\textbf{Output} & AC output (typically) & DC output (typically) \\
\textbf{Efficiency} & Lower (70-80\%) & Higher (80-95\%) \\
\textbf{Size} & Larger and heavier & Compact and lightweight \\
\textbf{Applications} & Computers, servers, critical equipment &
Electronic devices, chargers \\
\end{longtable}
}

\end{solutionbox}
\begin{mnemonicbox}
``BBOSS'' - Backup Battery Only in UPS, Small Size in
SMPS

\end{mnemonicbox}
\subsection*{Question 3(b) OR [4
marks]}\label{q3b}

\textbf{Draw and explain the circuit of three phase Half Wave rectifier.
Draw the waveforms.}

\begin{solutionbox}

\textbf{Three Phase Half Wave Rectifier:}

\begin{center}
\textbf{Mermaid Diagram (Code)}
\begin{verbatim}
{Shaded}
{Highlighting}[]
graph LR
    R[R Phase] {-{-}{-} D1[Diode D1] {-}{-}{-} Load[Load]}
    Y[Y Phase] {-{-}{-} D2[Diode D2] {-}{-}{-} Load}
    B[B Phase] {-{-}{-} D3[Diode D3] {-}{-}{-} Load}
    Load {-{-}{-} N[Neutral]}
{Highlighting}
{Shaded}
\end{verbatim}
\end{center}

\textbf{Waveform:}

\begin{verbatim}
     Voltage
       \^{}
       |   
       |    /{    /    /    /    /    /}
       |   /  {  /    /    /    /    /  }
       |{-{-}/{-}{-}{-}{-}/{-}{-}{-}{-}/{-}{-}{-}{-}/{-}{-}{-}{-}/{-}{-}{-}{-}/{-}{-}{-}{-}{-}{-} Time}
       |   R    Y    B    R    Y    B    R
       |   
    Input (Three phase)
    
     Voltage
       \^{}
       |   
       |    /{    /    /    /    /    /}
       |   /  {  /    /    /    /    /  }
       |{-{-}/{-}{-}{-}{-}/{-}{-}{-}{-}/{-}{-}{-}{-}/{-}{-}{-}{-}/{-}{-}{-}{-}/{-}{-}{-}{-}{-}{-} Time}
       |      
       |   
    Output DC (Less ripple)
\end{verbatim}

\textbf{Working principle:}

\begin{itemize}
\tightlist
\item
  \textbf{Conduction sequence}: Each diode conducts when its phase
  voltage is highest
\item
  \textbf{Conduction angle}: Each diode conducts for 120^\circ
\item
  \textbf{Output ripple}: 3 pulses per cycle, lower ripple than single
  phase
\item
  \textbf{Ripple frequency}: 3 times the input frequency
\end{itemize}

\end{solutionbox}
\begin{mnemonicbox}
``CROP'' - Conduction of 120^\circ, Ripple reduced, Output
smoother, Pulses tripled

\end{mnemonicbox}
\subsection*{Question 3(c) OR [7
marks]}\label{q3c}

\textbf{Define chopper. With the help of circuit diagram explain class D
chopper.}

\begin{solutionbox}

\textbf{Definition of Chopper:} A chopper is a DC to DC converter that
converts fixed DC input voltage to variable DC output voltage using
high-frequency switching.

\textbf{Class D Chopper (Two-quadrant chopper):}

\begin{center}
\textbf{Mermaid Diagram (Code)}
\begin{verbatim}
{Shaded}
{Highlighting}[]
graph LR
    VS[DC Source] {-{-}{-} S1[Switch S1] {-}{-}{-} L[Inductor]}
    L {-{-}{-} Load[Load] {-}{-}{-} VS}
    Load {-{-}{-} D1[Diode D1] {-}{-}{-} S1}
    Load {-{-}{-} S2[Switch S2] {-}{-}{-} D2[Diode D2] {-}{-}{-} VS}
{Highlighting}
{Shaded}
\end{verbatim}
\end{center}

\textbf{Working principle:}

\begin{itemize}
\tightlist
\item
  \textbf{First quadrant operation (forward motoring):}

  \begin{itemize}
  \tightlist
  \item
    S1 ON, S2 OFF: Energy flows from source to load
  \item
    S1 OFF, S2 OFF: Current freewheels through D2
  \end{itemize}
\item
  \textbf{Second quadrant operation (forward regeneration):}

  \begin{itemize}
  \tightlist
  \item
    S1 OFF, S2 ON: Energy flows from load to source
  \item
    S1 OFF, S2 OFF: Current freewheels through D1
  \end{itemize}
\end{itemize}

\textbf{Applications:}

\begin{itemize}
\tightlist
\item
  \textbf{DC motor drives}: Providing forward motoring and regenerative
  braking
\item
  \textbf{Battery charging}: Controlling charging current
\item
  \textbf{Renewable energy}: Interfacing with solar panels
\end{itemize}

\end{solutionbox}
\begin{mnemonicbox}
``FRED'' - Forward motoring, Regenerative braking,
Energy flow control, Dual quadrant operation

\end{mnemonicbox}
\subsection*{Question 4(a) [3 marks]}\label{q4a}

\textbf{Describe the use of SCR as a static switch.}

\begin{solutionbox}

\textbf{SCR as Static Switch:}

\begin{center}
\textbf{Mermaid Diagram (Code)}
\begin{verbatim}
{Shaded}
{Highlighting}[]
graph LR
    VS[Supply] {-{-}{-} SCR[SCR] {-}{-}{-} Load[Load]}
    GC[Gate Control] {-{-}{-} SCR}
{Highlighting}
{Shaded}
\end{verbatim}
\end{center}

\textbf{Key features:}

\begin{itemize}
\tightlist
\item
  \textbf{No moving parts}: Purely electronic switching
\item
  \textbf{Fast switching}: Microsecond response time
\item
  \textbf{High reliability}: Longer lifetime than mechanical switches
\item
  \textbf{Controlled turn-on}: Precise control via gate signal
\end{itemize}

\textbf{Advantages over mechanical switches:}

\begin{itemize}
\tightlist
\item
  \textbf{No arcing}: No contact bounce or wear
\item
  \textbf{Silent operation}: No mechanical noise
\item
  \textbf{EMI reduction}: Less electromagnetic interference
\end{itemize}

\end{solutionbox}
\begin{mnemonicbox}
``FANS'' - Fast switching, Arc-free operation, No
mechanical wear, Silent operation

\end{mnemonicbox}
\subsection*{Question 4(b) [4 marks]}\label{q4b}

\textbf{Draw the circuit diagram of A.C. Power control using DIAC and
TRIAC and explain its working.}

\begin{solutionbox}

\textbf{AC Power Control using DIAC and TRIAC:}

\begin{center}
\textbf{Mermaid Diagram (Code)}
\begin{verbatim}
{Shaded}
{Highlighting}[]
graph LR
    AC[AC Supply] {-{-}{-} TRIAC[TRIAC] {-}{-}{-} Load[Load]}
    AC {-{-}{-} R[Resistor] {-}{-}{-} C[Capacitor] {-}{-}{-} DIAC[DIAC] {-}{-}{-} G[TRIAC Gate]}
    G {-{-}{-} TRIAC}
{Highlighting}
{Shaded}
\end{verbatim}
\end{center}

\textbf{Working principle:}

\begin{itemize}
\tightlist
\item
  \textbf{RC network}: Controls firing angle by delaying gate pulse
\item
  \textbf{Capacitor charging}: C charges through R during each
  half-cycle
\item
  \textbf{DIAC breakdown}: When capacitor voltage reaches DIAC breakover
  voltage
\item
  \textbf{TRIAC triggering}: DIAC conducts and triggers TRIAC
\item
  \textbf{Power control}: Varying R changes firing angle and thus power
  delivered
\end{itemize}

\textbf{Applications:}

\begin{itemize}
\tightlist
\item
  \textbf{Light dimmers}: Controlling brightness of lamps
\item
  \textbf{Fan speed control}: Regulating fan speed
\item
  \textbf{Heater control}: Adjusting heating elements
\end{itemize}

\end{solutionbox}
\begin{mnemonicbox}
``CRAFT'' - Capacitor charges, Reaches breakover,
Activates DIAC, Fires TRIAC, Transfers power

\end{mnemonicbox}
\subsection*{Question 4(c) [7 marks]}\label{q4c}

\textbf{Explain the working principle of induction heating also write
the applications of induction heating.}

\begin{solutionbox}

\textbf{Working Principle of Induction Heating:}

\begin{center}
\textbf{Mermaid Diagram (Code)}
\begin{verbatim}
{Shaded}
{Highlighting}[]
graph LR
    Power[AC Power Supply] {-{-}{-} Inv[High Frequency Inverter]}
    Inv {-{-}{-} Coil[Induction Coil]}
    Coil {-{-}{-} Workpiece[Metal Workpiece]}

    subgraph "Physical Process"
    Coil {-.{-} Magnetic[Alternating Magnetic Field]}
    Magnetic {-.{-} Eddy[Eddy Currents]}
    Eddy {-.{-} Heat[Heat Generation]}
    end
{Highlighting}
{Shaded}
\end{verbatim}
\end{center}

\textbf{Working principle:}

\begin{itemize}
\tightlist
\item
  \textbf{High-frequency current}: Passes through induction coil
\item
  \textbf{Electromagnetic induction}: Creates alternating magnetic field
\item
  \textbf{Eddy currents}: Induced in workpiece
\item
  \textbf{Resistance heating}: Eddy currents generate heat due to
  resistance
\item
  \textbf{Skin effect}: Heat concentrated near surface
\item
  \textbf{Non-contact heating}: No physical contact between coil and
  workpiece
\end{itemize}

\textbf{Applications of Induction Heating:}

\begin{itemize}
\tightlist
\item
  \textbf{Metal heat treatment}: Hardening, annealing, tempering
\item
  \textbf{Metal melting}: Foundry operations
\item
  \textbf{Welding and brazing}: Joining metal components
\item
  \textbf{Forging}: Heating before forming
\item
  \textbf{Domestic cooking}: Induction cooktops
\item
  \textbf{Semiconductor processing}: Crystal growth
\end{itemize}

\end{solutionbox}
\begin{mnemonicbox}
``MASTER'' - Magnetic field, Alternating current,
Surface heating, Temperature control, Eddy currents, Resistance heating

\end{mnemonicbox}
\subsection*{Question 4(a) OR [3
marks]}\label{q4a}

\textbf{Explain working of photo relay circuit using LDR.}

\begin{solutionbox}

\textbf{Photo Relay Circuit using LDR:}

\begin{center}
\textbf{Mermaid Diagram (Code)}
\begin{verbatim}
{Shaded}
{Highlighting}[]
graph LR
    VS[Supply] {-{-}{-} R1[Resistor R1] {-}{-}{-} LDR[LDR]}
    LDR {-{-}{-} GND[Ground]}
    R1 {-{-}{-} B[Transistor Base]}
    VS {-{-}{-} RC[Collector Resistor] {-}{-}{-} C[Transistor Collector]}
    C {-{-}{-} Relay[Relay Coil] {-}{-}{-} GND}
    E[Transistor Emitter] {-{-}{-} GND}
{Highlighting}
{Shaded}
\end{verbatim}
\end{center}

\textbf{Working principle:}

\begin{itemize}
\tightlist
\item
  \textbf{Light-dependent resistor}: Resistance decreases with
  increasing light
\item
  \textbf{Voltage divider}: LDR and R1 form voltage divider
\item
  \textbf{Transistor switching}: Base voltage controls transistor
  conduction
\item
  \textbf{Relay operation}: Transistor drives relay coil
\item
  \textbf{Threshold adjustment}: Can be set using variable resistor
\end{itemize}

\textbf{Applications:}

\begin{itemize}
\tightlist
\item
  \textbf{Automatic street lighting}: Turns on lights at dusk
\item
  \textbf{Day/night switching}: Controls devices based on ambient light
\item
  \textbf{Security systems}: Light-activated alarms
\end{itemize}

\end{solutionbox}
\begin{mnemonicbox}
``LARK'' - Light controls, Activates transistor,
Relay switches, Keeps circuit automated

\end{mnemonicbox}
\subsection*{Question 4(b) OR [4
marks]}\label{q4b}

\textbf{Explain the operation of timer circuit using 555 timer IC.}

\begin{solutionbox}

\textbf{555 Timer Circuit (Monostable):}

\begin{center}
\textbf{Mermaid Diagram (Code)}
\begin{verbatim}
{Shaded}
{Highlighting}[]
graph TD
    VCC[+VCC] {-{-}{-} R[Resistor R] {-}{-}{-} D8[Pin 8 VCC]}
    D8 {-{-}{-} D4[Pin 4 Reset]}
    D8 {-{-}{-} D7[Pin 7 Discharge]}
    R {-{-}{-} D7}
    D7 {-{-}{-} C[Capacitor C] {-}{-}{-} GND[Ground]}
    Trigger[Trigger Input] {-{-}{-} D2[Pin 2 Trigger]}
    D3[Pin 3 Output] {-{-}{-} Output[Output]}
    D1[Pin 1 GND] {-{-}{-} GND}
    D5[Pin 5 Control] {-{-}{-} CC[Control Capacitor] {-}{-}{-} GND}
    D6[Pin 6 Threshold] {-{-}{-} D7}
{Highlighting}
{Shaded}
\end{verbatim}
\end{center}

\textbf{Working principle:}

\begin{itemize}
\tightlist
\item
  \textbf{Trigger input}: Active low trigger at pin 2
\item
  \textbf{Timing components}: R and C determine timing period (T =
  1.1RC)
\item
  \textbf{Output high}: When triggered, output goes high
\item
  \textbf{Capacitor charging}: C charges through R
\item
  \textbf{Threshold detection}: When voltage reaches 2/3 VCC, output
  goes low
\item
  \textbf{Timer reset}: Circuit can be reset using pin 4
\end{itemize}

\textbf{Applications:}

\begin{itemize}
\tightlist
\item
  \textbf{Delay circuits}: Creating time delays
\item
  \textbf{Pulse generation}: Generating precise pulses
\item
  \textbf{Timing control}: Sequential timing operations
\end{itemize}

\end{solutionbox}
\begin{mnemonicbox}
``TRACT'' - Trigger activates, Resistor-capacitor
timing, Accurate delay, Capacitor charges, Threshold detection

\end{mnemonicbox}
\subsection*{Question 4(c) OR [7
marks]}\label{q4c}

\textbf{Explain the working principle of dielectric heating also write
the applications of dielectric heating.}

\begin{solutionbox}

\textbf{Working Principle of Dielectric Heating:}

\begin{center}
\textbf{Mermaid Diagram (Code)}
\begin{verbatim}
{Shaded}
{Highlighting}[]
graph LR
    RF[RF Generator] {-{-}{-} Electrodes[Electrodes]}

    subgraph "Material Between Electrodes"
    Electrodes {-{-}{-} Electric[Alternating Electric Field]}
    Electric {-{-}{-} Dipoles[Molecular Dipoles]}
    Dipoles {-{-}{-} Oscillation[Dipole Oscillation]}
    Oscillation {-{-}{-} Friction[Molecular Friction]}
    Friction {-{-}{-} Heat[Heat Generation]}
    end
{Highlighting}
{Shaded}
\end{verbatim}
\end{center}

\textbf{Working principle:}

\begin{itemize}
\tightlist
\item
  \textbf{High-frequency electric field}: Applied between electrodes
\item
  \textbf{Dielectric material}: Placed between electrodes
\item
  \textbf{Molecular polarization}: Dipoles align with electric field
\item
  \textbf{Field oscillation}: Rapid reversal of field direction
\item
  \textbf{Molecular friction}: Dipoles rotate rapidly causing friction
\item
  \textbf{Volumetric heating}: Heat generated throughout material
\item
  \textbf{Frequency range}: Typically 10-100 MHz
\end{itemize}

\textbf{Applications of Dielectric Heating:}

\begin{itemize}
\tightlist
\item
  \textbf{Food processing}: Baking, drying, pasteurization
\item
  \textbf{Wood industry}: Gluing, drying timber
\item
  \textbf{Textile drying}: Removing moisture from fabrics
\item
  \textbf{Plastic welding}: Joining thermoplastics
\item
  \textbf{Medical applications}: Therapeutic diathermy
\item
  \textbf{Paper industry}: Drying paper products
\end{itemize}

\end{solutionbox}
\begin{mnemonicbox}
``DIPOLE'' - Dielectric material, Intense electric
field, Polarization of molecules, Oscillation causes, Linkage of heat,
Even heating throughout

\end{mnemonicbox}
\subsection*{Question 5(a) [3 marks]}\label{q5a}

\textbf{Define AC drive. State applications of AC drives.}

\begin{solutionbox}

\textbf{Definition of AC Drive:} An AC drive is an electronic device
that controls the speed, torque, and direction of an AC motor by varying
the frequency and voltage supplied to the motor.

\textbf{Applications of AC Drives:}

{\def\LTcaptype{none} % do not increment counter
\begin{longtable}[]{@{}ll@{}}
\toprule\noalign{}
Application Area & Examples \\
\midrule\noalign{}
\endhead
\bottomrule\noalign{}
\endlastfoot
\textbf{Industrial} & Conveyor systems, pumps, fans, compressors \\
\textbf{HVAC} & Blowers, cooling towers, air handling units \\
\textbf{Water treatment} & Pumps, mixers, aerators \\
\textbf{Mining} & Crushers, conveyors, pumps \\
\textbf{Textile} & Spinning machines, looms, winders \\
\textbf{Material handling} & Cranes, elevators, escalators \\
\end{longtable}
}

\end{solutionbox}
\begin{mnemonicbox}
``PITCHW'' - Pumps, Industrial machinery, Textile
machines, Conveyor systems, HVAC systems, Water treatment

\end{mnemonicbox}
\subsection*{Question 5(b) [4 marks]}\label{q5b}

\textbf{Draw and explain any one method for speed control of DC shunt
motor.}

\begin{solutionbox}

\textbf{Armature Voltage Control Method for DC Shunt Motor:}

\begin{center}
\textbf{Mermaid Diagram (Code)}
\begin{verbatim}
{Shaded}
{Highlighting}[]
graph LR
    AC[AC Supply] {-{-}{-} B[Bridge Rectifier]}
    B {-{-}{-} SCR[SCR] {-}{-}{-} A[Armature]}
    A {-{-}{-} B}
    AC {-{-}{-} F[Field Circuit]}
    F {-{-}{-} Field[Field Winding]}
    GC[Gate Control] {-{-}{-} SCR}
{Highlighting}
{Shaded}
\end{verbatim}
\end{center}

\textbf{Working principle:}

\begin{itemize}
\tightlist
\item
  \textbf{Constant field current}: Field supply maintained constant
\item
  \textbf{Variable armature voltage}: Controlled by SCR
\item
  \textbf{Speed equation}: N ∝ (V_{a} - I_{a}R_{a})/Φ
\item
  \textbf{Speed control}: By changing armature voltage V_{a}
\item
  \textbf{Torque control}: Armature current controls torque
\end{itemize}

\textbf{Advantages:}

\begin{itemize}
\tightlist
\item
  \textbf{Wide speed range}: Can achieve speeds below and above base
  speed
\item
  \textbf{Smooth control}: Continuous speed adjustment
\item
  \textbf{High efficiency}: Low power loss in control circuit
\end{itemize}

\end{solutionbox}
\begin{mnemonicbox}
``SAVE'' - SCR controls, Armature voltage varies,
Velocity changes, Efficient operation

\end{mnemonicbox}
\subsection*{Question 5(c) [7 marks]}\label{q5c}

\textbf{Draw the block diagram of PLC and explain the function of each
block.}

\begin{solutionbox}

\textbf{PLC Block Diagram:}

\begin{center}
\textbf{Mermaid Diagram (Code)}
\begin{verbatim}
{Shaded}
{Highlighting}[]
graph TD
    PS[Power Supply] {-{-}{-} CPU[Central Processing Unit]}
    CPU {-{-}{-} MEM[Memory]}
    CPU {-{-}{-} INP[Input Module]}
    CPU {-{-}{-} OUT[Output Module]}
    CPU {-{-}{-} COM[Communication Module]}
    INP {-{-}{-} Input[Input Devices]}
    OUT {-{-}{-} Output[Output Devices]}
    COM {-{-}{-} Network[Network/HMI]}
    PROG[Programming Device] {-{-}{-} COM}
{Highlighting}
{Shaded}
\end{verbatim}
\end{center}

\textbf{Functions of each block:}

{\def\LTcaptype{none} % do not increment counter
\begin{longtable}[]{@{}
  >{\raggedright\arraybackslash}p{(\linewidth - 2\tabcolsep) * \real{0.4118}}
  >{\raggedright\arraybackslash}p{(\linewidth - 2\tabcolsep) * \real{0.5882}}@{}}
\toprule\noalign{}
\begin{minipage}[b]{\linewidth}\raggedright
Block
\end{minipage} & \begin{minipage}[b]{\linewidth}\raggedright
Function
\end{minipage} \\
\midrule\noalign{}
\endhead
\bottomrule\noalign{}
\endlastfoot
\textbf{Power Supply} & Converts main AC supply to DC required for
internal circuits \\
\textbf{CPU} & Executes program, processes I/O, performs calculations \\
\textbf{Memory} & Stores program, data, and I/O status (RAM, ROM,
EEPROM) \\
\textbf{Input Module} & Interfaces with input devices, provides
isolation, signal conditioning \\
\textbf{Output Module} & Drives output devices, provides isolation and
protection \\
\textbf{Communication Module} & Connects PLC to networks, other PLCs,
and programming devices \\
\textbf{Programming Device} & Used to develop, edit, and monitor PLC
programs \\
\end{longtable}
}

\textbf{Advantages of PLC:}

\begin{itemize}
\tightlist
\item
  \textbf{Reliability}: Solid-state components with high MTBF
\item
  \textbf{Flexibility}: Easily reprogrammable for different applications
\item
  \textbf{Communication}: Network capabilities for distributed control
\item
  \textbf{Diagnostics}: Built-in diagnostics and troubleshooting
\end{itemize}

\end{solutionbox}
\begin{mnemonicbox}
``PRIME-C'' - Power supply, RAM/ROM memory, Input
module, Microprocessor (CPU), Execution of program, Communication
interface

\end{mnemonicbox}
\subsection*{Question 5(a) OR [3
marks]}\label{q5a}

\textbf{State the applications of stepper motor.}

\begin{solutionbox}

{\def\LTcaptype{none} % do not increment counter
\begin{longtable}[]{@{}ll@{}}
\toprule\noalign{}
Application Area & Examples \\
\midrule\noalign{}
\endhead
\bottomrule\noalign{}
\endlastfoot
\textbf{Precision positioning} & CNC machines, 3D printers, robotic
arms \\
\textbf{Office equipment} & Printers, scanners, photocopiers \\
\textbf{Medical devices} & Surgical robots, fluid pumps, sample
handlers \\
\textbf{Automotive} & Headlight adjustment, idle control, mirror
control \\
\textbf{Aerospace} & Satellite positioning, antenna control \\
\textbf{Consumer electronics} & Cameras (focus/zoom), gaming
controllers \\
\end{longtable}
}

\end{solutionbox}
\begin{mnemonicbox}
``POMAC'' - Positioning systems, Office equipment,
Medical devices, Automotive controls, Consumer electronics

\end{mnemonicbox}
\subsection*{Question 5(b) OR [4
marks]}\label{q5b}

\textbf{Draw and explain the circuit to control speed of a DC series
motor.}

\begin{solutionbox}

\textbf{Speed Control of DC Series Motor using SCR:}

\begin{center}
\textbf{Mermaid Diagram (Code)}
\begin{verbatim}
{Shaded}
{Highlighting}[]
graph LR
    AC[AC Supply] {-{-}{-} B[Bridge Rectifier]}
    B {-{-}{-} SCR[SCR] {-}{-}{-} A[Armature]}
    A {-{-}{-} SF[Series Field]}
    SF {-{-}{-} B}
    GC[Gate Control] {-{-}{-} SCR}
{Highlighting}
{Shaded}
\end{verbatim}
\end{center}

\textbf{Working principle:}

\begin{itemize}
\tightlist
\item
  \textbf{Series connection}: Field winding in series with armature
\item
  \textbf{SCR control}: Phase-controlled SCR regulates average voltage
\item
  \textbf{Speed equation}: N ∝ (V - I(Ra+Rf))/IΦ
\item
  \textbf{Speed-torque relation}: Non-linear relationship
\item
  \textbf{Application}: Used when high starting torque required
\end{itemize}

\textbf{Advantages:}

\begin{itemize}
\tightlist
\item
  \textbf{High starting torque}: Ideal for traction applications
\item
  \textbf{Simple control}: Basic circuit design
\item
  \textbf{Cost-effective}: Fewer components than other methods
\end{itemize}

\end{solutionbox}
\begin{mnemonicbox}
``SCAT'' - Series connection, Current controls flux,
Average voltage controlled by SCR, Torque highest at low speeds

\end{mnemonicbox}
\subsection*{Question 5(c) OR [7
marks]}\label{q5c}

\textbf{Discuss the BLDC motor in brief.}

\begin{solutionbox}

\textbf{BLDC Motor (Brushless DC Motor):}

\begin{center}
\textbf{Mermaid Diagram (Code)}
\begin{verbatim}
{Shaded}
{Highlighting}[]
graph LR
    subgraph "BLDC Motor Construction"
    Stator[Stator with Windings]
    Rotor[Rotor with Permanent Magnets]
    Hall[Hall Sensors]
    end

    subgraph "Control System"
    Controller[Electronic Controller]
    Driver[Power Driver]
    Feedback[Position Feedback]
    end
    
    Controller {-{-}{-} Driver}
    Driver {-{-}{-} Stator}
    Hall {-{-}{-} Feedback}
    Feedback {-{-}{-} Controller}
{Highlighting}
{Shaded}
\end{verbatim}
\end{center}

\textbf{Construction:}

\begin{itemize}
\tightlist
\item
  \textbf{Stator}: Contains windings (typically 3-phase)
\item
  \textbf{Rotor}: Permanent magnets on rotor
\item
  \textbf{Position sensing}: Hall effect sensors or encoders
\item
  \textbf{Controller}: Electronic commutation controller
\end{itemize}

\textbf{Working principle:}

\begin{itemize}
\tightlist
\item
  \textbf{Electronic commutation}: Replaces mechanical brushes
\item
  \textbf{Sequencing}: Controller energizes stator coils in sequence
\item
  \textbf{Position feedback}: Hall sensors determine rotor position
\item
  \textbf{Phase energizing}: Proper phase energized based on rotor
  position
\end{itemize}

\textbf{Advantages:}

\begin{itemize}
\tightlist
\item
  \textbf{High efficiency}: No brush friction losses
\item
  \textbf{Low maintenance}: No brush wear
\item
  \textbf{Longer lifespan}: Reliable operation
\item
  \textbf{Better speed-torque characteristics}: Flat curve
\item
  \textbf{Low noise}: Quiet operation
\item
  \textbf{Better heat dissipation}: Windings on stator
\end{itemize}

\textbf{Applications:}

\begin{itemize}
\tightlist
\item
  \textbf{Computer cooling fans}: CPU/GPU coolers
\item
  \textbf{Hard disk drives}: Spindle motors
\item
  \textbf{Electric vehicles}: Propulsion systems
\item
  \textbf{Drones}: Propeller motors
\item
  \textbf{Home appliances}: Washing machines, refrigerators
\item
  \textbf{Industrial automation}: Precision control systems
\end{itemize}

\end{solutionbox}
\begin{mnemonicbox}
``COPPER'' - Commutation electronic, Operation
efficient, Permanent magnets, Position sensors, Electronic control,
Reliable performance

\end{mnemonicbox}

\end{document}
