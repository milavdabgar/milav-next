\documentclass[10pt,a4paper]{article}

% content/resources/templates/preamble.tex
\usepackage[margin=0.6in]{geometry}
\author{Milav Dabgar}
\usepackage{amsmath,amssymb,amsthm}
\usepackage{booktabs}
\usepackage{multirow}
\usepackage{xcolor}
\usepackage{tcolorbox}
\tcbuselibrary{breakable,skins}
\usepackage[colorlinks=true,linkcolor=blue]{hyperref}
\usepackage{titlesec}
\usepackage{enumitem}
\usepackage{tikz}
\usepackage{pgfplots}
\usepackage{circuitikz}
\usepackage[version=4]{mhchem}
\usepackage{longtable}
\usepackage{array}
\usepackage{float}
\usepackage{caption}
\usepackage{listings}

\lstset{
  basicstyle=\small\ttfamily,
  breaklines=true,
  breakatwhitespace=false,
  postbreak=\mbox{\textcolor{red}{$\hookrightarrow$}\space},
  float=false,
  numbers=left,
  numberstyle=\tiny\color{gray},
  numbersep=10pt,
  xleftmargin=2em,
  keywordstyle=\color{blue},
  commentstyle=\color{green!60!black},
  stringstyle=\color{purple},
  backgroundcolor=\color{gray!5},
  showstringspaces=false,
  tabsize=2,
  captionpos=b,
  keepspaces=true,
  columns=flexible
}

\pgfplotsset{compat=1.18}
\usetikzlibrary{shapes,arrows,positioning,calc,patterns,decorations.pathmorphing,decorations.markings,arrows.meta}

% Color scheme
\definecolor{headcolor}{RGB}{0,102,204}
\definecolor{keycolor}{RGB}{220,20,60}
\definecolor{solutioncolor}{RGB}{34,139,34}
\definecolor{mnemoniccolor}{RGB}{148,0,211}
\definecolor{codecolor}{RGB}{0,0,100}

% Spacing
\setlength{\parskip}{3pt}
\setlist[itemize]{nosep}
\setlist[enumerate]{nosep}

% Title formatting
\titleformat{\section}{\Large\bfseries\color{headcolor}}{\thesection}{1em}{}
\titleformat{\subsection}{\large\bfseries\color{headcolor}}{\thesubsection}{1em}{}

% Pandoc tightlist compatibility
\providecommand{\tightlist}{%
  \setlength{\itemsep}{0pt}\setlength{\parskip}{0pt}}

% Pandoc longtable compatibility
\newcounter{none}
\def\thenone{}


% content/resources/templates/english-boxes.tex
% This file is currently empty - it exists to maintain consistency with the import structure.
% Add custom environments here if needed in the future.


\begin{document}

\begin{center}
{\Huge\bfseries\color{headcolor} Subject Name Solutions}\\[5pt]
{\LARGE 4331103 -- Winter 2024}\\[3pt]
{\large Semester 1 Study Material}\\[3pt]
{\normalsize\textit{Detailed Solutions and Explanations}}
\end{center}

\vspace{10pt}

\subsection*{Question 1(a) [3 marks]}\label{q1a}

\textbf{Draw the structure of IGBT and explain it.}

\begin{solutionbox}
IGBT combines MOSFET's input with BJT's output
characteristics.

\begin{center}
\textbf{Mermaid Diagram (Code)}
\begin{verbatim}
{Shaded}
{Highlighting}[]
graph LR
    A[Gate] {-{-}{} B[Oxide Layer]}
    C[Emitter] {-{-}{} D[N+]}
    D {-{-}{} E[P Body]}
    E {-{-}{} F[N{-} Drift Region]}
    F {-{-}{} G[P+ Substrate]}
    G {-{-}{} H[Collector]}
{Highlighting}
{Shaded}
\end{verbatim}
\end{center}

\begin{itemize}
\tightlist
\item
  \textbf{Gate-Oxide Layer}: Controls device switching
\item
  \textbf{N+ Emitter}: Source of electrons
\item
  \textbf{P+ Collector}: Forms BJT section
\end{itemize}

\end{solutionbox}
\begin{mnemonicbox}
``MOSFET Input, BJT Output, IGBT Throughout''

\end{mnemonicbox}
\subsection*{Question 1(b) [4 marks]}\label{q1b}

\textbf{Draw and explain the construction of SCR. Also draw the
characteristic curve of it.}

\begin{solutionbox}
SCR is a four-layer PNPN semiconductor device with
three terminals.

\begin{center}
\textbf{Mermaid Diagram (Code)}
\begin{verbatim}
{Shaded}
{Highlighting}[]
graph LR
    A[Anode] {-{-}{} B[P Layer]}
    B {-{-}{} C[N Layer]}
    C {-{-}{} D[P Layer]}
    D {-{-}{} E[N Layer]}
    E {-{-}{} F[Cathode]}
    G[Gate] {-{-}{} D}
{Highlighting}
{Shaded}
\end{verbatim}
\end{center}

\textbf{Characteristic Curve:}

\begin{verbatim}
     I
     \^{}
     |                     Forward
     |                   Conduction
     |                      /
     |                     /
     |                    /
     |          Breakover|
     |               *   |
     |             /     |
     |Forward     |      |
     |Blocking    |      |
     |            |      |
     +{-{-}{-}{-}{-}{-}{-}{-}{-}{-}{-}{-}+{-}{-}{-}{-}{-}{-}+{-}{-}{-}{-} V}
     |            |
     |            |
     |Reverse     |
     |Blocking    |
     |            |
     |            V
\end{verbatim}

\begin{itemize}
\tightlist
\item
  \textbf{P-N-P-N Layers}: Forms two transistors (PNP, NPN)
\item
  \textbf{Gate Terminal}: Triggers conduction
\item
  \textbf{Holding Current}: Minimum to maintain conduction
\end{itemize}

\end{solutionbox}
\begin{mnemonicbox}
``PNPN Layers Form Two BJT Pairs''

\end{mnemonicbox}
\subsection*{Question 1(c) [7 marks]}\label{q1c}

\textbf{Explain the working of solid state relay using Opto TRIAC,
Opto-SCR and Opto-transistor with the help of circuit diagram.}

\begin{solutionbox}
Solid state relays use optocouplers for electrical
isolation between control and load circuits.

\begin{center}
\textbf{Mermaid Diagram (Code)}
\begin{verbatim}
{Shaded}
{Highlighting}[]
graph LR
    A[Control Circuit] {-{-}{} B[LED]}
    B {-{-}{} C[Opto{-}isolator]}
    C {-{-}{} D[Power Switching Element]}
    D {-{-}{} E[Load Circuit]}

    subgraph "Types"
    F[Opto{-TRIAC]}
    G[Opto{-SCR]}
    H[Opto{-Transistor]}
    end
{Highlighting}
{Shaded}
\end{verbatim}
\end{center}

{\def\LTcaptype{none} % do not increment counter
\begin{longtable}[]{@{}
  >{\raggedright\arraybackslash}p{(\linewidth - 8\tabcolsep) * \real{0.1562}}
  >{\raggedright\arraybackslash}p{(\linewidth - 8\tabcolsep) * \real{0.2188}}
  >{\raggedright\arraybackslash}p{(\linewidth - 8\tabcolsep) * \real{0.1719}}
  >{\raggedright\arraybackslash}p{(\linewidth - 8\tabcolsep) * \real{0.2344}}
  >{\raggedright\arraybackslash}p{(\linewidth - 8\tabcolsep) * \real{0.2188}}@{}}
\toprule\noalign{}
\begin{minipage}[b]{\linewidth}\raggedright
SSR Type
\end{minipage} & \begin{minipage}[b]{\linewidth}\raggedright
Input Circuit
\end{minipage} & \begin{minipage}[b]{\linewidth}\raggedright
Isolation
\end{minipage} & \begin{minipage}[b]{\linewidth}\raggedright
Output Circuit
\end{minipage} & \begin{minipage}[b]{\linewidth}\raggedright
Applications
\end{minipage} \\
\midrule\noalign{}
\endhead
\bottomrule\noalign{}
\endlastfoot
Opto-TRIAC & DC control signal & LED + TRIAC detector & TRIAC power
switch & AC loads \\
Opto-SCR & DC control signal & LED + photo-SCR & SCR power switch & DC
loads \\
Opto-Transistor & DC control signal & LED + phototransistor & Power
transistor & Low power DC \\
\end{longtable}
}

\begin{itemize}
\tightlist
\item
  \textbf{Working Principle}: Control signal activates LED \rightarrow Light
  triggers photo-sensitive device \rightarrow Switches power circuit
\item
  \textbf{Zero-Crossing Detection}: Reduces EMI by switching at zero
  voltage
\item
  \textbf{No Mechanical Parts}: Increases reliability and life
\end{itemize}

\end{solutionbox}
\begin{mnemonicbox}
``LED Illuminates, Photo-device Conducts, Power
Flows''

\end{mnemonicbox}
\subsection*{Question 1(c OR) [7
marks]}\label{question-1c-or-7-marks}

\textbf{Describe the working and constructional features of SCR, GTO and
power MOSFET with the help of characteristic curve.}

\begin{solutionbox}

{\def\LTcaptype{none} % do not increment counter
\begin{longtable}[]{@{}
  >{\raggedright\arraybackslash}p{(\linewidth - 6\tabcolsep) * \real{0.1290}}
  >{\raggedright\arraybackslash}p{(\linewidth - 6\tabcolsep) * \real{0.2258}}
  >{\raggedright\arraybackslash}p{(\linewidth - 6\tabcolsep) * \real{0.3387}}
  >{\raggedright\arraybackslash}p{(\linewidth - 6\tabcolsep) * \real{0.3065}}@{}}
\toprule\noalign{}
\begin{minipage}[b]{\linewidth}\raggedright
Device
\end{minipage} & \begin{minipage}[b]{\linewidth}\raggedright
Construction
\end{minipage} & \begin{minipage}[b]{\linewidth}\raggedright
Characteristic Curve
\end{minipage} & \begin{minipage}[b]{\linewidth}\raggedright
Working Principle
\end{minipage} \\
\midrule\noalign{}
\endhead
\bottomrule\noalign{}
\endlastfoot
SCR & PNPN 4-layer with gate & Latching - once ON stays ON & Gate pulse
triggers, requires external commutation to turn OFF \\
GTO & Modified SCR with better gate control & Similar to SCR but can be
turned OFF by gate & Negative gate pulse extracts carriers, turns OFF \\
Power MOSFET & Vertical structure with many cells & Non-latching -
requires gate bias & Gate voltage creates channel, removed voltage turns
OFF \\
\end{longtable}
}

\begin{center}
\textbf{Mermaid Diagram (Code)}
\begin{verbatim}
{Shaded}
{Highlighting}[]
graph TD
    subgraph "SCR"
    A1[Anode] {-{-}{} P1[P Layer]}
    P1 {-{-}{} N1[N Layer]}
    N1 {-{-}{} P2[P Layer]}
    P2 {-{-}{} N2[N Layer]}
    N2 {-{-}{} K1[Cathode]}
    G1[Gate] {-{-}{} P2}
    end

    subgraph "GTO"
    A2[Anode] {-{-}{} P3[P Layer]}
    P3 {-{-}{} N3[N Layer]}
    N3 {-{-}{} P4[P Layer]}
    P4 {-{-}{} N4[N Layer]}
    N4 {-{-}{} K2[Cathode]}
    G2[Gate] {-{-}{} P4}
    end
    
    subgraph "Power MOSFET"
    S[Source] {-{-}{} N5[N+ Source]}
    N5 {-{-}{} P5[P Body]}
    P5 {-{-}{} N6[N{-} Drift]}
    N6 {-{-}{} N7[N+ Substrate]}
    N7 {-{-}{} D[Drain]}
    G3[Gate] {-{-}{-}{} P5}
    end
{Highlighting}
{Shaded}
\end{verbatim}
\end{center}

\begin{itemize}
\tightlist
\item
  \textbf{SCR}: High current capability, latching behavior
\item
  \textbf{GTO}: Self turn-off capability, higher switching speed
\item
  \textbf{MOSFET}: Voltage-controlled, fast switching, no secondary
  breakdown
\end{itemize}

\end{solutionbox}
\begin{mnemonicbox}
``SCR Latches, GTO Self-Extinguishes, MOSFET
Channels''

\end{mnemonicbox}
\subsection*{Question 2(a) [3 marks]}\label{q2a}

\textbf{Explain the methods to protect SCR against over current in
details.}

\begin{solutionbox}
SCR over-current protection prevents device damage due
to excessive current.

{\def\LTcaptype{none} % do not increment counter
\begin{longtable}[]{@{}
  >{\raggedright\arraybackslash}p{(\linewidth - 4\tabcolsep) * \real{0.3519}}
  >{\raggedright\arraybackslash}p{(\linewidth - 4\tabcolsep) * \real{0.3519}}
  >{\raggedright\arraybackslash}p{(\linewidth - 4\tabcolsep) * \real{0.2963}}@{}}
\toprule\noalign{}
\begin{minipage}[b]{\linewidth}\raggedright
Protection Method
\end{minipage} & \begin{minipage}[b]{\linewidth}\raggedright
Working Principle
\end{minipage} & \begin{minipage}[b]{\linewidth}\raggedright
Implementation
\end{minipage} \\
\midrule\noalign{}
\endhead
\bottomrule\noalign{}
\endlastfoot
Fast-acting Fuses & Melts quickly during fault & Series with SCR \\
Circuit Breakers & Trips when current exceeds threshold & Main circuit
protection \\
Current-limiting Reactors & Limits di/dt and peak current & Series with
SCR \\
\end{longtable}
}

\begin{itemize}
\tightlist
\item
  \textbf{Heat Sinks}: Help dissipate excess heat
\item
  \textbf{Snubber Circuits}: Reduce current spikes during switching
\end{itemize}

\end{solutionbox}
\begin{mnemonicbox}
``Fuses Fast, Reactors Restrict, Breakers Break''

\end{mnemonicbox}
\subsection*{Question 2(b) [4 marks]}\label{q2b}

\textbf{Explain any two methods to turn ON the SCR.}

\begin{solutionbox}
SCR can be turned ON through different triggering
methods.

{\def\LTcaptype{none} % do not increment counter
\begin{longtable}[]{@{}
  >{\raggedright\arraybackslash}p{(\linewidth - 4\tabcolsep) * \real{0.3167}}
  >{\raggedright\arraybackslash}p{(\linewidth - 4\tabcolsep) * \real{0.4000}}
  >{\raggedright\arraybackslash}p{(\linewidth - 4\tabcolsep) * \real{0.2833}}@{}}
\toprule\noalign{}
\begin{minipage}[b]{\linewidth}\raggedright
Triggering Method
\end{minipage} & \begin{minipage}[b]{\linewidth}\raggedright
Circuit Implementation
\end{minipage} & \begin{minipage}[b]{\linewidth}\raggedright
Characteristics
\end{minipage} \\
\midrule\noalign{}
\endhead
\bottomrule\noalign{}
\endlastfoot
Gate Triggering & Pulse applied between gate-cathode & Most common,
controlled \\
Voltage Triggering & Anode voltage exceeds breakover voltage & No gate
control, emergency \\
\end{longtable}
}

\begin{center}
\textbf{Mermaid Diagram (Code)}
\begin{verbatim}
{Shaded}
{Highlighting}[]
graph TD
    subgraph "Gate Triggering"
    DC[DC Source] {-{-}{} R1[Resistor]}
    R1 {-{-}{} SW[Switch]}
    SW {-{-}{} G[Gate]}
    K[Cathode] {-{-}{} GND[Ground]}
    end

    subgraph "Voltage Triggering"
    VS[Voltage Source] {-{-}{} SCR[SCR Anode]}
    SCR {-{-}{} RL[Load]}
    RL {-{-}{} GND2[Ground]}
    end
{Highlighting}
{Shaded}
\end{verbatim}
\end{center}

\begin{itemize}
\tightlist
\item
  \textbf{Gate Triggering}: Controls firing angle precisely
\item
  \textbf{Voltage Triggering}: Happens when forward voltage exceeds
  breakover voltage
\end{itemize}

\end{solutionbox}
\begin{mnemonicbox}
``Gate Gets Control, Voltage Ventures Automatically''

\end{mnemonicbox}
\subsection*{Question 2(c) [7 marks]}\label{q2c}

\textbf{Enlist the various methods to turn OFF the SCR and explain each
of it using circuit diagram in brief.}

\begin{solutionbox}
SCR commutation methods are techniques to turn OFF a
conducting SCR.

{\def\LTcaptype{none} % do not increment counter
\begin{longtable}[]{@{}
  >{\raggedright\arraybackslash}p{(\linewidth - 4\tabcolsep) * \real{0.3774}}
  >{\raggedright\arraybackslash}p{(\linewidth - 4\tabcolsep) * \real{0.3585}}
  >{\raggedright\arraybackslash}p{(\linewidth - 4\tabcolsep) * \real{0.2642}}@{}}
\toprule\noalign{}
\begin{minipage}[b]{\linewidth}\raggedright
Commutation Method
\end{minipage} & \begin{minipage}[b]{\linewidth}\raggedright
Circuit Principle
\end{minipage} & \begin{minipage}[b]{\linewidth}\raggedright
Applications
\end{minipage} \\
\midrule\noalign{}
\endhead
\bottomrule\noalign{}
\endlastfoot
Natural Commutation & AC source crosses zero & AC circuits \\
Forced Commutation & External components force current to zero & DC
circuits \\
Class A (Self) & Parallel LC oscillator & Simple circuits \\
Class B (Resonant) & LC circuit in series with SCR & Medium power \\
Class C (Complementary) & Second SCR to divert current & High power \\
Class D (Auxiliary) & Auxiliary SCR + LC & Controlled timing \\
Class E (External) & External voltage source & Reliable but complex \\
\end{longtable}
}

\begin{center}
\textbf{Mermaid Diagram (Code)}
\begin{verbatim}
{Shaded}
{Highlighting}[]
graph LR
    subgraph "Natural Commutation"
    direction LR
    AC[AC Source] {-{-}{} SCR1[SCR]}
    SCR1 {-{-}{} L1[Load]}
    L1 {-{-}{} AC}
    end

    subgraph "Class B Commutation"
    direction LR
    DC[DC Source] {-{-}{} SCR2[SCR]}
    SCR2 {-{-}{} L2[Load]}
    C[Capacitor] {-{-}{-}{} SCR2}
    L[Inductor] {-{-}{-}{} C}
    SW[Switch] {-{-}{-}{} L}
    end
{Highlighting}
{Shaded}
\end{verbatim}
\end{center}

\begin{itemize}
\tightlist
\item
  \textbf{Natural Commutation}: Current naturally falls to zero in AC
  cycles
\item
  \textbf{Forced Commutation}: Artificially brings current to zero in DC
  circuits
\item
  \textbf{Communication Classes}: A through E progressively more complex
  and reliable
\end{itemize}

\end{solutionbox}
\begin{mnemonicbox}
``Natural Zeros, Forced Components, Classes Advance
Reliability''

\end{mnemonicbox}
\subsection*{Question 2(a OR) [3
marks]}\label{question-2a-or-3-marks}

\textbf{Explain the methods to protect SCR against over voltage in
details.}

\begin{solutionbox}
Over-voltage protection prevents damage from voltage
transients.

{\def\LTcaptype{none} % do not increment counter
\begin{longtable}[]{@{}lll@{}}
\toprule\noalign{}
Protection Method & Working Principle & Implementation \\
\midrule\noalign{}
\endhead
\bottomrule\noalign{}
\endlastfoot
Snubber Circuits & RC network limits dv/dt & Parallel with SCR \\
Metal Oxide Varistors & Clamps voltage spikes & Parallel with SCR \\
Zener Diodes & Breaks down at set voltage & Anode-cathode protection \\
\end{longtable}
}

\begin{center}
\textbf{Mermaid Diagram (Code)}
\begin{verbatim}
{Shaded}
{Highlighting}[]
graph LR
    subgraph "Snubber Circuit"
    direction LR
    A1[Anode] {-{-}{-} R[Resistor]}
    R {-{-}{-} C[Capacitor]}
    C {-{-}{-} K1[Cathode]}
    end
{Highlighting}
{Shaded}
\end{verbatim}
\end{center}

\begin{itemize}
\tightlist
\item
  \textbf{Snubber Circuit}: Limits voltage rise rate (dv/dt)
\item
  \textbf{MOV}: Absorbs energy from voltage spikes
\item
  \textbf{Thyristor Rating}: Always use components with margin above
  circuit voltage
\end{itemize}

\end{solutionbox}
\begin{mnemonicbox}
``Snubbers Slow, Varistors Clamp, Zeners Zap''

\end{mnemonicbox}
\subsection*{Question 2(b OR) [4
marks]}\label{question-2b-or-4-marks}

\textbf{Explain triggering of Thyristor in detail.}

\begin{solutionbox}
Thyristor triggering involves activating the device
from blocking to conduction state.

{\def\LTcaptype{none} % do not increment counter
\begin{longtable}[]{@{}
  >{\raggedright\arraybackslash}p{(\linewidth - 4\tabcolsep) * \real{0.3800}}
  >{\raggedright\arraybackslash}p{(\linewidth - 4\tabcolsep) * \real{0.3800}}
  >{\raggedright\arraybackslash}p{(\linewidth - 4\tabcolsep) * \real{0.2400}}@{}}
\toprule\noalign{}
\begin{minipage}[b]{\linewidth}\raggedright
Triggering Method
\end{minipage} & \begin{minipage}[b]{\linewidth}\raggedright
Working Mechanism
\end{minipage} & \begin{minipage}[b]{\linewidth}\raggedright
Advantages
\end{minipage} \\
\midrule\noalign{}
\endhead
\bottomrule\noalign{}
\endlastfoot
Gate Triggering & Low power pulse at gate-cathode & Precise control \\
R-C Phase Shift & Varies phase angle for control & Simple circuit \\
UJT Triggering & Relaxation oscillator generates pulses & Stable
timing \\
Light Triggering & Photons generate carriers (LASCR) & Electrical
isolation \\
\end{longtable}
}

\begin{center}
\textbf{Mermaid Diagram (Code)}
\begin{verbatim}
{Shaded}
{Highlighting}[]
graph TD
    subgraph "UJT Triggering Circuit"
    direction LR
    DC[DC Source] {-{-}{} R1[Resistor]}
    R1 {-{-}{} UJT[UJT Emitter]}
    UJT {-{-}{} C[Capacitor]}
    C {-{-}{} GND[Ground]}
    UJT {-{-} "Base 1" {-}{-}{} R2[Resistor]}
    R2 {-{-}{} GND}
    UJT {-{-} "Base 2" {-}{-}{} R3[Resistor]}
    R3 {-{-}{} DC}
    UJT {-{-} "Pulse Output" {-}{-}{} T[Transformer]}
    T {-{-}{} G[SCR Gate]}
    end
{Highlighting}
{Shaded}
\end{verbatim}
\end{center}

\begin{itemize}
\tightlist
\item
  \textbf{Gate Current}: Must exceed latching current
\item
  \textbf{Gate Pulse}: Width and amplitude critical for reliable
  triggering
\item
  \textbf{Triggering Angle}: Controls power delivered to load
\end{itemize}

\end{solutionbox}
\begin{mnemonicbox}
``Gate Gets Going, RC Rhythmically, UJT Uniformly,
Light Liberates''

\end{mnemonicbox}
\subsection*{Question 2(c OR) [7
marks]}\label{question-2c-or-7-marks}

\textbf{Design and explain snubber circuit for SCR. Also explain the
importance of it.}

\begin{solutionbox}
Snubber circuits protect SCR from voltage transients
and control switching behavior.

\begin{center}
\textbf{Mermaid Diagram (Code)}
\begin{verbatim}
{Shaded}
{Highlighting}[]
graph LR
    A[Anode] {-{-}{-} R[Resistor]}
    R {-{-}{-} C[Capacitor]}
    C {-{-}{-} K[Cathode]}
    A {-{-}{-} SCR[SCR]}
    SCR {-{-}{-} K}
    A {-{-}{-} L[Inductor]}
    L {-{-}{-} Load[Load]}
    Load {-{-}{-} K}
{Highlighting}
{Shaded}
\end{verbatim}
\end{center}

{\def\LTcaptype{none} % do not increment counter
\begin{longtable}[]{@{}lll@{}}
\toprule\noalign{}
Component & Function & Selection Criteria \\
\midrule\noalign{}
\endhead
\bottomrule\noalign{}
\endlastfoot
Resistor (R) & Limits discharge current & R \textgreater{} E/I_{(}max_{)} \\
Capacitor (C) & Absorbs voltage transients & C = I_{(}load_{)}/(dv/dt) \\
Optional Diode & Provides discharge path & Fast recovery type \\
\end{longtable}
}

\textbf{Design Steps:}

\begin{enumerate}
\tightlist
\item
  Calculate maximum dv/dt from SCR datasheet
\item
  Determine load current and circuit voltage
\item
  Select C to limit dv/dt below SCR rating
\item
  Select R to limit discharge current and provide damping
\end{enumerate}

\textbf{Importance:}

\begin{itemize}
\tightlist
\item
  \textbf{dv/dt Protection}: Prevents false triggering
\item
  \textbf{Turn-off Support}: Improves commutation
\item
  \textbf{Switching Loss Reduction}: Reduces power dissipation
\item
  \textbf{EMI Reduction}: Smooths voltage transitions
\end{itemize}

\end{solutionbox}
\begin{mnemonicbox}
``Resistor Restrains, Capacitor Catches, Diode
Directs''

\end{mnemonicbox}
\subsection*{Question 3(a) [3 marks]}\label{q3a}

\textbf{Explain the working of three phase Full Wave Rectifier using
circuit diagram.}

\begin{solutionbox}
Three-phase full-wave rectifier converts three-phase AC
to DC with six diodes.

\begin{center}
\textbf{Mermaid Diagram (Code)}
\begin{verbatim}
{Shaded}
{Highlighting}[]
graph LR
    subgraph "Three{-Phase Source"}
    A[Phase A]
    B[Phase B]
    C[Phase C]
    end

    subgraph "Bridge Rectifier"
    D1[D1]
    D2[D2]
    D3[D3]
    D4[D4]
    D5[D5]
    D6[D6]
    end
    
    A {-{-}{} D1}
    B {-{-}{} D3}
    C {-{-}{} D5}
    D1 {-{-}{} P["{}+"]}
    D3 {-{-}{} P}
    D5 {-{-}{} P}
    N["{{-}"] {-}{-}{} D2}
    N {-{-}{} D4}
    N {-{-}{} D6}
    D2 {-{-}{} A}
    D4 {-{-}{} B}
    D6 {-{-}{} C}
    
    P {-{-}{} RL[Load]}
    RL {-{-}{} N}
{Highlighting}
{Shaded}
\end{verbatim}
\end{center}

\begin{itemize}
\tightlist
\item
  \textbf{Six Diodes}: Three for positive, three for negative
  half-cycles
\item
  \textbf{Conduction}: Each diode conducts for 120^\circ per cycle
\item
  \textbf{Output}: Low ripple (4.2\%) compared to single-phase
\end{itemize}

\end{solutionbox}
\begin{mnemonicbox}
``Six Diodes, Three Phases, Smooth DC''

\end{mnemonicbox}
\subsection*{Question 3(b) [4 marks]}\label{q3b}

\textbf{Differentiate single phase and poly phase rectifier circuit.}

\begin{solutionbox}

{\def\LTcaptype{none} % do not increment counter
\begin{longtable}[]{@{}
  >{\raggedright\arraybackslash}p{(\linewidth - 4\tabcolsep) * \real{0.1930}}
  >{\raggedright\arraybackslash}p{(\linewidth - 4\tabcolsep) * \real{0.4211}}
  >{\raggedright\arraybackslash}p{(\linewidth - 4\tabcolsep) * \real{0.3860}}@{}}
\toprule\noalign{}
\begin{minipage}[b]{\linewidth}\raggedright
Parameter
\end{minipage} & \begin{minipage}[b]{\linewidth}\raggedright
Single Phase Rectifier
\end{minipage} & \begin{minipage}[b]{\linewidth}\raggedright
Poly Phase Rectifier
\end{minipage} \\
\midrule\noalign{}
\endhead
\bottomrule\noalign{}
\endlastfoot
Input & Single AC source & Multiple AC sources (3 or more) \\
Diodes Required & 2 (half-wave), 4 (full-wave) & 3 (half-wave), 6
(full-wave) \\
Ripple Factor & 0.482 (full-wave) & 0.042 (3-phase full-wave) \\
Transformer Utilization & Lower (0.812) & Higher (0.955) \\
Output Waveform & Pulsating & Much smoother \\
Efficiency & Lower & Higher \\
Applications & Low power applications & Industrial power supplies \\
\end{longtable}
}

\begin{itemize}
\tightlist
\item
  \textbf{Form Factor}: Lower in poly-phase (better quality DC)
\item
  \textbf{Power Handling}: Polyphase handles higher power more
  efficiently
\item
  \textbf{Circuit Complexity}: Polyphase more complex but better
  performance
\end{itemize}

\end{solutionbox}
\begin{mnemonicbox}
``Single Pulses Heavily, Poly Provides Smoothly''

\end{mnemonicbox}
\subsection*{Question 3(c) [7 marks]}\label{q3c}

\textbf{Describe the application of series, parallel and bridge type
Inverter.}

\begin{solutionbox}

{\def\LTcaptype{none} % do not increment counter
\begin{longtable}[]{@{}
  >{\raggedright\arraybackslash}p{(\linewidth - 6\tabcolsep) * \real{0.2344}}
  >{\raggedright\arraybackslash}p{(\linewidth - 6\tabcolsep) * \real{0.2812}}
  >{\raggedright\arraybackslash}p{(\linewidth - 6\tabcolsep) * \real{0.2188}}
  >{\raggedright\arraybackslash}p{(\linewidth - 6\tabcolsep) * \real{0.2656}}@{}}
\toprule\noalign{}
\begin{minipage}[b]{\linewidth}\raggedright
Inverter Type
\end{minipage} & \begin{minipage}[b]{\linewidth}\raggedright
Circuit Topology
\end{minipage} & \begin{minipage}[b]{\linewidth}\raggedright
Applications
\end{minipage} & \begin{minipage}[b]{\linewidth}\raggedright
Characteristics
\end{minipage} \\
\midrule\noalign{}
\endhead
\bottomrule\noalign{}
\endlastfoot
Series Inverter & Resonant LC with load in series & Induction heating,
Ultrasonic generators & • High frequency• Voltage source•
Self-commutating \\
Parallel Inverter & Resonant LC with load in parallel & Uninterruptible
power supplies, Solar inverters & • Current source• Better efficiency•
Wider load range \\
Bridge Inverter & H-bridge with 4 switches & Motor drives, Grid-tied
systems, General purpose & • Voltage/current source• Most versatile•
Various control methods \\
\end{longtable}
}

\begin{center}
\textbf{Mermaid Diagram (Code)}
\begin{verbatim}
{Shaded}
{Highlighting}[]
graph TD
    subgraph "Series Inverter"
    DC1[DC Source] {-{-}{} S1[SCR]}
    S1 {-{-}{} L1[Inductor]}
    L1 {-{-}{} C1[Capacitor]}
    C1 {-{-}{} RL1[Load]}
    RL1 {-{-}{} DC1}
    end

    subgraph "Parallel Inverter"
    DC2[DC Source] {-{-}{} L2[Inductor]}
    L2 {-{-}{} S2[SCR]}
    S2 {-{-}{} RL2[Load]}
    C2[Capacitor] {-{-}{} RL2}
    RL2 {-{-}{} DC2}
    end
    
    subgraph "Bridge Inverter"
    DC3[DC Source] {-{-}{} Q1[Q1]}
    DC3 {-{-}{} Q3[Q3]}
    Q1 {-{-}{} Q2[Q2]}
    Q3 {-{-}{} Q4[Q4]}
    Q2 {-{-}{} DC3}
    Q4 {-{-}{} DC3}
    Q1 {-{-} "Load" {-}{-}{} Q4}
    Q3 {-{-} "Load" {-}{-}{} Q2}
    end
{Highlighting}
{Shaded}
\end{verbatim}
\end{center}

\begin{itemize}
\tightlist
\item
  \textbf{Series Inverter}: Best for fixed-frequency, fixed-load
  applications
\item
  \textbf{Parallel Inverter}: Handles load variations better
\item
  \textbf{Bridge Inverter}: Most widely used for general applications
\end{itemize}

\end{solutionbox}
\begin{mnemonicbox}
``Series Sings at High Frequency, Parallel Performs
with Variety, Bridge Brings Versatility''

\end{mnemonicbox}
\subsection*{Question 3(a OR) [3
marks]}\label{question-3a-or-3-marks}

\textbf{Explain the working of three phase Half Wave Rectifier using
circuit diagram.}

\begin{solutionbox}
Three-phase half-wave rectifier uses three diodes to
convert three-phase AC to DC.

\begin{center}
\textbf{Mermaid Diagram (Code)}
\begin{verbatim}
{Shaded}
{Highlighting}[]
graph LR
    subgraph "Three{-Phase Source"}
    A[Phase A]
    B[Phase B]
    C[Phase C]
    N[Neutral]
    end

    subgraph "Half{-Wave Rectifier"}
    D1[D1]
    D2[D2]
    D3[D3]
    end
    
    A {-{-}{} D1}
    B {-{-}{} D2}
    C {-{-}{} D3}
    D1 {-{-}{} P["{}+"]}
    D2 {-{-}{} P}
    D3 {-{-}{} P}
    P {-{-}{} RL[Load]}
    RL {-{-}{} N}
{Highlighting}
{Shaded}
\end{verbatim}
\end{center}

\begin{itemize}
\tightlist
\item
  \textbf{Three Diodes}: Each conducts during positive half-cycle of its
  phase
\item
  \textbf{Conduction}: Each diode conducts for 120^\circ per cycle
\item
  \textbf{Output}: 13.4\% ripple (higher than full-wave)
\end{itemize}

\end{solutionbox}
\begin{mnemonicbox}
``Three Diodes, Three Phases, One Direction''

\end{mnemonicbox}
\subsection*{Question 3(b OR) [4
marks]}\label{question-3b-or-4-marks}

\textbf{Enlist the different types of charging technology and compare
it.}

\begin{solutionbox}

{\def\LTcaptype{none} % do not increment counter
\begin{longtable}[]{@{}
  >{\raggedright\arraybackslash}p{(\linewidth - 6\tabcolsep) * \real{0.3030}}
  >{\raggedright\arraybackslash}p{(\linewidth - 6\tabcolsep) * \real{0.2879}}
  >{\raggedright\arraybackslash}p{(\linewidth - 6\tabcolsep) * \real{0.1818}}
  >{\raggedright\arraybackslash}p{(\linewidth - 6\tabcolsep) * \real{0.2273}}@{}}
\toprule\noalign{}
\begin{minipage}[b]{\linewidth}\raggedright
Charging Technology
\end{minipage} & \begin{minipage}[b]{\linewidth}\raggedright
Working Principle
\end{minipage} & \begin{minipage}[b]{\linewidth}\raggedright
Advantages
\end{minipage} & \begin{minipage}[b]{\linewidth}\raggedright
Disadvantages
\end{minipage} \\
\midrule\noalign{}
\endhead
\bottomrule\noalign{}
\endlastfoot
Constant Current (CC) & Fixed current until voltage threshold & Simple,
low cost & Longer charging time \\
Constant Voltage (CV) & Fixed voltage with declining current & Fast
initial charge & Current not limited at start \\
CC-CV & Starts with CC, switches to CV & Optimal charging profile &
Requires controller circuit \\
Pulse Charging & Current pulses with rest periods & Reduces heat,
extends battery life & Complex control circuit \\
Trickle Charging & Very low constant current & Maintains charge & Not
suitable for main charging \\
Fast Charging & High current with intelligent control & Significantly
reduced charging time & Heat generation, battery stress \\
Wireless Charging & Inductive coupling & Convenient, no cables & Lower
efficiency, alignment issues \\
\end{longtable}
}

\begin{itemize}
\tightlist
\item
  \textbf{Battery Types}: Different technologies suit different battery
  chemistries
\item
  \textbf{Charging Profiles}: Must match battery specifications to avoid
  damage
\item
  \textbf{Temperature Management}: Critical factor in charging
  efficiency and safety
\end{itemize}

\end{solutionbox}
\begin{mnemonicbox}
``Current Consistently, Voltage Varies, Pulse Pauses,
Trickle Tops, Fast Finishes''

\end{mnemonicbox}
\subsection*{Question 3(c OR) [7
marks]}\label{question-3c-or-7-marks}

\textbf{Explain the working of Solar Photovoltaic (PV) based power
generation with the help of block diagram.}

\begin{solutionbox}
Solar PV systems convert sunlight directly into
electricity through the photovoltaic effect.

\begin{center}
\textbf{Mermaid Diagram (Code)}
\begin{verbatim}
{Shaded}
{Highlighting}[]
graph LR
    S[Sunlight] {-{-}{} PV[Solar PV Panels]}
    PV {-{-}{} C[Charge Controller]}
    C {-{-}{} B[Battery Bank]}
    C {-{-}{} I[Inverter]}
    B {-{-}{} I}
    I {-{-}{} L[AC Loads]}
    C {-{-}{} DC[DC Loads]}
{Highlighting}
{Shaded}
\end{verbatim}
\end{center}

{\def\LTcaptype{none} % do not increment counter
\begin{longtable}[]{@{}
  >{\raggedright\arraybackslash}p{(\linewidth - 4\tabcolsep) * \real{0.3929}}
  >{\raggedright\arraybackslash}p{(\linewidth - 4\tabcolsep) * \real{0.3571}}
  >{\raggedright\arraybackslash}p{(\linewidth - 4\tabcolsep) * \real{0.2500}}@{}}
\toprule\noalign{}
\begin{minipage}[b]{\linewidth}\raggedright
Component
\end{minipage} & \begin{minipage}[b]{\linewidth}\raggedright
Function
\end{minipage} & \begin{minipage}[b]{\linewidth}\raggedright
Types
\end{minipage} \\
\midrule\noalign{}
\endhead
\bottomrule\noalign{}
\endlastfoot
Solar Panels & Convert light to DC electricity & Monocrystalline,
Polycrystalline, Thin-film \\
Charge Controller & Regulates battery charging & PWM, MPPT \\
Battery Bank & Stores energy & Lead-acid, Lithium-ion, Flow \\
Inverter & Converts DC to AC & Pure sine wave, Modified sine wave \\
Distribution System & Delivers power to loads & Off-grid, Grid-tied,
Hybrid \\
\end{longtable}
}

\begin{itemize}
\tightlist
\item
  \textbf{Photovoltaic Effect}: Light energy creates electron flow in
  semiconductor material
\item
  \textbf{Maximum Power Point Tracking}: Optimizes power extraction
  under varying conditions
\item
  \textbf{Grid Integration}: Can operate standalone or connected to
  utility grid
\end{itemize}

\end{solutionbox}
\begin{mnemonicbox}
``Sunlight Strikes Semiconductors, Controllers
Charge, Batteries Bank, Inverters Interface''

\end{mnemonicbox}
\subsection*{Question 4(a) [3 marks]}\label{q4a}

\textbf{State the merits and demerits of Induction heating.}

\begin{solutionbox}

{\def\LTcaptype{none} % do not increment counter
\begin{longtable}[]{@{}
  >{\raggedright\arraybackslash}p{(\linewidth - 2\tabcolsep) * \real{0.4833}}
  >{\raggedright\arraybackslash}p{(\linewidth - 2\tabcolsep) * \real{0.5167}}@{}}
\toprule\noalign{}
\begin{minipage}[b]{\linewidth}\raggedright
Merits of Induction Heating
\end{minipage} & \begin{minipage}[b]{\linewidth}\raggedright
Demerits of Induction Heating
\end{minipage} \\
\midrule\noalign{}
\endhead
\bottomrule\noalign{}
\endlastfoot
Rapid heating without direct contact & High initial installation cost \\
Precise temperature control & Requires electrical power source \\
Energy efficient (80-90\%) & Limited to electrically conductive
materials \\
Clean and pollution-free & Requires proper cooling systems \\
Localized heating possible & EMI generation may affect nearby
electronics \\
Uniform heating throughout material & May require specialized coil
designs \\
\end{longtable}
}

\begin{itemize}
\tightlist
\item
  \textbf{Working Principle}: Eddy currents induced in workpiece
  generate heat
\item
  \textbf{Applications}: Melting, hardening, annealing, welding
\end{itemize}

\end{solutionbox}
\begin{mnemonicbox}
``Fast, Focused, Efficient but Costly, Conductive,
Complex''

\end{mnemonicbox}
\subsection*{Question 4(b) [4 marks]}\label{q4b}

\textbf{Draw the circuit of sequential timer using IC-555 and explain
its working.}

\begin{solutionbox}
Sequential timer provides multiple timed outputs in
sequence.

\begin{center}
\textbf{Mermaid Diagram (Code)}
\begin{verbatim}
{Shaded}
{Highlighting}[]
graph TD
    VCC[+VCC] {-{-}{} R1[R1]}
    R1 {-{-}{} RST1[Reset IC1]}
    VCC {-{-}{} R2[R2]}
    R2 {-{-}{} TR1[Trigger IC1]}
    VCC {-{-}{} R3[R3]}
    R3 {-{-}{} THR1[Threshold IC1]}

    IC1[555 Timer 1] {-{-} "Output" {-}{-}{} C1[C1]}
    C1 {-{-}{} TR2[Trigger IC2]}
    
    IC2[555 Timer 2] {-{-} "Output" {-}{-}{} C2[C2]}
    C2 {-{-}{} TR3[Trigger IC3]}
    
    IC3[555 Timer 3] {-{-} "Output" {-}{-}{} LOAD[Load]}
{Highlighting}
{Shaded}
\end{verbatim}
\end{center}

\textbf{Working:}

\begin{enumerate}
\tightlist
\item
  First 555 timer operates in monostable mode
\item
  Output triggers second timer when first timing cycle completes
\item
  Second timer triggers third timer
\item
  Each timer's period determined by its RC time constant
\end{enumerate}

\begin{itemize}
\tightlist
\item
  \textbf{RC Values}: T = 1.1 \times R \times C determines each stage's timing
\item
  \textbf{Cascading}: Multiple stages provide sequential timing events
\item
  \textbf{Applications}: Process control, industrial sequencing
\end{itemize}

\end{solutionbox}
\begin{mnemonicbox}
``One Timer Triggers Another Sequentially''

\end{mnemonicbox}
\subsection*{Question 4(c) [7 marks]}\label{q4c}

\textbf{Draw the schematic circuit for single phase AC power control
using TRIAC and explain it in detail.}

\begin{solutionbox}
TRIAC-based AC power control regulates power to loads
through phase angle control.

\begin{center}
\textbf{Mermaid Diagram (Code)}
\begin{verbatim}
{Shaded}
{Highlighting}[]
graph LR
    AC[AC Supply] {-{-}{} F[Fuse]}
    F {-{-}{} T[TRIAC]}
    T {-{-}{} L[Load]}
    L {-{-}{} AC}

    AC {-{-} "Phase Detection" {-}{-}{} ZC[Zero{-}Crossing Detector]}
    ZC {-{-}{} TC[Timing Circuit]}
    TC {-{-}{} G[Gate Drive]}
    G {-{-}{} T}
{Highlighting}
{Shaded}
\end{verbatim}
\end{center}

{\def\LTcaptype{none} % do not increment counter
\begin{longtable}[]{@{}
  >{\raggedright\arraybackslash}p{(\linewidth - 4\tabcolsep) * \real{0.2750}}
  >{\raggedright\arraybackslash}p{(\linewidth - 4\tabcolsep) * \real{0.2500}}
  >{\raggedright\arraybackslash}p{(\linewidth - 4\tabcolsep) * \real{0.4750}}@{}}
\toprule\noalign{}
\begin{minipage}[b]{\linewidth}\raggedright
Component
\end{minipage} & \begin{minipage}[b]{\linewidth}\raggedright
Function
\end{minipage} & \begin{minipage}[b]{\linewidth}\raggedright
Selection Criteria
\end{minipage} \\
\midrule\noalign{}
\endhead
\bottomrule\noalign{}
\endlastfoot
TRIAC & Bidirectional power switch & Current rating \textgreater{} load
current \\
DIAC & Triggers TRIAC symmetrically & Breakover voltage \textless{}
trigger voltage \\
RC Network & Phase shifting for firing angle & R determines firing angle
range \\
Snubber Circuit & dv/dt protection & Based on TRIAC specifications \\
\end{longtable}
}

\textbf{Operation Principle:}

\begin{enumerate}
\tightlist
\item
  RC network creates phase shift from AC input
\item
  DIAC breaks over when capacitor voltage reaches threshold
\item
  DIAC triggers TRIAC at specific phase angle
\item
  Varying R changes phase angle, controlling power
\end{enumerate}

\begin{itemize}
\tightlist
\item
  \textbf{Firing Angle}: 0^\circ (full power) to 180^\circ (zero power)
\item
  \textbf{Applications}: Light dimmers, heater control, motor speed
  control
\item
  \textbf{Advantages}: Smooth control, no moving parts, high reliability
\end{itemize}

\end{solutionbox}
\begin{mnemonicbox}
``Resistance Changes Phase, DIAC Delivers Pulse,
TRIAC Transmits Power''

\end{mnemonicbox}
\subsection*{Question 4(a OR) [3
marks]}\label{question-4a-or-3-marks}

\textbf{Enlist the merits and demerits of Dielectric heating.}

\begin{solutionbox}

{\def\LTcaptype{none} % do not increment counter
\begin{longtable}[]{@{}
  >{\raggedright\arraybackslash}p{(\linewidth - 2\tabcolsep) * \real{0.4839}}
  >{\raggedright\arraybackslash}p{(\linewidth - 2\tabcolsep) * \real{0.5161}}@{}}
\toprule\noalign{}
\begin{minipage}[b]{\linewidth}\raggedright
Merits of Dielectric Heating
\end{minipage} & \begin{minipage}[b]{\linewidth}\raggedright
Demerits of Dielectric Heating
\end{minipage} \\
\midrule\noalign{}
\endhead
\bottomrule\noalign{}
\endlastfoot
Uniform heating throughout material & High initial equipment cost \\
Rapid heating (even for insulators) & High frequency power source
required \\
Selective heating possible & Not effective for conductive materials \\
Energy efficient for certain materials & RF radiation safety concerns \\
Clean and pollution-free & Complex impedance matching requirements \\
Works with non-conductive materials & Power loss in transmission
lines \\
\end{longtable}
}

\begin{itemize}
\tightlist
\item
  \textbf{Working Principle}: Dipole rotation in high-frequency electric
  field generates heat
\item
  \textbf{Applications}: Plastic welding, wood drying, food processing
\end{itemize}

\end{solutionbox}
\begin{mnemonicbox}
``Uniform, Rapid, Insulator-friendly but Expensive,
Complex, RF-intensive''

\end{mnemonicbox}
\subsection*{Question 4(b OR) [4
marks]}\label{question-4b-or-4-marks}

\textbf{Draw the circuit diagram of photo-electric relay using LDR and
explain its working.}

\begin{solutionbox}
Photo-electric relay uses light-dependent resistor to
detect light and control a relay.

\begin{center}
\textbf{Mermaid Diagram (Code)}
\begin{verbatim}
{Shaded}
{Highlighting}[]
graph LR
    VCC[+VCC] {-{-}{} R1[Load Resistor]}
    R1 {-{-}{} C[Collector]}
    VCC {-{-}{} RL[Relay Coil]}
    RL {-{-}{} C}
    C {-{-}{} Q[Transistor]}
    Q {-{-}{} GND[Ground]}
    B[Base] {-{-}{} Q}
    R2[Base Resistor] {-{-}{} B}
    VCC {-{-}{} LDR[LDR]}
    LDR {-{-}{} R2}
    RL {-{-} "Diode" {-}{-}{} VCC}
{Highlighting}
{Shaded}
\end{verbatim}
\end{center}

\textbf{Working:}

\begin{enumerate}
\tightlist
\item
  LDR resistance decreases when light falls on it
\item
  Voltage divider (LDR + R2) provides base current to transistor
\item
  Transistor turns ON when sufficient base current flows
\item
  Relay activates when transistor conducts
\end{enumerate}

\begin{itemize}
\tightlist
\item
  \textbf{Light Threshold}: Adjustable via potentiometer
\item
  \textbf{Applications}: Automatic lighting, counting systems, alarm
  systems
\item
  \textbf{LDR Characteristics}: Resistance inversely proportional to
  light intensity
\end{itemize}

\end{solutionbox}
\begin{mnemonicbox}
``Light Lowers Resistance, Transistor Turns, Relay
Responds''

\end{mnemonicbox}
\subsection*{Question 4(c OR) [7
marks]}\label{question-4c-or-7-marks}

\textbf{Draw the circuit of DC power control using SCR with UJT in
triggering circuit and explain in detail.}

\begin{solutionbox}
UJT-triggered SCR circuit provides precise control of
DC power to loads.

\begin{center}
\textbf{Mermaid Diagram (Code)}
\begin{verbatim}
{Shaded}
{Highlighting}[]
graph LR
    DC[DC Source] {-{-}{} F[Fuse]}
    F {-{-}{} SCR[SCR]}
    SCR {-{-}{} L[Load]}
    L {-{-}{} DC}

    DC {-{-}{} R1[R1]}
    R1 {-{-}{} P[Potentiometer]}
    P {-{-}{} C1[Timing Capacitor]}
    C1 {-{-}{} E[UJT Emitter]}
    E {-{-}{} UJT[UJT]}
    UJT {-{-} "Base 1" {-}{-}{} R2[R2]}
    R2 {-{-}{} GND[Ground]}
    UJT {-{-} "Base 2" {-}{-}{} R3[R3]}
    R3 {-{-}{} DC}
    UJT {-{-} "Pulse Output" {-}{-}{} T[Transformer]}
    T {-{-}{} G[SCR Gate]}
    G {-{-}{} K[SCR Cathode]}
{Highlighting}
{Shaded}
\end{verbatim}
\end{center}

{\def\LTcaptype{none} % do not increment counter
\begin{longtable}[]{@{}
  >{\raggedright\arraybackslash}p{(\linewidth - 4\tabcolsep) * \real{0.2750}}
  >{\raggedright\arraybackslash}p{(\linewidth - 4\tabcolsep) * \real{0.2500}}
  >{\raggedright\arraybackslash}p{(\linewidth - 4\tabcolsep) * \real{0.4750}}@{}}
\toprule\noalign{}
\begin{minipage}[b]{\linewidth}\raggedright
Component
\end{minipage} & \begin{minipage}[b]{\linewidth}\raggedright
Function
\end{minipage} & \begin{minipage}[b]{\linewidth}\raggedright
Selection Criteria
\end{minipage} \\
\midrule\noalign{}
\endhead
\bottomrule\noalign{}
\endlastfoot
UJT & Generates trigger pulses & η (intrinsic standoff ratio) =
0.5-0.8 \\
R_{1}+P & Timing resistor & Controls charging rate of C_{1} \\
C_{1} & Timing capacitor & Determines pulse frequency \\
Transformer & Isolates UJT circuit from SCR & Pulse transmission
capability \\
SCR & Main power control & Current rating \textgreater{} load current \\
\end{longtable}
}

\textbf{Working Principle:}

\begin{enumerate}
\tightlist
\item
  UJT relaxation oscillator generates pulses
\item
  Potentiometer varies charging rate, changing pulse frequency
\item
  Pulses are coupled through transformer to SCR gate
\item
  SCR conducts for portion of cycle based on trigger timing
\end{enumerate}

\begin{itemize}
\tightlist
\item
  \textbf{Control Range}: From minimum to maximum power
\item
  \textbf{Advantages}: Precise control, high efficiency
\item
  \textbf{Applications}: DC motor control, heating elements, battery
  chargers
\end{itemize}

\end{solutionbox}
\begin{mnemonicbox}
``Resistor Regulates Rate, UJT Unleashes Pulses, SCR
Switches Current''

\end{mnemonicbox}
\subsection*{Question 5(a) [3 marks]}\label{q5a}

\textbf{Explain the hall effect sensor in BLDC driver circuit.}

\begin{solutionbox}
Hall effect sensors detect rotor position in BLDC
motors for precise commutation timing.

\begin{center}
\textbf{Mermaid Diagram (Code)}
\begin{verbatim}
{Shaded}
{Highlighting}[]
graph LR
    subgraph "BLDC Motor"
    R[Rotor with Magnets]
    S[Stator Windings]
    H1[Hall Sensor 1]
    H2[Hall Sensor 2]
    H3[Hall Sensor 3]
    end

    H1 {-{-} "Position Signal" {-}{-}{} C[Controller]}
    H2 {-{-} "Position Signal" {-}{-}{} C}
    H3 {-{-} "Position Signal" {-}{-}{} C}
    C {-{-} "Commutation Signal" {-}{-}{} D[Driver Circuit]}
    D {-{-} "Phase Current" {-}{-}{} S}
{Highlighting}
{Shaded}
\end{verbatim}
\end{center}

{\def\LTcaptype{none} % do not increment counter
\begin{longtable}[]{@{}
  >{\raggedright\arraybackslash}p{(\linewidth - 4\tabcolsep) * \real{0.4194}}
  >{\raggedright\arraybackslash}p{(\linewidth - 4\tabcolsep) * \real{0.3226}}
  >{\raggedright\arraybackslash}p{(\linewidth - 4\tabcolsep) * \real{0.2581}}@{}}
\toprule\noalign{}
\begin{minipage}[b]{\linewidth}\raggedright
Hall Sensor
\end{minipage} & \begin{minipage}[b]{\linewidth}\raggedright
Function
\end{minipage} & \begin{minipage}[b]{\linewidth}\raggedright
Output
\end{minipage} \\
\midrule\noalign{}
\endhead
\bottomrule\noalign{}
\endlastfoot
Position Detection & Senses magnetic field of rotor & Digital
(ON/OFF) \\
Placement & 120^\circ apart for 3-phase motors & Provides 6 unique states \\
Signal Processing & Inputs to microcontroller & Determines switching
sequence \\
\end{longtable}
}

\begin{itemize}
\tightlist
\item
  \textbf{Working Principle}: Voltage generated perpendicular to current
  and magnetic field
\item
  \textbf{Commutation Sequence}: Each sensor pattern corresponds to
  specific switching combination
\end{itemize}

\end{solutionbox}
\begin{mnemonicbox}
``Magnet Moves, Hall Senses, Controller Commutates''

\end{mnemonicbox}
\subsection*{Question 5(b) [4 marks]}\label{q5b}

\textbf{Draw and explain solid state circuit to control speed of single
phase Induction motor using TRIAC.}

\begin{solutionbox}
TRIAC-based speed control for induction motors uses
phase control principles.

\begin{center}
\textbf{Mermaid Diagram (Code)}
\begin{verbatim}
{Shaded}
{Highlighting}[]
graph LR
    AC[AC Supply] {-{-}{} F[Fuse]}
    F {-{-}{} T[TRIAC]}
    T {-{-}{} M[Induction Motor]}
    M {-{-}{} AC}

    AC {-{-} "Zero Crossing" {-}{-}{} ZC[Zero{-}Crossing Detector]}
    ZC {-{-}{} MC[Microcontroller]}
    MC {-{-}{} OI[Opto{-}Isolator]}
    OI {-{-}{} T}
    S[Speed Control] {-{-}{} MC}
{Highlighting}
{Shaded}
\end{verbatim}
\end{center}

\textbf{Working Principle:}

\begin{enumerate}
\tightlist
\item
  Zero-crossing detector identifies voltage zero-crossings
\item
  Microcontroller calculates delay based on speed setting
\item
  After delay, gate pulse sent through opto-isolator to TRIAC
\item
  TRIAC conducts for remainder of half-cycle
\item
  Varying firing angle controls voltage to motor, adjusting speed
\end{enumerate}

\begin{itemize}
\tightlist
\item
  \textbf{TRIAC Rating}: Must handle starting current (5-7\times running
  current)
\item
  \textbf{Speed Range}: Limited at low end due to motor characteristics
\item
  \textbf{Applications}: Fans, pumps, small machine tools
\end{itemize}

\end{solutionbox}
\begin{mnemonicbox}
``Zero Detected, Delay Determined, TRIAC Triggered''

\end{mnemonicbox}
\subsection*{Question 5(c) [7 marks]}\label{q5c}

\textbf{Explain the construction and working of BLDC motor using
diagram. Also enlist its applications.}

\begin{solutionbox}
Brushless DC motors use electronic commutation instead
of mechanical brushes.

\begin{center}
\textbf{Mermaid Diagram (Code)}
\begin{verbatim}
{Shaded}
{Highlighting}[]
graph LR
    subgraph "BLDC Motor Construction"
    S[Stator with Windings]
    R[Rotor with Permanent Magnets]
    H[Hall Effect Sensors]
    end

    subgraph "Control System"
    HS[Hall Sensor Signals] {-{-}{} C[Controller]}
    C {-{-}{} D[Driver Circuit]}
    D {-{-}{} S}
    end
{Highlighting}
{Shaded}
\end{verbatim}
\end{center}

{\def\LTcaptype{none} % do not increment counter
\begin{longtable}[]{@{}lll@{}}
\toprule\noalign{}
Component & Function & Types/Variations \\
\midrule\noalign{}
\endhead
\bottomrule\noalign{}
\endlastfoot
Stator & Contains copper windings & Slotted/slotless designs \\
Rotor & Permanent magnets & Surface/interior mounted \\
Hall Sensors & Position detection & 60^\circ/120^\circ configurations \\
Controller & Commutation logic & Microcontroller-based \\
Driver & Power switching & MOSFET/IGBT-based \\
\end{longtable}
}

\textbf{Working Principle:}

\begin{enumerate}
\tightlist
\item
  Hall sensors detect rotor position
\item
  Controller determines correct energizing sequence
\item
  Driver powers appropriate stator windings
\item
  Magnetic interaction produces rotation
\item
  Process repeats continuously
\end{enumerate}

\textbf{Applications:}

\begin{itemize}
\tightlist
\item
  Computer cooling fans and hard drives
\item
  Electric vehicles and hybrid cars
\item
  Industrial automation and robotics
\item
  Medical equipment (pumps, ventilators)
\item
  Drones and RC models
\item
  Home appliances (washers, refrigerators)
\item
  Precision instruments
\end{itemize}

\end{solutionbox}
\begin{mnemonicbox}
``Magnets Move, Sensors See, Electronics Energize''

\end{mnemonicbox}
\subsection*{Question 5(a OR) [3
marks]}\label{question-5a-or-3-marks}

\textbf{Explain the working of variable frequency drive (VFD).}

\begin{solutionbox}
Variable Frequency Drives control motor speed by
varying the frequency and voltage.

\begin{center}
\textbf{Mermaid Diagram (Code)}
\begin{verbatim}
{Shaded}
{Highlighting}[]
graph LR
    AC[AC Supply] {-{-}{} R[Rectifier]}
    R {-{-}{} DC[DC Bus]}
    DC {-{-}{} I[Inverter]}
    I {-{-}{} M[Motor]}

    C[Controller] {-{-}{} I}
    S[Speed Reference] {-{-}{} C}
    F[Feedback] {-{-}{} C}
{Highlighting}
{Shaded}
\end{verbatim}
\end{center}

{\def\LTcaptype{none} % do not increment counter
\begin{longtable}[]{@{}lll@{}}
\toprule\noalign{}
VFD Section & Function & Components \\
\midrule\noalign{}
\endhead
\bottomrule\noalign{}
\endlastfoot
Rectifier & Converts AC to DC & Diodes or SCRs \\
DC Bus & Filters and stores energy & Capacitors, inductors \\
Inverter & Converts DC to variable AC & IGBTs or MOSFETs \\
Controller & Manages frequency/voltage & Microprocessor \\
\end{longtable}
}

\begin{itemize}
\tightlist
\item
  \textbf{V/f Control}: Maintains constant V/f ratio for stable torque
\item
  \textbf{Operating Range}: Typically 10-200\% of rated speed
\item
  \textbf{Efficiency}: High efficiency across wide speed range
\end{itemize}

\end{solutionbox}
\begin{mnemonicbox}
``Rectify to DC, Invert to AC, Vary Frequency''

\end{mnemonicbox}
\subsection*{Question 5(b OR) [4
marks]}\label{question-5b-or-4-marks}

\textbf{Draw and explain the circuit to control speed of Universal
motor.}

\begin{solutionbox}
Universal motors can run on AC or DC and allow simple
speed control methods.

\begin{center}
\textbf{Mermaid Diagram (Code)}
\begin{verbatim}
{Shaded}
{Highlighting}[]
graph LR
    AC[AC Supply] {-{-}{} F[Fuse]}
    F {-{-}{} T[TRIAC]}
    T {-{-}{} M[Universal Motor]}
    M {-{-}{} AC}

    AC {-{-}{} R1[R1]}
    R1 {-{-}{} DIAC[DIAC]}
    DIAC {-{-}{} G[TRIAC Gate]}
    R1 {-{-}{} C1[C1]}
    C1 {-{-}{} P[Potentiometer]}
    P {-{-}{} F}
{Highlighting}
{Shaded}
\end{verbatim}
\end{center}

\textbf{Working Principle:}

\begin{enumerate}
\tightlist
\item
  RC network creates phase shift from input voltage
\item
  Potentiometer adjusts phase shift amount
\item
  DIAC triggers when voltage reaches breakover
\item
  TRIAC conducts for remainder of half-cycle
\item
  Adjusting potentiometer varies firing angle and motor speed
\end{enumerate}

\begin{itemize}
\tightlist
\item
  \textbf{Speed Range}: Wide control range (10-100\%)
\item
  \textbf{Torque Characteristics}: Decreases somewhat at lower speeds
\item
  \textbf{Applications}: Power tools, household appliances, sewing
  machines
\end{itemize}

\end{solutionbox}
\begin{mnemonicbox}
``Resistance Changes Phase, DIAC Delivers, TRIAC
Conducts''

\end{mnemonicbox}
\subsection*{Question 5(c OR) [7
marks]}\label{question-5c-or-7-marks}

\textbf{Draw the block diagram of PLC and explain the function of each
block in brief. And enlist the advantages and applications of it.}

\begin{solutionbox}
Programmable Logic Controllers (PLCs) are industrial
computers for automation control.

\begin{center}
\textbf{Mermaid Diagram (Code)}
\begin{verbatim}
{Shaded}
{Highlighting}[]
graph LR
    subgraph "PLC System"
    PS[Power Supply]
    CPU[Central Processing Unit]
    IM[Input Modules]
    OM[Output Modules]
    MEM[Memory]
    COM[Communication Interface]
    end

    PS {-{-}{} CPU}
    PS {-{-}{} IM}
    PS {-{-}{} OM}
    PS {-{-}{} COM}
    
    IM {-{-}{} CPU}
    CPU {-{-}{} OM}
    CPU {{-}{-}{} MEM}
    CPU {{-}{-}{} COM}
    
    FS[Field Sensors] {-{-}{} IM}
    OM {-{-}{} ACT[Actuators]}
    COM {{-}{-}{} HMI[HMI/SCADA]}
    COM {{-}{-}{} NET[Network]}
{Highlighting}
{Shaded}
\end{verbatim}
\end{center}

{\def\LTcaptype{none} % do not increment counter
\begin{longtable}[]{@{}
  >{\raggedright\arraybackslash}p{(\linewidth - 4\tabcolsep) * \real{0.2558}}
  >{\raggedright\arraybackslash}p{(\linewidth - 4\tabcolsep) * \real{0.2326}}
  >{\raggedright\arraybackslash}p{(\linewidth - 4\tabcolsep) * \real{0.5116}}@{}}
\toprule\noalign{}
\begin{minipage}[b]{\linewidth}\raggedright
PLC Block
\end{minipage} & \begin{minipage}[b]{\linewidth}\raggedright
Function
\end{minipage} & \begin{minipage}[b]{\linewidth}\raggedright
Types/Characteristics
\end{minipage} \\
\midrule\noalign{}
\endhead
\bottomrule\noalign{}
\endlastfoot
Power Supply & Provides regulated power & Typically 24VDC or
110/220VAC \\
CPU & Executes program, processes I/O & Scan-based operation \\
Input Modules & Interface with field sensors & Digital, analog,
special \\
Output Modules & Control field devices & Relay, transistor, triac \\
Memory & Stores program and data & RAM, EEPROM, Flash \\
Communication & Network connectivity & Ethernet, Profibus, Modbus \\
\end{longtable}
}

\textbf{Advantages:}

\begin{itemize}
\tightlist
\item
  Reliability in harsh industrial environments
\item
  Flexibility for reprogramming
\item
  Compact size compared to relay-based systems
\item
  Built-in diagnostics and troubleshooting
\item
  Modular expandability
\item
  High-speed operation
\item
  Cost-effective for complex control systems
\end{itemize}

\textbf{Applications:}

\begin{itemize}
\tightlist
\item
  Manufacturing production lines
\item
  Process control in plants
\item
  Material handling systems
\item
  Building automation
\item
  Power generation and distribution
\item
  Water/wastewater treatment
\item
  Packaging machinery
\item
  Food processing
\end{itemize}

\end{solutionbox}
\begin{mnemonicbox}
``Power Provides, CPU Computes, Inputs Inform,
Outputs Operate, Memory Maintains''

\end{mnemonicbox}

\end{document}
