\documentclass[10pt,a4paper]{article}

% content/resources/templates/preamble.tex
\usepackage[margin=0.6in]{geometry}
\author{Milav Dabgar}
\usepackage{amsmath,amssymb,amsthm}
\usepackage{booktabs}
\usepackage{multirow}
\usepackage{xcolor}
\usepackage{tcolorbox}
\tcbuselibrary{breakable,skins}
\usepackage[colorlinks=true,linkcolor=blue]{hyperref}
\usepackage{titlesec}
\usepackage{enumitem}
\usepackage{tikz}
\usepackage{pgfplots}
\usepackage{circuitikz}
\usepackage[version=4]{mhchem}
\usepackage{longtable}
\usepackage{array}
\usepackage{float}
\usepackage{caption}
\usepackage{listings}

\lstset{
  basicstyle=\small\ttfamily,
  breaklines=true,
  breakatwhitespace=false,
  postbreak=\mbox{\textcolor{red}{$\hookrightarrow$}\space},
  float=false,
  numbers=left,
  numberstyle=\tiny\color{gray},
  numbersep=10pt,
  xleftmargin=2em,
  keywordstyle=\color{blue},
  commentstyle=\color{green!60!black},
  stringstyle=\color{purple},
  backgroundcolor=\color{gray!5},
  showstringspaces=false,
  tabsize=2,
  captionpos=b,
  keepspaces=true,
  columns=flexible
}

\pgfplotsset{compat=1.18}
\usetikzlibrary{shapes,arrows,positioning,calc,patterns,decorations.pathmorphing,decorations.markings,arrows.meta}

% Color scheme
\definecolor{headcolor}{RGB}{0,102,204}
\definecolor{keycolor}{RGB}{220,20,60}
\definecolor{solutioncolor}{RGB}{34,139,34}
\definecolor{mnemoniccolor}{RGB}{148,0,211}
\definecolor{codecolor}{RGB}{0,0,100}

% Spacing
\setlength{\parskip}{3pt}
\setlist[itemize]{nosep}
\setlist[enumerate]{nosep}

% Title formatting
\titleformat{\section}{\Large\bfseries\color{headcolor}}{\thesection}{1em}{}
\titleformat{\subsection}{\large\bfseries\color{headcolor}}{\thesubsection}{1em}{}

% Pandoc tightlist compatibility
\providecommand{\tightlist}{%
  \setlength{\itemsep}{0pt}\setlength{\parskip}{0pt}}

% Pandoc longtable compatibility
\newcounter{none}
\def\thenone{}


% content/resources/templates/english-boxes.tex
% This file is currently empty - it exists to maintain consistency with the import structure.
% Add custom environments here if needed in the future.


\begin{document}

\begin{center}
{\Huge\bfseries\color{headcolor} Subject Name Solutions}\\[5pt]
{\LARGE 4331103 -- Summer 2023}\\[3pt]
{\large Semester 1 Study Material}\\[3pt]
{\normalsize\textit{Detailed Solutions and Explanations}}
\end{center}

\vspace{10pt}

\subsection*{Question 1(a) [3 marks]}\label{q1a}

\textbf{Draw and Explain the V-I Characteristics of TRIAC.}

\begin{solutionbox}
TRIAC (Triode for Alternating Current) is a
bidirectional three-terminal semiconductor device that can conduct
current in either direction when triggered.

\textbf{Diagram:}

\begin{verbatim}
      I
      ↑
      │        MT2
      │        /│{}
      │       / │ {}
      │      /  │  {}
Quadrant III /   G   { Quadrant I}
      │    /    │    {}
      │   /     │     {}
──────┼──/──────┼──────{──── V}
      │ /       │       {}
      │/        │        {}
      │{        │        /}
      │ {       │       /}
      │  {      │      /}
Quadrant IV {   │     / Quadrant II}
      │    {    │    /}
      │     {   │   /}
      ↓      {  │  /}
             { │ /}
              {│/}
              MT1
\end{verbatim}

\begin{itemize}
\tightlist
\item
  \textbf{Bidirectional operation}: TRIAC conducts in both directions
  (positive and negative half cycles)
\item
  \textbf{Quadrant operation}: Functions in all four quadrants based on
  polarity of MT2 and gate
\item
  \textbf{Triggering voltage}: Breakdown occurs at \pmVBO in either
  direction
\item
  \textbf{Holding current}: Minimum current to maintain conduction
\end{itemize}

\end{solutionbox}
\begin{mnemonicbox}
``Two Rectifiers In A Case''

\end{mnemonicbox}
\subsection*{Question 1(b) [4 marks]}\label{q1b}

\textbf{Explain working of SCR using two transistor analogy.}

\begin{solutionbox}
SCR (Silicon Controlled Rectifier) can be represented
as interconnected PNP and NPN transistors.

\textbf{Diagram:}

\begin{verbatim}
         Anode
           │
           │
         ┌─┴─┐
         │   │
     ┌───┤ P ├───┐
     │   │   │   │
     │   └───┘   │
     │           │
     │   ┌───┐   │
     └───┤ N ├───┘
         │   │
     ┌───┤   ├───┐
     │   └───┘   │
     │           │
     │   ┌───┐   │
     └───┤ P ├───┘
         │   │
         └─┬─┘
           │
           │
        Cathode
\end{verbatim}

\begin{itemize}
\tightlist
\item
  \textbf{Two-transistor structure}: PNP (Q1) and NPN (Q2) connected
  such that collector of each transistor drives the base of other
\item
  \textbf{Regenerative feedback}: Once both transistors start
  conducting, they keep each other in saturation
\item
  \textbf{Triggering}: Applying gate current to Q2 base starts the
  regenerative process
\item
  \textbf{Latching}: Once triggered, SCR remains ON even if gate signal
  is removed
\end{itemize}

\end{solutionbox}
\begin{mnemonicbox}
``Pull Neat Path''

\end{mnemonicbox}
\subsection*{Question 1(c) [7 marks]}\label{q1c}

\textbf{Draw the circuit diagram of photo electric relay using LDR and
explain it Working.}

\begin{solutionbox}
A photoelectric relay using LDR (Light Dependent
Resistor) is a light-activated switching circuit.

\textbf{Circuit Diagram:}

\begin{verbatim}
     +Vcc
      │
      ├───────┐
      │       │
      R1      │
      │       │
      │       │
      ├───────┤
      │       │
      │      LDR
      │       │
      │       │
    ┌─┴─┐     │
    │ B │     │
 ───┤   ├─────┘
    │ C │
    └─┬─┘
      │
      │
      ├─────────┐
      │         │
      R2      Relay
      │         │
      │         │
      └─────────┴───── GND
\end{verbatim}

\begin{itemize}
\tightlist
\item
  \textbf{Light sensing}: LDR resistance decreases in presence of light
\item
  \textbf{Transistor operation}: When light falls on LDR, voltage at
  transistor base changes
\item
  \textbf{Relay switching}: Transistor conducts/cuts off based on light,
  activating/deactivating relay
\item
  \textbf{Threshold adjustment}: Potentiometer R1 sets light sensitivity
  level
\item
  \textbf{Applications}: Automatic street lights, burglar alarms,
  automatic door openers
\end{itemize}

\end{solutionbox}
\begin{mnemonicbox}
``Light Detects Readily''

\end{mnemonicbox}
\subsection*{Question 1(c OR) [7
marks]}\label{question-1c-or-7-marks}

\textbf{Draw the gate pulse trigger circuit using UJT for SCR and
explain its working.}

\begin{solutionbox}
UJT (Unijunction Transistor) provides reliable trigger
pulses for SCR.

\textbf{Circuit Diagram:}

\begin{verbatim}
        +Vcc
         │
         │
         R1
         │
         │
    ┌────┴────┐
    │         │
    │        B2
    │    UJT  │
    │         │
    │    B1   │
    └────┬────┘
         │
         R3
         │
     ┌───┴────┐
     │        │
     C       SCR Gate
     │        │
     │        │
     └────────┴──── GND
\end{verbatim}

\begin{itemize}
\tightlist
\item
  \textbf{RC timing}: R1 and C form charging circuit that determines
  pulse frequency
\item
  \textbf{UJT operation}: UJT fires when capacitor voltage reaches peak
  point voltage
\item
  \textbf{Pulse generation}: UJT discharges capacitor producing sharp
  trigger pulse
\item
  \textbf{SCR triggering}: Pulse applied to SCR gate turns it ON at
  specific points in AC cycle
\item
  \textbf{Frequency control}: Adjusting R1 changes pulse frequency for
  phase control
\end{itemize}

\end{solutionbox}
\begin{mnemonicbox}
``Uniform Junctions Trigger''

\end{mnemonicbox}
\subsection*{Question 2(a) [3 marks]}\label{q2a}

\textbf{State Triggering methods of SCR.}

\begin{solutionbox}

{\def\LTcaptype{none} % do not increment counter
\begin{longtable}[]{@{}
  >{\raggedright\arraybackslash}p{(\linewidth - 4\tabcolsep) * \real{0.3654}}
  >{\raggedright\arraybackslash}p{(\linewidth - 4\tabcolsep) * \real{0.4038}}
  >{\raggedright\arraybackslash}p{(\linewidth - 4\tabcolsep) * \real{0.2308}}@{}}
\toprule\noalign{}
\begin{minipage}[b]{\linewidth}\raggedright
Triggering Method
\end{minipage} & \begin{minipage}[b]{\linewidth}\raggedright
Operating Principle
\end{minipage} & \begin{minipage}[b]{\linewidth}\raggedright
Advantages
\end{minipage} \\
\midrule\noalign{}
\endhead
\bottomrule\noalign{}
\endlastfoot
Gate Triggering & Current applied to gate terminal & Most common,
precise control \\
Thermal Triggering & Temperature rise causes leakage & Simple, no
external circuit \\
Light Triggering & Photons create electron-hole pairs & Electrical
isolation, used in LASCRs \\
dv/dt Triggering & Rapid voltage rise causes turn-on & Useful for
protection circuits \\
Forward Voltage Triggering & Exceeding breakover voltage & No gate
connection needed \\
\end{longtable}
}

\end{solutionbox}
\begin{mnemonicbox}
``Good Triggers Let Devices Fire''

\end{mnemonicbox}
\subsection*{Question 2(b) [4 marks]}\label{q2b}

\textbf{What is Commutation of SCR? Explain class-E commutation.}

\begin{solutionbox}
Commutation is the process of turning OFF an SCR by
reducing its anode current below holding current.

\textbf{Class-E Commutation (Complementary Commutation):}

\begin{verbatim}
              L1
    AC   ┌────┐────┐
Source   │    │    │
    ┌────┴─┐  │  ┌─┴────┐
    │      │  │  │      │
    │ SCR1 │  │  │ SCR2 │
    │      │  │  │      │
    └────┬─┘  │  └─┬────┘
         │    │    │
         └────┘────┘
              Load
\end{verbatim}

\begin{itemize}
\tightlist
\item
  \textbf{Complementary switching}: Uses another SCR in opposite
  half-cycle
\item
  \textbf{Natural commutation}: AC source crosses zero, anode current
  falls below holding current
\item
  \textbf{Application}: AC power control circuits, cycloconverters
\item
  \textbf{Advantage}: No additional commutation components required
\end{itemize}

\end{solutionbox}
\begin{mnemonicbox}
``Complementary Elements''

\end{mnemonicbox}
\subsection*{Question 2(c) [7 marks]}\label{q2c}

\textbf{Draw and explain Snubber Circuit for SCR.}

\begin{solutionbox}
A snubber circuit protects SCR from voltage transients
and dv/dt turn-on.

\textbf{Circuit Diagram:}

\begin{verbatim}
         ┌─────────┐
         │         │
AC    ┌──┴──┐     Rs    
Source│     │     ├───┐
  ┌───┤ SCR ├─────┘   │
  │   │     │         │
  │   └──┬──┘         │
  │      │          Cs│
  │      │            │
  │      │            │
  │      └────────────┘
  │
  └─────────────────── Load
\end{verbatim}

\begin{itemize}
\tightlist
\item
  \textbf{RC network}: Series resistor (Rs) and capacitor (Cs) connected
  across SCR
\item
  \textbf{Transient suppression}: Capacitor absorbs voltage spikes that
  could damage SCR
\item
  \textbf{dv/dt protection}: Prevents false triggering due to rapid
  voltage rise
\item
  \textbf{Turn-off assistance}: Helps in commutation by providing
  alternate current path
\item
  \textbf{Component selection}: Cs based on load current, Rs limits
  discharge current
\end{itemize}

\end{solutionbox}
\begin{mnemonicbox}
``Safely Neutralizes Unwanted Breakover''

\end{mnemonicbox}
\subsection*{Question 2(a OR) [3
marks]}\label{question-2a-or-3-marks}

\textbf{Explain over current protection method of SCR.}

\begin{solutionbox}

{\def\LTcaptype{none} % do not increment counter
\begin{longtable}[]{@{}
  >{\raggedright\arraybackslash}p{(\linewidth - 4\tabcolsep) * \real{0.3654}}
  >{\raggedright\arraybackslash}p{(\linewidth - 4\tabcolsep) * \real{0.3654}}
  >{\raggedright\arraybackslash}p{(\linewidth - 4\tabcolsep) * \real{0.2692}}@{}}
\toprule\noalign{}
\begin{minipage}[b]{\linewidth}\raggedright
Protection Method
\end{minipage} & \begin{minipage}[b]{\linewidth}\raggedright
Working Principle
\end{minipage} & \begin{minipage}[b]{\linewidth}\raggedright
Applications
\end{minipage} \\
\midrule\noalign{}
\endhead
\bottomrule\noalign{}
\endlastfoot
Fuses & Melts when current exceeds rating & Simple, economical
protection \\
Circuit Breakers & Trips on overload, can be reset & Reusable
protection \\
Current Limiting Reactors & Limits fault current magnitude & Industrial
power control \\
Electronic Current Limiting & Senses current and controls gate & Precise
protection \\
Crowbar Circuit & Shorts power supply on overload & Protects sensitive
loads \\
\end{longtable}
}

\end{solutionbox}
\begin{mnemonicbox}
``Fault Current Causes Equipment Damage''

\end{mnemonicbox}
\subsection*{Question 2(b OR) [4
marks]}\label{question-2b-or-4-marks}

\textbf{Explain the working of opto-SCR.}

\begin{solutionbox}
An opto-SCR (or Light Activated SCR) combines a light
source and SCR in an isolated package.

\textbf{Diagram:}

\begin{verbatim}
      ┌───────────────┐
      │   ┌───┐       │
      │   │   │       │
 LED  │   │ ◄─┼───┐   │
Anode ├───┤LED│   │   │
      │   │   │   │   │
      │   └───┘   │   │
 LED  │           │   │
Cathod├───────────┘   │
      │               │
      │      ┌───┐    │
      │      │   │ SCR│
  SCR │      │ S ├───Anode
 Gate ├──────┤   │    │
      │      │ C │    │
      │      │ R │    │
  SCR │      └───┘    │
Cathod├───────────────┘
      │               │
      └───────────────┘
\end{verbatim}

\begin{itemize}
\tightlist
\item
  \textbf{Electrical isolation}: LED optically triggers SCR without
  electrical connection
\item
  \textbf{Noise immunity}: Immune to electrical noise and interference
\item
  \textbf{High-voltage isolation}: Separates control and power circuits
\item
  \textbf{Applications}: Industrial control, high-voltage switching
\end{itemize}

\end{solutionbox}
\begin{mnemonicbox}
``Light Activates Silicon Control''

\end{mnemonicbox}
\subsection*{Question 2(c OR) [7
marks]}\label{question-2c-or-7-marks}

\textbf{What is force commutation? Explain any two.}

\begin{solutionbox}
Force commutation is artificially turning OFF an SCR by
reducing its anode current below holding level.

\textbf{1. Class A Commutation (Self-Commutation):}

\begin{verbatim}
    ┌───┐
    │   │    L
    │   ├────┐─────┐
AC  │   │    │     │
Sour┤   │    │   SCR
    │   │    │     │
    └───┘    C     │
             │     │
             └─────┘
               Load
\end{verbatim}

\begin{itemize}
\tightlist
\item
  \textbf{LC resonant circuit}: Parallel L-C across SCR creates
  oscillations
\item
  \textbf{Reverse current}: L-C circuit forces reverse current through
  SCR
\item
  \textbf{Applications}: Inverters, choppers
\end{itemize}

\textbf{2. Class B Commutation (Resonant Pulse Commutation):}

\begin{verbatim}
              Commutating
                Switch
    ┌───┐      ┌───┐
    │   │      │   │
AC  │   │    L │   │
Sour┤   ├────┐─┴──┐
    │   │    │    │
    └───┘    │    │
            SCR   C
             │    │
             └────┘
              Load
\end{verbatim}

\begin{itemize}
\tightlist
\item
  \textbf{External switch}: Additional SCR or switch triggers
  commutation
\item
  \textbf{Energy storage}: L-C circuit stores energy then reverses SCR
  current
\item
  \textbf{Applications}: DC choppers, controlled rectifiers
\end{itemize}

\end{solutionbox}
\begin{mnemonicbox}
``Force Circuit Reversal''

\end{mnemonicbox}
\subsection*{Question 3(a) [3 marks]}\label{q3a}

\textbf{Explain 1-φ full Wave bridge-controlled rectifier using four
diodes \& one SCR.}

\begin{solutionbox}
This circuit combines diodes and an SCR for controlled
single-phase full-wave rectification.

\textbf{Circuit Diagram:}

\begin{verbatim}
         D1        D2
     ┌───┬───┐───┬───┐
     │   │   │   │   │
     │   ▼   │   ▼   │
     │       │       │
AC   │       │       │    Load
Sourc┤       │       ├───R───┐
     │       │       │       │
     │   ▲   │   ▲   │       │
     │   │   │   │   │       │
     └───┴───┘───┴───┘       │
        D3   SCR    D4       │
                            GND
\end{verbatim}

\begin{itemize}
\tightlist
\item
  \textbf{Bridge configuration}: Four diodes arranged in bridge with one
  replaced by SCR
\item
  \textbf{Variable output}: SCR controls conduction angle and thus
  output voltage
\item
  \textbf{Economical design}: Uses only one SCR instead of two or four
\item
  \textbf{Efficiency}: Higher than half-wave controlled rectifier
\end{itemize}

\end{solutionbox}
\begin{mnemonicbox}
``Blend Diodes Smartly''

\end{mnemonicbox}
\subsection*{Question 3(b) [4 marks]}\label{q3b}

\textbf{What is Chopper? What are its application?}

\begin{solutionbox}

{\def\LTcaptype{none} % do not increment counter
\begin{longtable}[]{@{}
  >{\raggedright\arraybackslash}p{(\linewidth - 2\tabcolsep) * \real{0.3810}}
  >{\raggedright\arraybackslash}p{(\linewidth - 2\tabcolsep) * \real{0.6190}}@{}}
\toprule\noalign{}
\begin{minipage}[b]{\linewidth}\raggedright
Aspect
\end{minipage} & \begin{minipage}[b]{\linewidth}\raggedright
Description
\end{minipage} \\
\midrule\noalign{}
\endhead
\bottomrule\noalign{}
\endlastfoot
Definition & DC-DC converter that converts fixed DC input to variable DC
output \\
Working Principle & Periodically switches DC input ON/OFF at high
frequency \\
Types & Step-down (Buck), Step-up (Boost), Buck-Boost, Cuk \\
Control Methods & PWM, Frequency modulation, Current-limit control \\
Applications & DC motor speed control, Battery chargers, UPS, Solar
systems, Electric vehicles \\
\end{longtable}
}

\end{solutionbox}
\begin{mnemonicbox}
``Chops Current Perfectly''

\end{mnemonicbox}
\subsection*{Question 3(c) [7 marks]}\label{q3c}

\textbf{Draw and explain the circuit diagram of static switch using SCR
for 1-φ A.C. Load.}

\begin{solutionbox}
A static switch using SCR provides non-mechanical
switching for AC loads.

\textbf{Circuit Diagram:}

\begin{verbatim}
              SCR1
             ┌──┐
     ┌───────┤  ├────┐
     │       └──┘    │
     │               │
AC   │               │   AC
Sourc┤               ├── Load
     │               │
     │       ┌──┐    │
     └───────┤  ├────┘
             └──┘
              SCR2
               │
               │
               │
          ┌────┴────┐
          │ Trigger │
          │ Circuit │
          └─────────┘
\end{verbatim}

\begin{itemize}
\tightlist
\item
  \textbf{Antiparallel SCRs}: Two SCRs connected in inverse parallel for
  bidirectional conduction
\item
  \textbf{Gate control}: Properly timed gate signals control power to
  load
\item
  \textbf{Zero-crossing switching}: SCRs naturally turn OFF at zero
  crossing
\item
  \textbf{Applications}: Heater control, motor soft-starting, lighting
  control
\item
  \textbf{Advantages}: No moving parts, silent operation, long life
\end{itemize}

\end{solutionbox}
\begin{mnemonicbox}
``Solid Switching Technology''

\end{mnemonicbox}
\subsection*{Question 3(a OR) [3
marks]}\label{question-3a-or-3-marks}

\textbf{Explain basic principle of DC Chopper.}

\begin{solutionbox}

{\def\LTcaptype{none} % do not increment counter
\begin{longtable}[]{@{}ll@{}}
\toprule\noalign{}
Component & Function \\
\midrule\noalign{}
\endhead
\bottomrule\noalign{}
\endlastfoot
Switching Device & SCR, MOSFET, IGBT switches DC at high frequency \\
Control Circuit & Generates PWM gate signals to control ON/OFF time \\
Duty Cycle & Ratio of ON time to total time period determines output \\
Output Filter & Smooths chopped output to reduce ripple \\
Working Principle & Average voltage = Input voltage \times Duty cycle \\
\end{longtable}
}

\end{solutionbox}
\begin{mnemonicbox}
``Direct Current Control''

\end{mnemonicbox}
\subsection*{Question 3(b OR) [4
marks]}\label{question-3b-or-4-marks}

\textbf{Write short note on: Un-interrupted Power Supply (UPS).}

\begin{solutionbox}
UPS provides emergency power when main supply fails.

\textbf{Block Diagram:}

\begin{verbatim}
    ┌─────────┐    ┌─────────┐    ┌─────────┐
    │  Mains  │    │Rectifier│    │ Inverter│
    │  Input  ├────┤  \& DC   ├────┤  \& AC   ├─── Output
    │ (AC)    │    │ Section │    │ Section │    (AC)
    └─────────┘    └─────────┘    └─────────┘
                        │
                    ┌───┴───┐
                    │Battery│
                    │System │
                    └───────┘
\end{verbatim}

\begin{itemize}
\tightlist
\item
  \textbf{Backup power}: Provides continuous power during outages
\item
  \textbf{Types}: Online, Offline, Line-interactive UPS
\item
  \textbf{Protection}: Against power surges, sags, and frequency
  variations
\item
  \textbf{Applications}: Computers, medical equipment,
  telecommunications
\end{itemize}

\end{solutionbox}
\begin{mnemonicbox}
``Uninterrupted Power Securely''

\end{mnemonicbox}
\subsection*{Question 3(c OR) [7
marks]}\label{question-3c-or-7-marks}

\textbf{Draw the block diagram of SMPS and explain the function of each
block.}

\begin{solutionbox}
Switched-Mode Power Supply converts AC to regulated DC
efficiently.

\textbf{Block Diagram:}

\begin{verbatim}
    ┌─────────┐    ┌─────────┐    ┌─────────┐    ┌─────────┐    ┌─────────┐
    │  Mains  │    │  Input  │    │High{-Freq│    │Output   │    │Output   │}
    │  Input  ├────┤Rectifier├────┤Switching├────┤Rectifier├────┤ Filter  ├─── DC Output
    │  (AC)   │    │\& Filter │    │ Circuit │    │\& Filter │    │         │
    └─────────┘    └─────────┘    └─────────┘    └─────────┘    └─────────┘
                                       │
                                  ┌────┴────┐
                                  │ Control │
                                  │ Circuit │
                                  └─────────┘
\end{verbatim}

\begin{itemize}
\tightlist
\item
  \textbf{Input rectifier}: Converts AC to unregulated DC
\item
  \textbf{High-frequency switching}: Converts DC to high-frequency AC
  using transistors
\item
  \textbf{Transformer}: Provides isolation and voltage scaling
\item
  \textbf{Output rectifier}: Converts high-frequency AC to DC
\item
  \textbf{Filter}: Smooths DC output to reduce ripple
\item
  \textbf{Control circuit}: Regulates output through feedback
\end{itemize}

\end{solutionbox}
\begin{mnemonicbox}
``Switch Mode Power System''

\end{mnemonicbox}
\subsection*{Question 4(a) [3 marks]}\label{q4a}

\textbf{Draw the circuit diagram using TRIAC for speed control of 1-φ DC
Shunt motor and Explain its working.}

\begin{solutionbox}
TRIAC-based speed control for a DC shunt motor provides
efficient variable speed.

\textbf{Circuit Diagram:}

\begin{verbatim}
     ┌────────┐   ┌────────┐      ┌───────┐
AC   │        │   │        │      │ DC    │
Sourc┤ TRIAC  ├───┤ Bridge ├──────┤ Shunt │
     │        │   │Rectifir│      │ Motor │
     └────────┘   └────────┘      └───────┘
         │
     ┌───┴───┐
     │ DIAC  │
     │       │
     └───┬───┘
         │
     ┌───┴───┐
     │       │
     │   R   │
     │       │
     └───┬───┘
         │
     ┌───┴───┐
     │   C   │
     │       │
     └───────┘
\end{verbatim}

\begin{itemize}
\tightlist
\item
  \textbf{Phase control}: TRIAC varies effective voltage through phase
  angle control
\item
  \textbf{Rectification}: Bridge rectifier converts AC to DC for motor
\item
  \textbf{Speed variation}: Motor speed proportional to applied voltage
\item
  \textbf{RC timing}: RC network determines firing angle of TRIAC
\end{itemize}

\end{solutionbox}
\begin{mnemonicbox}
``TRIAC Regulates Speed''

\end{mnemonicbox}
\subsection*{Question 4(b) [4 marks]}\label{q4b}

\textbf{Draw and explain the circuit diagram four stage sequential timer
using IC-556.}

\begin{solutionbox}
IC-556 dual timer can be configured as a multi-stage
sequential timer.

\textbf{Circuit Diagram:}

\begin{verbatim}
    Vcc
     │
     ├─────┬─────┬─────┬─────┐
     │     │     │     │     │
    R1    R2    R3    R4     │
     │     │     │     │     │
     ├─────┴─────┴─────┴─────┤
     │                       │
     │       IC{-556          │}
     │                       │
     ├───┬───┬───┬───────────┤
     │   │   │   │           │
     C1  C2  C3  C4          │
     │   │   │   │           │
     └───┴───┴───┴───────────┘
         │   │   │
         O1  O2  O3  O4
\end{verbatim}

\begin{itemize}
\tightlist
\item
  \textbf{Dual timer IC}: IC-556 contains two 555 timer circuits
\item
  \textbf{Cascaded configuration}: Output of one stage triggers the next
\item
  \textbf{Timing control}: RC time constants determine duration of each
  stage
\item
  \textbf{Applications}: Industrial sequencing, process control,
  automation
\end{itemize}

\end{solutionbox}
\begin{mnemonicbox}
``Sequential Steps Timed Precisely''

\end{mnemonicbox}
\subsection*{Question 4(c) [7 marks]}\label{q4c}

\textbf{Explain induction heating.}

\begin{solutionbox}
Induction heating is a non-contact heating process
using electromagnetic induction.

\textbf{Diagram:}

\begin{verbatim}
    ┌───────────────┐
    │ High{-Frequency│}
    │ Power Supply  │
    └───────┬───────┘
            │
    ┌───────┴───────┐
    │    Induction  │
    │     Coil      │
    └───────┬───────┘
            │
    ┌───────┴───────┐
    │   Workpiece   │
    │  (Conductive  │
    │   Material)   │
    └───────────────┘
\end{verbatim}

{\def\LTcaptype{none} % do not increment counter
\begin{longtable}[]{@{}
  >{\raggedright\arraybackslash}p{(\linewidth - 2\tabcolsep) * \real{0.4583}}
  >{\raggedright\arraybackslash}p{(\linewidth - 2\tabcolsep) * \real{0.5417}}@{}}
\toprule\noalign{}
\begin{minipage}[b]{\linewidth}\raggedright
Principle
\end{minipage} & \begin{minipage}[b]{\linewidth}\raggedright
Description
\end{minipage} \\
\midrule\noalign{}
\endhead
\bottomrule\noalign{}
\endlastfoot
Electromagnetic Induction & AC in coil creates alternating magnetic
field \\
Eddy Currents & Magnetic field induces currents in workpiece \\
Resistive Heating & Eddy currents generate heat due to material
resistance \\
Skin Effect & Current concentrates near surface at high frequencies \\
Applications & Heat treatment, melting, forging, brazing, cooking \\
\end{longtable}
}

\end{solutionbox}
\begin{mnemonicbox}
``Induced Heating Efficiently''

\end{mnemonicbox}
\subsection*{Question 4(a OR) [3
marks]}\label{question-4a-or-3-marks}

\textbf{Draw and explain three stage IC555 timer circuit.}

\begin{solutionbox}
A three-stage timer using IC555 provides sequential
timing operations.

\textbf{Circuit Diagram:}

\begin{verbatim}
                Vcc
                 │
         ┌───────┴───────┐
         │ Reset         │
    ┌────┤4          8├──┐
    │    │            │  │
    │ ┌──┤2  IC555   3├──┴────┐
    │ │  │             │      │
  R1│ │  │7            │      │
    │ │  │             │     R4
    │ │  │6            │      │
    ├─┘  │             │      │
    │   C1            C2      │
    │    │             │      │
    └────┴──────┬──────┴──────┘
                │
                O1
\end{verbatim}

\begin{itemize}
\tightlist
\item
  \textbf{Monostable mode}: Each stage operates in monostable mode with
  fixed time delay
\item
  \textbf{Cascaded connection}: Output of first timer triggers second,
  and so on
\item
  \textbf{Timing components}: R-C network determines time delay of each
  stage
\item
  \textbf{Applications}: Automatic sequencing, process timing,
  industrial control
\end{itemize}

\end{solutionbox}
\begin{mnemonicbox}
``Time Intervals Created''

\end{mnemonicbox}
\subsection*{Question 4(b OR) [4
marks]}\label{question-4b-or-4-marks}

\textbf{Explain the principle of dielectric heating.}

\begin{solutionbox}

{\def\LTcaptype{none} % do not increment counter
\begin{longtable}[]{@{}
  >{\raggedright\arraybackslash}p{(\linewidth - 2\tabcolsep) * \real{0.4583}}
  >{\raggedright\arraybackslash}p{(\linewidth - 2\tabcolsep) * \real{0.5417}}@{}}
\toprule\noalign{}
\begin{minipage}[b]{\linewidth}\raggedright
Principle
\end{minipage} & \begin{minipage}[b]{\linewidth}\raggedright
Description
\end{minipage} \\
\midrule\noalign{}
\endhead
\bottomrule\noalign{}
\endlastfoot
High-Frequency Electric Field & Material placed between electrodes with
RF voltage (1-100 MHz) \\
Molecular Friction & Dipole molecules vibrate/rotate trying to align
with alternating field \\
Heat Generation & Internal friction between molecules generates heat
uniformly \\
Non-Conductive Materials & Effective for heating non-conductive
materials (plastics, wood, food) \\
Applications & Plastic welding, wood drying, food processing (microwave
ovens) \\
\end{longtable}
}

\end{solutionbox}
\begin{mnemonicbox}
``Dielectric Energy Heats''

\end{mnemonicbox}
\subsection*{Question 4(c OR) [7
marks]}\label{question-4c-or-7-marks}

\textbf{Make comparison between Induction heating and Dielectric
heating.}

\begin{solutionbox}

{\def\LTcaptype{none} % do not increment counter
\begin{longtable}[]{@{}
  >{\raggedright\arraybackslash}p{(\linewidth - 4\tabcolsep) * \real{0.2245}}
  >{\raggedright\arraybackslash}p{(\linewidth - 4\tabcolsep) * \real{0.3878}}
  >{\raggedright\arraybackslash}p{(\linewidth - 4\tabcolsep) * \real{0.3878}}@{}}
\toprule\noalign{}
\begin{minipage}[b]{\linewidth}\raggedright
Parameter
\end{minipage} & \begin{minipage}[b]{\linewidth}\raggedright
Induction Heating
\end{minipage} & \begin{minipage}[b]{\linewidth}\raggedright
Dielectric Heating
\end{minipage} \\
\midrule\noalign{}
\endhead
\bottomrule\noalign{}
\endlastfoot
Basic Principle & Electromagnetic induction & High-frequency electric
field \\
Suitable Materials & Conductive materials (metals) & Non-conductive
materials (plastics, wood) \\
Frequency Range & 1 kHz to 1 MHz & 1 MHz to 1 GHz \\
Heating Mechanism & Eddy currents and hysteresis & Molecular friction
(dipole rotation) \\
Heat Distribution & Surface heating (skin effect) & Volumetric (uniform
throughout) \\
Efficiency & 80-90\% for magnetic materials & 50-70\% depending on
material \\
Applications & Metal melting, forging, heat treatment & Plastic welding,
food processing, drying \\
Equipment & Induction coil, work piece & Electrodes, dielectric
material \\
\end{longtable}
}

\end{solutionbox}
\begin{mnemonicbox}
``ICED'' - Induction Conductive, Eddy currents;
Dielectric, Dipoles

\end{mnemonicbox}
\subsection*{Question 5(a) [3 marks]}\label{q5a}

\textbf{Explain Construction and working of Universal Motor.}

\begin{solutionbox}
Universal motor operates on both AC and DC power
sources.

\textbf{Diagram:}

\begin{verbatim}
       ┌───┐
       │   │
       │   │
       │   │
 ┌─────┴───┴─────┐
 │ Field Winding │
 │   ┌─────┐     │
 │   │     │     │
 │   │Rotor│     │
 │   │     │     │
 │   └─────┘     │
 │               │
 └───────────────┘
      Brushes
\end{verbatim}

\begin{itemize}
\tightlist
\item
  \textbf{Series connection}: Field winding in series with armature
  winding
\item
  \textbf{Construction}: Stator with field winding, rotor with
  commutator and brushes
\item
  \textbf{Operating principle}: Same direction torque on both AC and DC
\item
  \textbf{Characteristics}: High starting torque, high speed at low load
\item
  \textbf{Applications}: Portable tools, household appliances, blenders
\end{itemize}

\end{solutionbox}
\begin{mnemonicbox}
``Universally Motorized''

\end{mnemonicbox}
\subsection*{Question 5(b) [4 marks]}\label{q5b}

\textbf{Draw and explain the construction of DC servo motor.}

\begin{solutionbox}
DC servo motor provides precise position or speed
control.

\textbf{Diagram:}

\begin{verbatim}
     ┌─────────────┐
     │  Permanent  │
     │   Magnet    │
     │   Stator    │
     │   ┌─────┐   │
     │   │     │   │
     │   │Rotor│   │
     │   │     │   │
     │   └─────┘   │
     │             │
     └─────┬───────┘
           │
     ┌─────┴─────┐
     │  Encoder  │
     │  Feedback │
     └───────────┘
\end{verbatim}

\begin{itemize}
\tightlist
\item
  \textbf{Construction}: Permanent magnet stator, lightweight rotor,
  feedback device
\item
  \textbf{Control system}: Closed-loop control with position/velocity
  feedback
\item
  \textbf{Low inertia}: Allows quick response and precise positioning
\item
  \textbf{Applications}: Robotics, CNC machines, positioning systems
\item
  \textbf{Features}: High torque-to-inertia ratio, fast response,
  accuracy
\end{itemize}

\end{solutionbox}
\begin{mnemonicbox}
``Servo System Control''

\end{mnemonicbox}
\subsection*{Question 5(c) [7 marks]}\label{q5c}

\textbf{Draw the block diagram of Programmable logic Control (PLC) and
explain the Function of each block.}

\begin{solutionbox}
PLC is an industrial digital computer for automation
control.

\textbf{Block Diagram:}

\begin{verbatim}
    ┌─────────────┐    ┌─────────────┐    ┌─────────────┐
    │             │    │             │    │             │
    │   Input     │    │  Central    │    │   Output    │
    │   Modules   ├────┤ Processing  ├────┤   Modules   │
    │             │    │   Unit      │    │             │
    └─────────────┘    └─────┬───────┘    └─────────────┘
                             │
         ┌───────────────────┼──────────────────┐
         │                   │                  │
    ┌────┴─────┐       ┌─────┴─────┐       ┌────┴─────┐
    │  Memory  │       │Programming│       │  Power   │
    │  Unit    │       │  Device   │       │  Supply  │
    └──────────┘       └───────────┘       └──────────┘
\end{verbatim}

\begin{itemize}
\tightlist
\item
  \textbf{CPU (Central Processing Unit)}: Executes program, processes
  I/O data, makes decisions
\item
  \textbf{Input modules}: Convert field signals (sensors, switches) to
  digital signals for CPU
\item
  \textbf{Output modules}: Convert CPU commands to actuator signals
  (motors, valves)
\item
  \textbf{Memory unit}: Stores program and data (ROM for OS, RAM for
  user program)
\item
  \textbf{Programming device}: PC or console for program development and
  monitoring
\item
  \textbf{Power supply}: Provides regulated power to PLC components
\end{itemize}

\end{solutionbox}
\begin{mnemonicbox}
``Programs Logic Completely''

\end{mnemonicbox}
\subsection*{Question 5(a OR) [3
marks]}\label{question-5a-or-3-marks}

\textbf{Draw and explain the construction of Stepper motor.}

\begin{solutionbox}
Stepper motor rotates in discrete steps for precise
positioning.

\textbf{Diagram:}

\begin{verbatim}
      ┌───────────┐
      │           │
      │  Stator   │
      │  ┌─────┐  │
      │  │     │  │
      │  │Rotor│  │
      │  │     │  │
      │  └─────┘  │
      │           │
      └───────────┘
          Phases
\end{verbatim}

\begin{itemize}
\tightlist
\item
  \textbf{Stator}: Contains multiple coil windings (phases)
\item
  \textbf{Rotor}: Permanent magnet or variable reluctance type
\item
  \textbf{Types}: Permanent magnet, variable reluctance, hybrid
\item
  \textbf{Step angle}: Typically 1.8^\circ (200 steps/rev) or 0.9^\circ (400
  steps/rev)
\item
  \textbf{Applications}: Printers, disk drives, robotics, CNC machines
\end{itemize}

\end{solutionbox}
\begin{mnemonicbox}
``Steps Precisely Moved''

\end{mnemonicbox}
\subsection*{Question 5(b OR) [4
marks]}\label{question-5b-or-4-marks}

\textbf{Draw explain solid state circuit to control DC shunt Motor
Speed.}

\begin{solutionbox}
Solid-state circuit provides efficient and smooth DC
motor speed control.

\textbf{Circuit Diagram:}

\begin{verbatim}
     +Vdc
      │
      │           ┌────────┐
      ├───────────┤ Field  │
      │           │ Winding│
      │           └────────┘
      │
    ┌─┴─┐
    │   │     ┌───────┐
    │PWM├─────┤ MOSFET│    ┌──────┐
    │   │     │Driver │────┤MOSFET│
    └───┘     └───────┘    │      │
                          ┌┴──────┴┐
                          │Armature│
                          │Winding │
                          └────────┘
\end{verbatim}

\begin{itemize}
\tightlist
\item
  \textbf{PWM controller}: Generates variable duty cycle pulses to
  control speed
\item
  \textbf{MOSFET driver}: Provides gate drive to power MOSFET
\item
  \textbf{Power MOSFET}: Controls current through armature winding
\item
  \textbf{Feedback}: Tachogenerator or encoder provides speed feedback
\item
  \textbf{Advantages}: Efficient, smooth control, wide speed range
\end{itemize}

\end{solutionbox}
\begin{mnemonicbox}
``Power With MOSFET''

\end{mnemonicbox}
\subsection*{Question 5(c OR) [7
marks]}\label{question-5c-or-7-marks}

\textbf{Explain the Working of VFD (Variable Frequency Drive).}

\begin{solutionbox}
VFD controls AC motor speed by varying frequency and
voltage.

\textbf{Block Diagram:}

\begin{verbatim}
    ┌─────────┐    ┌─────────┐    ┌─────────┐    ┌─────────┐
    │  AC     │    │Rectifier│    │DC Link  │    │Inverter │    ┌─────────┐
    │  Input  ├────┤ Circuit ├────┤Capacitor├────┤ Circuit ├────┤   AC    │
    │         │    │         │    │         │    │         │    │  Motor  │
    └─────────┘    └─────────┘    └─────────┘    └─────────┘    └─────────┘
                                       │
                                  ┌────┴────┐
                                  │ Control │
                                  │ Circuit │
                                  └─────────┘
\end{verbatim}

{\def\LTcaptype{none} % do not increment counter
\begin{longtable}[]{@{}
  >{\raggedright\arraybackslash}p{(\linewidth - 2\tabcolsep) * \real{0.5238}}
  >{\raggedright\arraybackslash}p{(\linewidth - 2\tabcolsep) * \real{0.4762}}@{}}
\toprule\noalign{}
\begin{minipage}[b]{\linewidth}\raggedright
Component
\end{minipage} & \begin{minipage}[b]{\linewidth}\raggedright
Function
\end{minipage} \\
\midrule\noalign{}
\endhead
\bottomrule\noalign{}
\endlastfoot
Rectifier & Converts AC input to DC (diode bridge or active front
end) \\
DC Link & Filters DC and stores energy (capacitors, sometimes
inductors) \\
Inverter & Converts DC to variable frequency AC (IGBTs with PWM) \\
Control Circuit & Regulates frequency/voltage based on speed
requirement \\
Braking Circuit & Dissipates regenerative energy during deceleration \\
\end{longtable}
}

\begin{itemize}
\tightlist
\item
  \textbf{Speed control}: Motor speed proportional to frequency (RPM =
  120f/P)
\item
  \textbf{Torque control}: Maintains V/f ratio for constant torque
\item
  \textbf{Energy savings}: Reduces energy consumption at lower speeds
\item
  \textbf{Applications}: Pumps, fans, conveyors, process control
\item
  \textbf{Features}: Soft start, overcurrent protection, regenerative
  braking
\end{itemize}

\end{solutionbox}
\begin{mnemonicbox}
``Vary Frequency, Drive motor''

\end{mnemonicbox}

\end{document}
