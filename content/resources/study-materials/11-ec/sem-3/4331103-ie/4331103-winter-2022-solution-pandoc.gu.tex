\documentclass[10pt,a4paper]{article}

% content/resources/templates/preamble.tex
\usepackage[margin=0.6in]{geometry}
\author{Milav Dabgar}
\usepackage{amsmath,amssymb,amsthm}
\usepackage{booktabs}
\usepackage{multirow}
\usepackage{xcolor}
\usepackage{tcolorbox}
\tcbuselibrary{breakable,skins}
\usepackage[colorlinks=true,linkcolor=blue]{hyperref}
\usepackage{titlesec}
\usepackage{enumitem}
\usepackage{tikz}
\usepackage{pgfplots}
\usepackage{circuitikz}
\usepackage[version=4]{mhchem}
\usepackage{longtable}
\usepackage{array}
\usepackage{float}
\usepackage{caption}
\usepackage{listings}

\lstset{
  basicstyle=\small\ttfamily,
  breaklines=true,
  breakatwhitespace=false,
  postbreak=\mbox{\textcolor{red}{$\hookrightarrow$}\space},
  float=false,
  numbers=left,
  numberstyle=\tiny\color{gray},
  numbersep=10pt,
  xleftmargin=2em,
  keywordstyle=\color{blue},
  commentstyle=\color{green!60!black},
  stringstyle=\color{purple},
  backgroundcolor=\color{gray!5},
  showstringspaces=false,
  tabsize=2,
  captionpos=b,
  keepspaces=true,
  columns=flexible
}

\pgfplotsset{compat=1.18}
\usetikzlibrary{shapes,arrows,positioning,calc,patterns,decorations.pathmorphing,decorations.markings,arrows.meta}

% Color scheme
\definecolor{headcolor}{RGB}{0,102,204}
\definecolor{keycolor}{RGB}{220,20,60}
\definecolor{solutioncolor}{RGB}{34,139,34}
\definecolor{mnemoniccolor}{RGB}{148,0,211}
\definecolor{codecolor}{RGB}{0,0,100}

% Spacing
\setlength{\parskip}{3pt}
\setlist[itemize]{nosep}
\setlist[enumerate]{nosep}

% Title formatting
\titleformat{\section}{\Large\bfseries\color{headcolor}}{\thesection}{1em}{}
\titleformat{\subsection}{\large\bfseries\color{headcolor}}{\thesubsection}{1em}{}

% Pandoc tightlist compatibility
\providecommand{\tightlist}{%
  \setlength{\itemsep}{0pt}\setlength{\parskip}{0pt}}

% Pandoc longtable compatibility
\newcounter{none}
\def\thenone{}


% content/resources/templates/gujarati-boxes.tex
\usepackage{fontspec}
\usepackage{polyglossia}

% Set Gujarati as main language (document is primarily in Gujarati)
% Note: gloss-gujarati.ldf doesn't exist in polyglossia, but it will use hyphenation patterns
\setdefaultlanguage{gujarati}
\setotherlanguage{english}

% Configure Gujarati font properly
% Use Language=Default to prevent polyglossia from trying to add language-specific features
% that don't exist for Gujarati, which causes "empty feature" warnings
\newfontfamily\gujaratifont[Script=Gujarati,AutoFakeBold=2.5,AutoFakeSlant=0.3]{Noto Sans Gujarati}
\setmainfont[Script=Gujarati,AutoFakeBold=2.5,AutoFakeSlant=0.3]{Noto Sans Gujarati}
% Use Noto Sans Gujarati for monospace to support Gujarati in text
\setmonofont[Scale=0.9]{Noto Sans Gujarati}

% Configure English to use the same font
\newfontfamily\englishfont[Script=Gujarati,AutoFakeBold=2.5,AutoFakeSlant=0.3]{Noto Sans Gujarati}

% Translations for polyglossia
\gappto\captionsgujarati{
  \renewcommand{\tablename}{કોષ્ટક}
  \renewcommand{\figurename}{આકૃતિ}
}

% Helper for TikZ nodes to ensure Gujarati font
\newcommand{\gu}[1]{{\gujaratifont #1}}

% Custom environments
\newtcolorbox{solutionbox}{
    breakable,
    enhanced,
    colback=solutioncolor!5!white,
    colframe=solutioncolor!75!black,
    fonttitle=\bfseries,
    title=જવાબ
}

\newtcolorbox{solutionboxnobreak}{
 colback=solutioncolor!5!white,
 colframe=solutioncolor!75!black,
 fonttitle=\bfseries,
 title=જવાબ
}

\newtcolorbox{keyformula}{
 breakable,
 enhanced,
 colback=keycolor!5!white,
 colframe=keycolor!75!black,
 fonttitle=\bfseries,
 title=રાસાયણિક સમીકરણ/સૂત્ર
}

\newtcolorbox{mnemonicbox}{
 breakable,
 enhanced,
 colback=mnemoniccolor!5!white,
 colframe=mnemoniccolor!75!black,
 fonttitle=\bfseries,
 title=મેમરી ટ્રીક
}


\begin{document}

\begin{center}
{\Huge\bfseries\color{headcolor} Subject Name (Gujarati)}\\[5pt]
{\LARGE 4331103 -- Winter 2022}\\[3pt]
{\large Semester 1 Study Material}\\[3pt]
{\normalsize\textit{Detailed Solutions and Explanations}}
\end{center}

\vspace{10pt}

\subsection*{પ્રશ્ન 1(અ) [3
ગુણ]}\label{uxaaauxab0uxab6uxaa8-1uxa85-3-uxa97uxaa3}

\textbf{SCR ની રચના દોરો અને સમજાવો.}

\begin{solutionbox}
SCR (સિલિકોન કંટ્રોલ્ડ રેક્ટિફાયર) એ ચાર-લેયર PNPN સેમિકન્ડક્ટર
ડિવાઇસ છે જેમાં ત્રણ ટર્મિનલ્સ છે: એનોડ, કેથોડ અને ગેટ.

\textbf{ડાયાગ્રામ:}

\begin{center}
\textbf{Mermaid Diagram (Code)}
\begin{verbatim}
{Shaded}
{Highlighting}[]
graph LR
    A[Anode] {-{-}{-} P1[P{-}layer]}
    P1 {-{-}{-} N1[N{-}layer]}
    N1 {-{-}{-} P2[P{-}layer]}
    P2 {-{-}{-} N2[N{-}layer]}
    N2 {-{-}{-} K[Cathode]}
    G[Gate] {-{-}{-} P2}
{Highlighting}
{Shaded}
\end{verbatim}
\end{center}

\begin{itemize}
\tightlist
\item
  \textbf{P-N-P-N લેયર્સ}: ચાર અલ્ટરનેટિંગ સેમિકન્ડક્ટર લેયર્સ
\item
  \textbf{ગેટ ટર્મિનલ}: ડિવાઇસના ટર્ન-ઓન ને નિયંત્રિત કરે છે
\item
  \textbf{કરંટ ફ્લો}: ટ્રિગર થવા પર એનોડથી કેથોડ તરફ
\end{itemize}

\end{solutionbox}
\begin{mnemonicbox}
``સિલિકોન કંટ્રોલ્સ રેક્ટિફિકેશન'' - SCR માત્ર ટ્રિગર થવા પર
એક દિશામાં પ્રવાહ નિયંત્રિત કરે છે.

\end{mnemonicbox}
\subsection*{પ્રશ્ન 1(બ) [4
ગુણ]}\label{uxaaauxab0uxab6uxaa8-1uxaac-4-uxa97uxaa3}

\textbf{TRIAC ની રચના દોરો અને સમજાવો.}

\begin{solutionbox}
TRIAC (ટ્રાયોડ ફોર અલ્ટરનેટિંગ કરંટ) એ બાયડાયરેક્શનલ ત્રણ-ટર્મિનલ
સેમિકન્ડક્ટર ડિવાઇસ છે જે ટ્રિગર થતાં બંને દિશામાં કન્ડક્ટ કરે છે.

\textbf{ડાયાગ્રામ:}

\begin{center}
\textbf{Mermaid Diagram (Code)}
\begin{verbatim}
{Shaded}
{Highlighting}[]
graph LR
    MT1[Main Terminal 1] {-{-}{-} N1[N{-}layer]}
    N1 {-{-}{-} P1[P{-}layer]}
    P1 {-{-}{-} N2[N{-}layer]}
    N2 {-{-}{-} P2[P{-}layer]}
    P2 {-{-}{-} N3[N{-}layer]}
    N3 {-{-}{-} MT2[Main Terminal 2]}
    G[Gate] {-{-}{-} P1}
{Highlighting}
{Shaded}
\end{verbatim}
\end{center}

\begin{itemize}
\tightlist
\item
  \textbf{બાયડાયરેક્શનલ ઓપરેશન}: ટ્રિગર થવા પર બંને દિશામાં કન્ડક્ટ કરે છે
\item
  \textbf{ગેટ કંટ્રોલ}: એક ગેટ બંને દિશામાં કન્ડક્શન નિયંત્રિત કરે છે
\item
  \textbf{ઇક્વિવેલન્ટ સર્કિટ}: એન્ટિ-પેરેલલમાં જોડાયેલા બે SCR જેવું કાર્ય કરે છે
\item
  \textbf{AC એપ્લિકેશન્સ}: AC પાવર કંટ્રોલ એપ્લિકેશન્સમાં વ્યાપકપણે ઉપયોગ થાય છે
\end{itemize}

\end{solutionbox}
\begin{mnemonicbox}
``ટ્રાય-દિશા AC કંટ્રોલર'' - AC સર્કિટમાં બંને દિશામાં કરંટ
નિયંત્રિત કરે છે.

\end{mnemonicbox}
\subsection*{પ્રશ્ન 1(ક) [7
ગુણ]}\label{uxaaauxab0uxab6uxaa8-1uxa95-7-uxa97uxaa3}

\textbf{ઓપ્ટો-આઈસોલેટર, ઓપ્ટો-TRIAC, ઓપ્ટો-SCR, અને ઓપ્ટો-ટ્રાન્ઝિસ્ટરની રચના,
કાર્યપદ્ધતિ વર્ણવો અને તેના ઉપયોગો લખો.}

\begin{solutionbox}
ઓપ્ટો-આઈસોલેટર્સ આઇસોલેટેડ સર્કિટ્સ વચ્ચે ઇલેક્ટ્રિકલ સિગ્નલ્સ ટ્રાન્સફર
કરવા માટે પ્રકાશનો ઉપયોગ કરે છે.

\textbf{ડાયાગ્રામ:}

\begin{center}
\textbf{Mermaid Diagram (Code)}
\begin{verbatim}
{Shaded}
{Highlighting}[]
graph LR
    subgraph Input
        LED[LED]
    end
    subgraph Output
        PD[Photo Detector]
    end
    LED {-{-} "Light" {-}{-}{} PD}
    style Input fill:\#f9f,stroke:\#333
    style Output fill:\#bbf,stroke:\#333
{Highlighting}
{Shaded}
\end{verbatim}
\end{center}

{\def\LTcaptype{none} % do not increment counter
\begin{longtable}[]{@{}
  >{\raggedright\arraybackslash}p{(\linewidth - 6\tabcolsep) * \real{0.1778}}
  >{\raggedright\arraybackslash}p{(\linewidth - 6\tabcolsep) * \real{0.3111}}
  >{\raggedright\arraybackslash}p{(\linewidth - 6\tabcolsep) * \real{0.2000}}
  >{\raggedright\arraybackslash}p{(\linewidth - 6\tabcolsep) * \real{0.3111}}@{}}
\toprule\noalign{}
\begin{minipage}[b]{\linewidth}\raggedright
ડિવાઇસ
\end{minipage} & \begin{minipage}[b]{\linewidth}\raggedright
રચના
\end{minipage} & \begin{minipage}[b]{\linewidth}\raggedright
કાર્યપદ્ધતિ
\end{minipage} & \begin{minipage}[b]{\linewidth}\raggedright
ઉપયોગો
\end{minipage} \\
\midrule\noalign{}
\endhead
\bottomrule\noalign{}
\endlastfoot
ઓપ્ટો-આઈસોલેટર & LED + ફોટોડિટેક્ટર & જ્યારે ઇનપુટ કરંટ પ્રવાહિત થાય છે ત્યારે LED
પ્રકાશ ઉત્સર્જિત કરે છે; ફોટોડિટેક્ટર આઉટપુટ સર્કિટને સક્રિય કરે છે & સિગ્નલ આઇસોલેશન,
મેડિકલ ઉપકરણો, ઔદ્યોગિક નિયંત્રણો \\
ઓપ્ટો-TRIAC & LED + ફોટો-TRIAC & LED પ્રકાશ દ્વારા TRIAC ને ટ્રિગર કરે છે;
ઇલેક્ટ્રિકલ આઇસોલેશન પ્રદાન કરે છે & AC પાવર કંટ્રોલ, સોલિડ સ્ટેટ રિલે, મોટર
કંટ્રોલ \\
ઓપ્ટો-SCR & LED + ફોટો-SCR & LED SCR ને ટ્રિગર કરવા માટે પ્રકાશ ઉત્સર્જિત કરે છે;
ઉચ્ચ આઇસોલેશન પ્રદાન કરે છે & DC સ્વિચિંગ, ઔદ્યોગિક નિયંત્રણો, ઉચ્ચ વોલ્ટેજ
આઇસોલેશન \\
ઓપ્ટો-ટ્રાન્ઝિસ્ટર & LED + ફોટો-ટ્રાન્ઝિસ્ટર & LED પ્રકાશ ફોટોટ્રાન્ઝિસ્ટરના બેઝ
કરંટને નિયંત્રિત કરે છે & એન્કોડર્સ, લેવલ ડિટેક્શન, પોઝિશન સેન્સિંગ \\
\end{longtable}
}

\begin{itemize}
\tightlist
\item
  \textbf{ઇલેક્ટ્રિકલ આઇસોલેશન}: ઇનપુટ અને આઉટપુટ વચ્ચે સંપૂર્ણ અલગતા
\item
  \textbf{નોઇઝ ઇમ્યુનિટી}: ઇલેક્ટ્રિકલ નોઇઝ પ્રત્યે ઉચ્ચ પ્રતિરોધ
\item
  \textbf{સ્પીડ}: માઇક્રોસેકન્ડ રેન્જમાં રિસ્પોન્સ ટાઇમ
\end{itemize}

\end{solutionbox}
\begin{mnemonicbox}
``LOST'' - Light Operates Semiconductor Terminals
બધા ઓપ્ટો-ડિવાઇસમાં.

\end{mnemonicbox}
\subsection*{પ્રશ્ન 1(ક) OR [7
ગુણ]}\label{uxaaauxab0uxab6uxaa8-1uxa95-or-7-uxa97uxaa3}

\textbf{બે ટ્રાન્ઝીસ્ટર એનાલોગી વડે SCRનું કાર્ય સમજાવો અને SCRનાં ઇન્ડસ્ટ્રીયલ
ઉપયોગો લખો.}

\begin{solutionbox}
SCR ને બે ઇન્ટરકનેક્ટેડ ટ્રાન્ઝિસ્ટર તરીકે મોડેલ કરી શકાય છે: PNP
(T1) અને NPN (T2).

\textbf{ડાયાગ્રામ:}

\begin{center}
\textbf{Mermaid Diagram (Code)}
\begin{verbatim}
{Shaded}
{Highlighting}[]
graph TD
    A[Anode] {-{-}{-} E1[Emitter T1]}
    B1[Base T1] {-{-}{-} C2[Collector T2]}
    C1[Collector T1] {-{-}{-} B2[Base T2]}
    E2[Emitter T2] {-{-}{-} K[Cathode]}
    G[Gate] {-{-}{-} B2}
{Highlighting}
{Shaded}
\end{verbatim}
\end{center}

\textbf{કાર્ય સિદ્ધાંત:}

{\def\LTcaptype{none} % do not increment counter
\begin{longtable}[]{@{}
  >{\raggedright\arraybackslash}p{(\linewidth - 2\tabcolsep) * \real{0.3529}}
  >{\raggedright\arraybackslash}p{(\linewidth - 2\tabcolsep) * \real{0.6471}}@{}}
\toprule\noalign{}
\begin{minipage}[b]{\linewidth}\raggedright
સ્ટેપ
\end{minipage} & \begin{minipage}[b]{\linewidth}\raggedright
ઓપરેશન
\end{minipage} \\
\midrule\noalign{}
\endhead
\bottomrule\noalign{}
\endlastfoot
પ્રારંભિક સ્થિતિ & બંને ટ્રાન્ઝિસ્ટર OFF હોય છે \\
ગેટ ટ્રિગરિંગ & ગેટમાં (T2ના B2માં) કરંટ ઇન્જેક્ટ કરવામાં આવે છે \\
રિજનરેટિવ એક્શન & T2 ON થાય છે \rightarrow T1 બેઝને કરંટ મળે છે \rightarrow T1 ON થાય છે \rightarrow T2 બેઝને વધુ
કરંટ મળે છે \\
લેચિંગ & ગેટ સિગ્નલ દૂર કરવામાં આવે તો પણ સ્વ-ટકાઉ કરંટ પ્રવાહ ચાલુ રહે છે \\
\end{longtable}
}

\textbf{SCRના ઔદ્યોગિક ઉપયોગો:}

\begin{itemize}
\tightlist
\item
  \textbf{પાવર કંટ્રોલ}: AC/DC મોટર સ્પીડ કંટ્રોલ
\item
  \textbf{સ્વિચિંગ}: સ્ટેટિક સ્વિચ, સોલિડ-સ્ટેટ રિલે
\item
  \textbf{ઇન્વર્ટર}: DC થી AC રૂપાંતર
\item
  \textbf{પ્રોટેક્શન}: ઓવરવોલ્ટેજ પ્રોટેક્શન સર્કિટ
\item
  \textbf{લાઇટિંગ}: લાઇટ ડિમર, ઇલ્યુમિનેશન કંટ્રોલ
\end{itemize}

\end{solutionbox}
\begin{mnemonicbox}
``POWER'' - Power control, Overvoltage protection,
Welding machines, Electronic converters, Regulated supplies.

\end{mnemonicbox}
\subsection*{પ્રશ્ન 2(અ) [3
ગુણ]}\label{uxaaauxab0uxab6uxaa8-2uxa85-3-uxa97uxaa3}

\textbf{એસ.સી.આર માં ટ્રિગરીંગ વ્યાખ્યાયીત કરી.કોઈ પણ બે ટ્રિગરીંગ ટેકનિક
સમજાવો.}

\begin{solutionbox}
ટ્રિગરિંગ એ SCRને તેના ગેટ ટર્મિનલ પર યોગ્ય સિગ્નલ લાગુ કરીને ON
કરવાની પ્રક્રિયા છે.

\textbf{બે ટ્રિગરિંગ ટેકનિક:}

{\def\LTcaptype{none} % do not increment counter
\begin{longtable}[]{@{}ll@{}}
\toprule\noalign{}
ટેકનિક & વિગત \\
\midrule\noalign{}
\endhead
\bottomrule\noalign{}
\endlastfoot
ગેટ ટ્રિગરિંગ & ગેટ-કેથોડ સર્કિટમાં ડાયરેક્ટ કરંટ પલ્સ આપવામાં આવે છે \\
લાઇટ ટ્રિગરિંગ & જંક્શન પર અથડાતા ફોટોન્સ કન્ડક્શન માટે ઊર્જા આપે છે \\
\end{longtable}
}

\begin{itemize}
\tightlist
\item
  \textbf{ગેટ ટ્રિગરિંગ}: ઇલેક્ટ્રિકલ પલ્સનો ઉપયોગ કરતી સૌથી સામાન્ય પદ્ધતિ
\item
  \textbf{લાઇટ ટ્રિગરિંગ}: ફોટોસેન્સિટિવ સેમિકન્ડક્ટર ગુણધર્મોનો ઉપયોગ કરે છે
\end{itemize}

\end{solutionbox}
\begin{mnemonicbox}
``GET'' - Gate Electrical Triggering સૌથી સામાન્ય
પદ્ધતિ છે.

\end{mnemonicbox}
\subsection*{પ્રશ્ન 2(બ) [4
ગુણ]}\label{uxaaauxab0uxab6uxaa8-2uxaac-4-uxa97uxaa3}

\textbf{ફોર્સ્ડ કોમ્યુટેશન અને નેચરલ કોમ્યુટેશન વચ્ચેનો તફાવત લખો.}

\begin{solutionbox}

{\def\LTcaptype{none} % do not increment counter
\begin{longtable}[]{@{}
  >{\raggedright\arraybackslash}p{(\linewidth - 4\tabcolsep) * \real{0.2157}}
  >{\raggedright\arraybackslash}p{(\linewidth - 4\tabcolsep) * \real{0.3725}}
  >{\raggedright\arraybackslash}p{(\linewidth - 4\tabcolsep) * \real{0.4118}}@{}}
\toprule\noalign{}
\begin{minipage}[b]{\linewidth}\raggedright
પેરામીટર
\end{minipage} & \begin{minipage}[b]{\linewidth}\raggedright
ફોર્સ્ડ કોમ્યુટેશન
\end{minipage} & \begin{minipage}[b]{\linewidth}\raggedright
નેચરલ કોમ્યુટેશન
\end{minipage} \\
\midrule\noalign{}
\endhead
\bottomrule\noalign{}
\endlastfoot
વ્યાખ્યા & એક્સટર્નલ સર્કિટરી SCRને ફોર્સ કરીને OFF કરે છે & કરંટ હોલ્ડિંગ વેલ્યુથી નીચે
જતાં SCR કુદરતી રીતે OFF થાય છે \\
એપ્લિકેશન & DC સર્કિટ્સ & AC સર્કિટ્સ \\
કોમ્પોનન્ટ્સ & વધારાના કોમ્પોનન્ટ્સની જરૂર પડે છે (કેપેસિટર, ઇન્ડક્ટર) & કોઈ વધારાના
કોમ્પોનન્ટ્સની જરૂર નથી \\
જટિલતા & જટિલ સર્કિટ ડિઝાઇન & સરળ સર્કિટ ડિઝાઇન \\
ઊર્જા & ટર્ન-ઓફ માટે બાહ્ય ઊર્જાની જરૂર પડે છે & કોઈ બાહ્ય ઊર્જાની જરૂર નથી \\
\end{longtable}
}

\begin{itemize}
\tightlist
\item
  \textbf{ફોર્સ્ડ કોમ્યુટેશન}: બાહ્ય સર્કિટનો ઉપયોગ કરીને SCRને સક્રિયપણે બંધ કરે છે
\item
  \textbf{નેચરલ કોમ્યુટેશન}: જ્યારે AC કરંટ શૂન્ય ક્રોસ કરે છે ત્યારે SCR બંધ થાય છે
\end{itemize}

\end{solutionbox}
\begin{mnemonicbox}
``FACE'' - Forced Active Commutation requires
External components.

\end{mnemonicbox}
\subsection*{પ્રશ્ન 2(ક) [7
ગુણ]}\label{uxaaauxab0uxab6uxaa8-2uxa95-7-uxa97uxaa3}

\textbf{SCR માટે સ્નબર સર્કિટ ડીઝાઈન કરો.}

\begin{solutionbox}
સ્નબર સર્કિટ SCRને ઊંચા dV/dt થી રક્ષણ આપે છે અને વોલ્ટેજ વૃદ્ધિના
દરને મર્યાદિત કરે છે.

\textbf{ડાયાગ્રામ:}

\begin{center}
\textbf{Mermaid Diagram (Code)}
\begin{verbatim}
{Shaded}
{Highlighting}[]
graph LR
    A[Anode] {-{-}{-} R[Resistance]}
    R {-{-}{-} C[Capacitance]}
    C {-{-}{-} K[Cathode]}
    A {-{-}{-} SCR[SCR]}
    SCR {-{-}{-} K}
{Highlighting}
{Shaded}
\end{verbatim}
\end{center}

\textbf{ડિઝાઇન સ્ટેપ્સ:}

{\def\LTcaptype{none} % do not increment counter
\begin{longtable}[]{@{}ll@{}}
\toprule\noalign{}
સ્ટેપ & ગણતરી \\
\midrule\noalign{}
\endhead
\bottomrule\noalign{}
\endlastfoot
1. dV/dt રેટિંગની ગણતરી કરો & ડેટાશીટમાંથી (V/μs) \\
2. R વેલ્યુ નક્કી કરો & R = V_{1}/IL જ્યાં V_{1} એ સપ્લાય વોલ્ટેજ અને IL એ લોડ કરંટ છે \\
3. C વેલ્યુ નક્કી કરો & C = 1/(R \times (dV/dt)max) \\
4. RC ટાઇમ કોન્સ્ટન્ટ & τ = R \times C (SCR ટર્ન-ઓફ ટાઇમ કરતાં વધારે હોવું જોઈએ) \\
\end{longtable}
}

\begin{itemize}
\tightlist
\item
  \textbf{રેઝિસ્ટન્સ R}: કેપેસિટરના ડિસ્ચાર્જ કરંટને મર્યાદિત કરે છે
\item
  \textbf{કેપેસિટન્સ C}: ટ્રાન્ઝિયન્ટ એનર્જીને શોષે છે અને dV/dt ને મર્યાદિત કરે છે
\item
  \textbf{પ્રોટેક્શન}: ખોટા ટ્રિગરિંગ અને નુકસાનને રોકે છે
\item
  \textbf{પાવર રેટિંગ}: R પાસે પૂરતી પાવર રેટિંગ હોવી જોઈએ
\end{itemize}

\end{solutionbox}
\begin{mnemonicbox}
``RCSS'' - Resistance-Capacitance Saves Silicon from
Stress.

\end{mnemonicbox}
\subsection*{પ્રશ્ન 2(અ) OR [3
ગુણ]}\label{uxaaauxab0uxab6uxaa8-2uxa85-or-3-uxa97uxaa3}

\textbf{એસ.સી.આર માટેનું ક્લાસ-ઈ કોમ્યુટેશન સમજાવો.}

\begin{solutionbox}
કોમ્યુટેશન એ SCRના એનોડ કરંટને હોલ્ડિંગ કરંટ લેવલથી નીચે ઘટાડીને તેને
OFF કરવાની પ્રક્રિયા છે.

\textbf{ક્લાસ-E કોમ્યુટેશન:}

\textbf{ડાયાગ્રામ:}

\begin{center}
\textbf{Mermaid Diagram (Code)}
\begin{verbatim}
{Shaded}
{Highlighting}[]
graph LR
    S[Supply] {-{-}{-} L[Load]}
    L {-{-}{-} SCR[SCR]}
    L {-{-}{-} C[Capacitor]}
    C {-{-}{-} A[Auxiliary SCR]}
    A {-{-}{-} S}
{Highlighting}
{Shaded}
\end{verbatim}
\end{center}

\begin{itemize}
\tightlist
\item
  \textbf{ઓક્ઝિલરી SCR}: કોમ્યુટેશન પ્રક્રિયાને નિયંત્રિત કરે છે
\item
  \textbf{રેઝોનન્ટ સર્કિટ}: LC રેઝોનન્ટ સર્કિટ બનાવે છે
\item
  \textbf{ઓપરેશન}: ઓક્ઝિલરી SCR મેઇન SCRને રિવર્સ-બાયસ કરવા માટે કેપેસિટર
  ડિસ્ચાર્જને ટ્રિગર કરે છે
\item
  \textbf{એપ્લિકેશન}: ઇન્વર્ટર અને ચોપરમાં ઉપયોગ થાય છે
\end{itemize}

\end{solutionbox}
\begin{mnemonicbox}
``ACE'' - Auxiliary Capacitor Extinguishes
conduction.

\end{mnemonicbox}
\subsection*{પ્રશ્ન 2(બ) OR [4
ગુણ]}\label{uxaaauxab0uxab6uxaa8-2uxaac-or-4-uxa97uxaa3}

\textbf{થાઈરિસ્ટરનું ટ્રિગરીંગ વિગતવાર સમજાવો.}

\begin{solutionbox}

{\def\LTcaptype{none} % do not increment counter
\begin{longtable}[]{@{}ll@{}}
\toprule\noalign{}
ટ્રિગરિંગ મેથડ & કાર્ય સિદ્ધાંત \\
\midrule\noalign{}
\endhead
\bottomrule\noalign{}
\endlastfoot
ગેટ ટ્રિગરિંગ & ગેટ અને કેથોડ વચ્ચે ઇલેક્ટ્રિકલ પલ્સ આપવામાં આવે છે \\
તાપમાન ટ્રિગરિંગ & જંક્શન તાપમાન ટર્ન-ઓન થવા માટે વધે છે \\
લાઇટ ટ્રિગરિંગ & ફોટોન્સ જંક્શન પર ઇલેક્ટ્રોન-હોલ જોડી બનાવે છે \\
dV/dt ટ્રિગરિંગ & ઝડપી વોલ્ટેજ વૃદ્ધિ કેપેસિટિવ કરંટ પ્રવાહ થવા માટે કારણભૂત છે \\
ફોરવર્ડ વોલ્ટેજ ટ્રિગરિંગ & બ્રેકઓવર વોલ્ટેજને વટાવવાથી એવેલાન્ચ કન્ડક્શન થાય છે \\
\end{longtable}
}

\begin{itemize}
\tightlist
\item
  \textbf{ગેટ ટ્રિગરિંગ}: સૌથી સામાન્ય અને નિયંત્રિત પદ્ધતિ
\item
  \textbf{પેરામીટર કંટ્રોલ}: પલ્સ પહોળાઈ, એમ્પ્લિટ્યુડ અને રાઈઝ ટાઈમ
\item
  \textbf{ગેટ સેન્સિટિવિટી}: તાપમાન સાથે બદલાય છે
\item
  \textbf{પ્રોટેક્શન}: અનિચ્છનીય ટ્રિગરિંગથી રક્ષણ જરૂરી છે
\end{itemize}

\end{solutionbox}
\begin{mnemonicbox}
``VITAL'' - Voltage, Illumination, Temperature And
Level બધી ટ્રિગરિંગ પદ્ધતિઓ છે.

\end{mnemonicbox}
\subsection*{પ્રશ્ન 2(ક) OR [7
ગુણ]}\label{uxaaauxab0uxab6uxaa8-2uxa95-or-7-uxa97uxaa3}

\textbf{એસ.સી.આર ને ઓવર વૉલ્ટેજ અને ઓવર કરંટ થી બચાવવા માટેની મેથડ વિગતવાર
સમજાવો.}

\begin{solutionbox}

\textbf{ઓવરવોલ્ટેજ પ્રોટેક્શન:}

\textbf{ડાયાગ્રામ:}

\begin{center}
\textbf{Mermaid Diagram (Code)}
\begin{verbatim}
{Shaded}
{Highlighting}[]
graph LR
    S[Supply] {-{-}{-} F[Fuse]}
    F {-{-}{-} V[Varistor]}
    V {-{-}{-} SCR[SCR]}
    SCR {-{-}{-} L[Load]}
    V {-{-}{-} RC[RC Snubber]}
    RC {-{-}{-} SCR}
{Highlighting}
{Shaded}
\end{verbatim}
\end{center}

{\def\LTcaptype{none} % do not increment counter
\begin{longtable}[]{@{}
  >{\raggedright\arraybackslash}p{(\linewidth - 2\tabcolsep) * \real{0.5000}}
  >{\raggedright\arraybackslash}p{(\linewidth - 2\tabcolsep) * \real{0.5000}}@{}}
\toprule\noalign{}
\begin{minipage}[b]{\linewidth}\raggedright
પ્રોટેક્શન મેથડ
\end{minipage} & \begin{minipage}[b]{\linewidth}\raggedright
કાર્ય સિદ્ધાંત
\end{minipage} \\
\midrule\noalign{}
\endhead
\bottomrule\noalign{}
\endlastfoot
RC સ્નબર સર્કિટ & વોલ્ટેજના ઉછાળાનો દર (dV/dt) મર્યાદિત કરે છે \\
વોલ્ટેજ ક્લેમ્પિંગ & જેનર ડાયોડ અથવા MOVsનો ઉપયોગ કરીને મહત્તમ વોલ્ટેજ મર્યાદિત કરે
છે \\
ક્રોબાર પ્રોટેક્શન & વોલ્ટેજ થ્રેશોલ્ડને વટાવે ત્યારે જાણીજોઈને શોર્ટ-સર્કિટ કરે છે \\
\end{longtable}
}

\textbf{ઓવરકરંટ પ્રોટેક્શન:}

\textbf{ડાયાગ્રામ:}

\begin{center}
\textbf{Mermaid Diagram (Code)}
\begin{verbatim}
{Shaded}
{Highlighting}[]
graph LR
    S[Supply] {-{-}{-} F[Fuse/Circuit Breaker]}
    F {-{-}{-} R[Current Limiting Resistor]}
    R {-{-}{-} SCR[SCR]}
    SCR {-{-}{-} L[Load]}
{Highlighting}
{Shaded}
\end{verbatim}
\end{center}

{\def\LTcaptype{none} % do not increment counter
\begin{longtable}[]{@{}ll@{}}
\toprule\noalign{}
પ્રોટેક્શન મેથડ & કાર્ય સિદ્ધાંત \\
\midrule\noalign{}
\endhead
\bottomrule\noalign{}
\endlastfoot
ફ્યુઝ/સર્કિટ બ્રેકર & ફોલ્ટ સ્થિતિઓ દરમિયાન સર્કિટને ડિસ્કનેક્ટ કરે છે \\
કરંટ લિમિટિંગ રિએક્ટર & ફોલ્ટ કરંટની માત્રા મર્યાદિત કરે છે \\
ઇલેક્ટ્રોનિક કરંટ લિમિટિંગ & સેન્સિંગ અને કંટ્રોલ સર્કિટ્સ કરંટને મર્યાદિત કરે છે \\
\end{longtable}
}

\begin{itemize}
\tightlist
\item
  \textbf{કોઓર્ડિનેશન}: પ્રોટેક્શન ડિવાઇસ સંકલનમાં કામ કરવી જોઈએ
\item
  \textbf{રિસ્પોન્સ ટાઇમ}: અસરકારક સુરક્ષા માટે મહત્વપૂર્ણ છે
\item
  \textbf{મલ્ટીપલ લેયર્સ}: ક્રિટિકલ એપ્લિકેશન માટે, કેટલીક પદ્ધતિઓને સંયોજિત કરવામાં
  આવે છે
\end{itemize}

\end{solutionbox}
\begin{mnemonicbox}
``SCOPE'' - Snubbers, Clamps, Overload sensors,
Protectors, and Electronic limiters.

\end{mnemonicbox}
\subsection*{પ્રશ્ન 3(અ) [3
ગુણ]}\label{uxaaauxab0uxab6uxaa8-3uxa85-3-uxa97uxaa3}

\textbf{સિંગલ ફેઝ રેક્ટિફાયર અને થ્રી ફેઝ રેક્ટિફાયર વચ્ચેનો તફાવત લખો.}

\begin{solutionbox}

{\def\LTcaptype{none} % do not increment counter
\begin{longtable}[]{@{}lll@{}}
\toprule\noalign{}
પેરામીટર & સિંગલ ફેઝ રેક્ટિફાયર & પોલી ફેઝ રેક્ટિફાયર \\
\midrule\noalign{}
\endhead
\bottomrule\noalign{}
\endlastfoot
ઇનપુટ & સિંગલ ફેઝ AC સપ્લાય & મલ્ટીપલ ફેઝ (સામાન્ય રીતે 3-ફેઝ) AC સપ્લાય \\
આઉટપુટ રિપલ & ઊંચી રિપલ સામગ્રી & નીચી રિપલ સામગ્રી \\
કાર્યક્ષમતા & ઓછી કાર્યક્ષમતા & ઊંચી કાર્યક્ષમતા \\
પાવર રેટિંગ & ઓછા પાવર એપ્લિકેશન માટે યોગ્ય & ઊંચા પાવર એપ્લિકેશન માટે યોગ્ય \\
ટ્રાન્સફોર્મર ઉપયોગિતા & ઓછો ઉપયોગિતા ફેક્ટર & ઊંચો ઉપયોગિતા ફેક્ટર \\
\end{longtable}
}

\begin{itemize}
\tightlist
\item
  \textbf{રિપલ ફેક્ટર}: સિંગલ ફેઝમાં પોલી ફેઝની તુલનામાં ઊંચી રિપલ હોય છે
\item
  \textbf{ફોર્મ ફેક્ટર}: પોલી ફેઝ સિસ્ટમમાં વધુ સારો
\item
  \textbf{સાઇઝ/વજન}: પોલી ફેઝ સિસ્ટમમાં વધુ સારો પાવર/વજન રેશિયો હોય છે
\end{itemize}

\end{solutionbox}
\begin{mnemonicbox}
``PERCH'' - Poly phase has Efficiency, Ripple
improvement, Capacity, and Higher ratings.

\end{mnemonicbox}
\subsection*{પ્રશ્ન 3(બ) [4
ગુણ]}\label{uxaaauxab0uxab6uxaa8-3uxaac-4-uxa97uxaa3}

\textbf{થ્રી ફેઝ હાફ વેવ રેક્ટિફાયર નો સર્કિટ ડાયગ્રામ દોરી તેની કાર્યપદ્ધતિ
સમજાવો.}

\begin{solutionbox}
થ્રી-ફેઝ હાફ-વેવ રેક્ટિફાયર ત્રણ ડાયોડનો ઉપયોગ કરીને થ્રી-ફેઝ ACને
પલ્સેટિંગ DCમાં રૂપાંતરિત કરે છે.

\textbf{ડાયાગ્રામ:}

\begin{center}
\textbf{Mermaid Diagram (Code)}
\begin{verbatim}
{Shaded}
{Highlighting}[]
graph TD
    A[Phase A] {-{-}{-} D1[Diode 1]}
    B[Phase B] {-{-}{-} D2[Diode 2]}
    C[Phase C] {-{-}{-} D3[Diode 3]}
    D1 {-{-}{-} O[Output +]}
    D2 {-{-}{-} O}
    D3 {-{-}{-} O}
    N[Neutral] {-{-}{-} ON[Output {-}]}
{Highlighting}
{Shaded}
\end{verbatim}
\end{center}

\textbf{કાર્યપદ્ધતિ:}

\begin{itemize}
\tightlist
\item
  દરેક ડાયોડ ત્યારે કન્ડક્ટ કરે છે જ્યારે તેનું ફેઝ વોલ્ટેજ સૌથી વધુ પોઝિટિવ હોય છે
\item
  દરેક ડાયોડનો કન્ડક્શન એંગલ 120^\circ છે
\item
  રિપલ ફ્રિક્વન્સી ઇનપુટ ફ્રિક્વન્સીની 3 ગણી છે
\item
  એવરેજ આઉટપુટ વોલ્ટેજ = 3Vm/2π (જ્યાં Vm પીક ફેઝ વોલ્ટેજ છે)
\item
  રિપલ ફેક્ટર = 0.17 (સિંગલ-ફેઝ હાફ-વેવ કરતાં ઘણો ઓછો)
\end{itemize}

\end{solutionbox}
\begin{mnemonicbox}
``THREE-D'' - THREE Diodes ક્રમશઃ કન્ડક્ટ કરે છે.

\end{mnemonicbox}
\subsection*{પ્રશ્ન 3(ક) [7
ગુણ]}\label{uxaaauxab0uxab6uxaa8-3uxa95-7-uxa97uxaa3}

\textbf{બ્લોક ડાયાગ્રામની મદદથી યુપીએસ અને એસએમપીએસની કામગીરીનું વર્ણન કરો.}

\begin{solutionbox}

\textbf{UPS (અનઇન્ટેરપ્ટેબલ પાવર સપ્લાય):}

\textbf{ડાયાગ્રામ:}

\begin{center}
\textbf{Mermaid Diagram (Code)}
\begin{verbatim}
{Shaded}
{Highlighting}[]
graph LR
    AC[AC Input] {-{-}{-} R[Rectifier]}
    R {-{-}{-} BC[Battery Charger]}
    BC {-{-}{-} B[Battery]}
    B {-{-}{-} I[Inverter]}
    R {-{-}{-} I}
    I {-{-}{-} F[Filter]}
    F {-{-}{-} L[Load]}
    AC {-.Bypass.{-}{} L}
{Highlighting}
{Shaded}
\end{verbatim}
\end{center}

{\def\LTcaptype{none} % do not increment counter
\begin{longtable}[]{@{}ll@{}}
\toprule\noalign{}
બ્લોક & કાર્ય \\
\midrule\noalign{}
\endhead
\bottomrule\noalign{}
\endlastfoot
રેક્ટિફાયર & બેટરી ચાર્જિંગ અને ઇન્વર્ટર માટે ACને DCમાં રૂપાંતરિત કરે છે \\
બેટરી & પાવર ફેલ્યોર દરમિયાન બેકઅપ માટે ઊર્જા સંગ્રહ કરે છે \\
ઇન્વર્ટર & લોડને પાવર આપવા માટે DCને ACમાં રૂપાંતરિત કરે છે \\
ફિલ્ટર & આઉટપુટ વેવફોર્મને સુવ્યવસ્થિત કરે છે \\
બાયપાસ & મેઇન્ટેનન્સ દરમિયાન ડાયરેક્ટ AC પ્રદાન કરે છે \\
\end{longtable}
}

\textbf{SMPS (સ્વિચ્ડ મોડ પાવર સપ્લાય):}

\textbf{ડાયાગ્રામ:}

\begin{center}
\textbf{Mermaid Diagram (Code)}
\begin{verbatim}
{Shaded}
{Highlighting}[]
graph LR
    AC[AC Input] {-{-}{-} R[Rectifier \& Filter]}
    R {-{-}{-} SW[High Frequency Switch]}
    SW {-{-}{-} T[HF Transformer]}
    T {-{-}{-} RF[Rectifier \& Filter]}
    RF {-{-}{-} L[Load]}
    FB[Feedback] {-{-}{-} SW}
    RF {-{-}{-} FB}
{Highlighting}
{Shaded}
\end{verbatim}
\end{center}

{\def\LTcaptype{none} % do not increment counter
\begin{longtable}[]{@{}ll@{}}
\toprule\noalign{}
બ્લોક & કાર્ય \\
\midrule\noalign{}
\endhead
\bottomrule\noalign{}
\endlastfoot
રેક્ટિફાયર \& ફિલ્ટર & ACને અનરેગ્યુલેટેડ DCમાં રૂપાંતરિત કરે છે \\
હાઇ ફ્રિક્વન્સી સ્વિચ & DCને હાઇ-ફ્રિક્વન્સી પલ્સમાં વિભાજિત કરે છે \\
HF ટ્રાન્સફોર્મર & આઇસોલેશન અને વોલ્ટેજ ટ્રાન્સફોર્મેશન પ્રદાન કરે છે \\
આઉટપુટ રેક્ટિફાયર \& ફિલ્ટર & હાઇ-ફ્રિક્વન્સી ACને સ્મૂથ DCમાં રૂપાંતરિત કરે છે \\
ફીડબેક સર્કિટ & સ્વિચને નિયંત્રિત કરીને આઉટપુટ વોલ્ટેજને નિયંત્રિત કરે છે \\
\end{longtable}
}

\begin{itemize}
\tightlist
\item
  \textbf{UPS કાર્યક્ષમતા}: 80-90\%, બેકઅપ પાવર પ્રદાન કરે છે
\item
  \textbf{SMPS કાર્યક્ષમતા}: 70-90\%, લિનિયર સપ્લાય કરતાં ઘણી નાની
\item
  \textbf{નિયમન}: બંને નિયંત્રિત આઉટપુટ વોલ્ટેજ પ્રદાન કરે છે
\end{itemize}

\end{solutionbox}
\begin{mnemonicbox}
``BRIEF'' - Battery backup, Rectification,
Inversion, Efficient switching, Feedback control.

\end{mnemonicbox}
\subsection*{પ્રશ્ન 3(અ) OR [3
ગુણ]}\label{uxaaauxab0uxab6uxaa8-3uxa85-or-3-uxa97uxaa3}

\textbf{ચોપર સર્કિટના સિદ્ધાંત અને કાર્યને સમજાવો.}

\begin{solutionbox}
ચોપર એ DC-થી-DC કન્વર્ટર છે જે ફિક્સ્ડ DC ઇનપુટ વોલ્ટેજને વેરિએબલ DC
આઉટપુટ વોલ્ટેજમાં રૂપાંતરિત કરે છે.

\textbf{ડાયાગ્રામ:}

\begin{center}
\textbf{Mermaid Diagram (Code)}
\begin{verbatim}
{Shaded}
{Highlighting}[]
graph LR
    DC[DC Source] {-{-}{-} S[Switch/SCR]}
    S {-{-}{-} L[Load]}
    L {-{-}{-} DC}
{Highlighting}
{Shaded}
\end{verbatim}
\end{center}

\textbf{સિદ્ધાંત:}

\begin{itemize}
\item
  સ્વિચ (સામાન્ય રીતે SCR, MOSFET, અથવા IGBT) ઝડપથી સ્રોતને લોડ સાથે જોડે છે અને
  અલગ કરે છે
\item
  આઉટપુટ વોલ્ટેજ ડ્યુટી સાયકલ દ્વારા નિયંત્રિત થાય છે (ON સમય / કુલ સમય)
\item
  સરેરાશ આઉટપુટ વોલ્ટેજ = ઇનપુટ વોલ્ટેજ \times ડ્યુટી સાયકલ
\item
  \textbf{ટાઇમ રેશિયો કંટ્રોલ}: ફ્રિક્વન્સી સ્થિર રાખીને ડ્યુટી સાયકલ બદલે છે
\item
  \textbf{ફ્રિક્વન્સી મોડ્યુલેશન}: ON સમય સ્થિર રાખીને ફ્રિક્વન્સી બદલે છે
\item
  \textbf{એપ્લિકેશન}: DC મોટર કંટ્રોલ, બેટરી-પાવર્ડ વાહનો
\end{itemize}

\end{solutionbox}
\begin{mnemonicbox}
``CHOP'' - Control High-speed Operation with Pulses.

\end{mnemonicbox}
\subsection*{પ્રશ્ન 3(બ) OR [4
ગુણ]}\label{uxaaauxab0uxab6uxaa8-3uxaac-or-4-uxa97uxaa3}

\textbf{સિંગલ-ફેઝ અને પોલી-ફેઝ રેક્ટિફાયર સર્કિટની તુલના કરો.}

\begin{solutionbox}

{\def\LTcaptype{none} % do not increment counter
\begin{longtable}[]{@{}lll@{}}
\toprule\noalign{}
પેરામીટર & સિંગલ-ફેઝ રેક્ટિફાયર & પોલી-ફેઝ રેક્ટિફાયર \\
\midrule\noalign{}
\endhead
\bottomrule\noalign{}
\endlastfoot
સપ્લાય & સિંગલ-ફેઝ AC & ત્રણ અથવા વધુ ફેઝ AC \\
આઉટપુટ વેવફોર્મ & વધુ પલ્સેટિંગ & સ્મૂધર (ઓછું પલ્સેટિંગ) \\
રિપલ કન્ટેન્ટ & ઊંચી (ફુલ વેવ માટે 0.48) & નીચી (3-ફેઝ ફુલ વેવ માટે 0.042) \\
ફિલ્ટરિંગ & વધુ ફિલ્ટરિંગની જરૂર & ઓછા ફિલ્ટરિંગની જરૂર \\
પાવર હેન્ડલિંગ & મર્યાદિત પાવર હેન્ડલિંગ & ઊંચુ પાવર હેન્ડલિંગ \\
ટ્રાન્સફોર્મર ઉપયોગિતા & 0.812 (ફુલ વેવ) & 0.955 (3-ફેઝ ફુલ વેવ) \\
કાર્યક્ષમતા & નીચી & ઊંચી \\
સાઇઝ & સમાન પાવર માટે નાની & ઊંચા પાવર માટે વધુ કોમ્પેક્ટ \\
\end{longtable}
}

\begin{itemize}
\tightlist
\item
  \textbf{હાર્મોનિક કન્ટેન્ટ}: પોલી-ફેઝ સિસ્ટમમાં નીચી
\item
  \textbf{TUF (ટ્રાન્સફોર્મર ઉપયોગિતા ફેક્ટર)}: પોલી-ફેઝ સિસ્ટમમાં ઊંચી
\item
  \textbf{કોસ્ટ-ઇફેક્ટિવનેસ}: ઊંચા પાવર માટે પોલી-ફેઝ વધુ આર્થિક
\end{itemize}

\end{solutionbox}
\begin{mnemonicbox}
``PERIPHERY'' - Poly-phase Efficiency Ripple
Improvement Power Handling Economy Rating Yield.

\end{mnemonicbox}
\subsection*{પ્રશ્ન 3(ક) OR [7
ગુણ]}\label{uxaaauxab0uxab6uxaa8-3uxa95-or-7-uxa97uxaa3}

\textbf{બ્લોક ડાયાગ્રામની મદદથી સૌર ફોટોવોલ્ટેઇક (PV) આધારિત પાવર જનરેશનની
કામગીરીનું વર્ણન કરો.}

\begin{solutionbox}
સોલર PV પાવર જનરેશન સેમિકન્ડક્ટર મટીરિયલનો ઉપયોગ કરીને
સૂર્યપ્રકાશને સીધો ઇલેક્ટ્રિસિટીમાં રૂપાંતરિત કરે છે.

\textbf{ડાયાગ્રામ:}

\begin{center}
\textbf{Mermaid Diagram (Code)}
\begin{verbatim}
{Shaded}
{Highlighting}[]
graph LR
    Sun((Sunlight)) {-{-}{-} PV[PV Array]}
    PV {-{-}{-} CC[Charge Controller]}
    CC {-{-}{-} B[Battery Bank]}
    B {-{-}{-} I[Inverter]}
    I {-{-}{-} L[AC Load]}
    B {-{-}{-} DCL[DC Load]}
    I {-{-}{-} G[Grid Connection]}
{Highlighting}
{Shaded}
\end{verbatim}
\end{center}

{\def\LTcaptype{none} % do not increment counter
\begin{longtable}[]{@{}ll@{}}
\toprule\noalign{}
કોમ્પોનન્ટ & કાર્ય \\
\midrule\noalign{}
\endhead
\bottomrule\noalign{}
\endlastfoot
PV એરે & ફોટોવોલ્ટેઇક ઇફેક્ટ દ્વારા સૌર ઊર્જાને DC ઇલેક્ટ્રિસિટીમાં રૂપાંતરિત કરે છે \\
ચાર્જ કંટ્રોલર & બેટરી ચાર્જિંગને નિયંત્રિત કરે છે અને ઓવરચાર્જિંગને રોકે છે \\
બેટરી બેંક & રાત્રે અથવા વાદળી સ્થિતિઓ દરમિયાન ઉપયોગ માટે ઊર્જા સંગ્રહિત કરે છે \\
ઇન્વર્ટર & AC લોડને પાવર આપવા માટે DCને ACમાં રૂપાંતરિત કરે છે \\
ગ્રિડ કનેક્શન & વધારાના પાવરને ગ્રિડમાં ફીડ કરવા માટે વૈકલ્પિક કનેક્શન \\
\end{longtable}
}

\textbf{કાર્ય સિદ્ધાંત:}

\begin{itemize}
\item
  \textbf{ફોટોવોલ્ટેઇક ઇફેક્ટ}: સૂર્યપ્રકાશના ફોટોન્સ સેમિકન્ડક્ટરમાં ઇલેક્ટ્રોન્સને મુક્ત
  કરે છે
\item
  \textbf{સેલ સ્ટ્રક્ચર}: P-N જંક્શન ઇલેક્ટ્રિક ફિલ્ડ બનાવે છે
\item
  \textbf{વોલ્ટેજ જનરેશન}: ટિપિકલ સેલ 0.5-0.6V DC ઉત્પન્ન કરે છે
\item
  \textbf{એરે કોન્ફિગરેશન}: ઇચ્છિત વોલ્ટેજ/કરંટ માટે સીરીઝ-પેરેલલ કનેક્શન
\item
  \textbf{કાર્યક્ષમતા}: સામાન્ય રીતે કોમર્શિયલ પેનલ માટે 15-22\%
\item
  \textbf{એપ્લિકેશન}: રેસિડેન્શિયલ, કોમર્શિયલ, ઔદ્યોગિક પાવર જનરેશન
\end{itemize}

\end{solutionbox}
\begin{mnemonicbox}
``SOLAR'' - Semiconductors Oriented
Light-to-electricity Array Regulation.

\end{mnemonicbox}
\subsection*{પ્રશ્ન 4(અ) [3
ગુણ]}\label{uxaaauxab0uxab6uxaa8-4uxa85-3-uxa97uxaa3}

\textbf{સ્ટેટિક સ્વીચના ફાયદા લખો.}

\begin{solutionbox}

{\def\LTcaptype{none} % do not increment counter
\begin{longtable}[]{@{}l@{}}
\toprule\noalign{}
સ્ટેટિક સ્વીચના ફાયદા \\
\midrule\noalign{}
\endhead
\bottomrule\noalign{}
\endlastfoot
કોઈ મૂવિંગ પાર્ટ્સ નથી - ઊંચી વિશ્વસનીયતા \\
સાયલેન્ટ ઓપરેશન \\
ફાસ્ટ સ્વિચિંગ રિસ્પોન્સ (માઇક્રોસેકન્ડ) \\
લાંબી ઓપરેશનલ લાઇફ \\
કોઈ કોન્ટેક્ટ બાઉન્સ અથવા આર્કિંગ નથી \\
કોમ્પેક્ટ સાઇઝ \\
ડિજિટલ કંટ્રોલ સિસ્ટમ સાથે સુસંગત \\
ઓછી મેઇન્ટેનન્સ આવશ્યકતાઓ \\
\end{longtable}
}

\begin{itemize}
\tightlist
\item
  \textbf{વિશ્વસનીયતા}: કોઈ મિકેનિકલ ઘસારો નથી
\item
  \textbf{સ્પીડ}: મિકેનિકલ સ્વિચ કરતાં ઘણી ઝડપી
\item
  \textbf{આઇસોલેશન}: ઇલેક્ટ્રિકલ આઇસોલેશન પ્રદાન કરી શકે છે
\end{itemize}

\end{solutionbox}
\begin{mnemonicbox}
``SAFE'' - Speed, Arc-free, Fast response,
Endurance.

\end{mnemonicbox}
\subsection*{પ્રશ્ન 4(બ) [4
ગુણ]}\label{uxaaauxab0uxab6uxaa8-4uxaac-4-uxa97uxaa3}

\textbf{DIAC-TRIAC નો ઉપયોગ કરીને A.C. પાવર કંટ્રોલનો સર્કિટ ડાયાગ્રામ દોરો
અને તેને સમજાવો.}

\begin{solutionbox}
DIAC-TRIAC સર્કિટ રેઝિસ્ટિવ અને ઇન્ડક્ટિવ લોડ માટે સ્મૂથ AC પાવર
કંટ્રોલ પ્રદાન કરે છે.

\textbf{ડાયાગ્રામ:}

\begin{center}
\textbf{Mermaid Diagram (Code)}
\begin{verbatim}
{Shaded}
{Highlighting}[]
graph LR
    AC[AC Supply] {-{-}{-} L[Load]}
    L {-{-}{-} T[TRIAC]}
    T {-{-}{-} AC}
    AC {-{-}{-} R1[Resistor R1]}
    R1 {-{-}{-} C[Capacitor C]}
    C {-{-}{-} D[DIAC]}
    D {-{-}{-} G[TRIAC Gate]}
    G {-{-}{-} T}
    R2[Variable Resistor R2] {-{-}{-} C}
    R2 {-{-}{-} T}
{Highlighting}
{Shaded}
\end{verbatim}
\end{center}

\textbf{કાર્યપદ્ધતિ:}

\begin{itemize}
\item
  વેરિએબલ રેઝિસ્ટર R2 કેપેસિટર Cના ચાર્જિંગ રેટને નિયંત્રિત કરે છે
\item
  જ્યારે કેપેસિટર વોલ્ટેજ DIAC બ્રેકઓવર વોલ્ટેજ પર પહોંચે છે, ત્યારે DIAC કન્ડક્ટ કરે છે
\item
  DIAC TRIAC ગેટને ટ્રિગર પલ્સ આપે છે
\item
  TRIAC બાકીના હાફ-સાયકલ માટે કન્ડક્ટ કરે છે
\item
  પ્રક્રિયા બંને હાફ-સાયકલ માટે પુનરાવર્તિત થાય છે
\item
  \textbf{ફેઝ કંટ્રોલ}: ફાયરિંગ એન્ગલ બદલીને પાવર નિયંત્રિત કરે છે
\item
  \textbf{એપ્લિકેશન}: લાઇટ ડિમર્સ, હીટર કંટ્રોલ, મોટર સ્પીડ કંટ્રોલ
\item
  \textbf{પાવર રેન્જ}: લગભગ-શૂન્યથી પૂર્ણ પાવર સુધી નિયંત્રિત કરી શકે છે
\end{itemize}

\end{solutionbox}
\begin{mnemonicbox}
``DIRECT'' - DIAC Initiates Regulated Energy Control
in TRIAC.

\end{mnemonicbox}
\subsection*{પ્રશ્ન 4(ક) [7
ગુણ]}\label{uxaaauxab0uxab6uxaa8-4uxa95-7-uxa97uxaa3}

\textbf{ટ્રિગરિંગ સર્કિટમાં UJT સાથે SCR નો ઉપયોગ કરીને DC પાવર કંટ્રોલ સર્કિટના
કાર્યનું વર્ણન કરો}

\begin{solutionbox}
UJT-ટ્રિગર્ડ SCR સર્કિટ લોડમાં DC પાવરનું ચોક્કસ નિયંત્રણ પ્રદાન કરે
છે.

\textbf{ડાયાગ્રામ:}

\begin{center}
\textbf{Mermaid Diagram (Code)}
\begin{verbatim}
{Shaded}
{Highlighting}[]
graph LR
    DC[DC Source] {-{-}{-} L[Load]}
    L {-{-}{-} SCR[SCR]}
    SCR {-{-}{-} DC}
    DC {-{-}{-} R1[Resistor R1]}
    R1 {-{-}{-} R2[Variable Resistor R2]}
    R2 {-{-}{-} C[Capacitor C]}
    C {-{-}{-} E[UJT Emitter]}
    B1[UJT Base 1] {-{-}{-} R3[Resistor R3]}
    B2[UJT Base 2] {-{-}{-} R4[Resistor R4]}
    R3 {-{-}{-} DC}
    R4 {-{-}{-} G[SCR Gate]}
    G {-{-}{-} SCR}
    E {-{-}{-} B1}
    E {-{-}{-} B2}
{Highlighting}
{Shaded}
\end{verbatim}
\end{center}

\textbf{કાર્ય સિદ્ધાંત:}

{\def\LTcaptype{none} % do not increment counter
\begin{longtable}[]{@{}
  >{\raggedright\arraybackslash}p{(\linewidth - 2\tabcolsep) * \real{0.3889}}
  >{\raggedright\arraybackslash}p{(\linewidth - 2\tabcolsep) * \real{0.6111}}@{}}
\toprule\noalign{}
\begin{minipage}[b]{\linewidth}\raggedright
સ્ટેજ
\end{minipage} & \begin{minipage}[b]{\linewidth}\raggedright
ઓપરેશન
\end{minipage} \\
\midrule\noalign{}
\endhead
\bottomrule\noalign{}
\endlastfoot
ચાર્જિંગ & R1 અને R2 કેપેસિટર Cના ચાર્જિંગ રેટને નિયંત્રિત કરે છે \\
UJT ફાયરિંગ & જ્યારે કેપેસિટર વોલ્ટેજ UJT ફાયરિંગ લેવલ પર પહોંચે, ત્યારે UJT કન્ડક્ટ કરે
છે \\
પલ્સ જનરેશન & UJT R4 પર શાર્પ ટ્રિગર પલ્સ જનરેટ કરે છે \\
SCR ટ્રિગરિંગ & પલ્સ SCR ગેટને ટ્રિગર કરે છે, SCRને ON કરી દે છે \\
પાવર કંટ્રોલ & વેરિએબલ રેઝિસ્ટર R2 ટાઈમિંગને એડજસ્ટ કરે છે, એવરેજ પાવરને કંટ્રોલ કરે
છે \\
\end{longtable}
}

\begin{itemize}
\tightlist
\item
  \textbf{ચોક્કસ કંટ્રોલ}: UJT સ્થિર, અનુમાનિત ટ્રિગરિંગ પ્રદાન કરે છે
\item
  \textbf{એપ્લિકેશન}: બેટરી ચાર્જર, DC મોટર સ્પીડ કંટ્રોલ, તાપમાન નિયંત્રણ
\item
  \textbf{ફાયદા}: ઓછી કિંમત, ઉચ્ચ વિશ્વસનીયતા, સારી તાપમાન સ્થિરતા
\item
  \textbf{કંટ્રોલ રેન્જ}: લગભગ-શૂન્યથી પૂર્ણ પાવર સુધીની વિશાળ રેન્જ
\end{itemize}

\end{solutionbox}
\begin{mnemonicbox}
``SCRUP'' - SCR Using Pulse from UJT for Power
control.

\end{mnemonicbox}
\subsection*{પ્રશ્ન 4(અ) OR [3
ગુણ]}\label{uxaaauxab0uxab6uxaa8-4uxa85-or-3-uxa97uxaa3}

\textbf{ડાઈ-ઈલેક્ટ્રીક હિટીંગના ઉપયોગો વર્ણવો.}

\begin{solutionbox}

{\def\LTcaptype{none} % do not increment counter
\begin{longtable}[]{@{}l@{}}
\toprule\noalign{}
ડાઈલેક્ટ્રિક હિટીંગના ઉપયોગો \\
\midrule\noalign{}
\endhead
\bottomrule\noalign{}
\endlastfoot
પ્લાસ્ટિક વેલ્ડિંગ અને સીલિંગ \\
લાકડાના ગ્લુઇંગ અને ક્યુરિંગ \\
ફૂડ પ્રોસેસિંગ (પ્રી-કુકિંગ, ડિફ્રોસ્ટિંગ) \\
ટેક્સટાઇલ ડ્રાઇંગ અને પ્રોસેસિંગ \\
પેપર અને બોર્ડ ડ્રાઇંગ \\
ફાર્માસ્યુટિકલ પ્રોડક્ટ્સ ડ્રાઇંગ \\
મેડિકલ એપ્લિકેશન (હાઇપરથર્મિયા ટ્રીટમેન્ટ) \\
રબર વલ્કેનાઇઝેશન \\
\end{longtable}
}

\begin{itemize}
\tightlist
\item
  \textbf{મટીરિયલ રિક્વાયરમેન્ટ}: પોલર મોલેક્યુલ્સ ધરાવતા નબળા કન્ડક્ટર્સ સાથે શ્રેષ્ઠ
  કામ કરે છે
\item
  \textbf{ફ્રિક્વન્સી રેન્જ}: સામાન્ય રીતે 10-100 MHz
\item
  \textbf{ફાયદા}: યુનિફોર્મ હીટિંગ, ઝડપી પ્રોસેસિંગ, ઊર્જા કાર્યક્ષમતા
\end{itemize}

\end{solutionbox}
\begin{mnemonicbox}
``POWER'' - Plastics, Organics, Wood, Edibles, and
Rubber processing.

\end{mnemonicbox}
\subsection*{પ્રશ્ન 4(બ) OR [4
ગુણ]}\label{uxaaauxab0uxab6uxaa8-4uxaac-or-4-uxa97uxaa3}

\textbf{ત્રણ તબક્કાના IC555 ટાઈમર સર્કિટ દોરો અને સમજાવો.}

\begin{solutionbox}
ત્રણ-સ્ટેજ IC555 ટાઈમર સર્કિટ સિક્વેન્શિયલ ટાઈમિંગ ઓપરેશન્સ પ્રદાન
કરે છે.

\textbf{ડાયાગ્રામ:}

\begin{center}
\textbf{Mermaid Diagram (Code)}
\begin{verbatim}
{Shaded}
{Highlighting}[]
graph LR
    subgraph "Timer 1"
        IC1[555 Timer] 
    end
    subgraph "Timer 2"
        IC2[555 Timer]
    end
    subgraph "Timer 3"
        IC3[555 Timer]
    end
    TR[Trigger Input] {-{-}{} IC1}
    IC1 {-{-}{} O1[Output 1]}
    O1 {-{-}{} IC2}
    IC2 {-{-}{} O2[Output 2]}
    O2 {-{-}{} IC3}
    IC3 {-{-}{} O3[Output 3]}
{Highlighting}
{Shaded}
\end{verbatim}
\end{center}

\textbf{કાર્યપદ્ધતિ:}

\begin{itemize}
\item
  પ્રથમ ટાઈમર બાહ્ય ટ્રિગર દ્વારા સક્રિય થાય છે
\item
  પ્રથમ ટાઈમરનો આઉટપુટ બીજા ટાઈમરને ટ્રિગર કરે છે
\item
  બીજા ટાઈમરનો આઉટપુટ ત્રીજા ટાઈમરને ટ્રિગર કરે છે
\item
  દરેક ટાઈમર સ્વતંત્ર રીતે એડજસ્ટ કરી શકાય છે
\item
  \textbf{એપ્લિકેશન}: ઔદ્યોગિક સિક્વેન્સિંગ, પ્રોસેસ કંટ્રોલ, એનિમેશન ઇફેક્ટ્સ
\item
  \textbf{ટાઈમિંગ રેન્જ}: યોગ્ય કોમ્પોનન્ટ પસંદગી સાથે માઇક્રોસેકન્ડથી કલાકો સુધી
\item
  \textbf{ફીચર્સ}: સ્થિર ટાઈમિંગ, સપ્લાય વેરિએશન્સથી પ્રતિકાર
\item
  \textbf{ફાયદા}: સરળ ડિઝાઇન, વિશ્વસનીય ઓપરેશન, ઓછી કિંમત
\end{itemize}

\end{solutionbox}
\begin{mnemonicbox}
``THREE-SET'' - THREE Stage Electronic Timers in
sequence.

\end{mnemonicbox}
\subsection*{પ્રશ્ન 4(ક) OR [7
ગુણ]}\label{uxaaauxab0uxab6uxaa8-4uxa95-or-7-uxa97uxaa3}

\textbf{ઇન્ડક્શન હીટિંગના કાર્ય સિદ્ધાંતનું વર્ણન કરો. અને ઇન્ડક્શન હીટિંગના
ફાયદાઓ-ગેરફાયદાઓની યાદી બનાવો.}

\begin{solutionbox}
ઇન્ડક્શન હીટિંગ ઇલેક્ટ્રિકલી કન્ડક્ટિવ મટીરિયલ્સને ગરમ કરવા માટે
ઇલેક્ટ્રોમેગ્નેટિક ઇન્ડક્શનનો ઉપયોગ કરે છે.

\textbf{ડાયાગ્રામ:}

\begin{center}
\textbf{Mermaid Diagram (Code)}
\begin{verbatim}
{Shaded}
{Highlighting}[]
graph LR
    PS[Power Supply] {-{-}{} INV[Inverter]}
    INV {-{-}{} LC[Matching Circuit]}
    LC {-{-}{} WC[Work Coil]}
    WC {-{-}{} W[Workpiece]}
    FC[Feedback Control] {-{-}{} INV}
{Highlighting}
{Shaded}
\end{verbatim}
\end{center}

\textbf{કાર્ય સિદ્ધાંત:}

\begin{itemize}
\tightlist
\item
  વર્ક કોઇલમાં હાઇ ફ્રિક્વન્સી AC અલ્ટરનેટિંગ મેગ્નેટિક ફિલ્ડ બનાવે છે
\item
  મેગ્નેટિક ફિલ્ડ વર્કપીસમાં એડી કરંટ પ્રેરિત કરે છે
\item
  મટીરિયલના રેઝિસ્ટન્સને કારણે એડી કરંટ ગરમી ઉત્પન્ન કરે છે
\item
  હીટિંગ બાહ્ય સ્રોતથી નહીં, પરંતુ વર્કપીસની અંદર થાય છે
\end{itemize}

{\def\LTcaptype{none} % do not increment counter
\begin{longtable}[]{@{}ll@{}}
\toprule\noalign{}
ફાયદા & ગેરફાયદા \\
\midrule\noalign{}
\endhead
\bottomrule\noalign{}
\endlastfoot
ઝડપી હીટિંગ & ઊંચી પ્રારંભિક ઉપકરણ કિંમત \\
ઊર્જા કાર્યક્ષમ (80-90\%) & ઇલેક્ટ્રિકલી કન્ડક્ટિવ મટીરિયલ્સ પૂરતું મર્યાદિત \\
ચોક્કસ તાપમાન કંટ્રોલ & હાઇ-ફ્રિક્વન્સી પાવર સપ્લાયની જરૂર છે \\
કોઈ દહન વિના ક્લીન પ્રોસેસ & ચોક્કસ એપ્લિકેશન માટે જટિલ કોઇલ ડિઝાઇન \\
લોકેલાઇઝ્ડ હીટિંગ શક્ય & ઊંચી પાવર આવશ્યકતાઓ \\
સુસંગત, પુનરાવર્તનીય પરિણામો & વોટર કૂલિંગ સિસ્ટમની જરૂર છે \\
પર્યાવરણને અનુકૂળ & ઇલેક્ટ્રોમેગ્નેટિક ઇન્ટરફેરન્સ મુદ્દાઓ \\
સુધારેલી કાર્ય સ્થિતિઓ & મર્યાદિત પેનિટ્રેશન ડેપ્થ \\
\end{longtable}
}

\begin{itemize}
\tightlist
\item
  \textbf{ફ્રિક્વન્સી રેન્જ}: એપ્લિકેશન પર આધારિત 1 kHz થી 1 MHz
\item
  \textbf{એપ્લિકેશન}: હીટ ટ્રીટમેન્ટ, મેલ્ટિંગ, બ્રેઝિંગ, સોલ્ડરિંગ
\end{itemize}

\end{solutionbox}
\begin{mnemonicbox}
``EDDY'' - Electromagnetic Device Develops Yield of
heat.

\end{mnemonicbox}
\subsection*{પ્રશ્ન 5(અ) [3
ગુણ]}\label{uxaaauxab0uxab6uxaa8-5uxa85-3-uxa97uxaa3}

\textbf{ડીસી શન્ટ મોટર સ્પીડને નિયંત્રિત કરવા માટે સોલિડ સ્ટેટ સર્કિટ દોરો અને
સમજાવો.}

\begin{solutionbox}
DC શન્ટ મોટર સ્પીડ કંટ્રોલ માટેની સોલિડ-સ્ટેટ સર્કિટ આર્મેચર વોલ્ટેજને
કંટ્રોલ કરવા માટે SCRનો ઉપયોગ કરે છે.

\textbf{ડાયાગ્રામ:}

\begin{center}
\textbf{Mermaid Diagram (Code)}
\begin{verbatim}
{Shaded}
{Highlighting}[]
graph LR
    AC[AC Supply] {-{-}{-} BR[Bridge Rectifier]}
    BR {-{-}{-} SCR[SCR]}
    SCR {-{-}{-} A[Armature]}
    A {-{-}{-} BR}
    BR {-{-}{-} F[Field Winding]}
    F {-{-}{-} BR}
    RC[Firing Circuit] {-{-}{-} SCR}
{Highlighting}
{Shaded}
\end{verbatim}
\end{center}

\begin{itemize}
\tightlist
\item
  \textbf{આર્મેચર વોલ્ટેજ કંટ્રોલ}: SCR આર્મેચરને વોલ્ટેજ કંટ્રોલ કરે છે
\item
  \textbf{ફિલ્ડ વાઇન્ડિંગ}: સીધો DC સપ્લાયથી જોડાયેલ
\item
  \textbf{સ્પીડ કંટ્રોલ}: SCR ફાયરિંગ એંગલ બદલીને
\item
  \textbf{ફાયદા}: સ્મૂથ કંટ્રોલ, ઊંચી કાર્યક્ષમતા, કોમ્પેક્ટ સાઇઝ
\end{itemize}

\end{solutionbox}
\begin{mnemonicbox}
``SAFE'' - SCR Armature Firing for Efficient
control.

\end{mnemonicbox}
\subsection*{પ્રશ્ન 5(બ) [4
ગુણ]}\label{uxaaauxab0uxab6uxaa8-5uxaac-4-uxa97uxaa3}

\textbf{સ્ટેપર મોટરના કાર્ય સિદ્ધાંતને સમજાવો.}

\begin{solutionbox}
સ્ટેપર મોટર ઇલેક્ટ્રિકલ પલ્સને ડિસ્ક્રીટ મિકેનિકલ મૂવમેન્ટમાં રૂપાંતરિત
કરે છે.

\textbf{ડાયાગ્રામ:}

\begin{center}
\textbf{Mermaid Diagram (Code)}
\begin{verbatim}
{Shaded}
{Highlighting}[]
graph TD
    subgraph "Stepper Motor"
        R[Rotor]
        S1[Stator Winding 1]
        S2[Stator Winding 2]
        S3[Stator Winding 3]
        S4[Stator Winding 4]
    end
{Highlighting}
{Shaded}
\end{verbatim}
\end{center}

\textbf{કાર્ય સિદ્ધાંત:}

\begin{itemize}
\tightlist
\item
  ક્રમમાં સ્ટેટર વાઇન્ડિંગ્સને એનર્જાઇઝ કરવાથી રોટેટિંગ મેગ્નેટિક ફિલ્ડ બને છે
\item
  પર્માનન્ટ મેગ્નેટ રોટર મેગ્નેટિક ફિલ્ડ સાથે એલાઇન થાય છે
\item
  દરેક પલ્સ ``સ્ટેપ'' એંગલ દ્વારા ચોક્કસ રોટેશન બનાવે છે
\item
  સ્ટેપ એંગલ મોટર કન્સ્ટ્રક્શન દ્વારા નિર્ધારિત થાય છે (સામાન્ય રીતે 1.8^\circ અથવા 0.9^\circ)
\end{itemize}

{\def\LTcaptype{none} % do not increment counter
\begin{longtable}[]{@{}ll@{}}
\toprule\noalign{}
પ્રકાર & ખાસિયતો \\
\midrule\noalign{}
\endhead
\bottomrule\noalign{}
\endlastfoot
વેરિએબલ રિલક્ટન્સ & કોઈ પર્માનન્ટ મેગ્નેટ નથી, મેગ્નેટિક રિલક્ટન્સ પર આધાર રાખે છે \\
પર્માનન્ટ મેગ્નેટ & પર્માનન્ટ મેગ્નેટ રોટરનો ઉપયોગ કરે છે \\
હાઇબ્રિડ & બંને પ્રકારની ખાસિયતો સંયોજિત કરે છે \\
\end{longtable}
}

\begin{itemize}
\tightlist
\item
  \textbf{ચોક્કસ પોઝિશનિંગ}: ચોક્કસ ઇન્ક્રિમેન્ટ સ્ટેપ્સમાં મૂવમેન્ટ
\item
  \textbf{ઓપન-લૂપ કંટ્રોલ}: પોઝિશન કંટ્રોલ માટે કોઈ ફીડબેક જરૂરી નથી
\item
  \textbf{હોલ્ડિંગ ટોર્ક}: એનર્જાઇઝ્ડ હોય ત્યારે પોઝિશન જાળવે છે
\end{itemize}

\end{solutionbox}
\begin{mnemonicbox}
``STEP'' - Sequential Triggering Enables Precise
positioning.

\end{mnemonicbox}
\subsection*{પ્રશ્ન 5(ક) [7
ગુણ]}\label{uxaaauxab0uxab6uxaa8-5uxa95-7-uxa97uxaa3}

\textbf{PLC નો બ્લોક ડાયાગ્રામ દોરો અને દરેક બ્લોકની કામગીરી સમજાવો.}

\begin{solutionbox}
પ્રોગ્રામેબલ લોજિક કંટ્રોલર (PLC) એ ઔદ્યોગિક પ્રોસેસના ઓટોમેશન માટે
વપરાતું ડિજિટલ કમ્પ્યુટર છે.

\textbf{ડાયાગ્રામ:}

\begin{center}
\textbf{Mermaid Diagram (Code)}
\begin{verbatim}
{Shaded}
{Highlighting}[]
graph TD
    PS[Power Supply] {-{-}{-} CPU[Central Processing Unit]}
    I[Input Modules] {-{-}{-} CPU}
    CPU {-{-}{-} O[Output Modules]}
    M[Memory] {-{-}{-} CPU}
    P[Programming Device] {-{-}{-} CPU}
    C[Communication Module] {-{-}{-} CPU}
{Highlighting}
{Shaded}
\end{verbatim}
\end{center}

{\def\LTcaptype{none} % do not increment counter
\begin{longtable}[]{@{}ll@{}}
\toprule\noalign{}
બ્લોક & કાર્ય \\
\midrule\noalign{}
\endhead
\bottomrule\noalign{}
\endlastfoot
પાવર સપ્લાય & આંતરિક ઉપયોગ માટે મુખ્ય ACને DCમાં રૂપાંતરિત કરે છે \\
CPU & પ્રોગ્રામ એક્ઝિક્યુટ કરે છે, ડેટા પ્રોસેસ કરે છે, ઓપરેશન્સ મેનેજ કરે છે \\
ઇનપુટ મોડ્યુલ્સ & સેન્સર, સ્વિચ અને ફિલ્ડ ડિવાઇસ સાથે ઇન્ટરફેસ \\
આઉટપુટ મોડ્યુલ્સ & એક્ચ્યુએટર, મોટર, વાલ્વ અને ઇન્ડિકેટર કંટ્રોલ કરે છે \\
મેમરી & પ્રોગ્રામ અને ડેટા સ્ટોર કરે છે (ROM, RAM, EEPROM) \\
પ્રોગ્રામિંગ ડિવાઇસ & પ્રોગ્રામિંગ માટે એક્સટર્નલ કમ્પ્યુટર અથવા ટર્મિનલ \\
કમ્યુનિકેશન મોડ્યુલ & અન્ય PLCs, SCADA, HMI સાથે ઇન્ટરફેસ \\
\end{longtable}
}

\begin{itemize}
\tightlist
\item
  \textbf{સ્કેન સાયકલ}: ઇનપુટ સ્કેનિંગ \rightarrow પ્રોગ્રામ એક્ઝિક્યુશન \rightarrow આઉટપુટ અપડેટિંગ
\item
  \textbf{ફાયદા}: વિશ્વસનીયતા, ફ્લેક્સિબિલિટી, મોડ્યુલર ડિઝાઇન, સરળ ટ્રબલશૂટિંગ
\item
  \textbf{એપ્લિકેશન}: મેન્યુફેક્ચરિંગ ઓટોમેશન, પ્રોસેસ કંટ્રોલ, મટીરિયલ હેન્ડલિંગ
\item
  \textbf{પ્રોગ્રામિંગ}: લેડર લોજિક, ફંક્શન બ્લોક ડાયાગ્રામ, સ્ટ્રક્ચર્ડ ટેક્સ્ટ
\end{itemize}

\end{solutionbox}
\begin{mnemonicbox}
``PILOT'' - Processing Inputs and Logic for Outputs
with Timing control.

\end{mnemonicbox}
\subsection*{પ્રશ્ન 5(અ) OR [3
ગુણ]}\label{uxaaauxab0uxab6uxaa8-5uxa85-or-3-uxa97uxaa3}

\textbf{ડીસી સર્વો મોટરનું બંધારણ દોરો અને સમજાવો.}

\begin{solutionbox}
DC સર્વો મોટર ચોક્કસ પોઝિશન અને સ્પીડ કંટ્રોલ માટે ડિઝાઇન કરવામાં
આવે છે.

\textbf{ડાયાગ્રામ:}

\begin{center}
\textbf{Mermaid Diagram (Code)}
\begin{verbatim}
{Shaded}
{Highlighting}[]
graph TD
    subgraph "DC Servo Motor"
        A[Armature]
        F[Field Winding]
        S[Shaft]
        FB[Feedback Device]
    end
{Highlighting}
{Shaded}
\end{verbatim}
\end{center}

\textbf{કોમ્પોનન્ટ્સ:}

\begin{itemize}
\item
  \textbf{આર્મેચર}: ઝડપી પ્રતિસાદ માટે લો ઇનર્શિયા
\item
  \textbf{ફિલ્ડ સિસ્ટમ}: મેગ્નેટિક ફિલ્ડ પ્રદાન કરે છે (આધુનિક મોટરમાં પર્માનન્ટ
  મેગ્નેટ્સ)
\item
  \textbf{ફીડબેક ડિવાઇસ}: પોઝિશન સેન્સર (એન્કોડર/રિઝોલ્વર/ટેકોમીટર)
\item
  \textbf{હાઉસિંગ}: બેરિંગ્સ અને માઉન્ટિંગ પ્રોવિઝન્સ ધરાવે છે
\item
  \textbf{હાઇ ટોર્ક-ટુ-ઇનર્શિયા રેશિયો}: ઝડપી સ્ટાર્ટ અને સ્ટોપની મંજૂરી આપે છે
\item
  \textbf{લિનિયર ટોર્ક-સ્પીડ કેરેક્ટરિસ્ટિક્સ}: ચોક્કસ કંટ્રોલને સક્ષમ બનાવે છે
\end{itemize}

\end{solutionbox}
\begin{mnemonicbox}
``SAFE'' - Sensitive Armature with Feedback for
Exactness.

\end{mnemonicbox}
\subsection*{પ્રશ્ન 5(બ) OR [4
ગુણ]}\label{uxaaauxab0uxab6uxaa8-5uxaac-or-4-uxa97uxaa3}

\textbf{ડીસી સીરીઝ મોટરની ઝડપને નિયંત્રિત કરવા માટે સર્કિટ દોરો અને સમજાવો.}

\begin{solutionbox}
SCRનો ઉપયોગ કરીને DC સીરીઝ મોટર સ્પીડ કંટ્રોલ સર્કિટ.

\textbf{ડાયાગ્રામ:}

\begin{center}
\textbf{Mermaid Diagram (Code)}
\begin{verbatim}
{Shaded}
{Highlighting}[]
graph LR
    AC[AC Supply] {-{-}{-} BR[Bridge Rectifier]}
    BR {-{-}{-} SCR[SCR]}
    SCR {-{-}{-} S[Series Field]}
    S {-{-}{-} A[Armature]}
    A {-{-}{-} BR}
    FC[Firing Circuit] {-{-}{-} SCR}
    P[Potentiometer] {-{-}{-} FC}
{Highlighting}
{Shaded}
\end{verbatim}
\end{center}

\textbf{કાર્યપદ્ધતિ:}

\begin{itemize}
\item
  બ્રિજ રેક્ટિફાયર ACને DCમાં રૂપાંતરિત કરે છે
\item
  SCR મોટરને એવરેજ વોલ્ટેજ કંટ્રોલ કરે છે
\item
  ફાયરિંગ એંગલ પોટેન્શિયોમીટર દ્વારા નિયંત્રિત થાય છે
\item
  સીરીઝ ફિલ્ડ અને આર્મેચર કરંટ સમાન છે
\item
  ઓછા લોડ પર સ્પીડ વોલ્ટેજના વિપરીત બદલાય છે
\item
  \textbf{આર્મેચર વોલ્ટેજ કંટ્રોલ}: સ્પીડ કંટ્રોલ માટે પ્રાથમિક પદ્ધતિ
\item
  \textbf{ટોર્ક કેરેક્ટરિસ્ટિક્સ}: ઉચ્ચ સ્ટાર્ટિંગ ટોર્ક જાળવવામાં આવે છે
\item
  \textbf{સ્પીડ રેન્જ}: સ્થિર ઓપરેશન માટે સામાન્ય રીતે 3:1
\end{itemize}

\end{solutionbox}
\begin{mnemonicbox}
``SCRAM'' - SCR Controls Rectified Armature and
Motor speed.

\end{mnemonicbox}
\subsection*{પ્રશ્ન 5(ક) OR [7
ગુણ]}\label{uxaaauxab0uxab6uxaa8-5uxa95-or-7-uxa97uxaa3}

\textbf{સ્ટેપર મોટર નું બંધારણ અને કાર્યપદ્ધતિ સમજાવી તેના ઉપયોગો જણાવો}

\begin{solutionbox}
સ્ટેપર મોટર એ ઇલેક્ટ્રોમેકેનિકલ ડિવાઇસ છે જે ઇલેક્ટ્રિકલ પલ્સને ડિસ્ક્રીટ
મિકેનિકલ મૂવમેન્ટમાં રૂપાંતરિત કરે છે.

\textbf{બંધારણ:}

\textbf{ડાયાગ્રામ:}

\begin{center}
\textbf{Mermaid Diagram (Code)}
\begin{verbatim}
{Shaded}
{Highlighting}[]
graph TD
    subgraph "Stepper Motor"
        R[Rotor {- Permanent Magnet]}
        S[Stator {- Electromagnetic Coils]}
        SH[Shaft]
    end
{Highlighting}
{Shaded}
\end{verbatim}
\end{center}

{\def\LTcaptype{none} % do not increment counter
\begin{longtable}[]{@{}ll@{}}
\toprule\noalign{}
કોમ્પોનન્ટ & વિગત \\
\midrule\noalign{}
\endhead
\bottomrule\noalign{}
\endlastfoot
સ્ટેટર & ફેઝમાં ગોઠવાયેલા મલ્ટિપલ કોઇલ વાઇન્ડિંગ્સ ધરાવે છે \\
રોટર & પર્માનન્ટ મેગ્નેટ અથવા સોફ્ટ આયર્ન (રિલક્ટન્સ પ્રકાર) \\
બેરિંગ્સ & શાફ્ટને સપોર્ટ કરે છે અને રોટેશનની મંજૂરી આપે છે \\
હાઉસિંગ & બધા કોમ્પોનન્ટ્સ ધારણ કરતું મિકેનિકલ સ્ટ્રક્ચર \\
લીડ્સ & સ્ટેટર વાઇન્ડિંગ્સ સાથે ઇલેક્ટ્રિકલ કનેક્શન \\
\end{longtable}
}

\textbf{કાર્ય સિદ્ધાંત:}

\begin{itemize}
\tightlist
\item
  ડિજિટલ પલ્સ ક્રમમાં સ્ટેટર વાઇન્ડિંગ્સને એનર્જાઇઝ કરે છે
\item
  મેગ્નેટિક ફિલ્ડ સ્ટેટરની આસપાસ સ્ટેપ્સમાં ફરે છે
\item
  રોટર ચોક્કસ એંગ્યુલર સ્ટેપ્સમાં મેગ્નેટિક ફિલ્ડને અનુસરે છે
\item
  દિશા એનર્જાઈઝેશનના ક્રમ દ્વારા નિયંત્રિત થાય છે
\item
  સ્પીડ પલ્સ ફ્રિક્વન્સી દ્વારા નિયંત્રિત થાય છે
\end{itemize}

\textbf{સ્ટેપર મોટરના પ્રકાર:}

{\def\LTcaptype{none} % do not increment counter
\begin{longtable}[]{@{}ll@{}}
\toprule\noalign{}
પ્રકાર & ખાસિયતો \\
\midrule\noalign{}
\endhead
\bottomrule\noalign{}
\endlastfoot
વેરિએબલ રિલક્ટન્સ & કોઈ પર્માનન્ટ મેગ્નેટ નહીં, ઉચ્ચ સ્પીડ, ઓછો ટોર્ક \\
પર્માનન્ટ મેગ્નેટ & સરળ ડિઝાઇન, મધ્યમ ટોર્ક, ઓછી રેઝોલ્યુશન \\
હાઇબ્રિડ & બંને ડિઝાઇન્સને સંયોજિત કરે છે, ઉચ્ચ રેઝોલ્યુશન, સારો ટોર્ક \\
\end{longtable}
}

\textbf{ઉપયોગો:}

\begin{itemize}
\tightlist
\item
  CNC મશીન અને 3D પ્રિન્ટર્સ
\item
  રોબોટિક્સ અને ઓટોમેશન
\item
  કેમેરા લેન્સ ફોકસિંગ મિકેનિઝમ
\item
  પ્રિસિઝન પોઝિશનિંગ સિસ્ટમ
\item
  મેડિકલ ઇક્વિપમેન્ટ
\item
  ઓફિસ ઇક્વિપમેન્ટ (પ્રિન્ટર, સ્કેનર)
\item
  ઓટોમોટિવ એપ્લિકેશન (હેડલાઇટ પોઝિશનિંગ)
\item
  નાના કન્ઝ્યુમર ડિવાઇસિસ
\end{itemize}

\end{solutionbox}
\begin{mnemonicbox}
``REACT'' - Rotation Exactly At Controlled Timing.

\end{mnemonicbox}

\end{document}
