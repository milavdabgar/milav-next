\documentclass[10pt,a4paper]{article}

% content/resources/templates/preamble.tex
\usepackage[margin=0.6in]{geometry}
\author{Milav Dabgar}
\usepackage{amsmath,amssymb,amsthm}
\usepackage{booktabs}
\usepackage{multirow}
\usepackage{xcolor}
\usepackage{tcolorbox}
\tcbuselibrary{breakable,skins}
\usepackage[colorlinks=true,linkcolor=blue]{hyperref}
\usepackage{titlesec}
\usepackage{enumitem}
\usepackage{tikz}
\usepackage{pgfplots}
\usepackage{circuitikz}
\usepackage[version=4]{mhchem}
\usepackage{longtable}
\usepackage{array}
\usepackage{float}
\usepackage{caption}
\usepackage{listings}

\lstset{
  basicstyle=\small\ttfamily,
  breaklines=true,
  breakatwhitespace=false,
  postbreak=\mbox{\textcolor{red}{$\hookrightarrow$}\space},
  float=false,
  numbers=left,
  numberstyle=\tiny\color{gray},
  numbersep=10pt,
  xleftmargin=2em,
  keywordstyle=\color{blue},
  commentstyle=\color{green!60!black},
  stringstyle=\color{purple},
  backgroundcolor=\color{gray!5},
  showstringspaces=false,
  tabsize=2,
  captionpos=b,
  keepspaces=true,
  columns=flexible
}

\pgfplotsset{compat=1.18}
\usetikzlibrary{shapes,arrows,positioning,calc,patterns,decorations.pathmorphing,decorations.markings,arrows.meta}

% Color scheme
\definecolor{headcolor}{RGB}{0,102,204}
\definecolor{keycolor}{RGB}{220,20,60}
\definecolor{solutioncolor}{RGB}{34,139,34}
\definecolor{mnemoniccolor}{RGB}{148,0,211}
\definecolor{codecolor}{RGB}{0,0,100}

% Spacing
\setlength{\parskip}{3pt}
\setlist[itemize]{nosep}
\setlist[enumerate]{nosep}

% Title formatting
\titleformat{\section}{\Large\bfseries\color{headcolor}}{\thesection}{1em}{}
\titleformat{\subsection}{\large\bfseries\color{headcolor}}{\thesubsection}{1em}{}

% Pandoc tightlist compatibility
\providecommand{\tightlist}{%
  \setlength{\itemsep}{0pt}\setlength{\parskip}{0pt}}

% Pandoc longtable compatibility
\newcounter{none}
\def\thenone{}


% content/resources/templates/english-boxes.tex
% This file is currently empty - it exists to maintain consistency with the import structure.
% Add custom environments here if needed in the future.


\begin{document}

\begin{center}
{\Huge\bfseries\color{headcolor} Subject Name Solutions}\\[5pt]
{\LARGE 4331103 -- Summer 2024}\\[3pt]
{\large Semester 1 Study Material}\\[3pt]
{\normalsize\textit{Detailed Solutions and Explanations}}
\end{center}

\vspace{10pt}

\subsection*{Question 1(a) [3 marks]}\label{q1a}

\textbf{Explain two transistor analogies of SCR.}

\begin{solutionbox}
SCR can be represented as a two-transistor model with
interconnected PNP and NPN transistors.

\textbf{Diagram:}

\begin{center}
\textbf{Mermaid Diagram (Code)}
\begin{verbatim}
{Shaded}
{Highlighting}[]
graph LR
    A[Anode] {-{-}{-} B1[PNP Base]}
    B1 {-{-}{-} C1[PNP Collector]}
    C1 {-{-}{-} E2[NPN Emitter]}
    E2 {-{-}{-} B2[NPN Base]}
    B2 {-{-}{-} C2[NPN Collector]}
    C2 {-{-}{-} K[Cathode]}
    G[Gate] {-{-}{-} B2}
    E1[PNP Emitter] {-{-}{-} A}
    E1 {-{-}{-} B2}
    C2 {-{-}{-} B1}
{Highlighting}
{Shaded}
\end{verbatim}
\end{center}

\begin{itemize}
\tightlist
\item
  \textbf{Regenerative action}: When gate current triggers NPN, it
  causes PNP to conduct, creating self-sustaining current
\item
  \textbf{Latching mechanism}: Once both transistors are ON, gate loses
  control as feedback path maintains conduction
\end{itemize}

\end{solutionbox}
\begin{mnemonicbox}
``Push-Pull Network Triggers Sustained Conduction''

\end{mnemonicbox}
\subsection*{Question 1(b) [4 marks]}\label{q1b}

\textbf{Explain working and characteristic of IGBT.}

\begin{solutionbox}
IGBT (Insulated Gate Bipolar Transistor) combines
MOSFET input characteristics with BJT output capabilities.

\textbf{Diagram:}

\begin{center}
\textbf{Mermaid Diagram (Code)}
\begin{verbatim}
{Shaded}
{Highlighting}[]
graph LR
    G[Gate] {-{-}{-} MOS[MOSFET Section]}
    MOS {-{-}{-} BJT[BJT Section]}
    BJT {-{-}{-} C[Collector]}
    E[Emitter] {-{-}{-} BJT}
{Highlighting}
{Shaded}
\end{verbatim}
\end{center}

\textbf{Characteristics Table:}

{\def\LTcaptype{none} % do not increment counter
\begin{longtable}[]{@{}ll@{}}
\toprule\noalign{}
Feature & Characteristic \\
\midrule\noalign{}
\endhead
\bottomrule\noalign{}
\endlastfoot
Switching & Fast turn-on, moderate turn-off \\
Control & Voltage-controlled like MOSFET \\
Conduction & Low forward voltage drop like BJT \\
Applications & High voltage, medium frequency switching \\
\end{longtable}
}

\begin{itemize}
\tightlist
\item
  \textbf{Input advantage}: Voltage-controlled gate with high impedance
  requires minimal drive power
\item
  \textbf{Output advantage}: Low on-state voltage drop even at high
  current densities
\end{itemize}

\end{solutionbox}
\begin{mnemonicbox}
``MOSFET Input, BJT Output, Makes Perfect Power
Switch''

\end{mnemonicbox}
\subsection*{Question 1(c) [7 marks]}\label{q1c}

\textbf{Explain construction, working and characteristic of DIAC.}

\begin{solutionbox}
DIAC (DIode for Alternating Current) is a bidirectional
triggering device used in thyristor control circuits.

\textbf{Diagram:}

\begin{center}
\textbf{Mermaid Diagram (Code)}
\begin{verbatim}
{Shaded}
{Highlighting}[]
graph LR
    A[Terminal A] {-{-}{-} P1[P{-}region]}
    P1 {-{-}{-} N1[N{-}region]}
    N1 {-{-}{-} P2[P{-}region]}
    P2 {-{-}{-} N2[N{-}region]}
    N2 {-{-}{-} B[Terminal B]}
{Highlighting}
{Shaded}
\end{verbatim}
\end{center}

\textbf{Characteristics Curve:}

\begin{verbatim}
                    I
                    \^{}
                    |      /
                    |     /
                    |    /
            {-{-}{-}{-}{-}{-}{-}{-}+{-}{-}{-}/{-}{-}{-}{-}{-}{-}{-}{-}{-} V}
                   /|
                  / |
                 /  |
                /   |
               Break{-over voltage}
\end{verbatim}

\textbf{Construction \& Operation Table:}

{\def\LTcaptype{none} % do not increment counter
\begin{longtable}[]{@{}ll@{}}
\toprule\noalign{}
Feature & Description \\
\midrule\noalign{}
\endhead
\bottomrule\noalign{}
\endlastfoot
Structure & Five-layer P-N-P-N with no gate terminal \\
Operation & Blocks current until break-over voltage is reached \\
Breakover & Typically 30-40V in either direction \\
Symmetry & Identical response in both directions \\
Application & Trigger device for TRIACs in AC circuits \\
\end{longtable}
}

\begin{itemize}
\tightlist
\item
  \textbf{Blocking state}: Below breakover voltage, high resistance
  prevents current flow
\item
  \textbf{Conducting state}: Above breakover voltage, negative
  resistance region enables sudden conduction
\item
  \textbf{Bidirectional}: Functions identically for positive and
  negative voltages
\end{itemize}

\end{solutionbox}
\begin{mnemonicbox}
``Break Voltage Both Ways, Then Current Flows''

\end{mnemonicbox}
\subsection*{Question 1(c) OR [7
marks]}\label{q1c}

\textbf{Explain construction and working of Opto-Isolator and Opto-SCR}

\begin{solutionbox}
Opto-devices use light to transfer signals while
maintaining electrical isolation between circuits.

\textbf{Opto-Isolator Diagram:}

\begin{center}
\textbf{Mermaid Diagram (Code)}
\begin{verbatim}
{Shaded}
{Highlighting}[]
graph LR
    A[Input] {-{-}{-} L[LED]}
    L {-{-}{-} G[Glass/Plastic]}
    G {-{-}{-} D[Phototransistor]}
    D {-{-}{-} O[Output]}
{Highlighting}
{Shaded}
\end{verbatim}
\end{center}

\textbf{Opto-SCR Diagram:}

\begin{center}
\textbf{Mermaid Diagram (Code)}
\begin{verbatim}
{Shaded}
{Highlighting}[]
graph LR
    A[Input] {-{-}{-} L[LED]}
    L {-{-}{-} G[Glass/Plastic]}
    G {-{-}{-} S[Light{-}sensitive SCR]}
    S {-{-}{-} O[Output]}
{Highlighting}
{Shaded}
\end{verbatim}
\end{center}

\textbf{Comparison Table:}

{\def\LTcaptype{none} % do not increment counter
\begin{longtable}[]{@{}
  >{\raggedright\arraybackslash}p{(\linewidth - 4\tabcolsep) * \real{0.2727}}
  >{\raggedright\arraybackslash}p{(\linewidth - 4\tabcolsep) * \real{0.4242}}
  >{\raggedright\arraybackslash}p{(\linewidth - 4\tabcolsep) * \real{0.3030}}@{}}
\toprule\noalign{}
\begin{minipage}[b]{\linewidth}\raggedright
Feature
\end{minipage} & \begin{minipage}[b]{\linewidth}\raggedright
Opto-Isolator
\end{minipage} & \begin{minipage}[b]{\linewidth}\raggedright
Opto-SCR
\end{minipage} \\
\midrule\noalign{}
\endhead
\bottomrule\noalign{}
\endlastfoot
Input & LED & LED \\
Output device & Phototransistor/photodiode & Light-sensitive SCR \\
Isolation & 2-5 kV & 2-5 kV \\
Current handling & Low-medium (100mA) & High (several amps) \\
Applications & Digital signal isolation & Power control, AC switching \\
\end{longtable}
}

\begin{itemize}
\tightlist
\item
  \textbf{Electrical isolation}: Complete electrical separation provides
  noise immunity and safety
\item
  \textbf{Signal transfer}: Light coupling eliminates ground loops and
  voltage level issues
\item
  \textbf{Triggering}: Light replaces gate current for SCR activation in
  Opto-SCR
\end{itemize}

\end{solutionbox}
\begin{mnemonicbox}
``Light Jumps Gaps While Electricity Stays Home''

\end{mnemonicbox}
\subsection*{Question 2(a) [3 marks]}\label{q2a}

\textbf{Draw symbol and give application of 1) UJT 2) SCS 3) MCT.}

\begin{solutionbox}

\textbf{UJT (Unijunction Transistor):}

\begin{verbatim}
    B2
     |
     |
     Z
    /|
   / |
B1{-{-}{-}+{-}{-}{-}E}
\end{verbatim}

\textbf{SCS (Silicon Controlled Switch):}

\begin{verbatim}
      A
      |
      |
  G2{-{-}+}
      |
      |
  G1{-{-}+}
      |
      |
      C
\end{verbatim}

\textbf{MCT (MOS-Controlled Thyristor):}

\begin{verbatim}
      A
      |
     \_|\_
 G{-{-}|\_\_\_|}
     \_|\_
     {\_/}
      |
      |
      C
\end{verbatim}

\textbf{Applications Table:}

{\def\LTcaptype{none} % do not increment counter
\begin{longtable}[]{@{}ll@{}}
\toprule\noalign{}
Device & Applications \\
\midrule\noalign{}
\endhead
\bottomrule\noalign{}
\endlastfoot
UJT & Relaxation oscillators, timing circuits, SCR triggering \\
SCS & Low power switching, level detection, pulse generation \\
MCT & High power switching, motor control, inverters \\
\end{longtable}
}

\end{solutionbox}
\begin{mnemonicbox}
``Unique timing, Controlled switching, Master power''

\end{mnemonicbox}
\subsection*{Question 2(b) [4 marks]}\label{q2b}

\textbf{Explain importance of gate protection for SCR.}

\begin{solutionbox}
Gate protection circuits safeguard SCR against spurious
triggering and voltage spikes.

\textbf{Gate Protection Circuit:}

\begin{verbatim}
        R
    .{-{-}{-}{-}www{-}{-}{-}{-}.}
    |           |
    |     D     |
 {-{-}{-}+{-}{-}{-}{-}{-}|{-}{-}{-}{-}+{-}{-}{-} To SCR Gate}
    |           |
    {{-}{-}{-}{-}{-}{-}{-}{-}{-}{-}{-}}
\end{verbatim}

\textbf{Protection Table:}

{\def\LTcaptype{none} % do not increment counter
\begin{longtable}[]{@{}
  >{\raggedright\arraybackslash}p{(\linewidth - 4\tabcolsep) * \real{0.2432}}
  >{\raggedright\arraybackslash}p{(\linewidth - 4\tabcolsep) * \real{0.5135}}
  >{\raggedright\arraybackslash}p{(\linewidth - 4\tabcolsep) * \real{0.2432}}@{}}
\toprule\noalign{}
\begin{minipage}[b]{\linewidth}\raggedright
Problem
\end{minipage} & \begin{minipage}[b]{\linewidth}\raggedright
Protection Method
\end{minipage} & \begin{minipage}[b]{\linewidth}\raggedright
Purpose
\end{minipage} \\
\midrule\noalign{}
\endhead
\bottomrule\noalign{}
\endlastfoot
Reverse voltage & Diode across gate & Prevents gate-cathode junction
damage \\
Noise & RC filter & Blocks high-frequency transients \\
dV/dt triggering & RC snubber & Controls rate of voltage rise \\
False triggering & Gate resistor & Limits gate current and avoids noise
triggering \\
\end{longtable}
}

\begin{itemize}
\tightlist
\item
  \textbf{Junction protection}: Prevents reverse voltage damage to
  gate-cathode junction
\item
  \textbf{Noise immunity}: Filters out electrical noise that could cause
  unwanted triggering
\end{itemize}

\end{solutionbox}
\begin{mnemonicbox}
``Guard the Gate to Prevent Problems''

\end{mnemonicbox}
\subsection*{Question 2(c) [7 marks]}\label{q2c}

\textbf{List out various methods of triggering SCR and explain any three
of them.}

\begin{solutionbox}
SCR triggering methods convert the device from blocking
to conducting state through gate activation.

\textbf{Triggering Methods Table:}

{\def\LTcaptype{none} % do not increment counter
\begin{longtable}[]{@{}
  >{\raggedright\arraybackslash}p{(\linewidth - 4\tabcolsep) * \real{0.2424}}
  >{\raggedright\arraybackslash}p{(\linewidth - 4\tabcolsep) * \real{0.3333}}
  >{\raggedright\arraybackslash}p{(\linewidth - 4\tabcolsep) * \real{0.4242}}@{}}
\toprule\noalign{}
\begin{minipage}[b]{\linewidth}\raggedright
Method
\end{minipage} & \begin{minipage}[b]{\linewidth}\raggedright
Principle
\end{minipage} & \begin{minipage}[b]{\linewidth}\raggedright
Applications
\end{minipage} \\
\midrule\noalign{}
\endhead
\bottomrule\noalign{}
\endlastfoot
Gate triggering & Direct current to gate & Most common method \\
Thermal triggering & Temperature increase & Thermal protection \\
Light triggering & Photons on junction & Remote activation \\
dV/dt triggering & Fast voltage rise & Often undesirable triggering \\
Voltage triggering & Exceeding breakover voltage & Protection
circuits \\
RF triggering & Radio frequency signals & Wireless control \\
\end{longtable}
}

\textbf{1. Gate Current Triggering:}

\begin{verbatim}
          A
          |
       \_\_\_|\_\_\_
      |   |   |
      |   R   |
G{-{-}{-}{-}{-}+{-}{-}{-}|{-}{-}+}
      |\_\_\_|\_\_\_|
          |
          K
\end{verbatim}

\begin{itemize}
\tightlist
\item
  \textbf{Direct control}: Small gate current initiates large anode
  current flow
\item
  \textbf{Current range}: 10-100mA typically required depending on SCR
  rating
\end{itemize}

\textbf{2. Light Triggering (LASCR):}

\begin{verbatim}
          A
          |
       \_\_\_|\_\_\_
      |  {  |}
      |  {  | {-}{-} Light}
G{-{-}{-}{-}{-}+{-}{-}{-}|{-}{-}+}
      |\_\_\_|\_\_\_|
          |
          K
\end{verbatim}

\begin{itemize}
\tightlist
\item
  \textbf{Optical control}: Photons generate carriers at junction
\item
  \textbf{Isolation}: Provides electrical isolation between control and
  power circuit
\end{itemize}

\textbf{3. dV/dt Triggering:}

\begin{verbatim}
         dV
         {-{-} = high}
         dt
          A
          |
       \_\_\_|\_\_\_
      |   |   |
      |       |
G{-{-}{-}{-}{-}+{-}{-}{-}|{-}{-}+}
      |\_\_\_|\_\_\_|
          |
          K
\end{verbatim}

\begin{itemize}
\tightlist
\item
  \textbf{Rate sensitivity}: Rapid voltage rise causes junction
  capacitance charging
\item
  \textbf{Prevention}: Snubber circuits (RC networks) control voltage
  rise rate
\end{itemize}

\end{solutionbox}
\begin{mnemonicbox}
``Gates, Light, and Voltage Changes Turn SCRs On''

\end{mnemonicbox}
\subsection*{Question 2(a) OR [3
marks]}\label{q2a}

\textbf{Explain working of solid state relay using opto-SCR.}

\begin{solutionbox}
Solid state relays (SSRs) use opto-SCR for contactless
switching with electrical isolation.

\textbf{SSR Block Diagram:}

\begin{center}
\textbf{Mermaid Diagram (Code)}
\begin{verbatim}
{Shaded}
{Highlighting}[]
graph LR
    I[Control Input] {-{-}{} LED[LED]}
    LED {-{-}{} OSCR[Opto{-}SCR]}
    OSCR {-{-}{} ZC[Zero Crossing Circuit]}
    ZC {-{-}{} TS[Thyristor Switch]}
    TS {-{-}{} O[Output Load]}
{Highlighting}
{Shaded}
\end{verbatim}
\end{center}

\textbf{Operation Table:}

{\def\LTcaptype{none} % do not increment counter
\begin{longtable}[]{@{}
  >{\raggedright\arraybackslash}p{(\linewidth - 4\tabcolsep) * \real{0.2692}}
  >{\raggedright\arraybackslash}p{(\linewidth - 4\tabcolsep) * \real{0.3846}}
  >{\raggedright\arraybackslash}p{(\linewidth - 4\tabcolsep) * \real{0.3462}}@{}}
\toprule\noalign{}
\begin{minipage}[b]{\linewidth}\raggedright
Stage
\end{minipage} & \begin{minipage}[b]{\linewidth}\raggedright
Function
\end{minipage} & \begin{minipage}[b]{\linewidth}\raggedright
Benefit
\end{minipage} \\
\midrule\noalign{}
\endhead
\bottomrule\noalign{}
\endlastfoot
Input stage & Drives LED using control signal & Low power control \\
Isolation & Light bridges electrical gap & Safety and noise immunity \\
Triggering & Light activates SCR & No mechanical contacts \\
Switching & Thyristors conduct load current & No arcing or contact
wear \\
\end{longtable}
}

\begin{itemize}
\tightlist
\item
  \textbf{Silent operation}: No mechanical noise during switching
\item
  \textbf{Long life}: No contact degradation as in electromechanical
  relays
\end{itemize}

\end{solutionbox}
\begin{mnemonicbox}
``Light Links Logic to Load''

\end{mnemonicbox}
\subsection*{Question 2(b) OR [4
marks]}\label{q2b}

\textbf{Define snubber circuit and explain importance of snubber
circuit.}

\begin{solutionbox}
A snubber circuit is a protective network that
suppresses voltage and current transients in switching devices.

\textbf{Basic RC Snubber:}

\begin{verbatim}
          A
          |
     C    |
    |{-{-}{-}{-}{-}|}
    |     |
    |     Z SCR
    |     Z
    |     |
   {-{-}{-}    |}
   {-{-}{-} R  |}
    |     |
    |{-{-}{-}{-}{-}|}
          |
          K
\end{verbatim}

\textbf{Importance Table:}

{\def\LTcaptype{none} % do not increment counter
\begin{longtable}[]{@{}
  >{\raggedright\arraybackslash}p{(\linewidth - 4\tabcolsep) * \real{0.2857}}
  >{\raggedright\arraybackslash}p{(\linewidth - 4\tabcolsep) * \real{0.2571}}
  >{\raggedright\arraybackslash}p{(\linewidth - 4\tabcolsep) * \real{0.4571}}@{}}
\toprule\noalign{}
\begin{minipage}[b]{\linewidth}\raggedright
Function
\end{minipage} & \begin{minipage}[b]{\linewidth}\raggedright
Benefit
\end{minipage} & \begin{minipage}[b]{\linewidth}\raggedright
Implementation
\end{minipage} \\
\midrule\noalign{}
\endhead
\bottomrule\noalign{}
\endlastfoot
dV/dt suppression & Prevents false triggering & RC circuit across SCR \\
Voltage spike reduction & Protects from overvoltage & Capacitor absorbs
energy \\
Oscillation damping & Reduces EMI & Resistor provides damping \\
Turn-off assistance & Improves commutation & Diverts current during
turn-off \\
\end{longtable}
}

\begin{itemize}
\tightlist
\item
  \textbf{Circuit protection}: Extends thyristor life by limiting stress
  on the device
\item
  \textbf{Noise reduction}: Minimizes electromagnetic interference in
  surrounding circuits
\end{itemize}

\end{solutionbox}
\begin{mnemonicbox}
``Suppress Noise Upsetting Balanced Behaviors Easily
Restored''

\end{mnemonicbox}
\subsection*{Question 2(c) OR [7
marks]}\label{q2c}

\textbf{List various commutation methods of SCR and explain any two of
them}

\begin{solutionbox}
Commutation is the process of turning OFF an SCR by
reducing its anode current below holding value.

\textbf{Commutation Methods Table:}

{\def\LTcaptype{none} % do not increment counter
\begin{longtable}[]{@{}lll@{}}
\toprule\noalign{}
Method & Principle & Applications \\
\midrule\noalign{}
\endhead
\bottomrule\noalign{}
\endlastfoot
Natural & AC zero crossing & AC power control \\
Forced & External circuit & DC applications \\
Class A & LC resonance & Inverters \\
Class B & Auxiliary SCR & DC choppers \\
Class C & LC with load & Variable frequency \\
Class D & Auxiliary source & Motor control \\
Class E & External pulse & Electronic circuits \\
\end{longtable}
}

\textbf{1. Natural Commutation:}

\begin{verbatim}
        AC
        {}
        |
        Z SCR
        Z
        |
        R Load
        |
       {-{-}{-}}
       GND
\end{verbatim}

\begin{itemize}
\tightlist
\item
  \textbf{Zero crossing}: SCR turns off when AC crosses zero and anode
  current falls below holding
\item
  \textbf{Simplicity}: No additional components required for commutation
\item
  \textbf{Limitation}: Works only in AC circuits at fixed frequency
\end{itemize}

\textbf{2. Forced Commutation (Class B):}

\begin{verbatim}
    +Vdc
      |
      |    C
      Z    |
 SCR1 Z    |
      |{-{-}{-}{-}+{-}{-}{-}{-},}
      |    |    |
      R    Z SCR2
 Load |    Z    |
      |    |    |
     {-{-}{-}  {-}{-}{-}  {-}{-}{-}}
     GND  GND  GND
\end{verbatim}

\begin{itemize}
\tightlist
\item
  \textbf{Auxiliary SCR}: Second SCR (SCR2) discharges capacitor to
  reverse bias main SCR
\item
  \textbf{Timing control}: Precise control over when SCR turns off
\item
  \textbf{Application}: Used in DC circuits where natural commutation
  isn't possible
\end{itemize}

\end{solutionbox}
\begin{mnemonicbox}
``Nature Follows Current, Forced Creates Current
Collapse''

\end{mnemonicbox}
\subsection*{Question 3(a) [3 marks]}\label{q3a}

\textbf{Explain advantages of polyphase rectifier over single phase
rectifier.}

\begin{solutionbox}
Polyphase rectifiers offer significant improvements
over single-phase designs in power applications.

\textbf{Advantages Table:}

{\def\LTcaptype{none} % do not increment counter
\begin{longtable}[]{@{}lll@{}}
\toprule\noalign{}
Parameter & Single Phase & Polyphase \\
\midrule\noalign{}
\endhead
\bottomrule\noalign{}
\endlastfoot
Ripple factor & Higher (0.482 for FW) & Lower (0.042 for 3-phase) \\
Form factor & Higher & Lower \\
Efficiency & Lower & Higher (better transformer utilization) \\
Power rating & Limited & Higher power handling \\
Harmonic content & More & Less (smoother DC) \\
\end{longtable}
}

\begin{itemize}
\tightlist
\item
  \textbf{Output smoothness}: Significantly less ripple requiring
  smaller filtering components
\item
  \textbf{Transformer utilization}: Better utilization factor (0.955 vs
  0.812) reduces transformer size
\end{itemize}

\end{solutionbox}
\begin{mnemonicbox}
``More Phases Mean Smoother Power''

\end{mnemonicbox}
\subsection*{Question 3(b) [4 marks]}\label{q3b}

\textbf{Write short note on UPS.}

\begin{solutionbox}
UPS (Uninterruptible Power Supply) provides continuous
power during main supply failure.

\textbf{UPS Block Diagram:}

\begin{center}
\textbf{Mermaid Diagram (Code)}
\begin{verbatim}
{Shaded}
{Highlighting}[]
graph LR
    A[AC Input] {-{-}{} R[Rectifier]}
    R {-{-}{} C[DC Bus]}
    C {-{-}{} I[Inverter]}
    I {-{-}{} O[AC Output]}
    C {{-}{-}{} B[Battery Bank]}
    S[Static Switch] {-{-}{} O}
    A {-{-}{} S}
{Highlighting}
{Shaded}
\end{verbatim}
\end{center}

\textbf{Types of UPS Table:}

{\def\LTcaptype{none} % do not increment counter
\begin{longtable}[]{@{}
  >{\raggedright\arraybackslash}p{(\linewidth - 4\tabcolsep) * \real{0.1935}}
  >{\raggedright\arraybackslash}p{(\linewidth - 4\tabcolsep) * \real{0.3548}}
  >{\raggedright\arraybackslash}p{(\linewidth - 4\tabcolsep) * \real{0.4516}}@{}}
\toprule\noalign{}
\begin{minipage}[b]{\linewidth}\raggedright
Type
\end{minipage} & \begin{minipage}[b]{\linewidth}\raggedright
Operation
\end{minipage} & \begin{minipage}[b]{\linewidth}\raggedright
Applications
\end{minipage} \\
\midrule\noalign{}
\endhead
\bottomrule\noalign{}
\endlastfoot
Online & Always through battery/inverter & Critical systems, medical \\
Offline & Switches to battery on failure & Personal computers, small
offices \\
Line-interactive & Voltage regulation + backup & Servers, network
equipment \\
\end{longtable}
}

\begin{itemize}
\tightlist
\item
  \textbf{Backup time}: Typically 5-30 minutes depending on battery
  capacity
\item
  \textbf{Protection}: Surge protection, voltage regulation, and
  frequency stabilization
\end{itemize}

\end{solutionbox}
\begin{mnemonicbox}
``Power Constantly Protected Under Switch''

\end{mnemonicbox}
\subsection*{Question 3(c) [7 marks]}\label{q3c}

\textbf{Give function of Inverter and explain basic principle of
Inverter also explain series inverter with neat diagram and waveform.}

\begin{solutionbox}
Inverters convert DC power to AC power by switching DC
through a transformer or directly to create alternating waveforms.

\textbf{Function Table:}

{\def\LTcaptype{none} % do not increment counter
\begin{longtable}[]{@{}ll@{}}
\toprule\noalign{}
Function & Description \\
\midrule\noalign{}
\endhead
\bottomrule\noalign{}
\endlastfoot
DC to AC conversion & Transforms steady DC to alternating AC \\
Frequency control & Generates variable frequency output \\
Voltage regulation & Maintains stable output despite load variations \\
Wave shaping & Produces sine, square, or modified sine waves \\
\end{longtable}
}

\textbf{Basic Principle Diagram:}

\begin{center}
\textbf{Mermaid Diagram (Code)}
\begin{verbatim}
{Shaded}
{Highlighting}[]
graph LR
    D[DC Source] {-{-}{} S[Switching Circuit]}
    S {-{-}{} T[Transformer/Filter]}
    T {-{-}{} A[AC Output]}
    C[Control Circuit] {-{-}{} S}
{Highlighting}
{Shaded}
\end{verbatim}
\end{center}

\textbf{Series Inverter Circuit:}

\begin{verbatim}
    +Vdc
      |
      |
    \_\_|\_\_
   |     |
   C     L
   |     |
   |     |
   |     |
   |    \_|\_
   |    { /}
   |    SCR
   |    \_|\_
   |     |
   |     |
   |     |
  {-{-}{-}   {-}{-}{-}}
  GND   GND
\end{verbatim}

\textbf{Waveforms:}

\begin{verbatim}
Voltage
   \^{}
   |     \_\_\_\_
   |    /    {}
   |\_\_\_/      {\_\_\_\_}
   |
   |           \_\_\_\_
   |          /    {}
   |\_\_\_\_\_\_\_\_\_/      {\_\_\_\_}
   +{-{-}{-}{-}{-}{-}{-}{-}{-}{-}{-}{-}{-}{-}{-}{-}{-}{-}{-}{-}{-}{-} Time}
   
Current
   \^{}
   |    /{}
   |   /  {}
   |\_\_/    {\_\_/\_\_}
   |           {  }
   |            {  }
   |             {  }
   |              {/}
   +{-{-}{-}{-}{-}{-}{-}{-}{-}{-}{-}{-}{-}{-}{-}{-}{-}{-}{-}{-}{-}{-} Time}
\end{verbatim}

\begin{itemize}
\tightlist
\item
  \textbf{Oscillation}: Series LC circuit creates resonant oscillation
  when SCR triggers
\item
  \textbf{Commutation}: SCR turns off naturally when current reverses
  through resonance
\item
  \textbf{Frequency}: Determined by LC values: f = 1/(2π\sqrtLC)
\end{itemize}

\end{solutionbox}
\begin{mnemonicbox}
``Direct Current Switches To Alternating Current
Through Resonant Circuit''

\end{mnemonicbox}
\subsection*{Question 3(a) OR [3
marks]}\label{q3a}

\textbf{Explain basic principle of chopper.}

\begin{solutionbox}
A chopper is a DC-to-DC converter that switches DC
input on/off to produce controllable average DC output.

\textbf{Basic Chopper Circuit:}

\begin{verbatim}
    +Vdc
      |
      |
     \_|\_
     { /}
      S Switch
     \_|\_
      |
      |
      R Load
      |
      |
     {-{-}{-}}
     GND
\end{verbatim}

\textbf{Principle Table:}

{\def\LTcaptype{none} % do not increment counter
\begin{longtable}[]{@{}
  >{\raggedright\arraybackslash}p{(\linewidth - 4\tabcolsep) * \real{0.3667}}
  >{\raggedright\arraybackslash}p{(\linewidth - 4\tabcolsep) * \real{0.3333}}
  >{\raggedright\arraybackslash}p{(\linewidth - 4\tabcolsep) * \real{0.3000}}@{}}
\toprule\noalign{}
\begin{minipage}[b]{\linewidth}\raggedright
Parameter
\end{minipage} & \begin{minipage}[b]{\linewidth}\raggedright
Relation
\end{minipage} & \begin{minipage}[b]{\linewidth}\raggedright
Control
\end{minipage} \\
\midrule\noalign{}
\endhead
\bottomrule\noalign{}
\endlastfoot
Output voltage & Vo = Vdc \times (Ton/T) & Duty cycle adjustment \\
Duty cycle & k = Ton/T & Controls output voltage \\
Frequency & f = 1/T & Affects ripple \\
Voltage regulation & Varies with load & Feedback control adjusts duty
cycle \\
\end{longtable}
}

\begin{itemize}
\tightlist
\item
  \textbf{Switching action}: Rapidly turns ON/OFF to chop DC input
\item
  \textbf{Pulse width modulation}: Controls voltage by varying ON-time
  ratio
\end{itemize}

\end{solutionbox}
\begin{mnemonicbox}
``Chopping Creates Controllable DC''

\end{mnemonicbox}
\subsection*{Question 3(b) OR [4
marks]}\label{q3b}

\textbf{Draw the block diagram of SMPS and explain function of each
block.}

\begin{solutionbox}
SMPS (Switched Mode Power Supply) converts input power
to regulated output using high-frequency switching.

\textbf{SMPS Block Diagram:}

\begin{center}
\textbf{Mermaid Diagram (Code)}
\begin{verbatim}
{Shaded}
{Highlighting}[]
graph LR
    A[AC Input] {-{-}{} F[EMI Filter]}
    F {-{-}{} R[Rectifier \& Filter]}
    R {-{-}{} S[Switching Circuit]}
    S {-{-}{} T[Transformer]}
    T {-{-}{} O[Output Rectifier]}
    O {-{-}{} OF[Output Filter]}
    OF {-{-}{} OUT[DC Output]}
    FB[Feedback Control] {-{-}{} S}
    OUT {-{-}{} FB}
{Highlighting}
{Shaded}
\end{verbatim}
\end{center}

\textbf{Blocks Function Table:}

{\def\LTcaptype{none} % do not increment counter
\begin{longtable}[]{@{}ll@{}}
\toprule\noalign{}
Block & Function \\
\midrule\noalign{}
\endhead
\bottomrule\noalign{}
\endlastfoot
EMI Filter & Suppresses noise from entering/leaving SMPS \\
Rectifier \& Filter & Converts AC to unregulated DC \\
Switching Circuit & Chops DC at high frequency (20-200kHz) \\
Transformer & Provides isolation and voltage transformation \\
Output Rectifier & Converts high-frequency AC back to DC \\
Output Filter & Smooths DC output and removes ripple \\
Feedback Control & Regulates output by adjusting duty cycle \\
\end{longtable}
}

\begin{itemize}
\tightlist
\item
  \textbf{High efficiency}: 70-90\% vs 30-60\% for linear supplies
\item
  \textbf{Small size}: High frequency allows smaller transformer and
  components
\end{itemize}

\end{solutionbox}
\begin{mnemonicbox}
``Filter, Rectify, Switch Through Transformer,
Rectify, Filter''

\end{mnemonicbox}
\subsection*{Question 3(c) OR [7
marks]}\label{q3c}

\textbf{Explain 1 phase half wave rectifier with waveform also explain 3
phase full wave rectifier with waveform.}

\begin{solutionbox}
Rectifiers convert AC to DC by allowing current flow in
one direction while blocking reverse flow.

\textbf{1-Phase Half Wave Rectifier:}

\begin{verbatim}
      AC
      {}
      |
     \_|\_
     { /}
      D
     \_|\_
      |
      R Load
      |
     {-{-}{-}}
     GND
\end{verbatim}

\textbf{1-Phase Half Wave Waveforms:}

\begin{verbatim}
Input AC
   \^{}
   |    /{         /}
   |   /  {       /  }
   |\_\_/    {\_\_\_\_\_/    \_\_\_\_}
   |
   |        /{         /}
   |       /  {       /  }
   |{     /         /    }
   +{-{-}{-}{-}{-}{-}{-}{-}{-}{-}{-}{-}{-}{-}{-}{-}{-}{-}{-}{-}{-}{-}{-} Time}
   
Output DC
   \^{}
   |    /{         /}
   |   /  {       /  }
   |\_\_/    {\_\_\_\_\_/    \_\_\_\_}
   |
   |                      
   |                      
   |
   +{-{-}{-}{-}{-}{-}{-}{-}{-}{-}{-}{-}{-}{-}{-}{-}{-}{-}{-}{-}{-}{-}{-} Time}
\end{verbatim}

\textbf{3-Phase Full Wave Rectifier:}

\begin{verbatim}
    A o{-{-}{-}D1{-}{-}{-}.}
               |
               |
    B o{-{-}{-}D3{-}{-}{-}+{-}{-}{-}o +Vdc}
               |
               |
    C o{-{-}{-}D5{-}{-}{-}.}
               
    A o{-{-}{-}D2{-}{-}{-}.}
               |
               |
    B o{-{-}{-}D4{-}{-}{-}+{-}{-}{-}o {-}Vdc}
               |
               |
    C o{-{-}{-}D6{-}{-}{-}.}
\end{verbatim}

\textbf{3-Phase Full Wave Waveforms:}

\begin{verbatim}
3{-Phase Input}
   \^{}
   |    /{    /    /    /}
   |   /  {  /    /    /  }
   |\_\_/\_\_\_\_{/\_\_\_\_/\_\_\_\_/\_\_\_\_}
   |  { / / / / / / /}
   |   {/  /  /  /  /  /}
   |
   +{-{-}{-}{-}{-}{-}{-}{-}{-}{-}{-}{-}{-}{-}{-}{-}{-}{-}{-}{-}{-}{-}{-} Time}
   
Rectified Output
   \^{}
   |   nnnnnnnnnnnnnnnnnnnnnn
   |  n  n  n  n  n  n  n  n
   |\_n\_\_\_\_n\_\_\_\_n\_\_\_\_n\_\_\_\_n\_\_\_
   |
   |
   |
   |
   +{-{-}{-}{-}{-}{-}{-}{-}{-}{-}{-}{-}{-}{-}{-}{-}{-}{-}{-}{-}{-}{-}{-} Time}
\end{verbatim}

\textbf{Comparison Table:}

{\def\LTcaptype{none} % do not increment counter
\begin{longtable}[]{@{}lll@{}}
\toprule\noalign{}
Parameter & 1-Phase Half Wave & 3-Phase Full Wave \\
\midrule\noalign{}
\endhead
\bottomrule\noalign{}
\endlastfoot
Ripple factor & 1.21 & 0.042 \\
Rectification efficiency & 40.6\% & 95.5\% \\
TUF & 0.287 & 0.955 \\
Peak inverse voltage & Vm & 2.09Vm \\
Form factor & 1.57 & 1.0007 \\
\end{longtable}
}

\begin{itemize}
\tightlist
\item
  \textbf{1-Phase Half Wave}: Simplest design but with high ripple and
  poor efficiency
\item
  \textbf{3-Phase Full Wave}: Much smoother output with 6 pulses per
  cycle
\end{itemize}

\end{solutionbox}
\begin{mnemonicbox}
``Half Passes Only Peaks, Three Phases Fill Valleys''

\end{mnemonicbox}
\subsection*{Question 4(a) [3 marks]}\label{q4a}

\textbf{Describe working of solar photovoltaic based power generation
with block diagram.}

\begin{solutionbox}
Solar PV power generation converts sunlight directly
into electricity through photovoltaic effect.

\textbf{Solar PV System Block Diagram:}

\begin{center}
\textbf{Mermaid Diagram (Code)}
\begin{verbatim}
{Shaded}
{Highlighting}[]
graph LR
    S[Solar Panel Array] {-{-}{} C[Charge Controller]}
    C {-{-}{} B[Battery Bank]}
    B {-{-}{} I[Inverter]}
    I {-{-}{} L[AC Loads]}
    C {-{-}{} D[DC Loads]}
{Highlighting}
{Shaded}
\end{verbatim}
\end{center}

\textbf{Component Table:}

{\def\LTcaptype{none} % do not increment counter
\begin{longtable}[]{@{}ll@{}}
\toprule\noalign{}
Component & Function \\
\midrule\noalign{}
\endhead
\bottomrule\noalign{}
\endlastfoot
Solar panels & Convert sunlight to DC electricity \\
Charge controller & Regulates charging, prevents overcharge \\
Battery bank & Stores energy for later use \\
Inverter & Converts DC to AC for household appliances \\
Distribution panel & Routes electricity to loads \\
\end{longtable}
}

\begin{itemize}
\tightlist
\item
  \textbf{Energy conversion}: Photons excite electrons in semiconductor
  material creating current
\item
  \textbf{Scalability}: System size can be adjusted based on power
  requirements
\end{itemize}

\end{solutionbox}
\begin{mnemonicbox}
``Sunlight Produces Voltage, Batteries Invert Loads''

\end{mnemonicbox}
\subsection*{Question 4(b) [4 marks]}\label{q4b}

\textbf{Explain use of SCR as static switch.}

\begin{solutionbox}
SCR functions as a solid-state switch with no moving
parts for reliable and fast switching.

\textbf{SCR Static Switch Circuit:}

\begin{verbatim}
    +Vdc
      |
      |
     \_|\_
     { /}
     SCR
     \_|\_
      |
     {-{-}{-}  Trigger}
      |   Circuit
      R   |
 Load |   |
      |{-{-}{-}|}
      |
     {-{-}{-}}
     GND
\end{verbatim}

\textbf{Applications Table:}

{\def\LTcaptype{none} % do not increment counter
\begin{longtable}[]{@{}lll@{}}
\toprule\noalign{}
Application & Advantage & Implementation \\
\midrule\noalign{}
\endhead
\bottomrule\noalign{}
\endlastfoot
Power control & Precise control, no arcing & Phase angle control \\
Motor starting & Smooth acceleration & Gradual voltage increase \\
Circuit protection & Fast response & Current sensing trigger \\
Heating control & Energy efficient & Zero-crossing switching \\
\end{longtable}
}

\begin{itemize}
\tightlist
\item
  \textbf{Latching action}: Once triggered, continues to conduct until
  current falls below holding value
\item
  \textbf{High reliability}: No mechanical wear due to absence of moving
  parts
\end{itemize}

\end{solutionbox}
\begin{mnemonicbox}
``Semiconductor Switching Controls Running Loads''

\end{mnemonicbox}
\subsection*{Question 4(c) [7 marks]}\label{q4c}

\textbf{Describe the working principle of Induction heating and
dielectric heating also give comparison of Induction heating and
dielectric heating.}

\begin{solutionbox}
Both heating methods use electromagnetic principles to
generate heat without direct contact.

\textbf{Induction Heating Diagram:}

\begin{center}
\textbf{Mermaid Diagram (Code)}
\begin{verbatim}
{Shaded}
{Highlighting}[]
graph LR
    A[AC Power] {-{-}{} B[High Frequency Generator]}
    B {-{-}{} C[Work Coil]}
    C {-{-}{} D[Magnetic Field]}
    D {-{-}{} E[Eddy Currents in Workpiece]}
    E {-{-}{} F[Heat Generation]}
{Highlighting}
{Shaded}
\end{verbatim}
\end{center}

\textbf{Dielectric Heating Diagram:}

\begin{center}
\textbf{Mermaid Diagram (Code)}
\begin{verbatim}
{Shaded}
{Highlighting}[]
graph LR
    A[RF Generator] {-{-}{} B[Applicator Plates]}
    B {-{-}{} C[Electric Field]}
    C {-{-}{} D[Molecular Friction in Material]}
    D {-{-}{} E[Heat Generation]}
{Highlighting}
{Shaded}
\end{verbatim}
\end{center}

\textbf{Comparison Table:}

{\def\LTcaptype{none} % do not increment counter
\begin{longtable}[]{@{}
  >{\raggedright\arraybackslash}p{(\linewidth - 4\tabcolsep) * \real{0.2245}}
  >{\raggedright\arraybackslash}p{(\linewidth - 4\tabcolsep) * \real{0.3878}}
  >{\raggedright\arraybackslash}p{(\linewidth - 4\tabcolsep) * \real{0.3878}}@{}}
\toprule\noalign{}
\begin{minipage}[b]{\linewidth}\raggedright
Parameter
\end{minipage} & \begin{minipage}[b]{\linewidth}\raggedright
Induction Heating
\end{minipage} & \begin{minipage}[b]{\linewidth}\raggedright
Dielectric Heating
\end{minipage} \\
\midrule\noalign{}
\endhead
\bottomrule\noalign{}
\endlastfoot
Principle & Eddy currents and hysteresis & Molecular friction from
oscillating field \\
Materials & Conductive metals & Non-conductive materials (plastics,
wood) \\
Frequency & 1-100 kHz & 10-100 MHz \\
Penetration & Surface and shallow depth & Uniform through material \\
Efficiency & 80-90\% & 50-70\% \\
Applications & Metal hardening, melting, forging & Plastic welding, food
processing, drying \\
\end{longtable}
}

\begin{itemize}
\tightlist
\item
  \textbf{Induction heating}: Works through electromagnetic induction
  creating eddy currents in conductive materials
\item
  \textbf{Dielectric heating}: Causes rapid oscillation of polar
  molecules creating internal friction and heat
\end{itemize}

\end{solutionbox}
\begin{mnemonicbox}
``Induction Makes Metals Hot, Dielectrics Heat
Non-Metals''

\end{mnemonicbox}
\subsection*{Question 4(a) OR [3
marks]}\label{q4a}

\textbf{Draw and explain the circuit diagram of photo electric relay
using photo diode.}

\begin{solutionbox}
Photo-electric relay uses light detection to control
switching operations automatically.

\textbf{Circuit Diagram:}

\begin{verbatim}
    +Vcc
      |
      R1
      |
      |{-{-}{-}{-}{-}{-}{-}+}
      |       |
     {-{-}{-}      |}
     / {      |}
    Photo    \_|\_
    Diode    | |\_
     {-{-}{-}     |/  |}
      |      |   |
      |      |{  |}
      |      |   |
      +{-{-}{-}{-}{-}{-}+   |}
      |          |
      R2         |
      |          |
     {-{-}{-}         |       +Vcc}
     GND         |        |
                 |       \_|\_
                 +{-{-}{-}{-}{-}{-}{-}|  |}
                         | /
                         |{}
                         |/
                         |
                         Z
                         Z Load
                         |
                        {-{-}{-}}
                        GND
\end{verbatim}

\textbf{Operation Table:}

{\def\LTcaptype{none} % do not increment counter
\begin{longtable}[]{@{}llll@{}}
\toprule\noalign{}
Light Condition & Photodiode State & Transistor State & Relay Action \\
\midrule\noalign{}
\endhead
\bottomrule\noalign{}
\endlastfoot
Dark & High resistance & OFF & De-energized \\
Light & Low resistance (conducts) & ON & Energized \\
\end{longtable}
}

\begin{itemize}
\tightlist
\item
  \textbf{Light detection}: Photodiode conducts when illuminated,
  changing bias on transistor
\item
  \textbf{Switching}: Transistor amplifies small photodiode current to
  drive relay coil
\end{itemize}

\end{solutionbox}
\begin{mnemonicbox}
``Light Drives Diode, Diode Drives Transistor,
Transistor Drives Relay''

\end{mnemonicbox}
\subsection*{Question 4(b) OR [4
marks]}\label{q4b}

\textbf{Draw the circuit diagram of AC power control using DIAC-TRIAC
and explain it.}

\begin{solutionbox}
DIAC-TRIAC circuit enables smooth control of AC power
through phase angle adjustment.

\textbf{Circuit Diagram:}

\begin{verbatim}
    AC      R1      DIAC
    {       www     \_|\_}
    |       |       / {}
    |{-{-}{-}{-}{-}{-}{-}|{-}{-}{-}{-}{-}{-}{-}|{-}{-}{-}{-}{-}{-}{-}{-}.}
    |       |       {\_/      |}
    |       |        |       |
    |       |        |       |
    |       C        |      \_|\_
    |       |        |     |   |
    |       |        |     |   |
    |       |        {{-}{-}{-}{-}{-}|GT |}
    |       |              |   |
    |       |              |\_\_\_|
    |       |                | TRIAC
    |       |                |
    Z       |                |
    Z Load  |                |
    |       |                |
    |{-{-}{-}{-}{-}{-}{-}+{-}{-}{-}{-}{-}{-}{-}{-}{-}{-}{-}{-}{-}{-}{-}{-}}
    |
   {-{-}{-}}
   GND
\end{verbatim}

\textbf{Operation Table:}

{\def\LTcaptype{none} % do not increment counter
\begin{longtable}[]{@{}ll@{}}
\toprule\noalign{}
Component & Function \\
\midrule\noalign{}
\endhead
\bottomrule\noalign{}
\endlastfoot
R1-C & Variable time constant for phase delay \\
DIAC & Triggers TRIAC when capacitor voltage reaches breakover \\
TRIAC & Controls load current based on triggering point \\
Load & Receives partial AC waveform based on phase control \\
\end{longtable}
}

\begin{itemize}
\tightlist
\item
  \textbf{Phase control}: RC network creates delay in triggering point
  within AC cycle
\item
  \textbf{Bidirectional operation}: Works on both halves of AC cycle
\end{itemize}

\end{solutionbox}
\begin{mnemonicbox}
``Delay Initiates At Capacitor, Triggers Reliable
Independent AC Control''

\end{mnemonicbox}
\subsection*{Question 4(c) OR [7
marks]}\label{q4c}

\textbf{Explain IC555 three stage sequential timer working with
waveform.}

\begin{solutionbox}
A three-stage sequential timer uses multiple 555 ICs to
generate timed sequences for process control.

\textbf{Circuit Diagram:}

\begin{center}
\textbf{Mermaid Diagram (Code)}
\begin{verbatim}
{Shaded}
{Highlighting}[]
graph LR
    T[Trigger] {-{-}{} IC1[555 Timer 1]}
    IC1 {-{-}{} O1[Output 1]}
    IC1 {-{-}{} D1[Delay]}
    D1 {-{-}{} IC2[555 Timer 2]}
    IC2 {-{-}{} O2[Output 2]}
    IC2 {-{-}{} D2[Delay]}
    D2 {-{-}{} IC3[555 Timer 3]}
    IC3 {-{-}{} O3[Output 3]}
    IC3 {-{-}{} R[Reset]}
    R {-.{-}{} IC1}
{Highlighting}
{Shaded}
\end{verbatim}
\end{center}

\textbf{Waveform:}

\begin{verbatim}
Trigger
   \_
\_\_|  |\_\_\_\_\_\_\_\_\_\_\_\_\_\_\_\_\_\_\_\_\_\_\_\_\_\_\_
   
Output 1
    \_\_\_\_\_\_\_\_\_\_\_\_
\_\_\_|            |\_\_\_\_\_\_\_\_\_\_\_\_\_\_\_\_
             
Output 2
              \_\_\_\_\_\_\_\_\_\_\_\_
\_\_\_\_\_\_\_\_\_\_\_\_\_|            |\_\_\_\_\_\_
                       
Output 3
                        \_\_\_\_\_\_\_\_
\_\_\_\_\_\_\_\_\_\_\_\_\_\_\_\_\_\_\_\_\_\_\_|        |
   
   {{-}T1{-}|{-}{-}T2{-}{-}|{-}{-}T3{-}{-}|{-}T4{-}}
\end{verbatim}

\textbf{Sequential Operation Table:}

{\def\LTcaptype{none} % do not increment counter
\begin{longtable}[]{@{}llll@{}}
\toprule\noalign{}
Stage & Action & Duration & Next Stage Trigger \\
\midrule\noalign{}
\endhead
\bottomrule\noalign{}
\endlastfoot
Initial & All outputs LOW & - & External trigger \\
Stage 1 & Output 1 HIGH & T1 (R1\timesC1) & Output 1 falling edge \\
Stage 2 & Output 2 HIGH & T2 (R2\timesC2) & Output 2 falling edge \\
Stage 3 & Output 3 HIGH & T3 (R3\timesC3) & Output 3 falling edge \\
Reset & All outputs LOW & T4 (reset time) & New external trigger \\
\end{longtable}
}

\begin{itemize}
\tightlist
\item
  \textbf{Cascading connection}: Output of first timer triggers second,
  and so on
\item
  \textbf{Timing control}: Each stage duration independently adjustable
  with RC values
\item
  \textbf{Applications}: Industrial sequencing, process control,
  automated systems
\end{itemize}

\end{solutionbox}
\begin{mnemonicbox}
``First Stage Finishes, Second Starts, Third
Succeeds''

\end{mnemonicbox}
\subsection*{Question 5(a) [3 marks]}\label{q5a}

\textbf{Draw and explain solid state control of DC shunt motor.}

\begin{solutionbox}
Solid-state DC motor control uses SCRs to regulate
voltage applied to the motor.

\textbf{Circuit Diagram:}

\begin{verbatim}
    AC      Bridge      SCR
    {       Rect      \_\_\_|\_\_\_}
    |       \_\_\_\_     |       |
    |{-{-}{-}{-}{-}{-}|\_\_\_\_|{-}{-}{-}{-}|       |{-}{-}{-}{-}.}
    |                |\_\_\_\_\_\_\_|    |
    |                    |        |
    |                   \_|\_       |
    |                   { /       |}
    |                   Gate      |
    |                 Circuit     |
    |                    |        |
    |                    |        |
    |       Field        |        |
    |       Winding      |     \_\_\_|\_\_\_
    |      \_\_\_\_\_\_        |    |       |
    |     |      |       |    | DC    |
    |{-{-}{-}{-}{-}|\_\_\_\_\_\_|{-}{-}{-}{-}{-}{-}{-}+{-}{-}{-}{-}| Motor |}
    |                         |\_\_\_\_\_\_\_|
    |                            |
   {-{-}{-}                          {-}{-}{-}}
   GND                          GND
\end{verbatim}

\textbf{Control Method Table:}

{\def\LTcaptype{none} % do not increment counter
\begin{longtable}[]{@{}lll@{}}
\toprule\noalign{}
Method & Operation & Advantage \\
\midrule\noalign{}
\endhead
\bottomrule\noalign{}
\endlastfoot
Phase control & Varies SCR firing angle & Smooth speed control \\
Chopper control & Pulse width modulation & High efficiency \\
Closed-loop & Feedback from tachometer & Precise speed regulation \\
\end{longtable}
}

\begin{itemize}
\tightlist
\item
  \textbf{Speed regulation}: Controls armature voltage to vary motor
  speed
\item
  \textbf{Torque control}: Maintains high starting torque with current
  limiting
\end{itemize}

\end{solutionbox}
\begin{mnemonicbox}
``SCR Controls Current Delivering Motor Power''

\end{mnemonicbox}
\subsection*{Question 5(b) [4 marks]}\label{q5b}

\textbf{Explain working principle of stepper motor.}

\begin{solutionbox}
Stepper motors convert digital pulses into precise
mechanical rotation through electromagnetic principles.

\textbf{Stepper Motor Structure:}

\begin{center}
\textbf{Mermaid Diagram (Code)}
\begin{verbatim}
{Shaded}
{Highlighting}[]
graph LR
    C[Controller] {-{-}{} D[Driver]}
    D {-{-}{} P[Phase Windings]}
    P {-{-}{} R[Rotor Movement]}
{Highlighting}
{Shaded}
\end{verbatim}
\end{center}

\textbf{Operation Principle Table:}

{\def\LTcaptype{none} % do not increment counter
\begin{longtable}[]{@{}lll@{}}
\toprule\noalign{}
Step Type & Rotation Angle & Control Method \\
\midrule\noalign{}
\endhead
\bottomrule\noalign{}
\endlastfoot
Full step & Typically 1.8^\circ or 0.9^\circ & One phase at a time \\
Half step & Half of full step & Two phases alternating \\
Micro-step & Fraction of full step & PWM current control \\
Wave drive & Full step angle & One phase energized \\
\end{longtable}
}

\begin{itemize}
\tightlist
\item
  \textbf{Digital positioning}: Each pulse rotates motor by precise
  angle
\item
  \textbf{Holding torque}: Maintains position when energized without
  rotation
\end{itemize}

\end{solutionbox}
\begin{mnemonicbox}
``Pulses Produce Precise Positional Steps''

\end{mnemonicbox}
\subsection*{Question 5(c) [7 marks]}\label{q5c}

\textbf{Draw the block diagram of PLC and explain function of each
block.}

\begin{solutionbox}
Programmable Logic Controller (PLC) is an industrial
digital computer for automation control.

\textbf{PLC Block Diagram:}

\begin{center}
\textbf{Mermaid Diagram (Code)}
\begin{verbatim}
{Shaded}
{Highlighting}[]
graph TD
    P[Power Supply] {-{-}{} CPU[Central Processing Unit]}
    CPU {-{-}{} M[Memory]}
    CPU {-{-}{} I[Input Module]}
    CPU {-{-}{} O[Output Module]}
    I {-{-}{} S[Input Sensors/Switches]}
    O {-{-}{} A[Actuators/Motors]}
    CPU {-{-}{} C[Communication Module]}
    CPU {-{-}{} P[Programming Device]}
{Highlighting}
{Shaded}
\end{verbatim}
\end{center}

\textbf{PLC Components Table:}

{\def\LTcaptype{none} % do not increment counter
\begin{longtable}[]{@{}ll@{}}
\toprule\noalign{}
Component & Function \\
\midrule\noalign{}
\endhead
\bottomrule\noalign{}
\endlastfoot
Power Supply & Converts main power to DC required by PLC \\
CPU & Executes program and makes decisions based on I/O \\
Memory & Stores program and data (ROM, RAM, EEPROM) \\
Input Module & Interfaces with sensors, switches, encoders \\
Output Module & Controls actuators, motors, valves, indicators \\
Communication Module & Connects to other PLCs, computers, networks \\
Programming Device & Used to write, edit, monitor PLC programs \\
\end{longtable}
}

\begin{itemize}
\tightlist
\item
  \textbf{Scan cycle}: Reads inputs, executes program, updates outputs
  continuously
\item
  \textbf{Programming languages}: Ladder logic, function block,
  structured text, etc.
\item
  \textbf{Advantages}: Reliability, flexibility, expandability,
  diagnostic capabilities
\end{itemize}

\end{solutionbox}
\begin{mnemonicbox}
``Power Centralizes Processing, Inputs/Outputs Make
Automation''

\end{mnemonicbox}
\subsection*{Question 5(a) OR [3
marks]}\label{q5a}

\textbf{Draw and explain construction of DC Servo motor.}

\begin{solutionbox}
DC servo motors provide precise position control with
feedback for automation and robotics.

\textbf{Construction Diagram:}

\begin{verbatim}
      Feedback
      Device
        |
        v
    .{-{-}{-}{-}{-}{-}{-}{-}.      Shaft}
    |        |{-{-}{-}{-}{-}{-}{-}{-}{-}{-}}
    |        |
    | Motor  |
    |        |
    |        |
    {{-}{-}{-}{-}{-}{-}{-}{-}}
        \^{}
        |
     Control
     Signal
\end{verbatim}

\textbf{Construction Table:}

{\def\LTcaptype{none} % do not increment counter
\begin{longtable}[]{@{}ll@{}}
\toprule\noalign{}
Component & Function \\
\midrule\noalign{}
\endhead
\bottomrule\noalign{}
\endlastfoot
Armature & Rotates within magnetic field \\
Field magnets & Creates magnetic field (often permanent magnets) \\
Commutator & Transfers power to rotating armature \\
Feedback device & Encoder/tachometer for position/speed feedback \\
Brushes & Connect power to commutator \\
\end{longtable}
}

\begin{itemize}
\tightlist
\item
  \textbf{Low inertia}: Special design allows rapid
  acceleration/deceleration
\item
  \textbf{High torque-to-inertia ratio}: Responds quickly to control
  signals
\end{itemize}

\end{solutionbox}
\begin{mnemonicbox}
``Precise Position Feedback Drives Exact Control''

\end{mnemonicbox}
\subsection*{Question 5(b) OR [4
marks]}\label{q5b}

\textbf{Explain working of BLDC motor.}

\begin{solutionbox}
Brushless DC (BLDC) motors use electronic commutation
instead of mechanical brushes and commutator.

\textbf{BLDC Operation Diagram:}

\begin{center}
\textbf{Mermaid Diagram (Code)}
\begin{verbatim}
{Shaded}
{Highlighting}[]
graph LR
    PS[Power Supply] {-{-}{} C[Controller]}
    C {-{-}{} D[Driver Circuit]}
    D {-{-}{} W[Stator Windings]}
    HS[Hall Sensors] {-{-}{} C}
    W {-{-}{} R[Rotor Rotation]}
    R {-{-}{} HS}
{Highlighting}
{Shaded}
\end{verbatim}
\end{center}

\textbf{Working Principle Table:}

{\def\LTcaptype{none} % do not increment counter
\begin{longtable}[]{@{}ll@{}}
\toprule\noalign{}
Component & Function \\
\midrule\noalign{}
\endhead
\bottomrule\noalign{}
\endlastfoot
Stator & Fixed windings that generate rotating magnetic field \\
Rotor & Permanent magnets that follow rotating field \\
Electronic controller & Replaces mechanical commutation \\
Hall sensors & Detect rotor position for synchronized switching \\
Driver circuit & Provides sequence of currents to stator coils \\
\end{longtable}
}

\begin{itemize}
\tightlist
\item
  \textbf{Commutation}: Electronic switching sequences power to stator
  windings
\item
  \textbf{Efficiency}: Higher efficiency due to elimination of brush
  losses
\item
  \textbf{Reliability}: No brush wear or sparking, longer lifespan
\end{itemize}

\end{solutionbox}
\begin{mnemonicbox}
``Electronic Switching Creates Rotation Without
Brushes''

\end{mnemonicbox}
\subsection*{Question 5(c) OR [7
marks]}\label{q5c}

\textbf{Explain construction and working of VFD.}

\begin{solutionbox}
Variable Frequency Drive (VFD) controls AC motor speed
by varying frequency and voltage.

\textbf{VFD Construction Diagram:}

\begin{center}
\textbf{Mermaid Diagram (Code)}
\begin{verbatim}
{Shaded}
{Highlighting}[]
graph LR
    A[AC Input] {-{-}{} R[Rectifier]}
    R {-{-}{} D[DC Bus/Filter]}
    D {-{-}{} I[Inverter]}
    I {-{-}{} M[Motor]}
    C[Control Circuit] {-{-}{} I}
    F[Feedback] {-{-}{} C}
{Highlighting}
{Shaded}
\end{verbatim}
\end{center}

\textbf{Construction and Working Table:}

{\def\LTcaptype{none} % do not increment counter
\begin{longtable}[]{@{}
  >{\raggedright\arraybackslash}p{(\linewidth - 4\tabcolsep) * \real{0.2903}}
  >{\raggedright\arraybackslash}p{(\linewidth - 4\tabcolsep) * \real{0.3871}}
  >{\raggedright\arraybackslash}p{(\linewidth - 4\tabcolsep) * \real{0.3226}}@{}}
\toprule\noalign{}
\begin{minipage}[b]{\linewidth}\raggedright
Section
\end{minipage} & \begin{minipage}[b]{\linewidth}\raggedright
Components
\end{minipage} & \begin{minipage}[b]{\linewidth}\raggedright
Function
\end{minipage} \\
\midrule\noalign{}
\endhead
\bottomrule\noalign{}
\endlastfoot
Rectifier & Diodes/SCRs & Converts AC to DC \\
DC Bus & Capacitors, inductors & Filters and smooths DC \\
Inverter & IGBTs/transistors & Converts DC to variable frequency AC \\
Control circuit & Microprocessor & Controls switching frequency and
patterns \\
Cooling system & Fans, heat sinks & Maintains safe operating
temperature \\
Protection circuits & Sensors, relays & Prevents damage from faults \\
\end{longtable}
}

\begin{itemize}
\tightlist
\item
  \textbf{Speed control}: V/f ratio maintained to provide constant
  torque
\item
  \textbf{Energy savings}: Adjusts power to actual load requirements
\item
  \textbf{Soft start}: Gradual acceleration prevents mechanical shock
\end{itemize}

\end{solutionbox}
\begin{mnemonicbox}
``Rectify, Filter, Invert Frequency For Motor
Control''

\end{mnemonicbox}

\end{document}
