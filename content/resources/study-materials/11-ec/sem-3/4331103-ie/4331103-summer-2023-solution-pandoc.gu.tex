\documentclass[10pt,a4paper]{article}

% content/resources/templates/preamble.tex
\usepackage[margin=0.6in]{geometry}
\author{Milav Dabgar}
\usepackage{amsmath,amssymb,amsthm}
\usepackage{booktabs}
\usepackage{multirow}
\usepackage{xcolor}
\usepackage{tcolorbox}
\tcbuselibrary{breakable,skins}
\usepackage[colorlinks=true,linkcolor=blue]{hyperref}
\usepackage{titlesec}
\usepackage{enumitem}
\usepackage{tikz}
\usepackage{pgfplots}
\usepackage{circuitikz}
\usepackage[version=4]{mhchem}
\usepackage{longtable}
\usepackage{array}
\usepackage{float}
\usepackage{caption}
\usepackage{listings}

\lstset{
  basicstyle=\small\ttfamily,
  breaklines=true,
  breakatwhitespace=false,
  postbreak=\mbox{\textcolor{red}{$\hookrightarrow$}\space},
  float=false,
  numbers=left,
  numberstyle=\tiny\color{gray},
  numbersep=10pt,
  xleftmargin=2em,
  keywordstyle=\color{blue},
  commentstyle=\color{green!60!black},
  stringstyle=\color{purple},
  backgroundcolor=\color{gray!5},
  showstringspaces=false,
  tabsize=2,
  captionpos=b,
  keepspaces=true,
  columns=flexible
}

\pgfplotsset{compat=1.18}
\usetikzlibrary{shapes,arrows,positioning,calc,patterns,decorations.pathmorphing,decorations.markings,arrows.meta}

% Color scheme
\definecolor{headcolor}{RGB}{0,102,204}
\definecolor{keycolor}{RGB}{220,20,60}
\definecolor{solutioncolor}{RGB}{34,139,34}
\definecolor{mnemoniccolor}{RGB}{148,0,211}
\definecolor{codecolor}{RGB}{0,0,100}

% Spacing
\setlength{\parskip}{3pt}
\setlist[itemize]{nosep}
\setlist[enumerate]{nosep}

% Title formatting
\titleformat{\section}{\Large\bfseries\color{headcolor}}{\thesection}{1em}{}
\titleformat{\subsection}{\large\bfseries\color{headcolor}}{\thesubsection}{1em}{}

% Pandoc tightlist compatibility
\providecommand{\tightlist}{%
  \setlength{\itemsep}{0pt}\setlength{\parskip}{0pt}}

% Pandoc longtable compatibility
\newcounter{none}
\def\thenone{}


% content/resources/templates/gujarati-boxes.tex
\usepackage{fontspec}
\usepackage{polyglossia}

% Set Gujarati as main language (document is primarily in Gujarati)
% Note: gloss-gujarati.ldf doesn't exist in polyglossia, but it will use hyphenation patterns
\setdefaultlanguage{gujarati}
\setotherlanguage{english}

% Configure Gujarati font properly
% Use Language=Default to prevent polyglossia from trying to add language-specific features
% that don't exist for Gujarati, which causes "empty feature" warnings
\newfontfamily\gujaratifont[Script=Gujarati,AutoFakeBold=2.5,AutoFakeSlant=0.3]{Noto Sans Gujarati}
\setmainfont[Script=Gujarati,AutoFakeBold=2.5,AutoFakeSlant=0.3]{Noto Sans Gujarati}
% Use Noto Sans Gujarati for monospace to support Gujarati in text
\setmonofont[Scale=0.9]{Noto Sans Gujarati}

% Configure English to use the same font
\newfontfamily\englishfont[Script=Gujarati,AutoFakeBold=2.5,AutoFakeSlant=0.3]{Noto Sans Gujarati}

% Translations for polyglossia
\gappto\captionsgujarati{
  \renewcommand{\tablename}{કોષ્ટક}
  \renewcommand{\figurename}{આકૃતિ}
}

% Helper for TikZ nodes to ensure Gujarati font
\newcommand{\gu}[1]{{\gujaratifont #1}}

% Custom environments
\newtcolorbox{solutionbox}{
    breakable,
    enhanced,
    colback=solutioncolor!5!white,
    colframe=solutioncolor!75!black,
    fonttitle=\bfseries,
    title=જવાબ
}

\newtcolorbox{solutionboxnobreak}{
 colback=solutioncolor!5!white,
 colframe=solutioncolor!75!black,
 fonttitle=\bfseries,
 title=જવાબ
}

\newtcolorbox{keyformula}{
 breakable,
 enhanced,
 colback=keycolor!5!white,
 colframe=keycolor!75!black,
 fonttitle=\bfseries,
 title=રાસાયણિક સમીકરણ/સૂત્ર
}

\newtcolorbox{mnemonicbox}{
 breakable,
 enhanced,
 colback=mnemoniccolor!5!white,
 colframe=mnemoniccolor!75!black,
 fonttitle=\bfseries,
 title=મેમરી ટ્રીક
}


\begin{document}

\begin{center}
{\Huge\bfseries\color{headcolor} Subject Name (Gujarati)}\\[5pt]
{\LARGE 4331103 -- Summer 2023}\\[3pt]
{\large Semester 1 Study Material}\\[3pt]
{\normalsize\textit{Detailed Solutions and Explanations}}
\end{center}

\vspace{10pt}

\subsection*{પ્રશ્ન 1(a) [3
ગુણ]}\label{q1a}

\textbf{TRAIC ની V-I લાક્ષણિકતા દોરો અને સમજાવો.}

\begin{solutionbox}
TRIAC (ટ્રાયોડ ફોર ઓલ્ટરનેટિંગ કરંટ) એ દ્વિદિશાત્મક ત્રણ-ટર્મિનલ
સેમિકન્ડક્ટર ઉપકરણ છે જે ટ્રિગર થાય ત્યારે કોઈપણ દિશામાં વિદ્યુત પ્રવાહ પસાર કરી શકે
છે.

\textbf{આકૃતિ:}

\begin{verbatim}
      I
      ↑
      │        MT2
      │        /│{}
      │       / │ {}
      │      /  │  {}
Quadrant III /   G   { Quadrant I}
      │    /    │    {}
      │   /     │     {}
──────┼──/──────┼──────{──── V}
      │ /       │       {}
      │/        │        {}
      │{        │        /}
      │ {       │       /}
      │  {      │      /}
Quadrant IV {   │     / Quadrant II}
      │    {    │    /}
      │     {   │   /}
      ↓      {  │  /}
             { │ /}
              {│/}
              MT1
\end{verbatim}

\begin{itemize}
\tightlist
\item
  \textbf{દ્વિદિશાત્મક કાર્યપદ્ધતિ}: TRIAC બંને દિશામાં વીજપ્રવાહ પસાર કરે છે
  (પોઝિટિવ અને નેગેટિવ હાફ સાયકલ્સ)
\item
  \textbf{ક્વોડ્રન્ટ ઓપરેશન}: MT2 અને ગેટની ધ્રુવતા પર આધારિત તમામ ચાર ક્વોડ્રન્ટમાં
  કામ કરે છે
\item
  \textbf{ટ્રિગરિંગ વોલ્ટેજ}: કોઈપણ દિશામાં \pmVBO ખાતે બ્રેકડાઉન થાય છે
\item
  \textbf{હોલ્ડિંગ કરંટ}: કન્ડક્શન જાળવી રાખવા માટે ન્યૂનતમ વિદ્યુત પ્રવાહ
\end{itemize}

\textbf{સ્મરણવાક્ય:} ``ટુ રેક્ટિફાયર્સ ઇન અ કેસ''

\end{solutionbox}
\subsection*{પ્રશ્ન 1(b) [4
ગુણ]}\label{q1b}

\textbf{બે ટ્રાણઝિસ્ટ્ર સામ્યતાનો ઉપયોગ કરીને SCR નું કાર્ય સમજાવો.}

\begin{solutionbox}
SCR (સિલિકોન કંટ્રોલ્ડ રેક્ટિફાયર) ને ઇન્ટરકનેક્ટેડ PNP અને NPN
ટ્રાન્ઝિસ્ટર તરીકે રજૂ કરી શકાય છે.

\textbf{આકૃતિ:}

\begin{verbatim}
         Anode
           │
           │
         ┌─┴─┐
         │   │
     ┌───┤ P ├───┐
     │   │   │   │
     │   └───┘   │
     │           │
     │   ┌───┐   │
     └───┤ N ├───┘
         │   │
     ┌───┤   ├───┐
     │   └───┘   │
     │           │
     │   ┌───┐   │
     └───┤ P ├───┘
         │   │
         └─┬─┘
           │
           │
        Cathode
\end{verbatim}

\begin{itemize}
\tightlist
\item
  \textbf{બે-ટ્રાન્ઝિસ્ટર સ્ટ્રક્ચર}: PNP (Q1) અને NPN (Q2) એવી રીતે જોડાયેલા છે કે
  દરેક ટ્રાન્ઝિસ્ટરનો કલેક્ટર બીજાના બેઝને ડ્રાઇવ કરે છે
\item
  \textbf{રિજનરેટિવ ફીડબેક}: એકવાર બંને ટ્રાન્ઝિસ્ટર કન્ડક્ટ કરવાનું શરૂ કરે, તેઓ
  એકબીજાને સેચુરેશનમાં રાખે છે
\item
  \textbf{ટ્રિગરિંગ}: Q2 બેઝમાં ગેટ કરંટ લાગુ કરવાથી રિજનરેટિવ પ્રક્રિયા શરૂ થાય છે
\item
  \textbf{લેચિંગ}: એકવાર ટ્રિગર થયા પછી, ગેટ સિગ્નલ દૂર કરવામાં આવે તો પણ SCR ON
  રહે છે
\end{itemize}

\textbf{સ્મરણવાક્ય:} ``પુલ નીટ પાથ''

\end{solutionbox}
\subsection*{પ્રશ્ન 1(c) [7
ગુણ]}\label{q1c}

\textbf{LDR નો ઉપયોગ કરીને ફોટો ઇલેક્ટ્રિક રિલેનો સર્કિટ ડાયાગ્રામ દોરો અને તેને
કાર્યકારી સમજાવો.}

\begin{solutionbox}
LDR (લાઇટ ડિપેન્ડન્ટ રેઝિસ્ટર)નો ઉપયોગ કરતું ફોટોઇલેક્ટ્રિક રિલે એ
પ્રકાશ-સક્રિય સ્વિચિંગ સર્કિટ છે.

\textbf{સર્કિટ ડાયાગ્રામ:}

\begin{verbatim}
     +Vcc
      │
      ├───────┐
      │       │
      R1      │
      │       │
      │       │
      ├───────┤
      │       │
      │      LDR
      │       │
      │       │
    ┌─┴─┐     │
    │ B │     │
 ───┤   ├─────┘
    │ C │
    └─┬─┘
      │
      │
      ├─────────┐
      │         │
      R2      Relay
      │         │
      │         │
      └─────────┴───── GND
\end{verbatim}

\begin{itemize}
\tightlist
\item
  \textbf{પ્રકાશ સેન્સિંગ}: પ્રકાશની હાજરીમાં LDR રેઝિસ્ટન્સ ઘટે છે
\item
  \textbf{ટ્રાન્ઝિસ્ટર ઓપરેશન}: જ્યારે LDR પર પ્રકાશ પડે છે, ત્યારે ટ્રાન્ઝિસ્ટર બેઝ
  પરનું વોલ્ટેજ બદલાય છે
\item
  \textbf{રિલે સ્વિચિંગ}: ટ્રાન્ઝિસ્ટર પ્રકાશના આધારે કન્ડક્ટ/કટ ઓફ થાય છે, જેથી રિલે
  સક્રિય/નિષ્ક્રિય થાય છે
\item
  \textbf{થ્રેશોલ્ડ એડજસ્ટમેન્ટ}: પોટેન્શિયોમીટર R1 પ્રકાશ સંવેદનશીલતા સેટ કરે છે
\item
  \textbf{એપ્લિકેશન્સ}: ઓટોમેટિક સ્ટ્રીટ લાઇટ્સ, ચોર-અલાર્મ, ઓટોમેટિક ડોર ઓપનર
\end{itemize}

\textbf{સ્મરણવાક્ય:} ``લાઇટ ડિટેક્ટ્સ રેડિલી''

\end{solutionbox}
\subsection*{પ્રશ્ન 1(c OR) [7
ગુણ]}\label{uxaaauxab0uxab6uxaa8-1c-or-7-uxa97uxaa3}

\textbf{SCR માટે UJT નો ઉપયોગ કરીને ગેટ પલ્સ ટ્રિગર સર્કિટ દોરો અને તેનું કાર્ય
સમજાવો.}

\begin{solutionbox}
UJT (યુનિજંક્શન ટ્રાન્ઝિસ્ટર) SCR માટે વિશ્વસનીય ટ્રિગર પલ્સ પ્રદાન
કરે છે.

\textbf{સર્કિટ ડાયાગ્રામ:}

\begin{verbatim}
        +Vcc
         │
         │
         R1
         │
         │
    ┌────┴────┐
    │         │
    │        B2
    │    UJT  │
    │         │
    │    B1   │
    └────┬────┘
         │
         R3
         │
     ┌───┴────┐
     │        │
     C       SCR Gate
     │        │
     │        │
     └────────┴──── GND
\end{verbatim}

\begin{itemize}
\tightlist
\item
  \textbf{RC ટાઇમિંગ}: R1 અને C ચાર્જિંગ સર્કિટ બનાવે છે જે પલ્સ ફ્રિક્વન્સી નક્કી કરે
  છે
\item
  \textbf{UJT ઓપરેશન}: કેપેસિટર વોલ્ટેજ પીક પોઇન્ટ વોલ્ટેજમાં પહોંચે ત્યારે UJT ફાયર
  થાય છે
\item
  \textbf{પલ્સ જનરેશન}: UJT કેપેસિટરને ડિસ્ચાર્જ કરે છે જેથી તીવ્ર ટ્રિગર પલ્સ પેદા
  થાય છે
\item
  \textbf{SCR ટ્રિગરિંગ}: AC સાયકલમાં ચોક્કસ બિંદુઓએ SCR ચાલુ કરવા માટે પલ્સ ગેટ
  પર લાગુ કરવામાં આવે છે
\item
  \textbf{ફ્રિક્વન્સી કંટ્રોલ}: ફેઝ કંટ્રોલ માટે R1 બદલવાથી પલ્સ ફ્રિક્વન્સી બદલાય છે
\end{itemize}

\textbf{સ્મરણવાક્ય:} ``યુનિફોર્મ જંક્શન્સ ટ્રિગર''

\end{solutionbox}
\subsection*{પ્રશ્ન 2(a) [3
ગુણ]}\label{q2a}

\textbf{SCR ની ટ્રિગરિંગ પદ્ધતિઓ સમજાવો.}

\begin{solutionbox}

{\def\LTcaptype{none} % do not increment counter
\begin{longtable}[]{@{}
  >{\raggedright\arraybackslash}p{(\linewidth - 4\tabcolsep) * \real{0.3654}}
  >{\raggedright\arraybackslash}p{(\linewidth - 4\tabcolsep) * \real{0.4038}}
  >{\raggedright\arraybackslash}p{(\linewidth - 4\tabcolsep) * \real{0.2308}}@{}}
\toprule\noalign{}
\begin{minipage}[b]{\linewidth}\raggedright
ટ્રિગરિંગ પદ્ધતિ
\end{minipage} & \begin{minipage}[b]{\linewidth}\raggedright
કાર્ય સિદ્ધાંત
\end{minipage} & \begin{minipage}[b]{\linewidth}\raggedright
ફાયદા
\end{minipage} \\
\midrule\noalign{}
\endhead
\bottomrule\noalign{}
\endlastfoot
ગેટ ટ્રિગરિંગ & ગેટ ટર્મિનલ પર વિદ્યુત પ્રવાહ લાગુ & સૌથી સામાન્ય, ચોક્કસ નિયંત્રણ \\
થર્મલ ટ્રિગરિંગ & તાપમાન વધવાથી લીકેજ થાય છે & સરળ, કોઈ બાહ્ય સર્કિટ નથી \\
લાઇટ ટ્રિગરિંગ & ફોટોન્સ ઇલેક્ટ્રોન-હોલ જોડી બનાવે છે & ઇલેક્ટ્રિકલ આઇસોલેશન, LASCR
માં વપરાય છે \\
dv/dt ટ્રિગરિંગ & ઝડપી વોલ્ટેજ વૃદ્ધિ ટર્ન-ઓન થવાનું કારણ બને છે & પ્રોટેક્શન સર્કિટ
માટે ઉપયોગી \\
ફોરવર્ડ વોલ્ટેજ ટ્રિગરિંગ & બ્રેકઓવર વોલ્ટેજ વટાવવાથી & કોઈ ગેટ કનેક્શનની જરૂર નથી \\
\end{longtable}
}

\textbf{સ્મરણવાક્ય:} ``ગુડ ટ્રિગર્સ લેટ ડિવાઇસેસ ફાયર''

\end{solutionbox}
\subsection*{પ્રશ્ન 2(b) [4
ગુણ]}\label{q2b}

\textbf{SCR નું કમ્યુટેશન શું છે? વર્ગ-E કમ્યુટેશન સમજાવો.}

\begin{solutionbox}
કમ્યુટેશન એ SCR ના એનોડ કરંટને હોલ્ડિંગ કરંટથી નીચે ઘટાડીને તેને બંધ
કરવાની પ્રક્રિયા છે.

\textbf{ક્લાસ-E કમ્યુટેશન (કોમ્પ્લિમેન્ટરી કમ્યુટેશન):}

\begin{verbatim}
              L1
    AC   ┌────┐────┐
Source   │    │    │
    ┌────┴─┐  │  ┌─┴────┐
    │      │  │  │      │
    │ SCR1 │  │  │ SCR2 │
    │      │  │  │      │
    └────┬─┘  │  └─┬────┘
         │    │    │
         └────┘────┘
              Load
\end{verbatim}

\begin{itemize}
\tightlist
\item
  \textbf{કોમ્પ્લિમેન્ટરી સ્વિચિંગ}: વિરુદ્ધ હાફ-સાયકલમાં બીજા SCR નો ઉપયોગ કરે છે
\item
  \textbf{નેચરલ કમ્યુટેશન}: AC સ્ત્રોત ઝીરો ક્રોસ કરે ત્યારે, એનોડ કરંટ હોલ્ડિંગ કરંટ
  કરતાં નીચે પડે છે
\item
  \textbf{એપ્લિકેશન}: AC પાવર કંટ્રોલ સર્કિટ્સ, સાયક્લોકન્વર્ટર્સ
\item
  \textbf{ફાયદો}: કોઈ વધારાના કમ્યુટેશન ઘટકોની આવશ્યકતા નથી
\end{itemize}

\textbf{સ્મરણવાક્ય:} ``કોમ્પ્લિમેન્ટરી એલિમેન્ટ્સ''

\end{solutionbox}
\subsection*{પ્રશ્ન 2(c) [7
ગુણ]}\label{q2c}

\textbf{SCR માટે સ્નબર સર્કિટ દોરો અને સમજાવો.}

\begin{solutionbox}
સ્નબર સર્કિટ SCR ને વોલ્ટેજ ટ્રાન્ઝિયન્ટ્સ અને dv/dt ટર્ન-ઓનથી રક્ષણ
આપે છે.

\textbf{સર્કિટ ડાયાગ્રામ:}

\begin{verbatim}
         ┌─────────┐
         │         │
AC    ┌──┴──┐     Rs    
Source│     │     ├───┐
  ┌───┤ SCR ├─────┘   │
  │   │     │         │
  │   └──┬──┘         │
  │      │          Cs│
  │      │            │
  │      │            │
  │      └────────────┘
  │
  └─────────────────── Load
\end{verbatim}

\begin{itemize}
\tightlist
\item
  \textbf{RC નેટવર્ક}: SCR પર શ્રેણીબદ્ધ રેસિસ્ટર (Rs) અને કેપેસિટર (Cs) જોડાયેલા છે
\item
  \textbf{ટ્રાન્ઝિયન્ટ સપ્રેશન}: કેપેસિટર વોલ્ટેજ સ્પાઇક્સને અવશોષિત કરે છે જે SCR ને
  નુકસાન પહોંચાડી શકે છે
\item
  \textbf{dv/dt પ્રોટેક્શન}: ઝડપી વોલ્ટેજ વધારાને કારણે ખોટા ટ્રિગરિંગને અટકાવે છે
\item
  \textbf{ટર્ન-ઓફ આસિસ્ટન્સ}: વૈકલ્પિક કરંટ પાથ પ્રદાન કરીને કમ્યુટેશનમાં મદદ કરે છે
\item
  \textbf{કમ્પોનન્ટ પસંદગી}: Cs લોડ કરંટ પર આધારિત, Rs ડિસ્ચાર્જ કરંટને મર્યાદિત
  કરે છે
\end{itemize}

\textbf{સ્મરણવાક્ય:} ``સેફલી ન્યુટ્રલાઇઝીસ અનવોન્ટેડ બ્રેકઓવર''

\end{solutionbox}
\subsection*{પ્રશ્ન 2(a OR) [3
ગુણ]}\label{uxaaauxab0uxab6uxaa8-2a-or-3-uxa97uxaa3}

\textbf{SCR ની વર્તમાન સંરક્ષણ પદ્ધતિ વિશે સમજાવો.}

\begin{solutionbox}

{\def\LTcaptype{none} % do not increment counter
\begin{longtable}[]{@{}
  >{\raggedright\arraybackslash}p{(\linewidth - 4\tabcolsep) * \real{0.3654}}
  >{\raggedright\arraybackslash}p{(\linewidth - 4\tabcolsep) * \real{0.3654}}
  >{\raggedright\arraybackslash}p{(\linewidth - 4\tabcolsep) * \real{0.2692}}@{}}
\toprule\noalign{}
\begin{minipage}[b]{\linewidth}\raggedright
સંરક્ષણ પદ્ધતિ
\end{minipage} & \begin{minipage}[b]{\linewidth}\raggedright
કાર્ય સિદ્ધાંત
\end{minipage} & \begin{minipage}[b]{\linewidth}\raggedright
એપ્લિકેશન્સ
\end{minipage} \\
\midrule\noalign{}
\endhead
\bottomrule\noalign{}
\endlastfoot
ફ્યુઝ & કરંટ રેટિંગ વટાવે ત્યારે પીગળે છે & સરળ, આર્થિક સંરક્ષણ \\
સર્કિટ બ્રેકર & ઓવરલોડ પર ટ્રિપ થાય છે, રીસેટ કરી શકાય છે & ફરીથી ઉપયોગ કરી
શકાય તેવું સંરક્ષણ \\
કરંટ લિમિટિંગ રિએક્ટર & ફોલ્ટ કરંટ મેગ્નિટ્યુડને મર્યાદિત કરે છે & ઔદ્યોગિક પાવર
કંટ્રોલ \\
ઇલેક્ટ્રોનિક કરંટ લિમિટિંગ & કરંટને સેન્સ કરે છે અને ગેટને નિયંત્રિત કરે છે & ચોક્કસ
સંરક્ષણ \\
ક્રોબાર સર્કિટ & ઓવરલોડ પર પાવર સપ્લાય શોર્ટ કરે છે & સંવેદનશીલ લોડ્સનું રક્ષણ કરે
છે \\
\end{longtable}
}

\textbf{સ્મરણવાક્ય:} ``ફોલ્ટ કરંટ કોઝીસ ઇક્વિપમેન્ટ ડેમેજ''

\end{solutionbox}
\subsection*{પ્રશ્ન 2(b OR) [4
ગુણ]}\label{uxaaauxab0uxab6uxaa8-2b-or-4-uxa97uxaa3}

\textbf{ઓપ્ટો-એસસીઆરની કામગીરી સમજાવો.}

\begin{solutionbox}
ઓપ્ટો-SCR (અથવા લાઇટ એક્ટિવેટેડ SCR) એક આઇસોલેટેડ પેકેજમાં લાઇટ
સોર્સ અને SCR ને જોડે છે.

\textbf{આકૃતિ:}

\begin{verbatim}
      ┌───────────────┐
      │   ┌───┐       │
      │   │   │       │
 LED  │   │ ◄─┼───┐   │
Anode ├───┤LED│   │   │
      │   │   │   │   │
      │   └───┘   │   │
 LED  │           │   │
Cathod├───────────┘   │
      │               │
      │      ┌───┐    │
      │      │   │ SCR│
  SCR │      │ S ├───Anode
 Gate ├──────┤   │    │
      │      │ C │    │
      │      │ R │    │
  SCR │      └───┘    │
Cathod├───────────────┘
      │               │
      └───────────────┘
\end{verbatim}

\begin{itemize}
\tightlist
\item
  \textbf{ઇલેક્ટ્રિકલ આઇસોલેશન}: LED ઇલેક્ટ્રિકલ કનેક્શન વિના ઓપ્ટિકલી SCR ને ટ્રિગર
  કરે છે
\item
  \textbf{નોઇઝ ઇમ્યુનિટી}: ઇલેક્ટ્રિકલ નોઇઝ અને ઇન્ટરફેરન્સથી રક્ષિત
\item
  \textbf{હાઇ-વોલ્ટેજ આઇસોલેશન}: કંટ્રોલ અને પાવર સર્કિટ્સને અલગ કરે છે
\item
  \textbf{એપ્લિકેશન્સ}: ઔદ્યોગિક નિયંત્રણ, હાઇ-વોલ્ટેજ સ્વિચિંગ
\end{itemize}

\textbf{સ્મરણવાક્ય:} ``લાઇટ એક્ટિવેટ્સ સિલિકોન કંટ્રોલ''

\end{solutionbox}
\subsection*{પ્રશ્ન 2(c OR) [7
ગુણ]}\label{uxaaauxab0uxab6uxaa8-2c-or-7-uxa97uxaa3}

\textbf{ફોર્સ કમ્યુટેશન શું છે? કોઈપણ બે સમજાવો.}

\begin{solutionbox}
ફોર્સ કમ્યુટેશન એ SCR ના એનોડ કરંટને હોલ્ડિંગ લેવલથી નીચે ઘટાડીને
કૃત્રિમ રીતે બંધ કરવાની પ્રક્રિયા છે.

\textbf{1. ક્લાસ A કમ્યુટેશન (સેલ્ફ-કમ્યુટેશન):}

\begin{verbatim}
    ┌───┐
    │   │    L
    │   ├────┐─────┐
AC  │   │    │     │
Sour┤   │    │   SCR
    │   │    │     │
    └───┘    C     │
             │     │
             └─────┘
               Load
\end{verbatim}

\begin{itemize}
\tightlist
\item
  \textbf{LC રેઝોનન્ટ સર્કિટ}: SCR ની આસપાસ સમાંતર L-C દોલનો પેદા કરે છે
\item
  \textbf{રિવર્સ કરંટ}: L-C સર્કિટ SCR દ્વારા રિવર્સ કરંટને દબાણ આપે છે
\item
  \textbf{એપ્લિકેશન્સ}: ઇન્વર્ટર્સ, ચોપર્સ
\end{itemize}

\textbf{2. ક્લાસ B કમ્યુટેશન (રેઝોનન્ટ પલ્સ કમ્યુટેશન):}

\begin{verbatim}
              Commutating
                Switch
    ┌───┐      ┌───┐
    │   │      │   │
AC  │   │    L │   │
Sour┤   ├────┐─┴──┐
    │   │    │    │
    └───┘    │    │
            SCR   C
             │    │
             └────┘
              Load
\end{verbatim}

\begin{itemize}
\tightlist
\item
  \textbf{એક્સટર્નલ સ્વિચ}: વધારાનો SCR અથવા સ્વિચ કમ્યુટેશનને ટ્રિગર કરે છે
\item
  \textbf{એનર્જી સ્ટોરેજ}: L-C સર્કિટ ઊર્જાને સંગ્રહિત કરે છે પછી SCR કરંટને રિવર્સ
  કરે છે
\item
  \textbf{એપ્લિકેશન્સ}: DC ચોપર્સ, કંટ્રોલ્ડ રેક્ટિફાયર્સ
\end{itemize}

\textbf{સ્મરણવાક્ય:} ``ફોર્સ સર્કિટ રિવર્સલ''

\end{solutionbox}
\subsection*{પ્રશ્ન 3(a) [3
ગુણ]}\label{q3a}

\textbf{ચાર ડાયોડનો ઉપયોગ કરીને 1-φ ફુલ વેવ બ્રિજ કોન્ટ્રોલએદ રેક્ટિફાયર
સમજાવો.}

\begin{solutionbox}
આ સર્કિટ કંટ્રોલ્ડ સિંગલ-ફેઝ ફુલ-વેવ રેક્ટિફિકેશન માટે ડાયોડ્સ અને SCR
ને જોડે છે.

\textbf{સર્કિટ ડાયાગ્રામ:}

\begin{verbatim}
         D1        D2
     ┌───┬───┐───┬───┐
     │   │   │   │   │
     │   ▼   │   ▼   │
     │       │       │
AC   │       │       │    Load
Sourc┤       │       ├───R───┐
     │       │       │       │
     │   ▲   │   ▲   │       │
     │   │   │   │   │       │
     └───┴───┘───┴───┘       │
        D3   SCR    D4       │
                            GND
\end{verbatim}

\begin{itemize}
\tightlist
\item
  \textbf{બ્રિજ કોન્ફિગરેશન}: ચાર ડાયોડ્સ બ્રિજમાં ગોઠવવામાં આવ્યા છે જેમાંથી એક
  SCR દ્વારા બદલાયેલ છે
\item
  \textbf{વેરિએબલ આઉટપુટ}: SCR કન્ડક્શન એંગલ અને તેથી આઉટપુટ વોલ્ટેજને નિયંત્રિત કરે છે
\item
  \textbf{આર્થિક ડિઝાઇન}: બે અથવા ચારને બદલે માત્ર એક SCR વાપરે છે
\item
  \textbf{કાર્યક્ષમતા}: હાફ-વેવ કંટ્રોલ્ડ રેક્ટિફાયર કરતાં વધુ
\end{itemize}

\textbf{સ્મરણવાક્ય:} ``બ્લેન્ડ ડાયોડ્સ સ્માર્ટલી''

\end{solutionbox}
\subsection*{પ્રશ્ન 3(b) [4
ગુણ]}\label{q3b}

\textbf{ચોપર શું છે? તેની ઉપયોગો જણાવો.}

\begin{solutionbox}

{\def\LTcaptype{none} % do not increment counter
\begin{longtable}[]{@{}
  >{\raggedright\arraybackslash}p{(\linewidth - 2\tabcolsep) * \real{0.3810}}
  >{\raggedright\arraybackslash}p{(\linewidth - 2\tabcolsep) * \real{0.6190}}@{}}
\toprule\noalign{}
\begin{minipage}[b]{\linewidth}\raggedright
પાસા
\end{minipage} & \begin{minipage}[b]{\linewidth}\raggedright
વર્ણન
\end{minipage} \\
\midrule\noalign{}
\endhead
\bottomrule\noalign{}
\endlastfoot
વ્યાખ્યા & DC-DC કન્વર્ટર જે ફિક્સ્ડ DC ઇનપુટને વેરિએબલ DC આઉટપુટમાં રૂપાંતરિત કરે છે \\
કાર્ય સિદ્ધાંત & પીરિયોડિકલી ઉચ્ચ આવૃત્તિએ DC ઇનપુટને ચાલુ/બંધ કરે છે \\
પ્રકારો & સ્ટેપ-ડાઉન (બક), સ્ટેપ-અપ (બૂસ્ટ), બક-બૂસ્ટ, ક્યુક \\
કંટ્રોલ મેથડ્સ & PWM, ફ્રિક્વન્સી મોડ્યુલેશન, કરંટ-લિમિટ કંટ્રોલ \\
એપ્લિકેશન્સ & DC મોટર સ્પીડ કંટ્રોલ, બેટરી ચાર્જર્સ, UPS, સોલાર સિસ્ટમ્સ, ઇલેક્ટ્રિક
વાહનો \\
\end{longtable}
}

\textbf{સ્મરણવાક્ય:} ``ચોપ્સ કરંટ પરફેક્ટલી''

\end{solutionbox}
\subsection*{પ્રશ્ન 3(c) [7
ગુણ]}\label{q3c}

\textbf{1-φ A.C. લોડ માટે SCR નો ઉપયોગ કરીને સ્ટેટિક સ્વીચના સર્કિટ ડાયાગ્રામ
દોરો અને સમજાવો.}

\begin{solutionbox}
SCR નો ઉપયોગ કરતું સ્ટેટિક સ્વિચ AC લોડ્સ માટે નોન-મિકેનિકલ
સ્વિચિંગ પ્રદાન કરે છે.

\textbf{સર્કિટ ડાયાગ્રામ:}

\begin{verbatim}
              SCR1
             ┌──┐
     ┌───────┤  ├────┐
     │       └──┘    │
     │               │
AC   │               │   AC
Sourc┤               ├── Load
     │               │
     │       ┌──┐    │
     └───────┤  ├────┘
             └──┘
              SCR2
               │
               │
               │
          ┌────┴────┐
          │ Trigger │
          │ Circuit │
          └─────────┘
\end{verbatim}

\begin{itemize}
\tightlist
\item
  \textbf{એન્ટિપેરેલલ SCRs}: બાઇડિરેક્શનલ કન્ડક્શન માટે ત્રણ SCRs ઇન્વર્સ પેરેલલમાં
  જોડાયેલા છે
\item
  \textbf{ગેટ કંટ્રોલ}: યોગ્ય સમયના ગેટ સિગ્નલ્સ લોડને પાવર નિયંત્રિત કરે છે
\item
  \textbf{ઝીરો-ક્રોસિંગ સ્વિચિંગ}: SCRs કુદરતી રીતે ઝીરો ક્રોસિંગ પર બંધ થાય છે
\item
  \textbf{એપ્લિકેશન્સ}: હીટર કંટ્રોલ, મોટર સોફ્ટ-સ્ટાર્ટિંગ, લાઇટિંગ કંટ્રોલ
\item
  \textbf{ફાયદા}: કોઈ મૂવિંગ પાર્ટ્સ નહીં, સાયલેન્ટ ઓપરેશન, લોંગ લાઇફ
\end{itemize}

\textbf{સ્મરણવાક્ય:} ``સોલિડ સ્વિચિંગ ટેક્નોલોજી''

\end{solutionbox}
\subsection*{પ્રશ્ન 3(a OR) [3
ગુણ]}\label{uxaaauxab0uxab6uxaa8-3a-or-3-uxa97uxaa3}

\textbf{ડીસી ચોપરનો મૂળ સિદ્ધાંત સમજાવો.}

\begin{solutionbox}

{\def\LTcaptype{none} % do not increment counter
\begin{longtable}[]{@{}ll@{}}
\toprule\noalign{}
ઘટક & કાર્ય \\
\midrule\noalign{}
\endhead
\bottomrule\noalign{}
\endlastfoot
સ્વિચિંગ ડિવાઇસ & SCR, MOSFET, IGBT ઉચ્ચ આવૃત્તિએ DC સ્વિચ કરે છે \\
કંટ્રોલ સર્કિટ & ON/OFF સમયને નિયંત્રિત કરવા માટે PWM ગેટ સિગ્નલ્સ જનરેટ કરે છે \\
ડ્યુટી સાયકલ & કુલ સમયગાળા પર ON સમયનો ગુણોત્તર આઉટપુટ નક્કી કરે છે \\
આઉટપુટ ફિલ્ટર & રિપલ ઘટાડવા માટે ચોપ્ડ આઉટપુટને સ્મૂધ કરે છે \\
કાર્ય સિદ્ધાંત & સરેરાશ વોલ્ટેજ = ઇનપુટ વોલ્ટેજ \times ડ્યુટી સાયકલ \\
\end{longtable}
}

\textbf{સ્મરણવાક્ય:} ``ડાયરેક્ટ કરંટ કંટ્રોલ''

\end{solutionbox}
\subsection*{પ્રશ્ન 3(b OR) [4
ગુણ]}\label{uxaaauxab0uxab6uxaa8-3b-or-4-uxa97uxaa3}

\textbf{આના પર ટૂંકી નોંધ લખો: અન-ઇન્ટરપ્ટેડ પાવર સપ્લાય (UPS).}

\begin{solutionbox}
UPS મુખ્ય સપ્લાય નિષ્ફળ જાય ત્યારે ઇમરજન્સી પાવર પ્રદાન કરે છે.

\textbf{બ્લોક ડાયાગ્રામ:}

\begin{verbatim}
    ┌─────────┐    ┌─────────┐    ┌─────────┐
    │  Mains  │    │Rectifier│    │ Inverter│
    │  Input  ├────┤  \& DC   ├────┤  \& AC   ├─── Output
    │ (AC)    │    │ Section │    │ Section │    (AC)
    └─────────┘    └─────────┘    └─────────┘
                        │
                    ┌───┴───┐
                    │Battery│
                    │System │
                    └───────┘
\end{verbatim}

\begin{itemize}
\tightlist
\item
  \textbf{બેકઅપ પાવર}: આઉટેજ દરમિયાન સતત પાવર પ્રદાન કરે છે
\item
  \textbf{પ્રકારો}: ઓનલાઇન, ઓફલાઇન, લાઇન-ઇન્ટરેક્ટિવ UPS
\item
  \textbf{સુરક્ષા}: પાવર સર્જ, સેગ્સ અને ફ્રિક્વન્સી વેરિએશન્સ સામે
\item
  \textbf{એપ્લિકેશન્સ}: કોમ્પ્યુટર્સ, મેડિકલ ઇક્વિપમેન્ટ, ટેલિકોમ્યુનિકેશન્સ
\end{itemize}

\textbf{સ્મરણવાક્ય:} ``અનઇન્ટરપ્ટેડ પાવર સિક્યોરલી''

\end{solutionbox}
\subsection*{પ્રશ્ન 3(c OR) [7
ગુણ]}\label{uxaaauxab0uxab6uxaa8-3c-or-7-uxa97uxaa3}

\textbf{SMPS ના બ્લોક ડાયાગ્રામ દોરો અને દરેક બ્લોકનું કાર્ય સમજાવો.}

\begin{solutionbox}
સ્વિચ્ડ-મોડ પાવર સપ્લાય કુશળતાથી AC ને રેગ્યુલેટેડ DC માં રૂપાંતરિત કરે
છે.

\textbf{બ્લોક ડાયાગ્રામ:}

\begin{verbatim}
    ┌─────────┐    ┌─────────┐    ┌─────────┐    ┌─────────┐    ┌─────────┐
    │  Mains  │    │  Input  │    │High{-Freq│    │Output   │    │Output   │}
    │  Input  ├────┤Rectifier├────┤Switching├────┤Rectifier├────┤ Filter  ├─── DC Output
    │  (AC)   │    │\& Filter │    │ Circuit │    │\& Filter │    │         │
    └─────────┘    └─────────┘    └─────────┘    └─────────┘    └─────────┘
                                       │
                                  ┌────┴────┐
                                  │ Control │
                                  │ Circuit │
                                  └─────────┘
\end{verbatim}

\begin{itemize}
\tightlist
\item
  \textbf{ઇનપુટ રેક્ટિફાયર}: AC ને અનરેગ્યુલેટેડ DC માં રૂપાંતરિત કરે છે
\item
  \textbf{હાઇ-ફ્રિક્વન્સી સ્વિચિંગ}: ટ્રાન્ઝિસ્ટરનો ઉપયોગ કરીને DC ને હાઇ-ફ્રિક્વન્સી
  AC માં રૂપાંતરિત કરે છે
\item
  \textbf{ટ્રાન્સફોર્મર}: આઇસોલેશન અને વોલ્ટેજ સ્કેલિંગ પ્રદાન કરે છે
\item
  \textbf{આઉટપુટ રેક્ટિફાયર}: હાઇ-ફ્રિક્વન્સી AC ને DC માં રૂપાંતરિત કરે છે
\item
  \textbf{ફિલ્ટર}: રિપલ ઘટાડવા માટે DC આઉટપુટને સ્મૂધ કરે છે
\item
  \textbf{કંટ્રોલ સર્કિટ}: ફીડબેક દ્વારા આઉટપુટને રેગ્યુલેટ કરે છે
\end{itemize}

\textbf{સ્મરણવાક્ય:} ``સ્વિચ મોડ પાવર સિસ્ટમ''

\end{solutionbox}
\subsection*{પ્રશ્ન 4(a) [3
ગુણ]}\label{q4a}

\textbf{1-φ DC શન્ટ મોટરના ગતિ નિયંત્રણ માટે TRIAC નો ઉપયોગ કરીને સર્કિટ
ડાયાગ્રામ દોરો અને તેની કામગીરી સમજાવો.}

\begin{solutionbox}
TRIAC-આધારિત સ્પીડ કંટ્રોલ DC શન્ટ મોટર માટે કાર્યક્ષમ વેરિએબલ
સ્પીડ પ્રદાન કરે છે.

\textbf{સર્કિટ ડાયાગ્રામ:}

\begin{verbatim}
     ┌────────┐   ┌────────┐      ┌───────┐
AC   │        │   │        │      │ DC    │
Sourc┤ TRIAC  ├───┤ Bridge ├──────┤ Shunt │
     │        │   │Rectifir│      │ Motor │
     └────────┘   └────────┘      └───────┘
         │
     ┌───┴───┐
     │ DIAC  │
     │       │
     └───┬───┘
         │
     ┌───┴───┐
     │       │
     │   R   │
     │       │
     └───┬───┘
         │
     ┌───┴───┐
     │   C   │
     │       │
     └───────┘
\end{verbatim}

\begin{itemize}
\tightlist
\item
  \textbf{ફેઝ કંટ્રોલ}: TRIAC ફેઝ એંગલ કંટ્રોલ દ્વારા અસરકારક વોલ્ટેજ બદલે છે
\item
  \textbf{રેક્ટિફિકેશન}: બ્રિજ રેક્ટિફાયર AC ને DC માં મોટર માટે રૂપાંતરિત કરે છે
\item
  \textbf{સ્પીડ વેરિએશન}: લાગુ કરેલા વોલ્ટેજના પ્રમાણમાં મોટર સ્પીડ
\item
  \textbf{RC ટાઇમિંગ}: RC નેટવર્ક TRIAC ના ફાયરિંગ એંગલને નક્કી કરે છે
\end{itemize}

\textbf{સ્મરણવાક્ય:} ``TRIAC રેગ્યુલેટ્સ સ્પીડ''

\end{solutionbox}
\subsection*{પ્રશ્ન 4(b) [4
ગુણ]}\label{q4b}

\textbf{IC-556 નો ઉપયોગ કરીને ચાર તબક્કાના ક્રમિક ટાઈમર સર્કિટ ડાયાગ્રામ દોરો
અને સમજાવો.}

\begin{solutionbox}
IC-556 ડ્યુઅલ ટાઇમરને મલ્ટી-સ્ટેજ સિક્વેન્શિયલ ટાઇમર તરીકે કોન્ફિગર
કરી શકાય છે.

\textbf{સર્કિટ ડાયાગ્રામ:}

\begin{verbatim}
    Vcc
     │
     ├─────┬─────┬─────┬─────┐
     │     │     │     │     │
    R1    R2    R3    R4     │
     │     │     │     │     │
     ├─────┴─────┴─────┴─────┤
     │                       │
     │       IC{-556          │}
     │                       │
     ├───┬───┬───┬───────────┤
     │   │   │   │           │
     C1  C2  C3  C4          │
     │   │   │   │           │
     └───┴───┴───┴───────────┘
         │   │   │
         O1  O2  O3  O4
\end{verbatim}

\begin{itemize}
\tightlist
\item
  \textbf{ડ્યુઅલ ટાઇમર IC}: IC-556 બે 555 ટાઇમર સર્કિટ્સ ધરાવે છે
\item
  \textbf{કેસ્કેડેડ કોન્ફિગરેશન}: એક સ્ટેજનો આઉટપુટ આગલાને ટ્રિગર કરે છે
\item
  \textbf{ટાઇમિંગ કંટ્રોલ}: RC ટાઇમ કોન્સ્ટન્ટ્સ દરેક સ્ટેજની અવધિ નક્કી કરે છે
\item
  \textbf{એપ્લિકેશન્સ}: ઔદ્યોગિક સિક્વન્સિંગ, પ્રક્રિયા નિયંત્રણ, ઓટોમેશન
\end{itemize}

\textbf{સ્મરણવાક્ય:} ``સિક્વેન્શિયલ સ્ટેપ્સ ટાઇમ્ડ પ્રિસાઇઝલી''

\end{solutionbox}
\subsection*{પ્રશ્ન 4(c) [7
ગુણ]}\label{q4c}

\textbf{ઇન્ડક્શન હીટિંગ સમજાવો.}

\begin{solutionbox}
ઇન્ડક્શન હીટિંગ ઇલેક્ટ્રોમેગ્નેટિક ઇન્ડક્શનનો ઉપયોગ કરીને નોન-કોન્ટેક્ટ
હીટિંગ પ્રક્રિયા છે.

\textbf{આકૃતિ:}

\begin{verbatim}
    ┌───────────────┐
    │ High{-Frequency│}
    │ Power Supply  │
    └───────┬───────┘
            │
    ┌───────┴───────┐
    │    Induction  │
    │     Coil      │
    └───────┬───────┘
            │
    ┌───────┴───────┐
    │   Workpiece   │
    │  (Conductive  │
    │   Material)   │
    └───────────────┘
\end{verbatim}

{\def\LTcaptype{none} % do not increment counter
\begin{longtable}[]{@{}ll@{}}
\toprule\noalign{}
સિદ્ધાંત & વર્ણન \\
\midrule\noalign{}
\endhead
\bottomrule\noalign{}
\endlastfoot
ઇલેક્ટ્રોમેગ્નેટિક ઇન્ડક્શન & કોઇલમાં AC પરિવર્તનશીલ ચુંબકીય ક્ષેત્ર બનાવે છે \\
એડી કરંટ્સ & ચુંબકીય ક્ષેત્ર વર્કપીસમાં કરંટ પ્રેરિત કરે છે \\
રેસિસ્ટિવ હીટિંગ & મટિરિયલ રેસિસ્ટન્સને કારણે એડી કરંટ ગરમી પેદા કરે છે \\
સ્કિન ઇફેક્ટ & ઉચ્ચ આવૃત્તિઓ પર કરંટ સપાટીની નજીક કેન્દ્રિત થાય છે \\
એપ્લિકેશન્સ & હીટ ટ્રીટમેન્ટ, મેલ્ટિંગ, ફોર્જિંગ, બ્રેઝિંગ, કુકિંગ \\
\end{longtable}
}

\textbf{સ્મરણવાક્ય:} ``ઇન્ડ્યુસ્ડ હીટિંગ ઇફિશિયન્ટલી''

\end{solutionbox}
\subsection*{પ્રશ્ન 4(a OR) [3
ગુણ]}\label{uxaaauxab0uxab6uxaa8-4a-or-3-uxa97uxaa3}

\textbf{ત્રણ તબક્કાના IC555 ટાઈમર સર્કિટ દોરો અને સમજાવો.}

\begin{solutionbox}
IC555 નો ઉપયોગ કરતો ત્રણ-સ્ટેજ ટાઇમર ક્રમિક ટાઇમિંગ ઓપરેશન્સ
પ્રદાન કરે છે.

\textbf{સર્કિટ ડાયાગ્રામ:}

\begin{verbatim}
                Vcc
                 │
         ┌───────┴───────┐
         │ Reset         │
    ┌────┤4          8├──┐
    │    │            │  │
    │ ┌──┤2  IC555   3├──┴────┐
    │ │  │             │      │
  R1│ │  │7            │      │
    │ │  │             │     R4
    │ │  │6            │      │
    ├─┘  │             │      │
    │   C1            C2      │
    │    │             │      │
    └────┴──────┬──────┴──────┘
                │
                O1
\end{verbatim}

\begin{itemize}
\tightlist
\item
  \textbf{મોનોસ્ટેબલ મોડ}: દરેક સ્ટેજ ફિક્સ્ડ ટાઇમ ડિલે સાથે મોનોસ્ટેબલ મોડમાં કામ
  કરે છે
\item
  \textbf{કેસ્કેડેડ કનેક્શન}: પ્રથમ ટાઇમરનો આઉટપુટ બીજાને ટ્રિગર કરે છે, વગેરે
\item
  \textbf{ટાઇમિંગ કોમ્પોનન્ટ્સ}: R-C નેટવર્ક દરેક સ્ટેજનો ટાઇમ ડિલે નક્કી કરે છે
\item
  \textbf{એપ્લિકેશન્સ}: ઓટોમેટિક સિક્વન્સિંગ, પ્રોસેસ ટાઇમિંગ, ઔદ્યોગિક નિયંત્રણ
\end{itemize}

\textbf{સ્મરણવાક્ય:} ``ટાઇમ ઇન્ટરવલ્સ ક્રિએટેડ''

\end{solutionbox}
\subsection*{પ્રશ્ન 4(b OR) [4
ગુણ]}\label{uxaaauxab0uxab6uxaa8-4b-or-4-uxa97uxaa3}

\textbf{ડાઇલેક્ટ્રિક હીટિંગનો સિદ્ધાંત સમજાવો.}

\begin{solutionbox}

{\def\LTcaptype{none} % do not increment counter
\begin{longtable}[]{@{}
  >{\raggedright\arraybackslash}p{(\linewidth - 2\tabcolsep) * \real{0.4583}}
  >{\raggedright\arraybackslash}p{(\linewidth - 2\tabcolsep) * \real{0.5417}}@{}}
\toprule\noalign{}
\begin{minipage}[b]{\linewidth}\raggedright
સિદ્ધાંત
\end{minipage} & \begin{minipage}[b]{\linewidth}\raggedright
વર્ણન
\end{minipage} \\
\midrule\noalign{}
\endhead
\bottomrule\noalign{}
\endlastfoot
હાઇ-ફ્રિક્વન્સી ઇલેક્ટ્રિક ફિલ્ડ & મટિરિયલ RF વોલ્ટેજ (1-100 MHz) સાથે ઇલેક્ટ્રોડ્સ
વચ્ચે મૂકવામાં આવે છે \\
મોલેક્યુલર ફ્રિક્શન & ડિપોલ અણુઓ અલ્ટરનેટિંગ ફિલ્ડ સાથે એલાઇન થવાનો પ્રયાસ કરતી વખતે
કંપન/ફરતા રહે છે \\
હીટ જનરેશન & અણુઓ વચ્ચે આંતરિક ઘર્ષણથી સમાન રીતે ગરમી ઉત્પન્ન થાય છે \\
નોન-કન્ડક્ટિવ મટિરિયલ્સ & નોન-કન્ડક્ટિવ મટિરિયલ્સ (પ્લાસ્ટિક, લાકડું, ખોરાક) ગરમ
કરવા માટે અસરકારક \\
એપ્લિકેશન્સ & પ્લાસ્ટિક વેલ્ડિંગ, લાકડું સૂકવવું, ફૂડ પ્રોસેસિંગ (માઇક્રોવેવ ઓવન) \\
\end{longtable}
}

\textbf{સ્મરણવાક્ય:} ``ડાઇલેક્ટ્રિક એનર્જી હીટ્સ''

\end{solutionbox}
\subsection*{પ્રશ્ન 4(c OR) [7
ગુણ]}\label{uxaaauxab0uxab6uxaa8-4c-or-7-uxa97uxaa3}

\textbf{ઇન્ડક્શન હીટિંગ અને ડાઇલેક્ટ્રિક હીટિંગ વચ્ચે સરખામણી કરો.}

\begin{solutionbox}

{\def\LTcaptype{none} % do not increment counter
\begin{longtable}[]{@{}
  >{\raggedright\arraybackslash}p{(\linewidth - 4\tabcolsep) * \real{0.2245}}
  >{\raggedright\arraybackslash}p{(\linewidth - 4\tabcolsep) * \real{0.3878}}
  >{\raggedright\arraybackslash}p{(\linewidth - 4\tabcolsep) * \real{0.3878}}@{}}
\toprule\noalign{}
\begin{minipage}[b]{\linewidth}\raggedright
પેરામીટર
\end{minipage} & \begin{minipage}[b]{\linewidth}\raggedright
ઇન્ડક્શન હીટિંગ
\end{minipage} & \begin{minipage}[b]{\linewidth}\raggedright
ડાઇલેક્ટ્રિક હીટિંગ
\end{minipage} \\
\midrule\noalign{}
\endhead
\bottomrule\noalign{}
\endlastfoot
મૂળભૂત સિદ્ધાંત & ઇલેક્ટ્રોમેગ્નેટિક ઇન્ડક્શન & હાઇ-ફ્રિક્વન્સી ઇલેક્ટ્રિક ફિલ્ડ \\
યોગ્ય મટિરિયલ્સ & કન્ડક્ટિવ મટિરિયલ્સ (મેટલ્સ) & નોન-કન્ડક્ટિવ મટિરિયલ્સ (પ્લાસ્ટિક,
લાકડું) \\
ફ્રિક્વન્સી રેન્જ & 1 kHz થી 1 MHz & 1 MHz થી 1 GHz \\
હીટિંગ મિકેનિઝમ & એડી કરંટ્સ અને હિસ્ટેરિસિસ & મોલેક્યુલર ફ્રિક્શન (ડિપોલ રોટેશન) \\
હીટ ડિસ્ટ્રિબ્યુશન & સરફેસ હીટિંગ (સ્કિન ઇફેક્ટ) & વોલ્યુમેટ્રિક (સમગ્ર સમાન) \\
કાર્યક્ષમતા & મેગ્નેટિક મટિરિયલ્સ માટે 80-90\% & મટિરિયલ પર આધારિત 50-70\% \\
એપ્લિકેશન્સ & મેટલ મેલ્ટિંગ, ફોર્જિંગ, હીટ ટ્રીટમેન્ટ & પ્લાસ્ટિક વેલ્ડિંગ, ફૂડ પ્રોસેસિંગ,
ડ્રાયિંગ \\
ઇક્વિપમેન્ટ & ઇન્ડક્શન કોઇલ, વર્ક પીસ & ઇલેક્ટ્રોડ્સ, ડાઇલેક્ટ્રિક મટિરિયલ \\
\end{longtable}
}

\textbf{સ્મરણવાક્ય:} ``ICED'' - ઇન્ડક્શન કન્ડક્ટિવ, એડી કરંટ્સ; ડાઇલેક્ટ્રિક,
ડિપોલ્સ

\end{solutionbox}
\subsection*{પ્રશ્ન 5(a) [3
ગુણ]}\label{q5a}

\textbf{યુનિવર્સલ મોટરનું બાંધકામ અને કાર્ય સમજાવો.}

\begin{solutionbox}
યુનિવર્સલ મોટર AC અને DC બંને પાવર સોર્સ પર કામ કરે છે.

\textbf{આકૃતિ:}

\begin{verbatim}
       ┌───┐
       │   │
       │   │
       │   │
 ┌─────┴───┴─────┐
 │ Field Winding │
 │   ┌─────┐     │
 │   │     │     │
 │   │Rotor│     │
 │   │     │     │
 │   └─────┘     │
 │               │
 └───────────────┘
      Brushes
\end{verbatim}

\begin{itemize}
\tightlist
\item
  \textbf{સીરીઝ કનેક્શન}: ફિલ્ડ વાઇન્ડિંગ આર્મેચર વાઇન્ડિંગ સાથે શ્રેણીમાં
\item
  \textbf{બાંધકામ}: ફિલ્ડ વાઇન્ડિંગ સાથે સ્ટેટર, કોમ્યુટેટર અને બ્રશ સાથે રોટર
\item
  \textbf{કાર્ય સિદ્ધાંત}: AC અને DC બંને પર સમાન દિશા ટોર્ક
\item
  \textbf{લાક્ષણિકતાઓ}: ઉચ્ચ સ્ટાર્ટિંગ ટોર્ક, ઓછા લોડ પર ઉચ્ચ ગતિ
\item
  \textbf{એપ્લિકેશન્સ}: પોર્ટેબલ ટૂલ્સ, ઘરેલું ઉપકરણો, બ્લેન્ડર્સ
\end{itemize}

\textbf{સ્મરણવાક્ય:} ``યુનિવર્સલી મોટરાઇઝ્ડ''

\end{solutionbox}
\subsection*{પ્રશ્ન 5(b) [4
ગુણ]}\label{q5b}

\textbf{ડીસી સર્વો મોટરનું બાંધકામ દોરો અને સમજાવો.}

\begin{solutionbox}
DC સર્વો મોટર ચોક્કસ પોઝિશન અથવા સ્પીડ કંટ્રોલ પ્રદાન કરે છે.

\textbf{આકૃતિ:}

\begin{verbatim}
     ┌─────────────┐
     │  Permanent  │
     │   Magnet    │
     │   Stator    │
     │   ┌─────┐   │
     │   │     │   │
     │   │Rotor│   │
     │   │     │   │
     │   └─────┘   │
     │             │
     └─────┬───────┘
           │
     ┌─────┴─────┐
     │  Encoder  │
     │  Feedback │
     └───────────┘
\end{verbatim}

\begin{itemize}
\tightlist
\item
  \textbf{બાંધકામ}: પરમેનન્ટ મેગ્નેટ સ્ટેટર, હળવા રોટર, ફીડબેક ડિવાઇસ
\item
  \textbf{કંટ્રોલ સિસ્ટમ}: પોઝિશન/વેલોસિટી ફીડબેક સાથે ક્લોઝ્ડ-લૂપ કંટ્રોલ
\item
  \textbf{લો ઇનર્શિયા}: ઝડપી પ્રતિસાદ અને ચોક્કસ પોઝિશનિંગની મંજૂરી આપે છે
\item
  \textbf{એપ્લિકેશન્સ}: રોબોટિક્સ, CNC મશીન્સ, પોઝિશનિંગ સિસ્ટમ્સ
\item
  \textbf{ફીચર્સ}: ઉચ્ચ ટોર્ક-ટુ-ઇનર્શિયા રેશિયો, ફાસ્ટ રિસ્પોન્સ, એક્યુરસી
\end{itemize}

\textbf{સ્મરણવાક્ય:} ``સર્વો સિસ્ટમ કંટ્રોલ''

\end{solutionbox}
\subsection*{પ્રશ્ન 5(c) [7
ગુણ]}\label{q5c}

\textbf{પ્રોગ્રામેબલ લોજિક કંટ્રોલ (PLC) નો બ્લોક ડાયાગ્રામ દોરો અને દરેક બ્લોકની
કામગીરી સમજાવો.}

\begin{solutionbox}
PLC ઓટોમેશન કંટ્રોલ માટે ઔદ્યોગિક ડિજિટલ કોમ્પ્યુટર છે.

\textbf{બ્લોક ડાયાગ્રામ:}

\begin{verbatim}
    ┌─────────────┐    ┌─────────────┐    ┌─────────────┐
    │             │    │             │    │             │
    │   Input     │    │  Central    │    │   Output    │
    │   Modules   ├────┤ Processing  ├────┤   Modules   │
    │             │    │   Unit      │    │             │
    └─────────────┘    └─────┬───────┘    └─────────────┘
                             │
         ┌───────────────────┼──────────────────┐
         │                   │                  │
    ┌────┴─────┐       ┌─────┴─────┐       ┌────┴─────┐
    │  Memory  │       │Programming│       │  Power   │
    │  Unit    │       │  Device   │       │  Supply  │
    └──────────┘       └───────────┘       └──────────┘
\end{verbatim}

\begin{itemize}
\tightlist
\item
  \textbf{CPU (સેન્ટ્રલ પ્રોસેસિંગ યુનિટ)}: પ્રોગ્રામ એક્ઝિક્યુટ કરે છે, I/O ડેટા પ્રોસેસ
  કરે છે, નિર્ણયો લે છે
\item
  \textbf{ઇનપુટ મોડ્યુલ્સ}: ફિલ્ડ સિગ્નલ્સ (સેન્સર્સ, સ્વિચેસ) ને CPU માટે ડિજિટલ
  સિગ્નલ્સમાં રૂપાંતરિત કરે છે
\item
  \textbf{આઉટપુટ મોડ્યુલ્સ}: CPU કમાન્ડ્સને એક્ટ્યુએટર સિગ્નલ્સ (મોટર્સ, વાલ્વ્સ) માં
  રૂપાંતરિત કરે છે
\item
  \textbf{મેમોરી યુનિટ}: પ્રોગ્રામ અને ડેટા સ્ટોર કરે છે (OS માટે ROM, યુઝર પ્રોગ્રામ
  માટે RAM)
\item
  \textbf{પ્રોગ્રામિંગ ડિવાઇસ}: પ્રોગ્રામ ડેવલપમેન્ટ અને મોનિટરિંગ માટે PC અથવા
  કન્સોલ
\item
  \textbf{પાવર સપ્લાય}: PLC કોમ્પોનન્ટ્સને રેગ્યુલેટેડ પાવર પ્રદાન કરે છે
\end{itemize}

\textbf{સ્મરણવાક્ય:} ``પ્રોગ્રામ્સ લોજિક કમ્પ્લીટલી''

\end{solutionbox}
\subsection*{પ્રશ્ન 5(a OR) [3
ગુણ]}\label{uxaaauxab0uxab6uxaa8-5a-or-3-uxa97uxaa3}

\textbf{સ્ટેપર મોટરનું બાંધકામ દોરો અને સમજાવો.}

\begin{solutionbox}
સ્ટેપર મોટર ચોક્કસ પોઝિશનિંગ માટે ડિસ્ક્રીટ સ્ટેપ્સમાં ફરે છે.

\textbf{આકૃતિ:}

\begin{verbatim}
      ┌───────────┐
      │           │
      │  Stator   │
      │  ┌─────┐  │
      │  │     │  │
      │  │Rotor│  │
      │  │     │  │
      │  └─────┘  │
      │           │
      └───────────┘
          Phases
\end{verbatim}

\begin{itemize}
\tightlist
\item
  \textbf{સ્ટેટર}: મલ્ટિપલ કોઇલ વાઇન્ડિંગ્સ (ફેઝીસ) ધરાવે છે
\item
  \textbf{રોટર}: પરમેનન્ટ મેગ્નેટ અથવા વેરિએબલ રિલક્ટન્સ પ્રકાર
\item
  \textbf{પ્રકારો}: પરમેનન્ટ મેગ્નેટ, વેરિએબલ રિલક્ટન્સ, હાઇબ્રિડ
\item
  \textbf{સ્ટેપ એંગલ}: સામાન્ય રીતે 1.8^\circ (200 સ્ટેપ્સ/રેવ) અથવા 0.9^\circ (400
  સ્ટેપ્સ/રેવ)
\item
  \textbf{એપ્લિકેશન્સ}: પ્રિન્ટર્સ, ડિસ્ક ડ્રાઇવ્સ, રોબોટિક્સ, CNC મશીન્સ
\end{itemize}

\textbf{સ્મરણવાક્ય:} ``સ્ટેપ્સ પ્રિસાઇઝલી મૂવ્ડ''

\end{solutionbox}
\subsection*{પ્રશ્ન 5(b OR) [4
ગુણ]}\label{uxaaauxab0uxab6uxaa8-5b-or-4-uxa97uxaa3}

\textbf{ડીસી શન્ટ મોટર સ્પીડને નિયંત્રિત કરવા માટે સોલિડ સ્ટેટ સર્કિટ સમજાવો.}

\begin{solutionbox}
સોલિડ-સ્ટેટ સર્કિટ DC મોટર સ્પીડ કંટ્રોલ માટે કાર્યક્ષમ અને સ્મૂધ
કંટ્રોલ પ્રદાન કરે છે.

\textbf{સર્કિટ ડાયાગ્રામ:}

\begin{verbatim}
     +Vdc
      │
      │           ┌────────┐
      ├───────────┤ Field  │
      │           │ Winding│
      │           └────────┘
      │
    ┌─┴─┐
    │   │     ┌───────┐
    │PWM├─────┤ MOSFET│    ┌──────┐
    │   │     │Driver │────┤MOSFET│
    └───┘     └───────┘    │      │
                          ┌┴──────┴┐
                          │Armature│
                          │Winding │
                          └────────┘
\end{verbatim}

\begin{itemize}
\tightlist
\item
  \textbf{PWM કંટ્રોલર}: ગતિ નિયંત્રિત કરવા માટે વેરિએબલ ડ્યુટી સાયકલ પલ્સ જનરેટ
  કરે છે
\item
  \textbf{MOSFET ડ્રાઇવર}: પાવર MOSFET માટે ગેટ ડ્રાઇવ પ્રદાન કરે છે
\item
  \textbf{પાવર MOSFET}: આર્મેચર વાઇન્ડિંગમાં કરંટ નિયંત્રિત કરે છે
\item
  \textbf{ફીડબેક}: ટેકોજનરેટર અથવા એન્કોડર સ્પીડ ફીડબેક પ્રદાન કરે છે
\item
  \textbf{ફાયદા}: કાર્યક્ષમ, સરળ નિયંત્રણ, વિશાળ ગતિ રેન્જ
\end{itemize}

\textbf{સ્મરણવાક્ય:} ``પાવર વિથ MOSFET''

\end{solutionbox}
\subsection*{પ્રશ્ન 5(c OR) [7
ગુણ]}\label{uxaaauxab0uxab6uxaa8-5c-or-7-uxa97uxaa3}

\textbf{VFD (વેરિયેબલ ફ્રીક્વન્સી ડ્રાઇવ) ની કામગીરી સમજાવો.}

\begin{solutionbox}
VFD ફ્રિક્વન્સી અને વોલ્ટેજમાં ફેરફાર કરીને AC મોટર સ્પીડ નિયંત્રિત
કરે છે.

\textbf{બ્લોક ડાયાગ્રામ:}

\begin{verbatim}
    ┌─────────┐    ┌─────────┐    ┌─────────┐    ┌─────────┐
    │  AC     │    │Rectifier│    │DC Link  │    │Inverter │    ┌─────────┐
    │  Input  ├────┤ Circuit ├────┤Capacitor├────┤ Circuit ├────┤   AC    │
    │         │    │         │    │         │    │         │    │  Motor  │
    └─────────┘    └─────────┘    └─────────┘    └─────────┘    └─────────┘
                                       │
                                  ┌────┴────┐
                                  │ Control │
                                  │ Circuit │
                                  └─────────┘
\end{verbatim}

{\def\LTcaptype{none} % do not increment counter
\begin{longtable}[]{@{}
  >{\raggedright\arraybackslash}p{(\linewidth - 2\tabcolsep) * \real{0.5238}}
  >{\raggedright\arraybackslash}p{(\linewidth - 2\tabcolsep) * \real{0.4762}}@{}}
\toprule\noalign{}
\begin{minipage}[b]{\linewidth}\raggedright
ઘટક
\end{minipage} & \begin{minipage}[b]{\linewidth}\raggedright
કાર્ય
\end{minipage} \\
\midrule\noalign{}
\endhead
\bottomrule\noalign{}
\endlastfoot
રેક્ટિફાયર & AC ઇનપુટને DC માં રૂપાંતરિત કરે છે (ડાયોડ બ્રિજ અથવા એક્ટિવ ફ્રન્ટ
એન્ડ) \\
DC લિંક & DC ને ફિલ્ટર કરે છે અને ઊર્જા સંગ્રહિત કરે છે (કેપેસિટર્સ, ક્યારેક ઇન્ડક્ટર્સ) \\
ઇન્વર્ટર & DC ને વેરિએબલ ફ્રિક્વન્સી AC માં રૂપાંતરિત કરે છે (PWM સાથે IGBTs) \\
કંટ્રોલ સર્કિટ & સ્પીડ જરૂરિયાત આધારિત ફ્રિક્વન્સી/વોલ્ટેજને રેગ્યુલેટ કરે છે \\
બ્રેકિંગ સર્કિટ & ડિસેલરેશન દરમિયાન રિજનરેટિવ ઊર્જાને વેડફે છે \\
\end{longtable}
}

\begin{itemize}
\tightlist
\item
  \textbf{સ્પીડ કંટ્રોલ}: મોટર સ્પીડ ફ્રિક્વન્સીના પ્રમાણમાં (RPM = 120f/P)
\item
  \textbf{ટોર્ક કંટ્રોલ}: કોન્સ્ટન્ટ ટોર્ક માટે V/f રેશિયો જાળવે છે
\item
  \textbf{એનર્જી સેવિંગ્સ}: ઓછી ગતિએ ઊર્જા વપરાશ ઘટાડે છે
\item
  \textbf{એપ્લિકેશન્સ}: પંપ્સ, ફેન્સ, કન્વેયર્સ, પ્રોસેસ કંટ્રોલ
\item
  \textbf{ફીચર્સ}: સોફ્ટ સ્ટાર્ટ, ઓવરકરંટ પ્રોટેક્શન, રિજનરેટિવ બ્રેકિંગ
\end{itemize}

\textbf{સ્મરણવાક્ય:} ``વેરી ફ્રિક્વન્સી, ડ્રાઇવ મોટર''

\end{solutionbox}

\end{document}
