\documentclass{article}

% content/resources/templates/preamble.tex
\usepackage[margin=0.6in]{geometry}
\author{Milav Dabgar}
\usepackage{amsmath,amssymb,amsthm}
\usepackage{booktabs}
\usepackage{multirow}
\usepackage{xcolor}
\usepackage{tcolorbox}
\tcbuselibrary{breakable,skins}
\usepackage[colorlinks=true,linkcolor=blue]{hyperref}
\usepackage{titlesec}
\usepackage{enumitem}
\usepackage{tikz}
\usepackage{pgfplots}
\usepackage{circuitikz}
\usepackage[version=4]{mhchem}
\usepackage{longtable}
\usepackage{array}
\usepackage{float}
\usepackage{caption}
\usepackage{listings}

\lstset{
  basicstyle=\small\ttfamily,
  breaklines=true,
  breakatwhitespace=false,
  postbreak=\mbox{\textcolor{red}{$\hookrightarrow$}\space},
  float=false,
  numbers=left,
  numberstyle=\tiny\color{gray},
  numbersep=10pt,
  xleftmargin=2em,
  keywordstyle=\color{blue},
  commentstyle=\color{green!60!black},
  stringstyle=\color{purple},
  backgroundcolor=\color{gray!5},
  showstringspaces=false,
  tabsize=2,
  captionpos=b,
  keepspaces=true,
  columns=flexible
}

\pgfplotsset{compat=1.18}
\usetikzlibrary{shapes,arrows,positioning,calc,patterns,decorations.pathmorphing,decorations.markings,arrows.meta}

% Color scheme
\definecolor{headcolor}{RGB}{0,102,204}
\definecolor{keycolor}{RGB}{220,20,60}
\definecolor{solutioncolor}{RGB}{34,139,34}
\definecolor{mnemoniccolor}{RGB}{148,0,211}
\definecolor{codecolor}{RGB}{0,0,100}

% Spacing
\setlength{\parskip}{3pt}
\setlist[itemize]{nosep}
\setlist[enumerate]{nosep}

% Title formatting
\titleformat{\section}{\Large\bfseries\color{headcolor}}{\thesection}{1em}{}
\titleformat{\subsection}{\large\bfseries\color{headcolor}}{\thesubsection}{1em}{}

% Pandoc tightlist compatibility
\providecommand{\tightlist}{%
  \setlength{\itemsep}{0pt}\setlength{\parskip}{0pt}}

% Pandoc longtable compatibility
\newcounter{none}
\def\thenone{}


% content/resources/templates/gujarati-boxes.tex
\usepackage{fontspec}
\usepackage{polyglossia}

% Set Gujarati as main language (document is primarily in Gujarati)
% Note: gloss-gujarati.ldf doesn't exist in polyglossia, but it will use hyphenation patterns
\setdefaultlanguage{gujarati}
\setotherlanguage{english}

% Configure Gujarati font properly
% Use Language=Default to prevent polyglossia from trying to add language-specific features
% that don't exist for Gujarati, which causes "empty feature" warnings
\newfontfamily\gujaratifont[Script=Gujarati,AutoFakeBold=2.5,AutoFakeSlant=0.3]{Noto Sans Gujarati}
\setmainfont[Script=Gujarati,AutoFakeBold=2.5,AutoFakeSlant=0.3]{Noto Sans Gujarati}
% Use Noto Sans Gujarati for monospace to support Gujarati in text
\setmonofont[Scale=0.9]{Noto Sans Gujarati}

% Configure English to use the same font
\newfontfamily\englishfont[Script=Gujarati,AutoFakeBold=2.5,AutoFakeSlant=0.3]{Noto Sans Gujarati}

% Translations for polyglossia
\gappto\captionsgujarati{
  \renewcommand{\tablename}{કોષ્ટક}
  \renewcommand{\figurename}{આકૃતિ}
}

% Helper for TikZ nodes to ensure Gujarati font
\newcommand{\gu}[1]{{\gujaratifont #1}}

% Custom environments
\newtcolorbox{solutionbox}{
    breakable,
    enhanced,
    colback=solutioncolor!5!white,
    colframe=solutioncolor!75!black,
    fonttitle=\bfseries,
    title=જવાબ
}

\newtcolorbox{solutionboxnobreak}{
 colback=solutioncolor!5!white,
 colframe=solutioncolor!75!black,
 fonttitle=\bfseries,
 title=જવાબ
}

\newtcolorbox{keyformula}{
 breakable,
 enhanced,
 colback=keycolor!5!white,
 colframe=keycolor!75!black,
 fonttitle=\bfseries,
 title=રાસાયણિક સમીકરણ/સૂત્ર
}

\newtcolorbox{mnemonicbox}{
 breakable,
 enhanced,
 colback=mnemoniccolor!5!white,
 colframe=mnemoniccolor!75!black,
 fonttitle=\bfseries,
 title=મેમરી ટ્રીક
}


% Custom commands for GTU solutions
% This file defines semantic commands for consistent formatting

% Question command with automatic formatting
\newcommand{\question}[2]{%
  \section*{Question #1}%
  \textbf{#2}%
}

% OR question variant
\newcommand{\questionor}[2]{%
  \section*{Question #1 OR}%
  \textbf{#2}%
}

% Proper table environment with caption
\newenvironment{answertable}[1]{%
  \begin{table}[htbp]
  \centering
  \caption{#1}
}{%
  \end{table}
}

% Proper figure environment for diagrams
\newenvironment{answerdiagram}[1]{%
  \begin{figure}[htbp]
  \centering
  \caption{#1}
}{%
  \end{figure}
}

% Semantic markup for key terms
\newcommand{\keyword}[1]{\textbf{#1}}
\newcommand{\code}[1]{\texttt{#1}}
\newcommand{\classname}[1]{\texttt{#1}}
\newcommand{\methodname}[1]{\texttt{#1}}

% Proper quotation marks
\newcommand{\mnemonic}[1]{``#1''}


\title{ઔદ્યોગિક ઇલેક્ટ્રોનિક્સ (4331103) - શિયાળુ 2022 સોલ્યુશન}
\date{March 1, 2023}

\begin{document}
\maketitle

\questionmarks{1(અ)}{3}{SCR ની રચના દોરો અને સમજાવો.}

\begin{solutionbox}
SCR (સિલિકોન કંટ્રોલ્ડ રેક્ટિફાયર) એ ચાર-લેયર PNPN સેમિકન્ડક્ટર ડિવાઇસ છે જેમાં ત્રણ ટર્મિનલ્સ છે: એનોડ, કેથોડ અને ગેટ.

\begin{center}
\begin{tikzpicture}[auto, node distance=1.5cm]
    \node [gtu block, minimum width=2cm, fill=blue!10] (P1) {P};
    \node [gtu block, minimum width=2cm, fill=red!10, below=0cm of P1] (N1) {N};
    \node [gtu block, minimum width=2cm, fill=blue!10, below=0cm of N1] (P2) {P};
    \node [gtu block, minimum width=2cm, fill=red!10, below=0cm of P2] (N2) {N};
    
    \draw [thick] (P1.north) -- ++(0,0.5) node[above] {Anode (A)};
    \draw [thick] (N2.south) -- ++(0,-0.5) node[below] {Cathode (K)};
    \draw [thick] (P2.east) -- ++(0.5,0) node[right] {Gate (G)};
\end{tikzpicture}
\captionof{figure}{SCR ની રચના}
\end{center}

\begin{itemize}
    \item \keyword{P-N-P-N લેયર્સ}: ચાર અલ્ટરનેટિંગ સેમિકન્ડક્ટર લેયર્સ.
    \item \keyword{ગેટ ટર્મિનલ}: ડિવાઇસના ટર્ન-ઓન ને નિયંત્રિત કરે છે.
    \item \keyword{કરંટ ફ્લો}: ટ્રિગર થવા પર એનોડથી કેથોડ તરફ.
\end{itemize}
\end{solutionbox}

\begin{mnemonicbox}
\mnemonic{Silicon Controls Rectification: SCR controls current flow in one direction only when triggered.}
\end{mnemonicbox}

\questionmarks{1(બ)}{4}{TRIAC ની રચના દોરો અને સમજાવો.}

\begin{solutionbox}
TRIAC (ટ્રાયોડ ફોર અલ્ટરનેટિંગ કરંટ) એ બાયડાયરેક્શનલ ત્રણ-ટર્મિનલ સેમિકન્ડક્ટર ડિવાઇસ છે જે ટ્રિગર થતાં બંને દિશામાં કન્ડક્ટ કરે છે.

\begin{center}
\begin{tikzpicture}[auto, node distance=0cm]
    \node [gtu block, minimum width=2.5cm, fill=red!10] (N4) {N4};
    \node [gtu block, minimum width=3cm, fill=blue!10, below=0cm of N4] (P1) {P1};
    \node [gtu block, minimum width=3cm, fill=red!10, below=0cm of P1] (N1) {N1};
    \node [gtu block, minimum width=3cm, fill=blue!10, below=0cm of N1] (P2) {P2};
    \node [gtu block, minimum width=1.5cm, fill=red!10, below=0cm of P2, xshift=-0.75cm] (N2) {N2};
    \node [gtu block, minimum width=1.5cm, fill=red!10, below=0cm of P2, xshift=0.75cm] (N3) {N3};
    
    \draw [thick] (P1.north) -- ++(0,0.5) node[above] {MT1};
    \draw [thick] (P2.south) -- ++(0,-0.5) node[below] {MT2};
    \draw [thick] (P2.east) -- ++(0.5,0) node[right] {Gate (G)};
\end{tikzpicture}
\captionof{figure}{TRIAC ની રચના}
\end{center}

\begin{itemize}
    \item \keyword{બાયડાયરેક્શનલ ઓપરેશન}: ટ્રિગર થવા પર બંને દિશામાં કન્ડક્ટ કરે છે.
    \item \keyword{ગેટ કંટ્રોલ}: એક ગેટ બંને દિશામાં કન્ડક્શન નિયંત્રિત કરે છે.
    \item \keyword{ઇક્વિવેલન્ટ સર્કિટ}: એન્ટિ-પેરેલલમાં જોડાયેલા બે SCR જેવું કાર્ય કરે છે.
    \item \keyword{AC એપ્લિકેશન્સ}: AC પાવર કંટ્રોલ એપ્લિકેશન્સમાં વ્યાપકપણે ઉપયોગ થાય છે.
\end{itemize}
\end{solutionbox}

\begin{mnemonicbox}
\mnemonic{TRI-direction AC controller: Controls current in both directions in AC circuits.}
\end{mnemonicbox}

\questionmarks{1(ક)}{7}{ઓપ્ટો-આઈસોલેટર, ઓપ્ટો-TRIAC, ઓપ્ટો-SCR, અને ઓપ્ટો-ટ્રાન્ઝિસ્ટરની રચના, કાર્યપદ્ધતિ વર્ણવો અને તેના ઉપયોગો લખો.}

\begin{solutionbox}
ઓપ્ટો-આઈસોલેટર્સ આઇસોલેટેડ સર્કિટ્સ વચ્ચે ઇલેક્ટ્રિકલ સિગ્નલ્સ ટ્રાન્સફર કરવા માટે પ્રકાશનો ઉપયોગ કરે છે.

\begin{center}
\begin{tikzpicture}[auto, node distance=2cm]
    \node [gtu block, fill=yellow!10] (LED) {LED};
    \node [gtu block, right=of LED, fill=green!10] (PD) {Photo Detector};
    \draw [gtu arrow, dashed] (LED) -- node {Light} (PD);
    \node [left=0.5cm of LED] (Input) {Input};
    \node [right=0.5cm of PD] (Output) {Output};
    \draw [thick] (Input) -- (LED);
    \draw [thick] (PD) -- (Output);
\end{tikzpicture}
\captionof{figure}{બેઝિક ઓપ્ટો-આઈસોલેટર બ્લોક ડાયાગ્રામ}
\end{center}

\begin{center}
\captionof{table}{ઓપ્ટો-આઈસોલેટરના પ્રકારો અને ઉપયોગો}
\begin{tabulary}{\linewidth}{|L|L|L|L|}
\hline
\textbf{ડિવાઇસ} & \textbf{રચના} & \textbf{કાર્યપદ્ધતિ} & \textbf{ઉપયોગો} \\ \hline
ઓપ્ટો-આઈસોલેટર & LED + ફોટોડિટેક્ટર & જ્યારે ઇનપુટ કરંટ પ્રવાહિત થાય છે ત્યારે LED પ્રકાશ ઉત્સર્જિત કરે છે; ફોટોડિટેક્ટર આઉટપુટ સર્કિટને સક્રિય કરે છે & સિગ્નલ આઇસોલેશન, મેડિકલ ઉપકરણો, ઔદ્યોગિક નિયંત્રણો \\ \hline
ઓપ્ટો-TRIAC & LED + ફોટો-TRIAC & LED પ્રકાશ દ્વારા TRIAC ને ટ્રિગર કરે છે; ઇલેક્ટ્રિકલ આઇસોલેશન પ્રદાન કરે છે & AC પાવર કંટ્રોલ, સોલિડ સ્ટેટ રિલે, મોટર કંટ્રોલ \\ \hline
ઓપ્ટો-SCR & LED + ફોટો-SCR & LED SCR ને ટ્રિગર કરવા માટે પ્રકાશ ઉત્સર્જિત કરે છે; ઉચ્ચ આઇસોલેશન પ્રદાન કરે છે & DC સ્વિચિંગ, ઔદ્યોગિક નિયંત્રણો, ઉચ્ચ વોલ્ટેજ આઇસોલેશન \\ \hline
ઓપ્ટો-ટ્રાન્ઝિસ્ટર & LED + ફોટો-ટ્રાન્ઝિસ્ટર & LED પ્રકાશ ફોટોટ્રાન્ઝિસ્ટરના બેઝ કરંટને નિયંત્રિત કરે છે & એન્કોડર્સ, લેવલ ડિટેક્શન, પોઝિશન સેન્સિંગ \\ \hline
\end{tabulary}
\end{center}

\begin{itemize}
    \item \keyword{ઇલેક્ટ્રિકલ આઇસોલેશન}: ઇનપુટ અને આઉટપુટ વચ્ચે સંપૂર્ણ અલગતા.
    \item \keyword{નોઇઝ ઇમ્યુનિટી}: ઇલેક્ટ્રિકલ નોઇઝ પ્રત્યે ઉચ્ચ પ્રતિરોધ.
    \item \keyword{સ્પીડ}: માઇક્રોસેકન્ડ રેન્જમાં રિસ્પોન્સ ટાઇમ.
\end{itemize}
\end{solutionbox}

\begin{mnemonicbox}
\mnemonic{LOST: Light Operates Semiconductor Terminals in all opto-devices.}
\end{mnemonicbox}

\questionmarks{1(ક OR)}{7}{બે ટ્રાન્ઝીસ્ટર એનાલોગી વડે SCRનું કાર્ય સમજાવો અને SCRનાં ઇન્ડસ્ટ્રીયલ ઉપયોગો લખો.}

\begin{solutionbox}
SCR ને બે ઇન્ટરકનેક્ટેડ ટ્રાન્ઝિસ્ટર તરીકે મોડેલ કરી શકાય છે: PNP (T1) અને NPN (T2).

\begin{center}
\begin{tikzpicture}[auto, node distance=2cm]
    \node (A) {Anode};
    \node [draw, circle, below=0.5cm of A] (T1) {PNP};
    \node [draw, circle, below=1cm of T1] (T2) {NPN};
    \node [below=0.5cm of T2] (K) {Cathode};
    \node [left=1cm of T2] (G) {Gate};
    
    \draw (A) -- (T1);
    \draw (T1) -- (T2);
    \draw (T2) -- (K);
    \draw (G) -- (T2);
    \draw (T1) to[bend left] (T2);
    \draw (T2) to[bend right] (T1);
\end{tikzpicture}
\captionof{figure}{SCR ની બે ટ્રાન્ઝિસ્ટર એનાલોગી}
\end{center}

\textbf{કાર્ય સિદ્ધાંત:}
\begin{center}
\begin{tabulary}{\linewidth}{|L|L|}
\hline
\textbf{સ્ટેપ} & \textbf{ઓપરેશન} \\ \hline
પ્રારંભિક સ્થિતિ & બંને ટ્રાન્ઝિસ્ટર OFF હોય છે \\ \hline
ગેટ ટ્રિગરિંગ & ગેટમાં (T2ના B2માં) કરંટ ઇન્જેક્ટ કરવામાં આવે છે \\ \hline
રિજનરેટિવ એક્શન & T2 ON થાય છે $\to$ T1 બેઝને કરંટ મળે છે $\to$ T1 ON થાય છે $\to$ T2 બેઝને વધુ કરંટ મળે છે \\ \hline
લેચિંગ & ગેટ સિગ્નલ દૂર કરવામાં આવે તો પણ સ્વ-ટકાઉ કરંટ પ્રવાહ ચાલુ રહે છે \\ \hline
\end{tabulary}
\end{center}

\textbf{SCRના ઔદ્યોગિક ઉપયોગો:}
\begin{itemize}
    \item \keyword{પાવર કંટ્રોલ}: AC/DC મોટર સ્પીડ કંટ્રોલ.
    \item \keyword{સ્વિચિંગ}: સ્ટેટિક સ્વિચ, સોલિડ-સ્ટેટ રિલે.
    \item \keyword{ઇન્વર્ટર}: DC થી AC રૂપાંતર.
    \item \keyword{પ્રોટેક્શન}: ઓવરવોલ્ટેજ પ્રોટેક્શન સર્કિટ.
    \item \keyword{લાઇટિંગ}: લાઇટ ડિમર, ઇલ્યુમિનેશન કંટ્રોલ.
\end{itemize}
\end{solutionbox}

\begin{mnemonicbox}
\mnemonic{POWER: Power control, Overvoltage protection, Welding machines, Electronic converters, Regulated supplies.}
\end{mnemonicbox}

\questionmarks{2(અ)}{3}{એસ.સી.આર માં ટ્રિગરીંગ વ્યાખ્યાયીત કરી.કોઈ પણ બે ટ્રિગરીંગ ટેકનિક સમજાવો.}

\begin{solutionbox}
ટ્રિગરિંગ એ SCRને તેના ગેટ ટર્મિનલ પર યોગ્ય સિગ્નલ લાગુ કરીને ON કરવાની પ્રક્રિયા છે.

\textbf{બે ટ્રિગરિંગ ટેકનિક:}
\begin{center}
\captionof{table}{ટ્રિગરિંગ ટેકનિક}
\begin{tabulary}{\linewidth}{|L|L|}
\hline
\textbf{ટેકનિક} & \textbf{વિગત} \\ \hline
ગેટ ટ્રિગરિંગ & ગેટ-કેથોડ સર્કિટમાં ડાયરેક્ટ કરંટ પલ્સ આપવામાં આવે છે \\ \hline
લાઇટ ટ્રિગરિંગ & જંક્શન પર અથડાતા ફોટોન્સ કન્ડક્શન માટે ઊર્જા આપે છે \\ \hline
\end{tabulary}
\end{center}

\begin{itemize}
    \item \keyword{ગેટ ટ્રિગરિંગ}: ઇલેક્ટ્રિકલ પલ્સનો ઉપયોગ કરતી સૌથી સામાન્ય પદ્ધતિ.
    \item \keyword{લાઇટ ટ્રિગરિંગ}: ફોટોસેન્સિટિવ સેમિકન્ડક્ટર ગુણધર્મોનો ઉપયોગ કરે છે.
\end{itemize}
\end{solutionbox}

\begin{mnemonicbox}
\mnemonic{GET: Gate Electrical Triggering is the most common method.}
\end{mnemonicbox}

\questionmarks{2(બ)}{4}{ફોર્સ્ડ કોમ્યુટેશન અને નેચરલ કોમ્યુટેશન વચ્ચેનો તફાવત લખો.}

\begin{solutionbox}
\begin{center}
\captionof{table}{ફોર્સ્ડ vs નેચરલ કોમ્યુટેશન}
\begin{tabulary}{\linewidth}{|L|L|L|}
\hline
\textbf{પેરામીટર} & \textbf{ફોર્સ્ડ કોમ્યુટેશન} & \textbf{નેચરલ કોમ્યુટેશન} \\ \hline
વ્યાખ્યા & એક્સટર્નલ સર્કિટરી SCRને ફોર્સ કરીને OFF કરે છે & કરંટ હોલ્ડિંગ વેલ્યુથી નીચે જતાં SCR કુદરતી રીતે OFF થાય છે \\ \hline
એપ્લિકેશન & DC સર્કિટ્સ & AC સર્કિટ્સ \\ \hline
કોમ્પોનન્ટ્સ & વધારાના કોમ્પોનન્ટ્સની જરૂર પડે છે (કેપેસિટર, ઇન્ડક્ટર) & કોઈ વધારાના કોમ્પોનન્ટ્સની જરૂર નથી \\ \hline
જટિલતા & જટિલ સર્કિટ ડિઝાઇન & સરળ સર્કિટ ડિઝાઇન \\ \hline
ઊર્જા & ટર્ન-ઓફ માટે બાહ્ય ઊર્જાની જરૂર પડે છે & કોઈ બાહ્ય ઊર્જાની જરૂર નથી \\ \hline
\end{tabulary}
\end{center}

\begin{itemize}
    \item \keyword{ફોર્સ્ડ કોમ્યુટેશન}: બાહ્ય સર્કિટનો ઉપયોગ કરીને SCRને સક્રિયપણે બંધ કરે છે.
    \item \keyword{નેચરલ કોમ્યુટેશન}: જ્યારે AC કરંટ શૂન્ય ક્રોસ કરે છે ત્યારે SCR બંધ થાય છે.
\end{itemize}
\end{solutionbox}

\begin{mnemonicbox}
\mnemonic{FACE: Forced Active Commutation requires External components.}
\end{mnemonicbox}

\questionmarks{2(ક)}{7}{SCR માટે સ્નબર સર્કિટ ડીઝાઈન કરો.}

\begin{solutionbox}
સ્નબર સર્કિટ SCRને ઊંચા $dV/dt$ થી રક્ષણ આપે છે અને વોલ્ટેજ વૃદ્ધિના દરને મર્યાદિત કરે છે.

\begin{center}
\begin{tikzpicture}[auto, node distance=2cm]
    \node (A) [label=above:Anode] {};
    \node (K) [label=below:Cathode, below=3cm of A] {};
    \node [gtu block, minimum width=1cm, minimum height=1.5cm] (SCR) at ($(A)!0.5!(K)$) {SCR};
    \draw (A) -- (SCR);
    \draw (SCR) -- (K);
    \draw (A) -- ++(2,0) coordinate (B);
    \draw (B) to[R, l=$R_s$] ++(0,-1.5) coordinate (C);
    \draw (C) to[C, l=$C_s$] ++(0,-1.5) coordinate (D);
    \draw (D) -- (K);
\end{tikzpicture}
\captionof{figure}{RC સ્નબર સર્કિટ}
\end{center}

\textbf{ડિઝાઇન સ્ટેપ્સ:}
\begin{center}
\begin{tabulary}{\linewidth}{|L|L|}
\hline
\textbf{સ્ટેપ} & \textbf{ગણતરી} \\ \hline
1. $dV/dt$ રેટિંગની ગણતરી કરો & ડેટાશીટમાંથી (V/$\mu$s) \\ \hline
2. R વેલ્યુ નક્કી કરો & $R = V_1/I_L$ જ્યાં $V_1$ એ સપ્લાય વોલ્ટેજ અને $I_L$ એ લોડ કરંટ છે \\ \hline
3. C વેલ્યુ નક્કી કરો & $C = 1/(R \times (dV/dt)_{max})$ \\ \hline
4. RC ટાઇમ કોન્સ્ટન્ટ & $\tau = R \times C$ (SCR ટર્ન-ઓફ ટાઇમ કરતાં વધારે હોવું જોઈએ) \\ \hline
\end{tabulary}
\end{center}

\begin{itemize}
    \item \keyword{રેઝિસ્ટન્સ R}: કેપેસિટરના ડિસ્ચાર્જ કરંટને મર્યાદિત કરે છે.
    \item \keyword{કેપેસિટન્સ C}: ટ્રાન્ઝિયન્ટ એનર્જીને શોષે છે અને $dV/dt$ ને મર્યાદિત કરે છે.
    \item \keyword{પ્રોટેક્શન}: ખોટા ટ્રિગરિંગ અને નુકસાનને રોકે છે.
\end{itemize}
\end{solutionbox}

\begin{mnemonicbox}
\mnemonic{RCSS: Resistance-Capacitance Saves Silicon from Stress.}
\end{mnemonicbox}

\questionmarks{2(અ OR)}{3}{એસ.સી.આર માટેનું ક્લાસ-ઈ કોમ્યુટેશન સમજાવો.}

\begin{solutionbox}
કોમ્યુટેશન એ SCRના એનોડ કરંટને હોલ્ડિંગ કરંટ લેવલથી નીચે ઘટાડીને તેને OFF કરવાની પ્રક્રિયા છે.

\textbf{ક્લાસ-E કોમ્યુટેશન:}
\begin{center}
\begin{tikzpicture}[auto, node distance=2cm]
    \node (S) {Supply};
    \node [gtu block, right=of S] (L) {Load};
    \node [gtu block, right=of L, label=below:Main SCR] (SCR) {SCR};
    \node [gtu block, below=of L] (C) {Capacitor};
    \node [gtu block, below=of C, label=below:Aux SCR] (A) {Aux SCR};
    \draw [thick] (S) -- (L);
    \draw [thick] (L) -- (SCR);
    \draw [thick] (L) -- (C);
    \draw [thick] (C) -- (A);
    \draw [thick] (A) -| (S);
\end{tikzpicture}
\captionof{figure}{ક્લાસ-E કોમ્યુટેશન સર્કિટ (ખ્યાલ)}
\end{center}

\begin{itemize}
    \item \keyword{ઓક્ઝિલરી SCR}: કોમ્યુટેશન પ્રક્રિયાને નિયંત્રિત કરે છે.
    \item \keyword{રેઝોનન્ટ સર્કિટ}: LC રેઝોનન્ટ સર્કિટ બનાવે છે.
    \item \keyword{ઓપરેશન}: ઓક્ઝિલરી SCR મેઇન SCRને રિવર્સ-બાયસ કરવા માટે કેપેસિટર ડિસ્ચાર્જને ટ્રિગર કરે છે.
    \item \keyword{એપ્લિકેશન}: ઇન્વર્ટર અને ચોપરમાં ઉપયોગ થાય છે.
\end{itemize}
\end{solutionbox}

\begin{mnemonicbox}
\mnemonic{ACE: Auxiliary Capacitor Extinguishes conduction.}
\end{mnemonicbox}

\questionmarks{2(બ OR)}{4}{થાઈરિસ્ટરનું ટ્રિગરીંગ વિગતવાર સમજાવો.}

\begin{solutionbox}
\begin{center}
\captionof{table}{થાઇરિસ્ટર ટ્રિગરિંગ પદ્ધતિઓ}
\begin{tabulary}{\linewidth}{|L|L|}
\hline
\textbf{ટ્રિગરિંગ મેથડ} & \textbf{કાર્ય સિદ્ધાંત} \\ \hline
ગેટ ટ્રિગરિંગ & ગેટ અને કેથોડ વચ્ચે ઇલેક્ટ્રિકલ પલ્સ આપવામાં આવે છે \\ \hline
તાપમાન ટ્રિગરિંગ & જંક્શન તાપમાન ટર્ન-ઓન થવા માટે વધે છે \\ \hline
લાઇટ ટ્રિગરિંગ & ફોટોન્સ જંક્શન પર ઇલેક્ટ્રોન-હોલ જોડી બનાવે છે \\ \hline
$dV/dt$ ટ્રિગરિંગ & ઝડપી વોલ્ટેજ વૃદ્ધિ કેપેસિટિવ કરંટ પ્રવાહ થવા માટે કારણભૂત છે \\ \hline
ફોરવર્ડ વોલ્ટેજ ટ્રિગરિંગ & બ્રેકઓવર વોલ્ટેજને વટાવવાથી એવેલાન્ચ કન્ડક્શન થાય છે \\ \hline
\end{tabulary}
\end{center}

\begin{itemize}
    \item \keyword{ગેટ ટ્રિગરિંગ}: સૌથી સામાન્ય અને નિયંત્રિત પદ્ધતિ.
    \item \keyword{પેરામીટર કંટ્રોલ}: પલ્સ પહોળાઈ, એમ્પ્લિટ્યુડ અને રાઈઝ ટાઈમ.
    \item \keyword{ગેટ સેન્સિટિવિટી}: તાપમાન સાથે બદલાય છે.
\end{itemize}
\end{solutionbox}

\begin{mnemonicbox}
\mnemonic{VITAL: Voltage, Illumination, Temperature And Level are all triggering methods.}
\end{mnemonicbox}

\questionmarks{2(ક OR)}{7}{એસ.સી.આર ને ઓવર વૉલ્ટેજ અને ઓવર કરંટ થી બચાવવા માટેની મેથડ વિગતવાર સમજાવો.}

\begin{solutionbox}
\textbf{ઓવરવોલ્ટેજ પ્રોટેક્શન:}
\begin{center}
\captionof{figure}{ઓવરવોલ્ટેજ પ્રોટેક્શન સ્કીમ}
\begin{tikzpicture}[auto, node distance=1.5cm]
    \node (S) {Supply};
    \node [right=of S] (F) {Fuse};
    \node [right=of F] (J) {};
    \node [above=of J] (V) {Varistor};
    \node [right=of J] (SCR) {SCR};
    \node [right=of SCR] (L) {Load};
    \node [below=of J] (RC) {RC Snubber};
    \draw (S) -- (F) -- (J) -- (SCR) -- (L);
    \draw (J) -- (V);
    \draw (J) -- (RC);
\end{tikzpicture}
\end{center}

\begin{center}
\captionof{table}{ઓવરવોલ્ટેજ પ્રોટેક્શન પદ્ધતિઓ}
\begin{tabulary}{\linewidth}{|L|L|}
\hline
\textbf{પ્રોટેક્શન મેથડ} & \textbf{કાર્ય સિદ્ધાંત} \\ \hline
RC સ્નબર સર્કિટ & વોલ્ટેજના ઉછાળાનો દર ($dV/dt$) મર્યાદિત કરે છે \\ \hline
વોલ્ટેજ ક્લેમ્પિંગ & જેનર ડાયોડ અથવા MOVsનો ઉપયોગ કરીને મહત્તમ વોલ્ટેજ મર્યાદિત કરે છે \\ \hline
ક્રોબાર પ્રોટેક્શન & વોલ્ટેજ થ્રેશોલ્ડને વટાવે ત્યારે જાણીજોઈને શોર્ટ-સર્કિટ કરે છે \\ \hline
\end{tabulary}
\end{center}

\textbf{ઓવરકરંટ પ્રોટેક્શન:}
\begin{center}
\captionof{table}{ઓવરકરંટ પ્રોટેક્શન પદ્ધતિઓ}
\begin{tabulary}{\linewidth}{|L|L|}
\hline
\textbf{પ્રોટેક્શન મેથડ} & \textbf{કાર્ય સિદ્ધાંત} \\ \hline
ફ્યુઝ/સર્કિટ બ્રેકર & ફોલ્ટ સ્થિતિઓ દરમિયાન સર્કિટને ડિસ્કનેક્ટ કરે છે \\ \hline
કરંટ લિમિટિંગ રિએક્ટર & ફોલ્ટ કરંટની માત્રા મર્યાદિત કરે છે \\ \hline
ઇલેક્ટ્રોનિક કરંટ લિમિટિંગ & સેન્સિંગ અને કંટ્રોલ સર્કિટ્સ કરંટને મર્યાદિત કરે છે \\ \hline
\end{tabulary}
\end{center}

\begin{itemize}
    \item \keyword{કોઓર્ડિનેશન}: પ્રોટેક્શન ડિવાઇસ સંકલનમાં કામ કરવી જોઈએ.
    \item \keyword{રિસ્પોન્સ ટાઇમ}: અસરકારક સુરક્ષા માટે મહત્વપૂર્ણ છે.
\end{itemize}
\end{solutionbox}

\begin{mnemonicbox}
\mnemonic{SCOPE: Snubbers, Clamps, Overload sensors, Protectors, and Electronic limiters.}
\end{mnemonicbox}

\questionmarks{3(અ)}{3}{સિંગલ ફેઝ રેક્ટિફાયર અને થ્રી ફેઝ રેક્ટિફાયર વચ્ચેનો તફાવત લખો.}

\begin{solutionbox}
\begin{center}
\captionof{table}{સિંગલ ફેઝ vs પોલી ફેઝ રેક્ટિફાયર}
\begin{tabulary}{\linewidth}{|L|L|L|}
\hline
\textbf{પેરામીટર} & \textbf{સિંગલ ફેઝ રેક્ટિફાયર} & \textbf{પોલી ફેઝ રેક્ટિફાયર} \\ \hline
ઇનપુટ & સિંગલ ફેઝ AC સપ્લાય & મલ્ટીપલ ફેઝ (સામાન્ય રીતે 3-ફેઝ) AC સપ્લાય \\ \hline
આઉટપુટ રિપલ & ઊંચી રિપલ સામગ્રી & નીચી રિપલ સામગ્રી \\ \hline
કાર્યક્ષમતા & ઓછી કાર્યક્ષમતા & ઊંચી કાર્યક્ષમતા \\ \hline
પાવર રેટિંગ & ઓછા પાવર એપ્લિકેશન માટે યોગ્ય & ઊંચા પાવર એપ્લિકેશન માટે યોગ્ય \\ \hline
ટ્રાન્સફોર્મર ઉપયોગિતા & ઓછો ઉપયોગિતા ફેક્ટર & ઊંચો ઉપયોગિતા ફેક્ટર \\ \hline
\end{tabulary}
\end{center}

\begin{itemize}
    \item \keyword{રિપલ ફેક્ટર}: સિંગલ ફેઝમાં પોલી ફેઝની તુલનામાં ઊંચી રિપલ હોય છે.
    \item \keyword{ફોર્મ ફેક્ટર}: પોલી ફેઝ સિસ્ટમમાં વધુ સારો.
    \item \keyword{સાઇઝ/વજન}: પોલી ફેઝ સિસ્ટમમાં વધુ સારો પાવર/વજન રેશિયો હોય છે.
\end{itemize}
\end{solutionbox}

\begin{mnemonicbox}
\mnemonic{PERCH: Poly phase has Efficiency, Ripple improvement, Capacity, and Higher ratings.}
\end{mnemonicbox}

\questionmarks{3(બ)}{4}{થ્રી ફેઝ હાફ વેવ રેક્ટિફાયર નો સર્કિટ ડાયગ્રામ દોરી તેની કાર્યપદ્ધતિ સમજાવો.}

\begin{solutionbox}
થ્રી-ફેઝ હાફ-વેવ રેક્ટિફાયર ત્રણ ડાયોડનો ઉપયોગ કરીને થ્રી-ફેઝ ACને પલ્સેટિંગ DCમાં રૂપાંતરિત કરે છે.

\begin{center}
\begin{tikzpicture}[auto, node distance=1.5cm]
    \node (A) {Phase A};
    \node [below=1cm of A] (B) {Phase B};
    \node [below=1cm of B] (C) {Phase C};
    \node [below=1cm of C] (N) {Neutral};
    \node [right=2cm of A, draw, circle] (D1) {D1};
    \node [right=2cm of B, draw, circle] (D2) {D2};
    \node [right=2cm of C, draw, circle] (D3) {D3};
    \draw (A) -- (D1);
    \draw (B) -- (D2);
    \draw (C) -- (D3);
    \node [right=2cm of D2] (O) {Output +};
    \node [right=6.5cm of N] (ON) {Output -};
    \draw (D1) -| (O);
    \draw (D2) -- (O);
    \draw (D3) -| (O);
    \draw (N) -- (ON);
    \draw (O) to[R, l=Load] (ON);
\end{tikzpicture}
\captionof{figure}{3-ફેઝ હાફ વેવ રેક્ટિફાયર}
\end{center}

\textbf{કાર્યપદ્ધતિ:}
\begin{itemize}
    \item દરેક ડાયોડ ત્યારે કન્ડક્ટ કરે છે જ્યારે તેનું ફેઝ વોલ્ટેજ સૌથી વધુ પોઝિટિવ હોય છે.
    \item દરેક ડાયોડનો કન્ડક્શન એંગલ $120^\circ$ છે.
    \item રિપલ ફ્રિક્વન્સી ઇનપુટ ફ્રિક્વન્સીની 3 ગણી છે.
    \item એવરેજ આઉટપુટ વોલ્ટેજ = $3V_m/2\pi$.
    \item રિપલ ફેક્ટર = 0.17 (સિંગલ-ફેઝ હાફ-વેવ કરતાં ઘણો ઓછો).
\end{itemize}
\end{solutionbox}

\begin{mnemonicbox}
\mnemonic{THREE-D: THREE Diodes conducting sequentially.}
\end{mnemonicbox}

\questionmarks{3(ક)}{7}{બ્લોક ડાયાગ્રામની મદદથી યુપીએસ અને એસએમપીએસની કામગીરીનું વર્ણન કરો.}

\begin{solutionbox}
\textbf{UPS (અનઇન્ટેરપ્ટેબલ પાવર સપ્લાય):}
\begin{center}
\begin{tikzpicture}[auto, node distance=1.5cm]
    \node [gtu block] (R) {Rectifier};
    \node [gtu block, right=of R] (B) {Battery};
    \node [gtu block, right=of B] (I) {Inverter};
    \node [gtu block, right=of I] (F) {Filter};
    \node [left=of R] (AC) {AC Input};
    \node [right=of F] (L) {Load};
    \draw [gtu arrow] (AC) -- (R);
    \draw [gtu arrow] (R) -- (B);
    \draw [gtu arrow] (B) -- (I);
    \draw [gtu arrow] (I) -- (F);
    \draw [gtu arrow] (F) -- (L);
    \draw [gtu arrow, dashed] (AC) to[bend left] node {Bypass} (L);
\end{tikzpicture}
\captionof{figure}{UPS બ્લોક ડાયાગ્રામ}
\end{center}

\begin{center}
\captionof{table}{UPS બ્લોક અને કાર્યો}
\begin{tabulary}{\linewidth}{|L|L|}
\hline
\textbf{બ્લોક} & \textbf{કાર્ય} \\ \hline
રેક્ટિફાયર & બેટરી ચાર્જિંગ અને ઇન્વર્ટર માટે ACને DCમાં રૂપાંતરિત કરે છે \\ \hline
બેટરી & પાવર ફેલ્યોર દરમિયાન બેકઅપ માટે ઊર્જા સંગ્રહ કરે છે \\ \hline
ઇન્વર્ટર & લોડને પાવર આપવા માટે DCને ACમાં રૂપાંતરિત કરે છે \\ \hline
ફિલ્ટર & આઉટપુટ વેવફોર્મને સુવ્યવસ્થિત કરે છે \\ \hline
બાયપાસ & મેઇન્ટેનન્સ દરમિયાન ડાયરેક્ટ AC પ્રદાન કરે છે \\ \hline
\end{tabulary}
\end{center}

\textbf{SMPS (સ્વિચ્ડ મોડ પાવર સપ્લાય):}
\begin{center}
\begin{tikzpicture}[auto, node distance=1.2cm]
    \node [gtu block] (RF1) {Rectifier \& Filter};
    \node [gtu block, right=of RF1] (SW) {Switch};
    \node [gtu block, right=of SW] (T) {Transformer};
    \node [gtu block, right=of T] (RF2) {Rectif/Filter};
    \node [below=of SW] (FB) {Feedback};
    \node [left=of RF1] (AC) {AC In};
    \node [right=of RF2] (DC) {DC Out};
    \draw [gtu arrow] (AC) -- (RF1);
    \draw [gtu arrow] (RF1) -- (SW);
    \draw [gtu arrow] (SW) -- (T);
    \draw [gtu arrow] (T) -- (RF2);
    \draw [gtu arrow] (RF2) -- (DC);
    \draw [gtu arrow] (RF2) |- (FB);
    \draw [gtu arrow] (FB) -- (SW);
\end{tikzpicture}
\captionof{figure}{SMPS બ્લોક ડાયાગ્રામ}
\end{center}

\begin{itemize}
    \item \keyword{UPS કાર્યક્ષમતા}: 80-90\%, બેકઅપ પાવર પ્રદાન કરે છે.
    \item \keyword{SMPS કાર્યક્ષમતા}: 70-90\%, લિનિયર સપ્લાય કરતાં ઘણી નાની.
    \item \keyword{નિયમન}: બંને નિયંત્રિત આઉટપુટ વોલ્ટેજ પ્રદાન કરે છે.
\end{itemize}
\end{solutionbox}

\begin{mnemonicbox}
\mnemonic{BRIEF: Battery backup, Rectification, Inversion, Efficient switching, Feedback control.}
\end{mnemonicbox}

\questionmarks{3(અ OR)}{3}{ચોપર સર્કિટના સિદ્ધાંત અને કાર્યને સમજાવો.}

\begin{solutionbox}
ચોપર એ DC-થી-DC કન્વર્ટર છે જે ફિક્સ્ડ DC ઇનપુટ વોલ્ટેજને વેરિએબલ DC આઉટપુટ વોલ્ટેજમાં રૂપાંતરિત કરે છે.

\begin{center}
\begin{tikzpicture}[auto, node distance=2cm]
    \node (DC) {DC Source};
    \node [gtu block, right=of DC] (S) {Switch/SCR};
    \node [gtu block, right=of S] (L) {Load};
    \draw [thick] (DC) -- (S);
    \draw [thick] (S) -- (L);
    \draw [thick] (L.south) -- ++(0,-0.5) -| (DC.south);
\end{tikzpicture}
\captionof{figure}{બેઝિક ચોપર સર્કિટ}
\end{center}

\textbf{સિદ્ધાંત:}
\begin{itemize}
    \item સ્વિચ (સામાન્ય રીતે SCR, MOSFET, અથવા IGBT) ઝડપથી સ્રોતને લોડ સાથે જોડે છે અને અલગ કરે છે.
    \item આઉટપુટ વોલ્ટેજ ડ્યુટી સાયકલ દ્વારા નિયંત્રિત થાય છે (ON સમય / કુલ સમય).
    \item એવરેજ આઉટપુટ વોલ્ટેજ = ઇનપુટ વોલ્ટેજ $\times$ ડ્યુટી સાયકલ.
    \item \keyword{ટાઇમ રેશિયો કંટ્રોલ}: ફ્રિક્વન્સી સ્થિર રાખીને ડ્યુટી સાયકલ બદલે છે.
    \item \keyword{ફ્રિક્વન્સી મોડ્યુલેશન}: ON સમય સ્થિર રાખીને ફ્રિક્વન્સી બદલે છે.
\end{itemize}
\end{solutionbox}

\begin{mnemonicbox}
\mnemonic{CHOP: Control High-speed Operation with Pulses.}
\end{mnemonicbox}

\questionmarks{3(બ OR)}{4}{સિંગલ-ફેઝ અને પોલી-ફેઝ રેક્ટિફાયર સર્કિટની તુલના કરો.}

\begin{solutionbox}
\begin{center}
\captionof{table}{સિંગલ-ફેઝ vs પોલી-ફેઝ રેક્ટિફાયર}
\begin{tabulary}{\linewidth}{|L|L|L|}
\hline
\textbf{પેરામીટર} & \textbf{સિંગલ-ફેઝ રેક્ટિફાયર} & \textbf{પોલી-ફેઝ રેક્ટિફાયર} \\ \hline
સપ્લાય & સિંગલ-ફેઝ AC & ત્રણ અથવા વધુ ફેઝ AC \\ \hline
આઉટપુટ વેવફોર્મ & વધુ પલ્સેટિંગ & સ્મૂધર (ઓછું પલ્સેટિંગ) \\ \hline
રિપલ કન્ટેન્ટ & ઊંચી (ફુલ વેવ માટે 0.48) & નીચી (3-ફેઝ ફુલ વેવ માટે 0.042) \\ \hline
ફિલ્ટરિંગ & વધુ ફિલ્ટરિંગની જરૂર & ઓછા ફિલ્ટરિંગની જરૂર \\ \hline
પાવર હેન્ડલિંગ & મર્યાદિત પાવર હેન્ડલિંગ & ઊંચુ પાવર હેન્ડલિંગ \\ \hline
ટ્રાન્સફોર્મર ઉપયોગિતા & 0.812 (ફુલ વેવ) & 0.955 (3-ફેઝ ફુલ વેવ) \\ \hline
કાર્યક્ષમતા & નીચી & ઊંચી \\ \hline
સાઇઝ & સમાન પાવર માટે નાની & ઊંચા પાવર માટે વધુ કોમ્પેક્ટ \\ \hline
\end{tabulary}
\end{center}

\begin{itemize}
    \item \keyword{હાર્મોનિક કન્ટેન્ટ}: પોલી-ફેઝ સિસ્ટમમાં નીચી.
    \item \keyword{TUF}: પોલી-ફેઝ સિસ્ટમમાં ઊંચી.
    \item \keyword{કોસ્ટ-ઇફેક્ટિવનેસ}: ઊંચા પાવર માટે પોલી-ફેઝ વધુ આર્થિક.
\end{itemize}
\end{solutionbox}

\begin{mnemonicbox}
\mnemonic{PERIPHERY: Poly-phase Efficiency Ripple Improvement Power Handling Economy Rating Yield.}
\end{mnemonicbox}

\questionmarks{3(ક OR)}{7}{બ્લોક ડાયાગ્રામની મદદથી સૌર ફોટોવોલ્ટેઇક (PV) આધારિત પાવર જનરેશનની કામગીરીનું વર્ણન કરો.}

\begin{solutionbox}
સોલર PV પાવર જનરેશન સેમિકન્ડક્ટર મટીરિયલનો ઉપયોગ કરીને સૂર્યપ્રકાશને સીધો ઇલેક્ટ્રિસિટીમાં રૂપાંતરિત કરે છે.

\begin{center}
\begin{tikzpicture}[auto, node distance=1.5cm]
    \node [draw, circle, fill=yellow!20] (Sun) {Sun};
    \node [gtu block, right=of Sun] (PV) {PV Array};
    \node [gtu block, right=of PV] (CC) {Charge Ctrl};
    \node [gtu block, right=of CC] (B) {Battery};
    \node [gtu block, below=of B] (I) {Inverter};
    \node [right=of I] (L) {AC Load};
    \node [below=of CC] (DCL) {DC Load};
    \draw [gtu arrow, dashed] (Sun) -- (PV);
    \draw [gtu arrow] (PV) -- (CC);
    \draw [gtu arrow] (CC) -- (B);
    \draw [gtu arrow] (B) -- (I);
    \draw [gtu arrow] (I) -- (L);
    \draw [gtu arrow] (CC) -- (DCL);
\end{tikzpicture}
\captionof{figure}{સોલર PV પાવર જનરેશન બ્લોક ડાયાગ્રામ}
\end{center}

\begin{center}
\captionof{table}{PV સિસ્ટમ કોમ્પોનન્ટ્સ}
\begin{tabulary}{\linewidth}{|L|L|}
\hline
\textbf{કોમ્પોનન્ટ} & \textbf{કાર્ય} \\ \hline
PV એરે & ફોટોવોલ્ટેઇક ઇફેક્ટ દ્વારા સૌર ઊર્જાને DC ઇલેક્ટ્રિસિટીમાં રૂપાંતરિત કરે છે \\ \hline
ચાર્જ કંટ્રોલર & બેટરી ચાર્જિંગને નિયંત્રિત કરે છે અને ઓવરચાર્જિંગને રોકે છે \\ \hline
બેટરી બેંક & રાત્રે અથવા વાદળી સ્થિતિઓ દરમિયાન ઉપયોગ માટે ઊર્જા સંગ્રહિત કરે છે \\ \hline
ઇન્વર્ટર & AC લોડને પાવર આપવા માટે DCને ACમાં રૂપાંતરિત કરે છે \\ \hline
ગ્રિડ કનેક્શન & વધારાના પાવરને ગ્રિડમાં ફીડ કરવા માટે વૈકલ્પિક કનેક્શન \\ \hline
\end{tabulary}
\end{center}

\begin{itemize}
    \item \keyword{ફોટોવોલ્ટેઇક ઇફેક્ટ}: સૂર્યપ્રકાશના ફોટોન્સ સેમિકન્ડક્ટરમાં ઇલેક્ટ્રોન્સને મુક્ત કરે છે.
    \item \keyword{કાર્યક્ષમતા}: સામાન્ય રીતે કોમર્શિયલ પેનલ માટે 15-22\%.
\end{itemize}
\end{solutionbox}

\begin{mnemonicbox}
\mnemonic{SOLAR: Semiconductors Oriented Light-to-electricity Array Regulation.}
\end{mnemonicbox}

\questionmarks{4(અ)}{3}{સ્ટેટિક સ્વીચના ફાયદા લખો.}

\begin{solutionbox}
\begin{center}
\captionof{table}{સ્ટેટિક સ્વિચના ફાયદા}
\begin{tabulary}{\linewidth}{|L|}
\hline
\textbf{ફીચર્સ} \\ \hline
કોઈ મૂવિંગ પાર્ટ્સ નથી - ઊંચી વિશ્વસનીયતા \\ \hline
સાયલેન્ટ ઓપરેશન \\ \hline
ફાસ્ટ સ્વિચિંગ રિસ્પોન્સ (માઇક્રોસેકન્ડ) \\ \hline
લાંબી ઓપરેશનલ લાઇફ \\ \hline
કોઈ કોન્ટેક્ટ બાઉન્સ અથવા આર્કિંગ નથી \\ \hline
કોમ્પેક્ટ સાઇઝ \\ \hline
ડિજિટલ કંટ્રોલ સિસ્ટમ સાથે સુસંગત \\ \hline
ઓછી મેઇન્ટેનન્સ આવશ્યકતાઓ \\ \hline
\end{tabulary}
\end{center}

\begin{itemize}
    \item \keyword{વિશ્વસનીયતા}: કોઈ મિકેનિકલ ઘસારો નથી.
    \item \keyword{સ્પીડ}: મિકેનિકલ સ્વિચ કરતાં ઘણી ઝડપી.
    \item \keyword{આઇસોલેશન}: ઇલેક્ટ્રિકલ આઇસોલેશન પ્રદાન કરી શકે છે.
\end{itemize}
\end{solutionbox}

\begin{mnemonicbox}
\mnemonic{SAFE: Speed, Arc-free, Fast response, Endurance.}
\end{mnemonicbox}

\questionmarks{4(બ)}{4}{DIAC-TRIAC નો ઉપયોગ કરીને A.C. પાવર કંટ્રોલનો સર્કિટ ડાયાગ્રામ દોરો અને તેને સમજાવો.}

\begin{solutionbox}
DIAC-TRIAC સર્કિટ રેઝિસ્ટિવ અને ઇન્ડક્ટિવ લોડ માટે સ્મૂથ AC પાવર કંટ્રોલ પ્રદાન કરે છે.

\begin{center}
\begin{tikzpicture}[auto, node distance=2cm]
    \node (AC) {AC Supply};
    \node [right=of AC] (L) {Load};
    \node [right=of L, draw] (TRIAC) {TRIAC};
    \node [below=of TRIAC] (G) {Gate};
    \node [left=of G] (DIAC) {DIAC};
    \node [left=of DIAC] (C) {C};
    \node [above=of C] (R2) {Var R};
    \node [above=of R2] (R1) {R1};
    \draw (AC) -- (L) -- (TRIAC.north);
    \draw (TRIAC.south) |- (AC);
    \draw (TRIAC.south) -- (G);
    \draw (G) -- (DIAC);
    \draw (DIAC) -- (C);
    \draw (C) -- (R2) -- (R1) -- (L);
\end{tikzpicture}
\captionof{figure}{DIAC-TRIAC ફેઝ કંટ્રોલ}
\end{center}

\textbf{કાર્યપદ્ધતિ:}
\begin{itemize}
    \item વેરિએબલ રેઝિસ્ટર R2 કેપેસિટર Cના ચાર્જિંગ રેટને નિયંત્રિત કરે છે.
    \item જ્યારે કેપેસિટર વોલ્ટેજ DIAC બ્રેકઓવર વોલ્ટેજ પર પહોંચે છે, ત્યારે DIAC કન્ડક્ટ કરે છે.
    \item DIAC TRIAC ગેટને ટ્રિગર પલ્સ આપે છે.
    \item TRIAC બાકીના હાફ-સાયકલ માટે કન્ડક્ટ કરે છે.
    \item \keyword{ફેઝ કંટ્રોલ}: ફાયરિંગ એંગલ બદલીને પાવર નિયંત્રિત કરે છે.
\end{itemize}
\end{solutionbox}

\begin{mnemonicbox}
\mnemonic{DIRECT: DIAC Initiates Regulated Energy Control in TRIAC.}
\end{mnemonicbox}

\questionmarks{4(ક)}{7}{ટ્રિગરિંગ સર્કિટમાં UJT સાથે SCR નો ઉપયોગ કરીને DC પાવર કંટ્રોલ સર્કિટના કાર્યનું વર્ણન કરો}

\begin{solutionbox}
UJT-ટ્રિગર્ડ SCR સર્કિટ લોડમાં DC પાવરનું ચોક્કસ નિયંત્રણ પ્રદાન કરે છે.

\begin{center}
\begin{tikzpicture}[auto, node distance=2cm]
    \node (DC) {DC};
    \node [right=of DC] (L) {Load};
    \node [right=of L] (SCR) {SCR};
    \node [below=of SCR] (G) {Gate};
    \node [left=of G] (R4) {R4};
    \node [above=of R4] (UJT) {UJT};
    \node [left=of UJT] (C) {C};
    \node [above=of C] (R2) {Var R};
    \draw (DC) -- (L) -- (SCR.north);
    \draw (SCR.south) |- (DC);
    \draw (UJT) -- (R4) -- (G);
\end{tikzpicture}
\captionof{figure}{UJT ટ્રિગરિંગ સર્કિટ (પ્રોટોટાઇપ)}
\end{center}

\begin{center}
\captionof{table}{UJT ટ્રિગરિંગ ઓપરેશન}
\begin{tabulary}{\linewidth}{|L|L|}
\hline
\textbf{સ્ટેજ} & \textbf{ઓપરેશન} \\ \hline
ચાર્જિંગ & R1 અને R2 કેપેસિટર Cના ચાર્જિંગ રેટને નિયંત્રિત કરે છે \\ \hline
UJT ફાયરિંગ & જ્યારે કેપેસિટર વોલ્ટેજ UJT ફાયરિંગ લેવલ પર પહોંચે, ત્યારે UJT કન્ડક્ટ કરે છે \\ \hline
પલ્સ જનરેશન & UJT R4 પર શાર્પ ટ્રિગર પલ્સ જનરેટ કરે છે \\ \hline
SCR ટ્રિગરિંગ & પલ્સ SCR ગેટને ટ્રિગર કરે છે, SCRને ON કરી દે છે \\ \hline
પાવર કંટ્રોલ & વેરિએબલ રેઝિસ્ટર R2 ટાઈમિંગને એડજસ્ટ કરે છે, એવરેજ પાવરને કંટ્રોલ કરે છે \\ \hline
\end{tabulary}
\end{center}

\begin{itemize}
    \item \keyword{ચોક્કસ કંટ્રોલ}: UJT સ્થિર, અનુમાનિત ટ્રિગરિંગ પ્રદાન કરે છે.
    \item \keyword{ફાયદા}: ઓછી કિંમત, ઉચ્ચ વિશ્વસનીયતા, સારી તાપમાન સ્થિરતા.
\end{itemize}
\end{solutionbox}

\begin{mnemonicbox}
\mnemonic{SCRUP: SCR Using Pulse from UJT for Power control.}
\end{mnemonicbox}

\questionmarks{4(અ OR)}{3}{ડાઈ-ઈલેક્ટ્રીક હિટીંગના ઉપયોગો વર્ણવો.}

\begin{solutionbox}
\begin{center}
\captionof{table}{ડાઈલેક્ટ્રિક હિટીંગના ઉપયોગો}
\begin{tabulary}{\linewidth}{|L|}
\hline
\textbf{ઉપયોગો} \\ \hline
પ્લાસ્ટિક વેલ્ડિંગ અને સીલિંગ \\ \hline
લાકડાના ગ્લુઇંગ અને ક્યુરિંગ \\ \hline
ફૂડ પ્રોસેસિંગ (પ્રી-કુકિંગ, ડિફ્રોસ્ટિંગ) \\ \hline
ટેક્સટાઇલ ડ્રાઇંગ અને પ્રોસેસિંગ \\ \hline
પેપર અને બોર્ડ ડ્રાઇંગ \\ \hline
ફાર્માસ્યુટિકલ પ્રોડક્ટ્સ ડ્રાઇંગ \\ \hline
મેડિકલ એપ્લિકેશન (હાઇપરથર્મિયા ટ્રીટમેન્ટ) \\ \hline
રબર વલ્કેનાઇઝેશન \\ \hline
\end{tabulary}
\end{center}

\begin{itemize}
    \item \keyword{મટીરિયલ રિક્વાયરમેન્ટ}: પોલર મોલેક્યુલ્સ ધરાવતા નબળા કન્ડક્ટર્સ સાથે શ્રેષ્ઠ કામ કરે છે.
    \item \keyword{ફ્રિક્વન્સી રેન્જ}: સામાન્ય રીતે 10-100 MHz.
\end{itemize}
\end{solutionbox}

\begin{mnemonicbox}
\mnemonic{POWER: Plastics, Organics, Wood, Edibles, and Rubber processing.}
\end{mnemonicbox}

\questionmarks{4(બ OR)}{4}{ત્રણ તબક્કાના IC555 ટાઈમર સર્કિટ દોરો અને સમજાવો.}

\begin{solutionbox}
ત્રણ-સ્ટેજ IC555 ટાઈમર સર્કિટ સિક્વેન્શિયલ ટાઈમિંગ ઓપરેશન્સ પ્રદાન કરે છે.

\begin{center}
\begin{tikzpicture}[auto, node distance=2cm]
    \node [gtu block] (IC1) {Timer 1};
    \node [gtu block, right=of IC1] (IC2) {Timer 2};
    \node [gtu block, right=of IC2] (IC3) {Timer 3};
    \node [left=of IC1] (Trig) {Trigger};
    \node [below=of IC1] (O1) {Out 1};
    \node [below=of IC2] (O2) {Out 2};
    \node [below=of IC3] (O3) {Out 3};
    \draw [gtu arrow] (Trig) -- (IC1);
    \draw [gtu arrow] (IC1) -- (IC2);
    \draw [gtu arrow] (IC2) -- (IC3);
    \draw [gtu arrow] (IC1) -- (O1);
    \draw [gtu arrow] (IC2) -- (O2);
    \draw [gtu arrow] (IC3) -- (O3);
\end{tikzpicture}
\captionof{figure}{સિક્વેન્શિયલ ટાઈમર બ્લોક ડાયાગ્રામ}
\end{center}

\textbf{કાર્યપદ્ધતિ:}
\begin{itemize}
    \item પ્રથમ ટાઈમર બાહ્ય ટ્રિગર દ્વારા સક્રિય થાય છે.
    \item પ્રથમ ટાઈમરનો આઉટપુટ બીજા ટાઈમરને ટ્રિગર કરે છે.
    \item બીજા ટાઈમરનો આઉટપુટ ત્રીજા ટાઈમરને ટ્રિગર કરે છે.
    \item દરેક ટાઈમર સ્વતંત્ર રીતે એડજસ્ટ કરી શકાય છે.
    \item \keyword{એપ્લિકેશન}: ઔદ્યોગિક સિક્વેન્સિંગ, પ્રોસેસ કંટ્રોલ, એનિમેશન ઇફેક્ટ્સ.
\end{itemize}
\end{solutionbox}

\begin{mnemonicbox}
\mnemonic{THREE-SET: THREE Stage Electronic Timers in sequence.}
\end{mnemonicbox}

\questionmarks{4(ક OR)}{7}{ઇન્ડક્શન હીટિંગના કાર્ય સિદ્ધાંતનું વર્ણન કરો. અને ઇન્ડક્શન હીટિંગના ફાયદાઓ-ગેરફાયદાઓની યાદી બનાવો.}

\begin{solutionbox}
ઇન્ડક્શન હીટિંગ ઇલેક્ટ્રિકલી કન્ડક્ટિવ મટીરિયલ્સને ગરમ કરવા માટે ઇલેક્ટ્રોમેગ્નેટિક ઇન્ડક્શનનો ઉપયોગ કરે છે.

\begin{center}
\begin{tikzpicture}[auto, node distance=1.5cm]
    \node [gtu block] (PS) {Power Supply};
    \node [gtu block, right=of PS] (INV) {Inverter};
    \node [gtu block, right=of INV] (LC) {Matching};
    \node [right=of LC, draw, decorate, decoration={coil}] (WC) {Coil};
    \node [right=of WC] (W) {Workpiece};
    \draw [gtu arrow] (PS) -- (INV);
    \draw [gtu arrow] (INV) -- (LC);
    \draw [gtu arrow] (LC) -- (WC);
\end{tikzpicture}
\captionof{figure}{ઇન્ડક્શન હીટિંગ સિસ્ટમ}
\end{center}

\begin{center}
\captionof{table}{ઇન્ડક્શન હીટિંગના ફાયદા અને ગેરફાયદા}
\begin{tabulary}{\linewidth}{|L|L|}
\hline
\textbf{ફાયદા} & \textbf{ગેરફાયદા} \\ \hline
ઝડપી હીટિંગ & ઊંચી પ્રારંભિક ઉપકરણ કિંમત \\ \hline
ઊર્જા કાર્યક્ષમ (80-90\%) & ઇલેક્ટ્રિકલી કન્ડક્ટિવ મટીરિયલ્સ પૂરતું મર્યાદિત \\ \hline
ચોક્કસ તાપમાન કંટ્રોલ & હાઇ-ફ્રિક્વન્સી પાવર સપ્લાયની જરૂર છે \\ \hline
કોઈ દહન વિના ક્લીન પ્રોસેસ & ચોક્કસ એપ્લિકેશન માટે જટિલ કોઇલ ડિઝાઇન \\ \hline
લોકેલાઇઝ્ડ હીટિંગ શક્ય & ઊંચી પાવર આવશ્યકતાઓ \\ \hline
સુસંગત, પુનરાવર્તનીય પરિણામો & વોટર કૂલિંગ સિસ્ટમની જરૂર છે \\ \hline
પર્યાવરણને અનુકૂળ & ઇલેક્ટ્રોમેગ્નેટિક ઇન્ટરફેરન્સ મુદ્દાઓ \\ \hline
સુધારેલી કાર્ય સ્થિતિઓ & મર્યાદિત પેનિટ્રેશન ડેપ્થ \\ \hline
\end{tabulary}
\end{center}
\end{solutionbox}

\begin{mnemonicbox}
\mnemonic{EDDY: Electromagnetic Device Develops Yield of heat.}
\end{mnemonicbox}

\questionmarks{5(અ)}{3}{ડીસી શન્ટ મોટર સ્પીડને નિયંત્રિત કરવા માટે સોલિડ સ્ટેટ સર્કિટ દોરો અને સમજાવો.}

\begin{solutionbox}
DC શન્ટ મોટર સ્પીડ કંટ્રોલ માટેની સોલિડ-સ્ટેટ સર્કિટ આર્મેચર વોલ્ટેજને કંટ્રોલ કરવા માટે SCRનો ઉપયોગ કરે છે.

\begin{center}
\begin{tikzpicture}[auto, node distance=2cm]
    \node (AC) {AC};
    \node [right=of AC, draw] (BR) {Full Wave Bridge};
    \node [right=of BR] (SCR) {SCR};
    \node [right=of SCR] (A) {Armature};
    \node [below=of A] (F) {Field};
    \draw (AC) -- (BR);
    \draw (BR) -- (SCR) -- (A);
    \draw (BR) |- (F);
\end{tikzpicture}
\captionof{figure}{સોલિડ સ્ટેટ સ્પીડ કંટ્રોલ}
\end{center}

\begin{itemize}
    \item \keyword{આર્મેચર વોલ્ટેજ કંટ્રોલ}: SCR આર્મેચરને વોલ્ટેજ કંટ્રોલ કરે છે.
    \item \keyword{ફિલ્ડ વાઇન્ડિંગ}: સીધો DC સપ્લાયથી જોડાયેલ.
    \item \keyword{સ્પીડ કંટ્રોલ}: SCR ફાયરિંગ એંગલ બદલીને.
\end{itemize}
\end{solutionbox}

\begin{mnemonicbox}
\mnemonic{SAFE: SCR Armature Firing for Efficient control.}
\end{mnemonicbox}

\questionmarks{5(બ)}{4}{સ્ટેપર મોટરના કાર્ય સિદ્ધાંતને સમજાવો.}

\begin{solutionbox}
સ્ટેપર મોટર ઇલેક્ટ્રિકલ પલ્સને ડિસ્ક્રીટ મિકેનિકલ મૂવમેન્ટમાં રૂપાંતરિત કરે છે.

\begin{center}
\begin{tikzpicture}[auto, node distance=1cm]
    \node [draw, circle, minimum size=1.5cm] (R) {Rotor};
    \node [above=of R] (S1) {Stator 1};
    \node [right=of R] (S2) {Stator 2};
    \node [below=of R] (S3) {Stator 3};
    \node [left=of R] (S4) {Stator 4};
    \draw [dashed] (R) -- (S1);
    \draw [dashed] (R) -- (S2);
    \draw [dashed] (R) -- (S3);
    \draw [dashed] (R) -- (S4);
\end{tikzpicture}
\captionof{figure}{સ્ટેપર મોટર કન્સેપ્ચ્યુઅલ ડાયાગ્રામ}
\end{center}

\textbf{કાર્ય સિદ્ધાંત:}
\begin{itemize}
    \item ક્રમમાં સ્ટેટર વાઇન્ડિંગ્સને એનર્જાઇઝ કરવાથી રોટેટિંગ મેગ્નેટિક ફિલ્ડ બને છે.
    \item પર્માનન્ટ મેગ્નેટ રોટર મેગ્નેટિક ફિલ્ડ સાથે એલાઇન થાય છે.
    \item દરેક પલ્સ "સ્ટેપ" એંગલ દ્વારા ચોક્કસ રોટેશન બનાવે છે.
    \item સ્ટેપ એંગલ મોટર કન્સ્ટ્રક્શન દ્વારા નિર્ધારિત થાય છે (સામાન્ય રીતે $1.8^\circ$ અથવા $0.9^\circ$).
\end{itemize}

\begin{center}
\captionof{table}{સ્ટેપર મોટરના પ્રકાર}
\begin{tabulary}{\linewidth}{|L|L|}
\hline
\textbf{પ્રકાર} & \textbf{ખાસિયતો} \\ \hline
વેરિએબલ રિલક્ટન્સ & કોઈ પર્માનન્ટ મેગ્નેટ નથી, મેગ્નેટિક રિલક્ટન્સ પર આધાર રાખે છે \\ \hline
પર્માનન્ટ મેગ્નેટ & પર્માનન્ટ મેગ્નેટ રોટરનો ઉપયોગ કરે છે \\ \hline
હાઇબ્રિડ & બંને પ્રકારની ખાસિયતો સંયોજિત કરે છે \\ \hline
\end{tabulary}
\end{center}
\end{solutionbox}

\begin{mnemonicbox}
\mnemonic{STEP: Sequential Triggering Enables Precise positioning.}
\end{mnemonicbox}

\questionmarks{5(ક)}{7}{PLC નો બ્લોક ડાયાગ્રામ દોરો અને દરેક બ્લોકની કામગીરી સમજાવો.}

\begin{solutionbox}
\begin{center}
\begin{tikzpicture}[auto, node distance=1.5cm]
    \node [gtu block, minimum width=3cm] (CPU) {CPU};
    \node [gtu block, left=of CPU] (IN) {Input Modules};
    \node [gtu block, right=of CPU] (OUT) {Output Modules};
    \node [gtu block, above=of CPU] (PS) {Power Supply};
    \node [gtu block, below=of CPU] (MEM) {Memory};
    \node [gtu block, below=of IN] (PROG) {Program Device};
    \draw [gtu arrow] (IN) -- (CPU);
    \draw [gtu arrow] (CPU) -- (OUT);
    \draw [gtu arrow] (PS) -- (CPU);
    \draw [gtu arrow] (CPU) -- (MEM);
    \draw [gtu arrow] (PROG) -| (CPU);
\end{tikzpicture}
\captionof{figure}{PLC બ્લોક ડાયાગ્રામ}
\end{center}

\begin{center}
\captionof{table}{PLC બ્લોક કાર્યો}
\begin{tabulary}{\linewidth}{|L|L|}
\hline
\textbf{બ્લોક} & \textbf{કાર્ય} \\ \hline
પાવર સપ્લાય & આંતરિક ઉપયોગ માટે મુખ્ય ACને DCમાં રૂપાંતરિત કરે છે \\ \hline
CPU & પ્રોગ્રામ એક્ઝિક્યુટ કરે છે, ડેટા પ્રોસેસ કરે છે, ઓપરેશન્સ મેનેજ કરે છે \\ \hline
ઇનપુટ મોડ્યુલ્સ & સેન્સર, સ્વિચ અને ફિલ્ડ ડિવાઇસ સાથે ઇન્ટરફેસ \\ \hline
આઉટપુટ મોડ્યુલ્સ & એક્ચ્યુએટર, મોટર, વાલ્વ અને ઇન્ડિકેટર કંટ્રોલ કરે છે \\ \hline
મેમરી & પ્રોગ્રામ અને ડેટા સ્ટોર કરે છે (ROM, RAM, EEPROM) \\ \hline
પ્રોગ્રામિંગ ડિવાઇસ & પ્રોગ્રામિંગ માટે એક્સટર્નલ કમ્પ્યુટર અથવા ટર્મિનલ \\ \hline
કમ્યુનિકેશન મોડ્યુલ & અન્ય PLCs, SCADA, HMI સાથે ઇન્ટરફેસ \\ \hline
\end{tabulary}
\end{center}
\end{solutionbox}

\begin{mnemonicbox}
\mnemonic{PILOT: Processing Inputs and Logic for Outputs with Timing control.}
\end{mnemonicbox}

\questionmarks{5(અ OR)}{3}{ડીસી સર્વો મોટરનું બંધારણ દોરો અને સમજાવો.}

\begin{solutionbox}
DC સર્વો મોટર ચોક્કસ પોઝિશન અને સ્પીડ કંટ્રોલ માટે ડિઝાઇન કરવામાં આવે છે.

\begin{center}
\begin{tikzpicture}[auto, node distance=1cm]
    \node [gtu block] (M) {Motor Body};
    \node [gtu block, right=of M] (F) {Feedback (Encoder)};
    \draw [thick] (M) -- (F);
    \draw [thick] (M.west) -- ++(-1,0) node[left] {Shaft};
\end{tikzpicture}
\captionof{figure}{DC સર્વો મોટર}
\end{center}

\textbf{કોમ્પોનન્ટ્સ:}
\begin{itemize}
    \item \keyword{આર્મેચર}: ઝડપી પ્રતિસાદ માટે લો ઇનર્શિયા.
    \item \keyword{ફિલ્ડ સિસ્ટમ}: મેગ્નેટિક ફિલ્ડ પ્રદાન કરે છે.
    \item \keyword{ફીડબેક ડિવાઇસ}: પોઝિશન સેન્સર (એન્કોડર/રિઝોલ્વર/ટેકોમીટર).
    \item \keyword{હાઇ ટોર્ક-ટુ-ઇનર્શિયા રેશિયો}: ઝડપી સ્ટાર્ટ અને સ્ટોપની મંજૂરી આપે છે.
\end{itemize}
\end{solutionbox}

\begin{mnemonicbox}
\mnemonic{SAFE: Sensitive Armature with Feedback for Exactness.}
\end{mnemonicbox}

\questionmarks{5(બ OR)}{4}{ડીસી સીરીઝ મોટરની ઝડપને નિયંત્રિત કરવા માટે સર્કિટ દોરો અને સમજાવો.}

\begin{solutionbox}
SCRનો ઉપયોગ કરીને DC સીરીઝ મોટર સ્પીડ કંટ્રોલ સર્કિટ.

\begin{center}
\begin{tikzpicture}[auto, node distance=2cm]
    \node (AC) {AC Supply};
    \node [right=of AC, draw] (BR) {Bridge Rectifier};
    \node [right=of BR] (SCR) {SCR};
    \node [right=of SCR, draw] (SF) {Series Field};
    \node [right=of SF, draw, circle] (A) {Armature};
    \draw (AC) -- (BR);
    \draw (BR) -- (SCR) -- (SF) -- (A);
    \draw (A) |- (BR);
\end{tikzpicture}
\captionof{figure}{DC સીરીઝ મોટર સ્પીડ કંટ્રોલ}
\end{center}

\begin{itemize}
    \item બ્રિજ રેક્ટિફાયર ACને DCમાં રૂપાંતરિત કરે છે.
    \item SCR મોટરને એવરેજ વોલ્ટેજ કંટ્રોલ કરે છે.
    \item ફાયરિંગ એંગલ પોટેન્શિયોમીટર દ્વારા નિયંત્રિત થાય છે.
    \item સીરીઝ ફિલ્ડ અને આર્મેચર કરંટ સમાન છે.
\end{itemize}
\end{solutionbox}

\begin{mnemonicbox}
\mnemonic{SCRAM: SCR Controls Rectified Armature and Motor speed.}
\end{mnemonicbox}

\questionmarks{5(ક OR)}{7}{સ્ટેપર મોટર નું બંધારણ અને કાર્યપદ્ધતિ સમજાવી તેના ઉપયોગો જણાવો}

\begin{solutionbox}
સ્ટેપર મોટર એ ઇલેક્ટ્રોમેકેનિકલ ડિવાઇસ છે જે ઇલેક્ટ્રિકલ પલ્સને ડિસ્ક્રીટ મિકેનિકલ મૂવમેન્ટમાં રૂપાંતરિત કરે છે.

\textbf{બંધારણ:}
\begin{itemize}
    \item \keyword{સ્ટેટર}: ફેઝમાં ગોઠવાયેલા મલ્ટિપલ કોઇલ વાઇન્ડિંગ્સ ધરાવે છે.
    \item \keyword{રોટર}: પર્માનન્ટ મેગ્નેટ અથવા સોફ્ટ આયર્ન (રિલક્ટન્સ પ્રકાર).
    \item \keyword{બેરિંગ્સ}: શાફ્ટને સપોર્ટ કરે છે અને રોટેશનની મંજૂરી આપે છે.
\end{itemize}

\textbf{ઉપયોગો:}
\begin{itemize}
    \item CNC મશીન અને 3D પ્રિન્ટર્સ.
    \item રોબોટિક્સ અને ઓટોમેશન.
    \item મેડિકલ ઇક્વિપમેન્ટ.
    \item ઓફિસ ઇક્વિપમેન્ટ (પ્રિન્ટર, સ્કેનર).
\end{itemize}
\end{solutionbox}

\begin{mnemonicbox}
\mnemonic{REACT: Rotation Exactly At Controlled Timing.}
\end{mnemonicbox}

\end{document}
