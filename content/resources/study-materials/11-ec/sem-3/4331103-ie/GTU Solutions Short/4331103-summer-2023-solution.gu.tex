\documentclass{article}

% content/resources/templates/preamble.tex
\usepackage[margin=0.6in]{geometry}
\author{Milav Dabgar}
\usepackage{amsmath,amssymb,amsthm}
\usepackage{booktabs}
\usepackage{multirow}
\usepackage{xcolor}
\usepackage{tcolorbox}
\tcbuselibrary{breakable,skins}
\usepackage[colorlinks=true,linkcolor=blue]{hyperref}
\usepackage{titlesec}
\usepackage{enumitem}
\usepackage{tikz}
\usepackage{pgfplots}
\usepackage{circuitikz}
\usepackage[version=4]{mhchem}
\usepackage{longtable}
\usepackage{array}
\usepackage{float}
\usepackage{caption}
\usepackage{listings}

\lstset{
  basicstyle=\small\ttfamily,
  breaklines=true,
  breakatwhitespace=false,
  postbreak=\mbox{\textcolor{red}{$\hookrightarrow$}\space},
  float=false,
  numbers=left,
  numberstyle=\tiny\color{gray},
  numbersep=10pt,
  xleftmargin=2em,
  keywordstyle=\color{blue},
  commentstyle=\color{green!60!black},
  stringstyle=\color{purple},
  backgroundcolor=\color{gray!5},
  showstringspaces=false,
  tabsize=2,
  captionpos=b,
  keepspaces=true,
  columns=flexible
}

\pgfplotsset{compat=1.18}
\usetikzlibrary{shapes,arrows,positioning,calc,patterns,decorations.pathmorphing,decorations.markings,arrows.meta}

% Color scheme
\definecolor{headcolor}{RGB}{0,102,204}
\definecolor{keycolor}{RGB}{220,20,60}
\definecolor{solutioncolor}{RGB}{34,139,34}
\definecolor{mnemoniccolor}{RGB}{148,0,211}
\definecolor{codecolor}{RGB}{0,0,100}

% Spacing
\setlength{\parskip}{3pt}
\setlist[itemize]{nosep}
\setlist[enumerate]{nosep}

% Title formatting
\titleformat{\section}{\Large\bfseries\color{headcolor}}{\thesection}{1em}{}
\titleformat{\subsection}{\large\bfseries\color{headcolor}}{\thesubsection}{1em}{}

% Pandoc tightlist compatibility
\providecommand{\tightlist}{%
  \setlength{\itemsep}{0pt}\setlength{\parskip}{0pt}}

% Pandoc longtable compatibility
\newcounter{none}
\def\thenone{}


% content/resources/templates/gujarati-boxes.tex
\usepackage{fontspec}
\usepackage{polyglossia}

% Set Gujarati as main language (document is primarily in Gujarati)
% Note: gloss-gujarati.ldf doesn't exist in polyglossia, but it will use hyphenation patterns
\setdefaultlanguage{gujarati}
\setotherlanguage{english}

% Configure Gujarati font properly
% Use Language=Default to prevent polyglossia from trying to add language-specific features
% that don't exist for Gujarati, which causes "empty feature" warnings
\newfontfamily\gujaratifont[Script=Gujarati,AutoFakeBold=2.5,AutoFakeSlant=0.3]{Noto Sans Gujarati}
\setmainfont[Script=Gujarati,AutoFakeBold=2.5,AutoFakeSlant=0.3]{Noto Sans Gujarati}
% Use Noto Sans Gujarati for monospace to support Gujarati in text
\setmonofont[Scale=0.9]{Noto Sans Gujarati}

% Configure English to use the same font
\newfontfamily\englishfont[Script=Gujarati,AutoFakeBold=2.5,AutoFakeSlant=0.3]{Noto Sans Gujarati}

% Translations for polyglossia
\gappto\captionsgujarati{
  \renewcommand{\tablename}{કોષ્ટક}
  \renewcommand{\figurename}{આકૃતિ}
}

% Helper for TikZ nodes to ensure Gujarati font
\newcommand{\gu}[1]{{\gujaratifont #1}}

% Custom environments
\newtcolorbox{solutionbox}{
    breakable,
    enhanced,
    colback=solutioncolor!5!white,
    colframe=solutioncolor!75!black,
    fonttitle=\bfseries,
    title=જવાબ
}

\newtcolorbox{solutionboxnobreak}{
 colback=solutioncolor!5!white,
 colframe=solutioncolor!75!black,
 fonttitle=\bfseries,
 title=જવાબ
}

\newtcolorbox{keyformula}{
 breakable,
 enhanced,
 colback=keycolor!5!white,
 colframe=keycolor!75!black,
 fonttitle=\bfseries,
 title=રાસાયણિક સમીકરણ/સૂત્ર
}

\newtcolorbox{mnemonicbox}{
 breakable,
 enhanced,
 colback=mnemoniccolor!5!white,
 colframe=mnemoniccolor!75!black,
 fonttitle=\bfseries,
 title=મેમરી ટ્રીક
}


% Custom commands for GTU solutions
% This file defines semantic commands for consistent formatting

% Question command with automatic formatting
\newcommand{\question}[2]{%
  \section*{Question #1}%
  \textbf{#2}%
}

% OR question variant
\newcommand{\questionor}[2]{%
  \section*{Question #1 OR}%
  \textbf{#2}%
}

% Proper table environment with caption
\newenvironment{answertable}[1]{%
  \begin{table}[htbp]
  \centering
  \caption{#1}
}{%
  \end{table}
}

% Proper figure environment for diagrams
\newenvironment{answerdiagram}[1]{%
  \begin{figure}[htbp]
  \centering
  \caption{#1}
}{%
  \end{figure}
}

% Semantic markup for key terms
\newcommand{\keyword}[1]{\textbf{#1}}
\newcommand{\code}[1]{\texttt{#1}}
\newcommand{\classname}[1]{\texttt{#1}}
\newcommand{\methodname}[1]{\texttt{#1}}

% Proper quotation marks
\newcommand{\mnemonic}[1]{``#1''}


\title{ઇન્ડસ્ટ્રિયલ ઇલેક્ટ્રોનિક્સ (4331103) - ગ્રીષ્મ 2023 સોલ્યુશન}
\date{જુલાઈ 21, 2023}

\begin{document}
\maketitle

\questionmarks{1(a)}{3}{TRAIC ની V-I લાક્ષણિકતા દોરો અને સમજાવો.}

\begin{solutionbox}

TRIAC (ટ્રાયોડ ફોર ઓલ્ટરનેટિંગ કરંટ) એ દ્વિદિશાત્મક ત્રણ-ટર્મિનલ સેમિકન્ડક્ટર ઉપકરણ છે જે ટ્રિગર થાય ત્યારે કોઈપણ દિશામાં વિદ્યુત પ્રવાહ પસાર કરી શકે છે.

\textbf{આકૃતિ:}

\begin{center}
\begin{tikzpicture}
    % Axes
    \draw[->] (-4,0) -- (4,0) node[right] {$V$};
    \draw[->] (0,-4) -- (0,4) node[above] {$I$};
    
    % Quadrant Labels
    \node at (2,2) {Quadrant I};
    \node at (-2,2) {Quadrant II};
    \node at (-2,-2) {Quadrant III};
    \node at (2,-2) {Quadrant IV};
    
    % Curve in Q1
    \draw[thick, blue] (0,0) -- (1,0.2) to[out=10,in=260] (2.5,3);
    \draw[thick, blue, dashed] (0,0) -- (2.5,0.5) -- (2.5,3); % Leakage and breakdown
    \node[right] at (2.5,3) {કન્ડક્શન};
    \node[below] at (2.5,0) {$V_{BO}$};
    
    % Curve in Q3
    \draw[thick, blue] (0,0) -- (-1,-0.2) to[out=190,in=80] (-2.5,-3);
    \draw[thick, blue, dashed] (0,0) -- (-2.5,-0.5) -- (-2.5,-3);
    \node[left] at (-2.5,-3) {કન્ડક્શન};
    \node[above] at (-2.5,0) {$-V_{BO}$};
    
    % Holding Current
    \draw[dotted] (-2.5, -0.5) -- (2.5, 0.5);
    \node[right, font=\footnotesize] at (0, 0.5) {$I_H$};

\end{tikzpicture}
\captionof{figure}{TRIAC ની V-I લાક્ષણિકતા}
\end{center}

\begin{itemize}
    \item \keyword{દ્વિદિશાત્મક કાર્યપદ્ધતિ}: TRIAC બંને દિશામાં વીજપ્રવાહ પસાર કરે છે (પોઝિટિવ અને નેગેટિવ હાફ સાયકલ્સ)
    \item \keyword{ક્વોડ્રન્ટ ઓપરેશન}: MT2 અને ગેટની ધ્રુવતા પર આધારિત તમામ ચાર ક્વોડ્રન્ટમાં કામ કરે છે
    \item \keyword{ટ્રિગરિંગ વોલ્ટેજ}: કોઈપણ દિશામાં $\pm V_{BO}$ ખાતે બ્રેકડાઉન થાય છે
    \item \keyword{હોલ્ડિંગ કરંટ}: કન્ડક્શન જાળવી રાખવા માટે ન્યૂનતમ વિદ્યુત પ્રવાહ
\end{itemize}

\end{solutionbox}
\mnemonicbox{ટુ રેક્ટિફાયર્સ ઇન અ કેસ}

\questionmarks{1(b)}{4}{બે ટ્રાણઝિસ્ટ્ર સામ્યતાનો ઉપયોગ કરીને SCR નું કાર્ય સમજાવો.}

\begin{solutionbox}

SCR (સિલિકોન કંટ્રોલ્ડ રેક્ટિફાયર) ને ઇન્ટરકનેક્ટેડ PNP અને NPN ટ્રાન્ઝિસ્ટર તરીકે રજૂ કરી શકાય છે.

\textbf{આકૃતિ:}

\begin{center}
\begin{tikzpicture}[gtu block/.style={draw, rectangle, minimum width=2cm, minimum height=1cm, align=center}]
    % Transistors representation
    \node (anode) at (0,4) {Anode};
    \node (cathode) at (0,-4) {Cathode};
    
    % PNP Transistor Q1
    \draw (0,2) node[pnp, rotate=-90] (Q1) {};
    \node[right] at (Q1) {$Q_1$ (PNP)};
    
    % NPN Transistor Q2
    \draw (0,-2) node[npn, rotate=-90] (Q2) {};
    \node[right] at (Q2) {$Q_2$ (NPN)};
    
    % Connections
    \draw (anode) -- (Q1.E);
    \draw (Q1.C) -- (Q2.B);
    \draw (Q2.C) -- (Q1.B); % Feedback loop
    \draw (Q2.E) -- (cathode);
    
    % Gate
    \draw (Q2.B) -- ++(-1.5,0) node[left] {Gate};

\end{tikzpicture}
\captionof{figure}{SCR ની બે ટ્રાન્ઝિસ્ટર સામ્યતા}
\end{center}

\begin{itemize}
    \item \keyword{બે-ટ્રાન્ઝિસ્ટર સ્ટ્રક્ચર}: PNP ($Q_1$) અને NPN ($Q_2$) એવી રીતે જોડાયેલા છે કે દરેક ટ્રાન્ઝિસ્ટરનો કલેક્ટર બીજાના બેઝને ડ્રાઇવ કરે છે
    \item \keyword{રિજનરેટિવ ફીડબેક}: એકવાર બંને ટ્રાન્ઝિસ્ટર કન્ડક્ટ કરવાનું શરૂ કરે, તેઓ એકબીજાને સેચુરેશનમાં રાખે છે
    \item \keyword{ટ્રિગરિંગ}: $Q_2$ બેઝમાં ગેટ કરંટ લાગુ કરવાથી રિજનરેટિવ પ્રક્રિયા શરૂ થાય છે
    \item \keyword{લેચિંગ}: એકવાર ટ્રિગર થયા પછી, ગેટ સિગ્નલ દૂર કરવામાં આવે તો પણ SCR ON રહે છે
\end{itemize}

\end{solutionbox}
\mnemonicbox{પુલ નીટ પાથ}

\questionmarks{1(c)}{7}{LDR નો ઉપયોગ કરીને ફોટો ઇલેક્ટ્રિક રિલેનો સર્કિટ ડાયાગ્રામ દોરો અને તેને કાર્યકારી સમજાવો.}

\begin{solutionbox}

LDR (લાઇટ ડિપેન્ડન્ટ રેઝિસ્ટર)નો ઉપયોગ કરતું ફોટોઇલેક્ટ્રિક રિલે એ પ્રકાશ-સક્રિય સ્વિચિંગ સર્કિટ છે.

\textbf{સર્કિટ ડાયાગ્રામ:}

\begin{center}
\begin{tikzpicture}
    % Power rails
    \draw (-2,4) -- (4,4) node[right] {$+V_{CC}$};
    \draw (-2,0) -- (4,0) node[right] {GND};
    
    % Potential Divider R1 and LDR
    \draw (0,4) to[R, l=$R_1$] (0,2) to[photoresistor, l=LDR] (0,0);
    
    % Transistor
    \draw (2,1) node[npn] (Q1) {};
    \draw (0,2) -- (Q1.B);
    \draw (Q1.E) -- (2,0);
    
    % Relay Coil
    \draw (2,4) to[L, l=Relay, name=L1] (Q1.C);
    
    % Diode across Relay
    \draw (2.5, 3.5) node[diode, rotate=90] (D1) {}; 
    \draw (2,3.8) -- (2.5, 3.8) -- (D1);
    \draw (2.5, 1.2) -- (D1);

\end{tikzpicture}
\captionof{figure}{ફોટો ઇલેક્ટ્રિક રિલે સર્કિટ}
\end{center}

\begin{itemize}
    \item \keyword{પ્રકાશ સેન્સિંગ}: પ્રકાશની હાજરીમાં LDR રેઝિસ્ટન્સ ઘટે છે
    \item \keyword{ટ્રાન્ઝિસ્ટર ઓપરેશન}: જ્યારે LDR પર પ્રકાશ પડે છે, ત્યારે ટ્રાન્ઝિસ્ટર બેઝ પરનું વોલ્ટેજ બદલાય છે
    \item \keyword{રિલે સ્વિચિંગ}: ટ્રાન્ઝિસ્ટર પ્રકાશના આધારે કન્ડક્ટ/કટ ઓફ થાય છે, જેથી રિલે સક્રિય/નિષ્ક્રિય થાય છે
    \item \keyword{થ્રેશોલ્ડ એડજસ્ટમેન્ટ}: પોટેન્શિયોમીટર $R_1$ પ્રકાશ સંવેદનશીલતા સેટ કરે છે
    \item \keyword{એપ્લિકેશન્સ}: ઓટોમેટિક સ્ટ્રીટ લાઇટ્સ, ચોર-અલાર્મ, ઓટોમેટિક ડોર ઓપનર
\end{itemize}

\end{solutionbox}
\mnemonicbox{લાઇટ ડિટેક્ટ્સ રેડિલી}

\questionmarks{1(c OR)}{7}{SCR માટે UJT નો ઉપયોગ કરીને ગેટ પલ્સ ટ્રિગર સર્કિટ દોરો અને તેનું કાર્ય સમજાવો.}

\begin{solutionbox}

UJT (યુનિજંક્શન ટ્રાન્ઝિસ્ટર) SCR માટે વિશ્વસનીય ટ્રિગર પલ્સ પ્રદાન કરે છે.

\textbf{સર્કિટ ડાયાગ્રામ:}

\begin{center}
\begin{tikzpicture}
    % Supply
    \draw (-1,5) -- (5,5) node[right] {$+V_{CC}$};
    \draw (-1,0) -- (5,0) node[right] {GND};
    
    % RC charging
    \draw (0,5) to[vR, l=$R_1$] (0,2.5) to[C, l=$C$] (0,0);
    
    % UJT
    \draw (2,2.5) node[ujt] (U1) {};
    \draw (0,2.5) -- (U1.E);
    \draw (U1.B2) to[R, l=$R_2$] (2,5);
    \draw (U1.B1) to[R, l=$R_3$] (2,0);
    
    % SCR Triggering
    \draw (4,0) node[thyristor] (S1) {};
    \draw (U1.B1) -- (S1); 
    \draw (2,0.5) -- (3,0.5) -- (3,0.2) node[below] {To SCR Gate};

\end{tikzpicture}
\captionof{figure}{SCR માટે UJT ટ્રિગર સર્કિટ}
\end{center}

\begin{itemize}
    \item \keyword{RC ટાઇમિંગ}: $R_1$ અને $C$ ચાર્જિંગ સર્કિટ બનાવે છે જે પલ્સ ફ્રિક્વન્સી નક્કી કરે છે
    \item \keyword{UJT ઓપરેશન}: કેપેસિટર વોલ્ટેજ પીક પોઇન્ટ વોલ્ટેજમાં પહોંચે ત્યારે UJT ફાયર થાય છે
    \item \keyword{પલ્સ જનરેશન}: UJT કેપેસિટરને ડિસ્ચાર્જ કરે છે જેથી તીવ્ર ટ્રિગર પલ્સ પેદા થાય છે
    \item \keyword{SCR ટ્રિગરિંગ}: AC સાયકલમાં ચોક્કસ બિંદુઓએ SCR ચાલુ કરવા માટે પલ્સ ગેટ પર લાગુ કરવામાં આવે છે
    \item \keyword{ફ્રિક્વન્સી કંટ્રોલ}: ફેઝ કંટ્રોલ માટે $R_1$ બદલવાથી પલ્સ ફ્રિક્વન્સી બદલાય છે
\end{itemize}

\end{solutionbox}
\mnemonicbox{યુનિફોર્મ જંક્શન્સ ટ્રિગર}

\questionmarks{2(a)}{3}{SCR ની ટ્રિગરિંગ પદ્ધતિઓ સમજાવો.}

\begin{solutionbox}

\begin{center}

\begin{tabulary}{\linewidth}{|L|L|L|}
\hline \textbf{ટ્રિગરિંગ પદ્ધતિ} & \textbf{કાર્ય સિદ્ધાંત} & \textbf{ફાયદા} \\ \hline
ગેટ ટ્રિગરિંગ & ગેટ ટર્મિનલ પર વિદ્યુત પ્રવાહ લાગુ & સૌથી સામાન્ય, ચોક્કસ નિયંત્રણ \\
થર્મલ ટ્રિગરિંગ & તાપમાન વધવાથી લીકેજ થાય છે & સરળ, કોઈ બાહ્ય સર્કિટ નથી \\
લાઇટ ટ્રિગરિંગ & ફોટોન્સ ઇલેક્ટ્રોન-હોલ જોડી બનાવે છે & ઇલેક્ટ્રિકલ આઇસોલેશન, LASCR માં વપરાય છે \\
dv/dt ટ્રિગરિંગ & ઝડપી વોલ્ટેજ વૃદ્ધિ ટર્ન-ઓન થવાનું કારણ બને છે & પ્રોટેક્શન સર્કિટ માટે ઉપયોગી \\
ફોરવર્ડ વોલ્ટેજ ટ્રિગરિંગ & બ્રેકઓવર વોલ્ટેજ વટાવવાથી & કોઈ ગેટ કનેક્શનની જરૂર નથી \\
\end{tabulary}
\captionof{table}{SCR ટ્રિગરિંગ પદ્ધતિઓ}
\end{center}

\end{solutionbox}
\mnemonicbox{ગુડ ટ્રિગર્સ લેટ ડિવાઇસેસ ફાયર}

\questionmarks{2(b)}{4}{SCR નું કમ્યુટેશન શું છે? વર્ગ-E કમ્યુટેશન સમજાવો.}

\begin{solutionbox}

કમ્યુટેશન એ SCR ના એનોડ કરંટને હોલ્ડિંગ કરંટથી નીચે ઘટાડીને તેને બંધ કરવાની પ્રક્રિયા છે.

\textbf{ક્લાસ-E કમ્યુટેશન (કોમ્પ્લિમેન્ટરી કમ્યુટેશન):}

\begin{center}
\begin{tikzpicture}
    % Two SCRs
    \draw (0,4) -- (0,3) node[thyristor] (S1) {} -- (0,0);
    \draw (4,4) -- (4,3) node[thyristor] (S2) {} -- (4,0);
    
    % Commutating Capacitor
    \draw (0,2) -- (2,2) to[C, l=$C$] (2,2) -- (4,2);
    
    % Loads
    \draw (0,0) to[R, l=$R_{L1}$] (0,-2) -- (2,-2);
    \draw (4,0) to[R, l=$R_{L2}$] (4,-2) -- (2,-2);
    \draw (2,-2) -- (2,-3) node[ground] {};
    
    % Supply
    \draw (0,4) -- (2,4) -- (2,5) node[above] {$+V_{DC}$};
    \draw (4,4) -- (2,4);

\end{tikzpicture}
\captionof{figure}{ક્લાસ-E કમ્યુટેશન સર્કિટ}
\end{center}

\begin{itemize}
    \item \keyword{કોમ્પ્લિમેન્ટરી સ્વિચિંગ}: વિરુદ્ધ હાફ-સાયકલમાં બીજા SCR નો ઉપયોગ કરે છે
    \item \keyword{નેચરલ કમ્યુટેશન}: AC સ્ત્રોત ઝીરો ક્રોસ કરે ત્યારે, એનોડ કરંટ હોલ્ડિંગ કરંટ કરતાં નીચે પડે છે
    \item \keyword{એપ્લિકેશન}: AC પાવર કંટ્રોલ સર્કિટ્સ, સાયક્લોકન્વર્ટર્સ
    \item \keyword{ફાયદો}: કોઈ વધારાના કમ્યુટેશન ઘટકોની આવશ્યકતા નથી
\end{itemize}

\end{solutionbox}
\mnemonicbox{કોમ્પ્લિમેન્ટરી એલિમેન્ટ્સ}

\questionmarks{2(c)}{7}{SCR માટે સ્નબર સર્કિટ દોરો અને સમજાવો.}

\begin{solutionbox}

સ્નબર સર્કિટ SCR ને વોલ્ટેજ ટ્રાન્ઝિયન્ટ્સ અને dv/dt ટર્ન-ઓનથી રક્ષણ આપે છે.

\textbf{સર્કિટ ડાયાગ્રામ:}

\begin{center}
\begin{tikzpicture}
    % SCR
    \draw (0,3) node[thyristor, rotate=-90] (S1) {};
    \node at (0,3.5) {SCR};
    
    % Snubber circuit parallel to SCR
    \draw (S1) -- (0,4) -- (2,4) to[R, l=$R_S$] (2,3) to[C, l=$C_S$] (2,1) -- (2,0) -- (S1);
    
    % Load and Source
    \draw (-2,4) to[sV, l=AC] (-2,0) -- (S1);
    \draw (-2,4) -- (S1);

\end{tikzpicture}
\captionof{figure}{SCR માટે સ્નબર સર્કિટ}
\end{center}

\begin{itemize}
    \item \keyword{RC નેટવર્ક}: SCR પર શ્રેણીબદ્ધ રેસિસ્ટર ($R_S$) અને કેપેસિટર ($C_S$) જોડાયેલા છે
    \item \keyword{ટ્રાન્ઝિયન્ટ સપ્રેશન}: કેપેસિટર વોલ્ટેજ સ્પાઇક્સને અવશોષિત કરે છે જે SCR ને નુકસાન પહોંચાડી શકે છે
    \item \keyword{dv/dt પ્રોટેક્શન}: ઝડપી વોલ્ટેજ વધારાને કારણે ખોટા ટ્રિગરિંગને અટકાવે છે
    \item \keyword{ટર્ન-ઓફ આસિસ્ટન્સ}: વૈકલ્પિક કરંટ પાથ પ્રદાન કરીને કમ્યુટેશનમાં મદદ કરે છે
    \item \keyword{કમ્પોનન્ટ પસંદગી}: $C_S$ લોડ કરંટ પર આધારિત, $R_S$ ડિસ્ચાર્જ કરંટને મર્યાદિત કરે છે
\end{itemize}

\end{solutionbox}
\mnemonicbox{સેફલી ન્યુટ્રલાઇઝીસ અનવોન્ટેડ બ્રેકઓવર}

\questionmarks{2(a OR)}{3}{SCR ની વર્તમાન સંરક્ષણ પદ્ધતિ વિશે સમજાવો.}

\begin{solutionbox}

\begin{center}

\begin{tabulary}{\linewidth}{|L|L|L|}
\hline \textbf{સંરક્ષણ પદ્ધતિ} & \textbf{કાર્ય સિદ્ધાંત} & \textbf{એપ્લિકેશન્સ} \\ \hline
ફ્યુઝ & કરંટ રેટિંગ વટાવે ત્યારે પીગળે છે & સરળ, આર્થિક સંરક્ષણ \\
સર્કિટ બ્રેકર & ઓવરલોડ પર ટ્રિપ થાય છે, રીસેટ કરી શકાય છે & ફરીથી ઉપયોગ કરી શકાય તેવું સંરક્ષણ \\
કરંટ લિમિટિંગ રિએક્ટર & ફોલ્ટ કરંટ મેગ્નિટ્યુડને મર્યાદિત કરે છે & ઔદ્યોગિક પાવર કંટ્રોલ \\
ઇલેક્ટ્રોનિક કરંટ લિમિટિંગ & કરંટને સેન્સ કરે છે અને ગેટને નિયંત્રિત કરે છે & ચોક્કસ સંરક્ષણ \\
ક્રોબાર સર્કિટ & ઓવરલોડ પર પાવર સપ્લાય શોર્ટ કરે છે & સંવેદનશીલ લોડ્સનું રક્ષણ કરે છે \\
\end{tabulary}
\captionof{table}{SCR સંરક્ષણ પદ્ધતિઓ}
\end{center}

\end{solutionbox}
\mnemonicbox{ફોલ્ટ કરંટ કોઝીસ ઇક્વિપમેન્ટ ડેમેજ}

\questionmarks{2(b OR)}{4}{ઓપ્ટો-એસસીઆરની કામગીરી સમજાવો.}

\begin{solutionbox}

ઓપ્ટો-SCR (અથવા લાઇટ એક્ટિવેટેડ SCR) એક આઇસોલેટેડ પેકેજમાં લાઇટ સોર્સ અને SCR ને જોડે છે.

\textbf{આકૃતિ:}

\begin{center}
\begin{tikzpicture}
    % Package Box
    \draw[dashed] (0,0) rectangle (5,3);
    
    % LED input
    \draw (0.5,1.5) to[led, l=LED] (0.5,0.5); 
    \draw (0.5, 2.5) -- (0.5, 1.5);
    \node[left] at (0.5, 2.5) {Anode};
    \node[left] at (0.5, 0.5) {Cathode};

    % Light arrows
    \draw[->, wave] (1.5, 1.5) -- (3, 1.5);
    
    % SCR output
    \draw (4,1.5) node[thyristor, rotate=-90] (S1) {};
    \draw (4, 2.5) -- (S1);
    \draw (4, 0.5) -- (S1);
    \node[right] at (4, 2.5) {Anode};
    \node[right] at (4, 0.5) {Cathode};
    \draw (S1) -- (3.5, 1.5) node[left] {Gate};
\end{tikzpicture}
\captionof{figure}{ઓપ્ટો-SCR સ્ટ્રક્ચર}
\end{center}

\begin{itemize}
    \item \keyword{ઇલેક્ટ્રિકલ આઇસોલેશન}: LED ઇલેક્ટ્રિકલ કનેક્શન વિના ઓપ્ટિકલી SCR ને ટ્રિગર કરે છે
    \item \keyword{નોઇઝ ઇમ્યુનિટી}: ઇલેક્ટ્રિકલ નોઇઝ અને ઇન્ટરફેરન્સથી રક્ષિત
    \item \keyword{હાઇ-વોલ્ટેજ આઇસોલેશન}: કંટ્રોલ અને પાવર સર્કિટ્સને અલગ કરે છે
    \item \keyword{એપ્લિકેશન્સ}: ઔદ્યોગિક નિયંત્રણ, હાઇ-વોલ્ટેજ સ્વિચિંગ
\end{itemize}

\end{solutionbox}
\mnemonicbox{લાઇટ એક્ટિવેટ્સ સિલિકોન કંટ્રોલ}

\questionmarks{2(c OR)}{7}{ફોર્સ કમ્યુટેશન શું છે? કોઈપણ બે સમજાવો.}

\begin{solutionbox}

ફોર્સ કમ્યુટેશન એ SCR ના એનોડ કરંટને હોલ્ડિંગ લેવલથી નીચે ઘટાડીને કૃત્રિમ રીતે બંધ કરવાની પ્રક્રિયા છે.

\textbf{1. ક્લાસ A કમ્યુટેશન (સેલ્ફ-કમ્યુટેશન):}

\begin{center}
\begin{tikzpicture}
    % Load resonant
    \draw (0,0) -- (0,3) to[L, l=$L$] (2,3) -- (2,2) to[C, l=$C$] (2,0) -- (0,0);
    \draw (2,3) -- (4,3) node[thyristor, rotate=-90] (S1) {} -- (4,0) -- (2,0);
    
    % DC Source
    \draw (-2,3) to[battery] (-2,0) -- (0,0);
    \draw (-2,3) -- (0,3);

\end{tikzpicture}
\captionof{figure}{ક્લાસ A કમ્યુટેશન}
\end{center}

\begin{itemize}
    \item \keyword{LC રેઝોનન્ટ સર્કિટ}: SCR ની આસપાસ સમાંતર L-C દોલનો પેદા કરે છે
    \item \keyword{રિવર્સ કરંટ}: L-C સર્કિટ SCR દ્વારા રિવર્સ કરંટને દબાણ આપે છે
    \item \keyword{એપ્લિકેશન્સ}: ઇન્વર્ટર્સ, ચોપર્સ
\end{itemize}

\textbf{2. ક્લાસ B કમ્યુટેશન (રેઝોનન્ટ પલ્સ કમ્યુટેશન):}

\begin{center}
\begin{tikzpicture}
    % Main SCR path
    \draw (0,4) to[L, l=$L$] (2,4) -- (2,3) node[thyristor] (S1) {} -- (2,0);
    
    % Commutating path
    \draw (2,4) -- (4,4) to[C, l=$C$] (4,2) -- (2,2);
    
    % Source
    \draw (0,4) node[left] {$+V_s$};
    \draw (0,0) node[left] {GND};

\end{tikzpicture}
\captionof{figure}{ક્લાસ B કમ્યુટેશન}
\end{center}

\begin{itemize}
    \item \keyword{એક્સટર્નલ સ્વિચ}: વધારાનો SCR અથવા સ્વિચ કમ્યુટેશનને ટ્રિગર કરે છે
    \item \keyword{એનર્જી સ્ટોરેજ}: L-C સર્કિટ ઊર્જાને સંગ્રહિત કરે છે પછી SCR કરંટને રિવર્સ કરે છે
    \item \keyword{એપ્લિકેશન્સ}: DC ચોપર્સ, કંટ્રોલ્ડ રેક્ટિફાયર્સ
\end{itemize}

\end{solutionbox}
\mnemonicbox{ફોર્સ સર્કિટ રિવર્સલ}

\questionmarks{3(a)}{3}{ચાર ડાયોડનો ઉપયોગ કરીને 1-Phase ફુલ વેવ બ્રિજ કોન્ટ્રોલએદ રેક્ટિફાયર સમજાવો.}

\begin{solutionbox}

આ સર્કિટ કંટ્રોલ્ડ સિંગલ-ફેઝ ફુલ-વેવ રેક્ટિફિકેશન માટે ડાયોડ્સ અને SCR ને જોડે છે.

\textbf{સર્કિટ ડાયાગ્રામ:}

\begin{center}
\begin{tikzpicture}
    % Diode Bridge with Series SCR as per Q description interpretation
    % Or simple bridge with one SCR leg. But text says "using four diodes AND one SCR".
    % This implies diode bridge feeding series SCR or similar. Let's stick to English diagram logic.
    
    % Diode Bridge
    \draw (-2,0) to[D] (0,2);
    \draw (0,2) to[D] (2,0);
    \draw (-2,0) to[D] (0,-2);
    \draw (0,-2) to[D] (2,0);
    
    % AC Input
    \draw (-3,0) to[sV, l=$V_{in}$] (-2,0);
    \draw (2,0) -- (3,0); 
    
    % SCR in series with Load
    \draw (0,2) -- (0,3) node[thyristor, rotate=-90] (S2) {} -- (0,4);
    \draw (0,-2) -- (0,-4);
    
    % Load
    \draw (3,3) to[R, l=Load] (3,-3);
    \draw (0,4) -| (3,3);
    \draw (0,-4) -| (3,-3);

\end{tikzpicture}
\captionof{figure}{1-Phase ફુલ વેવ રેક્ટિફાયર (4 ડાયોડ્સ અને 1 SCR)}
\end{center}

\begin{itemize}
    \item \keyword{બ્રિજ કોન્ફિગરેશન}: ચાર ડાયોડ્સ બ્રિજમાં ગોઠવવામાં આવ્યા છે જેમાંથી એક SCR દ્વારા બદલાયેલ છે અથવા શ્રેણીમાં છે
    \item \keyword{વેરિએબલ આઉટપુટ}: SCR કન્ડક્શન એંગલ અને તેથી આઉટપુટ વોલ્ટેજને નિયંત્રિત કરે છે
    \item \keyword{આર્થિક ડિઝાઇન}: બે અથવા ચારને બદલે માત્ર એક SCR વાપરે છે
    \item \keyword{કાર્યક્ષમતા}: હાફ-વેવ કંટ્રોલ્ડ રેક્ટિફાયર કરતાં વધુ
\end{itemize}

\end{solutionbox}
\mnemonicbox{બ્લેન્ડ ડાયોડ્સ સ્માર્ટલી}

\questionmarks{3(b)}{4}{ચોપર શું છે? તેની ઉપયોગો જણાવો.}

\begin{solutionbox}

\begin{center}

\begin{tabulary}{\linewidth}{|L|L|}
\hline \textbf{પાસા} & \textbf{વર્ણન} \\ \hline
વ્યાખ્યા & DC-DC કન્વર્ટર જે ફિક્સ્ડ DC ઇનપુટને વેરિએબલ DC આઉટપુટમાં રૂપાંતરિત કરે છે \\
કાર્ય સિદ્ધાંત & પીરિયોડિકલી ઉચ્ચ આવૃત્તિએ DC ઇનપુટને ચાલુ/બંધ કરે છે \\
પ્રકારો & સ્ટેપ-ડાઉન (બક), સ્ટેપ-અપ (બૂસ્ટ), બક-બૂસ્ટ, ક્યુક \\
કંટ્રોલ મેથડ્સ & PWM, ફ્રિક્વન્સી મોડ્યુલેશન, કરંટ-લિમિટ કંટ્રોલ \\
એપ્લિકેશન્સ & DC મોટર સ્પીડ કંટ્રોલ, બેટરી ચાર્જર્સ, UPS, સોલાર સિસ્ટમ્સ, ઇલેક્ટ્રિક વાહનો \\
\end{tabulary}
\captionof{table}{ચોપરના મૂળભૂત સિદ્ધાંતો અને ઉપયોગો}
\end{center}

\end{solutionbox}
\mnemonicbox{ચોપ્સ કરંટ પરફેક્ટલી}

\questionmarks{3(c)}{7}{1-Phase A.C. લોડ માટે SCR નો ઉપયોગ કરીને સ્ટેટિક સ્વીચના સર્કિટ ડાયાગ્રામ દોરો અને સમજાવો.}

\begin{solutionbox}

SCR નો ઉપયોગ કરતું સ્ટેટિક સ્વિચ AC લોડ્સ માટે નોન-મિકેનિકલ સ્વિચિંગ પ્રદાન કરે છે.

\textbf{સર્કિટ ડાયાગ્રામ:}

\begin{center}
\begin{tikzpicture}
    % Antiparallel SCRs
    \draw (0,0) -- (2,0);
    \draw (2,1) node[thyristor, rotate=-180] (S1) {};
    \draw (2,-1) node[thyristor] (S2) {};
    
    \draw (2,0) -- (S1);
    \draw (2,0) -- (S2);
    \draw (4,0) -- (S1);
    \draw (4,0) -- (S2);
    
    \draw (4,0) -- (6,0) to[R, l=Load] (6,-2) -- (0,-2) to[sV, l=AC] (0,0);
    
    % Trigger block
    \draw (2,-3) rectangle (4,-4) node[pos=0.5] {Trigger Circuit};
    \draw (S1) -- (3, -3);
    \draw (S2) -- (3, -3);
\end{tikzpicture}
\captionof{figure}{SCRs નો ઉપયોગ કરીને સ્ટેટિક AC સ્વિચ}
\end{center}

\begin{itemize}
    \item \keyword{એન્ટિપેરેલલ SCRs}: બાઇડિરેક્શનલ કન્ડક્શન માટે ત્રણ SCRs ઇન્વર્સ પેરેલલમાં જોડાયેલા છે
    \item \keyword{ગેટ કંટ્રોલ}: યોગ્ય સમયના ગેટ સિગ્નલ્સ લોડને પાવર નિયંત્રિત કરે છે
    \item \keyword{ઝીરો-ક્રોસિંગ સ્વિચિંગ}: SCRs કુદરતી રીતે ઝીરો ક્રોસિંગ પર બંધ થાય છે
    \item \keyword{એપ્લિકેશન્સ}: હીટર કંટ્રોલ, મોટર સોફ્ટ-સ્ટાર્ટિંગ, લાઇટિંગ કંટ્રોલ
    \item \keyword{ફાયદા}: કોઈ મૂવિંગ પાર્ટ્સ નહીં, સાયલેન્ટ ઓપરેશન, લોંગ લાઇફ
\end{itemize}

\end{solutionbox}
\mnemonicbox{સોલિડ સ્વિચિંગ ટેક્નોલોજી}

\questionmarks{3(a OR)}{3}{ડીસી ચોપરનો મૂળ સિદ્ધાંત સમજાવો.}

\begin{solutionbox}

\begin{center}

\begin{tabulary}{\linewidth}{|L|L|}
\hline \textbf{ઘટક} & \textbf{કાર્ય} \\ \hline
સ્વિચિંગ ડિવાઇસ & SCR, MOSFET, IGBT ઉચ્ચ આવૃત્તિએ DC સ્વિચ કરે છે \\
કંટ્રોલ સર્કિટ & ON/OFF સમયને નિયંત્રિત કરવા માટે PWM ગેટ સિગ્નલ્સ જનરેટ કરે છે \\
ડ્યુટી સાયકલ & કુલ સમયગાળા પર ON સમયનો ગુણોત્તર આઉટપુટ નક્કી કરે છે \\
આઉટપુટ ફિલ્ટર & રિપલ ઘટાડવા માટે ચોપ્ડ આઉટપુટને સ્મૂધ કરે છે \\
કાર્ય સિદ્ધાંત & સરેરાશ વોલ્ટેજ = ઇનપુટ વોલ્ટેજ $\times$ ડ્યુટી સાયકલ \\
\end{tabulary}
\captionof{table}{DC ચોપરનો સિદ્ધાંત}
\end{center}

\end{solutionbox}
\mnemonicbox{ડાયરેક્ટ કરંટ કંટ્રોલ}

\questionmarks{3(b OR)}{4}{આના પર ટૂંકી નોંધ લખો: અન-ઇન્ટરપ્ટેડ પાવર સપ્લાય (UPS).}

\begin{solutionbox}

UPS મુખ્ય સપ્લાય નિષ્ફળ જાય ત્યારે ઇમરજન્સી પાવર પ્રદાન કરે છે.

\textbf{બ્લોક ડાયાગ્રામ:}

\begin{center}
\begin{minipage}{\linewidth}
\centering
\begin{tikzpicture}[gtu block/.style={draw, rectangle, minimum width=2.5cm, minimum height=1.5cm, align=center}, node distance=3.5cm]
    \node[gtu block] (rect) {Rectifier \\ \& DC Section};
    \node[gtu block, right of=rect] (inv) {Inverter \\ \& AC Section};
    \node[left of=rect, align=center] (input) {Mains \\ Input (AC)};
    \node[right of=inv, align=center] (output) {Output \\ (AC)};
    \node[gtu block, below of=rect] (batt) {Battery \\ System};
    \draw[->] (input) -- (rect);
    \draw[->] (rect) -- (inv);
    \draw[->] (inv) -- (output);
    \draw[<->] (rect) -- (batt);
\end{tikzpicture}
\captionof{figure}{UPS બ્લોક ડાયાગ્રામ}
\end{minipage}
\end{center}

\begin{itemize}
    \item \keyword{બેકઅપ પાવર}: આઉટેજ દરમિયાન સતત પાવર પ્રદાન કરે છે
    \item \keyword{પ્રકારો}: ઓનલાઇન, ઓફલાઇન, લાઇન-ઇન્ટરેક્ટિવ UPS
    \item \keyword{સુરક્ષા}: પાવર સર્જ, સેગ્સ અને ફ્રિક્વન્સી વેરિએશન્સ સામે
    \item \keyword{એપ્લિકેશન્સ}: કોમ્પ્યુટર્સ, મેડિકલ ઇક્વિપમેન્ટ, ટેલિકોમ્યુનિકેશન્સ
\end{itemize}

\end{solutionbox}
\mnemonicbox{અનઇન્ટરપ્ટેડ પાવર સિક્યોરલી}

\questionmarks{3(c OR)}{7}{SMPS ના બ્લોક ડાયાગ્રામ દોરો અને દરેક બ્લોકનું કાર્ય સમજાવો.}

\begin{solutionbox}

સ્વિચ્ડ-મોડ પાવર સપ્લાય કુશળતાથી AC ને રેગ્યુલેટેડ DC માં રૂપાંતરિત કરે છે.

\textbf{બ્લોક ડાયાગ્રામ:}

\begin{center}
\begin{tikzpicture}[gtu block/.style={draw, rectangle, minimum width=2cm, minimum height=1.2cm, align=center}, auto, node distance=3cm]
    \node[gtu block] (rect1) {Input \\ Rectifier \\ \& Filter};
    \node[gtu block, right of=rect1] (switch) {High-Freq \\ Switching};
    \node[gtu block, right of=switch] (rect2) {Output \\ Rectifier \\ \& Filter};
    \node[right of=rect2] (out) {DC Output};
    \node[left of=rect1] (in) {AC Input};
    
    \node[gtu block, below of=switch] (ctrl) {Control \\ Circuit};
    
    \draw[->] (in) -- (rect1);
    \draw[->] (rect1) -- (switch);
    \draw[->] (switch) -- (rect2);
    \draw[->] (rect2) -- (out);
    
    \draw[->] (rect2) |- (ctrl); % Feedback
    \draw[->] (ctrl) -- (switch); % PWM Control
\end{tikzpicture}
\captionof{figure}{SMPS બ્લોક ડાયાગ્રામ}
\end{center}

\begin{itemize}
    \item \keyword{ઇનપુટ રેક્ટિફાયર}: AC ને અનરેગ્યુલેટેડ DC માં રૂપાંતરિત કરે છે
    \item \keyword{હાઇ-ફ્રિક્વન્સી સ્વિચિંગ}: ટ્રાન્ઝિસ્ટરનો ઉપયોગ કરીને DC ને હાઇ-ફ્રિક્વન્સી AC માં રૂપાંતરિત કરે છે
    \item \keyword{ટ્રાન્સફોર્મર}: આઇસોલેશન અને વોલ્ટેજ સ્કેલિંગ પ્રદાન કરે છે
    \item \keyword{આઉટપુટ રેક્ટિફાયર}: હાઇ-ફ્રિક્વન્સી AC ને DC માં રૂપાંતરિત કરે છે
    \item \keyword{ફિલ્ટર}: રિપલ ઘટાડવા માટે DC આઉટપુટને સ્મૂધ કરે છે
    \item \keyword{કંટ્રોલ સર્કિટ}: ફીડબેક દ્વારા આઉટપુટને રેગ્યુલેટ કરે છે
\end{itemize}

\end{solutionbox}
\mnemonicbox{સ્વિચ મોડ પાવર સિસ્ટમ}

\questionmarks{4(a)}{3}{1-Phase DC શન્ટ મોટરના ગતિ નિયંત્રણ માટે TRIAC નો ઉપયોગ કરીને સર્કિટ ડાયાગ્રામ દોરો અને તેની કામગીરી સમજાવો.}

\begin{solutionbox}

TRIAC-આધારિત સ્પીડ કંટ્રોલ DC શન્ટ મોટર માટે કાર્યક્ષમ વેરિએબલ સ્પીડ પ્રદાન કરે છે.

\textbf{સર્કિટ ડાયાગ્રામ:}

\begin{center}
\begin{tikzpicture}
    % AC Source
    \draw (0,4) to[sV, l=AC] (0,0);
    
    % TRIAC control part
    \draw (0,4) -- (2,4);
    \draw (2,4) node[triac, rotate=-90] (TR) {};
    \draw (TR) -- (4,4);
    
    % DIAC Trigger
    \draw (2,4) -- (2,3) to[D, l=DIAC] (2,2) -- (TR);
    
    % Bridge Rectifier
    \draw (4,5) to[D] (5,4);
    \draw (4,3) to[D] (5,4);
    \draw (5,4) -- (6,4) to[R, l=Motor] (6,0) -- (4,0);
    
    % RC Network
    \draw (2,4) to[R, l=$R$] (1,3) to[C, l=$C$] (1,0) -- (0,0);

\end{tikzpicture}
\captionof{figure}{DC મોટર માટે TRIAC સ્પીડ કંટ્રોલ}
\end{center}

\begin{itemize}
    \item \keyword{ફેઝ કંટ્રોલ}: TRIAC ફેઝ એંગલ કંટ્રોલ દ્વારા અસરકારક વોલ્ટેજ બદલે છે
    \item \keyword{રેક્ટિફિકેશન}: બ્રિજ રેક્ટિફાયર AC ને DC માં મોટર માટે રૂપાંતરિત કરે છે
    \item \keyword{સ્પીડ વેરિએશન}: લાગુ કરેલા વોલ્ટેજના પ્રમાણમાં મોટર સ્પીડ
    \item \keyword{RC ટાઇમિંગ}: RC નેટવર્ક TRIAC ના ફાયરિંગ એંગલને નક્કી કરે છે
\end{itemize}

\end{solutionbox}
\mnemonicbox{TRIAC રેગ્યુલેટ્સ સ્પીડ}

\questionmarks{4(b)}{4}{IC-556 નો ઉપયોગ કરીને ચાર તબક્કાના ક્રમિક ટાઈમર સર્કિટ ડાયાગ્રામ દોરો અને સમજાવો.}

\begin{solutionbox}

IC-556 ડ્યુઅલ ટાઇમરને મલ્ટી-સ્ટેજ સિક્વેન્શિયલ ટાઇમર તરીકે કોન્ફિગર કરી શકાય છે.

\textbf{સર્કિટ ડાયાગ્રામ:}

\begin{center}
\begin{tikzpicture}
    % Vcc line
    \draw (0,4) -- (8,4) node[right] {$V_{CC}$};
    \draw (0,0) -- (8,0) node[right] {GND};
    
    % 4 Stages abstractly using 555 blocks 
    \foreach \x in {1, 3, 5, 7} {
        \draw (\x, 2) node[draw, rectangle, minimum size=1cm] (T\x) {Timer};
        \draw (\x, 4) -- (T\x.north);
        \draw (\x, 0) -- (T\x.south);
        \draw (\x, 2.5) -- (\x-0.5, 2.5) to[R] (\x-0.5, 4); % R
        \draw (\x, 1.5) -- (\x-0.5, 1.5) to[C] (\x-0.5, 0); % C
    }
    
    % Cascading
    \draw[->] (T1) -- (T3);
    \draw[->] (T3) -- (T5);
    \draw[->] (T5) -- (T7);

\end{tikzpicture}
\captionof{figure}{ક્રમિક ટાઇમર સ્કીમેટિક કન્સેપ્ટ}
\end{center}

\begin{itemize}
    \item \keyword{ડ્યુઅલ ટાઇમર IC}: IC-556 બે 555 ટાઇમર સર્કિટ્સ ધરાવે છે
    \item \keyword{કેસ્કેડેડ કોન્ફિગરેશન}: એક સ્ટેજનો આઉટપુટ આગલાને ટ્રિગર કરે છે
    \item \keyword{ટાઇમિંગ કંટ્રોલ}: RC ટાઇમ કોન્સ્ટન્ટ્સ દરેક સ્ટેજની અવધિ નક્કી કરે છે
    \item \keyword{એપ્લિકેશન્સ}: ઔદ્યોગિક સિક્વન્સિંગ, પ્રક્રિયા નિયંત્રણ, ઓટોમેશન
\end{itemize}

\end{solutionbox}
\mnemonicbox{સિક્વેન્શિયલ સ્ટેપ્સ ટાઇમ્ડ પ્રિસાઇઝલી}

\questionmarks{4(c)}{7}{ઇન્ડક્શન હીટિંગ સમજાવો.}

\begin{solutionbox}

ઇન્ડક્શન હીટિંગ ઇલેક્ટ્રોમેગ્નેટિક ઇન્ડક્શનનો ઉપયોગ કરીને નોન-કોન્ટેક્ટ હીટિંગ પ્રક્રિયા છે.

\textbf{આકૃતિ:}

\begin{center}
\begin{tikzpicture}
    % Coil
    \foreach \y in {0, 0.5, 1, 1.5, 2} {
        \draw[thick] (0,\y) ellipse (1.5 and 0.2);
    }
    \draw (-1.5, 0) -- (-1.5, 2); 
    \draw (1.5, 0) -- (1.5, 2); 
    
    % Workpiece
    \fill[gray!50] (-0.5, 0.2) rectangle (0.5, 1.8);
    \node at (0, 1) {Workpiece};
    
    % Power Supply
    \draw (-2, 1) node[draw] (PS) {HF Supply};
    \draw (PS) -- (-1.5, 1);

\end{tikzpicture}
\captionof{figure}{ઇન્ડક્શન હીટિંગ સેટઅપ}
\end{center}

\begin{center}

\begin{tabulary}{\linewidth}{|L|L|}
\hline \textbf{સિદ્ધાંત} & \textbf{વર્ણન} \\ \hline
ઇલેક્ટ્રોમેગ્નેટિક ઇન્ડક્શન & કોઇલમાં AC પરિવર્તનશીલ ચુંબકીય ક્ષેત્ર બનાવે છે \\
એડી કરંટ્સ & ચુંબકીય ક્ષેત્ર વર્કપીસમાં કરંટ પ્રેરિત કરે છે \\
રેસિસ્ટિવ હીટિંગ & મટિરિયલ રેસિસ્ટન્સને કારણે એડી કરંટ ગરમી પેદા કરે છે \\
સ્કિન ઇફેક્ટ & ઉચ્ચ આવૃત્તિઓ પર કરંટ સપાટીની નજીક કેન્દ્રિત થાય છે \\
એપ્લિકેશન્સ & હીટ ટ્રીટમેન્ટ, મેલ્ટિંગ, ફોર્જિંગ, બ્રેઝિંગ, કુકિંગ \\
\end{tabulary}
\captionof{table}{ઇન્ડક્શન હીટિંગ સિદ્ધાંતો}
\end{center}

\end{solutionbox}
\mnemonicbox{ઇન્ડ્યુસ્ડ હીટિંગ ઇફિશિયન્ટલી}

\questionmarks{4(a OR)}{3}{ત્રણ તબક્કાના IC555 ટાઈમર સર્કિટ દોરો અને સમજાવો.}

\begin{solutionbox}

IC555 નો ઉપયોગ કરતો ત્રણ-સ્ટેજ ટાઇમર ક્રમિક ટાઇમિંગ ઓપરેશન્સ પ્રદાન કરે છે.

\textbf{સર્કિટ ડાયાગ્રામ:}

\begin{center}
\begin{tikzpicture}
    % Similar to sequential above but 3 stages
    \foreach \x/\n in {0/1, 3/2, 6/3} {
        \draw (\x,0) rectangle (\x+2, 2);
        \node at (\x+1, 1) {IC 555 (\n)};
        \draw (\x+1, 2) -- (\x+1, 3) node[above] {$V_{CC}$};
        \draw (\x+1, 0) -- (\x+1, -1) node[below] {GND};
        
        \ifnum \x<6
            \draw[->] (\x+2, 1) -- (\x+3, 1);
        \fi
    }
\end{tikzpicture}
\captionof{figure}{ત્રણ તબક્કાનું IC555 ટાઇમર}
\end{center}

\begin{itemize}
    \item \keyword{મોનોસ્ટેબલ મોડ}: દરેક સ્ટેજ ફિક્સ્ડ ટાઇમ ડિલે સાથે મોનોસ્ટેબલ મોડમાં કામ કરે છે
    \item \keyword{કેસ્કેડેડ કનેક્શન}: પ્રથમ ટાઇમરનો આઉટપુટ બીજાને ટ્રિગર કરે છે, વગેરે
    \item \keyword{ટાઇમિંગ કોમ્પોનન્ટ્સ}: R-C નેટવર્ક દરેક સ્ટેજનો ટાઇમ ડિલે નક્કી કરે છે
    \item \keyword{એપ્લિકેશન્સ}: ઓટોમેટિક સિક્વન્સિંગ, પ્રોસેસ ટાઇમિંગ, ઔદ્યોગિક નિયંત્રણ
\end{itemize}

\end{solutionbox}
\mnemonicbox{ટાઇમ ઇન્ટરવલ્સ ક્રિએટેડ}

\questionmarks{4(b OR)}{4}{ડાઇલેક્ટ્રિક હીટિંગનો સિદ્ધાંત સમજાવો.}

\begin{solutionbox}

\begin{center}

\begin{tabulary}{\linewidth}{|L|L|}
\hline \textbf{સિદ્ધાંત} & \textbf{વર્ણન} \\ \hline
હાઇ-ફ્રિક્વન્સી ઇલેક્ટ્રિક ફિલ્ડ & મટિરિયલ RF વોલ્ટેજ (1-100 MHz) સાથે ઇલેક્ટ્રોડ્સ વચ્ચે મૂકવામાં આવે છે \\
મોલેક્યુલર ફ્રિક્શન & ડિપોલ અણુઓ અલ્ટરનેટિંગ ફિલ્ડ સાથે એલાઇન થવાનો પ્રયાસ કરતી વખતે કંપન/ફરતા રહે છે \\
હીટ જનરેશન & અણુઓ વચ્ચે આંતરિક ઘર્ષણથી સમાન રીતે ગરમી ઉત્પન્ન થાય છે \\
નોન-કન્ડક્ટિવ મટિરિયલ્સ & નોન-કન્ડક્ટિવ મટિરિયલ્સ (પ્લાસ્ટિક, લાકડું, ખોરાક) ગરમ કરવા માટે અસરકારક \\
એપ્લિકેશન્સ & પ્લાસ્ટિક વેલ્ડિંગ, લાકડું સૂકવવું, ફૂડ પ્રોસેસિંગ (માઇક્રોવેવ ઓવન) \\
\end{tabulary}
\captionof{table}{ડાઇલેક્ટ્રિક હીટિંગ સિદ્ધાંતો}
\end{center}

\end{solutionbox}
\mnemonicbox{ડાઇલેક્ટ્રિક એનર્જી હીટ્સ}

\questionmarks{4(c OR)}{7}{ઇન્ડક્શન હીટિંગ અને ડાઇલેક્ટ્રિક હીટિંગ વચ્ચે સરખામણી કરો.}

\begin{solutionbox}

\begin{center}

\begin{tabulary}{\linewidth}{|L|L|L|}
\hline \textbf{પેરામીટર} & \textbf{ઇન્ડક્શન હીટિંગ} & \textbf{ડાઇલેક્ટ્રિક હીટિંગ} \\ \hline
મૂળભૂત સિદ્ધાંત & ઇલેક્ટ્રોમેગ્નેટિક ઇન્ડક્શન & હાઇ-ફ્રિક્વન્સી ઇલેક્ટ્રિક ફિલ્ડ \\
યોગ્ય મટિરિયલ્સ & કન્ડક્ટિવ મટિરિયલ્સ (મેટલ્સ) & નોન-કન્ડક્ટિવ મટિરિયલ્સ (પ્લાસ્ટિક, લાકડું) \\
ફ્રિક્વન્સી રેન્જ & 1 kHz થી 1 MHz & 1 MHz થી 1 GHz \\
હીટિંગ મિકેનિઝમ & એડી કરંટ્સ અને હિસ્ટેરિસિસ & મોલેક્યુલર ફ્રિક્શન (ડિપોલ રોટેશન) \\
હીટ ડિસ્ટ્રિબ્યુશન & સરફેસ હીટિંગ (સ્કિન ઇફેક્ટ) & વોલ્યુમેટ્રિક (સમગ્ર સમાન) \\
કાર્યક્ષમતા & મેગ્નેટિક મટિરિયલ્સ માટે 80-90\% & મટિરિયલ પર આધારિત 50-70\% \\
એપ્લિકેશન્સ & મેટલ મેલ્ટિંગ, ફોર્જિંગ, હીટ ટ્રીટમેન્ટ & પ્લાસ્ટિક વેલ્ડિંગ, ફૂડ પ્રોસેસિંગ, ડ્રાયિંગ \\
ઇક્વિપમેન્ટ & ઇન્ડક્શન કોઇલ, વર્ક પીસ & ઇલેક્ટ્રોડ્સ, ડાઇલેક્ટ્રિક મટિરિયલ \\
\end{tabulary}
\captionof{table}{ઇન્ડક્શન vs ડાઇલેક્ટ્રિક હીટિંગ સરખામણી}
\end{center}

\end{solutionbox}
\mnemonicbox{ICED}

\questionmarks{5(a)}{3}{યુનિવર્સલ મોટરનું બાંધકામ અને કાર્ય સમજાવો.}

\begin{solutionbox}

યુનિવર્સલ મોટર AC અને DC બંને પાવર સોર્સ પર કામ કરે છે.

\textbf{આકૃતિ:}

\begin{center}
\begin{tikzpicture}
    % Stator
    \draw (-2, 2) rectangle (2, 3) node[pos=0.5] {Stator Field Winding};
    \draw (-2, -3) rectangle (2, -2) node[pos=0.5] {Stator Field Winding};
    
    % Rotor
    \draw (0,0) circle (1.5);
    \node at (0,0) {Rotor (Armature)};
    
    % Brushes
    \draw (-1.7, 0) rectangle (-1.5, 0.5);
    \draw (1.5, 0) rectangle (1.7, 0.5);
    
    % Series connection
    \draw (-2, 2.5) -- (-3, 2.5) -- (-3, 0) -- (-1.7, 0);
    \draw (1.7, 0) -- (3, 0) -- (3, -2.5) -- (2, -2.5);

\end{tikzpicture}
\captionof{figure}{યુનિવર્સલ મોટર બાંધકામ}
\end{center}

\begin{itemize}
    \item \keyword{સીરીઝ કનેક્શન}: ફિલ્ડ વાઇન્ડિંગ આર્મેચર વાઇન્ડિંગ સાથે શ્રેણીમાં
    \item \keyword{બાંધકામ}: ફિલ્ડ વાઇન્ડિંગ સાથે સ્ટેટર, કોમ્યુટેટર અને બ્રશ સાથે રોટર
    \item \keyword{કાર્ય સિદ્ધાંત}: AC અને DC બંને પર સમાન દિશા ટોર્ક
    \item \keyword{લાક્ષણિકતાઓ}: ઉચ્ચ સ્ટાર્ટિંગ ટોર્ક, ઓછા લોડ પર ઉચ્ચ ગતિ
    \item \keyword{એપ્લિકેશન્સ}: પોર્ટેબલ ટૂલ્સ, ઘરેલું ઉપકરણો, બ્લેન્ડર્સ
\end{itemize}

\end{solutionbox}
\mnemonicbox{યુનિવર્સલી મોટરાઇઝ્ડ}

\questionmarks{5(b)}{4}{ડીસી સર્વો મોટરનું બાંધકામ દોરો અને સમજાવો.}

\begin{solutionbox}

DC સર્વો મોટર ચોક્કસ પોઝિશન અથવા સ્પીડ કંટ્રોલ પ્રદાન કરે છે.

\textbf{આકૃતિ:}

\begin{center}
\begin{tikzpicture}
    % Motor Body
    \draw (0,0) rectangle (4,2);
    \node[align=center] at (2,1) {DC Motor \\ (PM Stator)};
    
    % Shaft
    \draw (4,1) -- (5,1);
    
    % Encoder/Feedback
    \draw (-1,0.5) rectangle (0,1.5);
    \node at (-0.5, 1) {Encoder};
    \draw (0,1) -- (-1,1); 

\end{tikzpicture}
\captionof{figure}{DC સર્વો મોટર}
\end{center}

\begin{itemize}
    \item \keyword{બાંધકામ}: પરમેનન્ટ મેગ્નેટ સ્ટેટર, હળવા રોટર, ફીડબેક ડિવાઇસ
    \item \keyword{કંટ્રોલ સિસ્ટમ}: પોઝિશન/વેલોસિટી ફીડબેક સાથે ક્લોઝ્ડ-લૂપ કંટ્રોલ
    \item \keyword{લો ઇનર્શિયા}: ઝડપી પ્રતિસાદ અને ચોક્કસ પોઝિશનિંગની મંજૂરી આપે છે
    \item \keyword{એપ્લિકેશન્સ}: રોબોટિક્સ, CNC મશીન્સ, પોઝિશનિંગ સિસ્ટમ્સ
    \item \keyword{ફીચર્સ}: ઉચ્ચ ટોર્ક-ટુ-ઇનર્શિયા રેશિયો, ફાસ્ટ રિસ્પોન્સ, એક્યુરસી
\end{itemize}

\end{solutionbox}
\mnemonicbox{સર્વો સિસ્ટમ કંટ્રોલ}

\questionmarks{5(c)}{7}{પ્રોગ્રામેબલ લોજિક કંટ્રોલ (PLC) નો બ્લોક ડાયાગ્રામ દોરો અને દરેક બ્લોકની કામગીરી સમજાવો.}

\begin{solutionbox}

PLC ઓટોમેશન કંટ્રોલ માટે ઔદ્યોગિક ડિજિટલ કોમ્પ્યુટર છે.

\textbf{બ્લોક ડાયાગ્રામ:}

\begin{center}
\begin{tikzpicture}[gtu block/.style={draw, rectangle, minimum width=2.5cm, minimum height=1.5cm, align=center}, node distance=3.5cm]
    \node[gtu block] (cpu) {Central \\ Processing \\ Unit (CPU)};
    \node[gtu block, left of=cpu] (in) {Input \\ Modules};
    \node[gtu block, right of=cpu] (out) {Output \\ Modules};
    \node[gtu block, below of=cpu] (mem) {Memory \\ Unit};
    \node[gtu block, below of=in] (prog) {Programming \\ Device};
    \node[gtu block, below of=out] (pwr) {Power \\ Supply};
    
    \draw[<->] (in) -- (cpu);
    \draw[<->] (cpu) -- (out);
    \draw[<->] (cpu) -- (mem);
    \draw[<->] (prog) -- (cpu);
    \draw[->] (pwr) -- (cpu);

\end{tikzpicture}
\captionof{figure}{PLC બ્લોક ડાયાગ્રામ}
\end{center}

\begin{itemize}
    \item \keyword{CPU}: પ્રોગ્રામ એક્ઝિક્યુટ કરે છે, I/O ડેટા પ્રોસેસ કરે છે, નિર્ણયો લે છે
    \item \keyword{ઇનપુટ મોડ્યુલ્સ}: ફિલ્ડ સિગ્નલ્સ (સેન્સર્સ, સ્વિચેસ) ને CPU માટે ડિજિટલ સિગ્નલ્સમાં રૂપાંતરિત કરે છે
    \item \keyword{આઉટપુટ મોડ્યુલ્સ}: CPU કમાન્ડ્સને એક્ટ્યુએટર સિગ્નલ્સ (મોટર્સ, વાલ્વ્સ) માં રૂપાંતરિત કરે છે
    \item \keyword{મેમોરી યુનિટ}: પ્રોગ્રામ અને ડેટા સ્ટોર કરે છે (OS માટે ROM, યુઝર પ્રોગ્રામ માટે RAM)
    \item \keyword{પ્રોગ્રામિંગ ડિવાઇસ}: પ્રોગ્રામ ડેવલપમેન્ટ અને મોનિટરિંગ માટે PC અથવા કન્સોલ
    \item \keyword{પાવર સપ્લાય}: PLC કોમ્પોનન્ટ્સને રેગ્યુલેટેડ પાવર પ્રદાન કરે છે
\end{itemize}

\end{solutionbox}
\mnemonicbox{પ્રોગ્રામ્સ લોજિક કમ્પ્લીટલી}

\questionmarks{5(a OR)}{3}{સ્ટેપર મોટરનું બાંધકામ દોરો અને સમજાવો.}

\begin{solutionbox}

સ્ટેપર મોટર ચોક્કસ પોઝિશનિંગ માટે ડિસ્ક્રીટ સ્ટેપ્સમાં ફરે છે.

\textbf{આકૃતિ:}

\begin{center}
\begin{tikzpicture}
    % Stator
    \draw (0,0) circle (2);
    \foreach \x in {0, 90, 180, 270} {
        \draw (\x:1.5) -- (\x:2); % Poles
        \node at (\x:2.3) {Pole};
    }
    
    % Rotor
    \draw (0,0) circle (1);
    \node at (0,0) {N};
    \node at (0, -0.5) {S};

\end{tikzpicture}
\captionof{figure}{સ્ટેપર મોટર બાંધકામ}
\end{center}

\begin{itemize}
    \item \keyword{સ્ટેટર}: મલ્ટિપલ કોઇલ વાઇન્ડિંગ્સ (ફેઝીસ) ધરાવે છે
    \item \keyword{રોટર}: પરમેનન્ટ મેગ્નેટ અથવા વેરિએબલ રિલક્ટન્સ પ્રકાર
    \item \keyword{પ્રકારો}: પરમેનન્ટ મેગ્નેટ, વેરિએબલ રિલક્ટન્સ, હાઇબ્રિડ
    \item \keyword{સ્ટેપ એંગલ}: સામાન્ય રીતે 1.8$^{\circ}$ (200 સ્ટેપ્સ/રેવ) અથવા 0.9$^{\circ}$ (400 સ્ટેપ્સ/રેવ)
    \item \keyword{એપ્લિકેશન્સ}: પ્રિન્ટર્સ, ડિસ્ક ડ્રાઇવ્સ, રોબોટિક્સ, CNC મશીન્સ
\end{itemize}

\end{solutionbox}
\mnemonicbox{સ્ટેપ્સ પ્રિસાઇઝલી મૂવ્ડ}

\questionmarks{5(b OR)}{4}{ડીસી શન્ટ મોટર સ્પીડને નિયંત્રિત કરવા માટે સોલિડ સ્ટેટ સર્કિટ સમજાવો.}

\begin{solutionbox}

સોલિડ-સ્ટેટ સર્કિટ DC મોટર સ્પીડ કંટ્રોલ માટે કાર્યક્ષમ અને સ્મૂધ કંટ્રોલ પ્રદાન કરે છે.

\textbf{સર્કિટ ડાયાગ્રામ:}

\begin{center}
\begin{tikzpicture}
    % Field
    \draw (0,4) to[L, l=Field] (2,4);
    
    % Armature + MOSFET
    \draw (0,2) to[R, l=Armature] (2,2) -- (2,1) node[nigbt, rotate=-90] (Q1) {} -- (2,0); 
    \draw (0,2) -- (0,0);
    
    % Gate Drive
    \draw (Q1) -- (1.5, 1) node[left] {PWM Driver};

\end{tikzpicture}
\captionof{figure}{સોલિડ સ્ટેટ DC મોટર કંટ્રોલ}
\end{center}

\begin{itemize}
    \item \keyword{PWM કંટ્રોલર}: ગતિ નિયંત્રિત કરવા માટે વેરિએબલ ડ્યુટી સાયકલ પલ્સ જનરેટ કરે છે
    \item \keyword{MOSFET ડ્રાઇવર}: પાવર MOSFET માટે ગેટ ડ્રાઇવ પ્રદાન કરે છે
    \item \keyword{પાવર MOSFET}: આર્મેચર વાઇન્ડિંગમાં કરંટ નિયંત્રિત કરે છે
    \item \keyword{ફીડબેક}: ટેકોજનરેટર અથવા એન્કોડર સ્પીડ ફીડબેક પ્રદાન કરે છે
    \item \keyword{ફાયદા}: કાર્યક્ષમ, સરળ નિયંત્રણ, વિશાળ ગતિ રેન્જ
\end{itemize}

\end{solutionbox}
\mnemonicbox{પાવર વિથ MOSFET}

\questionmarks{5(c OR)}{7}{VFD (વેરિયેબલ ફ્રીક્વન્સી ડ્રાઇવ) ની કામગીરી સમજાવો.}

\begin{solutionbox}

VFD ફ્રિક્વન્સી અને વોલ્ટેજમાં ફેરફાર કરીને AC મોટર સ્પીડ નિયંત્રિત કરે છે.

\textbf{બ્લોક ડાયાગ્રામ:}

\begin{center}
\begin{tikzpicture}[gtu block/.style={draw, rectangle, minimum width=2.2cm, minimum height=1.2cm, align=center}, node distance=3cm]
    \node[gtu block] (rect) {Rectifier};
    \node[gtu block, right of=rect] (dc) {DC Link};
    \node[gtu block, right of=dc] (inv) {Inverter};
    \node[right of=inv] (motor) {AC Motor};
    \node[left of=rect] (ac) {AC Main};
    
    \draw[->] (ac) -- (rect);
    \draw[->] (rect) -- (dc);
    \draw[->] (dc) -- (inv);
    \draw[->] (inv) -- (motor);
    
    \node[gtu block, below of=dc] (ctrl) {Control \\ Circuit};
    \draw[->] (ctrl) -- (inv);

\end{tikzpicture}
\captionof{figure}{VFD બ્લોક ડાયાગ્રામ}
\end{center}

\begin{center}

\begin{tabulary}{\linewidth}{|L|L|}
\hline \textbf{ઘટક} & \textbf{કાર્ય} \\ \hline
રેક્ટિફાયર & AC ઇનપુટને DC માં રૂપાંતરિત કરે છે (ડાયોડ બ્રિજ અથવા એક્ટિવ ફ્રન્ટ એન્ડ) \\
DC લિંક & DC ને ફિલ્ટર કરે છે અને ઊર્જા સંગ્રહિત કરે છે (કેપેસિટર્સ, ક્યારેક ઇન્ડક્ટર્સ) \\
ઇન્વર્ટર & DC ને વેરિએબલ ફ્રિક્વન્સી AC માં રૂપાંતરિત કરે છે (PWM સાથે IGBTs) \\
કંટ્રોલ સર્કિટ & સ્પીડ જરૂરિયાત આધારિત ફ્રિક્વન્સી/વોલ્ટેજને રેગ્યુલેટ કરે છે \\
બ્રેકિંગ સર્કિટ & ડિસેલરેશન દરમિયાન રિજનરેટિવ ઊર્જાને વેડફે છે \\
\end{tabulary}
\captionof{table}{VFD ઘટકો}
\end{center}

\begin{itemize}
    \item \keyword{સ્પીડ કંટ્રોલ}: મોટર સ્પીડ ફ્રિક્વન્સીના પ્રમાણમાં ($RPM = 120f/P$)
    \item \keyword{ટોર્ક કંટ્રોલ}: કોન્સ્ટન્ટ ટોર્ક માટે V/f રેશિયો જાળવે છે
    \item \keyword{એનર્જી સેવિંગ્સ}: ઓછી ગતિએ ઊર્જા વપરાશ ઘટાડે છે
    \item \keyword{એપ્લિકેશન્સ}: પંપ્સ, ફેન્સ, કન્વેયર્સ, પ્રોસેસ કંટ્રોલ
    \item \keyword{ફીચર્સ}: સોફ્ટ સ્ટાર્ટ, ઓવરકરંટ પ્રોટેક્શન, રિજનરેટિવ બ્રેકિંગ
\end{itemize}

\end{solutionbox}
\mnemonicbox{વેરી ફ્રિક્વન્સી, ડ્રાઇવ મોટર}

\end{document}
