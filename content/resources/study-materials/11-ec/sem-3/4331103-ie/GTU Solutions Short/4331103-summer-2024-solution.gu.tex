\documentclass{article}

% content/resources/templates/preamble.tex
\usepackage[margin=0.6in]{geometry}
\author{Milav Dabgar}
\usepackage{amsmath,amssymb,amsthm}
\usepackage{booktabs}
\usepackage{multirow}
\usepackage{xcolor}
\usepackage{tcolorbox}
\tcbuselibrary{breakable,skins}
\usepackage[colorlinks=true,linkcolor=blue]{hyperref}
\usepackage{titlesec}
\usepackage{enumitem}
\usepackage{tikz}
\usepackage{pgfplots}
\usepackage{circuitikz}
\usepackage[version=4]{mhchem}
\usepackage{longtable}
\usepackage{array}
\usepackage{float}
\usepackage{caption}
\usepackage{listings}

\lstset{
  basicstyle=\small\ttfamily,
  breaklines=true,
  breakatwhitespace=false,
  postbreak=\mbox{\textcolor{red}{$\hookrightarrow$}\space},
  float=false,
  numbers=left,
  numberstyle=\tiny\color{gray},
  numbersep=10pt,
  xleftmargin=2em,
  keywordstyle=\color{blue},
  commentstyle=\color{green!60!black},
  stringstyle=\color{purple},
  backgroundcolor=\color{gray!5},
  showstringspaces=false,
  tabsize=2,
  captionpos=b,
  keepspaces=true,
  columns=flexible
}

\pgfplotsset{compat=1.18}
\usetikzlibrary{shapes,arrows,positioning,calc,patterns,decorations.pathmorphing,decorations.markings,arrows.meta}

% Color scheme
\definecolor{headcolor}{RGB}{0,102,204}
\definecolor{keycolor}{RGB}{220,20,60}
\definecolor{solutioncolor}{RGB}{34,139,34}
\definecolor{mnemoniccolor}{RGB}{148,0,211}
\definecolor{codecolor}{RGB}{0,0,100}

% Spacing
\setlength{\parskip}{3pt}
\setlist[itemize]{nosep}
\setlist[enumerate]{nosep}

% Title formatting
\titleformat{\section}{\Large\bfseries\color{headcolor}}{\thesection}{1em}{}
\titleformat{\subsection}{\large\bfseries\color{headcolor}}{\thesubsection}{1em}{}

% Pandoc tightlist compatibility
\providecommand{\tightlist}{%
  \setlength{\itemsep}{0pt}\setlength{\parskip}{0pt}}

% Pandoc longtable compatibility
\newcounter{none}
\def\thenone{}


% content/resources/templates/gujarati-boxes.tex
\usepackage{fontspec}
\usepackage{polyglossia}

% Set Gujarati as main language (document is primarily in Gujarati)
% Note: gloss-gujarati.ldf doesn't exist in polyglossia, but it will use hyphenation patterns
\setdefaultlanguage{gujarati}
\setotherlanguage{english}

% Configure Gujarati font properly
% Use Language=Default to prevent polyglossia from trying to add language-specific features
% that don't exist for Gujarati, which causes "empty feature" warnings
\newfontfamily\gujaratifont[Script=Gujarati,AutoFakeBold=2.5,AutoFakeSlant=0.3]{Noto Sans Gujarati}
\setmainfont[Script=Gujarati,AutoFakeBold=2.5,AutoFakeSlant=0.3]{Noto Sans Gujarati}
% Use Noto Sans Gujarati for monospace to support Gujarati in text
\setmonofont[Scale=0.9]{Noto Sans Gujarati}

% Configure English to use the same font
\newfontfamily\englishfont[Script=Gujarati,AutoFakeBold=2.5,AutoFakeSlant=0.3]{Noto Sans Gujarati}

% Translations for polyglossia
\gappto\captionsgujarati{
  \renewcommand{\tablename}{કોષ્ટક}
  \renewcommand{\figurename}{આકૃતિ}
}

% Helper for TikZ nodes to ensure Gujarati font
\newcommand{\gu}[1]{{\gujaratifont #1}}

% Custom environments
\newtcolorbox{solutionbox}{
    breakable,
    enhanced,
    colback=solutioncolor!5!white,
    colframe=solutioncolor!75!black,
    fonttitle=\bfseries,
    title=જવાબ
}

\newtcolorbox{solutionboxnobreak}{
 colback=solutioncolor!5!white,
 colframe=solutioncolor!75!black,
 fonttitle=\bfseries,
 title=જવાબ
}

\newtcolorbox{keyformula}{
 breakable,
 enhanced,
 colback=keycolor!5!white,
 colframe=keycolor!75!black,
 fonttitle=\bfseries,
 title=રાસાયણિક સમીકરણ/સૂત્ર
}

\newtcolorbox{mnemonicbox}{
 breakable,
 enhanced,
 colback=mnemoniccolor!5!white,
 colframe=mnemoniccolor!75!black,
 fonttitle=\bfseries,
 title=મેમરી ટ્રીક
}


% Custom commands for GTU solutions
% This file defines semantic commands for consistent formatting

% Question command with automatic formatting
\newcommand{\question}[2]{%
  \section*{Question #1}%
  \textbf{#2}%
}

% OR question variant
\newcommand{\questionor}[2]{%
  \section*{Question #1 OR}%
  \textbf{#2}%
}

% Proper table environment with caption
\newenvironment{answertable}[1]{%
  \begin{table}[htbp]
  \centering
  \caption{#1}
}{%
  \end{table}
}

% Proper figure environment for diagrams
\newenvironment{answerdiagram}[1]{%
  \begin{figure}[htbp]
  \centering
  \caption{#1}
}{%
  \end{figure}
}

% Semantic markup for key terms
\newcommand{\keyword}[1]{\textbf{#1}}
\newcommand{\code}[1]{\texttt{#1}}
\newcommand{\classname}[1]{\texttt{#1}}
\newcommand{\methodname}[1]{\texttt{#1}}

% Proper quotation marks
\newcommand{\mnemonic}[1]{``#1''}


\title{ઔદ્યોગિક ઇલેક્ટ્રોનિક્સ (4331103) - ઉનાળુ-2024 પરીક્ષા ઉકેલ}
\date{June 12, 2024}

\begin{document}
\maketitle

\questionmarks{1(a)}{3}{SCR ની બે ટ્રાન્ઝિસ્ટર સામ્યતા સમજાવો.}

\begin{solutionbox}
SCR એ પરસ્પર જોડાયેલા PNP અને NPN ટ્રાન્ઝિસ્ટર તરીકે રજૂ કરી શકાય છે.

\begin{center}
\begin{tikzpicture}[gtu block/.style={draw, rectangle, minimum width=2cm, minimum height=1cm, align=center}]
    % Transistors representation
    \node (anode) at (0,4) {Anode};
    \node (cathode) at (0,-4) {Cathode};
    
    % PNP Transistor Q1
    \draw (0,2) node[pnp, rotate=-90] (Q1) {};
    \node[right] at (Q1) {$Q_1$ (PNP)};
    
    % NPN Transistor Q2
    \draw (0,-2) node[npn, rotate=-90] (Q2) {};
    \node[right] at (Q2) {$Q_2$ (NPN)};
    
    % Connections
    \draw (anode) -- (Q1.E);
    \draw (Q1.C) -- (Q2.B);
    \draw (Q2.C) -- (Q1.B); % Feedback loop
    \draw (Q2.E) -- (cathode);
    
    % Gate
    \draw (Q2.B) -- ++(-1.5,0) node[left] {Gate};

\end{tikzpicture}
\captionof{figure}{SCR ની બે ટ્રાન્ઝિસ્ટર સામ્યતા}
\end{center}

\begin{itemize}
    \item \keyword{પુનઃઉત્પાદક ક્રિયા}: જ્યારે ગેટ પ્રવાહ NPN ને ટ્રિગર કરે છે, તે PNP ને વહન કરવા માટે કારણભૂત બને છે, જે સ્વ-ટકાઉ પ્રવાહ બનાવે છે
    \item \keyword{લેચિંગ મિકેનિઝમ}: એકવાર બંને ટ્રાન્ઝિસ્ટર ચાલુ થઈ જાય, ગેટ નિયંત્રણ ગુમાવે છે કારણ કે ફીડબેક પાથ વહન જાળવી રાખે છે
\end{itemize}
\end{solutionbox}
\mnemonicbox{પુશ-પુલ નેટવર્ક સતત વહન ટ્રિગર કરે છે}

\questionmarks{1(b)}{4}{IGBT ની કામગીરી અને લાક્ષણિકતા સમજાવો.}

\begin{solutionbox}
IGBT (ઇન્સુલેટેડ ગેટ બાયપોલર ટ્રાન્ઝિસ્ટર) MOSFET ઇનપુટ લાક્ષણિકતાઓને BJT આઉટપુટ ક્ષમતાઓ સાથે જોડે છે.

\begin{center}
\begin{tikzpicture}
    % IGBT Symbol
    \draw (0,0) node[nigbt] (igbt) {};
    \node[right] at (igbt.C) {Collector};
    \node[right] at (igbt.E) {Emitter};
    \node[left] at (igbt.G) {Gate};
    
    % Equivalent Circuit Representation
    \draw (4, 1) node[nmos] (mos) {};
    \draw (5.5, 0) node[pnp] (bjt) {};
    
    \node[above] at (mos.D) {Drain};
    \node[left] at (mos.G) {Gate};
    \node[right] at (bjt.E) {Emitter (IGBT Collect)};
    \node[below] at (bjt.C) {Collector (IGBT Emit)};
    
    % Connections for equivalent circuit concept
    \draw[dashed] (mos.S) -- (bjt.B); 
\end{tikzpicture}
\captionof{figure}{IGBT પ્રતીક અને બંધારણ}
\end{center}

\begin{center}
\begin{tabulary}{\linewidth}{|L|L|}
\hline \textbf{વિશેષતા} & \textbf{લાક્ષણિકતા} \\ \hline
સ્વિચિંગ & ઝડપી ચાલુ થવું, મધ્યમ બંધ થવું \\
નિયંત્રણ & MOSFET જેવું વોલ્ટેજ-નિયંત્રિત \\
વહન & BJT જેવું ઓછું ફોરવર્ડ વોલ્ટેજ ડ્રોપ \\
ઉપયોગો & ઉચ્ચ વોલ્ટેજ, મધ્યમ આવૃત્તિ સ્વિચિંગ \\
\hline
\end{tabulary}
\captionof{table}{IGBT લાક્ષણિકતાઓ}
\end{center}

\begin{itemize}
    \item \keyword{ઇનપુટ ફાયદો}: ઉચ્ચ અવરોધ સાથે વોલ્ટેજ-નિયંત્રિત ગેટ જેને લઘુત્તમ ડ્રાઇવ પાવરની જરૂર છે
    \item \keyword{આઉટપુટ ફાયદો}: ઉચ્ચ વિદ્યુત ઘનતા પર પણ ઓછો ઓન-સ્ટેટ વોલ્ટેજ ડ્રોપ
\end{itemize}
\end{solutionbox}
\mnemonicbox{MOSFET ઇનપુટ, BJT આઉટપુટ, સંપૂર્ણ પાવર સ્વિચ બનાવે છે}

\questionmarks{1(c)}{7}{DIAC નું બાંધકામ, કાર્ય અને લાક્ષણિકતા સમજાવો.}

\begin{solutionbox}
DIAC (ડાયોડ ફોર ઓલ્ટરનેટિંગ કરંટ) એ દ્વિદિશ ટ્રિગરિંગ ઉપકરણ છે જે થાઇરિસ્ટર નિયંત્રણ સર્કિટોમાં વપરાય છે.

\begin{center}
\begin{tikzpicture}
    % Construction
    \draw (0,0) rectangle (4,2);
    \draw (0,1.5) -- (4,1.5);
    \draw (0,0.5) -- (4,0.5);
    \node at (2, 1.75) {N2 (Terminal B)};
    \node at (2, 1) {P2 - N1 - P1};
    \node at (2, 0.25) {N1 (Terminal A)};
    
    % Symbol
    \draw (6,1) to[diac] (8,1);
    \node[left] at (6,1) {MT1};
    \node[right] at (8,1) {MT2};
\end{tikzpicture}
\captionof{figure}{DIAC બાંધકામ અને પ્રતીક}
\end{center}

\begin{center}
\begin{tikzpicture}
    % V-I Curve
    \draw[->] (-3,0) -- (3,0) node[right] {$V$};
    \draw[->] (0,-3) -- (0,3) node[above] {$I$};
    
    \draw[blue, thick] (0,0) -- (1,0.1) -- (2, 2) -- (1.5, 2.5);
    \draw[blue, thick] (0,0) -- (-1,-0.1) -- (-2, -2) -- (-1.5, -2.5);
    
    \node[right] at (2,2) {$V_{BO}$};
    \node[left] at (-2,-2) {$-V_{BO}$};
\end{tikzpicture}
\captionof{figure}{DIAC V-I લાક્ષણિકતાઓ}
\end{center}

\begin{center}
\begin{tabulary}{\linewidth}{|L|L|}
\hline \textbf{વિશેષતા} & \textbf{વર્ણન} \\ \hline
સ્ટ્રક્ચર & ગેટ ટર્મિનલ વગરનું પાંચ સ્તરીય P-N-P-N \\
કાર્ય & બ્રેક-ઓવર વોલ્ટેજ પહોંચતા સુધી પ્રવાહને અવરોધે છે \\
બ્રેકઓવર & સામાન્ય રીતે બંને દિશામાં 30-40V \\
સમમિતિ & બંને દિશાઓમાં સમાન પ્રતિક્રિયા \\
ઉપયોગ & AC સર્કિટમાં TRIAC માટે ટ્રિગર ઉપકરણ \\
\hline
\end{tabulary}
\captionof{table}{DIAC વિશેષતાઓ}
\end{center}

\begin{itemize}
    \item \keyword{અવરોધ અવસ્થા}: બ્રેકઓવર વોલ્ટેજથી નીચે, ઉચ્ચ અવરોધ પ્રવાહને રોકે છે
    \item \keyword{વહન અવસ્થા}: બ્રેકઓવર વોલ્ટેજથી ઉપર, નકારાત્મક અવરોધ વિસ્તાર અચાનક વહન સક્ષમ કરે છે
    \item \keyword{દ્વિદિશીય}: હકારાત્મક અને નકારાત્મક વોલ્ટેજ માટે સમાન રીતે કાર્ય કરે છે
\end{itemize}
\end{solutionbox}
\mnemonicbox{બંને દિશામાં બ્રેક વોલ્ટેજ, પછી પ્રવાહ વહે છે}

\questionmarks{1(c) OR}{7}{ઓપ્ટો-આઇસોલેટર અને ઓપ્ટો-એસસીઆરનું બાંધકામ અને કાર્ય સમજાવો.}

\begin{solutionbox}
ઓપ્ટો-ઉપકરણો સર્કિટો વચ્ચે વિદ્યુત અલગાવ જાળવતા સિગ્નલો ટ્રાન્સફર કરવા માટે પ્રકાશનો ઉપયોગ કરે છે.

\begin{center}
\begin{tikzpicture}
    % Opto-Isolator
    \draw[dashed] (-1,-1) rectangle (3,2);
    \node at (1,2.3) {Opto-Isolator};
    \draw (0,0.5) node[led, rotate=90] (led) {};
    \draw (led.anode) -- (0, 1.5) node[above] {In+};
    \draw (led.cathode) -- (0, -0.5) node[below] {In-};
    
    \draw (2,0.5) node[npn, photo] (photo) {};
    \draw (photo.C) -- (2,1.5) node[above] {VCC};
    \draw (photo.E) -- (2,-0.5) node[below] {Out};
    
    \draw[->, wave] (0.5, 0.5) -- (1.5, 0.5);
\end{tikzpicture}
\captionof{figure}{ઓપ્ટો-આઇસોલેટર}
\end{center}

\begin{center}
\begin{tikzpicture}
    % Opto-SCR
    \draw[dashed] (-1,-1) rectangle (3,2);
    \node at (1,2.3) {Opto-SCR};
    \draw (0,0.5) node[led, rotate=90] (led) {};
    \draw (led.anode) -- (0, 1.5) node[above] {In+};
    \draw (led.cathode) -- (0, -0.5) node[below] {In-};
    
    \draw (2,0.5) node[thyristor] (scr) {};
    \draw (scr.A) -- (2,1.5) node[above] {A};
    \draw (scr.K) -- (2,-0.5) node[below] {K};
    
    \draw[->, wave] (0.5, 0.5) -- (1.5, 0.5);
\end{tikzpicture}
\captionof{figure}{ઓપ્ટો-SCR}
\end{center}

\begin{center}
\begin{tabulary}{\linewidth}{|L|L|L|}
\hline \textbf{વિશેષતા} & \textbf{ઓપ્ટો-આઇસોલેટર} & \textbf{ઓપ્ટો-SCR} \\ \hline
ઇનપુટ & LED & LED \\
આઉટપુટ ઉપકરણ & ફોટોટ્રાન્ઝિસ્ટર/ફોટોડાયોડ & પ્રકાશ-સંવેદનશીલ SCR \\
અલગાવ & 2-5 kV & 2-5 kV \\
વિદ્યુત પ્રવાહ & ઓછો-મધ્યમ (100mA) & ઉચ્ચ (ઘણા એમ્પિયર) \\
ઉપયોગો & ડિજિટલ સિગ્નલ આઇસોલેશન & પાવર નિયંત્રણ, AC સ્વિચિંગ \\
\hline
\end{tabulary}
\captionof{table}{ઓપ્ટો-ઉપકરણોની તુલના}
\end{center}

\begin{itemize}
    \item \keyword{વિદ્યુત આઇસોલેશન}: સંપૂર્ણ વિદ્યુત અલગતા અવાજ પ્રતિરક્ષા અને સુરક્ષા પ્રદાન કરે છે
    \item \keyword{સિગ્નલ ટ્રાન્સફર}: પ્રકાશ કપલિંગ ગ્રાઉન્ડ લૂપ્સ અને વોલ્ટેજ સ્તરના મુદ્દાઓને દૂર કરે છે
    \item \keyword{ટ્રિગરિંગ}: ઓપ્ટો-SCRમાં પ્રકાશ ગેટ વિદ્યુત પ્રવાહને SCR સક્રિયકરણ માટે બદલે છે
\end{itemize}
\end{solutionbox}
\mnemonicbox{પ્રકાશ અંતર કૂદે છે જ્યારે વિદ્યુત ઘરે રહે છે}

\questionmarks{2(a)}{3}{1) UJT 2) SCS 3) MCT નું પ્રતીક દોરો અને ઉપયોગ આપો.}

\begin{solutionbox}
\begin{center}
\begin{tabulary}{\linewidth}{|C|C|L|}
\hline \textbf{ઉપકરણ} & \textbf{પ્રતીક} & \textbf{ઉપયોગો} \\ \hline
UJT & 
\begin{tikzpicture}[scale=0.5, baseline]
    \draw (0,0) node[ujt] {};
\end{tikzpicture} & 
રિલેક્સેશન ઓસિલેટર, ટાઇમિંગ સર્કિટ, SCR ટ્રિગરિંગ \\ \hline
SCS & 
\begin{tikzpicture}[scale=0.5, baseline]
    \draw (0,0) -- (0,-0.5) node[thyristor] (T) {} -- (0,-1.5);
    \draw (T.west) -- (-0.5, -0.7) node[left] {G1};
    \draw (0.3, -0.3) -- (0.5, -0.3) node[right] {G2}; 
\end{tikzpicture} & 
ઓછી પાવર સ્વિચિંગ, લેવલ ડિટેક્શન, પલ્સ જનરેશન \\ \hline
MCT & 
\begin{tikzpicture}[scale=0.5, baseline]
    \draw (0,0) node[thyristor] (T) {};
    \draw (T.west) -- ++(-0.5,0) node[left] {G};
    \node at (0.8,0) {MCT};
\end{tikzpicture} & 
ઉચ્ચ પાવર સ્વિચિંગ, મોટર નિયંત્રણ, ઇન્વર્ટર \\ \hline
\end{tabulary}
\captionof{table}{પાવર ઉપકરણોના પ્રતીકો અને ઉપયોગો}
\end{center}
\end{solutionbox}
\mnemonicbox{અનોખી ટાઇમિંગ, નિયંત્રિત સ્વિચિંગ, મુખ્ય પાવર}

\questionmarks{2(b)}{4}{SCR માટે ગેટ પ્રોટેક્શનનું મહત્વ સમજાવો.}

\begin{solutionbox}
ગેટ પ્રોટેક્શન સર્કિટ SCRને નકલી ટ્રિગરિંગ અને વોલ્ટેજ સ્પાઇક્સથી સુરક્ષિત રાખે છે.

\begin{center}
\begin{tikzpicture}
    \draw (0,0) node[thyristor] (S) {};
    \draw (S.G) -- (-2, -0.7); % Gate line
    
    % Protection components
    \draw (-1, -0.7) to[R, l=$R_g$] (-1, -2) node[ground]{}; % Gate resistor
    \draw (-1.5, -0.7) to[D, l=D] (-1.5, -2) node[ground]{}; % Protecting Diode
    
    \node at (0, 1) {Anode};
    \node at (0, -2) {Cathode};
\end{tikzpicture}
\captionof{figure}{ગેટ પ્રોટેક્શન સર્કિટ}
\end{center}

\begin{center}
\begin{tabulary}{\linewidth}{|L|L|L|}
\hline \textbf{સમસ્યા} & \textbf{સુરક્ષા પદ્ધતિ} & \textbf{હેતુ} \\ \hline
રિવર્સ વોલ્ટેજ & ગેટમાં ડાયોડ & ગેટ-કેથોડ જંક્શન નુકસાન અટકાવે છે \\
નોઇઝ & RC ફિલ્ટર & ઉચ્ચ-આવૃત્તિ ક્ષણિક અવરોધે છે \\
dV/dt ટ્રિગરિંગ & RC સ્નબર & વોલ્ટેજ વધારાનો દર નિયંત્રિત કરે છે \\
ખોટું ટ્રિગરિંગ & ગેટ રેસિસ્ટર & ગેટ કરંટને મર્યાદિત કરે છે અને નોઇઝ ટ્રિગરિંગ ટાળે છે \\
\hline
\end{tabulary}
\captionof{table}{ગેટ પ્રોટેક્શન પદ્ધતિઓ}
\end{center}

\begin{itemize}
    \item \keyword{જંક્શન સુરક્ષા}: ગેટ-કેથોડ જંક્શનને રિવર્સ વોલ્ટેજ નુકસાનથી બચાવે છે
    \item \keyword{નોઇઝ પ્રતિરક્ષા}: વિદ્યુત ઘોંઘાટને ફિલ્ટર કરે છે જે અનિચ્છનીય ટ્રિગરિંગનું કારણ બની શકે છે
\end{itemize}
\end{solutionbox}
\mnemonicbox{ગેટની રક્ષા કરો સમસ્યાઓ અટકાવવા માટે}

\questionmarks{2(c)}{7}{SCR ને ટ્રિગર કરવાની વિવિધ પદ્ધતિઓની યાદી બનાવો અને તેમાંથી કોઈપણ ત્રણ સમજાવો.}

\begin{solutionbox}
SCR ટ્રિગરિંગ પદ્ધતિઓ ગેટ સક્રિયકરણ દ્વારા ઉપકરણને અવરોધનથી વહન અવસ્થામાં રૂપાંતરિત કરે છે.

\begin{center}
\begin{tabulary}{\linewidth}{|L|L|L|}
\hline \textbf{પદ્ધતિ} & \textbf{સિદ્ધાંત} & \textbf{ઉપયોગો} \\ \hline
ગેટ ટ્રિગરિંગ & ગેટમાં સીધો પ્રવાહ & સૌથી સામાન્ય પદ્ધતિ \\
થર્મલ ટ્રિગરિંગ & તાપમાન વધારો & થર્મલ પ્રોટેક્શન \\
પ્રકાશ ટ્રિગરિંગ & જંક્શન પર ફોટોન & રિમોટ સક્રિયકરણ \\
dV/dt ટ્રિગરિંગ & ઝડપી વોલ્ટેજ વધારો & ઘણીવાર અનિચ્છનીય ટ્રિગરિંગ \\
વોલ્ટેજ ટ્રિગરિંગ & બ્રેકઓવર વોલ્ટેજ ઓળંગવું & પ્રોટેક્શન સર્કિટ \\
RF ટ્રિગરિંગ & રેડિયો ફ્રિક્વન્સી સિગ્નલ & વાયરલેસ કંટ્રોલ \\
\hline
\end{tabulary}
\captionof{table}{ટ્રિગરિંગ પદ્ધતિઓ ઓવરવ્યૂ}
\end{center}

\textbf{1. ગેટ કરંટ ટ્રિગરિંગ:}
\begin{center}
\begin{tikzpicture}
    \draw (0,0) node[thyristor] (S) {};
    \draw (S.G) -- (-2, -0.7) to[battery] (-2, -2) -- (S.K);
    \node at (-1.5, -1) {$I_g$};
\end{tikzpicture}
\end{center}
\begin{itemize}
    \item \keyword{સીધું નિયંત્રણ}: નાનો ગેટ પ્રવાહ મોટા એનોડ પ્રવાહને શરૂ કરે છે
    \item \keyword{પ્રવાહ રેન્જ}: SCR રેટિંગ પર આધાર રાખીને સામાન્ય રીતે 10-100mA જરૂરી
\end{itemize}

\textbf{2. પ્રકાશ ટ્રિગરિંગ (LASCR):}
\begin{center}
\begin{tikzpicture}
    \draw (0,0) node[thyristor] (S) {};
    \draw[->, wave] (-1.5, 0) -- (-0.5, 0);
    \node at (-2, 0) {Light};
\end{tikzpicture}
\end{center}
\begin{itemize}
    \item \keyword{ઓપ્ટિકલ કંટ્રોલ}: ફોટોન્સ જંક્શન પર કેરિયર્સ ઉત્પન્ન કરે છે
    \item \keyword{અલગાવ}: કંટ્રોલ અને પાવર સર્કિટ વચ્ચે વિદ્યુત અલગાવ પ્રદાન કરે છે
\end{itemize}

\textbf{3. dV/dt ટ્રિગરિંગ:}
\begin{center}
\begin{tikzpicture}
    \draw (0,0) node[thyristor] (S) {};
    \draw (0,1) node[above] {High $dV/dt$};
\end{tikzpicture}
\end{center}
\begin{itemize}
    \item \keyword{રેટ સંવેદનશીલતા}: ઝડપી વોલ્ટેજ વધારો જંક્શન કેપેસિટન્સ ચાર્જિંગનું કારણ બને છે
    \item \keyword{નિવારણ}: સ્નબર સર્કિટ (RC નેટવર્ક) વોલ્ટેજ વધારાના દરને નિયંત્રિત કરે છે
\end{itemize}
\end{solutionbox}
\mnemonicbox{ગેટ, પ્રકાશ, અને વોલ્ટેજ પરિવર્તન SCRને ચાલુ કરે છે}

\questionmarks{2(a) OR}{3}{ઓપ્ટો-એસસીઆરનો ઉપયોગ કરીને સોલિડ સ્ટેટ રિલેનું કાર્ય સમજાવો.}

\begin{solutionbox}
સોલિડ સ્ટેટ રિલે (SSRs) વિદ્યુત અલગાવ સાથે સંપર્ક વગરના સ્વિચિંગ માટે ઓપ્ટો-SCRનો ઉપયોગ કરે છે.

\begin{center}
\begin{tikzpicture}[gtu block/.style={draw, rectangle, minimum width=2cm, minimum height=1cm, align=center}, node distance=3cm]
    \node[gtu block] (in) {Control \\ Input};
    \node[gtu block, right of=in] (iso) {Opto-Iso \\ (LED+SCR)};
    \node[gtu block, right of=iso] (sw) {Thyristor \\ Switch};
    \node[right of=sw] (load) {Load};
    
    \draw[->] (in) -- (iso);
    \draw[->] (iso) -- (sw);
    \draw[->] (sw) -- (load);
\end{tikzpicture}
\captionof{figure}{SSR બ્લોક ડાયાગ્રામ}
\end{center}

\begin{center}
\begin{tabulary}{\linewidth}{|L|L|L|}
\hline \textbf{સ્ટેજ} & \textbf{કાર્ય} & \textbf{લાભ} \\ \hline
ઇનપુટ સ્ટેજ & કંટ્રોલ સિગ્નલનો ઉપયોગ કરીને LED ચલાવે છે & ઓછી શક્તિ નિયંત્રણ \\
અલગાવ & પ્રકાશ વિદ્યુત અંતર પુલ કરે છે & સુરક્ષા અને અવાજ પ્રતિરક્ષા \\
ટ્રિગરિંગ & પ્રકાશ SCRને સક્રિય કરે છે & યાંત્રિક સંપર્કો નથી \\
સ્વિચિંગ & થાઇરિસ્ટર લોડ કરંટનું વહન કરે છે & આર્કિંગ કે સંપર્ક ઘસારો નથી \\
\hline
\end{tabulary}
\captionof{table}{SSR ઓપરેશન}
\end{center}

\begin{itemize}
    \item \keyword{મૌન ઓપરેશન}: સ્વિચિંગ દરમિયાન કોઈ યાંત્રિક અવાજ નથી
    \item \keyword{લાંબુ આયુષ્ય}: ઇલેક્ટ્રોમેકેનિકલ રિલેની જેમ સંપર્ક અવનતિ નથી
\end{itemize}
\end{solutionbox}
\mnemonicbox{પ્રકાશ લોજિકને લોડ સાથે જોડે છે}

\questionmarks{2(b) OR}{4}{સ્નબર સર્કિટ વ્યાખ્યાયિત કરો અને સ્નબર સર્કિટનું મહત્વ સમજાવો.}

\begin{solutionbox}
સ્નબર સર્કિટ એ સુરક્ષાત્મક નેટવર્ક છે જે સ્વિચિંગ ઉપકરણોમાં વોલ્ટેજ અને કરંટ ક્ષણિકોને દબાવે છે.

\begin{center}
\begin{tikzpicture}
    \draw (0,0) node[thyristor] (S) {};
    \draw (S.north) -- (0, 2);
    \draw (S.south) -- (0, -2);
    
    % Snubber R-C
    \draw (0,1.5) -- (2,1.5) to[R, l=$R_s$] (2,0) to[C, l=$C_s$] (2,-1.5) -- (0,-1.5);
\end{tikzpicture}
\captionof{figure}{RC સ્નબર સર્કિટ}
\end{center}

\begin{center}
\begin{tabulary}{\linewidth}{|L|L|L|}
\hline \textbf{કાર્ય} & \textbf{લાભ} & \textbf{અમલીકરણ} \\ \hline
dV/dt દમન & ખોટા ટ્રિગરિંગને રોકે છે & SCR આસપાસ RC સર્કિટ \\
વોલ્ટેજ સ્પાઇક ઘટાડો & ઓવરવોલ્ટેજથી રક્ષણ & કેપેસિટર ઊર્જા શોષે છે \\
ઓસીલેશન ડેમ્પિંગ & EMI ઘટાડે છે & રેસિસ્ટર ડેમ્પિંગ પ્રદાન કરે છે \\
ટર્ન-ઓફ સહાય & કોમ્યુટેશન સુધારે છે & ટર્ન-ઓફ દરમિયાન પ્રવાહ વાળે છે \\
\hline
\end{tabulary}
\captionof{table}{સ્નબર મહત્વ}
\end{center}

\begin{itemize}
    \item \keyword{સર્કિટ સુરક્ષા}: ઉપકરણ પર તણાવને મર્યાદિત કરીને થાઇરિસ્ટરનું આયુષ્ય વધારે છે
    \item \keyword{અવાજ ઘટાડો}: આસપાસની સર્કિટોમાં ઇલેક્ટ્રોમેગ્નેટિક ઇન્ટરફેરન્સ ઘટાડે છે
\end{itemize}
\end{solutionbox}
\mnemonicbox{અવાજ દબાવો, સંતુલિત વર્તન સરળતાથી પુનઃસ્થાપિત થાય}

\questionmarks{2(c) OR}{7}{SCR ની વિવિધ કોમ્યુટેશન પદ્ધતિઓની યાદી બનાવો અને તેમાંથી કોઈપણ બે સમજાવો}

\begin{solutionbox}
કોમ્યુટેશન એ એનોડ પ્રવાહને હોલ્ડિંગ વેલ્યુ નીચે ઘટાડીને SCRને બંધ કરવાની પ્રક્રિયા છે.

\begin{center}
\begin{tabulary}{\linewidth}{|L|L|L|}
\hline \textbf{પદ્ધતિ} & \textbf{સિદ્ધાંત} & \textbf{ઉપયોગો} \\ \hline
નૈસર્ગિક & AC શૂન્ય ક્રોસિંગ & AC પાવર કંટ્રોલ \\
ફોર્સ્ડ & બાહ્ય સર્કિટ & DC એપ્લિકેશન \\
વર્ગ A & LC રેઝોનન્સ & ઇન્વર્ટર \\
વર્ગ B & ઓક્ઝિલરી SCR & DC ચોપર \\
વર્ગ C & લોડ સાથે LC & વેરિએબલ ફ્રિક્વન્સી \\
વર્ગ D & ઓક્ઝિલરી સ્ત્રોત & મોટર કંટ્રોલ \\
વર્ગ E & બાહ્ય પલ્સ & ઇલેક્ટ્રોનિક સર્કિટ \\
\hline
\end{tabulary}
\captionof{table}{કોમ્યુટેશન પદ્ધતિઓ}
\end{center}

\textbf{1. નૈસર્ગિક કોમ્યુટેશન:}
\begin{center}
\begin{tikzpicture}
    \draw (0,0) to[sV, l=AC] (0,2) -- (2,2) to[thyristor] (2,0) to[R, l=Load] (0,0);
\end{tikzpicture}
\captionof{figure}{નૈસર્ગિક કોમ્યુટેશન}
\end{center}
\begin{itemize}
    \item \keyword{શૂન્ય ક્રોસિંગ}: જ્યારે AC શૂન્ય પાર કરે છે અને એનોડ કરંટ હોલ્ડિંગથી નીચે પડે છે ત્યારે SCR બંધ થાય છે
    \item \keyword{સરળતા}: કોમ્યુટેશન માટે કોઈ વધારાના ઘટકોની જરૂર નથી
    \item \keyword{મર્યાદા}: ફક્ત AC સર્કિટમાં નિશ્ચિત આવૃત્તિ પર કામ કરે છે
\end{itemize}

\textbf{2. ફોર્સ્ડ કોમ્યુટેશન (વર્ગ B):}
\begin{center}
\begin{tikzpicture}
    \draw (0,3) -- (2,3) node[above] {Vdc} -- (4,3);
    \draw (2,3) to[thyristor, l=SCR1] (2,0);
    \draw (4,3) to[C] (4,1.5) -- (2,1.5); % Capacitor coupling
    \draw (4,3) -- (6,3) to[thyristor, l=SCR2] (6,0);
    \draw (2,0) to[R, l=Load] (2,-2) node[ground]{};
    \draw (6,0) -- (6,-2) node[ground]{};
\end{tikzpicture}
\captionof{figure}{વર્ગ B કોમ્યુટેશન}
\end{center}
\begin{itemize}
    \item \keyword{ઓક્ઝિલરી SCR}: બીજું SCR (SCR2) મુખ્ય SCRને રિવર્સ બાયસ કરવા કેપેસિટર ડિસ્ચાર્જ કરે છે
    \item \keyword{ટાઇમિંગ કંટ્રોલ}: SCR ક્યારે બંધ થાય તેના પર ચોક્કસ નિયંત્રણ
    \item \keyword{એપ્લિકેશન}: DC સર્કિટમાં વપરાય છે જ્યાં નૈસર્ગિક કોમ્યુટેશન શક્ય નથી
\end{itemize}
\end{solutionbox}
\mnemonicbox{પ્રકૃતિ પ્રવાહને અનુસરે છે, ફોર્સ્ડ પ્રવાહ કોલેપ્સ બનાવે છે}

\questionmarks{3(a)}{3}{સિંગલ ફેઝ રેક્ટિફાયર કરતાં પોલિફેસ રેક્ટિફાયરના ફાયદા સમજાવો.}

\begin{solutionbox}
પોલિફેઝ રેક્ટિફાયર પાવર એપ્લિકેશનમાં સિંગલ-ફેઝ ડિઝાઇન કરતાં નોંધપાત્ર સુધારા આપે છે.

\begin{center}
\begin{tabulary}{\linewidth}{|L|L|L|}
\hline \textbf{પેરામીટર} & \textbf{સિંગલ ફેઝ} & \textbf{પોલિફેઝ} \\ \hline
રિપલ ફેક્ટર & ઊંચો (FW માટે 0.482) & નીચો (3-ફેઝ માટે 0.042) \\
ફોર્મ ફેક્ટર & ઊંચો & નીચો \\
કાર્યક્ષમતા & ઓછી & ઊંચી (ટ્રાન્સફોર્મર વધુ સારી રીતે વપરાય છે) \\
પાવર રેટિંગ & મર્યાદિત & ઊંચું પાવર હેન્ડલિંગ \\
હાર્મોનિક કન્ટેન્ટ & વધુ & ઓછું (વધુ સરળ DC) \\
\hline
\end{tabulary}
\captionof{table}{સિંગલ ફેઝ vs પોલિફેઝ રેક્ટિફાયર}
\end{center}

\begin{itemize}
    \item \keyword{આઉટપુટ સ્મૂધનેસ}: નોંધપાત્ર રીતે ઓછો રિપલ જેને નાના ફિલ્ટરિંગ ઘટકોની જરૂર પડે છે
    \item \keyword{ટ્રાન્સફોર્મર ઉપયોગ}: વધુ સારો ઉપયોગ ફેક્ટર (0.955 vs 0.812) ટ્રાન્સફોર્મર કદ ઘટાડે છે
\end{itemize}
\end{solutionbox}
\mnemonicbox{વધુ ફેઝ એટલે વધુ સરળ પાવર}

\questionmarks{3(b)}{4}{UPS પર ટૂંકી નોંધ લખો.}

\begin{solutionbox}
UPS (અનઇન્ટરપ્ટિબલ પાવર સપ્લાય) મુખ્ય પાવર સપ્લાય નિષ્ફળ થાય ત્યારે સતત પાવર પ્રદાન કરે છે.

\begin{center}
\begin{tikzpicture}[gtu block/.style={draw, rectangle, minimum width=2cm, minimum height=1cm, align=center}, node distance=3cm]
    \node[gtu block] (rect) {Rectifier};
    \node[gtu block, right of=rect] (dc) {DC Bus};
    \node[gtu block, right of=dc] (inv) {Inverter};
    \node[gtu block, below of=dc] (batt) {Battery};
    
    \draw[->] (-1.5, 0) node[left]{AC In} -- (rect);
    \draw[->] (rect) -- (dc);
    \draw[->] (dc) -- (inv);
    \draw[->] (inv) -- (4.5, 0) node[right]{AC Out};
    \draw[<->] (dc) -- (batt);
\end{tikzpicture}
\captionof{figure}{બેઝિક UPS બ્લોક ડાયાગ્રામ}
\end{center}

\begin{center}
\begin{tabulary}{\linewidth}{|L|L|L|}
\hline \textbf{પ્રકાર} & \textbf{ઓપરેશન} & \textbf{એપ્લિકેશન} \\ \hline
ઓનલાઇન & હંમેશા બેટરી/ઇન્વર્ટર દ્વારા & ક્રિટિકલ સિસ્ટમ, મેડિકલ \\
ઓફલાઇન & નિષ્ફળતા પર બેટરી પર સ્વિચ & પર્સનલ કમ્પ્યુટર, નાના ઓફિસ \\
લાઇન-ઇન્ટરેક્ટિવ & વોલ્ટેજ રેગ્યુલેશન + બેકઅપ & સર્વર, નેટવર્ક ઇક્વિપમેન્ટ \\
\hline
\end{tabulary}
\captionof{table}{UPS પ્રકારો}
\end{center}

\begin{itemize}
    \item \keyword{બેકઅપ સમય}: બેટરી ક્ષમતા પર આધાર રાખીને સામાન્ય રીતે 5-30 મિનિટ
    \item \keyword{સુરક્ષા}: સર્જ પ્રોટેક્શન, વોલ્ટેજ રેગ્યુલેશન, અને ફ્રિક્વન્સી સ્ટેબિલાઇઝેશન
\end{itemize}
\end{solutionbox}
\mnemonicbox{પાવર સતત સ્વિચ હેઠળ સુરક્ષિત}

\questionmarks{3(c)}{7}{ઇન્વર્ટરનું કાર્ય આપો અને ઇન્વર્ટરના મૂળભૂત સિદ્ધાંતને સમજાવો પણ સુઘડ ડાયાગ્રામ અને વેવફોર્મ સાથે શ્રેણી ઇન્વર્ટર સમજાવો.}

\begin{solutionbox}
ઇન્વર્ટર ડીસી પાવરને એસી પાવરમાં રૂપાંતરિત કરે છે, ડીસીને ટ્રાન્સફોર્મર દ્વારા કે સીધા જ સ્વિચ કરીને વૈકલ્પિક તરંગ બનાવે છે.

\begin{center}
\begin{tabulary}{\linewidth}{|L|L|}
\hline \textbf{કાર્ય} & \textbf{વર્ણન} \\ \hline
DC થી AC રૂપાંતરણ & સ્થિર DC ને વૈકલ્પિક AC માં રૂપાંતરિત કરે છે \\
આવૃત્તિ નિયંત્રણ & ચલિત આવૃત્તિ આઉટપુટ ઉત્પન્ન કરે છે \\
વોલ્ટેજ નિયંત્રણ & લોડ વેરિએશન છતાં સ્થિર આઉટપુટ જાળવે છે \\
વેવ શેપિંગ & સાઇન, સ્ક્વેર, કે મોડિફાઇડ સાઇન વેવ્સ ઉત્પન્ન કરે છે \\
\hline
\end{tabulary}
\captionof{table}{ઇન્વર્ટર કાર્યો}
\end{center}

\textbf{શ્રેણી ઇન્વર્ટર સર્કિટ:}
\begin{center}
\begin{tikzpicture}
    % Series Inverter
    \draw (0,4) -- (4,4) node[right]{+Vdc};
    \draw (0,0) -- (4,0) node[right]{GND};
    
    \draw (2,4) -- (2,3) to[L, l=$L$] (2,2) to[C, l=$C$] (2,1) -- (2,0.5);
    \draw (2,0.5) node[thyristor] (T) {};
    \draw (T.south) -- (2, -1) to[R, l=Load] (2,-2) node[ground]{};
\end{tikzpicture}
\captionof{figure}{શ્રેણી ઇન્વર્ટર}
\end{center}
\begin{itemize}
    \item \keyword{ઓસીલેશન}: SCR ટ્રિગર થતાં શ્રેણી LC સર્કિટ રેઝોનન્ટ ઓસીલેશન બનાવે છે
    \item \keyword{કોમ્યુટેશન}: રેઝોનન્સ દ્વારા કરંટ રિવર્સ થાય ત્યારે SCR આપમેળે બંધ થાય છે
    \item \keyword{આવૃત્તિ}: LC વેલ્યુ દ્વારા નક્કી થાય છે: $f = 1/(2\pi\sqrt{LC})$
\end{itemize}
\end{solutionbox}
\mnemonicbox{ડાયરેક્ટ કરંટ સ્વિચ થઈને રેઝોનન્ટ સર્કિટ દ્વારા ઓલ્ટરનેટિંગ કરંટ બને છે}

\questionmarks{3(a) OR}{3}{ચોપરના મૂળ સિદ્ધાંતને સમજાવો.}

\begin{solutionbox}
ચોપર એ DC-થી-DC કન્વર્ટર છે જે નિયંત્રિત સરેરાશ DC આઉટપુટ ઉત્પન્ન કરવા માટે DC ઇનપુટને ચાલુ/બંધ કરે છે.

\begin{center}
\begin{tikzpicture}
    \draw (0,2) to[battery, l=$V_{dc}$] (0,0);
    \draw (0,2) -- (2,2) node[switch] (S) {} -- (4,2);
    \draw (4,2) to[R, l=Load] (4,0) -- (0,0);
    \node at (2,2.5) {Switch (Chopper)};
\end{tikzpicture}
\captionof{figure}{બેઝિક ચોપર સર્કિટ}
\end{center}

\begin{center}
\begin{tabulary}{\linewidth}{|L|L|L|}
\hline \textbf{પેરામીટર} & \textbf{સંબંધ} & \textbf{નિયંત્રણ} \\ \hline
આઉટપુટ વોલ્ટેજ & $V_o = V_{dc} \times (T_{on}/T)$ & ડ્યુટી સાયકલ એડજસ્ટમેન્ટ \\
ડ્યુટી સાયકલ & $k = T_{on}/T$ & આઉટપુટ વોલ્ટેજ નિયંત્રિત કરે છે \\
આવૃત્તિ & $f = 1/T$ & રિપલ પર અસર કરે છે \\
વોલ્ટેજ રેગ્યુલેશન & લોડ સાથે બદલાય છે & ફીડબેક કંટ્રોલ ડ્યુટી સાયકલ એડજસ્ટ કરે છે \\
\hline
\end{tabulary}
\captionof{table}{ચોપર સિદ્ધાંત}
\end{center}

\begin{itemize}
    \item \keyword{સ્વિચિંગ એક્શન}: DC ઇનપુટને ચોપ કરવા માટે ઝડપથી ON/OFF થાય છે
    \item \keyword{પલ્સ વિડ્થ મોડ્યુલેશન}: ON-ટાઇમ રેશિઓને બદલીને વોલ્ટેજ નિયંત્રિત કરે છે
\end{itemize}
\end{solutionbox}
\mnemonicbox{ચોપિંગ નિયંત્રિત DC બનાવે છે}

\questionmarks{3(b) OR}{4}{SMPS ના બ્લોક ડાયાગ્રામ દોરો અને દરેક બ્લોકનું કાર્ય સમજાવો.}

\begin{solutionbox}
SMPS (સ્વિચ્ડ મોડ પાવર સપ્લાય) ઉચ્ચ-આવૃત્તિ સ્વિચિંગનો ઉપયોગ કરીને ઇનપુટ પાવરને નિયંત્રિત આઉટપુટમાં રૂપાંતરિત કરે છે.

\begin{center}
\begin{tikzpicture}[gtu block/.style={draw, rectangle, minimum width=2cm, minimum height=1.2cm, align=center}, node distance=2.5cm, auto]
    \node[gtu block] (rect) {Rectifier \\ \& Filter};
    \node[gtu block, right of=rect] (switch) {HF \\ Switching};
    \node[gtu block, right of=switch] (trans) {Trans- \\ former};
    \node[gtu block, right of=trans] (outrect) {Output \\ Rectifier};
    \node[right of=outrect] (load) {DC Out};
    
    \node[gtu block, below of=switch] (ctrl) {Control \\ PWM};
    
    \draw[->] (-1.5, 0) node[left]{AC} -- (rect);
    \draw[->] (rect) -- (switch);
    \draw[->] (switch) -- (trans);
    \draw[->] (trans) -- (outrect);
    \draw[->] (outrect) -- (load);
    \draw[->] (outrect) |- (ctrl);
    \draw[->] (ctrl) -- (switch);
\end{tikzpicture}
\captionof{figure}{SMPS બ્લોક ડાયાગ્રામ}
\end{center}

\begin{center}
\begin{tabulary}{\linewidth}{|L|L|}
\hline \textbf{બ્લોક} & \textbf{કાર્ય} \\ \hline
EMI ફિલ્ટર & SMPSમાં પ્રવેશતા/છોડતા અવાજને દબાવે છે \\
રેક્ટિફાયર અને ફિલ્ટર & ACને અનિયમિત DCમાં રૂપાંતરિત કરે છે \\
સ્વિચિંગ સર્કિટ & ઉચ્ચ આવૃત્તિ (20-200kHz) પર DC ચોપ કરે છે \\
ટ્રાન્સફોર્મર & અલગાવ અને વોલ્ટેજ ટ્રાન્સફોર્મેશન પ્રદાન કરે છે \\
આઉટપુટ રેક્ટિફાયર & ઉચ્ચ-આવૃત્તિ ACને પાછો DCમાં રૂપાંતરિત કરે છે \\
આઉટપુટ ફિલ્ટર & DC આઉટપુટને સ્મૂધ કરે છે અને રિપલ દૂર કરે છે \\
ફીડબેક કંટ્રોલ & ડ્યુટી સાયકલ એડજસ્ટ કરીને આઉટપુટ નિયંત્રિત કરે છે \\
\hline
\end{tabulary}
\captionof{table}{SMPS બ્લોક કાર્યો}
\end{center}

\begin{itemize}
    \item \keyword{ઉચ્ચ કાર્યક્ષમતા}: લિનિયર સપ્લાય માટે 30-60\% ની સરખામણીએ 70-90\%
    \item \keyword{નાનું કદ}: ઉચ્ચ આવૃત્તિ નાના ટ્રાન્સફોર્મર અને ઘટકોની મંજૂરી આપે છે
\end{itemize}
\end{solutionbox}
\mnemonicbox{ફિલ્ટર, રેક્ટિફાય, ટ્રાન્સફોર્મર મારફતે સ્વિચ, રેક્ટિફાય, ફિલ્ટર}

\questionmarks{3(c) OR}{7}{વેવફોર્મ સાથે 1 ફેઝ હાફ વેવ રેક્ટિફાયર સમજાવો પણ વેવફોર્મ સાથે 3 ફેઝ ફુલ વેવ રેક્ટિફાયર સમજાવો.}

\begin{solutionbox}
રેક્ટિફાયર એક દિશામાં પ્રવાહની મંજૂરી આપીને અને રિવર્સ ફ્લોને અવરોધીને AC થી DC માં રૂપાંતરિત કરે છે.

\textbf{1-ફેઝ હાફ વેવ રેક્ટિફાયર:}
\begin{center}
\begin{tikzpicture}
    \draw (0,0) to[sV, l=AC] (0,2) to[D] (2,2) to[R, l=Load] (2,0) -- (0,0);
\end{tikzpicture}
\end{center}

\textbf{3-ફેઝ ફુલ વેવ રેક્ટિફાયર:}
\begin{center}
\begin{tikzpicture}
    % 3 Phase bridge
    \draw (0,0) -- (0,4);
    \draw (2,0) -- (2,4);
    \draw (4,0) -- (4,4);
    
    % Diodes on top
    \draw (0,4) to[D] (1,5) -- (5,5);
    \draw (2,4) to[D] (3,5) -- (5,5);
    \draw (4,4) to[D] (5,5) -- (6,5) node[right]{+Vdc};
    
    % Diodes on bottom
    \draw (0,0) to[D, invert] (1,-1) -- (5,-1);
    \draw (2,0) to[D, invert] (3,-1) -- (5,-1);
    \draw (4,0) to[D, invert] (5,-1) -- (6,-1) node[right]{-Vdc};
    
    % Inputs
    \node at (0,2) {R};
    \node at (2,2) {Y};
    \node at (4,2) {B};
\end{tikzpicture}
\end{center}

\begin{center}
\begin{tabulary}{\linewidth}{|L|L|L|}
\hline \textbf{પેરામીટર} & \textbf{1-ફેઝ હાફ વેવ} & \textbf{3-ફેઝ ફુલ વેવ} \\ \hline
રિપલ ફેક્ટર & 1.21 & 0.042 \\
રેક્ટિફિકેશન કાર્યક્ષમતા & 40.6\% & 95.5\% \\
TUF & 0.287 & 0.955 \\
પીક ઇન્વર્સ વોલ્ટેજ & $V_m$ & $2.09V_m$ \\
ફોર્મ ફેક્ટર & 1.57 & 1.0007 \\
\hline
\end{tabulary}
\captionof{table}{રેક્ટિફાયર તુલના}
\end{center}

\begin{itemize}
    \item \keyword{1-ફેઝ હાફ વેવ}: સૌથી સરળ ડિઝાઇન પરંતુ ઉચ્ચ રિપલ અને ઓછી કાર્યક્ષમતા સાથે
    \item \keyword{3-ફેઝ ફુલ વેવ}: એક ચક્ર દીઠ 6 પલ્સ સાથે ઘણો સરળ આઉટપુટ
\end{itemize}
\end{solutionbox}
\mnemonicbox{અર્ધ માત્ર શિખરો પસાર કરે છે, ત્રણ ફેઝ ખીણો ભરે છે}

\questionmarks{4(a)}{3}{બ્લોક ડાયાગ્રામ સાથે સૌર ફોટોવોલ્ટેઇક આધારિત પાવર જનરેશનની કામગીરીનું વર્ણન કરો.}

\begin{solutionbox}
સોલર PV પાવર જનરેશન ફોટોવોલ્ટાઇક ઇફેક્ટ દ્વારા સૂર્યપ્રકાશને સીધો વિદ્યુતમાં રૂપાંતરિત કરે છે.

\begin{center}
\begin{tikzpicture}[gtu block/.style={draw, rectangle, minimum width=2cm, minimum height=1cm, align=center}, node distance=2.5cm]
    \node[gtu block] (solar) {Solar \\ Array};
    \node[gtu block, right of=solar] (charge) {Charge \\ Controller};
    \node[gtu block, below of=charge] (batt) {Battery};
    \node[gtu block, right of=charge] (inv) {Inverter};
    \node[right of=inv] (ac) {AC Load};
    \node[above of=charge] (dc) {DC Load};
    
    \draw[->] (solar) -- (charge);
    \draw[<->] (charge) -- (batt);
    \draw[->] (charge) -- (inv);
    \draw[->] (inv) -- (ac);
    \draw[->] (charge) -- (dc);
\end{tikzpicture}
\captionof{figure}{સોલર PV સિસ્ટમ}
\end{center}

\begin{center}
\begin{tabulary}{\linewidth}{|L|L|}
\hline \textbf{ઘટક} & \textbf{કાર્ય} \\ \hline
સોલર પેનલ & સૂર્યપ્રકાશને DC વિદ્યુતમાં રૂપાંતરિત કરે છે \\
ચાર્જ કંટ્રોલર & ચાર્જિંગને નિયંત્રિત કરે છે, ઓવરચાર્જ અટકાવે છે \\
બેટરી બેંક & પછીના ઉપયોગ માટે ઊર્જા સંગ્રહિત કરે છે \\
ઇન્વર્ટર & ઘરેલું ઉપકરણો માટે DC ને AC માં રૂપાંતરિત કરે છે \\
ડિસ્ટ્રિબ્યુશન પેનલ & વિદ્યુતને લોડ તરફ રૂટ કરે છે \\
\hline
\end{tabulary}
\captionof{table}{PV ઘટકો}
\end{center}

\begin{itemize}
    \item \keyword{ઊર્જા રૂપાંતરણ}: ફોટોન્સ અર્ધવાહક સામગ્રીમાં ઇલેક્ટ્રોનને ઉત્તેજિત કરીને પ્રવાહ બનાવે છે
    \item \keyword{સ્કેલેબિલિટી}: પાવર જરૂરિયાતો અનુસાર સિસ્ટમનું કદ સમાયોજિત કરી શકાય છે
\end{itemize}
\end{solutionbox}
\mnemonicbox{સૂર્યપ્રકાશ વોલ્ટેજ ઉત્પન્ન કરે છે, બેટરી લોડને મદદ કરે છે}

\questionmarks{4(b)}{4}{સ્ટેટિક સ્વીચ તરીકે SCR નો ઉપયોગ સમજાવો.}

\begin{solutionbox}
SCR વિશ્વસનીય અને ઝડપી સ્વિચિંગ માટે કોઈ હલનચલન ભાગો વગરના સોલિડ-સ્ટેટ સ્વિચ તરીકે કાર્ય કરે છે.

\begin{center}
\begin{tikzpicture}
    \draw (0,3) -- (2,3) node[thyristor] (T) {} -- (2,0);
    \draw (0,3) node[left] {+V} -- (0,0) to[R, l=Load] (2,0) node[ground]{};
    \draw (T.west) -- (1, 2.3) node[left] {Trigger};
\end{tikzpicture}
\captionof{figure}{SCR સ્ટેટિક સ્વિચ કન્સેપ્ટ}
\end{center}

\begin{center}
\begin{tabulary}{\linewidth}{|L|L|L|}
\hline \textbf{એપ્લિકેશન} & \textbf{ફાયદો} & \textbf{અમલીકરણ} \\ \hline
પાવર કંટ્રોલ & ચોક્સાઈપૂર્ણ નિયંત્રણ, આર્કિંગ નથી & ફેઝ એંગલ કંટ્રોલ \\
મોટર સ્ટાર્ટિંગ & સરળ એક્સેલરેશન & ક્રમશઃ વોલ્ટેજ વધારો \\
સર્કિટ પ્રોટેક્શન & ઝડપી પ્રતિસાદ & કરંટ સેન્સિંગ ટ્રિગર \\
હીટિંગ કંટ્રોલ & ઊર્જા કાર્યક્ષમ & શૂન્ય-ક્રોસિંગ સ્વિચિંગ \\
\hline
\end{tabulary}
\captionof{table}{સ્ટેટિક સ્વિચ એપ્લિકેશન્સ}
\end{center}

\begin{itemize}
    \item \keyword{લેચિંગ એક્શન}: એકવાર ટ્રિગર થયા પછી, પ્રવાહ હોલ્ડિંગ વેલ્યુથી નીચે પડે ત્યાં સુધી વહન ચાલુ રાખે છે
    \item \keyword{ઉચ્ચ વિશ્વસનીયતા}: હલનચલન ભાગોની ગેરહાજરીને કારણે કોઈ યાંત્રિક ઘસારો નથી
\end{itemize}
\end{solutionbox}
\mnemonicbox{સેમિકન્ડક્ટર સ્વિચિંગ ચાલતા લોડને નિયંત્રિત કરે છે}

\questionmarks{4(c)}{7}{ઇન્ડક્શન હીટિંગ અને ડાઇલેક્ટ્રિક હીટિંગના કાર્ય સિદ્ધાંતનું વર્ણન કરો પણ ઇન્ડક્શન હીટિંગ અને ડાઇલેક્ટ્રિક હીટિંગની તુલના આપો.}

\begin{solutionbox}
બંને હીટિંગ પદ્ધતિઓ સીધા સંપર્ક વિના ગરમી ઉત્પન્ન કરવા માટે વિદ્યુતચુંબકીય સિદ્ધાંતોનો ઉપયોગ કરે છે.

\begin{center}
\begin{tikzpicture}
    % Induction
    \node at (0,3) {\textbf{Induction Heating}};
    \draw (0,0) ellipse (1.5 and 0.5);
    \foreach \y in {0.2, 0.4, 0.6} \draw (0,\y) ellipse (1.5 and 0.5);
    \node at (0,1.5) {Coil};
    \fill[gray!30] (-0.5, -0.5) rectangle (0.5, 1);
    \node at (0,0) {Metal};
    
    % Dielectric
    \node at (5,3) {\textbf{Dielectric Heating}};
    \draw (4,0) rectangle (6, 2);
    \fill[yellow!30] (4.2, 0.2) rectangle (5.8, 1.8);
    \node at (5,1) {Dielectric};
    \draw (4,2) -- (4,2.5) -- (6,2.5) -- (6,2); % Plate 1
    \draw (4,0) -- (4,-0.5) -- (6,-0.5) -- (6,0); % Plate 2
\end{tikzpicture}
\captionof{figure}{હીટિંગ સિદ્ધાંતો}
\end{center}

\begin{center}
\begin{tabulary}{\linewidth}{|L|L|L|}
\hline \textbf{પેરામીટર} & \textbf{ઇન્ડક્શન હીટિંગ} & \textbf{ડાઇલેક્ટ્રિક હીટિંગ} \\ \hline
સિદ્ધાંત & એડી કરંટ અને હિસ્ટેરેસિસ & દોલન ક્ષેત્રથી અણુ ઘર્ષણ \\
સામગ્રી & વાહક ધાતુઓ & અવાહક સામગ્રી (પ્લાસ્ટિક, લાકડું) \\
આવૃત્તિ & 1-100 kHz & 10-100 MHz \\
પ્રવેશ & સપાટી અને છીછરી ઊંડાઈ & સામગ્રી દ્વારા એકસરખું \\
કાર્યક્ષમતા & 80-90\% & 50-70\% \\
ઉપયોગો & ધાતુ હાર્ડનિંગ, ઓગાળવું, ફોર્જિંગ & પ્લાસ્ટિક વેલ્ડિંગ, ફૂડ પ્રોસેસિંગ, સૂકવવું \\
\hline
\end{tabulary}
\captionof{table}{હીટિંગ પદ્ધતિઓની તુલના}
\end{center}

\begin{itemize}
    \item \keyword{ઇન્ડક્શન હીટિંગ}: વાહક સામગ્રીમાં એડી કરંટ બનાવતા વિદ્યુતચુંબકીય પ્રેરણ દ્વારા કાર્ય કરે છે
    \item \keyword{ડાઇલેક્ટ્રિક હીટિંગ}: પોલર અણુઓના ઝડપી દોલનનું કારણ બને છે જે આંતરિક ઘર્ષણ અને ગરમી પેદા કરે છે
\end{itemize}
\end{solutionbox}
\mnemonicbox{ઇન્ડક્શન ધાતુઓને ગરમ કરે છે, ડાઇલેક્ટ્રિક્સ બિન-ધાતુઓને ગરમ કરે છે}

\questionmarks{4(a) OR}{3}{ફોટો ડાયોડનો ઉપયોગ કરીને ફોટો ઇલેક્ટ્રિક રિલેના સર્કિટ ડાયાગ્રામ દોરો અને સમજાવો.}

\begin{solutionbox}
ફોટો-ઇલેક્ટ્રિક રિલે આપમેળે સ્વિચિંગ ઓપરેશન નિયંત્રિત કરવા માટે પ્રકાશ શોધનો ઉપયોગ કરે છે.

\begin{center}
\begin{tikzpicture}
    % Photo Relay
    \draw (0,4) -- (4,4) node[right]{Vcc};
    \draw (0,0) -- (4,0) node[right]{GND};
    
    \draw (1,4) to[R, l=$R_1$] (1,2) to[photodiode] (1,0); 
    \draw (3,2) node[npn] (Q) {};
    \draw (1,2) -- (Q.B);
    \draw (Q.E) -- (3,0);
    \draw (3,4) to[L, l=Relay] (Q.C);
\end{tikzpicture}
\captionof{figure}{ફોટો-ઇલેક્ટ્રિક રિલે}
\end{center}

\begin{center}
\begin{tabulary}{\linewidth}{|L|L|L|L|}
\hline \textbf{પ્રકાશ સ્થિતિ} & \textbf{ફોટોડાયોડ સ્થિતિ} & \textbf{ટ્રાન્ઝિસ્ટર સ્થિતિ} & \textbf{રિલે એક્શન} \\ \hline
અંધારું & ઉચ્ચ અવરોધ & બંધ & ડી-એનર્જાઇઝ્ડ \\
પ્રકાશ & ઓછો અવરોધ (વહન કરે છે) & ચાલુ & એનર્જાઇઝ્ડ \\
\hline
\end{tabulary}
\captionof{table}{રિલે ઓપરેશન}
\end{center}

\begin{itemize}
    \item \keyword{પ્રકાશ શોધ}: પ્રકાશિત થયેલ ફોટોડાયોડ વહન કરે છે, ટ્રાન્ઝિસ્ટર પર બાયસ બદલે છે
    \item \keyword{સ્વિચિંગ}: ટ્રાન્ઝિસ્ટર રિલે કોઇલ ચલાવવા માટે નાના ફોટોડાયોડ પ્રવાહને વધારે છે
\end{itemize}
\end{solutionbox}
\mnemonicbox{પ્રકાશ ડાયોડને ચલાવે છે, ડાયોડ ટ્રાન્ઝિસ્ટરને ચલાવે છે, ટ્રાન્ઝિસ્ટર રિલેને ચલાવે છે}

\questionmarks{4(b) OR}{4}{DIAC-TRIAC નો ઉપયોગ કરીને AC પાવર કંટ્રોલનો સર્કિટ ડાયાગ્રામ દોરો અને તેને સમજાવો.}

\begin{solutionbox}
DIAC-TRIAC સર્કિટ ફેઝ એંગલ એડજસ્ટમેન્ટ દ્વારા AC પાવરને સરળ રીતે નિયંત્રિત કરવા દે છે.

\begin{center}
\begin{tikzpicture}
    \draw (0,4) to[sV, l=AC] (0,0);
    \draw (0,4) to[R, l=Load] (2,4) -- (4,4);
    \draw (4,4) to[triac] (4,0) -- (0,0);
    
    % Trigger
    \draw (2,4) to[R, l=$R_1$] (2,2) to[C, l=$C$] (2,0);
    \draw (2,2) to[D] (3,2) -- (3,1.3); \node at (2.5, 2.3) {DIAC};
\end{tikzpicture}
\captionof{figure}{DIAC-TRIAC પાવર કંટ્રોલ}
\end{center}

\begin{center}
\begin{tabulary}{\linewidth}{|L|L|}
\hline \textbf{ઘટક} & \textbf{કાર્ય} \\ \hline
$R_1-C$ & ફેઝ વિલંબ માટે વેરિએબલ ટાઇમ કોન્સ્ટન્ટ \\
DIAC & કેપેસિટર વોલ્ટેજ બ્રેકઓવર પહોંચે ત્યારે TRIAC ટ્રિગર કરે છે \\
TRIAC & ટ્રિગરિંગ પોઇન્ટ પર આધારિત લોડ કરંટ નિયંત્રિત કરે છે \\
લોડ & ફેઝ કંટ્રોલ પર આધારિત આંશિક AC વેવફોર્મ પ્રાપ્ત કરે છે \\
\hline
\end{tabulary}
\captionof{table}{સર્કિટ ઘટકો}
\end{center}

\begin{itemize}
    \item \keyword{ફેઝ કંટ્રોલ}: RC નેટવર્ક AC સાયકલની અંદર ટ્રિગરિંગ પોઇન્ટમાં વિલંબ બનાવે છે
    \item \keyword{દ્વિદિશીય ઓપરેશન}: AC સાયકલના બંને અર્ધ પર કામ કરે છે
\end{itemize}
\end{solutionbox}
\mnemonicbox{વિલંબ કેપેસિટર પર શરૂ થાય છે, વિશ્વસનીય સ્વતંત્ર AC કંટ્રોલ ટ્રિગર કરે છે}

\questionmarks{4(c) OR}{7}{વેવફોર્મ સાથે કામ કરતા IC555 ત્રણ તબક્કાના ક્રમિક ટાઈમરને સમજાવો.}

\begin{solutionbox}
ત્રણ-તબક્કાનો ક્રમિક ટાઇમર પ્રક્રિયા નિયંત્રણ માટે સમયબદ્ધ ક્રમ બનાવવા માટે બહુવિધ 555 ICનો ઉપયોગ કરે છે.

\begin{center}
\begin{tikzpicture}[gtu block/.style={draw, rectangle, minimum width=1.5cm, minimum height=1cm}, node distance=2cm]
    \node[gtu block] (T1) {Timer 1};
    \node[gtu block, right of=T1] (T2) {Timer 2};
    \node[gtu block, right of=T2] (T3) {Timer 3};
    
    \draw[->] (-1,0) node[left]{Trig} -- (T1);
    \draw[->] (T1) -- (T2);
    \draw[->] (T2) -- (T3);
    
    \draw (T1) -- (0,1) node[above]{Out1};
    \draw (T2) -- (2,1) node[above]{Out2};
    \draw (T3) -- (4,1) node[above]{Out3};
\end{tikzpicture}
\captionof{figure}{ક્રમિક ટાઈમર લોજિક}
\end{center}

\begin{center}
\begin{tabulary}{\linewidth}{|L|L|L|L|}
\hline \textbf{તબક્કો} & \textbf{ક્રિયા} & \textbf{અવધિ} & \textbf{આગલા તબક્કા ટ્રિગર} \\ \hline
પ્રારંભિક & બધા આઉટપુટ્સ LOW & - & બાહ્ય ટ્રિગર \\
તબક્કો 1 & આઉટપુટ 1 HIGH & T1 ($R_1 C_1$) & આઉટપુટ 1 ફોલિંગ એજ \\
તબક્કો 2 & આઉટપુટ 2 HIGH & T2 ($R_2 C_2$) & આઉટપુટ 2 ફોલિંગ એજ \\
તબક્કો 3 & આઉટપુટ 3 HIGH & T3 ($R_3 C_3$) & આઉટપુટ 3 ફોલિંગ એજ \\
રીસેટ & બધા આઉટપુટ્સ LOW & T4 (રીસેટ સમય) & નવો બાહ્ય ટ્રિગર \\
\hline
\end{tabulary}
\captionof{table}{ક્રમિક ટાઈમિંગ}
\end{center}

\begin{itemize}
    \item \keyword{કેસ્કેડિંગ કનેક્શન}: પહેલા ટાઇમરનો આઉટપુટ બીજાને ટ્રિગર કરે છે, અને આ રીતે આગળ વધે છે
    \item \keyword{ટાઇમિંગ કંટ્રોલ}: RC વેલ્યુ સાથે દરેક તબક્કાનો સમયગાળો સ્વતંત્ર રીતે સમાયોજિત કરી શકાય છે
\end{itemize}
\end{solutionbox}
\mnemonicbox{પ્રથમ તબક્કો સમાપ્ત થાય, બીજો શરૂ થાય, ત્રીજો અનુસરે}

\questionmarks{5(a)}{3}{ડીસી શંટ મોટરના સોલિડ સ્ટેટ કંટ્રોલ દોરો અને સમજાવો.}

\begin{solutionbox}
સોલિડ-સ્ટેટ DC મોટર કંટ્રોલ મોટરને આપવામાં આવતા વોલ્ટેજને નિયંત્રિત કરવા માટે SCRનો ઉપયોગ કરે છે.

\begin{center}
\begin{tikzpicture}
    % Bridge Rectifier feeding Motor
    \draw (0,0) node[draw, rectangle] (rect) {Bridge Rectifier};
    \draw (-2, 0) node[left]{AC} -- (rect);
    \draw (rect) -- (2,0) node[thyristor] (T) {};
    \draw (T.east) -- (4,0) node[draw, circle] (M) {M};
    \node at (4, -1) {DC Shunt Motor};
    
    % Field
    \draw (3, 1) to[L, l=Field] (5,1);
\end{tikzpicture}
\captionof{figure}{સોલિડ સ્ટેટ DC મોટર કંટ્રોલ}
\end{center}

\begin{center}
\begin{tabulary}{\linewidth}{|L|L|L|}
\hline \textbf{પદ્ધતિ} & \textbf{ઓપરેશન} & \textbf{ફાયદો} \\ \hline
ફેઝ કંટ્રોલ & SCR ફાયરિંગ એંગલ બદલે છે & સરળ ગતિ નિયંત્રણ \\
ચોપર કંટ્રોલ & પલ્સ વિડ્થ મોડ્યુલેશન & ઉચ્ચ કાર્યક્ષમતા \\
ક્લોઝ્ડ-લૂપ & ટેકોમીટરથી ફીડબેક & સચોટ ગતિ નિયમન \\
\hline
\end{tabulary}
\captionof{table}{કંટ્રોલ પદ્ધતિઓ}
\end{center}

\begin{itemize}
    \item \keyword{ગતિ નિયમન}: મોટરની ગતિ બદલવા માટે આર્મેચર વોલ્ટેજ નિયંત્રિત કરે છે
    \item \keyword{ટોર્ક કંટ્રોલ}: કરંટ મર્યાદિત કરીને ઉચ્ચ સ્ટાર્ટિંગ ટોર્ક જાળવે છે
\end{itemize}
\end{solutionbox}
\mnemonicbox{SCR પ્રવાહ નિયંત્રિત કરે છે મોટર પાવર વિતરણ માટે}

\questionmarks{5(b)}{4}{સ્ટેપર મોટરના કામના સિદ્ધાંતને સમજાવો.}

\begin{solutionbox}
સ્ટેપર મોટર્સ વિદ્યુતચુંબકીય સિદ્ધાંતો દ્વારા ડિજિટલ પલ્સને ચોક્કસ યાંત્રિક ફેરફારમાં રૂપાંતરિત કરે છે.

\begin{center}
\begin{tikzpicture}
    \draw (0,0) circle (1.5);
    \foreach \a in {0, 90, 180, 270} \draw (0,0) -- (\a:1.5);
    \node at (0,0) {Rotor};
    \node at (0,2) {Stator Coils};
\end{tikzpicture}
\captionof{figure}{સ્ટેપર મોટર કન્સેપ્ટ}
\end{center}

\begin{center}
\begin{tabulary}{\linewidth}{|L|L|L|}
\hline \textbf{સ્ટેપ પ્રકાર} & \textbf{રોટેશન એંગલ} & \textbf{કંટ્રોલ પદ્ધતિ} \\ \hline
ફુલ સ્ટેપ & સામાન્ય રીતે $1.8^{\circ}$ કે $0.9^{\circ}$ & એક સમયે એક ફેઝ \\
હાફ સ્ટેપ & ફુલ સ્ટેપનો અર્ધો & બે ફેઝ વૈકલ્પિક \\
માઇક્રો-સ્ટેપ & ફુલ સ્ટેપનો અંશ & PWM કરંટ કંટ્રોલ \\
વેવ ડ્રાઇવ & ફુલ સ્ટેપ એંગલ & એક ફેઝ એનર્જાઇઝ્ડ \\
\hline
\end{tabulary}
\captionof{table}{સ્ટેપિંગ મોડ્સ}
\end{center}

\begin{itemize}
    \item \keyword{ડિજિટલ પોઝિશનિંગ}: દરેક પલ્સ મોટરને ચોક્કસ ખૂણે ફેરવે છે
    \item \keyword{હોલ્ડિંગ ટોર્ક}: ફેરફાર વિના સ્થિતિ જાળવે છે જ્યારે એનર્જાઇઝ્ડ હોય
\end{itemize}
\end{solutionbox}
\mnemonicbox{પલ્સ ચોક્કસ સ્થિતિગત સ્ટેપ્સ ઉત્પન્ન કરે છે}

\questionmarks{5(c)}{7}{PLC ના બ્લોક ડાયાગ્રામ દોરો અને દરેક બ્લોકનું કાર્ય સમજાવો.}

\begin{solutionbox}
પ્રોગ્રામેબલ લોજિક કંટ્રોલર (PLC) એ ઓટોમેશન કંટ્રોલ માટેનું ઔદ્યોગિક ડિજિટલ કમ્પ્યુટર છે.

\begin{center}
\begin{tikzpicture}[gtu block/.style={draw, rectangle, minimum width=2.5cm, minimum height=1.2cm}, node distance=3cm]
    \node[gtu block] (cpu) {CPU};
    \node[gtu block, left of=cpu] (in) {Input Module};
    \node[gtu block, right of=cpu] (out) {Output Module};
    \node[gtu block, below of=cpu] (mem) {Memory};
    \node[gtu block, above of=cpu] (pwr) {Power Supply};
    
    \draw[<->] (in) -- (cpu);
    \draw[<->] (cpu) -- (out);
    \draw[<->] (cpu) -- (mem);
    \draw[->] (pwr) -- (cpu);
\end{tikzpicture}
\captionof{figure}{PLC આર્કિટેક્ચર}
\end{center}

\begin{center}
\begin{tabulary}{\linewidth}{|L|L|}
\hline \textbf{ઘટક} & \textbf{કાર્ય} \\ \hline
પાવર સપ્લાય & મુખ્ય પાવરને PLC માટે જરૂરી DC માં રૂપાંતરિત કરે છે \\
CPU & પ્રોગ્રામ ચલાવે છે અને I/O પર આધારિત નિર્ણયો કરે છે \\
મેમરી & પ્રોગ્રામ અને ડેટા સંગ્રહિત કરે છે (ROM, RAM, EEPROM) \\
ઇનપુટ મોડ્યુલ & સેન્સર, સ્વિચ, એન્કોડર સાથે ઇન્ટરફેસ કરે છે \\
આઉટપુટ મોડ્યુલ & એક્ચુએટર, મોટર, વાલ્વ, ઇન્ડિકેટર નિયંત્રિત કરે છે \\
કમ્યુનિકેશન મોડ્યુલ & અન્ય PLC, કમ્પ્યુટર, નેટવર્ક સાથે જોડાય છે \\
પ્રોગ્રામિંગ ડિવાઇસ & PLC પ્રોગ્રામ લખવા, એડિટ કરવા, મોનિટર કરવા માટે વપરાય છે \\
\hline
\end{tabulary}
\captionof{table}{PLC મોડ્યુલ્સ}
\end{center}

\begin{itemize}
    \item \keyword{સ્કેન સાયકલ}: સતત ઇનપુટ વાંચે છે, પ્રોગ્રામ ચલાવે છે, આઉટપુટ અપડેટ કરે છે
    \item \keyword{પ્રોગ્રામિંગ ભાષાઓ}: લેડર લોજિક, ફંક્શન બ્લોક, સ્ટ્રક્ચર્ડ ટેક્સ્ટ, વગેરે
\end{itemize}
\end{solutionbox}
\mnemonicbox{પાવર પ્રોસેસિંગને કેન્દ્રિત કરે છે, ઇનપુટ/આઉટપુટ ઓટોમેશન બનાવે છે}

\questionmarks{5(a) OR}{3}{ડીસી સર્વો મોટરનું બાંધકામ દોરો અને સમજાવો.}

\begin{solutionbox}
DC સર્વો મોટર્સ ઓટોમેશન અને રોબોટિક્સ માટે ફીડબેક સાથે ચોક્કસ પોઝિશન કંટ્રોલ પ્રદાન કરે છે.

\begin{center}
\begin{tikzpicture}
    \draw (0,0) rectangle (3,1.5);
    \node at (1.5, 0.75) {DC Motor};
    \draw (3, 0.75) -- (4, 0.75); % Shaft
    \draw (4, 0.5) rectangle (4.5, 1); % Encoder
    \node[above] at (4.25, 1) {Encoder};
\end{tikzpicture}
\captionof{figure}{DC સર્વો મોટર}
\end{center}

\begin{center}
\begin{tabulary}{\linewidth}{|L|L|}
\hline \textbf{ઘટક} & \textbf{કાર્ય} \\ \hline
આર્મેચર & ચુંબકીય ક્ષેત્રની અંદર ફરે છે \\
ફીલ્ડ મેગ્નેટ્સ & ચુંબકીય ક્ષેત્ર બનાવે છે (ઘણીવાર કાયમી ચુંબક) \\
કમ્યુટેટર & ફરતા આર્મેચરને પાવર ટ્રાન્સફર કરે છે \\
ફીડબેક ડિવાઇસ & પોઝિશન/સ્પીડ ફીડબેક માટે એન્કોડર/ટેકોમીટર \\
બ્રશ & કમ્યુટેટરને પાવર કનેક્ટ કરે છે \\
\hline
\end{tabulary}
\captionof{table}{સર્વો ઘટકો}
\end{center}

\begin{itemize}
    \item \keyword{ઓછી જડતા}: ખાસ ડિઝાઇન ઝડપી એક્સેલરેશન/ડિસેલરેશનની મંજૂરી આપે છે
    \item \keyword{ઉચ્ચ ટોર્ક-ટુ-ઇનર્શિયા રેશિઓ}: કંટ્રોલ સિગ્નલનો ઝડપથી જવાબ આપે છે
\end{itemize}
\end{solutionbox}
\mnemonicbox{ચોક્સાઈભર્યું પોઝિશન ફીડબેક સટીક નિયંત્રણ ચલાવે છે}

\questionmarks{5(b) OR}{4}{BLDC મોટરની કામગીરી સમજાવો.}

\begin{solutionbox}
બ્રશલેસ DC (BLDC) મોટર્સ યાંત્રિક બ્રશ અને કમ્યુટેટરને બદલે ઇલેક્ટ્રોનિક કમ્યુટેશનનો ઉપયોગ કરે છે.

\begin{center}
\begin{tabulary}{\linewidth}{|L|L|}
\hline \textbf{ઘટક} & \textbf{કાર્ય} \\ \hline
સ્ટેટર & ફિક્સ્ડ વાઇન્ડિંગ્સ જે ફરતું ચુંબકીય ક્ષેત્ર ઉત્પન્ન કરે છે \\
રોટર & કાયમી ચુંબક જે ફરતા ક્ષેત્રને અનુસરે છે \\
ઇલેક્ટ્રોનિક કંટ્રોલર & યાંત્રિક કમ્યુટેશનનું સ્થાન લે છે \\
હોલ સેન્સર & સિન્ક્રોનાઇઝ્ડ સ્વિચિંગ માટે રોટર પોઝિશન શોધે છે \\
ડ્રાઇવર સર્કિટ & સ્ટેટર કોઇલ્સમાં પ્રવાહનો ક્રમ પ્રદાન કરે છે \\
\hline
\end{tabulary}
\captionof{table}{BLDC ઘટકો}
\end{center}

\begin{itemize}
    \item \keyword{કમ્યુટેશન}: ઇલેક્ટ્રોનિક સ્વિચિંગ સિક્વન્સ સ્ટેટર વાઇન્ડિંગ્સમાં પાવર આપે છે
    \item \keyword{કાર્યક્ષમતા}: બ્રશ લોસિસના નિર્મૂલનને કારણે ઉચ્ચ કાર્યક્ષમતા
    \item \keyword{વિશ્વસનીયતા}: બ્રશનો ઘસારો કે સ્પાર્કિંગ નથી, લાંબુ આયુષ્ય
\end{itemize}
\end{solutionbox}
\mnemonicbox{ઇલેક્ટ્રોનિક સ્વિચિંગ બ્રશ વગર ફેરફાર બનાવે છે}

\questionmarks{5(c) OR}{7}{VFD નું બાંધકામ અને કાર્ય સમજાવો.}

\begin{solutionbox}
વેરિએબલ ફ્રિક્વન્સી ડ્રાઇવ (VFD) આવૃત્તિ અને વોલ્ટેજમાં ફેરફાર કરીને AC મોટરની ગતિ નિયંત્રિત કરે છે.

\begin{center}
\begin{tikzpicture}[gtu block/.style={draw, rectangle, minimum width=2cm, minimum height=1cm, align=center}, node distance=2.5cm]
    \node[gtu block] (rect) {Rectifier};
    \node[gtu block, right of=rect] (dc) {DC Link};
    \node[gtu block, right of=dc] (inv) {Inverter};
    \node[right of=inv] (motor) {AC Motor};
    
    \draw[->] (-1.5,0) node[left]{AC} -- (rect);
    \draw[->] (rect) -- (dc);
    \draw[->] (dc) -- (inv);
    \draw[->] (inv) -- (motor);
\end{tikzpicture}
\captionof{figure}{VFD બ્લોક ડાયાગ્રામ}
\end{center}

\begin{center}
\begin{tabulary}{\linewidth}{|L|L|L|}
\hline \textbf{વિભાગ} & \textbf{ઘટકો} & \textbf{કાર્ય} \\ \hline
રેક્ટિફાયર & ડાયોડ/SCRs & AC ને DC માં રૂપાંતરિત કરે છે \\
DC બસ & કેપેસિટર, ઇન્ડક્ટર & DC ને ફિલ્ટર અને સ્મૂધ કરે છે \\
ઇન્વર્ટર & IGBTs/ટ્રાન્ઝિસ્ટર & DC ને ચલિત આવૃત્તિ AC માં રૂપાંતરિત કરે છે \\
કંટ્રોલ સર્કિટ & માઇક્રોપ્રોસેસર & સ્વિચિંગ આવૃત્તિ અને પેટર્નને નિયંત્રિત કરે છે \\
કૂલિંગ સિસ્ટમ & ફેન, હીટ સિંક & સુરક્ષિત ઓપરેટિંગ તાપમાન જાળવે છે \\
પ્રોટેક્શન સર્કિટ & સેન્સર, રિલે & ફોલ્ટથી નુકસાન અટકાવે છે \\
\hline
\end{tabulary}
\captionof{table}{VFD માળખું}
\end{center}

\begin{itemize}
    \item \keyword{ગતિ નિયંત્રણ}: સતત ટોર્ક પ્રદાન કરવા માટે V/f રેશિઓ જાળવવામાં આવે છે
    \item \keyword{ઊર્જા બચત}: વાસ્તવિક લોડ જરૂરિયાતો અનુસાર પાવર સમાયોજિત કરે છે
\end{itemize}
\end{solutionbox}
\mnemonicbox{રેક્ટિફાય, ફિલ્ટર, મોટર કંટ્રોલ માટે આવૃત્તિ બદલો}

\end{document}


