\documentclass{article}

% content/resources/templates/preamble.tex
\usepackage[margin=0.6in]{geometry}
\author{Milav Dabgar}
\usepackage{amsmath,amssymb,amsthm}
\usepackage{booktabs}
\usepackage{multirow}
\usepackage{xcolor}
\usepackage{tcolorbox}
\tcbuselibrary{breakable,skins}
\usepackage[colorlinks=true,linkcolor=blue]{hyperref}
\usepackage{titlesec}
\usepackage{enumitem}
\usepackage{tikz}
\usepackage{pgfplots}
\usepackage{circuitikz}
\usepackage[version=4]{mhchem}
\usepackage{longtable}
\usepackage{array}
\usepackage{float}
\usepackage{caption}
\usepackage{listings}

\lstset{
  basicstyle=\small\ttfamily,
  breaklines=true,
  breakatwhitespace=false,
  postbreak=\mbox{\textcolor{red}{$\hookrightarrow$}\space},
  float=false,
  numbers=left,
  numberstyle=\tiny\color{gray},
  numbersep=10pt,
  xleftmargin=2em,
  keywordstyle=\color{blue},
  commentstyle=\color{green!60!black},
  stringstyle=\color{purple},
  backgroundcolor=\color{gray!5},
  showstringspaces=false,
  tabsize=2,
  captionpos=b,
  keepspaces=true,
  columns=flexible
}

\pgfplotsset{compat=1.18}
\usetikzlibrary{shapes,arrows,positioning,calc,patterns,decorations.pathmorphing,decorations.markings,arrows.meta}

% Color scheme
\definecolor{headcolor}{RGB}{0,102,204}
\definecolor{keycolor}{RGB}{220,20,60}
\definecolor{solutioncolor}{RGB}{34,139,34}
\definecolor{mnemoniccolor}{RGB}{148,0,211}
\definecolor{codecolor}{RGB}{0,0,100}

% Spacing
\setlength{\parskip}{3pt}
\setlist[itemize]{nosep}
\setlist[enumerate]{nosep}

% Title formatting
\titleformat{\section}{\Large\bfseries\color{headcolor}}{\thesection}{1em}{}
\titleformat{\subsection}{\large\bfseries\color{headcolor}}{\thesubsection}{1em}{}

% Pandoc tightlist compatibility
\providecommand{\tightlist}{%
  \setlength{\itemsep}{0pt}\setlength{\parskip}{0pt}}

% Pandoc longtable compatibility
\newcounter{none}
\def\thenone{}


% content/resources/templates/gujarati-boxes.tex
\usepackage{fontspec}
\usepackage{polyglossia}

% Set Gujarati as main language (document is primarily in Gujarati)
% Note: gloss-gujarati.ldf doesn't exist in polyglossia, but it will use hyphenation patterns
\setdefaultlanguage{gujarati}
\setotherlanguage{english}

% Configure Gujarati font properly
% Use Language=Default to prevent polyglossia from trying to add language-specific features
% that don't exist for Gujarati, which causes "empty feature" warnings
\newfontfamily\gujaratifont[Script=Gujarati,AutoFakeBold=2.5,AutoFakeSlant=0.3]{Noto Sans Gujarati}
\setmainfont[Script=Gujarati,AutoFakeBold=2.5,AutoFakeSlant=0.3]{Noto Sans Gujarati}
% Use Noto Sans Gujarati for monospace to support Gujarati in text
\setmonofont[Scale=0.9]{Noto Sans Gujarati}

% Configure English to use the same font
\newfontfamily\englishfont[Script=Gujarati,AutoFakeBold=2.5,AutoFakeSlant=0.3]{Noto Sans Gujarati}

% Translations for polyglossia
\gappto\captionsgujarati{
  \renewcommand{\tablename}{કોષ્ટક}
  \renewcommand{\figurename}{આકૃતિ}
}

% Helper for TikZ nodes to ensure Gujarati font
\newcommand{\gu}[1]{{\gujaratifont #1}}

% Custom environments
\newtcolorbox{solutionbox}{
    breakable,
    enhanced,
    colback=solutioncolor!5!white,
    colframe=solutioncolor!75!black,
    fonttitle=\bfseries,
    title=જવાબ
}

\newtcolorbox{solutionboxnobreak}{
 colback=solutioncolor!5!white,
 colframe=solutioncolor!75!black,
 fonttitle=\bfseries,
 title=જવાબ
}

\newtcolorbox{keyformula}{
 breakable,
 enhanced,
 colback=keycolor!5!white,
 colframe=keycolor!75!black,
 fonttitle=\bfseries,
 title=રાસાયણિક સમીકરણ/સૂત્ર
}

\newtcolorbox{mnemonicbox}{
 breakable,
 enhanced,
 colback=mnemoniccolor!5!white,
 colframe=mnemoniccolor!75!black,
 fonttitle=\bfseries,
 title=મેમરી ટ્રીક
}


% Custom commands for GTU solutions
% This file defines semantic commands for consistent formatting

% Question command with automatic formatting
\newcommand{\question}[2]{%
  \section*{Question #1}%
  \textbf{#2}%
}

% OR question variant
\newcommand{\questionor}[2]{%
  \section*{Question #1 OR}%
  \textbf{#2}%
}

% Proper table environment with caption
\newenvironment{answertable}[1]{%
  \begin{table}[htbp]
  \centering
  \caption{#1}
}{%
  \end{table}
}

% Proper figure environment for diagrams
\newenvironment{answerdiagram}[1]{%
  \begin{figure}[htbp]
  \centering
  \caption{#1}
}{%
  \end{figure}
}

% Semantic markup for key terms
\newcommand{\keyword}[1]{\textbf{#1}}
\newcommand{\code}[1]{\texttt{#1}}
\newcommand{\classname}[1]{\texttt{#1}}
\newcommand{\methodname}[1]{\texttt{#1}}

% Proper quotation marks
\newcommand{\mnemonic}[1]{``#1''}


\title{ઔદ્યોગિક ઇલેક્ટ્રોનિક્સ (4331103) - શિયાળો 2023 સોલ્યુશન}
\date{January 18, 2024}

\begin{document}
\maketitle

% ==========================================================================================
% Question 1
% ==========================================================================================
\questionmarks{1(a)}{3}{SCRનો સિમ્બોલ અને રચના દોરો. તદુપરાંત SCRના ઉપયોગો લખો.}

\begin{solutionbox}
\textbf{SCR સિમ્બોલ અને રચના:}

\begin{center}
    \begin{tikzpicture}[auto, node distance=1.5cm]
        % Symbol
        \begin{scope}[shift={(-3,0)}]
            \draw[thick] (0,2) -- (0,1.2); % Anode wire
            \draw[thick] (-0.8,1.2) -- (0.8,1.2); % Top bar
            \draw[thick] (0,1.2) -- (0,-0.5); % Vertical line
            \draw[thick] (-0.8,1.2) -- (0,-0.5) -- (0.8,1.2); % Triangle
            \draw[thick] (-0.8,-0.5) -- (0.8,-0.5); % Cathode bar
            \draw[thick] (0,-0.5) -- (0,-1.5); % Cathode wire
            \draw[thick] (0.4,-0.2) -- (1,-0.5) node[right] {Gate (G)};
            \node[above] at (0,2) {Anode (A)};
            \node[below] at (0,-1.5) {Cathode (K)};
            \node at (0,-2.5) {\textbf{Symbol}};
        \end{scope}

        % Construction
        \begin{scope}[shift={(3,0)}]
            \draw[fill=cyan!10] (-1.5,1.5) rectangle (1.5,2.5) node[midway] {P-Layer};
            \draw[fill=orange!10] (-1.5,0.5) rectangle (1.5,1.5) node[midway] {N-Layer};
            \draw[fill=cyan!10] (-1.5,-0.5) rectangle (1.5,0.5) node[midway] {P-Layer};
            \draw[fill=orange!10] (-1.5,-1.5) rectangle (1.5,-0.5) node[midway] {N-Layer};
            
            \draw[thick] (0,2.5) -- (0,3) node[above] {Anode (A)};
            \draw[thick] (0,-1.5) -- (0,-2) node[below] {Cathode (K)};
            \draw[thick] (1.5,0) -- (2,0) node[right] {Gate (G)};
            \node at (0,-2.5) {\textbf{Construction}};
        \end{scope}
    \end{tikzpicture}
    \captionof{figure}{SCR Symbol and Construction}
\end{center}

\textbf{SCRના ઉપયોગો:}
\begin{itemize}
    \item \keyword{પાવર કંટ્રોલ}: AC/DC પાવર રેગ્યુલેટર્સ
    \item \keyword{મોટર ડ્રાઈવ્સ}: મોટરની ગતિનું નિયંત્રણ
    \item \keyword{લાઈટિંગ કંટ્રોલ}: ડિમર સર્કિટ્સ
    \item \keyword{ઈન્વર્ટર્સ}: DC થી AC રૂપાંતરણ
\end{itemize}
\end{solutionbox}

\begin{mnemonicbox}
\mnemonic{PALS: પાવર કંટ્રોલ, એપ્લાયન્સ કંટ્રોલ, લાઈટિંગ સિસ્ટમ્સ, સ્પીડ રેગ્યુલેટર્સ}
\end{mnemonicbox}

\questionmarks{1(b)}{4}{પુરા નામ જણાવો (૧) SCS (૨) LASCR (3) MCT (૪) PUT.}

\begin{solutionbox}
\begin{center}
\captionof{table}{Full Forms of Devices}
\begin{tabulary}{\linewidth}{|L|L|}
\hline
\textbf{ડિવાઇસ} & \textbf{પૂરું નામ} \\ \hline
\textbf{SCS} & Silicon Controlled Switch \\ \hline
\textbf{LASCR} & Light Activated Silicon Controlled Rectifier \\ \hline
\textbf{MCT} & MOS Controlled Thyristor \\ \hline
\textbf{PUT} & Programmable Unijunction Transistor \\ \hline
\end{tabulary}
\end{center}
\end{solutionbox}

\begin{mnemonicbox}
\mnemonic{SLaMP: Silicon controlled switch, Light activated SCR, MOS controlled thyristor, Programmable UJT}
\end{mnemonicbox}

\questionmarks{1(c)}{7}{TRIACની V-I લાક્ષણિકતા દોરો અને સમજાવો. તદુપરાંત TRIACના ઉપયોગો લખો.}

\begin{solutionbox}
\textbf{TRIAC V-I લાક્ષણિકતા:}

\begin{center}
\begin{tikzpicture}[scale=0.8]
    % Axes
    \draw[->] (-4,0) -- (4,0) node[right] {$V_{MT2-MT1}$};
    \draw[->] (0,-4) -- (0,4) node[above] {$I_T$};
    
    % Quadrant I
    \draw[thick, blue] (0,0) -- (1,0.2) to[out=80,in=260] (1.5,3.5);
    \draw[blue] (1.5,3.5) -- (3,3.8) node[right] {Conduction};
    \draw[blue, dashed] (0,0) -- (2.5,0.5) node[right] {Blocking State};
    
    % Quadrant III
    \draw[thick, blue] (0,0) -- (-1,-0.2) to[out=260,in=80] (-1.5,-3.5);
    \draw[blue] (-1.5,-3.5) -- (-3,-3.8) node[left] {Conduction};
    \draw[blue, dashed] (0,0) -- (-2.5,-0.5) node[left] {Blocking State};
    
    % Labels
    \node at (2,2) {Quadrant I};
    \node at (-2,-2) {Quadrant III};
    \node at (2.8,0.3) {$+V_{BO}$};
    \node at (-2.8,-0.3) {$-V_{BO}$};
    \node at (0.3,1) {$+I_H$};
    \node at (-0.3,-1) {$-I_H$};
\end{tikzpicture}
\captionof{figure}{V-I Characteristics of TRIAC}
\end{center}

\textbf{TRIACની V-I લાક્ષણિકતા સમજૂતી:}
\begin{itemize}
    \item \keyword{દ્વિદિશાત્મક ઉપકરણ}: બંને દિશામાં વહન કરે છે.
    \item \keyword{ક્વાડ્રન્ટ ઓપરેશન}: પહેલા અને ત્રીજા ક્વાડ્રન્ટમાં કાર્ય કરે છે.
    \item \keyword{બ્રેકઓવર વોલ્ટેજ}: જ્યારે વોલ્ટેજ $\pm V_{bo}$ કરતાં વધે ત્યારે વહન શરૂ થાય.
    \item \keyword{હોલ્ડિંગ કરંટ}: ન્યૂનતમ પ્રવાહ જે વહનની સ્થિતિ જાળવી રાખે છે.
    \item \keyword{ગેટ ટ્રિગરિંગ}: પોઝિટિવ/નેગેટિવ ગેટ વોલ્ટેજથી ટ્રિગર થઈ શકે છે.
\end{itemize}

\textbf{TRIACના ઉપયોગો:}
\begin{itemize}
    \item \keyword{AC પાવર કંટ્રોલ}: લેમ્પ ડિમર્સ, હીટર કંટ્રોલ.
    \item \keyword{મોટર સ્પીડ કંટ્રોલ}: AC મોટર રેગ્યુલેટર્સ.
    \item \keyword{ફેન રેગ્યુલેટર્સ}: ઘરેલું પંખાની ગતિનું નિયંત્રણ.
    \item \keyword{લાઈટ ડિમર્સ}: એડજસ્ટેબલ લાઈટિંગ સિસ્ટમ્સ.
\end{itemize}
\end{solutionbox}

\begin{mnemonicbox}
\mnemonic{HALF: હીટર્સ, AC કંટ્રોલ, લાઈટિંગ સિસ્ટમ્સ, ફેન રેગ્યુલેટર્સ}
\end{mnemonicbox}

\questionmarks{1(c OR)}{7}{IGBT નું કન્સ્ટ્રકશન અને કાર્ય વિગતવાર સમજાવો.}

\begin{solutionbox}
\textbf{IGBT કન્સ્ટ્રકશન અને કાર્ય:}

\begin{center}
\begin{tikzpicture}
    % Substrate
    \draw[fill=red!10] (0,0) rectangle (6,4);
    
    % Layers
    \draw[fill=cyan!20] (0,3.5) rectangle (1.5,4) node[midway, scale=0.7] {N+ Emitter};
    \draw[fill=cyan!20] (4.5,3.5) rectangle (6,4) node[midway, scale=0.7] {N+ Emitter};
    
    \draw[fill=yellow!20] (0,2.5) rectangle (6,3.5) node[midway] {P Body};
    \draw[fill=green!10] (0,1.5) rectangle (6,2.5) node[midway] {N- Drift Region};
    \draw[fill=orange!20] (0,0.5) rectangle (6,1.5) node[midway] {N+ Buffer Layer};
    \draw[fill=magenta!20] (0,0) rectangle (6,0.5) node[midway] {P+ Collector};
    
    % Terminals
    \draw[thick] (0.75,4) -- (0.75,4.5) -- (3,4.5) -- (3,5) node[above] {Emitter (E)};
    \draw[thick] (5.25,4) -- (5.25,4.5) -- (3,4.5);
    
    \draw[thick] (3,0) -- (3,-0.5) node[below] {Collector (C)};
    
    \draw[fill=black] (2.8,3.7) rectangle (3.2,4.2); 
    \node at (3,3.95) [white, scale=0.5] {Gate};
    \draw[thick] (3.2,3.95) -- (3.5,3.95) -- (3.5,4.2) node[above] {Gate (G)};
    
\end{tikzpicture}
\captionof{figure}{Structure of IGBT}
\end{center}

\textbf{રચના વિગતો:}
\begin{itemize}
    \item \keyword{ત્રણ-ટર્મિનલ ડિવાઈસ}: ગેટ, એમિટર, કલેક્ટર.
    \item \keyword{મલ્ટિલેયર સ્ટ્રક્ચર}: N+, P, N-, N+ બફર, P+ સબસ્ટ્રેટ.
    \item \keyword{હાઈબ્રિડ ડિવાઈસ}: MOSFET ઈનપુટ અને BJT આઉટપુટ લાક્ષણિકતાઓનું સંયોજન.
\end{itemize}

\textbf{કાર્ય સિદ્ધાંત:}
\begin{itemize}
    \item \keyword{ગેટ કંટ્રોલ}: P-રીજનમાં ગેટ પર પોઝિટિવ વોલ્ટેજ ઇન્વર્ઝન લેયર બનાવે છે.
    \item \keyword{ચેનલ ફોર્મેશન}: ઇલેક્ટ્રોન્સ N+ એમિટરથી N- ડ્રિફ્ટ રીજન તરફ વહે છે.
    \item \keyword{કન્ડક્ટિવિટી મોડ્યુલેશન}: P-N- જંક્શન હોલ્સ ઇન્જેક્ટ કરે છે, રેઝિસ્ટન્સ ઘટાડે છે.
    \item \keyword{ટર્ન-ઓફ પ્રક્રિયા}: ગેટ વોલ્ટેજ દૂર કરવાથી ઇલેક્ટ્રોન ફ્લો બંધ થઈ જાય છે.
\end{itemize}

\textbf{IGBTના ફાયદા:}
\begin{itemize}
    \item \keyword{ઊંચી ઈનપુટ ઇમ્પીડન્સ}: સરળ વોલ્ટેજ નિયંત્રણ.
    \item \keyword{ઓછા કન્ડક્શન લોસ}: કાર્યક્ષમ પાવર હેન્ડલિંગ.
    \item \keyword{ઝડપી સ્વિચિંગ}: ઉચ્ચ ફ્રીક્વન્સી એપ્લિકેશન્સ માટે યોગ્ય.
\end{itemize}
\end{solutionbox}

\begin{mnemonicbox}
\mnemonic{GIVE: ગેટ કંટ્રોલ્ડ, ઇનપુટ હાઈ ઇમ્પીડન્સ, વોલ્ટેજ ડ્રિવન, એફિશિયન્ટ કન્ડક્શન}
\end{mnemonicbox}

% ==========================================================================================
% Question 2
% ==========================================================================================
\questionmarks{2(a)}{3}{UJTની મદદથી રિલેક્ષેશન ઓસિલેટર સર્કિટની ચર્ચા કરો.}

\begin{solutionbox}
\textbf{UJT રિલેક્ષેશન ઓસિલેટર:}

\begin{center}
\begin{tikzpicture}[auto, node distance=2cm]
    \draw (0,4) node[above] {$V_{CC}$} to[R, l=$R_3$] (0,3) -- (0,2);
    \draw (0,2) node[draw, circle, inner sep=1pt] (ujt) {UJT};
    \draw (0,2) -- (0,1) to[R, l=$R_2$] (0,0) node[ground] {};
    
    % Trigger part
    \draw (-3,4) to[R, l=$R_1$] (-3,2) -- (-3,1) to[C, l=$C$] (-3,0) node[ground] {};
    \draw (-3,2) -- (ujt.west);
    
    % Signal output from R1 (Base 1)
    \draw (0,1) -- (1,1) node[right] {Output};
\end{tikzpicture}
\captionof{figure}{UJT Relaxation Oscillator Circuit}
\end{center}

\textbf{કાર્ય સિદ્ધાંત:}
\begin{itemize}
    \item \keyword{કેપેસિટર ચાર્જિંગ}: C, R1 દ્વારા UJT ફાયરિંગ વોલ્ટેજ સુધી ચાર્જ થાય છે.
    \item \keyword{UJT ફાયર}: જ્યારે એમિટર વોલ્ટેજ પીક પોઈન્ટ વોલ્ટેજ સુધી પહોંચે ત્યારે.
    \item \keyword{ડિસ્ચાર્જ સાયકલ}: કેપેસિટર એમિટર-બેઝ1 જંક્શન દ્વારા ડિસ્ચાર્જ થાય છે.
    \item \keyword{ઓસિલેશન}: પ્રક્રિયા પુનરાવર્તિત થાય છે અને સોટૂથ વેવફોર્મ બનાવે છે.
\end{itemize}
\end{solutionbox}

\begin{mnemonicbox}
\mnemonic{CROP: કેપેસિટર ચાર્જ થાય, રીચ થ્રેશોલ્ડ, ઓસિલેટ થાય, પ્રોડ્યુસ સોટૂથ}
\end{mnemonicbox}

\questionmarks{2(b)}{4}{SCRની ટ્રીગરિંગ પદ્ધતિઓની ચર્ચા કરો.}

\begin{solutionbox}
\begin{center}
\captionof{table}{SCR Triggering Methods}
\begin{tabulary}{\linewidth}{|L|L|}
\hline
\textbf{ટ્રિગરિંગ પદ્ધતિ} & \textbf{કાર્ય સિદ્ધાંત} \\ \hline
\textbf{ગેટ ટ્રિગરિંગ} & ગેટ અને કેથોડ વચ્ચે પોઝિટિવ વોલ્ટેજ આપવામાં આવે છે \\ \hline
\textbf{થર્મલ ટ્રિગરિંગ} & તાપમાન વધારાથી બ્રેકઓવર વોલ્ટેજ ઘટે છે \\ \hline
\textbf{લાઈટ ટ્રિગરિંગ} & ફોટોન્સ LASCR માં ઇલેક્ટ્રોન-હોલ જોડ બનાવે છે \\ \hline
\textbf{dv/dt ટ્રિગરિંગ} & SCR પર ઝડપી વોલ્ટેજ વધારો કેપેસિટિવ કરંટ ઉત્પન્ન કરે છે \\ \hline
\textbf{બ્રેકઓવર ટ્રિગરિંગ} & ગેટ સિગ્નલ વિના વોલ્ટેજ બ્રેકઓવર વોલ્ટેજને ઓળંગે છે \\ \hline
\end{tabulary}
\end{center}

\textbf{મુખ્ય મુદ્દાઓ:}
\begin{itemize}
    \item \keyword{ગેટ ટ્રિગરિંગ}: સૌથી સામાન્ય પદ્ધતિ.
    \item \keyword{લાઈટ ટ્રિગરિંગ}: ઓપ્ટો-આઇસોલેટર્સમાં વપરાય છે.
    \item \keyword{dv/dt ટ્રિગરિંગ}: ઘણી વખત અવાંછનીય, સ્નબર સર્કિટની જરૂર પડે છે.
\end{itemize}
\end{solutionbox}

\begin{mnemonicbox}
\mnemonic{GLTDB: ગેટ, લાઈટ, થર્મલ, dv/dt, બ્રેકઓવર}
\end{mnemonicbox}

\questionmarks{2(c)}{7}{ક્લાસ એ પ્રકારની કોમ્યુટેશન પદ્ધતિ સમજાવો.}

\begin{solutionbox}
\textbf{ક્લાસ A કોમ્યુટેશન (LC સર્કિટ દ્વારા સેલ્ફ-કોમ્યુટેશન):}

\begin{center}
\begin{tikzpicture}[auto, node distance=2cm]
    \draw (0,3) node[left] {DC Source} to[short, o-] (1,3);
    \draw (1,3) to[Thyristor, l=SCR] (3,3);
    \draw (3,3) to[L, l=L] (5,3) to[C, l=C] (7,3);
    \draw (7,3) to[R, l=Load] (7,0);
    \draw (7,0) to[short, -o] (0,0);
    \draw (0,0) node[left] {GND/$-$};
\end{tikzpicture}
\captionof{figure}{Class A Commutation Circuit}
\end{center}

\textbf{કાર્ય સિદ્ધાંત:}
\begin{itemize}
    \item \keyword{પ્રારંભિક સ્થિતિ}: SCR વહન કરે છે, કેપેસિટર જમણી બાજુએ (+) પોલારિટી સાથે ચાર્જ થયેલ છે.
    \item \keyword{કોમ્યુટેશન શરૂઆત}: જ્યારે સ્વિચ SW બંધ થાય છે.
    \item \keyword{રેઝોનન્ટ સર્કિટ}: LC સર્કિટ રેઝોનન્ટ પાથ બનાવે છે.
    \item \keyword{રિવર્સ કરંટ}: કેપેસિટર ડિસ્ચાર્જ SCR મારફતે રિવર્સ કરંટ ઉત્પન્ન કરે છે.
    \item \keyword{ટર્ન-ઓફ}: જ્યારે કરંટ હોલ્ડિંગ કરંટથી નીચે પડે ત્યારે SCR બંધ થાય છે.
    \item \keyword{રિચાર્જિંગ}: કેપેસિટર વિપરીત પોલારિટી સાથે રિચાર્જ થાય છે.
\end{itemize}

\textbf{એપ્લિકેશન:}
\begin{itemize}
    \item \keyword{ઇન્વર્ટર સર્કિટ્સ}: DC થી AC રૂપાંતરણ.
    \item \keyword{ચોપર સર્કિટ્સ}: DC થી DC રૂપાંતરણ.
\end{itemize}
\end{solutionbox}

\begin{mnemonicbox}
\mnemonic{SCCRRT: સ્વિચ ક્લોઝ થાય, કેપેસિટર ડિસ્ચાર્જ થાય, કરંટ રિવર્સ થાય, SCR ટર્ન ઓફ થાય, રિચાર્જિંગ શરૂ થાય, ટર્ન-ઓફ પૂર્ણ થાય}
\end{mnemonicbox}

\questionmarks{2(a OR)}{3}{GTOનું પૂરું નામ જણાવો અને GTOની રચના દોરો.}

\begin{solutionbox}
\textbf{GTOનું પૂરું નામ:} Gate Turn-Off Thyristor

\textbf{GTOની રચના:}

\begin{center}
\begin{tikzpicture}
    % Layers
    \draw[fill=red!20] (0,3) rectangle (4,4) node[midway] {Anode (P+)};
    \draw[fill=blue!10] (0,2) rectangle (4,3) node[midway] {N Base};
    \draw[fill=red!10] (0,1) rectangle (4,2) node[midway] {P Base};
    \draw[fill=blue!20] (0,0) rectangle (4,1) node[midway] {Cathode (N+)};

    % Connections
    \draw[thick] (2,4) -- (2,4.5) node[above] {Anode (A)};
    \draw[thick] (2,0) -- (2,-0.5) node[below] {Cathode (K)};
    \draw[thick] (4,1.5) -- (4.5,1.5) node[right] {Gate (G)};
\end{tikzpicture}
\captionof{figure}{Structure of GTO}
\end{center}
\end{solutionbox}

\begin{mnemonicbox}
\mnemonic{PANG: P-એનોડ, એન્ડ, N-બેઝ, ગેટ-કંટ્રોલ્ડ થાયરિસ્ટર}
\end{mnemonicbox}

\questionmarks{2(b OR)}{4}{SCR માટેની સ્નબર સર્કિટની રચના અને જરૂરિયાતની ચર્ચા કરો.}

\begin{solutionbox}
\textbf{SCR માટે સ્નબર સર્કિટ:}

\begin{center}
\begin{tikzpicture}[auto, node distance=2cm]
    \draw (0,2) to[Thyristor, l=SCR] (0,0);
    \draw (-1,2) -- (0,2) -- (1,2);
    \draw (1,2) to[R, l=$R_s$] (1,1) to[C, l=$C_s$] (1,0); 
    \draw (-1,0) -- (0,0) -- (1,0);
    
    \node at (2,1) {Snubber Circuit};
\end{tikzpicture}
\captionof{figure}{Snubber Circuit}
\end{center}

\textbf{ડિઝાઇન જરૂરિયાતો:}
\begin{itemize}
    \item \keyword{રેઝિસ્ટર પસંદગી}: કેપેસિટર ડિસ્ચાર્જ કરંટને મર્યાદિત કરે છે.
    \item \keyword{કેપેસિટર પસંદગી}: વોલ્ટેજ વૃદ્ધિના દર (dv/dt)ને નિયંત્રિત કરે છે.
    \item \keyword{RC ટાઇમ કોન્સ્ટન્ટ}: રિસ્પોન્સ ટાઈમ નક્કી કરે છે.
\end{itemize}

\textbf{સ્નબર સર્કિટનો હેતુ:}
\begin{itemize}
    \item \keyword{dv/dt પ્રોટેક્શન}: ઝડપી વોલ્ટેજ પરિવર્તનને લીધે ખોટા ટ્રિગરિંગને અટકાવે છે.
    \item \keyword{વોલ્ટેજ સ્પાઈક સપ્રેશન}: ઇન્ડક્ટિવ વોલ્ટેજ સ્પાઈક્સને શોષે છે.
    \item \keyword{ટ્રાન્ઝિયન્ટ પ્રોટેક્શન}: સ્વિચિંગ દરમિયાન SCRને રક્ષણ આપે છે.
\end{itemize}
\end{solutionbox}

\begin{mnemonicbox}
\mnemonic{RAPE: રેઝિસ્ટર એન્ડ કેપેસિટર પ્રોટેક્ટ અગેઇન્સ્ટ એક્સેસિવ વોલ્ટેજ રાઇઝ}
\end{mnemonicbox}

\questionmarks{2(c OR)}{7}{ક્લાસ સી પ્રકારની કોમ્યુટેશન પદ્ધતિ સમજાવો.}

\begin{solutionbox}
\textbf{ક્લાસ C કોમ્યુટેશન (કોમ્પ્લિમેન્ટરી કોમ્યુટેશન):}

\begin{center}
\begin{tikzpicture}[auto, node distance=1.5cm]
    \draw (0,4) node[left] {$V_{DC}$} to[short, o-] (2,4) -- (4,4);
    
    \draw (2,4) to[R, l=$R_1$] (2,2) to[Thyristor, l=$SCR_1$] (2,0);
    \draw (4,4) to[R, l=$R_2$] (4,2) to[Thyristor, l=$SCR_2$] (4,0);
    \draw (2,0) -- (4,0) node[ground] {};
    
    \draw (2,2) to[C, l=C] (4,2);
\end{tikzpicture}
\captionof{figure}{Class C Commutation Circuit}
\end{center}

\textbf{કાર્ય સિદ્ધાંત:}
\begin{itemize}
    \item \keyword{પ્રારંભિક સ્થિતિ}: SCR1 વહન કરે છે, SCR2 બંધ છે.
    \item \keyword{કોમ્યુટેશન શરૂઆત}: SCR2 ટ્રિગર થાય છે.
    \item \keyword{લોડ ટ્રાન્સફર}: કરંટ SCR1 થી SCR2 માં ટ્રાન્સફર થાય છે.
    \item \keyword{વોલ્ટેજ રિવર્સલ}: SCR1 પર વોલ્ટેજ નેગેટિવ થાય છે.
    \item \keyword{ટર્ન-ઓફ}: જ્યારે કરંટ હોલ્ડિંગ કરંટથી નીચે પડે ત્યારે SCR1 બંધ થાય છે.
    \item \keyword{વૈકલ્પિક ઓપરેશન}: SCR1 અને SCR2 વૈકલ્પિક રીતે વહન કરે છે.
\end{itemize}

\textbf{એપ્લિકેશન:}
\begin{itemize}
    \item \keyword{ઇન્વર્ટર સર્કિટ્સ}: બ્રિજ ઇન્વર્ટરમાં વપરાય છે.
    \item \keyword{ડ્યુઅલ લોડ સિસ્ટમ્સ}: જ્યાં વૈકલ્પિક ઓપરેશનની જરૂર હોય.
\end{itemize}
\end{solutionbox}

\begin{mnemonicbox}
\mnemonic{TACTOR: ટ્રિગરિંગ ઓલ્ટરનેટ SCRs ક્રિએટ્સ ટર્ન-ઓફ એન્ડ રિવર્સલ}
\end{mnemonicbox}

% ==========================================================================================
% Question 3
% ==========================================================================================
\questionmarks{3(a)}{3}{પોલીફેઝ રેક્ટિફાયરના ફાયદા વર્ણવો.}

\begin{solutionbox}
\begin{center}
\captionof{table}{Advantages of Poly-phase Rectifier}
\begin{tabulary}{\linewidth}{|L|L|}
\hline
\textbf{ફાયદા} & \textbf{વર્ણન} \\ \hline
\textbf{ઉચ્ચ કાર્યક્ષમતા} & ઓછું પાવર લોસ અને ટ્રાન્સફોર્મર વપરાશમાં સુધારો \\ \hline
\textbf{ઓછો રિપલ ફેક્ટર} & વધુ સારો DC આઉટપુટ જેથી નાના ફિલ્ટર કોમ્પોનન્ટ્સ જોઈએ \\ \hline
\textbf{ઉચ્ચ પાવર હેન્ડલિંગ} & સિંગલ ફેઝ કરતાં વધુ પાવર લેવલ હેન્ડલ કરી શકે છે \\ \hline
\textbf{બેટર ટ્રાન્સફોર્મર ઉપયોગ} & ઉચ્ચ ટ્રાન્સફોર્મર ઉપયોગિતા ફેક્ટર \\ \hline
\textbf{ઓછી હાર્મોનિક સામગ્રી} & આઉટપુટમાં ઘટેલા હાર્મોનિક ડિસ્ટોર્શન \\ \hline
\end{tabulary}
\end{center}
\end{solutionbox}

\begin{mnemonicbox}
\mnemonic{HELPS: ઉચ્ચ કાર્યક્ષમતા, ઈવન આઉટપુટ, ઓછો રિપલ, પાવર હેન્ડલિંગ બેટર, નાના ફિલ્ટર}
\end{mnemonicbox}

\questionmarks{3(b)}{4}{સિંગલ ફેઇઝ હાફવેવ રેક્ટીફાયર સર્કિટ દોરો અને સમજાવો. વેવફોર્મ્સ દોરો.}

\begin{solutionbox}
\textbf{સિંગલ ફેઝ હાફ વેવ રેક્ટિફાયર:}

\begin{center}
\begin{tikzpicture}[auto, node distance=2cm]
    \draw (0,0) node[left] {AC Input} to[short, o-] (1,0) to[D*, l=D] (3,0) to[R, l=R] (3,-2) -- (1,-2) to[short, -o] (0,-2) node[left] {AC Input};
    \draw (3,0) -- (4,0) node[right] {+};
    \draw (3,-2) -- (4,-2) node[right] {-};
    \node at (4,-1) {Output};
\end{tikzpicture}
\captionof{figure}{Half Wave Rectifier Circuit}
\end{center}

\textbf{વેવફોર્મ:}

\begin{center}
\begin{tikzpicture}[xscale=1.5, yscale=0.8]
    % Input
    \draw[->] (0,0) -- (4.5,0) node[right] {Time};
    \draw[->] (0,-1.5) -- (0,1.5) node[above] {$V_{in}$};
    \draw[thick, gray, dashed] plot[domain=0:4*pi, samples=100] (\x/3.14, {sin(\x r)});
    \node at (2.2, 1.2) {Input AC};

    % Output
    \begin{scope}[yshift=-3cm]
        \draw[->] (0,0) -- (4.5,0) node[right] {Time};
        \draw[->] (0,-1.5) -- (0,1.5) node[above] {$V_{out}$};
        \draw[thick, blue] plot[domain=0:4*pi, samples=100] (\x/3.14, {max(0, sin(\x r))});
        \node at (2.2, 1.2) {Output DC (Pulsating)};
    \end{scope}
\end{tikzpicture}
\captionof{figure}{Half Wave Rectifier Waveforms}
\end{center}

\textbf{કાર્ય સિદ્ધાંત:}
\begin{itemize}
    \item \keyword{ફોરવર્ડ બાયસ}: ડાયોડ પોઝિટિવ હાફ-સાયકલ દરમિયાન વહન કરે છે.
    \item \keyword{રિવર્સ બાયસ}: ડાયોડ નેગેટિવ હાફ-સાયકલ દરમિયાન કરંટને અવરોધે છે.
    \item \keyword{આઉટપુટ}: પલ્સેટિંગ DC જેનો રિપલ ફેક્ટર ઊંચો હોય છે.
    \item \keyword{ફ્રિક્વન્સી}: આઉટપુટ ફ્રિક્વન્સી ઇનપુટ ફ્રિક્વન્સી જેટલી જ રહે છે.
\end{itemize}
\end{solutionbox}

\begin{mnemonicbox}
\mnemonic{PROF: પોઝિટિવ હાફ કન્ડક્ટ્સ, રિવર્સ હાફ બ્લોક્સ, આઉટપુટ ઇઝ પલ્સેટિંગ, ફ્રિક્વન્સી અનચેન્જ્ડ}
\end{mnemonicbox}

\questionmarks{3(c)}{7}{બધાજ પ્રકારના ઇન્વર્ટરની યાદી બનાવો. તેમાંથી સિંગલફેઝ ફુલ બ્રિજ ઇન્વર્ટર સમજાવો.}

\begin{solutionbox}
\textbf{ઇન્વર્ટરના પ્રકારો:}
\begin{enumerate}
    \item સર્કિટના આધારે: સીરીઝ, પેરેલલ, બ્રિજ
    \item ફેઝના આધારે: સિંગલ-ફેઝ, થ્રી-ફેઝ
    \item આઉટપુટના આધારે: સ્ક્વેર વેવ, મોડિફાઇડ સાઇન વેવ, પ્યોર સાઇન વેવ
    \item કોમ્યુટેશનના આધારે: SCR-બેઝ્ડ, ટ્રાન્ઝિસ્ટર-બેઝ્ડ
\end{enumerate}

\textbf{સિંગલ ફેઝ ફુલ બ્રિજ ઇન્વર્ટર:}

\begin{center}
\begin{tikzpicture}[auto, node distance=1.5cm]
    \draw (0,4) node[left] {$V_{DC}$} to[short, o-] (2,4) -- (4,4);
    
    \draw (2,4) to[switch, l=$S_1$] (2,2);
    \draw (2,2) to[switch, l=$S_2$] (2,0);
    
    \draw (4,4) to[switch, l=$S_3$] (4,2);
    \draw (4,2) to[switch, l=$S_4$] (4,0);
    
    \draw (2,0) -- (4,0) to[short, -o] (0,0) node[left] {GND};
    
    \draw (2,2) to[R, l=Load] (4,2);
\end{tikzpicture}
\captionof{figure}{Single Phase Full Bridge Inverter}
\end{center}

\textbf{કાર્ય સિદ્ધાંત:}
\begin{itemize}
    \item \keyword{પ્રથમ અર્ધ-સાયકલ}: S1 અને S4 ON, S2 અને S3 OFF.
    \item \keyword{બીજો અર્ધ-સાયકલ}: S2 અને S3 ON, S1 અને S4 OFF.
    \item \keyword{આઉટપુટ વેવફોર્મ}: લોડ પર AC સ્ક્વેર વેવ.
    \item \keyword{કંટ્રોલ મેથડ}: સ્વિચને 180° ફેઝ શિફ્ટ સાથે ગેટ સિગ્નલ આપવામાં આવે છે.
\end{itemize}

\textbf{ફાયદાઓ:}
\begin{itemize}
    \item \keyword{ઉચ્ચ આઉટપુટ પાવર}: હાફ બ્રિજની તુલનામાં બમણો આઉટપુટ.
    \item \keyword{બેટર વોલ્ટેજ ઉપયોગ}: લોડ પર સંપૂર્ણ DC બસ વોલ્ટેજ.
    \item \keyword{ઓછું કરંટ રેટિંગ}: દરેક સ્વિચ માત્ર લોડ કરંટ જ વહન કરે છે.
\end{itemize}
\end{solutionbox}

\begin{mnemonicbox}
\mnemonic{SOAP: સ્વિચેસ ઓપરેટ ઓલ્ટરનેટલી ઇન પેર્સ}
\end{mnemonicbox}

\questionmarks{3(a OR)}{3}{સરખાવો UPS અને SMPS.}

\begin{solutionbox}
\begin{center}
\captionof{table}{Comparison of UPS and SMPS}
\begin{tabulary}{\linewidth}{|L|L|L|}
\hline
\textbf{પેરામીટર} & \textbf{UPS} & \textbf{SMPS} \\ \hline
\textbf{મુખ્ય કાર્ય} & પાવર ફેઇલ થાય ત્યારે બેકઅપ પાવર આપે છે & AC થી રેગ્યુલેટેડ DC માં રૂપાંતર કરે છે \\ \hline
\textbf{બેટરી બેકઅપ} & બેકઅપ માટે બેટરી ધરાવે છે & કોઈ બેટરી બેકઅપ નથી \\ \hline
\textbf{આઉટપુટ} & AC આઉટપુટ (મોટેભાગે) & DC આઉટપુટ (મોટેભાગે) \\ \hline
\textbf{કાર્યક્ષમતા} & ઓછી (70-80\%) & ઉચ્ચ (80-95\%) \\ \hline
\textbf{સાઇઝ} & મોટું અને ભારે & કોમ્પેક્ટ અને હલકું \\ \hline
\textbf{એપ્લિકેશન} & કોમ્પ્યુટર, સર્વર, ક્રિટિકલ ઇક્વિપમેન્ટ & ઇલેક્ટ્રોનિક ડિવાઇસ, ચાર્જર \\ \hline
\end{tabulary}
\end{center}
\end{solutionbox}

\begin{mnemonicbox}
\mnemonic{BBOSS: બેકઅપ બેટરી ઓન્લી ઇન UPS, સ્મોલ સાઇઝ ઇન SMPS}
\end{mnemonicbox}

\questionmarks{3(b OR)}{4}{થ્રી ફેઇઝ હાફ વેવ રેક્ટીફાયર સર્કિટ દોરો અને સમજાવો. વેવફોર્મ્સદોરો.}

\begin{solutionbox}
\textbf{થ્રી ફેઝ હાફ વેવ રેક્ટિફાયર:}

\begin{center}
\begin{tikzpicture}[auto, node distance=1.5cm]
    \draw (0,3) node[left] {R} to[D*, l=$D_1$] (2,3) -- (2,2);
    \draw (0,2) node[left] {Y} to[D*, l=$D_2$] (2,2) -- (2,1);
    \draw (0,1) node[left] {B} to[D*, l=$D_3$] (2,1);
    
    \draw (2,1) -- (2,0) -- (3,0); % Common cathode
    \draw (3,0) to[R, l=Load] (3,-2) -- (0,-2) node[left] {Neutral};
\end{tikzpicture}
\captionof{figure}{Three Phase Half Wave Rectifier Circuit}
\end{center}

\textbf{વેવફોર્મ:}

\begin{center}
\begin{tikzpicture}[xscale=1.5, yscale=0.8]
    \draw[->] (0,0) -- (4.5,0) node[right] {Time};
    \draw[->] (0,-1.5) -- (0,2) node[above] {V};
    
    \foreach \x/\color in {0/red, 2/green, 4/blue} {
        \draw[thin, opacity=0.3] plot[domain=0:4*pi, samples=100] (\x/3.14, {sin((\x*180/3.14 - \x*60) r)});
    }
    
    % R phase
    \draw[dashed, red] plot[domain=0:4*pi, samples=100] (\x/3.14, {sin(\x r)});
    % Y phase
    \draw[dashed, blue] plot[domain=0:4*pi, samples=100] (\x/3.14, {sin((\x - 2.09)r)});
    % B phase
    \draw[dashed, green!60!black] plot[domain=0:4*pi, samples=100] (\x/3.14, {sin((\x - 4.18)r)});

    % Output envelope attempt
    \draw[thick] plot[domain=0:4*pi, samples=200] (\x/3.14, {max(sin(\x r), sin((\x - 2.09)r), sin((\x - 4.18)r))});
    
    \node at (2, 1.5) {Output Envelope};
\end{tikzpicture}
\captionof{figure}{Output Voltage Waveform}
\end{center}

\textbf{કાર્ય સિદ્ધાંત:}
\begin{itemize}
    \item \keyword{કન્ડક્શન સિક્વન્સ}: જ્યારે તેની ફેઝ વોલ્ટેજ સૌથી વધુ હોય ત્યારે દરેક ડાયોડ વહન કરે છે.
    \item \keyword{કન્ડક્શન એંગલ}: દરેક ડાયોડ 120° માટે વહન કરે છે.
    \item \keyword{આઉટપુટ રિપલ}: સાયકલ દીઠ 3 પલ્સ, સિંગલ ફેઝ કરતાં ઓછો રિપલ.
    \item \keyword{રિપલ ફ્રિક્વન્સી}: ઇનપુટ ફ્રિક્વન્સીથી 3 ગણી.
\end{itemize}
\end{solutionbox}

\begin{mnemonicbox}
\mnemonic{CROP: કન્ડક્શન ઓફ 120°, રિપલ રિડ્યુસ્ડ, આઉટપુટ સ્મૂધર, પલ્સ ટ્રિપલ્ડ}
\end{mnemonicbox}

\questionmarks{3(c OR)}{7}{ચોપરને વ્યાખ્યાયિત કરો. ક્લાસ ડી ચોપરનો પરિપથ દોરો અને સમજાવો.}

\begin{solutionbox}
\textbf{ચોપરની વ્યાખ્યા:}
ચોપર એ DC થી DC કન્વર્ટર છે જે ફિક્સ્ડ DC ઇનપુટ વોલ્ટેજને હાઈ-ફ્રિક્વન્સી સ્વિચિંગનો ઉપયોગ કરીને વેરિએબલ DC આઉટપુટ વોલ્ટેજમાં રૂપાંતરિત કરે છે.

\textbf{ક્લાસ D ચોપર (બે-ક્વાડ્રન્ટ ચોપર):}

\begin{center}
\begin{tikzpicture}[auto, node distance=1.5cm]
    \draw (0,4) node[left] {$V_s+$} -- (2,4) to[switch, l=$S_1$] (4,4);
    \draw (4,4) to[L, l=L] (4,2.5) to[R, l=Load] (4,1.5);
    \draw (4,1.5) to[switch, l=$S_2$] (2,1.5) -- (2,0) -- (0,0) node[left] {$V_s-$};
    
    % Diodes
    \draw (4,1.5) -- (5,1.5) to[D*, l=$D_2$] (5,4) -- (2,4); % Load- to Vs+
    \draw (0,0) -- (1,0) to[D*, l=$D_1$] (1,1.5) -- (4,4); % Vs- to Load+
\end{tikzpicture}
\captionof{figure}{Class D Chopper Circuit}
\end{center}

\textbf{કાર્ય સિદ્ધાંત:}
\begin{itemize}
    \item \keyword{પ્રથમ ક્વાડ્રન્ટ ઓપરેશન (ફોરવર્ડ મોટરિંગ)}: S1 ON, S2 ON. ઊર્જા સ્ત્રોતથી લોડ તરફ વહે છે.
    \item \keyword{બીજા ક્વાડ્રન્ટ ઓપરેશન (ફોરવર્ડ રિજનરેશન)}: S1 OFF, S2 OFF. કરંટ D2 દ્વારા ફ્રીવ્હીલ થાય છે. ઊર્જા લોડથી સ્ત્રોત તરફ વહે છે.
\end{itemize}

\textbf{એપ્લિકેશન:}
\begin{itemize}
    \item \keyword{DC મોટર ડ્રાઇવ}: ફોરવર્ડ મોટરિંગ અને રિજનરેટિવ બ્રેકિંગ પ્રદાન કરે છે.
    \item \keyword{બેટરી ચાર્જિંગ}: ચાર્જિંગ કરંટનું નિયંત્રણ.
    \item \keyword{રીન્યુએબલ એનર્જી}: સોલાર પેનલ સાથે ઇન્ટરફેસિંગ.
\end{itemize}
\end{solutionbox}

\begin{mnemonicbox}
\mnemonic{FRED: ફોરવર્ડ મોટરિંગ, રિજનરેટિવ બ્રેકિંગ, એનર્જી ફ્લો કંટ્રોલ, ડ્યુઅલ ક્વાડ્રન્ટ ઓપરેશન}
\end{mnemonicbox}

% ==========================================================================================
% Question 4
% ==========================================================================================
\questionmarks{4(a)}{3}{SCRનો સ્ટેટિક સ્વીચ તરીકેનો ઉપયોગ સમજાવો.}

\begin{solutionbox}
\textbf{SCR એઝ સ્ટેટિક સ્વિચ:}

\begin{center}
\begin{tikzpicture}[auto, node distance=2cm]
    \draw (0,0) node[left] {AC/DC Supply} to[short, o-] (1,0) to[Thyristor, l=SCR] (3,0) to[R, l=Load] (5,0) to[short, -o] (6,0) node[right] {Return};
    \draw (2,-1) node[below] {Gate Control} -- (2,0);
\end{tikzpicture}
\captionof{figure}{SCR Static Switch Application}
\end{center}

\textbf{મુખ્ય વિશેષતાઓ:}
\begin{itemize}
    \item \keyword{કોઈ મૂવિંગ પાર્ટ્સ નહીં}: શુદ્ધ ઇલેક્ટ્રોનિક સ્વિચિંગ.
    \item \keyword{ઝડપી સ્વિચિંગ}: માઇક્રોસેકન્ડ રિસ્પોન્સ ટાઈમ.
    \item \keyword{ઉચ્ચ વિશ્વસનીયતા}: મિકેનિકલ સ્વિચ કરતાં લાંબું આયુષ્ય.
    \item \keyword{નિયંત્રિત ટર્ન-ઓન}: ગેટ સિગ્નલ દ્વારા ચોક્કસ નિયંત્રણ.
\end{itemize}

\textbf{મિકેનિકલ સ્વિચ કરતાં ફાયદા:}
\begin{itemize}
    \item \keyword{કોઈ આર્કિંગ નહીં}: કોઈ કોન્ટેક્ટ બાઉન્સ કે ઘસારો નહીં.
    \item \keyword{સાયલેન્ટ ઓપરેશન}: કોઈ મિકેનિકલ અવાજ નહીં.
    \item \keyword{EMI ઘટાડો}: ઓછું ઇલેક્ટ્રોમેગ્નેટિક ઇન્ટરફેરન્સ.
\end{itemize}
\end{solutionbox}

\begin{mnemonicbox}
\mnemonic{FANS: ફાસ્ટ સ્વિચિંગ, આર્ક-ફ્રી ઓપરેશન, નો મિકેનિકલ વેર, સાયલેન્ટ ઓપરેશન}
\end{mnemonicbox}

\questionmarks{4(b)}{4}{DIAC અને TRIACનો ઉપયોગ કરી A.C પાવર કંટ્રોલનો સર્કિટ ડાયગ્રામ દોરો અને તેનું કાર્ય સમજાવો.}

\begin{solutionbox}
\textbf{DIAC અને TRIAC વડે AC પાવર કંટ્રોલ:}

\begin{center}
\begin{tikzpicture}[auto, node distance=2cm]
    \draw (0,3) node[left] {AC Supply} to[short, o-] (1,3) to[R, l=Load] (3,3) to[Triac, l=TRIAC, n=triac] (3,0) to[short, -o] (0,0);
    
    % Trigger circuit
    \draw (1,3) to[vR, l=$R_{var}$] (1,1.5) to[C, l=$C$] (1,0) -- (3,0);
    \draw (1,1.5) -- (2,1.5) to[diode, l=DIAC] (2,1) -- (triac.G);
\end{tikzpicture}
\captionof{figure}{AC Power Control Circuit}
\end{center}

\textbf{કાર્ય સિદ્ધાંત:}
\begin{itemize}
    \item \keyword{RC નેટવર્ક}: ગેટ પલ્સને વિલંબિત કરીને ફાયરિંગ એંગલનું નિયંત્રણ કરે છે.
    \item \keyword{કેપેસિટર ચાર્જિંગ}: C દરેક હાફ-સાયકલ દરમિયાન R મારફતે ચાર્જ થાય છે.
    \item \keyword{DIAC બ્રેકડાઉન}: જ્યારે કેપેસિટર વોલ્ટેજ DIAC બ્રેકઓવર વોલ્ટેજ સુધી પહોંચે.
    \item \keyword{TRIAC ટ્રિગરિંગ}: DIAC વહન કરે છે અને TRIAC ટ્રિગર કરે છે.
    \item \keyword{પાવર કંટ્રોલ}: R ને બદલવાથી ફાયરિંગ એંગલ અને પાવર ડિલિવરી બદલાય છે.
\end{itemize}

\textbf{એપ્લિકેશન:}
\begin{itemize}
    \item \keyword{લાઈટ ડિમર્સ}: લેમ્પની બ્રાઈટનેસ કંટ્રોલ.
    \item \keyword{ફેન સ્પીડ કંટ્રોલ}: પંખાની ગતિનું નિયંત્રણ.
    \item \keyword{હીટર કંટ્રોલ}: હીટિંગ એલિમેન્ટ્સ એડજસ્ટ કરવા.
\end{itemize}
\end{solutionbox}

\begin{mnemonicbox}
\mnemonic{CRAFT: કેપેસિટર ચાર્જેસ, રીચેસ બ્રેકઓવર, એક્ટિવેટ્સ DIAC, ફાયર્સ TRIAC, ટ્રાન્સફર્સ પાવર}
\end{mnemonicbox}

\questionmarks{4(c)}{7}{ઇન્ડક્શન હીટિંગનો કાર્યકારી સિદ્ધાંત સમજાવો તદુપરાંત ઇન્ડક્શન હીટિંગના ઉપયોગો લખો.}

\begin{solutionbox}
\textbf{ઇન્ડક્શન હીટિંગનો કાર્યકારી સિદ્ધાંત:}

\begin{center}
\begin{tikzpicture}
    % Coil
    \foreach \y in {0,0.5,1,1.5,2} {
        \draw[thick, orange] (0,\y) arc (180:360:1.5 and 0.2);
        \draw[thick, orange] (0,\y) arc (180:0:1.5 and 0.2);
    }
    \node at (3,1) [right, orange] {Induction Coil (AC)};
    
    % Workpiece
    \draw[fill=gray!50] (1.5,0.2) rectangle (1.5,1.8); 
    \draw[fill=gray!50] (0.5,0.2) rectangle (2.5,1.8);
    \node at (1.5,1) {Metal};
    
    % Field lines
    \draw[blue, dashed, ->] (1.5,1) ellipse (0.5 and 0.1);
    \node at (1.5,1.5) [scale=0.6, blue] {Eddy Currents};
    
\end{tikzpicture}
\captionof{figure}{Induction Heating Principle}
\end{center}

\textbf{કાર્ય સિદ્ધાંત:}
\begin{itemize}
    \item \keyword{હાઈ-ફ્રિક્વન્સી કરંટ}: ઇન્ડક્શન કોઈલમાંથી પસાર થાય છે.
    \item \keyword{ઇલેક્ટ્રોમેગ્નેટિક ઇન્ડક્શન}: ઓલ્ટરનેટિંગ મેગ્નેટિક ફિલ્ડ ઉત્પન્ન કરે છે.
    \item \keyword{એડી કરંટ}: વર્કપીસમાં પ્રેરિત થાય છે.
    \item \keyword{રેઝિસ્ટન્સ હીટિંગ}: એડી કરંટ રેઝિસ્ટન્સને કારણે ગરમી ઉત્પન્ન કરે છે.
    \item \keyword{સ્કિન ઇફેક્ટ}: સપાટીની નજીક ગરમી કેન્દ્રિત થાય છે.
    \item \keyword{નોન-કોન્ટેક્ટ હીટિંગ}: કોઈલ અને વર્કપીસ વચ્ચે કોઈ શારીરિક સંપર્ક નથી.
\end{itemize}

\textbf{ઇન્ડક્શન હીટિંગના ઉપયોગો:}
\begin{itemize}
    \item \keyword{મેટલ હીટ ટ્રીટમેન્ટ}: હાર્ડનિંગ, એનિલિંગ, ટેમ્પરિંગ.
    \item \keyword{મેટલ મેલ્ટિંગ}: ફાઉન્ડ્રી ઓપરેશન્સ.
    \item \keyword{વેલ્ડિંગ અને બ્રેઝિંગ}: મેટલ કોમ્પોનન્ટ્સની જોડાણ.
    \item \keyword{ફોર્જિંગ}: ફોર્મિંગ પહેલાં હીટિંગ.
    \item \keyword{ઘરેલું રસોઈ}: ઇન્ડક્શન કૂકટોપ.
    \item \keyword{સેમિકન્ડક્ટર પ્રોસેસિંગ}: ક્રિસ્ટલ ગ્રોથ.
\end{itemize}
\end{solutionbox}

\begin{mnemonicbox}
\mnemonic{MASTER: મેગ્નેટિક ફિલ્ડ, ઓલ્ટરનેટિંગ કરંટ, સરફેસ હીટિંગ, ટેમ્પરેચર કંટ્રોલ, એડી કરંટ્સ, રેઝિસ્ટન્સ હીટિંગ}
\end{mnemonicbox}

\questionmarks{4(a OR)}{3}{એલડીઆરનો ઉપયોગ કરીને ફોટો રિલે સર્કિટનું કાર્ય સમજાવો.}

\begin{solutionbox}
\textbf{LDR વાળો ફોટો રિલે સર્કિટ:}

\begin{center}
\begin{tikzpicture}[auto, node distance=2cm]
    \draw (0,4) node[above] {$V_{CC}$} -- (0,3) to[R, l=$R_1$] (0,1.5) to[vR, l=LDR] (0,0) node[ground] {};
    \draw (0,1.5) -- (1,1.5) to[T, l=NPN] (1,0) node[ground] {};
    \draw (2,4) -- (2,3) to[R, l=Relay] (2,2) -- (1,2) ;% Collector
    \draw (1,1.5) -- (1.5,1.5) node[right] {Base};
\end{tikzpicture}
\captionof{figure}{LDR Photo Relay Circuit}
\end{center}

\textbf{કાર્ય સિદ્ધાંત:}
\begin{itemize}
    \item \keyword{લાઈટ-ડિપેન્ડન્ટ રેઝિસ્ટર}: પ્રકાશ વધતાં રેઝિસ્ટન્સ ઘટે છે.
    \item \keyword{વોલ્ટેજ ડિવાઈડર}: LDR અને R1 વોલ્ટેજ ડિવાઈડર બનાવે છે.
    \item \keyword{ટ્રાન્ઝિસ્ટર સ્વિચિંગ}: બેઝ વોલ્ટેજ ટ્રાન્ઝિસ્ટર કન્ડક્શનને નિયંત્રિત કરે છે.
    \item \keyword{રિલે ઓપરેશન}: ટ્રાન્ઝિસ્ટર રિલે કોઈલને ડ્રાઈવ કરે છે.
    \item \keyword{થ્રેશોલ્ડ એડજસ્ટમેન્ટ}: વેરિએબલ રેઝિસ્ટર વડે સેટ કરી શકાય છે.
\end{itemize}

\textbf{એપ્લિકેશન:}
\begin{itemize}
    \item \keyword{ઓટોમેટિક સ્ટ્રીટ લાઈટિંગ}: સાંજ પડતાં લાઈટ ચાલુ કરે છે.
    \item \keyword{ડે/નાઈટ સ્વિચિંગ}: એમ્બિયન્ટ લાઈટના આધારે ડિવાઈસ કંટ્રોલ.
    \item \keyword{સિક્યોરિટી સિસ્ટમ}: લાઈટ-એક્ટિવેટેડ અલાર્મ.
\end{itemize}
\end{solutionbox}

\begin{mnemonicbox}
\mnemonic{LARK: લાઈટ કંટ્રોલ્સ, એક્ટિવેટ્સ ટ્રાન્ઝિસ્ટર, રિલે સ્વિચેસ, કીપ્સ સર્કિટ ઓટોમેટેડ}
\end{mnemonicbox}

\questionmarks{4(b OR)}{4}{555 ટાઈમર ICની મદદથી ટાઈમર સર્કિટનું કાર્ય સમજાવો.}

\begin{solutionbox}
\textbf{555 ટાઇમર સર્કિટ (મોનોસ્ટેબલ):}

\begin{center}
\begin{tikzpicture}
    \draw (0,0) rectangle (4,4);
    \node at (2,2) {555 IC};
    
    % Pins
    \node at (0,3.5) [left] {GND (1)};
    \node at (0,2.5) [left] {Trig (2)};
    \node at (0,1.5) [left] {Out (3)};
    \node at (0,0.5) [left] {Reset (4)};
    
    \node at (4,3.5) [right] {Vcc (8)};
    \node at (4,2.5) [right] {Disch (7)};
    \node at (4,1.5) [right] {Thres (6)};
    \node at (4,0.5) [right] {Cont (5)};
    
    % External components
    \draw (4,3.5) -- (5,3.5) -- (5,2.5) to[R, l=R] (5,1.5) to[C, l=C] (5,0) node[ground] {};
    \draw (4,2.5) -- (5,2.5); 
    
    \node at (2, -1) {Standard Monostable Configuration};
\end{tikzpicture}
\captionof{figure}{555 Timer Block Diagram}
\end{center}

\textbf{કાર્ય સિદ્ધાંત:}
\begin{itemize}
    \item \keyword{ટ્રિગર ઇનપુટ}: પિન 2 પર એક્ટિવ લો ટ્રિગર.
    \item \keyword{ટાઇમિંગ કોમ્પોનન્ટ્સ}: R અને C ટાઇમિંગ પીરિયડ નક્કી કરે છે ($T = 1.1RC$).
    \item \keyword{આઉટપુટ હાઈ}: ટ્રિગર થવા પર, આઉટપુટ હાઈ થાય છે.
    \item \keyword{કેપેસિટર ચાર્જિંગ}: C, R મારફતે ચાર્જ થાય છે.
    \item \keyword{થ્રેશોલ્ડ ડિટેક્શન}: જ્યારે વોલ્ટેજ 2/3 VCC સુધી પહોંચે, આઉટપુટ લો થાય છે.
    \item \keyword{ટાઇમર રિસેટ}: પિન 4 વડે સર્કિટ રિસેટ કરી શકાય છે.
\end{itemize}

\textbf{એપ્લિકેશન:}
\begin{itemize}
    \item \keyword{ડિલે સર્કિટ્સ}: ટાઈમ ડિલે બનાવવા.
    \item \keyword{પલ્સ જનરેશન}: ચોક્કસ પલ્સ જનરેટ કરવા.
    \item \keyword{ટાઇમિંગ કંટ્રોલ}: સિક્વેન્શિયલ ટાઇમિંગ ઓપરેશન્સ.
\end{itemize}
\end{solutionbox}

\begin{mnemonicbox}
\mnemonic{TRACT: ટ્રિગર એક્ટિવેટ્સ, રેઝિસ્ટર-કેપેસિટર ટાઇમિંગ, એક્યુરેટ ડિલે, કેપેસિટર ચાર્જેસ, થ્રેશોલ્ડ ડિટેક્શન}
\end{mnemonicbox}

\questionmarks{4(c OR)}{7}{ડાઈઇલેક્ટ્રીક હીટિંગનો કાર્યકારી સિદ્ધાંત સમજાવો તદુપરાંત ડાઈઇલેક્ટ્રીક હીટિંગના ઉપયોગો લખો.}

\begin{solutionbox}
\textbf{ડાઈઇલેક્ટ્રીક હીટિંગનો કાર્યકારી સિદ્ધાંત:}

\begin{center}
\begin{tikzpicture}
    % Electrodes
    \draw[fill=gray!30] (0,3) rectangle (4,3.5) node[midway] {Electrode (+)};
    \draw[fill=gray!30] (0,0) rectangle (4,0.5) node[midway] {Electrode (-)};
    
    % Material
    \draw[fill=yellow!10] (0.5,0.5) rectangle (3.5,3) node[midway, align=center] {Dielectric\\Material};
    
    % Dipoles
    \draw[->, thick, blue] (1, 1.5) -- (1, 2);
    \draw[->, thick, blue] (2, 2.5) -- (2, 2);
    \draw[->, thick, blue] (3, 1.5) -- (3, 2);
    \node at (2,1) [scale=0.7] {Dipoles aligning};
    
    % RF Gen
    \draw (-1,3.25) -- (0,3.25);
    \draw (-1,0.25) -- (0,0.25);
    \draw (-1,3.25) -- (-1,0.25);
    \node at (-1,1.75) [left] {RF Source};
\end{tikzpicture}
\captionof{figure}{Dielectric Heating Principle}
\end{center}

\textbf{કાર્ય સિદ્ધાંત:}
\begin{itemize}
    \item \keyword{ઉચ્ચ-ફ્રિક્વન્સી ઇલેક્ટ્રિક ફિલ્ડ}: ઇલેક્ટ્રોડ્સ વચ્ચે લાગુ કરવામાં આવે છે.
    \item \keyword{ડાઈઇલેક્ટ્રીક મટીરિયલ}: ઇલેક્ટ્રોડ્સ વચ્ચે મૂકવામાં આવે છે.
    \item \keyword{મોલેક્યુલર પોલરાઈઝેશન}: ડાયપોલ્સ ઇલેક્ટ્રિક ફિલ્ડ સાથે એલાઇન થાય છે.
    \item \keyword{ફિલ્ડ ઓસિલેશન}: ફિલ્ડની દિશાનું ઝડપી રિવર્સલ.
    \item \keyword{મોલેક્યુલર ફ્રિક્શન}: ડાયપોલ્સ ઝડપથી રોટેટ થઈને ફ્રિક્શન ઉત્પન્ન કરે છે.
    \item \keyword{વોલ્યુમેટ્રિક હીટિંગ}: સમગ્ર મટીરિયલમાં ગરમી ઉત્પન્ન થાય છે.
    \item \keyword{ફ્રિક્વન્સી રેન્જ}: સામાન્ય રીતે 10-100 MHz.
\end{itemize}

\textbf{ડાઈઇલેક્ટ્રીક હીટિંગના ઉપયોગો:}
\begin{itemize}
    \item \keyword{ફૂડ પ્રોસેસિંગ}: બેકિંગ, ડ્રાયિંગ, પાશ્ચરાઈઝેશન.
    \item \keyword{વુડ ઇન્ડસ્ટ્રી}: ગ્લુઈંગ, ટિમ્બર ડ્રાઈંગ.
    \item \keyword{ટેક્સટાઈલ ડ્રાઈંગ}: કાપડમાંથી ભેજ દૂર કરવો.
    \item \keyword{પ્લાસ્ટિક વેલ્ડિંગ}: થર્મોપ્લાસ્ટિક્સ જોડવા.
    \item \keyword{મેડિકલ એપ્લિકેશન}: થેરાપ્યુટિક ડાયથર્મી.
    \item \keyword{પેપર ઇન્ડસ્ટ્રી}: પેપર પ્રોડક્ટ્સ ડ્રાઈંગ.
\end{itemize}
\end{solutionbox}

\begin{mnemonicbox}
\mnemonic{DIPOLE: ડાઈઇલેક્ટ્રિક મટિરિયલ, ઇન્ટેન્સ ઇલેક્ટ્રિક ફિલ્ડ, પોલરાઈઝેશન ઓફ મોલેક્યુલ્સ, ઓસિલેશન કોઝેસ, લિંકેજ ઓફ હીટ, ઈવન હીટિંગ થ્રુઆઉટ}
\end{mnemonicbox}

% ==========================================================================================
% Question 5
% ==========================================================================================
\questionmarks{5(a)}{3}{AC ડ્રાઈવની વ્યાખ્યા આપો. AC ડ્રાઈવના ઉપયોગો જણાવો.}

\begin{solutionbox}
\textbf{AC ડ્રાઈવની વ્યાખ્યા:}
AC ડ્રાઈવ એ એક ઇલેક્ટ્રોનિક ઉપકરણ છે જે AC મોટરને સપ્લાય કરવામાં આવતા વોલ્ટેજ અને ફ્રિક્વન્સીમાં ફેરફાર કરીને મોટરની ગતિ, ટોર્ક અને દિશાનું નિયંત્રણ કરે છે.

\textbf{AC ડ્રાઈવના ઉપયોગો:}
\begin{center}
\captionof{table}{Applications of AC Drives}
\begin{tabulary}{\linewidth}{|L|L|}
\hline
\textbf{એપ્લિકેશન એરિયા} & \textbf{ઉદાહરણો} \\ \hline
\textbf{ઔદ્યોગિક} & કન્વેયર સિસ્ટમ્સ, પમ્પ્સ, ફેન્સ, કોમ્પ્રેસર્સ \\ \hline
\textbf{HVAC} & બ્લોઅર્સ, કુલિંગ ટાવર્સ, એર હેન્ડલિંગ યુનિટ્સ \\ \hline
\textbf{વોટર ટ્રીટમેન્ટ} & પમ્પ્સ, મિક્સર્સ, એરેટર્સ \\ \hline
\textbf{માઇનિંગ} & ક્રશર્સ, કન્વેયર્સ, પમ્પ્સ \\ \hline
\textbf{ટેક્સટાઈલ} & સ્પિનિંગ મશીન્સ, લૂમ્સ, વિન્ડર્સ \\ \hline
\textbf{મટીરિયલ હેન્ડલિંગ} & ક્રેન્સ, એલિવેટર્સ, એસ્કેલેટર્સ \\ \hline
\end{tabulary}
\end{center}
\end{solutionbox}

\begin{mnemonicbox}
\mnemonic{PITCHW: પમ્પ્સ, ઇન્ડસ્ટ્રીયલ મશીનરી, ટેક્સટાઈલ મશીન્સ, કન્વેયર સિસ્ટમ્સ, HVAC સિસ્ટમ્સ, વોટર ટ્રીટમેન્ટ}
\end{mnemonicbox}

\questionmarks{5(b)}{4}{ડીસી શંટ મોટરની સ્પીડ કંટ્રોલ માટેની કોઈ પણ એક રીત આકૃતિ દોરી સમજાવો.}

\begin{solutionbox}
\textbf{ડીસી શંટ મોટર માટે આર્મેચર વોલ્ટેજ કંટ્રોલ મેથડ:}

\begin{center}
\begin{tikzpicture}[auto, node distance=2cm]
    \draw (0,4) node[left] {AC Supply} to[short, o-] (1,4) to[D*, l=Bridge] (2,4) -- (3,4);
    \draw (3,4) to[Thyristor, l=SCR] (5,4) to[short] (5,3) to[circle, draw, l=Armature] (5,1) to[short] (3,1) -- (2,1) -- (1,1) to[short, -o] (0,1);
    
    % Field
    \draw (0,4) -- (0,5) to[R, l=Field] (5,5) -- (5,4); 
    
    % Proper Diagram
    \draw (0,0) node[left] {AC} to[short, o-] (1,0) to[D*, l=Bridge] (2,0);
    \draw (2,0) -- (2,2) to[Thyristor, l=SCR] (4,2) -- (4,1.5) node[draw, circle, minimum size=1cm] (M) {M};
    \draw (4,0.5) -- (4,0) -- (2,0);
    
    \draw (1,0) -- (1,-1) to[R, l=Field] (4,-1) -- (4,0);
    \node at (5,1) {Armature};
\end{tikzpicture}
\captionof{figure}{Armature Voltage Control Circuit}
\end{center}

\textbf{કાર્ય સિદ્ધાંત:}
\begin{itemize}
    \item \keyword{અચળ ફિલ્ડ કરંટ}: ફિલ્ડ સપ્લાય અચળ રાખવામાં આવે છે.
    \item \keyword{વેરિએબલ આર્મેચર વોલ્ટેજ}: SCR દ્વારા નિયંત્રિત.
    \item \keyword{સ્પીડ ઈક્વેશન}: $N \propto (V_a - I_aR_a)/\Phi$.
    \item \keyword{સ્પીડ કંટ્રોલ}: આર્મેચર વોલ્ટેજ $V_a$ બદલીને.
    \item \keyword{ટોર્ક કંટ્રોલ}: આર્મેચર કરંટ ટોર્કને નિયંત્રિત કરે છે.
\end{itemize}

\textbf{ફાયદાઓ:}
\begin{itemize}
    \item \keyword{wide સ્પીડ રેન્જ}: બેઝ સ્પીડની નીચે અને ઉપરની ઝડપ મેળવી શકાય છે.
    \item \keyword{સ્મૂધ કંટ્રોલ}: સતત સ્પીડ એડજસ્ટમેન્ટ.
    \item \keyword{ઉચ્ચ કાર્યક્ષમતા}: કંટ્રોલ સર્કિટમાં ઓછો પાવર લોસ.
\end{itemize}
\end{solutionbox}

\begin{mnemonicbox}
\mnemonic{SAVE: SCR કંટ્રોલ્સ, આર્મેચર વોલ્ટેજ વેરિઝ, વેલોસિટી ચેન્જિસ, એફિશિયન્ટ ઓપરેશન}
\end{mnemonicbox}

\questionmarks{5(c)}{7}{પી.એલ.સી ના બ્લોક ડાઈગ્રામ દોરો અને દરેક બ્લોકનું કાર્ય સમજાવો.}

\begin{solutionbox}
\textbf{PLC બ્લોક ડાયગ્રામ:}

\begin{center}
\begin{tikzpicture}[auto, node distance=1.5cm]
    % Central Unit
    \node [gtu block, minimum width=4cm, minimum height=3cm] (cpu) {};
    \node at (cpu.north) [below] {\textbf{CPU}};
    
    \node [gtu block, minimum width=3cm] (mem) at (cpu.center) [above=0.5cm] {Memory};
    
    % Modules
    \node [gtu block] (inp) [left=of cpu] {Input Module};
    \node [gtu block] (out) [right=of cpu] {Output Module};
    \node [gtu block] (ps) [above=of cpu] {Power Supply};
    \node [gtu block] (com) [below=of cpu] {Comm. Module};
    
    % Peripherals
    \node [gtu block, dashed] (dev_in) [left=of inp] {Input Devices};
    \node [gtu block, dashed] (dev_out) [right=of out] {Output Devices};
    \node [gtu block, dashed] (prog) [below=of com] {Programming Device};
    
    % Connections
    \draw [gtu arrow] (ps) -- (cpu);
    \draw [gtu arrow] (inp) -- (cpu);
    \draw [gtu arrow] (cpu) -- (out);
    \draw [gtu arrow] (cpu) -- (com);
    \draw [gtu arrow] (dev_in) -- (inp);
    \draw [gtu arrow] (out) -- (dev_out);
    \draw [gtu arrow] (prog) -- (com);
    
\end{tikzpicture}
\captionof{figure}{Block Diagram of PLC}
\end{center}

\textbf{દરેક બ્લોકનું કાર્ય:}
\begin{center}
\captionof{table}{Functions of PLC Blocks}
\begin{tabulary}{\linewidth}{|L|L|}
\hline
\textbf{બ્લોક} & \textbf{કાર્ય} \\ \hline
\textbf{પાવર સપ્લાય} & મેઈન AC સપ્લાયને ઇન્ટરનલ સર્કિટ માટે જરૂરી DC માં રૂપાંતરિત કરે છે \\ \hline
\textbf{CPU} & પ્રોગ્રામ એક્ઝીક્યુટ કરે છે, I/O પ્રોસેસ કરે છે, કેલ્ક્યુલેશન કરે છે \\ \hline
\textbf{મેમરી} & પ્રોગ્રામ, ડેટા અને I/O સ્ટેટસ સ્ટોર કરે છે (RAM, ROM, EEPROM) \\ \hline
\textbf{ઇનપુટ મોડ્યુલ} & ઇનપુટ ડિવાઈસ સાથે ઇન્ટરફેસ કરે છે, આઇસોલેશન, સિગ્નલ કન્ડિશનિંગ આપે છે \\ \hline
\textbf{આઉટપુટ મોડ્યુલ} & આઉટપુટ ડિવાઈસને ડ્રાઈવ કરે છે, આઇસોલેશન અને પ્રોટેક્શન આપે છે \\ \hline
\textbf{કોમ્યુનિકેશન મોડ્યુલ} & PLC ને નેટવર્ક, અન્ય PLC અને પ્રોગ્રામિંગ ડિવાઈસ સાથે જોડે છે \\ \hline
\textbf{પ્રોગ્રામિંગ ડિવાઈસ} & PLC પ્રોગ્રામ ડેવલપ, એડિટ અને મોનિટર કરવા માટે વપરાય છે \\ \hline
\end{tabulary}
\end{center}

\textbf{PLCના ફાયદાઓ:}
\begin{itemize}
    \item \keyword{રિલાયબિલિટી}: સોલિડ-સ્ટેટ કોમ્પોનન્ટ્સ ઉચ્ચ MTBF સાથે.
    \item \keyword{ફ્લેક્સિબિલિટી}: વિવિધ એપ્લિકેશન્સ માટે સરળતાથી રીપ્રોગ્રામ થઈ શકે છે.
    \item \keyword{કોમ્યુનિકેશન}: ડિસ્ટ્રિબ્યુટેડ કંટ્રોલ માટે નેટવર્ક ક્ષમતાઓ.
    \item \keyword{ડાયગ્નોસ્ટિક્સ}: બિલ્ટ-ઇન ડાયગ્નોસ્ટિક્સ અને ટ્રબલશૂટિંગ.
\end{itemize}
\end{solutionbox}

\begin{mnemonicbox}
\mnemonic{PRIME-C: પાવર સપ્લાય, RAM/ROM મેમરી, ઇનપુટ મોડ્યુલ, માઇક્રોપ્રોસેસર (CPU), એક્ઝિક્યુશન ઓફ પ્રોગ્રામ, કોમ્યુનિકેશન ઇન્ટરફેસ}
\end{mnemonicbox}

\questionmarks{5(a OR)}{3}{સ્તેપર મોટરના ઉપયોગો જણાવો.}

\begin{solutionbox}
\begin{center}
\captionof{table}{Applications of Stepper Motor}
\begin{tabulary}{\linewidth}{|L|L|}
\hline
\textbf{એપ્લિકેશન એરિયા} & \textbf{ઉદાહરણો} \\ \hline
\textbf{પ્રિસિઝન પોઝિશનિંગ} & CNC મશીન્સ, 3D પ્રિન્ટર્સ, રોબોટિક આર્મ્સ \\ \hline
\textbf{ઓફિસ ઇક્વિપમેન્ટ} & પ્રિન્ટર્સ, સ્કેનર્સ, ફોટોકોપિયર્સ \\ \hline
\textbf{મેડિકલ ડિવાઈસ} & સર્જિકલ રોબોટ્સ, ફ્લુઈડ પમ્પ્સ, સેમ્પલ હેન્ડલર્સ \\ \hline
\textbf{ઓટોમોટિવ} & હેડલાઈટ એડજસ્ટમેન્ટ, આઈડલ કંટ્રોલ, મિરર કંટ્રોલ \\ \hline
\textbf{એરોસ્પેસ} & સેટેલાઈટ પોઝિશનિંગ, એન્ટેના કંટ્રોલ \\ \hline
\textbf{કન્ઝ્યુમર ઇલેક્ટ્રોનિક્સ} & કેમેરા (ફોકસ/ઝૂમ), ગેમિંગ કંટ્રોલર્સ \\ \hline
\end{tabulary}
\end{center}
\end{solutionbox}

\begin{mnemonicbox}
\mnemonic{POMAC: પોઝિશનિંગ સિસ્ટમ્સ, ઓફિસ ઇક્વિપમેન્ટ, મેડિકલ ડિવાઈસ, ઓટોમોટિવ કંટ્રોલ્સ, કન્ઝ્યુમર ઇલેક્ટ્રોનિક્સ}
\end{mnemonicbox}

\questionmarks{5(b OR)}{4}{ડીસી સીરીઝ મોટરની ગતિને નિયંત્રિત કરવા માટે સર્કિટ દોરો અને સમજાવો.}

\begin{solutionbox}
\textbf{SCR વડે DC સીરીઝ મોટર સ્પીડ કંટ્રોલ:}

\begin{center}
\begin{tikzpicture}[auto, node distance=2cm]
    \draw (0,2) node[left] {AC} to[short, o-] (1,2) to[D*, l=Bridge] (2,2);
    \draw (2,2) to[Thyristor, l=SCR] (4,2) to[L, l=Field] (5,2) -- (5,1.5) node[draw, circle, minimum size=1cm] (M) {M};
    \draw (5,0.5) -- (5,0) -- (1,0) to[short, -o] (0,0);
    \node at (5.8,1) {Armature};
\end{tikzpicture}
\captionof{figure}{DC Series Motor Speed Control}
\end{center}

\textbf{કાર્ય સિદ્ધાંત:}
\begin{itemize}
    \item \keyword{સીરીઝ કનેક્શન}: ફિલ્ડ વાઈન્ડિંગ આર્મેચર સાથે સીરીઝમાં.
    \item \keyword{SCR કંટ્રોલ}: ફેઝ-કંટ્રોલ્ડ SCR એવરેજ વોલ્ટેજ રેગ્યુલેટ કરે છે.
    \item \keyword{સ્પીડ ઈક્વેશન}: $N \propto (V - I(R_a+R_f))/I\Phi$.
    \item \keyword{સ્પીડ-ટોર્ક રિલેશન}: નોન-લિનિયર રિલેશનશિપ.
    \item \keyword{એપ્લિકેશન}: જ્યાં હાઈ સ્ટાર્ટિંગ ટોર્ક જરૂરી હોય ત્યાં વપરાય છે.
\end{itemize}

\textbf{ફાયદાઓ:}
\begin{itemize}
    \item \keyword{હાઈ સ્ટાર્ટિંગ ટોર્ક}: ટ્રેક્શન એપ્લિકેશન્સ માટે આદર્શ.
    \item \keyword{સિમ્પલ કંટ્રોલ}: બેઝિક સર્કિટ ડિઝાઇન.
    \item \keyword{કોસ્ટ-ઇફેક્ટિવ}: અન્ય પદ્ધતિઓ કરતાં ઓછા કોમ્પોનન્ટ્સ.
\end{itemize}
\end{solutionbox}

\begin{mnemonicbox}
\mnemonic{SCAT: સીરીઝ કનેક્શન, કરંટ કંટ્રોલ્સ ફ્લક્સ, એવરેજ વોલ્ટેજ કંટ્રોલ્ડ બાય SCR, ટોર્ક હાઈએસ્ટ એટ લો સ્પીડ્સ}
\end{mnemonicbox}

\questionmarks{5(c OR)}{7}{BLDC મોટરની વિસ્તૃતમાં ચર્ચા કરો.}

\begin{solutionbox}
\textbf{BLDC મોટર (બ્રશલેસ DC મોટર):}

\begin{center}
\begin{tikzpicture}[auto, node distance=1.5cm]
    \node [gtu block] (controller) {Electronic Controller};
    \node [gtu block, right=of controller] (driver) {Power Driver};
    \node [gtu block, right=of driver] (stator) {Stator Coils};
    \node [gtu block, below=of stator] (rotor) {Rotor (Magnet)};
    \node [gtu block, left=of rotor] (hall) {Hall Sensors};
    
    \draw [gtu arrow] (controller) -- (driver);
    \draw [gtu arrow] (driver) -- (stator);
    \draw [gtu arrow, dashed] (stator) -- (rotor); % Magnetic coupling
    \draw [gtu arrow] (rotor) -- (hall); % Sensing
    \draw [gtu arrow] (hall) -| (controller); % Feedback
    
\end{tikzpicture}
\captionof{figure}{BLDC Motor Control System}
\end{center}

\textbf{રચના:}
\begin{itemize}
    \item \keyword{સ્ટેટર}: વાઈન્ડિંગ્સ ધરાવે છે (સામાન્ય રીતે 3-ફેઝ).
    \item \keyword{રોટર}: રોટર પર પર્મેનન્ટ મેગ્નેટ્સ.
    \item \keyword{પોઝિશન સેન્સિંગ}: હોલ ઇફેક્ટ સેન્સર્સ અથવા એન્કોડર્સ.
    \item \keyword{કંટ્રોલર}: ઇલેક્ટ્રોનિક કોમ્યુટેશન કંટ્રોલર.
\end{itemize}

\textbf{કાર્ય સિદ્ધાંત:}
\begin{itemize}
    \item \keyword{ઇલેક્ટ્રોનિક કોમ્યુટેશન}: મિકેનિકલ બ્રશની જગ્યાએ.
    \item \keyword{સિક્વન્સિંગ}: કંટ્રોલર સ્ટેટર કોઈલ્સને સિક્વન્સમાં એનર્જાઈઝ કરે છે.
    \item \keyword{પોઝિશન ફીડબેક}: હોલ સેન્સર્સ રોટર પોઝિશન નક્કી કરે છે.
    \item \keyword{ફેઝ એનર્જાઈઝિંગ}: રોટર પોઝિશનના આધારે યોગ્ય ફેઝ એનર્જાઈઝ થાય છે.
\end{itemize}

\textbf{ફાયદાઓ:}
\begin{itemize}
    \item \keyword{હાઈ એફિશિયન્સી}: કોઈ બ્રશ ફ્રિક્શન લોસ નહીં.
    \item \keyword{લો મેઈન્ટેનન્સ}: કોઈ બ્રશ વેર નહીં.
    \item \keyword{લાંબુ આયુષ્ય}: વિશ્વસનીય ઓપરેશન.
    \item \keyword{બેટર સ્પીડ-ટોર્ક કેરેક્ટરિસ્ટિક્સ}: ફ્લેટ કર્વ.
    \item \keyword{લો નોઈઝ}: શાંત ઓપરેશન.
    \item \keyword{બેટર હીટ ડિસિપેશન}: સ્ટેટર પર વાઈન્ડિંગ્સ.
\end{itemize}

\textbf{એપ્લિકેશન:}
\begin{itemize}
    \item \keyword{કોમ્પ્યુટર કૂલિંગ ફેન્સ}: CPU/GPU કૂલર્સ.
    \item \keyword{હાર્ડ ડિસ્ક ડ્રાઈવ્સ}: સ્પિન્ડલ મોટર્સ.
    \item \keyword{ઇલેક્ટ્રિક વ્હીકલ્સ}: પ્રોપલ્શન સિસ્ટમ્સ.
    \item \keyword{ડ્રોન્સ}: પ્રોપેલર મોટર્સ.
    \item \keyword{હોમ એપ્લાયન્સેસ}: વોશિંગ મશીન્સ, રેફ્રિજરેટર્સ.
    \item \keyword{ઔદ્યોગિક ઓટોમેશન}: પ્રિસિઝન કંટ્રોલ સિસ્ટમ્સ.
\end{itemize}
\end{solutionbox}

\begin{mnemonicbox}
\mnemonic{COPPER: કોમ્યુટેશન ઇલેક્ટ્રોનિક, ઓપરેશન એફિશિયન્ટ, પર્મેનન્ટ મેગ્નેટ્સ, પોઝિશન સેન્સર્સ, ઇલેક્ટ્રોનિક કંટ્રોલ, રિલાયબલ પરફોર્મન્સ}
\end{mnemonicbox}

\end{document}


\end{document}
