\documentclass[10pt,a4paper]{article}

% content/resources/templates/preamble.tex
\usepackage[margin=0.6in]{geometry}
\author{Milav Dabgar}
\usepackage{amsmath,amssymb,amsthm}
\usepackage{booktabs}
\usepackage{multirow}
\usepackage{xcolor}
\usepackage{tcolorbox}
\tcbuselibrary{breakable,skins}
\usepackage[colorlinks=true,linkcolor=blue]{hyperref}
\usepackage{titlesec}
\usepackage{enumitem}
\usepackage{tikz}
\usepackage{pgfplots}
\usepackage{circuitikz}
\usepackage[version=4]{mhchem}
\usepackage{longtable}
\usepackage{array}
\usepackage{float}
\usepackage{caption}
\usepackage{listings}

\lstset{
  basicstyle=\small\ttfamily,
  breaklines=true,
  breakatwhitespace=false,
  postbreak=\mbox{\textcolor{red}{$\hookrightarrow$}\space},
  float=false,
  numbers=left,
  numberstyle=\tiny\color{gray},
  numbersep=10pt,
  xleftmargin=2em,
  keywordstyle=\color{blue},
  commentstyle=\color{green!60!black},
  stringstyle=\color{purple},
  backgroundcolor=\color{gray!5},
  showstringspaces=false,
  tabsize=2,
  captionpos=b,
  keepspaces=true,
  columns=flexible
}

\pgfplotsset{compat=1.18}
\usetikzlibrary{shapes,arrows,positioning,calc,patterns,decorations.pathmorphing,decorations.markings,arrows.meta}

% Color scheme
\definecolor{headcolor}{RGB}{0,102,204}
\definecolor{keycolor}{RGB}{220,20,60}
\definecolor{solutioncolor}{RGB}{34,139,34}
\definecolor{mnemoniccolor}{RGB}{148,0,211}
\definecolor{codecolor}{RGB}{0,0,100}

% Spacing
\setlength{\parskip}{3pt}
\setlist[itemize]{nosep}
\setlist[enumerate]{nosep}

% Title formatting
\titleformat{\section}{\Large\bfseries\color{headcolor}}{\thesection}{1em}{}
\titleformat{\subsection}{\large\bfseries\color{headcolor}}{\thesubsection}{1em}{}

% Pandoc tightlist compatibility
\providecommand{\tightlist}{%
  \setlength{\itemsep}{0pt}\setlength{\parskip}{0pt}}

% Pandoc longtable compatibility
\newcounter{none}
\def\thenone{}


% content/resources/templates/gujarati-boxes.tex
\usepackage{fontspec}
\usepackage{polyglossia}

% Set Gujarati as main language (document is primarily in Gujarati)
% Note: gloss-gujarati.ldf doesn't exist in polyglossia, but it will use hyphenation patterns
\setdefaultlanguage{gujarati}
\setotherlanguage{english}

% Configure Gujarati font properly
% Use Language=Default to prevent polyglossia from trying to add language-specific features
% that don't exist for Gujarati, which causes "empty feature" warnings
\newfontfamily\gujaratifont[Script=Gujarati,AutoFakeBold=2.5,AutoFakeSlant=0.3]{Noto Sans Gujarati}
\setmainfont[Script=Gujarati,AutoFakeBold=2.5,AutoFakeSlant=0.3]{Noto Sans Gujarati}
% Use Noto Sans Gujarati for monospace to support Gujarati in text
\setmonofont[Scale=0.9]{Noto Sans Gujarati}

% Configure English to use the same font
\newfontfamily\englishfont[Script=Gujarati,AutoFakeBold=2.5,AutoFakeSlant=0.3]{Noto Sans Gujarati}

% Translations for polyglossia
\gappto\captionsgujarati{
  \renewcommand{\tablename}{કોષ્ટક}
  \renewcommand{\figurename}{આકૃતિ}
}

% Helper for TikZ nodes to ensure Gujarati font
\newcommand{\gu}[1]{{\gujaratifont #1}}

% Custom environments
\newtcolorbox{solutionbox}{
    breakable,
    enhanced,
    colback=solutioncolor!5!white,
    colframe=solutioncolor!75!black,
    fonttitle=\bfseries,
    title=જવાબ
}

\newtcolorbox{solutionboxnobreak}{
 colback=solutioncolor!5!white,
 colframe=solutioncolor!75!black,
 fonttitle=\bfseries,
 title=જવાબ
}

\newtcolorbox{keyformula}{
 breakable,
 enhanced,
 colback=keycolor!5!white,
 colframe=keycolor!75!black,
 fonttitle=\bfseries,
 title=રાસાયણિક સમીકરણ/સૂત્ર
}

\newtcolorbox{mnemonicbox}{
 breakable,
 enhanced,
 colback=mnemoniccolor!5!white,
 colframe=mnemoniccolor!75!black,
 fonttitle=\bfseries,
 title=મેમરી ટ્રીક
}


\begin{document}

\begin{center}
{\Huge\bfseries\color{headcolor} Subject Name (Gujarati)}\\[5pt]
{\LARGE 4331103 -- Winter 2023}\\[3pt]
{\large Semester 1 Study Material}\\[3pt]
{\normalsize\textit{Detailed Solutions and Explanations}}
\end{center}

\vspace{10pt}

\subsection*{પ્રશ્ન 1(અ) [3
ગુણ]}\label{uxaaauxab0uxab6uxaa8-1uxa85-3-uxa97uxaa3}

\textbf{SCRનો સિમ્બોલ અને રચના દોરો. તદુપરાંત SCRના ઉપયોગો લખો.}

\begin{solutionbox}

\textbf{SCR સિમ્બોલ અને રચના:}

\begin{verbatim}
    Anode (A)
       |
       v
       \_
      | |
      | |
  G {-|\_|  }
      | |
      |\_|
       |
       v
   Cathode (K)
\end{verbatim}

\textbf{રચના:}

\begin{center}
\textbf{Mermaid Diagram (Code)}
\begin{verbatim}
{Shaded}
{Highlighting}[]
graph LR
    A[Anode] {-{-}{-} P1[P{-}Layer]}
    P1 {-{-}{-} N1[N{-}Layer]}
    N1 {-{-}{-} P2[P{-}Layer]}
    P2 {-{-}{-} N2[N{-}Layer]}
    N2 {-{-}{-} K[Cathode]}
    G[Gate] {-{-}{-} P2}
{Highlighting}
{Shaded}
\end{verbatim}
\end{center}

\textbf{SCRના ઉપયોગો:}

\begin{itemize}
\tightlist
\item
  \textbf{પાવર કંટ્રોલ}: AC/DC પાવર રેગ્યુલેટર્સ
\item
  \textbf{મોટર ડ્રાઈવ્સ}: મોટરની ગતિનું નિયંત્રણ
\item
  \textbf{લાઈટિંગ કંટ્રોલ}: ડિમર સર્કિટ્સ
\item
  \textbf{ઈન્વર્ટર્સ}: DC થી AC રૂપાંતરણ
\end{itemize}

\end{solutionbox}
\begin{mnemonicbox}
``PALS'' - પાવર કંટ્રોલ, એપ્લાયન્સ કંટ્રોલ, લાઈટિંગ સિસ્ટમ્સ,
સ્પીડ રેગ્યુલેટર્સ

\end{mnemonicbox}
\subsection*{પ્રશ્ન 1(બ) [4
ગુણ]}\label{uxaaauxab0uxab6uxaa8-1uxaac-4-uxa97uxaa3}

\textbf{પુરા નામ જણાવો (૧) SCS (૨) LASCR (3) MCT (૪) PUT.}

\begin{solutionbox}

{\def\LTcaptype{none} % do not increment counter
\begin{longtable}[]{@{}ll@{}}
\toprule\noalign{}
ડિવાઇસ & પૂરું નામ \\
\midrule\noalign{}
\endhead
\bottomrule\noalign{}
\endlastfoot
\textbf{SCS} & Silicon Controlled Switch \\
\textbf{LASCR} & Light Activated Silicon Controlled Rectifier \\
\textbf{MCT} & MOS Controlled Thyristor \\
\textbf{PUT} & Programmable Unijunction Transistor \\
\end{longtable}
}

\end{solutionbox}
\begin{mnemonicbox}
``SLaMP'' - Silicon controlled switch, Light
activated SCR, MOS controlled thyristor, Programmable UJT

\end{mnemonicbox}
\subsection*{પ્રશ્ન 1(ક) [7
ગુણ]}\label{uxaaauxab0uxab6uxaa8-1uxa95-7-uxa97uxaa3}

\textbf{TRIACની V-I લાક્ષણિકતા દોરો અને સમજાવો. તદુપરાંત TRIACના ઉપયોગો
લખો.}

\begin{solutionbox}

\textbf{TRIAC V-I લાક્ષણિકતા:}

\begin{center}
\textbf{Mermaid Diagram (Code)}
\begin{verbatim}
{Shaded}
{Highlighting}[]
graph LR
    subgraph "V{-I Characteristics"}
    style V{-I fill:\#f9f9f9,stroke:\#333,stroke{-}width:1px}

    MT2((MT2)) {-{-}{-} O[O]}
    O {-{-}{-} MT1((MT1))}
    
    V1[V] {-{-}{-} I1[I]}
    
    G[Gate Triggering]
    
    quad1[I quadrant] {-{-}{-} quad3[III quadrant]}
    breakover1[Breakover voltage +Vbo] {-{-}{-} breakover2[Breakover voltage {-}Vbo]}
    
    holding1[Holding current +Ih] {-{-}{-} holding2[Holding current {-}Ih]}
    end
{Highlighting}
{Shaded}
\end{verbatim}
\end{center}

\textbf{TRIACની V-I લાક્ષણિકતા સમજૂતી:}

\begin{itemize}
\tightlist
\item
  \textbf{દ્વિદિશાત્મક ઉપકરણ}: બંને દિશામાં વહન કરે છે
\item
  \textbf{ક્વાડ્રન્ટ ઓપરેશન}: પહેલા અને ત્રીજા ક્વાડ્રન્ટમાં કાર્ય કરે છે
\item
  \textbf{બ્રેકઓવર વોલ્ટેજ}: જ્યારે વોલ્ટેજ \pmVbo કરતાં વધે ત્યારે વહન શરૂ થાય
\item
  \textbf{હોલ્ડિંગ કરંટ}: ન્યૂનતમ પ્રવાહ જે વહનની સ્થિતિ જાળવી રાખે છે
\item
  \textbf{ગેટ ટ્રિગરિંગ}: પોઝિટિવ/નેગેટિવ ગેટ વોલ્ટેજથી ટ્રિગર થઈ શકે છે
\end{itemize}

\textbf{TRIACના ઉપયોગો:}

\begin{itemize}
\tightlist
\item
  \textbf{AC પાવર કંટ્રોલ}: લેમ્પ ડિમર્સ, હીટર કંટ્રોલ
\item
  \textbf{મોટર સ્પીડ કંટ્રોલ}: AC મોટર રેગ્યુલેટર્સ
\item
  \textbf{ફેન રેગ્યુલેટર્સ}: ઘરેલું પંખાની ગતિનું નિયંત્રણ
\item
  \textbf{લાઈટ ડિમર્સ}: એડજસ્ટેબલ લાઈટિંગ સિસ્ટમ્સ
\end{itemize}

\end{solutionbox}
\begin{mnemonicbox}
``HALF'' - હીટર્સ, AC કંટ્રોલ, લાઈટિંગ સિસ્ટમ્સ, ફેન
રેગ્યુલેટર્સ

\end{mnemonicbox}
\subsection*{પ્રશ્ન 1(ક) OR [7
ગુણ]}\label{uxaaauxab0uxab6uxaa8-1uxa95-or-7-uxa97uxaa3}

\textbf{IGBT નું કન્સ્ટ્રકશન અને કાર્ય વિગતવાર સમજાવો.}

\begin{solutionbox}

\textbf{IGBT કન્સ્ટ્રકશન અને કાર્ય:}

\begin{center}
\textbf{Mermaid Diagram (Code)}
\begin{verbatim}
{Shaded}
{Highlighting}[]
graph LR
    G[Gate] {-{-}{-} E[Emitter]}
    E {-{-}{-} N+[N+ Layer]}
    N+ {-{-}{-} P[P Layer]}
    P {-{-}{-} N{-}[N{-} Drift Region]}
    N{- {-}{-}{-} N+B[N+ Buffer Layer]}
    N+B {-{-}{-} C[Collector]}
{Highlighting}
{Shaded}
\end{verbatim}
\end{center}

\textbf{રચના વિગતો:}

\begin{itemize}
\tightlist
\item
  \textbf{ત્રણ-ટર્મિનલ ડિવાઈસ}: ગેટ, એમિટર, કલેક્ટર
\item
  \textbf{મલ્ટિલેયર સ્ટ્રક્ચર}: N+, P, N-, N+ બફર, P+ સબસ્ટ્રેટ
\item
  \textbf{હાઈબ્રિડ ડિવાઈસ}: MOSFET ઈનપુટ અને BJT આઉટપુટ લાક્ષણિકતાઓનું સંયોજન
\end{itemize}

\textbf{કાર્ય સિદ્ધાંત:}

\begin{itemize}
\tightlist
\item
  \textbf{ગેટ કંટ્રોલ}: P-રીજનમાં ગેટ પર પોઝિટિવ વોલ્ટેજ ઇન્વર્ઝન લેયર બનાવે છે
\item
  \textbf{ચેનલ ફોર્મેશન}: ઇલેક્ટ્રોન્સ N+ એમિટરથી N- ડ્રિફ્ટ રીજન તરફ વહે છે
\item
  \textbf{કન્ડક્ટિવિટી મોડ્યુલેશન}: P-N- જંક્શન હોલ્સ ઇન્જેક્ટ કરે છે, રેઝિસ્ટન્સ ઘટાડે છે
\item
  \textbf{ટર્ન-ઓફ પ્રક્રિયા}: ગેટ વોલ્ટેજ દૂર કરવાથી ઇલેક્ટ્રોન ફ્લો બંધ થઈ જાય છે
\end{itemize}

\textbf{IGBTના ફાયદા:}

\begin{itemize}
\tightlist
\item
  \textbf{ઊંચી ઈનપુટ ઇમ્પીડન્સ}: સરળ વોલ્ટેજ નિયંત્રણ
\item
  \textbf{ઓછા કન્ડક્શન લોસ}: કાર્યક્ષમ પાવર હેન્ડલિંગ
\item
  \textbf{ઝડપી સ્વિચિંગ}: ઉચ્ચ ફ્રીક્વન્સી એપ્લિકેશન્સ માટે યોગ્ય
\end{itemize}

\end{solutionbox}
\begin{mnemonicbox}
``GIVE'' - ગેટ કંટ્રોલ્ડ, ઇનપુટ હાઈ ઇમ્પીડન્સ, વોલ્ટેજ ડ્રિવન,
એફિશિયન્ટ કન્ડક્શન

\end{mnemonicbox}
\subsection*{પ્રશ્ન 2(અ) [3
ગુણ]}\label{uxaaauxab0uxab6uxaa8-2uxa85-3-uxa97uxaa3}

\textbf{UJTની મદદથી રિલેક્ષેશન ઓસિલેટર સર્કિટની ચર્ચા કરો.}

\begin{solutionbox}

\textbf{UJT રિલેક્ષેશન ઓસિલેટર:}

\begin{center}
\textbf{Mermaid Diagram (Code)}
\begin{verbatim}
{Shaded}
{Highlighting}[]
graph LR
    VCC[VCC] {-{-}{-} R1[R1] {-}{-}{-} E[Emitter]}
    E {-{-}{-} C[Capacitor] {-}{-}{-} GND[GND]}
    E {-{-}{-} UJT[UJT]}
    UJT {-{-}{-} B1[Base 1] {-}{-}{-} R2[R2] {-}{-}{-} GND}
    UJT {-{-}{-} B2[Base 2] {-}{-}{-} R3[R3] {-}{-}{-} VCC}
    B1 {-{-}{-} Output[Output]}
{Highlighting}
{Shaded}
\end{verbatim}
\end{center}

\textbf{કાર્ય સિદ્ધાંત:}

\begin{itemize}
\tightlist
\item
  \textbf{કેપેસિટર ચાર્જિંગ}: C, R1 દ્વારા UJT ફાયરિંગ વોલ્ટેજ સુધી ચાર્જ થાય છે
\item
  \textbf{UJT ફાયર}: જ્યારે એમિટર વોલ્ટેજ પીક પોઈન્ટ વોલ્ટેજ સુધી પહોંચે ત્યારે
\item
  \textbf{ડિસ્ચાર્જ સાયકલ}: કેપેસિટર એમિટર-બેઝ1 જંક્શન દ્વારા ડિસ્ચાર્જ થાય છે
\item
  \textbf{ઓસિલેશન}: પ્રક્રિયા પુનરાવર્તિત થાય છે અને સોટૂથ વેવફોર્મ બનાવે છે
\end{itemize}

\end{solutionbox}
\begin{mnemonicbox}
``CROP'' - કેપેસિટર ચાર્જ થાય, રીચ થ્રેશોલ્ડ, ઓસિલેટ થાય,
પ્રોડ્યુસ સોટૂથ

\end{mnemonicbox}
\subsection*{પ્રશ્ન 2(બ) [4
ગુણ]}\label{uxaaauxab0uxab6uxaa8-2uxaac-4-uxa97uxaa3}

\textbf{SCRની ટ્રીગરિંગ પદ્ધતિઓની ચર્ચા કરો.}

\begin{solutionbox}

{\def\LTcaptype{none} % do not increment counter
\begin{longtable}[]{@{}ll@{}}
\toprule\noalign{}
ટ્રિગરિંગ પદ્ધતિ & કાર્ય સિદ્ધાંત \\
\midrule\noalign{}
\endhead
\bottomrule\noalign{}
\endlastfoot
\textbf{ગેટ ટ્રિગરિંગ} & ગેટ અને કેથોડ વચ્ચે પોઝિટિવ વોલ્ટેજ આપવામાં આવે છે \\
\textbf{થર્મલ ટ્રિગરિંગ} & તાપમાન વધારાથી બ્રેકઓવર વોલ્ટેજ ઘટે છે \\
\textbf{લાઈટ ટ્રિગરિંગ} & ફોટોન્સ LASCR માં ઇલેક્ટ્રોન-હોલ જોડ બનાવે છે \\
\textbf{dv/dt ટ્રિગરિંગ} & SCR પર ઝડપી વોલ્ટેજ વધારો કેપેસિટિવ કરંટ ઉત્પન્ન કરે
છે \\
\textbf{બ્રેકઓવર ટ્રિગરિંગ} & ગેટ સિગ્નલ વિના વોલ્ટેજ બ્રેકઓવર વોલ્ટેજને ઓળંગે છે \\
\end{longtable}
}

\textbf{મુખ્ય મુદ્દાઓ:}

\begin{itemize}
\tightlist
\item
  \textbf{ગેટ ટ્રિગરિંગ}: સૌથી સામાન્ય પદ્ધતિ
\item
  \textbf{લાઈટ ટ્રિગરિંગ}: ઓપ્ટો-આઇસોલેટર્સમાં વપરાય છે
\item
  \textbf{dv/dt ટ્રિગરિંગ}: ઘણી વખત અવાંછનીય, સ્નબર સર્કિટની જરૂર પડે છે
\end{itemize}

\end{solutionbox}
\begin{mnemonicbox}
``GLTDB'' - ગેટ, લાઈટ, થર્મલ, dv/dt, બ્રેકઓવર

\end{mnemonicbox}
\subsection*{પ્રશ્ન 2(ક) [7
ગુણ]}\label{uxaaauxab0uxab6uxaa8-2uxa95-7-uxa97uxaa3}

\textbf{ક્લાસ એ પ્રકારની કોમ્યુટેશન પદ્ધતિ સમજાવો.}

\begin{solutionbox}

\textbf{ક્લાસ A કોમ્યુટેશન (LC સર્કિટ દ્વારા સેલ્ફ-કોમ્યુટેશન):}

\begin{center}
\textbf{Mermaid Diagram (Code)}
\begin{verbatim}
{Shaded}
{Highlighting}[]
graph LR
    DC\_Source[DC Source] {-{-}{-} SCR[SCR] {-}{-}{-} Load[Load]}
    SCR {-{-}{-} L[Inductor] {-}{-}{-} C[Capacitor]}
    C {-{-}{-} SW[Switch] {-}{-}{-} DC\_Source}
{Highlighting}
{Shaded}
\end{verbatim}
\end{center}

\textbf{કાર્ય સિદ્ધાંત:}

\begin{itemize}
\tightlist
\item
  \textbf{પ્રારંભિક સ્થિતિ}: SCR વહન કરે છે, કેપેસિટર જમણી બાજુએ (+) પોલારિટી
  સાથે ચાર્જ થયેલ છે
\item
  \textbf{કોમ્યુટેશન શરૂઆત}: જ્યારે સ્વિચ SW બંધ થાય છે
\item
  \textbf{રેઝોનન્ટ સર્કિટ}: LC સર્કિટ રેઝોનન્ટ પાથ બનાવે છે
\item
  \textbf{રિવર્સ કરંટ}: કેપેસિટર ડિસ્ચાર્જ SCR મારફતે રિવર્સ કરંટ ઉત્પન્ન કરે છે
\item
  \textbf{ટર્ન-ઓફ}: જ્યારે કરંટ હોલ્ડિંગ કરંટથી નીચે પડે ત્યારે SCR બંધ થાય છે
\item
  \textbf{રિચાર્જિંગ}: કેપેસિટર વિપરીત પોલારિટી સાથે રિચાર્જ થાય છે
\end{itemize}

\textbf{એપ્લિકેશન:}

\begin{itemize}
\tightlist
\item
  \textbf{ઇન્વર્ટર સર્કિટ્સ}: DC થી AC રૂપાંતરણ
\item
  \textbf{ચોપર સર્કિટ્સ}: DC થી DC રૂપાંતરણ
\end{itemize}

\end{solutionbox}
\begin{mnemonicbox}
``SCCRRT'' - સ્વિચ ક્લોઝ થાય, કેપેસિટર ડિસ્ચાર્જ થાય, કરંટ
રિવર્સ થાય, SCR ટર્ન ઓફ થાય, રિચાર્જિંગ શરૂ થાય, ટર્ન-ઓફ પૂર્ણ થાય

\end{mnemonicbox}
\subsection*{પ્રશ્ન 2(અ) OR [3
ગુણ]}\label{uxaaauxab0uxab6uxaa8-2uxa85-or-3-uxa97uxaa3}

\textbf{GTOનું પૂરું નામ જણાવો અને GTOની રચના દોરો.}

\begin{solutionbox}

\textbf{GTOનું પૂરું નામ:} Gate Turn-Off Thyristor

\textbf{GTOની રચના:}

\begin{center}
\textbf{Mermaid Diagram (Code)}
\begin{verbatim}
{Shaded}
{Highlighting}[]
graph LR
    A[Anode] {-{-}{-} P1[P+ Anode Layer]}
    P1 {-{-}{-} N[N Base Layer]}
    N {-{-}{-} P2[P Base Layer]}
    P2 {-{-}{-} N2[N+ Cathode Layer]}
    N2 {-{-}{-} K[Cathode]}
    G[Gate] {-{-}{-} P2}
{Highlighting}
{Shaded}
\end{verbatim}
\end{center}

\end{solutionbox}
\begin{mnemonicbox}
``PANG'' - P-એનોડ, એન્ડ, N-બેઝ, ગેટ-કંટ્રોલ્ડ થાયરિસ્ટર

\end{mnemonicbox}
\subsection*{પ્રશ્ન 2(બ) OR [4
ગુણ]}\label{uxaaauxab0uxab6uxaa8-2uxaac-or-4-uxa97uxaa3}

\textbf{SCR માટેની સ્નબર સર્કિટની રચના અને જરૂરિયાતની ચર્ચા કરો.}

\begin{solutionbox}

\textbf{SCR માટે સ્નબર સર્કિટ:}

\begin{center}
\textbf{Mermaid Diagram (Code)}
\begin{verbatim}
{Shaded}
{Highlighting}[]
graph LR
    SCR[SCR] {-{-}{-} R[Resistor] {-}{-}{-} C[Capacitor]}
    C {-{-}{-} SCR}
{Highlighting}
{Shaded}
\end{verbatim}
\end{center}

\textbf{ડિઝાઇન જરૂરિયાતો:}

\begin{itemize}
\tightlist
\item
  \textbf{રેઝિસ્ટર પસંદગી}: કેપેસિટર ડિસ્ચાર્જ કરંટને મર્યાદિત કરે છે
\item
  \textbf{કેપેસિટર પસંદગી}: વોલ્ટેજ વૃદ્ધિના દર (dv/dt)ને નિયંત્રિત કરે છે
\item
  \textbf{RC ટાઇમ કોન્સ્ટન્ટ}: રિસ્પોન્સ ટાઈમ નક્કી કરે છે
\end{itemize}

\textbf{સ્નબર સર્કિટનો હેતુ:}

\begin{itemize}
\tightlist
\item
  \textbf{dv/dt પ્રોટેક્શન}: ઝડપી વોલ્ટેજ પરિવર્તનને લીધે ખોટા ટ્રિગરિંગને અટકાવે છે
\item
  \textbf{વોલ્ટેજ સ્પાઈક સપ્રેશન}: ઇન્ડક્ટિવ વોલ્ટેજ સ્પાઈક્સને શોષે છે
\item
  \textbf{ટ્રાન્ઝિયન્ટ પ્રોટેક્શન}: સ્વિચિંગ દરમિયાન SCRને રક્ષણ આપે છે
\end{itemize}

\end{solutionbox}
\begin{mnemonicbox}
``RAPE'' - રેઝિસ્ટર એન્ડ કેપેસિટર પ્રોટેક્ટ અગેઇન્સ્ટ એક્સેસિવ
વોલ્ટેજ રાઇઝ

\end{mnemonicbox}
\subsection*{પ્રશ્ન 2(ક) OR [7
ગુણ]}\label{uxaaauxab0uxab6uxaa8-2uxa95-or-7-uxa97uxaa3}

\textbf{ક્લાસ સી પ્રકારની કોમ્યુટેશન પદ્ધતિ સમજાવો.}

\begin{solutionbox}

\textbf{ક્લાસ C કોમ્યુટેશન (કોમ્પ્લિમેન્ટરી કોમ્યુટેશન):}

\begin{center}
\textbf{Mermaid Diagram (Code)}
\begin{verbatim}
{Shaded}
{Highlighting}[]
graph LR
    DC\_Source[DC Source] {-{-}{-} SCR1[SCR1] {-}{-}{-} Load1[Load 1]}
    DC\_Source {-{-}{-} SCR2[SCR2] {-}{-}{-} Load2[Load 2]}
    SCR1 {-{-}{-} SCR2}
{Highlighting}
{Shaded}
\end{verbatim}
\end{center}

\textbf{કાર્ય સિદ્ધાંત:}

\begin{itemize}
\tightlist
\item
  \textbf{પ્રારંભિક સ્થિતિ}: SCR1 વહન કરે છે, SCR2 બંધ છે
\item
  \textbf{કોમ્યુટેશન શરૂઆત}: SCR2 ટ્રિગર થાય છે
\item
  \textbf{લોડ ટ્રાન્સફર}: કરંટ SCR1 થી SCR2 માં ટ્રાન્સફર થાય છે
\item
  \textbf{વોલ્ટેજ રિવર્સલ}: SCR1 પર વોલ્ટેજ નેગેટિવ થાય છે
\item
  \textbf{ટર્ન-ઓફ}: જ્યારે કરંટ હોલ્ડિંગ કરંટથી નીચે પડે ત્યારે SCR1 બંધ થાય છે
\item
  \textbf{વૈકલ્પિક ઓપરેશન}: SCR1 અને SCR2 વૈકલ્પિક રીતે વહન કરે છે
\end{itemize}

\textbf{એપ્લિકેશન:}

\begin{itemize}
\tightlist
\item
  \textbf{ઇન્વર્ટર સર્કિટ્સ}: બ્રિજ ઇન્વર્ટરમાં વપરાય છે
\item
  \textbf{ડ્યુઅલ લોડ સિસ્ટમ્સ}: જ્યાં વૈકલ્પિક ઓપરેશનની જરૂર હોય
\end{itemize}

\end{solutionbox}
\begin{mnemonicbox}
``TACTOR'' - ટ્રિગરિંગ ઓલ્ટરનેટ SCRs ક્રિએટ્સ ટર્ન-ઓફ એન્ડ
રિવર્સલ

\end{mnemonicbox}
\subsection*{પ્રશ્ન 3(અ) [3
ગુણ]}\label{uxaaauxab0uxab6uxaa8-3uxa85-3-uxa97uxaa3}

\textbf{પોલીફેઝ રેક્ટિફાયરના ફાયદા વર્ણવો.}

\begin{solutionbox}

{\def\LTcaptype{none} % do not increment counter
\begin{longtable}[]{@{}ll@{}}
\toprule\noalign{}
ફાયદા & વર્ણન \\
\midrule\noalign{}
\endhead
\bottomrule\noalign{}
\endlastfoot
\textbf{ઉચ્ચ કાર્યક્ષમતા} & ઓછું પાવર લોસ અને ટ્રાન્સફોર્મર વપરાશમાં સુધારો \\
\textbf{ઓછો રિપલ ફેક્ટર} & વધુ સારો DC આઉટપુટ જેથી નાના ફિલ્ટર કોમ્પોનન્ટ્સ
જોઈએ \\
\textbf{ઉચ્ચ પાવર હેન્ડલિંગ} & સિંગલ ફેઝ કરતાં વધુ પાવર લેવલ હેન્ડલ કરી શકે છે \\
\textbf{બેટર ટ્રાન્સફોર્મર ઉપયોગ} & ઉચ્ચ ટ્રાન્સફોર્મર ઉપયોગિતા ફેક્ટર \\
\textbf{ઓછી હાર્મોનિક સામગ્રી} & આઉટપુટમાં ઘટેલા હાર્મોનિક ડિસ્ટોર્શન \\
\end{longtable}
}

\end{solutionbox}
\begin{mnemonicbox}
``HELPS'' - ઉચ્ચ કાર્યક્ષમતા, ઈવન આઉટપુટ, ઓછો રિપલ, પાવર
હેન્ડલિંગ બેટર, નાના ફિલ્ટર

\end{mnemonicbox}
\subsection*{પ્રશ્ન 3(બ) [4
ગુણ]}\label{uxaaauxab0uxab6uxaa8-3uxaac-4-uxa97uxaa3}

\textbf{સિંગલ ફેઇઝ હાફવેવ રેક્ટીફાયર સર્કિટ દોરો અને સમજાવો. વેવફોર્મ્સ દોરો.}

\begin{solutionbox}

\textbf{સિંગલ ફેઝ હાફ વેવ રેક્ટિફાયર:}

\begin{center}
\textbf{Mermaid Diagram (Code)}
\begin{verbatim}
{Shaded}
{Highlighting}[]
graph LR
    AC[AC Supply] {-{-}{-} D[Diode] {-}{-}{-} R[Load Resistor]}
    R {-{-}{-} AC}
{Highlighting}
{Shaded}
\end{verbatim}
\end{center}

\textbf{વેવફોર્મ:}

\begin{verbatim}
    Voltage
      \^{}
      |     /{      /      /}
      |    /  {    /      /  }
      |{-{-}{-}/{-}{-}{-}{-}{-}{-}/{-}{-}{-}{-}{-}{-}/{-}{-}{-}{-}{-}{-}{-}{-} Time}
      |         {              }
      |          {              }
      |
   Input AC
   
    Voltage
      \^{}
      |     /{      /      /}
      |    /  {    /      /  }
      |{-{-}{-}/{-}{-}{-}{-}{-}{-}/{-}{-}{-}{-}{-}{-}/{-}{-}{-}{-}{-}{-}{-}{-} Time}
      |    
      |    
      |
   Output DC (Pulsating)
\end{verbatim}

\textbf{કાર્ય સિદ્ધાંત:}

\begin{itemize}
\tightlist
\item
  \textbf{ફોરવર્ડ બાયસ}: ડાયોડ પોઝિટિવ હાફ-સાયકલ દરમિયાન વહન કરે છે
\item
  \textbf{રિવર્સ બાયસ}: ડાયોડ નેગેટિવ હાફ-સાયકલ દરમિયાન કરંટને અવરોધે છે
\item
  \textbf{આઉટપુટ}: પલ્સેટિંગ DC જેનો રિપલ ફેક્ટર ઊંચો હોય છે
\item
  \textbf{ફ્રિક્વન્સી}: આઉટપુટ ફ્રિક્વન્સી ઇનપુટ ફ્રિક્વન્સી જેટલી જ રહે છે
\end{itemize}

\end{solutionbox}
\begin{mnemonicbox}
``PROF'' - પોઝિટિવ હાફ કન્ડક્ટ્સ, રિવર્સ હાફ બ્લોક્સ,
આઉટપુટ ઇઝ પલ્સેટિંગ, ફ્રિક્વન્સી અનચેન્જ્ડ

\end{mnemonicbox}
\subsection*{પ્રશ્ન 3(ક) [7
ગુણ]}\label{uxaaauxab0uxab6uxaa8-3uxa95-7-uxa97uxaa3}

\textbf{બધાજ પ્રકારના ઇન્વર્ટરની યાદી બનાવો. તેમાંથી સિંગલફેઝ ફુલ બ્રિજ ઇન્વર્ટર
સમજાવો.}

\begin{solutionbox}

\textbf{ઇન્વર્ટરના પ્રકારો:}

\begin{enumerate}
\tightlist
\item
  સર્કિટના આધારે: સીરીઝ, પેરેલલ, બ્રિજ
\item
  ફેઝના આધારે: સિંગલ-ફેઝ, થ્રી-ફેઝ
\item
  આઉટપુટના આધારે: સ્ક્વેર વેવ, મોડિફાઇડ સાઇન વેવ, પ્યોર સાઇન વેવ
\item
  કોમ્યુટેશનના આધારે: SCR-બેઝ્ડ, ટ્રાન્ઝિસ્ટર-બેઝ્ડ
\end{enumerate}

\textbf{સિંગલ ફેઝ ફુલ બ્રિજ ઇન્વર્ટર:}

\begin{center}
\textbf{Mermaid Diagram (Code)}
\begin{verbatim}
{Shaded}
{Highlighting}[]
graph LR
    DC[DC Source] {-{-}{-} S1[Switch S1] {-}{-}{-} S2[Switch S2] {-}{-}{-} DC}
    S1 {-{-}{-} Load[Load] {-}{-}{-} S3[Switch S3]}
    S2 {-{-}{-} Load}
    S3 {-{-}{-} S4[Switch S4] {-}{-}{-} DC}
{Highlighting}
{Shaded}
\end{verbatim}
\end{center}

\textbf{કાર્ય સિદ્ધાંત:}

\begin{itemize}
\tightlist
\item
  \textbf{પ્રથમ અર્ધ-સાયકલ}: S1 અને S4 ON, S2 અને S3 OFF
\item
  \textbf{બીજો અર્ધ-સાયકલ}: S2 અને S3 ON, S1 અને S4 OFF
\item
  \textbf{આઉટપુટ વેવફોર્મ}: લોડ પર AC સ્ક્વેર વેવ
\item
  \textbf{કંટ્રોલ મેથડ}: સ્વિચને 180^\circ ફેઝ શિફ્ટ સાથે ગેટ સિગ્નલ આપવામાં આવે છે
\end{itemize}

\textbf{ફાયદાઓ:}

\begin{itemize}
\tightlist
\item
  \textbf{ઉચ્ચ આઉટપુટ પાવર}: હાફ બ્રિજની તુલનામાં બમણો આઉટપુટ
\item
  \textbf{બેટર વોલ્ટેજ ઉપયોગ}: લોડ પર સંપૂર્ણ DC બસ વોલ્ટેજ
\item
  \textbf{ઓછું કરંટ રેટિંગ}: દરેક સ્વિચ માત્ર લોડ કરંટ જ વહન કરે છે
\end{itemize}

\end{solutionbox}
\begin{mnemonicbox}
``SOAP'' - સ્વિચેસ ઓપરેટ ઓલ્ટરનેટલી ઇન પેર્સ

\end{mnemonicbox}
\subsection*{પ્રશ્ન 3(અ) OR [3
ગુણ]}\label{uxaaauxab0uxab6uxaa8-3uxa85-or-3-uxa97uxaa3}

\textbf{સરખાવો UPS અને SMPS.}

\begin{solutionbox}

{\def\LTcaptype{none} % do not increment counter
\begin{longtable}[]{@{}
  >{\raggedright\arraybackslash}p{(\linewidth - 4\tabcolsep) * \real{0.1325}}
  >{\raggedright\arraybackslash}p{(\linewidth - 4\tabcolsep) * \real{0.4458}}
  >{\raggedright\arraybackslash}p{(\linewidth - 4\tabcolsep) * \real{0.4217}}@{}}
\toprule\noalign{}
\begin{minipage}[b]{\linewidth}\raggedright
પેરામીટર
\end{minipage} & \begin{minipage}[b]{\linewidth}\raggedright
UPS (અનઇન્ટરપ્ટિબલ પાવર સપ્લાય)
\end{minipage} & \begin{minipage}[b]{\linewidth}\raggedright
SMPS (સ્વિચ્ડ મોડ પાવર સપ્લાય)
\end{minipage} \\
\midrule\noalign{}
\endhead
\bottomrule\noalign{}
\endlastfoot
\textbf{મુખ્ય કાર્ય} & પાવર ફેઇલ થાય ત્યારે બેકઅપ પાવર આપે છે & AC થી રેગ્યુલેટેડ DC
માં રૂપાંતર કરે છે \\
\textbf{બેટરી બેકઅપ} & બેકઅપ માટે બેટરી ધરાવે છે & કોઈ બેટરી બેકઅપ નથી \\
\textbf{આઉટપુટ} & AC આઉટપુટ (મોટેભાગે) & DC આઉટપુટ (મોટેભાગે) \\
\textbf{કાર્યક્ષમતા} & ઓછી (70-80\%) & ઉચ્ચ (80-95\%) \\
\textbf{સાઇઝ} & મોટું અને ભારે & કોમ્પેક્ટ અને હલકું \\
\textbf{એપ્લિકેશન} & કોમ્પ્યુટર, સર્વર, ક્રિટિકલ ઇક્વિપમેન્ટ & ઇલેક્ટ્રોનિક ડિવાઇસ,
ચાર્જર \\
\end{longtable}
}

\end{solutionbox}
\begin{mnemonicbox}
``BBOSS'' - બેકઅપ બેટરી ઓન્લી ઇન UPS, સ્મોલ સાઇઝ ઇન SMPS

\end{mnemonicbox}
\subsection*{પ્રશ્ન 3(બ) OR [4
ગુણ]}\label{uxaaauxab0uxab6uxaa8-3uxaac-or-4-uxa97uxaa3}

\textbf{થ્રી ફેઇઝ હાફ વેવ રેક્ટીફાયર સર્કિટ દોરો અને સમજાવો. વેવફોર્મ્સદોરો.}

\begin{solutionbox}

\textbf{થ્રી ફેઝ હાફ વેવ રેક્ટિફાયર:}

\begin{center}
\textbf{Mermaid Diagram (Code)}
\begin{verbatim}
{Shaded}
{Highlighting}[]
graph LR
    R[R Phase] {-{-}{-} D1[Diode D1] {-}{-}{-} Load[Load]}
    Y[Y Phase] {-{-}{-} D2[Diode D2] {-}{-}{-} Load}
    B[B Phase] {-{-}{-} D3[Diode D3] {-}{-}{-} Load}
    Load {-{-}{-} N[Neutral]}
{Highlighting}
{Shaded}
\end{verbatim}
\end{center}

\textbf{વેવફોર્મ:}

\begin{verbatim}
     Voltage
       \^{}
       |   
       |    /{    /    /    /    /    /}
       |   /  {  /    /    /    /    /  }
       |{-{-}/{-}{-}{-}{-}/{-}{-}{-}{-}/{-}{-}{-}{-}/{-}{-}{-}{-}/{-}{-}{-}{-}/{-}{-}{-}{-}{-}{-} Time}
       |   R    Y    B    R    Y    B    R
       |   
    Input (Three phase)
    
     Voltage
       \^{}
       |   
       |    /{    /    /    /    /    /}
       |   /  {  /    /    /    /    /  }
       |{-{-}/{-}{-}{-}{-}/{-}{-}{-}{-}/{-}{-}{-}{-}/{-}{-}{-}{-}/{-}{-}{-}{-}/{-}{-}{-}{-}{-}{-} Time}
       |      
       |   
    Output DC (Less ripple)
\end{verbatim}

\textbf{કાર્ય સિદ્ધાંત:}

\begin{itemize}
\tightlist
\item
  \textbf{કન્ડક્શન સિક્વન્સ}: જ્યારે તેની ફેઝ વોલ્ટેજ સૌથી વધુ હોય ત્યારે દરેક ડાયોડ
  વહન કરે છે
\item
  \textbf{કન્ડક્શન એંગલ}: દરેક ડાયોડ 120^\circ માટે વહન કરે છે
\item
  \textbf{આઉટપુટ રિપલ}: સાયકલ દીઠ 3 પલ્સ, સિંગલ ફેઝ કરતાં ઓછો રિપલ
\item
  \textbf{રિપલ ફ્રિક્વન્સી}: ઇનપુટ ફ્રિક્વન્સીથી 3 ગણી
\end{itemize}

\end{solutionbox}
\begin{mnemonicbox}
``CROP'' - કન્ડક્શન ઓફ 120^\circ, રિપલ રિડ્યુસ્ડ, આઉટપુટ સ્મૂધર,
પલ્સ ટ્રિપલ્ડ

\end{mnemonicbox}
\subsection*{પ્રશ્ન 3(ક) OR [7
ગુણ]}\label{uxaaauxab0uxab6uxaa8-3uxa95-or-7-uxa97uxaa3}

\textbf{ચોપરને વ્યાખ્યાયિત કરો. ક્લાસ ડી ચોપરનો પરિપથ દોરો અને સમજાવો.}

\begin{solutionbox}

\textbf{ચોપરની વ્યાખ્યા:} ચોપર એ DC થી DC કન્વર્ટર છે જે ફિક્સ્ડ DC ઇનપુટ વોલ્ટેજને
હાઈ-ફ્રિક્વન્સી સ્વિચિંગનો ઉપયોગ કરીને વેરિએબલ DC આઉટપુટ વોલ્ટેજમાં રૂપાંતરિત કરે છે.

\textbf{ક્લાસ D ચોપર (બે-ક્વાડ્રન્ટ ચોપર):}

\begin{center}
\textbf{Mermaid Diagram (Code)}
\begin{verbatim}
{Shaded}
{Highlighting}[]
graph LR
    VS[DC Source] {-{-}{-} S1[Switch S1] {-}{-}{-} L[Inductor]}
    L {-{-}{-} Load[Load] {-}{-}{-} VS}
    Load {-{-}{-} D1[Diode D1] {-}{-}{-} S1}
    Load {-{-}{-} S2[Switch S2] {-}{-}{-} D2[Diode D2] {-}{-}{-} VS}
{Highlighting}
{Shaded}
\end{verbatim}
\end{center}

\textbf{કાર્ય સિદ્ધાંત:}

\begin{itemize}
\tightlist
\item
  \textbf{પ્રથમ ક્વાડ્રન્ટ ઓપરેશન (ફોરવર્ડ મોટરિંગ):}

  \begin{itemize}
  \tightlist
  \item
    S1 ON, S2 OFF: ઊર્જા સ્ત્રોતથી લોડ તરફ વહે છે
  \item
    S1 OFF, S2 OFF: કરંટ D2 દ્વારા ફ્રીવ્હીલ થાય છે
  \end{itemize}
\item
  \textbf{બીજા ક્વાડ્રન્ટ ઓપરેશન (ફોરવર્ડ રિજનરેશન):}

  \begin{itemize}
  \tightlist
  \item
    S1 OFF, S2 ON: ઊર્જા લોડથી સ્ત્રોત તરફ વહે છે
  \item
    S1 OFF, S2 OFF: કરંટ D1 દ્વારા ફ્રીવ્હીલ થાય છે
  \end{itemize}
\end{itemize}

\textbf{એપ્લિકેશન:}

\begin{itemize}
\tightlist
\item
  \textbf{DC મોટર ડ્રાઇવ}: ફોરવર્ડ મોટરિંગ અને રિજનરેટિવ બ્રેકિંગ પ્રદાન કરે છે
\item
  \textbf{બેટરી ચાર્જિંગ}: ચાર્જિંગ કરંટનું નિયંત્રણ
\item
  \textbf{રીન્યુએબલ એનર્જી}: સોલાર પેનલ સાથે ઇન્ટરફેસિંગ
\end{itemize}

\end{solutionbox}
\begin{mnemonicbox}
``FRED'' - ફોરવર્ડ મોટરિંગ, રિજનરેટિવ બ્રેકિંગ, એનર્જી ફ્લો
કંટ્રોલ, ડ્યુઅલ ક્વાડ્રન્ટ ઓપરેશન

\end{mnemonicbox}
\subsection*{પ્રશ્ન 4(અ) [3
ગુણ]}\label{uxaaauxab0uxab6uxaa8-4uxa85-3-uxa97uxaa3}

\textbf{SCRનો સ્ટેટિક સ્વીચ તરીકેનો ઉપયોગ સમજાવો.}

\begin{solutionbox}

\textbf{SCR એઝ સ્ટેટિક સ્વિચ:}

\begin{center}
\textbf{Mermaid Diagram (Code)}
\begin{verbatim}
{Shaded}
{Highlighting}[]
graph LR
    VS[Supply] {-{-}{-} SCR[SCR] {-}{-}{-} Load[Load]}
    GC[Gate Control] {-{-}{-} SCR}
{Highlighting}
{Shaded}
\end{verbatim}
\end{center}

\textbf{મુખ્ય વિશેષતાઓ:}

\begin{itemize}
\tightlist
\item
  \textbf{કોઈ મૂવિંગ પાર્ટ્સ નહીં}: શુદ્ધ ઇલેક્ટ્રોનિક સ્વિચિંગ
\item
  \textbf{ઝડપી સ્વિચિંગ}: માઇક્રોસેકન્ડ રિસ્પોન્સ ટાઈમ
\item
  \textbf{ઉચ્ચ વિશ્વસનીયતા}: મિકેનિકલ સ્વિચ કરતાં લાંબું આયુષ્ય
\item
  \textbf{નિયંત્રિત ટર્ન-ઓન}: ગેટ સિગ્નલ દ્વારા ચોક્કસ નિયંત્રણ
\end{itemize}

\textbf{મિકેનિકલ સ્વિચ કરતાં ફાયદા:}

\begin{itemize}
\tightlist
\item
  \textbf{કોઈ આર્કિંગ નહીં}: કોઈ કોન્ટેક્ટ બાઉન્સ કે ઘસારો નહીં
\item
  \textbf{સાયલેન્ટ ઓપરેશન}: કોઈ મિકેનિકલ અવાજ નહીં
\item
  \textbf{EMI ઘટાડો}: ઓછું ઇલેક્ટ્રોમેગ્નેટિક ઇન્ટરફેરન્સ
\end{itemize}

\end{solutionbox}
\begin{mnemonicbox}
``FANS'' - ફાસ્ટ સ્વિચિંગ, આર્ક-ફ્રી ઓપરેશન, નો મિકેનિકલ
વેર, સાયલેન્ટ ઓપરેશન

\end{mnemonicbox}
\subsection*{પ્રશ્ન 4(બ) [4
ગુણ]}\label{uxaaauxab0uxab6uxaa8-4uxaac-4-uxa97uxaa3}

\textbf{DIAC અને TRIACનો ઉપયોગ કરી A.C પાવર કંટ્રોલનો સર્કિટ ડાયગ્રામ દોરો અને
તેનું કાર્ય સમજાવો.}

\begin{solutionbox}

\textbf{DIAC અને TRIAC વડે AC પાવર કંટ્રોલ:}

\begin{center}
\textbf{Mermaid Diagram (Code)}
\begin{verbatim}
{Shaded}
{Highlighting}[]
graph LR
    AC[AC Supply] {-{-}{-} TRIAC[TRIAC] {-}{-}{-} Load[Load]}
    AC {-{-}{-} R[Resistor] {-}{-}{-} C[Capacitor] {-}{-}{-} DIAC[DIAC] {-}{-}{-} G[TRIAC Gate]}
    G {-{-}{-} TRIAC}
{Highlighting}
{Shaded}
\end{verbatim}
\end{center}

\textbf{કાર્ય સિદ્ધાંત:}

\begin{itemize}
\tightlist
\item
  \textbf{RC નેટવર્ક}: ગેટ પલ્સને વિલંબિત કરીને ફાયરિંગ એંગલનું નિયંત્રણ કરે છે
\item
  \textbf{કેપેસિટર ચાર્જિંગ}: C દરેક હાફ-સાયકલ દરમિયાન R મારફતે ચાર્જ થાય છે
\item
  \textbf{DIAC બ્રેકડાઉન}: જ્યારે કેપેસિટર વોલ્ટેજ DIAC બ્રેકઓવર વોલ્ટેજ સુધી પહોંચે
\item
  \textbf{TRIAC ટ્રિગરિંગ}: DIAC વહન કરે છે અને TRIAC ટ્રિગર કરે છે
\item
  \textbf{પાવર કંટ્રોલ}: R ને બદલવાથી ફાયરિંગ એંગલ અને પાવર ડિલિવરી બદલાય છે
\end{itemize}

\textbf{એપ્લિકેશન:}

\begin{itemize}
\tightlist
\item
  \textbf{લાઈટ ડિમર્સ}: લેમ્પની બ્રાઈટનેસ કંટ્રોલ
\item
  \textbf{ફેન સ્પીડ કંટ્રોલ}: પંખાની ગતિનું નિયંત્રણ
\item
  \textbf{હીટર કંટ્રોલ}: હીટિંગ એલિમેન્ટ્સ એડજસ્ટ કરવા
\end{itemize}

\end{solutionbox}
\begin{mnemonicbox}
``CRAFT'' - કેપેસિટર ચાર્જેસ, રીચેસ બ્રેકઓવર, એક્ટિવેટ્સ DIAC,
ફાયર્સ TRIAC, ટ્રાન્સફર્સ પાવર

\end{mnemonicbox}
\subsection*{પ્રશ્ન 4(ક) [7
ગુણ]}\label{uxaaauxab0uxab6uxaa8-4uxa95-7-uxa97uxaa3}

\textbf{ઇન્ડક્શન હીટિંગનો કાર્યકારી સિદ્ધાંત સમજાવો તદુપરાંત ઇન્ડક્શન હીટિંગના
ઉપયોગો લખો.}

\begin{solutionbox}

\textbf{ઇન્ડક્શન હીટિંગનો કાર્યકારી સિદ્ધાંત:}

\begin{center}
\textbf{Mermaid Diagram (Code)}
\begin{verbatim}
{Shaded}
{Highlighting}[]
graph LR
    Power[AC Power Supply] {-{-}{-} Inv[High Frequency Inverter]}
    Inv {-{-}{-} Coil[Induction Coil]}
    Coil {-{-}{-} Workpiece[Metal Workpiece]}

    subgraph "Physical Process"
    Coil {-.{-} Magnetic[Alternating Magnetic Field]}
    Magnetic {-.{-} Eddy[Eddy Currents]}
    Eddy {-.{-} Heat[Heat Generation]}
    end
{Highlighting}
{Shaded}
\end{verbatim}
\end{center}

\textbf{કાર્ય સિદ્ધાંત:}

\begin{itemize}
\tightlist
\item
  \textbf{હાઈ-ફ્રિક્વન્સી કરંટ}: ઇન્ડક્શન કોઈલમાંથી પસાર થાય છે
\item
  \textbf{ઇલેક્ટ્રોમેગ્નેટિક ઇન્ડક્શન}: ઓલ્ટરનેટિંગ મેગ્નેટિક ફિલ્ડ ઉત્પન્ન કરે છે
\item
  \textbf{એડી કરંટ}: વર્કપીસમાં પ્રેરિત થાય છે
\item
  \textbf{રેઝિસ્ટન્સ હીટિંગ}: એડી કરંટ રેઝિસ્ટન્સને કારણે ગરમી ઉત્પન્ન કરે છે
\item
  \textbf{સ્કિન ઇફેક્ટ}: સપાટીની નજીક ગરમી કેન્દ્રિત થાય છે
\item
  \textbf{નોન-કોન્ટેક્ટ હીટિંગ}: કોઈલ અને વર્કપીસ વચ્ચે કોઈ શારીરિક સંપર્ક નથી
\end{itemize}

\textbf{ઇન્ડક્શન હીટિંગના ઉપયોગો:}

\begin{itemize}
\tightlist
\item
  \textbf{મેટલ હીટ ટ્રીટમેન્ટ}: હાર્ડનિંગ, એનિલિંગ, ટેમ્પરિંગ
\item
  \textbf{મેટલ મેલ્ટિંગ}: ફાઉન્ડ્રી ઓપરેશન્સ
\item
  \textbf{વેલ્ડિંગ અને બ્રેઝિંગ}: મેટલ કોમ્પોનન્ટ્સની જોડાણ
\item
  \textbf{ફોર્જિંગ}: ફોર્મિંગ પહેલાં હીટિંગ
\item
  \textbf{ઘરેલું રસોઈ}: ઇન્ડક્શન કૂકટોપ
\item
  \textbf{સેમિકન્ડક્ટર પ્રોસેસિંગ}: ક્રિસ્ટલ ગ્રોથ
\end{itemize}

\end{solutionbox}
\begin{mnemonicbox}
``MASTER'' - મેગ્નેટિક ફિલ્ડ, ઓલ્ટરનેટિંગ કરંટ, સરફેસ હીટિંગ,
ટેમ્પરેચર કંટ્રોલ, એડી કરંટ્સ, રેઝિસ્ટન્સ હીટિંગ

\end{mnemonicbox}
\subsection*{પ્રશ્ન 4(અ) OR [3
ગુણ]}\label{uxaaauxab0uxab6uxaa8-4uxa85-or-3-uxa97uxaa3}

\textbf{એલડીઆરનો ઉપયોગ કરીને ફોટો રિલે સર્કિટનું કાર્ય સમજાવો.}

\begin{solutionbox}

\textbf{LDR વાળો ફોટો રિલે સર્કિટ:}

\begin{center}
\textbf{Mermaid Diagram (Code)}
\begin{verbatim}
{Shaded}
{Highlighting}[]
graph LR
    VS[Supply] {-{-}{-} R1[Resistor R1] {-}{-}{-} LDR[LDR]}
    LDR {-{-}{-} GND[Ground]}
    R1 {-{-}{-} B[Transistor Base]}
    VS {-{-}{-} RC[Collector Resistor] {-}{-}{-} C[Transistor Collector]}
    C {-{-}{-} Relay[Relay Coil] {-}{-}{-} GND}
    E[Transistor Emitter] {-{-}{-} GND}
{Highlighting}
{Shaded}
\end{verbatim}
\end{center}

\textbf{કાર્ય સિદ્ધાંત:}

\begin{itemize}
\tightlist
\item
  \textbf{લાઈટ-ડિપેન્ડન્ટ રેઝિસ્ટર}: પ્રકાશ વધતાં રેઝિસ્ટન્સ ઘટે છે
\item
  \textbf{વોલ્ટેજ ડિવાઈડર}: LDR અને R1 વોલ્ટેજ ડિવાઈડર બનાવે છે
\item
  \textbf{ટ્રાન્ઝિસ્ટર સ્વિચિંગ}: બેઝ વોલ્ટેજ ટ્રાન્ઝિસ્ટર કન્ડક્શનને નિયંત્રિત કરે છે
\item
  \textbf{રિલે ઓપરેશન}: ટ્રાન્ઝિસ્ટર રિલે કોઈલને ડ્રાઈવ કરે છે
\item
  \textbf{થ્રેશોલ્ડ એડજસ્ટમેન્ટ}: વેરિએબલ રેઝિસ્ટર વડે સેટ કરી શકાય છે
\end{itemize}

\textbf{એપ્લિકેશન:}

\begin{itemize}
\tightlist
\item
  \textbf{ઓટોમેટિક સ્ટ્રીટ લાઈટિંગ}: સાંજ પડતાં લાઈટ ચાલુ કરે છે
\item
  \textbf{ડે/નાઈટ સ્વિચિંગ}: એમ્બિયન્ટ લાઈટના આધારે ડિવાઈસ કંટ્રોલ
\item
  \textbf{સિક્યોરિટી સિસ્ટમ}: લાઈટ-એક્ટિવેટેડ અલાર્મ
\end{itemize}

\end{solutionbox}
\begin{mnemonicbox}
``LARK'' - લાઈટ કંટ્રોલ્સ, એક્ટિવેટ્સ ટ્રાન્ઝિસ્ટર, રિલે
સ્વિચેસ, કીપ્સ સર્કિટ ઓટોમેટેડ

\end{mnemonicbox}
\subsection*{પ્રશ્ન 4(બ) OR [4
ગુણ]}\label{uxaaauxab0uxab6uxaa8-4uxaac-or-4-uxa97uxaa3}

\textbf{555 ટાઈમર ICની મદદથી ટાઈમર સર્કિટનું કાર્ય સમજાવો.}

\begin{solutionbox}

\textbf{555 ટાઇમર સર્કિટ (મોનોસ્ટેબલ):}

\begin{center}
\textbf{Mermaid Diagram (Code)}
\begin{verbatim}
{Shaded}
{Highlighting}[]
graph TD
    VCC[+VCC] {-{-}{-} R[Resistor R] {-}{-}{-} D8[Pin 8 VCC]}
    D8 {-{-}{-} D4[Pin 4 Reset]}
    D8 {-{-}{-} D7[Pin 7 Discharge]}
    R {-{-}{-} D7}
    D7 {-{-}{-} C[Capacitor C] {-}{-}{-} GND[Ground]}
    Trigger[Trigger Input] {-{-}{-} D2[Pin 2 Trigger]}
    D3[Pin 3 Output] {-{-}{-} Output[Output]}
    D1[Pin 1 GND] {-{-}{-} GND}
    D5[Pin 5 Control] {-{-}{-} CC[Control Capacitor] {-}{-}{-} GND}
    D6[Pin 6 Threshold] {-{-}{-} D7}
{Highlighting}
{Shaded}
\end{verbatim}
\end{center}

\textbf{કાર્ય સિદ્ધાંત:}

\begin{itemize}
\tightlist
\item
  \textbf{ટ્રિગર ઇનપુટ}: પિન 2 પર એક્ટિવ લો ટ્રિગર
\item
  \textbf{ટાઇમિંગ કોમ્પોનન્ટ્સ}: R અને C ટાઇમિંગ પીરિયડ નક્કી કરે છે (T = 1.1RC)
\item
  \textbf{આઉટપુટ હાઈ}: ટ્રિગર થવા પર, આઉટપુટ હાઈ થાય છે
\item
  \textbf{કેપેસિટર ચાર્જિંગ}: C, R મારફતે ચાર્જ થાય છે
\item
  \textbf{થ્રેશોલ્ડ ડિટેક્શન}: જ્યારે વોલ્ટેજ 2/3 VCC સુધી પહોંચે, આઉટપુટ લો થાય છે
\item
  \textbf{ટાઇમર રિસેટ}: પિન 4 વડે સર્કિટ રિસેટ કરી શકાય છે
\end{itemize}

\textbf{એપ્લિકેશન:}

\begin{itemize}
\tightlist
\item
  \textbf{ડિલે સર્કિટ્સ}: ટાઈમ ડિલે બનાવવા
\item
  \textbf{પલ્સ જનરેશન}: ચોક્કસ પલ્સ જનરેટ કરવા
\item
  \textbf{ટાઇમિંગ કંટ્રોલ}: સિક્વેન્શિયલ ટાઇમિંગ ઓપરેશન્સ
\end{itemize}

\end{solutionbox}
\begin{mnemonicbox}
``TRACT'' - ટ્રિગર એક્ટિવેટ્સ, રેઝિસ્ટર-કેપેસિટર ટાઇમિંગ,
એક્યુરેટ ડિલે, કેપેસિટર ચાર્જેસ, થ્રેશોલ્ડ ડિટેક્શન

\end{mnemonicbox}
\subsection*{પ્રશ્ન 4(ક) OR [7
ગુણ]}\label{uxaaauxab0uxab6uxaa8-4uxa95-or-7-uxa97uxaa3}

\textbf{ડાઈઇલેક્ટ્રીક હીટિંગનો કાર્યકારી સિદ્ધાંત સમજાવો તદુપરાંત ડાઈઇલેક્ટ્રીક
હીટિંગના ઉપયોગો લખો.}

\begin{solutionbox}

\textbf{ડાઈઇલેક્ટ્રીક હીટિંગનો કાર્યકારી સિદ્ધાંત:}

\begin{center}
\textbf{Mermaid Diagram (Code)}
\begin{verbatim}
{Shaded}
{Highlighting}[]
graph LR
    RF[RF Generator] {-{-}{-} Electrodes[Electrodes]}

    subgraph "Material Between Electrodes"
    Electrodes {-{-}{-} Electric[Alternating Electric Field]}
    Electric {-{-}{-} Dipoles[Molecular Dipoles]}
    Dipoles {-{-}{-} Oscillation[Dipole Oscillation]}
    Oscillation {-{-}{-} Friction[Molecular Friction]}
    Friction {-{-}{-} Heat[Heat Generation]}
    end
{Highlighting}
{Shaded}
\end{verbatim}
\end{center}

\textbf{કાર્ય સિદ્ધાંત:}

\begin{itemize}
\tightlist
\item
  \textbf{ઉચ્ચ-ફ્રિક્વન્સી ઇલેક્ટ્રિક ફિલ્ડ}: ઇલેક્ટ્રોડ્સ વચ્ચે લાગુ કરવામાં આવે છે
\item
  \textbf{ડાઈઇલેક્ટ્રીક મટીરિયલ}: ઇલેક્ટ્રોડ્સ વચ્ચે મૂકવામાં આવે છે
\item
  \textbf{મોલેક્યુલર પોલરાઈઝેશન}: ડાયપોલ્સ ઇલેક્ટ્રિક ફિલ્ડ સાથે એલાઇન થાય છે
\item
  \textbf{ફિલ્ડ ઓસિલેશન}: ફિલ્ડની દિશાનું ઝડપી રિવર્સલ
\item
  \textbf{મોલેક્યુલર ફ્રિક્શન}: ડાયપોલ્સ ઝડપથી રોટેટ થઈને ફ્રિક્શન ઉત્પન્ન કરે છે
\item
  \textbf{વોલ્યુમેટ્રિક હીટિંગ}: સમગ્ર મટીરિયલમાં ગરમી ઉત્પન્ન થાય છે
\item
  \textbf{ફ્રિક્વન્સી રેન્જ}: સામાન્ય રીતે 10-100 MHz
\end{itemize}

\textbf{ડાઈઇલેક્ટ્રીક હીટિંગના ઉપયોગો:}

\begin{itemize}
\tightlist
\item
  \textbf{ફૂડ પ્રોસેસિંગ}: બેકિંગ, ડ્રાયિંગ, પાશ્ચરાઈઝેશન
\item
  \textbf{વુડ ઇન્ડસ્ટ્રી}: ગ્લુઈંગ, ટિમ્બર ડ્રાઈંગ
\item
  \textbf{ટેક્સટાઈલ ડ્રાઈંગ}: કાપડમાંથી ભેજ દૂર કરવો
\item
  \textbf{પ્લાસ્ટિક વેલ્ડિંગ}: થર્મોપ્લાસ્ટિક્સ જોડવા
\item
  \textbf{મેડિકલ એપ્લિકેશન}: થેરાપ્યુટિક ડાયથર્મી
\item
  \textbf{પેપર ઇન્ડસ્ટ્રી}: પેપર પ્રોડક્ટ્સ ડ્રાઈંગ
\end{itemize}

\end{solutionbox}
\begin{mnemonicbox}
``DIPOLE'' - ડાઈઇલેક્ટ્રિક મટિરિયલ, ઇન્ટેન્સ ઇલેક્ટ્રિક ફિલ્ડ,
પોલરાઈઝેશન ઓફ મોલેક્યુલ્સ, ઓસિલેશન કોઝેસ, લિંકેજ ઓફ હીટ, ઈવન હીટિંગ થ્રુઆઉટ

\end{mnemonicbox}
\subsection*{પ્રશ્ન 5(અ) [3
ગુણ]}\label{uxaaauxab0uxab6uxaa8-5uxa85-3-uxa97uxaa3}

\textbf{AC ડ્રાઈવને વ્યાખ્યાયિત કરો. AC ડ્રાઈવના ઉપયોગો જણાવો.}

\begin{solutionbox}

\textbf{AC ડ્રાઈવની વ્યાખ્યા:} AC ડ્રાઈવ એક ઇલેક્ટ્રોનિક ડિવાઈસ છે જે AC મોટરને
આપવામાં આવતા ફ્રિક્વન્સી અને વોલ્ટેજમાં ફેરફાર કરીને AC મોટરની સ્પીડ, ટોર્ક અને દિશાનું
નિયંત્રણ કરે છે.

\textbf{AC ડ્રાઈવના ઉપયોગો:}

{\def\LTcaptype{none} % do not increment counter
\begin{longtable}[]{@{}ll@{}}
\toprule\noalign{}
એપ્લિકેશન એરિયા & ઉદાહરણો \\
\midrule\noalign{}
\endhead
\bottomrule\noalign{}
\endlastfoot
\textbf{ઔદ્યોગિક} & કન્વેયર સિસ્ટમ્સ, પમ્પ્સ, ફેન્સ, કોમ્પ્રેસર્સ \\
\textbf{HVAC} & બ્લોઅર્સ, કૂલિંગ ટાવર્સ, એર હેન્ડલિંગ યુનિટ્સ \\
\textbf{વોટર ટ્રીટમેન્ટ} & પમ્પ્સ, મિક્સર્સ, એરેટર્સ \\
\textbf{માઈનિંગ} & ક્રશર્સ, કન્વેયર્સ, પમ્પ્સ \\
\textbf{ટેક્સટાઈલ} & સ્પિનિંગ મશીન્સ, લૂમ્સ, વાઈન્ડર્સ \\
\textbf{મટિરિયલ હેન્ડલિંગ} & ક્રેન્સ, એલિવેટર્સ, એસ્કેલેટર્સ \\
\end{longtable}
}

\end{solutionbox}
\begin{mnemonicbox}
``PITCHW'' - પમ્પ્સ, ઇન્ડસ્ટ્રિયલ મશીનરી, ટેક્સટાઈલ મશીન્સ,
કન્વેયર સિસ્ટમ્સ, HVAC સિસ્ટમ્સ, વોટર ટ્રીટમેન્ટ

\end{mnemonicbox}
\subsection*{પ્રશ્ન 5(બ) [4
ગુણ]}\label{uxaaauxab0uxab6uxaa8-5uxaac-4-uxa97uxaa3}

\textbf{ડીસી સંટ મોટરની ગતિને નિયંત્રિત કરવા માટેની કોઈ એક પદ્ધતિ ની સર્કિટ
દોરો અને સમજાવો.}

\begin{solutionbox}

\textbf{DC શંટ મોટર માટે આર્મેચર વોલ્ટેજ કંટ્રોલ મેથડ:}

\begin{center}
\textbf{Mermaid Diagram (Code)}
\begin{verbatim}
{Shaded}
{Highlighting}[]
graph LR
    AC[AC Supply] {-{-}{-} B[Bridge Rectifier]}
    B {-{-}{-} SCR[SCR] {-}{-}{-} A[Armature]}
    A {-{-}{-} B}
    AC {-{-}{-} F[Field Circuit]}
    F {-{-}{-} Field[Field Winding]}
    GC[Gate Control] {-{-}{-} SCR}
{Highlighting}
{Shaded}
\end{verbatim}
\end{center}

\textbf{કાર્ય સિદ્ધાંત:}

\begin{itemize}
\tightlist
\item
  \textbf{કોન્સ્ટન્ટ ફિલ્ડ કરંટ}: ફિલ્ડ સપ્લાય સ્થિર રાખવામાં આવે છે
\item
  \textbf{વેરિએબલ આર્મેચર વોલ્ટેજ}: SCR દ્વારા નિયંત્રિત
\item
  \textbf{સ્પીડ ઈક્વેશન}: N ∝ (V_{a} - I_{a}R_{a})/Φ
\item
  \textbf{સ્પીડ કંટ્રોલ}: આર્મેચર વોલ્ટેજ V_{a} બદલીને
\item
  \textbf{ટોર્ક કંટ્રોલ}: આર્મેચર કરંટ ટોર્ક નિયંત્રિત કરે છે
\end{itemize}

\textbf{ફાયદાઓ:}

\begin{itemize}
\tightlist
\item
  \textbf{વાઈડ સ્પીડ રેન્જ}: બેઝ સ્પીડની નીચે અને ઉપર સ્પીડ મેળવી શકાય છે
\item
  \textbf{સ્મૂધ કંટ્રોલ}: સતત સ્પીડ એડજસ્ટમેન્ટ
\item
  \textbf{હાઈ એફિશિયન્સી}: કંટ્રોલ સર્કિટમાં ઓછો પાવર લોસ
\end{itemize}

\end{solutionbox}
\begin{mnemonicbox}
``SAVE'' - SCR કંટ્રોલ્સ, આર્મેચર વોલ્ટેજ વેરીસ, વેલોસિટી
ચેન્જેસ, એફિશિયન્ટ ઓપરેશન

\end{mnemonicbox}
\subsection*{પ્રશ્ન 5(ક) [7
ગુણ]}\label{uxaaauxab0uxab6uxaa8-5uxa95-7-uxa97uxaa3}

\textbf{PLCનો બ્લોક ડાયગ્રામ દોરો અને દરેક બ્લોકનું કાર્ય સમજાવો.}

\begin{solutionbox}

\textbf{PLC બ્લોક ડાયગ્રામ:}

\begin{center}
\textbf{Mermaid Diagram (Code)}
\begin{verbatim}
{Shaded}
{Highlighting}[]
graph TD
    PS[Power Supply] {-{-}{-} CPU[Central Processing Unit]}
    CPU {-{-}{-} MEM[Memory]}
    CPU {-{-}{-} INP[Input Module]}
    CPU {-{-}{-} OUT[Output Module]}
    CPU {-{-}{-} COM[Communication Module]}
    INP {-{-}{-} Input[Input Devices]}
    OUT {-{-}{-} Output[Output Devices]}
    COM {-{-}{-} Network[Network/HMI]}
    PROG[Programming Device] {-{-}{-} COM}
{Highlighting}
{Shaded}
\end{verbatim}
\end{center}

\textbf{દરેક બ્લોકનું કાર્ય:}

{\def\LTcaptype{none} % do not increment counter
\begin{longtable}[]{@{}
  >{\raggedright\arraybackslash}p{(\linewidth - 2\tabcolsep) * \real{0.4118}}
  >{\raggedright\arraybackslash}p{(\linewidth - 2\tabcolsep) * \real{0.5882}}@{}}
\toprule\noalign{}
\begin{minipage}[b]{\linewidth}\raggedright
બ્લોક
\end{minipage} & \begin{minipage}[b]{\linewidth}\raggedright
કાર્ય
\end{minipage} \\
\midrule\noalign{}
\endhead
\bottomrule\noalign{}
\endlastfoot
\textbf{પાવર સપ્લાય} & મેઈન AC સપ્લાયને ઇન્ટરનલ સર્કિટ માટે જરૂરી DC માં રૂપાંતરિત
કરે છે \\
\textbf{CPU} & પ્રોગ્રામ એક્ઝીક્યુટ કરે છે, I/O પ્રોસેસ કરે છે, કેલ્ક્યુલેશન કરે છે \\
\textbf{મેમરી} & પ્રોગ્રામ, ડેટા અને I/O સ્ટેટસ સ્ટોર કરે છે (RAM, ROM, EEPROM) \\
\textbf{ઇનપુટ મોડ્યુલ} & ઇનપુટ ડિવાઈસ સાથે ઇન્ટરફેસ કરે છે, આઇસોલેશન, સિગ્નલ
કન્ડિશનિંગ આપે છે \\
\textbf{આઉટપુટ મોડ્યુલ} & આઉટપુટ ડિવાઈસને ડ્રાઈવ કરે છે, આઇસોલેશન અને પ્રોટેક્શન આપે
છે \\
\textbf{કોમ્યુનિકેશન મોડ્યુલ} & PLC ને નેટવર્ક, અન્ય PLC અને પ્રોગ્રામિંગ ડિવાઈસ સાથે
જોડે છે \\
\textbf{પ્રોગ્રામિંગ ડિવાઈસ} & PLC પ્રોગ્રામ ડેવલપ, એડિટ અને મોનિટર કરવા માટે
વપરાય છે \\
\end{longtable}
}

\textbf{PLCના ફાયદાઓ:}

\begin{itemize}
\tightlist
\item
  \textbf{રિલાયબિલિટી}: સોલિડ-સ્ટેટ કોમ્પોનન્ટ્સ ઉચ્ચ MTBF સાથે
\item
  \textbf{ફ્લેક્સિબિલિટી}: વિવિધ એપ્લિકેશન્સ માટે સરળતાથી રીપ્રોગ્રામ થઈ શકે છે
\item
  \textbf{કોમ્યુનિકેશન}: ડિસ્ટ્રિબ્યુટેડ કંટ્રોલ માટે નેટવર્ક ક્ષમતાઓ
\item
  \textbf{ડાયગ્નોસ્ટિક્સ}: બિલ્ટ-ઇન ડાયગ્નોસ્ટિક્સ અને ટ્રબલશૂટિંગ
\end{itemize}

\end{solutionbox}
\begin{mnemonicbox}
``PRIME-C'' - પાવર સપ્લાય, RAM/ROM મેમરી, ઇનપુટ મોડ્યુલ,
માઇક્રોપ્રોસેસર (CPU), એક્ઝિક્યુશન ઓફ પ્રોગ્રામ, કોમ્યુનિકેશન ઇન્ટરફેસ

\end{mnemonicbox}
\subsection*{પ્રશ્ન 5(અ) OR [3
ગુણ]}\label{uxaaauxab0uxab6uxaa8-5uxa85-or-3-uxa97uxaa3}

\textbf{સ્ટેપર મોટરના ઉપયોગો જણાવો.}

\begin{solutionbox}

{\def\LTcaptype{none} % do not increment counter
\begin{longtable}[]{@{}ll@{}}
\toprule\noalign{}
એપ્લિકેશન એરિયા & ઉદાહરણો \\
\midrule\noalign{}
\endhead
\bottomrule\noalign{}
\endlastfoot
\textbf{પ્રિસિઝન પોઝિશનિંગ} & CNC મશીન્સ, 3D પ્રિન્ટર્સ, રોબોટિક આર્મ્સ \\
\textbf{ઓફિસ ઇક્વિપમેન્ટ} & પ્રિન્ટર્સ, સ્કેનર્સ, ફોટોકોપિયર્સ \\
\textbf{મેડિકલ ડિવાઈસ} & સર્જિકલ રોબોટ્સ, ફ્લુઈડ પમ્પ્સ, સેમ્પલ હેન્ડલર્સ \\
\textbf{ઓટોમોટિવ} & હેડલાઈટ એડજસ્ટમેન્ટ, આઈડલ કંટ્રોલ, મિરર કંટ્રોલ \\
\textbf{એરોસ્પેસ} & સેટેલાઈટ પોઝિશનિંગ, એન્ટેના કંટ્રોલ \\
\textbf{કન્ઝ્યુમર ઇલેક્ટ્રોનિક્સ} & કેમેરા (ફોકસ/ઝૂમ), ગેમિંગ કંટ્રોલર્સ \\
\end{longtable}
}

\end{solutionbox}
\begin{mnemonicbox}
``POMAC'' - પોઝિશનિંગ સિસ્ટમ્સ, ઓફિસ ઇક્વિપમેન્ટ, મેડિકલ
ડિવાઈસ, ઓટોમોટિવ કંટ્રોલ્સ, કન્ઝ્યુમર ઇલેક્ટ્રોનિક્સ

\end{mnemonicbox}
\subsection*{પ્રશ્ન 5(બ) OR [4
ગુણ]}\label{uxaaauxab0uxab6uxaa8-5uxaac-or-4-uxa97uxaa3}

\textbf{ડીસી સીરીઝ મોટરની ગતિને નિયંત્રિત કરવા માટે સર્કિટ દોરો અને સમજાવો.}

\begin{solutionbox}

\textbf{SCR વડે DC સીરીઝ મોટર સ્પીડ કંટ્રોલ:}

\begin{center}
\textbf{Mermaid Diagram (Code)}
\begin{verbatim}
{Shaded}
{Highlighting}[]
graph LR
    AC[AC Supply] {-{-}{-} B[Bridge Rectifier]}
    B {-{-}{-} SCR[SCR] {-}{-}{-} A[Armature]}
    A {-{-}{-} SF[Series Field]}
    SF {-{-}{-} B}
    GC[Gate Control] {-{-}{-} SCR}
{Highlighting}
{Shaded}
\end{verbatim}
\end{center}

\textbf{કાર્ય સિદ્ધાંત:}

\begin{itemize}
\tightlist
\item
  \textbf{સીરીઝ કનેક્શન}: ફિલ્ડ વાઈન્ડિંગ આર્મેચર સાથે સીરીઝમાં
\item
  \textbf{SCR કંટ્રોલ}: ફેઝ-કંટ્રોલ્ડ SCR એવરેજ વોલ્ટેજ રેગ્યુલેટ કરે છે
\item
  \textbf{સ્પીડ ઈક્વેશન}: N ∝ (V - I(Ra+Rf))/IΦ
\item
  \textbf{સ્પીડ-ટોર્ક રિલેશન}: નોન-લિનિયર રિલેશનશિપ
\item
  \textbf{એપ્લિકેશન}: જ્યાં હાઈ સ્ટાર્ટિંગ ટોર્ક જરૂરી હોય ત્યાં વપરાય છે
\end{itemize}

\textbf{ફાયદાઓ:}

\begin{itemize}
\tightlist
\item
  \textbf{હાઈ સ્ટાર્ટિંગ ટોર્ક}: ટ્રેક્શન એપ્લિકેશન્સ માટે આદર્શ
\item
  \textbf{સિમ્પલ કંટ્રોલ}: બેઝિક સર્કિટ ડિઝાઇન
\item
  \textbf{કોસ્ટ-ઇફેક્ટિવ}: અન્ય પદ્ધતિઓ કરતાં ઓછા કોમ્પોનન્ટ્સ
\end{itemize}

\end{solutionbox}
\begin{mnemonicbox}
``SCAT'' - સીરીઝ કનેક્શન, કરંટ કંટ્રોલ્સ ફ્લક્સ, એવરેજ વોલ્ટેજ
કંટ્રોલ્ડ બાય SCR, ટોર્ક હાઈએસ્ટ એટ લો સ્પીડ્સ

\end{mnemonicbox}
\subsection*{પ્રશ્ન 5(ક) OR [7
ગુણ]}\label{uxaaauxab0uxab6uxaa8-5uxa95-or-7-uxa97uxaa3}

\textbf{BLDC મોટરની વિસ્તૃતમાં ચર્ચા કરો.}

\begin{solutionbox}

\textbf{BLDC મોટર (બ્રશલેસ DC મોટર):}

\begin{center}
\textbf{Mermaid Diagram (Code)}
\begin{verbatim}
{Shaded}
{Highlighting}[]
graph LR
    subgraph "BLDC મોટર કન્સ્ટ્રક્શન"
    Stator[સ્ટેટર વિથ વાઈન્ડિંગ્સ]
    Rotor[રોટર વિથ પર્મેનન્ટ મેગ્નેટ્સ]
    Hall[હોલ સેન્સર્સ]
    end

    subgraph "કંટ્રોલ સિસ્ટમ"
    Controller[ઇલેક્ટ્રોનિક કંટ્રોલર]
    Driver[પાવર ડ્રાઈવર]
    Feedback[પોઝિશન ફીડબેક]
    end
    
    Controller {-{-}{-} Driver}
    Driver {-{-}{-} Stator}
    Hall {-{-}{-} Feedback}
    Feedback {-{-}{-} Controller}
{Highlighting}
{Shaded}
\end{verbatim}
\end{center}

\textbf{રચના:}

\begin{itemize}
\tightlist
\item
  \textbf{સ્ટેટર}: વાઈન્ડિંગ્સ ધરાવે છે (સામાન્ય રીતે 3-ફેઝ)
\item
  \textbf{રોટર}: રોટર પર પર્મેનન્ટ મેગ્નેટ્સ
\item
  \textbf{પોઝિશન સેન્સિંગ}: હોલ ઇફેક્ટ સેન્સર્સ અથવા એન્કોડર્સ
\item
  \textbf{કંટ્રોલર}: ઇલેક્ટ્રોનિક કોમ્યુટેશન કંટ્રોલર
\end{itemize}

\textbf{કાર્ય સિદ્ધાંત:}

\begin{itemize}
\tightlist
\item
  \textbf{ઇલેક્ટ્રોનિક કોમ્યુટેશન}: મિકેનિકલ બ્રશની જગ્યાએ
\item
  \textbf{સિક્વન્સિંગ}: કંટ્રોલર સ્ટેટર કોઈલ્સને સિક્વન્સમાં એનર્જાઈઝ કરે છે
\item
  \textbf{પોઝિશન ફીડબેક}: હોલ સેન્સર્સ રોટર પોઝિશન નક્કી કરે છે
\item
  \textbf{ફેઝ એનર્જાઈઝિંગ}: રોટર પોઝિશનના આધારે યોગ્ય ફેઝ એનર્જાઈઝ થાય છે
\end{itemize}

\textbf{ફાયદાઓ:}

\begin{itemize}
\tightlist
\item
  \textbf{હાઈ એફિશિયન્સી}: કોઈ બ્રશ ફ્રિક્શન લોસ નહીં
\item
  \textbf{લો મેઈન્ટેનન્સ}: કોઈ બ્રશ વેર નહીં
\item
  \textbf{લાંબુ આયુષ્ય}: વિશ્વસનીય ઓપરેશન
\item
  \textbf{બેટર સ્પીડ-ટોર્ક કેરેક્ટરિસ્ટિક્સ}: ફ્લેટ કર્વ
\item
  \textbf{લો નોઈઝ}: શાંત ઓપરેશન
\item
  \textbf{બેટર હીટ ડિસિપેશન}: સ્ટેટર પર વાઈન્ડિંગ્સ
\end{itemize}

\textbf{એપ્લિકેશન:}

\begin{itemize}
\tightlist
\item
  \textbf{કોમ્પ્યુટર કૂલિંગ ફેન્સ}: CPU/GPU કૂલર્સ
\item
  \textbf{હાર્ડ ડિસ્ક ડ્રાઈવ્સ}: સ્પિન્ડલ મોટર્સ
\item
  \textbf{ઇલેક્ટ્રિક વ્હીકલ્સ}: પ્રોપલ્શન સિસ્ટમ્સ
\item
  \textbf{ડ્રોન્સ}: પ્રોપેલર મોટર્સ
\item
  \textbf{હોમ એપ્લાયન્સેસ}: વોશિંગ મશીન્સ, રેફ્રિજરેટર્સ
\item
  \textbf{ઔદ્યોગિક ઓટોમેશન}: પ્રિસિઝન કંટ્રોલ સિસ્ટમ્સ
\end{itemize}

\end{solutionbox}
\begin{mnemonicbox}
``COPPER'' - કોમ્યુટેશન ઇલેક્ટ્રોનિક, ઓપરેશન એફિશિયન્ટ,
પર્મેનન્ટ મેગ્નેટ્સ, પોઝિશન સેન્સર્સ, ઇલેક્ટ્રોનિક કંટ્રોલ, રિલાયબલ પરફોર્મન્સ

\end{mnemonicbox}

\end{document}
