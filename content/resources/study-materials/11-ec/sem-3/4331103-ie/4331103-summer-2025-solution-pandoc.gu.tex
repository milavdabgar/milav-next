\documentclass[10pt,a4paper]{article}

% content/resources/templates/preamble.tex
\usepackage[margin=0.6in]{geometry}
\author{Milav Dabgar}
\usepackage{amsmath,amssymb,amsthm}
\usepackage{booktabs}
\usepackage{multirow}
\usepackage{xcolor}
\usepackage{tcolorbox}
\tcbuselibrary{breakable,skins}
\usepackage[colorlinks=true,linkcolor=blue]{hyperref}
\usepackage{titlesec}
\usepackage{enumitem}
\usepackage{tikz}
\usepackage{pgfplots}
\usepackage{circuitikz}
\usepackage[version=4]{mhchem}
\usepackage{longtable}
\usepackage{array}
\usepackage{float}
\usepackage{caption}
\usepackage{listings}

\lstset{
  basicstyle=\small\ttfamily,
  breaklines=true,
  breakatwhitespace=false,
  postbreak=\mbox{\textcolor{red}{$\hookrightarrow$}\space},
  float=false,
  numbers=left,
  numberstyle=\tiny\color{gray},
  numbersep=10pt,
  xleftmargin=2em,
  keywordstyle=\color{blue},
  commentstyle=\color{green!60!black},
  stringstyle=\color{purple},
  backgroundcolor=\color{gray!5},
  showstringspaces=false,
  tabsize=2,
  captionpos=b,
  keepspaces=true,
  columns=flexible
}

\pgfplotsset{compat=1.18}
\usetikzlibrary{shapes,arrows,positioning,calc,patterns,decorations.pathmorphing,decorations.markings,arrows.meta}

% Color scheme
\definecolor{headcolor}{RGB}{0,102,204}
\definecolor{keycolor}{RGB}{220,20,60}
\definecolor{solutioncolor}{RGB}{34,139,34}
\definecolor{mnemoniccolor}{RGB}{148,0,211}
\definecolor{codecolor}{RGB}{0,0,100}

% Spacing
\setlength{\parskip}{3pt}
\setlist[itemize]{nosep}
\setlist[enumerate]{nosep}

% Title formatting
\titleformat{\section}{\Large\bfseries\color{headcolor}}{\thesection}{1em}{}
\titleformat{\subsection}{\large\bfseries\color{headcolor}}{\thesubsection}{1em}{}

% Pandoc tightlist compatibility
\providecommand{\tightlist}{%
  \setlength{\itemsep}{0pt}\setlength{\parskip}{0pt}}

% Pandoc longtable compatibility
\newcounter{none}
\def\thenone{}


% content/resources/templates/gujarati-boxes.tex
\usepackage{fontspec}
\usepackage{polyglossia}

% Set Gujarati as main language (document is primarily in Gujarati)
% Note: gloss-gujarati.ldf doesn't exist in polyglossia, but it will use hyphenation patterns
\setdefaultlanguage{gujarati}
\setotherlanguage{english}

% Configure Gujarati font properly
% Use Language=Default to prevent polyglossia from trying to add language-specific features
% that don't exist for Gujarati, which causes "empty feature" warnings
\newfontfamily\gujaratifont[Script=Gujarati,AutoFakeBold=2.5,AutoFakeSlant=0.3]{Noto Sans Gujarati}
\setmainfont[Script=Gujarati,AutoFakeBold=2.5,AutoFakeSlant=0.3]{Noto Sans Gujarati}
% Use Noto Sans Gujarati for monospace to support Gujarati in text
\setmonofont[Scale=0.9]{Noto Sans Gujarati}

% Configure English to use the same font
\newfontfamily\englishfont[Script=Gujarati,AutoFakeBold=2.5,AutoFakeSlant=0.3]{Noto Sans Gujarati}

% Translations for polyglossia
\gappto\captionsgujarati{
  \renewcommand{\tablename}{કોષ્ટક}
  \renewcommand{\figurename}{આકૃતિ}
}

% Helper for TikZ nodes to ensure Gujarati font
\newcommand{\gu}[1]{{\gujaratifont #1}}

% Custom environments
\newtcolorbox{solutionbox}{
    breakable,
    enhanced,
    colback=solutioncolor!5!white,
    colframe=solutioncolor!75!black,
    fonttitle=\bfseries,
    title=જવાબ
}

\newtcolorbox{solutionboxnobreak}{
 colback=solutioncolor!5!white,
 colframe=solutioncolor!75!black,
 fonttitle=\bfseries,
 title=જવાબ
}

\newtcolorbox{keyformula}{
 breakable,
 enhanced,
 colback=keycolor!5!white,
 colframe=keycolor!75!black,
 fonttitle=\bfseries,
 title=રાસાયણિક સમીકરણ/સૂત્ર
}

\newtcolorbox{mnemonicbox}{
 breakable,
 enhanced,
 colback=mnemoniccolor!5!white,
 colframe=mnemoniccolor!75!black,
 fonttitle=\bfseries,
 title=મેમરી ટ્રીક
}


\begin{document}

\begin{center}
{\Huge\bfseries\color{headcolor} Subject Name (Gujarati)}\\[5pt]
{\LARGE 4331103 -- Summer 2025}\\[3pt]
{\large Semester 1 Study Material}\\[3pt]
{\normalsize\textit{Detailed Solutions and Explanations}}
\end{center}

\vspace{10pt}

\subsection*{પ્રશ્ન 1(a) [3
ગુણ]}\label{q1a}

\textbf{Opto-Isolators, Opto-TRIAC અને Opto-ટ્રાન્ઝિસ્ટરની લાક્ષણિકતાઓ દોરો.}

\begin{solutionbox}

\textbf{ઓપ્ટો-ઇલેક્ટ્રોનિક ઉપકરણોની લાક્ષણિકતાઓ:}

{\def\LTcaptype{none} % do not increment counter
\begin{longtable}[]{@{}
  >{\centering\arraybackslash}p{(\linewidth - 4\tabcolsep) * \real{0.3409}}
  >{\centering\arraybackslash}p{(\linewidth - 4\tabcolsep) * \real{0.2727}}
  >{\centering\arraybackslash}p{(\linewidth - 4\tabcolsep) * \real{0.3864}}@{}}
\toprule\noalign{}
\begin{minipage}[b]{\linewidth}\centering
Opto-Isolator
\end{minipage} & \begin{minipage}[b]{\linewidth}\centering
Opto-TRIAC
\end{minipage} & \begin{minipage}[b]{\linewidth}\centering
Opto-Transistor
\end{minipage} \\
\midrule\noalign{}
\endhead
\bottomrule\noalign{}
\endlastfoot
\pandocbounded{\includegraphics[keepaspectratio,alt={Opto-Isolator Characteristic}]{https://i.ibb.co/3p0M3Cj/opto-isolator.png}}
&
\pandocbounded{\includegraphics[keepaspectratio,alt={Opto-TRIAC Characteristic}]{https://i.ibb.co/bJ8wPdz/opto-triac.png}}
&
\pandocbounded{\includegraphics[keepaspectratio,alt={Opto-Transistor Characteristic}]{https://i.ibb.co/h2GYVsP/opto-transistor.png}} \\
LED કરંટ અને ફોટોડિટેક્ટર કરંટ વચ્ચે લીનિયર સંબંધ & થ્રેશોલ્ડ સાથે નોન-લીનિયર ટ્રિગરિંગ
રિસ્પોન્સ & લીનિયર કરંટ ટ્રાન્સફર લાક્ષણિકતા \\
CTR (કરંટ ટ્રાન્સફર રેશિયો) મુખ્ય પેરામીટર છે & ચોક્કસ કરંટ થ્રેશોલ્ડ પર ટ્રિગરિંગ થાય
છે & કલેક્ટર કરંટ બેઝ ઇલ્યુમિનેશન પર આધાર રાખે છે \\
\end{longtable}
}

\begin{itemize}
\tightlist
\item
  \textbf{CTR (કરંટ ટ્રાન્સફર રેશિયો)}: આઉટપુટ કરંટનો ઇનપુટ કરંટ સાથેનો ગુણોત્તર
\item
  \textbf{ટ્રિગર કરંટ}: ડિવાઈસને એક્ટિવેટ કરવા માટે જરૂરી ન્યૂનતમ કરંટ
\item
  \textbf{લિનિયારિટી}: આઉટપુટ ઇનપુટ લાઇટના પ્રમાણમાં કેટલું છે
\end{itemize}

\end{solutionbox}
\begin{mnemonicbox}
``LTL - લાઇટ ટ્રાન્સફર્સ લાઇક કરંટ ફ્લોઝ -- લીનિયર ફોર
આઇસોલેટર્સ/ટ્રાન્ઝિસ્ટર્સ, ટ્રિગર્ડ ફોર TRIACs''

\end{mnemonicbox}
\subsection*{પ્રશ્ન 1(b) [4
ગુણ]}\label{q1b}

\textbf{IGBT ની કાર્યકારી અને બાંધકામ સુવિધાઓનું વર્ણન કરો.}

\begin{solutionbox}

\textbf{IGBT સ્ટ્રક્ચર અને ઓપરેશન:}

\begin{center}
\textbf{Mermaid Diagram (Code)}
\begin{verbatim}
{Shaded}
{Highlighting}[]
graph LR
    A[Gate] {-{-}{} B[Metal Oxide]}
    B {-{-}{} C[P+ Body]}
    C {-{-}{} D[N{-} Drift Region]}
    D {-{-}{} E[P+ Collector/Substrate]}
    F[Emitter] {-{-}{} C}
    E {-{-}{} G[Collector]}
    style A fill:\#91a6ff
    style B fill:\#ffeead
    style C fill:\#ff9e9e
    style D fill:\#d9ffb3
    style E fill:\#ff9e9e
    style F fill:\#91a6ff
    style G fill:\#91a6ff
{Highlighting}
{Shaded}
\end{verbatim}
\end{center}

{\def\LTcaptype{none} % do not increment counter
\begin{longtable}[]{@{}ll@{}}
\toprule\noalign{}
ફીચર & વર્ણન \\
\midrule\noalign{}
\endhead
\bottomrule\noalign{}
\endlastfoot
સ્ટ્રક્ચર & MOSFET ઇનપુટને BJT આઉટપુટ સાથે જોડે છે \\
લેયર્સ & ગેટ/મેટલ ઓક્સાઇડ/P+ બોડી/N- ડ્રિફ્ટ/P+ કલેક્ટર \\
ફાયદાઓ & ઉચ્ચ ઇનપુટ ઇમ્પિડન્સ, ઓછું કન્ડક્શન લોસ \\
સ્વિચિંગ & BJT કરતાં ઝડપી, MOSFET કરતાં વધુ સારી પાવર હેન્ડલિંગ \\
\end{longtable}
}

\begin{itemize}
\tightlist
\item
  \textbf{વોલ્ટેજ કંટ્રોલ્ડ}: MOSFET જેવી ગેટ વોલ્ટેજ દ્વારા નિયંત્રિત ડિવાઇસ
\item
  \textbf{કન્ડક્ટિવિટી મોડ્યુલેશન}: P+ કલેક્ટર ડ્રિફ્ટ રિજિયનમાં હોલ્સ ઇન્જેક્ટ કરે છે
\item
  \textbf{લો ઓન-સ્ટેટ વોલ્ટેજ}: MOSFET કરતાં ઓછું કન્ડક્શન લોસ
\end{itemize}

\end{solutionbox}
\begin{mnemonicbox}
``IGBT MBC'' - ``ઇનપુટ ફ્રોમ MOS, બોડી હેન્ડલ્સ કરંટ,
કલેક્ટર એક્ટ્સ લાઇક BJT''

\end{mnemonicbox}
\subsection*{પ્રશ્ન 1(c) [7
ગુણ]}\label{q1c}

\textbf{બે-ટ્રાન્ઝિસ્ટર એનાલોજીનો ઉપયોગ કરીને SCR નું કાર્ય સમજાવો.}

\begin{solutionbox}

\textbf{SCR એઝ ટુ-ટ્રાન્ઝિસ્ટર મોડેલ:}

\begin{center}
\textbf{Mermaid Diagram (Code)}
\begin{verbatim}
{Shaded}
{Highlighting}[]
graph LR
    A[Anode] {-{-}{} B[P1]}
    B {-{-}{} C[N1]}
    C {-{-}{} D[P2]}
    D {-{-}{} E[N2]}
    E {-{-}{} F[Cathode]}
    G[Gate] {-{-}{} D}

    subgraph PNP Transistor
    B
    C
    D
    end
    
    subgraph NPN Transistor
    C
    D
    E
    end
    
    style A fill:\#91a6ff
    style B fill:\#ff9e9e
    style C fill:\#d9ffb3
    style D fill:\#ff9e9e
    style E fill:\#d9ffb3
    style F fill:\#91a6ff
    style G fill:\#91a6ff
{Highlighting}
{Shaded}
\end{verbatim}
\end{center}

\textbf{બે-ટ્રાન્ઝિસ્ટર સમજૂતી:}

{\def\LTcaptype{none} % do not increment counter
\begin{longtable}[]{@{}
  >{\raggedright\arraybackslash}p{(\linewidth - 4\tabcolsep) * \real{0.3235}}
  >{\raggedright\arraybackslash}p{(\linewidth - 4\tabcolsep) * \real{0.2941}}
  >{\raggedright\arraybackslash}p{(\linewidth - 4\tabcolsep) * \real{0.3824}}@{}}
\toprule\noalign{}
\begin{minipage}[b]{\linewidth}\raggedright
કોમ્પોનન્ટ
\end{minipage} & \begin{minipage}[b]{\linewidth}\raggedright
ફંક્શન
\end{minipage} & \begin{minipage}[b]{\linewidth}\raggedright
કનેક્શન્સ
\end{minipage} \\
\midrule\noalign{}
\endhead
\bottomrule\noalign{}
\endlastfoot
PNP (T1) & ઉપરનો ટ્રાન્ઝિસ્ટર & એમિટર એનોડથી, કલેક્ટર N1 થી, બેઝ P2-N1
જંક્શનથી \\
NPN (T2) & નીચેનો ટ્રાન્ઝિસ્ટર & એમિટર કેથોડથી, કલેક્ટર P1-N1 જંક્શનથી, બેઝ
ગેટથી \\
ફીડબેક & રિજનરેટિવ એક્શન & T1નો કલેક્ટર કરંટ = T2નો બેઝ કરંટ અને વાઇસ વર્સા \\
\end{longtable}
}

\begin{itemize}
\tightlist
\item
  \textbf{લેચિંગ મેકેનિઝમ}: એકવાર ટ્રિગર થયા પછી, ટ્રાન્ઝિસ્ટર એકબીજાને ON રાખે છે
\item
  \textbf{ટ્રિગરિંગ}: નાનો ગેટ કરંટ \rightarrow T2 ચાલુ થાય \rightarrow T1ને બેઝ કરંટ મળે \rightarrow બંને ચાલુ રહે
\item
  \textbf{હોલ્ડિંગ કરંટ}: રિજનરેટિવ એક્શન જાળવી રાખવા માટે જરૂરી ન્યૂનતમ કરંટ
\item
  \textbf{ટર્ન-ઓફ}: એનોડ કરંટ હોલ્ડિંગ કરંટથી નીચે જવો જોઈએ
\end{itemize}

\end{solutionbox}
\begin{mnemonicbox}
``PPFF'' - ``પોઝિટિવ ફીડબેક પર્પેચ્યુએટ્સ ફોરવર્ડ કન્ડક્શન''

\end{mnemonicbox}
\subsection*{પ્રશ્ન 1(c) OR [7
ગુણ]}\label{q1c}

\textbf{ઓપ્ટો-એસસીઆરનો ઉપયોગ કરીને સોલિડ સ્ટેટ રિલેનું કાર્ય સમજાવો.}

\begin{solutionbox}

\textbf{ઓપ્ટો-SCR સાથે સોલિડ સ્ટેટ રિલે:}

\begin{center}
\textbf{Mermaid Diagram (Code)}
\begin{verbatim}
{Shaded}
{Highlighting}[]
graph LR
    A[AC/DC Input] {-{-}{} B[LED]}
    B {-{-}{} C[Photo{-}SCR/Detector]}
    C {-{-}{} D[Main SCR/TRIAC]}
    D {-{-}{} E[Output Load]}
    F[Zero Crossing Circuit] {-{-}{} D}
    style A fill:\#b3e0ff
    style B fill:\#ffcccc
    style C fill:\#ffee99
    style D fill:\#ccffcc
    style E fill:\#dddddd
    style F fill:\#e6ccff
{Highlighting}
{Shaded}
\end{verbatim}
\end{center}

\textbf{કાર્ય સિદ્ધાંત અને ઘટકો:}

{\def\LTcaptype{none} % do not increment counter
\begin{longtable}[]{@{}
  >{\raggedright\arraybackslash}p{(\linewidth - 4\tabcolsep) * \real{0.2500}}
  >{\raggedright\arraybackslash}p{(\linewidth - 4\tabcolsep) * \real{0.3571}}
  >{\raggedright\arraybackslash}p{(\linewidth - 4\tabcolsep) * \real{0.3929}}@{}}
\toprule\noalign{}
\begin{minipage}[b]{\linewidth}\raggedright
સ્ટેજ
\end{minipage} & \begin{minipage}[b]{\linewidth}\raggedright
ફંક્શન
\end{minipage} & \begin{minipage}[b]{\linewidth}\raggedright
ફાયદો
\end{minipage} \\
\midrule\noalign{}
\endhead
\bottomrule\noalign{}
\endlastfoot
ઇનપુટ & ઓછા વોલ્ટેજનું કંટ્રોલ સિગ્નલ LED ને એક્ટિવેટ કરે છે & હાઇ પાવરથી આઇસોલેશન \\
ઓપ્ટો-કપલર & LED લાઇટ ફોટો-સેન્સિટિવ SCR ને ટ્રિગર કરે છે & ઇલેક્ટ્રિકલ આઇસોલેશન \\
ડ્રાઇવર સર્કિટ & ફોટો-SCR મુખ્ય સ્વિચિંગ ડિવાઇસને એક્ટિવેટ કરે છે & સ્વિચિંગ ક્ષમતાનું
એમ્પ્લિફિકેશન \\
આઉટપુટ સ્ટેજ & મુખ્ય SCR/TRIAC હાઇ-પાવર લોડને નિયંત્રિત કરે છે & લોડ કરંટને સંભાળે
છે \\
સ્નબર & RC સર્કિટ વોલ્ટેજ સ્પાઇક્સથી રક્ષણ આપે છે & ખોટા ટ્રિગરિંગને રોકે છે \\
\end{longtable}
}

\begin{itemize}
\tightlist
\item
  \textbf{ઇલેક્ટ્રિકલ આઇસોલેશન}: કંટ્રોલ અને પાવર સર્કિટ વચ્ચે સંપૂર્ણ અલગતા
  (\textgreater1000V)
\item
  \textbf{ઝીરો-ક્રોસિંગ}: માત્ર ઝીરો વોલ્ટેજ પર સ્વિચિંગ EMI/RFI નોઇઝ ઘટાડે છે
\item
  \textbf{સાયલેન્ટ ઓપરેશન}: પરંપરાગત રિલેથી વિપરીત, કોઈ મેકેનિકલ ક્લિક નથી
\item
  \textbf{લાંબી લાઇફ}: પરંપરાગત રિલેમાં જેવા મેકેનિકલ ઘસારો નથી
\end{itemize}

\end{solutionbox}
\begin{mnemonicbox}
``LIPO'' - ``લાઇટ ઇન, પાવર આઉટ - આઇસોલેશન ગેરંટેડ''

\end{mnemonicbox}
\subsection*{પ્રશ્ન 2(a) [3
ગુણ]}\label{q2a}

\textbf{SCR માટે સ્નબર સર્કિટનું કાર્ય સમજાવો.}

\begin{solutionbox}

\textbf{SCR માટે સ્નબર સર્કિટ:}

\begin{verbatim}
    +{-{-}{-}||{-}{-}{-}+}
    |   C1   |
    |        |
A{-{-}{-}+        +{-}{-}{-}R1{-}{-}{-}+}
|                     |
SCR                   |
|                     |
K{-{-}{-}{-}{-}{-}{-}{-}{-}{-}{-}{-}{-}{-}{-}{-}{-}{-}{-}{-}{-}+}
\end{verbatim}

{\def\LTcaptype{none} % do not increment counter
\begin{longtable}[]{@{}
  >{\raggedright\arraybackslash}p{(\linewidth - 4\tabcolsep) * \real{0.2619}}
  >{\raggedright\arraybackslash}p{(\linewidth - 4\tabcolsep) * \real{0.2143}}
  >{\raggedright\arraybackslash}p{(\linewidth - 4\tabcolsep) * \real{0.5238}}@{}}
\toprule\noalign{}
\begin{minipage}[b]{\linewidth}\raggedright
કોમ્પોનન્ટ
\end{minipage} & \begin{minipage}[b]{\linewidth}\raggedright
હેતુ
\end{minipage} & \begin{minipage}[b]{\linewidth}\raggedright
સાઇઝિંગ કન્સિડરેશન
\end{minipage} \\
\midrule\noalign{}
\endhead
\bottomrule\noalign{}
\endlastfoot
કેપેસિટર (C1) & dv/dt રેટને મર્યાદિત કરે છે & SCRની મહત્તમ dv/dt રેટિંગ પર
આધારિત \\
રેઝિસ્ટર (R1) & ડિસ્ચાર્જ કરંટને મર્યાદિત કરે છે & કેપેસિટર વેલ્યુ અને સ્વિચિંગ ફ્રિક્વન્સી
પર આધારિત \\
\end{longtable}
}

\begin{itemize}
\tightlist
\item
  \textbf{dv/dt પ્રોટેક્શન}: ઝડપી વોલ્ટેજ વધારાને કારણે ખોટા ટ્રિગરિંગને રોકે છે
\item
  \textbf{ટર્ન-ઓફ સપોર્ટ}: વૈકલ્પિક પાથ પ્રદાન કરીને કમ્યુટેશનમાં મદદ કરે છે
\item
  \textbf{એનર્જી એબ્સોર્પશન}: સ્વિચિંગ દરમિયાન ઇન્ડક્ટિવ લોડથી ઊર્જા શોષે છે
\end{itemize}

\end{solutionbox}
\begin{mnemonicbox}
``CARD'' - ``કેપેસિટર એન્ડ રેઝિસ્ટર ડેમ્પ અનવોન્ટેડ ટ્રિગરિંગ''

\end{mnemonicbox}
\subsection*{પ્રશ્ન 2(b) [4
ગુણ]}\label{q2b}

\textbf{ફોર્સ્ડ અને નેચરલ કોમ્યુટેશન વચ્ચેનો તફાવત લખો.}

\begin{solutionbox}

\textbf{કોમ્યુટેશન પદ્ધતિઓની તુલના:}

{\def\LTcaptype{none} % do not increment counter
\begin{longtable}[]{@{}
  >{\raggedright\arraybackslash}p{(\linewidth - 4\tabcolsep) * \real{0.2157}}
  >{\raggedright\arraybackslash}p{(\linewidth - 4\tabcolsep) * \real{0.3725}}
  >{\raggedright\arraybackslash}p{(\linewidth - 4\tabcolsep) * \real{0.4118}}@{}}
\toprule\noalign{}
\begin{minipage}[b]{\linewidth}\raggedright
પેરામીટર
\end{minipage} & \begin{minipage}[b]{\linewidth}\raggedright
ફોર્સ્ડ કોમ્યુટેશન
\end{minipage} & \begin{minipage}[b]{\linewidth}\raggedright
નેચરલ કોમ્યુટેશન
\end{minipage} \\
\midrule\noalign{}
\endhead
\bottomrule\noalign{}
\endlastfoot
વ્યાખ્યા & બાહ્ય સર્કિટ SCRને બંધ કરવા માટે દબાણ કરે છે & AC સ્ત્રોત કુદરતી રીતે કરંટને
શૂન્ય સુધી ઘટાડે છે \\
એપ્લિકેશન & મુખ્યત્વે DC સર્કિટ્સ & મુખ્યત્વે AC સર્કિટ્સ \\
કોમ્પોનન્ટ્સ & વધારાના ઘટકોની જરૂર પડે છે (કેપેસિટર, ઇન્ડક્ટર) & કોઈ વધારાના
ઘટકોની જરૂર નથી \\
કોમ્પ્લેક્સિટી & વધુ જટિલ સર્કિટ ડિઝાઇન & સરળ સર્કિટ ડિઝાઇન \\
એનર્જી & કોમ્યુટેશન માટે વધારાની ઊર્જા જરૂરી & હાલના સ્ત્રોત ઊર્જાનો ઉપયોગ કરે છે \\
કંટ્રોલ & ચોક્કસપણે નિયંત્રિત કરી શકાય છે & AC સાયકલના નિશ્ચિત બિંદુઓએ થાય છે \\
ખર્ચ & વધારાના ઘટકોને કારણે વધારે & ઓછી ખર્ચાળ અમલીકરણ \\
\end{longtable}
}

\begin{itemize}
\tightlist
\item
  \textbf{ટાઇમિંગ કંટ્રોલ}: ફોર્સ્ડ કોમ્યુટેશન વધુ સારો ટાઇમિંગ કંટ્રોલ આપે છે
\item
  \textbf{સર્કિટ સાઇઝ}: નેચરલ કોમ્યુટેશનથી નાની સર્કિટ સાઇઝ મળે છે
\item
  \textbf{વિશ્વસનીયતા}: નેચરલ કોમ્યુટેશનમાં નિષ્ફળ થવા માટે ઓછા ઘટકો છે
\end{itemize}

\end{solutionbox}
\begin{mnemonicbox}
``DANCE'' - ``DC નીડ્સ એક્ટિવ કોમ્યુટેશન, નેચરલ ફોર AC,
કોસ્ટ્સ એક્સ્ટ્રા ફોર ફોર્સ્ડ''

\end{mnemonicbox}
\subsection*{પ્રશ્ન 2(c) [7
ગુણ]}\label{q2c}

\textbf{બ્લોક ડાયાગ્રામની મદદથી યુપીએસની કામગીરીનું વર્ણન કરો.}

\begin{solutionbox}

\textbf{UPS બ્લોક ડાયાગ્રામ અને ઓપરેશન:}

\begin{center}
\textbf{Mermaid Diagram (Code)}
\begin{verbatim}
{Shaded}
{Highlighting}[]
graph LR
    A[AC Input] {-{-}{} B[Rectifier/Charger]}
    B {-{-}{} C[Battery Bank]}
    C {-{-}{} D[Inverter]}
    B {-{-}{} D}
    D {-{-}{} E[Output Filter]}
    E {-{-}{} F[AC Output]}
    G[Control Circuit] {-{-}{} B}
    G {-{-}{} D}
    H[Bypass Switch] {-{-}{} F}
    A {-{-}{} H}
    style A fill:\#b3e0ff
    style B fill:\#ffcccc
    style C fill:\#ffffb3
    style D fill:\#ccffcc
    style E fill:\#e6ccff
    style F fill:\#b3e0ff
    style G fill:\#ffee99
    style H fill:\#ffddbb
{Highlighting}
{Shaded}
\end{verbatim}
\end{center}

\textbf{UPS ઓપરેશન મોડ્સ:}

{\def\LTcaptype{none} % do not increment counter
\begin{longtable}[]{@{}
  >{\raggedright\arraybackslash}p{(\linewidth - 4\tabcolsep) * \real{0.1935}}
  >{\raggedright\arraybackslash}p{(\linewidth - 4\tabcolsep) * \real{0.4194}}
  >{\raggedright\arraybackslash}p{(\linewidth - 4\tabcolsep) * \real{0.3871}}@{}}
\toprule\noalign{}
\begin{minipage}[b]{\linewidth}\raggedright
મોડ
\end{minipage} & \begin{minipage}[b]{\linewidth}\raggedright
વર્ણન
\end{minipage} & \begin{minipage}[b]{\linewidth}\raggedright
પાવર પાથ
\end{minipage} \\
\midrule\noalign{}
\endhead
\bottomrule\noalign{}
\endlastfoot
નોર્મલ & AC સ્ત્રોત રેક્ટિફાયર અને ઇન્વર્ટર મારફતે લોડને પાવર આપે છે & AC ઇનપુટ \rightarrow
રેક્ટિફાયર \rightarrow ઇન્વર્ટર \rightarrow આઉટપુટ \\
બેટરી & AC નિષ્ફળ થાય ત્યારે બેટરી લોડને પાવર આપે છે & બેટરી \rightarrow ઇન્વર્ટર \rightarrow આઉટપુટ \\
બાયપાસ & મેઇન્ટેનન્સ માટે AC સીધા લોડ સાથે જોડાય છે & AC ઇનપુટ \rightarrow બાયપાસ સ્વિચ \rightarrow
આઉટપુટ \\
ચાર્જિંગ & નોર્મલ મોડમાં બેટરી ચાર્જ થાય છે & રેક્ટિફાયર \rightarrow બેટરી \\
\end{longtable}
}

\begin{itemize}
\tightlist
\item
  \textbf{ઓનલાઇન UPS}: પાવર હંમેશા રેક્ટિફાયર/ઇન્વર્ટર મારફતે વહે છે (ડબલ કન્વર્ઝન)
\item
  \textbf{ઓફલાઇન UPS}: પાવર સીધો લોડમાં જાય છે, પાવર નિષ્ફળ થાય ત્યારે બેટરી
  પર સ્વિચ થાય છે
\item
  \textbf{લાઇન-ઇન્ટરેક્ટિવ}: ઓફલાઇન જેવું પરંતુ વોલ્ટેજ રેગ્યુલેશન સાથે
\item
  \textbf{બેકઅપ ટાઇમ}: બેટરી ક્ષમતા અને લોડ જરૂરિયાતો પર આધાર રાખે છે
\end{itemize}

\end{solutionbox}
\begin{mnemonicbox}
``BRIC'' - ``બેટરી રેડી વ્હેન ઇનપુટ કટ્સ ઓફ''

\end{mnemonicbox}
\subsection*{પ્રશ્ન 2(a) OR [3
ગુણ]}\label{q2a}

\textbf{SCR ની પલ્સ ગેટ ટ્રિગરિંગ પદ્ધતિ સમજાવો.}

\begin{solutionbox}

\textbf{પલ્સ ગેટ ટ્રિગરિંગ મેથડ:}

\begin{verbatim}
      +{-{-}{-}{-}{-}+}
      |Pulse|
      |Gen. |
      +{-{-}+{-}{-}+}
         |
         v
A{-{-}{-}+{-}{-}{-}{-}{-}{-}{-}{-}+}
|   |        |
|   |  SCR   |
|   |        |
K{-{-}{-}+{-}{-}{-}{-}{-}{-}{-}{-}+}
\end{verbatim}

{\def\LTcaptype{none} % do not increment counter
\begin{longtable}[]{@{}lll@{}}
\toprule\noalign{}
પેરામીટર & સ્પેસિફિકેશન & ફાયદો \\
\midrule\noalign{}
\endhead
\bottomrule\noalign{}
\endlastfoot
પલ્સ વિડ્થ & 10-100 μs & યોગ્ય ટર્ન-ઓન સુનિશ્ચિત કરે છે \\
એમ્પ્લિટ્યુડ & થ્રેશોલ્ડથી 1-3V ઉપર & વિશ્વસનીય ટ્રિગરિંગ \\
રાઇઝ ટાઇમ & ફાસ્ટ (\textless1 μs) & ક્વિક ટર્ન-ઓન \\
ફ્રિક્વન્સી & સિંગલ અથવા ટ્રેન ઓફ પલ્સિસ & ટાઇમિંગ પર કંટ્રોલ \\
\end{longtable}
}

\begin{itemize}
\tightlist
\item
  \textbf{પ્રિસાઇઝ કંટ્રોલ}: SCR ટર્ન-ઓનનો ચોક્કસ સમય
\item
  \textbf{નોઇઝ ઇમ્યુનિટી}: ખોટા ટ્રિગરિંગને ઓછું સંવેદનશીલ
\item
  \textbf{પાવર એફિશિયન્સી}: ઓછો એવરેજ ગેટ પાવર વપરાશ
\item
  \textbf{આઇસોલેશન}: પલ્સ ટ્રાન્સફોર્મર અથવા ઓપ્ટો-આઇસોલેટર મારફતે કપલ કરી શકાય
  છે
\end{itemize}

\end{solutionbox}
\begin{mnemonicbox}
``TRAP'' - ``ટાઇમ્ડ, રિલાયબલ, એમ્પ્લિટ્યુડ-કંટ્રોલ્ડ પલ્સિસ''

\end{mnemonicbox}
\subsection*{પ્રશ્ન 2(b) OR [4
ગુણ]}\label{q2b}

\textbf{SCR ની કમ્યુટેશન પદ્ધતિઓની યાદી બનાવો અને કોઈપણ એકને વિગતવાર સમજાવો.}

\begin{solutionbox}

\textbf{SCR ની કમ્યુટેશન પદ્ધતિઓ:}

{\def\LTcaptype{none} % do not increment counter
\begin{longtable}[]{@{}lll@{}}
\toprule\noalign{}
પદ્ધતિ & સર્કિટ પ્રકાર & એપ્લિકેશન \\
\midrule\noalign{}
\endhead
\bottomrule\noalign{}
\endlastfoot
ક્લાસ A & LC દ્વારા સેલ્ફ-કોમ્યુટેટેડ & લો-પાવર ઇન્વર્ટર્સ \\
ક્લાસ B & AC સ્ત્રોત દ્વારા સેલ્ફ-કોમ્યુટેટેડ & AC પાવર કંટ્રોલ \\
ક્લાસ C & કોમ્પ્લિમેન્ટરી SCR કોમ્યુટેશન & DC ચોપર્સ \\
ક્લાસ D & એક્સટર્નલ પલ્સ કોમ્યુટેશન & DC/AC કન્વર્ટર્સ \\
ક્લાસ E & એક્સટર્નલ કેપેસિટર કોમ્યુટેશન & DC પાવર કંટ્રોલ \\
ક્લાસ F & લાઇન કોમ્યુટેશન & AC લાઇન કંટ્રોલ્ડ રેક્ટિફાયર્સ \\
\end{longtable}
}

\textbf{ક્લાસ E (કેપેસિટર કોમ્યુટેશન)ની વિગતવાર સમજૂતી:}

\begin{center}
\textbf{Mermaid Diagram (Code)}
\begin{verbatim}
{Shaded}
{Highlighting}[]
graph LR
    A[DC Source] {-{-}{} B[SCR1]}
    B {-{-}{} C[Load]}
    C {-{-}{} D[Ground]}
    A {-{-}{} E[Commutating Capacitor]}
    E {-{-}{} F[Auxiliary SCR2]}
    F {-{-}{} D}
    style A fill:\#b3e0ff
    style B fill:\#ffcccc
    style C fill:\#ffffb3
    style D fill:\#ccffcc
    style E fill:\#e6ccff
    style F fill:\#ffcccc
{Highlighting}
{Shaded}
\end{verbatim}
\end{center}

\begin{itemize}
\tightlist
\item
  \textbf{કાર્ય સિદ્ધાંત}: જ્યારે SCR1 ચાલુ હોય અને લોડ કરંટ વહન કરતો હોય, ત્યારે
  SCR2ને ફાયર કરવાથી પ્રી-ચાર્જ્ડ કેપેસિટર SCR1 પર જોડાય છે, જે તેને રિવર્સ બાયસ કરે
  છે
\item
  \textbf{ટર્ન-ઓફ ટાઇમ}: કેપેસિટર વેલ્યુ અને સર્કિટ રેઝિસ્ટન્સ દ્વારા નક્કી થાય છે
\item
  \textbf{એપ્લિકેશન્સ}: DC ચોપર્સ, પાવર કંટ્રોલ સર્કિટ્સ, ઇન્વર્ટર્સ
\item
  \textbf{ફાયદાઓ}: સરળ સર્કિટ, વિશ્વસનીય ઓપરેશન, કોસ્ટ-ઇફેક્ટિવ
\end{itemize}

\end{solutionbox}
\begin{mnemonicbox}
``CARE'' - ``કેપેસિટર એપ્લાઇઝ રિવર્સ વોલ્ટેજ ફોર
એક્સ્ટિંક્શન''

\end{mnemonicbox}
\subsection*{પ્રશ્ન 2(c) OR [7
ગુણ]}\label{q2c}

\textbf{બ્લોક ડાયાગ્રામની મદદથી SMPS ની કામગીરીનું વર્ણન કરો.}

\begin{solutionbox}

\textbf{SMPS બ્લોક ડાયાગ્રામ અને ઓપરેશન:}

\begin{center}
\textbf{Mermaid Diagram (Code)}
\begin{verbatim}
{Shaded}
{Highlighting}[]
graph LR
    A[AC Input] {-{-}{} B[EMI Filter]}
    B {-{-}{} C[Rectifier/PFC]}
    C {-{-}{} D[High Frequency Inverter]}
    D {-{-}{} E[HF Transformer]}
    E {-{-}{} F[Rectifier/Filter]}
    F {-{-}{} G[Output DC]}
    H[Feedback Control] {-{-}{} D}
    F {-{-}{} H}
    style A fill:\#b3e0ff
    style B fill:\#ffddbb
    style C fill:\#ffcccc
    style D fill:\#ccffcc
    style E fill:\#ffffb3
    style F fill:\#e6ccff
    style G fill:\#b3e0ff
    style H fill:\#ffee99
{Highlighting}
{Shaded}
\end{verbatim}
\end{center}

\textbf{SMPS કાર્ય સિદ્ધાંત:}

{\def\LTcaptype{none} % do not increment counter
\begin{longtable}[]{@{}
  >{\raggedright\arraybackslash}p{(\linewidth - 4\tabcolsep) * \real{0.2121}}
  >{\raggedright\arraybackslash}p{(\linewidth - 4\tabcolsep) * \real{0.3030}}
  >{\raggedright\arraybackslash}p{(\linewidth - 4\tabcolsep) * \real{0.4848}}@{}}
\toprule\noalign{}
\begin{minipage}[b]{\linewidth}\raggedright
બ્લોક
\end{minipage} & \begin{minipage}[b]{\linewidth}\raggedright
ફંક્શન
\end{minipage} & \begin{minipage}[b]{\linewidth}\raggedright
મુખ્ય ઘટકો
\end{minipage} \\
\midrule\noalign{}
\endhead
\bottomrule\noalign{}
\endlastfoot
EMI ફિલ્ટર & નોઇઝને દબાવે છે & ઇન્ડક્ટર્સ, કેપેસિટર્સ \\
રેક્ટિફાયર/PFC & AC ને DC માં રૂપાંતરિત કરે છે, પાવર ફેક્ટર સુધારે છે & ડાયોડ્સ, બૂસ્ટ
કન્વર્ટર \\
HF ઇન્વર્ટર & હાઇ-ફ્રીક્વન્સી AC બનાવે છે & સ્વિચિંગ ટ્રાન્ઝિસ્ટર્સ (MOSFET/IGBT) \\
HF ટ્રાન્સફોર્મર & આઇસોલેટ અને વોલ્ટેજ ટ્રાન્સફોર્મ કરે છે & ફેરાઇટ કોર ટ્રાન્સફોર્મર \\
આઉટપુટ સ્ટેજ & ક્લીન DC માટે રેક્ટિફાઇ અને ફિલ્ટર કરે છે & ફાસ્ટ ડાયોડ્સ, LC ફિલ્ટર \\
ફીડબેક & આઉટપુટ વોલ્ટેજ નિયંત્રિત કરે છે & ઓપ્ટો-આઇસોલેટર, PWM કંટ્રોલર \\
\end{longtable}
}

\begin{itemize}
\tightlist
\item
  \textbf{હાઇ એફિશિયન્સી}: લીનિયર પાવર સપ્લાય 50-60\% ની તુલનામાં 70-95\%
  કાર્યક્ષમ
\item
  \textbf{સાઇઝ રિડક્શન}: હાઇ-ફ્રીક્વન્સી ઓપરેશન નાના ટ્રાન્સફોર્મર્સને શક્ય બનાવે છે
\item
  \textbf{રેગ્યુલેશન}: ફીડબેક લૂપ ઇનપુટ/લોડ પરિવર્તન છતાં સ્થિર આઉટપુટ જાળવે છે
\item
  \textbf{પ્રોટેક્શન}: ઓવરકરંટ, ઓવરવોલ્ટેજ, અને થર્મલ પ્રોટેક્શન બિલ્ટ-ઇન
\end{itemize}

\end{solutionbox}
\begin{mnemonicbox}
``RELIEF'' - ``રેક્ટિફાય, એનર્જાઈઝ એટ હાઇ ફ્રીક્વન્સી,
આઇસોલેટ, એક્સટ્રેક્ટ DC, ફીડબેક''

\end{mnemonicbox}
\subsection*{પ્રશ્ન 3(a) [3
ગુણ]}\label{q3a}

\textbf{ઓવરવોલ્ટેજ સામે SCR ને સુરક્ષિત કરવાની પદ્ધતિ જણાવો.}

\begin{solutionbox}

\textbf{SCR ઓવરવોલ્ટેજ પ્રોટેક્શન મેથડ્સ:}

{\def\LTcaptype{none} % do not increment counter
\begin{longtable}[]{@{}lll@{}}
\toprule\noalign{}
પદ્ધતિ & સર્કિટ અમલીકરણ & પ્રોટેક્શન લેવલ \\
\midrule\noalign{}
\endhead
\bottomrule\noalign{}
\endlastfoot
સ્નબર સર્કિટ & SCR પર RC નેટવર્ક & dv/dt પ્રોટેક્શન \\
MOV (મેટલ ઓક્સાઇડ વેરિસ્ટર) & SCR પર કનેક્ટેડ & ટ્રાન્ઝિયન્ટ સપ્રેશન \\
વોલ્ટેજ ક્લેમ્પિંગ & શ્રેણીમાં ઝેનર ડાયોડ્સ & ફિક્સ્ડ વોલ્ટેજ લિમિટિંગ \\
ક્રોબાર સર્કિટ & સેન્સિંગ અને શન્ટિંગ સર્કિટ & સંપૂર્ણ શટડાઉન \\
\end{longtable}
}

\begin{itemize}
\tightlist
\item
  \textbf{વોલ્ટેજ રેટિંગ}: હંમેશા સામાન્ય ઓપરેટિંગ વોલ્ટેજથી 2-3 ગણી વોલ્ટેજ રેટિંગવાળા
  SCR નો ઉપયોગ કરો
\item
  \textbf{રેટ-ઓફ-રાઇઝ}: સ્નબર સર્કિટ્સ (dv/dt પ્રોટેક્શન) સાથે ફાસ્ટ ટ્રાન્ઝિયન્ટથી
  રક્ષણ કરો
\item
  \textbf{બ્રેકડાઉન વોલ્ટેજ}: SCR જંક્શનના રિવર્સ બ્રેકડાઉન વોલ્ટેજને ક્યારેય ઓળંગશો
  નહીં
\item
  \textbf{કોઓર્ડિનેટેડ પ્રોટેક્શન}: ક્રિટિકલ એપ્લિકેશન્સ માટે બહુવિધ પદ્ધતિઓનો ઉપયોગ
  કરો
\end{itemize}

\end{solutionbox}
\begin{mnemonicbox}
``SCRAM'' - ``સ્નબર સર્કિટ્સ રિડ્યુસ એબનોર્મલ મેક્સિમમ
વોલ્ટેજ''

\end{mnemonicbox}
\subsection*{પ્રશ્ન 3(b) [4
ગુણ]}\label{q3b}

\textbf{સિંગલ-ફેઝ રેક્ટિફાયર કરતાં પોલિફેઝ રેક્ટિફાયરના કોઈપણ ચાર ફાયદા જણાવો.}

\begin{solutionbox}

\textbf{પોલિફેઝ રેક્ટિફાયરના ફાયદાઓ:}

{\def\LTcaptype{none} % do not increment counter
\begin{longtable}[]{@{}
  >{\raggedright\arraybackslash}p{(\linewidth - 4\tabcolsep) * \real{0.3438}}
  >{\raggedright\arraybackslash}p{(\linewidth - 4\tabcolsep) * \real{0.4062}}
  >{\raggedright\arraybackslash}p{(\linewidth - 4\tabcolsep) * \real{0.2500}}@{}}
\toprule\noalign{}
\begin{minipage}[b]{\linewidth}\raggedright
ફાયદો
\end{minipage} & \begin{minipage}[b]{\linewidth}\raggedright
સમજૂતી
\end{minipage} & \begin{minipage}[b]{\linewidth}\raggedright
પ્રભાવ
\end{minipage} \\
\midrule\noalign{}
\endhead
\bottomrule\noalign{}
\endlastfoot
હાયર પાવર હેન્ડલિંગ & ફેઝ પર લોડ વિતરિત કરે છે & હાઇ-પાવર એપ્લિકેશન્સ માટે યોગ્ય \\
ઘટાડેલું રિપલ & ઓવરલેપિંગ ફેઝ આઉટપુટ રિપલ ઘટાડે છે & ઓછી ફિલ્ટરિંગની જરૂર \\
બેટર ટ્રાન્સફોર્મર યુટિલાઇઝેશન & ઉચ્ચ ટ્રાન્સફોર્મર યુટિલાઇઝેશન ફેક્ટર (0.955 vs
0.812) & વધુ અર્થવ્યવસ્થિત ડિઝાઇન \\
ઇમ્પ્રૂવ્ડ પાવર ફેક્ટર & બેટર લાઇન યુટિલાઇઝેશન & ઘટાડેલા લાઇન લોસિસ \\
લોઅર હાર્મોનિક કન્ટેન્ટ & હાર્મોનિક્સ ઉચ્ચ ફ્રિક્વન્સીથી શરૂ થાય છે & ઘટાડેલા EMI
મુદ્દાઓ \\
હાયર એફિશિયન્સી & બેટર ડિસ્ટ્રિબ્યુશનને કારણે ઘટાડેલા લોસિસ & ઓછા ઓપરેટિંગ ખર્ચ \\
\end{longtable}
}

\begin{itemize}
\tightlist
\item
  \textbf{ફોર્મ ફેક્ટર}: નીચો ફોર્મ ફેક્ટર એટલે વધુ સારી DC ક્વોલિટી
\item
  \textbf{રિપલ ફ્રિક્વન્સી}: ઉચ્ચ રિપલ ફ્રિક્વન્સી ફિલ્ટર કરવી સરળ છે
\item
  \textbf{બેલેન્સ્ડ લોડ}: પોલિફેઝ સપ્લાયમાંથી બેલેન્સ્ડ કરંટ ખેંચે છે
\item
  \textbf{સાઇઝ રિડક્શન}: નાના ફિલ્ટર ઘટકોની જરૂર પડે છે
\end{itemize}

\end{solutionbox}
\begin{mnemonicbox}
``HERBS'' - ``હાયર એફિશિયન્સી, ઇવન લોડ, રિડ્યુસ્ડ રિપલ,
બેટર PF, સ્મોલર ફિલ્ટર્સ''

\end{mnemonicbox}
\subsection*{પ્રશ્ન 3(c) [7
ગુણ]}\label{q3c}

\textbf{બ્લોક ડાયાગ્રામની મદદથી સૌર ફોટોવોલ્ટેઇક (PV) આધારિત પાવર જનરેશનની
કામગીરીનું વર્ણન કરો.}

\begin{solutionbox}

\textbf{સોલર PV પાવર જનરેશન સિસ્ટમ:}

\begin{center}
\textbf{Mermaid Diagram (Code)}
\begin{verbatim}
{Shaded}
{Highlighting}[]
graph LR
    A[Solar PV Array] {-{-}{} B[Charge Controller]}
    B {-{-}{} C[Battery Bank]}
    C {-{-}{} D[Inverter]}
    D {-{-}{} E[AC Loads]}
    D {-{-}{} F[Grid Connection]}
    B {-{-}{} G[DC Loads]}
    H[Maximum Power Point Tracker] {-{-}{} B}
    A {-{-}{} H}
    style A fill:\#ffffb3
    style B fill:\#ffcccc
    style C fill:\#b3e0ff
    style D fill:\#ccffcc
    style E fill:\#e6ccff
    style F fill:\#ffddbb
    style G fill:\#e6ccff
    style H fill:\#ffee99
{Highlighting}
{Shaded}
\end{verbatim}
\end{center}

\textbf{સિસ્ટમ ઘટકો અને કાર્યો:}

{\def\LTcaptype{none} % do not increment counter
\begin{longtable}[]{@{}
  >{\raggedright\arraybackslash}p{(\linewidth - 4\tabcolsep) * \real{0.3143}}
  >{\raggedright\arraybackslash}p{(\linewidth - 4\tabcolsep) * \real{0.2857}}
  >{\raggedright\arraybackslash}p{(\linewidth - 4\tabcolsep) * \real{0.4000}}@{}}
\toprule\noalign{}
\begin{minipage}[b]{\linewidth}\raggedright
ઘટક
\end{minipage} & \begin{minipage}[b]{\linewidth}\raggedright
કાર્ય
\end{minipage} & \begin{minipage}[b]{\linewidth}\raggedright
મુખ્ય ફીચર્સ
\end{minipage} \\
\midrule\noalign{}
\endhead
\bottomrule\noalign{}
\endlastfoot
PV એરે & સનલાઇટને DC ઇલેક્ટ્રિસિટીમાં રૂપાંતરિત કરે છે & મલ્ટિપલ સિરીઝ/પેરેલેલ કનેક્ટેડ
પેનલ્સ \\
MPPT & પાવર એક્સટ્રેક્શન મહત્તમ કરે છે & ઓપ્ટિમલ ઓપરેટિંગ પોઇન્ટ ટ્રેક કરે છે \\
ચાર્જ કંટ્રોલર & બેટરી ચાર્જિંગ મેનેજ કરે છે & ઓવરચાર્જિંગ/ડીપ ડિસ્ચાર્જ અટકાવે છે \\
બેટરી બેંક & એનર્જી સ્ટોરેજ & વિશ્વસનીયતા માટે ડીપ સાયકલ બેટરી \\
ઇન્વર્ટર & DC ને AC માં રૂપાંતરિત કરે છે & સંવેદનશીલ ઉપકરણો માટે પ્યોર સાઇન વેવ \\
ડિસ્ટ્રિબ્યુશન પેનલ & લોડ્સમાં પાવર રૂટ કરે છે & પ્રોટેક્શન ડિવાઇસિસ સમાવેશ કરે છે \\
\end{longtable}
}

\begin{itemize}
\tightlist
\item
  \textbf{ગ્રિડ-ટાઇડ સિસ્ટમ્સ}: યુટિલિટી ગ્રિડથી જોડાયેલ, વધારાની પાવર વેચી શકે
  છે
\item
  \textbf{ઓફ-ગ્રિડ સિસ્ટમ્સ}: બેટરી સ્ટોરેજ સાથે સ્ટેન્ડઅલોન સિસ્ટમ
\item
  \textbf{હાઇબ્રિડ સિસ્ટમ્સ}: બેટરી બેકઅપ સાથે બંને મોડમાં ચાલી શકે છે
\item
  \textbf{એફિશિયન્સી}: સૂર્યપ્રકાશથી વપરાશયોગ્ય વીજળી સુધીની સામાન્ય સિસ્ટમ
  કાર્યક્ષમતા 15-20\%
\end{itemize}

\end{solutionbox}
\begin{mnemonicbox}
``SIMPLE'' - ``સન ઇન, મેક્સિમમ પાવર, લોકલ એનર્જી''

\end{mnemonicbox}
\subsection*{પ્રશ્ન 3(a) OR [3
ગુણ]}\label{q3a}

\textbf{ઓવર કરંટ સામે SCR ને સુરક્ષિત કરવાની પદ્ધતિ જણાવો.}

\begin{solutionbox}

\textbf{SCR ઓવરકરંટ પ્રોટેક્શન મેથડ્સ:}

{\def\LTcaptype{none} % do not increment counter
\begin{longtable}[]{@{}lll@{}}
\toprule\noalign{}
મેથડ & અમલીકરણ & રિસ્પોન્સ ટાઇમ \\
\midrule\noalign{}
\endhead
\bottomrule\noalign{}
\endlastfoot
ફ્યુઝ & ફાસ્ટ-એક્ટિંગ સેમિકન્ડક્ટર ફ્યુઝ & ખૂબ ઝડપી (માઇક્રોસેકન્ડ) \\
સર્કિટ બ્રેકર & મેગ્નેટિક/થર્મલ બ્રેકર & મધ્યમ (મિલિસેકન્ડ) \\
કરંટ લિમિટિંગ રિએક્ટર & શ્રેણીમાં ઇન્ડક્ટર & તાત્કાલિક \\
ઇલેક્ટ્રોનિક કરંટ લિમિટિંગ & સેન્સિંગ અને કંટ્રોલ સર્કિટ & ઝડપી (માઇક્રોસેકન્ડ) \\
\end{longtable}
}

\begin{itemize}
\tightlist
\item
  \textbf{કરંટ રેટિંગ}: હંમેશા મહત્તમ ઓપરેટિંગ કરંટથી ઉપરની કરંટ રેટિંગવાળા SCR નો
  ઉપયોગ કરો
\item
  \textbf{di/dt પ્રોટેક્શન}: જંક્શન નુકસાન અટકાવવા માટે કરંટ વૃદ્ધિના દરને મર્યાદિત
  કરો
\item
  \textbf{થર્મલ મેનેજમેન્ટ}: થર્મલ રનવે અટકાવવા માટે યોગ્ય હીટસિંકિંગ
\item
  \textbf{કોઓર્ડિનેશન}: SCR ને નુકસાન થાય તે પહેલા પ્રોટેક્શન ડિવાઇસ કાર્ય કરવું
  જોઈએ
\end{itemize}

\end{solutionbox}
\begin{mnemonicbox}
``FIRE'' - ``ફ્યુઝ ઇમિડિયટલી રિસ્ટ્રિક્ટ એક્સેસિવ કરંટ''

\end{mnemonicbox}
\subsection*{પ્રશ્ન 3(b) OR [4
ગુણ]}\label{q3b}

\textbf{ડીસી ચોપરનો મૂળ સિદ્ધાંત સમજાવો.}

\begin{solutionbox}

\textbf{DC ચોપર બેઝિક પ્રિન્સિપલ:}

\begin{center}
\textbf{Mermaid Diagram (Code)}
\begin{verbatim}
{Shaded}
{Highlighting}[]
graph LR
    A[DC Input] {-{-}{} B[Switching Device]}
    B {-{-}{} C[Filter]}
    C {-{-}{} D[DC Output]}
    E[Control Circuit] {-{-}{} B}
    style A fill:\#b3e0ff
    style B fill:\#ffcccc
    style C fill:\#ffffb3
    style D fill:\#ccffcc
    style E fill:\#ffee99
{Highlighting}
{Shaded}
\end{verbatim}
\end{center}

{\def\LTcaptype{none} % do not increment counter
\begin{longtable}[]{@{}
  >{\raggedright\arraybackslash}p{(\linewidth - 4\tabcolsep) * \real{0.3438}}
  >{\raggedright\arraybackslash}p{(\linewidth - 4\tabcolsep) * \real{0.4062}}
  >{\raggedright\arraybackslash}p{(\linewidth - 4\tabcolsep) * \real{0.2500}}@{}}
\toprule\noalign{}
\begin{minipage}[b]{\linewidth}\raggedright
પેરામીટર
\end{minipage} & \begin{minipage}[b]{\linewidth}\raggedright
વર્ણન
\end{minipage} & \begin{minipage}[b]{\linewidth}\raggedright
પ્રભાવ
\end{minipage} \\
\midrule\noalign{}
\endhead
\bottomrule\noalign{}
\endlastfoot
ડ્યુટી સાયકલ (α) & કુલ પીરિયડમાં ON સમયનો ગુણોત્તર & આઉટપુટ વોલ્ટેજ નિયંત્રિત કરે
છે \\
સ્વિચિંગ ફ્રિક્વન્સી & દર સેકન્ડે ON/OFF સાયકલની સંખ્યા & રિપલ અને ફિલ્ટર સાઇઝને અસર
કરે છે \\
ચોપિંગ મેથડ & સ્ટેપ-અપ, સ્ટેપ-ડાઉન, બક-બૂસ્ટ & વોલ્ટેજ કન્વર્ઝન નક્કી કરે છે \\
કંટ્રોલ સ્ટ્રેટેજી & PWM, કરંટ મોડ, વગેરે & સિસ્ટમ રિસ્પોન્સને અસર કરે છે \\
\end{longtable}
}

\begin{itemize}
\tightlist
\item
  \textbf{બેઝિક ઇક્વેશન}: Vout = Vin \times ડ્યુટી સાયકલ (સ્ટેપ-ડાઉન ચોપર માટે)
\item
  \textbf{ઓપરેટિંગ પ્રિન્સિપલ}: રેપિડ સ્વિચિંગ એવરેજ વોલ્ટેજ નિયંત્રિત કરે છે
\item
  \textbf{ફાયદાઓ}: ઉચ્ચ કાર્યક્ષમતા, ચોક્કસ નિયંત્રણ, કોમ્પેક્ટ સાઇઝ
\item
  \textbf{એપ્લિકેશન્સ}: DC મોટર ડ્રાઇવ, બેટરી ચાર્જિંગ, DC વોલ્ટેજ રેગ્યુલેશન
\end{itemize}

\end{solutionbox}
\begin{mnemonicbox}
``DISC'' - ``ડ્યુટી સાયકલ ઇન્ફ્લુએન્સિસ સ્વિચિંગ ટુ કંટ્રોલ
આઉટપુટ''

\end{mnemonicbox}
\subsection*{પ્રશ્ન 3(c) OR [7
ગુણ]}\label{q3c}

\textbf{ડાયોડનો ઉપયોગ કરીને 3-Φ ફુલ વેવ રેક્ટિફાયરનું સર્કિટ ડાયાગ્રામ દોરો અને
સમજાવો.}

\begin{solutionbox}

\textbf{3-ફેઝ ફુલ વેવ ડાયોડ રેક્ટિફાયર (બ્રિજ કોન્ફિગરેશન):}

\begin{verbatim}
    D1      D3      D5
    /{      /      /}
   /  {    /      /  }
  /    {  /      /    }
R{-{-}{-}{-}{-}+{-}{-}+{-}{-}{-}{-}{-}{-}+{-}{-}+{-}{-}{-}{-}+{-}{-}{-}{-}+}
      |           |         |
      |           |     +   |
S{-{-}{-}{-}{-}+{-}{-}+{-}{-}{-}{-}{-}{-}+{-}{-}+{-}{-}{-}{-}| Load |}
      |  |      |  |    |   |
      |  |      |  |    +   |
T{-{-}{-}{-}{-}+{-}{-}+{-}{-}{-}{-}{-}{-}+{-}{-}+{-}{-}{-}{-}+{-}{-}{-}{-}+}
       {/       /       /}
       D2       D4       D6
\end{verbatim}

\textbf{વર્કિંગ પ્રિન્સિપલ:}

{\def\LTcaptype{none} % do not increment counter
\begin{longtable}[]{@{}lll@{}}
\toprule\noalign{}
ફેઝ & કન્ડક્શન પેટર્ન & આઉટપુટ કેરેક્ટરિસ્ટિક્સ \\
\midrule\noalign{}
\endhead
\bottomrule\noalign{}
\endlastfoot
0^\circ-60^\circ & D1 અને D6 કન્ડક્ટ & R અને T ફેઝિસ લોડ સાથે કનેક્ટેડ \\
60^\circ-120^\circ & D1 અને D2 કન્ડક્ટ & R અને S ફેઝિસ લોડ સાથે કનેક્ટેડ \\
120^\circ-180^\circ & D3 અને D2 કન્ડક્ટ & S અને R ફેઝિસ લોડ સાથે કનેક્ટેડ \\
180^\circ-240^\circ & D3 અને D4 કન્ડક્ટ & S અને T ફેઝિસ લોડ સાથે કનેક્ટેડ \\
240^\circ-300^\circ & D5 અને D4 કન્ડક્ટ & T અને S ફેઝિસ લોડ સાથે કનેક્ટેડ \\
300^\circ-360^\circ & D5 અને D6 કન્ડક્ટ & T અને R ફેઝિસ લોડ સાથે કનેક્ટેડ \\
\end{longtable}
}

\begin{itemize}
\tightlist
\item
  \textbf{રિપલ ફ્રિક્વન્સી}: ઇનપુટ ફ્રિક્વન્સીથી 6 ગણી (50/60Hz ઇનપુટ માટે
  300/360Hz)
\item
  \textbf{રિપલ ફેક્ટર}: આશરે 4.2\% (સિંગલ-ફેઝથી ઘણું ઓછું)
\item
  \textbf{એવરેજ આઉટપુટ વોલ્ટેજ}: Vdc = 1.35 \times Vrms (લાઇન વોલ્ટેજ)
\item
  \textbf{કન્ડક્શન એંગલ}: દરેક ડાયોડ સાયકલના 120^\circ માટે કન્ડક્ટ કરે છે
\end{itemize}

\end{solutionbox}
\begin{mnemonicbox}
``PRESTO'' - ``પેર્સ ઓફ ડાયોડ્સ રેક્ટિફાય એફિશિયન્ટલી,
સિક્સ ટાઇમ્સ પર સાયકલ આઉટપુટ''

\end{mnemonicbox}
\subsection*{પ્રશ્ન 4(a) [3
ગુણ]}\label{q4a}

\textbf{ઇન્ડક્શન હીટિંગની એપ્લિકેશનો લખો.}

\begin{solutionbox}

\textbf{ઇન્ડક્શન હીટિંગની એપ્લિકેશન્સ:}

{\def\LTcaptype{none} % do not increment counter
\begin{longtable}[]{@{}
  >{\raggedright\arraybackslash}p{(\linewidth - 4\tabcolsep) * \real{0.3953}}
  >{\raggedright\arraybackslash}p{(\linewidth - 4\tabcolsep) * \real{0.3256}}
  >{\raggedright\arraybackslash}p{(\linewidth - 4\tabcolsep) * \real{0.2791}}@{}}
\toprule\noalign{}
\begin{minipage}[b]{\linewidth}\raggedright
એપ્લિકેશન એરિયા
\end{minipage} & \begin{minipage}[b]{\linewidth}\raggedright
સ્પેસિફિક યુઝેસ
\end{minipage} & \begin{minipage}[b]{\linewidth}\raggedright
ફાયદાઓ
\end{minipage} \\
\midrule\noalign{}
\endhead
\bottomrule\noalign{}
\endlastfoot
મેટલ હીટ ટ્રીટમેન્ટ & હાર્ડનિંગ, એનિલિંગ, ટેમ્પરિંગ & ચોક્કસ નિયંત્રણ, લોકલાઇઝ્ડ
હીટિંગ \\
મેલ્ટિંગ & ફાઉન્ડ્રી ઓપરેશન્સ, કિંમતી ધાતુઓ & ક્લીન, કાર્યક્ષમ મેલ્ટિંગ \\
વેલ્ડિંગ & પાઇપ વેલ્ડિંગ, બ્રેઝિંગ, સોલ્ડરિંગ & કેન્દ્રિત ગરમી, નો કોન્ટેક્ટ \\
ફોર્જિંગ & બિલેટ્સ પ્રી-હીટિંગ, હોટ ફોર્મિંગ & રેપિડ હીટિંગ, એનર્જી એફિશિયન્ટ \\
ઘરેલું & ઇન્ડક્શન કુકટોપ & સલામતી, કાર્યક્ષમતા, નિયંત્રણ \\
મેડિકલ & હાઇપરથર્મિયા ટ્રીટમેન્ટ & કંટ્રોલ્ડ ડીપ ટિશ્યુ હીટિંગ \\
\end{longtable}
}

\begin{itemize}
\tightlist
\item
  \textbf{ઔદ્યોગિક ફાયદાઓ}: ઝડપી હીટિંગ, ઊર્જા કાર્યક્ષમતા, ક્લીન પ્રોસેસ
\item
  \textbf{કંટ્રોલ બેનિફિટ્સ}: ચોક્કસ તાપમાન નિયંત્રણ, પુનરાવર્તનીય પરિણામો
\item
  \textbf{પર્યાવરણીય અસર}: જીવાશ્મ બળતણ હીટિંગની તુલનામાં ઘટાડેલા ઉત્સર્જન
\item
  \textbf{મેટલર્જિકલ ક્વોલિટી}: ઘણા એપ્લિકેશન્સમાં સુધારેલા મટીરિયલ પ્રોપર્ટીઝ
\end{itemize}

\end{solutionbox}
\begin{mnemonicbox}
``HAMMER'' - ``હાર્ડનિંગ, એનિલિંગ, મેલ્ટિંગ, મેડિકલ,
એડી-કરંટ કુકિંગ, રિશેપિંગ મેટલ્સ''

\end{mnemonicbox}
\subsection*{પ્રશ્ન 4(b) [4
ગુણ]}\label{q4b}

\textbf{TRIAC અને DIAC નો ઉપયોગ કરીને AC લોડને નિયંત્રિત કરવાની સર્કિટ દોરો અને
સમજાવો.}

\begin{solutionbox}

\textbf{TRIAC અને DIAC સાથે AC લોડ કંટ્રોલ:}

\begin{verbatim}
      R1          C1
AC o{-{-}///{-}{-}+{-}{-}{-}||{-}{-}{-}+}
              |        |
              |  DIAC  |
              |   |    |
              |   v    |
              +{-{-}{-}+{-}{-}{-}{-}+}
              |        |
              | TRIAC  |
              |        |
AC o{-{-}{-}{-}{-}{-}{-}{-}{-}{-}+{-}{-}{-}{-}{-}{-}{-}{-}+{-}{-}{-}o LOAD}
\end{verbatim}

\textbf{સર્કિટ ઓપરેશન:}

{\def\LTcaptype{none} % do not increment counter
\begin{longtable}[]{@{}lll@{}}
\toprule\noalign{}
કોમ્પોનન્ટ & ફંક્શન & સર્કિટ પર અસર \\
\midrule\noalign{}
\endhead
\bottomrule\noalign{}
\endlastfoot
R1 & વેરિએબલ રેઝિસ્ટર & C1 ના ચાર્જિંગ રેટને નિયંત્રિત કરે છે \\
C1 & ટાઇમિંગ કેપેસિટર & ટ્રિગરિંગ માટે ફેઝ શિફ્ટ બનાવે છે \\
DIAC & બાય-ડિરેક્શનલ ટ્રિગર & શાર્પ ટ્રિગરિંગ પલ્સ પ્રદાન કરે છે \\
TRIAC & પાવર કંટ્રોલ ડિવાઇસ & લોડ માટે કરંટ નિયંત્રિત કરે છે \\
RC નેટવર્ક & ફેઝ-શિફ્ટ નેટવર્ક & ફાયરિંગ એંગલ નક્કી કરે છે \\
\end{longtable}
}

\begin{itemize}
\tightlist
\item
  \textbf{ફેઝ કંટ્રોલ}: R1 એડજસ્ટ કરવાથી જે ફેઝ એંગલ પર DIAC ટ્રિગર થાય છે તે
  બદલાય છે
\item
  \textbf{પાવર કંટ્રોલ}: ફાયરિંગ એંગલ બદલવાથી લોડનો એવરેજ પાવર નિયંત્રિત થાય છે
\item
  \textbf{બાય-ડિરેક્શનલ કંટ્રોલ}: AC ઇનપુટના બંને અર્ધ-ચક્રો પર કામ કરે છે
\item
  \textbf{એપ્લિકેશન્સ}: લાઇટ ડિમર, ફેન સ્પીડ કંટ્રોલ, હીટર કંટ્રોલ
\end{itemize}

\end{solutionbox}
\begin{mnemonicbox}
``CRAFT'' - ``કેપેસિટર અને રેઝિસ્ટર એડજસ્ટ ફાયરિંગ ટાઇમ''

\end{mnemonicbox}
\subsection*{પ્રશ્ન 4(c) [7
ગુણ]}\label{q4c}

\textbf{વર્કિંગ અને એપ્લિકેશન્સ સાથે સ્પોટ વેલ્ડીંગ સમજાવો.}

\begin{solutionbox}

\textbf{સ્પોટ વેલ્ડિંગ પ્રોસેસ અને એપ્લિકેશન્સ:}

\begin{center}
\textbf{Mermaid Diagram (Code)}
\begin{verbatim}
{Shaded}
{Highlighting}[]
graph LR
    A[Step 1: Material Positioning] {-{-}{} B[Step 2: Electrode Contact]}
    B {-{-}{} C[Step 3: Current Flow]}
    C {-{-}{} D[Step 4: Heat Generation]}
    D {-{-}{} E[Step 5: Weld Formation]}
    E {-{-}{} F[Step 6: Cooling]}
    style A fill:\#ffffb3
    style B fill:\#ffcccc
    style C fill:\#b3e0ff
    style D fill:\#e6ccff
    style E fill:\#ccffcc
    style F fill:\#ffddbb
{Highlighting}
{Shaded}
\end{verbatim}
\end{center}

\textbf{સ્પોટ વેલ્ડિંગ વર્કિંગ પ્રિન્સિપલ:}

{\def\LTcaptype{none} % do not increment counter
\begin{longtable}[]{@{}lll@{}}
\toprule\noalign{}
સ્ટેજ & પ્રોસેસ & પેરામીટર્સ \\
\midrule\noalign{}
\endhead
\bottomrule\noalign{}
\endlastfoot
સેટઅપ & મટીરિયલ ઇલેક્ટ્રોડ વચ્ચે મૂકવામાં આવે છે & શીટ થિકનેસ, મટીરિયલ ટાઇપ \\
કોન્ટેક્ટ & ઇલેક્ટ્રોડ્સ પ્રેશર લાગુ કરે છે & 200-1000 પાઉન્ડ પ્રેશર \\
કરંટ ફ્લો & વર્કપીસ મારફતે હાઇ કરંટ પસાર થાય છે & 1000-100,000 એમ્પિયર \\
હીટિંગ & રેઝિસ્ટન્સ લોકલાઇઝ્ડ હીટિંગ બનાવે છે & આશરે 2500^\circF તાપમાન \\
ફ્યુઝન & મટીરિયલ પીગળે છે અને નગેટ બનાવે છે & 0.1-1 સેકન્ડની અવધિ \\
કૂલિંગ & કૂલિંગ દરમિયાન પ્રેશર જાળવવામાં આવે છે & ઇલેક્ટ્રોડ કૂલિંગ મહત્વપૂર્ણ \\
\end{longtable}
}

\textbf{સ્પોટ વેલ્ડિંગના એપ્લિકેશન્સ:}

\begin{itemize}
\tightlist
\item
  \textbf{ઓટોમોટિવ}: કાર બોડી એસેમ્બલી, શીટ મેટલ જોઇનિંગ
\item
  \textbf{ઇલેક્ટ્રોનિક્સ}: બેટરી ટેબ્સ, નાના કોમ્પોનન્ટ એસેમ્બલી
\item
  \textbf{ઉપકરણો}: રેફ્રિજરેટર, વોશિંગ મશીન, ડિશવોશર
\item
  \textbf{એરોસ્પેસ}: એરક્રાફ્ટ પેનલ એસેમ્બલી, લાઇટવેઇટ સ્ટ્રક્ચર
\item
  \textbf{મેડિકલ}: સર્જિકલ ઇન્સ્ટ્રુમેન્ટ્સ, ઇમ્પ્લાન્ટેબલ ડિવાઇસિસ
\item
  \textbf{કન્ઝ્યુમર પ્રોડક્ટ્સ}: મેટલ ફર્નિચર, કન્ટેનર, રમકડાં
\end{itemize}

\end{solutionbox}
\begin{mnemonicbox}
``PCAFRI'' - ``પોઝિશન, કોમ્પ્રેસ, એપ્લાય કરંટ, ફોર્મ નગેટ,
રિલીઝ આફ્ટર કૂલિંગ, ઇન્સ્પેક્ટ''

\end{mnemonicbox}
\subsection*{પ્રશ્ન 4(a) OR [3
ગુણ]}\label{q4a}

\textbf{ડાઇલેક્ટ્રિક હીટિંગની એપ્લિકેશનો લખો.}

\begin{solutionbox}

\textbf{ડાઇલેક્ટ્રિક હીટિંગની એપ્લિકેશન્સ:}

{\def\LTcaptype{none} % do not increment counter
\begin{longtable}[]{@{}lll@{}}
\toprule\noalign{}
ઇન્ડસ્ટ્રી & એપ્લિકેશન્સ & ફાયદાઓ \\
\midrule\noalign{}
\endhead
\bottomrule\noalign{}
\endlastfoot
ફૂડ પ્રોસેસિંગ & ડિફ્રોસ્ટિંગ, કુકિંગ, પાસ્ટ્યુરાઇઝેશન & યુનિફોર્મ હીટિંગ, સ્પીડ \\
વુડ ઇન્ડસ્ટ્રી & ડ્રાઇંગ, ગ્લુ ક્યુરિંગ, ડિલેમિનેશન & રિડ્યુસ્ડ ટાઇમ, ઇમ્પ્રૂવ્ડ ક્વોલિટી \\
ટેક્સટાઇલ & યાર્ન, ફાઇબર, ફિનિશ્ડ ગુડ્સ ડ્રાઇંગ & એનર્જી એફિશિયન્સી, સ્પીડ \\
પ્લાસ્ટિક્સ & પ્રિહીટિંગ, મોલ્ડિંગ, વેલ્ડિંગ & યુનિફોર્મ હીટિંગ, નો સરફેસ ડેમેજ \\
ફાર્માસ્યુટિકલ & ડ્રાઇંગ, સ્ટેરિલાઇઝેશન & કંટ્રોલ્ડ પ્રોસેસ, સ્પીડ \\
પેપર & ડ્રાઇંગ, ગ્લુ સેટિંગ & યુનિફોર્મ મોઇસ્ચર રિમૂવલ \\
\end{longtable}
}

\begin{itemize}
\tightlist
\item
  \textbf{પ્રોસેસ બેનિફિટ્સ}: વોલ્યુમેટ્રિક હીટિંગ (માત્ર સરફેસ જ નહીં પણ સંપૂર્ણ વસ્તુને
  ગરમ કરે છે)
\item
  \textbf{સ્પીડ એડવાન્ટેજ}: પરંપરાગત હીટિંગથી નોંધપાત્ર રીતે ઝડપી
\item
  \textbf{ક્વોલિટી ઇમ્પ્રુવમેન્ટ}: વધુ યુનિફોર્મ હીટિંગ, બેટર પ્રોડક્ટ ક્વોલિટી
\item
  \textbf{એનર્જી એફિશિયન્સી}: મટીરિયલમાં ડાયરેક્ટ એનર્જી ટ્રાન્સફર
\end{itemize}

\end{solutionbox}
\begin{mnemonicbox}
``FITPP'' - ``ફૂડ, ઇન્સુલેશન ડ્રાઇંગ, ટેક્સટાઇલ, પ્લાસ્ટિક્સ,
ફાર્માસ્યુટિકલ પ્રોડક્ટ્સ''

\end{mnemonicbox}
\subsection*{પ્રશ્ન 4(b) OR [4
ગુણ]}\label{q4b}

\textbf{SCR ડીલે ટાઈમર પર ટૂંકી નોંધ લખો.}

\begin{solutionbox}

\textbf{SCR ડિલે ટાઇમર:}

\begin{center}
\textbf{Mermaid Diagram (Code)}
\begin{verbatim}
{Shaded}
{Highlighting}[]
graph LR
    A[Trigger Input] {-{-}{} B[RC Timing Circuit]}
    B {-{-}{} C[SCR]}
    C {-{-}{} D[Relay/Output Device]}
    E[Power Supply] {-{-}{} B}
    E {-{-}{} C}
    E {-{-}{} D}
    style A fill:\#b3e0ff
    style B fill:\#ffcccc
    style C fill:\#ffffb3
    style D fill:\#ccffcc
    style E fill:\#e6ccff
{Highlighting}
{Shaded}
\end{verbatim}
\end{center}

{\def\LTcaptype{none} % do not increment counter
\begin{longtable}[]{@{}lll@{}}
\toprule\noalign{}
કોમ્પોનન્ટ & ફંક્શન & સિલેક્શન ક્રાઇટેરિયા \\
\midrule\noalign{}
\endhead
\bottomrule\noalign{}
\endlastfoot
RC નેટવર્ક & ટાઇમ ડિલે નક્કી કરે છે & R\timesC આશરે ટાઇમિંગ આપે છે \\
SCR & સ્વિચિંગ એલિમેન્ટ & કરંટ રેટિંગ લોડ પર આધારિત \\
UJT/ટ્રિગર & ગેટ પલ્સ પ્રદાન કરે છે & વિશ્વસનીય ટ્રિગરિંગ સર્કિટ \\
આઉટપુટ સ્ટેજ & લોડને નિયંત્રિત કરે છે & રિલે અથવા ડાયરેક્ટ લોડ કનેક્શન \\
\end{longtable}
}

\begin{itemize}
\tightlist
\item
  \textbf{ટાઇમિંગ પ્રિન્સિપલ}: RC ચાર્જિંગ ટાઇમ ડિલે પીરિયડ નક્કી કરે છે
\item
  \textbf{એક્યુરેસી}: સામાન્ય રીતે સેટ ટાઇમના \pm5-10\%
\item
  \textbf{એપ્લિકેશન્સ}: ઔદ્યોગિક પ્રોસેસ કંટ્રોલ, સિક્વન્સ કંટ્રોલ, પ્રોટેક્શન સર્કિટ
\item
  \textbf{ફાયદાઓ}: સરળ ડિઝાઇન, વિશ્વસનીય ઓપરેશન, કોસ્ટ-ઇફેક્ટિવ
\end{itemize}

\end{solutionbox}
\begin{mnemonicbox}
``TIME'' - ``ટાઇમિંગ ઇઝ મેનેજ્ડ બાય ઇલેક્ટ્રોનિક્સ''

\end{mnemonicbox}
\subsection*{પ્રશ્ન 4(c) OR [7
ગુણ]}\label{q4c}

\textbf{સ્ટેટિક સ્વીચ તરીકે SCR નું કાર્ય સમજાવો. સ્ટેટિક સ્વીચના ફાયદા લખો.}

\begin{solutionbox}

\textbf{SCR એઝ સ્ટેટિક સ્વિચ:}

\begin{verbatim}
    +{-{-}{-}{-}{-}{-}{-}{-}{-}{-}{-}{-}{-}{-}{-}{-}{-}{-}+}
    |                  |
AC/DC o{-{-}{-}{-}{-}{-}+         |}
            SCR        LOAD
             |         |
Control o{-{-}{-}{-}|         |}
    |        |         |
    +{-{-}{-}{-}{-}{-}{-}{-}+{-}{-}{-}{-}{-}{-}{-}{-}{-}+}
\end{verbatim}

\textbf{વર્કિંગ પ્રિન્સિપલ:}

{\def\LTcaptype{none} % do not increment counter
\begin{longtable}[]{@{}lll@{}}
\toprule\noalign{}
મોડ & સ્ટેટ & કેરેક્ટરિસ્ટિક \\
\midrule\noalign{}
\endhead
\bottomrule\noalign{}
\endlastfoot
OFF સ્ટેટ & કોઈ ગેટ સિગ્નલ નહીં & હાઇ ઇમ્પિડન્સ, મિનિમલ લીકેજ \\
ON સ્ટેટ & ગેટ ટ્રિગર થયેલ & લો ઇમ્પિડન્સ, હાઇ કરંટ ફ્લો \\
ટર્ન-ON & ગેટ પલ્સ એપ્લાઇડ & ફાસ્ટ ટ્રાન્ઝિશન (μs રેન્જ) \\
ટર્ન-OFF & કરંટ હોલ્ડિંગથી નીચે પડે & AC માં ઓટોમેટિક, DC માં કમ્યુટેશનની જરૂર \\
\end{longtable}
}

\begin{itemize}
\tightlist
\item
  \textbf{DC ઓપરેશન}: ટર્ન-ઓફ માટે કમ્યુટેશન સર્કિટની જરૂર પડે છે
\item
  \textbf{AC ઓપરેશન}: ઝીરો ક્રોસિંગ પર નેચરલ ટર્ન-ઓફ
\item
  \textbf{કંટ્રોલ મેથડ્સ}: ડાયરેક્ટ ગેટ ડ્રાઇવ, પલ્સ ટ્રિગરિંગ, ઓપ્ટો-આઇસોલેશન
\item
  \textbf{પ્રોટેક્શન}: સ્નબર સર્કિટ, કરંટ લિમિટિંગની જરૂર પડે છે
\end{itemize}

\textbf{સ્ટેટિક સ્વિચના ફાયદાઓ:}

{\def\LTcaptype{none} % do not increment counter
\begin{longtable}[]{@{}
  >{\raggedright\arraybackslash}p{(\linewidth - 4\tabcolsep) * \real{0.2157}}
  >{\raggedright\arraybackslash}p{(\linewidth - 4\tabcolsep) * \real{0.2549}}
  >{\raggedright\arraybackslash}p{(\linewidth - 4\tabcolsep) * \real{0.5294}}@{}}
\toprule\noalign{}
\begin{minipage}[b]{\linewidth}\raggedright
ફાયદો
\end{minipage} & \begin{minipage}[b]{\linewidth}\raggedright
વર્ણન
\end{minipage} & \begin{minipage}[b]{\linewidth}\raggedright
મિકેનિકલ સાથે તુલના
\end{minipage} \\
\midrule\noalign{}
\endhead
\bottomrule\noalign{}
\endlastfoot
નો મુવિંગ પાર્ટ્સ & કોઈ મિકેનિકલ ઘસારો નહીં & લાંબી લાઇફટાઇમ (લાખો ઓપરેશન્સ) \\
સાયલન્ટ ઓપરેશન & સ્વિચિંગ દરમિયાન કોઈ ઓડિબલ નોઇઝ નહીં & અવાજ-સંવેદનશીલ
એપ્લિકેશન્સમાં મહત્વપૂર્ણ \\
ફાસ્ટ સ્વિચિંગ & માઇક્રોસેકન્ડ રેન્જ સ્વિચિંગ & મિકેનિકલ કોન્ટેક્ટ કરતાં ઘણું ઝડપી \\
નો આર્કિંગ & કોઈ કોન્ટેક્ટ બાઉન્સ કે આર્કિંગ નહીં & જોખમી વાતાવરણમાં વધુ સુરક્ષિત \\
સાઇઝ \& વેઇટ & કોમ્પેક્ટ અને હળવું & નોંધપાત્ર સ્પેસ સેવિંગ \\
EMI/RFI & ઓછું ઇલેક્ટ્રોમેગ્નેટિક ઇન્ટરફેરન્સ & સંવેદનશીલ ઇલેક્ટ્રોનિક્સ માટે બેટર \\
\end{longtable}
}

\begin{itemize}
\tightlist
\item
  \textbf{રિલાયબિલિટી}: ઉચ્ચ MTBF (મીન ટાઇમ બિટ્વીન ફેલ્યોર્સ)
\item
  \textbf{કંપેટિબિલિટી}: ઇલેક્ટ્રોનિક કંટ્રોલ સિસ્ટમ સાથે કામ કરે છે
\item
  \textbf{વોલ્ટેજ આઇસોલેશન}: ઓપ્ટો-આઇસોલેશન સમાવી શકે છે
\item
  \textbf{સર્જ હેન્ડલિંગ}: યોગ્ય ડિઝાઇન સાથે બેટર ટ્રાન્ઝિયન્ટ પ્રોટેક્શન
\end{itemize}

\end{solutionbox}
\begin{mnemonicbox}
``FANS'' - ``ફાસ્ટ સ્વિચિંગ, આર્ક-ફ્રી ઓપરેશન, નો મુવિંગ
પાર્ટ્સ, સાયલન્ટ ઓપરેશન''

\end{mnemonicbox}
\subsection*{પ્રશ્ન 5(a) [3
ગુણ]}\label{q5a}

\textbf{ડીસી ડ્રાઇવ શું છે? ડીસી ડ્રાઇવ્સનું વર્ગીકરણ આપો.}

\begin{solutionbox}

\textbf{DC ડ્રાઇવ વ્યાખ્યા અને વર્ગીકરણ:}

{\def\LTcaptype{none} % do not increment counter
\begin{longtable}[]{@{}
  >{\raggedright\arraybackslash}p{(\linewidth - 2\tabcolsep) * \real{0.3810}}
  >{\raggedright\arraybackslash}p{(\linewidth - 2\tabcolsep) * \real{0.6190}}@{}}
\toprule\noalign{}
\begin{minipage}[b]{\linewidth}\raggedright
પાસું
\end{minipage} & \begin{minipage}[b]{\linewidth}\raggedright
વર્ણન
\end{minipage} \\
\midrule\noalign{}
\endhead
\bottomrule\noalign{}
\endlastfoot
વ્યાખ્યા & DC મોટરની સ્પીડ, ટોર્ક અને દિશા નિયંત્રિત કરતી ઇલેક્ટ્રોનિક સિસ્ટમ \\
બેઝિક ફંક્શન & મોટર પેરામીટર્સને નિયંત્રિત કરવા માટે આર્મેચર વોલ્ટેજ અને/અથવા ફિલ્ડ
કરંટને નિયંત્રિત કરે છે \\
\end{longtable}
}

\textbf{DC ડ્રાઇવ્સનું વર્ગીકરણ:}

{\def\LTcaptype{none} % do not increment counter
\begin{longtable}[]{@{}lll@{}}
\toprule\noalign{}
વર્ગીકરણ આધાર & પ્રકારો & લાક્ષણિકતાઓ \\
\midrule\noalign{}
\endhead
\bottomrule\noalign{}
\endlastfoot
પાવર રેટિંગ & ફ્રેક્શનલ, ઇન્ટિગ્રલ, હાઇ પાવર & હોર્સપાવર રેટિંગ પર આધારિત \\
કંટ્રોલ મેથડ & ઓપન લૂપ, ક્લોઝ્ડ લૂપ & ફીડબેક મેકેનિઝમ પર આધારિત \\
ક્વોડ્રન્ટ ઓપરેશન & સિંગલ, ટુ, ફોર ક્વોડ્રન્ટ & સ્પીડ/ટોર્ક દિશા પર આધારિત \\
પાવર સપ્લાય & સિંગલ-ફેઝ, થ્રી-ફેઝ & ઇનપુટ પાવર કોન્ફિગરેશન પર આધારિત \\
કન્વર્ટર ટાઇપ & હાફ-વેવ, ફુલ-વેવ, ચોપર & પાવર કન્વર્ઝન મેથડ પર આધારિત \\
એપ્લિકેશન & જનરલ પર્પઝ, સર્વો, સ્પેશલાઇઝ્ડ & ઇન્ટેન્ડેડ યુઝ પર આધારિત \\
\end{longtable}
}

\begin{itemize}
\tightlist
\item
  \textbf{પાવર રેન્જ}: ફ્રેક્શનલ HP થી લઈને હજારો HP સુધી
\item
  \textbf{કંટ્રોલ પ્રિસિઝન}: બેઝિકથી હાઇ-પ્રિસિઝન (0.01\%)
\item
  \textbf{રિસ્પોન્સ ટાઇમ}: મિલિસેકન્ડથી માઇક્રોસેકન્ડ સુધી
\item
  \textbf{પ્રોટેક્શન}: વિવિધ બિલ્ટ-ઇન પ્રોટેક્શન ફીચર્સ
\end{itemize}

\end{solutionbox}
\begin{mnemonicbox}
``PQCAS'' - ``પાવર રેટિંગ, ક્વોડ્રન્ટ્સ, કંટ્રોલ ટાઇપ, AC
ઇનપુટ ફેઝિસ, સ્વિચિંગ મેથડ''

\end{mnemonicbox}
\subsection*{પ્રશ્ન 5(b) [4
ગુણ]}\label{q5b}

\textbf{વેરિએબલ રીલક્ટન્સ પ્રકાર સ્ટેપર મોટરનું બાંધકામ દોરો અને સમજાવો.}

\begin{solutionbox}

\textbf{વેરિએબલ રિલક્ટન્સ સ્ટેપર મોટર કન્સ્ટ્રક્શન:}

\begin{verbatim}
    +{-{-}{-}{-}{-}{-}{-}{-}{-}{-}{-}{-}{-}{-}{-}{-}{-}+}
    |                 |
    |     Stator      |
    |    +{-{-}{-}{-}{-}{-}{-}+    |}
    |    |       |    |
    |    |Rotor  |    |
    |    |       |    |
    |    +{-{-}{-}{-}{-}{-}{-}+    |}
    |                 |
    +{-{-}{-}{-}{-}{-}{-}{-}{-}{-}{-}{-}{-}{-}{-}{-}{-}+}
\end{verbatim}

{\def\LTcaptype{none} % do not increment counter
\begin{longtable}[]{@{}
  >{\raggedright\arraybackslash}p{(\linewidth - 4\tabcolsep) * \real{0.3143}}
  >{\raggedright\arraybackslash}p{(\linewidth - 4\tabcolsep) * \real{0.4000}}
  >{\raggedright\arraybackslash}p{(\linewidth - 4\tabcolsep) * \real{0.2857}}@{}}
\toprule\noalign{}
\begin{minipage}[b]{\linewidth}\raggedright
કોમ્પોનન્ટ
\end{minipage} & \begin{minipage}[b]{\linewidth}\raggedright
કન્સ્ટ્રક્શન
\end{minipage} & \begin{minipage}[b]{\linewidth}\raggedright
ફંક્શન
\end{minipage} \\
\midrule\noalign{}
\endhead
\bottomrule\noalign{}
\endlastfoot
સ્ટેટર & મલ્ટિપલ પોલ્સ અને વાઇન્ડિંગ્સ સાથે લેમિનેટેડ સ્ટીલ & એનર્જાઇઝ થવા પર મેગ્નેટિક
ફિલ્ડ બનાવે છે \\
રોટર & સોફ્ટ આયર્ન વિથ મલ્ટિપલ ટીથ, કોઈ પર્મેનન્ટ મેગ્નેટ્સ નહીં & એનર્જાઇઝ્ડ સ્ટેટર
પોલ્સ સાથે એલાઇન થાય છે \\
એર ગેપ & રોટર અને સ્ટેટર વચ્ચે નાની જગ્યા & સ્ટેપ એક્યુરેસી અને ટોર્કને અસર કરે છે \\
વાઇન્ડિંગ & સ્ટેટર પર મલ્ટિપલ ફેઝ વાઇન્ડિંગ્સ & ક્રમિક એનર્જાઇઝિંગ રોટેશન બનાવે છે \\
\end{longtable}
}

\begin{itemize}
\tightlist
\item
  \textbf{ટૂથ કોન્ફિગરેશન}: સામાન્ય રીતે રોટર ટીથ સ્ટેટર ટીથ કરતા ઓછી હોય છે
\item
  \textbf{સ્ટેપ એંગલ}: આના દ્વારા નક્કી થાય છે: સ્ટેપ એંગલ = 360^\circ \div (રોટર ટીથની
  સંખ્યા \times ફેઝની સંખ્યા)
\item
  \textbf{કન્સ્ટ્રક્શન સિમ્પ્લિસિટી}: રોટર પર કોઈ પર્મેનન્ટ મેગ્નેટ્સ કે વાઇન્ડિંગ્સ નથી
\item
  \textbf{ઓપરેટિંગ પ્રિન્સિપલ}: ફેઝિસ એનર્જાઇઝ થાય ત્યારે મેગ્નેટિક રિલક્ટન્સ પાથ
  મિનિમાઇઝ થવાનો પ્રયાસ કરે છે
\end{itemize}

\end{solutionbox}
\begin{mnemonicbox}
``STAR'' - ``સ્ટેટર એનર્જાઇઝીસ, ટીથ એલાઇન વિથ મિનિમમ
રિલક્ટન્સ''

\end{mnemonicbox}
\subsection*{પ્રશ્ન 5(c) [7
ગુણ]}\label{q5c}

\textbf{VFD (વેરિએબલ ફ્રીક્વન્સી ડ્રાઇવ) ની કામગીરી સમજાવો.}

\begin{solutionbox}

\textbf{વેરિએબલ ફ્રીક્વન્સી ડ્રાઇવ (VFD) વર્કિંગ:}

\begin{center}
\textbf{Mermaid Diagram (Code)}
\begin{verbatim}
{Shaded}
{Highlighting}[]
graph LR
    A[AC Input] {-{-}{} B[Rectifier]}
    B {-{-}{} C[DC Bus/Filter]}
    C {-{-}{} D[Inverter]}
    D {-{-}{} E[AC Motor]}
    F[Control System] {-{-}{} B}
    F {-{-}{} D}
    G[Operator Interface] {-{-}{} F}
    H[Feedback Sensors] {-{-}{} F}
    style A fill:\#b3e0ff
    style B fill:\#ffcccc
    style C fill:\#ffffb3
    style D fill:\#ccffcc
    style E fill:\#e6ccff
    style F fill:\#ffee99
    style G fill:\#ffddbb
    style H fill:\#d9ffb3
{Highlighting}
{Shaded}
\end{verbatim}
\end{center}

\textbf{VFD કોમ્પોનન્ટ્સ અને ફંક્શન્સ:}

{\def\LTcaptype{none} % do not increment counter
\begin{longtable}[]{@{}lll@{}}
\toprule\noalign{}
કોમ્પોનન્ટ & ફંક્શન & ફીચર્સ \\
\midrule\noalign{}
\endhead
\bottomrule\noalign{}
\endlastfoot
રેક્ટિફાયર & AC ને DC માં કન્વર્ટ કરે છે & 6-પલ્સ અથવા 12-પલ્સ ડિઝાઇન \\
DC બસ & ફિલ્ટર કરે છે અને એનર્જી સ્ટોર કરે છે & કેપેસિટર્સ અને ઇન્ડક્ટર્સ \\
ઇન્વર્ટર & વેરિએબલ ફ્રિક્વન્સી AC બનાવે છે & IGBT અથવા MOSFET આધારિત \\
કંટ્રોલ સિસ્ટમ & સમગ્ર ઓપરેશન મેનેજ કરે છે & માઇક્રોપ્રોસેસર આધારિત \\
HMI & યુઝર ઇન્ટરફેસ & ડિસ્પ્લે, કીપેડ, કમ્યુનિકેશન \\
પ્રોટેક્શન & સિસ્ટમ પ્રોટેક્શન & કરંટ, વોલ્ટેજ, તાપમાન સેન્સર \\
\end{longtable}
}

\textbf{વર્કિંગ પ્રિન્સિપલ:}

\begin{itemize}
\tightlist
\item
  \textbf{સ્પીડ કંટ્રોલ ઇક્વેશન}: મોટર સ્પીડ (RPM) = (ફ્રિક્વન્સી \times 120) \div પોલ્સની
  સંખ્યા
\item
  \textbf{ટોર્ક કંટ્રોલ}: V/F રેશિયો જાળવવાથી ટોર્ક આઉટપુટ નિયંત્રિત થાય છે
\item
  \textbf{સોફ્ટ સ્ટાર્ટ}: ક્રમશઃ ફ્રિક્વન્સી/વોલ્ટેજ રેમ્પ-અપ ઇનરશ કરંટ ઘટાડે છે
\item
  \textbf{બ્રેકિંગ મેથડ્સ}: રિજનરેટિવ, ડાયનેમિક, અથવા DC ઇન્જેક્શન બ્રેકિંગ
\item
  \textbf{એનર્જી સેવિંગ્સ}: ઘટાડેલી સ્પીડ પર નોંધપાત્ર ઊર્જા બચત
\item
  \textbf{એડવાન્સ્ડ ફીચર્સ}: PID કંટ્રોલ, નેટવર્ક કમ્યુનિકેશન, પ્રોગ્રામેબલ ફંક્શન્સ
\end{itemize}

\end{solutionbox}
\begin{mnemonicbox}
``DRIVE'' - ``DC કન્વર્ઝન, રેગ્યુલેશન, ઇન્વર્ટર ક્રિએટ્સ,
વેરિએબલ ફ્રિક્વન્સી, એફિશિયન્ટ મોટર કંટ્રોલ''

\end{mnemonicbox}
\subsection*{પ્રશ્ન 5(a) OR [3
ગુણ]}\label{q5a}

\textbf{હોલ ઇફેક્ટ સેન્સર શું છે અને ડીસી મોટર્સમાં તેમની ભૂમિકા શું છે?}

\begin{solutionbox}

\textbf{DC મોટર્સમાં હોલ ઇફેક્ટ સેન્સર:}

{\def\LTcaptype{none} % do not increment counter
\begin{longtable}[]{@{}ll@{}}
\toprule\noalign{}
પાસું & વર્ણન \\
\midrule\noalign{}
\endhead
\bottomrule\noalign{}
\endlastfoot
વ્યાખ્યા & મેગ્નેટિક ફિલ્ડને ડિટેક્ટ કરતા સેમિકન્ડક્ટર-આધારિત સેન્સર \\
સિદ્ધાંત & મેગ્નેટિક ફિલ્ડમાં કરંટ ફ્લોથી લંબરૂપે વોલ્ટેજ ડિફરન્સ ઉત્પન્ન થાય છે \\
સિગ્નલ આઉટપુટ & ડિજિટલ (ON/OFF) અથવા એનાલોગ (ફિલ્ડ સ્ટ્રેન્થના પ્રમાણમાં) \\
સાઇઝ & કોમ્પેક્ટ, મોટર હાઉસિંગમાં ઇન્ટિગ્રેટેડ થઈ શકે છે \\
\end{longtable}
}

\textbf{DC મોટર્સમાં રોલ:}

{\def\LTcaptype{none} % do not increment counter
\begin{longtable}[]{@{}lll@{}}
\toprule\noalign{}
ફંક્શન & એપ્લિકેશન & બેનિફિટ \\
\midrule\noalign{}
\endhead
\bottomrule\noalign{}
\endlastfoot
પોઝિશન સેન્સિંગ & રોટર પોઝિશન ડિટેક્શન & પ્રિસાઇઝ કોમ્યુટેશન ટાઇમિંગ \\
સ્પીડ મેઝરમેન્ટ & RPM કેલ્ક્યુલેશન માટે પલ્સ જનરેશન & એક્યુરેટ સ્પીડ ફીડબેક \\
ડિરેક્શન ડિટેક્શન & ફેઝ સિક્વન્સ મોનિટરિંગ & રોટેશન ડિરેક્શન કંટ્રોલ \\
કરંટ સેન્સિંગ & નોન-કોન્ટેક્ટ કરંટ મેઝરમેન્ટ & ઓવરલોડ પ્રોટેક્શન \\
\end{longtable}
}

\begin{itemize}
\tightlist
\item
  \textbf{BLDC મોટર્સ}: ઇલેક્ટ્રોનિક કોમ્યુટેશન (મિકેનિકલ કોમ્યુટેટરને રિપ્લેસ કરવા)
  માટે ક્રિટિકલ
\item
  \textbf{પ્રિસિઝન}: મિકેનિકલ સેન્સર કરતાં ઉચ્ચ ચોકસાઈ
\item
  \textbf{રિલાયબિલિટી}: કોઈ મિકેનિકલ ઘસારો નહીં, લાંબી સર્વિસ લાઇફ
\item
  \textbf{ઇન્ટિગ્રેશન}: ડ્રાઇવ ઇલેક્ટ્રોનિક્સ સાથે ઇન્ટિગ્રેટેડ થઈ શકે છે
\end{itemize}

\end{solutionbox}
\begin{mnemonicbox}
``MAPS'' - ``મેઝર્સ પોઝિશન, એઇડ્સ કોમ્યુટેશન, પ્રોવાઇડ્સ
સ્પીડ ડેટા, સેન્સિસ મેગ્નેટિક ફિલ્ડ્સ''

\end{mnemonicbox}
\subsection*{પ્રશ્ન 5(b) OR [4
ગુણ]}\label{q5b}

\textbf{સ્ટેપર મોટરના કાર્ય સિદ્ધાંતને સમજાવો.}

\begin{solutionbox}

\textbf{સ્ટેપર મોટર વર્કિંગ પ્રિન્સિપલ:}

\begin{center}
\textbf{Mermaid Diagram (Code)}
\begin{verbatim}
{Shaded}
{Highlighting}[]
graph TD
    A[Step 1: Energize Phase A] {-{-}{} B[Rotor aligns with Phase A]}
    B {-{-}{} C[Step 2: Energize Phase B]}
    C {-{-}{} D[Rotor aligns with Phase B]}
    D {-{-}{} E[Step 3: Energize Phase C]}
    E {-{-}{} F[Rotor aligns with Phase C]}
    F {-{-}{} G[Step 4: Energize Phase D]}
    G {-{-}{} H[Rotor aligns with Phase D]}
    H {-{-}{} A}
    style A fill:\#ffffb3
    style B fill:\#ffcccc
    style C fill:\#b3e0ff
    style D fill:\#ccffcc
    style E fill:\#e6ccff
    style F fill:\#ffddbb
    style G fill:\#d9ffb3
    style H fill:\#ffee99
{Highlighting}
{Shaded}
\end{verbatim}
\end{center}

{\def\LTcaptype{none} % do not increment counter
\begin{longtable}[]{@{}lll@{}}
\toprule\noalign{}
ઓપરેટિંગ મોડ & વર્ણન & ફાયદાઓ \\
\midrule\noalign{}
\endhead
\bottomrule\noalign{}
\endlastfoot
ફુલ સ્ટેપ & એક સમયે એક ફેઝ એનર્જાઇઝ્ડ & મેક્સિમમ ટોર્ક \\
હાફ સ્ટેપ & વારાફરતી એક અને બે ફેઝિસ એનર્જાઇઝ્ડ & ડબલ રેઝોલ્યુશન, સ્મૂધર \\
માઇક્રોસ્ટેપિંગ & ફેઝિસમાં પ્રોપોર્શનલ કરંટ & વેરી સ્મૂધ મોશન, હાઇ રેઝોલ્યુશન \\
વેવ ડ્રાઇવ & સિક્વેન્શિયલ સિંગલ ફેઝ એનર્જાઇઝેશન & લોઅર પાવર કન્ઝમ્પશન \\
\end{longtable}
}

\begin{itemize}
\tightlist
\item
  \textbf{પોઝિશન કંટ્રોલ}: ફીડબેક વગર ચોક્કસ એન્ગ્યુલર પોઝિશનિંગ
\item
  \textbf{સ્ટેપ એંગલ}: સામાન્ય સ્ટેપ એંગલ્સ 1.8^\circ (200 સ્ટેપ્સ/રેવ) અથવા 0.9^\circ (400
  સ્ટેપ્સ/રેવ)
\item
  \textbf{હોલ્ડિંગ ટોર્ક}: સ્ટેન્ડસ્ટિલ પર ફેઝિસ એનર્જાઇઝ્ડ હોય ત્યારે પોઝિશન જાળવે છે
\item
  \textbf{ઓપન-લૂપ કંટ્રોલ}: સામાન્ય રીતે પોઝિશન ફીડબેકની જરૂર નથી
\item
  \textbf{સ્પીડ-ટોર્ક}: સ્પીડ વધે તેમ ટોર્ક ઘટે છે
\end{itemize}

\end{solutionbox}
\begin{mnemonicbox}
``STEPS'' - ``સિક્વેન્શિયલ ટ્રિગરિંગ ઓફ ઇલેક્ટ્રોમેગ્નેટિક ફેઝિસ
કોઝિસ સ્ટેપિંગ''

\end{mnemonicbox}
\subsection*{પ્રશ્ન 5(c) OR [7
ગુણ]}\label{q5c}

\textbf{PLC નો બ્લોક ડાયાગ્રામ દોરો અને દરેક બ્લોકની કામગીરી સમજાવો.}

\begin{solutionbox}

\textbf{PLC બ્લોક ડાયાગ્રામ અને ફંક્શન્સ:}

\begin{center}
\textbf{Mermaid Diagram (Code)}
\begin{verbatim}
{Shaded}
{Highlighting}[]
graph TD
    A[Power Supply] {-{-}{} B[CPU/Processor]}
    C[Input Interface] {-{-}{} B}
    B {-{-}{} D[Output Interface]}
    B {-{-}{} E[Memory]}
    F[Programming Device] {-{-}{} B}
    G[Communication Interface] {-{-}{} B}
    style A fill:\#ffffb3
    style B fill:\#ffcccc
    style C fill:\#b3e0ff
    style D fill:\#ccffcc
    style E fill:\#e6ccff
    style F fill:\#ffddbb
    style G fill:\#d9ffb3
{Highlighting}
{Shaded}
\end{verbatim}
\end{center}

\textbf{દરેક બ્લોકનાં ફંક્શન્સ:}

{\def\LTcaptype{none} % do not increment counter
\begin{longtable}[]{@{}
  >{\raggedright\arraybackslash}p{(\linewidth - 4\tabcolsep) * \real{0.2059}}
  >{\raggedright\arraybackslash}p{(\linewidth - 4\tabcolsep) * \real{0.2941}}
  >{\raggedright\arraybackslash}p{(\linewidth - 4\tabcolsep) * \real{0.5000}}@{}}
\toprule\noalign{}
\begin{minipage}[b]{\linewidth}\raggedright
બ્લોક
\end{minipage} & \begin{minipage}[b]{\linewidth}\raggedright
ફંક્શન
\end{minipage} & \begin{minipage}[b]{\linewidth}\raggedright
લાક્ષણિકતાઓ
\end{minipage} \\
\midrule\noalign{}
\endhead
\bottomrule\noalign{}
\endlastfoot
પાવર સપ્લાય & મુખ્ય પાવરને સિસ્ટમ વોલ્ટેજમાં રૂપાંતરિત કરે છે & રેગ્યુલેટેડ, પ્રોટેક્ટેડ,
આઇસોલેશન સાથે \\
CPU/પ્રોસેસર & પ્રોગ્રામ એક્ઝિક્યુટ કરે છે, ઓપરેશન્સ નિયંત્રિત કરે છે & સ્પીડ સ્કેન ટાઇમમાં
માપવામાં આવે છે (ms) \\
ઇનપુટ ઇન્ટરફેસ & સેન્સર અને સ્વિચ સાથે કનેક્ટ કરે છે & ડિજિટલ/એનાલોગ, આઇસોલેશન,
ફિલ્ટરિંગ \\
આઉટપુટ ઇન્ટરફેસ & એક્ચુએટર અને ઇન્ડિકેટર સાથે કનેક્ટ કરે છે & રિલે/ટ્રાન્ઝિસ્ટર/ટ્રાયક
આઉટપુટ \\
મેમરી & પ્રોગ્રામ અને ડેટા સ્ટોર કરે છે & પ્રોગ્રામ, ડેટા, અને સિસ્ટમ મેમરી એરિયા \\
પ્રોગ્રામિંગ ડિવાઇસ & પ્રોગ્રામ્સ ડેવલપ અને લોડ કરવા માટે વપરાય છે & PC, હેન્ડહેલ્ડ
પ્રોગ્રામર, સોફ્ટવેર \\
કમ્યુનિકેશન & નેટવર્ક/અન્ય ડિવાઇસિસ સાથે કનેક્ટ કરે છે & ઔદ્યોગિક પ્રોટોકોલ, રિમોટ
I/O \\
\end{longtable}
}

\begin{itemize}
\tightlist
\item
  \textbf{સ્કેન સાયકલ}: ઇનપુટ વાંચવા, પ્રોગ્રામ એક્ઝિક્યુટ કરવા, આઉટપુટ અપડેટ
  કરવાની ક્રમિક પ્રક્રિયા
\item
  \textbf{પ્રોગ્રામિંગ લેંગ્વેજિસ}: લેડર ડાયાગ્રામ (LD), ફંક્શન બ્લોક ડાયાગ્રામ
  (FBD), સ્ટ્રક્ચર્ડ ટેક્સ્ટ (ST), ઇન્સ્ટ્રક્શન લિસ્ટ (IL), સિક્વેન્શિયલ ફંક્શન ચાર્ટ
  (SFC)
\item
  \textbf{મોડ્યુલરિટી}: વધારાના I/O મોડ્યુલ્સ સાથે વિસ્તૃત કરી શકાય છે
\item
  \textbf{રોબસ્ટનેસ}: કઠોર ઔદ્યોગિક પર્યાવરણ માટે ડિઝાઇન કરેલ
\item
  \textbf{રિલાયાબિલિટી}: સામાન્ય રીતે MTBF \textgreater100,000 કલાક
\end{itemize}

\end{solutionbox}
\begin{mnemonicbox}
``PICO MPC'' - ``પાવર, ઇનપુટ્સ, CPU, આઉટપુટ્સ, મેમરી,
પ્રોગ્રામિંગ ઇન્ટરફેસ, કમ્યુનિકેશન''

\end{mnemonicbox}

\end{document}
