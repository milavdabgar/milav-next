\documentclass[10pt,a4paper]{article}

% content/resources/templates/preamble.tex
\usepackage[margin=0.6in]{geometry}
\author{Milav Dabgar}
\usepackage{amsmath,amssymb,amsthm}
\usepackage{booktabs}
\usepackage{multirow}
\usepackage{xcolor}
\usepackage{tcolorbox}
\tcbuselibrary{breakable,skins}
\usepackage[colorlinks=true,linkcolor=blue]{hyperref}
\usepackage{titlesec}
\usepackage{enumitem}
\usepackage{tikz}
\usepackage{pgfplots}
\usepackage{circuitikz}
\usepackage[version=4]{mhchem}
\usepackage{longtable}
\usepackage{array}
\usepackage{float}
\usepackage{caption}
\usepackage{listings}

\lstset{
  basicstyle=\small\ttfamily,
  breaklines=true,
  breakatwhitespace=false,
  postbreak=\mbox{\textcolor{red}{$\hookrightarrow$}\space},
  float=false,
  numbers=left,
  numberstyle=\tiny\color{gray},
  numbersep=10pt,
  xleftmargin=2em,
  keywordstyle=\color{blue},
  commentstyle=\color{green!60!black},
  stringstyle=\color{purple},
  backgroundcolor=\color{gray!5},
  showstringspaces=false,
  tabsize=2,
  captionpos=b,
  keepspaces=true,
  columns=flexible
}

\pgfplotsset{compat=1.18}
\usetikzlibrary{shapes,arrows,positioning,calc,patterns,decorations.pathmorphing,decorations.markings,arrows.meta}

% Color scheme
\definecolor{headcolor}{RGB}{0,102,204}
\definecolor{keycolor}{RGB}{220,20,60}
\definecolor{solutioncolor}{RGB}{34,139,34}
\definecolor{mnemoniccolor}{RGB}{148,0,211}
\definecolor{codecolor}{RGB}{0,0,100}

% Spacing
\setlength{\parskip}{3pt}
\setlist[itemize]{nosep}
\setlist[enumerate]{nosep}

% Title formatting
\titleformat{\section}{\Large\bfseries\color{headcolor}}{\thesection}{1em}{}
\titleformat{\subsection}{\large\bfseries\color{headcolor}}{\thesubsection}{1em}{}

% Pandoc tightlist compatibility
\providecommand{\tightlist}{%
  \setlength{\itemsep}{0pt}\setlength{\parskip}{0pt}}

% Pandoc longtable compatibility
\newcounter{none}
\def\thenone{}


% content/resources/templates/gujarati-boxes.tex
\usepackage{fontspec}
\usepackage{polyglossia}

% Set Gujarati as main language (document is primarily in Gujarati)
% Note: gloss-gujarati.ldf doesn't exist in polyglossia, but it will use hyphenation patterns
\setdefaultlanguage{gujarati}
\setotherlanguage{english}

% Configure Gujarati font properly
% Use Language=Default to prevent polyglossia from trying to add language-specific features
% that don't exist for Gujarati, which causes "empty feature" warnings
\newfontfamily\gujaratifont[Script=Gujarati,AutoFakeBold=2.5,AutoFakeSlant=0.3]{Noto Sans Gujarati}
\setmainfont[Script=Gujarati,AutoFakeBold=2.5,AutoFakeSlant=0.3]{Noto Sans Gujarati}
% Use Noto Sans Gujarati for monospace to support Gujarati in text
\setmonofont[Scale=0.9]{Noto Sans Gujarati}

% Configure English to use the same font
\newfontfamily\englishfont[Script=Gujarati,AutoFakeBold=2.5,AutoFakeSlant=0.3]{Noto Sans Gujarati}

% Translations for polyglossia
\gappto\captionsgujarati{
  \renewcommand{\tablename}{કોષ્ટક}
  \renewcommand{\figurename}{આકૃતિ}
}

% Helper for TikZ nodes to ensure Gujarati font
\newcommand{\gu}[1]{{\gujaratifont #1}}

% Custom environments
\newtcolorbox{solutionbox}{
    breakable,
    enhanced,
    colback=solutioncolor!5!white,
    colframe=solutioncolor!75!black,
    fonttitle=\bfseries,
    title=જવાબ
}

\newtcolorbox{solutionboxnobreak}{
 colback=solutioncolor!5!white,
 colframe=solutioncolor!75!black,
 fonttitle=\bfseries,
 title=જવાબ
}

\newtcolorbox{keyformula}{
 breakable,
 enhanced,
 colback=keycolor!5!white,
 colframe=keycolor!75!black,
 fonttitle=\bfseries,
 title=રાસાયણિક સમીકરણ/સૂત્ર
}

\newtcolorbox{mnemonicbox}{
 breakable,
 enhanced,
 colback=mnemoniccolor!5!white,
 colframe=mnemoniccolor!75!black,
 fonttitle=\bfseries,
 title=મેમરી ટ્રીક
}


\begin{document}

\begin{center}
{\Huge\bfseries\color{headcolor} Subject Name (Gujarati)}\\[5pt]
{\LARGE 4331103 -- Summer 2024}\\[3pt]
{\large Semester 1 Study Material}\\[3pt]
{\normalsize\textit{Detailed Solutions and Explanations}}
\end{center}

\vspace{10pt}

\subsection*{પ્રશ્ન 1(a) [3
ગુણ]}\label{q1a}

\textbf{SCR ની બે ટ્રાન્ઝિસ્ટર સામ્યતા સમજાવો.}

\begin{solutionbox}
SCR એ પરસ્પર જોડાયેલા PNP અને NPN ટ્રાન્ઝિસ્ટર તરીકે રજૂ કરી શકાય
છે.

\textbf{આકૃતિ:}

\begin{center}
\textbf{Mermaid Diagram (Code)}
\begin{verbatim}
{Shaded}
{Highlighting}[]
graph LR
    A[Anode] {-{-}{-} B1[PNP Base]}
    B1 {-{-}{-} C1[PNP Collector]}
    C1 {-{-}{-} E2[NPN Emitter]}
    E2 {-{-}{-} B2[NPN Base]}
    B2 {-{-}{-} C2[NPN Collector]}
    C2 {-{-}{-} K[Cathode]}
    G[Gate] {-{-}{-} B2}
    E1[PNP Emitter] {-{-}{-} A}
    E1 {-{-}{-} B2}
    C2 {-{-}{-} B1}
{Highlighting}
{Shaded}
\end{verbatim}
\end{center}

\begin{itemize}
\tightlist
\item
  \textbf{પુનઃઉત્પાદક ક્રિયા}: જ્યારે ગેટ પ્રવાહ NPN ને ટ્રિગર કરે છે, તે PNP ને વહન
  કરવા માટે કારણભૂત બને છે, જે સ્વ-ટકાઉ પ્રવાહ બનાવે છે
\item
  \textbf{લેચિંગ મિકેનિઝમ}: એકવાર બંને ટ્રાન્ઝિસ્ટર ચાલુ થઈ જાય, ગેટ નિયંત્રણ ગુમાવે છે
  કારણ કે ફીડબેક પાથ વહન જાળવી રાખે છે
\end{itemize}

\textbf{યાદ રાખવા માટે સૂત્ર:} ``પુશ-પુલ નેટવર્ક સતત વહન ટ્રિગર કરે છે''

\end{solutionbox}
\subsection*{પ્રશ્ન 1(b) [4
ગુણ]}\label{q1b}

\textbf{IGBT ની કામગીરી અને લાક્ષણિકતા સમજાવો.}

\begin{solutionbox}
IGBT (ઇન્સુલેટેડ ગેટ બાયપોલર ટ્રાન્ઝિસ્ટર) MOSFET ઇનપુટ
લાક્ષણિકતાઓને BJT આઉટપુટ ક્ષમતાઓ સાથે જોડે છે.

\textbf{આકૃતિ:}

\begin{center}
\textbf{Mermaid Diagram (Code)}
\begin{verbatim}
{Shaded}
{Highlighting}[]
graph LR
    G[Gate] {-{-}{-} MOS[MOSFET Section]}
    MOS {-{-}{-} BJT[BJT Section]}
    BJT {-{-}{-} C[Collector]}
    E[Emitter] {-{-}{-} BJT}
{Highlighting}
{Shaded}
\end{verbatim}
\end{center}

\textbf{લાક્ષણિકતા કોષ્ટક:}

{\def\LTcaptype{none} % do not increment counter
\begin{longtable}[]{@{}ll@{}}
\toprule\noalign{}
વિશેષતા & લાક્ષણિકતા \\
\midrule\noalign{}
\endhead
\bottomrule\noalign{}
\endlastfoot
સ્વિચિંગ & ઝડપી ચાલુ થવું, મધ્યમ બંધ થવું \\
નિયંત્રણ & MOSFET જેવું વોલ્ટેજ-નિયંત્રિત \\
વહન & BJT જેવું ઓછું ફોરવર્ડ વોલ્ટેજ ડ્રોપ \\
ઉપયોગો & ઉચ્ચ વોલ્ટેજ, મધ્યમ આવૃત્તિ સ્વિચિંગ \\
\end{longtable}
}

\begin{itemize}
\tightlist
\item
  \textbf{ઇનપુટ ફાયદો}: ઉચ્ચ અવરોધ સાથે વોલ્ટેજ-નિયંત્રિત ગેટ જેને લઘુત્તમ ડ્રાઇવ
  પાવરની જરૂર છે
\item
  \textbf{આઉટપુટ ફાયદો}: ઉચ્ચ વિદ્યુત ઘનતા પર પણ ઓછો ઓન-સ્ટેટ વોલ્ટેજ ડ્રોપ
\end{itemize}

\textbf{યાદ રાખવા માટે સૂત્ર:} ``MOSFET ઇનપુટ, BJT આઉટપુટ, સંપૂર્ણ પાવર સ્વિચ
બનાવે છે''

\end{solutionbox}
\subsection*{પ્રશ્ન 1(c) [7
ગુણ]}\label{q1c}

\textbf{DIAC નું બાંધકામ, કાર્ય અને લાક્ષણિકતા સમજાવો.}

\begin{solutionbox}
DIAC (ડાયોડ ફોર ઓલ્ટરનેટિંગ કરંટ) એ દ્વિદિશ ટ્રિગરિંગ ઉપકરણ છે જે
થાઇરિસ્ટર નિયંત્રણ સર્કિટોમાં વપરાય છે.

\textbf{આકૃતિ:}

\begin{center}
\textbf{Mermaid Diagram (Code)}
\begin{verbatim}
{Shaded}
{Highlighting}[]
graph LR
    A[Terminal A] {-{-}{-} P1[P{-}region]}
    P1 {-{-}{-} N1[N{-}region]}
    N1 {-{-}{-} P2[P{-}region]}
    P2 {-{-}{-} N2[N{-}region]}
    N2 {-{-}{-} B[Terminal B]}
{Highlighting}
{Shaded}
\end{verbatim}
\end{center}

\textbf{લાક્ષણિકતા વક્ર:}

\begin{verbatim}
                    I
                    \^{}
                    |      /
                    |     /
                    |    /
            {-{-}{-}{-}{-}{-}{-}{-}+{-}{-}{-}/{-}{-}{-}{-}{-}{-}{-}{-}{-} V}
                   /|
                  / |
                 /  |
                /   |
               Break{-over voltage}
\end{verbatim}

\textbf{બાંધકામ અને કાર્ય કોષ્ટક:}

{\def\LTcaptype{none} % do not increment counter
\begin{longtable}[]{@{}ll@{}}
\toprule\noalign{}
વિશેષતા & વર્ણન \\
\midrule\noalign{}
\endhead
\bottomrule\noalign{}
\endlastfoot
સ્ટ્રક્ચર & ગેટ ટર્મિનલ વગરનું પાંચ સ્તરીય P-N-P-N \\
કાર્ય & બ્રેક-ઓવર વોલ્ટેજ પહોંચતા સુધી પ્રવાહને અવરોધે છે \\
બ્રેકઓવર & સામાન્ય રીતે બંને દિશામાં 30-40V \\
સમમિતિ & બંને દિશાઓમાં સમાન પ્રતિક્રિયા \\
ઉપયોગ & AC સર્કિટમાં TRIAC માટે ટ્રિગર ઉપકરણ \\
\end{longtable}
}

\begin{itemize}
\tightlist
\item
  \textbf{અવરોધ અવસ્થા}: બ્રેકઓવર વોલ્ટેજથી નીચે, ઉચ્ચ અવરોધ પ્રવાહને રોકે છે
\item
  \textbf{વહન અવસ્થા}: બ્રેકઓવર વોલ્ટેજથી ઉપર, નકારાત્મક અવરોધ વિસ્તાર અચાનક
  વહન સક્ષમ કરે છે
\item
  \textbf{દ્વિદિશીય}: હકારાત્મક અને નકારાત્મક વોલ્ટેજ માટે સમાન રીતે કાર્ય કરે છે
\end{itemize}

\textbf{યાદ રાખવા માટે સૂત્ર:} ``બંને દિશામાં બ્રેક વોલ્ટેજ, પછી પ્રવાહ વહે છે''

\end{solutionbox}
\subsection*{પ્રશ્ન 1(c) OR [7
ગુણ]}\label{q1c}

\textbf{ઓપ્ટો-આઇસોલેટર અને ઓપ્ટો-એસસીઆરનું બાંધકામ અને કાર્ય સમજાવો.}

\begin{solutionbox}
ઓપ્ટો-ઉપકરણો સર્કિટો વચ્ચે વિદ્યુત અલગાવ જાળવતા સિગ્નલો ટ્રાન્સફર
કરવા માટે પ્રકાશનો ઉપયોગ કરે છે.

\textbf{ઓપ્ટો-આઇસોલેટર આકૃતિ:}

\begin{center}
\textbf{Mermaid Diagram (Code)}
\begin{verbatim}
{Shaded}
{Highlighting}[]
graph LR
    A[Input] {-{-}{-} L[LED]}
    L {-{-}{-} G[Glass/Plastic]}
    G {-{-}{-} D[Phototransistor]}
    D {-{-}{-} O[Output]}
{Highlighting}
{Shaded}
\end{verbatim}
\end{center}

\textbf{ઓપ્ટો-SCR આકૃતિ:}

\begin{center}
\textbf{Mermaid Diagram (Code)}
\begin{verbatim}
{Shaded}
{Highlighting}[]
graph LR
    A[Input] {-{-}{-} L[LED]}
    L {-{-}{-} G[Glass/Plastic]}
    G {-{-}{-} S[Light{-}sensitive SCR]}
    S {-{-}{-} O[Output]}
{Highlighting}
{Shaded}
\end{verbatim}
\end{center}

\textbf{તુલના કોષ્ટક:}

{\def\LTcaptype{none} % do not increment counter
\begin{longtable}[]{@{}lll@{}}
\toprule\noalign{}
વિશેષતા & ઓપ્ટો-આઇસોલેટર & ઓપ્ટો-SCR \\
\midrule\noalign{}
\endhead
\bottomrule\noalign{}
\endlastfoot
ઇનપુટ & LED & LED \\
આઉટપુટ ઉપકરણ & ફોટોટ્રાન્ઝિસ્ટર/ફોટોડાયોડ & પ્રકાશ-સંવેદનશીલ SCR \\
અલગાવ & 2-5 kV & 2-5 kV \\
વિદ્યુત પ્રવાહ & ઓછો-મધ્યમ (100mA) & ઉચ્ચ (ઘણા એમ્પિયર) \\
ઉપયોગો & ડિજિટલ સિગ્નલ આઇસોલેશન & પાવર નિયંત્રણ, AC સ્વિચિંગ \\
\end{longtable}
}

\begin{itemize}
\tightlist
\item
  \textbf{વિદ્યુત આઇસોલેશન}: સંપૂર્ણ વિદ્યુત અલગતા અવાજ પ્રતિરક્ષા અને સુરક્ષા પ્રદાન
  કરે છે
\item
  \textbf{સિગ્નલ ટ્રાન્સફર}: પ્રકાશ કપલિંગ ગ્રાઉન્ડ લૂપ્સ અને વોલ્ટેજ સ્તરના મુદ્દાઓને
  દૂર કરે છે
\item
  \textbf{ટ્રિગરિંગ}: ઓપ્ટો-SCRમાં પ્રકાશ ગેટ વિદ્યુત પ્રવાહને SCR સક્રિયકરણ માટે
  બદલે છે
\end{itemize}

\textbf{યાદ રાખવા માટે સૂત્ર:} ``પ્રકાશ અંતર કૂદે છે જ્યારે વિદ્યુત ઘરે રહે છે''

\end{solutionbox}
\subsection*{પ્રશ્ન 2(a) [3
ગુણ]}\label{q2a}

\textbf{1) UJT 2) SCS 3) MCT નું પ્રતીક દોરો અને ઉપયોગ આપો.}

\begin{solutionbox}

\textbf{UJT (યુનિજંક્શન ટ્રાન્ઝિસ્ટર):}

\begin{verbatim}
    B2
     |
     |
     Z
    /|
   / |
B1{-{-}{-}+{-}{-}{-}E}
\end{verbatim}

\textbf{SCS (સિલિકોન કંટ્રોલ્ડ સ્વિચ):}

\begin{verbatim}
      A
      |
      |
  G2{-{-}+}
      |
      |
  G1{-{-}+}
      |
      |
      C
\end{verbatim}

\textbf{MCT (MOS-કંટ્રોલ્ડ થાઇરિસ્ટર):}

\begin{verbatim}
      A
      |
     \_|\_
 G{-{-}|\_\_\_|}
     \_|\_
     {\_/}
      |
      |
      C
\end{verbatim}

\textbf{ઉપયોગ કોષ્ટક:}

{\def\LTcaptype{none} % do not increment counter
\begin{longtable}[]{@{}ll@{}}
\toprule\noalign{}
ઉપકરણ & ઉપયોગો \\
\midrule\noalign{}
\endhead
\bottomrule\noalign{}
\endlastfoot
UJT & રિલેક્સેશન ઓસિલેટર, ટાઇમિંગ સર્કિટ, SCR ટ્રિગરિંગ \\
SCS & ઓછી પાવર સ્વિચિંગ, લેવલ ડિટેક્શન, પલ્સ જનરેશન \\
MCT & ઉચ્ચ પાવર સ્વિચિંગ, મોટર નિયંત્રણ, ઇન્વર્ટર \\
\end{longtable}
}

\textbf{યાદ રાખવા માટે સૂત્ર:} ``અનોખી ટાઇમિંગ, નિયંત્રિત સ્વિચિંગ, મુખ્ય પાવર''

\end{solutionbox}
\subsection*{પ્રશ્ન 2(b) [4
ગુણ]}\label{q2b}

\textbf{SCR માટે ગેટ પ્રોટેક્શનનું મહત્વ સમજાવો.}

\begin{solutionbox}
ગેટ પ્રોટેક્શન સર્કિટ SCRને નકલી ટ્રિગરિંગ અને વોલ્ટેજ સ્પાઇક્સથી
સુરક્ષિત રાખે છે.

\textbf{ગેટ પ્રોટેક્શન સર્કિટ:}

\begin{verbatim}
        R
    .{-{-}{-}{-}www{-}{-}{-}{-}.}
    |           |
    |     D     |
 {-{-}{-}+{-}{-}{-}{-}{-}|{-}{-}{-}{-}+{-}{-}{-} To SCR Gate}
    |           |
    {{-}{-}{-}{-}{-}{-}{-}{-}{-}{-}{-}}
\end{verbatim}

\textbf{સુરક્ષા કોષ્ટક:}

{\def\LTcaptype{none} % do not increment counter
\begin{longtable}[]{@{}lll@{}}
\toprule\noalign{}
સમસ્યા & સુરક્ષા પદ્ધતિ & હેતુ \\
\midrule\noalign{}
\endhead
\bottomrule\noalign{}
\endlastfoot
રિવર્સ વોલ્ટેજ & ગેટમાં ડાયોડ & ગેટ-કેથોડ જંક્શન નુકસાન અટકાવે છે \\
નોઇઝ & RC ફિલ્ટર & ઉચ્ચ-આવૃત્તિ ક્ષણિક અવરોધે છે \\
dV/dt ટ્રિગરિંગ & RC સ્નબર & વોલ્ટેજ વધારાનો દર નિયંત્રિત કરે છે \\
ખોટું ટ્રિગરિંગ & ગેટ રેસિસ્ટર & ગેટ કરંટને મર્યાદિત કરે છે અને નોઇઝ ટ્રિગરિંગ ટાળે છે \\
\end{longtable}
}

\begin{itemize}
\tightlist
\item
  \textbf{જંક્શન સુરક્ષા}: ગેટ-કેથોડ જંક્શનને રિવર્સ વોલ્ટેજ નુકસાનથી બચાવે છે
\item
  \textbf{નોઇઝ પ્રતિરક્ષા}: વિદ્યુત ઘોંઘાટને ફિલ્ટર કરે છે જે અનિચ્છનીય ટ્રિગરિંગનું
  કારણ બની શકે છે
\end{itemize}

\textbf{યાદ રાખવા માટે સૂત્ર:} ``ગેટની રક્ષા કરો સમસ્યાઓ અટકાવવા માટે''

\end{solutionbox}
\subsection*{પ્રશ્ન 2(c) [7
ગુણ]}\label{q2c}

\textbf{SCR ને ટ્રિગર કરવાની વિવિધ પદ્ધતિઓની યાદી બનાવો અને તેમાંથી કોઈપણ ત્રણ
સમજાવો.}

\begin{solutionbox}
SCR ટ્રિગરિંગ પદ્ધતિઓ ગેટ સક્રિયકરણ દ્વારા ઉપકરણને અવરોધનથી વહન
અવસ્થામાં રૂપાંતરિત કરે છે.

\textbf{ટ્રિગરિંગ પદ્ધતિઓ કોષ્ટક:}

{\def\LTcaptype{none} % do not increment counter
\begin{longtable}[]{@{}lll@{}}
\toprule\noalign{}
પદ્ધતિ & સિદ્ધાંત & ઉપયોગો \\
\midrule\noalign{}
\endhead
\bottomrule\noalign{}
\endlastfoot
ગેટ ટ્રિગરિંગ & ગેટમાં સીધો પ્રવાહ & સૌથી સામાન્ય પદ્ધતિ \\
થર્મલ ટ્રિગરિંગ & તાપમાન વધારો & થર્મલ પ્રોટેક્શન \\
પ્રકાશ ટ્રિગરિંગ & જંક્શન પર ફોટોન & રિમોટ સક્રિયકરણ \\
dV/dt ટ્રિગરિંગ & ઝડપી વોલ્ટેજ વધારો & ઘણીવાર અનિચ્છનીય ટ્રિગરિંગ \\
વોલ્ટેજ ટ્રિગરિંગ & બ્રેકઓવર વોલ્ટેજ ઓળંગવું & પ્રોટેક્શન સર્કિટ \\
RF ટ્રિગરિંગ & રેડિયો ફ્રિક્વન્સી સિગ્નલ & વાયરલેસ કંટ્રોલ \\
\end{longtable}
}

\textbf{1. ગેટ કરંટ ટ્રિગરિંગ:}

\begin{verbatim}
          A
          |
       \_\_\_|\_\_\_
      |   |   |
      |   R   |
G{-{-}{-}{-}{-}+{-}{-}{-}|{-}{-}+}
      |\_\_\_|\_\_\_|
          |
          K
\end{verbatim}

\begin{itemize}
\tightlist
\item
  \textbf{સીધું નિયંત્રણ}: નાનો ગેટ પ્રવાહ મોટા એનોડ પ્રવાહને શરૂ કરે છે
\item
  \textbf{પ્રવાહ રેન્જ}: SCR રેટિંગ પર આધાર રાખીને સામાન્ય રીતે 10-100mA જરૂરી
\end{itemize}

\textbf{2. પ્રકાશ ટ્રિગરિંગ (LASCR):}

\begin{verbatim}
          A
          |
       \_\_\_|\_\_\_
      |  {  |}
      |  {  | {-}{-} Light}
G{-{-}{-}{-}{-}+{-}{-}{-}|{-}{-}+}
      |\_\_\_|\_\_\_|
          |
          K
\end{verbatim}

\begin{itemize}
\tightlist
\item
  \textbf{ઓપ્ટિકલ કંટ્રોલ}: ફોટોન્સ જંક્શન પર કેરિયર્સ ઉત્પન્ન કરે છે
\item
  \textbf{અલગાવ}: કંટ્રોલ અને પાવર સર્કિટ વચ્ચે વિદ્યુત અલગાવ પ્રદાન કરે છે
\end{itemize}

\textbf{3. dV/dt ટ્રિગરિંગ:}

\begin{verbatim}
         dV
         {-{-} = high}
         dt
          A
          |
       \_\_\_|\_\_\_
      |   |   |
      |       |
G{-{-}{-}{-}{-}+{-}{-}{-}|{-}{-}+}
      |\_\_\_|\_\_\_|
          |
          K
\end{verbatim}

\begin{itemize}
\tightlist
\item
  \textbf{રેટ સંવેદનશીલતા}: ઝડપી વોલ્ટેજ વધારો જંક્શન કેપેસિટન્સ ચાર્જિંગનું કારણ બને છે
\item
  \textbf{નિવારણ}: સ્નબર સર્કિટ (RC નેટવર્ક) વોલ્ટેજ વધારાના દરને નિયંત્રિત કરે છે
\end{itemize}

\textbf{યાદ રાખવા માટે સૂત્ર:} ``ગેટ, પ્રકાશ, અને વોલ્ટેજ પરિવર્તન SCRને ચાલુ કરે
છે''

\end{solutionbox}
\subsection*{પ્રશ્ન 2(a) OR [3
ગુણ]}\label{q2a}

\textbf{ઓપ્ટો-એસસીઆરનો ઉપયોગ કરીને સોલિડ સ્ટેટ રિલેનું કાર્ય સમજાવો.}

\begin{solutionbox}
સોલિડ સ્ટેટ રિલે (SSRs) વિદ્યુત અલગાવ સાથે સંપર્ક વગરના સ્વિચિંગ
માટે ઓપ્ટો-SCRનો ઉપયોગ કરે છે.

\textbf{SSR બ્લોક ડાયાગ્રામ:}

\begin{center}
\textbf{Mermaid Diagram (Code)}
\begin{verbatim}
{Shaded}
{Highlighting}[]
graph LR
    I[Control Input] {-{-}{} LED[LED]}
    LED {-{-}{} OSCR[Opto{-}SCR]}
    OSCR {-{-}{} ZC[Zero Crossing Circuit]}
    ZC {-{-}{} TS[Thyristor Switch]}
    TS {-{-}{} O[Output Load]}
{Highlighting}
{Shaded}
\end{verbatim}
\end{center}

\textbf{ઓપરેશન કોષ્ટક:}

{\def\LTcaptype{none} % do not increment counter
\begin{longtable}[]{@{}lll@{}}
\toprule\noalign{}
સ્ટેજ & કાર્ય & લાભ \\
\midrule\noalign{}
\endhead
\bottomrule\noalign{}
\endlastfoot
ઇનપુટ સ્ટેજ & કંટ્રોલ સિગ્નલનો ઉપયોગ કરીને LED ચલાવે છે & ઓછી શક્તિ નિયંત્રણ \\
અલગાવ & પ્રકાશ વિદ્યુત અંતર પુલ કરે છે & સુરક્ષા અને અવાજ પ્રતિરક્ષા \\
ટ્રિગરિંગ & પ્રકાશ SCRને સક્રિય કરે છે & યાંત્રિક સંપર્કો નથી \\
સ્વિચિંગ & થાઇરિસ્ટર લોડ કરંટનું વહન કરે છે & આર્કિંગ કે સંપર્ક ઘસારો નથી \\
\end{longtable}
}

\begin{itemize}
\tightlist
\item
  \textbf{મૌન ઓપરેશન}: સ્વિચિંગ દરમિયાન કોઈ યાંત્રિક અવાજ નથી
\item
  \textbf{લાંબુ આયુષ્ય}: ઇલેક્ટ્રોમેકેનિકલ રિલેની જેમ સંપર્ક અવનતિ નથી
\end{itemize}

\textbf{યાદ રાખવા માટે સૂત્ર:} ``પ્રકાશ લોજિકને લોડ સાથે જોડે છે''

\end{solutionbox}
\subsection*{પ્રશ્ન 2(b) OR [4
ગુણ]}\label{q2b}

\textbf{સ્નબર સર્કિટ વ્યાખ્યાયિત કરો અને સ્નબર સર્કિટનું મહત્વ સમજાવો.}

\begin{solutionbox}
સ્નબર સર્કિટ એ સુરક્ષાત્મક નેટવર્ક છે જે સ્વિચિંગ ઉપકરણોમાં વોલ્ટેજ અને
કરંટ ક્ષણિકોને દબાવે છે.

\textbf{બેઝિક RC સ્નબર:}

\begin{verbatim}
          A
          |
     C    |
    |{-{-}{-}{-}{-}|}
    |     |
    |     Z SCR
    |     Z
    |     |
   {-{-}{-}    |}
   {-{-}{-} R  |}
    |     |
    |{-{-}{-}{-}{-}|}
          |
          K
\end{verbatim}

\textbf{મહત્વ કોષ્ટક:}

{\def\LTcaptype{none} % do not increment counter
\begin{longtable}[]{@{}lll@{}}
\toprule\noalign{}
કાર્ય & લાભ & અમલીકરણ \\
\midrule\noalign{}
\endhead
\bottomrule\noalign{}
\endlastfoot
dV/dt દમન & ખોટા ટ્રિગરિંગને રોકે છે & SCR આસપાસ RC સર્કિટ \\
વોલ્ટેજ સ્પાઇક ઘટાડો & ઓવરવોલ્ટેજથી રક્ષણ & કેપેસિટર ઊર્જા શોષે છે \\
ઓસીલેશન ડેમ્પિંગ & EMI ઘટાડે છે & રેસિસ્ટર ડેમ્પિંગ પ્રદાન કરે છે \\
ટર્ન-ઓફ સહાય & કોમ્યુટેશન સુધારે છે & ટર્ન-ઓફ દરમિયાન પ્રવાહ વાળે છે \\
\end{longtable}
}

\begin{itemize}
\tightlist
\item
  \textbf{સર્કિટ સુરક્ષા}: ઉપકરણ પર તણાવને મર્યાદિત કરીને થાઇરિસ્ટરનું આયુષ્ય વધારે
  છે
\item
  \textbf{અવાજ ઘટાડો}: આસપાસની સર્કિટોમાં ઇલેક્ટ્રોમેગ્નેટિક ઇન્ટરફેરન્સ ઘટાડે છે
\end{itemize}

\textbf{યાદ રાખવા માટે સૂત્ર:} ``અવાજ દબાવો, સંતુલિત વર્તન સરળતાથી પુનઃસ્થાપિત
થાય''

\end{solutionbox}
\subsection*{પ્રશ્ન 2(c) OR [7
ગુણ]}\label{q2c}

\textbf{SCR ની વિવિધ કોમ્યુટેશન પદ્ધતિઓની યાદી બનાવો અને તેમાંથી કોઈપણ બે
સમજાવો}

\begin{solutionbox}
કોમ્યુટેશન એ એનોડ પ્રવાહને હોલ્ડિંગ વેલ્યુ નીચે ઘટાડીને SCRને બંધ
કરવાની પ્રક્રિયા છે.

\textbf{કોમ્યુટેશન પદ્ધતિઓ કોષ્ટક:}

{\def\LTcaptype{none} % do not increment counter
\begin{longtable}[]{@{}lll@{}}
\toprule\noalign{}
પદ્ધતિ & સિદ્ધાંત & ઉપયોગો \\
\midrule\noalign{}
\endhead
\bottomrule\noalign{}
\endlastfoot
નૈસર્ગિક & AC શૂન્ય ક્રોસિંગ & AC પાવર કંટ્રોલ \\
ફોર્સ્ડ & બાહ્ય સર્કિટ & DC એપ્લિકેશન \\
વર્ગ A & LC રેઝોનન્સ & ઇન્વર્ટર \\
વર્ગ B & ઓક્ઝિલરી SCR & DC ચોપર \\
વર્ગ C & લોડ સાથે LC & વેરિએબલ ફ્રિક્વન્સી \\
વર્ગ D & ઓક્ઝિલરી સ્ત્રોત & મોટર કંટ્રોલ \\
વર્ગ E & બાહ્ય પલ્સ & ઇલેક્ટ્રોનિક સર્કિટ \\
\end{longtable}
}

\textbf{1. નૈસર્ગિક કોમ્યુટેશન:}

\begin{verbatim}
        AC
        {}
        |
        Z SCR
        Z
        |
        R Load
        |
       {-{-}{-}}
       GND
\end{verbatim}

\begin{itemize}
\tightlist
\item
  \textbf{શૂન્ય ક્રોસિંગ}: જ્યારે AC શૂન્ય પાર કરે છે અને એનોડ કરંટ હોલ્ડિંગથી નીચે પડે
  છે ત્યારે SCR બંધ થાય છે
\item
  \textbf{સરળતા}: કોમ્યુટેશન માટે કોઈ વધારાના ઘટકોની જરૂર નથી
\item
  \textbf{મર્યાદા}: ફક્ત AC સર્કિટમાં નિશ્ચિત આવૃત્તિ પર કામ કરે છે
\end{itemize}

\textbf{2. ફોર્સ્ડ કોમ્યુટેશન (વર્ગ B):}

\begin{verbatim}
    +Vdc
      |
      |    C
      Z    |
 SCR1 Z    |
      |{-{-}{-}{-}+{-}{-}{-}{-},}
      |    |    |
      R    Z SCR2
 Load |    Z    |
      |    |    |
     {-{-}{-}  {-}{-}{-}  {-}{-}{-}}
     GND  GND  GND
\end{verbatim}

\begin{itemize}
\tightlist
\item
  \textbf{ઓક્ઝિલરી SCR}: બીજું SCR (SCR2) મુખ્ય SCRને રિવર્સ બાયસ કરવા કેપેસિટર
  ડિસ્ચાર્જ કરે છે
\item
  \textbf{ટાઇમિંગ કંટ્રોલ}: SCR ક્યારે બંધ થાય તેના પર ચોક્કસ નિયંત્રણ
\item
  \textbf{એપ્લિકેશન}: DC સર્કિટમાં વપરાય છે જ્યાં નૈસર્ગિક કોમ્યુટેશન શક્ય નથી
\end{itemize}

\textbf{યાદ રાખવા માટે સૂત્ર:} ``પ્રકૃતિ પ્રવાહને અનુસરે છે, ફોર્સ્ડ પ્રવાહ કોલેપ્સ
બનાવે છે''

\end{solutionbox}
\subsection*{પ્રશ્ન 3(a) [3
ગુણ]}\label{q3a}

\textbf{સિંગલ ફેઝ રેક્ટિફાયર કરતાં પોલિફેસ રેક્ટિફાયરના ફાયદા સમજાવો.}

\begin{solutionbox}
પોલિફેઝ રેક્ટિફાયર પાવર એપ્લિકેશનમાં સિંગલ-ફેઝ ડિઝાઇન કરતાં
નોંધપાત્ર સુધારા આપે છે.

\textbf{ફાયદા કોષ્ટક:}

{\def\LTcaptype{none} % do not increment counter
\begin{longtable}[]{@{}lll@{}}
\toprule\noalign{}
પેરામીટર & સિંગલ ફેઝ & પોલિફેઝ \\
\midrule\noalign{}
\endhead
\bottomrule\noalign{}
\endlastfoot
રિપલ ફેક્ટર & ઊંચો (FW માટે 0.482) & નીચો (3-ફેઝ માટે 0.042) \\
ફોર્મ ફેક્ટર & ઊંચો & નીચો \\
કાર્યક્ષમતા & ઓછી & ઊંચી (ટ્રાન્સફોર્મર વધુ સારી રીતે વપરાય છે) \\
પાવર રેટિંગ & મર્યાદિત & ઊંચું પાવર હેન્ડલિંગ \\
હાર્મોનિક કન્ટેન્ટ & વધુ & ઓછું (વધુ સરળ DC) \\
\end{longtable}
}

\begin{itemize}
\tightlist
\item
  \textbf{આઉટપુટ સ્મૂધનેસ}: નોંધપાત્ર રીતે ઓછો રિપલ જેને નાના ફિલ્ટરિંગ ઘટકોની જરૂર
  પડે છે
\item
  \textbf{ટ્રાન્સફોર્મર ઉપયોગ}: વધુ સારો ઉપયોગ ફેક્ટર (0.955 vs 0.812)
  ટ્રાન્સફોર્મર કદ ઘટાડે છે
\end{itemize}

\textbf{યાદ રાખવા માટે સૂત્ર:} ``વધુ ફેઝ એટલે વધુ સરળ પાવર''

\end{solutionbox}
\subsection*{પ્રશ્ન 3(b) [4
ગુણ]}\label{q3b}

\textbf{UPS પર ટૂંકી નોંધ લખો.}

\begin{solutionbox}
UPS (અનઇન્ટરપ્ટિબલ પાવર સપ્લાય) મુખ્ય પાવર સપ્લાય નિષ્ફળ થાય
ત્યારે સતત પાવર પ્રદાન કરે છે.

\textbf{UPS બ્લોક ડાયાગ્રામ:}

\begin{center}
\textbf{Mermaid Diagram (Code)}
\begin{verbatim}
{Shaded}
{Highlighting}[]
graph LR
    A[AC Input] {-{-}{} R[Rectifier]}
    R {-{-}{} C[DC Bus]}
    C {-{-}{} I[Inverter]}
    I {-{-}{} O[AC Output]}
    C {{-}{-}{} B[Battery Bank]}
    S[Static Switch] {-{-}{} O}
    A {-{-}{} S}
{Highlighting}
{Shaded}
\end{verbatim}
\end{center}

\textbf{UPS પ્રકાર કોષ્ટક:}

{\def\LTcaptype{none} % do not increment counter
\begin{longtable}[]{@{}lll@{}}
\toprule\noalign{}
પ્રકાર & ઓપરેશન & એપ્લિકેશન \\
\midrule\noalign{}
\endhead
\bottomrule\noalign{}
\endlastfoot
ઓનલાઇન & હંમેશા બેટરી/ઇન્વર્ટર દ્વારા & ક્રિટિકલ સિસ્ટમ, મેડિકલ \\
ઓફલાઇન & નિષ્ફળતા પર બેટરી પર સ્વિચ & પર્સનલ કમ્પ્યુટર, નાના ઓફિસ \\
લાઇન-ઇન્ટરેક્ટિવ & વોલ્ટેજ રેગ્યુલેશન + બેકઅપ & સર્વર, નેટવર્ક ઇક્વિપમેન્ટ \\
\end{longtable}
}

\begin{itemize}
\tightlist
\item
  \textbf{બેકઅપ સમય}: બેટરી ક્ષમતા પર આધાર રાખીને સામાન્ય રીતે 5-30 મિનિટ
\item
  \textbf{સુરક્ષા}: સર્જ પ્રોટેક્શન, વોલ્ટેજ રેગ્યુલેશન, અને ફ્રિક્વન્સી સ્ટેબિલાઇઝેશન
\end{itemize}

\textbf{યાદ રાખવા માટે સૂત્ર:} ``પાવર સતત સ્વિચ હેઠળ સુરક્ષિત''

\end{solutionbox}
\subsection*{પ્રશ્ન 3(c) [7
ગુણ]}\label{q3c}

\textbf{ઇન્વર્ટરનું કાર્ય આપો અને ઇન્વર્ટરના મૂળભૂત સિદ્ધાંતને સમજાવો પણ સુઘડ ડાયાગ્રામ
અને વેવફોર્મ સાથે શ્રેણી ઇન્વર્ટર સમજાવો.}

\begin{solutionbox}
ઇન્વર્ટર ડીસી પાવરને એસી પાવરમાં રૂપાંતરિત કરે છે, ડીસીને
ટ્રાન્સફોર્મર દ્વારા કે સીધા જ સ્વિચ કરીને વૈકલ્પિક તરંગ બનાવે છે.

\textbf{કાર્ય કોષ્ટક:}

{\def\LTcaptype{none} % do not increment counter
\begin{longtable}[]{@{}ll@{}}
\toprule\noalign{}
કાર્ય & વર્ણન \\
\midrule\noalign{}
\endhead
\bottomrule\noalign{}
\endlastfoot
DC થી AC રૂપાંતરણ & સ્થિર DC ને વૈકલ્પિક AC માં રૂપાંતરિત કરે છે \\
આવૃત્તિ નિયંત્રણ & ચલિત આવૃત્તિ આઉટપુટ ઉત્પન્ન કરે છે \\
વોલ્ટેજ નિયંત્રણ & લોડ વેરિએશન છતાં સ્થિર આઉટપુટ જાળવે છે \\
વેવ શેપિંગ & સાઇન, સ્ક્વેર, કે મોડિફાઇડ સાઇન વેવ્સ ઉત્પન્ન કરે છે \\
\end{longtable}
}

\textbf{બેઝિક સિદ્ધાંત ડાયાગ્રામ:}

\begin{center}
\textbf{Mermaid Diagram (Code)}
\begin{verbatim}
{Shaded}
{Highlighting}[]
graph LR
    D[DC Source] {-{-}{} S[Switching Circuit]}
    S {-{-}{} T[Transformer/Filter]}
    T {-{-}{} A[AC Output]}
    C[Control Circuit] {-{-}{} S}
{Highlighting}
{Shaded}
\end{verbatim}
\end{center}

\textbf{શ્રેણી ઇન્વર્ટર સર્કિટ:}

\begin{verbatim}
    +Vdc
      |
      |
    \_\_|\_\_
   |     |
   C     L
   |     |
   |     |
   |     |
   |    \_|\_
   |    { /}
   |    SCR
   |    \_|\_
   |     |
   |     |
   |     |
  {-{-}{-}   {-}{-}{-}}
  GND   GND
\end{verbatim}

\textbf{વેવફોર્મ:}

\begin{verbatim}
Voltage
   \^{}
   |     \_\_\_\_
   |    /    {}
   |\_\_\_/      {\_\_\_\_}
   |
   |           \_\_\_\_
   |          /    {}
   |\_\_\_\_\_\_\_\_\_/      {\_\_\_\_}
   +{-{-}{-}{-}{-}{-}{-}{-}{-}{-}{-}{-}{-}{-}{-}{-}{-}{-}{-}{-}{-}{-} Time}
   
Current
   \^{}
   |    /{}
   |   /  {}
   |\_\_/    {\_\_/\_\_}
   |           {  }
   |            {  }
   |             {  }
   |              {/}
   +{-{-}{-}{-}{-}{-}{-}{-}{-}{-}{-}{-}{-}{-}{-}{-}{-}{-}{-}{-}{-}{-} Time}
\end{verbatim}

\begin{itemize}
\tightlist
\item
  \textbf{ઓસીલેશન}: SCR ટ્રિગર થતાં શ્રેણી LC સર્કિટ રેઝોનન્ટ ઓસીલેશન બનાવે છે
\item
  \textbf{કોમ્યુટેશન}: રેઝોનન્સ દ્વારા કરંટ રિવર્સ થાય ત્યારે SCR આપમેળે બંધ થાય છે
\item
  \textbf{આવૃત્તિ}: LC વેલ્યુ દ્વારા નક્કી થાય છે: f = 1/(2π\sqrtLC)
\end{itemize}

\textbf{યાદ રાખવા માટે સૂત્ર:} ``ડાયરેક્ટ કરંટ સ્વિચ થઈને રેઝોનન્ટ સર્કિટ દ્વારા
ઓલ્ટરનેટિંગ કરંટ બને છે''

\end{solutionbox}
\subsection*{પ્રશ્ન 3(a) OR [3
ગુણ]}\label{q3a}

\textbf{ચોપરના મૂળ સિદ્ધાંતને સમજાવો.}

\begin{solutionbox}
ચોપર એ DC-થી-DC કન્વર્ટર છે જે નિયંત્રિત સરેરાશ DC આઉટપુટ ઉત્પન્ન
કરવા માટે DC ઇનપુટને ચાલુ/બંધ કરે છે.

\textbf{બેઝિક ચોપર સર્કિટ:}

\begin{verbatim}
    +Vdc
      |
      |
     \_|\_
     { /}
      S Switch
     \_|\_
      |
      |
      R Load
      |
      |
     {-{-}{-}}
     GND
\end{verbatim}

\textbf{સિદ્ધાંત કોષ્ટક:}

{\def\LTcaptype{none} % do not increment counter
\begin{longtable}[]{@{}lll@{}}
\toprule\noalign{}
પેરામીટર & સંબંધ & નિયંત્રણ \\
\midrule\noalign{}
\endhead
\bottomrule\noalign{}
\endlastfoot
આઉટપુટ વોલ્ટેજ & Vo = Vdc \times (Ton/T) & ડ્યુટી સાયકલ એડજસ્ટમેન્ટ \\
ડ્યુટી સાયકલ & k = Ton/T & આઉટપુટ વોલ્ટેજ નિયંત્રિત કરે છે \\
આવૃત્તિ & f = 1/T & રિપલ પર અસર કરે છે \\
વોલ્ટેજ રેગ્યુલેશન & લોડ સાથે બદલાય છે & ફીડબેક કંટ્રોલ ડ્યુટી સાયકલ એડજસ્ટ કરે છે \\
\end{longtable}
}

\begin{itemize}
\tightlist
\item
  \textbf{સ્વિચિંગ એક્શન}: DC ઇનપુટને ચોપ કરવા માટે ઝડપથી ON/OFF થાય છે
\item
  \textbf{પલ્સ વિડ્થ મોડ્યુલેશન}: ON-ટાઇમ રેશિઓને બદલીને વોલ્ટેજ નિયંત્રિત કરે છે
\end{itemize}

\textbf{યાદ રાખવા માટે સૂત્ર:} ``ચોપિંગ નિયંત્રિત DC બનાવે છે''

\end{solutionbox}
\subsection*{પ્રશ્ન 3(b) OR [4
ગુણ]}\label{q3b}

\textbf{SMPS ના બ્લોક ડાયાગ્રામ દોરો અને દરેક બ્લોકનું કાર્ય સમજાવો.}

\begin{solutionbox}
SMPS (સ્વિચ્ડ મોડ પાવર સપ્લાય) ઉચ્ચ-આવૃત્તિ સ્વિચિંગનો ઉપયોગ કરીને
ઇનપુટ પાવરને નિયંત્રિત આઉટપુટમાં રૂપાંતરિત કરે છે.

\textbf{SMPS બ્લોક ડાયાગ્રામ:}

\begin{center}
\textbf{Mermaid Diagram (Code)}
\begin{verbatim}
{Shaded}
{Highlighting}[]
graph LR
    A[AC Input] {-{-}{} F[EMI Filter]}
    F {-{-}{} R[Rectifier \& Filter]}
    R {-{-}{} S[Switching Circuit]}
    S {-{-}{} T[Transformer]}
    T {-{-}{} O[Output Rectifier]}
    O {-{-}{} OF[Output Filter]}
    OF {-{-}{} OUT[DC Output]}
    FB[Feedback Control] {-{-}{} S}
    OUT {-{-}{} FB}
{Highlighting}
{Shaded}
\end{verbatim}
\end{center}

\textbf{બ્લોક્સ કાર્ય કોષ્ટક:}

{\def\LTcaptype{none} % do not increment counter
\begin{longtable}[]{@{}ll@{}}
\toprule\noalign{}
બ્લોક & કાર્ય \\
\midrule\noalign{}
\endhead
\bottomrule\noalign{}
\endlastfoot
EMI ફિલ્ટર & SMPSમાં પ્રવેશતા/છોડતા અવાજને દબાવે છે \\
રેક્ટિફાયર અને ફિલ્ટર & ACને અનિયમિત DCમાં રૂપાંતરિત કરે છે \\
સ્વિચિંગ સર્કિટ & ઉચ્ચ આવૃત્તિ (20-200kHz) પર DC ચોપ કરે છે \\
ટ્રાન્સફોર્મર & અલગાવ અને વોલ્ટેજ ટ્રાન્સફોર્મેશન પ્રદાન કરે છે \\
આઉટપુટ રેક્ટિફાયર & ઉચ્ચ-આવૃત્તિ ACને પાછો DCમાં રૂપાંતરિત કરે છે \\
આઉટપુટ ફિલ્ટર & DC આઉટપુટને સ્મૂધ કરે છે અને રિપલ દૂર કરે છે \\
ફીડબેક કંટ્રોલ & ડ્યુટી સાયકલ એડજસ્ટ કરીને આઉટપુટ નિયંત્રિત કરે છે \\
\end{longtable}
}

\begin{itemize}
\tightlist
\item
  \textbf{ઉચ્ચ કાર્યક્ષમતા}: લિનિયર સપ્લાય માટે 30-60\% ની સરખામણીએ 70-90\%
\item
  \textbf{નાનું કદ}: ઉચ્ચ આવૃત્તિ નાના ટ્રાન્સફોર્મર અને ઘટકોની મંજૂરી આપે છે
\end{itemize}

\textbf{યાદ રાખવા માટે સૂત્ર:} ``ફિલ્ટર, રેક્ટિફાય, ટ્રાન્સફોર્મર મારફતે સ્વિચ,
રેક્ટિફાય, ફિલ્ટર''

\end{solutionbox}
\subsection*{પ્રશ્ન 3(c) OR [7
ગુણ]}\label{q3c}

\textbf{વેવફોર્મ સાથે 1 ફેઝ હાફ વેવ રેક્ટિફાયર સમજાવો પણ વેવફોર્મ સાથે 3 ફેઝ ફુલ વેવ
રેક્ટિફાયર સમજાવો.}

\begin{solutionbox}
રેક્ટિફાયર એક દિશામાં પ્રવાહની મંજૂરી આપીને અને રિવર્સ ફ્લોને
અવરોધીને AC થી DC માં રૂપાંતરિત કરે છે.

\textbf{1-ફેઝ હાફ વેવ રેક્ટિફાયર:}

\begin{verbatim}
      AC
      {}
      |
     \_|\_
     { /}
      D
     \_|\_
      |
      R Load
      |
     {-{-}{-}}
     GND
\end{verbatim}

\textbf{1-ફેઝ હાફ વેવ વેવફોર્મ:}

\begin{verbatim}
Input AC
   \^{}
   |    /{         /}
   |   /  {       /  }
   |\_\_/    {\_\_\_\_\_/    \_\_\_\_}
   |
   |        /{         /}
   |       /  {       /  }
   |{     /         /    }
   +{-{-}{-}{-}{-}{-}{-}{-}{-}{-}{-}{-}{-}{-}{-}{-}{-}{-}{-}{-}{-}{-}{-} Time}
   
Output DC
   \^{}
   |    /{         /}
   |   /  {       /  }
   |\_\_/    {\_\_\_\_\_/    \_\_\_\_}
   |
   |                      
   |                      
   |
   +{-{-}{-}{-}{-}{-}{-}{-}{-}{-}{-}{-}{-}{-}{-}{-}{-}{-}{-}{-}{-}{-}{-} Time}
\end{verbatim}

\textbf{3-ફેઝ ફુલ વેવ રેક્ટિફાયર:}

\begin{verbatim}
    A o{-{-}{-}D1{-}{-}{-}.}
               |
               |
    B o{-{-}{-}D3{-}{-}{-}+{-}{-}{-}o +Vdc}
               |
               |
    C o{-{-}{-}D5{-}{-}{-}.}
               
    A o{-{-}{-}D2{-}{-}{-}.}
               |
               |
    B o{-{-}{-}D4{-}{-}{-}+{-}{-}{-}o {-}Vdc}
               |
               |
    C o{-{-}{-}D6{-}{-}{-}.}
\end{verbatim}

\textbf{3-ફેઝ ફુલ વેવ વેવફોર્મ:}

\begin{verbatim}
3{-Phase Input}
   \^{}
   |    /{    /    /    /}
   |   /  {  /    /    /  }
   |\_\_/\_\_\_\_{/\_\_\_\_/\_\_\_\_/\_\_\_\_}
   |  { / / / / / / /}
   |   {/  /  /  /  /  /}
   |
   +{-{-}{-}{-}{-}{-}{-}{-}{-}{-}{-}{-}{-}{-}{-}{-}{-}{-}{-}{-}{-}{-}{-} Time}
   
Rectified Output
   \^{}
   |   nnnnnnnnnnnnnnnnnnnnnn
   |  n  n  n  n  n  n  n  n
   |\_n\_\_\_\_n\_\_\_\_n\_\_\_\_n\_\_\_\_n\_\_\_
   |
   |
   |
   |
   +{-{-}{-}{-}{-}{-}{-}{-}{-}{-}{-}{-}{-}{-}{-}{-}{-}{-}{-}{-}{-}{-}{-} Time}
\end{verbatim}

\textbf{તુલના કોષ્ટક:}

{\def\LTcaptype{none} % do not increment counter
\begin{longtable}[]{@{}lll@{}}
\toprule\noalign{}
પેરામીટર & 1-ફેઝ હાફ વેવ & 3-ફેઝ ફુલ વેવ \\
\midrule\noalign{}
\endhead
\bottomrule\noalign{}
\endlastfoot
રિપલ ફેક્ટર & 1.21 & 0.042 \\
રેક્ટિફિકેશન કાર્યક્ષમતા & 40.6\% & 95.5\% \\
TUF & 0.287 & 0.955 \\
પીક ઇન્વર્સ વોલ્ટેજ & Vm & 2.09Vm \\
ફોર્મ ફેક્ટર & 1.57 & 1.0007 \\
\end{longtable}
}

\begin{itemize}
\tightlist
\item
  \textbf{1-ફેઝ હાફ વેવ}: સૌથી સરળ ડિઝાઇન પરંતુ ઉચ્ચ રિપલ અને ઓછી કાર્યક્ષમતા
  સાથે
\item
  \textbf{3-ફેઝ ફુલ વેવ}: એક ચક્ર દીઠ 6 પલ્સ સાથે ઘણો સરળ આઉટપુટ
\end{itemize}

\textbf{યાદ રાખવા માટે સૂત્ર:} ``અર્ધ માત્ર શિખરો પસાર કરે છે, ત્રણ ફેઝ ખીણો ભરે
છે''

\end{solutionbox}
\subsection*{પ્રશ્ન 4(a) [3
ગુણ]}\label{q4a}

\textbf{બ્લોક ડાયાગ્રામ સાથે સૌર ફોટોવોલ્ટેઇક આધારિત પાવર જનરેશનની કામગીરીનું
વર્ણન કરો.}

\begin{solutionbox}
સોલર PV પાવર જનરેશન ફોટોવોલ્ટાઇક ઇફેક્ટ દ્વારા સૂર્યપ્રકાશને સીધો
વિદ્યુતમાં રૂપાંતરિત કરે છે.

\textbf{સોલર PV સિસ્ટમ બ્લોક ડાયાગ્રામ:}

\begin{center}
\textbf{Mermaid Diagram (Code)}
\begin{verbatim}
{Shaded}
{Highlighting}[]
graph LR
    S[Solar Panel Array] {-{-}{} C[Charge Controller]}
    C {-{-}{} B[Battery Bank]}
    B {-{-}{} I[Inverter]}
    I {-{-}{} L[AC Loads]}
    C {-{-}{} D[DC Loads]}
{Highlighting}
{Shaded}
\end{verbatim}
\end{center}

\textbf{ઘટક કોષ્ટક:}

{\def\LTcaptype{none} % do not increment counter
\begin{longtable}[]{@{}ll@{}}
\toprule\noalign{}
ઘટક & કાર્ય \\
\midrule\noalign{}
\endhead
\bottomrule\noalign{}
\endlastfoot
સોલર પેનલ & સૂર્યપ્રકાશને DC વિદ્યુતમાં રૂપાંતરિત કરે છે \\
ચાર્જ કંટ્રોલર & ચાર્જિંગને નિયંત્રિત કરે છે, ઓવરચાર્જ અટકાવે છે \\
બેટરી બેંક & પછીના ઉપયોગ માટે ઊર્જા સંગ્રહિત કરે છે \\
ઇન્વર્ટર & ઘરેલું ઉપકરણો માટે DC ને AC માં રૂપાંતરિત કરે છે \\
ડિસ્ટ્રિબ્યુશન પેનલ & વિદ્યુતને લોડ તરફ રૂટ કરે છે \\
\end{longtable}
}

\begin{itemize}
\tightlist
\item
  \textbf{ઊર્જા રૂપાંતરણ}: ફોટોન્સ અર્ધવાહક સામગ્રીમાં ઇલેક્ટ્રોનને ઉત્તેજિત કરીને
  પ્રવાહ બનાવે છે
\item
  \textbf{સ્કેલેબિલિટી}: પાવર જરૂરિયાતો અનુસાર સિસ્ટમનું કદ સમાયોજિત કરી શકાય છે
\end{itemize}

\textbf{યાદ રાખવા માટે સૂત્ર:} ``સૂર્યપ્રકાશ વોલ્ટેજ ઉત્પન્ન કરે છે, બેટરી લોડને મદદ
કરે છે''

\end{solutionbox}
\subsection*{પ્રશ્ન 4(b) [4
ગુણ]}\label{q4b}

\textbf{સ્ટેટિક સ્વીચ તરીકે SCR નો ઉપયોગ સમજાવો.}

\begin{solutionbox}
SCR વિશ્વસનીય અને ઝડપી સ્વિચિંગ માટે કોઈ હલનચલન ભાગો વગરના
સોલિડ-સ્ટેટ સ્વિચ તરીકે કાર્ય કરે છે.

\textbf{SCR સ્ટેટિક સ્વિચ સર્કિટ:}

\begin{verbatim}
    +Vdc
      |
      |
     \_|\_
     { /}
     SCR
     \_|\_
      |
     {-{-}{-}  Trigger}
      |   Circuit
      R   |
 Load |   |
      |{-{-}{-}|}
      |
     {-{-}{-}}
     GND
\end{verbatim}

\textbf{એપ્લિકેશન કોષ્ટક:}

{\def\LTcaptype{none} % do not increment counter
\begin{longtable}[]{@{}lll@{}}
\toprule\noalign{}
એપ્લિકેશન & ફાયદો & અમલીકરણ \\
\midrule\noalign{}
\endhead
\bottomrule\noalign{}
\endlastfoot
પાવર કંટ્રોલ & ચોક્સાઈપૂર્ણ નિયંત્રણ, આર્કિંગ નથી & ફેઝ એંગલ કંટ્રોલ \\
મોટર સ્ટાર્ટિંગ & સરળ એક્સેલરેશન & ક્રમશઃ વોલ્ટેજ વધારો \\
સર્કિટ પ્રોટેક્શન & ઝડપી પ્રતિસાદ & કરંટ સેન્સિંગ ટ્રિગર \\
હીટિંગ કંટ્રોલ & ઊર્જા કાર્યક્ષમ & શૂન્ય-ક્રોસિંગ સ્વિચિંગ \\
\end{longtable}
}

\begin{itemize}
\tightlist
\item
  \textbf{લેચિંગ એક્શન}: એકવાર ટ્રિગર થયા પછી, પ્રવાહ હોલ્ડિંગ વેલ્યુથી નીચે પડે ત્યાં
  સુધી વહન ચાલુ રાખે છે
\item
  \textbf{ઉચ્ચ વિશ્વસનીયતા}: હલનચલન ભાગોની ગેરહાજરીને કારણે કોઈ યાંત્રિક ઘસારો
  નથી
\end{itemize}

\textbf{યાદ રાખવા માટે સૂત્ર:} ``સેમિકન્ડક્ટર સ્વિચિંગ ચાલતા લોડને નિયંત્રિત કરે છે''

\end{solutionbox}
\subsection*{પ્રશ્ન 4(c) [7
ગુણ]}\label{q4c}

\textbf{ઇન્ડક્શન હીટિંગ અને ડાઇલેક્ટ્રિક હીટિંગના કાર્ય સિદ્ધાંતનું વર્ણન કરો પણ
ઇન્ડક્શન હીટિંગ અને ડાઇલેક્ટ્રિક હીટિંગની તુલના આપો.}

\begin{solutionbox}
બંને હીટિંગ પદ્ધતિઓ સીધા સંપર્ક વિના ગરમી ઉત્પન્ન કરવા માટે
વિદ્યુતચુંબકીય સિદ્ધાંતોનો ઉપયોગ કરે છે.

\textbf{ઇન્ડક્શન હીટિંગ ડાયાગ્રામ:}

\begin{center}
\textbf{Mermaid Diagram (Code)}
\begin{verbatim}
{Shaded}
{Highlighting}[]
graph LR
    A[AC Power] {-{-}{} B[High Frequency Generator]}
    B {-{-}{} C[Work Coil]}
    C {-{-}{} D[Magnetic Field]}
    D {-{-}{} E[Eddy Currents in Workpiece]}
    E {-{-}{} F[Heat Generation]}
{Highlighting}
{Shaded}
\end{verbatim}
\end{center}

\textbf{ડાઇલેક્ટ્રિક હીટિંગ ડાયાગ્રામ:}

\begin{center}
\textbf{Mermaid Diagram (Code)}
\begin{verbatim}
{Shaded}
{Highlighting}[]
graph LR
    A[RF Generator] {-{-}{} B[Applicator Plates]}
    B {-{-}{} C[Electric Field]}
    C {-{-}{} D[Molecular Friction in Material]}
    D {-{-}{} E[Heat Generation]}
{Highlighting}
{Shaded}
\end{verbatim}
\end{center}

\textbf{તુલના કોષ્ટક:}

{\def\LTcaptype{none} % do not increment counter
\begin{longtable}[]{@{}
  >{\raggedright\arraybackslash}p{(\linewidth - 4\tabcolsep) * \real{0.2245}}
  >{\raggedright\arraybackslash}p{(\linewidth - 4\tabcolsep) * \real{0.3878}}
  >{\raggedright\arraybackslash}p{(\linewidth - 4\tabcolsep) * \real{0.3878}}@{}}
\toprule\noalign{}
\begin{minipage}[b]{\linewidth}\raggedright
પેરામીટર
\end{minipage} & \begin{minipage}[b]{\linewidth}\raggedright
ઇન્ડક્શન હીટિંગ
\end{minipage} & \begin{minipage}[b]{\linewidth}\raggedright
ડાઇલેક્ટ્રિક હીટિંગ
\end{minipage} \\
\midrule\noalign{}
\endhead
\bottomrule\noalign{}
\endlastfoot
સિદ્ધાંત & એડી કરંટ અને હિસ્ટેરેસિસ & દોલન ક્ષેત્રથી અણુ ઘર્ષણ \\
સામગ્રી & વાહક ધાતુઓ & અવાહક સામગ્રી (પ્લાસ્ટિક, લાકડું) \\
આવૃત્તિ & 1-100 kHz & 10-100 MHz \\
પ્રવેશ & સપાટી અને છીછરી ઊંડાઈ & સામગ્રી દ્વારા એકસરખું \\
કાર્યક્ષમતા & 80-90\% & 50-70\% \\
ઉપયોગો & ધાતુ હાર્ડનિંગ, ઓગાળવું, ફોર્જિંગ & પ્લાસ્ટિક વેલ્ડિંગ, ફૂડ પ્રોસેસિંગ, સૂકવવું \\
\end{longtable}
}

\begin{itemize}
\tightlist
\item
  \textbf{ઇન્ડક્શન હીટિંગ}: વાહક સામગ્રીમાં એડી કરંટ બનાવતા વિદ્યુતચુંબકીય પ્રેરણ
  દ્વારા કાર્ય કરે છે
\item
  \textbf{ડાઇલેક્ટ્રિક હીટિંગ}: પોલર અણુઓના ઝડપી દોલનનું કારણ બને છે જે આંતરિક ઘર્ષણ
  અને ગરમી પેદા કરે છે
\end{itemize}

\textbf{યાદ રાખવા માટે સૂત્ર:} ``ઇન્ડક્શન ધાતુઓને ગરમ કરે છે, ડાઇલેક્ટ્રિક્સ
બિન-ધાતુઓને ગરમ કરે છે''

\end{solutionbox}
\subsection*{પ્રશ્ન 4(a) OR [3
ગુણ]}\label{q4a}

\textbf{ફોટો ડાયોડનો ઉપયોગ કરીને ફોટો ઇલેક્ટ્રિક રિલેના સર્કિટ ડાયાગ્રામ દોરો
અને સમજાવો.}

\begin{solutionbox}
ફોટો-ઇલેક્ટ્રિક રિલે આપમેળે સ્વિચિંગ ઓપરેશન નિયંત્રિત કરવા માટે પ્રકાશ
શોધનો ઉપયોગ કરે છે.

\textbf{સર્કિટ ડાયાગ્રામ:}

\begin{verbatim}
    +Vcc
      |
      R1
      |
      |{-{-}{-}{-}{-}{-}{-}+}
      |       |
     {-{-}{-}      |}
     / {      |}
    Photo    \_|\_
    Diode    | |\_
     {-{-}{-}     |/  |}
      |      |   |
      |      |{  |}
      |      |   |
      +{-{-}{-}{-}{-}{-}+   |}
      |          |
      R2         |
      |          |
     {-{-}{-}         |       +Vcc}
     GND         |        |
                 |       \_|\_
                 +{-{-}{-}{-}{-}{-}{-}|  |}
                         | /
                         |{}
                         |/
                         |
                         Z
                         Z Load
                         |
                        {-{-}{-}}
                        GND
\end{verbatim}

\textbf{ઓપરેશન કોષ્ટક:}

{\def\LTcaptype{none} % do not increment counter
\begin{longtable}[]{@{}llll@{}}
\toprule\noalign{}
પ્રકાશ સ્થિતિ & ફોટોડાયોડ સ્થિતિ & ટ્રાન્ઝિસ્ટર સ્થિતિ & રિલે એક્શન \\
\midrule\noalign{}
\endhead
\bottomrule\noalign{}
\endlastfoot
અંધારું & ઉચ્ચ અવરોધ & બંધ & ડી-એનર્જાઇઝ્ડ \\
પ્રકાશ & ઓછો અવરોધ (વહન કરે છે) & ચાલુ & એનર્જાઇઝ્ડ \\
\end{longtable}
}

\begin{itemize}
\tightlist
\item
  \textbf{પ્રકાશ શોધ}: પ્રકાશિત થયેલ ફોટોડાયોડ વહન કરે છે, ટ્રાન્ઝિસ્ટર પર બાયસ
  બદલે છે
\item
  \textbf{સ્વિચિંગ}: ટ્રાન્ઝિસ્ટર રિલે કોઇલ ચલાવવા માટે નાના ફોટોડાયોડ પ્રવાહને
  વધારે છે
\end{itemize}

\textbf{યાદ રાખવા માટે સૂત્ર:} ``પ્રકાશ ડાયોડને ચલાવે છે, ડાયોડ ટ્રાન્ઝિસ્ટરને ચલાવે
છે, ટ્રાન્ઝિસ્ટર રિલેને ચલાવે છે''

\end{solutionbox}
\subsection*{પ્રશ્ન 4(b) OR [4
ગુણ]}\label{q4b}

\textbf{DIAC-TRIAC નો ઉપયોગ કરીને AC પાવર કંટ્રોલનો સર્કિટ ડાયાગ્રામ દોરો અને
તેને સમજાવો.}

\begin{solutionbox}
DIAC-TRIAC સર્કિટ ફેઝ એંગલ એડજસ્ટમેન્ટ દ્વારા AC પાવરને સરળ રીતે
નિયંત્રિત કરવા દે છે.

\textbf{સર્કિટ ડાયાગ્રામ:}

\begin{verbatim}
    AC      R1      DIAC
    {       www     \_|\_}
    |       |       / {}
    |{-{-}{-}{-}{-}{-}{-}|{-}{-}{-}{-}{-}{-}{-}|{-}{-}{-}{-}{-}{-}{-}{-}.}
    |       |       {\_/      |}
    |       |        |       |
    |       |        |       |
    |       C        |      \_|\_
    |       |        |     |   |
    |       |        |     |   |
    |       |        {{-}{-}{-}{-}{-}|GT |}
    |       |              |   |
    |       |              |\_\_\_|
    |       |                | TRIAC
    |       |                |
    Z       |                |
    Z Load  |                |
    |       |                |
    |{-{-}{-}{-}{-}{-}{-}+{-}{-}{-}{-}{-}{-}{-}{-}{-}{-}{-}{-}{-}{-}{-}{-}}
    |
   {-{-}{-}}
   GND
\end{verbatim}

\textbf{ઓપરેશન કોષ્ટક:}

{\def\LTcaptype{none} % do not increment counter
\begin{longtable}[]{@{}ll@{}}
\toprule\noalign{}
ઘટક & કાર્ય \\
\midrule\noalign{}
\endhead
\bottomrule\noalign{}
\endlastfoot
R1-C & ફેઝ વિલંબ માટે વેરિએબલ ટાઇમ કોન્સ્ટન્ટ \\
DIAC & કેપેસિટર વોલ્ટેજ બ્રેકઓવર પહોંચે ત્યારે TRIAC ટ્રિગર કરે છે \\
TRIAC & ટ્રિગરિંગ પોઇન્ટ પર આધારિત લોડ કરંટ નિયંત્રિત કરે છે \\
લોડ & ફેઝ કંટ્રોલ પર આધારિત આંશિક AC વેવફોર્મ પ્રાપ્ત કરે છે \\
\end{longtable}
}

\begin{itemize}
\tightlist
\item
  \textbf{ફેઝ કંટ્રોલ}: RC નેટવર્ક AC સાયકલની અંદર ટ્રિગરિંગ પોઇન્ટમાં વિલંબ બનાવે છે
\item
  \textbf{દ્વિદિશીય ઓપરેશન}: AC સાયકલના બંને અર્ધ પર કામ કરે છે
\end{itemize}

\textbf{યાદ રાખવા માટે સૂત્ર:} ``વિલંબ કેપેસિટર પર શરૂ થાય છે, વિશ્વસનીય સ્વતંત્ર AC
કંટ્રોલ ટ્રિગર કરે છે''

\end{solutionbox}
\subsection*{પ્રશ્ન 4(c) OR [7
ગુણ]}\label{q4c}

\textbf{વેવફોર્મ સાથે કામ કરતા IC555 ત્રણ તબક્કાના ક્રમિક ટાઈમરને સમજાવો.}

\begin{solutionbox}
ત્રણ-તબક્કાનો ક્રમિક ટાઇમર પ્રક્રિયા નિયંત્રણ માટે સમયબદ્ધ ક્રમ
બનાવવા માટે બહુવિધ 555 ICનો ઉપયોગ કરે છે.

\textbf{સર્કિટ ડાયાગ્રામ:}

\begin{center}
\textbf{Mermaid Diagram (Code)}
\begin{verbatim}
{Shaded}
{Highlighting}[]
graph LR
    T[Trigger] {-{-}{} IC1[555 Timer 1]}
    IC1 {-{-}{} O1[Output 1]}
    IC1 {-{-}{} D1[Delay]}
    D1 {-{-}{} IC2[555 Timer 2]}
    IC2 {-{-}{} O2[Output 2]}
    IC2 {-{-}{} D2[Delay]}
    D2 {-{-}{} IC3[555 Timer 3]}
    IC3 {-{-}{} O3[Output 3]}
    IC3 {-{-}{} R[Reset]}
    R {-.{-}{} IC1}
{Highlighting}
{Shaded}
\end{verbatim}
\end{center}

\textbf{વેવફોર્મ:}

\begin{verbatim}
Trigger
   \_
\_\_|  |\_\_\_\_\_\_\_\_\_\_\_\_\_\_\_\_\_\_\_\_\_\_\_\_\_\_\_
   
Output 1
    \_\_\_\_\_\_\_\_\_\_\_\_
\_\_\_|            |\_\_\_\_\_\_\_\_\_\_\_\_\_\_\_\_
             
Output 2
              \_\_\_\_\_\_\_\_\_\_\_\_
\_\_\_\_\_\_\_\_\_\_\_\_\_|            |\_\_\_\_\_\_
                       
Output 3
                        \_\_\_\_\_\_\_\_
\_\_\_\_\_\_\_\_\_\_\_\_\_\_\_\_\_\_\_\_\_\_\_|        |
   
   {{-}T1{-}|{-}{-}T2{-}{-}|{-}{-}T3{-}{-}|{-}T4{-}}
\end{verbatim}

\textbf{ક્રમિક ઓપરેશન કોષ્ટક:}

{\def\LTcaptype{none} % do not increment counter
\begin{longtable}[]{@{}llll@{}}
\toprule\noalign{}
તબક્કો & ક્રિયા & અવધિ & આગલા તબક્કા ટ્રિગર \\
\midrule\noalign{}
\endhead
\bottomrule\noalign{}
\endlastfoot
પ્રારંભિક & બધા આઉટપુટ્સ LOW & - & બાહ્ય ટ્રિગર \\
તબક્કો 1 & આઉટપુટ 1 HIGH & T1 (R1\timesC1) & આઉટપુટ 1 ફોલિંગ એજ \\
તબક્કો 2 & આઉટપુટ 2 HIGH & T2 (R2\timesC2) & આઉટપુટ 2 ફોલિંગ એજ \\
તબક્કો 3 & આઉટપુટ 3 HIGH & T3 (R3\timesC3) & આઉટપુટ 3 ફોલિંગ એજ \\
રીસેટ & બધા આઉટપુટ્સ LOW & T4 (રીસેટ સમય) & નવો બાહ્ય ટ્રિગર \\
\end{longtable}
}

\begin{itemize}
\tightlist
\item
  \textbf{કેસ્કેડિંગ કનેક્શન}: પહેલા ટાઇમરનો આઉટપુટ બીજાને ટ્રિગર કરે છે, અને આ રીતે
  આગળ વધે છે
\item
  \textbf{ટાઇમિંગ કંટ્રોલ}: RC વેલ્યુ સાથે દરેક તબક્કાનો સમયગાળો સ્વતંત્ર રીતે
  સમાયોજિત કરી શકાય છે
\item
  \textbf{ઉપયોગો}: ઔદ્યોગિક સિક્વન્સિંગ, પ્રક્રિયા નિયંત્રણ, સ્વચાલિત સિસ્ટમ
\end{itemize}

\textbf{યાદ રાખવા માટે સૂત્ર:} ``પ્રથમ તબક્કો સમાપ્ત થાય, બીજો શરૂ થાય, ત્રીજો
અનુસરે''

\end{solutionbox}
\subsection*{પ્રશ્ન 5(a) [3
ગુણ]}\label{q5a}

\textbf{ડીસી શંટ મોટરના સોલિડ સ્ટેટ કંટ્રોલ દોરો અને સમજાવો.}

\begin{solutionbox}
સોલિડ-સ્ટેટ DC મોટર કંટ્રોલ મોટરને આપવામાં આવતા વોલ્ટેજને નિયંત્રિત
કરવા માટે SCRનો ઉપયોગ કરે છે.

\textbf{સર્કિટ ડાયાગ્રામ:}

\begin{verbatim}
    AC      Bridge      SCR
    {       Rect      \_\_\_|\_\_\_}
    |       \_\_\_\_     |       |
    |{-{-}{-}{-}{-}{-}|\_\_\_\_|{-}{-}{-}{-}|       |{-}{-}{-}{-}.}
    |                |\_\_\_\_\_\_\_|    |
    |                    |        |
    |                   \_|\_       |
    |                   { /       |}
    |                   Gate      |
    |                 Circuit     |
    |                    |        |
    |                    |        |
    |       Field        |        |
    |       Winding      |     \_\_\_|\_\_\_
    |      \_\_\_\_\_\_        |    |       |
    |     |      |       |    | DC    |
    |{-{-}{-}{-}{-}|\_\_\_\_\_\_|{-}{-}{-}{-}{-}{-}{-}+{-}{-}{-}{-}| Motor |}
    |                         |\_\_\_\_\_\_\_|
    |                            |
   {-{-}{-}                          {-}{-}{-}}
   GND                          GND
\end{verbatim}

\textbf{કંટ્રોલ પદ્ધતિ કોષ્ટક:}

{\def\LTcaptype{none} % do not increment counter
\begin{longtable}[]{@{}lll@{}}
\toprule\noalign{}
પદ્ધતિ & ઓપરેશન & ફાયદો \\
\midrule\noalign{}
\endhead
\bottomrule\noalign{}
\endlastfoot
ફેઝ કંટ્રોલ & SCR ફાયરિંગ એંગલ બદલે છે & સરળ ગતિ નિયંત્રણ \\
ચોપર કંટ્રોલ & પલ્સ વિડ્થ મોડ્યુલેશન & ઉચ્ચ કાર્યક્ષમતા \\
ક્લોઝ્ડ-લૂપ & ટેકોમીટરથી ફીડબેક & સચોટ ગતિ નિયમન \\
\end{longtable}
}

\begin{itemize}
\tightlist
\item
  \textbf{ગતિ નિયમન}: મોટરની ગતિ બદલવા માટે આર્મેચર વોલ્ટેજ નિયંત્રિત કરે છે
\item
  \textbf{ટોર્ક કંટ્રોલ}: કરંટ મર્યાદિત કરીને ઉચ્ચ સ્ટાર્ટિંગ ટોર્ક જાળવે છે
\end{itemize}

\textbf{યાદ રાખવા માટે સૂત્ર:} ``SCR પ્રવાહ નિયંત્રિત કરે છે મોટર પાવર વિતરણ
માટે''

\end{solutionbox}
\subsection*{પ્રશ્ન 5(b) [4
ગુણ]}\label{q5b}

\textbf{સ્ટેપર મોટરના કામના સિદ્ધાંતને સમજાવો.}

\begin{solutionbox}
સ્ટેપર મોટર્સ વિદ્યુતચુંબકીય સિદ્ધાંતો દ્વારા ડિજિટલ પલ્સને ચોક્કસ
યાંત્રિક ફેરફારમાં રૂપાંતરિત કરે છે.

\textbf{સ્ટેપર મોટર સ્ટ્રક્ચર:}

\begin{center}
\textbf{Mermaid Diagram (Code)}
\begin{verbatim}
{Shaded}
{Highlighting}[]
graph LR
    C[Controller] {-{-}{} D[Driver]}
    D {-{-}{} P[Phase Windings]}
    P {-{-}{} R[Rotor Movement]}
{Highlighting}
{Shaded}
\end{verbatim}
\end{center}

\textbf{ઓપરેશન સિદ્ધાંત કોષ્ટક:}

{\def\LTcaptype{none} % do not increment counter
\begin{longtable}[]{@{}lll@{}}
\toprule\noalign{}
સ્ટેપ પ્રકાર & રોટેશન એંગલ & કંટ્રોલ પદ્ધતિ \\
\midrule\noalign{}
\endhead
\bottomrule\noalign{}
\endlastfoot
ફુલ સ્ટેપ & સામાન્ય રીતે 1.8^\circ કે 0.9^\circ & એક સમયે એક ફેઝ \\
હાફ સ્ટેપ & ફુલ સ્ટેપનો અર્ધો & બે ફેઝ વૈકલ્પિક \\
માઇક્રો-સ્ટેપ & ફુલ સ્ટેપનો અંશ & PWM કરંટ કંટ્રોલ \\
વેવ ડ્રાઇવ & ફુલ સ્ટેપ એંગલ & એક ફેઝ એનર્જાઇઝ્ડ \\
\end{longtable}
}

\begin{itemize}
\tightlist
\item
  \textbf{ડિજિટલ પોઝિશનિંગ}: દરેક પલ્સ મોટરને ચોક્કસ ખૂણે ફેરવે છે
\item
  \textbf{હોલ્ડિંગ ટોર્ક}: ફેરફાર વિના સ્થિતિ જાળવે છે જ્યારે એનર્જાઇઝ્ડ હોય
\end{itemize}

\textbf{યાદ રાખવા માટે સૂત્ર:} ``પલ્સ ચોક્કસ સ્થિતિગત સ્ટેપ્સ ઉત્પન્ન કરે છે''

\end{solutionbox}
\subsection*{પ્રશ્ન 5(c) [7
ગુણ]}\label{q5c}

\textbf{PLC ના બ્લોક ડાયાગ્રામ દોરો અને દરેક બ્લોકનું કાર્ય સમજાવો.}

\begin{solutionbox}
પ્રોગ્રામેબલ લોજિક કંટ્રોલર (PLC) એ ઓટોમેશન કંટ્રોલ માટેનું ઔદ્યોગિક
ડિજિટલ કમ્પ્યુટર છે.

\textbf{PLC બ્લોક ડાયાગ્રામ:}

\begin{center}
\textbf{Mermaid Diagram (Code)}
\begin{verbatim}
{Shaded}
{Highlighting}[]
graph TD
    P[Power Supply] {-{-}{} CPU[Central Processing Unit]}
    CPU {-{-}{} M[Memory]}
    CPU {-{-}{} I[Input Module]}
    CPU {-{-}{} O[Output Module]}
    I {-{-}{} S[Input Sensors/Switches]}
    O {-{-}{} A[Actuators/Motors]}
    CPU {-{-}{} C[Communication Module]}
    CPU {-{-}{} P[Programming Device]}
{Highlighting}
{Shaded}
\end{verbatim}
\end{center}

\textbf{PLC ઘટકો કોષ્ટક:}

{\def\LTcaptype{none} % do not increment counter
\begin{longtable}[]{@{}
  >{\raggedright\arraybackslash}p{(\linewidth - 2\tabcolsep) * \real{0.5238}}
  >{\raggedright\arraybackslash}p{(\linewidth - 2\tabcolsep) * \real{0.4762}}@{}}
\toprule\noalign{}
\begin{minipage}[b]{\linewidth}\raggedright
ઘટક
\end{minipage} & \begin{minipage}[b]{\linewidth}\raggedright
કાર્ય
\end{minipage} \\
\midrule\noalign{}
\endhead
\bottomrule\noalign{}
\endlastfoot
પાવર સપ્લાય & મુખ્ય પાવરને PLC માટે જરૂરી DC માં રૂપાંતરિત કરે છે \\
CPU & પ્રોગ્રામ ચલાવે છે અને I/O પર આધારિત નિર્ણયો કરે છે \\
મેમરી & પ્રોગ્રામ અને ડેટા સંગ્રહિત કરે છે (ROM, RAM, EEPROM) \\
ઇનપુટ મોડ્યુલ & સેન્સર, સ્વિચ, એન્કોડર સાથે ઇન્ટરફેસ કરે છે \\
આઉટપુટ મોડ્યુલ & એક્ચુએટર, મોટર, વાલ્વ, ઇન્ડિકેટર નિયંત્રિત કરે છે \\
કમ્યુનિકેશન મોડ્યુલ & અન્ય PLC, કમ્પ્યુટર, નેટવર્ક સાથે જોડાય છે \\
પ્રોગ્રામિંગ ડિવાઇસ & PLC પ્રોગ્રામ લખવા, એડિટ કરવા, મોનિટર કરવા માટે વપરાય
છે \\
\end{longtable}
}

\begin{itemize}
\tightlist
\item
  \textbf{સ્કેન સાયકલ}: સતત ઇનપુટ વાંચે છે, પ્રોગ્રામ ચલાવે છે, આઉટપુટ અપડેટ કરે છે
\item
  \textbf{પ્રોગ્રામિંગ ભાષાઓ}: લેડર લોજિક, ફંક્શન બ્લોક, સ્ટ્રક્ચર્ડ ટેક્સ્ટ, વગેરે
\item
  \textbf{ફાયદાઓ}: વિશ્વસનીયતા, લચીલાપણું, વિસ્તરણશીલતા, નિદાન ક્ષમતાઓ
\end{itemize}

\textbf{યાદ રાખવા માટે સૂત્ર:} ``પાવર પ્રોસેસિંગને કેન્દ્રિત કરે છે, ઇનપુટ/આઉટપુટ
ઓટોમેશન બનાવે છે''

\end{solutionbox}
\subsection*{પ્રશ્ન 5(a) OR [3
ગુણ]}\label{q5a}

\textbf{ડીસી સર્વો મોટરનું બાંધકામ દોરો અને સમજાવો.}

\begin{solutionbox}
DC સર્વો મોટર્સ ઓટોમેશન અને રોબોટિક્સ માટે ફીડબેક સાથે ચોક્કસ
પોઝિશન કંટ્રોલ પ્રદાન કરે છે.

\textbf{બાંધકામ ડાયાગ્રામ:}

\begin{verbatim}
      Feedback
      Device
        |
        v
    .{-{-}{-}{-}{-}{-}{-}{-}.      Shaft}
    |        |{-{-}{-}{-}{-}{-}{-}{-}{-}{-}}
    |        |
    | Motor  |
    |        |
    |        |
    {{-}{-}{-}{-}{-}{-}{-}{-}}
        \^{}
        |
     Control
     Signal
\end{verbatim}

\textbf{બાંધકામ કોષ્ટક:}

{\def\LTcaptype{none} % do not increment counter
\begin{longtable}[]{@{}ll@{}}
\toprule\noalign{}
ઘટક & કાર્ય \\
\midrule\noalign{}
\endhead
\bottomrule\noalign{}
\endlastfoot
આર્મેચર & ચુંબકીય ક્ષેત્રની અંદર ફરે છે \\
ફીલ્ડ મેગ્નેટ્સ & ચુંબકીય ક્ષેત્ર બનાવે છે (ઘણીવાર કાયમી ચુંબક) \\
કમ્યુટેટર & ફરતા આર્મેચરને પાવર ટ્રાન્સફર કરે છે \\
ફીડબેક ડિવાઇસ & પોઝિશન/સ્પીડ ફીડબેક માટે એન્કોડર/ટેકોમીટર \\
બ્રશ & કમ્યુટેટરને પાવર કનેક્ટ કરે છે \\
\end{longtable}
}

\begin{itemize}
\tightlist
\item
  \textbf{ઓછી જડતા}: ખાસ ડિઝાઇન ઝડપી એક્સેલરેશન/ડિસેલરેશનની મંજૂરી આપે છે
\item
  \textbf{ઉચ્ચ ટોર્ક-ટુ-ઇનર્શિયા રેશિઓ}: કંટ્રોલ સિગ્નલનો ઝડપથી જવાબ આપે છે
\end{itemize}

\textbf{યાદ રાખવા માટે સૂત્ર:} ``ચોક્સાઈભર્યું પોઝિશન ફીડબેક સટીક નિયંત્રણ ચલાવે
છે''

\end{solutionbox}
\subsection*{પ્રશ્ન 5(b) OR [4
ગુણ]}\label{q5b}

\textbf{BLDC મોટરની કામગીરી સમજાવો.}

\begin{solutionbox}
બ્રશલેસ DC (BLDC) મોટર્સ યાંત્રિક બ્રશ અને કમ્યુટેટરને બદલે ઇલેક્ટ્રોનિક
કમ્યુટેશનનો ઉપયોગ કરે છે.

\textbf{BLDC ઓપરેશન ડાયાગ્રામ:}

\begin{center}
\textbf{Mermaid Diagram (Code)}
\begin{verbatim}
{Shaded}
{Highlighting}[]
graph LR
    PS[Power Supply] {-{-}{} C[Controller]}
    C {-{-}{} D[Driver Circuit]}
    D {-{-}{} W[Stator Windings]}
    HS[Hall Sensors] {-{-}{} C}
    W {-{-}{} R[Rotor Rotation]}
    R {-{-}{} HS}
{Highlighting}
{Shaded}
\end{verbatim}
\end{center}

\textbf{કામગીરી સિદ્ધાંત કોષ્ટક:}

{\def\LTcaptype{none} % do not increment counter
\begin{longtable}[]{@{}ll@{}}
\toprule\noalign{}
ઘટક & કાર્ય \\
\midrule\noalign{}
\endhead
\bottomrule\noalign{}
\endlastfoot
સ્ટેટર & ફિક્સ્ડ વાઇન્ડિંગ્સ જે ફરતું ચુંબકીય ક્ષેત્ર ઉત્પન્ન કરે છે \\
રોટર & કાયમી ચુંબક જે ફરતા ક્ષેત્રને અનુસરે છે \\
ઇલેક્ટ્રોનિક કંટ્રોલર & યાંત્રિક કમ્યુટેશનનું સ્થાન લે છે \\
હોલ સેન્સર & સિન્ક્રોનાઇઝ્ડ સ્વિચિંગ માટે રોટર પોઝિશન શોધે છે \\
ડ્રાઇવર સર્કિટ & સ્ટેટર કોઇલ્સમાં પ્રવાહનો ક્રમ પ્રદાન કરે છે \\
\end{longtable}
}

\begin{itemize}
\tightlist
\item
  \textbf{કમ્યુટેશન}: ઇલેક્ટ્રોનિક સ્વિચિંગ સિક્વન્સ સ્ટેટર વાઇન્ડિંગ્સમાં પાવર આપે છે
\item
  \textbf{કાર્યક્ષમતા}: બ્રશ લોસિસના નિર્મૂલનને કારણે ઉચ્ચ કાર્યક્ષમતા
\item
  \textbf{વિશ્વસનીયતા}: બ્રશનો ઘસારો કે સ્પાર્કિંગ નથી, લાંબુ આયુષ્ય
\end{itemize}

\textbf{યાદ રાખવા માટે સૂત્ર:} ``ઇલેક્ટ્રોનિક સ્વિચિંગ બ્રશ વગર ફેરફાર બનાવે છે''

\end{solutionbox}
\subsection*{પ્રશ્ન 5(c) OR [7
ગુણ]}\label{q5c}

\textbf{VFD નું બાંધકામ અને કાર્ય સમજાવો.}

\begin{solutionbox}
વેરિએબલ ફ્રિક્વન્સી ડ્રાઇવ (VFD) આવૃત્તિ અને વોલ્ટેજમાં ફેરફાર કરીને
AC મોટરની ગતિ નિયંત્રિત કરે છે.

\textbf{VFD બાંધકામ ડાયાગ્રામ:}

\begin{center}
\textbf{Mermaid Diagram (Code)}
\begin{verbatim}
{Shaded}
{Highlighting}[]
graph LR
    A[AC Input] {-{-}{} R[Rectifier]}
    R {-{-}{} D[DC Bus/Filter]}
    D {-{-}{} I[Inverter]}
    I {-{-}{} M[Motor]}
    C[Control Circuit] {-{-}{} I}
    F[Feedback] {-{-}{} C}
{Highlighting}
{Shaded}
\end{verbatim}
\end{center}

\textbf{બાંધકામ અને કામગીરી કોષ્ટક:}

{\def\LTcaptype{none} % do not increment counter
\begin{longtable}[]{@{}lll@{}}
\toprule\noalign{}
વિભાગ & ઘટકો & કાર્ય \\
\midrule\noalign{}
\endhead
\bottomrule\noalign{}
\endlastfoot
રેક્ટિફાયર & ડાયોડ/SCRs & AC ને DC માં રૂપાંતરિત કરે છે \\
DC બસ & કેપેસિટર, ઇન્ડક્ટર & DC ને ફિલ્ટર અને સ્મૂધ કરે છે \\
ઇન્વર્ટર & IGBTs/ટ્રાન્ઝિસ્ટર & DC ને ચલિત આવૃત્તિ AC માં રૂપાંતરિત કરે છે \\
કંટ્રોલ સર્કિટ & માઇક્રોપ્રોસેસર & સ્વિચિંગ આવૃત્તિ અને પેટર્નને નિયંત્રિત કરે છે \\
કૂલિંગ સિસ્ટમ & ફેન, હીટ સિંક & સુરક્ષિત ઓપરેટિંગ તાપમાન જાળવે છે \\
પ્રોટેક્શન સર્કિટ & સેન્સર, રિલે & ફોલ્ટથી નુકસાન અટકાવે છે \\
\end{longtable}
}

\begin{itemize}
\tightlist
\item
  \textbf{ગતિ નિયંત્રણ}: સતત ટોર્ક પ્રદાન કરવા માટે V/f રેશિઓ જાળવવામાં આવે છે
\item
  \textbf{ઊર્જા બચત}: વાસ્તવિક લોડ જરૂરિયાતો અનુસાર પાવર સમાયોજિત કરે છે
\item
  \textbf{સોફ્ટ સ્ટાર્ટ}: ક્રમશઃ એક્સેલરેશન યાંત્રિક આઘાતને અટકાવે છે
\end{itemize}

\textbf{યાદ રાખવા માટે સૂત્ર:} ``રેક્ટિફાય, ફિલ્ટર, મોટર કંટ્રોલ માટે આવૃત્તિ
બદલો''

\end{solutionbox}

\end{document}
