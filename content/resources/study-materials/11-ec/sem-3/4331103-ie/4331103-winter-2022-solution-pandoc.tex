\documentclass[10pt,a4paper]{article}

% content/resources/templates/preamble.tex
\usepackage[margin=0.6in]{geometry}
\author{Milav Dabgar}
\usepackage{amsmath,amssymb,amsthm}
\usepackage{booktabs}
\usepackage{multirow}
\usepackage{xcolor}
\usepackage{tcolorbox}
\tcbuselibrary{breakable,skins}
\usepackage[colorlinks=true,linkcolor=blue]{hyperref}
\usepackage{titlesec}
\usepackage{enumitem}
\usepackage{tikz}
\usepackage{pgfplots}
\usepackage{circuitikz}
\usepackage[version=4]{mhchem}
\usepackage{longtable}
\usepackage{array}
\usepackage{float}
\usepackage{caption}
\usepackage{listings}

\lstset{
  basicstyle=\small\ttfamily,
  breaklines=true,
  breakatwhitespace=false,
  postbreak=\mbox{\textcolor{red}{$\hookrightarrow$}\space},
  float=false,
  numbers=left,
  numberstyle=\tiny\color{gray},
  numbersep=10pt,
  xleftmargin=2em,
  keywordstyle=\color{blue},
  commentstyle=\color{green!60!black},
  stringstyle=\color{purple},
  backgroundcolor=\color{gray!5},
  showstringspaces=false,
  tabsize=2,
  captionpos=b,
  keepspaces=true,
  columns=flexible
}

\pgfplotsset{compat=1.18}
\usetikzlibrary{shapes,arrows,positioning,calc,patterns,decorations.pathmorphing,decorations.markings,arrows.meta}

% Color scheme
\definecolor{headcolor}{RGB}{0,102,204}
\definecolor{keycolor}{RGB}{220,20,60}
\definecolor{solutioncolor}{RGB}{34,139,34}
\definecolor{mnemoniccolor}{RGB}{148,0,211}
\definecolor{codecolor}{RGB}{0,0,100}

% Spacing
\setlength{\parskip}{3pt}
\setlist[itemize]{nosep}
\setlist[enumerate]{nosep}

% Title formatting
\titleformat{\section}{\Large\bfseries\color{headcolor}}{\thesection}{1em}{}
\titleformat{\subsection}{\large\bfseries\color{headcolor}}{\thesubsection}{1em}{}

% Pandoc tightlist compatibility
\providecommand{\tightlist}{%
  \setlength{\itemsep}{0pt}\setlength{\parskip}{0pt}}

% Pandoc longtable compatibility
\newcounter{none}
\def\thenone{}


% content/resources/templates/english-boxes.tex
% This file is currently empty - it exists to maintain consistency with the import structure.
% Add custom environments here if needed in the future.


\begin{document}

\begin{center}
{\Huge\bfseries\color{headcolor} Subject Name Solutions}\\[5pt]
{\LARGE 4331103 -- Winter 2022}\\[3pt]
{\large Semester 1 Study Material}\\[3pt]
{\normalsize\textit{Detailed Solutions and Explanations}}
\end{center}

\vspace{10pt}

\subsection*{Question 1(a) [3 marks]}\label{q1a}

\textbf{Draw the construction of SCR and explain it.}

\begin{solutionbox}
SCR (Silicon Controlled Rectifier) is a four-layer PNPN
semiconductor device with three terminals: Anode, Cathode, and Gate.

\textbf{Diagram:}

\begin{center}
\textbf{Mermaid Diagram (Code)}
\begin{verbatim}
{Shaded}
{Highlighting}[]
graph LR
    A[Anode] {-{-}{-} P1[P{-}layer]}
    P1 {-{-}{-} N1[N{-}layer]}
    N1 {-{-}{-} P2[P{-}layer]}
    P2 {-{-}{-} N2[N{-}layer]}
    N2 {-{-}{-} K[Cathode]}
    G[Gate] {-{-}{-} P2}
{Highlighting}
{Shaded}
\end{verbatim}
\end{center}

\begin{itemize}
\tightlist
\item
  \textbf{P-N-P-N Layers}: Four alternating semiconductor layers
\item
  \textbf{Gate Terminal}: Controls turn-on of the device
\item
  \textbf{Current Flow}: Anode to cathode when triggered
\end{itemize}

\end{solutionbox}
\begin{mnemonicbox}
``Silicon Controls Rectification'' - SCR controls
current flow in one direction only when triggered.

\end{mnemonicbox}
\subsection*{Question 1(b) [4 marks]}\label{q1b}

\textbf{Draw construction of TRIAC and explain it.}

\begin{solutionbox}
TRIAC (Triode for Alternating Current) is a
bidirectional three-terminal semiconductor device that conducts in both
directions when triggered.

\textbf{Diagram:}

\begin{center}
\textbf{Mermaid Diagram (Code)}
\begin{verbatim}
{Shaded}
{Highlighting}[]
graph LR
    MT1[Main Terminal 1] {-{-}{-} N1[N{-}layer]}
    N1 {-{-}{-} P1[P{-}layer]}
    P1 {-{-}{-} N2[N{-}layer]}
    N2 {-{-}{-} P2[P{-}layer]}
    P2 {-{-}{-} N3[N{-}layer]}
    N3 {-{-}{-} MT2[Main Terminal 2]}
    G[Gate] {-{-}{-} P1}
{Highlighting}
{Shaded}
\end{verbatim}
\end{center}

\begin{itemize}
\tightlist
\item
  \textbf{Bidirectional Operation}: Conducts in both directions when
  triggered
\item
  \textbf{Gate Control}: Single gate controls conduction in both
  directions
\item
  \textbf{Equivalent Circuit}: Acts like two SCRs connected in
  anti-parallel
\item
  \textbf{AC Applications}: Widely used for AC power control
  applications
\end{itemize}

\end{solutionbox}
\begin{mnemonicbox}
``TRI-direction AC controller'' - Controls current in
both directions in AC circuits.

\end{mnemonicbox}
\subsection*{Question 1(c) [7 marks]}\label{q1c}

\textbf{Describe construction \& working of Opto-Isolators, Opto-TRIAC,
Opto-SCR, and Opto-transistor. And list their applications.}

\begin{solutionbox}
Opto-isolators use light to transfer electrical signals
between isolated circuits.

\textbf{Diagram:}

\begin{center}
\textbf{Mermaid Diagram (Code)}
\begin{verbatim}
{Shaded}
{Highlighting}[]
graph LR
    subgraph Input
        LED[LED]
    end
    subgraph Output
        PD[Photo Detector]
    end
    LED {-{-} "Light" {-}{-}{} PD}
    style Input fill:\#f9f,stroke:\#333
    style Output fill:\#bbf,stroke:\#333
{Highlighting}
{Shaded}
\end{verbatim}
\end{center}

{\def\LTcaptype{none} % do not increment counter
\begin{longtable}[]{@{}
  >{\raggedright\arraybackslash}p{(\linewidth - 6\tabcolsep) * \real{0.1778}}
  >{\raggedright\arraybackslash}p{(\linewidth - 6\tabcolsep) * \real{0.3111}}
  >{\raggedright\arraybackslash}p{(\linewidth - 6\tabcolsep) * \real{0.2000}}
  >{\raggedright\arraybackslash}p{(\linewidth - 6\tabcolsep) * \real{0.3111}}@{}}
\toprule\noalign{}
\begin{minipage}[b]{\linewidth}\raggedright
Device
\end{minipage} & \begin{minipage}[b]{\linewidth}\raggedright
Construction
\end{minipage} & \begin{minipage}[b]{\linewidth}\raggedright
Working
\end{minipage} & \begin{minipage}[b]{\linewidth}\raggedright
Applications
\end{minipage} \\
\midrule\noalign{}
\endhead
\bottomrule\noalign{}
\endlastfoot
Opto-Isolator & LED + Photodetector & LED emits light when input current
flows; photodetector activates output circuit & Signal isolation,
Medical equipment, Industrial controls \\
Opto-TRIAC & LED + Photo-TRIAC & LED triggers the TRIAC through light;
provides electrical isolation & AC power control, Solid state relays,
Motor controls \\
Opto-SCR & LED + Photo-SCR & LED emits light to trigger SCR; provides
high isolation & DC switching, Industrial controls, High voltage
isolation \\
Opto-transistor & LED + Photo-transistor & LED light controls base
current of phototransistor & Encoders, Level detection, Position
sensing \\
\end{longtable}
}

\begin{itemize}
\tightlist
\item
  \textbf{Electrical Isolation}: Complete separation between input and
  output
\item
  \textbf{Noise Immunity}: High resistance to electrical noise
\item
  \textbf{Speed}: Response times in microseconds range
\end{itemize}

\end{solutionbox}
\begin{mnemonicbox}
``LOST'' - Light Operates Semiconductor Terminals in
all opto-devices.

\end{mnemonicbox}
\subsection*{Question 1(c) OR [7
marks]}\label{q1c}

\textbf{Describe Explain working of SCR using two transistor analogies.
List the various industrial applications of SCR.}

\begin{solutionbox}
SCR can be modeled as two interconnected transistors:
PNP (T1) and NPN (T2).

\textbf{Diagram:}

\begin{center}
\textbf{Mermaid Diagram (Code)}
\begin{verbatim}
{Shaded}
{Highlighting}[]
graph TD
    A[Anode] {-{-}{-} E1[Emitter T1]}
    B1[Base T1] {-{-}{-} C2[Collector T2]}
    C1[Collector T1] {-{-}{-} B2[Base T2]}
    E2[Emitter T2] {-{-}{-} K[Cathode]}
    G[Gate] {-{-}{-} B2}
{Highlighting}
{Shaded}
\end{verbatim}
\end{center}

\textbf{Working Principle:}

{\def\LTcaptype{none} % do not increment counter
\begin{longtable}[]{@{}
  >{\raggedright\arraybackslash}p{(\linewidth - 2\tabcolsep) * \real{0.3529}}
  >{\raggedright\arraybackslash}p{(\linewidth - 2\tabcolsep) * \real{0.6471}}@{}}
\toprule\noalign{}
\begin{minipage}[b]{\linewidth}\raggedright
Step
\end{minipage} & \begin{minipage}[b]{\linewidth}\raggedright
Operation
\end{minipage} \\
\midrule\noalign{}
\endhead
\bottomrule\noalign{}
\endlastfoot
Initial State & Both transistors are OFF \\
Gate Triggering & Current injected into gate (B2 of T2) \\
Regenerative Action & T2 turns ON \rightarrow T1 base gets current \rightarrow T1 turns ON \rightarrow
More current to T2 base \\
Latching & Self-sustaining current flow continues even if gate signal is
removed \\
\end{longtable}
}

\textbf{Industrial Applications of SCR:}

\begin{itemize}
\tightlist
\item
  \textbf{Power Control}: AC/DC motor speed control
\item
  \textbf{Switching}: Static switches, solid-state relays
\item
  \textbf{Inverters}: DC to AC conversion
\item
  \textbf{Protection}: Overvoltage protection circuits
\item
  \textbf{Lighting}: Light dimmers, illumination control
\end{itemize}

\end{solutionbox}
\begin{mnemonicbox}
``POWER'' - Power control, Overvoltage protection,
Welding machines, Electronic converters, Regulated supplies.

\end{mnemonicbox}
\subsection*{Question 2(a) [3 marks]}\label{q2a}

\textbf{Define Triggering in SCR and explain any two triggering
techniques.}

\begin{solutionbox}
Triggering is the process of turning ON an SCR by
applying appropriate signal to its gate terminal.

\textbf{Two Triggering Techniques:}

{\def\LTcaptype{none} % do not increment counter
\begin{longtable}[]{@{}
  >{\raggedright\arraybackslash}p{(\linewidth - 2\tabcolsep) * \real{0.4583}}
  >{\raggedright\arraybackslash}p{(\linewidth - 2\tabcolsep) * \real{0.5417}}@{}}
\toprule\noalign{}
\begin{minipage}[b]{\linewidth}\raggedright
Technique
\end{minipage} & \begin{minipage}[b]{\linewidth}\raggedright
Description
\end{minipage} \\
\midrule\noalign{}
\endhead
\bottomrule\noalign{}
\endlastfoot
Gate Triggering & Direct current pulse applied to gate-cathode
circuit \\
Light Triggering & Photons striking junction provide energy for
conduction \\
\end{longtable}
}

\begin{itemize}
\tightlist
\item
  \textbf{Gate Triggering}: Most common method using electrical pulse
\item
  \textbf{Light Triggering}: Uses photosensitive semiconductor
  properties
\end{itemize}

\end{solutionbox}
\begin{mnemonicbox}
``GET'' - Gate Electrical Triggering is the most
common method.

\end{mnemonicbox}
\subsection*{Question 2(b) [4 marks]}\label{q2b}

\textbf{Write the differences between forced commutation and natural
commutation.}

\begin{solutionbox}

{\def\LTcaptype{none} % do not increment counter
\begin{longtable}[]{@{}
  >{\raggedright\arraybackslash}p{(\linewidth - 4\tabcolsep) * \real{0.2157}}
  >{\raggedright\arraybackslash}p{(\linewidth - 4\tabcolsep) * \real{0.3725}}
  >{\raggedright\arraybackslash}p{(\linewidth - 4\tabcolsep) * \real{0.4118}}@{}}
\toprule\noalign{}
\begin{minipage}[b]{\linewidth}\raggedright
Parameter
\end{minipage} & \begin{minipage}[b]{\linewidth}\raggedright
Forced Commutation
\end{minipage} & \begin{minipage}[b]{\linewidth}\raggedright
Natural Commutation
\end{minipage} \\
\midrule\noalign{}
\endhead
\bottomrule\noalign{}
\endlastfoot
Definition & External circuitry forces SCR to turn OFF & SCR turns OFF
naturally when current falls below holding value \\
Application & DC circuits & AC circuits \\
Components & Requires additional components (capacitors, inductors) & No
additional components needed \\
Complexity & Complex circuit design & Simple circuit design \\
Energy & External energy needed for turn-off & No external energy
needed \\
\end{longtable}
}

\begin{itemize}
\tightlist
\item
  \textbf{Forced Commutation}: Actively turns OFF SCR using external
  circuit
\item
  \textbf{Natural Commutation}: SCR turns OFF when AC current crosses
  zero
\end{itemize}

\end{solutionbox}
\begin{mnemonicbox}
``FACE'' - Forced Active Commutation requires
External components.

\end{mnemonicbox}
\subsection*{Question 2(c) [7 marks]}\label{q2c}

\textbf{Design the snubber circuit for SCR.}

\begin{solutionbox}
Snubber circuit protects SCR from high dV/dt and limits
rate of voltage rise.

\textbf{Diagram:}

\begin{center}
\textbf{Mermaid Diagram (Code)}
\begin{verbatim}
{Shaded}
{Highlighting}[]
graph LR
    A[Anode] {-{-}{-} R[Resistance]}
    R {-{-}{-} C[Capacitance]}
    C {-{-}{-} K[Cathode]}
    A {-{-}{-} SCR[SCR]}
    SCR {-{-}{-} K}
{Highlighting}
{Shaded}
\end{verbatim}
\end{center}

\textbf{Design Steps:}

{\def\LTcaptype{none} % do not increment counter
\begin{longtable}[]{@{}
  >{\raggedright\arraybackslash}p{(\linewidth - 2\tabcolsep) * \real{0.3158}}
  >{\raggedright\arraybackslash}p{(\linewidth - 2\tabcolsep) * \real{0.6842}}@{}}
\toprule\noalign{}
\begin{minipage}[b]{\linewidth}\raggedright
Step
\end{minipage} & \begin{minipage}[b]{\linewidth}\raggedright
Calculation
\end{minipage} \\
\midrule\noalign{}
\endhead
\bottomrule\noalign{}
\endlastfoot
1. Calculate dV/dt rating & From datasheet (V/μs) \\
2. Determine R value & R = V_{1}/IL where V_{1} is supply voltage and IL is
load current \\
3. Determine C value & C = 1/(R \times (dV/dt)max) \\
4. RC time constant & τ = R \times C (should be greater than SCR turn-off
time) \\
\end{longtable}
}

\begin{itemize}
\tightlist
\item
  \textbf{Resistance R}: Limits discharge current of capacitor
\item
  \textbf{Capacitance C}: Absorbs transient energy and limits dV/dt
\item
  \textbf{Protection}: Prevents false triggering and damage
\item
  \textbf{Power Rating}: R must have sufficient power rating
\end{itemize}

\end{solutionbox}
\begin{mnemonicbox}
``RCSS'' - Resistance-Capacitance Saves Silicon from
Stress.

\end{mnemonicbox}
\subsection*{Question 2(a) OR [3
marks]}\label{q2a}

\textbf{Define commutation and Explain class-E commutation for SCR.}

\begin{solutionbox}
Commutation is the process of turning OFF an SCR by
reducing its anode current below the holding current level.

\textbf{Class-E Commutation:}

\textbf{Diagram:}

\begin{center}
\textbf{Mermaid Diagram (Code)}
\begin{verbatim}
{Shaded}
{Highlighting}[]
graph LR
    S[Supply] {-{-}{-} L[Load]}
    L {-{-}{-} SCR[SCR]}
    L {-{-}{-} C[Capacitor]}
    C {-{-}{-} A[Auxiliary SCR]}
    A {-{-}{-} S}
{Highlighting}
{Shaded}
\end{verbatim}
\end{center}

\begin{itemize}
\tightlist
\item
  \textbf{Auxiliary SCR}: Controls the commutation process
\item
  \textbf{Resonant Circuit}: Forms LC resonant circuit
\item
  \textbf{Operation}: Auxiliary SCR triggers capacitor discharge to
  reverse-bias main SCR
\item
  \textbf{Application}: Used in inverters and choppers
\end{itemize}

\end{solutionbox}
\begin{mnemonicbox}
``ACE'' - Auxiliary Capacitor Extinguishes
conduction.

\end{mnemonicbox}
\subsection*{Question 2(b) OR [4
marks]}\label{q2b}

\textbf{Explain Triggering of Thyristor.}

\begin{solutionbox}

{\def\LTcaptype{none} % do not increment counter
\begin{longtable}[]{@{}
  >{\raggedright\arraybackslash}p{(\linewidth - 2\tabcolsep) * \real{0.5000}}
  >{\raggedright\arraybackslash}p{(\linewidth - 2\tabcolsep) * \real{0.5000}}@{}}
\toprule\noalign{}
\begin{minipage}[b]{\linewidth}\raggedright
Triggering Method
\end{minipage} & \begin{minipage}[b]{\linewidth}\raggedright
Working Principle
\end{minipage} \\
\midrule\noalign{}
\endhead
\bottomrule\noalign{}
\endlastfoot
Gate Triggering & Electrical pulse applied between gate and cathode \\
Temperature Triggering & Junction temperature increases to cause
turn-on \\
Light Triggering & Photons create electron-hole pairs at junctions \\
dV/dt Triggering & Rapid voltage rise causes capacitive current flow \\
Forward Voltage Triggering & Exceeding breakover voltage causes
avalanche conduction \\
\end{longtable}
}

\begin{itemize}
\tightlist
\item
  \textbf{Gate Triggering}: Most common and controllable method
\item
  \textbf{Parameter Control}: Pulse width, amplitude, and rise time
\item
  \textbf{Gate Sensitivity}: Varies with temperature
\item
  \textbf{Protection}: Required against unwanted triggering
\end{itemize}

\end{solutionbox}
\begin{mnemonicbox}
``VITAL'' - Voltage, Illumination, Temperature And
Level are all triggering methods.

\end{mnemonicbox}
\subsection*{Question 2(c) OR [7
marks]}\label{q2c}

\textbf{Explain methods to protect SCR against over voltage and current
in details.}

\begin{solutionbox}

\textbf{Overvoltage Protection:}

\textbf{Diagram:}

\begin{center}
\textbf{Mermaid Diagram (Code)}
\begin{verbatim}
{Shaded}
{Highlighting}[]
graph LR
    S[Supply] {-{-}{-} F[Fuse]}
    F {-{-}{-} V[Varistor]}
    V {-{-}{-} SCR[SCR]}
    SCR {-{-}{-} L[Load]}
    V {-{-}{-} RC[RC Snubber]}
    RC {-{-}{-} SCR}
{Highlighting}
{Shaded}
\end{verbatim}
\end{center}

{\def\LTcaptype{none} % do not increment counter
\begin{longtable}[]{@{}
  >{\raggedright\arraybackslash}p{(\linewidth - 2\tabcolsep) * \real{0.5000}}
  >{\raggedright\arraybackslash}p{(\linewidth - 2\tabcolsep) * \real{0.5000}}@{}}
\toprule\noalign{}
\begin{minipage}[b]{\linewidth}\raggedright
Protection Method
\end{minipage} & \begin{minipage}[b]{\linewidth}\raggedright
Working Principle
\end{minipage} \\
\midrule\noalign{}
\endhead
\bottomrule\noalign{}
\endlastfoot
RC Snubber Circuit & Limits rate of rise of voltage (dV/dt) \\
Voltage Clamping & Using Zener diodes or MOVs to limit maximum
voltage \\
Crowbar Protection & Deliberate short-circuit when voltage exceeds
threshold \\
\end{longtable}
}

\textbf{Overcurrent Protection:}

\textbf{Diagram:}

\begin{center}
\textbf{Mermaid Diagram (Code)}
\begin{verbatim}
{Shaded}
{Highlighting}[]
graph LR
    S[Supply] {-{-}{-} F[Fuse/Circuit Breaker]}
    F {-{-}{-} R[Current Limiting Resistor]}
    R {-{-}{-} SCR[SCR]}
    SCR {-{-}{-} L[Load]}
{Highlighting}
{Shaded}
\end{verbatim}
\end{center}

{\def\LTcaptype{none} % do not increment counter
\begin{longtable}[]{@{}
  >{\raggedright\arraybackslash}p{(\linewidth - 2\tabcolsep) * \real{0.5000}}
  >{\raggedright\arraybackslash}p{(\linewidth - 2\tabcolsep) * \real{0.5000}}@{}}
\toprule\noalign{}
\begin{minipage}[b]{\linewidth}\raggedright
Protection Method
\end{minipage} & \begin{minipage}[b]{\linewidth}\raggedright
Working Principle
\end{minipage} \\
\midrule\noalign{}
\endhead
\bottomrule\noalign{}
\endlastfoot
Fuses/Circuit Breakers & Disconnects circuit during fault conditions \\
Current Limiting Reactors & Limits fault current magnitude \\
Electronic Current Limiting & Sensing and control circuits limit
current \\
\end{longtable}
}

\begin{itemize}
\tightlist
\item
  \textbf{Coordination}: Protection devices must work in coordination
\item
  \textbf{Response Time}: Critical for effective protection
\item
  \textbf{Multiple Layers}: For critical applications, several methods
  are combined
\end{itemize}

\end{solutionbox}
\begin{mnemonicbox}
``SCOPE'' - Snubbers, Clamps, Overload sensors,
Protectors, and Electronic limiters.

\end{mnemonicbox}
\subsection*{Question 3(a) [3 marks]}\label{q3a}

\textbf{List the differences between single phase rectifier and poly
phase rectifier.}

\begin{solutionbox}

{\def\LTcaptype{none} % do not increment counter
\begin{longtable}[]{@{}
  >{\raggedright\arraybackslash}p{(\linewidth - 4\tabcolsep) * \real{0.1930}}
  >{\raggedright\arraybackslash}p{(\linewidth - 4\tabcolsep) * \real{0.4211}}
  >{\raggedright\arraybackslash}p{(\linewidth - 4\tabcolsep) * \real{0.3860}}@{}}
\toprule\noalign{}
\begin{minipage}[b]{\linewidth}\raggedright
Parameter
\end{minipage} & \begin{minipage}[b]{\linewidth}\raggedright
Single Phase Rectifier
\end{minipage} & \begin{minipage}[b]{\linewidth}\raggedright
Poly Phase Rectifier
\end{minipage} \\
\midrule\noalign{}
\endhead
\bottomrule\noalign{}
\endlastfoot
Input & Single phase AC supply & Multiple phase (usually 3-phase) AC
supply \\
Output Ripple & Higher ripple content & Lower ripple content \\
Efficiency & Lower efficiency & Higher efficiency \\
Power Rating & Suitable for low power applications & Suitable for high
power applications \\
Transformer Utilization & Lower utilization factor & Higher utilization
factor \\
\end{longtable}
}

\begin{itemize}
\tightlist
\item
  \textbf{Ripple Factor}: Single phase has higher ripple compared to
  poly phase
\item
  \textbf{Form Factor}: Better in poly phase systems
\item
  \textbf{Size/Weight}: Poly phase systems have better power/weight
  ratio
\end{itemize}

\end{solutionbox}
\begin{mnemonicbox}
``PERCH'' - Poly phase has Efficiency, Ripple
improvement, Capacity, and Higher ratings.

\end{mnemonicbox}
\subsection*{Question 3(b) [4 marks]}\label{q3b}

\textbf{Draw the circuit diagram of three phases Half Wave Rectifier and
explain its Working.}

\begin{solutionbox}
Three-phase half-wave rectifier converts three-phase AC
into pulsating DC using three diodes.

\textbf{Diagram:}

\begin{center}
\textbf{Mermaid Diagram (Code)}
\begin{verbatim}
{Shaded}
{Highlighting}[]
graph TD
    A[Phase A] {-{-}{-} D1[Diode 1]}
    B[Phase B] {-{-}{-} D2[Diode 2]}
    C[Phase C] {-{-}{-} D3[Diode 3]}
    D1 {-{-}{-} O[Output +]}
    D2 {-{-}{-} O}
    D3 {-{-}{-} O}
    N[Neutral] {-{-}{-} ON[Output {-}]}
{Highlighting}
{Shaded}
\end{verbatim}
\end{center}

\textbf{Working:}

\begin{itemize}
\tightlist
\item
  Each diode conducts when its phase voltage is most positive
\item
  Conduction angle of each diode is 120^\circ
\item
  Ripple frequency is 3 times the input frequency
\item
  Average output voltage = 3Vm/2π (where Vm is peak phase voltage)
\item
  Ripple factor = 0.17 (much lower than single-phase half-wave)
\end{itemize}

\end{solutionbox}
\begin{mnemonicbox}
``THREE-D'' - THREE Diodes conducting sequentially.

\end{mnemonicbox}
\subsection*{Question 3(c) [7 marks]}\label{q3c}

\textbf{Describe the working of UPS \& SMPS with the help of block
diagram.}

\begin{solutionbox}

\textbf{UPS (Uninterruptible Power Supply):}

\textbf{Diagram:}

\begin{center}
\textbf{Mermaid Diagram (Code)}
\begin{verbatim}
{Shaded}
{Highlighting}[]
graph LR
    AC[AC Input] {-{-}{-} R[Rectifier]}
    R {-{-}{-} BC[Battery Charger]}
    BC {-{-}{-} B[Battery]}
    B {-{-}{-} I[Inverter]}
    R {-{-}{-} I}
    I {-{-}{-} F[Filter]}
    F {-{-}{-} L[Load]}
    AC {-.Bypass.{-}{} L}
{Highlighting}
{Shaded}
\end{verbatim}
\end{center}

{\def\LTcaptype{none} % do not increment counter
\begin{longtable}[]{@{}ll@{}}
\toprule\noalign{}
Block & Function \\
\midrule\noalign{}
\endhead
\bottomrule\noalign{}
\endlastfoot
Rectifier & Converts AC to DC for battery charging and inverter \\
Battery & Stores energy for backup during power failure \\
Inverter & Converts DC to AC for powering load \\
Filter & Smooths output waveform \\
Bypass & Provides direct AC during maintenance \\
\end{longtable}
}

\textbf{SMPS (Switched Mode Power Supply):}

\textbf{Diagram:}

\begin{center}
\textbf{Mermaid Diagram (Code)}
\begin{verbatim}
{Shaded}
{Highlighting}[]
graph LR
    AC[AC Input] {-{-}{-} R[Rectifier \& Filter]}
    R {-{-}{-} SW[High Frequency Switch]}
    SW {-{-}{-} T[HF Transformer]}
    T {-{-}{-} RF[Rectifier \& Filter]}
    RF {-{-}{-} L[Load]}
    FB[Feedback] {-{-}{-} SW}
    RF {-{-}{-} FB}
{Highlighting}
{Shaded}
\end{verbatim}
\end{center}

{\def\LTcaptype{none} % do not increment counter
\begin{longtable}[]{@{}ll@{}}
\toprule\noalign{}
Block & Function \\
\midrule\noalign{}
\endhead
\bottomrule\noalign{}
\endlastfoot
Rectifier \& Filter & Converts AC to unregulated DC \\
High Frequency Switch & Chops DC into high-frequency pulses \\
HF Transformer & Provides isolation and voltage transformation \\
Output Rectifier \& Filter & Converts high-frequency AC to smooth DC \\
Feedback Circuit & Regulates output voltage by controlling switch \\
\end{longtable}
}

\begin{itemize}
\tightlist
\item
  \textbf{UPS Efficiency}: 80-90\%, provides backup power
\item
  \textbf{SMPS Efficiency}: 70-90\%, much smaller than linear supplies
\item
  \textbf{Regulation}: Both provide regulated output voltage
\end{itemize}

\end{solutionbox}
\begin{mnemonicbox}
``BRIEF'' - Battery backup, Rectification, Inversion,
Efficient switching, Feedback control.

\end{mnemonicbox}
\subsection*{Question 3(a) OR [3
marks]}\label{q3a}

\textbf{Explain the Principle \& working of Chopper circuits.}

\begin{solutionbox}
Chopper is a DC-to-DC converter that converts fixed DC
input voltage to variable DC output voltage.

\textbf{Diagram:}

\begin{center}
\textbf{Mermaid Diagram (Code)}
\begin{verbatim}
{Shaded}
{Highlighting}[]
graph LR
    DC[DC Source] {-{-}{-} S[Switch/SCR]}
    S {-{-}{-} L[Load]}
    L {-{-}{-} DC}
{Highlighting}
{Shaded}
\end{verbatim}
\end{center}

\textbf{Principle:}

\begin{itemize}
\item
  Switch (typically SCR, MOSFET, or IGBT) rapidly connects and
  disconnects source to load
\item
  Output voltage controlled by duty cycle (ON time / total time)
\item
  Average output voltage = Input voltage \times Duty cycle
\item
  \textbf{Time Ratio Control}: Varies duty cycle, keeping frequency
  constant
\item
  \textbf{Frequency Modulation}: Varies frequency, keeping ON time
  constant
\item
  \textbf{Applications}: DC motor control, battery-powered vehicles
\end{itemize}

\end{solutionbox}
\begin{mnemonicbox}
``CHOP'' - Control High-speed Operation with Pulses.

\end{mnemonicbox}
\subsection*{Question 3(b) OR [4
marks]}\label{q3b}

\textbf{Compare single-phase and Poly-phase rectifier circuits.}

\begin{solutionbox}

{\def\LTcaptype{none} % do not increment counter
\begin{longtable}[]{@{}
  >{\raggedright\arraybackslash}p{(\linewidth - 4\tabcolsep) * \real{0.1930}}
  >{\raggedright\arraybackslash}p{(\linewidth - 4\tabcolsep) * \real{0.4211}}
  >{\raggedright\arraybackslash}p{(\linewidth - 4\tabcolsep) * \real{0.3860}}@{}}
\toprule\noalign{}
\begin{minipage}[b]{\linewidth}\raggedright
Parameter
\end{minipage} & \begin{minipage}[b]{\linewidth}\raggedright
Single-Phase Rectifier
\end{minipage} & \begin{minipage}[b]{\linewidth}\raggedright
Poly-Phase Rectifier
\end{minipage} \\
\midrule\noalign{}
\endhead
\bottomrule\noalign{}
\endlastfoot
Supply & Single-phase AC & Three or more phase AC \\
Output Waveform & More pulsating & Smoother (less pulsating) \\
Ripple Content & Higher (0.48 for full wave) & Lower (0.042 for 3-phase
full wave) \\
Filtering & More filtering required & Less filtering required \\
Power Handling & Limited power handling & Higher power handling \\
Transformer Utilization & 0.812 (full wave) & 0.955 (3-phase full
wave) \\
Efficiency & Lower & Higher \\
Size & Smaller for same power & More compact for high power \\
\end{longtable}
}

\begin{itemize}
\tightlist
\item
  \textbf{Harmonic Content}: Lower in poly-phase systems
\item
  \textbf{TUF (Transformer Utilization Factor)}: Higher in poly-phase
  systems
\item
  \textbf{Cost-Effectiveness}: Poly-phase more economical for high power
\end{itemize}

\end{solutionbox}
\begin{mnemonicbox}
``PERIPHERY'' - Poly-phase Efficiency Ripple
Improvement Power Handling Economy Rating Yield.

\end{mnemonicbox}
\subsection*{Question 3(c) OR [7
marks]}\label{q3c}

\textbf{Describe the working of solar Photovoltaic (PV) based power
generation with the help of block diagram.}

\begin{solutionbox}
Solar PV power generation converts sunlight directly
into electricity using semiconductor materials.

\textbf{Diagram:}

\begin{center}
\textbf{Mermaid Diagram (Code)}
\begin{verbatim}
{Shaded}
{Highlighting}[]
graph LR
    Sun((Sunlight)) {-{-}{-} PV[PV Array]}
    PV {-{-}{-} CC[Charge Controller]}
    CC {-{-}{-} B[Battery Bank]}
    B {-{-}{-} I[Inverter]}
    I {-{-}{-} L[AC Load]}
    B {-{-}{-} DCL[DC Load]}
    I {-{-}{-} G[Grid Connection]}
{Highlighting}
{Shaded}
\end{verbatim}
\end{center}

{\def\LTcaptype{none} % do not increment counter
\begin{longtable}[]{@{}
  >{\raggedright\arraybackslash}p{(\linewidth - 2\tabcolsep) * \real{0.5238}}
  >{\raggedright\arraybackslash}p{(\linewidth - 2\tabcolsep) * \real{0.4762}}@{}}
\toprule\noalign{}
\begin{minipage}[b]{\linewidth}\raggedright
Component
\end{minipage} & \begin{minipage}[b]{\linewidth}\raggedright
Function
\end{minipage} \\
\midrule\noalign{}
\endhead
\bottomrule\noalign{}
\endlastfoot
PV Array & Converts solar energy to DC electricity through photovoltaic
effect \\
Charge Controller & Regulates battery charging and prevents
overcharging \\
Battery Bank & Stores energy for use during night or cloudy
conditions \\
Inverter & Converts DC to AC for powering AC loads \\
Grid Connection & Optional connection for feeding excess power to
grid \\
\end{longtable}
}

\textbf{Working Principle:}

\begin{itemize}
\item
  \textbf{Photovoltaic Effect}: Photons from sunlight knock electrons
  free in semiconductor
\item
  \textbf{Cell Structure}: P-N junction creates electric field
\item
  \textbf{Voltage Generation}: Typical cell produces 0.5-0.6V DC
\item
  \textbf{Array Configuration}: Series-parallel connections for desired
  voltage/current
\item
  \textbf{Efficiency}: Typically 15-22\% for commercial panels
\item
  \textbf{Applications}: Residential, commercial, industrial power
  generation
\end{itemize}

\end{solutionbox}
\begin{mnemonicbox}
``SOLAR'' - Semiconductors Oriented
Light-to-electricity Array Regulation.

\end{mnemonicbox}
\subsection*{Question 4(a) [3 marks]}\label{q4a}

\textbf{List the advantages of static switch.}

\begin{solutionbox}

{\def\LTcaptype{none} % do not increment counter
\begin{longtable}[]{@{}l@{}}
\toprule\noalign{}
Advantages of Static Switch \\
\midrule\noalign{}
\endhead
\bottomrule\noalign{}
\endlastfoot
No moving parts - higher reliability \\
Silent operation \\
Fast switching response (microseconds) \\
Longer operational life \\
No contact bounce or arcing \\
Compact size \\
Compatible with digital control systems \\
Lower maintenance requirements \\
\end{longtable}
}

\begin{itemize}
\tightlist
\item
  \textbf{Reliability}: No mechanical wear and tear
\item
  \textbf{Speed}: Much faster than mechanical switches
\item
  \textbf{Isolation}: Can provide electrical isolation
\end{itemize}

\end{solutionbox}
\begin{mnemonicbox}
``SAFE'' - Speed, Arc-free, Fast response, Endurance.

\end{mnemonicbox}
\subsection*{Question 4(b) [4 marks]}\label{q4b}

\textbf{Draw the circuit diagram of A.C. Power control using DIAC-TRIAC
and Explain it.}

\begin{solutionbox}
DIAC-TRIAC circuit provides smooth AC power control for
resistive and inductive loads.

\textbf{Diagram:}

\begin{center}
\textbf{Mermaid Diagram (Code)}
\begin{verbatim}
{Shaded}
{Highlighting}[]
graph LR
    AC[AC Supply] {-{-}{-} L[Load]}
    L {-{-}{-} T[TRIAC]}
    T {-{-}{-} AC}
    AC {-{-}{-} R1[Resistor R1]}
    R1 {-{-}{-} C[Capacitor C]}
    C {-{-}{-} D[DIAC]}
    D {-{-}{-} G[TRIAC Gate]}
    G {-{-}{-} T}
    R2[Variable Resistor R2] {-{-}{-} C}
    R2 {-{-}{-} T}
{Highlighting}
{Shaded}
\end{verbatim}
\end{center}

\textbf{Working:}

\begin{itemize}
\item
  Variable resistor R2 controls charging rate of capacitor C
\item
  When capacitor voltage reaches DIAC breakover voltage, DIAC conducts
\item
  DIAC delivers trigger pulse to TRIAC gate
\item
  TRIAC conducts for remainder of half-cycle
\item
  Process repeats for both half-cycles
\item
  \textbf{Phase Control}: Controls power by varying firing angle
\item
  \textbf{Applications}: Light dimmers, heater controls, motor speed
  control
\item
  \textbf{Power Range}: Can control from near-zero to full power
\end{itemize}

\end{solutionbox}
\begin{mnemonicbox}
``DIRECT'' - DIAC Initiates Regulated Energy Control
in TRIAC.

\end{mnemonicbox}
\subsection*{Question 4(c) [7 marks]}\label{q4c}

\textbf{Describe function of DC power control circuit using SCR with UJT
in triggering circuit.}

\begin{solutionbox}
UJT-triggered SCR circuit provides precise control of
DC power to the load.

\textbf{Diagram:}

\begin{center}
\textbf{Mermaid Diagram (Code)}
\begin{verbatim}
{Shaded}
{Highlighting}[]
graph LR
    DC[DC Source] {-{-}{-} L[Load]}
    L {-{-}{-} SCR[SCR]}
    SCR {-{-}{-} DC}
    DC {-{-}{-} R1[Resistor R1]}
    R1 {-{-}{-} R2[Variable Resistor R2]}
    R2 {-{-}{-} C[Capacitor C]}
    C {-{-}{-} E[UJT Emitter]}
    B1[UJT Base 1] {-{-}{-} R3[Resistor R3]}
    B2[UJT Base 2] {-{-}{-} R4[Resistor R4]}
    R3 {-{-}{-} DC}
    R4 {-{-}{-} G[SCR Gate]}
    G {-{-}{-} SCR}
    E {-{-}{-} B1}
    E {-{-}{-} B2}
{Highlighting}
{Shaded}
\end{verbatim}
\end{center}

\textbf{Working Principle:}

{\def\LTcaptype{none} % do not increment counter
\begin{longtable}[]{@{}
  >{\raggedright\arraybackslash}p{(\linewidth - 2\tabcolsep) * \real{0.3889}}
  >{\raggedright\arraybackslash}p{(\linewidth - 2\tabcolsep) * \real{0.6111}}@{}}
\toprule\noalign{}
\begin{minipage}[b]{\linewidth}\raggedright
Stage
\end{minipage} & \begin{minipage}[b]{\linewidth}\raggedright
Operation
\end{minipage} \\
\midrule\noalign{}
\endhead
\bottomrule\noalign{}
\endlastfoot
Charging & R1 and R2 control charging rate of capacitor C \\
UJT Firing & When capacitor voltage reaches UJT firing level, UJT
conducts \\
Pulse Generation & UJT generates sharp trigger pulse across R4 \\
SCR Triggering & Pulse triggers SCR gate, turning SCR ON \\
Power Control & Variable resistor R2 adjusts timing, controlling average
power \\
\end{longtable}
}

\begin{itemize}
\tightlist
\item
  \textbf{Precise Control}: UJT provides stable, predictable triggering
\item
  \textbf{Applications}: Battery chargers, DC motor speed control,
  temperature control
\item
  \textbf{Advantages}: Low cost, high reliability, good temperature
  stability
\item
  \textbf{Control Range}: Wide range from near-zero to full power
\end{itemize}

\end{solutionbox}
\begin{mnemonicbox}
``SCRUP'' - SCR Using Pulse from UJT for Power
control.

\end{mnemonicbox}
\subsection*{Question 4(a) OR [3
marks]}\label{q4a}

\textbf{Enlist applications of dielectric heating.}

\begin{solutionbox}

{\def\LTcaptype{none} % do not increment counter
\begin{longtable}[]{@{}l@{}}
\toprule\noalign{}
Applications of Dielectric Heating \\
\midrule\noalign{}
\endhead
\bottomrule\noalign{}
\endlastfoot
Plastic welding and sealing \\
Wood gluing and curing \\
Food processing (pre-cooking, defrosting) \\
Textile drying and processing \\
Paper and board drying \\
Pharmaceutical products drying \\
Medical applications (hyperthermia treatment) \\
Rubber vulcanization \\
\end{longtable}
}

\begin{itemize}
\tightlist
\item
  \textbf{Material Requirements}: Works best with poor conductors that
  have polar molecules
\item
  \textbf{Frequency Range}: Typically 10-100 MHz
\item
  \textbf{Advantages}: Uniform heating, faster processing, energy
  efficiency
\end{itemize}

\end{solutionbox}
\begin{mnemonicbox}
``POWER'' - Plastics, Organics, Wood, Edibles, and
Rubber processing.

\end{mnemonicbox}
\subsection*{Question 4(b) OR [4
marks]}\label{q4b}

\textbf{Draw and explain three stage IC555 timer circuit.}

\begin{solutionbox}
Three-stage IC555 timer circuit provides sequential
timing operations.

\textbf{Diagram:}

\begin{center}
\textbf{Mermaid Diagram (Code)}
\begin{verbatim}
{Shaded}
{Highlighting}[]
graph LR
    subgraph "Timer 1"
        IC1[555 Timer] 
    end
    subgraph "Timer 2"
        IC2[555 Timer]
    end
    subgraph "Timer 3"
        IC3[555 Timer]
    end
    TR[Trigger Input] {-{-}{} IC1}
    IC1 {-{-}{} O1[Output 1]}
    O1 {-{-}{} IC2}
    IC2 {-{-}{} O2[Output 2]}
    O2 {-{-}{} IC3}
    IC3 {-{-}{} O3[Output 3]}
{Highlighting}
{Shaded}
\end{verbatim}
\end{center}

\textbf{Working:}

\begin{itemize}
\item
  First timer activated by external trigger
\item
  Output of first timer triggers second timer
\item
  Output of second timer triggers third timer
\item
  Each timer can be independently adjusted
\item
  \textbf{Applications}: Industrial sequencing, process control,
  animation effects
\item
  \textbf{Timing Range}: Microseconds to hours with proper component
  selection
\item
  \textbf{Features}: Stable timing, immune to supply variations
\item
  \textbf{Advantages}: Simple design, reliable operation, low cost
\end{itemize}

\end{solutionbox}
\begin{mnemonicbox}
``THREE-SET'' - THREE Stage Electronic Timers in
sequence.

\end{mnemonicbox}
\subsection*{Question 4(c) OR [7
marks]}\label{q4c}

\textbf{Describe the working principle of Induction heating. And List
merits-demerits of Induction heating.}

\begin{solutionbox}
Induction heating uses electromagnetic induction to
heat electrically conductive materials.

\textbf{Diagram:}

\begin{center}
\textbf{Mermaid Diagram (Code)}
\begin{verbatim}
{Shaded}
{Highlighting}[]
graph LR
    PS[Power Supply] {-{-}{} INV[Inverter]}
    INV {-{-}{} LC[Matching Circuit]}
    LC {-{-}{} WC[Work Coil]}
    WC {-{-}{} W[Workpiece]}
    FC[Feedback Control] {-{-}{} INV}
{Highlighting}
{Shaded}
\end{verbatim}
\end{center}

\textbf{Working Principle:}

\begin{itemize}
\tightlist
\item
  High frequency AC in work coil creates alternating magnetic field
\item
  Magnetic field induces eddy currents in workpiece
\item
  Eddy currents generate heat due to material resistance
\item
  Heating occurs within the workpiece, not from external source
\end{itemize}

{\def\LTcaptype{none} % do not increment counter
\begin{longtable}[]{@{}
  >{\raggedright\arraybackslash}p{(\linewidth - 2\tabcolsep) * \real{0.4444}}
  >{\raggedright\arraybackslash}p{(\linewidth - 2\tabcolsep) * \real{0.5556}}@{}}
\toprule\noalign{}
\begin{minipage}[b]{\linewidth}\raggedright
Merits
\end{minipage} & \begin{minipage}[b]{\linewidth}\raggedright
Demerits
\end{minipage} \\
\midrule\noalign{}
\endhead
\bottomrule\noalign{}
\endlastfoot
Rapid heating & High initial equipment cost \\
Energy efficient (80-90\%) & Limited to electrically conductive
materials \\
Precise temperature control & Requires high-frequency power supply \\
Clean process with no combustion & Complex coil design for specific
applications \\
Localized heating possible & High power requirements \\
Consistent, repeatable results & Requires water cooling systems \\
Environmentally friendly & Electromagnetic interference issues \\
Improved working conditions & Limited penetration depth \\
\end{longtable}
}

\begin{itemize}
\tightlist
\item
  \textbf{Frequency Range}: 1 kHz to 1 MHz depending on application
\item
  \textbf{Applications}: Heat treatment, melting, brazing, soldering
\end{itemize}

\end{solutionbox}
\begin{mnemonicbox}
``EDDY'' - Electromagnetic Device Develops Yield of
heat.

\end{mnemonicbox}
\subsection*{Question 5(a) [3 marks]}\label{q5a}

\textbf{Draw \& explain solid state circuit to control dc shunt motor
speed.}

\begin{solutionbox}
Solid-state circuit for DC shunt motor speed control
uses SCR to control armature voltage.

\textbf{Diagram:}

\begin{center}
\textbf{Mermaid Diagram (Code)}
\begin{verbatim}
{Shaded}
{Highlighting}[]
graph LR
    AC[AC Supply] {-{-}{-} BR[Bridge Rectifier]}
    BR {-{-}{-} SCR[SCR]}
    SCR {-{-}{-} A[Armature]}
    A {-{-}{-} BR}
    BR {-{-}{-} F[Field Winding]}
    F {-{-}{-} BR}
    RC[Firing Circuit] {-{-}{-} SCR}
{Highlighting}
{Shaded}
\end{verbatim}
\end{center}

\begin{itemize}
\tightlist
\item
  \textbf{Armature Voltage Control}: SCR controls voltage to armature
\item
  \textbf{Field Winding}: Connected directly to DC supply
\item
  \textbf{Speed Control}: By varying SCR firing angle
\item
  \textbf{Advantages}: Smooth control, high efficiency, compact size
\end{itemize}

\end{solutionbox}
\begin{mnemonicbox}
``SAFE'' - SCR Armature Firing for Efficient control.

\end{mnemonicbox}
\subsection*{Question 5(b) [4 marks]}\label{q5b}

\textbf{Explain working principle of stepper motor.}

\begin{solutionbox}
Stepper motor converts electrical pulses into discrete
mechanical movements.

\textbf{Diagram:}

\begin{center}
\textbf{Mermaid Diagram (Code)}
\begin{verbatim}
{Shaded}
{Highlighting}[]
graph TD
    subgraph "Stepper Motor"
        R[Rotor]
        S1[Stator Winding 1]
        S2[Stator Winding 2]
        S3[Stator Winding 3]
        S4[Stator Winding 4]
    end
{Highlighting}
{Shaded}
\end{verbatim}
\end{center}

\textbf{Working Principle:}

\begin{itemize}
\tightlist
\item
  Energizing stator windings in sequence creates rotating magnetic field
\item
  Permanent magnet rotor aligns with magnetic field
\item
  Each pulse creates rotation by exact ``step'' angle
\item
  Step angle determined by motor construction (typically 1.8^\circ or 0.9^\circ)
\end{itemize}

{\def\LTcaptype{none} % do not increment counter
\begin{longtable}[]{@{}
  >{\raggedright\arraybackslash}p{(\linewidth - 2\tabcolsep) * \real{0.2727}}
  >{\raggedright\arraybackslash}p{(\linewidth - 2\tabcolsep) * \real{0.7273}}@{}}
\toprule\noalign{}
\begin{minipage}[b]{\linewidth}\raggedright
Type
\end{minipage} & \begin{minipage}[b]{\linewidth}\raggedright
Characteristics
\end{minipage} \\
\midrule\noalign{}
\endhead
\bottomrule\noalign{}
\endlastfoot
Variable Reluctance & No permanent magnet, relies on magnetic
reluctance \\
Permanent Magnet & Uses permanent magnet rotor \\
Hybrid & Combines features of both types \\
\end{longtable}
}

\begin{itemize}
\tightlist
\item
  \textbf{Precise Positioning}: Movement in exact increment steps
\item
  \textbf{Open-Loop Control}: No feedback needed for position control
\item
  \textbf{Holding Torque}: Maintains position when energized
\end{itemize}

\end{solutionbox}
\begin{mnemonicbox}
``STEP'' - Sequential Triggering Enables Precise
positioning.

\end{mnemonicbox}
\subsection*{Question 5(c) [7 marks]}\label{q5c}

\textbf{Draw the block diagram of PLC and explain the function of each
block.}

\begin{solutionbox}
Programmable Logic Controller (PLC) is a digital
computer used for automation of industrial processes.

\textbf{Diagram:}

\begin{center}
\textbf{Mermaid Diagram (Code)}
\begin{verbatim}
{Shaded}
{Highlighting}[]
graph TD
    PS[Power Supply] {-{-}{-} CPU[Central Processing Unit]}
    I[Input Modules] {-{-}{-} CPU}
    CPU {-{-}{-} O[Output Modules]}
    M[Memory] {-{-}{-} CPU}
    P[Programming Device] {-{-}{-} CPU}
    C[Communication Module] {-{-}{-} CPU}
{Highlighting}
{Shaded}
\end{verbatim}
\end{center}

{\def\LTcaptype{none} % do not increment counter
\begin{longtable}[]{@{}ll@{}}
\toprule\noalign{}
Block & Function \\
\midrule\noalign{}
\endhead
\bottomrule\noalign{}
\endlastfoot
Power Supply & Converts main AC to DC for internal use \\
CPU & Executes program, processes data, manages operations \\
Input Modules & Interface with sensors, switches, and field devices \\
Output Modules & Control actuators, motors, valves, and indicators \\
Memory & Stores program and data (ROM, RAM, EEPROM) \\
Programming Device & External computer or terminal for programming \\
Communication Module & Interfaces with other PLCs, SCADA, HMI \\
\end{longtable}
}

\begin{itemize}
\tightlist
\item
  \textbf{Scan Cycle}: Input scanning \rightarrow Program execution \rightarrow Output
  updating
\item
  \textbf{Advantages}: Reliability, flexibility, modular design, easy
  troubleshooting
\item
  \textbf{Applications}: Manufacturing automation, process control,
  material handling
\item
  \textbf{Programming}: Ladder logic, function block diagram, structured
  text
\end{itemize}

\end{solutionbox}
\begin{mnemonicbox}
``PILOT'' - Processing Inputs and Logic for Outputs
with Timing control.

\end{mnemonicbox}
\subsection*{Question 5(a) OR [3
marks]}\label{q5a}

\textbf{Draw and explain the construction of DC servo motor.}

\begin{solutionbox}
DC servo motor is designed for precise position and
speed control.

\textbf{Diagram:}

\begin{center}
\textbf{Mermaid Diagram (Code)}
\begin{verbatim}
{Shaded}
{Highlighting}[]
graph TD
    subgraph "DC Servo Motor"
        A[Armature]
        F[Field Winding]
        S[Shaft]
        FB[Feedback Device]
    end
{Highlighting}
{Shaded}
\end{verbatim}
\end{center}

\textbf{Components:}

\begin{itemize}
\item
  \textbf{Armature}: Low inertia for quick response
\item
  \textbf{Field System}: Provides magnetic field (permanent magnets in
  modern motors)
\item
  \textbf{Commutator \& Brushes}: Electrical connection to rotating
  armature
\item
  \textbf{Feedback Device}: Position sensor
  (encoder/resolver/tachometer)
\item
  \textbf{Housing}: Contains bearings and mounting provisions
\item
  \textbf{High Torque-to-Inertia Ratio}: Allows quick starts and stops
\item
  \textbf{Linear Torque-Speed Characteristics}: Enables precise control
\item
  \textbf{Low Electrical Time Constant}: Fast response to control
  signals
\end{itemize}

\end{solutionbox}
\begin{mnemonicbox}
``SAFE'' - Sensitive Armature with Feedback for
Exactness.

\end{mnemonicbox}
\subsection*{Question 5(b) OR [4
marks]}\label{q5b}

\textbf{Draw and explain the circuit to control speed of a DC series
motor.}

\begin{solutionbox}
DC series motor speed control circuit using SCR.

\textbf{Diagram:}

\begin{center}
\textbf{Mermaid Diagram (Code)}
\begin{verbatim}
{Shaded}
{Highlighting}[]
graph LR
    AC[AC Supply] {-{-}{-} BR[Bridge Rectifier]}
    BR {-{-}{-} SCR[SCR]}
    SCR {-{-}{-} S[Series Field]}
    S {-{-}{-} A[Armature]}
    A {-{-}{-} BR}
    FC[Firing Circuit] {-{-}{-} SCR}
    P[Potentiometer] {-{-}{-} FC}
{Highlighting}
{Shaded}
\end{verbatim}
\end{center}

\textbf{Working:}

\begin{itemize}
\item
  Bridge rectifier converts AC to DC
\item
  SCR controls average voltage to motor
\item
  Firing angle controlled by potentiometer
\item
  Series field and armature current is the same
\item
  Speed varies inversely with voltage at low loads
\item
  \textbf{Armature Voltage Control}: Primary method for speed control
\item
  \textbf{Torque Characteristics}: High starting torque maintained
\item
  \textbf{Speed Range}: Typically 3:1 for stable operation
\end{itemize}

\end{solutionbox}
\begin{mnemonicbox}
``SCRAM'' - SCR Controls Rectified Armature and Motor
speed.

\end{mnemonicbox}
\subsection*{Question 5(c) OR [7
marks]}\label{q5c}

\textbf{Explain construction, working of Stepper motor Give and its
applications}

\begin{solutionbox}
Stepper motor is an electromechanical device that
converts electrical pulses into discrete mechanical movements.

\textbf{Construction:}

\textbf{Diagram:}

\begin{center}
\textbf{Mermaid Diagram (Code)}
\begin{verbatim}
{Shaded}
{Highlighting}[]
graph TD
    subgraph "Stepper Motor"
        R[Rotor {- Permanent Magnet]}
        S[Stator {- Electromagnetic Coils]}
        SH[Shaft]
    end
{Highlighting}
{Shaded}
\end{verbatim}
\end{center}

{\def\LTcaptype{none} % do not increment counter
\begin{longtable}[]{@{}ll@{}}
\toprule\noalign{}
Component & Description \\
\midrule\noalign{}
\endhead
\bottomrule\noalign{}
\endlastfoot
Stator & Contains multiple coil windings arranged in phases \\
Rotor & Permanent magnet or soft iron (reluctance type) \\
Bearings & Support shaft and allow rotation \\
Housing & Mechanical structure holding all components \\
Leads & Electrical connections to stator windings \\
\end{longtable}
}

\textbf{Working Principle:}

\begin{itemize}
\tightlist
\item
  Digital pulses energize stator windings in sequence
\item
  Magnetic field rotates in steps around stator
\item
  Rotor follows magnetic field in precise angular steps
\item
  Direction controlled by sequence of energization
\item
  Speed controlled by pulse frequency
\end{itemize}

\textbf{Types of Stepper Motors:}

{\def\LTcaptype{none} % do not increment counter
\begin{longtable}[]{@{}ll@{}}
\toprule\noalign{}
Type & Characteristics \\
\midrule\noalign{}
\endhead
\bottomrule\noalign{}
\endlastfoot
Variable Reluctance & No permanent magnet, high speed, low torque \\
Permanent Magnet & Simpler design, moderate torque, lower resolution \\
Hybrid & Combines both designs, high resolution, good torque \\
\end{longtable}
}

\textbf{Applications:}

\begin{itemize}
\tightlist
\item
  CNC machines and 3D printers
\item
  Robotics and automation
\item
  Camera lens focusing mechanisms
\item
  Precision positioning systems
\item
  Medical equipment
\item
  Office equipment (printers, scanners)
\item
  Automotive applications (headlight positioning)
\item
  Small consumer devices
\end{itemize}

\end{solutionbox}
\begin{mnemonicbox}
``REACT'' - Rotation Exactly At Controlled Timing.

\end{mnemonicbox}

\end{document}
