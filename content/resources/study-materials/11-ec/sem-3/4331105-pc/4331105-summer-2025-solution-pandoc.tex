\documentclass[10pt,a4paper]{article}

% content/resources/templates/preamble.tex
\usepackage[margin=0.6in]{geometry}
\author{Milav Dabgar}
\usepackage{amsmath,amssymb,amsthm}
\usepackage{booktabs}
\usepackage{multirow}
\usepackage{xcolor}
\usepackage{tcolorbox}
\tcbuselibrary{breakable,skins}
\usepackage[colorlinks=true,linkcolor=blue]{hyperref}
\usepackage{titlesec}
\usepackage{enumitem}
\usepackage{tikz}
\usepackage{pgfplots}
\usepackage{circuitikz}
\usepackage[version=4]{mhchem}
\usepackage{longtable}
\usepackage{array}
\usepackage{float}
\usepackage{caption}
\usepackage{listings}

\lstset{
  basicstyle=\small\ttfamily,
  breaklines=true,
  breakatwhitespace=false,
  postbreak=\mbox{\textcolor{red}{$\hookrightarrow$}\space},
  float=false,
  numbers=left,
  numberstyle=\tiny\color{gray},
  numbersep=10pt,
  xleftmargin=2em,
  keywordstyle=\color{blue},
  commentstyle=\color{green!60!black},
  stringstyle=\color{purple},
  backgroundcolor=\color{gray!5},
  showstringspaces=false,
  tabsize=2,
  captionpos=b,
  keepspaces=true,
  columns=flexible
}

\pgfplotsset{compat=1.18}
\usetikzlibrary{shapes,arrows,positioning,calc,patterns,decorations.pathmorphing,decorations.markings,arrows.meta}

% Color scheme
\definecolor{headcolor}{RGB}{0,102,204}
\definecolor{keycolor}{RGB}{220,20,60}
\definecolor{solutioncolor}{RGB}{34,139,34}
\definecolor{mnemoniccolor}{RGB}{148,0,211}
\definecolor{codecolor}{RGB}{0,0,100}

% Spacing
\setlength{\parskip}{3pt}
\setlist[itemize]{nosep}
\setlist[enumerate]{nosep}

% Title formatting
\titleformat{\section}{\Large\bfseries\color{headcolor}}{\thesection}{1em}{}
\titleformat{\subsection}{\large\bfseries\color{headcolor}}{\thesubsection}{1em}{}

% Pandoc tightlist compatibility
\providecommand{\tightlist}{%
  \setlength{\itemsep}{0pt}\setlength{\parskip}{0pt}}

% Pandoc longtable compatibility
\newcounter{none}
\def\thenone{}


% content/resources/templates/english-boxes.tex
% This file is currently empty - it exists to maintain consistency with the import structure.
% Add custom environments here if needed in the future.


\begin{document}

\begin{center}
{\Huge\bfseries\color{headcolor} Subject Name Solutions}\\[5pt]
{\LARGE 4331105 -- Summer 2025}\\[3pt]
{\large Semester 1 Study Material}\\[3pt]
{\normalsize\textit{Detailed Solutions and Explanations}}
\end{center}

\vspace{10pt}

\subsection*{Question 1(a) [3 marks]}\label{q1a}

\textbf{How many keywords are there in C? Write any four keywords}

\begin{solutionbox}

{\def\LTcaptype{none} % do not increment counter
\begin{longtable}[]{@{}ll@{}}
\toprule\noalign{}
Total Keywords & Examples \\
\midrule\noalign{}
\endhead
\bottomrule\noalign{}
\endlastfoot
32 keywords & int, float, char, if \\
\end{longtable}
}

\textbf{Diagram:}

\begin{center}
\textbf{Mermaid Diagram (Code)}
\begin{verbatim}
{Shaded}
{Highlighting}[]
graph TD
    A[C Keywords {- 32 Total] {-}{-}{} B[Data Types: int, float, char, double]}
    A {-{-}{} C[Control: if, else, for, while]}
    A {-{-}{} D[Storage: static, extern, auto, register]}
{Highlighting}
{Shaded}
\end{verbatim}
\end{center}

\begin{itemize}
\tightlist
\item
  \textbf{32 keywords}: Total reserved words in C language
\item
  \textbf{Data type keywords}: int, float, char, double for variable
  declaration
\item
  \textbf{Control keywords}: if, else, for, while for program flow
\end{itemize}

\end{solutionbox}
\begin{mnemonicbox}
``Cats In Four Colors'' (char, int, float, const)

\end{mnemonicbox}
\begin{center}\rule{0.5\linewidth}{0.5pt}\end{center}

\subsection*{Question 1(b) [4 marks]}\label{q1b}

\textbf{What is variable? Explain rules for naming a variable with
example}

\begin{solutionbox}

\textbf{Variable Definition:}

{\def\LTcaptype{none} % do not increment counter
\begin{longtable}[]{@{}ll@{}}
\toprule\noalign{}
Aspect & Description \\
\midrule\noalign{}
\endhead
\bottomrule\noalign{}
\endlastfoot
Definition & Named memory location to store data \\
Purpose & Hold values that can change during program execution \\
Declaration & datatype variable\_name; \\
\end{longtable}
}

\textbf{Naming Rules:}

\begin{itemize}
\tightlist
\item
  \textbf{First character}: Must be letter or underscore (\_)
\item
  \textbf{Subsequent characters}: Letters, digits, underscore only
\item
  \textbf{Case sensitive}: `Age' and `age' are different
\item
  \textbf{No keywords}: Cannot use reserved words like `int', `float'
\end{itemize}

\textbf{Examples:}

\begin{verbatim}
int age;        // Valid
float \_salary;  // Valid
char name123;   // Valid
int 2number;    // Invalid {- starts with digit}
float for;      // Invalid {- keyword used}
\end{verbatim}

\end{solutionbox}
\begin{mnemonicbox}
``Letters First, No Keywords'' (LF-NK)

\end{mnemonicbox}
\begin{center}\rule{0.5\linewidth}{0.5pt}\end{center}

\subsection*{Question 1(c) [7 marks]}\label{q1c}

\textbf{Specify errors if any, in the following statements}

\begin{solutionbox}

{\def\LTcaptype{none} % do not increment counter
\begin{longtable}[]{@{}
  >{\raggedright\arraybackslash}p{(\linewidth - 4\tabcolsep) * \real{0.4231}}
  >{\raggedright\arraybackslash}p{(\linewidth - 4\tabcolsep) * \real{0.2692}}
  >{\raggedright\arraybackslash}p{(\linewidth - 4\tabcolsep) * \real{0.3077}}@{}}
\toprule\noalign{}
\begin{minipage}[b]{\linewidth}\raggedright
Statement
\end{minipage} & \begin{minipage}[b]{\linewidth}\raggedright
Error
\end{minipage} & \begin{minipage}[b]{\linewidth}\raggedright
Reason
\end{minipage} \\
\midrule\noalign{}
\endhead
\bottomrule\noalign{}
\endlastfoot
(1) fLoat x; & Invalid keyword & Correct: float x; \\
(2) int min, max = 20; & Partial initialization & Only max initialized,
min uninitialized \\
(3) long char c; & Invalid combination & Cannot combine long with
char \\
(4) iNt a; & Invalid keyword & Correct: int a; \\
(5) FLOAT

f=2; & Invalid keyword & Correct: float

f=2; \\

(6) double m ; n; & Missing datatype & Correct: double m, n; \\
(7) Int score (100)0; & Multiple errors & Invalid syntax, correct: int
score = 100; \\
\end{longtable}
}

\textbf{Key Points:}

\begin{itemize}
\tightlist
\item
  \textbf{Case sensitivity}: Keywords must be lowercase
\item
  \textbf{Multiple declaration}: Use comma separator
\item
  \textbf{Initialization syntax}: Use = operator
\end{itemize}

\end{solutionbox}
\begin{mnemonicbox}
``Keywords Lower Case Always'' (KLCA)

\end{mnemonicbox}
\begin{center}\rule{0.5\linewidth}{0.5pt}\end{center}

\subsection*{Question 1(c) OR [7
marks]}\label{q1c}

\textbf{What is algorithm? What is flowchart? Draw a flowchart to find
area and perimeter of circle.}

\begin{solutionbox}

\textbf{Definitions:}

{\def\LTcaptype{none} % do not increment counter
\begin{longtable}[]{@{}ll@{}}
\toprule\noalign{}
Term & Definition \\
\midrule\noalign{}
\endhead
\bottomrule\noalign{}
\endlastfoot
Algorithm & Step-by-step procedure to solve a problem \\
Flowchart & Visual representation of algorithm using symbols \\
\end{longtable}
}

\textbf{Flowchart for Circle Area and Perimeter:}

\begin{verbatim}
flowchart LR
    A[Start] {-{-} B[Input radius r]}
    B {-{-} C[Calculate area = π  r^{2}]}
    C {-{-} D[Calculate perimeter = 2  π  r]}
    D {-{-} E[Display area and perimeter]}
    E {-{-} F[End]}
\end{verbatim}

\textbf{Algorithm Steps:}

\begin{itemize}
\tightlist
\item
  \textbf{Step 1}: Start
\item
  \textbf{Step 2}: Input radius value
\item
  \textbf{Step 3}: Calculate area using formula π\timesr^{2}
\item
  \textbf{Step 4}: Calculate perimeter using formula 2\timesπ\timesr
\end{itemize}

\end{solutionbox}
\begin{mnemonicbox}
``Start Input Calculate Display End'' (SICDE)

\end{mnemonicbox}
\begin{center}\rule{0.5\linewidth}{0.5pt}\end{center}

\subsection*{Question 2(a) [3 marks]}\label{q2a}

\textbf{What is operator? List all the `C' operators.}

\begin{solutionbox}

\textbf{Operator Definition:}

{\def\LTcaptype{none} % do not increment counter
\begin{longtable}[]{@{}ll@{}}
\toprule\noalign{}
Aspect & Description \\
\midrule\noalign{}
\endhead
\bottomrule\noalign{}
\endlastfoot
Definition & Special symbols that perform operations on operands \\
Purpose & Manipulate data and variables \\
\end{longtable}
}

\textbf{C Operators List:}

{\def\LTcaptype{none} % do not increment counter
\begin{longtable}[]{@{}ll@{}}
\toprule\noalign{}
Category & Operators \\
\midrule\noalign{}
\endhead
\bottomrule\noalign{}
\endlastfoot
Arithmetic & +, -, *, /, \% \\
Relational & \textless, \textgreater, \textless=, \textgreater=, ==,
!= \\
Logical & \&\&, \textbar\textbar, ! \\
Assignment & =, +=, -=, *=, /= \\
Increment/Decrement & ++, -- \\
Conditional & ?: \\
\end{longtable}
}

\end{solutionbox}
\begin{mnemonicbox}
``Add Relate Logic Assign Increment Condition''
(ARLIC)

\end{mnemonicbox}
\begin{center}\rule{0.5\linewidth}{0.5pt}\end{center}

\subsection*{Question 2(b) [4 marks]}\label{q2b}

\textbf{State difference between while and do while loop.}

\begin{solutionbox}

{\def\LTcaptype{none} % do not increment counter
\begin{longtable}[]{@{}lll@{}}
\toprule\noalign{}
Aspect & while loop & do-while loop \\
\midrule\noalign{}
\endhead
\bottomrule\noalign{}
\endlastfoot
\textbf{Entry condition} & Pre-tested & Post-tested \\
\textbf{Minimum execution} & 0 times & At least 1 time \\
\textbf{Syntax} & while(condition) \{ \} & do \{ \} while(condition); \\
\textbf{Semicolon} & Not required after while & Required after while \\
\end{longtable}
}

\textbf{Example:}

\begin{verbatim}
// while loop
while(i {} 5) \{
    printf("\%d", i);
    i++;
\}

// do{-while loop  }
do \{
    printf("\%d", i);
    i++;
\} while(i {} 5);
\end{verbatim}

\textbf{Key Points:}

\begin{itemize}
\tightlist
\item
  \textbf{Pre-tested}: Condition checked before execution
\item
  \textbf{Post-tested}: Condition checked after execution
\end{itemize}

\end{solutionbox}
\begin{mnemonicbox}
``While Before, Do After'' (WB-DA)

\end{mnemonicbox}
\begin{center}\rule{0.5\linewidth}{0.5pt}\end{center}

\subsection*{Question 2(c) [7 marks]}\label{q2c}

\textbf{How is scanf() function used for formatted input? Explain with
example}

\begin{solutionbox}

\textbf{scanf() Function:}

{\def\LTcaptype{none} % do not increment counter
\begin{longtable}[]{@{}ll@{}}
\toprule\noalign{}
Feature & Description \\
\midrule\noalign{}
\endhead
\bottomrule\noalign{}
\endlastfoot
Purpose & Read formatted input from keyboard \\
Syntax & scanf(``format\_string'', \&variable); \\
Return & Number of successfully read inputs \\
\end{longtable}
}

\textbf{Format Specifiers:}

{\def\LTcaptype{none} % do not increment counter
\begin{longtable}[]{@{}ll@{}}
\toprule\noalign{}
Specifier & Data Type \\
\midrule\noalign{}
\endhead
\bottomrule\noalign{}
\endlastfoot
\%d & int \\
\%f & float \\
\%c & char \\
\%s & string \\
\end{longtable}
}

\textbf{Examples:}

\begin{verbatim}
int age;
float salary;
char grade;

scanf("\%d", \&age);           // Read integer
scanf("\%f", \&salary);        // Read float
scanf("\%c", \&grade);         // Read character
scanf("\%d \%f", \&age, \&salary); // Multiple inputs
\end{verbatim}

\textbf{Important Points:}

\begin{itemize}
\tightlist
\item
  \textbf{Address operator (\&)}: Required for variables
\item
  \textbf{Format string}: Must match data types
\item
  \textbf{Buffer issues}: Use fflush(stdin) if needed
\end{itemize}

\end{solutionbox}
\begin{mnemonicbox}
``Address Format Match'' (AFM)

\end{mnemonicbox}
\begin{center}\rule{0.5\linewidth}{0.5pt}\end{center}

\subsection*{Question 2(a) OR [3
marks]}\label{q2a}

\textbf{List arithmetic and relational operators of C language}

\begin{solutionbox}

{\def\LTcaptype{none} % do not increment counter
\begin{longtable}[]{@{}lll@{}}
\toprule\noalign{}
Operator Type & Operators & Purpose \\
\midrule\noalign{}
\endhead
\bottomrule\noalign{}
\endlastfoot
\textbf{Arithmetic} & +, -, *, /, \% & Mathematical operations \\
\textbf{Relational} & \textless, \textgreater, \textless=,
\textgreater=, ==, != & Comparison operations \\
\end{longtable}
}

\textbf{Examples:}

\begin{verbatim}
// Arithmetic
int a = 10 + 5;    // Addition
int b = 10 \% 3;    // Modulus (remainder)

// Relational
if(a {} b)          // Greater than
if(a == b)         // Equal to
\end{verbatim}

\end{solutionbox}
\begin{mnemonicbox}
``Add Multiply Compare'' (AMC)

\end{mnemonicbox}
\begin{center}\rule{0.5\linewidth}{0.5pt}\end{center}

\subsection*{Question 2(b) OR [4
marks]}\label{q2b}

\textbf{Draw flow chart of else if ladder.}

\begin{solutionbox}

\begin{verbatim}
flowchart LR
    A[Start] {-{-} B\{Condition 1?\}}
    B {-{-}|True| C[Statement 1]}
    B {-{-}|False| D\{Condition 2?\}}
    D {-{-}|True| E[Statement 2]}
    D {-{-}|False| F\{Condition 3?\}}
    F {-{-}|True| G[Statement 3]}
    F {-{-}|False| H[Else Statement]}
    C {-{-} I[End]}
    E {-{-} I}
    G {-{-} I}
    H {-{-} I}
\end{verbatim}

\textbf{Structure:}

\begin{itemize}
\tightlist
\item
  \textbf{Multiple conditions}: Checked sequentially
\item
  \textbf{First true}: Corresponding block executes
\item
  \textbf{Default case}: Else block for no match
\end{itemize}

\end{solutionbox}
\begin{mnemonicbox}
``Check First True Execute'' (CFTE)

\end{mnemonicbox}
\begin{center}\rule{0.5\linewidth}{0.5pt}\end{center}

\subsection*{Question 2(c) OR [7
marks]}\label{q2c}

\textbf{How is printf() function used for formatted output? Explain with
example}

\begin{solutionbox}

\textbf{printf() Function:}

{\def\LTcaptype{none} % do not increment counter
\begin{longtable}[]{@{}ll@{}}
\toprule\noalign{}
Feature & Description \\
\midrule\noalign{}
\endhead
\bottomrule\noalign{}
\endlastfoot
Purpose & Display formatted output on screen \\
Syntax & printf(``format\_string'', variables); \\
Return & Number of characters printed \\
\end{longtable}
}

\textbf{Format Specifiers:}

{\def\LTcaptype{none} % do not increment counter
\begin{longtable}[]{@{}lll@{}}
\toprule\noalign{}
Specifier & Usage & Example \\
\midrule\noalign{}
\endhead
\bottomrule\noalign{}
\endlastfoot
\%d & Integer & printf(``\%d'', 25); \\
\%f & Float & printf(``\%.2f'', 3.14); \\
\%c & Character & printf(``\%c'', `A'); \\
\%s & String & printf(``\%s'', ``Hello''); \\
\end{longtable}
}

\textbf{Advanced Formatting:}

\begin{verbatim}
int num = 123;
float pi = 3.14159;

printf("Number: \%5d{n}", num);      // Width specification
printf("Pi: \%.2f{n}", pi);          // Precision specification
printf("Hex: \%x{n}", num);          // Hexadecimal
printf("Left aligned: \%{-10dn}", num); // Left alignment
\end{verbatim}

\textbf{Escape Sequences:}

\begin{itemize}
\tightlist
\item
  **\n**: New line
\item
  **\t**: Tab space
\item
  \textbf{\textbackslash{}}: Backslash
\end{itemize}

\end{solutionbox}
\begin{mnemonicbox}
``Format Width Precision Align'' (FWPA)

\end{mnemonicbox}
\begin{center}\rule{0.5\linewidth}{0.5pt}\end{center}

\subsection*{Question 3(a) [3 marks]}\label{q3a}

\textbf{List Logical operators and explain it}

\begin{solutionbox}

{\def\LTcaptype{none} % do not increment counter
\begin{longtable}[]{@{}llll@{}}
\toprule\noalign{}
Operator & Symbol & Description & Truth Table \\
\midrule\noalign{}
\endhead
\bottomrule\noalign{}
\endlastfoot
\textbf{AND} & \&\& & True if both operands true & T\&\&T = T, others =
F \\
\textbf{OR} & \textbar\textbar{} & True if any operand true &
F\textbar\textbar

F = F, others = T \\

\textbf{NOT} & ! & Inverts the condition & !T = F, !F = T \\
\end{longtable}
}

\textbf{Examples:}

\begin{verbatim}
int

a = 5,

b = 10;


if(a {} 0 \&\& b {} 0)    // Both conditions must be true
if(a {} 15 || b {} 5)   // At least one condition true  
if(!(a {} 10))         // Negation of condition
\end{verbatim}

\end{solutionbox}
\begin{mnemonicbox}
``And Or Not'' (AON)

\end{mnemonicbox}
\begin{center}\rule{0.5\linewidth}{0.5pt}\end{center}

\subsection*{Question 3(b) [4 marks]}\label{q3b}

\textbf{Explain for loop with example.}

\begin{solutionbox}

\textbf{For Loop Structure:}

{\def\LTcaptype{none} % do not increment counter
\begin{longtable}[]{@{}ll@{}}
\toprule\noalign{}
Component & Purpose \\
\midrule\noalign{}
\endhead
\bottomrule\noalign{}
\endlastfoot
Initialization & Set starting value \\
Condition & Test for continuation \\
Update & Modify loop variable \\
\end{longtable}
}

\textbf{Syntax:}

\begin{verbatim}
for(initialization; condition; update) \{
    statements;
\}
\end{verbatim}

\textbf{Example:}

\begin{verbatim}
// Print numbers 1 to 5
for(int

i = 1; i {=} 5; i++) \{

    printf("\%d ", i);
\}
// Output: 1 2 3 4 5
\end{verbatim}

\textbf{Execution Flow:}

\begin{itemize}
\tightlist
\item
  \textbf{Step 1}: Initialize i = 1
\item
  \textbf{Step 2}: Check condition i \textless= 5
\item
  \textbf{Step 3}: Execute statements
\item
  \textbf{Step 4}: Update i++, repeat from step 2
\end{itemize}

\end{solutionbox}
\begin{mnemonicbox}
``Initialize Check Execute Update'' (ICEU)

\end{mnemonicbox}
\begin{center}\rule{0.5\linewidth}{0.5pt}\end{center}

\subsection*{Question 3(c) [7 marks]}\label{q3c}

\textbf{Write a program to find maximum out of three integer numbers x
and y.}

\begin{solutionbox}

\begin{verbatim}
\#include {stdio.h}

int main() \{
    int x, y, z, max;
    
    printf("Enter three numbers: ");
    scanf("\%d \%d \%d", \&x, \&y, \&z);
    
    max = x;  // Assume first number is maximum
    
    if(y {} max) \{
        max = y;
    \}
    if(z {} max) \{
        max = z;
    \}
    
    printf("Maximum number is: \%d", max);
    
    return 0;
\}
\end{verbatim}

\textbf{Algorithm Steps:}

{\def\LTcaptype{none} % do not increment counter
\begin{longtable}[]{@{}ll@{}}
\toprule\noalign{}
Step & Action \\
\midrule\noalign{}
\endhead
\bottomrule\noalign{}
\endlastfoot
1 & Input three numbers \\
2 & Assume first as maximum \\
3 & Compare with second, update if larger \\
4 & Compare with third, update if larger \\
5 & Display maximum \\
\end{longtable}
}

\textbf{Alternative Method:}

\begin{verbatim}
max = (x {} y) ? ((x {} z) ? x : z) : ((y {} z) ? y : z);
\end{verbatim}

\end{solutionbox}
\begin{mnemonicbox}
``Assume Compare Update Display'' (ACUD)

\end{mnemonicbox}
\begin{center}\rule{0.5\linewidth}{0.5pt}\end{center}

\subsection*{Question 3(a) OR [3
marks]}\label{q3a}

\textbf{Explain conditional operator with example.}

\begin{solutionbox}

\textbf{Conditional Operator (Ternary):}

{\def\LTcaptype{none} % do not increment counter
\begin{longtable}[]{@{}ll@{}}
\toprule\noalign{}
Feature & Description \\
\midrule\noalign{}
\endhead
\bottomrule\noalign{}
\endlastfoot
Symbol & ?: \\
Syntax & condition ? value1 : value2 \\
Purpose & Shortcut for if-else \\
\end{longtable}
}

\textbf{Examples:}

\begin{verbatim}
int

a = 10,

b = 20;

int max = (a {} b) ? a : b;        // max = 20

char grade = (marks {=} 60) ? {P} : {F};
printf("Status: \%s", (age {=} 18) ? "Adult" : "Minor");
\end{verbatim}

\textbf{Equivalent if-else:}

\begin{verbatim}
if(a {} b)
    max = a;
else
    max = b;
\end{verbatim}

\textbf{Advantages:}

\begin{itemize}
\tightlist
\item
  \textbf{Concise}: Single line expression
\item
  \textbf{Efficient}: Faster execution
\end{itemize}

\end{solutionbox}
\begin{mnemonicbox}
``Question Mark Colon Choice'' (QMCC)

\end{mnemonicbox}
\begin{center}\rule{0.5\linewidth}{0.5pt}\end{center}

\subsection*{Question 3(b) OR [4
marks]}\label{q3b}

\textbf{Explain while loop with example.}

\begin{solutionbox}

\textbf{While Loop:}

{\def\LTcaptype{none} % do not increment counter
\begin{longtable}[]{@{}ll@{}}
\toprule\noalign{}
Feature & Description \\
\midrule\noalign{}
\endhead
\bottomrule\noalign{}
\endlastfoot
Type & Entry-controlled loop \\
Syntax & while(condition) \{ statements; \} \\
Execution & Repeats while condition is true \\
\end{longtable}
}

\textbf{Example:}

\begin{verbatim}
int i = 1;
while(i {=} 5) \{
    printf("\%d ", i);
    i++;
\}
// Output: 1 2 3 4 5
\end{verbatim}

\textbf{Important Points:}

\begin{itemize}
\tightlist
\item
  \textbf{Initialization}: Before loop
\item
  \textbf{Condition}: Checked at beginning\\
\item
  \textbf{Update}: Inside loop body
\item
  \textbf{Infinite loop}: If condition never becomes false
\end{itemize}

\textbf{Flowchart Structure:}

\begin{verbatim}
flowchart LR
    A[Initialize] {-{-} B\{Condition?\}}
    B {-{-}|True| C[Execute Statements]}
    C {-{-} D[Update Variable]}
    D {-{-} B}
    B {-{-}|False| E[Exit Loop]}
\end{verbatim}

\end{solutionbox}
\begin{mnemonicbox}
``Initialize Check Execute Update'' (ICEU)

\end{mnemonicbox}
\begin{center}\rule{0.5\linewidth}{0.5pt}\end{center}

\subsection*{Question 3(c) OR [7
marks]}\label{q3c}

\textbf{WAP to read an integer from key board and print whether given
number is odd or even.}

\begin{solutionbox}

\begin{verbatim}
\#include {stdio.h}

int main() \{
    int number;
    
    printf("Enter an integer: ");
    scanf("\%d", \&number);
    
    if(number \% 2 == 0) \{
        printf("\%d is Even number", number);
    \}
    else \{
        printf("\%d is Odd number", number);
    \}
    
    return 0;
\}
\end{verbatim}

\textbf{Logic Explanation:}

{\def\LTcaptype{none} % do not increment counter
\begin{longtable}[]{@{}ll@{}}
\toprule\noalign{}
Concept & Description \\
\midrule\noalign{}
\endhead
\bottomrule\noalign{}
\endlastfoot
\textbf{Modulus operator (\%)} & Returns remainder after division \\
\textbf{Even condition} & number \% 2 == 0 \\
\textbf{Odd condition} & number \% 2 != 0 \\
\end{longtable}
}

\textbf{Alternative Methods:}

\begin{verbatim}
// Method 2: Using conditional operator
printf("\%d is \%s", number, (number \% 2 == 0) ? "Even" : "Odd");

// Method 3: Using bitwise AND
if(number \& 1)
    printf("Odd");
else
    printf("Even");
\end{verbatim}

\textbf{Sample Output:}

\begin{verbatim}
Enter an integer: 7
7 is Odd number
\end{verbatim}

\end{solutionbox}
\begin{mnemonicbox}
``Modulus Two Zero Even'' (MTZE)

\end{mnemonicbox}
\begin{center}\rule{0.5\linewidth}{0.5pt}\end{center}

\subsection*{Question 4(a) [3 marks]}\label{q4a}

**Evaluate following arithmetic expressions: 30/4*4 -- 20\%6 + 17/2**

\begin{solutionbox}

\textbf{Step-by-step Evaluation:}

{\def\LTcaptype{none} % do not increment counter
\begin{longtable}[]{@{}llll@{}}
\toprule\noalign{}
Step & Expression & Calculation & Result \\
\midrule\noalign{}
\endhead
\bottomrule\noalign{}
\endlastfoot
1 & 30/4*4 & (30/4)\emph{4 = 7}4 & 28 \\
2 & 20\%6 & 20 mod 6 & 2 \\
3 & 17/2 & Integer division & 8 \\
4 & Final & 28 - 2 + 8 & 34 \\
\end{longtable}
}

\textbf{Operator Precedence:}

{\def\LTcaptype{none} % do not increment counter
\begin{longtable}[]{@{}ll@{}}
\toprule\noalign{}
Priority & Operators \\
\midrule\noalign{}
\endhead
\bottomrule\noalign{}
\endlastfoot
High & *, /, \% (Left to right) \\
Low & +, - (Left to right) \\
\end{longtable}
}

\textbf{Complete Calculation:}

\begin{verbatim}
30/4*4 – 20%6 + 17/2
= 7*4 - 2 + 8      // Division and modulus first
= 28 - 2 + 8       // Multiplication
= 26 + 8           // Left to right for +,-
= 34               // Final answer
\end{verbatim}

\end{solutionbox}
\begin{mnemonicbox}
``Multiply Divide Before Add Subtract'' (MDBAS)

\end{mnemonicbox}
\begin{center}\rule{0.5\linewidth}{0.5pt}\end{center}

\subsection*{Question 4(b) [4 marks]}\label{q4b}

\textbf{WAP to find sum and average of an array of 5 integer numbers.}

\begin{solutionbox}

\begin{verbatim}
\#include {stdio.h}

int main() \{
    int numbers[5];
    int sum = 0;
    float average;
    
    printf("Enter 5 integers:{n}");
    for(int i = 0; i {} 5; i++) \{
        scanf("\%d", \&numbers[i]);
        sum += numbers[i];
    \}
    
    average = (float)sum / 5;
    
    printf("Sum = \%d{n}", sum);
    printf("Average = \%.2f", average);
    
    return 0;
\}
\end{verbatim}

\textbf{Algorithm:}

{\def\LTcaptype{none} % do not increment counter
\begin{longtable}[]{@{}ll@{}}
\toprule\noalign{}
Step & Action \\
\midrule\noalign{}
\endhead
\bottomrule\noalign{}
\endlastfoot
1 & Declare array of 5 integers \\
2 & Initialize sum to 0 \\
3 & Input 5 numbers using loop \\
4 & Add each number to sum \\
5 & Calculate average = sum/5 \\
6 & Display results \\
\end{longtable}
}

\textbf{Key Points:}

\begin{itemize}
\tightlist
\item
  \textbf{Type casting}: (float)sum for accurate division
\item
  \textbf{Loop usage}: Efficient for repetitive input
\end{itemize}

\end{solutionbox}
\begin{mnemonicbox}
``Declare Input Add Calculate Display'' (DIACD)

\end{mnemonicbox}
\begin{center}\rule{0.5\linewidth}{0.5pt}\end{center}

\subsection*{Question 4(c) [7 marks]}\label{q4c}

\textbf{Define pointer. Explain how pointers are declared and
initialized with example.}

\begin{solutionbox}

\textbf{Pointer Definition:}

{\def\LTcaptype{none} % do not increment counter
\begin{longtable}[]{@{}ll@{}}
\toprule\noalign{}
Aspect & Description \\
\midrule\noalign{}
\endhead
\bottomrule\noalign{}
\endlastfoot
Definition & Variable that stores memory address of another variable \\
Purpose & Direct memory access and dynamic memory allocation \\
Symbol & * (asterisk) for declaration and dereferencing \\
\end{longtable}
}

\textbf{Declaration and Initialization:}

\begin{verbatim}
// Declaration
int *ptr;           // Pointer to integer
float *fptr;        // Pointer to float
char *cptr;         // Pointer to character

// Initialization
int num = 10;
int *ptr = \&num;    // Initialize with address of num

// Alternative
int *ptr;
ptr = \&num;         // Assign address later
\end{verbatim}

\textbf{Example Program:}

\begin{verbatim}
\#include {stdio.h}

int main() \{
    int num = 25;
    int *ptr = \&num;
    
    printf("Value of num: \%d{n}", num);
    printf("Address of num: \%p{n}", \&num);
    printf("Value of ptr: \%p{n}", ptr);
    printf("Value pointed by ptr: \%d{n}", *ptr);
    
    return 0;
\}
\end{verbatim}

\textbf{Key Operators:}

\begin{itemize}
\tightlist
\item
  \textbf{\& (Address-of)}: Gets address of variable
\item
  \textbf{* (Dereference)}: Gets value at address
\end{itemize}

\textbf{Memory Diagram:}

\begin{verbatim}
num: [25] at address 1000
ptr: [1000] at address 2000
\end{verbatim}

\end{solutionbox}
\begin{mnemonicbox}
``Address Star Dereference'' (ASD)

\end{mnemonicbox}
\begin{center}\rule{0.5\linewidth}{0.5pt}\end{center}

\subsection*{Question 4(a) OR [3
marks]}\label{q4a}

\textbf{Evaluate following arithmetic expressions: 50 / 3 \% 3 + 5 * 7}

\begin{solutionbox}

\textbf{Step-by-step Evaluation:}

{\def\LTcaptype{none} % do not increment counter
\begin{longtable}[]{@{}llll@{}}
\toprule\noalign{}
Step & Expression & Calculation & Result \\
\midrule\noalign{}
\endhead
\bottomrule\noalign{}
\endlastfoot
1 & 50/3 & Integer division & 16 \\
2 & 16\%3 & 16 mod 3 & 1 \\
3 & 5*7 & Multiplication & 35 \\
4 & Final & 1 + 35 & 36 \\
\end{longtable}
}

\textbf{Complete Calculation:}

\begin{verbatim}
50 / 3 % 3 + 5 * 7
= 16 % 3 + 35      // Division and multiplication first
= 1 + 35           // Modulus operation
= 36               // Final answer
\end{verbatim}

\textbf{Operator Precedence Applied:}

\begin{itemize}
\tightlist
\item
  \textbf{High priority}: /, \%, * (left to right)
\item
  \textbf{Low priority}: + (left to right)
\end{itemize}

\end{solutionbox}
\begin{mnemonicbox}
``Divide Mod Multiply Add'' (DMMA)

\end{mnemonicbox}
\begin{center}\rule{0.5\linewidth}{0.5pt}\end{center}

\subsection*{Question 4(b) OR [4
marks]}\label{q4b}

\textbf{WAP to find the largest number in an array of N integers.}

\begin{solutionbox}

\begin{verbatim}
\#include {stdio.h}

int main() \{
    int n, i;
    int largest;
    
    printf("Enter number of elements: ");
    scanf("\%d", \&n);
    
    int arr[n];
    
    printf("Enter \%d numbers:{n}", n);
    for(i = 0; i {} n; i++) \{
        scanf("\%d", \&arr[i]);
    \}
    
    largest = arr[0];  // Assume first element is largest
    
    for(i = 1; i {} n; i++) \{
        if(arr[i] {} largest) \{
            largest = arr[i];
        \}
    \}
    
    printf("Largest number is: \%d", largest);
    
    return 0;
\}
\end{verbatim}

\textbf{Algorithm:}

{\def\LTcaptype{none} % do not increment counter
\begin{longtable}[]{@{}ll@{}}
\toprule\noalign{}
Step & Action \\
\midrule\noalign{}
\endhead
\bottomrule\noalign{}
\endlastfoot
1 & Input array size \\
2 & Input array elements \\
3 & Assume first element as largest \\
4 & Compare with remaining elements \\
5 & Update largest if bigger found \\
6 & Display result \\
\end{longtable}
}

\end{solutionbox}
\begin{mnemonicbox}
``Input Assume Compare Update Display'' (IACUD)

\end{mnemonicbox}
\begin{center}\rule{0.5\linewidth}{0.5pt}\end{center}

\subsection*{Question 4(c) OR [7
marks]}\label{q4c}

\textbf{Define array. Explain the need for array variable. Explain 1-D
array with example}

\begin{solutionbox}

\textbf{Array Definition:}

{\def\LTcaptype{none} % do not increment counter
\begin{longtable}[]{@{}ll@{}}
\toprule\noalign{}
Aspect & Description \\
\midrule\noalign{}
\endhead
\bottomrule\noalign{}
\endlastfoot
Definition & Collection of similar data type elements \\
Storage & Consecutive memory locations \\
Access & Using index/subscript \\
\end{longtable}
}

\textbf{Need for Arrays:}

{\def\LTcaptype{none} % do not increment counter
\begin{longtable}[]{@{}ll@{}}
\toprule\noalign{}
Problem & Solution with Array \\
\midrule\noalign{}
\endhead
\bottomrule\noalign{}
\endlastfoot
Store multiple values & Single array variable \\
Avoid multiple variables & arr[100] instead of a1, a2, \ldots,
a100 \\
Efficient processing & Loop-based operations \\
Memory organization & Contiguous allocation \\
\end{longtable}
}

\textbf{1-D Array Declaration:}

\begin{verbatim}
datatype arrayname[size];

// Examples
int marks[5];           // Array of 5 integers
float prices[10];       // Array of 10 floats
char name[20];         // Array of 20 characters
\end{verbatim}

\textbf{Array Initialization:}

\begin{verbatim}
// Method 1: At declaration
int numbers[5] = \{10, 20, 30, 40, 50\;}

// Method 2: Individual assignment
int arr[3];
arr[0] = 5;
arr[1] = 15;
arr[2] = 25;
\end{verbatim}

\textbf{Example Program:}

\begin{verbatim}
\#include {stdio.h}

int main() \{
    int marks[5] = \{85, 90, 78, 92, 88\;}
    int i, sum = 0;
    
    printf("Student marks:{n}");
    for(i = 0; i {} 5; i++) \{
        printf("Subject \%d: \%d{n}", i+1, marks[i]);
        sum += marks[i];
    \}
    
    printf("Total marks: \%d", sum);
    return 0;
\}
\end{verbatim}

\textbf{Memory Layout:}

\begin{verbatim}
marks[0] marks[1] marks[2] marks[3] marks[4]
  [85]     [90]     [78]     [92]     [88]
 1000     1004     1008     1012     1016
\end{verbatim}

\end{solutionbox}
\begin{mnemonicbox}
``Similar Data Consecutive Index'' (SDCI)

\end{mnemonicbox}
\begin{center}\rule{0.5\linewidth}{0.5pt}\end{center}

\subsection*{Question 5(a) [3 marks]}\label{q5a}

\textbf{Give an example of if \ldots{} else statement.}

\begin{solutionbox}

\textbf{If-else Example:}

\begin{verbatim}
\#include {stdio.h}

int main() \{
    int age;
    
    printf("Enter your age: ");
    scanf("\%d", \&age);
    
    if(age {=} 18) \{
        printf("You are eligible to vote");
    \}
    else \{
        printf("You are not eligible to vote");
    \}
    
    return 0;
\}
\end{verbatim}

\textbf{Structure:}

{\def\LTcaptype{none} % do not increment counter
\begin{longtable}[]{@{}ll@{}}
\toprule\noalign{}
Component & Purpose \\
\midrule\noalign{}
\endhead
\bottomrule\noalign{}
\endlastfoot
\textbf{if} & Tests condition \\
\textbf{condition} & Boolean expression \\
\textbf{if-block} & Executes when condition true \\
\textbf{else-block} & Executes when condition false \\
\end{longtable}
}

\textbf{Sample Outputs:}

\begin{verbatim}
Input: 20    Output: You are eligible to vote
Input: 16    Output: You are not eligible to vote
\end{verbatim}

\end{solutionbox}
\begin{mnemonicbox}
``If True Else False'' (ITEF)

\end{mnemonicbox}
\begin{center}\rule{0.5\linewidth}{0.5pt}\end{center}

\subsection*{Question 5(b) [4 marks]}\label{q5b}

\textbf{WAP to check the category of given character.}

\begin{solutionbox}

\begin{verbatim}
\#include {stdio.h}
\#include {ctype.h}

int main() \{
    char ch;
    
    printf("Enter a character: ");
    scanf("\%c", \&ch);
    
    if(isdigit(ch)) \{
        printf("{}\%c{ is a Digit"}, ch);
    \}
    else if(isupper(ch)) \{
        printf("{}\%c{ is an Uppercase letter"}, ch);
    \}
    else if(islower(ch)) \{
        printf("{}\%c{ is a Lowercase letter"}, ch);
    \}
    else \{
        printf("{}\%c{ is a Special symbol"}, ch);
    \}
    
    return 0;
\}
\end{verbatim}

\textbf{Character Categories:}

{\def\LTcaptype{none} % do not increment counter
\begin{longtable}[]{@{}lll@{}}
\toprule\noalign{}
Function & Category & Range \\
\midrule\noalign{}
\endhead
\bottomrule\noalign{}
\endlastfoot
isdigit() & Digit & 0-9 \\
isupper() & Uppercase & A-Z \\
islower() & Lowercase & a-z \\
Others & Special symbols & !@\#\$\%\^{}\&* etc. \\
\end{longtable}
}

\textbf{Alternative Method:}

\begin{verbatim}
if(ch {=} {0} \&\& ch {=} {9})
    printf("Digit");
else if(ch {=} {A} \&\& ch {=} {Z})
    printf("Uppercase");
else if(ch {=} {a} \&\& ch {=} {z})
    printf("Lowercase");
else
    printf("Special symbol");
\end{verbatim}

\end{solutionbox}
\begin{mnemonicbox}
``Digit Upper Lower Special'' (DULS)

\end{mnemonicbox}
\begin{center}\rule{0.5\linewidth}{0.5pt}\end{center}

\subsection*{Question 5(c) [7 marks]}\label{q5c}

\textbf{What is structure? Explain its syntax with suitable example}

\begin{solutionbox}

\textbf{Structure Definition:}

{\def\LTcaptype{none} % do not increment counter
\begin{longtable}[]{@{}ll@{}}
\toprule\noalign{}
Aspect & Description \\
\midrule\noalign{}
\endhead
\bottomrule\noalign{}
\endlastfoot
Definition & User-defined data type combining different data types \\
Purpose & Group related data under single name \\
Keyword & struct \\
\end{longtable}
}

\textbf{Syntax:}

\begin{verbatim}
struct structure\_name \{
    datatype member1;
    datatype member2;
    ...
\;}
\end{verbatim}

\textbf{Example - Student Structure:}

\begin{verbatim}
\#include {stdio.h}

struct Student \{
    int roll\_no;
    char name[50];
    float marks;
    char grade;
\;}

int main() \{
    struct Student s1;
    
    // Input data
    printf("Enter roll number: ");
    scanf("\%d", \&s1.roll\_no);
    
    printf("Enter name: ");
    scanf("\%s", s1.name);
    
    printf("Enter marks: ");
    scanf("\%f", \&s1.marks);
    
    printf("Enter grade: ");
    scanf(" \%c", \&s1.grade);
    
    // Display data
    printf("{n}Student Details:{n}");
    printf("Roll No: \%d{n}", s1.roll\_no);
    printf("Name: \%s{n}", s1.name);
    printf("Marks: \%.2f{n}", s1.marks);
    printf("Grade: \%c{n}", s1.grade);
    
    return 0;
\}
\end{verbatim}

\textbf{Structure Features:}

{\def\LTcaptype{none} % do not increment counter
\begin{longtable}[]{@{}ll@{}}
\toprule\noalign{}
Feature & Description \\
\midrule\noalign{}
\endhead
\bottomrule\noalign{}
\endlastfoot
\textbf{Dot operator (.)} & Access structure members \\
\textbf{Memory allocation} & Total size = sum of all members \\
\textbf{Initialization} & Can initialize at declaration \\
\end{longtable}
}

\textbf{Structure Initialization:}

\begin{verbatim}
struct Student s1 = \{101, "John", 85.5, {A}\;}
\end{verbatim}

\textbf{Memory Layout:}

\begin{verbatim}
s1: [roll_no][name...][marks][grade]
     4 bytes  50 bytes 4 bytes 1 byte
\end{verbatim}

\end{solutionbox}
\begin{mnemonicbox}
``Group Related Data Together'' (GRDT)

\end{mnemonicbox}
\begin{center}\rule{0.5\linewidth}{0.5pt}\end{center}

\subsection*{Question 5(a) OR [3
marks]}\label{q5a}

\textbf{WAP to Print all numbers between -5 \& +5.}

\begin{solutionbox}

\begin{verbatim}
\#include {stdio.h}

int main() \{
    int i;
    
    printf("Numbers between {-5 and +5:}{n}");
    
    for(i = {-}5; i {=} 5; i++) \{
        printf("\%d ", i);
    \}
    
    return 0;
\}
\end{verbatim}

\textbf{Output:}

\begin{verbatim}
Numbers between -5 and +5:
-5 -4 -3 -2 -1 0 1 2 3 4 5
\end{verbatim}

\textbf{Alternative Methods:}

\begin{verbatim}
// Method 2: Using while loop
int i = {-}5;
while(i {=} 5) \{
    printf("\%d ", i);
    i++;
\}

// Method 3: Two separate loops
for(i = {-}5; i {} 0; i++)
    printf("\%d ", i);
printf("0 ");
for(i = 1; i {=} 5; i++)
    printf("\%d ", i);
\end{verbatim}

\end{solutionbox}
\begin{mnemonicbox}
``Start Negative End Positive'' (SNEP)

\end{mnemonicbox}
\begin{center}\rule{0.5\linewidth}{0.5pt}\end{center}

\subsection*{Question 5(b) OR [4
marks]}\label{q5b}

\textbf{WAP to find roots of quadratic equation.}

\begin{solutionbox}

\begin{verbatim}
\#include {stdio.h}
\#include {math.h}

int main() \{
    float a, b, c;
    float discriminant, root1, root2;
    
    printf("Enter coefficients (a, b, c): ");
    scanf("\%f \%f \%f", \&a, \&b, \&c);
    
    discriminant = b*b {-} 4*a*c;
    
    if(discriminant {} 0) \{
        root1 = ({-}b + sqrt(discriminant)) / (2*a);
        root2 = ({-}b {-} sqrt(discriminant)) / (2*a);
        printf("Roots are real and different{n}");
        printf("Root1 = \%.2f{n}", root1);
        printf("Root2 = \%.2f{n}", root2);
    \}
    else if(discriminant == 0) \{
        root1 = {-}b / (2*a);
        printf("Roots are real and equal{n}");
        printf("Root = \%.2f{n}", root1);
    \}
    else \{
        float realPart = {-}b / (2*a);
        float imagPart = sqrt({-}discriminant) / (2*a);
        printf("Roots are complex{n}");
        printf("Root1 = \%.2f + \%.2fi{n}", realPart, imagPart);
        printf("Root2 = \%.2f {- }\%.2fi{n}", realPart, imagPart);
    \}
    
    return 0;
\}
\end{verbatim}

\textbf{Quadratic Formula Analysis:}

{\def\LTcaptype{none} % do not increment counter
\begin{longtable}[]{@{}ll@{}}
\toprule\noalign{}
Discriminant & Nature of Roots \\
\midrule\noalign{}
\endhead
\bottomrule\noalign{}
\endlastfoot
\textbf{b^{2}-4ac \textgreater{} 0} & Real and different \\
\textbf{b^{2}-4ac = 0} & Real and equal \\
\textbf{b^{2}-4ac \textless{} 0} & Complex (imaginary) \\
\end{longtable}
}

\textbf{Formula:} x = (-b \pm \sqrt(b^{2}-4ac)) / 2a

\textbf{Sample Output:}

\begin{verbatim}
Enter coefficients: 1 -7 12
Roots are real and different
Root1 = 4.00
Root2 = 3.00
\end{verbatim}

\end{solutionbox}
\begin{mnemonicbox}
``Discriminant Decides Root Nature'' (DDRN)

\end{mnemonicbox}
\begin{center}\rule{0.5\linewidth}{0.5pt}\end{center}

\subsection*{Question 5(c) OR [7
marks]}\label{q5c}

\textbf{Explain following built-in functions with examples}

\begin{solutionbox}

\textbf{Function Explanations:}

{\def\LTcaptype{none} % do not increment counter
\begin{longtable}[]{@{}llll@{}}
\toprule\noalign{}
Function & Purpose & Header File & Example \\
\midrule\noalign{}
\endhead
\bottomrule\noalign{}
\endlastfoot
clrscr() & Clear screen & conio.h & clrscr(); \\
sqrt() & Square root & math.h & sqrt(16) = 4.0 \\
strlen() & String length & string.h & strlen(``Hello'') = 5 \\
isdigit() & Check if digit & ctype.h & isdigit(`5') = true \\
isalpha() & Check if alphabet & ctype.h & isalpha(`A') = true \\
toupper() & Convert to uppercase & ctype.h & toupper(`a') = `A' \\
tolower() & Convert to lowercase & ctype.h & tolower(`B') = `b' \\
\end{longtable}
}

\textbf{Example Program:}

\begin{verbatim}
\#include {stdio.h}
\#include {conio.h}
\#include {math.h}
\#include {string.h}
\#include {ctype.h}

int main() \{
    clrscr();  // Clear screen
    
    // sqrt() example
    float num = 25.0;
    printf("Square root of \%.1f = \%.2f{n}", num, sqrt(num));
    
    // strlen() example
    char str[] = "Programming";
    printf("Length of {}\%s{ = }\%d{n}", str, strlen(str));
    
    // Character functions
    char ch = {a};
    printf("{}\%c{ is digit: }\%s{n}", ch, isdigit(ch) ? "Yes" : "No");
    printf("{}\%c{ is alphabet: }\%s{n}", ch, isalpha(ch) ? "Yes" : "No");
    printf("Uppercase of {}\%c{ = }\%c{}{n}", ch, toupper(ch));
    
    ch = {B};
    printf("Lowercase of {}\%c{ = }\%c{}{n}", ch, tolower(ch));
    
    return 0;
\}
\end{verbatim}

\textbf{Function Categories:}

\begin{center}
\textbf{Mermaid Diagram (Code)}
\begin{verbatim}
{Shaded}
{Highlighting}[]
graph TD
    A[Built{-in Functions] {-}{-}{} B[Screen Control]}
    A {-{-}{} C[Mathematical]}
    A {-{-}{} D[String]}
    A {-{-}{} E[Character]}
    
    B {-{-}{} F["clrscr()"]}
    C {-{-}{} G["sqrt()"]}
    D {-{-}{} H["strlen()"]}
    E {-{-}{} I["isdigit(), isalpha()"]}
    E {-{-}{} J["toupper(), tolower()"]}
{Highlighting}
{Shaded}
\end{verbatim}
\end{center}

\textbf{Key Points:}

\begin{itemize}
\tightlist
\item
  \textbf{Header files}: Must include appropriate headers
\item
  \textbf{Return values}: Most functions return specific types
\item
  \textbf{Parameter types}: Check function parameter requirements
\end{itemize}

\end{solutionbox}
\begin{mnemonicbox}
``Clear Math String Character'' (CMSC)

\end{mnemonicbox}

\end{document}
