\documentclass{article}

% content/resources/templates/preamble.tex
\usepackage[margin=0.6in]{geometry}
\author{Milav Dabgar}
\usepackage{amsmath,amssymb,amsthm}
\usepackage{booktabs}
\usepackage{multirow}
\usepackage{xcolor}
\usepackage{tcolorbox}
\tcbuselibrary{breakable,skins}
\usepackage[colorlinks=true,linkcolor=blue]{hyperref}
\usepackage{titlesec}
\usepackage{enumitem}
\usepackage{tikz}
\usepackage{pgfplots}
\usepackage{circuitikz}
\usepackage[version=4]{mhchem}
\usepackage{longtable}
\usepackage{array}
\usepackage{float}
\usepackage{caption}
\usepackage{listings}

\lstset{
  basicstyle=\small\ttfamily,
  breaklines=true,
  breakatwhitespace=false,
  postbreak=\mbox{\textcolor{red}{$\hookrightarrow$}\space},
  float=false,
  numbers=left,
  numberstyle=\tiny\color{gray},
  numbersep=10pt,
  xleftmargin=2em,
  keywordstyle=\color{blue},
  commentstyle=\color{green!60!black},
  stringstyle=\color{purple},
  backgroundcolor=\color{gray!5},
  showstringspaces=false,
  tabsize=2,
  captionpos=b,
  keepspaces=true,
  columns=flexible
}

\pgfplotsset{compat=1.18}
\usetikzlibrary{shapes,arrows,positioning,calc,patterns,decorations.pathmorphing,decorations.markings,arrows.meta}

% Color scheme
\definecolor{headcolor}{RGB}{0,102,204}
\definecolor{keycolor}{RGB}{220,20,60}
\definecolor{solutioncolor}{RGB}{34,139,34}
\definecolor{mnemoniccolor}{RGB}{148,0,211}
\definecolor{codecolor}{RGB}{0,0,100}

% Spacing
\setlength{\parskip}{3pt}
\setlist[itemize]{nosep}
\setlist[enumerate]{nosep}

% Title formatting
\titleformat{\section}{\Large\bfseries\color{headcolor}}{\thesection}{1em}{}
\titleformat{\subsection}{\large\bfseries\color{headcolor}}{\thesubsection}{1em}{}

% Pandoc tightlist compatibility
\providecommand{\tightlist}{%
  \setlength{\itemsep}{0pt}\setlength{\parskip}{0pt}}

% Pandoc longtable compatibility
\newcounter{none}
\def\thenone{}


% content/resources/templates/english-boxes.tex

% Custom environments
\newtcolorbox{solutionbox}{
 breakable,
 enhanced,
 colback=solutioncolor!5!white,
 colframe=solutioncolor!75!black,
 fonttitle=\bfseries,
 title=Solution
}

\newtcolorbox{solutionboxnobreak}{
 colback=solutioncolor!5!white,
 colframe=solutioncolor!75!black,
 fonttitle=\bfseries,
 title=Solution
}

\newtcolorbox{keyformula}{
 breakable,
 enhanced,
 colback=keycolor!5!white,
 colframe=keycolor!75!black,
 fonttitle=\bfseries,
 title=Key Formula
}

\newtcolorbox{mnemonicboxenv}{
 breakable,
 enhanced,
 colback=mnemoniccolor!5!white,
 colframe=mnemoniccolor!75!black,
 fonttitle=\bfseries,
 title=Mnemonic
}

\newcommand{\mnemonicbox}[1]{%
  \begin{mnemonicboxenv}
    #1
  \end{mnemonicboxenv}
}


% Custom commands for GTU solutions
% This file defines semantic commands for consistent formatting

% Question command with automatic formatting
\newcommand{\question}[2]{%
  \section*{Question #1}%
  \textbf{#2}%
}

% OR question variant
\newcommand{\questionor}[2]{%
  \section*{Question #1 OR}%
  \textbf{#2}%
}

% Proper table environment with caption
\newenvironment{answertable}[1]{%
  \begin{table}[htbp]
  \centering
  \caption{#1}
}{%
  \end{table}
}

% Proper figure environment for diagrams
\newenvironment{answerdiagram}[1]{%
  \begin{figure}[htbp]
  \centering
  \caption{#1}
}{%
  \end{figure}
}

% Semantic markup for key terms
\newcommand{\keyword}[1]{\textbf{#1}}
\newcommand{\code}[1]{\texttt{#1}}
\newcommand{\classname}[1]{\texttt{#1}}
\newcommand{\methodname}[1]{\texttt{#1}}

% Proper quotation marks
\newcommand{\mnemonic}[1]{``#1''}


\title{Programming in C (4331105) - Winter 2023 Solution}
\date{May 21, 2024}

\begin{document}
\maketitle

\questionmarks{1}{a}{3}
\textbf{Define algorithm and write an algorithm to find area of circle.}

\begin{solutionbox}
    \textbf{Answer}:
    An algorithm is a step-by-step procedure or set of rules for solving a specific problem or accomplishing a particular task.

    \textbf{Algorithm to find area of circle:}

    \begin{enumerate}
        \item Start
        \item Input radius (r) of the circle
        \item Calculate area = $\pi \times r^2$
        \item Display the area
        \item Stop
    \end{enumerate}

    \begin{mnemonicbox}"Start, Read, Calculate, Display, Stop"\end{mnemonicbox}
\end{solutionbox}

\questionmarks{1}{b}{4}
\textbf{Define flowchart and draw a flowchart to find minimum of three numbers.}

\begin{solutionbox}
    \textbf{Answer}:
    A flowchart is a visual representation of an algorithm using standardized symbols and shapes connected by arrows to show the sequence of steps.

    \textbf{Flowchart to find minimum of three numbers:}

    \begin{center}
    \begin{tikzpicture}[gtu flow]
        \node[gtu start] (start) {Start};
        \node[gtu input, below=of start] (input) {Input A, B, C};
        \node[gtu decision, below=of input] (cond1) {A $<$ B?};
        
        \node[gtu decision, below left=of cond1, xshift=-1cm] (cond2) {A $<$ C?};
        \node[gtu decision, below right=of cond1, xshift=1cm] (cond3) {B $<$ C?};
        
        \node[gtu process, below left=of cond2] (minA) {min = A};
        \node[gtu process, below right=of cond2] (minC1) {min = C};
        
        \node[gtu process, below left=of cond3] (minB) {min = B};
        \node[gtu process, below right=of cond3] (minC2) {min = C};
        
        \node[gtu output, below=of cond2, yshift=-3cm] (print) {Display min};
        \node[gtu stop, below=of print] (stop) {Stop};

        \draw[gtu arrow] (start) -- (input);
        \draw[gtu arrow] (input) -- (cond1);
        \draw[gtu arrow] (cond1) -| node[above] {Yes} (cond2);
        \draw[gtu arrow] (cond1) -| node[above] {No} (cond3);
        
        \draw[gtu arrow] (cond2) -| node[above] {Yes} (minA);
        \draw[gtu arrow] (cond2) -| node[above] {No} (minC1);
        
        \draw[gtu arrow] (cond3) -| node[above] {Yes} (minB);
        \draw[gtu arrow] (cond3) -| node[above] {No} (minC2);
        
        \draw[gtu arrow] (minA) |- (print);
        \draw[gtu arrow] (minC1) |- (print);
        \draw[gtu arrow] (minB) |- (print);
        \draw[gtu arrow] (minC2) |- (print);
        \draw[gtu arrow] (print) -- (stop);
    \end{tikzpicture}
    \end{center}

    \begin{itemize}
        \item \textbf{Comparison Strategy}: First compare A and B, then compare with C
        \item \textbf{Branching Logic}: Use if-else structure to find smallest value
    \end{itemize}

    \begin{mnemonicbox}"Compare pairs, find the rare small value everywhere"\end{mnemonicbox}
\end{solutionbox}

\questionmarks{1}{c}{7}
\textbf{Write a program to calculate simple interest using below equation. I=PRN/100 Where P=Principle amount, R=Rate of interest and N=Period.}

\begin{solutionbox}
    \textbf{Answer}:

\begin{lstlisting}[language=C]
#include <stdio.h>

int main() {
    float P, R, N, I;
    
    // Input principal amount, rate of interest and time period
    printf("Enter Principal amount: ");
    scanf("%f", &P);
    
    printf("Enter Rate of interest: ");
    scanf("%f", &R);
    
    printf("Enter Time period (in years): ");
    scanf("%f", &N);
    
    // Calculate Simple Interest
    I = (P * R * N) / 100;
    
    // Display the result
    printf("Simple Interest = %.2f\n", I);
    
    return 0;
}
\end{lstlisting}

    \textbf{Diagram:}

    \begin{center}
    \begin{tikzpicture}[gtu flow]
        \node[gtu block] (formula) {Formula: $I = \frac{P \times R \times N}{100}$};
        \node[gtu input, left=of formula, yshift=1cm] (P) {P};
        \node[gtu input, left=of formula] (R) {R};
        \node[gtu input, left=of formula, yshift=-1cm] (N) {N};
        \node[gtu output, right=of formula] (I) {I};

        \draw[gtu arrow] (P) -- (formula);
        \draw[gtu arrow] (R) -- (formula);
        \draw[gtu arrow] (N) -- (formula);
        \draw[gtu arrow] (formula) -- (I);
    \end{tikzpicture}
    \end{center}

    \begin{itemize}
        \item \textbf{Floating-point variables}: Store decimal values for precision
        \item \textbf{User interaction}: Clear prompts for input
        \item \textbf{Result formatting}: \%.2f displays two decimal places
    \end{itemize}

    \begin{mnemonicbox}"Principal, Rate and Number, divided by Hundred gives Interest"\end{mnemonicbox}
\end{solutionbox}

\orquestionmarks{1}{c}{7}
\textbf{Write a program to read radius(R) and height(H) from keyboard and print calculated the volume(V) of cylinder using V=$\pi$R$^2$H}

\begin{solutionbox}
    \textbf{Answer}:

\begin{lstlisting}[language=C]
#include <stdio.h>

int main() {
    float radius, height, volume;
    const float PI = 3.14159;
    
    // Input radius and height
    printf("Enter radius of cylinder: ");
    scanf("%f", &radius);
    
    printf("Enter height of cylinder: ");
    scanf("%f", &height);
    
    // Calculate volume of cylinder
    volume = PI * radius * radius * height;
    
    // Display the result
    printf("Volume of cylinder = %.2f\n", volume);
    
    return 0;
}
\end{lstlisting}

    \textbf{Diagram:}

    \begin{center}
    \begin{tikzpicture}[gtu flow]
        \node[gtu input] (input) {Input radius, height};
        \node[gtu process, right=of input] (calc) {Volume = $\pi \times r^2 \times h$};
        \node[gtu output, right=of calc] (output) {Display volume};

        \draw[gtu arrow] (input) -- (calc);
        \draw[gtu arrow] (calc) -- (output);
    \end{tikzpicture}
    \end{center}

    \begin{itemize}
        \item \textbf{Constants}: PI defined as constant for clarity
        \item \textbf{Formula}: Use R$^2$ by multiplying radius twice
        \item \textbf{Input validation}: Assumes positive values for radius and height
    \end{itemize}

    \begin{mnemonicbox}"Radius squared times height times Pi, gives cylinder volume, don't ask why"\end{mnemonicbox}
\end{solutionbox}

\questionmarks{2}{a}{3}
\textbf{List out different operators supported in C programming language.}

\begin{solutionbox}
    \textbf{Answer}:

    \begin{tabulary}{\linewidth}{|L|L|}
        \hline
        \textbf{Category} & \textbf{Operators} \\
        \hline
        Arithmetic & +, -, *, /, \% (addition, subtraction, multiplication, division, modulus) \\
        \hline
        Relational & ==, !=, $>$, $<$, $>=$, $<=$ (equal, not equal, greater than, less than, greater than or equal to, less than or equal to) \\
        \hline
        Logical & \&\&, $||$, ! (AND, OR, NOT) \\
        \hline
        Assignment & =, +=, -=, *=, /=, \%= (assign, plus-assign, minus-assign, etc.) \\
        \hline
        Increment/Decrement & ++, -- (increment, decrement) \\
        \hline
        Bitwise & \&, $|$, \^{}, \~{}, $<<$, $>>$ (AND, OR, XOR, complement, left shift, right shift) \\
        \hline
        Conditional & ? : (ternary operator) \\
        \hline
        Special & sizeof(), \&, *, -$>$, . (size, address, pointer, structure) \\
        \hline
    \end{tabulary}

    \begin{mnemonicbox}"ARABIA CS" (Arithmetic, Relational, Assignment, Bitwise, Increment, Assignment, Conditional, Special)\end{mnemonicbox}
\end{solutionbox}

\questionmarks{2}{b}{4}
\textbf{Explain Relational operator and Increment/Decrement operator with example.}

\begin{solutionbox}
    \textbf{Answer}:

    \begin{tabulary}{\linewidth}{|L|L|L|L|}
        \hline
        \textbf{Operator Type} & \textbf{Description} & \textbf{Example} & \textbf{Output} \\
        \hline
        Relational & Compare two values to test the relationship between them & \code{int a = 5, b = 10;} \code{printf("\%d", a < b);} & \code{1} (true) \\
        \hline
         & Equal to (==) & \code{printf("\%d", 5 == 5);} & \code{1} (true) \\
        \hline
         & Not equal to (!=) & \code{printf("\%d", 5 != 10);} & \code{1} (true) \\
        \hline
         & Greater/Less than & \code{printf("\%d \%d", 5 > 3, 5 < 3);} & \code{1 0} \\
        \hline
        Increment & Increases value by 1. Pre-increment (++x): increment then use. Post-increment (x++): use then increment & \code{int x = 5;} \code{printf("\%d ", ++x);} \code{printf("\%d", x);} & \code{6 6} \\
        \hline
        Decrement & Decreases value by 1. Pre-decrement (--x): decrement then use. Post-decrement (x--): use then decrement & \code{int y = 5;} \code{printf("\%d ", y--);} \code{printf("\%d", y);} & \code{5 4} \\
        \hline
    \end{tabulary}

    \begin{itemize}
        \item \textbf{Relational operators}: Return 1 (true) or 0 (false)
        \item \textbf{Increment/Decrement}: Changes variable value and returns a value
    \end{itemize}

    \begin{mnemonicbox}"Relational tells if TRUE or LIE, Increment/Decrement makes values rise or DIE"\end{mnemonicbox}
\end{solutionbox}

\questionmarks{2}{c}{7}
\textbf{Write a program to print sum and average of 1 to 100.}

\begin{solutionbox}
    \textbf{Answer}:

\begin{lstlisting}[language=C]
#include <stdio.h>

int main() {
    int i, sum = 0;
    float average;
    
    // Calculate sum of numbers from 1 to 100
    for(i = 1; i <= 100; i++) {
        sum += i;
    }
    
    // Calculate average
    average = (float)sum / 100;
    
    // Display the results
    printf("Sum of numbers from 1 to 100 = %d\n", sum);
    printf("Average of numbers from 1 to 100 = %.2f\n", average);
    
    return 0;
}
\end{lstlisting}

    \textbf{Diagram:}

    \begin{center}
    \begin{tikzpicture}[gtu flow]
        \node[gtu start] (start) {Start};
        \node[gtu process, below=of start] (init) {sum = 0, i = 1};
        \node[gtu decision, below=of init] (cond) {i $\le$ 100?};
        \node[gtu process, below=of cond] (body) {sum += i\\i++};
        \node[gtu process, right=of cond] (avg) {avg = sum / 100};
        \node[gtu output, right=of avg] (output) {Print sum, avg};
        \node[gtu stop, below=of output] (stop) {Stop};

        \draw[gtu arrow] (start) -- (init);
        \draw[gtu arrow] (init) -- (cond);
        \draw[gtu arrow] (cond) -- node[left] {Yes} (body);
        \draw[gtu arrow] (body.west) -- ++(-0.5,0) |- (cond.west);
        \draw[gtu arrow] (cond) -- node[above] {No} (avg);
        \draw[gtu arrow] (avg) -- (output);
        \draw[gtu arrow] (output) -- (stop);
    \end{tikzpicture}
    \end{center}

    \begin{itemize}
        \item \textbf{Loop counter}: Variable i tracks numbers 1 to 100
        \item \textbf{Sum calculation}: Accumulates values in sum variable
        \item \textbf{Type casting}: (float) converts sum to floating-point for accurate division
    \end{itemize}

    \begin{mnemonicbox}"Sum One to Hundred, then Divide for Average"\end{mnemonicbox}
\end{solutionbox}

\orquestionmarks{2}{a}{3}
\textbf{State the difference between gets(S) and scanf("\%s",S) where S is string.}

\begin{solutionbox}
    \textbf{Answer}:

    \begin{tabulary}{\linewidth}{|L|L|L|}
        \hline
        \textbf{Feature} & \textbf{gets(S)} & \textbf{scanf("\%s",S)} \\
        \hline
        Input termination & Reads until newline character (\textbackslash n) & Reads until whitespace (space, tab, newline) \\
        \hline
        Whitespace handling & Can read string with spaces & Stops reading at first whitespace \\
        \hline
        Buffer overflow & No bounds checking (unsafe) & No bounds checking (unsafe) \\
        \hline
        Return value & Returns S on success, NULL on error & Returns number of items successfully read \\
        \hline
        Replacement & fgets() is safer alternative & scanf("\%ns",S) with width limit is safer \\
        \hline
    \end{tabulary}

    \begin{itemize}
        \item \textbf{Safety concern}: Both functions can cause buffer overflow
        \item \textbf{Practical usage}: gets() for full lines, scanf() for single words
    \end{itemize}

    \begin{mnemonicbox}"gets Gets Everything Till newline, scanf Stops Catching After Finding whitespace"\end{mnemonicbox}
\end{solutionbox}

\orquestionmarks{2}{b}{4}
\textbf{Explain Logical operator and Assignment operator with example.}

\begin{solutionbox}
    \textbf{Answer}:

    \begin{tabulary}{\linewidth}{|L|L|L|L|}
        \hline
        \textbf{Operator Type} & \textbf{Description} & \textbf{Example} & \textbf{Output} \\
        \hline
        Logical & Perform logical operations on conditions & \code{int a = 5, b = 10;} & \\
        \hline
         & Logical AND (\&\&) & \code{printf("\%d", (a > 0) \&\& (b > 0));} & \code{1} (true) \\
        \hline
         & Logical OR ($||$) & \code{printf("\%d", (a > 10) || (b > 5));} & \code{1} (true) \\
        \hline
         & Logical NOT (!) & \code{printf("\%d", !(a == b));} & \code{1} (true) \\
        \hline
        Assignment & Assign values to variables & \code{int x = 10;} & \code{x = 10} \\
        \hline
         & Simple assignment (=) & \code{x = 20;} & \code{x = 20} \\
        \hline
         & Add and assign (+=) & \code{x += 5;} & \code{x = 25} \\
        \hline
         & Subtract and assign (-=) & \code{x -= 10;} & \code{x = 15} \\
        \hline
         & Multiply and assign (*=) & \code{x *= 2;} & \code{x = 30} \\
        \hline
         & Divide and assign (/=) & \code{x /= 3;} & \code{x = 10} \\
        \hline
    \end{tabulary}

    \begin{itemize}
        \item \textbf{Logical operators}: Used in decision making
        \item \textbf{Short-circuit evaluation}: \&\& and $||$ evaluate only what's necessary
        \item \textbf{Compound assignment}: Combines operation and assignment
    \end{itemize}

    \begin{mnemonicbox}"AND needs all TRUE, OR needs just one; Assignment takes right, puts it on the left throne"\end{mnemonicbox}
\end{solutionbox}

\orquestionmarks{2}{c}{7}
\textbf{Write a program to print all the integers between given two floating point numbers.}

\begin{solutionbox}
    \textbf{Answer}:

\begin{lstlisting}[language=C]
#include <stdio.h>
#include <math.h>

int main() {
    float num1, num2;
    int start, end, i;
    
    // Input two floating point numbers
    printf("Enter first floating point number: ");
    scanf("%f", &num1);
    
    printf("Enter second floating point number: ");
    scanf("%f", &num2);
    
    // Find the ceil of smaller number and floor of larger number
    if(num1 < num2) {
        start = ceil(num1);
        end = floor(num2);
    } else {
        start = ceil(num2);
        end = floor(num1);
    }
    
    // Print all integers between the two numbers
    printf("Integers between %.2f and %.2f are:\n", num1, num2);
    for(i = start; i <= end; i++) {
        printf("%d ", i);
    }
    printf("\n");
    
    return 0;
}
\end{lstlisting}

    \textbf{Diagram:}

    \begin{center}
    \begin{tikzpicture}[gtu flow]
        \node[gtu input] (input) {Input num1, num2};
        \node[gtu decision, below=of input] (cond) {num1 $<$ num2?};
        \node[gtu process, below left=of cond] (case1) {start = ceil(num1)\\end = floor(num2)};
        \node[gtu process, below right=of cond] (case2) {start = ceil(num2)\\end = floor(num1)};
        \node[gtu process, below=of cond, yshift=-3.5cm] (print) {Print integers start to end};

        \draw[gtu arrow] (input) -- (cond);
        \draw[gtu arrow] (cond) -| node[above] {Yes} (case1);
        \draw[gtu arrow] (cond) -| node[above] {No} (case2);
        \draw[gtu arrow] (case1) -- (case1 |- print.north);
        \draw[gtu arrow] (case2) -- (case2 |- print.north);
    \end{tikzpicture}
    \end{center}

    \begin{itemize}
        \item \textbf{Math functions}: ceil() rounds up, floor() rounds down
        \item \textbf{Range determination}: Works regardless of input order
        \item \textbf{Integer extraction}: Only prints whole numbers between floats
    \end{itemize}

    \begin{mnemonicbox}"Ceiling the small, flooring the big, then print every Integer in between"\end{mnemonicbox}
\end{solutionbox}

\questionmarks{3}{a}{3}
\textbf{Explain multiple if-else statement with example.}

\begin{solutionbox}
    \textbf{Answer}:
    Multiple if-else statements allow testing several conditions in sequence, where each condition is checked only if the previous conditions are false.

\begin{lstlisting}[language=C]
#include <stdio.h>

int main() {
    int marks;
    
    printf("Enter marks (0-100): ");
    scanf("%d", &marks);
    
    if(marks >= 80) {
        printf("Grade: A\n");
    } else if(marks >= 70) {
        printf("Grade: B\n");
    } else if(marks >= 60) {
        printf("Grade: C\n");
    } else if(marks >= 50) {
        printf("Grade: D\n");
    } else {
        printf("Grade: F\n");
    }
    
    return 0;
}
\end{lstlisting}

    \textbf{Diagram:}

    \begin{center}
    \begin{tikzpicture}[gtu flow]
        \node[gtu input] (input) {Input marks};
        \node[gtu decision, below=of input] (cond1) {marks $\ge$ 80?};
        \node[gtu process, right=of cond1] (gradeA) {Grade: A};
        \node[gtu decision, below=of cond1] (cond2) {marks $\ge$ 70?};
        \node[gtu process, right=of cond2] (gradeB) {Grade: B};
        \node[gtu decision, below=of cond2] (cond3) {marks $\ge$ 60?};
        \node[gtu process, right=of cond3] (gradeC) {Grade: C};
        \node[gtu decision, below=of cond3] (cond4) {marks $\ge$ 50?};
        \node[gtu process, right=of cond4] (gradeD) {Grade: D};
        \node[gtu process, below=of cond4] (gradeF) {Grade: F};
        
        \draw[gtu arrow] (input) -- (cond1);
        \draw[gtu arrow] (cond1) -- node[above] {Yes} (gradeA);
        \draw[gtu arrow] (cond1) -- node[right] {No} (cond2);
        \draw[gtu arrow] (cond2) -- node[above] {Yes} (gradeB);
        \draw[gtu arrow] (cond2) -- node[right] {No} (cond3);
        \draw[gtu arrow] (cond3) -- node[above] {Yes} (gradeC);
        \draw[gtu arrow] (cond3) -- node[right] {No} (cond4);
        \draw[gtu arrow] (cond4) -- node[above] {Yes} (gradeD);
        \draw[gtu arrow] (cond4) -- node[right] {No} (gradeF);
    \end{tikzpicture}
    \end{center}

    \begin{itemize}
        \item \textbf{Sequential testing}: Only one block executes
        \item \textbf{Efficiency}: Stops checking after finding true condition
    \end{itemize}

    \begin{mnemonicbox}"If this THEN that, ELSE IF another THEN something else"\end{mnemonicbox}
\end{solutionbox}

\questionmarks{3}{b}{4}
\textbf{State the working of while loop and for loop.}

\begin{solutionbox}
    \textbf{Answer}:

    \begin{tabulary}{\linewidth}{|L|L|L|L|}
        \hline
        \textbf{Loop Type} & \textbf{Working} & \textbf{Syntax} & \textbf{Use Cases} \\
        \hline
        while loop & 1. Test condition \newline 2. If true, execute body \newline 3. Repeat steps 1-2 until condition is false & \code{while(condition) \{}\newline \code{    // statements}\newline \code{\}} & When number of iterations is unknown beforehand \\
        \hline
        for loop & 1. Execute initialization once \newline 2. Test condition \newline 3. If true, execute body \newline 4. Execute update statement \newline 5. Repeat steps 2-4 until condition is false & \code{for(init; cond; update) \{}\newline \code{    // statements}\newline \code{\}} & When number of iterations is known beforehand \\
        \hline
    \end{tabulary}

    \textbf{Comparison:}

    \begin{center}
    \begin{tikzpicture}[gtu flow]
        % While Loop
        \node[gtu start] (w_start) {Start While};
        \node[gtu decision, below=of w_start] (w_cond) {Condition?};
        \node[gtu process, below=of w_cond] (w_body) {Body};
        \node[gtu stop, right=of w_cond] (w_end) {End};

        \draw[gtu arrow] (w_start) -- (w_cond);
        \draw[gtu arrow] (w_cond) -- node[left] {Yes} (w_body);
        \draw[gtu arrow] (w_body.west) -- ++(-0.5,0) |- (w_cond.west);
        \draw[gtu arrow] (w_cond) -- node[above] {No} (w_end);

        % For Loop - Shifted right
        \node[gtu start, right=of w_end, xshift=1cm] (f_start) {Start For};
        \node[gtu process, below=of f_start] (f_init) {Init};
        \node[gtu decision, below=of f_init] (f_cond) {Condition?};
        \node[gtu process, below=of f_cond] (f_body) {Body};
        \node[gtu process, left=of f_body] (f_update) {Update};
        \node[gtu stop, right=of f_cond] (f_end) {End};

        \draw[gtu arrow] (f_start) -- (f_init);
        \draw[gtu arrow] (f_init) -- (f_cond);
        \draw[gtu arrow] (f_cond) -- node[left] {Yes} (f_body);
        \draw[gtu arrow] (f_body) -- (f_update);
        \draw[gtu arrow] (f_update) |- (f_cond);
        \draw[gtu arrow] (f_cond) -- node[above] {No} (f_end);
    \end{tikzpicture}
    \end{center}

    \begin{itemize}
        \item \textbf{Entry control}: Both check condition before execution
        \item \textbf{Components}: for loop combines initialization, condition, and update
    \end{itemize}

    \begin{mnemonicbox}"WHILE checks THEN acts, FOR initializes CHECKS acts UPDATES"\end{mnemonicbox}
\end{solutionbox}

\questionmarks{3}{c}{7}
\textbf{Write a program to find factorial of a given number.}

\begin{solutionbox}
    \textbf{Answer}:

\begin{lstlisting}[language=C]
#include <stdio.h>

int main() {
    int num, i;
    unsigned long long factorial = 1;
    
    // Input a number
    printf("Enter a positive integer: ");
    scanf("%d", &num);
    
    // Check if the number is negative
    if(num < 0) {
        printf("Error: Factorial is not defined for negative numbers.\n");
    } else {
        // Calculate factorial
        for(i = 1; i <= num; i++) {
            factorial *= i;
        }
        
        printf("Factorial of %d = %llu\n", num, factorial);
    }
    
    return 0;
}
\end{lstlisting}

    \textbf{Diagram:}

    \begin{center}
    \begin{tikzpicture}[gtu flow]
        \node[gtu start] (start) {Start};
        \node[gtu input, below=of start] (input) {Input n};
        \node[gtu decision, below=of input] (check) {n $<$ 0?};
        \node[gtu output, right=of check] (error) {Error};
        \node[gtu process, below=of check] (init) {fact = 1, i = 1};
        \node[gtu decision, below=of init] (loop) {i $\le$ n?};
        \node[gtu process, right=of loop] (calc) {fact *= i\\i++};
        \node[gtu output, below=of loop] (output) {Print fact};
        \node[gtu stop, below=of output] (stop) {Stop};

        \draw[gtu arrow] (start) -- (input);
        \draw[gtu arrow] (input) -- (check);
        \draw[gtu arrow] (check) -- node[above] {Yes} (error);
        \draw[gtu arrow] (error) |- (stop);
        \draw[gtu arrow] (check) -- node[left] {No} (init);
        \draw[gtu arrow] (init) -- (loop);
        \draw[gtu arrow] (loop) -- node[above] {Yes} (calc);
        \draw[gtu arrow] (calc) |- (init.south east); 
        \draw[gtu arrow] (loop) -- node[left] {No} (output);
        \draw[gtu arrow] (output) -- (stop);
    \end{tikzpicture}
    \end{center}

    \begin{itemize}
        \item \textbf{Data type}: unsigned long long for large factorials
        \item \textbf{Error handling}: Checks for negative input
        \item \textbf{Loop implementation}: Multiply successive integers
    \end{itemize}

    \begin{mnemonicbox}"Factorial Formula: Multiply From One to Number"\end{mnemonicbox}
\end{solutionbox}

\orquestionmarks{3}{a}{3}
\textbf{Explain the working of switch-case statement with example.}

\begin{solutionbox}
    \textbf{Answer}:
    The switch-case statement is a multi-way decision maker that tests the value of an expression against various case values and executes the matching case block.

\begin{lstlisting}[language=C]
#include <stdio.h>

int main() {
    int day;
    
    printf("Enter day number (1-7): ");
    scanf("%d", &day);
    
    switch(day) {
        case 1: printf("Monday\n"); break;
        case 2: printf("Tuesday\n"); break;
        case 3: printf("Wednesday\n"); break;
        case 4: printf("Thursday\n"); break;
        case 5: printf("Friday\n"); break;
        case 6: printf("Saturday\n"); break;
        case 7: printf("Sunday\n"); break;
        default: printf("Invalid day number\n");
    }
    
    return 0;
}
\end{lstlisting}

    \textbf{Diagram:}

    \begin{center}
    \begin{tikzpicture}[gtu flow]
        \node[gtu input] (input) {Input day};
        \node[gtu decision, below=of input] (switch) {switch(day)};
        
        \node[gtu process, below left=of switch] (c1) {case 1};
        \node[gtu process, below=of switch] (c2) {case ...};
        \node[gtu process, below right=of switch] (def) {default};
        
        \node[gtu output, below=of c1] (o1) {Mon};
        \node[gtu output, below=of c2] (o2) {...};
        \node[gtu output, below=of def] (odef) {Invalid};
        
        \node[gtu stop, below=of o2] (end) {End};

        \draw[gtu arrow] (input) -- (switch);
        \draw[gtu arrow] (switch) -- (c1);
        \draw[gtu arrow] (switch) -- (c2);
        \draw[gtu arrow] (switch) -- (def);
        
        \draw[gtu arrow] (c1) -- (o1);
        \draw[gtu arrow] (c2) -- (o2);
        \draw[gtu arrow] (def) -- (odef);
        
        \draw[gtu arrow] (o1) -- (end);
        \draw[gtu arrow] (o2) -- (end);
        \draw[gtu arrow] (odef) -- (end);
    \end{tikzpicture}
    \end{center}

    \begin{itemize}
        \item \textbf{Expression evaluation}: Only integer or character types
        \item \textbf{Case matching}: Executes matching case until break
        \item \textbf{Default case}: Executes when no case matches
    \end{itemize}

    \begin{mnemonicbox}"SWITCH value, CASE match, BREAK out, DEFAULT rescue"\end{mnemonicbox}
\end{solutionbox}

\orquestionmarks{3}{b}{4}
\textbf{Define break and continue keyword.}

\begin{solutionbox}
    \textbf{Answer}:

    \begin{tabulary}{\linewidth}{|L|L|L|L|}
        \hline
        \textbf{Keyword} & \textbf{Definition} & \textbf{Purpose} & \textbf{Example} \\
        \hline
        break & Terminates the innermost loop or switch statement immediately & To exit a loop prematurely when a certain condition is met & \code{for(i=1; i<=10; i++) \{} \newline \code{    if(i == 5) break;} \newline \code{    printf("\%d ", i);} \newline \code{\} // Output: 1 2 3 4} \\
        \hline
        continue & Skips the rest of the current iteration and jumps to the next iteration of the loop & To skip specific iterations without terminating the loop & \code{for(i=1; i<=10; i++) \{} \newline \code{    if(i == 5) continue;} \newline \code{    printf("\%d ", i);} \newline \code{\} // Output: 1 2 3 4 6 7 8 9 10} \\
        \hline
    \end{tabulary}

    \textbf{Behavioral Comparison:}

    \begin{center}
    \begin{tikzpicture}[gtu flow]
        % Break
        \node[gtu start] (b_check) {Condition};
        \node[gtu decision, below=of b_check] (b_dec) {break?};
        \node[gtu stop, left=of b_dec] (b_exit) {Exit Loop};
        \node[gtu process, right=of b_dec] (b_cont) {Continue};
        
        \draw[gtu arrow] (b_check) -- (b_dec);
        \draw[gtu arrow] (b_dec) -- node[above] {Yes} (b_exit);
        \draw[gtu arrow] (b_dec) -- node[above] {No} (b_cont);

        % Continue
        \node[gtu start, right=of b_cont, xshift=1cm] (c_check) {Condition};
        \node[gtu decision, below=of c_check] (c_dec) {continue?};
        \node[gtu process, left=of c_dec] (c_skip) {Skip Body};
        \node[gtu process, right=of c_dec] (c_exec) {Execute Body};
        \node[gtu process, below=of c_skip] (c_next) {Next Iteration};
        
        \draw[gtu arrow] (c_check) -- (c_dec);
        \draw[gtu arrow] (c_dec) -- node[above] {Yes} (c_skip);
        \draw[gtu arrow] (c_dec) -- node[above] {No} (c_exec);
        \draw[gtu arrow] (c_skip) -- (c_next);
        \draw[gtu arrow] (c_exec) |- (c_next);
    \end{tikzpicture}
    \end{center}

    \begin{itemize}
        \item \textbf{Scope}: Both affect only the innermost loop
        \item \textbf{Control transfer}: break exits loop, continue jumps to next iteration
    \end{itemize}

    \begin{mnemonicbox}"BREAK leaves the room, CONTINUE skips to the next dance move"\end{mnemonicbox}
\end{solutionbox}

\orquestionmarks{3}{c}{7}
\textbf{Write a program to read number of lines (n) from keyboard and print the triangle shown below.}

\begin{verbatim}
1
1 2
1 2 3
1 2 3 4
1 2 3 4 5
\end{verbatim}

\begin{solutionbox}
    \textbf{Answer}:

\begin{lstlisting}[language=C]
#include <stdio.h>

int main() {
    int n, i, j;
    
    // Input number of lines
    printf("Enter number of lines: ");
    scanf("%d", &n);
    
    // Print the triangle pattern
    for(i = 1; i <= n; i++) {
        // Print numbers from 1 to i in each row
        for(j = 1; j <= i; j++) {
            printf("%d ", j);
        }
        printf("\n");
    }
    
    return 0;
}
\end{lstlisting}

    \textbf{Program Flow:}

    \begin{center}
    \begin{tikzpicture}[gtu flow]
        \node[gtu input] (input) {Input n};
        \node[gtu process, below=of input] (outer) {i = 1};
        \node[gtu decision, below=of outer] (check_i) {i $\le$ n?};
        
        \node[gtu process, right=of check_i] (inner) {j = 1};
        \node[gtu decision, right=of inner] (check_j) {j $\le$ i?};
        \node[gtu output, below=of check_j] (print_j) {Print j};
        \node[gtu process, left=of print_j] (inc_j) {j++};
        
        \node[gtu output, below=of check_i, yshift=-2cm] (newline) {Print \\n};
        \node[gtu process, left=of newline] (inc_i) {i++};
        \node[gtu stop, left=of check_i] (stop) {Stop};

        \draw[gtu arrow] (input) -- (outer);
        \draw[gtu arrow] (outer) -- (check_i);
        \draw[gtu arrow] (check_i) -- node[above] {Yes} (inner);
        \draw[gtu arrow] (inner) -- (check_j);
        \draw[gtu arrow] (check_j) -- node[right] {Yes} (print_j);
        \draw[gtu arrow] (print_j) -- (inc_j);
        \draw[gtu arrow] (inc_j) |- (check_j);
        
        \draw[gtu arrow] (check_j) -- node[right] {No} (newline); % simplified path
        \draw[gtu arrow] (newline) -- (inc_i);
        \draw[gtu arrow] (inc_i) |- (check_i);
        \draw[gtu arrow] (check_i) -- node[above] {No} (stop);
    \end{tikzpicture}
    \end{center}

    \begin{itemize}
        \item \textbf{Nested loops}: Outer loop for rows, inner loop for columns
        \item \textbf{Pattern logic}: Row number determines how many numbers to print
        \item \textbf{Number sequence}: Each row prints 1 to row number
    \end{itemize}

    \begin{mnemonicbox}"Rows decide COUNTer limit, COLumns print ONE to ROW"\end{mnemonicbox}
\end{solutionbox}

\questionmarks{4}{a}{3}
\textbf{Explain nested if-else statement with example.}

\begin{solutionbox}
    \textbf{Answer}:
    Nested if-else statements are if-else constructs placed inside another if or else block, allowing more complex conditional logic and multiple levels of decision making.

\begin{lstlisting}[language=C]
#include <stdio.h>

int main() {
    int age;
    char hasID;
    
    printf("Enter age: ");
    scanf("%d", &age);
    
    printf("Do you have ID? (Y/N): ");
    scanf(" %c", &hasID);
    
    if(age >= 18) {
        if(hasID == 'Y' || hasID == 'y') {
            printf("You can vote!\n");
        } else {
            printf("You need ID to vote.\n");
        }
    } else {
        printf("You must be 18 or older to vote.\n");
    }
    
    return 0;
}
\end{lstlisting}

    \textbf{Decision Tree:}

    \begin{center}
    \begin{tikzpicture}[gtu flow]
        \node[gtu input] (input) {Input age, hasID};
        \node[gtu decision, below=of input] (age_check) {age $\ge$ 18?};
        \node[gtu decision, left=of age_check, xshift=-2cm] (id_check) {hasID == 'Y' or 'y'?};
        \node[gtu process, below left=of id_check] (vote) {You can vote!};
        \node[gtu process, below right=of id_check] (need_id) {Need ID to vote};
        \node[gtu process, right=of age_check] (underage) {Must be 18+ to vote};

        \draw[gtu arrow] (input) -- (age_check);
        \draw[gtu arrow] (age_check) -- node[above] {Yes} (id_check);
        \draw[gtu arrow] (age_check) -- node[above] {No} (underage);
        \draw[gtu arrow] (id_check) -- node[left] {Yes} (vote);
        \draw[gtu arrow] (id_check) -- node[right] {No} (need_id);
    \end{tikzpicture}
    \end{center}

    \begin{itemize}
        \item \textbf{Hierarchical conditions}: Evaluates conditions in layers
        \item \textbf{Indentation}: Improves readability of nested structures
        \item \textbf{Multi-factor decisions}: Combines multiple criteria
    \end{itemize}

    \begin{mnemonicbox}"If INSIDE if, check DEEPER conditions"\end{mnemonicbox}
\end{solutionbox}

\questionmarks{4}{b}{4}
\textbf{Describe initialization of one-dimensional array.}

\begin{solutionbox}
    \textbf{Answer}:

    \begin{tabulary}{\linewidth}{|L|L|L|L|}
        \hline
        \textbf{Initialization Method} & \textbf{Syntax} & \textbf{Example} & \textbf{Description} \\
        \hline
        Declaration with size & \code{type name[size];} & \code{int marks[5];} & Creates array with specified size, elements have garbage values \\
        \hline
        Declaration with initialization & \code{type name[size] = \{values\};} & \code{int ages[4] = \{21, 19, 25, 32\};} & Creates and initializes array with specific values \\
        \hline
        Partial initialization & \code{type name[size] = \{values\};} & \code{int nums[5] = \{1, 2\};} & Initializes first elements, rest become zero \\
        \hline
        Size inference & \code{type name[] = \{values\};} & \code{int scores[] = \{95, 88, 72\};} & Size determined by number of initializers \\
        \hline
        Individual element & \code{name[index] = value;} & \code{marks[0] = 85;} & Assigns value to specific element \\
        \hline
    \end{tabulary}

    \textbf{Array Visualization:}

    \begin{center}
    \begin{tikzpicture}[gtu block]
        \matrix[nodes={draw, minimum size=1cm, anchor=center}, column sep=0cm, row sep=0cm] (arr) {
            \node (n0) {10}; & \node (n1) {20}; & \node (n2) {30}; & \node (n3) {40}; & \node (n4) {50}; \\
        };
        \node[below=0.2cm of n0] {[0]};
        \node[below=0.2cm of n1] {[1]};
        \node[below=0.2cm of n2] {[2]};
        \node[below=0.2cm of n3] {[3]};
        \node[below=0.2cm of n4] {[4]};
        \node[left=0.5cm of n0] {int numbers[5] =};
    \end{tikzpicture}
    \end{center}

    \begin{itemize}
        \item \textbf{Zero-indexing}: First element at index 0
        \item \textbf{Contiguous memory}: Elements stored sequentially
        \item \textbf{Size limitation}: Size must be known at compile time
    \end{itemize}

    \begin{mnemonicbox}"Declare SIZE first, then FILL with values or let COMPILER COUNT"\end{mnemonicbox}
\end{solutionbox}

\questionmarks{4}{c}{7}
\textbf{Define Array and write a program to reverse a string.}

\begin{solutionbox}
    \textbf{Answer}:
    An array is a collection of similar data items stored at contiguous memory locations and accessed using a common name.

\begin{lstlisting}[language=C]
#include <stdio.h>
#include <string.h>

int main() {
    char str[100], reversed[100];
    int i, j, length;
    
    // Input a string
    printf("Enter a string: ");
    gets(str);
    
    // Find the length of string
    length = strlen(str);
    
    // Reverse the string
    for(i = length - 1, j = 0; i >= 0; i--, j++) {
        reversed[j] = str[i];
    }
    
    // Add null terminator
    reversed[j] = '\0';
    
    // Display the reversed string
    printf("Reversed string: %s\n", reversed);
    
    return 0;
}
\end{lstlisting}

    \textbf{Algorithm Visualization:}

    \begin{center}
    \begin{tikzpicture}[gtu flow]
        \node[gtu start] (orig) {Original: "HELLO"};
        \node[gtu process, right=of orig] (h) {H};
        \node[gtu process, right=of h] (e) {E};
        \node[gtu process, right=of e] (l1) {L};
        \node[gtu process, right=of l1] (l2) {L};
        \node[gtu process, right=of l2] (o) {O};
        
        \node[gtu process, below=of h] (r4) {rev[4]};
        \node[gtu process, below=of e] (r3) {rev[3]};
        \node[gtu process, below=of l1] (r2) {rev[2]};
        \node[gtu process, below=of l2] (r1) {rev[1]};
        \node[gtu process, below=of o] (r0) {rev[0]};
        
        \node[gtu output, below=of r2] (final) {Reversed: "OLLEH"};

        \draw[gtu arrow] (orig) -- (h);
        \draw[gtu arrow] (h) -- (e);
        \draw[gtu arrow] (e) -- (l1);
        \draw[gtu arrow] (l1) -- (l2);
        \draw[gtu arrow] (l2) -- (o);
        
        \draw[gtu arrow] (h) -- (r4);
        \draw[gtu arrow] (e) -- (r3);
        \draw[gtu arrow] (l1) -- (r2);
        \draw[gtu arrow] (l2) -- (r1);
        \draw[gtu arrow] (o) -- (r0);
        
        \draw[gtu arrow] (r0) -- (final);
        \draw[gtu arrow] (r1) -- (final);
        \draw[gtu arrow] (r2) -- (final);
        \draw[gtu arrow] (r3) -- (final);
        \draw[gtu arrow] (r4) -- (final);
    \end{tikzpicture}
    \end{center}

    \begin{itemize}
        \item \textbf{Character array}: Stores string with null terminator
        \item \textbf{Two-pointer technique}: One for original, one for reversed
        \item \textbf{Zero-based indexing}: Arrays start at index 0
    \end{itemize}

    \begin{mnemonicbox}"Start from END, place at BEGIN, stop at ZERO"\end{mnemonicbox}
\end{solutionbox}

\orquestionmarks{4}{a}{3}
\textbf{Explain do while loop with example}

\begin{solutionbox}
    \textbf{Answer}:
    The do-while loop is an exit-controlled loop that executes the loop body at least once before checking the condition.

\begin{lstlisting}[language=C]
#include <stdio.h>

int main() {
    int num, sum = 0;
    
    do {
        printf("Enter a number (0 to stop): ");
        scanf("%d", &num);
        sum += num;
    } while(num != 0);
    
    printf("Sum of all entered numbers: %d\n", sum);
    
    return 0;
}
\end{lstlisting}

    \textbf{Loop Execution Flow:}

    \begin{center}
    \begin{tikzpicture}[gtu flow]
        \node[gtu start] (start) {Start};
        \node[gtu process, below=of start] (init) {sum = 0};
        \node[gtu input, below=of init] (input) {Input num};
        \node[gtu process, below=of input] (add) {sum += num};
        \node[gtu decision, below=of add] (cond) {num != 0?};
        \node[gtu output, right=of cond] (print) {Display sum};
        \node[gtu stop, below=of print] (stop) {Stop};

        \draw[gtu arrow] (start) -- (init);
        \draw[gtu arrow] (init) -- (input);
        \draw[gtu arrow] (input) -- (add);
        \draw[gtu arrow] (add) -- (cond);
        \draw[gtu arrow] (cond) -- node[left] {Yes} ++(-2,0) |- (input);
        \draw[gtu arrow] (cond) -- node[above] {No} (print);
        \draw[gtu arrow] (print) -- (stop);
    \end{tikzpicture}
    \end{center}

    \begin{itemize}
        \item \textbf{Execution order}: Body first, condition check later
        \item \textbf{Guaranteed execution}: Loop body always executes at least once
        \item \textbf{Termination}: Condition evaluated at bottom of loop
    \end{itemize}

    \begin{mnemonicbox}"DO first, ask questions WHILE later"\end{mnemonicbox}
\end{solutionbox}

\orquestionmarks{4}{b}{4}
\textbf{Define pointer and describe pointer with example.}

\begin{solutionbox}
    \textbf{Answer}:
    A pointer is a variable that stores the memory address of another variable.

    \begin{tabulary}{\linewidth}{|L|L|L|}
        \hline
        \textbf{Pointer Concept} & \textbf{Description} & \textbf{Example} \\
        \hline
        Declaration & Data\_type *pointer\_name; & \code{int *ptr;} \\
        \hline
        Initialization & Assign address of a variable & \code{int num = 10; int *ptr = \&num;} \\
        \hline
        Dereference & Access the value at the address & \code{printf("\%d", *ptr);} // Prints 10 \\
        \hline
        Address operator & Gets address of a variable & \code{printf("\%p", \&num);} // Prints address \\
        \hline
        Null pointer & Pointer that points to nothing & \code{int *ptr = NULL;} \\
        \hline
    \end{tabulary}

    \textbf{Pointer Visualization:}

    \begin{center}
    \begin{tikzpicture}[gtu block]
        \node[draw, rectangle split, rectangle split parts=2] (num) {Address: 1000 \nodepart{second} Value: 10};
        \node[above=0.1cm of num] {Variable 'num'};
        
        \node[draw, rectangle split, rectangle split parts=2, right=3cm of num] (ptr) {Address: 2000 \nodepart{second} Value: 1000};
        \node[above=0.1cm of ptr] {Pointer 'ptr'};
        
        \draw[gtu arrow] (ptr.south) -- ++(0,-0.5) -| (num.south);
    \end{tikzpicture}
    \end{center}

    \begin{itemize}
        \item \textbf{Indirect access}: Access variables through their addresses
        \item \textbf{Memory manipulation}: Direct memory access for efficiency
        \item \textbf{Dynamic memory}: Enables allocation/deallocation during runtime
    \end{itemize}

    \begin{mnemonicbox}"Pointers POINT to ADDRESS, STARS dereference to VALUES"\end{mnemonicbox}
\end{solutionbox}

\orquestionmarks{4}{c}{7}
\textbf{Define pointer and write a program to exchange two integers using pointer arguments.}

\begin{solutionbox}
    \textbf{Answer}:
    A pointer is a variable that contains the memory address of another variable, allowing indirect access and manipulation of data.

\begin{lstlisting}[language=C]
#include <stdio.h>

// Function to swap two integers using pointers
void swap(int *a, int *b) {
    int temp = *a;
    *a = *b;
    *b = temp;
}

int main() {
    int num1, num2;
    
    // Input two integers
    printf("Enter first number: ");
    scanf("%d", &num1);
    
    printf("Enter second number: ");
    scanf("%d", &num2);
    
    printf("Before swapping: num1 = %d, num2 = %d\n", num1, num2);
    
    // Call swap function with addresses of num1 and num2
    swap(&num1, &num2);
    
    printf("After swapping: num1 = %d, num2 = %d\n", num1, num2);
    
    return 0;
}
\end{lstlisting}

    \textbf{Swap Process Visualization:}

    \begin{center}
    \begin{tikzpicture}[gtu flow]
        \node[gtu block] (init) {a points to num1\\b points to num2};
        \node[gtu process, right=of init] (step1) {temp = *a};
        \node[gtu process, right=of step1] (step2) {*a = *b};
        \node[gtu process, below=of step2] (step3) {*b = temp};
        \node[gtu output, left=of step3] (final) {Values exchanged};

        \draw[gtu arrow] (init) -- (step1);
        \draw[gtu arrow] (step1) -- (step2);
        \draw[gtu arrow] (step2) -- (step3);
        \draw[gtu arrow] (step3) -- (final);
    \end{tikzpicture}
    \end{center}

    \begin{itemize}
        \item \textbf{Pass by reference}: Pointers allow functions to modify original variables
        \item \textbf{Temporary variable}: Required for swapping without data loss
        \item \textbf{Function parameter}: Pointer arguments pass addresses
    \end{itemize}

    \begin{mnemonicbox}"Grab by ADDRESS, change the CONTENT, without being PRESENT"\end{mnemonicbox}
\end{solutionbox}

\questionmarks{5}{a}{3}
\textbf{Write a program to find the numbers which are divisible by 7 in between the numbers 50 and 500.}

\begin{solutionbox}
    \textbf{Answer}:

\begin{lstlisting}[language=C]
#include <stdio.h>

int main() {
    int i, count = 0;
    
    printf("Numbers divisible by 7 between 50 and 500:\n");
    
    // Find and print numbers divisible by 7
    for(i = 50; i <= 500; i++) {
        if(i % 7 == 0) {
            printf("%d ", i);
            count++;
            
            // Print 10 numbers per line for better readability
            if(count % 10 == 0)
                printf("\n");
        }
    }
    
    printf("\nTotal count: %d\n", count);
    
    return 0;
}
\end{lstlisting}

    \textbf{Algorithm Visualization:}

    \begin{center}
    \begin{tikzpicture}[gtu flow]
        \node[gtu start] (start) {Start};
        \node[gtu process, below=of start] (init) {i = 50, count = 0};
        \node[gtu decision, below=of init] (loop) {i $\le$ 500?};
        \node[gtu decision, right=of loop] (div) {i \% 7 == 0?};
        \node[gtu process, right=of div] (print) {Print i\\count++};
        \node[gtu process, below=of div] (inc) {i++};
        \node[gtu output, below=of loop] (total) {Print total count};
        \node[gtu stop, below=of total] (stop) {Stop};

        \draw[gtu arrow] (start) -- (init);
        \draw[gtu arrow] (init) -- (loop);
        \draw[gtu arrow] (loop) -- node[above] {Yes} (div);
        \draw[gtu arrow] (div) -- node[above] {Yes} (print);
        \draw[gtu arrow] (div) -- node[left] {No} (inc);
        \draw[gtu arrow] (print) |- (inc);
        \draw[gtu arrow] (inc) -| (loop.south);
        \draw[gtu arrow] (loop) -- node[left] {No} (total);
        \draw[gtu arrow] (total) -- (stop);
    \end{tikzpicture}
    \end{center}

    \begin{itemize}
        \item \textbf{Modulo operator}: i \% 7 == 0 checks divisibility
        \item \textbf{Formatting output}: Line breaks for readability
        \item \textbf{Counter variable}: Tracks how many numbers found
    \end{itemize}

    \begin{mnemonicbox}"DIVide by SEVEN, ZERO remainder wins"\end{mnemonicbox}
\end{solutionbox}

\questionmarks{5}{b}{4}
\textbf{Write a program which reads an integer from keyboard and prints whether given number is odd or even.}

\begin{solutionbox}
    \textbf{Answer}:

\begin{lstlisting}[language=C]
#include <stdio.h>

int main() {
    int number;
    
    // Input an integer
    printf("Enter an integer: ");
    scanf("%d", &number);
    
    // Check if the number is even or odd
    if(number % 2 == 0) {
        printf("%d is an even number.\n", number);
    } else {
        printf("%d is an odd number.\n", number);
    }
    
    return 0;
}
\end{lstlisting}

    \textbf{Decision Logic:}

    \begin{center}
    \begin{tikzpicture}[gtu flow]
        \node[gtu input] (input) {Input number};
        \node[gtu decision, below=of input] (check) {num \% 2 == 0?};
        \node[gtu output, below left=of check] (even) {Even};
        \node[gtu output, below right=of check] (odd) {Odd};
        \node[gtu stop, below=of check, yshift=-2cm] (stop) {End};

        \draw[gtu arrow] (input) -- (check);
        \draw[gtu arrow] (check) -- node[left] {Yes} (even);
        \draw[gtu arrow] (check) -- node[right] {No} (odd);
        \draw[gtu arrow] (even) -- (stop);
        \draw[gtu arrow] (odd) -- (stop);
    \end{tikzpicture}
    \end{center}

    \begin{itemize}
        \item \textbf{Modulo test}: Even numbers have remainder 0 when divided by 2
        \item \textbf{Binary representation}: Last bit is 0 for even, 1 for odd
    \end{itemize}

    \begin{mnemonicbox}"EVEN with ZERO end, ODD with ONE bend"\end{mnemonicbox}
\end{solutionbox}

\questionmarks{5}{c}{7}
\textbf{Define structure? Explain how it differs from array? Develop a structure named book to save following information about books. Book title, Name of author, Price and Number of pages.}

\begin{solutionbox}
    \textbf{Answer}:
    A structure is a user-defined data type that allows grouping of variables of different data types under a single name.

    \textbf{Difference between Structure and Array:}

    \begin{tabulary}{\linewidth}{|L|L|L|}
        \hline
        \textbf{Feature} & \textbf{Structure} & \textbf{Array} \\
        \hline
        Data type & Can store different data types & Stores elements of same data type \\
        \hline
        Access & Members accessed using dot (.) operator & Elements accessed using index [] \\
        \hline
        Memory allocation & Memory may not be contiguous & Memory is always contiguous \\
        \hline
        Size & Size can vary for each member & Size is same for all elements \\
        \hline
        Declaration & Uses struct keyword & Uses square brackets [] \\
        \hline
        Purpose & Organizes related heterogeneous data & Organizes homogeneous data \\
        \hline
    \end{tabulary}

    \textbf{Book Structure Program:}

\begin{lstlisting}[language=C]
#include <stdio.h>
#include <string.h>

// Define the structure
struct Book {
    char title[100];
    char author[50];
    float price;
    int pages;
};

int main() {
    // Declare a variable of type struct Book
    struct Book myBook;
    
    // Assign values to the structure members
    strcpy(myBook.title, "C Programming");
    strcpy(myBook.author, "Dennis Ritchie");
    myBook.price = 350.50;
    myBook.pages = 285;
    
    // Display book information
    printf("Book Details:\n");
    printf("Title: %s\n", myBook.title);
    printf("Author: %s\n", myBook.author);
    printf("Price: %.2f\n", myBook.price);
    printf("Pages: %d\n", myBook.pages);
    
    return 0;
}
\end{lstlisting}

    \textbf{Structure Visualization:}

    \begin{center}
    \begin{tikzpicture}[gtu block]
        \node[draw, rectangle split, rectangle split parts=4] (struct) {
            title: "C Programming" \nodepart{second}
            author: "Dennis Ritchie" \nodepart{third}
            price: 350.50 \nodepart{fourth}
            pages: 285
        };
        \node[above=0.1cm of struct] {struct Book myBook};
    \end{tikzpicture}
    \end{center}

    \begin{itemize}
        \item \textbf{Structure definition}: Creates template for data
        \item \textbf{Member access}: Use dot operator (structure.member)
        \item \textbf{String handling}: Uses string functions for character arrays
    \end{itemize}

    \begin{mnemonicbox}"STRUCTURE groups DIFFERENT, ARRAY repeats SAME"\end{mnemonicbox}
\end{solutionbox}

\orquestionmarks{5}{a}{3}
\textbf{Write a program which reads a real number from keyboard and prints a smallest integer greater than it.}

\begin{solutionbox}
    \textbf{Answer}:

\begin{lstlisting}[language=C]
#include <stdio.h>
#include <math.h>

int main() {
    float number;
    int result;
    
    // Input a real number
    printf("Enter a real number: ");
    scanf("%f", &number);
    
    // Find smallest integer greater than the input
    result = ceil(number);
    
    // Display the result
    printf("Smallest integer greater than %.2f is %d\n", number, result);
    
    return 0;
}
\end{lstlisting}

    \textbf{Function Behavior:}

    \begin{center}
    \begin{tikzpicture}[gtu flow]
        \node[gtu input] (input) {Input real number};
        \node[gtu process, right=of input] (ceil) {Apply ceil function};
        \node[gtu output, right=of ceil] (result) {Display result};

        \draw[gtu arrow] (input) -- (ceil);
        \draw[gtu arrow] (ceil) -- (result);
    \end{tikzpicture}
    \end{center}

    \begin{itemize}
        \item \textbf{Math function}: ceil() rounds up to next integer
        \item \textbf{Result type}: Returns smallest integer greater than input
        \item \textbf{Handling edge cases}: Works with negative numbers
    \end{itemize}

    \begin{mnemonicbox}"CEILING function, UP we go, NEXT integer we show"\end{mnemonicbox}
\end{solutionbox}

\orquestionmarks{5}{b}{4}
\textbf{Write a program which reads character from keyboard and prints its ASCII value.}

\begin{solutionbox}
    \textbf{Answer}:

\begin{lstlisting}[language=C]
#include <stdio.h>

int main() {
    char ch;
    
    // Input a character
    printf("Enter a character: ");
    scanf("%c", &ch);
    
    // Display ASCII value of the character
    printf("ASCII value of '%c' is %d\n", ch, ch);
    
    return 0;
}
\end{lstlisting}

    \textbf{Program Visualization:}

    \begin{center}
    \begin{tikzpicture}[gtu flow]
        \node[gtu input] (input) {Input character 'A'};
        \node[gtu process, right=of input] (conv) {Internal: 65};
        \node[gtu output, right=of conv] (output) {Print 65};

        \draw[gtu arrow] (input) -- (conv);
        \draw[gtu arrow] (conv) -- (output);
    \end{tikzpicture}
    \end{center}

    \begin{itemize}
        \item \textbf{Character storage}: Characters stored as integers in memory
        \item \textbf{Type conversion}: Automatic conversion from char to int
        \item \textbf{Extended ASCII}: Values from 0 to 255 for 8-bit characters
    \end{itemize}

    \begin{mnemonicbox}"CHARS have NUMBERS underneath, PRINT shows BOTH sides"\end{mnemonicbox}
\end{solutionbox}

\orquestionmarks{5}{c}{7}
\textbf{Define function? Explain its advantage. Write function to calculate the square of a given integer number.}

\begin{solutionbox}
    \textbf{Answer}:
    A function is a self-contained block of code designed to perform a specific task. It takes input, processes it, and returns an output.

    \textbf{Advantages of Functions:}

    \begin{tabulary}{\linewidth}{|L|L|}
        \hline
        \textbf{Advantage} & \textbf{Description} \\
        \hline
        Code reusability & Write once, use many times \\
        \hline
        Modularity & Break complex problems into manageable parts \\
        \hline
        Maintainability & Easier to debug and modify isolated code \\
        \hline
        Abstraction & Hide implementation details \\
        \hline
        Readability & Makes code more organized and understandable \\
        \hline
        Scope control & Variables local to functions reduce naming conflicts \\
        \hline
    \end{tabulary}

    \textbf{Program with Square Function:}

\begin{lstlisting}[language=C]
#include <stdio.h>

// Function to calculate square of an integer
int square(int num) {
    return num * num;
}

int main() {
    int number, result;
    
    // Input an integer
    printf("Enter an integer: ");
    scanf("%d", &number);
    
    // Call the square function
    result = square(number);
    
    // Display the result
    printf("Square of %d is %d\n", number, result);
    
    return 0;
}
\end{lstlisting}

    \textbf{Function Flow:}

    \begin{center}
    \begin{tikzpicture}[gtu flow]
        \node[gtu process] (main) {main function};
        \node[gtu process, right=of main] (sq) {square function};
        \node[gtu stop, right=of sq] (end) {End};

        \draw[gtu arrow, bend left] (main) to node[above] {call(num)} (sq);
        \draw[gtu arrow, bend left] (sq) to node[below] {return result} (main);
        \draw[gtu arrow] (main) -- (end);
    \end{tikzpicture}
    \end{center}

    \begin{itemize}
        \item \textbf{Function prototype}: Declares function signature
        \item \textbf{Parameters}: Input values passed to function
        \item \textbf{Return value}: Output or result from function
    \end{itemize}

    \begin{mnemonicbox}"Functions ENCAPSULATE tasks, take INPUTS, give OUTPUTS"\end{mnemonicbox}
\end{solutionbox}

\end{document}
