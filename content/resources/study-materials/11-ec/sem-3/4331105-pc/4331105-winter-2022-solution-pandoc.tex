\documentclass[10pt,a4paper]{article}

% content/resources/templates/preamble.tex
\usepackage[margin=0.6in]{geometry}
\author{Milav Dabgar}
\usepackage{amsmath,amssymb,amsthm}
\usepackage{booktabs}
\usepackage{multirow}
\usepackage{xcolor}
\usepackage{tcolorbox}
\tcbuselibrary{breakable,skins}
\usepackage[colorlinks=true,linkcolor=blue]{hyperref}
\usepackage{titlesec}
\usepackage{enumitem}
\usepackage{tikz}
\usepackage{pgfplots}
\usepackage{circuitikz}
\usepackage[version=4]{mhchem}
\usepackage{longtable}
\usepackage{array}
\usepackage{float}
\usepackage{caption}
\usepackage{listings}

\lstset{
  basicstyle=\small\ttfamily,
  breaklines=true,
  breakatwhitespace=false,
  postbreak=\mbox{\textcolor{red}{$\hookrightarrow$}\space},
  float=false,
  numbers=left,
  numberstyle=\tiny\color{gray},
  numbersep=10pt,
  xleftmargin=2em,
  keywordstyle=\color{blue},
  commentstyle=\color{green!60!black},
  stringstyle=\color{purple},
  backgroundcolor=\color{gray!5},
  showstringspaces=false,
  tabsize=2,
  captionpos=b,
  keepspaces=true,
  columns=flexible
}

\pgfplotsset{compat=1.18}
\usetikzlibrary{shapes,arrows,positioning,calc,patterns,decorations.pathmorphing,decorations.markings,arrows.meta}

% Color scheme
\definecolor{headcolor}{RGB}{0,102,204}
\definecolor{keycolor}{RGB}{220,20,60}
\definecolor{solutioncolor}{RGB}{34,139,34}
\definecolor{mnemoniccolor}{RGB}{148,0,211}
\definecolor{codecolor}{RGB}{0,0,100}

% Spacing
\setlength{\parskip}{3pt}
\setlist[itemize]{nosep}
\setlist[enumerate]{nosep}

% Title formatting
\titleformat{\section}{\Large\bfseries\color{headcolor}}{\thesection}{1em}{}
\titleformat{\subsection}{\large\bfseries\color{headcolor}}{\thesubsection}{1em}{}

% Pandoc tightlist compatibility
\providecommand{\tightlist}{%
  \setlength{\itemsep}{0pt}\setlength{\parskip}{0pt}}

% Pandoc longtable compatibility
\newcounter{none}
\def\thenone{}


% content/resources/templates/english-boxes.tex
% This file is currently empty - it exists to maintain consistency with the import structure.
% Add custom environments here if needed in the future.


\begin{document}

\begin{center}
{\Huge\bfseries\color{headcolor} Subject Name Solutions}\\[5pt]
{\LARGE 4331105 -- Winter 2022}\\[3pt]
{\large Semester 1 Study Material}\\[3pt]
{\normalsize\textit{Detailed Solutions and Explanations}}
\end{center}

\vspace{10pt}

\subsection*{Question 1(a) [3 marks]}\label{q1a}

\textbf{List basic data types of C language with their range}

\begin{solutionbox}

{\def\LTcaptype{none} % do not increment counter
\begin{longtable}[]{@{}
  >{\raggedright\arraybackslash}p{(\linewidth - 4\tabcolsep) * \real{0.3438}}
  >{\raggedright\arraybackslash}p{(\linewidth - 4\tabcolsep) * \real{0.4375}}
  >{\raggedright\arraybackslash}p{(\linewidth - 4\tabcolsep) * \real{0.2188}}@{}}
\toprule\noalign{}
\begin{minipage}[b]{\linewidth}\raggedright
Data Type
\end{minipage} & \begin{minipage}[b]{\linewidth}\raggedright
Size (bytes)
\end{minipage} & \begin{minipage}[b]{\linewidth}\raggedright
Range
\end{minipage} \\
\midrule\noalign{}
\endhead
\bottomrule\noalign{}
\endlastfoot
char & 1 & -128 to 127 \\
int & 2 or 4 & -32,768 to 32,767 (2 bytes) or -2,147,483,648 to
2,147,483,647 (4 bytes) \\
float & 4 & 3.4E-38 to 3.4E+38 \\
double & 8 & 1.7E-308 to 1.7E+308 \\
\end{longtable}
}

\end{solutionbox}
\begin{mnemonicbox}
``CIFD - Computer Is Fundamentally Digital''

\end{mnemonicbox}
\subsection*{Question 1(b) [4 marks]}\label{q1b}

\textbf{Explain rules for naming a variable.}

\begin{solutionbox}

{\def\LTcaptype{none} % do not increment counter
\begin{longtable}[]{@{}
  >{\raggedright\arraybackslash}p{(\linewidth - 2\tabcolsep) * \real{0.4000}}
  >{\raggedright\arraybackslash}p{(\linewidth - 2\tabcolsep) * \real{0.6000}}@{}}
\toprule\noalign{}
\begin{minipage}[b]{\linewidth}\raggedright
Rule
\end{minipage} & \begin{minipage}[b]{\linewidth}\raggedright
Example
\end{minipage} \\
\midrule\noalign{}
\endhead
\bottomrule\noalign{}
\endlastfoot
Must start with letter or underscore & valid: \_count, name / invalid:
1score \\
Can contain letters, digits, underscores & valid: user\_1 / invalid:
user-1 \\
Cannot use keywords & valid: integer / invalid: int \\
Case sensitive & total and TOTAL are different \\
\end{longtable}
}

\textbf{Diagram:}

\begin{verbatim}
┌───────────────────────────────┐
│ Variable Naming Rules         │
├───────────────────────────────┤
│ [A{-Z, a{-}z, \_]  [A{-}Z, a{-}z, 0{-}9, \_]* │}
└───────────────────────────────┘
\end{verbatim}

\end{solutionbox}
\begin{mnemonicbox}
``LUCK - Letters Underscore Case Keywords''

\end{mnemonicbox}
\subsection*{Question 1(c) [7 marks]}\label{q1c}

\textbf{Define flowchart. Draw flowchart to find minimum of two integer
numbers N1 and N2.}

\begin{solutionbox}

A flowchart is a graphical representation of an algorithm showing the
sequence of steps using standard symbols connected by arrows.

\begin{verbatim}
flowchart LR
    A([Start]) {-{-} B[/Input N1, N2/]}
    B {-{-} C\{N1  N2?\}}
    C {-{-}|Yes| D[min = N1]}
    C {-{-}|No| E[min = N2]}
    D {-{-} F[/Output min/]}
    E {-{-} F}
    F {-{-} G([End])}
\end{verbatim}

\begin{itemize}
\tightlist
\item
  \textbf{Flowchart symbols}: Visual representation of logical steps
\item
  \textbf{Decision diamond}: Tests condition to determine flow path
\item
  \textbf{Process boxes}: Contain calculations or operations
\end{itemize}

\end{solutionbox}
\begin{mnemonicbox}
``FAST - Flow Analysis Shown Through-charts''

\end{mnemonicbox}
\subsection*{Question 1(c OR) [7
marks]}\label{question-1c-or-7-marks}

\textbf{Define algorithm. Write an algorithm to calculate area and
circumference of circle.}

\begin{solutionbox}

An algorithm is a step-by-step procedure to solve a particular problem
in a finite sequence of well-defined instructions.

\textbf{Algorithm for circle calculations:}

\begin{verbatim}
1. START
2. Input radius r
3. Calculate area = π * r * r
4. Calculate circumference = 2 * π * r
5. Output area, circumference
6. STOP
\end{verbatim}

{\def\LTcaptype{none} % do not increment counter
\begin{longtable}[]{@{}lll@{}}
\toprule\noalign{}
Step & Operation & Formula \\
\midrule\noalign{}
\endhead
\bottomrule\noalign{}
\endlastfoot
1 & Get radius & Input r \\
2 & Calculate area & A = π \times r^{2} \\
3 & Calculate circumference & C = 2 \times π \times r \\
4 & Display results & Output A, C \\
\end{longtable}
}

\end{solutionbox}
\begin{mnemonicbox}
``SICS - Steps In Clear Sequence''

\end{mnemonicbox}
\subsection*{Question 2(a) [3 marks]}\label{q2a}

\textbf{Differentiate printf() and scanf().}

\begin{solutionbox}

{\def\LTcaptype{none} % do not increment counter
\begin{longtable}[]{@{}
  >{\raggedright\arraybackslash}p{(\linewidth - 4\tabcolsep) * \real{0.3214}}
  >{\raggedright\arraybackslash}p{(\linewidth - 4\tabcolsep) * \real{0.3571}}
  >{\raggedright\arraybackslash}p{(\linewidth - 4\tabcolsep) * \real{0.3214}}@{}}
\toprule\noalign{}
\begin{minipage}[b]{\linewidth}\raggedright
Feature
\end{minipage} & \begin{minipage}[b]{\linewidth}\raggedright
printf()
\end{minipage} & \begin{minipage}[b]{\linewidth}\raggedright
scanf()
\end{minipage} \\
\midrule\noalign{}
\endhead
\bottomrule\noalign{}
\endlastfoot
Purpose & Outputs data to screen & Inputs data from keyboard \\
Format & printf(``format'', variables) & scanf(``format'',
\&variables) \\
Returns & Number of chars printed & Number of items successfully read \\
Addressing & Uses variable names & Uses address of variables (\&var) \\
\end{longtable}
}

\end{solutionbox}
\begin{mnemonicbox}
``IO-AR - Input Output-Address Returns''

\end{mnemonicbox}
\subsection*{Question 2(b) [4 marks]}\label{q2b}

\textbf{Develop a C program to find maximum among two numbers using
conditional operator.}

\begin{solutionbox}

\begin{verbatim}
\#include {stdio.h}

int main() \{
    int num1, num2, max;
    
    printf("Enter two numbers: ");
    scanf("\%d \%d", \&num1, \&num2);
    
    max = (num1 {} num2) ? num1 : num2;
    
    printf("Maximum number is: \%d", max);
    
    return 0;
\}
\end{verbatim}

\textbf{Diagram:}

\begin{verbatim}
┌─────────────┐     ┌─────────────┐      ┌─────────────┐
│ Input       │────{│ Condition   │─────│ Output      │}
│ num1, num2  │     │ num1 { num2?│      │ max         │}
└─────────────┘     └─────────────┘      └─────────────┘
\end{verbatim}

\end{solutionbox}
\begin{mnemonicbox}
``CTO - Condition Then Output''

\end{mnemonicbox}
\subsection*{Question 2(c) [7 marks]}\label{q2c}

\textbf{Explain arithmetic \& relational operators with examples.}

\begin{solutionbox}

{\def\LTcaptype{none} % do not increment counter
\begin{longtable}[]{@{}llll@{}}
\toprule\noalign{}
Type & Operators & Example & Result \\
\midrule\noalign{}
\endhead
\bottomrule\noalign{}
\endlastfoot
\textbf{Arithmetic Operators} & & & \\
Addition & + & 5 + 3 & 8 \\
Subtraction & - & 5 - 3 & 2 \\
Multiplication & * & 5 * 3 & 15 \\
Division & / & 5 / 3 & 1 (integer division) \\
Modulus & \% & 5 \% 3 & 2 (remainder) \\
\textbf{Relational Operators} & & & \\
Equal to & == & 5 == 3 & 0 (false) \\
Not equal to & != & 5 != 3 & 1 (true) \\
Greater than & \textgreater{} & 5 \textgreater{} 3 & 1 (true) \\
Less than & \textless{} & 5 \textless{} 3 & 0 (false) \\
Greater than or equal & \textgreater= & 5 \textgreater= 5 & 1 (true) \\
Less than or equal & \textless= & 5 \textless= 3 & 0 (false) \\
\end{longtable}
}

\end{solutionbox}
\begin{mnemonicbox}
``ASMDCRO - Add Subtract Multiply Divide Compare
Return Output''

\end{mnemonicbox}
\subsection*{Question 2(a OR) [3
marks]}\label{question-2a-or-3-marks}

\textbf{Considering precedence of operators, write down each step of
evaluation and final answer if expression (25/3) * 4 -- 10 \% 3 + 9/2 is
evaluated.}

\begin{solutionbox}

{\def\LTcaptype{none} % do not increment counter
\begin{longtable}[]{@{}llll@{}}
\toprule\noalign{}
Step & Operation & Calculation & Result \\
\midrule\noalign{}
\endhead
\bottomrule\noalign{}
\endlastfoot
1 & Parentheses (25/3) & 25/3 = 8 (integer division) & 8 \\
2 & Modulus 10 \% 3 & 10 \% 3 = 1 & 1 \\
3 & Division 9/2 & 9/2 = 4 (integer division) & 4 \\
4 & Multiplication 8 * 4 & 8 * 4 = 32 & 32 \\
5 & Subtraction 32 - 1 & 32 - 1 = 31 & 31 \\
6 & Addition 31 + 4 & 31 + 4 = 35 & 35 \\
\end{longtable}
}

Final answer = 35

\end{solutionbox}
\begin{mnemonicbox}
``PEMDAS - Parentheses, Exponents,
Multiplication/Division, Addition/Subtraction''

\end{mnemonicbox}
\subsection*{Question 2(b OR) [4
marks]}\label{question-2b-or-4-marks}

\textbf{Develop a C program to find roots of an algebraic equation}

\begin{solutionbox}

\begin{verbatim}
\#include {stdio.h}
\#include {math.h}

int main() \{
    float a, b, c;
    float discriminant, root1, root2;
    
    printf("Enter coefficients a, b, c: ");
    scanf("\%f \%f \%f", \&a, \&b, \&c);
    
    discriminant = b*b {-} 4*a*c;
    
    if (discriminant {} 0) \{
        root1 = ({-}b + sqrt(discriminant)) / (2*a);
        root2 = ({-}b {-} sqrt(discriminant)) / (2*a);
        printf("Roots: \%.2f and \%.2f", root1, root2);
    \} else if (discriminant == 0) \{
        root1 = {-}b / (2*a);
        printf("Root: \%.2f", root1);
    \} else \{
        printf("No real roots");
    \}
    
    return 0;
\}
\end{verbatim}

\textbf{Diagram:}

\begin{verbatim}
flowchart LR
    A[Input a,b,c] {-{-} B[Calculate d = b^{2}{-}4ac]}
    B {-{-} C\{d  0?\}}
    C {-{-}|Yes| D[Two real roots]}
    C {-{-}|No| E\{d = 0?\}}
    E {-{-}|Yes| F[One real root]}
    E {-{-}|No| G[No real roots]}
\end{verbatim}

\end{solutionbox}
\begin{mnemonicbox}
``QDR - Quadratic Discriminant Roots''

\end{mnemonicbox}
\subsection*{Question 2(c OR) [7
marks]}\label{question-2c-or-7-marks}

\textbf{Explain logical \& bit-wise operators with examples.}

\begin{solutionbox}

{\def\LTcaptype{none} % do not increment counter
\begin{longtable}[]{@{}llll@{}}
\toprule\noalign{}
Type & Operators & Example & Result \\
\midrule\noalign{}
\endhead
\bottomrule\noalign{}
\endlastfoot
\textbf{Logical Operators} & & & \\
Logical AND & \&\& & (5\textgreater3) \&\& (4\textless7) & 1 (true) \\
Logical OR & \textbar\textbar{} & (5\textless3) \textbar\textbar{}
(4\textless7) & 1 (true) \\
Logical NOT & ! & !(5\textgreater3) & 0 (false) \\
\textbf{Bitwise Operators} & & & \\
Bitwise AND & \& & 5 \& 3 (101 \& 011) & 1 (001) \\
Bitwise OR & \textbar{} & 5 \textbar{} 3 (101 \textbar{} 011) & 7
(111) \\
Bitwise XOR & \^{} & 5 \^{} 3 (101 \^{} 011) & 6 (110) \\
Bitwise NOT & \textasciitilde{} & \textasciitilde5 (\textasciitilde{}
00000101) & -6 (11111010) \\
Left Shift & \textless\textless{} & 5 \textless\textless{} 1 (101
\textless\textless{} 1) & 10 (1010) \\
Right Shift & \textgreater\textgreater{} & 5 \textgreater\textgreater{}
1 (101 \textgreater\textgreater{} 1) & 2 (10) \\
\end{longtable}
}

\end{solutionbox}
\begin{mnemonicbox}
``LAND BORNS - Logical AND OR NOT, Bitwise OR AND NOT
Shift''

\end{mnemonicbox}
\subsection*{Question 3(a) [3 marks]}\label{q3a}

\textbf{Explain the use of `go to' statement with example}

\begin{solutionbox}

The \texttt{goto} statement allows unconditional jump to a labeled
statement in the program.

\begin{verbatim}
\#include {stdio.h}

int main() \{
    int i = 0;
    
start:
    printf("\%d ", i);
    i++;
    if (i {} 5)
        goto start;
    
    return 0;
\}
// Output: 0 1 2 3 4
\end{verbatim}

\textbf{Diagram:}

\begin{verbatim}
     ┌─────────┐
     │ Start   │
     └────┬────┘
          │
          ▼
┌─────────────────┐
│  print(i)       │
│  i++            │
└────────┬────────┘
         │
         ▼
     ┌────────┐    Yes
     │ i { 5? ├───────┐}
     └────┬───┘       │
          │No         │
          ▼           │
     ┌────────┐       │
     │  End   │       │
     └────────┘       │
                      │
     ┌────────────────┘
     │   
     ▼   
  goto start
\end{verbatim}

\end{solutionbox}
\begin{mnemonicbox}
``JUMP - Just Unconditionally Move Program-counter''

\end{mnemonicbox}
\subsection*{Question 3(b) [4 marks]}\label{q3b}

\textbf{Develop a C program to check whether the entered number is even
or odd.}

\begin{solutionbox}

\begin{verbatim}
\#include {stdio.h}

int main() \{
    int num;
    
    printf("Enter a number: ");
    scanf("\%d", \&num);
    
    if (num \% 2 == 0)
        printf("\%d is even", num);
    else
        printf("\%d is odd", num);
        
    return 0;
\}
\end{verbatim}

\textbf{Diagram:}

\begin{verbatim}
flowchart LR
    A([Start]) {-{-} B[/Input num/]}
    B {-{-} C\{num \% 2 == 0?\}}
    C {-{-}|Yes| D[/Print num is even/]}
    C {-{-}|No| E[/Print num is odd/]}
    D {-{-} F([End])}
    E {-{-} F}
\end{verbatim}

\end{solutionbox}
\begin{mnemonicbox}
``MODE - Modulo Odd-Even Determination''

\end{mnemonicbox}
\subsection*{Question 3(c) [7 marks]}\label{q3c}

\textbf{Draw flowchart and explain else if ladder with example.}

\begin{solutionbox}

The else-if ladder allows checking multiple conditions in sequence,
executing the block associated with the first true condition.

\begin{verbatim}
flowchart LR
    A([Start]) {-{-} B[/Input marks/]}
    B {-{-} C\{marks = 90?\}}
    C {-{-}|Yes| D[grade = A]}
    C {-{-}|No| E\{marks = 80?\}}
    E {-{-}|Yes| F[grade = B]}
    E {-{-}|No| G\{marks = 70?\}}
    G {-{-}|Yes| H[grade = C]}
    G {-{-}|No| I\{marks = 60?\}}
    I {-{-}|Yes| J[grade = D]}
    I {-{-}|No| K[grade = F]}
    D \& F \& H \& J \& K {-{-} L[/Output grade/]}
    L {-{-} M([End])}
\end{verbatim}

\begin{verbatim}
\#include {stdio.h}

int main() \{
    int marks;
    char grade;
    
    printf("Enter marks: ");
    scanf("\%d", \&marks);
    
    if (marks {=} 90)
        grade = {A};
    else if (marks {=} 80)
        grade = {B};
    else if (marks {=} 70)
        grade = {C};
    else if (marks {=} 60)
        grade = {D};
    else
        grade = {F};
        
    printf("Grade: \%c", grade);
    
    return 0;
\}
\end{verbatim}

\begin{itemize}
\tightlist
\item
  \textbf{Multiple conditions}: Checks conditions sequentially
\item
  \textbf{First match}: Only executes code for first true condition
\item
  \textbf{Default case}: Final else handles all remaining cases
\end{itemize}

\end{solutionbox}
\begin{mnemonicbox}
``CAFE - Condition Assess First Eligible''

\end{mnemonicbox}
\subsection*{Question 3(a OR) [3
marks]}\label{question-3a-or-3-marks}

\textbf{Explain the use of continue and break statement.}

\begin{solutionbox}

{\def\LTcaptype{none} % do not increment counter
\begin{longtable}[]{@{}lll@{}}
\toprule\noalign{}
Statement & Purpose & Effect \\
\midrule\noalign{}
\endhead
\bottomrule\noalign{}
\endlastfoot
break & Exit a loop or switch & Terminates entire loop immediately \\
continue & Skip current iteration & Jumps to next iteration of loop \\
\end{longtable}
}

\begin{verbatim}
// break example
for(int

i=1; i{=}10; i++) \{

    if(i == 6)
        break;      // Exits loop when i=6
    printf("\%d ", i); // Output: 1 2 3 4 5
\}

// continue example
for(int

i=1; i{=}10; i++) \{

    if(i \% 2 == 0)
        continue;   // Skips even numbers
    printf("\%d ", i); // Output: 1 3 5 7 9
\}
\end{verbatim}

\textbf{Diagram:}

\begin{verbatim}
  break                        continue
┌─────────┐                   ┌─────────┐
│ Loop    │                   │ Loop    │
│ ┌─────┐ │                   │ ┌─────┐ │
│ │  1  │ │                   │ │  1  │ │
│ └─────┘ │                   │ └─────┘ │
│ ┌─────┐ │                   │ ┌─────┐ │
│ │  2  │ │◄───┐              │ │  2  │ │◄───┐
│ └─────┘ │    │              │ └─────┘ │    │
│ ┌─────┐ │    │ break        │ ┌─────┐ │    │ continue 
│ │  3  ├─┘    │              │ │  3  ├─┘    │
│ └─────┘      │              │ └─────┘      │
└─────────────┐│              └───────┬──────┘
 Exit Loop    ││               Next   │
              ┘│             Iteration│
               └──────────────────────┘
\end{verbatim}

\end{solutionbox}
\begin{mnemonicbox}
``BEST - Break Exits, Skip with conTinue''

\end{mnemonicbox}
\subsection*{Question 3(b OR) [4
marks]}\label{question-3b-or-4-marks}

\textbf{Develop a C program to print sum of 1 to 10 numbers using for
loop.}

\begin{solutionbox}

\begin{verbatim}
\#include {stdio.h}

int main() \{
    int i, sum = 0;
    
    for(i = 1; i {=} 10; i++) \{
        sum += i;
    \}
    
    printf("Sum of numbers from 1 to 10: \%d", sum);
    
    return 0;
\}
\end{verbatim}

\textbf{Diagram:}

\begin{verbatim}
flowchart LR
    A([Start]) {-{-} B[sum = 0]}
    B {-{-} C[i = 1]}
    C {-{-} D\{i = 10?\}}
    D {-{-}|Yes| E[sum = sum + i]}
    E {-{-} F[i++]}
    F {-{-} D}
    D {-{-}|No| G[/Print sum/]}
    G {-{-} H([End])}
\end{verbatim}

\end{solutionbox}
\begin{mnemonicbox}
``SILA - Sum Increment Loop Add''

\end{mnemonicbox}
\subsection*{Question 3(c OR) [7
marks]}\label{question-3c-or-7-marks}

\textbf{Draw flowchart and explain switch statement with example.}

\begin{solutionbox}

The switch statement selects one code block from multiple options based
on a variable's value.

\begin{verbatim}
flowchart LR
    A([Start]) {-{-} B[/Input choice/]}
    B {-{-} C\{switch choice\}}
    C {-{-}|case 1| D[/Print Addition/]}
    C {-{-}|case 2| E[/Print Subtraction/]}
    C {-{-}|case 3| F[/Print Multiplication/]}
    C {-{-}|case 4| G[/Print Division/]}
    C {-{-}|default| H[/Print Invalid/]}
    D \& E \& F \& G \& H {-{-} I([End])}
\end{verbatim}

\begin{verbatim}
\#include {stdio.h}

int main() \{
    int choice;
    
    printf("Enter operation (1{-4): "});
    scanf("\%d", \&choice);
    
    switch(choice) \{
        case 1:
            printf("Addition selected");
            break;
        case 2:
            printf("Subtraction selected");
            break;
        case 3:
            printf("Multiplication selected");
            break;
        case 4:
            printf("Division selected");
            break;
        default:
            printf("Invalid choice");
    \}
    
    return 0;
\}
\end{verbatim}

\begin{itemize}
\tightlist
\item
  \textbf{Expression}: Takes integer or character expression
\item
  \textbf{Case labels}: Must be constant expressions
\item
  \textbf{Break statement}: Prevents fall-through to next case
\item
  \textbf{Default}: Handles values not matching any case
\end{itemize}

\end{solutionbox}
\begin{mnemonicbox}
``SCBD - Switch Cases Break Default''

\end{mnemonicbox}
\subsection*{Question 4(a) [3 marks]}\label{q4a}

\textbf{Develop a C program to convert uppercase alphabet to lowercase
alphabet.}

\begin{solutionbox}

\begin{verbatim}
\#include {stdio.h}

int main() \{
    char upper, lower;
    
    printf("Enter uppercase letter: ");
    scanf("\%c", \&upper);
    
    lower = upper + 32;
    // Alternatively: lower = tolower(upper);
    
    printf("Lowercase letter: \%c", lower);
    
    return 0;
\}
\end{verbatim}

\textbf{Diagram:}

\begin{verbatim}
┌─────────────┐
│ Input {A   │}
└──────┬──────┘
       │
       ▼
┌─────────────┐
│ ASCII code  │
│     65      │
└──────┬──────┘
       │ +32
       ▼
┌─────────────┐
│ ASCII code  │
│     97      │
└──────┬──────┘
       │
       ▼
┌─────────────┐
│ Output {a  │}
└─────────────┘
\end{verbatim}

\end{solutionbox}
\begin{mnemonicbox}
``ASCII-32 - Add 32 to Shift Characters Into
Lowercase''

\end{mnemonicbox}
\subsection*{Question 4(b) [4 marks]}\label{q4b}

\textbf{What is pointer? Explain with example.}

\begin{solutionbox}

A pointer is a variable that stores the memory address of another
variable.

{\def\LTcaptype{none} % do not increment counter
\begin{longtable}[]{@{}lll@{}}
\toprule\noalign{}
Concept & Syntax & Description \\
\midrule\noalign{}
\endhead
\bottomrule\noalign{}
\endlastfoot
Declaration & \texttt{int\ *p;} & Declares pointer p to int \\
Initialization & \texttt{p\ =\ \&var;} & Store address of var in p \\
Dereferencing & \texttt{*p\ =\ 10;} & Access/modify pointed value \\
Pointer arithmetic & \texttt{p++} & Move to next memory location \\
\end{longtable}
}

\begin{verbatim}
\#include {stdio.h}

int main() \{
    int num = 10;
    int *ptr;
    
    ptr = \&num;  // Store address of num in ptr
    
    printf("Value of num: \%d{n}", num);
    printf("Address of num: \%p{n}", \&num);
    printf("Value of ptr: \%p{n}", ptr);
    printf("Value pointed by ptr: \%d{n}", *ptr);
    
    *ptr = 20;  // Change value using pointer
    printf("New value of num: \%d{n}", num);
    
    return 0;
\}
\end{verbatim}

\textbf{Diagram:}

\begin{verbatim}
Memory:
┌─────────────┐
│    num      │ 1000  ┌───────────┐
│    (10)     │◄──────┤   *ptr    │ 2000
└─────────────┘       │  (1000)   │
                      └───────────┘
\end{verbatim}

\end{solutionbox}
\begin{mnemonicbox}
``SAID - Store Address to Indirectly Dereference''

\end{mnemonicbox}
\subsection*{Question 4(c) [7 marks]}\label{q4c}

\textbf{Draw flowchart and explain for loop with example.}

\begin{solutionbox}

The for loop is used to repeat a block of code a specified number of
times.

\begin{verbatim}
flowchart LR
    A([Start]) {-{-} B[Initialization: i=1]}
    B {-{-} C\{Condition: i=5?\}}
    C {-{-}|True| D[Body: Print i]}
    D {-{-} E[Update: i++]}
    E {-{-} C}
    C {-{-}|False| F([End])}
\end{verbatim}

\begin{verbatim}
\#include {stdio.h}

int main() \{
    int i;
    
    // Syntax: for(initialization; condition; update)
    for(i = 1; i {=} 5; i++) \{
        printf("\%d ", i);
    \}
    // Output: 1 2 3 4 5
    
    return 0;
\}
\end{verbatim}

\begin{itemize}
\tightlist
\item
  \textbf{Initialization}: Executes once before loop starts
\item
  \textbf{Condition}: Checked before each iteration
\item
  \textbf{Update}: Executes after each iteration
\item
  \textbf{Body}: Code block that repeats
\end{itemize}

\end{solutionbox}
\begin{mnemonicbox}
``ICU-B - Initialize, Check, Update, Body''

\end{mnemonicbox}
\subsection*{Question 4(a OR) [3
marks]}\label{question-4a-or-3-marks}

\textbf{Develop a C program to find area of a triangle (½ * base *
height)?}

\begin{solutionbox}

\begin{verbatim}
\#include {stdio.h}

int main() \{
    float base, height, area;
    
    printf("Enter base of triangle: ");
    scanf("\%f", \&base);
    
    printf("Enter height of triangle: ");
    scanf("\%f", \&height);
    
    area = 0.5 * base * height;
    
    printf("Area of triangle: \%.2f", area);
    
    return 0;
\}
\end{verbatim}

\textbf{Diagram:}

\begin{verbatim}
flowchart LR
    A([Start]) {-{-} B[/Input base, height/]}
    B {-{-} C[area = 0.5 * base * height]}
    C {-{-} D[/Output area/]}
    D {-{-} E([End])}
\end{verbatim}

\end{solutionbox}
\begin{mnemonicbox}
``BHA - Base times Height divided by two equals
Area''

\end{mnemonicbox}
\subsection*{Question 4(b OR) [4
marks]}\label{question-4b-or-4-marks}

\textbf{Explain declaration and initialization of pointer.}

\begin{solutionbox}

{\def\LTcaptype{none} % do not increment counter
\begin{longtable}[]{@{}
  >{\raggedright\arraybackslash}p{(\linewidth - 6\tabcolsep) * \real{0.2683}}
  >{\raggedright\arraybackslash}p{(\linewidth - 6\tabcolsep) * \real{0.1951}}
  >{\raggedright\arraybackslash}p{(\linewidth - 6\tabcolsep) * \real{0.2195}}
  >{\raggedright\arraybackslash}p{(\linewidth - 6\tabcolsep) * \real{0.3171}}@{}}
\toprule\noalign{}
\begin{minipage}[b]{\linewidth}\raggedright
Operation
\end{minipage} & \begin{minipage}[b]{\linewidth}\raggedright
Syntax
\end{minipage} & \begin{minipage}[b]{\linewidth}\raggedright
Example
\end{minipage} & \begin{minipage}[b]{\linewidth}\raggedright
Description
\end{minipage} \\
\midrule\noalign{}
\endhead
\bottomrule\noalign{}
\endlastfoot
Declaration & datatype *pointer\_name; & \texttt{int\ *ptr;} & Creates a
pointer variable \\
Initialization & pointer\_name = \&variable; & \texttt{ptr\ =\ \&num;} &
Assigns address to pointer \\
Combined & datatype *pointer\_name = \&variable; &
\texttt{int\ *ptr\ =\ \&num;} & Declaration with initialization \\
NULL pointer & pointer\_name = NULL; & \texttt{ptr\ =\ NULL;} & Safe
initialization when no address is available \\
\end{longtable}
}

\begin{verbatim}
\#include {stdio.h}

int main() \{
    int num = 10;           // Regular variable
    int *ptr1;              // Declaration only
    int *ptr2 = \&num;       // Declaration with initialization
    
    ptr1 = \&num;            // Initialization of ptr1
    
    printf("num value: \%d{n}", num);
    printf("num address: \%p{n}", \&num);
    printf("ptr1 value: \%p{n}", ptr1);
    printf("ptr2 value: \%p{n}", ptr2);
    printf("Value via ptr1: \%d{n}", *ptr1);
    printf("Value via ptr2: \%d{n}", *ptr2);
    
    return 0;
\}
\end{verbatim}

\textbf{Diagram:}

\begin{verbatim}
Pointer Declaration:
┌───────────┐
│ int *ptr; │
└───────────┘

Pointer Initialization:
┌─────────────┐       ┌──────────┐
│ ptr = \&num; │      │ num (10) │
└─────────────┘       └──────────┘
\end{verbatim}

\end{solutionbox}
\begin{mnemonicbox}
``PAIN - Pointer Allocate, Initialize, Navigate''

\end{mnemonicbox}
\subsection*{Question 4(c OR) [7
marks]}\label{question-4c-or-7-marks}

\textbf{Draw flowchart and explain while loop with example.}

\begin{solutionbox}

The while loop repeats a block of code as long as a specified condition
is true.

\begin{verbatim}
flowchart LR
    A([Start]) {-{-} B[i = 1]}
    B {-{-} C\{i = 5?\}}
    C {-{-}|True| D[/Print i/]}
    D {-{-} E[i++]}
    E {-{-} C}
    C {-{-}|False| F([End])}
\end{verbatim}

\begin{verbatim}
\#include {stdio.h}

int main() \{
    int i = 1;
    
    // Syntax: while(condition) \{ body \}
    while(i {=} 5) \{
        printf("\%d ", i);
        i++;
    \}
    // Output: 1 2 3 4 5
    
    return 0;
\}
\end{verbatim}

\begin{itemize}
\tightlist
\item
  \textbf{Initialization}: Must be done before loop
\item
  \textbf{Condition}: Evaluated at beginning of each iteration
\item
  \textbf{Body}: Executes only if condition is true
\item
  \textbf{Update}: Must be inside loop body
\end{itemize}

\end{solutionbox}
\begin{mnemonicbox}
``CUBE - Condition check, Update inside Body, Exit
when false''

\end{mnemonicbox}
\subsection*{Question 5(a) [3 marks]}\label{q5a}

\textbf{Build a structure to store book information: book\_no,
book\_title, book\_author, book\_price}

\begin{solutionbox}

\begin{verbatim}
\#include {stdio.h}
\#include {string.h}

struct Book \{
    int book\_no;
    char book\_title[50];
    char book\_author[50];
    float book\_price;
\;}

int main() \{
    struct Book book1;
    
    book1.book\_no = 101;
    strcpy(book1.book\_title, "Programming in C");
    strcpy(book1.book\_author, "Dennis Ritchie");
    book1.book\_price = 450.50;
    
    printf("Book No: \%d{n}", book1.book\_no);
    printf("Title: \%s{n}", book1.book\_title);
    printf("Author: \%s{n}", book1.book\_author);
    printf("Price: \%.2f", book1.book\_price);
    
    return 0;
\}
\end{verbatim}

\textbf{Diagram:}

\begin{verbatim}
struct Book
┌───────────────────────────────┐
│ book\_no      (int)            │
├───────────────────────────────┤
│ book\_title   (char[50])       │
├───────────────────────────────┤
│ book\_author  (char[50])       │
├───────────────────────────────┤
│ book\_price   (float)          │
└───────────────────────────────┘
\end{verbatim}

\end{solutionbox}
\begin{mnemonicbox}
``SNAP - Structure Needs All Properties''

\end{mnemonicbox}
\subsection*{Question 5(b) [4 marks]}\label{q5b}

\textbf{Explain following functions with example. (1) sqrt() (2) pow()
(3) strlen() (4) strcpy()}

\begin{solutionbox}

{\def\LTcaptype{none} % do not increment counter
\begin{longtable}[]{@{}
  >{\raggedright\arraybackslash}p{(\linewidth - 6\tabcolsep) * \real{0.2703}}
  >{\raggedright\arraybackslash}p{(\linewidth - 6\tabcolsep) * \real{0.2432}}
  >{\raggedright\arraybackslash}p{(\linewidth - 6\tabcolsep) * \real{0.2432}}
  >{\raggedright\arraybackslash}p{(\linewidth - 6\tabcolsep) * \real{0.2432}}@{}}
\toprule\noalign{}
\begin{minipage}[b]{\linewidth}\raggedright
Function
\end{minipage} & \begin{minipage}[b]{\linewidth}\raggedright
Library
\end{minipage} & \begin{minipage}[b]{\linewidth}\raggedright
Purpose
\end{minipage} & \begin{minipage}[b]{\linewidth}\raggedright
Example
\end{minipage} \\
\midrule\noalign{}
\endhead
\bottomrule\noalign{}
\endlastfoot
sqrt() & math.h & Calculates square root & \texttt{sqrt(16)} returns
\texttt{4.0} \\
pow() & math.h & Raises to power & \texttt{pow(2,\ 3)} returns
\texttt{8.0} \\
strlen() & string.h & Finds string length & \texttt{strlen("hello")}
returns \texttt{5} \\
strcpy() & string.h & Copies string & \texttt{strcpy(dest,\ "hello")}
copies ``hello'' to dest \\
\end{longtable}
}

\begin{verbatim}
\#include {stdio.h}
\#include {math.h}
\#include {string.h}

int main() \{
    double sqrtResult = sqrt(25);
    double powResult = pow(2, 4);
    char str[] = "Programming";
    char dest[20];
    int length = strlen(str);
    
    strcpy(dest, str);
    
    printf("sqrt(25) = \%.2f{n}", sqrtResult);
    printf("pow(2, 4) = \%.2f{n}", powResult);
    printf("Length of {}\%s{ = }\%d{n}", str, length);
    printf("Copied string: \%s{n}", dest);
    
    return 0;
\}
\end{verbatim}

\end{solutionbox}
\begin{mnemonicbox}
``SPSS - Square-root Power String-length
String-copy''

\end{mnemonicbox}
\subsection*{Question 5(c) [7 marks]}\label{q5c}

\textbf{Explain arrays and array initialization. Give example.}

\begin{solutionbox}

An array is a collection of similar data elements stored at contiguous
memory locations.

{\def\LTcaptype{none} % do not increment counter
\begin{longtable}[]{@{}
  >{\raggedright\arraybackslash}p{(\linewidth - 4\tabcolsep) * \real{0.3200}}
  >{\raggedright\arraybackslash}p{(\linewidth - 4\tabcolsep) * \real{0.3200}}
  >{\raggedright\arraybackslash}p{(\linewidth - 4\tabcolsep) * \real{0.3600}}@{}}
\toprule\noalign{}
\begin{minipage}[b]{\linewidth}\raggedright
Method
\end{minipage} & \begin{minipage}[b]{\linewidth}\raggedright
Syntax
\end{minipage} & \begin{minipage}[b]{\linewidth}\raggedright
Example
\end{minipage} \\
\midrule\noalign{}
\endhead
\bottomrule\noalign{}
\endlastfoot
Declaration & \texttt{data\_type\ array\_name[size];} &
\texttt{int\ marks[5];} \\
Initialization at declaration &
\texttt{data\_type\ array\_name[size]\ =\ \{values\};} &
\texttt{int\ marks[5]\ =\ \{95,\ 80,\ 85,\ 75,\ 90\};} \\
Individual element & \texttt{array\_name[index]\ =\ value;} &
\texttt{marks[0]\ =\ 95;} \\
Partial initialization & \texttt{int\ arr[5]\ =\ \{1,\ 2\};} &
Remaining elements are 0 \\
Without size & \texttt{int\ arr[]\ =\ \{1,\ 2,\ 3\};} & Size
determined by elements \\
\end{longtable}
}

\begin{verbatim}
\#include {stdio.h}

int main() \{
    // Array declaration and initialization
    int numbers[5] = \{10, 20, 30, 40, 50\;}
    
    // Accessing array elements
    printf("First element: \%d{n}", numbers[0]);
    printf("Third element: \%d{n}", numbers[2]);
    
    // Changing array element
    numbers[1] = 25;
    
    // Printing all elements
    printf("Array elements: ");
    for(int i = 0; i {} 5; i++) \{
        printf("\%d ", numbers[i]);
    \}
    
    return 0;
\}
\end{verbatim}

\textbf{Diagram:}

\begin{verbatim}
Array in memory:
┌────┬────┬────┬────┬────┐
│ 10 │ 20 │ 30 │ 40 │ 50 │
└────┴────┴────┴────┴────┘
   0    1    2    3    4   indices
\end{verbatim}

\end{solutionbox}
\begin{mnemonicbox}
``CASED - Contiguous Arrangement of Similar Elements
with Direct-access''

\end{mnemonicbox}
\subsection*{Question 5(a OR) [3
marks]}\label{question-5a-or-3-marks}

\textbf{Write the difference between array and structure.}

\begin{solutionbox}

{\def\LTcaptype{none} % do not increment counter
\begin{longtable}[]{@{}
  >{\raggedright\arraybackslash}p{(\linewidth - 4\tabcolsep) * \real{0.3333}}
  >{\raggedright\arraybackslash}p{(\linewidth - 4\tabcolsep) * \real{0.2593}}
  >{\raggedright\arraybackslash}p{(\linewidth - 4\tabcolsep) * \real{0.4074}}@{}}
\toprule\noalign{}
\begin{minipage}[b]{\linewidth}\raggedright
Feature
\end{minipage} & \begin{minipage}[b]{\linewidth}\raggedright
Array
\end{minipage} & \begin{minipage}[b]{\linewidth}\raggedright
Structure
\end{minipage} \\
\midrule\noalign{}
\endhead
\bottomrule\noalign{}
\endlastfoot
Data types & Same data type only & Different data types allowed \\
Access & Using index: \texttt{arr[0]} & Using dot operator:
\texttt{emp.id} \\
Memory & Contiguous allocation & May not be contiguous \\
Size & Fixed at declaration & Sum of member sizes \\
Initialization & \texttt{int\ arr[3]\ =\ \{1,2,3\};} &
\texttt{struct\ emp\ e\ =\ \{101,"John",5000\};} \\
Purpose & Collection of similar items & Collection of related items \\
\end{longtable}
}

\textbf{Diagram:}

\begin{verbatim}
Array:                  Structure:
┌───┬───┬───┐           ┌───────────────┐
│ 1 │ 2 │ 3 │           │ id: 101       │
└───┴───┴───┘           ├───────────────┤
  int  int int          │ name: "John"  │
                        ├───────────────┤
                        │ salary: 5000.0│
                        └───────────────┘
                          int  char[]  float
\end{verbatim}

\end{solutionbox}
\begin{mnemonicbox}
``HASDIP - Homogeneous vs.~Assorted, Same
vs.~Different, Index vs.~Point''

\end{mnemonicbox}
\subsection*{Question 5(b OR) [4
marks]}\label{question-5b-or-4-marks}

\textbf{What is user defined function? Explain with example.}

\begin{solutionbox}

A user-defined function is a code block that performs a specific task,
created by the programmer to reuse and organize code.

{\def\LTcaptype{none} % do not increment counter
\begin{longtable}[]{@{}lll@{}}
\toprule\noalign{}
Component & Description & Example \\
\midrule\noalign{}
\endhead
\bottomrule\noalign{}
\endlastfoot
Return type & Data type returned by function & \texttt{int},
\texttt{void}, etc. \\
Function name & Identifier for the function & \texttt{sum},
\texttt{findMax} \\
Parameters & Input values in parentheses & \texttt{(int\ a,\ int\ b)} \\
Function body & Code inside curly braces &
\texttt{\{\ return\ a+b;\ \}} \\
\end{longtable}
}

\begin{verbatim}
\#include {stdio.h}

// Function declaration
int sum(int a, int b);

int main() \{
    int num1 = 5, num2 = 10;
    int result;
    
    // Function call
    result = sum(num1, num2);
    
    printf("Sum = \%d", result);
    
    return 0;
\}

// Function definition
int sum(int a, int b) \{
    return a + b;
\}
\end{verbatim}

\textbf{Diagram:}

\begin{verbatim}
flowchart TD
    A[Main function] {-{-}|Call with arguments| B[User{-}defined function]}
    B {-{-}|Return result| A}
\end{verbatim}

\end{solutionbox}
\begin{mnemonicbox}
``CRPB - Create, Return, Pass, Body''

\end{mnemonicbox}
\subsection*{Question 5(c OR) [7
marks]}\label{question-5c-or-7-marks}

\textbf{Develop a C program to find sum of array elements and average of
it.}

\begin{solutionbox}

\begin{verbatim}
\#include {stdio.h}

int main() \{
    int arr[100], n, i;
    int sum = 0;
    float avg;
    
    printf("Enter number of elements: ");
    scanf("\%d", \&n);
    
    printf("Enter \%d elements:{n}", n);
    for(i = 0; i {} n; i++) \{
        scanf("\%d", \&arr[i]);
        sum += arr[i];  // Add each element to sum
    \}
    
    avg = (float)sum / n;  // Calculate average
    
    printf("Sum of array elements: \%d{n}", sum);
    printf("Average of array elements: \%.2f", avg);
    
    return 0;
\}
\end{verbatim}

\textbf{Diagram:}

\begin{verbatim}
flowchart LR
    A([Start]) {-{-} B[/Input size n/]}
    B {-{-} C[/Input n elements/]}
    C {-{-} D[sum = 0]}
    D {-{-} E[i = 0]}
    E {-{-} F\{i  n?\}}
    F {-{-}|Yes| G["sum += arr[i]"]}
    G {-{-} H[i++]}
    H {-{-} F}
    F {-{-}|No| I[avg = sum / n]}
    I {-{-} J[/Output sum, avg/]}
    J {-{-} K([End])}
\end{verbatim}

{\def\LTcaptype{none} % do not increment counter
\begin{longtable}[]{@{}lll@{}}
\toprule\noalign{}
Step & Operation & Example (for array [5,10,15,20]) \\
\midrule\noalign{}
\endhead
\bottomrule\noalign{}
\endlastfoot
1 & Input array & [5,10,15,20] \\
2 & Initialize sum = 0 & sum = 0 \\
3 & Add each element & sum = 0+5+10+15+20 = 50 \\
4 & Divide by count & avg = 50/4 = 12.5 \\
\end{longtable}
}

\end{solutionbox}
\begin{mnemonicbox}
``LISA - Loop, Increment, Sum, Average''

\end{mnemonicbox}

\end{document}
