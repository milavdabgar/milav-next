\documentclass{article}

% content/resources/templates/preamble.tex
\usepackage[margin=0.6in]{geometry}
\author{Milav Dabgar}
\usepackage{amsmath,amssymb,amsthm}
\usepackage{booktabs}
\usepackage{multirow}
\usepackage{xcolor}
\usepackage{tcolorbox}
\tcbuselibrary{breakable,skins}
\usepackage[colorlinks=true,linkcolor=blue]{hyperref}
\usepackage{titlesec}
\usepackage{enumitem}
\usepackage{tikz}
\usepackage{pgfplots}
\usepackage{circuitikz}
\usepackage[version=4]{mhchem}
\usepackage{longtable}
\usepackage{array}
\usepackage{float}
\usepackage{caption}
\usepackage{listings}

\lstset{
  basicstyle=\small\ttfamily,
  breaklines=true,
  breakatwhitespace=false,
  postbreak=\mbox{\textcolor{red}{$\hookrightarrow$}\space},
  float=false,
  numbers=left,
  numberstyle=\tiny\color{gray},
  numbersep=10pt,
  xleftmargin=2em,
  keywordstyle=\color{blue},
  commentstyle=\color{green!60!black},
  stringstyle=\color{purple},
  backgroundcolor=\color{gray!5},
  showstringspaces=false,
  tabsize=2,
  captionpos=b,
  keepspaces=true,
  columns=flexible
}

\pgfplotsset{compat=1.18}
\usetikzlibrary{shapes,arrows,positioning,calc,patterns,decorations.pathmorphing,decorations.markings,arrows.meta}

% Color scheme
\definecolor{headcolor}{RGB}{0,102,204}
\definecolor{keycolor}{RGB}{220,20,60}
\definecolor{solutioncolor}{RGB}{34,139,34}
\definecolor{mnemoniccolor}{RGB}{148,0,211}
\definecolor{codecolor}{RGB}{0,0,100}

% Spacing
\setlength{\parskip}{3pt}
\setlist[itemize]{nosep}
\setlist[enumerate]{nosep}

% Title formatting
\titleformat{\section}{\Large\bfseries\color{headcolor}}{\thesection}{1em}{}
\titleformat{\subsection}{\large\bfseries\color{headcolor}}{\thesubsection}{1em}{}

% Pandoc tightlist compatibility
\providecommand{\tightlist}{%
  \setlength{\itemsep}{0pt}\setlength{\parskip}{0pt}}

% Pandoc longtable compatibility
\newcounter{none}
\def\thenone{}


% content/resources/templates/english-boxes.tex

% Custom environments
\newtcolorbox{solutionbox}{
 breakable,
 enhanced,
 colback=solutioncolor!5!white,
 colframe=solutioncolor!75!black,
 fonttitle=\bfseries,
 title=Solution
}

\newtcolorbox{solutionboxnobreak}{
 colback=solutioncolor!5!white,
 colframe=solutioncolor!75!black,
 fonttitle=\bfseries,
 title=Solution
}

\newtcolorbox{keyformula}{
 breakable,
 enhanced,
 colback=keycolor!5!white,
 colframe=keycolor!75!black,
 fonttitle=\bfseries,
 title=Key Formula
}

\newtcolorbox{mnemonicboxenv}{
 breakable,
 enhanced,
 colback=mnemoniccolor!5!white,
 colframe=mnemoniccolor!75!black,
 fonttitle=\bfseries,
 title=Mnemonic
}

\newcommand{\mnemonicbox}[1]{%
  \begin{mnemonicboxenv}
    #1
  \end{mnemonicboxenv}
}


% Custom commands for GTU solutions
% This file defines semantic commands for consistent formatting

% Question command with automatic formatting
\newcommand{\question}[2]{%
  \section*{Question #1}%
  \textbf{#2}%
}

% OR question variant
\newcommand{\questionor}[2]{%
  \section*{Question #1 OR}%
  \textbf{#2}%
}

% Proper table environment with caption
\newenvironment{answertable}[1]{%
  \begin{table}[htbp]
  \centering
  \caption{#1}
}{%
  \end{table}
}

% Proper figure environment for diagrams
\newenvironment{answerdiagram}[1]{%
  \begin{figure}[htbp]
  \centering
  \caption{#1}
}{%
  \end{figure}
}

% Semantic markup for key terms
\newcommand{\keyword}[1]{\textbf{#1}}
\newcommand{\code}[1]{\texttt{#1}}
\newcommand{\classname}[1]{\texttt{#1}}
\newcommand{\methodname}[1]{\texttt{#1}}

% Proper quotation marks
\newcommand{\mnemonic}[1]{``#1''}


\title{Programming in C (4331105) - Summer 2025 Solution}
\date{May 20, 2025}

\begin{document}
\maketitle

\questionmarks{1}{a}{3}
\textbf{How many keywords are there in C? Write any four keywords}

\begin{solutionbox}
    \textbf{Answer}:

    \begin{tabulary}{\linewidth}{|L|L|}
        \hline
        \textbf{Total Keywords} & \textbf{Examples} \\
        \hline
        32 keywords & int, float, char, if \\
        \hline
    \end{tabulary}

    \textbf{Diagram:}

    \begin{center}
    \begin{tikzpicture}[gtu flow]
        \node[gtu block] (keywords) {C Keywords - 32 Total};
        \node[gtu block, below left=of keywords, xshift=-1cm] (types) {Data Types: int, float, char, double};
        \node[gtu block, below=of keywords] (control) {Control: if, else, for, while};
        \node[gtu block, below right=of keywords, xshift=1cm] (storage) {Storage: static, extern, auto, register};

        \draw[gtu arrow] (keywords) -- (types);
        \draw[gtu arrow] (keywords) -- (control);
        \draw[gtu arrow] (keywords) -- (storage);
    \end{tikzpicture}
    \end{center}

    \begin{itemize}
        \item \textbf{32 keywords}: Total reserved words in C language
        \item \textbf{Data type keywords}: int, float, char, double for variable declaration
        \item \textbf{Control keywords}: if, else, for, while for program flow
    \end{itemize}

    \begin{mnemonicbox}"Cats In Four Colors" (char, int, float, const)\end{mnemonicbox}
\end{solutionbox}

\questionmarks{1}{b}{4}
\textbf{What is variable? Explain rules for naming a variable with example}

\begin{solutionbox}
    \textbf{Answer}:

    \textbf{Variable Definition:}

    \begin{tabulary}{\linewidth}{|L|L|}
        \hline
        \textbf{Aspect} & \textbf{Description} \\
        \hline
        Definition & Named memory location to store data \\
        \hline
        Purpose & Hold values that can change during program execution \\
        \hline
        Declaration & \code{datatype variable\_name;} \\
        \hline
    \end{tabulary}

    \textbf{Naming Rules:}

    \begin{itemize}
        \item \textbf{First character}: Must be letter or underscore (\_)
        \item \textbf{Subsequent characters}: Letters, digits, underscore only
        \item \textbf{Case sensitive}: 'Age' and 'age' are different
        \item \textbf{No keywords}: Cannot use reserved words like 'int', 'float'
    \end{itemize}

    \textbf{Examples:}

\begin{lstlisting}[language=C]
int age;        // Valid
float _salary;  // Valid
char name123;   // Valid
int 2number;    // Invalid - starts with digit
float for;      // Invalid - keyword used
\end{lstlisting}

    \begin{mnemonicbox}"Letters First, No Keywords" (LF-NK)\end{mnemonicbox}
\end{solutionbox}

\questionmarks{1}{c}{7}
\textbf{Specify errors if any, in the following statements}

\begin{solutionbox}
    \textbf{Answer}:

    \begin{tabulary}{\linewidth}{|L|L|L|}
        \hline
        \textbf{Statement} & \textbf{Error} & \textbf{Reason} \\
        \hline
        (1) \code{fLoat x;} & Invalid keyword & Correct: \code{float x;} \\
        \hline
        (2) \code{int min, max = 20;} & Partial initialization & Only max initialized, min uninitialized \\
        \hline
        (3) \code{long char c;} & Invalid combination & Cannot combine long with char \\
        \hline
        (4) \code{iNt a;} & Invalid keyword & Correct: \code{int a;} \\
        \hline
        (5) \code{FLOAT f=2;} & Invalid keyword & Correct: \code{float f=2;} \\
        \hline
        (6) \code{double m ; n;} & Missing datatype & Correct: \code{double m, n;} \\
        \hline
        (7) \code{Int score (100)0;} & Multiple errors & Invalid syntax, correct: \code{int score = 100;} \\
        \hline
    \end{tabulary}

    \textbf{Key Points:}

    \begin{itemize}
        \item \textbf{Case sensitivity}: Keywords must be lowercase
        \item \textbf{Multiple declaration}: Use comma separator
        \item \textbf{Initialization syntax}: Use = operator
    \end{itemize}

    \begin{mnemonicbox}"Keywords Lower Case Always" (KLCA)\end{mnemonicbox}
\end{solutionbox}

\orquestionmarks{1}{c}{7}
\textbf{What is algorithm? What is flowchart? Draw a flowchart to find area and perimeter of circle.}

\begin{solutionbox}
    \textbf{Answer}:

    \textbf{Definitions:}

    \begin{tabulary}{\linewidth}{|L|L|}
        \hline
        \textbf{Term} & \textbf{Definition} \\
        \hline
        Algorithm & Step-by-step procedure to solve a problem \\
        \hline
        Flowchart & Visual representation of algorithm using symbols \\
        \hline
    \end{tabulary}

    \textbf{Flowchart for Circle Area and Perimeter:}

    \begin{center}
    \begin{tikzpicture}[gtu flow]
        \node[gtu start] (start) {Start};
        \node[gtu input, below=of start] (input) {Input radius r};
        \node[gtu process, below=of input] (calc_area) {Calculate area = $\pi \times r^2$};
        \node[gtu process, below=of calc_area] (calc_perim) {Calculate perimeter = $2 \times \pi \times r$};
        \node[gtu output, below=of calc_perim] (display) {Display area and perimeter};
        \node[gtu stop, below=of display] (end) {End};

        \draw[gtu arrow] (start) -- (input);
        \draw[gtu arrow] (input) -- (calc_area);
        \draw[gtu arrow] (calc_area) -- (calc_perim);
        \draw[gtu arrow] (calc_perim) -- (display);
        \draw[gtu arrow] (display) -- (end);
    \end{tikzpicture}
    \end{center}

    \textbf{Algorithm Steps:}

    \begin{itemize}
        \item \textbf{Step 1}: Start
        \item \textbf{Step 2}: Input radius value
        \item \textbf{Step 3}: Calculate area using formula $\pi \times r^2$
        \item \textbf{Step 4}: Calculate perimeter using formula $2 \times \pi \times r$
    \end{itemize}

    \begin{mnemonicbox}"Start Input Calculate Display End" (SICDE)\end{mnemonicbox}
\end{solutionbox}

\questionmarks{2}{a}{3}
\textbf{What is operator? List all the 'C' operators.}

\begin{solutionbox}
    \textbf{Answer}:

    \textbf{Operator Definition:}

    \begin{tabulary}{\linewidth}{|L|L|}
        \hline
        \textbf{Aspect} & \textbf{Description} \\
        \hline
        Definition & Special symbols that perform operations on operands \\
        \hline
        Purpose & Manipulate data and variables \\
        \hline
    \end{tabulary}

    \textbf{C Operators List:}

    \begin{tabulary}{\linewidth}{|L|L|}
        \hline
        \textbf{Category} & \textbf{Operators} \\
        \hline
        Arithmetic & +, -, *, /, \% \\
        \hline
        Relational & <, >, <=, >=, ==, != \\
        \hline
        Logical & \&\&, ||, ! \\
        \hline
        Assignment & =, +=, -=, *=, /= \\
        \hline
        Increment/Decrement & ++, -- \\
        \hline
        Conditional & ?: \\
        \hline
    \end{tabulary}

    \begin{mnemonicbox}"Add Relate Logic Assign Increment Condition" (ARLIC)\end{mnemonicbox}
\end{solutionbox}

\questionmarks{2}{b}{4}
\textbf{State difference between while and do while loop.}

\begin{solutionbox}
    \textbf{Answer}:

    \begin{tabulary}{\linewidth}{|L|L|L|}
        \hline
        \textbf{Aspect} & \textbf{while loop} & \textbf{do-while loop} \\
        \hline
        \textbf{Entry condition} & Pre-tested & Post-tested \\
        \hline
        \textbf{Minimum execution} & 0 times & At least 1 time \\
        \hline
        \textbf{Syntax} & \code{while(condition) \{ \}} & \code{do \{ \} while(condition);} \\
        \hline
        \textbf{Semicolon} & Not required after while & Required after while \\
        \hline
    \end{tabulary}

    \textbf{Example:}

\begin{lstlisting}[language=C]
// while loop
while(i < 5) {
    printf("%d", i);
    i++;
}

// do-while loop  
do {
    printf("%d", i);
    i++;
} while(i < 5);
\end{lstlisting}

    \textbf{Key Points:}

    \begin{itemize}
        \item \textbf{Pre-tested}: Condition checked before execution
        \item \textbf{Post-tested}: Condition checked after execution
    \end{itemize}

    \begin{mnemonicbox}"While Before, Do After" (WB-DA)\end{mnemonicbox}
\end{solutionbox}

\questionmarks{2}{c}{7}
\textbf{How is scanf() function used for formatted input? Explain with example}

\begin{solutionbox}
    \textbf{Answer}:

    \textbf{scanf() Function:}

    \begin{tabulary}{\linewidth}{|L|L|}
        \hline
        \textbf{Feature} & \textbf{Description} \\
        \hline
        Purpose & Read formatted input from keyboard \\
        \hline
        Syntax & \code{scanf("format\_string", \&variable);} \\
        \hline
        Return & Number of successfully read inputs \\
        \hline
    \end{tabulary}

    \textbf{Format Specifiers:}

    \begin{tabulary}{\linewidth}{|L|L|}
        \hline
        \textbf{Specifier} & \textbf{Data Type} \\
        \hline
        \%d & int \\
        \hline
        \%f & float \\
        \hline
        \%c & char \\
        \hline
        \%s & string \\
        \hline
    \end{tabulary}

    \textbf{Examples:}

\begin{lstlisting}[language=C]
int age;
float salary;
char grade;

scanf("%d", &age);           // Read integer
scanf("%f", &salary);        // Read float
scanf("%c", &grade);         // Read character
scanf("%d %f", &age, &salary); // Multiple inputs
\end{lstlisting}

    \textbf{Important Points:}

    \begin{itemize}
        \item \textbf{Address operator (\&)}: Required for variables
        \item \textbf{Format string}: Must match data types
        \item \textbf{Buffer issues}: Use \code{fflush(stdin)} if needed
    \end{itemize}

    \begin{mnemonicbox}"Address Format Match" (AFM)\end{mnemonicbox}
\end{solutionbox}

\orquestionmarks{2}{a}{3}
\textbf{List arithmetic and relational operators of C language}

\begin{solutionbox}
    \textbf{Answer}:

    \begin{tabulary}{\linewidth}{|L|L|L|}
        \hline
        \textbf{Operator Type} & \textbf{Operators} & \textbf{Purpose} \\
        \hline
        \textbf{Arithmetic} & +, -, *, /, \% & Mathematical operations \\
        \hline
        \textbf{Relational} & <, >, <=, >=, ==, != & Comparison operations \\
        \hline
    \end{tabulary}

    \textbf{Examples:}

\begin{lstlisting}[language=C]
// Arithmetic
int a = 10 + 5;    // Addition
int b = 10 % 3;    // Modulus (remainder)

// Relational
if(a > b)          // Greater than
if(a == b)         // Equal to
\end{lstlisting}

    \begin{mnemonicbox}"Add Multiply Compare" (AMC)\end{mnemonicbox}
\end{solutionbox}

\questionmarks{2}{b}{4}
\textbf{Draw flow chart of else if ladder.}

\begin{solutionbox}
    \textbf{Answer}:

    \begin{center}
    \begin{tikzpicture}[gtu flow]
        \node[gtu start] (start) {Start};
        \node[gtu decision, alias=cond1, below=of start] {Condition 1?};
        \node[gtu process, alias=stmt1, right=of cond1, xshift=2cm] {Statement 1};
        
        \node[gtu decision, alias=cond2, below=of cond1, yshift=-1cm] {Condition 2?};
        \node[gtu process, alias=stmt2, right=of cond2, xshift=2cm] {Statement 2};
        
        \node[gtu decision, alias=cond3, below=of cond2, yshift=-1cm] {Condition 3?};
        \node[gtu process, alias=stmt3, right=of cond3, xshift=2cm] {Statement 3};
        
        \node[gtu process, alias=else, below=of cond3] {Else Statement};
        \node[gtu stop, below=of else] (end) {End};

        \draw[gtu arrow] (start) -- (cond1);
        \draw[gtu arrow] (cond1) -- node[above] {True} (stmt1);
        \draw[gtu arrow] (cond1) -- node[right] {False} (cond2);
        
        \draw[gtu arrow] (cond2) -- node[above] {True} (stmt2);
        \draw[gtu arrow] (cond2) -- node[right] {False} (cond3);
        
        \draw[gtu arrow] (cond3) -- node[above] {True} (stmt3);
        \draw[gtu arrow] (cond3) -- node[right] {False} (else);
        
        % Connect all statements to End
        \draw[gtu arrow] (stmt1) -| (end);
        \draw[gtu arrow] (stmt2) -| (end);
        \draw[gtu arrow] (stmt3) -| (end);
        \draw[gtu arrow] (else) -- (end);
    \end{tikzpicture}
    \end{center}

    \textbf{Structure:}

    \begin{itemize}
        \item \textbf{Multiple conditions}: Checked sequentially
        \item \textbf{First true}: Corresponding block executes
        \item \textbf{Default case}: Else block for no match
    \end{itemize}

    \begin{mnemonicbox}"Check First True Execute" (CFTE)\end{mnemonicbox}
\end{solutionbox}

\questionmarks{2}{c}{7}
\textbf{How is printf() function used for formatted output? Explain with example}

\begin{solutionbox}
    \textbf{Answer}:

    \textbf{printf() Function:}

    \begin{tabulary}{\linewidth}{|L|L|}
        \hline
        \textbf{Feature} & \textbf{Description} \\
        \hline
        Purpose & Display formatted output on screen \\
        \hline
        Syntax & \code{printf("format\_string", variables);} \\
        \hline
        Return & Number of characters printed \\
        \hline
    \end{tabulary}

    \textbf{Format Specifiers:}

    \begin{tabulary}{\linewidth}{|L|L|L|}
        \hline
        \textbf{Specifier} & \textbf{Usage} & \textbf{Example} \\
        \hline
        \%d & Integer & \code{printf("\%d", 25);} \\
        \hline
        \%f & Float & \code{printf("\%.2f", 3.14);} \\
        \hline
        \%c & Character & \code{printf("\%c", 'A');} \\
        \hline
        \%s & String & \code{printf("\%s", "Hello");} \\
        \hline
    \end{tabulary}

    \textbf{Advanced Formatting:}

\begin{lstlisting}[language=C]
int num = 123;
float pi = 3.14159;

printf("Number: %5d\n", num);      // Width specification
printf("Pi: %.2f\n", pi);          // Precision specification
printf("Hex: %x\n", num);          // Hexadecimal
printf("Left aligned: %-10d\n", num); // Left alignment
\end{lstlisting}

    \textbf{Escape Sequences:}

    \begin{itemize}
        \item \textbf{\textbackslash n}: New line
        \item \textbf{\textbackslash t}: Tab space
        \item \textbf{\textbackslash\textbackslash}: Backslash
    \end{itemize}

    \begin{mnemonicbox}"Format Width Precision Align" (FWPA)\end{mnemonicbox}
\end{solutionbox}

\questionmarks{3}{a}{3}
\textbf{List Logical operators and explain it}

\begin{solutionbox}
    \textbf{Answer}:

    \begin{tabulary}{\linewidth}{|L|C|L|L|}
        \hline
        \textbf{Operator} & \textbf{Symbol} & \textbf{Description} & \textbf{Truth Table} \\
        \hline
        \textbf{AND} & \&\& & True if both operands true & T\&\&T = T, others = F \\
        \hline
        \textbf{OR} & || & True if any operand true & F||F = F, others = T \\
        \hline
        \textbf{NOT} & ! & Inverts the condition & !T = F, !F = T \\
        \hline
    \end{tabulary}

    \textbf{Examples:}

\begin{lstlisting}[language=C]
int a = 5, b = 10;

if(a > 0 && b > 0)    // Both conditions must be true
if(a > 15 || b > 5)   // At least one condition true  
if(!(a > 10))         // Negation of condition
\end{lstlisting}

    \begin{mnemonicbox}"And Or Not" (AON)\end{mnemonicbox}
\end{solutionbox}

\questionmarks{3}{b}{4}
\textbf{Explain for loop with example.}

\begin{solutionbox}
    \textbf{Answer}:

    \textbf{For Loop Structure:}

    \begin{tabulary}{\linewidth}{|L|L|}
        \hline
        \textbf{Component} & \textbf{Purpose} \\
        \hline
        Initialization & Set starting value \\
        \hline
        Condition & Test for continuation \\
        \hline
        Update & Modify loop variable \\
        \hline
    \end{tabulary}

    \textbf{Syntax:}

\begin{lstlisting}[language=C]
for(initialization; condition; update) {
    statements;
}
\end{lstlisting}

    \textbf{Example:}

\begin{lstlisting}[language=C]
// Print numbers 1 to 5
for(int i = 1; i <= 5; i++) {
    printf("%d ", i);
}
// Output: 1 2 3 4 5
\end{lstlisting}

    \textbf{Execution Flow:}

    \begin{itemize}
        \item \textbf{Step 1}: Initialize i = 1
        \item \textbf{Step 2}: Check condition i <= 5
        \item \textbf{Step 3}: Execute statements
        \item \textbf{Step 4}: Update i++, repeat from step 2
    \end{itemize}

    \begin{mnemonicbox}"Initialize Check Execute Update" (ICEU)\end{mnemonicbox}
\end{solutionbox}

\questionmarks{3}{c}{7}
\textbf{Write a program to find maximum out of three integer numbers x and y.}

\begin{solutionbox}
    \textbf{Answer}:

\begin{lstlisting}[language=C]
#include <stdio.h>

int main() {
    int x, y, z, max;
    
    printf("Enter three numbers: ");
    scanf("%d %d %d", &x, &y, &z);
    
    max = x;  // Assume first number is maximum
    
    if(y > max) {
        max = y;
    }
    if(z > max) {
        max = z;
    }
    
    printf("Maximum number is: %d", max);
    
    return 0;
}
\end{lstlisting}

    \textbf{Algorithm Steps:}

    \begin{tabulary}{\linewidth}{|C|L|}
        \hline
        \textbf{Step} & \textbf{Action} \\
        \hline
        1 & Input three numbers \\
        \hline
        2 & Assume first as maximum \\
        \hline
        3 & Compare with second, update if larger \\
        \hline
        4 & Compare with third, update if larger \\
        \hline
        5 & Display maximum \\
        \hline
    \end{tabulary}

    \textbf{Alternative Method:}

\begin{lstlisting}[language=C]
max = (x > y) ? ((x > z) ? x : z) : ((y > z) ? y : z);
\end{lstlisting}

    \begin{mnemonicbox}"Assume Compare Update Display" (ACUD)\end{mnemonicbox}
\end{solutionbox}

\orquestionmarks{3}{a}{3}
\textbf{Explain conditional operator with example.}

\begin{solutionbox}
    \textbf{Answer}:

    \textbf{Conditional Operator (Ternary):}

    \begin{tabulary}{\linewidth}{|L|L|}
        \hline
        \textbf{Feature} & \textbf{Description} \\
        \hline
        Symbol & ?: \\
        \hline
        Syntax & \code{condition ? value1 : value2} \\
        \hline
        Purpose & Shortcut for if-else \\
        \hline
    \end{tabulary}

    \textbf{Examples:}

\begin{lstlisting}[language=C]
int a = 10, b = 20;
int max = (a > b) ? a : b;        // max = 20

char grade = (marks >= 60) ? 'P' : 'F';
printf("Status: %s", (age >= 18) ? "Adult" : "Minor");
\end{lstlisting}

    \textbf{Equivalent if-else:}

\begin{lstlisting}[language=C]
if(a > b)
    max = a;
else
    max = b;
\end{lstlisting}

    \textbf{Advantages:}

    \begin{itemize}
        \item \textbf{Concise}: Single line expression
        \item \textbf{Efficient}: Faster execution
    \end{itemize}

    \begin{mnemonicbox}"Question Mark Colon Choice" (QMCC)\end{mnemonicbox}
\end{solutionbox}

\questionmarks{3}{b}{4}
\textbf{Explain while loop with example.}

\begin{solutionbox}
    \textbf{Answer}:

    \textbf{While Loop:}

    \begin{tabulary}{\linewidth}{|L|L|}
        \hline
        \textbf{Feature} & \textbf{Description} \\
        \hline
        Type & Entry-controlled loop \\
        \hline
        Syntax & \code{while(condition) \{ statements; \}} \\
        \hline
        Execution & Repeats while condition is true \\
        \hline
    \end{tabulary}

    \textbf{Example:}

\begin{lstlisting}[language=C]
int i = 1;
while(i <= 5) {
    printf("%d ", i);
    i++;
}
// Output: 1 2 3 4 5
\end{lstlisting}

    \textbf{Important Points:}

    \begin{itemize}
        \item \textbf{Initialization}: Before loop
        \item \textbf{Condition}: Checked at beginning  
        \item \textbf{Update}: Inside loop body
        \item \textbf{Infinite loop}: If condition never becomes false
    \end{itemize}

    \textbf{Flowchart Structure:}

    \begin{center}
    \begin{tikzpicture}[gtu flow]
        \node[gtu start] (start) {Initialize};
        \node[gtu decision, alias=cond, below=of start] {Condition?};
        \node[gtu process, below=of cond] (body) {Execute Statements};
        \node[gtu process, below=of body] (update) {Update Variable};
        \node[gtu stop, right=of cond, xshift=2cm] (end) {Exit Loop};

        \draw[gtu arrow] (start) -- (cond);
        \draw[gtu arrow] (cond) -- node[left] {True} (body);
        \draw[gtu arrow] (body) -- (update);
        \draw[gtu arrow] (update.west) -- ++(-1,0) |- (cond.west);
        \draw[gtu arrow] (cond) -- node[above] {False} (end);
    \end{tikzpicture}
    \end{center}

    \begin{mnemonicbox}"Initialize Check Execute Update" (ICEU)\end{mnemonicbox}
\end{solutionbox}

\questionmarks{3}{c}{7}
\textbf{WAP to read an integer from key board and print whether given number is odd or even.}

\begin{solutionbox}
    \textbf{Answer}:

\begin{lstlisting}[language=C]
#include <stdio.h>

int main() {
    int number;
    
    printf("Enter an integer: ");
    scanf("%d", &number);
    
    if(number % 2 == 0) {
        printf("%d is Even number", number);
    }
    else {
        printf("%d is Odd number", number);
    }
    
    return 0;
}
\end{lstlisting}

    \textbf{Logic Explanation:}

    \begin{tabulary}{\linewidth}{|L|L|}
        \hline
        \textbf{Concept} & \textbf{Description} \\
        \hline
        \textbf{Modulus operator (\%)} & Returns remainder after division \\
        \hline
        \textbf{Even condition} & \code{number \% 2 == 0} \\
        \hline
        \textbf{Odd condition} & \code{number \% 2 != 0} \\
        \hline
    \end{tabulary}

    \textbf{Alternative Methods:}

\begin{lstlisting}[language=C]
// Method 2: Using conditional operator
printf("%d is %s", number, (number % 2 == 0) ? "Even" : "Odd");

// Method 3: Using bitwise AND
if(number & 1)
    printf("Odd");
else
    printf("Even");
\end{lstlisting}

    \textbf{Sample Output:}

\begin{lstlisting}
Enter an integer: 7
7 is Odd number
\end{lstlisting}

    \begin{mnemonicbox}"Modulus Two Zero Even" (MTZE)\end{mnemonicbox}
\end{solutionbox}

\questionmarks{4}{a}{3}
\textbf{Evaluate following arithmetic expressions: 30/4*4 – 20\%6 + 17/2}

\begin{solutionbox}
    \textbf{Answer}:

    \textbf{Step-by-step Evaluation:}

    \begin{tabulary}{\linewidth}{|C|L|L|C|}
        \hline
        \textbf{Step} & \textbf{Expression} & \textbf{Calculation} & \textbf{Result} \\
        \hline
        1 & 30/4*4 & (30/4)*4 = 7*4 & 28 \\
        \hline
        2 & 20\%6 & 20 mod 6 & 2 \\
        \hline
        3 & 17/2 & Integer division & 8 \\
        \hline
        4 & Final & 28 - 2 + 8 & 34 \\
        \hline
    \end{tabulary}

    \textbf{Operator Precedence:}

    \begin{tabulary}{\linewidth}{|L|L|}
        \hline
        \textbf{Priority} & \textbf{Operators} \\
        \hline
        High & *, /, \% (Left to right) \\
        \hline
        Low & +, - (Left to right) \\
        \hline
    \end{tabulary}

    \textbf{Complete Calculation:}

\begin{lstlisting}
30/4*4 - 20%6 + 17/2
= 7*4 - 2 + 8      // Division and modulus first
= 28 - 2 + 8       // Multiplication
= 26 + 8           // Left to right for +,-
= 34               // Final answer
\end{lstlisting}

    \begin{mnemonicbox}"Multiply Divide Before Add Subtract" (MDBAS)\end{mnemonicbox}
\end{solutionbox}

\questionmarks{4}{b}{4}
\textbf{WAP to find sum and average of an array of 5 integer numbers.}

\begin{solutionbox}
    \textbf{Answer}:

\begin{lstlisting}[language=C]
#include <stdio.h>

int main() {
    int numbers[5];
    int sum = 0;
    float average;
    
    printf("Enter 5 integers:\n");
    for(int i = 0; i < 5; i++) {
        scanf("%d", &numbers[i]);
        sum += numbers[i];
    }
    
    average = (float)sum / 5;
    
    printf("Sum = %d\n", sum);
    printf("Average = %.2f", average);
    
    return 0;
}
\end{lstlisting}

    \textbf{Algorithm:}

    \begin{enumerate}
        \item Declare array of 5 integers
        \item Initialize sum to 0
        \item Input 5 numbers using loop
        \item Add each number to sum
        \item Calculate average = sum/5
        \item Display results
    \end{enumerate}

    \textbf{Key Points:}

    \begin{itemize}
        \item \textbf{Type casting}: \code{(float)sum} for accurate division
        \item \textbf{Loop usage}: Efficient for repetitive input
    \end{itemize}

    \begin{mnemonicbox}"Declare Input Add Calculate Display" (DIACD)\end{mnemonicbox}
\end{solutionbox}

\questionmarks{4}{c}{7}
\textbf{Define pointer. Explain how pointers are declared and initialized with example.}

\begin{solutionbox}
    \textbf{Answer}:

    \textbf{Pointer Definition:}

    \begin{tabulary}{\linewidth}{|L|L|}
        \hline
        \textbf{Aspect} & \textbf{Description} \\
        \hline
        Definition & Variable that stores memory address of another variable \\
        \hline
        Purpose & Direct memory access and dynamic memory allocation \\
        \hline
        Symbol & * (asterisk) for declaration and dereferencing \\
        \hline
    \end{tabulary}

    \textbf{Declaration and Initialization:}

\begin{lstlisting}[language=C]
// Declaration
int *ptr;           // Pointer to integer
float *fptr;        // Pointer to float
char *cptr;         // Pointer to character

// Initialization
int num = 10;
int *ptr = &num;    // Initialize with address of num

// Alternative
int *ptr;
ptr = &num;         // Assign address later
\end{lstlisting}

    \textbf{Example Program:}

\begin{lstlisting}[language=C]
#include <stdio.h>

int main() {
    int num = 25;
    int *ptr = &num;
    
    printf("Value of num: %d\n", num);
    printf("Address of num: %p\n", &num);
    printf("Value of ptr: %p\n", ptr);
    printf("Value pointed by ptr: %d\n", *ptr);
    
    return 0;
}
\end{lstlisting}

    \textbf{Memory Diagram:}

    \begin{center}
    \begin{tikzpicture}[gtu flow]
        \node[gtu block, minimum width=2.5cm] (num) {\textbf{num} \\ Value: 25 \\ Addr: 1000};
        \node[gtu block, minimum width=2.5cm, right=of num, xshift=2cm] (ptr) {\textbf{ptr} \\ Value: 1000 \\ Addr: 2000};
        
        \draw[gtu arrow] (ptr.west) -- node[above] {Points to} (num.east);
    \end{tikzpicture}
    \end{center}

    \begin{mnemonicbox}"Address Star Dereference" (ASD)\end{mnemonicbox}
\end{solutionbox}

\orquestionmarks{4}{a}{3}
\textbf{Evaluate following arithmetic expressions: 50 / 3 \% 3 + 5 * 7}

\begin{solutionbox}
    \textbf{Answer}:

    \textbf{Step-by-step Evaluation:}

    \begin{tabulary}{\linewidth}{|C|L|L|C|}
        \hline
        \textbf{Step} & \textbf{Expression} & \textbf{Calculation} & \textbf{Result} \\
        \hline
        1 & 50/3 & Integer division & 16 \\
        \hline
        2 & 16\%3 & 16 mod 3 & 1 \\
        \hline
        3 & 5*7 & Multiplication & 35 \\
        \hline
        4 & Final & 1 + 35 & 36 \\
        \hline
    \end{tabulary}

    \textbf{Complete Calculation:}

\begin{lstlisting}
50 / 3 % 3 + 5 * 7
= 16 % 3 + 35      // Division and multiplication first
= 1 + 35           // Modulus operation
= 36               // Final answer
\end{lstlisting}

    \textbf{Operator Precedence Applied:}

    \begin{itemize}
        \item \textbf{High priority}: /, \%, * (left to right)
        \item \textbf{Low priority}: + (left to right)
    \end{itemize}

    \begin{mnemonicbox}"Divide Mod Multiply Add" (DMMA)\end{mnemonicbox}
\end{solutionbox}

\questionmarks{4}{b}{4}
\textbf{WAP to find the largest number in an array of N integers.}

\begin{solutionbox}
    \textbf{Answer}:

\begin{lstlisting}[language=C]
#include <stdio.h>

int main() {
    int n, i;
    int largest;
    
    printf("Enter number of elements: ");
    scanf("%d", &n);
    
    int arr[n];
    
    printf("Enter %d numbers:\n", n);
    for(i = 0; i < n; i++) {
        scanf("%d", &arr[i]);
    }
    
    largest = arr[0];  // Assume first element is largest
    
    for(i = 1; i < n; i++) {
        if(arr[i] > largest) {
            largest = arr[i];
        }
    }
    
    printf("Largest number is: %d", largest);
    
    return 0;
}
\end{lstlisting}

    \textbf{Algorithm:}

    \begin{enumerate}
        \item Input array size
        \item Input array elements
        \item Assume first element as largest
        \item Compare with remaining elements
        \item Update largest if bigger found
        \item Display result
    \end{enumerate}

    \begin{mnemonicbox}"Input Assume Compare Update Display" (IACUD)\end{mnemonicbox}
\end{solutionbox}

\questionmarks{4}{c}{7}
\textbf{Define array. Explain the need for array variable. Explain 1-D array with example}

\begin{solutionbox}
    \textbf{Answer}:

    \textbf{Array Definition:}

    \begin{tabulary}{\linewidth}{|L|L|}
        \hline
        \textbf{Aspect} & \textbf{Description} \\
        \hline
        Definition & Collection of similar data type elements \\
        \hline
        Storage & Consecutive memory locations \\
        \hline
        Access & Using index/subscript \\
        \hline
    \end{tabulary}

    \textbf{Need for Arrays:}

    \begin{tabulary}{\linewidth}{|L|L|}
        \hline
        \textbf{Problem} & \textbf{Solution with Array} \\
        \hline
        Store multiple values & Single array variable \\
        \hline
        Avoid multiple variables & \code{arr[100]} instead of a1, a2, ..., a100 \\
        \hline
        Efficient processing & Loop-based operations \\
        \hline
        Memory organization & Contiguous allocation \\
        \hline
    \end{tabulary}

    \textbf{1-D Array Declaration:}

\begin{lstlisting}[language=C]
datatype arrayname[size];

// Examples
int marks[5];           // Array of 5 integers
float prices[10];       // Array of 10 floats
char name[20];         // Array of 20 characters
\end{lstlisting}

    \textbf{Array Initialization:}

\begin{lstlisting}[language=C]
// Method 1: At declaration
int numbers[5] = {10, 20, 30, 40, 50};

// Method 2: Individual assignment
int arr[3];
arr[0] = 5;
arr[1] = 15;
arr[2] = 25;
\end{lstlisting}

    \textbf{Example Program:}

\begin{lstlisting}[language=C]
#include <stdio.h>

int main() {
    int marks[5] = {85, 90, 78, 92, 88};
    int i, sum = 0;
    
    printf("Student marks:\n");
    for(i = 0; i < 5; i++) {
        printf("Subject %d: %d\n", i+1, marks[i]);
        sum += marks[i];
    }
    
    printf("Total marks: %d", sum);
    return 0;
}
\end{lstlisting}

    \textbf{Memory Layout:}

    \begin{center}
    \begin{tikzpicture}[gtu flow]
        \foreach \x/\val/\addr in {0/85/1000, 1/90/1004, 2/78/1008, 3/92/1012, 4/88/1016} {
            \node[draw, rectangle, minimum size=1cm] (n\x) at (\x*1.5, 0) {\val};
            \node[below=0.1cm of n\x] {\footnotesize marks[\x]};
            \node[above=0.1cm of n\x] {\footnotesize \addr};
        }
    \end{tikzpicture}
    \end{center}

    \begin{mnemonicbox}"Similar Data Consecutive Index" (SDCI)\end{mnemonicbox}
\end{solutionbox}

\questionmarks{5}{a}{3}
\textbf{Give an example of if … else statement.}

\begin{solutionbox}
    \textbf{Answer}:

\begin{lstlisting}[language=C]
#include <stdio.h>

int main() {
    int age;
    
    printf("Enter your age: ");
    scanf("%d", &age);
    
    if(age >= 18) {
        printf("You are eligible to vote");
    }
    else {
        printf("You are not eligible to vote");
    }
    
    return 0;
}
\end{lstlisting}

    \textbf{Structure:}

    \begin{tabulary}{\linewidth}{|L|L|}
        \hline
        \textbf{Component} & \textbf{Purpose} \\
        \hline
        \textbf{if} & Tests condition \\
        \hline
        \textbf{condition} & Boolean expression \\
        \hline
        \textbf{if-block} & Executes when condition true \\
        \hline
        \textbf{else-block} & Executes when condition false \\
        \hline
    \end{tabulary}

    \textbf{Sample Outputs:}

\begin{lstlisting}
Input: 20    Output: You are eligible to vote
Input: 16    Output: You are not eligible to vote
\end{lstlisting}

    \begin{mnemonicbox}"If True Else False" (ITEF)\end{mnemonicbox}
\end{solutionbox}

\questionmarks{5}{b}{4}
\textbf{WAP to check the category of given character.}

\begin{solutionbox}
    \textbf{Answer}:

\begin{lstlisting}[language=C]
#include <stdio.h>
#include <ctype.h>

int main() {
    char ch;
    
    printf("Enter a character: ");
    scanf("%c", &ch);
    
    if(isdigit(ch)) {
        printf("'%c' is a Digit", ch);
    }
    else if(isupper(ch)) {
        printf("'%c' is an Uppercase letter", ch);
    }
    else if(islower(ch)) {
        printf("'%c' is a Lowercase letter", ch);
    }
    else {
        printf("'%c' is a Special symbol", ch);
    }
    
    return 0;
}
\end{lstlisting}

    \textbf{Character Categories:}

    \begin{tabulary}{\linewidth}{|L|L|L|}
        \hline
        \textbf{Function} & \textbf{Category} & \textbf{Range} \\
        \hline
        isdigit() & Digit & 0-9 \\
        \hline
        isupper() & Uppercase & A-Z \\
        \hline
        islower() & Lowercase & a-z \\
        \hline
        Others & Special symbols & !@\#\$\%\^{}\&* etc. \\
        \hline
    \end{tabulary}

    \textbf{Alternative Method:}

\begin{lstlisting}[language=C]
if(ch >= '0' && ch <= '9')
    printf("Digit");
else if(ch >= 'A' && ch <= 'Z')
    printf("Uppercase");
else if(ch >= 'a' && ch <= 'z')
    printf("Lowercase");
else
    printf("Special symbol");
\end{lstlisting}

    \begin{mnemonicbox}"Digit Upper Lower Special" (DULS)\end{mnemonicbox}
\end{solutionbox}

\questionmarks{5}{c}{7}
\textbf{What is structure? Explain its syntax with suitable example}

\begin{solutionbox}
    \textbf{Answer}:

    \textbf{Structure Definition:}

    \begin{tabulary}{\linewidth}{|L|L|}
        \hline
        \textbf{Aspect} & \textbf{Description} \\
        \hline
        Definition & User-defined data type combining different data types \\
        \hline
        Purpose & Group related data under single name \\
        \hline
        Keyword & \code{struct} \\
        \hline
    \end{tabulary}

    \textbf{Syntax:}

\begin{lstlisting}[language=C]
struct structure_name {
    datatype member1;
    datatype member2;
    ...
};
\end{lstlisting}

    \textbf{Example - Student Structure:}

\begin{lstlisting}[language=C]
#include <stdio.h>

struct Student {
    int roll_no;
    char name[50];
    float marks;
    char grade;
};

int main() {
    struct Student s1;
    
    // Input data
    printf("Enter roll number: ");
    scanf("%d", &s1.roll_no);
    
    printf("Enter name: ");
    scanf("%s", s1.name);
    
    printf("Enter marks: ");
    scanf("%f", &s1.marks);
    
    printf("Enter grade: ");
    scanf(" %c", &s1.grade);
    
    // Display data
    printf("\nStudent Details:\n");
    printf("Roll No: %d\n", s1.roll_no);
    printf("Name: %s\n", s1.name);
    printf("Marks: %.2f\n", s1.marks);
    printf("Grade: %c\n", s1.grade);
    
    return 0;
}
\end{lstlisting}

    \textbf{Structure Features:}

    \begin{itemize}
        \item \textbf{Dot operator (.)}: Access structure members
        \item \textbf{Memory allocation}: Total size = sum of all members
        \item \textbf{Initialization}: Can initialize at declaration
    \end{itemize}

    \textbf{Structure Initialization:}

\begin{lstlisting}[language=C]
struct Student s1 = {101, "John", 85.5, 'A'};
\end{lstlisting}

    \textbf{Memory Layout:}

    \begin{center}
    \begin{tikzpicture}[gtu flow]
        \node[gtu block, minimum width=4cm] (struct) {
            \textbf{s1 Structure} \\
            \rule{3.8cm}{0.4pt} \\
            roll\_no (4 bytes) \\
            name (50 bytes) \\
            marks (4 bytes) \\
            grade (1 byte)
        };
    \end{tikzpicture}
    \end{center}

    \begin{mnemonicbox}"Group Related Data Together" (GRDT)\end{mnemonicbox}
\end{solutionbox}

\orquestionmarks{5}{a}{3}
\textbf{WAP to Print all numbers between -5 \& +5.}

\begin{solutionbox}
    \textbf{Answer}:

\begin{lstlisting}[language=C]
#include <stdio.h>

int main() {
    int i;
    
    printf("Numbers between -5 and +5:\n");
    
    for(i = -5; i <= 5; i++) {
        printf("%d ", i);
    }
    
    return 0;
}
\end{lstlisting}

    \textbf{Output:}

\begin{lstlisting}
Numbers between -5 and +5:
-5 -4 -3 -2 -1 0 1 2 3 4 5
\end{lstlisting}

    \textbf{Alternative Methods:}

\begin{lstlisting}[language=C]
// Method 2: Using while loop
int i = -5;
while(i <= 5) {
    printf("%d ", i);
    i++;
}

// Method 3: Two separate loops
for(i = -5; i < 0; i++)
    printf("%d ", i);
printf("0 ");
for(i = 1; i <= 5; i++)
    printf("%d ", i);
\end{lstlisting}

    \begin{mnemonicbox}"Start Negative End Positive" (SNEP)\end{mnemonicbox}
\end{solutionbox}

\questionmarks{5}{b}{4}
\textbf{WAP to find roots of quadratic equation.}

\begin{solutionbox}
    \textbf{Answer}:

\begin{lstlisting}[language=C]
#include <stdio.h>
#include <math.h>

int main() {
    float a, b, c;
    float discriminant, root1, root2;
    
    printf("Enter coefficients (a, b, c): ");
    scanf("%f %f %f", &a, &b, &c);
    
    discriminant = b*b - 4*a*c;
    
    if(discriminant > 0) {
        root1 = (-b + sqrt(discriminant)) / (2*a);
        root2 = (-b - sqrt(discriminant)) / (2*a);
        printf("Roots are real and different\n");
        printf("Root1 = %.2f\n", root1);
        printf("Root2 = %.2f\n", root2);
    }
    else if(discriminant == 0) {
        root1 = -b / (2*a);
        printf("Roots are real and equal\n");
        printf("Root = %.2f\n", root1);
    }
    else {
        float realPart = -b / (2*a);
        float imagPart = sqrt(-discriminant) / (2*a);
        printf("Roots are complex\n");
        printf("Root1 = %.2f + %.2fi\n", realPart, imagPart);
        printf("Root2 = %.2f - %.2fi\n", realPart, imagPart);
    }
    
    return 0;
}
\end{lstlisting}

    \textbf{Quadratic Formula Analysis:}

    \begin{tabulary}{\linewidth}{|L|L|}
        \hline
        \textbf{Discriminant} & \textbf{Nature of Roots} \\
        \hline
        $b^2-4ac > 0$ & Real and different \\
        \hline
        $b^2-4ac = 0$ & Real and equal \\
        \hline
        $b^2-4ac < 0$ & Complex (imaginary) \\
        \hline
    \end{tabulary}

    \textbf{Formula:} $x = \frac{-b \pm \sqrt{b^2-4ac}}{2a}$

    \begin{mnemonicbox}"Discriminant Decides Root Nature" (DDRN)\end{mnemonicbox}
\end{solutionbox}

\questionmarks{5}{c}{7}
\textbf{Explain following built-in functions with examples}

\begin{solutionbox}
    \textbf{Answer}:

    \textbf{Function Explanations:}

    \begin{tabulary}{\linewidth}{|L|L|L|L|}
        \hline
        \textbf{Function} & \textbf{Purpose} & \textbf{Header File} & \textbf{Example} \\
        \hline
        \code{clrscr()} & Clear screen & conio.h & \code{clrscr();} \\
        \hline
        \code{sqrt()} & Square root & math.h & \code{sqrt(16) = 4.0} \\
        \hline
        \code{strlen()} & String length & string.h & \code{strlen("Hello") = 5} \\
        \hline
        \code{isdigit()} & Check if digit & ctype.h & \code{isdigit('5') = true} \\
        \hline
        \code{isalpha()} & Check if alphabet & ctype.h & \code{isalpha('A') = true} \\
        \hline
        \code{toupper()} & Convert to uppercase & ctype.h & \code{toupper('a') = 'A'} \\
        \hline
        \code{tolower()} & Convert to lowercase & ctype.h & \code{tolower('B') = 'b'} \\
        \hline
    \end{tabulary}

    \textbf{Example Program:}

\begin{lstlisting}[language=C]
#include <stdio.h>
#include <math.h>
#include <string.h>
#include <ctype.h>

int main() {
    // clrscr(); // Not standard in modern compilers
    
    // sqrt() example
    float num = 25.0;
    printf("Square root of %.1f = %.2f\n", num, sqrt(num));
    
    // strlen() example
    char str[] = "Programming";
    printf("Length of '%s' = %d\n", str, strlen(str));
    
    // Character functions
    char ch = 'a';
    printf("'%c' is digit: %s\n", ch, isdigit(ch) ? "Yes" : "No");
    printf("'%c' is alphabet: %s\n", ch, isalpha(ch) ? "Yes" : "No");
    printf("Uppercase of '%c' = '%c'\n", ch, toupper(ch));
    
    return 0;
}
\end{lstlisting}

    \textbf{Function Categories:}

    \begin{center}
    \begin{tikzpicture}[gtu flow]
        \node[gtu block] (builtin) {Built-in Functions};
        \node[gtu block, below left=of builtin, xshift=-2cm] (screen) {Screen Control: clrscr()};
        \node[gtu block, below left=of builtin, xshift=1cm] (math) {Mathematical: sqrt()};
        \node[gtu block, below right=of builtin, xshift=-1cm] (string) {String: strlen()};
        \node[gtu block, below right=of builtin, xshift=2cm] (char) {Character: isdigit, isalpha, toupper, tolower};

        \draw[gtu arrow] (builtin) -- (screen);
        \draw[gtu arrow] (builtin) -- (math);
        \draw[gtu arrow] (builtin) -- (string);
        \draw[gtu arrow] (builtin) -- (char);
    \end{tikzpicture}
    \end{center}

    \begin{mnemonicbox}"Clear Math String Character" (CMSC)\end{mnemonicbox}
\end{solutionbox}

\end{document}
