\documentclass{article}

% content/resources/templates/preamble.tex
\usepackage[margin=0.6in]{geometry}
\author{Milav Dabgar}
\usepackage{amsmath,amssymb,amsthm}
\usepackage{booktabs}
\usepackage{multirow}
\usepackage{xcolor}
\usepackage{tcolorbox}
\tcbuselibrary{breakable,skins}
\usepackage[colorlinks=true,linkcolor=blue]{hyperref}
\usepackage{titlesec}
\usepackage{enumitem}
\usepackage{tikz}
\usepackage{pgfplots}
\usepackage{circuitikz}
\usepackage[version=4]{mhchem}
\usepackage{longtable}
\usepackage{array}
\usepackage{float}
\usepackage{caption}
\usepackage{listings}

\lstset{
  basicstyle=\small\ttfamily,
  breaklines=true,
  breakatwhitespace=false,
  postbreak=\mbox{\textcolor{red}{$\hookrightarrow$}\space},
  float=false,
  numbers=left,
  numberstyle=\tiny\color{gray},
  numbersep=10pt,
  xleftmargin=2em,
  keywordstyle=\color{blue},
  commentstyle=\color{green!60!black},
  stringstyle=\color{purple},
  backgroundcolor=\color{gray!5},
  showstringspaces=false,
  tabsize=2,
  captionpos=b,
  keepspaces=true,
  columns=flexible
}

\pgfplotsset{compat=1.18}
\usetikzlibrary{shapes,arrows,positioning,calc,patterns,decorations.pathmorphing,decorations.markings,arrows.meta}

% Color scheme
\definecolor{headcolor}{RGB}{0,102,204}
\definecolor{keycolor}{RGB}{220,20,60}
\definecolor{solutioncolor}{RGB}{34,139,34}
\definecolor{mnemoniccolor}{RGB}{148,0,211}
\definecolor{codecolor}{RGB}{0,0,100}

% Spacing
\setlength{\parskip}{3pt}
\setlist[itemize]{nosep}
\setlist[enumerate]{nosep}

% Title formatting
\titleformat{\section}{\Large\bfseries\color{headcolor}}{\thesection}{1em}{}
\titleformat{\subsection}{\large\bfseries\color{headcolor}}{\thesubsection}{1em}{}

% Pandoc tightlist compatibility
\providecommand{\tightlist}{%
  \setlength{\itemsep}{0pt}\setlength{\parskip}{0pt}}

% Pandoc longtable compatibility
\newcounter{none}
\def\thenone{}


% content/resources/templates/gujarati-boxes.tex
\usepackage{fontspec}
\usepackage{polyglossia}

% Set Gujarati as main language (document is primarily in Gujarati)
% Note: gloss-gujarati.ldf doesn't exist in polyglossia, but it will use hyphenation patterns
\setdefaultlanguage{gujarati}
\setotherlanguage{english}

% Configure Gujarati font properly
% Use Language=Default to prevent polyglossia from trying to add language-specific features
% that don't exist for Gujarati, which causes "empty feature" warnings
\newfontfamily\gujaratifont[Script=Gujarati,AutoFakeBold=2.5,AutoFakeSlant=0.3]{Noto Sans Gujarati}
\setmainfont[Script=Gujarati,AutoFakeBold=2.5,AutoFakeSlant=0.3]{Noto Sans Gujarati}
% Use Noto Sans Gujarati for monospace to support Gujarati in text
\setmonofont[Scale=0.9]{Noto Sans Gujarati}

% Configure English to use the same font
\newfontfamily\englishfont[Script=Gujarati,AutoFakeBold=2.5,AutoFakeSlant=0.3]{Noto Sans Gujarati}

% Translations for polyglossia
\gappto\captionsgujarati{
  \renewcommand{\tablename}{કોષ્ટક}
  \renewcommand{\figurename}{આકૃતિ}
}

% Helper for TikZ nodes to ensure Gujarati font
\newcommand{\gu}[1]{{\gujaratifont #1}}

% Custom environments
\newtcolorbox{solutionbox}{
    breakable,
    enhanced,
    colback=solutioncolor!5!white,
    colframe=solutioncolor!75!black,
    fonttitle=\bfseries,
    title=જવાબ
}

\newtcolorbox{solutionboxnobreak}{
 colback=solutioncolor!5!white,
 colframe=solutioncolor!75!black,
 fonttitle=\bfseries,
 title=જવાબ
}

\newtcolorbox{keyformula}{
 breakable,
 enhanced,
 colback=keycolor!5!white,
 colframe=keycolor!75!black,
 fonttitle=\bfseries,
 title=રાસાયણિક સમીકરણ/સૂત્ર
}

\newtcolorbox{mnemonicbox}{
 breakable,
 enhanced,
 colback=mnemoniccolor!5!white,
 colframe=mnemoniccolor!75!black,
 fonttitle=\bfseries,
 title=મેમરી ટ્રીક
}


% Custom commands for GTU solutions
% This file defines semantic commands for consistent formatting

% Question command with automatic formatting
\newcommand{\question}[2]{%
  \section*{Question #1}%
  \textbf{#2}%
}

% OR question variant
\newcommand{\questionor}[2]{%
  \section*{Question #1 OR}%
  \textbf{#2}%
}

% Proper table environment with caption
\newenvironment{answertable}[1]{%
  \begin{table}[htbp]
  \centering
  \caption{#1}
}{%
  \end{table}
}

% Proper figure environment for diagrams
\newenvironment{answerdiagram}[1]{%
  \begin{figure}[htbp]
  \centering
  \caption{#1}
}{%
  \end{figure}
}

% Semantic markup for key terms
\newcommand{\keyword}[1]{\textbf{#1}}
\newcommand{\code}[1]{\texttt{#1}}
\newcommand{\classname}[1]{\texttt{#1}}
\newcommand{\methodname}[1]{\texttt{#1}}

% Proper quotation marks
\newcommand{\mnemonic}[1]{``#1''}


\title{Programming in C (4331105) - Summer 2025 Solution}
\date{May 20, 2025}

\begin{document}
\maketitle

\questionmarks{1}{a}{3}
\textbf{C માં કેટલા keywords છે? કોઈપણ ચાર keywords લખો}

\begin{solutionbox}
    \textbf{જવાબ}:

    \begin{tabulary}{\linewidth}{|L|L|}
        \hline
        \textbf{કુલ Keywords} & \textbf{ઉદાહરણો} \\
        \hline
        32 keywords & int, float, char, if \\
        \hline
    \end{tabulary}

    \textbf{આકૃતિ:}

    \begin{center}
    \begin{tikzpicture}[gtu flow]
        \node[gtu block] (keywords) {C Keywords - 32 કુલ};
        \node[gtu block, below left=of keywords, xshift=-1cm] (types) {Data Types: int, float, char, double};
        \node[gtu block, below=of keywords] (control) {Control: if, else, for, while};
        \node[gtu block, below right=of keywords, xshift=1cm] (storage) {Storage: static, extern, auto, register};

        \draw[gtu arrow] (keywords) -- (types);
        \draw[gtu arrow] (keywords) -- (control);
        \draw[gtu arrow] (keywords) -- (storage);
    \end{tikzpicture}
    \end{center}

    \begin{itemize}
        \item \textbf{32 keywords}: C ભાષામાં કુલ આરક્ષિત શબ્દો
        \item \textbf{Data type keywords}: int, float, char, double વેરિએબલ જાહેર કરવા માટે
        \item \textbf{Control keywords}: if, else, for, while પ્રોગ્રામ ફ્લો માટે
    \end{itemize}

    \begin{mnemonicbox}"બિલાડી ચાર રંગમાં" (char, int, float, const)\end{mnemonicbox}
\end{solutionbox}

\questionmarks{1}{b}{4}
\textbf{વેરિએબલ શું છે? ઉદાહરણ સાથે વેરિએબલને નામ આપવાના નિયમો સમજાવો}

\begin{solutionbox}
    \textbf{જવાબ}:

    \textbf{વેરિએબલ વ્યાખ્યા:}

    \begin{tabulary}{\linewidth}{|L|L|}
        \hline
        \textbf{પાસાં} & \textbf{વર્ણન} \\
        \hline
        વ્યાખ્યા & ડેટા સ્ટોર કરવા માટે નામવાળી મેમરી લોકેશન \\
        \hline
        હેતુ & પ્રોગ્રામ દરમિયાન બદલાતા મૂલ્યો રાખવા \\
        \hline
        જાહેરાત & \code{datatype variable\_name;} \\
        \hline
    \end{tabulary}

    \textbf{નામકરણના નિયમો:}

    \begin{itemize}
        \item \textbf{પ્રથમ અક્ષર}: અક્ષર અથવા underscore (\_) હોવો જોઈએ
        \item \textbf{પછીના અક્ષરો}: અક્ષરો, અંકો, underscore માત્ર
        \item \textbf{Case sensitive}: 'Age' અને 'age' અલગ છે
        \item \textbf{Keywords નહીં}: 'int', 'float' જેવા આરક્ષિત શબ્દો વાપરી શકાતા નથી
    \end{itemize}

    \textbf{ઉદાહરણો:}

\begin{lstlisting}[language=C]
int age;        // યોગ્ય
float _salary;  // યોગ્ય
char name123;   // યોગ્ય
int 2number;    // ખોટું - અંકથી શરૂ
float for;      // ખોટું - keyword વપરાયું
\end{lstlisting}

    \begin{mnemonicbox}"અક્ષર પહેલાં, keywords નહીં"\end{mnemonicbox}
\end{solutionbox}

\questionmarks{1}{c}{7}
\textbf{નીચેના statements માં errors શોધો}

\begin{solutionbox}
    \textbf{જવાબ}:

    \begin{tabulary}{\linewidth}{|L|L|L|}
        \hline
        \textbf{Statement} & \textbf{Error} & \textbf{કારણ} \\
        \hline
        (1) \code{fLoat x;} & ખોટો keyword & સાચું: \code{float x;} \\
        \hline
        (2) \code{int min, max = 20;} & અર્ધ initialization & માત્ર max initialize, min નહીં \\
        \hline
        (3) \code{long char c;} & ખોટું combination & long ને char સાથે વાપરી શકાતું નથી \\
        \hline
        (4) \code{iNt a;} & ખોટો keyword & સાચું: \code{int a;} \\
        \hline
        (5) \code{FLOAT f=2;} & ખોટો keyword & સાચું: \code{float f=2;} \\
        \hline
        (6) \code{double m ; n;} & Missing datatype & સાચું: \code{double m, n;} \\
        \hline
        (7) \code{Int score (100)0;} & અનેક errors & ખોટું syntax, સાચું: \code{int score = 100;} \\
        \hline
    \end{tabulary}

    \textbf{મુખ્ય મુદ્દાઓ:}

    \begin{itemize}
        \item \textbf{Case sensitivity}: Keywords નાના અક્ષરમાં હોવા જોઈએ
        \item \textbf{Multiple declaration}: Comma separator વાપરો
        \item \textbf{Initialization syntax}: = operator વાપરો
    \end{itemize}

    \begin{mnemonicbox}"Keywords હંમેશા નાના અક્ષરે"\end{mnemonicbox}
\end{solutionbox}

\orquestionmarks{1}{c}{7}
\textbf{અલ્ગોરિધમ શું છે? ફ્લોચાર્ટ શું છે? વર્તુળનો વિસ્તાર અને પરિમિતિ શોધવા માટે ફ્લોચાર્ટ દોરો.}

\begin{solutionbox}
    \textbf{જવાબ}:

    \textbf{વ્યાખ્યાઓ:}

    \begin{tabulary}{\linewidth}{|L|L|}
        \hline
        \textbf{શબ્દ} & \textbf{વ્યાખ્યા} \\
        \hline
        Algorithm & સમસ્યા હલ કરવાની પગલાબદ્ધ પ્રક્રિયા \\
        \hline
        Flowchart & Algorithm નું પ્રતીકો વડે દ્રશ્ય પ્રતિનિધિત્વ \\
        \hline
    \end{tabulary}

    \textbf{વર્તુળ માટે ફ્લોચાર્ટ:}

    \begin{center}
    \begin{tikzpicture}[gtu flow]
        \node[gtu start] (start) {શરૂઆત};
        \node[gtu input, below=of start] (input) {radius r input કરો};
        \node[gtu process, below=of input] (calc_area) {area = $\pi \times r^2$ ગણતરી કરો};
        \node[gtu process, below=of calc_area] (calc_perim) {perimeter = $2 \times \pi \times r$ ગણતરી કરો};
        \node[gtu output, below=of calc_perim] (display) {area અને perimeter દર્શાવો};
        \node[gtu stop, below=of display] (end) {અંત};

        \draw[gtu arrow] (start) -- (input);
        \draw[gtu arrow] (input) -- (calc_area);
        \draw[gtu arrow] (calc_area) -- (calc_perim);
        \draw[gtu arrow] (calc_perim) -- (display);
        \draw[gtu arrow] (display) -- (end);
    \end{tikzpicture}
    \end{center}

    \textbf{Algorithm ના પગલાં:}

    \begin{itemize}
        \item \textbf{પગલું 1}: શરૂઆત કરો
        \item \textbf{પગલું 2}: Radius નું મૂલ્ય input કરો
        \item \textbf{પગલું 3}: $\pi \times r^2$ સૂત્ર વડે area ગણો
        \item \textbf{પગલું 4}: $2 \times \pi \times r$ સૂત્ર વડે perimeter ગણો
    \end{itemize}

    \begin{mnemonicbox}"શરૂ Input ગણતરી દર્શાવો અંત"\end{mnemonicbox}
\end{solutionbox}

\questionmarks{2}{a}{3}
\textbf{ઓપરેટર શું છે? C ના બધા operators ની યાદી બનાવો.}

\begin{solutionbox}
    \textbf{જવાબ}:

    \textbf{ઓપરેટર વ્યાખ્યા:}

    \begin{tabulary}{\linewidth}{|L|L|}
        \hline
        \textbf{પાસાં} & \textbf{વર્ણન} \\
        \hline
        વ્યાખ્યા & Operands પર operations કરતા ખાસ પ્રતીકો \\
        \hline
        હેતુ & ડેટા અને વેરિએબલ્સ સાથે કામ કરવા \\
        \hline
    \end{tabulary}

    \textbf{C ઓપરેટર્સ યાદી:}

    \begin{tabulary}{\linewidth}{|L|L|}
        \hline
        \textbf{વર્ગ} & \textbf{Operators} \\
        \hline
        Arithmetic & +, -, *, /, \% \\
        \hline
        Relational & <, >, <=, >=, ==, != \\
        \hline
        Logical & \&\&, ||, ! \\
        \hline
        Assignment & =, +=, -=, *=, /= \\
        \hline
        Increment/Decrement & ++, -- \\
        \hline
        Conditional & ?: \\
        \hline
    \end{tabulary}

    \begin{mnemonicbox}"ગણતરી, સંબંધ, તર્ક, અસાઇન, વધારો, શરત"\end{mnemonicbox}
\end{solutionbox}

\questionmarks{2}{b}{4}
\textbf{while અને do while loop વચ્ચે તફાવત લખો.}

\begin{solutionbox}
    \textbf{જવાબ}:

    \begin{tabulary}{\linewidth}{|L|L|L|}
        \hline
        \textbf{પાસાં} & \textbf{while loop} & \textbf{do-while loop} \\
        \hline
        \textbf{Entry condition} & Pre-tested & Post-tested \\
        \hline
        \textbf{ન્યૂનતમ execution} & 0 વખત & ઓછામાં ઓછું 1 વખત \\
        \hline
        \textbf{Syntax} & \code{while(condition) \{ \}} & \code{do \{ \} while(condition);} \\
        \hline
        \textbf{Semicolon} & while પછી જરૂરી નથી & while પછી જરૂરી છે \\
        \hline
    \end{tabulary}

    \textbf{ઉદાહરણ:}

\begin{lstlisting}[language=C]
// while loop
while(i < 5) {
    printf("%d", i);
    i++;
}

// do-while loop  
do {
    printf("%d", i);
    i++;
} while(i < 5);
\end{lstlisting}

    \textbf{મુખ્ય મુદ્દાઓ:}

    \begin{itemize}
        \item \textbf{Pre-tested}: Execution પહેલાં condition ચકાસાય
        \item \textbf{Post-tested}: Execution પછી condition ચકાસાય
    \end{itemize}

    \begin{mnemonicbox}"While પહેલાં, Do પછી"\end{mnemonicbox}
\end{solutionbox}

\questionmarks{2}{c}{7}
\textbf{scanf() function નો formatted input માટે કેવી રીતે ઉપયોગ થાય છે? ઉદાહરણ સાથે સમજાવો}

\begin{solutionbox}
    \textbf{જવાબ}:

    \textbf{scanf() Function:}

    \begin{tabulary}{\linewidth}{|L|L|}
        \hline
        \textbf{લક્ષણ} & \textbf{વર્ણન} \\
        \hline
        હેતુ & Keyboard થી formatted input વાંચવા \\
        \hline
        Syntax & \code{scanf("format\_string", \&variable);} \\
        \hline
        Return & સફળતાપૂર્વક વંચાયેલા inputs ની સંખ્યા \\
        \hline
    \end{tabulary}

    \textbf{Format Specifiers:}

    \begin{tabulary}{\linewidth}{|L|L|}
        \hline
        \textbf{Specifier} & \textbf{Data Type} \\
        \hline
        \%d & int \\
        \hline
        \%f & float \\
        \hline
        \%c & char \\
        \hline
        \%s & string \\
        \hline
    \end{tabulary}

    \textbf{ઉદાહરણો:}

\begin{lstlisting}[language=C]
int age;
float salary;
char grade;

scanf("%d", &age);           // Integer વાંચો
scanf("%f", &salary);        // Float વાંચો
scanf("%c", &grade);         // Character વાંચો
scanf("%d %f", &age, &salary); // બહુવિધ inputs
\end{lstlisting}

    \textbf{મહત્વના મુદ્દાઓ:}

    \begin{itemize}
        \item \textbf{Address operator (\&)}: Variables માટે જરૂરી
        \item \textbf{Format string}: Data types સાથે match થવું જોઈએ
        \item \textbf{Buffer issues}: જરૂર પડે તો \code{fflush(stdin)} વાપરો
    \end{itemize}

    \begin{mnemonicbox}"Address Format Match"\end{mnemonicbox}
\end{solutionbox}

\orquestionmarks{2}{a}{3}
\textbf{C ભાષાના arithmetic અને relational operators ની યાદી બનાવો}

\begin{solutionbox}
    \textbf{જવાબ}:

    \begin{tabulary}{\linewidth}{|L|L|L|}
        \hline
        \textbf{Operator Type} & \textbf{Operators} & \textbf{હેતુ} \\
        \hline
        \textbf{Arithmetic} & +, -, *, /, \% & ગાણિતિક operations \\
        \hline
        \textbf{Relational} & <, >, <=, >=, ==, != & Comparison operations \\
        \hline
    \end{tabulary}

    \textbf{ઉદાહરણો:}

\begin{lstlisting}[language=C]
// Arithmetic
int a = 10 + 5;    // Addition
int b = 10 % 3;    // Modulus (remainder)

// Relational
if(a > b)          // મોટું
if(a == b)         // બરાબર
\end{lstlisting}

    \begin{mnemonicbox}"ગણતરી સરખામણી"\end{mnemonicbox}
\end{solutionbox}

\questionmarks{2}{b}{4}
\textbf{else if ladder નો flow chart દોરો.}

\begin{solutionbox}
    \textbf{જવાબ}:

    \begin{center}
    \begin{tikzpicture}[gtu flow]
        \node[gtu start] (start) {શરૂઆત};
        \node[gtu decision, alias=cond1, below=of start] {શરત 1?};
        \node[gtu process, alias=stmt1, right=of cond1, xshift=2cm] {Statement 1};
        
        \node[gtu decision, alias=cond2, below=of cond1, yshift=-1cm] {શરત 2?};
        \node[gtu process, alias=stmt2, right=of cond2, xshift=2cm] {Statement 2};
        
        \node[gtu decision, alias=cond3, below=of cond2, yshift=-1cm] {શરત 3?};
        \node[gtu process, alias=stmt3, right=of cond3, xshift=2cm] {Statement 3};
        
        \node[gtu process, alias=else, below=of cond3] {Else Statement};
        \node[gtu stop, below=of else] (end) {અંત};

        \draw[gtu arrow] (start) -- (cond1);
        \draw[gtu arrow] (cond1) -- node[above] {સાચું} (stmt1);
        \draw[gtu arrow] (cond1) -- node[right] {ખોટું} (cond2);
        
        \draw[gtu arrow] (cond2) -- node[above] {સાચું} (stmt2);
        \draw[gtu arrow] (cond2) -- node[right] {ખોટું} (cond3);
        
        \draw[gtu arrow] (cond3) -- node[above] {સાચું} (stmt3);
        \draw[gtu arrow] (cond3) -- node[right] {ખોટું} (else);
        
        % Connect all statements to End
        \draw[gtu arrow] (stmt1) -| (end);
        \draw[gtu arrow] (stmt2) -| (end);
        \draw[gtu arrow] (stmt3) -| (end);
        \draw[gtu arrow] (else) -- (end);
    \end{tikzpicture}
    \end{center}

    \textbf{માળખું:}

    \begin{itemize}
        \item \textbf{બહુવિધ શરતો}: ક્રમમાં ચકાસાય છે
        \item \textbf{પ્રથમ સાચી}: તેનો block execute થાય
        \item \textbf{Default case}: કોઈ match ન મળે તો else block
    \end{itemize}

    \begin{mnemonicbox}"પ્રથમ સાચી શોધો execute કરો"\end{mnemonicbox}
\end{solutionbox}

\questionmarks{2}{c}{7}
\textbf{printf() function નો formatted output માટે કેવી રીતે ઉપયોગ થાય છે? ઉદાહરણ સાથે સમજાવો}

\begin{solutionbox}
    \textbf{જવાબ}:

    \textbf{printf() Function:}

    \begin{tabulary}{\linewidth}{|L|L|}
        \hline
        \textbf{લક્ષણ} & \textbf{વર્ણન} \\
        \hline
        હેતુ & Screen પર formatted output દર્શાવવા \\
        \hline
        Syntax & \code{printf("format\_string", variables);} \\
        \hline
        Return & Print કરાયેલા characters ની સંખ્યા \\
        \hline
    \end{tabulary}

    \textbf{Format Specifiers:}

    \begin{tabulary}{\linewidth}{|L|L|L|}
        \hline
        \textbf{Specifier} & \textbf{વપરાશ} & \textbf{ઉદાહરણ} \\
        \hline
        \%d & Integer & \code{printf("\%d", 25);} \\
        \hline
        \%f & Float & \code{printf("\%.2f", 3.14);} \\
        \hline
        \%c & Character & \code{printf("\%c", 'A');} \\
        \hline
        \%s & String & \code{printf("\%s", "Hello");} \\
        \hline
    \end{tabulary}

    \textbf{Advanced Formatting:}

\begin{lstlisting}[language=C]
int num = 123;
float pi = 3.14159;

printf("Number: %5d\n", num);      // Width specification
printf("Pi: %.2f\n", pi);          // Precision specification
printf("Hex: %x\n", num);          // Hexadecimal
printf("Left aligned: %-10d\n", num); // Left alignment
\end{lstlisting}

    \textbf{Escape Sequences:}

    \begin{itemize}
        \item \textbf{\textbackslash n}: નવી લીટી
        \item \textbf{\textbackslash t}: Tab space
        \item \textbf{\textbackslash\textbackslash}: Backslash
    \end{itemize}

    \begin{mnemonicbox}"Format Width Precision Align"\end{mnemonicbox}
\end{solutionbox}

\questionmarks{3}{a}{3}
\textbf{Logical operators ની યાદી બનાવો અને તેને સમજાવો}

\begin{solutionbox}
    \textbf{જવાબ}:

    \begin{tabulary}{\linewidth}{|L|C|L|L|}
        \hline
        \textbf{Operator} & \textbf{Symbol} & \textbf{વર્ણન} & \textbf{Truth Table} \\
        \hline
        \textbf{AND} & \&\& & બંને operands સાચા હોય તો સાચું & T\&\&T = T, બાકી = F \\
        \hline
        \textbf{OR} & || & કોઈપણ operand સાચો હોય તો સાચું & F||F = F, બાકી = T \\
        \hline
        \textbf{NOT} & ! & Condition ઉલટાવે છે & !T = F, !F = T \\
        \hline
    \end{tabulary}

    \textbf{ઉદાહરણો:}

\begin{lstlisting}[language=C]
int a = 5, b = 10;

if(a > 0 && b > 0)    // બંને શરતો સાચી હોવી જોઈએ
if(a > 15 || b > 5)   // ઓછામાં ઓછી એક શરત સાચી
if(!(a > 10))         // શરતનું નકારણ
\end{lstlisting}

    \begin{mnemonicbox}"અને અથવા નહીં"\end{mnemonicbox}
\end{solutionbox}

\questionmarks{3}{b}{4}
\textbf{for loop ને ઉદાહરણ સાથે સમજાવો.}

\begin{solutionbox}
    \textbf{જવાબ}:

    \textbf{For Loop માળખું:}

    \begin{tabulary}{\linewidth}{|L|L|}
        \hline
        \textbf{ઘટક} & \textbf{હેતુ} \\
        \hline
        Initialization & શરુઆતી મૂલ્ય સેટ કરવું \\
        \hline
        Condition & ચાલુ રાખવા માટે ટેસ્ટ \\
        \hline
        Update & Loop variable ને બદલવું \\
        \hline
    \end{tabulary}

    \textbf{Syntax:}

\begin{lstlisting}[language=C]
for(initialization; condition; update) {
    statements;
}
\end{lstlisting}

    \textbf{ઉદાહરણ:}

\begin{lstlisting}[language=C]
// 1 થી 5 સુધીના નંબર print કરો
for(int i = 1; i <= 5; i++) {
    printf("%d ", i);
}
// Output: 1 2 3 4 5
\end{lstlisting}

    \textbf{Execution Flow:}

    \begin{itemize}
        \item \textbf{પગલું 1}: i = 1 initialize કરો
        \item \textbf{પગલું 2}: i <= 5 condition ચકાસો
        \item \textbf{પગલું 3}: Statements execute કરો
        \item \textbf{પગલું 4}: i++ update કરો, પગલું 2 પર પાછા
    \end{itemize}

    \begin{mnemonicbox}"Initialize ચકાસો Execute Update"\end{mnemonicbox}
\end{solutionbox}

\questionmarks{3}{c}{7}
\textbf{ત્રણ પૂર્ણાંક સંખ્યાઓ x અને y માંથી મહત્તમ શોધવા માટે પ્રોગ્રામ લખો.}

\begin{solutionbox}
    \textbf{જવાબ}:

\begin{lstlisting}[language=C]
#include <stdio.h>

int main() {
    int x, y, z, max;
    
    printf("ત્રણ સંખ્યાઓ દાખલ કરો: ");
    scanf("%d %d %d", &x, &y, &z);
    
    max = x;  // પ્રથમ સંખ્યાને મહત્તમ માનો
    
    if(y > max) {
        max = y;
    }
    if(z > max) {
        max = z;
    }
    
    printf("મહત્તમ સંખ્યા છે: %d", max);
    
    return 0;
}
\end{lstlisting}

    \textbf{Algorithm ના પગલાં:}

    \begin{tabulary}{\linewidth}{|C|L|}
        \hline
        \textbf{પગલું} & \textbf{કાર્ય} \\
        \hline
        1 & ત્રણ સંખ્યાઓ input કરો \\
        \hline
        2 & પ્રથમને મહત્તમ માનો \\
        \hline
        3 & બીજી સાથે સરખામણી, મોટી હોય તો update \\
        \hline
        4 & ત્રીજી સાથે સરખામણી, મોટી હોય તો update \\
        \hline
        5 & મહત્તમ દર્શાવો \\
        \hline
    \end{tabulary}

    \textbf{વૈકલ્પિક પદ્ધતિ:}

\begin{lstlisting}[language=C]
max = (x > y) ? ((x > z) ? x : z) : ((y > z) ? y : z);
\end{lstlisting}

    \begin{mnemonicbox}"માનો સરખાવો Update દર્શાવો"\end{mnemonicbox}
\end{solutionbox}

\orquestionmarks{3}{a}{3}
\textbf{conditional operator ને ઉદાહરણ સાથે સમજાવો.}

\begin{solutionbox}
    \textbf{જવાબ}:

    \textbf{Conditional Operator (Ternary):}

    \begin{tabulary}{\linewidth}{|L|L|}
        \hline
        \textbf{લક્ષણ} & \textbf{વર્ણન} \\
        \hline
        Symbol & ?: \\
        \hline
        Syntax & \code{condition ? value1 : value2} \\
        \hline
        હેતુ & if-else નો ટૂંકો રસ્તો \\
        \hline
    \end{tabulary}

    \textbf{ઉદાહરણો:}

\begin{lstlisting}[language=C]
int a = 10, b = 20;
int max = (a > b) ? a : b;        // max = 20

char grade = (marks >= 60) ? 'P' : 'F';
printf("Status: %s", (age >= 18) ? "Adult" : "Minor");
\end{lstlisting}

    \textbf{સમાન if-else:}

\begin{lstlisting}[language=C]
if(a > b)
    max = a;
else
    max = b;
\end{lstlisting}

    \textbf{ફાયદાઓ:}

    \begin{itemize}
        \item \textbf{સંક્ષિપ્ત}: એક લીટીમાં expression
        \item \textbf{કાર્યક્ષમ}: ઝડપી execution
    \end{itemize}

    \begin{mnemonicbox}"પ્રશ્નચિહ્ન કોલન પસંદગી"\end{mnemonicbox}
\end{solutionbox}

\questionmarks{3}{b}{4}
\textbf{while loop ને ઉદાહરણ સાથે સમજાવો.}

\begin{solutionbox}
    \textbf{જવાબ}:

    \textbf{While Loop:}

    \begin{tabulary}{\linewidth}{|L|L|}
        \hline
        \textbf{લક્ષણ} & \textbf{વર્ણન} \\
        \hline
        પ્રકાર & Entry-controlled loop \\
        \hline
        Syntax & \code{while(condition) \{ statements; \}} \\
        \hline
        Execution & શરત સાચી હોય ત્યાં સુધી repeat \\
        \hline
    \end{tabulary}

    \textbf{ઉદાહરણ:}

\begin{lstlisting}[language=C]
int i = 1;
while(i <= 5) {
    printf("%d ", i);
    i++;
}
// Output: 1 2 3 4 5
\end{lstlisting}

    \textbf{મહત્વના મુદ્દાઓ:}

    \begin{itemize}
        \item \textbf{Initialization}: Loop પહેલાં
        \item \textbf{Condition}: શરૂઆતમાં ચકાસાય
        \item \textbf{Update}: Loop body અંદર
        \item \textbf{Infinite loop}: જો condition ક્યારેય false ન બને
    \end{itemize}

    \textbf{Flowchart માળખું:}

    \begin{center}
    \begin{tikzpicture}[gtu flow]
        \node[gtu start] (start) {Initialize};
        \node[gtu decision, alias=cond, below=of start] {Condition?};
        \node[gtu process, below=of cond] (body) {Statements Execute કરો};
        \node[gtu process, below=of body] (update) {Variable Update કરો};
        \node[gtu stop, right=of cond, xshift=2cm] (end) {Loop બહાર નીકળો};

        \draw[gtu arrow] (start) -- (cond);
        \draw[gtu arrow] (cond) -- node[left] {સાચું} (body);
        \draw[gtu arrow] (body) -- (update);
        \draw[gtu arrow] (update.west) -- ++(-1,0) |- (cond.west);
        \draw[gtu arrow] (cond) -- node[above] {ખોટું} (end);
    \end{tikzpicture}
    \end{center}

    \begin{mnemonicbox}"Initialize ચકાસો Execute Update"\end{mnemonicbox}
\end{solutionbox}

\questionmarks{3}{c}{7}
\textbf{કીબોર્ડમાંથી પૂર્ણાંક વાંચવા માટે અને આપેલ સંખ્યા odd હોય કે even હોય તે શોધવા માટેનો પ્રોગ્રામ લખો.}

\begin{solutionbox}
    \textbf{જવાબ}:

\begin{lstlisting}[language=C]
#include <stdio.h>

int main() {
    int number;
    
    printf("એક પૂર્ણાંક દાખલ કરો: ");
    scanf("%d", &number);
    
    if(number % 2 == 0) {
        printf("%d એ સમ સંખ્યા છે", number);
    }
    else {
        printf("%d એ વિષમ સંખ્યા છે", number);
    }
    
    return 0;
}
\end{lstlisting}

    \textbf{તર્ક સમજૂતી:}

    \begin{tabulary}{\linewidth}{|L|L|}
        \hline
        \textbf{ખ્યાલ} & \textbf{વર્ણન} \\
        \hline
        \textbf{Modulus operator (\%)} & ભાગાકાર પછી બાકી આપે છે \\
        \hline
        \textbf{સમ શરત} & \code{number \% 2 == 0} \\
        \hline
        \textbf{વિષમ શરત} & \code{number \% 2 != 0} \\
        \hline
    \end{tabulary}

    \textbf{વૈકલ્પિક પદ્ધતિઓ:}

\begin{lstlisting}[language=C]
// પદ્ધતિ 2: Conditional operator વાપરીને
printf("%d એ %s છે", number, (number % 2 == 0) ? "સમ" : "વિષમ");

// પદ્ધતિ 3: Bitwise AND વાપરીને
if(number & 1)
    printf("વિષમ");
else
    printf("સમ");
\end{lstlisting}

    \textbf{Sample Output:}

\begin{lstlisting}
એક પૂર્ણાંક દાખલ કરો: 7
7 એ વિષમ સંખ્યા છે
\end{lstlisting}

    \begin{mnemonicbox}"Modulus બે શૂન્ય સમ"\end{mnemonicbox}
\end{solutionbox}

\questionmarks{4}{a}{3}
\textbf{નીચેના arithmetic expressions નું મૂલ્યાંકન કરો: 30/4*4 – 20\%6 + 17/2}

\begin{solutionbox}
    \textbf{જવાબ}:

    \textbf{પગલાબદ્ધ મૂલ્યાંકન:}

    \begin{tabulary}{\linewidth}{|C|L|L|C|}
        \hline
        \textbf{પગલું} & \textbf{Expression} & \textbf{ગણતરી} & \textbf{પરિણામ} \\
        \hline
        1 & 30/4*4 & (30/4)*4 = 7*4 & 28 \\
        \hline
        2 & 20\%6 & 20 mod 6 & 2 \\
        \hline
        3 & 17/2 & Integer division & 8 \\
        \hline
        4 & અંતિમ & 28 - 2 + 8 & 34 \\
        \hline
    \end{tabulary}

    \textbf{Operator પ્રાધાન્યતા:}

    \begin{tabulary}{\linewidth}{|L|L|}
        \hline
        \textbf{પ્રાધાન્યતા} & \textbf{Operators} \\
        \hline
        ઉચ્ચ & *, /, \% (ડાબેથી જમણે) \\
        \hline
        નીચી & +, - (ડાબેથી જમણે) \\
        \hline
    \end{tabulary}

    \textbf{સંપૂર્ણ ગણતરી:}

\begin{lstlisting}
30/4*4 - 20%6 + 17/2
= 7*4 - 2 + 8      // પહેલાં division અને modulus
= 28 - 2 + 8       // Multiplication
= 26 + 8           // +,- માટે ડાબેથી જમણે
= 34               // અંતિમ જવાબ
\end{lstlisting}

    \begin{mnemonicbox}"ગુણા ભાગ પહેલાં બાદબાકી પછી"\end{mnemonicbox}
\end{solutionbox}

\questionmarks{4}{b}{4}
\textbf{5 પૂર્ણાંક સંખ્યાઓની array નો સરવાળો અને સરેરાશ શોધવા માટેનો પ્રોગ્રામ લખો.}

\begin{solutionbox}
    \textbf{જવાબ}:

\begin{lstlisting}[language=C]
#include <stdio.h>

int main() {
    int numbers[5];
    int sum = 0;
    float average;
    
    printf("5 પૂર્ણાંકો દાખલ કરો:\n");
    for(int i = 0; i < 5; i++) {
        scanf("%d", &numbers[i]);
        sum += numbers[i];
    }
    
    average = (float)sum / 5;
    
    printf("સરવાળો = %d\n", sum);
    printf("સરેરાશ = %.2f", average);
    
    return 0;
}
\end{lstlisting}

    \textbf{Algorithm:}

    \begin{enumerate}
        \item 5 integers ની array જાહેર કરો
        \item Sum ને 0 થી initialize કરો
        \item Loop વાપરીને 5 numbers input કરો
        \item દરેક number ને sum માં ઉમેરો
        \item Average = sum/5 ગણો
        \item પરિણામો દર્શાવો
    \end{enumerate}

    \textbf{મુખ્ય મુદ્દાઓ:}

    \begin{itemize}
        \item \textbf{Type casting}: \code{(float)sum} ચોક્કસ division માટે
        \item \textbf{Loop વપરાશ}: Repetitive input માટે કાર્યક્ષમ
    \end{itemize}

    \begin{mnemonicbox}"જાહેર Input ઉમેરો ગણો દર્શાવો"\end{mnemonicbox}
\end{solutionbox}

\questionmarks{4}{c}{7}
\textbf{Pointer વ્યાખ્યાયિત કરો. Pointers કેવી રીતે declared અને initialized કરવામાં આવે છે તે ઉદાહરણ સાથે સમજાવો.}

\begin{solutionbox}
    \textbf{જવાબ}:

    \textbf{Pointer વ્યાખ્યા:}

    \begin{tabulary}{\linewidth}{|L|L|}
        \hline
        \textbf{પાસાં} & \textbf{વર્ણન} \\
        \hline
        વ્યાખ્યા & બીજા variable નું memory address સ્ટોર કરતું variable \\
        \hline
        હેતુ & સીધી memory access અને dynamic memory allocation \\
        \hline
        Symbol & * (asterisk) declaration અને dereferencing માટે \\
        \hline
    \end{tabulary}

    \textbf{Declaration અને Initialization:}

\begin{lstlisting}[language=C]
// Declaration
int *ptr;           // Integer નો pointer
float *fptr;        // Float નો pointer
char *cptr;         // Character નો pointer

// Initialization
int num = 10;
int *ptr = &num;    // num ના address સાથે initialize

// વૈકલ્પિક
int *ptr;
ptr = &num;         // પછીથી address assign
\end{lstlisting}

    \textbf{ઉદાહરણ પ્રોગ્રામ:}

\begin{lstlisting}[language=C]
#include <stdio.h>

int main() {
    int num = 25;
    int *ptr = &num;
    
    printf("num નું મૂલ્ય: %d\n", num);
    printf("num નું address: %p\n", &num);
    printf("ptr નું મૂલ્ય: %p\n", ptr);
    printf("ptr દ્વારા pointed મૂલ્ય: %d\n", *ptr);
    
    return 0;
}
\end{lstlisting}

    \textbf{Memory Diagram:}

    \begin{center}
    \begin{tikzpicture}[gtu flow]
        \node[gtu block, minimum width=2.5cm] (num) {\textbf{num} \\ Value: 25 \\ Addr: 1000};
        \node[gtu block, minimum width=2.5cm, right=of num, xshift=2cm] (ptr) {\textbf{ptr} \\ Value: 1000 \\ Addr: 2000};
        
        \draw[gtu arrow] (ptr.west) -- node[above] {point કરે છે} (num.east);
    \end{tikzpicture}
    \end{center}

    \begin{mnemonicbox}"Address Star Dereference"\end{mnemonicbox}
\end{solutionbox}

\orquestionmarks{4}{a}{3}
\textbf{નીચેના arithmetic expressions નું મૂલ્યાંકન કરો: 50 / 3 \% 3 + 5 * 7}

\begin{solutionbox}
    \textbf{જવાબ}:

    \textbf{પગલાબદ્ધ મૂલ્યાંકન:}

    \begin{tabulary}{\linewidth}{|C|L|L|C|}
        \hline
        \textbf{પગલું} & \textbf{Expression} & \textbf{ગણતરી} & \textbf{પરિણામ} \\
        \hline
        1 & 50/3 & Integer division & 16 \\
        \hline
        2 & 16\%3 & 16 mod 3 & 1 \\
        \hline
        3 & 5*7 & Multiplication & 35 \\
        \hline
        4 & અંતિમ & 1 + 35 & 36 \\
        \hline
    \end{tabulary}

    \textbf{સંપૂર્ણ ગણતરી:}

\begin{lstlisting}
50 / 3 % 3 + 5 * 7
= 16 % 3 + 35      // પહેલાં division અને multiplication
= 1 + 35           // Modulus operation
= 36               // અંતિમ જવાબ
\end{lstlisting}

    \textbf{Operator પ્રાધાન્યતા લાગુ:}

    \begin{itemize}
        \item \textbf{ઉચ્ચ પ્રાધાન્યતા}: /, \%, * (ડાબેથી જમણે)
        \item \textbf{નીચી પ્રાધાન્યતા}: + (ડાબેથી જમણે)
    \end{itemize}

    \begin{mnemonicbox}"ભાગ Mod ગુણા ઉમેરો"\end{mnemonicbox}
\end{solutionbox}

\questionmarks{4}{b}{4}
\textbf{N પૂર્ણાંકોની array માં સૌથી મોટી સંખ્યા શોધવા માટેનો પ્રોગ્રામ લખો.}

\begin{solutionbox}
    \textbf{જવાબ}:

\begin{lstlisting}[language=C]
#include <stdio.h>

int main() {
    int n, i;
    int largest;
    
    printf("elements ની સંખ્યા દાખલ કરો: ");
    scanf("%d", &n);
    
    int arr[n];
    
    printf("%d સંખ્યાઓ દાખલ કરો:\n", n);
    for(i = 0; i < n; i++) {
        scanf("%d", &arr[i]);
    }
    
    largest = arr[0];  // પ્રથમ element ને largest માનો
    
    for(i = 1; i < n; i++) {
        if(arr[i] > largest) {
            largest = arr[i];
        }
    }
    
    printf("સૌથી મોટી સંખ્યા છે: %d", largest);
    
    return 0;
}
\end{lstlisting}

    \textbf{Algorithm:}

    \begin{enumerate}
        \item Array નું size input કરો
        \item Array elements input કરો
        \item પ્રથમ element ને largest માનો
        \item બાકીના elements સાથે સરખામણી કરો
        \item જો મોટું મળે તો largest update કરો
        \item પરિણામ દર્શાવો
    \end{enumerate}

    \begin{mnemonicbox}"Input માનો સરખાવો Update દર્શાવો"\end{mnemonicbox}
\end{solutionbox}

\questionmarks{4}{c}{7}
\textbf{Array વ્યાખ્યાયિત કરો. Array variable ની જરૂરિયાત સમજાવો. 1-D array ને ઉદાહરણ સાથે સમજાવો}

\begin{solutionbox}
    \textbf{જવાબ}:

    \textbf{Array વ્યાખ્યા:}

    \begin{tabulary}{\linewidth}{|L|L|}
        \hline
        \textbf{પાસાં} & \textbf{વર્ણન} \\
        \hline
        વ્યાખ્યા & સમાન data type ના elements નો સંગ્રહ \\
        \hline
        Storage & સતત memory locations માં \\
        \hline
        Access & Index/subscript વાપરીને \\
        \hline
    \end{tabulary}

    \textbf{Arrays ની જરૂરિયાત:}

    \begin{tabulary}{\linewidth}{|L|L|}
        \hline
        \textbf{સમસ્યા} & \textbf{Array સાથે ઉકેલ} \\
        \hline
        બહુવિધ values સ્ટોર કરવા & એક જ array variable \\
        \hline
        ઘણા variables ટાળવા & \code{arr[100]} બદલે a1, a2, ..., a100 \\
        \hline
        કાર્યક્ષમ processing & Loop-based operations \\
        \hline
        Memory organization & Contiguous allocation \\
        \hline
    \end{tabulary}

    \textbf{1-D Array Declaration:}

\begin{lstlisting}[language=C]
datatype arrayname[size];

// ઉદાહરણો
int marks[5];           // 5 integers ની Array
float prices[10];       // 10 floats ની Array
char name[20];         // 20 characters ની Array
\end{lstlisting}

    \textbf{Array Initialization:}

\begin{lstlisting}[language=C]
// પદ્ધતિ 1: Declaration વખતે
int numbers[5] = {10, 20, 30, 40, 50};

// પદ્ધતિ 2: વ્યક્તિગત assignment
int arr[3];
arr[0] = 5;
arr[1] = 15;
arr[2] = 25;
\end{lstlisting}

    \textbf{ઉદાહરણ પ્રોગ્રામ:}

\begin{lstlisting}[language=C]
#include <stdio.h>

int main() {
    int marks[5] = {85, 90, 78, 92, 88};
    int i, sum = 0;
    
    printf("વિદ્યાર્થીના ગુણ:\n");
    for(i = 0; i < 5; i++) {
        printf("વિષય %d: %d\n", i+1, marks[i]);
        sum += marks[i];
    }
    
    printf("કુલ ગુણ: %d", sum);
    return 0;
}
\end{lstlisting}

    \textbf{Memory Layout:}

    \begin{center}
    \begin{tikzpicture}[gtu flow]
        \foreach \x/\val/\addr in {0/85/1000, 1/90/1004, 2/78/1008, 3/92/1012, 4/88/1016} {
            \node[draw, rectangle, minimum size=1cm] (n\x) at (\x*1.5, 0) {\val};
            \node[below=0.1cm of n\x] {\footnotesize marks[\x]};
            \node[above=0.1cm of n\x] {\footnotesize \addr};
        }
    \end{tikzpicture}
    \end{center}

    \begin{mnemonicbox}"સમાન ડેટા સતત Index"\end{mnemonicbox}
\end{solutionbox}

\questionmarks{5}{a}{3}
\textbf{if … else statement નું ઉદાહરણ આપો.}

\begin{solutionbox}
    \textbf{જવાબ}:

\begin{lstlisting}[language=C]
#include <stdio.h>

int main() {
    int age;
    
    printf("તમારી ઉંમર દાખલ કરો: ");
    scanf("%d", &age);
    
    if(age >= 18) {
        printf("તમે મતદાન માટે લાયક છો");
    }
    else {
        printf("તમે મતદાન માટે લાયક નથી");
    }
    
    return 0;
}
\end{lstlisting}

    \textbf{માળખું:}

    \begin{tabulary}{\linewidth}{|L|L|}
        \hline
        \textbf{ઘટક} & \textbf{હેતુ} \\
        \hline
        \textbf{if} & શરત ટેસ્ટ કરે છે \\
        \hline
        \textbf{condition} & Boolean expression \\
        \hline
        \textbf{if-block} & શરત સાચી હોય ત્યારે execute \\
        \hline
        \textbf{else-block} & શરત ખોટી હોય ત્યારે execute \\
        \hline
    \end{tabulary}

    \textbf{Sample Outputs:}

\begin{lstlisting}
Input: 20    Output: તમે મતદાન માટે લાયક છો
Input: 16    Output: તમે મતદાન માટે લાયક નથી
\end{lstlisting}

    \begin{mnemonicbox}"જો સાચું નહીંતર ખોટું"\end{mnemonicbox}
\end{solutionbox}

\questionmarks{5}{b}{4}
\textbf{આપેલ character ની category ચકાસવા માટેનો પ્રોગ્રામ લખો.}

\begin{solutionbox}
    \textbf{જવાબ}:

\begin{lstlisting}[language=C]
#include <stdio.h>
#include <ctype.h>

int main() {
    char ch;
    
    printf("એક character દાખલ કરો: ");
    scanf("%c", &ch);
    
    if(isdigit(ch)) {
        printf("'%c' એ અંક છે", ch);
    }
    else if(isupper(ch)) {
        printf("'%c' એ મોટા અક્ષર છે", ch);
    }
    else if(islower(ch)) {
        printf("'%c' એ નાના અક્ષર છે", ch);
    }
    else {
        printf("'%c' એ વિશેષ પ્રતીક છે", ch);
    }
    
    return 0;
}
\end{lstlisting}

    \textbf{Character Categories:}

    \begin{tabulary}{\linewidth}{|L|L|L|}
        \hline
        \textbf{Function} & \textbf{વર્ગ} & \textbf{Range} \\
        \hline
        isdigit() & અંક & 0-9 \\
        \hline
        isupper() & મોટા અક્ષર & A-Z \\
        \hline
        islower() & નાના અક્ષર & a-z \\
        \hline
        Others & વિશેષ પ્રતીકો & !@\#\$\%\^{}\&* etc. \\
        \hline
    \end{tabulary}

    \textbf{વૈકલ્પિક પદ્ધતિ:}

\begin{lstlisting}[language=C]
if(ch >= '0' && ch <= '9')
    printf("અંક");
else if(ch >= 'A' && ch <= 'Z')
    printf("મોટા અક્ષર");
else if(ch >= 'a' && ch <= 'z')
    printf("નાના અક્ષર");
else
    printf("વિશેષ પ્રતીક");
\end{lstlisting}

    \begin{mnemonicbox}"અંક મોટા નાના વિશેષ"\end{mnemonicbox}
\end{solutionbox}

\questionmarks{5}{c}{7}
\textbf{Structure શું છે? તેની syntax યોગ્ય ઉદાહરણ સાથે સમજાવો}

\begin{solutionbox}
    \textbf{જવાબ}:

    \textbf{Structure વ્યાખ્યા:}

    \begin{tabulary}{\linewidth}{|L|L|}
        \hline
        \textbf{પાસાં} & \textbf{વર્ણન} \\
        \hline
        વ્યાખ્યા & વિવિધ data types ને જોડીને બનાવેલ user-defined data type \\
        \hline
        હેતુ & સંબંધિત data ને એક જ નામ હેઠળ જૂથ બનાવવા \\
        \hline
        Keyword & \code{struct} \\
        \hline
    \end{tabulary}

    \textbf{Syntax:}

\begin{lstlisting}[language=C]
struct structure_name {
    datatype member1;
    datatype member2;
    ...
};
\end{lstlisting}

    \textbf{ઉદાહરણ - Student Structure:}

\begin{lstlisting}[language=C]
#include <stdio.h>

struct Student {
    int roll_no;
    char name[50];
    float marks;
    char grade;
};

int main() {
    struct Student s1;
    
    // Data input
    printf("રોલ નંબર દાખલ કરો: ");
    scanf("%d", &s1.roll_no);
    
    printf("નામ દાખલ કરો: ");
    scanf("%s", s1.name);
    
    printf("ગુણ દાખલ કરો: ");
    scanf("%f", &s1.marks);
    
    printf("ગ્રેડ દાખલ કરો: ");
    scanf(" %c", &s1.grade);
    
    // Data display
    printf("\nવિદ્યાર્થીની વિગતો:\n");
    printf("રોલ નં: %d\n", s1.roll_no);
    printf("નામ: %s\n", s1.name);
    printf("ગુણ: %.2f\n", s1.marks);
    printf("ગ્રેડ: %c\n", s1.grade);
    
    return 0;
}
\end{lstlisting}

    \textbf{Structure લક્ષણો:}

    \begin{itemize}
        \item \textbf{Dot operator (.)}: Structure members ને access કરવા
        \item \textbf{Memory allocation}: કુલ size = બધા members નો સરવાળો
        \item \textbf{Initialization}: Declaration વખતે initialize કરી શકાય
    \end{itemize}

    \textbf{Structure Initialization:}

\begin{lstlisting}[language=C]
struct Student s1 = {101, "John", 85.5, 'A'};
\end{lstlisting}

    \textbf{Memory Layout:}

    \begin{center}
    \begin{tikzpicture}[gtu flow]
        \node[gtu block, minimum width=4cm] (struct) {
            \textbf{s1 Structure} \\
            \rule{3.8cm}{0.4pt} \\
            roll\_no (4 bytes) \\
            name (50 bytes) \\
            marks (4 bytes) \\
            grade (1 byte)
        };
    \end{tikzpicture}
    \end{center}

    \begin{mnemonicbox}"સંબંધિત ડેટાને એકસાથે જૂથ બનાવો"\end{mnemonicbox}
\end{solutionbox}

\orquestionmarks{5}{a}{3}
\textbf{-5 અને +5 વચ્ચેના બધા numbers print કરવા માટેનો પ્રોગ્રામ લખો.}

\begin{solutionbox}
    \textbf{જવાબ}:

\begin{lstlisting}[language=C]
#include <stdio.h>

int main() {
    int i;
    
    printf("-5 અને +5 વચ્ચેના નંબરો:\n");
    
    for(i = -5; i <= 5; i++) {
        printf("%d ", i);
    }
    
    return 0;
}
\end{lstlisting}

    \textbf{Output:}

\begin{lstlisting}
-5 અને +5 વચ્ચેના નંબરો:
-5 -4 -3 -2 -1 0 1 2 3 4 5
\end{lstlisting}

    \textbf{વૈકલ્પિક પદ્ધતિઓ:}

\begin{lstlisting}[language=C]
// પદ્ધતિ 2: while loop વાપરીને
int i = -5;
while(i <= 5) {
    printf("%d ", i);
    i++;
}

// પદ્ધતિ 3: બે અલગ loops
for(i = -5; i < 0; i++)
    printf("%d ", i);
printf("0 ");
for(i = 1; i <= 5; i++)
    printf("%d ", i);
\end{lstlisting}

    \begin{mnemonicbox}"નકારાત્મકથી શરૂ સકારાત્મકે અંત"\end{mnemonicbox}
\end{solutionbox}

\questionmarks{5}{b}{4}
\textbf{Quadratic equation ના roots શોધવા માટેનો પ્રોગ્રામ લખો.}

\begin{solutionbox}
    \textbf{જવાબ}:

\begin{lstlisting}[language=C]
#include <stdio.h>
#include <math.h>

int main() {
    float a, b, c;
    float discriminant, root1, root2;
    
    printf("coefficients (a, b, c) દાખલ કરો: ");
    scanf("%f %f %f", &a, &b, &c);
    
    discriminant = b*b - 4*a*c;
    
    if(discriminant > 0) {
        root1 = (-b + sqrt(discriminant)) / (2*a);
        root2 = (-b - sqrt(discriminant)) / (2*a);
        printf("Roots વાસ્તવિક અને અલગ છે\n");
        printf("Root1 = %.2f\n", root1);
        printf("Root2 = %.2f\n", root2);
    }
    else if(discriminant == 0) {
        root1 = -b / (2*a);
        printf("Roots વાસ્તવિક અને સમાન છે\n");
        printf("Root = %.2f\n", root1);
    }
    else {
        float realPart = -b / (2*a);
        float imagPart = sqrt(-discriminant) / (2*a);
        printf("Roots સંકુલ છે\n");
        printf("Root1 = %.2f + %.2fi\n", realPart, imagPart);
        printf("Root2 = %.2f - %.2fi\n", realPart, imagPart);
    }
    
    return 0;
}
\end{lstlisting}

    \textbf{Quadratic Formula વિશ્લેષણ:}

    \begin{tabulary}{\linewidth}{|L|L|}
        \hline
        \textbf{Discriminant} & \textbf{Roots નો પ્રકાર} \\
        \hline
        $b^2-4ac > 0$ & વાસ્તવિક અને અલગ \\
        \hline
        $b^2-4ac = 0$ & વાસ્તવિક અને સમાન \\
        \hline
        $b^2-4ac < 0$ & સંકુલ (કાલ્પનિક) \\
        \hline
    \end{tabulary}

    \textbf{સૂત્ર:} $x = \frac{-b \pm \sqrt{b^2-4ac}}{2a}$

    \begin{mnemonicbox}"Discriminant Roots નો પ્રકાર નક્કી કરે છે"\end{mnemonicbox}
\end{solutionbox}

\questionmarks{5}{c}{7}
\textbf{નીચેના built-in functions ને ઉદાહરણો સાથે સમજાવો}

\begin{solutionbox}
    \textbf{જવાબ}:

    \textbf{Function સમજૂતીઓ:}

    \begin{tabulary}{\linewidth}{|L|L|L|L|}
        \hline
        \textbf{Function} & \textbf{હેતુ} & \textbf{Header File} & \textbf{ઉદાહરણ} \\
        \hline
        \code{clrscr()} & Screen સાફ કરવા & conio.h & \code{clrscr();} \\
        \hline
        \code{sqrt()} & વર્ગમૂળ & math.h & \code{sqrt(16) = 4.0} \\
        \hline
        \code{strlen()} & String ની લંબાઈ & string.h & \code{strlen("Hello") = 5} \\
        \hline
        \code{isdigit()} & અંક છે કે કેમ ચકાસવા & ctype.h & \code{isdigit('5') = true} \\
        \hline
        \code{isalpha()} & અક્ષર છે કે કેમ ચકાસવા & ctype.h & \code{isalpha('A') = true} \\
        \hline
        \code{toupper()} & મોટા અક્ષરમાં બદલવા & ctype.h & \code{toupper('a') = 'A'} \\
        \hline
        \code{tolower()} & નાના અક્ષરમાં બદલવા & ctype.h & \code{tolower('B') = 'b'} \\
        \hline
    \end{tabulary}

    \textbf{ઉદાહરણ પ્રોગ્રામ:}

\begin{lstlisting}[language=C]
#include <stdio.h>
#include <math.h>
#include <string.h>
#include <ctype.h>

int main() {
    // clrscr(); // Not standard in modern compilers
    
    // sqrt() ઉદાહરણ
    float num = 25.0;
    printf("%.1f નું વર્ગમૂળ = %.2f\n", num, sqrt(num));
    
    // strlen() ઉદાહરણ
    char str[] = "Programming";
    printf("'%s' ની લંબાઈ = %d\n", str, strlen(str));
    
    // Character functions
    char ch = 'a';
    printf("'%c' અંક છે: %s\n", ch, isdigit(ch) ? "હા" : "ના");
    printf("'%c' અક્ષર છે: %s\n", ch, isalpha(ch) ? "હા" : "ના");
    printf("toupper('%c') = '%c'\n", ch, toupper(ch));
    
    return 0;
}
\end{lstlisting}

    \textbf{Function વર્ગીકરણ:}

    \begin{center}
    \begin{tikzpicture}[gtu flow]
        \node[gtu block] (builtin) {Built-in Functions};
        \node[gtu block, below left=of builtin, xshift=-2cm] (screen) {Screen Control: clrscr()};
        \node[gtu block, below left=of builtin, xshift=1cm] (math) {Mathematical: sqrt()};
        \node[gtu block, below right=of builtin, xshift=-1cm] (string) {String: strlen()};
        \node[gtu block, below right=of builtin, xshift=2cm] (char) {Character: isdigit, isalpha, toupper, tolower};

        \draw[gtu arrow] (builtin) -- (screen);
        \draw[gtu arrow] (builtin) -- (math);
        \draw[gtu arrow] (builtin) -- (string);
        \draw[gtu arrow] (builtin) -- (char);
    \end{tikzpicture}
    \end{center}

    \begin{mnemonicbox}"સાફ ગણિત String Character"\end{mnemonicbox}
\end{solutionbox}

\end{document}
