\documentclass[10pt,a4paper]{article}

% content/resources/templates/preamble.tex
\usepackage[margin=0.6in]{geometry}
\author{Milav Dabgar}
\usepackage{amsmath,amssymb,amsthm}
\usepackage{booktabs}
\usepackage{multirow}
\usepackage{xcolor}
\usepackage{tcolorbox}
\tcbuselibrary{breakable,skins}
\usepackage[colorlinks=true,linkcolor=blue]{hyperref}
\usepackage{titlesec}
\usepackage{enumitem}
\usepackage{tikz}
\usepackage{pgfplots}
\usepackage{circuitikz}
\usepackage[version=4]{mhchem}
\usepackage{longtable}
\usepackage{array}
\usepackage{float}
\usepackage{caption}
\usepackage{listings}

\lstset{
  basicstyle=\small\ttfamily,
  breaklines=true,
  breakatwhitespace=false,
  postbreak=\mbox{\textcolor{red}{$\hookrightarrow$}\space},
  float=false,
  numbers=left,
  numberstyle=\tiny\color{gray},
  numbersep=10pt,
  xleftmargin=2em,
  keywordstyle=\color{blue},
  commentstyle=\color{green!60!black},
  stringstyle=\color{purple},
  backgroundcolor=\color{gray!5},
  showstringspaces=false,
  tabsize=2,
  captionpos=b,
  keepspaces=true,
  columns=flexible
}

\pgfplotsset{compat=1.18}
\usetikzlibrary{shapes,arrows,positioning,calc,patterns,decorations.pathmorphing,decorations.markings,arrows.meta}

% Color scheme
\definecolor{headcolor}{RGB}{0,102,204}
\definecolor{keycolor}{RGB}{220,20,60}
\definecolor{solutioncolor}{RGB}{34,139,34}
\definecolor{mnemoniccolor}{RGB}{148,0,211}
\definecolor{codecolor}{RGB}{0,0,100}

% Spacing
\setlength{\parskip}{3pt}
\setlist[itemize]{nosep}
\setlist[enumerate]{nosep}

% Title formatting
\titleformat{\section}{\Large\bfseries\color{headcolor}}{\thesection}{1em}{}
\titleformat{\subsection}{\large\bfseries\color{headcolor}}{\thesubsection}{1em}{}

% Pandoc tightlist compatibility
\providecommand{\tightlist}{%
  \setlength{\itemsep}{0pt}\setlength{\parskip}{0pt}}

% Pandoc longtable compatibility
\newcounter{none}
\def\thenone{}


% content/resources/templates/gujarati-boxes.tex
\usepackage{fontspec}
\usepackage{polyglossia}

% Set Gujarati as main language (document is primarily in Gujarati)
% Note: gloss-gujarati.ldf doesn't exist in polyglossia, but it will use hyphenation patterns
\setdefaultlanguage{gujarati}
\setotherlanguage{english}

% Configure Gujarati font properly
% Use Language=Default to prevent polyglossia from trying to add language-specific features
% that don't exist for Gujarati, which causes "empty feature" warnings
\newfontfamily\gujaratifont[Script=Gujarati,AutoFakeBold=2.5,AutoFakeSlant=0.3]{Noto Sans Gujarati}
\setmainfont[Script=Gujarati,AutoFakeBold=2.5,AutoFakeSlant=0.3]{Noto Sans Gujarati}
% Use Noto Sans Gujarati for monospace to support Gujarati in text
\setmonofont[Scale=0.9]{Noto Sans Gujarati}

% Configure English to use the same font
\newfontfamily\englishfont[Script=Gujarati,AutoFakeBold=2.5,AutoFakeSlant=0.3]{Noto Sans Gujarati}

% Translations for polyglossia
\gappto\captionsgujarati{
  \renewcommand{\tablename}{કોષ્ટક}
  \renewcommand{\figurename}{આકૃતિ}
}

% Helper for TikZ nodes to ensure Gujarati font
\newcommand{\gu}[1]{{\gujaratifont #1}}

% Custom environments
\newtcolorbox{solutionbox}{
    breakable,
    enhanced,
    colback=solutioncolor!5!white,
    colframe=solutioncolor!75!black,
    fonttitle=\bfseries,
    title=જવાબ
}

\newtcolorbox{solutionboxnobreak}{
 colback=solutioncolor!5!white,
 colframe=solutioncolor!75!black,
 fonttitle=\bfseries,
 title=જવાબ
}

\newtcolorbox{keyformula}{
 breakable,
 enhanced,
 colback=keycolor!5!white,
 colframe=keycolor!75!black,
 fonttitle=\bfseries,
 title=રાસાયણિક સમીકરણ/સૂત્ર
}

\newtcolorbox{mnemonicbox}{
 breakable,
 enhanced,
 colback=mnemoniccolor!5!white,
 colframe=mnemoniccolor!75!black,
 fonttitle=\bfseries,
 title=મેમરી ટ્રીક
}


\begin{document}

\begin{center}
{\Huge\bfseries\color{headcolor} Subject Name (Gujarati)}\\[5pt]
{\LARGE 4331105 -- Summer 2023}\\[3pt]
{\large Semester 1 Study Material}\\[3pt]
{\normalsize\textit{Detailed Solutions and Explanations}}
\end{center}

\vspace{10pt}

\subsection*{પ્રશ્ન 1(અ) [3
ગુણ]}\label{uxaaauxab0uxab6uxaa8-1uxa85-3-uxa97uxaa3}

\textbf{C લેંગ્વેજના કોઈ પણ છ કીવર્ડ લખો.}

\begin{solutionbox}


{\def\LTcaptype{none} % do not increment counter
\vspace{-5pt}
\captionof{table}{C લેંગ્વેજના છ કીવર્ડ}
\vspace{-10pt}
\begin{longtable}[]{@{}ll@{}}
\toprule\noalign{}
કીવર્ડ & ઉપયોગ \\
\midrule\noalign{}
\endhead
\bottomrule\noalign{}
\endlastfoot
int & પૂર્ણાંક ડેટા પ્રકાર \\
float & અપૂર્ણાંક ડેટા પ્રકાર \\
if & શરતી નિવેદન \\
while & લૂપ સ્ટ્રક્ચર \\
return & ફંક્શનમાંથી મૂલ્ય પાછું મેળવવા માટે \\
void & ખાલી રિટર્ન પ્રકાર દર્શાવવા \\
\end{longtable}
}

\end{solutionbox}
\begin{mnemonicbox}
``I Feel When Running Very Ill'' (int, float, while,
return, void, if)

\end{mnemonicbox}
\subsection*{પ્રશ્ન 1(બ) [4
ગુણ]}\label{uxaaauxab0uxab6uxaa8-1uxaac-4-uxa97uxaa3}

\textbf{વેરિયેબલની વ્યાખ્યા લખો. C પ્રોગ્રામિંગમાં વેરિયેબલના નામ માટેના નિયમો
લખો.}

\begin{solutionbox}

\textbf{વેરિયેબલ}: એક નામાંકિત મેમરી સ્થાન જેનો ઉપયોગ પ્રોગ્રામના અમલ દરમિયાન
સુધારી શકાય તેવા ડેટાને સંગ્રહિત કરવા માટે થાય છે.


{\def\LTcaptype{none} % do not increment counter
\vspace{-5pt}
\captionof{table}{C માં વેરિયેબલના નામકરણના નિયમો}
\vspace{-10pt}
\begin{longtable}[]{@{}ll@{}}
\toprule\noalign{}
નિયમ & ઉદાહરણ \\
\midrule\noalign{}
\endhead
\bottomrule\noalign{}
\endlastfoot
અક્ષર/અંડરસ્કોરથી શરૂ થવું જોઈએ & name, \_value \\
અક્ષરો, અંકો, અંડરસ્કોર સમાવી શકે & user\_1, count99 \\
ખાલી જગ્યા કે વિશેષ અક્ષરો ન હોવા જોઈએ & ✓: total\_sum, ✗: total-sum \\
કેસ સેન્સિટિવ છે & Name \neq name \\
રિઝર્વ કીવર્ડ્સનો ઉપયોગ ન કરી શકાય & ✗: int, while \\
મહત્તમ 31 અક્ષરો (સ્ટાન્ડર્ડ) & studentRegistrationNumber \\
\end{longtable}
}

\end{solutionbox}
\begin{mnemonicbox}
``Letters Lead, No Special Keys'' (અક્ષરથી શરૂ, વિશેષ
અક્ષરો નહીં, કીવર્ડ્સ નહીં)

\end{mnemonicbox}
\subsection*{પ્રશ્ન 1(ક) [7
ગુણ]}\label{uxaaauxab0uxab6uxaa8-1uxa95-7-uxa97uxaa3}

\textbf{ફ્લોચાર્ટની વ્યાખ્યા લખો. ફ્લોચાર્ટના સિમ્બોલ દોરો અને સમજાવો. નીચેના
સમીકરણનો ઉપયોગ કરીને સિમ્પલ ઇન્ટરેસ્ટની ગણતરી કરવા માટેનો પ્રોગ્રામ લખો.
I=PRN/100 જ્યાં

P=પ્રિન્સીપલ રકમ,

R= વ્યાજનો દર અને

N= સમયગાળો.}


\begin{solutionbox}

\textbf{ફ્લોચાર્ટ}: એક પ્રશ્નનો ઉકેલ કરવા માટે જરૂરી ક્રમિક ઓપરેશન્સને દર્શાવવા માટે
પ્રમાણભૂત પ્રતીકોનો ઉપયોગ કરીને અલ્ગોરિધમની ગ્રાફિકલ રજૂઆત.


{\def\LTcaptype{none} % do not increment counter
\vspace{-5pt}
\captionof{table}{ફ્લોચાર્ટ સિમ્બોલ}
\vspace{-10pt}
\begin{longtable}[]{@{}
  >{\raggedright\arraybackslash}p{(\linewidth - 4\tabcolsep) * \real{0.3913}}
  >{\raggedright\arraybackslash}p{(\linewidth - 4\tabcolsep) * \real{0.2174}}
  >{\raggedright\arraybackslash}p{(\linewidth - 4\tabcolsep) * \real{0.3913}}@{}}
\toprule\noalign{}
\begin{minipage}[b]{\linewidth}\raggedright
સિમ્બોલ
\end{minipage} & \begin{minipage}[b]{\linewidth}\raggedright
નામ
\end{minipage} & \begin{minipage}[b]{\linewidth}\raggedright
ઉપયોગ
\end{minipage} \\
\midrule\noalign{}
\endhead
\bottomrule\noalign{}
\endlastfoot
\pandocbounded{\includegraphics[keepaspectratio,alt={Oval}]{https://goat.liveuml.com/svg/S2w}}
& ટર્મિનલ & શરૂઆત/અંત \\
\pandocbounded{\includegraphics[keepaspectratio,alt={Rectangle}]{https://goat.liveuml.com/svg/JYuxD}}
& પ્રોસેસ & ગણતરી \\
\pandocbounded{\includegraphics[keepaspectratio,alt={Parallelogram}]{https://goat.liveuml.com/svg/JY1ZD}}
& ઈનપુટ/આઉટપુટ & ડેટા વાંચવો/દર્શાવવો \\
\pandocbounded{\includegraphics[keepaspectratio,alt={Diamond}]{https://goat.liveuml.com/svg/CYuxD}}
& નિર્ણય & શરતો \\
\pandocbounded{\includegraphics[keepaspectratio,alt={Arrow}]{https://goat.liveuml.com/svg/KYsxD}}
& ફ્લો લાઈન & ક્રમ બતાવે છે \\
\end{longtable}
}

\textbf{સિમ્પલ ઇન્ટરેસ્ટનું ફ્લોચાર્ટ:}

\begin{verbatim}
flowchart LR
    A([Start]) {-{-} B[/Input P, R, N/]}
    B {-{-} C[Calculate I = P*R*N/100]}
    C {-{-} D[/Display I/]}
    D {-{-} E([End])}
\end{verbatim}

\textbf{પ્રોગ્રામ:}

\begin{verbatim}
\#include {stdio.h}
void main()
\{
    float p, r, n, i;
    
    printf("Enter principal amount: ");
    scanf("\%f", \&p);
    
    printf("Enter rate of interest: ");
    scanf("\%f", \&r);
    
    printf("Enter time period in years: ");
    scanf("\%f", \&n);
    
    i = (p * r * n) / 100;
    
    printf("Simple Interest = \%.2f", i);
\}
\end{verbatim}

\end{solutionbox}
\begin{mnemonicbox}
``Please Return Nice Interest'' (Principal, Rate,
Number of years, Interest)

\end{mnemonicbox}
\subsection*{પ્રશ્ન 1(ક) OR [7
ગુણ]}\label{uxaaauxab0uxab6uxaa8-1uxa95-or-7-uxa97uxaa3}

\textbf{અલગોરિધમની વ્યાખ્યા લખો. સિલિન્ડરનું ઘનફળ શોધવા માટેનું અલગોરિધમ લખો.
યુઝર પાસેથી સિલિન્ડરની ત્રિજ્યા(R) અને ઊંચાઈ(H) ઈનપુટ લઇ સિલિન્ડરના વોલ્યુમ(V)ની
ગણતરી નીચેના સમીકરણનો ઉપયોગ કરીને પ્રિન્ટ કરવા માટેનો પ્રોગ્રામ લખો. V=πR^{2}H}

\begin{solutionbox}

\textbf{અલગોરિધમ}: મર્યાદિત સમયમાં કોઈ સમસ્યાનો ઉકેલ કરવા માટેની પગલાવાર
પ્રક્રિયા.

\textbf{સિલિન્ડરના ઘનફળ માટેનું અલગોરિધમ:}

\begin{enumerate}
\tightlist
\item
  શરૂ કરો
\item
  ત્રિજ્યા (R) અને ઊંચાઈ (H) ઇનપુટ લો
\item
  V = π \times R^{2} \times H સૂત્રનો ઉપયોગ કરીને ઘનફળની ગણતરી કરો
\item
  ઘનફળ પ્રદર્શિત કરો
\item
  સમાપ્ત
\end{enumerate}

\textbf{ડાયગ્રામ: સિલિન્ડર}

\begin{verbatim}
    +{-{-}{-}{-}{-}{-}+}
    |      |
    |      | H
    |      |
    +{-{-}{-}{-}{-}{-}+}
       R 
\end{verbatim}

\textbf{પ્રોગ્રામ:}

\begin{verbatim}
\#include {stdio.h}
void main()
\{
    float radius, height, volume;
    float pi = 3.14159;
    
    printf("Enter radius of cylinder: ");
    scanf("\%f", \&radius);
    
    printf("Enter height of cylinder: ");
    scanf("\%f", \&height);
    
    volume = pi * radius * radius * height;
    
    printf("Volume of cylinder = \%.2f", volume);
\}
\end{verbatim}

\end{solutionbox}
\begin{mnemonicbox}
``Round Hat Volume'' (Radius, Height, Volume)

\end{mnemonicbox}
\subsection*{પ્રશ્ન 2(અ) [3
ગુણ]}\label{uxaaauxab0uxab6uxaa8-2uxa85-3-uxa97uxaa3}

\textbf{C પ્રોગ્રામિંગ ભાષામાં સપોર્ટ કરતા વિવિધ ઓપરેટરોની યાદી બનાવો.}

\begin{solutionbox}


{\def\LTcaptype{none} % do not increment counter
\vspace{-5pt}
\captionof{table}{C પ્રોગ્રામિંગમાં ઓપરેટર્સ}
\vspace{-10pt}
\begin{longtable}[]{@{}lll@{}}
\toprule\noalign{}
ઓપરેટર પ્રકાર & ઉદાહરણો & ઉપયોગ \\
\midrule\noalign{}
\endhead
\bottomrule\noalign{}
\endlastfoot
એરિથમેટિક & +, -, *, /, \% & ગાણિતિક ઓપરેશન્સ \\
રિલેશનલ & \textless, \textgreater, ==, !=, \textless=, \textgreater= &
મૂલ્યોની સરખામણી \\
લોજીકલ & \&\&, \textbar\textbar, ! & શરતોને જોડવા \\
એસાઇનમેન્ટ & =, +=, -=, *=, /= & મૂલ્યો આપવા \\
ઇનક્રિમેન્ટ/ડિક્રિમેન્ટ & ++, -- & 1 વધારવું/ઘટાડવું \\
બિટવાઇઝ & \&, \textbar, \^{}, \textasciitilde, \textless\textless,
\textgreater\textgreater{} & બિટ મેનિપ્યુલેશન \\
કન્ડિશનલ & ?: & ટૂંકા if-else \\
\end{longtable}
}

\end{solutionbox}
\begin{mnemonicbox}
``All Relationships Lead Ancestors Incrementally
Beyond Conditions'' (દરેક પ્રકારનો પ્રથમ અક્ષર)

\end{mnemonicbox}
\subsection*{પ્રશ્ન 2(બ) [4
ગુણ]}\label{uxaaauxab0uxab6uxaa8-2uxaac-4-uxa97uxaa3}

\textbf{1 થી 50 નો સરવાળો અને સરેરાશ પ્રિન્ટ કરવા માટેનો પ્રોગ્રામ લખો.}

\begin{solutionbox}

\textbf{પ્રોગ્રામ:}

\begin{verbatim}
\#include {stdio.h}
void main()
\{
    int i, sum = 0;
    float avg;
    
    for(i = 1; i {=} 50; i++)
    \{
        sum = sum + i;
    \}
    
    avg = (float)sum / 50;
    
    printf("Sum of numbers from 1 to 50 = \%d{n}", sum);
    printf("Average of numbers from 1 to 50 = \%.2f", avg);
\}
\end{verbatim}

\textbf{પ્રક્રિયા ડાયગ્રામ:}

\begin{verbatim}
flowchart LR
    A([Start]) {-{-} B[Set sum = 0]}
    B {-{-} C[Loop i from 1 to 50]}
    C {-{-} D[Add i to sum]}
    D {-{-} E\{i  50?\}}
    E {-{-}|Yes| C}
    E {-{-}|No| F[Calculate avg = sum/50]}
    F {-{-} G[/Display sum and avg/]}
    G {-{-} H([End])}
\end{verbatim}

\end{solutionbox}
\begin{mnemonicbox}
``Summing And Dividing'' (Sum, Average, Division)

\end{mnemonicbox}
\subsection*{પ્રશ્ન 2(ક) [7
ગુણ]}\label{uxaaauxab0uxab6uxaa8-2uxa95-7-uxa97uxaa3}

\textbf{એરીથમેટીક અને રિલેશનલ ઓપરેટર ઉદાહરણ સાથે સમજાવો.}

\begin{solutionbox}

\textbf{એરિથમેટિક ઓપરેટર્સ:}


{\def\LTcaptype{none} % do not increment counter
\vspace{-5pt}
\captionof{table}{C માં એરિથમેટિક ઓપરેટર}
\vspace{-10pt}
\begin{longtable}[]{@{}llll@{}}
\toprule\noalign{}
ઓપરેટર & ઓપરેશન & ઉદાહરણ & પરિણામ \\
\midrule\noalign{}
\endhead
\bottomrule\noalign{}
\endlastfoot
+ & સરવાળો & 5 + 3 & 8 \\
- & બાદબાકી & 7 - 2 & 5 \\
* & ગુણાકાર & 4 * 3 & 12 \\
/ & ભાગાકાર & 8 / 4 & 2 \\
\% & મોડ્યુલસ (બાકી) & 7 \% 3 & 1 \\
\end{longtable}
}

\textbf{રિલેશનલ ઓપરેટર્સ:}


{\def\LTcaptype{none} % do not increment counter
\vspace{-5pt}
\captionof{table}{C માં રિલેશનલ ઓપરેટર}
\vspace{-10pt}
\begin{longtable}[]{@{}llll@{}}
\toprule\noalign{}
ઓપરેટર & અર્થ & ઉદાહરણ & પરિણામ \\
\midrule\noalign{}
\endhead
\bottomrule\noalign{}
\endlastfoot
\textless{} & કરતાં ઓછું & 5 \textless{} 8 & 1 (સાચું) \\
\textgreater{} & કરતાં વધુ & 9 \textgreater{} 3 & 1 (સાચું) \\
== & બરાબર & 4 == 4 & 1 (સાચું) \\
!= & અસમાન & 7 != 3 & 1 (સાચું) \\
\textless= & કરતાં ઓછું અથવા બરાબર & 4 \textless= 4 & 1 (સાચું) \\
\textgreater= & કરતાં વધુ અથવા બરાબર & 6 \textgreater= 9 & 0 (ખોટું) \\
\end{longtable}
}

\textbf{કોડ ઉદાહરણ:}

\begin{verbatim}
\#include {stdio.h}
void main()
\{
int

a = 10,

b = 5;

    
    // એરિથમેટિક ઓપરેટર્સ
    printf("a + b = \%d{n}", a + b);   // 15
    printf("a {- b = }\%d{n}", a {-} b);   // 5
    printf("a * b = \%d{n}", a * b);   // 50
    printf("a / b = \%d{n}", a / b);   // 2
    printf("a \%\% b = \%d{n}", a \% b);  // 0
    
    // રિલેશનલ ઓપરેટર્સ
    printf("a { b: }\%d{n}", a {} b);    // 0 (ખોટું)
    printf("a { b: }\%d{n}", a {} b);    // 1 (સાચું)
printf("a == b: \%d{n}",

a == b);  // 0 (ખોટું)

    printf("a != b: \%d{n}", a != b);  // 1 (સાચું)
\}
\end{verbatim}

\end{solutionbox}
\begin{mnemonicbox}
``Add Subtract Multiply Divide Remainder''
(એરિથમેટિક), ``Less Greater Equal Not'' (રિલેશનલ)

\end{mnemonicbox}
\subsection*{પ્રશ્ન 2(અ) OR [3
ગુણ]}\label{uxaaauxab0uxab6uxaa8-2uxa85-or-3-uxa97uxaa3}

\textbf{gets(S) અને scanf(``\%s'',S) ફંક્શન વચ્ચેનો તફાવત લખો જ્યાં S સ્ટ્રીંગ છે.}

\begin{solutionbox}


{\def\LTcaptype{none} % do not increment counter
\vspace{-5pt}
\captionof{table}{gets(S) અને scanf(``\%s'',S) વચ્ચેનો તફાવત}
\vspace{-10pt}
\begin{longtable}[]{@{}
  >{\raggedright\arraybackslash}p{(\linewidth - 4\tabcolsep) * \real{0.2727}}
  >{\raggedright\arraybackslash}p{(\linewidth - 4\tabcolsep) * \real{0.2727}}
  >{\raggedright\arraybackslash}p{(\linewidth - 4\tabcolsep) * \real{0.4545}}@{}}
\toprule\noalign{}
\begin{minipage}[b]{\linewidth}\raggedright
લક્ષણ
\end{minipage} & \begin{minipage}[b]{\linewidth}\raggedright
gets(S)
\end{minipage} & \begin{minipage}[b]{\linewidth}\raggedright
scanf(``\%s'',S)
\end{minipage} \\
\midrule\noalign{}
\endhead
\bottomrule\noalign{}
\endlastfoot
સ્પેસ હેન્ડલિંગ & શબ્દો વચ્ચે સ્પેસ વાંચે છે & સ્પેસ પર વાંચવાનું બંધ કરે છે \\
ઇનપુટ સમાપ્તિ & ન્યૂલાઇન પર સમાપ્ત થાય છે & વ્હાઇટસ્પેસ પર સમાપ્ત થાય છે \\
બફર ઓવરફ્લો & અસુરક્ષિત, લંબાઈ ચકાસણી નથી & વિડ્થ લિમિટ સાથે સુરક્ષિત \\
ઉદાહરણ વર્તન & ``Hello World'' \rightarrow ``Hello World'' & ``Hello World'' \rightarrow
``Hello'' \\
સુરક્ષા & ઓવરફ્લો જોખમને કારણે અવમૂલ્યિત & વિડ્થ સ્પેસિફાયર સાથે વધુ સારું \\
\end{longtable}
}

\end{solutionbox}
\begin{mnemonicbox}
``Gets Spaces, Scanf Stops'' (gets સ્પેસ વાંચે છે, scanf
સ્પેસ પર અટકે છે)

\end{mnemonicbox}
\subsection*{પ્રશ્ન 2(બ) OR [4
ગુણ]}\label{uxaaauxab0uxab6uxaa8-2uxaac-or-4-uxa97uxaa3}

\textbf{બે સંખ્યાની અદલાબદલી કરવાનો પ્રોગ્રામ લખો.}

\begin{solutionbox}

\textbf{પ્રોગ્રામ:}

\begin{verbatim}
\#include {stdio.h}
void main()
\{
    int a, b, temp;
    
    printf("Enter value of a: ");
    scanf("\%d", \&a);
    
    printf("Enter value of b: ");
    scanf("\%d", \&b);
    
printf("Before swapping:

a = \%d,

b = \%d{n}", a, b);

    
    // ટેમ્પ વેરિયેબલનો ઉપયોગ કરીને અદલાબદલી
    temp = a;
    a = b;
    b = temp;
    
printf("After swapping:

a = \%d,

b = \%d", a, b);

\}
\end{verbatim}

\textbf{અદલાબદલી ડાયગ્રામ:}

\begin{verbatim}
flowchart LR
    A["a = 5"] {-{-} |Step 1: temp = a| C["temp = 5"]}
B["b = 10"] {-{-} |Step 2:

a = b| A1["a = 10"]}

C {-{-} |Step 3:

b = temp| B1["b = 5"]}

\end{verbatim}

\end{solutionbox}
\begin{mnemonicbox}
``Temporary Assists Swapping'' (ટેમ્પ વેરિયેબલ અદલાબદલી
માટે મદદ કરે છે)

\end{mnemonicbox}
\subsection*{પ્રશ્ન 2(ક) OR [7
ગુણ]}\label{uxaaauxab0uxab6uxaa8-2uxa95-or-7-uxa97uxaa3}

\textbf{લોજીકલ ઓપરેટર અને બીટ-વાઈસ ઓપરેટર ઉદાહરણ સાથે સમજાવો.}

\begin{solutionbox}

\textbf{લોજીકલ ઓપરેટર્સ:}


{\def\LTcaptype{none} % do not increment counter
\vspace{-5pt}
\captionof{table}{C માં લોજીકલ ઓપરેટર}
\vspace{-10pt}
\begin{longtable}[]{@{}llll@{}}
\toprule\noalign{}
ઓપરેટર & વર્ણન & ઉદાહરણ & પરિણામ \\
\midrule\noalign{}
\endhead
\bottomrule\noalign{}
\endlastfoot
\&\& & લોજીકલ AND & (5\textgreater3) \&\& (8\textgreater6) & 1 (બંને
સાચાં) \\
\textbar\textbar{} & લોજીકલ OR & (5\textless3) \textbar\textbar{}
(8\textgreater6) & 1 (એક સાચું) \\
! & લોજીકલ NOT & !(5\textgreater3) & 0 (સાચાને ખોટામાં ફેરવે) \\
\end{longtable}
}

\textbf{બિટવાઇઝ ઓપરેટર્સ:}


{\def\LTcaptype{none} % do not increment counter
\vspace{-5pt}
\captionof{table}{C માં બિટવાઇઝ ઓપરેટર}
\vspace{-10pt}
\begin{longtable}[]{@{}llll@{}}
\toprule\noalign{}
ઓપરેટર & વિગત & ઉદાહરણ & બાઇનરી પરિણામ \\
\midrule\noalign{}
\endhead
\bottomrule\noalign{}
\endlastfoot
\& & બિટવાઇઝ AND & 5 \& 3 & 101 \& 011 = 001 (1) \\
\textbar{} & બિટવાઇઝ OR & 5 \textbar{} 3 & 101 \textbar{} 011 = 111
(7) \\
\^{} & બિટવાઇઝ XOR & 5 \^{} 3 & 101 \^{} 011 = 110 (6) \\
\textasciitilde{} & બિટવાઇઝ NOT & \textasciitilde5 & \textasciitilde0101
= 1010 (-6) \\
\textless\textless{} & લેફ્ટ શિફ્ટ & 5 \textless\textless{} 1 & 101
\textless\textless{} 1 = 1010 (10) \\
\textgreater\textgreater{} & રાઇટ શિફ્ટ & 5 \textgreater\textgreater{} 1
& 101 \textgreater\textgreater{} 1 = 10 (2) \\
\end{longtable}
}

\textbf{કોડ ઉદાહરણ:}

\begin{verbatim}
\#include {stdio.h}
void main()
\{
int

a = 5,

b = 3;

    
    // લોજીકલ ઓપરેટર્સ
    printf("a{3 \&\& b5: }\%d{n}", (a{}3) \&\& (b{}5));  // 1 (સાચું)
    printf("a{3 || b1: }\%d{n}", (a{}3) || (b{}1));  // 1 (સાચું)
    printf("!(a{b): }\%d{n}", !(a{}b));              // 0 (ખોટું)
    
    // બિટવાઇઝ ઓપરેટર્સ
    printf("a \& b: \%d{n}", a \& b);   // 1
    printf("a | b: \%d{n}", a | b);   // 7
    printf("a \^{ b: }\%d{n}", a \^{} b);   // 6
    printf("{a: }\%d{n}", {}a);         // {-6}
    printf("a { 1: }\%d{n}", a {} 1); // 10
    printf("a { 1: }\%d{n}", a {} 1); // 2
\}
\end{verbatim}

\end{solutionbox}
\begin{mnemonicbox}
``AND OR NOT'' (લોજીકલ ઓપરેટર્સ), ``AND OR XOR NOT
SHIFT'' (બિટવાઇઝ ઓપરેટર્સ)

\end{mnemonicbox}
\subsection*{પ્રશ્ન 3(અ) [3
ગુણ]}\label{uxaaauxab0uxab6uxaa8-3uxa85-3-uxa97uxaa3}

\textbf{ઉદાહરણ સાથે multiple if-else સ્ટેટમેન્ટ સમજાવો.}

\begin{solutionbox}

\textbf{Multiple if-else}: શરતોનો ક્રમ અનુસાર ચકાસણી થાય છે જ્યાં સૌથી પહેલી
સાચી શરત મળે ત્યાં સુધી.

\textbf{સ્ટ્રક્ચર:}

\begin{verbatim}
if (condition1)
    statement1;
else if (condition2)
    statement2;
else if (condition3)
    statement3;
else
    default\_statement;
\end{verbatim}

\textbf{કોડ ઉદાહરણ:}

\begin{verbatim}
\#include {stdio.h}
void main()
\{
    int marks;
    
    printf("Enter marks: ");
    scanf("\%d", \&marks);
    
    if (marks {=} 80)
        printf("Grade: A");
    else if (marks {=} 70)
        printf("Grade: B");
    else if (marks {=} 60)
        printf("Grade: C");
    else if (marks {=} 50)
        printf("Grade: D");
    else
        printf("Grade: F");
\}
\end{verbatim}

\textbf{ડાયગ્રામ:}

\begin{verbatim}
flowchart LR
    A[Start] {-{-} B\{marks = 80?\}}
    B {-{-}|Yes| C[Grade A]}
    B {-{-}|No| D\{marks = 70?\}}
    D {-{-}|Yes| E[Grade B]}
    D {-{-}|No| F\{marks = 60?\}}
    F {-{-}|Yes| G[Grade C]}
    F {-{-}|No| H\{marks = 50?\}}
    H {-{-}|Yes| I[Grade D]}
    H {-{-}|No| J[Grade F]}
\end{verbatim}

\end{solutionbox}
\begin{mnemonicbox}
``Check Each Condition in Sequence'' (CECS)

\end{mnemonicbox}
\subsection*{પ્રશ્ન 3(બ) [4
ગુણ]}\label{uxaaauxab0uxab6uxaa8-3uxaac-4-uxa97uxaa3}

\textbf{While લૂપ અને for લૂપનું વર્કિંગ જણાવો.}

\begin{solutionbox}


{\def\LTcaptype{none} % do not increment counter
\vspace{-5pt}
\captionof{table}{While લૂપ vs For લૂપ}
\vspace{-10pt}
\begin{longtable}[]{@{}
  >{\raggedright\arraybackslash}p{(\linewidth - 4\tabcolsep) * \real{0.2903}}
  >{\raggedright\arraybackslash}p{(\linewidth - 4\tabcolsep) * \real{0.3871}}
  >{\raggedright\arraybackslash}p{(\linewidth - 4\tabcolsep) * \real{0.3226}}@{}}
\toprule\noalign{}
\begin{minipage}[b]{\linewidth}\raggedright
લક્ષણ
\end{minipage} & \begin{minipage}[b]{\linewidth}\raggedright
While લૂપ
\end{minipage} & \begin{minipage}[b]{\linewidth}\raggedright
For લૂપ
\end{minipage} \\
\midrule\noalign{}
\endhead
\bottomrule\noalign{}
\endlastfoot
સિન્ટેક્સ & \texttt{while(condition)\ \{\ statements;\ \}} &
\texttt{for(init;\ condition;\ update)\ \{\ statements;\ \}} \\
ક્યારે વાપરવું & જ્યારે પુનરાવર્તનની સંખ્યા અજ્ઞાત હોય & જ્યારે પુનરાવર્તનની સંખ્યા
જાણીતી હોય \\
ઇનિશિયલાઇઝેશન & લૂપની બહાર & લૂપના ડિક્લેરેશનમાં \\
અપડેટ & લૂપ બોડીની અંદર કરવું જોઈએ & લૂપ ડિક્લેરેશનમાં આપોઆપ થાય છે \\
એક્ઝિટ કંટ્રોલ & માત્ર શરૂઆતમાં & માત્ર શરૂઆતમાં \\
ઉદાહરણ & યુઝર ઇનપુટ ચકાસવા & નિશ્ચિત વખત પુનરાવર્તન કરવા \\
\end{longtable}
}

\textbf{While લૂપ ફ્લો:}

\begin{verbatim}
flowchart LR
    A([Start]) {-{-} B[Initialize]}
    B {-{-} C\{Condition\}}
    C {-{-}|True| D[Body]}
    D {-{-} C}
    C {-{-}|False| E([End])}
\end{verbatim}

\textbf{For લૂપ ફ્લો:}

\begin{verbatim}
flowchart LR
    A([Start]) {-{-} B[Initialize]}
    B {-{-} C\{Condition\}}
    C {-{-}|True| D[Body]}
    D {-{-} E[Update]}
    E {-{-} C}
    C {-{-}|False| F([End])}
\end{verbatim}

\end{solutionbox}
\begin{mnemonicbox}
``While Checks Then Acts'' (WCTA), ``For Initializes
Tests Updates'' (FITU)

\end{mnemonicbox}
\subsection*{પ્રશ્ન 3(ક) [7
ગુણ]}\label{uxaaauxab0uxab6uxaa8-3uxa95-7-uxa97uxaa3}

\textbf{આપેલ સંખ્યાના ફેક્ટોરિયલ શોધવા માટેનો પ્રોગ્રામ લખો.}

\begin{solutionbox}

\textbf{પ્રોગ્રામ:}

\begin{verbatim}
\#include {stdio.h}
void main()
\{
    int num, i;
    unsigned long fact = 1;
    
    printf("Enter a number: ");
    scanf("\%d", \&num);
    
    if (num {} 0)
        printf("Factorial not defined for negative numbers");
    else
    \{
        for(i = 1; i {=} num; i++)
        \{
            fact = fact * i;
        \}
        printf("Factorial of \%d = \%lu", num, fact);
    \}
\}
\end{verbatim}

\textbf{ફેક્ટોરિયલ ગણતરી કોષ્ટક:} ઉદાહરણ તરીકે, જો num = 5:

{\def\LTcaptype{none} % do not increment counter
\begin{longtable}[]{@{}llll@{}}
\toprule\noalign{}
પુનરાવર્તન & i & fact = fact * i & નવી fact કિંમત \\
\midrule\noalign{}
\endhead
\bottomrule\noalign{}
\endlastfoot
પ્રારંભિક & - & - & 1 \\
1 & 1 & 1 * 1 & 1 \\
2 & 2 & 1 * 2 & 2 \\
3 & 3 & 2 * 3 & 6 \\
4 & 4 & 6 * 4 & 24 \\
5 & 5 & 24 * 5 & 120 \\
\end{longtable}
}

\textbf{ફેક્ટોરિયલ ગણતરી ડાયગ્રામ:}

\begin{verbatim}
flowchart LR
    A([Start]) {-{-} B[/Input num/]}
    B {-{-} C\{num  0?\}}
    C {-{-}|Yes| D[/Error message/]}
    C {-{-}|No| E[fact = 1]}
    E {-{-} F[Loop i from 1 to num]}
    F {-{-} G[fact = fact * i]}
    G {-{-} H\{i  num?\}}
    H {-{-}|Yes| F}
    H {-{-}|No| I[/Display fact/]}
    I {-{-} J([End])}
\end{verbatim}

\end{solutionbox}
\begin{mnemonicbox}
``Find And Count The Numbers!'' (FACTN! - Factorial)

\end{mnemonicbox}
\subsection*{પ્રશ્ન 3(અ) OR [3
ગુણ]}\label{uxaaauxab0uxab6uxaa8-3uxa85-or-3-uxa97uxaa3}

\textbf{ઉદાહરણ સાથે switch-case સ્ટેટમેન્ટની કામગીરી સમજાવો.}

\begin{solutionbox}

\textbf{Switch-Case}: એક પસંદગી નિવેદન જે મૂલ્યોની યાદી (કેસ) સામે વેરિયેબલની
સમાનતા ચકાસવાની મંજૂરી આપે છે.

\textbf{સ્ટ્રક્ચર:}

\begin{verbatim}
switch(expression) \{
    case value1:
        statements1;
        break;
    case value2:
        statements2;
        break;
    default:
        default\_statements;
\}
\end{verbatim}

\textbf{કોડ ઉદાહરણ:}

\begin{verbatim}
\#include {stdio.h}
void main()
\{
    int day;
    
    printf("Enter day number (1{-7): "});
    scanf("\%d", \&day);
    
    switch(day) \{
        case 1:
            printf("Monday");
            break;
        case 2:
            printf("Tuesday");
            break;
        case 3:
            printf("Wednesday");
            break;
        case 4:
            printf("Thursday");
            break;
        case 5:
            printf("Friday");
            break;
        case 6:
            printf("Saturday");
            break;
        case 7:
            printf("Sunday");
            break;
        default:
            printf("Invalid day");
    \}
\}
\end{verbatim}

\textbf{Switch-Case ડાયગ્રામ:}

\begin{verbatim}
flowchart TD
    A[Start] {-{-} B[/Input day/]}
    B {-{-} C\{Switch day\}}
    C {-{-}|case 1| D[Monday]}
    C {-{-}|case 2| E[Tuesday]}
    C {-{-}|case 3| F[Wednesday]}
    C {-{-}|case 4| G[Thursday]}
    C {-{-}|case 5| H[Friday]}
    C {-{-}|case 6| I[Saturday]}
    C {-{-}|case 7| J[Sunday]}
    C {-{-}|default| K[Invalid day]}
    D \& E \& F \& G \& H \& I \& J \& K {-{-} L[End]}
\end{verbatim}

\end{solutionbox}
\begin{mnemonicbox}
``Select Value, Exit with Break'' (SVEB)

\end{mnemonicbox}
\subsection*{પ્રશ્ન 3(બ) OR [4
ગુણ]}\label{uxaaauxab0uxab6uxaa8-3uxaac-or-4-uxa97uxaa3}

\textbf{break અને continue સ્ટેટમેન્ટ ઉપયોગ લખો.}

\begin{solutionbox}


{\def\LTcaptype{none} % do not increment counter
\vspace{-5pt}
\captionof{table}{Break vs Continue Keywords}
\vspace{-10pt}
\begin{longtable}[]{@{}
  >{\raggedright\arraybackslash}p{(\linewidth - 4\tabcolsep) * \real{0.3462}}
  >{\raggedright\arraybackslash}p{(\linewidth - 4\tabcolsep) * \real{0.2692}}
  >{\raggedright\arraybackslash}p{(\linewidth - 4\tabcolsep) * \real{0.3846}}@{}}
\toprule\noalign{}
\begin{minipage}[b]{\linewidth}\raggedright
લક્ષણ
\end{minipage} & \begin{minipage}[b]{\linewidth}\raggedright
break
\end{minipage} & \begin{minipage}[b]{\linewidth}\raggedright
continue
\end{minipage} \\
\midrule\noalign{}
\endhead
\bottomrule\noalign{}
\endlastfoot
ઉદ્દેશ & વર્તમાન લૂપ/સ્વિચમાંથી બહાર નીકળે છે & વર્તમાન પુનરાવર્તન છોડી, આગલા
પુનરાવર્તનમાં જાય છે \\
લૂપ પર અસર & લૂપને સમાપ્ત કરે છે & આગલા પુનરાવર્તનમાં આગળ વધે છે \\
ક્યાં વપરાય છે & લૂપ્સ \& સ્વિચ સ્ટેટમેન્ટ્સ & માત્ર લૂપ્સમાં \\
કંટ્રોલ ફ્લો & લૂપ પછીના સ્ટેટમેન્ટ પર જાય છે & લૂપની શરત ચકાસણી પર જાય છે \\
ઉપયોગનું ઉદાહરણ & શરત પૂરી થાય ત્યારે લૂપમાંથી નીકળવું & ચોક્કસ પુનરાવર્તનો છોડવા \\
\end{longtable}
}

\textbf{ફ્લો ડાયગ્રામ - break:}

\begin{verbatim}
flowchart LR
    A([Start]) {-{-} B[Loop]}
    B {-{-} C\{Condition\}}
    C {-{-}|True| D[break]}
    C {-{-}|False| E[Loop statements]}
    E {-{-} B}
    D {-{-} F[Statements after loop]}
    F {-{-} G([End])}
\end{verbatim}

\textbf{ફ્લો ડાયગ્રામ - continue:}

\begin{verbatim}
flowchart LR
    A([Start]) {-{-} B[Loop]}
    B {-{-} C\{Condition\}}
    C {-{-}|True| D[continue]}
    C {-{-}|False| E[Loop statements]}
    D {-{-} B}
    E {-{-} B}
    B {-{-}|Loop ends| F([End])}
\end{verbatim}

\end{solutionbox}
\begin{mnemonicbox}
``Break Exits, Continue Skips'' (BECS)

\end{mnemonicbox}
\subsection*{પ્રશ્ન 3(ક) OR [7
ગુણ]}\label{uxaaauxab0uxab6uxaa8-3uxa95-or-7-uxa97uxaa3}

\textbf{કીબોર્ડ પરથી લીટીઓની સંખ્યા (n) વાંચી અને નીચે દર્શાવેલ ત્રિકોણ પ્રિન્ટ
કરવા માટેનો પ્રોગ્રામ લખો. ઉદાહરણ તરીકે, n=5}

\begin{verbatim}
1 2 3 4 5
1 2 3 4
1 2 3
1 2
1
\end{verbatim}

\begin{solutionbox}

\textbf{પ્રોગ્રામ:}

\begin{verbatim}
\#include {stdio.h}
void main()
\{
    int n, i, j;
    
    printf("Enter number of lines: ");
    scanf("\%d", \&n);
    
    for(i = n; i {=} 1; i{-{-})}
    \{
        for(j = 1; j {=} i; j++)
        \{
            printf("\%d ", j);
        \}
        printf("{n}");
    \}
\}
\end{verbatim}

\textbf{પેટર્ન લોજિક કોષ્ટક:} n = 5 માટે:

{\def\LTcaptype{none} % do not increment counter
\begin{longtable}[]{@{}lll@{}}
\toprule\noalign{}
i & j & આઉટપુટ \\
\midrule\noalign{}
\endhead
\bottomrule\noalign{}
\endlastfoot
5 & j=1 થી 5 & 1 2 3 4 5 \\
4 & j=1 થી 4 & 1 2 3 4 \\
3 & j=1 થી 3 & 1 2 3 \\
2 & j=1 થી 2 & 1 2 \\
1 & j=1 થી 1 & 1 \\
\end{longtable}
}

\textbf{પેટર્ન વિઝ્યુલાઇઝેશન:}

\begin{verbatim}
1 2 3 4 5
1 2 3 4
1 2 3
1 2
1
\end{verbatim}

\textbf{પ્રોગ્રામ ફ્લો:}

\begin{verbatim}
flowchart LR
    A([Start]) {-{-} B[/Input n/]}
    B {-{-} C[outer loop: i = n to 1]}
    C {-{-} D[inner loop: j = 1 to i]}
    D {-{-} E[/Print j/]}
    E {-{-} F\{j  i?\}}
    F {-{-}|Yes| D}
    F {-{-}|No| G[/Print newline/]}
    G {-{-} H\{i  1?\}}
    H {-{-}|Yes| C}
    H {-{-}|No| I([End])}
\end{verbatim}

\end{solutionbox}
\begin{mnemonicbox}
``Decreasing Rows With Increasing Values'' (DRWIV)

\end{mnemonicbox}
\subsection*{પ્રશ્ન 4(અ) [3
ગુણ]}\label{uxaaauxab0uxab6uxaa8-4uxa85-3-uxa97uxaa3}

\textbf{નેસ્ટેડ if-else સ્ટેટમેન્ટ ઉદાહરણ સાથે સમજાવો.}

\begin{solutionbox}

\textbf{નેસ્ટેડ if-else}: બીજા if અથવા else બ્લોકની અંદરનું if-else સ્ટેટમેન્ટ.

\textbf{સ્ટ્રક્ચર:}

\begin{verbatim}
if (condition1) \{
    if (condition2) \{
        statements1;
    \} else \{
        statements2;
    \}
\} else \{
    statements3;
\}
\end{verbatim}

\textbf{કોડ ઉદાહરણ:}

\begin{verbatim}
\#include {stdio.h}
void main()
\{
    int age, weight;
    
    printf("Enter age: ");
    scanf("\%d", \&age);
    
    if (age {=} 18) \{
        printf("Enter weight: ");
        scanf("\%d", \&weight);
        
        if (weight {=} 50) \{
            printf("Eligible to donate blood");
        \} else \{
            printf("Underweight, not eligible");
        \}
    \} else \{
        printf("Age below 18, not eligible");
    \}
\}
\end{verbatim}

\textbf{નેસ્ટેડ if-else ડાયગ્રામ:}

\begin{verbatim}
flowchart LR
    A[Start] {-{-} B\{age = 18?\}}
    B {-{-}|Yes| C\{weight = 50?\}}
    B {-{-}|No| D[Not eligible: Age]}
    C {-{-}|Yes| E[Eligible]}
    C {-{-}|No| F[Not eligible: Weight]}
    D \& E \& F {-{-} G[End]}
\end{verbatim}

\end{solutionbox}
\begin{mnemonicbox}
``Check Outside Then Inside'' (COTI)

\end{mnemonicbox}
\subsection*{પ્રશ્ન 4(બ) [4
ગુણ]}\label{uxaaauxab0uxab6uxaa8-4uxaac-4-uxa97uxaa3}

\textbf{Pointer arguments નો ઉપયોગ કરીને બે પૂર્ણાંક સંખ્યાની અદલાબદલી કરવાનો
પ્રોગ્રામ લખો.}

\begin{solutionbox}

\textbf{પ્રોગ્રામ:}

\begin{verbatim}
\#include {stdio.h}
void main()
\{
    int a, b, temp;
    int *p1, *p2;
    
    printf("Enter value of a: ");
    scanf("\%d", \&a);
    
    printf("Enter value of b: ");
    scanf("\%d", \&b);
    
    p1 = \&a;  // p1 a ને પોઇન્ટ કરે છે
    p2 = \&b;  // p2 b ને પોઇન્ટ કરે છે
    
printf("Before swapping:

a = \%d,

b = \%d{n}", a, b);

    
    // પોઇન્ટર્સનો ઉપયોગ કરીને અદલાબદલી
    temp = *p1;
    *p1 = *p2;
    *p2 = temp;
    
printf("After swapping:

a = \%d,

b = \%d", a, b);

\}
\end{verbatim}

\textbf{પોઇન્ટર અદલાબદલી ડાયગ્રામ:}

\begin{verbatim}
        +{-{-}{-}+        +{-}{-}{-}+}
        | 5 |{{-}{-}{-}{-}{-}{-}{-}|p1 |}
   a {- +{-}{-}{-}+        +{-}{-}{-}+}
   
        +{-{-}{-}+        +{-}{-}{-}+}
        | 10|{{-}{-}{-}{-}{-}{-}{-}|p2 |}
   b {- +{-}{-}{-}+        +{-}{-}{-}+}
   
   After swapping:
   
        +{-{-}{-}+        +{-}{-}{-}+}
        | 10|{{-}{-}{-}{-}{-}{-}{-}|p1 |}
   a {- +{-}{-}{-}+        +{-}{-}{-}+}
   
        +{-{-}{-}+        +{-}{-}{-}+}
        | 5 |{{-}{-}{-}{-}{-}{-}{-}|p2 |}
   b {- +{-}{-}{-}+        +{-}{-}{-}+}
\end{verbatim}

\end{solutionbox}
\begin{mnemonicbox}
``Pointers Exchange Memory Values'' (PEMV)

\end{mnemonicbox}
\subsection*{પ્રશ્ન 4(ક) [7
ગુણ]}\label{uxaaauxab0uxab6uxaa8-4uxa95-7-uxa97uxaa3}

\textbf{Array ની વ્યાખ્યા લખો. One dimensional array નું initialization અને
declaration સમજાવો.}

\begin{solutionbox}

\textbf{Array}: એક જ ડેટા પ્રકારના તત્વોનો સમૂહ જે સળંગ મેમરી સ્થાનોમાં સંગ્રહિત
થાય છે અને ઇન્ડેક્સ વડે ઍક્સેસ થાય છે.


{\def\LTcaptype{none} % do not increment counter
\vspace{-5pt}
\captionof{table}{Array ડિક્લેરેશન \& ઇનિશિયલાઇઝેશન}
\vspace{-10pt}
\begin{longtable}[]{@{}
  >{\raggedright\arraybackslash}p{(\linewidth - 4\tabcolsep) * \real{0.3929}}
  >{\raggedright\arraybackslash}p{(\linewidth - 4\tabcolsep) * \real{0.2857}}
  >{\raggedright\arraybackslash}p{(\linewidth - 4\tabcolsep) * \real{0.3214}}@{}}
\toprule\noalign{}
\begin{minipage}[b]{\linewidth}\raggedright
ઓપરેશન
\end{minipage} & \begin{minipage}[b]{\linewidth}\raggedright
સિન્ટેક્સ
\end{minipage} & \begin{minipage}[b]{\linewidth}\raggedright
ઉદાહરણ
\end{minipage} \\
\midrule\noalign{}
\endhead
\bottomrule\noalign{}
\endlastfoot
ડિક્લેરેશન & data\_type array\_name[size]; & int marks[5]; \\
ડિક્લેરેશન સમયે ઇનિશિયલાઇઝેશન & data\_type array\_name[size] =
\{values\}; & int nums[4] = \{10, 20, 30, 40\}; \\
આંશિક ઇનિશિયલાઇઝેશન & data\_type array\_name[size] = \{values\}; & int
nums[5] = \{10, 20\}; \\
સાઇઝ વિના & data\_type array\_name[] = \{values\}; & int nums[]
= \{10, 20, 30\}; \\
વ્યક્તિગત તત્વ & array\_name[index] = value; & marks[0] = 95; \\
\end{longtable}
}

\textbf{કોડ ઉદાહરણ:}

\begin{verbatim}
\#include {stdio.h}
void main()
\{
    // ડિક્લેરેશન
    int marks[5];
    
    // ડિક્લેરેશન પછી ઇનિશિયલાઇઝેશન
    marks[0] = 85;
    marks[1] = 90;
    marks[2] = 78;
    marks[3] = 92;
    marks[4] = 88;
    
    // ડિક્લેરેશન સાથે ઇનિશિયલાઇઝેશન
    int scores[] = \{95, 89, 76, 82, 91\;}
    
    // એરે તત્વો ઍક્સેસ કરવા
    printf("marks[2] = \%d{n}", marks[2]);
    printf("scores[3] = \%d", scores[3]);
\}
\end{verbatim}

\textbf{એરે રજૂઆત:}

\begin{verbatim}
marks: [85][90][78][92][88]
        |   |   |   |   |
        0   1   2   3   4  (indices)
\end{verbatim}

\textbf{મેમરી રજૂઆત:}

\begin{verbatim}
flowchart LR
    A["marks[0]{br /85"] {-}{-}{-} B["marks[1]br /90"]}
    B {-{-}{-} C["marks[2]br /78"]}
    C {-{-}{-} D["marks[3]br /92"]}
    D {-{-}{-} E["marks[4]br /88"]}
\end{verbatim}

\end{solutionbox}
\begin{mnemonicbox}
``Declare, Initialize, Access With Index'' (DIAWI)

\end{mnemonicbox}
\subsection*{પ્રશ્ન 4(અ) OR [3
ગુણ]}\label{uxaaauxab0uxab6uxaa8-4uxa85-or-3-uxa97uxaa3}

\textbf{do while loop ઉદાહરણ સાથે સમજાવો.}

\begin{solutionbox}

\textbf{do-while loop}: એક લૂપ જે શરતની ચકાસણી કરતા પહેલા ઓછામાં ઓછી એકવાર
લૂપ બોડી ચલાવે છે.

\textbf{સ્ટ્રક્ચર:}

\begin{verbatim}
do \{
    statements;
\} while(condition);
\end{verbatim}

\textbf{કોડ ઉદાહરણ:}

\begin{verbatim}
\#include {stdio.h}
void main()
\{
    int num, sum = 0;
    
    do \{
        printf("Enter a number (0 to stop): ");
        scanf("\%d", \&num);
        sum += num;
    \} while(num != 0);
    
    printf("Sum of entered numbers = \%d", sum);
\}
\end{verbatim}

\textbf{do-while લૂપ ફ્લો:}

\begin{verbatim}
flowchart TD
    A([Start]) {-{-} B[Body statements]}
    B {-{-} C\{Condition\}}
    C {-{-}|True| B}
    C {-{-}|False| D([End])}
\end{verbatim}

\textbf{while લૂપથી મુખ્ય તફાવતો:}

\begin{itemize}
\tightlist
\item
  બોડી ઓછામાં ઓછી એકવાર ચલાવે છે
\item
  સ્ટેટમેન્ટ્સ ચલાવ્યા પછી કંડીશન ચેક કરે છે
\item
  કંડીશન પછી સેમિકોલોન જરૂરી છે
\end{itemize}

\end{solutionbox}
\begin{mnemonicbox}
``Do First, Check Later'' (DFCL)

\end{mnemonicbox}
\subsection*{પ્રશ્ન 4(બ) OR [4
ગુણ]}\label{uxaaauxab0uxab6uxaa8-4uxaac-or-4-uxa97uxaa3}

\textbf{નીચે આપેલ ફંકશન ઉદાહરણ સાથે સમજાવો:} \textbf{(1) gets() (2) puts()
(3) strlen() (4) strcpy()}

\begin{solutionbox}


{\def\LTcaptype{none} % do not increment counter
\vspace{-5pt}
\captionof{table}{C માં સ્ટ્રિંગ ફંકશન્સ}
\vspace{-10pt}
\begin{longtable}[]{@{}
  >{\raggedright\arraybackslash}p{(\linewidth - 6\tabcolsep) * \real{0.2778}}
  >{\raggedright\arraybackslash}p{(\linewidth - 6\tabcolsep) * \real{0.2500}}
  >{\raggedright\arraybackslash}p{(\linewidth - 6\tabcolsep) * \real{0.2222}}
  >{\raggedright\arraybackslash}p{(\linewidth - 6\tabcolsep) * \real{0.2500}}@{}}
\toprule\noalign{}
\begin{minipage}[b]{\linewidth}\raggedright
ફંકશન
\end{minipage} & \begin{minipage}[b]{\linewidth}\raggedright
હેતુ
\end{minipage} & \begin{minipage}[b]{\linewidth}\raggedright
સિન્ટેક્સ
\end{minipage} & \begin{minipage}[b]{\linewidth}\raggedright
ઉદાહરણ
\end{minipage} \\
\midrule\noalign{}
\endhead
\bottomrule\noalign{}
\endlastfoot
gets() & સ્પેસ સાથે સ્ટ્રિંગ વાંચે છે & gets(string); & gets(name); \\
puts() & ન્યૂલાઇન સાથે સ્ટ્રિંગ દર્શાવે છે & puts(string); & puts(name); \\
strlen() & સ્ટ્રિંગની લંબાઈ આપે છે & strlen(string); & n = strlen(name); \\
strcpy() & સોર્સને ડેસ્ટિનેશનમાં કોપી કરે છે & strcpy(dest, src); & strcpy(str1,
str2); \\
\end{longtable}
}

\textbf{કોડ ઉદાહરણ:}

\begin{verbatim}
\#include {stdio.h}
\#include {string.h}
void main()
\{
    char name[50], copy[50];
    int length;
    
    printf("Enter your name: ");
    gets(name);           // સ્પેસ સાથે નામ વાંચે છે
    
    puts("Your name is:"); // ન્યૂલાઇન સાથે દર્શાવે છે
    puts(name);
    
    length = strlen(name); // સ્ટ્રિંગની લંબાઈ મેળવે છે
    printf("Length: \%d{n}", length);
    
    strcpy(copy, name);    // name ને copy માં કોપી કરે છે
    printf("Copied string: \%s", copy);
\}
\end{verbatim}

\end{solutionbox}
\begin{mnemonicbox}
``Gets Puts String's Length and Copies'' (GPSLC)

\end{mnemonicbox}
\subsection*{પ્રશ્ન 4(ક) OR [7
ગુણ]}\label{uxaaauxab0uxab6uxaa8-4uxa95-or-7-uxa97uxaa3}

\textbf{Recursion ની વ્યાખ્યા આપી ઉદાહરણ સાથે સમજાવો. Recursion નો ઉપયોગ
કરીને આપેલા નંબરનો ફેક્ટોરીયલ શોધવાનો પ્રોગ્રામ લખો.}

\begin{solutionbox}

\textbf{Recursion}: એક પ્રક્રિયા જેમાં ફંક્શન સીધી કે પરોક્ષ રીતે પોતાને જ ચોક્કસ
શરત પૂરી થાય ત્યાં સુધી કૉલ કરે છે.

\textbf{Recursion ના ઘટકો:}

\begin{enumerate}
\tightlist
\item
  બેઝ કેસ: રિકર્ઝન રોકવા માટેની શરત
\item
  રિકર્સિવ કેસ: ફંકશન પોતે જ પોતાને કૉલ કરે છે
\end{enumerate}

\textbf{કોડ ઉદાહરણ:}

\begin{verbatim}
\#include {stdio.h}

// ફેક્ટોરિયલ શોધવા માટે રિકર્સિવ ફંક્શન
unsigned long factorial(int n)
\{
    // બેઝ કેસ
if (n == 0 ||

n == 1)

        return 1;
    
    // રિકર્સિવ કેસ
    else
        return n * factorial(n{-}1);
\}

void main()
\{
    int num;
    unsigned long result;
    
    printf("Enter a number: ");
    scanf("\%d", \&num);
    
    if (num {} 0)
        printf("Factorial not defined for negative numbers");
    else
    \{
        result = factorial(num);
        printf("Factorial of \%d = \%lu", num, result);
    \}
\}
\end{verbatim}

\textbf{રિકર્સિવ ફેક્ટોરિયલ ગણતરી:} factorial(5) માટે


{\def\LTcaptype{none} % do not increment counter
\vspace{-5pt}
\captionof{table}{રિકર્ઝન ટ્રેસ}
\vspace{-10pt}
\begin{longtable}[]{@{}lll@{}}
\toprule\noalign{}
કૉલ & રિટર્ન & ગણતરી \\
\midrule\noalign{}
\endhead
\bottomrule\noalign{}
\endlastfoot
factorial(5) & 5 \times factorial(4) & 5 \times 24 = 120 \\
factorial(4) & 4 \times factorial(3) & 4 \times 6 = 24 \\
factorial(3) & 3 \times factorial(2) & 3 \times 2 = 6 \\
factorial(2) & 2 \times factorial(1) & 2 \times 1 = 2 \\
factorial(1) & 1 & બેઝ કેસ \\
\end{longtable}
}

\textbf{રિકર્ઝન ડાયગ્રામ:}

\begin{verbatim}
flowchart LR
    A["factorial(5)"] {-{-} B["5 * factorial(4)"]}
    B {-{-} C["4 * factorial(3)"]}
    C {-{-} D["3 * factorial(2)"]}
    D {-{-} E["2 * factorial(1)"]}
    E {-{-} F["return 1"]}
    F {-{-} G["return 2"]}
    G {-{-} H["return 6"]}
    H {-{-} I["return 24"]}
    I {-{-} J["return 120"]}
\end{verbatim}

\end{solutionbox}
\begin{mnemonicbox}
``Function Calling Itself, Bottoming Out'' (FCIBO)

\end{mnemonicbox}
\subsection*{પ્રશ્ન 5(અ) [3
ગુણ]}\label{uxaaauxab0uxab6uxaa8-5uxa85-3-uxa97uxaa3}

\textbf{array અને structure વચ્ચેનો તફાવત લખો.}

\begin{solutionbox}


{\def\LTcaptype{none} % do not increment counter
\vspace{-5pt}
\captionof{table}{Array vs Structure}
\vspace{-10pt}
\begin{longtable}[]{@{}
  >{\raggedright\arraybackslash}p{(\linewidth - 4\tabcolsep) * \real{0.3333}}
  >{\raggedright\arraybackslash}p{(\linewidth - 4\tabcolsep) * \real{0.2593}}
  >{\raggedright\arraybackslash}p{(\linewidth - 4\tabcolsep) * \real{0.4074}}@{}}
\toprule\noalign{}
\begin{minipage}[b]{\linewidth}\raggedright
લક્ષણ
\end{minipage} & \begin{minipage}[b]{\linewidth}\raggedright
Array
\end{minipage} & \begin{minipage}[b]{\linewidth}\raggedright
Structure
\end{minipage} \\
\midrule\noalign{}
\endhead
\bottomrule\noalign{}
\endlastfoot
ડેટા પ્રકાર & બધા તત્વો માટે એક જ ડેટા પ્રકાર & વિવિધ ડેટા પ્રકાર સંગ્રહી શકે છે \\
ઍક્સેસ & ઇન્ડેક્સનો ઉપયોગ (arr[0]) & મેમ્બર નામનો ઉપયોગ (s.name) \\
મેમરી ફાળવણી & સળંગ & સળંગ પરંતુ વિવિધ સાઇઝ \\
સાઇઝ & ડિક્લેરેશન સમયે ફિક્સ સાઇઝ & બધા મેમ્બર્સની સાઇઝનો સરવાળો \\
હેતુ & સમાન વસ્તુઓનો સંગ્રહ & વિવિધ પ્રકારના સંબંધિત ડેટાનું ગ્રુપિંગ \\
ડિક્લેરેશન & \texttt{int\ arr[5];} &
\texttt{struct\ student\ \{\ int\ id;\ char\ name[20];\ \};} \\
\end{longtable}
}

\textbf{ડાયગ્રામ:}

\begin{verbatim}
flowchart TD
    subgraph Array
    direction LR
    A["[0]{br /int"] {-}{-}{-} B["[1]br /int"] {-}{-}{-} C["[2]br /int"]}
    end
    
    subgraph Structure
    direction LR
    D["id{br /int"] {-}{-}{-} E["namebr /char[]"] {-}{-}{-} F["agebr /int"]}
    end
\end{verbatim}

\end{solutionbox}
\begin{mnemonicbox}
``Arrays for Same, Structures for Different'' (ASSD)

\end{mnemonicbox}
\subsection*{પ્રશ્ન 5(બ) [4
ગુણ]}\label{uxaaauxab0uxab6uxaa8-5uxaac-4-uxa97uxaa3}

\textbf{આપેલ 10 કિંમતમાંથી મહત્તમ કિંમત શોધવાનો C પ્રોગ્રામ array નો ઉપયોગ કરીને
લખો.}

\begin{solutionbox}

\textbf{પ્રોગ્રામ:}

\begin{verbatim}
\#include {stdio.h}
void main()
\{
    int arr[10], i, max;
    
    // 10 કિંમતો ઇનપુટ
    printf("Enter 10 values:{n}");
    for(i = 0; i {} 10; i++)
    \{
        printf("Enter value \%d: ", i+1);
        scanf("\%d", \&arr[i]);
    \}
    
    // મહત્તમ કિંમત શોધવી
    max = arr[0];  // પ્રથમ તત્વ મહત્તમ માની લો
    for(i = 1; i {} 10; i++)
    \{
        if(arr[i] {} max)
            max = arr[i];
    \}
    
    printf("Maximum value is: \%d", max);
\}
\end{verbatim}

\textbf{અલ્ગોરિધમ ફ્લો:}

\begin{verbatim}
flowchart LR
    A([Start]) {-{-} B[/Input 10 values/]}
    B {-{-} C[Set max = first element]}
    C {-{-} D[Loop i from 1 to 9]}
    D {-{-} E\{"arr[i]  max?"\}}
    E {-{-}|Yes| F["max = arr[i]"]}
    E {-{-}|No| G[Continue]}
    F \& G {-{-} H\{i  9?\}}
    H {-{-}|Yes| D}
    H {-{-}|No| I[/Display max/]}
    I {-{-} J([End])}
\end{verbatim}

\end{solutionbox}
\begin{mnemonicbox}
``Compare And Replace Maximum'' (CARM)

\end{mnemonicbox}
\subsection*{પ્રશ્ન 5(ક) [7
ગુણ]}\label{uxaaauxab0uxab6uxaa8-5uxa95-7-uxa97uxaa3}

\textbf{Structure ને વ્યાખ્યા લખો. Book નામથી એક structure બનાવો કે જેમાં book
વિશેની માહિતી Book title, Name of author, Price and Number of pages સ્ટોર
કરી શકાય.}

\begin{solutionbox}

\textbf{Structure}: વિવિધ ડેટા પ્રકારના સંબંધિત વેરિયેબલ્સને એક જ નામ હેઠળ ગ્રુપ
કરતું યુઝર-ડિફાઇન્ડ ડેટા પ્રકાર.

\textbf{Book Structure કોડ:}

\begin{verbatim}
\#include {stdio.h}

struct book \{
    char title[50];
    char author[30];
    float price;
    int pages;
\;}

void main()
\{
    struct book b1;
    
    // પુસ્તકની વિગતો ઇનપુટ
    printf("Enter book title: ");
    gets(b1.title);
    
    printf("Enter author name: ");
    gets(b1.author);
    
    printf("Enter price: ");
    scanf("\%f", \&b1.price);
    
    printf("Enter number of pages: ");
    scanf("\%d", \&b1.pages);
    
    // પુસ્તકની વિગતો પ્રદર્શિત
    printf("{n}Book Details:{n}");
    printf("Title: \%s{n}", b1.title);
    printf("Author: \%s{n}", b1.author);
    printf("Price: Rs. \%.2f{n}", b1.price);
    printf("Pages: \%d", b1.pages);
\}
\end{verbatim}

\textbf{Structure મેમરી રજૂઆત:}

\begin{verbatim}
+{-{-}{-}{-}{-}{-}{-}{-}{-}{-}{-}{-}{-}{-}{-}{-}{-}{-}{-}{-}{-}{-}{-}{-}+}
| struct book            |
+{-{-}{-}{-}{-}{-}{-}{-}{-}{-}{-}{-}{-}{-}{-}{-}{-}{-}{-}{-}{-}{-}{-}{-}+}
| title[50]  |           |
|            |  "C Prog" |
+{-{-}{-}{-}{-}{-}{-}{-}{-}{-}{-}{-}{-}{-}{-}{-}{-}{-}{-}{-}{-}{-}{-}{-}+}
| author[30] |           |
|            |  "Dennis" |
+{-{-}{-}{-}{-}{-}{-}{-}{-}{-}{-}{-}{-}{-}{-}{-}{-}{-}{-}{-}{-}{-}{-}{-}+}
| price      |   450.50  |
+{-{-}{-}{-}{-}{-}{-}{-}{-}{-}{-}{-}{-}{-}{-}{-}{-}{-}{-}{-}{-}{-}{-}{-}+}
| pages      |    320    |
+{-{-}{-}{-}{-}{-}{-}{-}{-}{-}{-}{-}{-}{-}{-}{-}{-}{-}{-}{-}{-}{-}{-}{-}+}
\end{verbatim}

\textbf{Structure ડાયગ્રામ:}

\begin{verbatim}
classDiagram
    class book \{
        char title[50]
        char author[30]
        float price
        int pages
    \}
\end{verbatim}

\end{solutionbox}
\begin{mnemonicbox}
``Title Author Price Pages'' (TAPP)

\end{mnemonicbox}
\subsection*{પ્રશ્ન 5(અ) OR [3
ગુણ]}\label{uxaaauxab0uxab6uxaa8-5uxa85-or-3-uxa97uxaa3}

\textbf{સ્ટ્રીંગ શું છે? સ્ટ્રીંગ ઉપર કયા ઓપરેશન પરફોર્મ થાય છે.}

\begin{solutionbox}

\textbf{સ્ટ્રીંગ}: NULL કેરેક્ટર `\textbackslash0' દ્વારા સમાપ્ત થતા અક્ષરોની
શ્રેણી.


{\def\LTcaptype{none} % do not increment counter
\vspace{-5pt}
\captionof{table}{C માં સ્ટ્રીંગ ઓપરેશન્સ}
\vspace{-10pt}
\begin{longtable}[]{@{}lll@{}}
\toprule\noalign{}
ઓપરેશન & ફંક્શન & ઉદાહરણ \\
\midrule\noalign{}
\endhead
\bottomrule\noalign{}
\endlastfoot
ઇનપુટ & gets(), scanf() & gets(str), scanf(``\%s'', str) \\
આઉટપુટ & puts(), printf() & puts(str), printf(``\%s'', str) \\
લંબાઈ & strlen() & len = strlen(str) \\
કોપી & strcpy() & strcpy(dest, src) \\
જોડાણ & strcat() & strcat(str1, str2) \\
સરખામણી & strcmp() & result = strcmp(str1, str2) \\
શોધ & strchr(), strstr() & ptr = strchr(str, `a') \\
રૂપાંતર & strlwr(), strupr() & strlwr(str), strupr(str) \\
\end{longtable}
}

\textbf{સ્ટ્રીંગ રજૂઆત:}

\begin{verbatim}
+{-{-}{-}+{-}{-}{-}+{-}{-}{-}+{-}{-}{-}+{-}{-}{-}+{-}{-}{-}+}
| H | e | l | l | o | {0|}
+{-{-}{-}+{-}{-}{-}+{-}{-}{-}+{-}{-}{-}+{-}{-}{-}+{-}{-}{-}+}
\end{verbatim}

\end{solutionbox}
\begin{mnemonicbox}
``Input Output Length Copy Concat Compare Search
Convert'' (IOLCCSC)

\end{mnemonicbox}
\subsection*{પ્રશ્ન 5(બ) OR [4
ગુણ]}\label{uxaaauxab0uxab6uxaa8-5uxaac-or-4-uxa97uxaa3}

\textbf{A to Z ની ASCII વેલ્યુ પ્રિન્ટ કરવા માટેનો પ્રોગ્રામ લખો.}

\begin{solutionbox}

\textbf{પ્રોગ્રામ:}

\begin{verbatim}
\#include {stdio.h}
void main()
\{
    char ch;
    
    printf("ASCII values from A to Z:{n}");
    printf("Character{t}ASCII Value{n}");
    printf("{-{-}{-}{-}{-}{-}{-}{-}{-}{-}{-}{-}{-}{-}{-}{-}{-}{-}{-}{-}{-}{-}{-}}{n}");
    
    for(ch = {A}; ch {=} {Z}; ch++)
    \{
        printf("    \%c{tt}   \%d{n}", ch, ch);
    \}
\}
\end{verbatim}

\textbf{સેમ્પલ આઉટપુટ કોષ્ટક:}

{\def\LTcaptype{none} % do not increment counter
\begin{longtable}[]{@{}ll@{}}
\toprule\noalign{}
Character & ASCII Value \\
\midrule\noalign{}
\endhead
\bottomrule\noalign{}
\endlastfoot
A & 65 \\
B & 66 \\
\ldots{} & \ldots{} \\
Z & 90 \\
\end{longtable}
}

\textbf{ASCII ચાર્ટ રજૂઆત:}

\begin{verbatim}
ASCII Values:
A(65) B(66) C(67) ... Z(90)
\end{verbatim}

\end{solutionbox}
\begin{mnemonicbox}
``Alphabets Sequentially Creating Integer Indices''
(ASCII)

\end{mnemonicbox}
\subsection*{પ્રશ્ન 5(ક) OR [7
ગુણ]}\label{uxaaauxab0uxab6uxaa8-5uxa95-or-7-uxa97uxaa3}

\textbf{user defined અને library function શું છે? દરેકના બે ઉદાહરણ સાથે
સમજાવો.}

\begin{solutionbox}

\textbf{Library Functions}: C ભાષા દ્વારા પૂરા પાડવામાં આવતા પહેલેથી
વ્યાખ્યાયિત ફંક્શન્સ જે ઉપયોગ માટે તૈયાર છે.

\textbf{User-Defined Functions}: પ્રોગ્રામર દ્વારા ચોક્કસ કાર્યો કરવા માટે
બનાવેલા ફંક્શન્સ.


{\def\LTcaptype{none} % do not increment counter
\vspace{-5pt}
\captionof{table}{Library vs User-Defined Functions}
\vspace{-10pt}
\begin{longtable}[]{@{}lll@{}}
\toprule\noalign{}
લક્ષણ & Library Functions & User-Defined Functions \\
\midrule\noalign{}
\endhead
\bottomrule\noalign{}
\endlastfoot
વ્યાખ્યા & હેડર ફાઈલોમાં પહેલેથી વ્યાખ્યાયિત & પ્રોગ્રામર દ્વારા બનાવવામાં આવે છે \\
ડિક્લેરેશન & વ્યાખ્યા કરવાની જરૂર નથી & વ્યાખ્યા કરવી જ જોઈએ \\
ઉદાહરણો & printf(), scanf(), strlen() & calculateArea(), findMax() \\
હેડર ફાઇલ્સ & stdio.h, string.h, math.h, etc. & કોઈ હેડર જરૂરી નથી \\
હેતુ & સામાન્ય કાર્યો & કસ્ટમાઇઝ્ડ કાર્યો \\
\end{longtable}
}

\textbf{Library Functions ના ઉદાહરણો:}

\begin{enumerate}
\tightlist
\item
  \textbf{strlen() - સ્ટ્રિંગ લંબાઈ}
\end{enumerate}

\begin{verbatim}
\#include {stdio.h}
\#include {string.h}
void main()
\{
    char str[] = "Hello";
    int length = strlen(str);  // Library function
    printf("Length of string: \%d", length);
\}
\end{verbatim}

\begin{enumerate}
\tightlist
\item
  \textbf{sqrt() - વર્ગમૂળ}
\end{enumerate}

\begin{verbatim}
\#include {stdio.h}
\#include {math.h}
void main()
\{
    float num = 25, result;
    result = sqrt(num);  // Library function
    printf("Square root of \%.0f = \%.2f", num, result);
\}
\end{verbatim}

\textbf{User-Defined Functions ના ઉદાહરણો:}

\begin{enumerate}
\tightlist
\item
  \textbf{calculateArea() - લંબચોરસનું ક્ષેત્રફળ}
\end{enumerate}

\begin{verbatim}
\#include {stdio.h}

// User{-defined function}
float calculateArea(float length, float width)
\{
    return length * width;
\}

void main()
\{
    float length = 10.5, width = 5.5, area;
    area = calculateArea(length, width);  // User function call
    printf("Area of rectangle = \%.2f", area);
\}
\end{verbatim}

\begin{enumerate}
\tightlist
\item
  \textbf{findMax() - ત્રણ સંખ્યાઓમાંથી મહત્તમ}
\end{enumerate}

\begin{verbatim}
\#include {stdio.h}

// User{-defined function}
int findMax(int a, int b, int c)
\{
    if(a {=} b \&\& a {=} c)
        return a;
    else if(b {=} a \&\& b {=} c)
        return b;
    else
        return c;
\}

void main()
\{
int

x = 10,

y = 25,

z = 15, max;

    max = findMax(x, y, z);  // User function call
    printf("Maximum number is: \%d", max);
\}
\end{verbatim}

\end{solutionbox}
\begin{mnemonicbox}
``Libraries Provide, Users Create'' (LPUC)

\end{mnemonicbox}

\end{document}
