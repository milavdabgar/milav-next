\documentclass{article}

% content/resources/templates/preamble.tex
\usepackage[margin=0.6in]{geometry}
\author{Milav Dabgar}
\usepackage{amsmath,amssymb,amsthm}
\usepackage{booktabs}
\usepackage{multirow}
\usepackage{xcolor}
\usepackage{tcolorbox}
\tcbuselibrary{breakable,skins}
\usepackage[colorlinks=true,linkcolor=blue]{hyperref}
\usepackage{titlesec}
\usepackage{enumitem}
\usepackage{tikz}
\usepackage{pgfplots}
\usepackage{circuitikz}
\usepackage[version=4]{mhchem}
\usepackage{longtable}
\usepackage{array}
\usepackage{float}
\usepackage{caption}
\usepackage{listings}

\lstset{
  basicstyle=\small\ttfamily,
  breaklines=true,
  breakatwhitespace=false,
  postbreak=\mbox{\textcolor{red}{$\hookrightarrow$}\space},
  float=false,
  numbers=left,
  numberstyle=\tiny\color{gray},
  numbersep=10pt,
  xleftmargin=2em,
  keywordstyle=\color{blue},
  commentstyle=\color{green!60!black},
  stringstyle=\color{purple},
  backgroundcolor=\color{gray!5},
  showstringspaces=false,
  tabsize=2,
  captionpos=b,
  keepspaces=true,
  columns=flexible
}

\pgfplotsset{compat=1.18}
\usetikzlibrary{shapes,arrows,positioning,calc,patterns,decorations.pathmorphing,decorations.markings,arrows.meta}

% Color scheme
\definecolor{headcolor}{RGB}{0,102,204}
\definecolor{keycolor}{RGB}{220,20,60}
\definecolor{solutioncolor}{RGB}{34,139,34}
\definecolor{mnemoniccolor}{RGB}{148,0,211}
\definecolor{codecolor}{RGB}{0,0,100}

% Spacing
\setlength{\parskip}{3pt}
\setlist[itemize]{nosep}
\setlist[enumerate]{nosep}

% Title formatting
\titleformat{\section}{\Large\bfseries\color{headcolor}}{\thesection}{1em}{}
\titleformat{\subsection}{\large\bfseries\color{headcolor}}{\thesubsection}{1em}{}

% Pandoc tightlist compatibility
\providecommand{\tightlist}{%
  \setlength{\itemsep}{0pt}\setlength{\parskip}{0pt}}

% Pandoc longtable compatibility
\newcounter{none}
\def\thenone{}


% content/resources/templates/gujarati-boxes.tex
\usepackage{fontspec}
\usepackage{polyglossia}

% Set Gujarati as main language (document is primarily in Gujarati)
% Note: gloss-gujarati.ldf doesn't exist in polyglossia, but it will use hyphenation patterns
\setdefaultlanguage{gujarati}
\setotherlanguage{english}

% Configure Gujarati font properly
% Use Language=Default to prevent polyglossia from trying to add language-specific features
% that don't exist for Gujarati, which causes "empty feature" warnings
\newfontfamily\gujaratifont[Script=Gujarati,AutoFakeBold=2.5,AutoFakeSlant=0.3]{Noto Sans Gujarati}
\setmainfont[Script=Gujarati,AutoFakeBold=2.5,AutoFakeSlant=0.3]{Noto Sans Gujarati}
% Use Noto Sans Gujarati for monospace to support Gujarati in text
\setmonofont[Scale=0.9]{Noto Sans Gujarati}

% Configure English to use the same font
\newfontfamily\englishfont[Script=Gujarati,AutoFakeBold=2.5,AutoFakeSlant=0.3]{Noto Sans Gujarati}

% Translations for polyglossia
\gappto\captionsgujarati{
  \renewcommand{\tablename}{કોષ્ટક}
  \renewcommand{\figurename}{આકૃતિ}
}

% Helper for TikZ nodes to ensure Gujarati font
\newcommand{\gu}[1]{{\gujaratifont #1}}

% Custom environments
\newtcolorbox{solutionbox}{
    breakable,
    enhanced,
    colback=solutioncolor!5!white,
    colframe=solutioncolor!75!black,
    fonttitle=\bfseries,
    title=જવાબ
}

\newtcolorbox{solutionboxnobreak}{
 colback=solutioncolor!5!white,
 colframe=solutioncolor!75!black,
 fonttitle=\bfseries,
 title=જવાબ
}

\newtcolorbox{keyformula}{
 breakable,
 enhanced,
 colback=keycolor!5!white,
 colframe=keycolor!75!black,
 fonttitle=\bfseries,
 title=રાસાયણિક સમીકરણ/સૂત્ર
}

\newtcolorbox{mnemonicbox}{
 breakable,
 enhanced,
 colback=mnemoniccolor!5!white,
 colframe=mnemoniccolor!75!black,
 fonttitle=\bfseries,
 title=મેમરી ટ્રીક
}


% Custom commands for GTU solutions
% This file defines semantic commands for consistent formatting

% Question command with automatic formatting
\newcommand{\question}[2]{%
  \section*{Question #1}%
  \textbf{#2}%
}

% OR question variant
\newcommand{\questionor}[2]{%
  \section*{Question #1 OR}%
  \textbf{#2}%
}

% Proper table environment with caption
\newenvironment{answertable}[1]{%
  \begin{table}[htbp]
  \centering
  \caption{#1}
}{%
  \end{table}
}

% Proper figure environment for diagrams
\newenvironment{answerdiagram}[1]{%
  \begin{figure}[htbp]
  \centering
  \caption{#1}
}{%
  \end{figure}
}

% Semantic markup for key terms
\newcommand{\keyword}[1]{\textbf{#1}}
\newcommand{\code}[1]{\texttt{#1}}
\newcommand{\classname}[1]{\texttt{#1}}
\newcommand{\methodname}[1]{\texttt{#1}}

% Proper quotation marks
\newcommand{\mnemonic}[1]{``#1''}


\title{Programming In C (4331105) - Winter 2022 Solution}
\date{March 28, 2023}

\begin{document}
\maketitle

\questionmarks{1}{a}{3}
\textbf{C ભાષાના મૂળભૂત data types તેમની range સાથે જણાવો}

\begin{solutionbox}
    \textbf{જવાબ}:

    \begin{tabulary}{\linewidth}{|L|L|L|}
        \hline
        \textbf{Data Type} & \textbf{Size (bytes)} & \textbf{Range} \\
        \hline
        char & 1 & -128 થી 127 \\
        \hline
        int & 2 or 4 & -32,768 થી 32,767 (2 bytes) અથવા -2,147,483,648 થી 2,147,483,647 (4 bytes) \\
        \hline
        float & 4 & 3.4E-38 થી 3.4E+38 \\
        \hline
        double & 8 & 1.7E-308 થી 1.7E+308 \\
        \hline
    \end{tabulary}

    \begin{mnemonicbox}"CIFD - Computer Is Fundamentally Digital"\end{mnemonicbox}
\end{solutionbox}

\questionmarks{1}{b}{4}
\textbf{વેરીએબલના નામ માટેના નિયમો સમજાવો.}

\begin{solutionbox}
    \textbf{જવાબ}:

    \begin{tabulary}{\linewidth}{|L|L|}
        \hline
        \textbf{નિયમ} & \textbf{ઉદાહરણ} \\
        \hline
        અક્ષર અથવા અંડરસ્કોરથી શરૂ થવું જોઈએ & માન્ય: \code{\_count}, \code{name} / અમાન્ય: \code{1score} \\
        \hline
        અક્ષરો, અંકો, અંડરસ્કોર હોઈ શકે છે & માન્ય: \code{user\_1} / અમાન્ય: \code{user-1} \\
        \hline
        Keywords નો ઉપયોગ કરી શકાતો નથી & માન્ય: \code{integer} / અમાન્ય: \code{int} \\
        \hline
        Case sensitive છે & \code{total} અને \code{TOTAL} અલગ છે \\
        \hline
    \end{tabulary}

    \textbf{આકૃતિ:}

    \begin{center}
    \begin{tikzpicture}[gtu flow]
        \node[gtu block, align=center] (rules) {\textbf{વેરિએબલ નામકરણ નિયમો} \\ \rule{4cm}{0.4pt} \\ \code{[A-Z, a-z, \_]} $\rightarrow$ \code{[A-Z, a-z, 0-9, \_]*}};
    \end{tikzpicture}
    \end{center}

    \begin{mnemonicbox}"LUCK - Letters Underscore Case Keywords"\end{mnemonicbox}
\end{solutionbox}

\questionmarks{1}{c}{7}
\textbf{ફ્લોચાર્ટની વ્યાખ્યા લખો. બે પૂર્ણાંક નંબરો N1 અને N2 માંથી લઘુત્તમ શોધવા માટેનો ફ્લોચાર્ટ દોરો.}

\begin{solutionbox}
    \textbf{જવાબ}:

    ફ્લોચાર્ટ એ અલ્ગોરિધમનું ચિત્રાત્મક નિરૂપણ છે જે પ્રમાણભૂત પ્રતીકોનો ઉપયોગ કરીને પ્રક્રિયાના પગલાં બતાવે છે.

    \begin{center}
    \begin{tikzpicture}[gtu flow]
        \node[gtu start] (start) {શરૂઆત};
        \node[gtu input, right=of start] (input) {Input N1, N2};
        \node[gtu decision, right=of input] (dec) {N1 < N2?};
        \node[gtu process, above right=of dec] (min1) {min = N1};
        \node[gtu process, below right=of dec] (min2) {min = N2};
        \node[gtu output, right=of dec, xshift=3cm] (output) {Output min};
        \node[gtu stop, right=of output] (end) {અંત};

        \draw[gtu arrow] (start) -- (input);
        \draw[gtu arrow] (input) -- (dec);
        \draw[gtu arrow] (dec) |- node[near start, above] {હા} (min1);
        \draw[gtu arrow] (dec) |- node[near start, below] {ના} (min2);
        \draw[gtu arrow] (min1) -| (output);
        \draw[gtu arrow] (min2) -| (output);
        \draw[gtu arrow] (output) -- (end);
    \end{tikzpicture}
    \end{center}

    \begin{itemize}
        \item \textbf{ફ્લોચાર્ટ પ્રતીકો}: તાર્કિક પગલાંનું દ્રશ્ય નિરૂપણ
        \item \textbf{નિર્ણય ડાયમંડ}: પ્રવાહનો માર્ગ નક્કી કરવા શરત તપાસે છે
        \item \textbf{પ્રક્રિયા બોક્સ}: ગણતરીઓ અથવા કામગીરી સમાવે છે
    \end{itemize}

    \begin{mnemonicbox}"FAST - Flow Analysis Shown Through-charts"\end{mnemonicbox}
\end{solutionbox}

\orquestionmarks{1}{c}{7}
\textbf{અલગોરિથમની વ્યાખ્યા લખો. વર્તુળનું ક્ષેત્રફળ અને પરિઘ ગણવા માટેનો અલગોરિથમ લખો.}

\begin{solutionbox}
    \textbf{જવાબ}:

    અલ્ગોરિધમ એ કોઈ ચોક્કસ સમસ્યા ઉકેલવા માટેની સુનિશ્ચિત સૂચનાઓની પગલાવાર પ્રક્રિયા છે.

    \textbf{વર્તુળની ગણતરી માટે અલ્ગોરિધમ:}

\begin{lstlisting}
1. START
2. Input radius r
3. Calculate area = PI * r * r
4. Calculate circumference = 2 * PI * r
5. Output area, circumference
6. STOP
\end{lstlisting}

    \begin{tabulary}{\linewidth}{|L|L|L|}
        \hline
        \textbf{પગલું} & \textbf{કામગીરી} & \textbf{સૂત્ર} \\
        \hline
        1 & ત્રિજ્યા મેળવો & Input r \\
        \hline
        2 & ક્ષેત્રફળ ગણો & $A = \pi \times r^2$ \\
        \hline
        3 & પરિઘ ગણો & $C = 2 \times \pi \times r$ \\
        \hline
        4 & પરિણામ દર્શાવો & Output A, C \\
        \hline
    \end{tabulary}

    \begin{mnemonicbox}"SICS - Steps In Clear Sequence"\end{mnemonicbox}
\end{solutionbox}

\questionmarks{2}{a}{3}
\textbf{printf() અને scanf() વચ્ચેનો તફાવત લખો.}

\begin{solutionbox}
    \textbf{જવાબ}:

    \begin{tabulary}{\linewidth}{|L|L|L|}
        \hline
        \textbf{ફીચર} & \textbf{printf()} & \textbf{scanf()} \\
        \hline
        ઉદ્દેશ & સ્ક્રીન પર ડેટા આઉટપુટ કરે છે & કીબોર્ડથી ડેટા ઇનપુટ લે છે \\
        \hline
        ફોર્મેટ & \code{printf("format", variables)} & \code{scanf("format", \&variables)} \\
        \hline
        Returns & પ્રિન્ટ થયેલ અક્ષરોની સંખ્યા & સફળતાપૂર્વક વાંચેલ વસ્તુઓની સંખ્યા \\
        \hline
        એડ્રેસિંગ & વેરિએબલ નામોનો ઉપયોગ કરે છે & વેરિએબલ્સના એડ્રેસ (\&var) નો ઉપયોગ કરે છે \\
        \hline
    \end{tabulary}

    \begin{mnemonicbox}"IO-AR - Input Output-Address Returns"\end{mnemonicbox}
\end{solutionbox}

\questionmarks{2}{b}{4}
\textbf{Conditional operator નો ઉપયોગ કરીને બે નંબરોમાંથી મહત્તમ શોધવા માટે C પ્રોગ્રામ લખો.}

\begin{solutionbox}
    \textbf{જવાબ}:

\begin{lstlisting}[language=C]
#include <stdio.h>

int main() {
    int num1, num2, max;
    
    printf("Enter two numbers: ");
    scanf("%d %d", &num1, &num2);
    
    max = (num1 > num2) ? num1 : num2;
    
    printf("Maximum number is: %d", max);
    
    return 0;
}
\end{lstlisting}

    \textbf{આકૃતિ:}

    \begin{center}
    \begin{tikzpicture}[gtu flow]
        \node[gtu input] (input) {\textbf{Input} \\ num1, num2};
        \node[gtu decision, right=of input] (cond) {\textbf{Condition} \\ num1 > num2?};
        \node[gtu output, right=of cond] (output) {\textbf{Output} \\ max};

        \draw[gtu arrow] (input) -- (cond);
        \draw[gtu arrow] (cond) -- (output);
    \end{tikzpicture}
    \end{center}

    \begin{mnemonicbox}"CTO - Condition Then Output"\end{mnemonicbox}
\end{solutionbox}

\questionmarks{2}{c}{7}
\textbf{એરિથમેટિક અને રીલેશનલ ઓપરેટર્સ ઉદાહરણ સાથે સમજાવો.}

\begin{solutionbox}
    \textbf{જવાબ}:

    \begin{tabulary}{\linewidth}{|L|C|L|L|}
        \hline
        \textbf{પ્રકાર} & \textbf{Operators} & \textbf{ઉદાહરણ} & \textbf{પરિણામ} \\
        \hline
        \multicolumn{4}{|l|}{\textbf{એરિથમેટિક ઓપરેટર્સ}} \\
        \hline
        સરવાળો & + & 5 + 3 & 8 \\
        \hline
        બાદબાકી & - & 5 - 3 & 2 \\
        \hline
        ગુણાકાર & * & 5 * 3 & 15 \\
        \hline
        ભાગાકાર & / & 5 / 3 & 1 (પૂર્ણાંક ભાગાકાર) \\
        \hline
        મોડ્યુલસ & \% & 5 \% 3 & 2 (શેષ) \\
        \hline
        \multicolumn{4}{|l|}{\textbf{રીલેશનલ ઓપરેટર્સ}} \\
        \hline
        સમાન & == & 5 == 3 & 0 (ખોટું) \\
        \hline
        સમાન નથી & != & 5 != 3 & 1 (સાચું) \\
        \hline
        મોટું & > & 5 > 3 & 1 (સાચું) \\
        \hline
        નાનું & < & 5 < 3 & 0 (ખોટું) \\
        \hline
        મોટું અથવા સમાન & >= & 5 >= 5 & 1 (સાચું) \\
        \hline
        નાનું અથવા સમાન & <= & 5 <= 3 & 0 (ખોટું) \\
        \hline
    \end{tabulary}

    \begin{mnemonicbox}"ASMDCRO - Add Subtract Multiply Divide Compare Return Output"\end{mnemonicbox}
\end{solutionbox}

\orquestionmarks{2}{a}{3}
\textbf{ઓપરેટરની અગ્રતા (Precedence) ધ્યાનમાં રાખીને, જો expression (25/3) * 4 – 10 \% 3 + 9/2 નું મૂલ્યાંકન કરવામાં આવે તો દરેક સ્ટેપ અને છેલ્લો જવાબ લખો.}

\begin{solutionbox}
    \textbf{જવાબ}:

    \begin{tabulary}{\linewidth}{|L|L|L|L|}
        \hline
        \textbf{પગલું} & \textbf{કામગીરી} & \textbf{ગણતરી} & \textbf{પરિણામ} \\
        \hline
        1 & કૌંસ (25/3) & 25/3 = 8 (પૂર્ણાંક ભાગાકાર) & 8 \\
        \hline
        2 & મોડ્યુલસ 10 \% 3 & 10 \% 3 = 1 & 1 \\
        \hline
        3 & ભાગાકાર 9/2 & 9/2 = 4 (પૂર્ણાંક ભાગાકાર) & 4 \\
        \hline
        4 & ગુણાકાર 8 * 4 & 8 * 4 = 32 & 32 \\
        \hline
        5 & બાદબાકી 32 - 1 & 32 - 1 = 31 & 31 \\
        \hline
        6 & સરવાળો 31 + 4 & 31 + 4 = 35 & 35 \\
        \hline
    \end{tabulary}

    અંતિમ જવાબ = 35

    \begin{mnemonicbox}"PEMDAS - Parentheses, Exponents, Multiplication/Division, Addition/Subtraction"\end{mnemonicbox}
\end{solutionbox}

\orquestionmarks{2}{b}{4}
\textbf{એલ્જેબ્રીક સમીકરણના ઉકેલ માટેનો C પ્રોગ્રામ લખો.}

\begin{solutionbox}
    \textbf{જવાબ}:

\begin{lstlisting}[language=C]
#include <stdio.h>
#include <math.h>

int main() {
    float a, b, c;
    float discriminant, root1, root2;
    
    printf("Enter coefficients a, b, c: ");
    scanf("%f %f %f", &a, &b, &c);
    
    discriminant = b*b - 4*a*c;
    
    if (discriminant > 0) {
        root1 = (-b + sqrt(discriminant)) / (2*a);
        root2 = (-b - sqrt(discriminant)) / (2*a);
        printf("Roots: %.2f and %.2f", root1, root2);
    } else if (discriminant == 0) {
        root1 = -b / (2*a);
        printf("Root: %.2f", root1);
    } else {
        printf("No real roots");
    }
    
    return 0;
}
\end{lstlisting}

    \textbf{આકૃતિ:}

    \begin{center}
    \begin{tikzpicture}[gtu flow]
        \node[gtu input] (input) {Input a,b,c};
        \node[gtu process, below=of input] (calc) {Calculate $d = b^2-4ac$};
        \node[gtu decision, below=of calc] (dec1) {$d > 0$?};
        \node[gtu process, left=of dec1] (real2) {બે વાસ્તવિક ઉકેલ};
        \node[gtu decision, right=of dec1] (dec2) {$d = 0$?};
        \node[gtu process, below=of dec2] (real1) {એક વાસ્તવિક ઉકેલ};
        \node[gtu process, right=of dec2, xshift=1cm] (imag) {વાસ્તવિક ઉકેલ નથી};

        \draw[gtu arrow] (input) -- (calc);
        \draw[gtu arrow] (calc) -- (dec1);
        \draw[gtu arrow] (dec1) -- node[above] {હા} (real2);
        \draw[gtu arrow] (dec1) -- node[above] {ના} (dec2);
        \draw[gtu arrow] (dec2) -- node[right] {હા} (real1);
        \draw[gtu arrow] (dec2) -- node[above] {ના} (imag);
    \end{tikzpicture}
    \end{center}

    \begin{mnemonicbox}"QDR - Quadratic Discriminant Roots"\end{mnemonicbox}
\end{solutionbox}

\orquestionmarks{2}{c}{7}
\textbf{લોજીકલ અને બિટવાઈઝ ઓપરેટર્સ ઉદાહરણ સાથે સમજાવો.}

\begin{solutionbox}
    \textbf{જવાબ}:

    \begin{tabulary}{\linewidth}{|L|C|L|L|}
        \hline
        \textbf{પ્રકાર} & \textbf{Operators} & \textbf{ઉદાહરણ} & \textbf{પરિણામ} \\
        \hline
        \multicolumn{4}{|l|}{\textbf{લોજીકલ ઓપરેટર્સ}} \\
        \hline
        લોજીકલ AND & \&\& & (5>3) \&\& (4<7) & 1 (સાચું) \\
        \hline
        લોજીકલ OR & || & (5<3) || (4<7) & 1 (સાચું) \\
        \hline
        લોજીકલ NOT & ! & !(5>3) & 0 (ખોટું) \\
        \hline
        \multicolumn{4}{|l|}{\textbf{બિટવાઈઝ ઓપરેટર્સ}} \\
        \hline
        બિટવાઈઝ AND & \& & 5 \& 3 (101 \& 011) & 1 (001) \\
        \hline
        બિટવાઈઝ OR & | & 5 | 3 (101 | 011) & 7 (111) \\
        \hline
        બિટવાઈઝ XOR & \^{} & 5 \^{} 3 (101 \^{} 011) & 6 (110) \\
        \hline
        બિટવાઈઝ NOT & \~{} & \~{}5 (\~{} 00000101) & -6 (11111010) \\
        \hline
        Left Shift & << & 5 << 1 (101 << 1) & 10 (1010) \\
        \hline
        Right Shift & >> & 5 >> 1 (101 >> 1) & 2 (10) \\
        \hline
    \end{tabulary}

    \begin{mnemonicbox}"LAND BORNS - Logical AND OR NOT, Bitwise OR AND NOT Shift"\end{mnemonicbox}
\end{solutionbox}

\questionmarks{3}{a}{3}
\textbf{'go to' statement નો ઉપયોગ ઉદાહરણ સાથે સમજાવો.}

\begin{solutionbox}
    \textbf{જવાબ}:

    \code{goto} statement પ્રોગ્રામમાં લેબલ થયેલ statement પર સીધો જમ્પ કરવાની મંજૂરી આપે છે.

\begin{lstlisting}[language=C]
#include <stdio.h>

int main() {
    int i = 0;
    
start:
    printf("%d ", i);
    i++;
    if (i < 5)
        goto start;
    
    return 0;
}
// આઉટપુટ: 0 1 2 3 4
\end{lstlisting}

    \textbf{આકૃતિ:}

    \begin{center}
    \begin{tikzpicture}[gtu flow]
        \node[gtu start] (start) {Start};
        \node[gtu process, below=of start] (body) {print(i)\\ i++};
        \node[gtu decision, below=of body] (cond) {i < 5?};
        \node[gtu stop, below=of cond] (end) {End};

        \draw[gtu arrow] (start) -- (body);
        \draw[gtu arrow] (body) -- (cond);
        \draw[gtu arrow] (cond) -- node[right] {ના} (end);
        \draw[gtu arrow] (cond.east) -- ++(1,0) |- node[near start, right] {હા} (start);
    \end{tikzpicture}
    \end{center}

    \begin{mnemonicbox}"JUMP - Just Unconditionally Move Program-counter"\end{mnemonicbox}
\end{solutionbox}

\questionmarks{3}{b}{4}
\textbf{આપેલ નંબર બેકી છે કે એકી તે શોધવા માટે C પ્રોગ્રામ લખો.}

\begin{solutionbox}
    \textbf{જવાબ}:

\begin{lstlisting}[language=C]
#include <stdio.h>

int main() {
    int num;
    
    printf("Enter a number: ");
    scanf("%d", &num);
    
    if (num % 2 == 0)
        printf("%d is even", num);
    else
        printf("%d is odd", num);
        
    return 0;
}
\end{lstlisting}

    \textbf{આકૃતિ:}

    \begin{center}
    \begin{tikzpicture}[gtu flow]
        \node[gtu start] (start) {Start};
        \node[gtu input, right=of start] (input) {Input num};
        \node[gtu decision, right=of input] (cond) {num \% 2 == 0?};
        \node[gtu output, above right=of cond] (even) {Print "Even"};
        \node[gtu output, below right=of cond] (odd) {Print "Odd"};
        \node[gtu stop, right=of cond, xshift=3cm] (end) {End};

        \draw[gtu arrow] (start) -- (input);
        \draw[gtu arrow] (input) -- (cond);
        \draw[gtu arrow] (cond) |- node[near start, above] {હા} (even);
        \draw[gtu arrow] (cond) |- node[near start, below] {ના} (odd);
        \draw[gtu arrow] (even) -| (end);
        \draw[gtu arrow] (odd) -| (end);
    \end{tikzpicture}
    \end{center}

    \begin{mnemonicbox}"MODE - Modulo Odd-Even Determination"\end{mnemonicbox}
\end{solutionbox}

\questionmarks{3}{c}{7}
\textbf{else if ladder ફ્લોચાર્ટ દોરી ઉદાહરણ સાથે સમજાવો.}

\begin{solutionbox}
    \textbf{જવાબ}:

    Else-if ladder ક્રમમાં બહુવિધ શરતો તપાસવા માટે વપરાય છે, જેમાં પ્રથમ સાચી શરત સાથે સંકળાયેલ બ્લોક એક્ઝિક્યુટ થાય છે.

    \begin{center}
    \begin{tikzpicture}[gtu flow]
        \node[gtu start] (start) {Start};
        \node[gtu input, below=of start] (input) {Input marks};
        \node[gtu decision, below=of input] (cond1) {marks $\ge$ 90?};
        \node[gtu process, right=of cond1, xshift=1.5cm] (gradeA) {grade = A};
        
        \node[gtu decision, below=of cond1, yshift=-1cm] (cond2) {marks $\ge$ 80?};
        \node[gtu process, right=of cond2, xshift=1.5cm] (gradeB) {grade = B};
        
        \node[gtu decision, below=of cond2, yshift=-1cm] (cond3) {marks $\ge$ 70?};
        \node[gtu process, right=of cond3, xshift=1.5cm] (gradeC) {grade = C};

        \node[gtu decision, below=of cond3, yshift=-1cm] (cond4) {marks $\ge$ 60?};
        \node[gtu process, right=of cond4, xshift=1.5cm] (gradeD) {grade = D};
        
        \node[gtu process, below=of cond4] (gradeF) {grade = F};
        
        \node[gtu output, right=of gradeC, xshift=2cm] (output) {Output grade};
        \node[gtu stop, below=of output] (end) {End};

        \draw[gtu arrow] (start) -- (input);
        \draw[gtu arrow] (input) -- (cond1);
        \draw[gtu arrow] (cond1) -- node[above] {હા} (gradeA);
        \draw[gtu arrow] (cond1) -- node[left] {ના} (cond2);
        
        \draw[gtu arrow] (cond2) -- node[above] {હા} (gradeB);
        \draw[gtu arrow] (cond2) -- node[left] {ના} (cond3);
        
        \draw[gtu arrow] (cond3) -- node[above] {હા} (gradeC);
        \draw[gtu arrow] (cond3) -- node[left] {ના} (cond4);

        \draw[gtu arrow] (cond4) -- node[above] {હા} (gradeD);
        \draw[gtu arrow] (cond4) -- node[left] {ના} (gradeF);
        
        \draw[gtu arrow] (gradeA) -| (output);
        \draw[gtu arrow] (gradeB) -| (output);
        \draw[gtu arrow] (gradeC) -- (output);
        \draw[gtu arrow] (gradeD) -| (output);
        \draw[gtu arrow] (gradeF) -| (output);
        \draw[gtu arrow] (output) -- (end);
    \end{tikzpicture}
    \end{center}

\begin{lstlisting}[language=C]
#include <stdio.h>

int main() {
    int marks;
    char grade;
    
    printf("Enter marks: ");
    scanf("%d", &marks);
    
    if (marks >= 90)
        grade = 'A';
    else if (marks >= 80)
        grade = 'B';
    else if (marks >= 70)
        grade = 'C';
    else if (marks >= 60)
        grade = 'D';
    else
        grade = 'F';
        
    printf("Grade: %c", grade);
    
    return 0;
}
\end{lstlisting}

    \begin{itemize}
        \item \textbf{એકાધિક શરતો}: ક્રમાનુસાર શરતો તપાસે છે
        \item \textbf{પ્રથમ મેચ}: માત્ર પ્રથમ સાચી શરત માટે કોડ એક્ઝિક્યુટ થાય છે
        \item \textbf{ડિફોલ્ટ કેસ}: છેલ્લે else બાકીના તમામ કેસો સંભાળે છે
    \end{itemize}

    \begin{mnemonicbox}"CAFE - Condition Assess First Eligible"\end{mnemonicbox}
\end{solutionbox}

\orquestionmarks{3}{a}{3}
\textbf{continue અને break statement નો ઉપયોગ સમજાવો.}

\begin{solutionbox}
    \textbf{જવાબ}:

    \begin{tabulary}{\linewidth}{|L|L|L|}
        \hline
        \textbf{Statement} & \textbf{હેતુ} & \textbf{અસર} \\
        \hline
        break & લૂપ અથવા સ્વિચમાંથી બહાર નીકળવા & સમગ્ર લૂપને તરત જ સમાપ્ત કરે છે \\
        \hline
        continue & વર્તમાન પુનરાવર્તન છોડવા & લૂપના આગલા પુનરાવર્તન પર જાય છે \\
        \hline
    \end{tabulary}

\begin{lstlisting}[language=C]
// break example
for(int i=1; i<=10; i++) {
    if(i == 6)
        break;      // i=6 થાય ત્યારે લૂપમાંથી બહાર નીકળે છે
    printf("%d ", i); // આઉટપુટ: 1 2 3 4 5
}

// continue example
for(int i=1; i<=10; i++) {
    if(i % 2 == 0)
        continue;   // બેકી સંખ્યાઓ છોડે છે
    printf("%d ", i); // આઉટપુટ: 1 3 5 7 9
}
\end{lstlisting}

    \textbf{આકૃતિ:}

    \begin{center}
    \begin{tikzpicture}[gtu flow]
        % Break Diagram
        \node[gtu block] (loop1) {\textbf{Loop} \\ \code{1, 2, 3...}};
        \node[gtu process, right=of loop1] (break) {break};
        \node[gtu stop, below=of loop1] (exit1) {Exit Loop};
        \draw[gtu arrow] (loop1) -- (break);
        \draw[gtu arrow] (break) |- (exit1);
        \node[above=of loop1] {\textbf{Break}};

        % Continue Diagram
        \node[gtu block, right=of break, xshift=2cm] (loop2) {\textbf{Loop} \\ \code{1, 2...}};
        \node[gtu process, right=of loop2] (continue) {continue};
        \node[above=of loop2] {\textbf{Continue}};
        \draw[gtu arrow] (loop2) -- (continue);
        \draw[gtu arrow] (continue.north) -- ++(0,0.5) -| (loop2.north);
        \node[above of=continue, yshift=0.5cm] {Next Iteration};
    \end{tikzpicture}
    \end{center}

    \begin{mnemonicbox}"BEST - Break Exits, Skip with conTinue"\end{mnemonicbox}
\end{solutionbox}

\orquestionmarks{3}{b}{4}
\textbf{For loop નો ઉપયોગ કરીને 1 થી 10 નંબરોનો સરવાળો પ્રિન્ટ કરવા માટે C પ્રોગ્રામ લખો.}

\begin{solutionbox}
    \textbf{જવાબ}:

\begin{lstlisting}[language=C]
#include <stdio.h>

int main() {
    int i, sum = 0;
    
    for(i = 1; i <= 10; i++) {
        sum += i;
    }
    
    printf("Sum of numbers from 1 to 10: %d", sum);
    
    return 0;
}
\end{lstlisting}

    \textbf{આકૃતિ:}

    \begin{center}
    \begin{tikzpicture}[gtu flow]
        \node[gtu start] (start) {Start};
        \node[gtu process, right=of start] (init) {sum = 0\\i = 1};
        \node[gtu decision, right=of init] (cond) {i $\le$ 10?};
        \node[gtu process, below=of cond] (body) {sum += i\\i++};
        \node[gtu output, right=of cond] (output) {Print sum};
        \node[gtu stop, right=of output] (end) {End};

        \draw[gtu arrow] (start) -- (init);
        \draw[gtu arrow] (init) -- (cond);
        \draw[gtu arrow] (cond) -- node[right] {હા} (body);
        \draw[gtu arrow] (body.west) -- ++(-0.5,0) |- (cond.west);
        \draw[gtu arrow] (cond) -- node[above] {ના} (output);
        \draw[gtu arrow] (output) -- (end);
    \end{tikzpicture}
    \end{center}

    \begin{mnemonicbox}"SILA - Sum Increment Loop Add"\end{mnemonicbox}
\end{solutionbox}

\orquestionmarks{3}{c}{7}
\textbf{સ્વિચ સ્ટેટમેન્ટ ફ્લોચાર્ટ દોરી ઉદાહરણ સાથે સમજાવો.}

\begin{solutionbox}
    \textbf{જવાબ}:

    સ્વિચ સ્ટેટમેન્ટ વેરિએબલની કિંમતના આધારે બહુવિધ વિકલ્પોમાંથી એક કોડ બ્લોક પસંદ કરે છે.

    \begin{center}
    \begin{tikzpicture}[gtu flow]
        \node[gtu start] (start) {Start};
        \node[gtu input, below=of start] (input) {Input choice};
        \node[gtu decision, below=of input] (switch) {switch choice};
        
        \node[gtu process, below left=of switch, xshift=-2cm] (c1) {case 1: Add};
        \node[gtu process, below left=of switch] (c2) {case 2: Sub};
        \node[gtu process, below right=of switch] (c3) {case 3: Mul};
        \node[gtu process, below right=of switch, xshift=2cm] (c4) {case 4: Div};
        \node[gtu process, below=of switch, yshift=-3cm] (def) {default: Invalid};
        
        \node[gtu stop, below=of def] (end) {End};

        \draw[gtu arrow] (start) -- (input);
        \draw[gtu arrow] (input) -- (switch);
        \draw[gtu arrow] (switch) -- (c1);
        \draw[gtu arrow] (switch) -- (c2);
        \draw[gtu arrow] (switch) -- (c3);
        \draw[gtu arrow] (switch) -- (c4);
        \draw[gtu arrow] (switch) -- (def);
        
        \draw[gtu arrow] (c1) |- (end);
        \draw[gtu arrow] (c2) |- (end);
        \draw[gtu arrow] (c3) |- (end);
        \draw[gtu arrow] (c4) |- (end);
        \draw[gtu arrow] (def) -- (end);
    \end{tikzpicture}
    \end{center}

\begin{lstlisting}[language=C]
#include <stdio.h>

int main() {
    int choice;
    
    printf("Enter operation (1-4): ");
    scanf("%d", &choice);
    
    switch(choice) {
        case 1:
            printf("Addition selected");
            break;
        case 2:
            printf("Subtraction selected");
            break;
        case 3:
            printf("Multiplication selected");
            break;
        case 4:
            printf("Division selected");
            break;
        default:
            printf("Invalid choice");
    }
    
    return 0;
}
\end{lstlisting}

    \begin{itemize}
        \item \textbf{Expression}: પૂર્ણાંક અથવા અક્ષર expression લે છે
        \item \textbf{Case labels}: અચળ expressions હોવા જોઈએ
        \item \textbf{Break statement}: આગલા કેસમાં પડતા અટકાવે છે
        \item \textbf{Default}: કોઈપણ કેસ સાથે મેચ ન થતી કિંમતો સંભાળે છે
    \end{itemize}

    \begin{mnemonicbox}"SCBD - Switch Cases Break Default"\end{mnemonicbox}
\end{solutionbox}

\questionmarks{4}{a}{3}
\textbf{Develop a C program to convert uppercase alphabet to lowercase alphabet.}

\begin{solutionbox}
    \textbf{Answer}:

\begin{lstlisting}[language=C]
#include <stdio.h>

int main() {
    char upper, lower;
    
    printf("Enter uppercase letter: ");
    scanf("%c", &upper);
    
    lower = upper + 32;
    // Alternatively: lower = tolower(upper);
    
    printf("Lowercase letter: %c", lower);
    
    return 0;
}
\end{lstlisting}

    \textbf{Diagram:}

    \begin{center}
    \begin{tikzpicture}[gtu flow]
        \node[gtu input] (input) {Input 'A'};
        \node[gtu block, below=of input] (ascii1) {ASCII code\\65};
        \node[gtu block, below=of ascii1] (ascii2) {ASCII code\\97};
        \node[gtu output, below=of ascii2] (output) {Output 'a'};

        \draw[gtu arrow] (input) -- (ascii1);
        \draw[gtu arrow] (ascii1) -- node[right] {+32} (ascii2);
        \draw[gtu arrow] (ascii2) -- (output);
    \end{tikzpicture}
    \end{center}

    \begin{mnemonicbox}"ASCII-32 - Add 32 to Shift Characters Into Lowercase"\end{mnemonicbox}
\end{solutionbox}

\questionmarks{4}{b}{4}
\textbf{What is pointer? Explain with example.}

\begin{solutionbox}
    \textbf{Answer}:

    પોઇન્ટર એ એક વેરિએબલ છે જે બીજા વેરિએબલનું મેમરી એડ્રેસ સ્ટોર કરે છે.

    \begin{tabulary}{\linewidth}{|L|L|L|}
        \hline
        \textbf{Concept} & \textbf{Syntax} & \textbf{Description} \\
        \hline
        Declaration & \code{int *p;} & int પોઇન્ટર p ડિક્લેર કરે છે \\
        \hline
        Initialization & \code{p = \&var;} & var નું એડ્રેસ p માં સ્ટોર કરે છે \\
        \hline
        Dereferencing & \code{*p = 10;} & પોઇન્ટેડ વેલ્યુ ને એક્સેસ/મોડિફાય કરે છે \\
        \hline
        Pointer arithmetic & \code{p++} & પછીના મેમરી લોકેશન પર જાય છે \\
        \hline
    \end{tabulary}

\begin{lstlisting}[language=C]
#include <stdio.h>

int main() {
    int num = 10;
    int *ptr;
    
    ptr = &num;  // Store address of num in ptr
    
    printf("Value of num: %d\n", num);
    printf("Address of num: %p\n", &num);
    printf("Value of ptr: %p\n", ptr);
    printf("Value pointed by ptr: %d\n", *ptr);
    
    *ptr = 20;  // Change value using pointer
    printf("New value of num: %d\n", num);
    
    return 0;
}
\end{lstlisting}

    \textbf{Memory Diagram:}

    \begin{center}
    \begin{tikzpicture}[gtu flow]
        \node[gtu block, minimum width=2.5cm] (num) {\textbf{num}\\Value: 10\\Addr: 1000};
        \node[gtu block, minimum width=2.5cm, right=of num, xshift=2cm] (ptr) {\textbf{ptr}\\Value: 1000\\Addr: 2000};
        
        \draw[gtu arrow] (ptr.west) -- node[above] {Points to} (num.east);
    \end{tikzpicture}
    \end{center}

    \begin{mnemonicbox}"SAID - Store Address to Indirectly Dereference"\end{mnemonicbox}
\end{solutionbox}

\questionmarks{4}{c}{7}
\textbf{Draw flowchart and explain for loop with example.}

\begin{solutionbox}
    \textbf{Answer}:

    for loop નો ઉપયોગ કોડ બ્લોકને ચોક્કસ સંખ્યામાં પુનરાવર્તિત કરવા માટે થાય છે.

    \begin{center}
    \begin{tikzpicture}[gtu flow]
        \node[gtu start] (start) {Start};
        \node[gtu process, right=of start] (init) {Initialization\\i=1};
        \node[gtu decision, right=of init] (cond) {Condition\\i $\le$ 5?};
        \node[gtu process, below=of cond] (body) {Body\\Print i};
        \node[gtu process, left=of body] (update) {Update\\i++};
        \node[gtu stop, right=of cond, xshift=2cm] (end) {End};

        \draw[gtu arrow] (start) -- (init);
        \draw[gtu arrow] (init) -- (cond);
        \draw[gtu arrow] (cond) -- node[right] {True} (body);
        \draw[gtu arrow] (body) -- (update);
        \draw[gtu arrow] (update) |- (cond);
        \draw[gtu arrow] (cond) -- node[above] {False} (end);
    \end{tikzpicture}
    \end{center}

\begin{lstlisting}[language=C]
#include <stdio.h>

int main() {
    int i;
    
    // Syntax: for(initialization; condition; update)
    for(i = 1; i <= 5; i++) {
        printf("%d ", i);
    }
    // Output: 1 2 3 4 5
    
    return 0;
}
\end{lstlisting}

    \begin{itemize}
        \item \textbf{Initialization}: લૂપ શરૂ થતા પહેલા એકવાર એક્ઝિક્યુટ થાય છે
        \item \textbf{Condition}: દરેક પુનરાવર્તન પહેલાં ચેક થાય છે
        \item \textbf{Update}: દરેક પુનરાવર્તન પછી એક્ઝિક્યુટ થાય છે
        \item \textbf{Body}: કોડ બ્લોક જે પુનરાવર્તિત થાય છે
    \end{itemize}

    \begin{mnemonicbox}"ICU-B - Initialize, Check, Update, Body"\end{mnemonicbox}
\end{solutionbox}

\orquestionmarks{4}{a}{3}
\textbf{Develop a C program to find area of a triangle (0.5 * base * height)?}

\begin{solutionbox}
    \textbf{Answer}:

\begin{lstlisting}[language=C]
#include <stdio.h>

int main() {
    float base, height, area;
    
    printf("Enter base of triangle: ");
    scanf("%f", &base);
    
    printf("Enter height of triangle: ");
    scanf("%f", &height);
    
    area = 0.5 * base * height;
    
    printf("Area of triangle: %.2f", area);
    
    return 0;
}
\end{lstlisting}

    \textbf{Diagram:}

    \begin{center}
    \begin{tikzpicture}[gtu flow]
        \node[gtu start] (start) {Start};
        \node[gtu input, right=of start] (input) {Input base, height};
        \node[gtu process, right=of input] (calc) {area = $0.5 \times base \times height$};
        \node[gtu output, right=of calc] (output) {Output area};
        \node[gtu stop, right=of output] (end) {End};

        \draw[gtu arrow] (start) -- (input);
        \draw[gtu arrow] (input) -- (calc);
        \draw[gtu arrow] (calc) -- (output);
        \draw[gtu arrow] (output) -- (end);
    \end{tikzpicture}
    \end{center}

    \begin{mnemonicbox}"BHA - Base times Height divided by two equals Area"\end{mnemonicbox}
\end{solutionbox}

\orquestionmarks{4}{b}{4}
\textbf{Explain declaration and initialization of pointer.}

\begin{solutionbox}
    \textbf{Answer}:

    \begin{tabulary}{\linewidth}{|L|L|L|L|}
        \hline
        \textbf{Operation} & \textbf{Syntax} & \textbf{Example} & \textbf{Description} \\
        \hline
        Declaration & \code{datatype *name;} & \code{int *ptr;} & પોઇન્ટર વેરિએબલ બનાવે છે \\
        \hline
        Initialization & \code{name = \&var;} & \code{ptr = \&num;} & પોઇન્ટરને એડ્રેસ અસાઇન કરે છે \\
        \hline
        Combined & \code{datatype *name = \&var;} & \code{int *ptr = \&num;} & ઇનિશિયલાઇઝેશન સાથે ડેક્લેરેશન \\
        \hline
        NULL pointer & \code{name = NULL;} & \code{ptr = NULL;} & સેફ ઇનિશિયલાઇઝેશન \\
        \hline
    \end{tabulary}

\begin{lstlisting}[language=C]
#include <stdio.h>

int main() {
    int num = 10;           // Regular variable
    int *ptr1;              // Declaration only
    int *ptr2 = &num;       // Declaration with initialization
    
    ptr1 = &num;            // Initialization of ptr1
    
    printf("num value: %d\n", num);
    printf("num address: %p\n", &num);
    printf("ptr1 value: %p\n", ptr1);
    printf("ptr2 value: %p\n", ptr2);
    
    return 0;
}
\end{lstlisting}

    \textbf{Diagram:}

    \begin{center}
    \begin{tikzpicture}[gtu flow]
        \node[gtu block] (decl) {\textbf{Declaration} \\ \code{int *ptr;}};
        \node[gtu block, right=of decl, xshift=2cm] (init) {\textbf{Initialization} \\ \code{ptr = \&num;}};
        \node[gtu block, right=of init] (var) {\textbf{Variable} \\ \code{num (10)}};
        
        \draw[gtu arrow] (decl) -- (init);
        \draw[gtu arrow] (init) -- node[above] {points to} (var);
    \end{tikzpicture}
    \end{center}

    \begin{mnemonicbox}"PAIN - Pointer Allocate, Initialize, Navigate"\end{mnemonicbox}
\end{solutionbox}

\orquestionmarks{4}{c}{7}
\textbf{Draw flowchart and explain while loop with example.}

\begin{solutionbox}
    \textbf{Answer}:

    જો શરત સાચી હોય ત્યાં સુધી while loop કોડ બ્લોકનું પુનરાવર્તન કરે છે.

    \begin{center}
    \begin{tikzpicture}[gtu flow]
        \node[gtu start] (start) {Start};
        \node[gtu process, right=of start] (init) {i = 1};
        \node[gtu decision, right=of init] (cond) {i $\le$ 5?};
        \node[gtu process, below=of cond] (body) {Print i\\i++};
        \node[gtu stop, right=of cond, xshift=2cm] (end) {End};

        \draw[gtu arrow] (start) -- (init);
        \draw[gtu arrow] (init) -- (cond);
        \draw[gtu arrow] (cond) -- node[right] {True} (body);
        \draw[gtu arrow] (body.west) -- ++(-0.5,0) |- (cond.west);
        \draw[gtu arrow] (cond) -- node[above] {False} (end);
    \end{tikzpicture}
    \end{center}

\begin{lstlisting}[language=C]
#include <stdio.h>

int main() {
    int i = 1;
    
    // Syntax: while(condition)
    while(i <= 5) {
        printf("%d ", i);
        i++;
    }
    // Output: 1 2 3 4 5
    
    return 0;
}
\end{lstlisting}

    \begin{itemize}
        \item \textbf{Entry controlled}: એક્ઝિક્યુશન પહેલાં કન્ડિશન ચેક થાય છે
        \item \textbf{Infinite loop}: જો કન્ડિશન ક્યારેય ખોટી ન થાય
        \item \textbf{Components}: ઇનિશિયલાઇઝેશન, કન્ડિશન, બોડી, અપડેટ
    \end{itemize}

    \begin{mnemonicbox}"WET - While Entry Test"\end{mnemonicbox}
\end{solutionbox}

\questionmarks{5}{a}{3}
\textbf{બુકની આપેલ માહિતી સ્ટોર કરવાનું સ્ટ્રક્ચર બનાવો: book\_no, book\_title, book\_author, book\_price}

\begin{solutionbox}
    \textbf{Answer}:

\begin{lstlisting}[language=C]
#include <stdio.h>
#include <string.h>

struct Book {
    int book_no;
    char book_title[50];
    char book_author[50];
    float book_price;
};

int main() {
    struct Book book1;
    
    book1.book_no = 101;
    strcpy(book1.book_title, "Programming in C");
    strcpy(book1.book_author, "Dennis Ritchie");
    book1.book_price = 450.50;
    
    printf("Book No: %d\n", book1.book_no);
    printf("Title: %s\n", book1.book_title);
    printf("Author: %s\n", book1.book_author);
    printf("Price: %.2f", book1.book_price);
    
    return 0;
}
\end{lstlisting}

    \textbf{આકૃતિ:}

    \begin{center}
    \begin{tikzpicture}[gtu flow]
        \node[gtu block, minimum width=4cm] (struct) {
            \textbf{struct Book} \\
            \rule{3.8cm}{0.4pt} \\
            book\_no (int) \\
            book\_title (char[50]) \\
            book\_author (char[50]) \\
            book\_price (float)
        };
    \end{tikzpicture}
    \end{center}

    \begin{mnemonicbox}"SNAP - Structure Needs All Properties"\end{mnemonicbox}
\end{solutionbox}

\questionmarks{5}{b}{4}
\textbf{(1)sqrt() (2)pow() (3)strlen() (4)strcpy() ફંક્શનો ઉદાહરણ સાથે સમજાવો.}

\begin{solutionbox}
    \textbf{Answer}:

    \begin{tabulary}{\linewidth}{|L|L|L|L|}
        \hline
        \textbf{ફંક્શન} & \textbf{લાઇબ્રેરી} & \textbf{હેતુ} & \textbf{ઉદાહરણ} \\
        \hline
        sqrt() & math.h & વર્ગમૂળ ગણે છે & \code{sqrt(16)} $\to$ \code{4.0} \\
        \hline
        pow() & math.h & પાવર ગણે છે & \code{pow(2, 3)} $\to$ \code{8.0} \\
        \hline
        strlen() & string.h & સ્ટ્રિંગની લંબાઈ શોધે છે & \code{strlen("Hi")} $\to$ \code{2} \\
        \hline
        strcpy() & string.h & સ્ટ્રિંગની કોપી કરે છે & \code{strcpy(d, s)} \\
        \hline
    \end{tabulary}

\begin{lstlisting}[language=C]
#include <stdio.h>
#include <math.h>
#include <string.h>

int main() {
    double sqrtResult = sqrt(25);
    double powResult = pow(2, 4);
    char str[] = "Programming";
    char dest[20];
    int length = strlen(str);
    
    strcpy(dest, str);
    
    printf("sqrt(25) = %.2f\n", sqrtResult);
    printf("pow(2, 4) = %.2f\n", powResult);
    printf("Length of '%s' = %d\n", str, length);
    printf("Copied string: %s\n", dest);
    
    return 0;
}
\end{lstlisting}

    \begin{mnemonicbox}"SPSS - Square-root Power String-length String-copy"\end{mnemonicbox}
\end{solutionbox}

\questionmarks{5}{c}{7}
\textbf{એરે અને એરેનું ઈનીશ્યલાઈઝેશન ઉદાહરણ સાથે સમજાવો.}

\begin{solutionbox}
    \textbf{Answer}:

    એરે એ સમાન ડેટા એલિમેન્ટ્સનો સમૂહ છે જે સતત મેમરી લોકેશન પર સ્ટોર થાય છે.

    \begin{tabulary}{\linewidth}{|L|L|L|}
        \hline
        \textbf{પદ્ધતિ} & \textbf{સિન્ટેક્સ} & \textbf{ઉદાહરણ} \\
        \hline
        ડેક્લેરેશન & \code{type name[size];} & \code{int marks[5];} \\
        \hline
        ઇનિશિયલાઇઝેશન & \code{type name[] = \{v1,..\};} & \code{int a[]=\{1,2\};} \\
        \hline
        એલિમેન્ટ એક્સેસ & \code{name[index] = val;} & \code{marks[0]=90;} \\
        \hline
    \end{tabulary}

\begin{lstlisting}[language=C]
#include <stdio.h>

int main() {
    // એરે ડેક્લેરેશન અને ઇનિશિયલાઇઝેશન
    int numbers[5] = {10, 20, 30, 40, 50};
    
    // એરે એલિમેન્ટ્સ એક્સેસ કરવા
    printf("First element: %d\n", numbers[0]);
    
    numbers[1] = 25; // એરે એલિમેન્ટ બદલવી
    
    printf("Array elements: ");
    for(int i = 0; i < 5; i++) {
        printf("%d ", numbers[i]);
    }
    
    return 0;
}
\end{lstlisting}

    \textbf{આકૃતિ:}

    \begin{center}
    \begin{tikzpicture}[gtu flow]
        \matrix[nodes={draw, minimum size=8mm}, column sep=-\pgflinewidth] {
            \node {10}; & \node {20}; & \node {30}; & \node {40}; & \node {50}; \\
        };
        \foreach \i in {0,1,2,3,4}
            \node at (-1.7+\i*0.8, -0.6) {\scriptsize \i};
    \end{tikzpicture}
    \end{center}

    \begin{mnemonicbox}"CASED - Contiguous Arrangement of Similar Elements with Direct-access"\end{mnemonicbox}
\end{solutionbox}

\orquestionmarks{5}{a}{3}
\textbf{એરે અને સ્ટ્રક્ચર નો તફાવત લખો.}

\begin{solutionbox}
    \textbf{Answer}:

    \begin{tabulary}{\linewidth}{|L|L|L|}
        \hline
        \textbf{ફીચર} & \textbf{એરે} & \textbf{સ્ટ્રક્ચર} \\
        \hline
        ડેટા ટાઈપ્સ & માત્ર એક જ ડેટા ટાઈપ (સમાન) & અલગ અલગ ડેટા ટાઈપ્સ મંજૂર \\
        \hline
        એક્સેસ & ઇન્ડેક્સ દ્વારા: \code{arr[0]} & ડોટ ઓપરેટર દ્વારા: \code{emp.id} \\
        \hline
        મેમરી & સતત ફાળવણી & સતત ન પણ હોઈ શકે \\
        \hline
        હેતુ & સમાન આઇટમ્સનો સંગ્રહ & સંબંધિત આઇટમ્સનો સંગ્રહ \\
        \hline
    \end{tabulary}

    \textbf{આકૃતિ:}

    \begin{center}
    \begin{tikzpicture}[gtu flow]
        \node[gtu block] (array) {Array [1, 2, 3]};
        \node[gtu block, right=of array] (struct) {Struct \{id:1, name:'A'\}};
    \end{tikzpicture}
    \end{center}

    \begin{mnemonicbox}"HASDIP - Homogeneous vs. Assorted, Same vs. Different, Index vs. Point"\end{mnemonicbox}
\end{solutionbox}

\orquestionmarks{5}{b}{4}
\textbf{યુઝર ડીફાઈન ફંક્શન એટલે શું? ઉદાહરણ સાથે સમજાવો.}

\begin{solutionbox}
    \textbf{Answer}:

    યુઝર-ડિફાઇન્ડ ફંક્શન એ એક કોડ બ્લોક છે જે ચોક્કસ કાર્ય કરે છે.

    \begin{tabulary}{\linewidth}{|L|L|}
        \hline
        \textbf{કોમ્પોનન્ટ} & \textbf{વિગત} \\
        \hline
        રિટર્ન ટાઇપ & પરત કરવામાં આવતા ડેટાનો પ્રકાર (e.g., int) \\
        \hline
        ફંક્શન નામ & ઓળખકર્તા (e.g., sum) \\
        \hline
        પેરામીટર્સ & ઇનપુટ વેલ્યુઝ (e.g., int a, int b) \\
        \hline
        બોડી & એક્ઝિક્યુટ થતો કોડ \\
        \hline
    \end{tabulary}

\begin{lstlisting}[language=C]
#include <stdio.h>

// ફંક્શન ડેક્લેરેશન
int sum(int a, int b);

int main() {
    int res = sum(5, 10); // ફંક્શન કોલ
    printf("Sum = %d", res);
    return 0;
}

// ફંક્શન ડેફિનિશન
int sum(int a, int b) {
    return a + b;
}
\end{lstlisting}

    \textbf{આકૃતિ:}

    \begin{center}
    \begin{tikzpicture}[gtu flow]
        \node[gtu block] (main) {Main Function\\Call sum(5, 10)};
        \node[gtu block, right=of main, xshift=2cm] (func) {Function sum(a, b)\\return a+b};

        \draw[gtu arrow] (main.north east) to[bend left] node[above] {Arguments} (func.north west);
        \draw[gtu arrow] (func.south west) to[bend left] node[below] {Return} (main.south east);
    \end{tikzpicture}
    \end{center}

    \begin{mnemonicbox}"CRPB - Create, Return, Pass, Body"\end{mnemonicbox}
\end{solutionbox}

\orquestionmarks{5}{c}{7}
\textbf{એરેના ઘટકોનો સરવાળો અને સરેરાશ શોધવા માટેનો C પ્રોગ્રામ લખો.}

\begin{solutionbox}
    \textbf{Answer}:

\begin{lstlisting}[language=C]
#include <stdio.h>

int main() {
    int arr[100], n, i;
    int sum = 0;
    float avg;
    
    printf("Enter number of elements: ");
    scanf("%d", &n);
    
    printf("Enter %d elements:\n", n);
    for(i = 0; i < n; i++) {
        scanf("%d", &arr[i]);
        sum += arr[i];  // દરેક એલિમેન્ટ ઉમેરો
    }
    
    avg = (float)sum / n;  // સરેરાશ ગણતરી
    
    printf("Sum: %d\n", sum);
    printf("Avg: %.2f", avg);
    
    return 0;
}
\end{lstlisting}

    \textbf{આકૃતિ:}

    \begin{center}
    \begin{tikzpicture}[gtu flow]
        \node[gtu start] (start) {Start};
        \node[gtu input, right=of start] (inputn) {Input n};
        \node[gtu input, right=of inputn] (inputarr) {Input elements};
        \node[gtu process, below=of start] (init) {sum = 0, i = 0};
        \node[gtu decision, below=of inputn] (cond) {i < n?};
        \node[gtu process, below=of inputarr] (sum) {sum += arr[i]\\i++};
        \node[gtu process, below=of cond] (calc) {avg = sum / n};
        \node[gtu output, below=of sum] (output) {Output sum, avg};
        \node[gtu stop, below=of calc] (end) {End};

        \draw[gtu arrow] (start) -- (inputn);
        \draw[gtu arrow] (inputn) -- (inputarr);
        \draw[gtu arrow] (inputarr) |- (init);
        \draw[gtu arrow] (init) -- (cond);
        \draw[gtu arrow] (cond) -- node[above] {Yes} (sum);
        \draw[gtu arrow] (sum) |- (0,-5) -| (cond);
        \draw[gtu arrow] (cond) -- node[left] {No} (calc);
        \draw[gtu arrow] (calc) -- (output);
        \draw[gtu arrow] (output) -- (end);
    \end{tikzpicture}
    \end{center}

    \begin{mnemonicbox}"LISA - Loop, Increment, Sum, Average"\end{mnemonicbox}
\end{solutionbox}

\end{document}
