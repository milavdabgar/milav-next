\documentclass[10pt,a4paper]{article}

% content/resources/templates/preamble.tex
\usepackage[margin=0.6in]{geometry}
\author{Milav Dabgar}
\usepackage{amsmath,amssymb,amsthm}
\usepackage{booktabs}
\usepackage{multirow}
\usepackage{xcolor}
\usepackage{tcolorbox}
\tcbuselibrary{breakable,skins}
\usepackage[colorlinks=true,linkcolor=blue]{hyperref}
\usepackage{titlesec}
\usepackage{enumitem}
\usepackage{tikz}
\usepackage{pgfplots}
\usepackage{circuitikz}
\usepackage[version=4]{mhchem}
\usepackage{longtable}
\usepackage{array}
\usepackage{float}
\usepackage{caption}
\usepackage{listings}

\lstset{
  basicstyle=\small\ttfamily,
  breaklines=true,
  breakatwhitespace=false,
  postbreak=\mbox{\textcolor{red}{$\hookrightarrow$}\space},
  float=false,
  numbers=left,
  numberstyle=\tiny\color{gray},
  numbersep=10pt,
  xleftmargin=2em,
  keywordstyle=\color{blue},
  commentstyle=\color{green!60!black},
  stringstyle=\color{purple},
  backgroundcolor=\color{gray!5},
  showstringspaces=false,
  tabsize=2,
  captionpos=b,
  keepspaces=true,
  columns=flexible
}

\pgfplotsset{compat=1.18}
\usetikzlibrary{shapes,arrows,positioning,calc,patterns,decorations.pathmorphing,decorations.markings,arrows.meta}

% Color scheme
\definecolor{headcolor}{RGB}{0,102,204}
\definecolor{keycolor}{RGB}{220,20,60}
\definecolor{solutioncolor}{RGB}{34,139,34}
\definecolor{mnemoniccolor}{RGB}{148,0,211}
\definecolor{codecolor}{RGB}{0,0,100}

% Spacing
\setlength{\parskip}{3pt}
\setlist[itemize]{nosep}
\setlist[enumerate]{nosep}

% Title formatting
\titleformat{\section}{\Large\bfseries\color{headcolor}}{\thesection}{1em}{}
\titleformat{\subsection}{\large\bfseries\color{headcolor}}{\thesubsection}{1em}{}

% Pandoc tightlist compatibility
\providecommand{\tightlist}{%
  \setlength{\itemsep}{0pt}\setlength{\parskip}{0pt}}

% Pandoc longtable compatibility
\newcounter{none}
\def\thenone{}


% content/resources/templates/english-boxes.tex
% This file is currently empty - it exists to maintain consistency with the import structure.
% Add custom environments here if needed in the future.


\begin{document}

\begin{center}
{\Huge\bfseries\color{headcolor} Subject Name Solutions}\\[5pt]
{\LARGE 4331105 -- Winter 2023}\\[3pt]
{\large Semester 1 Study Material}\\[3pt]
{\normalsize\textit{Detailed Solutions and Explanations}}
\end{center}

\vspace{10pt}

\subsection*{Question 1(a) [3 marks]}\label{q1a}

\textbf{Define algorithm and write an algorithm to find area of circle.}

\begin{solutionbox}
An algorithm is a step-by-step procedure or set of
rules for solving a specific problem or accomplishing a particular task.

\textbf{Algorithm to find area of circle:}

\begin{verbatim}
Step 1: Start
Step 2: Input radius (r) of the circle
Step 3: Calculate area = π \times r^{2}
Step 4: Display the area
Step 5: Stop
\end{verbatim}

\end{solutionbox}
\begin{mnemonicbox}
``Start, Read, Calculate, Display, Stop''

\end{mnemonicbox}
\subsection*{Question 1(b) [4 marks]}\label{q1b}

\textbf{Define flowchart and draw a flowchart to find minimum of three
numbers.}

\begin{solutionbox}
A flowchart is a visual representation of an algorithm
using standardized symbols and shapes connected by arrows to show the
sequence of steps.

\textbf{Flowchart to find minimum of three numbers:}

\begin{verbatim}
flowchart LR
    A([Start]) {-{-} B[/Input three numbers A, B, C/]}
    B {-{-} C\{Is A  B?\}}
    C {-{-}|Yes| D\{Is A  C?\}}
    C {-{-}|No| E\{Is B  C?\}}
    D {-{-}|Yes| F[min = A]}
    D {-{-}|No| G[min = C]}
    E {-{-}|Yes| H[min = B]}
    E {-{-}|No| I[min = C]}
    F {-{-} J[/Display min/]}
    G {-{-} J}
    H {-{-} J}
    I {-{-} J}
    J {-{-} K([Stop])}
\end{verbatim}

\begin{itemize}
\tightlist
\item
  \textbf{Comparison Strategy}: First compare A and B, then compare with
  C
\item
  \textbf{Branching Logic}: Use if-else structure to find smallest value
\end{itemize}

\end{solutionbox}
\begin{mnemonicbox}
``Compare pairs, find the rare small value
everywhere''

\end{mnemonicbox}
\subsection*{Question 1(c) [7 marks]}\label{q1c}

\textbf{Write a program to calculate simple interest using below
equation.

I=PRN/100 Where

P=Principle amount,

R=Rate of interest and

N=Period.}

\begin{solutionbox}

\begin{verbatim}
\#include {stdio.h}

int main() \{
    float P, R, N, I;
    
    // Input principal amount, rate of interest and time period
    printf("Enter Principal amount: ");
    scanf("\%f", \&P);
    
    printf("Enter Rate of interest: ");
    scanf("\%f", \&R);
    
    printf("Enter Time period (in years): ");
    scanf("\%f", \&N);
    
    // Calculate Simple Interest
    I = (P * R * N) / 100;
    
    // Display the result
    printf("Simple Interest = \%.2f{n}", I);
    
    return 0;
\}
\end{verbatim}

\textbf{Diagram:}

\begin{verbatim}
flowchart LR
    P["Principal (P)"] {-{-} Formula["I = (P  R  N) / 100"]}
    R["Rate (R)"] {-{-} Formula}
    N["Period (N)"] {-{-} Formula}
    Formula {-{-} Interest["Interest (I)"]}
\end{verbatim}

\begin{itemize}
\tightlist
\item
  \textbf{Floating-point variables}: Store decimal values for precision
\item
  \textbf{User interaction}: Clear prompts for input
\item
  \textbf{Result formatting}: \%.2f displays two decimal places
\end{itemize}

\end{solutionbox}
\begin{mnemonicbox}
``Principal, Rate and Number, divided by Hundred
gives Interest''

\end{mnemonicbox}
\subsection*{Question 1(c OR) [7
marks]}\label{question-1c-or-7-marks}

\textbf{Write a program to read radius(R) and height(H) from keyboard
and print calculated the volume(V) of cylinder using V=πR^{2}H}

\begin{solutionbox}

\begin{verbatim}
\#include {stdio.h}

int main() \{
    float radius, height, volume;
    const float PI = 3.14159;
    
    // Input radius and height
    printf("Enter radius of cylinder: ");
    scanf("\%f", \&radius);
    
    printf("Enter height of cylinder: ");
    scanf("\%f", \&height);
    
    // Calculate volume of cylinder
    volume = PI * radius * radius * height;
    
    // Display the result
    printf("Volume of cylinder = \%.2f{n}", volume);
    
    return 0;
\}
\end{verbatim}

\textbf{Diagram:}

\begin{verbatim}
flowchart LR
    A[/Input radius, height/] {-{-} B["Calculate volume = π  radius^{2}  height"]}
    B {-{-} C[/Display volume/]}
\end{verbatim}

\begin{itemize}
\tightlist
\item
  \textbf{Constants}: PI defined as constant for clarity
\item
  \textbf{Formula}: Use R^{2} by multiplying radius twice
\item
  \textbf{Input validation}: Assumes positive values for radius and
  height
\end{itemize}

\end{solutionbox}
\begin{mnemonicbox}
``Radius squared times height times Pi, gives
cylinder volume, don't ask why''

\end{mnemonicbox}
\subsection*{Question 2(a) [3 marks]}\label{q2a}

\textbf{List out different operators supported in C programming
language.}

\begin{solutionbox}

{\def\LTcaptype{none} % do not increment counter
\begin{longtable}[]{@{}
  >{\raggedright\arraybackslash}p{(\linewidth - 2\tabcolsep) * \real{0.4762}}
  >{\raggedright\arraybackslash}p{(\linewidth - 2\tabcolsep) * \real{0.5238}}@{}}
\toprule\noalign{}
\begin{minipage}[b]{\linewidth}\raggedright
Category
\end{minipage} & \begin{minipage}[b]{\linewidth}\raggedright
Operators
\end{minipage} \\
\midrule\noalign{}
\endhead
\bottomrule\noalign{}
\endlastfoot
Arithmetic & +, -, *, /, \% (addition, subtraction, multiplication,
division, modulus) \\
Relational & ==, !=, \textgreater, \textless, \textgreater=, \textless=
(equal, not equal, greater than, less than, greater than or equal to,
less than or equal to) \\
Logical & \&\&, \textbar\textbar, ! (AND, OR, NOT) \\
Assignment & =, +=, -=, *=, /=, \%= (assign, plus-assign, minus-assign,
etc.) \\
Increment/Decrement & ++, -- (increment, decrement) \\
Bitwise & \&, \textbar, \^{}, \textasciitilde, \textless\textless,
\textgreater\textgreater{} (AND, OR, XOR, complement, left shift, right
shift) \\
Conditional & ? : (ternary operator) \\
Special & sizeof(), \&, *, -\textgreater, . (size, address, pointer,
structure) \\
\end{longtable}
}

\end{solutionbox}
\begin{mnemonicbox}
``ARABIA CS'' (Arithmetic, Relational, Assignment,
Bitwise, Increment, Assignment, Conditional, Special)

\end{mnemonicbox}
\subsection*{Question 2(b) [4 marks]}\label{q2b}

\textbf{Explain Relational operator and Increment/Decrement operator
with example.}

\begin{solutionbox}

{\def\LTcaptype{none} % do not increment counter
\begin{longtable}[]{@{}
  >{\raggedright\arraybackslash}p{(\linewidth - 6\tabcolsep) * \real{0.3333}}
  >{\raggedright\arraybackslash}p{(\linewidth - 6\tabcolsep) * \real{0.2889}}
  >{\raggedright\arraybackslash}p{(\linewidth - 6\tabcolsep) * \real{0.2000}}
  >{\raggedright\arraybackslash}p{(\linewidth - 6\tabcolsep) * \real{0.1778}}@{}}
\toprule\noalign{}
\begin{minipage}[b]{\linewidth}\raggedright
Operator Type
\end{minipage} & \begin{minipage}[b]{\linewidth}\raggedright
Description
\end{minipage} & \begin{minipage}[b]{\linewidth}\raggedright
Example
\end{minipage} & \begin{minipage}[b]{\linewidth}\raggedright
Output
\end{minipage} \\
\midrule\noalign{}
\endhead
\bottomrule\noalign{}
\endlastfoot
Relational & Compare two values to test the relationship between them &
\texttt{int\ a\ =\ 5,\ b\ =\ 10;}\texttt{printf("\%d",\ a\ \textless{}\ b);}
& \texttt{1} (true) \\
& Equal to (==) & \texttt{printf("\%d",\ 5\ ==\ 5);} & \texttt{1}
(true) \\
& Not equal to (!=) & \texttt{printf("\%d",\ 5\ !=\ 10);} & \texttt{1}
(true) \\
& Greater/Less than &
\texttt{printf("\%d\ \%d",\ 5\ \textgreater{}\ 3,\ 5\ \textless{}\ 3);}
& \texttt{1\ 0} \\
Increment & Increases value by 1Pre-increment (++x): increment then
usePost-increment (x++): use then increment &
\texttt{int\ x\ =\ 5;}\texttt{printf("\%d\ ",\ ++x);}\texttt{printf("\%d",\ x);}
& \texttt{6\ 6} \\
Decrement & Decreases value by 1Pre-decrement (--x): decrement then
usePost-decrement (x--): use then decrement &
\texttt{int\ y\ =\ 5;}\texttt{printf("\%d\ ",\ y-\/-);}\texttt{printf("\%d",\ y);}
& \texttt{5\ 4} \\
\end{longtable}
}

\begin{itemize}
\tightlist
\item
  \textbf{Relational operators}: Return 1 (true) or 0 (false)
\item
  \textbf{Increment/Decrement}: Changes variable value and returns a
  value
\end{itemize}

\end{solutionbox}
\begin{mnemonicbox}
``Relational tells if TRUE or LIE,
Increment/Decrement makes values rise or DIE''

\end{mnemonicbox}
\subsection*{Question 2(c) [7 marks]}\label{q2c}

\textbf{Write a program to print sum and average of 1 to 100.}

\begin{solutionbox}

\begin{verbatim}
\#include {stdio.h}

int main() \{
    int i, sum = 0;
    float average;
    
    // Calculate sum of numbers from 1 to 100
    for(i = 1; i {=} 100; i++) \{
        sum += i;
    \}
    
    // Calculate average
    average = (float)sum / 100;
    
    // Display the results
    printf("Sum of numbers from 1 to 100 = \%d{n}", sum);
    printf("Average of numbers from 1 to 100 = \%.2f{n}", average);
    
    return 0;
\}
\end{verbatim}

\textbf{Diagram:}

\begin{verbatim}
flowchart LR
    A([Start]) {-{-} B[Initialize sum = 0]}
    B {-{-} C[Set i = 1]}
    C {-{-} D\{Is i = 100?\}}
    D {-{-}|Yes| E[sum = sum + i]}
    E {-{-} F[i = i + 1]}
    F {-{-} D}
    D {-{-}|No| G[Calculate average = sum / 100]}
    G {-{-} H[Display sum and average]}
    H {-{-} I([Stop])}
\end{verbatim}

\begin{itemize}
\tightlist
\item
  \textbf{Loop counter}: Variable i tracks numbers 1 to 100
\item
  \textbf{Sum calculation}: Accumulates values in sum variable
\item
  \textbf{Type casting}: (float) converts sum to floating-point for
  accurate division
\end{itemize}

\end{solutionbox}
\begin{mnemonicbox}
``Sum One to Hundred, then Divide for Average''

\end{mnemonicbox}
\subsection*{Question 2(a OR) [3
marks]}\label{question-2a-or-3-marks}

\textbf{State the difference between gets(S) and scanf(``\%s'',S) where
S is string.}

\begin{solutionbox}

{\def\LTcaptype{none} % do not increment counter
\begin{longtable}[]{@{}
  >{\raggedright\arraybackslash}p{(\linewidth - 4\tabcolsep) * \real{0.2727}}
  >{\raggedright\arraybackslash}p{(\linewidth - 4\tabcolsep) * \real{0.2727}}
  >{\raggedright\arraybackslash}p{(\linewidth - 4\tabcolsep) * \real{0.4545}}@{}}
\toprule\noalign{}
\begin{minipage}[b]{\linewidth}\raggedright
Feature
\end{minipage} & \begin{minipage}[b]{\linewidth}\raggedright
gets(S)
\end{minipage} & \begin{minipage}[b]{\linewidth}\raggedright
scanf(``\%s'',S)
\end{minipage} \\
\midrule\noalign{}
\endhead
\bottomrule\noalign{}
\endlastfoot
Input termination & Reads until newline character (\n) & Reads until
whitespace (space, tab, newline) \\
Whitespace handling & Can read string with spaces & Stops reading at
first whitespace \\
Buffer overflow & No bounds checking (unsafe) & No bounds checking
(unsafe) \\
Return value & Returns S on success, NULL on error & Returns number of
items successfully read \\
Replacement & fgets() is safer alternative & scanf(``\%ns'',S) with
width limit is safer \\
\end{longtable}
}

\begin{itemize}
\tightlist
\item
  \textbf{Safety concern}: Both functions can cause buffer overflow
\item
  \textbf{Practical usage}: gets() for full lines, scanf() for single
  words
\end{itemize}

\end{solutionbox}
\begin{mnemonicbox}
``gets Gets Everything Till newline, scanf Stops
Catching After Finding whitespace''

\end{mnemonicbox}
\subsection*{Question 2(b OR) [4
marks]}\label{question-2b-or-4-marks}

\textbf{Explain Logical operator and Assignment operator with example.}

\begin{solutionbox}

{\def\LTcaptype{none} % do not increment counter
\begin{longtable}[]{@{}
  >{\raggedright\arraybackslash}p{(\linewidth - 6\tabcolsep) * \real{0.3333}}
  >{\raggedright\arraybackslash}p{(\linewidth - 6\tabcolsep) * \real{0.2889}}
  >{\raggedright\arraybackslash}p{(\linewidth - 6\tabcolsep) * \real{0.2000}}
  >{\raggedright\arraybackslash}p{(\linewidth - 6\tabcolsep) * \real{0.1778}}@{}}
\toprule\noalign{}
\begin{minipage}[b]{\linewidth}\raggedright
Operator Type
\end{minipage} & \begin{minipage}[b]{\linewidth}\raggedright
Description
\end{minipage} & \begin{minipage}[b]{\linewidth}\raggedright
Example
\end{minipage} & \begin{minipage}[b]{\linewidth}\raggedright
Output
\end{minipage} \\
\midrule\noalign{}
\endhead
\bottomrule\noalign{}
\endlastfoot
Logical & Perform logical operations on conditions &
\texttt{int\ a\ =\ 5,\ b\ =\ 10;} & \\
& Logical AND (\&\&) &
\texttt{printf("\%d",\ (a\ \textgreater{}\ 0)\ \&\&\ (b\ \textgreater{}\ 0));}
& \texttt{1} (true) \\
& Logical OR (\textbackslash{} & \textbackslash{} & ) \\
& Logical NOT (!) & \texttt{printf("\%d",\ !(a\ ==\ b));} & \texttt{1}
(true) \\
Assignment & Assign values to variables & \texttt{int\ x\ =\ 10;} &
\texttt{x\ =\ 10} \\
& Simple assignment (=) & \texttt{x\ =\ 20;} & \texttt{x\ =\ 20} \\
& Add and assign (+=) & \texttt{x\ +=\ 5;} & \texttt{x\ =\ 25} \\
& Subtract and assign (-=) & \texttt{x\ -=\ 10;} & \texttt{x\ =\ 15} \\
& Multiply and assign (*=) & \texttt{x\ *=\ 2;} & \texttt{x\ =\ 30} \\
& Divide and assign (/=) & \texttt{x\ /=\ 3;} & \texttt{x\ =\ 10} \\
\end{longtable}
}

\begin{itemize}
\tightlist
\item
  \textbf{Logical operators}: Used in decision making
\item
  \textbf{Short-circuit evaluation}: \&\& and \textbar\textbar{}
  evaluate only what's necessary
\item
  \textbf{Compound assignment}: Combines operation and assignment
\end{itemize}

\end{solutionbox}
\begin{mnemonicbox}
``AND needs all TRUE, OR needs just one; Assignment
takes right, puts it on the left throne''

\end{mnemonicbox}
\subsection*{Question 2(c OR) [7
marks]}\label{question-2c-or-7-marks}

\textbf{Write a program to print all the integers between given two
floating point numbers.}

\begin{solutionbox}

\begin{verbatim}
\#include {stdio.h}
\#include {math.h}

int main() \{
    float num1, num2;
    int start, end, i;
    
    // Input two floating point numbers
    printf("Enter first floating point number: ");
    scanf("\%f", \&num1);
    
    printf("Enter second floating point number: ");
    scanf("\%f", \&num2);
    
    // Find the ceil of smaller number and floor of larger number
    if(num1 {} num2) \{
        start = ceil(num1);
        end = floor(num2);
    \} else \{
        start = ceil(num2);
        end = floor(num1);
    \}
    
    // Print all integers between the two numbers
    printf("Integers between \%.2f and \%.2f are:{n}", num1, num2);
    for(i = start; i {=} end; i++) \{
        printf("\%d ", i);
    \}
    printf("{n}");
    
    return 0;
\}
\end{verbatim}

\textbf{Diagram:}

\begin{verbatim}
flowchart LR
    A[/num1, num2 ઇનપુટ લો/] {-{-} B\{શું num1  num2?\}}
    B {-{-}|હા| C["start = ceil(num1)br /end = floor(num2)"]}
    B {-{-}|ના| D["start = ceil(num2)br /end = floor(num1)"]}
    C {-{-} E[start થી end સુધીના પૂર્ણાંકો પ્રિન્ટ કરો]}
    D {-{-} E}
\end{verbatim}

\begin{itemize}
\tightlist
\item
  \textbf{Math functions}: ceil() rounds up, floor() rounds down
\item
  \textbf{Range determination}: Works regardless of input order
\item
  \textbf{Integer extraction}: Only prints whole numbers between floats
\end{itemize}

\end{solutionbox}
\begin{mnemonicbox}
``Ceiling the small, flooring the big, then print
every Integer in between''

\end{mnemonicbox}
\subsection*{Question 3(a) [3 marks]}\label{q3a}

\textbf{Explain multiple if-else statement with example.}

\begin{solutionbox}

Multiple if-else statements allow testing several conditions in
sequence, where each condition is checked only if the previous
conditions are false.

\begin{verbatim}
\#include {stdio.h}

int main() \{
    int marks;
    
    printf("Enter marks (0{-100): "});
    scanf("\%d", \&marks);
    
    if(marks {=} 80) \{
        printf("Grade: A{n}");
    \} else if(marks {=} 70) \{
        printf("Grade: B{n}");
    \} else if(marks {=} 60) \{
        printf("Grade: C{n}");
    \} else if(marks {=} 50) \{
        printf("Grade: D{n}");
    \} else \{
        printf("Grade: F{n}");
    \}
    
    return 0;
\}
\end{verbatim}

\textbf{Diagram:}

\begin{verbatim}
flowchart LR
    A[/Input marks/] {-{-} B\{marks = 80?\}}
    B {-{-}|Yes| C[Grade: A]}
    B {-{-}|No| D\{marks = 70?\}}
    D {-{-}|Yes| E[Grade: B]}
    D {-{-}|No| F\{marks = 60?\}}
    F {-{-}|Yes| G[Grade: C]}
    F {-{-}|No| H\{marks = 50?\}}
    H {-{-}|Yes| I[Grade: D]}
    H {-{-}|No| J[Grade: F]}
\end{verbatim}

\begin{itemize}
\tightlist
\item
  \textbf{Sequential testing}: Only one block executes
\item
  \textbf{Efficiency}: Stops checking after finding true condition
\end{itemize}

\end{solutionbox}
\begin{mnemonicbox}
``If this THEN that, ELSE IF another THEN something
else''

\end{mnemonicbox}
\subsection*{Question 3(b) [4 marks]}\label{q3b}

\textbf{State the working of while loop and for loop.}

\begin{solutionbox}

{\def\LTcaptype{none} % do not increment counter
\begin{longtable}[]{@{}
  >{\raggedright\arraybackslash}p{(\linewidth - 6\tabcolsep) * \real{0.2821}}
  >{\raggedright\arraybackslash}p{(\linewidth - 6\tabcolsep) * \real{0.2308}}
  >{\raggedright\arraybackslash}p{(\linewidth - 6\tabcolsep) * \real{0.2051}}
  >{\raggedright\arraybackslash}p{(\linewidth - 6\tabcolsep) * \real{0.2821}}@{}}
\toprule\noalign{}
\begin{minipage}[b]{\linewidth}\raggedright
Loop Type
\end{minipage} & \begin{minipage}[b]{\linewidth}\raggedright
Working
\end{minipage} & \begin{minipage}[b]{\linewidth}\raggedright
Syntax
\end{minipage} & \begin{minipage}[b]{\linewidth}\raggedright
Use Cases
\end{minipage} \\
\midrule\noalign{}
\endhead
\bottomrule\noalign{}
\endlastfoot
while loop & 1. Test condition2. If true, execute body3. Repeat steps
1-2 until condition is false &
\texttt{while(condition)\ \{}\texttt{//\ statements}\texttt{\}} & When
number of iterations is unknown beforehand \\
for loop & 1. Execute initialization once2. Test condition3. If true,
execute body4. Execute update statement5. Repeat steps 2-4 until
condition is false &
\texttt{for(initialization;\ condition;\ update)\ \{}\texttt{//\ statements}\texttt{\}}
& When number of iterations is known beforehand \\
\end{longtable}
}

\textbf{Comparison:}

\begin{verbatim}
flowchart TD
    subgraph "while loop"
    A1[Start] {-{-} B1\{Conditionbr /True?\}}
    B1 {-{-}|Yes| C1[Executebr /Body]}
    C1 {-{-} B1}
    B1 {-{-}|No| D1[End]}
    end

    subgraph "for loop"
    A2[Initialization] {-{-} B2\{Conditionbr /True?\}}
    B2 {-{-}|Yes| C2[Executebr /Body]}
    C2 {-{-} D2[Update]}
    D2 {-{-} B2}
    B2 {-{-}|No| E2[End]}
    end
\end{verbatim}

\begin{itemize}
\tightlist
\item
  \textbf{Entry control}: Both check condition before execution
\item
  \textbf{Components}: for loop combines initialization, condition, and
  update
\end{itemize}

\end{solutionbox}
\begin{mnemonicbox}
``WHILE checks THEN acts, FOR initializes CHECKS acts
UPDATES''

\end{mnemonicbox}
\subsection*{Question 3(c) [7 marks]}\label{q3c}

\textbf{Write a program to find factorial of a given number.}

\begin{solutionbox}

\begin{verbatim}
\#include {stdio.h}

int main() \{
    int num, i;
    unsigned long long factorial = 1;
    
    // Input a number
    printf("Enter a positive integer: ");
    scanf("\%d", \&num);
    
    // Check if the number is negative
    if(num {} 0) \{
        printf("Error: Factorial is not defined for negative numbers.{n}");
    \} else \{
        // Calculate factorial
        for(i = 1; i {=} num; i++) \{
            factorial *= i;
        \}
        
        printf("Factorial of \%d = \%llu{n}", num, factorial);
    \}
    
    return 0;
\}
\end{verbatim}

\textbf{Diagram:}

\begin{verbatim}
flowchart LR
    A([Start]) {-{-} B[/Input number/]}
    B {-{-} C\{Is number  0?\}}
    C {-{-}|Yes| D[/Display error message/]}
    C {-{-}|No| E[Initialize factorial = 1]}
    E {-{-} F[Set i = 1]}
    F {-{-} G\{Is i = number?\}}
    G {-{-}|Yes| H[factorial = factorial * i]}
    H {-{-} I[i = i + 1]}
    I {-{-} G}
    G {-{-}|No| J[/Display factorial/]}
    D {-{-} K([Stop])}
    J {-{-} K}
\end{verbatim}

\begin{itemize}
\tightlist
\item
  \textbf{Data type}: unsigned long long for large factorials
\item
  \textbf{Error handling}: Checks for negative input
\item
  \textbf{Loop implementation}: Multiply successive integers
\end{itemize}

\end{solutionbox}
\begin{mnemonicbox}
``Factorial Formula: Multiply From One to Number''

\end{mnemonicbox}
\subsection*{Question 3(a OR) [3
marks]}\label{question-3a-or-3-marks}

\textbf{Explain the working of switch-case statement with example.}

\begin{solutionbox}

The switch-case statement is a multi-way decision maker that tests the
value of an expression against various case values and executes the
matching case block.

\begin{verbatim}
\#include {stdio.h}

int main() \{
    int day;
    
    printf("Enter day number (1{-7): "});
    scanf("\%d", \&day);
    
    switch(day) \{
        case 1:
            printf("Monday{n}");
            break;
        case 2:
            printf("Tuesday{n}");
            break;
        case 3:
            printf("Wednesday{n}");
            break;
        case 4:
            printf("Thursday{n}");
            break;
        case 5:
            printf("Friday{n}");
            break;
        case 6:
            printf("Saturday{n}");
            break;
        case 7:
            printf("Sunday{n}");
            break;
        default:
            printf("Invalid day number{n}");
    \}
    
    return 0;
\}
\end{verbatim}

\textbf{Diagram:}

\begin{verbatim}
flowchart LR
    A[/Input day/] {-{-} B\{"switch(day)"\}}
    B {-{-} C1\{case 1\}}
    B {-{-} C2\{case 2\}}
    B {-{-} C3\{...\}}
    B {-{-} C4\{case 7\}}
    B {-{-} C5\{default\}}
    C1 {-{-}|match| D1[Print {-} Monday]}
    C2 {-{-}|match| D2[Print {-} Tuesday]}
    C3 {-{-}|match| D3[...]}
    C4 {-{-}|match| D4[Print {-} Sunday]}
    C5 {-{-}|no match| D5[Print {-} Invalid day]}
    D1 {-{-} E[break]}
    D2 {-{-} E}
    D3 {-{-} E}
    D4 {-{-} E}
    D5 {-{-} F([End])}
    E {-{-} F}
\end{verbatim}

\begin{itemize}
\tightlist
\item
  \textbf{Expression evaluation}: Only integer or character types
\item
  \textbf{Case matching}: Executes matching case until break
\item
  \textbf{Default case}: Executes when no case matches
\end{itemize}

\end{solutionbox}
\begin{mnemonicbox}
``SWITCH value, CASE match, BREAK out, DEFAULT
rescue''

\end{mnemonicbox}
\subsection*{Question 3(b OR) [4
marks]}\label{question-3b-or-4-marks}

\textbf{Define break and continue keyword.}

\begin{solutionbox}

{\def\LTcaptype{none} % do not increment counter
\begin{longtable}[]{@{}
  >{\raggedright\arraybackslash}p{(\linewidth - 6\tabcolsep) * \real{0.2308}}
  >{\raggedright\arraybackslash}p{(\linewidth - 6\tabcolsep) * \real{0.3077}}
  >{\raggedright\arraybackslash}p{(\linewidth - 6\tabcolsep) * \real{0.2308}}
  >{\raggedright\arraybackslash}p{(\linewidth - 6\tabcolsep) * \real{0.2308}}@{}}
\toprule\noalign{}
\begin{minipage}[b]{\linewidth}\raggedright
Keyword
\end{minipage} & \begin{minipage}[b]{\linewidth}\raggedright
Definition
\end{minipage} & \begin{minipage}[b]{\linewidth}\raggedright
Purpose
\end{minipage} & \begin{minipage}[b]{\linewidth}\raggedright
Example
\end{minipage} \\
\midrule\noalign{}
\endhead
\bottomrule\noalign{}
\endlastfoot
break & Terminates the innermost loop or switch statement immediately &
To exit a loop prematurely when a certain condition is met &
\texttt{c\ for(i=1;\ i\textless{}=10;\ i++)\ \{\ if(i\ ==\ 5)\ break;\ printf("\%d\ ",\ i);\ \}\ //\ Output:\ 1\ 2\ 3\ 4} \\
continue & Skips the rest of the current iteration and jumps to the next
iteration of the loop & To skip specific iterations without terminating
the loop &
\texttt{c\ for(i=1;\ i\textless{}=10;\ i++)\ \{\ if(i\ ==\ 5)\ continue;\ printf("\%d\ ",\ i);\ \}\ //\ Output:\ 1\ 2\ 3\ 4\ 6\ 7\ 8\ 9\ 10} \\
\end{longtable}
}

\textbf{Behavioral Comparison:}

\begin{verbatim}
flowchart TD
    subgraph "break"
    A1[Enter Loop] {-{-} B1\{Conditionbr /for break?\}}
    B1 {-{-}|Yes| C1[Exit Loop]}
    B1 {-{-}|No| D1[Continuebr /Execution]}
    D1 {-{-} E1[Nextbr /Iteration]}
    E1 {-{-} B1}
    end

    subgraph "continue"
    A2[Enter Loop] {-{-} B2\{Conditionbr /for continue?\}}
    B2 {-{-}|Yes| C2[Skip Restbr /of Loop Body]}
    B2 {-{-}|No| D2[Continuebr /Execution]}
    C2 {-{-} E2[Nextbr /Iteration]}
    D2 {-{-} E2}
    E2 {-{-} B2}
    end
\end{verbatim}

\begin{itemize}
\tightlist
\item
  \textbf{Scope}: Both affect only the innermost loop
\item
  \textbf{Control transfer}: break exits loop, continue jumps to next
  iteration
\end{itemize}

\end{solutionbox}
\begin{mnemonicbox}
``BREAK leaves the room, CONTINUE skips to the next
dance move''

\end{mnemonicbox}
\subsection*{Question 3(c OR) [7
marks]}\label{question-3c-or-7-marks}

\textbf{Write a program to read number of lines (n) from keyboard and
print the triangle shown below.}

\textbf{For Example, n=5}

\begin{verbatim}
1
1 2
1 2 3
1 2 3 4
1 2 3 4 5
\end{verbatim}

\begin{solutionbox}

\begin{verbatim}
\#include {stdio.h}

int main() \{
    int n, i, j;
    
    // Input number of lines
    printf("Enter number of lines: ");
    scanf("\%d", \&n);
    
    // Print the triangle pattern
    for(i = 1; i {=} n; i++) \{
        // Print numbers from 1 to i in each row
        for(j = 1; j {=} i; j++) \{
            printf("\%d ", j);
        \}
        printf("{n}");
    \}
    
    return 0;
\}
\end{verbatim}

\textbf{Pattern Visualization:}

\begin{verbatim}
Row 1: 1
Row 2: 1 2
Row 3: 1 2 3
Row 4: 1 2 3 4
Row 5: 1 2 3 4 5
\end{verbatim}

\textbf{Program Flow:}

\begin{verbatim}
flowchart LR
    A[/Input n/] {-{-} B[Set i = 1]}
    B {-{-} C\{Is i = n?\}}
    C {-{-}|Yes| D[Set j = 1]}
    D {-{-} E\{Is j = i?\}}
    E {-{-}|Yes| F[/Print j/]}
    F {-{-} G[j = j + 1]}
    G {-{-} E}
    E {-{-}|No| H[/Print newline/]}
    H {-{-} I[i = i + 1]}
    I {-{-} C}
    C {-{-}|No| J([Stop])}
\end{verbatim}

\begin{itemize}
\tightlist
\item
  \textbf{Nested loops}: Outer loop for rows, inner loop for columns
\item
  \textbf{Pattern logic}: Row number determines how many numbers to
  print
\item
  \textbf{Number sequence}: Each row prints 1 to row number
\end{itemize}

\end{solutionbox}
\begin{mnemonicbox}
``Rows decide COUNTer limit, COLumns print ONE to
ROW''

\end{mnemonicbox}
\subsection*{Question 4(a) [3 marks]}\label{q4a}

\textbf{Explain nested if-else statement with example.}

\begin{solutionbox}

Nested if-else statements are if-else constructs placed inside another
if or else block, allowing more complex conditional logic and multiple
levels of decision making.

\begin{verbatim}
\#include {stdio.h}

int main() \{
    int age;
    char hasID;
    
    printf("Enter age: ");
    scanf("\%d", \&age);
    
    printf("Do you have ID? (Y/N): ");
    scanf(" \%c", \&hasID);
    
    if(age {=} 18) \{
        if(hasID == {Y} || hasID == {y}) \{
            printf("You can vote!{n}");
        \} else \{
            printf("You need ID to vote.{n}");
        \}
    \} else \{
        printf("You must be 18 or older to vote.{n}");
    \}
    
    return 0;
\}
\end{verbatim}

\textbf{Decision Tree:}

\begin{verbatim}
flowchart LR
    A[/Input age and hasID/] {-{-} B\{age = 18?\}}
    B {-{-}|Yes| C\{hasID == Ybr /or y?\}}
    C {-{-}|Yes| D[You can vote!]}
    C {-{-}|No| E[You need ID to vote]}
    B {-{-}|No| F[You must be 18 or older to vote]}
\end{verbatim}

\begin{itemize}
\tightlist
\item
  \textbf{Hierarchical conditions}: Evaluates conditions in layers
\item
  \textbf{Indentation}: Improves readability of nested structures
\item
  \textbf{Multi-factor decisions}: Combines multiple criteria
\end{itemize}

\end{solutionbox}
\begin{mnemonicbox}
``If INSIDE if, check DEEPER conditions''

\end{mnemonicbox}
\subsection*{Question 4(b) [4 marks]}\label{q4b}

\textbf{Describe initialization of one-dimensional array.}

\begin{solutionbox}

{\def\LTcaptype{none} % do not increment counter
\begin{longtable}[]{@{}
  >{\raggedright\arraybackslash}p{(\linewidth - 6\tabcolsep) * \real{0.4444}}
  >{\raggedright\arraybackslash}p{(\linewidth - 6\tabcolsep) * \real{0.1481}}
  >{\raggedright\arraybackslash}p{(\linewidth - 6\tabcolsep) * \real{0.1667}}
  >{\raggedright\arraybackslash}p{(\linewidth - 6\tabcolsep) * \real{0.2407}}@{}}
\toprule\noalign{}
\begin{minipage}[b]{\linewidth}\raggedright
Initialization Method
\end{minipage} & \begin{minipage}[b]{\linewidth}\raggedright
Syntax
\end{minipage} & \begin{minipage}[b]{\linewidth}\raggedright
Example
\end{minipage} & \begin{minipage}[b]{\linewidth}\raggedright
Description
\end{minipage} \\
\midrule\noalign{}
\endhead
\bottomrule\noalign{}
\endlastfoot
Declaration with size & \texttt{data\_type\ array\_name[size];} &
\texttt{int\ marks[5];} & Creates array with specified size,
elements have garbage values \\
Declaration with initialization &
\texttt{data\_type\ array\_name[size]\ =\ \{values\};} &
\texttt{int\ ages[4]\ =\ \{21,\ 19,\ 25,\ 32\};} & Creates and
initializes array with specific values \\
Partial initialization &
\texttt{data\_type\ array\_name[size]\ =\ \{values\};} &
\texttt{int\ nums[5]\ =\ \{1,\ 2\};} & Initializes first elements,
rest become zero \\
Size inference & \texttt{data\_type\ array\_name[]\ =\ \{values\};}
& \texttt{int\ scores[]\ =\ \{95,\ 88,\ 72,\ 84,\ 91\};} & Size
determined by number of initializers \\
Individual element & \texttt{array\_name[index]\ =\ value;} &
\texttt{marks[0]\ =\ 85;} & Assigns value to specific element \\
\end{longtable}
}

\textbf{Array Visualization:}

\begin{verbatim}
int numbers[5] = {10, 20, 30, 40, 50};
\end{verbatim}

\begin{verbatim}
┌─────┬─────┬─────┬─────┬─────┐
│ 10  │ 20  │ 30  │ 40  │ 50  │
└─────┴─────┴─────┴─────┴─────┘
  [0]   [1]   [2]   [3]   [4]   \leftarrow indices
\end{verbatim}

\begin{itemize}
\tightlist
\item
  \textbf{Zero-indexing}: First element at index 0
\item
  \textbf{Contiguous memory}: Elements stored sequentially
\item
  \textbf{Size limitation}: Size must be known at compile time
\end{itemize}

\end{solutionbox}
\begin{mnemonicbox}
``Declare SIZE first, then FILL with values or let
COMPILER COUNT''

\end{mnemonicbox}
\subsection*{Question 4(c) [7 marks]}\label{q4c}

\textbf{Define Array and write a program to reverse a string.}

\begin{solutionbox}

An array is a collection of similar data items stored at contiguous
memory locations and accessed using a common name.

\begin{verbatim}
\#include {stdio.h}
\#include {string.h}

int main() \{
    char str[100], reversed[100];
    int i, j, length;
    
    // Input a string
    printf("Enter a string: ");
    gets(str);
    
    // Find the length of string
    length = strlen(str);
    
    // Reverse the string
for(i = length {-} 1,

j = 0; i {=} 0; i{-{-},} j++) \{

        reversed[j] = str[i];
    \}
    
    // Add null terminator
    reversed[j] = {}{0}{};
    
    // Display the reversed string
    printf("Reversed string: \%s{n}", reversed);
    
    return 0;
\}
\end{verbatim}

\textbf{Algorithm Visualization:}

\begin{verbatim}
flowchart LR
    A["Original: {HELLO"] {-}{-} B["H"] \& C["E"] \& D["L"] \& E["L"] \& F["O"]}
    F {-{-} G["reversed[0]"]}
    E {-{-} H["reversed[1]"]}
    D {-{-} I["reversed[2]"]}
    C {-{-} J["reversed[3]"]}
    B {-{-} K["reversed[4]"]}
    G \& H \& I \& J \& K {-{-} L["Reversed: OLLEH"]}
\end{verbatim}

\begin{itemize}
\tightlist
\item
  \textbf{Character array}: Stores string with null terminator
\item
  \textbf{Two-pointer technique}: One for original, one for reversed
\item
  \textbf{Zero-based indexing}: Arrays start at index 0
\end{itemize}

\end{solutionbox}
\begin{mnemonicbox}
``Start from END, place at BEGIN, stop at ZERO''

\end{mnemonicbox}
\subsection*{Question 4(a OR) [3
marks]}\label{question-4a-or-3-marks}

\textbf{Explain do while loop with example}

\begin{solutionbox}

The do-while loop is an exit-controlled loop that executes the loop body
at least once before checking the condition.

\begin{verbatim}
\#include {stdio.h}

int main() \{
    int num, sum = 0;
    
    do \{
        printf("Enter a number (0 to stop): ");
        scanf("\%d", \&num);
        sum += num;
    \} while(num != 0);
    
    printf("Sum of all entered numbers: \%d{n}", sum);
    
    return 0;
\}
\end{verbatim}

\textbf{Loop Execution Flow:}

\begin{verbatim}
flowchart LR
    A([Start]) {-{-} B[sum = 0]}
    B {-{-} C[/Input num/]}
    C {-{-} D[sum = sum + num]}
    D {-{-} E\{Is num != 0?\}}
    E {-{-}|Yes| C}
    E {-{-}|No| F[/Display sum/]}
    F {-{-} G([Stop])}
\end{verbatim}

\textbf{Key Characteristics:}

\begin{itemize}
\tightlist
\item
  \textbf{Execution order}: Body first, condition check later
\item
  \textbf{Guaranteed execution}: Loop body always executes at least once
\item
  \textbf{Termination}: Condition evaluated at bottom of loop
\end{itemize}

\end{solutionbox}
\begin{mnemonicbox}
``DO first, ask questions WHILE later''

\end{mnemonicbox}
\subsection*{Question 4(b OR) [4
marks]}\label{question-4b-or-4-marks}

\textbf{Define pointer and describe pointer with example.}

\begin{solutionbox}

A pointer is a variable that stores the memory address of another
variable.

{\def\LTcaptype{none} % do not increment counter
\begin{longtable}[]{@{}
  >{\raggedright\arraybackslash}p{(\linewidth - 4\tabcolsep) * \real{0.4359}}
  >{\raggedright\arraybackslash}p{(\linewidth - 4\tabcolsep) * \real{0.3333}}
  >{\raggedright\arraybackslash}p{(\linewidth - 4\tabcolsep) * \real{0.2308}}@{}}
\toprule\noalign{}
\begin{minipage}[b]{\linewidth}\raggedright
Pointer Concept
\end{minipage} & \begin{minipage}[b]{\linewidth}\raggedright
Description
\end{minipage} & \begin{minipage}[b]{\linewidth}\raggedright
Example
\end{minipage} \\
\midrule\noalign{}
\endhead
\bottomrule\noalign{}
\endlastfoot
Declaration & Data\_type *pointer\_name; & \texttt{int\ *ptr;} \\
Initialization & Assign address of a variable &
\texttt{int\ num\ =\ 10;\ int\ *ptr\ =\ \&num;} \\
Dereference & Access the value at the address &
\texttt{printf("\%d",\ *ptr);} // Prints 10 \\
Address operator & Gets address of a variable &
\texttt{printf("\%p",\ \&num);} // Prints address \\
Null pointer & Pointer that points to nothing &
\texttt{int\ *ptr\ =\ NULL;} \\
\end{longtable}
}

\textbf{Pointer Visualization:}

\begin{verbatim}
Memory:
┌──────┬───────┐    ┌──────┬───────┐
│ \&num │ 1000  │    │ \&ptr │ 2000  │
├──────┼───────┤    ├──────┼───────┤
│ num  │   10  │    │ ptr  │ 1000  │
└──────┴───────┘    └──────┴───────┘
                      │
                      └──────{ Points to address of num}
\end{verbatim}

\begin{itemize}
\tightlist
\item
  \textbf{Indirect access}: Access variables through their addresses
\item
  \textbf{Memory manipulation}: Direct memory access for efficiency
\item
  \textbf{Dynamic memory}: Enables allocation/deallocation during
  runtime
\end{itemize}

\end{solutionbox}
\begin{mnemonicbox}
``Pointers POINT to ADDRESS, STARS dereference to
VALUES''

\end{mnemonicbox}
\subsection*{Question 4(c OR) [7
marks]}\label{question-4c-or-7-marks}

\textbf{Define pointer and write a program to exchange two integers
using pointer arguments.}

\begin{solutionbox}

A pointer is a variable that contains the memory address of another
variable, allowing indirect access and manipulation of data.

\begin{verbatim}
\#include {stdio.h}

// Function to swap two integers using pointers
void swap(int *a, int *b) \{
    int temp = *a;
    *a = *b;
    *b = temp;
\}

int main() \{
    int num1, num2;
    
    // Input two integers
    printf("Enter first number: ");
    scanf("\%d", \&num1);
    
    printf("Enter second number: ");
    scanf("\%d", \&num2);
    
    printf("Before swapping: num1 = \%d, num2 = \%d{n}", num1, num2);
    
    // Call swap function with addresses of num1 and num2
    swap(\&num1, \&num2);
    
    printf("After swapping: num1 = \%d, num2 = \%d{n}", num1, num2);
    
    return 0;
\}
\end{verbatim}

\textbf{Swap Process Visualization:}

\begin{verbatim}
flowchart LR
    A[a points to num1{br /b points to num2] {-}{-} B[temp = *a]}
    B {-{-} C[*a = *b]}
    C {-{-} D[*b = temp]}
    D {-{-} E[Values exchanged]}
\end{verbatim}

\textbf{Memory Changes:}

\begin{verbatim}
Before swap:
num1 = 5, num2 = 10
a --> num1, b --> num2

Step 1: temp = *a
temp = 5, num1 = 5, num2 = 10

Step 2: *a = *b
temp = 5, num1 = 10, num2 = 10

Step 3: *b = temp
temp = 5, num1 = 10, num2 = 5

After swap:
num1 = 10, num2 = 5
\end{verbatim}

\begin{itemize}
\tightlist
\item
  \textbf{Pass by reference}: Pointers allow functions to modify
  original variables
\item
  \textbf{Temporary variable}: Required for swapping without data loss
\item
  \textbf{Function parameter}: Pointer arguments pass addresses
\end{itemize}

\end{solutionbox}
\begin{mnemonicbox}
``Grab by ADDRESS, change the CONTENT, without being
PRESENT''

\end{mnemonicbox}
\subsection*{Question 5(a) [3 marks]}\label{q5a}

\textbf{Write a program to find the numbers which are divisible by 7 in
between the numbers 50 and 500.}

\begin{solutionbox}

\begin{verbatim}
\#include {stdio.h}

int main() \{
    int i, count = 0;
    
    printf("Numbers divisible by 7 between 50 and 500:{n}");
    
    // Find and print numbers divisible by 7
    for(i = 50; i {=} 500; i++) \{
        if(i \% 7 == 0) \{
            printf("\%d ", i);
            count++;
            
            // Print 10 numbers per line for better readability
            if(count \% 10 == 0)
                printf("{n}");
        \}
    \}
    
    printf("{n}Total count: \%d{n}", count);
    
    return 0;
\}
\end{verbatim}

\textbf{Algorithm Visualization:}

\begin{verbatim}
flowchart LR
A([Start]) {-{-} B[Set

i = 50, count = 0]}

    B {-{-} C\{Is i = 500?\}}
    C {-{-}|Yes| D\{Is i \% 7 == 0?\}}
    D {-{-}|Yes| E[Print ibr /count++]}
    D {-{-}|No| F[i++]}
    E {-{-} F}
    F {-{-} C}
    C {-{-}|No| G[Print total count]}
    G {-{-} H([Stop])}
\end{verbatim}

\begin{itemize}
\tightlist
\item
  \textbf{Modulo operator}: i \% 7 == 0 checks divisibility
\item
  \textbf{Formatting output}: Line breaks for readability
\item
  \textbf{Counter variable}: Tracks how many numbers found
\end{itemize}

\end{solutionbox}
\begin{mnemonicbox}
``DIVide by SEVEN, ZERO remainder wins''

\end{mnemonicbox}
\subsection*{Question 5(b) [4 marks]}\label{q5b}

\textbf{Write a program which reads an integer from keyboard and prints
whether given number is odd or even.}

\begin{solutionbox}

\begin{verbatim}
\#include {stdio.h}

int main() \{
    int number;
    
    // Input an integer
    printf("Enter an integer: ");
    scanf("\%d", \&number);
    
    // Check if the number is even or odd
    if(number \% 2 == 0) \{
        printf("\%d is an even number.{n}", number);
    \} else \{
        printf("\%d is an odd number.{n}", number);
    \}
    
    return 0;
\}
\end{verbatim}

\textbf{Decision Logic:}

\begin{verbatim}
flowchart LR
    A[/Input number/] {-{-} B\{Is number \% 2 == 0?\}}
    B {-{-}|Yes| C[/Print "number is even"/]}
    B {-{-}|No| D[/Print "number is odd"/]}
    C {-{-} E([End])}
    D {-{-} E}
\end{verbatim}

\textbf{Modulo Division Table for Small Numbers:}

{\def\LTcaptype{none} % do not increment counter
\begin{longtable}[]{@{}lll@{}}
\toprule\noalign{}
Number & Number \% 2 & Even/Odd \\
\midrule\noalign{}
\endhead
\bottomrule\noalign{}
\endlastfoot
0 & 0 & Even \\
1 & 1 & Odd \\
2 & 0 & Even \\
3 & 1 & Odd \\
4 & 0 & Even \\
\end{longtable}
}

\begin{itemize}
\tightlist
\item
  \textbf{Modulo test}: Even numbers have remainder 0 when divided by 2
\item
  \textbf{Binary representation}: Last bit is 0 for even, 1 for odd
\item
  \textbf{Simple algorithm}: Works for all integers including negatives
\end{itemize}

\end{solutionbox}
\begin{mnemonicbox}
``EVEN with ZERO end, ODD with ONE bend''

\end{mnemonicbox}
\subsection*{Question 5(c) [7 marks]}\label{q5c}

\textbf{Define structure? Explain how it differs from array? Develop a
structure named book to save following information about books. Book
title, Name of author, Price and Number of pages.}

\begin{solutionbox}

A structure is a user-defined data type that allows grouping of
variables of different data types under a single name.

\textbf{Difference between Structure and Array:}

{\def\LTcaptype{none} % do not increment counter
\begin{longtable}[]{@{}
  >{\raggedright\arraybackslash}p{(\linewidth - 4\tabcolsep) * \real{0.3333}}
  >{\raggedright\arraybackslash}p{(\linewidth - 4\tabcolsep) * \real{0.4074}}
  >{\raggedright\arraybackslash}p{(\linewidth - 4\tabcolsep) * \real{0.2593}}@{}}
\toprule\noalign{}
\begin{minipage}[b]{\linewidth}\raggedright
Feature
\end{minipage} & \begin{minipage}[b]{\linewidth}\raggedright
Structure
\end{minipage} & \begin{minipage}[b]{\linewidth}\raggedright
Array
\end{minipage} \\
\midrule\noalign{}
\endhead
\bottomrule\noalign{}
\endlastfoot
Data type & Can store different data types & Stores elements of same
data type \\
Access & Members accessed using dot (.) operator & Elements accessed
using index [] \\
Memory allocation & Memory may not be contiguous & Memory is always
contiguous \\
Size & Size can vary for each member & Size is same for all elements \\
Declaration & Uses struct keyword & Uses square brackets [] \\
Purpose & Organizes related heterogeneous data & Organizes homogeneous
data \\
\end{longtable}
}

\textbf{Book Structure Program:}

\begin{verbatim}
\#include {stdio.h}
\#include {string.h}

// Define the structure
struct Book \{
    char title[100];
    char author[50];
    float price;
    int pages;
\;}

int main() \{
    // Declare a variable of type struct Book
    struct Book myBook;
    
    // Assign values to the structure members
    strcpy(myBook.title, "C Programming");
    strcpy(myBook.author, "Dennis Ritchie");
    myBook.price = 350.50;
    myBook.pages = 285;
    
    // Display book information
    printf("Book Details:{n}");
    printf("Title: \%s{n}", myBook.title);
    printf("Author: \%s{n}", myBook.author);
    printf("Price: \%.2f{n}", myBook.price);
    printf("Pages: \%d{n}", myBook.pages);
    
    return 0;
\}
\end{verbatim}

\textbf{Structure Visualization:}

\begin{verbatim}
struct Book myBook
┌───────────────────┬──────────────────────────────┐
│ Member            │ Value                        │
├───────────────────┼──────────────────────────────┤
│ title             │ "C Programming"              │
├───────────────────┼──────────────────────────────┤
│ author            │ "Dennis Ritchie"             │
├───────────────────┼──────────────────────────────┤
│ price             │ 350.50                       │
├───────────────────┼──────────────────────────────┤
│ pages             │ 285                          │
└───────────────────┴──────────────────────────────┘
\end{verbatim}

\begin{itemize}
\tightlist
\item
  \textbf{Structure definition}: Creates template for data
\item
  \textbf{Member access}: Use dot operator (structure.member)
\item
  \textbf{String handling}: Uses string functions for character arrays
\end{itemize}

\end{solutionbox}
\begin{mnemonicbox}
``STRUCTURE groups DIFFERENT, ARRAY repeats SAME''

\end{mnemonicbox}
\subsection*{Question 5(a OR) [3
marks]}\label{question-5a-or-3-marks}

\textbf{Write a program which reads a real number from keyboard and
prints a smallest integer greater than it.}

\begin{solutionbox}

\begin{verbatim}
\#include {stdio.h}
\#include {math.h}

int main() \{
    float number;
    int result;
    
    // Input a real number
    printf("Enter a real number: ");
    scanf("\%f", \&number);
    
    // Find smallest integer greater than the input
    result = ceil(number);
    
    // Display the result
    printf("Smallest integer greater than \%.2f is \%d{n}", number, result);
    
    return 0;
\}
\end{verbatim}

\textbf{Function Behavior:}

\begin{verbatim}
flowchart LR
    A[/Input real number/] {-{-} B[Apply ceil function]}
    B {-{-} C[/Display result/]}
\end{verbatim}

\textbf{Examples of ceil() function:}

\begin{verbatim}
Real Number | ceil() Result
------------|-------------
    3.14    |      4
    5.0     |      5
   -2.7     |     -2
\end{verbatim}

\begin{itemize}
\tightlist
\item
  \textbf{Math function}: ceil() rounds up to next integer
\item
  \textbf{Result type}: Returns smallest integer greater than input
\item
  \textbf{Handling edge cases}: Works with negative numbers
\end{itemize}

\end{solutionbox}
\begin{mnemonicbox}
``CEILING function, UP we go, NEXT integer we show''

\end{mnemonicbox}
\subsection*{Question 5(b OR) [4
marks]}\label{question-5b-or-4-marks}

\textbf{Write a program which reads character from keyboard and prints
its ASCII value.}

\begin{solutionbox}

\begin{verbatim}
\#include {stdio.h}

int main() \{
    char ch;
    
    // Input a character
    printf("Enter a character: ");
    scanf("\%c", \&ch);
    
    // Display ASCII value of the character
    printf("ASCII value of {}\%c{ is }\%d{n}", ch, ch);
    
    return 0;
\}
\end{verbatim}

\textbf{Program Visualization:}

\begin{verbatim}
flowchart LR
    A[/Input character/] {-{-} B[/Print character and its ASCII value/]}
\end{verbatim}

\textbf{ASCII Table Sample:}

\begin{verbatim}
Character | ASCII Value
----------|------------
    'A'    |     65
    'a'    |     97
    '0'    |     48
    ' '    |     32
\end{verbatim}

\begin{itemize}
\tightlist
\item
  \textbf{Character storage}: Characters stored as integers in memory
\item
  \textbf{Type conversion}: Automatic conversion from char to int
\item
  \textbf{Extended ASCII}: Values from 0 to 255 for 8-bit characters
\end{itemize}

\end{solutionbox}
\begin{mnemonicbox}
``CHARS have NUMBERS underneath, PRINT shows BOTH
sides''

\end{mnemonicbox}
\subsection*{Question 5(c OR) [7
marks]}\label{question-5c-or-7-marks}

\textbf{Define function? Explain its advantage. Write function to
calculate the square of a given integer number.}

\begin{solutionbox}

A function is a self-contained block of code designed to perform a
specific task. It takes input, processes it, and returns an output.

\textbf{Advantages of Functions:}

{\def\LTcaptype{none} % do not increment counter
\begin{longtable}[]{@{}ll@{}}
\toprule\noalign{}
Advantage & Description \\
\midrule\noalign{}
\endhead
\bottomrule\noalign{}
\endlastfoot
Code reusability & Write once, use many times \\
Modularity & Break complex problems into manageable parts \\
Maintainability & Easier to debug and modify isolated code \\
Abstraction & Hide implementation details \\
Readability & Makes code more organized and understandable \\
Scope control & Variables local to functions reduce naming conflicts \\
\end{longtable}
}

\textbf{Program with Square Function:}

\begin{verbatim}
\#include {stdio.h}

// Function to calculate square of an integer
int square(int num) \{
    return num * num;
\}

int main() \{
    int number, result;
    
    // Input an integer
    printf("Enter an integer: ");
    scanf("\%d", \&number);
    
    // Call the square function
    result = square(number);
    
    // Display the result
    printf("Square of \%d is \%d{n}", number, result);
    
    return 0;
\}
\end{verbatim}

\textbf{Function Flow:}

\begin{verbatim}
flowchart LR
    A[main function] {-{-}|call with number| B[square function]}
    B {-{-}|return num * num| C[main function]}
    C {-{-}|display result| D[End]}
\end{verbatim}

\textbf{Function Components:}

\begin{verbatim}
Return Type    Function Name    Parameters
    ↓               ↓               ↓
   int           square          (int num)
                    ↓
              Function Body
            {                
             return num * num;   \leftarrow Function Logic
            }
\end{verbatim}

\begin{itemize}
\tightlist
\item
  \textbf{Function prototype}: Declares function signature
\item
  \textbf{Parameters}: Input values passed to function
\item
  \textbf{Return value}: Output or result from function
\end{itemize}

\end{solutionbox}
\begin{mnemonicbox}
``Functions ENCAPSULATE tasks, take INPUTS, give
OUTPUTS''

\end{mnemonicbox}

\end{document}
