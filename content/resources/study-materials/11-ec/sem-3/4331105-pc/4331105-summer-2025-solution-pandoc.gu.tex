\documentclass[10pt,a4paper]{article}

% content/resources/templates/preamble.tex
\usepackage[margin=0.6in]{geometry}
\author{Milav Dabgar}
\usepackage{amsmath,amssymb,amsthm}
\usepackage{booktabs}
\usepackage{multirow}
\usepackage{xcolor}
\usepackage{tcolorbox}
\tcbuselibrary{breakable,skins}
\usepackage[colorlinks=true,linkcolor=blue]{hyperref}
\usepackage{titlesec}
\usepackage{enumitem}
\usepackage{tikz}
\usepackage{pgfplots}
\usepackage{circuitikz}
\usepackage[version=4]{mhchem}
\usepackage{longtable}
\usepackage{array}
\usepackage{float}
\usepackage{caption}
\usepackage{listings}

\lstset{
  basicstyle=\small\ttfamily,
  breaklines=true,
  breakatwhitespace=false,
  postbreak=\mbox{\textcolor{red}{$\hookrightarrow$}\space},
  float=false,
  numbers=left,
  numberstyle=\tiny\color{gray},
  numbersep=10pt,
  xleftmargin=2em,
  keywordstyle=\color{blue},
  commentstyle=\color{green!60!black},
  stringstyle=\color{purple},
  backgroundcolor=\color{gray!5},
  showstringspaces=false,
  tabsize=2,
  captionpos=b,
  keepspaces=true,
  columns=flexible
}

\pgfplotsset{compat=1.18}
\usetikzlibrary{shapes,arrows,positioning,calc,patterns,decorations.pathmorphing,decorations.markings,arrows.meta}

% Color scheme
\definecolor{headcolor}{RGB}{0,102,204}
\definecolor{keycolor}{RGB}{220,20,60}
\definecolor{solutioncolor}{RGB}{34,139,34}
\definecolor{mnemoniccolor}{RGB}{148,0,211}
\definecolor{codecolor}{RGB}{0,0,100}

% Spacing
\setlength{\parskip}{3pt}
\setlist[itemize]{nosep}
\setlist[enumerate]{nosep}

% Title formatting
\titleformat{\section}{\Large\bfseries\color{headcolor}}{\thesection}{1em}{}
\titleformat{\subsection}{\large\bfseries\color{headcolor}}{\thesubsection}{1em}{}

% Pandoc tightlist compatibility
\providecommand{\tightlist}{%
  \setlength{\itemsep}{0pt}\setlength{\parskip}{0pt}}

% Pandoc longtable compatibility
\newcounter{none}
\def\thenone{}


% content/resources/templates/gujarati-boxes.tex
\usepackage{fontspec}
\usepackage{polyglossia}

% Set Gujarati as main language (document is primarily in Gujarati)
% Note: gloss-gujarati.ldf doesn't exist in polyglossia, but it will use hyphenation patterns
\setdefaultlanguage{gujarati}
\setotherlanguage{english}

% Configure Gujarati font properly
% Use Language=Default to prevent polyglossia from trying to add language-specific features
% that don't exist for Gujarati, which causes "empty feature" warnings
\newfontfamily\gujaratifont[Script=Gujarati,AutoFakeBold=2.5,AutoFakeSlant=0.3]{Noto Sans Gujarati}
\setmainfont[Script=Gujarati,AutoFakeBold=2.5,AutoFakeSlant=0.3]{Noto Sans Gujarati}
% Use Noto Sans Gujarati for monospace to support Gujarati in text
\setmonofont[Scale=0.9]{Noto Sans Gujarati}

% Configure English to use the same font
\newfontfamily\englishfont[Script=Gujarati,AutoFakeBold=2.5,AutoFakeSlant=0.3]{Noto Sans Gujarati}

% Translations for polyglossia
\gappto\captionsgujarati{
  \renewcommand{\tablename}{કોષ્ટક}
  \renewcommand{\figurename}{આકૃતિ}
}

% Helper for TikZ nodes to ensure Gujarati font
\newcommand{\gu}[1]{{\gujaratifont #1}}

% Custom environments
\newtcolorbox{solutionbox}{
    breakable,
    enhanced,
    colback=solutioncolor!5!white,
    colframe=solutioncolor!75!black,
    fonttitle=\bfseries,
    title=જવાબ
}

\newtcolorbox{solutionboxnobreak}{
 colback=solutioncolor!5!white,
 colframe=solutioncolor!75!black,
 fonttitle=\bfseries,
 title=જવાબ
}

\newtcolorbox{keyformula}{
 breakable,
 enhanced,
 colback=keycolor!5!white,
 colframe=keycolor!75!black,
 fonttitle=\bfseries,
 title=રાસાયણિક સમીકરણ/સૂત્ર
}

\newtcolorbox{mnemonicbox}{
 breakable,
 enhanced,
 colback=mnemoniccolor!5!white,
 colframe=mnemoniccolor!75!black,
 fonttitle=\bfseries,
 title=મેમરી ટ્રીક
}


\begin{document}

\begin{center}
{\Huge\bfseries\color{headcolor} Subject Name (Gujarati)}\\[5pt]
{\LARGE 4331105 -- Summer 2025}\\[3pt]
{\large Semester 1 Study Material}\\[3pt]
{\normalsize\textit{Detailed Solutions and Explanations}}
\end{center}

\vspace{10pt}

\subsection*{પ્રશ્ન 1(a) [3
ગુણ]}\label{q1a}

\textbf{C માં કેટલા keywords છે? કોઈપણ ચાર keywords લખો}

\begin{solutionbox}

{\def\LTcaptype{none} % do not increment counter
\begin{longtable}[]{@{}ll@{}}
\toprule\noalign{}
કુલ Keywords & ઉદાહરણો \\
\midrule\noalign{}
\endhead
\bottomrule\noalign{}
\endlastfoot
32 keywords & int, float, char, if \\
\end{longtable}
}

\textbf{આકૃતિ:}

\begin{center}
\textbf{Mermaid Diagram (Code)}
\begin{verbatim}
{Shaded}
{Highlighting}[]
graph TD
    A[C Keywords {- 32 કુલ] {-}{-}{} B[Data Types: int, float, char, double]}
    A {-{-}{} C[Control: if, else, for, while]}
    A {-{-}{} D[Storage: static, extern, auto, register]}
{Highlighting}
{Shaded}
\end{verbatim}
\end{center}

\begin{itemize}
\tightlist
\item
  \textbf{32 keywords}: C ભાષામાં કુલ આરક્ષિત શબ્દો
\item
  \textbf{Data type keywords}: int, float, char, double વેરિએબલ જાહેર કરવા
  માટે
\item
  \textbf{Control keywords}: if, else, for, while પ્રોગ્રામ ફ્લો માટે
\end{itemize}

\end{solutionbox}
\begin{mnemonicbox}
``બિલાડી ચાર રંગમાં'' (char, int, float, const)

\end{mnemonicbox}
\begin{center}\rule{0.5\linewidth}{0.5pt}\end{center}

\subsection*{પ્રશ્ન 1(b) [4
ગુણ]}\label{q1b}

\textbf{વેરિએબલ શું છે? ઉદાહરણ સાથે વેરિએબલને નામ આપવાના નિયમો સમજાવો}

\begin{solutionbox}

\textbf{વેરિએબલ વ્યાખ્યા:}

{\def\LTcaptype{none} % do not increment counter
\begin{longtable}[]{@{}ll@{}}
\toprule\noalign{}
પાસાં & વર્ણન \\
\midrule\noalign{}
\endhead
\bottomrule\noalign{}
\endlastfoot
વ્યાખ્યા & ડેટા સ્ટોર કરવા માટે નામવાળી મેમરી લોકેશન \\
હેતુ & પ્રોગ્રામ દરમિયાન બદલાતા મૂલ્યો રાખવા \\
જાહેરાત & datatype variable\_name; \\
\end{longtable}
}

\textbf{નામકરણના નિયમો:}

\begin{itemize}
\tightlist
\item
  \textbf{પ્રથમ અક્ષર}: અક્ષર અથવા underscore (\_) હોવો જોઈએ
\item
  \textbf{પછીના અક્ષરો}: અક્ષરો, અંકો, underscore માત્ર
\item
  \textbf{Case sensitive}: `Age' અને `age' અલગ છે
\item
  \textbf{Keywords નહીં}: `int', `float' જેવા આરક્ષિત શબ્દો વાપરી શકાતા નથી
\end{itemize}

\textbf{ઉદાહરણો:}

\begin{verbatim}
int age;        // યોગ્ય
float \_salary;  // યોગ્ય
char name123;   // યોગ્ય
int 2number;    // ખોટું {- અંકથી શરૂ}
float for;      // ખોટું {- keyword વપરાયું}
\end{verbatim}

\end{solutionbox}
\begin{mnemonicbox}
``અક્ષર પહેલાં, keywords નહીં''

\end{mnemonicbox}
\begin{center}\rule{0.5\linewidth}{0.5pt}\end{center}

\subsection*{પ્રશ્ન 1(c) [7
ગુણ]}\label{q1c}

\textbf{નીચેના statements માં errors શોધો}

\begin{solutionbox}

{\def\LTcaptype{none} % do not increment counter
\begin{longtable}[]{@{}
  >{\raggedright\arraybackslash}p{(\linewidth - 4\tabcolsep) * \real{0.4231}}
  >{\raggedright\arraybackslash}p{(\linewidth - 4\tabcolsep) * \real{0.2692}}
  >{\raggedright\arraybackslash}p{(\linewidth - 4\tabcolsep) * \real{0.3077}}@{}}
\toprule\noalign{}
\begin{minipage}[b]{\linewidth}\raggedright
Statement
\end{minipage} & \begin{minipage}[b]{\linewidth}\raggedright
Error
\end{minipage} & \begin{minipage}[b]{\linewidth}\raggedright
કારણ
\end{minipage} \\
\midrule\noalign{}
\endhead
\bottomrule\noalign{}
\endlastfoot
(1) fLoat x; & ખોટો keyword & સાચું: float x; \\
(2) int min, max = 20; & અર્ધ initialization & માત્ર max initialize, min
નહીં \\
(3) long char c; & ખોટું combination & long ને char સાથે વાપરી શકાતું નથી \\
(4) iNt a; & ખોટો keyword & સાચું: int a; \\
(5) FLOAT

f=2; & ખોટો keyword & સાચું: float

f=2; \\

(6) double m ; n; & Missing datatype & સાચું: double m, n; \\
(7) Int score (100)0; & અનેક errors & ખોટું syntax, સાચું: int score =
100; \\
\end{longtable}
}

\textbf{મુખ્ય મુદ્દાઓ:}

\begin{itemize}
\tightlist
\item
  \textbf{Case sensitivity}: Keywords નાના અક્ષરમાં હોવા જોઈએ
\item
  \textbf{Multiple declaration}: Comma separator વાપરો
\item
  \textbf{Initialization syntax}: = operator વાપરો
\end{itemize}

\end{solutionbox}
\begin{mnemonicbox}
``Keywords હંમેશા નાના અક્ષરે''

\end{mnemonicbox}
\begin{center}\rule{0.5\linewidth}{0.5pt}\end{center}

\subsection*{પ્રશ્ન 1(c) OR [7
ગુણ]}\label{q1c}

\textbf{અલ્ગોરિધમ શું છે? ફ્લોચાર્ટ શું છે? વર્તુળનો વિસ્તાર અને પરિમિતિ શોધવા માટે
ફ્લોચાર્ટ દોરો.}

\begin{solutionbox}

\textbf{વ્યાખ્યાઓ:}

{\def\LTcaptype{none} % do not increment counter
\begin{longtable}[]{@{}ll@{}}
\toprule\noalign{}
શબ્દ & વ્યાખ્યા \\
\midrule\noalign{}
\endhead
\bottomrule\noalign{}
\endlastfoot
Algorithm & સમસ્યા હલ કરવાની પગલાબદ્ધ પ્રક્રિયા \\
Flowchart & Algorithm નું પ્રતીકો વડે દ્રશ્ય પ્રતિનિધિત્વ \\
\end{longtable}
}

\textbf{વર્તુળ માટે ફ્લોચાર્ટ:}

\begin{verbatim}
flowchart LR
    A[શરૂઆત] {-{-} B[radius r input કરો]}
    B {-{-} C[area = π  r^{2} ગણતરી કરો]}
    C {-{-} D[perimeter = 2  π  r ગણતરી કરો]}
    D {-{-} E[area અને perimeter દર્શાવો]}
    E {-{-} F[અંત]}
\end{verbatim}

\textbf{Algorithm ના પગલાં:}

\begin{itemize}
\tightlist
\item
  \textbf{પગલું 1}: શરૂઆત કરો
\item
  \textbf{પગલું 2}: Radius નું મૂલ્ય input કરો
\item
  \textbf{પગલું 3}: π\timesr^{2} સૂત્ર વડે area ગણો
\item
  \textbf{પગલું 4}: 2\timesπ\timesr સૂત્ર વડે perimeter ગણો
\end{itemize}

\end{solutionbox}
\begin{mnemonicbox}
``શરૂ Input ગણતરી દર્શાવો અંત''

\end{mnemonicbox}
\begin{center}\rule{0.5\linewidth}{0.5pt}\end{center}

\subsection*{પ્રશ્ન 2(a) [3
ગુણ]}\label{q2a}

\textbf{ઓપરેટર શું છે? C ના બધા operators ની યાદી બનાવો.}

\begin{solutionbox}

\textbf{ઓપરેટર વ્યાખ્યા:}

{\def\LTcaptype{none} % do not increment counter
\begin{longtable}[]{@{}ll@{}}
\toprule\noalign{}
પાસાં & વર્ણન \\
\midrule\noalign{}
\endhead
\bottomrule\noalign{}
\endlastfoot
વ્યાખ્યા & Operands પર operations કરતા ખાસ પ્રતીકો \\
હેતુ & ડેટા અને વેરિએબલ્સ સાથે કામ કરવા \\
\end{longtable}
}

\textbf{C ઓપરેટર્સ યાદી:}

{\def\LTcaptype{none} % do not increment counter
\begin{longtable}[]{@{}ll@{}}
\toprule\noalign{}
વર્ગ & Operators \\
\midrule\noalign{}
\endhead
\bottomrule\noalign{}
\endlastfoot
Arithmetic & +, -, *, /, \% \\
Relational & \textless, \textgreater, \textless=, \textgreater=, ==,
!= \\
Logical & \&\&, \textbar\textbar, ! \\
Assignment & =, +=, -=, *=, /= \\
Increment/Decrement & ++, -- \\
Conditional & ?: \\
\end{longtable}
}

\end{solutionbox}
\begin{mnemonicbox}
``ગણતરી સંબંધ તર્ક અસાઇન વધારો શરત''

\end{mnemonicbox}
\begin{center}\rule{0.5\linewidth}{0.5pt}\end{center}

\subsection*{પ્રશ્ન 2(b) [4
ગુણ]}\label{q2b}

\textbf{while અને do while loop વચ્ચે તફાવત લખો.}

\begin{solutionbox}

{\def\LTcaptype{none} % do not increment counter
\begin{longtable}[]{@{}lll@{}}
\toprule\noalign{}
પાસાં & while loop & do-while loop \\
\midrule\noalign{}
\endhead
\bottomrule\noalign{}
\endlastfoot
\textbf{Entry condition} & Pre-tested & Post-tested \\
\textbf{ન્યૂનતમ execution} & 0 વખત & ઓછામાં ઓછું 1 વખત \\
\textbf{Syntax} & while(condition) \{ \} & do \{ \} while(condition); \\
\textbf{Semicolon} & while પછી જરૂરી નથી & while પછી જરૂરી છે \\
\end{longtable}
}

\textbf{ઉદાહરણ:}

\begin{verbatim}
// while loop
while(i {} 5) \{
    printf("\%d", i);
    i++;
\}

// do{-while loop  }
do \{
    printf("\%d", i);
    i++;
\} while(i {} 5);
\end{verbatim}

\textbf{મુખ્ય મુદ્દાઓ:}

\begin{itemize}
\tightlist
\item
  \textbf{Pre-tested}: Execution પહેલાં condition ચકાસાય
\item
  \textbf{Post-tested}: Execution પછી condition ચકાસાય
\end{itemize}

\end{solutionbox}
\begin{mnemonicbox}
``While પહેલાં, Do પછી''

\end{mnemonicbox}
\begin{center}\rule{0.5\linewidth}{0.5pt}\end{center}

\subsection*{પ્રશ્ન 2(c) [7
ગુણ]}\label{q2c}

\textbf{scanf() function નો formatted input માટે કેવી રીતે ઉપયોગ થાય છે?
ઉદાહરણ સાથે સમજાવો}

\begin{solutionbox}

\textbf{scanf() Function:}

{\def\LTcaptype{none} % do not increment counter
\begin{longtable}[]{@{}ll@{}}
\toprule\noalign{}
લક્ષણ & વર્ણન \\
\midrule\noalign{}
\endhead
\bottomrule\noalign{}
\endlastfoot
હેતુ & Keyboard થી formatted input વાંચવા \\
Syntax & scanf(``format\_string'', \&variable); \\
Return & સફળતાપૂર્વક વંચાયેલા inputs ની સંખ્યા \\
\end{longtable}
}

\textbf{Format Specifiers:}

{\def\LTcaptype{none} % do not increment counter
\begin{longtable}[]{@{}ll@{}}
\toprule\noalign{}
Specifier & Data Type \\
\midrule\noalign{}
\endhead
\bottomrule\noalign{}
\endlastfoot
\%d & int \\
\%f & float \\
\%c & char \\
\%s & string \\
\end{longtable}
}

\textbf{ઉદાહરણો:}

\begin{verbatim}
int age;
float salary;
char grade;

scanf("\%d", \&age);           // Integer વાંચો
scanf("\%f", \&salary);        // Float વાંચો
scanf("\%c", \&grade);         // Character વાંચો
scanf("\%d \%f", \&age, \&salary); // બહુવિધ inputs
\end{verbatim}

\textbf{મહત્વના મુદ્દાઓ:}

\begin{itemize}
\tightlist
\item
  \textbf{Address operator (\&)}: Variables માટે જરૂરી
\item
  \textbf{Format string}: Data types સાથે match થવું જોઈએ
\item
  \textbf{Buffer issues}: જરૂર પડે તો fflush(stdin) વાપરો
\end{itemize}

\end{solutionbox}
\begin{mnemonicbox}
``Address Format Match''

\end{mnemonicbox}
\begin{center}\rule{0.5\linewidth}{0.5pt}\end{center}

\subsection*{પ્રશ્ન 2(a) OR [3
ગુણ]}\label{q2a}

\textbf{C ભાષાના arithmetic અને relational operators ની યાદી બનાવો}

\begin{solutionbox}

{\def\LTcaptype{none} % do not increment counter
\begin{longtable}[]{@{}lll@{}}
\toprule\noalign{}
Operator Type & Operators & હેતુ \\
\midrule\noalign{}
\endhead
\bottomrule\noalign{}
\endlastfoot
\textbf{Arithmetic} & +, -, *, /, \% & ગાણિતિક operations \\
\textbf{Relational} & \textless, \textgreater, \textless=,
\textgreater=, ==, != & Comparison operations \\
\end{longtable}
}

\textbf{ઉદાહરણો:}

\begin{verbatim}
// Arithmetic
int a = 10 + 5;    // Addition
int b = 10 \% 3;    // Modulus (remainder)

// Relational
if(a {} b)          // મોટું
if(a == b)         // બરાબર
\end{verbatim}

\end{solutionbox}
\begin{mnemonicbox}
``ગણતરી સરખામણી''

\end{mnemonicbox}
\begin{center}\rule{0.5\linewidth}{0.5pt}\end{center}

\subsection*{પ્રશ્ન 2(b) OR [4
ગુણ]}\label{q2b}

\textbf{else if ladder નો flow chart દોરો.}

\begin{solutionbox}

\begin{verbatim}
flowchart LR
    A[શરૂઆત] {-{-} B\{શરત 1?\}}
    B {-{-}|સાચું| C[Statement 1]}
    B {-{-}|ખોટું| D\{શરત 2?\}}
    D {-{-}|સાચું| E[Statement 2]}
    D {-{-}|ખોટું| F\{શરત 3?\}}
    F {-{-}|સાચું| G[Statement 3]}
    F {-{-}|ખોટું| H[Else Statement]}
    C {-{-} I[અંત]}
    E {-{-} I}
    G {-{-} I}
    H {-{-} I}
\end{verbatim}

\textbf{માળખું:}

\begin{itemize}
\tightlist
\item
  \textbf{બહુવિધ શરતો}: ક્રમમાં ચકાસાય છે
\item
  \textbf{પ્રથમ સાચી}: તેનો block execute થાય
\item
  \textbf{Default case}: કોઈ match ન મળે તો else block
\end{itemize}

\end{solutionbox}
\begin{mnemonicbox}
``પ્રથમ સાચી શોધો execute કરો''

\end{mnemonicbox}
\begin{center}\rule{0.5\linewidth}{0.5pt}\end{center}

\subsection*{પ્રશ્ન 2(c) OR [7
ગુણ]}\label{q2c}

\textbf{printf() function નો formatted output માટે કેવી રીતે ઉપયોગ થાય છે?
ઉદાહરણ સાથે સમજાવો}

\begin{solutionbox}

\textbf{printf() Function:}

{\def\LTcaptype{none} % do not increment counter
\begin{longtable}[]{@{}ll@{}}
\toprule\noalign{}
લક્ષણ & વર્ણન \\
\midrule\noalign{}
\endhead
\bottomrule\noalign{}
\endlastfoot
હેતુ & Screen પર formatted output દર્શાવવા \\
Syntax & printf(``format\_string'', variables); \\
Return & Print કરાયેલા characters ની સંખ્યા \\
\end{longtable}
}

\textbf{Format Specifiers:}

{\def\LTcaptype{none} % do not increment counter
\begin{longtable}[]{@{}lll@{}}
\toprule\noalign{}
Specifier & વપરાશ & ઉદાહરણ \\
\midrule\noalign{}
\endhead
\bottomrule\noalign{}
\endlastfoot
\%d & Integer & printf(``\%d'', 25); \\
\%f & Float & printf(``\%.2f'', 3.14); \\
\%c & Character & printf(``\%c'', `A'); \\
\%s & String & printf(``\%s'', ``Hello''); \\
\end{longtable}
}

\textbf{Advanced Formatting:}

\begin{verbatim}
int num = 123;
float pi = 3.14159;

printf("Number: \%5d{n}", num);      // Width specification
printf("Pi: \%.2f{n}", pi);          // Precision specification
printf("Hex: \%x{n}", num);          // Hexadecimal
printf("Left aligned: \%{-10dn}", num); // Left alignment
\end{verbatim}

\textbf{Escape Sequences:}

\begin{itemize}
\tightlist
\item
  **\n**: નવી લીટી
\item
  **\t**: Tab space
\item
  \textbf{\textbackslash{}}: Backslash
\end{itemize}

\end{solutionbox}
\begin{mnemonicbox}
``Format Width Precision Align''

\end{mnemonicbox}
\begin{center}\rule{0.5\linewidth}{0.5pt}\end{center}

\subsection*{પ્રશ્ન 3(a) [3
ગુણ]}\label{q3a}

\textbf{Logical operators ની યાદી બનાવો અને તેને સમજાવો}

\begin{solutionbox}

{\def\LTcaptype{none} % do not increment counter
\begin{longtable}[]{@{}llll@{}}
\toprule\noalign{}
Operator & Symbol & વર્ણન & Truth Table \\
\midrule\noalign{}
\endhead
\bottomrule\noalign{}
\endlastfoot
\textbf{AND} & \&\& & બંને operands સાચા હોય તો સાચું & T\&\&T = T, બાકી =
F \\
\textbf{OR} & \textbar\textbar{} & કોઈપણ operand સાચો હોય તો સાચું &
F\textbar\textbar

F = F, બાકી = T \\

\textbf{NOT} & ! & Condition ઉલટાવે છે & !T = F, !F = T \\
\end{longtable}
}

\textbf{ઉદાહરણો:}

\begin{verbatim}
int

a = 5,

b = 10;


if(a {} 0 \&\& b {} 0)    // બંને શરતો સાચી હોવી જોઈએ
if(a {} 15 || b {} 5)   // ઓછામાં ઓછી એક શરત સાચી
if(!(a {} 10))         // શરતનું નકારણ
\end{verbatim}

\end{solutionbox}
\begin{mnemonicbox}
``અને અથવા નહીં''

\end{mnemonicbox}
\begin{center}\rule{0.5\linewidth}{0.5pt}\end{center}

\subsection*{પ્રશ્ન 3(b) [4
ગુણ]}\label{q3b}

\textbf{for loop ને ઉદાહરણ સાથે સમજાવો.}

\begin{solutionbox}

\textbf{For Loop માળખું:}

{\def\LTcaptype{none} % do not increment counter
\begin{longtable}[]{@{}ll@{}}
\toprule\noalign{}
ઘટક & હેતુ \\
\midrule\noalign{}
\endhead
\bottomrule\noalign{}
\endlastfoot
Initialization & શરુઆતી મૂલ્ય સેટ કરવું \\
Condition & ચાલુ રાખવા માટે ટેસ્ટ \\
Update & Loop variable ને બદલવું \\
\end{longtable}
}

\textbf{Syntax:}

\begin{verbatim}
for(initialization; condition; update) \{
    statements;
\}
\end{verbatim}

\textbf{ઉદાહરણ:}

\begin{verbatim}
// 1 થી 5 સુધીના નંબર print કરો
for(int

i = 1; i {=} 5; i++) \{

    printf("\%d ", i);
\}
// Output: 1 2 3 4 5
\end{verbatim}

\textbf{Execution Flow:}

\begin{itemize}
\tightlist
\item
  \textbf{પગલું 1}: i = 1 initialize કરો
\item
  \textbf{પગલું 2}: i \textless= 5 condition ચકાસો
\item
  \textbf{પગલું 3}: Statements execute કરો
\item
  \textbf{પગલું 4}: i++ update કરો, પગલું 2 પર પાછા
\end{itemize}

\end{solutionbox}
\begin{mnemonicbox}
``Initialize ચકાસો Execute Update''

\end{mnemonicbox}
\begin{center}\rule{0.5\linewidth}{0.5pt}\end{center}

\subsection*{પ્રશ્ન 3(c) [7
ગુણ]}\label{q3c}

\textbf{ત્રણ પૂર્ણાંક સંખ્યાઓ x અને y માંથી મહત્તમ શોધવા માટે પ્રોગ્રામ લખો.}

\begin{solutionbox}

\begin{verbatim}
\#include {stdio.h}

int main() \{
    int x, y, z, max;
    
    printf("ત્રણ સંખ્યાઓ દાખલ કરો: ");
    scanf("\%d \%d \%d", \&x, \&y, \&z);
    
    max = x;  // પ્રથમ સંખ્યાને મહત્તમ માનો
    
    if(y {} max) \{
        max = y;
    \}
    if(z {} max) \{
        max = z;
    \}
    
    printf("મહત્તમ સંખ્યા છે: \%d", max);
    
    return 0;
\}
\end{verbatim}

\textbf{Algorithm ના પગલાં:}

{\def\LTcaptype{none} % do not increment counter
\begin{longtable}[]{@{}ll@{}}
\toprule\noalign{}
પગલું & કાર્ય \\
\midrule\noalign{}
\endhead
\bottomrule\noalign{}
\endlastfoot
1 & ત્રણ સંખ્યાઓ input કરો \\
2 & પ્રથમને મહત્તમ માનો \\
3 & બીજી સાથે સરખામણી, મોટી હોય તો update \\
4 & ત્રીજી સાથે સરખામણી, મોટી હોય તો update \\
5 & મહત્તમ દર્શાવો \\
\end{longtable}
}

\textbf{વૈકલ્પિક પદ્ધતિ:}

\begin{verbatim}
max = (x {} y) ? ((x {} z) ? x : z) : ((y {} z) ? y : z);
\end{verbatim}

\end{solutionbox}
\begin{mnemonicbox}
``માનો સરખાવો Update દર્શાવો''

\end{mnemonicbox}
\begin{center}\rule{0.5\linewidth}{0.5pt}\end{center}

\subsection*{પ્રશ્ન 3(a) OR [3
ગુણ]}\label{q3a}

\textbf{conditional operator ને ઉદાહરણ સાથે સમજાવો.}

\begin{solutionbox}

\textbf{Conditional Operator (Ternary):}

{\def\LTcaptype{none} % do not increment counter
\begin{longtable}[]{@{}ll@{}}
\toprule\noalign{}
લક્ષણ & વર્ણન \\
\midrule\noalign{}
\endhead
\bottomrule\noalign{}
\endlastfoot
Symbol & ?: \\
Syntax & condition ? value1 : value2 \\
હેતુ & if-else નો ટૂંકો રસ્તો \\
\end{longtable}
}

\textbf{ઉદાહરણો:}

\begin{verbatim}
int

a = 10,

b = 20;

int max = (a {} b) ? a : b;        // max = 20

char grade = (marks {=} 60) ? {P} : {F};
printf("Status: \%s", (age {=} 18) ? "Adult" : "Minor");
\end{verbatim}

\textbf{સમાન if-else:}

\begin{verbatim}
if(a {} b)
    max = a;
else
    max = b;
\end{verbatim}

\textbf{ફાયદાઓ:}

\begin{itemize}
\tightlist
\item
  \textbf{સંક્ષિપ્ત}: એક લીટીમાં expression
\item
  \textbf{કાર્યક્ષમ}: ઝડપી execution
\end{itemize}

\end{solutionbox}
\begin{mnemonicbox}
``પ્રશ્નચિહ્ન કોલન પસંદગી''

\end{mnemonicbox}
\begin{center}\rule{0.5\linewidth}{0.5pt}\end{center}

\subsection*{પ્રશ્ન 3(b) OR [4
ગુણ]}\label{q3b}

\textbf{while loop ને ઉદાહરણ સાથે સમજાવો.}

\begin{solutionbox}

\textbf{While Loop:}

{\def\LTcaptype{none} % do not increment counter
\begin{longtable}[]{@{}ll@{}}
\toprule\noalign{}
લક્ષણ & વર્ણન \\
\midrule\noalign{}
\endhead
\bottomrule\noalign{}
\endlastfoot
પ્રકાર & Entry-controlled loop \\
Syntax & while(condition) \{ statements; \} \\
Execution & શરત સાચી હોય ત્યાં સુધી repeat \\
\end{longtable}
}

\textbf{ઉદાહરણ:}

\begin{verbatim}
int i = 1;
while(i {=} 5) \{
    printf("\%d ", i);
    i++;
\}
// Output: 1 2 3 4 5
\end{verbatim}

\textbf{મહત્વના મુદ્દાઓ:}

\begin{itemize}
\tightlist
\item
  \textbf{Initialization}: Loop પહેલાં
\item
  \textbf{Condition}: શરૂઆતમાં ચકાસાય
\item
  \textbf{Update}: Loop body અંદર
\item
  \textbf{Infinite loop}: જો condition ક્યારેય false ન બને
\end{itemize}

\textbf{Flowchart માળખું:}

\begin{verbatim}
flowchart LR
    A[Initialize] {-{-} B\{Condition?\}}
    B {-{-}|સાચું| C[Statements Execute કરો]}
    C {-{-} D[Variable Update કરો]}
    D {-{-} B}
    B {-{-}|ખોટું| E[Loop બહાર નીકળો]}
\end{verbatim}

\end{solutionbox}
\begin{mnemonicbox}
``Initialize ચકાસો Execute Update''

\end{mnemonicbox}
\begin{center}\rule{0.5\linewidth}{0.5pt}\end{center}

\subsection*{પ્રશ્ન 3(c) OR [7
ગુણ]}\label{q3c}

\textbf{કીબોર્ડમાંથી પૂર્ણાંક વાંચવા માટે અને આપેલ સંખ્યા odd હોય કે even હોય તે
શોધવા માટેનો પ્રોગ્રામ લખો.}

\begin{solutionbox}

\begin{verbatim}
\#include {stdio.h}

int main() \{
    int number;
    
    printf("એક પૂર્ણાંક દાખલ કરો: ");
    scanf("\%d", \&number);
    
    if(number \% 2 == 0) \{
        printf("\%d એ સમ સંખ્યા છે", number);
    \}
    else \{
        printf("\%d એ વિષમ સંખ્યા છે", number);
    \}
    
    return 0;
\}
\end{verbatim}

\textbf{તર્ક સમજૂતી:}

{\def\LTcaptype{none} % do not increment counter
\begin{longtable}[]{@{}ll@{}}
\toprule\noalign{}
ખ્યાલ & વર્ણન \\
\midrule\noalign{}
\endhead
\bottomrule\noalign{}
\endlastfoot
\textbf{Modulus operator (\%)} & ભાગાકાર પછી બાકી આપે છે \\
\textbf{સમ શરત} & number \% 2 == 0 \\
\textbf{વિષમ શરત} & number \% 2 != 0 \\
\end{longtable}
}

\textbf{વૈકલ્પિક પદ્ધતિઓ:}

\begin{verbatim}
// પદ્ધતિ 2: Conditional operator વાપરીને
printf("\%d એ \%s છે", number, (number \% 2 == 0) ? "સમ" : "વિષમ");

// પદ્ધતિ 3: Bitwise AND વાપરીને
if(number \& 1)
    printf("વિષમ");
else
    printf("સમ");
\end{verbatim}

\textbf{Sample Output:}

\begin{verbatim}
એક પૂર્ણાંક દાખલ કરો: 7
7 એ વિષમ સંખ્યા છે
\end{verbatim}

\end{solutionbox}
\begin{mnemonicbox}
``Modulus બે શૂન્ય સમ''

\end{mnemonicbox}
\begin{center}\rule{0.5\linewidth}{0.5pt}\end{center}

\subsection*{પ્રશ્ન 4(a) [3
ગુણ]}\label{q4a}

**નીચેના arithmetic expressions નું મૂલ્યાંકન કરો: 30/4*4 -- 20\%6 + 17/2**

\begin{solutionbox}

\textbf{પગલાબદ્ધ મૂલ્યાંકન:}

{\def\LTcaptype{none} % do not increment counter
\begin{longtable}[]{@{}llll@{}}
\toprule\noalign{}
પગલું & Expression & ગણતરી & પરિણામ \\
\midrule\noalign{}
\endhead
\bottomrule\noalign{}
\endlastfoot
1 & 30/4*4 & (30/4)\emph{4 = 7}4 & 28 \\
2 & 20\%6 & 20 mod 6 & 2 \\
3 & 17/2 & Integer division & 8 \\
4 & અંતિમ & 28 - 2 + 8 & 34 \\
\end{longtable}
}

\textbf{Operator પ્રાધાન્યતા:}

{\def\LTcaptype{none} % do not increment counter
\begin{longtable}[]{@{}ll@{}}
\toprule\noalign{}
પ્રાધાન્યતા & Operators \\
\midrule\noalign{}
\endhead
\bottomrule\noalign{}
\endlastfoot
ઉચ્ચ & *, /, \% (ડાબેથી જમણે) \\
નીચી & +, - (ડાબેથી જમણે) \\
\end{longtable}
}

\textbf{સંપૂર્ણ ગણતરી:}

\begin{verbatim}
30/4*4 – 20%6 + 17/2
= 7*4 - 2 + 8      // પહેલાં division અને modulus
= 28 - 2 + 8       // Multiplication
= 26 + 8           // +,- માટે ડાબેથી જમણે
= 34               // અંતિમ જવાબ
\end{verbatim}

\end{solutionbox}
\begin{mnemonicbox}
``ગુણા ભાગ પહેલાં બાદબાકી પછી''

\end{mnemonicbox}
\begin{center}\rule{0.5\linewidth}{0.5pt}\end{center}

\subsection*{પ્રશ્ન 4(b) [4
ગુણ]}\label{q4b}

\textbf{5 પૂર્ણાંક સંખ્યાઓની array નો સરવાળો અને સરેરાશ શોધવા માટેનો પ્રોગ્રામ
લખો.}

\begin{solutionbox}

\begin{verbatim}
\#include {stdio.h}

int main() \{
    int numbers[5];
    int sum = 0;
    float average;
    
    printf("5 પૂર્ણાંકો દાખલ કરો:{n}");
    for(int i = 0; i {} 5; i++) \{
        scanf("\%d", \&numbers[i]);
        sum += numbers[i];
    \}
    
    average = (float)sum / 5;
    
    printf("સરવાળો = \%d{n}", sum);
    printf("સરેરાશ = \%.2f", average);
    
    return 0;
\}
\end{verbatim}

\textbf{Algorithm:}

{\def\LTcaptype{none} % do not increment counter
\begin{longtable}[]{@{}ll@{}}
\toprule\noalign{}
પગલું & કાર્ય \\
\midrule\noalign{}
\endhead
\bottomrule\noalign{}
\endlastfoot
1 & 5 integers ની array જાહેર કરો \\
2 & Sum ને 0 થી initialize કરો \\
3 & Loop વાપરીને 5 numbers input કરો \\
4 & દરેક number ને sum માં ઉમેરો \\
5 & Average = sum/5 ગણો \\
6 & પરિણામો દર્શાવો \\
\end{longtable}
}

\textbf{મુખ્ય મુદ્દાઓ:}

\begin{itemize}
\tightlist
\item
  \textbf{Type casting}: (float)sum ચોક્કસ division માટે
\item
  \textbf{Loop વપરાશ}: Repetitive input માટે કાર્યક્ષમ
\end{itemize}

\end{solutionbox}
\begin{mnemonicbox}
``જાહેર Input ઉમેરો ગણો દર્શાવો''

\end{mnemonicbox}
\begin{center}\rule{0.5\linewidth}{0.5pt}\end{center}

\subsection*{પ્રશ્ન 4(c) [7
ગુણ]}\label{q4c}

\textbf{Pointer વ્યાખ્યાયિત કરો. Pointers કેવી રીતે declared અને initialized
કરવામાં આવે છે તે ઉદાહરણ સાથે સમજાવો.}

\begin{solutionbox}

\textbf{Pointer વ્યાખ્યા:}

{\def\LTcaptype{none} % do not increment counter
\begin{longtable}[]{@{}ll@{}}
\toprule\noalign{}
પાસાં & વર્ણન \\
\midrule\noalign{}
\endhead
\bottomrule\noalign{}
\endlastfoot
વ્યાખ્યા & બીજા variable નું memory address સ્ટોર કરતું variable \\
હેતુ & સીધી memory access અને dynamic memory allocation \\
Symbol & * (asterisk) declaration અને dereferencing માટે \\
\end{longtable}
}

\textbf{Declaration અને Initialization:}

\begin{verbatim}
// Declaration
int *ptr;           // Integer નો pointer
float *fptr;        // Float નો pointer
char *cptr;         // Character નો pointer

// Initialization
int num = 10;
int *ptr = \&num;    // num ના address સાથે initialize

// વૈકલ્પિક
int *ptr;
ptr = \&num;         // પછીથી address assign
\end{verbatim}

\textbf{ઉદાહરણ પ્રોગ્રામ:}

\begin{verbatim}
\#include {stdio.h}

int main() \{
    int num = 25;
    int *ptr = \&num;
    
    printf("num નું મૂલ્ય: \%d{n}", num);
    printf("num નું address: \%p{n}", \&num);
    printf("ptr નું મૂલ્ય: \%p{n}", ptr);
    printf("ptr દ્વારા pointed મૂલ્ય: \%d{n}", *ptr);
    
    return 0;
\}
\end{verbatim}

\textbf{મુખ્ય Operators:}

\begin{itemize}
\tightlist
\item
  \textbf{\& (Address-of)}: Variable નું address મેળવે છે
\item
  \textbf{* (Dereference)}: Address પરનું મૂલ્ય મેળવે છે
\end{itemize}

\textbf{Memory Diagram:}

\begin{verbatim}
num: [25] at address 1000
ptr: [1000] at address 2000
\end{verbatim}

\end{solutionbox}
\begin{mnemonicbox}
``Address Star Dereference''

\end{mnemonicbox}
\begin{center}\rule{0.5\linewidth}{0.5pt}\end{center}

\subsection*{પ્રશ્ન 4(a) OR [3
ગુણ]}\label{q4a}

\textbf{નીચેના arithmetic expressions નું મૂલ્યાંકન કરો: 50 / 3 \% 3 + 5 * 7}

\begin{solutionbox}

\textbf{પગલાબદ્ધ મૂલ્યાંકન:}

{\def\LTcaptype{none} % do not increment counter
\begin{longtable}[]{@{}llll@{}}
\toprule\noalign{}
પગલું & Expression & ગણતરી & પરિણામ \\
\midrule\noalign{}
\endhead
\bottomrule\noalign{}
\endlastfoot
1 & 50/3 & Integer division & 16 \\
2 & 16\%3 & 16 mod 3 & 1 \\
3 & 5*7 & Multiplication & 35 \\
4 & અંતિમ & 1 + 35 & 36 \\
\end{longtable}
}

\textbf{સંપૂર્ણ ગણતરી:}

\begin{verbatim}
50 / 3 % 3 + 5 * 7
= 16 % 3 + 35      // પહેલાં division અને multiplication
= 1 + 35           // Modulus operation
= 36               // અંતિમ જવાબ
\end{verbatim}

\textbf{Operator પ્રાધાન્યતા લાગુ:}

\begin{itemize}
\tightlist
\item
  \textbf{ઉચ્ચ પ્રાધાન્યતા}: /, \%, * (ડાબેથી જમણે)
\item
  \textbf{નીચી પ્રાધાન્યતા}: + (ડાબેથી જમણે)
\end{itemize}

\end{solutionbox}
\begin{mnemonicbox}
``ભાગ Mod ગુણા ઉમેરો''

\end{mnemonicbox}
\begin{center}\rule{0.5\linewidth}{0.5pt}\end{center}

\subsection*{પ્રશ્ન 4(b) OR [4
ગુણ]}\label{q4b}

\textbf{N પૂર્ણાંકોની array માં સૌથી મોટી સંખ્યા શોધવા માટેનો પ્રોગ્રામ લખો.}

\begin{solutionbox}

\begin{verbatim}
\#include {stdio.h}

int main() \{
    int n, i;
    int largest;
    
    printf("elements ની સંખ્યા દાખલ કરો: ");
    scanf("\%d", \&n);
    
    int arr[n];
    
    printf("\%d સંખ્યાઓ દાખલ કરો:{n}", n);
    for(i = 0; i {} n; i++) \{
        scanf("\%d", \&arr[i]);
    \}
    
    largest = arr[0];  // પ્રથમ element ને largest માનો
    
    for(i = 1; i {} n; i++) \{
        if(arr[i] {} largest) \{
            largest = arr[i];
        \}
    \}
    
    printf("સૌથી મોટી સંખ્યા છે: \%d", largest);
    
    return 0;
\}
\end{verbatim}

\textbf{Algorithm:}

{\def\LTcaptype{none} % do not increment counter
\begin{longtable}[]{@{}ll@{}}
\toprule\noalign{}
પગલું & કાર્ય \\
\midrule\noalign{}
\endhead
\bottomrule\noalign{}
\endlastfoot
1 & Array નું size input કરો \\
2 & Array elements input કરો \\
3 & પ્રથમ element ને largest માનો \\
4 & બાકીના elements સાથે સરખામણી કરો \\
5 & જો મોટું મળે તો largest update કરો \\
6 & પરિણામ દર્શાવો \\
\end{longtable}
}

\end{solutionbox}
\begin{mnemonicbox}
``Input માનો સરખાવો Update દર્શાવો''

\end{mnemonicbox}
\begin{center}\rule{0.5\linewidth}{0.5pt}\end{center}

\subsection*{પ્રશ્ન 4(c) OR [7
ગુણ]}\label{q4c}

\textbf{Array વ્યાખ્યાયિત કરો. Array variable ની જરૂરિયાત સમજાવો. 1-D array
ને ઉદાહરણ સાથે સમજાવો}

\begin{solutionbox}

\textbf{Array વ્યાખ્યા:}

{\def\LTcaptype{none} % do not increment counter
\begin{longtable}[]{@{}ll@{}}
\toprule\noalign{}
પાસાં & વર્ણન \\
\midrule\noalign{}
\endhead
\bottomrule\noalign{}
\endlastfoot
વ્યાખ્યા & સમાન data type ના elements નો સંગ્રહ \\
Storage & સતત memory locations માં \\
Access & Index/subscript વાપરીને \\
\end{longtable}
}

\textbf{Arrays ની જરૂરિયાત:}

{\def\LTcaptype{none} % do not increment counter
\begin{longtable}[]{@{}ll@{}}
\toprule\noalign{}
સમસ્યા & Array સાથે ઉકેલ \\
\midrule\noalign{}
\endhead
\bottomrule\noalign{}
\endlastfoot
બહુવિધ values સ્ટોર કરવા & એક જ array variable \\
ઘણા variables ટાળવા & arr[100] બદલે a1, a2, \ldots, a100 \\
કાર્યક્ષમ processing & Loop-based operations \\
Memory organization & Contiguous allocation \\
\end{longtable}
}

\textbf{1-D Array Declaration:}

\begin{verbatim}
datatype arrayname[size];

// ઉદાહરણો
int marks[5];           // 5 integers ની Array
float prices[10];       // 10 floats ની Array
char name[20];         // 20 characters ની Array
\end{verbatim}

\textbf{Array Initialization:}

\begin{verbatim}
// પદ્ધતિ 1: Declaration વખતે
int numbers[5] = \{10, 20, 30, 40, 50\;}

// પદ્ધતિ 2: વ્યક્તિગત assignment
int arr[3];
arr[0] = 5;
arr[1] = 15;
arr[2] = 25;
\end{verbatim}

\textbf{ઉદાહરણ પ્રોગ્રામ:}

\begin{verbatim}
\#include {stdio.h}

int main() \{
    int marks[5] = \{85, 90, 78, 92, 88\;}
    int i, sum = 0;
    
    printf("વિદ્યાર્થીના ગુણ:{n}");
    for(i = 0; i {} 5; i++) \{
        printf("વિષય \%d: \%d{n}", i+1, marks[i]);
        sum += marks[i];
    \}
    
    printf("કુલ ગુણ: \%d", sum);
    return 0;
\}
\end{verbatim}

\textbf{Memory Layout:}

\begin{verbatim}
marks[0] marks[1] marks[2] marks[3] marks[4]
  [85]     [90]     [78]     [92]     [88]
 1000     1004     1008     1012     1016
\end{verbatim}

\end{solutionbox}
\begin{mnemonicbox}
``સમાન ડેટા સતત Index''

\end{mnemonicbox}
\begin{center}\rule{0.5\linewidth}{0.5pt}\end{center}

\subsection*{પ્રશ્ન 5(a) [3
ગુણ]}\label{q5a}

\textbf{if \ldots{} else statement નું ઉદાહરણ આપો.}

\begin{solutionbox}

\textbf{If-else ઉદાહરણ:}

\begin{verbatim}
\#include {stdio.h}

int main() \{
    int age;
    
    printf("તમારી ઉંમર દાખલ કરો: ");
    scanf("\%d", \&age);
    
    if(age {=} 18) \{
        printf("તમે મતદાન માટે લાયક છો");
    \}
    else \{
        printf("તમે મતદાન માટે લાયક નથી");
    \}
    
    return 0;
\}
\end{verbatim}

\textbf{માળખું:}

{\def\LTcaptype{none} % do not increment counter
\begin{longtable}[]{@{}ll@{}}
\toprule\noalign{}
ઘટક & હેતુ \\
\midrule\noalign{}
\endhead
\bottomrule\noalign{}
\endlastfoot
\textbf{if} & શરત ટેસ્ટ કરે છે \\
\textbf{condition} & Boolean expression \\
\textbf{if-block} & શરત સાચી હોય ત્યારે execute \\
\textbf{else-block} & શરત ખોટી હોય ત્યારે execute \\
\end{longtable}
}

\textbf{Sample Outputs:}

\begin{verbatim}
Input: 20    Output: તમે મતદાન માટે લાયક છો
Input: 16    Output: તમે મતદાન માટે લાયક નથી
\end{verbatim}

\end{solutionbox}
\begin{mnemonicbox}
``જો સાચું નહીંતર ખોટું''

\end{mnemonicbox}
\begin{center}\rule{0.5\linewidth}{0.5pt}\end{center}

\subsection*{પ્રશ્ન 5(b) [4
ગુણ]}\label{q5b}

\textbf{આપેલ character ની category ચકાસવા માટેનો પ્રોગ્રામ લખો.}

\begin{solutionbox}

\begin{verbatim}
\#include {stdio.h}
\#include {ctype.h}

int main() \{
    char ch;
    
    printf("એક character દાખલ કરો: ");
    scanf("\%c", \&ch);
    
    if(isdigit(ch)) \{
        printf("{}\%c{ એ અંક છે"}, ch);
    \}
    else if(isupper(ch)) \{
        printf("{}\%c{ એ મોટા અક્ષર છે"}, ch);
    \}
    else if(islower(ch)) \{
        printf("{}\%c{ એ નાના અક્ષર છે"}, ch);
    \}
    else \{
        printf("{}\%c{ એ વિશેષ પ્રતીક છે"}, ch);
    \}
    
    return 0;
\}
\end{verbatim}

\textbf{Character Categories:}

{\def\LTcaptype{none} % do not increment counter
\begin{longtable}[]{@{}lll@{}}
\toprule\noalign{}
Function & વર્ગ & Range \\
\midrule\noalign{}
\endhead
\bottomrule\noalign{}
\endlastfoot
isdigit() & અંક & 0-9 \\
isupper() & મોટા અક્ષર & A-Z \\
islower() & નાના અક્ષર & a-z \\
Others & વિશેષ પ્રતીકો & !@\#\$\%\^{}\&* etc. \\
\end{longtable}
}

\textbf{વૈકલ્પિક પદ્ધતિ:}

\begin{verbatim}
if(ch {=} {0} \&\& ch {=} {9})
    printf("અંક");
else if(ch {=} {A} \&\& ch {=} {Z})
    printf("મોટા અક્ષર");
else if(ch {=} {a} \&\& ch {=} {z})
    printf("નાના અક્ષર");
else
    printf("વિશેષ પ્રતીક");
\end{verbatim}

\end{solutionbox}
\begin{mnemonicbox}
``અંક મોટા નાના વિશેષ''

\end{mnemonicbox}
\begin{center}\rule{0.5\linewidth}{0.5pt}\end{center}

\subsection*{પ્રશ્ન 5(c) [7
ગુણ]}\label{q5c}

\textbf{Structure શું છે? તેની syntax યોગ્ય ઉદાહરણ સાથે સમજાવો}

\begin{solutionbox}

\textbf{Structure વ્યાખ્યા:}

{\def\LTcaptype{none} % do not increment counter
\begin{longtable}[]{@{}ll@{}}
\toprule\noalign{}
પાસાં & વર્ણન \\
\midrule\noalign{}
\endhead
\bottomrule\noalign{}
\endlastfoot
વ્યાખ્યા & વિવિધ data types ને જોડીને બનાવેલ user-defined data type \\
હેતુ & સંબંધિત data ને એક જ નામ હેઠળ જૂથ બનાવવા \\
Keyword & struct \\
\end{longtable}
}

\textbf{Syntax:}

\begin{verbatim}
struct structure\_name \{
    datatype member1;
    datatype member2;
    ...
\;}
\end{verbatim}

\textbf{ઉદાહરણ - Student Structure:}

\begin{verbatim}
\#include {stdio.h}

struct Student \{
    int roll\_no;
    char name[50];
    float marks;
    char grade;
\;}

int main() \{
    struct Student s1;
    
    // Data input
    printf("રોલ નંબર દાખલ કરો: ");
    scanf("\%d", \&s1.roll\_no);
    
    printf("નામ દાખલ કરો: ");
    scanf("\%s", s1.name);
    
    printf("ગુણ દાખલ કરો: ");
    scanf("\%f", \&s1.marks);
    
    printf("ગ્રેડ દાખલ કરો: ");
    scanf(" \%c", \&s1.grade);
    
    // Data display
    printf("{n}વિદ્યાર્થીની વિગતો:{n}");
    printf("રોલ નં: \%d{n}", s1.roll\_no);
    printf("નામ: \%s{n}", s1.name);
    printf("ગુણ: \%.2f{n}", s1.marks);
    printf("ગ્રેડ: \%c{n}", s1.grade);
    
    return 0;
\}
\end{verbatim}

\textbf{Structure લક્ષણો:}

{\def\LTcaptype{none} % do not increment counter
\begin{longtable}[]{@{}ll@{}}
\toprule\noalign{}
લક્ષણ & વર્ણન \\
\midrule\noalign{}
\endhead
\bottomrule\noalign{}
\endlastfoot
\textbf{Dot operator (.)} & Structure members ને access કરવા \\
\textbf{Memory allocation} & કુલ size = બધા members નો સરવાળો \\
\textbf{Initialization} & Declaration વખતે initialize કરી શકાય \\
\end{longtable}
}

\textbf{Structure Initialization:}

\begin{verbatim}
struct Student s1 = \{101, "John", 85.5, {A}\;}
\end{verbatim}

\textbf{Memory Layout:}

\begin{verbatim}
s1: [roll_no][name...][marks][grade]
     4 bytes  50 bytes 4 bytes 1 byte
\end{verbatim}

\end{solutionbox}
\begin{mnemonicbox}
``સંબંધિત ડેટાને એકસાથે જૂથ બનાવો''

\end{mnemonicbox}
\begin{center}\rule{0.5\linewidth}{0.5pt}\end{center}

\subsection*{પ્રશ્ન 5(a) OR [3
ગુણ]}\label{q5a}

\textbf{-5 અને +5 વચ્ચેના બધા numbers print કરવા માટેનો પ્રોગ્રામ લખો.}

\begin{solutionbox}

\begin{verbatim}
\#include {stdio.h}

int main() \{
    int i;
    
    printf("{-5 અને +5 વચ્ચેના નંબરો:}{n}");
    
    for(i = {-}5; i {=} 5; i++) \{
        printf("\%d ", i);
    \}
    
    return 0;
\}
\end{verbatim}

\textbf{Output:}

\begin{verbatim}
-5 અને +5 વચ્ચેના નંબરો:
-5 -4 -3 -2 -1 0 1 2 3 4 5
\end{verbatim}

\textbf{વૈકલ્પિક પદ્ધતિઓ:}

\begin{verbatim}
// પદ્ધતિ 2: while loop વાપરીને
int i = {-}5;
while(i {=} 5) \{
    printf("\%d ", i);
    i++;
\}

// પદ્ધતિ 3: બે અલગ loops
for(i = {-}5; i {} 0; i++)
    printf("\%d ", i);
printf("0 ");
for(i = 1; i {=} 5; i++)
    printf("\%d ", i);
\end{verbatim}

\end{solutionbox}
\begin{mnemonicbox}
``નકારાત્મકથી શરૂ સકારાત્મકે અંત''

\end{mnemonicbox}
\begin{center}\rule{0.5\linewidth}{0.5pt}\end{center}

\subsection*{પ્રશ્ન 5(b) OR [4
ગુણ]}\label{q5b}

\textbf{Quadratic equation ના roots શોધવા માટેનો પ્રોગ્રામ લખો.}

\begin{solutionbox}

\begin{verbatim}
\#include {stdio.h}
\#include {math.h}

int main() \{
    float a, b, c;
    float discriminant, root1, root2;
    
    printf("coefficients (a, b, c) દાખલ કરો: ");
    scanf("\%f \%f \%f", \&a, \&b, \&c);
    
    discriminant = b*b {-} 4*a*c;
    
    if(discriminant {} 0) \{
        root1 = ({-}b + sqrt(discriminant)) / (2*a);
        root2 = ({-}b {-} sqrt(discriminant)) / (2*a);
        printf("Roots વાસ્તવિક અને અલગ છે{n}");
        printf("Root1 = \%.2f{n}", root1);
        printf("Root2 = \%.2f{n}", root2);
    \}
    else if(discriminant == 0) \{
        root1 = {-}b / (2*a);
        printf("Roots વાસ્તવિક અને સમાન છે{n}");
        printf("Root = \%.2f{n}", root1);
    \}
    else \{
        float realPart = {-}b / (2*a);
        float imagPart = sqrt({-}discriminant) / (2*a);
        printf("Roots સંકુલ છે{n}");
        printf("Root1 = \%.2f + \%.2fi{n}", realPart, imagPart);
        printf("Root2 = \%.2f {- }\%.2fi{n}", realPart, imagPart);
    \}
    
    return 0;
\}
\end{verbatim}

\textbf{Quadratic Formula વિશ્લેષણ:}

{\def\LTcaptype{none} % do not increment counter
\begin{longtable}[]{@{}ll@{}}
\toprule\noalign{}
Discriminant & Roots નો પ્રકાર \\
\midrule\noalign{}
\endhead
\bottomrule\noalign{}
\endlastfoot
\textbf{b^{2}-4ac \textgreater{} 0} & વાસ્તવિક અને અલગ \\
\textbf{b^{2}-4ac = 0} & વાસ્તવિક અને સમાન \\
\textbf{b^{2}-4ac \textless{} 0} & સંકુલ (કાલ્પનિક) \\
\end{longtable}
}

\textbf{સૂત્ર:} x = (-b \pm \sqrt(b^{2}-4ac)) / 2a

\textbf{Sample Output:}

\begin{verbatim}
coefficients દાખલ કરો: 1 -7 12
Roots વાસ્તવિક અને અલગ છે
Root1 = 4.00
Root2 = 3.00
\end{verbatim}

\end{solutionbox}
\begin{mnemonicbox}
``Discriminant Roots નો પ્રકાર નક્કી કરે છે''

\end{mnemonicbox}
\begin{center}\rule{0.5\linewidth}{0.5pt}\end{center}

\subsection*{પ્રશ્ન 5(c) OR [7
ગુણ]}\label{q5c}

\textbf{નીચેના built-in functions ને ઉદાહરણો સાથે સમજાવો}

\begin{solutionbox}

\textbf{Function સમજૂતીઓ:}

{\def\LTcaptype{none} % do not increment counter
\begin{longtable}[]{@{}llll@{}}
\toprule\noalign{}
Function & હેતુ & Header File & ઉદાહરણ \\
\midrule\noalign{}
\endhead
\bottomrule\noalign{}
\endlastfoot
clrscr() & Screen સાફ કરવા & conio.h & clrscr(); \\
sqrt() & વર્ગમૂળ & math.h & sqrt(16) = 4.0 \\
strlen() & String ની લંબાઈ & string.h & strlen(``Hello'') = 5 \\
isdigit() & અંક છે કે કેમ ચકાસવા & ctype.h & isdigit(`5') = true \\
isalpha() & અક્ષર છે કે કેમ ચકાસવા & ctype.h & isalpha(`A') = true \\
toupper() & મોટા અક્ષરમાં બદલવા & ctype.h & toupper(`a') = `A' \\
tolower() & નાના અક્ષરમાં બદલવા & ctype.h & tolower(`B') = `b' \\
\end{longtable}
}

\textbf{ઉદાહરણ પ્રોગ્રામ:}

\begin{verbatim}
\#include {stdio.h}
\#include {conio.h}
\#include {math.h}
\#include {string.h}
\#include {ctype.h}

int main() \{
    clrscr();  // Screen સાફ કરો
    
    // sqrt() ઉદાહરણ
    float num = 25.0;
    printf("\%.1f નું વર્ગમૂળ = \%.2f{n}", num, sqrt(num));
    
    // strlen() ઉદાહરણ
    char str[] = "Programming";
    printf("{}\%s{ ની લંબાઈ = }\%d{n}", str, strlen(str));
    
    // Character functions
    char ch = {a};
    printf("{}\%c{ અંક છે: }\%s{n}", ch, isdigit(ch) ? "હા" : "ના");
    printf("{}\%c{ અક્ષર છે: }\%s{n}", ch, isalpha(ch) ? "હા" : "ના");
    printf("{}\%c{ નું મોટું અક્ષર = }\%c{}{n}", ch, toupper(ch));
    
    ch = {B};
    printf("{}\%c{ નું નાનું અક્ષર = }\%c{}{n}", ch, tolower(ch));
    
    return 0;
\}
\end{verbatim}

\textbf{Function વર્ગીકરણ:}

\begin{center}
\textbf{Mermaid Diagram (Code)}
\begin{verbatim}
{Shaded}
{Highlighting}[]
graph TD
    A[Built{-in Functions] {-}{-}{} B[Screen Control]}
    A {-{-}{} C[Mathematical]}
    A {-{-}{} D[String]}
    A {-{-}{} E[Character]}
    
    B {-{-}{} F["clrscr()"]}
    C {-{-}{} G["sqrt()"]}
    D {-{-}{} H["strlen()"]}
    E {-{-}{} I["isdigit(), isalpha()"]}
    E {-{-}{} J["toupper(), tolower()"]}
{Highlighting}
{Shaded}
\end{verbatim}
\end{center}

\textbf{મુખ્ય મુદ્દાઓ:}

\begin{itemize}
\tightlist
\item
  \textbf{Header files}: યોગ્ય headers include કરવા જરૂરી
\item
  \textbf{Return values}: મોટાભાગના functions ચોક્કસ types return કરે છે
\item
  \textbf{Parameter types}: Function parameter requirements ચકાસો
\end{itemize}

\end{solutionbox}
\begin{mnemonicbox}
``સાફ ગણિત String Character''

\end{mnemonicbox}

\end{document}
