\documentclass[10pt,a4paper]{article}

% content/resources/templates/preamble.tex
\usepackage[margin=0.6in]{geometry}
\author{Milav Dabgar}
\usepackage{amsmath,amssymb,amsthm}
\usepackage{booktabs}
\usepackage{multirow}
\usepackage{xcolor}
\usepackage{tcolorbox}
\tcbuselibrary{breakable,skins}
\usepackage[colorlinks=true,linkcolor=blue]{hyperref}
\usepackage{titlesec}
\usepackage{enumitem}
\usepackage{tikz}
\usepackage{pgfplots}
\usepackage{circuitikz}
\usepackage[version=4]{mhchem}
\usepackage{longtable}
\usepackage{array}
\usepackage{float}
\usepackage{caption}
\usepackage{listings}

\lstset{
  basicstyle=\small\ttfamily,
  breaklines=true,
  breakatwhitespace=false,
  postbreak=\mbox{\textcolor{red}{$\hookrightarrow$}\space},
  float=false,
  numbers=left,
  numberstyle=\tiny\color{gray},
  numbersep=10pt,
  xleftmargin=2em,
  keywordstyle=\color{blue},
  commentstyle=\color{green!60!black},
  stringstyle=\color{purple},
  backgroundcolor=\color{gray!5},
  showstringspaces=false,
  tabsize=2,
  captionpos=b,
  keepspaces=true,
  columns=flexible
}

\pgfplotsset{compat=1.18}
\usetikzlibrary{shapes,arrows,positioning,calc,patterns,decorations.pathmorphing,decorations.markings,arrows.meta}

% Color scheme
\definecolor{headcolor}{RGB}{0,102,204}
\definecolor{keycolor}{RGB}{220,20,60}
\definecolor{solutioncolor}{RGB}{34,139,34}
\definecolor{mnemoniccolor}{RGB}{148,0,211}
\definecolor{codecolor}{RGB}{0,0,100}

% Spacing
\setlength{\parskip}{3pt}
\setlist[itemize]{nosep}
\setlist[enumerate]{nosep}

% Title formatting
\titleformat{\section}{\Large\bfseries\color{headcolor}}{\thesection}{1em}{}
\titleformat{\subsection}{\large\bfseries\color{headcolor}}{\thesubsection}{1em}{}

% Pandoc tightlist compatibility
\providecommand{\tightlist}{%
  \setlength{\itemsep}{0pt}\setlength{\parskip}{0pt}}

% Pandoc longtable compatibility
\newcounter{none}
\def\thenone{}


% content/resources/templates/english-boxes.tex
% This file is currently empty - it exists to maintain consistency with the import structure.
% Add custom environments here if needed in the future.


\begin{document}

\begin{center}
{\Huge\bfseries\color{headcolor} Subject Name Solutions}\\[5pt]
{\LARGE 4331105 -- Summer 2024}\\[3pt]
{\large Semester 1 Study Material}\\[3pt]
{\normalsize\textit{Detailed Solutions and Explanations}}
\end{center}

\vspace{10pt}

\subsection*{Question 1(a) [3 marks]}\label{q1a}

\textbf{Define keyword. List any four keywords for C language.}

\begin{solutionbox}
A keyword is a predefined, reserved word in C that has
special meaning to the compiler and cannot be used as an identifier.


{\def\LTcaptype{none} % do not increment counter
\vspace{-5pt}
\captionof{table}{Common C Keywords}
\vspace{-10pt}
\begin{longtable}[]{@{}ll@{}}
\toprule\noalign{}
Keyword & Purpose \\
\midrule\noalign{}
\endhead
\bottomrule\noalign{}
\endlastfoot
int & Integer data type \\
float & Floating-point data type \\
char & Character data type \\
if & Conditional statement \\
for & Loop statement \\
while & Loop statement \\
void & Return type/parameter \\
return & Return value from function \\
\end{longtable}
}

\begin{itemize}
\tightlist
\item
  \textbf{Reserved words}: Keywords cannot be used as variable names
\item
  \textbf{Pre-defined}: They have fixed meaning in the language
\item
  \textbf{Case-sensitive}: All keywords must be in lowercase
\end{itemize}

\end{solutionbox}
\begin{mnemonicbox}
``If VoId FoR WhIle'' (first letters of important
keywords)

\end{mnemonicbox}
\subsection*{Question 1(b) [4 marks]}\label{q1b}

\textbf{Explain rules for naming a variable.}

\begin{solutionbox}
Variables in C must follow specific naming rules to be
valid identifiers.


{\def\LTcaptype{none} % do not increment counter
\vspace{-5pt}
\captionof{table}{Variable Naming Rules in C}
\vspace{-10pt}
\begin{longtable}[]{@{}
  >{\raggedright\arraybackslash}p{(\linewidth - 6\tabcolsep) * \real{0.1176}}
  >{\raggedright\arraybackslash}p{(\linewidth - 6\tabcolsep) * \real{0.2549}}
  >{\raggedright\arraybackslash}p{(\linewidth - 6\tabcolsep) * \real{0.2941}}
  >{\raggedright\arraybackslash}p{(\linewidth - 6\tabcolsep) * \real{0.3333}}@{}}
\toprule\noalign{}
\begin{minipage}[b]{\linewidth}\raggedright
Rule
\end{minipage} & \begin{minipage}[b]{\linewidth}\raggedright
Description
\end{minipage} & \begin{minipage}[b]{\linewidth}\raggedright
Valid Example
\end{minipage} & \begin{minipage}[b]{\linewidth}\raggedright
Invalid Example
\end{minipage} \\
\midrule\noalign{}
\endhead
\bottomrule\noalign{}
\endlastfoot
First character & Must be a letter or underscore & age, \_count &
1value \\
Subsequent characters & Letters, digits, or underscores & user\_1,
total99 & user@1 \\
Case sensitivity & Uppercase and lowercase are different & Value \neq value
& - \\
Keywords & Cannot use reserved keywords & counter & int \\
Length & Should be meaningful but not too long & studentMarks & sm \\
Special characters & Not allowed & firstName & first-name \\
\end{longtable}
}

\begin{itemize}
\tightlist
\item
  \textbf{Descriptive names}: Use meaningful names that indicate purpose
\item
  \textbf{Consistent style}: Follow a consistent naming convention
\item
  \textbf{No spaces}: Use underscores or camelCase instead
\end{itemize}

\end{solutionbox}
\begin{mnemonicbox}
``FLASKS'' (First Letter, Letters/digits, Avoid
keywords, Sensitive case, Keep meaningful, Skip special chars)

\end{mnemonicbox}
\subsection*{Question 1(c) [7 marks]}\label{q1c}

\textbf{Define flowchart. Draw flowchart to find minimum of three
integer numbers N1, N2 and N3.}

\begin{solutionbox}
A flowchart is a graphical representation of an
algorithm showing the steps as boxes and their order by connecting them
with arrows.

\textbf{Diagram:}

\begin{verbatim}
flowchart LR
    A([Start]) {-{-} B[/Input N1, N2, N3/]}
    B {-{-} C\{Is N1  N2?\}}
    C {-{-}|Yes| D[min = N1]}
    C {-{-}|No| E[min = N2]}
    D {-{-} F\{Is min  N3?\}}
    E {-{-} F}
    F {-{-}|Yes| G[min remains same]}
    F {-{-}|No| H[min = N3]}
    G {-{-} I[/Output min/]}
    H {-{-} I}
    I {-{-} J([End])}
\end{verbatim}

\begin{itemize}
\tightlist
\item
  \textbf{Symbols used}: Oval (start/end), Parallelogram (input/output),
  Diamond (decision), Rectangle (process)
\item
  \textbf{Decision points}: Compare values systematically
\item
  \textbf{Logical flow}: Arrows show the sequence of operations
\end{itemize}

\end{solutionbox}
\begin{mnemonicbox}
``Start-Input-Compare-Output-End'' (SICOE)

\end{mnemonicbox}
\subsection*{Question 1(c) OR [7
marks]}\label{q1c}

\textbf{Define algorithm. Write an algorithm to find minimum of three
integer numbers N1, N2 and N3.}

\begin{solutionbox}
An algorithm is a step-by-step procedure or finite set
of well-defined instructions to solve a particular problem.

\textbf{Algorithm to find minimum of three numbers:}

\begin{verbatim}
Step 1: Start
Step 2: Input three numbers N1, N2, and N3
Step 3: Set min = N1 (assume first number is minimum)
Step 4: If N2 < min, then set min = N2
Step 5: If N3 < min, then set min = N3
Step 6: Output min as the minimum number
Step 7: End
\end{verbatim}


{\def\LTcaptype{none} % do not increment counter
\vspace{-5pt}
\captionof{table}{Algorithm Characteristics}
\vspace{-10pt}
\begin{longtable}[]{@{}ll@{}}
\toprule\noalign{}
Characteristic & Description \\
\midrule\noalign{}
\endhead
\bottomrule\noalign{}
\endlastfoot
Finiteness & Algorithm must terminate after finite steps \\
Definiteness & Each step must be precisely defined \\
Input & Algorithm takes zero or more inputs \\
Output & Algorithm produces one or more outputs \\
Effectiveness & Steps must be simple and executable \\
\end{longtable}
}

\begin{itemize}
\tightlist
\item
  \textbf{Sequential steps}: Follows a logical order
\item
  \textbf{Comparative approach}: Systematically finds minimum
\item
  \textbf{Simplicity}: Easy to understand and implement
\end{itemize}

\end{solutionbox}
\begin{mnemonicbox}
``FIDEO'' (Finiteness, Input, Definiteness,
Effectiveness, Output)

\end{mnemonicbox}
\subsection*{Question 2(a) [3 marks]}\label{q2a}

\textbf{Differentiate gets() and puts().}

\begin{solutionbox}
gets() and puts() are standard library functions in C
for input and output operations with strings.


{\def\LTcaptype{none} % do not increment counter
\vspace{-5pt}
\captionof{table}{Comparison of gets() and puts()}
\vspace{-10pt}
\begin{longtable}[]{@{}
  >{\raggedright\arraybackslash}p{(\linewidth - 4\tabcolsep) * \real{0.3600}}
  >{\raggedright\arraybackslash}p{(\linewidth - 4\tabcolsep) * \real{0.3200}}
  >{\raggedright\arraybackslash}p{(\linewidth - 4\tabcolsep) * \real{0.3200}}@{}}
\toprule\noalign{}
\begin{minipage}[b]{\linewidth}\raggedright
Feature
\end{minipage} & \begin{minipage}[b]{\linewidth}\raggedright
gets()
\end{minipage} & \begin{minipage}[b]{\linewidth}\raggedright
puts()
\end{minipage} \\
\midrule\noalign{}
\endhead
\bottomrule\noalign{}
\endlastfoot
Purpose & Reads string from stdin & Writes string to stdout \\
Prototype & char \emph{gets(char }str) & int puts(const char *str) \\
Behavior & Reads until newline & Adds newline automatically \\
Return value & Returns str on success, NULL on failure & Returns
non-negative on success, EOF on error \\
Safety & Unsafe (buffer overflow risk) & Safe \\
Recommended & No (deprecated) & Yes \\
\end{longtable}
}

\begin{itemize}
\tightlist
\item
  \textbf{Input/Output}: gets() for input, puts() for output
\item
  \textbf{Termination}: gets() stops at newline, puts() adds newline
\item
  \textbf{Security}: gets() has no buffer limit check
\end{itemize}

\end{solutionbox}
\begin{mnemonicbox}
``Gets In, Puts Out'' (gets reads in, puts writes
out)

\end{mnemonicbox}
\subsection*{Question 2(b) [4 marks]}\label{q2b}

\textbf{Develop a C program to find whether the entered number is even
or odd using conditional operator.}

\begin{solutionbox}
This program uses the conditional operator to check if
a number is even or odd.

\begin{verbatim}
\#include {stdio.h}

int main() \{
    int num;
    
    printf("Enter a number: ");
    scanf("\%d", \&num);
    
    // Using conditional operator to check even or odd
    (num \% 2 == 0) ? printf("\%d is even{n}", num) : printf("\%d is odd{n}", num);
    
    return 0;
\}
\end{verbatim}

\textbf{Diagram:}

\begin{verbatim}
flowchart LR
    A([Start]) {-{-} B[/Input num/]}
    B {-{-} C\{"num \% 2 == 0?"\}}
    C {-{-}|True| D[/Output "num is even"/]}
    C {-{-}|False| E[/Output "num is odd"/]}
    D {-{-} F([End])}
    E {-{-} F}
\end{verbatim}

\begin{itemize}
\tightlist
\item
  \textbf{Conditional operator}: ? : is a ternary operator
\item
  \textbf{Modulus operation}: \% gives remainder after division
\item
  \textbf{Test condition}: num \% 2 == 0 checks for even number
\end{itemize}

\end{solutionbox}
\begin{mnemonicbox}
``REMinder 0 = Even'' (Remainder 0 means Even)

\end{mnemonicbox}
\subsection*{Question 2(c) [7 marks]}\label{q2c}

\textbf{Explain logical \& relational operators with examples.}

\begin{solutionbox}
Logical and relational operators are used to create
conditions and make decisions in C programs.


{\def\LTcaptype{none} % do not increment counter
\vspace{-5pt}
\captionof{table}{Relational Operators}
\vspace{-10pt}
\begin{longtable}[]{@{}llll@{}}
\toprule\noalign{}
Operator & Meaning & Example & Result \\
\midrule\noalign{}
\endhead
\bottomrule\noalign{}
\endlastfoot
== & Equal to & 5 == 5 & true (1) \\
!= & Not equal to & 5 != 3 & true (1) \\
\textgreater{} & Greater than & 7 \textgreater{} 3 & true (1) \\
\textless{} & Less than & 2 \textless{} 8 & true (1) \\
\textgreater= & Greater than or equal to & 4 \textgreater= 4 & true
(1) \\
\textless= & Less than or equal to & 6 \textless= 9 & true (1) \\
\end{longtable}
}


{\def\LTcaptype{none} % do not increment counter
\vspace{-5pt}
\captionof{table}{Logical Operators}
\vspace{-10pt}
\begin{longtable}[]{@{}llll@{}}
\toprule\noalign{}
Operator & Meaning & Example & Result \\
\midrule\noalign{}
\endhead
\bottomrule\noalign{}
\endlastfoot
\&\& & Logical AND & (5\textgreater3) \&\& (8\textgreater5) & true
(1) \\
\textbar\textbar{} & Logical OR & (5\textgreater7) \textbar\textbar{}
(3\textless6) & true (1) \\
! & Logical NOT & !(5\textgreater7) & true (1) \\
\end{longtable}
}

\textbf{Code Example:}

\begin{verbatim}
int age = 20;
int score = 75;

// Using both relational and logical operators
if ((age {=} 18) \&\& (score {} 70)) \{
    printf("Eligible");
\}
\end{verbatim}

\begin{itemize}
\tightlist
\item
  \textbf{Comparison}: Relational operators compare values
\item
  \textbf{Combining conditions}: Logical operators connect multiple
  conditions
\item
  \textbf{Truth value}: All operators return 1 (true) or 0 (false)
\end{itemize}

\end{solutionbox}
\begin{mnemonicbox}
``CORNL'' (Compare with relational, OR/AND/NOT with
logical)

\end{mnemonicbox}
\subsection*{Question 2(a) OR [3
marks]}\label{q2a}

\textbf{Considering precedence of operators, write down each step of
evaluation and final answer if expression 16 + ( 216 / ( ( 3 + 6 ) * 12
) ) -10 is evaluated.}

\begin{solutionbox}
Let's evaluate the expression 16 + ( 216 / ( ( 3 + 6 )
* 12 ) ) - 10 step by step following operator precedence.


{\def\LTcaptype{none} % do not increment counter
\vspace{-5pt}
\captionof{table}{Step-by-Step Evaluation}
\vspace{-10pt}
\begin{longtable}[]{@{}lll@{}}
\toprule\noalign{}
Step & Operation & Expression after this step \\
\midrule\noalign{}
\endhead
\bottomrule\noalign{}
\endlastfoot
1 & Calculate (3 + 6) & 16 + ( 216 / ( 9 * 12 ) ) - 10 \\
2 & Calculate (9 * 12) & 16 + ( 216 / 108 ) - 10 \\
3 & Calculate (216 / 108) & 16 + 2 - 10 \\
4 & Calculate 16 + 2 & 18 - 10 \\
5 & Calculate 18 - 10 & 8 \\
\end{longtable}
}

\textbf{Final Answer: 8}

\textbf{Diagram:}

\begin{verbatim}
flowchart LR
    A["16 + ( 216 / ( ( 3 + 6 ) * 12 ) ) {- 10"] {-}{-} B["16 + ( 216 / ( 9 * 12 ) ) {-} 10"]}
    B {-{-} C["16 + ( 216 / 108 ) {-} 10"]}
    C {-{-} D[16 + 2 {-} 10]}
    D {-{-} E[18 {-} 10]}
    E {-{-} F[8]}
\end{verbatim}

\begin{itemize}
\tightlist
\item
  \textbf{Parentheses first}: Innermost parentheses evaluated first
\item
  \textbf{Multiplication before division}: Calculate from left to right
\item
  \textbf{Addition and subtraction last}: From left to right
\end{itemize}

\end{solutionbox}
\begin{mnemonicbox}
``PEMDAS'' (Parentheses, Exponents,
Multiplication/Division, Addition/Subtraction)

\end{mnemonicbox}
\subsection*{Question 2(b) OR [4
marks]}\label{q2b}

\textbf{Write a C program to find circumference and area of a circle.}

\begin{solutionbox}
This program calculates the area and circumference of a
circle based on its radius.

\begin{verbatim}
\#include {stdio.h}
\#define PI 3.14159

int main() \{
    float radius, area, circumference;
    
    printf("Enter the radius of circle: ");
    scanf("\%f", \&radius);
    
    // Calculate area and circumference
    area = PI * radius * radius;
    circumference = 2 * PI * radius;
    
    printf("Area of circle = \%.2f square units{n}", area);
    printf("Circumference of circle = \%.2f units{n}", circumference);
    
    return 0;
\}
\end{verbatim}

\textbf{Diagram:}

\begin{verbatim}
flowchart LR
    A([Start]) {-{-} B[/Input radius/]}
    B {-{-} C[area = PI * radius * radius]}
    C {-{-} D[circumference = 2 * PI * radius]}
    D {-{-} E[/Output area and circumference/]}
    E {-{-} F([End])}
\end{verbatim}

\begin{itemize}
\tightlist
\item
  \textbf{Formula}: Area = π \times r^{2} and Circumference = 2π \times r
\item
  \textbf{Constant definition}: Using \#define for PI
\item
  \textbf{Float variables}: For decimal precision
\end{itemize}

\end{solutionbox}
\begin{mnemonicbox}
``PIR^{2}'' for area, ``2PIR'' for circumference

\end{mnemonicbox}
\subsection*{Question 2(c) OR [7
marks]}\label{q2c}

\textbf{Explain arithmetic \& bit-wise operators with examples.}

\begin{solutionbox}
Arithmetic operators perform mathematical operations
while bit-wise operators manipulate individual bits of integers.


{\def\LTcaptype{none} % do not increment counter
\vspace{-5pt}
\captionof{table}{Arithmetic Operators}
\vspace{-10pt}
\begin{longtable}[]{@{}llll@{}}
\toprule\noalign{}
Operator & Description & Example & Result \\
\midrule\noalign{}
\endhead
\bottomrule\noalign{}
\endlastfoot
+ & Addition & 5 + 3 & 8 \\
- & Subtraction & 7 - 2 & 5 \\
* & Multiplication & 4 * 3 & 12 \\
/ & Division & 10 / 3 & 3 (integer division) \\
\% & Modulus (Remainder) & 10 \% 3 & 1 \\
++ & Increment & a++ & Adds 1 after using value \\
-- & Decrement & --b & Subtracts 1 before using value \\
\end{longtable}
}


{\def\LTcaptype{none} % do not increment counter
\vspace{-5pt}
\captionof{table}{Bitwise Operators}
\vspace{-10pt}
\begin{longtable}[]{@{}llll@{}}
\toprule\noalign{}
Operator & Description & Example (binary) & Result \\
\midrule\noalign{}
\endhead
\bottomrule\noalign{}
\endlastfoot
\& & Bitwise AND & 5 (101) \& 3 (011) & 1 (001) \\
\textbar{} & Bitwise OR & 5 (101) \textbar{} 3 (011) & 7 (111) \\
\^{} & Bitwise XOR & 5 (101) \^{} 3 (011) & 6 (110) \\
\textasciitilde{} & Bitwise NOT & \textasciitilde5 (101) & -6 (depends
on bits) \\
\textless\textless{} & Left Shift & 5 \textless\textless{} 1 & 10
(1010) \\
\textgreater\textgreater{} & Right Shift & 5 \textgreater\textgreater{}
1 & 2 (10) \\
\end{longtable}
}

\textbf{Code Example:}

\begin{verbatim}
int

a = 5,

b = 3;

printf("a + b = \%d{n}", a + b);      // 8
printf("a \& b = \%d{n}", a \& b);      // 1
printf("a { 1 = }\%d{n}", a {} 1);    // 10
\end{verbatim}

\begin{itemize}
\tightlist
\item
  \textbf{Mathematical operations}: Arithmetic operators for
  calculations
\item
  \textbf{Bit manipulation}: Bitwise operators work at binary level
\item
  \textbf{Efficiency}: Bitwise operations are faster for certain tasks
\end{itemize}

\end{solutionbox}
\begin{mnemonicbox}
``SAME BARON'' (Subtraction Addition Multiplication,
Bitwise AND/OR/NOT)

\end{mnemonicbox}
\subsection*{Question 3(a) [3 marks]}\label{q3a}

\textbf{Explain the use of `go to' statement with example.}

\begin{solutionbox}
The goto statement is used to transfer program control
unconditionally to a labeled statement.

\begin{verbatim}
\#include {stdio.h}

int main() \{
    int num, sum = 0;
    
    printf("Enter a positive number: ");
    scanf("\%d", \&num);
    
    if (num {=} 0) \{
        goto error;
    \}
    
    sum = num * (num + 1) / 2;
    printf("Sum of first \%d numbers = \%d{n}", num, sum);
    goto end;
    
    error:
        printf("Error: Please enter a positive number!{n}");
    
    end:
        return 0;
\}
\end{verbatim}

\textbf{Diagram:}

\begin{verbatim}
flowchart LR
    A([Start]) {-{-} B[/Input num/]}
    B {-{-} C\{num = 0?\}}
    C {-{-}|Yes| D[/Output error message/]}
    C {-{-}|No| E[Calculate sum]}
    E {-{-} F[/Output sum/]}
    D {-{-} G([End])}
    F {-{-} G}
\end{verbatim}

\begin{itemize}
\tightlist
\item
  \textbf{Label declaration}: Labels end with colon (:)
\item
  \textbf{Jump statement}: goto transfers control to label
\item
  \textbf{Caution}: Excessive use creates ``spaghetti code''
\end{itemize}

\end{solutionbox}
\begin{mnemonicbox}
``JUMPing LABEL'' (Jump to a labeled statement)

\end{mnemonicbox}
\subsection*{Question 3(b) [4 marks]}\label{q3b}

\textbf{The marks obtained by the student in 5 different subjects are
input through keyboard. The student gets grade as per following rules:
Percentage above or equal to 90- Grade A. Percentage between 80 and 89-
Grade B. Percentage between 70 and 79-Grade C. Percentage between 60 and
69-Grade D. Percentage between 50 and 59-Grade E. Percentage less than
50- Grade F. Write a C program to display the grade obtained by the
student.}

\begin{solutionbox}
This program calculates the grade based on the average
marks in 5 subjects.

\begin{verbatim}
\#include {stdio.h}

int main() \{
    int marks[5], total = 0, i;
    float percentage;
    char grade;
    
    // Input marks for 5 subjects
    for (i = 0; i {} 5; i++) \{
        printf("Enter marks for subject \%d (out of 100): ", i+1);
        scanf("\%d", \&marks[i]);
        total += marks[i];
    \}
    
    // Calculate percentage
    percentage = total / 5.0;
    
    // Determine grade
    if (percentage {=} 90)
        grade = {A};
    else if (percentage {=} 80)
        grade = {B};
    else if (percentage {=} 70)
        grade = {C};
    else if (percentage {=} 60)
        grade = {D};
    else if (percentage {=} 50)
        grade = {E};
    else
        grade = {F};
    
    printf("Percentage: \%.2f\%\%{n}", percentage);
    printf("Grade: \%c{n}", grade);
    
    return 0;
\}
\end{verbatim}


{\def\LTcaptype{none} % do not increment counter
\vspace{-5pt}
\captionof{table}{Grading Criteria}
\vspace{-10pt}
\begin{longtable}[]{@{}ll@{}}
\toprule\noalign{}
Percentage Range & Grade \\
\midrule\noalign{}
\endhead
\bottomrule\noalign{}
\endlastfoot
\geq 90 & A \\
80-89 & B \\
70-79 & C \\
60-69 & D \\
50-59 & E \\
\textless{} 50 & F \\
\end{longtable}
}

\begin{itemize}
\tightlist
\item
  \textbf{Input array}: Stores marks of 5 subjects
\item
  \textbf{Percentage calculation}: Sum divided by number of subjects
\item
  \textbf{Grade determination}: Using if-else ladder
\end{itemize}

\end{solutionbox}
\begin{mnemonicbox}
``ABCDEF-90-80-70-60-50'' (Grades with their
percentage thresholds)

\end{mnemonicbox}
\subsection*{Question 3(c) [7 marks]}\label{q3c}

\textbf{Draw flowchart and explain nested if-else with example.}

\begin{solutionbox}
Nested if-else is a control structure where an if or
else statement contains another if-else statement within it.

\textbf{Diagram:}

\begin{verbatim}
flowchart LR
    A([Start]) {-{-} B[/Input age, score/]}
    B {-{-} C\{age = 18?\}}
    C {-{-}|Yes| D\{score = 60?\}}
    C {-{-}|No| E[/Output "Not eligible: Age criteria not met"/]}
    D {-{-}|Yes| F[/Output "Eligible for admission"/]}
    D {-{-}|No| G[/Output "Not eligible: Score criteria not met"/]}
    E {-{-} H([End])}
    F {-{-} H}
    G {-{-} H}
\end{verbatim}

\textbf{Code Example:}

\begin{verbatim}
\#include {stdio.h}

int main() \{
    int age, score;
    
    printf("Enter age: ");
    scanf("\%d", \&age);
    printf("Enter score: ");
    scanf("\%d", \&score);
    
    if (age {=} 18) \{
        if (score {=} 60) \{
            printf("Eligible for admission");
        \} else \{
            printf("Not eligible: Score criteria not met");
        \}
    \} else \{
        printf("Not eligible: Age criteria not met");
    \}
    
    return 0;
\}
\end{verbatim}

\begin{itemize}
\tightlist
\item
  \textbf{Multiple conditions}: Tests several conditions in sequence
\item
  \textbf{Hierarchical decision}: Inner condition only evaluated if
  outer is true
\item
  \textbf{Indentation}: Proper indentation helps in understanding
  structure
\end{itemize}

\end{solutionbox}
\begin{mnemonicbox}
``CONE'' (Check Outer, Nest Evaluation inside)

\end{mnemonicbox}
\subsection*{Question 3(a) OR [3
marks]}\label{q3a}

\textbf{Explain the use of continue and break statement.}

\begin{solutionbox}
The break and continue statements control the flow of
loops in different ways.


{\def\LTcaptype{none} % do not increment counter
\vspace{-5pt}
\captionof{table}{Comparison of break and continue}
\vspace{-10pt}
\begin{longtable}[]{@{}
  >{\raggedright\arraybackslash}p{(\linewidth - 4\tabcolsep) * \real{0.3462}}
  >{\raggedright\arraybackslash}p{(\linewidth - 4\tabcolsep) * \real{0.2692}}
  >{\raggedright\arraybackslash}p{(\linewidth - 4\tabcolsep) * \real{0.3846}}@{}}
\toprule\noalign{}
\begin{minipage}[b]{\linewidth}\raggedright
Feature
\end{minipage} & \begin{minipage}[b]{\linewidth}\raggedright
break
\end{minipage} & \begin{minipage}[b]{\linewidth}\raggedright
continue
\end{minipage} \\
\midrule\noalign{}
\endhead
\bottomrule\noalign{}
\endlastfoot
Purpose & Exits the loop immediately & Skips current iteration \\
Effect on loop & Terminates completely & Proceeds to next iteration \\
Applicable in & switch, for, while, do-while & for, while, do-while \\
Usage & When condition met and no more iterations needed & When current
iteration should be skipped \\
\end{longtable}
}

\textbf{Example with break:}

\begin{verbatim}
for (int

i = 1; i {=} 10; i++) \{

    if (i == 5)
        break;    // Exit loop when i equals 5
    printf("\%d ", i);  // Outputs: 1 2 3 4
\}
\end{verbatim}

\textbf{Example with continue:}

\begin{verbatim}
for (int

i = 1; i {=} 10; i++) \{

    if (i \% 2 == 0)
        continue;  // Skip even numbers
    printf("\%d ", i);  // Outputs: 1 3 5 7 9
\}
\end{verbatim}

\begin{itemize}
\tightlist
\item
  \textbf{Loop control}: Both used to manage loop execution
\item
  \textbf{Break exits}: Completely stops the loop
\item
  \textbf{Continue skips}: Only skips current iteration
\end{itemize}

\end{solutionbox}
\begin{mnemonicbox}
``BEC'' (Break Exits Completely, Continue only
current)

\end{mnemonicbox}
\subsection*{Question 3(b) OR [4
marks]}\label{q3b}

\textbf{Write a program using for loop to print this output:}

\begin{verbatim}
1
1 2
1 2 3
1 2 3 4
\end{verbatim}

\begin{solutionbox}
This program uses nested for loops to print the pattern
of numbers.

\begin{verbatim}
\#include {stdio.h}

int main() \{
    int i, j;
    
    // Outer loop for rows (1 to 4)
    for (i = 1; i {=} 4; i++) \{
        // Inner loop for columns (1 to i)
        for (j = 1; j {=} i; j++) \{
            printf("\%d ", j);
        \}
        printf("{n}");  // Move to next line after each row
    \}
    
    return 0;
\}
\end{verbatim}

\textbf{Diagram:}

\begin{verbatim}
flowchart LR
    A([Start]) {-{-} B[i = 1]}
    B {-{-} C\{i = 4?\}}
    C {-{-}|Yes| D[j = 1]}
    D {-{-} E\{j = i?\}}
    E {-{-}|Yes| F[/Print j/]}
    F {-{-} G[j++]}
    G {-{-} E}
    E {-{-}|No| H[/Print newline/]}
    H {-{-} I[i++]}
    I {-{-} C}
    C {-{-}|No| J([End])}
\end{verbatim}

\begin{itemize}
\tightlist
\item
  \textbf{Nested loops}: Outer loop for rows, inner for columns
\item
  \textbf{Dynamic limit}: Inner loop runs j from 1 to current i
\item
  \textbf{Incremental pattern}: Each row has one more number
\end{itemize}

\end{solutionbox}
\begin{mnemonicbox}
``RICI'' (Row Increases, Column Increases based on
row number)

\end{mnemonicbox}
\subsection*{Question 3(c) OR [7
marks]}\label{q3c}

\textbf{Draw flowchart and explain switch statement with example.}

\begin{solutionbox}
The switch statement is a multi-way decision maker that
tests a variable against various case values.

\textbf{Diagram:}

\begin{verbatim}
flowchart LR
    A([Start]) {-{-} B[/Input choice/]}
    B {-{-} C\{Switch choice\}}
    C {-{-}|case 1| D[/Output "Option 1 selected"/]}
    C {-{-}|case 2| E[/Output "Option 2 selected"/]}
    C {-{-}|case 3| F[/Output "Option 3 selected"/]}
    C {-{-}|default| G[/Output "Invalid option"/]}
    D {-{-} H([End])}
    E {-{-} H}
    F {-{-} H}
    G {-{-} H}
\end{verbatim}

\textbf{Code Example:}

\begin{verbatim}
\#include {stdio.h}

int main() \{
    int choice;
    
    printf("Menu:{n}");
    printf("1. Add{n}");
    printf("2. Subtract{n}");
    printf("3. Multiply{n}");
    printf("Enter your choice (1{-3): "});
    scanf("\%d", \&choice);
    
    switch (choice) \{
        case 1:
            printf("Addition selected{n}");
            break;
        case 2:
            printf("Subtraction selected{n}");
            break;
        case 3:
            printf("Multiplication selected{n}");
            break;
        default:
            printf("Invalid choice{n}");
    \}
    
    return 0;
\}
\end{verbatim}

\begin{itemize}
\tightlist
\item
  \textbf{Multiple cases}: Tests one variable against multiple values
\item
  \textbf{Break statement}: Prevents fall-through to next case
\item
  \textbf{Default case}: Handles values not matching any case
\item
  \textbf{Case order}: Can be in any order, default usually last
\end{itemize}

\end{solutionbox}
\begin{mnemonicbox}
``CASED'' (Check All Switch Expression's
Destinations)

\end{mnemonicbox}
\subsection*{Question 4(a) [3 marks]}\label{q4a}

**Develop a C program to convert temperature from Celsius to Fahrenheit
using formula fahrenheit= ((celsius*9)/5)+32.**

\begin{solutionbox}
This program converts a temperature value from Celsius
to Fahrenheit.

\begin{verbatim}
\#include {stdio.h}

int main() \{
    float celsius, fahrenheit;
    
    printf("Enter temperature in Celsius: ");
    scanf("\%f", \&celsius);
    
    // Convert Celsius to Fahrenheit
    fahrenheit = ((celsius * 9) / 5) + 32;
    
    printf("\%.2f Celsius = \%.2f Fahrenheit{n}", celsius, fahrenheit);
    
    return 0;
\}
\end{verbatim}

\textbf{Diagram:}

\begin{verbatim}
flowchart LR
    A([Start]) {-{-} B[/Input celsius/]}
    B {-{-} C["fahrenheit = ((celsius * 9) / 5) + 32"]}
    C {-{-} D[/Output celsius and fahrenheit/]}
    D {-{-} E([End])}
\end{verbatim}

\begin{itemize}
\tightlist
\item
  \textbf{Formula}: F = ((C \times 9) \div 5) + 32
\item
  \textbf{Float variables}: For decimal precision
\item
  \textbf{Formatted output}: Using \%.2f for two decimal places
\end{itemize}

\end{solutionbox}
\begin{mnemonicbox}
``C95+32=F'' (Celsius \times 9 \div 5 + 32 = Fahrenheit)

\end{mnemonicbox}
\subsection*{Question 4(b) [4 marks]}\label{q4b}

\textbf{What is pointer? Explain with example.}

\begin{solutionbox}
A pointer is a variable that stores the memory address
of another variable.

\textbf{Diagram:}

\begin{verbatim}
Memory:
+--------+      +--------+
| ptr    |----->| var    |
| 0x1000 |      | 0x2000 |
+--------+      +--------+
  Address         Value: 10
  contains
  0x2000
\end{verbatim}

\textbf{Code Example:}

\begin{verbatim}
\#include {stdio.h}

int main() \{
    int var = 10;    // Regular variable
    int *ptr;        // Pointer variable
    
    ptr = \&var;      // Store address of var in ptr
    
    printf("Value of var: \%d{n}", var);       // Output: 10
    printf("Address of var: \%p{n}", \&var);    // Output: memory address
    printf("Value of ptr: \%p{n}", ptr);       // Output: same memory address
    printf("Value at address stored in ptr: \%d{n}", *ptr); // Output: 10
    
    // Modify value using pointer
    *ptr = 20;
    printf("New value of var: \%d{n}", var);   // Output: 20
    
    return 0;
\}
\end{verbatim}


{\def\LTcaptype{none} % do not increment counter
\vspace{-5pt}
\captionof{table}{Pointer Operations}
\vspace{-10pt}
\begin{longtable}[]{@{}llll@{}}
\toprule\noalign{}
Operation & Symbol & Description & Example \\
\midrule\noalign{}
\endhead
\bottomrule\noalign{}
\endlastfoot
Address-of & \& & Gets address of variable & \&var \\
Dereference & * & Accesses value at address & *ptr \\
Declaration & * & Creates pointer variable & int *ptr; \\
Assignment & = & Assigns address to pointer & ptr = \&var; \\
\end{longtable}
}

\begin{itemize}
\tightlist
\item
  \textbf{Memory address}: Pointer stores location, not value
\item
  \textbf{Indirection}: Access value indirectly using address
\item
  \textbf{Memory manipulation}: Allows dynamic memory access
\end{itemize}

\end{solutionbox}
\begin{mnemonicbox}
``ADA'' (Address Dereferencing Access)

\end{mnemonicbox}
\subsection*{Question 4(c) [7 marks]}\label{q4c}

\textbf{Draw flowchart and explain do-while loop with example.}

\begin{solutionbox}
The do-while loop is a post-test loop that executes its
body at least once before checking the condition.

\textbf{Diagram:}

\begin{verbatim}
flowchart LR
    A([Start]) {-{-} B[/Initialize counter i = 1/]}
    B {-{-} C[/Execute loop body: Print i/]}
    C {-{-} D[/Increment i: i++/]}
    D {-{-} E\{i = 5?\}}
    E {-{-}|Yes| C}
    E {-{-}|No| F([End])}
\end{verbatim}

\textbf{Code Example:}

\begin{verbatim}
\#include {stdio.h}

int main() \{
    int i = 1;
    
    do \{
        printf("\%d ", i);
        i++;
    \} while (i {=} 5);  // Condition checked after first execution
    
    // Output: 1 2 3 4 5
    
    return 0;
\}
\end{verbatim}


{\def\LTcaptype{none} % do not increment counter
\vspace{-5pt}
\captionof{table}{Characteristics of do-while Loop}
\vspace{-10pt}
\begin{longtable}[]{@{}ll@{}}
\toprule\noalign{}
Characteristic & Description \\
\midrule\noalign{}
\endhead
\bottomrule\noalign{}
\endlastfoot
Execution order & Body first, then condition \\
Minimum iterations & At least one \\
Condition check & At the end of loop \\
Termination & When condition becomes false \\
Syntax & do \{ statements; \} while (condition); \\
\end{longtable}
}

\begin{itemize}
\tightlist
\item
  \textbf{Post-test loop}: Condition evaluated after loop body
\item
  \textbf{Guaranteed execution}: Loop body always runs at least once
\item
  \textbf{Semicolon}: Required after while condition
\end{itemize}

\end{solutionbox}
\begin{mnemonicbox}
``DECAT'' (Do Execute Check After That)

\end{mnemonicbox}
\subsection*{Question 4(a) OR [3
marks]}\label{q4a}

\textbf{Develop a C program to find area of a triangle (½ * base *
height)?}

\begin{solutionbox}
This program calculates the area of a triangle using
the formula Area = ½ \times base \times height.

\begin{verbatim}
\#include {stdio.h}

int main() \{
    float base, height, area;
    
    printf("Enter base of triangle: ");
    scanf("\%f", \&base);
    printf("Enter height of triangle: ");
    scanf("\%f", \&height);
    
    // Calculate area
    area = 0.5 * base * height;
    
    printf("Area of triangle = \%.2f square units{n}", area);
    
    return 0;
\}
\end{verbatim}

\textbf{Diagram:}

\begin{verbatim}
flowchart LR
    A([Start]) {-{-} B[/Input base, height/]}
    B {-{-} C[area = 0.5 * base * height]}
    C {-{-} D[/Output area/]}
    D {-{-} E([End])}
\end{verbatim}

\begin{itemize}
\tightlist
\item
  \textbf{Formula}: Area = ½ \times base \times height
\item
  \textbf{Float variables}: For decimal precision
\item
  \textbf{User input}: Gets base and height from user
\end{itemize}

\end{solutionbox}
\begin{mnemonicbox}
``Half-BH'' (Half times Base times Height)

\end{mnemonicbox}
\subsection*{Question 4(b) OR [4
marks]}\label{q4b}

\textbf{Explain declaration and initialization of pointer.}

\begin{solutionbox}
Pointer declaration and initialization involve creating
a pointer variable and assigning it a memory address.


{\def\LTcaptype{none} % do not increment counter
\vspace{-5pt}
\captionof{table}{Pointer Declaration and Initialization}
\vspace{-10pt}
\begin{longtable}[]{@{}
  >{\raggedright\arraybackslash}p{(\linewidth - 6\tabcolsep) * \real{0.2683}}
  >{\raggedright\arraybackslash}p{(\linewidth - 6\tabcolsep) * \real{0.1951}}
  >{\raggedright\arraybackslash}p{(\linewidth - 6\tabcolsep) * \real{0.2195}}
  >{\raggedright\arraybackslash}p{(\linewidth - 6\tabcolsep) * \real{0.3171}}@{}}
\toprule\noalign{}
\begin{minipage}[b]{\linewidth}\raggedright
Operation
\end{minipage} & \begin{minipage}[b]{\linewidth}\raggedright
Syntax
\end{minipage} & \begin{minipage}[b]{\linewidth}\raggedright
Example
\end{minipage} & \begin{minipage}[b]{\linewidth}\raggedright
Explanation
\end{minipage} \\
\midrule\noalign{}
\endhead
\bottomrule\noalign{}
\endlastfoot
Declaration & data\_type *pointer\_name; & int *ptr; & Creates pointer
to int \\
Initialization & pointer\_name = \&variable; & ptr = \# & Assigns
address of num to ptr \\
Combined & data\_type *pointer\_name = \&variable; & int *ptr = \# &
Declares and initializes together \\
Null pointer & pointer\_name = NULL; & ptr = NULL; & Points to nothing
(safe practice) \\
\end{longtable}
}

\textbf{Code Example:}

\begin{verbatim}
\#include {stdio.h}

int main() \{
    // Declaration
    int *ptr1;
    
    // Declaration and initialization together
    int num = 10;
    int *ptr2 = \&num;
    
    // Initialization with NULL
    int *ptr3 = NULL;
    
    printf("Value at address ptr2: \%d{n}", *ptr2);  // Output: 10
    
    return 0;
\}
\end{verbatim}

\begin{itemize}
\tightlist
\item
  \textbf{Asterisk syntax}: * used in declaration to create pointer
\item
  \textbf{Address operator}: \& gets address of variable
\item
  \textbf{NULL initialization}: Safe practice to avoid wild pointers
\item
  \textbf{Pointer type}: Must match the data type it points to
\end{itemize}

\end{solutionbox}
\begin{mnemonicbox}
``DINA'' (Declare, Initialize with NULL or Address)

\end{mnemonicbox}
\subsection*{Question 4(c) OR [7
marks]}\label{q4c}

\textbf{Draw flowchart and explain while loop with example.}

\begin{solutionbox}
The while loop is a pre-test loop that executes its
body repeatedly as long as the condition remains true.

\textbf{Diagram:}

\begin{verbatim}
flowchart LR
    A([Start]) {-{-} B[/Initialize counter i = 1/]}
    B {-{-} C\{i = 5?\}}
    C {-{-}|Yes| D[/Execute loop body: Print i/]}
    D {-{-} E[/Increment i: i++/]}
    E {-{-} C}
    C {-{-}|No| F([End])}
\end{verbatim}

\textbf{Code Example:}

\begin{verbatim}
\#include {stdio.h}

int main() \{
    int i = 1;
    
    while (i {=} 5) \{  // Condition checked before each execution
        printf("\%d ", i);
        i++;
    \}
    
    // Output: 1 2 3 4 5
    
    return 0;
\}
\end{verbatim}


{\def\LTcaptype{none} % do not increment counter
\vspace{-5pt}
\captionof{table}{Characteristics of while Loop}
\vspace{-10pt}
\begin{longtable}[]{@{}ll@{}}
\toprule\noalign{}
Characteristic & Description \\
\midrule\noalign{}
\endhead
\bottomrule\noalign{}
\endlastfoot
Execution order & Condition first, then body \\
Minimum iterations & Zero (if condition initially false) \\
Condition check & At the beginning of loop \\
Termination & When condition becomes false \\
Syntax & while (condition) \{ statements; \} \\
\end{longtable}
}

\begin{itemize}
\tightlist
\item
  \textbf{Pre-test loop}: Condition evaluated before loop body
\item
  \textbf{Zero iterations possible}: Body may never execute if condition
  initially false
\item
  \textbf{Loop variable}: Must be initialized before loop
\item
  \textbf{Infinite loop}: Occurs if condition never becomes false
\end{itemize}

\end{solutionbox}
\begin{mnemonicbox}
``CELT'' (Check, Execute, Loop, Terminate)

\end{mnemonicbox}
\subsection*{Question 5(a) [3 marks]}\label{q5a}

\textbf{Build a structure to store book information: book\_no,
book\_title, book\_author, book\_price}

\begin{solutionbox}
This program creates a structure to store book
information with the specified fields.

\begin{verbatim}
\#include {stdio.h}
\#include {string.h}

// Define structure for book information
struct Book \{
    int book\_no;
    char book\_title[50];
    char book\_author[30];
    float book\_price;
\;}

int main() \{
    // Declare a variable of Book structure
    struct Book book1;
    
    // Assign values to structure members
    book1.book\_no = 101;
    strcpy(book1.book\_title, "Programming in C");
    strcpy(book1.book\_author, "Dennis Ritchie");
    book1.book\_price = 450.75;
    
    // Display book information
    printf("Book No: \%d{n}", book1.book\_no);
    printf("Title: \%s{n}", book1.book\_title);
    printf("Author: \%s{n}", book1.book\_author);
    printf("Price: Rs. \%.2f{n}", book1.book\_price);
    
    return 0;
\}
\end{verbatim}

\textbf{Diagram:}

\begin{verbatim}
+-----------------+
| struct Book     |
+-----------------+
| int book_no     |
| char book_title |
| char book_author|
| float book_price|
+-----------------+
\end{verbatim}

\begin{itemize}
\tightlist
\item
  \textbf{Structure definition}: Uses struct keyword to define composite
  data type
\item
  \textbf{Member access}: Using dot (.) operator to access members
\item
  \textbf{String copying}: strcpy() for character arrays
\end{itemize}

\end{solutionbox}
\begin{mnemonicbox}
``NTAP'' (Number, Title, Author, Price)

\end{mnemonicbox}
\subsection*{Question 5(b) [4 marks]}\label{q5b}

\textbf{Explain following functions with example. (1) sqrt() (2) pow()
(3) strlen() (4) strcpy()}

\begin{solutionbox}
These are standard library functions in C, used for
mathematical calculations and string manipulations.


{\def\LTcaptype{none} % do not increment counter
\vspace{-5pt}
\captionof{table}{Library Functions}
\vspace{-10pt}
\begin{longtable}[]{@{}
  >{\raggedright\arraybackslash}p{(\linewidth - 8\tabcolsep) * \real{0.2041}}
  >{\raggedright\arraybackslash}p{(\linewidth - 8\tabcolsep) * \real{0.2653}}
  >{\raggedright\arraybackslash}p{(\linewidth - 8\tabcolsep) * \real{0.1837}}
  >{\raggedright\arraybackslash}p{(\linewidth - 8\tabcolsep) * \real{0.1837}}
  >{\raggedright\arraybackslash}p{(\linewidth - 8\tabcolsep) * \real{0.1633}}@{}}
\toprule\noalign{}
\begin{minipage}[b]{\linewidth}\raggedright
Function
\end{minipage} & \begin{minipage}[b]{\linewidth}\raggedright
Header File
\end{minipage} & \begin{minipage}[b]{\linewidth}\raggedright
Purpose
\end{minipage} & \begin{minipage}[b]{\linewidth}\raggedright
Example
\end{minipage} & \begin{minipage}[b]{\linewidth}\raggedright
Output
\end{minipage} \\
\midrule\noalign{}
\endhead
\bottomrule\noalign{}
\endlastfoot
sqrt() & math.h & Square root of a number & sqrt(16) & 4.0 \\
pow() & math.h & Raises number to power & pow(2, 3) & 8.0 \\
strlen() & string.h & Length of string & strlen(``Hello'') & 5 \\
strcpy() & string.h & Copies one string to another & strcpy(dest,
``Hello'') & dest contains ``Hello'' \\
\end{longtable}
}

\textbf{Code Example:}

\begin{verbatim}
\#include {stdio.h}
\#include {math.h}
\#include {string.h}

int main() \{
    // sqrt() and pow() examples
    printf("Square root of 25: \%.2f{n}", sqrt(25));
    printf("2 raised to power 4: \%.2f{n}", pow(2, 4));
    
    // strlen() example
    char str[] = "C Programming";
    printf("Length of string: \%d{n}", strlen(str));
    
    // strcpy() example
    char source[] = "Hello";
    char destination[10];
    strcpy(destination, source);
    printf("Copied string: \%s{n}", destination);
    
    return 0;
\}
\end{verbatim}

\begin{itemize}
\tightlist
\item
  \textbf{Math functions}: sqrt() and pow() for mathematical
  calculations
\item
  \textbf{String functions}: strlen() and strcpy() for string
  manipulations
\item
  \textbf{Header files}: Required to use these functions
\item
  \textbf{Return types}: sqrt() and pow() return double, strlen()
  returns size\_t
\end{itemize}

\end{solutionbox}
\begin{mnemonicbox}
``MPSL'' (Math Power and String Length)

\end{mnemonicbox}
\subsection*{Question 5(c) [7 marks]}\label{q5c}

\textbf{Explain arrays and array initialization. Give example.}

\begin{solutionbox}
An array is a collection of elements of the same data
type stored in contiguous memory locations.


{\def\LTcaptype{none} % do not increment counter
\vspace{-5pt}
\captionof{table}{Array Types and Initialization Methods}
\vspace{-10pt}
\begin{longtable}[]{@{}
  >{\raggedright\arraybackslash}p{(\linewidth - 6\tabcolsep) * \real{0.1500}}
  >{\raggedright\arraybackslash}p{(\linewidth - 6\tabcolsep) * \real{0.1625}}
  >{\raggedright\arraybackslash}p{(\linewidth - 6\tabcolsep) * \real{0.3875}}
  >{\raggedright\arraybackslash}p{(\linewidth - 6\tabcolsep) * \real{0.3000}}@{}}
\toprule\noalign{}
\begin{minipage}[b]{\linewidth}\raggedright
Array Type
\end{minipage} & \begin{minipage}[b]{\linewidth}\raggedright
Declaration
\end{minipage} & \begin{minipage}[b]{\linewidth}\raggedright
Initialization at Declaration
\end{minipage} & \begin{minipage}[b]{\linewidth}\raggedright
Separate Initialization
\end{minipage} \\
\midrule\noalign{}
\endhead
\bottomrule\noalign{}
\endlastfoot
Integer & int arr[5]; & int arr[5] = \{10, 20, 30, 40, 50\}; &
arr[0] = 10; arr[1] = 20; etc. \\
Character & char str[10]; & char str[10] = ``Hello''; &
strcpy(str, ``Hello''); \\
Float & float values[3]; & float values[3] = \{1.5, 2.5, 3.5\};
& values[0] = 1.5; etc. \\
Partial & int nums[5]; & int nums[5] = \{1, 2\}; & Remaining set
to 0 \\
Size inference & - & int nums[] = \{1, 2, 3\}; & Size determined by
initializer \\
\end{longtable}
}

\textbf{Code Example:}

\begin{verbatim}
\#include {stdio.h}

int main() \{
    // Array declaration and initialization
    int numbers[5] = \{10, 20, 30, 40, 50\;}
    
    // Access and display array elements
    printf("Array elements: ");
    for (int i = 0; i {} 5; i++) \{
        printf("\%d ", numbers[i]);
    \}
    printf("{n}");
    
    // Modifying array element
    numbers[2] = 35;
    printf("Modified element at index 2: \%d{n}", numbers[2]);
    
    return 0;
\}
\end{verbatim}

\textbf{Diagram:}

\begin{verbatim}
Array: numbers[5]
+-----+-----+-----+-----+-----+
| 10  | 20  | 30  | 40  | 50  |
+-----+-----+-----+-----+-----+
  [0]   [1]   [2]   [3]   [4]
\end{verbatim}

\begin{itemize}
\tightlist
\item
  \textbf{Zero-based indexing}: First element at index 0
\item
  \textbf{Contiguous memory}: Elements stored adjacently
\item
  \textbf{Fixed size}: Size defined at compile time
\item
  \textbf{Element access}: Using index with square brackets
\end{itemize}

\end{solutionbox}
\begin{mnemonicbox}
``DICE'' (Declaration, Initialization, Contiguous
storage, Element access)

\end{mnemonicbox}
\subsection*{Question 5(a) OR [3
marks]}\label{q5a}

\textbf{Explain declaration of structure with example.}

\begin{solutionbox}
Structure declaration in C involves defining a new data
type that combines different data types under a single name.


{\def\LTcaptype{none} % do not increment counter
\vspace{-5pt}
\captionof{table}{Structure Declaration Methods}
\vspace{-10pt}
\begin{longtable}[]{@{}
  >{\raggedright\arraybackslash}p{(\linewidth - 4\tabcolsep) * \real{0.3200}}
  >{\raggedright\arraybackslash}p{(\linewidth - 4\tabcolsep) * \real{0.3200}}
  >{\raggedright\arraybackslash}p{(\linewidth - 4\tabcolsep) * \real{0.3600}}@{}}
\toprule\noalign{}
\begin{minipage}[b]{\linewidth}\raggedright
Method
\end{minipage} & \begin{minipage}[b]{\linewidth}\raggedright
Syntax
\end{minipage} & \begin{minipage}[b]{\linewidth}\raggedright
Example
\end{minipage} \\
\midrule\noalign{}
\endhead
\bottomrule\noalign{}
\endlastfoot
Basic declaration & struct tag\_name \{ members; \}; & struct Student \{
int id; char name[20]; \}; \\
With variables & struct tag\_name \{ members; \} variables; & struct
Point \{ int x, y; \} p1, p2; \\
Without tag & struct \{ members; \} variables; & struct \{ float real,
imag; \} c1; \\
Typedef & typedef struct \{ members; \} alias; & typedef struct \{ int
h, w; \} Rectangle; \\
\end{longtable}
}

\textbf{Code Example:}

\begin{verbatim}
\#include {stdio.h}

// Structure declaration
struct Student \{
    int id;
    char name[30];
    float percentage;
\;}

int main() \{
    // Declaring structure variable
    struct Student s1;
    
    // Assigning values to structure members
    s1.id = 101;
    strcpy(s1.name, "John");
    s1.percentage = 85.5;
    
    // Displaying structure members
    printf("Student ID: \%d{n}", s1.id);
    printf("Name: \%s{n}", s1.name);
    printf("Percentage: \%.2f\%\%{n}", s1.percentage);
    
    return 0;
\}
\end{verbatim}

\begin{itemize}
\tightlist
\item
  \textbf{Structure keyword}: struct used to define new data type
\item
  \textbf{Member access}: . (dot) operator to access members
\item
  \textbf{Heterogeneous data}: Can combine different data types
\item
  \textbf{Custom data type}: Creates user-defined data type
\end{itemize}

\end{solutionbox}
\begin{mnemonicbox}
``SMUVT'' (Structure Mostly Uses Various Types)

\end{mnemonicbox}
\subsection*{Question 5(b) OR [4
marks]}\label{q5b}

\textbf{What is user defined function? Explain with example.}

\begin{solutionbox}
A user-defined function is a block of code written by
the programmer to perform a specific task, which can be called from
other parts of the program.


{\def\LTcaptype{none} % do not increment counter
\vspace{-5pt}
\captionof{table}{Function Components}
\vspace{-10pt}
\begin{longtable}[]{@{}lll@{}}
\toprule\noalign{}
Component & Description & Example \\
\midrule\noalign{}
\endhead
\bottomrule\noalign{}
\endlastfoot
Return type & Data type returned by function & int, float, void \\
Function name & Identifier for the function & add, findMax \\
Parameters & Input values in parentheses & (int a, int b) \\
Function body & Code inside curly braces & \{ return a + b; \} \\
Function call & Invoking the function & result = add(5, 3); \\
\end{longtable}
}

\textbf{Code Example:}

\begin{verbatim}
\#include {stdio.h}

// User{-defined function declaration}
int findMax(int a, int b);

int main() \{
    int num1 = 10, num2 = 20, max;
    
    // Function call
    max = findMax(num1, num2);
    
    printf("Maximum between \%d and \%d is \%d{n}", num1, num2, max);
    
    return 0;
\}

// Function definition
int findMax(int a, int b) \{
    // Function body
    if (a {} b)
        return a;
    else
        return b;
\}
\end{verbatim}

\textbf{Diagram:}

\begin{verbatim}
flowchart TD
    A[main function] {-{-}|calls with num1, num2| B[findMax function]}
    B {-{-}|returns maximum value| A}
\end{verbatim}

\begin{itemize}
\tightlist
\item
  \textbf{Modular code}: Break large program into smaller parts
\item
  \textbf{Reusability}: Call function multiple times from different
  places
\item
  \textbf{Declaration vs Definition}: Declaration tells compiler about
  function, definition contains actual code
\item
  \textbf{Parameters}: Pass values to function when calling
\end{itemize}

\end{solutionbox}
\begin{mnemonicbox}
``CDRP'' (Create, Define, Return, Pass)

\end{mnemonicbox}
\subsection*{Question 5(c) OR [7
marks]}\label{q5c}

\textbf{Develop a C program to arrange elements of an array of 10
numbers in ascending order.}

\begin{solutionbox}
This program sorts an array of 10 integers in ascending
order using bubble sort algorithm.

\begin{verbatim}
\#include {stdio.h}

int main() \{
    int arr[10], i, j, temp;
    
    // Input array elements
    printf("Enter 10 integers: {n}");
    for (i = 0; i {} 10; i++) \{
        scanf("\%d", \&arr[i]);
    \}
    
    // Bubble sort algorithm for ascending order
    for (i = 0; i {} 9; i++) \{
        for (j = 0; j {} 9 {-} i; j++) \{
            if (arr[j] {} arr[j + 1]) \{
                // Swap if current element is greater than next
                temp = arr[j];
                arr[j] = arr[j + 1];
                arr[j + 1] = temp;
            \}
        \}
    \}
    
    // Display sorted array
    printf("Array in ascending order: {n}");
    for (i = 0; i {} 10; i++) \{
        printf("\%d ", arr[i]);
    \}
    
    return 0;
\}
\end{verbatim}

\textbf{Diagram:}

\begin{verbatim}
flowchart LR
    A([Start]) {-{-} B[/Input 10 array elements/]}
    B {-{-} C[i = 0]}
    C {-{-} D\{i  9?\}}
    D {-{-}|Yes| E[j = 0]}
    E {-{-} F\{j  9{-}i?\}}
    F {-{-}|Yes| G\{"arr[j]  arr[j+1]?"\}}
    G {-{-}|Yes| H["Swap arr[j] and arr[j+1]"]}
    G {-{-}|No| I[j++]}
    H {-{-} I}
    I {-{-} F}
    F {-{-}|No| J[i++]}
    J {-{-} D}
    D {-{-}|No| K[/Output sorted array/]}
    K {-{-} L([End])}
\end{verbatim}

\begin{itemize}
\tightlist
\item
  \textbf{Bubble sort}: Compare adjacent elements and swap if needed
\item
  \textbf{Nested loops}: Outer loop for passes, inner loop for
  comparisons
\item
  \textbf{Optimization}: Each pass fixes at least one element, so inner
  loop runs fewer times
\item
  \textbf{Temporary variable}: Used for swapping elements
\end{itemize}

\end{solutionbox}
\begin{mnemonicbox}
``BSCOT'' (Bubble Sort Compares and Orders Things)

\end{mnemonicbox}

\end{document}
