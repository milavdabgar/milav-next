\documentclass[10pt,a4paper]{article}

% content/resources/templates/preamble.tex
\usepackage[margin=0.6in]{geometry}
\author{Milav Dabgar}
\usepackage{amsmath,amssymb,amsthm}
\usepackage{booktabs}
\usepackage{multirow}
\usepackage{xcolor}
\usepackage{tcolorbox}
\tcbuselibrary{breakable,skins}
\usepackage[colorlinks=true,linkcolor=blue]{hyperref}
\usepackage{titlesec}
\usepackage{enumitem}
\usepackage{tikz}
\usepackage{pgfplots}
\usepackage{circuitikz}
\usepackage[version=4]{mhchem}
\usepackage{longtable}
\usepackage{array}
\usepackage{float}
\usepackage{caption}
\usepackage{listings}

\lstset{
  basicstyle=\small\ttfamily,
  breaklines=true,
  breakatwhitespace=false,
  postbreak=\mbox{\textcolor{red}{$\hookrightarrow$}\space},
  float=false,
  numbers=left,
  numberstyle=\tiny\color{gray},
  numbersep=10pt,
  xleftmargin=2em,
  keywordstyle=\color{blue},
  commentstyle=\color{green!60!black},
  stringstyle=\color{purple},
  backgroundcolor=\color{gray!5},
  showstringspaces=false,
  tabsize=2,
  captionpos=b,
  keepspaces=true,
  columns=flexible
}

\pgfplotsset{compat=1.18}
\usetikzlibrary{shapes,arrows,positioning,calc,patterns,decorations.pathmorphing,decorations.markings,arrows.meta}

% Color scheme
\definecolor{headcolor}{RGB}{0,102,204}
\definecolor{keycolor}{RGB}{220,20,60}
\definecolor{solutioncolor}{RGB}{34,139,34}
\definecolor{mnemoniccolor}{RGB}{148,0,211}
\definecolor{codecolor}{RGB}{0,0,100}

% Spacing
\setlength{\parskip}{3pt}
\setlist[itemize]{nosep}
\setlist[enumerate]{nosep}

% Title formatting
\titleformat{\section}{\Large\bfseries\color{headcolor}}{\thesection}{1em}{}
\titleformat{\subsection}{\large\bfseries\color{headcolor}}{\thesubsection}{1em}{}

% Pandoc tightlist compatibility
\providecommand{\tightlist}{%
  \setlength{\itemsep}{0pt}\setlength{\parskip}{0pt}}

% Pandoc longtable compatibility
\newcounter{none}
\def\thenone{}


% content/resources/templates/english-boxes.tex
% This file is currently empty - it exists to maintain consistency with the import structure.
% Add custom environments here if needed in the future.


\begin{document}

\begin{center}
{\Huge\bfseries\color{headcolor} Subject Name Solutions}\\[5pt]
{\LARGE 4331105 -- Summer 2023}\\[3pt]
{\large Semester 1 Study Material}\\[3pt]
{\normalsize\textit{Detailed Solutions and Explanations}}
\end{center}

\vspace{10pt}

\subsection*{Question 1(a) [3 marks]}\label{q1a}

\textbf{List any six keywords of C language.}

\begin{solutionbox}


{\def\LTcaptype{none} % do not increment counter
\vspace{-5pt}
\captionof{table}{Six Keywords in C Language}
\vspace{-10pt}
\begin{longtable}[]{@{}ll@{}}
\toprule\noalign{}
Keyword & Purpose \\
\midrule\noalign{}
\endhead
\bottomrule\noalign{}
\endlastfoot
int & Integer data type \\
float & Floating-point data type \\
if & Conditional statement \\
while & Loop structure \\
return & Returns value from function \\
void & Specifies empty return type \\
\end{longtable}
}

\end{solutionbox}
\begin{mnemonicbox}
``I Feel When Running Very Ill'' (int, float, while,
return, void, if)

\end{mnemonicbox}
\subsection*{Question 1(b) [4 marks]}\label{q1b}

\textbf{Define variable. List the rule for naming of variable in c
programming.}

\begin{solutionbox}

\textbf{Variable}: A named memory location used to store data that can
be modified during program execution.


{\def\LTcaptype{none} % do not increment counter
\vspace{-5pt}
\captionof{table}{Rules for Variable Naming in C}
\vspace{-10pt}
\begin{longtable}[]{@{}ll@{}}
\toprule\noalign{}
Rule & Example \\
\midrule\noalign{}
\endhead
\bottomrule\noalign{}
\endlastfoot
Must begin with letter/underscore & name, \_value \\
Can contain letters, digits, underscore & user\_1, count99 \\
No spaces or special characters & ✓: total\_sum, ✗: total-sum \\
Case sensitive & Name \neq name \\
Cannot use reserved keywords & ✗: int, while \\
Maximum 31 characters (standard) & studentRegistrationNumber \\
\end{longtable}
}

\end{solutionbox}
\begin{mnemonicbox}
``Letters Lead, No Special Keys'' (begins with
letter, no special chars, no keywords)

\end{mnemonicbox}
\subsection*{Question 1(c) [7 marks]}\label{q1c}

\textbf{Define flowchart. Draw and Explain flowchart symbols. Write a
program to calculate simple interest using below equation. I=PRN/100
Where

P=Principal amount,

R=Rate of interest and

N=Period.}


\begin{solutionbox}

\textbf{Flowchart}: A graphical representation of an algorithm that uses
standard symbols to show the sequence of operations needed to solve a
problem.


{\def\LTcaptype{none} % do not increment counter
\vspace{-5pt}
\captionof{table}{Flowchart Symbols}
\vspace{-10pt}
\begin{longtable}[]{@{}
  >{\raggedright\arraybackslash}p{(\linewidth - 4\tabcolsep) * \real{0.3478}}
  >{\raggedright\arraybackslash}p{(\linewidth - 4\tabcolsep) * \real{0.2609}}
  >{\raggedright\arraybackslash}p{(\linewidth - 4\tabcolsep) * \real{0.3913}}@{}}
\toprule\noalign{}
\begin{minipage}[b]{\linewidth}\raggedright
Symbol
\end{minipage} & \begin{minipage}[b]{\linewidth}\raggedright
Name
\end{minipage} & \begin{minipage}[b]{\linewidth}\raggedright
Purpose
\end{minipage} \\
\midrule\noalign{}
\endhead
\bottomrule\noalign{}
\endlastfoot
\pandocbounded{\includegraphics[keepaspectratio,alt={Oval}]{https://goat.liveuml.com/svg/S2w}}
& Terminal & Start/End \\
\pandocbounded{\includegraphics[keepaspectratio,alt={Rectangle}]{https://goat.liveuml.com/svg/JYuxD}}
& Process & Calculations \\
\pandocbounded{\includegraphics[keepaspectratio,alt={Parallelogram}]{https://goat.liveuml.com/svg/JY1ZD}}
& Input/Output & Read/Display data \\
\pandocbounded{\includegraphics[keepaspectratio,alt={Diamond}]{https://goat.liveuml.com/svg/CYuxD}}
& Decision & Conditions \\
\pandocbounded{\includegraphics[keepaspectratio,alt={Arrow}]{https://goat.liveuml.com/svg/KYsxD}}
& Flow Line & Shows sequence \\
\end{longtable}
}

\textbf{Simple Interest Flowchart:}

\begin{verbatim}
flowchart LR
    A([Start]) {-{-} B[/Input P, R, N/]}
    B {-{-} C[Calculate I = P*R*N/100]}
    C {-{-} D[/Display I/]}
    D {-{-} E([End])}
\end{verbatim}

\textbf{Program:}

\begin{verbatim}
\#include {stdio.h}
void main()
\{
    float p, r, n, i;
    
    printf("Enter principal amount: ");
    scanf("\%f", \&p);
    
    printf("Enter rate of interest: ");
    scanf("\%f", \&r);
    
    printf("Enter time period in years: ");
    scanf("\%f", \&n);
    
    i = (p * r * n) / 100;
    
    printf("Simple Interest = \%.2f", i);
\}
\end{verbatim}

\end{solutionbox}
\begin{mnemonicbox}
``Please Return Nice Interest'' (Principal, Rate,
Number of years, Interest)

\end{mnemonicbox}
\subsection*{Question 1(c) OR [7
marks]}\label{q1c}

\textbf{Define algorithm. Write algorithm for finding volume of
cylinder. Write a program to read radius(R) and height(H) from user and
print calculated the volume(V) of cylinder using. V=ΠR^{2}H.}

\begin{solutionbox}

\textbf{Algorithm}: A step-by-step procedure to solve a problem in a
finite amount of time.

\textbf{Algorithm for Cylinder Volume:}

\begin{enumerate}
\tightlist
\item
  Start
\item
  Input radius (R) and height (H)
\item
  Calculate volume using formula V = π \times R^{2} \times H
\item
  Display the volume
\item
  End
\end{enumerate}

\textbf{Diagram: Cylinder}

\begin{verbatim}
    +{-{-}{-}{-}{-}{-}+}
    |      |
    |      | H
    |      |
    +{-{-}{-}{-}{-}{-}+}
       R 
\end{verbatim}

\textbf{Program:}

\begin{verbatim}
\#include {stdio.h}
void main()
\{
    float radius, height, volume;
    float pi = 3.14159;
    
    printf("Enter radius of cylinder: ");
    scanf("\%f", \&radius);
    
    printf("Enter height of cylinder: ");
    scanf("\%f", \&height);
    
    volume = pi * radius * radius * height;
    
    printf("Volume of cylinder = \%.2f", volume);
\}
\end{verbatim}

\end{solutionbox}
\begin{mnemonicbox}
``Round Hat Volume'' (Radius, Height, Volume)

\end{mnemonicbox}
\subsection*{Question 2(a) [3 marks]}\label{q2a}

\textbf{List out different operators supported in C programming
language.}

\begin{solutionbox}


{\def\LTcaptype{none} % do not increment counter
\vspace{-5pt}
\captionof{table}{Operators in C Programming}
\vspace{-10pt}
\begin{longtable}[]{@{}lll@{}}
\toprule\noalign{}
Operator Type & Examples & Use \\
\midrule\noalign{}
\endhead
\bottomrule\noalign{}
\endlastfoot
Arithmetic & +, -, *, /, \% & Mathematical operations \\
Relational & \textless, \textgreater, ==, !=, \textless=, \textgreater=
& Compare values \\
Logical & \&\&, \textbar\textbar, ! & Combine conditions \\
Assignment & =, +=, -=, *=, /= & Assign values \\
Increment/Decrement & ++, -- & Increase/decrease by 1 \\
Bitwise & \&, \textbar, \^{}, \textasciitilde, \textless\textless,
\textgreater\textgreater{} & Bit manipulation \\
Conditional & ?: & Short if-else \\
\end{longtable}
}

\end{solutionbox}
\begin{mnemonicbox}
``All Relationships Lead Ancestors Incrementally
Beyond Conditions'' (first letter of each type)

\end{mnemonicbox}
\subsection*{Question 2(b) [4 marks]}\label{q2b}

\textbf{Write a program to print sum and average of 1 to 50.}

\begin{solutionbox}

\textbf{Program:}

\begin{verbatim}
\#include {stdio.h}
void main()
\{
    int i, sum = 0;
    float avg;
    
    for(i = 1; i {=} 50; i++)
    \{
        sum = sum + i;
    \}
    
    avg = (float)sum / 50;
    
    printf("Sum of numbers from 1 to 50 = \%d{n}", sum);
    printf("Average of numbers from 1 to 50 = \%.2f", avg);
\}
\end{verbatim}

\textbf{Process Diagram:}

\begin{verbatim}
flowchart LR
    A([Start]) {-{-} B[Set sum = 0]}
    B {-{-} C[Loop i from 1 to 50]}
    C {-{-} D[Add i to sum]}
    D {-{-} E\{i  50?\}}
    E {-{-}|Yes| C}
    E {-{-}|No| F[Calculate avg = sum/50]}
    F {-{-} G[/Display sum and avg/]}
    G {-{-} H([End])}
\end{verbatim}

\end{solutionbox}
\begin{mnemonicbox}
``Summing And Dividing'' (Sum, Average, Division)

\end{mnemonicbox}
\subsection*{Question 2(c) [7 marks]}\label{q2c}

\textbf{Explain arithmetic \& relational operators with example.}

\begin{solutionbox}

\textbf{Arithmetic Operators:}


{\def\LTcaptype{none} % do not increment counter
\vspace{-5pt}
\captionof{table}{Arithmetic Operators in C}
\vspace{-10pt}
\begin{longtable}[]{@{}llll@{}}
\toprule\noalign{}
Operator & Operation & Example & Result \\
\midrule\noalign{}
\endhead
\bottomrule\noalign{}
\endlastfoot
+ & Addition & 5 + 3 & 8 \\
- & Subtraction & 7 - 2 & 5 \\
* & Multiplication & 4 * 3 & 12 \\
/ & Division & 8 / 4 & 2 \\
\% & Modulus (Remainder) & 7 \% 3 & 1 \\
\end{longtable}
}

\textbf{Relational Operators:}


{\def\LTcaptype{none} % do not increment counter
\vspace{-5pt}
\captionof{table}{Relational Operators in C}
\vspace{-10pt}
\begin{longtable}[]{@{}llll@{}}
\toprule\noalign{}
Operator & Meaning & Example & Result \\
\midrule\noalign{}
\endhead
\bottomrule\noalign{}
\endlastfoot
\textless{} & Less than & 5 \textless{} 8 & 1 (true) \\
\textgreater{} & Greater than & 9 \textgreater{} 3 & 1 (true) \\
== & Equal to & 4 == 4 & 1 (true) \\
!= & Not equal to & 7 != 3 & 1 (true) \\
\textless= & Less than or equal to & 4 \textless= 4 & 1 (true) \\
\textgreater= & Greater than or equal to & 6 \textgreater= 9 & 0
(false) \\
\end{longtable}
}

\textbf{Code Example:}

\begin{verbatim}
\#include {stdio.h}
void main()
\{
int

a = 10,

b = 5;

    
    // Arithmetic operators
    printf("a + b = \%d{n}", a + b);   // 15
    printf("a {- b = }\%d{n}", a {-} b);   // 5
    printf("a * b = \%d{n}", a * b);   // 50
    printf("a / b = \%d{n}", a / b);   // 2
    printf("a \%\% b = \%d{n}", a \% b);  // 0
    
    // Relational operators
    printf("a { b: }\%d{n}", a {} b);    // 0 (false)
    printf("a { b: }\%d{n}", a {} b);    // 1 (true)
printf("a == b: \%d{n}",

a == b);  // 0 (false)

    printf("a != b: \%d{n}", a != b);  // 1 (true)
\}
\end{verbatim}

\end{solutionbox}
\begin{mnemonicbox}
``Add Subtract Multiply Divide Remainder''
(arithmetic), ``Less Greater Equal Not'' (relational)

\end{mnemonicbox}
\subsection*{Question 2(a) OR [3
marks]}\label{q2a}

\textbf{State the difference between gets(S) and scanf(``\%s'',S) where
S is string.}

\begin{solutionbox}


{\def\LTcaptype{none} % do not increment counter
\vspace{-5pt}
\captionof{table}{Difference between gets(S) and scanf(``\%s'',S)}
\vspace{-10pt}
\begin{longtable}[]{@{}
  >{\raggedright\arraybackslash}p{(\linewidth - 4\tabcolsep) * \real{0.2727}}
  >{\raggedright\arraybackslash}p{(\linewidth - 4\tabcolsep) * \real{0.2727}}
  >{\raggedright\arraybackslash}p{(\linewidth - 4\tabcolsep) * \real{0.4545}}@{}}
\toprule\noalign{}
\begin{minipage}[b]{\linewidth}\raggedright
Feature
\end{minipage} & \begin{minipage}[b]{\linewidth}\raggedright
gets(S)
\end{minipage} & \begin{minipage}[b]{\linewidth}\raggedright
scanf(``\%s'',S)
\end{minipage} \\
\midrule\noalign{}
\endhead
\bottomrule\noalign{}
\endlastfoot
Space handling & Reads spaces between words & Stops reading at space \\
Input termination & Ends at newline character & Ends at whitespace \\
Buffer overflow & Unsafe, no length check & Safer with width limit \\
Example behavior & ``Hello World'' \rightarrow ``Hello World'' & ``Hello World'' \rightarrow
``Hello'' \\
Security & Deprecated due to overflow risks & Better with width
specifier \\
\end{longtable}
}

\end{solutionbox}
\begin{mnemonicbox}
``Gets Spaces, Scanf Stops'' (gets reads spaces,
scanf stops at spaces)

\end{mnemonicbox}
\subsection*{Question 2(b) OR [4
marks]}\label{q2b}

\textbf{Write a program to swap two numbers.}

\begin{solutionbox}

\textbf{Program:}

\begin{verbatim}
\#include {stdio.h}
void main()
\{
    int a, b, temp;
    
    printf("Enter value of a: ");
    scanf("\%d", \&a);
    
    printf("Enter value of b: ");
    scanf("\%d", \&b);
    
printf("Before swapping:

a = \%d,

b = \%d{n}", a, b);

    
    // Swapping using temp variable
    temp = a;
    a = b;
    b = temp;
    
printf("After swapping:

a = \%d,

b = \%d", a, b);

\}
\end{verbatim}

\textbf{Swapping Diagram:}

\begin{verbatim}
flowchart LR
    A["a = 5"] {-{-} |Step 1: temp = a| C["temp = 5"]}
B["b = 10"] {-{-} |Step 2:

a = b| A1["a = 10"]}

C {-{-} |Step 3:

b = temp| B1["b = 5"]}

\end{verbatim}

\end{solutionbox}
\begin{mnemonicbox}
``Temporary Assists Swapping'' (Temp variable enables
swapping)

\end{mnemonicbox}
\subsection*{Question 2(c) OR [7
marks]}\label{q2c}

\textbf{Explain Logical operator and bit-wise operator with example.}

\begin{solutionbox}

\textbf{Logical Operators:}


{\def\LTcaptype{none} % do not increment counter
\vspace{-5pt}
\captionof{table}{Logical Operators in C}
\vspace{-10pt}
\begin{longtable}[]{@{}llll@{}}
\toprule\noalign{}
Operator & Description & Example & Result \\
\midrule\noalign{}
\endhead
\bottomrule\noalign{}
\endlastfoot
\&\& & Logical AND & (5\textgreater3) \&\& (8\textgreater6) & 1 (both
true) \\
\textbar\textbar{} & Logical OR & (5\textless3) \textbar\textbar{}
(8\textgreater6) & 1 (one true) \\
! & Logical NOT & !(5\textgreater3) & 0 (inverts true to false) \\
\end{longtable}
}

\textbf{Bitwise Operators:}


{\def\LTcaptype{none} % do not increment counter
\vspace{-5pt}
\captionof{table}{Bitwise Operators in C}
\vspace{-10pt}
\begin{longtable}[]{@{}llll@{}}
\toprule\noalign{}
Operator & Description & Example & Binary Result \\
\midrule\noalign{}
\endhead
\bottomrule\noalign{}
\endlastfoot
\& & Bitwise AND & 5 \& 3 & 101 \& 011 = 001 (1) \\
\textbar{} & Bitwise OR & 5 \textbar{} 3 & 101 \textbar{} 011 = 111
(7) \\
\^{} & Bitwise XOR & 5 \^{} 3 & 101 \^{} 011 = 110 (6) \\
\textasciitilde{} & Bitwise NOT & \textasciitilde5 & \textasciitilde0101
= 1010 (-6) \\
\textless\textless{} & Left Shift & 5 \textless\textless{} 1 & 101
\textless\textless{} 1 = 1010 (10) \\
\textgreater\textgreater{} & Right Shift & 5 \textgreater\textgreater{}
1 & 101 \textgreater\textgreater{} 1 = 10 (2) \\
\end{longtable}
}

\textbf{Code Example:}

\begin{verbatim}
\#include {stdio.h}
void main()
\{
int

a = 5,

b = 3;

    
    // Logical operators
    printf("a{3 \&\& b5: }\%d{n}", (a{}3) \&\& (b{}5));  // 1 (true)
    printf("a{3 || b1: }\%d{n}", (a{}3) || (b{}1));  // 1 (true)
    printf("!(a{b): }\%d{n}", !(a{}b));              // 0 (false)
    
    // Bitwise operators
    printf("a \& b: \%d{n}", a \& b);   // 1
    printf("a | b: \%d{n}", a | b);   // 7
    printf("a \^{ b: }\%d{n}", a \^{} b);   // 6
    printf("{a: }\%d{n}", {}a);         // {-6}
    printf("a { 1: }\%d{n}", a {} 1); // 10
    printf("a { 1: }\%d{n}", a {} 1); // 2
\}
\end{verbatim}

\end{solutionbox}
\begin{mnemonicbox}
``AND OR NOT'' (logical operators), ``AND OR XOR NOT
SHIFT'' (bitwise operators)

\end{mnemonicbox}
\subsection*{Question 3(a) [3 marks]}\label{q3a}

\textbf{Explain multiple if-else statement with example.}

\begin{solutionbox}

\textbf{Multiple if-else}: Series of if-else statements where each
condition is checked sequentially until a true condition is found.

\textbf{Structure:}

\begin{verbatim}
if (condition1)
    statement1;
else if (condition2)
    statement2;
else if (condition3)
    statement3;
else
    default\_statement;
\end{verbatim}

\textbf{Code Example:}

\begin{verbatim}
\#include {stdio.h}
void main()
\{
    int marks;
    
    printf("Enter marks: ");
    scanf("\%d", \&marks);
    
    if (marks {=} 80)
        printf("Grade: A");
    else if (marks {=} 70)
        printf("Grade: B");
    else if (marks {=} 60)
        printf("Grade: C");
    else if (marks {=} 50)
        printf("Grade: D");
    else
        printf("Grade: F");
\}
\end{verbatim}

\textbf{Diagram:}

\begin{verbatim}
flowchart LR
    A[Start] {-{-} B\{marks = 80?\}}
    B {-{-}|Yes| C[Grade A]}
    B {-{-}|No| D\{marks = 70?\}}
    D {-{-}|Yes| E[Grade B]}
    D {-{-}|No| F\{marks = 60?\}}
    F {-{-}|Yes| G[Grade C]}
    F {-{-}|No| H\{marks = 50?\}}
    H {-{-}|Yes| I[Grade D]}
    H {-{-}|No| J[Grade F]}
\end{verbatim}

\end{solutionbox}
\begin{mnemonicbox}
``Check Each Condition in Sequence'' (CECS)

\end{mnemonicbox}
\subsection*{Question 3(b) [4 marks]}\label{q3b}

\textbf{State the working of while loop and for loop.}

\begin{solutionbox}


{\def\LTcaptype{none} % do not increment counter
\vspace{-5pt}
\captionof{table}{While Loop vs For Loop}
\vspace{-10pt}
\begin{longtable}[]{@{}
  >{\raggedright\arraybackslash}p{(\linewidth - 4\tabcolsep) * \real{0.2903}}
  >{\raggedright\arraybackslash}p{(\linewidth - 4\tabcolsep) * \real{0.3871}}
  >{\raggedright\arraybackslash}p{(\linewidth - 4\tabcolsep) * \real{0.3226}}@{}}
\toprule\noalign{}
\begin{minipage}[b]{\linewidth}\raggedright
Feature
\end{minipage} & \begin{minipage}[b]{\linewidth}\raggedright
While Loop
\end{minipage} & \begin{minipage}[b]{\linewidth}\raggedright
For Loop
\end{minipage} \\
\midrule\noalign{}
\endhead
\bottomrule\noalign{}
\endlastfoot
Syntax & \texttt{while(condition)\ \{\ statements;\ \}} &
\texttt{for(init;\ condition;\ update)\ \{\ statements;\ \}} \\
When to use & When number of iterations is unknown & When number of
iterations is known \\
Initialization & Before the loop & Inside the loop declaration \\
Update & Must be done inside the loop body & Automatically in loop
declaration \\
Exit control & Only at the beginning & Only at the beginning \\
Example & Validating user input & Iterating fixed number of times \\
\end{longtable}
}

\textbf{While Loop Flow:}

\begin{verbatim}
flowchart LR
    A([Start]) {-{-} B[Initialize]}
    B {-{-} C\{Condition\}}
    C {-{-}|True| D[Body]}
    D {-{-} C}
    C {-{-}|False| E([End])}
\end{verbatim}

\textbf{For Loop Flow:}

\begin{verbatim}
flowchart LR
    A([Start]) {-{-} B[Initialize]}
    B {-{-} C\{Condition\}}
    C {-{-}|True| D[Body]}
    D {-{-} E[Update]}
    E {-{-} C}
    C {-{-}|False| F([End])}
\end{verbatim}

\end{solutionbox}
\begin{mnemonicbox}
``While Checks Then Acts'' (WCTA), ``For Initializes
Tests Updates'' (FITU)

\end{mnemonicbox}
\subsection*{Question 3(c) [7 marks]}\label{q3c}

\textbf{Write a program to find factorial of a given number.}

\begin{solutionbox}

\textbf{Program:}

\begin{verbatim}
\#include {stdio.h}
void main()
\{
    int num, i;
    unsigned long fact = 1;
    
    printf("Enter a number: ");
    scanf("\%d", \&num);
    
    if (num {} 0)
        printf("Factorial not defined for negative numbers");
    else
    \{
        for(i = 1; i {=} num; i++)
        \{
            fact = fact * i;
        \}
        printf("Factorial of \%d = \%lu", num, fact);
    \}
\}
\end{verbatim}

\textbf{Factorial Calculation Table:} For example, if num = 5:

{\def\LTcaptype{none} % do not increment counter
\begin{longtable}[]{@{}llll@{}}
\toprule\noalign{}
Iteration & i & fact = fact * i & New fact value \\
\midrule\noalign{}
\endhead
\bottomrule\noalign{}
\endlastfoot
Initial & - & - & 1 \\
1 & 1 & 1 * 1 & 1 \\
2 & 2 & 1 * 2 & 2 \\
3 & 3 & 2 * 3 & 6 \\
4 & 4 & 6 * 4 & 24 \\
5 & 5 & 24 * 5 & 120 \\
\end{longtable}
}

\textbf{Factorial Calculation Diagram:}

\begin{verbatim}
flowchart LR
    A([Start]) {-{-} B[/Input num/]}
    B {-{-} C\{num  0?\}}
    C {-{-}|Yes| D[/Error message/]}
    C {-{-}|No| E[fact = 1]}
    E {-{-} F[Loop i from 1 to num]}
    F {-{-} G[fact = fact * i]}
    G {-{-} H\{i  num?\}}
    H {-{-}|Yes| F}
    H {-{-}|No| I[/Display fact/]}
    I {-{-} J([End])}
\end{verbatim}

\end{solutionbox}
\begin{mnemonicbox}
``Find And Count The Numbers!'' (FACTN! - Factorial)

\end{mnemonicbox}
\subsection*{Question 3(a) OR [3
marks]}\label{q3a}

\textbf{Explain the working of switch-case statement with example.}

\begin{solutionbox}

\textbf{Switch-Case}: A selection statement that allows a variable to be
tested for equality against a list of values (cases).

\textbf{Structure:}

\begin{verbatim}
switch(expression) \{
    case value1:
        statements1;
        break;
    case value2:
        statements2;
        break;
    default:
        default\_statements;
\}
\end{verbatim}

\textbf{Code Example:}

\begin{verbatim}
\#include {stdio.h}
void main()
\{
    int day;
    
    printf("Enter day number (1{-7): "});
    scanf("\%d", \&day);
    
    switch(day) \{
        case 1:
            printf("Monday");
            break;
        case 2:
            printf("Tuesday");
            break;
        case 3:
            printf("Wednesday");
            break;
        case 4:
            printf("Thursday");
            break;
        case 5:
            printf("Friday");
            break;
        case 6:
            printf("Saturday");
            break;
        case 7:
            printf("Sunday");
            break;
        default:
            printf("Invalid day");
    \}
\}
\end{verbatim}

\textbf{Switch-Case Diagram:}

\begin{verbatim}
flowchart TD
    A[Start] {-{-} B[/Input day/]}
    B {-{-} C\{Switch day\}}
    C {-{-}|case 1| D[Monday]}
    C {-{-}|case 2| E[Tuesday]}
    C {-{-}|case 3| F[Wednesday]}
    C {-{-}|case 4| G[Thursday]}
    C {-{-}|case 5| H[Friday]}
    C {-{-}|case 6| I[Saturday]}
    C {-{-}|case 7| J[Sunday]}
    C {-{-}|default| K[Invalid day]}
    D \& E \& F \& G \& H \& I \& J \& K {-{-} L[End]}
\end{verbatim}

\end{solutionbox}
\begin{mnemonicbox}
``Select Value, Exit with Break'' (SVEB)

\end{mnemonicbox}
\subsection*{Question 3(b) OR [4
marks]}\label{q3b}

\textbf{State the use of break and continue keyword.}

\begin{solutionbox}


{\def\LTcaptype{none} % do not increment counter
\vspace{-5pt}
\captionof{table}{Break vs Continue Keywords}
\vspace{-10pt}
\begin{longtable}[]{@{}
  >{\raggedright\arraybackslash}p{(\linewidth - 4\tabcolsep) * \real{0.3462}}
  >{\raggedright\arraybackslash}p{(\linewidth - 4\tabcolsep) * \real{0.2692}}
  >{\raggedright\arraybackslash}p{(\linewidth - 4\tabcolsep) * \real{0.3846}}@{}}
\toprule\noalign{}
\begin{minipage}[b]{\linewidth}\raggedright
Feature
\end{minipage} & \begin{minipage}[b]{\linewidth}\raggedright
break
\end{minipage} & \begin{minipage}[b]{\linewidth}\raggedright
continue
\end{minipage} \\
\midrule\noalign{}
\endhead
\bottomrule\noalign{}
\endlastfoot
Purpose & Exits from current loop/switch & Skips current iteration,
continues next iteration \\
Effect on loop & Terminates the loop & Proceeds to next iteration \\
Where used & Loops \& switch statements & Only in loops \\
Control flow & Passes to statement after loop & Goes to loop condition
check \\
Example use & Exit loop when condition met & Skip specific iterations \\
\end{longtable}
}

\textbf{Flow Diagram - break:}

\begin{verbatim}
flowchart LR
    A([Start]) {-{-} B[Loop]}
    B {-{-} C\{Condition\}}
    C {-{-}|True| D[break]}
    C {-{-}|False| E[Loop statements]}
    E {-{-} B}
    D {-{-} F[Statements after loop]}
    F {-{-} G([End])}
\end{verbatim}

\textbf{Flow Diagram - continue:}

\begin{verbatim}
flowchart LR
    A([Start]) {-{-} B[Loop]}
    B {-{-} C\{Condition\}}
    C {-{-}|True| D[continue]}
    C {-{-}|False| E[Loop statements]}
    D {-{-} B}
    E {-{-} B}
    B {-{-}|Loop ends| F([End])}
\end{verbatim}

\end{solutionbox}
\begin{mnemonicbox}
``Break Exits, Continue Skips'' (BECS)

\end{mnemonicbox}
\subsection*{Question 3(c) OR [7
marks]}\label{q3c}

\textbf{Write a program to read number of lines (n) from keyboard and
print the triangle shown below. For Example, n=5}

\begin{verbatim}
1 2 3 4 5
1 2 3 4
1 2 3
1 2
1
\end{verbatim}

\begin{solutionbox}

\textbf{Program:}

\begin{verbatim}
\#include {stdio.h}
void main()
\{
    int n, i, j;
    
    printf("Enter number of lines: ");
    scanf("\%d", \&n);
    
    for(i = n; i {=} 1; i{-{-})}
    \{
        for(j = 1; j {=} i; j++)
        \{
            printf("\%d ", j);
        \}
        printf("{n}");
    \}
\}
\end{verbatim}

\textbf{Pattern Logic Table:} For n = 5:

{\def\LTcaptype{none} % do not increment counter
\begin{longtable}[]{@{}lll@{}}
\toprule\noalign{}
i & j & Output \\
\midrule\noalign{}
\endhead
\bottomrule\noalign{}
\endlastfoot
5 & j=1 to 5 & 1 2 3 4 5 \\
4 & j=1 to 4 & 1 2 3 4 \\
3 & j=1 to 3 & 1 2 3 \\
2 & j=1 to 2 & 1 2 \\
1 & j=1 to 1 & 1 \\
\end{longtable}
}

\textbf{Pattern Visualization:}

\begin{verbatim}
1 2 3 4 5
1 2 3 4
1 2 3
1 2
1
\end{verbatim}

\textbf{Program Flow:}

\begin{verbatim}
flowchart LR
    A([Start]) {-{-} B[/Input n/]}
    B {-{-} C[outer loop: i = n to 1]}
    C {-{-} D[inner loop: j = 1 to i]}
    D {-{-} E[/Print j/]}
    E {-{-} F\{j  i?\}}
    F {-{-}|Yes| D}
    F {-{-}|No| G[/Print newline/]}
    G {-{-} H\{i  1?\}}
    H {-{-}|Yes| C}
    H {-{-}|No| I([End])}
\end{verbatim}

\end{solutionbox}
\begin{mnemonicbox}
``Decreasing Rows With Increasing Values'' (DRWIV)

\end{mnemonicbox}
\subsection*{Question 4(a) [3 marks]}\label{q4a}

\textbf{Explain nested if-else statement with example.}

\begin{solutionbox}

\textbf{Nested if-else}: An if-else statement inside another if or else
block.

\textbf{Structure:}

\begin{verbatim}
if (condition1) \{
    if (condition2) \{
        statements1;
    \} else \{
        statements2;
    \}
\} else \{
    statements3;
\}
\end{verbatim}

\textbf{Code Example:}

\begin{verbatim}
\#include {stdio.h}
void main()
\{
    int age, weight;
    
    printf("Enter age: ");
    scanf("\%d", \&age);
    
    if (age {=} 18) \{
        printf("Enter weight: ");
        scanf("\%d", \&weight);
        
        if (weight {=} 50) \{
            printf("Eligible to donate blood");
        \} else \{
            printf("Underweight, not eligible");
        \}
    \} else \{
        printf("Age below 18, not eligible");
    \}
\}
\end{verbatim}

\textbf{Nested if-else Diagram:}

\begin{verbatim}
flowchart LR
    A[Start] {-{-} B\{age = 18?\}}
    B {-{-}|Yes| C\{weight = 50?\}}
    B {-{-}|No| D[Not eligible: Age]}
    C {-{-}|Yes| E[Eligible]}
    C {-{-}|No| F[Not eligible: Weight]}
    D \& E \& F {-{-} G[End]}
\end{verbatim}

\end{solutionbox}
\begin{mnemonicbox}
``Check Outside Then Inside'' (COTI)

\end{mnemonicbox}
\subsection*{Question 4(b) [4 marks]}\label{q4b}

\textbf{Write a program to exchange two integer numbers using pointer
arguments.}

\begin{solutionbox}

\textbf{Program:}

\begin{verbatim}
\#include {stdio.h}
void main()
\{
    int a, b, temp;
    int *p1, *p2;
    
    printf("Enter value of a: ");
    scanf("\%d", \&a);
    
    printf("Enter value of b: ");
    scanf("\%d", \&b);
    
    p1 = \&a;  // p1 points to a
    p2 = \&b;  // p2 points to b
    
printf("Before swapping:

a = \%d,

b = \%d{n}", a, b);

    
    // Swapping using pointers
    temp = *p1;
    *p1 = *p2;
    *p2 = temp;
    
printf("After swapping:

a = \%d,

b = \%d", a, b);

\}
\end{verbatim}

\textbf{Pointer Swapping Diagram:}

\begin{verbatim}
        +{-{-}{-}+        +{-}{-}{-}+}
        | 5 |{{-}{-}{-}{-}{-}{-}{-}|p1 |}
   a {- +{-}{-}{-}+        +{-}{-}{-}+}
   
        +{-{-}{-}+        +{-}{-}{-}+}
        | 10|{{-}{-}{-}{-}{-}{-}{-}|p2 |}
   b {- +{-}{-}{-}+        +{-}{-}{-}+}
   
   After swapping:
   
        +{-{-}{-}+        +{-}{-}{-}+}
        | 10|{{-}{-}{-}{-}{-}{-}{-}|p1 |}
   a {- +{-}{-}{-}+        +{-}{-}{-}+}
   
        +{-{-}{-}+        +{-}{-}{-}+}
        | 5 |{{-}{-}{-}{-}{-}{-}{-}|p2 |}
   b {- +{-}{-}{-}+        +{-}{-}{-}+}
\end{verbatim}

\end{solutionbox}
\begin{mnemonicbox}
``Pointers Exchange Memory Values'' (PEMV)

\end{mnemonicbox}
\subsection*{Question 4(c) [7 marks]}\label{q4c}

\textbf{Define Array. Explain initialization \& declaration of
one-dimensional array.}

\begin{solutionbox}

\textbf{Array}: A collection of elements of the same data type stored in
contiguous memory locations and accessed using indices.


{\def\LTcaptype{none} % do not increment counter
\vspace{-5pt}
\captionof{table}{Array Declaration \& Initialization}
\vspace{-10pt}
\begin{longtable}[]{@{}
  >{\raggedright\arraybackslash}p{(\linewidth - 4\tabcolsep) * \real{0.3929}}
  >{\raggedright\arraybackslash}p{(\linewidth - 4\tabcolsep) * \real{0.2857}}
  >{\raggedright\arraybackslash}p{(\linewidth - 4\tabcolsep) * \real{0.3214}}@{}}
\toprule\noalign{}
\begin{minipage}[b]{\linewidth}\raggedright
Operation
\end{minipage} & \begin{minipage}[b]{\linewidth}\raggedright
Syntax
\end{minipage} & \begin{minipage}[b]{\linewidth}\raggedright
Example
\end{minipage} \\
\midrule\noalign{}
\endhead
\bottomrule\noalign{}
\endlastfoot
Declaration & data\_type array\_name[size]; & int marks[5]; \\
Initialization at declaration & data\_type array\_name[size] =
\{values\}; & int nums[4] = \{10, 20, 30, 40\}; \\
Partial initialization & data\_type array\_name[size] = \{values\};
& int nums[5] = \{10, 20\}; \\
Without size & data\_type array\_name[] = \{values\}; & int
nums[] = \{10, 20, 30\}; \\
Individual element & array\_name[index] = value; & marks[0] =
95; \\
\end{longtable}
}

\textbf{Code Example:}

\begin{verbatim}
\#include {stdio.h}
void main()
\{
    // Declaration
    int marks[5];
    
    // Initialization after declaration
    marks[0] = 85;
    marks[1] = 90;
    marks[2] = 78;
    marks[3] = 92;
    marks[4] = 88;
    
    // Declaration with initialization
    int scores[] = \{95, 89, 76, 82, 91\;}
    
    // Accessing array elements
    printf("marks[2] = \%d{n}", marks[2]);
    printf("scores[3] = \%d", scores[3]);
\}
\end{verbatim}

\textbf{Array Representation:}

\begin{verbatim}
marks: [85][90][78][92][88]
        |   |   |   |   |
        0   1   2   3   4  (indices)
\end{verbatim}

\textbf{Memory Representation:}

\begin{verbatim}
flowchart LR
    A["marks[0]{br /85"] {-}{-}{-} B["marks[1]br /90"]}
    B {-{-}{-} C["marks[2]br /78"]}
    C {-{-}{-} D["marks[3]br /92"]}
    D {-{-}{-} E["marks[4]br /88"]}
\end{verbatim}

\end{solutionbox}
\begin{mnemonicbox}
``Declare, Initialize, Access With Index'' (DIAWI)

\end{mnemonicbox}
\subsection*{Question 4(a) OR [3
marks]}\label{q4a}

\textbf{Explain do while loop with example.}

\begin{solutionbox}

\textbf{do-while loop}: A loop that executes the body at least once
before checking the condition.

\textbf{Structure:}

\begin{verbatim}
do \{
    statements;
\} while(condition);
\end{verbatim}

\textbf{Code Example:}

\begin{verbatim}
\#include {stdio.h}
void main()
\{
    int num, sum = 0;
    
    do \{
        printf("Enter a number (0 to stop): ");
        scanf("\%d", \&num);
        sum += num;
    \} while(num != 0);
    
    printf("Sum of entered numbers = \%d", sum);
\}
\end{verbatim}

\textbf{do-while Loop Flow:}

\begin{verbatim}
flowchart LR
    A([Start]) {-{-} B[Body statements]}
    B {-{-} C\{Condition\}}
    C {-{-}|True| B}
    C {-{-}|False| D([End])}
\end{verbatim}

\textbf{Key Differences from while loop:}

\begin{itemize}
\tightlist
\item
  Body executes at least once
\item
  Condition checked after execution
\item
  Semicolon required after condition
\end{itemize}

\end{solutionbox}
\begin{mnemonicbox}
``Do First, Check Later'' (DFCL)

\end{mnemonicbox}
\subsection*{Question 4(b) OR [4
marks]}\label{q4b}

\textbf{Explain following functions with example:} \textbf{(1) gets()
(2) puts() (3) strlen() (4) strcpy()}

\begin{solutionbox}


{\def\LTcaptype{none} % do not increment counter
\vspace{-5pt}
\captionof{table}{String Functions in C}
\vspace{-10pt}
\begin{longtable}[]{@{}
  >{\raggedright\arraybackslash}p{(\linewidth - 6\tabcolsep) * \real{0.2778}}
  >{\raggedright\arraybackslash}p{(\linewidth - 6\tabcolsep) * \real{0.2500}}
  >{\raggedright\arraybackslash}p{(\linewidth - 6\tabcolsep) * \real{0.2222}}
  >{\raggedright\arraybackslash}p{(\linewidth - 6\tabcolsep) * \real{0.2500}}@{}}
\toprule\noalign{}
\begin{minipage}[b]{\linewidth}\raggedright
Function
\end{minipage} & \begin{minipage}[b]{\linewidth}\raggedright
Purpose
\end{minipage} & \begin{minipage}[b]{\linewidth}\raggedright
Syntax
\end{minipage} & \begin{minipage}[b]{\linewidth}\raggedright
Example
\end{minipage} \\
\midrule\noalign{}
\endhead
\bottomrule\noalign{}
\endlastfoot
gets() & Reads string with spaces & gets(string); & gets(name); \\
puts() & Displays string with newline & puts(string); & puts(name); \\
strlen() & Returns string length & strlen(string); & n =
strlen(name); \\
strcpy() & Copies source to destination & strcpy(dest, src); &
strcpy(str1, str2); \\
\end{longtable}
}

\textbf{Code Example:}

\begin{verbatim}
\#include {stdio.h}
\#include {string.h}
void main()
\{
    char name[50], copy[50];
    int length;
    
    printf("Enter your name: ");
    gets(name);           // Read name with spaces
    
    puts("Your name is:"); // Display with newline
    puts(name);
    
    length = strlen(name); // Get string length
    printf("Length: \%d{n}", length);
    
    strcpy(copy, name);    // Copy name to copy
    printf("Copied string: \%s", copy);
\}
\end{verbatim}

\end{solutionbox}
\begin{mnemonicbox}
``Gets Puts String's Length and Copies'' (GPSLC)

\end{mnemonicbox}
\subsection*{Question 4(c) OR [7
marks]}\label{q4c}

\textbf{Define recursion and explain with suitable example. Write a
program to find factorial of a given number using recursion.}

\begin{solutionbox}

\textbf{Recursion}: A process where a function calls itself directly or
indirectly until a specific condition is met.

\textbf{Recursion Components:}

\begin{enumerate}
\tightlist
\item
  Base case: Condition to stop recursion
\item
  Recursive case: Function calling itself
\end{enumerate}

\textbf{Code Example:}

\begin{verbatim}
\#include {stdio.h}

// Recursive function to find factorial
unsigned long factorial(int n)
\{
    // Base case
if (n == 0 ||

n == 1)

        return 1;
    
    // Recursive case
    else
        return n * factorial(n{-}1);
\}

void main()
\{
    int num;
    unsigned long result;
    
    printf("Enter a number: ");
    scanf("\%d", \&num);
    
    if (num {} 0)
        printf("Factorial not defined for negative numbers");
    else
    \{
        result = factorial(num);
        printf("Factorial of \%d = \%lu", num, result);
    \}
\}
\end{verbatim}

\textbf{Recursive Factorial Calculation:} For factorial(5)


{\def\LTcaptype{none} % do not increment counter
\vspace{-5pt}
\captionof{table}{Recursion Trace}
\vspace{-10pt}
\begin{longtable}[]{@{}lll@{}}
\toprule\noalign{}
Call & Returns & Calculation \\
\midrule\noalign{}
\endhead
\bottomrule\noalign{}
\endlastfoot
factorial(5) & 5 \times factorial(4) & 5 \times 24 = 120 \\
factorial(4) & 4 \times factorial(3) & 4 \times 6 = 24 \\
factorial(3) & 3 \times factorial(2) & 3 \times 2 = 6 \\
factorial(2) & 2 \times factorial(1) & 2 \times 1 = 2 \\
factorial(1) & 1 & Base case \\
\end{longtable}
}

\textbf{Recursion Diagram:}

\begin{verbatim}
flowchart LR
    A["factorial(5)"] {-{-} B["5 * factorial(4)"]}
    B {-{-} C["4 * factorial(3)"]}
    C {-{-} D["3 * factorial(2)"]}
    D {-{-} E["2 * factorial(1)"]}
    E {-{-} F["return 1"]}
    F {-{-} G["return 2"]}
    G {-{-} H["return 6"]}
    H {-{-} I["return 24"]}
    I {-{-} J["return 120"]}
\end{verbatim}

\end{solutionbox}
\begin{mnemonicbox}
``Function Calling Itself, Bottoming Out'' (FCIBO)

\end{mnemonicbox}
\subsection*{Question 5(a) [3 marks]}\label{q5a}

\textbf{Write the difference between array and structure.}

\begin{solutionbox}


{\def\LTcaptype{none} % do not increment counter
\vspace{-5pt}
\captionof{table}{Array vs Structure}
\vspace{-10pt}
\begin{longtable}[]{@{}
  >{\raggedright\arraybackslash}p{(\linewidth - 4\tabcolsep) * \real{0.3333}}
  >{\raggedright\arraybackslash}p{(\linewidth - 4\tabcolsep) * \real{0.2593}}
  >{\raggedright\arraybackslash}p{(\linewidth - 4\tabcolsep) * \real{0.4074}}@{}}
\toprule\noalign{}
\begin{minipage}[b]{\linewidth}\raggedright
Feature
\end{minipage} & \begin{minipage}[b]{\linewidth}\raggedright
Array
\end{minipage} & \begin{minipage}[b]{\linewidth}\raggedright
Structure
\end{minipage} \\
\midrule\noalign{}
\endhead
\bottomrule\noalign{}
\endlastfoot
Data type & Same data type for all elements & Can store different data
types \\
Access & Using index (arr[0]) & Using member name (s.name) \\
Memory allocation & Contiguous & Contiguous but different sizes \\
Size & Fixed size at declaration & Sum of sizes of all members \\
Purpose & Collection of similar items & Grouping related data of
different types \\
Declaration & \texttt{int\ arr[5];} &
\texttt{struct\ student\ \{\ int\ id;\ char\ name[20];\ \};} \\
\end{longtable}
}

\textbf{Diagram:}

\begin{verbatim}
flowchart TD
    subgraph Array
    direction LR
    A["[0]{br /int"] {-}{-}{-} B["[1]br /int"] {-}{-}{-} C["[2]br /int"]}
    end
    
    subgraph Structure
    direction LR
    D["id{br /int"] {-}{-}{-} E["namebr /char[]"] {-}{-}{-} F["agebr /int"]}
    end
\end{verbatim}

\end{solutionbox}
\begin{mnemonicbox}
``Arrays for Same, Structures for Different'' (ASSD)

\end{mnemonicbox}
\subsection*{Question 5(b) [4 marks]}\label{q5b}

\textbf{Write a C program using array that find the maximum value from
given 10 values.}

\begin{solutionbox}

\textbf{Program:}

\begin{verbatim}
\#include {stdio.h}
void main()
\{
    int arr[10], i, max;
    
    // Input 10 values
    printf("Enter 10 values:{n}");
    for(i = 0; i {} 10; i++)
    \{
        printf("Enter value \%d: ", i+1);
        scanf("\%d", \&arr[i]);
    \}
    
    // Find maximum value
    max = arr[0];  // Assume first element is maximum
    for(i = 1; i {} 10; i++)
    \{
        if(arr[i] {} max)
            max = arr[i];
    \}
    
    printf("Maximum value is: \%d", max);
\}
\end{verbatim}

\textbf{Algorithm Flow:}

\begin{verbatim}
flowchart LR
    A([Start]) {-{-} B[/Input 10 values/]}
    B {-{-} C[Set max = first element]}
    C {-{-} D[Loop i from 1 to 9]}
    D {-{-} E\{"arr[i]  max?"\}}
    E {-{-}|Yes| F["max = arr[i]"]}
    E {-{-}|No| G[Continue]}
    F \& G {-{-} H\{i  9?\}}
    H {-{-}|Yes| D}
    H {-{-}|No| I[/Display max/]}
    I {-{-} J([End])}
\end{verbatim}

\end{solutionbox}
\begin{mnemonicbox}
``Compare And Replace Maximum'' (CARM)

\end{mnemonicbox}
\subsection*{Question 5(c) [7 marks]}\label{q5c}

\textbf{Define structure? Develop a structure named book to save
following information about books. Book title, Name of author, Price and
Number of pages.}

\begin{solutionbox}

\textbf{Structure}: A user-defined data type that groups related
variables of different data types under a single name.

\textbf{Book Structure Code:}

\begin{verbatim}
\#include {stdio.h}

struct book \{
    char title[50];
    char author[30];
    float price;
    int pages;
\;}

void main()
\{
    struct book b1;
    
    // Input book details
    printf("Enter book title: ");
    gets(b1.title);
    
    printf("Enter author name: ");
    gets(b1.author);
    
    printf("Enter price: ");
    scanf("\%f", \&b1.price);
    
    printf("Enter number of pages: ");
    scanf("\%d", \&b1.pages);
    
    // Display book details
    printf("{n}Book Details:{n}");
    printf("Title: \%s{n}", b1.title);
    printf("Author: \%s{n}", b1.author);
    printf("Price: Rs. \%.2f{n}", b1.price);
    printf("Pages: \%d", b1.pages);
\}
\end{verbatim}

\textbf{Structure Memory Representation:}

\begin{verbatim}
+{-{-}{-}{-}{-}{-}{-}{-}{-}{-}{-}{-}{-}{-}{-}{-}{-}{-}{-}{-}{-}{-}{-}{-}+}
| struct book            |
+{-{-}{-}{-}{-}{-}{-}{-}{-}{-}{-}{-}{-}{-}{-}{-}{-}{-}{-}{-}{-}{-}{-}{-}+}
| title[50]  |           |
|            |  "C Prog" |
+{-{-}{-}{-}{-}{-}{-}{-}{-}{-}{-}{-}{-}{-}{-}{-}{-}{-}{-}{-}{-}{-}{-}{-}+}
| author[30] |           |
|            |  "Dennis" |
+{-{-}{-}{-}{-}{-}{-}{-}{-}{-}{-}{-}{-}{-}{-}{-}{-}{-}{-}{-}{-}{-}{-}{-}+}
| price      |   450.50  |
+{-{-}{-}{-}{-}{-}{-}{-}{-}{-}{-}{-}{-}{-}{-}{-}{-}{-}{-}{-}{-}{-}{-}{-}+}
| pages      |    320    |
+{-{-}{-}{-}{-}{-}{-}{-}{-}{-}{-}{-}{-}{-}{-}{-}{-}{-}{-}{-}{-}{-}{-}{-}+}
\end{verbatim}

\textbf{Structure Diagram:}

\begin{verbatim}
classDiagram
    class book \{
        char title[50]
        char author[30]
        float price
        int pages
    \}
\end{verbatim}

\end{solutionbox}
\begin{mnemonicbox}
``Title Author Price Pages'' (TAPP)

\end{mnemonicbox}
\subsection*{Question 5(a) OR [3
marks]}\label{q5a}

\textbf{What is a string? What are the operations that can be performed
on string?}

\begin{solutionbox}

\textbf{String}: A sequence of characters terminated by a null character
`\textbackslash0'.


{\def\LTcaptype{none} % do not increment counter
\vspace{-5pt}
\captionof{table}{String Operations in C}
\vspace{-10pt}
\begin{longtable}[]{@{}lll@{}}
\toprule\noalign{}
Operation & Function & Example \\
\midrule\noalign{}
\endhead
\bottomrule\noalign{}
\endlastfoot
Input & gets(), scanf() & gets(str), scanf(``\%s'', str) \\
Output & puts(), printf() & puts(str), printf(``\%s'', str) \\
Length & strlen() & len = strlen(str) \\
Copy & strcpy() & strcpy(dest, src) \\
Concatenate & strcat() & strcat(str1, str2) \\
Compare & strcmp() & result = strcmp(str1, str2) \\
Search & strchr(), strstr() & ptr = strchr(str, `a') \\
Convert & strlwr(), strupr() & strlwr(str), strupr(str) \\
\end{longtable}
}

\textbf{String Representation:}

\begin{verbatim}
+{-{-}{-}+{-}{-}{-}+{-}{-}{-}+{-}{-}{-}+{-}{-}{-}+{-}{-}{-}+}
| H | e | l | l | o | {0|}
+{-{-}{-}+{-}{-}{-}+{-}{-}{-}+{-}{-}{-}+{-}{-}{-}+{-}{-}{-}+}
\end{verbatim}

\end{solutionbox}
\begin{mnemonicbox}
``Input Output Length Copy Concat Compare Search
Convert'' (IOLCCSC)

\end{mnemonicbox}
\subsection*{Question 5(b) OR [4
marks]}\label{q5b}

\textbf{Write a program prints its ASCII value from A to Z.}

\begin{solutionbox}

\textbf{Program:}

\begin{verbatim}
\#include {stdio.h}
void main()
\{
    char ch;
    
    printf("ASCII values from A to Z:{n}");
    printf("Character{t}ASCII Value{n}");
    printf("{-{-}{-}{-}{-}{-}{-}{-}{-}{-}{-}{-}{-}{-}{-}{-}{-}{-}{-}{-}{-}{-}{-}}{n}");
    
    for(ch = {A}; ch {=} {Z}; ch++)
    \{
        printf("    \%c{tt}   \%d{n}", ch, ch);
    \}
\}
\end{verbatim}

\textbf{Sample Output Table:}

{\def\LTcaptype{none} % do not increment counter
\begin{longtable}[]{@{}ll@{}}
\toprule\noalign{}
Character & ASCII Value \\
\midrule\noalign{}
\endhead
\bottomrule\noalign{}
\endlastfoot
A & 65 \\
B & 66 \\
\ldots{} & \ldots{} \\
Z & 90 \\
\end{longtable}
}

\textbf{ASCII Chart Representation:}

\begin{verbatim}
ASCII Values:
A(65) B(66) C(67) ... Z(90)
\end{verbatim}

\textbf{Mnemaid:} ``Alphabets Sequentially Creating Integer Indices''
(ASCII)

\end{solutionbox}
\subsection*{Question 5(c) OR [7
marks]}\label{q5c}

\textbf{What is user defined and library function? Explain with two
examples of each.}

\begin{solutionbox}

\textbf{Library Functions}: Pre-defined functions provided by C language
that are ready to use.

\textbf{User-Defined Functions}: Functions created by the programmer to
perform specific tasks.


{\def\LTcaptype{none} % do not increment counter
\vspace{-5pt}
\captionof{table}{Library vs User-Defined Functions}
\vspace{-10pt}
\begin{longtable}[]{@{}lll@{}}
\toprule\noalign{}
Feature & Library Functions & User-Defined Functions \\
\midrule\noalign{}
\endhead
\bottomrule\noalign{}
\endlastfoot
Definition & Pre-defined in header files & Created by programmer \\
Declaration & No need to define & Must be defined \\
Examples & printf(), scanf(), strlen() & calculateArea(), findMax() \\
Header files & stdio.h, string.h, math.h, etc. & No header required \\
Purpose & Common tasks & Customized tasks \\
\end{longtable}
}

\textbf{Examples of Library Functions:}

\begin{enumerate}
\tightlist
\item
  \textbf{strlen() - String Length}
\end{enumerate}

\begin{verbatim}
\#include {stdio.h}
\#include {string.h}
void main()
\{
    char str[] = "Hello";
    int length = strlen(str);  // Library function
    printf("Length of string: \%d", length);
\}
\end{verbatim}

\begin{enumerate}
\tightlist
\item
  \textbf{sqrt() - Square Root}
\end{enumerate}

\begin{verbatim}
\#include {stdio.h}
\#include {math.h}
void main()
\{
    float num = 25, result;
    result = sqrt(num);  // Library function
    printf("Square root of \%.0f = \%.2f", num, result);
\}
\end{verbatim}

\textbf{Examples of User-Defined Functions:}

\begin{enumerate}
\tightlist
\item
  \textbf{calculateArea() - Area of Rectangle}
\end{enumerate}

\begin{verbatim}
\#include {stdio.h}

// User{-defined function}
float calculateArea(float length, float width)
\{
    return length * width;
\}

void main()
\{
    float length = 10.5, width = 5.5, area;
    area = calculateArea(length, width);  // User function call
    printf("Area of rectangle = \%.2f", area);
\}
\end{verbatim}

\begin{enumerate}
\tightlist
\item
  \textbf{findMax() - Maximum of Three Numbers}
\end{enumerate}

\begin{verbatim}
\#include {stdio.h}

// User{-defined function}
int findMax(int a, int b, int c)
\{
    if(a {=} b \&\& a {=} c)
        return a;
    else if(b {=} a \&\& b {=} c)
        return b;
    else
        return c;
\}

void main()
\{
int

x = 10,

y = 25,

z = 15, max;

    max = findMax(x, y, z);  // User function call
    printf("Maximum number is: \%d", max);
\}
\end{verbatim}

\end{solutionbox}
\begin{mnemonicbox}
``Libraries Provide, Users Create'' (LPUC)

\end{mnemonicbox}

\end{document}
