\documentclass[10pt,a4paper]{article}

% content/resources/templates/preamble.tex
\usepackage[margin=0.6in]{geometry}
\author{Milav Dabgar}
\usepackage{amsmath,amssymb,amsthm}
\usepackage{booktabs}
\usepackage{multirow}
\usepackage{xcolor}
\usepackage{tcolorbox}
\tcbuselibrary{breakable,skins}
\usepackage[colorlinks=true,linkcolor=blue]{hyperref}
\usepackage{titlesec}
\usepackage{enumitem}
\usepackage{tikz}
\usepackage{pgfplots}
\usepackage{circuitikz}
\usepackage[version=4]{mhchem}
\usepackage{longtable}
\usepackage{array}
\usepackage{float}
\usepackage{caption}
\usepackage{listings}

\lstset{
  basicstyle=\small\ttfamily,
  breaklines=true,
  breakatwhitespace=false,
  postbreak=\mbox{\textcolor{red}{$\hookrightarrow$}\space},
  float=false,
  numbers=left,
  numberstyle=\tiny\color{gray},
  numbersep=10pt,
  xleftmargin=2em,
  keywordstyle=\color{blue},
  commentstyle=\color{green!60!black},
  stringstyle=\color{purple},
  backgroundcolor=\color{gray!5},
  showstringspaces=false,
  tabsize=2,
  captionpos=b,
  keepspaces=true,
  columns=flexible
}

\pgfplotsset{compat=1.18}
\usetikzlibrary{shapes,arrows,positioning,calc,patterns,decorations.pathmorphing,decorations.markings,arrows.meta}

% Color scheme
\definecolor{headcolor}{RGB}{0,102,204}
\definecolor{keycolor}{RGB}{220,20,60}
\definecolor{solutioncolor}{RGB}{34,139,34}
\definecolor{mnemoniccolor}{RGB}{148,0,211}
\definecolor{codecolor}{RGB}{0,0,100}

% Spacing
\setlength{\parskip}{3pt}
\setlist[itemize]{nosep}
\setlist[enumerate]{nosep}

% Title formatting
\titleformat{\section}{\Large\bfseries\color{headcolor}}{\thesection}{1em}{}
\titleformat{\subsection}{\large\bfseries\color{headcolor}}{\thesubsection}{1em}{}

% Pandoc tightlist compatibility
\providecommand{\tightlist}{%
  \setlength{\itemsep}{0pt}\setlength{\parskip}{0pt}}

% Pandoc longtable compatibility
\newcounter{none}
\def\thenone{}


% content/resources/templates/english-boxes.tex
% This file is currently empty - it exists to maintain consistency with the import structure.
% Add custom environments here if needed in the future.


\begin{document}

\begin{center}
{\Huge\bfseries\color{headcolor} Subject Name Solutions}\\[5pt]
{\LARGE 4331101 -- Winter 2024}\\[3pt]
{\large Semester 1 Study Material}\\[3pt]
{\normalsize\textit{Detailed Solutions and Explanations}}
\end{center}

\vspace{10pt}

\subsection*{Question 1(a) [3 marks]}\label{q1a}

\textbf{Define (i) Node (ii) Branch and (iii) Loop for electronic
network.}

\begin{solutionbox}

\textbf{Node}:

\begin{itemize}
\tightlist
\item
  \textbf{Junction point} where two or more branches meet in a network
\item
  Points where elements are connected together
\item
  Current sum of all branches at a node equals zero
\end{itemize}

\textbf{Branch}:

\begin{itemize}
\tightlist
\item
  \textbf{Single element} (R, L, or C) or path connecting two nodes
\item
  Each branch has a specific current flowing through it
\item
  Active branches contain sources; passive branches contain R, L, C
\end{itemize}

\textbf{Loop}:

\begin{itemize}
\tightlist
\item
  \textbf{Closed path} in a network formed by connected branches
\item
  No node is encountered more than once
\item
  Used in loop analysis for solving networks
\end{itemize}

\end{solutionbox}
\begin{mnemonicbox}
``NBL: Nodes join, Branches connect, Loops circle''

\end{mnemonicbox}
\subsection*{Question 1(b) [4 marks]}\label{q1b}

\textbf{Three resistors of 200 Ω, 300 Ω and 500 Ω are connected in
parallel across 100 V supply. Find (i) Current flowing through each
resistor and Total current (ii) Equivalent Resistance}

\begin{solutionbox}

\textbf{Table of Calculations:}

{\def\LTcaptype{none} % do not increment counter
\begin{longtable}[]{@{}llll@{}}
\toprule\noalign{}
Parameter & Formula & Calculation & Result \\
\midrule\noalign{}
\endhead
\bottomrule\noalign{}
\endlastfoot
I_{1} (200Ω) & I = V/R & 100V/200Ω & 0.5A \\
I_{2} (300Ω) & I = V/R & 100V/300Ω & 0.333A \\
I_{3} (500Ω) & I = V/R & 100V/500Ω & 0.2A \\
I_{(}_{t}_{o}_{t}_{a}_{l}_{)} & I_{1}+I_{2}+I_{3} & 0.5+0.333+0.2 & 1.033A \\
R_{(}_{e}q_{)} & 1/R_{(}_{e}q_{)} = 1/R_{1}+1/R_{2}+1/R_{3} & 1/200+1/300+1/500 & 96.77Ω \\
\end{longtable}
}

\end{solutionbox}
\begin{mnemonicbox}
``Parallel paths divide current inversely with
resistance''

\end{mnemonicbox}
\subsection*{Question 1(c) [7 marks]}\label{q1c}

\textbf{Explain Series and Parallel connection for Capacitors}

\begin{solutionbox}

\textbf{Capacitors in Series:}

\begin{center}
\textbf{Mermaid Diagram (Code)}
\begin{verbatim}
{Shaded}
{Highlighting}[]
graph LR
    A["{+"] {-}{-}{-} B[C_{1}] {-}{-}{-} C[C_{2}] {-}{-}{-} D[C_{3}] {-}{-}{-} E["{}{-}"]}
{Highlighting}
{Shaded}
\end{verbatim}
\end{center}


{\def\LTcaptype{none} % do not increment counter
\vspace{-5pt}
\captionof{table}{Series Capacitors Properties}
\vspace{-10pt}
\begin{longtable}[]{@{}
  >{\raggedright\arraybackslash}p{(\linewidth - 4\tabcolsep) * \real{0.3125}}
  >{\raggedright\arraybackslash}p{(\linewidth - 4\tabcolsep) * \real{0.2812}}
  >{\raggedright\arraybackslash}p{(\linewidth - 4\tabcolsep) * \real{0.4062}}@{}}
\toprule\noalign{}
\begin{minipage}[b]{\linewidth}\raggedright
Property
\end{minipage} & \begin{minipage}[b]{\linewidth}\raggedright
Formula
\end{minipage} & \begin{minipage}[b]{\linewidth}\raggedright
Description
\end{minipage} \\
\midrule\noalign{}
\endhead
\bottomrule\noalign{}
\endlastfoot
Equivalent Capacitance & 1/C_{(}_{e}q_{)} = 1/C_{1} + 1/C_{2} + 1/C_{3} & Always smaller
than smallest capacitor \\
Charge &

Q = Q_{1} = Q_{2} = Q_{3} & Same on all capacitors \\

Voltage & V = V_{1} + V_{2} + V_{3} & Divides according to 1/C ratio \\
Energy & E = CV^{2}/2 & Distributed across capacitors \\
\end{longtable}
}

\textbf{Capacitors in Parallel:}

\begin{center}
\textbf{Mermaid Diagram (Code)}
\begin{verbatim}
{Shaded}
{Highlighting}[]
graph LR
    A["{+"] {-}{-}{-} B["{}+"]}
    B {-{-}{-} C[C_{1}] {-}{-}{-} D["{}{-}"]}
    B {-{-}{-} E[C_{2}] {-}{-}{-} D}
    B {-{-}{-} F[C_{3}] {-}{-}{-} D}
    A {-{-}{-} D}
{Highlighting}
{Shaded}
\end{verbatim}
\end{center}


{\def\LTcaptype{none} % do not increment counter
\vspace{-5pt}
\captionof{table}{Parallel Capacitors Properties}
\vspace{-10pt}
\begin{longtable}[]{@{}
  >{\raggedright\arraybackslash}p{(\linewidth - 4\tabcolsep) * \real{0.3125}}
  >{\raggedright\arraybackslash}p{(\linewidth - 4\tabcolsep) * \real{0.2812}}
  >{\raggedright\arraybackslash}p{(\linewidth - 4\tabcolsep) * \real{0.4062}}@{}}
\toprule\noalign{}
\begin{minipage}[b]{\linewidth}\raggedright
Property
\end{minipage} & \begin{minipage}[b]{\linewidth}\raggedright
Formula
\end{minipage} & \begin{minipage}[b]{\linewidth}\raggedright
Description
\end{minipage} \\
\midrule\noalign{}
\endhead
\bottomrule\noalign{}
\endlastfoot
Equivalent Capacitance & C_{(}_{e}q_{)} = C_{1} + C_{2} + C_{3} & Sum of individual
capacitances \\
Charge & Q = Q_{1} + Q_{2} + Q_{3} & Distributes according to C value \\
Voltage &

V = V_{1} = V_{2} = V_{3} & Same across all capacitors \\

Energy & E = CV^{2}/2 & Sum of individual energies \\
\end{longtable}
}

\end{solutionbox}
\begin{mnemonicbox}
``Series caps add reciprocally, parallel caps add
directly''

\end{mnemonicbox}
\subsection*{Question 1(c) OR [7
marks]}\label{q1c}

\textbf{Explain Series and Parallel connection for Inductors.}

\begin{solutionbox}

\textbf{Inductors in Series:}

\begin{center}
\textbf{Mermaid Diagram (Code)}
\begin{verbatim}
{Shaded}
{Highlighting}[]
graph LR
    A["{+"] {-}{-}{-} B[L_{1}] {-}{-}{-} C[L_{2}] {-}{-}{-} D[L_{3}] {-}{-}{-} E["{}{-}"]}
{Highlighting}
{Shaded}
\end{verbatim}
\end{center}


{\def\LTcaptype{none} % do not increment counter
\vspace{-5pt}
\captionof{table}{Series Inductors Properties}
\vspace{-10pt}
\begin{longtable}[]{@{}
  >{\raggedright\arraybackslash}p{(\linewidth - 4\tabcolsep) * \real{0.3125}}
  >{\raggedright\arraybackslash}p{(\linewidth - 4\tabcolsep) * \real{0.2812}}
  >{\raggedright\arraybackslash}p{(\linewidth - 4\tabcolsep) * \real{0.4062}}@{}}
\toprule\noalign{}
\begin{minipage}[b]{\linewidth}\raggedright
Property
\end{minipage} & \begin{minipage}[b]{\linewidth}\raggedright
Formula
\end{minipage} & \begin{minipage}[b]{\linewidth}\raggedright
Description
\end{minipage} \\
\midrule\noalign{}
\endhead
\bottomrule\noalign{}
\endlastfoot
Equivalent Inductance & L_{(}_{e}q_{)} = L_{1} + L_{2} + L_{3} & Sum of individual
inductances \\
Current &

I = I_{1} = I_{2} = I_{3} & Same through all inductors \\

Voltage & V = V_{1} + V_{2} + V_{3} & Divides according to L ratio \\
Energy & E = LI^{2}/2 & Sum of individual energies \\
\end{longtable}
}

\textbf{Inductors in Parallel:}

\begin{center}
\textbf{Mermaid Diagram (Code)}
\begin{verbatim}
{Shaded}
{Highlighting}[]
graph LR
    A["{+"] {-}{-}{-} B["{}+"]}
    B {-{-}{-} C[L_{1}] {-}{-}{-} D["{}{-}"]}
    B {-{-}{-} E[L_{2}] {-}{-}{-} D}
    B {-{-}{-} F[L_{3}] {-}{-}{-} D}
    A {-{-}{-} D}
{Highlighting}
{Shaded}
\end{verbatim}
\end{center}


{\def\LTcaptype{none} % do not increment counter
\vspace{-5pt}
\captionof{table}{Parallel Inductors Properties}
\vspace{-10pt}
\begin{longtable}[]{@{}
  >{\raggedright\arraybackslash}p{(\linewidth - 4\tabcolsep) * \real{0.3125}}
  >{\raggedright\arraybackslash}p{(\linewidth - 4\tabcolsep) * \real{0.2812}}
  >{\raggedright\arraybackslash}p{(\linewidth - 4\tabcolsep) * \real{0.4062}}@{}}
\toprule\noalign{}
\begin{minipage}[b]{\linewidth}\raggedright
Property
\end{minipage} & \begin{minipage}[b]{\linewidth}\raggedright
Formula
\end{minipage} & \begin{minipage}[b]{\linewidth}\raggedright
Description
\end{minipage} \\
\midrule\noalign{}
\endhead
\bottomrule\noalign{}
\endlastfoot
Equivalent Inductance & 1/L_{(}_{e}q_{)} = 1/L_{1} + 1/L_{2} + 1/L_{3} & Always smaller
than smallest inductor \\
Current & I = I_{1} + I_{2} + I_{3} & Divides according to 1/L ratio \\
Voltage &

V = V_{1} = V_{2} = V_{3} & Same across all inductors \\

Energy & E = LI^{2}/2 & Distributed across inductors \\
\end{longtable}
}

\end{solutionbox}
\begin{mnemonicbox}
``Series inductors add directly, parallel inductors
add reciprocally''

\end{mnemonicbox}
\subsection*{Question 2(a) [3 marks]}\label{q2a}

\textbf{Classify network elements.}

\begin{solutionbox}


{\def\LTcaptype{none} % do not increment counter
\vspace{-5pt}
\captionof{table}{Classification of Network Elements}
\vspace{-10pt}
\begin{longtable}[]{@{}
  >{\raggedright\arraybackslash}p{(\linewidth - 4\tabcolsep) * \real{0.3704}}
  >{\raggedright\arraybackslash}p{(\linewidth - 4\tabcolsep) * \real{0.2593}}
  >{\raggedright\arraybackslash}p{(\linewidth - 4\tabcolsep) * \real{0.3704}}@{}}
\toprule\noalign{}
\begin{minipage}[b]{\linewidth}\raggedright
Category
\end{minipage} & \begin{minipage}[b]{\linewidth}\raggedright
Types
\end{minipage} & \begin{minipage}[b]{\linewidth}\raggedright
Examples
\end{minipage} \\
\midrule\noalign{}
\endhead
\bottomrule\noalign{}
\endlastfoot
\textbf{Active vs Passive} & Active & Voltage/current sources,
transistors \\
& Passive & Resistors, capacitors, inductors \\
\textbf{Linear vs Non-linear} & Linear & Resistors, ideal sources \\
& Non-linear & Diodes, transistors \\
\textbf{Bilateral vs Unilateral} & Bilateral & Resistors, capacitors,
inductors \\
& Unilateral & Diodes, transistors \\
\textbf{Lumped vs Distributed} & Lumped & Discrete R, L, C components \\
& Distributed & Transmission lines \\
\end{longtable}
}

\end{solutionbox}
\begin{mnemonicbox}
``ALBU: Active/passive, Linear/non-linear,
Bilateral/unilateral, lumped/distributed''

\end{mnemonicbox}
\subsection*{Question 2(b) [4 marks]}\label{q2b}

\textbf{Three resistances of 10, 30 and 70 ohms are connected in star.
Find equivalent resistances in delta connection.}

\begin{solutionbox}

\textbf{Diagram: Star to Delta Conversion}

\begin{verbatim}
graph TB
    subgraph Star Connection
        A((1)) {-{-}{-} R1[10Ω]}
        B((2)) {-{-}{-} R2[30Ω]}
        C((3)) {-{-}{-} R3[70Ω]}
        R1 {-{-}{-} D((0))}
        R2 {-{-}{-} D}
        R3 {-{-}{-} D}
    end

    subgraph Delta Connection
        A1((1)) {-{-}{-} R12[R_{1}_{2}]}
        A1 {-{-}{-} R31[R_{3}_{1}]}
        B1((2)) {-{-}{-} R12}
        B1 {-{-}{-} R23[R_{2}_{3}]}
        C1((3)) {-{-}{-} R23}
        C1 {-{-}{-} R31}
    end
\end{verbatim}


{\def\LTcaptype{none} % do not increment counter
\vspace{-5pt}
\captionof{table}{Star-Delta Conversion Formulas and Calculations}
\vspace{-10pt}
\begin{longtable}[]{@{}llll@{}}
\toprule\noalign{}
Delta Resistance & Formula & Calculation & Result \\
\midrule\noalign{}
\endhead
\bottomrule\noalign{}
\endlastfoot
R_{1}_{2} & (R_{1}\timesR_{2}+R_{2}\timesR_{3}+R_{3}\timesR_{1})/R_{3} & (10\times30+30\times70+70\times10)/70 & 47.14Ω \\
R_{2}_{3} & (R_{1}\timesR_{2}+R_{2}\timesR_{3}+R_{3}\timesR_{1})/R_{1} & (10\times30+30\times70+70\times10)/10 & 330Ω \\
R_{3}_{1} & (R_{1}\timesR_{2}+R_{2}\timesR_{3}+R_{3}\timesR_{1})/R_{2} & (10\times30+30\times70+70\times10)/30 & 110Ω \\
\end{longtable}
}

\end{solutionbox}
\begin{mnemonicbox}
``Star-Delta: Product sum over opposite resistor''

\end{mnemonicbox}
\subsection*{Question 2(c) [7 marks]}\label{q2c}

\textbf{Explain π network.}

\begin{solutionbox}

\textbf{Diagram: π (Pi) Network}

\begin{center}
\textbf{Mermaid Diagram (Code)}
\begin{verbatim}
{Shaded}
{Highlighting}[]
graph LR
    A[Input] {-{-}{-} B((Node 1))}
    B {-{-}{-} C[Z_{1}] {-}{-}{-} D((Node 2))}
    D {-{-}{-} E[Output]}
    B {-{-}{-} F[Z_{3}] {-}{-}{-} G((Ground))}
    D {-{-}{-} H[Z_{2}] {-}{-}{-} G}
{Highlighting}
{Shaded}
\end{verbatim}
\end{center}


{\def\LTcaptype{none} % do not increment counter
\vspace{-5pt}
\captionof{table}{π Network Characteristics}
\vspace{-10pt}
\begin{longtable}[]{@{}
  >{\raggedright\arraybackslash}p{(\linewidth - 2\tabcolsep) * \real{0.4583}}
  >{\raggedright\arraybackslash}p{(\linewidth - 2\tabcolsep) * \real{0.5417}}@{}}
\toprule\noalign{}
\begin{minipage}[b]{\linewidth}\raggedright
Parameter
\end{minipage} & \begin{minipage}[b]{\linewidth}\raggedright
Description
\end{minipage} \\
\midrule\noalign{}
\endhead
\bottomrule\noalign{}
\endlastfoot
\textbf{Structure} & Two shunt impedances (Z_{3}, Z_{2}) and one series
impedance (Z_{1}) \\
\textbf{Transmission Parameters} & A = 1 + Z_{1}/Z_{2}, B = Z_{1}, C = 1/Z_{2} +
1/Z_{3} + Z_{1}/(Z_{2}\timesZ_{3}), D = 1 + Z_{1}/Z_{3} \\
\textbf{Impedance Parameters} & Z_{1}_{1} = Z_{1} + Z_{3}, Z_{1}_{2} = Z_{1}, Z_{2}_{1} = Z_{1}, Z_{2}_{2} =
Z_{1} + Z_{2} \\
\textbf{Image Impedance} & Z_{0}π = \sqrt(Z_{1}Z_{2}Z_{3}/(Z_{2}+Z_{3})) \\
\textbf{Applications} & Matching networks, filters, attenuators \\
\textbf{Conversion} & Can be converted to T-network \\
\end{longtable}
}

\end{solutionbox}
\begin{mnemonicbox}
``π has two legs down, one branch across''

\end{mnemonicbox}
\subsection*{Question 2(a) OR [3
marks]}\label{q2a}

\textbf{List the types of network.}

\begin{solutionbox}


{\def\LTcaptype{none} % do not increment counter
\vspace{-5pt}
\captionof{table}{Types of Networks}
\vspace{-10pt}
\begin{longtable}[]{@{}ll@{}}
\toprule\noalign{}
Category & Types \\
\midrule\noalign{}
\endhead
\bottomrule\noalign{}
\endlastfoot
\textbf{Based on Linearity} & Linear Networks, Non-linear Networks \\
\textbf{Based on Elements} & Passive Networks, Active Networks \\
\textbf{Based on Parameters} & Time-variant, Time-invariant Networks \\
\textbf{Based on Configuration} & T-Network, π-Network, Lattice
Network \\
\textbf{Based on Ports} & One-port, Two-port, Multi-port Networks \\
\textbf{Based on Symmetry} & Symmetrical, Asymmetrical Networks \\
\textbf{Based on Reciprocity} & Reciprocal, Non-reciprocal Networks \\
\end{longtable}
}

\end{solutionbox}
\begin{mnemonicbox}
``LEPCPS: Linearity, Elements, Parameters,
Configuration, Ports, Symmetry''

\end{mnemonicbox}
\subsection*{Question 2(b) OR [4
marks]}\label{q2b}

\textbf{Three resistances of 20, 50 and 100 ohms are connected in delta.
Find equivalent resistances in star connection.}

\begin{solutionbox}

\textbf{Diagram: Delta to Star Conversion}

\begin{verbatim}
graph TB
    subgraph Delta Connection
        A((1)) {-{-}{-} R12[20Ω]}
        A {-{-}{-} R31[100Ω]}
        B((2)) {-{-}{-} R12}
        B {-{-}{-} R23[50Ω]}
        C((3)) {-{-}{-} R23}
        C {-{-}{-} R31}
    end

    subgraph Star Connection
        A1((1)) {-{-}{-} R1[R_{1}]}
        B1((2)) {-{-}{-} R2[R_{2}]}
        C1((3)) {-{-}{-} R3[R_{3}]}
        R1 {-{-}{-} D((0))}
        R2 {-{-}{-} D}
        R3 {-{-}{-} D}
    end
\end{verbatim}


{\def\LTcaptype{none} % do not increment counter
\vspace{-5pt}
\captionof{table}{Delta-Star Conversion Formulas and Calculations}
\vspace{-10pt}
\begin{longtable}[]{@{}llll@{}}
\toprule\noalign{}
Star Resistance & Formula & Calculation & Result \\
\midrule\noalign{}
\endhead
\bottomrule\noalign{}
\endlastfoot
R_{1} & (R_{1}_{2}\timesR_{3}_{1})/(R_{1}_{2}+R_{2}_{3}+R_{3}_{1}) & (20\times100)/(20+50+100) & 11.76Ω \\
R_{2} & (R_{1}_{2}\timesR_{2}_{3})/(R_{1}_{2}+R_{2}_{3}+R_{3}_{1}) & (20\times50)/(20+50+100) & 5.88Ω \\
R_{3} & (R_{2}_{3}\timesR_{3}_{1})/(R_{1}_{2}+R_{2}_{3}+R_{3}_{1}) & (50\times100)/(20+50+100) & 29.41Ω \\
\end{longtable}
}

\end{solutionbox}
\begin{mnemonicbox}
``Delta-Star: Product of adjacent pairs over sum of
all''

\end{mnemonicbox}
\subsection*{Question 2(c) OR [7
marks]}\label{q2c}

\textbf{Explain T network.}

\begin{solutionbox}

\textbf{Diagram: T Network}

\begin{center}
\textbf{Mermaid Diagram (Code)}
\begin{verbatim}
{Shaded}
{Highlighting}[]
graph LR
    A[Input] {-{-}{-} B[Z_{1}] {-}{-}{-} C((Node))}
    C {-{-}{-} D[Z_{2}] {-}{-}{-} E[Output]}
    C {-{-}{-} F[Z_{3}] {-}{-}{-} G((Ground))}
{Highlighting}
{Shaded}
\end{verbatim}
\end{center}


{\def\LTcaptype{none} % do not increment counter
\vspace{-5pt}
\captionof{table}{T Network Characteristics}
\vspace{-10pt}
\begin{longtable}[]{@{}
  >{\raggedright\arraybackslash}p{(\linewidth - 2\tabcolsep) * \real{0.4583}}
  >{\raggedright\arraybackslash}p{(\linewidth - 2\tabcolsep) * \real{0.5417}}@{}}
\toprule\noalign{}
\begin{minipage}[b]{\linewidth}\raggedright
Parameter
\end{minipage} & \begin{minipage}[b]{\linewidth}\raggedright
Description
\end{minipage} \\
\midrule\noalign{}
\endhead
\bottomrule\noalign{}
\endlastfoot
\textbf{Structure} & Two series impedances (Z_{1}, Z_{2}) and one shunt
impedance (Z_{3}) \\
\textbf{Transmission Parameters} & A = 1 + Z_{1}/Z_{3}, B = Z_{1} + Z_{2} + Z_{1}Z_{2}/Z_{3},
C = 1/Z_{3},

D = 1 + Z_{2}/Z_{3} \\

\textbf{Impedance Parameters} & Z_{1}_{1} = Z_{1} + Z_{3}, Z_{1}_{2} = Z_{3}, Z_{2}_{1} = Z_{3}, Z_{2}_{2} =
Z_{2} + Z_{3} \\
\textbf{Image Impedance} & Z_{0}T = \sqrt(Z_{1}Z_{2} + Z_{1}Z_{3} + Z_{2}Z_{3}) \\
\textbf{Applications} & Matching networks, filters, attenuators \\
\textbf{Conversion} & Can be converted to π-network \\
\end{longtable}
}

\end{solutionbox}
\begin{mnemonicbox}
``T has two arms across, one leg down''

\end{mnemonicbox}
\subsection*{Question 3(a) [3 marks]}\label{q3a}

\textbf{Explain Kirchhoff's law.}

\begin{solutionbox}

\textbf{Kirchhoff's Current Law (KCL):}

\begin{itemize}
\tightlist
\item
  \textbf{Sum of currents} entering a node equals sum of currents
  leaving it
\item
  Algebraic sum of currents at any node is zero
\item
  \sumI = 0 (currents entering positive, leaving negative)
\end{itemize}

\textbf{Kirchhoff's Voltage Law (KVL):}

\begin{itemize}
\tightlist
\item
  \textbf{Sum of voltage drops} around any closed loop equals zero
\item
  \sumV = 0 (voltage rises positive, drops negative)
\item
  Based on conservation of energy
\end{itemize}

\textbf{Diagram of Kirchhoff's Laws:}

\begin{center}
\textbf{Mermaid Diagram (Code)}
\begin{verbatim}
{Shaded}
{Highlighting}[]
graph TD
    subgraph KCL
        A((Node)) {-{-}{-} B[I_{1}]}
        A {-{-}{-} C[I_{2}]}
        A {-{-}{-} D[I_{3}]}
        A {-{-}{-} E[I_{4}]}
    end

    subgraph KVL
    direction LR
        F[V_{1] {-}{-}{-} G[V_{2}] {-}{-}{-} H[V_{3}] {-}{-}{-} I[V_{4}] {-}{-}{-} F}
    end
{Highlighting}
{Shaded}
\end{verbatim}
\end{center}

\end{solutionbox}
\begin{mnemonicbox}
``Current converges, Voltage voyages in a loop''

\end{mnemonicbox}
\subsection*{Question 3(b) [4 marks]}\label{q3b}

\textbf{Explain Nodal analysis.}

\begin{solutionbox}

\textbf{Diagram: Nodal Analysis Concept}

\begin{center}
\textbf{Mermaid Diagram (Code)}
\begin{verbatim}
{Shaded}
{Highlighting}[]
graph LR
    A[Step 1: Identify nodes] {-{-}{} B[Step 2: Select reference node]}
    B {-{-}{} C[Step 3: Assign node voltages]}
    C {-{-}{} D[Step 4: Apply KCL at each node]}
    D {-{-}{} E[Step 5: Solve equations]}
{Highlighting}
{Shaded}
\end{verbatim}
\end{center}


{\def\LTcaptype{none} % do not increment counter
\vspace{-5pt}
\captionof{table}{Nodal Analysis Method}
\vspace{-10pt}
\begin{longtable}[]{@{}ll@{}}
\toprule\noalign{}
Step & Description \\
\midrule\noalign{}
\endhead
\bottomrule\noalign{}
\endlastfoot
1. Select reference node & Usually ground (0V) \\
2. Assign voltages & Label remaining node voltages (V_{1}, V_{2}, etc.) \\
3. Apply KCL & Write KCL equation at each non-reference node \\
4. Express currents & Use Ohm's Law to express branch currents \\
5. Solve equations & Find node voltages using simultaneous equations \\
\end{longtable}
}

\textbf{Example: For nodes with voltages V_{1} and V_{2}:}

\begin{itemize}
\tightlist
\item
  KCL at node 1: (V_{1}-0)/R_{1} + (V_{1}-V_{2})/R_{2} + I_{1} = 0
\item
  KCL at node 2: (V_{2}-V_{1})/R_{2} + (V_{2}-0)/R_{3} + I_{2} = 0
\end{itemize}

\end{solutionbox}
\begin{mnemonicbox}
``Nodal needs KCL to analyze voltage''

\end{mnemonicbox}
\subsection*{Question 3(c) [7 marks]}\label{q3c}

\textbf{Use Thevenin's theorem to find current through the 5 Ω resistor
for given circuit.}

\begin{solutionbox}

\textbf{Diagram: Original Circuit and Thevenin Equivalent}

\begin{verbatim}
+{-{-}+     +{-}{-}+}
|  |     |  |
12V 20Ω  8V 10Ω
|  |     |  |
+{-{-}+{-}{-}+{-}{-}+{-}{-}+}
    |      |
    +{-{-}+{-}{-}{-}+}
       |
       5Ω
       |
      {-{-}{-}}
       {-}
\end{verbatim}

\textbf{Steps to Find Thevenin Equivalent:}


{\def\LTcaptype{none} % do not increment counter
\vspace{-5pt}
\captionof{table}{Thevenin's Theorem Process and Calculations}
\vspace{-10pt}
\begin{longtable}[]{@{}
  >{\raggedright\arraybackslash}p{(\linewidth - 6\tabcolsep) * \real{0.1667}}
  >{\raggedright\arraybackslash}p{(\linewidth - 6\tabcolsep) * \real{0.2500}}
  >{\raggedright\arraybackslash}p{(\linewidth - 6\tabcolsep) * \real{0.3611}}
  >{\raggedright\arraybackslash}p{(\linewidth - 6\tabcolsep) * \real{0.2222}}@{}}
\toprule\noalign{}
\begin{minipage}[b]{\linewidth}\raggedright
Step
\end{minipage} & \begin{minipage}[b]{\linewidth}\raggedright
Process
\end{minipage} & \begin{minipage}[b]{\linewidth}\raggedright
Calculation
\end{minipage} & \begin{minipage}[b]{\linewidth}\raggedright
Result
\end{minipage} \\
\midrule\noalign{}
\endhead
\bottomrule\noalign{}
\endlastfoot
1. Remove load (5Ω) & Calculate open-circuit voltage (Voc) & Voc =
Voltage divider formula & Vth = 9.33V \\
2. Replace voltage sources with shorts & Calculate equivalent resistance
(Req) & Req = 20Ω & \\
3. Draw Thevenin equivalent & Connect Vth and Rth in series with load &
& \\
4. Calculate load current &

I = Vth/(Rth+RL) &

I = 9.33/(6.67+5) &

I =

0.8A \\
\end{longtable}
}

\end{solutionbox}
\begin{mnemonicbox}
``Thevenin transforms: Find Voc and Req, then
calculate I''

\end{mnemonicbox}
\subsection*{Question 3(a) OR [3
marks]}\label{q3a}

\textbf{State and explain Maximum Power Transfer Theorem.}

\begin{solutionbox}

\textbf{Maximum Power Transfer Theorem:}

\begin{itemize}
\tightlist
\item
  Maximum power is transferred from source to load when \textbf{load
  resistance equals source internal resistance} (RL = Rth)
\item
  Only 50\% efficiency is achieved at maximum power transfer
\item
  Applies to DC and AC circuits (with complex impedances)
\end{itemize}

\textbf{Diagram: Maximum Power Transfer}

\begin{center}
\textbf{Mermaid Diagram (Code)}
\begin{verbatim}
{Shaded}
{Highlighting}[]
graph LR
    A[Source Circuit] {-{-}{-} B[Rth]}
    B {-{-}{-} C[RL]}
    A {-{-}{-} D[Vth]}
    D {-{-}{-} C}
    E[Power Transfer Curve] {-{-}{-} F[Peak at RL = Rth]}
{Highlighting}
{Shaded}
\end{verbatim}
\end{center}

\textbf{Formula: P = (Vth^{2}\timesRL)/(Rth+RL)^{2}}

\end{solutionbox}
\begin{mnemonicbox}
``Match the load to the source for maximum power
transfer''

\end{mnemonicbox}
\subsection*{Question 3(b) OR [4
marks]}\label{q3b}

\textbf{Explain method of drawing dual network using any circuit.}

\begin{solutionbox}

\textbf{Diagram: Original and Dual Network Example}

\begin{verbatim}
Original:       Dual:
R1              C1
o{-{-}{-}www{-}{-}{-}o     o{-}{-}{-}||{-}{-}{-}o}
|         |     |        |
C1        R2    L1       L2
|         |     |        |
o{-{-}{-}||{-}{-}{-}o      o{-}{-}{-}www{-}{-}o}
    L1               R1
\end{verbatim}


{\def\LTcaptype{none} % do not increment counter
\vspace{-5pt}
\captionof{table}{Dual Network Conversion Rules}
\vspace{-10pt}
\begin{longtable}[]{@{}lll@{}}
\toprule\noalign{}
Original Element & Dual Element & Example \\
\midrule\noalign{}
\endhead
\bottomrule\noalign{}
\endlastfoot
Series connection & Parallel connection & Series R \rightarrow Parallel C \\
Parallel connection & Series connection & Parallel C \rightarrow Series L \\
Voltage source & Current source & V source \rightarrow I source \\
Current source & Voltage source & I source \rightarrow V source \\
Resistor (R) & Conductance (1/R) & R \rightarrow G (1/R) \\
Inductor (L) & Capacitor (1/L) & L \rightarrow C (1/L) \\
Capacitor (C) & Inductor (1/C) & C \rightarrow L (1/C) \\
\end{longtable}
}

\textbf{Duality Process:}

\begin{enumerate}
\tightlist
\item
  Redraw network with meshes as nodes and nodes as meshes
\item
  Replace elements with their duals
\item
  Interchange series and parallel connections
\end{enumerate}

\end{solutionbox}
\begin{mnemonicbox}
``Duality swaps: Series\leftrightarrowParallel, V\leftrightarrowI, R\leftrightarrowG, L\leftrightarrowC''

\end{mnemonicbox}
\subsection*{Question 3(c) OR [7
marks]}\label{q3c}

\textbf{Find out Norton's equivalent circuit for the given network. Find
out load current if (i) R_{(}L_{)} = 3 KΩ (ii) R_{(}L_{)} = 1.5 Ω}

\begin{solutionbox}

\textbf{Diagram: Original Circuit and Norton Equivalent}

\begin{verbatim}
    +{-{-}+}
    |  |
    6V 9KΩ
    |  |
+{-{-}{-}+{-}{-}+{-}{-}{-}{-}{-}+}
|            |
3KΩ         6KΩ
|            |
+{-{-}{-}{-}{-}+{-}{-}{-}{-}{-}{-}+}
      |
      RL
      |
     {-{-}{-}}
      {-}
\end{verbatim}


{\def\LTcaptype{none} % do not increment counter
\vspace{-5pt}
\captionof{table}{Norton's Theorem Process and Calculations}
\vspace{-10pt}
\begin{longtable}[]{@{}
  >{\raggedright\arraybackslash}p{(\linewidth - 6\tabcolsep) * \real{0.1667}}
  >{\raggedright\arraybackslash}p{(\linewidth - 6\tabcolsep) * \real{0.2500}}
  >{\raggedright\arraybackslash}p{(\linewidth - 6\tabcolsep) * \real{0.3611}}
  >{\raggedright\arraybackslash}p{(\linewidth - 6\tabcolsep) * \real{0.2222}}@{}}
\toprule\noalign{}
\begin{minipage}[b]{\linewidth}\raggedright
Step
\end{minipage} & \begin{minipage}[b]{\linewidth}\raggedright
Process
\end{minipage} & \begin{minipage}[b]{\linewidth}\raggedright
Calculation
\end{minipage} & \begin{minipage}[b]{\linewidth}\raggedright
Result
\end{minipage} \\
\midrule\noalign{}
\endhead
\bottomrule\noalign{}
\endlastfoot
1. Calculate short-circuit current (Isc) & Short load terminals and find
current & Isc = Source current through short & In = 0.5mA \\
2. Calculate Norton resistance (Rn) & Replace sources with internal
resistance & Rn = 9KΩ & \\
3. Draw Norton equivalent & Connect In and Rn in parallel & & \\
4. Calculate load current (RL = 3KΩ) &

I = In \times Rn/(Rn + RL) &

I = 0.5mA

\times 3KΩ/(3KΩ + 3KΩ) & I = 0.25mA \\
5. Calculate load current (RL = 1.5Ω) &

I = In \times Rn/(Rn + RL) &

I =

0.5mA \times 3KΩ/(3KΩ + 1.5Ω) & I = 0.33mA \\
\end{longtable}
}

\end{solutionbox}
\begin{mnemonicbox}
``Norton needs Isc and Req to make a current source''

\end{mnemonicbox}
\subsection*{Question 4(a) [3 marks]}\label{q4a}

\textbf{Derive the equation of Quality factor Q for a coil.}

\begin{solutionbox}

\textbf{Diagram: Coil Equivalent Circuit}

\begin{verbatim}
     R       L
o{-{-}{-}www{-}{-}{-}OOOOOO{-}{-}{-}o}
\end{verbatim}

\textbf{Derivation of Q factor for a coil:}


{\def\LTcaptype{none} % do not increment counter
\vspace{-5pt}
\captionof{table}{Q Factor Derivation for Coil}
\vspace{-10pt}
\begin{longtable}[]{@{}lll@{}}
\toprule\noalign{}
Step & Expression & Explanation \\
\midrule\noalign{}
\endhead
\bottomrule\noalign{}
\endlastfoot
1. Impedance & Z = R + jωL & Complex impedance of coil \\
2. Reactive power & PX = (ωL)I^{2} & Power stored in inductor \\
3. Real power & PR = RI^{2} & Power dissipated in resistance \\
4. Quality factor & Q = PX/PR & Ratio of stored to dissipated power \\
5. Substitution & Q = (ωL)I^{2}/RI^{2} & Substitute expressions \\
6. Final equation & Q = ωL/R & Simplify to get Q factor \\
\end{longtable}
}

\end{solutionbox}
\begin{mnemonicbox}
``Quality coils: ωL/R shows energy saving ability''

\end{mnemonicbox}
\subsection*{Question 4(b) [4 marks]}\label{q4b}

\textbf{A series RLC circuit has R = 50 Ω, L = 0.2 H and C = 10 μF.
Calculate (i) Q factor, (ii) BW, (iii) Upper cut off and lower cut off
frequencies.}

\begin{solutionbox}

\textbf{Diagram: Series RLC Circuit}

\begin{verbatim}
R=50Ω

L=0.2H

o{-{-}{-}{-}www{-}{-}{-}{-}{-}{-}OOOOOO{-}{-}{-}{-}+}
                        |
                        |
                       {-{-}{-}}
                       {-{-}{-} C=10μF}
                        |
                        |
o{-{-}{-}{-}{-}{-}{-}{-}{-}{-}{-}{-}{-}{-}{-}{-}{-}{-}{-}{-}{-}{-}{-}+}
\end{verbatim}


{\def\LTcaptype{none} % do not increment counter
\vspace{-5pt}
\captionof{table}{Calculations for Series RLC Circuit}
\vspace{-10pt}
\begin{longtable}[]{@{}
  >{\raggedright\arraybackslash}p{(\linewidth - 6\tabcolsep) * \real{0.2683}}
  >{\raggedright\arraybackslash}p{(\linewidth - 6\tabcolsep) * \real{0.2195}}
  >{\raggedright\arraybackslash}p{(\linewidth - 6\tabcolsep) * \real{0.3171}}
  >{\raggedright\arraybackslash}p{(\linewidth - 6\tabcolsep) * \real{0.1951}}@{}}
\toprule\noalign{}
\begin{minipage}[b]{\linewidth}\raggedright
Parameter
\end{minipage} & \begin{minipage}[b]{\linewidth}\raggedright
Formula
\end{minipage} & \begin{minipage}[b]{\linewidth}\raggedright
Calculation
\end{minipage} & \begin{minipage}[b]{\linewidth}\raggedright
Result
\end{minipage} \\
\midrule\noalign{}
\endhead
\bottomrule\noalign{}
\endlastfoot
Resonant frequency (fr) & fr = 1/(2π\sqrtLC) & 1/(2π\sqrt(0.2\times10\times10^{-}^{6})) & 112.5
Hz \\
Quality factor (Q) & Q = (1/R)\sqrt(L/C) & (1/50)\sqrt(0.2/10\times10^{-}^{6}) & 28.28 \\
Bandwidth (BW) & BW = fr/Q & 112.5/28.28 & 3.98 Hz \\
Lower cutoff (f_{1}) & f_{1} = fr - BW/2 & 112.5 - 3.98/2 & 110.51 Hz \\
Upper cutoff (f_{2}) & f_{2} = fr + BW/2 & 112.5 + 3.98/2 & 114.49 Hz \\
\end{longtable}
}

\end{solutionbox}
\begin{mnemonicbox}
``Q defines BW, which sets cutoff frequencies''

\end{mnemonicbox}
\subsection*{Question 4(c) [7 marks]}\label{q4c}

\textbf{Explain Mutual Inductance along with Co-efficient of mutual
inductance. Also derive the equation of K.}

\begin{solutionbox}

\textbf{Diagram: Mutual Inductance Between Two Coils}

\begin{center}
\textbf{Mermaid Diagram (Code)}
\begin{verbatim}
{Shaded}
{Highlighting}[]
graph LR
    A[Input] {-{-}{-} B[L_{1}]}
    B {-{-}{-} C[Output 1]}
    D[Input] {-{-}{-} E[L_{2}]}
    E {-{-}{-} F[Output 2]}
    B {-.{-} E}
    linkStyle 4 stroke{-width:2px,stroke{-}dasharray: 5 5}
{Highlighting}
{Shaded}
\end{verbatim}
\end{center}

\textbf{Mutual Inductance (M):}

\begin{itemize}
\tightlist
\item
  When current in one coil induces voltage in nearby coil
\item
  Coupling between coils depends on position, orientation, and medium
\item
  Mutual inductance M in henries (H)
\end{itemize}


{\def\LTcaptype{none} % do not increment counter
\vspace{-5pt}
\captionof{table}{Mutual Inductance Equations}
\vspace{-10pt}
\begin{longtable}[]{@{}
  >{\raggedright\arraybackslash}p{(\linewidth - 4\tabcolsep) * \real{0.3333}}
  >{\raggedright\arraybackslash}p{(\linewidth - 4\tabcolsep) * \real{0.2727}}
  >{\raggedright\arraybackslash}p{(\linewidth - 4\tabcolsep) * \real{0.3939}}@{}}
\toprule\noalign{}
\begin{minipage}[b]{\linewidth}\raggedright
Parameter
\end{minipage} & \begin{minipage}[b]{\linewidth}\raggedright
Formula
\end{minipage} & \begin{minipage}[b]{\linewidth}\raggedright
Description
\end{minipage} \\
\midrule\noalign{}
\endhead
\bottomrule\noalign{}
\endlastfoot
Induced voltage & v_{2} = M(di_{1}/dt) & Voltage induced in coil 2 due to
current in coil 1 \\
Mutual inductance & M = k\sqrt(L_{1}L_{2}) & Mutual inductance related to
self-inductances \\
Coupling coefficient (k) & k = M/\sqrt(L_{1}L_{2}) & Measure of coupling between
coils (0 \leq k \leq 1) \\
Total inductance & Lt = L_{1} + L_{2} \pm 2M & Total inductance depends on
direction of coupling \\
\end{longtable}
}

\textbf{Derivation of Coupling Coefficient (k):}

\begin{itemize}
\tightlist
\item
  From M = k\sqrt(L_{1}L_{2})
\item
  Rearranging: k = M/\sqrt(L_{1}L_{2})
\item
  k = 1 for perfect coupling
\item
  k = 0 for no coupling
\item
  Typically 0.1 to 0.9 for real circuits
\end{itemize}

\end{solutionbox}
\begin{mnemonicbox}
``M measures magnetic linkage, k shows coupling
quality''

\end{mnemonicbox}
\subsection*{Question 4(a) OR [3
marks]}\label{q4a}

\textbf{Explain the types of coupling for coupled circuit.}

\begin{solutionbox}

\textbf{Diagram: Types of Coupling}

\begin{verbatim}
graph TB
    A[Types of Coupling] {-{-} B[Tight Coupling]}
    A {-{-} C[Loose Coupling]}
    A {-{-} D[Critical Coupling]}
    A {-{-} E[Direct Coupling]}
    A {-{-} F[Inductive Coupling]}
    A {-{-} G[Capacitive Coupling]}
\end{verbatim}


{\def\LTcaptype{none} % do not increment counter
\vspace{-5pt}
\captionof{table}{Types of Coupling}
\vspace{-10pt}
\begin{longtable}[]{@{}
  >{\raggedright\arraybackslash}p{(\linewidth - 4\tabcolsep) * \real{0.3261}}
  >{\raggedright\arraybackslash}p{(\linewidth - 4\tabcolsep) * \real{0.3696}}
  >{\raggedright\arraybackslash}p{(\linewidth - 4\tabcolsep) * \real{0.3043}}@{}}
\toprule\noalign{}
\begin{minipage}[b]{\linewidth}\raggedright
Coupling Type
\end{minipage} & \begin{minipage}[b]{\linewidth}\raggedright
Characteristics
\end{minipage} & \begin{minipage}[b]{\linewidth}\raggedright
Applications
\end{minipage} \\
\midrule\noalign{}
\endhead
\bottomrule\noalign{}
\endlastfoot
\textbf{Tight Coupling} & k \textgreater{} 0.5, high energy transfer &
Transformers \\
\textbf{Loose Coupling} & k \textless{} 0.5, selective frequency
response & RF tuning circuits \\
\textbf{Critical Coupling} & k adjusted for optimal bandwidth & RF
filters \\
\textbf{Direct Coupling} & Components directly connected & Audio
amplifiers \\
\textbf{Inductive Coupling} & Magnetic field transfers energy &
Transformers, wireless charging \\
\textbf{Capacitive Coupling} & Electric field transfers energy & Signal
coupling between stages \\
\end{longtable}
}

\end{solutionbox}
\begin{mnemonicbox}
``TLCLIC: Tight, Loose, Critical, Direct, Inductive,
Capacitive''

\end{mnemonicbox}
\subsection*{Question 4(b) OR [4
marks]}\label{q4b}

\textbf{A parallel resonant circuit having inductance of 10 mH with
quality factor

Q = 100, resonant frequency Fr = 50 KHz. Find out (i)

Required capacitance C, (ii) Resistance R of the coil, (iii) BW.}

\begin{solutionbox}

\textbf{Diagram: Parallel Resonant Circuit}

\begin{verbatim}
        L=10mH  
o{-{-}{-}{-}{-}{-}{-}OOOOOO{-}{-}{-}{-}{-}{-}{-}{-}+}
|                     |
|        R            |
|       www           |
|                     |
|                    {-{-}{-}}
|                    {-{-}{-} C=?}
|                     |
o{-{-}{-}{-}{-}{-}{-}{-}{-}{-}{-}{-}{-}{-}{-}{-}{-}{-}{-}{-}{-}+}
\end{verbatim}


{\def\LTcaptype{none} % do not increment counter
\vspace{-5pt}
\captionof{table}{Calculations for Parallel Resonant Circuit}
\vspace{-10pt}
\begin{longtable}[]{@{}
  >{\raggedright\arraybackslash}p{(\linewidth - 6\tabcolsep) * \real{0.2683}}
  >{\raggedright\arraybackslash}p{(\linewidth - 6\tabcolsep) * \real{0.2195}}
  >{\raggedright\arraybackslash}p{(\linewidth - 6\tabcolsep) * \real{0.3171}}
  >{\raggedright\arraybackslash}p{(\linewidth - 6\tabcolsep) * \real{0.1951}}@{}}
\toprule\noalign{}
\begin{minipage}[b]{\linewidth}\raggedright
Parameter
\end{minipage} & \begin{minipage}[b]{\linewidth}\raggedright
Formula
\end{minipage} & \begin{minipage}[b]{\linewidth}\raggedright
Calculation
\end{minipage} & \begin{minipage}[b]{\linewidth}\raggedright
Result
\end{minipage} \\
\midrule\noalign{}
\endhead
\bottomrule\noalign{}
\endlastfoot
Resonant frequency & fr = 1/(2π\sqrtLC) & 50 kHz = 1/(2π\sqrt(10\times10^{-}^{3}\timesC)) & \\
Capacitance (C) &

C = 1/(4π^{2}fr^{2}L) &

C = 1/(4π^{2}\times(50\times10^{3})^{2}\times10\times10^{-}^{3}) &

C =

1.01 nF \\
Resistance (R) &

Q = ωL/R & 100 = 2π\times50\times10^{3}\times10\times10^{-}^{3}/R &

R = 31.4 Ω \\

Bandwidth (BW) & BW = fr/Q & BW = 50\times10^{3}/100 & BW = 500 Hz \\
\end{longtable}
}

\end{solutionbox}
\begin{mnemonicbox}
``Parallel resonance parameters: C from fr, R from Q,
BW from fr/Q''

\end{mnemonicbox}
\subsection*{Question 4(c) OR [7
marks]}\label{q4c}

\textbf{Explain Band width and Selectivity of a series RLC circuit. Also
establish the relation between Q factor and BW for series resonance
circuit.}

\begin{solutionbox}

\textbf{Diagram: Frequency Response of Series RLC Circuit}

\begin{center}
\textbf{Mermaid Diagram (Code)}
\begin{verbatim}
{Shaded}
{Highlighting}[]
graph LR
    A[Frequency f] {-{-}{} B[Impedance Z]}
    B {-{-}{} C[Resonance at fr]}
    B {-{-}{} D[f1: Lower cutoff]}
    B {-{-}{} E[f2: Upper cutoff]}
    F[BW = f2 {- f1] {-}{-}{} G[Q = fr/BW]}
{Highlighting}
{Shaded}
\end{verbatim}
\end{center}

\textbf{Bandwidth (BW):}

\begin{itemize}
\tightlist
\item
  \textbf{Frequency range} between half-power points
\item
  At half-power points, impedance is \sqrt2 times minimum value
\item
  BW = f_{2} - f_{1}, where f_{1} and f_{2} are lower and upper cutoff frequencies
\end{itemize}

\textbf{Selectivity:}

\begin{itemize}
\tightlist
\item
  \textbf{Ability to reject} frequencies outside the bandwidth
\item
  Higher Q means higher selectivity and narrower bandwidth
\item
  Measured by steepness of response curve
\end{itemize}


{\def\LTcaptype{none} % do not increment counter
\vspace{-5pt}
\captionof{table}{Series RLC Bandwidth Parameters}
\vspace{-10pt}
\begin{longtable}[]{@{}
  >{\raggedright\arraybackslash}p{(\linewidth - 4\tabcolsep) * \real{0.3333}}
  >{\raggedright\arraybackslash}p{(\linewidth - 4\tabcolsep) * \real{0.2727}}
  >{\raggedright\arraybackslash}p{(\linewidth - 4\tabcolsep) * \real{0.3939}}@{}}
\toprule\noalign{}
\begin{minipage}[b]{\linewidth}\raggedright
Parameter
\end{minipage} & \begin{minipage}[b]{\linewidth}\raggedright
Formula
\end{minipage} & \begin{minipage}[b]{\linewidth}\raggedright
Description
\end{minipage} \\
\midrule\noalign{}
\endhead
\bottomrule\noalign{}
\endlastfoot
Bandwidth (BW) & BW = f_{2} - f_{1} & Difference between upper and lower
cutoff points \\
Half-power points & Z = \sqrt2 \times Z_{m}ᵢ_{n} & Points where power drops to half of
maximum \\
Resonant frequency & fr = 1/(2π\sqrtLC) & Center frequency \\
Quality factor & Q = ω_{o}L/R & Energy storage vs.~dissipation ratio \\
\end{longtable}
}

\textbf{Derivation of Q-BW Relationship:}

\begin{itemize}
\tightlist
\item
  At resonance, impedance Z = R
\item
  At cutoff frequencies, Z = \sqrt2R
\item
  This occurs when reactance XL - XC = \pmR
\item
  At f_{1}: ωL - 1/ωC = -R
\item
  At f_{2}: ωL - 1/ωC = +R
\item
  Solving these equations: BW = R/2πL = fr/Q
\item
  Therefore, Q = fr/BW
\end{itemize}

\end{solutionbox}
\begin{mnemonicbox}
``Quality inversely proportional to bandwidth''

\end{mnemonicbox}
\subsection*{Question 5(a) [3 marks]}\label{q5a}

\textbf{Design a symmetrical T type attenuator to give attenuation of 60
dB and work in to the load of 500 Ω resistance.}

\begin{solutionbox}

\textbf{Diagram: Symmetrical T-type Attenuator}

\begin{verbatim}
        R1/2          R1/2
   o{-{-}{-}{-}www{-}{-}{-}{-}{-}o{-}{-}{-}{-}www{-}{-}{-}{-}o}
   |            |           |
   |            |           |
   |           R2           |
   |            |           |
   |            |           |
  IN           {-{-}{-}         OUT}
                {-}
\end{verbatim}


{\def\LTcaptype{none} % do not increment counter
\vspace{-5pt}
\captionof{table}{Attenuator Design}
\vspace{-10pt}
\begin{longtable}[]{@{}llll@{}}
\toprule\noalign{}
Parameter & Formula & Calculation & Result \\
\midrule\noalign{}
\endhead
\bottomrule\noalign{}
\endlastfoot
Attenuation (N) &

N = 10\^{}(dB/20) & 10\^{}(60/20) &

N = 1000 \\

Z_{0} & Given & 500 Ω & 500 Ω \\
R_{1} & R_{1} = 2Z_{0}(N-1)/(N+1) & 2\times500\times(1000-1)/(1000+1) & R_{1} = 998 Ω \\
R_{2} & R_{2} = Z_{0}(N+1)/(N-1) & 500\times(1000+1)/(1000-1) & R_{2} = 0.5 Ω \\
\end{longtable}
}

\end{solutionbox}
\begin{mnemonicbox}
``T attenuator: R_{1} series divides, R_{2} shunts''

\end{mnemonicbox}
\subsection*{Question 5(b) [4 marks]}\label{q5b}

\textbf{Compare Band pass and Band stop filters.}

\begin{solutionbox}

\textbf{Diagram: Band Pass vs Band Stop Response}

\begin{center}
\textbf{Mermaid Diagram (Code)}
\begin{verbatim}
{Shaded}
{Highlighting}[]
graph LR
    A[Frequency] {-{-}{} B[Gain]}
    B {-{-}{}|Band Pass| C[Pass in middle band]}
    B {-{-}{}|Band Stop| D[Reject in middle band]}
{Highlighting}
{Shaded}
\end{verbatim}
\end{center}


{\def\LTcaptype{none} % do not increment counter
\vspace{-5pt}
\captionof{table}{Comparison of Band Pass and Band Stop Filters}
\vspace{-10pt}
\begin{longtable}[]{@{}
  >{\raggedright\arraybackslash}p{(\linewidth - 4\tabcolsep) * \real{0.2245}}
  >{\raggedright\arraybackslash}p{(\linewidth - 4\tabcolsep) * \real{0.3878}}
  >{\raggedright\arraybackslash}p{(\linewidth - 4\tabcolsep) * \real{0.3878}}@{}}
\toprule\noalign{}
\begin{minipage}[b]{\linewidth}\raggedright
Parameter
\end{minipage} & \begin{minipage}[b]{\linewidth}\raggedright
Band Pass Filter
\end{minipage} & \begin{minipage}[b]{\linewidth}\raggedright
Band Stop Filter
\end{minipage} \\
\midrule\noalign{}
\endhead
\bottomrule\noalign{}
\endlastfoot
\textbf{Frequency Response} & Passes frequencies within specific band &
Rejects frequencies within specific band \\
\textbf{Circuit Structure} & Series \& parallel resonant circuits &
Series \& parallel resonant circuits \\
\textbf{Cut-off Frequencies} & Has lower (f_{1}) and upper (f_{2}) cut-offs &
Has lower (f_{1}) and upper (f_{2}) cut-offs \\
\textbf{Bandwidth} & BW = f_{2} - f_{1} & BW = f_{2} - f_{1} \\
\textbf{Applications} & Radio tuning, audio equalization & Noise
elimination, harmonic suppression \\
\textbf{Implementation} & Series/parallel combination of HPF \& LPF &
Parallel/series combination of HPF \& LPF \\
\textbf{Phase Response} & 0^\circ at resonance & 180^\circ at resonance \\
\end{longtable}
}

\end{solutionbox}
\begin{mnemonicbox}
``Pass the middle or Stop the middle''

\end{mnemonicbox}
\subsection*{Question 5(c) [7 marks]}\label{q5c}

\textbf{Explain constant K Low Pass Filter.}

\begin{solutionbox}

\textbf{Diagram: Constant K Low Pass Filter T and π Sections}

\begin{verbatim}
T{-section:                    π{-}section:}
    L/2         L/2               L
o{-{-}{-}OOOO{-}{-}{-}{-}o{-}{-}{-}OOOO{-}{-}{-}{-}{-}{-}o  o{-}{-}{-}OOOO{-}{-}{-}o}
            |                  |      |
            C                  C/2    C/2
            |                  |      |
            |                  |      |
o{-{-}{-}{-}{-}{-}{-}{-}{-}{-}{-}o{-}{-}{-}{-}{-}{-}{-}{-}{-}{-}{-}{-}{-}o  o{-}{-}{-}{-}{-}{-}{-}{-}{-}{-}o}
\end{verbatim}

\textbf{Constant K Low Pass Filter:}

\begin{itemize}
\tightlist
\item
  \textbf{Passes frequencies} below cutoff frequency (fc)
\item
  Attenuates frequencies above fc
\item
  ``Constant K'' means product of series and shunt impedances is
  constant at all frequencies (Z_{1}Z_{2} = K^{2})
\end{itemize}


{\def\LTcaptype{none} % do not increment counter
\vspace{-5pt}
\captionof{table}{T and π Section Parameters}
\vspace{-10pt}
\begin{longtable}[]{@{}lll@{}}
\toprule\noalign{}
Parameter & T-section & π-section \\
\midrule\noalign{}
\endhead
\bottomrule\noalign{}
\endlastfoot
Series arm & L/2 at each end & L in center \\
Shunt arm & C in center & C/2 at each end \\
Cutoff frequency & fc = 1/(π\sqrtLC) & fc = 1/(π\sqrtLC) \\
Characteristic impedance & Z_{0} = \sqrt(L/C) & Z_{0} = \sqrt(L/C) \\
Design equation for L &

L = Z_{0}/πfc &

L = Z_{0}/πfc \\

Design equation for C &

C = 1/(πfcZ_{0}) &

C = 1/(πfcZ_{0}) \\

\end{longtable}
}

\textbf{Frequency Response:}

\begin{itemize}
\tightlist
\item
  Passes DC and low frequencies with minimal attenuation
\item
  Attenuation increases rapidly above cutoff frequency
\item
  Phase shift increases with frequency
\end{itemize}

\end{solutionbox}
\begin{mnemonicbox}
``Constant K LPF: L series blocks high, C shunt
shorts high''

\end{mnemonicbox}
\subsection*{Question 5(a) OR [3
marks]}\label{q5a}

\textbf{Design a high pass filter with T section having a cut-off
frequency of 2 KHz with a load resistance of 500 Ω.}

\begin{solutionbox}

\textbf{Diagram: High Pass T-section Filter}

\begin{verbatim}
      C/2          C/2
   o{-{-}{-}||{-}{-}{-}{-}o{-}{-}{-}||{-}{-}{-}o}
   |         |        |
   |         L        |
   |        OOO       |
   |         |        |
  IN        {-{-}{-}      OUT}
             {-}
\end{verbatim}


{\def\LTcaptype{none} % do not increment counter
\vspace{-5pt}
\captionof{table}{High Pass Filter Design}
\vspace{-10pt}
\begin{longtable}[]{@{}
  >{\raggedright\arraybackslash}p{(\linewidth - 6\tabcolsep) * \real{0.2683}}
  >{\raggedright\arraybackslash}p{(\linewidth - 6\tabcolsep) * \real{0.2195}}
  >{\raggedright\arraybackslash}p{(\linewidth - 6\tabcolsep) * \real{0.3171}}
  >{\raggedright\arraybackslash}p{(\linewidth - 6\tabcolsep) * \real{0.1951}}@{}}
\toprule\noalign{}
\begin{minipage}[b]{\linewidth}\raggedright
Parameter
\end{minipage} & \begin{minipage}[b]{\linewidth}\raggedright
Formula
\end{minipage} & \begin{minipage}[b]{\linewidth}\raggedright
Calculation
\end{minipage} & \begin{minipage}[b]{\linewidth}\raggedright
Result
\end{minipage} \\
\midrule\noalign{}
\endhead
\bottomrule\noalign{}
\endlastfoot
Cutoff frequency (fc) & Given & 2 kHz & 2 kHz \\
Load resistance (R_{0}) & Given & 500 Ω & 500 Ω \\
Series capacitance (C/2) &

C = 1/(πfcR_{0}) &

C = 1/(π\times2\times10^{3}\times500) &

C =

0.318 μF \\
Total capacitance (C) & 2 \times (C/2) & 2 \times 0.159 μF & C = 0.318 μF \\
Shunt inductance (L) &

L = R_{0}/(πfc) &

L = 500/(π\times2\times10^{3}) &

L = 79.6 mH \\

\end{longtable}
}

\end{solutionbox}
\begin{mnemonicbox}
``High pass T: C blocks DC in series, L passes high
in shunt''

\end{mnemonicbox}
\subsection*{Question 5(b) OR [4
marks]}\label{q5b}

\textbf{Give classification of filters.}

\begin{solutionbox}

\textbf{Diagram: Filter Classification}

\begin{center}
\textbf{Mermaid Diagram (Code)}
\begin{verbatim}
{Shaded}
{Highlighting}[]
graph TD
    A[Filters] {-{-}{} B[By Function]}
    A {-{-}{} C[By Design]}
    A {-{-}{} D[By Implementation]}
    B {-{-}{} B1[Low Pass]}
    B {-{-}{} B2[High Pass]}
    B {-{-}{} B3[Band Pass]}
    B {-{-}{} B4[Band Stop]}
    B {-{-}{} B5[All Pass]}
    C {-{-}{} C1[Passive]}
    C {-{-}{} C2[Active]}
    C1 {-{-}{} C11[Constant{-}k]}
    C1 {-{-}{} C12[m{-}derived]}
    C1 {-{-}{} C13[Composite]}
    D {-{-}{} D1[Analog]}
    D {-{-}{} D2[Digital]}
{Highlighting}
{Shaded}
\end{verbatim}
\end{center}


{\def\LTcaptype{none} % do not increment counter
\vspace{-5pt}
\captionof{table}{Classification of Filters}
\vspace{-10pt}
\begin{longtable}[]{@{}
  >{\raggedright\arraybackslash}p{(\linewidth - 4\tabcolsep) * \real{0.4286}}
  >{\raggedright\arraybackslash}p{(\linewidth - 4\tabcolsep) * \real{0.1667}}
  >{\raggedright\arraybackslash}p{(\linewidth - 4\tabcolsep) * \real{0.4048}}@{}}
\toprule\noalign{}
\begin{minipage}[b]{\linewidth}\raggedright
Classification By
\end{minipage} & \begin{minipage}[b]{\linewidth}\raggedright
Types
\end{minipage} & \begin{minipage}[b]{\linewidth}\raggedright
Characteristics
\end{minipage} \\
\midrule\noalign{}
\endhead
\bottomrule\noalign{}
\endlastfoot
\textbf{Function} & Low Pass & Passes frequencies below cutoff \\
& High Pass & Passes frequencies above cutoff \\
& Band Pass & Passes frequencies within a band \\
& Band Stop & Rejects frequencies within a band \\
& All Pass & Passes all frequencies but modifies phase \\
\textbf{Design} & Passive & Uses passive elements (R, L, C) \\
& Active & Uses active components (op-amps) \\
\textbf{Response} & Butterworth & Maximally flat response \\
& Chebyshev & Ripple in passband, steeper rolloff \\
& Bessel & Linear phase response \\
& Elliptic & Ripple in both passband and stopband \\
\textbf{Implementation} & Passive Filter Types & Constant-k, m-derived,
composite \\
\end{longtable}
}

\end{solutionbox}
\begin{mnemonicbox}
``FLHBA: Function (Low/High/Band/All-pass), Design,
Response, Implementation''

\end{mnemonicbox}
\subsection*{Question 5(c) OR [7
marks]}\label{q5c}

\textbf{Explain constant K High Pass Filter.}

\begin{solutionbox}

\textbf{Diagram: Constant K High Pass Filter T and π Sections}

\begin{verbatim}
T{-section:                     π{-}section:}
     C/2          C/2               C
o{-{-}{-}{-}||{-}{-}{-}{-}{-}{-}o{-}{-}{-}{-}||{-}{-}{-}{-}{-}{-}{-}o   o{-}{-}{-}{-}||{-}{-}{-}{-}o}
             |                   |      |
             L                   L/2    L/2
             |                   |      |
             |                   |      |
o{-{-}{-}{-}{-}{-}{-}{-}{-}{-}{-}{-}o{-}{-}{-}{-}{-}{-}{-}{-}{-}{-}{-}{-}{-}{-}o  o{-}{-}{-}{-}{-}{-}{-}{-}{-}{-}o}
\end{verbatim}

\textbf{Constant K High Pass Filter:}

\begin{itemize}
\tightlist
\item
  \textbf{Passes frequencies} above cutoff frequency (fc)
\item
  Attenuates frequencies below fc
\item
  ``Constant K'' means product of series and shunt impedances is
  constant at all frequencies (Z_{1}Z_{2} = K^{2})
\end{itemize}


{\def\LTcaptype{none} % do not increment counter
\vspace{-5pt}
\captionof{table}{T and π Section Parameters}
\vspace{-10pt}
\begin{longtable}[]{@{}lll@{}}
\toprule\noalign{}
Parameter & T-section & π-section \\
\midrule\noalign{}
\endhead
\bottomrule\noalign{}
\endlastfoot
Series arm & C/2 at each end & C in center \\
Shunt arm & L in center & L/2 at each end \\
Cutoff frequency & fc = 1/(π\sqrtLC) & fc = 1/(π\sqrtLC) \\
Characteristic impedance & Z_{0} = \sqrt(L/C) & Z_{0} = \sqrt(L/C) \\
Design equation for L &

L = Z_{0}/(πfc) &

L = Z_{0}/(πfc) \\

Design equation for C &

C = 1/(πfcZ_{0}) &

C = 1/(πfcZ_{0}) \\

\end{longtable}
}

\textbf{Frequency Response:}

\begin{itemize}
\tightlist
\item
  Blocks DC and low frequencies
\item
  Passes high frequencies with minimal attenuation
\item
  Attenuation increases as frequency decreases below cutoff
\item
  Phase shift approaches 0^\circ at very high frequencies
\end{itemize}

\end{solutionbox}
\begin{mnemonicbox}
``Constant K HPF: C series blocks low, L shunt passes
high''

\end{mnemonicbox}

\end{document}
