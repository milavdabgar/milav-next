\documentclass[10pt,a4paper]{article}

% content/resources/templates/preamble.tex
\usepackage[margin=0.6in]{geometry}
\author{Milav Dabgar}
\usepackage{amsmath,amssymb,amsthm}
\usepackage{booktabs}
\usepackage{multirow}
\usepackage{xcolor}
\usepackage{tcolorbox}
\tcbuselibrary{breakable,skins}
\usepackage[colorlinks=true,linkcolor=blue]{hyperref}
\usepackage{titlesec}
\usepackage{enumitem}
\usepackage{tikz}
\usepackage{pgfplots}
\usepackage{circuitikz}
\usepackage[version=4]{mhchem}
\usepackage{longtable}
\usepackage{array}
\usepackage{float}
\usepackage{caption}
\usepackage{listings}

\lstset{
  basicstyle=\small\ttfamily,
  breaklines=true,
  breakatwhitespace=false,
  postbreak=\mbox{\textcolor{red}{$\hookrightarrow$}\space},
  float=false,
  numbers=left,
  numberstyle=\tiny\color{gray},
  numbersep=10pt,
  xleftmargin=2em,
  keywordstyle=\color{blue},
  commentstyle=\color{green!60!black},
  stringstyle=\color{purple},
  backgroundcolor=\color{gray!5},
  showstringspaces=false,
  tabsize=2,
  captionpos=b,
  keepspaces=true,
  columns=flexible
}

\pgfplotsset{compat=1.18}
\usetikzlibrary{shapes,arrows,positioning,calc,patterns,decorations.pathmorphing,decorations.markings,arrows.meta}

% Color scheme
\definecolor{headcolor}{RGB}{0,102,204}
\definecolor{keycolor}{RGB}{220,20,60}
\definecolor{solutioncolor}{RGB}{34,139,34}
\definecolor{mnemoniccolor}{RGB}{148,0,211}
\definecolor{codecolor}{RGB}{0,0,100}

% Spacing
\setlength{\parskip}{3pt}
\setlist[itemize]{nosep}
\setlist[enumerate]{nosep}

% Title formatting
\titleformat{\section}{\Large\bfseries\color{headcolor}}{\thesection}{1em}{}
\titleformat{\subsection}{\large\bfseries\color{headcolor}}{\thesubsection}{1em}{}

% Pandoc tightlist compatibility
\providecommand{\tightlist}{%
  \setlength{\itemsep}{0pt}\setlength{\parskip}{0pt}}

% Pandoc longtable compatibility
\newcounter{none}
\def\thenone{}


% content/resources/templates/gujarati-boxes.tex
\usepackage{fontspec}
\usepackage{polyglossia}

% Set Gujarati as main language (document is primarily in Gujarati)
% Note: gloss-gujarati.ldf doesn't exist in polyglossia, but it will use hyphenation patterns
\setdefaultlanguage{gujarati}
\setotherlanguage{english}

% Configure Gujarati font properly
% Use Language=Default to prevent polyglossia from trying to add language-specific features
% that don't exist for Gujarati, which causes "empty feature" warnings
\newfontfamily\gujaratifont[Script=Gujarati,AutoFakeBold=2.5,AutoFakeSlant=0.3]{Noto Sans Gujarati}
\setmainfont[Script=Gujarati,AutoFakeBold=2.5,AutoFakeSlant=0.3]{Noto Sans Gujarati}
% Use Noto Sans Gujarati for monospace to support Gujarati in text
\setmonofont[Scale=0.9]{Noto Sans Gujarati}

% Configure English to use the same font
\newfontfamily\englishfont[Script=Gujarati,AutoFakeBold=2.5,AutoFakeSlant=0.3]{Noto Sans Gujarati}

% Translations for polyglossia
\gappto\captionsgujarati{
  \renewcommand{\tablename}{કોષ્ટક}
  \renewcommand{\figurename}{આકૃતિ}
}

% Helper for TikZ nodes to ensure Gujarati font
\newcommand{\gu}[1]{{\gujaratifont #1}}

% Custom environments
\newtcolorbox{solutionbox}{
    breakable,
    enhanced,
    colback=solutioncolor!5!white,
    colframe=solutioncolor!75!black,
    fonttitle=\bfseries,
    title=જવાબ
}

\newtcolorbox{solutionboxnobreak}{
 colback=solutioncolor!5!white,
 colframe=solutioncolor!75!black,
 fonttitle=\bfseries,
 title=જવાબ
}

\newtcolorbox{keyformula}{
 breakable,
 enhanced,
 colback=keycolor!5!white,
 colframe=keycolor!75!black,
 fonttitle=\bfseries,
 title=રાસાયણિક સમીકરણ/સૂત્ર
}

\newtcolorbox{mnemonicbox}{
 breakable,
 enhanced,
 colback=mnemoniccolor!5!white,
 colframe=mnemoniccolor!75!black,
 fonttitle=\bfseries,
 title=મેમરી ટ્રીક
}


\begin{document}

\begin{center}
{\Huge\bfseries\color{headcolor} Subject Name (Gujarati)}\\[5pt]
{\LARGE 4331101 -- Winter 2023}\\[3pt]
{\large Semester 1 Study Material}\\[3pt]
{\normalsize\textit{Detailed Solutions and Explanations}}
\end{center}

\vspace{10pt}

\subsection*{પ્રશ્ન 1(a) [3
ગુણ]}\label{q1a}

\textbf{યોગ્ય રેખાકૃતિ સાથે સ્ત્રોત પરિવર્તન સમજાવો.}

\begin{solutionbox}
સ્ત્રોત પરિવર્તન એ વોલ્ટેજ સ્ત્રોતને કરંટ સ્ત્રોતમાં અથવા તેનાથી
વિપરીત રૂપાંતરિત કરવાની પદ્ધતિ છે જેમાં બાહ્ય સર્કિટનું વર્તન બદલાતું નથી.

\textbf{આકૃતિ:}

\begin{center}
\textbf{Mermaid Diagram (Code)}
\begin{verbatim}
{Shaded}
{Highlighting}[]
graph LR
    subgraph "Voltage Source Circuit"
    VS[V] {-{-}{-} RS[R]}
    end
    subgraph "Current Source Circuit"
    IS[I] {-.{-} RP[R]}
    end

    VS {-{-}{-} IS}
    
    class VS,IS fill:\#f96
{Highlighting}
{Shaded}
\end{verbatim}
\end{center}

\begin{itemize}
\tightlist
\item
  \textbf{વોલ્ટેજથી કરંટ સ્ત્રોત}: I = V/R, સમાન R સમાંતરમાં
\item
  \textbf{કરંટથી વોલ્ટેજ સ્ત્રોત}: V = I\timesR, સમાન R શ્રેણીમાં
\end{itemize}

\end{solutionbox}
\begin{mnemonicbox}
``મૂલ્ય રહે છે, રેસિસ્ટન્સ બદલાય છે'' (V=IR હંમેશા લાગુ પડે છે)

\end{mnemonicbox}
\subsection*{પ્રશ્ન 1(b) [4
ગુણ]}\label{q1b}

\textbf{શ્રેણીમાં જોડાયેલા બે કેપેસિટર માટે વોલ્ટેજ, કરંટ અને પાવર સંબંધ મેળવો.}

\begin{solutionbox}


{\def\LTcaptype{none} % do not increment counter
\vspace{-5pt}
\captionof{table}{શ્રેણીમાં કેપેસિટર્સ}
\vspace{-10pt}
\begin{longtable}[]{@{}lll@{}}
\toprule\noalign{}
પરિમાણ & સૂત્ર & સમજૂતી \\
\midrule\noalign{}
\endhead
\bottomrule\noalign{}
\endlastfoot
કુલ કેપેસિટન્સ & 1/CT = 1/C_{1} + 1/C_{2} & પ્રતિરોધી યોગ \\
વોલ્ટેજ વિતરણ & V_{1}/V_{2} = C_{2}/C_{1} & કેપેસિટન્સ રેશિયોના વ્યસ્ત \\
કરંટ &

I = I_{1} = I_{2} & બધા દ્વારા સમાન કરંટ વહે છે \\

ચાર્જ &

Q = Q_{1} = Q_{2} & દરેક કેપેસિટર પર સમાન ચાર્જ \\

પાવર &

P = VI = V^{2}/Xc & જ્યાં Xc = 1/2πfC \\

\end{longtable}
}

\begin{itemize}
\tightlist
\item
  \textbf{વોલ્ટેજ વિભાજન}: V_{1} = V \times C_{2}/(C_{1}+C_{2})
\item
  \textbf{ચાર્જ સંગ્રહ}: Q = C_{1}C_{2}V/(C_{1}+C_{2})
\end{itemize}

\end{solutionbox}
\begin{mnemonicbox}
``શ્રેણીમાં કેપેસિટર્સ: કરંટ સમાન, કેપેસિટન્સ ઘટે''

\end{mnemonicbox}
\subsection*{પ્રશ્ન 1(c) [7
ગુણ]}\label{q1c}

\textbf{રેસિસ્ટરના શ્રેણી અને સમાંતર જોડાણ વચ્ચેનો તફાવત આપો અને સમાંતર જોડાણના કુલ
રેસિસ્ટન્સનું સમીકરણ મેળવો.}

\begin{solutionbox}


{\def\LTcaptype{none} % do not increment counter
\vspace{-5pt}
\captionof{table}{શ્રેણી વિરુદ્ધ સમાંતર રેસિસ્ટર્સ}
\vspace{-10pt}
\begin{longtable}[]{@{}
  >{\raggedright\arraybackslash}p{(\linewidth - 4\tabcolsep) * \real{0.2200}}
  >{\raggedright\arraybackslash}p{(\linewidth - 4\tabcolsep) * \real{0.3600}}
  >{\raggedright\arraybackslash}p{(\linewidth - 4\tabcolsep) * \real{0.4200}}@{}}
\toprule\noalign{}
\begin{minipage}[b]{\linewidth}\raggedright
પરિમાણ
\end{minipage} & \begin{minipage}[b]{\linewidth}\raggedright
શ્રેણી જોડાણ
\end{minipage} & \begin{minipage}[b]{\linewidth}\raggedright
સમાંતર જોડાણ
\end{minipage} \\
\midrule\noalign{}
\endhead
\bottomrule\noalign{}
\endlastfoot
કુલ રેસિસ્ટન્સ & વધે છે (RT = R_{1} + R_{2} + \ldots) & ઘટે છે (RT \textless{} સૌથી
નાના R) \\
કરંટ & બધામાં સમાન (I) & વિભાજન થાય (IT = I_{1} + I_{2} + \ldots) \\
વોલ્ટેજ & વિભાજન થાય (VT = V_{1} + V_{2} + \ldots) & બધા પર સમાન (V) \\
પાવર & PT = P_{1} + P_{2} + \ldots{} & PT = P_{1} + P_{2} + \ldots{} \\
\end{longtable}
}

\textbf{સમાંતર રેસિસ્ટન્સ માટેનું વ્યુત્પત્તિ:}

કિરચોફના કરંટ નિયમ અનુસાર: IT = I_{1} + I_{2} + \ldots{} + In

I = V/R બદલતાં: V/RT = V/R_{1} + V/R_{2} + \ldots{} + V/Rn

V થી ભાગીને: 1/RT = 1/R_{1} + 1/R_{2} + \ldots{} + 1/Rn

બે રેસિસ્ટર્સ માટે: 1/RT = 1/R_{1} + 1/R_{2}, જે આપે છે RT = R_{1}R_{2}/(R_{1}+R_{2})

\end{solutionbox}
\begin{mnemonicbox}
``સમાંતરમાં, વ્યસ્ત મૂલ્યો ઉમેરાય છે''

\end{mnemonicbox}
\subsection*{પ્રશ્ન 1(c) OR [7
ગુણ]}\label{q1c}

\textbf{1) યુનિલેટરલ, બાયલેટરલ નેટવર્ક, મેશ અને લૂપ વ્યાખ્યાયિત કરો.} \textbf{2)
વોલ્ટેજ ડિવિઝન સર્કિટ દોરો અને સમીકરણ લખો.}

\begin{solutionbox}


{\def\LTcaptype{none} % do not increment counter
\vspace{-5pt}
\captionof{table}{નેટવર્ક વ્યાખ્યાઓ}
\vspace{-10pt}
\begin{longtable}[]{@{}lll@{}}
\toprule\noalign{}
પદ & વ્યાખ્યા & ઉદાહરણ \\
\midrule\noalign{}
\endhead
\bottomrule\noalign{}
\endlastfoot
યુનિલેટરલ નેટવર્ક & માત્ર એક દિશામાં કરંટ પસાર થવા દે છે & ડાયોડ સર્કિટ \\
બાયલેટરલ નેટવર્ક & બંને દિશામાં કરંટ પસાર થવા દે છે & RLC સર્કિટ \\
મેશ & સપાટ નેટવર્ક પાથ જેમાં કોઈ બીજો પાથ નથી & એક બંધ પાથ \\
લૂપ & નેટવર્કમાં કોઈપણ બંધ પાથ & અન્ય તત્વો શામેલ કરી શકે \\
\end{longtable}
}

\textbf{વોલ્ટેજ ડિવિઝન સર્કિટ:}

\begin{center}
\textbf{Mermaid Diagram (Code)}
\begin{verbatim}
{Shaded}
{Highlighting}[]
graph LR
    A[Input] {-{-}{-} R1[R_{1}] {-}{-}{-} B[Output V_{0}] {-}{-}{-} R2[R_{2}] {-}{-}{-} C[Ground]}
{Highlighting}
{Shaded}
\end{verbatim}
\end{center}

\textbf{વોલ્ટેજ ડિવિઝન સમીકરણ:} Vo = Vin \times R_{2}/(R_{1}+R_{2})

\begin{itemize}
\tightlist
\item
  \textbf{સમાનુપાતિક}: રેસિસ્ટન્સ જેના પર વોલ્ટેજ માપવામાં આવે છે
\item
  \textbf{વ્યસ્ત સમાનુપાતિક}: કુલ રેસિસ્ટન્સ
\end{itemize}

\end{solutionbox}
\begin{mnemonicbox}
``આઉટપુટ વોલ્ટેજ ઈનપુટ ગુણ્યા રેસિસ્ટન્સના ગુણોત્તર''

\end{mnemonicbox}
\subsection*{પ્રશ્ન 2(a) [3
ગુણ]}\label{q2a}

\textbf{T-type નેટવર્કને π-type નેટવર્કમાં કન્વર્ટ કરવા માટે સમીકરણો મેળવો.}

\begin{solutionbox}

\textbf{આકૃતિ: T થી π રૂપાંતરણ}

\begin{verbatim}
    A    Z_{1    B         A            B}
     o{-{-}{-}//{-}{-}o          o         o}
     |          |          {       /}
     |          |    ={          /}
     Z_{3         Z_{2}              /}
     |          |            Z_{1_{2} Z_{2}_{3}}
     o{-{-}{-}{-}{-}{-}{-}{-}{-}{-}o              /}
     C                         o
                               C
\end{verbatim}

\textbf{રૂપાંતરણ સમીકરણો:}

\begin{itemize}
\tightlist
\item
  Z_{1}_{2} = (Z_{1}Z_{2} + Z_{2}Z_{3} + Z_{3}Z_{1})/Z_{3}
\item
  Z_{2}_{3} = (Z_{1}Z_{2} + Z_{2}Z_{3} + Z_{3}Z_{1})/Z_{1}
\item
  Z_{3}_{1} = (Z_{1}Z_{2} + Z_{2}Z_{3} + Z_{3}Z_{1})/Z_{2}
\end{itemize}

જ્યાં Z_{1}, Z_{2}, Z_{3} એ T-નેટવર્કના ઇમ્પીડન્સ છે અને Z_{1}_{2}, Z_{2}_{3}, Z_{3}_{1} એ π-નેટવર્કના
ઇમ્પીડન્સ છે.

\end{solutionbox}
\begin{mnemonicbox}
``બધા ગુણનનો સરવાળો વિભાજિત સામેના દ્વારા''

\end{mnemonicbox}
\subsection*{પ્રશ્ન 2(b) [4
ગુણ]}\label{q2b}

\textbf{ઓપન સર્કિટ ઇમ્પીડન્સ પેરામીટર (Z પેરામીટર) સમજાવો.}

\begin{solutionbox}

\textbf{Z-પેરામીટર્સ}: આને ઓપન-સર્કિટ ઇમ્પીડન્સ પેરામીટર્સ પણ કહેવામાં આવે છે કારણ કે
તેઓ આઉટપુટ પોર્ટ્સને ખુલ્લા રાખીને માપવામાં આવે છે.


{\def\LTcaptype{none} % do not increment counter
\vspace{-5pt}
\captionof{table}{Z-પેરામીટર સમીકરણો}
\vspace{-10pt}
\begin{longtable}[]{@{}
  >{\raggedright\arraybackslash}p{(\linewidth - 4\tabcolsep) * \real{0.3056}}
  >{\raggedright\arraybackslash}p{(\linewidth - 4\tabcolsep) * \real{0.3333}}
  >{\raggedright\arraybackslash}p{(\linewidth - 4\tabcolsep) * \real{0.3611}}@{}}
\toprule\noalign{}
\begin{minipage}[b]{\linewidth}\raggedright
પેરામીટર
\end{minipage} & \begin{minipage}[b]{\linewidth}\raggedright
વ્યાખ્યા
\end{minipage} & \begin{minipage}[b]{\linewidth}\raggedright
ગણતરી
\end{minipage} \\
\midrule\noalign{}
\endhead
\bottomrule\noalign{}
\endlastfoot
Z_{1}_{1} & આઉટપુટ ખુલ્લું હોય ત્યારે ઇનપુટ ઇમ્પીડન્સ & Z_{1}_{1} = V_{1}/I_{1} (જ્યારે I_{2}=0) \\
Z_{1}_{2} & પોર્ટ 2 થી પોર્ટ 1 સુધીનો ટ્રાન્સફર ઇમ્પીડન્સ & Z_{1}_{2} = V_{1}/I_{2} (જ્યારે
I_{1}=0) \\
Z_{2}_{1} & પોર્ટ 1 થી પોર્ટ 2 સુધીનો ટ્રાન્સફર ઇમ્પીડન્સ & Z_{2}_{1} = V_{2}/I_{1} (જ્યારે
I_{2}=0) \\
Z_{2}_{2} & ઇનપુટ ખુલ્લું હોય ત્યારે આઉટપુટ ઇમ્પીડન્સ & Z_{2}_{2} = V_{2}/I_{2} (જ્યારે I_{1}=0) \\
\end{longtable}
}

\textbf{મેટ્રિક્સ ફોર્મ:} [V_{1}] = [Z_{1}_{1} Z_{1}_{2}] \times [I_{1}] [V_{2}]
[Z_{2}_{1} Z_{2}_{2}] [I_{2}]

\begin{itemize}
\tightlist
\item
  \textbf{સિમેટ્રિકલ નેટવર્ક}: Z_{1}_{2} = Z_{2}_{1}
\item
  \textbf{એકમો}: ઓહ્મ (Ω)
\end{itemize}

\end{solutionbox}
\begin{mnemonicbox}
``Vs તે Zs ગુણ્યા Is''

\end{mnemonicbox}
\subsection*{પ્રશ્ન 2(c) [7
ગુણ]}\label{q2c}

\textbf{સિમેટ્રિકલ T-type નેટવર્ક માટે કેરેક્ટેરિસ્ટિક ઇમ્પીડન્સ (Z_{0}_{t}) નું સૂત્ર મેળવો.}

\begin{solutionbox}

\textbf{આકૃતિ: સિમેટ્રિકલ T-નેટવર્ક}

\begin{verbatim}
         Z_{1/2        Z_{1}/2}
     o{-{-}{-}{-}//{-}{-}{-}{-}o{-}{-}{-}//{-}{-}{-}{-}o}
     |            |           |
     |            |           |
    Z_{0_{t}          Z_{2}          Z_{0}_{t}}
     |            |           |
     |            |           |
     o{-{-}{-}{-}{-}{-}{-}{-}{-}{-}{-}{-}o{-}{-}{-}{-}{-}{-}{-}{-}{-}{-}{-}o}
\end{verbatim}

\textbf{વ્યુત્પત્તિ:}

\begin{enumerate}
\tightlist
\item
  સિમેટ્રિકલ T-નેટવર્ક માટે, Z_{1} બે ભાગમાં સરખે ભાગે વિભાજિત થાય છે (દરેક Z_{1}/2)
\item
  ઇમેજ ઇમ્પીડન્સ મેચિંગ માટે: Z_{0}_{t} = Z_{0}_{t}'
\end{enumerate}

વોલ્ટેજ ડિવિઝન દ્વારા: V_{2}/V_{1} = Z_{0}_{t}/(Z_{1}/2 + Z_{0}_{t} + Z_{2}\textbar\textbar Z_{0}_{t})

મેચ્ડ કન્ડિશન માટે: Z_{0}_{t}^{2} = (Z_{1}/2)(Z_{1}/2 + Z_{2})

તેથી: Z_{0}_{t} = \sqrt[(Z_{1}/2)(Z_{1}/2 + Z_{2})] Z_{0}_{t} = \sqrt[Z_{1}^{2}/4 + Z_{1}Z_{2}/2] Z_{0}_{t} =
\sqrt[Z_{1}(Z_{1}+2Z_{2})/4]

\end{solutionbox}
\begin{mnemonicbox}
``Z_{1} અને તેની સાથે જોડાયેલા Z_{1}ના વર્ગમૂળ''

\end{mnemonicbox}
\subsection*{પ્રશ્ન 2(a) OR [3
ગુણ]}\label{q2a}

\textbf{π-type નેટવર્કને T-type નેટવર્કમાં કન્વર્ટ કરવા માટે સમીકરણો મેળવો.}

\begin{solutionbox}

\textbf{આકૃતિ: π થી T રૂપાંતરણ}

\begin{verbatim}
    A            B         A    Z_{1    B}
    o          o           o{-{-}{-}//{-}{-}{-}o}
     {        /            |          |}
      {      /             |          |}
      Z_{1_{2}    Z_{2}_{3}     =    Z_{3}         Z_{2}}
       {    /              |          |}
        {  /               o{-}{-}{-}{-}{-}{-}{-}{-}{-}{-}o}
         o                  C
         C
\end{verbatim}

\textbf{રૂપાંતરણ સમીકરણો:}

\begin{itemize}
\tightlist
\item
  Z_{1} = (Z_{1}_{2}Z_{3}_{1})/(Z_{1}_{2} + Z_{2}_{3} + Z_{3}_{1})
\item
  Z_{2} = (Z_{2}_{3}Z_{1}_{2})/(Z_{1}_{2} + Z_{2}_{3} + Z_{3}_{1})
\item
  Z_{3} = (Z_{3}_{1}Z_{2}_{3})/(Z_{1}_{2} + Z_{2}_{3} + Z_{3}_{1})
\end{itemize}

જ્યાં Z_{1}_{2}, Z_{2}_{3}, Z_{3}_{1} એ π-નેટવર્કના ઇમ્પીડન્સ છે અને Z_{1}, Z_{2}, Z_{3} એ T-નેટવર્કના
ઇમ્પીડન્સ છે.

\end{solutionbox}
\begin{mnemonicbox}
``આસન્ન જોડીઓના ગુણાકાર વિભાજિત બધાના સરવાળા દ્વારા''

\end{mnemonicbox}
\subsection*{પ્રશ્ન 2(b) OR [4
ગુણ]}\label{q2b}

\textbf{એડમિટન્સ પેરામીટર (Y પેરામીટર) સમજાવો.}

\begin{solutionbox}

\textbf{Y-પેરામીટર્સ}: આને શોર્ટ-સર્કિટ એડમિટન્સ પેરામીટર્સ પણ કહેવામાં આવે છે કારણ
કે તેઓ આઉટપુટ પોર્ટ્સને શોર્ટ રાખીને માપવામાં આવે છે.


{\def\LTcaptype{none} % do not increment counter
\vspace{-5pt}
\captionof{table}{Y-પેરામીટર સમીકરણો}
\vspace{-10pt}
\begin{longtable}[]{@{}
  >{\raggedright\arraybackslash}p{(\linewidth - 4\tabcolsep) * \real{0.3056}}
  >{\raggedright\arraybackslash}p{(\linewidth - 4\tabcolsep) * \real{0.3333}}
  >{\raggedright\arraybackslash}p{(\linewidth - 4\tabcolsep) * \real{0.3611}}@{}}
\toprule\noalign{}
\begin{minipage}[b]{\linewidth}\raggedright
પેરામીટર
\end{minipage} & \begin{minipage}[b]{\linewidth}\raggedright
વ્યાખ્યા
\end{minipage} & \begin{minipage}[b]{\linewidth}\raggedright
ગણતરી
\end{minipage} \\
\midrule\noalign{}
\endhead
\bottomrule\noalign{}
\endlastfoot
Y_{1}_{1} & આઉટપુટ શોર્ટેડ હોય ત્યારે ઇનપુટ એડમિટન્સ & Y_{1}_{1} = I_{1}/V_{1} (જ્યારે V_{2}=0) \\
Y_{1}_{2} & પોર્ટ 2 થી પોર્ટ 1 સુધીનો ટ્રાન્સફર એડમિટન્સ & Y_{1}_{2} = I_{1}/V_{2} (જ્યારે
V_{1}=0) \\
Y_{2}_{1} & પોર્ટ 1 થી પોર્ટ 2 સુધીનો ટ્રાન્સફર એડમિટન્સ & Y_{2}_{1} = I_{2}/V_{1} (જ્યારે
V_{2}=0) \\
Y_{2}_{2} & ઇનપુટ શોર્ટેડ હોય ત્યારે આઉટપુટ એડમિટન્સ & Y_{2}_{2} = I_{2}/V_{2} (જ્યારે V_{1}=0) \\
\end{longtable}
}

\textbf{મેટ્રિક્સ ફોર્મ:} [I_{1}] = [Y_{1}_{1} Y_{1}_{2}] \times [V_{1}] [I_{2}]
[Y_{2}_{1} Y_{2}_{2}] [V_{2}]

\begin{itemize}
\tightlist
\item
  \textbf{સિમેટ્રિકલ નેટવર્ક}: Y_{1}_{2} = Y_{2}_{1}
\item
  \textbf{એકમો}: સીમેન્સ (S)
\end{itemize}

\end{solutionbox}
\begin{mnemonicbox}
``Is તે Ys ગુણ્યા Vs''

\end{mnemonicbox}
\subsection*{પ્રશ્ન 2(c) OR [7
ગુણ]}\label{q2c}

\textbf{સિમેટ્રિકલ π-type નેટવર્ક માટે કેરેક્ટેરિસ્ટિક ઇમ્પીડન્સ (Z_{0}π) નું સૂત્ર મેળવો.}

\begin{solutionbox}

\textbf{આકૃતિ: સિમેટ્રિકલ π-નેટવર્ક}

\begin{verbatim}
     o{-{-}{-}{-}{-}{-}{-}{-}{-}{-}{-}o{-}{-}{-}{-}{-}{-}{-}{-}{-}{-}o}
     |           |          |
     |           |          |
    2Z_{3         Z_{1}         2Z_{3}}
     |           |          |
     |           |          |
     o{-{-}{-}{-}Z_{0}π{-}{-}{-}{-}o{-}{-}{-}{-}Z_{0}π{-}{-}{-}o}
\end{verbatim}

\textbf{વ્યુત્પત્તિ:}

\begin{enumerate}
\tightlist
\item
  સિમેટ્રિકલ π-નેટવર્ક માટે, શંટ આર્મ્સમાં એડમિટન્સ Y_{1} બે સરખા ભાગમાં વહેંચાય છે (Y_{3} =
  Y_{1}/2)
\item
  ઇમેજ ઇમ્પીડન્સ મેચિંગ માટે: Z_{0}π = Z_{0}π'
\end{enumerate}

કરંટ ડિવિઝન દ્વારા: I_{2}/I_{1} = Z_{0}π/(Z_{0}π + Z_{1} + Z_{0}π\textbar\textbar2Z_{3})

મેચ્ડ કન્ડિશન માટે: Z_{0}π^{2} = Z_{1}(2Z_{3})/(Z_{1} + 2Z_{3})

સરળીકરણ: Z_{0}π = \sqrt[Z_{1}(2Z_{3})/(Z_{1} + 2Z_{3})] Z_{0}π = \sqrt[2Z_{1}Z_{3}/(Z_{1} + 2Z_{3})]

\end{solutionbox}
\begin{mnemonicbox}
``પાઈનો ઇમ્પીડન્સ તે જુએ છે તેનો જ્યામિતીય મધ્યવર્તી''

\end{mnemonicbox}
\subsection*{પ્રશ્ન 3(a) [3
ગુણ]}\label{q3a}

\textbf{ડ્યુઆલિટીનો સિદ્ધાંત સમજાવો.}

\begin{solutionbox}

\textbf{ડ્યુઆલિટીનો સિદ્ધાંત}: દરેક ઇલેક્ટ્રીકલ નેટવર્ક માટે, એક ડ્યુઅલ નેટવર્ક
અસ્તિત્વમાં છે જેનું વર્તન સમાન છે પરંતુ તત્વો બદલાયેલા છે.


{\def\LTcaptype{none} % do not increment counter
\vspace{-5pt}
\captionof{table}{ડ્યુઅલ તત્વ જોડીઓ}
\vspace{-10pt}
\begin{longtable}[]{@{}ll@{}}
\toprule\noalign{}
મૂળ સર્કિટ & ડ્યુઅલ સર્કિટ \\
\midrule\noalign{}
\endhead
\bottomrule\noalign{}
\endlastfoot
વોલ્ટેજ (V) & કરંટ (I) \\
કરંટ (I) & વોલ્ટેજ (V) \\
રેસિસ્ટન્સ (R) & કંડક્ટન્સ (G) \\
ઇન્ડક્ટન્સ (L) & કેપેસિટન્સ (C) \\
શ્રેણી જોડાણ & સમાંતર જોડાણ \\
KVL & KCL \\
મેશ એનાલિસિસ & નોડલ એનાલિસિસ \\
\end{longtable}
}

\begin{itemize}
\tightlist
\item
  \textbf{નેટવર્ક ટ્રાન્સફોર્મેશન}: દરેક તત્વને તેના ડ્યુઅલથી બદલો
\item
  \textbf{ટોપોલોજી ટ્રાન્સફોર્મેશન}: દરેક નોડને લૂપથી અને દરેક લૂપને નોડથી બદલો
\end{itemize}

\end{solutionbox}
\begin{mnemonicbox}
``શ્રેણીથી સમાંતર, સ્ત્રોત બદલે ડ્યુઅલ, V બને I અને I બને V''

\end{mnemonicbox}
\subsection*{પ્રશ્ન 3(b) [4
ગુણ]}\label{q3b}

\textbf{થેવેનિનનો પ્રમેય જણાવો અને સમજાવો.}

\begin{solutionbox}

\textbf{થેવેનિનનો પ્રમેય}: કોઈપણ લીનીયર બે-ટર્મિનલ નેટવર્કને શ્રેણીમાં વોલ્ટેજ સ્ત્રોત
(Vth) અને રેસિસ્ટન્સ (Rth) ધરાવતા સમકક્ષ સર્કિટથી બદલી શકાય છે.

\textbf{આકૃતિ:}

\begin{center}
\textbf{Mermaid Diagram (Code)}
\begin{verbatim}
{Shaded}
{Highlighting}[]
graph TD
    subgraph "Original Network"
    direction LR
    A[Complex Network] {-{-}{-} R1[Load]}
    end
    subgraph "Thevenin Equivalent"
    direction LR
    VTH[Vth] {-{-}{-} RTH[Rth] {-}{-}{-} RL[Load]}
    end
{Highlighting}
{Shaded}
\end{verbatim}
\end{center}

\textbf{થેવેનિન સમકક્ષ શોધવું:}

\begin{enumerate}
\tightlist
\item
  લોડ રેસિસ્ટન્સ દૂર કરો
\item
  ઓપન-સર્કિટ વોલ્ટેજ (Vth) ગણો
\item
  Rth શોધવા માટે:

  \begin{itemize}
  \tightlist
  \item
બધા સ્ત્રોતોને નિષ્ક્રિય કરો (V=0,

I=0)

  \item
    ટર્મિનલ્સ વચ્ચેનો રેસિસ્ટન્સ ગણો
  \end{itemize}
\end{enumerate}

\end{solutionbox}
\begin{mnemonicbox}
``વોલ્ટેજ માટે ખુલ્લું, રેસિસ્ટન્સ માટે મૃત''

\end{mnemonicbox}
\subsection*{પ્રશ્ન 3(c) [7
ગુણ]}\label{q3c}

\textbf{ઉદાહરણ સાથે KCL અને KVL જણાવો અને સમજાવો.}

\begin{solutionbox}


{\def\LTcaptype{none} % do not increment counter
\vspace{-5pt}
\captionof{table}{કિરચોફના નિયમો}
\vspace{-10pt}
\begin{longtable}[]{@{}
  >{\raggedright\arraybackslash}p{(\linewidth - 6\tabcolsep) * \real{0.1042}}
  >{\raggedright\arraybackslash}p{(\linewidth - 6\tabcolsep) * \real{0.2292}}
  >{\raggedright\arraybackslash}p{(\linewidth - 6\tabcolsep) * \real{0.3958}}
  >{\raggedright\arraybackslash}p{(\linewidth - 6\tabcolsep) * \real{0.2708}}@{}}
\toprule\noalign{}
\begin{minipage}[b]{\linewidth}\raggedright
નિયમ
\end{minipage} & \begin{minipage}[b]{\linewidth}\raggedright
અભિધાન
\end{minipage} & \begin{minipage}[b]{\linewidth}\raggedright
ગાણિતિક રૂપ
\end{minipage} & \begin{minipage}[b]{\linewidth}\raggedright
અમલીકરણ
\end{minipage} \\
\midrule\noalign{}
\endhead
\bottomrule\noalign{}
\endlastfoot
KCL & નોડમાં પ્રવેશતા કરંટનો સરવાળો નોડથી બહાર નીકળતા કરંટના સરવાળા બરાબર છે &
\sumIin = \sumIout & નોડલ એનાલિસિસ \\
KVL & કોઈપણ બંધ લૂપ ફરતે વોલ્ટેજ ડ્રોપનો સરવાળો શૂન્ય છે & \sumV = 0 & મેશ
એનાલિસિસ \\
\end{longtable}
}

\textbf{KCL ઉદાહરણ:}

\begin{verbatim}
        I_{1}
        ↓
        o
       / {}
      /   {}
     I_{2    I_{3}}
    /       {}
   o         o

નોડ પર: I_{1 = I_{2} + I_{3}}
\end{verbatim}

\textbf{KVL ઉદાહરણ:}

\begin{verbatim}
    +    R_{1     +}
    o{-{-}{-}///{-}{-}o}
    |           |
   V_{1           R_{2}}
    |           |
    o{-{-}{-}{-}{-}{-}{-}{-}{-}{-}{-}o}
    {-           {-}}

લૂપની આસપાસ: V_{1 {-} I_{1} {-} I_{2} = 0}
\end{verbatim}

\end{solutionbox}
\begin{mnemonicbox}
``નોડ પર કરંટનો સરવાળો શૂન્ય, લૂપ આસપાસ વોલ્ટેજના પણ''

\end{mnemonicbox}
\subsection*{પ્રશ્ન 3(a) OR [3
ગુણ]}\label{q3a}

\textbf{મેશ એનાલિસિસ દ્વારા નેટવર્કનું સોલ્યુશન સમજાવો.}

\begin{solutionbox}

\textbf{મેશ એનાલિસિસ}: એક સર્કિટ એનાલિસિસ પદ્ધતિ જે અજાણી કરંટ અને વોલ્ટેજને
શોધવા માટે મેશ કરંટનો ચલ તરીકે ઉપયોગ કરે છે.

\textbf{આકૃતિ: સિમ્પલ ટુ-મેશ સર્કિટ}

\begin{verbatim}
    +   R_{1    +   R_{3}   +}
    o{-{-}//{-}{-}o{-}{-}//{-}{-}o}
    |         |         |
   V_{1        R_{2}        V_{2}}
    |         |         |
    o{-{-}{-}{-}{-}{-}{-}{-}{-}o{-}{-}{-}{-}{-}{-}{-}{-}{-}o}
    {-         {-}         {-}}
     Mesh 1     Mesh 2
\end{verbatim}

\textbf{પગલાં:}

\begin{enumerate}
\tightlist
\item
  મેશ (બંધ લૂપ) ઓળખો
\item
  ઘડિયાળના કાંટાની દિશામાં મેશ કરંટ (I_{1}, I_{2}) આપો
\item
  દરેક મેશ પર KVL લાગુ કરો
\item
  પરિણામી સમકાલીન સમીકરણોનો ઉકેલ મેળવો
\end{enumerate}

\textbf{ઉદાહરણ સમીકરણો:}

\begin{itemize}
\tightlist
\item
  મેશ 1: V_{1} = I_{1}(R_{1}+R_{2}) - I_{2}R_{2}
\item
  મેશ 2: -V_{2} = -I_{1}R_{2} + I_{2}(R_{2}+R_{3})
\end{itemize}

\end{solutionbox}
\begin{mnemonicbox}
``આપો, KVL લાગુ કરો, ગોઠવો, અને ઉકેલો''

\end{mnemonicbox}
\subsection*{પ્રશ્ન 3(b) OR [4
ગુણ]}\label{q3b}

\textbf{નોર્ટનનો પ્રમેય જણાવો અને સમજાવો.}

\begin{solutionbox}

\textbf{નોર્ટનનો પ્રમેય}: કોઈપણ લીનીયર બે-ટર્મિનલ નેટવર્કને સમાંતરમાં કરંટ સ્ત્રોત
(IN) અને રેસિસ્ટન્સ (RN) ધરાવતા સમકક્ષ સર્કિટથી બદલી શકાય છે.

\textbf{આકૃતિ:}

\begin{center}
\textbf{Mermaid Diagram (Code)}
\begin{verbatim}
{Shaded}
{Highlighting}[]
graph TD
    subgraph "Original Network"
    direction LR
    A[Complex Network] {-{-}{-} R1[Load]}
    end
    subgraph "Norton Equivalent"
    direction LR
    IN[In] {-.{-} RN[Rn] {-}{-}{-} RL[Load]}
    end
{Highlighting}
{Shaded}
\end{verbatim}
\end{center}

\textbf{નોર્ટન સમકક્ષ શોધવું:}

\begin{enumerate}
\tightlist
\item
  લોડ રેસિસ્ટન્સ દૂર કરો
\item
  શોર્ટ-સર્કિટ કરંટ (IN) ગણો
\item
  RN શોધવા માટે:

  \begin{itemize}
  \tightlist
  \item
બધા સ્ત્રોતોને નિષ્ક્રિય કરો (V=0,

I=0)

  \item
    ટર્મિનલ્સ વચ્ચેનો રેસિસ્ટન્સ ગણો (RN = Rth)
  \end{itemize}
\end{enumerate}

\end{solutionbox}
\begin{mnemonicbox}
``કરંટ માટે શોર્ટ, રેસિસ્ટન્સ માટે મૃત''

\end{mnemonicbox}
\subsection*{પ્રશ્ન 3(c) OR [7
ગુણ]}\label{q3c}

\textbf{મહત્તમ પાવર ટ્રાન્સફર પ્રમેય જણાવો અને સમજાવો. મહત્તમ પાવર ટ્રાન્સફર
માટેની સ્થિતિ મેળવો.}

\begin{solutionbox}

\textbf{મહત્તમ પાવર ટ્રાન્સફર પ્રમેય}: જ્યારે લોડનો રેસિસ્ટન્સ નેટવર્કના થેવેનિન સમકક્ષ
રેસિસ્ટન્સ બરાબર હોય ત્યારે લોડને મહત્તમ પાવર મળે છે.

\textbf{આકૃતિ:}

\begin{center}
\textbf{Mermaid Diagram (Code)}
\begin{verbatim}
{Shaded}
{Highlighting}[]
graph LR
    A[Vth] {-{-}{-} B[Rth] {-}{-}{-} C[RL]}
{Highlighting}
{Shaded}
\end{verbatim}
\end{center}

\textbf{વ્યુત્પત્તિ:}

\begin{enumerate}
\item
  લોડને મળતો પાવર: P = I^{2}RL
\item
  સર્કિટમાં કરંટ: I = Vth/(Rth + RL)
\item
  બદલતાં: P = Vth^{2}RL/(Rth + RL)^{2}
\item
  RL ના સંદર્ભમાં ડિફરેન્શિએટ કરીને શૂન્ય સુયોજિત કરતાં: dP/dRL = 0
\item
  આ આપે છે: RL = Rth
\item
  મહત્તમ પાવર: Pmax = Vth^{2}/(4Rth)
\end{enumerate}

\end{solutionbox}
\begin{mnemonicbox}
``મેચ કરો, મહત્તમ બનાવો''

\end{mnemonicbox}
\subsection*{પ્રશ્ન 4(a) [3
ગુણ]}\label{q4a}

\textbf{કોઇલ માટે Q પરિબળનું સમીકરણ મેળવો.}

\begin{solutionbox}

\textbf{Q ફેક્ટર (ક્વોલિટી ફેક્ટર)} કોઇલ માટે ઇન્ડક્ટિવ રિએક્ટન્સનો રેસિસ્ટન્સ સાથેનો
ગુણોત્તર દર્શાવે છે.

\textbf{આકૃતિ: રેસિસ્ટન્સ સાથેની કોઇલ}

\begin{verbatim}
    o{-{-}{-}{-}///{-}{-}{-}{-}uuuu{-}{-}{-}{-}o}
         R          L
\end{verbatim}

\textbf{વ્યુત્પત્તિ:}

\begin{enumerate}
\tightlist
\item
  રેસિસ્ટન્સ સાથેની ઇન્ડક્ટર માટે, ઇમ્પીડન્સ Z = R + jωL
\item
  Q ફેક્ટર વ્યાખ્યા: Q = રિએક્ટિવ પાવર / એક્ટિવ પાવર
\item
  Q = ωL/R
\end{enumerate}

જ્યાં:

\begin{itemize}
\tightlist
\item
  L = ઇન્ડક્ટન્સ હેનરીમાં
\item
  R = શ્રેણી રેસિસ્ટન્સ ઓહ્મમાં
\item
  ω = 2πf, એન્ગ્યુલર ફ્રીક્વન્સી
\end{itemize}

\end{solutionbox}
\begin{mnemonicbox}
``ક્વોલિટી તે રિએક્ટન્સ ભાગે રેસિસ્ટન્સ''

\end{mnemonicbox}
\subsection*{પ્રશ્ન 4(b) [4
ગુણ]}\label{q4b}

\textbf{સમાંતર RLC સર્કિટ માટે રેઝોનન્ટ ફ્રીક્વન્સીનું સમીકરણ મેળવો.}

\begin{solutionbox}

\textbf{આકૃતિ: સમાંતર RLC સર્કિટ}

\begin{verbatim}
        o{-{-}{-}{-}{-}o}
        |     |
        R     |
        |     |
        o     |
        |     |
        C     L
        |     |
        o{-{-}{-}{-}{-}o}
\end{verbatim}

\textbf{વ્યુત્પત્તિ:}

\begin{enumerate}
\tightlist
\item
સમાંતર RLC નો એડમિટન્સ:

Y = 1/R + jωC + 1/jωL = 1/R + j(ωC - 1/ωL)

\item
  રેઝોનન્સ પર, કાલ્પનિક ભાગ શૂન્ય છે: ωC - 1/ωL = 0
\item
  ω માટે ઉકેલતાં: ω^{2} = 1/LC
\item
  તેથી: ω = 1/\sqrt(LC)
\item
  રેઝોનન્સ ફ્રીક્વન્સી: fr = 1/(2π\sqrt(LC))
\end{enumerate}

\textbf{નોંધ:} R બેન્ડવિડ્થને અસર કરે છે પરંતુ રેઝોનન્સ ફ્રીક્વન્સીને નહીં.

\end{solutionbox}
\begin{mnemonicbox}
``એક ભાગે બે પાઈ ગુણ્યા LC ના વર્ગમૂળ''

\end{mnemonicbox}
\subsection*{પ્રશ્ન 4(c) [7
ગુણ]}\label{q4c}

\textbf{જરૂરી ડાયાગ્રામ સાથે કપલ્ડ સર્કિટના પ્રકારો લખો અને આયર્ન કોર ટ્રાન્સફોર્મર
સમજાવો.}

\begin{solutionbox}


{\def\LTcaptype{none} % do not increment counter
\vspace{-5pt}
\captionof{table}{કપલ્ડ સર્કિટના પ્રકાર}
\vspace{-10pt}
\begin{longtable}[]{@{}lll@{}}
\toprule\noalign{}
પ્રકાર & કપલિંગ માધ્યમ & અમલીકરણ \\
\midrule\noalign{}
\endhead
\bottomrule\noalign{}
\endlastfoot
ડાયરેક્ટ કપલિંગ & વાહકથી જોડાયેલ & DC એમ્પ્લિફાયર્સ \\
કેપેસિટિવ કપલિંગ & કેપેસિટર & AC સિગ્નલ કપલિંગ \\
ઇન્ડક્ટિવ કપલિંગ & ચુંબકીય ક્ષેત્ર & ટ્રાન્સફોર્મર્સ \\
રેસિસ્ટિવ કપલિંગ & રેસિસ્ટર & ઓછી આવૃત્તિના સિગ્નલ \\
\end{longtable}
}

\textbf{આકૃતિ: આયર્ન કોર ટ્રાન્સફોર્મર}

\begin{center}
\textbf{Mermaid Diagram (Code)}
\begin{verbatim}
{Shaded}
{Highlighting}[]
graph LR
    subgraph "Primary"
    V1[V_{1] {-}{-}{-} L1[uuuu]}
    end

    subgraph "Iron Core"
    Core[" "]
    end
    
    subgraph "Secondary"
    L2[uuuu] {-{-}{-} V2[V_{2}]}
    end
    
    L1 {-{-}{-} Core {-}{-}{-} L2}
{Highlighting}
{Shaded}
\end{verbatim}
\end{center}

\textbf{આયર્ન કોર ટ્રાન્સફોર્મર:}

\begin{itemize}
\tightlist
\item
  \textbf{સિદ્ધાંત}: આયર્ન કોર દ્વારા મ્યુચ્યુઅલ ઇન્ડક્ટન્સ
\item
  \textbf{કાર્ય}: ઇલેક્ટ્રોમેગ્નેટિક ઇન્ડક્શન દ્વારા સર્કિટ્સ વચ્ચે ઊર્જા ટ્રાન્સફર કરે છે
\item
  \textbf{કપલિંગ કોઇફિશિયન્ટ}: k \approx 1 (લગભગ પરફેક્ટ કપલિંગ)
\item
  \textbf{ટર્ન્સ રેશિયો}: V_{2}/V_{1} = N_{2}/N_{1}
\item
  \textbf{ફાયદા}: ઉચ્ચ કાર્યક્ષમતા, સારું કપલિંગ
\end{itemize}

\end{solutionbox}
\begin{mnemonicbox}
``પ્રાથમિક ઉત્તેજિત કરે, કોર વહન કરે, સેકન્ડરી પહોંચાડે''

\end{mnemonicbox}
\subsection*{પ્રશ્ન 4(a) OR [3
ગુણ]}\label{q4a}

\textbf{કેપેસિટર માટે Q પરિબળનું સમીકરણ મેળવો.}

\begin{solutionbox}

\textbf{Q ફેક્ટર (ક્વોલિટી ફેક્ટર)} કેપેસિટર માટે કેપેસિટિવ રિએક્ટન્સનો રેસિસ્ટન્સ
સાથેનો ગુણોત્તર દર્શાવે છે.

\textbf{આકૃતિ: રેસિસ્ટન્સ સાથેની કેપેસિટર}

\begin{verbatim}
    o{-{-}{-}{-}///{-}{-}{-}{-}||{-}{-}{-}{-}o}
         R          C
\end{verbatim}

\textbf{વ્યુત્પત્તિ:}

\begin{enumerate}
\tightlist
\item
  સીરીઝ રેસિસ્ટન્સ સાથેની કેપેસિટર માટે, ઇમ્પીડન્સ Z = R - j/(ωC)
\item
  Q ફેક્ટર વ્યાખ્યા: Q = રિએક્ટિવ પાવર / એક્ટિવ પાવર
\item
  Q = 1/(ωCR)
\end{enumerate}

જ્યાં:

\begin{itemize}
\tightlist
\item
  C = કેપેસિટન્સ ફેરડમાં
\item
  R = સીરીઝ રેસિસ્ટન્સ ઓહ્મમાં
\item
  ω = 2πf, એન્ગ્યુલર ફ્રીક્વન્સી
\end{itemize}

\end{solutionbox}
\begin{mnemonicbox}
``ક્વોલિટી તે એક ભાગે રેસિસ્ટન્સ ગુણ્યા રિએક્ટન્સ''

\end{mnemonicbox}
\subsection*{પ્રશ્ન 4(b) OR [4
ગુણ]}\label{q4b}

\textbf{શ્રેણી રેઝોનન્સ સર્કિટ માટે રેઝોનન્સ ફ્રીક્વન્સીનું સમીકરણ મેળવો.}

\begin{solutionbox}

\textbf{આકૃતિ: શ્રેણી RLC સર્કિટ}

\begin{verbatim}
    o{-{-}{-}{-}///{-}{-}{-}{-}uuuu{-}{-}{-}{-}||{-}{-}{-}{-}o}
         R          L       C
\end{verbatim}

\textbf{વ્યુત્પત્તિ:}

\begin{enumerate}
\tightlist
\item
શ્રેણી RLC નો ઇમ્પીડન્સ:

Z = R + jωL - j/(ωC) = R + j(ωL - 1/ωC)

\item
  રેઝોનન્સ પર, કાલ્પનિક ભાગ શૂન્ય છે: ωL - 1/ωC = 0
\item
  ω માટે ઉકેલતાં: ω^{2} = 1/LC
\item
  તેથી: ω = 1/\sqrt(LC)
\item
  રેઝોનન્સ ફ્રીક્વન્સી: fr = 1/(2π\sqrt(LC))
\end{enumerate}

\textbf{મુખ્ય મુદ્દાઓ:}

\begin{itemize}
\tightlist
\item
  રેઝોનન્સ પર, ઇમ્પીડન્સ માત્ર રેસિસ્ટિવ છે: Z = R
\item
  સર્કિટ રેસિસ્ટર જેવું દેખાય છે
\item
  રેઝોનન્સ પર કરંટ મહત્તમ છે
\end{itemize}

\end{solutionbox}
\begin{mnemonicbox}
``એક ભાગે બે પાઈ ગુણ્યા LC ના વર્ગમૂળ''

\end{mnemonicbox}
\subsection*{પ્રશ્ન 4(c) OR [7
ગુણ]}\label{q4c}

\textbf{ચુંબકીય રીતે જોડાયેલા કોઇલની પેર વચ્ચે કોએફિસિયન્ટ ઓફ કપલિંગનું સમીકરણ
મેળવો.}

\begin{solutionbox}

\textbf{આકૃતિ: ચુંબકીય રીતે જોડાયેલા કોઇલ્સ}

\begin{verbatim}
       uuuu   k   uuuu
    o{-{-}WWWW{-}{-}{-}{-}{-}{-}{-}WWWW{-}{-}o}
       L_{1          L_{2}    }
\end{verbatim}

\textbf{વ્યુત્પત્તિ:}

\begin{enumerate}
\tightlist
\item
  મ્યુચ્યુઅલ ઇન્ડક્ટન્સ (M) વ્યક્તિગત ઇન્ડક્ટન્સથી સંબંધિત છે: M = k\sqrt(L_{1}L_{2})
\item
  k માટે ઉકેલીને: k = M/\sqrt(L_{1}L_{2})
\end{enumerate}

જ્યાં:

\begin{itemize}
\tightlist
\item
  k = કોએફિસિયન્ટ ઓફ કપલિંગ (0 \leq k \leq 1)
\item
  M = મ્યુચ્યુઅલ ઇન્ડક્ટન્સ હેનરીમાં
\item
  L_{1}, L_{2} = કોઇલ્સના સેલ્ફ-ઇન્ડક્ટન્સ હેનરીમાં
\end{itemize}


{\def\LTcaptype{none} % do not increment counter
\vspace{-5pt}
\captionof{table}{કપલિંગ કોએફિસિયન્ટના મૂલ્યો}
\vspace{-10pt}
\begin{longtable}[]{@{}lll@{}}
\toprule\noalign{}
k નું મૂલ્ય & કપલિંગનો પ્રકાર & અમલીકરણ \\
\midrule\noalign{}
\endhead
\bottomrule\noalign{}
\endlastfoot
k = 0 & કોઈ કપલિંગ નહીં & અલગ સર્કિટ્સ \\
0 \textless{} k \textless{} 0.5 & લૂઝ કપલિંગ & RF ટ્રાન્સફોર્મર્સ \\
0.5 \textless{} k \textless{} 1 & ટાઇટ કપલિંગ & પાવર ટ્રાન્સફોર્મર્સ \\
k = 1 & પરફેક્ટ કપલિંગ & આદર્શ ટ્રાન્સફોર્મર \\
\end{longtable}
}

\end{solutionbox}
\begin{mnemonicbox}
``મ્યુચ્યુઅલ ભાગે ગુણાકારના વર્ગમૂળ''

\end{mnemonicbox}
\subsection*{પ્રશ્ન 5(a) [3
ગુણ]}\label{q5a}

\textbf{Neper અને dB ને વ્યાખ્યાયિત કરો. નેપર અને ડીબી વચ્ચે સંબંધ સ્થાપિત કરો.}

\begin{solutionbox}


{\def\LTcaptype{none} % do not increment counter
\vspace{-5pt}
\captionof{table}{Neper અને dB વ્યાખ્યાઓ}
\vspace{-10pt}
\begin{longtable}[]{@{}
  >{\raggedright\arraybackslash}p{(\linewidth - 6\tabcolsep) * \real{0.1765}}
  >{\raggedright\arraybackslash}p{(\linewidth - 6\tabcolsep) * \real{0.3529}}
  >{\raggedright\arraybackslash}p{(\linewidth - 6\tabcolsep) * \real{0.2647}}
  >{\raggedright\arraybackslash}p{(\linewidth - 6\tabcolsep) * \real{0.2059}}@{}}
\toprule\noalign{}
\begin{minipage}[b]{\linewidth}\raggedright
એકમ
\end{minipage} & \begin{minipage}[b]{\linewidth}\raggedright
વ્યાખ્યા
\end{minipage} & \begin{minipage}[b]{\linewidth}\raggedright
સૂત્ર
\end{minipage} & \begin{minipage}[b]{\linewidth}\raggedright
ઉપયોગ
\end{minipage} \\
\midrule\noalign{}
\endhead
\bottomrule\noalign{}
\endlastfoot
Neper (Np) & કુદરતી લોગેરિધમિક ગુણોત્તર & N = ln(V_{1}/V_{2}) અથવા ln(I_{1}/I_{2}) &
પાવર સિસ્ટમ એનાલિસિસ \\
Decibel (dB) & સામાન્ય લોગેરિધમિક ગુણોત્તર & dB = 20log_{1}_{0}(V_{1}/V_{2}) અથવા
10log_{1}_{0}(P_{1}/P_{2}) & સિગ્નલ લેવલ માપન \\
\end{longtable}
}

\textbf{સંબંધ:}

\begin{enumerate}
\tightlist
\item
  N = ln(V_{1}/V_{2})
\item
  dB = 20log_{1}_{0}(V_{1}/V_{2})
\item
  જેમ ln(x) = 2.303 \times log_{1}_{0}(x)
\item
તેથી:

N = 2.303 \times dB/20 = 0.1152 \times dB

\item
  વિપરીતરીતે: dB = 8.686 \times N
\end{enumerate}

\end{solutionbox}
\begin{mnemonicbox}
``એક Neper એ 8.686 dB છે''

\end{mnemonicbox}
\subsection*{પ્રશ્ન 5(b) [4
ગુણ]}\label{q5b}

\textbf{વિવિધ પ્રકારના એટેન્યુએટરનું વર્ગીકરણ કરો.}

\begin{solutionbox}


{\def\LTcaptype{none} % do not increment counter
\vspace{-5pt}
\captionof{table}{એટેન્યુએટરના પ્રકાર}
\vspace{-10pt}
\begin{longtable}[]{@{}
  >{\raggedright\arraybackslash}p{(\linewidth - 6\tabcolsep) * \real{0.1304}}
  >{\raggedright\arraybackslash}p{(\linewidth - 6\tabcolsep) * \real{0.2391}}
  >{\raggedright\arraybackslash}p{(\linewidth - 6\tabcolsep) * \real{0.3478}}
  >{\raggedright\arraybackslash}p{(\linewidth - 6\tabcolsep) * \real{0.2826}}@{}}
\toprule\noalign{}
\begin{minipage}[b]{\linewidth}\raggedright
પ્રકાર
\end{minipage} & \begin{minipage}[b]{\linewidth}\raggedright
રચના
\end{minipage} & \begin{minipage}[b]{\linewidth}\raggedright
લાક્ષણિકતાઓ
\end{minipage} & \begin{minipage}[b]{\linewidth}\raggedright
ઉપયોગો
\end{minipage} \\
\midrule\noalign{}
\endhead
\bottomrule\noalign{}
\endlastfoot
T-type & T આકારમાં ત્રણ રેસિસ્ટર & ફિક્સ્ડ ઇમ્પીડન્સ, સારું બેલેન્સ & સિગ્નલ લેવલ
કંટ્રોલ \\
π-type (Pi) & π આકારમાં ત્રણ રેસિસ્ટર & બેહતર આઇસોલેશન, વધુ સામાન્ય & RF સિગ્નલ
એટેન્યુએશન \\
L-type & L આકારમાં બે રેસિસ્ટર & સરળ, અસંતુલિત & બેસિક લેવલ એડજસ્ટમેન્ટ \\
Bridged T & બ્રિજિંગ રેસિસ્ટર સાથે T & સતત ઇમ્પીડન્સ & ઓડિયો એપ્લિકેશન્સ \\
Balanced & સિમેટ્રિકલ ડિઝાઇન & સારો CMRR & બેલેન્સ્ડ ટ્રાન્સમિશન \\
Lattice & હીરા આકારનું & બેલેન્સ્ડ, સિમેટ્રિકલ & ટેલીફોન સિસ્ટમ્સ \\
\end{longtable}
}

\textbf{આકૃતિ: મૂળભૂત એટેન્યુએટર પ્રકાર}

\begin{center}
\textbf{Mermaid Diagram (Code)}
\begin{verbatim}
{Shaded}
{Highlighting}[]
graph TD
    subgraph "T{-type"}
    direction LR
    T1[o]{-{-}{-}TR1[R_{1}]{-}{-}{-}T2[o]}
    TR2[R_{2]}
    T2{-{-}{-}TR2{-}{-}{-}T3[o]}
    end

    subgraph "π{-type"}
    direction LR
    P1[o]{-{-}{-}PR1[R_{1}]{-}{-}{-}P2[o]}
    PR2[R_{2]}
    P1{-{-}{-}PR2}
    PR3[R_{3]}
    PR2{-{-}{-}P2}
    end
{Highlighting}
{Shaded}
\end{verbatim}
\end{center}

\end{solutionbox}
\begin{mnemonicbox}
``Tees, Pies અને Ells સિગ્નલને સારી રીતે એટેન્યુએટ કરે છે''

\end{mnemonicbox}
\subsection*{પ્રશ્ન 5(c) [7
ગુણ]}\label{q5c}

\textbf{નીચે બતાવેલ લો-પાસ ફિલ્ટરની કટ-ઓફ આવૃત્તિ અને નોમિનલ ઈંપીડન્સ નક્કી કરો.}

\begin{solutionbox}

\textbf{આકૃતિ: લો-પાસ ફિલ્ટર સેક્શન્સ}

\begin{verbatim}
    T{-section:                   π{-}section:}
    o{-{-}{-}L/2{-}{-}o{-}{-}{-}L/2{-}{-}{-}{-}o       o{-}{-}{-}{-}{-}{-}{-}{-}{-}{-}{-}{-}o{-}{-}{-}{-}{-}{-}{-}{-}{-}{-}{-}o}
             |                   |           |           |
             C                   C/2         |          C/2
             |                   |           |           |
    o{-{-}{-}{-}{-}{-}{-}{-}o{-}{-}{-}{-}{-}{-}{-}{-}{-}{-}{-}o      o{-}{-}{-}{-}{-}L{-}{-}{-}{-}{-}{-}o{-}{-}{-}{-}{-}{-}{-}{-}{-}{-}{-}o}
\end{verbatim}

\textbf{T-સેક્શન માટે:}

\begin{itemize}
\tightlist
\item
  કટ-ઓફ ફ્રીક્વન્સી: fc = 1/(π\sqrt(LC))
\item
  નોમિનલ ઇમ્પીડન્સ: R_{0} = \sqrt(L/C)
\item
જ્યાં

L = 10 mH,

C = 0.1 μF

\end{itemize}

ગણતરી: fc = 1/(π\sqrt(10\times10^{-}^{3} \times 0.1\times10^{-}^{6})) = 1/(π\sqrt(10^{-}^{9})) = 1/(π\times10^{-}^{4}·^{5}) =
3.18 kHz R_{0} = \sqrt(10\times10^{-}^{3}/0.1\times10^{-}^{6}) = \sqrt(10^{5}) = 316.23 Ω

\textbf{π-સેક્શન માટે:}

\begin{itemize}
\tightlist
\item
  કટ-ઓફ ફ્રીક્વન્સી: fc = 1/(π\sqrt(LC))
\item
  નોમિનલ ઇમ્પીડન્સ: R_{0} = \sqrt(L/C)
\item
  T-સેક્શન જેવા જ મૂલ્યો
\end{itemize}

\end{solutionbox}
\begin{mnemonicbox}
``કટ-ઓફ ફ્રીક્વન્સી એ LC ના વર્ગમૂળના વ્યસ્ત છે''

\end{mnemonicbox}
\subsection*{પ્રશ્ન 5(a) OR [3
ગુણ]}\label{q5a}

\textbf{કોન્સ્ટન્ટ-કે પ્રકારના ફિલ્ટર્સની મર્યાદા સમજાવો.}

\begin{solutionbox}


{\def\LTcaptype{none} % do not increment counter
\vspace{-5pt}
\captionof{table}{કોન્સ્ટન્ટ-k ફિલ્ટર્સની મર્યાદાઓ}
\vspace{-10pt}
\begin{longtable}[]{@{}
  >{\raggedright\arraybackslash}p{(\linewidth - 4\tabcolsep) * \real{0.3636}}
  >{\raggedright\arraybackslash}p{(\linewidth - 4\tabcolsep) * \real{0.3939}}
  >{\raggedright\arraybackslash}p{(\linewidth - 4\tabcolsep) * \real{0.2424}}@{}}
\toprule\noalign{}
\begin{minipage}[b]{\linewidth}\raggedright
મર્યાદા
\end{minipage} & \begin{minipage}[b]{\linewidth}\raggedright
વિવરણ
\end{minipage} & \begin{minipage}[b]{\linewidth}\raggedright
અસર
\end{minipage} \\
\midrule\noalign{}
\endhead
\bottomrule\noalign{}
\endlastfoot
ઇમ્પીડન્સ મેચિંગ & ઇમ્પીડન્સ ફ્રીક્વન્સી સાથે બદલાય છે & સિગ્નલ પરાવર્તન, પાવર
નુકસાન \\
એટેન્યુએશન બેન્ડ & કટ-ઓફ પર ધીમું પરિવર્તન & નબળી ફ્રીક્વન્સી સિલેક્ટિવિટી \\
ફેઝ રિસ્પોન્સ & નોન-લિનિયર ફેઝ લાક્ષણિકતા & સિગ્નલ ડિસ્ટોર્શન \\
પાસબેન્ડ રિપલ & પાસબેન્ડમાં અસમાન રિસ્પોન્સ & સિગ્નલ એમ્પ્લિટ્યુડ વેરિએશન \\
રોલ-ઓફ રેટ & ધીમો રોલ-ઓફ (20 dB/decade) & નબળું સ્ટોપ-બેન્ડ રિજેક્શન \\
\end{longtable}
}

\begin{itemize}
\tightlist
\item
  \textbf{મુખ્ય સમસ્યા}: પાસ બેન્ડથી સ્ટોપ બેન્ડમાં નબળું પરિવર્તન
\item
  \textbf{સુધારો}: m-derived ફિલ્ટર્સનો ઉપયોગ
\end{itemize}

\end{solutionbox}
\begin{mnemonicbox}
``નબળું મેચિંગ અને ટ્રાન્ઝિશન ડિસ્ટોર્શનમાં પરિણમે''

\end{mnemonicbox}
\subsection*{પ્રશ્ન 5(b) OR [4
ગુણ]}\label{q5b}

\textbf{T-પ્રકાર કોન્સ્ટન્ટ-કે હાઇ પાસ ફિલ્ટર માટે કટ-ઓફ આવૃત્તિનું સમીકરણ મેળવો.}

\begin{solutionbox}

\textbf{આકૃતિ: T-પ્રકાર કોન્સ્ટન્ટ-k હાઇ પાસ ફિલ્ટર}

\begin{verbatim}
    o{-{-}{-}{-}C/2{-}{-}{-}{-}o{-}{-}{-}{-}C/2{-}{-}{-}{-}o}
                |               
                L
                |
    o{-{-}{-}{-}{-}{-}{-}{-}{-}{-}{-}o{-}{-}{-}{-}{-}{-}{-}{-}{-}{-}{-}o}
\end{verbatim}

\textbf{વ્યુત્પત્તિ:}

\begin{enumerate}
\tightlist
\item
  હાઇ-પાસ ફિલ્ટર માટે, સીરીઝ એલિમેન્ટ્સ કેપેસિટર છે અને શંટ એલિમેન્ટ્સ ઇન્ડક્ટર છે
\item
  ટ્રાન્સફર ફંક્શન: H(jω) = Z_{2}/(Z_{1} + Z_{2})
\item
  જ્યાં Z_{1} = 1/(jωC) અને Z_{2} = jωL
\item
  કટ-ઓફ માટે ઇમ્પીડન્સ કન્ડિશન: Z_{1}/Z_{2} = 4 અથવા Z_{1}/4Z_{2} = 1
\item
  બદલવાથી: 1/(jωC) = 4jωL
\item
  ω માટે ઉકેલવાથી: ω^{2} = 1/(4LC)
\item
  કટ-ઓફ ફ્રીક્વન્સી: fc = 1/(4π\sqrt(LC))
\end{enumerate}

\end{solutionbox}
\begin{mnemonicbox}
``હાઇ પાસ એક ભાગે ચાર પાઈ એલ-સી નીચેની ફ્રીક્વન્સી કાપે''

\end{mnemonicbox}
\subsection*{પ્રશ્ન 5(c) OR [7
ગુણ]}\label{q5c}

\textbf{વ્યાખ્યાઓ અને લાક્ષણિકતાઓના ગ્રાફનો ઉપયોગ કરીને ફિલ્ટર્સનું વર્ગીકરણ આપો.}

\begin{solutionbox}


{\def\LTcaptype{none} % do not increment counter
\vspace{-5pt}
\captionof{table}{ફિલ્ટર વર્ગીકરણ}
\vspace{-10pt}
\begin{longtable}[]{@{}
  >{\raggedright\arraybackslash}p{(\linewidth - 6\tabcolsep) * \real{0.3023}}
  >{\raggedright\arraybackslash}p{(\linewidth - 6\tabcolsep) * \real{0.1860}}
  >{\raggedright\arraybackslash}p{(\linewidth - 6\tabcolsep) * \real{0.1860}}
  >{\raggedright\arraybackslash}p{(\linewidth - 6\tabcolsep) * \real{0.3256}}@{}}
\toprule\noalign{}
\begin{minipage}[b]{\linewidth}\raggedright
ફિલ્ટર પ્રકાર
\end{minipage} & \begin{minipage}[b]{\linewidth}\raggedright
પસાર કરે છે
\end{minipage} & \begin{minipage}[b]{\linewidth}\raggedright
અટકાવે છે
\end{minipage} & \begin{minipage}[b]{\linewidth}\raggedright
અમલીકરણો
\end{minipage} \\
\midrule\noalign{}
\endhead
\bottomrule\noalign{}
\endlastfoot
લો-પાસ & fc નીચેની ફ્રીક્વન્સીઓ & fc ઉપરની ફ્રીક્વન્સીઓ & ઓડિયો એમ્પ્લિફાયર્સ,
પાવર સપ્લાઈ \\
હાઇ-પાસ & fc ઉપરની ફ્રીક્વન્સીઓ & fc નીચેની ફ્રીક્વન્સીઓ & નોઈઝ એલિમિનેશન, ટ્રેબલ
કંટ્રોલ \\
બેન્ડ-પાસ & fL અને fH વચ્ચેની રેન્જ & રેન્જની બહારની ફ્રીક્વન્સીઓ & રેડિયો ટ્યુનિંગ,
ઇક્વલાઇઝર્સ \\
બેન્ડ-સ્ટોપ & રેન્જની બહારની ફ્રીક્વન્સીઓ & fL અને fH વચ્ચેની રેન્જ & નોઈઝ એલિમિનેશન,
નોચ ફિલ્ટર્સ \\
ઓલ-પાસ & યુનિટી ગેઇન સાથે બધી ફ્રીક્વન્સીઓ & કોઈ નહીં (માત્ર ફેઝ બદલે છે) & ફેઝ
કરેક્શન, ટાઇમ ડિલે \\
\end{longtable}
}

\textbf{લાક્ષણિક રિસ્પોન્સ ગ્રાફ:}

\begin{center}
\textbf{Mermaid Diagram (Code)}
\begin{verbatim}
{Shaded}
{Highlighting}[]
graph TD
    subgraph "Low{-Pass"}
    LP[High{br/{}│{}br/{}Gain{}br/{}│{}br/{}Low] {-}{-}{-} LPf[Frequency ]}

    style LP stroke{-width:0, fill:\#fff}
    style LPf stroke{-width:0, fill:\#fff}
    end
    
    subgraph "High{-Pass"}
    HP[High{br/{}│{}br/{}Gain{}br/{}│{}br/{}Low] {-}{-}{-} HPf[Frequency ]}
    
    style HP stroke{-width:0, fill:\#fff}
    style HPf stroke{-width:0, fill:\#fff}
    end
    
    subgraph "Band{-Pass"}
    BP[High{br/{}│{}br/{}Gain{}br/{}│{}br/{}Low] {-}{-}{-} BPf[Frequency ]}
    
    style BP stroke{-width:0, fill:\#fff}
    style BPf stroke{-width:0, fill:\#fff}
    end
    
    subgraph "Band{-Stop"}
    BS[High{br/{}│{}br/{}Gain{}br/{}│{}br/{}Low] {-}{-}{-} BSf[Frequency ]}
    
    style BS stroke{-width:0, fill:\#fff}
    style BSf stroke{-width:0, fill:\#fff}
    end
{Highlighting}
{Shaded}
\end{verbatim}
\end{center}

\textbf{ફિલ્ટર અમલીકરણો:}

\begin{itemize}
\tightlist
\item
  \textbf{પેસિવ}: R, L, C ઘટકોનો ઉપયોગ કરે છે
\item
  \textbf{એક્ટિવ}: RC નેટવર્ક સાથે ઓપ-એમ્પ્સનો ઉપયોગ કરે છે
\item
  \textbf{ડિજિટલ}: DSP એલ્ગોરિધમનો ઉપયોગ કરે છે
\end{itemize}

\end{solutionbox}
\begin{mnemonicbox}
``લો-હાઇ-બેન્ડ-સ્ટોપ સિગ્નલને પરફેક્ટ બનાવે છે''

\end{mnemonicbox}

\end{document}
