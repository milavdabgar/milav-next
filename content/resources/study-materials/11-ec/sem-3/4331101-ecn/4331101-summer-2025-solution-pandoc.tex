\documentclass[10pt,a4paper]{article}

% content/resources/templates/preamble.tex
\usepackage[margin=0.6in]{geometry}
\author{Milav Dabgar}
\usepackage{amsmath,amssymb,amsthm}
\usepackage{booktabs}
\usepackage{multirow}
\usepackage{xcolor}
\usepackage{tcolorbox}
\tcbuselibrary{breakable,skins}
\usepackage[colorlinks=true,linkcolor=blue]{hyperref}
\usepackage{titlesec}
\usepackage{enumitem}
\usepackage{tikz}
\usepackage{pgfplots}
\usepackage{circuitikz}
\usepackage[version=4]{mhchem}
\usepackage{longtable}
\usepackage{array}
\usepackage{float}
\usepackage{caption}
\usepackage{listings}

\lstset{
  basicstyle=\small\ttfamily,
  breaklines=true,
  breakatwhitespace=false,
  postbreak=\mbox{\textcolor{red}{$\hookrightarrow$}\space},
  float=false,
  numbers=left,
  numberstyle=\tiny\color{gray},
  numbersep=10pt,
  xleftmargin=2em,
  keywordstyle=\color{blue},
  commentstyle=\color{green!60!black},
  stringstyle=\color{purple},
  backgroundcolor=\color{gray!5},
  showstringspaces=false,
  tabsize=2,
  captionpos=b,
  keepspaces=true,
  columns=flexible
}

\pgfplotsset{compat=1.18}
\usetikzlibrary{shapes,arrows,positioning,calc,patterns,decorations.pathmorphing,decorations.markings,arrows.meta}

% Color scheme
\definecolor{headcolor}{RGB}{0,102,204}
\definecolor{keycolor}{RGB}{220,20,60}
\definecolor{solutioncolor}{RGB}{34,139,34}
\definecolor{mnemoniccolor}{RGB}{148,0,211}
\definecolor{codecolor}{RGB}{0,0,100}

% Spacing
\setlength{\parskip}{3pt}
\setlist[itemize]{nosep}
\setlist[enumerate]{nosep}

% Title formatting
\titleformat{\section}{\Large\bfseries\color{headcolor}}{\thesection}{1em}{}
\titleformat{\subsection}{\large\bfseries\color{headcolor}}{\thesubsection}{1em}{}

% Pandoc tightlist compatibility
\providecommand{\tightlist}{%
  \setlength{\itemsep}{0pt}\setlength{\parskip}{0pt}}

% Pandoc longtable compatibility
\newcounter{none}
\def\thenone{}


% content/resources/templates/english-boxes.tex
% This file is currently empty - it exists to maintain consistency with the import structure.
% Add custom environments here if needed in the future.


\begin{document}

\begin{center}
{\Huge\bfseries\color{headcolor} Subject Name Solutions}\\[5pt]
{\LARGE 4331101 -- Summer 2025}\\[3pt]
{\large Semester 1 Study Material}\\[3pt]
{\normalsize\textit{Detailed Solutions and Explanations}}
\end{center}

\vspace{10pt}

\subsection*{Question 1(a) [3 marks]}\label{q1a}

\textbf{Define following terms. (i) Active elements (ii) Bilateral
elements (iii) Linear elements}

\begin{solutionbox}

{\def\LTcaptype{none} % do not increment counter
\begin{longtable}[]{@{}
  >{\raggedright\arraybackslash}p{(\linewidth - 2\tabcolsep) * \real{0.3333}}
  >{\raggedright\arraybackslash}p{(\linewidth - 2\tabcolsep) * \real{0.6667}}@{}}
\toprule\noalign{}
\begin{minipage}[b]{\linewidth}\raggedright
Term
\end{minipage} & \begin{minipage}[b]{\linewidth}\raggedright
Definition
\end{minipage} \\
\midrule\noalign{}
\endhead
\bottomrule\noalign{}
\endlastfoot
\textbf{Active elements} & Electronic components that can supply energy
or power to a circuit (like batteries, generators, op-amps) \\
\textbf{Bilateral elements} & Components that allow current flow equally
in both directions with same characteristics (like resistors,
capacitors, inductors) \\
\textbf{Linear elements} & Components whose current-voltage relationship
follows a straight line and obeys the principle of superposition (like
resistors following Ohm's law) \\
\end{longtable}
}

\end{solutionbox}
\begin{mnemonicbox}
``ABL: Active powers Batteries, Bilateral flows Both
ways, Linear stays Lawful''

\end{mnemonicbox}
\subsection*{Question 1(b) [4 marks]}\label{q1b}

\textbf{Capacitors of 10µF, 20µF and 30µF are connected in series and
supply of 200V DC is given. Find voltage across each capacitor.}

\begin{solutionbox}

For series-connected capacitors:

\begin{enumerate}
\tightlist
\item
  Find equivalent capacitance: 1/Ceq = 1/C_{1} + 1/C_{2} + 1/C_{3}
\item
  Voltage division: VC = (C_{1}/C) \times V
\end{enumerate}

\textbf{Calculation:} 1/Ceq = 1/10 + 1/20 + 1/30 = 0.1 + 0.05 + 0.033 =
0.183 Ceq = 5.46 μF

{\def\LTcaptype{none} % do not increment counter
\begin{longtable}[]{@{}llll@{}}
\toprule\noalign{}
Capacitor & Formula & Calculation & Voltage \\
\midrule\noalign{}
\endhead
\bottomrule\noalign{}
\endlastfoot
C_{1} = 10μF & V_{1} = (Ceq/C_{1}) \times V & (5.46/10) \times 200 = 109.2V & 109.2V \\
C_{2} = 20μF & V_{2} = (Ceq/C_{2}) \times V & (5.46/20) \times 200 = 54.6V & 54.6V \\
C_{3} = 30μF & V_{3} = (Ceq/C_{3}) \times V & (5.46/30) \times 200 = 36.4V & 36.4V \\
\end{longtable}
}

\end{solutionbox}
\begin{mnemonicbox}
``Smaller Capacitors get Larger Voltages''

\end{mnemonicbox}
\subsection*{Question 1(c) [7 marks]}\label{q1c}

\textbf{Explain Node pair voltage method for graph theory.}

\begin{solutionbox}

Node pair voltage method is a systematic approach to analyze electrical
networks.

\textbf{Procedure:}

\begin{enumerate}
\tightlist
\item
  Select a reference node (ground)
\item
  Identify the node voltages (N-1 unknowns for N nodes)
\item
  Apply KCL at each non-reference node
\item
  Express branch currents in terms of node voltages
\item
  Solve the equations for node voltages
\end{enumerate}

\textbf{Diagram:}

\begin{center}
\textbf{Mermaid Diagram (Code)}
\begin{verbatim}
{Shaded}
{Highlighting}[]
graph LR
    A[Select reference node] {-{-}{} B[Identify node voltages]}
    B {-{-}{} C[Apply KCL at each node]}
    C {-{-}{} D[Express branch currents using node voltages]}
    D {-{-}{} E[Solve equations for node voltages]}
    E {-{-}{} F[Calculate branch currents]}
{Highlighting}
{Shaded}
\end{verbatim}
\end{center}

\textbf{Key advantages:}

\begin{itemize}
\tightlist
\item
  \textbf{Fewer equations}: Only (n-1) equations for n nodes
\item
  \textbf{Computational efficiency}: Reduces system complexity
\item
  \textbf{Direct voltage solutions}: Provides node voltages directly
\item
  \textbf{Systematic approach}: Works for any network topology
\end{itemize}

\end{solutionbox}
\begin{mnemonicbox}
``GARCS: Ground, Assign voltages, Relate with KCL,
Calculate currents, Solve equations''

\end{mnemonicbox}
\subsection*{Question 1(c) OR [7
marks]}\label{q1c}

\textbf{Explain voltage division method with necessary equations.}

\begin{solutionbox}

Voltage division is a method to calculate how voltage distributes across
series components.

\textbf{Principle:} In a series circuit, voltage divides proportionally
to component resistances/impedances.

\textbf{Formula:} For a resistor R_{1} in a series circuit with total
resistance RT: V_{1} = (R_{1}/RT) \times VS

\textbf{Diagram:}

\begin{verbatim}
      +{-{-}{-}+}
VS {-{-}{-}|   |{-}{-} R1 {-}{-}|}
      +{-{-}{-}+        |}
                   |  V1
                   |
                   |
      +{-{-}{-}+        |}
      |   |{-{-} R2 {-}{-}|}
      +{-{-}{-}+}
        |
        |
      {-{-}{-}{-}{-}}
       {-{-}{-}}
        {-}
\end{verbatim}

\textbf{Mathematical explanation:}

\begin{itemize}
\tightlist
\item
  For resistors: V_{1} = (R_{1}/RT) \times VS
\item
  For capacitors: V_{1} = (1/C_{1})/(1/CT) \times VS = (CT/C_{1}) \times VS
\item
  For inductors: V_{1} = (L_{1}/LT) \times VS
\item
  For complex impedances: V_{1} = (Z_{1}/ZT) \times VS
\end{itemize}

\textbf{Examples:}

\begin{enumerate}
\tightlist
\item
  Voltage across a 1kΩ resistor in series with 4kΩ with 5V source =
  (1/5)\times5V = 1V
\item
  Voltage across a 10μF capacitor in series with 40μF with 10V source =
  (1/10)/(1/8)\times10V = 8V
\end{enumerate}

\end{solutionbox}
\begin{mnemonicbox}
``The BIGGER the RESISTANCE, the BIGGER the VOLTAGE
drop''

\end{mnemonicbox}
\subsection*{Question 2(a) [3 marks]}\label{q2a}

\textbf{Write open circuit impedance parameters of Two port network.}

\begin{solutionbox}

\textbf{Open Circuit Impedance Parameters:}

{\def\LTcaptype{none} % do not increment counter
\begin{longtable}[]{@{}
  >{\raggedright\arraybackslash}p{(\linewidth - 4\tabcolsep) * \real{0.2821}}
  >{\raggedright\arraybackslash}p{(\linewidth - 4\tabcolsep) * \real{0.2564}}
  >{\raggedright\arraybackslash}p{(\linewidth - 4\tabcolsep) * \real{0.4615}}@{}}
\toprule\noalign{}
\begin{minipage}[b]{\linewidth}\raggedright
Parameter
\end{minipage} & \begin{minipage}[b]{\linewidth}\raggedright
Equation
\end{minipage} & \begin{minipage}[b]{\linewidth}\raggedright
Physical Meaning
\end{minipage} \\
\midrule\noalign{}
\endhead
\bottomrule\noalign{}
\endlastfoot
\textbf{Z_{1}_{1}} & Z_{1}_{1} = V_{1}/I_{1} (when I_{2}=0) & Input impedance with output
open-circuited \\
\textbf{Z_{1}_{2}} & Z_{1}_{2} = V_{1}/I_{2} (when I_{1}=0) & Transfer impedance from port 2
to port 1 \\
\textbf{Z_{2}_{1}} & Z_{2}_{1} = V_{2}/I_{1} (when I_{2}=0) & Transfer impedance from port 1
to port 2 \\
\textbf{Z_{2}_{2}} & Z_{2}_{2} = V_{2}/I_{2} (when I_{1}=0) & Output impedance with input
open-circuited \\
\end{longtable}
}

\end{solutionbox}
\begin{mnemonicbox}
``ZIPO: Z-parameters with Inputs and outputs, Ports
Open where needed''

\end{mnemonicbox}
\subsection*{Question 2(b) [4 marks]}\label{q2b}

\textbf{Derive conversion from T-type network to \prod-type network.}

\begin{solutionbox}

\textbf{T to \prod Network Conversion:}

\textbf{Diagram:}

\begin{verbatim}
   T{-Network           -Network}
      Z1                   Y1
  o{-{-}{-}///{-}{-}{-}o       o{-}{-}{-}///{-}{-}{-}o}
  |           |       |           |
  |           |       |           |
 Z3          Z2      Y3          Y2
  |           |       |           |
  |           |       |           |
  o{-{-}{-}{-}{-}{-}{-}{-}{-}{-}{-}o       o{-}{-}{-}{-}{-}{-}{-}{-}{-}{-}{-}o}
\end{verbatim}

\textbf{Conversion Equations:}

{\def\LTcaptype{none} % do not increment counter
\begin{longtable}[]{@{}
  >{\raggedright\arraybackslash}p{(\linewidth - 4\tabcolsep) * \real{0.2955}}
  >{\raggedright\arraybackslash}p{(\linewidth - 4\tabcolsep) * \real{0.2045}}
  >{\raggedright\arraybackslash}p{(\linewidth - 4\tabcolsep) * \real{0.5000}}@{}}
\toprule\noalign{}
\begin{minipage}[b]{\linewidth}\raggedright
\prod-Parameter
\end{minipage} & \begin{minipage}[b]{\linewidth}\raggedright
Formula
\end{minipage} & \begin{minipage}[b]{\linewidth}\raggedright
Based on T-Parameters
\end{minipage} \\
\midrule\noalign{}
\endhead
\bottomrule\noalign{}
\endlastfoot
Y_{1} = 1/Z_{1} & Y_{1} = Z_{2}/(Z_{1}Z_{2}+Z_{2}Z_{3}+Z_{3}Z_{1}) & Reciprocal of Z_{1} modified by
network \\
Y_{2} = 1/Z_{2} & Y_{2} = Z_{1}/(Z_{1}Z_{2}+Z_{2}Z_{3}+Z_{3}Z_{1}) & Reciprocal of Z_{2} modified by
network \\
Y_{3} = 1/Z_{3} & Y_{3} = Z_{3}/(Z_{1}Z_{2}+Z_{2}Z_{3}+Z_{3}Z_{1}) & Reciprocal of Z_{3} modified by
network \\
\end{longtable}
}

\textbf{Derivation Steps:}

\begin{enumerate}
\tightlist
\item
  Define determinant Δ = Z_{1}Z_{2}+Z_{2}Z_{3}+Z_{3}Z_{1}
\item
  Use network theory to derive Y_{1} = Z_{2}/Δ
\item
  Similarly, Y_{2} = Z_{1}/Δ
\item
  And Y_{3} = Z_{3}/Δ
\end{enumerate}

\end{solutionbox}
\begin{mnemonicbox}
``Delta Divides: Y_{1} gets Z_{2}, Y_{2} gets Z_{1}, Y_{3} gets Z_{3}''

\end{mnemonicbox}
\subsection*{Question 2(c) [7 marks]}\label{q2c}

\textbf{Three resistances of 1, 1 and 1 ohms are connected in Delta.
Find equivalent resistances in star connection.}

\begin{solutionbox}

\textbf{Delta to Star Conversion:}

\textbf{Diagram:}

\begin{verbatim}
   Delta Network           Star Network
       R1                      ra
   o{-{-}{-}///{-}{-}o           o{-}{-}{-}///{-}{-}{-}o}
   |          |           |           |
   |          |           |           |
   |          |           |           |
   |          |          rb           rc
   |          |           |           |
  R3         R2           |           |
   |          |           |           |
   |          |           o{-{-}{-}{-}{-}{-}{-}{-}{-}{-}{-}o}
   o{-{-}{-}{-}{-}{-}{-}{-}{-}{-}o}
\end{verbatim}

\textbf{Conversion Formulas:}

\begin{itemize}
\tightlist
\item
  ra = (R_{1}\timesR_{3})/(R_{1}+R_{2}+R_{3})
\item
  rb = (R_{1}\timesR_{2})/(R_{1}+R_{2}+R_{3})
\item
  rc = (R_{2}\timesR_{3})/(R_{1}+R_{2}+R_{3})
\end{itemize}

\textbf{Calculation:} Given: R_{1} = R_{2} = R_{3} = 1Ω Sum of resistances:
R_{1}+R_{2}+R_{3} = 3Ω

{\def\LTcaptype{none} % do not increment counter
\begin{longtable}[]{@{}llll@{}}
\toprule\noalign{}
Star Resistor & Formula & Calculation & Result \\
\midrule\noalign{}
\endhead
\bottomrule\noalign{}
\endlastfoot
ra & (R_{1}\timesR_{3})/(R_{1}+R_{2}+R_{3}) & (1\times1)/3 & 0.333Ω \\
rb & (R_{1}\timesR_{2})/(R_{1}+R_{2}+R_{3}) & (1\times1)/3 & 0.333Ω \\
rc & (R_{2}\timesR_{3})/(R_{1}+R_{2}+R_{3}) & (1\times1)/3 & 0.333Ω \\
\end{longtable}
}

\end{solutionbox}
\begin{mnemonicbox}
``Product Over Sum: Each star arm gets the product of
adjacent delta sides divided by the sum of all''

\end{mnemonicbox}
\subsection*{Question 2(a) OR [3
marks]}\label{q2a}

\textbf{Define. (i) Transfer Impedance (ii) Image Impedance (iii)
Driving point Impedance}

\begin{solutionbox}

{\def\LTcaptype{none} % do not increment counter
\begin{longtable}[]{@{}
  >{\raggedright\arraybackslash}p{(\linewidth - 2\tabcolsep) * \real{0.3333}}
  >{\raggedright\arraybackslash}p{(\linewidth - 2\tabcolsep) * \real{0.6667}}@{}}
\toprule\noalign{}
\begin{minipage}[b]{\linewidth}\raggedright
Term
\end{minipage} & \begin{minipage}[b]{\linewidth}\raggedright
Definition
\end{minipage} \\
\midrule\noalign{}
\endhead
\bottomrule\noalign{}
\endlastfoot
\textbf{Transfer Impedance} & Ratio of output voltage at one port to
input current at another port when all other ports are open-circuited
(Z_{2}_{1} = V_{2}/I_{1} when I_{2}=0) \\
\textbf{Image Impedance} & Input impedance at port when the output port
is terminated with its own image impedance, creating infinite chain with
same impedance at all points \\
\textbf{Driving point Impedance} & Input impedance seen when looking
into a specified port or terminal pair (Z_{1}_{1} = V_{1}/I_{1} for port 1) \\
\end{longtable}
}

\end{solutionbox}
\begin{mnemonicbox}
``TID: Transfer relates ports, Image creates
reflections, Driving point looks inward''

\end{mnemonicbox}
\subsection*{Question 2(b) OR [4
marks]}\label{q2b}

\textbf{Get the equation for characteristics impedance Z for a standard
`T' network.}

\begin{solutionbox}

\textbf{Characteristic Impedance of `T' network:}

\textbf{Diagram:}

\begin{verbatim}
        Z1/2           Z1/2
    o{-{-}{-}///{-}{-}{-}{-}o{-}{-}{-}///{-}{-}{-}o}
    |            |           |
    |            |           |
    A           Z2           B
    |            |           |
    |            |           |
    o{-{-}{-}{-}{-}{-}{-}{-}{-}{-}{-}{-}o{-}{-}{-}{-}{-}{-}{-}{-}{-}{-}{-}o}
\end{verbatim}

\textbf{Derivation:} For a symmetrical T-network with series impedance
Z_{1} (split as Z_{1}/2 on each side) and shunt impedance Z_{2}:

Z_{0} = \sqrt(Z_{1}Z_{2} + Z_{1}^{2}/4)

\textbf{Steps:}

\begin{enumerate}
\tightlist
\item
  ABCD parameters for T-network:

  \begin{itemize}
  \tightlist
  \item
    A = 1 + Z_{1}/2Z_{2}
  \item
    B = Z_{1} + Z_{1}^{2}/4Z_{2}
  \item
    C = 1/Z_{2}
  \item
    D = 1 + Z_{1}/2Z_{2}
  \end{itemize}
\item
  From transmission line theory, Z_{0} = \sqrt(B/C)
\item
  Substituting: Z_{0} = \sqrt((Z_{1} + Z_{1}^{2}/4Z_{2})/(1/Z_{2}))
\item
  Simplifying: Z_{0} = \sqrt(Z_{1}Z_{2} + Z_{1}^{2}/4)
\end{enumerate}

\end{solutionbox}
\begin{mnemonicbox}
``Square root of Z-products plus quarter-square''

\end{mnemonicbox}
\subsection*{Question 2(c) OR [7
marks]}\label{q2c}

\textbf{Three resistances of 6, 15 and 10 ohms are connected in star.
Find equivalent resistances in delta connection.}

\begin{solutionbox}

\textbf{Star to Delta Conversion:}

\textbf{Diagram:}

\begin{verbatim}
   Star Network           Delta Network
       ra                      R1
   o{-{-}{-}///{-}{-}o           o{-}{-}{-}///{-}{-}{-}o}
   |          |           |           |
   |          |           |           |
   |          |           |           |
   |          |          R3          R2
  rb         rc           |           |
   |          |           |           |
   |          |           |           |
   o{-{-}{-}{-}{-}{-}{-}{-}{-}{-}o           o{-}{-}{-}{-}{-}{-}{-}{-}{-}{-}{-}o}
\end{verbatim}

\textbf{Conversion Formulas:}

\begin{itemize}
\tightlist
\item
  R_{1} = (ra\timesrb + rb\timesrc + rc\timesra)/ra
\item
  R_{2} = (ra\timesrb + rb\timesrc + rc\timesra)/rb
\item
  R_{3} = (ra\timesrb + rb\timesrc + rc\timesra)/rc
\end{itemize}

\textbf{Calculation:} Given: ra = 6Ω, rb = 15Ω, rc = 10Ω Sum of products
= (6\times15) + (15\times10) + (10\times6) = 90 + 150 + 60 = 300

{\def\LTcaptype{none} % do not increment counter
\begin{longtable}[]{@{}llll@{}}
\toprule\noalign{}
Delta Resistor & Formula & Calculation & Result \\
\midrule\noalign{}
\endhead
\bottomrule\noalign{}
\endlastfoot
R_{1} & (ra\timesrb + rb\timesrc + rc\timesra)/ra & 300/6 & 50Ω \\
R_{2} & (ra\timesrb + rb\timesrc + rc\timesra)/rb & 300/15 & 20Ω \\
R_{3} & (ra\timesrb + rb\timesrc + rc\timesra)/rc & 300/10 & 30Ω \\
\end{longtable}
}

\end{solutionbox}
\begin{mnemonicbox}
``Sum of Products Over Each: Delta side gets all
products divided by opposite star arm''

\end{mnemonicbox}
\subsection*{Question 3(a) [3 marks]}\label{q3a}

\textbf{Analyze the circuit (R1, R2 and R3 Connected in series with dc
supply) to calculate loop current using KVL.}

\begin{solutionbox}

\textbf{KVL for Series Circuit:}

\textbf{Diagram:}

\begin{verbatim}
      +{-{-}{-}+}
VS {-{-}{-}|   |{-}{-} R1 {-}{-}+{-}{-} R2 {-}{-}+{-}{-} R3 {-}{-}+}
      +{-{-}{-}+        |        |        |}
                   |        |        |
                   |        |        |
                  I|       I|       I|
                   |        |        |
                   |        |        |
                   +{-{-}{-}{-}{-}{-}{-}{-}+{-}{-}{-}{-}{-}{-}{-}{-}+}
                          {-{-}{-}{-}{-}}
                           {-{-}{-}}
                            {-}
\end{verbatim}

\textbf{KVL Equation:} VS - IR_{1} - IR_{2} - IR_{3} = 0 \textbf{Loop Current:} I
= VS/(R_{1} + R_{2} + R_{3})

\textbf{Steps:}

\begin{enumerate}
\tightlist
\item
  Identify all elements in the loop: VS, R_{1}, R_{2}, R_{3}
\item
  Apply KVL: Sum of voltage rises = Sum of voltage drops
\item
Solve for I:

I = VS/RT where RT = R_{1} + R_{2} + R_{3}

\end{enumerate}

\end{solutionbox}
\begin{mnemonicbox}
``KVL: Kirchhoff's Voltage Loop requires total
resistance''

\end{mnemonicbox}
\subsection*{Question 3(b) [4 marks]}\label{q3b}

\textbf{State Norton's theorem}

\begin{solutionbox}

\textbf{Norton's Theorem:}

Any linear electrical network consisting of voltage sources, current
sources, and resistances can be replaced by an equivalent circuit
consisting of a current source IN in parallel with a resistance RN.

\textbf{Diagram:}

\begin{verbatim}
    Original Network          Norton Equivalent
         +{-{-}{-}{-}{-}+                    +{-}{-}{-}{-}{-}+}
         |     |                    |     |
         |  A  |                    | IN  |
         |     |                    |     |
         +{-{-}+{-}{-}+                    +{-}{-}+{-}{-}+}
            |                          |
      +{-{-}{-}{-}{-}+{-}{-}{-}{-}{-}+              +{-}{-}{-}{-}{-}+{-}{-}{-}{-}{-}+}
      |     |     |              |     |     |
      |     Z     |       ={     |    RN     |}
      |     |     |              |     |     |
      +{-{-}{-}{-}{-}+{-}{-}{-}{-}{-}+              +{-}{-}{-}{-}{-}+{-}{-}{-}{-}{-}+}
            |                          |
         +{-{-}+{-}{-}+                    +{-}{-}+{-}{-}+}
         |     |                    |     |
         |  B  |                    |     |
         |     |                    |     |
         +{-{-}{-}{-}{-}+                    +{-}{-}{-}{-}{-}+}
\end{verbatim}

\textbf{How to find Norton equivalent:}

\begin{enumerate}
\tightlist
\item
  \textbf{Norton Current (IN)}: Short-circuit current flowing through
  the load terminals
\item
  \textbf{Norton Resistance (RN)}: Input resistance seen at the
  terminals with all sources replaced by their internal resistances
\end{enumerate}

\end{solutionbox}
\begin{mnemonicbox}
``SCIP: Short-Circuit current In Parallel with
equivalent resistance''

\end{mnemonicbox}
\subsection*{Question 3(c) [7 marks]}\label{q3c}

\textbf{Explain the steps to calculate the current in any branch of the
ckt using superposition theorem}

\begin{solutionbox}

\textbf{Superposition Theorem Application:}

\textbf{Principle:} In a linear circuit with multiple sources, the
response in any element equals the sum of responses caused by each
source acting alone.

\textbf{Steps:}

\begin{enumerate}
\tightlist
\item
  Consider only one source at a time
\item
  Replace other voltage sources with short circuits
\item
  Replace other current sources with open circuits
\item
  Calculate partial current for each source
\item
  Add all partial currents (algebraically) for final current
\end{enumerate}

\textbf{Diagram:}

\begin{verbatim}
flowchart LR
    A[Select one source] {-{-} B[Replace other sources]}
    B {-{-} C[Calculate partial current]}
    C {-{-} D[Repeat for all sources]}
    D {-{-} E[Sum partial currents]}
\end{verbatim}

\textbf{Mathematical Expression:} I = I_{1} + I_{2} + I_{3} + \ldots{} + In where
I_{1}, I_{2}, etc. are partial currents due to individual sources

\textbf{Example calculation:} For a branch with current contributions:
I_{1} = 2A (from source 1) I_{2} = -1A (from source 2) I_{3} = 0.5A (from source
3) Total current = 2A + (-1A) + 0.5A = 1.5A

\end{solutionbox}
\begin{mnemonicbox}
``OSACI: One Source Active, Calculate and Integrate''

\end{mnemonicbox}
\subsection*{Question 3(a) OR [3
marks]}\label{q3a}

\textbf{Analyze the circuit (R1, R2 and R3 Connected in parallel with dc
supply) to calculate node voltage using KCL.}

\begin{solutionbox}

\textbf{KCL for Parallel Circuit:}

\textbf{Diagram:}

\begin{verbatim}
                 I1
      +{-{-}{-}+    +{-}{-}{-}+}
VS {-{-}{-}|   |{-}{-}{-}{-}| R1|{-}{-}{-}{-}+}
      +{-{-}{-}+    +{-}{-}{-}+    |}
                        |
                 I2     |
                +{-{-}{-}+   |}
                | R2|{-{-}{-}+{-}{-}{-} V (Node)}
                +{-{-}{-}+   |}
                        |
                 I3     |
                +{-{-}{-}+   |}
                | R3|{-{-}{-}+}
                +{-{-}{-}+   |}
                        |
                       {-{-}{-}}
                        {-}
\end{verbatim}

\textbf{KCL Equation:} I_{1} + I_{2} + I_{3} = 0 \textbf{Node Voltage:} V = VS
(because parallel elements have same voltage)

\textbf{Steps:}

\begin{enumerate}
\tightlist
\item
  Identify node voltage V
\item
  Express branch currents: I_{1} = V/R_{1}, I_{2} = V/R_{2}, I_{3} = V/R_{3}
\item
  Apply KCL: V/R_{1} + V/R_{2} + V/R_{3} = VS/RT where 1/RT = 1/R_{1} + 1/R_{2} + 1/R_{3}
\end{enumerate}

\end{solutionbox}
\begin{mnemonicbox}
``KCL: Kirchhoff's Current Law means parallel voltage
equals source''

\end{mnemonicbox}
\subsection*{Question 3(b) OR [4
marks]}\label{q3b}

\textbf{State Maximum power transfer theorem.}

\begin{solutionbox}

\textbf{Maximum Power Transfer Theorem:}

For a source with internal resistance, maximum power is transferred to
the load when the load resistance equals the source's internal
resistance.

\textbf{Diagram:}

\begin{verbatim}
    +{-{-}{-}+     Rsource      +{-}{-}{-}+}
    |   |{-{-}{-}{-}////{-}{-}{-}{-}{-}{-}|   |}
    | V |                  | RL|
    |   |                  |   |
    +{-{-}{-}+                  +{-}{-}{-}+}
     {-{-}{-}                    {-}{-}{-}}
      {-                      {-}}
\end{verbatim}

\textbf{Mathematical expression:}

\begin{itemize}
\tightlist
\item
  Maximum power transfer occurs when RL = Rsource
\item
  Maximum power: Pmax = V^{2}/(4\timesRsource)
\end{itemize}

\textbf{Key points:}

\begin{itemize}
\tightlist
\item
  \textbf{Efficiency}: Only 50\% at maximum power transfer
\item
  \textbf{AC Circuits}: Load impedance must be complex conjugate of
  source impedance
\item
  \textbf{Applications}: Signal transmission, audio systems, RF circuits
\end{itemize}

\end{solutionbox}
\begin{mnemonicbox}
``MEET: Maximum Efficiency Equals when
Thevenin-matched''

\end{mnemonicbox}
\subsection*{Question 3(c) OR [7
marks]}\label{q3c}

\textbf{Explain the steps to calculate Vth, Rth and load current in the
ckt using Thevenin's theorem}

\begin{solutionbox}

\textbf{Thevenin's Theorem Application:}

\textbf{Principle:} Any linear electrical network with voltage and
current sources can be replaced by an equivalent circuit with a single
voltage source Vth and a series resistance Rth.

\textbf{Steps:}

\begin{enumerate}
\tightlist
\item
  Remove the load resistance from the circuit
\item
  Calculate open-circuit voltage (Vth) across the load terminals
\item
  Replace all sources with their internal resistances (voltage sources
  as short circuits, current sources as open circuits)
\item
  Calculate equivalent resistance (Rth) seen from the load terminals
\item
  Draw the Thevenin equivalent circuit with Vth and Rth
\item
  Reconnect the load and calculate load current: IL = Vth/(Rth + RL)
\end{enumerate}

\textbf{Diagram:}

\begin{verbatim}
flowchart LR
    A[Remove load] {-{-} B[Find Vth]}
    B {-{-} C[Replace sources with internal resistances]}
    C {-{-} D[Calculate Rth]}
    D {-{-} E[Draw Thevenin equivalent]}
    E {-{-} F[Reconnect load and calculate IL]}
\end{verbatim}

\textbf{Example calculation:}

\begin{itemize}
\tightlist
\item
  If Vth = 12V
\item
  Rth = 3Ω
\item
  RL = 6Ω
\item
  Then IL = 12V/(3Ω + 6Ω) = 12V/9Ω = 1.33A
\end{itemize}

\end{solutionbox}
\begin{mnemonicbox}
``VORTE: Voltage Open, Resistance with sources
Transformed, Equivalent circuit''

\end{mnemonicbox}
\subsection*{Question 4(a) [3 marks]}\label{q4a}

\textbf{Define resonance.}

\begin{solutionbox}

\textbf{Resonance:}

Resonance is a phenomenon in which a circuit responds with maximum
amplitude to an applied signal at a specific frequency called the
resonant frequency.

\textbf{Key characteristics:}

\begin{itemize}
\tightlist
\item
  Impedance becomes purely resistive
\item
  Inductive reactance equals capacitive reactance (XL = XC)
\item
  Voltage and current are in phase
\item
  Circuit stores and releases energy between L and C components
\end{itemize}

\textbf{Applications:}

\begin{itemize}
\tightlist
\item
  Tuning circuits
\item
  Filters
\item
  Oscillators
\item
  Wireless communications
\end{itemize}

\end{solutionbox}
\begin{mnemonicbox}
``MAX-IN-PHASE: Maximum response when Inductive and
capacitive reactances are equal and PHASEs cancel''

\end{mnemonicbox}
\subsection*{Question 4(b) [4 marks]}\label{q4b}

\textbf{Derive an equation for Quality factor of coil.}

\begin{solutionbox}

\textbf{Quality Factor (Q) of a Coil:}

\textbf{Definition:} Q-factor is the ratio of energy stored to energy
dissipated per cycle in a resonant circuit.

\textbf{Derivation:} For a coil with inductance L and resistance R:

\begin{enumerate}
\tightlist
\item
  Energy stored in inductor: WL = ½LI^{2}
\item
  Power dissipated in resistance: P = I^{2}R
\item
Time period:

T = 1/f = 2π/ω

\item
  Energy dissipated per cycle: Wd = P\timesT = I^{2}R\times(2π/ω)
\item
  Q = 2π(Energy stored/Energy dissipated per cycle)
\item
  Q = 2π(½LI^{2})/(I^{2}R\times2π/ω) = ωL/R
\end{enumerate}

\textbf{Final Equation:} Q = ωL/R = 2πfL/R

\textbf{Significance:}

\begin{itemize}
\tightlist
\item
  Higher Q indicates lower energy loss
\item
  Q increases with frequency
\item
  Q decreases with resistance
\end{itemize}

\end{solutionbox}
\begin{mnemonicbox}
``Omega-L over R gives Quality''

\end{mnemonicbox}
\subsection*{Question 4(c) [7 marks]}\label{q4c}

\textbf{An RLC series circuit has R=1 KΩ, L=100 mH and C=10µF. If a
voltage of 100 V is applied across series combination, determine: (i)
Resonance frequency (ii) `Q' factor}

\begin{solutionbox}

\textbf{RLC Series Circuit Analysis:}

\textbf{Diagram:}

\begin{verbatim}
         L=100mH
      +{-{-}{-}uuuu{-}{-}{-}+}
      |           |
      |           |
100V

R=1kΩ

C=10µF

      |           |
      |           |
      +{-{-}{-}{-}{-}{-}{-}{-}{-}{-}{-}+}
\end{verbatim}

\textbf{Calculations:}

\begin{enumerate}
\tightlist
\item
  \textbf{Resonance frequency:}
\end{enumerate}

\begin{itemize}
\tightlist
\item
  Formula: fr = 1/(2π\sqrt(LC))
\item
  fr = 1/(2π\sqrt(100\times10^{-}^{3} \times 10\times10^{-}^{6}))
\item
  fr = 1/(2π\sqrt(1\times10^{-}^{6}))
\item
  fr = 1/(2π \times 1\times10^{-}^{3})
\item
  fr = 159.15 Hz
\end{itemize}

\begin{enumerate}
\tightlist
\item
  \textbf{Quality factor (Q):}
\end{enumerate}

\begin{itemize}
\tightlist
\item
  Formula: Q = (1/R)\sqrt(L/C)
\item
  Q = (1/1000)\sqrt(100\times10^{-}^{3}/10\times10^{-}^{6})
\item
  Q = (1/1000)\sqrt(10^{4})
\item
  Q = (1/1000) \times 100
\item
  Q = 0.1
\end{itemize}

{\def\LTcaptype{none} % do not increment counter
\begin{longtable}[]{@{}llll@{}}
\toprule\noalign{}
Parameter & Formula & Calculation & Result \\
\midrule\noalign{}
\endhead
\bottomrule\noalign{}
\endlastfoot
Resonant frequency (fr) & 1/(2π\sqrt(LC)) & 1/(2π\sqrt(1\times10^{-}^{6})) & 159.15 Hz \\
Quality factor (Q) & (1/R)\sqrt(L/C) & (1/1000)\sqrt(10^{4}) & 0.1 \\
\end{longtable}
}

\end{solutionbox}
\begin{mnemonicbox}
``Frequency from LC, Quality from LCR''

\end{mnemonicbox}
\subsection*{Question 4(a) OR [3
marks]}\label{q4a}

\textbf{Define Mutual Inductance.}

\begin{solutionbox}

\textbf{Mutual Inductance:}

Mutual inductance is the property of a circuit whereby a change in
current in one coil induces a voltage in another coil due to the
magnetic coupling between them.

\textbf{Mathematical expression:}

\begin{itemize}
\tightlist
\item
  Voltage induced in coil 2: V_{2} = -M(dI_{1}/dt)
\item
  M = k\sqrt(L_{1}L_{2}) where k is the coupling coefficient (0\leqk\leq1)
\item
  Unit: Henry (H)
\end{itemize}

\textbf{Key properties:}

\begin{itemize}
\tightlist
\item
  Depends on coil geometry, distance and orientation
\item
  Proportional to both inductances
\item
  Basis for transformers and coupled circuits
\item
  Can be positive or negative based on mutual flux direction
\end{itemize}

\end{solutionbox}
\begin{mnemonicbox}
``MICK: Mutual Inductance links Coils through
K-coupling''

\end{mnemonicbox}
\subsection*{Question 4(b) OR [4
marks]}\label{q4b}

\textbf{Derive equation of coefficient of coupling}

\begin{solutionbox}

\textbf{Coefficient of Coupling (k):}

\textbf{Definition:} The coefficient of coupling (k) is a measure of the
magnetic coupling between two coils, ranging from 0 (no coupling) to 1
(perfect coupling).

\textbf{Derivation:}

\begin{enumerate}
\tightlist
\item
  Define mutual inductance: M = magnetic flux linkage / current
\item
  For two coils with self-inductances L_{1} and L_{2}:

  \begin{itemize}
  \tightlist
  \item
    Flux linkage in coil 1 due to current in coil 1: λ_{1}_{1} = L_{1}I_{1}
  \item
    Flux linkage in coil 2 due to current in coil 2: λ_{2}_{2} = L_{2}I_{2}
  \item
    Flux linkage in coil 2 due to current in coil 1: λ_{2}_{1} = MI_{1}
  \end{itemize}
\item
  The coupling coefficient k represents the fraction of flux from coil 1
  that links with coil 2
\item
  From electromagnetic theory: M = k\sqrt(L_{1}L_{2})
\item
  Rearranging: k = M/\sqrt(L_{1}L_{2})
\end{enumerate}

\textbf{Final Equation:} k = M/\sqrt(L_{1}L_{2})

\textbf{Key points:}

\begin{itemize}
\tightlist
\item
  k = 0: No magnetic coupling
\item
  0 \textless{} k \textless{} 1: Partial coupling
\item
  k = 1: Perfect coupling (all flux links both coils)
\end{itemize}

\end{solutionbox}
\begin{mnemonicbox}
``M divided by Geometric Mean of Ls''

\end{mnemonicbox}
\subsection*{Question 4(c) OR [7
marks]}\label{q4c}

\textbf{Derive resonance frequency of parallel resonance circuit.}

\begin{solutionbox}

\textbf{Parallel Resonance Frequency Derivation:}

\textbf{Diagram:}

\begin{verbatim}
              +{-{-}{-}+}
              |   |
           +{-{-}+{-}{-}{-}+{-}{-}+}
           |          |
           |          |
      L    |          |   C
    uuuuu  |          |  ||| 
           |          |  |||
           |          |  |||
           |          |  |||
           +{-{-}{-}{-}{-}{-}{-}{-}{-}{-}+{-}{-}{-}+}
              |   |
              +{-{-}{-}+}
               R
\end{verbatim}

\textbf{Derivation steps:}

\begin{enumerate}
\item
For a parallel RLC circuit, the admittance is:

Y = 1/Z = 1/R + 1/jωL +

  jωC
\item
  At resonance, the imaginary part becomes zero: Im(Y) = 0 1/jωL + jωC =
  0 -j/ωL + jωC = 0 1/ωL = ωC ω^{2}LC = 1
\item
  For the ideal case (with infinite resistance): ω_{0} = 1/\sqrt(LC) f_{0} =
  1/(2π\sqrt(LC))
\item
  For the real case (with resistance R): If R is in series with L, the
  resonant frequency becomes: f_{0} = (1/2π)\sqrt(1/LC - R^{2}/L^{2})
\end{enumerate}

\textbf{Final Equation:}

\begin{itemize}
\tightlist
\item
  Ideal case: f_{0} = 1/(2π\sqrt(LC))
\item
  Real case (R in series with L): f_{0} = (1/2π)\sqrt(1/LC - R^{2}/L^{2})
\end{itemize}

\textbf{Key characteristics of parallel resonance:}

\begin{itemize}
\tightlist
\item
  Maximum impedance at resonance
\item
  Minimum current drawn from source
\item
  Current circulates between L and C
\item
  Also called ``anti-resonance'' or ``rejector circuit''
\end{itemize}

\end{solutionbox}
\begin{mnemonicbox}
``ONE over LC SQRT: The frequency where parallel
paths balance''

\end{mnemonicbox}
\subsection*{Question 5(a) [3 marks]}\label{q5a}

\textbf{Classify various types of attenuators.}

\begin{solutionbox}

\textbf{Types of Attenuators:}

{\def\LTcaptype{none} % do not increment counter
\begin{longtable}[]{@{}
  >{\raggedright\arraybackslash}p{(\linewidth - 4\tabcolsep) * \real{0.1818}}
  >{\raggedright\arraybackslash}p{(\linewidth - 4\tabcolsep) * \real{0.3333}}
  >{\raggedright\arraybackslash}p{(\linewidth - 4\tabcolsep) * \real{0.4848}}@{}}
\toprule\noalign{}
\begin{minipage}[b]{\linewidth}\raggedright
Type
\end{minipage} & \begin{minipage}[b]{\linewidth}\raggedright
Structure
\end{minipage} & \begin{minipage}[b]{\linewidth}\raggedright
Characteristics
\end{minipage} \\
\midrule\noalign{}
\endhead
\bottomrule\noalign{}
\endlastfoot
\textbf{T-type} & Series-shunt-series & Symmetric, good for matching,
widely used \\
\textbf{\prod-type} & Shunt-series-shunt & Symmetric, alternative to
T-type \\
\textbf{Lattice} & Balanced bridge & Symmetrical, used in balanced
lines \\
\textbf{L-type} & Series-shunt & Asymmetric, simpler design \\
\textbf{Bridged-T} & T with bridged shunt & Better frequency response,
complex \\
\textbf{O-type} & Series-shunt-series-shunt & Improved rejection
characteristics \\
\end{longtable}
}

\end{solutionbox}
\begin{mnemonicbox}
``TL\prodBO: Top attenuators Let \prod signals Balance
Output''

\end{mnemonicbox}
\subsection*{Question 5(b) [4 marks]}\label{q5b}

\textbf{Derive relation between Decibel and Neper}

\begin{solutionbox}

\textbf{Decibel to Neper Conversion:}

\textbf{Definitions:}

\begin{itemize}
\tightlist
\item
  \textbf{Decibel (dB)}: Power ratio logarithm using base 10 (common
  logarithm)
\item
  \textbf{Neper (Np)}: Voltage/current ratio logarithm using base e
  (natural logarithm)
\end{itemize}

\textbf{Derivation:}

\begin{enumerate}
\tightlist
\item
  Power ratio in dB: Loss(dB) = 10 log_{1}_{0}(P_{1}/P_{2})
\item
  Voltage ratio in dB: Loss(dB) = 20 log_{1}_{0}(V_{1}/V_{2})
\item
  Voltage ratio in Nepers: Loss(Np) = ln(V_{1}/V_{2})
\item
  Converting between logarithm bases: log_{1}_{0}(x) = ln(x)/ln(10)
\item
  Substitute: Loss(dB) = 20 ln(V_{1}/V_{2})/ln(10) = 20 Loss(Np)/ln(10)
\end{enumerate}

\textbf{Final Relation:}

\begin{itemize}
\tightlist
\item
  1 Neper = ln(10)/20 \times 10 dB = 8.686 dB
\item
  1 dB = 0.115 Neper
\end{itemize}


{\def\LTcaptype{none} % do not increment counter
\begin{longtable}[]{@{}lll@{}}
\toprule\noalign{}
Conversion & Formula & Value \\
\midrule\noalign{}
\endhead
\bottomrule\noalign{}
\endlastfoot
Neper to dB & 1 Np = (20/ln10) dB & 1 Np = 8.686 dB \\
dB to Neper & 1 dB = (ln10/20) Np & 1 dB = 0.115 Np \\
\end{longtable}
}

\end{solutionbox}
\begin{mnemonicbox}
``8.686: Eight Point Six Nepers Buy Ten decibels''

\end{mnemonicbox}
\subsection*{Question 5(c) [7 marks]}\label{q5c}

\textbf{Design T type attenuator which provides 20 dB attenuation and
having characteristics Impedance of 600 ohm.}

\begin{solutionbox}

\textbf{T-Type Attenuator Design:}

\textbf{Diagram:}

\begin{verbatim}
       Z1/2         Z1/2
    o{-{-}{-}///{-}{-}{-}o{-}{-}{-}///{-}{-}{-}o}
    |            |           |
    |            |           |
   R0           Z2          R0
    |            |           |
    |            |           |
    o{-{-}{-}{-}{-}{-}{-}{-}{-}{-}{-}{-}o{-}{-}{-}{-}{-}{-}{-}{-}{-}{-}{-}o}
\end{verbatim}

\textbf{Design Steps:}

\begin{enumerate}
\item
Calculate attenuation ratio N from dB:

N = 10\^{}(dB/20) =

  10\^{}(20/20) = 10
\item
  Calculate R_{1} and R_{2} using formulas:

  \begin{itemize}
  \tightlist
  \item
    R_{1} = R_{0} \times [(N^{2} - 1)/(N^{2} + 1)]
  \item
    R_{2} = R_{0} \times [2N/(N^{2} - 1)]
  \end{itemize}
\end{enumerate}

\textbf{Calculation:}

Given:

\begin{itemize}
\tightlist
\item
  Attenuation = 20 dB
\item
  Characteristic impedance = 600 Ω
\end{itemize}

{\def\LTcaptype{none} % do not increment counter
\begin{longtable}[]{@{}llll@{}}
\toprule\noalign{}
Parameter & Formula & Calculation & Result \\
\midrule\noalign{}
\endhead
\bottomrule\noalign{}
\endlastfoot
N & 10\^{}(dB/20) & 10\^{}(20/20) & 10 \\
R_{1} & R_{0}[(N^{2} - 1)/(N^{2} + 1)] & 600[(10^{2} - 1)/(10^{2} + 1)] & 588.2
Ω \\
Z_{1}/2 & R_{1}/2 & 588.2/2 & 294.1 Ω \\
R_{2} & R_{0}[2N/(N^{2} - 1)] & 600[2\times10/(10^{2} - 1)] & 121.2 Ω \\
\end{longtable}
}

\textbf{Final T-network values:}

\begin{itemize}
\tightlist
\item
  Each series arm (Z_{1}/2): 294.1 Ω
\item
  Shunt arm (Z_{2}): 121.2 Ω
\end{itemize}

\end{solutionbox}
\begin{mnemonicbox}
``N-squared minus ONE over N-squared plus ONE for
series resistance''

\end{mnemonicbox}
\subsection*{Question 5(a) OR [3
marks]}\label{q5a}

\textbf{State limitations of constant K low pass filters}

\begin{solutionbox}

\textbf{Limitations of Constant-K Low Pass Filters:}

{\def\LTcaptype{none} % do not increment counter
\begin{longtable}[]{@{}
  >{\raggedright\arraybackslash}p{(\linewidth - 2\tabcolsep) * \real{0.4800}}
  >{\raggedright\arraybackslash}p{(\linewidth - 2\tabcolsep) * \real{0.5200}}@{}}
\toprule\noalign{}
\begin{minipage}[b]{\linewidth}\raggedright
Limitation
\end{minipage} & \begin{minipage}[b]{\linewidth}\raggedright
Description
\end{minipage} \\
\midrule\noalign{}
\endhead
\bottomrule\noalign{}
\endlastfoot
\textbf{Poor cutoff transition} & Gradual transition from pass band to
stop band instead of sharp cutoff \\
\textbf{Uneven impedance} & Impedance varies with frequency, causing
matching problems \\
\textbf{Attenuation ripple} & Non-uniform attenuation in both pass band
and stop band \\
\textbf{Phase distortion} & Non-linear phase response causing signal
distortion \\
\textbf{Fixed termination} & Designed for specific load impedance;
performance deteriorates with other loads \\
\textbf{Limited selectivity} & Poor selectivity compared to modern
filter designs \\
\end{longtable}
}

\end{solutionbox}
\begin{mnemonicbox}
``PUAPFL: Poor transition, Uneven impedance,
Attenuation ripple, Phase distortion, Fixed termination, Limited
selectivity''

\end{mnemonicbox}
\subsection*{Question 5(b) OR [4
marks]}\label{q5b}

\textbf{Give classification of filters showing frequency response curves
For each of them}

\begin{solutionbox}

\textbf{Classification of Filters:}

{\def\LTcaptype{none} % do not increment counter
\begin{longtable}[]{@{}lll@{}}
\toprule\noalign{}
Filter Type & Frequency Response Curve & Characteristics \\
\midrule\noalign{}
\endhead
\bottomrule\noalign{}
\endlastfoot
\textbf{Low Pass} & ```goat & \\
\end{longtable}
}

\begin{verbatim}
    |\\
    |  \\
    |    \\________
    |
    +---------------
       fc
``` | Passes frequencies below cutoff fc, blocks higher frequencies |
\end{verbatim}

\textbf{High Pass} \textbar{}
\texttt{goat\ \ \ \ \ \ \ \ \textbar{}\ \ \ \ \ \ \ \ \_\_\_\_\_\_\_\ \ \ \ \ \ \ \ \textbar{}\ \ \ \ \ \ /\ \ \ \ \ \ \ \ \textbar{}\ \ \ \ /\ \ \ \ \ \ \ \ \textbar{}\ \ /\ \ \ \ \ \ \ \ \textbar{}/\ \ \ \ \ \ \ \ +-\/-\/-\/-\/-\/-\/-\/-\/-\/-\/-\/-\/-\/-\/-\ \ \ \ \ \ \ \ \ \ \ fc}
\textbar{} Blocks frequencies below cutoff fc, passes higher frequencies
\textbar{}\\
\textbf{Band Pass} \textbar{}
\texttt{goat\ \ \ \ \ \ \ \ \textbar{}\ \ \ \ \ \ /\textbackslash{}\ \ \ \ \ \ \ \ \textbar{}\ \ \ \ \ /\ \ \textbackslash{}\ \ \ \ \ \ \ \ \textbar{}\ \ \ \ /\ \ \ \ \textbackslash{}\ \ \ \ \ \ \ \ \textbar{}\ \ \ /\ \ \ \ \ \ \textbackslash{}\ \ \ \ \ \ \ \ \textbar{}\_\_/\ \ \ \ \ \ \ \ \textbackslash{}\_\_\_\ \ \ \ \ \ \ \ +-\/-\/-\/-\/-\/-\/-\/-\/-\/-\/-\/-\/-\/-\/-\ \ \ \ \ \ \ \ \ \ \ f1\ \ \ \ f2}
\textbar{} Passes frequencies between f1 and f2, blocks others
\textbar{}\\
\textbf{Band Stop} \textbar{}
\texttt{goat\ \ \ \ \ \ \ \ \textbar{}\_\_\_\ \ \ \ \ \ \ \ \_\_\_\ \ \ \ \ \ \ \ \textbar{}\ \ \ \textbackslash{}\ \ \ \ \ \ /\ \ \ \ \ \ \ \ \textbar{}\ \ \ \ \textbackslash{}\ \ \ \ /\ \ \ \ \ \ \ \ \textbar{}\ \ \ \ \ \textbackslash{}\ \ /\ \ \ \ \ \ \ \ \textbar{}\ \ \ \ \ \ \textbackslash{}/\ \ \ \ \ \ \ \ +-\/-\/-\/-\/-\/-\/-\/-\/-\/-\/-\/-\/-\/-\/-\ \ \ \ \ \ \ \ \ \ \ f1\ \ \ \ f2}
\textbar{} Blocks frequencies between f1 and f2, passes others
\textbar{}

\end{solutionbox}
\begin{mnemonicbox}
``LHBS: Low lets low tones, High lets high tones,
Band-pass selects middle, Band-Stop rejects middle''

\end{mnemonicbox}
\subsection*{Question 5(c) OR [7
marks]}\label{q5c}

\textbf{Derive equation for designing a constant K low pass filters.}

\begin{solutionbox}

\textbf{Constant-K Low Pass Filter Design:}

\textbf{Diagram:}

\begin{verbatim}
    T{-section:              π{-}section:}
    
       L/2         L/2
    o{-{-}{-}uuu{-}{-}{-}{-}o{-}{-}{-}{-}uuu{-}{-}{-}o     o{-}{-}{-}{-}{-}{-}{-}{-}{-}{-}{-}o}
    |          |          |     |           |
    |          |          |     |           |
    |          C          |     C/2        C/2
    |          |          |     |           |
    |          |          |     |           |
    o{-{-}{-}{-}{-}{-}{-}{-}{-}{-}o{-}{-}{-}{-}{-}{-}{-}{-}{-}{-}o     o{-}{-}{-}{-}{-}uuu{-}{-}{-}o}
                                      L
\end{verbatim}

\textbf{Design Theory:} A constant-K filter has impedance product Z_{1}Z_{2} =
k^{2} (constant) at all frequencies.

\textbf{Derivation Steps:}

\begin{enumerate}
\tightlist
\item
  For a T-section low-pass filter:

  \begin{itemize}
  \tightlist
  \item
    Series impedance Z_{1} = jωL
  \item
    Shunt impedance Z_{2} = 1/jωC
  \end{itemize}
\item
  Product Z_{1}Z_{2} must be constant:

  \begin{itemize}
  \tightlist
  \item
    Z_{1}Z_{2} = jωL \times 1/jωC = L/C = k^{2}
  \end{itemize}
\item
  Characteristic impedance at zero frequency:

  \begin{itemize}
  \tightlist
  \item
    R_{0} = \sqrt(L/C)
  \end{itemize}
\item
  Cut-off frequency occurs when:

  \begin{itemize}
  \tightlist
  \item
Z_{1} = 2Z_{0} at

ω = ωc

  \item
    jωcL = 2R_{0} = 2\sqrt(L/C)
  \item
    ωc^{2} = 4/LC
  \item
    ωc = 2/\sqrt(LC)
  \item
    fc = 1/π\sqrt(LC)
  \end{itemize}
\item
  Design equations:

  \begin{itemize}
  \tightlist
  \item
    L = R_{0}/πfc
  \item
    C = 1/(πfcR_{0})
  \end{itemize}
\end{enumerate}

\textbf{Final Equations:}

\begin{itemize}
\tightlist
\item
  Cut-off frequency: fc = 1/π\sqrt(LC)
\item
  Inductance: L = R_{0}/πfc
\item
  Capacitance: C = 1/(πfcR_{0})
\end{itemize}

\textbf{T-section values:}

\begin{itemize}
\tightlist
\item
  Series inductance: L/2 each arm
\item
  Shunt capacitance: C
\end{itemize}

\textbf{π-section values:}

\begin{itemize}
\tightlist
\item
  Series inductance: L
\item
  Shunt capacitance: C/2 each arm
\end{itemize}

\end{solutionbox}
\begin{mnemonicbox}
``One over Pi-Root-LC: The frequency where we Cut''

\end{mnemonicbox}

\end{document}
