\documentclass[10pt,a4paper]{article}

% content/resources/templates/preamble.tex
\usepackage[margin=0.6in]{geometry}
\author{Milav Dabgar}
\usepackage{amsmath,amssymb,amsthm}
\usepackage{booktabs}
\usepackage{multirow}
\usepackage{xcolor}
\usepackage{tcolorbox}
\tcbuselibrary{breakable,skins}
\usepackage[colorlinks=true,linkcolor=blue]{hyperref}
\usepackage{titlesec}
\usepackage{enumitem}
\usepackage{tikz}
\usepackage{pgfplots}
\usepackage{circuitikz}
\usepackage[version=4]{mhchem}
\usepackage{longtable}
\usepackage{array}
\usepackage{float}
\usepackage{caption}
\usepackage{listings}

\lstset{
  basicstyle=\small\ttfamily,
  breaklines=true,
  breakatwhitespace=false,
  postbreak=\mbox{\textcolor{red}{$\hookrightarrow$}\space},
  float=false,
  numbers=left,
  numberstyle=\tiny\color{gray},
  numbersep=10pt,
  xleftmargin=2em,
  keywordstyle=\color{blue},
  commentstyle=\color{green!60!black},
  stringstyle=\color{purple},
  backgroundcolor=\color{gray!5},
  showstringspaces=false,
  tabsize=2,
  captionpos=b,
  keepspaces=true,
  columns=flexible
}

\pgfplotsset{compat=1.18}
\usetikzlibrary{shapes,arrows,positioning,calc,patterns,decorations.pathmorphing,decorations.markings,arrows.meta}

% Color scheme
\definecolor{headcolor}{RGB}{0,102,204}
\definecolor{keycolor}{RGB}{220,20,60}
\definecolor{solutioncolor}{RGB}{34,139,34}
\definecolor{mnemoniccolor}{RGB}{148,0,211}
\definecolor{codecolor}{RGB}{0,0,100}

% Spacing
\setlength{\parskip}{3pt}
\setlist[itemize]{nosep}
\setlist[enumerate]{nosep}

% Title formatting
\titleformat{\section}{\Large\bfseries\color{headcolor}}{\thesection}{1em}{}
\titleformat{\subsection}{\large\bfseries\color{headcolor}}{\thesubsection}{1em}{}

% Pandoc tightlist compatibility
\providecommand{\tightlist}{%
  \setlength{\itemsep}{0pt}\setlength{\parskip}{0pt}}

% Pandoc longtable compatibility
\newcounter{none}
\def\thenone{}


% content/resources/templates/gujarati-boxes.tex
\usepackage{fontspec}
\usepackage{polyglossia}

% Set Gujarati as main language (document is primarily in Gujarati)
% Note: gloss-gujarati.ldf doesn't exist in polyglossia, but it will use hyphenation patterns
\setdefaultlanguage{gujarati}
\setotherlanguage{english}

% Configure Gujarati font properly
% Use Language=Default to prevent polyglossia from trying to add language-specific features
% that don't exist for Gujarati, which causes "empty feature" warnings
\newfontfamily\gujaratifont[Script=Gujarati,AutoFakeBold=2.5,AutoFakeSlant=0.3]{Noto Sans Gujarati}
\setmainfont[Script=Gujarati,AutoFakeBold=2.5,AutoFakeSlant=0.3]{Noto Sans Gujarati}
% Use Noto Sans Gujarati for monospace to support Gujarati in text
\setmonofont[Scale=0.9]{Noto Sans Gujarati}

% Configure English to use the same font
\newfontfamily\englishfont[Script=Gujarati,AutoFakeBold=2.5,AutoFakeSlant=0.3]{Noto Sans Gujarati}

% Translations for polyglossia
\gappto\captionsgujarati{
  \renewcommand{\tablename}{કોષ્ટક}
  \renewcommand{\figurename}{આકૃતિ}
}

% Helper for TikZ nodes to ensure Gujarati font
\newcommand{\gu}[1]{{\gujaratifont #1}}

% Custom environments
\newtcolorbox{solutionbox}{
    breakable,
    enhanced,
    colback=solutioncolor!5!white,
    colframe=solutioncolor!75!black,
    fonttitle=\bfseries,
    title=જવાબ
}

\newtcolorbox{solutionboxnobreak}{
 colback=solutioncolor!5!white,
 colframe=solutioncolor!75!black,
 fonttitle=\bfseries,
 title=જવાબ
}

\newtcolorbox{keyformula}{
 breakable,
 enhanced,
 colback=keycolor!5!white,
 colframe=keycolor!75!black,
 fonttitle=\bfseries,
 title=રાસાયણિક સમીકરણ/સૂત્ર
}

\newtcolorbox{mnemonicbox}{
 breakable,
 enhanced,
 colback=mnemoniccolor!5!white,
 colframe=mnemoniccolor!75!black,
 fonttitle=\bfseries,
 title=મેમરી ટ્રીક
}


\begin{document}

\begin{center}
{\Huge\bfseries\color{headcolor} Subject Name (Gujarati)}\\[5pt]
{\LARGE 4331101 -- Summer 2024}\\[3pt]
{\large Semester 1 Study Material}\\[3pt]
{\normalsize\textit{Detailed Solutions and Explanations}}
\end{center}

\vspace{10pt}

\subsection*{પ્રશ્ન 1(અ) [3
માર્ક્સ]}\label{uxaaauxab0uxab6uxaa8-1uxa85-3-uxaaeuxab0uxa95uxab8}

\textbf{યોગ્ય આકૃતિ સાથે નોડ, બ્રાન્ચ અને લૂપ વ્યાખ્યાયિત કરો.}

\begin{solutionbox}

\textbf{આકૃતિ:}

\begin{center}
\textbf{Mermaid Diagram (Code)}
\begin{verbatim}
{Shaded}
{Highlighting}[]
graph LR
    A((Node A)) {-{-}{-} B\{Branch 1\}}
    A {-{-}{-} C\{Branch 2\}}
    A {-{-}{-} D\{Branch 3\}}
    B {-{-}{-} E((Node B))}
    C {-{-}{-} F((Node C))}
    D {-{-}{-} G((Node D))}
    E {-{-}{-} H\{Branch 4\}}
    H {-{-}{-} F}
    G {-{-}{-} I\{Branch 5\}}
    I {-{-}{-} F}

    subgraph Loop X
    A {-{-}{} B {-}{-}{} E {-}{-}{} H {-}{-}{} F {-}{-}{} C {-}{-}{} A}
    end
{Highlighting}
{Shaded}
\end{verbatim}
\end{center}

\begin{itemize}
\tightlist
\item
  \textbf{નોડ}: એક બિંદુ જ્યાં બે કે વધુ સર્કિટ તત્વો એકબીજા સાથે જોડાય છે
\item
  \textbf{બ્રાન્ચ}: બે નોડ્સને જોડતું એક સિંગલ એલિમેન્ટ
\item
  \textbf{લૂપ}: સર્કિટમાં કોઈપણ બંધ પાથ જ્યાં કોઈ નોડ એક કરતાં વધુ વખત આવતો નથી
\end{itemize}

\end{solutionbox}
\begin{mnemonicbox}
``NBA સર્કિટ'' - Nodes જંક્શનો છે, Branches રસ્તાઓ છે,
Loops વૈકલ્પિક માર્ગો છે

\end{mnemonicbox}
\subsection*{પ્રશ્ન 1(બ) [4
માર્ક્સ]}\label{uxaaauxab0uxab6uxaa8-1uxaac-4-uxaaeuxab0uxa95uxab8}

\textbf{નેટવર્ક માટે ``ટ્રી'' અને ``ગ્રાફ'' સમજાવો.}

\begin{solutionbox}

\textbf{આકૃતિ:}

\begin{center}
\textbf{Mermaid Diagram (Code)}
\begin{verbatim}
{Shaded}
{Highlighting}[]
graph TD
    subgraph Network Graph
    direction LR    
    A((A)) {-{-}{-} B((B))}
    A {-{-}{-} C((C))}
    B {-{-}{-} D((D))}
    C {-{-}{-} D}
    B {-{-}{-} C}
    end

    subgraph Tree of Network
    direction LR    
    E((A)) {-{-}{-} F((B))}
    E {-{-}{-} G((C))}
    F {-{-}{-} H((D))}
    end
{Highlighting}
{Shaded}
\end{verbatim}
\end{center}

{\def\LTcaptype{none} % do not increment counter
\begin{longtable}[]{@{}
  >{\raggedright\arraybackslash}p{(\linewidth - 4\tabcolsep) * \real{0.4091}}
  >{\raggedright\arraybackslash}p{(\linewidth - 4\tabcolsep) * \real{0.3182}}
  >{\raggedright\arraybackslash}p{(\linewidth - 4\tabcolsep) * \real{0.2727}}@{}}
\toprule\noalign{}
\begin{minipage}[b]{\linewidth}\raggedright
લક્ષણ
\end{minipage} & \begin{minipage}[b]{\linewidth}\raggedright
ગ્રાફ
\end{minipage} & \begin{minipage}[b]{\linewidth}\raggedright
ટ્રી
\end{minipage} \\
\midrule\noalign{}
\endhead
\bottomrule\noalign{}
\endlastfoot
\textbf{વ્યાખ્યા} & નેટવર્કનું સંપૂર્ણ ટોપોલોજિકલ રજૂઆત & કનેક્ટેડ સબગ્રાફ જેમાં બધા
નોડ્સ હોય પણ લૂપ ન હોય \\
\textbf{તત્વો} & બધી બ્રાન્ચ અને નોડ્સ ધરાવે છે & N-1 બ્રાન્ચ ધરાવે છે જ્યાં N નોડ્સની
સંખ્યા છે \\
\textbf{લૂપ્સ} & લૂપ્સ ધરાવે છે & કોઈ લૂપ્સ નથી \\
\textbf{ઉપયોગ} & સંપૂર્ણ સર્કિટ એનાલિસિસ માટે વપરાય છે & નેટવર્ક ગણતરીઓને સરળ
બનાવવા માટે વપરાય છે \\
\end{longtable}
}

\end{solutionbox}
\begin{mnemonicbox}
``GRAND Tree'' - Graph માં Routes And Nodes with
Detours છે, Tree માં ફક્ત સિંગલ Routes છે

\end{mnemonicbox}
\subsection*{પ્રશ્ન 1(ક) [7
માર્ક્સ]}\label{uxaaauxab0uxab6uxaa8-1uxa95-7-uxaaeuxab0uxa95uxab8}

\textbf{યોગ્ય આકૃતિનો ઉપયોગ કરી ``મેષ કરંટ મેથડ'' સમજાવો.}

\begin{solutionbox}

\textbf{આકૃતિ:}

\begin{center}
\textbf{Mermaid Diagram (Code)}
\begin{verbatim}
{Shaded}
{Highlighting}[]
graph LR
    subgraph Mesh 1
    A(({+)) {-}{-} R1 {-}{-}{} B(({}+))}
    B {-{-} R3 {-}{-}{} C(({}+))}
    C {-{-} R5 {-}{-}{} A}
    end

    subgraph Mesh 2
    B {-{-} R2 {-}{-}{} D(({}+))}
    D {-{-} R4 {-}{-}{} C}
    C {-{-} R3 {-}{-}{} B}
    end
    
    style Mesh 1 fill:\#f9f,stroke:\#333,stroke{-width:2px}
    style Mesh 2 fill:\#bbf,stroke:\#333,stroke{-width:2px}
{Highlighting}
{Shaded}
\end{verbatim}
\end{center}

{\def\LTcaptype{none} % do not increment counter
\begin{longtable}[]{@{}ll@{}}
\toprule\noalign{}
પગલું & વર્ણન \\
\midrule\noalign{}
\endhead
\bottomrule\noalign{}
\endlastfoot
1 & સર્કિટમાં સ્વતંત્ર મેશ ઓળખો \\
2 & મેશ કરંટ્સ (I_{1}, I_{2}, વગેરે) ઘડિયાળના કાંટાની દિશામાં અસાઇન કરો \\
3 & દરેક મેશ માટે KVL લાગુ કરો \\
4 & ઇક્વેશન્સ બનાવો: ΣR·I(સ્વયં) - ΣR·I(અડીને) = ΣV \\
5 & સિમલ્ટેનિયસ ઇક્વેશન્સ ઉકેલો \\
\end{longtable}
}

\begin{itemize}
\tightlist
\item
  \textbf{ફાયદો}: બ્રાન્ચ કરંટ મેથડ કરતાં ઓછા ઇક્વેશન્સ
\item
  \textbf{ઉપયોગ}: પ્લેનર નેટવર્ક્સ માટે શ્રેષ્ઠ
\item
  \textbf{મર્યાદા}: નોન-પ્લેનર નેટવર્ક્સ માટે ઓછું કાર્યક્ષમ
\end{itemize}

\end{solutionbox}
\begin{mnemonicbox}
``MIAMI'' - Meshes Identified, Assign currents, Make
equations, Intersection currents calculated, Solve કરો

\end{mnemonicbox}
\subsection*{પ્રશ્ન 1(ક) [7 માર્ક્સ
(વિકલ્પ)]}\label{uxaaauxab0uxab6uxaa8-1uxa95-7-uxaaeuxab0uxa95uxab8-uxab5uxa95uxab2uxaaa}

\textbf{યોગ્ય રેખાકૃતિનો ઉપયોગ કરીને ``નોડ પેર વોલ્ટેજ પદ્ધતિ'' સમજાવો.}

\begin{solutionbox}

\textbf{આકૃતિ:}

\begin{center}
\textbf{Mermaid Diagram (Code)}
\begin{verbatim}
{Shaded}
{Highlighting}[]
graph LR
    A((Node 1)) {-{-} I1 {-}{-}{} B((Node 2))}
    A {-{-} I2 {-}{-}{} C((Node 3))}
    B {-{-} I3 {-}{-}{} C}
    B {-{-} I4 {-}{-}{} D((Reference))}
    C {-{-} I5 {-}{-}{} D}
    A {-{-} I6 {-}{-}{} D}
{Highlighting}
{Shaded}
\end{verbatim}
\end{center}

{\def\LTcaptype{none} % do not increment counter
\begin{longtable}[]{@{}ll@{}}
\toprule\noalign{}
પગલું & વર્ણન \\
\midrule\noalign{}
\endhead
\bottomrule\noalign{}
\endlastfoot
1 & રેફરન્સ નોડ (ગ્રાઉન્ડ) પસંદ કરો \\
2 & બાકીના નોડ્સને નોડ વોલ્ટેજ (V_{1}, V_{2}, વગેરે) અસાઇન કરો \\
3 & દરેક નોડ પર KCL લાગુ કરો (રેફરન્સ સિવાય) \\
4 & ઓહ્મના નિયમનો ઉપયોગ કરીને કરંટ્સને નોડ વોલ્ટેજમાં વ્યક્ત કરો \\
5 & સિમલ્ટેનિયસ ઇક્વેશન્સ ઉકેલો \\
\end{longtable}
}

\begin{itemize}
\tightlist
\item
  \textbf{ફાયદો}: ઘણા મેશવાળા સર્કિટ્સ માટે મેશ મેથડ કરતાં ઓછા ઇક્વેશન્સ
\item
  \textbf{ઉપયોગ}: નોન-પ્લેનર સર્કિટ્સ માટે કાર્યક્ષમ
\item
  \textbf{મુખ્ય ઇક્વેશન}: ΣG·V(સ્વયં) - ΣG·V(અડીને) = ΣI
\end{itemize}

\end{solutionbox}
\begin{mnemonicbox}
``GRAND'' - Ground node fixed, Remaining nodes
numbered, Apply KCL, Note voltage differences, Derive solutions

\end{mnemonicbox}
\subsection*{પ્રશ્ન 2(અ) [3
માર્ક્સ]}\label{uxaaauxab0uxab6uxaa8-2uxa85-3-uxaaeuxab0uxa95uxab8}

\textbf{KCL ઉદાહરણ સાથે સમજાવો.}

\begin{solutionbox}

\textbf{આકૃતિ:}

\begin{verbatim}
    I1    
  +{-{-}{-}{-}{-}+}
  |     |
  |     I3 ↓  
I2 ↓    |
  |     |
  +{-{-}{-}{-}{-}+}
    I4 ↑
\end{verbatim}

\textbf{કિરચોફનો કરંટ લૉ (KCL)}: કોઈપણ નોડ પર પ્રવેશતા અને છોડતા તમામ કરંટ્સનો
અલજેબ્રાઇક સરવાળો શૂન્ય હોય છે.

{\def\LTcaptype{none} % do not increment counter
\begin{longtable}[]{@{}ll@{}}
\toprule\noalign{}
ગાણિતિક સ્વરૂપ & ઉદાહરણ ઉપયોગ \\
\midrule\noalign{}
\endhead
\bottomrule\noalign{}
\endlastfoot
ΣI = 0 & નોડ પર: I_{1} - I_{2} - I_{3} + I_{4} = 0 \\
ΣIin = ΣIout & પ્રવેશતા કરંટ્સ = બહાર નીકળતા કરંટ્સ \\
\end{longtable}
}

\end{solutionbox}
\begin{mnemonicbox}
``ZINC'' - Zero Is Net Current at a node

\end{mnemonicbox}
\subsection*{પ્રશ્ન 2(બ) [4
માર્ક્સ]}\label{uxaaauxab0uxab6uxaa8-2uxaac-4-uxaaeuxab0uxa95uxab8}

\textbf{યોગ્ય આકૃતિનો ઉપયોગ કરી Z-પેરામીટર, Y-પેરામીટર h-પેરામીટર અને
ABCD-પેરામીટર સમજાવો.}

\begin{solutionbox}

\textbf{આકૃતિ:}

\begin{verbatim}
        +{-{-}{-}{-}{-}+}
   V1   |     |   V2
       |  2  |   
+{-{-}{-}{-}{-}{-}+|  P  |+{-}{-}{-}{-}{-}+}
   I1   |  O  |   I2
       |  R  |   
        |  T  |
        +{-{-}{-}{-}{-}+}
\end{verbatim}

{\def\LTcaptype{none} % do not increment counter
\begin{longtable}[]{@{}
  >{\raggedright\arraybackslash}p{(\linewidth - 6\tabcolsep) * \real{0.2683}}
  >{\raggedright\arraybackslash}p{(\linewidth - 6\tabcolsep) * \real{0.2927}}
  >{\raggedright\arraybackslash}p{(\linewidth - 6\tabcolsep) * \real{0.2683}}
  >{\raggedright\arraybackslash}p{(\linewidth - 6\tabcolsep) * \real{0.1707}}@{}}
\toprule\noalign{}
\begin{minipage}[b]{\linewidth}\raggedright
પેરામીટર
\end{minipage} & \begin{minipage}[b]{\linewidth}\raggedright
વ્યાખ્યા
\end{minipage} & \begin{minipage}[b]{\linewidth}\raggedright
સમીકરણો
\end{minipage} & \begin{minipage}[b]{\linewidth}\raggedright
ઉપયોગ
\end{minipage} \\
\midrule\noalign{}
\endhead
\bottomrule\noalign{}
\endlastfoot
\textbf{Z} & ઇમ્પિડન્સ પેરામીટર્સ & V_{1} = Z_{1}_{1}I_{1} + Z_{1}_{2}I_{2}, V_{2} = Z_{2}_{1}I_{1} + Z_{2}_{2}I_{2} &
હાઇ ઇમ્પિડન્સ સર્કિટ્સ \\
\textbf{Y} & એડમિટન્સ પેરામીટર્સ & I_{1} = Y_{1}_{1}V_{1} + Y_{1}_{2}V_{2}, I_{2} = Y_{2}_{1}V_{1} + Y_{2}_{2}V_{2} &
લો ઇમ્પિડન્સ સર્કિટ્સ \\
\textbf{h} & હાઇબ્રિડ પેરામીટર્સ & V_{1} = h_{1}_{1}I_{1} + h_{1}_{2}V_{2}, I_{2} = h_{2}_{1}I_{1} + h_{2}_{2}V_{2} &
ટ્રાન્ઝિસ્ટર સર્કિટ્સ \\
\textbf{ABCD} & ટ્રાન્સમિશન પેરામીટર્સ & V_{1} = AV_{2} - BI_{2}, I_{1} = CV_{2} - DI_{2} &
કેસ્કેડેડ નેટવર્ક્સ \\
\end{longtable}
}

\end{solutionbox}
\begin{mnemonicbox}
``ZANY HAB'' - Z for high impedance, A for low,
hy-brid for transistors, ABCD for Cascades

\end{mnemonicbox}
\subsection*{પ્રશ્ન 2(ક) [7
માર્ક્સ]}\label{uxaaauxab0uxab6uxaa8-2uxa95-7-uxaaeuxab0uxa95uxab8}

\textbf{π-ટાઈપ નેટવર્કને T-ટાઈપ નેટવર્ક અને T-ટાઈપ નેટવર્કને π-ટાઈપ નેટવર્કમાં
રૂપાંતરિત કરવા માટેના સમીકરણો મેળવો.}

\begin{solutionbox}

\textbf{આકૃતિ:}

\begin{center}
\textbf{Mermaid Diagram (Code)}
\begin{verbatim}
{Shaded}
{Highlighting}[]
graph TD
    subgraph T{-Network}
    A1((1)) {-{-} Z1 {-}{-}{} O1((O))}
    B1((2)) {-{-} Z2 {-}{-}{} O1}
    C1((3)) {-{-} Z3 {-}{-}{} O1}
    end

    subgraph π{-Network}
    A2((1)) {-{-} Y1 {-}{-}{} B2((2))}
    B2 {-{-} Y2 {-}{-}{} C2((3))}
    C2 {-{-} Y3 {-}{-}{} A2}
    end
{Highlighting}
{Shaded}
\end{verbatim}
\end{center}

{\def\LTcaptype{none} % do not increment counter
\begin{longtable}[]{@{}
  >{\raggedright\arraybackslash}p{(\linewidth - 2\tabcolsep) * \real{0.5455}}
  >{\raggedright\arraybackslash}p{(\linewidth - 2\tabcolsep) * \real{0.4545}}@{}}
\toprule\noalign{}
\begin{minipage}[b]{\linewidth}\raggedright
રૂપાંતરણ
\end{minipage} & \begin{minipage}[b]{\linewidth}\raggedright
ફોર્મ્યુલા
\end{minipage} \\
\midrule\noalign{}
\endhead
\bottomrule\noalign{}
\endlastfoot
\textbf{π થી T} & Z_{1} = (Z_{1}_{2}·Z_{3}_{1})/(Z_{1}_{2}+Z_{2}_{3}+Z_{3}_{1}) Z_{2} =
(Z_{1}_{2}·Z_{2}_{3})/(Z_{1}_{2}+Z_{2}_{3}+Z_{3}_{1}) Z_{3} = (Z_{2}_{3}·Z_{3}_{1})/(Z_{1}_{2}+Z_{2}_{3}+Z_{3}_{1}) \\
\textbf{T થી π} & Z_{1}_{2} = (Z_{1}·Z_{2}+Z_{2}·Z_{3}+Z_{3}·Z_{1})/Z_{3} Z_{2}_{3} =
(Z_{1}·Z_{2}+Z_{2}·Z_{3}+Z_{3}·Z_{1})/Z_{1} Z_{3}_{1} = (Z_{1}·Z_{2}+Z_{2}·Z_{3}+Z_{3}·Z_{1})/Z_{2} \\
\end{longtable}
}

\begin{itemize}
\tightlist
\item
  \textbf{ઉપયોગ}: નેટવર્ક સરળીકરણ અને વિશ્લેષણ
\item
  \textbf{શરત}: બંને નેટવર્ક્સ ટર્મિનલ્સ પર સમાન હોવા જોઈએ
\item
  \textbf{મર્યાદા}: ફક્ત લીનિયર નેટવર્ક્સ માટે લાગુ પડે છે
\end{itemize}

\end{solutionbox}
\begin{mnemonicbox}
``TRIP'' - T and π networks Relate Impedances
through Products and sums

\end{mnemonicbox}
\subsection*{પ્રશ્ન 2(અ OR) [3
માર્ક્સ]}\label{uxaaauxab0uxab6uxaa8-2uxa85-or-3-uxaaeuxab0uxa95uxab8}

\textbf{KVL ઉદાહરણ સાથે સમજાવો.}

\begin{solutionbox}

\textbf{આકૃતિ:}

\begin{verbatim}
    +{-{-}R1{-}{-}+}
    |      |
   V1     R2
    |      |
    +{-{-}R3{-}{-}+}
\end{verbatim}

\textbf{કિરચોફનો વોલ્ટેજ લૉ (KVL)}: કોઈપણ બંધ લૂપમાં તમામ વોલ્ટેજનો અલજેબ્રાઇક
સરવાળો શૂન્ય હોય છે.

{\def\LTcaptype{none} % do not increment counter
\begin{longtable}[]{@{}ll@{}}
\toprule\noalign{}
ગાણિતિક સ્વરૂપ & ઉદાહરણ ઉપયોગ \\
\midrule\noalign{}
\endhead
\bottomrule\noalign{}
\endlastfoot
ΣV = 0 & લૂપમાં: V_{1} - IR_{1} - IR_{2} - IR_{3} = 0 \\
ΣVrises = ΣVdrops & વોલ્ટેજ વધારા = વોલ્ટેજ ઘટાડા \\
\end{longtable}
}

\end{solutionbox}
\begin{mnemonicbox}
``ZERO'' - Zero is Every voltage Round a loop's
Output

\end{mnemonicbox}
\subsection*{પ્રશ્ન 2(બ OR) [4
માર્ક્સ]}\label{uxaaauxab0uxab6uxaa8-2uxaac-or-4-uxaaeuxab0uxa95uxab8}

\textbf{વિવિધ ઈલેક્ટ્રોનિક્સ નેટવર્કનું વર્ગીકરણ કરો અને સમજાવો.}

\begin{solutionbox}

{\def\LTcaptype{none} % do not increment counter
\begin{longtable}[]{@{}
  >{\raggedright\arraybackslash}p{(\linewidth - 4\tabcolsep) * \real{0.3889}}
  >{\raggedright\arraybackslash}p{(\linewidth - 4\tabcolsep) * \real{0.3611}}
  >{\raggedright\arraybackslash}p{(\linewidth - 4\tabcolsep) * \real{0.2500}}@{}}
\toprule\noalign{}
\begin{minipage}[b]{\linewidth}\raggedright
નેટવર્ક પ્રકાર
\end{minipage} & \begin{minipage}[b]{\linewidth}\raggedright
વર્ણન
\end{minipage} & \begin{minipage}[b]{\linewidth}\raggedright
ઉદાહરણ
\end{minipage} \\
\midrule\noalign{}
\endhead
\bottomrule\noalign{}
\endlastfoot
\textbf{લીનિયર vs નોન-લીનિયર} & સમાનુપાતિકતાના સિદ્ધાંતનું પાલન કરે/ન કરે &
રેઝિસ્ટર્સ vs ડાયોડ્સ \\
\textbf{પેસિવ vs એક્ટિવ} & ઊર્જા પ્રદાન કરતા નથી/કરે છે & RC સર્કિટ vs
એમ્પ્લિફાયર \\
\textbf{બાયલેટરલ vs યુનિલેટરલ} & બંને દિશામાં સમાન/અલગ ગુણધર્મો & રેઝિસ્ટર્સ vs
ડાયોડ્સ \\
\textbf{લમ્પ્ડ vs ડિસ્ટ્રિબ્યુટેડ} & પેરામીટર્સ કેન્દ્રિત/ફેલાયેલા છે & RC સર્કિટ vs
ટ્રાન્સમિશન લાઇન \\
\textbf{ટાઇમ વેરિઅન્ટ vs ઇન્વેરિઅન્ટ} & પેરામીટર્સ સમય સાથે બદલાય/ન બદલાય &
ઇલેક્ટ્રોનિક સ્વિચ vs ફિક્સ્ડ રેઝિસ્ટર \\
\end{longtable}
}

\textbf{આકૃતિ:}

\begin{verbatim}
graph TB
    A[ઇલેક્ટ્રોનિક નેટવર્ક્સ]
    A {-{-} B[લીનિયરતા આધારિત]}
    A {-{-} C[ઊર્જા આધારિત]}
    A {-{-} D[દિશાત્મકતા આધારિત]}
    A {-{-} E[પેરામીટર્સ આધારિત]}
    A {-{-} F[સમય આધારિત]}

    B {-{-} G[લીનિયર]}
    B {-{-} H[નોન{-}લીનિયર]}
    C {-{-} I[એક્ટિવ]}
    C {-{-} J[પેસિવ]}
    D {-{-} K[બાયલેટરલ]}
    D {-{-} L[યુનિલેટરલ]}
    E {-{-} M[લમ્પ્ડ]}
    E {-{-} N[ડિસ્ટ્રિબ્યુટેડ]}
    F {-{-} O[ટાઇમ{-}ઇન્વેરિઅન્ટ]}
    F {-{-} P[ટાઇમ{-}વેરિઅન્ટ]}
\end{verbatim}

\end{solutionbox}
\begin{mnemonicbox}
``PLANT'' - Proportionality for Linear, Lively for
Active, All directions for bilateral, Near for lumped, Time-fixed for
invariant

\end{mnemonicbox}
\subsection*{પ્રશ્ન 2(ક OR) [7
માર્ક્સ]}\label{uxaaauxab0uxab6uxaa8-2uxa95-or-7-uxaaeuxab0uxa95uxab8}

\textbf{T-નેટવર્ક અને π-નેટવર્ક માટે કૅરૅક્ટરીસટીક્સ ઇમપીડંસનું સમીકરણ મેળવો.}

\begin{solutionbox}

\textbf{આકૃતિ:}

\begin{center}
\textbf{Mermaid Diagram (Code)}
\begin{verbatim}
{Shaded}
{Highlighting}[]
graph TD
    subgraph T{-Network}
    A1((1)) {-{-} Z1 {-}{-}{} O1((O))}
    O1 {-{-} Z3 {-}{-}{} C1((2))}
    O1 {-{-} Z2 {-}{-}{} B1}
    end

    subgraph π{-Network}
    A2((1)) {-{-} Y1 {-}{-}{} B2}
    B2 {-{-} Y2 {-}{-}{} C2((2))}
    C2 {-{-} Y3 {-}{-}{} A2}
    end
{Highlighting}
{Shaded}
\end{verbatim}
\end{center}

{\def\LTcaptype{none} % do not increment counter
\begin{longtable}[]{@{}
  >{\raggedright\arraybackslash}p{(\linewidth - 4\tabcolsep) * \real{0.1475}}
  >{\raggedright\arraybackslash}p{(\linewidth - 4\tabcolsep) * \real{0.5574}}
  >{\raggedright\arraybackslash}p{(\linewidth - 4\tabcolsep) * \real{0.2951}}@{}}
\toprule\noalign{}
\begin{minipage}[b]{\linewidth}\raggedright
નેટવર્ક
\end{minipage} & \begin{minipage}[b]{\linewidth}\raggedright
કૅરૅક્ટરીસટીક્સ ઇમપીડંસ સમીકરણ
\end{minipage} & \begin{minipage}[b]{\linewidth}\raggedright
મેળવવાના પગલાં
\end{minipage} \\
\midrule\noalign{}
\endhead
\bottomrule\noalign{}
\endlastfoot
\textbf{T-નેટવર્ક} & Z_{0}T = \sqrt[(Z_{1}+Z_{2})(Z_{2}+Z_{3})] & 1. સિમેટ્રિકલ લોડ Z_{0} લાગુ
કરો 2. ઇનપુટ ઇમ્પીડન્સ શોધો 3. ઇમ્પીડન્સ મેચિંગ માટે, Zin = Z_{0} 4. Z_{0} માટે ઉકેલો \\
\textbf{π-નેટવર્ક} & Z_{0}π = 1/\sqrt[(Y_{1}+Y_{3})(Y_{2}+Y_{3})] & 1. સિમેટ્રિકલ લોડ Z_{0}
લાગુ કરો 2. ઇનપુટ ઇમ્પીડન્સ શોધો 3. ઇમ્પીડન્સ મેચિંગ માટે, Zin = Z_{0} 4. Z_{0} માટે
ઉકેલો \\
\end{longtable}
}

\begin{itemize}
\tightlist
\item
  \textbf{સંબંધ}: Z_{0}T \times Z_{0}π = Z_{1}·Z_{3}
\item
  \textbf{ઉપયોગ}: ઇમ્પીડન્સ મેચિંગ અને ફિલ્ટર્સ
\item
  \textbf{મર્યાદા}: ફક્ત સિમેટ્રિકલ નેટવર્ક્સ માટે માન્ય
\end{itemize}

\end{solutionbox}
\begin{mnemonicbox}
``TIPSZ'' - T-networks and π-networks Impedances are
Products and Square roots of Z values

\end{mnemonicbox}
\subsection*{પ્રશ્ન 3(અ) [3
માર્ક્સ]}\label{uxaaauxab0uxab6uxaa8-3uxa85-3-uxaaeuxab0uxa95uxab8}

\textbf{ડ્યુઆલિટી ના સિદ્ધાંતને ઉદાહરણ સાથે સમજાવો.}

\begin{solutionbox}

\textbf{આકૃતિ:}

\begin{verbatim}
Original Circuit          Dual Circuit
   +{-{-}{-}R1{-}{-}{-}+               +{-}{-}{-}G1{-}{-}{-}+}
   |        |               |        |
  V1       R2        ={    I1       G2}
   |        |               |        |
   +{-{-}{-}R3{-}{-}{-}+               +{-}{-}{-}G3{-}{-}{-}+}
\end{verbatim}

\textbf{ડ્યુઆલિટીનો સિદ્ધાંત}: દરેક ઇલેક્ટ્રિકલ નેટવર્ક માટે, એક ડ્યુઅલ નેટવર્ક
અસ્તિત્વમાં છે જ્યાં:

{\def\LTcaptype{none} % do not increment counter
\begin{longtable}[]{@{}lll@{}}
\toprule\noalign{}
ઓરિજિનલ & ડ્યુઅલ & ઉદાહરણ \\
\midrule\noalign{}
\endhead
\bottomrule\noalign{}
\endlastfoot
વોલ્ટેજ (V) & કરંટ (I) & 10V સોર્સ \rightarrow 10A સોર્સ \\
કરંટ (I) & વોલ્ટેજ (V) & 5A \rightarrow 5V \\
રેઝિસ્ટન્સ (R) & કન્ડક્ટન્સ (G) & 100Ω \rightarrow 100S \\
સીરીઝ કનેક્શન & પેરેલલ કનેક્શન & સીરીઝ રેઝિસ્ટર્સ \rightarrow પેરેલલ કન્ડક્ટર્સ \\
KVL & KCL & ΣV = 0 \rightarrow ΣI = 0 \\
\end{longtable}
}

\end{solutionbox}
\begin{mnemonicbox}
``VIGOR'' - Voltage to current, Impedance to
admittance, Graph remains, Open to closed, Resistors to conductors

\end{mnemonicbox}
\subsection*{પ્રશ્ન 3(બ) [4
માર્ક્સ]}\label{uxaaauxab0uxab6uxaa8-3uxaac-4-uxaaeuxab0uxa95uxab8}

\textbf{થેવેનિનના થિયરમનો ઉપયોગ કરીને સર્કિટમાં લોડ પ્રવાહની ગણતરી કરવાનાં પગલાં
સમજાવો.}

\begin{solutionbox}

\textbf{આકૃતિ:}

\begin{verbatim}
flowchart LR
    A[ઓરિજિનલ સર્કિટ] {-{-} B[લોડ દૂર કરો]}
    B {-{-} C[Voc શોધો]}
    B {-{-} D[Rth શોધો]}
    C {-{-} E[થેવેનિન ઇક્વિવેલન્ટ]}
    D {-{-} E}
    E {-{-} F[લોડ ફરી જોડો]}
    F {-{-} G[IL = Vth/Rth+RL ગણો]}

    style E fill:\#bbf,stroke:\#333
\end{verbatim}

{\def\LTcaptype{none} % do not increment counter
\begin{longtable}[]{@{}ll@{}}
\toprule\noalign{}
પગલું & વર્ણન \\
\midrule\noalign{}
\endhead
\bottomrule\noalign{}
\endlastfoot
1 & સર્કિટમાંથી લોડ રેઝિસ્ટરને દૂર કરો \\
2 & લોડ ટર્મિનલ વચ્ચે ઓપન-સર્કિટ વોલ્ટેજ (Vth) શોધો \\
3 & સર્કિટમાં પાછા જોતા થેવેનિન રેઝિસ્ટન્સ (Rth) ગણો \\
4 & થેવેનિન ઇક્વિવેલન્ટ સર્કિટ (Rth સાથે સીરીઝમાં Vth) દોરો \\
5 & થેવેનિન સર્કિટ પર લોડ રેઝિસ્ટર (RL) ફરીથી જોડો \\
6 & લોડ કરંટ ગણો: IL = Vth/(Rth+RL) \\
\end{longtable}
}

\end{solutionbox}
\begin{mnemonicbox}
``REVOLT'' - Remove load, Evaluate Voc, Obtain Rth,
Look at Thevenin circuit, Use I = V/R formula

\end{mnemonicbox}
\subsection*{પ્રશ્ન 3(ક) [7
માર્ક્સ]}\label{uxaaauxab0uxab6uxaa8-3uxa95-7-uxaaeuxab0uxa95uxab8}

\textbf{સુપરપોઝિશન થિયરમનો ઉપયોગ કરીને લોડ રેઝિસ્ટરમાંથી પસાર થતો વિદ્યુતપ્રવાહ
શોધો.}

\begin{solutionbox}

\textbf{આકૃતિ:}

\begin{verbatim}
    4Ω         10Ω
    ┌──────┬───────┐
    │      │       │
 12V┘     6Ω      ┌12A
    │    IL↓       │
    │      │       │
    └──────┴───────┘
\end{verbatim}


{\def\LTcaptype{none} % do not increment counter
\vspace{-5pt}
\captionof{table}{પગલા-દર-પગલા ઉકેલ:}
\vspace{-10pt}
\begin{longtable}[]{@{}
  >{\raggedright\arraybackslash}p{(\linewidth - 4\tabcolsep) * \real{0.1875}}
  >{\raggedright\arraybackslash}p{(\linewidth - 4\tabcolsep) * \real{0.4062}}
  >{\raggedright\arraybackslash}p{(\linewidth - 4\tabcolsep) * \real{0.4062}}@{}}
\toprule\noalign{}
\begin{minipage}[b]{\linewidth}\raggedright
પગલું
\end{minipage} & \begin{minipage}[b]{\linewidth}\raggedright
વર્ણન
\end{minipage} & \begin{minipage}[b]{\linewidth}\raggedright
ગણતરી
\end{minipage} \\
\midrule\noalign{}
\endhead
\bottomrule\noalign{}
\endlastfoot
1 & ફક્ત 12V સોર્સ ધ્યાનમાં લો (12A ને ઓપન સાથે બદલો) & I_{1} = 12/(4+6+10) =
0.6A 6Ω માંથી I_{1} = 0.6A \\
2 & ફક્ત 12A સોર્સ ધ્યાનમાં લો (12V ને શોર્ટ સાથે બદલો) & I_{2} = -12\times10/(4+10+6)
= -6A 6Ω માંથી I_{2} = -12\times4/(4+10+6) = -2.4A \\
3 & સુપરપોઝિશન લાગુ કરો & IL = I_{1} + I_{2} = 0.6 + (-2.4) = -1.8A \\
\end{longtable}
}

\end{solutionbox}
\begin{solutionbox}
IL = -1.8A (6Ω લોડ રેઝિસ્ટરમાં ઉપર તરફ વહેતો કરંટ)

\end{solutionbox}
\begin{mnemonicbox}
``SONAR'' - Sources Only one at a time, Neutralize
others, Add Results

\end{mnemonicbox}
\subsection*{પ્રશ્ન 3(અ OR) [3
માર્ક્સ]}\label{uxaaauxab0uxab6uxaa8-3uxa85-or-3-uxaaeuxab0uxa95uxab8}

\textbf{મહત્તમ પાવર ટ્રાન્સફર થિયરમનું નિવેદન લખો. એસી અને ડીસી નેટવર્ક માટે મહત્તમ
પાવર ટ્રાન્સફર માટેની શરતો શું છે?}

\begin{solutionbox}

\textbf{મહત્તમ પાવર ટ્રાન્સફર થિયરમ}: જ્યારે લોડ ઇમ્પીડન્સ સોર્સ આંતરિક ઇમ્પીડન્સના
કોમ્પ્લેક્સ કોન્જુગેટ જેટલી હોય ત્યારે સોર્સથી લોડમાં મહત્તમ પાવર ટ્રાન્સફર થાય છે.

{\def\LTcaptype{none} % do not increment counter
\begin{longtable}[]{@{}
  >{\raggedright\arraybackslash}p{(\linewidth - 2\tabcolsep) * \real{0.2745}}
  >{\raggedright\arraybackslash}p{(\linewidth - 2\tabcolsep) * \real{0.7255}}@{}}
\toprule\noalign{}
\begin{minipage}[b]{\linewidth}\raggedright
નેટવર્ક પ્રકાર
\end{minipage} & \begin{minipage}[b]{\linewidth}\raggedright
મહત્તમ પાવર ટ્રાન્સફર માટેની શરત
\end{minipage} \\
\midrule\noalign{}
\endhead
\bottomrule\noalign{}
\endlastfoot
\textbf{ડીસી નેટવર્ક્સ} & RL = Rth (લોડ રેઝિસ્ટન્સ થેવેનિન રેઝિસ્ટન્સ જેટલી હોય) \\
\textbf{એસી નેટવર્ક્સ} & ZL = Zth* (લોડ ઇમ્પીડન્સ થેવેનિન ઇમ્પીડન્સના કોમ્પ્લેક્સ
કોન્જુગેટ જેટલી હોય) RL = Rth અને XL = -Xth \\
\end{longtable}
}

\textbf{આકૃતિ:}

\begin{verbatim}
   Rth     
    ┌─/{//─┐}
    │        │
 Vth┘       RL
    │        │
    └────────┘
   DC Network

   Rth     Xth
    ┌─/{//─┬─XX─┐}
    │        │    │
 Vth┘       RL   XL
    │        │    │
    └────────┴────┘
    AC Network
\end{verbatim}

\end{solutionbox}
\begin{mnemonicbox}
``MATCH'' - Maximum power At Terminals when
Conjugate impedances are Honored

\end{mnemonicbox}
\subsection*{પ્રશ્ન 3(બ OR) [4
માર્ક્સ]}\label{uxaaauxab0uxab6uxaa8-3uxaac-or-4-uxaaeuxab0uxa95uxab8}

\textbf{નોટોનના થિયરમનો ઉપયોગ કરીને સર્કિટમાં લોડ પ્રવાહની ગણતરી કરવાનાં પગલાં
સમજાવો.}

\begin{solutionbox}

\textbf{આકૃતિ:}

\begin{verbatim}
flowchart LR
    A[ઓરિજિનલ સર્કિટ] {-{-} B[લોડ ટર્મિનલ્સ શોર્ટ કરો]}
    B {-{-} C[Isc શોધો]}
    B {-{-} D[Rn=Rth શોધો]}
    C {-{-} E[નોર્ટન ઇક્વિવેલન્ટ]}
    D {-{-} E}
    E {-{-} F[લોડ ફરી જોડો]}
    F {-{-} G[IL = In/Rn+RL ગણો]}

    style E fill:\#bbf,stroke:\#333
\end{verbatim}

{\def\LTcaptype{none} % do not increment counter
\begin{longtable}[]{@{}ll@{}}
\toprule\noalign{}
પગલું & વર્ણન \\
\midrule\noalign{}
\endhead
\bottomrule\noalign{}
\endlastfoot
1 & સર્કિટમાંથી લોડ રેઝિસ્ટરને દૂર કરો \\
2 & લોડ ટર્મિનલ્સ વચ્ચે શોર્ટ-સર્કિટ કરંટ (In) શોધો \\
3 & સર્કિટમાં પાછા જોતા નોર્ટન રેઝિસ્ટન્સ (Rn) ગણો \\
4 & નોર્ટન ઇક્વિવેલન્ટ સર્કિટ (Rn સાથે પેરેલલમાં In) દોરો \\
5 & નોર્ટન સર્કિટ પર લોડ રેઝિસ્ટર (RL) ફરીથી જોડો \\
6 & લોડ કરંટ ગણો: IL = In\timesRn/(Rn+RL) \\
\end{longtable}
}

\end{solutionbox}
\begin{mnemonicbox}
``SENIOR'' - Short terminals, Evaluate Isc, Notice
Rn value, Implement Norton circuit, Obtain result

\end{mnemonicbox}
\subsection*{પ્રશ્ન 3(ક OR) [7
માર્ક્સ]}\label{uxaaauxab0uxab6uxaa8-3uxa95-or-7-uxaaeuxab0uxa95uxab8}

\textbf{આપેલ નેટવર્ક પર રેસીપ્રોસીટી થિયરમ કેવી રીતે લાગુ થાય છે તે દર્શાવો.}

\begin{solutionbox}

\textbf{આકૃતિ:}

\begin{verbatim}
    2Ω         2Ω
    ┌──────┬───────┐
    │      │       │
 10V┘     4Ω      2Ω
    │      │       │
    └──────┴───────┘
\end{verbatim}


{\def\LTcaptype{none} % do not increment counter
\vspace{-5pt}
\captionof{table}{રેસીપ્રોસીટી થિયરમ લાગુ કરવું:}
\vspace{-10pt}
\begin{longtable}[]{@{}
  >{\raggedright\arraybackslash}p{(\linewidth - 6\tabcolsep) * \real{0.1429}}
  >{\raggedright\arraybackslash}p{(\linewidth - 6\tabcolsep) * \real{0.2619}}
  >{\raggedright\arraybackslash}p{(\linewidth - 6\tabcolsep) * \real{0.2619}}
  >{\raggedright\arraybackslash}p{(\linewidth - 6\tabcolsep) * \real{0.3333}}@{}}
\toprule\noalign{}
\begin{minipage}[b]{\linewidth}\raggedright
પગલું
\end{minipage} & \begin{minipage}[b]{\linewidth}\raggedright
સર્કિટ 1
\end{minipage} & \begin{minipage}[b]{\linewidth}\raggedright
સર્કિટ 2
\end{minipage} & \begin{minipage}[b]{\linewidth}\raggedright
ચકાસણી
\end{minipage} \\
\midrule\noalign{}
\endhead
\bottomrule\noalign{}
\endlastfoot
1 & ડાબી બાજુ 10V સોર્સ, જમણી બાજુ I_{1} શોધો & જમણી બાજુ 10V સોર્સ, ડાબી બાજુ I_{2}
શોધો & I_{1} = I_{2} રેસીપ્રોસીટી પુષ્ટિ કરે છે \\
2 & KVL વાપરીને મેશ ઇક્વેશન્સ બનાવો & બદલાયેલ સોર્સ માટે નવા મેશ ઇક્વેશન્સ બનાવો &
બંને સિસ્ટમ ઉકેલો \\
3 & I_{1} = 10\times2/(2\times4 + 2\times2 + 4\times2) = 0.625A & I_{2} = 10\times2/(2\times4 + 2\times2 + 4\times2) =
0.625A & I_{1} = I_{2} = 0.625A ✓ \\
\end{longtable}
}

\textbf{સિદ્ધાંત}: બાયલેટરલ તત્વો ધરાવતા પેસિવ નેટવર્કમાં, જો બ્રાન્ચ 1માં વોલ્ટેજ
સોર્સ E બ્રાન્ચ 2માં કરંટ I ઉત્પન્ન કરે, તો બ્રાન્ચ 2માં મૂકેલો તે જ વોલ્ટેજ સોર્સ E
બ્રાન્ચ 1માં તે જ કરંટ I ઉત્પન્ન કરશે.

\end{solutionbox}
\begin{mnemonicbox}
``RESPECT'' - Rewire sources, Exchange positions,
See if currents Preserve Equality when Circuit Transformed

\end{mnemonicbox}
\subsection*{પ્રશ્ન 4(અ) [3
માર્ક્સ]}\label{uxaaauxab0uxab6uxaa8-4uxa85-3-uxaaeuxab0uxa95uxab8}

\textbf{કપલ્ડ સર્કિટ સમજાવો.}

\begin{solutionbox}

\textbf{આકૃતિ:}

\begin{verbatim}
    L1         L2
    ┌─OOOO─┐   ┌─OOOO─┐
    │      │   │      │
 V1 ┘      │   │     RL
    │   M  │   │      │
    └──────┘   └──────┘
    Primary    Secondary
\end{verbatim}

\textbf{કપલ્ડ સર્કિટ}: એક સર્કિટ જ્યાં મ્યુચ્યુઅલ ઇન્ડક્ટન્સ દ્વારા ઇન્ડક્ટર્સ વચ્ચે ઊર્જા
ટ્રાન્સફર થાય છે.

{\def\LTcaptype{none} % do not increment counter
\begin{longtable}[]{@{}
  >{\raggedright\arraybackslash}p{(\linewidth - 2\tabcolsep) * \real{0.4583}}
  >{\raggedright\arraybackslash}p{(\linewidth - 2\tabcolsep) * \real{0.5417}}@{}}
\toprule\noalign{}
\begin{minipage}[b]{\linewidth}\raggedright
પેરામીટર
\end{minipage} & \begin{minipage}[b]{\linewidth}\raggedright
વર્ણન
\end{minipage} \\
\midrule\noalign{}
\endhead
\bottomrule\noalign{}
\endlastfoot
\textbf{મ્યુચ્યુઅલ ઇન્ડક્ટન્સ (M)} & કોઇલ્સ વચ્ચે મેગ્નેટિક કપલિંગનું માપ \\
\textbf{કપલિંગ કોઇફિશિયન્ટ (k)} & k = M/\sqrt(L_{1}L_{2}), 0 (કોઈ કપલિંગ નહીં) થી 1
(પરફેક્ટ કપલિંગ) સુધી \\
\textbf{ઉપયોગો} & ટ્રાન્સફોર્મર, ફિલ્ટર્સ, ટ્યુન્ડ સર્કિટ્સ \\
\end{longtable}
}

\end{solutionbox}
\begin{mnemonicbox}
``MICE'' - Mutual Inductance Creates Energy transfer

\end{mnemonicbox}
\subsection*{પ્રશ્ન 4(બ) [4
માર્ક્સ]}\label{uxaaauxab0uxab6uxaa8-4uxaac-4-uxaaeuxab0uxa95uxab8}

\textbf{કપલ્ડ સર્કિટ માટે co-efficient of coupling નું સમીકરણ મેળવો.}

\begin{solutionbox}

\textbf{આકૃતિ:}

\begin{center}
\textbf{Mermaid Diagram (Code)}
\begin{verbatim}
{Shaded}
{Highlighting}[]
graph LR
    A[મેગ્નેટિક ફ્લક્સ લિંકેજ] {-{-}{} B[મ્યુચ્યુઅલ ઇન્ડક્ટન્સ]}
    B {-{-}{} C[કપલિંગ કોઇફિશિયન્ટ]}

    subgraph Formula Derivation
    D["φ12 = કોઇલ 1 થી 2 ફ્લક્સ"]
    E["M = N2·φ12/I1"]
    F["k = M/(L1·L2)"]
    end
{Highlighting}
{Shaded}
\end{verbatim}
\end{center}

{\def\LTcaptype{none} % do not increment counter
\begin{longtable}[]{@{}lll@{}}
\toprule\noalign{}
પગલું & વર્ણન & સમીકરણ \\
\midrule\noalign{}
\endhead
\bottomrule\noalign{}
\endlastfoot
1 & મ્યુચ્યુઅલ ઇન્ડક્ટન્સ વ્યાખ્યાયિત કરો & M = N_{2}·φ_{1}_{2}/I_{1} \\
2 & સેલ્ફ-ઇન્ડક્ટન્સ વ્યાખ્યાયિત કરો & L_{1} = N_{1}·φ_{1}_{1}/I_{1}, L_{2} = N_{2}·φ_{2}_{2}/I_{2} \\
3 & મહત્તમ શક્ય M & Mmax = \sqrt(L_{1}·L_{2}) \\
4 & કપલિંગ કોઇફિશિયન્ટ વ્યાખ્યાયિત કરો & k = M/\sqrt(L_{1}·L_{2}) \\
\end{longtable}
}

\begin{itemize}
\tightlist
\item
  \textbf{રેન્જ}: 0 \leq k \leq 1
\item
  \textbf{ભૌતિક અર્થ}: એક કોઇલનો કેટલો ફ્લક્સ બીજી કોઇલ સાથે લિંક થાય છે તેનું
  પ્રમાણ
\item
  \textbf{પરફેક્ટ કપલિંગ}: k = 1, જ્યારે બધો ફ્લક્સ બંને કોઇલ્સને લિંક કરે છે
\end{itemize}

\end{solutionbox}
\begin{mnemonicbox}
``MASK'' - Mutual inductance And Self inductances
create K

\end{mnemonicbox}
\subsection*{પ્રશ્ન 4(ક) [7
માર્ક્સ]}\label{uxaaauxab0uxab6uxaa8-4uxa95-7-uxaaeuxab0uxa95uxab8}

\textbf{સિરીઝ રેઝોનન્સ સર્કિટની રેઝોનન્સ ફ્રિક્વન્સીનું સમીકરણ તારવો. R=20Ω, L=1H,
C=1μF સાથે સિરીઝ RLC સર્કિટની રેઝોનન્ટ ફ્રિક્વન્સી, Q ફેક્ટર અને બેન્ડવિડ્થની ગણતરી
કરો.}

\begin{solutionbox}

\textbf{આકૃતિ:}

\begin{verbatim}
     R       L
    ┌─/{//─┬─OOOO─┐}
    │        │      │
  V ┘        │      │
    │        │      │
    └────────┴──||──┘
                C
\end{verbatim}

\textbf{મેળવણી:}

{\def\LTcaptype{none} % do not increment counter
\begin{longtable}[]{@{}lll@{}}
\toprule\noalign{}
પગલું & વર્ણન & સમીકરણ \\
\midrule\noalign{}
\endhead
\bottomrule\noalign{}
\endlastfoot
1 & સિરીઝ RLC ની ઇમ્પીડન્સ & Z = R + j(ωL - 1/ωC) \\
2 & રેઝોનન્સ પર, Im(Z) = 0 & ωL - 1/ωC = 0 \\
3 & રેઝોનન્ટ ફ્રિક્વન્સી માટે ઉકેલો & ω_{0} = 1/\sqrt(LC) અથવા f_{0} = 1/(2π\sqrt(LC)) \\
\end{longtable}
}

\textbf{ગણતરીઓ:}

{\def\LTcaptype{none} % do not increment counter
\begin{longtable}[]{@{}
  >{\raggedright\arraybackslash}p{(\linewidth - 6\tabcolsep) * \real{0.2683}}
  >{\raggedright\arraybackslash}p{(\linewidth - 6\tabcolsep) * \real{0.2195}}
  >{\raggedright\arraybackslash}p{(\linewidth - 6\tabcolsep) * \real{0.3171}}
  >{\raggedright\arraybackslash}p{(\linewidth - 6\tabcolsep) * \real{0.1951}}@{}}
\toprule\noalign{}
\begin{minipage}[b]{\linewidth}\raggedright
પેરામીટર
\end{minipage} & \begin{minipage}[b]{\linewidth}\raggedright
ફોર્મ્યુલા
\end{minipage} & \begin{minipage}[b]{\linewidth}\raggedright
ગણતરી
\end{minipage} & \begin{minipage}[b]{\linewidth}\raggedright
પરિણામ
\end{minipage} \\
\midrule\noalign{}
\endhead
\bottomrule\noalign{}
\endlastfoot
રેઝોનન્ટ ફ્રિક્વન્સી & f_{0} = 1/(2π\sqrt(LC)) & f_{0} = 1/(2π\sqrt(1\times10^{-}^{6})) & 159.15 Hz \\
Q ફેક્ટર &

Q = ω_{0}L/R &

Q = 2π\times159.15\times1/20 & 50 \\

બેન્ડવિડ્થ & BW = f_{0}/Q & BW = 159.15/50 & 3.18 Hz \\
\end{longtable}
}

\end{solutionbox}
\begin{mnemonicbox}
``FQBR'' - Frequency from reactances, Q from
resistance ratio, Bandwidth from Resonance divided by Q

\end{mnemonicbox}
\subsection*{પ્રશ્ન 4(અ OR) [3
માર્ક્સ]}\label{uxaaauxab0uxab6uxaa8-4uxa85-or-3-uxaaeuxab0uxa95uxab8}

\textbf{Quality factor સમજાઓ.}

\begin{solutionbox}

\textbf{આકૃતિ:}

\begin{center}
\textbf{Mermaid Diagram (Code)}
\begin{verbatim}
{Shaded}
{Highlighting}[]
graph TD
    A[Quality Factor] {-{-}{} B[ઊર્જા સંગ્રહ]}
    A {-{-}{} C[પાવર લોસ]}
    A {-{-}{} D[સિલેક્ટિવિટી]}
    A {-{-}{} E[બેન્ડવિડ્થ]}

    style A fill:\#bbf,stroke:\#333
{Highlighting}
{Shaded}
\end{verbatim}
\end{center}

\textbf{ક્વોલિટી ફેક્ટર (Q)}: એક ડાયમેન્શનલેસ પેરામીટર જે બતાવે છે કે રેઝોનેટર કેટલો
અન્ડર-ડેમ્પ્ડ છે, અથવા વૈકલ્પિક રીતે, રેઝોનેટરની બેન્ડવિડ્થ તેની કેન્દ્ર ફ્રિક્વન્સી સાપેક્ષે
કેટલી છે.

{\def\LTcaptype{none} % do not increment counter
\begin{longtable}[]{@{}
  >{\raggedright\arraybackslash}p{(\linewidth - 2\tabcolsep) * \real{0.3243}}
  >{\raggedright\arraybackslash}p{(\linewidth - 2\tabcolsep) * \real{0.6757}}@{}}
\toprule\noalign{}
\begin{minipage}[b]{\linewidth}\raggedright
વ્યાખ્યા
\end{minipage} & \begin{minipage}[b]{\linewidth}\raggedright
ગાણિતિક અભિવ્યક્તિ
\end{minipage} \\
\midrule\noalign{}
\endhead
\bottomrule\noalign{}
\endlastfoot
\textbf{ઊર્જા પરિપ્રેક્ષ્ય} & Q = 2π \times સંગ્રહિત ઊર્જા / સાયકલ દીઠ વેડફાતી ઊર્જા \\
\textbf{સર્કિટ પરિપ્રેક્ષ્ય} & Q = X/R (જ્યાં X રિએક્ટન્સ છે, R રેઝિસ્ટન્સ છે) \\
\textbf{ફ્રિક્વન્સી પરિપ્રેક્ષ્ય} & Q = f_{0}/BW (જ્યાં f_{0} રેઝોનન્ટ ફ્રિક્વન્સી છે, BW
બેન્ડવિડ્થ છે) \\
\end{longtable}
}

\end{solutionbox}
\begin{mnemonicbox}
``QSEL'' - Quality shows Energy vs.~Loss and
Selectivity

\end{mnemonicbox}
\subsection*{પ્રશ્ન 4(બ OR) [4
માર્ક્સ]}\label{uxaaauxab0uxab6uxaa8-4uxaac-or-4-uxaaeuxab0uxa95uxab8}

\textbf{કેપેસીટર માટે Quality factor નું સમીકરણ તારવો.}

\begin{solutionbox}

\textbf{આકૃતિ:}

\begin{verbatim}
    Ideal C    ESR
      ||      /{//}
      ||        R
     ┌||┐      ┌─┐
     │  │      │ │
     │  │      │ │
     └──┘      └─┘
    Real capacitor model
\end{verbatim}

\textbf{મેળવણી:}

{\def\LTcaptype{none} % do not increment counter
\begin{longtable}[]{@{}
  >{\raggedright\arraybackslash}p{(\linewidth - 4\tabcolsep) * \real{0.2069}}
  >{\raggedright\arraybackslash}p{(\linewidth - 4\tabcolsep) * \real{0.4483}}
  >{\raggedright\arraybackslash}p{(\linewidth - 4\tabcolsep) * \real{0.3448}}@{}}
\toprule\noalign{}
\begin{minipage}[b]{\linewidth}\raggedright
પગલું
\end{minipage} & \begin{minipage}[b]{\linewidth}\raggedright
વર્ણન
\end{minipage} & \begin{minipage}[b]{\linewidth}\raggedright
સમીકરણ
\end{minipage} \\
\midrule\noalign{}
\endhead
\bottomrule\noalign{}
\endlastfoot
1 & સંગ્રહિત ઊર્જા વ્યાખ્યાયિત કરો & Estored = CV^{2}/2 \\
2 & સાયકલ દીઠ ઊર્જા લોસ વ્યાખ્યાયિત કરો & Eloss = πCV^{2}/ωCR = πV^{2}/ωR \\
3 & Q ફેક્ટર વ્યાખ્યાયિત કરો & Q = 2π \times Estored / Eloss \\
4 & સબસ્ટિટ્યૂટ કરો અને સિમ્પ્લિફાય કરો &

Q = 2π \times (CV^{2}/2) \div (πV^{2}/ωR) = ωCR \\

\end{longtable}
}

\textbf{ફાઈનલ ઈક્વેશન:} Q = ωCR = 1/(ωRC) = 1/tanδ

જ્યાં:

\begin{itemize}
\tightlist
\item
  ω = એન્ગ્યુલર ફ્રિક્વન્સી (2πf)
\item
  R = ઇક્વિવેલન્ટ સિરીઝ રેઝિસ્ટન્સ (ESR)
\item
  C = કેપેસિટન્સ
\item
  tanδ = ડિસિપેશન ફેક્ટર
\end{itemize}

\end{solutionbox}
\begin{mnemonicbox}
``CORE'' - Capacitors' Quality equals One over
Resistance times Capacitance

\end{mnemonicbox}
\subsection*{પ્રશ્ન 4(ક OR) [7
માર્ક્સ]}\label{uxaaauxab0uxab6uxaa8-4uxa95-or-7-uxaaeuxab0uxa95uxab8}

\textbf{પેરેલલ રેઝોનન્સ સર્કિટની રેઝોનન્સ ફ્રિક્વન્સીનું સમીકરણ તારવો. R=30Ω, L=1H,
C=1μF સાથે પેરેલલ RLC સર્કિટની રેઝોનન્ટ ફ્રિક્વન્સી, Q ફેક્ટર અને બેન્ડવિડ્થની ગણતરી
કરો.}

\begin{solutionbox}

\textbf{આકૃતિ:}

\begin{verbatim}
             L
    ┌────┬─OOOO─┬────┐
    │    │      │    │
 V ┘   R │      │ C  │
    │    │      │    │
    └────┴──────┴────┘
\end{verbatim}

\textbf{મેળવણી:}

{\def\LTcaptype{none} % do not increment counter
\begin{longtable}[]{@{}lll@{}}
\toprule\noalign{}
પગલું & વર્ણન & સમીકરણ \\
\midrule\noalign{}
\endhead
\bottomrule\noalign{}
\endlastfoot
1 & પેરેલલ RLC ની એડમિટન્સ & Y = 1/R + 1/jωL + jωC \\
2 & રેઝોનન્સ પર, Im(Y) = 0 & 1/jωL + jωC = 0 \\
3 & રેઝોનન્ટ ફ્રિક્વન્સી માટે ઉકેલો & ω_{0} = 1/\sqrt(LC) અથવા f_{0} = 1/(2π\sqrt(LC)) \\
\end{longtable}
}

\textbf{ગણતરીઓ:}

{\def\LTcaptype{none} % do not increment counter
\begin{longtable}[]{@{}
  >{\raggedright\arraybackslash}p{(\linewidth - 6\tabcolsep) * \real{0.2683}}
  >{\raggedright\arraybackslash}p{(\linewidth - 6\tabcolsep) * \real{0.2195}}
  >{\raggedright\arraybackslash}p{(\linewidth - 6\tabcolsep) * \real{0.3171}}
  >{\raggedright\arraybackslash}p{(\linewidth - 6\tabcolsep) * \real{0.1951}}@{}}
\toprule\noalign{}
\begin{minipage}[b]{\linewidth}\raggedright
પેરામીટર
\end{minipage} & \begin{minipage}[b]{\linewidth}\raggedright
ફોર્મ્યુલા
\end{minipage} & \begin{minipage}[b]{\linewidth}\raggedright
ગણતરી
\end{minipage} & \begin{minipage}[b]{\linewidth}\raggedright
પરિણામ
\end{minipage} \\
\midrule\noalign{}
\endhead
\bottomrule\noalign{}
\endlastfoot
રેઝોનન્ટ ફ્રિક્વન્સી & f_{0} = 1/(2π\sqrt(LC)) & f_{0} = 1/(2π\sqrt(1\times10^{-}^{6})) & 159.15 Hz \\
Q ફેક્ટર &

Q = R/ω_{0}L &

Q = 30/(2π\times159.15\times1) & 0.03 \\

બેન્ડવિડ્થ & BW = f_{0}/Q & BW = 159.15/0.03 & 5305 Hz \\
\end{longtable}
}

\end{solutionbox}
\begin{mnemonicbox}
``FPQB'' - Frequency from Parallel elements, Q from
Resistance divided by reactance, Bandwidth from division

\end{mnemonicbox}
\subsection*{પ્રશ્ન 5(અ) [3
માર્ક્સ]}\label{uxaaauxab0uxab6uxaa8-5uxa85-3-uxaaeuxab0uxa95uxab8}

\textbf{T પ્રકાર એટેન્યુએટર સમજાવો.}

\begin{solutionbox}

\textbf{આકૃતિ:}

\begin{verbatim}
    Z1
    ┌─/{//─┐}
    │        │
 In ┘    Z3  │      Out
    │   /{//│       │}
    └─/{//─┘       │}
        Z2
\end{verbatim}

\textbf{T-પ્રકાર એટેન્યુએટર}: T કોન્ફિગરેશનમાં સિગ્નલની એમ્પ્લિટ્યુડ ઘટાડવા માટે
વપરાતું પેસિવ નેટવર્ક.

{\def\LTcaptype{none} % do not increment counter
\begin{longtable}[]{@{}lll@{}}
\toprule\noalign{}
કમ્પોનન્ટ & વર્ણન & ફોર્મ્યુલા \\
\midrule\noalign{}
\endhead
\bottomrule\noalign{}
\endlastfoot
\textbf{Z1, Z2} & સિરીઝ આર્મ્સ & Z1 = Z2 = Z_{0}(N-1)/(N+1) \\
\textbf{Z3} & શન્ટ આર્મ & Z3 = 2Z_{0}/(N^{2}-1) \\
\textbf{N} & એટેન્યુએશન રેશિયો & N = 10\^{}(dB/20) \\
\end{longtable}
}

\begin{itemize}
\tightlist
\item
  \textbf{લક્ષણ}: મેચ્ડ સોર્સ અને લોડ માટે સિમેટ્રિકલ
\item
  \textbf{ઉપયોગો}: સિગ્નલ લેવલ કંટ્રોલ, ઇમ્પિડન્સ મેચિંગ
\item
  \textbf{ફાયદો}: યોગ્ય ડિઝાઇન સાથે ઇમ્પિડન્સ મેચિંગ જાળવે છે
\end{itemize}

\end{solutionbox}
\begin{mnemonicbox}
``TSAR'' - T-shape with Series Arms and Resistance
in middle

\end{mnemonicbox}
\subsection*{પ્રશ્ન 5(બ) [4
માર્ક્સ]}\label{uxaaauxab0uxab6uxaa8-5uxaac-4-uxaaeuxab0uxa95uxab8}

\textbf{વિવિધ પેસિવ ફિલ્ટર સર્કિટસનું વર્ગીકરણ કરો.}

\begin{solutionbox}

\textbf{આકૃતિ:}

\begin{center}
\textbf{Mermaid Diagram (Code)}
\begin{verbatim}
{Shaded}
{Highlighting}[]
graph TD
    A[પેસિવ ફિલ્ટર્સ]
    A {-{-}{} B[ફ્રિક્વન્સી રિસ્પોન્સ આધારિત]}
    A {-{-}{} C[કોન્ફિગરેશન આધારિત]}

    B {-{-}{} D[લો પાસ]}
    B {-{-}{} E[હાઇ પાસ]}
    B {-{-}{} F[બેન્ડ પાસ]}
    B {-{-}{} G[બેન્ડ સ્ટોપ]}
    
    C {-{-}{} H[T{-}સેક્શન]}
    C {-{-}{} I[π{-}સેક્શન]}
    C {-{-}{} J[L{-}સેક્શન]}
    C {-{-}{} K[લેટિસ]}
{Highlighting}
{Shaded}
\end{verbatim}
\end{center}

{\def\LTcaptype{none} % do not increment counter
\begin{longtable}[]{@{}
  >{\raggedright\arraybackslash}p{(\linewidth - 6\tabcolsep) * \real{0.2453}}
  >{\raggedright\arraybackslash}p{(\linewidth - 6\tabcolsep) * \real{0.1887}}
  >{\raggedright\arraybackslash}p{(\linewidth - 6\tabcolsep) * \real{0.3019}}
  >{\raggedright\arraybackslash}p{(\linewidth - 6\tabcolsep) * \real{0.2642}}@{}}
\toprule\noalign{}
\begin{minipage}[b]{\linewidth}\raggedright
ફિલ્ટર પ્રકાર
\end{minipage} & \begin{minipage}[b]{\linewidth}\raggedright
કાર્ય
\end{minipage} & \begin{minipage}[b]{\linewidth}\raggedright
ટિપિકલ સર્કિટ
\end{minipage} & \begin{minipage}[b]{\linewidth}\raggedright
ઉપયોગો
\end{minipage} \\
\midrule\noalign{}
\endhead
\bottomrule\noalign{}
\endlastfoot
\textbf{લો પાસ} & નીચી ફ્રિક્વન્સી પસાર કરે & RC, RL સર્કિટ્સ & ઓડિયો ફિલ્ટર્સ,
પાવર સપ્લાય \\
\textbf{હાઇ પાસ} & ઊંચી ફ્રિક્વન્સી પસાર કરે & CR, LR સર્કિટ્સ & નોઇઝ ફિલ્ટરિંગ,
સિગ્નલ કન્ડિશનિંગ \\
\textbf{બેન્ડ પાસ} & ફ્રિક્વન્સીનો બેન્ડ પસાર કરે & RLC સર્કિટ્સ & રેડિયો ટ્યુનિંગ,
સિગ્નલ સિલેક્શન \\
\textbf{બેન્ડ સ્ટોપ} & ફ્રિક્વન્સીનો બેન્ડ બ્લોક કરે & પેરેલલ RLC & ઇન્ટરફેરન્સ
રિજેક્શન \\
\end{longtable}
}

\end{solutionbox}
\begin{mnemonicbox}
``LHBB'' - Low High Band Band filters for Pass and
Block

\end{mnemonicbox}
\subsection*{પ્રશ્ન 5(ક) [7
માર્ક્સ]}\label{uxaaauxab0uxab6uxaa8-5uxa95-7-uxaaeuxab0uxa95uxab8}

\textbf{કટ ઓફ ફ્રિક્વન્સી=1000Hz અને 500Ω લોડ ધરાવતા T-section સાથે કોન્સ્ટન્ટ-k
ટાઈપ લો પાસ અને હાઇ પાસ ફિલ્ટર ડિઝાઈન કરો.}

\begin{solutionbox}

\textbf{આકૃતિ:}

\begin{verbatim}
Low Pass T{-Filter          High Pass T{-}Filter}
     L/2       L/2              C/2       C/2
   {-{-}OOOO{-}{-}{-}{-}{-}{-}OOOO{-}{-}        {-}{-}{-}||{-}{-}{-}{-}{-}{-}{-}{-}||{-}{-}{-}}
   |                |        |                |
   |                |        |                |
   |       C        |        |       L        |
   |       ||       |        |      OOOO      |
   |                |        |                |
 {-{-}{-}{-}{-}{-}{-}{-}{-}{-}{-}{-}{-}{-}{-}{-}{-}{-}{-}{-}{-}{-}    {-}{-}{-}{-}{-}{-}{-}{-}{-}{-}{-}{-}{-}{-}{-}{-}{-}{-}{-}{-}{-}{-}}
\end{verbatim}

\textbf{ડિઝાઇન ગણતરીઓ:}

કોન્સ્ટન્ટ-k T-ટાઇપ લો પાસ ફિલ્ટર માટે:

{\def\LTcaptype{none} % do not increment counter
\begin{longtable}[]{@{}llll@{}}
\toprule\noalign{}
પેરામીટર & ફોર્મ્યુલા & ગણતરી & મૂલ્ય \\
\midrule\noalign{}
\endhead
\bottomrule\noalign{}
\endlastfoot
કટ-ઓફ ફ્રિક્વન્સી & fc = 1000 Hz & આપેલ & 1000 Hz \\
લોડ ઇમ્પિડન્સ & R_{0} = 500 Ω & આપેલ & 500 Ω \\
સિરીઝ ઇન્ડક્ટર &

L = R_{0}/πfc &

L = 500/(π\times1000) & 159.15 mH \\

હાલ્ફ સેક્શન્સ & L/2 & 159.15/2 & 79.58 mH \\
શન્ટ કેપેસિટર &

C = 1/(πfcR_{0}) &

C = 1/(π\times1000\times500) & 0.636 μF \\

\end{longtable}
}

કોન્સ્ટન્ટ-k T-ટાઇપ હાઇ પાસ ફિલ્ટર માટે:

{\def\LTcaptype{none} % do not increment counter
\begin{longtable}[]{@{}llll@{}}
\toprule\noalign{}
પેરામીટર & ફોર્મ્યુલા & ગણતરી & મૂલ્ય \\
\midrule\noalign{}
\endhead
\bottomrule\noalign{}
\endlastfoot
સિરીઝ કેપેસિટર &

C = 1/(4πfcR_{0}) &

C = 1/(4π\times1000\times500) & 0.159 μF \\

હાલ્ફ સેક્શન્સ & C/2 & 0.159/2 & 0.0795 μF \\
શન્ટ ઇન્ડક્ટર &

L = R_{0}/(4πfc) &

L = 500/(4π\times1000) & 39.79 mH \\

\end{longtable}
}

\end{solutionbox}
\begin{mnemonicbox}
``FRED'' - Frequency Ratio determines Element
Dimensions

\end{mnemonicbox}
\subsection*{પ્રશ્ન 5(અ OR) [3
માર્ક્સ]}\label{uxaaauxab0uxab6uxaa8-5uxa85-or-3-uxaaeuxab0uxa95uxab8}

\textbf{π પ્રકાર એટેન્યુએટર સમજાઓ.}

\begin{solutionbox}

\textbf{આકૃતિ:}

\begin{verbatim}
          Z2
          │
    ┌────┐│┌────┐
    │    ││     │
 In ┘  Z1│││Z3  │      Out
    │    ││     │
    └────┘│└────┘
          │
\end{verbatim}

\textbf{π-પ્રકાર એટેન્યુએટર}: π કોન્ફિગરેશનમાં સિગ્નલની એમ્પ્લિટ્યુડ ઘટાડવા માટે
વપરાતું પેસિવ નેટવર્ક.

{\def\LTcaptype{none} % do not increment counter
\begin{longtable}[]{@{}lll@{}}
\toprule\noalign{}
કમ્પોનન્ટ & વર્ણન & ફોર્મ્યુલા \\
\midrule\noalign{}
\endhead
\bottomrule\noalign{}
\endlastfoot
\textbf{Z2} & સિરીઝ આર્મ & Z2 = 2Z_{0}/(N^{2}-1) \\
\textbf{Z1, Z3} & શન્ટ આર્મ્સ & Z1 = Z3 = Z_{0}(N+1)/(N-1) \\
\textbf{N} & એટેન્યુએશન રેશિયો & N = 10\^{}(dB/20) \\
\end{longtable}
}

\begin{itemize}
\tightlist
\item
  \textbf{લક્ષણ}: મેચ્ડ સોર્સ અને લોડ માટે સિમેટ્રિકલ
\item
  \textbf{ઉપયોગો}: સિગ્નલ લેવલ કંટ્રોલ, ઇમ્પિડન્સ મેચિંગ
\item
  \textbf{ફાયદો}: ઇનપુટ અને આઉટપુટ વચ્ચે સારું આઇસોલેશન
\end{itemize}

\end{solutionbox}
\begin{mnemonicbox}
``PASS'' - Pi-Attenuator has Series in middle and
Shunt arms outside

\end{mnemonicbox}
\subsection*{પ્રશ્ન 5(બ OR) [4
માર્ક્સ]}\label{uxaaauxab0uxab6uxaa8-5uxaac-or-4-uxaaeuxab0uxa95uxab8}

\textbf{વિવિધ પ્રકારના એટેન્યુએટરનું વર્ગીકરણ કરો.}

\begin{solutionbox}

\textbf{આકૃતિ:}

\begin{center}
\textbf{Mermaid Diagram (Code)}
\begin{verbatim}
{Shaded}
{Highlighting}[]
graph TD
    A[એટેન્યુએટર્સ]
    A {-{-}{} B[સ્ટ્રક્ચર આધારિત]}
    A {-{-}{} C[ફંક્શન આધારિત]}

    B {-{-}{} D[T{-}પ્રકાર]}
    B {-{-}{} E[π{-}પ્રકાર]}
    B {-{-}{} F[L{-}પ્રકાર]}
    B {-{-}{} G[બ્રિજ્ડ{-}T]}
    B {-{-}{} H[લેટિસ]}
    
    C {-{-}{} I[ફિક્સ્ડ]}
    C {-{-}{} J[વેરિએબલ]}
    C {-{-}{} K[સ્ટેપ્ડ]}
    C {-{-}{} L[પ્રોગ્રામેબલ]}
{Highlighting}
{Shaded}
\end{verbatim}
\end{center}

{\def\LTcaptype{none} % do not increment counter
\begin{longtable}[]{@{}llll@{}}
\toprule\noalign{}
એટેન્યુએટર પ્રકાર & લક્ષણો & ઉપયોગો & ફાયદા \\
\midrule\noalign{}
\endhead
\bottomrule\noalign{}
\endlastfoot
\textbf{T-પ્રકાર} & સિરીઝ-શન્ટ-સિરીઝ & ઓડિયો સિસ્ટમ્સ & સરળ ડિઝાઇન \\
\textbf{π-પ્રકાર} & શન્ટ-સિરીઝ-શન્ટ & RF સર્કિટ્સ & વધુ સારું આઇસોલેશન \\
\textbf{L-પ્રકાર} & સિરીઝ-શન્ટ & સરળ મેચિંગ & ઇમ્પિડન્સ ટ્રાન્સફોર્મેશન \\
\textbf{બ્રિજ્ડ-T} & બેલેન્સ્ડ સ્ટ્રક્ચર & ટેસ્ટ ઇક્વિપમેન્ટ & મિનિમલ ડિસ્ટોર્શન \\
\textbf{બેલેન્સ્ડ} & સિમેટ્રિક ડ્યુઅલ પાથ & ડિફરેન્શિયલ સિગ્નલ્સ & કોમન મોડ
રિજેક્શન \\
\end{longtable}
}

\end{solutionbox}
\begin{mnemonicbox}
``TPLBV'' - T, Pi, L, Bridged-T, and Variable
attenuators

\end{mnemonicbox}
\subsection*{પ્રશ્ન 5(ક OR) [7
માર્ક્સ]}\label{uxaaauxab0uxab6uxaa8-5uxa95-or-7-uxaaeuxab0uxa95uxab8}

\textbf{40dBનું એટેન્યુએશન આપવા અને 500Ω ના લોડમાં કામ કરવા માટે સપ્રમાણ T
પ્રકારના એટેન્યુએટર અને π પ્રકારનું એટેન્યુએટર ડિઝાઇન કરો.}

\begin{solutionbox}

\textbf{આકૃતિ:}

\begin{verbatim}
T{-type Attenuator           π{-}type Attenuator}
     R1       R1                     R2
   {-{-}//{-}{-}{-}{-}//{-}{-}              {-}{-}//{-}{-}}
   |               |            |        |
   |               |            |        |
   |       R2      |            R1      R1
   |      /{/     |           //    //}
   |               |            |        |
   {-{-}{-}{-}{-}{-}{-}{-}{-}{-}{-}{-}{-}{-}{-}{-}{-}            {-}{-}{-}{-}{-}{-}{-}{-}{-}{-}}
\end{verbatim}

\textbf{ડિઝાઇન ગણતરીઓ:}

{\def\LTcaptype{none} % do not increment counter
\begin{longtable}[]{@{}llll@{}}
\toprule\noalign{}
પગલું & ફોર્મ્યુલા & ગણતરી & મૂલ્ય \\
\midrule\noalign{}
\endhead
\bottomrule\noalign{}
\endlastfoot
આપેલ & એટેન્યુએશન = 40 dB & - & 40 dB \\
પગલું 1 & N = 10\^{}(dB/20) & 10\^{}(40/20) & 100 \\
પગલું 2 & K = (N-1)/(N+1) & (100-1)/(100+1) & 0.98 \\
\end{longtable}
}

T-પ્રકાર એટેન્યુએટર માટે:

{\def\LTcaptype{none} % do not increment counter
\begin{longtable}[]{@{}llll@{}}
\toprule\noalign{}
કમ્પોનન્ટ & ફોર્મ્યુલા & ગણતરી & મૂલ્ય \\
\midrule\noalign{}
\endhead
\bottomrule\noalign{}
\endlastfoot
R_{1} (સિરીઝ) & Z_{0}·K & 500 \times 0.98 & 490 Ω \\
R_{2} (શન્ટ) & Z_{0}/(K·(N-K)) & 500/(0.98\times(100-0.98)) & 5.15 Ω \\
\end{longtable}
}

π-પ્રકાર એટેન્યુએટર માટે:

{\def\LTcaptype{none} % do not increment counter
\begin{longtable}[]{@{}llll@{}}
\toprule\noalign{}
કમ્પોનન્ટ & ફોર્મ્યુલા & ગણતરી & મૂલ્ય \\
\midrule\noalign{}
\endhead
\bottomrule\noalign{}
\endlastfoot
R_{1} (શન્ટ) & Z_{0}/K & 500/0.98 & 510.2 Ω \\
R_{2} (સિરીઝ) & Z_{0}·K·(N-K) & 500 \times 0.98 \times (100-0.98) & 48,541 Ω \\
\end{longtable}
}

\end{solutionbox}
\begin{mnemonicbox}
``DANK'' - dB Attenuation is Number K, which
determines resistor values

\end{mnemonicbox}

\end{document}
