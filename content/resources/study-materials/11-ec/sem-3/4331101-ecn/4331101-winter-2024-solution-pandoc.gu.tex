\documentclass[10pt,a4paper]{article}

% content/resources/templates/preamble.tex
\usepackage[margin=0.6in]{geometry}
\author{Milav Dabgar}
\usepackage{amsmath,amssymb,amsthm}
\usepackage{booktabs}
\usepackage{multirow}
\usepackage{xcolor}
\usepackage{tcolorbox}
\tcbuselibrary{breakable,skins}
\usepackage[colorlinks=true,linkcolor=blue]{hyperref}
\usepackage{titlesec}
\usepackage{enumitem}
\usepackage{tikz}
\usepackage{pgfplots}
\usepackage{circuitikz}
\usepackage[version=4]{mhchem}
\usepackage{longtable}
\usepackage{array}
\usepackage{float}
\usepackage{caption}
\usepackage{listings}

\lstset{
  basicstyle=\small\ttfamily,
  breaklines=true,
  breakatwhitespace=false,
  postbreak=\mbox{\textcolor{red}{$\hookrightarrow$}\space},
  float=false,
  numbers=left,
  numberstyle=\tiny\color{gray},
  numbersep=10pt,
  xleftmargin=2em,
  keywordstyle=\color{blue},
  commentstyle=\color{green!60!black},
  stringstyle=\color{purple},
  backgroundcolor=\color{gray!5},
  showstringspaces=false,
  tabsize=2,
  captionpos=b,
  keepspaces=true,
  columns=flexible
}

\pgfplotsset{compat=1.18}
\usetikzlibrary{shapes,arrows,positioning,calc,patterns,decorations.pathmorphing,decorations.markings,arrows.meta}

% Color scheme
\definecolor{headcolor}{RGB}{0,102,204}
\definecolor{keycolor}{RGB}{220,20,60}
\definecolor{solutioncolor}{RGB}{34,139,34}
\definecolor{mnemoniccolor}{RGB}{148,0,211}
\definecolor{codecolor}{RGB}{0,0,100}

% Spacing
\setlength{\parskip}{3pt}
\setlist[itemize]{nosep}
\setlist[enumerate]{nosep}

% Title formatting
\titleformat{\section}{\Large\bfseries\color{headcolor}}{\thesection}{1em}{}
\titleformat{\subsection}{\large\bfseries\color{headcolor}}{\thesubsection}{1em}{}

% Pandoc tightlist compatibility
\providecommand{\tightlist}{%
  \setlength{\itemsep}{0pt}\setlength{\parskip}{0pt}}

% Pandoc longtable compatibility
\newcounter{none}
\def\thenone{}


% content/resources/templates/gujarati-boxes.tex
\usepackage{fontspec}
\usepackage{polyglossia}

% Set Gujarati as main language (document is primarily in Gujarati)
% Note: gloss-gujarati.ldf doesn't exist in polyglossia, but it will use hyphenation patterns
\setdefaultlanguage{gujarati}
\setotherlanguage{english}

% Configure Gujarati font properly
% Use Language=Default to prevent polyglossia from trying to add language-specific features
% that don't exist for Gujarati, which causes "empty feature" warnings
\newfontfamily\gujaratifont[Script=Gujarati,AutoFakeBold=2.5,AutoFakeSlant=0.3]{Noto Sans Gujarati}
\setmainfont[Script=Gujarati,AutoFakeBold=2.5,AutoFakeSlant=0.3]{Noto Sans Gujarati}
% Use Noto Sans Gujarati for monospace to support Gujarati in text
\setmonofont[Scale=0.9]{Noto Sans Gujarati}

% Configure English to use the same font
\newfontfamily\englishfont[Script=Gujarati,AutoFakeBold=2.5,AutoFakeSlant=0.3]{Noto Sans Gujarati}

% Translations for polyglossia
\gappto\captionsgujarati{
  \renewcommand{\tablename}{કોષ્ટક}
  \renewcommand{\figurename}{આકૃતિ}
}

% Helper for TikZ nodes to ensure Gujarati font
\newcommand{\gu}[1]{{\gujaratifont #1}}

% Custom environments
\newtcolorbox{solutionbox}{
    breakable,
    enhanced,
    colback=solutioncolor!5!white,
    colframe=solutioncolor!75!black,
    fonttitle=\bfseries,
    title=જવાબ
}

\newtcolorbox{solutionboxnobreak}{
 colback=solutioncolor!5!white,
 colframe=solutioncolor!75!black,
 fonttitle=\bfseries,
 title=જવાબ
}

\newtcolorbox{keyformula}{
 breakable,
 enhanced,
 colback=keycolor!5!white,
 colframe=keycolor!75!black,
 fonttitle=\bfseries,
 title=રાસાયણિક સમીકરણ/સૂત્ર
}

\newtcolorbox{mnemonicbox}{
 breakable,
 enhanced,
 colback=mnemoniccolor!5!white,
 colframe=mnemoniccolor!75!black,
 fonttitle=\bfseries,
 title=મેમરી ટ્રીક
}


\begin{document}

\begin{center}
{\Huge\bfseries\color{headcolor} Subject Name (Gujarati)}\\[5pt]
{\LARGE 4331101 -- Winter 2024}\\[3pt]
{\large Semester 1 Study Material}\\[3pt]
{\normalsize\textit{Detailed Solutions and Explanations}}
\end{center}

\vspace{10pt}

\subsection*{પ્રશ્ન 1(a) [3
ગુણ]}\label{q1a}

\textbf{ઇલેક્ટ્રોનીક નેટવર્ક માટે (i) નોડ (ii) બ્રાંચ અને (iii) લૂપ ની વ્યાખ્યા
આપો.}

\begin{solutionbox}

\textbf{નોડ}:

\begin{itemize}
\tightlist
\item
  \textbf{જંક્શન પોઈન્ટ} જ્યાં બે અથવા વધુ બ્રાંચ નેટવર્કમાં મળે છે
\item
  એવા બિંદુઓ જ્યાં ઘટકો જોડાયેલા હોય છે
\item
  નોડ પર બધી બ્રાંચોનો કરંટ સરવાળો શૂન્ય થાય છે
\end{itemize}

\textbf{બ્રાંચ}:

\begin{itemize}
\tightlist
\item
  \textbf{સિંગલ ઘટક} (R, L, અથવા C) અથવા બે નોડ્સને જોડતો પાથ
\item
  દરેક બ્રાંચમાં એક ચોક્કસ કરંટ વહે છે
\item
  એક્ટીવ બ્રાંચમાં સોર્સ હોય છે; પેસિવ બ્રાંચમાં R, L, C હોય છે
\end{itemize}

\textbf{લૂપ}:

\begin{itemize}
\tightlist
\item
  નેટવર્કમાં જોડાયેલા બ્રાંચોથી બનતો \textbf{બંધ પાથ}
\item
  કોઈ નોડ એક કરતાં વધુ વખત આવતું નથી
\item
  નેટવર્ક ઉકેલવા માટે લૂપ એનાલિસિસમાં વપરાય છે
\end{itemize}

\end{solutionbox}
\begin{mnemonicbox}
``NBL: નોડ્સ જોડાય, બ્રાંચેસ કનેક્ટ, લૂપ્સ સર્કલ''

\end{mnemonicbox}
\subsection*{પ્રશ્ન 1(b) [4
ગુણ]}\label{q1b}

\textbf{200 Ω, 300 Ω અને 500 Ω ના રેઝીસ્ટર 100 V ના સપ્લાય સાથે પેરેલલમાં
જોડાયેલા છે. તો (i) દરેક રેઝીસ્ટરમાંથી પસાર થતો કરંટ તથા કુલ કરંટ (ii) ઇક્વીવેલન્ટ
રેઝીસ્ટર શોધો.}

\begin{solutionbox}

\textbf{ગણતરીઓનું કોષ્ટક:}

{\def\LTcaptype{none} % do not increment counter
\begin{longtable}[]{@{}llll@{}}
\toprule\noalign{}
પેરામીટર & ફોર્મ્યુલા & ગણતરી & પરિણામ \\
\midrule\noalign{}
\endhead
\bottomrule\noalign{}
\endlastfoot
I_{1} (200Ω) & I = V/R & 100V/200Ω & 0.5A \\
I_{2} (300Ω) & I = V/R & 100V/300Ω & 0.333A \\
I_{3} (500Ω) & I = V/R & 100V/500Ω & 0.2A \\
I_{(}_{t}_{o}_{t}_{a}_{l}_{)} & I_{1}+I_{2}+I_{3} & 0.5+0.333+0.2 & 1.033A \\
R_{(}_{e}q_{)} & 1/R_{(}_{e}q_{)} = 1/R_{1}+1/R_{2}+1/R_{3} & 1/200+1/300+1/500 & 96.77Ω \\
\end{longtable}
}

\end{solutionbox}
\begin{mnemonicbox}
``પેરેલલ પાથ કરંટને અવરોધના વ્યસ્ત પ્રમાણમાં વહેંચે છે''

\end{mnemonicbox}
\subsection*{પ્રશ્ન 1(c) [7
ગુણ]}\label{q1c}

\textbf{કેપેસીટર માટે સિરીઝ અને પેરેલલ જોડાણ સમજાવો.}

\begin{solutionbox}

\textbf{સિરીઝમાં કેપેસીટર:}

\begin{center}
\textbf{Mermaid Diagram (Code)}
\begin{verbatim}
{Shaded}
{Highlighting}[]
graph LR
    A["{+"] {-}{-}{-} B[C_{1}] {-}{-}{-} C[C_{2}] {-}{-}{-} D[C_{3}] {-}{-}{-} E["{}{-}"]}
{Highlighting}
{Shaded}
\end{verbatim}
\end{center}


{\def\LTcaptype{none} % do not increment counter
\vspace{-5pt}
\captionof{table}{સિરીઝ કેપેસીટરોની વિશેષતાઓ}
\vspace{-10pt}
\begin{longtable}[]{@{}
  >{\raggedright\arraybackslash}p{(\linewidth - 4\tabcolsep) * \real{0.3125}}
  >{\raggedright\arraybackslash}p{(\linewidth - 4\tabcolsep) * \real{0.2812}}
  >{\raggedright\arraybackslash}p{(\linewidth - 4\tabcolsep) * \real{0.4062}}@{}}
\toprule\noalign{}
\begin{minipage}[b]{\linewidth}\raggedright
વિશેષતા
\end{minipage} & \begin{minipage}[b]{\linewidth}\raggedright
ફોર્મ્યુલા
\end{minipage} & \begin{minipage}[b]{\linewidth}\raggedright
વર્ણન
\end{minipage} \\
\midrule\noalign{}
\endhead
\bottomrule\noalign{}
\endlastfoot
ઇક્વિવેલન્ટ કેપેસિટન્સ & 1/C_{(}_{e}q_{)} = 1/C_{1} + 1/C_{2} + 1/C_{3} & હંમેશા નાનામાં નાના
કેપેસિટર કરતાં નાનું \\
ચાર્જ &

Q = Q_{1} = Q_{2} = Q_{3} & બધા કેપેસિટર પર સરખો \\

વોલ્ટેજ & V = V_{1} + V_{2} + V_{3} & 1/C ના રેશિયો પ્રમાણે વહેંચાય છે \\
ઊર્જા & E = CV^{2}/2 & કેપેસિટર્સમાં વહેંચાયેલી \\
\end{longtable}
}

\textbf{પેરેલલમાં કેપેસીટર:}

\begin{center}
\textbf{Mermaid Diagram (Code)}
\begin{verbatim}
{Shaded}
{Highlighting}[]
graph LR
    A["{+"] {-}{-}{-} B["{}+"]}
    B {-{-}{-} C[C_{1}] {-}{-}{-} D["{}{-}"]}
    B {-{-}{-} E[C_{2}] {-}{-}{-} D}
    B {-{-}{-} F[C_{3}] {-}{-}{-} D}
    A {-{-}{-} D}
{Highlighting}
{Shaded}
\end{verbatim}
\end{center}


{\def\LTcaptype{none} % do not increment counter
\vspace{-5pt}
\captionof{table}{પેરેલલ કેપેસીટરોની વિશેષતાઓ}
\vspace{-10pt}
\begin{longtable}[]{@{}lll@{}}
\toprule\noalign{}
વિશેષતા & ફોર્મ્યુલા & વર્ણન \\
\midrule\noalign{}
\endhead
\bottomrule\noalign{}
\endlastfoot
ઇક્વિવેલન્ટ કેપેસિટન્સ & C_{(}_{e}q_{)} = C_{1} + C_{2} + C_{3} & વ્યક્તિગત કેપેસિટન્સનો સરવાળો \\
ચાર્જ & Q = Q_{1} + Q_{2} + Q_{3} & C ની કિંમત અનુસાર વહેંચાય છે \\
વોલ્ટેજ &

V = V_{1} = V_{2} = V_{3} & બધા કેપેસિટર પર સરખો \\

ઊર્જા & E = CV^{2}/2 & વ્યક્તિગત ઊર્જાનો સરવાળો \\
\end{longtable}
}

\end{solutionbox}
\begin{mnemonicbox}
``સિરીઝ કેપ્સમાં વ્યસ્ત સરવાળો, પેરેલલ કેપ્સમાં સીધો સરવાળો''

\end{mnemonicbox}
\subsection*{પ્રશ્ન 1(c) OR [7
ગુણ]}\label{q1c}

\textbf{ઇન્ડક્ટર માટે સિરીઝ અને પેરેલલ જોડાણ સમજાવો.}

\begin{solutionbox}

\textbf{સિરીઝમાં ઇન્ડક્ટર:}

\begin{center}
\textbf{Mermaid Diagram (Code)}
\begin{verbatim}
{Shaded}
{Highlighting}[]
graph LR
    A["{+"] {-}{-}{-} B[L_{1}] {-}{-}{-} C[L_{2}] {-}{-}{-} D[L_{3}] {-}{-}{-} E["{}{-}"]}
{Highlighting}
{Shaded}
\end{verbatim}
\end{center}


{\def\LTcaptype{none} % do not increment counter
\vspace{-5pt}
\captionof{table}{સિરીઝ ઇન્ડક્ટરોની વિશેષતાઓ}
\vspace{-10pt}
\begin{longtable}[]{@{}lll@{}}
\toprule\noalign{}
વિશેષતા & ફોર્મ્યુલા & વર્ણન \\
\midrule\noalign{}
\endhead
\bottomrule\noalign{}
\endlastfoot
ઇક્વિવેલન્ટ ઇન્ડક્ટન્સ & L_{(}_{e}q_{)} = L_{1} + L_{2} + L_{3} & વ્યક્તિગત ઇન્ડક્ટન્સનો સરવાળો \\
કરંટ &

I = I_{1} = I_{2} = I_{3} & બધા ઇન્ડક્ટર પર સરખો \\

વોલ્ટેજ & V = V_{1} + V_{2} + V_{3} & L ના રેશિયો અનુસાર વહેંચાય છે \\
ઊર્જા & E = LI^{2}/2 & વ્યક્તિગત ઊર્જાનો સરવાળો \\
\end{longtable}
}

\textbf{પેરેલલમાં ઇન્ડક્ટર:}

\begin{center}
\textbf{Mermaid Diagram (Code)}
\begin{verbatim}
{Shaded}
{Highlighting}[]
graph LR
    A["{+"] {-}{-}{-} B["{}+"]}
    B {-{-}{-} C[L_{1}] {-}{-}{-} D["{}{-}"]}
    B {-{-}{-} E[L_{2}] {-}{-}{-} D}
    B {-{-}{-} F[L_{3}] {-}{-}{-} D}
    A {-{-}{-} D}
{Highlighting}
{Shaded}
\end{verbatim}
\end{center}


{\def\LTcaptype{none} % do not increment counter
\vspace{-5pt}
\captionof{table}{પેરેલલ ઇન્ડક્ટરોની વિશેષતાઓ}
\vspace{-10pt}
\begin{longtable}[]{@{}
  >{\raggedright\arraybackslash}p{(\linewidth - 4\tabcolsep) * \real{0.3125}}
  >{\raggedright\arraybackslash}p{(\linewidth - 4\tabcolsep) * \real{0.2812}}
  >{\raggedright\arraybackslash}p{(\linewidth - 4\tabcolsep) * \real{0.4062}}@{}}
\toprule\noalign{}
\begin{minipage}[b]{\linewidth}\raggedright
વિશેષતા
\end{minipage} & \begin{minipage}[b]{\linewidth}\raggedright
ફોર્મ્યુલા
\end{minipage} & \begin{minipage}[b]{\linewidth}\raggedright
વર્ણન
\end{minipage} \\
\midrule\noalign{}
\endhead
\bottomrule\noalign{}
\endlastfoot
ઇક્વિવેલન્ટ ઇન્ડક્ટન્સ & 1/L_{(}_{e}q_{)} = 1/L_{1} + 1/L_{2} + 1/L_{3} & હંમેશા નાનામાં નાના
ઇન્ડક્ટર કરતાં નાનું \\
કરંટ & I = I_{1} + I_{2} + I_{3} & 1/L ના રેશિયો અનુસાર વહેંચાય છે \\
વોલ્ટેજ &

V = V_{1} = V_{2} = V_{3} & બધા ઇન્ડક્ટર પર સરખો \\

ઊર્જા & E = LI^{2}/2 & ઇન્ડક્ટરોમાં વહેંચાયેલી \\
\end{longtable}
}

\end{solutionbox}
\begin{mnemonicbox}
``સિરીઝ ઇન્ડક્ટરોમાં સીધો સરવાળો, પેરેલલ ઇન્ડક્ટરોમાં વ્યસ્ત
સરવાળો''

\end{mnemonicbox}
\subsection*{પ્રશ્ન 2(a) [3
ગુણ]}\label{q2a}

\textbf{નેટવર્ક એલીમેન્ટને વર્ગીકૃત કરો.}

\begin{solutionbox}


{\def\LTcaptype{none} % do not increment counter
\vspace{-5pt}
\captionof{table}{નેટવર્ક એલીમેન્ટનું વર્ગીકરણ}
\vspace{-10pt}
\begin{longtable}[]{@{}lll@{}}
\toprule\noalign{}
શ્રેણી & પ્રકારો & ઉદાહરણો \\
\midrule\noalign{}
\endhead
\bottomrule\noalign{}
\endlastfoot
\textbf{એક્ટિવ vs પેસિવ} & એક્ટિવ & વોલ્ટેજ/કરંટ સોર્સ, ટ્રાન્ઝિસ્ટર \\
& પેસિવ & રેઝિસ્ટર, કેપેસિટર, ઇન્ડક્ટર \\
\textbf{લિનિયર vs નોન-લિનિયર} & લિનિયર & રેઝિસ્ટર, આદર્શ સોર્સ \\
& નોન-લિનિયર & ડાયોડ, ટ્રાન્ઝિસ્ટર \\
\textbf{બાઇલેટરલ vs યુનિલેટરલ} & બાઇલેટરલ & રેઝિસ્ટર, કેપેસિટર, ઇન્ડક્ટર \\
& યુનિલેટરલ & ડાયોડ, ટ્રાન્ઝિસ્ટર \\
\textbf{લમ્પ્ડ vs ડિસ્ટ્રિબ્યુટેડ} & લમ્પ્ડ & ડિસક્રીટ R, L, C ઘટકો \\
& ડિસ્ટ્રિબ્યુટેડ & ટ્રાન્સમિશન લાઇન \\
\end{longtable}
}

\end{solutionbox}
\begin{mnemonicbox}
``ALBU: એક્ટિવ/પેસિવ, લિનિયર/નોન-લિનિયર,
બાઇલેટરલ/યુનિલેટરલ, લમ્પ્ડ/ડિસ્ટ્રિબ્યુટેડ''

\end{mnemonicbox}
\subsection*{પ્રશ્ન 2(b) [4
ગુણ]}\label{q2b}

\textbf{10, 30 અને 70 ohms ના રેઝીસ્ટર સ્ટારમાં કનેક્ટ કરેલા છે. ડેલ્ટા કનેક્શનનાં
ઇક્વીવેલન્ટ રેઝીસ્ટર શોધો.}

\begin{solutionbox}

\textbf{આકૃતિ: સ્ટાર થી ડેલ્ટા રૂપાંતરણ}

\begin{verbatim}
graph TB
    subgraph Star Connection
        A((1)) {-{-}{-} R1[10Ω]}
        B((2)) {-{-}{-} R2[30Ω]}
        C((3)) {-{-}{-} R3[70Ω]}
        R1 {-{-}{-} D((0))}
        R2 {-{-}{-} D}
        R3 {-{-}{-} D}
    end

    subgraph Delta Connection
        A1((1)) {-{-}{-} R12[R_{1}_{2}]}
        A1 {-{-}{-} R31[R_{3}_{1}]}
        B1((2)) {-{-}{-} R12}
        B1 {-{-}{-} R23[R_{2}_{3}]}
        C1((3)) {-{-}{-} R23}
        C1 {-{-}{-} R31}
    end
\end{verbatim}


{\def\LTcaptype{none} % do not increment counter
\vspace{-5pt}
\captionof{table}{સ્ટાર-ડેલ્ટા રૂપાંતરણ ફોર્મ્યુલા અને ગણતરીઓ}
\vspace{-10pt}
\begin{longtable}[]{@{}llll@{}}
\toprule\noalign{}
ડેલ્ટા રેઝીસ્ટન્સ & ફોર્મ્યુલા & ગણતરી & પરિણામ \\
\midrule\noalign{}
\endhead
\bottomrule\noalign{}
\endlastfoot
R_{1}_{2} & (R_{1}\timesR_{2}+R_{2}\timesR_{3}+R_{3}\timesR_{1})/R_{3} & (10\times30+30\times70+70\times10)/70 & 47.14Ω \\
R_{2}_{3} & (R_{1}\timesR_{2}+R_{2}\timesR_{3}+R_{3}\timesR_{1})/R_{1} & (10\times30+30\times70+70\times10)/10 & 330Ω \\
R_{3}_{1} & (R_{1}\timesR_{2}+R_{2}\timesR_{3}+R_{3}\timesR_{1})/R_{2} & (10\times30+30\times70+70\times10)/30 & 110Ω \\
\end{longtable}
}

\end{solutionbox}
\begin{mnemonicbox}
``સ્ટાર-ડેલ્ટા: ગુણાકારનો સરવાળો વિરુદ્ધ રેઝ

\end{mnemonicbox}
\subsection*{પ્રશ્ન 2(c) [7
ગુણ]}\label{q2c}

\textbf{π નેટવર્ક સમજાવો.}

\begin{solutionbox}

\textbf{આકૃતિ: π (પાઈ) નેટવર્ક}

\begin{verbatim}
       Z1
   o{-{-}{-}{-}www{-}{-}{-}{-}o}
   |           |
   |           |
   |           |
  Z3          Z2
   |           |
   |           |
   o{-{-}{-}{-}{-}o{-}{-}{-}{-}{-}o}
      Ground
\end{verbatim}


{\def\LTcaptype{none} % do not increment counter
\vspace{-5pt}
\captionof{table}{π નેટવર્ક વિશેષતાઓ}
\vspace{-10pt}
\begin{longtable}[]{@{}
  >{\raggedright\arraybackslash}p{(\linewidth - 2\tabcolsep) * \real{0.4583}}
  >{\raggedright\arraybackslash}p{(\linewidth - 2\tabcolsep) * \real{0.5417}}@{}}
\toprule\noalign{}
\begin{minipage}[b]{\linewidth}\raggedright
પેરામીટર
\end{minipage} & \begin{minipage}[b]{\linewidth}\raggedright
વર્ણન
\end{minipage} \\
\midrule\noalign{}
\endhead
\bottomrule\noalign{}
\endlastfoot
\textbf{સ્ટ્રક્ચર} & બે શન્ટ ઇમ્પિડન્સ (Z_{3}, Z_{2}) અને એક સિરીઝ ઇમ્પિડન્સ (Z_{1}) \\
\textbf{ટ્રાન્સમિશન પેરામીટર્સ} & A = 1 + Z_{1}/Z_{2}, B = Z_{1}, C = 1/Z_{2} + 1/Z_{3} +
Z_{1}/(Z_{2}\timesZ_{3}), D = 1 + Z_{1}/Z_{3} \\
\textbf{ઇમ્પિડન્સ પેરામીટર્સ} & Z_{1}_{1} = Z_{1} + Z_{3}, Z_{1}_{2} = Z_{1}, Z_{2}_{1} = Z_{1}, Z_{2}_{2} = Z_{1}
+ Z_{2} \\
\textbf{ઇમેજ ઇમ્પિડન્સ} & Z_{0}π = \sqrt(Z_{1}Z_{2}Z_{3}/(Z_{2}+Z_{3})) \\
\textbf{એપ્લિકેશન} & મેચિંગ નેટવર્ક, ફિલ્ટર, એટેન્યુએટર \\
\textbf{રૂપાંતરણ} & T-નેટવર્કમાં રૂપાંતરિત કરી શકાય છે \\
\end{longtable}
}

\end{solutionbox}
\begin{mnemonicbox}
``π ના બે પગ નીચે, એક શાખા આડી''

\end{mnemonicbox}
\subsection*{પ્રશ્ન 2(a) OR [3
ગુણ]}\label{q2a}

\textbf{નેટવર્કનાં પ્રકારો જણાવો.}

\begin{solutionbox}


{\def\LTcaptype{none} % do not increment counter
\vspace{-5pt}
\captionof{table}{નેટવર્કના પ્રકારો}
\vspace{-10pt}
\begin{longtable}[]{@{}ll@{}}
\toprule\noalign{}
શ્રેણી & પ્રકારો \\
\midrule\noalign{}
\endhead
\bottomrule\noalign{}
\endlastfoot
\textbf{લિનિયારિટી આધારિત} & લિનિયર નેટવર્ક, નોન-લિનિયર નેટવર્ક \\
\textbf{ઘટકો આધારિત} & પેસિવ નેટવર્ક, એક્ટિવ નેટવર્ક \\
\textbf{પેરામીટર આધારિત} & ટાઇમ-વેરિયન્ટ, ટાઇમ-ઇન્વેરિયન્ટ નેટવર્ક \\
\textbf{કોન્ફિગરેશન આધારિત} & T-નેટવર્ક, π-નેટવર્ક, લેટિસ નેટવર્ક \\
\textbf{પોર્ટ આધારિત} & વન-પોર્ટ, ટુ-પોર્ટ, મલ્ટિ-પોર્ટ નેટવર્ક \\
\textbf{સિમેટ્રી આધારિત} & સિમેટ્રિકલ, એસિમેટ્રિકલ નેટવર્ક \\
\textbf{રેસિપ્રોસિટી આધારિત} & રેસિપ્રોકલ, નોન-રેસિપ્રોકલ નેટવર્ક \\
\end{longtable}
}

\end{solutionbox}
\begin{mnemonicbox}
``LEPCPS: લિનિયારિટી, એલિમેન્ટ્સ, પેરામીટર્સ, કોન્ફિગરેશન,
પોર્ટ્સ, સિમેટ્રી''

\end{mnemonicbox}
\subsection*{પ્રશ્ન 2(b) OR [4
ગુણ]}\label{q2b}

\textbf{20, 50 અને 100 ohms ના રેઝીસ્ટર ડેલ્ટામાં કનેક્ટ કરેલા છે. સ્ટાર કનેક્શનનાં
ઇક્વીવેલન્ટ રેઝીસ્ટર શોધો.}

\begin{solutionbox}

\textbf{આકૃતિ: ડેલ્ટા થી સ્ટાર રૂપાંતરણ}

\begin{verbatim}
     Delta Connection          Star Connection
        1                         1
        o                         o
       / {                       /}
      /   {                     /}
R12=20Ω   R31=100Ω          R1=?
    /       {                 /}
   /         {               /}
  o{-{-}{-}{-}{-}{-}{-}{-}{-}{-}{-}o             o{-}{-}{-}{-}{-}{-}{-}{-}{-}{-}{-}o}
  2    R23=50Ω 3             2    R2=?  0
                                     {}
                                      {}
                                       {}
                                        {}
                                         {}
                                          o
                                          3
                                         /
                                        /
                                       /
                                    R3=?
\end{verbatim}


{\def\LTcaptype{none} % do not increment counter
\vspace{-5pt}
\captionof{table}{ડેલ્ટા-સ્ટાર રૂપાંતરણ ફોર્મ્યુલા અને ગણતરીઓ}
\vspace{-10pt}
\begin{longtable}[]{@{}llll@{}}
\toprule\noalign{}
સ્ટાર રેઝીસ્ટન્સ & ફોર્મ્યુલા & ગણતરી & પરિણામ \\
\midrule\noalign{}
\endhead
\bottomrule\noalign{}
\endlastfoot
R_{1} & (R_{1}_{2}\timesR_{3}_{1})/(R_{1}_{2}+R_{2}_{3}+R_{3}_{1}) & (20\times100)/(20+50+100) & 11.76Ω \\
R_{2} & (R_{1}_{2}\timesR_{2}_{3})/(R_{1}_{2}+R_{2}_{3}+R_{3}_{1}) & (20\times50)/(20+50+100) & 5.88Ω \\
R_{3} & (R_{2}_{3}\timesR_{3}_{1})/(R_{1}_{2}+R_{2}_{3}+R_{3}_{1}) & (50\times100)/(20+50+100) & 29.41Ω \\
\end{longtable}
}

\end{solutionbox}
\begin{mnemonicbox}
``ડેલ્ટા-સ્ટાર: આજુબાજુના જોડાનો ગુણાકાર બધાના સરવાળા
ઉપર''

\end{mnemonicbox}
\subsection*{પ્રશ્ન 2(c) OR [7
ગુણ]}\label{q2c}

\textbf{T નેટવર્ક સમજાવો.}

\begin{solutionbox}

\textbf{આકૃતિ: T નેટવર્ક}

\begin{verbatim}
         Z1        Z2
     o{-{-}{-}www{-}{-}{-}o{-}{-}{-}www{-}{-}{-}o}
               |
               |
               Z3
               |
               |
               o
             Ground
\end{verbatim}


{\def\LTcaptype{none} % do not increment counter
\vspace{-5pt}
\captionof{table}{T નેટવર્ક વિશેષતાઓ}
\vspace{-10pt}
\begin{longtable}[]{@{}
  >{\raggedright\arraybackslash}p{(\linewidth - 2\tabcolsep) * \real{0.4583}}
  >{\raggedright\arraybackslash}p{(\linewidth - 2\tabcolsep) * \real{0.5417}}@{}}
\toprule\noalign{}
\begin{minipage}[b]{\linewidth}\raggedright
પેરામીટર
\end{minipage} & \begin{minipage}[b]{\linewidth}\raggedright
વર્ણન
\end{minipage} \\
\midrule\noalign{}
\endhead
\bottomrule\noalign{}
\endlastfoot
\textbf{સ્ટ્રક્ચર} & બે સિરીઝ ઇમ્પિડન્સ (Z_{1}, Z_{2}) અને એક શન્ટ ઇમ્પિડન્સ (Z_{3}) \\
\textbf{ટ્રાન્સમિશન પેરામીટર્સ} & A = 1 + Z_{1}/Z_{3}, B = Z_{1} + Z_{2} + Z_{1}Z_{2}/Z_{3}, C =
1/Z_{3}, D = 1 + Z_{2}/Z_{3} \\
\textbf{ઇમ્પિડન્સ પેરામીટર્સ} & Z_{1}_{1} = Z_{1} + Z_{3}, Z_{1}_{2} = Z_{3}, Z_{2}_{1} = Z_{3}, Z_{2}_{2} = Z_{2}
+ Z_{3} \\
\textbf{ઇમેજ ઇમ્પિડન્સ} & Z_{0}T = \sqrt(Z_{1}Z_{2} + Z_{1}Z_{3} + Z_{2}Z_{3}) \\
\textbf{એપ્લિકેશન} & મેચિંગ નેટવર્ક, ફિલ્ટર, એટેન્યુએટર \\
\textbf{રૂપાંતરણ} & π-નેટવર્કમાં રૂપાંતરિત કરી શકાય છે \\
\end{longtable}
}

\end{solutionbox}
\begin{mnemonicbox}
``T ની બે બાહુ આડી, એક પગ નીચે''

\end{mnemonicbox}
\subsection*{પ્રશ્ન 3(a) [3
ગુણ]}\label{q3a}

\textbf{Kirchhoff's law સમજાવો.}

\begin{solutionbox}

\textbf{Kirchhoff's Current Law (KCL):}

\begin{itemize}
\tightlist
\item
  નોડમાં \textbf{પ્રવેશતા કરંટનો સરવાળો} તે નોડમાંથી નીકળતા કરંટના સરવાળા બરાબર
  હોય છે
\item
  કોઈપણ નોડ પર કરંટનો બીજગણિતીય સરવાળો શૂન્ય હોય છે
\item
  \sumI = 0 (પ્રવેશતા કરંટ પોઝિટિવ, નીકળતા નેગેટિવ)
\end{itemize}

\textbf{Kirchhoff's Voltage Law (KVL):}

\begin{itemize}
\tightlist
\item
  કોઈપણ બંધ લૂપમાં \textbf{વોલ્ટેજ ડ્રોપનો સરવાળો} શૂન્ય થાય છે
\item
  \sumV = 0 (વોલ્ટેજ વૃદ્ધિ પોઝિટિવ, ડ્રોપ નેગેટિવ)
\item
  ઊર્જાના સંરક્ષણ પર આધારિત છે
\end{itemize}

\textbf{આકૃતિ: Kirchhoff's Laws}

\begin{verbatim}
      KCL:                  KVL:
      I1                    V1
                           ↑
      o                     o
     ↑↓                    ↗ ↘
    I4 I2                 V4   V2
     ↑↓                    ↖ ↙
      o                     o
                           ↓
      I3                    V3
\end{verbatim}

\end{solutionbox}
\begin{mnemonicbox}
``કરંટ કન્વર્જ, વોલ્ટેજ વોયેજ ઈન અ લૂપ''

\end{mnemonicbox}
\subsection*{પ્રશ્ન 3(b) [4
ગુણ]}\label{q3b}

\textbf{Nodal analysis સમજાવો.}

\begin{solutionbox}

\textbf{આકૃતિ: નોડલ એનાલિસિસ કોન્સેપ્ટ}

\begin{verbatim}
   Step 1: Identify nodes
           ↓
   Step 2: Select reference node
           ↓
   Step 3: Assign node voltages
           ↓
   Step 4: Apply KCL at each node
           ↓
   Step 5: Solve equations
\end{verbatim}


{\def\LTcaptype{none} % do not increment counter
\vspace{-5pt}
\captionof{table}{નોડલ એનાલિસિસ મેથડ}
\vspace{-10pt}
\begin{longtable}[]{@{}ll@{}}
\toprule\noalign{}
સ્ટેપ & વર્ણન \\
\midrule\noalign{}
\endhead
\bottomrule\noalign{}
\endlastfoot
1. રેફરન્સ નોડ પસંદ કરો & સામાન્ય રીતે ગ્રાઉન્ડ (0V) \\
2. વોલ્ટેજ અસાઇન કરો & બાકીના નોડ વોલ્ટેજને લેબલ કરો (V_{1}, V_{2}, વગેરે) \\
3. KCL લાગુ કરો & દરેક નોન-રેફરન્સ નોડ પર KCL સમીકરણ લખો \\
4. કરંટને એક્સપ્રેસ કરો & ઓહ્મના નિયમનો ઉપયોગ કરીને બ્રાન્ચ કરંટ એક્સપ્રેસ કરો \\
5. સમીકરણો ઉકેલો & સિમલ્ટેનિયસ ઇક્વેશન વડે નોડ વોલ્ટેજ શોધો \\
\end{longtable}
}

\textbf{ઉદાહરણ: V_{1} અને V_{2} વોલ્ટેજવાળા નોડ્સ માટે:}

\begin{itemize}
\tightlist
\item
  નોડ 1 પર KCL: (V_{1}-0)/R_{1} + (V_{1}-V_{2})/R_{2} + I_{1} = 0
\item
  નોડ 2 પર KCL: (V_{2}-V_{1})/R_{2} + (V_{2}-0)/R_{3} + I_{2} = 0
\end{itemize}

\end{solutionbox}
\begin{mnemonicbox}
``નોડલ વોલ્ટેજ એનાલિસિસ માટે KCL જરૂરી છે''

\end{mnemonicbox}
\subsection*{પ્રશ્ન 3(c) [7
ગુણ]}\label{q3c}

\textbf{Thevenin's theorem નો ઉપયોગ કરીને ઉપર દશાર્વેલ સર્કિટ માટે 5 Ω રેઝીસ્ટર
માંથી પસાર થતો કરંટ શોધો.}

\begin{solutionbox}

\textbf{આકૃતિ: મૂળ સર્કિટ અને થેવેનિન ઇક્વિવેલન્ટ}

\begin{verbatim}
+{-{-}+     +{-}{-}+}
|  |     |  |
12V 20Ω  8V 10Ω
|  |     |  |
+{-{-}+{-}{-}+{-}{-}+{-}{-}+}
    |      |
    +{-{-}+{-}{-}{-}+}
       |
       5Ω
       |
      {-{-}{-}}
       {-}
\end{verbatim}

\textbf{થેવેનિન ઇક્વિવેલન્ટ શોધવા માટેના સ્ટેપ્સ:}


{\def\LTcaptype{none} % do not increment counter
\vspace{-5pt}
\captionof{table}{થેવેનિનના સિદ્ધાંતની પ્રક્રિયા અને ગણતરીઓ}
\vspace{-10pt}
\begin{longtable}[]{@{}
  >{\raggedright\arraybackslash}p{(\linewidth - 6\tabcolsep) * \real{0.1667}}
  >{\raggedright\arraybackslash}p{(\linewidth - 6\tabcolsep) * \real{0.2500}}
  >{\raggedright\arraybackslash}p{(\linewidth - 6\tabcolsep) * \real{0.3611}}
  >{\raggedright\arraybackslash}p{(\linewidth - 6\tabcolsep) * \real{0.2222}}@{}}
\toprule\noalign{}
\begin{minipage}[b]{\linewidth}\raggedright
સ્ટેપ
\end{minipage} & \begin{minipage}[b]{\linewidth}\raggedright
પ્રક્રિયા
\end{minipage} & \begin{minipage}[b]{\linewidth}\raggedright
ગણતરી
\end{minipage} & \begin{minipage}[b]{\linewidth}\raggedright
પરિણામ
\end{minipage} \\
\midrule\noalign{}
\endhead
\bottomrule\noalign{}
\endlastfoot
1. લોડ (5Ω) દૂર કરો & ઓપન-સર્કિટ વોલ્ટેજ (Voc) ગણો & Voc = વોલ્ટેજ ડિવાઇડર
ફોર્મ્યુલા & Vth = 9.33V \\
2. વોલ્ટેજ સોર્સને શોર્ટ કરો & ઇક્વિવેલન્ટ રેઝિસ્ટન્સ (Req) ગણો & Req = 20Ω & \\
3. થેવેનિન ઇક્વિવેલન્ટ દોરો & Vth અને Rth ને લોડ સાથે સિરીઝમાં જોડો & & \\
4. લોડ કરંટ ગણો &

I = Vth/(Rth+RL) &

I = 9.33/(6.67+5) &

I = 0.8A \\

\end{longtable}
}

\end{solutionbox}
\begin{mnemonicbox}
``થેવેનિન ટ્રાન્સફોર્મ: Voc અને Req શોધી, પછી I ગણો''

\end{mnemonicbox}
\subsection*{પ્રશ્ન 3(a) OR [3
ગુણ]}\label{q3a}

\textbf{Maximum Power Transfer Theorem જણાવો અને સમજાવો.}

\begin{solutionbox}

\textbf{Maximum Power Transfer Theorem:}

\begin{itemize}
\tightlist
\item
  મહત્તમ પાવર સોર્સથી લોડમાં ત્યારે ટ્રાન્સફર થાય છે જ્યારે \textbf{લોડ રેઝીસ્ટન્સ
  સોર્સના આંતરિક રેઝીસ્ટન્સ સમાન હોય} (RL = Rth)
\item
  મહત્તમ પાવર ટ્રાન્સફર પર માત્ર 50\% કાર્યક્ષમતા પ્રાપ્ત થાય છે
\item
  DC અને AC સર્કિટ બંને માટે લાગુ પડે છે (કોમ્પ્લેક્સ ઇમ્પિડન્સ સાથે)
\end{itemize}

\textbf{આકૃતિ: મહત્તમ પાવર ટ્રાન્સફર}

\begin{verbatim}
   Source        Load
  +{-{-}{-}{-}{-}+      +{-}{-}{-}{-}{-}+}
  |     |      |     |
  | Vth |{-{-}{-}{-}{-}{-}|     |}
  |     |  Rth |     | RL
  |     |{-{-}{-}{-}{-}{-}|     |}
  |     |      |     |
  +{-{-}{-}{-}{-}+      +{-}{-}{-}{-}{-}+}

  Power Transfer Curve:
       \^{}
       |       *
  Power|      / {}
       |     /   {}
       |    /     {}
       |   /       {}
       |  /         {}
       | /           {}
       |/             {}
       +{-{-}{-}{-}{-}{-}{-}{-}{-}{-}{-}{-}{-}{-}{-}{-}}
           RL = Rth       RL
\end{verbatim}

\textbf{ફોર્મ્યુલા: P = (Vth^{2}\timesRL)/(Rth+RL)^{2}}

\end{solutionbox}
\begin{mnemonicbox}
``મહત્તમ પાવર ટ્રાન્સફર માટે લોડને સોર્સ સાથે મેચ કરો''

\end{mnemonicbox}
\subsection*{પ્રશ્ન 3(b) OR [4
ગુણ]}\label{q3b}

\textbf{કોઈપણ સર્કિટનો ઉપયોગ કરીને ડ્યુઅલ નેટવર્ક દોરવાની પદ્ધતિ સમજાવો.}

\begin{solutionbox}

\textbf{આકૃતિ: મૂળ અને ડ્યુઅલ નેટવર્ક ઉદાહરણ}

\begin{verbatim}
Original:       Dual:
R1              C1
o{-{-}{-}www{-}{-}{-}o     o{-}{-}{-}||{-}{-}{-}o}
|         |     |        |
C1        R2    L1       L2
|         |     |        |
o{-{-}{-}||{-}{-}{-}{-}o     o{-}{-}{-}www{-}{-}o}
    L1               R1
\end{verbatim}


{\def\LTcaptype{none} % do not increment counter
\vspace{-5pt}
\captionof{table}{ડ્યુઅલ નેટવર્ક રૂપાંતરણ નિયમો}
\vspace{-10pt}
\begin{longtable}[]{@{}lll@{}}
\toprule\noalign{}
મૂળ ઘટક & ડ્યુઅલ ઘટક & ઉદાહરણ \\
\midrule\noalign{}
\endhead
\bottomrule\noalign{}
\endlastfoot
સિરીઝ કનેક્શન & પેરેલલ કનેક્શન & સિરીઝ R \rightarrow પેરેલલ C \\
પેરેલલ કનેક્શન & સિરીઝ કનેક્શન & પેરેલલ C \rightarrow સિરીઝ L \\
વોલ્ટેજ સોર્સ & કરંટ સોર્સ & V સોર્સ \rightarrow I સોર્સ \\
કરંટ સોર્સ & વોલ્ટેજ સોર્સ & I સોર્સ \rightarrow V સોર્સ \\
રેઝીસ્ટર (R) & કંડક્ટન્સ (1/R) & R \rightarrow G (1/R) \\
ઇન્ડક્ટર (L) & કેપેસિટર (1/L) & L \rightarrow C (1/L) \\
કેપેસિટર (C) & ઇન્ડક્ટર (1/C) & C \rightarrow L (1/C) \\
\end{longtable}
}

\textbf{ડ્યુઅલિટી પ્રક્રિયા:}

\begin{enumerate}
\tightlist
\item
  મેશ્સને નોડ્સ તરીકે અને નોડ્સને મેશ્સ તરીકે રિડ્રો કરો
\item
  ઘટકોને તેમના ડ્યુઅલ સાથે બદલો
\item
  સિરીઝ અને પેરેલલ કનેક્શન્સને અદલાબદલી કરો
\end{enumerate}

\end{solutionbox}
\begin{mnemonicbox}
``ડ્યુઅલિટી સ્વેપ્સ: સિરીઝ\leftrightarrowપેરેલલ, V\leftrightarrowI, R\leftrightarrowG, L\leftrightarrowC''

\end{mnemonicbox}
\subsection*{પ્રશ્ન 3(c) OR [7
ગુણ]}\label{q3c}

\textbf{ઉપર આપેલ નેટવર્ક માટે નોર્ટનની ઇક્વીવેલન્ટ સર્કિટ શોધો. લોડ કરંટ શોધો જો
(i) RL=3 KΩ (ii) RL=1.5 Ω}

\begin{solutionbox}

\textbf{આકૃતિ: મૂળ સર્કિટ અને નોર્ટન ઇક્વિવેલન્ટ}

\begin{verbatim}
    +{-{-}+}
    |  |
    6V 9KΩ
    |  |
+{-{-}{-}+{-}{-}+{-}{-}{-}{-}{-}+}
|            |
3KΩ         6KΩ
|            |
+{-{-}{-}{-}{-}+{-}{-}{-}{-}{-}{-}+}
      |
      RL
      |
     {-{-}{-}}
      {-}
\end{verbatim}


{\def\LTcaptype{none} % do not increment counter
\vspace{-5pt}
\captionof{table}{નોર્ટનના સિદ્ધાંતની પ્રક્રિયા અને ગણતરીઓ}
\vspace{-10pt}
\begin{longtable}[]{@{}
  >{\raggedright\arraybackslash}p{(\linewidth - 6\tabcolsep) * \real{0.1667}}
  >{\raggedright\arraybackslash}p{(\linewidth - 6\tabcolsep) * \real{0.2500}}
  >{\raggedright\arraybackslash}p{(\linewidth - 6\tabcolsep) * \real{0.3611}}
  >{\raggedright\arraybackslash}p{(\linewidth - 6\tabcolsep) * \real{0.2222}}@{}}
\toprule\noalign{}
\begin{minipage}[b]{\linewidth}\raggedright
સ્ટેપ
\end{minipage} & \begin{minipage}[b]{\linewidth}\raggedright
પ્રક્રિયા
\end{minipage} & \begin{minipage}[b]{\linewidth}\raggedright
ગણતરી
\end{minipage} & \begin{minipage}[b]{\linewidth}\raggedright
પરિણામ
\end{minipage} \\
\midrule\noalign{}
\endhead
\bottomrule\noalign{}
\endlastfoot
1. શોર્ટ-સર્કિટ કરંટ (Isc) ગણો & લોડ ટર્મિનલ્સને શોર્ટ કરો અને કરંટ શોધો & Isc =
શોર્ટ મારફતે સોર્સ કરંટ & In = 0.5mA \\
2. નોર્ટન રેઝીસ્ટન્સ (Rn) ગણો & સોર્સને આંતરિક રેઝીસ્ટન્સ સાથે બદલો & Rn = 9KΩ & \\
3. નોર્ટન ઇક્વિવેલન્ટ દોરો & In અને Rn ને પેરેલલમાં જોડો & & \\
4. લોડ કરંટ (RL = 3KΩ) ગણો &

I = In \times Rn/(Rn + RL) &

I = 0.5mA \times 3KΩ/(3KΩ

+ 3KΩ) & I = 0.25mA \\
5. લોડ કરંટ (RL = 1.5Ω) ગણો &

I = In \times Rn/(Rn + RL) &

I = 0.5mA \times

3KΩ/(3KΩ + 1.5Ω) & I = 0.33mA \\
\end{longtable}
}

\end{solutionbox}
\begin{mnemonicbox}
``નોર્ટનને કરંટ સોર્સ બનાવવા Isc અને Req જોઈએ''

\end{mnemonicbox}
\subsection*{પ્રશ્ન 4(a) [3
ગુણ]}\label{q4a}

\textbf{કોઇલ માટે ક્વોલિટી ફેક્ટર Q નું સમીકરણ મેળવો.}

\begin{solutionbox}

\textbf{આકૃતિ: કોઇલ ઇક્વિવેલન્ટ સર્કિટ}

\begin{verbatim}
     R       L
o{-{-}{-}www{-}{-}{-}OOOOOO{-}{-}{-}o}
\end{verbatim}

\textbf{કોઇલ માટે Q ફેક્ટરની ડેરિવેશન:}


{\def\LTcaptype{none} % do not increment counter
\vspace{-5pt}
\captionof{table}{કોઇલ માટે Q ફેક્ટર ડેરિવેશન}
\vspace{-10pt}
\begin{longtable}[]{@{}lll@{}}
\toprule\noalign{}
સ્ટેપ & અભિવ્યક્તિ & સમજૂતી \\
\midrule\noalign{}
\endhead
\bottomrule\noalign{}
\endlastfoot
1. ઇમ્પિડન્સ & Z = R + jωL & કોઇલનું કોમ્પ્લેક્સ ઇમ્પિડન્સ \\
2. રિએક્ટિવ પાવર & PX = (ωL)I^{2} & ઇન્ડક્ટરમાં સંગ્રહિત પાવર \\
3. રીઅલ પાવર & PR = RI^{2} & રેઝીસ્ટન્સમાં વેડફાતો પાવર \\
4. ક્વોલિટી ફેક્ટર & Q = PX/PR & સંગ્રહિત અને વેડફાતા પાવરનો રેશિયો \\
5. સબ્સ્ટિટ્યુશન & Q = (ωL)I^{2}/RI^{2} & અભિવ્યક્તિઓ સબ્સ્ટિટ્યુટ કરો \\
6. ફાઇનલ ઇક્વેશન & Q = ωL/R & Q ફેક્ટર મેળવવા સરળ કરો \\
\end{longtable}
}

\end{solutionbox}
\begin{mnemonicbox}
``ક્વોલિટી કોઇલ્સ: ωL/R ઊર્જા બચાવવાની ક્ષમતા દર્શાવે છે''

\end{mnemonicbox}
\subsection*{પ્રશ્ન 4(b) [4
ગુણ]}\label{q4b}

\textbf{શ્રેણી RLC સર્કિટમાં R=50 Ω, L=0.2 H અને C=10 μF છે. (i)Q પરિબળ, (ii)
BW, (iii) અપર કટ ઓફ અને લોઅર કટ ઓફ ફ્રીક્વન્સીઝની ગણતરી કરો.}

\begin{solutionbox}

\textbf{આકૃતિ: સિરીઝ RLC સર્કિટ}

\begin{verbatim}
R=50Ω

L=0.2H

o{-{-}{-}{-}www{-}{-}{-}{-}{-}OOOOOO{-}{-}{-}{-}{-}+}
                        |
                        |
                       {-{-}{-}}
                       {-{-}{-} C=10μF}
                        |
                        |
o{-{-}{-}{-}{-}{-}{-}{-}{-}{-}{-}{-}{-}{-}{-}{-}{-}{-}{-}{-}{-}{-}{-}+}
\end{verbatim}


{\def\LTcaptype{none} % do not increment counter
\vspace{-5pt}
\captionof{table}{સિરીઝ RLC સર્કિટ માટે ગણતરીઓ}
\vspace{-10pt}
\begin{longtable}[]{@{}
  >{\raggedright\arraybackslash}p{(\linewidth - 6\tabcolsep) * \real{0.2683}}
  >{\raggedright\arraybackslash}p{(\linewidth - 6\tabcolsep) * \real{0.2195}}
  >{\raggedright\arraybackslash}p{(\linewidth - 6\tabcolsep) * \real{0.3171}}
  >{\raggedright\arraybackslash}p{(\linewidth - 6\tabcolsep) * \real{0.1951}}@{}}
\toprule\noalign{}
\begin{minipage}[b]{\linewidth}\raggedright
પેરામીટર
\end{minipage} & \begin{minipage}[b]{\linewidth}\raggedright
ફોર્મ્યુલા
\end{minipage} & \begin{minipage}[b]{\linewidth}\raggedright
ગણતરી
\end{minipage} & \begin{minipage}[b]{\linewidth}\raggedright
પરિણામ
\end{minipage} \\
\midrule\noalign{}
\endhead
\bottomrule\noalign{}
\endlastfoot
રેઝોનન્ટ ફ્રીક્વન્સી (fr) & fr = 1/(2π\sqrtLC) & 1/(2π\sqrt(0.2\times10\times10^{-}^{6})) & 112.5
Hz \\
ક્વોલિટી ફેક્ટર (Q) & Q = (1/R)\sqrt(L/C) & (1/50)\sqrt(0.2/10\times10^{-}^{6}) & 28.28 \\
બેન્ડવિડ્થ (BW) & BW = fr/Q & 112.5/28.28 & 3.98 Hz \\
લોઅર કટઓફ (f_{1}) & f_{1} = fr - BW/2 & 112.5 - 3.98/2 & 110.51 Hz \\
અપર કટઓફ (f_{2}) & f_{2} = fr + BW/2 & 112.5 + 3.98/2 & 114.49 Hz \\
\end{longtable}
}

\end{solutionbox}
\begin{mnemonicbox}
``Q કટઓફ ફ્રીક્વન્સી માટે BW નિર્ધારિત કરે છે''

\end{mnemonicbox}
\subsection*{પ્રશ્ન 4(c) [7
ગુણ]}\label{q4c}

\textbf{મ્યુચ્યુઅલ ઇન્ડક્ટન્સના કો-એફીસીએન્ટ સાથે મ્યુચ્યુઅલ ઇન્ડક્ટન્સ સમજાવો. K નું
સમીકરણ પણ મેળવો.}

\begin{solutionbox}

\textbf{આકૃતિ: બે કોઇલ વચ્ચે મ્યુચ્યુઅલ ઇન્ડક્ટન્સ}

\begin{verbatim}
   Coil 1          Coil 2
    OOOO            OOOO
   O    O          O    O
Input O    O        O    O Output
   O    O {    O    O}
    OOOO            OOOO
\end{verbatim}

\textbf{મ્યુચ્યુઅલ ઇન્ડક્ટન્સ (M):}

\begin{itemize}
\tightlist
\item
  જ્યારે એક કોઇલમાં કરંટ નજીકની કોઇલમાં વોલ્ટેજ પ્રેરિત કરે છે
\item
  કોઇલ્સ વચ્ચેની કપલિંગ તેમની સ્થિતિ, ઓરિયેન્ટેશન અને માધ્યમ પર નિર્ભર કરે છે
\item
  મ્યુચ્યુઅલ ઇન્ડક્ટન્સ M હેનરી (H)માં
\end{itemize}


{\def\LTcaptype{none} % do not increment counter
\vspace{-5pt}
\captionof{table}{મ્યુચ્યુઅલ ઇન્ડક્ટન્સ સમીકરણો}
\vspace{-10pt}
\begin{longtable}[]{@{}
  >{\raggedright\arraybackslash}p{(\linewidth - 4\tabcolsep) * \real{0.3333}}
  >{\raggedright\arraybackslash}p{(\linewidth - 4\tabcolsep) * \real{0.2727}}
  >{\raggedright\arraybackslash}p{(\linewidth - 4\tabcolsep) * \real{0.3939}}@{}}
\toprule\noalign{}
\begin{minipage}[b]{\linewidth}\raggedright
પેરામીટર
\end{minipage} & \begin{minipage}[b]{\linewidth}\raggedright
ફોર્મ્યુલા
\end{minipage} & \begin{minipage}[b]{\linewidth}\raggedright
વર્ણન
\end{minipage} \\
\midrule\noalign{}
\endhead
\bottomrule\noalign{}
\endlastfoot
પ્રેરિત વોલ્ટેજ & v_{2} = M(di_{1}/dt) & કોઇલ 1માં કરંટને લીધે કોઇલ 2માં પ્રેરિત વોલ્ટેજ \\
મ્યુચ્યુઅલ ઇન્ડક્ટન્સ & M = k\sqrt(L_{1}L_{2}) & સેલ્ફ-ઇન્ડક્ટન્સ સાથે સંબંધિત મ્યુચ્યુઅલ ઇન્ડક્ટન્સ \\
કપલિંગ કોઇફિશિયન્ટ (k) & k = M/\sqrt(L_{1}L_{2}) & કોઇલ્સ વચ્ચેની કપલિંગનું માપ (0 \leq k \leq
1) \\
કુલ ઇન્ડક્ટન્સ & Lt = L_{1} + L_{2} \pm 2M & કુલ ઇન્ડક્ટન્સ કપલિંગની દિશા પર નિર્ભર \\
\end{longtable}
}

\textbf{કપલિંગ કોઇફિશિયન્ટ (k)ની ડેરિવેશન:}

\begin{itemize}
\tightlist
\item
  M = k\sqrt(L_{1}L_{2}) માંથી
\item
  ફરી ગોઠવતા: k = M/\sqrt(L_{1}L_{2})
\item
  k = 1 પરફેક્ટ કપલિંગ માટે
\item
  k = 0 નો કપલિંગ માટે
\item
  વાસ્તવિક સર્કિટ માટે સામાન્ય રીતે 0.1 થી 0.9
\end{itemize}

\end{solutionbox}
\begin{mnemonicbox}
``M મેગ્નેટિક લિંકેજ માપે, k કપલિંગની ક્વોલિટી દર્શાવે''

\end{mnemonicbox}
\subsection*{પ્રશ્ન 4(a) OR [3
ગુણ]}\label{q4a}

\textbf{કપલ સર્કિટ માટે કપ્લીંગના પ્રકારો સમજાવો.}

\begin{solutionbox}

\textbf{આકૃતિ: કપલિંગના પ્રકારો}

\begin{verbatim}
   Types of Coupling
          |
  +{-{-}{-}{-}{-}{-}{-}+{-}{-}{-}{-}{-}{-}{-}{-}{-}{-}+{-}{-}{-}{-}{-}{-}{-}+{-}{-}{-}{-}{-}{-}{-}{-}{-}+{-}{-}{-}{-}{-}{-}{-}{-}+}
  |       |          |       |         |        |
Tight     Loose   Critical Direct Inductive Capacitive
Coupling Coupling Coupling Coupling Coupling Coupling
\end{verbatim}


{\def\LTcaptype{none} % do not increment counter
\vspace{-5pt}
\captionof{table}{કપલિંગના પ્રકારો}
\vspace{-10pt}
\begin{longtable}[]{@{}
  >{\raggedright\arraybackslash}p{(\linewidth - 4\tabcolsep) * \real{0.3261}}
  >{\raggedright\arraybackslash}p{(\linewidth - 4\tabcolsep) * \real{0.3696}}
  >{\raggedright\arraybackslash}p{(\linewidth - 4\tabcolsep) * \real{0.3043}}@{}}
\toprule\noalign{}
\begin{minipage}[b]{\linewidth}\raggedright
કપલિંગનો પ્રકાર
\end{minipage} & \begin{minipage}[b]{\linewidth}\raggedright
લક્ષણો
\end{minipage} & \begin{minipage}[b]{\linewidth}\raggedright
એપ્લિકેશન
\end{minipage} \\
\midrule\noalign{}
\endhead
\bottomrule\noalign{}
\endlastfoot
\textbf{ટાઇટ કપલિંગ} & k \textgreater{} 0.5, ઉચ્ચ ઊર્જા ટ્રાન્સફર &
ટ્રાન્સફોર્મર \\
\textbf{લૂઝ કપલિંગ} & k \textless{} 0.5, સિલેક્ટિવ ફ્રીક્વન્સી રિસ્પોન્સ & RF
ટ્યુનિંગ સર્કિટ \\
\textbf{ક્રિટિકલ કપલિંગ} & k ઓપ્ટિમલ બેન્ડવિડ્થ માટે એડજસ્ટ કરેલું & RF ફિલ્ટર \\
\textbf{ડાયરેક્ટ કપલિંગ} & ઘટકો સીધા જોડાયેલા & ઓડિયો એમ્પ્લિફાયર \\
\textbf{ઇન્ડક્ટિવ કપલિંગ} & મેગ્નેટિક ફિલ્ડ ઊર્જા ટ્રાન્સફર કરે છે & ટ્રાન્સફોર્મર,
વાયરલેસ ચાર્જિંગ \\
\textbf{કેપેસિટિવ કપલિંગ} & ઇલેક્ટ્રિક ફિલ્ડ ઊર્જા ટ્રાન્સફર કરે છે & સ્ટેજ વચ્ચે સિગ્નલ
કપલિંગ \\
\end{longtable}
}

\end{solutionbox}
\begin{mnemonicbox}
``TLCLIC: ટાઇટ, લૂઝ, ક્રિટિકલ, ડાયરેક્ટ, ઇન્ડક્ટિવ,
કેપેસિટિવ''

\end{mnemonicbox}
\subsection*{પ્રશ્ન 4(b) OR [4
ગુણ]}\label{q4b}

\textbf{ગુણવત્તા પરિબળ Q = 100, રેઝોનન્ટ ફ્રિકવન્સી Fr = 50 KHz સાથે 10 mH નું
ઇન્ડક્ટન્સ ધરાવતું સમાંતર રેઝોનન્ટ સર્કિટ. શોધો (i) જરૂરી કેપેસીટન્સ C, (ii) કોઇલનો
પ્રતિકાર R, (iii) BW.}

\begin{solutionbox}

\textbf{આકૃતિ: પેરેલલ રેઝોનન્ટ સર્કિટ}

\begin{verbatim}
        L=10mH  
o{-{-}{-}{-}{-}{-}{-}OOOOOO{-}{-}{-}{-}{-}{-}{-}{-}+}
|                     |
|        R            |
|       www           |
|                     |
|                    {-{-}{-}}
|                    {-{-}{-} C=?}
|                     |
o{-{-}{-}{-}{-}{-}{-}{-}{-}{-}{-}{-}{-}{-}{-}{-}{-}{-}{-}{-}{-}+}
\end{verbatim}


{\def\LTcaptype{none} % do not increment counter
\vspace{-5pt}
\captionof{table}{પેરેલલ રેઝોનન્ટ સર્કિટ માટે ગણતરીઓ}
\vspace{-10pt}
\begin{longtable}[]{@{}
  >{\raggedright\arraybackslash}p{(\linewidth - 6\tabcolsep) * \real{0.2683}}
  >{\raggedright\arraybackslash}p{(\linewidth - 6\tabcolsep) * \real{0.2195}}
  >{\raggedright\arraybackslash}p{(\linewidth - 6\tabcolsep) * \real{0.3171}}
  >{\raggedright\arraybackslash}p{(\linewidth - 6\tabcolsep) * \real{0.1951}}@{}}
\toprule\noalign{}
\begin{minipage}[b]{\linewidth}\raggedright
પેરામીટર
\end{minipage} & \begin{minipage}[b]{\linewidth}\raggedright
ફોર્મ્યુલા
\end{minipage} & \begin{minipage}[b]{\linewidth}\raggedright
ગણતરી
\end{minipage} & \begin{minipage}[b]{\linewidth}\raggedright
પરિણામ
\end{minipage} \\
\midrule\noalign{}
\endhead
\bottomrule\noalign{}
\endlastfoot
રેઝોનન્ટ ફ્રીક્વન્સી & fr = 1/(2π\sqrtLC) & 50 kHz = 1/(2π\sqrt(10\times10^{-}^{3}\timesC)) & \\
કેપેસિટન્સ (C) &

C = 1/(4π^{2}fr^{2}L) &

C = 1/(4π^{2}\times(50\times10^{3})^{2}\times10\times10^{-}^{3}) &

C = 1.01

nF \\
રેઝિસ્ટન્સ (R) &

Q = ωL/R & 100 = 2π\times50\times10^{3}\times10\times10^{-}^{3}/R &

R = 31.4 Ω \\

બેન્ડવિડ્થ (BW) & BW = fr/Q & BW = 50\times10^{3}/100 & BW = 500 Hz \\
\end{longtable}
}

\end{solutionbox}
\begin{mnemonicbox}
``પેરેલલ રેઝોનન્સ પેરામીટર્સ: C fr માંથી, R Q માંથી, BW fr/Q
માંથી''

\end{mnemonicbox}
\subsection*{પ્રશ્ન 4(c) OR [7
ગુણ]}\label{q4c}

\textbf{સીરીઝ RLC સર્કિટની Band width અને Selectivity સમજાવો. શ્રેણી રેઝોનન્સ
સર્કિટ માટે Q પરિબળ અને BW વચ્ચેનો સંબંધ પણ સ્થાપિત કરો.}

\begin{solutionbox}

\textbf{આકૃતિ: સિરીઝ RLC સર્કિટનો ફ્રીક્વન્સી રિસ્પોન્સ}

\begin{verbatim}
   Impedance
       \^{}
       |       
       |       
       |       
       |      /|{}
       |     / | {}
       |    /  |  {}
       |   /   |   {}
       |  /    |    {}
       | /     |     {}
       |/      |      {}
       +{-{-}{-}{-}{-}{-}{-}+{-}{-}{-}{-}{-}{-}{-}+{-}{-}}
              fr        Frequency
             f1  f2
             {{-}{-}{-}{-}}
               BW
\end{verbatim}

\textbf{બેન્ડવિડ્થ (BW):}

\begin{itemize}
\tightlist
\item
  હાફ-પાવર પોઇન્ટ વચ્ચેની \textbf{ફ્રીક્વન્સી રેન્જ}
\item
  હાફ-પાવર પોઇન્ટ પર ઇમ્પિડન્સ લઘુતમ મૂલ્યના \sqrt2 ગણું હોય છે
\item
  BW = f_{2} - f_{1}, જ્યાં f_{1} અને f_{2} લોઅર અને અપર કટઓફ ફ્રીક્વન્સી છે
\end{itemize}

\textbf{સિલેક્ટિવિટી:}

\begin{itemize}
\tightlist
\item
  બેન્ડવિડ્થ બહારની ફ્રીક્વન્સીઓને \textbf{નકારવાની ક્ષમતા}
\item
  ઉચ્ચ Q એટલે વધુ સિલેક્ટિવિટી અને સાંકડી બેન્ડવિડ્થ
\item
  રિસ્પોન્સ કર્વની તીવ્રતા દ્વારા માપવામાં આવે છે
\end{itemize}


{\def\LTcaptype{none} % do not increment counter
\vspace{-5pt}
\captionof{table}{સિરીઝ RLC બેન્ડવિડ્થ પેરામીટર્સ}
\vspace{-10pt}
\begin{longtable}[]{@{}lll@{}}
\toprule\noalign{}
પેરામીટર & ફોર્મ્યુલા & વર્ણન \\
\midrule\noalign{}
\endhead
\bottomrule\noalign{}
\endlastfoot
બેન્ડવિડ્થ (BW) & BW = f_{2} - f_{1} & અપર અને લોઅર કટઓફ પોઇન્ટ વચ્ચેનો તફાવત \\
હાફ-પાવર પોઇન્ટ & Z = \sqrt2 \times Z_{m}ᵢ_{n} & જ્યાં પાવર મહત્તમના અર્ધા જેટલો થાય છે \\
રેઝોનન્ટ ફ્રીક્વન્સી & fr = 1/(2π\sqrtLC) & સેન્ટર ફ્રીક્વન્સી \\
ક્વોલિટી ફેક્ટર & Q = ω_{o}L/R & ઊર્જા સંગ્રહ vs.~વેડફાટ રેશિયો \\
\end{longtable}
}

\textbf{Q-BW સંબંધની ડેરિવેશન:}

\begin{itemize}
\tightlist
\item
  રેઝોનન્સ પર ઇમ્પિડન્સ Z = R
\item
  કટઓફ ફ્રીક્વન્સી પર Z = \sqrt2R
\item
  આ ત્યારે થાય છે જ્યારે રિએક્ટન્સ XL - XC = \pmR
\item
  f_{1} પર: ωL - 1/ωC = -R
\item
  f_{2} પર: ωL - 1/ωC = +R
\item
  આ સમીકરણો ઉકેલતા: BW = R/2πL = fr/Q
\item
  આથી, Q = fr/BW
\end{itemize}

\end{solutionbox}
\begin{mnemonicbox}
``ક્વોલિટી બેન્ડવિડ્થના વ્યસ્ત પ્રમાણમાં''

\end{mnemonicbox}
\subsection*{પ્રશ્ન 5(a) [3
ગુણ]}\label{q5a}

\textbf{60 ડીબીનું એટેન્યુએશન આપવા અને 500 Ω પ્રતિકારના લોડમાં કામ કરવા માટે
સપ્રમાણ T પ્રકારના એટેન્યુએટરને ડિઝાઇન કરો.}

\begin{solutionbox}

\textbf{આકૃતિ: સપ્રમાણ T-ટાઇપ એટેન્યુએટર}

\begin{verbatim}
        R1/2          R1/2
   o{-{-}{-}{-}www{-}{-}{-}{-}{-}o{-}{-}{-}{-}www{-}{-}{-}{-}o}
   |            |           |
   |            |           |
   |           R2           |
   |            |           |
   |            |           |
  IN           {-{-}{-}         OUT}
                {-}
\end{verbatim}


{\def\LTcaptype{none} % do not increment counter
\vspace{-5pt}
\captionof{table}{એટેન્યુએટર ડિઝાઇન}
\vspace{-10pt}
\begin{longtable}[]{@{}llll@{}}
\toprule\noalign{}
પેરામીટર & ફોર્મ્યુલા & ગણતરી & પરિણામ \\
\midrule\noalign{}
\endhead
\bottomrule\noalign{}
\endlastfoot
એટેન્યુએશન (N) &

N = 10\^{}(dB/20) & 10\^{}(60/20) &

N = 1000 \\

Z_{0} & આપેલ & 500 Ω & 500 Ω \\
R_{1} & R_{1} = 2Z_{0}(N-1)/(N+1) & 2\times500\times(1000-1)/(1000+1) & R_{1} = 998 Ω \\
R_{2} & R_{2} = Z_{0}(N+1)/(N-1) & 500\times(1000+1)/(1000-1) & R_{2} = 0.5 Ω \\
\end{longtable}
}

\end{solutionbox}
\begin{mnemonicbox}
``T એટેન્યુએટર: R_{1} સિરીઝ ડિવાઇડ કરે, R_{2} શન્ટ કરે''

\end{mnemonicbox}
\subsection*{પ્રશ્ન 5(b) [4
ગુણ]}\label{q5b}

\textbf{બેન્ડ પાસ અને બેન્ડ સ્ટોપ ફિલ્ટર્સને સરખાવો.}

\begin{solutionbox}

\textbf{આકૃતિ: બેન્ડ પાસ vs બેન્ડ સ્ટોપ રિસ્પોન્સ}

\begin{verbatim}
           f1    f2
            v    v
  Gain     Band Pass         Band Stop
   \^{          \_\_\_              \_   \_}
   |         /   {            | \_/ |}
   |        /     {           |     |}
   |       /       {          |     |}
   |      /         {         |     |}
   |     /           {        |     |}
   |    /             {       |     |}
   |\_\_\_/               {\_\_\_\_\_\_|     |\_\_\_\_\_\_}
   +{-{-}{-}{-}{-}{-}{-}{-}{-}{-}{-}{-}{-}{-}{-}{-}{-}{-}{-}{-}{-}{-}{-}{-}{-}{-}{-}{-}{-}{-}{-}{-}{-}{-}}
              Frequency
\end{verbatim}


{\def\LTcaptype{none} % do not increment counter
\vspace{-5pt}
\captionof{table}{બેન્ડ પાસ અને બેન્ડ સ્ટોપ ફિલ્ટર્સની તુલના}
\vspace{-10pt}
\begin{longtable}[]{@{}
  >{\raggedright\arraybackslash}p{(\linewidth - 4\tabcolsep) * \real{0.2245}}
  >{\raggedright\arraybackslash}p{(\linewidth - 4\tabcolsep) * \real{0.3878}}
  >{\raggedright\arraybackslash}p{(\linewidth - 4\tabcolsep) * \real{0.3878}}@{}}
\toprule\noalign{}
\begin{minipage}[b]{\linewidth}\raggedright
પેરામીટર
\end{minipage} & \begin{minipage}[b]{\linewidth}\raggedright
બેન્ડ પાસ ફિલ્ટર
\end{minipage} & \begin{minipage}[b]{\linewidth}\raggedright
બેન્ડ સ્ટોપ ફિલ્ટર
\end{minipage} \\
\midrule\noalign{}
\endhead
\bottomrule\noalign{}
\endlastfoot
\textbf{ફ્રીક્વન્સી રિસ્પોન્સ} & ચોક્કસ બેન્ડમાંની ફ્રીક્વન્સીઓ પસાર કરે છે & ચોક્કસ
બેન્ડમાંની ફ્રીક્વન્સીઓ નકારે છે \\
\textbf{સર્કિટ સ્ટ્રક્ચર} & સિરીઝ \& પેરેલલ રેઝોનન્ટ સર્કિટ & સિરીઝ \& પેરેલલ
રેઝોનન્ટ સર્કિટ \\
\textbf{કટ-ઓફ ફ્રીક્વન્સી} & લોઅર (f_{1}) અને અપર (f_{2}) કટ-ઓફ છે & લોઅર (f_{1}) અને
અપર (f_{2}) કટ-ઓફ છે \\
\textbf{બેન્ડવિડ્થ} & BW = f_{2} - f_{1} & BW = f_{2} - f_{1} \\
\textbf{એપ્લિકેશન} & રેડિયો ટ્યુનિંગ, ઓડિયો ઇક્વલાઇઝેશન & નોઇઝ એલિમિનેશન,
હાર્મોનિક સપ્રેશન \\
\textbf{ઇમ્પ્લિમેન્ટેશન} & HPF \& LPF ની સિરીઝ/પેરેલલ કોમ્બિનેશન & HPF \& LPF ની
પેરેલલ/સિરીઝ કોમ્બિનેશન \\
\textbf{ફેઝ રિસ્પોન્સ} & રેઝોનન્સ પર 0^\circ & રેઝોનન્સ પર 180^\circ \\
\end{longtable}
}

\end{solutionbox}
\begin{mnemonicbox}
``મધ્યમાં પાસ કરો અથવા મધ્યમાં સ્ટોપ કરો''

\end{mnemonicbox}
\subsection*{પ્રશ્ન 5(c) [7
ગુણ]}\label{q5c}

\textbf{Constant K લો પાસ ફિલ્ટર સમજાવો.}

\begin{solutionbox}

\textbf{આકૃતિ: Constant K લો પાસ ફિલ્ટર T અને π સેક્શન}

\begin{verbatim}
T{-section:                    π{-}section:}
    L/2         L/2               L
o{-{-}{-}OOOO{-}{-}{-}{-}o{-}{-}{-}OOOO{-}{-}{-}{-}{-}{-}o  o{-}{-}{-}OOOO{-}{-}{-}o}
            |                  |      |
            C                  C/2    C/2
            |                  |      |
            |                  |      |
o{-{-}{-}{-}{-}{-}{-}{-}{-}{-}{-}o{-}{-}{-}{-}{-}{-}{-}{-}{-}{-}{-}{-}{-}o  o{-}{-}{-}{-}{-}{-}{-}{-}{-}{-}o}
\end{verbatim}

\textbf{Constant K લો પાસ ફિલ્ટર:}

\begin{itemize}
\tightlist
\item
  કટઓફ ફ્રીક્વન્સી (fc) \textbf{નીચેની ફ્રીક્વન્સીઓ} પસાર કરે છે
\item
  fc ઉપરની ફ્રીક્વન્સીઓ ઘટાડે છે
\item
  ``Constant K'' નો અર્થ છે કે સિરીઝ અને શન્ટ ઈમ્પિડન્સના ગુણાકારો બધી ફ્રીક્વન્સી
  પર સ્થિર રહે છે (Z_{1}Z_{2} = K^{2})
\end{itemize}


{\def\LTcaptype{none} % do not increment counter
\vspace{-5pt}
\captionof{table}{T અને π સેક્શન પેરામીટર્સ}
\vspace{-10pt}
\begin{longtable}[]{@{}lll@{}}
\toprule\noalign{}
પેરામીટર & T-સેક્શન & π-સેક્શન \\
\midrule\noalign{}
\endhead
\bottomrule\noalign{}
\endlastfoot
સિરીઝ આર્મ & દરેક છેડે L/2 & મધ્યમાં L \\
શન્ટ આર્મ & મધ્યમાં C & દરેક છેડે C/2 \\
કટઓફ ફ્રીક્વન્સી & fc = 1/(π\sqrtLC) & fc = 1/(π\sqrtLC) \\
કેરેક્ટરિસ્ટિક ઇમ્પિડન્સ & Z_{0} = \sqrt(L/C) & Z_{0} = \sqrt(L/C) \\
L માટે ડિઝાઇન ઇક્વેશન &

L = Z_{0}/πfc &

L = Z_{0}/πfc \\

C માટે ડિઝાઇન ઇક્વેશન &

C = 1/(πfcZ_{0}) &

C = 1/(πfcZ_{0}) \\

\end{longtable}
}

\textbf{ફ્રીક્વન્સી રિસ્પોન્સ:}

\begin{itemize}
\tightlist
\item
  DC અને લો ફ્રીક્વન્સીઓ ન્યૂનતમ એટેન્યુએશન સાથે પસાર કરે છે
\item
  કટઓફ ફ્રીક્વન્સી ઉપર એટેન્યુએશન ઝડપથી વધે છે
\item
  ફેઝ શિફ્ટ ફ્રીક્વન્સી સાથે વધે છે
\end{itemize}

\end{solutionbox}
\begin{mnemonicbox}
``Constant K LPF: L સિરીઝ હાઈ બ્લોક, C શન્ટ હાઈ શોર્ટ''

\end{mnemonicbox}
\subsection*{પ્રશ્ન 5(a) OR [3
ગુણ]}\label{q5a}

\textbf{500 Ω ના લોડ પ્રતિકાર સાથે 2 KHz ની કટ-ઓફ આવતર્ન ધરાવતા T સવિભાગ
સાથે ઉચ્ચ પાસ ફિલ્ટર ડિઝાઇન કરો.}

\begin{solutionbox}

\textbf{આકૃતિ: હાઇ પાસ T-સેક્શન ફિલ્ટર}

\begin{verbatim}
      C/2          C/2
   o{-{-}{-}||{-}{-}{-}{-}o{-}{-}{-}||{-}{-}{-}o}
   |         |        |
   |         L        |
   |        OOO       |
   |         |        |
  IN        {-{-}{-}      OUT}
             {-}
\end{verbatim}


{\def\LTcaptype{none} % do not increment counter
\vspace{-5pt}
\captionof{table}{હાઇ પાસ ફિલ્ટર ડિઝાઇન}
\vspace{-10pt}
\begin{longtable}[]{@{}
  >{\raggedright\arraybackslash}p{(\linewidth - 6\tabcolsep) * \real{0.2683}}
  >{\raggedright\arraybackslash}p{(\linewidth - 6\tabcolsep) * \real{0.2195}}
  >{\raggedright\arraybackslash}p{(\linewidth - 6\tabcolsep) * \real{0.3171}}
  >{\raggedright\arraybackslash}p{(\linewidth - 6\tabcolsep) * \real{0.1951}}@{}}
\toprule\noalign{}
\begin{minipage}[b]{\linewidth}\raggedright
પેરામીટર
\end{minipage} & \begin{minipage}[b]{\linewidth}\raggedright
ફોર્મ્યુલા
\end{minipage} & \begin{minipage}[b]{\linewidth}\raggedright
ગણતરી
\end{minipage} & \begin{minipage}[b]{\linewidth}\raggedright
પરિણામ
\end{minipage} \\
\midrule\noalign{}
\endhead
\bottomrule\noalign{}
\endlastfoot
કટઓફ ફ્રીક્વન્સી (fc) & આપેલ & 2 kHz & 2 kHz \\
લોડ રેઝીસ્ટન્સ (R_{0}) & આપેલ & 500 Ω & 500 Ω \\
સિરીઝ કેપેસિટન્સ (C/2) &

C = 1/(πfcR_{0}) &

C = 1/(π\times2\times10^{3}\times500) &

C = 0.318

μF \\
કુલ કેપેસિટન્સ (C) & 2 \times (C/2) & 2 \times 0.159 μF & C = 0.318 μF \\
શન્ટ ઇન્ડક્ટન્સ (L) &

L = R_{0}/(πfc) &

L = 500/(π\times2\times10^{3}) &

L = 79.6 mH \\

\end{longtable}
}

\end{solutionbox}
\begin{mnemonicbox}
``હાઇ પાસ T: C સિરીઝમાં DC બ્લોક, L શન્ટમાં હાઇ પાસ''

\end{mnemonicbox}
\subsection*{પ્રશ્ન 5(b) OR [4
ગુણ]}\label{q5b}

\textbf{ફિલ્ટર્સનું વર્ગીકરણ આપો.}

\begin{solutionbox}

\textbf{આકૃતિ: ફિલ્ટર વર્ગીકરણ}

\begin{verbatim}
                     Filters
                        |
     +{-{-}{-}{-}{-}{-}{-}{-}{-}{-}+{-}{-}{-}{-}{-}{-}{-}+{-}{-}{-}{-}{-}{-}{-}{-}{-}{-}+}
     |          |                  |
  By Function By Design      By Implementation
     |          |                  |
+{-{-}{-}{-}+{-}{-}{-}{-}+     |              +{-}{-}{-}+{-}{-}{-}+}
|         |     |              |       |
Low Pass  High Pass...     Analog    Digital
          |     |              |       |
   Band Pass    +{-{-}{-}{-}{-}+{-}{-}{-}{-}{-}{-}{-}{-}+}
       |              |        |
   Band Stop      Passive    Active
\end{verbatim}


{\def\LTcaptype{none} % do not increment counter
\vspace{-5pt}
\captionof{table}{ફિલ્ટર્સનું વર્ગીકરણ}
\vspace{-10pt}
\begin{longtable}[]{@{}
  >{\raggedright\arraybackslash}p{(\linewidth - 4\tabcolsep) * \real{0.4286}}
  >{\raggedright\arraybackslash}p{(\linewidth - 4\tabcolsep) * \real{0.1667}}
  >{\raggedright\arraybackslash}p{(\linewidth - 4\tabcolsep) * \real{0.4048}}@{}}
\toprule\noalign{}
\begin{minipage}[b]{\linewidth}\raggedright
વર્ગીકરણ દ્વારા
\end{minipage} & \begin{minipage}[b]{\linewidth}\raggedright
પ્રકારો
\end{minipage} & \begin{minipage}[b]{\linewidth}\raggedright
વિશેષતાઓ
\end{minipage} \\
\midrule\noalign{}
\endhead
\bottomrule\noalign{}
\endlastfoot
\textbf{ફંક્શન} & લો પાસ & કટઓફની નીચેની ફ્રીક્વન્સીઓ પસાર કરે \\
& હાઇ પાસ & કટઓફની ઉપરની ફ્રીક્વન્સીઓ પસાર કરે \\
& બેન્ડ પાસ & બેન્ડની અંદરની ફ્રીક્વન્સીઓ પસાર કરે \\
& બેન્ડ સ્ટોપ & બેન્ડની અંદરની ફ્રીક્વન્સીઓ નકારે \\
& ઓલ પાસ & બધી ફ્રીક્વન્સીઓ પસાર કરે પણ ફેઝ સુધારે \\
\textbf{ડિઝાઇન} & પેસિવ & પેસિવ ઘટકો (R, L, C) વાપરે \\
& એક્ટિવ & એક્ટિવ ઘટકો (ઓપ-એમ્પ્સ) વાપરે \\
\textbf{રિસ્પોન્સ} & બટરવર્થ & મેક્સિમલી ફલેટ રિસ્પોન્સ \\
& ચેબિશેવ & પાસબેન્ડમાં રિપલ, સ્ટીપર રોલઓફ \\
& બેસેલ & લિનિયર ફેઝ રિસ્પોન્સ \\
& એલિપ્ટિક & પાસબેન્ડ અને સ્ટોપબેન્ડ બંનેમાં રિપલ \\
\textbf{ઇમ્પ્લિમેન્ટેશન} & પેસિવ ફિલ્ટર પ્રકારો & Constant-k, m-derived,
composite \\
\end{longtable}
}

\end{solutionbox}
\begin{mnemonicbox}
``FLHBA: ફંક્શન (લો/હાઇ/બેન્ડ/ઓલ-પાસ), ડિઝાઇન, રિસ્પોન્સ,
ઇમ્પ્લિમેન્ટેશન''

\end{mnemonicbox}
\subsection*{પ્રશ્ન 5(c) OR [7
ગુણ]}\label{q5c}

\textbf{Constant K હાઇ પાસ ફિલ્ટર સમજાવો.}

\begin{solutionbox}

\textbf{આકૃતિ: Constant K હાઇ પાસ ફિલ્ટર T અને π સેક્શન}

\begin{verbatim}
T{-section:                     π{-}section:}
     C/2          C/2               C
o{-{-}{-}{-}||{-}{-}{-}{-}{-}{-}o{-}{-}{-}{-}||{-}{-}{-}{-}{-}{-}{-}o   o{-}{-}{-}{-}||{-}{-}{-}{-}o}
             |                   |      |
             L                   L/2    L/2
             |                   |      |
             |                   |      |
o{-{-}{-}{-}{-}{-}{-}{-}{-}{-}{-}{-}o{-}{-}{-}{-}{-}{-}{-}{-}{-}{-}{-}{-}{-}{-}o  o{-}{-}{-}{-}{-}{-}{-}{-}{-}{-}o}
\end{verbatim}

\textbf{Constant K હાઇ પાસ ફિલ્ટર:}

\begin{itemize}
\tightlist
\item
  કટઓફ ફ્રીક્વન્સી (fc) \textbf{ઉપરની ફ્રીક્વન્સીઓ} પસાર કરે છે
\item
  fc નીચેની ફ્રીક્વન્સીઓ ઘટાડે છે
\item
  ``Constant K'' નો અર્થ છે કે સિરીઝ અને શન્ટ ઈમ્પિડન્સના ગુણાકારો બધી ફ્રીક્વન્સી
  પર સ્થિર રહે છે (Z_{1}Z_{2} = K^{2})
\end{itemize}


{\def\LTcaptype{none} % do not increment counter
\vspace{-5pt}
\captionof{table}{T અને π સેક્શન પેરામીટર્સ}
\vspace{-10pt}
\begin{longtable}[]{@{}lll@{}}
\toprule\noalign{}
પેરામીટર & T-સેક્શન & π-સેક્શન \\
\midrule\noalign{}
\endhead
\bottomrule\noalign{}
\endlastfoot
સિરીઝ આર્મ & દરેક છેડે C/2 & મધ્યમાં C \\
શન્ટ આર્મ & મધ્યમાં L & દરેક છેડે L/2 \\
કટઓફ ફ્રીક્વન્સી & fc = 1/(π\sqrtLC) & fc = 1/(π\sqrtLC) \\
કેરેક્ટરિસ્ટિક ઇમ્પિડન્સ & Z_{0} = \sqrt(L/C) & Z_{0} = \sqrt(L/C) \\
L માટે ડિઝાઇન ઇક્વેશન &

L = Z_{0}/(πfc) &

L = Z_{0}/(πfc) \\

C માટે ડિઝાઇન ઇક્વેશન &

C = 1/(πfcZ_{0}) &

C = 1/(πfcZ_{0}) \\

\end{longtable}
}

\textbf{ફ્રીક્વન્સી રિસ્પોન્સ:}

\begin{itemize}
\tightlist
\item
  DC અને લો ફ્રીક્વન્સીઓ બ્લોક કરે છે
\item
  હાઇ ફ્રીક્વન્સીઓ ન્યૂનતમ એટેન્યુએશન સાથે પસાર કરે છે
\item
  કટઓફ ફ્રીક્વન્સી નીચે જતાં એટેન્યુએશન વધે છે
\item
  ખૂબ ઊંચી ફ્રીક્વન્સીઓ પર ફેઝ શિફ્ટ 0^\circ તરફ જાય છે
\end{itemize}

\end{solutionbox}
\begin{mnemonicbox}
``Constant K HPF: C સિરીઝ લો બ્લોક, L શન્ટ હાઇ પાસ''

\end{mnemonicbox}

\end{document}
