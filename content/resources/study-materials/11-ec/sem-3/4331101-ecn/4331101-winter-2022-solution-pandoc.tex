\documentclass[10pt,a4paper]{article}

% content/resources/templates/preamble.tex
\usepackage[margin=0.6in]{geometry}
\author{Milav Dabgar}
\usepackage{amsmath,amssymb,amsthm}
\usepackage{booktabs}
\usepackage{multirow}
\usepackage{xcolor}
\usepackage{tcolorbox}
\tcbuselibrary{breakable,skins}
\usepackage[colorlinks=true,linkcolor=blue]{hyperref}
\usepackage{titlesec}
\usepackage{enumitem}
\usepackage{tikz}
\usepackage{pgfplots}
\usepackage{circuitikz}
\usepackage[version=4]{mhchem}
\usepackage{longtable}
\usepackage{array}
\usepackage{float}
\usepackage{caption}
\usepackage{listings}

\lstset{
  basicstyle=\small\ttfamily,
  breaklines=true,
  breakatwhitespace=false,
  postbreak=\mbox{\textcolor{red}{$\hookrightarrow$}\space},
  float=false,
  numbers=left,
  numberstyle=\tiny\color{gray},
  numbersep=10pt,
  xleftmargin=2em,
  keywordstyle=\color{blue},
  commentstyle=\color{green!60!black},
  stringstyle=\color{purple},
  backgroundcolor=\color{gray!5},
  showstringspaces=false,
  tabsize=2,
  captionpos=b,
  keepspaces=true,
  columns=flexible
}

\pgfplotsset{compat=1.18}
\usetikzlibrary{shapes,arrows,positioning,calc,patterns,decorations.pathmorphing,decorations.markings,arrows.meta}

% Color scheme
\definecolor{headcolor}{RGB}{0,102,204}
\definecolor{keycolor}{RGB}{220,20,60}
\definecolor{solutioncolor}{RGB}{34,139,34}
\definecolor{mnemoniccolor}{RGB}{148,0,211}
\definecolor{codecolor}{RGB}{0,0,100}

% Spacing
\setlength{\parskip}{3pt}
\setlist[itemize]{nosep}
\setlist[enumerate]{nosep}

% Title formatting
\titleformat{\section}{\Large\bfseries\color{headcolor}}{\thesection}{1em}{}
\titleformat{\subsection}{\large\bfseries\color{headcolor}}{\thesubsection}{1em}{}

% Pandoc tightlist compatibility
\providecommand{\tightlist}{%
  \setlength{\itemsep}{0pt}\setlength{\parskip}{0pt}}

% Pandoc longtable compatibility
\newcounter{none}
\def\thenone{}


% content/resources/templates/english-boxes.tex
% This file is currently empty - it exists to maintain consistency with the import structure.
% Add custom environments here if needed in the future.


\begin{document}

\begin{center}
{\Huge\bfseries\color{headcolor} Subject Name Solutions}\\[5pt]
{\LARGE 4331101 -- Winter 2022}\\[3pt]
{\large Semester 1 Study Material}\\[3pt]
{\normalsize\textit{Detailed Solutions and Explanations}}
\end{center}

\vspace{10pt}

\subsection*{Question 1(a) [3 marks]}\label{q1a}

\textbf{Define: 1) Branch 2) Junction 3) Mesh}

\begin{solutionbox}

\begin{itemize}
\tightlist
\item
  \textbf{Branch}: A branch is a single circuit element or a combination
  of elements connected between two nodes of a network.
\item
  \textbf{Junction}: A junction (or node) is a point in a circuit where
  two or more circuit elements are connected together.
\item
  \textbf{Mesh}: A mesh is a closed path in a network where no other
  closed path exists inside it.
\end{itemize}

\end{solutionbox}
\begin{mnemonicbox}
``BJM: Branches Join at junctions to Make meshes''

\end{mnemonicbox}
\subsection*{Question 1(b) [4 marks]}\label{q1b}

\textbf{Write voltage division and current division rule with necessary
circuit diagram}

\begin{solutionbox}

\textbf{Voltage Division Rule}: In a series circuit, voltage across any
component is proportional to its resistance.

\begin{center}
\textbf{Mermaid Diagram (Code)}
\begin{verbatim}
{Shaded}
{Highlighting}[]
graph LR
    A(({+)) {-}{-}{-} B[R1] {-}{-}{-} C[R2] {-}{-}{-} D((–))}
    E[V1] {-.{-} B}
    F[V2] {-.{-} C}
    G[VS] {-.{-} A}
{Highlighting}
{Shaded}
\end{verbatim}
\end{center}

\begin{itemize}
\tightlist
\item
  \textbf{Formula}: V_{1} = VS \times (R_{1}/(R_{1}+R_{2}))
\item
  \textbf{Application}: Used to find individual voltage drops across
  series components
\end{itemize}

\textbf{Current Division Rule}: In a parallel circuit, current through
any branch is inversely proportional to its resistance.

\begin{center}
\textbf{Mermaid Diagram (Code)}
\begin{verbatim}
{Shaded}
{Highlighting}[]
graph LR
    A(({+)) {-}{-}{-} B {-}{-}{-} C((–))}
    B {-{-}{-} D[R1] {-}{-}{-} C}
    B {-{-}{-} E[R2] {-}{-}{-} C}
    F[I1] {-.{-} D}
    G[I2] {-.{-} E}
    H[IS] {-.{-} A}
{Highlighting}
{Shaded}
\end{verbatim}
\end{center}

\begin{itemize}
\tightlist
\item
  \textbf{Formula}: I_{1} = IS \times (R_{2}/(R_{1}+R_{2}))
\item
  \textbf{Key concept}: Current takes path of least resistance
\end{itemize}

\end{solutionbox}
\begin{mnemonicbox}
``VoSe CuPa: Voltage divides in Series, Current
divides in Parallel''

\end{mnemonicbox}
\subsection*{Question 1(c) [7 marks]}\label{q1c}

\textbf{Draw Graph and Tree for a network shown in fig(1). Show link
currents on a graph. Also write Tie-set schedule for a tree of network
shown in fig.~(1)}

\begin{solutionbox}

\textbf{Graph of the Network}:

\begin{center}
\textbf{Mermaid Diagram (Code)}
\begin{verbatim}
{Shaded}
{Highlighting}[]
graph LR
    A((A)) {-{-}{-} B((B))}
    A {-{-}{-} C((C))}
    A {-{-}{-} D((D))}
    B {-{-}{-} C}
    B {-{-}{-} D}
    C {-{-}{-} D}
    A {-{-} 1 {-}{-}{-} B}
    A {-{-} 3 {-}{-}{-} C}
    B {-{-} 2 {-}{-}{-} D}
    C {-{-} 5 {-}{-}{-} D}
    B {-{-} 6 {-}{-}{-} C}
    A {-{-} 7 {-}{-}{-} D}
    style A fill:\#f9f,stroke:\#333,stroke{-width:2px}
    style B fill:\#f9f,stroke:\#333,stroke{-width:2px}
    style C fill:\#f9f,stroke:\#333,stroke{-width:2px}
    style D fill:\#f9f,stroke:\#333,stroke{-width:2px}
{Highlighting}
{Shaded}
\end{verbatim}
\end{center}

\textbf{Tree of the Network} (shown with bold edges):

\begin{center}
\textbf{Mermaid Diagram (Code)}
\begin{verbatim}
{Shaded}
{Highlighting}[]
graph LR
    A((A)) {-{-}{-} B((B))}
    A {-{-}{-} C((C))}
    C {-{-}{-} D((D))}
    style A fill:\#f9f,stroke:\#333,stroke{-width:2px}
    style B fill:\#f9f,stroke:\#333,stroke{-width:2px}
    style C fill:\#f9f,stroke:\#333,stroke{-width:2px}
    style D fill:\#f9f,stroke:\#333,stroke{-width:2px}
    linkStyle 0 stroke{-width:4px,stroke:green}
    linkStyle 1 stroke{-width:4px,stroke:green}
    linkStyle 2 stroke{-width:4px,stroke:green}
{Highlighting}
{Shaded}
\end{verbatim}
\end{center}

\textbf{Link Currents} (shown on remaining branches that are not part of
the tree):

\begin{itemize}
\tightlist
\item
  Link 1: Branch 2 (BD)
\item
  Link 2: Branch 6 (BC)
\item
  Link 3: Branch 7 (AD)
\item
  Link 4: Branch 5 (CD)
\end{itemize}

\textbf{Tie-set Schedule}:

{\def\LTcaptype{none} % do not increment counter
\begin{longtable}[]{@{}
  >{\raggedright\arraybackslash}p{(\linewidth - 14\tabcolsep) * \real{0.1463}}
  >{\raggedright\arraybackslash}p{(\linewidth - 14\tabcolsep) * \real{0.1220}}
  >{\raggedright\arraybackslash}p{(\linewidth - 14\tabcolsep) * \real{0.1220}}
  >{\raggedright\arraybackslash}p{(\linewidth - 14\tabcolsep) * \real{0.1220}}
  >{\raggedright\arraybackslash}p{(\linewidth - 14\tabcolsep) * \real{0.1220}}
  >{\raggedright\arraybackslash}p{(\linewidth - 14\tabcolsep) * \real{0.1220}}
  >{\raggedright\arraybackslash}p{(\linewidth - 14\tabcolsep) * \real{0.1220}}
  >{\raggedright\arraybackslash}p{(\linewidth - 14\tabcolsep) * \real{0.1220}}@{}}
\toprule\noalign{}
\begin{minipage}[b]{\linewidth}\raggedright
Link/Tree Branch
\end{minipage} & \begin{minipage}[b]{\linewidth}\raggedright
Branch 1 (AB)
\end{minipage} & \begin{minipage}[b]{\linewidth}\raggedright
Branch 3 (AC)
\end{minipage} & \begin{minipage}[b]{\linewidth}\raggedright
Branch 4 (CD)
\end{minipage} & \begin{minipage}[b]{\linewidth}\raggedright
Branch 2 (BD)
\end{minipage} & \begin{minipage}[b]{\linewidth}\raggedright
Branch 6 (BC)
\end{minipage} & \begin{minipage}[b]{\linewidth}\raggedright
Branch 7 (AD)
\end{minipage} & \begin{minipage}[b]{\linewidth}\raggedright
Branch 5 (CD)
\end{minipage} \\
\midrule\noalign{}
\endhead
\bottomrule\noalign{}
\endlastfoot
Link 1 (BD) & 1 & 0 & 0 & 1 & 0 & 0 & 0 \\
Link 2 (BC) & 1 & 1 & 0 & 0 & 1 & 0 & 0 \\
Link 3 (AD) & 0 & 0 & 1 & 0 & 0 & 1 & 0 \\
Link 4 (CD) & 0 & 0 & 1 & 0 & 0 & 0 & 1 \\
\end{longtable}
}

\end{solutionbox}
\begin{mnemonicbox}
``TGLT: Trees Generate Link-current Tie-sets''

\end{mnemonicbox}
\subsection*{Question 1(c) OR [7
marks]}\label{q1c}

\textbf{Draw Graph and Tree for a network shown in fig(1). Show branch
voltages on tree. Also write cut-set schedule for a tree of network
shown on fig.(1)}

\begin{solutionbox}

\textbf{Graph of the Network}:

\begin{center}
\textbf{Mermaid Diagram (Code)}
\begin{verbatim}
{Shaded}
{Highlighting}[]
graph LR
    A((A)) {-{-}{-} B((B))}
    A {-{-}{-} C((C))}
    A {-{-}{-} D((D))}
    B {-{-}{-} C}
    B {-{-}{-} D}
    C {-{-}{-} D}
    A {-{-} 1 {-}{-}{-} B}
    A {-{-} 3 {-}{-}{-} C}
    B {-{-} 2 {-}{-}{-} D}
    C {-{-} 5 {-}{-}{-} D}
    B {-{-} 6 {-}{-}{-} C}
    A {-{-} 7 {-}{-}{-} D}
    style A fill:\#f9f,stroke:\#333,stroke{-width:2px}
    style B fill:\#f9f,stroke:\#333,stroke{-width:2px}
    style C fill:\#f9f,stroke:\#333,stroke{-width:2px}
    style D fill:\#f9f,stroke:\#333,stroke{-width:2px}
{Highlighting}
{Shaded}
\end{verbatim}
\end{center}

\textbf{Tree of the Network} (shown with bold edges and branch
voltages):

\begin{center}
\textbf{Mermaid Diagram (Code)}
\begin{verbatim}
{Shaded}
{Highlighting}[]
graph LR
    A((A)) {-{-}"V_{1}"{-}{-}{} B((B))}
    A {-{-}"V_{3}"{-}{-}{} C((C))}
    C {-{-}"V_{4}"{-}{-}{} D((D))}
    style A fill:\#f9f,stroke:\#333,stroke{-width:2px}
    style B fill:\#f9f,stroke:\#333,stroke{-width:2px}
    style C fill:\#f9f,stroke:\#333,stroke{-width:2px}
    style D fill:\#f9f,stroke:\#333,stroke{-width:2px}
    linkStyle 0 stroke{-width:4px,stroke:green}
    linkStyle 1 stroke{-width:4px,stroke:green}
    linkStyle 2 stroke{-width:4px,stroke:green}
{Highlighting}
{Shaded}
\end{verbatim}
\end{center}

\textbf{Cut-set Schedule}:

{\def\LTcaptype{none} % do not increment counter
\begin{longtable}[]{@{}
  >{\raggedright\arraybackslash}p{(\linewidth - 14\tabcolsep) * \real{0.1322}}
  >{\raggedright\arraybackslash}p{(\linewidth - 14\tabcolsep) * \real{0.1240}}
  >{\raggedright\arraybackslash}p{(\linewidth - 14\tabcolsep) * \real{0.1240}}
  >{\raggedright\arraybackslash}p{(\linewidth - 14\tabcolsep) * \real{0.1240}}
  >{\raggedright\arraybackslash}p{(\linewidth - 14\tabcolsep) * \real{0.1240}}
  >{\raggedright\arraybackslash}p{(\linewidth - 14\tabcolsep) * \real{0.1240}}
  >{\raggedright\arraybackslash}p{(\linewidth - 14\tabcolsep) * \real{0.1240}}
  >{\raggedright\arraybackslash}p{(\linewidth - 14\tabcolsep) * \real{0.1240}}@{}}
\toprule\noalign{}
\begin{minipage}[b]{\linewidth}\raggedright
Cut-set/Branch
\end{minipage} & \begin{minipage}[b]{\linewidth}\raggedright
Branch 1 (AB)
\end{minipage} & \begin{minipage}[b]{\linewidth}\raggedright
Branch 3 (AC)
\end{minipage} & \begin{minipage}[b]{\linewidth}\raggedright
Branch 4 (CD)
\end{minipage} & \begin{minipage}[b]{\linewidth}\raggedright
Branch 2 (BD)
\end{minipage} & \begin{minipage}[b]{\linewidth}\raggedright
Branch 6 (BC)
\end{minipage} & \begin{minipage}[b]{\linewidth}\raggedright
Branch 7 (AD)
\end{minipage} & \begin{minipage}[b]{\linewidth}\raggedright
Branch 5 (CD)
\end{minipage} \\
\midrule\noalign{}
\endhead
\bottomrule\noalign{}
\endlastfoot
Cut-set 1 (AB) & 1 & 0 & 0 & -1 & -1 & 0 & 0 \\
Cut-set 2 (AC) & 0 & 1 & 0 & 0 & 1 & -1 & 0 \\
Cut-set 3 (CD) & 0 & 0 & 1 & 1 & 0 & 1 & 1 \\
\end{longtable}
}

\end{solutionbox}
\begin{mnemonicbox}
``CGVS: Cut-sets Generate Voltage Sources''

\end{mnemonicbox}
\subsection*{Question 2(a) [3 marks]}\label{q2a}

\textbf{Define: 1) Active and passive network 2)Unilateral and Bilateral
network.}

\begin{solutionbox}

\begin{itemize}
\item
  \textbf{Active Network}: A network containing one or more sources of
  EMF (voltage/current sources) that supply energy to the circuit.
\item
  \textbf{Passive Network}: A network containing only passive elements
  like resistors, capacitors, and inductors with no energy sources.
\item
  \textbf{Unilateral Network}: A network in which the properties and
  performance change when input and output terminals are interchanged.
\item
  \textbf{Bilateral Network}: A network in which the properties and
  performance remain unchanged when input and output terminals are
  interchanged.
\end{itemize}

\textbf{Diagram}:

\begin{center}
\textbf{Mermaid Diagram (Code)}
\begin{verbatim}
{Shaded}
{Highlighting}[]
graph LR
    subgraph "Network Types"
    A[Active: Contains sources]
    B[Passive: No sources]
    C[Unilateral: Diodes/Transistors]
    D[Bilateral: R, L, C elements]
    end
{Highlighting}
{Shaded}
\end{verbatim}
\end{center}

\end{solutionbox}
\begin{mnemonicbox}
``APUB: Active Provides energy, Unilateral Blocks
reversal''

\end{mnemonicbox}
\subsection*{Question 2(b) [4 marks]}\label{q2b}

\textbf{Write equation for Z parameter and derive Z11, Z12, Z21, Z22
from that equation.}

\begin{solutionbox}

Z-parameters define the relationship between port voltages and currents
in a two-port network:

\textbf{Equations}:

\begin{itemize}
\tightlist
\item
  V_{1} = Z_{1}_{1}I_{1} + Z_{1}_{2}I_{2}
\item
  V_{2} = Z_{2}_{1}I_{1} + Z_{2}_{2}I_{2}
\end{itemize}

\textbf{Derivation}:

\begin{itemize}
\tightlist
\item
  \textbf{Z_{1}_{1} = V_{1}/I_{1}} (with I_{2} = 0): Input impedance with output port
  open-circuited
\item
  \textbf{Z_{1}_{2} = V_{1}/I_{2}} (with I_{1} = 0): Reverse transfer impedance with
  input port open-circuited
\item
  \textbf{Z_{2}_{1} = V_{2}/I_{1}} (with I_{2} = 0): Forward transfer impedance with
  output port open-circuited
\item
  \textbf{Z_{2}_{2} = V_{2}/I_{2}} (with I_{1} = 0): Output impedance with input port
  open-circuited
\end{itemize}

\end{solutionbox}
\begin{mnemonicbox}
``Z Impedance: Open circuit gives correct
Parameters''

\end{mnemonicbox}
\subsection*{Question 2(c) [7 marks]}\label{q2c}

\textbf{Derive equation of characteristic impedance(ZOT) for a standard
T network.}

\begin{solutionbox}

For a standard T-network:

\begin{center}
\textbf{Mermaid Diagram (Code)}
\begin{verbatim}
{Shaded}
{Highlighting}[]
graph LR
    A((Port{-1)) {-}{-}{-} B[Z1] {-}{-}{-} C((Junction))}
    C {-{-}{-} D[Z2] {-}{-}{-} E((Port{-}2))}
    C {-{-}{-} F[Z3] {-}{-}{-} G((Ground))}
{Highlighting}
{Shaded}
\end{verbatim}
\end{center}

\textbf{Derivation Steps}:

\begin{enumerate}
\tightlist
\item
  For a symmetric T-network, Z_{1} = Z_{2}
\item
  Under matched condition, input impedance equals characteristic
  impedance
\item
  Z_{0}_{t} = Z_{1} + (Z_{1}\timesZ_{3})/(Z_{1} + Z_{3})
\item
  For balanced T-network where Z_{1} = Z_{2} = Z/2 and Z_{3} = Z:
\item
  Z_{0}_{t} = Z/2 + (Z/2\timesZ)/(Z/2 + Z)
\item
  Z_{0}_{t} = Z/2 + (Z^{2}/2)/(Z + Z/2)
\item
  Z_{0}_{t} = Z/2 + (Z^{2}/2)/(3Z/2)
\item
  Z_{0}_{t} = Z/2 + Z^{2}/3Z
\item
  Z_{0}_{t} = Z/2 + Z/3
\item
  Z_{0}_{t} = (3Z + 2Z)/6
\item
  Z_{0}_{t} = \sqrt(Z_{1}(Z_{1} + 2Z_{3}))
\end{enumerate}

\textbf{Final Equation}: Z_{0}_{t} = \sqrt(Z_{1}(Z_{1} + 2Z_{3}))

\end{solutionbox}
\begin{mnemonicbox}
``TO Impedance: Two arms Over middle branch''

\end{mnemonicbox}
\subsection*{Question 2(a) OR [3
marks]}\label{q2a}

\textbf{Define: 1)Driving point impedance 2) Transfer impedance}

\begin{solutionbox}

\begin{itemize}
\item
  \textbf{Driving Point Impedance}: The ratio of voltage to current at
  the same port/pair of terminals when all other independent sources are
  set to zero.
\item
  \textbf{Transfer Impedance}: The ratio of voltage at one port to the
  current at another port when all other independent sources are set to
  zero.
\end{itemize}

\textbf{Diagram}:

\begin{center}
\textbf{Mermaid Diagram (Code)}
\begin{verbatim}
{Shaded}
{Highlighting}[]
graph LR
    subgraph "Impedance Types"
    A[Driving Point: V_{1/I_{1} or V_{2}/I_{2}]}
    B[Transfer: V_{2/I_{1} or V_{1}/I_{2}]}
    end
{Highlighting}
{Shaded}
\end{verbatim}
\end{center}

\end{solutionbox}
\begin{mnemonicbox}
``DTSS: Driving at Terminal Same, Transfer at
Separate''

\end{mnemonicbox}
\subsection*{Question 2(b) OR [4
marks]}\label{q2b}

\textbf{Explain Kirchhoff's voltage law with example.}

\begin{solutionbox}

\textbf{Kirchhoff's Voltage Law (KVL)}: The algebraic sum of all
voltages around any closed loop in a circuit is zero.

\textbf{Mathematically}: \sumV = 0 (around a closed loop)

\textbf{Circuit Example}:

\begin{center}
\textbf{Mermaid Diagram (Code)}
\begin{verbatim}
{Shaded}
{Highlighting}[]
graph LR
    A(({+)) {-}{-}"10V"{-}{-}{} B}
    B {-{-}"R_{1} = 2Ω"{-}{-}{} C}
    C {-{-}"R_{2} = 3Ω"{-}{-}{} D}
    D {-{-}"R_{3} = 5Ω"{-}{-}{} A}
    style A fill:\#f9f,stroke:\#333,stroke{-width:2px}
    style B fill:\#f9f,stroke:\#333,stroke{-width:2px}
    style C fill:\#f9f,stroke:\#333,stroke{-width:2px}
    style D fill:\#f9f,stroke:\#333,stroke{-width:2px}
{Highlighting}
{Shaded}
\end{verbatim}
\end{center}

If I = 1A, then:

\begin{itemize}
\tightlist
\item
  V_{1} = 1A \times 2Ω = 2V
\item
  V_{2} = 1A \times 3Ω = 3V
\item
  V_{3} = 1A \times 5Ω = 5V
\end{itemize}

Applying KVL: 10V - 2V - 3V - 5V = 0 ✓

\end{solutionbox}
\begin{mnemonicbox}
``VACZ: Voltages Around Closed loop are Zero''

\end{mnemonicbox}
\subsection*{Question 2(c) OR [7
marks]}\label{q2c}

\textbf{Derive equation to convert π network into T network.}

\begin{solutionbox}

\textbf{π Network to T Network Conversion}:

\begin{center}
\textbf{Mermaid Diagram (Code)}
\begin{verbatim}
{Shaded}
{Highlighting}[]
graph TD
    subgraph "π Network"
    A1((A)) {-{-}{-} B1((B))}
    A1 {-{-}{-} Y1[Ya] {-}{-}{-} C1}
    B1 {-{-}{-} Y2[Yb] {-}{-}{-} C1}
    A1 {-{-}{-} Y3[Yc] {-}{-}{-} B1}
    C1((C))
    end

    subgraph "T Network"
    A2((A)) {-{-}{-} Z1[Za] {-}{-}{-} D2((D))}
    B2((B)) {-{-}{-} Z2[Zb] {-}{-}{-} D2}
    D2 {-{-}{-} Z3[Zc] {-}{-}{-} C2((C))}
    end
{Highlighting}
{Shaded}
\end{verbatim}
\end{center}

\textbf{Conversion Equations}:

\begin{enumerate}
\tightlist
\item
  Za = (Ya \times Yc) / Y∆
\item
  Zb = (Yb \times Yc) / Y∆
\item
  Zc = (Ya \times Yb) / Y∆
\end{enumerate}

Where Y∆ = Ya + Yb + Yc

\textbf{Derivation}:

\begin{enumerate}
\tightlist
\item
  Start with Y-parameters of π-network
\item
  Express Y-parameters in terms of branch admittances
\item
  Convert to Z-parameters using matrix inversion
\item
  Express T-network impedances in terms of Z-parameters
\item
  Simplify to get the conversion formulas above
\end{enumerate}

\end{solutionbox}
\begin{mnemonicbox}
``PIE to TEA: Product over sum for opposite branch''

\end{mnemonicbox}
\subsection*{Question 3(a) [3 marks]}\label{q3a}

\textbf{Explain Kirchhoff's current law with example.}

\begin{solutionbox}

\textbf{Kirchhoff's Current Law (KCL)}: The algebraic sum of all
currents entering and leaving a node must equal zero.

\textbf{Mathematically}: \sumI = 0 (at any node)

\textbf{Circuit Example}:

\begin{center}
\textbf{Mermaid Diagram (Code)}
\begin{verbatim}
{Shaded}
{Highlighting}[]
graph TD
    A[I_{1 = 5A] {-}{-}{} B((Node))}
    C[I_{2 = 2A] {-}{-}{} B}
    B {-{-}{} D[I_{3} = 3A]}
    B {-{-}{} E[I_{4} = 4A]}
    style B fill:\#f9f,stroke:\#333,stroke{-width:2px}
{Highlighting}
{Shaded}
\end{verbatim}
\end{center}

Applying KCL at node B:

\begin{itemize}
\tightlist
\item
  Currents entering: I_{1} + I_{2} = 5A + 2A = 7A
\item
  Currents leaving: I_{3} + I_{4} = 3A + 4A = 7A
\item
  Therefore: I_{1} + I_{2} - I_{3} - I_{4} = 5 + 2 - 3 - 4 = 0 ✓
\end{itemize}

\end{solutionbox}
\begin{mnemonicbox}
``CuNoZ: Currents at Node are Zero''

\end{mnemonicbox}
\subsection*{Question 3(b) [4 marks]}\label{q3b}

\textbf{Explain mesh analysis with required equations.}

\begin{solutionbox}

\textbf{Mesh Analysis}: A circuit analysis technique that uses mesh
currents as variables to solve a circuit with multiple loops.

\textbf{Steps}:

\begin{enumerate}
\tightlist
\item
  Identify all meshes (closed loops) in the circuit
\item
  Assign a mesh current to each mesh
\item
  Apply KVL to each mesh
\item
  Solve the resulting system of equations
\end{enumerate}

\textbf{Example Circuit}:

\begin{center}
\textbf{Mermaid Diagram (Code)}
\begin{verbatim}
{Shaded}
{Highlighting}[]
graph LR
    A((A)) {-{-} R_{1} {-}{-}{-} B((B))}
    B {-{-} R_{3} {-}{-}{-} C((C))}
    A {-{-} R_{2} {-}{-}{-} C}
    A {-{-} V_{1} {-}{-}{-} D}
    D {-{-} {}+ {-}{-}{-} A}
    C {-{-} V_{2} {-}{-}{-} E}
    E {-{-} {}+ {-}{-}{-} C}
    style A fill:\#f9f,stroke:\#333,stroke{-width:2px}
    style B fill:\#f9f,stroke:\#333,stroke{-width:2px}
    style C fill:\#f9f,stroke:\#333,stroke{-width:2px}
{Highlighting}
{Shaded}
\end{verbatim}
\end{center}

\textbf{Equations}:

\begin{itemize}
\tightlist
\item
  Mesh 1: V_{1} = I_{1}R_{1} + I_{1}R_{2} - I_{2}R_{2}
\item
  Mesh 2: V_{2} = I_{2}R_{2} + I_{2}R_{3} - I_{1}R_{2}
\end{itemize}

\end{solutionbox}
\begin{mnemonicbox}
``MILK: Mesh Is Loop with KVL''

\end{mnemonicbox}
\subsection*{Question 3(c) [7 marks]}\label{q3c}

\textbf{State and explain Thevenin's theorem.}

\begin{solutionbox}

\textbf{Thevenin's Theorem}: Any linear network with voltage and current
sources can be replaced by an equivalent circuit consisting of a voltage
source (VTH) in series with a resistance (RTH).

\begin{center}
\textbf{Mermaid Diagram (Code)}
\begin{verbatim}
{Shaded}
{Highlighting}[]
graph TD
    subgraph "Original Network"
    A((A)) {-{-}{-} B[Complex Network] {-}{-}{-} C((B))}
    end
    subgraph "Thevenin Equivalent"
    D((A)) {-{-}{-} E[VTH] {-}{-}{-} F(({}+))}
    F {-{-}{-} G[RTH] {-}{-}{-} H((B))}
    end
{Highlighting}
{Shaded}
\end{verbatim}
\end{center}

\textbf{Steps to Find Thevenin Equivalent}:

\begin{enumerate}
\tightlist
\item
  Remove the load from the terminals of interest
\item
  Calculate the open-circuit voltage (VOC) across these terminals (=
  VTH)
\item
  Calculate the resistance looking back into the circuit with all
  sources replaced by their internal resistances (= RTH)
\item
  The Thevenin equivalent consists of VTH in series with RTH
\end{enumerate}

\textbf{Example Application}:

\begin{itemize}
\tightlist
\item
  Original complex circuit with load RL
\item
  Remove RL and find VOC = VTH
\item
  Deactivate sources and find RTH
\item
  Reconnect RL to simplified Thevenin equivalent
\end{itemize}

\end{solutionbox}
\begin{mnemonicbox}
``TORV: Thevenin's Open-circuit Resistance and
Voltage''

\end{mnemonicbox}
\subsection*{Question 3(a) OR [3
marks]}\label{q3a}

\textbf{State and explain reciprocity theorem.}

\begin{solutionbox}

\textbf{Reciprocity Theorem}: In a linear, bilateral network, if a
voltage source in one branch produces a current in another branch, then
the same voltage source, if placed in the second branch, will produce
the same current in the first branch.

\begin{center}
\textbf{Mermaid Diagram (Code)}
\begin{verbatim}
{Shaded}
{Highlighting}[]
graph LR
    subgraph "Original Circuit"
    direction LR
    A((A)) {-{-}{-} B[V] {-}{-}{-} C((B))}
    C {-{-}{-} D[Network] {-}{-}{-} E((C))}
    E {-{-}{-} F[Ammeter] {-}{-}{-} A}
    end

    subgraph "Reciprocal Circuit"
    direction LR
    G((A)) {-{-}{-} H[Ammeter] {-}{-}{-} I((B))}
    I {-{-}{-} J[Network] {-}{-}{-} K((C))}
    K {-{-}{-} L[V] {-}{-}{-} G}
    end
{Highlighting}
{Shaded}
\end{verbatim}
\end{center}

\textbf{Mathematically}: If a voltage V_{1} in branch 1 produces current I_{2}
in branch 2, then voltage V_{1} in branch 2 will produce current I_{2} in
branch 1.

\textbf{Limitations}: Applies only to networks with:

\begin{itemize}
\tightlist
\item
  Linear elements
\item
  Bilateral elements (no diodes, transistors)
\item
  Single independent source
\end{itemize}

\end{solutionbox}
\begin{mnemonicbox}
``RESWAP: REciprocity SWAPs Position with identical
results''

\end{mnemonicbox}
\subsection*{Question 3(b) OR [4
marks]}\label{q3b}

\textbf{Explain nodal analysis with required equations.}

\begin{solutionbox}

\textbf{Nodal Analysis}: A circuit analysis technique that uses node
voltages as variables to solve a circuit.

\textbf{Steps}:

\begin{enumerate}
\tightlist
\item
  Choose a reference node (ground)
\item
  Assign voltage variables to remaining nodes
\item
  Apply KCL at each non-reference node
\item
  Solve the resulting system of equations
\end{enumerate}

\textbf{Example Circuit}:

\begin{center}
\textbf{Mermaid Diagram (Code)}
\begin{verbatim}
{Shaded}
{Highlighting}[]
graph LR
    A((Node 1)) {-{-} G_{1} {-}{-}{-} B((Ground))}
    C((Node 2)) {-{-} G_{2} {-}{-}{-} B}
    A {-{-} G_{3} {-}{-}{-} C}
    A {-{-} I_{1} {-}{-}{} B}
    C {-{-} I_{2} {-}{-}{} B}
    style A fill:\#f9f,stroke:\#333,stroke{-width:2px}
    style C fill:\#f9f,stroke:\#333,stroke{-width:2px}
    style B fill:\#f9f,stroke:\#333,stroke{-width:2px}
{Highlighting}
{Shaded}
\end{verbatim}
\end{center}

\textbf{Equations}:

\begin{itemize}
\tightlist
\item
  Node 1: I_{1} = V_{1}G_{1} + (V_{1}-V_{2})G_{3}
\item
  Node 2: I_{2} = V_{2}G_{2} + (V_{2}-V_{1})G_{3}
\end{itemize}

\end{solutionbox}
\begin{mnemonicbox}
``NKCV: Nodal uses KCL with Voltage variables''

\end{mnemonicbox}
\subsection*{Question 3(c) OR [7
marks]}\label{q3c}

\textbf{State and prove maximum power transfer theorem.}

\begin{solutionbox}

\textbf{Maximum Power Transfer Theorem}: A load connected to a source
will extract maximum power when its resistance equals the internal
resistance of the source.

\begin{center}
\textbf{Mermaid Diagram (Code)}
\begin{verbatim}
{Shaded}
{Highlighting}[]
graph LR
    A(({+)) {-}{-}{-} B[VS] {-}{-}{-} C((X))}
    C {-{-}{-} D[RS] {-}{-}{-} E((Y))}
    E {-{-}{-} F[RL] {-}{-}{-} G((Z))}
    G {-{-}{-} A}
    style C fill:\#f9f,stroke:\#333,stroke{-width:2px}
    style E fill:\#f9f,stroke:\#333,stroke{-width:2px}
{Highlighting}
{Shaded}
\end{verbatim}
\end{center}

\textbf{Proof}:

\begin{enumerate}
\tightlist
\item
  Current in the circuit: I = VS/(RS + RL)
\item
Power delivered to load:

P = I^{2}RL = (VS^{2}RL)/(RS + RL)^{2}

\item
  For maximum power, dP/dRL = 0
\item
  Solving: (VS^{2}(RS + RL)^{2} - VS^{2}RL·2(RS + RL))/(RS + RL)^{4} = 0
\item
  Simplifying: (RS + RL)^{2} = 2RL(RS + RL)
\item
  Further simplifying: RS + RL = 2RL
\item
  Therefore: RS = RL
\end{enumerate}

\textbf{Maximum Power}: Pmax = VS^{2}/(4RS)

\end{solutionbox}
\begin{mnemonicbox}
``MaRLRS: Maximum power when load Resistance equals
Source Resistance''

\end{mnemonicbox}
\subsection*{Question 4(a) [3 marks]}\label{q4a}

\textbf{Why series resonance circuit act as voltage amplifier and
parallel resonance circuit act as current amplifier?}

\begin{solutionbox}

\textbf{Series Resonance as Voltage Amplifier}:

\begin{itemize}
\tightlist
\item
  At resonance, series circuit impedance is minimum (just R)
\item
  Voltage across L or C can be much larger than source voltage
\item
  Voltage magnification factor = Q = XL/R = 1/R\sqrt(L/C)
\item
  Voltage across L or C = Q \times Source voltage
\end{itemize}

\textbf{Parallel Resonance as Current Amplifier}:

\begin{itemize}
\tightlist
\item
  At resonance, parallel circuit impedance is maximum
\item
  Current in L or C can be much larger than source current
\item
  Current magnification factor = Q = R/XL = R\sqrt(C/L)
\item
  Current through L or C = Q \times Source current
\end{itemize}

\textbf{Table}:

{\def\LTcaptype{none} % do not increment counter
\begin{longtable}[]{@{}lll@{}}
\toprule\noalign{}
Circuit Type & Impedance at Resonance & Amplification \\
\midrule\noalign{}
\endhead
\bottomrule\noalign{}
\endlastfoot
Series & Minimum (R only) & Voltage (VL or VC = Q\timesVS) \\
Parallel & Maximum (R^{2}/r) & Current (IL or IC = Q\timesIS) \\
\end{longtable}
}

\end{solutionbox}
\begin{mnemonicbox}
``SeVoPa: Series Voltage, Parallel current
amplification''

\end{mnemonicbox}
\subsection*{Question 4(b) [4 marks]}\label{q4b}

\textbf{Derive equation of Q of coil.}

\begin{solutionbox}

\textbf{Q-factor of a Coil}:

\begin{center}
\textbf{Mermaid Diagram (Code)}
\begin{verbatim}
{Shaded}
{Highlighting}[]
graph LR
    A((A)) {-{-}{-} B[R] {-}{-}{-} C((B))}
    C {-{-}{-} D[L] {-}{-}{-} A}
    style C fill:\#f9f,stroke:\#333,stroke{-width:2px}
{Highlighting}
{Shaded}
\end{verbatim}
\end{center}

\textbf{Derivation}:

\begin{enumerate}
\tightlist
\item
  Q-factor is defined as: Q = Energy stored / Energy dissipated per
  cycle
\item
  Energy stored in inductor = (1/2)LI^{2}
\item
  Power dissipated in resistor = I^{2}R
\item
  Energy dissipated per cycle = Power \times Time period = I^{2}R \times (1/f)
\item
  Therefore: Q = ((1/2)LI^{2}) / (I^{2}R \times (1/f))
\item
  Simplifying: Q = 2π \times (1/2)LI^{2} \times f / (I^{2}R)
\item
Q = 2πf \times L /

R = ωL / R

\end{enumerate}

\textbf{Final Equation}: Q = ωL / R = 2πfL / R = XL / R

\end{solutionbox}
\begin{mnemonicbox}
``QualityEDR: Quality equals Energy stored Divided by
energy lost per Radian''

\end{mnemonicbox}
\subsection*{Question 4(c) [7 marks]}\label{q4c}

\textbf{Derive equation of series resonance frequency for series R-L-C
circuit.}

\begin{solutionbox}

\textbf{Series R-L-C Circuit}:

\begin{center}
\textbf{Mermaid Diagram (Code)}
\begin{verbatim}
{Shaded}
{Highlighting}[]
graph LR
    A((Input)) {-{-}{-} B[R] {-}{-}{-} C[L] {-}{-}{-} D[C] {-}{-}{-} E((Output))}
    style A fill:\#f9f,stroke:\#333,stroke{-width:2px}
    style E fill:\#f9f,stroke:\#333,stroke{-width:2px}
{Highlighting}
{Shaded}
\end{verbatim}
\end{center}

\textbf{Derivation}:

\begin{enumerate}
\tightlist
\item
  Impedance of series RLC circuit: Z = R + j(XL - XC)
\item
  Where: XL = ωL and XC = 1/ωC
\item
  At resonance, XL = XC (inductive and capacitive reactances are equal)
\item
  Therefore: ωL = 1/ωC
\item
  Solving for ω: ω^{2} = 1/LC
\item
  Resonant frequency: ω_{0} = 1/\sqrt(LC)
\item
  In terms of frequency f: f_{0} = 1/(2π\sqrt(LC))
\end{enumerate}

\textbf{Characteristics at Resonance}:

\begin{itemize}
\tightlist
\item
  Impedance is minimum (purely resistive: Z = R)
\item
  Current is maximum (I = V/R)
\item
  Power factor is unity (circuit appears resistive)
\item
  Voltages across L and C are equal and opposite
\end{itemize}

\end{solutionbox}
\begin{mnemonicbox}
``RES: Reactances Equal at Series resonance''

\end{mnemonicbox}
\subsection*{Question 4(a) OR [3
marks]}\label{q4a}

\textbf{What is coupled circuits? Define self-inductance and mutual
inductance.}

\begin{solutionbox}

\textbf{Coupled Circuits}: Two or more circuits that are magnetically
linked such that energy can be transferred between them through their
mutual magnetic field.

\begin{center}
\textbf{Mermaid Diagram (Code)}
\begin{verbatim}
{Shaded}
{Highlighting}[]
graph TD
    subgraph "Primary"
    A((A)) {-{-}{-} B[L1] {-}{-}{-} C((B))}
    end

    subgraph "Secondary"
    D((C)) {-{-}{-} E[L2] {-}{-}{-} F((D))}
    end
    
    G[M] {-.{-}{} B}
    G {-.{-}{} E}
{Highlighting}
{Shaded}
\end{verbatim}
\end{center}

\textbf{Self-inductance (L)}: The property of a circuit whereby a change
in current produces a self-induced EMF in the same circuit. L = Φ/I
(ratio of magnetic flux to the current producing it)

\textbf{Mutual inductance (M)}: The property of a circuit whereby a
change in current in one circuit induces an EMF in another circuit. M =
Φ_{2}_{1}/I_{1} (ratio of flux in circuit 2 due to current in circuit 1)

\end{solutionbox}
\begin{mnemonicbox}
``SiMu: Self in Mine, Mutual in Yours''

\end{mnemonicbox}
\subsection*{Question 4(b) OR [4
marks]}\label{q4b}

\textbf{Derive equation for co-efficient of coupling (K).}

\begin{solutionbox}

\textbf{Coefficient of Coupling (k)}:

\begin{center}
\textbf{Mermaid Diagram (Code)}
\begin{verbatim}
{Shaded}
{Highlighting}[]
graph LR
    subgraph "Coupled Coils"
    A((A)) {-{-}{-} B[L1] {-}{-}{-} C((B))}
    D((C)) {-{-}{-} E[L2] {-}{-}{-} F((D))}
    G[M] {-.{-}{} B}
    G {-.{-}{} E}
    end
{Highlighting}
{Shaded}
\end{verbatim}
\end{center}

\textbf{Derivation}:

\begin{enumerate}
\tightlist
\item
  The mutual inductance (M) between two coils depends on:

  \begin{itemize}
  \tightlist
  \item
    Self-inductances of the coils (L_{1} and L_{2})
  \item
    Physical arrangement (proximity and orientation)
  \end{itemize}
\item
  Maximum possible mutual inductance: M_{m}_{a}_{x} = \sqrt(L_{1}L_{2})
\item
  Coefficient of coupling is defined as: k = M/M_{m}_{a}_{x}
\item
  Therefore: k = M/\sqrt(L_{1}L_{2})
\end{enumerate}

\textbf{Characteristics}:

\begin{itemize}
\tightlist
\item
  k ranges from 0 (no coupling) to 1 (perfect coupling)
\item
  k depends on geometry, orientation, and medium
\item
  Typical transformers: k = 0.95 to 0.99
\item
  Air-core coils: k = 0.01 to 0.5
\end{itemize}

\end{solutionbox}
\begin{mnemonicbox}
``KMutual: K Measures Mutual linkage proportion''

\end{mnemonicbox}
\subsection*{Question 4(c) OR [7
marks]}\label{q4c}

\textbf{A series RLC circuit has R=30Ω, L=0.5H, and C=5µF. Calculate (i)
series resonance frequency (2) Q Factor (3)BW}

\begin{solutionbox}

\textbf{Given}:

\begin{itemize}
\tightlist
\item
  Resistance, R = 30Ω
\item
  Inductance, L = 0.5H
\item
Capacitance,

C = 5µF = 5\times10^{-}^{6}F

\end{itemize}

\textbf{Calculations}:

\textbf{(i) Series Resonance Frequency}:

\begin{itemize}
\tightlist
\item
  f_{0} = 1/(2π\sqrt(LC))
\item
  f_{0} = 1/(2π\sqrt(0.5 \times 5\times10^{-}^{6}))
\item
  f_{0} = 1/(2π\sqrt(2.5\times10^{-}^{6}))
\item
  f_{0} = 1/(2π \times 1.58\times10^{-}^{3})
\item
  f_{0} = 1/(9.9\times10^{-}^{3})
\item
  f_{0} = 100.76 Hz
\item
  f_{0} \approx 100 Hz
\end{itemize}

\textbf{(ii) Q Factor}:

\begin{itemize}
\tightlist
\item
  Q = (1/R)\sqrt(L/C)
\item
  Q = (1/30)\sqrt(0.5/(5\times10^{-}^{6}))
\item
  Q = (1/30)\sqrt(100,000)
\item
  Q = (1/30) \times 316.23
\item
  Q = 10.54
\end{itemize}

\textbf{(iii) Bandwidth (BW)}:

\begin{itemize}
\tightlist
\item
  BW = f_{0}/Q
\item
  BW = 100.76/10.54
\item
  BW = 9.56 Hz
\end{itemize}

\textbf{Table}:

{\def\LTcaptype{none} % do not increment counter
\begin{longtable}[]{@{}lll@{}}
\toprule\noalign{}
Parameter & Formula & Value \\
\midrule\noalign{}
\endhead
\bottomrule\noalign{}
\endlastfoot
Resonant Frequency (f_{0}) & 1/(2π\sqrt(LC)) & 100 Hz \\
Quality Factor (Q) & (1/R)\sqrt(L/C) & 10.54 \\
Bandwidth (BW) & f_{0}/Q & 9.56 Hz \\
\end{longtable}
}

\end{solutionbox}
\begin{mnemonicbox}
``RQB: Resonance Quality determines Bandwidth''

\end{mnemonicbox}
\subsection*{Question 5(a) [3 marks]}\label{q5a}

\textbf{Classify various types of attenuators.}

\begin{solutionbox}

\textbf{Attenuators}: Network of resistors designed to reduce
(attenuate) signal level without distortion.

\textbf{Types of Attenuators}:

\begin{center}
\textbf{Mermaid Diagram (Code)}
\begin{verbatim}
{Shaded}
{Highlighting}[]
graph TD
    A[Attenuators] {-{-}{} B[Fixed Attenuators]}
    A {-{-}{} C[Variable Attenuators]}
    B {-{-}{} D[T{-}type]}
    B {-{-}{} E[π{-}type]}
    B {-{-}{} F[Bridged{-}T]}
    B {-{-}{} G[Lattice]}
    C {-{-}{} H[Step Attenuators]}
    C {-{-}{} I[Continuously Variable]}
{Highlighting}
{Shaded}
\end{verbatim}
\end{center}

Based on configuration:

\begin{itemize}
\tightlist
\item
  \textbf{T-type}: Three resistor T-shaped configuration
\item
  \textbf{π-type}: Three resistor π-shaped configuration
\item
  \textbf{Bridged-T}: T-type with a resistor bridging across
\item
  \textbf{Lattice}: Balanced configuration with four resistors
\end{itemize}

Based on symmetry:

\begin{itemize}
\tightlist
\item
  \textbf{Symmetrical}: Equal input and output impedance
\item
  \textbf{Asymmetrical}: Different input and output impedance
\end{itemize}

\end{solutionbox}
\begin{mnemonicbox}
``ATP Fixed: Attenuator Types include Pad, Tee,
Lattice''

\end{mnemonicbox}
\subsection*{Question 5(b) [4 marks]}\label{q5b}

\textbf{Derive relation between attenuator and neper.}

\begin{solutionbox}

\textbf{Relationship between Attenuation and Neper}:

\begin{itemize}
\item
  \textbf{Attenuation (α)}: Ratio of input voltage (or current) to
  output voltage (or current), expressed in different units.
\item
  \textbf{Neper (Np)}: Natural logarithmic unit of ratios, used mainly
  in transmission line theory.
\end{itemize}

\textbf{Derivation}:

\begin{enumerate}
\tightlist
\item
  For a voltage ratio V_{1}/V_{2}:

  \begin{itemize}
  \tightlist
  \item
    Attenuation in Nepers = ln(V_{1}/V_{2})
  \item
    Attenuation in Decibels = 20log_{1}_{0}(V_{1}/V_{2})
  \end{itemize}
\item
  For a power ratio P_{1}/P_{2}:

  \begin{itemize}
  \tightlist
  \item
    Attenuation in Nepers = (1/2)ln(P_{1}/P_{2})
  \item
    Attenuation in Decibels = 10log_{1}_{0}(P_{1}/P_{2})
  \end{itemize}
\item
  Relationship between dB and Neper:

  \begin{itemize}
  \tightlist
  \item
    1 Neper = 8.686 dB
  \item
    1 dB = 0.115 Neper
  \end{itemize}
\end{enumerate}

\textbf{Table}:

{\def\LTcaptype{none} % do not increment counter
\begin{longtable}[]{@{}lll@{}}
\toprule\noalign{}
Unit & Voltage Ratio & Power Ratio \\
\midrule\noalign{}
\endhead
\bottomrule\noalign{}
\endlastfoot
Neper (Np) & ln(V_{1}/V_{2}) & (1/2)ln(P_{1}/P_{2}) \\
Decibel (dB) & 20log_{1}_{0}(V_{1}/V_{2}) & 10log_{1}_{0}(P_{1}/P_{2}) \\
\end{longtable}
}

\end{solutionbox}
\begin{mnemonicbox}
``NED: Neper Equals Decibel divided by 8.686''

\end{mnemonicbox}
\subsection*{Question 5(c) [7 marks]}\label{q5c}

\textbf{Derive equations of R1 and R2 for symmetrical T attenuator.}

\begin{solutionbox}

\textbf{Symmetrical T Attenuator}:

\begin{center}
\textbf{Mermaid Diagram (Code)}
\begin{verbatim}
{Shaded}
{Highlighting}[]
graph LR
    A((Input)) {-{-}{-} B[R1] {-}{-}{-} C((Junction))}
    C {-{-}{-} D[R1] {-}{-}{-} E((Output))}
    C {-{-}{-} F[R2] {-}{-}{-} G((Ground))}
    style A fill:\#f9f,stroke:\#333,stroke{-width:2px}
    style E fill:\#f9f,stroke:\#333,stroke{-width:2px}
    style C fill:\#f9f,stroke:\#333,stroke{-width:2px}
{Highlighting}
{Shaded}
\end{verbatim}
\end{center}

\textbf{Derivation}:

\begin{enumerate}
\tightlist
\item
  For a symmetrical T-attenuator with characteristic impedance Z_{0}:

  \begin{itemize}
  \tightlist
  \item
    Input and output impedance must both equal Z_{0}
  \item
Attenuation ratio

N = V_{1}/V_{2} = I_{2}/I_{1}

  \end{itemize}
\item
  From circuit analysis:

  \begin{itemize}
  \tightlist
  \item
    Z_{0} = R_{1} + (R_{2}(R_{1}))/(R_{2}+R_{1})
  \item
    N = (R_{1} + R_{2} + R_{1})/R_{2} = (2R_{1}+R_{2})/R_{2}
  \end{itemize}
\item
  Solving for R_{1} and R_{2}:

  \begin{itemize}
  \tightlist
  \item
    R_{1} = Z_{0}(N-1)/(N+1)
  \item
    R_{2} = 2Z_{0}N/(N^{2}-1)
  \end{itemize}
\item
  For attenuation in dB (α):

  \begin{itemize}
  \tightlist
  \item
    N = 10\^{}(α/20)
  \item
    R_{1} = Z_{0}·tanh(α/2)
  \item
    R_{2} = Z_{0}/sinh(α)
  \end{itemize}
\end{enumerate}

\textbf{Final Equations}:

\begin{itemize}
\tightlist
\item
  R_{1} = Z_{0}(N-1)/(N+1)
\item
  R_{2} = 2Z_{0}N/(N^{2}-1)
\end{itemize}

\end{solutionbox}
\begin{mnemonicbox}
``TSR: T-attenuator Symmetry Requires equal R1
values''

\end{mnemonicbox}
\subsection*{Question 5(a) OR [3
marks]}\label{q5a}

\textbf{Draw circuit diagram of symmetrical Bridge T and symmetrical
Lattice attenuator.}

\begin{solutionbox}

\textbf{Symmetrical Bridge-T Attenuator}:

\begin{verbatim}
                R1
   A o{-{-}{-}{-}{-}{-}{-}{-}///{-}{-}{-}{-}{-}{-}{-}{-}o B}
              |      |
              |      |
              {      /}
              /  R3  {}
       R2     {      /}
   o{-{-}{-}///{-}{-}+     +{-}{-}{-}{-}o}
   |                      |
   o{-{-}{-}{-}{-}{-}{-}{-}{-}{-}{-}{-}{-}{-}{-}{-}{-}{-}{-}{-}{-}{-}o}
   C                      D
\end{verbatim}

\textbf{Symmetrical Lattice Attenuator}:

\begin{verbatim}
           R1
   A o{-{-}{-}{-}///{-}{-}{-}{-}o B}
      {            /}
       {          /}
        {        /}
     R2  {      / R2}
          {    /}
           {  /}
            {/}
            /{}
           /  {}
          /    {}
         /      {}
    R1  /        {}
     C o{-{-}///{-}{-}o D}
\end{verbatim}

\textbf{Characteristics}:

\begin{enumerate}
\tightlist
\item
  \textbf{Bridge-T}: Combines features of T and π attenuators, suitable
  for high-frequency applications
\item
  \textbf{Lattice}: Balanced configuration with excellent phase and
  frequency response, commonly used in balanced lines
\end{enumerate}

\end{solutionbox}
\begin{mnemonicbox}
``BL-BA: Bridge Ladder, Balanced Attenuators''

\end{mnemonicbox}
\subsection*{Question 5(b) OR [4
marks]}\label{q5b}

\textbf{Write classification of filter based on frequency with their
frequency responses showing pass band and stop band.}

\begin{solutionbox}

\textbf{Classification of Filters Based on Frequency}:

\begin{center}
\textbf{Mermaid Diagram (Code)}
\begin{verbatim}
{Shaded}
{Highlighting}[]
graph TD
    A[Passive Filters] {-{-}{} B[Low Pass Filter]}
    A {-{-}{} C[High Pass Filter]}
    A {-{-}{} D[Band Pass Filter]}
    A {-{-}{} E[Band Stop Filter]}
    A {-{-}{} F[All Pass Filter]}
{Highlighting}
{Shaded}
\end{verbatim}
\end{center}

\textbf{Frequency Responses}:

\begin{enumerate}
\item
  \textbf{Low Pass Filter}: Passes frequencies below cutoff, attenuates
  above

\begin{verbatim}
Gain |
   1 |****
     |    ****
     |        ****
   0 |------------****----
     |
     +----------------------
        0     fc        f \rightarrow
\end{verbatim}
\item
  \textbf{High Pass Filter}: Passes frequencies above cutoff, attenuates
  below

\begin{verbatim}
Gain |
   1 |            ****
     |        ****
     |    ****
   0 |****-----------------
     |
     +----------------------
        0     fc        f \rightarrow
\end{verbatim}
\item
  \textbf{Band Pass Filter}: Passes frequencies within a specific band

\begin{verbatim}
Gain |
   1 |        ****
     |    ****    ****
     |   *          *
   0 |***-------------***--
     |
     +----------------------
        0   f1   f2     f \rightarrow
\end{verbatim}
\item
  \textbf{Band Stop Filter}: Rejects frequencies within a specific band

\begin{verbatim}
Gain |
  1 |***             ***
     |   *           *
     |    ***     ***
  0 |        *****
     |
     +----------------------
        0   f1   f2     f \rightarrow
\end{verbatim}
\end{enumerate}

\end{solutionbox}
\begin{mnemonicbox}
``LHBBA: Low High Band-pass Band-stop All-pass''

\end{mnemonicbox}
\subsection*{Question 5(c) OR [7
marks]}\label{q5c}

\textbf{Draw the circuit for T-section and π-section constant-K low pass
filter and Derive equation of cut-off frequency.}

\begin{solutionbox}

\textbf{T-section Constant-K Low Pass Filter}:

\begin{verbatim}
           L/2            L/2
   o{-{-}{-}{-}{-}{-}UUUUUU{-}{-}{-}{-}{-}{-}UUUUUU{-}{-}{-}{-}{-}{-}{-}o}
   |                               |
   |                               |
   |                               |
   |              C                |
   |              |                |
   o{-{-}{-}{-}{-}{-}{-}{-}{-}{-}{-}{-}{-}{-}+{-}{-}{-}{-}{-}{-}{-}{-}{-}{-}{-}{-}{-}{-}{-}{-}o}
   Input                        Output
\end{verbatim}

\textbf{π-section Constant-K Low Pass Filter}:

\begin{verbatim}
          L
   o{-{-}{-}{-}{-}{-}UUUUUU{-}{-}{-}{-}{-}{-}{-}{-}{-}o}
   |                     |
   |                     |
   |                     |
   |                     |
  {-{-}{-}                   {-}{-}{-}}
  {-{-}{-} C/2               {-}{-}{-} C/2}
   |                     |
   |                     |
   o{-{-}{-}{-}{-}{-}{-}{-}{-}{-}{-}{-}{-}{-}{-}{-}{-}{-}{-}{-}{-}o}
   Input              Output
\end{verbatim}

\textbf{Derivation of Cutoff Frequency}:

\begin{enumerate}
\tightlist
\item
  For a constant-K filter:

  \begin{itemize}
  \tightlist
  \item
    Z_{1} \times Z_{2} = R_{0}^{2} (characteristic impedance squared)
  \item
    Z_{1} = jωL (series impedance)
  \item
    Z_{2} = 1/jωC (shunt impedance)
  \end{itemize}
\item
  Therefore:

  \begin{itemize}
  \tightlist
  \item
    R_{0}^{2} = Z_{1} \times Z_{2} = jωL \times 1/jωC = L/C
  \item
    R_{0} = \sqrt(L/C)
  \end{itemize}
\item
  Pass band condition:

  \begin{itemize}
  \tightlist
  \item
    -1 \textless{} Z_{1}/4Z_{2} \textless{} 0
  \item
    -1 \textless{} jωL/(4 \times 1/jωC) \textless{} 0
  \item
    -1 \textless{} -ω^{2}LC/4 \textless{} 0
  \end{itemize}
\item
  At cutoff frequency:

  \begin{itemize}
  \tightlist
  \item
    ω^{2}LC/4 = 1
  \item
    ωc^{2} = 4/LC
  \item
    ωc = 2/\sqrt(LC)
  \item
    fc = ωc/2π = 1/π\sqrt(LC)
  \end{itemize}
\end{enumerate}

\textbf{Final Equation}:

\begin{itemize}
\tightlist
\item
  Cutoff frequency fc = 1/π\sqrt(LC)
\end{itemize}

\end{solutionbox}
\begin{mnemonicbox}
``KCLP: Konstant-k Cutoff in Low Pass depends on L
and C product''

\end{mnemonicbox}

\end{document}
