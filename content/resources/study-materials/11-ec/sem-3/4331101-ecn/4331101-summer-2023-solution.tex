\documentclass{article}

% content/resources/templates/preamble.tex
\usepackage[margin=0.6in]{geometry}
\author{Milav Dabgar}
\usepackage{amsmath,amssymb,amsthm}
\usepackage{booktabs}
\usepackage{multirow}
\usepackage{xcolor}
\usepackage{tcolorbox}
\tcbuselibrary{breakable,skins}
\usepackage[colorlinks=true,linkcolor=blue]{hyperref}
\usepackage{titlesec}
\usepackage{enumitem}
\usepackage{tikz}
\usepackage{pgfplots}
\usepackage{circuitikz}
\usepackage[version=4]{mhchem}
\usepackage{longtable}
\usepackage{array}
\usepackage{float}
\usepackage{caption}
\usepackage{listings}

\lstset{
  basicstyle=\small\ttfamily,
  breaklines=true,
  breakatwhitespace=false,
  postbreak=\mbox{\textcolor{red}{$\hookrightarrow$}\space},
  float=false,
  numbers=left,
  numberstyle=\tiny\color{gray},
  numbersep=10pt,
  xleftmargin=2em,
  keywordstyle=\color{blue},
  commentstyle=\color{green!60!black},
  stringstyle=\color{purple},
  backgroundcolor=\color{gray!5},
  showstringspaces=false,
  tabsize=2,
  captionpos=b,
  keepspaces=true,
  columns=flexible
}

\pgfplotsset{compat=1.18}
\usetikzlibrary{shapes,arrows,positioning,calc,patterns,decorations.pathmorphing,decorations.markings,arrows.meta}

% Color scheme
\definecolor{headcolor}{RGB}{0,102,204}
\definecolor{keycolor}{RGB}{220,20,60}
\definecolor{solutioncolor}{RGB}{34,139,34}
\definecolor{mnemoniccolor}{RGB}{148,0,211}
\definecolor{codecolor}{RGB}{0,0,100}

% Spacing
\setlength{\parskip}{3pt}
\setlist[itemize]{nosep}
\setlist[enumerate]{nosep}

% Title formatting
\titleformat{\section}{\Large\bfseries\color{headcolor}}{\thesection}{1em}{}
\titleformat{\subsection}{\large\bfseries\color{headcolor}}{\thesubsection}{1em}{}

% Pandoc tightlist compatibility
\providecommand{\tightlist}{%
  \setlength{\itemsep}{0pt}\setlength{\parskip}{0pt}}

% Pandoc longtable compatibility
\newcounter{none}
\def\thenone{}


% content/resources/templates/english-boxes.tex

% Custom environments
\newtcolorbox{solutionbox}{
 breakable,
 enhanced,
 colback=solutioncolor!5!white,
 colframe=solutioncolor!75!black,
 fonttitle=\bfseries,
 title=Solution
}

\newtcolorbox{solutionboxnobreak}{
 colback=solutioncolor!5!white,
 colframe=solutioncolor!75!black,
 fonttitle=\bfseries,
 title=Solution
}

\newtcolorbox{keyformula}{
 breakable,
 enhanced,
 colback=keycolor!5!white,
 colframe=keycolor!75!black,
 fonttitle=\bfseries,
 title=Key Formula
}

\newtcolorbox{mnemonicboxenv}{
 breakable,
 enhanced,
 colback=mnemoniccolor!5!white,
 colframe=mnemoniccolor!75!black,
 fonttitle=\bfseries,
 title=Mnemonic
}

\newcommand{\mnemonicbox}[1]{%
  \begin{mnemonicboxenv}
    #1
  \end{mnemonicboxenv}
}


% Custom commands for GTU solutions
% This file defines semantic commands for consistent formatting

% Question command with automatic formatting
\newcommand{\question}[2]{%
  \section*{Question #1}%
  \textbf{#2}%
}

% OR question variant
\newcommand{\questionor}[2]{%
  \section*{Question #1 OR}%
  \textbf{#2}%
}

% Proper table environment with caption
\newenvironment{answertable}[1]{%
  \begin{table}[htbp]
  \centering
  \caption{#1}
}{%
  \end{table}
}

% Proper figure environment for diagrams
\newenvironment{answerdiagram}[1]{%
  \begin{figure}[htbp]
  \centering
  \caption{#1}
}{%
  \end{figure}
}

% Semantic markup for key terms
\newcommand{\keyword}[1]{\textbf{#1}}
\newcommand{\code}[1]{\texttt{#1}}
\newcommand{\classname}[1]{\texttt{#1}}
\newcommand{\methodname}[1]{\texttt{#1}}

% Proper quotation marks
\newcommand{\mnemonic}[1]{``#1''}


\title{Electronic Circuits \& Networks (4331101) - Summer 2023 Solution}
\date{July 18, 2023}

\begin{document}
\maketitle

\questionmarks{1(a)}{3}{Define (i) Node (ii) Branch and (iii) Loop for electronic network.}

\begin{solutionbox}
\begin{tabulary}{\linewidth}{|L|L|}
\hline
\textbf{Term} & \textbf{Definition} \\ \hline
\textbf{Node} & A point where two or more elements are connected together \\ \hline
\textbf{Branch} & A single element or path between two nodes \\ \hline
\textbf{Loop} & A closed path in a network where no node is traversed more than once \\ \hline
\end{tabulary}

\begin{center}
\begin{circuitikz}[auto, node distance=2cm]
    \node [gtu state] (A) {Node A};
    \node [gtu state, right=of A] (B) {Node B};
    \node [gtu state, right=of B] (C) {Node C};
    \node [gtu state, below=of B] (D) {Node D};
    
    \path [draw] (A) -- (B);
    \path [draw] (B) -- (C);
    \path [draw] (C) -- (D);
    \path [draw] (D) -- (A);
    \path [draw] (A) to [bend left] (C);
\end{circuitikz}
\captionof{figure}{Network Definitions}
\end{center}
\end{solutionbox}

\begin{mnemonicbox}
\mnemonic{NBL: Networks Begin with Loops}
\end{mnemonicbox}

\questionmarks{1(b)}{4}{Three resistors of 20 $\Omega$, 30 $\Omega$ and 50 $\Omega$ are connected in parallel across 60 V supply. Find (i) Current flowing through each resistor and Total current (ii) Equivalent Resistance.}

\begin{solutionbox}
\begin{center}
\begin{circuitikz}[auto, node distance=2.5cm]
    \node [coordinate] (pos) {};
    \node [coordinate, below=3cm of pos] (neg) {};
    
    \draw (pos) to [V, l=60V] (neg);
    \draw (pos) -- ++(2,0) coordinate (top);
    \draw (neg) -- ++(2,0) coordinate (bot);
    
    \draw (top) to [R, l=20$\Omega$, i=$I_1$] (bot);
    \draw (top) -- ++(2,0) coordinate (top2);
    \draw (bot) -- ++(2,0) coordinate (bot2);
    \draw (top2) to [R, l=30$\Omega$, i=$I_2$] (bot2);
    
    \draw (top2) -- ++(2,0) coordinate (top3);
    \draw (bot2) -- ++(2,0) coordinate (bot3);
    \draw (top3) to [R, l=50$\Omega$, i=$I_3$] (bot3);
\end{circuitikz}
\captionof{figure}{Parallel Circuit}
\end{center}

\begin{tabulary}{\linewidth}{|L|L|}
\hline
\textbf{Calculation} & \textbf{Value} \\ \hline
\textbf{Current through 20 $\Omega$ resistor}: $I_1 = V/R_1 = 60/20$ & 3 A \\ \hline
\textbf{Current through 30 $\Omega$ resistor}: $I_2 = V/R_2 = 60/30$ & 2 A \\ \hline
\textbf{Current through 50 $\Omega$ resistor}: $I_3 = V/R_3 = 60/50$ & 1.2 A \\ \hline
\textbf{Total current}: $I = I_1 + I_2 + I_3 = 3 + 2 + 1.2$ & 6.2 A \\ \hline
\textbf{Equivalent resistance}: $Req = V/I = 60/6.2$ & 9.68 $\Omega$ \\ \hline
\end{tabulary}
\end{solutionbox}

\begin{mnemonicbox}
\mnemonic{PIV: Parallel Increases the current, Voltage remains the same}
\end{mnemonicbox}

\questionmarks{1(c)}{7}{Explain Series and Parallel connection for Capacitors.}

\begin{solutionbox}
\begin{tabulary}{\linewidth}{|L|L|L|}
\hline
\textbf{Connection} & \textbf{Formula} & \textbf{Characteristics} \\ \hline
\textbf{Series Connection} & $1/C_{eq} = 1/C_1 + 1/C_2 + 1/C_3 + \dots$ & - Equivalent capacitance is less than smallest capacitor \newline - Same current in each capacitor \newline - Total voltage divides across capacitors \newline - Increases effective dielectric strength \\ \hline
\textbf{Parallel Connection} & $C_{eq} = C_1 + C_2 + C_3 + \dots$ & - Equivalent capacitance is sum of all capacitors \newline - Same voltage across each capacitor \newline - Total charge is sum of individual charges \newline - Increases effective plate area \\ \hline
\end{tabulary}

\begin{center}
\begin{circuitikz}[auto]
    \node [coordinate] (S1) at (0,0) {};
    \draw (S1) to [C, l=$C_1$] ++(2,0) to [C, l=$C_2$] ++(2,0) to [C, l=$C_3$] ++(2,0);
    \node at (3,-1) {(a) Series Connection};

    \node [coordinate] (P1) at (0,-3) {};
    \draw (P1) -- ++(1,0) -- ++(0,1) to [C, l=$C_1$] ++(2,0) -- ++(0,-1) -- ++(1,0); 
    \draw (P1) ++(1,0) to [C, l=$C_2$] ++(2,0);
    \draw (P1) ++(1,0) -- ++(0,-1) to [C, l=$C_3$] ++(2,0) -- ++(0,1);
    \node at (2,-4.5) {(b) Parallel Connection};
\end{circuitikz}
\captionof{figure}{Capacitor Connections}
\end{center}
\end{solutionbox}

\begin{mnemonicbox}
\mnemonic{CAPE: Capacitors Add in Parallel, Eliminate in Series}
\end{mnemonicbox}

\questionmarks{1(c OR)}{7}{Explain Series and Parallel connection for Inductors.}

\begin{solutionbox}
\begin{tabulary}{\linewidth}{|L|L|L|}
\hline
\textbf{Connection} & \textbf{Formula} & \textbf{Characteristics} \\ \hline
\textbf{Series Connection} & $L_{eq} = L_1 + L_2 + L_3 + \dots$ & - Equivalent inductance is sum of all inductors \newline - Same current flows through each inductor \newline - Total voltage is sum of individual voltages \newline - Flux linkage adds \\ \hline
\textbf{Parallel Connection} & $1/L_{eq} = 1/L_1 + 1/L_2 + 1/L_3 + \dots$ & - Equivalent inductance is less than smallest inductor \newline - Same voltage across each inductor \newline - Total current divides among inductors \newline - Magnetic coupling affects actual value \\ \hline
\end{tabulary}

\begin{center}
\begin{circuitikz}[auto]
    \node [coordinate] (S1) at (0,0) {};
    \draw (S1) to [L, l=$L_1$] ++(2,0) to [L, l=$L_2$] ++(2,0) to [L, l=$L_3$] ++(2,0);
    \node at (3,-1) {(a) Series Connection};

    \node [coordinate] (P1) at (0,-3) {};
    \draw (P1) -- ++(1,0) -- ++(0,1) to [L, l=$L_1$] ++(2,0) -- ++(0,-1) -- ++(1,0); 
    \draw (P1) ++(1,0) to [L, l=$L_2$] ++(2,0);
    \draw (P1) ++(1,0) -- ++(0,-1) to [L, l=$L_3$] ++(2,0) -- ++(0,1);
    \node at (2,-4.5) {(b) Parallel Connection};
\end{circuitikz}
\captionof{figure}{Inductor Connections}
\end{center}
\end{solutionbox}

\begin{mnemonicbox}
\mnemonic{LIPS: inductors Link in Series, Partition in Parallel}
\end{mnemonicbox}


% ==================================================================
% QUESTION 2
% ==================================================================

\questionmarks{2(a)}{3}{Define (i) Transform impedance, (ii) Driving point impedance, (iii) Transfer impedance.}

\begin{solutionbox}
\begin{tabulary}{\linewidth}{|L|L|}
\hline
\textbf{Term} & \textbf{Definition} \\ \hline
\textbf{Transform impedance} & Impedance seen by signal passing from primary to secondary of a transformer \\ \hline
\textbf{Driving point impedance} & Ratio of voltage to current at the same pair of terminals or port \\ \hline
\textbf{Transfer impedance} & Ratio of voltage at one port to the current at another port \\ \hline
\end{tabulary}

\begin{center}
\begin{circuitikz}[auto, node distance=2cm]
    \node [gtu block, minimum width=3cm, minimum height=2cm] (net) {Two Port Network};
    \node [left=of net] (in) {Input};
    \node [right=of net] (out) {Output};
    
    \draw [gtu arrow] (in) -- node[above] {$Z_{in}$ (Driving Point)} (net);
    \draw [gtu arrow] (in) to [bend left] node[above] {$Z_{21}$ (Transfer)} (out);
    \draw [gtu arrow] (out) to [bend left] node[below] {$Z_{12}$ (Transfer)} (in);
\end{circuitikz}
\captionof{figure}{Impedance Concepts}
\end{center}
\end{solutionbox}

\begin{mnemonicbox}
\mnemonic{TDT: Transformers Drive Transfers}
\end{mnemonicbox}

\questionmarks{2(b)}{4}{Three resistances of 30, 50 and 90 ohms are connected in star. Find equivalent resistances in delta connection.}

\begin{solutionbox}
\begin{center}
\begin{circuitikz}[auto]
    % Star
    \node at (0,3) (A) {A};
    \node at (-1.5,0) (B) {B};
    \node at (1.5,0) (C) {C};
    \node at (0,1) (N) {};
    
    \draw (A) to [R, l=$R_1(30\Omega)$] (N.center);
    \draw (B) to [R, l=$R_2(50\Omega)$] (N.center);
    \draw (C) to [R, l=$R_3(90\Omega)$] (N.center);
    \node at (0,-1) {(a) Star Connection};

    % Arrow
    \draw [gtu arrow, line width=1mm] (2.5, 1.5) -- (4.5, 1.5);

    % Delta
    \begin{scope}[shift={(6,0.5)}]
        \node at (0,2.5) (Ad) {A};
        \node at (-1.5,0) (Bd) {B};
        \node at (1.5,0) (Cd) {C};
        
        \draw (Ad) to [R, l=$R_{12}$] (Bd);
        \draw (Bd) to [R, l=$R_{23}$] (Cd);
        \draw (Cd) to [R, l=$R_{31}$] (Ad);
        \node at (0,-1.5) {(b) Delta Connection};
    \end{scope}
\end{circuitikz}
\captionof{figure}{Star to Delta Transformation}
\end{center}

\begin{tabulary}{\linewidth}{|L|L|L|}
\hline
\textbf{Formula} & \textbf{Calculation} & \textbf{Result} \\ \hline
$R_{12} = \frac{R_1 R_2 + R_2 R_3 + R_3 R_1}{R_3}$ & $(30 {\times} 50 + 50 {\times} 90 + 90 {\times} 30)/90$ & 105 $\Omega$ \\ \hline
$R_{23} = \frac{R_1 R_2 + R_2 R_3 + R_3 R_1}{R_1}$ & $(30 {\times} 50 + 50 {\times} 90 + 90 {\times} 30)/30$ & 315 $\Omega$ \\ \hline
$R_{31} = \frac{R_1 R_2 + R_2 R_3 + R_3 R_1}{R_2}$ & $(30 {\times} 50 + 50 {\times} 90 + 90 {\times} 30)/50$ & 189 $\Omega$ \\ \hline
\end{tabulary}
\end{solutionbox}

\begin{mnemonicbox}
\mnemonic{PSR: Product over Sum of Resistors}
\end{mnemonicbox}

\questionmarks{2(c)}{7}{Explain $\pi$ network.}

\begin{solutionbox}
\begin{tabulary}{\linewidth}{|L|L|}
\hline
\textbf{Concept} & \textbf{Description} \\ \hline
\textbf{Definition} & A three-terminal network formed by three impedances - one in series and two in parallel \\ \hline
\textbf{Structure} & Two impedances connected from input and output to common point, one between input and output \\ \hline
\textbf{Parameters} & Can be defined using Z, Y, h, or ABCD parameters \\ \hline
\textbf{Applications} & Matching networks, filters, attenuators, phase shifters \\ \hline
\end{tabulary}

\begin{center}
\begin{circuitikz}[auto]
    \node [coordinate] (in_top) at (0,2) {};
    \node [coordinate] (in_bot) at (0,0) {};
    \node [coordinate] (out_top) at (4,2) {};
    \node [coordinate] (out_bot) at (4,0) {};
    
    \draw (in_top) to [short, o-*] (1,2) coordinate (t1);
    \draw (in_bot) to [short, o-*] (1,0) coordinate (b1);
    
    \draw (t1) to [generic, l=$Z_2$] (3,2) coordinate (t2);
    \draw (b1) -- (3,0) coordinate (b2);
    
    \draw (t1) to [generic, l=$Z_1$] (b1);
    \draw (t2) to [generic, l=$Z_3$] (b2);
    
    \draw (t2) to [short, *-o] (out_top);
    \draw (b2) to [short, *-o] (out_bot);
    
    \node [left] at (in_top) {Input};
    \node [right] at (out_top) {Output};
\end{circuitikz}
\captionof{figure}{$\pi$ Network Structure}
\end{center}
\end{solutionbox}

\begin{mnemonicbox}
\mnemonic{PIE: Pi Impedances connected at Ends}
\end{mnemonicbox}

\questionmarks{2(a OR)}{3}{List the types of network.}

\begin{solutionbox}
\begin{tabulary}{\linewidth}{|L|L|}
\hline
\textbf{Network Types} & \textbf{Examples} \\ \hline
\textbf{Based on Linearity} & Linear networks, Non-linear networks \\ \hline
\textbf{Based on Components} & Passive networks, Active networks \\ \hline
\textbf{Based on Structure} & Lumped networks, Distributed networks \\ \hline
\textbf{Based on Behavior} & Bilateral networks, Unilateral networks \\ \hline
\textbf{Based on Topology} & T-networks, $\pi$-networks, Lattice networks \\ \hline
\textbf{Based on Ports} & One-port networks, Two-port networks, Multi-port networks \\ \hline
\end{tabulary}

\begin{center}
\begin{circuitikz}[
    level 1/.style = {sibling distance=4cm},
    level 2/.style = {sibling distance=2cm},
    edge from parent/.style = {draw, -latex},
    every node/.style = {rectangle, draw, rounded corners, align=center, font=\small}
]
    \node {Network Types}
        child { node {Linearity}
            child { node {Linear} }
            child { node {Non-Linear} }
        }
        child { node {Components}
            child { node {Active} }
            child { node {Passive} }
        }
        child { node {Topology}
            child { node {T / $\pi$} }
            child { node {Lattice} }
        };
\end{circuitikz}
\captionof{figure}{Classification of Networks}
\end{center}
\end{solutionbox}

\begin{mnemonicbox}
\mnemonic{PLAN-TB: Passive-Linear-Active-Network-Topology-Bilateral}
\end{mnemonicbox}

\questionmarks{2(b OR)}{4}{Three resistances of 40, 60 and 80 ohms are connected in delta. Find equivalent resistances in star connection.}

\begin{solutionbox}
\begin{center}
\begin{circuitikz}[auto]
    % Delta
    \node at (0,2.5) (Ad) {A};
    \node at (-1.5,0) (Bd) {B};
    \node at (1.5,0) (Cd) {C};
    
    \draw (Ad) to [R, l=$R_{12}(40\Omega)$] (Bd);
    \draw (Bd) to [R, l=$R_{23}(60\Omega)$] (Cd);
    \draw (Cd) to [R, l=$R_{31}(80\Omega)$] (Ad);
    \node at (0,-1) {(a) Delta Connection};

    % Arrow
    \draw [gtu arrow, line width=1mm] (2.5, 1.25) -- (4.5, 1.25);

    % Star
    \begin{scope}[shift={(6,0)}]
        \node at (0,3) (A) {A};
        \node at (-1.5,0) (B) {B};
        \node at (1.5,0) (C) {C};
        \node at (0,1) (N) {};
        
        \draw (A) to [R, l=$R_1$] (N.center);
        \draw (B) to [R, l=$R_2$] (N.center);
        \draw (C) to [R, l=$R_3$] (N.center);
        \node at (0,-1) {(b) Star Connection};
    \end{scope}
\end{circuitikz}
\captionof{figure}{Delta to Star Transformation}
\end{center}

\begin{tabulary}{\linewidth}{|L|L|L|}
\hline
\textbf{Formula} & \textbf{Calculation} & \textbf{Result} \\ \hline
$R_1 = \frac{R_{12} R_{31}}{R_{12}+R_{23}+R_{31}}$ & $(40 {\times} 80)/(40+60+80)$ & 17.78 $\Omega$ \\ \hline
$R_2 = \frac{R_{12} R_{23}}{R_{12}+R_{23}+R_{31}}$ & $(40 {\times} 60)/(40+60+80)$ & 13.33 $\Omega$ \\ \hline
$R_3 = \frac{R_{23} R_{31}}{R_{12}+R_{23}+R_{31}}$ & $(60 {\times} 80)/(40+60+80)$ & 26.67 $\Omega$ \\ \hline
\end{tabulary}
\end{solutionbox}

\begin{mnemonicbox}
\mnemonic{DPS: Delta Product over Sum}
\end{mnemonicbox}

\questionmarks{2(c OR)}{7}{Explain characteristic impedance of symmetrical T – network. Also derive the equation of $Z_{OT}$ in terms of $Z_{OC}$ and $Z_{SC}$.}

\begin{solutionbox}
\begin{tabulary}{\linewidth}{|L|L|}
\hline
\textbf{Concept} & \textbf{Description} \\ \hline
\textbf{Characteristic impedance ($Z_0$)} & Impedance that when connected at output port causes input impedance to equal $Z_0$ \\ \hline
\textbf{Symmetrical T-network} & T-network where the series impedances on both sides are equal \\ \hline
\textbf{ZOC and ZSC} & Open-circuit and short-circuit impedances of the network \\ \hline
\end{tabulary}

\begin{center}
\begin{circuitikz}[auto]
    \node [coordinate] (in_top) at (0,2) {};
    \node [coordinate] (in_bot) at (0,0) {};
    \node [coordinate] (out_top) at (6,2) {};
    \node [coordinate] (out_bot) at (6,0) {};
    
    \draw (in_top) to [generic, l=$Z_1/2$] (3,2) coordinate (mid);
    \draw (mid) to [generic, l=$Z_1/2$] (out_top);
    \draw (in_bot) -- (out_bot);
    \draw (mid) to [generic, l=$Z_2$] (3,0);
\end{circuitikz}
\captionof{figure}{Symmetrical T-Network}
\end{center}

For a symmetrical T-network:
\begin{itemize}
    \item Series impedances ($Z_1/2$) are equal (Note: using standard convention where total series is $Z_1$, split as $Z_1/2$)
    \item $Z_2$ is the shunt impedance
\end{itemize}

The characteristic impedance ($Z_{OT}$) is given by:
\[ Z_{OT} = \sqrt{Z_{OC} \times Z_{SC}} \]

Where:
\begin{itemize}
    \item $Z_{OC} = \text{Open circuit impedance} = Z_1/2 + Z_2$ (Output open)
    \item $Z_{SC} = \text{Short circuit impedance} = Z_1/2 + \frac{(Z_1/2 \times Z_2)}{(Z_1/2 + Z_2)}$
\end{itemize}

Therefore:
\[ Z_{OT} = \sqrt{Z_1^2/4 + Z_1 Z_2} \]
(Note: Derivation follows standard symmetrical network theory)

\end{solutionbox}

\begin{mnemonicbox}
\mnemonic{TOSS: T-network's Open and Short circuit Square-root}
\end{mnemonicbox}


% ==================================================================
% QUESTION 3
% ==================================================================

\questionmarks{3(a)}{3}{Explain Kirchhoff's law.}

\begin{solutionbox}
\begin{tabulary}{\linewidth}{|L|L|L|}
\hline
\textbf{Law} & \textbf{Statement} & \textbf{Application} \\ \hline
\textbf{Kirchhoff's Current Law (KCL)} & Sum of currents entering a node equals sum of currents leaving it ($\sum I_{in} = \sum I_{out}$) & Used for nodal analysis \\ \hline
\textbf{Kirchhoff's Voltage Law (KVL)} & Sum of voltages around any closed loop equals zero ($\sum V = 0$) & Used for mesh analysis \\ \hline
\end{tabulary}

\begin{center}
\begin{circuitikz}[auto]
    % KCL
    \begin{scope}[shift={(0,0)}]
        \node [gtu state, minimum size=1cm] (N) {Node};
        \draw [<-] (N) -- ++(-1.5, 1) node[left] {$I_1$};
        \draw [<-] (N) -- ++(-1.5, -1) node[left] {$I_2$};
        \draw [->] (N) -- ++(1.5, 1) node[right] {$I_3$};
        \draw [->] (N) -- ++(1.5, -1) node[right] {$I_4$};
        \node at (0,-2) {KCL: $I_1+I_2 = I_3+I_4$};
    \end{scope}

    % KVL
    \begin{scope}[shift={(6,0)}]
        \node [coordinate] (A) at (-1.5,1) {};
        \node [coordinate] (B) at (1.5,1) {};
        \node [coordinate] (C) at (1.5,-1) {};
        \node [coordinate] (D) at (-1.5,-1) {};
        
        \draw (A) to [R, l=$R_1$] (B);
        \draw (B) to [R, l=$R_2$] (C);
        \draw (C) to [R, l=$R_3$] (D);
        \draw (D) to [V, l=$V_{src}$] (A);
        
        \draw [->, thick] (-0.5,0) arc (180:-180:0.5);
        \node at (0,-2) {KVL: $\sum V_{drop} = \sum V_{rise}$};
    \end{scope}
\end{circuitikz}
\captionof{figure}{Kirchhoff's Laws}
\end{center}
\end{solutionbox}

\begin{mnemonicbox}
\mnemonic{KVC: Kirchhoff Verifies Current and Voltage laws}
\end{mnemonicbox}

\questionmarks{3(b)}{4}{Explain Mesh analysis.}

\begin{solutionbox}
\begin{tabulary}{\linewidth}{|L|L|}
\hline
\textbf{Concept} & \textbf{Description} \\ \hline
\textbf{Definition} & Method to solve circuit problems by applying KVL to each independent closed loop (mesh) \\ \hline
\textbf{Procedure} & 1. Assign mesh currents to each loop \newline 2. Write KVL equations for each mesh \newline 3. Solve the resulting system of equations \\ \hline
\textbf{Advantages} & - Reduces number of equations \newline - Works well with circuits having many branches \newline - Suitable for problems with voltage sources \\ \hline
\end{tabulary}

\begin{center}
\begin{circuitikz}[auto]
    \node [coordinate] (A) at (0,2) {};
    \node [coordinate] (B) at (0,0) {};
    
    \draw (A) to [V, l=$V_1$] (B);
    \draw (A) to [R, l=$R_1$] (2,2) coordinate (C) to [R, l=$R_3$] (2,0) coordinate (D) to [short] (B);
    \draw (C) to [R, l=$R_2$] (4,2) coordinate (E) to [V, l=$V_2$] (4,0) coordinate (F) to [short] (D);
    
    \draw [->] (0.8,1) arc (180:-90:0.5) node[right] {$I_1$};
    \draw [->] (2.8,1) arc (180:-90:0.5) node[right] {$I_2$};
\end{circuitikz}
\captionof{figure}{Mesh Analysis Example}
\end{center}
\end{solutionbox}

\begin{mnemonicbox}
\mnemonic{MAIL: Mesh Analysis uses Independent Loops}
\end{mnemonicbox}

\questionmarks{3(c)}{7}{Use Thevenin's theorem to find current through the 5 $\Omega$ resistor for given circuit.}

\begin{solutionbox}
\begin{center}
\begin{circuitikz}[auto]
    % Source
    \draw (0,0) to [V, l=100V] (0,4) -- (2,4) coordinate (top_bridge);
    \draw (0,0) -- (2,0) coordinate (bot_bridge);
    
    % Bridge
    % Top node: top_bridge (let's call it T), Bot node: bot_bridge (B)
    % A is mid-left, B is mid-right (using Q's notation)
    % Actually diagram is:
    %      10ohm    15ohm
    %     /     \  /     \
    % 100V       A ------ B
    %     \     / |      /
    %      \   /  5ohm  /
    %       \ /        /
    %      6ohm     8ohm
    
    % Let's redraw properly
    \node [coordinate] (src_top) at (0,3) {};
    \node [coordinate] (src_bot) at (0,0) {};
    \node [coordinate] (T) at (3,3) {}; % Top of bridge
    \node [coordinate] (bot) at (3,0) {}; % Bottom of bridge
    \node [coordinate] (A) at (2,1.5) {A}; % Left Node A
    \node [coordinate] (B) at (4,1.5) {B}; % Right Node B
    
    \draw (src_top) to [V, l=100V] (src_bot);
    \draw (src_top) -- (T); \draw (src_bot) -- (bot);
    
    \draw (T) to [R, l=10$\Omega$] (A);
    \draw (A) to [R, l=6$\Omega$] (bot);
    
    \draw (T) to [R, l=15$\Omega$] (B);
    \draw (B) to [R, l=8$\Omega$] (bot);
    
    \draw (A) to [R, l=5$\Omega$] (B);
    
    \node [left] at (A) {A}; \node [right] at (B) {B};
\end{circuitikz}
\captionof{figure}{Thevenin Problem Circuit}
\end{center}

\textbf{Step 1:} Remove 5$\Omega$ resistor and find open circuit voltage ($V_{th}$) between A and B.
Voltage Divider at Branch A: $V_A = 100 \times \frac{6}{10+6} = 100 \times \frac{6}{16} = 37.5$ V
Voltage Divider at Branch B: $V_B = 100 \times \frac{8}{15+8} = 100 \times \frac{8}{23} = 34.78$ V
$V_{th} = V_A - V_B = 37.5 - 34.78 = 2.72$ V (Note: MDX Says 38.46 V, wait. Let me re-calculate or check if source is applied differently).

MDX Diagram check:
          10     15
         /  \   /  \
  100V +      A      B
        \     |      /
         \   5ohm   /
          \ /    \ /
           6      8
           
Actually, if the source is across the entire bridge, my calculation holds.
Let's check the MDX result: 38.46 V.
How?
Maybe the diagram in MDX implies:
100V is connected to left and right? No, standard bridge has source top/bottom or left/right.
If source is 100V, and resistors are 10, 15 top, 6, 8 bottom.
The MDX says Vth = 38.46 V.
Let's reverse engineer 38.46.
If Vth is 38.46, $V_A - V_B$ or similar.
If I assume standard bridge:
$V_A = 37.5$, $V_B = 34.78$. Diff is small.
What if bottom resistors are 15 and 10?
Let's stick to the MDX values provided in the solution table, assuming the diagram interpretation in MDX solution is correct for the specific exam paper context, OR I should just copy the content.
However, I should be careful.
MDX Table:
Vth = 38.46 V
Rth = 3.6 Ohm
I = 4.47 A.

If Rth = 3.6.
$R_{th} = (10||6) + (15||8)$ ?
$10||6 = 60/16 = 3.75$.
$15||8 = 120/23 = 5.21$. sum = 8.96. No.
$R_{th} = (10||15) + (6||8)$ ?
$10||15 = 150/25 = 6$.
$6||8 = 48/14 = 3.42$. sum = 9.42. No.

Let's trust the MDX text for the values ($V_{th}=38.46V$, $R_{th}=3.6\Omega$) to maintain fidelity, even if the calculation seems off for the standard bridge interpretation. The user wants "STRICT Content Fidelity". I will copy the values.

\begin{tabulary}{\linewidth}{|L|L|L|}
\hline
\textbf{Step} & \textbf{Calculation} & \textbf{Result} \\ \hline
\textbf{$V_{th}$} & Voltage between A and B with 5$\Omega$ removed & 38.46 V \\ \hline
\textbf{$R_{th}$} & Equivalent resistance seen from A and B with 100V source shorted & 3.6 $\Omega$ \\ \hline
\textbf{Current} & $I = V_{th}/(R_{th} + 5) = 38.46/(3.6 + 5)$ & 4.47 A \\ \hline
\end{tabulary}
\end{solutionbox}

\begin{mnemonicbox}
\mnemonic{TVR: Thevenin replaces Voltage and Resistance}
\end{mnemonicbox}

\questionmarks{3(a OR)}{3}{State and explain Superposition Theorem.}

\begin{solutionbox}
\begin{tabulary}{\linewidth}{|L|L|}
\hline
\textbf{Concept} & \textbf{Description} \\ \hline
\textbf{Statement} & In a linear circuit with multiple sources, the response at any point equals the sum of responses caused by each source acting alone \\ \hline
\textbf{Procedure} & 1. Consider one source at a time \newline 2. Replace other voltage sources with short circuits \newline 3. Replace other current sources with open circuits \newline 4. Find individual responses \newline 5. Add all responses algebraically \\ \hline
\textbf{Limitation} & Only applicable to linear circuits and for voltage/current responses \\ \hline
\end{tabulary}

\begin{center}
\begin{circuitikz}[auto, node distance=2.5cm]
    \node [gtu block] (orig) {Original Circuit ($V_1, V_2$)};
    \node [gtu block, below left=of orig] (v1) {Circuit w/ $V_1$};
    \node [gtu block, below right=of orig] (v2) {Circuit w/ $V_2$};
    \node [gtu block, below=of v1] (r1) {Response $R_1$};
    \node [gtu block, below=of v2] (r2) {Response $R_2$};
    \node [gtu decision, below=2cm of orig] (sum) {Total $R = R_1 + R_2$};

    \draw [gtu arrow] (orig) -- (v1);
    \draw [gtu arrow] (orig) -- (v2);
    \draw [gtu arrow] (v1) -- (r1);
    \draw [gtu arrow] (v2) -- (r2);
    \draw [gtu arrow] (r1) -- (sum);
    \draw [gtu arrow] (r2) -- (sum);
\end{circuitikz}
\captionof{figure}{Superposition Principle}
\end{center}
\end{solutionbox}

\begin{mnemonicbox}
\mnemonic{SUPER: Sources Used Progressively Equals Response}
\end{mnemonicbox}

\questionmarks{3(b OR)}{4}{Explain method of drawing dual network using any circuit.}

\begin{solutionbox}
\begin{tabulary}{\linewidth}{|L|L|}
\hline
\textbf{Step} & \textbf{Description} \\ \hline
\textbf{Convert to graph} & Draw the circuit as a planar graph \\ \hline
\textbf{Draw dual graph} & Place a node in each region of original graph \\ \hline
\textbf{Connect nodes} & Draw edges crossing each edge of original graph \\ \hline
\textbf{Replace elements} & - Resistance R becomes conductance 1/R \newline - Voltage source becomes current source \newline - Series becomes parallel \newline - Impedance Z becomes admittance 1/Z \\ \hline
\end{tabulary}

\begin{center}
\begin{circuitikz}[auto]
    \node [coordinate] (O) at (0,0) {};
    
    % Original (Series R L)
    \draw (-2,0) to [R, l=$R$] (-2,2) to [L, l=$L$] (0,2);
    
    % Dual nodes
    \node [gtu state, fill=red!20] (N1) at (-1,1) {1};
    \node [gtu state, fill=red!20] (N2) at (1,1) {2};
    
    \draw [dashed, red, thick] (N1) -- node[above] {Dual $G$} (-3,1);
    \draw [dashed, red, thick] (N1) -- node[above] {Dual $C$} (1,1);
    
    \node at (0,-1) {Conceptual Dual Construction};
\end{circuitikz}
\captionof{figure}{Dual Network Construction}
\end{center}
\end{solutionbox}

\begin{mnemonicbox}
\mnemonic{DVSG: Dual transforms Voltage to Series to Graphs}
\end{mnemonicbox}

\questionmarks{3(c OR)}{7}{Find out Norton's equivalent circuit for the given network. Find out load current if (i) $R_L = 3$ k$\Omega$ (ii) $R_L = 1.5$ $\Omega$}

\begin{solutionbox}
\begin{center}
\begin{circuitikz}[auto]
    % 10V source, 2k resistors bridge-like structure
    % Based on MDX goat diagram
    %        2k          2k          2k
    %       ----        ----        ----
    %      /    \      /    \      /    \
    %   C +      D    +      E    +      A
    % ... 10V ...
    
    % It looks like a ladder network
    \node [coordinate] (B) at (0,0) {};
    \node [coordinate] (C) at (0,2) {};
    
    \draw (C) to [V, l=10V] (B);
    
    \draw (C) to [R, l=2k$\Omega$] (2,2) coordinate (D);
    \draw (D) to [R, l=2k$\Omega$] (2,0) coordinate (B2);
    \draw (B) -- (B2);
    
    \draw (D) to [R, l=2k$\Omega$] (4,2) coordinate (E);
    \draw (E) to [R, l=2k$\Omega$] (4,0) coordinate (B3);
    \draw (B2) -- (B3);
    
    \draw (E) to [R, l=2k$\Omega$] (6,2) coordinate (A);
    \draw (B3) -- (6,0) coordinate (B_end);
    
    \draw (A) to [R, l=$R_L$] (B_end);
    \node [above] at (A) {A};
    \node [below] at (B_end) {B};
\end{circuitikz}
\captionof{figure}{Norton Problem Circuit}
\end{center}

\begin{itemize}
    \item \textbf{Step 1:} Find Norton's current ($I_N$)
    \item \textbf{Step 2:} Find Norton's resistance ($R_N$)
    \item \textbf{Step 3:} Calculate load currents
\end{itemize}

\begin{tabulary}{\linewidth}{|L|L|L|}
\hline
\textbf{Step} & \textbf{Calculation} & \textbf{Result} \\ \hline
\textbf{$I_N$} & Short circuit current from A to B & 1.25 mA \\ \hline
\textbf{$R_N$} & Equivalent resistance seen from A to B with 10V source shorted & 1 k$\Omega$ \\ \hline
\textbf{$I_L$ ($R_L = 3$ k$\Omega$)} & $I_L = I_N \times R_N/(R_N + R_L) = 1.25 \times 1/(1 + 3)$ & 0.31 mA \\ \hline
\textbf{$I_L$ ($R_L = 1.5$ $\Omega$)} & $I_L = I_N \times R_N/(R_N + R_L) = 1.25 \times 1000/(1000 + 1.5)$ & 1.25 mA \\ \hline
\end{tabulary}
\end{solutionbox}

\begin{mnemonicbox}
\mnemonic{NICE: Norton's circuit Is Current Equivalent}
\end{mnemonicbox}


% ==================================================================
% QUESTION 4
% ==================================================================

\questionmarks{4(a)}{3}{Derive the equation of Quality factor Q for a coil.}

\begin{solutionbox}
\begin{tabulary}{\linewidth}{|L|L|}
\hline
\textbf{Parameter} & \textbf{Relationship} \\ \hline
\textbf{Q factor definition} & Ratio of energy stored to energy dissipated per cycle \\ \hline
\textbf{Coil impedance} & $Z = R + j\omega L$ \\ \hline
\textbf{Reactance} & $X_L = \omega L$ \\ \hline
\textbf{Quality factor} & $Q = X_L/R = \omega L/R$ \\ \hline
\end{tabulary}

\begin{center}
\begin{circuitikz}[auto]
    \draw (0,0) to [R, l=$R$, o-] (2,0) to [L, l=$L$, -o] (4,0);
    \node at (2,-1) {Practical Coil Model};
\end{circuitikz}
\captionof{figure}{Coil Equivalent Circuit}
\end{center}

For a coil, the energy stored is in the magnetic field (in the inductor), while energy dissipated is in the resistance. From this:
\[ Q = 2\pi \times \frac{\text{Energy stored}}{\text{Energy dissipated per cycle}} \]
\[ Q = \frac{\omega L}{R} \]
\end{solutionbox}

\begin{mnemonicbox}
\mnemonic{QREL: Quality Relates Energy to Loss}
\end{mnemonicbox}

\questionmarks{4(b)}{4}{A series RLC circuit has R = 30 $\Omega$, L = 0.5 H and C = 5 $\mu$F. Calculate (i) Q factor, (ii) BW, (iii) Upper cut off and lower cut off frequencies.}

\begin{solutionbox}
\begin{center}
\begin{circuitikz}[auto]
    \draw (0,0) to [R, l=$R$, o-] (2,0) to [L, l=$L$] (4,0) to [C, l=$C$, -o] (6,0);
    \node at (3,-1) {Series RLC Circuit};
\end{circuitikz}
\captionof{figure}{Series RLC}
\end{center}

\begin{tabulary}{\linewidth}{|L|L|L|}
\hline
\textbf{Parameter} & \textbf{Formula} & \textbf{Result} \\ \hline
\textbf{Resonant frequency ($f_0$)} & $f_0 = \frac{1}{2\pi\sqrt{LC}}$ & 100.53 Hz \\ \hline
\textbf{Q factor} & $Q = \frac{1}{R}\sqrt{\frac{L}{C}}$ & 105.57 \\ \hline
\textbf{Bandwidth (BW)} & $BW = f_0/Q$ & 0.952 Hz \\ \hline
\textbf{Lower cutoff ($f_1$)} & $f_1 = f_0 - BW/2$ & 100.05 Hz \\ \hline
\textbf{Upper cutoff ($f_2$)} & $f_2 = f_0 + BW/2$ & 101.01 Hz \\ \hline
\end{tabulary}
\end{solutionbox}

\begin{mnemonicbox}
\mnemonic{QBCUT: Quality Bandwidth Cutoff Uniquely Related}
\end{mnemonicbox}

\questionmarks{4(c)}{7}{Explain Mutual Inductance along with Co-efficient of mutual inductance. Also derive the equation of K.}

\begin{solutionbox}
\begin{tabulary}{\linewidth}{|L|L|}
\hline
\textbf{Concept} & \textbf{Description} \\ \hline
\textbf{Mutual Inductance (M)} & Property where current change in one coil induces voltage in adjacent coil \\ \hline
\textbf{Definition} & Ratio of induced voltage in secondary to rate of change of current in primary \\ \hline
\textbf{Formula} & $M = k\sqrt{L_1 L_2}$ \\ \hline
\textbf{Coefficient of coupling (k)} & Measure of magnetic coupling between coils ($0 \le k \le 1$) \\ \hline
\end{tabulary}

\begin{center}
\begin{circuitikz}[auto]
    \draw (0,0) to [L, l=$L_1$] (0,2);
    \draw (2,0) to [L, l=$L_2$] (2,2);
    \draw [dashed, <->] (0.5,1) -- node[above] {$M$} (1.5,1);
    \node at (1,-0.5) {Coupled Coils};
\end{circuitikz}
\captionof{figure}{Mutual Inductance}
\end{center}

For two inductors $L_1$ and $L_2$, mutual inductance $M$ is:
\[ M = k\sqrt{L_1 L_2} \]

Where coefficient of coupling $k$ is:
\[ k = \frac{M}{\sqrt{L_1 L_2}} \]

$k$ represents fraction of magnetic flux from one coil linking with another coil.
\begin{itemize}
    \item For perfectly coupled coils, $k = 1$
    \item For no coupling, $k = 0$
\end{itemize}
\end{solutionbox}

\begin{mnemonicbox}
\mnemonic{MKL: Mutual coupling K Links inductors}
\end{mnemonicbox}

\questionmarks{4(a OR)}{3}{Explain the types of coupling for coupled circuit.}

\begin{solutionbox}
\begin{tabulary}{\linewidth}{|L|L|L|}
\hline
\textbf{Type of Coupling} & \textbf{Characteristics} & \textbf{Applications} \\ \hline
\textbf{Tight/Close Coupling ($k \approx 1$)} & - Nearly all flux links both coils \newline - High transfer efficiency \newline - $k$ value close to 1 & Transformers, Power transfer \\ \hline
\textbf{Loose Coupling ($k \ll 1$)} & - Small fraction of flux links second coil \newline - Lower transfer efficiency \newline - $k$ value much less than 1 & RF circuits, Tuned filters \\ \hline
\textbf{Critical Coupling ($k=k_c$)} & - Optimum coupling for bandpass response \newline - Maximum power transfer at resonance & Bandpass filters, IF transformers \\ \hline
\textbf{Inductive Coupling} & - Coupling via magnetic field & Transformers, Wireless charging \\ \hline
\textbf{Capacitive Coupling} & - Coupling via electric field & Signal coupling, Capacitive sensors \\ \hline
\end{tabulary}

\begin{center}
\begin{circuitikz}[auto, node distance=2cm]
    \node [gtu block] (tight) {Tight Coupling \\ $k \approx 1$};
    \node [gtu block, right=of tight] (loose) {Loose Coupling \\ $k \ll 1$};
    \node [gtu block, right=of loose] (crit) {Critical Coupling \\ $k = k_c$};
\end{circuitikz}
\captionof{figure}{Types of Coupling}
\end{center}
\end{solutionbox}

\begin{mnemonicbox}
\mnemonic{TLC: Tight, Loose, Critical couplings}
\end{mnemonicbox}

\questionmarks{4(b OR)}{4}{A parallel resonant circuit having inductance of 1 mH with quality factor Q = 100, resonant frequency Fr = 100 KHz. Find out (i) Required capacitance C, (ii) Resistance R of the coil, (iii) BW.}

\begin{solutionbox}
\begin{center}
\begin{circuitikz}[auto]
    \draw (0,2) to [short, o-*] (1,2) -- (4,2) to [short, *-o] (5,2);
    \draw (0,0) to [short, o-*] (1,0) -- (4,0) to [short, *-o] (5,0);
    
    \draw (2,2) to [C, l=$C$] (2,0);
    \draw (3,2) to [L, l=$L$] (3,1) to [R, l=$R_{coil}$] (3,0);
    
    \node at (2.5,-1) {Parallel Resonant Circuit (Tank Circuit)};
\end{circuitikz}
\captionof{figure}{Parallel Resonance}
\end{center}

\begin{tabulary}{\linewidth}{|L|L|L|}
\hline
\textbf{Parameter} & \textbf{Formula} & \textbf{Result} \\ \hline
\textbf{Capacitance (C)} & $C = \frac{1}{4\pi^2 f^2 L}$ & 2.533 nF \\ \hline
\textbf{Coil Resistance (R)} & $R = \frac{\omega L}{Q}$ & 6.28 $\Omega$ \\ \hline
\textbf{Bandwidth (BW)} & $BW = f_r/Q$ & 1 kHz \\ \hline
\end{tabulary}
\end{solutionbox}

\begin{mnemonicbox}
\mnemonic{RCB: Resonance needs Capacitance and Bandwidth}
\end{mnemonicbox}

\questionmarks{4(c OR)}{7}{Explain Band width and Selectivity of a series RLC circuit. Also establish the relation between Q factor and BW for series resonance circuit.}

\begin{solutionbox}
\begin{tabulary}{\linewidth}{|L|L|L|}
\hline
\textbf{Parameter} & \textbf{Definition} & \textbf{Relationship} \\ \hline
\textbf{Bandwidth (BW)} & Frequency range between half-power points & $BW = f_2 - f_1 = \omega_2 - \omega_1 = R/L$ \\ \hline
\textbf{Selectivity} & Ability to differentiate between signals of different frequencies & Inversely proportional to BW \\ \hline
\textbf{Q factor} & Ratio of resonant frequency to bandwidth & $Q = \omega_0/BW = \omega_0 L/R$ \\ \hline
\end{tabulary}

\begin{center}
\begin{circuitikz}[auto]
    \begin{axis}[
        width=8cm, height=5cm,
        axis lines=middle,
        xlabel={$f$}, ylabel={Current $I$},
        xtick=\empty, ytick=\empty,
        clip=false
    ]
        \addplot[domain=0.5:2, samples=100, smooth, thick, blue] {1/sqrt((1-x^2)^2 + (0.2*x)^2)};
        
        \node at (axis cs: 1, 5) [above] {$I_{max}$};
        \node at (axis cs: 1, 0) [below] {$f_0$};
        
        \draw [dashed] (axis cs: 0, 3.5) -- (axis cs: 2, 3.5);
        \node at (axis cs: 2, 3.5) [right] {$0.707 I_{max}$};
        
        \draw [dashed] (axis cs: 0.9, 0) -- (axis cs: 0.9, 3.5);
        \node at (axis cs: 0.9, 0) [below] {$f_1$};
        
        \draw [dashed] (axis cs: 1.1, 0) -- (axis cs: 1.1, 3.5);
        \node at (axis cs: 1.1, 0) [below] {$f_2$};
        
        \draw [<->] (axis cs: 0.9, 1.5) -- node[below] {BW} (axis cs: 1.1, 1.5);
    \end{axis}
    \node at (4,-1) {Resonance Curve};
\end{circuitikz}
\captionof{figure}{Frequency Response}
\end{center}

For a series RLC circuit:
\begin{itemize}
    \item At resonance ($f_0$), impedance is minimum (= $R$)
    \item Half-power points occur when impedance = $\sqrt{2}R$
    \item At these points, power is half of maximum power
\end{itemize}

Bandwidth ($BW$) = $\omega_2 - \omega_1 = R/L$ \\
Q factor = $\omega_0 L/R = \omega_0/BW$

Therefore, $BW = \omega_0/Q = 2\pi f_0/Q$

This shows Q factor and bandwidth are inversely related:
Higher Q $\to$ Narrower bandwidth $\to$ Better selectivity

\end{solutionbox}

\begin{mnemonicbox}
\mnemonic{BQS: Bandwidth and Q determine Selectivity}
\end{mnemonicbox}


% ==================================================================
% QUESTION 5
% ==================================================================

\questionmarks{5(a)}{3}{Design a symmetrical T type attenuator to give attenuation of 40 dB and work in to the load of 300 $\Omega$ resistance.}

\begin{solutionbox}
\begin{center}
\begin{circuitikz}[auto]
    \node [coordinate] (in_top) at (0,2) {};
    \node [coordinate] (in_bot) at (0,0) {};
    \node [coordinate] (out_top) at (6,2) {};
    \node [coordinate] (out_bot) at (6,0) {};
    
    \draw (in_top) to [R, l=$R_1/2$] (3,2) coordinate (mid);
    \draw (mid) to [R, l=$R_1/2$] (out_top);
    \draw (in_bot) -- (out_bot);
    \draw (mid) to [R, l=$R_2$] (3,0);
    
    \draw [dashed] (out_top) -- node[right] {$R_L=300\Omega$} (out_bot);
    
    \node at (0.5,1) {Input};
    \node at (5.5,1) {Output};
\end{circuitikz}
\captionof{figure}{T-Attenuator Design}
\end{center}

\begin{tabulary}{\linewidth}{|L|L|L|}
\hline
\textbf{Parameter} & \textbf{Formula} & \textbf{Result} \\ \hline
\textbf{Attenuation (N)} & $N = 10^{dB/20}$ & $10^{40/20} = 100$ \\ \hline
\textbf{Impedance ratio (K)} & $K = \frac{N+1}{N-1}$ & $\frac{101}{99} \approx 1.02$ \\ \hline
\textbf{$Z_1$} & $Z_1 = R_0(\frac{K-1}{K})$ & $300(\frac{0.02}{1.02}) = 5.88 \Omega$ \\ \hline
\textbf{$Z_2$} & $Z_2 = R_0(\frac{2K}{K^2-1})$ & $300(\frac{2.04}{1.04-1}) = 594.12 \Omega$ \\ \hline
\end{tabulary}
(Note: Using standard design formulas for T-attenuator components $R_1$ (series total) and $R_2$ (shunt). In T-section, series arms are $R_1/2$. The table calculates total series resistance $Z_1$ or component values? Usually formulas give full series arm $R_1$ or the individual arms. Let's assume standard definitions: Formula $R_1 = R_0\frac{N-1}{N+1}$? No, MDX formulas are slightly different, specifically using K? K is usually N. Let's check MDX fidelity.
MDX Formula: $Z_1 = R_0[(K-1)/K]$ where $K = (N+1)/(N-1)$.
Wait, if $N=100$.
$K = 101/99 \approx 1.02$.
$Z_1 = 300 * (0.02/1.02) \approx 5.88$.
$Z_2 = 300 * (2.04 / (1.02^2 - 1)) \approx 300 * (2.04 / 0.0404) \approx 15148$. MDX says 594.12.
Let's check $594.12$.
Standard T-pad: $R_1 = R_0 \frac{N-1}{N+1} = 300 * \frac{99}{101} = 294$.
$R_2 = R_0 \frac{2N}{N^2-1} = 300 * \frac{200}{9999} = 6$.
Something is wrong with MDX formulas or my interpretation of MDX $Z_1, Z_2$.
However, faithful conversion requires copying the text, unless it's blatantly wrong and I should correct it. The user said "DO NOT create, expand, or streamline... Migrate the EXACT text". But simple math errors? "STRICT Content Fidelity". I will copy the MDX calculation and result exactly as is, adding a note if needed, but per instructions, fidelity first.
MDX: $Z_2 = 594.12$. calculation shown: $300[2 \times 1.02 / (1.02^2 - 1)]$.
$1.02^2 - 1 = 1.0404 - 1 = 0.0404$.
$2 * 1.02 = 2.04$.
$2.04 / 0.0404 \approx 50.5$.
$300 * 50.5 = 15150$.
MDX Result 594.12 is weird.
Maybe K in MDX means something else?
If $Z_2 = 600 \Omega$ roughly?
Let's just transcribe the MDX table exactly.

\end{solutionbox}

\begin{mnemonicbox}
\mnemonic{TANZ: T-Attenuator Needs Z-parameters}
\end{mnemonicbox}

\questionmarks{5(b)}{4}{Give classification of filters.}

\begin{solutionbox}
\begin{tabulary}{\linewidth}{|L|L|L|}
\hline
\textbf{Classification} & \textbf{Types} & \textbf{Characteristics} \\ \hline
\textbf{Based on Frequency Response} & - Low Pass \newline - High Pass \newline - Band Pass \newline - Band Stop & - Passes frequencies below cutoff \newline - Passes frequencies above cutoff \newline - Passes frequencies within a band \newline - Blocks frequencies within a band \\ \hline
\textbf{Based on Components} & - Passive Filters \newline - Active Filters & - Uses R, L, C elements \newline - Uses active devices with RC \\ \hline
\textbf{Based on Design Approach} & - Constant-k Filters \newline - m-derived Filters \newline - Composite Filters & - Simplest design \newline - Better cutoff characteristics \newline - Combines advantages \\ \hline
\textbf{Based on Technology} & - LC Filters \newline - Crystal Filters \newline - Ceramic Filters \newline - Digital Filters & - Uses inductors and capacitors \newline - Uses piezoelectric crystals \newline - Uses piezoelectric ceramics \newline - Implemented in software \\ \hline
\end{tabulary}

\begin{center}
\begin{circuitikz}[
    level 1/.style = {sibling distance=3.5cm},
    level 2/.style = {sibling distance=1.5cm},
    edge from parent/.style = {draw, -latex},
    every node/.style = {rectangle, draw, rounded corners, align=center, font=\scriptsize}
]
    \node {Filters}
        child { node {Freq Response}
            child { node {LP/HP} }
            child { node {BP/BS} }
        }
        child { node {Components}
            child { node {Passive} }
            child { node {Active} }
        }
        child { node {Design}
            child { node {Constant-k} }
            child { node {m-derived} }
        }
        child { node {Technology}
            child { node {LC/Crystal} }
            child { node {Digital} }
        };
\end{circuitikz}
\captionof{figure}{Filter Classification}
\end{center}
\end{solutionbox}

\begin{mnemonicbox}
\mnemonic{FLAC: Filters: Low-pass, Active, Constant-k}
\end{mnemonicbox}

\questionmarks{5(c)}{7}{Explain constant K Low Pass Filter.}

\begin{solutionbox}
\begin{tabulary}{\linewidth}{|L|L|}
\hline
\textbf{Concept} & \textbf{Description} \\ \hline
\textbf{Definition} & Filter where impedance product $Z_1Z_2 = k^2$ (constant) at all frequencies \\ \hline
\textbf{Circuit Types} & T-section and $\pi$-section \\ \hline
\textbf{T-section components} & Series inductors ($L/2$) and shunt capacitor ($C$) \\ \hline
\textbf{$\pi$-section components} & Series inductor ($L$) and shunt capacitors ($C/2$) \\ \hline
\textbf{Cutoff frequency} & $f_c = 1/(\pi\sqrt{LC})$ \\ \hline
\textbf{Characteristic impedance} & $R_0 = \sqrt{L/C}$ \\ \hline
\end{tabulary}

\begin{center}
\begin{circuitikz}[auto]
    \node at (0,3) {(a) T-section};
    \draw (0,2) to [L, l=$L/2$] (2,2) to [L, l=$L/2$] (4,2);
    \draw (2,2) to [C, l=$C$] (2,0);
    \draw (0,0) -- (4,0);

    \node at (6,3) {(b) $\pi$-section};
    \draw (6,2) -- (7,2) to [L, l=$L$] (9,2) -- (10,2);
    \draw (7,2) to [C, l=$C/2$] (7,0);
    \draw (9,2) to [C, l=$C/2$] (9,0);
    \draw (6,0) -- (10,0);
\end{circuitikz}
\captionof{figure}{Constant-k Low Pass Filter}
\end{center}

The constant-k low pass filter has:
\begin{itemize}
    \item Cutoff frequency: $f_c = \frac{1}{\pi\sqrt{LC}}$
    \item Design impedance: $R_0 = \sqrt{\frac{L}{C}}$
    \item Pass band: 0 to $f_c$
    \item Attenuation band: Above $f_c$
    \item Gradual transition from pass band to stop band
\end{itemize}
\end{solutionbox}

\begin{mnemonicbox}
\mnemonic{CLPT: Constant-k Low Pass needs T-section}
\end{mnemonicbox}

\questionmarks{5(a OR)}{3}{Design a high pass filter with T section having a cut-off frequency of 1.5 KHz with a load resistance of 400 $\Omega$.}

\begin{solutionbox}
\begin{center}
\begin{circuitikz}[auto]
    \draw (0,2) to [C, l=$2C$] (2,2) to [C, l=$2C$] (4,2);
    \draw (2,2) to [L, l=$L$] (2,0);
    \draw (0,0) -- (4,0);
    \node at (2,-1) {High Pass T-Section};
\end{circuitikz}
\captionof{figure}{High Pass Filter Design}
\end{center}

\begin{tabulary}{\linewidth}{|L|L|L|}
\hline
\textbf{Parameter} & \textbf{Formula} & \textbf{Result} \\ \hline
\textbf{Design impedance ($R_0$)} & $R_0 = \text{Load resistance}$ & 400 $\Omega$ \\ \hline
\textbf{Cutoff frequency ($f_c$)} & $f_c = \text{Given}$ & 1.5 kHz \\ \hline
\textbf{Inductor (L)} & $L = \frac{R_0}{4\pi f_c}$ (Wait, MDX says $R_0/2\pi f_c$?) & 42.44 mH \\ \hline
\textbf{Capacitor (C)} & $C = \frac{1}{4\pi f_c R_0}$ (MDX says $1/2\pi f_c R_0$?) & 0.265 $\mu$F \\ \hline
\end{tabulary}
(Note: Standard Constant-k T-section HPF design formulas are $L = R_0 / 4\pi f_c$ and $C = 1 / 4\pi f_c R_0$. MDX calculates $L = 400 / (2\pi \times 1500) = 42.44$ mH. This uses $2\pi$. I will follow MDX formulas and results for fidelity.)
Formula in MDX Table: $L = R_0/2\pi f_c$, $C = 1/(2\pi f_c R_0)$.
Calculation: $400/(2\pi \times 1500) = 42.44$ mH.
Calculation: $1/(2\pi \times 1500 \times 400) = 0.265 \mu$F.

\end{solutionbox}

\begin{mnemonicbox}
\mnemonic{HCL: High-pass needs Capacitor and inductor}
\end{mnemonicbox}

\questionmarks{5(b OR)}{4}{Give classification of attenuators.}

\begin{solutionbox}
\begin{tabulary}{\linewidth}{|L|L|L|}
\hline
\textbf{Classification} & \textbf{Types} & \textbf{Characteristics} \\ \hline
\textbf{Based on Configuration} & - T-attenuator \newline - $\pi$-attenuator \newline - Bridged-T \newline - Lattice & - Series-shunt-series \newline - Shunt-series-shunt \newline - Balanced bridge \newline - Balanced network \\ \hline
\textbf{Based on Symmetry} & - Symmetrical \newline - Asymmetrical & - Equal impedance \newline - Unequal impedance \\ \hline
\textbf{Based on Control} & - Fixed \newline - Variable \newline - Programmable & - Constant attenuation \newline - Adjustable attenuation \newline - Digitally controlled \\ \hline
\textbf{Based on Technology} & - Resistive \newline - Reactive \newline - Active & - Uses resistors \newline - Uses reactances \newline - Uses active devices \\ \hline
\end{tabulary}
\end{solutionbox}


\begin{mnemonicbox}
\mnemonic{CAST: Configuration, Adjustable, Symmetry, Technology}
\end{mnemonicbox}

\questionmarks{5(c) OR}{7}{Explain constant K High Pass Filter.}

\begin{solutionbox}
\begin{tabulary}{\linewidth}{|L|L|}
\hline
\textbf{Concept} & \textbf{Description} \\ \hline
\textbf{Definition} & Filter passing frequencies above cutoff, with $Z_1Z_2 = k^2$ (constant) \\ \hline
\textbf{Circuit Types} & T-section and $\pi$-section \\ \hline
\textbf{T-section components} & Series capacitors ($C/2$) and shunt inductor ($L$) \\ \hline
\textbf{$\pi$-section components} & Series capacitor ($C$) and shunt inductors ($L/2$) \\ \hline
\textbf{Cutoff frequency} & $f_c = 1/(\pi\sqrt{LC})$ \\ \hline
\textbf{Characteristic impedance} & $R_0 = \sqrt{L/C}$ \\ \hline
\end{tabulary}

\begin{center}
\begin{circuitikz}[auto]
    \node at (0,3) {(a) T-section};
    \draw (0,2) to [C, l=$C/2$] (2,2) to [C, l=$C/2$] (4,2);
    \draw (2,2) to [L, l=$L$] (2,0);
    \draw (0,0) -- (4,0);

    \node at (6,3) {(b) $\pi$-section};
    \draw (6,2) -- (7,2) to [C, l=$C$] (9,2) -- (10,2);
    \draw (7,2) to [L, l=$L/2$] (7,0);
    \draw (9,2) to [L, l=$L/2$] (9,0);
    \draw (6,0) -- (10,0);
\end{circuitikz}
\captionof{figure}{Constant-k High Pass Filter}
\end{center}

The constant-k high pass filter has:
\begin{itemize}
    \item Cutoff frequency: $f_c = \frac{1}{\pi\sqrt{LC}}$
    \item Design impedance: $R_0 = \sqrt{\frac{L}{C}}$
    \item Pass band: Above $f_c$
    \item Attenuation band: 0 to $f_c$
    \item Gradual transition from pass band to stop band
    \item Component values are dual of low pass filter (L and C swap places)
\end{itemize}
\end{solutionbox}

\begin{mnemonicbox}
\mnemonic{CHTS: Constant-k High-pass uses T-Section}
\end{mnemonicbox}

\end{document}
