\documentclass[10pt,a4paper]{article}

% content/resources/templates/preamble.tex
\usepackage[margin=0.6in]{geometry}
\author{Milav Dabgar}
\usepackage{amsmath,amssymb,amsthm}
\usepackage{booktabs}
\usepackage{multirow}
\usepackage{xcolor}
\usepackage{tcolorbox}
\tcbuselibrary{breakable,skins}
\usepackage[colorlinks=true,linkcolor=blue]{hyperref}
\usepackage{titlesec}
\usepackage{enumitem}
\usepackage{tikz}
\usepackage{pgfplots}
\usepackage{circuitikz}
\usepackage[version=4]{mhchem}
\usepackage{longtable}
\usepackage{array}
\usepackage{float}
\usepackage{caption}
\usepackage{listings}

\lstset{
  basicstyle=\small\ttfamily,
  breaklines=true,
  breakatwhitespace=false,
  postbreak=\mbox{\textcolor{red}{$\hookrightarrow$}\space},
  float=false,
  numbers=left,
  numberstyle=\tiny\color{gray},
  numbersep=10pt,
  xleftmargin=2em,
  keywordstyle=\color{blue},
  commentstyle=\color{green!60!black},
  stringstyle=\color{purple},
  backgroundcolor=\color{gray!5},
  showstringspaces=false,
  tabsize=2,
  captionpos=b,
  keepspaces=true,
  columns=flexible
}

\pgfplotsset{compat=1.18}
\usetikzlibrary{shapes,arrows,positioning,calc,patterns,decorations.pathmorphing,decorations.markings,arrows.meta}

% Color scheme
\definecolor{headcolor}{RGB}{0,102,204}
\definecolor{keycolor}{RGB}{220,20,60}
\definecolor{solutioncolor}{RGB}{34,139,34}
\definecolor{mnemoniccolor}{RGB}{148,0,211}
\definecolor{codecolor}{RGB}{0,0,100}

% Spacing
\setlength{\parskip}{3pt}
\setlist[itemize]{nosep}
\setlist[enumerate]{nosep}

% Title formatting
\titleformat{\section}{\Large\bfseries\color{headcolor}}{\thesection}{1em}{}
\titleformat{\subsection}{\large\bfseries\color{headcolor}}{\thesubsection}{1em}{}

% Pandoc tightlist compatibility
\providecommand{\tightlist}{%
  \setlength{\itemsep}{0pt}\setlength{\parskip}{0pt}}

% Pandoc longtable compatibility
\newcounter{none}
\def\thenone{}


% content/resources/templates/english-boxes.tex
% This file is currently empty - it exists to maintain consistency with the import structure.
% Add custom environments here if needed in the future.


\begin{document}

\begin{center}
{\Huge\bfseries\color{headcolor} Subject Name Solutions}\\[5pt]
{\LARGE 4331101 -- Winter 2023}\\[3pt]
{\large Semester 1 Study Material}\\[3pt]
{\normalsize\textit{Detailed Solutions and Explanations}}
\end{center}

\vspace{10pt}

\subsection*{Question 1(a) [3 marks]}\label{q1a}

\textbf{Explain Source transformation with appropriate diagram.}

\begin{solutionbox}
Source transformation is a technique to convert voltage
source to current source or vice-versa without changing the external
circuit behavior.

\textbf{Diagram:}

\begin{center}
\textbf{Mermaid Diagram (Code)}
\begin{verbatim}
{Shaded}
{Highlighting}[]
graph LR
    subgraph "Voltage Source Circuit"
    VS[V] {-{-}{-} RS[R]}
    end
    subgraph "Current Source Circuit"
    IS[I] {-.{-} RP[R]}
    end

    VS {-{-}{-} IS}
    
    class VS,IS fill:\#f96
{Highlighting}
{Shaded}
\end{verbatim}
\end{center}

\begin{itemize}
\tightlist
\item
  \textbf{Voltage to Current Source}: I = V/R, same R in parallel
\item
  \textbf{Current to Voltage Source}: V = I\timesR, same R in series
\end{itemize}

\end{solutionbox}
\begin{mnemonicbox}
``Value Stays, Resistance Shifts'' (V=IR always
applies)

\end{mnemonicbox}
\subsection*{Question 1(b) [4 marks]}\label{q1b}

\textbf{Determine voltage, current and power relationship for two
capacitor connected in series.}

\begin{solutionbox}


{\def\LTcaptype{none} % do not increment counter
\vspace{-5pt}
\captionof{table}{Capacitors in Series}
\vspace{-10pt}
\begin{longtable}[]{@{}lll@{}}
\toprule\noalign{}
Parameter & Formula & Explanation \\
\midrule\noalign{}
\endhead
\bottomrule\noalign{}
\endlastfoot
Total Capacitance & 1/CT = 1/C_{1} + 1/C_{2} & Reciprocal sum \\
Voltage Distribution & V_{1}/V_{2} = C_{2}/C_{1} & Inverse to capacitance ratio \\
Current &

I = I_{1} = I_{2} & Same current flows through all \\

Charge &

Q = Q_{1} = Q_{2} & Same charge on each capacitor \\

Power &

P = VI = V^{2}/Xc & Where Xc = 1/2πfC \\

\end{longtable}
}

\begin{itemize}
\tightlist
\item
  \textbf{Voltage division}: V_{1} = V \times C_{2}/(C_{1}+C_{2})
\item
  \textbf{Charge storage}: Q = C_{1}C_{2}V/(C_{1}+C_{2})
\end{itemize}

\end{solutionbox}
\begin{mnemonicbox}
``Capacitors in Series: Currents Same, Capacitance
Shrinks''

\end{mnemonicbox}
\subsection*{Question 1(c) [7 marks]}\label{q1c}

\textbf{State difference between Series and parallel connection of
resistor and derive the equation of total resistance of parallel
connection.}

\begin{solutionbox}


{\def\LTcaptype{none} % do not increment counter
\vspace{-5pt}
\captionof{table}{Series vs Parallel Resistors}
\vspace{-10pt}
\begin{longtable}[]{@{}
  >{\raggedright\arraybackslash}p{(\linewidth - 4\tabcolsep) * \real{0.2200}}
  >{\raggedright\arraybackslash}p{(\linewidth - 4\tabcolsep) * \real{0.3600}}
  >{\raggedright\arraybackslash}p{(\linewidth - 4\tabcolsep) * \real{0.4200}}@{}}
\toprule\noalign{}
\begin{minipage}[b]{\linewidth}\raggedright
Parameter
\end{minipage} & \begin{minipage}[b]{\linewidth}\raggedright
Series Connection
\end{minipage} & \begin{minipage}[b]{\linewidth}\raggedright
Parallel Connection
\end{minipage} \\
\midrule\noalign{}
\endhead
\bottomrule\noalign{}
\endlastfoot
Total Resistance & Increases (RT = R_{1} + R_{2} + \ldots) & Decreases (RT
\textless{} smallest R) \\
Current & Same through all (I) & Divides (IT = I_{1} + I_{2} + \ldots) \\
Voltage & Divides (VT = V_{1} + V_{2} + \ldots) & Same across all (V) \\
Power & PT = P_{1} + P_{2} + \ldots{} & PT = P_{1} + P_{2} + \ldots{} \\
\end{longtable}
}

\textbf{Derivation for Parallel Resistance:}

By Kirchhoff's Current Law: IT = I_{1} + I_{2} + \ldots{} + In

Substituting

I = V/R: V/RT = V/R_{1} + V/R_{2} + \ldots{} + V/Rn


Dividing by V: 1/RT = 1/R_{1} + 1/R_{2} + \ldots{} + 1/Rn

For two resistors: 1/RT = 1/R_{1} + 1/R_{2}, which gives RT = R_{1}R_{2}/(R_{1}+R_{2})

\end{solutionbox}
\begin{mnemonicbox}
``In Parallel, Reciprocals Add''

\end{mnemonicbox}
\subsection*{Question 1(c) OR [7
marks]}\label{q1c}

\textbf{1) Define unilateral, bilateral network, Mesh and Loop.}
\textbf{2) Draw voltage division circuit and write equation.}

\begin{solutionbox}


{\def\LTcaptype{none} % do not increment counter
\vspace{-5pt}
\captionof{table}{Network Definitions}
\vspace{-10pt}
\begin{longtable}[]{@{}
  >{\raggedright\arraybackslash}p{(\linewidth - 4\tabcolsep) * \real{0.2222}}
  >{\raggedright\arraybackslash}p{(\linewidth - 4\tabcolsep) * \real{0.4444}}
  >{\raggedright\arraybackslash}p{(\linewidth - 4\tabcolsep) * \real{0.3333}}@{}}
\toprule\noalign{}
\begin{minipage}[b]{\linewidth}\raggedright
Term
\end{minipage} & \begin{minipage}[b]{\linewidth}\raggedright
Definition
\end{minipage} & \begin{minipage}[b]{\linewidth}\raggedright
Example
\end{minipage} \\
\midrule\noalign{}
\endhead
\bottomrule\noalign{}
\endlastfoot
Unilateral Network & Allows current in one direction only & Diode
circuit \\
Bilateral Network & Allows current in both directions & RLC circuit \\
Mesh & Planar network path with no other path inside it & Single closed
path \\
Loop & Any closed path in a network & Can contain other elements \\
\end{longtable}
}

\textbf{Voltage Division Circuit:}

\begin{center}
\textbf{Mermaid Diagram (Code)}
\begin{verbatim}
{Shaded}
{Highlighting}[]
graph TD
    A[Input] {-{-}{-} R1[R_{1}] {-}{-}{-} B[Output V_{0}] {-}{-}{-} R2[R_{2}] {-}{-}{-} C[Ground]}
{Highlighting}
{Shaded}
\end{verbatim}
\end{center}

\textbf{Voltage Division Equation:} Vo = Vin \times R_{2}/(R_{1}+R_{2})

\begin{itemize}
\tightlist
\item
  \textbf{Proportional to}: Resistance across which voltage is measured
\item
  \textbf{Inversely proportional to}: Total resistance
\end{itemize}

\end{solutionbox}
\begin{mnemonicbox}
``Voltage Output equals Input times Resistance
Ratio''

\end{mnemonicbox}
\subsection*{Question 2(a) [3 marks]}\label{q2a}

\textbf{Derive equations to convert T-type network into π-type network}

\begin{solutionbox}

\textbf{Diagram: T to π Conversion}

\begin{verbatim}
    A    Z_{1    B         A            B}
     o{-{-}{-}//{-}{-}o          o         o}
     |          |          {       /}
     |          |    ={          /}
     Z_{3         Z_{2}              /}
     |          |            Z_{1_{2} Z_{2}_{3}}
     o{-{-}{-}{-}{-}{-}{-}{-}{-}{-}o              /}
     C                         o
                               C
\end{verbatim}

\textbf{Conversion Equations:}

\begin{itemize}
\tightlist
\item
  Z_{1}_{2} = (Z_{1}Z_{2} + Z_{2}Z_{3} + Z_{3}Z_{1})/Z_{3}
\item
  Z_{2}_{3} = (Z_{1}Z_{2} + Z_{2}Z_{3} + Z_{3}Z_{1})/Z_{1}
\item
  Z_{3}_{1} = (Z_{1}Z_{2} + Z_{2}Z_{3} + Z_{3}Z_{1})/Z_{2}
\end{itemize}

Where Z_{1}, Z_{2}, Z_{3} are T-network impedances and Z_{1}_{2}, Z_{2}_{3}, Z_{3}_{1} are
π-network impedances.

\end{solutionbox}
\begin{mnemonicbox}
``Sum of all products divided by the opposite''

\end{mnemonicbox}
\subsection*{Question 2(b) [4 marks]}\label{q2b}

\textbf{Explain Open circuit Impedance Parameter (Z Parameter)}

\begin{solutionbox}

\textbf{Z-Parameters}: Also called open-circuit impedance parameters
because they're measured with output ports open.


{\def\LTcaptype{none} % do not increment counter
\vspace{-5pt}
\captionof{table}{Z-Parameter Equations}
\vspace{-10pt}
\begin{longtable}[]{@{}
  >{\raggedright\arraybackslash}p{(\linewidth - 4\tabcolsep) * \real{0.3056}}
  >{\raggedright\arraybackslash}p{(\linewidth - 4\tabcolsep) * \real{0.3333}}
  >{\raggedright\arraybackslash}p{(\linewidth - 4\tabcolsep) * \real{0.3611}}@{}}
\toprule\noalign{}
\begin{minipage}[b]{\linewidth}\raggedright
Parameter
\end{minipage} & \begin{minipage}[b]{\linewidth}\raggedright
Definition
\end{minipage} & \begin{minipage}[b]{\linewidth}\raggedright
Calculation
\end{minipage} \\
\midrule\noalign{}
\endhead
\bottomrule\noalign{}
\endlastfoot
Z_{1}_{1} & Input impedance with output open & Z_{1}_{1} = V_{1}/I_{1} (when I_{2}=0) \\
Z_{1}_{2} & Transfer impedance from port 2 to port 1 & Z_{1}_{2} = V_{1}/I_{2} (when
I_{1}=0) \\
Z_{2}_{1} & Transfer impedance from port 1 to port 2 & Z_{2}_{1} = V_{2}/I_{1} (when
I_{2}=0) \\
Z_{2}_{2} & Output impedance with input open & Z_{2}_{2} = V_{2}/I_{2} (when I_{1}=0) \\
\end{longtable}
}

\textbf{Matrix Form:} [V_{1}] = [Z_{1}_{1} Z_{1}_{2}] \times [I_{1}] [V_{2}]
[Z_{2}_{1} Z_{2}_{2}] [I_{2}]

\begin{itemize}
\tightlist
\item
  \textbf{Symmetrical Network}: Z_{1}_{2} = Z_{2}_{1}
\item
  \textbf{Units}: Ohms (Ω)
\end{itemize}

\end{solutionbox}
\begin{mnemonicbox}
``Vs equal Zs times Is''

\end{mnemonicbox}
\subsection*{Question 2(c) [7 marks]}\label{q2c}

\textbf{Derive the expressions for the characteristic impedance (Z_{0}_{t})
for Symmetrical T Network.}

\begin{solutionbox}

\textbf{Diagram: Symmetrical T-Network}

\begin{verbatim}
         Z_{1/2        Z_{1}/2}
     o{-{-}{-}{-}//{-}{-}{-}{-}o{-}{-}{-}//{-}{-}{-}{-}o}
     |            |           |
     |            |           |
    Z_{0_{t}          Z_{2}          Z_{0}_{t}}
     |            |           |
     |            |           |
     o{-{-}{-}{-}{-}{-}{-}{-}{-}{-}{-}{-}o{-}{-}{-}{-}{-}{-}{-}{-}{-}{-}{-}o}
\end{verbatim}

\textbf{Derivation:}

\begin{enumerate}
\tightlist
\item
  For symmetrical T-network, Z_{1} is split equally across two arms (Z_{1}/2
  each)
\item
  For image impedance matching: Z_{0}_{t} = Z_{0}_{t}'
\end{enumerate}

By voltage division: V_{2}/V_{1} = Z_{0}_{t}/(Z_{1}/2 + Z_{0}_{t} + Z_{2}\textbar\textbar Z_{0}_{t})

For matched condition: Z_{0}_{t}^{2} = (Z_{1}/2)(Z_{1}/2 + Z_{2})

Therefore: Z_{0}_{t} = \sqrt[(Z_{1}/2)(Z_{1}/2 + Z_{2})] Z_{0}_{t} = \sqrt[Z_{1}^{2}/4 + Z_{1}Z_{2}/2]
Z_{0}_{t} = \sqrt[Z_{1}(Z_{1}+2Z_{2})/4]

\end{solutionbox}
\begin{mnemonicbox}
``The square root of Z_{1} times what Z_{1} meets''

\end{mnemonicbox}
\subsection*{Question 2(a) OR [3
marks]}\label{q2a}

\textbf{Derive equations to convert π-type network into T-type network.}

\begin{solutionbox}

\textbf{Diagram: π to T Conversion}

\begin{verbatim}
    A            B         A    Z_{1    B}
    o          o           o{-{-}{-}//{-}{-}{-}o}
     {        /            |          |}
      {      /             |          |}
      Z_{1_{2}    Z_{2}_{3}     =    Z_{3}         Z_{2}}
       {    /              |          |}
        {  /               o{-}{-}{-}{-}{-}{-}{-}{-}{-}{-}o}
         o                  C
         C
\end{verbatim}

\textbf{Conversion Equations:}

\begin{itemize}
\tightlist
\item
  Z_{1} = (Z_{1}_{2}Z_{3}_{1})/(Z_{1}_{2} + Z_{2}_{3} + Z_{3}_{1})
\item
  Z_{2} = (Z_{2}_{3}Z_{1}_{2})/(Z_{1}_{2} + Z_{2}_{3} + Z_{3}_{1})
\item
  Z_{3} = (Z_{3}_{1}Z_{2}_{3})/(Z_{1}_{2} + Z_{2}_{3} + Z_{3}_{1})
\end{itemize}

Where Z_{1}_{2}, Z_{2}_{3}, Z_{3}_{1} are π-network impedances and Z_{1}, Z_{2}, Z_{3} are
T-network impedances.

\end{solutionbox}
\begin{mnemonicbox}
``Product of adjacent pairs divided by sum of all''

\end{mnemonicbox}
\subsection*{Question 2(b) OR [4
marks]}\label{q2b}

\textbf{Explain Admittance Parameter (Y Parameter).}

\begin{solutionbox}

\textbf{Y-Parameters}: Also called short-circuit admittance parameters
because they're measured with output ports shorted.


{\def\LTcaptype{none} % do not increment counter
\vspace{-5pt}
\captionof{table}{Y-Parameter Equations}
\vspace{-10pt}
\begin{longtable}[]{@{}
  >{\raggedright\arraybackslash}p{(\linewidth - 4\tabcolsep) * \real{0.3056}}
  >{\raggedright\arraybackslash}p{(\linewidth - 4\tabcolsep) * \real{0.3333}}
  >{\raggedright\arraybackslash}p{(\linewidth - 4\tabcolsep) * \real{0.3611}}@{}}
\toprule\noalign{}
\begin{minipage}[b]{\linewidth}\raggedright
Parameter
\end{minipage} & \begin{minipage}[b]{\linewidth}\raggedright
Definition
\end{minipage} & \begin{minipage}[b]{\linewidth}\raggedright
Calculation
\end{minipage} \\
\midrule\noalign{}
\endhead
\bottomrule\noalign{}
\endlastfoot
Y_{1}_{1} & Input admittance with output shorted & Y_{1}_{1} = I_{1}/V_{1} (when V_{2}=0) \\
Y_{1}_{2} & Transfer admittance from port 2 to port 1 & Y_{1}_{2} = I_{1}/V_{2} (when
V_{1}=0) \\
Y_{2}_{1} & Transfer admittance from port 1 to port 2 & Y_{2}_{1} = I_{2}/V_{1} (when
V_{2}=0) \\
Y_{2}_{2} & Output admittance with input shorted & Y_{2}_{2} = I_{2}/V_{2} (when V_{1}=0) \\
\end{longtable}
}

\textbf{Matrix Form:} [I_{1}] = [Y_{1}_{1} Y_{1}_{2}] \times [V_{1}] [I_{2}]
[Y_{2}_{1} Y_{2}_{2}] [V_{2}]

\begin{itemize}
\tightlist
\item
  \textbf{Symmetrical Network}: Y_{1}_{2} = Y_{2}_{1}
\item
  \textbf{Units}: Siemens (S)
\end{itemize}

\end{solutionbox}
\begin{mnemonicbox}
``Is equal Ys times Vs''

\end{mnemonicbox}
\subsection*{Question 2(c) OR [7
marks]}\label{q2c}

\textbf{Derive the expressions for the characteristic impedance (Z_{0}π)
for Symmetrical π Network.}

\begin{solutionbox}

\textbf{Diagram: Symmetrical π-Network}

\begin{verbatim}
     o{-{-}{-}{-}{-}{-}{-}{-}{-}{-}{-}o{-}{-}{-}{-}{-}{-}{-}{-}{-}{-}o}
     |           |          |
     |           |          |
    2Z_{3         Z_{1}         2Z_{3}}
     |           |          |
     |           |          |
     o{-{-}{-}{-}Z_{0}π{-}{-}{-}{-}o{-}{-}{-}{-}Z_{0}π{-}{-}{-}o}
\end{verbatim}

\textbf{Derivation:}

\begin{enumerate}
\tightlist
\item
  For symmetrical π-network, admittance Y_{1} in shunt arms is split into 2
  equal parts (Y_{3} = Y_{1}/2)
\item
  For image impedance matching: Z_{0}π = Z_{0}π'
\end{enumerate}

By current division: I_{2}/I_{1} = Z_{0}π/(Z_{0}π + Z_{1} + Z_{0}π\textbar\textbar2Z_{3})

For matched condition: Z_{0}π^{2} = Z_{1}(2Z_{3})/(Z_{1} + 2Z_{3})

Simplifying: Z_{0}π = \sqrt[Z_{1}(2Z_{3})/(Z_{1} + 2Z_{3})] Z_{0}π = \sqrt[2Z_{1}Z_{3}/(Z_{1} +
2Z_{3})]

\end{solutionbox}
\begin{mnemonicbox}
``Pi's impedance equals the geometric mean of what it
sees''

\end{mnemonicbox}
\subsection*{Question 3(a) [3 marks]}\label{q3a}

\textbf{Explain principal of duality.}

\begin{solutionbox}

\textbf{Principle of Duality}: For every electrical network, there
exists a dual network with similar behavior but with interchanged
elements.


{\def\LTcaptype{none} % do not increment counter
\vspace{-5pt}
\captionof{table}{Dual Element Pairs}
\vspace{-10pt}
\begin{longtable}[]{@{}ll@{}}
\toprule\noalign{}
Original Circuit & Dual Circuit \\
\midrule\noalign{}
\endhead
\bottomrule\noalign{}
\endlastfoot
Voltage (V) & Current (I) \\
Current (I) & Voltage (V) \\
Resistance (R) & Conductance (G) \\
Inductance (L) & Capacitance (C) \\
Series Connection & Parallel Connection \\
KVL & KCL \\
Mesh Analysis & Nodal Analysis \\
\end{longtable}
}

\begin{itemize}
\tightlist
\item
  \textbf{Network Transformation}: Replace each element with its dual
\item
  \textbf{Topology Transformation}: Replace each node with a loop and
  each loop with a node
\end{itemize}

\end{solutionbox}
\begin{mnemonicbox}
``Series to Parallel, Source turns dual, V becomes I
and I becomes V''

\end{mnemonicbox}
\subsection*{Question 3(b) [4 marks]}\label{q3b}

\textbf{State and Explain Thevenin's Theorem.}

\begin{solutionbox}

\textbf{Thevenin's Theorem}: Any linear two-terminal network can be
replaced by an equivalent circuit consisting of a voltage source (Vth)
in series with a resistance (Rth).

\textbf{Diagram:}

\begin{center}
\textbf{Mermaid Diagram (Code)}
\begin{verbatim}
{Shaded}
{Highlighting}[]
graph TD
    subgraph "Original Network"
    direction LR
    A[Complex Network] {-{-}{-} R1[Load]}
    end
    subgraph "Thevenin Equivalent"
    direction LR
    VTH[Vth] {-{-}{-} RTH[Rth] {-}{-}{-} RL[Load]}
    end
{Highlighting}
{Shaded}
\end{verbatim}
\end{center}

\textbf{Finding Thevenin Equivalent:}

\begin{enumerate}
\tightlist
\item
  Remove the load resistance
\item
  Calculate open-circuit voltage (Vth)
\item
  Find Rth by:

  \begin{itemize}
  \tightlist
  \item
Deactivating all sources (V=0,

I=0)

  \item
    Calculate resistance between terminals
  \end{itemize}
\end{enumerate}

\end{solutionbox}
\begin{mnemonicbox}
``Open for Voltage, Dead for Resistance''

\end{mnemonicbox}
\subsection*{Question 3(c) [7 marks]}\label{q3c}

\textbf{State and explain KCL and KVL with example.}

\begin{solutionbox}


{\def\LTcaptype{none} % do not increment counter
\vspace{-5pt}
\captionof{table}{Kirchhoff's Laws}
\vspace{-10pt}
\begin{longtable}[]{@{}
  >{\raggedright\arraybackslash}p{(\linewidth - 6\tabcolsep) * \real{0.1042}}
  >{\raggedright\arraybackslash}p{(\linewidth - 6\tabcolsep) * \real{0.2292}}
  >{\raggedright\arraybackslash}p{(\linewidth - 6\tabcolsep) * \real{0.3958}}
  >{\raggedright\arraybackslash}p{(\linewidth - 6\tabcolsep) * \real{0.2708}}@{}}
\toprule\noalign{}
\begin{minipage}[b]{\linewidth}\raggedright
Law
\end{minipage} & \begin{minipage}[b]{\linewidth}\raggedright
Statement
\end{minipage} & \begin{minipage}[b]{\linewidth}\raggedright
Mathematical Form
\end{minipage} & \begin{minipage}[b]{\linewidth}\raggedright
Application
\end{minipage} \\
\midrule\noalign{}
\endhead
\bottomrule\noalign{}
\endlastfoot
KCL & Sum of currents entering a node equals sum of currents leaving it
& \sumIin = \sumIout & Node Analysis \\
KVL & Sum of voltage drops around any closed loop equals zero & \sumV = 0 &
Mesh Analysis \\
\end{longtable}
}

\textbf{KCL Example:}

\begin{verbatim}
        I_{1}
        ↓
        o
       / {}
      /   {}
     I_{2    I_{3}}
    /       {}
   o         o

At node: I_{1 = I_{2} + I_{3}}
\end{verbatim}

\textbf{KVL Example:}

\begin{verbatim}
    +    R_{1     +}
    o{-{-}{-}///{-}{-}o}
    |           |
   V_{1           R_{2}}
    |           |
    o{-{-}{-}{-}{-}{-}{-}{-}{-}{-}{-}o}
    {-           {-}}

Around loop: V_{1 {-} I_{1} {-} I_{2} = 0}
\end{verbatim}

\end{solutionbox}
\begin{mnemonicbox}
``Currents at nodes sum to zero, Voltages round loops
also do''

\end{mnemonicbox}
\subsection*{Question 3(a) OR [3
marks]}\label{q3a}

\textbf{Explain the solution of a network by Mesh Analysis.}

\begin{solutionbox}

\textbf{Mesh Analysis}: A circuit analysis method that uses mesh
currents as variables to solve for unknown currents and voltages.

\textbf{Diagram: Simple Two-Mesh Circuit}

\begin{verbatim}
    +   R_{1    +   R_{3}   +}
    o{-{-}//{-}{-}o{-}{-}//{-}{-}o}
    |         |         |
   V_{1        R_{2}        V_{2}}
    |         |         |
    o{-{-}{-}{-}{-}{-}{-}{-}{-}o{-}{-}{-}{-}{-}{-}{-}{-}{-}o}
    {-         {-}         {-}}
     Mesh 1     Mesh 2
\end{verbatim}

\textbf{Steps:}

\begin{enumerate}
\tightlist
\item
  Identify meshes (closed loops)
\item
  Assign clockwise mesh currents (I_{1}, I_{2})
\item
  Apply KVL to each mesh
\item
  Solve the resulting simultaneous equations
\end{enumerate}

\textbf{Example Equations:}

\begin{itemize}
\tightlist
\item
  Mesh 1: V_{1} = I_{1}(R_{1}+R_{2}) - I_{2}R_{2}
\item
  Mesh 2: -V_{2} = -I_{1}R_{2} + I_{2}(R_{2}+R_{3})
\end{itemize}

\end{solutionbox}
\begin{mnemonicbox}
``Assign, Apply KVL, Arrange, and Solve''

\end{mnemonicbox}
\subsection*{Question 3(b) OR [4
marks]}\label{q3b}

\textbf{State and Explain Norton's Theorem.}

\begin{solutionbox}

\textbf{Norton's Theorem}: Any linear two-terminal network can be
replaced by an equivalent circuit consisting of a current source (IN) in
parallel with a resistance (RN).

\textbf{Diagram:}

\begin{center}
\textbf{Mermaid Diagram (Code)}
\begin{verbatim}
{Shaded}
{Highlighting}[]
graph TD
    subgraph "Original Network"
    direction LR
    A[Complex Network] {-{-}{-} R1[Load]}
    end
    subgraph "Norton Equivalent"
    direction LR
    IN[In] {-.{-} RN[Rn] {-}{-}{-} RL[Load]}
    end
{Highlighting}
{Shaded}
\end{verbatim}
\end{center}

\textbf{Finding Norton Equivalent:}

\begin{enumerate}
\tightlist
\item
  Remove the load resistance
\item
  Calculate short-circuit current (IN)
\item
  Find RN by:

  \begin{itemize}
  \tightlist
  \item
Deactivating all sources (V=0,

I=0)

  \item
    Calculate resistance between terminals (RN = Rth)
  \end{itemize}
\end{enumerate}

\end{solutionbox}
\begin{mnemonicbox}
``Short for Current, Dead for Resistance''

\end{mnemonicbox}
\subsection*{Question 3(c) OR [7
marks]}\label{q3c}

\textbf{State and explain Maximum power transfer theorem. Derive
condition for maximum power transfer.}

\begin{solutionbox}

\textbf{Maximum Power Transfer Theorem}: A load receives maximum power
when its resistance equals the Thevenin equivalent resistance of the
network.

\textbf{Diagram:}

\begin{center}
\textbf{Mermaid Diagram (Code)}
\begin{verbatim}
{Shaded}
{Highlighting}[]
graph LR
    A[Vth] {-{-}{-} B[Rth] {-}{-}{-} C[RL]}
{Highlighting}
{Shaded}
\end{verbatim}
\end{center}

\textbf{Derivation:}

\begin{enumerate}
\item
  Power delivered to load: P = I^{2}RL
\item
  Current through circuit: I = Vth/(Rth + RL)
\item
  Substituting: P = Vth^{2}RL/(Rth + RL)^{2}
\item
  Differentiating with respect to RL and setting to zero: dP/dRL = 0
\item
  This gives: RL = Rth
\item
  Maximum power: Pmax = Vth^{2}/(4Rth)
\end{enumerate}

\end{solutionbox}
\begin{mnemonicbox}
``Match to maximize''

\end{mnemonicbox}
\subsection*{Question 4(a) [3 marks]}\label{q4a}

\textbf{Derive equation of Q factor for coil.}

\begin{solutionbox}

\textbf{Q Factor (Quality Factor)} for a coil represents the ratio of
inductive reactance to resistance.

\textbf{Diagram: Coil with Resistance}

\begin{verbatim}
    o{-{-}{-}{-}///{-}{-}{-}{-}uuuu{-}{-}{-}{-}o}
         R          L
\end{verbatim}

\textbf{Derivation:}

\begin{enumerate}
\tightlist
\item
  For an inductor with resistance, impedance Z = R + jωL
\item
  Q factor is defined as: Q = Reactive Power / Active Power
\item
  Q = ωL/R
\end{enumerate}

Where:

\begin{itemize}
\tightlist
\item
  L = Inductance in Henries
\item
  R = Series resistance in Ohms
\item
  ω = 2πf, Angular frequency
\end{itemize}

\end{solutionbox}
\begin{mnemonicbox}
``Quality equals Reactance over Resistance''

\end{mnemonicbox}
\subsection*{Question 4(b) [4 marks]}\label{q4b}

\textbf{Derive the formula for resonant frequency for a parallel RLC
circuit.}

\begin{solutionbox}

\textbf{Diagram: Parallel RLC Circuit}

\begin{verbatim}
        o{-{-}{-}{-}{-}o}
        |     |
        R     |
        |     |
        o     |
        |     |
        C     L
        |     |
        o{-{-}{-}{-}{-}o}
\end{verbatim}

\textbf{Derivation:}

\begin{enumerate}
\tightlist
\item
Admittance of parallel RLC:

Y = 1/R + jωC + 1/jωL = 1/R + j(ωC - 1/ωL)

\item
  At resonance, imaginary part is zero: ωC - 1/ωL = 0
\item
  Solving for ω: ω^{2} = 1/LC
\item
  Therefore: ω = 1/\sqrt(LC)
\item
  Resonance frequency: fr = 1/(2π\sqrt(LC))
\end{enumerate}

\textbf{Note:} R affects bandwidth but not resonance frequency.

\end{solutionbox}
\begin{mnemonicbox}
``One over Two Pi times Square Root of LC''

\end{mnemonicbox}
\subsection*{Question 4(c) [7 marks]}\label{q4c}

\textbf{Write types of coupled circuits with necessary diagram and
explain iron core transformer.}

\begin{solutionbox}


{\def\LTcaptype{none} % do not increment counter
\vspace{-5pt}
\captionof{table}{Types of Coupled Circuits}
\vspace{-10pt}
\begin{longtable}[]{@{}lll@{}}
\toprule\noalign{}
Type & Coupling Medium & Application \\
\midrule\noalign{}
\endhead
\bottomrule\noalign{}
\endlastfoot
Direct Coupling & Conductively connected & DC amplifiers \\
Capacitive Coupling & Capacitor & AC signal coupling \\
Inductive Coupling & Magnetic field & Transformers \\
Resistive Coupling & Resistor & Low-frequency signals \\
\end{longtable}
}

\textbf{Diagram: Iron Core Transformer}

\begin{center}
\textbf{Mermaid Diagram (Code)}
\begin{verbatim}
{Shaded}
{Highlighting}[]
graph LR
    subgraph "Primary"
    V1[V_{1] {-}{-}{-} L1[uuuu]}
    end

    subgraph "Iron Core"
    Core[" "]
    end
    
    subgraph "Secondary"
    L2[uuuu] {-{-}{-} V2[V_{2}]}
    end
    
    L1 {-{-}{-} Core {-}{-}{-} L2}
{Highlighting}
{Shaded}
\end{verbatim}
\end{center}

\textbf{Iron Core Transformer:}

\begin{itemize}
\tightlist
\item
  \textbf{Principle}: Mutual inductance through iron core
\item
  \textbf{Function}: Transfers energy between circuits by
  electromagnetic induction
\item
  \textbf{Coupling Coefficient}: k \approx 1 (near perfect coupling)
\item
  \textbf{Turns Ratio}: V_{2}/V_{1} = N_{2}/N_{1}
\item
  \textbf{Advantages}: High efficiency, good coupling
\end{itemize}

\end{solutionbox}
\begin{mnemonicbox}
``Primary excites, Core conducts, Secondary
delivers''

\end{mnemonicbox}
\subsection*{Question 4(a) OR [3
marks]}\label{q4a}

\textbf{Derive equation of Q factor for capacitor.}

\begin{solutionbox}

\textbf{Q Factor (Quality Factor)} for a capacitor represents the ratio
of capacitive reactance to resistance.

\textbf{Diagram: Capacitor with Resistance}

\begin{verbatim}
    o{-{-}{-}{-}///{-}{-}{-}{-}||{-}{-}{-}{-}o}
         R          C
\end{verbatim}

\textbf{Derivation:}

\begin{enumerate}
\tightlist
\item
  For a capacitor with series resistance, impedance Z = R - j/(ωC)
\item
  Q factor is defined as: Q = Reactive Power / Active Power
\item
  Q = 1/(ωCR)
\end{enumerate}

Where:

\begin{itemize}
\tightlist
\item
  C = Capacitance in Farads
\item
  R = Series resistance in Ohms
\item
  ω = 2πf, Angular frequency
\end{itemize}

\end{solutionbox}
\begin{mnemonicbox}
``Quality equals One over Resistance times
Reactance''

\end{mnemonicbox}
\subsection*{Question 4(b) OR [4
marks]}\label{q4b}

\textbf{Derive the equation of resonance frequency for a series
resonance circuit.}

\begin{solutionbox}

\textbf{Diagram: Series RLC Circuit}

\begin{verbatim}
    o{-{-}{-}{-}///{-}{-}{-}{-}uuuu{-}{-}{-}{-}||{-}{-}{-}{-}o}
         R          L       C
\end{verbatim}

\textbf{Derivation:}

\begin{enumerate}
\tightlist
\item
Impedance of series RLC:

Z = R + jωL - j/(ωC) = R + j(ωL - 1/ωC)

\item
  At resonance, imaginary part is zero: ωL - 1/ωC = 0
\item
  Solving for ω: ω^{2} = 1/LC
\item
  Therefore: ω = 1/\sqrt(LC)
\item
  Resonance frequency: fr = 1/(2π\sqrt(LC))
\end{enumerate}

\textbf{Key Points:}

\begin{itemize}
\tightlist
\item
  At resonance, impedance is purely resistive: Z = R
\item
  Circuit appears as a resistor
\item
  Current is maximum at resonance
\end{itemize}

\end{solutionbox}
\begin{mnemonicbox}
``One over Two Pi times Square Root of LC''

\end{mnemonicbox}
\subsection*{Question 4(c) OR [7
marks]}\label{q4c}

\textbf{Derive the Expression for coefficient coupling between pair of
magnetically coupled coils.}

\begin{solutionbox}

\textbf{Diagram: Magnetically Coupled Coils}

\begin{verbatim}
       uuuu   k   uuuu
    o{-{-}WWWW{-}{-}{-}{-}{-}{-}{-}WWWW{-}{-}o}
       L_{1          L_{2}    }
\end{verbatim}

\textbf{Derivation:}

\begin{enumerate}
\tightlist
\item
  Mutual inductance (M) relates to individual inductances by: M =
  k\sqrt(L_{1}L_{2})
\item
  Solving for k: k = M/\sqrt(L_{1}L_{2})
\end{enumerate}

Where:

\begin{itemize}
\tightlist
\item
  k = Coefficient of coupling (0 \leq k \leq 1)
\item
  M = Mutual inductance in Henries
\item
  L_{1}, L_{2} = Self-inductances of coils in Henries
\end{itemize}


{\def\LTcaptype{none} % do not increment counter
\vspace{-5pt}
\captionof{table}{Coupling Coefficient Values}
\vspace{-10pt}
\begin{longtable}[]{@{}lll@{}}
\toprule\noalign{}
Value of k & Coupling Type & Application \\
\midrule\noalign{}
\endhead
\bottomrule\noalign{}
\endlastfoot
k = 0 & No coupling & Separate circuits \\
0 \textless{} k \textless{} 0.5 & Loose coupling & RF transformers \\
0.5 \textless{} k \textless{} 1 & Tight coupling & Power transformers \\
k = 1 & Perfect coupling & Ideal transformer \\
\end{longtable}
}

\end{solutionbox}
\begin{mnemonicbox}
``Mutual over square root of product''

\end{mnemonicbox}
\subsection*{Question 5(a) [3 marks]}\label{q5a}

\textbf{Define Neper and dB. Establish relationship between Neper and
dB.}

\begin{solutionbox}


{\def\LTcaptype{none} % do not increment counter
\vspace{-5pt}
\captionof{table}{Neper and dB Definitions}
\vspace{-10pt}
\begin{longtable}[]{@{}
  >{\raggedright\arraybackslash}p{(\linewidth - 6\tabcolsep) * \real{0.1765}}
  >{\raggedright\arraybackslash}p{(\linewidth - 6\tabcolsep) * \real{0.3529}}
  >{\raggedright\arraybackslash}p{(\linewidth - 6\tabcolsep) * \real{0.2647}}
  >{\raggedright\arraybackslash}p{(\linewidth - 6\tabcolsep) * \real{0.2059}}@{}}
\toprule\noalign{}
\begin{minipage}[b]{\linewidth}\raggedright
Unit
\end{minipage} & \begin{minipage}[b]{\linewidth}\raggedright
Definition
\end{minipage} & \begin{minipage}[b]{\linewidth}\raggedright
Formula
\end{minipage} & \begin{minipage}[b]{\linewidth}\raggedright
Usage
\end{minipage} \\
\midrule\noalign{}
\endhead
\bottomrule\noalign{}
\endlastfoot
Neper (Np) & Natural logarithmic ratio & N = ln(V_{1}/V_{2}) or ln(I_{1}/I_{2}) &
Power system analysis \\
Decibel (dB) & Common logarithmic ratio & dB = 20log_{1}_{0}(V_{1}/V_{2}) or
10log_{1}_{0}(P_{1}/P_{2}) & Signal level measurement \\
\end{longtable}
}

\textbf{Relationship:}

\begin{enumerate}
\tightlist
\item
  N = ln(V_{1}/V_{2})
\item
  dB = 20log_{1}_{0}(V_{1}/V_{2})
\item
  Since ln(x) = 2.303 \times log_{1}_{0}(x)
\item
Therefore:

N = 2.303 \times dB/20 = 0.1152 \times dB

\item
  Conversely: dB = 8.686 \times N
\end{enumerate}

\end{solutionbox}
\begin{mnemonicbox}
``A Neper is 8.686 dB''

\end{mnemonicbox}
\subsection*{Question 5(b) [4 marks]}\label{q5b}

\textbf{Classify various types of Attenuators.}

\begin{solutionbox}


{\def\LTcaptype{none} % do not increment counter
\vspace{-5pt}
\captionof{table}{Types of Attenuators}
\vspace{-10pt}
\begin{longtable}[]{@{}
  >{\raggedright\arraybackslash}p{(\linewidth - 6\tabcolsep) * \real{0.1304}}
  >{\raggedright\arraybackslash}p{(\linewidth - 6\tabcolsep) * \real{0.2391}}
  >{\raggedright\arraybackslash}p{(\linewidth - 6\tabcolsep) * \real{0.3478}}
  >{\raggedright\arraybackslash}p{(\linewidth - 6\tabcolsep) * \real{0.2826}}@{}}
\toprule\noalign{}
\begin{minipage}[b]{\linewidth}\raggedright
Type
\end{minipage} & \begin{minipage}[b]{\linewidth}\raggedright
Structure
\end{minipage} & \begin{minipage}[b]{\linewidth}\raggedright
Characteristics
\end{minipage} & \begin{minipage}[b]{\linewidth}\raggedright
Applications
\end{minipage} \\
\midrule\noalign{}
\endhead
\bottomrule\noalign{}
\endlastfoot
T-type & Three resistors in T formation & Fixed impedance, good balance
& Signal level control \\
π-type (Pi) & Three resistors in π formation & Better isolation, more
common & RF signal attenuation \\
L-type & Two resistors in L formation & Simple, unbalanced & Basic level
adjustment \\
Bridged T & T with bridging resistor & Constant impedance & Audio
applications \\
Balanced & Symmetrical design & Good CMRR & Balanced transmission \\
Lattice & Diamond-shaped & Balanced, symmetrical & Telephone systems \\
\end{longtable}
}

\textbf{Diagram: Basic Attenuator Types}

\begin{center}
\textbf{Mermaid Diagram (Code)}
\begin{verbatim}
{Shaded}
{Highlighting}[]
graph TD
    subgraph "T{-type"}
    direction LR
    T1[o]{-{-}{-}TR1[R_{1}]{-}{-}{-}T2[o]}
    TR2[R_{2]}
    T2{-{-}{-}TR2{-}{-}{-}T3[o]}
    end

    subgraph "π{-type"}
    direction LR
    P1[o]{-{-}{-}PR1[R_{1}]{-}{-}{-}P2[o]}
    PR2[R_{2]}
    P1{-{-}{-}PR2}
    PR3[R_{3]}
    PR2{-{-}{-}P2}
    end
{Highlighting}
{Shaded}
\end{verbatim}
\end{center}

\end{solutionbox}
\begin{mnemonicbox}
``Tees, Pies and Ells attenuate the signals well''

\end{mnemonicbox}
\subsection*{Question 5(c) [7 marks]}\label{q5c}

\textbf{Determine the cut-off frequency and the nominal impedance of
each of the low-pass filter sections shown below.}

\begin{solutionbox}

\textbf{Diagram: Low-Pass Filter Sections}

\begin{verbatim}
    T{-section:                   π{-}section:}
    o{-{-}{-}L/2{-}{-}{-}o{-}{-}{-}L/2{-}{-}{-}o       o{-}{-}{-}{-}{-}{-}{-}{-}{-}{-}{-}{-}o{-}{-}{-}{-}{-}{-}{-}{-}{-}{-}{-}o}
             |                   |           |           |
             C                   C/2         |          C/2
             |                   |           |           |
    o{-{-}{-}{-}{-}{-}{-}{-}o{-}{-}{-}{-}{-}{-}{-}{-}{-}{-}{-}o      o{-}{-}{-}{-}{-}L{-}{-}{-}{-}{-}{-}o{-}{-}{-}{-}{-}{-}{-}{-}{-}{-}{-}o}
\end{verbatim}

\textbf{For T-section:}

\begin{itemize}
\tightlist
\item
  Cut-off frequency: fc = 1/(π\sqrt(LC))
\item
  Nominal impedance: R_{0} = \sqrt(L/C)
\item
Where

L = 10 mH,

C = 0.1 μF

\end{itemize}

Calculation: fc = 1/(π\sqrt(10\times10^{-}^{3} \times 0.1\times10^{-}^{6})) = 1/(π\sqrt(10^{-}^{9})) =
1/(π\times10^{-}^{4}·^{5}) = 3.18 kHz R_{0} = \sqrt(10\times10^{-}^{3}/0.1\times10^{-}^{6}) = \sqrt(10^{5}) = 316.23 Ω

\textbf{For π-section:}

\begin{itemize}
\tightlist
\item
  Cut-off frequency: fc = 1/(π\sqrt(LC))
\item
  Nominal impedance: R_{0} = \sqrt(L/C)
\item
  Same values as T-section
\end{itemize}

\end{solutionbox}
\begin{mnemonicbox}
``Cut-off frequency is inverse to the square root of
LC''

\end{mnemonicbox}
\subsection*{Question 5(a) OR [3
marks]}\label{q5a}

\textbf{Explain the limitation of constant k type filters.}

\begin{solutionbox}


{\def\LTcaptype{none} % do not increment counter
\vspace{-5pt}
\captionof{table}{Limitations of Constant-k Filters}
\vspace{-10pt}
\begin{longtable}[]{@{}
  >{\raggedright\arraybackslash}p{(\linewidth - 4\tabcolsep) * \real{0.3636}}
  >{\raggedright\arraybackslash}p{(\linewidth - 4\tabcolsep) * \real{0.3939}}
  >{\raggedright\arraybackslash}p{(\linewidth - 4\tabcolsep) * \real{0.2424}}@{}}
\toprule\noalign{}
\begin{minipage}[b]{\linewidth}\raggedright
Limitation
\end{minipage} & \begin{minipage}[b]{\linewidth}\raggedright
Description
\end{minipage} & \begin{minipage}[b]{\linewidth}\raggedright
Effect
\end{minipage} \\
\midrule\noalign{}
\endhead
\bottomrule\noalign{}
\endlastfoot
Impedance Matching & Impedance varies with frequency & Signal
reflection, power loss \\
Attenuation Band & Gradual transition at cut-off & Poor frequency
selectivity \\
Phase Response & Non-linear phase characteristic & Signal distortion \\
Passband Ripple & Non-uniform response in passband & Signal amplitude
variation \\
Roll-off Rate & Slow roll-off (20 dB/decade) & Poor stop-band
rejection \\
\end{longtable}
}

\begin{itemize}
\tightlist
\item
  \textbf{Main issue}: Poor transition from pass to stop band
\item
  \textbf{Improvement}: Using m-derived filters
\end{itemize}

\end{solutionbox}
\begin{mnemonicbox}
``Poor Matching And Transition Results In
Distortion''

\end{mnemonicbox}
\subsection*{Question 5(b) OR [4
marks]}\label{q5b}

\textbf{Derive equation of cut-off frequency for T-type Constant-k high
Pass filter.}

\begin{solutionbox}

\textbf{Diagram: T-type Constant-k High Pass Filter}

\begin{verbatim}
    o{-{-}{-}C/2{-}{-}{-}o{-}{-}{-}{-}C/2{-}{-}{-}{-}{-}{-}o}
              |               
              L
              |
    o{-{-}{-}{-}{-}{-}{-}{-}{-}o{-}{-}{-}{-}{-}{-}{-}{-}{-}{-}{-}{-}{-}o}
\end{verbatim}

\textbf{Derivation:}

\begin{enumerate}
\tightlist
\item
  For high-pass filter, series elements are capacitors and shunt
  elements are inductors
\item
  Transfer function: H(jω) = Z_{2}/(Z_{1} + Z_{2})
\item
  Where Z_{1} = 1/(jωC) and Z_{2} = jωL
\item
  Impedance condition for cut-off: Z_{1}/Z_{2} = 4 or Z_{1}/4Z_{2} = 1
\item
  Substituting: 1/(jωC) = 4jωL
\item
  Solving for ω: ω^{2} = 1/(4LC)
\item
  Cut-off frequency: fc = 1/(4π\sqrt(LC))
\end{enumerate}

\end{solutionbox}
\begin{mnemonicbox}
``High Pass cuts frequencies below one over four pi
L-C''

\end{mnemonicbox}
\subsection*{Question 5(c) OR [7
marks]}\label{q5c}

\textbf{Give classification of filters using definitions and
characteristics graphs for each.}

\begin{solutionbox}


{\def\LTcaptype{none} % do not increment counter
\vspace{-5pt}
\captionof{table}{Filter Classification}
\vspace{-10pt}
\begin{longtable}[]{@{}
  >{\raggedright\arraybackslash}p{(\linewidth - 6\tabcolsep) * \real{0.3023}}
  >{\raggedright\arraybackslash}p{(\linewidth - 6\tabcolsep) * \real{0.1860}}
  >{\raggedright\arraybackslash}p{(\linewidth - 6\tabcolsep) * \real{0.1860}}
  >{\raggedright\arraybackslash}p{(\linewidth - 6\tabcolsep) * \real{0.3256}}@{}}
\toprule\noalign{}
\begin{minipage}[b]{\linewidth}\raggedright
Filter Type
\end{minipage} & \begin{minipage}[b]{\linewidth}\raggedright
Passes
\end{minipage} & \begin{minipage}[b]{\linewidth}\raggedright
Blocks
\end{minipage} & \begin{minipage}[b]{\linewidth}\raggedright
Applications
\end{minipage} \\
\midrule\noalign{}
\endhead
\bottomrule\noalign{}
\endlastfoot
Low-Pass & Frequencies below fc & Frequencies above fc & Audio
amplifiers, power supplies \\
High-Pass & Frequencies above fc & Frequencies below fc & Noise
elimination, treble control \\
Band-Pass & Range between fL and fH & Frequencies outside range & Radio
tuning, equalizers \\
Band-Stop & Frequencies outside range & Range between fL and fH & Noise
elimination, notch filters \\
All-Pass & All frequencies with unity gain & None (changes only phase) &
Phase correction, time delay \\
\end{longtable}
}

\textbf{Characteristic Response Graphs:}

\begin{center}
\textbf{Mermaid Diagram (Code)}
\begin{verbatim}
{Shaded}
{Highlighting}[]
graph TD
    subgraph "Low{-Pass"}
    LP[High{br/{}│{}br/{}Gain{}br/{}│{}br/{}Low] {-}{-}{-} LPf[Frequency ]}

    style LP stroke{-width:0, fill:\#fff}
    style LPf stroke{-width:0, fill:\#fff}
    end
    
    subgraph "High{-Pass"}
    HP[High{br/{}│{}br/{}Gain{}br/{}│{}br/{}Low] {-}{-}{-} HPf[Frequency ]}
    
    style HP stroke{-width:0, fill:\#fff}
    style HPf stroke{-width:0, fill:\#fff}
    end
    
    subgraph "Band{-Pass"}
    BP[High{br/{}│{}br/{}Gain{}br/{}│{}br/{}Low] {-}{-}{-} BPf[Frequency ]}
    
    style BP stroke{-width:0, fill:\#fff}
    style BPf stroke{-width:0, fill:\#fff}
    end
    
    subgraph "Band{-Stop"}
    BS[High{br/{}│{}br/{}Gain{}br/{}│{}br/{}Low] {-}{-}{-} BSf[Frequency ]}
    
    style BS stroke{-width:0, fill:\#fff}
    style BSf stroke{-width:0, fill:\#fff}
    end
{Highlighting}
{Shaded}
\end{verbatim}
\end{center}

\textbf{Filter Implementations:}

\begin{itemize}
\tightlist
\item
  \textbf{Passive}: Uses R, L, C components
\item
  \textbf{Active}: Uses op-amps with RC networks
\item
  \textbf{Digital}: Uses DSP algorithms
\end{itemize}

\end{solutionbox}
\begin{mnemonicbox}
``Low-High-Band-Stop makes Signals Perfect''

\end{mnemonicbox}

\end{document}
