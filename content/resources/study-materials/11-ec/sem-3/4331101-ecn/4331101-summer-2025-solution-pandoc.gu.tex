\documentclass[10pt,a4paper]{article}

% content/resources/templates/preamble.tex
\usepackage[margin=0.6in]{geometry}
\author{Milav Dabgar}
\usepackage{amsmath,amssymb,amsthm}
\usepackage{booktabs}
\usepackage{multirow}
\usepackage{xcolor}
\usepackage{tcolorbox}
\tcbuselibrary{breakable,skins}
\usepackage[colorlinks=true,linkcolor=blue]{hyperref}
\usepackage{titlesec}
\usepackage{enumitem}
\usepackage{tikz}
\usepackage{pgfplots}
\usepackage{circuitikz}
\usepackage[version=4]{mhchem}
\usepackage{longtable}
\usepackage{array}
\usepackage{float}
\usepackage{caption}
\usepackage{listings}

\lstset{
  basicstyle=\small\ttfamily,
  breaklines=true,
  breakatwhitespace=false,
  postbreak=\mbox{\textcolor{red}{$\hookrightarrow$}\space},
  float=false,
  numbers=left,
  numberstyle=\tiny\color{gray},
  numbersep=10pt,
  xleftmargin=2em,
  keywordstyle=\color{blue},
  commentstyle=\color{green!60!black},
  stringstyle=\color{purple},
  backgroundcolor=\color{gray!5},
  showstringspaces=false,
  tabsize=2,
  captionpos=b,
  keepspaces=true,
  columns=flexible
}

\pgfplotsset{compat=1.18}
\usetikzlibrary{shapes,arrows,positioning,calc,patterns,decorations.pathmorphing,decorations.markings,arrows.meta}

% Color scheme
\definecolor{headcolor}{RGB}{0,102,204}
\definecolor{keycolor}{RGB}{220,20,60}
\definecolor{solutioncolor}{RGB}{34,139,34}
\definecolor{mnemoniccolor}{RGB}{148,0,211}
\definecolor{codecolor}{RGB}{0,0,100}

% Spacing
\setlength{\parskip}{3pt}
\setlist[itemize]{nosep}
\setlist[enumerate]{nosep}

% Title formatting
\titleformat{\section}{\Large\bfseries\color{headcolor}}{\thesection}{1em}{}
\titleformat{\subsection}{\large\bfseries\color{headcolor}}{\thesubsection}{1em}{}

% Pandoc tightlist compatibility
\providecommand{\tightlist}{%
  \setlength{\itemsep}{0pt}\setlength{\parskip}{0pt}}

% Pandoc longtable compatibility
\newcounter{none}
\def\thenone{}


% content/resources/templates/gujarati-boxes.tex
\usepackage{fontspec}
\usepackage{polyglossia}

% Set Gujarati as main language (document is primarily in Gujarati)
% Note: gloss-gujarati.ldf doesn't exist in polyglossia, but it will use hyphenation patterns
\setdefaultlanguage{gujarati}
\setotherlanguage{english}

% Configure Gujarati font properly
% Use Language=Default to prevent polyglossia from trying to add language-specific features
% that don't exist for Gujarati, which causes "empty feature" warnings
\newfontfamily\gujaratifont[Script=Gujarati,AutoFakeBold=2.5,AutoFakeSlant=0.3]{Noto Sans Gujarati}
\setmainfont[Script=Gujarati,AutoFakeBold=2.5,AutoFakeSlant=0.3]{Noto Sans Gujarati}
% Use Noto Sans Gujarati for monospace to support Gujarati in text
\setmonofont[Scale=0.9]{Noto Sans Gujarati}

% Configure English to use the same font
\newfontfamily\englishfont[Script=Gujarati,AutoFakeBold=2.5,AutoFakeSlant=0.3]{Noto Sans Gujarati}

% Translations for polyglossia
\gappto\captionsgujarati{
  \renewcommand{\tablename}{કોષ્ટક}
  \renewcommand{\figurename}{આકૃતિ}
}

% Helper for TikZ nodes to ensure Gujarati font
\newcommand{\gu}[1]{{\gujaratifont #1}}

% Custom environments
\newtcolorbox{solutionbox}{
    breakable,
    enhanced,
    colback=solutioncolor!5!white,
    colframe=solutioncolor!75!black,
    fonttitle=\bfseries,
    title=જવાબ
}

\newtcolorbox{solutionboxnobreak}{
 colback=solutioncolor!5!white,
 colframe=solutioncolor!75!black,
 fonttitle=\bfseries,
 title=જવાબ
}

\newtcolorbox{keyformula}{
 breakable,
 enhanced,
 colback=keycolor!5!white,
 colframe=keycolor!75!black,
 fonttitle=\bfseries,
 title=રાસાયણિક સમીકરણ/સૂત્ર
}

\newtcolorbox{mnemonicbox}{
 breakable,
 enhanced,
 colback=mnemoniccolor!5!white,
 colframe=mnemoniccolor!75!black,
 fonttitle=\bfseries,
 title=મેમરી ટ્રીક
}


\begin{document}

\begin{center}
{\Huge\bfseries\color{headcolor} Subject Name (Gujarati)}\\[5pt]
{\LARGE 4331101 -- Summer 2025}\\[3pt]
{\large Semester 1 Study Material}\\[3pt]
{\normalsize\textit{Detailed Solutions and Explanations}}
\end{center}

\vspace{10pt}

\subsection*{પ્રશ્ન 1(a) [3
ગુણ]}\label{q1a}

\textbf{નીચેના શબ્દો વ્યાખ્યાયિત કરો. (i) એકટીવ એલિમેન્ટસ (ii) બાયલેટરલ એલિમેન્ટસ
(iii) લિનિયર એલિમેંટ્સ}

\begin{solutionbox}

{\def\LTcaptype{none} % do not increment counter
\begin{longtable}[]{@{}
  >{\raggedright\arraybackslash}p{(\linewidth - 2\tabcolsep) * \real{0.3333}}
  >{\raggedright\arraybackslash}p{(\linewidth - 2\tabcolsep) * \real{0.6667}}@{}}
\toprule\noalign{}
\begin{minipage}[b]{\linewidth}\raggedright
શબ્દ
\end{minipage} & \begin{minipage}[b]{\linewidth}\raggedright
વ્યાખ્યા
\end{minipage} \\
\midrule\noalign{}
\endhead
\bottomrule\noalign{}
\endlastfoot
\textbf{એકટીવ એલિમેન્ટસ} & એલેક્ટ્રોનિક ઘટકો જે સર્કિટમાં ઊર્જા અથવા પાવર આપી શકે
છે (જેમ કે બેટરી, જનરેટર, ઓપ-એમ્પ) \\
\textbf{બાયલેટરલ એલિમેન્ટસ} & ઘટકો જે બંને દિશામાં સમાન લાક્ષણિકતાઓ સાથે કરંટને
સરખી રીતે વહેવા દે છે (જેમ કે રેસિસ્ટર, કેપેસિટર, ઇન્ડક્ટર) \\
\textbf{લિનિયર એલિમેંટ્સ} & ઘટકો જેમનો કરંટ-વોલ્ટેજ સંબંધ સીધી લાઇનનું અનુસરણ કરે છે
અને સુપરપોઝિશનના સિદ્ધાંતનું પાલન કરે છે (જેમ કે ઓહ્મના નિયમનું અનુસરણ કરતા રેસિસ્ટર) \\
\end{longtable}
}

\end{solutionbox}
\begin{mnemonicbox}
``ABL: Active powers Batteries, Bilateral flows Both
ways, Linear stays Lawful''

\end{mnemonicbox}
\subsection*{પ્રશ્ન 1(b) [4
ગુણ]}\label{q1b}

\textbf{10µf, 20 µf અને 30µf ના કેપેસિટર શ્રેણીમાં જોડાયેલા છે અને 200 V DCનો
પુરવઠો આપવામાં આવે છે. દરેક કેપેસિટરમાં વોલ્ટેજ શોધો.}

\begin{solutionbox}

શ્રેણીમાં જોડાયેલા કેપેસિટર માટે:

\begin{enumerate}
\tightlist
\item
  સમતુલ્ય કેપેસિટન્સ શોધો: 1/Ceq = 1/C_{1} + 1/C_{2} + 1/C_{3}
\item
  વોલ્ટેજ વિભાજન: VC = (C_{1}/C) \times V
\end{enumerate}

\textbf{ગણતરી:} 1/Ceq = 1/10 + 1/20 + 1/30 = 0.1 + 0.05 + 0.033 = 0.183
Ceq = 5.46 μF

{\def\LTcaptype{none} % do not increment counter
\begin{longtable}[]{@{}llll@{}}
\toprule\noalign{}
કેપેસિટર & સૂત્ર & ગણતરી & વોલ્ટેજ \\
\midrule\noalign{}
\endhead
\bottomrule\noalign{}
\endlastfoot
C_{1} = 10μF & V_{1} = (Ceq/C_{1}) \times V & (5.46/10) \times 200 = 109.2V & 109.2V \\
C_{2} = 20μF & V_{2} = (Ceq/C_{2}) \times V & (5.46/20) \times 200 = 54.6V & 54.6V \\
C_{3} = 30μF & V_{3} = (Ceq/C_{3}) \times V & (5.46/30) \times 200 = 36.4V & 36.4V \\
\end{longtable}
}

\end{solutionbox}
\begin{mnemonicbox}
``નાના કેપેસિટરમાં મોટો વોલ્ટેજ મળે''

\end{mnemonicbox}
\subsection*{પ્રશ્ન 1(c) [7
ગુણ]}\label{q1c}

\textbf{ગ્રાફ થિયરી માટે નોડ પેર વોલ્ટેજ પદ્ધતિ સમજાવો.}

\begin{solutionbox}

નોડ પેર વોલ્ટેજ પદ્ધતિ એ ઇલેક્ટ્રિકલ નેટવર્ક્સનું વિશ્લેષણ કરવા માટેની પદ્ધતિસરની પદ્ધતિ
છે.

\textbf{પ્રક્રિયા:}

\begin{enumerate}
\tightlist
\item
  સંદર્ભ નોડ પસંદ કરો (ગ્રાઉન્ડ)
\item
  નોડ વોલ્ટેજને ઓળખો (N નોડ માટે N-1 અજ્ઞાત)
\item
  દરેક બિન-સંદર્ભ નોડ પર KCL લાગુ કરો
\item
  નોડ વોલ્ટેજના સંદર્ભમાં શાખા કરંટ વ્યક્ત કરો
\item
  નોડ વોલ્ટેજ માટે સમીકરણોનો ઉકેલ કરો
\end{enumerate}

\textbf{આકૃતિ:}

\begin{center}
\textbf{Mermaid Diagram (Code)}
\begin{verbatim}
{Shaded}
{Highlighting}[]
graph LR
    A[સંદર્ભ નોડ પસંદ કરો] {-{-}{} B[નોડ વોલ્ટેજ ઓળખો]}
    B {-{-}{} C[દરેક નોડ પર KCL લાગુ કરો]}
    C {-{-}{} D[નોડ વોલ્ટેજનો ઉપયોગ કરીને શાખા કરંટ વ્યક્ત કરો]}
    D {-{-}{} E[નોડ વોલ્ટેજ માટે સમીકરણો ઉકેલો]}
    E {-{-}{} F[શાખા કરંટની ગણતરી કરો]}
{Highlighting}
{Shaded}
\end{verbatim}
\end{center}

\textbf{મુખ્ય ફાયદા:}

\begin{itemize}
\tightlist
\item
  \textbf{ઓછા સમીકરણો}: n નોડ માટે ફક્ત (n-1) સમીકરણો
\item
  \textbf{કમ્પ્યુટેશનલ કાર્યક્ષમતા}: સિસ્ટમની જટિલતા ઘટાડે છે
\item
  \textbf{સીધા વોલ્ટેજ ઉકેલ}: સીધા નોડ વોલ્ટેજ પ્રદાન કરે છે
\item
  \textbf{પદ્ધતિસરનો અભિગમ}: કોઈપણ નેટવર્ક ટોપોલોજી માટે કામ કરે છે
\end{itemize}

\end{solutionbox}
\begin{mnemonicbox}
``GARCS: Ground, Assign voltages, Relate with KCL,
Calculate currents, Solve equations''

\end{mnemonicbox}
\subsection*{પ્રશ્ન 1(c) OR [7
ગુણ]}\label{q1c}

\textbf{જરૂરી સમીકરણો સાથે વોલ્ટેજ વિભાજન પદ્ધતિ સમજાવો.}

\begin{solutionbox}

વોલ્ટેજ વિભાજન એ શ્રેણી ઘટકોમાં વોલ્ટેજ કેવી રીતે વિતરિત થાય છે તે ગણવાની એક પદ્ધતિ
છે.

\textbf{સિદ્ધાંત:} શ્રેણી સર્કિટમાં, વોલ્ટેજ ઘટક પ્રતિરોધ/ઇમ્પીડન્સના પ્રમાણમાં
વિભાજિત થાય છે.

\textbf{સૂત્ર:} કુલ પ્રતિરોધ RT સાથે શ્રેણી સર્કિટમાં એક પ્રતિરોધ R_{1} માટે: V_{1} =
(R_{1}/RT) \times VS

\textbf{આકૃતિ:}

\begin{verbatim}
      +{-{-}{-}+}
VS {-{-}{-}|   |{-}{-} R1 {-}{-}|}
      +{-{-}{-}+        |}
                   |  V1
                   |
                   |
      +{-{-}{-}+        |}
      |   |{-{-} R2 {-}{-}|}
      +{-{-}{-}+}
        |
        |
      {-{-}{-}{-}{-}}
       {-{-}{-}}
        {-}
\end{verbatim}

\textbf{ગાણિતિક સમજૂતી:}

\begin{itemize}
\tightlist
\item
  પ્રતિરોધક માટે: V_{1} = (R_{1}/RT) \times VS
\item
  કેપેસિટર માટે: V_{1} = (1/C_{1})/(1/CT) \times VS = (CT/C_{1}) \times VS
\item
  ઇન્ડક્ટર માટે: V_{1} = (L_{1}/LT) \times VS
\item
  જટિલ ઇમ્પીડન્સ માટે: V_{1} = (Z_{1}/ZT) \times VS
\end{itemize}

\textbf{ઉદાહરણો:}

\begin{enumerate}
\tightlist
\item
  5V સ્ત્રોત સાથે 4kΩ ની શ્રેણીમાં 1kΩ પ્રતિરોધક પર વોલ્ટેજ = (1/5)\times5V = 1V
\item
  10V સ્ત્રોત સાથે 40μF ની શ્રેણીમાં 10μF કેપેસિટર પર વોલ્ટેજ = (1/10)/(1/8)\times10V =
  8V
\end{enumerate}

\end{solutionbox}
\begin{mnemonicbox}
``જેટલો મોટો પ્રતિરોધ, તેટલો મોટો વોલ્ટેજ ડ્રોપ''

\end{mnemonicbox}
\subsection*{પ્રશ્ન 2(a) [3
ગુણ]}\label{q2a}

\textbf{ટુ પોર્ટ નેટવર્કના ઓપન સર્કિટ ઈમ્પીડેન્સ પેરામીટર્સ લખો.}

\begin{solutionbox}

\textbf{ઓપન સર્કિટ ઈમ્પીડેન્સ પેરામીટર્સ:}

{\def\LTcaptype{none} % do not increment counter
\begin{longtable}[]{@{}
  >{\raggedright\arraybackslash}p{(\linewidth - 4\tabcolsep) * \real{0.3235}}
  >{\raggedright\arraybackslash}p{(\linewidth - 4\tabcolsep) * \real{0.2941}}
  >{\raggedright\arraybackslash}p{(\linewidth - 4\tabcolsep) * \real{0.3824}}@{}}
\toprule\noalign{}
\begin{minipage}[b]{\linewidth}\raggedright
પેરામીટર
\end{minipage} & \begin{minipage}[b]{\linewidth}\raggedright
સમીકરણ
\end{minipage} & \begin{minipage}[b]{\linewidth}\raggedright
ભૌતિક અર્થ
\end{minipage} \\
\midrule\noalign{}
\endhead
\bottomrule\noalign{}
\endlastfoot
\textbf{Z_{1}_{1}} & Z_{1}_{1} = V_{1}/I_{1} (જ્યારે I_{2}=0) & આઉટપુટ ઓપન-સર્કિટેડ હોય ત્યારે ઇનપુટ
ઇમ્પીડન્સ \\
\textbf{Z_{1}_{2}} & Z_{1}_{2} = V_{1}/I_{2} (જ્યારે I_{1}=0) & પોર્ટ 2 થી પોર્ટ 1 સુધી ટ્રાન્સફર
ઇમ્પીડન્સ \\
\textbf{Z_{2}_{1}} & Z_{2}_{1} = V_{2}/I_{1} (જ્યારે I_{2}=0) & પોર્ટ 1 થી પોર્ટ 2 સુધી ટ્રાન્સફર
ઇમ્પીડન્સ \\
\textbf{Z_{2}_{2}} & Z_{2}_{2} = V_{2}/I_{2} (જ્યારે I_{1}=0) & ઇનપુટ ઓપન-સર્કિટેડ હોય ત્યારે આઉટપુટ
ઇમ્પીડન્સ \\
\end{longtable}
}

\end{solutionbox}
\begin{mnemonicbox}
``ZIPO: Z-parameters with Inputs and outputs, Ports
Open where needed''

\end{mnemonicbox}
\subsection*{પ્રશ્ન 2(b) [4
ગુણ]}\label{q2b}

\textbf{ટી-ટાઈપ નેટવર્કમાંથી \prod-પ્રકાર નેટવર્કમાં રૂપાંતરણ મેળવો.}

\begin{solutionbox}

\textbf{T થી \prod નેટવર્ક રૂપાંતરણ:}

\textbf{આકૃતિ:}

\begin{verbatim}
   T{-Network           -Network}
      Z1                   Y1
  o{-{-}{-}///{-}{-}{-}o       o{-}{-}{-}///{-}{-}{-}o}
  |           |       |           |
  |           |       |           |
 Z3          Z2      Y3          Y2
  |           |       |           |
  |           |       |           |
  o{-{-}{-}{-}{-}{-}{-}{-}{-}{-}{-}o       o{-}{-}{-}{-}{-}{-}{-}{-}{-}{-}{-}o}
\end{verbatim}

\textbf{રૂપાંતરણ સમીકરણો:}

{\def\LTcaptype{none} % do not increment counter
\begin{longtable}[]{@{}
  >{\raggedright\arraybackslash}p{(\linewidth - 4\tabcolsep) * \real{0.2955}}
  >{\raggedright\arraybackslash}p{(\linewidth - 4\tabcolsep) * \real{0.2045}}
  >{\raggedright\arraybackslash}p{(\linewidth - 4\tabcolsep) * \real{0.5000}}@{}}
\toprule\noalign{}
\begin{minipage}[b]{\linewidth}\raggedright
\prod-પેરામીટર
\end{minipage} & \begin{minipage}[b]{\linewidth}\raggedright
સૂત્ર
\end{minipage} & \begin{minipage}[b]{\linewidth}\raggedright
T-પેરામીટર્સ પર આધારિત
\end{minipage} \\
\midrule\noalign{}
\endhead
\bottomrule\noalign{}
\endlastfoot
Y_{1} = 1/Z_{1} & Y_{1} = Z_{2}/(Z_{1}Z_{2}+Z_{2}Z_{3}+Z_{3}Z_{1}) & નેટવર્ક દ્વારા સંશોધિત Z_{1}નો
રેસિપ્રોકલ \\
Y_{2} = 1/Z_{2} & Y_{2} = Z_{1}/(Z_{1}Z_{2}+Z_{2}Z_{3}+Z_{3}Z_{1}) & નેટવર્ક દ્વારા સંશોધિત Z_{2}નો
રેસિપ્રોકલ \\
Y_{3} = 1/Z_{3} & Y_{3} = Z_{3}/(Z_{1}Z_{2}+Z_{2}Z_{3}+Z_{3}Z_{1}) & નેટવર્ક દ્વારા સંશોધિત Z_{3}નો
રેસિપ્રોકલ \\
\end{longtable}
}

\textbf{ડેરિવેશન સ્ટેપ્સ:}

\begin{enumerate}
\tightlist
\item
  ડિટર્મિનન્ટ Δ = Z_{1}Z_{2}+Z_{2}Z_{3}+Z_{3}Z_{1} વ્યાખ્યાયિત કરો
\item
  નેટવર્ક થિયરી વાપરીને Y_{1} = Z_{2}/Δ તારવો
\item
  તે જ રીતે, Y_{2} = Z_{1}/Δ
\item
  અને Y_{3} = Z_{3}/Δ
\end{enumerate}

\end{solutionbox}
\begin{mnemonicbox}
``ડેલ્ટા ડિવાઇડ: Y_{1}ને Z_{2} મળે, Y_{2}ને Z_{1} મળે, Y_{3}ને Z_{3} મળે''

\end{mnemonicbox}
\subsection*{પ્રશ્ન 2(c) [7
ગુણ]}\label{q2c}

\textbf{ડેલ્ટામાં 1, 1 અને 1 ઓહ્મના ત્રણ રેસીસ્ટર જોડાયેલા છે. સમકક્ષ સ્ટાર નેટવર્ક
શોધો.}

\begin{solutionbox}

\textbf{ડેલ્ટા થી સ્ટાર રૂપાંતરણ:}

\textbf{આકૃતિ:}

\begin{verbatim}
   Delta Network           Star Network
       R1                      ra
   o{-{-}{-}///{-}{-}o           o{-}{-}{-}///{-}{-}{-}o}
   |          |           |           |
   |          |           |           |
   |          |           |           |
   |          |          rb           rc
   |          |           |           |
  R3         R2           |           |
   |          |           |           |
   |          |           o{-{-}{-}{-}{-}{-}{-}{-}{-}{-}{-}o}
   o{-{-}{-}{-}{-}{-}{-}{-}{-}{-}o}
\end{verbatim}

\textbf{રૂપાંતરણ સૂત્રો:}

\begin{itemize}
\tightlist
\item
  ra = (R_{1}\timesR_{3})/(R_{1}+R_{2}+R_{3})
\item
  rb = (R_{1}\timesR_{2})/(R_{1}+R_{2}+R_{3})
\item
  rc = (R_{2}\timesR_{3})/(R_{1}+R_{2}+R_{3})
\end{itemize}

\textbf{ગણતરી:} આપેલું: R_{1} = R_{2} = R_{3} = 1Ω પ્રતિરોધનો સરવાળો: R_{1}+R_{2}+R_{3} = 3Ω

{\def\LTcaptype{none} % do not increment counter
\begin{longtable}[]{@{}llll@{}}
\toprule\noalign{}
સ્ટાર પ્રતિરોધક & સૂત્ર & ગણતરી & પરિણામ \\
\midrule\noalign{}
\endhead
\bottomrule\noalign{}
\endlastfoot
ra & (R_{1}\timesR_{3})/(R_{1}+R_{2}+R_{3}) & (1\times1)/3 & 0.333Ω \\
rb & (R_{1}\timesR_{2})/(R_{1}+R_{2}+R_{3}) & (1\times1)/3 & 0.333Ω \\
rc & (R_{2}\timesR_{3})/(R_{1}+R_{2}+R_{3}) & (1\times1)/3 & 0.333Ω \\
\end{longtable}
}

\end{solutionbox}
\begin{mnemonicbox}
``પ્રોડક્ટ ઓવર સમ: દરેક સ્ટાર આર્મને નજીકના ડેલ્ટા બાજુઓના
ગુણાકારને બધાના સરવાળા વડે ભાગવાથી મળે છે''

\end{mnemonicbox}
\subsection*{પ્રશ્ન 2(a) OR [3
ગુણ]}\label{q2a}

\textbf{વ્યાખ્યાયિત કરો. (i) ટ્રાન્સફર ઇમ્પીડન્સ (ii) ઇમેજ ઇમ્પીડન્સ (iii)
ડ્રાઇવિંગ પોઈન્ટ ઇમ્પીડન્સ}

\begin{solutionbox}

{\def\LTcaptype{none} % do not increment counter
\begin{longtable}[]{@{}
  >{\raggedright\arraybackslash}p{(\linewidth - 2\tabcolsep) * \real{0.3333}}
  >{\raggedright\arraybackslash}p{(\linewidth - 2\tabcolsep) * \real{0.6667}}@{}}
\toprule\noalign{}
\begin{minipage}[b]{\linewidth}\raggedright
શબ્દ
\end{minipage} & \begin{minipage}[b]{\linewidth}\raggedright
વ્યાખ્યા
\end{minipage} \\
\midrule\noalign{}
\endhead
\bottomrule\noalign{}
\endlastfoot
\textbf{ટ્રાન્સફર ઇમ્પીડન્સ} & એક પોર્ટ પર આઉટપુટ વોલ્ટેજનો બીજા પોર્ટ પર ઈનપુટ
કરંટના ગુણોત્તર જ્યારે અન્ય બધા પોર્ટ ઓપન-સર્કિટેડ હોય (Z_{2}_{1} = V_{2}/I_{1} જ્યારે I_{2}=0) \\
\textbf{ઇમેજ ઇમ્પીડન્સ} & જ્યારે આઉટપુટ પોર્ટ તેના પોતાના ઇમેજ ઇમ્પીડન્સ સાથે ટર્મિનેટ
કરવામાં આવે ત્યારે પોર્ટ પર ઇનપુટ ઇમ્પીડન્સ, જે તમામ પોઇન્ટ્સ પર સમાન ઇમ્પીડન્સ સાથે
અનંત ચેઇન બનાવે છે \\
\textbf{ડ્રાઇવિંગ પોઈન્ટ ઇમ્પીડન્સ} & જ્યારે નિર્દિષ્ટ પોર્ટ અથવા ટર્મિનલ જોડીમાં
જોતા હોઈએ ત્યારે દેખાતી ઇનપુટ ઇમ્પીડન્સ (Z_{1}_{1} = V_{1}/I_{1} પોર્ટ 1 માટે) \\
\end{longtable}
}

\end{solutionbox}
\begin{mnemonicbox}
``TID: Transfer relates ports, Image creates
reflections, Driving point looks inward''

\end{mnemonicbox}
\subsection*{પ્રશ્ન 2(b) OR [4
ગુણ]}\label{q2b}

\textbf{સ્ટાન્ડર્ડ `T' નેટવર્ક માટે કેરેક્ટરીસ્ટીક ઇમ્પીડન્સ Z માટે સમીકરણ મેળવો.}

\begin{solutionbox}

\textbf{`T' નેટવર્કની કેરેક્ટરીસ્ટીક ઇમ્પીડન્સ:}

\textbf{આકૃતિ:}

\begin{verbatim}
        Z1/2         Z1/2
    o{-{-}{-}///{-}{-}{-}o{-}{-}{-}{-}///{-}{-}{-}o}
    |            |           |
    |            |           |
    A           Z2           B
    |            |           |
    |            |           |
    o{-{-}{-}{-}{-}{-}{-}{-}{-}{-}{-}{-}o{-}{-}{-}{-}{-}{-}{-}{-}{-}{-}{-}o}
\end{verbatim}

\textbf{ડેરિવેશન:} સિમેટ્રિકલ T-નેટવર્ક માટે સીરીઝ ઇમ્પીડન્સ Z_{1} (દરેક બાજુ પર Z_{1}/2
તરીકે વિભાજિત) અને શંટ ઇમ્પીડન્સ Z_{2} સાથે:

Z_{0} = \sqrt(Z_{1}Z_{2} + Z_{1}^{2}/4)

\textbf{સ્ટેપ્સ:}

\begin{enumerate}
\tightlist
\item
  T-નેટવર્ક માટે ABCD પેરામીટર્સ:

  \begin{itemize}
  \tightlist
  \item
    A = 1 + Z_{1}/2Z_{2}
  \item
    B = Z_{1} + Z_{1}^{2}/4Z_{2}
  \item
    C = 1/Z_{2}
  \item
    D = 1 + Z_{1}/2Z_{2}
  \end{itemize}
\item
  ટ્રાન્સમિશન લાઇન થિયરી માંથી, Z_{0} = \sqrt(B/C)
\item
  સબસ્ટિટ્યુટિંગ: Z_{0} = \sqrt((Z_{1} + Z_{1}^{2}/4Z_{2})/(1/Z_{2}))
\item
  સરળીકરણ: Z_{0} = \sqrt(Z_{1}Z_{2} + Z_{1}^{2}/4)
\end{enumerate}

\end{solutionbox}
\begin{mnemonicbox}
``Z-પ્રોડક્ટ પ્લસ ક્વાર્ટર-સ્ક્વેરનું વર્ગમૂળ''

\end{mnemonicbox}
\subsection*{પ્રશ્ન 2(c) OR [7
ગુણ]}\label{q2c}

\textbf{6, 15 અને 10 ઓહ્મના ત્રણ રેસીસ્ટર સ્ટાર માં જોડાયેલા છે. સમકક્ષ ડેલ્ટા નેટવર્ક
શોધો.}

\begin{solutionbox}

\textbf{સ્ટાર થી ડેલ્ટા રૂપાંતરણ:}

\textbf{આકૃતિ:}

\begin{verbatim}
   Star Network           Delta Network
       ra                      R1
   o{-{-}{-}///{-}{-}o           o{-}{-}{-}///{-}{-}{-}o}
   |          |           |           |
   |          |           |           |
   |          |           |           |
   |          |          R3          R2
  rb         rc           |           |
   |          |           |           |
   |          |           |           |
   o{-{-}{-}{-}{-}{-}{-}{-}{-}{-}o           o{-}{-}{-}{-}{-}{-}{-}{-}{-}{-}{-}o}
\end{verbatim}

\textbf{રૂપાંતરણ સૂત્રો:}

\begin{itemize}
\tightlist
\item
  R_{1} = (ra\timesrb + rb\timesrc + rc\timesra)/ra
\item
  R_{2} = (ra\timesrb + rb\timesrc + rc\timesra)/rb
\item
  R_{3} = (ra\timesrb + rb\timesrc + rc\timesra)/rc
\end{itemize}

\textbf{ગણતરી:} આપેલું: ra = 6Ω, rb = 15Ω, rc = 10Ω પ્રોડક્ટનો સરવાળો =
(6\times15) + (15\times10) + (10\times6) = 90 + 150 + 60 = 300

{\def\LTcaptype{none} % do not increment counter
\begin{longtable}[]{@{}llll@{}}
\toprule\noalign{}
ડેલ્ટા પ્રતિરોધક & સૂત્ર & ગણતરી & પરિણામ \\
\midrule\noalign{}
\endhead
\bottomrule\noalign{}
\endlastfoot
R_{1} & (ra\timesrb + rb\timesrc + rc\timesra)/ra & 300/6 & 50Ω \\
R_{2} & (ra\timesrb + rb\timesrc + rc\timesra)/rb & 300/15 & 20Ω \\
R_{3} & (ra\timesrb + rb\timesrc + rc\timesra)/rc & 300/10 & 30Ω \\
\end{longtable}
}

\end{solutionbox}
\begin{mnemonicbox}
``પ્રોડક્ટ્સ સમ ઓવર ઓપોઝિટ: ડેલ્ટા બાજુને સામેના સ્ટાર આર્મ વડે
ભાગેલા બધા પ્રોડક્ટ્સ મળે છે''

\end{mnemonicbox}
\subsection*{પ્રશ્ન 3(a) [3
ગુણ]}\label{q3a}

\textbf{KVL નો ઉપયોગ કરીને લૂપ કરંટની ગણતરી કરવા માટે સર્કિટ (R1, R2 અને R3 dc
સપ્લાય સાથે શ્રેણીમાં જોડાયેલા) નું વિશ્લેષણ કરો}

\begin{solutionbox}

\textbf{શ્રેણી સર્કિટ માટે KVL:}

\textbf{આકૃતિ:}

\begin{verbatim}
      +{-{-}{-}+}
VS {-{-}{-}|   |{-}{-} R1 {-}{-}+{-}{-} R2 {-}{-}+{-}{-} R3 {-}{-}+}
      +{-{-}{-}+        |        |        |}
                   |        |        |
                   |        |        |
                  I|       I|       I|
                   |        |        |
                   |        |        |
                   +{-{-}{-}{-}{-}{-}{-}{-}+{-}{-}{-}{-}{-}{-}{-}{-}+}
                          {-{-}{-}{-}{-}}
                           {-{-}{-}}
                            {-}
\end{verbatim}

\textbf{KVL સમીકરણ:} VS - IR_{1} - IR_{2} - IR_{3} = 0 \textbf{લૂપ કરંટ:} I =
VS/(R_{1} + R_{2} + R_{3})

\textbf{સ્ટેપ્સ:}

\begin{enumerate}
\tightlist
\item
  લૂપમાં બધા ઘટકોને ઓળખો: VS, R_{1}, R_{2}, R_{3}
\item
  KVL લાગુ કરો: વોલ્ટેજ વૃદ્ધિનો સરવાળો = વોલ્ટેજ ડ્રોપનો સરવાળો
\item
I માટે ઉકેલ:

I = VS/RT જ્યાં RT = R_{1} + R_{2} + R_{3}

\end{enumerate}

\end{solutionbox}
\begin{mnemonicbox}
``KVL: કિરચોફનો વોલ્ટેજ લૂપ કુલ પ્રતિરોધની જરૂર પડે છે''

\end{mnemonicbox}
\subsection*{પ્રશ્ન 3(b) [4
ગુણ]}\label{q3b}

\textbf{નોર્ટનનું થીયરમ લખો.}

\begin{solutionbox}

\textbf{નોર્ટનનું થીયરમ:}

વોલ્ટેજ સ્ત્રોત, કરંટ સ્ત્રોત અને પ્રતિરોધ વાળા કોઈપણ લિનિયર ઇલેક્ટ્રિકલ નેટવર્કને IN
કરંટ સ્ત્રોત અને RN પ્રતિરોધ સમાંતર જોડાયેલા સમકક્ષ સર્કિટ દ્વારા બદલી શકાય છે.

\textbf{આકૃતિ:}

\begin{verbatim}
    Original Network          Norton Equivalent
         +{-{-}{-}{-}{-}+                    +{-}{-}{-}{-}{-}+}
         |     |                    |     |
         |  A  |                    | IN  |
         |     |                    |     |
         +{-{-}+{-}{-}+                    +{-}{-}+{-}{-}+}
            |                          |
      +{-{-}{-}{-}{-}+{-}{-}{-}{-}{-}+              +{-}{-}{-}{-}{-}+{-}{-}{-}{-}{-}+}
      |     |     |              |     |     |
      |     Z     |       ={     |    RN     |}
      |     |     |              |     |     |
      +{-{-}{-}{-}{-}+{-}{-}{-}{-}{-}+              +{-}{-}{-}{-}{-}+{-}{-}{-}{-}{-}+}
            |                          |
         +{-{-}+{-}{-}+                    +{-}{-}+{-}{-}+}
         |     |                    |     |
         |  B  |                    |     |
         |     |                    |     |
         +{-{-}{-}{-}{-}+                    +{-}{-}{-}{-}{-}+}
\end{verbatim}

\textbf{નોર્ટન સમકક્ષ કેવી રીતે શોધવું:}

\begin{enumerate}
\tightlist
\item
  \textbf{નોર્ટન કરંટ (IN)}: લોડ ટર્મિનલ્સ વચ્ચે શોર્ટ-સર્કિટ કરંટ
\item
  \textbf{નોર્ટન રેસિસ્ટન્સ (RN)}: બધા સ્ત્રોતોને તેમના આંતરિક પ્રતિરોધ સાથે બદલીને
  ટર્મિનલ્સથી જોતા ઈનપુટ રેસિસ્ટન્સ
\end{enumerate}

\end{solutionbox}
\begin{mnemonicbox}
``SCIP: Short-Circuit current In Parallel with
equivalent resistance''

\end{mnemonicbox}
\subsection*{પ્રશ્ન 3(c) [7
ગુણ]}\label{q3c}

\textbf{સુપરપોઝિશન પ્રમેયનો ઉપયોગ કરીને ckt ની કોઈપણ શાખામાં કરંટની ગણતરી
કરવાનાં પગલાં સમજાવો}

\begin{solutionbox}

\textbf{સુપરપોઝિશન થીયરમનો ઉપયોગ:}

\textbf{સિદ્ધાંત:} એક લિનિયર સર્કિટમાં બહુવિધ સ્ત્રોત સાથે, કોઈપણ તત્વમાં પ્રતિભાવ
દરેક સ્ત્રોત એકલા કાર્ય કરતા હોય ત્યારે થતા પ્રતિભાવોના સરવાળા બરાબર હોય છે.

\textbf{સ્ટેપ્સ:}

\begin{enumerate}
\tightlist
\item
  એક સમયે એક જ સ્ત્રોત ધ્યાનમાં લો
\item
  અન્ય વોલ્ટેજ સ્ત્રોતને શોર્ટ સર્કિટ સાથે બદલો
\item
  અન્ય કરંટ સ્ત્રોતને ઓપન સર્કિટ સાથે બદલો
\item
  દરેક સ્ત્રોત માટે આંશિક કરંટની ગણતરી કરો
\item
  તમામ આંશિક કરંટને (બીજગણિતીય રીતે) એકસાથે ઉમેરો
\end{enumerate}

\textbf{આકૃતિ:}

\begin{verbatim}
flowchart LR
    A[એક સ્ત્રોત પસંદ કરો] {-{-} B[અન્ય સ્ત્રોતોને બદલો]}
    B {-{-} C[આંશિક કરંટની ગણતરી કરો]}
    C {-{-} D[બધા સ્ત્રોત માટે પુનરાવર્તન કરો]}
    D {-{-} E[આંશિક કરંટનો સરવાળો કરો]}
\end{verbatim}

\textbf{ગાણિતિક અભિવ્યક્તિ:} I = I_{1} + I_{2} + I_{3} + \ldots{} + In જ્યાં I_{1}, I_{2},
વગેરે વ્યક્તિગત સ્ત્રોતોના કારણે આંશિક કરંટ છે

\textbf{ઉદાહરણ ગણતરી:} કરંટ યોગદાન સાથે શાખા માટે: I_{1} = 2A (સ્ત્રોત 1 થી) I_{2}
= -1A (સ્ત્રોત 2 થી) I_{3} = 0.5A (સ્ત્રોત 3 થી) કુલ કરંટ = 2A + (-1A) + 0.5A =
1.5A

\end{solutionbox}
\begin{mnemonicbox}
``OSACI: One Source Active, Calculate and
Integrate''

\end{mnemonicbox}
\subsection*{પ્રશ્ન 3(a) OR [3
ગુણ]}\label{q3a}

\textbf{KCL નો ઉપયોગ કરીને નોડ વોલ્ટેજની ગણતરી કરવા માટે સર્કિટ (R1, R2 અને R3
ડીસી સપ્લાય સાથે સમાંતર જોડાયેલ) નું વિશ્લેષણ કરો}

\begin{solutionbox}

\textbf{સમાંતર સર્કિટ માટે KCL:}

\textbf{આકૃતિ:}

\begin{verbatim}
                 I1
      +{-{-}{-}+    +{-}{-}{-}+}
VS {-{-}{-}|   |{-}{-}{-}{-}| R1|{-}{-}{-}{-}+}
      +{-{-}{-}+    +{-}{-}{-}+    |}
                        |
                 I2     |
                +{-{-}{-}+   |}
                | R2|{-{-}{-}+{-}{-}{-} V (Node)}
                +{-{-}{-}+   |}
                        |
                 I3     |
                +{-{-}{-}+   |}
                | R3|{-{-}{-}+}
                +{-{-}{-}+   |}
                        |
                       {-{-}{-}}
                        {-}
\end{verbatim}

\textbf{KCL સમીકરણ:} I_{1} + I_{2} + I_{3} = 0 \textbf{નોડ વોલ્ટેજ:} V = VS (કારણ કે
સમાંતર ઘટકોમાં સમાન વોલ્ટેજ હોય છે)

\textbf{સ્ટેપ્સ:}

\begin{enumerate}
\tightlist
\item
  નોડ વોલ્ટેજ V ને ઓળખો
\item
  શાખા કરંટને વ્યક્ત કરો: I_{1} = V/R_{1}, I_{2} = V/R_{2}, I_{3} = V/R_{3}
\item
  KCL લાગુ કરો: V/R_{1} + V/R_{2} + V/R_{3} = VS/RT જ્યાં 1/RT = 1/R_{1} + 1/R_{2} + 1/R_{3}
\end{enumerate}

\end{solutionbox}
\begin{mnemonicbox}
``KCL: કિરચોફનો કરંટ નિયમ સમાંતર વોલ્ટેજ સ્ત્રોત જેટલો જ
બતાવે છે''

\end{mnemonicbox}
\subsection*{પ્રશ્ન 3(b) OR [4
ગુણ]}\label{q3b}

\textbf{મહત્તમ પાવર ટ્રાન્સફર થીયરમ લખો.}

\begin{solutionbox}

\textbf{મહત્તમ પાવર ટ્રાન્સફર થીયરમ:}

આંતરિક પ્રતિરોધ ધરાવતા સ્ત્રોત માટે, જ્યારે લોડ પ્રતિરોધ સ્ત્રોતના આંતરિક પ્રતિરોધ
બરાબર હોય ત્યારે લોડમાં મહત્તમ પાવર ટ્રાન્સફર થાય છે.

\textbf{આકૃતિ:}

\begin{verbatim}
    +{-{-}{-}+     Rsource      +{-}{-}{-}+}
    |   |{-{-}{-}{-}////{-}{-}{-}{-}{-}{-}|   |}
    | V |                  | RL|
    |   |                  |   |
    +{-{-}{-}+                  +{-}{-}{-}+}
     {-{-}{-}                    {-}{-}{-}}
      {-                      {-}}
\end{verbatim}

\textbf{ગાણિતિક અભિવ્યક્તિ:}

\begin{itemize}
\tightlist
\item
  મહત્તમ પાવર ટ્રાન્સફર થાય ત્યારે RL = Rsource
\item
  મહત્તમ પાવર: Pmax = V^{2}/(4\timesRsource)
\end{itemize}

\textbf{મુખ્ય મુદ્દાઓ:}

\begin{itemize}
\tightlist
\item
  \textbf{કાર્યક્ષમતા}: મહત્તમ પાવર ટ્રાન્સફર પર માત્ર 50\%
\item
  \textbf{AC સર્કિટ્સ}: લોડ ઇમ્પીડન્સ સ્ત્રોત ઇમ્પીડન્સનો કોમ્પ્લેક્સ કોન્જુગેટ હોવો
  જોઈએ
\item
  \textbf{ઉપયોગો}: સિગ્નલ ટ્રાન્સમિશન, ઓડિયો સિસ્ટમ્સ, RF સર્કિટ્સ
\end{itemize}

\end{solutionbox}
\begin{mnemonicbox}
``MEET: Maximum Efficiency Equals when
Thevenin-matched''

\end{mnemonicbox}
\subsection*{પ્રશ્ન 3(c) OR [7
ગુણ]}\label{q3c}

\textbf{થેવેનિનના પ્રમેયનો ઉપયોગ કરીને ckt માં Vth, Rth અને લોડ કરંટની ગણતરી
કરવાનાં પગલાં સમજાવો.}

\begin{solutionbox}

\textbf{થેવેનિનના થીયરમનો ઉપયોગ:}

\textbf{સિદ્ધાંત:} વોલ્ટેજ અને કરંટ સ્ત્રોત ધરાવતા કોઈપણ લિનિયર ઇલેક્ટ્રિકલ નેટવર્કને
એક સિંગલ વોલ્ટેજ સ્ત્રોત Vth અને શ્રેણી પ્રતિરોધ Rth વાળા સમકક્ષ સર્કિટ દ્વારા બદલી
શકાય છે.

\textbf{સ્ટેપ્સ:}

\begin{enumerate}
\tightlist
\item
  સર્કિટમાંથી લોડ પ્રતિરોધ દૂર કરો
\item
  લોડ ટર્મિનલ્સ વચ્ચે ઓપન-સર્કિટ વોલ્ટેજ (Vth) ની ગણતરી કરો
\item
  બધા સ્ત્રોતોને તેમના આંતરિક પ્રતિરોધ સાથે બદલો (વોલ્ટેજ સ્ત્રોતને શોર્ટ સર્કિટ તરીકે,
  કરંટ સ્ત્રોતને ઓપન સર્કિટ તરીકે)
\item
  લોડ ટર્મિનલ્સથી જોતા સમકક્ષ પ્રતિરોધ (Rth) ની ગણતરી કરો
\item
  Vth અને Rth સાથે થેવેનિન સમકક્ષ સર્કિટ દોરો
\item
  લોડને ફરીથી જોડો અને લોડ કરંટની ગણતરી કરો: IL = Vth/(Rth + RL)
\end{enumerate}

\textbf{આકૃતિ:}

\begin{verbatim}
flowchart LR
    A[લોડ દૂર કરો] {-{-} B[Vth શોધો]}
    B {-{-} C[સ્ત્રોતોને આંતરિક પ્રતિરોધ સાથે બદલો]}
    C {-{-} D[Rth ની ગણતરી કરો]}
    D {-{-} E[થેવેનિન સમકક્ષ દોરો]}
    E {-{-} F[લોડ ફરીથી જોડીને IL ની ગણતરી કરો]}
\end{verbatim}

\textbf{ઉદાહરણ ગણતરી:}

\begin{itemize}
\tightlist
\item
  જો Vth = 12V
\item
  Rth = 3Ω
\item
  RL = 6Ω
\item
  પછી IL = 12V/(3Ω + 6Ω) = 12V/9Ω = 1.33A
\end{itemize}

\end{solutionbox}
\begin{mnemonicbox}
``VORTE: Voltage Open, Resistance with sources
Transformed, Equivalent circuit''

\end{mnemonicbox}
\subsection*{પ્રશ્ન 4(a) [3
ગુણ]}\label{q4a}

\textbf{રેઝોનન્સ વ્યાખ્યાયિત કરો.}

\begin{solutionbox}

\textbf{રેઝોનન્સ:}

રેઝોનન્સ એ એક ઘટના છે જેમાં સર્કિટ ચોક્કસ ફ્રિક્વન્સી પર, જેને રેઝોનન્ટ ફ્રિક્વન્સી
કહેવામાં આવે છે, એપ્લાઈડ સિગ્નલનો મહત્તમ એમ્પ્લિટ્યુડ સાથે પ્રતિસાદ આપે છે.

\textbf{મુખ્ય લાક્ષણિકતાઓ:}

\begin{itemize}
\tightlist
\item
  ઇમ્પીડન્સ માત્ર રેઝિસ્ટિવ બને છે
\item
  ઇન્ડક્ટિવ રિએક્ટન્સ કેપેસિટિવ રિએક્ટન્સ બરાબર થાય છે (XL = XC)
\item
  વોલ્ટેજ અને કરંટ એક જ ફેઝમાં હોય છે
\item
  સર્કિટ L અને C ઘટકો વચ્ચે ઊર્જા સંગ્રહિત કરે છે અને છોડે છે
\end{itemize}

\textbf{ઉપયોગો:}

\begin{itemize}
\tightlist
\item
  ટ્યુનિંગ સર્કિટ્સ
\item
  ફિલ્ટર્સ
\item
  ઓસીલેટર્સ
\item
  વાયરલેસ કોમ્યુનિકેશન
\end{itemize}

\end{solutionbox}
\begin{mnemonicbox}
``MAX-IN-PHASE: Maximum response when Inductive and
capacitive reactances are equal and PHASEs cancel''

\end{mnemonicbox}
\subsection*{પ્રશ્ન 4(b) [4
ગુણ]}\label{q4b}

\textbf{કોઇલના ક્વાલિટી ફેક્ટર માટે સમીકરણ મેળવો.}

\begin{solutionbox}

\textbf{કોઇલનો ક્વાલિટી ફેક્ટર (Q):}

\textbf{વ્યાખ્યા:} Q-ફેક્ટર એ રેઝોનન્ટ સર્કિટમાં સંગ્રહિત ઊર્જાનું એક ચક્ર દીઠ વેડફાતી
ઊર્જા સાથેનો ગુણોત્તર છે.

\textbf{ડેરિવેશન:} ઇન્ડક્ટન્સ L અને રેઝિસ્ટન્સ R વાળી કોઇલ માટે:

\begin{enumerate}
\tightlist
\item
  ઇન્ડક્ટરમાં સંગ્રહિત ઊર્જા: WL = ½LI^{2}
\item
  રેઝિસ્ટન્સમાં વેડફાતી પાવર: P = I^{2}R
\item
સમય અવધિ:

T = 1/f = 2π/ω

\item
  એક ચક્ર દીઠ વેડફાતી ઊર્જા: Wd = P\timesT = I^{2}R\times(2π/ω)
\item
  Q = 2π(સંગ્રહિત ઊર્જા/એક ચક્ર દીઠ વેડફાતી ઊર્જા)
\item
  Q = 2π(½LI^{2})/(I^{2}R\times2π/ω) = ωL/R
\end{enumerate}

\textbf{અંતિમ સમીકરણ:} Q = ωL/R = 2πfL/R

\textbf{મહત્વ:}

\begin{itemize}
\tightlist
\item
  ઉચ્ચ Q ઓછી ઊર્જા ખોટ સૂચવે છે
\item
  Q ફ્રિક્વન્સી સાથે વધે છે
\item
  Q રેઝિસ્ટન્સ સાથે ઘટે છે
\end{itemize}

\end{solutionbox}
\begin{mnemonicbox}
``ઓમેગા-L ડિવાઇડેડ બાય R ગિવ્સ ક્વાલિટી''

\end{mnemonicbox}
\subsection*{પ્રશ્ન 4(c) [7
ગુણ]}\label{q4c}

\textbf{RLC શ્રેણીના સર્કિટમાં R=1 KΩ, L=100 mH અને C=10µF છે. જો શ્રેણીના
સંયોજનમાં 100 V નો વોલ્ટેજ લાગુ કરવામાં આવે તો, નક્કી કરો: (i) રેઝોનન્સ ફ્રીક્વન્સી
(ii) `Q' પરિબળ}

\begin{solutionbox}

\textbf{RLC શ્રેણી સર્કિટ વિશ્લેષણ:}

\textbf{આકૃતિ:}

\begin{verbatim}
         L=100mH
      +{-{-}{-}uuuu{-}{-}{-}{-}+}
      |           |
      |           |
100V

R=1kΩ

C=10µF

      |           |
      |           |
      +{-{-}{-}{-}{-}{-}{-}{-}{-}{-}{-}+}
\end{verbatim}

\textbf{ગણતરી:}

\begin{enumerate}
\tightlist
\item
  \textbf{રેઝોનન્સ ફ્રીક્વન્સી:}
\end{enumerate}

\begin{itemize}
\tightlist
\item
  સૂત્ર: fr = 1/(2π\sqrt(LC))
\item
  fr = 1/(2π\sqrt(100\times10^{-}^{3} \times 10\times10^{-}^{6}))
\item
  fr = 1/(2π\sqrt(1\times10^{-}^{6}))
\item
  fr = 1/(2π \times 1\times10^{-}^{3})
\item
  fr = 159.15 Hz
\end{itemize}

\begin{enumerate}
\tightlist
\item
  \textbf{ક્વોલિટી ફેક્ટર (Q):}
\end{enumerate}

\begin{itemize}
\tightlist
\item
  સૂત્ર: Q = (1/R)\sqrt(L/C)
\item
  Q = (1/1000)\sqrt(100\times10^{-}^{3}/10\times10^{-}^{6})
\item
  Q = (1/1000)\sqrt(10^{4})
\item
  Q = (1/1000) \times 100
\item
  Q = 0.1
\end{itemize}

{\def\LTcaptype{none} % do not increment counter
\begin{longtable}[]{@{}llll@{}}
\toprule\noalign{}
પેરામીટર & સૂત્ર & ગણતરી & પરિણામ \\
\midrule\noalign{}
\endhead
\bottomrule\noalign{}
\endlastfoot
રેઝોનન્ટ ફ્રિક્વન્સી (fr) & 1/(2π\sqrt(LC)) & 1/(2π\sqrt(1\times10^{-}^{6})) & 159.15 Hz \\
ક્વોલિટી ફેક્ટર (Q) & (1/R)\sqrt(L/C) & (1/1000)\sqrt(10^{4}) & 0.1 \\
\end{longtable}
}

\end{solutionbox}
\begin{mnemonicbox}
``ફ્રિક્વન્સી LC માંથી, ક્વોલિટી LCR માંથી''

\end{mnemonicbox}
\subsection*{પ્રશ્ન 4(a) OR [3
ગુણ]}\label{q4a}

\textbf{મ્યુચ્યુઅલ ઇન્ડક્ટન્સ વ્યાખ્યાયિત કરો.}

\begin{solutionbox}

\textbf{મ્યુચ્યુઅલ ઇન્ડક્ટન્સ:}

મ્યુચ્યુઅલ ઇન્ડક્ટન્સ એ સર્કિટનો એવો ગુણધર્મ છે જેના કારણે એક કોઇલમાં કરંટમાં ફેરફાર
થવાથી તેમની વચ્ચેના મેગ્નેટિક કપલીંગને કારણે બીજી કોઇલમાં વોલ્ટેજ પ્રેરિત થાય છે.

\textbf{ગાણિતિક અભિવ્યક્તિ:}

\begin{itemize}
\tightlist
\item
  કોઇલ 2 માં પ્રેરિત વોલ્ટેજ: V_{2} = -M(dI_{1}/dt)
\item
  M = k\sqrt(L_{1}L_{2}) જ્યાં k કપલિંગ કોએફિશિયન્ટ છે (0\leqk\leq1)
\item
  એકમ: હેનરી (H)
\end{itemize}

\textbf{મુખ્ય ગુણધર્મો:}

\begin{itemize}
\tightlist
\item
  કોઇલ જ્યોમેટ્રી, અંતર અને ઓરિએન્ટેશન પર આધાર રાખે છે
\item
  બંને ઇન્ડક્ટન્સના પ્રમાણમાં હોય છે
\item
  ટ્રાન્સફોર્મર અને કપલ્ડ સર્કિટ્સનો આધાર છે
\item
  મ્યુચ્યુઅલ ફ્લક્સની દિશાના આધારે પોઝિટિવ અથવા નેગેટિવ હોઈ શકે છે
\end{itemize}

\end{solutionbox}
\begin{mnemonicbox}
``MICK: Mutual Inductance links Coils through
K-coupling''

\end{mnemonicbox}
\subsection*{પ્રશ્ન 4(b) OR [4
ગુણ]}\label{q4b}

\textbf{કોએફીશિયન્ટ ઓફ કપલિંગનું સમીકરણ મેળવો}

\begin{solutionbox}

\textbf{કોએફિશિયન્ટ ઓફ કપલિંગ (k):}

\textbf{વ્યાખ્યા:} કોએફિશિયન્ટ ઓફ કપલિંગ (k) એ બે કોઇલ્સ વચ્ચેના મેગ્નેટિક કપલિંગનું
માપ છે, જે 0 (કોઈ કપલિંગ નહીં) થી 1 (પૂર્ણ કપલિંગ) સુધી હોય છે.

\textbf{ડેરિવેશન:}

\begin{enumerate}
\tightlist
\item
  મ્યુચ્યુઅલ ઇન્ડક્ટન્સ વ્યાખ્યાયિત કરો: M = મેગ્નેટિક ફ્લક્સ લિંકેજ / કરંટ
\item
  સેલ્ફ-ઇન્ડક્ટન્સ L_{1} અને L_{2} વાળી બે કોઇલ્સ માટે:

  \begin{itemize}
  \tightlist
  \item
    કોઇલ 1 માં કરંટ 1 ના કારણે કોઇલ 1 માં ફ્લક્સ લિંકેજ: λ_{1}_{1} = L_{1}I_{1}
  \item
    કોઇલ 2 માં કરંટ 2 ના કારણે કોઇલ 2 માં ફ્લક્સ લિંકેજ: λ_{2}_{2} = L_{2}I_{2}
  \item
    કોઇલ 1 માં કરંટ ના કારણે કોઇલ 2 માં ફ્લક્સ લિંકેજ: λ_{2}_{1} = MI_{1}
  \end{itemize}
\item
  કપલિંગ કોએફિશિયન્ટ k એ કોઇલ 1 માંથી ફ્લક્સનો અંશ જે કોઇલ 2 સાથે જોડાય છે તેનું
  પ્રતિનિધિત્વ કરે છે
\item
  ઇલેક્ટ્રોમેગ્નેટિક થિયરી માંથી: M = k\sqrt(L_{1}L_{2})
\item
  ફરીથી ગોઠવણ: k = M/\sqrt(L_{1}L_{2})
\end{enumerate}

\textbf{અંતિમ સમીકરણ:} k = M/\sqrt(L_{1}L_{2})

\textbf{મુખ્ય મુદ્દાઓ:}

\begin{itemize}
\tightlist
\item
  k = 0: કોઈ મેગ્નેટિક કપલિંગ નહીં
\item
  0 \textless{} k \textless{} 1: આંશિક કપલિંગ
\item
  k = 1: પૂર્ણ કપલિંગ (બધો ફ્લક્સ બંને કોઇલ્સને જોડે છે)
\end{itemize}

\end{solutionbox}
\begin{mnemonicbox}
``M ડિવાઇડેડ બાય જીઓમેટ્રિક મીન ઓફ Ls''

\end{mnemonicbox}
\subsection*{પ્રશ્ન 4(c) OR [7
ગુણ]}\label{q4c}

\textbf{સમાંતર રેઝોનન્સ સર્કિટની રેઝોનન્સ ફ્રીક્વન્સી મેળવો.}

\begin{solutionbox}

\textbf{સમાંતર રેઝોનન્સ ફ્રીક્વન્સી ડેરિવેશન:}

\textbf{આકૃતિ:}

\begin{verbatim}
              +{-{-}{-}+}
              |   |
           +{-{-}+{-}{-}{-}+{-}{-}+}
           |          |
           |          |
      L    |          |   C
    uuuuu  |          |  ||| 
           |          |  |||
           |          |  |||
           |          |  |||
           +{-{-}{-}{-}{-}{-}{-}{-}{-}{-}+{-}{-}{-}+}
              |   |
              +{-{-}{-}+}
               R
\end{verbatim}

\textbf{ડેરિવેશન સ્ટેપ્સ:}

\begin{enumerate}
\item
સમાંતર RLC સર્કિટ માટે, એડમિટન્સ છે:

Y = 1/Z = 1/R + 1/jωL + jωC

\item
  રેઝોનન્સ પર, કાલ્પનિક ભાગ શૂન્ય થાય છે: Im(Y) = 0 1/jωL + jωC = 0 -j/ωL +
  jωC = 0 1/ωL = ωC ω^{2}LC = 1
\item
  આદર્શ કિસ્સા માટે (અનંત પ્રતિરોધ સાથે): ω_{0} = 1/\sqrt(LC) f_{0} = 1/(2π\sqrt(LC))
\item
  વાસ્તવિક કિસ્સા માટે (પ્રતિરોધ R સાથે): જો R, L ની શ્રેણીમાં હોય, તો રેઝોનન્ટ
  ફ્રિક્વન્સી થાય છે: f_{0} = (1/2π)\sqrt(1/LC - R^{2}/L^{2})
\end{enumerate}

\textbf{અંતિમ સમીકરણ:}

\begin{itemize}
\tightlist
\item
  આદર્શ કિસ્સા: f_{0} = 1/(2π\sqrt(LC))
\item
  વાસ્તવિક કિસ્સા (R, L ની શ્રેણીમાં): f_{0} = (1/2π)\sqrt(1/LC - R^{2}/L^{2})
\end{itemize}

\textbf{સમાંતર રેઝોનન્સની મુખ્ય લાક્ષણિકતાઓ:}

\begin{itemize}
\tightlist
\item
  રેઝોનન્સ પર મહત્તમ ઇમ્પીડન્સ
\item
  સ્ત્રોતમાંથી લેવાતો ન્યૂનતમ કરંટ
\item
  L અને C વચ્ચે કરંટ પરિભ્રમણ કરે છે
\item
  ``એન્ટી-રેઝોનન્સ'' અથવા ``રિજેક્ટર સર્કિટ'' તરીકે પણ ઓળખાય છે
\end{itemize}

\end{solutionbox}
\begin{mnemonicbox}
``ONE over LC SQRT: The frequency where parallel
paths balance''

\end{mnemonicbox}
\subsection*{પ્રશ્ન 5(a) [3
ગુણ]}\label{q5a}

\textbf{વિવિધ પ્રકારના એટેન્યુએટરનું વર્ગીકરણ કરો.}

\begin{solutionbox}

\textbf{એટેન્યુએટરના પ્રકારો:}

{\def\LTcaptype{none} % do not increment counter
\begin{longtable}[]{@{}lll@{}}
\toprule\noalign{}
પ્રકાર & સંરચના & લાક્ષણિકતાઓ \\
\midrule\noalign{}
\endhead
\bottomrule\noalign{}
\endlastfoot
\textbf{T-પ્રકાર} & શ્રેણી-શંટ-શ્રેણી & સિમેટ્રિક, મેચિંગ માટે સારું, વ્યાપકપણે
વપરાતું \\
\textbf{\prod-પ્રકાર} & શંટ-શ્રેણી-શંટ & સિમેટ્રિક, T-પ્રકારનો વિકલ્પ \\
\textbf{લેટિસ} & બેલેન્સ્ડ બ્રિજ & સિમેટ્રિકલ, બેલેન્સ્ડ લાઇન્સમાં વપરાય છે \\
\textbf{L-પ્રકાર} & શ્રેણી-શંટ & એસિમેટ્રિક, સરળ ડિઝાઈન \\
\textbf{બ્રિજ્ડ-T} & બ્રિજ્ડ શંટ સાથે T & સારો ફ્રિક્વન્સી રિસ્પોન્સ, જટિલ \\
\textbf{O-પ્રકાર} & શ્રેણી-શંટ-શ્રેણી-શંટ & સુધારેલા રિજેક્શન લક્ષણો \\
\end{longtable}
}

\end{solutionbox}
\begin{mnemonicbox}
``TL\prodBO: Top attenuators Let \prod signals Balance
Output''

\end{mnemonicbox}
\subsection*{પ્રશ્ન 5(b) [4
ગુણ]}\label{q5b}

\textbf{ડેસિબલ અને નેપર વચ્ચેનો સંબંધ મેળવો}

\begin{solutionbox}

\textbf{ડેસિબલ થી નેપર રૂપાંતરણ:}

\textbf{વ્યાખ્યાઓ:}

\begin{itemize}
\tightlist
\item
  \textbf{ડેસિબલ (dB)}: બેઝ 10 (કોમન લોગેરિધમ) વાપરીને પાવર રેશિયો લોગેરિધમ
\item
  \textbf{નેપર (Np)}: બેઝ e (નેચરલ લોગેરિધમ) વાપરીને વોલ્ટેજ/કરંટ રેશિયો લોગેરિધમ
\end{itemize}

\textbf{ડેરિવેશન:}

\begin{enumerate}
\tightlist
\item
  dB માં પાવર રેશિયો: Loss(dB) = 10 log_{1}_{0}(P_{1}/P_{2})
\item
  dB માં વોલ્ટેજ રેશિયો: Loss(dB) = 20 log_{1}_{0}(V_{1}/V_{2})
\item
  નેપર માં વોલ્ટેજ રેશિયો: Loss(Np) = ln(V_{1}/V_{2})
\item
  લોગેરિધમ બેઝ વચ્ચે રૂપાંતરણ: log_{1}_{0}(x) = ln(x)/ln(10)
\item
  સબસ્ટિટ્યુટ: Loss(dB) = 20 ln(V_{1}/V_{2})/ln(10) = 20 Loss(Np)/ln(10)
\end{enumerate}

\textbf{અંતિમ સંબંધ:}

\begin{itemize}
\tightlist
\item
  1 નેપર = ln(10)/20 \times 10 dB = 8.686 dB
\item
  1 dB = 0.115 નેપર
\end{itemize}


{\def\LTcaptype{none} % do not increment counter
\begin{longtable}[]{@{}lll@{}}
\toprule\noalign{}
રૂપાંતરણ & સૂત્ર & મૂલ્ય \\
\midrule\noalign{}
\endhead
\bottomrule\noalign{}
\endlastfoot
નેપર થી dB & 1 Np = (20/ln10) dB & 1 Np = 8.686 dB \\
dB થી નેપર & 1 dB = (ln10/20) Np & 1 dB = 0.115 Np \\
\end{longtable}
}

\end{solutionbox}
\begin{mnemonicbox}
``8.686: Eight Point Six Nepers Buy Ten decibels''

\end{mnemonicbox}
\subsection*{પ્રશ્ન 5(c) [7
ગુણ]}\label{q5c}

\textbf{ડિઝાઇન T પ્રકારનું એટેન્યુએટર જેનો 20 ડીબી એટેન્યુએશન અને કેરેક્ટરીસ્ટીક
ઇમ્પીડન્સ 600 ઓહ્મ છે.}

\begin{solutionbox}

\textbf{T-પ્રકારના એટેન્યુએટર ડિઝાઇન:}

\textbf{આકૃતિ:}

\begin{verbatim}
       Z1/2         Z1/2
    o{-{-}{-}///{-}{-}{-}{-}o{-}{-}{-}///{-}{-}{-}o}
    |            |           |
    |            |           |
   R0           Z2          R0
    |            |           |
    |            |           |
    o{-{-}{-}{-}{-}{-}{-}{-}{-}{-}{-}{-}o{-}{-}{-}{-}{-}{-}{-}{-}{-}{-}{-}o}
\end{verbatim}

\textbf{ડિઝાઇન સ્ટેપ્સ:}

\begin{enumerate}
\item
dB માંથી એટેન્યુએશન રેશિયો N ની ગણતરી કરો:

N = 10\^{}(dB/20) =

  10\^{}(20/20) = 10
\item
  સૂત્રો વાપરીને R_{1} અને R_{2} ની ગણતરી કરો:

  \begin{itemize}
  \tightlist
  \item
    R_{1} = R_{0} \times [(N^{2} - 1)/(N^{2} + 1)]
  \item
    R_{2} = R_{0} \times [2N/(N^{2} - 1)]
  \end{itemize}
\end{enumerate}

\textbf{ગણતરી:}

આપેલું:

\begin{itemize}
\tightlist
\item
  એટેન્યુએશન = 20 dB
\item
  કેરેક્ટરીસ્ટીક ઇમ્પીડન્સ = 600 Ω
\end{itemize}

{\def\LTcaptype{none} % do not increment counter
\begin{longtable}[]{@{}llll@{}}
\toprule\noalign{}
પેરામીટર & સૂત્ર & ગણતરી & પરિણામ \\
\midrule\noalign{}
\endhead
\bottomrule\noalign{}
\endlastfoot
N & 10\^{}(dB/20) & 10\^{}(20/20) & 10 \\
R_{1} & R_{0}[(N^{2} - 1)/(N^{2} + 1)] & 600[(10^{2} - 1)/(10^{2} + 1)] & 588.2
Ω \\
Z_{1}/2 & R_{1}/2 & 588.2/2 & 294.1 Ω \\
R_{2} & R_{0}[2N/(N^{2} - 1)] & 600[2\times10/(10^{2} - 1)] & 121.2 Ω \\
\end{longtable}
}

\textbf{અંતિમ T-નેટવર્ક મૂલ્યો:}

\begin{itemize}
\tightlist
\item
  દરેક શ્રેણી આર્મ (Z_{1}/2): 294.1 Ω
\item
  શંટ આર્મ (Z_{2}): 121.2 Ω
\end{itemize}

\end{solutionbox}
\begin{mnemonicbox}
``N-squared minus ONE over N-squared plus ONE for
series resistance''

\end{mnemonicbox}
\subsection*{પ્રશ્ન 5(a) OR [3
ગુણ]}\label{q5a}

\textbf{કોંસ્ટંટ K લો પાસ ફિલ્ટર્સની મર્યાદાઓ લખો.}

\begin{solutionbox}

\textbf{કોન્સ્ટન્ટ-K લો પાસ ફિલ્ટર્સની મર્યાદાઓ:}

{\def\LTcaptype{none} % do not increment counter
\begin{longtable}[]{@{}
  >{\raggedright\arraybackslash}p{(\linewidth - 2\tabcolsep) * \real{0.4800}}
  >{\raggedright\arraybackslash}p{(\linewidth - 2\tabcolsep) * \real{0.5200}}@{}}
\toprule\noalign{}
\begin{minipage}[b]{\linewidth}\raggedright
મર્યાદા
\end{minipage} & \begin{minipage}[b]{\linewidth}\raggedright
વર્ણન
\end{minipage} \\
\midrule\noalign{}
\endhead
\bottomrule\noalign{}
\endlastfoot
\textbf{ખરાબ કટઓફ ટ્રાન્ઝિશન} & તીક્ષ્ણ કટઓફને બદલે પાસ બેન્ડથી સ્ટોપ બેન્ડમાં ક્રમિક
પરિવર્તન \\
\textbf{અસમાન ઇમ્પીડન્સ} & ઇમ્પીડન્સ ફ્રિક્વન્સી સાથે બદલાય છે, જેના કારણે મેચિંગ
સમસ્યાઓ ઉદ્ભવે છે \\
\textbf{એટેન્યુએશન રિપલ} & પાસ બેન્ડ અને સ્ટોપ બેન્ડ બંનેમાં બિન-સમાન એટેન્યુએશન \\
\textbf{ફેઝ ડિસ્ટોર્શન} & નોન-લિનિયર ફેઝ રિસ્પોન્સ જે સિગ્નલ ડિસ્ટોર્શન ઉત્પન્ન કરે
છે \\
\textbf{ફિક્સ્ડ ટર્મિનેશન} & વિશિષ્ટ લોડ ઇમ્પીડન્સ માટે ડિઝાઇન; અન્ય લોડ સાથે
પ્રદર્શન બગડે છે \\
\textbf{સીમિત સિલેક્ટિવિટી} & આધુનિક ફિલ્ટર ડિઝાઇનની તુલનામાં ખરાબ
સિલેક્ટિવિટી \\
\end{longtable}
}

\end{solutionbox}
\begin{mnemonicbox}
``PUAPFL: Poor transition, Uneven impedance,
Attenuation ripple, Phase distortion, Fixed termination, Limited
selectivity''

\end{mnemonicbox}
\subsection*{પ્રશ્ન 5(b) OR [4
ગુણ]}\label{q5b}

\textbf{ફ્રીક્વન્સી રિસ્પોન્સ વક્ર દશાર્વીને ફિલ્ટર્સનું વર્ગીકરણ આપો.}

\begin{solutionbox}

\textbf{ફિલ્ટર્સનું વર્ગીકરણ:}

{\def\LTcaptype{none} % do not increment counter
\begin{longtable}[]{@{}lll@{}}
\toprule\noalign{}
ફિલ્ટર પ્રકાર & ફ્રિક્વન્સી રિસ્પોન્સ વક્ર & લાક્ષણિકતાઓ \\
\midrule\noalign{}
\endhead
\bottomrule\noalign{}
\endlastfoot
\textbf{લો પાસ} & ```goat & \\
\end{longtable}
}

\begin{verbatim}
    |\\
    |  \\
    |    \\________
    |
    +---------------
       fc
``` | કટઓફ fc નીચેની ફ્રિક્વન્સી પસાર કરે છે, ઉચ્ચ ફ્રિક્વન્સી અવરોધે છે |
\end{verbatim}

\textbf{હાઇ પાસ} \textbar{}
\texttt{goat\ \ \ \ \ \ \ \ \textbar{}\ \ \ \ \ \ \ \ \_\_\_\_\_\_\_\ \ \ \ \ \ \ \ \textbar{}\ \ \ \ \ \ /\ \ \ \ \ \ \ \ \textbar{}\ \ \ \ /\ \ \ \ \ \ \ \ \textbar{}\ \ /\ \ \ \ \ \ \ \ \textbar{}/\ \ \ \ \ \ \ \ +-\/-\/-\/-\/-\/-\/-\/-\/-\/-\/-\/-\/-\/-\/-\ \ \ \ \ \ \ \ \ \ \ fc}
\textbar{} કટઓફ fc નીચેની ફ્રિક્વન્સી અવરોધે છે, ઉચ્ચ ફ્રિક્વન્સી પસાર કરે છે
\textbar{}\\
\textbf{બેન્ડ પાસ} \textbar{}
\texttt{goat\ \ \ \ \ \ \ \ \textbar{}\ \ \ \ \ \ /\textbackslash{}\ \ \ \ \ \ \ \ \textbar{}\ \ \ \ \ /\ \ \textbackslash{}\ \ \ \ \ \ \ \ \textbar{}\ \ \ \ /\ \ \ \ \textbackslash{}\ \ \ \ \ \ \ \ \textbar{}\ \ \ /\ \ \ \ \ \ \textbackslash{}\ \ \ \ \ \ \ \ \textbar{}\_\_/\ \ \ \ \ \ \ \ \textbackslash{}\_\_\_\ \ \ \ \ \ \ \ +-\/-\/-\/-\/-\/-\/-\/-\/-\/-\/-\/-\/-\/-\/-\ \ \ \ \ \ \ \ \ \ \ f1\ \ \ \ f2}
\textbar{} f1 અને f2 વચ્ચેની ફ્રિક્વન્સી પસાર કરે છે, અન્યને અવરોધે છે \textbar{}\\
\textbf{બેન્ડ સ્ટોપ} \textbar{}
\texttt{goat\ \ \ \ \ \ \ \ \textbar{}\_\_\_\ \ \ \ \ \ \ \ \_\_\_\ \ \ \ \ \ \ \ \textbar{}\ \ \ \textbackslash{}\ \ \ \ \ \ /\ \ \ \ \ \ \ \ \textbar{}\ \ \ \ \textbackslash{}\ \ \ \ /\ \ \ \ \ \ \ \ \textbar{}\ \ \ \ \ \textbackslash{}\ \ /\ \ \ \ \ \ \ \ \textbar{}\ \ \ \ \ \ \textbackslash{}/\ \ \ \ \ \ \ \ +-\/-\/-\/-\/-\/-\/-\/-\/-\/-\/-\/-\/-\/-\/-\ \ \ \ \ \ \ \ \ \ \ f1\ \ \ \ f2}
\textbar{} f1 અને f2 વચ્ચેની ફ્રિક્વન્સી અવરોધે છે, અન્યને પસાર કરે છે \textbar{}

\end{solutionbox}
\begin{mnemonicbox}
``LHBS: Low lets low tones, High lets high tones,
Band-pass selects middle, Band-Stop rejects middle''

\end{mnemonicbox}
\subsection*{પ્રશ્ન 5(c) OR [7
ગુણ]}\label{q5c}

\textbf{કોંસ્ટંટ K લો પાસ ફિલ્ટર્સ ડિઝાઇન કરવા માટે સમીકરણ મેળવો.}

\begin{solutionbox}

\textbf{કોન્સ્ટન્ટ-K લો પાસ ફિલ્ટર ડિઝાઇન:}

\textbf{આકૃતિ:}

\begin{verbatim}
    T{-section:              π{-}section:}
    
       L/2         L/2
    o{-{-}{-}uuu{-}{-}{-}{-}o{-}{-}{-}uuu{-}{-}{-}o      o{-}{-}{-}{-}{-}{-}{-}{-}{-}{-}{-}o}
    |          |          |     |           |
    |          |          |     |           |
    |          C          |     C/2        C/2
    |          |          |     |           |
    |          |          |     |           |
    o{-{-}{-}{-}{-}{-}{-}{-}{-}{-}o{-}{-}{-}{-}{-}{-}{-}{-}{-}{-}o     o{-}{-}{-}{-}{-}uuu{-}{-}{-}o}
                                      L
\end{verbatim}

\textbf{ડિઝાઇન થિયરી:} કોન્સ્ટન્ટ-K ફિલ્ટરમાં ઇમ્પીડન્સ પ્રોડક્ટ Z_{1}Z_{2} = k^{2} (અચળ)
બધી ફ્રિક્વન્સી પર રહે છે.

\textbf{ડેરિવેશન સ્ટેપ્સ:}

\begin{enumerate}
\tightlist
\item
  T-સેક્શન લો-પાસ ફિલ્ટર માટે:

  \begin{itemize}
  \tightlist
  \item
    સીરીઝ ઇમ્પીડન્સ Z_{1} = jωL
  \item
    શંટ ઇમ્પીડન્સ Z_{2} = 1/jωC
  \end{itemize}
\item
  Z_{1}Z_{2} પ્રોડક્ટ અચળ હોવું જોઈએ:

  \begin{itemize}
  \tightlist
  \item
    Z_{1}Z_{2} = jωL \times 1/jωC = L/C = k^{2}
  \end{itemize}
\item
  ઝીરો ફ્રિક્વન્સી પર કેરેક્ટરીસ્ટીક ઇમ્પીડન્સ:

  \begin{itemize}
  \tightlist
  \item
    R_{0} = \sqrt(L/C)
  \end{itemize}
\item
  કટ-ઓફ ફ્રિક્વન્સી ત્યારે આવે છે જ્યારે:

  \begin{itemize}
  \tightlist
  \item
Z_{1} = 2Z_{0} at

ω = ωc

  \item
    jωcL = 2R_{0} = 2\sqrt(L/C)
  \item
    ωc^{2} = 4/LC
  \item
    ωc = 2/\sqrt(LC)
  \item
    fc = 1/π\sqrt(LC)
  \end{itemize}
\item
  ડિઝાઇન સમીકરણો:

  \begin{itemize}
  \tightlist
  \item
    L = R_{0}/πfc
  \item
    C = 1/(πfcR_{0})
  \end{itemize}
\end{enumerate}

\textbf{અંતિમ સમીકરણો:}

\begin{itemize}
\tightlist
\item
  કટ-ઓફ ફ્રિક્વન્સી: fc = 1/π\sqrt(LC)
\item
  ઇન્ડક્ટન્સ: L = R_{0}/πfc
\item
  કેપેસિટન્સ: C = 1/(πfcR_{0})
\end{itemize}

\textbf{T-સેક્શન મૂલ્યો:}

\begin{itemize}
\tightlist
\item
  સીરીઝ ઇન્ડક્ટન્સ: દરેક આર્મ માં L/2
\item
  શંટ કેપેસિટન્સ: C
\end{itemize}

\textbf{π-સેક્શન મૂલ્યો:}

\begin{itemize}
\tightlist
\item
  સીરીઝ ઇન્ડક્ટન્સ: L
\item
  શંટ કેપેસિટન્સ: દરેક આર્મ માં C/2
\end{itemize}

\end{solutionbox}
\begin{mnemonicbox}
``One over Pi-Root-LC: The frequency where we Cut''

\end{mnemonicbox}

\end{document}
