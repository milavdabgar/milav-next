\documentclass[10pt,a4paper]{article}

% content/resources/templates/preamble.tex
\usepackage[margin=0.6in]{geometry}
\author{Milav Dabgar}
\usepackage{amsmath,amssymb,amsthm}
\usepackage{booktabs}
\usepackage{multirow}
\usepackage{xcolor}
\usepackage{tcolorbox}
\tcbuselibrary{breakable,skins}
\usepackage[colorlinks=true,linkcolor=blue]{hyperref}
\usepackage{titlesec}
\usepackage{enumitem}
\usepackage{tikz}
\usepackage{pgfplots}
\usepackage{circuitikz}
\usepackage[version=4]{mhchem}
\usepackage{longtable}
\usepackage{array}
\usepackage{float}
\usepackage{caption}
\usepackage{listings}

\lstset{
  basicstyle=\small\ttfamily,
  breaklines=true,
  breakatwhitespace=false,
  postbreak=\mbox{\textcolor{red}{$\hookrightarrow$}\space},
  float=false,
  numbers=left,
  numberstyle=\tiny\color{gray},
  numbersep=10pt,
  xleftmargin=2em,
  keywordstyle=\color{blue},
  commentstyle=\color{green!60!black},
  stringstyle=\color{purple},
  backgroundcolor=\color{gray!5},
  showstringspaces=false,
  tabsize=2,
  captionpos=b,
  keepspaces=true,
  columns=flexible
}

\pgfplotsset{compat=1.18}
\usetikzlibrary{shapes,arrows,positioning,calc,patterns,decorations.pathmorphing,decorations.markings,arrows.meta}

% Color scheme
\definecolor{headcolor}{RGB}{0,102,204}
\definecolor{keycolor}{RGB}{220,20,60}
\definecolor{solutioncolor}{RGB}{34,139,34}
\definecolor{mnemoniccolor}{RGB}{148,0,211}
\definecolor{codecolor}{RGB}{0,0,100}

% Spacing
\setlength{\parskip}{3pt}
\setlist[itemize]{nosep}
\setlist[enumerate]{nosep}

% Title formatting
\titleformat{\section}{\Large\bfseries\color{headcolor}}{\thesection}{1em}{}
\titleformat{\subsection}{\large\bfseries\color{headcolor}}{\thesubsection}{1em}{}

% Pandoc tightlist compatibility
\providecommand{\tightlist}{%
  \setlength{\itemsep}{0pt}\setlength{\parskip}{0pt}}

% Pandoc longtable compatibility
\newcounter{none}
\def\thenone{}


% content/resources/templates/english-boxes.tex
% This file is currently empty - it exists to maintain consistency with the import structure.
% Add custom environments here if needed in the future.


\begin{document}

\begin{center}
{\Huge\bfseries\color{headcolor} Subject Name Solutions}\\[5pt]
{\LARGE 4331101 -- Summer 2023}\\[3pt]
{\large Semester 1 Study Material}\\[3pt]
{\normalsize\textit{Detailed Solutions and Explanations}}
\end{center}

\vspace{10pt}

\subsection*{Question 1(a) [3 marks]}\label{q1a}

\textbf{Define (i) Node (ii) Branch and (iii) Loop for electronic
network.}

\begin{solutionbox}

{\def\LTcaptype{none} % do not increment counter
\begin{longtable}[]{@{}
  >{\raggedright\arraybackslash}p{(\linewidth - 2\tabcolsep) * \real{0.3333}}
  >{\raggedright\arraybackslash}p{(\linewidth - 2\tabcolsep) * \real{0.6667}}@{}}
\toprule\noalign{}
\begin{minipage}[b]{\linewidth}\raggedright
Term
\end{minipage} & \begin{minipage}[b]{\linewidth}\raggedright
Definition
\end{minipage} \\
\midrule\noalign{}
\endhead
\bottomrule\noalign{}
\endlastfoot
\textbf{Node} & A point where two or more elements are connected
together \\
\textbf{Branch} & A single element or path between two nodes \\
\textbf{Loop} & A closed path in a network where no node is traversed
more than once \\
\end{longtable}
}

\textbf{Diagram:}

\begin{center}
\textbf{Mermaid Diagram (Code)}
\begin{verbatim}
{Shaded}
{Highlighting}[]
graph LR
    A((Node A)) {-{-}{-} B((Node B)) {-}{-}{-} C((Node C)) {-}{-}{-} D((Node D)) {-}{-}{-} A}
    A {-{-}{-} C}
    style A fill:\#f9f,stroke:\#333,stroke{-width:2px}
    style B fill:\#bbf,stroke:\#333,stroke{-width:2px}
    style C fill:\#f9f,stroke:\#333,stroke{-width:2px}
    style D fill:\#bbf,stroke:\#333,stroke{-width:2px}
{Highlighting}
{Shaded}
\end{verbatim}
\end{center}

\end{solutionbox}
\begin{mnemonicbox}
``NBL: Networks Begin with Loops''

\end{mnemonicbox}
\subsection*{Question 1(b) [4 marks]}\label{q1b}

\textbf{Three resistors of 20 Ω, 30 Ω and 50 Ω are connected in parallel
across 60 V supply. Find (i) Current flowing through each resistor and
Total current (ii) Equivalent Resistance.}

\begin{solutionbox}

\textbf{Diagram:}

\begin{center}
\textbf{Mermaid Diagram (Code)}
\begin{verbatim}
{Shaded}
{Highlighting}[]
graph LR
    A[60V] {-{-}{-} B(({}+))}
    B {-{-}{-} C[20Ω] {-}{-}{-} D(({}{-}))}
    B {-{-}{-} E[30Ω] {-}{-}{-} D}
    B {-{-}{-} F[50Ω] {-}{-}{-} D}
    D {-{-}{-} A}
{Highlighting}
{Shaded}
\end{verbatim}
\end{center}

{\def\LTcaptype{none} % do not increment counter
\begin{longtable}[]{@{}ll@{}}
\toprule\noalign{}
Calculation & Value \\
\midrule\noalign{}
\endhead
\bottomrule\noalign{}
\endlastfoot
\textbf{Current through 20 Ω resistor}: I_{1} = V/R_{1} = 60/20 & 3 A \\
\textbf{Current through 30 Ω resistor}: I_{2} = V/R_{2} = 60/30 & 2 A \\
\textbf{Current through 50 Ω resistor}: I_{3} = V/R_{3} = 60/50 & 1.2 A \\
\textbf{Total current}: I = I_{1} + I_{2} + I_{3} = 3 + 2 + 1.2 & 6.2 A \\
\textbf{Equivalent resistance}: Req = V/I = 60/6.2 & 9.68 Ω \\
\end{longtable}
}

\end{solutionbox}
\begin{mnemonicbox}
``PIV: Parallel Increases the current, Voltage
remains the same''

\end{mnemonicbox}
\subsection*{Question 1(c) [7 marks]}\label{q1c}

\textbf{Explain Series and Parallel connection for Capacitors.}

\begin{solutionbox}

{\def\LTcaptype{none} % do not increment counter
\begin{longtable}[]{@{}
  >{\raggedright\arraybackslash}p{(\linewidth - 4\tabcolsep) * \real{0.3158}}
  >{\raggedright\arraybackslash}p{(\linewidth - 4\tabcolsep) * \real{0.2368}}
  >{\raggedright\arraybackslash}p{(\linewidth - 4\tabcolsep) * \real{0.4474}}@{}}
\toprule\noalign{}
\begin{minipage}[b]{\linewidth}\raggedright
Connection
\end{minipage} & \begin{minipage}[b]{\linewidth}\raggedright
Formula
\end{minipage} & \begin{minipage}[b]{\linewidth}\raggedright
Characteristics
\end{minipage} \\
\midrule\noalign{}
\endhead
\bottomrule\noalign{}
\endlastfoot
\textbf{Series Connection} & 1/C\_eq = 1/C_{1} + 1/C_{2} + 1/C_{3} + \ldots{} & -
Equivalent capacitance is less than smallest capacitor- Same current in
each capacitor- Total voltage divides across capacitors- Increases
effective dielectric strength \\
\textbf{Parallel Connection} & C\_eq = C_{1} + C_{2} + C_{3} + \ldots{} & -
Equivalent capacitance is sum of all capacitors- Same voltage across
each capacitor- Total charge is sum of individual charges- Increases
effective plate area \\
\end{longtable}
}

\textbf{Diagram:}

\begin{center}
\textbf{Mermaid Diagram (Code)}
\begin{verbatim}
{Shaded}
{Highlighting}[]
graph TD
    subgraph Series
        direction LR
        A["{+"] {-}{-}{-} B[C_{1}] {-}{-}{-} C[C_{2}] {-}{-}{-} D[C_{3}] {-}{-}{-} E["{}{-}"]}
    end

    subgraph Parallel
        F["{+"] {-}{-}{-} G[C_{1}] {-}{-}{-} H["{}{-}"]}
        F {-{-}{-} I[C_{2}] {-}{-}{-} H}
        F {-{-}{-} J[C_{3}] {-}{-}{-} H}
    end
{Highlighting}
{Shaded}
\end{verbatim}
\end{center}

\end{solutionbox}
\begin{mnemonicbox}
``CAPE: Capacitors Add in Parallel, Eliminate in
Series''

\end{mnemonicbox}
\subsection*{Question 1(c) OR [7
marks]}\label{q1c}

\textbf{Explain Series and Parallel connection for Inductors.}

\begin{solutionbox}

{\def\LTcaptype{none} % do not increment counter
\begin{longtable}[]{@{}
  >{\raggedright\arraybackslash}p{(\linewidth - 4\tabcolsep) * \real{0.3158}}
  >{\raggedright\arraybackslash}p{(\linewidth - 4\tabcolsep) * \real{0.2368}}
  >{\raggedright\arraybackslash}p{(\linewidth - 4\tabcolsep) * \real{0.4474}}@{}}
\toprule\noalign{}
\begin{minipage}[b]{\linewidth}\raggedright
Connection
\end{minipage} & \begin{minipage}[b]{\linewidth}\raggedright
Formula
\end{minipage} & \begin{minipage}[b]{\linewidth}\raggedright
Characteristics
\end{minipage} \\
\midrule\noalign{}
\endhead
\bottomrule\noalign{}
\endlastfoot
\textbf{Series Connection} & L\_eq = L_{1} + L_{2} + L_{3} + \ldots{} & -
Equivalent inductance is sum of all inductors- Same current flows
through each inductor- Total voltage is sum of individual voltages- Flux
linkage adds \\
\textbf{Parallel Connection} & 1/L\_eq = 1/L_{1} + 1/L_{2} + 1/L_{3} + \ldots{} &
- Equivalent inductance is less than smallest inductor- Same voltage
across each inductor- Total current divides among inductors- Magnetic
coupling affects actual value \\
\end{longtable}
}

\textbf{Diagram:}

\begin{center}
\textbf{Mermaid Diagram (Code)}
\begin{verbatim}
{Shaded}
{Highlighting}[]
graph TD
    subgraph Series
        direction LR    
        A["{+"] {-}{-}{-} B((L_{1})) {-}{-}{-} C((L_{2})) {-}{-}{-} D((L_{3})) {-}{-}{-} E["{}{-}"]}
    end

    subgraph Parallel
        F["{+"] {-}{-}{-} G((L_{1})) {-}{-}{-} H["{}{-}"]}
        F {-{-}{-} I((L_{2})) {-}{-}{-} H}
        F {-{-}{-} J((L_{3})) {-}{-}{-} H}
    end
{Highlighting}
{Shaded}
\end{verbatim}
\end{center}

\end{solutionbox}
\begin{mnemonicbox}
``LIPS: inductors Link in Series, Partition in
Parallel''

\end{mnemonicbox}
\subsection*{Question 2(a) [3 marks]}\label{q2a}

\textbf{Define (i) Transform impedance, (ii) Driving point impedance,
(iii) Transfer impedance.}

\begin{solutionbox}

{\def\LTcaptype{none} % do not increment counter
\begin{longtable}[]{@{}
  >{\raggedright\arraybackslash}p{(\linewidth - 2\tabcolsep) * \real{0.3333}}
  >{\raggedright\arraybackslash}p{(\linewidth - 2\tabcolsep) * \real{0.6667}}@{}}
\toprule\noalign{}
\begin{minipage}[b]{\linewidth}\raggedright
Term
\end{minipage} & \begin{minipage}[b]{\linewidth}\raggedright
Definition
\end{minipage} \\
\midrule\noalign{}
\endhead
\bottomrule\noalign{}
\endlastfoot
\textbf{Transform impedance} & Impedance seen by signal passing from
primary to secondary of a transformer \\
\textbf{Driving point impedance} & Ratio of voltage to current at the
same pair of terminals or port \\
\textbf{Transfer impedance} & Ratio of voltage at one port to the
current at another port \\
\end{longtable}
}

\textbf{Diagram:}

\begin{center}
\textbf{Mermaid Diagram (Code)}
\begin{verbatim}
{Shaded}
{Highlighting}[]
graph TD
    A[Input] {-{-}{-} B[Two Port Network] {-}{-}{-} C[Output]}
    D[Z11: Driving point impedance] {-.{-}{} B}
    E[Z21: Transfer impedance] {-.{-}{} B}
    F[Z12: Transfer impedance] {-.{-}{} B}
    G[Z22: Driving point impedance] {-.{-}{} B}
{Highlighting}
{Shaded}
\end{verbatim}
\end{center}

\end{solutionbox}
\begin{mnemonicbox}
``TDT: Transformers Drive Transfers''

\end{mnemonicbox}
\subsection*{Question 2(b) [4 marks]}\label{q2b}

\textbf{Three resistances of 30, 50 and 90 ohms are connected in star.
Find equivalent resistances in delta connection.}

\begin{solutionbox}

\textbf{Diagram:}

\begin{center}
\textbf{Mermaid Diagram (Code)}
\begin{verbatim}
{Shaded}
{Highlighting}[]
graph LR
    A((A)) {-{-}{-} B[R_{1}=30Ω] {-}{-}{-} D((D))}
    B((B)) {-{-}{-} C[R_{2}=50Ω] {-}{-}{-} D}
    C((C)) {-{-}{-} E[R_{3}=90Ω] {-}{-}{-} D}

    subgraph Equivalent Delta
        A {-{-}{-} F[R_{1}_{2}] {-}{-}{-} B}
        B {-{-}{-} G[R_{2}_{3}] {-}{-}{-} C}
        C {-{-}{-} H[R_{3}_{1}] {-}{-}{-} A}
    end
{Highlighting}
{Shaded}
\end{verbatim}
\end{center}

{\def\LTcaptype{none} % do not increment counter
\begin{longtable}[]{@{}
  >{\raggedright\arraybackslash}p{(\linewidth - 4\tabcolsep) * \real{0.6182}}
  >{\raggedright\arraybackslash}p{(\linewidth - 4\tabcolsep) * \real{0.2364}}
  >{\raggedright\arraybackslash}p{(\linewidth - 4\tabcolsep) * \real{0.1455}}@{}}
\toprule\noalign{}
\begin{minipage}[b]{\linewidth}\raggedright
Star to Delta Conversion Formula
\end{minipage} & \begin{minipage}[b]{\linewidth}\raggedright
Calculation
\end{minipage} & \begin{minipage}[b]{\linewidth}\raggedright
Result
\end{minipage} \\
\midrule\noalign{}
\endhead
\bottomrule\noalign{}
\endlastfoot
R_{1}_{2} = (R_{1}\timesR_{2} + R_{2}\timesR_{3} + R_{3}\timesR_{1})/R_{3} & (30\times50 + 50\times90 + 90\times30)/90 & 105 Ω \\
R_{2}_{3} = (R_{1}\timesR_{2} + R_{2}\timesR_{3} + R_{3}\timesR_{1})/R_{1} & (30\times50 + 50\times90 + 90\times30)/30 & 315 Ω \\
R_{3}_{1} = (R_{1}\timesR_{2} + R_{2}\timesR_{3} + R_{3}\timesR_{1})/R_{2} & (30\times50 + 50\times90 + 90\times30)/50 & 189 Ω \\
\end{longtable}
}

\end{solutionbox}
\begin{mnemonicbox}
``PSR: Product over Sum of Resistors''

\end{mnemonicbox}
\subsection*{Question 2(c) [7 marks]}\label{q2c}

\textbf{Explain π network.}

\begin{solutionbox}

{\def\LTcaptype{none} % do not increment counter
\begin{longtable}[]{@{}
  >{\raggedright\arraybackslash}p{(\linewidth - 2\tabcolsep) * \real{0.4091}}
  >{\raggedright\arraybackslash}p{(\linewidth - 2\tabcolsep) * \real{0.5909}}@{}}
\toprule\noalign{}
\begin{minipage}[b]{\linewidth}\raggedright
Concept
\end{minipage} & \begin{minipage}[b]{\linewidth}\raggedright
Description
\end{minipage} \\
\midrule\noalign{}
\endhead
\bottomrule\noalign{}
\endlastfoot
\textbf{Definition} & A three-terminal network formed by three
impedances - one in series and two in parallel \\
\textbf{Structure} & Two impedances connected from input and output to
common point, one between input and output \\
\textbf{Parameters} & Can be defined using Z, Y, h, or ABCD
parameters \\
\textbf{Applications} & Matching networks, filters, attenuators, phase
shifters \\
\end{longtable}
}

\textbf{Diagram:}

\begin{center}
\textbf{Mermaid Diagram (Code)}
\begin{verbatim}
{Shaded}
{Highlighting}[]
graph LR
    A[Input] {-{-}{-} C[Z_{2}] {-}{-}{-} B[Output]}
    A {-{-}{-} D[Z_{1}] {-}{-}{-} E[Common/Ground]}
    B {-{-}{-} F[Z_{3}] {-}{-}{-} E}

    style D fill:\#bbf,stroke:\#333,stroke{-width:2px}
    style C fill:\#f96,stroke:\#333,stroke{-width:2px}
    style F fill:\#bbf,stroke:\#333,stroke{-width:2px}
{Highlighting}
{Shaded}
\end{verbatim}
\end{center}

\end{solutionbox}
\begin{mnemonicbox}
``PIE: Pi Impedances connected at Ends''

\end{mnemonicbox}
\subsection*{Question 2(a) OR [3
marks]}\label{q2a}

\textbf{List the types of network.}

\begin{solutionbox}

{\def\LTcaptype{none} % do not increment counter
\begin{longtable}[]{@{}
  >{\raggedright\arraybackslash}p{(\linewidth - 2\tabcolsep) * \real{0.5833}}
  >{\raggedright\arraybackslash}p{(\linewidth - 2\tabcolsep) * \real{0.4167}}@{}}
\toprule\noalign{}
\begin{minipage}[b]{\linewidth}\raggedright
Network Types
\end{minipage} & \begin{minipage}[b]{\linewidth}\raggedright
Examples
\end{minipage} \\
\midrule\noalign{}
\endhead
\bottomrule\noalign{}
\endlastfoot
\textbf{Based on Linearity} & Linear networks, Non-linear networks \\
\textbf{Based on Components} & Passive networks, Active networks \\
\textbf{Based on Structure} & Lumped networks, Distributed networks \\
\textbf{Based on Behavior} & Bilateral networks, Unilateral networks \\
\textbf{Based on Topology} & T-networks, π-networks, Lattice networks \\
\textbf{Based on Ports} & One-port networks, Two-port networks,
Multi-port networks \\
\end{longtable}
}

\textbf{Diagram:}

\begin{center}
\textbf{Mermaid Diagram (Code)}
\begin{verbatim}
{Shaded}
{Highlighting}[]
graph TD
    A[Network Types] {-{-}{} B[Linear/Non{-}linear]}
    A {-{-}{} C[Passive/Active]}
    A {-{-}{} D[Lumped/Distributed]}
    A {-{-}{} E[Bilateral/Unilateral]}
    A {-{-}{} F[T/π/Lattice]}
    A {-{-}{} G[One{-}port/Two{-}port/Multi{-}port]}
{Highlighting}
{Shaded}
\end{verbatim}
\end{center}

\end{solutionbox}
\begin{mnemonicbox}
``PLAN-TB:
Passive-Linear-Active-Network-Topology-Bilateral''

\end{mnemonicbox}
\subsection*{Question 2(b) OR [4
marks]}\label{q2b}

\textbf{Three resistances of 40, 60 and 80 ohms are connected in delta.
Find equivalent resistances in star connection.}

\begin{solutionbox}

\textbf{Diagram:}

\begin{center}
\textbf{Mermaid Diagram (Code)}
\begin{verbatim}
{Shaded}
{Highlighting}[]
graph TD
    A((A)) {-{-}{-} B[R_{1}_{2}=40Ω] {-}{-}{-} B((B))}
    B {-{-}{-} C[R_{2}_{3}=60Ω] {-}{-}{-} C((C))}
    C {-{-}{-} D[R_{3}_{1}=80Ω] {-}{-}{-} A}

    subgraph Equivalent Star
        A {-{-}{-} E[R_{1}] {-}{-}{-} G((D))}
        B {-{-}{-} F[R_{2}] {-}{-}{-} G}
        C {-{-}{-} H[R_{3}] {-}{-}{-} G}
    end
{Highlighting}
{Shaded}
\end{verbatim}
\end{center}

{\def\LTcaptype{none} % do not increment counter
\begin{longtable}[]{@{}lll@{}}
\toprule\noalign{}
Delta to Star Conversion Formula & Calculation & Result \\
\midrule\noalign{}
\endhead
\bottomrule\noalign{}
\endlastfoot
R_{1} = (R_{1}_{2}\timesR_{3}_{1})/(R_{1}_{2}+R_{2}_{3}+R_{3}_{1}) & (40\times80)/(40+60+80) & 17.78 Ω \\
R_{2} = (R_{1}_{2}\timesR_{2}_{3})/(R_{1}_{2}+R_{2}_{3}+R_{3}_{1}) & (40\times60)/(40+60+80) & 13.33 Ω \\
R_{3} = (R_{2}_{3}\timesR_{3}_{1})/(R_{1}_{2}+R_{2}_{3}+R_{3}_{1}) & (60\times80)/(40+60+80) & 26.67 Ω \\
\end{longtable}
}

\end{solutionbox}
\begin{mnemonicbox}
``DPS: Delta Product over Sum''

\end{mnemonicbox}
\subsection*{Question 2(c) OR [7
marks]}\label{q2c}

\textbf{Explain characteristic impedance of symmetrical T -- network.
Also derive the equation of ZOT in terms of ZOC and ZSC.}

\begin{solutionbox}

{\def\LTcaptype{none} % do not increment counter
\begin{longtable}[]{@{}
  >{\raggedright\arraybackslash}p{(\linewidth - 2\tabcolsep) * \real{0.4091}}
  >{\raggedright\arraybackslash}p{(\linewidth - 2\tabcolsep) * \real{0.5909}}@{}}
\toprule\noalign{}
\begin{minipage}[b]{\linewidth}\raggedright
Concept
\end{minipage} & \begin{minipage}[b]{\linewidth}\raggedright
Description
\end{minipage} \\
\midrule\noalign{}
\endhead
\bottomrule\noalign{}
\endlastfoot
\textbf{Characteristic impedance (Z_{0})} & Impedance that when connected
at output port causes input impedance to equal Z_{0} \\
\textbf{Symmetrical T-network} & T-network where the series impedances
on both sides are equal \\
\textbf{ZOC and ZSC} & Open-circuit and short-circuit impedances of the
network \\
\end{longtable}
}

\textbf{Diagram:}

\begin{center}
\textbf{Mermaid Diagram (Code)}
\begin{verbatim}
{Shaded}
{Highlighting}[]
graph LR
    A[Input] {-{-}{-} B[Z_{1}] {-}{-}{-} C((Middle)) {-}{-}{-} D[Z_{1}] {-}{-}{-} E[Output]}
    C {-{-}{-} F[Z_{2}] {-}{-}{-} G[Ground]}
    H[Z_{0] {-}.{-}{} E}
{Highlighting}
{Shaded}
\end{verbatim}
\end{center}

For a symmetrical T-network:

\begin{itemize}
\tightlist
\item
  Series impedances (Z_{1}) are equal
\item
  Z_{2} is the shunt impedance
\end{itemize}

The characteristic impedance (Z_{0}ᵀ) is given by: Z_{0}ᵀ = \sqrt(Z_{0}ᶜ \times Z_{0}ˢᶜ)

Where:

\begin{itemize}
\tightlist
\item
  Z_{0}ᶜ = Open circuit impedance = Z_{1} + Z_{2} + (Z_{1}\timesZ_{2})/Z_{1} = Z_{1} + Z_{2}
\item
  Z_{0}ˢᶜ = Short circuit impedance = Z_{1}^{2}/Z_{2}
\end{itemize}

Therefore: Z_{0}ᵀ = \sqrt[(Z_{1} + Z_{2}) \times Z_{1}^{2}/Z_{2}] = \sqrt[Z_{1}^{2} + Z_{1}\timesZ_{2}]

\end{solutionbox}
\begin{mnemonicbox}
``TOSS: T-network's Open and Short circuit
Square-root''

\end{mnemonicbox}
\subsection*{Question 3(a) [3 marks]}\label{q3a}

\textbf{Explain Kirchhoff's law.}

\begin{solutionbox}

{\def\LTcaptype{none} % do not increment counter
\begin{longtable}[]{@{}
  >{\raggedright\arraybackslash}p{(\linewidth - 4\tabcolsep) * \real{0.1724}}
  >{\raggedright\arraybackslash}p{(\linewidth - 4\tabcolsep) * \real{0.3793}}
  >{\raggedright\arraybackslash}p{(\linewidth - 4\tabcolsep) * \real{0.4483}}@{}}
\toprule\noalign{}
\begin{minipage}[b]{\linewidth}\raggedright
Law
\end{minipage} & \begin{minipage}[b]{\linewidth}\raggedright
Statement
\end{minipage} & \begin{minipage}[b]{\linewidth}\raggedright
Application
\end{minipage} \\
\midrule\noalign{}
\endhead
\bottomrule\noalign{}
\endlastfoot
\textbf{Kirchhoff's Current Law (KCL)} & Sum of currents entering a node
equals sum of currents leaving it & Used for nodal analysis \\
\textbf{Kirchhoff's Voltage Law (KVL)} & Sum of voltages around any
closed loop equals zero & Used for mesh analysis \\
\end{longtable}
}

\textbf{Diagram:}

\begin{center}
\textbf{Mermaid Diagram (Code)}
\begin{verbatim}
{Shaded}
{Highlighting}[]
graph TD
    subgraph "Current Law"
        A((Node)) {-{-}{-} B[I_{1}]}
        A {-{-}{-} C[I_{2}]}
        A {-{-}{-} D[I_{3}]}
        A {-{-}{-} E[I_{4}]}
    end

    subgraph "Voltage Law"
            direction LR
        F[V_{1] {-}{-}{-} G[V_{2}] {-}{-}{-} H[V_{3}] {-}{-}{-} I[V_{4}] {-}{-}{-} F}
    end
{Highlighting}
{Shaded}
\end{verbatim}
\end{center}

\end{solutionbox}
\begin{mnemonicbox}
``KVC: Kirchhoff Verifies Current and Voltage laws''

\end{mnemonicbox}
\subsection*{Question 3(b) [4 marks]}\label{q3b}

\textbf{Explain Mesh analysis.}

\begin{solutionbox}

{\def\LTcaptype{none} % do not increment counter
\begin{longtable}[]{@{}
  >{\raggedright\arraybackslash}p{(\linewidth - 2\tabcolsep) * \real{0.4091}}
  >{\raggedright\arraybackslash}p{(\linewidth - 2\tabcolsep) * \real{0.5909}}@{}}
\toprule\noalign{}
\begin{minipage}[b]{\linewidth}\raggedright
Concept
\end{minipage} & \begin{minipage}[b]{\linewidth}\raggedright
Description
\end{minipage} \\
\midrule\noalign{}
\endhead
\bottomrule\noalign{}
\endlastfoot
\textbf{Definition} & Method to solve circuit problems by applying KVL
to each independent closed loop (mesh) \\
\textbf{Procedure} & 1. Assign mesh currents to each loop2. Write KVL
equations for each mesh3. Solve the resulting system of equations \\
\textbf{Advantages} & - Reduces number of equations- Works well with
circuits having many branches- Suitable for problems with voltage
sources \\
\end{longtable}
}

\textbf{Diagram:}

\begin{center}
\textbf{Mermaid Diagram (Code)}
\begin{verbatim}
{Shaded}
{Highlighting}[]
graph LR
    A(({+)) {-}{-}{-} B[R_{1}] {-}{-}{-} C((B)) {-}{-}{-} D[R_{2}] {-}{-}{-} E((C))}
    E {-{-}{-} F[R_{3}] {-}{-}{-} G((D)) {-}{-}{-} H[R_{4}] {-}{-}{-} A}
    C {-{-}{-} I[R_{5}] {-}{-}{-} G}

    J[I_{1] {-}.{-}{} B}
    K[I_{2] {-}.{-}{} D}
    L[I_{3] {-}.{-}{} I}
{Highlighting}
{Shaded}
\end{verbatim}
\end{center}

\end{solutionbox}
\begin{mnemonicbox}
``MAIL: Mesh Analysis uses Independent Loops''

\end{mnemonicbox}
\subsection*{Question 3(c) [7 marks]}\label{q3c}

\textbf{Use Thevenin's theorem to find current through the 5 Ω resistor
for given circuit.}

\begin{solutionbox}

\textbf{Diagram:}

\begin{verbatim}
          10Ω        15Ω
          {-{-}{-}{-}      {-}{-}{-}{-}}
         /    {    /    }
        /      {  /      }
  100V +        A        B
        {       |        /}
         {     {-}{-}{-}      /}
          {   | 5Ω|    /}
           {  |   |   /}
            { {-}{-}{-}   /}
             {|   |/}
              6Ω  8Ω
\end{verbatim}

\textbf{Step 1:} Remove 5Ω resistor and find open circuit voltage (V_{t}_{h})
\textbf{Step 2:} Find Thevenin's equivalent resistance (R_{t}_{h})
\textbf{Step 3:} Calculate current through 5Ω resistor

{\def\LTcaptype{none} % do not increment counter
\begin{longtable}[]{@{}
  >{\raggedright\arraybackslash}p{(\linewidth - 4\tabcolsep) * \real{0.2222}}
  >{\raggedright\arraybackslash}p{(\linewidth - 4\tabcolsep) * \real{0.4815}}
  >{\raggedright\arraybackslash}p{(\linewidth - 4\tabcolsep) * \real{0.2963}}@{}}
\toprule\noalign{}
\begin{minipage}[b]{\linewidth}\raggedright
Step
\end{minipage} & \begin{minipage}[b]{\linewidth}\raggedright
Calculation
\end{minipage} & \begin{minipage}[b]{\linewidth}\raggedright
Result
\end{minipage} \\
\midrule\noalign{}
\endhead
\bottomrule\noalign{}
\endlastfoot
\textbf{V_{t}_{h}} & Voltage between A and B with 5Ω removed & 38.46 V \\
\textbf{R_{t}_{h}} & Equivalent resistance seen from A and B with 100V source
shorted & 3.6 Ω \\
\textbf{Current} & I = V_{t}_{h}/(R_{t}_{h} + 5) = 38.46/(3.6 + 5) & 4.47 A \\
\end{longtable}
}

\end{solutionbox}
\begin{mnemonicbox}
``TVR: Thevenin replaces Voltage and Resistance''

\end{mnemonicbox}
\subsection*{Question 3(a) OR [3
marks]}\label{q3a}

\textbf{State and explain Superposition Theorem.}

\begin{solutionbox}

{\def\LTcaptype{none} % do not increment counter
\begin{longtable}[]{@{}
  >{\raggedright\arraybackslash}p{(\linewidth - 2\tabcolsep) * \real{0.4091}}
  >{\raggedright\arraybackslash}p{(\linewidth - 2\tabcolsep) * \real{0.5909}}@{}}
\toprule\noalign{}
\begin{minipage}[b]{\linewidth}\raggedright
Concept
\end{minipage} & \begin{minipage}[b]{\linewidth}\raggedright
Description
\end{minipage} \\
\midrule\noalign{}
\endhead
\bottomrule\noalign{}
\endlastfoot
\textbf{Statement} & In a linear circuit with multiple sources, the
response at any point equals the sum of responses caused by each source
acting alone \\
\textbf{Procedure} & 1. Consider one source at a time2. Replace other
voltage sources with short circuits3. Replace other current sources with
open circuits4. Find individual responses5. Add all responses
algebraically \\
\textbf{Limitation} & Only applicable to linear circuits and for
voltage/current responses \\
\end{longtable}
}

\textbf{Diagram:}

\begin{center}
\textbf{Mermaid Diagram (Code)}
\begin{verbatim}
{Shaded}
{Highlighting}[]
graph LR
    A[Original Circuit] {-{-}{} B[Circuit with V_{1} only]}
    A {-{-}{} C[Circuit with V_{2} only]}
    B {-{-}{} D[Response R_{1}]}
    C {-{-}{} E[Response R_{2}]}
    D {-{-}{} F[Total Response = R_{1} + R_{2}]}
    E {-{-}{} F}
{Highlighting}
{Shaded}
\end{verbatim}
\end{center}

\end{solutionbox}
\begin{mnemonicbox}
``SUPER: Sources Used Progressively Equals Response''

\end{mnemonicbox}
\subsection*{Question 3(b) OR [4
marks]}\label{q3b}

\textbf{Explain method of drawing dual network using any circuit.}

\begin{solutionbox}

{\def\LTcaptype{none} % do not increment counter
\begin{longtable}[]{@{}
  >{\raggedright\arraybackslash}p{(\linewidth - 2\tabcolsep) * \real{0.3158}}
  >{\raggedright\arraybackslash}p{(\linewidth - 2\tabcolsep) * \real{0.6842}}@{}}
\toprule\noalign{}
\begin{minipage}[b]{\linewidth}\raggedright
Step
\end{minipage} & \begin{minipage}[b]{\linewidth}\raggedright
Description
\end{minipage} \\
\midrule\noalign{}
\endhead
\bottomrule\noalign{}
\endlastfoot
\textbf{Convert to graph} & Draw the circuit as a planar graph \\
\textbf{Draw dual graph} & Place a node in each region of original
graph \\
\textbf{Connect nodes} & Draw edges crossing each edge of original
graph \\
\textbf{Replace elements} & - Resistance R becomes conductance 1/R-
Voltage source becomes current source- Series becomes parallel-
Impedance Z becomes admittance 1/Z \\
\end{longtable}
}

\textbf{Diagram:}

\begin{verbatim}
Original Circuit     Dual Circuit
   +{-{-}{-}R1{-}{-}{-}+         +{-}{-}{-}G1{-}{-}{-}+}
   |        |         |        |
  V1       R2        I1       G2
   |        |         |        |
   +{-{-}{-}R3{-}{-}{-}+         +{-}{-}{-}G3{-}{-}{-}+}
\end{verbatim}

\end{solutionbox}
\begin{mnemonicbox}
``DVSG: Dual transforms Voltage to Series to Graphs''

\end{mnemonicbox}
\subsection*{Question 3(c) OR [7
marks]}\label{q3c}

\textbf{Find out Norton's equivalent circuit for the given network. Find
out load current if (i) RL = 3 KΩ (ii) RL = 1.5 Ω}

\begin{solutionbox}

\textbf{Diagram:}

\begin{verbatim}
        2kΩ          2kΩ          2kΩ
       {-{-}{-}{-}{-}        {-}{-}{-}{-}{-}        {-}{-}{-}{-}{-}}
      /     {      /           /     }
     /       {    /           /       }
  C +         D  +         E  +         A
     {                               }
      {                               }
       {          |         | |         |}
       |         | |         | |         |
      10V         2kΩ         2kΩ         RL
       |         | |         | |         |
       |         | |         | |         |
       +         + +         + +         +
       B         B B         B B         B
\end{verbatim}

\textbf{Step 1:} Find Norton's current (IN) \textbf{Step 2:} Find
Norton's resistance (RN) \textbf{Step 3:} Calculate load currents

{\def\LTcaptype{none} % do not increment counter
\begin{longtable}[]{@{}
  >{\raggedright\arraybackslash}p{(\linewidth - 4\tabcolsep) * \real{0.2222}}
  >{\raggedright\arraybackslash}p{(\linewidth - 4\tabcolsep) * \real{0.4815}}
  >{\raggedright\arraybackslash}p{(\linewidth - 4\tabcolsep) * \real{0.2963}}@{}}
\toprule\noalign{}
\begin{minipage}[b]{\linewidth}\raggedright
Step
\end{minipage} & \begin{minipage}[b]{\linewidth}\raggedright
Calculation
\end{minipage} & \begin{minipage}[b]{\linewidth}\raggedright
Result
\end{minipage} \\
\midrule\noalign{}
\endhead
\bottomrule\noalign{}
\endlastfoot
\textbf{IN} & Short circuit current from A to B & 1.25 mA \\
\textbf{RN} & Equivalent resistance seen from A to B with 10V source
shorted & 1 kΩ \\
\textbf{IL (RL = 3 KΩ)} & IL = IN \times RN/(RN + RL) = 1.25 \times 1/(1 + 3) &
0.31 mA \\
\textbf{IL (RL = 1.5 Ω)} & IL = IN \times RN/(RN + RL) = 1.25 \times 1000/(1000 +
1.5) & 1.25 mA \\
\end{longtable}
}

\end{solutionbox}
\begin{mnemonicbox}
``NICE: Norton's circuit Is Current Equivalent''

\end{mnemonicbox}
\subsection*{Question 4(a) [3 marks]}\label{q4a}

\textbf{Derive the equation of Quality factor Q for a coil.}

\begin{solutionbox}

{\def\LTcaptype{none} % do not increment counter
\begin{longtable}[]{@{}
  >{\raggedright\arraybackslash}p{(\linewidth - 2\tabcolsep) * \real{0.4400}}
  >{\raggedright\arraybackslash}p{(\linewidth - 2\tabcolsep) * \real{0.5600}}@{}}
\toprule\noalign{}
\begin{minipage}[b]{\linewidth}\raggedright
Parameter
\end{minipage} & \begin{minipage}[b]{\linewidth}\raggedright
Relationship
\end{minipage} \\
\midrule\noalign{}
\endhead
\bottomrule\noalign{}
\endlastfoot
\textbf{Q factor definition} & Ratio of energy stored to energy
dissipated per cycle \\
\textbf{Coil impedance} & Z = R + jωL \\
\textbf{Reactance} & XL = ωL \\
\textbf{Quality factor} & Q = XL/R = ωL/R \\
\end{longtable}
}

\textbf{Diagram:}

\begin{verbatim}
    +{-{-}{-}R{-}{-}{-}+}
    |       |
    +{-{-}L{-}{-}{-}{-}+}
\end{verbatim}

For a coil, the energy stored is in the magnetic field (in the
inductor), while energy dissipated is in the resistance. From this:

Q = 2π \times (Energy stored)/(Energy dissipated per cycle)

Q = ωL/R


\end{solutionbox}
\begin{mnemonicbox}
``QREL: Quality Relates Energy to Loss''

\end{mnemonicbox}
\subsection*{Question 4(b) [4 marks]}\label{q4b}

\textbf{A series RLC circuit has R = 30 Ω, L = 0.5 H and C = 5 µF.
Calculate (i) Q factor, (ii) BW, (iii) Upper cut off and lower cut off
frequencies.}

\begin{solutionbox}

\textbf{Diagram:}

\begin{center}
\textbf{Mermaid Diagram (Code)}
\begin{verbatim}
{Shaded}
{Highlighting}[]
graph LR
    A[Input] {-{-}{-} B[R=30Ω] {-}{-}{-} C[L=0.5H] {-}{-}{-} D[C=5µF] {-}{-}{-} E[Output]}
{Highlighting}
{Shaded}
\end{verbatim}
\end{center}

{\def\LTcaptype{none} % do not increment counter
\begin{longtable}[]{@{}
  >{\raggedright\arraybackslash}p{(\linewidth - 6\tabcolsep) * \real{0.2683}}
  >{\raggedright\arraybackslash}p{(\linewidth - 6\tabcolsep) * \real{0.2195}}
  >{\raggedright\arraybackslash}p{(\linewidth - 6\tabcolsep) * \real{0.3171}}
  >{\raggedright\arraybackslash}p{(\linewidth - 6\tabcolsep) * \real{0.1951}}@{}}
\toprule\noalign{}
\begin{minipage}[b]{\linewidth}\raggedright
Parameter
\end{minipage} & \begin{minipage}[b]{\linewidth}\raggedright
Formula
\end{minipage} & \begin{minipage}[b]{\linewidth}\raggedright
Calculation
\end{minipage} & \begin{minipage}[b]{\linewidth}\raggedright
Result
\end{minipage} \\
\midrule\noalign{}
\endhead
\bottomrule\noalign{}
\endlastfoot
\textbf{Resonant frequency (f_{0})} & f_{0} = 1/(2π\sqrtLC) & 1/(2π\sqrt(0.5\times5\times10^{-}^{6}))
& 100.53 Hz \\
\textbf{Q factor} & Q = (1/R)\sqrt(L/C) & (1/30)\sqrt(0.5/(5\times10^{-}^{6})) & 105.57 \\
\textbf{Bandwidth (BW)} & BW = f_{0}/Q & 100.53/105.57 & 0.952 Hz \\
\textbf{Lower cutoff (f_{1})} & f_{1} = f_{0} - BW/2 & 100.53 - 0.952/2 & 100.05
Hz \\
\textbf{Upper cutoff (f_{2})} & f_{2} = f_{0} + BW/2 & 100.53 + 0.952/2 & 101.01
Hz \\
\end{longtable}
}

\end{solutionbox}
\begin{mnemonicbox}
``QBCUT: Quality Bandwidth Cutoff Uniquely Related''

\end{mnemonicbox}
\subsection*{Question 4(c) [7 marks]}\label{q4c}

\textbf{Explain Mutual Inductance along with Co-efficient of mutual
inductance. Also derive the equation of K.}

\begin{solutionbox}

{\def\LTcaptype{none} % do not increment counter
\begin{longtable}[]{@{}
  >{\raggedright\arraybackslash}p{(\linewidth - 2\tabcolsep) * \real{0.4091}}
  >{\raggedright\arraybackslash}p{(\linewidth - 2\tabcolsep) * \real{0.5909}}@{}}
\toprule\noalign{}
\begin{minipage}[b]{\linewidth}\raggedright
Concept
\end{minipage} & \begin{minipage}[b]{\linewidth}\raggedright
Description
\end{minipage} \\
\midrule\noalign{}
\endhead
\bottomrule\noalign{}
\endlastfoot
\textbf{Mutual Inductance (M)} & Property where current change in one
coil induces voltage in adjacent coil \\
\textbf{Definition} & Ratio of induced voltage in secondary to rate of
change of current in primary \\
\textbf{Formula} & M = k\sqrt(L_{1}L_{2}) \\
\textbf{Coefficient of coupling (k)} & Measure of magnetic coupling
between coils (0 \leq k \leq 1) \\
\end{longtable}
}

\textbf{Diagram:}

\begin{center}
\textbf{Mermaid Diagram (Code)}
\begin{verbatim}
{Shaded}
{Highlighting}[]
graph LR
    A[I_{1] {-}{-}{-} B[L_{1}] {-}{-}{-} C}
    D[I_{2] {-}{-}{-} E[L_{2}] {-}{-}{-} F}

    G[Mutual Inductance M] {-.{-}{} B}
    G {-.{-}{} E}
{Highlighting}
{Shaded}
\end{verbatim}
\end{center}

For two inductors L_{1} and L_{2}, mutual inductance M is: M = k\sqrt(L_{1}L_{2})

Where coefficient of coupling k is: k = M/\sqrt(L_{1}L_{2})

k represents fraction of magnetic flux from one coil linking with
another coil. For perfectly coupled coils,

k = 1 For no coupling,

k = 0


\end{solutionbox}
\begin{mnemonicbox}
``MKL: Mutual coupling K Links inductors''

\end{mnemonicbox}
\subsection*{Question 4(a) OR [3
marks]}\label{q4a}

\textbf{Explain the types of coupling for coupled circuit.}

\begin{solutionbox}

{\def\LTcaptype{none} % do not increment counter
\begin{longtable}[]{@{}
  >{\raggedright\arraybackslash}p{(\linewidth - 4\tabcolsep) * \real{0.3673}}
  >{\raggedright\arraybackslash}p{(\linewidth - 4\tabcolsep) * \real{0.3469}}
  >{\raggedright\arraybackslash}p{(\linewidth - 4\tabcolsep) * \real{0.2857}}@{}}
\toprule\noalign{}
\begin{minipage}[b]{\linewidth}\raggedright
Type of Coupling
\end{minipage} & \begin{minipage}[b]{\linewidth}\raggedright
Characteristics
\end{minipage} & \begin{minipage}[b]{\linewidth}\raggedright
Applications
\end{minipage} \\
\midrule\noalign{}
\endhead
\bottomrule\noalign{}
\endlastfoot
\textbf{Tight/Close Coupling (k\approx1)} & - Nearly all flux links both
coils- High transfer efficiency- k value close to 1 & Transformers,
Power transfer \\
\textbf{Loose Coupling (k≪1)} & - Small fraction of flux links second
coil- Lower transfer efficiency- k value much less than 1 & RF circuits,
Tuned filters \\
\textbf{Critical Coupling (k=kc)} & - Optimum coupling for bandpass
response- Maximum power transfer at resonance & Bandpass filters, IF
transformers \\
\textbf{Inductive Coupling} & - Coupling via magnetic field &
Transformers, Wireless charging \\
\textbf{Capacitive Coupling} & - Coupling via electric field & Signal
coupling, Capacitive sensors \\
\end{longtable}
}

\textbf{Diagram:}

\begin{center}
\textbf{Mermaid Diagram (Code)}
\begin{verbatim}
{Shaded}
{Highlighting}[]
graph LR
    subgraph "Tight Coupling"
        direction LR
        A1[Primary] {-{-}{-} B1[k  1] {-}{-}{-} C1[Secondary]}
    end

    subgraph "Loose Coupling"
        direction LR
        A2[Primary] {-{-}{-} B2[k ≪ 1] {-}{-}{-} C2[Secondary]}
    end
    
    subgraph "Critical Coupling"
        direction LR    
        A3[Primary] {-{-}{-} B3[k = kc] {-}{-}{-} C3[Secondary]}
    end
{Highlighting}
{Shaded}
\end{verbatim}
\end{center}

\end{solutionbox}
\begin{mnemonicbox}
``TLC: Tight, Loose, Critical couplings''

\end{mnemonicbox}
\subsection*{Question 4(b) OR [4
marks]}\label{q4b}

\textbf{A parallel resonant circuit having inductance of 1 mH with
quality factor

Q = 100, resonant frequency Fr = 100 KHz. Find out (i)

Required capacitance C, (ii) Resistance R of the coil, (iii) BW.}

\begin{solutionbox}

\textbf{Diagram:}

\begin{center}
\textbf{Mermaid Diagram (Code)}
\begin{verbatim}
{Shaded}
{Highlighting}[]
graph TD
    A[Input] {-{-}{-} B((Node))}
    B {-{-}{-} C[L=1mH]}
    B {-{-}{-} D[C=?]}
    B {-{-}{-} E[Output]}
    C {-{-}{-} F[R=?]}
{Highlighting}
{Shaded}
\end{verbatim}
\end{center}

{\def\LTcaptype{none} % do not increment counter
\begin{longtable}[]{@{}
  >{\raggedright\arraybackslash}p{(\linewidth - 6\tabcolsep) * \real{0.2683}}
  >{\raggedright\arraybackslash}p{(\linewidth - 6\tabcolsep) * \real{0.2195}}
  >{\raggedright\arraybackslash}p{(\linewidth - 6\tabcolsep) * \real{0.3171}}
  >{\raggedright\arraybackslash}p{(\linewidth - 6\tabcolsep) * \real{0.1951}}@{}}
\toprule\noalign{}
\begin{minipage}[b]{\linewidth}\raggedright
Parameter
\end{minipage} & \begin{minipage}[b]{\linewidth}\raggedright
Formula
\end{minipage} & \begin{minipage}[b]{\linewidth}\raggedright
Calculation
\end{minipage} & \begin{minipage}[b]{\linewidth}\raggedright
Result
\end{minipage} \\
\midrule\noalign{}
\endhead
\bottomrule\noalign{}
\endlastfoot
\textbf{Capacitance (C)} & C = 1/(4π^{2}f^{2}L) & 1/(4π^{2}\times(100\times10^{3})^{2}\times1\times10^{-}^{3}) &
2.533 nF \\
\textbf{Coil Resistance (R)} & R = ωL/Q & 2π\times100\times10^{3}\times1\times10^{-}^{3}/100 & 6.28
Ω \\
\textbf{Bandwidth (BW)} & BW = fr/Q & 100\times10^{3}/100 & 1 kHz \\
\end{longtable}
}

\end{solutionbox}
\begin{mnemonicbox}
``RCB: Resonance needs Capacitance and Bandwidth''

\end{mnemonicbox}
\subsection*{Question 4(c) OR [7
marks]}\label{q4c}

\textbf{Explain Band width and Selectivity of a series RLC circuit. Also
establish the relation between Q factor and BW for series resonance
circuit.}

\begin{solutionbox}

{\def\LTcaptype{none} % do not increment counter
\begin{longtable}[]{@{}
  >{\raggedright\arraybackslash}p{(\linewidth - 4\tabcolsep) * \real{0.2973}}
  >{\raggedright\arraybackslash}p{(\linewidth - 4\tabcolsep) * \real{0.3243}}
  >{\raggedright\arraybackslash}p{(\linewidth - 4\tabcolsep) * \real{0.3784}}@{}}
\toprule\noalign{}
\begin{minipage}[b]{\linewidth}\raggedright
Parameter
\end{minipage} & \begin{minipage}[b]{\linewidth}\raggedright
Definition
\end{minipage} & \begin{minipage}[b]{\linewidth}\raggedright
Relationship
\end{minipage} \\
\midrule\noalign{}
\endhead
\bottomrule\noalign{}
\endlastfoot
\textbf{Bandwidth (BW)} & Frequency range between half-power points & BW
= f_{2} - f_{1} = ω_{2} - ω_{1} = R/L \\
\textbf{Selectivity} & Ability to differentiate between signals of
different frequencies & Inversely proportional to BW \\
\textbf{Q factor} & Ratio of resonant frequency to bandwidth & Q = ω_{0}/BW
= ω_{0}L/R \\
\end{longtable}
}

\textbf{Diagram:}

\begin{center}
\textbf{Mermaid Diagram (Code)}
\begin{verbatim}
{Shaded}
{Highlighting}[]
graph LR
    A[Input] {-{-}{-} B[R] {-}{-}{-} C[L] {-}{-}{-} D[C] {-}{-}{-} E[Output]}

    subgraph "Frequency Response"
        F[Amplitude] {-{-}{-} G[f_{0}]}
        H[BW = f_{2 {-} f_{1}] {-}.{-}{} G}
    end
{Highlighting}
{Shaded}
\end{verbatim}
\end{center}

For a series RLC circuit:

\begin{itemize}
\tightlist
\item
  At resonance (f_{0}), impedance is minimum (= R)
\item
  Half-power points occur when impedance = \sqrt2\timesR
\item
  At these points, power is half of maximum power
\end{itemize}

Bandwidth (BW) = ω_{2} - ω_{1} = R/L Q factor = ω_{0}L/R = ω_{0}/BW

Therefore, BW = ω_{0}/Q = 2πf_{0}/Q

This shows Q factor and bandwidth are inversely related: Higher Q \rightarrow
Narrower bandwidth \rightarrow Better selectivity

\end{solutionbox}
\begin{mnemonicbox}
``BQS: Bandwidth and Q determine Selectivity''

\end{mnemonicbox}
\subsection*{Question 5(a) [3 marks]}\label{q5a}

\textbf{Design a symmetrical T type attenuator to give attenuation of 40
dB and work in to the load of 300 Ω resistance.}

\begin{solutionbox}

\textbf{Diagram:}

\begin{center}
\textbf{Mermaid Diagram (Code)}
\begin{verbatim}
{Shaded}
{Highlighting}[]
graph LR
    A[Input] {-{-}{-} B[Z_{1}/2] {-}{-}{-} C((Node)) {-}{-}{-} D[Z_{1}/2] {-}{-}{-} E[Output]}
    C {-{-}{-} F[Z_{2}] {-}{-}{-} G[Ground]}
    H[300Ω] {-.{-}{} E}
{Highlighting}
{Shaded}
\end{verbatim}
\end{center}

{\def\LTcaptype{none} % do not increment counter
\begin{longtable}[]{@{}llll@{}}
\toprule\noalign{}
Parameter & Formula & Calculation & Result \\
\midrule\noalign{}
\endhead
\bottomrule\noalign{}
\endlastfoot
\textbf{Attenuation (N)} & N = 10\^{}(dB/20) & 10\^{}(40/20) & 100 \\
\textbf{Impedance ratio (K)} & K = (N+1)/(N-1) & (100+1)/(100-1) &
1.02 \\
\textbf{Z_{1}} & Z_{1} = R_{0}[(K-1)/K] & 300[(1.02-1)/1.02] & 5.88 Ω \\
\textbf{Z_{2}} & Z_{2} = R_{0}[2K/(K^{2}-1)] & 300[2\times1.02/(1.02^{2}-1)] &
594.12 Ω \\
\end{longtable}
}

\end{solutionbox}
\begin{mnemonicbox}
``TANZ: T-Attenuator Needs Z-parameters''

\end{mnemonicbox}
\subsection*{Question 5(b) [4 marks]}\label{q5b}

\textbf{Give classification of filters.}

\begin{solutionbox}

{\def\LTcaptype{none} % do not increment counter
\begin{longtable}[]{@{}
  >{\raggedright\arraybackslash}p{(\linewidth - 4\tabcolsep) * \real{0.4000}}
  >{\raggedright\arraybackslash}p{(\linewidth - 4\tabcolsep) * \real{0.1750}}
  >{\raggedright\arraybackslash}p{(\linewidth - 4\tabcolsep) * \real{0.4250}}@{}}
\toprule\noalign{}
\begin{minipage}[b]{\linewidth}\raggedright
Classification
\end{minipage} & \begin{minipage}[b]{\linewidth}\raggedright
Types
\end{minipage} & \begin{minipage}[b]{\linewidth}\raggedright
Characteristics
\end{minipage} \\
\midrule\noalign{}
\endhead
\bottomrule\noalign{}
\endlastfoot
\textbf{Based on Frequency Response} & - Low Pass- High Pass- Band Pass-
Band Stop & - Passes frequencies below cutoff- Passes frequencies above
cutoff- Passes frequencies within a band- Blocks frequencies within a
band \\
\textbf{Based on Components} & - Passive Filters- Active Filters & -
Uses R, L, C elements- Uses active devices with RC \\
\textbf{Based on Design Approach} & - Constant-k Filters- m-derived
Filters- Composite Filters & - Simplest design- Better cutoff
characteristics- Combines advantages \\
\textbf{Based on Technology} & - LC Filters- Crystal Filters- Ceramic
Filters- Digital Filters & - Uses inductors and capacitors- Uses
piezoelectric crystals- Uses piezoelectric ceramics- Implemented in
software \\
\end{longtable}
}

\textbf{Diagram:}

\begin{center}
\textbf{Mermaid Diagram (Code)}
\begin{verbatim}
{Shaded}
{Highlighting}[]
graph TD
    A[Filters] {-{-}{} B[Frequency Response]}
    A {-{-}{} C[Components]}
    A {-{-}{} D[Design Approach]}
    A {-{-}{} E[Technology]}

    B {-{-}{} F[Low Pass]}
    B {-{-}{} G[High Pass]}
    B {-{-}{} H[Band Pass]}
    B {-{-}{} I[Band Stop]}
{Highlighting}
{Shaded}
\end{verbatim}
\end{center}

\end{solutionbox}
\begin{mnemonicbox}
``FLAC: Filters: Low-pass, Active, Constant-k''

\end{mnemonicbox}
\subsection*{Question 5(c) [7 marks]}\label{q5c}

\textbf{Explain constant K Low Pass Filter.}

\begin{solutionbox}

{\def\LTcaptype{none} % do not increment counter
\begin{longtable}[]{@{}
  >{\raggedright\arraybackslash}p{(\linewidth - 2\tabcolsep) * \real{0.4091}}
  >{\raggedright\arraybackslash}p{(\linewidth - 2\tabcolsep) * \real{0.5909}}@{}}
\toprule\noalign{}
\begin{minipage}[b]{\linewidth}\raggedright
Concept
\end{minipage} & \begin{minipage}[b]{\linewidth}\raggedright
Description
\end{minipage} \\
\midrule\noalign{}
\endhead
\bottomrule\noalign{}
\endlastfoot
\textbf{Definition} & Filter where impedance product Z_{1}Z_{2} = k^{2}
(constant) at all frequencies \\
\textbf{Circuit Types} & T-section and π-section \\
\textbf{T-section components} & Series inductors (L/2) and shunt
capacitor (C) \\
\textbf{π-section components} & Series inductor (L) and shunt capacitors
(C/2) \\
\textbf{Cutoff frequency} & fc = 1/π\sqrt(LC) \\
\textbf{Characteristic impedance} & R_{0} = \sqrt(L/C) \\
\end{longtable}
}

\textbf{Diagram:}

\begin{center}
\textbf{Mermaid Diagram (Code)}
\begin{verbatim}
{Shaded}
{Highlighting}[]
graph TD
    subgraph "T{-section"}
        A1[Input] {-{-}{-} B1[L/2] {-}{-}{-} C1((Node)) {-}{-}{-} D1[L/2] {-}{-}{-} E1[Output]}
        C1 {-{-}{-} F1[C] {-}{-}{-} G1[Ground]}
    end

    subgraph "π{-section"}
        A2[Input] {-{-}{-} B2((Node))}
        B2 {-{-}{-} C2[C/2] {-}{-}{-} G2[Ground]}
        B2 {-{-}{-} D2[L] {-}{-}{-} E2((Node)) {-}{-}{-} F2[Output]}
        E2 {-{-}{-} H2[C/2] {-}{-}{-} G2}
    end
{Highlighting}
{Shaded}
\end{verbatim}
\end{center}

The constant-k low pass filter has:

\begin{itemize}
\tightlist
\item
  Cutoff frequency: fc = 1/π\sqrt(LC)
\item
  Design impedance: R_{0} = \sqrt(L/C)
\item
  Pass band: 0 to fc
\item
  Attenuation band: Above fc
\item
  Gradual transition from pass band to stop band
\end{itemize}

\end{solutionbox}
\begin{mnemonicbox}
``CLPT: Constant-k Low Pass needs T-section''

\end{mnemonicbox}
\subsection*{Question 5(a) OR [3
marks]}\label{q5a}

\textbf{Design a high pass filter with T section having a cut-off
frequency of 1.5 KHz with a load resistance of 400 Ω.}

\begin{solutionbox}

\textbf{Diagram:}

\begin{center}
\textbf{Mermaid Diagram (Code)}
\begin{verbatim}
{Shaded}
{Highlighting}[]
graph LR
    A[Input] {-{-}{-} B[C/2] {-}{-}{-} C((Node)) {-}{-}{-} D[C/2] {-}{-}{-} E[Output]}
    C {-{-}{-} F[L] {-}{-}{-} G[Ground]}
    H[400Ω] {-.{-}{} E}
{Highlighting}
{Shaded}
\end{verbatim}
\end{center}

{\def\LTcaptype{none} % do not increment counter
\begin{longtable}[]{@{}llll@{}}
\toprule\noalign{}
Parameter & Formula & Calculation & Result \\
\midrule\noalign{}
\endhead
\bottomrule\noalign{}
\endlastfoot
\textbf{Design impedance (R_{0})} & R_{0} = Load resistance & Given & 400 Ω \\
\textbf{Cutoff frequency (fc)} & fc = Given & Given & 1.5 kHz \\
\textbf{Inductor (L)} & L = R_{0}/2πfc & 400/(2π\times1500) & 42.44 mH \\
\textbf{Capacitor (C)} & C = 1/(2πfcR_{0}) & 1/(2π\times1500\times400) & 0.265 µF \\
\end{longtable}
}

\end{solutionbox}
\begin{mnemonicbox}
``HCL: High-pass needs Capacitor and inductor''

\end{mnemonicbox}
\subsection*{Question 5(b) OR [4
marks]}\label{q5b}

\textbf{Give classification of attenuators.}

\begin{solutionbox}

{\def\LTcaptype{none} % do not increment counter
\begin{longtable}[]{@{}
  >{\raggedright\arraybackslash}p{(\linewidth - 4\tabcolsep) * \real{0.4000}}
  >{\raggedright\arraybackslash}p{(\linewidth - 4\tabcolsep) * \real{0.1750}}
  >{\raggedright\arraybackslash}p{(\linewidth - 4\tabcolsep) * \real{0.4250}}@{}}
\toprule\noalign{}
\begin{minipage}[b]{\linewidth}\raggedright
Classification
\end{minipage} & \begin{minipage}[b]{\linewidth}\raggedright
Types
\end{minipage} & \begin{minipage}[b]{\linewidth}\raggedright
Characteristics
\end{minipage} \\
\midrule\noalign{}
\endhead
\bottomrule\noalign{}
\endlastfoot
\textbf{Based on Configuration} & - T-attenuator- π-attenuator-
Bridged-T- Lattice & - Series-shunt-series- Shunt-series-shunt- Balanced
bridge- Balanced network \\
\textbf{Based on Symmetry} & - Symmetrical- Asymmetrical & - Equal
impedance- Unequal impedance \\
\textbf{Based on Control} & - Fixed- Variable- Programmable & - Constant
attenuation- Adjustable attenuation- Digitally controlled \\
\textbf{Based on Technology} & - Resistive- Reactive- Active & - Uses
resistors- Uses reactances- Uses active devices \\
\end{longtable}
}

\textbf{Diagram:}

\begin{center}
\textbf{Mermaid Diagram (Code)}
\begin{verbatim}
{Shaded}
{Highlighting}[]
graph TD
    A[Attenuators] {-{-}{} B[Configuration]}
    A {-{-}{} C[Symmetry]}
    A {-{-}{} D[Control]}
    A {-{-}{} E[Technology]}

    B {-{-}{} F[T{-}type]}
    B {-{-}{} G[π{-}type]}
    B {-{-}{} H[Bridged{-}T]}
    B {-{-}{} I[Lattice]}
{Highlighting}
{Shaded}
\end{verbatim}
\end{center}

\end{solutionbox}
\begin{mnemonicbox}
``CAST: Configuration, Adjustable, Symmetry,
Technology''

\end{mnemonicbox}
\subsection*{Question 5(c) OR [7
marks]}\label{q5c}

\textbf{Explain constant K High Pass Filter.}

\begin{solutionbox}

{\def\LTcaptype{none} % do not increment counter
\begin{longtable}[]{@{}
  >{\raggedright\arraybackslash}p{(\linewidth - 2\tabcolsep) * \real{0.4091}}
  >{\raggedright\arraybackslash}p{(\linewidth - 2\tabcolsep) * \real{0.5909}}@{}}
\toprule\noalign{}
\begin{minipage}[b]{\linewidth}\raggedright
Concept
\end{minipage} & \begin{minipage}[b]{\linewidth}\raggedright
Description
\end{minipage} \\
\midrule\noalign{}
\endhead
\bottomrule\noalign{}
\endlastfoot
\textbf{Definition} & Filter passing frequencies above cutoff, with Z_{1}Z_{2}
= k^{2} (constant) \\
\textbf{Circuit Types} & T-section and π-section \\
\textbf{T-section components} & Series capacitors (C/2) and shunt
inductor (L) \\
\textbf{π-section components} & Series capacitor (C) and shunt inductors
(L/2) \\
\textbf{Cutoff frequency} & fc = 1/π\sqrt(LC) \\
\textbf{Characteristic impedance} & R_{0} = \sqrt(L/C) \\
\end{longtable}
}

\textbf{Diagram:}

\begin{center}
\textbf{Mermaid Diagram (Code)}
\begin{verbatim}
{Shaded}
{Highlighting}[]
graph TD
    subgraph "T{-section"}
        A1[Input] {-{-}{-} B1[C/2] {-}{-}{-} C1((Node)) {-}{-}{-} D1[C/2] {-}{-}{-} E1[Output]}
        C1 {-{-}{-} F1[L] {-}{-}{-} G1[Ground]}
    end

    subgraph "π{-section"}
        A2[Input] {-{-}{-} B2((Node))}
        B2 {-{-}{-} C2[L/2] {-}{-}{-} G2[Ground]}
        B2 {-{-}{-} D2[C] {-}{-}{-} E2((Node)) {-}{-}{-} F2[Output]}
        E2 {-{-}{-} H2[L/2] {-}{-}{-} G2}
    end
{Highlighting}
{Shaded}
\end{verbatim}
\end{center}

The constant-k high pass filter has:

\begin{itemize}
\tightlist
\item
  Cutoff frequency: fc = 1/π\sqrt(LC)
\item
  Design impedance: R_{0} = \sqrt(L/C)
\item
  Pass band: Above fc
\item
  Attenuation band: 0 to fc
\item
  Gradual transition from pass band to stop band
\item
  Component values are dual of low pass filter (L and C swap places)
\end{itemize}

\end{solutionbox}
\begin{mnemonicbox}
``CHTS: Constant-k High-pass uses T-Section''

\end{mnemonicbox}

\end{document}
