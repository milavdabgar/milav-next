\documentclass[10pt,a4paper]{article}

% content/resources/templates/preamble.tex
\usepackage[margin=0.6in]{geometry}
\author{Milav Dabgar}
\usepackage{amsmath,amssymb,amsthm}
\usepackage{booktabs}
\usepackage{multirow}
\usepackage{xcolor}
\usepackage{tcolorbox}
\tcbuselibrary{breakable,skins}
\usepackage[colorlinks=true,linkcolor=blue]{hyperref}
\usepackage{titlesec}
\usepackage{enumitem}
\usepackage{tikz}
\usepackage{pgfplots}
\usepackage{circuitikz}
\usepackage[version=4]{mhchem}
\usepackage{longtable}
\usepackage{array}
\usepackage{float}
\usepackage{caption}
\usepackage{listings}

\lstset{
  basicstyle=\small\ttfamily,
  breaklines=true,
  breakatwhitespace=false,
  postbreak=\mbox{\textcolor{red}{$\hookrightarrow$}\space},
  float=false,
  numbers=left,
  numberstyle=\tiny\color{gray},
  numbersep=10pt,
  xleftmargin=2em,
  keywordstyle=\color{blue},
  commentstyle=\color{green!60!black},
  stringstyle=\color{purple},
  backgroundcolor=\color{gray!5},
  showstringspaces=false,
  tabsize=2,
  captionpos=b,
  keepspaces=true,
  columns=flexible
}

\pgfplotsset{compat=1.18}
\usetikzlibrary{shapes,arrows,positioning,calc,patterns,decorations.pathmorphing,decorations.markings,arrows.meta}

% Color scheme
\definecolor{headcolor}{RGB}{0,102,204}
\definecolor{keycolor}{RGB}{220,20,60}
\definecolor{solutioncolor}{RGB}{34,139,34}
\definecolor{mnemoniccolor}{RGB}{148,0,211}
\definecolor{codecolor}{RGB}{0,0,100}

% Spacing
\setlength{\parskip}{3pt}
\setlist[itemize]{nosep}
\setlist[enumerate]{nosep}

% Title formatting
\titleformat{\section}{\Large\bfseries\color{headcolor}}{\thesection}{1em}{}
\titleformat{\subsection}{\large\bfseries\color{headcolor}}{\thesubsection}{1em}{}

% Pandoc tightlist compatibility
\providecommand{\tightlist}{%
  \setlength{\itemsep}{0pt}\setlength{\parskip}{0pt}}

% Pandoc longtable compatibility
\newcounter{none}
\def\thenone{}


% content/resources/templates/english-boxes.tex
% This file is currently empty - it exists to maintain consistency with the import structure.
% Add custom environments here if needed in the future.


\begin{document}

\begin{center}
{\Huge\bfseries\color{headcolor} Subject Name Solutions}\\[5pt]
{\LARGE 4331101 -- Summer 2024}\\[3pt]
{\large Semester 1 Study Material}\\[3pt]
{\normalsize\textit{Detailed Solutions and Explanations}}
\end{center}

\vspace{10pt}

\subsection*{Question 1(a) [3 marks]}\label{q1a}

\textbf{Define node, branch and loop with suitable diagram.}

\begin{solutionbox}

\textbf{Diagram:}

\begin{center}
\textbf{Mermaid Diagram (Code)}
\begin{verbatim}
{Shaded}
{Highlighting}[]
graph LR
    A((Node A)) {-{-}{-} B\{Branch 1\}}
    A {-{-}{-} C\{Branch 2\}}
    A {-{-}{-} D\{Branch 3\}}
    B {-{-}{-} E((Node B))}
    C {-{-}{-} F((Node C))}
    D {-{-}{-} G((Node D))}
    E {-{-}{-} H\{Branch 4\}}
    H {-{-}{-} F}
    G {-{-}{-} I\{Branch 5\}}
    I {-{-}{-} F}

    subgraph Loop X
    A {-{-}{} B {-}{-}{} E {-}{-}{} H {-}{-}{} F {-}{-}{} C {-}{-}{} A}
    end
{Highlighting}
{Shaded}
\end{verbatim}
\end{center}

\begin{itemize}
\tightlist
\item
  \textbf{Node}: A point where two or more circuit elements join
  together
\item
  \textbf{Branch}: A single element connecting two nodes
\item
  \textbf{Loop}: Any closed path in a circuit where no node is
  encountered more than once
\end{itemize}

\end{solutionbox}
\begin{mnemonicbox}
``NBA circuit'' - Nodes are junctions, Branches are
roads, Loops are Alternate paths

\end{mnemonicbox}
\subsection*{Question 1(b) [4 marks]}\label{q1b}

\textbf{Explain ``Tree'' and ``Graph'' of a network.}

\begin{solutionbox}

\textbf{Diagram:}

\begin{center}
\textbf{Mermaid Diagram (Code)}
\begin{verbatim}
{Shaded}
{Highlighting}[]
graph TD
    subgraph Network Graph
    direction LR    
    A((A)) {-{-}{-} B((B))}
    A {-{-}{-} C((C))}
    B {-{-}{-} D((D))}
    C {-{-}{-} D}
    B {-{-}{-} C}
    end

    subgraph Tree of Network
    direction LR    
    E((A)) {-{-}{-} F((B))}
    E {-{-}{-} G((C))}
    F {-{-}{-} H((D))}
    end
{Highlighting}
{Shaded}
\end{verbatim}
\end{center}

{\def\LTcaptype{none} % do not increment counter
\begin{longtable}[]{@{}
  >{\raggedright\arraybackslash}p{(\linewidth - 4\tabcolsep) * \real{0.4091}}
  >{\raggedright\arraybackslash}p{(\linewidth - 4\tabcolsep) * \real{0.3182}}
  >{\raggedright\arraybackslash}p{(\linewidth - 4\tabcolsep) * \real{0.2727}}@{}}
\toprule\noalign{}
\begin{minipage}[b]{\linewidth}\raggedright
Feature
\end{minipage} & \begin{minipage}[b]{\linewidth}\raggedright
Graph
\end{minipage} & \begin{minipage}[b]{\linewidth}\raggedright
Tree
\end{minipage} \\
\midrule\noalign{}
\endhead
\bottomrule\noalign{}
\endlastfoot
\textbf{Definition} & Complete topological representation of network &
Connected subgraph containing all nodes but no loops \\
\textbf{Elements} & Contains all branches and nodes & Contains N-1
branches where N is number of nodes \\
\textbf{Loops} & Contains loops & No loops \\
\textbf{Application} & Used for complete circuit analysis & Used for
simplifying network calculations \\
\end{longtable}
}

\end{solutionbox}
\begin{mnemonicbox}
``GRAND Tree'' - Graph has Routes And Nodes with
Detours, Tree has only single Routes

\end{mnemonicbox}
\subsection*{Question 1(c) [7 marks]}\label{q1c}

\textbf{Explain ``Mesh current Method'' using suitable diagram.}

\begin{solutionbox}

\textbf{Diagram:}

\begin{center}
\textbf{Mermaid Diagram (Code)}
\begin{verbatim}
{Shaded}
{Highlighting}[]
graph LR
    subgraph Mesh 1
    A(({+)) {-}{-} R1 {-}{-}{} B(({}+))}
    B {-{-} R3 {-}{-}{} C(({}+))}
    C {-{-} R5 {-}{-}{} A}
    end

    subgraph Mesh 2
    B {-{-} R2 {-}{-}{} D(({}+))}
    D {-{-} R4 {-}{-}{} C}
    C {-{-} R3 {-}{-}{} B}
    end
    
    style Mesh 1 fill:\#f9f,stroke:\#333,stroke{-width:2px}
    style Mesh 2 fill:\#bbf,stroke:\#333,stroke{-width:2px}
{Highlighting}
{Shaded}
\end{verbatim}
\end{center}

{\def\LTcaptype{none} % do not increment counter
\begin{longtable}[]{@{}ll@{}}
\toprule\noalign{}
Step & Description \\
\midrule\noalign{}
\endhead
\bottomrule\noalign{}
\endlastfoot
1 & Identify independent meshes in the circuit \\
2 & Assign mesh currents (I_{1}, I_{2}, etc.) in clockwise direction \\
3 & Apply KVL to each mesh \\
4 & Form equations using: ΣR·I(own) - ΣR·I(adjacent) = ΣV \\
5 & Solve the simultaneous equations \\
\end{longtable}
}

\begin{itemize}
\tightlist
\item
  \textbf{Advantage}: Fewer equations than branch current method
\item
  \textbf{Application}: Best for planar networks
\item
  \textbf{Limitation}: Less efficient for non-planar networks
\end{itemize}

\end{solutionbox}
\begin{mnemonicbox}
``MIAMI'' - Meshes Identified, Assign currents, Make
equations, Intersection currents calculated, Solve

\end{mnemonicbox}
\subsection*{Question 1(c OR) [7
marks]}\label{question-1c-or-7-marks}

\textbf{Explain ``Node pair voltage Method'' using suitable diagram.}

\begin{solutionbox}

\textbf{Diagram:}

\begin{center}
\textbf{Mermaid Diagram (Code)}
\begin{verbatim}
{Shaded}
{Highlighting}[]
graph LR
    A((Node 1)) {-{-} I1 {-}{-}{} B((Node 2))}
    A {-{-} I2 {-}{-}{} C((Node 3))}
    B {-{-} I3 {-}{-}{} C}
    B {-{-} I4 {-}{-}{} D((Reference))}
    C {-{-} I5 {-}{-}{} D}
    A {-{-} I6 {-}{-}{} D}
{Highlighting}
{Shaded}
\end{verbatim}
\end{center}

{\def\LTcaptype{none} % do not increment counter
\begin{longtable}[]{@{}ll@{}}
\toprule\noalign{}
Step & Description \\
\midrule\noalign{}
\endhead
\bottomrule\noalign{}
\endlastfoot
1 & Select a reference node (ground) \\
2 & Assign node voltages (V_{1}, V_{2}, etc.) to remaining nodes \\
3 & Apply KCL at each node (except reference) \\
4 & Express currents in terms of node voltages using Ohm's Law \\
5 & Solve the simultaneous equations \\
\end{longtable}
}

\begin{itemize}
\tightlist
\item
  \textbf{Advantage}: Fewer equations than mesh method for circuits with
  many meshes
\item
  \textbf{Application}: Efficient for non-planar circuits
\item
  \textbf{Key equation}: ΣG·V(own) - ΣG·V(adjacent) = ΣI
\end{itemize}

\end{solutionbox}
\begin{mnemonicbox}
``GRAND'' - Ground node fixed, Remaining nodes
numbered, Apply KCL, Note voltage differences, Derive solutions

\end{mnemonicbox}
\subsection*{Question 2(a) [3 marks]}\label{q2a}

\textbf{Explain KCL with example.}

\begin{solutionbox}

\textbf{Diagram:}

\begin{verbatim}
    I1    
  +{-{-}{-}{-}{-}+}
  |     |
  |     I3 ↓  
I2 ↓    |
  |     |
  +{-{-}{-}{-}{-}+}
    I4 ↑
\end{verbatim}

\textbf{Kirchhoff's Current Law (KCL)}: The algebraic sum of all
currents entering and leaving a node is zero.

{\def\LTcaptype{none} % do not increment counter
\begin{longtable}[]{@{}ll@{}}
\toprule\noalign{}
Mathematical Form & Example Application \\
\midrule\noalign{}
\endhead
\bottomrule\noalign{}
\endlastfoot
ΣI = 0 & At node: I_{1} - I_{2} - I_{3} + I_{4} = 0 \\
ΣIin = ΣIout & Currents entering = Currents leaving \\
\end{longtable}
}

\end{solutionbox}
\begin{mnemonicbox}
``ZINC'' - Zero Is Net Current at a node

\end{mnemonicbox}
\subsection*{Question 2(b) [4 marks]}\label{q2b}

\textbf{Explain Z-parameter, Y-parameter, h-parameter and ABCD-parameter
using suitable network.}

\begin{solutionbox}

\textbf{Diagram:}

\begin{verbatim}
        +{-{-}{-}{-}{-}+}
   V1   |     |   V2
       |  2  |   
+{-{-}{-}{-}{-}{-}+|  P  |+{-}{-}{-}{-}{-}+}
   I1   |  O  |   I2
       |  R  |   
        |  T  |
        +{-{-}{-}{-}{-}+}
\end{verbatim}

{\def\LTcaptype{none} % do not increment counter
\begin{longtable}[]{@{}
  >{\raggedright\arraybackslash}p{(\linewidth - 6\tabcolsep) * \real{0.2683}}
  >{\raggedright\arraybackslash}p{(\linewidth - 6\tabcolsep) * \real{0.2927}}
  >{\raggedright\arraybackslash}p{(\linewidth - 6\tabcolsep) * \real{0.2683}}
  >{\raggedright\arraybackslash}p{(\linewidth - 6\tabcolsep) * \real{0.1707}}@{}}
\toprule\noalign{}
\begin{minipage}[b]{\linewidth}\raggedright
Parameter
\end{minipage} & \begin{minipage}[b]{\linewidth}\raggedright
Definition
\end{minipage} & \begin{minipage}[b]{\linewidth}\raggedright
Equations
\end{minipage} & \begin{minipage}[b]{\linewidth}\raggedright
Usage
\end{minipage} \\
\midrule\noalign{}
\endhead
\bottomrule\noalign{}
\endlastfoot
\textbf{Z} & Impedance parameters & V_{1} = Z_{1}_{1}I_{1} + Z_{1}_{2}I_{2}, V_{2} = Z_{2}_{1}I_{1} +
Z_{2}_{2}I_{2} & High impedance circuits \\
\textbf{Y} & Admittance parameters & I_{1} = Y_{1}_{1}V_{1} + Y_{1}_{2}V_{2}, I_{2} = Y_{2}_{1}V_{1} +
Y_{2}_{2}V_{2} & Low impedance circuits \\
\textbf{h} & Hybrid parameters & V_{1} = h_{1}_{1}I_{1} + h_{1}_{2}V_{2}, I_{2} = h_{2}_{1}I_{1} + h_{2}_{2}V_{2}
& Transistor circuits \\
\textbf{ABCD} & Transmission parameters & V_{1} = AV_{2} - BI_{2}, I_{1} = CV_{2} - DI_{2}
& Cascaded networks \\
\end{longtable}
}

\end{solutionbox}
\begin{mnemonicbox}
``ZANY HAB'' - Z for high impedance, A for low,
hy-brid for transistors, ABCD for Cascades

\end{mnemonicbox}
\subsection*{Question 2(c) [7 marks]}\label{q2c}

\textbf{Derive the equations to convert π-type network into T-type
network and T-type network into π-type network.}

\begin{solutionbox}

\textbf{Diagram:}

\begin{center}
\textbf{Mermaid Diagram (Code)}
\begin{verbatim}
{Shaded}
{Highlighting}[]
graph TD
    subgraph T{-Network}
    A1((1)) {-{-} Z1 {-}{-}{} O1((O))}
    B1((2)) {-{-} Z2 {-}{-}{} O1}
    C1((3)) {-{-} Z3 {-}{-}{} O1}
    end

    subgraph π{-Network}
    A2((1)) {-{-} Y1 {-}{-}{} B2((2))}
    B2 {-{-} Y2 {-}{-}{} C2((3))}
    C2 {-{-} Y3 {-}{-}{} A2}
    end
{Highlighting}
{Shaded}
\end{verbatim}
\end{center}

{\def\LTcaptype{none} % do not increment counter
\begin{longtable}[]{@{}
  >{\raggedright\arraybackslash}p{(\linewidth - 2\tabcolsep) * \real{0.5455}}
  >{\raggedright\arraybackslash}p{(\linewidth - 2\tabcolsep) * \real{0.4545}}@{}}
\toprule\noalign{}
\begin{minipage}[b]{\linewidth}\raggedright
Conversion
\end{minipage} & \begin{minipage}[b]{\linewidth}\raggedright
Formulas
\end{minipage} \\
\midrule\noalign{}
\endhead
\bottomrule\noalign{}
\endlastfoot
\textbf{π to T} & Z_{1} = (Z_{1}_{2}·Z_{3}_{1})/(Z_{1}_{2}+Z_{2}_{3}+Z_{3}_{1}) Z_{2} =
(Z_{1}_{2}·Z_{2}_{3})/(Z_{1}_{2}+Z_{2}_{3}+Z_{3}_{1}) Z_{3} = (Z_{2}_{3}·Z_{3}_{1})/(Z_{1}_{2}+Z_{2}_{3}+Z_{3}_{1}) \\
\textbf{T to π} & Z_{1}_{2} = (Z_{1}·Z_{2}+Z_{2}·Z_{3}+Z_{3}·Z_{1})/Z_{3} Z_{2}_{3} =
(Z_{1}·Z_{2}+Z_{2}·Z_{3}+Z_{3}·Z_{1})/Z_{1} Z_{3}_{1} = (Z_{1}·Z_{2}+Z_{2}·Z_{3}+Z_{3}·Z_{1})/Z_{2} \\
\end{longtable}
}

\begin{itemize}
\tightlist
\item
  \textbf{Application}: Network simplification and analysis
\item
  \textbf{Condition}: Both networks must be equivalent at terminals
\item
  \textbf{Limitation}: Only applies for linear networks
\end{itemize}

\end{solutionbox}
\begin{mnemonicbox}
``TRIP'' - T and π networks Relate Impedances through
Products and sums

\end{mnemonicbox}
\subsection*{Question 2(a OR) [3
marks]}\label{question-2a-or-3-marks}

\textbf{Explain KVL with example.}

\begin{solutionbox}

\textbf{Diagram:}

\begin{verbatim}
    +{-{-}R1{-}{-}+}
    |      |
   V1     R2
    |      |
    +{-{-}R3{-}{-}+}
\end{verbatim}

\textbf{Kirchhoff's Voltage Law (KVL)}: The algebraic sum of all
voltages around any closed loop is zero.

{\def\LTcaptype{none} % do not increment counter
\begin{longtable}[]{@{}ll@{}}
\toprule\noalign{}
Mathematical Form & Example Application \\
\midrule\noalign{}
\endhead
\bottomrule\noalign{}
\endlastfoot
ΣV = 0 & In loop: V_{1} - IR_{1} - IR_{2} - IR_{3} = 0 \\
ΣVrises = ΣVdrops & Voltage rises = Voltage drops \\
\end{longtable}
}

\end{solutionbox}
\begin{mnemonicbox}
``ZERO'' - Zero is Every voltage Round a loop's
Output

\end{mnemonicbox}
\subsection*{Question 2(b OR) [4
marks]}\label{question-2b-or-4-marks}

\textbf{Classify and explain various Electronics network.}

\begin{solutionbox}

{\def\LTcaptype{none} % do not increment counter
\begin{longtable}[]{@{}
  >{\raggedright\arraybackslash}p{(\linewidth - 4\tabcolsep) * \real{0.3889}}
  >{\raggedright\arraybackslash}p{(\linewidth - 4\tabcolsep) * \real{0.3611}}
  >{\raggedright\arraybackslash}p{(\linewidth - 4\tabcolsep) * \real{0.2500}}@{}}
\toprule\noalign{}
\begin{minipage}[b]{\linewidth}\raggedright
Network Type
\end{minipage} & \begin{minipage}[b]{\linewidth}\raggedright
Description
\end{minipage} & \begin{minipage}[b]{\linewidth}\raggedright
Example
\end{minipage} \\
\midrule\noalign{}
\endhead
\bottomrule\noalign{}
\endlastfoot
\textbf{Linear vs Non-linear} & Follows/doesn't follow proportionality
principle & Resistors vs Diodes \\
\textbf{Passive vs Active} & Don't/do supply energy & RC circuit vs
Amplifier \\
\textbf{Bilateral vs Unilateral} & Same/different properties in either
direction & Resistors vs Diodes \\
\textbf{Lumped vs Distributed} & Parameters concentrated/spread & RC
circuit vs Transmission line \\
\textbf{Time variant vs Invariant} & Parameters change/don't change with
time & Electronic switch vs Fixed resistor \\
\end{longtable}
}

\textbf{Diagram:}

\begin{verbatim}
graph TB
    A[Electronic Networks]
    A {-{-} B[Based on Linearity]}
    A {-{-} C[Based on Energy]}
    A {-{-} D[Based on Directionality]}
    A {-{-} E[Based on Parameters]}
    A {-{-} F[Based on Time]}

    B {-{-} G[Linear]}
    B {-{-} H[Non{-}linear]}
    C {-{-} I[Active]}
    C {-{-} J[Passive]}
    D {-{-} K[Bilateral]}
    D {-{-} L[Unilateral]}
    E {-{-} M[Lumped]}
    E {-{-} N[Distributed]}
    F {-{-} O[Time{-}invariant]}
    F {-{-} P[Time{-}variant]}
\end{verbatim}

\end{solutionbox}
\begin{mnemonicbox}
``PLANT'' - Proportionality for Linear, Lively for
Active, All directions for bilateral, Near for lumped, Time-fixed for
invariant

\end{mnemonicbox}
\subsection*{Question 2(c OR) [7
marks]}\label{question-2c-or-7-marks}

\textbf{Derive the equation of characteristic impedance for T-network
and π-network.}

\begin{solutionbox}

\textbf{Diagram:}

\begin{center}
\textbf{Mermaid Diagram (Code)}
\begin{verbatim}
{Shaded}
{Highlighting}[]
graph TD
    subgraph T{-Network}
    A1((1)) {-{-} Z1 {-}{-}{} O1((O))}
    O1 {-{-} Z3 {-}{-}{} C1((2))}
    O1 {-{-} Z2 {-}{-}{} B1}
    end

    subgraph π{-Network}
    A2((1)) {-{-} Y1 {-}{-}{} B2}
    B2 {-{-} Y2 {-}{-}{} C2((2))}
    C2 {-{-} Y3 {-}{-}{} A2}
    end
{Highlighting}
{Shaded}
\end{verbatim}
\end{center}

{\def\LTcaptype{none} % do not increment counter
\begin{longtable}[]{@{}
  >{\raggedright\arraybackslash}p{(\linewidth - 4\tabcolsep) * \real{0.1475}}
  >{\raggedright\arraybackslash}p{(\linewidth - 4\tabcolsep) * \real{0.5574}}
  >{\raggedright\arraybackslash}p{(\linewidth - 4\tabcolsep) * \real{0.2951}}@{}}
\toprule\noalign{}
\begin{minipage}[b]{\linewidth}\raggedright
Network
\end{minipage} & \begin{minipage}[b]{\linewidth}\raggedright
Characteristic Impedance Equation
\end{minipage} & \begin{minipage}[b]{\linewidth}\raggedright
Derivation Steps
\end{minipage} \\
\midrule\noalign{}
\endhead
\bottomrule\noalign{}
\endlastfoot
\textbf{T-Network} & Z_{0}T = \sqrt[(Z_{1}+Z_{2})(Z_{2}+Z_{3})] & 1. Apply symmetrical
load Z_{0} 2. Find input impedance 3. For impedance matching, Zin = Z_{0} 4.
Solve for Z_{0} \\
\textbf{π-Network} & Z_{0}π = 1/\sqrt[(Y_{1}+Y_{3})(Y_{2}+Y_{3})] & 1. Apply
symmetrical load Z_{0} 2. Find input impedance 3. For impedance matching,
Zin = Z_{0} 4. Solve for Z_{0} \\
\end{longtable}
}

\begin{itemize}
\tightlist
\item
  \textbf{Relation}: Z_{0}T \times Z_{0}π = Z_{1}·Z_{3}
\item
  \textbf{Application}: Impedance matching and filters
\item
  \textbf{Limitation}: Valid only for symmetrical networks
\end{itemize}

\end{solutionbox}
\begin{mnemonicbox}
``TIPSZ'' - T-networks and π-networks Impedances are
Products and Square roots of Z values

\end{mnemonicbox}
\subsection*{Question 3(a) [3 marks]}\label{q3a}

\textbf{Explain the principle of duality with example.}

\begin{solutionbox}

\textbf{Diagram:}

\begin{verbatim}
Original Circuit          Dual Circuit
   +{-{-}{-}R1{-}{-}{-}+               +{-}{-}{-}G1{-}{-}{-}+}
   |        |               |        |
  V1       R2        ={    I1       G2}
   |        |               |        |
   +{-{-}{-}R3{-}{-}{-}+               +{-}{-}{-}G3{-}{-}{-}+}
\end{verbatim}

\textbf{Principle of Duality}: For every electrical network, there
exists a dual network where:

{\def\LTcaptype{none} % do not increment counter
\begin{longtable}[]{@{}
  >{\raggedright\arraybackslash}p{(\linewidth - 4\tabcolsep) * \real{0.4000}}
  >{\raggedright\arraybackslash}p{(\linewidth - 4\tabcolsep) * \real{0.2400}}
  >{\raggedright\arraybackslash}p{(\linewidth - 4\tabcolsep) * \real{0.3600}}@{}}
\toprule\noalign{}
\begin{minipage}[b]{\linewidth}\raggedright
Original
\end{minipage} & \begin{minipage}[b]{\linewidth}\raggedright
Dual
\end{minipage} & \begin{minipage}[b]{\linewidth}\raggedright
Example
\end{minipage} \\
\midrule\noalign{}
\endhead
\bottomrule\noalign{}
\endlastfoot
Voltage (V) & Current (I) & 10V source \rightarrow 10A source \\
Current (I) & Voltage (V) & 5A \rightarrow 5V \\
Resistance (R) & Conductance (G) & 100Ω \rightarrow 100S \\
Series connection & Parallel connection & Series resistors \rightarrow Parallel
conductors \\
KVL & KCL & ΣV = 0 \rightarrow ΣI = 0 \\
\end{longtable}
}

\end{solutionbox}
\begin{mnemonicbox}
``VIGOR'' - Voltage to current, Impedance to
admittance, Graph remains, Open to closed, Resistors to conductors

\end{mnemonicbox}
\subsection*{Question 3(b) [4 marks]}\label{q3b}

\textbf{Explain the steps to calculate the load current in the circuit
using Thevenin's Theorem.}

\begin{solutionbox}

\textbf{Diagram:}

\begin{verbatim}
flowchart LR
    A[Original Circuit] {-{-} B[Remove Load]}
    B {-{-} C[Find Voc]}
    B {-{-} D[Find Rth]}
    C {-{-} E[Thevenin Equivalent]}
    D {-{-} E}
    E {-{-} F[Reconnect Load]}
    F {-{-} G[Calculate IL = Vth/Rth+RL]}

    style E fill:\#bbf,stroke:\#333
\end{verbatim}

{\def\LTcaptype{none} % do not increment counter
\begin{longtable}[]{@{}ll@{}}
\toprule\noalign{}
Step & Description \\
\midrule\noalign{}
\endhead
\bottomrule\noalign{}
\endlastfoot
1 & Remove the load resistor from the circuit \\
2 & Find open-circuit voltage (Vth) across load terminals \\
3 & Calculate Thevenin resistance (Rth) looking back into circuit \\
4 & Draw Thevenin equivalent circuit (Vth in series with Rth) \\
5 & Reconnect load resistor (RL) to Thevenin circuit \\
6 & Calculate load current: IL = Vth/(Rth+RL) \\
\end{longtable}
}

\end{solutionbox}
\begin{mnemonicbox}
``REVOLT'' - Remove load, Evaluate Voc, Obtain Rth,
Look at Thevenin circuit, Use I = V/R formula

\end{mnemonicbox}
\subsection*{Question 3(c) [7 marks]}\label{q3c}

\textbf{Find the current through load resistor using superposition
theorem.}

\begin{solutionbox}

\textbf{Diagram:}

\begin{verbatim}
    4Ω         10Ω
    ┌──────┬───────┐
    │      │       │
 12V┘     6Ω      ┌12A
    │    IL↓       │
    │      │       │
    └──────┴───────┘
\end{verbatim}


{\def\LTcaptype{none} % do not increment counter
\vspace{-5pt}
\captionof{table}{Step-by-Step Solution:}
\vspace{-10pt}
\begin{longtable}[]{@{}
  >{\raggedright\arraybackslash}p{(\linewidth - 4\tabcolsep) * \real{0.1875}}
  >{\raggedright\arraybackslash}p{(\linewidth - 4\tabcolsep) * \real{0.4062}}
  >{\raggedright\arraybackslash}p{(\linewidth - 4\tabcolsep) * \real{0.4062}}@{}}
\toprule\noalign{}
\begin{minipage}[b]{\linewidth}\raggedright
Step
\end{minipage} & \begin{minipage}[b]{\linewidth}\raggedright
Description
\end{minipage} & \begin{minipage}[b]{\linewidth}\raggedright
Calculation
\end{minipage} \\
\midrule\noalign{}
\endhead
\bottomrule\noalign{}
\endlastfoot
1 & Consider 12V source only (replace 12A with open) & I_{1} = 12/(4+6+10)
= 0.6A I_{1} through 6Ω = 0.6A \\
2 & Consider 12A source only (replace 12V with short) & I_{2} =
-12\times10/(4+10+6) = -6A I_{2} through 6Ω = -12\times4/(4+10+6) = -2.4A \\
3 & Apply superposition & IL = I_{1} + I_{2} = 0.6 + (-2.4) = -1.8A \\
\end{longtable}
}

\end{solutionbox}
\begin{solutionbox}
IL = -1.8A (current flowing upward through 6Ω load
resistor)

\end{solutionbox}
\begin{mnemonicbox}
``SONAR'' - Sources Only one at a time, Neutralize
others, Add Results

\end{mnemonicbox}
\subsection*{Question 3(a OR) [3
marks]}\label{question-3a-or-3-marks}

\textbf{Write Maximum Power Transfer Theorem statement. What are the
conditions for maximum power transfer for AC and DC networks?}

\begin{solutionbox}

\textbf{Maximum Power Transfer Theorem}: Maximum power is transferred
from source to load when the load impedance is equal to the complex
conjugate of the source internal impedance.

{\def\LTcaptype{none} % do not increment counter
\begin{longtable}[]{@{}
  >{\raggedright\arraybackslash}p{(\linewidth - 2\tabcolsep) * \real{0.2745}}
  >{\raggedright\arraybackslash}p{(\linewidth - 2\tabcolsep) * \real{0.7255}}@{}}
\toprule\noalign{}
\begin{minipage}[b]{\linewidth}\raggedright
Network Type
\end{minipage} & \begin{minipage}[b]{\linewidth}\raggedright
Condition for Maximum Power Transfer
\end{minipage} \\
\midrule\noalign{}
\endhead
\bottomrule\noalign{}
\endlastfoot
\textbf{DC Networks} & RL = Rth (Load resistance equals Thevenin
resistance) \\
\textbf{AC Networks} & ZL = Zth* (Load impedance equals complex
conjugate of Thevenin impedance) RL = Rth and XL = -Xth \\
\end{longtable}
}

\textbf{Diagram:}

\begin{verbatim}
   Rth     
    ┌─/{//─┐}
    │        │
 Vth┘       RL
    │        │
    └────────┘
   DC Network

   Rth     Xth
    ┌─/{//─┬─XX─┐}
    │        │    │
 Vth┘       RL   XL
    │        │    │
    └────────┴────┘
    AC Network
\end{verbatim}

\end{solutionbox}
\begin{mnemonicbox}
``MATCH'' - Maximum power At Terminals when Conjugate
impedances are Honored

\end{mnemonicbox}
\subsection*{Question 3(b OR) [4
marks]}\label{question-3b-or-4-marks}

\textbf{Explain the steps to calculate the load current in the circuit
using Norton's Theorem.}

\begin{solutionbox}

\textbf{Diagram:}

\begin{verbatim}
flowchart LR
    A[Original Circuit] {-{-} B[Short Load Terminals]}
    B {-{-} C[Find Isc]}
    B {-{-} D[Find Rn=Rth]}
    C {-{-} E[Norton Equivalent]}
    D {-{-} E}
    E {-{-} F[Reconnect Load]}
    F {-{-} G[Calculate IL = In/Rn+RL]}

    style E fill:\#bbf,stroke:\#333
\end{verbatim}

{\def\LTcaptype{none} % do not increment counter
\begin{longtable}[]{@{}ll@{}}
\toprule\noalign{}
Step & Description \\
\midrule\noalign{}
\endhead
\bottomrule\noalign{}
\endlastfoot
1 & Remove the load resistor from the circuit \\
2 & Find short-circuit current (In) across load terminals \\
3 & Calculate Norton resistance (Rn) looking back into circuit \\
4 & Draw Norton equivalent circuit (In in parallel with Rn) \\
5 & Reconnect load resistor (RL) to Norton circuit \\
6 & Calculate load current: IL = In\timesRn/(Rn+RL) \\
\end{longtable}
}

\end{solutionbox}
\begin{mnemonicbox}
``SENIOR'' - Short terminals, Evaluate Isc, Notice Rn
value, Implement Norton circuit, Obtain result

\end{mnemonicbox}
\subsection*{Question 3(c OR) [7
marks]}\label{question-3c-or-7-marks}

\textbf{Demonstrate how the reciprocity theorem is applied to a given
network.}

\begin{solutionbox}

\textbf{Diagram:}

\begin{verbatim}
    2Ω         2Ω
    ┌──────┬───────┐
    │      │       │
 10V┘     4Ω      2Ω
    │      │       │
    └──────┴───────┘
\end{verbatim}


{\def\LTcaptype{none} % do not increment counter
\vspace{-5pt}
\captionof{table}{Applying Reciprocity Theorem:}
\vspace{-10pt}
\begin{longtable}[]{@{}
  >{\raggedright\arraybackslash}p{(\linewidth - 6\tabcolsep) * \real{0.1429}}
  >{\raggedright\arraybackslash}p{(\linewidth - 6\tabcolsep) * \real{0.2619}}
  >{\raggedright\arraybackslash}p{(\linewidth - 6\tabcolsep) * \real{0.2619}}
  >{\raggedright\arraybackslash}p{(\linewidth - 6\tabcolsep) * \real{0.3333}}@{}}
\toprule\noalign{}
\begin{minipage}[b]{\linewidth}\raggedright
Step
\end{minipage} & \begin{minipage}[b]{\linewidth}\raggedright
Circuit 1
\end{minipage} & \begin{minipage}[b]{\linewidth}\raggedright
Circuit 2
\end{minipage} & \begin{minipage}[b]{\linewidth}\raggedright
Verification
\end{minipage} \\
\midrule\noalign{}
\endhead
\bottomrule\noalign{}
\endlastfoot
1 & 10V source at left, Find I_{1} at right & 10V source at right, Find I_{2}
at left & I_{1} = I_{2} confirms reciprocity \\
2 & Create mesh equations using KVL & Create new mesh equations for
swapped source & Solve both systems \\
3 & I_{1} = 10\times2/(2\times4 + 2\times2 + 4\times2) = 0.625A & I_{2} = 10\times2/(2\times4 + 2\times2 + 4\times2) =
0.625A & I_{1} = I_{2} = 0.625A ✓ \\
\end{longtable}
}

\textbf{Principle}: In a passive network containing only bilateral
elements, if voltage source E in branch 1 produces current I in branch
2, then the same voltage source E placed in branch 2 will produce the
same current I in branch 1.

\end{solutionbox}
\begin{mnemonicbox}
``RESPECT'' - Rewire sources, Exchange positions, See
if currents Preserve Equality when Circuit Transformed

\end{mnemonicbox}
\subsection*{Question 4(a) [3 marks]}\label{q4a}

\textbf{Explain coupled circuit.}

\begin{solutionbox}

\textbf{Diagram:}

\begin{verbatim}
    L1         L2
    ┌─OOOO─┐   ┌─OOOO─┐
    │      │   │      │
 V1 ┘      │   │     RL
    │   M  │   │      │
    └──────┘   └──────┘
    Primary    Secondary
\end{verbatim}

\textbf{Coupled Circuit}: A circuit where energy is transferred between
inductors through mutual inductance.

{\def\LTcaptype{none} % do not increment counter
\begin{longtable}[]{@{}
  >{\raggedright\arraybackslash}p{(\linewidth - 2\tabcolsep) * \real{0.4583}}
  >{\raggedright\arraybackslash}p{(\linewidth - 2\tabcolsep) * \real{0.5417}}@{}}
\toprule\noalign{}
\begin{minipage}[b]{\linewidth}\raggedright
Parameter
\end{minipage} & \begin{minipage}[b]{\linewidth}\raggedright
Description
\end{minipage} \\
\midrule\noalign{}
\endhead
\bottomrule\noalign{}
\endlastfoot
\textbf{Mutual Inductance (M)} & Measure of magnetic coupling between
coils \\
\textbf{Coupling Coefficient (k)} & k = M/\sqrt(L_{1}L_{2}), ranges from 0 (no
coupling) to 1 (perfect coupling) \\
\textbf{Applications} & Transformers, filters, tuned circuits \\
\end{longtable}
}

\end{solutionbox}
\begin{mnemonicbox}
``MICE'' - Mutual Inductance Creates Energy transfer

\end{mnemonicbox}
\subsection*{Question 4(b) [4 marks]}\label{q4b}

\textbf{Derive the equation of co-efficient of coupling for coupled
circuit.}

\begin{solutionbox}

\textbf{Diagram:}

\begin{center}
\textbf{Mermaid Diagram (Code)}
\begin{verbatim}
{Shaded}
{Highlighting}[]
graph LR
    A[Magnetic Flux Linkage] {-{-}{} B[Mutual Inductance]}
    B {-{-}{} C[Coupling Coefficient]}

    subgraph Formula Derivation
    D["φ12 = Flux from coil 1 to 2"]
    E["M = N2·φ12/I1"]
    F["k = M/(L1·L2)"]
    end
{Highlighting}
{Shaded}
\end{verbatim}
\end{center}

{\def\LTcaptype{none} % do not increment counter
\begin{longtable}[]{@{}lll@{}}
\toprule\noalign{}
Step & Description & Equation \\
\midrule\noalign{}
\endhead
\bottomrule\noalign{}
\endlastfoot
1 & Define mutual inductance & M = N_{2}·φ_{1}_{2}/I_{1} \\
2 & Define self-inductances & L_{1} = N_{1}·φ_{1}_{1}/I_{1}, L_{2} = N_{2}·φ_{2}_{2}/I_{2} \\
3 & Maximum possible M & Mmax = \sqrt(L_{1}·L_{2}) \\
4 & Define coupling coefficient & k = M/\sqrt(L_{1}·L_{2}) \\
\end{longtable}
}

\begin{itemize}
\tightlist
\item
  \textbf{Range}: 0 \leq k \leq 1
\item
  \textbf{Physical meaning}: Fraction of flux from one coil linking with
  the other coil
\item
  \textbf{Perfect coupling}: k = 1, when all flux links both coils
\end{itemize}

\end{solutionbox}
\begin{mnemonicbox}
``MASK'' - Mutual inductance And Self inductances
create K

\end{mnemonicbox}
\subsection*{Question 4(c) [7 marks]}\label{q4c}

\textbf{Derive equation of resonance frequency for series resonance.
Calculate resonant frequency, Q factor and bandwidth of series RLC
circuit with

R=20Ω,

L=1H,

C=1μF.}


\begin{solutionbox}

\textbf{Diagram:}

\begin{verbatim}
     R       L
    ┌─/{//─┬─OOOO─┐}
    │        │      │
  V ┘        │      │
    │        │      │
    └────────┴──||──┘
                C
\end{verbatim}

\textbf{Derivation:}

{\def\LTcaptype{none} % do not increment counter
\begin{longtable}[]{@{}lll@{}}
\toprule\noalign{}
Step & Description & Equation \\
\midrule\noalign{}
\endhead
\bottomrule\noalign{}
\endlastfoot
1 & Impedance of series RLC & Z = R + j(ωL - 1/ωC) \\
2 & At resonance, Im(Z) = 0 & ωL - 1/ωC = 0 \\
3 & Solve for resonant frequency & ω_{0} = 1/\sqrt(LC) or f_{0} = 1/(2π\sqrt(LC)) \\
\end{longtable}
}

\textbf{Calculations:}

{\def\LTcaptype{none} % do not increment counter
\begin{longtable}[]{@{}
  >{\raggedright\arraybackslash}p{(\linewidth - 6\tabcolsep) * \real{0.2683}}
  >{\raggedright\arraybackslash}p{(\linewidth - 6\tabcolsep) * \real{0.2195}}
  >{\raggedright\arraybackslash}p{(\linewidth - 6\tabcolsep) * \real{0.3171}}
  >{\raggedright\arraybackslash}p{(\linewidth - 6\tabcolsep) * \real{0.1951}}@{}}
\toprule\noalign{}
\begin{minipage}[b]{\linewidth}\raggedright
Parameter
\end{minipage} & \begin{minipage}[b]{\linewidth}\raggedright
Formula
\end{minipage} & \begin{minipage}[b]{\linewidth}\raggedright
Calculation
\end{minipage} & \begin{minipage}[b]{\linewidth}\raggedright
Result
\end{minipage} \\
\midrule\noalign{}
\endhead
\bottomrule\noalign{}
\endlastfoot
Resonant frequency & f_{0} = 1/(2π\sqrt(LC)) & f_{0} = 1/(2π\sqrt(1\times10^{-}^{6})) & 159.15
Hz \\
Q factor &

Q = ω_{0}L/R &

Q = 2π\times159.15\times1/20 & 50 \\

Bandwidth & BW = f_{0}/Q & BW = 159.15/50 & 3.18 Hz \\
\end{longtable}
}

\end{solutionbox}
\begin{mnemonicbox}
``FQBR'' - Frequency from reactances, Q from
resistance ratio, Bandwidth from Resonance divided by Q

\end{mnemonicbox}
\subsection*{Question 4(a OR) [3
marks]}\label{question-4a-or-3-marks}

\textbf{Explain Quality factor.}

\begin{solutionbox}

\textbf{Diagram:}

\begin{center}
\textbf{Mermaid Diagram (Code)}
\begin{verbatim}
{Shaded}
{Highlighting}[]
graph TD
    A[Quality Factor] {-{-}{} B[Energy Storage]}
    A {-{-}{} C[Power Loss]}
    A {-{-}{} D[Selectivity]}
    A {-{-}{} E[Bandwidth]}

    style A fill:\#bbf,stroke:\#333
{Highlighting}
{Shaded}
\end{verbatim}
\end{center}

\textbf{Quality Factor (Q)}: A dimensionless parameter that indicates
how under-damped a resonator is, or alternatively, characterizes a
resonator's bandwidth relative to its center frequency.

{\def\LTcaptype{none} % do not increment counter
\begin{longtable}[]{@{}
  >{\raggedright\arraybackslash}p{(\linewidth - 2\tabcolsep) * \real{0.3243}}
  >{\raggedright\arraybackslash}p{(\linewidth - 2\tabcolsep) * \real{0.6757}}@{}}
\toprule\noalign{}
\begin{minipage}[b]{\linewidth}\raggedright
Definition
\end{minipage} & \begin{minipage}[b]{\linewidth}\raggedright
Mathematical Expression
\end{minipage} \\
\midrule\noalign{}
\endhead
\bottomrule\noalign{}
\endlastfoot
Energy perspective & Q = 2π \times Energy stored / Energy dissipated per
cycle \\
Circuit perspective & Q = X/R (where X is reactance, R is resistance) \\
Frequency perspective & Q = f_{0}/BW (where f_{0} is resonant frequency, BW is
bandwidth) \\
\end{longtable}
}

\end{solutionbox}
\begin{mnemonicbox}
``QSEL'' - Quality shows Energy vs.~Loss and
Selectivity

\end{mnemonicbox}
\subsection*{Question 4(b OR) [4
marks]}\label{question-4b-or-4-marks}

\textbf{Derive the equation of quality factor for a capacitor.}

\begin{solutionbox}

\textbf{Diagram:}

\begin{verbatim}
    Ideal C    ESR
      ||      /{//}
      ||        R
     ┌||┐      ┌─┐
     │  │      │ │
     │  │      │ │
     └──┘      └─┘
    Real capacitor model
\end{verbatim}

\textbf{Derivation:}

{\def\LTcaptype{none} % do not increment counter
\begin{longtable}[]{@{}lll@{}}
\toprule\noalign{}
Step & Description & Equation \\
\midrule\noalign{}
\endhead
\bottomrule\noalign{}
\endlastfoot
1 & Define energy stored & Estored = CV^{2}/2 \\
2 & Define energy loss per cycle & Eloss = πCV^{2}/ωCR = πV^{2}/ωR \\
3 & Define Q factor & Q = 2π \times Estored / Eloss \\
4 & Substitute and simplify &

Q = 2π \times (CV^{2}/2) \div (πV^{2}/ωR) = ωCR \\

\end{longtable}
}

\textbf{Final equation:} Q = ωCR = 1/(ωRC) = 1/tanδ

Where:

\begin{itemize}
\tightlist
\item
  ω = Angular frequency (2πf)
\item
  R = Equivalent series resistance (ESR)
\item
  C = Capacitance
\item
  tanδ = Dissipation factor
\end{itemize}

\end{solutionbox}
\begin{mnemonicbox}
``CORE'' - Capacitors' Quality equals One over
Resistance times Capacitance

\end{mnemonicbox}
\subsection*{Question 4(c OR) [7
marks]}\label{question-4c-or-7-marks}

\textbf{Derive equation of resonance frequency for parallel resonance.
Calculate resonant frequency, Q factor and bandwidth of parallel RLC
circuit with

R=30Ω,

L=1H,

C=1μF.}


\begin{solutionbox}

\textbf{Diagram:}

\begin{verbatim}
             L
    ┌────┬─OOOO─┬────┐
    │    │      │    │
  V ┘  R │      │ C  │
    │    │      │    │
    └────┴──────┴────┘
\end{verbatim}

\textbf{Derivation:}

{\def\LTcaptype{none} % do not increment counter
\begin{longtable}[]{@{}lll@{}}
\toprule\noalign{}
Step & Description & Equation \\
\midrule\noalign{}
\endhead
\bottomrule\noalign{}
\endlastfoot
1 & Admittance of parallel RLC & Y = 1/R + 1/jωL + jωC \\
2 & At resonance, Im(Y) = 0 & 1/jωL + jωC = 0 \\
3 & Solve for resonant frequency & ω_{0} = 1/\sqrt(LC) or f_{0} = 1/(2π\sqrt(LC)) \\
\end{longtable}
}

\textbf{Calculations:}

{\def\LTcaptype{none} % do not increment counter
\begin{longtable}[]{@{}
  >{\raggedright\arraybackslash}p{(\linewidth - 6\tabcolsep) * \real{0.2683}}
  >{\raggedright\arraybackslash}p{(\linewidth - 6\tabcolsep) * \real{0.2195}}
  >{\raggedright\arraybackslash}p{(\linewidth - 6\tabcolsep) * \real{0.3171}}
  >{\raggedright\arraybackslash}p{(\linewidth - 6\tabcolsep) * \real{0.1951}}@{}}
\toprule\noalign{}
\begin{minipage}[b]{\linewidth}\raggedright
Parameter
\end{minipage} & \begin{minipage}[b]{\linewidth}\raggedright
Formula
\end{minipage} & \begin{minipage}[b]{\linewidth}\raggedright
Calculation
\end{minipage} & \begin{minipage}[b]{\linewidth}\raggedright
Result
\end{minipage} \\
\midrule\noalign{}
\endhead
\bottomrule\noalign{}
\endlastfoot
Resonant frequency & f_{0} = 1/(2π\sqrt(LC)) & f_{0} = 1/(2π\sqrt(1\times10^{-}^{6})) & 159.15
Hz \\
Q factor &

Q = R/ω_{0}L &

Q = 30/(2π\times159.15\times1) & 0.03 \\

Bandwidth & BW = f_{0}/Q & BW = 159.15/0.03 & 5305 Hz \\
\end{longtable}
}

\end{solutionbox}
\begin{mnemonicbox}
``FPQB'' - Frequency from Parallel elements, Q from
Resistance divided by reactance, Bandwidth from division

\end{mnemonicbox}
\subsection*{Question 5(a) [3 marks]}\label{q5a}

\textbf{Explain the T type attenuator.}

\begin{solutionbox}

\textbf{Diagram:}

\begin{verbatim}
    Z1
    ┌─/{//─┐}
    │        │
 In ┘    Z3  │      Out
    │   /{//│       │}
    └─/{//─┘       │}
        Z2
\end{verbatim}

\textbf{T-type Attenuator}: A passive network in T configuration used to
reduce signal amplitude.

{\def\LTcaptype{none} % do not increment counter
\begin{longtable}[]{@{}lll@{}}
\toprule\noalign{}
Component & Description & Formula \\
\midrule\noalign{}
\endhead
\bottomrule\noalign{}
\endlastfoot
\textbf{Z1, Z2} & Series arms & Z1 = Z2 = Z_{0}(N-1)/(N+1) \\
\textbf{Z3} & Shunt arm & Z3 = 2Z_{0}/(N^{2}-1) \\
\textbf{N} & Attenuation ratio & N = 10\^{}(dB/20) \\
\end{longtable}
}

\begin{itemize}
\tightlist
\item
  \textbf{Characteristic}: Symmetrical for matched source and load
\item
  \textbf{Applications}: Signal level control, impedance matching
\item
  \textbf{Advantage}: Maintains impedance matching with proper design
\end{itemize}

\end{solutionbox}
\begin{mnemonicbox}
``TSAR'' - T-shape with Series Arms and Resistance in
middle

\end{mnemonicbox}
\subsection*{Question 5(b) [4 marks]}\label{q5b}

\textbf{Classify the various passive filter circuits.}

\begin{solutionbox}

\textbf{Diagram:}

\begin{center}
\textbf{Mermaid Diagram (Code)}
\begin{verbatim}
{Shaded}
{Highlighting}[]
graph TD
    A[Passive Filters]
    A {-{-}{} B[Based on Frequency Response]}
    A {-{-}{} C[Based on Configuration]}

    B {-{-}{} D[Low Pass]}
    B {-{-}{} E[High Pass]}
    B {-{-}{} F[Band Pass]}
    B {-{-}{} G[Band Stop]}
    
    C {-{-}{} H[T{-}section]}
    C {-{-}{} I[π{-}section]}
    C {-{-}{} J[L{-}section]}
    C {-{-}{} K[Lattice]}
{Highlighting}
{Shaded}
\end{verbatim}
\end{center}

{\def\LTcaptype{none} % do not increment counter
\begin{longtable}[]{@{}
  >{\raggedright\arraybackslash}p{(\linewidth - 6\tabcolsep) * \real{0.2453}}
  >{\raggedright\arraybackslash}p{(\linewidth - 6\tabcolsep) * \real{0.1887}}
  >{\raggedright\arraybackslash}p{(\linewidth - 6\tabcolsep) * \real{0.3019}}
  >{\raggedright\arraybackslash}p{(\linewidth - 6\tabcolsep) * \real{0.2642}}@{}}
\toprule\noalign{}
\begin{minipage}[b]{\linewidth}\raggedright
Filter Type
\end{minipage} & \begin{minipage}[b]{\linewidth}\raggedright
Function
\end{minipage} & \begin{minipage}[b]{\linewidth}\raggedright
Typical Circuit
\end{minipage} & \begin{minipage}[b]{\linewidth}\raggedright
Applications
\end{minipage} \\
\midrule\noalign{}
\endhead
\bottomrule\noalign{}
\endlastfoot
\textbf{Low Pass} & Passes low frequencies & RC, RL circuits & Audio
filters, Power supplies \\
\textbf{High Pass} & Passes high frequencies & CR, LR circuits & Noise
filtering, Signal conditioning \\
\textbf{Band Pass} & Passes a band of frequencies & RLC circuits & Radio
tuning, Signal selection \\
\textbf{Band Stop} & Blocks a band of frequencies & Parallel RLC &
Interference rejection \\
\end{longtable}
}

\end{solutionbox}
\begin{mnemonicbox}
``LHBB'' - Low High Band Band filters for Pass and
Block

\end{mnemonicbox}
\subsection*{Question 5(c) [7 marks]}\label{q5c}

\textbf{Design constant-k type low pass and High pass filter with
T-section having cutoff frequency= 1000Hz \& load of 500Ω.}

\begin{solutionbox}

\textbf{Diagram:}

\begin{verbatim}
Low Pass T{-Filter          High Pass T{-}Filter}
     L/2       L/2              C/2       C/2
   {-{-}OOOO{-}{-}{-}{-}{-}{-}OOOO{-}{-}        {-}{-}{-}||{-}{-}{-}{-}{-}{-}{-}{-}||{-}{-}{-}}
   |                |        |                |
   |                |        |                |
   |       C        |        |       L        |
   |       ||       |        |      OOOO      |
   |                |        |                |
 {-{-}{-}{-}{-}{-}{-}{-}{-}{-}{-}{-}{-}{-}{-}{-}{-}{-}{-}{-}{-}{-}    {-}{-}{-}{-}{-}{-}{-}{-}{-}{-}{-}{-}{-}{-}{-}{-}{-}{-}{-}{-}{-}{-}}
\end{verbatim}

\textbf{Design Calculations:}

For Constant-k T-type low pass filter:

{\def\LTcaptype{none} % do not increment counter
\begin{longtable}[]{@{}llll@{}}
\toprule\noalign{}
Parameter & Formula & Calculation & Value \\
\midrule\noalign{}
\endhead
\bottomrule\noalign{}
\endlastfoot
Cut-off frequency & fc = 1000 Hz & Given & 1000 Hz \\
Load impedance & R_{0} = 500 Ω & Given & 500 Ω \\
Series inductor &

L = R_{0}/πfc &

L = 500/(π\times1000) & 159.15 mH \\

Half sections & L/2 & 159.15/2 & 79.58 mH \\
Shunt capacitor &

C = 1/(πfcR_{0}) &

C = 1/(π\times1000\times500) & 0.636 μF \\

\end{longtable}
}

For Constant-k T-type high pass filter:

{\def\LTcaptype{none} % do not increment counter
\begin{longtable}[]{@{}llll@{}}
\toprule\noalign{}
Parameter & Formula & Calculation & Value \\
\midrule\noalign{}
\endhead
\bottomrule\noalign{}
\endlastfoot
Series capacitor &

C = 1/(4πfcR_{0}) &

C = 1/(4π\times1000\times500) & 0.159 μF \\

Half sections & C/2 & 0.159/2 & 0.0795 μF \\
Shunt inductor &

L = R_{0}/(4πfc) &

L = 500/(4π\times1000) & 39.79 mH \\

\end{longtable}
}

\end{solutionbox}
\begin{mnemonicbox}
``FRED'' - Frequency Ratio determines Element
Dimensions

\end{mnemonicbox}
\subsection*{Question 5(a OR) [3
marks]}\label{question-5a-or-3-marks}

\textbf{Explain the π type attenuator.}

\begin{solutionbox}

\textbf{Diagram:}

\begin{verbatim}
          Z2
          │
    ┌────┐│┌────┐
    │    ││     │
 In ┘  Z1│││Z3  │  Out
    │    ││     │
    └────┘│└────┘
          │
\end{verbatim}

\textbf{π-type Attenuator}: A passive network in π configuration used to
reduce signal amplitude.

{\def\LTcaptype{none} % do not increment counter
\begin{longtable}[]{@{}lll@{}}
\toprule\noalign{}
Component & Description & Formula \\
\midrule\noalign{}
\endhead
\bottomrule\noalign{}
\endlastfoot
\textbf{Z2} & Series arm & Z2 = 2Z_{0}/(N^{2}-1) \\
\textbf{Z1, Z3} & Shunt arms & Z1 = Z3 = Z_{0}(N+1)/(N-1) \\
\textbf{N} & Attenuation ratio & N = 10\^{}(dB/20) \\
\end{longtable}
}

\begin{itemize}
\tightlist
\item
  \textbf{Characteristic}: Symmetrical for matched source and load
\item
  \textbf{Applications}: Signal level control, impedance matching
\item
  \textbf{Advantage}: Good isolation between input and output
\end{itemize}

\end{solutionbox}
\begin{mnemonicbox}
``PASS'' - Pi-Attenuator has Series in middle and
Shunt arms outside

\end{mnemonicbox}
\subsection*{Question 5(b OR) [4
marks]}\label{question-5b-or-4-marks}

\textbf{Classify various types of attenuators.}

\begin{solutionbox}

\textbf{Diagram:}

\begin{center}
\textbf{Mermaid Diagram (Code)}
\begin{verbatim}
{Shaded}
{Highlighting}[]
graph TD
    A[Attenuators]
    A {-{-}{} B[Based on Structure]}
    A {-{-}{} C[Based on Function]}

    B {-{-}{} D[T{-}type]}
    B {-{-}{} E[π{-}type]}
    B {-{-}{} F[L{-}type]}
    B {-{-}{} G[Bridged{-}T]}
    B {-{-}{} H[Lattice]}
    
    C {-{-}{} I[Fixed]}
    C {-{-}{} J[Variable]}
    C {-{-}{} K[Stepped]}
    C {-{-}{} L[Programmable]}
{Highlighting}
{Shaded}
\end{verbatim}
\end{center}

{\def\LTcaptype{none} % do not increment counter
\begin{longtable}[]{@{}
  >{\raggedright\arraybackslash}p{(\linewidth - 6\tabcolsep) * \real{0.2833}}
  >{\raggedright\arraybackslash}p{(\linewidth - 6\tabcolsep) * \real{0.2833}}
  >{\raggedright\arraybackslash}p{(\linewidth - 6\tabcolsep) * \real{0.2333}}
  >{\raggedright\arraybackslash}p{(\linewidth - 6\tabcolsep) * \real{0.2000}}@{}}
\toprule\noalign{}
\begin{minipage}[b]{\linewidth}\raggedright
Attenuator Type
\end{minipage} & \begin{minipage}[b]{\linewidth}\raggedright
Characteristics
\end{minipage} & \begin{minipage}[b]{\linewidth}\raggedright
Applications
\end{minipage} & \begin{minipage}[b]{\linewidth}\raggedright
Advantages
\end{minipage} \\
\midrule\noalign{}
\endhead
\bottomrule\noalign{}
\endlastfoot
\textbf{T-type} & Series-Shunt-Series & Audio systems & Simple design \\
\textbf{π-type} & Shunt-Series-Shunt & RF circuits & Better isolation \\
\textbf{L-type} & Series-Shunt & Simple matching & Impedance
transformation \\
\textbf{Bridged-T} & Balanced structure & Test equipment & Minimal
distortion \\
\textbf{Balanced} & Symmetric dual paths & Differential signals & Common
mode rejection \\
\end{longtable}
}

\end{solutionbox}
\begin{mnemonicbox}
``TPLBV'' - T, Pi, L, Bridged-T, and Variable
attenuators

\end{mnemonicbox}
\subsection*{Question 5(c OR) [7
marks]}\label{question-5c-or-7-marks}

\textbf{Design a symmetrical T type attenuator and π type attenuator to
give attenuation of 40dB and to work into the load of 500Ω.}

\begin{solutionbox}

\textbf{Diagram:}

\begin{verbatim}
T{-type Attenuator           π{-}type Attenuator}
     R1       R1                     R2
   {-{-}//{-}{-}{-}{-}//{-}{-}              {-}{-}//{-}{-}}
   |               |            |        |
   |               |            |        |
   |       R2      |            R1      R1
   |      /{/     |           //    //}
   |               |            |        |
   {-{-}{-}{-}{-}{-}{-}{-}{-}{-}{-}{-}{-}{-}{-}{-}{-}            {-}{-}{-}{-}{-}{-}{-}{-}{-}{-}}
\end{verbatim}

\textbf{Design Calculations:}

{\def\LTcaptype{none} % do not increment counter
\begin{longtable}[]{@{}llll@{}}
\toprule\noalign{}
Step & Formula & Calculation & Value \\
\midrule\noalign{}
\endhead
\bottomrule\noalign{}
\endlastfoot
Given & Attenuation = 40 dB & - & 40 dB \\
Step 1 & N = 10\^{}(dB/20) & 10\^{}(40/20) & 100 \\
Step 2 & K = (N-1)/(N+1) & (100-1)/(100+1) & 0.98 \\
\end{longtable}
}

For T-type attenuator:

{\def\LTcaptype{none} % do not increment counter
\begin{longtable}[]{@{}llll@{}}
\toprule\noalign{}
Component & Formula & Calculation & Value \\
\midrule\noalign{}
\endhead
\bottomrule\noalign{}
\endlastfoot
R_{1} (series) & Z_{0}·K & 500 \times 0.98 & 490 Ω \\
R_{2} (shunt) & Z_{0}/(K·(N-K)) & 500/(0.98\times(100-0.98)) & 5.15 Ω \\
\end{longtable}
}

For π-type attenuator:

{\def\LTcaptype{none} % do not increment counter
\begin{longtable}[]{@{}llll@{}}
\toprule\noalign{}
Component & Formula & Calculation & Value \\
\midrule\noalign{}
\endhead
\bottomrule\noalign{}
\endlastfoot
R_{1} (shunt) & Z_{0}/K & 500/0.98 & 510.2 Ω \\
R_{2} (series) & Z_{0}·K·(N-K) & 500 \times 0.98 \times (100-0.98) & 48,541 Ω \\
\end{longtable}
}

\end{solutionbox}
\begin{mnemonicbox}
``DANK'' - dB Attenuation is Number K, which
determines resistor values

\end{mnemonicbox}

\end{document}
