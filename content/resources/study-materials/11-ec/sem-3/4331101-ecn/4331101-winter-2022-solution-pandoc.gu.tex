\documentclass[10pt,a4paper]{article}

% content/resources/templates/preamble.tex
\usepackage[margin=0.6in]{geometry}
\author{Milav Dabgar}
\usepackage{amsmath,amssymb,amsthm}
\usepackage{booktabs}
\usepackage{multirow}
\usepackage{xcolor}
\usepackage{tcolorbox}
\tcbuselibrary{breakable,skins}
\usepackage[colorlinks=true,linkcolor=blue]{hyperref}
\usepackage{titlesec}
\usepackage{enumitem}
\usepackage{tikz}
\usepackage{pgfplots}
\usepackage{circuitikz}
\usepackage[version=4]{mhchem}
\usepackage{longtable}
\usepackage{array}
\usepackage{float}
\usepackage{caption}
\usepackage{listings}

\lstset{
  basicstyle=\small\ttfamily,
  breaklines=true,
  breakatwhitespace=false,
  postbreak=\mbox{\textcolor{red}{$\hookrightarrow$}\space},
  float=false,
  numbers=left,
  numberstyle=\tiny\color{gray},
  numbersep=10pt,
  xleftmargin=2em,
  keywordstyle=\color{blue},
  commentstyle=\color{green!60!black},
  stringstyle=\color{purple},
  backgroundcolor=\color{gray!5},
  showstringspaces=false,
  tabsize=2,
  captionpos=b,
  keepspaces=true,
  columns=flexible
}

\pgfplotsset{compat=1.18}
\usetikzlibrary{shapes,arrows,positioning,calc,patterns,decorations.pathmorphing,decorations.markings,arrows.meta}

% Color scheme
\definecolor{headcolor}{RGB}{0,102,204}
\definecolor{keycolor}{RGB}{220,20,60}
\definecolor{solutioncolor}{RGB}{34,139,34}
\definecolor{mnemoniccolor}{RGB}{148,0,211}
\definecolor{codecolor}{RGB}{0,0,100}

% Spacing
\setlength{\parskip}{3pt}
\setlist[itemize]{nosep}
\setlist[enumerate]{nosep}

% Title formatting
\titleformat{\section}{\Large\bfseries\color{headcolor}}{\thesection}{1em}{}
\titleformat{\subsection}{\large\bfseries\color{headcolor}}{\thesubsection}{1em}{}

% Pandoc tightlist compatibility
\providecommand{\tightlist}{%
  \setlength{\itemsep}{0pt}\setlength{\parskip}{0pt}}

% Pandoc longtable compatibility
\newcounter{none}
\def\thenone{}


% content/resources/templates/gujarati-boxes.tex
\usepackage{fontspec}
\usepackage{polyglossia}

% Set Gujarati as main language (document is primarily in Gujarati)
% Note: gloss-gujarati.ldf doesn't exist in polyglossia, but it will use hyphenation patterns
\setdefaultlanguage{gujarati}
\setotherlanguage{english}

% Configure Gujarati font properly
% Use Language=Default to prevent polyglossia from trying to add language-specific features
% that don't exist for Gujarati, which causes "empty feature" warnings
\newfontfamily\gujaratifont[Script=Gujarati,AutoFakeBold=2.5,AutoFakeSlant=0.3]{Noto Sans Gujarati}
\setmainfont[Script=Gujarati,AutoFakeBold=2.5,AutoFakeSlant=0.3]{Noto Sans Gujarati}
% Use Noto Sans Gujarati for monospace to support Gujarati in text
\setmonofont[Scale=0.9]{Noto Sans Gujarati}

% Configure English to use the same font
\newfontfamily\englishfont[Script=Gujarati,AutoFakeBold=2.5,AutoFakeSlant=0.3]{Noto Sans Gujarati}

% Translations for polyglossia
\gappto\captionsgujarati{
  \renewcommand{\tablename}{કોષ્ટક}
  \renewcommand{\figurename}{આકૃતિ}
}

% Helper for TikZ nodes to ensure Gujarati font
\newcommand{\gu}[1]{{\gujaratifont #1}}

% Custom environments
\newtcolorbox{solutionbox}{
    breakable,
    enhanced,
    colback=solutioncolor!5!white,
    colframe=solutioncolor!75!black,
    fonttitle=\bfseries,
    title=જવાબ
}

\newtcolorbox{solutionboxnobreak}{
 colback=solutioncolor!5!white,
 colframe=solutioncolor!75!black,
 fonttitle=\bfseries,
 title=જવાબ
}

\newtcolorbox{keyformula}{
 breakable,
 enhanced,
 colback=keycolor!5!white,
 colframe=keycolor!75!black,
 fonttitle=\bfseries,
 title=રાસાયણિક સમીકરણ/સૂત્ર
}

\newtcolorbox{mnemonicbox}{
 breakable,
 enhanced,
 colback=mnemoniccolor!5!white,
 colframe=mnemoniccolor!75!black,
 fonttitle=\bfseries,
 title=મેમરી ટ્રીક
}


\begin{document}

\begin{center}
{\Huge\bfseries\color{headcolor} Subject Name (Gujarati)}\\[5pt]
{\LARGE 4331101 -- Winter 2022}\\[3pt]
{\large Semester 1 Study Material}\\[3pt]
{\normalsize\textit{Detailed Solutions and Explanations}}
\end{center}

\vspace{10pt}

\subsection*{પ્રશ્ન 1(અ) [3
ગુણ]}\label{uxaaauxab0uxab6uxaa8-1uxa85-3-uxa97uxaa3}

\textbf{વ્યાખ્યા આપો. : ૧) બ્રાંચ ૨) જંક્શન ૩) મેશ}

\begin{solutionbox}

\begin{itemize}
\tightlist
\item
  \textbf{બ્રાંચ}: બ્રાંચ એટલે એક અથવા વધારે સર્કિટ તત્વો જે નેટવર્કના બે નોડ્સ વચ્ચે
  જોડાયેલા હોય.
\item
  \textbf{જંક્શન}: જંક્શન (અથવા નોડ) એટલે એવું બિંદુ જ્યાં બે અથવા વધારે સર્કિટ તત્વો
  એકબીજા સાથે જોડાયેલા હોય.
\item
  \textbf{મેશ}: મેશ એટલે નેટવર્કમાં એક બંધ પથ જેમાં અન્ય કોઈ બંધ પથ તેની અંદર ન હોય.
\end{itemize}

\end{solutionbox}
\begin{mnemonicbox}
``BJM: Branches Join at junctions to Make meshes''

\end{mnemonicbox}
\subsection*{પ્રશ્ન 1(બ) [4
ગુણ]}\label{uxaaauxab0uxab6uxaa8-1uxaac-4-uxa97uxaa3}

\textbf{જરુરી સર્કિટ સાથે વોલ્ટેજ અને કરંટ ડિવિઝન નો નિયમ લખો.}

\begin{solutionbox}

\textbf{વોલ્ટેજ ડિવિઝન નિયમ}: સિરીઝ સર્કિટમાં, કોઈપણ ઘટક પરનો વોલ્ટેજ તેના
રેઝિસ્ટન્સના પ્રમાણમાં હોય છે.

\begin{center}
\textbf{Mermaid Diagram (Code)}
\begin{verbatim}
{Shaded}
{Highlighting}[]
graph LR
    A(({+)) {-}{-}{-} B[R1] {-}{-}{-} C[R2] {-}{-}{-} D((–))}
    E[V1] {-.{-} B}
    F[V2] {-.{-} C}
    G[VS] {-.{-} A}
{Highlighting}
{Shaded}
\end{verbatim}
\end{center}

\begin{itemize}
\tightlist
\item
  \textbf{સૂત્ર}: V_{1} = VS \times (R_{1}/(R_{1}+R_{2}))
\item
  \textbf{ઉપયોગ}: સિરીઝ ઘટકો પરના વ્યક્તિગત વોલ્ટેજ ડ્રોપ્સ શોધવા માટે વપરાય છે
\end{itemize}

\textbf{કરંટ ડિવિઝન નિયમ}: પેરેલલ સર્કિટમાં, કોઈપણ શાખામાંથી પસાર થતો કરંટ તેના
રેઝિસ્ટન્સના વ્યસ્ત પ્રમાણમાં હોય છે.

\begin{center}
\textbf{Mermaid Diagram (Code)}
\begin{verbatim}
{Shaded}
{Highlighting}[]
graph LR
    A(({+)) {-}{-}{-} B {-}{-}{-} C((–))}
    B {-{-}{-} D[R1] {-}{-}{-} C}
    B {-{-}{-} E[R2] {-}{-}{-} C}
    F[I1] {-.{-} D}
    G[I2] {-.{-} E}
    H[IS] {-.{-} A}
{Highlighting}
{Shaded}
\end{verbatim}
\end{center}

\begin{itemize}
\tightlist
\item
  \textbf{સૂત્ર}: I_{1} = IS \times (R_{2}/(R_{1}+R_{2}))
\item
  \textbf{મુખ્ય સિદ્ધાંત}: કરંટ ઓછા રેઝિસ્ટન્સનો માર્ગ પસંદ કરે છે
\end{itemize}

\end{solutionbox}
\begin{mnemonicbox}
``VoSe CuPa: Voltage divides in Series, Current
divides in Parallel''

\end{mnemonicbox}
\subsection*{પ્રશ્ન 1(ક) [7
ગુણ]}\label{uxaaauxab0uxab6uxaa8-1uxa95-7-uxa97uxaa3}

\textbf{Fig. (૧) માં બતાવેલ નેટવર્ક માટે ગ્રાફ અને ટ્રી દોરો. ગ્રાફ પર લિંક કરંટ
બતાવો. સાથે ટ્રી માટે ટાઇ-સેટ સેડ્યુલ લખો.}

\begin{solutionbox}

\textbf{નેટવર્કનો ગ્રાફ}:

\begin{center}
\textbf{Mermaid Diagram (Code)}
\begin{verbatim}
{Shaded}
{Highlighting}[]
graph LR
    A((A)) {-{-}{-} B((B))}
    A {-{-}{-} C((C))}
    A {-{-}{-} D((D))}
    B {-{-}{-} C}
    B {-{-}{-} D}
    C {-{-}{-} D}
    A {-{-} 1 {-}{-}{-} B}
    A {-{-} 3 {-}{-}{-} C}
    B {-{-} 2 {-}{-}{-} D}
    C {-{-} 5 {-}{-}{-} D}
    B {-{-} 6 {-}{-}{-} C}
    A {-{-} 7 {-}{-}{-} D}
    style A fill:\#f9f,stroke:\#333,stroke{-width:2px}
    style B fill:\#f9f,stroke:\#333,stroke{-width:2px}
    style C fill:\#f9f,stroke:\#333,stroke{-width:2px}
    style D fill:\#f9f,stroke:\#333,stroke{-width:2px}
{Highlighting}
{Shaded}
\end{verbatim}
\end{center}

\textbf{નેટવર્કનું ટ્રી} (બોલ્ડ એજ સાથે બતાવેલ):

\begin{center}
\textbf{Mermaid Diagram (Code)}
\begin{verbatim}
{Shaded}
{Highlighting}[]
graph LR
    A((A)) {-{-}{-} B((B))}
    A {-{-}{-} C((C))}
    C {-{-}{-} D((D))}
    style A fill:\#f9f,stroke:\#333,stroke{-width:2px}
    style B fill:\#f9f,stroke:\#333,stroke{-width:2px}
    style C fill:\#f9f,stroke:\#333,stroke{-width:2px}
    style D fill:\#f9f,stroke:\#333,stroke{-width:2px}
    linkStyle 0 stroke{-width:4px,stroke:green}
    linkStyle 1 stroke{-width:4px,stroke:green}
    linkStyle 2 stroke{-width:4px,stroke:green}
{Highlighting}
{Shaded}
\end{verbatim}
\end{center}

\textbf{લિંક કરંટ} (બાકીની શાખાઓ પર બતાવેલ જે ટ્રીનો ભાગ નથી):

\begin{itemize}
\tightlist
\item
  લિંક 1: શાખા 2 (BD)
\item
  લિંક 2: શાખા 6 (BC)
\item
  લિંક 3: શાખા 7 (AD)
\item
  લિંક 4: શાખા 5 (CD)
\end{itemize}

\textbf{ટાઇ-સેટ સેડ્યુલ}:

{\def\LTcaptype{none} % do not increment counter
\begin{longtable}[]{@{}
  >{\raggedright\arraybackslash}p{(\linewidth - 14\tabcolsep) * \real{0.1463}}
  >{\raggedright\arraybackslash}p{(\linewidth - 14\tabcolsep) * \real{0.1220}}
  >{\raggedright\arraybackslash}p{(\linewidth - 14\tabcolsep) * \real{0.1220}}
  >{\raggedright\arraybackslash}p{(\linewidth - 14\tabcolsep) * \real{0.1220}}
  >{\raggedright\arraybackslash}p{(\linewidth - 14\tabcolsep) * \real{0.1220}}
  >{\raggedright\arraybackslash}p{(\linewidth - 14\tabcolsep) * \real{0.1220}}
  >{\raggedright\arraybackslash}p{(\linewidth - 14\tabcolsep) * \real{0.1220}}
  >{\raggedright\arraybackslash}p{(\linewidth - 14\tabcolsep) * \real{0.1220}}@{}}
\toprule\noalign{}
\begin{minipage}[b]{\linewidth}\raggedright
લિંક/ટ્રી શાખા
\end{minipage} & \begin{minipage}[b]{\linewidth}\raggedright
શાખા 1 (AB)
\end{minipage} & \begin{minipage}[b]{\linewidth}\raggedright
શાખા 3 (AC)
\end{minipage} & \begin{minipage}[b]{\linewidth}\raggedright
શાખા 4 (CD)
\end{minipage} & \begin{minipage}[b]{\linewidth}\raggedright
શાખા 2 (BD)
\end{minipage} & \begin{minipage}[b]{\linewidth}\raggedright
શાખા 6 (BC)
\end{minipage} & \begin{minipage}[b]{\linewidth}\raggedright
શાખા 7 (AD)
\end{minipage} & \begin{minipage}[b]{\linewidth}\raggedright
શાખા 5 (CD)
\end{minipage} \\
\midrule\noalign{}
\endhead
\bottomrule\noalign{}
\endlastfoot
લિંક 1 (BD) & 1 & 0 & 0 & 1 & 0 & 0 & 0 \\
લિંક 2 (BC) & 1 & 1 & 0 & 0 & 1 & 0 & 0 \\
લિંક 3 (AD) & 0 & 0 & 1 & 0 & 0 & 1 & 0 \\
લિંક 4 (CD) & 0 & 0 & 1 & 0 & 0 & 0 & 1 \\
\end{longtable}
}

\end{solutionbox}
\begin{mnemonicbox}
``TGLT: Trees Generate Link-current Tie-sets''

\end{mnemonicbox}
\subsection*{પ્રશ્ન 1(ક) OR [7
ગુણ]}\label{uxaaauxab0uxab6uxaa8-1uxa95-or-7-uxa97uxaa3}

\textbf{Fig. (૧) માં બતાવેલ નેટવર્ક માટે ગ્રાફ અને ટ્રી દોરો. ટ્રી પર બ્રાંચ વોલ્ટેજ
બતાવો. સાથે ટ્રી માટે કટ-સેટ સેડ્યુલ લખો.}

\begin{solutionbox}

\textbf{નેટવર્કનો ગ્રાફ}:

\begin{center}
\textbf{Mermaid Diagram (Code)}
\begin{verbatim}
{Shaded}
{Highlighting}[]
graph LR
    A((A)) {-{-}{-} B((B))}
    A {-{-}{-} C((C))}
    A {-{-}{-} D((D))}
    B {-{-}{-} C}
    B {-{-}{-} D}
    C {-{-}{-} D}
    A {-{-} 1 {-}{-}{-} B}
    A {-{-} 3 {-}{-}{-} C}
    B {-{-} 2 {-}{-}{-} D}
    C {-{-} 5 {-}{-}{-} D}
    B {-{-} 6 {-}{-}{-} C}
    A {-{-} 7 {-}{-}{-} D}
    style A fill:\#f9f,stroke:\#333,stroke{-width:2px}
    style B fill:\#f9f,stroke:\#333,stroke{-width:2px}
    style C fill:\#f9f,stroke:\#333,stroke{-width:2px}
    style D fill:\#f9f,stroke:\#333,stroke{-width:2px}
{Highlighting}
{Shaded}
\end{verbatim}
\end{center}

\textbf{નેટવર્કનું ટ્રી} (બોલ્ડ એજ સાથે બતાવેલ અને બ્રાંચ વોલ્ટેજ સાથે):

\begin{center}
\textbf{Mermaid Diagram (Code)}
\begin{verbatim}
{Shaded}
{Highlighting}[]
graph LR
    A((A)) {-{-}"V_{1}"{-}{-}{} B((B))}
    A {-{-}"V_{3}"{-}{-}{} C((C))}
    C {-{-}"V_{4}"{-}{-}{} D((D))}
    style A fill:\#f9f,stroke:\#333,stroke{-width:2px}
    style B fill:\#f9f,stroke:\#333,stroke{-width:2px}
    style C fill:\#f9f,stroke:\#333,stroke{-width:2px}
    style D fill:\#f9f,stroke:\#333,stroke{-width:2px}
    linkStyle 0 stroke{-width:4px,stroke:green}
    linkStyle 1 stroke{-width:4px,stroke:green}
    linkStyle 2 stroke{-width:4px,stroke:green}
{Highlighting}
{Shaded}
\end{verbatim}
\end{center}

\textbf{કટ-સેટ સેડ્યુલ}:

{\def\LTcaptype{none} % do not increment counter
\begin{longtable}[]{@{}
  >{\raggedright\arraybackslash}p{(\linewidth - 14\tabcolsep) * \real{0.1322}}
  >{\raggedright\arraybackslash}p{(\linewidth - 14\tabcolsep) * \real{0.1240}}
  >{\raggedright\arraybackslash}p{(\linewidth - 14\tabcolsep) * \real{0.1240}}
  >{\raggedright\arraybackslash}p{(\linewidth - 14\tabcolsep) * \real{0.1240}}
  >{\raggedright\arraybackslash}p{(\linewidth - 14\tabcolsep) * \real{0.1240}}
  >{\raggedright\arraybackslash}p{(\linewidth - 14\tabcolsep) * \real{0.1240}}
  >{\raggedright\arraybackslash}p{(\linewidth - 14\tabcolsep) * \real{0.1240}}
  >{\raggedright\arraybackslash}p{(\linewidth - 14\tabcolsep) * \real{0.1240}}@{}}
\toprule\noalign{}
\begin{minipage}[b]{\linewidth}\raggedright
કટ-સેટ/શાખા
\end{minipage} & \begin{minipage}[b]{\linewidth}\raggedright
શાખા 1 (AB)
\end{minipage} & \begin{minipage}[b]{\linewidth}\raggedright
શાખા 3 (AC)
\end{minipage} & \begin{minipage}[b]{\linewidth}\raggedright
શાખા 4 (CD)
\end{minipage} & \begin{minipage}[b]{\linewidth}\raggedright
શાખા 2 (BD)
\end{minipage} & \begin{minipage}[b]{\linewidth}\raggedright
શાખા 6 (BC)
\end{minipage} & \begin{minipage}[b]{\linewidth}\raggedright
શાખા 7 (AD)
\end{minipage} & \begin{minipage}[b]{\linewidth}\raggedright
શાખા 5 (CD)
\end{minipage} \\
\midrule\noalign{}
\endhead
\bottomrule\noalign{}
\endlastfoot
કટ-સેટ 1 (AB) & 1 & 0 & 0 & -1 & -1 & 0 & 0 \\
કટ-સેટ 2 (AC) & 0 & 1 & 0 & 0 & 1 & -1 & 0 \\
કટ-સેટ 3 (CD) & 0 & 0 & 1 & 1 & 0 & 1 & 1 \\
\end{longtable}
}

\end{solutionbox}
\begin{mnemonicbox}
``CGVS: Cut-sets Generate Voltage Sources''

\end{mnemonicbox}
\subsection*{પ્રશ્ન 2(અ) [3
ગુણ]}\label{uxaaauxab0uxab6uxaa8-2uxa85-3-uxa97uxaa3}

\textbf{વ્યાખ્યા આપો: ૧) એક્ટિવ અને પેસિવ નેટ્વર્ક ૨) યુનિલેટરલ અને બાઇ-લેટરલ
નેટવર્ક.}

\begin{solutionbox}

\begin{itemize}
\item
  \textbf{એક્ટિવ નેટવર્ક}: એવું નેટવર્ક જેમાં એક કે વધારે EMF સ્રોત (વોલ્ટેજ/કરંટ સ્રોત)
  હોય જે સર્કિટને ઊર્જા પૂરી પાડે છે.
\item
  \textbf{પેસિવ નેટવર્ક}: એવું નેટવર્ક જેમાં માત્ર પેસિવ તત્વો જેવા કે રેઝિસ્ટર, કેપેસિટર
  અને ઇન્ડક્ટર હોય, કોઈ ઊર્જા સ્રોત ન હોય.
\item
  \textbf{યુનિલેટરલ નેટવર્ક}: એવું નેટવર્ક જેમાં ઇનપુટ અને આઉટપુટ ટર્મિનલ્સ બદલવાથી તેની
  પ્રોપર્ટી અને પરફોર્મન્સ બદલાય છે.
\item
  \textbf{બાઇલેટરલ નેટવર્ક}: એવું નેટવર્ક જેમાં ઇનપુટ અને આઉટપુટ ટર્મિનલ્સ બદલવાથી તેની
  પ્રોપર્ટી અને પરફોર્મન્સ સમાન રહે છે.
\end{itemize}

\textbf{આકૃતિ}:

\begin{center}
\textbf{Mermaid Diagram (Code)}
\begin{verbatim}
{Shaded}
{Highlighting}[]
graph LR
    subgraph "નેટવર્કના પ્રકાર"
    A[એક્ટિવ: સ્રોત ધરાવે છે]
    B[પેસિવ: સ્રોત નથી]
    C[યુનિલેટરલ: ડાયોડ/ટ્રાન્ઝિસ્ટર]
    D[બાઇલેટરલ: R, L, C તત્વો]
    end
{Highlighting}
{Shaded}
\end{verbatim}
\end{center}

\end{solutionbox}
\begin{mnemonicbox}
``APUB: Active Provides energy, Unilateral Blocks
reversal''

\end{mnemonicbox}
\subsection*{પ્રશ્ન 2(બ) [4
ગુણ]}\label{uxaaauxab0uxab6uxaa8-2uxaac-4-uxa97uxaa3}

\textbf{Z પેરામિટર માટે સમીકરણ લખો અને Z11, Z12, Z21, Z22 એ સમીકરણો પરથી
તારવો.}

\begin{solutionbox}

Z-પેરામિટર્સ બે-પોર્ટ નેટવર્કમાં પોર્ટ વોલ્ટેજ અને કરંટ વચ્ચેનો સંબંધ વ્યાખ્યાયિત કરે છે:

\textbf{સમીકરણો}:

\begin{itemize}
\tightlist
\item
  V_{1} = Z_{1}_{1}I_{1} + Z_{1}_{2}I_{2}
\item
  V_{2} = Z_{2}_{1}I_{1} + Z_{2}_{2}I_{2}
\end{itemize}

\textbf{તારણ}:

\begin{itemize}
\tightlist
\item
  \textbf{Z_{1}_{1} = V_{1}/I_{1}} (I_{2} = 0 સાથે): આઉટપુટ પોર્ટ ઓપન-સર્કિટ હોય ત્યારે ઇનપુટ
  ઇમ્પીડન્સ
\item
  \textbf{Z_{1}_{2} = V_{1}/I_{2}} (I_{1} = 0 સાથે): ઇનપુટ પોર્ટ ઓપન-સર્કિટ હોય ત્યારે રિવર્સ
  ટ્રાન્સફર ઇમ્પીડન્સ
\item
  \textbf{Z_{2}_{1} = V_{2}/I_{1}} (I_{2} = 0 સાથે): આઉટપુટ પોર્ટ ઓપન-સર્કિટ હોય ત્યારે
  ફોરવર્ડ ટ્રાન્સફર ઇમ્પીડન્સ
\item
  \textbf{Z_{2}_{2} = V_{2}/I_{2}} (I_{1} = 0 સાથે): ઇનપુટ પોર્ટ ઓપન-સર્કિટ હોય ત્યારે આઉટપુટ
  ઇમ્પીડન્સ
\end{itemize}

\end{solutionbox}
\begin{mnemonicbox}
``Z Impedance: Open circuit gives correct
Parameters''

\end{mnemonicbox}
\subsection*{પ્રશ્ન 2(ક) [7
ગુણ]}\label{uxaaauxab0uxab6uxaa8-2uxa95-7-uxa97uxaa3}

\textbf{સ્ટાન્ડર્ડ T નેટવર્ક માટે કેરક્ટરિસ્ટિક ઇમ્પિડન્સ (ZOT) નુ સમીકરણ તારવો.}

\begin{solutionbox}

સ્ટાન્ડર્ડ T-નેટવર્ક માટે:

\begin{center}
\textbf{Mermaid Diagram (Code)}
\begin{verbatim}
{Shaded}
{Highlighting}[]
graph LR
    A((Port{-1)) {-}{-}{-} B[Z1] {-}{-}{-} C((Junction))}
    C {-{-}{-} D[Z2] {-}{-}{-} E((Port{-}2))}
    C {-{-}{-} F[Z3] {-}{-}{-} G((Ground))}
{Highlighting}
{Shaded}
\end{verbatim}
\end{center}

\textbf{તારણના પગલાં}:

\begin{enumerate}
\tightlist
\item
  સિમેટ્રિક T-નેટવર્ક માટે, Z_{1} = Z_{2}
\item
  મેચ્ડ કન્ડિશન હેઠળ, ઇનપુટ ઇમ્પિડન્સ કેરેક્ટરિસ્ટિક ઇમ્પિડન્સ બરાબર હોય
\item
  Z_{0}_{t} = Z_{1} + (Z_{1}\timesZ_{3})/(Z_{1} + Z_{3})
\item
  બેલેન્સ્ડ T-નેટવર્ક જ્યાં Z_{1} = Z_{2} = Z/2 અને Z_{3} = Z માટે:
\item
  Z_{0}_{t} = Z/2 + (Z/2\timesZ)/(Z/2 + Z)
\item
  Z_{0}_{t} = Z/2 + (Z^{2}/2)/(Z + Z/2)
\item
  Z_{0}_{t} = Z/2 + (Z^{2}/2)/(3Z/2)
\item
  Z_{0}_{t} = Z/2 + Z^{2}/3Z
\item
  Z_{0}_{t} = Z/2 + Z/3
\item
  Z_{0}_{t} = (3Z + 2Z)/6
\item
  Z_{0}_{t} = \sqrt(Z_{1}(Z_{1} + 2Z_{3}))
\end{enumerate}

\textbf{અંતિમ સમીકરણ}: Z_{0}_{t} = \sqrt(Z_{1}(Z_{1} + 2Z_{3}))

\end{solutionbox}
\begin{mnemonicbox}
``TO Impedance: Two arms Over middle branch''

\end{mnemonicbox}
\subsection*{પ્રશ્ન 2(અ) OR [3
ગુણ]}\label{uxaaauxab0uxab6uxaa8-2uxa85-or-3-uxa97uxaa3}

\textbf{વ્યાખ્યા આપો. ૧) ડ્રાઇવીંગ પોઇંટ ઇમ્પીડન્સ ૨) ટ્રાન્સફર ઇમ્પીડન્સ}

\begin{solutionbox}

\begin{itemize}
\item
  \textbf{ડ્રાઇવિંગ પોઇંટ ઇમ્પીડન્સ}: જ્યારે અન્ય બધા સ્વતંત્ર સ્રોત શૂન્ય પર સેટ હોય
  ત્યારે સમાન પોર્ટ/ટર્મિનલના જોડા પર વોલ્ટેજ અને કરંટનો ગુણોત્તર.
\item
  \textbf{ટ્રાન્સફર ઇમ્પીડન્સ}: જ્યારે અન્ય બધા સ્વતંત્ર સ્રોત શૂન્ય પર સેટ હોય ત્યારે
  એક પોર્ટ પર વોલ્ટેજ અને બીજા પોર્ટ પર કરંટનો ગુણોત્તર.
\end{itemize}

\textbf{આકૃતિ}:

\begin{center}
\textbf{Mermaid Diagram (Code)}
\begin{verbatim}
{Shaded}
{Highlighting}[]
graph LR
    subgraph "ઇમ્પીડન્સના પ્રકાર"
    A[ડ્રાઇવિંગ પોઇંટ: V_{1/I_{1} અથવા V_{2}/I_{2}]}
    B[ટ્રાન્સફર: V_{2/I_{1} અથવા V_{1}/I_{2}]}
    end
{Highlighting}
{Shaded}
\end{verbatim}
\end{center}

\end{solutionbox}
\begin{mnemonicbox}
``DTSS: Driving at Terminal Same, Transfer at
Separate''

\end{mnemonicbox}
\subsection*{પ્રશ્ન 2(બ) OR [4
ગુણ]}\label{uxaaauxab0uxab6uxaa8-2uxaac-or-4-uxa97uxaa3}

\textbf{કિર્ચોફનો વોલ્ટેજ લો ઉદાહરણ સાથે સમજાવો.}

\begin{solutionbox}

\textbf{કિર્ચોફનો વોલ્ટેજ લો (KVL)}: સર્કિટમાં કોઈપણ બંધ લૂપની આસપાસના તમામ
વોલ્ટેજનો અલજેબ્રાઇક સરવાળો શૂન્ય હોય છે.

\textbf{ગણિતમાં}: \sumV = 0 (બંધ લૂપ આસપાસ)

\textbf{સર્કિટ ઉદાહરણ}:

\begin{center}
\textbf{Mermaid Diagram (Code)}
\begin{verbatim}
{Shaded}
{Highlighting}[]
graph LR
    A(({+)) {-}{-}"10V"{-}{-}{} B}
    B {-{-}"R_{1} = 2Ω"{-}{-}{} C}
    C {-{-}"R_{2} = 3Ω"{-}{-}{} D}
    D {-{-}"R_{3} = 5Ω"{-}{-}{} A}
    style A fill:\#f9f,stroke:\#333,stroke{-width:2px}
    style B fill:\#f9f,stroke:\#333,stroke{-width:2px}
    style C fill:\#f9f,stroke:\#333,stroke{-width:2px}
    style D fill:\#f9f,stroke:\#333,stroke{-width:2px}
{Highlighting}
{Shaded}
\end{verbatim}
\end{center}

જો I = 1A, તો:

\begin{itemize}
\tightlist
\item
  V_{1} = 1A \times 2Ω = 2V
\item
  V_{2} = 1A \times 3Ω = 3V
\item
  V_{3} = 1A \times 5Ω = 5V
\end{itemize}

KVL લાગુ કરતાં: 10V - 2V - 3V - 5V = 0 ✓

\end{solutionbox}
\begin{mnemonicbox}
``VACZ: Voltages Around Closed loop are Zero''

\end{mnemonicbox}
\subsection*{પ્રશ્ન 2(ક) OR [7
ગુણ]}\label{uxaaauxab0uxab6uxaa8-2uxa95-or-7-uxa97uxaa3}

\textbf{Π નેટવર્ક માથી T નેટવર્ક મા બદલવાના સમીકણ તારવો.}

\begin{solutionbox}

\textbf{π નેટવર્કને T નેટવર્કમાં રૂપાંતરણ}:

\begin{center}
\textbf{Mermaid Diagram (Code)}
\begin{verbatim}
{Shaded}
{Highlighting}[]
graph TD
    subgraph "π નેટવર્ક"
    A1((A)) {-{-}{-} B1((B))}
    A1 {-{-}{-} Y1[Ya] {-}{-}{-} C1}
    B1 {-{-}{-} Y2[Yb] {-}{-}{-} C1}
    A1 {-{-}{-} Y3[Yc] {-}{-}{-} B1}
    C1((C))
    end

    subgraph "T નેટવર્ક"
    A2((A)) {-{-}{-} Z1[Za] {-}{-}{-} D2((D))}
    B2((B)) {-{-}{-} Z2[Zb] {-}{-}{-} D2}
    D2 {-{-}{-} Z3[Zc] {-}{-}{-} C2((C))}
    end
{Highlighting}
{Shaded}
\end{verbatim}
\end{center}

\textbf{રૂપાંતરણ સમીકરણો}:

\begin{enumerate}
\tightlist
\item
  Za = (Ya \times Yc) / Y∆
\item
  Zb = (Yb \times Yc) / Y∆
\item
  Zc = (Ya \times Yb) / Y∆
\end{enumerate}

જ્યાં Y∆ = Ya + Yb + Yc

\textbf{તારણ}:

\begin{enumerate}
\tightlist
\item
  π-નેટવર્કના Y-પેરામિટર્સથી શરૂઆત કરો
\item
  શાખા એડમિટન્સના સંદર્ભમાં Y-પેરામિટર્સને વ્યક્ત કરો
\item
  મેટ્રિક્સ ઇન્વર્ઝનનો ઉપયોગ કરીને Z-પેરામિટર્સમાં રૂપાંતરિત કરો
\item
  Z-પેરામિટર્સના સંદર્ભમાં T-નેટવર્ક ઇમ્પિડન્સને વ્યક્ત કરો
\item
  સરળ બનાવીને ઉપરના રૂપાંતરણ સૂત્રો મેળવો
\end{enumerate}

\end{solutionbox}
\begin{mnemonicbox}
``PIE to TEA: Product over sum for opposite branch''

\end{mnemonicbox}
\subsection*{પ્રશ્ન 3(અ) [3
ગુણ]}\label{uxaaauxab0uxab6uxaa8-3uxa85-3-uxa97uxaa3}

\textbf{કિર્ચોફનો કરંટ લો ઉદાહરણ સાથે સમજાવો.}

\begin{solutionbox}

\textbf{કિર્ચોફનો કરંટ લો (KCL)}: કોઈપણ નોડમાં પ્રવેશતા અને છોડતા તમામ કરંટનો
અલજેબ્રાઇક સરવાળો શૂન્ય હોવો જોઈએ.

\textbf{ગણિતમાં}: \sumI = 0 (કોઈપણ નોડ પર)

\textbf{સર્કિટ ઉદાહરણ}:

\begin{center}
\textbf{Mermaid Diagram (Code)}
\begin{verbatim}
{Shaded}
{Highlighting}[]
graph TD
    A[I_{1 = 5A] {-}{-}{} B((Node))}
    C[I_{2 = 2A] {-}{-}{} B}
    B {-{-}{} D[I_{3} = 3A]}
    B {-{-}{} E[I_{4} = 4A]}
    style B fill:\#f9f,stroke:\#333,stroke{-width:2px}
{Highlighting}
{Shaded}
\end{verbatim}
\end{center}

નોડ B પર KCL લાગુ કરતાં:

\begin{itemize}
\tightlist
\item
  પ્રવેશતા કરંટ: I_{1} + I_{2} = 5A + 2A = 7A
\item
  છોડતા કરંટ: I_{3} + I_{4} = 3A + 4A = 7A
\item
  તેથી: I_{1} + I_{2} - I_{3} - I_{4} = 5 + 2 - 3 - 4 = 0 ✓
\end{itemize}

\end{solutionbox}
\begin{mnemonicbox}
``CuNoZ: Currents at Node are Zero''

\end{mnemonicbox}
\subsection*{પ્રશ્ન 3(બ) [4
ગુણ]}\label{uxaaauxab0uxab6uxaa8-3uxaac-4-uxa97uxaa3}

\textbf{જરુરી સમીકરણો સાથે મેશ એનાલિસિસ સમજાવો.}

\begin{solutionbox}

\textbf{મેશ એનાલિસિસ}: એક સર્કિટ એનાલિસિસ તકનીક જે મલ્ટિપલ લૂપ્સ વાળી સર્કિટને
ઉકેલવા માટે મેશ કરંટ્સનો ઉપયોગ કરે છે.

\textbf{પગલાં}:

\begin{enumerate}
\tightlist
\item
  સર્કિટમાં બધા મેશ (બંધ લૂપ) ઓળખો
\item
  દરેક મેશને મેશ કરંટ સોંપો
\item
  દરેક મેશ પર KVL લાગુ કરો
\item
  પરિણામી સમીકરણ સિસ્ટમને ઉકેલો
\end{enumerate}

\textbf{ઉદાહરણ સર્કિટ}:

\begin{center}
\textbf{Mermaid Diagram (Code)}
\begin{verbatim}
{Shaded}
{Highlighting}[]
graph LR
    A((A)) {-{-} R_{1} {-}{-}{-} B((B))}
    B {-{-} R_{3} {-}{-}{-} C((C))}
    A {-{-} R_{2} {-}{-}{-} C}
    A {-{-} V_{1} {-}{-}{-} D}
    D {-{-} {}+ {-}{-}{-} A}
    C {-{-} V_{2} {-}{-}{-} E}
    E {-{-} {}+ {-}{-}{-} C}
    style A fill:\#f9f,stroke:\#333,stroke{-width:2px}
    style B fill:\#f9f,stroke:\#333,stroke{-width:2px}
    style C fill:\#f9f,stroke:\#333,stroke{-width:2px}
{Highlighting}
{Shaded}
\end{verbatim}
\end{center}

\textbf{સમીકરણો}:

\begin{itemize}
\tightlist
\item
  મેશ 1: V_{1} = I_{1}R_{1} + I_{1}R_{2} - I_{2}R_{2}
\item
  મેશ 2: V_{2} = I_{2}R_{2} + I_{2}R_{3} - I_{1}R_{2}
\end{itemize}

\end{solutionbox}
\begin{mnemonicbox}
``MILK: Mesh Is Loop with KVL''

\end{mnemonicbox}
\subsection*{પ્રશ્ન 3(ક) [7
ગુણ]}\label{uxaaauxab0uxab6uxaa8-3uxa95-7-uxa97uxaa3}

\textbf{થીવીનીન નો થીયરમ લખો અને સમજાવો.}

\begin{solutionbox}

\textbf{થીવીનીનનો સિદ્ધાંત}: કોઈપણ લીનીયર નેટવર્ક જેમાં વોલ્ટેજ અને કરંટ સ્રોત હોય
તેને એક વોલ્ટેજ સ્રોત (VTH) અને એક રેઝિસ્ટન્સ (RTH) સીરીઝમાં ધરાવતા તુલ્ય સર્કિટ
દ્વારા બદલી શકાય છે.

\begin{center}
\textbf{Mermaid Diagram (Code)}
\begin{verbatim}
{Shaded}
{Highlighting}[]
graph TD
    subgraph "મૂળ નેટવર્ક"
    A((A)) {-{-}{-} B[જટિલ નેટવર્ક] {-}{-}{-} C((B))}
    end
    subgraph "થીવીનીન સમકક્ષ"
    D((A)) {-{-}{-} E[VTH] {-}{-}{-} F(({}+))}
    F {-{-}{-} G[RTH] {-}{-}{-} H((B))}
    end
{Highlighting}
{Shaded}
\end{verbatim}
\end{center}

\textbf{થીવીનીન સમકક્ષ શોધવાના પગલાં}:

\begin{enumerate}
\tightlist
\item
  જે ટર્મિનલ માટે સમકક્ષ શોધવાની છે તેમાંથી લોડ દૂર કરો
\item
  આ ટર્મિનલ્સ વચ્ચે ઓપન-સર્કિટ વોલ્ટેજ (VOC) ગણો (= VTH)
\item
  તમામ સ્રોતોને તેમના આંતરિક રેઝિસ્ટન્સ દ્વારા બદલીને સર્કિટમાં પાછા જોતા રેઝિસ્ટન્સ
  ગણો (= RTH)
\item
  થીવીનીન સમકક્ષ VTH અને RTH સીરીઝમાં ધરાવે છે
\end{enumerate}

\textbf{ઉદાહરણ ઍપ્લિકેશન}:

\begin{itemize}
\tightlist
\item
  લોડ RL સાથે મૂળ જટિલ સર્કિટ
\item
  RL દૂર કરો અને VOC = VTH શોધો
\item
  સ્રોતોને નિષ્ક્રિય કરો અને RTH શોધો
\item
  સરળીકૃત થીવીનીન સમકક્ષ સાથે RL ફરીથી જોડો
\end{itemize}

\end{solutionbox}
\begin{mnemonicbox}
``TORV: Thevenin's Open-circuit Resistance and
Voltage''

\end{mnemonicbox}
\subsection*{પ્રશ્ન 3(અ) OR [3
ગુણ]}\label{uxaaauxab0uxab6uxaa8-3uxa85-or-3-uxa97uxaa3}

\textbf{રેસિપ્રોસિટી થીયરમ લખો અને સમજાવો.}

\begin{solutionbox}

\textbf{રેસિપ્રોસિટી સિદ્ધાંત}: એક લીનિયર, બાઇલેટરલ નેટવર્કમાં, જો એક શાખામાં
વોલ્ટેજ સ્રોત બીજી શાખામાં કરંટ ઉત્પન્ન કરે છે, તો તે જ વોલ્ટેજ સ્રોત, જો બીજી શાખામાં
મૂકવામાં આવે, તો તે પ્રથમ શાખામાં સમાન કરંટ ઉત્પન્ન કરશે.

\begin{center}
\textbf{Mermaid Diagram (Code)}
\begin{verbatim}
{Shaded}
{Highlighting}[]
graph LR
    subgraph "મૂળ સર્કિટ"
    direction LR
    A((A)) {-{-}{-} B[V] {-}{-}{-} C((B))}
    C {-{-}{-} D[નેટવર્ક] {-}{-}{-} E((C))}
    E {-{-}{-} F[એમીટર] {-}{-}{-} A}
    end

    subgraph "રેસિપ્રોકલ સર્કિટ"
    direction LR
    G((A)) {-{-}{-} H[એમીટર] {-}{-}{-} I((B))}
    I {-{-}{-} J[નેટવર્ક] {-}{-}{-} K((C))}
    K {-{-}{-} L[V] {-}{-}{-} G}
    end
{Highlighting}
{Shaded}
\end{verbatim}
\end{center}

\textbf{ગણિતમાં}: જો શાખા 1માં વોલ્ટેજ V_{1} શાખા 2માં કરંટ I_{2} ઉત્પન્ન કરે છે, તો
શાખા 2માં વોલ્ટેજ V_{1} શાખા 1માં કરંટ I_{2} ઉત્પન્ન કરશે.

\textbf{મર્યાદાઓ}: ફક્ત નીચેના લક્ષણો ધરાવતા નેટવર્ક માટે લાગુ પડે છે:

\begin{itemize}
\tightlist
\item
  લીનિયર તત્વો
\item
  બાઇલેટરલ તત્વો (ડાયોડ, ટ્રાન્ઝિસ્ટર નહીં)
\item
  એક સ્વતંત્ર સ્રોત
\end{itemize}

\end{solutionbox}
\begin{mnemonicbox}
``RESWAP: REciprocity SWAPs Position with identical
results''

\end{mnemonicbox}
\subsection*{પ્રશ્ન 3(બ) OR [4
ગુણ]}\label{uxaaauxab0uxab6uxaa8-3uxaac-or-4-uxa97uxaa3}

\textbf{જરુરી સમિકરણો સાથે નોડલ એનાલિસિસ સમજાવો.}

\begin{solutionbox}

\textbf{નોડલ એનાલિસિસ}: એક સર્કિટ એનાલિસિસ તકનીક જે સર્કિટ ઉકેલવા માટે નોડ
વોલ્ટેજનો ઉપયોગ કરે છે.

\textbf{પગલાં}:

\begin{enumerate}
\tightlist
\item
  રેફરન્સ નોડ (ગ્રાઉન્ડ) પસંદ કરો
\item
  બાકીના નોડ્સને વોલ્ટેજ વેરિયેબલ સોંપો
\item
  દરેક નોન-રેફરન્સ નોડ પર KCL લાગુ કરો
\item
  પરિણામી સમીકરણ સિસ્ટમને ઉકેલો
\end{enumerate}

\textbf{ઉદાહરણ સર્કિટ}:

\begin{center}
\textbf{Mermaid Diagram (Code)}
\begin{verbatim}
{Shaded}
{Highlighting}[]
graph LR
    A((નોડ 1)) {-{-} G_{1} {-}{-}{-} B((ગ્રાઉન્ડ))}
    C((નોડ 2)) {-{-} G_{2} {-}{-}{-} B}
    A {-{-} G_{3} {-}{-}{-} C}
    A {-{-} I_{1} {-}{-}{} B}
    C {-{-} I_{2} {-}{-}{} B}
    style A fill:\#f9f,stroke:\#333,stroke{-width:2px}
    style C fill:\#f9f,stroke:\#333,stroke{-width:2px}
    style B fill:\#f9f,stroke:\#333,stroke{-width:2px}
{Highlighting}
{Shaded}
\end{verbatim}
\end{center}

\textbf{સમીકરણો}:

\begin{itemize}
\tightlist
\item
  નોડ 1: I_{1} = V_{1}G_{1} + (V_{1}-V_{2})G_{3}
\item
  નોડ 2: I_{2} = V_{2}G_{2} + (V_{2}-V_{1})G_{3}
\end{itemize}

\end{solutionbox}
\begin{mnemonicbox}
``NKCV: Nodal uses KCL with Voltage variables''

\end{mnemonicbox}
\subsection*{પ્રશ્ન 3(ક) OR [7
ગુણ]}\label{uxaaauxab0uxab6uxaa8-3uxa95-or-7-uxa97uxaa3}

\textbf{મેક્સિમમ પાવર ટ્રાન્સફર થીયરમ લખો અને સમજાવો.}

\begin{solutionbox}

\textbf{મહત્તમ પાવર ટ્રાન્સફર સિદ્ધાંત}: એક સ્રોત સાથે જોડાયેલ લોડ મહત્તમ પાવર
ત્યારે મેળવશે જ્યારે તેનો રેઝિસ્ટન્સ સ્રોતના આંતરિક રેઝિસ્ટન્સ બરાબર હોય.

\begin{center}
\textbf{Mermaid Diagram (Code)}
\begin{verbatim}
{Shaded}
{Highlighting}[]
graph LR
    A(({+)) {-}{-}{-} B[VS] {-}{-}{-} C((X))}
    C {-{-}{-} D[RS] {-}{-}{-} E((Y))}
    E {-{-}{-} F[RL] {-}{-}{-} G((Z))}
    G {-{-}{-} A}
    style C fill:\#f9f,stroke:\#333,stroke{-width:2px}
    style E fill:\#f9f,stroke:\#333,stroke{-width:2px}
{Highlighting}
{Shaded}
\end{verbatim}
\end{center}

\textbf{પ્રમાણ}:

\begin{enumerate}
\tightlist
\item
  સર્કિટમાં કરંટ: I = VS/(RS + RL)
\item
લોડમાં પહોંચતો પાવર:

P = I^{2}RL = (VS^{2}RL)/(RS + RL)^{2}

\item
  મહત્તમ પાવર માટે, dP/dRL = 0
\item
  ઉકેલતાં: (VS^{2}(RS + RL)^{2} - VS^{2}RL·2(RS + RL))/(RS + RL)^{4} = 0
\item
  સરળ કરતાં: (RS + RL)^{2} = 2RL(RS + RL)
\item
  વધુ સરળ કરતાં: RS + RL = 2RL
\item
  તેથી: RS = RL
\end{enumerate}

\textbf{મહત્તમ પાવર}: Pmax = VS^{2}/(4RS)

\end{solutionbox}
\begin{mnemonicbox}
``MaRLRS: Maximum power when load Resistance equals
Source Resistance''

\end{mnemonicbox}
\subsection*{પ્રશ્ન 4(અ) [3
ગુણ]}\label{uxaaauxab0uxab6uxaa8-4uxa85-3-uxa97uxaa3}

\textbf{શા માટે સિરિઝ રેઝોનંસ સર્કિટ વોલ્ટેજ એમ્પ્લિફાયર અને પેરેલલ રેઝોનંસ સર્કિટ કરંટ
એમ્પ્લિફાયર તરીકે વર્તે છે?}

\begin{solutionbox}

\textbf{સિરીઝ રેઝોનન્સ વોલ્ટેજ એમ્પ્લિફાયર તરીકે}:

\begin{itemize}
\tightlist
\item
  રેઝોનન્સ પર, સિરીઝ સર્કિટ ઇમ્પીડન્સ ન્યૂનતમ (માત્ર R) હોય છે
\item
  L અથવા C પરનો વોલ્ટેજ સ્રોત વોલ્ટેજ કરતાં ઘણો વધારે હોઈ શકે
\item
  વોલ્ટેજ મેગ્નિફિકેશન ફેક્ટર = Q = XL/R = 1/R\sqrt(L/C)
\item
  L અથવા C પરનો વોલ્ટેજ = Q \times સ્રોત વોલ્ટેજ
\end{itemize}

\textbf{પેરેલલ રેઝોનન્સ કરંટ એમ્પ્લિફાયર તરીકે}:

\begin{itemize}
\tightlist
\item
  રેઝોનન્સ પર, પેરેલલ સર્કિટ ઇમ્પીડન્સ મહત્તમ હોય છે
\item
  L અથવા C માંથી પસાર થતો કરંટ સ્રોત કરંટ કરતાં ઘણો વધારે હોઈ શકે
\item
  કરંટ મેગ્નિફિકેશન ફેક્ટર = Q = R/XL = R\sqrt(C/L)
\item
  L અથવા C માંથી પસાર થતો કરંટ = Q \times સ્રોત કરંટ
\end{itemize}

\textbf{કોષ્ટક}:

{\def\LTcaptype{none} % do not increment counter
\begin{longtable}[]{@{}lll@{}}
\toprule\noalign{}
સર્કિટ પ્રકાર & રેઝોનન્સ પર ઇમ્પીડન્સ & એમ્પ્લિફિકેશન \\
\midrule\noalign{}
\endhead
\bottomrule\noalign{}
\endlastfoot
સિરીઝ & ન્યૂનતમ (માત્ર R) & વોલ્ટેજ (VL અથવા VC = Q\timesVS) \\
પેરેલલ & મહત્તમ (R^{2}/r) & કરંટ (IL અથવા IC = Q\timesIS) \\
\end{longtable}
}

\end{solutionbox}
\begin{mnemonicbox}
``SeVoPa: Series Voltage, Parallel current
amplification''

\end{mnemonicbox}
\subsection*{પ્રશ્ન 4(બ) [4
ગુણ]}\label{uxaaauxab0uxab6uxaa8-4uxaac-4-uxa97uxaa3}

\textbf{કોઇલ ના Q નુ સમીકરણ તારવો.}

\begin{solutionbox}

\textbf{કોઇલનો Q-ફેક્ટર}:

\begin{center}
\textbf{Mermaid Diagram (Code)}
\begin{verbatim}
{Shaded}
{Highlighting}[]
graph LR
    A((A)) {-{-}{-} B[R] {-}{-}{-} C((B))}
    C {-{-}{-} D[L] {-}{-}{-} A}
    style C fill:\#f9f,stroke:\#333,stroke{-width:2px}
{Highlighting}
{Shaded}
\end{verbatim}
\end{center}

\textbf{તારણ}:

\begin{enumerate}
\tightlist
\item
  Q-ફેક્ટર વ્યાખ્યાયિત: Q = સ્ટોર થયેલી ઊર્જા / પ્રતિ સાયકલ વેડફાયેલી ઊર્જા
\item
  ઇન્ડક્ટરમાં સંગ્રહિત ઊર્જા = (1/2)LI^{2}
\item
  રેઝિસ્ટરમાં વેડફાયેલી પાવર = I^{2}R
\item
  પ્રતિ સાયકલ વેડફાયેલી ઊર્જા = પાવર \times સમય અવધિ = I^{2}R \times (1/f)
\item
  તેથી: Q = ((1/2)LI^{2}) / (I^{2}R \times (1/f))
\item
  સરળ કરતાં: Q = 2π \times (1/2)LI^{2} \times f / (I^{2}R)
\item
Q = 2πf \times L /

R = ωL / R

\end{enumerate}

\textbf{અંતિમ સમીકરણ}: Q = ωL / R = 2πfL / R = XL / R

\end{solutionbox}
\begin{mnemonicbox}
``QualityEDR: Quality equals Energy stored Divided
by energy lost per Radian''

\end{mnemonicbox}
\subsection*{પ્રશ્ન 4(ક) [7
ગુણ]}\label{uxaaauxab0uxab6uxaa8-4uxa95-7-uxa97uxaa3}

\textbf{સિરિઝ R-L-C સર્કિટ માટે સિરિઝ રેઝોનંસ ફ્રિક્વંસી નુ સમીકરણ તારવો.}

\begin{solutionbox}

\textbf{સિરીઝ R-L-C સર્કિટ}:

\begin{center}
\textbf{Mermaid Diagram (Code)}
\begin{verbatim}
{Shaded}
{Highlighting}[]
graph LR
    A((Input)) {-{-}{-} B[R] {-}{-}{-} C[L] {-}{-}{-} D[C] {-}{-}{-} E((Output))}
    style A fill:\#f9f,stroke:\#333,stroke{-width:2px}
    style E fill:\#f9f,stroke:\#333,stroke{-width:2px}
{Highlighting}
{Shaded}
\end{verbatim}
\end{center}

\textbf{તારણ}:

\begin{enumerate}
\tightlist
\item
  સિરીઝ RLC સર્કિટની ઇમ્પીડન્સ: Z = R + j(XL - XC)
\item
  જ્યાં: XL = ωL અને XC = 1/ωC
\item
  રેઝોનન્સ પર, XL = XC (ઇન્ડક્ટિવ અને કેપેસિટિવ રિએક્ટન્સ સમાન હોય છે)
\item
  તેથી: ωL = 1/ωC
\item
  ω માટે ઉકેલતાં: ω^{2} = 1/LC
\item
  રેઝોનન્ટ ફ્રિક્વન્સી: ω_{0} = 1/\sqrt(LC)
\item
  ફ્રિક્વન્સી f ના સંદર્ભમાં: f_{0} = 1/(2π\sqrt(LC))
\end{enumerate}

\textbf{રેઝોનન્સ પર લક્ષણો}:

\begin{itemize}
\tightlist
\item
  ઇમ્પીડન્સ ન્યૂનતમ (સંપૂર્ણ રેઝિસ્ટિવ: Z = R)
\item
  કરંટ મહત્તમ (I = V/R)
\item
  પાવર ફેક્ટર એકમ (સર્કિટ રેઝિસ્ટિવ લાગે છે)
\item
  L અને C પરના વોલ્ટેજ સમાન અને વિપરીત હોય છે
\end{itemize}

\end{solutionbox}
\begin{mnemonicbox}
``RES: Reactances Equal at Series resonance''

\end{mnemonicbox}
\subsection*{પ્રશ્ન 4(અ) OR [3
ગુણ]}\label{uxaaauxab0uxab6uxaa8-4uxa85-or-3-uxa97uxaa3}

\textbf{કપલ્ડ સર્કિટ શુ છે? સેલ્ફ ઇંડક્ટંસ અને મ્યુચ્યુઅલ ઇંદક્ટંસ ની વ્યાખ્યા આપો.}

\begin{solutionbox}

\textbf{કપલ્ડ સર્કિટ્સ}: બે અથવા વધુ સર્કિટ્સ જે મેગ્નેટિક રીતે જોડાયેલી હોય, જેથી
તેમની પરસ્પર મેગ્નેટિક ફીલ્ડ દ્વારા ઊર્જા એકમાંથી બીજામાં ટ્રાન્સફર થઈ શકે.

\begin{center}
\textbf{Mermaid Diagram (Code)}
\begin{verbatim}
{Shaded}
{Highlighting}[]
graph TD
    subgraph "પ્રાઇમરી"
    A((A)) {-{-}{-} B[L1] {-}{-}{-} C((B))}
    end

    subgraph "સેકન્ડરી"
    D((C)) {-{-}{-} E[L2] {-}{-}{-} F((D))}
    end
    
    G[M] {-.{-}{} B}
    G {-.{-}{} E}
{Highlighting}
{Shaded}
\end{verbatim}
\end{center}

\textbf{સેલ્ફ-ઇન્ડક્ટન્સ (L)}: એક સર્કિટનો ગુણધર્મ જેના દ્વારા કરંટમાં ફેરફારથી તે જ
સર્કિટમાં સેલ્ફ-ઇન્ડ્યુસ્ડ EMF ઉત્પન્ન થાય છે. L = Φ/I (મેગ્નેટિક ફ્લક્સનો તેને ઉત્પન્ન કરતા
કરંટ સાથેનો ગુણોત્તર)

\textbf{મ્યુચ્યુઅલ ઇન્ડક્ટન્સ (M)}: એક સર્કિટનો ગુણધર્મ જેના દ્વારા એક સર્કિટમાં કરંટમાં
ફેરફારથી બીજી સર્કિટમાં EMF ઇન્ડ્યુસ કરે છે. M = Φ_{2}_{1}/I_{1} (સર્કિટ 1 માં કરંટને કારણે
સર્કિટ 2 માં ફ્લક્સનો ગુણોત્તર)

\end{solutionbox}
\begin{mnemonicbox}
``SiMu: Self in Mine, Mutual in Yours''

\end{mnemonicbox}
\subsection*{પ્રશ્ન 4(બ) OR [4
ગુણ]}\label{uxaaauxab0uxab6uxaa8-4uxaac-or-4-uxa97uxaa3}

\textbf{કો-એફિસિઅંટ ઓફ કપલિંગ(K) નુ સમીકરણ તારવો.}

\begin{solutionbox}

\textbf{કપલિંગનો ગુણાંક (k)}:

\begin{center}
\textbf{Mermaid Diagram (Code)}
\begin{verbatim}
{Shaded}
{Highlighting}[]
graph LR
    subgraph "કપલ્ડ કોઇલ્સ"
    A((A)) {-{-}{-} B[L1] {-}{-}{-} C((B))}
    D((C)) {-{-}{-} E[L2] {-}{-}{-} F((D))}
    G[M] {-.{-}{} B}
    G {-.{-}{} E}
    end
{Highlighting}
{Shaded}
\end{verbatim}
\end{center}

\textbf{તારણ}:

\begin{enumerate}
\tightlist
\item
  બે કોઇલ્સ વચ્ચેનો મ્યુચ્યુઅલ ઇન્ડક્ટન્સ (M) આના પર આધારિત છે:

  \begin{itemize}
  \tightlist
  \item
    કોઇલ્સનો સેલ્ફ-ઇન્ડક્ટન્સ (L_{1} અને L_{2})
  \item
    ભૌતિક ગોઠવણ (નજીકતા અને દિશા)
  \end{itemize}
\item
  મહત્તમ શક્ય મ્યુચ્યુઅલ ઇન્ડક્ટન્સ: M_{m}_{a}_{x} = \sqrt(L_{1}L_{2})
\item
  કપલિંગનો ગુણાંક વ્યાખ્યાયિત: k = M/M_{m}_{a}_{x}
\item
  તેથી: k = M/\sqrt(L_{1}L_{2})
\end{enumerate}

\textbf{લક્ષણો}:

\begin{itemize}
\tightlist
\item
  k ની રેન્જ 0 (કોઈ કપલિંગ નહીં) થી 1 (પૂર્ણ કપલિંગ) સુધી
\item
  k ભૂમિતિ, દિશાનિર્દેશન અને માધ્યમ પર આધારિત છે
\item
  સામાન્ય ટ્રાન્સફોર્મર: k = 0.95 થી 0.99
\item
  એર-કોર કોઇલ્સ: k = 0.01 થી 0.5
\end{itemize}

\end{solutionbox}
\begin{mnemonicbox}
``KMutual: K Measures Mutual linkage proportion''

\end{mnemonicbox}
\subsection*{પ્રશ્ન 4(ક) OR [7
ગુણ]}\label{uxaaauxab0uxab6uxaa8-4uxa95-or-7-uxa97uxaa3}

\textbf{સિરિઝા RLC સર્કિટ મા R=30Ω, L=0.5H, અને C=5µF છે. (૧) સિરિઝ રેઝોનંસ
ફ્રિલક્વંસિ (૨) Q ફેક્ટર (૩)BW ની ગણતરી કરો.}

\begin{solutionbox}

\textbf{આપેલ}:

\begin{itemize}
\tightlist
\item
  રેઝિસ્ટન્સ, R = 30Ω
\item
  ઇન્ડક્ટન્સ, L = 0.5H
\item
કેપેસિટન્સ,

C = 5µF = 5\times10^{-}^{6}F

\end{itemize}

\textbf{ગણતરી}:

\textbf{(૧) સિરીઝ રેઝોનન્સ ફ્રિક્વન્સી}:

\begin{itemize}
\tightlist
\item
  f_{0} = 1/(2π\sqrt(LC))
\item
  f_{0} = 1/(2π\sqrt(0.5 \times 5\times10^{-}^{6}))
\item
  f_{0} = 1/(2π\sqrt(2.5\times10^{-}^{6}))
\item
  f_{0} = 1/(2π \times 1.58\times10^{-}^{3})
\item
  f_{0} = 1/(9.9\times10^{-}^{3})
\item
  f_{0} = 100.76 Hz
\item
  f_{0} \approx 100 Hz
\end{itemize}

\textbf{(૨) Q ફેક્ટર}:

\begin{itemize}
\tightlist
\item
  Q = (1/R)\sqrt(L/C)
\item
  Q = (1/30)\sqrt(0.5/(5\times10^{-}^{6}))
\item
  Q = (1/30)\sqrt(100,000)
\item
  Q = (1/30) \times 316.23
\item
  Q = 10.54
\end{itemize}

\textbf{(૩) બેન્ડવિડ્થ (BW)}:

\begin{itemize}
\tightlist
\item
  BW = f_{0}/Q
\item
  BW = 100.76/10.54
\item
  BW = 9.56 Hz
\end{itemize}

\textbf{કોષ્ટક}:

{\def\LTcaptype{none} % do not increment counter
\begin{longtable}[]{@{}lll@{}}
\toprule\noalign{}
પેરામીટર & સૂત્ર & મૂલ્ય \\
\midrule\noalign{}
\endhead
\bottomrule\noalign{}
\endlastfoot
રેઝોનન્ટ ફ્રિક્વન્સી (f_{0}) & 1/(2π\sqrt(LC)) & 100 Hz \\
ક્વોલિટી ફેક્ટર (Q) & (1/R)\sqrt(L/C) & 10.54 \\
બેન્ડવિડ્થ (BW) & f_{0}/Q & 9.56 Hz \\
\end{longtable}
}

\end{solutionbox}
\begin{mnemonicbox}
``RQB: Resonance Quality determines Bandwidth''

\end{mnemonicbox}
\subsection*{પ્રશ્ન 5(અ) [3
ગુણ]}\label{uxaaauxab0uxab6uxaa8-5uxa85-3-uxa97uxaa3}

\textbf{એટેન્યુટર નુ વર્ગીકરણ કરો.}

\begin{solutionbox}

\textbf{એટેન્યુએટર્સ}: રેઝિસ્ટર્સનું નેટવર્ક જે વિકૃતિ વિના સિગ્નલ લેવલને ઘટાડવા
(એટેન્યુએટ) માટે ડિઝાઇન કરવામાં આવે છે.

\textbf{એટેન્યુએટર્સના પ્રકાર}:

\begin{center}
\textbf{Mermaid Diagram (Code)}
\begin{verbatim}
{Shaded}
{Highlighting}[]
graph TD
    A[એટેન્યુએટર્સ] {-{-}{} B[ફિક્સ્ડ એટેન્યુએટર્સ]}
    A {-{-}{} C[વેરિએબલ એટેન્યુએટર્સ]}
    B {-{-}{} D[T{-}પ્રકાર]}
    B {-{-}{} E[π{-}પ્રકાર]}
    B {-{-}{} F[બ્રિજ્ડ{-}T]}
    B {-{-}{} G[લેટિસ]}
    C {-{-}{} H[સ્ટેપ એટેન્યુએટર્સ]}
    C {-{-}{} I[કન્ટિન્યુઅસલી વેરિએબલ]}
{Highlighting}
{Shaded}
\end{verbatim}
\end{center}

કોન્ફિગરેશન આધારિત:

\begin{itemize}
\tightlist
\item
  \textbf{T-પ્રકાર}: ત્રણ રેઝિસ્ટર T-આકારની કોન્ફિગરેશન
\item
  \textbf{π-પ્રકાર}: ત્રણ રેઝિસ્ટર π-આકારની કોન્ફિગરેશન
\item
  \textbf{બ્રિજ્ડ-T}: T-પ્રકાર સાથે એક રેઝિસ્ટર આરપાર જોડાય
\item
  \textbf{લેટિસ}: ચાર રેઝિસ્ટર્સ સાથે બેલેન્સ્ડ કોન્ફિગરેશન
\end{itemize}

સિમેટ્રી આધારિત:

\begin{itemize}
\tightlist
\item
  \textbf{સિમેટ્રિકલ}: સમાન ઈનપુટ અને આઉટપુટ ઇમ્પીડન્સ
\item
  \textbf{અસિમેટ્રિકલ}: અલગ ઈનપુટ અને આઉટપુટ ઇમ્પીડન્સ
\end{itemize}

\end{solutionbox}
\begin{mnemonicbox}
``ATP Fixed: Attenuator Types include Pad, Tee,
Lattice''

\end{mnemonicbox}
\subsection*{પ્રશ્ન 5(બ) [4
ગુણ]}\label{uxaaauxab0uxab6uxaa8-5uxaac-4-uxa97uxaa3}

\textbf{એટેન્યુએશન અને નેપર વચ્ચેનો સંબંધ તારવો.}

\begin{solutionbox}

\textbf{એટેન્યુએશન અને નેપર વચ્ચેનો સંબંધ}:

\begin{itemize}
\item
  \textbf{એટેન્યુએશન (α)}: ઇનપુટ વોલ્ટેજ (અથવા કરંટ)નો આઉટપુટ વોલ્ટેજ (અથવા કરંટ)
  સાથેનો ગુણોત્તર, વિવિધ એકમોમાં વ્યક્ત.
\item
  \textbf{નેપર (Np)}: ગુણોત્તરનો નેચરલ લોગેરિધમિક એકમ, મુખ્યત્વે ટ્રાન્સમિશન લાઇન
  થિયરીમાં વપરાય છે.
\end{itemize}

\textbf{તારણ}:

\begin{enumerate}
\tightlist
\item
  વોલ્ટેજ ગુણોત્તર V_{1}/V_{2} માટે:

  \begin{itemize}
  \tightlist
  \item
    નેપરમાં એટેન્યુએશન = ln(V_{1}/V_{2})
  \item
    ડેસિબલમાં એટેન્યુએશન = 20log_{1}_{0}(V_{1}/V_{2})
  \end{itemize}
\item
  પાવર ગુણોત્તર P_{1}/P_{2} માટે:

  \begin{itemize}
  \tightlist
  \item
    નેપરમાં એટેન્યુએશન = (1/2)ln(P_{1}/P_{2})
  \item
    ડેસિબલમાં એટેન્યુએશન = 10log_{1}_{0}(P_{1}/P_{2})
  \end{itemize}
\item
  dB અને નેપર વચ્ચેનો સંબંધ:

  \begin{itemize}
  \tightlist
  \item
    1 નેપર = 8.686 dB
  \item
    1 dB = 0.115 નેપર
  \end{itemize}
\end{enumerate}

\textbf{કોષ્ટક}:

{\def\LTcaptype{none} % do not increment counter
\begin{longtable}[]{@{}lll@{}}
\toprule\noalign{}
એકમ & વોલ્ટેજ ગુણોત્તર & પાવર ગુણોત્તર \\
\midrule\noalign{}
\endhead
\bottomrule\noalign{}
\endlastfoot
નેપર (Np) & ln(V_{1}/V_{2}) & (1/2)ln(P_{1}/P_{2}) \\
ડેસિબલ (dB) & 20log_{1}_{0}(V_{1}/V_{2}) & 10log_{1}_{0}(P_{1}/P_{2}) \\
\end{longtable}
}

\end{solutionbox}
\begin{mnemonicbox}
``NED: Neper Equals Decibel divided by 8.686''

\end{mnemonicbox}
\subsection*{પ્રશ્ન 5(ક) [7
ગુણ]}\label{uxaaauxab0uxab6uxaa8-5uxa95-7-uxa97uxaa3}

\textbf{સિમેટ્રિકલ T એટેન્યુએટર માટે R1 અને R2 ના સમીકરણો તારવો.}

\begin{solutionbox}

\textbf{સિમેટ્રિકલ T એટેન્યુએટર}:

\begin{center}
\textbf{Mermaid Diagram (Code)}
\begin{verbatim}
{Shaded}
{Highlighting}[]
graph LR
    A((Input)) {-{-}{-} B[R1] {-}{-}{-} C((Junction))}
    C {-{-}{-} D[R1] {-}{-}{-} E((Output))}
    C {-{-}{-} F[R2] {-}{-}{-} G((Ground))}
    style A fill:\#f9f,stroke:\#333,stroke{-width:2px}
    style E fill:\#f9f,stroke:\#333,stroke{-width:2px}
    style C fill:\#f9f,stroke:\#333,stroke{-width:2px}
{Highlighting}
{Shaded}
\end{verbatim}
\end{center}

\textbf{તારણ}:

\begin{enumerate}
\tightlist
\item
  કેરેક્ટરિસ્ટિક ઇમ્પીડન્સ Z_{0} સાથેના સિમેટ્રિકલ T-એટેન્યુએટર માટે:

  \begin{itemize}
  \tightlist
  \item
    ઇનપુટ અને આઉટપુટ ઇમ્પીડન્સ બંને Z_{0} બરાબર હોવા જોઈએ
  \item
એટેન્યુએશન રેશિયો

N = V_{1}/V_{2} = I_{2}/I_{1}

  \end{itemize}
\item
  સર્કિટ એનાલિસિસથી:

  \begin{itemize}
  \tightlist
  \item
    Z_{0} = R_{1} + (R_{2}(R_{1}))/(R_{2}+R_{1})
  \item
    N = (R_{1} + R_{2} + R_{1})/R_{2} = (2R_{1}+R_{2})/R_{2}
  \end{itemize}
\item
  R_{1} અને R_{2} માટે ઉકેલ:

  \begin{itemize}
  \tightlist
  \item
    R_{1} = Z_{0}(N-1)/(N+1)
  \item
    R_{2} = 2Z_{0}N/(N^{2}-1)
  \end{itemize}
\item
  dB (α) માં એટેન્યુએશન માટે:

  \begin{itemize}
  \tightlist
  \item
    N = 10\^{}(α/20)
  \item
    R_{1} = Z_{0}·tanh(α/2)
  \item
    R_{2} = Z_{0}/sinh(α)
  \end{itemize}
\end{enumerate}

\textbf{અંતિમ સમીકરણો}:

\begin{itemize}
\tightlist
\item
  R_{1} = Z_{0}(N-1)/(N+1)
\item
  R_{2} = 2Z_{0}N/(N^{2}-1)
\end{itemize}

\end{solutionbox}
\begin{mnemonicbox}
``TSR: T-attenuator Symmetry Requires equal R1
values''

\end{mnemonicbox}
\subsection*{પ્રશ્ન 5(અ) OR [3
ગુણ]}\label{uxaaauxab0uxab6uxaa8-5uxa85-or-3-uxa97uxaa3}

\textbf{સિમેટ્રિકલ બિ્રજ T અને સિમેટ્રિકલ લેટિસ એટેન્યુએટર ની સર્કિટ દોરો.}

\begin{solutionbox}

\textbf{સિમેટ્રિકલ બ્રિજ-T એટેન્યુએટર}:

\begin{verbatim}
                R1
   A o{-{-}{-}{-}{-}{-}{-}{-}///{-}{-}{-}{-}{-}{-}{-}{-}o B}
              |      |
              |      |
              {      /}
              /  R3  {}
       R2     {      /}
   o{-{-}{-}///{-}{-}+     +{-}{-}{-}{-}o}
   |                      |
   o{-{-}{-}{-}{-}{-}{-}{-}{-}{-}{-}{-}{-}{-}{-}{-}{-}{-}{-}{-}{-}{-}o}
   C                      D
\end{verbatim}

\textbf{સિમેટ્રિકલ લેટિસ એટેન્યુએટર}:

\begin{verbatim}
           R1
   A o{-{-}{-}{-}///{-}{-}{-}{-}o B}
      {            /}
       {          /}
        {        /}
     R2  {      / R2}
          {    /}
           {  /}
            {/}
            /{}
           /  {}
          /    {}
         /      {}
    R1  /        {}
     C o{-{-}///{-}{-}o D}
\end{verbatim}

\textbf{લક્ષણો}:

\begin{enumerate}
\tightlist
\item
  \textbf{બ્રિજ-T}: T અને π એટેન્યુએટર્સની વિશેષતાઓ સંયોજિત કરે છે, ઉચ્ચ-ફ્રિક્વન્સી
  એપ્લિકેશન માટે યોગ્ય
\item
  \textbf{લેટિસ}: ઉત્તમ ફેઝ અને ફ્રિક્વન્સી રિસ્પોન્સ સાથેની બેલેન્સ્ડ કોન્ફિગરેશન,
  સામાન્ય રીતે બેલેન્સ્ડ લાઇન્સમાં વપરાય છે
\end{enumerate}

\end{solutionbox}
\begin{mnemonicbox}
``BL-BA: Bridge Ladder, Balanced Attenuators''

\end{mnemonicbox}
\subsection*{પ્રશ્ન 5(બ) OR [4
ગુણ]}\label{uxaaauxab0uxab6uxaa8-5uxaac-or-4-uxa97uxaa3}

\textbf{ફ્રિક્વંસી ને આધારે ફિલ્ટર નુ વર્ગીકરણ કરો અને સાથે પાસ બેંડ અને સ્ટોપ બેંડ
દર્શાવતા ફ્રિક્વંસી રિસ્પોંસ દોરો.}

\begin{solutionbox}

\textbf{ફ્રિક્વન્સી આધારિત ફિલ્ટરનું વર્ગીકરણ}:

\begin{center}
\textbf{Mermaid Diagram (Code)}
\begin{verbatim}
{Shaded}
{Highlighting}[]
graph TD
    A[પેસિવ ફિલ્ટર્સ] {-{-}{} B[લો પાસ ફિલ્ટર]}
    A {-{-}{} C[હાઇ પાસ ફિલ્ટર]}
    A {-{-}{} D[બેન્ડ પાસ ફિલ્ટર]}
    A {-{-}{} E[બેન્ડ સ્ટોપ ફિલ્ટર]}
    A {-{-}{} F[ઓલ પાસ ફિલ્ટર]}
{Highlighting}
{Shaded}
\end{verbatim}
\end{center}

\textbf{ફ્રિક્વન્સી રિસ્પોન્સ}:

\begin{enumerate}
\item
  \textbf{લો પાસ ફિલ્ટર}: કટઓફ નીચેની ફ્રિક્વન્સી પસાર કરે, ઉપરની એટેન્યુએટ કરે

\begin{verbatim}
Gain |
   1 |****
     |    ****
     |        ****
   0 |------------****----
     |
     +----------------------
        0     fc        f \rightarrow
\end{verbatim}
\item
  \textbf{હાઇ પાસ ફિલ્ટર}: કટઓફ ઉપરની ફ્રિક્વન્સી પસાર કરે, નીચેની એટેન્યુએટ કરે

\begin{verbatim}
Gain |
   1 |            ****
     |        ****
     |    ****
   0 |****-----------------
     |
     +----------------------
        0     fc        f \rightarrow
\end{verbatim}
\item
  \textbf{બેન્ડ પાસ ફિલ્ટર}: ચોક્કસ બેન્ડની અંદરની ફ્રિક્વન્સી પસાર કરે

\begin{verbatim}
Gain |
   1 |        ****
     |    ****    ****
     |   *          *
   0 |***-------------***--
     |
     +----------------------
        0   f1   f2     f \rightarrow
\end{verbatim}
\item
  \textbf{બેન્ડ સ્ટોપ ફિલ્ટર}: ચોક્કસ બેન્ડની અંદરની ફ્રિક્વન્સી રિજેક્ટ કરે

\begin{verbatim}
Gain |
   1 |***             ***
     |   *           *
     |    ***     ***
   0 |        *****
     |
     +----------------------
        0   f1   f2     f \rightarrow
\end{verbatim}
\end{enumerate}

\end{solutionbox}
\begin{mnemonicbox}
``LHBBA: Low High Band-pass Band-stop All-pass''

\end{mnemonicbox}
\subsection*{પ્રશ્ન 5(ક) OR [7
ગુણ]}\label{uxaaauxab0uxab6uxaa8-5uxa95-or-7-uxa97uxaa3}

\textbf{Constant-k લો પાસ ફિલ્ટર ના T સેક્શન અને Π સેક્શન દોરો અને કટ ઓફ
ફ્રિક્વંસીનુ સમીકરણ તારવો.}

\begin{solutionbox}

\textbf{T-સેક્શન Constant-K લો પાસ ફિલ્ટર}:

\begin{verbatim}
           L/2            L/2
   o{-{-}{-}{-}{-}{-}UUUUUU{-}{-}{-}{-}{-}{-}UUUUUU{-}{-}{-}{-}{-}{-}{-}o}
   |                               |
   |                               |
   |                               |
   |              C                |
   |              |                |
   o{-{-}{-}{-}{-}{-}{-}{-}{-}{-}{-}{-}{-}{-}+{-}{-}{-}{-}{-}{-}{-}{-}{-}{-}{-}{-}{-}{-}{-}{-}o}
   Input                        Output
\end{verbatim}

\textbf{π-સેક્શન Constant-K લો પાસ ફિલ્ટર}:

\begin{verbatim}
          L
   o{-{-}{-}{-}{-}{-}UUUUUU{-}{-}{-}{-}{-}{-}{-}{-}{-}o}
   |                     |
   |                     |
   |                     |
   |                     |
  {-{-}{-}                   {-}{-}{-}}
  {-{-}{-} C/2               {-}{-}{-} C/2}
   |                     |
   |                     |
   o{-{-}{-}{-}{-}{-}{-}{-}{-}{-}{-}{-}{-}{-}{-}{-}{-}{-}{-}{-}{-}o}
   Input              Output
\end{verbatim}

\textbf{કટઓફ ફ્રિક્વન્સીનું તારણ}:

\begin{enumerate}
\tightlist
\item
  Constant-K ફિલ્ટર માટે:

  \begin{itemize}
  \tightlist
  \item
    Z_{1} \times Z_{2} = R_{0}^{2} (કેરેક્ટરિસ્ટિક ઇમ્પીડન્સ વર્ગ)
  \item
    Z_{1} = jωL (સિરીઝ ઇમ્પીડન્સ)
  \item
    Z_{2} = 1/jωC (શન્ટ ઇમ્પીડન્સ)
  \end{itemize}
\item
  તેથી:

  \begin{itemize}
  \tightlist
  \item
    R_{0}^{2} = Z_{1} \times Z_{2} = jωL \times 1/jωC = L/C
  \item
    R_{0} = \sqrt(L/C)
  \end{itemize}
\item
  પાસ બેન્ડ કન્ડિશન:

  \begin{itemize}
  \tightlist
  \item
    -1 \textless{} Z_{1}/4Z_{2} \textless{} 0
  \item
    -1 \textless{} jωL/(4 \times 1/jωC) \textless{} 0
  \item
    -1 \textless{} -ω^{2}LC/4 \textless{} 0
  \end{itemize}
\item
  કટઓફ ફ્રિક્વન્સી પર:

  \begin{itemize}
  \tightlist
  \item
    ω^{2}LC/4 = 1
  \item
    ωc^{2} = 4/LC
  \item
    ωc = 2/\sqrt(LC)
  \item
    fc = ωc/2π = 1/π\sqrt(LC)
  \end{itemize}
\end{enumerate}

\textbf{અંતિમ સમીકરણ}:

\begin{itemize}
\tightlist
\item
  કટઓફ ફ્રિક્વન્સી fc = 1/π\sqrt(LC)
\end{itemize}

\end{solutionbox}
\begin{mnemonicbox}
``KCLP: Konstant-k Cutoff in Low Pass depends on L
and C product''

\end{mnemonicbox}

\end{document}
