\documentclass{article}

% content/resources/templates/preamble.tex
\usepackage[margin=0.6in]{geometry}
\author{Milav Dabgar}
\usepackage{amsmath,amssymb,amsthm}
\usepackage{booktabs}
\usepackage{multirow}
\usepackage{xcolor}
\usepackage{tcolorbox}
\tcbuselibrary{breakable,skins}
\usepackage[colorlinks=true,linkcolor=blue]{hyperref}
\usepackage{titlesec}
\usepackage{enumitem}
\usepackage{tikz}
\usepackage{pgfplots}
\usepackage{circuitikz}
\usepackage[version=4]{mhchem}
\usepackage{longtable}
\usepackage{array}
\usepackage{float}
\usepackage{caption}
\usepackage{listings}

\lstset{
  basicstyle=\small\ttfamily,
  breaklines=true,
  breakatwhitespace=false,
  postbreak=\mbox{\textcolor{red}{$\hookrightarrow$}\space},
  float=false,
  numbers=left,
  numberstyle=\tiny\color{gray},
  numbersep=10pt,
  xleftmargin=2em,
  keywordstyle=\color{blue},
  commentstyle=\color{green!60!black},
  stringstyle=\color{purple},
  backgroundcolor=\color{gray!5},
  showstringspaces=false,
  tabsize=2,
  captionpos=b,
  keepspaces=true,
  columns=flexible
}

\pgfplotsset{compat=1.18}
\usetikzlibrary{shapes,arrows,positioning,calc,patterns,decorations.pathmorphing,decorations.markings,arrows.meta}

% Color scheme
\definecolor{headcolor}{RGB}{0,102,204}
\definecolor{keycolor}{RGB}{220,20,60}
\definecolor{solutioncolor}{RGB}{34,139,34}
\definecolor{mnemoniccolor}{RGB}{148,0,211}
\definecolor{codecolor}{RGB}{0,0,100}

% Spacing
\setlength{\parskip}{3pt}
\setlist[itemize]{nosep}
\setlist[enumerate]{nosep}

% Title formatting
\titleformat{\section}{\Large\bfseries\color{headcolor}}{\thesection}{1em}{}
\titleformat{\subsection}{\large\bfseries\color{headcolor}}{\thesubsection}{1em}{}

% Pandoc tightlist compatibility
\providecommand{\tightlist}{%
  \setlength{\itemsep}{0pt}\setlength{\parskip}{0pt}}

% Pandoc longtable compatibility
\newcounter{none}
\def\thenone{}


% content/resources/templates/gujarati-boxes.tex
\usepackage{fontspec}
\usepackage{polyglossia}

% Set Gujarati as main language (document is primarily in Gujarati)
% Note: gloss-gujarati.ldf doesn't exist in polyglossia, but it will use hyphenation patterns
\setdefaultlanguage{gujarati}
\setotherlanguage{english}

% Configure Gujarati font properly
% Use Language=Default to prevent polyglossia from trying to add language-specific features
% that don't exist for Gujarati, which causes "empty feature" warnings
\newfontfamily\gujaratifont[Script=Gujarati,AutoFakeBold=2.5,AutoFakeSlant=0.3]{Noto Sans Gujarati}
\setmainfont[Script=Gujarati,AutoFakeBold=2.5,AutoFakeSlant=0.3]{Noto Sans Gujarati}
% Use Noto Sans Gujarati for monospace to support Gujarati in text
\setmonofont[Scale=0.9]{Noto Sans Gujarati}

% Configure English to use the same font
\newfontfamily\englishfont[Script=Gujarati,AutoFakeBold=2.5,AutoFakeSlant=0.3]{Noto Sans Gujarati}

% Translations for polyglossia
\gappto\captionsgujarati{
  \renewcommand{\tablename}{કોષ્ટક}
  \renewcommand{\figurename}{આકૃતિ}
}

% Helper for TikZ nodes to ensure Gujarati font
\newcommand{\gu}[1]{{\gujaratifont #1}}

% Custom environments
\newtcolorbox{solutionbox}{
    breakable,
    enhanced,
    colback=solutioncolor!5!white,
    colframe=solutioncolor!75!black,
    fonttitle=\bfseries,
    title=જવાબ
}

\newtcolorbox{solutionboxnobreak}{
 colback=solutioncolor!5!white,
 colframe=solutioncolor!75!black,
 fonttitle=\bfseries,
 title=જવાબ
}

\newtcolorbox{keyformula}{
 breakable,
 enhanced,
 colback=keycolor!5!white,
 colframe=keycolor!75!black,
 fonttitle=\bfseries,
 title=રાસાયણિક સમીકરણ/સૂત્ર
}

\newtcolorbox{mnemonicbox}{
 breakable,
 enhanced,
 colback=mnemoniccolor!5!white,
 colframe=mnemoniccolor!75!black,
 fonttitle=\bfseries,
 title=મેમરી ટ્રીક
}


% Custom commands for GTU solutions
% This file defines semantic commands for consistent formatting

% Question command with automatic formatting
\newcommand{\question}[2]{%
  \section*{Question #1}%
  \textbf{#2}%
}

% OR question variant
\newcommand{\questionor}[2]{%
  \section*{Question #1 OR}%
  \textbf{#2}%
}

% Proper table environment with caption
\newenvironment{answertable}[1]{%
  \begin{table}[htbp]
  \centering
  \caption{#1}
}{%
  \end{table}
}

% Proper figure environment for diagrams
\newenvironment{answerdiagram}[1]{%
  \begin{figure}[htbp]
  \centering
  \caption{#1}
}{%
  \end{figure}
}

% Semantic markup for key terms
\newcommand{\keyword}[1]{\textbf{#1}}
\newcommand{\code}[1]{\texttt{#1}}
\newcommand{\classname}[1]{\texttt{#1}}
\newcommand{\methodname}[1]{\texttt{#1}}

% Proper quotation marks
\newcommand{\mnemonic}[1]{``#1''}


\title{Electronic Circuits \& Networks (4331101) - Summer 2023 Solution}
\date{July 18, 2023}

\begin{document}
\maketitle

\questionmarks{1(a)}{3}{ઇલેક્ટ્રોનિક નેટવર્ક માટે વ્યાખ્યા આપો. (i) નોડ (ii) બ્રાંચ (iii) લૂપ}

\begin{solutionbox}
\begin{tabulary}{\linewidth}{|L|L|}
\hline
\textbf{શબ્દ} & \textbf{વ્યાખ્યા} \\ \hline
\textbf{નોડ} & એક બિંદુ જ્યાં બે કે વધુ તત્વો એકબીજા સાથે જોડાયેલા હોય \\ \hline
\textbf{બ્રાંચ} & બે નોડ વચ્ચેનો એક તત્વ અથવા પાથ \\ \hline
\textbf{લૂપ} & નેટવર્કમાં બંધ પાથ જ્યાં કોઈ નોડને એક કરતાં વધુ વખત ક્રોસ ન કરાય \\ \hline
\end{tabulary}

\begin{center}
\begin{circuitikz}[auto, node distance=2cm]
    \node [gtu state] (A) {Node A};
    \node [gtu state, right=of A] (B) {Node B};
    \node [gtu state, right=of B] (C) {Node C};
    \node [gtu state, below=of B] (D) {Node D};
    
    \path [draw] (A) -- (B);
    \path [draw] (B) -- (C);
    \path [draw] (C) -- (D);
    \path [draw] (D) -- (A);
    \path [draw] (A) to [bend left] (C);
\end{circuitikz}
\captionof{figure}{Network Definitions}
\end{center}
\end{solutionbox}

\begin{mnemonicbox}
\mnemonic{NBL: નેટવર્ક્સ બિગિન વિથ લૂપ્સ}
\end{mnemonicbox}

\questionmarks{1(b)}{4}{20 $\Omega$, 30 $\Omega$ અને 50 $\Omega$ નાં રેઝીસ્ટર 60 V નાં સપ્લાય સાથે પેરેલલમાં જોડાયેલા છે. તો (i) દરેક રેઝીસ્ટરમાંથી પસાર થતો કરંટ તથા કુલ કરંટ (ii) ઇક્વીવેલન્ટ રેઝીસ્ટર શોધો.}

\begin{solutionbox}
\begin{center}
\begin{circuitikz}[auto, node distance=2.5cm]
    \node [coordinate] (pos) {};
    \node [coordinate, below=3cm of pos] (neg) {};
    
    \draw (pos) to [V, l=60V] (neg);
    \draw (pos) -- ++(2,0) coordinate (top);
    \draw (neg) -- ++(2,0) coordinate (bot);
    
    \draw (top) to [R, l=20$\Omega$, i=$I_1$] (bot);
    \draw (top) -- ++(2,0) coordinate (top2);
    \draw (bot) -- ++(2,0) coordinate (bot2);
    \draw (top2) to [R, l=30$\Omega$, i=$I_2$] (bot2);
    
    \draw (top2) -- ++(2,0) coordinate (top3);
    \draw (bot2) -- ++(2,0) coordinate (bot3);
    \draw (top3) to [R, l=50$\Omega$, i=$I_3$] (bot3);
\end{circuitikz}
\captionof{figure}{Parallel Circuit}
\end{center}

\begin{tabulary}{\linewidth}{|L|L|}
\hline
\textbf{ગણતરી} & \textbf{મૂલ્ય} \\ \hline
\textbf{20 $\Omega$ રેઝીસ્ટરમાંથી પસાર થતો કરંટ}: $I_1 = V/R_1 = 60/20$ & 3 A \\ \hline
\textbf{30 $\Omega$ રેઝીસ્ટરમાંથી પસાર થતો કરંટ}: $I_2 = V/R_2 = 60/30$ & 2 A \\ \hline
\textbf{50 $\Omega$ રેઝીસ્ટરમાંથી પસાર થતો કરંટ}: $I_3 = V/R_3 = 60/50$ & 1.2 A \\ \hline
\textbf{કુલ કરંટ}: $I = I_1 + I_2 + I_3 = 3 + 2 + 1.2$ & 6.2 A \\ \hline
\textbf{ઇક્વીવેલન્ટ રેઝીસ્ટન્સ}: $Req = V/I = 60/6.2$ & 9.68 $\Omega$ \\ \hline
\end{tabulary}
\end{solutionbox}

\begin{mnemonicbox}
\mnemonic{PIV: પેરેલલ ઇન્ક્રીઝીસ ધ કરંટ, વોલ્ટેજ રીમેઇન્સ ધ સેમ}
\end{mnemonicbox}

\questionmarks{1(c)}{7}{કેપેસીટર માટે સિરિઝ અને પેરેલલ જોડાણ સમજાવો.}

\begin{solutionbox}
\begin{tabulary}{\linewidth}{|L|L|L|}
\hline
\textbf{જોડાણ} & \textbf{સૂત્ર} & \textbf{લક્ષણો} \\ \hline
\textbf{સિરિઝ જોડાણ} & $1/C_{eq} = 1/C_1 + 1/C_2 + 1/C_3 + \dots$ & - ઇક્વીવેલન્ટ કેપેસિટન્સ સૌથી નાના કેપેસિટરથી ઓછું \newline - દરેક કેપેસિટરમાં સમાન કરંટ \newline - કુલ વોલ્ટેજ કેપેસિટરો વચ્ચે વહેંચાય છે \newline - ડાયલેક્ટ્રીક સ્ટ્રેન્થ વધારે છે \\ \hline
\textbf{પેરેલલ જોડાણ} & $C_{eq} = C_1 + C_2 + C_3 + \dots$ & - ઇક્વીવેલન્ટ કેપેસિટન્સ બધા કેપેસિટરોનો સરવાળો \newline - દરેક કેપેસિટર પર સમાન વોલ્ટેજ \newline - કુલ ચાર્જ વ્યક્તિગત ચાર્જનો સરવાળો \newline - પ્લેટનું ક્ષેત્રફળ વધારે છે \\ \hline
\end{tabulary}

\begin{center}
\begin{circuitikz}[auto]
    \node [coordinate] (S1) at (0,0) {};
    \draw (S1) to [C, l=$C_1$] ++(2,0) to [C, l=$C_2$] ++(2,0) to [C, l=$C_3$] ++(2,0);
    \node at (3,-1) {(a) Series Connection};

    \node [coordinate] (P1) at (0,-3) {};
    \draw (P1) -- ++(1,0) -- ++(0,1) to [C, l=$C_1$] ++(2,0) -- ++(0,-1) -- ++(1,0); 
    \draw (P1) ++(1,0) to [C, l=$C_2$] ++(2,0);
    \draw (P1) ++(1,0) -- ++(0,-1) to [C, l=$C_3$] ++(2,0) -- ++(0,1);
    \node at (2,-4.5) {(b) Parallel Connection};
\end{circuitikz}
\captionof{figure}{Capacitor Connections}
\end{center}
\end{solutionbox}

\begin{mnemonicbox}
\mnemonic{CAPE: કેપેસિટર્સ એડ ઇન પેરેલલ, એલિમિનેટ ઇન સિરિઝ}
\end{mnemonicbox}

\questionmarks{1(c OR)}{7}{ઇન્ડક્ટર માટે સિરિઝ અને પેરેલલ જોડાણ સમજાવો.}

\begin{solutionbox}
\begin{tabulary}{\linewidth}{|L|L|L|}
\hline
\textbf{જોડાણ} & \textbf{સૂત્ર} & \textbf{લક્ષણો} \\ \hline
\textbf{સિરિઝ જોડાણ} & $L_{eq} = L_1 + L_2 + L_3 + \dots$ & - ઇક્વીવેલન્ટ ઇન્ડક્ટન્સ બધા ઇન્ડક્ટરોનો સરવાળો \newline - દરેક ઇન્ડક્ટરમાં સમાન કરંટ \newline - કુલ વોલ્ટેજ વ્યક્તિગત વોલ્ટેજનો સરવાળો \newline - ફ્લક્સ લિંકેજ વધે છે \\ \hline
\textbf{પેરેલલ જોડાણ} & $1/L_{eq} = 1/L_1 + 1/L_2 + 1/L_3 + \dots$ & - ઇક્વીવેલન્ટ ઇન્ડક્ટન્સ સૌથી નાના ઇન્ડક્ટરથી ઓછું \newline - દરેક ઇન્ડક્ટર પર સમાન વોલ્ટેજ \newline - કુલ કરંટ ઇન્ડક્ટરો વચ્ચે વહેંચાય છે \newline - મેગ્નેટિક કપલિંગ વાસ્તવિક મૂલ્યને અસર કરે છે \\ \hline
\end{tabulary}

\begin{center}
\begin{circuitikz}[auto]
    \node [coordinate] (S1) at (0,0) {};
    \draw (S1) to [L, l=$L_1$] ++(2,0) to [L, l=$L_2$] ++(2,0) to [L, l=$L_3$] ++(2,0);
    \node at (3,-1) {(a) Series Connection};

    \node [coordinate] (P1) at (0,-3) {};
    \draw (P1) -- ++(1,0) -- ++(0,1) to [L, l=$L_1$] ++(2,0) -- ++(0,-1) -- ++(1,0); 
    \draw (P1) ++(1,0) to [L, l=$L_2$] ++(2,0);
    \draw (P1) ++(1,0) -- ++(0,-1) to [L, l=$L_3$] ++(2,0) -- ++(0,1);
    \node at (2,-4.5) {(b) Parallel Connection};
\end{circuitikz}
\captionof{figure}{Inductor Connections}
\end{center}
\end{solutionbox}

\begin{mnemonicbox}
\mnemonic{LIPS: ઇન્ડક્ટર્સ લિંક ઇન સિરિઝ, પાર્ટિશન ઇન પેરેલલ}
\end{mnemonicbox}


% ==================================================================
% QUESTION 2
% ==================================================================

\questionmarks{2(a)}{3}{વ્યાખ્યા આપો. (i) ટ્રાન્સફોર્મઇમ્પીડન્સ, (ii) ડ્રાઇવિંગ પોઇન્ટ ઇમ્પીડન્સ, (iii) ટ્રાન્સફર ઇમ્પીડન્સ.}

\begin{solutionbox}
\begin{tabulary}{\linewidth}{|L|L|}
\hline
\textbf{શબ્દ} & \textbf{વ્યાખ્યા} \\ \hline
\textbf{ટ્રાન્સફોર્મઇમ્પીડન્સ} & ટ્રાન્સફોર્મરમાં પ્રાથમિકથી ગૌણ તરફ જતા સિગ્નલ દ્વારા જોવામાં આવતા ઇમ્પીડન્સ \\ \hline
\textbf{ડ્રાઇવિંગ પોઇન્ટ ઇમ્પીડન્સ} & એક જ પોર્ટ પર વોલ્ટેજનો કરંટ સાથેનો ગુણોત્તર \\ \hline
\textbf{ટ્રાન્સફર ઇમ્પીડન્સ} & એક પોર્ટ પર વોલ્ટેજનો બીજા પોર્ટના કરંટ સાથેનો ગુણોત્તર \\ \hline
\end{tabulary}

\begin{center}
\begin{circuitikz}[auto, node distance=2cm]
    \node [gtu block, minimum width=3cm, minimum height=2cm] (net) {Two Port Network};
    \node [left=of net] (in) {Input};
    \node [right=of net] (out) {Output};
    
    \draw [gtu arrow] (in) -- node[above] {$Z_{in}$ (Driving Point)} (net);
    \draw [gtu arrow] (in) to [bend left] node[above] {$Z_{21}$ (Transfer)} (out);
    \draw [gtu arrow] (out) to [bend left] node[below] {$Z_{12}$ (Transfer)} (in);
\end{circuitikz}
\captionof{figure}{Impedance Concepts}
\end{center}
\end{solutionbox}

\begin{mnemonicbox}
\mnemonic{TDT: ટ્રાન્સફોર્મર્સ ડ્રાઇવ ટ્રાન્સફર્સ}
\end{mnemonicbox}

\questionmarks{2(b)}{4}{30, 50 અને 90 ohms ના રેઝીસ્ટર સ્ટારમાં કનેક્ટ કરેલા છે. ડેલ્ટા કનેક્શનનાં ઇક્વીવેલન્ટ રેઝીસ્ટર શોધો.}

\begin{solutionbox}
\begin{center}
\begin{circuitikz}[auto]
    % Star
    \node at (0,3) (A) {A};
    \node at (-1.5,0) (B) {B};
    \node at (1.5,0) (C) {C};
    \node at (0,1) (N) {};
    
    \draw (A) to [R, l=$R_1(30\Omega)$] (N.center);
    \draw (B) to [R, l=$R_2(50\Omega)$] (N.center);
    \draw (C) to [R, l=$R_3(90\Omega)$] (N.center);
    \node at (0,-1) {(a) Star Connection};

    % Arrow
    \draw [gtu arrow, line width=1mm] (2.5, 1.5) -- (4.5, 1.5);

    % Delta
    \begin{scope}[shift={(6,0.5)}]
        \node at (0,2.5) (Ad) {A};
        \node at (-1.5,0) (Bd) {B};
        \node at (1.5,0) (Cd) {C};
        
        \draw (Ad) to [R, l=$R_{12}$] (Bd);
        \draw (Bd) to [R, l=$R_{23}$] (Cd);
        \draw (Cd) to [R, l=$R_{31}$] (Ad);
        \node at (0,-1.5) {(b) Delta Connection};
    \end{scope}
\end{circuitikz}
\captionof{figure}{Star to Delta Transformation}
\end{center}

\begin{tabulary}{\linewidth}{|L|L|L|}
\hline
\textbf{સ્ટાર થી ડેલ્ટા કન્વર્ઝન ફોર્મ્યુલા} & \textbf{ગણતરી} & \textbf{પરિણામ} \\ \hline
$R_{12} = \frac{R_1 R_2 + R_2 R_3 + R_3 R_1}{R_3}$ & $(30 {\times} 50 + 50 {\times} 90 + 90 {\times} 30)/90$ & 105 $\Omega$ \\ \hline
$R_{23} = \frac{R_1 R_2 + R_2 R_3 + R_3 R_1}{R_1}$ & $(30 {\times} 50 + 50 {\times} 90 + 90 {\times} 30)/30$ & 315 $\Omega$ \\ \hline
$R_{31} = \frac{R_1 R_2 + R_2 R_3 + R_3 R_1}{R_2}$ & $(30 {\times} 50 + 50 {\times} 90 + 90 {\times} 30)/50$ & 189 $\Omega$ \\ \hline
\end{tabulary}
\end{solutionbox}

\begin{mnemonicbox}
\mnemonic{PSR: પ્રોડક્ટ ઓવર સમ ઓફ રેસિસ્ટર્સ}
\end{mnemonicbox}

\questionmarks{2(c)}{7}{Π નેટવર્ક સમજાવો.}

\begin{solutionbox}
\begin{tabulary}{\linewidth}{|L|L|}
\hline
\textbf{વિભાવના} & \textbf{વર્ણન} \\ \hline
\textbf{વ્યાખ્યા} & ત્રણ-ટર્મિનલ નેટવર્ક જે ત્રણ ઇમ્પીડન્સથી બનેલું હોય - એક સિરીઝમાં અને બે પેરેલલમાં \\ \hline
\textbf{સ્ટ્રક્ચર} & બે ઇમ્પીડન્સ ઇનપુટ અને આઉટપુટથી કોમન બિંદુ સુધી જોડાયેલા, એક ઇનપુટ અને આઉટપુટ વચ્ચે \\ \hline
\textbf{પેરામીટર્સ} & Z, Y, h, અથવા ABCD પેરામીટર્સનો ઉપયોગ કરીને વ્યાખ્યાયિત કરી શકાય છે \\ \hline
\textbf{એપ્લિકેશન્સ} & મેચિંગ નેટવર્ક્સ, ફિલ્ટર્સ, એટેન્યુએટર્સ, ફેઝ શિફ્ટર્સ \\ \hline
\end{tabulary}

\begin{center}
\begin{circuitikz}[auto]
    \node [coordinate] (in_top) at (0,2) {};
    \node [coordinate] (in_bot) at (0,0) {};
    \node [coordinate] (out_top) at (4,2) {};
    \node [coordinate] (out_bot) at (4,0) {};
    
    \draw (in_top) to [short, o-*] (1,2) coordinate (t1);
    \draw (in_bot) to [short, o-*] (1,0) coordinate (b1);
    
    \draw (t1) to [generic, l=$Z_2$] (3,2) coordinate (t2);
    \draw (b1) -- (3,0) coordinate (b2);
    
    \draw (t1) to [generic, l=$Z_1$] (b1);
    \draw (t2) to [generic, l=$Z_3$] (b2);
    
    \draw (t2) to [short, *-o] (out_top);
    \draw (b2) to [short, *-o] (out_bot);
    
    \node [left] at (in_top) {Input};
    \node [right] at (out_top) {Output};
\end{circuitikz}
\captionof{figure}{$\pi$ Network Structure}
\end{center}
\end{solutionbox}

\begin{mnemonicbox}
\mnemonic{PIE: પાઈ ઇમ્પીડન્સીસ કનેક્ટેડ એટ એન્ડ્સ}
\end{mnemonicbox}

\questionmarks{2(a OR)}{3}{નેટવર્કનાં પ્રકારો જણાવો.}

\begin{solutionbox}
\begin{tabulary}{\linewidth}{|L|L|}
\hline
\textbf{નેટવર્ક પ્રકારો} & \textbf{ઉદાહરણો} \\ \hline
\textbf{લિનિયરતા આધારિત} & લિનિયર નેટવર્ક્સ, નોન-લિનિયર નેટવર્ક્સ \\ \hline
\textbf{ઘટકો આધારિત} & પેસિવ નેટવર્ક્સ, એક્ટિવ નેટવર્ક્સ \\ \hline
\textbf{સ્ટ્રક્ચર આધારિત} & લમ્પ્ડ નેટવર્ક્સ, ડિસ્ટ્રિબ્યુટેડ નેટવર્ક્સ \\ \hline
\textbf{વર્તણૂક આધારિત} & બાઇલેટરલ નેટવર્ક્સ, યુનિલેટરલ નેટવર્ક્સ \\ \hline
\textbf{ટોપોલોજી આધારિત} & T-નેટવર્ક્સ, $\pi$-નેટવર્ક્સ, લેટિસ નેટવર્ક્સ \\ \hline
\textbf{પોર્ટ્સ આધારિત} & વન-પોર્ટ નેટવર્ક્સ, ટુ-પોર્ટ નેટવર્ક્સ, મલ્ટિ-પોર્ટ નેટવર્ક્સ \\ \hline
\end{tabulary}

\begin{center}
\begin{circuitikz}[
    level 1/.style = {sibling distance=4cm},
    level 2/.style = {sibling distance=2cm},
    edge from parent/.style = {draw, -latex},
    every node/.style = {rectangle, draw, rounded corners, align=center, font=\small}
]
    \node {Network Types}
        child { node {Linearity}
            child { node {Linear} }
            child { node {Non-Linear} }
        }
        child { node {Components}
            child { node {Active} }
            child { node {Passive} }
        }
        child { node {Topology}
            child { node {T / $\pi$} }
            child { node {Lattice} }
        };
\end{circuitikz}
\captionof{figure}{Classification of Networks}
\end{center}
\end{solutionbox}

\begin{mnemonicbox}
\mnemonic{PLAN-TB: પેસિવ-લિનિયર-એક્ટિવ-નેટવર્ક-ટોપોલોજી-બાઇલેટરલ}
\end{mnemonicbox}

\questionmarks{2(b OR)}{4}{40, 60 અને 80 ohms ના રેઝીસ્ટર ડેલ્ટામાં કનેક્ટ કરેલા છે. સ્ટાર કનેક્શનનાં ઇક્વીવેલન્ટ રેઝીસ્ટર શોધો.}

\begin{solutionbox}
\begin{center}
\begin{circuitikz}[auto]
    % Delta
    \node at (0,2.5) (Ad) {A};
    \node at (-1.5,0) (Bd) {B};
    \node at (1.5,0) (Cd) {C};
    
    \draw (Ad) to [R, l=$R_{12}(40\Omega)$] (Bd);
    \draw (Bd) to [R, l=$R_{23}(60\Omega)$] (Cd);
    \draw (Cd) to [R, l=$R_{31}(80\Omega)$] (Ad);
    \node at (0,-1) {(a) Delta Connection};

    % Arrow
    \draw [gtu arrow, line width=1mm] (2.5, 1.25) -- (4.5, 1.25);

    % Star
    \begin{scope}[shift={(6,0)}]
        \node at (0,3) (A) {A};
        \node at (-1.5,0) (B) {B};
        \node at (1.5,0) (C) {C};
        \node at (0,1) (N) {};
        
        \draw (A) to [R, l=$R_1$] (N.center);
        \draw (B) to [R, l=$R_2$] (N.center);
        \draw (C) to [R, l=$R_3$] (N.center);
        \node at (0,-1) {(b) Star Connection};
    \end{scope}
\end{circuitikz}
\captionof{figure}{Delta to Star Transformation}
\end{center}

\begin{tabulary}{\linewidth}{|L|L|L|}
\hline
\textbf{ડેલ્ટા થી સ્ટાર કન્વર્ઝન ફોર્મ્યુલા} & \textbf{ગણતરી} & \textbf{પરિણામ} \\ \hline
$R_1 = \frac{R_{12} R_{31}}{R_{12}+R_{23}+R_{31}}$ & $(40 {\times} 80)/(40+60+80)$ & 17.78 $\Omega$ \\ \hline
$R_2 = \frac{R_{12} R_{23}}{R_{12}+R_{23}+R_{31}}$ & $(40 {\times} 60)/(40+60+80)$ & 13.33 $\Omega$ \\ \hline
$R_3 = \frac{R_{23} R_{31}}{R_{12}+R_{23}+R_{31}}$ & $(60 {\times} 80)/(40+60+80)$ & 26.67 $\Omega$ \\ \hline
\end{tabulary}
\end{solutionbox}

\begin{mnemonicbox}
\mnemonic{DPS: ડેલ્ટા પ્રોડક્ટ ઓવર સમ}
\end{mnemonicbox}

\questionmarks{2(c OR)}{7}{symmetrical T – network માટે કેરેક્ટરાસ્ટીક ઇમ્પીડન્સ સમજાવો. ZOT નું સૂત્ર ZOC and ZSC ના રૂપમાં તારવો.}

\begin{solutionbox}
\begin{tabulary}{\linewidth}{|L|L|}
\hline
\textbf{વિભાવના} & \textbf{વર્ણન} \\ \hline
\textbf{કેરેક્ટરાસ્ટીક ઇમ્પીડન્સ ($Z_0$)} & આઉટપુટ પોર્ટ પર જોડાયેલું ઇમ્પીડન્સ જેના કારણે ઇનપુટ ઇમ્પીડન્સ $Z_0$ ની બરાબર થાય \\ \hline
\textbf{સિમેટ્રિકલ T-નેટવર્ક} & T-નેટવર્ક જેમાં બંને બાજુના સિરીઝ ઇમ્પીડન્સ સમાન હોય \\ \hline
\textbf{ZOC અને ZSC} & નેટવર્કના ઓપન-સર્કિટ અને શોર્ટ-સર્કિટ ઇમ્પીડન્સીસ \\ \hline
\end{tabulary}

\begin{center}
\begin{circuitikz}[auto]
    \node [coordinate] (in_top) at (0,2) {};
    \node [coordinate] (in_bot) at (0,0) {};
    \node [coordinate] (out_top) at (6,2) {};
    \node [coordinate] (out_bot) at (6,0) {};
    
    \draw (in_top) to [generic, l=$Z_1/2$] (3,2) coordinate (mid);
    \draw (mid) to [generic, l=$Z_1/2$] (out_top);
    \draw (in_bot) -- (out_bot);
    \draw (mid) to [generic, l=$Z_2$] (3,0);
\end{circuitikz}
\captionof{figure}{Symmetrical T-Network}
\end{center}

સિમેટ્રિકલ T-નેટવર્ક માટે:
\begin{itemize}
    \item સિરીઝ ઇમ્પીડન્સીસ ($Z_1/2$) સમાન હોય છે
    \item $Z_2$ એ શન્ટ ઇમ્પીડન્સ છે
\end{itemize}

કેરેક્ટરાસ્ટીક ઇમ્પીડન્સ ($Z_{OT}$) આ રીતે આપવામાં આવે છે:
\[ Z_{OT} = \sqrt{Z_{OC} \times Z_{SC}} \]

જ્યાં:
\begin{itemize}
    \item $Z_{OC} = \text{ઓપન સર્કિટ ઇમ્પીડન્સ} = Z_1/2 + Z_2$ (આઉટપુટ ઓપન)
    \item $Z_{SC} = \text{શોર્ટ સર્કિટ ઇમ્પીડન્સ} = Z_1/2 + \frac{(Z_1/2 \times Z_2)}{(Z_1/2 + Z_2)}$
\end{itemize}

તેથી:
\[ Z_{OT} = \sqrt{Z_1^2/4 + Z_1 Z_2} \]

\end{solutionbox}

\begin{mnemonicbox}
\mnemonic{TOSS: T-નેટવર્ક્સ ઓપન એન્ડ શોર્ટ સર્કિટ સ્ક્વેર-રૂટ}
\end{mnemonicbox}


% ==================================================================
% QUESTION 3
% ==================================================================

\questionmarks{3(a)}{3}{Kirchhoff's law સમજાવો.}

\begin{solutionbox}
\begin{tabulary}{\linewidth}{|L|L|L|}
\hline
\textbf{નિયમ} & \textbf{વિધાન} & \textbf{ઉપયોગ} \\ \hline
\textbf{Kirchhoff's Current Law (KCL)} & નોડમાં પ્રવેશતા કરંટનો સરવાળો નોડમાંથી નીકળતા કરંટના સરવાળા બરાબર હોય & નોડલ એનાલિસિસ માટે ઉપયોગી \\ \hline
\textbf{Kirchhoff's Voltage Law (KVL)} & કોઈપણ બંધ લૂપની આસપાસ વોલ્ટેજનો સરવાળો શૂન્ય હોય & મેશ એનાલિસિસ માટે ઉપયોગી \\ \hline
\end{tabulary}

\begin{center}
\begin{circuitikz}[auto]
    % KCL
    \begin{scope}[shift={(0,0)}]
        \node [gtu state, minimum size=1cm] (N) {Node};
        \draw [<-] (N) -- ++(-1.5, 1) node[left] {$I_1$};
        \draw [<-] (N) -- ++(-1.5, -1) node[left] {$I_2$};
        \draw [->] (N) -- ++(1.5, 1) node[right] {$I_3$};
        \draw [->] (N) -- ++(1.5, -1) node[right] {$I_4$};
        \node at (0,-2) {KCL: $I_1+I_2 = I_3+I_4$};
    \end{scope}

    % KVL
    \begin{scope}[shift={(6,0)}]
        \node [coordinate] (A) at (-1.5,1) {};
        \node [coordinate] (B) at (1.5,1) {};
        \node [coordinate] (C) at (1.5,-1) {};
        \node [coordinate] (D) at (-1.5,-1) {};
        
        \draw (A) to [R, l=$R_1$] (B);
        \draw (B) to [R, l=$R_2$] (C);
        \draw (C) to [R, l=$R_3$] (D);
        \draw (D) to [V, l=$V_{src}$] (A);
        
        \draw [->, thick] (-0.5,0) arc (180:-180:0.5);
        \node at (0,-2) {KVL: $\sum V_{drop} = \sum V_{rise}$};
    \end{scope}
\end{circuitikz}
\captionof{figure}{Kirchhoff's Laws}
\end{center}
\end{solutionbox}

\begin{mnemonicbox}
\mnemonic{KVC: કિરચોફ વેરિફાઈસ કરંટ એન્ડ વોલ્ટેજ લોઝ}
\end{mnemonicbox}

\questionmarks{3(b)}{4}{Mesh analysis સમજાવો.}

\begin{solutionbox}
\begin{tabulary}{\linewidth}{|L|L|}
\hline
\textbf{વિભાવના} & \textbf{વર્ણન} \\ \hline
\textbf{વ્યાખ્યા} & દરેક સ્વતંત્ર બંધ લૂપ (મેશ) માટે KVL લાગુ પાડીને સર્કિટ સમસ્યાઓ ઉકેલવાની પદ્ધતિ \\ \hline
\textbf{પ્રક્રિયા} & 1. દરેક લૂપને મેશ કરંટ આપો \newline 2. દરેક મેશ માટે KVL સમીકરણો લખો \newline 3. પરિણામી સમીકરણોની સિસ્ટમ ઉકેલો \\ \hline
\textbf{ફાયદાઓ} & - સમીકરણોની સંખ્યા ઘટાડે છે \newline - ઘણી શાખાઓ વાળા સર્કિટ્સ માટે સારું કામ કરે છે \newline - વોલ્ટેજ સ્ત્રોતો વાળી સમસ્યાઓ માટે યોગ્ય \\ \hline
\end{tabulary}

\begin{center}
\begin{circuitikz}[auto]
    \node [coordinate] (A) at (0,2) {};
    \node [coordinate] (B) at (0,0) {};
    
    \draw (A) to [V, l=$V_1$] (B);
    \draw (A) to [R, l=$R_1$] (2,2) coordinate (C) to [R, l=$R_3$] (2,0) coordinate (D) to [short] (B);
    \draw (C) to [R, l=$R_2$] (4,2) coordinate (E) to [V, l=$V_2$] (4,0) coordinate (F) to [short] (D);
    
    \draw [->] (0.8,1) arc (180:-90:0.5) node[right] {$I_1$};
    \draw [->] (2.8,1) arc (180:-90:0.5) node[right] {$I_2$};
\end{circuitikz}
\captionof{figure}{Mesh Analysis Example}
\end{center}
\end{solutionbox}

\begin{mnemonicbox}
\mnemonic{MAIL: મેશ એનાલિસિસ યુઝિસ ઇન્ડિપેન્ડન્ટ લૂપ્સ}
\end{mnemonicbox}

\questionmarks{3(c)}{7}{Thevenin's theorem નો ઉપયોગ કરીને ઉપર દશાર્વેલ સર્કિટ માટે 5 $\Omega$ રેઝીસ્ટર માંથી પસાર થતો કરંટ શોધો.}

\begin{solutionbox}
\begin{center}
\begin{circuitikz}[auto]
    % Source
    \draw (0,0) to [V, l=100V] (0,4) -- (2,4) coordinate (top_bridge);
    \draw (0,0) -- (2,0) coordinate (bot_bridge);
    
    % Bridge
    % Top node: top_bridge (let's call it T), Bot node: bot_bridge (B)
    % A is mid-left, B is mid-right (using Q's notation)
    % Actually diagram is:
    %      10ohm    15ohm
    %     /     \  /     \
    % 100V       A ------ B
    %     \     / |      /
    %      \   /  5ohm  /
    %       \ /        /
    %      6ohm     8ohm
    
    % Let's redraw properly
    \node [coordinate] (src_top) at (0,3) {};
    \node [coordinate] (src_bot) at (0,0) {};
    \node [coordinate] (T) at (3,3) {}; % Top of bridge
    \node [coordinate] (bot) at (3,0) {}; % Bottom of bridge
    \node [coordinate] (A) at (2,1.5) {A}; % Left Node A
    \node [coordinate] (B) at (4,1.5) {B}; % Right Node B
    
    \draw (src_top) to [V, l=100V] (src_bot);
    \draw (src_top) -- (T); \draw (src_bot) -- (bot);
    
    \draw (T) to [R, l=10$\Omega$] (A);
    \draw (A) to [R, l=6$\Omega$] (bot);
    
    \draw (T) to [R, l=15$\Omega$] (B);
    \draw (B) to [R, l=8$\Omega$] (bot);
    
    \draw (A) to [R, l=5$\Omega$] (B);
    
    \node [left] at (A) {A}; \node [right] at (B) {B};
\end{circuitikz}
\captionof{figure}{Thevenin Problem Circuit}
\end{center}

\textbf{સ્ટેપ 1:} 5$\Omega$ રેઝીસ્ટર દૂર કરીને ઓપન સર્કિટ વોલ્ટેજ ($V_{th}$) શોધો
\textbf{સ્ટેપ 2:} થેવેનિનનું ઇક્વિવેલન્ટ રેઝિસ્ટન્સ ($R_{th}$) શોધો
\textbf{સ્ટેપ 3:} 5$\Omega$ રેઝીસ્ટરમાંથી પસાર થતો કરંટ ગણો

\begin{tabulary}{\linewidth}{|L|L|L|}
\hline
\textbf{સ્ટેપ} & \textbf{ગણતરી} & \textbf{પરિણામ} \\ \hline
\textbf{$V_{th}$} & A અને B વચ્ચેનું વોલ્ટેજ જ્યારે 5$\Omega$ દૂર કરવામાં આવે & 38.46 V \\ \hline
\textbf{$R_{th}$} & A અને B થી જોવાતું ઇક્વિવેલન્ટ રેઝિસ્ટન્સ જ્યારે 100V સ્ત્રોત શોર્ટ કરવામાં આવે & 3.6 $\Omega$ \\ \hline
\textbf{કરંટ} & $I = V_{th}/(R_{th} + 5) = 38.46/(3.6 + 5)$ & 4.47 A \\ \hline
\end{tabulary}
\end{solutionbox}

\begin{mnemonicbox}
\mnemonic{TVR: થેવેનિન રિપ્લેસીસ વોલ્ટેજ એન્ડ રેઝીસ્ટન્સ}
\end{mnemonicbox}

\questionmarks{3(a OR)}{3}{Superposition Theorem જણાવો અને સમજાવો.}

\begin{solutionbox}
\begin{tabulary}{\linewidth}{|L|L|}
\hline
\textbf{વિભાવના} & \textbf{વર્ણન} \\ \hline
\textbf{વિધાન} & લિનિયર સર્કિટમાં બહુવિધ સ્ત્રોતો સાથે, કોઈપણ બિંદુ પર પ્રતિભાવ દરેક સ્ત્રોત એકલા કાર્ય કરતા હોય ત્યારે થતા પ્રતિભાવોના સરવાળા બરાબર હોય છે \\ \hline
\textbf{પ્રક્રિયા} & 1. એક સમયે એક સ્ત્રોત ધ્યાનમાં લો \newline 2. અન્ય વોલ્ટેજ સ્ત્રોતોને શોર્ટ સર્કિટથી બદલો \newline 3. અન્ય કરંટ સ્ત્રોતોને ઓપન સર્કિટથી બદલો \newline 4. વ્યક્તિગત પ્રતિભાવો શોધો \newline 5. બધા પ્રતિભાવોને બીજગણિતીય રીતે ઉમેરો \\ \hline
\textbf{મર્યાદા} & માત્ર લિનિયર સર્કિટ્સ અને વોલ્ટેજ/કરંટ પ્રતિભાવો માટે જ લાગુ \\ \hline
\end{tabulary}

\begin{center}
\begin{circuitikz}[auto, node distance=2.5cm]
    \node [gtu block] (orig) {Original Circuit ($V_1, V_2$)};
    \node [gtu block, below left=of orig] (v1) {Circuit w/ $V_1$};
    \node [gtu block, below right=of orig] (v2) {Circuit w/ $V_2$};
    \node [gtu block, below=of v1] (r1) {Response $R_1$};
    \node [gtu block, below=of v2] (r2) {Response $R_2$};
    \node [gtu decision, below=2cm of orig] (sum) {Total $R = R_1 + R_2$};

    \draw [gtu arrow] (orig) -- (v1);
    \draw [gtu arrow] (orig) -- (v2);
    \draw [gtu arrow] (v1) -- (r1);
    \draw [gtu arrow] (v2) -- (r2);
    \draw [gtu arrow] (r1) -- (sum);
    \draw [gtu arrow] (r2) -- (sum);
\end{circuitikz}
\captionof{figure}{Superposition Principle}
\end{center}
\end{solutionbox}

\begin{mnemonicbox}
\mnemonic{SUPER: સોર્સિસ યુઝ્ડ પ્રોગ્રેસિવલી ઈક્વલ્સ રિસ્પોન્સ}
\end{mnemonicbox}

\questionmarks{3(b OR)}{4}{કોઈપણ સર્કિટનો ઉપયોગ કરીને ડ્યુઅલ નેટવર્ક દોરવાની પદ્ધતિ સમજાવો.}

\begin{solutionbox}
\begin{tabulary}{\linewidth}{|L|L|}
\hline
\textbf{સ્ટેપ} & \textbf{વર્ણન} \\ \hline
\textbf{ગ્રાફમાં રૂપાંતરણ} & સર્કિટને પ્લેનર ગ્રાફ તરીકે દોરો \\ \hline
\textbf{ડ્યુઅલ ગ્રાફ દોરો} & મૂળ ગ્રાફના દરેક ક્ષેત્રમાં એક નોડ મૂકો \\ \hline
\textbf{નોડ્સ જોડો} & મૂળ ગ્રાફની દરેક એજને ક્રોસ કરતી એજ દોરો \\ \hline
\textbf{ઘટકોને બદલો} & - રેઝિસ્ટન્સ R કન્ડક્ટન્સ 1/R બને \newline - વોલ્ટેજ સોર્સ કરંટ સોર્સ બને \newline - સિરીઝ પેરેલલ બને \newline - ઇમ્પીડન્સ Z એડમિટન્સ 1/Z બને \\ \hline
\end{tabulary}

\begin{center}
\begin{circuitikz}[auto]
    \node [coordinate] (O) at (0,0) {};
    
    % Original (Series R L)
    \draw (-2,0) to [R, l=$R$] (-2,2) to [L, l=$L$] (0,2);
    
    % Dual nodes
    \node [gtu state, fill=red!20] (N1) at (-1,1) {1};
    \node [gtu state, fill=red!20] (N2) at (1,1) {2};
    
    \draw [dashed, red, thick] (N1) -- node[above] {Dual $G$} (-3,1);
    \draw [dashed, red, thick] (N1) -- node[above] {Dual $C$} (1,1);
    
    \node at (0,-1) {Conceptual Dual Construction};
\end{circuitikz}
\captionof{figure}{Dual Network Construction}
\end{center}
\end{solutionbox}

\begin{mnemonicbox}
\mnemonic{DVSG: ડ્યુઅલ ટ્રાન્સફોર્મ્સ વોલ્ટેજ ટુ સિરીઝ ટુ ગ્રાફ્સ}
\end{mnemonicbox}

\questionmarks{3(c OR)}{7}{ઉપર આપેલ નેટવર્ક માટે નોર્ટનની ઇક્વીવેલન્ટ સર્કિટ શોધો. લોડ કરંટ શોધો જો (i) $R_L = 3$ k$\Omega$ (ii) $R_L = 1.5$ $\Omega$}

\begin{solutionbox}
\begin{center}
\begin{circuitikz}[auto]
    % 10V source, 2k resistors bridge-like structure
    % Based on MDX goat diagram
    %        2k          2k          2k
    %       ----        ----        ----
    %      /    \      /    \      /    \
    %   C +      D    +      E    +      A
    % ... 10V ...
    
    % It looks like a ladder network
    \node [coordinate] (B) at (0,0) {};
    \node [coordinate] (C) at (0,2) {};
    
    \draw (C) to [V, l=10V] (B);
    
    \draw (C) to [R, l=2k$\Omega$] (2,2) coordinate (D);
    \draw (D) to [R, l=2k$\Omega$] (2,0) coordinate (B2);
    \draw (B) -- (B2);
    
    \draw (D) to [R, l=2k$\Omega$] (4,2) coordinate (E);
    \draw (E) to [R, l=2k$\Omega$] (4,0) coordinate (B3);
    \draw (B2) -- (B3);
    
    \draw (E) to [R, l=2k$\Omega$] (6,2) coordinate (A);
    \draw (B3) -- (6,0) coordinate (B_end);
    
    \draw (A) to [R, l=$R_L$] (B_end);
    \node [above] at (A) {A};
    \node [below] at (B_end) {B};
\end{circuitikz}
\captionof{figure}{Norton Problem Circuit}
\end{center}

\begin{itemize}
    \item \textbf{સ્ટેપ 1:} નોર્ટનનો કરંટ ($I_N$) શોધો
    \item \textbf{સ્ટેપ 2:} નોર્ટનનું રેઝિસ્ટન્સ ($R_N$) શોધો
    \item \textbf{સ્ટેપ 3:} લોડ કરંટ્સ ગણો
\end{itemize}

\begin{tabulary}{\linewidth}{|L|L|L|}
\hline
\textbf{સ્ટેપ} & \textbf{ગણતરી} & \textbf{પરિણામ} \\ \hline
\textbf{$I_N$} & A થી B સુધીનો શોર્ટ સર્કિટ કરંટ & 1.25 mA \\ \hline
\textbf{$R_N$} & A થી B સુધી જોવાતું ઇક્વિવેલન્ટ રેઝિસ્ટન્સ જ્યારે 10V સ્ત્રોત શોર્ટ કરવામાં આવે & 1 k$\Omega$ \\ \hline
\textbf{$I_L$ ($R_L = 3$ k$\Omega$)} & $I_L = I_N \times R_N/(R_N + R_L) = 1.25 \times 1/(1 + 3)$ & 0.31 mA \\ \hline
\textbf{$I_L$ ($R_L = 1.5$ $\Omega$)} & $I_L = I_N \times R_N/(R_N + R_L) = 1.25 \times 1000/(1000 + 1.5)$ & 1.25 mA \\ \hline
\end{tabulary}
\end{solutionbox}

\begin{mnemonicbox}
\mnemonic{NICE: નોર્ટન્સ સર્કિટ ઈઝ કરંટ ઈક્વિવેલન્ટ}
\end{mnemonicbox}


% ==================================================================
% QUESTION 4
% ==================================================================

\questionmarks{4(a)}{3}{કોઇલ માટે ક્વોલિટી ફેક્ટર Q નું સમીકરણ મેળવો.}

\begin{solutionbox}
\begin{tabulary}{\linewidth}{|L|L|}
\hline
\textbf{પેરામીટર} & \textbf{સંબંધ} \\ \hline
\textbf{Q ફેક્ટર વ્યાખ્યા} & સંગ્રહિત ઊર્જા અને પ્રતિ ચક્ર વેડફાતી ઊર્જાનો ગુણોત્તર \\ \hline
\textbf{કોઇલ ઇમ્પીડન્સ} & $Z = R + j\omega L$ \\ \hline
\textbf{રિએક્ટન્સ} & $X_L = \omega L$ \\ \hline
\textbf{ક્વોલિટી ફેક્ટર} & $Q = X_L/R = \omega L/R$ \\ \hline
\end{tabulary}

\begin{center}
\begin{circuitikz}[auto]
    \draw (0,0) to [R, l=$R$, o-] (2,0) to [L, l=$L$, -o] (4,0);
    \node at (2,-1) {Practical Coil Model};
\end{circuitikz}
\captionof{figure}{Coil Equivalent Circuit}
\end{center}

કોઇલ માટે, સંગ્રહિત ઊર્જા ચુંબકીય ક્ષેત્રમાં (ઇન્ડક્ટરમાં) હોય છે, જ્યારે વેડફાતી ઊર્જા રેઝિસ્ટન્સમાં હોય છે. આમાંથી:

\[ Q = 2\pi \times \frac{\text{સંગ્રહિત ઊર્જા}}{\text{પ્રતિ ચક્ર વેડફાતી ઊર્જા}} \]
\[ Q = \frac{\omega L}{R} \]
\end{solutionbox}

\begin{mnemonicbox}
\mnemonic{QREL: ક્વોલિટી રિલેટ્સ એનર્જી ટુ લોસ}
\end{mnemonicbox}

\questionmarks{4(b)}{4}{શ્રેણી RLC સર્કિટમાં R=30 $\Omega$, L=0.5 H અને C=5 $\mu$F છે. (i)Q પરિબળ, (ii) BW, (iii) અપર કટ ઓફ અને લોઅર કટ ઓફ ફ્રીક્વન્સીઝની ગણતરી કરો.}

\begin{solutionbox}
\begin{center}
\begin{circuitikz}[auto]
    \draw (0,0) to [R, l=$R$, o-] (2,0) to [L, l=$L$] (4,0) to [C, l=$C$, -o] (6,0);
    \node at (3,-1) {Series RLC Circuit};
\end{circuitikz}
\captionof{figure}{Series RLC}
\end{center}

\begin{tabulary}{\linewidth}{|L|L|L|L|}
\hline
\textbf{પેરામીટર} & \textbf{સૂત્ર} & \textbf{ગણતરી} & \textbf{પરિણામ} \\ \hline
\textbf{રેઝોનન્ટ ફ્રીક્વન્સી ($f_0$)} & $f_0 = 1/(2\pi\sqrt{LC})$ & $1/(2\pi\sqrt{0.5 \times 5 \times 10^{-6}})$ & 100.53 Hz \\ \hline
\textbf{Q ફેક્ટર} & $Q = (1/R)\sqrt{L/C}$ & $(1/30)\sqrt{0.5/(5 \times 10^{-6})}$ & 105.57 \\ \hline
\textbf{બેન્ડવિડ્થ (BW)} & $BW = f_0/Q$ & $100.53/105.57$ & 0.952 Hz \\ \hline
\textbf{લોઅર કટઓફ ($f_1$)} & $f_1 = f_0 - BW/2$ & $100.53 - 0.952/2$ & 100.05 Hz \\ \hline
\textbf{અપર કટઓફ ($f_2$)} & $f_2 = f_0 + BW/2$ & $100.53 + 0.952/2$ & 101.01 Hz \\ \hline
\end{tabulary}
\end{solutionbox}

\begin{mnemonicbox}
\mnemonic{QBCUT: ક્વોલિટી બેન્ડવિડ્થ કટઓફ યુનિકલી રિલેટેડ}
\end{mnemonicbox}

\questionmarks{4(c)}{7}{મ્યુચ્યુઅલ ઇન્ડક્ટન્સના કો-એફીસીએન્ટ સાથે મ્યુચ્યુઅલ ઇન્ડક્ટન્સ સમજાવો. K નું સમીકરણ પણ મેળવો.}

\begin{solutionbox}
\begin{tabulary}{\linewidth}{|L|L|}
\hline
\textbf{વિભાવના} & \textbf{વર્ણન} \\ \hline
\textbf{મ્યુચ્યુઅલ ઇન્ડક્ટન્સ ($M$)} & ગુણધર્મ જ્યાં એક કોઇલમાં કરંટ બદલાવથી પાસેની કોઇલમાં વોલ્ટેજ ઉત્પન્ન થાય છે \\ \hline
\textbf{વ્યાખ્યા} & પ્રાથમિક કોઇલમાં કરંટના બદલાવના દરના સાપેક્ષ ગૌણ કોઇલમાં પ્રેરિત વોલ્ટેજનો ગુણોત્તર \\ \hline
\textbf{સૂત્ર} & $M = k\sqrt{L_1 L_2}$ \\ \hline
\textbf{કપલિંગ ગુણાંક ($k$)} & કોઇલ્સ વચ્ચે ચુંબકીય કપલિંગનું માપ ($0 \le k \le 1$) \\ \hline
\end{tabulary}

\begin{center}
\begin{circuitikz}[auto]
    \draw (0,0) to [L, l=$L_1$] (0,2);
    \draw (2,0) to [L, l=$L_2$] (2,2);
    \draw [dashed, <->] (0.5,1) -- node[above] {$M$} (1.5,1);
    \node at (1,-0.5) {Coupled Coils};
\end{circuitikz}
\captionof{figure}{Mutual Inductance}
\end{center}

બે ઇન્ડક્ટર્સ $L_1$ અને $L_2$ માટે, મ્યુચ્યુઅલ ઇન્ડક્ટન્સ $M$ છે:
\[ M = k\sqrt{L_1 L_2} \]

જ્યાં કપલિંગ ગુણાંક $k$ છે:
\[ k = \frac{M}{\sqrt{L_1 L_2}} \]

\begin{itemize}
    \item $k$ એક કોઇલથી બીજી કોઇલ સાથે જોડાતા ચુંબકીય ફ્લક્સના અંશનું પ્રતિનિધિત્વ કરે છે
    \item સંપૂર્ણ કપલ કોઇલ્સ માટે, $k = 1$
    \item કોઈ કપલિંગ નથી ત્યારે, $k = 0$
\end{itemize}
\end{solutionbox}

\begin{mnemonicbox}
\mnemonic{MKL: મ્યુચ્યુઅલ કપલિંગ K લિંક્સ ઇન્ડક્ટર્સ}
\end{mnemonicbox}

\questionmarks{4(a OR)}{3}{કપલ સર્કિટ માટેકપ્લીંગના પ્રકારો સમજાવો.}

\begin{solutionbox}
\begin{tabulary}{\linewidth}{|L|L|L|}
\hline
\textbf{કપલિંગના પ્રકાર} & \textbf{લક્ષણો} & \textbf{ઉપયોગો} \\ \hline
\textbf{ટાઇટ/ક્લોઝ કપલિંગ ($k \approx 1$)} & - લગભગ બધો ફ્લક્સ બંને કોઇલ્સને જોડે છે \newline - ઉચ્ચ ટ્રાન્સફર ક્ષમતા \newline - $k$ વેલ્યુ 1 ની નજીક & ટ્રાન્સફોર્મર્સ, પાવર ટ્રાન્સફર \\ \hline
\textbf{લૂઝ કપલિંગ ($k \ll 1$)} & - ફ્લક્સનો નાનો અંશ બીજી કોઇલને જોડે છે \newline - ઓછી ટ્રાન્સફર ક્ષમતા \newline - $k$ વેલ્યુ 1 કરતા ઘણી ઓછી & RF સર્કિટ્સ, ટ્યુન્ડ ફિલ્ટર્સ \\ \hline
\textbf{ક્રિટિકલ કપલિંગ ($k=k_c$)} & - બેન્ડપાસ પ્રતિભાવ માટે શ્રેષ્ઠ કપલિંગ \newline - રેઝોનન્સ પર મહત્તમ પાવર ટ્રાન્સફર & બેન્ડપાસ ફિલ્ટર્સ, IF ટ્રાન્સફોર્મર્સ \\ \hline
\textbf{ઇન્ડક્ટિવ કપલિંગ} & - ચુંબકીય ક્ષેત્ર દ્વારા કપલિંગ & ટ્રાન્સફોર્મર્સ, વાયરલેસ ચાર્જિંગ \\ \hline
\textbf{કેપેસિટિવ કપલિંગ} & - વિદ્યુત ક્ષેત્ર દ્વારા કપલિંગ & સિગ્નલ કપલિંગ, કેપેસિટિવ સેન્સર્સ \\ \hline
\end{tabulary}

\begin{center}
\begin{circuitikz}[auto, node distance=2cm]
    \node [gtu block] (tight) {Tight Coupling \\ $k \approx 1$};
    \node [gtu block, right=of tight] (loose) {Loose Coupling \\ $k \ll 1$};
    \node [gtu block, right=of loose] (crit) {Critical Coupling \\ $k = k_c$};
\end{circuitikz}
\captionof{figure}{Types of Coupling}
\end{center}
\end{solutionbox}

\begin{mnemonicbox}
\mnemonic{TLC: ટાઇટ, લૂઝ, ક્રિટિકલ કપલિંગ્સ}
\end{mnemonicbox}

\questionmarks{4(b OR)}{4}{ગુણવત્તા પરિબળ Q = 100, રેઝોનન્ટ ફ્રિકવન્સી Fr = 100 KHz સાથે 1 mH નું ઇન્ડક્ટન્સ ધરાવતું સમાંતર રેઝોનન્ટ સર્કિટ. શોધો (i) જરૂરી કેપેસીટન્સ C, (ii) કોઇલનો પ્રતિકાર R, (iii) BW.}

\begin{solutionbox}
\begin{center}
\begin{circuitikz}[auto]
    \draw (0,2) to [short, o-*] (1,2) -- (4,2) to [short, *-o] (5,2);
    \draw (0,0) to [short, o-*] (1,0) -- (4,0) to [short, *-o] (5,0);
    
    \draw (2,2) to [C, l=$C$] (2,0);
    \draw (3,2) to [L, l=$L$] (3,1) to [R, l=$R_{coil}$] (3,0);
    
    \node at (2.5,-1) {Parallel Resonant Circuit (Tank Circuit)};
\end{circuitikz}
\captionof{figure}{Parallel Resonance}
\end{center}

\begin{tabulary}{\linewidth}{|L|L|L|L|}
\hline
\textbf{પેરામીટર} & \textbf{સૂત્ર} & \textbf{ગણતરી} & \textbf{પરિણામ} \\ \hline
\textbf{કેપેસિટન્સ (C)} & $C = 1/(4\pi^2 f^2 L)$ & $1/(4\pi^2 \times (10^5)^2 \times 10^{-3})$ & 2.533 nF \\ \hline
\textbf{કોઇલ રેઝિસ્ટન્સ (R)} & $R = \omega L / Q$ & $2\pi \times 10^5 \times 10^{-3} / 100$ & 6.28 $\Omega$ \\ \hline
\textbf{બેન્ડવિડ્થ (BW)} & $BW = f_r/Q$ & $100 \text{ kHz} / 100$ & 1 kHz \\ \hline
\end{tabulary}
\end{solutionbox}

\begin{mnemonicbox}
\mnemonic{RCB: રેઝોનન્સ નીડ્સ કેપેસિટન્સ એન્ડ બેન્ડવિડ્થ}
\end{mnemonicbox}

\questionmarks{4(c OR)}{7}{series RLC સર્કિટની Band width અને Selectivity સમજાવો. શ્રેણી રેઝોનન્સ સર્કિટ માટે Q પરિબળ અને BW વચ્ચેનો સંબંધ પણ સ્થાપિત કરો.}

\begin{solutionbox}
\begin{tabulary}{\linewidth}{|L|L|L|}
\hline
\textbf{પેરામીટર} & \textbf{વ્યાખ્યા} & \textbf{સંબંધ} \\ \hline
\textbf{બેન્ડવિડ્થ (BW)} & હાફ-પાવર પોઇન્ટ્સ વચ્ચેનો ફ્રીક્વન્સી રેન્જ & $BW = f_2 - f_1 = R/L$ (rad/s $R/L$) \\ \hline
\textbf{સિલેક્ટિવિટી} & વિવિધ ફ્રીક્વન્સીઓના સિગ્નલ્સને અલગ કરવાની ક્ષમતા & BW સાથે વ્યસ્ત પ્રમાણમાં \\ \hline
\textbf{Q ફેક્ટર} & રેઝોનન્ટ ફ્રીક્વન્સીનો બેન્ડવિડ્થ સાથેનો ગુણોત્તર & $Q = \omega_0/BW$ \\ \hline
\end{tabulary}

\begin{center}
\begin{circuitikz}[auto]
    \begin{axis}[
        width=8cm, height=5cm,
        axis lines=middle,
        xlabel={$f$}, ylabel={Current $I$},
        xtick=\empty, ytick=\empty,
        clip=false
    ]
        \addplot[domain=0.5:2, samples=100, smooth, thick, blue] {1/sqrt((1-x^2)^2 + (0.2*x)^2)};
        
        \node at (axis cs: 1, 5) [above] {$I_{max}$};
        \node at (axis cs: 1, 0) [below] {$f_0$};
        
        \draw [dashed] (axis cs: 0, 3.5) -- (axis cs: 2, 3.5);
        \node at (axis cs: 2, 3.5) [right] {$0.707 I_{max}$};
        
        \draw [dashed] (axis cs: 0.9, 0) -- (axis cs: 0.9, 3.5);
        \node at (axis cs: 0.9, 0) [below] {$f_1$};
        
        \draw [dashed] (axis cs: 1.1, 0) -- (axis cs: 1.1, 3.5);
        \node at (axis cs: 1.1, 0) [below] {$f_2$};
        
        \draw [<->] (axis cs: 0.9, 1.5) -- node[below] {BW} (axis cs: 1.1, 1.5);
    \end{axis}
    \node at (4,-1) {Resonance Curve};
\end{circuitikz}
\captionof{figure}{Frequency Response}
\end{center}

સિરીઝ RLC સર્કિટ માટે:

\begin{itemize}
    \item રેઝોનન્સ ($f_0$) પર, ઇમ્પીડન્સ ન્યૂનતમ છે ($= R$)
    \item હાફ-પાવર પોઇન્ટ્સ ત્યારે આવે છે જ્યારે ઇમ્પીડન્સ $= \sqrt{2}R$
    \item આ બિંદુઓ પર, પાવર મહત્તમ પાવરનો અડધો હોય છે
\end{itemize}

બેન્ડવિડ્થ ($BW$) = $\omega_2 - \omega_1 = R/L$
Q ફેક્ટર = $\omega_0 L/R = \omega_0/BW$

તેથી, $BW = \omega_0/Q = 2\pi f_0/Q$

આ દર્શાવે છે કે Q ફેક્ટર અને બેન્ડવિડ્થ વ્યસ્ત રીતે સંબંધિત છે:
ઉચ્ચ Q $\rightarrow$ સાંકડી બેન્ડવિડ્થ $\rightarrow$ વધુ સારી સિલેક્ટિવિટી

\end{solutionbox}

\begin{mnemonicbox}
\mnemonic{BQS: બેન્ડવિડ્થ અને Q નક્કી કરે છે સિલેક્ટિવિટી}
\end{mnemonicbox}


% ==================================================================
% QUESTION 5
% ==================================================================

\questionmarks{5(a)}{3}{40 ડીબીનું એટેન્યુએશન આપવા અને 300 $\Omega$ પ્રતિકારના લોડમાં કામ કરવા માટે સપ્રમાણ T પ્રકારના એટેન્યુએટરને ડિઝાઇન કરો.}

\begin{solutionbox}
\begin{center}
\begin{circuitikz}[auto]
    \node [coordinate] (in_top) at (0,2) {};
    \node [coordinate] (in_bot) at (0,0) {};
    \node [coordinate] (out_top) at (6,2) {};
    \node [coordinate] (out_bot) at (6,0) {};
    
    \draw (in_top) to [R, l=$R_1/2$] (3,2) coordinate (mid);
    \draw (mid) to [R, l=$R_1/2$] (out_top);
    \draw (in_bot) -- (out_bot);
    \draw (mid) to [R, l=$R_2$] (3,0);
    
    \draw [dashed] (out_top) -- node[right] {$R_L=300\Omega$} (out_bot);
    
    \node at (0.5,1) {Input};
    \node at (5.5,1) {Output};
\end{circuitikz}
\captionof{figure}{T-Attenuator Design}
\end{center}

\begin{tabulary}{\linewidth}{|L|L|L|L|}
\hline
\textbf{પેરામીટર} & \textbf{સૂત્ર} & \textbf{ગણતરી} & \textbf{પરિણામ} \\ \hline
\textbf{એટેન્યુએશન (N)} & $N = 10^{(dB/20)}$ & $10^{(40/20)}$ & 100 \\ \hline
\textbf{ઇમ્પીડન્સ રેશિયો (K)} & $K = (N+1)/(N-1)$ & $(100+1)/(100-1)$ & 1.02 \\ \hline
\textbf{Z1} & $Z_1 = R_0[(K-1)/K]$ & $300[(1.02-1)/1.02]$ & 5.88 $\Omega$ \\ \hline
\textbf{Z2} & $Z_2 = R_0[2K/(K^2-1)]$ & $300[2 \times 1.02/(1.02^2-1)]$ & 594.12 $\Omega$ \\ \hline
\end{tabulary}
(નોંધ: T-એટેન્યુએટર ઘટકો $R_1$ (કુલ શ્રેણી) અને $R_2$ (શન્ટ) માટે પ્રમાણભૂત ડિઝાઇન સૂત્રોનો ઉપયોગ કરીને. T-સેક્શનમાં, શ્રેણી આર્મ્સ $R_1/2$ છે. કોષ્ટક કુલ શ્રેણી પ્રતિકાર $Z_1$ અથવા ઘટક મૂલ્યોની ગણતરી કરે છે? સામાન્ય રીતે સૂત્રો સંપૂર્ણ શ્રેણી આર્મ $R_1$ અથવા વ્યક્તિગત આર્મ્સ આપે છે. ચાલો પ્રમાણભૂત વ્યાખ્યાઓ ધારીએ: સૂત્ર $R_1 = R_0\frac{N-1}{N+1}$? ના, MDX સૂત્રો સહેજ અલગ છે, ખાસ કરીને K નો ઉપયોગ કરીને? K સામાન્ય રીતે N હોય છે. ચાલો MDX ની ચોકસાઈ તપાસીએ.
MDX સૂત્ર: $Z_1 = R_0[(K-1)/K]$ જ્યાં $K = (N+1)/(N-1)$.
રાહ જુઓ, જો $N=100$.
$K = 101/99 \approx 1.02$.
$Z_1 = 300 * (0.02/1.02) \approx 5.88$.
$Z_2 = 300 * (2.04 / (1.02^2 - 1)) \approx 300 * (2.04 / 0.0404) \approx 15148$. MDX 594.12 કહે છે.
ચાલો $594.12$ તપાસીએ.
પ્રમાણભૂત T-પેડ: $R_1 = R_0 \frac{N-1}{N+1} = 300 * \frac{99}{101} = 294$.
$R_2 = R_0 \frac{2N}{N^2-1} = 300 * \frac{200}{9999} = 6$.
MDX સૂત્રો અથવા MDX $Z_1, Z_2$ ના મારા અર્થઘટનમાં કંઈક ખોટું છે.
જોકે, વિશ્વાસપૂર્વક રૂપાંતરણ માટે ટેક્સ્ટની નકલ કરવી જરૂરી છે, સિવાય કે તે સ્પષ્ટપણે ખોટું હોય અને મારે તેને સુધારવું જોઈએ. વપરાશકર્તાએ "બનાવશો નહીં, વિસ્તૃત કરશો નહીં, અથવા સુવ્યવસ્થિત કરશો નહીં... ચોક્કસ ટેક્સ્ટને સ્થાનાંતરિત કરો" એમ કહ્યું. પરંતુ સરળ ગણિતની ભૂલો? "સખત સામગ્રીની ચોકસાઈ". હું MDX ગણતરી અને પરિણામને બરાબર જેમ છે તેમ ટ્રાન્સક્રાઇબ કરીશ, જો જરૂરી હોય તો એક નોંધ ઉમેરીશ, પરંતુ સૂચનાઓ મુજબ, ચોકસાઈ પ્રથમ.
MDX: $Z_2 = 594.12$. દર્શાવેલ ગણતરી: $300[2 \times 1.02 / (1.02^2 - 1)]$.
$1.02^2 - 1 = 1.0404 - 1 = 0.0404$.
$2 * 1.02 = 2.04$.
$2.04 / 0.0404 \approx 50.5$.
$300 * 50.5 = 15150$.
MDX પરિણામ 594.12 વિચિત્ર છે.
કદાચ MDX માં K નો અર્થ કંઈક બીજો છે?
જો $Z_2 = 600 \Omega$ આશરે?
ચાલો MDX કોષ્ટકને બરાબર ટ્રાન્સક્રાઇબ કરીએ.)

\end{solutionbox}

\begin{mnemonicbox}
\mnemonic{TANZ: T-એટેન્યુએટર નીડ્સ Z-પેરામીટર્સ}
\end{mnemonicbox}

\questionmarks{5(b)}{4}{ફિલ્ટર્સનું વર્ગીકરણ આપો.}

\begin{solutionbox}
\begin{tabulary}{\linewidth}{|L|L|L|}
\hline
\textbf{વર્ગીકરણ} & \textbf{પ્રકારો} & \textbf{લક્ષણો} \\ \hline
\textbf{ફ્રીક્વન્સી રિસ્પોન્સ આધારિત} & - લો પાસ \newline - હાઇ પાસ \newline - બેન્ડ પાસ \newline - બેન્ડ સ્ટોપ & - કટઓફ નીચેની ફ્રીક્વન્સી પસાર કરે \newline - કટઓફ ઉપરની ફ્રીક્વન્સી પસાર કરે \newline - બેન્ડની અંદરની ફ્રીક્વન્સી પસાર કરે \newline - બેન્ડની અંદરની ફ્રીક્વન્સી અવરોધે \\ \hline
\textbf{ઘટકો આધારિત} & - પેસિવ ફિલ્ટર્સ \newline - એક્ટિવ ફિલ્ટર્સ & - R, L, C ઘટકોનો ઉપયોગ \newline - RC સાથે એક્ટિવ ડિવાઇસનો ઉપયોગ \\ \hline
\textbf{ડિઝાઇન અભિગમ આધારિત} & - કન્સ્ટન્ટ-k ફિલ્ટર્સ \newline - m-ડેરાઇવ્ડ ફિલ્ટર્સ \newline - કમ્પોઝિટ ફિલ્ટર્સ & - સરળતમ ડિઝાઇન \newline - વધુ સારા કટઓફ લક્ષણો \newline - ફાયદાઓનું સંયોજન \\ \hline
\textbf{ટેકનોલોજી આધારિત} & - LC ફિલ્ટર્સ \newline - ક્રિસ્ટલ ફિલ્ટર્સ \newline - સેરામિક ફિલ્ટર્સ \newline - ડિજિટલ ફિલ્ટર્સ & - ઇન્ડક્ટર અને કેપેસિટરનો ઉપયોગ \newline - પિઝોઇલેક્ટ્રિક ક્રિસ્ટલનો ઉપયોગ \newline - પિઝોઇલેક્ટ્રિક સેરામિકનો ઉપયોગ \newline - સોફ્ટવેરમાં અમલીકરણ \\ \hline
\end{tabulary}

\begin{center}
\begin{circuitikz}[
    level 1/.style = {sibling distance=3.5cm},
    level 2/.style = {sibling distance=1.5cm},
    edge from parent/.style = {draw, -latex},
    every node/.style = {rectangle, draw, rounded corners, align=center, font=\scriptsize}
]
    \node {Filters}
        child { node {Freq Response}
            child { node {LP/HP} }
            child { node {BP/BS} }
        }
        child { node {Components}
            child { node {Passive} }
            child { node {Active} }
        }
        child { node {Design}
            child { node {Constant-k} }
            child { node {m-derived} }
        }
        child { node {Technology}
            child { node {LC/Crystal} }
            child { node {Digital} }
        };
\end{circuitikz}
\captionof{figure}{Filter Classification}
\end{center}
\end{solutionbox}

\begin{mnemonicbox}
\mnemonic{FLAC: ફિલ્ટર્સ: લો-પાસ, એક્ટિવ, કન્સ્ટન્ટ-k}
\end{mnemonicbox}

\questionmarks{5(c)}{7}{constant K લો પાસ ફિલ્ટર સમજાવો.}

\begin{solutionbox}
\begin{tabulary}{\linewidth}{|L|L|}
\hline
\textbf{વિભાવના} & \textbf{વર્ણન} \\ \hline
\textbf{વ્યાખ્યા} & ફિલ્ટર જેમાં ઇમ્પીડન્સ પ્રોડક્ટ $Z_1Z_2 = k^2$ (અચળ) દરેક ફ્રીક્વન્સી પર \\ \hline
\textbf{સર્કિટ પ્રકાર} & T-સેક્શન અને $\pi$-સેક્શન \\ \hline
\textbf{T-સેક્શન ઘટકો} & સિરીઝ ઇન્ડક્ટર્સ ($L/2$) અને શન્ટ કેપેસિટર ($C$) \\ \hline
\textbf{π-સેક્શન ઘટકો} & સિરીઝ ઇન્ડક્ટર ($L$) અને શન્ટ કેપેસિટર્સ ($C/2$) \\ \hline
\textbf{કટઓફ ફ્રીક્વન્સી} & $f_c = 1/(\pi\sqrt{LC})$ \\ \hline
\textbf{કેરેક્ટરિસ્ટિક ઇમ્પીડન્સ} & $R_0 = \sqrt{L/C}$ \\ \hline
\end{tabulary}

\begin{center}
\begin{circuitikz}[auto]
    \node at (0,3) {(a) T-section};
    \draw (0,2) to [L, l=$L/2$] (2,2) to [L, l=$L/2$] (4,2);
    \draw (2,2) to [C, l=$C$] (2,0);
    \draw (0,0) -- (4,0);

    \node at (6,3) {(b) $\pi$-section};
    \draw (6,2) -- (7,2) to [L, l=$L$] (9,2) -- (10,2);
    \draw (7,2) to [C, l=$C/2$] (7,0);
    \draw (9,2) to [C, l=$C/2$] (9,0);
    \draw (6,0) -- (10,0);
\end{circuitikz}
\captionof{figure}{Constant-k Low Pass Filter}
\end{center}

કન્સ્ટન્ટ-k લો પાસ ફિલ્ટરના લક્ષણો:
\begin{itemize}
    \item કટઓફ ફ્રીક્વન્સી: $f_c = \frac{1}{\pi\sqrt{LC}}$
    \item ડિઝાઇન ઇમ્પીડન્સ: $R_0 = \sqrt{\frac{L}{C}}$
    \item પાસ બેન્ડ: 0 થી $f_c$
    \item એટેન્યુએશન બેન્ડ: $f_c$ ઉપર
    \item પાસ બેન્ડથી સ્ટોપ બેન્ડ સુધી ક્રમશઃ સંક્રમણ
\end{itemize}
\end{solutionbox}

\begin{mnemonicbox}
\mnemonic{CLPT: કન્સ્ટન્ટ-k લો પાસ નીડ્સ T-સેક્શન}
\end{mnemonicbox}

\questionmarks{5(a OR)}{3}{400 $\Omega$ ના લોડ પ્રતિકાર સાથે 1.5 KHz ની કટ-ઓફ આવર્તન ધરાવતા T વિભાગ સાથે ઉચ્ચ પાસ ફિલ્ટર ડિઝાઇન કરો.}

\begin{solutionbox}
\begin{center}
\begin{circuitikz}[auto]
    \draw (0,2) to [C, l=$2C$] (2,2) to [C, l=$2C$] (4,2);
    \draw (2,2) to [L, l=$L$] (2,0);
    \draw (0,0) -- (4,0);
    \node at (2,-1) {High Pass T-Section};
\end{circuitikz}
\captionof{figure}{High Pass Filter Design}
\end{center}

\begin{tabulary}{\linewidth}{|L|L|L|}
\hline
\textbf{પેરામીટર} & \textbf{સૂત્ર} & \textbf{પરિણામ} \\ \hline
\textbf{ડિઝાઇન ઇમ્પીડન્સ ($R_0$)} & $R_0 = \text{લોડ રેઝિસ્ટન્સ}$ & 400 $\Omega$ \\ \hline
\textbf{કટઓફ ફ્રીક્વન્સી ($f_c$)} & $f_c = \text{આપેલ}$ & 1.5 kHz \\ \hline
\textbf{ઇન્ડક્ટર (L)} & $L = \frac{R_0}{2\pi f_c}$ & 42.44 mH \\ \hline
\textbf{કેપેસિટર (C)} & $C = \frac{1}{2\pi f_c R_0}$ & 0.265 $\mu$F \\ \hline
\end{tabulary}
(Note: MDX uses $2\pi$ in calculations. Fidelity maintained.)
ફોર્મ્યુલા: $L = R_0/2\pi f_c$, $C = 1/(2\pi f_c R_0)$.
ગણતરી: $400/(2\pi \times 1500) = 42.44$ mH.
ગણતરી: $1/(2\pi \times 1500 \times 400) = 0.265 \mu$F.

\end{solutionbox}

\begin{mnemonicbox}
\mnemonic{HCL: હાઇ-પાસ નીડ્સ કેપેસિટર એન્ડ ઇન્ડક્ટર}
\end{mnemonicbox}

\questionmarks{5(b OR)}{4}{એટેન્યુએટરનું વર્ગીકરણ આપો.}

\begin{solutionbox}
\begin{tabulary}{\linewidth}{|L|L|L|}
\hline
\textbf{વર્ગીકરણ} & \textbf{પ્રકારો} & \textbf{લક્ષણો} \\ \hline
\textbf{કન્ફિગરેશન આધારિત} & - T-એટેન્યુએટર \newline - $\pi$-એટેન્યુએટર \newline - બ્રિજ્ડ-T \newline - લેટિસ & - સિરીઝ-શન્ટ-સિરીઝ \newline - શન્ટ-સિરીઝ-શન્ટ \newline - બેલેન્સ્ડ બ્રિજ \newline - બેલેન્સ્ડ નેટવર્ક \\ \hline
\textbf{સિમેટ્રી આધારિત} & - સિમેટ્રિકલ \newline - એસિમેટ્રિકલ & - સમાન ઇમ્પીડન્સ \newline - અસમાન ઇમ્પીડન્સ \\ \hline
\textbf{નિયંત્રણ આધારિત} & - ફિક્સ્ડ \newline - વેરિએબલ \newline - પ્રોગ્રામેબલ & - અચળ એટેન્યુએશન \newline - સમાયોજ્ય એટેન્યુએશન \newline - ડિજિટલી નિયંત્રિત \\ \hline
\textbf{ટેકનોલોજી આધારિત} & - રેઝિસ્ટિવ \newline - રિએક્ટિવ \newline - એક્ટિવ & - રેઝિસ્ટરનો ઉપયોગ \newline - રિએક્ટન્સનો ઉપયોગ \newline - એક્ટિવ ડિવાઇસનો ઉપયોગ \\ \hline
\end{tabulary}
\end{solutionbox}


\begin{mnemonicbox}
\mnemonic{CAST: કન્ફિગરેશન, એડજસ્ટેબલ, સિમેટ્રી, ટેકનોલોજી}
\end{mnemonicbox}

\questionmarks{5(c) OR}{7}{constant K હાઇ પાસ ફિલ્ટર સમજાવો.}

\begin{solutionbox}
\begin{tabulary}{\linewidth}{|L|L|}
\hline
\textbf{વિભાવના} & \textbf{વર્ણન} \\ \hline
\textbf{વ્યાખ્યા} & કટઓફ ઉપરની ફ્રીક્વન્સી પસાર કરતું ફિલ્ટર, જેમાં $Z_1Z_2 = k^2$ (અચળ) \\ \hline
\textbf{સર્કિટ પ્રકાર} & T-સેક્શન અને $\pi$-સેક્શન \\ \hline
\textbf{T-સેક્શન ઘટકો} & સિરીઝ કેપેસિટર્સ ($C/2$) અને શન્ટ ઇન્ડક્ટર ($L$) \\ \hline
\textbf{$\pi$-સેક્શન ઘટકો} & સિરીઝ કેપેસિટર ($C$) અને શન્ટ ઇન્ડક્ટર્સ ($L/2$) \\ \hline
\textbf{કટઓફ ફ્રીક્વન્સી} & $f_c = 1/(\pi\sqrt{LC})$ \\ \hline
\textbf{કેરેક્ટરિસ્ટિક ઇમ્પીડન્સ} & $R_0 = \sqrt{L/C}$ \\ \hline
\end{tabulary}

\begin{center}
\begin{circuitikz}[auto]
    \node at (0,3) {(a) T-section};
    \draw (0,2) to [C, l=$C/2$] (2,2) to [C, l=$C/2$] (4,2);
    \draw (2,2) to [L, l=$L$] (2,0);
    \draw (0,0) -- (4,0);

    \node at (6,3) {(b) $\pi$-section};
    \draw (6,2) -- (7,2) to [C, l=$C$] (9,2) -- (10,2);
    \draw (7,2) to [L, l=$L/2$] (7,0);
    \draw (9,2) to [L, l=$L/2$] (9,0);
    \draw (6,0) -- (10,0);
\end{circuitikz}
\captionof{figure}{Constant-k High Pass Filter}
\end{center}

કન્સ્ટન્ટ-k હાઇ પાસ ફિલ્ટરના લક્ષણો:
\begin{itemize}
    \item કટઓફ ફ્રીક્વન્સી: $f_c = \frac{1}{\pi\sqrt{LC}}$
    \item ડિઝાઇન ઇમ્પીડન્સ: $R_0 = \sqrt{\frac{L}{C}}$
    \item પાસ બેન્ડ: $f_c$ ઉપર
    \item એટેન્યુએશન બેન્ડ: 0 થી $f_c$
    \item પાસ બેન્ડથી સ્ટોપ બેન્ડ સુધી ક્રમશઃ સંક્રમણ
    \item ઘટક મૂલ્યો લો પાસ ફિલ્ટરના ડ્યુઅલ છે (L અને C જગ્યા બદલે છે)
\end{itemize}
\end{solutionbox}

\begin{mnemonicbox}
\mnemonic{CHTS: કન્સ્ટન્ટ-k હાઇ-પાસ યુઝિસ T-સેક્શન}
\end{mnemonicbox}

\end{document}
