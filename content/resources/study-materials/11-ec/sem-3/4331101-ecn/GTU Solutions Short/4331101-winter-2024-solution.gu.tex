\documentclass{article}

% content/resources/templates/preamble.tex
\usepackage[margin=0.6in]{geometry}
\author{Milav Dabgar}
\usepackage{amsmath,amssymb,amsthm}
\usepackage{booktabs}
\usepackage{multirow}
\usepackage{xcolor}
\usepackage{tcolorbox}
\tcbuselibrary{breakable,skins}
\usepackage[colorlinks=true,linkcolor=blue]{hyperref}
\usepackage{titlesec}
\usepackage{enumitem}
\usepackage{tikz}
\usepackage{pgfplots}
\usepackage{circuitikz}
\usepackage[version=4]{mhchem}
\usepackage{longtable}
\usepackage{array}
\usepackage{float}
\usepackage{caption}
\usepackage{listings}

\lstset{
  basicstyle=\small\ttfamily,
  breaklines=true,
  breakatwhitespace=false,
  postbreak=\mbox{\textcolor{red}{$\hookrightarrow$}\space},
  float=false,
  numbers=left,
  numberstyle=\tiny\color{gray},
  numbersep=10pt,
  xleftmargin=2em,
  keywordstyle=\color{blue},
  commentstyle=\color{green!60!black},
  stringstyle=\color{purple},
  backgroundcolor=\color{gray!5},
  showstringspaces=false,
  tabsize=2,
  captionpos=b,
  keepspaces=true,
  columns=flexible
}

\pgfplotsset{compat=1.18}
\usetikzlibrary{shapes,arrows,positioning,calc,patterns,decorations.pathmorphing,decorations.markings,arrows.meta}

% Color scheme
\definecolor{headcolor}{RGB}{0,102,204}
\definecolor{keycolor}{RGB}{220,20,60}
\definecolor{solutioncolor}{RGB}{34,139,34}
\definecolor{mnemoniccolor}{RGB}{148,0,211}
\definecolor{codecolor}{RGB}{0,0,100}

% Spacing
\setlength{\parskip}{3pt}
\setlist[itemize]{nosep}
\setlist[enumerate]{nosep}

% Title formatting
\titleformat{\section}{\Large\bfseries\color{headcolor}}{\thesection}{1em}{}
\titleformat{\subsection}{\large\bfseries\color{headcolor}}{\thesubsection}{1em}{}

% Pandoc tightlist compatibility
\providecommand{\tightlist}{%
  \setlength{\itemsep}{0pt}\setlength{\parskip}{0pt}}

% Pandoc longtable compatibility
\newcounter{none}
\def\thenone{}


% content/resources/templates/gujarati-boxes.tex
\usepackage{fontspec}
\usepackage{polyglossia}

% Set Gujarati as main language (document is primarily in Gujarati)
% Note: gloss-gujarati.ldf doesn't exist in polyglossia, but it will use hyphenation patterns
\setdefaultlanguage{gujarati}
\setotherlanguage{english}

% Configure Gujarati font properly
% Use Language=Default to prevent polyglossia from trying to add language-specific features
% that don't exist for Gujarati, which causes "empty feature" warnings
\newfontfamily\gujaratifont[Script=Gujarati,AutoFakeBold=2.5,AutoFakeSlant=0.3]{Noto Sans Gujarati}
\setmainfont[Script=Gujarati,AutoFakeBold=2.5,AutoFakeSlant=0.3]{Noto Sans Gujarati}
% Use Noto Sans Gujarati for monospace to support Gujarati in text
\setmonofont[Scale=0.9]{Noto Sans Gujarati}

% Configure English to use the same font
\newfontfamily\englishfont[Script=Gujarati,AutoFakeBold=2.5,AutoFakeSlant=0.3]{Noto Sans Gujarati}

% Translations for polyglossia
\gappto\captionsgujarati{
  \renewcommand{\tablename}{કોષ્ટક}
  \renewcommand{\figurename}{આકૃતિ}
}

% Helper for TikZ nodes to ensure Gujarati font
\newcommand{\gu}[1]{{\gujaratifont #1}}

% Custom environments
\newtcolorbox{solutionbox}{
    breakable,
    enhanced,
    colback=solutioncolor!5!white,
    colframe=solutioncolor!75!black,
    fonttitle=\bfseries,
    title=જવાબ
}

\newtcolorbox{solutionboxnobreak}{
 colback=solutioncolor!5!white,
 colframe=solutioncolor!75!black,
 fonttitle=\bfseries,
 title=જવાબ
}

\newtcolorbox{keyformula}{
 breakable,
 enhanced,
 colback=keycolor!5!white,
 colframe=keycolor!75!black,
 fonttitle=\bfseries,
 title=રાસાયણિક સમીકરણ/સૂત્ર
}

\newtcolorbox{mnemonicbox}{
 breakable,
 enhanced,
 colback=mnemoniccolor!5!white,
 colframe=mnemoniccolor!75!black,
 fonttitle=\bfseries,
 title=મેમરી ટ્રીક
}


% Custom commands for GTU solutions
% This file defines semantic commands for consistent formatting

% Question command with automatic formatting
\newcommand{\question}[2]{%
  \section*{Question #1}%
  \textbf{#2}%
}

% OR question variant
\newcommand{\questionor}[2]{%
  \section*{Question #1 OR}%
  \textbf{#2}%
}

% Proper table environment with caption
\newenvironment{answertable}[1]{%
  \begin{table}[htbp]
  \centering
  \caption{#1}
}{%
  \end{table}
}

% Proper figure environment for diagrams
\newenvironment{answerdiagram}[1]{%
  \begin{figure}[htbp]
  \centering
  \caption{#1}
}{%
  \end{figure}
}

% Semantic markup for key terms
\newcommand{\keyword}[1]{\textbf{#1}}
\newcommand{\code}[1]{\texttt{#1}}
\newcommand{\classname}[1]{\texttt{#1}}
\newcommand{\methodname}[1]{\texttt{#1}}

% Proper quotation marks
\newcommand{\mnemonic}[1]{``#1''}


\title{ઇલેક્ટ્રોનિક સર્કિટ્સ અને નેટવર્ક્સ (4331101) - શિયાળુ 2024 સોલ્યુશન}
\date{May 20, 2024}

\begin{document}
\maketitle

\section*{પ્રશ્ન 1(a) [3 ગુણ]}
\questionmarks{1(a)}{3}{ગુણ}

\textbf{ઇલેક્ટ્રોનીક નેટવર્ક માટે (i) નોડ (ii) બ્રાંચ અને (iii) લૂપ ની વ્યાખ્યા આપો.}

\begin{solutionbox}
\textbf{જવાબ}:

\textbf{નોડ}: 
\begin{itemize}
    \item \textbf{જંક્શન પોઈન્ટ} જ્યાં બે અથવા વધુ બ્રાંચ નેટવર્કમાં મળે છે
    \item એવા બિંદુઓ જ્યાં ઘટકો જોડાયેલા હોય છે
    \item નોડ પર બધી બ્રાંચોનો કરંટ સરવાળો શૂન્ય થાય છે
\end{itemize}

\textbf{બ્રાંચ}: 
\begin{itemize}
    \item \textbf{સિંગલ ઘટક} (R, L, અથવા C) અથવા બે નોડ્સને જોડતો પાથ
    \item દરેક બ્રાંચમાં એક ચોક્કસ કરંટ વહે છે
    \item એક્ટીવ બ્રાંચમાં સોર્સ હોય છે; પેસિવ બ્રાંચમાં R, L, C હોય છે
\end{itemize}

\textbf{લૂપ}: 
\begin{itemize}
    \item નેટવર્કમાં જોડાયેલા બ્રાંચોથી બનતો \textbf{બંધ પાથ}
    \item કોઈ નોડ એક કરતાં વધુ વખત આવતું નથી
    \item નેટવર્ક ઉકેલવા માટે લૂપ એનાલિસિસમાં વપરાય છે
\end{itemize}

\begin{mnemonicbox}
"NBL: નોડ્સ જોડાય, બ્રાંચેસ કનેક્ટ, લૂપ્સ સર્કલ"
\end{mnemonicbox}
\end{solutionbox}

\section*{પ્રશ્ન 1(b) [4 ગુણ]}
\questionmarks{1(b)}{4}{ગુણ}

\textbf{200 $\Omega$, 300 $\Omega$ અને 500 $\Omega$ ના રેઝીસ્ટર 100 V ના સપ્લાય સાથે પેરેલલમાં જોડાયેલા છે. તો (i) દરેક રેઝીસ્ટરમાંથી પસાર થતો કરંટ તથા કુલ કરંટ (ii) ઇક્વીવેલન્ટ રેઝીસ્ટર શોધો.}

\begin{solutionbox}
\textbf{જવાબ}:

\textbf{ગણતરીઓનું કોષ્ટક:}

\begin{tabulary}{\linewidth}{@{}|L|L|L|L|@{}}
    \hline
    \textbf{પેરામીટર} & \textbf{ફોર્મ્યુલા} & \textbf{ગણતરી} & \textbf{પરિણામ} \\
    \hline
    $I_1$ ($200\Omega$) & $I = V/R$ & $100V/200\Omega$ & $0.5A$ \\
    \hline
    $I_2$ ($300\Omega$) & $I = V/R$ & $100V/300\Omega$ & $0.333A$ \\
    \hline
    $I_3$ ($500\Omega$) & $I = V/R$ & $100V/500\Omega$ & $0.2A$ \\
    \hline
    $I_{(total)}$ & $I_1+I_2+I_3$ & $0.5+0.333+0.2$ & $1.033A$ \\
    \hline
    $R_{(eq)}$ & $1/R_{(eq)} = 1/R_1+1/R_2+1/R_3$ & $1/200+1/300+1/500$ & $96.77\Omega$ \\
    \hline
\end{tabulary}

\begin{mnemonicbox}
"પેરેલલ પાથ કરંટને અવરોધના વ્યસ્ત પ્રમાણમાં વહેંચે છે"
\end{mnemonicbox}
\end{solutionbox}

\section*{પ્રશ્ન 1(c) [7 ગુણ]}
\questionmarks{1(c)}{7}{ગુણ}

\textbf{કેપેસીટર માટે સિરીઝ અને પેરેલલ જોડાણ સમજાવો.}

\begin{solutionbox}
\textbf{જવાબ}:

\textbf{સિરીઝમાં કેપેસીટર:}

\begin{center}
\begin{circuitikz}
    \draw (0,0) to[C, l=$C_1$] (2,0) to[C, l=$C_2$] (4,0) to[C, l=$C_3$] (6,0);
    \node[left] at (0,0) {$+$};
    \node[right] at (6,0) {$-$};
\end{circuitikz}
\captionof{figure}{સિરીઝ કેપેસીટરો}
\end{center}

\textbf{કોષ્ટક: સિરીઝ કેપેસીટરોની વિશેષતાઓ}

\begin{tabulary}{\linewidth}{@{}|L|L|L|@{}}
    \hline
    \textbf{વિશેષતા} & \textbf{ફોર્મ્યુલા} & \textbf{વર્ણન} \\
    \hline
    ઇક્વિવેલન્ટ કેપેસિટન્સ & $1/C_{(eq)} = 1/C_1 + 1/C_2 + 1/C_3$ & હંમેશા નાનામાં નાના કેપેસિટર કરતાં નાનું \\
    \hline
    ચાર્જ & $Q = Q_1 = Q_2 = Q_3$ & બધા કેપેસિટર પર સરખો \\
    \hline
    વોલ્ટેજ & $V = V_1 + V_2 + V_3$ & $1/C$ ના રેશિયો પ્રમાણે વહેંચાય છે \\
    \hline
    ઊર્જા & $E = CV^2/2$ & કેપેસિટર્સમાં વહેંચાયેલી \\
    \hline
\end{tabulary}

\textbf{પેરેલલમાં કેપેસીટર:}

\begin{center}
\begin{circuitikz}
    \draw (0,3) to[short, o-] (1,3) -- (5,3) to[short, -o] (6,3);
    \draw (0,0) to[short, o-] (1,0) -- (5,0) to[short, -o] (6,0);
    \draw (2,3) to[C, l=$C_1$] (2,0);
    \draw (3,3) to[C, l=$C_2$] (3,0);
    \draw (4,3) to[C, l=$C_3$] (4,0);
    \node[left] at (0,3) {$+$};
    \node[left] at (0,0) {$-$};
\end{circuitikz}
\captionof{figure}{પેરેલલ કેપેસીટરો}
\end{center}

\textbf{કોષ્ટક: પેરેલલ કેપેસીટરોની વિશેષતાઓ}

\begin{tabulary}{\linewidth}{@{}|L|L|L|@{}}
    \hline
    \textbf{વિશેષતા} & \textbf{ફોર્મ્યુલા} & \textbf{વર્ણન} \\
    \hline
    ઇક્વિવેલન્ટ કેપેસિટન્સ & $C_{(eq)} = C_1 + C_2 + C_3$ & વ્યક્તિગત કેપેસિટન્સનો સરવાળો \\
    \hline
    ચાર્જ & $Q = Q_1 + Q_2 + Q_3$ & $C$ ની કિંમત અનુસાર વહેંચાય છે \\
    \hline
    વોલ્ટેજ & $V = V_1 = V_2 = V_3$ & બધા કેપેસિટર પર સરખો \\
    \hline
    ઊર્જા & $E = CV^2/2$ & વ્યક્તિગત ઊર્જાનો સરવાળો \\
    \hline
\end{tabulary}

\begin{mnemonicbox}
"સિરીઝ કેપ્સમાં વ્યસ્ત સરવાળો, પેરેલલ કેપ્સમાં સીધો સરવાળો"
\end{mnemonicbox}
\end{solutionbox}

\section*{પ્રશ્ન 1(c) OR [7 ગુણ]}
\questionmarks{1(c) OR}{7}{ગુણ}

\textbf{ઇન્ડક્ટર માટે સિરીઝ અને પેરેલલ જોડાણ સમજાવો.}

\begin{solutionbox}
\textbf{જવાબ}:

\textbf{સિરીઝમાં ઇન્ડક્ટર:}

\begin{center}
\begin{circuitikz}
    \draw (0,0) to[L, l=$L_1$] (2,0) to[L, l=$L_2$] (4,0) to[L, l=$L_3$] (6,0);
    \node[left] at (0,0) {$+$};
    \node[right] at (6,0) {$-$};
\end{circuitikz}
\captionof{figure}{સિરીઝ ઇન્ડક્ટરો}
\end{center}

\textbf{કોષ્ટક: સિરીઝ ઇન્ડક્ટરોની વિશેષતાઓ}

\begin{tabulary}{\linewidth}{@{}|L|L|L|@{}}
    \hline
    \textbf{વિશેષતા} & \textbf{ફોર્મ્યુલા} & \textbf{વર્ણન} \\
    \hline
    ઇક્વિવેલન્ટ ઇન્ડક્ટન્સ & $L_{(eq)} = L_1 + L_2 + L_3$ & વ્યક્તિગત ઇન્ડક્ટન્સનો સરવાળો \\
    \hline
    કરંટ & $I = I_1 = I_2 = I_3$ & બધા ઇન્ડક્ટર પર સરખો \\
    \hline
    વોલ્ટેજ & $V = V_1 + V_2 + V_3$ & $L$ ના રેશિયો અનુસાર વહેંચાય છે \\
    \hline
    ઊર્જા & $E = LI^2/2$ & વ્યક્તિગત ઊર્જાનો સરવાળો \\
    \hline
\end{tabulary}

\textbf{પેરેલલમાં ઇન્ડક્ટર:}

\begin{center}
\begin{circuitikz}
    \draw (0,3) to[short, o-] (1,3) -- (5,3) to[short, -o] (6,3);
    \draw (0,0) to[short, o-] (1,0) -- (5,0) to[short, -o] (6,0);
    \draw (2,3) to[L, l=$L_1$] (2,0);
    \draw (3,3) to[L, l=$L_2$] (3,0);
    \draw (4,3) to[L, l=$L_3$] (4,0);
    \node[left] at (0,3) {$+$};
    \node[left] at (0,0) {$-$};
\end{circuitikz}
\captionof{figure}{પેરેલલ ઇન્ડક્ટરો}
\end{center}

\textbf{કોષ્ટક: પેરેલલ ઇન્ડક્ટરોની વિશેષતાઓ}

\begin{tabulary}{\linewidth}{@{}|L|L|L|@{}}
    \hline
    \textbf{વિશેષતા} & \textbf{ફોર્મ્યુલા} & \textbf{વર્ણન} \\
    \hline
    ઇક્વિવેલન્ટ ઇન્ડક્ટન્સ & $1/L_{(eq)} = 1/L_1 + 1/L_2 + 1/L_3$ & હંમેશા નાનામાં નાના ઇન્ડક્ટર કરતાં નાનું \\
    \hline
    કરંટ & $I = I_1 + I_2 + I_3$ & $1/L$ ના રેશિયો અનુસાર વહેંચાય છે \\
    \hline
    વોલ્ટેજ & $V = V_1 = V_2 = V_3$ & બધા ઇન્ડક્ટર પર સરખો \\
    \hline
    ઊર્જા & $E = LI^2/2$ & ઇન્ડક્ટરોમાં વહેંચાયેલી \\
    \hline
\end{tabulary}

\begin{mnemonicbox}
"સિરીઝ ઇન્ડક્ટરોમાં સીધો સરવાળો, પેરેલલ ઇન્ડક્ટરોમાં વ્યસ્ત સરવાળો"
\end{mnemonicbox}
\end{solutionbox}

\section*{પ્રશ્ન 2(a) [3 ગુણ]}
\questionmarks{2(a)}{3}{ગુણ}

\textbf{નેટવર્ક એલીમેન્ટને વર્ગીકૃત કરો.}

\begin{solutionbox}
\textbf{જવાબ}:

\textbf{કોષ્ટક: નેટવર્ક એલીમેન્ટનું વર્ગીકરણ}

\begin{tabulary}{\linewidth}{@{}|L|L|L|@{}}
    \hline
    \textbf{શ્રેણી} & \textbf{પ્રકારો} & \textbf{ઉદાહરણો} \\
    \hline
    \textbf{એક્ટિવ vs પેસિવ} & એક્ટિવ & વોલ્ટેજ/કરંટ સોર્સ, ટ્રાન્ઝિસ્ટર \\
    \cline{2-3}
     & પેસિવ & રેઝિસ્ટર, કેપેસિટર, ઇન્ડક્ટર \\
    \hline
    \textbf{લિનિયર vs નોન-લિનિયર} & લિનિયર & રેઝિસ્ટર, આદર્શ સોર્સ \\
    \cline{2-3}
     & નોન-લિનિયર & ડાયોડ, ટ્રાન્ઝિસ્ટર \\
    \hline
    \textbf{બાઇલેટરલ vs યુનિલેટરલ} & બાઇલેટરલ & રેઝિસ્ટર, કેપેસિટર, ઇન્ડક્ટર \\
    \cline{2-3}
     & યુનિલેટરલ & ડાયોડ, ટ્રાન્ઝિસ્ટર \\
    \hline
    \textbf{લમ્પ્ડ vs ડિસ્ટ્રિબ્યુટેડ} & લમ્પ્ડ & ડિસક્રીટ R, L, C ઘટકો \\
    \cline{2-3}
     & ડિસ્ટ્રિબ્યુટેડ & ટ્રાન્સમિશન લાઇન \\
    \hline
\end{tabulary}

\begin{mnemonicbox}
"ALBU: એક્ટિવ/પેસિવ, લિનિયર/નોન-લિનિયર, બાઇલેટરલ/યુનિલેટરલ, લમ્પ્ડ/ડિસ્ટ્રિબ્યુટેડ"
\end{mnemonicbox}
\end{solutionbox}

\section*{પ્રશ્ન 2(b) [4 ગુણ]}
\questionmarks{2(b)}{4}{ગુણ}

\textbf{10, 30 અને 70 ohms ના રેઝીસ્ટર સ્ટારમાં કનેક્ટ કરેલા છે. ડેલ્ટા કનેક્શનનાં ઇક્વીવેલન્ટ રેઝીસ્ટર શોધો.}

\begin{solutionbox}
\textbf{જવાબ}:

\textbf{આકૃતિ: સ્ટાર થી ડેલ્ટા રૂપાંતરણ}

\begin{center}
\begin{circuitikz}[scale=0.8]
    % Star
    \node at (0,3) {Star};
    \draw (0,2) to[R, l=$R_1(10\Omega)$] (0,0);
    \draw (0,0) to[R, l=$R_2(30\Omega)$] (-1.73, -1);
    \draw (0,0) to[R, l=$R_3(70\Omega)$] (1.73, -1);
    \node at (0, 0.3) {0};
    \node at (0, 2.3) {1};
    \node at (-1.9, -1.2) {2};
    \node at (1.9, -1.2) {3};

    \draw[->, thick] (2.5, 0.5) -- (4.5, 0.5);

    % Delta
    \node at (7,3) {Delta};
    \draw (7,2) to[R, l=$R_{12}$] (5.27, -1) to[R, l=$R_{23}$] (8.73, -1) to[R, l=$R_{31}$] (7,2);
    \node at (7, 2.3) {1};
    \node at (5.1, -1.2) {2};
    \node at (8.9, -1.2) {3};
\end{circuitikz}
\captionof{figure}{સ્ટાર થી ડેલ્ટા રૂપાંતરણ}
\end{center}

\textbf{કોષ્ટક: સ્ટાર-ડેલ્ટા રૂપાંતરણ ફોર્મ્યુલા અને ગણતરીઓ}

\begin{tabulary}{\linewidth}{@{}|L|L|L|L|@{}}
    \hline
    \textbf{ડેલ્ટા રેઝીસ્ટન્સ} & \textbf{ફોર્મ્યુલા} & \textbf{ગણતરી} & \textbf{પરિણામ} \\
    \hline
    $R_{12}$ & $(R_1R_2+R_2R_3+R_3R_1)/R_3$ & $(10\times30+30\times70+70\times10)/70$ & $47.14\Omega$ \\
    \hline
    $R_{23}$ & $(R_1R_2+R_2R_3+R_3R_1)/R_1$ & $(10\times30+30\times70+70\times10)/10$ & $330\Omega$ \\
    \hline
    $R_{31}$ & $(R_1R_2+R_2R_3+R_3R_1)/R_2$ & $(10\times30+30\times70+70\times10)/30$ & $110\Omega$ \\
    \hline
\end{tabulary}

\begin{mnemonicbox}
"સ્ટાર-ડેલ્ટા: ગુણાકારનો સરવાળો વિરુદ્ધ રેઝિસ્ટર ઉપર"
\end{mnemonicbox}
\end{solutionbox}

\section*{પ્રશ્ન 2(c) [7 ગુણ]}
\questionmarks{2(c)}{7}{ગુણ}

\textbf{$\pi$ નેટવર્ક સમજાવો.}

\begin{solutionbox}
\textbf{જવાબ}:

\textbf{આકૃતિ: $\pi$ (પાઈ) નેટવર્ક}

\begin{center}
\begin{circuitikz}
    \draw (0,2) to[short, o-] (1,2) to[short] (1,3) to[generic, l=$Z_1$] (4,3) to[short] (4,2) to[short, -o] (5,2);
    \draw (1,2) to[generic, l=$Z_3$] (1,0);
    \draw (4,2) to[generic, l=$Z_2$] (4,0);
    \draw (0,0) to[short, o-] (5,0) to[short, -o] (5,0);
    \node at (0.5, 2.5) {Node 1};
    \node at (4.5, 2.5) {Node 2};
\end{circuitikz}
\captionof{figure}{$\pi$ નેટવર્ક}
\end{center}

\textbf{કોષ્ટક: $\pi$ નેટવર્ક વિશેષતાઓ}

\begin{tabulary}{\linewidth}{@{}|L|L|@{}}
    \hline
    \textbf{પેરામીટર} & \textbf{વર્ણન} \\
    \hline
    \textbf{સ્ટ્રક્ચર} & બે શન્ટ ઇમ્પિડન્સ ($Z_3$, $Z_2$) અને એક સિરીઝ ઇમ્પિડન્સ ($Z_1$) \\
    \hline
    \textbf{ટ્રાન્સમિશન પેરામીટર્સ} & $A = 1 + Z_1/Z_2$, $B = Z_1$, $C = 1/Z_2 + 1/Z_3 + Z_1/(Z_2Z_3)$, $D = 1 + Z_1/Z_3$ \\
    \hline
    \textbf{ઇમ્પિડન્સ પેરામીટર્સ} & $Z_{11} = Z_1 + Z_3$, $Z_{12} = Z_1$, $Z_{21} = Z_1$, $Z_{22} = Z_1 + Z_2$ \\
    \hline
    \textbf{ઇમેજ ઇમ્પિડન્સ} & $Z_{0\pi} = \sqrt{Z_1Z_2Z_3/(Z_2+Z_3)}$ \\
    \hline
    \textbf{એપ્લિકેશન} & મેચિંગ નેટવર્ક, ફિલ્ટર, એટેન્યુએટર \\
    \hline
    \textbf{રૂપાંતરણ} & T-નેટવર્કમાં રૂપાંતરિત કરી શકાય છે \\
    \hline
\end{tabulary}

\begin{mnemonicbox}
"$\pi$ ના બે પગ નીચે, એક શાખા આડી"
\end{mnemonicbox}
\end{solutionbox}

\section*{પ્રશ્ન 2(a) OR [3 ગુણ]}
\questionmarks{2(a) OR}{3}{ગુણ}

\textbf{નેટવર્કનાં પ્રકારો જણાવો.}

\begin{solutionbox}
\textbf{જવાબ}:

\textbf{કોષ્ટક: નેટવર્કના પ્રકારો}

\begin{tabulary}{\linewidth}{@{}|L|L|@{}}
    \hline
    \textbf{શ્રેણી} & \textbf{પ્રકારો} \\
    \hline
    \textbf{લિનિયારિટી આધારિત} & લિનિયર નેટવર્ક, નોન-લિનિયર નેટવર્ક \\
    \hline
    \textbf{ઘટકો આધારિત} & પેસિવ નેટવર્ક, એક્ટિવ નેટવર્ક \\
    \hline
    \textbf{પેરામીટર આધારિત} & ટાઇમ-વેરિયન્ટ, ટાઇમ-ઇન્વેરિયન્ટ નેટવર્ક \\
    \hline
    \textbf{કોન્ફિગરેશન આધારિત} & T-નેટવર્ક, $\pi$-નેટવર્ક, લેટિસ નેટવર્ક \\
    \hline
    \textbf{પોર્ટ આધારિત} & વન-પોર્ટ, ટુ-પોર્ટ, મલ્ટિ-પોર્ટ નેટવર્ક \\
    \hline
    \textbf{સિમેટ્રી આધારિત} & સિમેટ્રિકલ, એસિમેટ્રિકલ નેટવર્ક \\
    \hline
    \textbf{રેસિપ્રોસિટી આધારિત} & રેસિપ્રોકલ, નોન-રેસિપ્રોકલ નેટવર્ક \\
    \hline
\end{tabulary}

\begin{mnemonicbox}
"LEPCPS: લિનિયારિટી, એલિમેન્ટ્સ, પેરામીટર્સ, કોન્ફિગરેશન, પોર્ટ્સ, સિમેટ્રી"
\end{mnemonicbox}
\end{solutionbox}

\section*{પ્રશ્ન 2(b) OR [4 ગુણ]}
\questionmarks{2(b) OR}{4}{ગુણ}

\textbf{20, 50 અને 100 ohms ના રેઝીસ્ટર ડેલ્ટામાં કનેક્ટ કરેલા છે. સ્ટાર કનેક્શનનાં ઇક્વીવેલન્ટ રેઝીસ્ટર શોધો.}

\begin{solutionbox}
\textbf{જવાબ}:

\textbf{આકૃતિ: ડેલ્ટા થી સ્ટાર રૂપાંતરણ}

\begin{center}
\begin{circuitikz}[scale=0.8]
    % Delta
    \node at (0,3) {Delta};
    \draw (0,2) to[R, l=$R_{12}(20\Omega)$] (-1.73, -1) to[R, l=$R_{23}(50\Omega)$] (1.73, -1) to[R, l=$R_{31}(100\Omega)$] (0,2);
    \node at (0, 2.3) {1};
    \node at (-1.9, -1.2) {2};
    \node at (1.9, -1.2) {3};
    
    \draw[->, thick] (2.5, 0.5) -- (4.5, 0.5);

    % Star
    \node at (7,3) {Star};
    \draw (7,2) to[R, l=$R_1$] (7,0);
    \draw (7,0) to[R, l=$R_2$] (5.27, -1);
    \draw (7,0) to[R, l=$R_3$] (8.73, -1);
    \node at (7, 0.3) {0};
    \node at (7, 2.3) {1};
    \node at (5.1, -1.2) {2};
    \node at (8.9, -1.2) {3};
\end{circuitikz}
\captionof{figure}{ડેલ્ટા થી સ્ટાર રૂપાંતરણ}
\end{center}

\textbf{કોષ્ટક: ડેલ્ટા-સ્ટાર રૂપાંતરણ ફોર્મ્યુલા અને ગણતરીઓ}

\begin{tabulary}{\linewidth}{@{}|L|L|L|L|@{}}
    \hline
    \textbf{સ્ટાર રેઝીસ્ટન્સ} & \textbf{ફોર્મ્યુલા} & \textbf{ગણતરી} & \textbf{પરિણામ} \\
    \hline
    $R_1$ & $(R_{12}R_{31})/(R_{12}+R_{23}+R_{31})$ & $(20\times100)/(20+50+100)$ & $11.76\Omega$ \\
    \hline
    $R_2$ & $(R_{12}R_{23})/(R_{12}+R_{23}+R_{31})$ & $(20\times50)/(20+50+100)$ & $5.88\Omega$ \\
    \hline
    $R_3$ & $(R_{23}R_{31})/(R_{12}+R_{23}+R_{31})$ & $(50\times100)/(20+50+100)$ & $29.41\Omega$ \\
    \hline
\end{tabulary}

\begin{mnemonicbox}
"ડેલ્ટા-સ્ટાર: આજુબાજુના જોડાનો ગુણાકાર બધાના સરવાળા ઉપર"
\end{mnemonicbox}
\end{solutionbox}

\section*{Question 2(c) OR [7 marks]}
\questionmarks{2(c) OR}{7}{marks}

\textbf{T નેટવર્ક સમજાવો.}

\begin{solutionbox}
\textbf{જવાબ}:

\textbf{આકૃતિ: T નેટવર્ક}

\begin{center}
\begin{circuitikz}
    \draw (0,2) to[short, o-] (1,2) to[generic, l=$Z_1$] (3,2) coordinate(C) to[generic, l=$Z_2$] (5,2) to[short, -o] (6,2);
    \draw (C) to[generic, l=$Z_3$] (3,0) -- (3,0) coordinate(G);
    \draw (0,0) to[short, o-] (6,0) to[short, -o] (6,0);
    \node at (3, -0.3) {Ground};
\end{circuitikz}
\captionof{figure}{T નેટવર્ક}
\end{center}

\textbf{કોષ્ટક: T નેટવર્ક વિશેષતાઓ}

\begin{tabulary}{\linewidth}{@{}|L|L|@{}}
    \hline
    \textbf{પેરામીટર} & \textbf{વર્ણન} \\
    \hline
    \textbf{સ્ટ્રક્ચર} & બે સિરીઝ ઇમ્પિડન્સ ($Z_1$, $Z_2$) અને એક શન્ટ ઇમ્પિડન્સ ($Z_3$) \\
    \hline
    \textbf{ટ્રાન્સમિશન પેરામીટર્સ} & $A = 1 + Z_1/Z_3$, $B = Z_1 + Z_2 + Z_1Z_2/Z_3$, $C = 1/Z_3$, $D = 1 + Z_2/Z_3$ \\
    \hline
    \textbf{ઇમ્પિડન્સ પેરામીટર્સ} & $Z_{11} = Z_1 + Z_3$, $Z_{12} = Z_3$, $Z_{21} = Z_3$, $Z_{22} = Z_2 + Z_3$ \\
    \hline
    \textbf{ઇમેજ ઇમ્પિડન્સ} & $Z_{0T} = \sqrt{Z_1Z_2 + Z_1Z_3 + Z_2Z_3}$ \\
    \hline
    \textbf{એપ્લિકેશન} & મેચિંગ નેટવર્ક, ફિલ્ટર, એટેન્યુએટર \\
    \hline
    \textbf{રૂપાંતરણ} & $\pi$-નેટવર્કમાં રૂપાંતરિત કરી શકાય છે \\
    \hline
\end{tabulary}

\begin{mnemonicbox}
"T ની બે બાહુ આડી, એક પગ નીચે"
\end{mnemonicbox}
\end{solutionbox}

\section*{Question 3(a) [3 marks]}
\questionmarks{3(a)}{3}{marks}

\textbf{Kirchhoff's law સમજાવો.}

\begin{solutionbox}
\textbf{જવાબ}:

\textbf{Kirchhoff's Current Law (KCL):}
\begin{itemize}
    \item નોડમાં \textbf{પ્રવેશતા કરંટનો સરવાળો} તે નોડમાંથી નીકળતા કરંટના સરવાળા બરાબર હોય છે
    \item કોઈપણ નોડ પર કરંટનો બીજગણિતીય સરવાળો શૂન્ય હોય છે
    \item $\sum I = 0$ (પ્રવેશતા કરંટ પોઝિટિવ, નીકળતા નેગેટિવ)
\end{itemize}

\textbf{Kirchhoff's Voltage Law (KVL):}
\begin{itemize}
    \item કોઈપણ બંધ લૂપમાં \textbf{વોલ્ટેજ ડ્રોપનો સરવાળો} શૂન્ય થાય છે
    \item $\sum V = 0$ (વોલ્ટેજ વૃદ્ધિ પોઝિટિવ, ડ્રોપ નેગેટિવ)
    \item ઊર્જાના સંરક્ષણ પર આધારિત છે
\end{itemize}

\textbf{આકૃતિ: Kirchhoff's Laws}

\begin{center}
\begin{circuitikz}[scale=0.8]
    % KCL
    \draw (0,0) node[circ]{} node[right]{Node};
    \draw[<-] (0,0) -- (-1.5, 1) node[left]{$I_1$};
    \draw[->] (0,0) -- (1.5, 1) node[right]{$I_2$};
    \draw[->] (0,0) -- (-1.5, -1) node[left]{$I_3$};
    \draw[<-] (0,0) -- (1.5, -1) node[right]{$I_4$};
    \node at (0, -2) {KCL: $\sum I = 0$};

    % KVL
    \draw (4, -1) to[V, l=$V_1$] (4, 1) to[R, l=$V_2$] (6, 1) to[R, l=$V_3$] (6, -1) -- (4, -1);
    \node at (5, -2) {KVL: $\sum V = 0$};
\end{circuitikz}
\captionof{figure}{Kirchhoff's Laws}
\end{center}

\begin{mnemonicbox}
"કરંટ કન્વર્જ, વોલ્ટેજ વોયેજ ઈન અ લૂપ"
\end{mnemonicbox}
\end{solutionbox}

\section*{Question 3(b) [4 marks]}
\questionmarks{3(b)}{4}{marks}

\textbf{Nodal analysis સમજાવો.}

\begin{solutionbox}
\textbf{જવાબ}:

\textbf{આકૃતિ: નોડલ એનાલિસિસ કોન્સેપ્ટ}

\begin{center}
\captionof{figure}{નોડલ એનાલિસિસ કોન્સેપ્ટ}
\end{center}

\textbf{કોષ્ટક: નોડલ એનાલિસિસ મેથડ}

\begin{tabulary}{\linewidth}{@{}|L|L|@{}}
    \hline
    \textbf{સ્ટેપ} & \textbf{વર્ણન} \\
    \hline
    1. રેફરન્સ નોડ પસંદ કરો & સામાન્ય રીતે ગ્રાઉન્ડ (0V) \\
    \hline
    2. વોલ્ટેજ અસાઇન કરો & બાકીના નોડ વોલ્ટેજને લેબલ કરો ($V_1$, $V_2$, વગેરે) \\
    \hline
    3. KCL લાગુ કરો & દરેક નોન-રેફરન્સ નોડ પર KCL સમીકરણ લખો \\
    \hline
    4. કરંટને એક્સપ્રેસ કરો & ઓહ્મના નિયમનો ઉપયોગ કરીને બ્રાન્ચ કરંટ એક્સપ્રેસ કરો \\
    \hline
    5. સમીકરણો ઉકેલો & સિમલ્ટેનિયસ ઇક્વેશન વડે નોડ વોલ્ટેજ શોધો \\
    \hline
\end{tabulary}

\textbf{ઉદાહરણ: $V_1$ અને $V_2$ વોલ્ટેજવાળા નોડ્સ માટે:}
\begin{itemize}
    \item નોડ 1 પર KCL: $(V_1-0)/R_1 + (V_1-V_2)/R_2 + I_1 = 0$
    \item નોડ 2 પર KCL: $(V_2-V_1)/R_2 + (V_2-0)/R_3 + I_2 = 0$
\end{itemize}

\begin{mnemonicbox}
"નોડલ વોલ્ટેજ એનાલિસિસ માટે KCL જરૂરી છે"
\end{mnemonicbox}
\end{solutionbox}

\section*{Question 3(c) [7 marks]}
\questionmarks{3(c)}{7}{marks}

\textbf{Thevenin's theorem નો ઉપયોગ કરીને ઉપર દશાર્વેલ સર્કિટ માટે 5 $\Omega$ રેઝીસ્ટર માંથી પસાર થતો કરંટ શોધો.}

\begin{solutionbox}
\textbf{જવાબ}:

\textbf{આકૃતિ: મૂળ સર્કિટ અને થેવેનિન ઇક્વિવેલન્ટ}

\begin{center}
\begin{circuitikz}[scale=0.8]
    % Original
    \draw (0,0) node[ground]{} to[short] (6,0);
    \draw (0,0) to[V, l=12V, invert] (0,3) to[R, l=20$\Omega$] (3,3)to[R, l=10$\Omega$] (3,0);
    \draw (3,3) to[short] (6,3);
    \draw (6,0) to[V, l=8V] (6,3);
    \draw (3,0) to[short] (6,0); \node at (3,0) [circ] {};
    % Load tap
    \draw (4.5, 3) to[short] (4.5, 1.5) to[R, l=5$\Omega$] (4.5, 0.5) to[short] (4.5, 0);
    \node at (4.5, 0) [ground]{};
    \node at (3, -0.5) {Original Circuit};

    % Thevenin
    \draw (8, 0) to[V, l=$V_{th}$] (8, 3) to[R, l=$R_{th}$] (10, 3) to[R, l=$R_L(5\Omega)$] (10, 0) -- (8,0);
    \node at (9, -0.5) {Thevenin Equivalent};
\end{circuitikz}
\captionof{figure}{થેવેનિન ઇક્વિવેલન્ટ}
\end{center}

\textbf{થેવેનિન ઇક્વિવેલન્ટ શોધવા માટેના સ્ટેપ્સ:}

\textbf{કોષ્ટક: થેવેનિનના સિદ્ધાંતની પ્રક્રિયા અને ગણતરીઓ}

\begin{tabulary}{\linewidth}{@{}|L|L|L|L|@{}}
    \hline
    \textbf{સ્ટેપ} & \textbf{પ્રક્રિયા} & \textbf{ગણતરી} & \textbf{પરિણામ} \\
    \hline
    1. લોડ (5$\Omega$) દૂર કરો & ઓપન-સર્કિટ વોલ્ટેજ ($V_{oc}$) ગણો & $V_{oc} = V_{12} + (V_8-V_{12}) \times \frac{20}{20+10}$ & $V_{th} = 9.33V$ \\
    \hline
    2. વોલ્ટેજ સોર્સને શોર્ટ કરો & ઇક્વિવેલન્ટ રેઝિસ્ટન્સ ($R_{eq}$) ગણો & $R_{eq} = 20\Omega || 10\Omega$ & $R_{th} = 6.67\Omega$ \\
    \hline
    3. થેવેનિન ઇક્વિવેલન્ટ દોરો & $V_{th}$ અને $R_{th}$ ને લોડ સાથે સિરીઝમાં જોડો & - & - \\
    \hline
    4. લોડ કરંટ ગણો & $I = V_{th}/(R_{th}+R_L)$ & $I = 9.33/(6.67+5)$ & $I = 0.8A$ \\
    \hline
\end{tabulary}

\begin{mnemonicbox}
"થેવેનિન ટ્રાન્સફોર્મ: Voc અને Req શોધી, પછી I ગણો"
\end{mnemonicbox}
\end{solutionbox}

\section*{Question 3(a) OR [3 marks]}
\questionmarks{3(a) OR}{3}{ગુણ}

\textbf{Maximum Power Transfer Theorem જણાવો અને સમજાવો.}

\begin{solutionbox}
\textbf{જવાબ}:

\textbf{Maximum Power Transfer Theorem:}
\begin{itemize}
    \item મહત્તમ પાવર સોર્સથી લોડમાં ત્યારે ટ્રાન્સફર થાય છે જ્યારે \textbf{લોડ રેઝીસ્ટન્સ સોર્સના આંતરિક રેઝીસ્ટન્સ સમાન હોય} ($R_L = R_{th}$)
    \item મહત્તમ પાવર ટ્રાન્સફર પર માત્ર 50\% કાર્યક્ષમતા પ્રાપ્ત થાય છે
    \item DC અને AC સર્કિટ બંને માટે લાગુ પડે છે (કોમ્પ્લેક્સ ઇમ્પિડન્સ સાથે)
\end{itemize}

\textbf{આકૃતિ: મહત્તમ પાવર ટ્રાન્સફર}

\begin{center}
\begin{circuitikz}[scale=0.8]
    \draw (0,0) to[V, l=$V_{th}$] (0,3) to[R, l=$R_{th}$] (3,3) to[R, l=$R_L$] (3,0) -- (0,0);
\end{circuitikz}

\captionof{figure}{મહત્તમ પાવર ટ્રાન્સફર}
\end{center}

\textbf{ફોર્મ્યુલા}: $P = \frac{V_{th}^2 \times R_L}{(R_{th}+R_L)^2}$

\begin{mnemonicbox}
"મહત્તમ પાવર ટ્રાન્સફર માટે લોડને સોર્સ સાથે મેચ કરો"
\end{mnemonicbox}
\end{solutionbox}

\section*{Question 3(b) OR [4 marks]}
\questionmarks{3(b) OR}{4}{ગુણ}

\textbf{કોઈપણ સર્કિટનો ઉપયોગ કરીને ડ્યુઅલ નેટવર્ક દોરવાની પદ્ધતિ સમજાવો.}

\begin{solutionbox}
\textbf{જવાબ}:

\textbf{આકૃતિ: મૂળ અને ડ્યુઅલ નેટવર્ક ઉદાહરણ}

\begin{center}
\begin{circuitikz}[scale=0.7]
    \node at (0,3) {Original (Series RLC)};
    \draw (0,0) to[V, l=$V$] (0,2) to[R, l=$R$] (2,2) to[L, l=$L$] (4,2) to[C, l=$C$] (4,0) -- (0,0);
\end{circuitikz}
\hfill
\begin{circuitikz}[scale=0.7]
    \node at (0,3) {Dual (Parallel GCL)};
    \draw (0,0) to[I, l=$I$] (0,2) -- (4,2);
    \draw (1,2) to[R, l=$G$] (1,0);
    \draw (2,2) to[C, l=$C'$] (2,0);
    \draw (3,2) to[L, l=$L'$] (3,0);
    \draw (0,0) -- (4,0);
\end{circuitikz}
\captionof{figure}{ડ્યુઅલ નેટવર્ક}
\end{center}

\textbf{કોષ્ટક: ડ્યુઅલ નેટવર્ક રૂપાંતરણ નિયમો}

\begin{tabulary}{\linewidth}{@{}|L|L|L|@{}}
    \hline
    \textbf{મૂળ ઘટક} & \textbf{ડ્યુઅલ ઘટક} & \textbf{ઉદાહરણ} \\
    \hline
    સિરીઝ કનેક્શન & પેરેલલ કનેક્શન & સિરીઝ R $\to$ પેરેલલ C \\
    \hline
    પેરેલલ કનેક્શન & સિરીઝ કનેક્શન & પેરેલલ C $\to$ સિરીઝ L \\
    \hline
    વોલ્ટેજ સોર્સ & કરંટ સોર્સ & V સોર્સ $\to$ I સોર્સ \\
    \hline
    કરંટ સોર્સ & વોલ્ટેજ સોર્સ & I સોર્સ $\to$ V સોર્સ \\
    \hline
    રેઝીસ્ટર (R) & કંડક્ટન્સ (1/R) & R $\to$ G (1/R) \\
    \hline
    ઇન્ડક્ટર (L) & કેપેસિટર (1/L) & L $\to$ C (1/L) \\
    \hline
    કેપેસિટર (C) & ઇન્ડક્ટર (1/C) & C $\to$ L (1/C) \\
    \hline
\end{tabulary}

\textbf{ડ્યુઅલિટી પ્રક્રિયા:}
\begin{enumerate}
    \item મેશ્સને નોડ્સ તરીકે અને નોડ્સને મેશ્સ તરીકે રિડ્રો કરો
    \item ઘટકોને તેમના ડ્યુઅલ સાથે બદલો
    \item સિરીઝ અને પેરેલલ કનેક્શન્સને અદલાબદલી કરો
\end{enumerate}

\begin{mnemonicbox}
"ડ્યુઅલિટી સ્વેપ્સ: સિરીઝ\textleftrightarrow પેરેલલ, V\textleftrightarrow I, R\textleftrightarrow G, L\textleftrightarrow C"
\end{mnemonicbox}
\end{solutionbox}

\section*{Question 3(c) OR [7 marks]}
\questionmarks{3(c) OR}{7}{ગુણ}

\textbf{ઉપર આપેલ નેટવર્ક માટે નોર્ટનની ઇક્વીવેલન્ટ સર્કિટ શોધો. લોડ કરંટ શોધો જો (i) $R_L=3 K\Omega$ (ii) $R_L=1.5 \Omega$}

\begin{solutionbox}
\textbf{જવાબ}:

\textbf{આકૃતિ: મૂળ સર્કિટ અને નોર્ટન ઇક્વિવેલન્ટ}

\begin{center}
\begin{circuitikz}[scale=0.8]
    % Original
    \draw (0,0) node[ground]{} to[short] (6,0);
    \draw (0,0) to[V, l=6V, invert] (0,3) to[R, l=9$K\Omega$] (3,3);
    \draw (3,3) to[R, l=3$K\Omega$] (3,0);
    \draw (3,3) to[R, l=6$K\Omega$] (6,3) to[R, l=$R_L$] (6,0);
    \node at (3, -1) {Original Circuit};
    
    % Norton
    \draw (8,0) to[short] (12,0);
    \draw (8,0) to[I, l=$I_N$] (8,3) -- (12,3);
    \draw (10,3) to[R, l=$R_N$] (10,0);
    \draw (12,3) to[R, l=$R_L$] (12,0);
    \node at (10, -1) {Norton Equivalent};
\end{circuitikz}
\captionof{figure}{નોર્ટન ઇક્વિવેલન્ટ}
\end{center}

\textbf{કોષ્ટક: નોર્ટનના સિદ્ધાંતની પ્રક્રિયા અને ગણતરીઓ}

\begin{tabulary}{\linewidth}{@{}|L|L|L|L|@{}}
    \hline
    \textbf{સ્ટેપ} & \textbf{પ્રક્રિયા} & \textbf{ગણતરી} & \textbf{પરિણામ} \\
    \hline
    1. શોર્ટ-સર્કિટ કરંટ ($I_{sc}$) ગણો & લોડ ટર્મિનલ્સને શોર્ટ કરો અને કરંટ શોધો & - & $I_n = 0.5mA$ \\
    \hline
    2. નોર્ટન રેઝીસ્ટન્સ ($R_n$) ગણો & સોર્સને આંતરિક રેઝીસ્ટન્સ સાથે બદલો & - & $R_n = 3K\Omega$ \\
    \hline
    3. નોર્ટન ઇક્વિવેલન્ટ દોરો & $I_n$ અને $R_n$ ને પેરેલલમાં જોડો & - & - \\
    \hline
    4. લોડ કરંટ ($R_L = 3K\Omega$) ગણો & $I = I_n \times R_n/(R_n + R_L)$ & $I = 0.5mA \times 3K/(3K + 3K)$ & $I = 0.25mA$ \\
    \hline
    5. લોડ કરંટ ($R_L = 1.5\Omega$) ગણો & $I = I_n \times R_n/(R_n + R_L)$ & $I = 0.5mA \times 3K/(3K + 1.5)$ & $I = 0.33mA$ \\
    \hline
\end{tabulary}

\begin{mnemonicbox}
"નોર્ટનને કરંટ સોર્સ બનાવવા Isc અને Req જોઈએ"
\end{mnemonicbox}
\end{solutionbox}

\section*{Question 4(a) [3 marks]}
\questionmarks{4(a)}{3}{ગુણ}

\textbf{કોઇલ માટે ક્વોલિટી ફેક્ટર Q નું સમીકરણ મેળવો.}

\begin{solutionbox}
\textbf{જવાબ}:

\textbf{આકૃતિ: કોઇલ ઇક્વિવેલન્ટ સર્કિટ}

\begin{center}
\begin{circuitikz}
    \draw (0,0) to[short, o-] (1,0) to[R, l=$R$] (3,0) to[L, l=$L$] (5,0) to[short, -o] (6,0);
\end{circuitikz}
\captionof{figure}{કોઇલ ઇક્વિવેલન્ટ સર્કિટ}
\end{center}

\textbf{કોઇલ માટે Q ફેક્ટરની ડેરિવેશન:}

\begin{tabulary}{\linewidth}{@{}|L|L|L|@{}}
    \hline
    \textbf{સ્ટેપ} & \textbf{અભિવ્યક્તિ} & \textbf{સમજૂતી} \\
    \hline
    1. ઇમ્પિડન્સ & $Z = R + j\omega L$ & કોઇલનું કોમ્પ્લેક્સ ઇમ્પિડન્સ \\
    \hline
    2. રિએક્ટિવ પાવર & $P_X = (\omega L)I^2$ & ઇન્ડક્ટરમાં સંગ્રહિત પાવર \\
    \hline
    3. રીઅલ પાવર & $P_R = RI^2$ & રેઝીસ્ટન્સમાં વેડફાતો પાવર \\
    \hline
    4. ક્વોલિટી ફેક્ટર & $Q = P_X/P_R$ & સંગ્રહિત અને વેડફાતા પાવરનો રેશિયો \\
    \hline
    5. સબ્સ્ટિટ્યુશન & $Q = (\omega L)I^2/RI^2$ & અભિવ્યક્તિઓ સબ્સ્ટિટ્યુટ કરો \\
    \hline
    6. ફાઇનલ ઇક્વેશન & $Q = \omega L/R$ & Q ફેક્ટર મેળવવા સરળ કરો \\
    \hline
\end{tabulary}

\begin{mnemonicbox}
"ક્વોલિટી કોઇલ્સ: omega L/R ઊર્જા બચાવવાની ક્ષમતા દર્શાવે છે"
\end{mnemonicbox}
\end{solutionbox}

\section*{Question 4(b) [4 marks]}
\questionmarks{4(b)}{4}{ગુણ}

\textbf{શ્રેણી RLC સર્કિટમાં R=50 $\Omega$, L=0.2 H અને C=10 $\mu$F છે. (i)Q પરિબળ, (ii) BW, (iii) અપર કટ ઓફ અને લોઅર કટ ઓફ ફ્રીક્વન્સીઝની ગણતરી કરો.}

\begin{solutionbox}
\textbf{જવાબ}:

\textbf{આકૃતિ: સિરીઝ RLC સર્કિટ}

\begin{center}
\begin{circuitikz}
    \draw (0,0) to[short, o-] (1,0) to[R, l=$R=50\Omega$] (3,0) to[L, l=$L=0.2H$] (5,0) to[C, l=$C=10\mu F$] (7,0) to[short, -o] (8,0);
\end{circuitikz}
\captionof{figure}{સિરીઝ RLC સર્કિટ}
\end{center}

\textbf{કોષ્ટક: સિરીઝ RLC સર્કિટ માટે ગણતરીઓ}

\begin{tabulary}{\linewidth}{@{}|L|L|L|L|@{}}
    \hline
    \textbf{પેરામીટર} & \textbf{ફોર્મ્યુલા} & \textbf{ગણતરી} & \textbf{પરિણામ} \\
    \hline
    રેઝોનન્ટ ફ્રીક્વન્સી ($f_r$) & $f_r = 1/(2\pi\sqrt{LC})$ & $1/(2\pi\sqrt{0.2\times10\times10^{-6}})$ & $112.5$ Hz \\
    \hline
    ક્વોલિટી ફેક્ટર ($Q$) & $Q = (1/R)\sqrt{L/C}$ & $(1/50)\sqrt{0.2/10\times10^{-6}}$ & $28.28$ \\
    \hline
    બેન્ડવિડ્થ ($BW$) & $BW = f_r/Q$ & $112.5/28.28$ & $3.98$ Hz \\
    \hline
    લોઅર કટઓફ ($f_1$) & $f_1 = f_r - BW/2$ & $112.5 - 3.98/2$ & $110.51$ Hz \\
    \hline
    અપર કટઓફ ($f_2$) & $f_2 = f_r + BW/2$ & $112.5 + 3.98/2$ & $114.49$ Hz \\
    \hline
\end{tabulary}

\begin{mnemonicbox}
"Q કટઓફ ફ્રીક્વન્સી માટે BW નિર્ધારિત કરે છે"
\end{mnemonicbox}
\end{solutionbox}

\section*{Question 4(c) [7 marks]}
\questionmarks{4(c)}{7}{ગુણ}

\textbf{મ્યુચ્યુઅલ ઇન્ડક્ટન્સના કો-એફીસીએન્ટ સાથે મ્યુચ્યુઅલ ઇન્ડક્ટન્સ સમજાવો. K નું સમીકરણ પણ મેળવો.}

\begin{solutionbox}
\textbf{જવાબ}:

\textbf{આકૃતિ: બે કોઇલ વચ્ચે મ્યુચ્યુઅલ ઇન્ડક્ટન્સ}

\begin{center}
\begin{circuitikz}
    \draw (0,0) to[L, l=$L_1$] (0,2);
    \draw (2,0) to[L, l=$L_2$] (2,2);
    \draw [dashed, <->] (0.5, 1) -- (1.5, 1) node[midway, above] {$M$};
    \node at (-0.5, 1) {Input};
    \node at (2.5, 1) {Output};
\end{circuitikz}
\captionof{figure}{મ્યુચ્યુઅલ ઇન્ડક્ટન્સ}
\end{center}

\textbf{મ્યુચ્યુઅલ ઇન્ડક્ટન્સ (M):}
\begin{itemize}
    \item જ્યારે એક કોઇલમાં કરંટ નજીકની કોઇલમાં વોલ્ટેજ પ્રેરિત કરે છે
    \item કોઇલ્સ વચ્ચેની કપલિંગ તેમની સ્થિતિ, ઓરિયેન્ટેશન અને માધ્યમ પર નિર્ભર કરે છે
    \item મ્યુચ્યુઅલ ઇન્ડક્ટન્સ M હેનરી (H)માં
\end{itemize}

\textbf{કોષ્ટક: મ્યુચ્યુઅલ ઇન્ડક્ટન્સ સમીકરણો}

\begin{tabulary}{\linewidth}{@{}|L|L|L|@{}}
    \hline
    \textbf{પેરામીટર} & \textbf{ફોર્મ્યુલા} & \textbf{વર્ણન} \\
    \hline
    પ્રેરિત વોલ્ટેજ & $v_2 = M(di_1/dt)$ & કોઇલ 1માં કરંટને લીધે કોઇલ 2માં પ્રેરિત વોલ્ટેજ \\
    \hline
    મ્યુચ્યુઅલ ઇન્ડક્ટન્સ & $M = k\sqrt{L_1L_2}$ & સેલ્ફ-ઇન્ડક્ટન્સ સાથે સંબંધિત મ્યુચ્યુઅલ ઇન્ડક્ટન્સ \\
    \hline
    કપલિંગ કોઇફિશિયન્ટ (k) & $k = M/\sqrt{L_1L_2}$ & કોઇલ્સ વચ્ચેની કપલિંગનું માપ ($0 \le k \le 1$) \\
    \hline
    કુલ ઇન્ડક્ટન્સ & $L_t = L_1 + L_2 \pm 2M$ & કુલ ઇન્ડક્ટન્સ કપલિંગની દિશા પર નિર્ભર \\
    \hline
\end{tabulary}

\textbf{કપલિંગ કોઇફિશિયન્ટ (k)ની ડેરિવેશન:}
\begin{itemize}
    \item $M = k\sqrt{L_1L_2}$ માંથી
    \item ફરી ગોઠવતા: $k = M/\sqrt{L_1L_2}$
    \item $k = 1$ પરફેક્ટ કપલિંગ માટે
    \item $k = 0$ નો કપલિંગ માટે
    \item વાસ્તવિક સર્કિટ માટે સામાન્ય રીતે 0.1 થી 0.9
\end{itemize}

\begin{mnemonicbox}
"M મેગ્નેટિક લિંકેજ માપે, k કપલિંગની ક્વોલિટી દર્શાવે"
\end{mnemonicbox}
\end{solutionbox}

\section*{Question 4(a) OR [3 marks]}
\questionmarks{4(a) OR}{3}{ગુણ}

\textbf{કપલ સર્કિટ માટે કપ્લીંગના પ્રકારો સમજાવો.}

\begin{solutionbox}
\textbf{જવાબ}:

\textbf{આકૃતિ: કપલિંગના પ્રકારો}

\begin{center}
\captionof{figure}{કપલિંગના પ્રકારો}
\end{center}

\textbf{કોષ્ટક: કપલિંગના પ્રકારો}

\begin{tabulary}{\linewidth}{@{}|L|L|L|@{}}
    \hline
    \textbf{કપલિંગનો પ્રકાર} & \textbf{લક્ષણો} & \textbf{એપ્લિકેશન} \\
    \hline
    \textbf{ટાઇટ કપલિંગ} & $k > 0.5$, ઉચ્ચ ઊર્જા ટ્રાન્સફર & ટ્રાન્સફોર્મર \\
    \hline
    \textbf{લૂઝ કપલિંગ} & $k < 0.5$, સિલેક્ટિવ ફ્રીક્વન્સી રિસ્પોન્સ & RF ટ્યુનિંગ સર્કિટ \\
    \hline
    \textbf{ક્રિટિકલ કપલિંગ} & k ઓપ્ટિમલ બેન્ડવિડ્થ માટે એડજસ્ટ કરેલું & RF ફિલ્ટર \\
    \hline
    \textbf{ડાયરેક્ટ કપલિંગ} & ઘટકો સીધા જોડાયેલા & ઓડિયો એમ્પ્લિફાયર \\
    \hline
    \textbf{ઇન્ડક્ટિવ કપલિંગ} & મેગ્નેટિક ફિલ્ડ ઊર્જા ટ્રાન્સફર કરે છે & ટ્રાન્સફોર્મર, વાયરલેસ ચાર્જિંગ \\
    \hline
    \textbf{કેપેસિટિવ કપલિંગ} & ઇલેક્ટ્રિક ફિલ્ડ ઊર્જા ટ્રાન્સફર કરે છે & સ્ટેજ વચ્ચે સિગ્નલ કપલિંગ \\
    \hline
\end{tabulary}

\begin{mnemonicbox}
"TLCLIC: ટાઇટ, લૂઝ, ક્રિટિકલ, ડાયરેક્ટ, ઇન્ડક્ટિવ, કેપેસિટિવ"
\end{mnemonicbox}
\end{solutionbox}

\section*{Question 4(b) OR [4 marks]}
\questionmarks{4(b) OR}{4}{ગુણ}

\textbf{ગુણવત્તા પરિબળ Q = 100, રેઝોનન્ટ ફ્રિકવન્સી Fr = 50 KHz સાથે 10 mH નું ઇન્ડક્ટન્સ ધરાવતું સમાંતર રેઝોનન્ટ સર્કિટ. શોધો (i) જરૂરી કેપેસીટન્સ C, (ii) કોઇલનો પ્રતિકાર R, (iii) BW.}

\begin{solutionbox}
\textbf{જવાબ}:

\textbf{આકૃતિ: પેરેલલ રેઝોનન્ટ સર્કિટ}

\begin{center}
\begin{circuitikz}
    \draw (0,2) -- (4,2);
    \draw (0,0) -- (4,0);
    \draw (1,2) to[L, l=$L=10mH$] (1,1) to[R, l=$R$] (1,0); 
    \draw (3,2) to[C, l=$C=?$] (3,0); 
    \node at (0,1) {Source};
\end{circuitikz}
\captionof{figure}{પેરેલલ રેઝોનન્ટ સર્કિટ}
\end{center}

\textbf{કોષ્ટક: પેરેલલ રેઝોનન્ટ સર્કિટ માટે ગણતરીઓ}

\begin{tabulary}{\linewidth}{@{}|L|L|L|L|@{}}
    \hline
    \textbf{પેરામીટર} & \textbf{ફોર્મ્યુલા} & \textbf{ગણતરી} & \textbf{પરિણામ} \\
    \hline
    રેઝોનન્ટ ફ્રીક્વન્સી & $f_r = 1/(2\pi\sqrt{LC})$ & $50 kHz = 1/(2\pi\sqrt{10\times10^{-3}\times C})$ & - \\
    \hline
    કેપેસિટન્સ (C) & $C = 1/(4\pi^2 f_r^2 L)$ & $C = 1/(4\pi^2 \times (50\times10^3)^2 \times 10\times10^{-3})$ & $C = 1.01$ nF \\
    \hline
    રેઝિસ્ટન્સ (R) & $Q = \omega L/R$ & $100 = 2\pi \times 50\times10^3 \times 10\times10^{-3}/R$ & $R = 31.4 \Omega$ \\
    \hline
    બેન્ડવિડ્થ (BW) & $BW = f_r/Q$ & $BW = 50\times10^3/100$ & $BW = 500$ Hz \\
    \hline
\end{tabulary}

\begin{mnemonicbox}
"પેરેલલ રેઝોનન્સ પેરામીટર્સ: C fr માંથી, R Q માંથી, BW fr/Q માંથી"
\end{mnemonicbox}
\end{solutionbox}

\section*{Question 4(c) OR [7 marks]}
\questionmarks{4(c) OR}{7}{ગુણ}

\textbf{સીરીઝ RLC સર્કિટની Band width અને Selectivity સમજાવો. શ્રેણી રેઝોનન્સ સર્કિટ માટે Q પરિબળ અને BW વચ્ચેનો સંબંધ પણ સ્થાપિત કરો.}

\begin{solutionbox}
\textbf{જવાબ}:

\textbf{આકૃતિ: સિરીઝ RLC સર્કિટનો ફ્રીક્વન્સી રિસ્પોન્સ}

\begin{center}
\textit{નોંધ: રેઝોનન્સ કર્વ ઘંટડાના આકારનું હોય છે જેમાં કરંટ $f_r$ પર મહત્તમ હોય છે અને બેન્ડવિડ્થ $f_1$ અને $f_2$ વચ્ચે છે.}

    \item હાફ-પાવર પોઇન્ટ વચ્ચેની \textbf{ફ્રીક્વન્સી રેન્જ}
    \item હાફ-પાવર પોઇન્ટ પર ઇમ્પિડન્સ લઘુતમ મૂલ્યના $\sqrt{2}$ ગણું હોય છે
    \item $BW = f_2 - f_1$, જ્યાં $f_1$ અને $f_2$ લોઅર અને અપર કટઓફ ફ્રીક્વન્સી છે
\end{itemize}

\textbf{સિલેક્ટિવિટી:}
\begin{itemize}
    \item બેન્ડવિડ્થ બહારની ફ્રીક્વન્સીઓને \textbf{નકારવાની ક્ષમતા}
    \item ઉચ્ચ Q એટલે વધુ સિલેક્ટિવિટી અને સાંકડી બેન્ડવિડ્થ
    \item રિસ્પોન્સ કર્વની તીવ્રતા દ્વારા માપવામાં આવે છે
\end{itemize}

\textbf{કોષ્ટક: સિરીઝ RLC બેન્ડવિડ્થ પેરામીટર્સ}

\begin{tabulary}{\linewidth}{@{}|L|L|L|@{}}
    \hline
    \textbf{પેરામીટર} & \textbf{ફોર્મ્યુલા} & \textbf{વર્ણન} \\
    \hline
    બેન્ડવિડ્થ (BW) & $BW = f_2 - f_1$ & અપર અને લોઅર કટઓફ પોઇન્ટ વચ્ચેનો તફાવત \\
    \hline
    હાફ-પાવર પોઇન્ટ & $Z = \sqrt{2} \times Z_{min}$ & જ્યાં પાવર મહત્તમના અર્ધા જેટલો થાય છે \\
    \hline
    રેઝોનન્ટ ફ્રીક્વન્સી & $f_r = 1/(2\pi\sqrt{LC})$ & સેન્ટર ફ્રીક્વન્સી \\
    \hline
    ક્વોલિટી ફેક્ટર & $Q = \omega_0 L/R$ & ઊર્જા સંગ્રહ vs. વેડફાટ રેશિયો \\
    \hline
\end{tabulary}

\textbf{Q-BW સંબંધની ડેરિવેશન:}
\begin{itemize}
    \item રેઝોનન્સ પર, ઇમ્પિડન્સ $Z = R$
    \item કટઓફ ફ્રીક્વન્સી પર $Z = \sqrt{2}R$
    \item આ ત્યારે થાય છે જ્યારે રિએક્ટન્સ $X_L - X_C = \pm R$
    \item $f_1$ પર: $\omega L - 1/\omega C = -R$
    \item $f_2$ પર: $\omega L - 1/\omega C = +R$
    \item આ સમીકરણો ઉકેલતા: $BW = R/2\pi L = f_r/Q$
    \item આથી, $Q = f_r/BW$
\end{itemize}

\begin{mnemonicbox}
"ક્વોલિટી બેન્ડવિડ્થના વ્યસ્ત પ્રમાણમાં"
\end{mnemonicbox}
\end{solutionbox}

\section*{Question 5(a) [3 marks]}
\questionmarks{5(a)}{3}{ગુણ}

\textbf{60 ડીબીનું એટેન્યુએશન આપવા અને 500 $\Omega$ પ્રતિકારના લોડમાં કામ કરવા માટે સપ્રમાણ T પ્રકારના એટેન્યુએટરને ડિઝાઇન કરો.}

\begin{solutionbox}
\textbf{જવાબ}:

\textbf{આકૃતિ: સપ્રમાણ T-ટાઇપ એટેન્યુએટર}

\begin{center}
\begin{circuitikz}
    \draw (0,2) to[short, o-] (1,2) to[R, l=$R_1$] (3,2) to[R, l=$R_1$] (5,2) to[short, -o] (6,2);
    \draw (3,2) to[R, l=$R_2$] (3,0);
    \draw (0,0) to[short, o-] (6,0) to[short, -o] (6,0);
\end{circuitikz}
\captionof{figure}{T-ટાઇપ એટેન્યુએટર}
\end{center}

\textbf{કોષ્ટક: એટેન્યુએટર ડિઝાઇન}

\begin{tabulary}{\linewidth}{@{}|L|L|L|L|@{}}
    \hline
    \textbf{પેરામીટર} & \textbf{ફોર્મ્યુલા} & \textbf{ગણતરી} & \textbf{પરિણામ} \\
    \hline
    એટેન્યુએશન (N) & $N = 10^{(dB/20)}$ & $10^{(60/20)}$ & $N = 1000$ \\
    \hline
    $Z_0$ & આપેલ & $500 \Omega$ & $500 \Omega$ \\
    \hline
    $R_1$ & $R_1 = 2Z_0(N-1)/(N+1)$ & $2\times500\times(1000-1)/(1000+1)$ & $R_1 = 998 \Omega$ \\
    \hline
    $R_2$ & $R_2 = Z_0(N+1)/(N-1)$ & $500\times(1000+1)/(1000-1)$ & $R_2 = 0.5 \Omega$ \\
    \hline
\end{tabulary}

\begin{mnemonicbox}
"T એટેન્યુએટર: R1 સિરીઝ ડિવાઈડ્સ, R2 શન્ટ્સ"
\end{mnemonicbox}
\end{solutionbox}

\section*{Question 5(b) [4 marks]}
\questionmarks{5(b)}{4}{ગુણ}

\textbf{Compare Band pass and Band stop filters.}

\begin{solutionbox}
\textbf{Answer}:

\textbf{Diagram: Band Pass vs Band Stop Response}

\begin{center}
\hfill
\captionof{figure}{Filter Responses}
\end{center}

\textbf{Table: Comparison of Band Pass and Band Stop Filters}

\begin{tabulary}{\linewidth}{@{}|L|L|L|@{}}
    \hline
    \textbf{Parameter} & \textbf{Band Pass Filter} & \textbf{Band Stop Filter} \\
    \hline
    \textbf{Frequency Response} & Passes frequencies within specific band & Rejects frequencies within specific band \\
    \hline
    \textbf{Circuit Structure} & Series \& parallel resonant circuits & Series \& parallel resonant circuits \\
    \hline
    \textbf{Cut-off Frequencies} & Has lower ($f_1$) and upper ($f_2$) cut-offs & Has lower ($f_1$) and upper ($f_2$) cut-offs \\
    \hline
    \textbf{Bandwidth} & $BW = f_2 - f_1$ & $BW = f_2 - f_1$ \\
    \hline
    \textbf{Applications} & Radio tuning, audio equalization & Noise elimination, harmonic suppression \\
    \hline
    \textbf{Implementation} & Series/parallel combination of HPF \& LPF & Parallel/series combination of HPF \& LPF \\
    \hline
    \textbf{Phase Response} & 0$^{\circ}$ at resonance & 180$^{\circ}$ at resonance \\
    \hline
\end{tabulary}

\end{solutionbox}

\end{document}
