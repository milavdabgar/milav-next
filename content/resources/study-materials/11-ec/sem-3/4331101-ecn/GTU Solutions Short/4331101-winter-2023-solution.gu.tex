\documentclass{article}

% content/resources/templates/preamble.tex
\usepackage[margin=0.6in]{geometry}
\author{Milav Dabgar}
\usepackage{amsmath,amssymb,amsthm}
\usepackage{booktabs}
\usepackage{multirow}
\usepackage{xcolor}
\usepackage{tcolorbox}
\tcbuselibrary{breakable,skins}
\usepackage[colorlinks=true,linkcolor=blue]{hyperref}
\usepackage{titlesec}
\usepackage{enumitem}
\usepackage{tikz}
\usepackage{pgfplots}
\usepackage{circuitikz}
\usepackage[version=4]{mhchem}
\usepackage{longtable}
\usepackage{array}
\usepackage{float}
\usepackage{caption}
\usepackage{listings}

\lstset{
  basicstyle=\small\ttfamily,
  breaklines=true,
  breakatwhitespace=false,
  postbreak=\mbox{\textcolor{red}{$\hookrightarrow$}\space},
  float=false,
  numbers=left,
  numberstyle=\tiny\color{gray},
  numbersep=10pt,
  xleftmargin=2em,
  keywordstyle=\color{blue},
  commentstyle=\color{green!60!black},
  stringstyle=\color{purple},
  backgroundcolor=\color{gray!5},
  showstringspaces=false,
  tabsize=2,
  captionpos=b,
  keepspaces=true,
  columns=flexible
}

\pgfplotsset{compat=1.18}
\usetikzlibrary{shapes,arrows,positioning,calc,patterns,decorations.pathmorphing,decorations.markings,arrows.meta}

% Color scheme
\definecolor{headcolor}{RGB}{0,102,204}
\definecolor{keycolor}{RGB}{220,20,60}
\definecolor{solutioncolor}{RGB}{34,139,34}
\definecolor{mnemoniccolor}{RGB}{148,0,211}
\definecolor{codecolor}{RGB}{0,0,100}

% Spacing
\setlength{\parskip}{3pt}
\setlist[itemize]{nosep}
\setlist[enumerate]{nosep}

% Title formatting
\titleformat{\section}{\Large\bfseries\color{headcolor}}{\thesection}{1em}{}
\titleformat{\subsection}{\large\bfseries\color{headcolor}}{\thesubsection}{1em}{}

% Pandoc tightlist compatibility
\providecommand{\tightlist}{%
  \setlength{\itemsep}{0pt}\setlength{\parskip}{0pt}}

% Pandoc longtable compatibility
\newcounter{none}
\def\thenone{}


% content/resources/templates/gujarati-boxes.tex
\usepackage{fontspec}
\usepackage{polyglossia}

% Set Gujarati as main language (document is primarily in Gujarati)
% Note: gloss-gujarati.ldf doesn't exist in polyglossia, but it will use hyphenation patterns
\setdefaultlanguage{gujarati}
\setotherlanguage{english}

% Configure Gujarati font properly
% Use Language=Default to prevent polyglossia from trying to add language-specific features
% that don't exist for Gujarati, which causes "empty feature" warnings
\newfontfamily\gujaratifont[Script=Gujarati,AutoFakeBold=2.5,AutoFakeSlant=0.3]{Noto Sans Gujarati}
\setmainfont[Script=Gujarati,AutoFakeBold=2.5,AutoFakeSlant=0.3]{Noto Sans Gujarati}
% Use Noto Sans Gujarati for monospace to support Gujarati in text
\setmonofont[Scale=0.9]{Noto Sans Gujarati}

% Configure English to use the same font
\newfontfamily\englishfont[Script=Gujarati,AutoFakeBold=2.5,AutoFakeSlant=0.3]{Noto Sans Gujarati}

% Translations for polyglossia
\gappto\captionsgujarati{
  \renewcommand{\tablename}{કોષ્ટક}
  \renewcommand{\figurename}{આકૃતિ}
}

% Helper for TikZ nodes to ensure Gujarati font
\newcommand{\gu}[1]{{\gujaratifont #1}}

% Custom environments
\newtcolorbox{solutionbox}{
    breakable,
    enhanced,
    colback=solutioncolor!5!white,
    colframe=solutioncolor!75!black,
    fonttitle=\bfseries,
    title=જવાબ
}

\newtcolorbox{solutionboxnobreak}{
 colback=solutioncolor!5!white,
 colframe=solutioncolor!75!black,
 fonttitle=\bfseries,
 title=જવાબ
}

\newtcolorbox{keyformula}{
 breakable,
 enhanced,
 colback=keycolor!5!white,
 colframe=keycolor!75!black,
 fonttitle=\bfseries,
 title=રાસાયણિક સમીકરણ/સૂત્ર
}

\newtcolorbox{mnemonicbox}{
 breakable,
 enhanced,
 colback=mnemoniccolor!5!white,
 colframe=mnemoniccolor!75!black,
 fonttitle=\bfseries,
 title=મેમરી ટ્રીક
}


% Custom commands for GTU solutions
% This file defines semantic commands for consistent formatting

% Question command with automatic formatting
\newcommand{\question}[2]{%
  \section*{Question #1}%
  \textbf{#2}%
}

% OR question variant
\newcommand{\questionor}[2]{%
  \section*{Question #1 OR}%
  \textbf{#2}%
}

% Proper table environment with caption
\newenvironment{answertable}[1]{%
  \begin{table}[htbp]
  \centering
  \caption{#1}
}{%
  \end{table}
}

% Proper figure environment for diagrams
\newenvironment{answerdiagram}[1]{%
  \begin{figure}[htbp]
  \centering
  \caption{#1}
}{%
  \end{figure}
}

% Semantic markup for key terms
\newcommand{\keyword}[1]{\textbf{#1}}
\newcommand{\code}[1]{\texttt{#1}}
\newcommand{\classname}[1]{\texttt{#1}}
\newcommand{\methodname}[1]{\texttt{#1}}

% Proper quotation marks
\newcommand{\mnemonic}[1]{``#1''}


\title{ઇલેક્ટ્રોનિક સર્કિટ્સ અને નેટવર્ક્સ (4331101) - શિયાળો 2023 સોલ્યુશન}
\date{જાન્યુઆરી 11, 2023}

\begin{document}
\maketitle

\section*{પ્રશ્ન 1(a) [3 ગુણ]}
\questionmarks{1(a)}{3}{યોગ્ય રેખાકૃતિ સાથે સ્ત્રોત પરિવર્તન સમજાવો.}

\begin{solutionbox}
\textbf{સ્ત્રોત પરિવર્તન}: વોલ્ટેજ સ્ત્રોતને કરંટ સ્ત્રોતમાં અથવા તેનાથી વિપરીત રૂપાંતરિત કરવાની પદ્ધતિ છે જેમાં બાહ્ય સર્કિટનું વર્તન બદલાતું નથી.

\begin{center}
\begin{circuitikz}[american, scale=0.9]
    % Voltage Source
    \draw (0,0) to[V, l=$V_S$] (0,2) to[R, l=$R_S$] (2,2) to[short, -o] (3,2) node[right]{A};
    \draw (0,0) to[short, -o] (3,0) node[right]{B};
    \node at (1.5, -0.5) {\gu{વોલ્ટેજ સ્ત્રોત}};

    \draw[<->, thick] (3.5, 1) -- (4.5, 1);

    % Current Source
    \draw (6,0) to[I, l=$I_S$] (6,2) to[short, -o] (8,2) node[right]{A};
    \draw (6,0) to[short, -o] (8,0) node[right]{B};
    \draw (7,2) to[R, l=$R_P$] (7,0);
    \node at (7, -0.5) {\gu{કરંટ સ્ત્રોત}};
\end{circuitikz}
\captionof{figure}{સ્ત્રોત પરિવર્તન}
\end{center}

\textbf{સૂત્રો}:
\begin{itemize}
    \item \textbf{વોલ્ટેજથી કરંટ}: $I_S = V_S/R_S$, સમાન $R_P = R_S$ સમાંતરમાં.
    \item \textbf{કરંટથી વોલ્ટેજ}: $V_S = I_S \times R_P$, સમાન $R_S = R_P$ શ્રેણીમાં.
\end{itemize}

\begin{mnemonicbox}
"મૂલ્ય રહે છે, રેસિસ્ટન્સ બદલાય છે" (V=IR હંમેશા લાગુ પડે છે)
\end{mnemonicbox}
\end{solutionbox}

\section*{પ્રશ્ન 1(b) [4 ગુણ]}
\questionmarks{1(b)}{4}{શ્રેણીમાં જોડાયેલા બે કેપેસિટર માટે વોલ્ટેજ, કરંટ અને પાવર સંબંધ મેળવો.}

\begin{solutionbox}
\textbf{શ્રેણીમાં કેપેસિટર્સ}:

\begin{center}
\begin{circuitikz}[american]
    \draw (0,0) to[V, l=$V$] (0,2) to[C, l=$C_1$, v=$V_1$] (2,2) to[C, l=$C_2$, v=$V_2$] (4,2) to[short] (4,0) -- (0,0);
\end{circuitikz}
\end{center}

\begin{tabulary}{\linewidth}{@{}|L|L|L|@{}}
    \hline
    \textbf{પરિમાણ} & \textbf{સૂત્ર} & \textbf{સમજૂતી} \\
    \hline
    કુલ કેપેસિટન્સ & $1/C_T = 1/C_1 + 1/C_2$ & પ્રતિરોધી યોગ \\
    \hline
    વોલ્ટેજ વિતરણ & $V_1/V_2 = C_2/C_1$ & કેપેસિટન્સ રેશિયોના વ્યસ્ત \\
    \hline
    કરંટ & $I = I_1 = I_2$ & બધા દ્વારા સમાન કરંટ વહે છે \\
    \hline
    ચાર્જ & $Q = Q_1 = Q_2$ & દરેક કેપેસિટર પર સમાન ચાર્જ \\
    \hline
    પાવર & $P = VI = V^2/X_c$ & જ્યાં $X_c = 1/2\pi fC$ \\
    \hline
\end{tabulary}

\textbf{સંબંધો}:
\begin{itemize}
    \item \textbf{વોલ્ટેજ વિભાજન}: $V_1 = V \times \frac{C_2}{C_1+C_2}$
    \item \textbf{ચાર્જ સંગ્રહ}: $Q = C_{eq}V = \frac{C_1C_2}{C_1+C_2}V$
\end{itemize}

\begin{mnemonicbox}
"શ્રેણીમાં કેપેસિટર્સ: કરંટ સમાન, કેપેસિટન્સ ઘટે"
\end{mnemonicbox}
\end{solutionbox}

\section*{પ્રશ્ન 1(c) [7 ગુણ]}
\questionmarks{1(c)}{7}{રેસિસ્ટરના શ્રેણી અને સમાંતર જોડાણ વચ્ચેનો તફાવત આપો અને સમાંતર જોડાણના કુલ રેસિસ્ટન્સનું સમીકરણ મેળવો.}

\begin{solutionbox}
\textbf{શ્રેણી અને સમાંતર રેસિસ્ટર્સ વચ્ચેનો તફાવત}:

\begin{tabulary}{\linewidth}{@{}|L|L|L|@{}}
    \hline
    \textbf{પરિમાણ} & \textbf{શ્રેણી જોડાણ} & \textbf{સમાંતર જોડાણ} \\
    \hline
    કુલ રેસિસ્ટન્સ & વધે છે ($R_T = R_1 + R_2 + \dots$) & ઘટે છે ($R_T <$ સૌથી નાના $R$) \\
    \hline
    કરંટ & બધામાં સમાન ($I$) & વિભાજન થાય ($I_T = I_1 + I_2 + \dots$) \\
    \hline
    વોલ્ટેજ & વિભાજન થાય ($V_T = V_1 + V_2 + \dots$) & બધા પર સમાન ($V$) \\
    \hline
    પાવર & $P_T = P_1 + P_2 + \dots$ & $P_T = P_1 + P_2 + \dots$ \\
    \hline
\end{tabulary}

\textbf{સમાંતર રેસિસ્ટન્સ માટેનું વ્યુત્પત્તિ}:

\begin{center}
\begin{circuitikz}[american, scale=0.8]
    \draw (0,0) to[I, l=$I_T$] (0,2) -- (1,2);
    \draw (1,2) -- (1,3) to[R, l=$R_1$, i=$I_1$] (3,3) -- (3,2);
    \draw (1,2) -- (1,1) to[R, l=$R_2$, i=$I_2$] (3,1) -- (3,2);
    \draw (3,2) -- (4,2) to[short] (4,0) -- (0,0);
    \node at (2, 3.5) {$V$};
\end{circuitikz}
\end{center}

\begin{enumerate}
    \item કિરચોફના કરંટ નિયમ (KCL) અનુસાર:
    \[ I_T = I_1 + I_2 + \dots + I_n \]
    \item ઓહ્મના નિયમ ($I = V/R$) મુકતાં:
    \[ \frac{V}{R_T} = \frac{V}{R_1} + \frac{V}{R_2} + \dots + \frac{V}{R_n} \]
    \item વોલ્ટેજ $V$ બધા રેસિસ્ટર્સ માટે સમાન હોવાથી, $V$ વડે ભાગતા:
    \[ \frac{1}{R_T} = \frac{1}{R_1} + \frac{1}{R_2} + \dots + \frac{1}{R_n} \]
    \item બે રેસિસ્ટર્સ માટે:
    \[ \frac{1}{R_T} = \frac{1}{R_1} + \frac{1}{R_2} \implies R_T = \frac{R_1R_2}{R_1+R_2} \]
\end{enumerate}

\begin{mnemonicbox}
"સમાંતરમાં, વ્યસ્ત મૂલ્યો ઉમેરાય છે"
\end{mnemonicbox}
\end{solutionbox}

\section*{પ્રશ્ન 1(c) OR [7 ગુણ]}
\questionmarks{1(c) OR}{7}{1) યુનિલેટરલ, બાયલેટરલ નેટવર્ક, મેશ અને લૂપ વ્યાખ્યાયિત કરો.\\ 2) વોલ્ટેજ ડિવિઝન સર્કિટ દોરો અને સમીકરણ લખો.}

\begin{solutionbox}
\textbf{1) વ્યાખ્યાઓ}:

\begin{tabulary}{\linewidth}{@{}|L|L|L|@{}}
    \hline
    \textbf{પદ} & \textbf{વ્યાખ્યા} & \textbf{ઉદાહરણ} \\
    \hline
    યુનિલેટરલ નેટવર્ક & માત્ર એક દિશામાં કરંટ પસાર થવા દે છે. દિશા સાથે લાક્ષણિકતાઓ બદલાય છે. & ડાયોડ, ટ્રાન્ઝિસ્ટર સર્કિટ \\
    \hline
    બાયલેટરલ નેટવર્ક & બંને દિશામાં કરંટ પસાર થવા દે છે. લાક્ષણિકતાઓ દિશાથી સ્વતંત્ર છે. & RLC સર્કિટ \\
    \hline
    મેશ & સપાટ નેટવર્ક પાથ જેમાં કોઈ બીજો પાથ નથી (મૂળભૂત લૂપ). & એક બંધ પાથ \\
    \hline
    લૂપ & નેટવર્કમાં કોઈપણ બંધ પાથ જ્યાં છેલ્લો નોડ પહેલા નોડ જેવો જ હોય. & બાહ્ય પરિમિતિ \\
    \hline
\end{tabulary}

\textbf{2) વોલ્ટેજ ડિવિઝન સર્કિટ}:

\begin{center}
\begin{circuitikz}[american]
    \draw (0,0) to[V, l=$V_{in}$] (0,4) to[R, l=$R_1$] (3,4) to[R, l=$R_2$, v=$V_o$] (3,0) -- (0,0);
    \draw (3,4) to[short, -o] (4,4) node[right]{+};
    \draw (3,0) to[short, -o] (4,0) node[right]{-};
    \node at (4.5, 2) {$V_o$};
\end{circuitikz}
\end{center}

\textbf{સમીકરણ}:
\[ V_o = V_{in} \times \frac{R_2}{R_1 + R_2} \]

\begin{itemize}
    \item રેસિસ્ટર પર વોલ્ટેજ તેના રેસિસ્ટન્સ અને કુલ શ્રેણી રેસિસ્ટન્સના પ્રમાણમાં હોય છે.
\end{itemize}

\begin{mnemonicbox}
"આઉટપુટ વોલ્ટેજ ઈનપુટ ગુણ્યા રેસિસ્ટન્સના ગુણોત્તર"
\end{mnemonicbox}
\end{solutionbox}

\section*{પ્રશ્ન 2(a) [3 ગુણ]}
\questionmarks{2(a)}{3}{T-type નેટવર્કને $\pi$-type નેટવર્કમાં કન્વર્ટ કરવા માટે સમીકરણો મેળવો.}

\begin{solutionbox}
\textbf{T થી $\pi$ રૂપાંતરણ}:

\begin{center}
\begin{circuitikz}[american, scale=0.7]
    % T network
    \draw (0,2) node[left]{A} to[short, o-] (1,2) to[R, l=$Z_1$] (3,2) coordinate(C);
    \draw (C) to[R, l=$Z_2$] (5,2) to[short, -o] (6,2) node[right]{B};
    \draw (C) to[R, l=$Z_3$] (3,0) node[below]{C};
    
    \draw[->, thick] (6.5, 1) -- (7.5, 1);
    
    % Pi network
    \draw (8,2) node[left]{A} to[short, o-] (9,2) coordinate(A) to[R, l=$Z_{12}$] (13,2) coordinate(B) to[short, -o] (14,2) node[right]{B};
    \draw (A) to[R, l=$Z_{31}$] (11,0) coordinate(G);
    \draw (B) to[R, l=$Z_{23}$] (11,0); 
    \node at (11, -0.3) {C};
\end{circuitikz}
\end{center}

\textbf{રૂપાંતરણ સમીકરણો}:
\begin{align*}
    Z_{12} &= \frac{Z_1Z_2 + Z_2Z_3 + Z_3Z_1}{Z_3} \\
    Z_{23} &= \frac{Z_1Z_2 + Z_2Z_3 + Z_3Z_1}{Z_1} \\
    Z_{31} &= \frac{Z_1Z_2 + Z_2Z_3 + Z_3Z_1}{Z_2}
\end{align*}

\begin{mnemonicbox}
"બધા ગુણનનો સરવાળો વિભાજિત સામેના દ્વારા"
\end{mnemonicbox}
\end{solutionbox}

\section*{પ્રશ્ન 2(b) [4 ગુણ]}
\questionmarks{2(b)}{4}{ઓપન સર્કિટ ઇમ્પીડન્સ પેરામીટર (Z પેરામીટર) સમજાવો.}

\begin{solutionbox}
\textbf{Z-પેરામીટર્સ}: આને ઓપન-સર્કિટ ઇમ્પીડન્સ પેરામીટર્સ પણ કહેવામાં આવે છે.

\textbf{વ્યાખ્યાયિત સમીકરણો}:
\begin{align*}
    V_1 &= Z_{11}I_1 + Z_{12}I_2 \\
    V_2 &= Z_{21}I_1 + Z_{22}I_2
\end{align*}
મેટ્રિક્સ ફોર્મમાં:
\[ \begin{bmatrix} V_1 \\ V_2 \end{bmatrix} = \begin{bmatrix} Z_{11} & Z_{12} \\ Z_{21} & Z_{22} \end{bmatrix} \begin{bmatrix} I_1 \\ I_2 \end{bmatrix} \]

\textbf{પેરામીટર વ્યાખ્યાઓ} (જ્યારે અન્ય પોર્ટ ખુલ્લું હોય, $I=0$):
\begin{tabulary}{\linewidth}{@{}|L|L|L|@{}}
    \hline
    \textbf{પેરામીટર} & \textbf{નામ} & \textbf{સૂત્ર} \\
    \hline
    $Z_{11}$ & ઇનપુટ ઇમ્પીડન્સ & $V_1/I_1 |_{I_2=0}$ \\
    \hline
    $Z_{12}$ & રિવર્સ ટ્રાન્સફર ઇમ્પીડન્સ & $V_1/I_2 |_{I_1=0}$ \\
    \hline
    $Z_{21}$ & ફોરવર્ડ ટ્રાન્સફર ઇમ્પીડન્સ & $V_2/I_1 |_{I_2=0}$ \\
    \hline
    $Z_{22}$ & આઉટપુટ ઇમ્પીડન્સ & $V_2/I_2 |_{I_1=0}$ \\
    \hline
\end{tabulary}

\begin{mnemonicbox}
"Vs તે Zs ગુણ્યા Is"
\end{mnemonicbox}
\end{solutionbox}

\section*{પ્રશ્ન 2(c) [7 ગુણ]}
\questionmarks{2(c)}{7}{સિમેટ્રિકલ T-type નેટવર્ક માટે કેરેક્ટેરિસ્ટિક ઇમ્પીડન્સ ($Z_{0T}$) નું સૂત્ર મેળવો.}

\begin{solutionbox}
\textbf{સિમેટ્રિકલ T-નેટવર્ક}:

\begin{center}
\begin{circuitikz}[american]
    \draw (0,2) to[short, o-] (1,2) to[R, l=$Z_1/2$] (3,2) coordinate(Mid);
    \draw (Mid) to[R, l=$Z_1/2$] (5,2) to[short, -o] (6,2);
    \draw (Mid) to[R, l=$Z_2$] (3,0) coordinate(G);
    \draw (0,0) to[short, o-] (6,0) to[short, -o] (6,0);
    
    % Termination
    \draw (6,2) -- (7,2) to[R, l=$Z_{0T}$] (7,0) -- (6,0);
    
    % Input Z label
    \draw[->] (-0.5, 1) -- (0.5, 1) node[midway, above]{$Z_{in}=Z_{0T}$};
\end{circuitikz}
\end{center}

\textbf{વ્યુત્પત્તિ}:
\begin{enumerate}
    \item કેરેક્ટેરિસ્ટિક ઇમ્પીડન્સ $Z_{0T}$ માં ટર્મિનેટ થયેલ સિમેટ્રિકલ નેટવર્ક માટે, ઇનપુટ ઇમ્પીડન્સ પણ $Z_{0T}$ હોય છે.
    \item ટર્મિનલ્સ A-B થી જોતા ઇનપુટ ઇમ્પીડન્સ:
    \[ Z_{in} = \frac{Z_1}{2} + \left( Z_2 \parallel \left(\frac{Z_1}{2} + Z_{0T}\right) \right) \]
    \item $Z_{in} = Z_{0T}$ સેટ કરતા અને ઉકેલતા:
    \[ Z_{0T} = \sqrt{\frac{Z_1^2}{4} + Z_1Z_2} \]
\end{enumerate}

\begin{mnemonicbox}
"Z1 અને તેની સાથે જોડાયેલા Z1 ના વર્ગમૂળ"
\end{mnemonicbox}
\end{solutionbox}

\section*{પ્રશ્ન 2(a) OR [3 ગુણ]}
\questionmarks{2(a) OR}{3}{$\pi$-type નેટવર્કને T-type નેટવર્કમાં કન્વર્ટ કરવા માટે સમીકરણો મેળવો.}

\begin{solutionbox}
\textbf{$\pi$ થી T રૂપાંતરણ}:

\begin{center}
\begin{circuitikz}[american, scale=0.7]
    % Pi network
    \draw (0,2) node[left]{A} to[short, o-] (1,2) coordinate(A) to[R, l=$Z_{12}$] (5,2) coordinate(B) to[short, -o] (6,2) node[right]{B};
    \draw (A) to[R, l=$Z_{31}$] (3,0) coordinate(G);
    \draw (B) to[R, l=$Z_{23}$] (3,0); 
    \node at (3, -0.3) {C};

    \draw[->, thick] (6.5, 1) -- (7.5, 1);

    % T network
    \draw (8,2) node[left]{A} to[short, o-] (9,2) to[R, l=$Z_1$] (11,2) coordinate(C);
    \draw (C) to[R, l=$Z_2$] (13,2) to[short, -o] (14,2) node[right]{B};
    \draw (C) to[R, l=$Z_3$] (11,0) node[below]{C};
\end{circuitikz}
\end{center}

\textbf{રૂપાંતરણ સમીકરણો}:
\begin{align*}
    Z_1 &= \frac{Z_{12}Z_{31}}{Z_{12} + Z_{23} + Z_{31}} \\
    Z_2 &= \frac{Z_{12}Z_{23}}{Z_{12} + Z_{23} + Z_{31}} \\
    Z_3 &= \frac{Z_{31}Z_{23}}{Z_{12} + Z_{23} + Z_{31}}
\end{align*}

\begin{mnemonicbox}
"આસન્ન જોડીઓના ગુણાકાર વિભાજિત બધાના સરવાળા દ્વારા"
\end{mnemonicbox}
\end{solutionbox}

\section*{પ્રશ્ન 2(b) OR [4 ગુણ]}
\questionmarks{2(b) OR}{4}{એડમિટન્સ પેરામીટર (Y પેરામીટર) સમજાવો.}

\begin{solutionbox}
\textbf{Y-પેરામીટર્સ}: આને શોર્ટ-સર્કિટ એડમિટન્સ પેરામીટર્સ પણ કહેવામાં આવે છે.

\textbf{વ્યાખ્યાયિત સમીકરણો}:
\begin{align*}
    I_1 &= Y_{11}V_1 + Y_{12}V_2 \\
    I_2 &= Y_{21}V_1 + Y_{22}V_2
\end{align*}
મેટ્રિક્સ ફોર્મમાં:
\[ \begin{bmatrix} I_1 \\ I_2 \end{bmatrix} = \begin{bmatrix} Y_{11} & Y_{12} \\ Y_{21} & Y_{22} \end{bmatrix} \begin{bmatrix} V_1 \\ V_2 \end{bmatrix} \]

\textbf{પેરામીટર વ્યાખ્યાઓ} (જ્યારે અન્ય પોર્ટ શોર્ટેડ હોય, $V=0$):
\begin{tabulary}{\linewidth}{@{}|L|L|L|@{}}
    \hline
    \textbf{પેરામીટર} & \textbf{નામ} & \textbf{સૂત્ર} \\
    \hline
    $Y_{11}$ & ઇનપુટ એડમિટન્સ & $I_1/V_1 |_{V_2=0}$ \\
    \hline
    $Y_{12}$ & રિવર્સ ટ્રાન્સફર એડમિટન્સ & $I_1/V_2 |_{V_1=0}$ \\
    \hline
    $Y_{21}$ & ફોરવર્ડ ટ્રાન્સફર એડમિટન્સ & $I_2/V_1 |_{V_2=0}$ \\
    \hline
    $Y_{22}$ & આઉટપુટ એડમિટન્સ & $I_2/V_2 |_{V_1=0}$ \\
    \hline
\end{tabulary}

\begin{mnemonicbox}
"Is તે Ys ગુણ્યા Vs"
\end{mnemonicbox}
\end{solutionbox}

\section*{પ્રશ્ન 2(c) OR [7 ગુણ]}
\questionmarks{2(c) OR}{7}{સિમેટ્રિકલ $\pi$-type નેટવર્ક માટે કેરેક્ટેરિસ્ટિક ઇમ્પીડન્સ ($Z_{0\pi}$) નું સૂત્ર મેળવો.}

\begin{solutionbox}
\textbf{સિમેટ્રિકલ $\pi$-નેટવર્ક}:

\begin{center}
\begin{circuitikz}[american]
    \draw (0,2) to[short, o-] (1,2) to[short] (1,3) to[R, l=$Z_1$] (5,3) to[short] (5,2) to[short, -o] (6,2);
    \draw (1,2) to[R, l=$2Z_2$] (1,0);
    \draw (5,2) to[R, l=$2Z_2$] (5,0);
    \draw (0,0) to[short, o-] (6,0) to[short, -o] (6,0);
    \node at (1, -0.5) {\gu{નોંધ: શંટ} $2Z_2$ \gu{છે}};
\end{circuitikz}
\end{center}

\textbf{વ્યુત્પત્તિ}:
\begin{enumerate}
    \item સિમેટ્રિકલ $\pi$-નેટવર્ક માટે, શંટ આર્મ્સમાં એડમિટન્સ $Y_1$ બે સરખા ભાગમાં વહેંચાય છે (અહીં $Y_3 = Y_1/2$ અથવા $Z_{shunt} = 2Z_3$).
    \item ઇમેજ ઇમ્પીડન્સ મેચિંગ માટે $Z_{0\pi} = \sqrt{Z_{SC}Z_{OC}}$ વાપરી શકાય.
    \item MDX મુજબ:
    \[ Z_{0\pi} = \sqrt{\frac{2Z_1Z_3}{Z_1 + 2Z_3}} \]
\end{enumerate}

\begin{mnemonicbox}
"પાઈનો ઇમ્પીડન્સ તે જુએ છે તેનો જ્યામિતીય મધ્યવર્તી"
\end{mnemonicbox}
\end{solutionbox}

\section*{પ્રશ્ન 3(a) [3 ગુણ]}
\questionmarks{3(a)}{3}{ડ્યુઆલિટીનો સિદ્ધાંત સમજાવો.}

\begin{solutionbox}
\textbf{ડ્યુઆલિટીનો સિદ્ધાંત}: દરેક ઇલેક્ટ્રીકલ નેટવર્ક માટે, એક ડ્યુઅલ નેટવર્ક અસ્તિત્વમાં છે જેનું વર્તન સમાન છે પરંતુ તત્વો બદલાયેલા છે. જો એક વિધાન એક સર્કિટ માટે સાચું હોય, તો તેનું ડ્યુઅલ વિધાન ડ્યુઅલ સર્કિટ માટે સાચું છે.

\textbf{ડ્યુઅલ તત્વ જોડીઓ}:
\begin{tabulary}{\linewidth}{@{}|L|L|@{}}
    \hline
    \textbf{મૂળ સર્કિટ} & \textbf{ડ્યુઅલ સર્કિટ} \\
    \hline
    વોલ્ટેજ ($V$) & કરંટ ($I$) \\
    રેસિસ્ટન્સ ($R$) & કંડક્ટન્સ ($G$) \\
    ઇન્ડક્ટન્સ ($L$) & કેપેસિટન્સ ($C$) \\
    શ્રેણી જોડાણ & સમાંતર જોડાણ \\
    KVL & KCL \\
    ઓપન સર્કિટ & શોર્ટ સર્કિટ \\
    \hline
\end{tabulary}

\begin{mnemonicbox}
"શ્રેણીથી સમાંતર, સ્ત્રોત બદલે ડ્યુઅલ, V બને I અને I બને V"
\end{mnemonicbox}
\end{solutionbox}

\section*{પ્રશ્ન 3(b) [4 ગુણ]}
\questionmarks{3(b)}{4}{થેવેનિનનો પ્રમેય જણાવો અને સમજાવો.}

\begin{solutionbox}
\textbf{થેવેનિનનો પ્રમેય}: કોઈપણ લીનીયર બે-ટર્મિનલ નેટવર્કને શ્રેણીમાં વોલ્ટેજ સ્ત્રોત ($V_{TH}$) અને રેસિસ્ટન્સ ($R_{TH}$) ધરાવતા સમકક્ષ સર્કિટથી બદલી શકાય છે.

\begin{center}
\begin{circuitikz}[american]
    \draw (0,0) node[draw, minimum width=2cm, minimum height=1.5cm] {\gu{લીનીયર નેટવર્ક}} to[short, -o] (2,0.5) node[right]{A};
    \draw (1, -0.75) to[short, -o] (2,-0.75) node[right]{B};
    
    \draw[->, thick] (3, 0) -- (4, 0) node[midway, above]{\gu{સમકક્ષ}};
    
    \draw (5,-0.75) to[V, l=$V_{TH}$] (5,0.5) to[R, l=$R_{TH}$] (7,0.5) to[short, -o] (8,0.5) node[right]{A};
    \draw (5,-0.75) to[short, -o] (8,-0.75) node[right]{B};
\end{circuitikz}
\end{center}

\textbf{પ્રક્રિયા}:
\begin{enumerate}
    \item $V_{TH}$ શોધો: ટર્મિનલ્સ A-B વચ્ચેનો ઓપન-સર્કિટ વોલ્ટેજ.
    \item $R_{TH}$ શોધો: બધા સ્ત્રોતોને નિષ્ક્રિય કરીને A-B થી દેખાતો રેસિસ્ટન્સ.
\end{enumerate}

\begin{mnemonicbox}
"વોલ્ટેજ માટે ખુલ્લું, રેસિસ્ટન્સ માટે મૃત"
\end{mnemonicbox}
\end{solutionbox}

\section*{પ્રશ્ન 3(c) [7 ગુણ]}
\questionmarks{3(c)}{7}{ઉદાહરણ સાથે KCL અને KVL જણાવો અને સમજાવો.}

\begin{solutionbox}
\textbf{કિરચોફનો કરંટ નિયમ (KCL)}: નોડમાં પ્રવેશતા કરંટનો સરવાળો નોડથી બહાર નીકળતા કરંટના સરવાળા બરાબર છે.
\[ \sum I_{in} = \sum I_{out} \]

\textit{ઉદાહરણ}:
\begin{center}
\begin{circuitikz}
    \draw (0,0) node[circ, label=above:Node]{} coordinate(N);
    \draw (N) -- ++(-1.5, 1) node[left]{$I_1$} [<-];
    \draw (N) -- ++(-1.5, -1) node[left]{$I_2$} [->];
    \draw (N) -- ++(1.5, 0) node[right]{$I_3$} [->];
\end{circuitikz}
\end{center}
સમીકરણ: $I_1 = I_2 + I_3$

\noindent\rule{\linewidth}{0.4pt}

\textbf{કિરચોફનો વોલ્ટેજ નિયમ (KVL)}: કોઈપણ બંધ લૂપ ફરતે વોલ્ટેજ ડ્રોપનો સરવાળો શૂન્ય છે.
\[ \sum V = 0 \]

\textit{ઉદાહરણ}:
\begin{center}
\begin{circuitikz}[american]
    \draw (0,0) to[V, l=$V_S$] (0,2) to[R, l=$R_1$, v=$V_1$] (3,2) to[R, l=$R_2$, v=$V_2$] (3,0) -- (0,0);
\end{circuitikz}
\end{center}
સમીકરણ: $V_S - I R_1 - I R_2 = 0$

\begin{mnemonicbox}
"નોડ પર કરંટનો સરવાળો શૂન્ય, લૂપ આસપાસ વોલ્ટેજના પણ"
\end{mnemonicbox}
\end{solutionbox}

\section*{પ્રશ્ન 3(a) OR [3 ગુણ]}
\questionmarks{3(a) OR}{3}{મેશ એનાલિસિસ દ્વારા નેટવર્કનું સોલ્યુશન સમજાવો.}

\begin{solutionbox}
\textbf{મેશ એનાલિસિસ}: સર્કિટમાં વોલ્ટેજ ડ્રોપ અને કરંટ શોધવા માટે મેશ કરંટનો ઉપયોગ કરીને ઇલેક્ટ્રિકલ સર્કિટનું વિશ્લેષણ કરવાની પદ્ધતિ છે. તે KVL પર આધારિત છે.

\textbf{ઉદાહરણ સર્કિટ}:
\begin{center}
\begin{circuitikz}[american, scale=0.8]
    \draw (0,0) to[V, l=$V_1$] (0,2) to[R, l=$R_1$] (2,2) coordinate(M) to[R, l=$R_3$] (4,2) to[V, l=$V_2$] (4,0) -- (0,0);
    \draw (M) to[R, l=$R_2$] (2,0);
    
    % Loops
    \draw[->] (0.5, 0.5) arc (-45:225:0.5) node[midway, left]{$I_1$};
    \draw[->] (2.5, 0.5) arc (-45:225:0.5) node[midway, left]{$I_2$};
    \node at (1, 1) {\gu{મેશ 1}};
    \node at (3, 1) {\gu{મેશ 2}};
\end{circuitikz}
\end{center}

\textbf{પ્રક્રિયા}:
\begin{enumerate}
    \item મેશ (મૂળભૂત લૂપ્સ) ઓળખો.
    \item દરેક મેશને મેશ કરંટ ($I_1, I_2$) સોંપો.
    \item દરેક મેશ માટે KVL સમીકરણ લખો.
    \item મેશ કરંટ શોધવા માટે સમીકરણો ઉકેલો.
\end{enumerate}

\begin{mnemonicbox}
"સોંપો, KVL લાગુ કરો, ગોઠવો અને ઉકેલો"
\end{mnemonicbox}
\end{solutionbox}

\section*{પ્રશ્ન 3(b) OR [4 ગુણ]}
\questionmarks{3(b) OR}{4}{નોર્ટનનો પ્રમેય જણાવો અને સમજાવો.}

\begin{solutionbox}
\textbf{નોર્ટનનો પ્રમેય}: કોઈપણ લીનીયર, બાયલેટરલ નેટવર્કને સમાંતરમાં કરંટ સ્ત્રોત ($I_N$) અને રેસિસ્ટન્સ ($R_N$) ધરાવતા સમકક્ષ સર્કિટથી બદલી શકાય છે.

\begin{center}
\begin{circuitikz}[american]
    \draw (0,0) node[draw, minimum width=2cm, minimum height=1.5cm] {\gu{લીનીયર નેટવર્ક}} to[short, -o] (2,0.5) node[right]{A};
    \draw (1, -0.75) to[short, -o] (2,-0.75) node[right]{B};
    
    \draw[->, thick] (3, 0) -- (4, 0) node[midway, above]{\gu{સમકક્ષ}};
    
    \draw (5,-0.75) to[short, -o] (8,-0.75) node[right]{B};
    \draw (5,-0.75) to[I, l=$I_N$] (5,0.5) to[short, -o] (8,0.5) node[right]{A};
    \draw (7,0.5) to[R, l=$R_N$] (7,-0.75);
\end{circuitikz}
\captionof{figure}{નોર્ટન સમકક્ષ}
\end{center}

\begin{mnemonicbox}
"કરંટ માટે શોર્ટ, રેસિસ્ટન્સ માટે મૃત"
\end{mnemonicbox}
\end{solutionbox}

\section*{પ્રશ્ન 3(c) OR [7 ગુણ]}
\questionmarks{3(c) OR}{7}{મહત્તમ પાવર ટ્રાન્સફર પ્રમેય જણાવો અને સમજાવો. મહત્તમ પાવર ટ્રાન્સફર માટેની શરત મેળવો.}

\begin{solutionbox}
\textbf{મહત્તમ પાવર ટ્રાન્સફર પ્રમેય}: જ્યારે લોડ રેસિસ્ટન્સ ($R_L$) સ્ત્રોતના આંતરિક રેસિસ્ટન્સ ($R_{TH}$) જેટલો થાય છે ત્યારે DC સ્ત્રોત લોડને મહત્તમ પાવર પહોંચાડે છે.

\begin{center}
\begin{circuitikz}[american]
    \draw (0,0) to[V, l=$V_{TH}$] (0,2) to[R, l=$R_{TH}$] (2,2) to[R, l=$R_L$] (2,0) -- (0,0);
\end{circuitikz}
\end{center}

\textbf{વ્યુત્પત્તિ}:
\begin{enumerate}
    \item લોડ પાવર: $P_L = I^2 R_L$
    \item $R_L$ ની સાપેક્ષમાં વિકલન કરતા અને શૂન્ય સેટ કરતા:
    \[ \frac{dP_L}{dR_L} = 0 \implies R_L = R_{TH} \]
\end{enumerate}

\textbf{મહત્તમ પાવર સૂત્ર}:
\[ P_{max} = \frac{V_{TH}^2}{4R_{TH}} \]

\begin{mnemonicbox}
"મહત્તમ કરવા માટે મેચ કરો"
\end{mnemonicbox}
\end{solutionbox}

\section*{પ્રશ્ન 4(a) [3 ગુણ]}
\questionmarks{4(a)}{3}{કોઇલ માટે Q ફેક્ટરનું સમીકરણ મેળવો.}

\begin{solutionbox}
\textbf{Q ફેક્ટર}: કોઇલના રિએક્ટન્સ અને રેસિસ્ટન્સનો ગુણોત્તર છે.

\textbf{વ્યુત્પત્તિ}:
\begin{enumerate}
    \item કોઇલ ઇમ્પીડન્સ: $Z = R + j\omega L$
    \item $Q = \frac{\omega L}{R} = \frac{2\pi f L}{R}$
\end{enumerate}

\begin{mnemonicbox}
"ગુણવત્તા બરાબર રિએક્ટન્સ ભાગ્યા રેસિસ્ટન્સ"
\end{mnemonicbox}
\end{solutionbox}

\section*{પ્રશ્ન 4(b) [4 ગુણ]}
\questionmarks{4(b)}{4}{સમાંતર RLC સર્કિટ માટે રેઝોનન્ટ ફ્રિકવન્સીનું સૂત્ર મેળવો.}

\begin{solutionbox}
\textbf{સમાંતર RLC સર્કિટ}:

\begin{center}
\begin{circuitikz}[american, scale=0.8]
    \draw (0,2) to[short, o-] (1,2) -- (4,2) to[short, -o] (5,2);
    \draw (0,0) to[short, o-] (1,0) -- (4,0) to[short, -o] (5,0);
    \draw (1,2) to[R, l=$R$] (1,0);
    \draw (2.5,2) to[C, l=$C$] (2.5,0);
    \draw (4,2) to[L, l=$L$] (4,0);
\end{circuitikz}
\end{center}

\textbf{વ્યુત્પત્તિ}:
\begin{enumerate}
    \item રેઝોનન્સ પર, સુસેપ્ટન્સ શૂન્ય હોય છે:
    \[ \omega C - \frac{1}{\omega L} = 0 \]
    \[ \omega^2 = \frac{1}{LC} \implies f_r = \frac{1}{2\pi\sqrt{LC}} \]
\end{enumerate}

\begin{mnemonicbox}
"એક ભાગ્યા બે પાઈ ગુણ્યા LC નું વર્ગમૂળ"
\end{mnemonicbox}
\end{solutionbox}

\section*{પ્રશ્ન 4(c) [7 ગુણ]}
\questionmarks{4(c)}{7}{કપલ્ડ સર્કિટના પ્રકારો જરૂરી આકૃતિ સાથે લખો અને આયર્ન કોર ટ્રાન્સફોર્મર સમજાવો.}

\begin{solutionbox}
\textbf{કપલ્ડ સર્કિટના પ્રકારો}: ડાયરેક્ટ, કેપેસિટીવ, ઇન્ડક્ટિવ, રેઝિસ્ટિવ.

\textbf{આયર્ન કોર ટ્રાન્સફોર્મર}:

\begin{center}
\begin{circuitikz}[american]
    \draw (0,0) to[V, l=$V_1$] (0,2) -- (1,2) to[L, l=$N_1$, name=L1] (1,0) -- (0,0);
    \draw (4,0) to[V, l=$V_2$] (4,2) -- (3,2) to[L, l=$N_2$, name=L2] (3,0) -- (4,0);
    % Iron Core Lines
    \draw[thick] (1.6, 0.2) -- (1.6, 1.8);
    \draw[thick] (1.8, 0.2) -- (1.8, 1.8);
    \node at (2, 2.2) {\gu{આયર્ન કોર}};
\end{circuitikz}
\end{center}

\textbf{સમજૂતી}:
\begin{itemize}
    \item \textbf{સિદ્ધાંત}: મ્યુચ્યુઅલ ઇન્ડક્શન.
    \item \textbf{સમીકરણ}: $\frac{V_2}{V_1} = \frac{N_2}{N_1}$
    \item \textbf{ઉપયોગ}: પાવર ટ્રાન્સમિશન.
\end{itemize}

\begin{mnemonicbox}
"પ્રાથમિક ઉત્તેજિત કરે, કોર વહન કરે, ગૌણ પહોંચાડે"
\end{mnemonicbox}
\end{solutionbox}

\section*{પ્રશ્ન 4(a) OR [3 ગુણ]}
\questionmarks{4(a) OR}{3}{કેપેસિટર માટે Q ફેક્ટરનું સમીકરણ મેળવો.}

\begin{solutionbox}
\textbf{વ્યુત્પત્તિ}:
\begin{enumerate}
    \item $Q = \frac{1}{\omega C R} = \frac{1}{2\pi f C R}$
\end{enumerate}

\begin{mnemonicbox}
"ગુણવત્તા બરાબર એક ભાગ્યા રેસિસ્ટન્સ ગુણ્યા રિએક્ટન્સ"
\end{mnemonicbox}
\end{solutionbox}

\section*{પ્રશ્ન 4(b) OR [4 ગુણ]}
\questionmarks{4(b) OR}{4}{શ્રેણી રેઝોનન્સ સર્કિટ માટે રેઝોનન્સ ફ્રિકવન્સીનું સમીકરણ મેળવો.}

\begin{solutionbox}
\textbf{શ્રેણી RLC સર્કિટ}:
\begin{center}
\begin{circuitikz}[american]
    \draw (0,2) to[short, o-] (0.5,2) to[R, l=$R$] (2.5,2) to[L, l=$L$] (4.5,2) to[C, l=$C$] (6.5,2) to[short, -o] (7,2);
\end{circuitikz}
\end{center}

\textbf{વ્યુત્પત્તિ}:
\begin{enumerate}
    \item રેઝોનન્સ શરત: $X_L = X_C \implies f_r = \frac{1}{2\pi\sqrt{LC}}$
\end{enumerate}

\begin{mnemonicbox}
"એક ભાગ્યા બે પાઈ ગુણ્યા LC નું વર્ગમૂળ"
\end{mnemonicbox}
\end{solutionbox}

\section*{પ્રશ્ન 4(c) OR [7 ગુણ]}
\questionmarks{4(c) OR}{7}{ચુંબકીય રીતે જોડાયેલ કોઇલની જોડી વચ્ચેના કોએફિશિયન્ટ ઓફ કપલિંગ માટે સમીકરણ મેળવો.}

\begin{solutionbox}
\textbf{કોએફિશિયન્ટ ઓફ કપલિંગ (k)}:

\begin{center}
\begin{circuitikz}[american]
    \draw (0,0) to[L, l=$L_1$] (0,2);
    \draw (2,0) to[L, l=$L_2$] (2,2);
    \draw[<->, dashed] (0.5,1) -- (1.5,1) node[midway, above]{$M$};
\end{circuitikz}
\end{center}

\textbf{સૂત્ર}:
\[ k = \frac{M}{\sqrt{L_1 L_2}} \]

\begin{mnemonicbox}
"મ્યુચ્યુઅલ ભાગ્યા પ્રોડક્ટનું વર્ગમૂળ"
\end{mnemonicbox}
\end{solutionbox}

\section*{પ્રશ્ન 5(a) [3 ગુણ]}
\questionmarks{5(a)}{3}{નેપર અને dB વ્યાખ્યાયિત કરો. નેપર અને dB વચ્ચેનો સંબંધ સ્થાપિત કરો.}

\begin{solutionbox}
\textbf{વ્યાખ્યાઓ}:
\begin{itemize}
    \item \textbf{ નેપર (Np)}: નેચરલ લઘુગણક ($ln$) પર આધારિત એટેન્યુએશન એકમ.
    \item \textbf{ડેસિબેલ (dB)}: સામાન્ય લઘુગણક ($\log_{10}$) પર આધારિત એકમ.
\end{itemize}

\textbf{સંબંધ}:
\[ 1 \text{ Neper} = 8.686 \text{ dB} \]

\begin{mnemonicbox}
"એક નેપર એટલે 8.686 dB"
\end{mnemonicbox}
\end{solutionbox}

\section*{પ્રશ્ન 5(b) [4 ગુણ]}
\questionmarks{5(b)}{4}{વિવિધ પ્રકારના એટેન્યુએટરનું વર્ગીકરણ કરો.}

\begin{solutionbox}
\textbf{વર્ગીકરણ}:
\begin{center}
\begin{tikzpicture}[gtu tree]
    \node [gtu root] {\gu{એટેન્યુએટર્સ}}
        child { 
            node [gtu child] {\gu{અસમપ્રમાણ}} 
            child { node [gtu child] {L-Type} }
            child { node [gtu child] {T-Type} }
            child { node [gtu child] {$\pi$-Type} }
        }
        child { 
            node [gtu child] {\gu{સમપ્રમાણ}}
            child { node [gtu child] {\gu{સંતુલિત}} 
                child { node [gtu child] {H-Type} }
                child { node [gtu child] {O-Type} }
            }
            child { node [gtu child] {\gu{અસંતુલિત}}
                child { node [gtu child] {T-Type} }
                child { node [gtu child] {$\pi$-Type} }
                child { node [gtu child] {\gu{બ્રિજ્ડ}-T} }
                child { node [gtu child] {\gu{લેટીસ}} }
            }
        };
\end{tikzpicture}
\end{center}

\begin{mnemonicbox}
"Tees, Pies અને Ells સિગ્નલોને સારી રીતે ઘટાડે છે"
\end{mnemonicbox}
\end{solutionbox}

\section*{પ્રશ્ન 5(c) [7 ગુણ]}
\questionmarks{5(c)}{7}{નીચે દર્શાવેલ લો-પાસ ફિલ્ટર વિભાગોના કટ-ઓફ ફ્રિકવન્સી અને નોમિનલ ઇમ્પીડન્સ નક્કી કરો.}

\begin{solutionbox}
\textbf{ફિલ્ટર વિભાગો}: $L = 10 \text{ mH}$, $C = 0.1 \mu\text{F}$.

\textbf{ગણતરી}:
\[ f_c = \frac{1}{\pi\sqrt{LC}} \approx 10.06 \text{ kHz} \]
\[ R_0 = \sqrt{\frac{L}{C}} = 316.23 \Omega \]

\begin{mnemonicbox}
"કટ-ઓફ ફ્રિકવન્સી LC ના વર્ગમૂળના વ્યસ્ત છે"
\end{mnemonicbox}
\end{solutionbox}

\section*{પ્રશ્ન 5(a) OR [3 ગુણ]}
\questionmarks{5(a) OR}{3}{કોન્સ્ટન્ટ k ટાઈપ ફિલ્ટરની મર્યાદા સમજાવો.}

\begin{solutionbox}
\textbf{મર્યાદાઓ}:
\begin{enumerate}
    \item \textbf{ઇમ્પીડન્સ મેચિંગ}: $Z_0$ ફ્રિકવન્સી સાથે બદલાય છે.
    \item \textbf{કટ-ઓફ શાર્પનેસ}: કટ-ઓફ પછી ધીમી એટેન્યુએશન.
\end{enumerate}

\begin{mnemonicbox}
"ખરાબ મેચિંગ અને ટ્રાન્ઝિશન વિકૃતિમાં પરિણમે છે"
\end{mnemonicbox}
\end{solutionbox}

\section*{પ્રશ્ન 5(b) OR [4 ગુણ]}
\questionmarks{5(b) OR}{4}{T-type કોન્સ્ટન્ટ k હાઈ પાસ ફિલ્ટર માટે કટ-ઓફ ફ્રિકવન્સીનું સમીકરણ મેળવો.}

\begin{solutionbox}
\textbf{સમીકરણ}:
\[ f_c = \frac{1}{4\pi\sqrt{LC}} \]

\begin{mnemonicbox}
"હાઈ પાસ કટ કરે ફ્રિકવન્સી એક ભાગ્યા ચાર પાઈ L-C"
\end{mnemonicbox}
\end{solutionbox}

\section*{પ્રશ્ન 5(c) OR [7 ગુણ]}
\questionmarks{5(c) OR}{7}{દરેક માટે વ્યાખ્યાઓ અને લાક્ષણિકતાઓ ગ્રાફનો ઉપયોગ કરીને ફિલ્ટર્સનું વર્ગીકરણ કરો.}

\begin{solutionbox}
\textbf{ફિલ્ટર્સનું વર્ગીકરણ}: LPF, HPF, BPF, BSF, APF.

\textbf{લાક્ષણિક આલેખ}:
\begin{center}
\begin{tikzpicture}[scale=0.5]
    % LPF
    \begin{scope}
        \draw[->] (0,0) -- (2.5,0) node[right]{$f$};
        \draw[->] (0,0) -- (0,2) node[above]{\gu{ગેઈન}};
        \draw[thick, blue] (0,1.5) -- (1,1.5) -- (1,0) -- (2.5,0);
        \node at (1.25,-0.7) {LPF};
    \end{scope}

    % HPF
    \begin{scope}[shift={(4,0)}]
        \draw[->] (0,0) -- (2.5,0) node[right]{$f$};
        \draw[->] (0,0) -- (0,2) node[above]{\gu{ગેઈન}};
        \draw[thick, blue] (0,0) -- (1,0) -- (1,1.5) -- (2.5,1.5);
        \node at (1.25,-0.7) {HPF};
    \end{scope}
    
    % BPF
    \begin{scope}[shift={(8,0)}]
        \draw[->] (0,0) -- (2.5,0) node[right]{$f$};
        \draw[->] (0,0) -- (0,2) node[above]{\gu{ગેઈન}};
        \draw[thick, blue] (0,0) -- (0.7,0) -- (0.7,1.5) -- (1.8,1.5) -- (1.8,0) -- (2.5,0);
        \node at (1.25,-0.7) {BPF};
    \end{scope}
    
    % BSF
    \begin{scope}[shift={(12,0)}]
        \draw[->] (0,0) -- (2.5,0) node[right]{$f$};
        \draw[->] (0,0) -- (0,2) node[above]{\gu{ગેઈન}};
        \draw[thick, blue] (0,1.5) -- (0.7,1.5) -- (0.7,0) -- (1.8,0) -- (1.8,1.5) -- (2.5,1.5);
        \node at (1.25,-0.7) {BSF};
    \end{scope}
\end{tikzpicture}
\end{center}

\begin{mnemonicbox}
"લો-હાઈ-બેન્ડ-સ્ટોપ સિગ્નલ્સને પરફેક્ટ બનાવે છે"
\end{mnemonicbox}
\end{solutionbox}

\end{document}

