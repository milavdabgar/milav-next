\documentclass{article}

% content/resources/templates/preamble.tex
\usepackage[margin=0.6in]{geometry}
\author{Milav Dabgar}
\usepackage{amsmath,amssymb,amsthm}
\usepackage{booktabs}
\usepackage{multirow}
\usepackage{xcolor}
\usepackage{tcolorbox}
\tcbuselibrary{breakable,skins}
\usepackage[colorlinks=true,linkcolor=blue]{hyperref}
\usepackage{titlesec}
\usepackage{enumitem}
\usepackage{tikz}
\usepackage{pgfplots}
\usepackage{circuitikz}
\usepackage[version=4]{mhchem}
\usepackage{longtable}
\usepackage{array}
\usepackage{float}
\usepackage{caption}
\usepackage{listings}

\lstset{
  basicstyle=\small\ttfamily,
  breaklines=true,
  breakatwhitespace=false,
  postbreak=\mbox{\textcolor{red}{$\hookrightarrow$}\space},
  float=false,
  numbers=left,
  numberstyle=\tiny\color{gray},
  numbersep=10pt,
  xleftmargin=2em,
  keywordstyle=\color{blue},
  commentstyle=\color{green!60!black},
  stringstyle=\color{purple},
  backgroundcolor=\color{gray!5},
  showstringspaces=false,
  tabsize=2,
  captionpos=b,
  keepspaces=true,
  columns=flexible
}

\pgfplotsset{compat=1.18}
\usetikzlibrary{shapes,arrows,positioning,calc,patterns,decorations.pathmorphing,decorations.markings,arrows.meta}

% Color scheme
\definecolor{headcolor}{RGB}{0,102,204}
\definecolor{keycolor}{RGB}{220,20,60}
\definecolor{solutioncolor}{RGB}{34,139,34}
\definecolor{mnemoniccolor}{RGB}{148,0,211}
\definecolor{codecolor}{RGB}{0,0,100}

% Spacing
\setlength{\parskip}{3pt}
\setlist[itemize]{nosep}
\setlist[enumerate]{nosep}

% Title formatting
\titleformat{\section}{\Large\bfseries\color{headcolor}}{\thesection}{1em}{}
\titleformat{\subsection}{\large\bfseries\color{headcolor}}{\thesubsection}{1em}{}

% Pandoc tightlist compatibility
\providecommand{\tightlist}{%
  \setlength{\itemsep}{0pt}\setlength{\parskip}{0pt}}

% Pandoc longtable compatibility
\newcounter{none}
\def\thenone{}


% content/resources/templates/gujarati-boxes.tex
\usepackage{fontspec}
\usepackage{polyglossia}

% Set Gujarati as main language (document is primarily in Gujarati)
% Note: gloss-gujarati.ldf doesn't exist in polyglossia, but it will use hyphenation patterns
\setdefaultlanguage{gujarati}
\setotherlanguage{english}

% Configure Gujarati font properly
% Use Language=Default to prevent polyglossia from trying to add language-specific features
% that don't exist for Gujarati, which causes "empty feature" warnings
\newfontfamily\gujaratifont[Script=Gujarati,AutoFakeBold=2.5,AutoFakeSlant=0.3]{Noto Sans Gujarati}
\setmainfont[Script=Gujarati,AutoFakeBold=2.5,AutoFakeSlant=0.3]{Noto Sans Gujarati}
% Use Noto Sans Gujarati for monospace to support Gujarati in text
\setmonofont[Scale=0.9]{Noto Sans Gujarati}

% Configure English to use the same font
\newfontfamily\englishfont[Script=Gujarati,AutoFakeBold=2.5,AutoFakeSlant=0.3]{Noto Sans Gujarati}

% Translations for polyglossia
\gappto\captionsgujarati{
  \renewcommand{\tablename}{કોષ્ટક}
  \renewcommand{\figurename}{આકૃતિ}
}

% Helper for TikZ nodes to ensure Gujarati font
\newcommand{\gu}[1]{{\gujaratifont #1}}

% Custom environments
\newtcolorbox{solutionbox}{
    breakable,
    enhanced,
    colback=solutioncolor!5!white,
    colframe=solutioncolor!75!black,
    fonttitle=\bfseries,
    title=જવાબ
}

\newtcolorbox{solutionboxnobreak}{
 colback=solutioncolor!5!white,
 colframe=solutioncolor!75!black,
 fonttitle=\bfseries,
 title=જવાબ
}

\newtcolorbox{keyformula}{
 breakable,
 enhanced,
 colback=keycolor!5!white,
 colframe=keycolor!75!black,
 fonttitle=\bfseries,
 title=રાસાયણિક સમીકરણ/સૂત્ર
}

\newtcolorbox{mnemonicbox}{
 breakable,
 enhanced,
 colback=mnemoniccolor!5!white,
 colframe=mnemoniccolor!75!black,
 fonttitle=\bfseries,
 title=મેમરી ટ્રીક
}


% Custom commands for GTU solutions
% This file defines semantic commands for consistent formatting

% Question command with automatic formatting
\newcommand{\question}[2]{%
  \section*{Question #1}%
  \textbf{#2}%
}

% OR question variant
\newcommand{\questionor}[2]{%
  \section*{Question #1 OR}%
  \textbf{#2}%
}

% Proper table environment with caption
\newenvironment{answertable}[1]{%
  \begin{table}[htbp]
  \centering
  \caption{#1}
}{%
  \end{table}
}

% Proper figure environment for diagrams
\newenvironment{answerdiagram}[1]{%
  \begin{figure}[htbp]
  \centering
  \caption{#1}
}{%
  \end{figure}
}

% Semantic markup for key terms
\newcommand{\keyword}[1]{\textbf{#1}}
\newcommand{\code}[1]{\texttt{#1}}
\newcommand{\classname}[1]{\texttt{#1}}
\newcommand{\methodname}[1]{\texttt{#1}}

% Proper quotation marks
\newcommand{\mnemonic}[1]{``#1''}


\title{Electronic Circuits \& Networks (4331101) - Summer 2024 Solution}
\date{June 6, 2024}

\begin{document}
\maketitle

\questionmarks{1(a)}{3}{યોગ્ય આકૃતિ સાથે નોડ, બ્રાન્ચ અને લૂપ વ્યાખ્યાયિત કરો.}

\begin{solutionbox}
\begin{center}
\begin{circuitikz}[scale=1, transform shape]
    \ctikzset{bipoles/length=1cm}
    \draw (0,2) node[anchor=south] {Node A} to[short, *-*] (2,2) node[anchor=south] {Node B};
    \draw (2,2) to[R, l=Branch 1] (4,2) node[anchor=south] {Node C};
    \draw (2,2) to[R, l=Branch 2] (2,0);
    \draw (0,0) to[short, *-*] (2,0) node[anchor=north] {Node D};
    \draw (2,0) to[short, -*] (4,0);
    \draw (4,2) to[R, l=Branch 3] (4,0);
    \draw (0,2) to[V, l=Branch 4] (0,0);
    
    % Loop indication
    \draw[->, red, thick] (1,1) arc (160:-160:0.5) node[right] {Loop};
\end{circuitikz}
\captionof{figure}{નોડ, બ્રાન્ચ અને લૂપ દર્શાવતી સર્કિટ}
\end{center}

\begin{itemize}
    \item \textbf{નોડ}: એક બિંદુ જ્યાં બે કે વધુ સર્કિટ તત્વો એકબીજા સાથે જોડાય છે. આકૃતિમાં, બિંદુઓ A, B, C, અને D નોડ્સ છે.
    \item \textbf{બ્રાન્ચ}: બે નોડ્સને જોડતું એક સિંગલ એલિમેન્ટ. નોડ્સને જોડતા રેઝિસ્ટર્સ અને વોલ્ટેજ સોર્સ બ્રાન્ચ છે.
    \item \textbf{લૂપ}: સર્કિટમાં કોઈપણ બંધ પાથ જ્યાં કોઈ નોડ એક કરતાં વધુ વખત આવતો નથી. પાથ A-B-D-A એક લૂપ છે.
\end{itemize}

\begin{mnemonicbox}
\mnemonic{NBA circuit: Nodes are junctions, Branches are roads, Loops are Alternate paths}
\end{mnemonicbox}
\end{solutionbox}

\questionmarks{1(b)}{4}{નેટવર્ક માટે "ટ્રી" અને "ગ્રાફ" સમજાવો.}

\begin{solutionbox}
\begin{center}
\begin{tabular}{cc}
\begin{tikzpicture}[auto, node distance=2cm, >=latex]
    \node[gtu state] (A) {A};
    \node[gtu state] (B) [right of=A] {B};
    \node[gtu state] (C) [below of=A] {C};
    \node[gtu state] (D) [right of=C] {D};

    \draw (A) -- (B);
    \draw (A) -- (C);
    \draw (B) -- (D);
    \draw (C) -- (D);
    \draw (B) -- (C);
\end{tikzpicture}
&
\begin{tikzpicture}[auto, node distance=2cm, >=latex]
    \node[gtu state] (A) {A};
    \node[gtu state] (B) [right of=A] {B};
    \node[gtu state] (C) [below of=A] {C};
    \node[gtu state] (D) [right of=C] {D};

    \draw[thick] (A) -- (B);
    \draw[thick] (A) -- (C);
    \draw[thick] (B) -- (D);
    % Dotted lines for removed links
    \draw[dotted] (C) -- (D);
    \draw[dotted] (B) -- (C);
\end{tikzpicture} \\
(a) નેટવર્ક ગ્રાફ & (b) નેટવર્કનું ટ્રી
\end{tabular}
\captionof{figure}{નેટવર્કનો ગ્રાફ અને ટ્રી}
\end{center}

\begin{center}
\begin{tabulary}{\linewidth}{|L|L|L|}
\hline
\textbf{લક્ષણ} & \textbf{ગ્રાફ} & \textbf{ટ્રી} \\ \hline
\textbf{વ્યાખ્યા} & નેટવર્કનું સંપૂર્ણ ટોપોલોજિકલ રજૂઆત & કનેક્ટેડ સબગ્રાફ જેમાં બધા નોડ્સ હોય પણ લૂપ ન હોય \\ \hline
\textbf{તત્વો} & બધી બ્રાન્ચ અને નોડ્સ ધરાવે છે & $N-1$ બ્રાન્ચ ધરાવે છે જ્યાં $N$ નોડ્સની સંખ્યા છે \\ \hline
\textbf{લૂપ્સ} & લૂપ્સ ધરાવે છે & કોઈ લૂપ્સ નથી \\ \hline
\textbf{ઉપયોગ} & સંપૂર્ણ સર્કિટ એનાલિસિસ માટે વપરાય છે & નેટવર્ક ગણતરીઓને સરળ બનાવવા માટે વપરાય છે \\ \hline
\end{tabulary}
\end{center}

\begin{mnemonicbox}
\mnemonic{GRAND Tree: Graph has Routes And Nodes with Detours, Tree has only single Routes}
\end{mnemonicbox}
\end{solutionbox}

\questionmarks{1(c)}{7}{યોગ્ય આકૃતિનો ઉપયોગ કરી "મેષ કરંટ મેથડ" સમજાવો.}

\begin{solutionbox}
\begin{center}
\begin{circuitikz}[auto]
    \draw (0,0) to[V, l=$V_1$] (0,3) to[R, l=$R_1$] (3,3) to[R, l=$R_2$] (6,3) to[V, l=$V_2$] (6,0) -- (0,0);
    \draw (3,3) to[R, l=$R_3$] (3,0);
    
    \draw[->, thick, blue] (1.5, 1.5) arc (120:-120:0.5) node[right] {$I_1$};
    \node at (1.5, 0.5) {Mesh 1};
    
    \draw[->, thick, red] (4.5, 1.5) arc (120:-120:0.5) node[right] {$I_2$};
    \node at (4.5, 0.5) {Mesh 2};
\end{circuitikz}
\captionof{figure}{મેશ એનાલિસિસ સર્કિટ}
\end{center}

\begin{center}
\begin{tabulary}{\linewidth}{|C|L|}
\hline
\textbf{પગલું} & \textbf{વર્ણન} \\ \hline
1 & સર્કિટમાં સ્વતંત્ર મેશ ઓળખો \\ \hline
2 & મેશ કરંટ્સ ($I_1, I_2$, વગેરે) ઘડિયાળના કાંટાની દિશામાં અસાઇન કરો \\ \hline
3 & દરેક મેશ માટે KVL લાગુ કરો \\ \hline
4 & ઇક્વેશન્સ બનાવો: $\Sigma R \cdot I(\text{own}) - \Sigma R \cdot I(\text{adjacent}) = \Sigma V$ \\ \hline
5 & સિમલ્ટેનિયસ ઇક્વેશન્સ ઉકેલો \\ \hline
\end{tabulary}
\end{center}

\begin{itemize}
    \item \textbf{ફાયદો}: બ્રાન્ચ કરંટ મેથડ કરતાં ઓછા ઇક્વેશન્સ
    \item \textbf{ઉપયોગ}: પ્લેનર નેટવર્ક્સ માટે શ્રેષ્ઠ
    \item \textbf{મર્યાદા}: નોન-પ્લેનર નેટવર્ક્સ માટે ઓછું કાર્યક્ષમ
\end{itemize}

\begin{mnemonicbox}
\mnemonic{MIAMI: Meshes Identified, Assign currents, Make equations, Intersection currents calculated, Solve}
\end{mnemonicbox}
\end{solutionbox}

\questionmarks{1(c) OR}{7}{યોગ્ય રેખાકૃતિનો ઉપયોગ કરીને "નોડ પેર વોલ્ટેજ પદ્ધતિ" સમજાવો.}

\begin{solutionbox}
\begin{center}
\begin{circuitikz}[auto]
    \draw (0,0) node[ground]{} to[I, l=$I_s$] (0,3) node[anchor=south]{Node 1 ($V_1$)} to[R, l=$R_1$] (3,3) node[anchor=south]{Node 2 ($V_2$)} to[R, l=$R_2$] (3,0) node[ground]{};
    \draw (3,3) to[R, l=$R_3$] (6,3) node[anchor=south]{Node 3 ($V_3$)} to[R, l=$R_4$] (6,0) node[ground]{};
    \draw (0,3) to[R, l=$R_5$] (3,0); % Diagonal or extra branch
    
    % Node labels
    \node[circ] at (0,3) {};
    \node[circ] at (3,3) {};
    \node[circ] at (6,3) {};
\end{circuitikz}
\captionof{figure}{નોડલ એનાલિસિસ સર્કિટ}
\end{center}

\begin{center}
\begin{tabulary}{\linewidth}{|C|L|}
\hline
\textbf{પગલું} & \textbf{વર્ણન} \\ \hline
1 & રેફરન્સ નોડ (ગ્રાઉન્ડ) પસંદ કરો \\ \hline
2 & બાકીના નોડ્સને નોડ વોલ્ટેજ ($V_1, V_2$, વગેરે) અસાઇન કરો \\ \hline
3 & દરેક નોડ પર KCL લાગુ કરો (રેફરન્સ સિવાય) \\ \hline
4 & ઓહ્મના નિયમનો ઉપયોગ કરીને કરંટ્સને નોડ વોલ્ટેજમાં વ્યક્ત કરો \\ \hline
5 & સિમલ્ટેનિયસ ઇક્વેશન્સ ઉકેલો \\ \hline
\end{tabulary}
\end{center}

\begin{itemize}
    \item \textbf{ફાયદો}: ઘણા મેશવાળા સર્કિટ્સ માટે મેશ મેથડ કરતાં ઓછા ઇક્વેશન્સ
    \item \textbf{ઉપયોગ}: નોન-પ્લેનર સર્કિટ્સ માટે કાર્યક્ષમ
    \item \textbf{મુખ્ય ઇક્વેશન}: $\Sigma G \cdot V(\text{own}) - \Sigma G \cdot V(\text{adjacent}) = \Sigma I$
\end{itemize}

\begin{mnemonicbox}
\mnemonic{GRAND: Ground node fixed, Remaining nodes numbered, Apply KCL, Note voltage differences, Derive solutions}
\end{mnemonicbox}
\end{solutionbox}

\questionmarks{2(a)}{3}{KCL ઉદાહરણ સાથે સમજાવો.}

\begin{solutionbox}
\begin{center}
\begin{circuitikz}[scale=1]
    \node[circ] (A) at (2,2) {};
    \draw (2,2) node[above right] {Node};
    
    \draw[<-] (A) -- (0,2) node[left] {$I_1$};
    \draw[->] (A) -- (2,0) node[below] {$I_2$};
    \draw[->] (A) -- (4,2) node[right] {$I_3$};
    \draw[<-] (A) -- (2,4) node[above] {$I_4$};
\end{circuitikz}
\captionof{figure}{નોડ પર KCL}
\end{center}

\textbf{કિરચોફનો કરંટ લૉ (KCL)}: કોઈપણ નોડ પર પ્રવેશતા અને છોડતા તમામ કરંટ્સનો અલજેબ્રાઇક સરવાળો શૂન્ય હોય છે.

\begin{center}
\begin{tabulary}{\linewidth}{|L|L|}
\hline
\textbf{ગાણિતિક સ્વરૂપ} & \textbf{ઉદાહરણ ઉપયોગ} \\ \hline
$\Sigma I = 0$ & નોડ પર: $I_1 - I_2 - I_3 + I_4 = 0$ \\ \hline
$\Sigma I_{in} = \Sigma I_{out}$ & પ્રવેશતા કરંટ્સ = બહાર નીકળતા કરંટ્સ \\ \hline
\end{tabulary}
\end{center}

\begin{mnemonicbox}
\mnemonic{ZINC: Zero Is Net Current at a node}
\end{mnemonicbox}
\end{solutionbox}

\questionmarks{2(b)}{4}{યોગ્ય આકૃતિનો ઉપયોગ કરી Z-પેરામીટર, Y-પેરામીટર h-પેરામીટર અને ABCD-પેરામીટર સમજાવો.}

\begin{solutionbox}
\begin{center}
\begin{circuitikz}
    \draw (0,0) rectangle (4,3);
    \node at (2,1.5) {Two-Port Network};
    
    \draw (0,2.5) to[short, -o] (-1,2.5) node[left] {$+$};
    \draw (-1,0.5) node[left] {$-$} to[short, o-] (0,0.5);
    \draw (4,2.5) to[short, -o] (5,2.5) node[right] {$+$};
    \draw (5,0.5) node[right] {$-$} to[short, o-] (4,0.5);
    
    \node at (-1, 1.5) {$V_1$};
    \node at (5, 1.5) {$V_2$};
    
    \draw[->] (-0.5, 2.5) -- (0, 2.5) node[midway, above] {$I_1$};
    \draw[<-] (4, 2.5) -- (4.5, 2.5) node[midway, above] {$I_2$};
\end{circuitikz}
\captionof{figure}{ટુ-પોર્ટ નેટવર્ક}
\end{center}

\begin{center}
\begin{tabulary}{\linewidth}{|L|L|L|L|}
\hline
\textbf{પેરામીટર} & \textbf{વ્યાખ્યા} & \textbf{સમીકરણો} & \textbf{ઉપયોગ} \\ \hline
\textbf{Z} & ઇમ્પિડન્સ પેરામીટર્સ & $V_1 = Z_{11}I_1 + Z_{12}I_2$ \newline $V_2 = Z_{21}I_1 + Z_{22}I_2$ & હાઇ ઇમ્પિડન્સ સર્કિટ્સ \\ \hline
\textbf{Y} & એડમિટન્સ પેરામીટર્સ & $I_1 = Y_{11}V_1 + Y_{12}V_2$ \newline $I_2 = Y_{21}V_1 + Y_{22}V_2$ & લો ઇમ્પિડન્સ સર્કિટ્સ \\ \hline
\textbf{h} & હાઇબ્રિડ પેરામીટર્સ & $V_1 = h_{11}I_1 + h_{12}V_2$ \newline $I_2 = h_{21}I_1 + h_{22}V_2$ & ટ્રાન્ઝિસ્ટર સર્કિટ્સ \\ \hline
\textbf{ABCD} & ટ્રાન્સમિશન પેરામીટર્સ & $V_1 = AV_2 - BI_2$ \newline $I_1 = CV_2 - DI_2$ & કેસ્કેડેડ નેટવર્ક્સ \\ \hline
\end{tabulary}
\end{center}

\begin{mnemonicbox}
\mnemonic{ZANY HAB: Z for high impedance, A for low, hy-brid for transistors, ABCD for Cascades}
\end{mnemonicbox}
\end{solutionbox}

\questionmarks{2(c)}{7}{$\pi$-ટાઈપ નેટવર્કને T-ટાઈપ નેટવર્ક અને T-ટાઈપ નેટવર્કને $\pi$-ટાઈપ નેટવર્કમાં રૂપાંતરિત કરવા માટેના સમીકરણો મેળવો.}

\begin{solutionbox}
\begin{center}
\begin{tabular}{cc}
\begin{circuitikz}[scale=0.8]
    \node at (2,3) {\textbf{T-Network}};
    \draw (0,2) to[R, l=$Z_1$, o-] (2,2) to[R, l=$Z_2$, -o] (4,2);
    \draw (2,2) to[R, l=$Z_3$] (2,0) node[ground]{};
    \node at (0,2) [left] {1};
    \node at (4,2) [right] {2};
\end{circuitikz} &
\begin{circuitikz}[scale=0.8]
    \node at (2,3) {\textbf{$\pi$-Network}};
    \draw (0,2) to[R, l=$Z_{12}$, o-o] (4,2);
    \draw (0,2) to[R, l=$Z_{13}$] (0,0) node[ground]{};
    \draw (4,2) to[R, l=$Z_{23}$] (4,0) node[ground]{};
    \node at (0,2) [left] {1};
    \node at (4,2) [right] {2};
\end{circuitikz}
\end{tabular}
\captionof{figure}{T અને $\pi$ નેટવર્ક્સ}
\end{center}

\begin{center}
\begin{tabulary}{\linewidth}{|L|L|}
\hline
\textbf{રૂપાંતરણ} & \textbf{ફોર્મ્યુલા} \\ \hline
\textbf{$\pi$ થી T} & $Z_1 = \frac{Z_{12}Z_{13}}{Z_{12}+Z_{23}+Z_{13}}$ \newline $Z_2 = \frac{Z_{12}Z_{23}}{Z_{12}+Z_{23}+Z_{13}}$ \newline $Z_3 = \frac{Z_{23}Z_{13}}{Z_{12}+Z_{23}+Z_{13}}$ \\ \hline
\textbf{T થી $\pi$} & $Z_{12} = \frac{Z_1Z_2+Z_2Z_3+Z_3Z_1}{Z_3}$ \newline $Z_{23} = \frac{Z_1Z_2+Z_2Z_3+Z_3Z_1}{Z_1}$ \newline $Z_{13} = \frac{Z_1Z_2+Z_2Z_3+Z_3Z_1}{Z_2}$ \\ \hline
\end{tabulary}
\end{center}

\begin{itemize}
    \item \textbf{ઉપયોગ}: નેટવર્ક સરળીકરણ અને વિશ્લેષણ
    \item \textbf{શરત}: બંને નેટવર્ક્સ ટર્મિનલ્સ પર સમાન હોવા જોઈએ
    \item \textbf{મર્યાદા}: ફક્ત લીનિયર નેટવર્ક્સ માટે લાગુ પડે છે
\end{itemize}

\begin{mnemonicbox}
\mnemonic{TRIP: T and \pi{} networks Relate Impedances through Products and sums}
\end{mnemonicbox}
\end{solutionbox}

\questionmarks{2(a) OR}{3}{KVL ઉદાહરણ સાથે સમજાવો.}

\begin{solutionbox}
\begin{center}
\begin{circuitikz}[auto]
    \draw (0,0) to[V, l=$V_1$] (0,3) to[R, l=$R_1$] (3,3) to[R, l=$R_2$] (3,0) to[R, l=$R_3$] (0,0);
\end{circuitikz}
\captionof{figure}{KVL લૂપ}
\end{center}

\textbf{કિરચોફનો વોલ્ટેજ લૉ (KVL)}: કોઈપણ બંધ લૂપમાં તમામ વોલ્ટેજનો અલજેબ્રાઇક સરવાળો શૂન્ય હોય છે.

\begin{center}
\begin{tabulary}{\linewidth}{|L|L|}
\hline
\textbf{ગાણિતિક સ્વરૂપ} & \textbf{ઉદાહરણ ઉપયોગ} \\ \hline
$\Sigma V = 0$ & લૂપમાં: $V_1 - IR_1 - IR_2 - IR_3 = 0$ \\ \hline
$\Sigma V_{rises} = \Sigma V_{drops}$ & વોલ્ટેજ વધારા = વોલ્ટેજ ઘટાડા \\ \hline
\end{tabulary}
\end{center}

\begin{mnemonicbox}
\mnemonic{ZERO: Zero is Every voltage Round a loop's Output}
\end{mnemonicbox}
\end{solutionbox}

\questionmarks{2(b) OR}{4}{વિવિધ ઈલેક્ટ્રોનિક્સ નેટવર્કનું વર્ગીકરણ કરો અને સમજાવો.}

\begin{solutionbox}
\begin{center}
\begin{tabulary}{\linewidth}{|L|L|L|}
\hline
\textbf{નેટવર્ક પ્રકાર} & \textbf{વર્ણન} & \textbf{ઉદાહરણ} \\ \hline
\textbf{લીનિયર vs નોન-લીનિયર} & સમાનુપાતિકતાના સિદ્ધાંતનું પાલન કરે/ન કરે & રેઝિસ્ટર્સ vs ડાયોડ્સ \\ \hline
\textbf{પેસિવ vs એક્ટિવ} & ઊર્જા પ્રદાન કરતા નથી/કરે છે & RC સર્કિટ vs એમ્પ્લિફાયર \\ \hline
\textbf{બાયલેટરલ vs યુનિલેટરલ} & બંને દિશામાં સમાન/અલગ ગુણધર્મો & રેઝિસ્ટર્સ vs ડાયોડ્સ \\ \hline
\textbf{લમ્પ્ડ vs ડિસ્ટ્રિબ્યુટેડ} & પેરામીટર્સ કેન્દ્રિત/ફેલાયેલા છે & RC સર્કિટ vs ટ્રાન્સમિશન લાઇન \\ \hline
\textbf{ટાઇમ વેરિઅન્ટ vs ઇન્વેરિઅન્ટ} & પેરામીટર્સ સમય સાથે બદલાય/ન બદલાય & ઇલેક્ટ્રોનિક સ્વિચ vs ફિક્સ્ડ રેઝિસ્ટર \\ \hline
\end{tabulary}
\end{center}

\begin{center}
\begin{tikzpicture}[
    level 1/.style={sibling distance=2.5cm},
    level 2/.style={sibling distance=1.5cm},
    edge from parent/.style={draw, -latex},
    every node/.style={gtu block, align=center}
]
    \node {Electronic Networks}
        child {node {Based on Linearity}
            child {node {Linear}}
            child {node {Non-linear}}
        }
        child {node {Based on Energy}
            child {node {Active}}
            child {node {Passive}}
        }
        child {node {Based on Direction}
            child {node {Bilateral}}
            child {node {Unilateral}}
        };
\end{tikzpicture}
\captionof{figure}{ઇલેક્ટ્રોનિક નેટવર્ક્સનું વર્ગીકરણ}
\end{center}

\begin{mnemonicbox}
\mnemonic{PLANT: Proportionality for Linear, Lively for Active, All directions for bilateral, Near for lumped, Time-fixed for invariant}
\end{mnemonicbox}
\end{solutionbox}

\questionmarks{2(c) OR}{7}{T-નેટવર્ક અને $\pi$-નેટવર્ક માટે કૅરૅક્ટરીસટીક્સ ઇમપીડંસનું સમીકરણ મેળવો.}

\begin{solutionbox}
\begin{center}
\begin{tabular}{cc}
\begin{circuitikz}[scale=0.8]
    \draw (0,2) to[R, l=$Z_1/2$, o-] (2,2) to[R, l=$Z_1/2$, -o] (4,2);
    \draw (2,2) to[R, l=$Z_2$] (2,0) node[ground]{};
    \node at (2,-0.5) {T-Network};
\end{circuitikz} &
\begin{circuitikz}[scale=0.8]
    \draw (0,2) to[R, l=$Z_2$, o-o] (4,2);
    \draw (0,2) to[R, l=$2Z_1$] (0,0) node[ground]{};
    \draw (4,2) to[R, l=$2Z_1$] (4,0) node[ground]{};
    \node at (2,-0.5) {$\pi$-Network};
\end{circuitikz}
\end{tabular}
\captionof{figure}{સિમેટ્રિકલ T અને $\pi$ નેટવર્ક્સ}
\end{center}

\begin{center}
\begin{tabulary}{\linewidth}{|L|L|L|}
\hline
\textbf{નેટવર્ક} & \textbf{કૅરૅક્ટરીસટીક્સ ઇમપીડંસ સમીકરણ} & \textbf{મેળવવાના પગલાં} \\ \hline
\textbf{T-નેટવર્ક} & $Z_{0T} = \sqrt{Z_1(Z_1/4 + Z_2)}$ & 1. સિમેટ્રિકલ લોડ $Z_0$ લાગુ કરો \newline 2. ઇનપુટ ઇમ્પીડન્સ શોધો \newline 3. ઇમ્પીડન્સ મેચિંગ માટે, $Z_{in} = Z_0$ \newline 4. $Z_0$ માટે ઉકેલો \\ \hline
\textbf{$\pi$-નેટવર્ક} & $Z_{0\pi} = \sqrt{\frac{Z_1Z_2}{1 + Z_1/4Z_2}}$ & 1. સિમેટ્રિકલ લોડ $Z_0$ લાગુ કરો \newline 2. ઇનપુટ ઇમ્પીડન્સ શોધો \newline 3. ઇમ્પીડન્સ મેચિંગ માટે, $Z_{in} = Z_0$ \newline 4. $Z_0$ માટે ઉકેલો \\ \hline
\end{tabulary}
\end{center}

\begin{itemize}
    \item \textbf{સંબંધ}: $Z_{0T} \times Z_{0\pi} = Z_1 \cdot Z_2$
    \item \textbf{ઉપયોગ}: ઇમ્પીડન્સ મેચિંગ અને ફિલ્ટર્સ
    \item \textbf{મર્યાદા}: ફક્ત સિમેટ્રિકલ નેટવર્ક્સ માટે માન્ય
\end{itemize}

\begin{mnemonicbox}
\mnemonic{TIPSZ: T-networks and \pi{}-networks Impedances are Products and Square roots of Z values}
\end{mnemonicbox}
\end{solutionbox}

\questionmarks{3(a)}{3}{ડ્યુઆલિટીનો સિદ્ધાંત ઉદાહરણ સાથે સમજાવો.}

\begin{solutionbox}
\begin{center}
\begin{tabular}{cc}
\begin{circuitikz}[scale=0.8]
    \draw (0,2) to[V, l=$V_1$] (0,0);
    \draw (0,2) to[R, l=$R_1$] (2,2) to[R, l=$R_2$] (2,0);
    \draw (2,2) to[R, l=$R_3$] (4,2) to[short] (4,0) to[short] (0,0);
    \node at (2,-0.5) {Original Circuit};
\end{circuitikz} &
\begin{circuitikz}[scale=0.8]
    \draw (0,2) to[I, l=$I_1$] (0,0);
    \draw (0,2) -- (1,2) to[R, l=$G_1$] (1,0) -- (0,0);
    \draw (1,2) -- (2,2) to[R, l=$G_2$] (2,0) -- (1,0);
    \draw (2,2) -- (3,2) to[R, l=$G_3$] (3,0) -- (2,0);
    \node at (1.5,-0.5) {Dual Circuit};
\end{circuitikz}
\end{tabular}
\captionof{figure}{ડ્યુઆલિટીનો સિદ્ધાંત}
\end{center}

\textbf{ડ્યુઆલિટીનો સિદ્ધાંત}: દરેક ઇલેક્ટ્રિકલ નેટવર્ક માટે, એક ડ્યુઅલ નેટવર્ક અસ્તિત્વમાં છે જ્યાં:

\begin{center}
\begin{tabulary}{\linewidth}{|L|L|L|}
\hline
\textbf{ઓરિજિનલ} & \textbf{ડ્યુઅલ} & \textbf{ઉદાહરણ} \\ \hline
વોલ્ટેજ (V) & કરંટ (I) & 10V સોર્સ $\to$ 10A સોર્સ \\ \hline
કરંટ (I) & વોલ્ટેજ (V) & 5A $\to$ 5V \\ \hline
રેઝિસ્ટન્સ (R) & કન્ડક્ટન્સ (G) & 100$\Omega$ $\to$ 100S \\ \hline
સિરીઝ જોડાણ & પેરેલલ જોડાણ & સિરીઝ રેઝિસ્ટર્સ $\to$ પેરેલલ કન્ડક્ટર્સ \\ \hline
KVL & KCL & $\Sigma V = 0 \to \Sigma I = 0$ \\ \hline
\end{tabulary}
\end{center}

\begin{mnemonicbox}
\mnemonic{VIGOR: Voltage to current, Impedance to admittance, Graph remains, Open to closed, Resistors to conductors}
\end{mnemonicbox}
\end{solutionbox}

\questionmarks{3(b)}{4}{થેવેનીન થીયરમનો ઉપયોગ કરીને સર્કિટમાં લોડ કરંટની ગણતરી કરવા માટેના પગલાં સમજાવો.}

\begin{solutionbox}
\begin{center}
\begin{circuitikz}[auto]
    % Block diagram style
    \node[gtu block] (A) {Original Circuit};
    \node[gtu block, right=of A] (B) {Remove Load};
    \node[gtu block, right=of B] (E) {Thevenin Equivalent};
    \node[gtu block, right=of E] (F) {Reconnect Load};
    
    \draw[gtu arrow] (A) -- (B);
    \draw[gtu arrow] (B) -- node[above, font=\tiny] {Find $V_{th}, R_{th}$} (E);
    \draw[gtu arrow] (E) -- (F);
\end{circuitikz}
\captionof{figure}{થેવેનીન થીયરમની પ્રક્રિયા}
\end{center}

\begin{center}
\begin{tabulary}{\linewidth}{|C|L|}
\hline
\textbf{પગલું} & \textbf{વર્ણન} \\ \hline
1 & સર્કિટમાંથી લોડ રેઝિસ્ટર દૂર કરો \\ \hline
2 & લોડ ટર્મિનલ્સ પર ઓપન-સર્કિટ વોલ્ટેજ ($V_{th}$) શોધો \\ \hline
3 & સર્કિટમાં પાછા જોઈને થેવેનીન રેઝિસ્ટન્સ ($R_{th}$) શોધો \\ \hline
4 & થેવેનીન ઇક્વિવેલન્ટ સર્કિટ દોરો ($V_{th}$ સાથે $R_{th}$ સિરીઝમાં) \\ \hline
5 & લોડ રેઝિસ્ટર ($R_L$) ને થેવેનીન સર્કિટ સાથે ફરીથી જોડો \\ \hline
6 & લોડ કરંટ ગણો: $I_L = V_{th}/(R_{th}+R_L)$ \\ \hline
\end{tabulary}
\end{center}

\begin{mnemonicbox}
\mnemonic{REVOLT: Remove load, Evaluate Voc, Obtain Rth, Look at Thevenin circuit, Use I = V/R formula}
\end{mnemonicbox}
\end{solutionbox}

\questionmarks{3(c)}{7}{સુપરપોઝિશન થીયરમનો ઉપયોગ કરીને લોડ રેઝિસ્ટરમાંથી વહેતો કરંટ શોધો.}

\begin{solutionbox}
\begin{center}
\begin{circuitikz}[scale=1]
    \draw (0,0) to[V, l=12V, invert] (0,3) to[R, l=4$\Omega$] (3,3) node[label=above:A]{} to[R, l=10$\Omega$] (6,3) node[label=above:B]{};
    \draw (6,3) to[I, l=12A, invert] (6,0) -- (0,0);
    \draw (3,3) to[R, l=6$\Omega$, i=$I_L$] (3,0);
    \node at (3,-0.5) {આપેલ સર્કિટ};
\end{circuitikz}
\captionof{figure}{સુપરપોઝિશન પ્રશ્ન}
\end{center}

\textbf{સ્ટેપ-બાય-સ્ટેપ ઉકેલ:}

\begin{center}
\begin{tabulary}{\linewidth}{|C|L|L|}
\hline
\textbf{પગલું} & \textbf{વર્ણન} & \textbf{ગણતરી} \\ \hline
1 & માત્ર 12V સોર્સ ધ્યાનમાં લો (12A ઓપન કરો) & $I_1 = 12/(4+6+10) = 0.6A$ (10 ઓહ્મ સિરીઝમાં છે) \\ \hline
2 & માત્ર 12A સોર્સ ધ્યાનમાં લો (12V શોર્ટ કરો) & $I_2 = -12 \times 4 / (4+10+6) = -2.4A$ (કરંટ ડિવાઈડર) \\ \hline
3 & સુપરપોઝિશન લાગુ કરો & $I_L = I_1 + I_2 = 0.6 + (-2.4) = -1.8A$ \\ \hline
\end{tabulary}
\end{center}

\textbf{જવાબ}: $I_L = -1.8A$ (6$\Omega$ લોડ રેઝિસ્ટરમાંથી ઉપરની તરફ વહેતો કરંટ)

\begin{mnemonicbox}
\mnemonic{SONAR: Sources Only one at a time, Neutralize others, Add Results}
\end{mnemonicbox}
\end{solutionbox}

\questionmarks{3(a) OR}{3}{મેક્સિમમ પાવર ટ્રાન્સફર થીયરમનું વિધાન લખો. AC અને DC નેટવર્ક માટે મેક્સિમમ પાવર ટ્રાન્સફર માટેની શરતો શું છે?}

\begin{solutionbox}
\textbf{મેક્સિમમ પાવર ટ્રાન્સફર થીયરમ}: જ્યારે લોડ ઇમ્પિડન્સ એ સોર્સના ઇન્ટરનલ ઇમ્પિડન્સના કોમ્પ્લેક્સ કોન્જુકેટ બરાબર હોય ત્યારે સોર્સમાંથી લોડમાં મહત્તમ પાવર ટ્રાન્સફર થાય છે.

\begin{center}
\begin{tabulary}{\linewidth}{|L|L|}
\hline
\textbf{નેટવર્ક પ્રકાર} & \textbf{મેક્સિમમ પાવર ટ્રાન્સફર માટેની શરત} \\ \hline
\textbf{DC નેટવર્ક્સ} & $R_L = R_{th}$ (લોડ રેઝિસ્ટન્સ થેવેનીન રેઝિસ્ટન્સ બરાબર હોય) \\ \hline
\textbf{AC નેટવર્ક્સ} & $Z_L = Z_{th}^*$ (લોડ ઇમ્પિડન્સ થેવેનીન ઇમ્પિડન્સના કોમ્પ્લેક્સ કોન્જુકેટ બરાબર હોય) \\ \hline
\end{tabulary}
\end{center}

\begin{center}
\begin{tabular}{cc}
\begin{circuitikz}[scale=0.8]
    \draw (0,2) to[R, l=$R_{th}$] (2,2) to[R, l=$R_L$] (2,0) -- (0,0) to[V, l=$V_{th}$] (0,2);
    \node at (1,-0.5) {DC Network};
\end{circuitikz} &
\begin{circuitikz}[scale=0.8]
    \draw (0,2) to[R, l=$R_{th}$] (1.5,2) to[L, l=$X_{th}$] (3,2);
    \draw (3,2) to[L, l=$X_L$] (3,1) to[R, l=$R_L$] (3,0) -- (0,0) to[V, l=$V_{th}$] (0,2);
    \node at (1.5,-0.5) {AC Network};
\end{circuitikz}
\end{tabular}
\captionof{figure}{મેક્સિમમ પાવર ટ્રાન્સફર સર્કિટ્સ}
\end{center}

\begin{mnemonicbox}
\mnemonic{MATCH: Maximum power At Terminals when Conjugate impedances are Honored}
\end{mnemonicbox}
\end{solutionbox}

\questionmarks{3(b) OR}{4}{નોર્ટન થીયરમનો ઉપયોગ કરીને સર્કિટમાં લોડ કરંટ ગણવા માટેના પગલાં સમજાવો.}

\begin{solutionbox}
\begin{center}
\begin{circuitikz}[auto]
    \node[gtu block] (A) {Original Circuit};
    \node[gtu block, right=of A] (B) {Short Terminals};
    \node[gtu block, right=of B] (E) {Norton Equivalent};
    \node[gtu block, right=of E] (F) {Reconnect Load};
    
    \draw[gtu arrow] (A) -- (B);
    \draw[gtu arrow] (B) -- node[above, font=\tiny] {Find $I_{sc}, R_N$} (E);
    \draw[gtu arrow] (E) -- (F);
\end{circuitikz}
\captionof{figure}{નોર્ટન થીયરમ પગલાં}
\end{center}

\begin{center}
\begin{tabulary}{\linewidth}{|C|L|}
\hline
\textbf{પગલું} & \textbf{વર્ણન} \\ \hline
1 & સર્કિટમાંથી લોડ રેઝિસ્ટર દૂર કરો \\ \hline
2 & લોડ ટર્મિનલ્સ પર શોર્ટ-સર્કિટ કરંટ ($I_{sc}$ અથવા $I_N$) શોધો \\ \hline
3 & સર્કિટમાં પાછા જોઈને નોર્ટન રેઝિસ્ટન્સ ($R_N$) શોધો \\ \hline
4 & નોર્ટન ઇક્વિવેલન્ટ સર્કિટ દોરો ($I_N$ સાથે $R_N$ પેરેલલ) \\ \hline
5 & લોડ રેઝિસ્ટર ($R_L$) ને નોર્ટન સર્કિટ સાથે જોડો \\ \hline
6 & લોડ કરંટ ગણો: $I_L = I_N \times R_N/(R_N+R_L)$ \\ \hline
\end{tabulary}
\end{center}

\begin{mnemonicbox}
\mnemonic{SENIOR: Short terminals, Evaluate Isc, Notice Rn value, Implement Norton circuit, Obtain result}
\end{mnemonicbox}
\end{solutionbox}

\questionmarks{3(c) OR}{7}{રેસીપ્રોસીટી થીયરમ આપેલ નેટવર્ક માટે કેવી રીતે લાગુ પડે છે તે દર્શાવો.}

\begin{solutionbox}
\begin{center}
\begin{circuitikz}[scale=0.9]
    \draw (0,0) to[V, l=10V] (0,2) to[R, l=2$\Omega$] (3,2) to[R, l=2$\Omega$] (6,2);
    \draw (3,2) to[R, l=4$\Omega$] (3,0);
    \draw (6,2) to[short] (6,0) -- (0,0);
    
    % Nodes
    \node at (3,2) [above] {Node A};
    \node at (6,2) [right] {Output};
\end{circuitikz}
\captionof{figure}{રેસીપ્રોસીટી થીયરમ ઉદાહરણ}
\end{center}

\textbf{કોષ્ટક: રેસીપ્રોસીટી થીયરમ લાગુ કરવું:}

\begin{center}
\begin{tabulary}{\linewidth}{|C|L|L|L|}
\hline
\textbf{પગલું} & \textbf{સર્કિટ 1} & \textbf{સર્કિટ 2} & \textbf{ચકાસણી} \\ \hline
1 & ડાબી બાજુ 10V સોર્સ, જમણી બાજુ $I_1$ શોધો & જમણી બાજુ 10V સોર્સ, ડાબી બાજુ $I_2$ શોધો & $I_1 = I_2$ રેસીપ્રોસીટી સાબિત કરે છે \\ \hline
2 & KVL વડે મેશ ઇક્વેશન્સ બનાવો & સોર્સ બદલીને ઇક્વેશન્સ બનાવો & બંને સિસ્ટમ ઉકેલો \\ \hline
3 & $I_1 = 0.625A$ & $I_2 = 0.625A$ & $I_1 = I_2 = 0.625A$ \checkmark \\ \hline
\end{tabulary}
\end{center}

\textbf{સિદ્ધાંત}: પેસિવ, બાયલેટરલ નેટવર્કમાં, જો બ્રાન્ચ 1 માં વોલ્ટેજ સોર્સ E બ્રાન્ચ 2 માં કરંટ I ઉત્પન્ન કરે, તો તે જ વોલ્ટેજ સોર્સ E ને બ્રાન્ચ 2 માં મૂકવાથી બ્રાન્ચ 1 માં તેટલો જ કરંટ I ઉત્પન્ન થશે.

\begin{mnemonicbox}
\mnemonic{RESPECT: Rewire sources, Exchange positions, See if currents Preserve Equality when Circuit Transformed}
\end{mnemonicbox}
\end{solutionbox}


% Question 4
\questionmarks{4(a)}{3}{કપલ્ડ સર્કિટ સમજાવો.}

\begin{solutionbox}
\begin{center}
\begin{circuitikz}[scale=1]
    \draw (0,0) to[V, l=$V_1$] (0,2) to[L, l=$L_1$, name=L1] (0,0);
    \draw (2,0) to[R, l=$R_L$] (2,2) to[L, l=$L_2$, name=L2] (2,0);
    % Transformer core/coupling
    \draw ($(L1.north west)!0! (L1.north east)$) ++(0.7,0) -- ++(0,-1.5) node[midway, right] {$M$};
    \draw ($(L1.north west)!0! (L1.north east)$) ++(0.8,0) -- ++(0,-1.5);
\end{circuitikz}
\captionof{figure}{કપલ્ડ સર્કિટ}
\end{center}

\textbf{કપલ્ડ સર્કિટ}: એક સર્કિટ જ્યાં મ્યુચ્યુઅલ ઇન્ડક્ટન્સ દ્વારા ઇન્ડક્ટરો વચ્ચે ઊર્જા ટ્રાન્સફર થાય છે.

\begin{center}
\begin{tabulary}{\linewidth}{|L|L|}
\hline
\textbf{પેરામીટર} & \textbf{વર્ણન} \\ \hline
\textbf{મ્યુચ્યુઅલ ઇન્ડક્ટન્સ (M)} & કોઇલ વચ્ચેના મેગ્નેટિક કપલિંગનું માપ \\ \hline
\textbf{કોફિશિયન્ટ ઓફ કપલિંગ (k)} & $k = M/\sqrt{L_1L_2}$, 0 થી 1 ની વચ્ચે હોય છે \\ \hline
\textbf{ઉપયોગો} & ટ્રાન્સફોર્મર, ફિલ્ટર્સ, ટ્યુન્ડ સર્કિટ \\ \hline
\end{tabulary}
\end{center}

\begin{mnemonicbox}
\mnemonic{MICE: Mutual Inductance Creates Energy transfer}
\end{mnemonicbox}
\end{solutionbox}

\questionmarks{4(b)}{4}{કપલ્ડ સર્કિટ માટે કોફિશિયન્ટ ઓફ કપલિંગનું સમીકરણ તારવો.}

\begin{solutionbox}
\begin{center}
\begin{tikzpicture}[node distance=2cm, auto, >=latex]
    \node[gtu block] (A) {Magnetic Flux Linkage};
    \node[gtu block, right=of A] (B) {Mutual Inductance};
    \node[gtu block, right=of B] (C) {Coupling Coefficient};
    
    \draw[gtu arrow] (A) -- (B);
    \draw[gtu arrow] (B) -- (C);
\end{tikzpicture}
\captionof{figure}{તારવણી લોજિક}
\end{center}

\begin{center}
\begin{tabulary}{\linewidth}{|C|L|L|}
\hline
\textbf{પગલું} & \textbf{વર્ણન} & \textbf{સમીકરણ} \\ \hline
1 & મ્યુચ્યુઅલ ઇન્ડક્ટન્સ વ્યાખ્યાયિત કરો & $M = N_2 \cdot \phi_{12}/I_1$ \\ \hline
2 & સેલ્ફ ઇન્ડક્ટન્સ વ્યાખ્યાયિત કરો & $L_1 = N_1 \cdot \phi_{11}/I_1$, $L_2 = N_2 \cdot \phi_{22}/I_2$ \\ \hline
3 & મહત્તમ શક્ય M & $M_{max} = \sqrt{L_1 L_2}$ \\ \hline
4 & કપલિંગ કોફિશિયન્ટ વ્યાખ્યાયિત કરો & $k = M/\sqrt{L_1 L_2}$ \\ \hline
\end{tabulary}
\end{center}

\begin{itemize}
    \item \textbf{રેંજ}: $0 \leq k \leq 1$
    \item \textbf{ભૌતિક અર્થ}: એક કોઇલમાંથી ફ્લક્સનો અંશ જે અન્ય કોઇલ સાથે લિંક થાય છે
\end{itemize}

\begin{mnemonicbox}
\mnemonic{MASK: Mutual inductance And Self inductances create K}
\end{mnemonicbox}
\end{solutionbox}

\questionmarks{4(c)}{7}{સિરીઝ રેઝોનન્સ માટે રેઝોનન્સ ફ્રિકવન્સીનું સમીકરણ તારવો. R=20$\Omega$, L=1H, C=1$\mu$F સાથે સિરીઝ RLC સર્કિટની રેઝોનન્ટ ફ્રિકવન્સી, Q ફેક્ટર અને બેન્ડવિડ્થની ગણતરી કરો.}

\begin{solutionbox}
\begin{center}
\begin{circuitikz}
    \draw (0,0) to[V, l=$V$] (0,2) to[R, l=$R$] (2,2) to[L, l=$L$] (4,2) to[C, l=$C$] (4,0) -- (0,0);
\end{circuitikz}
\captionof{figure}{સિરીઝ RLC સર્કિટ}
\end{center}

\textbf{તારવણી:}
\begin{enumerate}
    \item સિરીઝ RLC નો ઇમ્પિડન્સ: $Z = R + j(\omega L - 1/\omega C)$
    \item રેઝોનન્સ પર, ઈમેજનરી ભાગ શૂન્ય હોય છે: $\omega L - 1/\omega C = 0$
    \item રેઝોનન્ટ ફ્રિકવન્સી માટે ઉકેલો: $\omega_0 = 1/\sqrt{LC}$ અથવા $f_0 = 1/(2\pi\sqrt{LC})$
\end{enumerate}

\textbf{ગણતરીઓ:}

\begin{center}
\begin{tabulary}{\linewidth}{|L|L|L|L|}
\hline
\textbf{પેરામીટર} & \textbf{ફોર્મ્યુલા} & \textbf{ગણતરી} & \textbf{પરિણામ} \\ \hline
રેઝોનન્ટ ફ્રિકવન્સી & $f_0 = 1/(2\pi\sqrt{LC})$ & $f_0 = 1/(2\pi\sqrt{1\times10^{-6}})$ & 159.15 Hz \\ \hline
Q ફેક્ટર & $Q = \omega_0 L/R$ & $Q = 2\pi\times159.15\times1/20$ & 50 \\ \hline
બેન્ડવિડ્થ & $BW = f_0/Q$ & $BW = 159.15/50$ & 3.18 Hz \\ \hline
\end{tabulary}
\end{center}

\begin{mnemonicbox}
\mnemonic{FQBR: Frequency from reactances, Q from resistance ratio, Bandwidth from Resonance divided by Q}
\end{mnemonicbox}
\end{solutionbox}

\questionmarks{4(a) OR}{3}{ક્વોલિટી ફેક્ટર સમજાવો.}

\begin{solutionbox}
\textbf{ક્વોલિટી ફેક્ટર (Q)}: એક ડાયમેન્શનલેસ પેરામીટર જે દર્શાવે છે કે રેઝોનેટર કેટલું અંડર-ડેમ્પ્ડ છે, અથવા રેઝોનેટરની બેન્ડવિડ્થ તેની સેન્ટર ફ્રિકવન્સીની સાપેક્ષે કેટલી છે.

\begin{center}
\begin{tabulary}{\linewidth}{|L|L|}
\hline
\textbf{વ્યાખ્યા} & \textbf{ગાણિતિક અભિવ્યક્તિ} \\ \hline
ઊર્જા પરિપ્રેક્ષ્ય & $Q = 2\pi \times \frac{\text{Energy stored}}{\text{Energy dissipated per cycle}}$ \\ \hline
સર્કિટ પરિપ્રેક્ષ્ય & $Q = X/R$ (જ્યાં X રિએક્ટન્સ, R રેઝિસ્ટન્સ) \\ \hline
ફ્રિકવન્સી પરિપ્રેક્ષ્ય & $Q = f_0/BW$ (જ્યાં $f_0$ રેઝોનન્ટ ફ્રિકવન્સી, BW બેન્ડવિડ્થ) \\ \hline
\end{tabulary}
\end{center}

\begin{mnemonicbox}
\mnemonic{QSEL: Quality shows Energy vs. Loss and Selectivity}
\end{mnemonicbox}
\end{solutionbox}

\questionmarks{4(b) OR}{4}{કેપેસીટર માટે ક્વોલિટી ફેક્ટરનું સમીકરણ તારવો.}

\begin{solutionbox}
\begin{center}
\begin{circuitikz}
    \draw (0,0) to[C, l=$C$] (0,2);
    \draw (2,0) to[R, l=$ESR$] (2,2);
    \draw (0,2) -- (2,2);
    \draw (0,0) -- (2,0);
\end{circuitikz}
\captionof{figure}{રિયલ કેપેસીટર મોડેલ}
\end{center}

\textbf{તારવણી:}

\begin{center}
\begin{tabulary}{\linewidth}{|C|L|L|}
\hline
\textbf{પગલું} & \textbf{વર્ણન} & \textbf{સમીકરણ} \\ \hline
1 & સંગ્રહિત ઊર્જા & $E_{stored} = CV^2/2$ \\ \hline
2 & એનર્જી લોસ & $E_{loss} = \pi CV^2/\omega CR = \pi V^2/\omega R$ \\ \hline
3 & Q ફેક્ટર & $Q = 2\pi \times E_{stored} / E_{loss}$ \\ \hline
4 & સાદું રૂપ આપો & $Q = \omega CR$ \\ \hline
\end{tabulary}
\end{center}

\textbf{અંતિમ સમીકરણ:} $Q = \omega CR = 1/(\omega RC) = 1/\tan\delta$ \newline
(નોંધ: MDX માં આપેલ ફોર્મ્યુલા મુજબ)

\begin{mnemonicbox}
\mnemonic{CORE: Capacitors' Quality equals One over Resistance times Capacitance}
\end{mnemonicbox}
\end{solutionbox}

\questionmarks{4(c) OR}{7}{પેરેલલ રેઝોનન્સ માટે રેઝોનન્સ ફ્રિકવન્સીનું સમીકરણ તારવો. R=30$\Omega$, L=1H, C=1$\mu$F સાથે પેરેલલ RLC સર્કિટની રેઝોનન્ટ ફ્રિકવન્સી, Q ફેક્ટર અને બેન્ડવિડ્થની ગણતરી કરો.}

\begin{solutionbox}
\begin{center}
\begin{circuitikz}
    \draw (0,0) to[I, l=$I$] (0,2) -- (6,2);
    \draw (2,2) to[R, l=$R$] (2,0);
    \draw (4,2) to[L, l=$L$] (4,0);
    \draw (6,2) to[C, l=$C$] (6,0);
    \draw (0,0) -- (6,0);
\end{circuitikz}
\captionof{figure}{પેરેલલ RLC સર્કિટ}
\end{center}

\textbf{તારવણી:}
\begin{enumerate}
    \item પેરેલલ RLC નો એડમિટન્સ: $Y = 1/R + 1/j\omega L + j\omega C$
    \item રેઝોનન્સ પર, ઈમેજનરી ભાગ શૂન્ય: $j(\omega C - 1/\omega L) = 0$
    \item ફ્રિકવન્સી: $\omega_0 = 1/\sqrt{LC}$ અથવા $f_0 = 1/(2\pi\sqrt{LC})$
\end{enumerate}

\textbf{ગણતરીઓ:}

\begin{center}
\begin{tabulary}{\linewidth}{|L|L|L|L|}
\hline
\textbf{પેરામીટર} & \textbf{ફોર્મ્યુલા} & \textbf{ગણતરી} & \textbf{પરિણામ} \\ \hline
રેઝોનન્ટ ફ્રિકવન્સી & $f_0 = 1/(2\pi\sqrt{LC})$ & $f_0 = 1/(2\pi\sqrt{1\times10^{-6}})$ & 159.15 Hz \\ \hline
Q ફેક્ટર & $Q = R/\omega_0 L$ & $Q = 30/(2\pi\times159.15\times1)$ & 0.03 \\ \hline
બેન્ડવિડ્થ & $BW = f_0/Q$ & $BW = 159.15/0.03$ & 5305 Hz \\ \hline
\end{tabulary}
\end{center}

\begin{mnemonicbox}
\mnemonic{FPQB: Frequency from Parallel elements, Q from Resistance divided by reactance, Bandwidth from division}
\end{mnemonicbox}
\end{solutionbox}

% Question 5
\questionmarks{5(a)}{3}{T ટાઈપ એટેન્યુએટર સમજાવો.}

\begin{solutionbox}
\begin{center}
\begin{circuitikz}[scale=1]
    \draw (0,2) to[R, l=$Z_1$, o-] (2,2) to[R, l=$Z_2$, -o] (4,2);
    \draw (2,2) to[R, l=$Z_3$] (2,0);
    \draw (0,0) to[short, o-o] (4,0);
    \node at (0,2) [left] {In};
    \node at (4,2) [right] {Out};
\end{circuitikz}
\captionof{figure}{T-ટાઈપ એટેન્યુએટર}
\end{center}

\textbf{T-ટાઈપ એટેન્યુએટર}: સિગ્નલ એમ્પ્લીટ્યુડ ઘટાડવા માટે વપરાતું T કોન્ફિગરેશનમાં પેસિવ નેટવર્ક.

\begin{center}
\begin{tabulary}{\linewidth}{|L|L|L|}
\hline
\textbf{કમ્પોનન્ટ} & \textbf{વર્ણન} & \textbf{ફોર્મ્યુલા} \\ \hline
\textbf{$Z_1, Z_2$} & સિરીઝ આર્મ્સ & $Z_1 = Z_2 = Z_0(N-1)/(N+1)$ \\ \hline
\textbf{$Z_3$} & શંટ આર્મ & $Z_3 = 2Z_0/(N^2-1)$ \\ \hline
\textbf{N} & એટેન્યુએશન રેશિયો & $N = 10^{(dB/20)}$ \\ \hline
\end{tabulary}
\end{center}

\begin{itemize}
    \item \textbf{લાક્ષણિકતા}: મેચ સોર્સ અને લોડ માટે સિમેટ્રિકલ
    \item \textbf{ઉપયોગો}: સિગ્નલ લેવલ કંટ્રોલ, ઇમ્પીડન્સ મેચિંગ
    \item \textbf{ફાયદો}: યોગ્ય ડિઝાઈન સાથે ઇમ્પીડન્સ મેચિંગ જાળવી રાખે છે
\end{itemize}

\begin{mnemonicbox}
\mnemonic{TSAR: T-shape with Series Arms and Resistance in middle}
\end{mnemonicbox}
\end{solutionbox}

\questionmarks{5(b)}{4}{વિવિધ પેસિવ ફિલ્ટર સર્કિટ્સનું વર્ગીકરણ કરો.}

\begin{solutionbox}
\begin{center}
\begin{tikzpicture}[
    level 1/.style={sibling distance=4cm},
    level 2/.style={sibling distance=1.5cm},
    edge from parent/.style={draw, -latex},
    every node/.style={gtu block, align=center, font=\footnotesize}
]
    \node {Passive Filters}
        child {node {Based on Frequency Response}
            child {node {Low Pass}}
            child {node {High Pass}}
            child {node {Band Pass}}
            child {node {Band Stop}}
        }
        child {node {Based on Configuration}
            child {node {T-section}}
            child {node {$\pi$-section}}
            child {node {L-section}}
            child {node {Lattice}}
        };
\end{tikzpicture}
\captionof{figure}{પેસિવ ફિલ્ટરનું વર્ગીકરણ}
\end{center}

\begin{center}
\begin{tabulary}{\linewidth}{|L|L|L|L|}
\hline
\textbf{ફિલ્ટર પ્રકાર} & \textbf{કાર્ય} & \textbf{લાક્ષણિક સર્કિટ} & \textbf{ઉપયોગ} \\ \hline
\textbf{Low Pass} & ઓછી ફ્રિકવન્સી પસાર કરે છે & RC, RL સર્કિટ & ઓડિયો ફિલ્ટર્સ, પાવર સપ્લાય \\ \hline
\textbf{High Pass} & ઊંચી ફ્રિકવન્સી પસાર કરે છે & CR, LR સર્કિટ & નોઈસ ફિલ્ટરિંગ \\ \hline
\textbf{Band Pass} & ફ્રિકવન્સી બેન્ડ પસાર કરે છે & RLC સર્કિટ & રેડિયો ટ્યુનિંગ \\ \hline
\textbf{Band Stop} & ફ્રિકવન્સી બેન્ડ બ્લોક કરે છે & પેરેલલ RLC & ઇન્ટરફિયરન્સ રિજેક્શન \\ \hline
\end{tabulary}
\end{center}

\begin{mnemonicbox}
\mnemonic{LHBB: Low High Band Band filters for Pass and Block}
\end{mnemonicbox}
\end{solutionbox}

\questionmarks{5(c)}{7}{કટઓફ ફ્રિકવન્સી 1000Hz અને 500$\Omega$ ના લોડ સાથે T-સેક્શન ધરાવતા કોન્સ્ટન્ટ-k ટાઈપ લો પાસ અને હાઈ પાસ ફિલ્ટર ડિઝાઈન કરો.}

\begin{solutionbox}
\begin{center}
\begin{tabular}{cc}
\begin{circuitikz}[scale=0.8]
    \draw (0,2) to[L, l=$L/2$] (2,2) to[L, l=$L/2$] (4,2);
    \draw (2,2) to[C, l=$C$] (2,0) node[ground]{};
    \node at (2,-0.5) {Low Pass T-Filter};
\end{circuitikz} &
\begin{circuitikz}[scale=0.8]
    \draw (0,2) to[C, l=$C/2$] (2,2) to[C, l=$C/2$] (4,2);
    \draw (2,2) to[L, l=$L$] (2,0) node[ground]{};
    \node at (2,-0.5) {High Pass T-Filter};
\end{circuitikz}
\end{tabular}
\captionof{figure}{કોન્સ્ટન્ટ-k ટાઈપ T-ફિલ્ટર્સ}
\end{center}

\textbf{ડિઝાઈન ગણતરીઓ:}

કોન્સ્ટન્ટ-k T-ટાઈપ લો પાસ ફિલ્ટર માટે:
\begin{center}
\begin{tabulary}{\linewidth}{|L|L|L|L|}
\hline
\textbf{પેરામીટર} & \textbf{ફોર્મ્યુલા} & \textbf{ગણતરી} & \textbf{કિંમત} \\ \hline
કટ-ઓફ ફ્રિકવન્સી & $f_c = 1000$ Hz & આપેલ & 1000 Hz \\ \hline
લોડ ઇમ્પિડન્સ & $R_0 = 500 \Omega$ & આપેલ & 500 $\Omega$ \\ \hline
સિરીઝ ઇન્ડક્ટર & $L = R_0/\pi f_c$ & $L = 500/(\pi \times 1000)$ & 159.15 mH \\ \hline
હાફ સેક્શન્સ & $L/2$ & $159.15/2$ & 79.58 mH \\ \hline
શંટ કેપેસીટર & $C = 1/(\pi f_c R_0)$ & $C = 1/(\pi \times 1000 \times 500)$ & 0.636 $\mu$F \\ \hline
\end{tabulary}
\end{center}

કોન્સ્ટન્ટ-k T-ટાઈપ હાઈ પાસ ફિલ્ટર માટે:
\begin{center}
\begin{tabulary}{\linewidth}{|L|L|L|L|}
\hline
\textbf{પેરામીટર} & \textbf{ફોર્મ્યુલા} & \textbf{ગણતરી} & \textbf{કિંમત} \\ \hline
સિરીઝ કેપેસીટર & $C = 1/(4\pi f_c R_0)$ & $C = 1/(4\pi \times 1000 \times 500)$ & 0.159 $\mu$F \\ \hline
હાફ સેક્શન્સ & $C/2$ & $0.159/2$ & 0.0795 $\mu$F \\ \hline
શંટ ઇન્ડક્ટર & $L = R_0/(4\pi f_c)$ & $L = 500/(4\pi \times 1000)$ & 39.79 mH \\ \hline
\end{tabulary}
\end{center}

\begin{mnemonicbox}
\mnemonic{FRED: Frequency Ratio determines Element Dimensions}
\end{mnemonicbox}
\end{solutionbox}

\questionmarks{5(a) OR}{3}{$\pi$ ટાઈપ એટેન્યુએટર સમજાવો.}

\begin{solutionbox}
\begin{center}
\begin{circuitikz}[scale=1]
    \draw (0,2) to[short, o-] (1,2) to[R, l=$Z_2$] (3,2) to[short, -o] (4,2);
    \draw (1,2) to[R, l=$Z_1$] (1,0);
    \draw (3,2) to[R, l=$Z_3$] (3,0);
    \draw (0,0) to[short, o-o] (4,0);
    
    \node at (0,2) [left] {In};
    \node at (4,2) [right] {Out};
\end{circuitikz}
\captionof{figure}{$\pi$-ટાઈપ એટેન્યુએટર}
\end{center}

\textbf{$\pi$-ટાઈપ એટેન્યુએટર}: સિગ્નલ એમ્પ્લીટ્યુડ ઘટાડવા માટે વપરાતું $\pi$ કોન્ફિગરેશનમાં પેસિવ નેટવર્ક.

\begin{center}
\begin{tabulary}{\linewidth}{|L|L|L|}
\hline
\textbf{કમ્પોનન્ટ} & \textbf{વર્ણન} & \textbf{ફોર્મ્યુલા} \\ \hline
\textbf{$Z_2$} & સિરીઝ આર્મ & $Z_2 = 2Z_0/(N^2-1)$ \\ \hline
\textbf{$Z_1, Z_3$} & શંટ આર્મ્સ & $Z_1 = Z_3 = Z_0(N+1)/(N-1)$ \\ \hline
\textbf{N} & એટેન્યુએશન રેશિયો & $N = 10^{(dB/20)}$ \\ \hline
\end{tabulary}
\end{center}

\begin{itemize}
    \item \textbf{લાક્ષણિકતા}: મેચ સોર્સ અને લોડ માટે સિમેટ્રિકલ
    \item \textbf{ઉપયોગો}: સિગ્નલ લેવલ કંટ્રોલ, ઇમ્પીડન્સ મેચિંગ
    \item \textbf{ફાયદો}: ઇનપુટ અને આઉટપુટ વચ્ચે સારું આઇસોલેશન
\end{itemize}

\begin{mnemonicbox}
\mnemonic{PASS: Pi-Attenuator has Series in middle and Shunt arms outside}
\end{mnemonicbox}
\end{solutionbox}

\questionmarks{5(b) OR}{4}{વિવિધ પ્રકારના એટેન્યુએટર્સનું વર્ગીકરણ કરો.}

\begin{solutionbox}
\begin{center}
\begin{tikzpicture}[
    level 1/.style={sibling distance=3cm},
    level 2/.style={sibling distance=1.5cm},
    edge from parent/.style={draw, -latex},
    every node/.style={gtu block, align=center, font=\footnotesize}
]
    \node {Attenuators}
        child {node {Based on Structure}
            child {node {T-type}}
            child {node {$\pi$-type}}
            child {node {L-type}}
            child {node {Bridged-T}}
            child {node {Lattice}}
        }
        child {node {Based on Function}
            child {node {Fixed}}
            child {node {Variable}}
            child {node {Stepped}}
            child {node {Programmable}}
        };
\end{tikzpicture}
\captionof{figure}{એટેન્યુએટર્સનું વર્ગીકરણ}
\end{center}


\captionof{table}{એટેન્યુએટર્સનું વર્ગીકરણ}
\begin{center}
\begin{tabulary}{\linewidth}{|L|L|L|L|}
\hline
\textbf{એટેન્યુએટર પ્રકાર} & \textbf{લાક્ષણિકતાઓ} & \textbf{ઉપયોગો} & \textbf{ફાયદા} \\ \hline
\textbf{T-type} & સિરીઝ-શંટ-સિરીઝ & ઓડિયો સિસ્ટમ્સ & સરળ ડિઝાઈન \\ \hline
\textbf{$\pi$-type} & શંટ-સિરીઝ-શંટ & RF સર્કિટ & સારું આઇસોલેશન \\ \hline
\textbf{L-type} & સિરીઝ-શંટ & સરળ મેચિંગ & ઇમ્પીડન્સ ટ્રાન્સફોર્મેશન \\ \hline
\textbf{Bridged-T} & બેલેન્સ્ડ સ્ટ્રક્ચર & ટેસ્ટ સાધનો & ન્યૂનતમ વિકૃતિ \\ \hline
\textbf{Balanced} & સિમેટ્રિકલ ડ્યુઅલ પાથ & ડિફરન્શિયલ સિગ્નલ & કોમન મોડ રિજેક્શન \\ \hline
\end{tabulary}
\end{center}

\begin{mnemonicbox}
\mnemonic{TPLBV: T, Pi, L, Bridged-T, and Variable attenuators}
\end{mnemonicbox}
\end{solutionbox}

\questionmarks{5(c) OR}{7}{40dB નું એટેન્યુએશન આપવા અને 500$\Omega$ ના લોડમાં કામ કરવા માટે સિમેટ્રિકલ T ટાઈપ એટેન્યુએટર અને $\pi$ ટાઈપ એટેન્યુએટર ડિઝાઈન કરો.}

\begin{solutionbox}
\begin{center}
\begin{tabular}{cc}
\begin{circuitikz}[scale=0.8]
    \node at (2,2.5) {T-type};
    \draw (0,2) to[R, l=$R_1$] (2,2) to[R, l=$R_1$] (4,2);
    \draw (2,2) to[R, l=$R_2$] (2,0) -- (4,0) -- (0,0);
\end{circuitikz} &
\begin{circuitikz}[scale=0.8]
    \node at (2,2.5) {$\pi$-type};
    \draw (0,2) to[R, l=$R_2$] (4,2);
    \draw (0,2) to[R, l=$R_1$] (0,0) -- (4,0);
    \draw (4,2) to[R, l=$R_1$] (4,0);
\end{circuitikz}
\end{tabular}
\captionof{figure}{ડિઝાઈન કરેલ એટેન્યુએટર્સ}
\end{center}

\textbf{ડિઝાઈન ગણતરીઓ:}

\captionof{table}{ગણતરીના પગલાં}
\begin{center}
\begin{tabulary}{\linewidth}{|L|L|L|L|}
\hline
\textbf{પગલું} & \textbf{ફોર્મ્યુલા} & \textbf{ગણતરી} & \textbf{કિંમત} \\ \hline
આપેલ & એટેન્યુએશન = 40 dB & - & 40 dB \\ \hline
Step 1 & $N = 10^{(dB/20)}$ & $10^{(40/20)}$ & 100 \\ \hline
Step 2 & $K = (N-1)/(N+1)$ & $(100-1)/(100+1)$ & 0.98 \\ \hline
\end{tabulary}
\end{center}

T-ટાઈપ એટેન્યુએટર માટે:
\captionof{table}{T-ટાઈપ એટેન્યુએટર કિંમતો}
\begin{center}
\begin{tabulary}{\linewidth}{|L|L|L|L|}
\hline
\textbf{કમ્પોનન્ટ} & \textbf{ફોર્મ્યુલા} & \textbf{ગણતરી} & \textbf{કિંમત} \\ \hline
$R_1$ (સિરીઝ) & $Z_0 \cdot K$ & $500 \times 0.98$ & 490 $\Omega$ \\ \hline
$R_2$ (શંટ) & $Z_0/(K \cdot (N-K))$ & $500/(0.98 \times (100-0.98))$ & 5.15 $\Omega$ \\ \hline
\end{tabulary}
\end{center}

$\pi$-ટાઈપ એટેન્યુએટર માટે:
\captionof{table}{$\pi$-ટાઈપ એટેન્યુએટર કિંમતો}
\begin{center}
\begin{tabulary}{\linewidth}{|L|L|L|L|}
\hline
\textbf{કમ્પોનન્ટ} & \textbf{ફોર્મ્યુલા} & \textbf{ગણતરી} & \textbf{કિંમત} \\ \hline
$R_1$ (શંટ) & $Z_0/K$ & $500/0.98$ & 510.2 $\Omega$ \\ \hline
$R_2$ (સિરીઝ) & $Z_0 \cdot K \cdot (N-K)$ & $500 \times 0.98 \times (100-0.98)$ & 48,541 $\Omega$ \\ \hline
\end{tabulary}
\end{center}

\begin{mnemonicbox}
\mnemonic{DANK: dB Attenuation is Number K, which determines resistor values}
\end{mnemonicbox}
\end{solutionbox}

\end{document}
