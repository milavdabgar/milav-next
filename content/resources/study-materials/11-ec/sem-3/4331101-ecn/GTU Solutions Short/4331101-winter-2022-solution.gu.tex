\documentclass{article}

% content/resources/templates/preamble.tex
\usepackage[margin=0.6in]{geometry}
\author{Milav Dabgar}
\usepackage{amsmath,amssymb,amsthm}
\usepackage{booktabs}
\usepackage{multirow}
\usepackage{xcolor}
\usepackage{tcolorbox}
\tcbuselibrary{breakable,skins}
\usepackage[colorlinks=true,linkcolor=blue]{hyperref}
\usepackage{titlesec}
\usepackage{enumitem}
\usepackage{tikz}
\usepackage{pgfplots}
\usepackage{circuitikz}
\usepackage[version=4]{mhchem}
\usepackage{longtable}
\usepackage{array}
\usepackage{float}
\usepackage{caption}
\usepackage{listings}

\lstset{
  basicstyle=\small\ttfamily,
  breaklines=true,
  breakatwhitespace=false,
  postbreak=\mbox{\textcolor{red}{$\hookrightarrow$}\space},
  float=false,
  numbers=left,
  numberstyle=\tiny\color{gray},
  numbersep=10pt,
  xleftmargin=2em,
  keywordstyle=\color{blue},
  commentstyle=\color{green!60!black},
  stringstyle=\color{purple},
  backgroundcolor=\color{gray!5},
  showstringspaces=false,
  tabsize=2,
  captionpos=b,
  keepspaces=true,
  columns=flexible
}

\pgfplotsset{compat=1.18}
\usetikzlibrary{shapes,arrows,positioning,calc,patterns,decorations.pathmorphing,decorations.markings,arrows.meta}

% Color scheme
\definecolor{headcolor}{RGB}{0,102,204}
\definecolor{keycolor}{RGB}{220,20,60}
\definecolor{solutioncolor}{RGB}{34,139,34}
\definecolor{mnemoniccolor}{RGB}{148,0,211}
\definecolor{codecolor}{RGB}{0,0,100}

% Spacing
\setlength{\parskip}{3pt}
\setlist[itemize]{nosep}
\setlist[enumerate]{nosep}

% Title formatting
\titleformat{\section}{\Large\bfseries\color{headcolor}}{\thesection}{1em}{}
\titleformat{\subsection}{\large\bfseries\color{headcolor}}{\thesubsection}{1em}{}

% Pandoc tightlist compatibility
\providecommand{\tightlist}{%
  \setlength{\itemsep}{0pt}\setlength{\parskip}{0pt}}

% Pandoc longtable compatibility
\newcounter{none}
\def\thenone{}


% content/resources/templates/gujarati-boxes.tex
\usepackage{fontspec}
\usepackage{polyglossia}

% Set Gujarati as main language (document is primarily in Gujarati)
% Note: gloss-gujarati.ldf doesn't exist in polyglossia, but it will use hyphenation patterns
\setdefaultlanguage{gujarati}
\setotherlanguage{english}

% Configure Gujarati font properly
% Use Language=Default to prevent polyglossia from trying to add language-specific features
% that don't exist for Gujarati, which causes "empty feature" warnings
\newfontfamily\gujaratifont[Script=Gujarati,AutoFakeBold=2.5,AutoFakeSlant=0.3]{Noto Sans Gujarati}
\setmainfont[Script=Gujarati,AutoFakeBold=2.5,AutoFakeSlant=0.3]{Noto Sans Gujarati}
% Use Noto Sans Gujarati for monospace to support Gujarati in text
\setmonofont[Scale=0.9]{Noto Sans Gujarati}

% Configure English to use the same font
\newfontfamily\englishfont[Script=Gujarati,AutoFakeBold=2.5,AutoFakeSlant=0.3]{Noto Sans Gujarati}

% Translations for polyglossia
\gappto\captionsgujarati{
  \renewcommand{\tablename}{કોષ્ટક}
  \renewcommand{\figurename}{આકૃતિ}
}

% Helper for TikZ nodes to ensure Gujarati font
\newcommand{\gu}[1]{{\gujaratifont #1}}

% Custom environments
\newtcolorbox{solutionbox}{
    breakable,
    enhanced,
    colback=solutioncolor!5!white,
    colframe=solutioncolor!75!black,
    fonttitle=\bfseries,
    title=જવાબ
}

\newtcolorbox{solutionboxnobreak}{
 colback=solutioncolor!5!white,
 colframe=solutioncolor!75!black,
 fonttitle=\bfseries,
 title=જવાબ
}

\newtcolorbox{keyformula}{
 breakable,
 enhanced,
 colback=keycolor!5!white,
 colframe=keycolor!75!black,
 fonttitle=\bfseries,
 title=રાસાયણિક સમીકરણ/સૂત્ર
}

\newtcolorbox{mnemonicbox}{
 breakable,
 enhanced,
 colback=mnemoniccolor!5!white,
 colframe=mnemoniccolor!75!black,
 fonttitle=\bfseries,
 title=મેમરી ટ્રીક
}


% Custom commands for GTU solutions
% This file defines semantic commands for consistent formatting

% Question command with automatic formatting
\newcommand{\question}[2]{%
  \section*{Question #1}%
  \textbf{#2}%
}

% OR question variant
\newcommand{\questionor}[2]{%
  \section*{Question #1 OR}%
  \textbf{#2}%
}

% Proper table environment with caption
\newenvironment{answertable}[1]{%
  \begin{table}[htbp]
  \centering
  \caption{#1}
}{%
  \end{table}
}

% Proper figure environment for diagrams
\newenvironment{answerdiagram}[1]{%
  \begin{figure}[htbp]
  \centering
  \caption{#1}
}{%
  \end{figure}
}

% Semantic markup for key terms
\newcommand{\keyword}[1]{\textbf{#1}}
\newcommand{\code}[1]{\texttt{#1}}
\newcommand{\classname}[1]{\texttt{#1}}
\newcommand{\methodname}[1]{\texttt{#1}}

% Proper quotation marks
\newcommand{\mnemonic}[1]{``#1''}


\title{ઇલેક્ટ્રોનિક સર્કિટ્સ એન્ડ નેટવર્ક્સ (4331101) - વિન્ટર 2022 સોલ્યુશન}
\date{February 23, 2023}

\begin{document}
\maketitle

\section*{પ્રશ્ન 1(અ) [3 ગુણ]}
\questionmarks{1(અ)}{3}{ગુણ}

\textbf{વ્યાખ્યા આપો. : ૧) બ્રાંચ ૨) જંક્શન ૩) મેશ}

\begin{solutionbox}
\begin{itemize}
    \item \textbf{બ્રાંચ}: બ્રાંચ એટલે એક અથવા વધારે સર્કિટ તત્વો જે નેટવર્કના બે નોડ્સ વચ્ચે જોડાયેલા હોય.
    \item \textbf{જંક્શન}: જંક્શન (અથવા નોડ) એટલે એવું બિંદુ જ્યાં બે અથવા વધારે સર્કિટ તત્વો એકબીજા સાથે જોડાયેલા હોય.
    \item \textbf{મેશ}: મેશ એટલે નેટવર્કમાં એક બંધ પથ જેમાં અન્ય કોઈ બંધ પથ તેની અંદર ન હોય.
\end{itemize}

\begin{mnemonicbox}
"BJM: Branches Join at junctions to Make meshes"
\end{mnemonicbox}
\end{solutionbox}

\section*{પ્રશ્ન 1(બ) [4 ગુણ]}
\questionmarks{1(બ)}{4}{ગુણ}

\textbf{જરુરી સર્કિટ સાથે વોલ્ટેજ અને કરંટ ડિવિઝન નો નિયમ લખો.}

\begin{solutionbox}
\textbf{વોલ્ટેજ ડિવિઝન નિયમ}: સિરીઝ સર્કિટમાં, કોઈપણ ઘટક પરનો વોલ્ટેજ તેના રેઝિસ્ટન્સના પ્રમાણમાં હોય છે.

\begin{center}
\begin{circuitikz}[american, scale=0.8]
    \draw (0,0) to[V, l=$V_S$, invert] (0,3)
          to[R, l=$R_1$, v=$V_1$] (3,3)
          to[R, l=$R_2$, v=$V_2$] (3,0) -- (0,0);
\end{circuitikz}
\captionof{figure}{વોલ્ટેજ ડિવિઝન સર્કિટ}
\end{center}

\begin{itemize}
    \item \textbf{સૂત્ર}: $V_1 = V_S \times \frac{R_1}{R_1+R_2}$
    \item \textbf{ઉપયોગ}: સિરીઝ ઘટકો પરના વ્યક્તિગત વોલ્ટેજ ડ્રોપ્સ શોધવા માટે વપરાય છે
\end{itemize}

\textbf{કરંટ ડિવિઝન નિયમ}: પેરેલલ સર્કિટમાં, કોઈપણ શાખામાંથી પસાર થતો કરંટ તેના રેઝિસ્ટન્સના વ્યસ્ત પ્રમાણમાં હોય છે.

\begin{center}
\begin{circuitikz}[american, scale=0.8]
    \draw (0,0) to[I, l=$I_S$] (0,3) -- (4,3);
    \draw (2,3) to[R, l=$R_1$, i=$I_1$] (2,0);
    \draw (4,3) to[R, l=$R_2$, i=$I_2$] (4,0);
    \draw (0,0) -- (4,0);
\end{circuitikz}
\captionof{figure}{કરંટ ડિવિઝન સર્કિટ}
\end{center}

\begin{itemize}
    \item \textbf{સૂત્ર}: $I_1 = I_S \times \frac{R_2}{R_1+R_2}$
    \item \textbf{મુખ્ય સિદ્ધાંત}: કરંટ ઓછા રેઝિસ્ટન્સનો માર્ગ પસંદ કરે છે
\end{itemize}

\begin{mnemonicbox}
"VoSe CuPa: Voltage divides in Series, Current divides in Parallel"
\end{mnemonicbox}
\end{solutionbox}

\section*{પ્રશ્ન 1(ક) [7 ગુણ]}
\questionmarks{1(ક)}{7}{ગુણ}

\textbf{Fig. (૧) માં બતાવેલ નેટવર્ક માટે ગ્રાફ અને ટ્રી દોરો. ગ્રાફ પર લિંક કરંટ બતાવો. સાથે ટ્રી માટે ટાઇ-સેટ સેડ્યુલ લખો.}

\begin{solutionbox}
\textbf{નેટવર્કનો ગ્રાફ}:

\begin{center}
\begin{tikzpicture}[auto, node distance=2cm, main/.style={circle,draw,fill=blue!10,minimum size=0.8cm,inner sep=0pt}]
    \node[main] (A) {A};
    \node[main] (B) [right of=A] {B};
    \node[main] (C) [below right of=A] {C};
    \node[main] (D) [below left of=C] {D};

    \path[draw]
    (A) edge node {1} (B)
    (A) edge node {3} (C)
    (A) edge node {7} (D)
    (B) edge node {2} (D)
    (B) edge node {6} (C)
    (C) edge node {5} (D);
\end{tikzpicture}
\captionof{figure}{નેટવર્કનો ગ્રાફ}
\end{center}

\textbf{નેટવર્કનું ટ્રી} (સોલિડ માં ટ્વીગ્સ, ડેશ માં લિંક્સ):

\begin{center}
\begin{tikzpicture}[auto, node distance=2cm, main/.style={circle,draw,fill=blue!10,minimum size=0.8cm,inner sep=0pt}]
    \node[main] (A) {A};
    \node[main] (B) [right of=A] {B};
    \node[main] (C) [below right of=A] {C};
    \node[main] (D) [below left of=C] {D};

    % Tree branches (Twigs) - Solid
    \draw[thick] (A) -- node {1} (B);
    \draw[thick] (A) -- node {3} (C);
    \draw[thick] (C) -- node {5} (D);

    % Links - Dashed with Link Currents
    \draw[dashed, ->] (B) -- node[near start] {\gu{લિંક} 1 (2)} (D);
    \draw[dashed, ->] (B) -- node[near start] {\gu{લિંક} 2 (6)} (C);
    \draw[dashed, ->] (A) -- node[swap] {\gu{લિંક} 3 (7)} (D);
\end{tikzpicture}
\captionof{figure}{ટ્રી અને લિંક્સ}
\end{center}

\textbf{ટાઇ-સેટ સેડ્યુલ}:

\begin{tabulary}{\linewidth}{@{}|L|C|C|C|C|C|C|C|@{}}
    \hline
    \textbf{લિંક/ટ્રી શાખા} & \textbf{શાખા 1 (AB)} & \textbf{શાખા 3 (AC)} & \textbf{શાખા 4 (CD)} & \textbf{શાખા 2 (BD)} & \textbf{શાખા 6 (BC)} & \textbf{શાખા 7 (AD)} & \textbf{શાખા 5 (CD)} \\
    \hline
    લિંક 1 (BD) & 1 & 0 & 0 & 1 & 0 & 0 & 0 \\
    \hline
    લિંક 2 (BC) & 1 & 1 & 0 & 0 & 1 & 0 & 0 \\
    \hline
    લિંક 3 (AD) & 0 & 0 & 1 & 0 & 0 & 1 & 0 \\
    \hline
    લિંક 4 (CD) & 0 & 0 & 1 & 0 & 0 & 0 & 1 \\
    \hline
\end{tabulary}

\begin{mnemonicbox}
"TGLT: Trees Generate Link-current Tie-sets"
\end{mnemonicbox}
\end{solutionbox}

\section*{પ્રશ્ન 1(ક) OR [7 ગુણ]}
\questionmarks{1(ક) OR}{7}{ગુણ}

\textbf{Fig. (૧) માં બતાવેલ નેટવર્ક માટે ગ્રાફ અને ટ્રી દોરો. ટ્રી પર બ્રાંચ વોલ્ટેજ બતાવો. સાથે ટ્રી માટે કટ-સેટ સેડ્યુલ લખો.}

\begin{solutionbox}
\textbf{નેટવર્કનો ગ્રાફ}: ઉપર મુજબ.

\textbf{નેટવર્કનું ટ્રી}:

\begin{center}
\begin{tikzpicture}[auto, node distance=2cm, main/.style={circle,draw,fill=blue!10,minimum size=0.8cm,inner sep=0pt}]
    \node[main] (A) {A};
    \node[main] (B) [right of=A] {B};
    \node[main] (C) [below right of=A] {C};
    \node[main] (D) [below left of=C] {D};

    % Tree branches with Voltages
    \draw[thick, ->] (A) -- node {$V_1$} (B);
    \draw[thick, ->] (A) -- node {$V_3$} (C);
    \draw[thick, ->] (C) -- node {$V_4$} (D);
    
    % Links just to show context, ghosted
    \draw[dotted] (B) -- (D);
    \draw[dotted] (B) -- (C);
    \draw[dotted] (A) -- (D);
\end{tikzpicture}
\captionof{figure}{બ્રાંચ વોલ્ટેજ સાથે ટ્રી}
\end{center}

\textbf{કટ-સેટ સેડ્યુલ}:

\begin{tabulary}{\linewidth}{@{}|L|C|C|C|C|C|C|C|@{}}
    \hline
    \textbf{કટ-સેટ/શાખા} & \textbf{શાખા 1 (AB)} & \textbf{શાખા 3 (AC)} & \textbf{શાખા 4 (CD)} & \textbf{શાખા 2 (BD)} & \textbf{શાખા 6 (BC)} & \textbf{શાખા 7 (AD)} & \textbf{શાખા 5 (CD)} \\
    \hline
    કટ-સેટ 1 (AB) & 1 & 0 & 0 & -1 & -1 & 0 & 0 \\
    \hline
    કટ-સેટ 2 (AC) & 0 & 1 & 0 & 0 & 1 & -1 & 0 \\
    \hline
    કટ-સેટ 3 (CD) & 0 & 0 & 1 & 1 & 0 & 1 & 1 \\
    \hline
\end{tabulary}

\begin{mnemonicbox}
"CGVS: Cut-sets Generate Voltage Sources"
\end{mnemonicbox}
\end{solutionbox}

\section*{પ્રશ્ન 2(અ) [3 ગુણ]}
\questionmarks{2(અ)}{3}{ગુણ}

\textbf{વ્યાખ્યા આપો: ૧) એક્ટિવ અને પેસિવ નેટ્વર્ક ૨) યુનિલેટરલ અને બાઇ-લેટરલ નેટવર્ક.}

\begin{solutionbox}
\begin{itemize}
    \item \textbf{એક્ટિવ નેટવર્ક}: એવું નેટવર્ક જેમાં એક કે વધારે EMF સ્રોત (વોલ્ટેજ/કરંટ સ્રોત) હોય જે સર્કિટને ઊર્જા પૂરી પાડે છે.
    \item \textbf{પેસિવ નેટવર્ક}: એવું નેટવર્ક જેમાં માત્ર પેસિવ તત્વો જેવા કે રેઝિસ્ટર, કેપેસિટર અને ઇન્ડક્ટર હોય, કોઈ ઊર્જા સ્રોત ન હોય.
    
    \item \textbf{યુનિલેટરલ નેટવર્ક}: એવું નેટવર્ક જેમાં ઇનપુટ અને આઉટપુટ ટર્મિનલ્સ બદલવાથી તેની પ્રોપર્ટી અને પરફોર્મન્સ બદલાય છે.
    \item \textbf{બાઇલેટરલ નેટવર્ક}: એવું નેટવર્ક જેમાં ઇનપુટ અને આઉટપુટ ટર્મિનલ્સ બદલવાથી તેની પ્રોપર્ટી અને પરફોર્મન્સ સમાન રહે છે.
\end{itemize}

\begin{center}
\begin{tikzpicture}[gtu tree]
    \node [gtu root] {\gu{નેટવર્કના પ્રકાર}}
        child { node [gtu child] {\gu{એક્ટિવ}\\\gu{(સ્રોત ધરાવે છે)}} }
        child { node [gtu child] {\gu{પેસિવ}\\\gu{(સ્રોત નથી)}} }
        child { node [gtu child] {\gu{યુનિલેટરલ}\\\gu{(ડાયોડ/ટ્રાન્ઝિસ્ટર)}} }
        child { node [gtu child] {\gu{બાઇલેટરલ}\\\gu{(R, L, C તત્વો)}} };
\end{tikzpicture}
\captionof{figure}{નેટવર્ક વર્ગીકરણ}
\end{center}

\begin{mnemonicbox}
"APUB: Active Provides energy, Unilateral Blocks reversal"
\end{mnemonicbox}
\end{solutionbox}

\section*{પ્રશ્ન 2(બ) [4 ગુણ]}
\questionmarks{2(બ)}{4}{ગુણ}

\textbf{Z પેરામિટર માટે સમીકરણ લખો અને Z11, Z12, Z21, Z22 એ સમીકરણો પરથી તારવો.}

\begin{solutionbox}
Z-પેરામિટર્સ બે-પોર્ટ નેટવર્કમાં પોર્ટ વોલ્ટેજ અને કરંટ વચ્ચેનો સંબંધ વ્યાખ્યાયિત કરે છે:

\textbf{સમીકરણો}:
\begin{align*}
    V_1 &= Z_{11}I_1 + Z_{12}I_2 \\
    V_2 &= Z_{21}I_1 + Z_{22}I_2
\end{align*}

\textbf{તારણ}:
\begin{itemize}
    \item \textbf{$Z_{11} = \frac{V_1}{I_1} \bigg|_{I_2=0}$}: આઉટપુટ પોર્ટ ઓપન-સર્કિટ હોય ત્યારે ઇનપુટ ઇમ્પીડન્સ.
    \item \textbf{$Z_{12} = \frac{V_1}{I_2} \bigg|_{I_1=0}$}: ઇનપુટ પોર્ટ ઓપન-સર્કિટ હોય ત્યારે રિવર્સ ટ્રાન્સફર ઇમ્પીડન્સ.
    \item \textbf{$Z_{21} = \frac{V_2}{I_1} \bigg|_{I_2=0}$}: આઉટપુટ પોર્ટ ઓપન-સર્કિટ હોય ત્યારે ફોરવર્ડ ટ્રાન્સફર ઇમ્પીડન્સ.
    \item \textbf{$Z_{22} = \frac{V_2}{I_2} \bigg|_{I_1=0}$}: ઇનપુટ પોર્ટ ઓપન-સર્કિટ હોય ત્યારે આઉટપુટ ઇમ્પીડન્સ.
\end{itemize}

\begin{mnemonicbox}
"Z Impedance: Open circuit gives correct Parameters"
\end{mnemonicbox}
\end{solutionbox}

\section*{પ્રશ્ન 2(ક) [7 ગુણ]}
\questionmarks{2(ક)}{7}{ગુણ}

\textbf{સ્ટાન્ડર્ડ T નેટવર્ક માટે કેરક્ટરિસ્ટિક ઇમ્પિડન્સ (ZOT) નુ સમીકરણ તારવો.}

\begin{solutionbox}
સ્ટાન્ડર્ડ T-નેટવર્ક માટે:

\begin{center}
\begin{circuitikz}[american, scale=0.9]
    \draw (0,2) to[short, o-] (1,2) to[generic, l=$Z_1$] (3,2) coordinate(C) to[generic, l=$Z_2$] (5,2) to[short, -o] (6,2);
    \draw (C) to[generic, l=$Z_3$] (3,0) -- (3,0) coordinate(G);
    \draw (0,0) to[short, o-] (6,0) to[short, -o] (6,0);
\end{circuitikz}
\captionof{figure}{T-નેટવર્ક}
\end{center}

\textbf{તારણના પગલાં}:
\begin{enumerate}
    \item સિમેટ્રિક T-નેટવર્ક માટે, $Z_1 = Z_2$.
    \item મેચ્ડ કન્ડિશન હેઠળ, ઇનપુટ ઇમ્પિડન્સ કેરેક્ટરિસ્ટિક ઇમ્પિડન્સ બરાબર હોય.
    \item $Z_{0T} = Z_1 + \frac{Z_1 \times Z_3}{Z_1 + Z_3}$
    \item બેલેન્સ્ડ T-નેટવર્ક જ્યાં $Z_1 = Z_2 = Z/2$ અને $Z_3 = Z$ માટે:
    \item $Z_{0T} = \frac{Z}{2} + \frac{\frac{Z}{2} \times Z}{\frac{Z}{2} + Z}$
    \item $Z_{0T} = \frac{Z}{2} + \frac{Z^2/2}{3Z/2}$
    \item $Z_{0T} = \frac{Z}{2} + \frac{Z}{3}$
    \item $Z_{0T} = \frac{3Z + 2Z}{6}$
    \item $Z_{0T} = \sqrt{Z_1(Z_1 + 2Z_3)}$
\end{enumerate}

\textbf{અંતિમ સમીકરણ}: $Z_{0T} = \sqrt{Z_1(Z_1 + 2Z_3)}$

\begin{mnemonicbox}
"TO Impedance: Two arms Over middle branch"
\end{mnemonicbox}
\end{solutionbox}

\section*{પ્રશ્ન 2(અ) OR [3 ગુણ]}
\questionmarks{2(અ) OR}{3}{ગુણ}

\textbf{વ્યાખ્યા આપો. ૧) ડ્રાઇવીંગ પોઇંટ ઇમ્પીડન્સ ૨) ટ્રાન્સફર ઇમ્પીડન્સ}

\begin{solutionbox}
\begin{itemize}
    \item \textbf{ડ્રાઇવિંગ પોઇંટ ઇમ્પીડન્સ}: જ્યારે અન્ય બધા સ્વતંત્ર સ્રોત શૂન્ય પર સેટ હોય ત્યારે સમાન પોર્ટ/ટર્મિનલના જોડા પર વોલ્ટેજ અને કરંટનો ગુણોત્તર ($Z_{11} = V_1/I_1$).
    \item \textbf{ટ્રાન્સફર ઇમ્પીડન્સ}: જ્યારે અન્ય બધા સ્વતંત્ર સ્રોત શૂન્ય પર સેટ હોય ત્યારે એક પોર્ટ પર વોલ્ટેજ અને બીજા પોર્ટ પર કરંટનો ગુણોત્તર ($Z_{21} = V_2/I_1$).
\end{itemize}

\begin{center}
\begin{tikzpicture}[gtu block/.style={rectangle, draw, fill=blue!10, rounded corners, minimum height=2em}]
    \node[gtu block] (A) {\gu{ઇમ્પીડન્સના પ્રકાર}};
    \node[gtu block] (B) [below left=of A] {\gu{ડ્રાઇવિંગ પોઇંટ} ($V_1/I_1$)};
    \node[gtu block] (C) [below right=of A] {\gu{ટ્રાન્સફર} ($V_2/I_1$)};
    \draw[->] (A) -- (B);
    \draw[->] (A) -- (C);
\end{tikzpicture}
\end{center}

\begin{mnemonicbox}
"DTSS: Driving at Terminal Same, Transfer at Separate"
\end{mnemonicbox}
\end{solutionbox}

\section*{પ્રશ્ન 2(બ) OR [4 ગુણ]}
\questionmarks{2(બ) OR}{4}{ગુણ}

\textbf{કિર્ચોફનો વોલ્ટેજ લો ઉદાહરણ સાથે સમજાવો.}

\begin{solutionbox}
\textbf{કિર્ચોફનો વોલ્ટેજ લો (KVL)}: સર્કિટમાં કોઈપણ બંધ લૂપની આસપાસના તમામ વોલ્ટેજનો અલજેબ્રાઇક સરવાળો શૂન્ય હોય છે.

\textbf{ગણિતમાં}: $\sum V = 0$ (બંધ લૂપ આસપાસ)

\textbf{સર્કિટ ઉદાહરણ}:

\begin{center}
\begin{circuitikz}[american, scale=0.8]
    \draw (0,0) to[V, l=10V, invert] (0,3)
          to[R, l={$R_1=2\Omega$}] (3,3)
          to[R, l={$R_2=3\Omega$}] (3,0)
          to[R, l={$R_3=5\Omega$}] (0,0);
\end{circuitikz}
\captionof{figure}{KVL ઉદાહરણ સર્કિટ}
\end{center}

જો $I = 1A$, તો:
\begin{itemize}
    \item $V_1 = 1A \times 2\Omega = 2V$
    \item $V_2 = 1A \times 3\Omega = 3V$
    \item $V_3 = 1A \times 5\Omega = 5V$
\end{itemize}

KVL લાગુ કરતાં: $10V - 2V - 3V - 5V = 0$ \checkmark

\begin{mnemonicbox}
"VACZ: Voltages Around Closed loop are Zero"
\end{mnemonicbox}
\end{solutionbox}

\section*{પ્રશ્ન 2(ક) OR [7 ગુણ]}
\questionmarks{2(ક) OR}{7}{ગુણ}

\textbf{$\Pi$ નેટવર્ક માથી T નેટવર્ક મા બદલવાના સમીકણ તારવો.}

\begin{solutionbox}
\textbf{$\pi$ નેટવર્કને T નેટવર્કમાં રૂપાંતરણ}:

\begin{center}
\begin{circuitikz}[american, scale=0.8]
    % Pi Network
    \draw (0,2) to[short, o-] (1,2) to[short] (1,3) to[generic, l=$Y_a$] (4,3) to[short] (4,2) to[short, -o] (5,2);
    \draw (1,2) to[generic, l=$Y_c$] (1,0);
    \draw (4,2) to[generic, l=$Y_b$] (4,0);
    \draw (0,0) to[short, o-] (5,0) to[short, -o] (5,0);
    \node at (2.5, -1) {$\pi$ \gu{નેટવર્ક}};
    
    % Arrow
    \draw[->, thick] (6, 1.5) -- (8, 1.5);
    
    % T Network
    \draw (9,2) to[short, o-] (10,2) to[generic, l=$Z_a$] (12,2) coordinate(C) to[generic, l=$Z_b$] (14,2) to[short, -o] (15,2);
    \draw (C) to[generic, l=$Z_c$] (12,0) -- (12,0);
    \draw (9,0) to[short, o-] (15,0) to[short, -o] (15,0);
    \node at (12, -1) {T \gu{નેટવર્ક}};
\end{circuitikz}
\captionof{figure}{રૂપાંતરણ આકૃતિ}
\end{center}

\textbf{રૂપાંતરણ સમીકરણો}:
\begin{itemize}
    \item $Z_a = \frac{Y_a \times Y_c}{Y_\Delta}$
    \item $Z_b = \frac{Y_b \times Y_c}{Y_\Delta}$
    \item $Z_c = \frac{Y_a \times Y_b}{Y_\Delta}$
\end{itemize}
જ્યાં $Y_\Delta = Y_a + Y_b + Y_c$

\textbf{તારણ}:
\begin{enumerate}
    \item $\pi$-નેટવર્કના Y-પેરામિટર્સથી શરૂઆત કરો
    \item શાખા એડમિટન્સના સંદર્ભમાં Y-પેરામિટર્સને વ્યક્ત કરો
    \item મેટ્રિક્સ ઇન્વર્ઝનનો ઉપયોગ કરીને Z-પેરામિટર્સમાં રૂપાંતરિત કરો
    \item Z-પેરામિટર્સના સંદર્ભમાં T-નેટવર્ક ઇમ્પિડન્સને વ્યક્ત કરો
    \item સરળ બનાવીને ઉપરના રૂપાંતરણ સૂત્રો મેળવો
\end{enumerate}

\begin{mnemonicbox}
"PIE to TEA: Product over sum for opposite branch"
\end{mnemonicbox}
\end{solutionbox}

\section*{પ્રશ્ન 3(અ) [3 ગુણ]}
\questionmarks{3(અ)}{3}{ગુણ}

\textbf{કિર્ચોફનો કરંટ લો ઉદાહરણ સાથે સમજાવો.}

\begin{solutionbox}
\textbf{કિર્ચોફનો કરંટ લો (KCL)}: કોઈપણ નોડમાં પ્રવેશતા અને છોડતા તમામ કરંટનો અલજેબ્રાઇક સરવાળો શૂન્ય હોવો જોઈએ.

\textbf{ગણિતમાં}: $\sum I = 0$ (કોઈપણ નોડ પર)

\textbf{સર્કિટ ઉદાહરણ}:
\begin{center}
\begin{circuitikz}[american, scale=0.8]
    \node[circ] (B) at (2,2) {};
    \draw (0,4) to[short, i=$I_1=5A$] (B);
    \draw (0,0) to[short, i=$I_2=2A$] (B);
    \draw (B) to[short, i=$I_3=3A$] (4,4);
    \draw (B) to[short, i=$I_4=4A$] (4,0);
    \node[above] at (B) {\gu{નોડ} B};
\end{circuitikz}
\end{center}

નોડ B પર KCL લાગુ કરતાં:
\begin{itemize}
    \item પ્રવેશતા કરંટ: $I_1 + I_2 = 5A + 2A = 7A$
    \item છોડતા કરંટ: $I_3 + I_4 = 3A + 4A = 7A$
    \item તેથી: $I_1 + I_2 - I_3 - I_4 = 5 + 2 - 3 - 4 = 0$ \checkmark
\end{itemize}

\begin{mnemonicbox}
"CuNoZ: Currents at Node are Zero"
\end{mnemonicbox}
\end{solutionbox}

\section*{પ્રશ્ન 3(બ) [4 ગુણ]}
\questionmarks{3(બ)}{4}{ગુણ}

\textbf{જરુરી સમીકરણો સાથે મેશ એનાલિસિસ સમજાવો.}

\begin{solutionbox}
\textbf{મેશ એનાલિસિસ}: એક સર્કિટ એનાલિસિસ તકનીક જે મલ્ટિપલ લૂપ્સ વાળી સર્કિટને ઉકેલવા માટે મેશ કરંટ્સનો ઉપયોગ કરે છે.

\textbf{પગલાં}:
\begin{enumerate}
    \item સર્કિટમાં બધા મેશ (બંધ લૂપ) ઓળખો
    \item દરેક મેશને મેશ કરંટ સોંપો
    \item દરેક મેશ પર KVL લાગુ કરો
    \item પરિણામી સમીકરણ સિસ્ટમને ઉકેલો
\end{enumerate}

\textbf{ઉદાહરણ સર્કિટ}:
\begin{center}
\begin{circuitikz}[american, scale=0.8]
    \draw (0,0) to[V, l=$V_1$, invert] (0,3)
          to[R, l=$R_1$] (3,3)
          to[R, l=$R_2$] (3,0) -- (0,0);
    \draw (3,3) to[R, l=$R_3$] (6,3)
          to[V, l=$V_2$] (6,0) -- (3,0);
    \draw[->] (1.5,1.5) arc (-60:240:0.5) node[midway, below] {$I_1$};
    \draw[->] (4.5,1.5) arc (-60:240:0.5) node[midway, below] {$I_2$};
\end{circuitikz}
\captionof{figure}{મેશ એનાલિસિસ ઉદાહરણ}
\end{center}

\textbf{સમીકરણો}:
\begin{itemize}
    \item મેશ 1: $V_1 = I_1R_1 + (I_1-I_2)R_2$
    \item મેશ 2: $V_2 = I_2R_3 + (I_2-I_1)R_2$
\end{itemize}

\begin{mnemonicbox}
"MILK: Mesh Is Loop with KVL"
\end{mnemonicbox}
\end{solutionbox}

\section*{પ્રશ્ન 3(ક) [7 ગુણ]}
\questionmarks{3(ક)}{7}{ગુણ}

\textbf{થેવેનિનનો પ્રમેય લખો અને સમજાવો.}

\begin{solutionbox}
\textbf{થેવેનિનનો પ્રમેય}: વોલ્ટેજ અને કરંટ સ્રોતો ધરાવતા કોઈપણ લીનિયર નેટવર્કને વોલ્ટેજ સ્રોત ($V_{TH}$) અને રેઝિસ્ટન્સ ($R_{TH}$) ના સિરીઝ જોડાણથી બદલી શકાય છે.

\begin{center}
\begin{circuitikz}[american]
    \draw (0,0) node[circ, label=left:B]{} to[short] (1,0) to[R, l=$R_{TH}$] (3,0) to[V, l=$V_{TH}$] (3,2) to[short] (0,2) node[circ, label=left:A]{};
    \draw (4,1) node{\Huge $\equiv$};
    \draw (5,0) node[draw, minimum width=2cm, minimum height=2cm] (box) {\gu{લીનિયર નેટવર્ક}};
    \node at (4.5, 2) {\gu{ઓરિજિનલ}};
    \node at (1.5, 2.5) {\gu{ઈક્વિવલેન્ટ}};
\end{circuitikz}
\captionof{figure}{થેવેનિન ઈક્વિવલેન્ટ}
\end{center}

\textbf{થેવેનિન ઈક્વિવલેન્ટ શોધવાના પગલાં}:
\begin{enumerate}
    \item જે ટર્મિનલ્સ વચ્ચે શોધવું હોય ત્યાંથી લોડ દૂર કરો
    \item આ ટર્મિનલ્સ પર ઓપન-સર્કિટ વોલ્ટેજ ($V_{OC}$) શોધો (= $V_{TH}$)
    \item સર્કિટમાં પાછા જોતા રેઝિસ્ટન્સ શોધો જ્યારે બધા સ્રોતોને તેમના આંતરિક રેઝિસ્ટન્સથી બદલવામાં આવે (= $R_{TH}$)
    \item થેવેનિન ઈક્વિવલેન્ટ માં $V_{TH}$ સાથે સિરીઝમાં $R_{TH}$ હોય છે
\end{enumerate}

\begin{mnemonicbox}
"TORV: Thevenin's Open-circuit Resistance and Voltage"
\end{mnemonicbox}
\end{solutionbox}

\section*{પ્રશ્ન 3(અ) OR [3 ગુણ]}
\questionmarks{3(અ) OR}{3}{ગુણ}

\textbf{રેસીપ્રોસિટી થીયરમ લખો અને સમજાવો.}

\begin{solutionbox}
\textbf{રેસીપ્રોસિટી પ્રમેય}: લીનિયર, બાઇલેટરલ નેટવર્કમાં, જો એક શાખામાં વોલ્ટેજ સ્રોત બીજી શાખામાં કરંટ ઉત્પન કરે છે, તો તે જ વોલ્ટેજ સ્રોત, જો બીજી શાખામાં મૂકવામાં આવે તો, પહેલી શાખામાં તેટલો જ કરંટ ઉત્પન કરશે.

\begin{center}
\begin{circuitikz}[american, scale=0.8]
    % Case 1
    \draw (0,0) to[V, l=$V$] (0,2) -- (1,2) to[twoport, t=Network] (3,2) -- (4,2) to[rmeter, t=A] (4,0) -- (0,0);
    \node at (2,-0.5) {\gu{મૂળ સ્થિતિ}};
    
    % Case 2
    \draw (6,0) to[rmeter, t=A] (6,2) -- (7,2) to[twoport, t=Network] (9,2) -- (10,2) to[V, l=$V$] (10,0) -- (6,0);
    \node at (8,-0.5) {\gu{રેસીપ્રોકલ સ્થિતિ}};
\end{circuitikz}
\end{center}

\textbf{ગણિતમાં}: જો શાખા 1 માં વોલ્ટેજ $V_1$ શાખા 2 માં કરંટ $I_2$ આપે છે, તો શાખા 2 માં વોલ્ટેજ $V_1$ શાખા 1 માં કરંટ $I_2$ આપશે.

\textbf{મર્યાદાઓ}: માત્ર આ નેટવર્ક્સ માટે લાગુ પડે છે:
\begin{itemize}
    \item લીનિયર ઘટકો
    \item બાઇલેટરલ ઘટકો (ડાયોડ, ટ્રાન્ઝિસ્ટર ન હોવા જોઈએ)
    \item એક જ સ્વતંત્ર સ્રોત
\end{itemize}

\begin{mnemonicbox}
"RESWAP: REciprocity SWAPs Position with identical results"
\end{mnemonicbox}
\end{solutionbox}

\section*{પ્રશ્ન 3(બ) OR [4 ગુણ]}
\questionmarks{3(બ) OR}{4}{ગુણ}

\textbf{જરુરી સમીકરણો સાથે નોડલ એનાલિસિસ સમજાવો.}

\begin{solutionbox}
\textbf{નોડલ એનાલિસિસ}: એક સર્કિટ એનાલિસિસ તકનીક જે સર્કિટ ઉકેલવા માટે નોડ વોલ્ટેજને વેરિએબલ તરીકે ઉપયોગ કરે છે.

\textbf{પગલાં}:
\begin{enumerate}
    \item રેફરન્સ નોડ (ગ્રાઉન્ડ) પસંદ કરો
    \item બાકીના નોડ્સને વોલ્ટેજ વેરિએબલ સોંપો
    \item દરેક નોન-રેફરન્સ નોડ પર KCL લાગુ કરો
    \item પરિણામી સમીકરણ સિસ્ટમને ઉકેલો
\end{enumerate}

\textbf{ઉદાહરણ સર્કિટ}:
\begin{center}
\begin{circuitikz}[american, scale=0.8]
    \draw (0,0) node[ground]{} to[I, l=$I_1$, invert] (0,3) node[circ, label={Node 1}]{} 
          to[R, l=$G_3$] (4,3) node[circ, label={Node 2}]{}
          to[I, l=$I_2$] (4,0) node[ground]{};
    \draw (0,3) to[R, l=$G_1$] (0,0);
    \draw (4,3) to[R, l=$G_2$] (4,0);
\end{circuitikz}
\captionof{figure}{નોડલ એનાલિસિસ}
\end{center}

\textbf{સમીકરણો}:
\begin{itemize}
    \item નોડ 1: $I_1 = V_1G_1 + (V_1-V_2)G_3$
    \item નોડ 2: $I_2 = V_2G_2 + (V_2-V_1)G_3$
\end{itemize}

\begin{mnemonicbox}
"NKCV: Nodal uses KCL with Voltage variables"
\end{mnemonicbox}
\end{solutionbox}

\section*{પ્રશ્ન 3(ક) OR [7 ગુણ]}
\questionmarks{3(ક) OR}{7}{ગુણ}

\textbf{મેક્સિમમ પાવર ટ્રાન્સફર થીયરમ લખો અને સાબિત કરો.}

\begin{solutionbox}
\textbf{મેક્સિમમ પાવર ટ્રાન્સફર પ્રમેય}: સ્રોત સાથે જોડાયેલ લોડ મહત્તમ પાવર ત્યારે જ ખેંચશે જ્યારે તેનો રેઝિસ્ટન્સ સ્રોતના આંતરિક રેઝિસ્ટન્સ બરાબર હશે.

\begin{center}
\begin{circuitikz}[american]
    \draw (0,0) to[V, l=$V_S$] (0,3) to[R, l=$R_S$] (3,3) 
          to[R, l=$R_L$] (3,0) -- (0,0);
\end{circuitikz}
\end{center}

\textbf{સાબિતી}:
\begin{enumerate}
    \item સર્કિટમાં કરંટ: $I = \frac{V_S}{R_S + R_L}$
    \item લોડને મળતો પાવર: $P = I^2R_L = \frac{V_S^2R_L}{(R_S + R_L)^2}$
    \item મહત્તમ પાવર માટે, $\frac{dP}{dR_L} = 0$
    \item ઉકેલતા: $\frac{V_S^2(R_S + R_L)^2 - V_S^2R_L \cdot 2(R_S + R_L)}{(R_S + R_L)^4} = 0$
    \item સાદુરૂપ: $(R_S + R_L)^2 = 2R_L(R_S + R_L)$
    \item વધુ સાદુરૂપ: $R_S + R_L = 2R_L$
    \item તેથી: $R_S = R_L$
\end{enumerate}

\textbf{મહત્તમ પાવર}: $P_{max} = \frac{V_S^2}{4R_S}$

\begin{mnemonicbox}
"MaRLRS: Maximum power when load Resistance equals Source Resistance"
\end{mnemonicbox}
\end{solutionbox}

\section*{પ્રશ્ન 4(અ) [3 ગુણ]}
\questionmarks{4(અ)}{3}{ગુણ}

\textbf{શુ કામ સિરીઝ રેઝોનન્સ સર્કિટ વોલ્ટેજ એમ્પ્લીફાયર તરીકે અને પેરેલલ રેઝોનન્સ સર્કિટ કરંટ એમ્પ્લીફાયર તરીકે વર્તે છે?}

\begin{solutionbox}
\textbf{વોલ્ટેજ એમ્પ્લીફાયર તરીકે સિરીઝ રેઝોનન્સ}:
\begin{itemize}
    \item રેઝોનન્સ પર, સિરીઝ સર્કિટ ઇમ્પિડન્સ ન્યૂનતમ છે (માત્ર R)
    \item L અથવા C પરનો વોલ્ટેજ સ્રોત વોલ્ટેજ કરતાં ઘણો વધારે હોઈ શકે છે
    \item વોલ્ટેજ મેગ્નિફિકેશન ફેક્ટર = $Q = \frac{X_L}{R} = \frac{1}{R}\sqrt{\frac{L}{C}}$
    \item L અથવા C પર વોલ્ટેજ = $Q \times$ સ્રોત વોલ્ટેજ
\end{itemize}

\textbf{કરંટ એમ્પ્લીફાયર તરીકે પેરેલલ રેઝોનન્સ}:
\begin{itemize}
    \item રેઝોનન્સ પર, પેરેલલ સર્કિટ ઇમ્પિડન્સ મહત્તમ છે
    \item L અથવા C માં કરંટ સ્રોત કરંટ કરતાં ઘણો વધારે હોઈ શકે છે
    \item કરંટ મેગ્નિફિકેશન ફેક્ટર = $Q = \frac{R}{X_L} = R\sqrt{\frac{C}{L}}$
    \item L અથવા C માં કરંટ = $Q \times$ સ્રોત કરંટ
\end{itemize}

\textbf{કોષ્ટક}:
\begin{tabulary}{\linewidth}{@{}|L|L|L|@{}}
    \hline
    \textbf{સર્કિટ પ્રકાર} & \textbf{રેઝોનન્સ પર ઇમ્પીડન્સ} & \textbf{એમ્પ્લીફિકેશન} \\
    \hline
    સિરીઝ & ન્યૂનતમ ($R$ માત્ર) & વોલ્ટેજ ($V_L$ કે $V_C = Q \times V_S$) \\
    \hline
    પેરેલલ & મહત્તમ ($R^2/r$) & કરંટ ($I_L$ કે $I_C = Q \times I_S$) \\
    \hline
\end{tabulary}

\begin{mnemonicbox}
"SeVoPa: Series Voltage, Parallel current amplification"
\end{mnemonicbox}
\end{solutionbox}

\section*{પ્રશ્ન 4(બ) [4 ગુણ]}
\questionmarks{4(બ)}{4}{ગુણ}

\textbf{કોઇલના Q નુ સમીકરણ તારવો.}

\begin{solutionbox}
\textbf{કોઇલનો Q-ફેક્ટર}:

\begin{center}
\begin{circuitikz}[american]
    \draw (0,0) to[R, l=$R$, o-o] (2,0) to[L, l=$L$, o-o] (4,0);
\end{circuitikz}
\end{center}

\textbf{તારણ}:
\begin{enumerate}
    \item Q-ફેક્ટર વ્યાખ્યાયિત છે: $Q = \frac{\text{Energy stored}}{\text{Energy dissipated per cycle}}$ (સંગ્રહિત ઊર્જા / વ્યય થતી ઊર્જા)
    \item ઇન્ડક્ટરમાં સંગ્રહિત ઊર્જા = $\frac{1}{2}LI^2$
    \item રેઝિસ્ટરમાં વ્યય થતો પાવર = $I^2R$
    \item પ્રતિ સાયકલ વ્યય થતી ઊર્જા = પાવર $\times$ સમયગાળો = $I^2R \times \frac{1}{f}$
    \item તેથી: $Q = \frac{\frac{1}{2}LI^2}{I^2R \times \frac{1}{f}}$
    \item સાદુરૂપ: $Q = \frac{2\pi \times \frac{1}{2}LI^2 \times f}{I^2R}$
    \item $Q = \frac{2\pi f \times L}{R} = \frac{\omega L}{R}$
\end{enumerate}

\textbf{અંતિમ સમીકરણ}: $Q = \frac{\omega L}{R} = \frac{2\pi fL}{R} = \frac{X_L}{R}$

\begin{mnemonicbox}
"QualityEDR: Quality equals Energy stored Divided by energy lost per Radian"
\end{mnemonicbox}
\end{solutionbox}

\section*{પ્રશ્ન 4(ક) [7 ગુણ]}
\questionmarks{4(ક)}{7}{ગુણ}

\textbf{સિરીઝ R-L-C સર્કિટ ની રેઝોનન્સ ફ્રિક્વન્સીનુ સમીકરણ તારવો. }

\begin{solutionbox}
\textbf{સિરીઝ R-L-C સર્કિટ}:

\begin{center}
\begin{circuitikz}[american]
    \draw (0,0) to[short, o-] (1,0) to[R, l=$R$] (3,0) to[L, l=$L$] (5,0) to[C, l=$C$] (7,0) to[short, -o] (8,0);
\end{circuitikz}
\captionof{figure}{સિરીઝ RLC સર્કિટ}
\end{center}

\textbf{તારણ}:
\begin{enumerate}
    \item સિરીઝ RLC સર્કિટનું ઇમ્પિડન્સ: $Z = R + j(X_L - X_C)$
    \item જ્યાં: $X_L = \omega L$ અને $X_C = \frac{1}{\omega C}$
    \item રેઝોનન્સ પર, $X_L = X_C$ (ઇન્ડક્ટિવ અને કેપેસિટિવ રિએક્ટન્સ સમાન છે)
    \item તેથી: $\omega L = \frac{1}{\omega C}$
    \item $\omega$ માટે ઉકેલતા: $\omega^2 = \frac{1}{LC}$
    \item રેઝોનન્ટ ફ્રિક્વન્સી: $\omega_0 = \frac{1}{\sqrt{LC}}$
    \item ફ્રિક્વન્સી f ના સંદર્ભમાં: $f_0 = \frac{1}{2\pi\sqrt{LC}}$
\end{enumerate}

\textbf{રેઝોનન્સ પર ખાસિયતો}:
\begin{itemize}
    \item ઇમ્પિડન્સ ન્યૂનતમ છે (સંપૂર્ણપણે રેઝિસ્ટિવ: $Z = R$)
    \item કરંટ મહત્તમ છે ($I = V/R$)
    \item પાવર ફેક્ટર યુનિટી છે (સર્કિટ રેઝિસ્ટિવ દેખાય છે)
    \item L અને C પરના વોલ્ટેજ સમાન અને વિરુદ્ધ છે
\end{itemize}

\begin{mnemonicbox}
"RES: Reactances Equal at Series resonance"
\end{mnemonicbox}
\end{solutionbox}

\section*{પ્રશ્ન 4(અ) OR [3 ગુણ]}
\questionmarks{4(અ) OR}{3}{ગુણ}

\textbf{કપલ્ડ સર્કિટ એટલે શું? સેલ્ફ ઇન્ડક્ટન્સ અને મ્યુચ્યુઅલ ઇન્ડક્ટન્સ ની વ્યાખ્યા આપો.}

\begin{solutionbox}
\textbf{કપલ્ડ સર્કિટ્સ}: બે કે તેથી વધુ સર્કિટ જે ચુંબકીય રીતે જોડાયેલી હોય જેથી તેમના પરસ્પર ચુંબકીય ક્ષેત્ર દ્વારા ઊર્જા એકમાંથી બીજામાં સ્થાનાંતરિત થઈ શકે.

\begin{center}
\begin{circuitikz}[american]
    \draw (0,0) to[L, l=$L_1$] (0,2);
    \draw (2,0) to[L, l=$L_2$] (2,2);
    \draw[<->, dashed] (0.5, 1) -- (1.5, 1) node[midway, above] {$M$};
\end{circuitikz}
\captionof{figure}{કપલ્ડ કોઇલ્સ}
\end{center}

\textbf{સેલ્ફ-ઇન્ડક્ટન્સ (L)}: સર્કિટનો ગુણધર્મ જેના દ્વારા કરંટમાં ફેરફાર તે જ સર્કિટમાં સેલ્ફ-ઇન્ડ્યુસ્ડ EMF ઉત્પન્ન કરે છે.
$L = \Phi/I$ (તે ઉત્પન્ન કરતા કરંટ સાથે મેગ્નેટિક ફ્લક્સનો ગુણોત્તર)

\textbf{મ્યુચ્યુઅલ ઇન્ડક્ટન્સ (M)}: સર્કિટનો ગુણધર્મ જેના દ્વારા એક સર્કિટમાં કરંટનો ફેરફાર બીજી સર્કિટમાં EMF ઉત્પન્ન કરે છે.
$M = \Phi_{21}/I_1$ (સર્કિટ 1 ના કરંટને લીધે સર્કિટ 2 માં ઉત્પન્ન થતા ફ્લક્સનો ગુણોત્તર)

\begin{mnemonicbox}
"SiMu: Self in Mine, Mutual in Yours"
\end{mnemonicbox}
\end{solutionbox}

\section*{પ્રશ્ન 4(બ) OR [4 ગુણ]}
\questionmarks{4(બ) OR}{4}{ગુણ}

\textbf{કોએફિશીયન્ટ ઓફ કપલિંગ (K) નુ સમીકરણ તારવો.}

\begin{solutionbox}
\textbf{કોએફિશિયન્ટ ઓફ કપલિંગ (k)}:

\textbf{તારણ}:
\begin{enumerate}
    \item બે કોઇલ વચ્ચેનો મ્યુચ્યુઅલ ઇન્ડક્ટન્સ (M) આધાર રાખે છે:
    \begin{itemize}
        \item કોઇલના સેલ્ફ-ઇન્ડક્ટન્સ ($L_1$ અને $L_2$)
        \item ભૌતિક ગોઠવણ (નિકટતા અને દિશા)
    \end{itemize}
    \item મહત્તમ શક્ય મ્યુચ્યુઅલ ઇન્ડક્ટન્સ: $M_{max} = \sqrt{L_1L_2}$
    \item કપલિંગનો કોએફિશિયન્ટ આ રીતે વ્યાખ્યાયિત થાય છે: $k = \frac{M}{M_{max}}$
    \item તેથી: $k = \frac{M}{\sqrt{L_1L_2}}$
\end{enumerate}

\textbf{લક્ષણો}:
\begin{itemize}
    \item $k$ ની કિંમત 0 (કપલિંગ નથી) થી 1 (પૂર્ણ કપલિંગ) સુધી હોય છે
    \item $k$ ભૂમિતિ, અભિગમ અને માધ્યમ પર આધાર રાખે છે
    \item સામાન્ય ટ્રાન્સફોર્મર: $k = 0.95$ થી $0.99$
    \item એર-કોર કોઇલ: $k = 0.01$ થી $0.5$
\end{itemize}

\begin{mnemonicbox}
"KMutual: K Measures Mutual linkage proportion"
\end{mnemonicbox}
\end{solutionbox}

\section*{પ્રશ્ન 4(ક) OR [7 ગુણ]}
\questionmarks{4(ક) OR}{7}{ગુણ}

\textbf{એક RLC સિરીઝ સર્કીટ મા R=30\Omega, L=0.5H, અને C=5$\mu$F આપેલ છે. તો ગણતરી કરો (૧) રેઝોનન્સ ફ્રિકવન્સી (૨) Q ફેક્ટર (૩) બેન્ડ વિડ્થ}

\begin{solutionbox}
\textbf{આપેલ છે}:
\begin{itemize}
    \item રેઝિસ્ટન્સ, $R = 30\Omega$
    \item ઇન્ડક્ટન્સ, $L = 0.5H$
    \item કેપેસિટન્સ, $C = 5\mu F = 5\times 10^{-6}F$
\end{itemize}

\textbf{ગણતરી}:

\textbf{(i) રેઝોનન્સ ફ્રિકવન્સી}:
\begin{itemize}
    \item $f_0 = \frac{1}{2\pi\sqrt{LC}}$
    \item $f_0 = \frac{1}{2\pi\sqrt{0.5 \times 5\times 10^{-6}}}$
    \item $f_0 = 100.76 \text{ Hz} \approx 100 \text{ Hz}$
\end{itemize}

\textbf{(ii) Q ફેક્ટર}:
\begin{itemize}
    \item $Q = \frac{1}{R}\sqrt{\frac{L}{C}}$
    \item $Q = \frac{1}{30}\sqrt{\frac{0.5}{5\times 10^{-6}}}$
    \item $Q = 10.54$
\end{itemize}

\textbf{(iii) બેન્ડવિડ્થ (BW)}:
\begin{itemize}
    \item $BW = \frac{f_0}{Q}$
    \item $BW = \frac{100.76}{10.54} = 9.56 \text{ Hz}$
\end{itemize}

\textbf{કોષ્ટક}:
\begin{tabulary}{\linewidth}{@{}|L|L|L|@{}}
    \hline
    \textbf{પેરામીટર} & \textbf{સૂત્ર} & \textbf{કિંમત} \\
    \hline
    રેઝોનન્ટ ફ્રિકવન્સી ($f_0$) & $\frac{1}{2\pi\sqrt{LC}}$ & 100 Hz \\
    \hline
    ક્વોલિટી ફેક્ટર ($Q$) & $\frac{1}{R}\sqrt{\frac{L}{C}}$ & 10.54 \\
    \hline
    બેન્ડવિડ્થ ($BW$) & $f_0/Q$ & 9.56 Hz \\
    \hline
\end{tabulary}

\begin{mnemonicbox}
"RQB: Resonance Quality determines Bandwidth"
\end{mnemonicbox}
\end{solutionbox}

\section*{પ્રશ્ન 5(અ) [3 ગુણ]}
\questionmarks{5(અ)}{3}{ગુણ}

\textbf{એટેન્યુએટરના પ્રકારોનું વર્ગીકરણ કરો.}

\begin{solutionbox}
\textbf{એટેન્યુએટર્સ}: રેઝિસ્ટરનું નેટવર્ક જે સિગ્નલ લેવલને ડિસ્ટોર્શન વગર ઘટાડવા (એટેન્યુએટ) માટે રચાયેલ છે.

\textbf{એટેન્યુએટર્સના પ્રકાર}:
\begin{center}
\begin{tikzpicture}[gtu tree]
    \node [gtu root] {Attenuators}
        child { 
            node [gtu child] {\gu{ફિક્સ્ડ}} 
            child { node [gtu child] {T-\gu{ટાઇપ}} }
            child { node [gtu child] {$\pi$-\gu{ટાઇપ}} }
            child { node [gtu child] {\gu{બ્રિજ્ડ}-T} }
            child { node [gtu child] {\gu{લેટિસ}} }
        }
        child { 
            node [gtu child] {\gu{વેરિએબલ}}
            child { node [gtu child] {\gu{સ્ટેપ}} }
            child { node [gtu child] {\gu{કન્ટિન્યુઅસ}} }
        };
\end{tikzpicture}
\captionof{figure}{એટેન્યુએટર્સનું વર્ગીકરણ}
\end{center}

\begin{itemize}
    \item \textbf{રચનાના આધારે}: T-ટાઇપ, $\pi$-ટાઇપ, બ્રિજ્ડ-T, લેટિસ
    \item \textbf{સપ્રમાણતાના આધારે}: સિમેટ્રિકલ (સમાન Z ઇન/આઉટ), અસિમેટ્રિકલ
\end{itemize}

\begin{mnemonicbox}
"ATP Fixed: Attenuator Types include Pad, Tee, Lattice"
\end{mnemonicbox}
\end{solutionbox}

\section*{પ્રશ્ન 5(બ) [4 ગુણ]}
\questionmarks{5(બ)}{4}{ગુણ}

\textbf{એટેન્યુએટર અને નેપર વચ્ચેનો સંબંધ તારવો.}

\begin{solutionbox}
\textbf{એટેન્યુએશન અને નેપર વચ્ચેનો સંબંધ}:

\begin{itemize}
    \item \textbf{એટેન્યુએશન ($\alpha$)}: ઇનપુટ વોલ્ટેજ (કે કરંટ) અને આઉટપુટ વોલ્ટેજ (કે કરંટ) નો ગુણોત્તર.
    \item \textbf{નેપર (Np)}: ગુણોત્તરોનું નેચરલ લોગેરિધમિક એકમ.
\end{itemize}

\textbf{તારણ}:
\begin{enumerate}
    \item વોલ્ટેજ ગુણોત્તર $V_1/V_2$ માટે:
    \begin{itemize}
        \item નેપરમાં એટેન્યુએશન = $\ln(V_1/V_2)$
        \item ડેસિબલમાં એટેન્યુએશન = $20\log_{10}(V_1/V_2)$
    \end{itemize}
    
    \item પાવર ગુણોત્તર $P_1/P_2$ માટે:
    \begin{itemize}
        \item નેપરમાં એટેન્યુએશન = $\frac{1}{2}\ln(P_1/P_2)$
        \item ડેસિબલમાં એટેન્યુએશન = $10\log_{10}(P_1/P_2)$
    \end{itemize}
    
    \item dB અને નેપર વચ્ચેનો સંબંધ:
    \begin{itemize}
        \item 1 નેપર = 8.686 dB
        \item 1 dB = 0.115 નેપર
    \end{itemize}
\end{enumerate}

\textbf{કોષ્ટક}:
\begin{tabulary}{\linewidth}{@{}|L|L|L|@{}}
    \hline
    \textbf{એકમ} & \textbf{વોલ્ટેજ ગુણોત્તર} & \textbf{પાવર ગુણોત્તર} \\
    \hline
    નેપર (Np) & $\ln(V_1/V_2)$ & $\frac{1}{2}\ln(P_1/P_2)$ \\
    \hline
    ડેસિબલ (dB) & $20\log_{10}(V_1/V_2)$ & $10\log_{10}(P_1/P_2)$ \\
    \hline
\end{tabulary}

\begin{mnemonicbox}
"NED: Neper Equals Decibel divided by 8.686"
\end{mnemonicbox}
\end{solutionbox}

\section*{પ્રશ્ન 5(ક) [7 ગુણ]}
\questionmarks{5(ક)}{7}{ગુણ}

\textbf{સિમેટ્રીકલ T એટેન્યુએટર માટે R1 અને R2 ના સમીકણો તારવો.}

\begin{solutionbox}
\textbf{સિમેટ્રિકલ T એટેન્યુએટર}:

\begin{center}
\begin{circuitikz}[american, scale=0.9]
    \draw (0,2) to[short, o-] (1,2) to[R, l=$R_1$] (3,2) coordinate(C) to[R, l=$R_1$] (5,2) to[short, -o] (6,2);
    \draw (C) to[R, l=$R_2$] (3,0) -- (3,0) coordinate(G);
    \draw (0,0) to[short, o-] (6,0) to[short, -o] (6,0);
\end{circuitikz}
\captionof{figure}{સિમેટ્રિકલ T એટેન્યુએટર}
\end{center}

\textbf{તારણ}:
\begin{enumerate}
    \item કેરેક્ટરિસ્ટિક ઇમ્પિડન્સ $Z_0$ સાથેના સિમેટ્રિકલ T-એટેન્યુએટર માટે:
    \begin{itemize}
        \item ઇનપુટ અને આઉટપુટ ઇમ્પિડન્સ બંને $Z_0$ બરાબર હોવા જોઈએ
        \item એટેન્યુએશન ગુણોત્તર $N = V_1/V_2 = I_2/I_1$
    \end{itemize}
    
    \item સર્કિટ એનાલિસિસથી:
    \begin{itemize}
        \item $R_1 = Z_0 \frac{N-1}{N+1}$
        \item $R_2 = \frac{2Z_0N}{N^2-1}$
    \end{itemize}
    
    \item dB માં એટેન્યુએશન ($\alpha$) માટે:
    \begin{itemize}
        \item $N = 10^{\alpha/20}$
        \item $R_1 = Z_0 \tanh(\alpha/2)$
        \item $R_2 = Z_0 / \sinh(\alpha)$
    \end{itemize}
\end{enumerate}

\textbf{અંતિમ સમીકરણો}:
\begin{itemize}
    \item $R_1 = Z_0 \frac{N-1}{N+1}$
    \item $R_2 = \frac{2Z_0N}{N^2-1}$
\end{itemize}

\begin{mnemonicbox}
"TSR: T-attenuator Symmetry Requires equal R1 values"
\end{mnemonicbox}
\end{solutionbox}

\section*{પ્રશ્ન 5(અ) OR [3 ગુણ]}
\questionmarks{5(અ) OR}{3}{ગુણ}

\textbf{સિમેટ્રીકલ બ્રિજ T અને સિમેટ્રીકલ લેટિસ એટેન્યુએટરની આકૃતિ દોરો.}

\begin{solutionbox}
\textbf{સિમેટ્રિકલ બ્રિજ-T એટેન્યુએટર}:
\begin{center}
\begin{circuitikz}[american, scale=0.8]
    \draw (0,2) to[short, o-] (1,2) to[R, l=$R_1$] (3,2) to[R, l=$R_1$] (5,2) to[short, -o] (6,2);
    \draw (3,2) to[R, l=$R_2$] (3,0);
    \draw (0,0) to[short, o-] (6,0) to[short, -o] (6,0);
    % Bridge resistor across the two R1s
    \draw (1,2) to[short] (1,3) to[R, l=$R_3$] (5,3) to[short] (5,2);
\end{circuitikz}
\captionof{figure}{બ્રિજ-T એટેન્યુએટર}
\end{center}

\textbf{સિમેટ્રિકલ લેટિસ એટેન્યુએટર}:
\begin{center}
\begin{circuitikz}[american, scale=0.8]
    \draw (0,3) node[left]{A} to[short, o-] (1,3) -- (2,3);
    \draw (0,0) node[left]{C} to[short, o-] (1,0) -- (2,0);
    
    % Lattice cross lines
    \draw (2,3) to[R, l=$R_1$] (5,3);
    \draw (2,0) to[R, l=$R_1$] (5,0);
    \draw (2,3) to[R, l=$R_2$] (5,0);
    \draw (2,0) to[R, l=$R_2$] (5,3);
    
    \draw (5,3) -- (6,3) to[short, -o] (7,3) node[right]{B};
    \draw (5,0) -- (6,0) to[short, -o] (7,0) node[right]{D};
\end{circuitikz}
\captionof{figure}{લેટિસ એટેન્યુએટર}
\end{center}

\textbf{લક્ષણો}:
\begin{itemize}
    \item \textbf{બ્રિજ-T}: T અને $\pi$ એટેન્યુએટર્સની સુવિધાઓને જોડે છે.
    \item \textbf{લેટિસ}: ઉત્તમ ફેઝ/ફ્રિક્વન્સી રિસ્પોન્સ સાથે બેલેન્સ્ડ ગોઠવણ.
\end{itemize}

\begin{mnemonicbox}
"BL-BA: Bridge Ladder, Balanced Attenuators"
\end{mnemonicbox}
\end{solutionbox}

\section*{પ્રશ્ન 5(બ) OR [4 ગુણ]}
\questionmarks{5(બ) OR}{4}{ગુણ}

\textbf{ફિલ્ટરનુ ફ્રિકવન્સી આધારિત વર્ગીકરણ લખો અને દરેક ના ફ્રિકવન્સી રિસ્પોન્સ મા પાસ બેન્ડ અને સ્ટોપ બેન્ડ બતાવો.}

\begin{solutionbox}
\textbf{ફ્રિક્વન્સી આધારિત ફિલ્ટર્સનું વર્ગીકરણ}:

\begin{center}
\begin{tikzpicture}[gtu tree]
    \node [gtu root] {\gu{ફિલ્ટર્સ}}
        child { node [gtu child] {\gu{લો પાસ}} }
        child { node [gtu child] {\gu{હાઈ પાસ}} }
        child { node [gtu child] {\gu{બેન્ડ પાસ}} }
        child { node [gtu child] {\gu{બેન્ડ સ્ટોપ}} }
        child { node [gtu child] {\gu{ઓલ પાસ}} };
\end{tikzpicture}
\end{center}

\textbf{ફ્રિક્વન્સી રિસ્પોન્સ}:

\begin{center}
\begin{tikzpicture}[scale=0.5]
    % LPF
    \begin{scope}
    \draw[->] (0,0) -- (3,0) node[right] {f};
    \draw[->] (0,0) -- (0,2) node[above] {G};
    \draw[thick, blue] (0,1.5) -- (1.5,1.5) -- (1.5,0);
    \node at (1.5,-0.5) {LPF};
    \end{scope}
    
    % HPF
    \begin{scope}[xshift=4cm]
    \draw[->] (0,0) -- (3,0) node[right] {f};
    \draw[->] (0,0) -- (0,2) node[above] {G};
    \draw[thick, blue] (0,0) -- (1.5,0) -- (1.5,1.5) -- (3,1.5);
    \node at (1.5,-0.5) {HPF};
    \end{scope}
    
    % BPF
    \begin{scope}[xshift=8cm]
    \draw[->] (0,0) -- (3,0) node[right] {f};
    \draw[->] (0,0) -- (0,2) node[above] {G};
    \draw[thick, blue] (0,0) -- (1,0) -- (1,1.5) -- (2,1.5) -- (2,0) -- (3,0);
    \node at (1.5,-0.5) {BPF};
    \end{scope}
    
    % BSF
    \begin{scope}[xshift=12cm]
    \draw[->] (0,0) -- (3,0) node[right] {f};
    \draw[->] (0,0) -- (0,2) node[above] {G};
    \draw[thick, blue] (0,1.5) -- (1,1.5) -- (1,0) -- (2,0) -- (2,1.5) -- (3,1.5);
    \node at (1.5,-0.5) {BSF};
    \end{scope}
\end{tikzpicture}
\captionof{figure}{આદર્શ ફ્રિક્વન્સી રિસ્પોન્સ}
\end{center}

\begin{mnemonicbox}
"LHBBA: Low High Band-pass Band-stop All-pass"
\end{mnemonicbox}
\end{solutionbox}

\section*{પ્રશ્ન 5(ક) OR [7 ગુણ]}
\questionmarks{5(ક) OR}{7}{ગુણ}

\textbf{T-સેક્શન અને $\pi$-સેક્શન કોન્સ્ટન્ટ-K લો પાસ ફિલ્ટરની આકૃતિ દોરો અને કટ-ઓફ ફ્રિકવન્સીનુ સમીકરણ તારવો.}

\begin{solutionbox}
\textbf{T-સેક્શન કોન્સ્ટન્ટ-K લો પાસ ફિલ્ટર}:
\begin{center}
\begin{circuitikz}[american, scale=0.9]
    \draw (0,2) to[short, o-] (1,2) to[L, l=$L/2$] (3,2) coordinate(C) to[L, l=$L/2$] (5,2) to[short, -o] (6,2);
    \draw (C) to[C, l=$C$] (3,0) -- (3,0) coordinate(G);
    \draw (0,0) to[short, o-] (6,0) to[short, -o] (6,0);
\end{circuitikz}
\captionof{figure}{T-સેક્શન LPF}
\end{center}

\textbf{$\pi$-સેક્શન કોન્સ્ટન્ટ-K લો પાસ ફિલ્ટર}:
\begin{center}
\begin{circuitikz}[american, scale=0.9]
    \draw (0,2) to[short, o-] (1,2) to[short] (1,3) to[L, l=$L$] (5,3) to[short] (5,2) to[short, -o] (6,2);
    \draw (1,2) to[C, l=$C/2$] (1,0);
    \draw (5,2) to[C, l=$C/2$] (5,0);
    \draw (0,0) to[short, o-] (6,0) to[short, -o] (6,0);
\end{circuitikz}
\captionof{figure}{$\pi$-સેક્શન LPF}
\end{center}

\textbf{કટ-ઓફ ફ્રિક્વન્સીનું તારણ}:
\begin{enumerate}
    \item કોન્સ્ટન્ટ-K ફિલ્ટર માટે:
    \begin{itemize}
        \item $Z_1 \times Z_2 = R_0^2$ (કેરેક્ટરિસ્ટિક ઇમ્પિડન્સ વર્ગ)
        \item $Z_1 = j\omega L$ (સિરીઝ ઇમ્પિડન્સ)
        \item $Z_2 = \frac{1}{j\omega C}$ (શંટ ઇમ્પિડન્સ)
    \end{itemize}
    \item $R_0^2 = j\omega L \times \frac{1}{j\omega C} = \frac{L}{C} \implies R_0 = \sqrt{L/C}$
    \item પાસ બેન્ડ શરત: $-1 < \frac{Z_1}{4Z_2} < 0$
    \item કટ-ઓફ ફ્રિક્વન્સી પર: $\frac{\omega^2 LC}{4} = 1$
    \item $\omega_c = \frac{2}{\sqrt{LC}}$
    \item $f_c = \frac{1}{\pi\sqrt{LC}}$
\end{enumerate}

\textbf{અંતિમ સમીકરણ}: $f_c = \frac{1}{\pi\sqrt{LC}}$

\begin{mnemonicbox}
"KCLP: Konstant-k Cutoff in Low Pass depends on L and C product"
\end{mnemonicbox}
\end{solutionbox}

\end{document}

