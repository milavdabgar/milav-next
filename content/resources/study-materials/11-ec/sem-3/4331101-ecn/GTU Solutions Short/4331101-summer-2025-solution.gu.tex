\documentclass{article}

% content/resources/templates/preamble.tex
\usepackage[margin=0.6in]{geometry}
\author{Milav Dabgar}
\usepackage{amsmath,amssymb,amsthm}
\usepackage{booktabs}
\usepackage{multirow}
\usepackage{xcolor}
\usepackage{tcolorbox}
\tcbuselibrary{breakable,skins}
\usepackage[colorlinks=true,linkcolor=blue]{hyperref}
\usepackage{titlesec}
\usepackage{enumitem}
\usepackage{tikz}
\usepackage{pgfplots}
\usepackage{circuitikz}
\usepackage[version=4]{mhchem}
\usepackage{longtable}
\usepackage{array}
\usepackage{float}
\usepackage{caption}
\usepackage{listings}

\lstset{
  basicstyle=\small\ttfamily,
  breaklines=true,
  breakatwhitespace=false,
  postbreak=\mbox{\textcolor{red}{$\hookrightarrow$}\space},
  float=false,
  numbers=left,
  numberstyle=\tiny\color{gray},
  numbersep=10pt,
  xleftmargin=2em,
  keywordstyle=\color{blue},
  commentstyle=\color{green!60!black},
  stringstyle=\color{purple},
  backgroundcolor=\color{gray!5},
  showstringspaces=false,
  tabsize=2,
  captionpos=b,
  keepspaces=true,
  columns=flexible
}

\pgfplotsset{compat=1.18}
\usetikzlibrary{shapes,arrows,positioning,calc,patterns,decorations.pathmorphing,decorations.markings,arrows.meta}

% Color scheme
\definecolor{headcolor}{RGB}{0,102,204}
\definecolor{keycolor}{RGB}{220,20,60}
\definecolor{solutioncolor}{RGB}{34,139,34}
\definecolor{mnemoniccolor}{RGB}{148,0,211}
\definecolor{codecolor}{RGB}{0,0,100}

% Spacing
\setlength{\parskip}{3pt}
\setlist[itemize]{nosep}
\setlist[enumerate]{nosep}

% Title formatting
\titleformat{\section}{\Large\bfseries\color{headcolor}}{\thesection}{1em}{}
\titleformat{\subsection}{\large\bfseries\color{headcolor}}{\thesubsection}{1em}{}

% Pandoc tightlist compatibility
\providecommand{\tightlist}{%
  \setlength{\itemsep}{0pt}\setlength{\parskip}{0pt}}

% Pandoc longtable compatibility
\newcounter{none}
\def\thenone{}


% content/resources/templates/gujarati-boxes.tex
\usepackage{fontspec}
\usepackage{polyglossia}

% Set Gujarati as main language (document is primarily in Gujarati)
% Note: gloss-gujarati.ldf doesn't exist in polyglossia, but it will use hyphenation patterns
\setdefaultlanguage{gujarati}
\setotherlanguage{english}

% Configure Gujarati font properly
% Use Language=Default to prevent polyglossia from trying to add language-specific features
% that don't exist for Gujarati, which causes "empty feature" warnings
\newfontfamily\gujaratifont[Script=Gujarati,AutoFakeBold=2.5,AutoFakeSlant=0.3]{Noto Sans Gujarati}
\setmainfont[Script=Gujarati,AutoFakeBold=2.5,AutoFakeSlant=0.3]{Noto Sans Gujarati}
% Use Noto Sans Gujarati for monospace to support Gujarati in text
\setmonofont[Scale=0.9]{Noto Sans Gujarati}

% Configure English to use the same font
\newfontfamily\englishfont[Script=Gujarati,AutoFakeBold=2.5,AutoFakeSlant=0.3]{Noto Sans Gujarati}

% Translations for polyglossia
\gappto\captionsgujarati{
  \renewcommand{\tablename}{કોષ્ટક}
  \renewcommand{\figurename}{આકૃતિ}
}

% Helper for TikZ nodes to ensure Gujarati font
\newcommand{\gu}[1]{{\gujaratifont #1}}

% Custom environments
\newtcolorbox{solutionbox}{
    breakable,
    enhanced,
    colback=solutioncolor!5!white,
    colframe=solutioncolor!75!black,
    fonttitle=\bfseries,
    title=જવાબ
}

\newtcolorbox{solutionboxnobreak}{
 colback=solutioncolor!5!white,
 colframe=solutioncolor!75!black,
 fonttitle=\bfseries,
 title=જવાબ
}

\newtcolorbox{keyformula}{
 breakable,
 enhanced,
 colback=keycolor!5!white,
 colframe=keycolor!75!black,
 fonttitle=\bfseries,
 title=રાસાયણિક સમીકરણ/સૂત્ર
}

\newtcolorbox{mnemonicbox}{
 breakable,
 enhanced,
 colback=mnemoniccolor!5!white,
 colframe=mnemoniccolor!75!black,
 fonttitle=\bfseries,
 title=મેમરી ટ્રીક
}


% Custom commands for GTU solutions
% This file defines semantic commands for consistent formatting

% Question command with automatic formatting
\newcommand{\question}[2]{%
  \section*{Question #1}%
  \textbf{#2}%
}

% OR question variant
\newcommand{\questionor}[2]{%
  \section*{Question #1 OR}%
  \textbf{#2}%
}

% Proper table environment with caption
\newenvironment{answertable}[1]{%
  \begin{table}[htbp]
  \centering
  \caption{#1}
}{%
  \end{table}
}

% Proper figure environment for diagrams
\newenvironment{answerdiagram}[1]{%
  \begin{figure}[htbp]
  \centering
  \caption{#1}
}{%
  \end{figure}
}

% Semantic markup for key terms
\newcommand{\keyword}[1]{\textbf{#1}}
\newcommand{\code}[1]{\texttt{#1}}
\newcommand{\classname}[1]{\texttt{#1}}
\newcommand{\methodname}[1]{\texttt{#1}}

% Proper quotation marks
\newcommand{\mnemonic}[1]{``#1''}


\title{ઇલેક્ટ્રોનિક સર્કિટ અને નેટવર્ક્સ (4331101) - ઉનાળુ 2025 ઉકેલ}
\date{May 9, 2025}

\begin{document}
\maketitle

\questionmarks{1(a)}{3}{નીચેના શબ્દો વ્યાખ્યાયિત કરો. (i) એકટીવ એલિમેન્ટસ (ii) બાયલેટરલ એલિમેન્ટસ (iii) લિનિયર એલિમેંટ્સ}

\begin{solutionbox}
\begin{center}
\begin{tabulary}{\linewidth}{|L|L|}
\hline
\textbf{શબ્દ} & \textbf{વ્યાખ્યા} \\ \hline
\textbf{એકટીવ એલિમેન્ટસ} & એલેક્ટ્રોનિક ઘટકો જે સર્કિટમાં ઊર્જા અથવા પાવર આપી શકે છે (જેમ કે બેટરી, જનરેટર, ઓપ-એમ્પ) \\ \hline
\textbf{બાયલેટરલ એલિમેન્ટસ} & ઘટકો જે બંને દિશામાં સમાન લાક્ષણિકતાઓ સાથે કરંટને સરખી રીતે વહેવા દે છે (જેમ કે રેસિસ્ટર, કેપેસિટર, ઇન્ડક્ટર) \\ \hline
\textbf{લિનિયર એલિમેંટ્સ} & ઘટકો જેમનો કરંટ-વોલ્ટેજ સંબંધ સીધી લાઇનનું અનુસરણ કરે છે અને સુપરપોઝિશનના સિદ્ધાંતનું પાલન કરે છે (જેમ કે ઓહ્મના નિયમનું અનુસરણ કરતા રેસિસ્ટર) \\ \hline
\end{tabulary}
\end{center}

\begin{mnemonicbox}
\mnemonic{ABL: Active powers Batteries, Bilateral flows Both ways, Linear stays Lawful}
\end{mnemonicbox}
\end{solutionbox}

\questionmarks{1(b)}{4}{10$\mu$F, 20$\mu$F અને 30$\mu$F ના કેપેસિટર શ્રેણીમાં જોડાયેલા છે અને 200V DCનો પુરવઠો આપવામાં આવે છે. દરેક કેપેસિટરમાં વોલ્ટેજ શોધો.}

\begin{solutionbox}
શ્રેણીમાં જોડાયેલા કેપેસિટર માટે:
\begin{enumerate}
    \item સમતુલ્ય કેપેસિટન્સ શોધો: $1/C_{eq} = 1/C_1 + 1/C_2 + 1/C_3$
    \item વોલ્ટેજ વિભાજન: $V_C = (C_{eq}/C_x) \times V$
\end{enumerate}

\textbf{ગણતરી:}
$1/C_{eq} = 1/10 + 1/20 + 1/30 = 0.1 + 0.05 + 0.033 = 0.183$
$C_{eq} = 5.46 \mu$F

\begin{center}
\begin{tabulary}{\linewidth}{|L|L|L|L|}
\hline
\textbf{કેપેસિટર} & \textbf{સૂત્ર} & \textbf{ગણતરી} & \textbf{વોલ્ટેજ} \\ \hline
$C_1 = 10\mu$F & $V_1 = (C_{eq}/C_1) \times V$ & $(5.46/10) \times 200 = 109.2$V & 109.2V \\ \hline
$C_2 = 20\mu$F & $V_2 = (C_{eq}/C_2) \times V$ & $(5.46/20) \times 200 = 54.6$V & 54.6V \\ \hline
$C_3 = 30\mu$F & $V_3 = (C_{eq}/C_3) \times V$ & $(5.46/30) \times 200 = 36.4$V & 36.4V \\ \hline
\end{tabulary}
\end{center}

\begin{mnemonicbox}
\mnemonic{નાના કેપેસિટરમાં મોટો વોલ્ટેજ મળે}
\end{mnemonicbox}
\end{solutionbox}

\questionmarks{1(c)}{7}{ગ્રાફ થિયરી માટે નોડ પેર વોલ્ટેજ પદ્ધતિ સમજાવો.}

\begin{solutionbox}
નોડ પેર વોલ્ટેજ પદ્ધતિ એ ઇલેક્ટ્રિકલ નેટવર્ક્સનું વિશ્લેષણ કરવા માટેની પદ્ધતિસરની પદ્ધતિ છે.

\textbf{પ્રક્રિયા:}
\begin{enumerate}
    \item સંદર્ભ નોડ પસંદ કરો (ગ્રાઉન્ડ)
    \item નોડ વોલ્ટેજને ઓળખો (N નોડ માટે N-1 અજ્ઞાત)
    \item દરેક બિન-સંદર્ભ નોડ પર KCL લાગુ કરો
    \item નોડ વોલ્ટેજના સંદર્ભમાં શાખા કરંટ વ્યક્ત કરો
    \item નોડ વોલ્ટેજ માટે સમીકરણોનો ઉકેલ કરો
\end{enumerate}

\begin{center}
\begin{tikzpicture}[node distance=2cm, auto, >=latex]
    \node[gtu block] (A) {સંદર્ભ નોડ પસંદ કરો};
    \node[gtu block, right=of A] (B) {નોડ વોલ્ટેજ ઓળખો};
    \node[gtu block, right=of B] (C) {દરેક નોડ પર KCL લાગુ કરો};
    \node[gtu block, below=of C] (D) {શાખા કરંટ વ્યક્ત કરો};
    \node[gtu block, left=of D] (E) {સમીકરણો ઉકેલો};
    \node[gtu block, left=of E] (F) {શાખા કરંટ ગણો};

    \draw[gtu arrow] (A) -- (B);
    \draw[gtu arrow] (B) -- (C);
    \draw[gtu arrow] (C) -- (D);
    \draw[gtu arrow] (D) -- (E);
    \draw[gtu arrow] (E) -- (F);
\end{tikzpicture}
\captionof{figure}{નોડ પેર વોલ્ટેજ પદ્ધતિ}
\end{center}

\textbf{મુખ્ય ફાયદા:}
\begin{itemize}
    \item \textbf{ઓછા સમીકરણો}: n નોડ માટે ફક્ત (n-1) સમીકરણો
    \item \textbf{કમ્પ્યુટેશનલ કાર્યક્ષમતા}: સિસ્ટમની જટિલતા ઘટાડે છે
    \item \textbf{સીધા વોલ્ટેજ ઉકેલ}: સીધા નોડ વોલ્ટેજ પ્રદાન કરે છે
    \item \textbf{પદ્ધતિસરનો અભિગમ}: કોઈપણ નેટવર્ક ટોપોલોજી માટે કામ કરે છે
\end{itemize}

\begin{mnemonicbox}
\mnemonic{GARCS: Ground, Assign voltages, Relate with KCL, Calculate currents, Solve equations}
\end{mnemonicbox}
\end{solutionbox}

\questionmarks{1(c) OR}{7}{જરૂરી સમીકરણો સાથે વોલ્ટેજ વિભાજન પદ્ધતિ સમજાવો.}

\begin{solutionbox}
વોલ્ટેજ વિભાજન એ શ્રેણી ઘટકોમાં વોલ્ટેજ કેવી રીતે વિતરિત થાય છે તે ગણવાની એક પદ્ધતિ છે.

\textbf{સિદ્ધાંત:}
શ્રેણી સર્કિટમાં, વોલ્ટેજ ઘટક પ્રતિરોધ/ઇમ્પીડન્સના પ્રમાણમાં વિભાજિત થાય છે.

\textbf{સૂત્ર:}
કુલ પ્રતિરોધ $R_T$ સાથે શ્રેણી સર્કિટમાં એક પ્રતિરોધ $R_1$ માટે:
$V_1 = (R_1/R_T) \times V_S$

\begin{center}
\begin{circuitikz}[scale=1]
    \draw (0,0) to[V, l=$V_S$] (0,3) -- (2,3) to[R, l=$R_1$, v=$V_1$] (2,1.5);
    \draw (2,1.5) to[R, l=$R_2$] (2,0) -- (0,0);
\end{circuitikz}
\captionof{figure}{વોલ્ટેજ ડિવાઈડર સર્કિટ}
\end{center}

\textbf{ગાણિતિક સમજૂતી:}
\begin{itemize}
    \item પ્રતિરોધક માટે: $V_1 = (R_1/R_T) \times V_S$
    \item કેપેસિટર માટે: $V_1 = (1/C_1)/(1/C_T) \times V_S = (C_T/C_1) \times V_S$
    \item ઇન્ડક્ટર માટે: $V_1 = (L_1/L_T) \times V_S$
    \item જટિલ ઇમ્પીડન્સ માટે: $V_1 = (Z_1/Z_T) \times V_S$
\end{itemize}

\textbf{ઉદાહરણો:}
\begin{enumerate}
    \item 5V સ્ત્રોત સાથે 4k$\Omega$ ની શ્રેણીમાં 1k$\Omega$ પ્રતિરોધક પર વોલ્ટેજ = $(1/5) \times 5$V = 1V
    \item 10V સ્ત્રોત સાથે 40$\mu$F ની શ્રેણીમાં 10$\mu$F કેપેસિટર પર વોલ્ટેજ = $(1/10)/(1/8) \times 10$V = 8V
\end{enumerate}

\begin{mnemonicbox}
\mnemonic{જેટલો મોટો પ્રતિરોધ, તેટલો મોટો વોલ્ટેજ ડ્રોપ}
\end{mnemonicbox}
\end{solutionbox}

\questionmarks{2(a)}{3}{ટુ પોર્ટ નેટવર્કના ઓપન સર્કિટ ઈમ્પીડેન્સ પેરામીટર્સ લખો.}

\begin{solutionbox}
\textbf{ઓપન સર્કિટ ઈમ્પીડેન્સ પેરામીટર્સ:}

\begin{center}
\begin{tabulary}{\linewidth}{|L|L|L|}
\hline
\textbf{પેરામીટર} & \textbf{સમીકરણ} & \textbf{ભૌતિક અર્થ} \\ \hline
\textbf{$Z_{11}$} & $Z_{11} = V_1/I_1$ (જ્યારે $I_2=0$) & આઉટપુટ ઓપન-સર્કિટેડ હોય ત્યારે ઇનપુટ ઇમ્પીડન્સ \\ \hline
\textbf{$Z_{12}$} & $Z_{12} = V_1/I_2$ (જ્યારે $I_1=0$) & પોર્ટ 2 થી પોર્ટ 1 સુધી ટ્રાન્સફર ઇમ્પીડન્સ \\ \hline
\textbf{$Z_{21}$} & $Z_{21} = V_2/I_1$ (જ્યારે $I_2=0$) & પોર્ટ 1 થી પોર્ટ 2 સુધી ટ્રાન્સફર ઇમ્પીડન્સ \\ \hline
\textbf{$Z_{22}$} & $Z_{22} = V_2/I_2$ (જ્યારે $I_1=0$) & ઇનપુટ ઓપન-સર્કિટેડ હોય ત્યારે આઉટપુટ ઇમ્પીડન્સ \\ \hline
\end{tabulary}
\end{center}

\end{solutionbox}

\questionmarks{2(b)}{4}{ટી-ટાઈપ નેટવર્કમાંથી $\Pi$-પ્રકાર નેટવર્કમાં રૂપાંતરણ મેળવો.}

\begin{solutionbox}
\textbf{T થી $\Pi$ નેટવર્ક રૂપાંતરણ:}

\begin{center}
\begin{tabular}{cc}
\begin{circuitikz}[scale=0.8]
    \node at (2,2.5) {\textbf{T-નેટવર્ક}};
    \draw (0,2) to[R, l=$Z_1$, o-] (2,2) to[R, l=$Z_2$, -o] (4,2);
    \draw (2,2) to[R, l=$Z_3$] (2,0) to[short] (0,0);
    \draw (2,0) to[short] (4,0);
\end{circuitikz} &
\begin{circuitikz}[scale=0.8]
    \node at (2,2.5) {\textbf{$\Pi$-નેટવર્ક}};
    \draw (0,2) to[R, l=$Y_1$ (series), o-o] (4,2);
    \draw (0,2) to[R, l=$Y_3$] (0,0);
    \draw (4,2) to[R, l=$Y_2$] (4,0);
    \draw (0,0) to[short] (4,0);
\end{circuitikz}
\end{tabular}
\captionof{figure}{T અને $\Pi$ નેટવર્ક રૂપાંતરણ}
\end{center}

\textbf{રૂપાંતરણ સમીકરણો:}

\begin{center}
\begin{tabulary}{\linewidth}{|L|L|L|}
\hline
\textbf{$\Pi$-પેરામીટર} & \textbf{સૂત્ર} & \textbf{T-પેરામીટર્સ પર આધારિત} \\ \hline
$Y_1 = 1/Z_{\pi1}$ & $Y_1 = Z_2/(Z_1Z_2+Z_2Z_3+Z_3Z_1)$ & $Z_1$ ના સમકક્ષનું વ્યસ્ત \\ \hline
$Y_2 = 1/Z_{\pi2}$ & $Y_2 = Z_1/(Z_1Z_2+Z_2Z_3+Z_3Z_1)$ & $Z_2$ ના સમકક્ષનું વ્યસ્ત \\ \hline
$Y_3 = 1/Z_{\pi3}$ & $Y_3 = Z_3/(Z_1Z_2+Z_2Z_3+Z_3Z_1)$ & $Z_3$ ના સમકક્ષનું વ્યસ્ત \\ \hline
\end{tabulary}
\end{center}

\textbf{ડેરિવેશન સ્ટેપ્સ:}
\begin{enumerate}
    \item ડિટર્મિનન્ટ $\Delta = Z_1Z_2+Z_2Z_3+Z_3Z_1$ વ્યાખ્યાયિત કરો
    \item નેટવર્ક થિયરી વાપરીને $Y_1 = Z_2/\Delta$ તારવો
    \item તે જ રીતે, $Y_2 = Z_1/\Delta$ અને $Y_3 = Z_3/\Delta$
\end{enumerate}

\begin{mnemonicbox}
\mnemonic{Delta Divides: Y1 gets Z2, Y2 gets Z1, Y3 gets Z3}
\end{mnemonicbox}
\end{solutionbox}

\questionmarks{2(c)}{7}{ડેલ્ટામાં 1, 1 અને 1 ઓહ્મના ત્રણ રેસીસ્ટર જોડાયેલા છે. સમકક્ષ સ્ટાર નેટવર્ક શોધો.}

\begin{solutionbox}
\textbf{ડેલ્ટા થી સ્ટાર રૂપાંતરણ:}

\begin{center}
\begin{tabular}{cc}
\begin{circuitikz}[scale=0.8]
    \node at (1.5,2) {\textbf{ડેલ્ટા}};
    \draw (0,0) to[R, l=$R_3$] (1.5,2.6) to[R, l=$R_1$] (3,0) to[R, l=$R_2$] (0,0);
\end{circuitikz} &
\begin{circuitikz}[scale=0.8]
    \node at (1.5,2) {\textbf{સ્ટાર}};
    \draw (0,0) node[anchor=east]{B} to[R, l=$r_b$] (1.5,1) to[R, l=$r_a$] (1.5,2.6) node[anchor=south]{A};
    \draw (1.5,1) to[R, l=$r_c$] (3,0) node[anchor=west]{C};
\end{circuitikz}
\end{tabular}
\captionof{figure}{ડેલ્ટા થી સ્ટાર રૂપાંતરણ}
\end{center}

\textbf{રૂપાંતરણ સૂત્રો:}
\begin{itemize}
    \item $r_a = (R_1 \times R_3)/(R_1+R_2+R_3)$
    \item $r_b = (R_1 \times R_2)/(R_1+R_2+R_3)$
    \item $r_c = (R_2 \times R_3)/(R_1+R_2+R_3)$
\end{itemize}

\textbf{ગણતરી:}
આપેલું: $R_1 = R_2 = R_3 = 1\Omega$
પ્રતિરોધનો સરવાળો: $R_1+R_2+R_3 = 3\Omega$

\begin{center}
\begin{tabulary}{\linewidth}{|L|L|L|L|}
\hline
\textbf{સ્ટાર પ્રતિરોધક} & \textbf{સૂત્ર} & \textbf{ગણતરી} & \textbf{પરિણામ} \\ \hline
$r_a$ & $(R_1 \times R_3)/\Sigma R$ & $(1 \times 1)/3$ & $0.333\Omega$ \\ \hline
$r_b$ & $(R_1 \times R_2)/\Sigma R$ & $(1 \times 1)/3$ & $0.333\Omega$ \\ \hline
$r_c$ & $(R_2 \times R_3)/\Sigma R$ & $(1 \times 1)/3$ & $0.333\Omega$ \\ \hline
\end{tabulary}
\end{center}

\begin{mnemonicbox}
\mnemonic{પ્રોડક્ટ ઓવર સમ: દરેક સ્ટાર આર્મને પ્રોડક્ટ ઑફ ડેલ્ટા સાઈડ્સ ભાગ્યા સરવાળો મળે છે}
\end{mnemonicbox}
\end{solutionbox}

\questionmarks{2(a) OR}{3}{વ્યાખ્યાયિત કરો. (i) ટ્રાન્સફર ઇમ્પીડન્સ (ii) ઇમેજ ઇમ્પીડન્સ (iii) ડ્રાઇવિંગ પોઈન્ટ ઇમ્પીડન્સ}

\begin{solutionbox}
\begin{center}
\begin{tabulary}{\linewidth}{|L|L|}
\hline
\textbf{શબ્દ} & \textbf{વ્યાખ્યા} \\ \hline
\textbf{ટ્રાન્સફર ઇમ્પીડન્સ} & એક પોર્ટ પર આઉટપુટ વોલ્ટેજનો બીજા પોર્ટ પર ઈનપુટ કરંટના ગુણોત્તર જ્યારે અન્ય બધા પોર્ટ ઓપન-સર્કિટેડ હોય ($Z_{21} = V_2/I_1$ જ્યારે $I_2=0$) \\ \hline
\textbf{ઇમેજ ઇમ્પીડન્સ} & જ્યારે આઉટપુટ પોર્ટ તેના પોતાના ઇમેજ ઇમ્પીડન્સ સાથે ટર્મિનેટ કરવામાં આવે ત્યારે પોર્ટ પર ઇનપુટ ઇમ્પીડન્સ, જે તમામ પોઇન્ટ્સ પર સમાન ઇમ્પીડન્સ સાથે અનંત ચેઇન બનાવે છે \\ \hline
\textbf{ડ્રાઇવિંગ પોઈન્ટ ઇમ્પીડન્સ} & જ્યારે નિર્દિષ્ટ પોર્ટ અથવા ટર્મિનલ જોડીમાં જોતા હોઈએ ત્યારે દેખાતી ઇનપુટ ઇમ્પીડન્સ ($Z_{11} = V_1/I_1$ પોર્ટ 1 માટે) \\ \hline
\end{tabulary}
\end{center}

\begin{mnemonicbox}
\mnemonic{TID: Transfer relates ports, Image creates reflections, Driving point looks inward}
\end{mnemonicbox}
\end{solutionbox}

\questionmarks{2(b) OR}{4}{સ્ટાન્ડર્ડ 'T' નેટવર્ક માટે કેરેક્ટરીસ્ટીક ઇમ્પીડન્સ Z માટે સમીકરણ મેળવો.}

\begin{solutionbox}
\textbf{'T' નેટવર્કની કેરેક્ટરીસ્ટીક ઇમ્પીડન્સ:}

\begin{center}
\begin{circuitikz}[scale=1]
    \draw (0,2) to[R, l=$Z_1/2$, o-] (2,2) to[R, l=$Z_1/2$, -o] (4,2);
    \draw (2,2) to[R, l=$Z_2$] (2,0) node[ground]{};
    \draw (0,0) to[short, o-o] (4,0);
\end{circuitikz}
\captionof{figure}{સિમેટ્રિકલ T-નેટવર્ક}
\end{center}

\textbf{ડેરિવેશન:}
સિમેટ્રિકલ T-નેટવર્ક માટે સીરીઝ ઇમ્પીડન્સ $Z_1$ (દરેક બાજુ પર $Z_1/2$ તરીકે વિભાજિત) અને શંટ ઇમ્પીડન્સ $Z_2$ સાથે:
$Z_0 = \sqrt{Z_1Z_2 + Z_1^2/4}$

\textbf{સ્ટેપ્સ:}
\begin{enumerate}
    \item T-નેટવર્ક માટે ABCD પેરામીટર્સ:
    \begin{itemize}
        \item $A = 1 + Z_1/2Z_2$
        \item $B = Z_1 + Z_1^2/4Z_2$
        \item $C = 1/Z_2$
        \item $D = 1 + Z_1/2Z_2$
    \end{itemize}
    \item ટ્રાન્સમિશન લાઇન થિયરી માંથી, $Z_0 = \sqrt{B/C}$
    \item સબસ્ટિટ્યુટિંગ: $Z_0 = \sqrt{(Z_1 + Z_1^2/4Z_2)/(1/Z_2)}$
    \item સરળીકરણ: $Z_0 = \sqrt{Z_1Z_2 + Z_1^2/4}$
\end{enumerate}

\begin{mnemonicbox}
\mnemonic{Z-પ્રોડક્ટ પ્લસ ક્વાર્ટર-સ્ક્વેરનું વર્ગમૂળ}
\end{mnemonicbox}
\end{solutionbox}

\questionmarks{2(c) OR}{7}{6, 15 અને 10 ઓહ્મના ત્રણ રેસીસ્ટર સ્ટાર માં જોડાયેલા છે. સમકક્ષ ડેલ્ટા નેટવર્ક શોધો.}

\begin{solutionbox}
\textbf{સ્ટાર થી ડેલ્ટા રૂપાંતરણ:}

\begin{center}
\begin{tabular}{cc}
\begin{circuitikz}[scale=0.8]
    \node at (1.5,2) {\textbf{સ્ટાર}};
    \draw (0,0) node[anchor=east]{B} to[R, l=$r_b$, *-] (1.5,1) to[R, l=$r_a$, -*] (1.5,2.6) node[anchor=south]{A};
    \draw (1.5,1) to[R, l=$r_c$, -*] (3,0) node[anchor=west]{C};
\end{circuitikz} &
\begin{circuitikz}[scale=0.8]
    \node at (1.5,2) {\textbf{ડેલ્ટા}};
    \draw (0,0) to[R, l=$R_3$] (1.5,2.6) to[R, l=$R_1$] (3,0) to[R, l=$R_2$] (0,0);
\end{circuitikz}
\end{tabular}
\captionof{figure}{સ્ટાર થી ડેલ્ટા રૂપાંતરણ}
\end{center}

\textbf{રૂપાંતરણ સૂત્રો:}
\begin{itemize}
    \item $R_1 = (r_a r_b + r_b r_c + r_c r_a)/r_a$
    \item $R_2 = (r_a r_b + r_b r_c + r_c r_a)/r_b$
    \item $R_3 = (r_a r_b + r_b r_c + r_c r_a)/r_c$
\end{itemize}

\textbf{ગણતરી:}
આપેલું: $r_a = 6\Omega, r_b = 15\Omega, r_c = 10\Omega$
પ્રોડક્ટનો સરવાળો = $(6 \times 15) + (15 \times 10) + (10 \times 6) = 90 + 150 + 60 = 300$

\begin{center}
\begin{tabulary}{\linewidth}{|L|L|L|L|}
\hline
\textbf{ડેલ્ટા પ્રતિરોધક} & \textbf{સૂત્ર} & \textbf{ગણતરી} & \textbf{પરિણામ} \\ \hline
$R_1$ & Sum of Products/$r_a$ & $300/6$ & $50\Omega$ \\ \hline
$R_2$ & Sum of Products/$r_b$ & $300/15$ & $20\Omega$ \\ \hline
$R_3$ & Sum of Products/$r_c$ & $300/10$ & $30\Omega$ \\ \hline
\end{tabulary}
\end{center}

\end{solutionbox}

\questionmarks{3(a)}{3}{KVL નો ઉપયોગ કરીને લૂપ કરંટની ગણતરી કરવા માટે સર્કિટ (R1, R2 અને R3 dc સપ્લાય સાથે શ્રેણીમાં જોડાયેલા) નું વિશ્લેષણ કરો}

\begin{solutionbox}
\textbf{શ્રેણી સર્કિટ માટે KVL:}

\begin{center}
\begin{circuitikz}[scale=1]
    \draw (0,0) to[V, l=$V_S$] (0,3) to[R, l=$R_1$] (3,3) to[R, l=$R_2$] (6,3) to[R, l=$R_3$] (6,0) -- (0,0);
    \draw[->, thick, blue] (2, 1.5) arc (120:-120:0.5) node[right] {$I$};
\end{circuitikz}
\captionof{figure}{KVL માટે શ્રેણી સર્કિટ}
\end{center}

\textbf{KVL સમીકરણ:} $V_S - IR_1 - IR_2 - IR_3 = 0$

\textbf{લૂપ કરંટ:} $I = V_S/(R_1 + R_2 + R_3)$

\textbf{સ્ટેપ્સ:}
\begin{enumerate}
    \item લૂપમાં બધા ઘટકોને ઓળખો: $V_S, R_1, R_2, R_3$
    \item KVL લાગુ કરો: વોલ્ટેજ વૃદ્ધિનો સરવાળો = વોલ્ટેજ ડ્રોપનો સરવાળો
    \item $I$ માટે ઉકેલ: $I = V_S/R_{eq}$ જ્યાં $R_{eq} = R_1 + R_2 + R_3$
\end{enumerate}

\begin{mnemonicbox}
\mnemonic{KVL: કિરચોફનો વોલ્ટેજ લૂપ કુલ પ્રતિરોધની જરૂર પડે છે}
\end{mnemonicbox}
\end{solutionbox}

\questionmarks{3(b)}{4}{નોર્ટનનું થીયરમ લખો.}

\begin{solutionbox}
\textbf{નોર્ટનનું થીયરમ:}
વોલ્ટેજ સ્ત્રોત, કરંટ સ્ત્રોત અને પ્રતિરોધ વાળા કોઈપણ લિનિયર ઇલેક્ટ્રિકલ નેટવર્કને $I_N$ કરંટ સ્ત્રોત અને $R_N$ પ્રતિરોધ સમાંતર જોડાયેલા સમકક્ષ સર્કિટ દ્વારા બદલી શકાય છે.

\begin{center}
\begin{circuitikz}[scale=0.9]
    \node[gtu block] (A) {Linear Network};
    \draw (A.east) -- ++(1,0) node[right] {Load};
    
    \draw[->, very thick] (3,0) -- (4,0) node[midway, above] {Equivalent};
    
    \draw (5,0) to[I, l=$I_N$] (5,2) -- (7,2);
    \draw (5,0) -- (7,0);
    \draw (7,2) to[R, l=$R_N$] (7,0);
    \draw (7,2) -- (8,2) node[right, o-o] {Load};
    \draw (7,0) -- (8,0);
\end{circuitikz}
\captionof{figure}{નોર્ટન સમકક્ષ સર્કિટ}
\end{center}

\textbf{નોર્ટન સમકક્ષ કેવી રીતે શોધવું:}
\begin{enumerate}
    \item \textbf{નોર્ટન કરંટ ($I_N$)}: લોડ ટર્મિનલ્સ વચ્ચે શોર્ટ-સર્કિટ કરંટ
    \item \textbf{નોર્ટન રેસિસ્ટન્સ ($R_N$)}: બધા સ્ત્રોતોને તેમના આંતરિક પ્રતિરોધ સાથે બદલીને ટર્મિનલ્સથી જોતા ઈનપુટ રેસિસ્ટન્સ
\end{enumerate}

\begin{mnemonicbox}
\mnemonic{SCIP: Short-Circuit current In Parallel with equivalent resistance}
\end{mnemonicbox}
\end{solutionbox}

\questionmarks{3(c)}{7}{સુપરપોઝિશન પ્રમેયનો ઉપયોગ કરીને ckt ની કોઈપણ શાખામાં કરંટની ગણતરી કરવાનાં પગલાં સમજાવો}

\begin{solutionbox}
\textbf{સુપરપોઝિશન થીયરમનો ઉપયોગ:}

\textbf{સિદ્ધાંત:}
એક લિનિયર સર્કિટમાં બહુવિધ સ્ત્રોત સાથે, કોઈપણ તત્વમાં પ્રતિભાવ દરેક સ્ત્રોત એકલા કાર્ય કરતા હોય ત્યારે થતા પ્રતિભાવોના સરવાળા બરાબર હોય છે.

\textbf{સ્ટેપ્સ:}
\begin{enumerate}
    \item એક સમયે એક જ સ્ત્રોત ધ્યાનમાં લો
    \item અન્ય વોલ્ટેજ સ્ત્રોતને શોર્ટ સર્કિટ સાથે બદલો
    \item અન્ય કરંટ સ્ત્રોતને ઓપન સર્કિટ સાથે બદલો
    \item દરેક સ્ત્રોત માટે આંશિક કરંટની ગણતરી કરો
    \item તમામ આંશિક કરંટને (બીજગણિતીય રીતે) એકસાથે ઉમેરો
\end{enumerate}

\begin{center}
\begin{tikzpicture}[node distance=2cm, auto, >=latex]
    \node[gtu block] (A) {એક સ્ત્રોત પસંદ કરો};
    \node[gtu block, right=of A] (B) {અન્ય સ્ત્રોતોને બદલો};
    \node[gtu block, right=of B] (C) {આંશિક કરંટ ગણો};
    \node[gtu block, below=of C] (D) {પુનરાવર્તન કરો};
    \node[gtu block, left=of D] (E) {સરવાળો કરો};
    
    \draw[gtu arrow] (A) -- (B);
    \draw[gtu arrow] (B) -- (C);
    \draw[gtu arrow] (C) -- (D);
    \draw[gtu arrow] (D) -- (E);
\end{tikzpicture}
\captionof{figure}{સુપરપોઝિશન પ્રક્રિયા}
\end{center}

\textbf{ગાણિતિક અભિવ્યક્તિ:}
$I = I_1 + I_2 + I_3 + ... + I_n$
જ્યાં $I_1, I_2$, વગેરે વ્યક્તિગત સ્ત્રોતોના કારણે આંશિક કરંટ છે

\textbf{ઉદાહરણ ગણતરી:}
કરંટ યોગદાન સાથે શાખા માટે:
$I_1 = 2A$ (સ્ત્રોત 1 થી), $I_2 = -1A$ (સ્ત્રોત 2 થી), $I_3 = 0.5A$ (સ્ત્રોત 3 થી)
કુલ કરંટ = $2A + (-1A) + 0.5A = 1.5A$

\begin{mnemonicbox}
\mnemonic{OSACI: One Source Active, Calculate and Integrate}
\end{mnemonicbox}
\end{solutionbox}

\questionmarks{3(a) OR}{3}{KCL નો ઉપયોગ કરીને નોડ વોલ્ટેજની ગણતરી કરવા માટે સર્કિટ (R1, R2 અને R3 ડીસી સપ્લાય સાથે સમાંતર જોડાયેલ) નું વિશ્લેષણ કરો}

\begin{solutionbox}
\textbf{સમાંતર સર્કિટ માટે KCL:}

\begin{center}
\begin{circuitikz}[scale=1]
    \draw (0,0) to[V, l=$V_S$] (0,3) -- (6,3);
    \draw (0,0) -- (6,0);
    \draw (2,3) to[R, l=$R_1$, i=$I_1$] (2,0);
    \draw (4,3) to[R, l=$R_2$, i=$I_2$] (4,0);
    \draw (6,3) to[R, l=$R_3$, i=$I_3$] (6,0);
    \node at (6,3) [label=above:Node V]{};
\end{circuitikz}
\captionof{figure}{KCL માટે સમાંતર સર્કિટ}
\end{center}

\textbf{KCL સમીકરણ:} $I_1 + I_2 + I_3 = I_{total}$ (જો કરંટ સ્ત્રોત હોય)
નોડ વોલ્ટેજ $V = V_S$ (કારણ કે સમાંતર ઘટકોમાં સમાન વોલ્ટેજ હોય છે).

\textbf{સ્ટેપ્સ:}
\begin{enumerate}
    \item નોડ વોલ્ટેજ $V$ ને ઓળખો
    \item શાખા કરંટને વ્યક્ત કરો: $I_1 = V/R_1, I_2 = V/R_2, I_3 = V/R_3$
    \item KCL લાગુ કરો. (જો વોલ્ટેજ સ્ત્રોત સીધો જોડાયેલ હોય તો $V$ જાણીતું છે)
\end{enumerate}

\begin{mnemonicbox}
\mnemonic{KCL: કિરચોફનો કરંટ નિયમ સમાંતર વોલ્ટેજ સ્ત્રોત જેટલો જ બતાવે છે}
\end{mnemonicbox}
\end{solutionbox}

\questionmarks{3(b) OR}{4}{મહત્તમ પાવર ટ્રાન્સફર થીયરમ લખો.}

\begin{solutionbox}
\textbf{મહત્તમ પાવર ટ્રાન્સફર થીયરમ:}
આંતરિક પ્રતિરોધ ધરાવતા સ્ત્રોત માટે, જ્યારે લોડ પ્રતિરોધ સ્ત્રોતના આંતરિક પ્રતિરોધ બરાબર હોય ત્યારે લોડમાં મહત્તમ પાવર ટ્રાન્સફર થાય છે.

\begin{center}
\begin{circuitikz}[scale=1]
    \draw (0,0) to[V, l=$V_{th}$] (0,2) to[R, l=$R_{th}$] (3,2) to[R, l=$R_L$] (3,0) -- (0,0);
\end{circuitikz}
\captionof{figure}{મહત્તમ પાવર ટ્રાન્સફર સર્કિટ}
\end{center}

\textbf{ગાણિતિક અભિવ્યક્તિ:}
\begin{itemize}
    \item મહત્તમ પાવર ટ્રાન્સફર થાય ત્યારે $R_L = R_{source}$ (અથવા $R_{th}$)
    \item મહત્તમ પાવર: $P_{max} = V_{th}^2/(4 R_{th})$
\end{itemize}

\textbf{મુખ્ય મુદ્દાઓ:}
\begin{itemize}
    \item \textbf{કાર્યક્ષમતા}: મહત્તમ પાવર ટ્રાન્સફર પર માત્ર 50\%
    \item \textbf{AC સર્કિટ્સ}: લોડ ઇમ્પીડન્સ સ્ત્રોત ઇમ્પીડન્સનો કોમ્પ્લેક્સ કોન્જુગેટ હોવો જોઈએ ($Z_L = Z_S^*$)
    \item \textbf{ઉપયોગો}: સિગ્નલ ટ્રાન્સમિશન, ઓડિયો સિસ્ટમ્સ
\end{itemize}

\begin{mnemonicbox}
\mnemonic{MEET: Maximum Efficiency Equals when Thevenin-matched}
\end{mnemonicbox}
\end{solutionbox}

\questionmarks{3(c) OR}{7}{થેવેનિનના પ્રમેયનો ઉપયોગ કરીને ckt માં Vth, Rth અને લોડ કરંટની ગણતરી કરવાનાં પગલાં સમજાવો.}

\begin{solutionbox}
\textbf{થેવેનિનના થીયરમનો ઉપયોગ:}

\textbf{સિદ્ધાંત:}
વોલ્ટેજ અને કરંટ સ્ત્રોત ધરાવતા કોઈપણ લિનિયર ઇલેક્ટ્રિકલ નેટવર્કને એક સિંગલ વોલ્ટેજ સ્ત્રોત $V_{th}$ અને શ્રેણી પ્રતિરોધ $R_{th}$ વાળા સમકક્ષ સર્કિટ દ્વારા બદલી શકાય છે.

\textbf{સ્ટેપ્સ:}
\begin{enumerate}
    \item સર્કિટમાંથી લોડ પ્રતિરોધ દૂર કરો
    \item લોડ ટર્મિનલ્સ વચ્ચે ઓપન-સર્કિટ વોલ્ટેજ ($V_{th}$) ની ગણતરી કરો
    \item બધા સ્ત્રોતોને તેમના આંતરિક પ્રતિરોધ સાથે બદલો (વોલ્ટેજ સ્ત્રોત શોર્ટ, કરંટ સ્ત્રોત ઓપન)
    \item લોડ ટર્મિનલ્સથી જોતા સમકક્ષ પ્રતિરોધ ($R_{th}$) ની ગણતરી કરો
    \item $V_{th}$ અને $R_{th}$ સાથે થેવેનિન સમકક્ષ સર્કિટ દોરો
    \item લોડને ફરીથી જોડો અને લોડ કરંટની ગણતરી કરો: $I_L = V_{th}/(R_{th} + R_L)$
\end{enumerate}

\begin{center}
\begin{tikzpicture}[node distance=2cm, auto, >=latex, every node/.style={gtu block}]
    \node (A) {લોડ દૂર કરો};
    \node[right=of A] (B) {Vth શોધો};
    \node[right=of B] (C) {સ્ત્રોત બદલો};
    \node[below=of C] (D) {Rth ગણો};
    \node[left=of D] (E) {સમકક્ષ દોરો};
    \node[left=of E] (F) {IL ગણો};
    
    \draw[gtu arrow] (A) -- (B);
    \draw[gtu arrow] (B) -- (C);
    \draw[gtu arrow] (C) -- (D);
    \draw[gtu arrow] (D) -- (E);
    \draw[gtu arrow] (E) -- (F);
\end{tikzpicture}
\captionof{figure}{થેવેનિન પ્રક્રિયા}
\end{center}

\textbf{ઉદાહરણ ગણતરી:}
\begin{itemize}
    \item જો $V_{th} = 12V$, $R_{th} = 3\Omega$, $R_L = 6\Omega$
    \item પછી $I_L = 12V/(3\Omega + 6\Omega) = 1.33A$
\end{itemize}

\end{solutionbox}

\questionmarks{4(a)}{3}{કપલ્ડ સર્કિટ દોરો. L1, L2 અને M માર્ક કરો.}

\begin{solutionbox}
\textbf{કપલ્ડ સર્કિટ ડાયાગ્રામ:}

\begin{center}
\begin{circuitikz}[scale=1]
    \draw (0,0) to[L, l=$L_1$, name=L1] (0,2);
    \draw (2,2) to[L, l=$L_2$, name=L2] (2,0);
    \draw (0,0) -- (2,0);
    \draw (0,2) -- (-1,2) node[left] {Input};
    \draw (2,2) -- (3,2) node[right] {Output};
    
    % Mutual inductance symbol
    \draw [densely dashed, <->] (0.4,1) -- (1.6,1) node[midway, above] {$M$};
    
    % Dot convention
    \node [circle, fill, inner sep=1pt] at (-0.2, 1.8) {};
    \node [circle, fill, inner sep=1pt] at (1.8, 1.8) {};
\end{circuitikz}
\captionof{figure}{મેગ્નેટિકલી કપલ્ડ સર્કિટ}
\end{center}

\textbf{ઘટકો:}
\begin{itemize}
    \item $L_1$: કોઇલ 1 નું સેલ્ફ-ઇન્ડક્ટન્સ
    \item $L_2$: કોઇલ 2 નું સેલ્ફ-ઇન્ડક્ટન્સ
    \item $M$: કોઇલ્સ વચ્ચે મ્યુચ્યુઅલ ઇન્ડક્ટન્સ
    \item ડોટ્સ પ્રેરિત વોલ્ટેજની પોલેરિટી સૂચવે છે
\end{itemize}

\begin{mnemonicbox}
\mnemonic{M-Link: Mutual inductance links two coils together}
\end{mnemonicbox}
\end{solutionbox}

\questionmarks{4(b)}{4}{કોએફીશિયન્ટ ઓફ કપલિંગ વ્યાખ્યાયિત કરો. K, M, L1, L2 વચ્ચેનો સંબંધ જણાવો.}

\begin{solutionbox}
\textbf{કોએફિશિયન્ટ ઓફ કપલિંગ ($K$):}
એક કોઇલ દ્વારા ઉત્પાદિત મેગ્નેટિક ફ્લક્સનો અંશ જે બીજી કોઇલ સાથે જોડાય છે. તે બે કોઇલ વચ્ચેના મેગ્નેટિક કપલિંગની હદ દર્શાવે છે.

\textbf{સંબંધ:}
$M = K\sqrt{L_1 L_2}$
અથવા
$K = M/\sqrt{L_1 L_2}$

\textbf{મુખ્ય ગુણધર્મો:}
\begin{itemize}
    \item રેન્જ: $0 \le K \le 1$
    \item $K=1$: પરફેક્ટલી કપલ્ડ (માળખું)
    \item $K=0$: નો કપલિંગ (મેગ્નેટિકલી આઇસોલેટેડ)
    \item $K<0.5$: લૂઝલી કપલ્ડ
\end{itemize}

\begin{mnemonicbox}
\mnemonic{K defines the Link: Ratio of Mutual to Geometric Mean of Selfs}
\end{mnemonicbox}
\end{solutionbox}

\questionmarks{4(c)}{7}{શ્રેણી રેઝોનન્સ સર્કિટની રેઝોનન્ટ ફ્રીક્વન્સી માટે સમીકરણ મેળવો. એક શ્રેણી RLC સર્કિટમાં R=10 ohm, L=0.1H અને C=10$\mu$F છે. રેઝોનન્ટ ફ્રીક્વન્સીની ગણતરી કરો.}

\begin{solutionbox}
\textbf{શ્રેણી રેઝોનન્સ ડેરિવેશન:}

\begin{center}
\begin{circuitikz}[scale=1]
    \draw (0,0) to[V, l=$V \sin \omega t$] (0,2) to[R, l=$R$] (2,2) to[L, l=$L$] (4,2) to[C, l=$C$] (6,2) -- (6,0) -- (0,0);
\end{circuitikz}
\captionof{figure}{શ્રેણી RLC સર્કિટ}
\end{center}

\textbf{રેઝોનન્સ માટે શરત:}
જ્યારે ઇન્ડક્ટિવ રિએક્ટન્સ કેપેસિટિવ રિએક્ટન્સ બરાબર થાય છે ($X_L = X_C$) ત્યારે રેઝોનન્સ થાય છે, જેનાથી સર્કિટ માત્ર રેઝિસ્ટિવ બને છે.

\textbf{ડેરિવેશન:}
\begin{enumerate}
    \item $X_L = 2\pi f L$ અને $X_C = 1/(2\pi f C)$
    \item રેઝોનન્સ પર ($f_r$): $X_L = X_C$
    \item $2\pi f_r L = 1/(2\pi f_r C)$
    \item $(2\pi f_r)^2 = 1/(LC)$
    \item $f_r = 1/(2\pi\sqrt{LC})$
\end{enumerate}

\textbf{ગણતરી:}
આપેલું: $R=10\Omega, L=0.1$H, $C=10\mu$F ($10 \times 10^{-6}$F)

$f_r = 1/(2\pi\sqrt{0.1 \times 10 \times 10^{-6}})$
$= 1/(2\pi\sqrt{10^{-6}})$
$= 1/(2\pi \times 10^{-3})$
$= 1000/2\pi$
$= 159.15$ Hz

\begin{mnemonicbox}
\mnemonic{Formula is inverse of 2-pi-root-LC}
\end{mnemonicbox}
\end{solutionbox}

\questionmarks{4(a) OR}{3}{ક્વોલિટી ફેક્ટર વ્યાખ્યાયિત કરો.}

\begin{solutionbox}
\textbf{ક્વોલિટી ફેક્ટર (Q-factor):}
તે મેરિટનો આંકડો છે જે રેઝોનન્સની તીક્ષ્ણતા અથવા રેઝોનન્ટ સર્કિટની સિલેક્ટિવિટીના માપદંડ તરીકે કાર્ય કરે છે.

\textbf{વ્યાખ્યાઓ:}
\begin{enumerate}
    \item રેઝોનન્સ પર L અથવા C માં વોલ્ટેજનો એપ્લાઈડ વોલ્ટેજ સાથેનો ગુણોત્તર (વોલ્ટેજ મેગ્નિફિકેશન).
    \item રિએક્ટિવ પાવરનો એક્ટિવ પાવર સાથેનો ગુણોત્તર ($Q = \text{રિએક્ટિવ પાવર} / \text{એક્ટિવ પાવર}$).
    \item સંગ્રહિત ઊર્જાનો એક ચક્ર દીઠ વેડફાતી ઊર્જા સાથેનો ગુણોત્તર ($Q = 2\pi \times (\text{મહત્તમ સંગ્રહિત ઊર્જા} / \text{ચક્ર દીઠ વેડફાતી ઊર્જા})$).
\end{enumerate}

\textbf{સૂત્ર:}
શ્રેણી RLC માટે: $Q = (1/R)\sqrt{L/C} = \omega_r L / R$

\begin{mnemonicbox}
\mnemonic{Quality magnifies Voltage and selects Frequencies}
\end{mnemonicbox}
\end{solutionbox}

\questionmarks{4(b) OR}{4}{કેપેસિટરનો ક્વોલિટી ફેક્ટર સમજાવો.}

\begin{solutionbox}
\textbf{કેપેસિટરનો ક્વોલિટી ફેક્ટર:}
તે કેપેસિટરની કાર્યક્ષમતા દર્શાવે છે, તેની સંગ્રહિત ઊર્જાની તેની ઊર્જાના નુકસાન સાથે સરખામણી કરે છે. વાસ્તવિક કેપેસિટરમાં થોડો લીકેજ રેઝિસ્ટન્સ અથવા સમકક્ષ શ્રેણી રેઝિસ્ટન્સ (ESR) હોય છે.

\begin{center}
\begin{circuitikz}[scale=1]
    \draw (0,0) to[C, l=$C$] (0,2) to[R, l=$R_{ESR}$] (2,2) -- (2,0) -- (0,0);
    \node at (1,-0.5) {વાસ્તવિક કેપેસિટર મોડેલ};
\end{circuitikz}
\captionof{figure}{વાસ્તવિક કેપેસિટર મોડેલ}
\end{center}

\textbf{સમીકરણ:}
$Q_C = X_C / R_{ESR} = 1 / (\omega C R_{ESR})$

\textbf{મહત્વ:}
\begin{itemize}
    \item ઉચ્ચ Q એટલે ઓછું નુકસાન (આદર્શ કેપેસિટરની નજીક).
    \item RF સર્કિટ્સમાં શાર્પ ટ્યુનિંગ માટે મહત્વપૂર્ણ.
    \item ડિસીપેશન ફેક્ટર ($D$) એ Q નું વ્યસ્ત છે ($D = 1/Q$).
\end{itemize}

\begin{mnemonicbox}
\mnemonic{Q is Reactance over Resistance implies Low Loss}
\end{mnemonicbox}
\end{solutionbox}

\questionmarks{4(c) OR}{7}{સમાંતર રેઝોનન્સ જરૂરી આકૃતિઓ અને વિશ્લેષણ સાથે સમજાવો.}

\begin{solutionbox}
\textbf{સમાંતર રેઝોનન્સ (ટાંક સર્કિટ):}

\begin{center}
\begin{circuitikz}[scale=1]
    \draw (0,0) to[I, l=$I$] (0,2) -- (2,2);
    \draw (2,2) to[R, l=$R$] (2,0) -- (0,0);
    \draw (2,2) -- (4,2) to[L, l=$L$] (4,0) -- (2,0);
    \draw (4,2) -- (6,2) to[C, l=$C$] (6,0) -- (4,0);
\end{circuitikz}
\captionof{figure}{આદર્શ સમાંતર RLC સર્કિટ}
\end{center}

\textbf{વિશ્લેષણ:}
\begin{enumerate}
    \item એડમિટન્સ $Y = 1/R + j(\omega C - 1/\omega L)$
    \item જ્યારે એડમિટન્સનો કાલ્પનિક ભાગ શૂન્ય હોય ત્યારે રેઝોનન્સ થાય છે (સસેપ્ટન્સ $B=0$).
    \item શરત: $\omega C - 1/\omega L = 0 \implies \omega C = 1/\omega L$
    \item રેઝોનન્ટ ફ્રીક્વન્સી: $f_r = 1/(2\pi\sqrt{LC})$ (આદર્શ કિસ્સા માટે શ્રેણી જેવું જ)
\end{enumerate}

\textbf{રેઝોનન્સ પર લાક્ષણિકતાઓ:}
\begin{itemize}
    \item **ઇમ્પીડન્સ**: મહત્તમ ($Z = R$ માત્ર રેઝિસ્ટિવ). પ્રાયોગિક ટાંકામાં (L શ્રેણી આંતરિક r સાથે), $Z_{dynamic} = L/(Cr)$.
    \item **કરંટ**: સ્ત્રોતમાંથી ન્યૂનતમ (વોલ્ટેજ સ્ત્રોત પર), પરંતુ L અને C વચ્ચેનો ફરતો કરંટ મેગ્નિફાઇ થાય છે (કરંટ મેગ્નિફિકેશન).
    \item **પાવર ફેક્ટર**: એકમ (Unity).
\end{itemize}

\end{solutionbox}

\questionmarks{5(a)}{3}{વિવિધ પ્રકારના એટેન્યુએટરનું વર્ગીકરણ કરો.}

\begin{solutionbox}
\textbf{એટેન્યુએટરના પ્રકારો:}

\begin{center}
\begin{tabulary}{\linewidth}{|L|L|L|}
\hline
\textbf{પ્રકાર} & \textbf{સંરચના} & \textbf{લાક્ષણિકતાઓ} \\ \hline
\textbf{T-પ્રકાર} & શ્રેણી-શંટ-શ્રેણી & સિમેટ્રિક, મેચિંગ માટે સારું, વ્યાપકપણે વપરાતું \\ \hline
\textbf{$\Pi$-પ્રકાર} & શંટ-શ્રેણી-શંટ & સિમેટ્રિક, T-પ્રકારનો વિકલ્પ \\ \hline
\textbf{લેટિસ} & બેલેન્સ્ડ બ્રિજ & સિમેટ્રિકલ, બેલેન્સ્ડ લાઇન્સમાં વપરાય છે \\ \hline
\textbf{L-પ્રકાર} & શ્રેણી-શંટ & એસિમેટ્રિક, સરળ ડિઝાઈન \\ \hline
\textbf{બ્રિજ્ડ-T} & બ્રિજ્ડ શંટ સાથે T & સારો ફ્રિક્વન્સી રિસ્પોન્સ, જટિલ \\ \hline
\textbf{O-પ્રકાર} & શ્રેણી-શંટ-શ્રેણી-શંટ & સુધારેલા રિજેક્શન લક્ષણો \\ \hline
\end{tabulary}
\end{center}

\begin{mnemonicbox}
\mnemonic{TL$\Pi$BO: Top attenuators Let $\Pi$ signals Balance Output}
\end{mnemonicbox}
\end{solutionbox}

\questionmarks{5(b)}{4}{ડેસીબલ અને નેપર વચ્ચેનો સંબંધ મેળવો}

\begin{solutionbox}
\textbf{ડેસીબલ થી નેપર રૂપાંતરણ:}

\textbf{વ્યાખ્યાઓ:}
\begin{itemize}
    \item \textbf{ડેસીબલ (dB)}: 10 ના બેઝ (common logarithm) નો ઉપયોગ કરીને પાવર રેશિયો લોગરીધમ
    \item \textbf{નેપર (Np)}: e ના બેઝ (natural logarithm) નો ઉપયોગ કરીને વોલ્ટેજ/કરંટ રેશિયો લોગરીધમ
\end{itemize}

\textbf{ડેરિવેશન:}
\begin{enumerate}
    \item dB માં પાવર રેશિયો: Loss(dB) = $10 \log_{10}(P_1/P_2)$
    \item dB માં વોલ્ટેજ રેશિયો: Loss(dB) = $20 \log_{10}(V_1/V_2)$
    \item Nepers માં વોલ્ટેજ રેશિયો: Loss(Np) = $\ln(V_1/V_2)$
    \item લોગરીધમ બેઝ વચ્ચે રૂપાંતરણ: $\log_{10}(x) = \ln(x)/\ln(10)$
    \item સબસ્ટિટ્યુશન: Loss(dB) = $20 \ln(V_1/V_2)/\ln(10) = 20 \text{Loss(Np)}/\ln(10)$
\end{enumerate}

\textbf{અંતિમ સંબંધ:}
\begin{itemize}
    \item $1 \text{ Neper} = (\ln(10)/20) \times 10 \text{ dB} \approx 20/2.303 \approx 8.686 \text{ dB}$
    \item $1 \text{ dB} = 0.115 \text{ Neper}$
\end{itemize}

\begin{center}
\begin{tabulary}{\linewidth}{|L|L|L|}
\hline
\textbf{રૂપાંતરણ} & \textbf{સૂત્ર} & \textbf{મૂલ્ય} \\ \hline
Neper to dB & 1 Np = $(20/\ln10)$ dB & 1 Np = 8.686 dB \\ \hline
dB to Neper & 1 dB = $(\ln10/20)$ Np & 1 dB = 0.115 Np \\ \hline
\end{tabulary}
\end{center}

\begin{mnemonicbox}
\mnemonic{8.686: Eight Point Six Nepers Buy Ten decibels}
\end{mnemonicbox}
\end{solutionbox}

\questionmarks{5(c)}{7}{600 ઓહ્મ કેરેક્ટરીસ્ટીક ઇમ્પીડન્સ ધરાવતું અને 20 dB એટેન્યુએશન પૂરું પાડતું T પ્રકારનું એટેન્યુએટર ડિઝાઈન કરો.}

\begin{solutionbox}
\textbf{T-પ્રકાર એટેન્યુએટર ડિઝાઈન:}

\begin{center}
\begin{circuitikz}[scale=1]
    \draw (0,2) to[R, l=$R_1/2$] (2,2) to[R, l=$R_1/2$] (4,2);
    \draw (2,2) to[R, l=$R_2$] (2,0);
    \draw (0,0) -- (4,0);
    \draw (0,2) -- (-0.5,2) node[left] {Input};
    \draw (4,2) -- (4.5,2) node[right] {Output};
    \node at (2, -0.5) {$R_1$ કુલ શ્રેણી, $R_1/2$ માં વિભાજિત};
\end{circuitikz}
\captionof{figure}{T-પ્રકાર એટેન્યુએટર}
\end{center}

\textbf{ડિઝાઈન સ્ટેપ્સ:}
\begin{enumerate}
    \item dB માંથી એટેન્યુએશન રેશિયો $N$ ગણો:
    $N = 10^{(dB/20)}$
    \item સૂત્રોનો ઉપયોગ કરીને $R_1$ (શ્રેણી) અને $R_2$ (શંટ) ગણો:
    $R_1 = R_0 \times [(N^2 - 1)/(N^2 + 1)]$
    $R_2 = R_0 \times [2N/(N^2 - 1)]$
\end{enumerate}

\textbf{ગણતરી:}
આપેલું: Attenuation = 20 dB, $Z_0 = 600 \Omega$

\begin{center}
\begin{tabulary}{\linewidth}{|L|L|L|L|}
\hline
\textbf{પેરામીટર} & \textbf{સૂત્ર} & \textbf{ગણતરી} & \textbf{પરિણામ} \\ \hline
$N$ & $10^{(dB/20)}$ & $10^{(20/20)}$ & 10 \\ \hline
$R_1$ (કુલ) & $R_0[(N^2 - 1)/(N^2 + 1)]$ & $600[(99)/(101)]$ & $588.1 \Omega$ \\ \hline
$R_1/2$ (દરેક આર્મ) & $R_1/2$ & $588.1/2$ & $294.05 \Omega$ \\ \hline
$R_2$ (શંટ) & $R_0[2N/(N^2 - 1)]$ & $600[20/99]$ & $121.2 \Omega$ \\ \hline
\end{tabulary}
\end{center}

\textbf{અંતિમ મૂલ્યો:}
દરેક શ્રેણી આર્મ = 294.05$\Omega$, શંટ આર્મ = 121.2$\Omega$.

\begin{mnemonicbox}
\mnemonic{N-squared minus ONE over N-squared plus ONE for series resistance}
\end{mnemonicbox}
\end{solutionbox}

\questionmarks{5(a) OR}{3}{કોન્સ્ટન્ટ K લો પાસ ફિલ્ટરની મર્યાદાઓ જણાવો}

\begin{solutionbox}
\textbf{કોન્સ્ટન્ટ-K લો પાસ ફિલ્ટર્સની મર્યાદાઓ:}

\begin{center}
\begin{tabulary}{\linewidth}{|L|L|}
\hline
\textbf{મર્યાદા} & \textbf{વર્ણન} \\ \hline
\textbf{નબળું કટઓફ ટ્રાન્ઝિશન} & શાર્પ કટઓફને બદલે પાસ બેન્ડથી સ્ટોપ બેન્ડમાં ધીમે ધીમે ફેરફાર \\ \hline
\textbf{અસમાન ઇમ્પીડન્સ} & ઇમ્પીડન્સ ફ્રિક્વન્સી સાથે બદલાય છે, મેચિંગ સમસ્યાઓ પેદા કરે છે ($Z_0$ અચળ નથી) \\ \hline
\textbf{એટેન્યુએશન રિપલ} & પાસ બેન્ડ અને સ્ટોપ બેન્ડ બંનેમાં અસમાન એટેન્યુએશન \\ \hline
\textbf{ફેઝ ડિસ્ટોર્શન} & નોન-લિનિયર ફેઝ રિસ્પોન્સ સિગ્નલ ડિસ્ટોર્શનનું કારણ બને છે \\ \hline
\textbf{ફિક્સ્ડ ટર્મિનેશન} & ચોક્કસ લોડ ઇમ્પીડન્સ ($R_0$) માટે રચાયેલ; અન્ય લોડ સાથે કામગીરી બગડે છે \\ \hline
\textbf{મર્યાદિત સિલેક્ટિવિટી} & આધુનિક ફિલ્ટર ડિઝાઈનની તુલનામાં નબળી પસંદગી (m-ડિરાઇવ્ડ ફિલ્ટર્સ વધુ સારા છે) \\ \hline
\end{tabulary}
\end{center}

\begin{mnemonicbox}
\mnemonic{PUAPFL: Poor transition, Uneven impedance, Attenuation ripple, Phase distortion, Fixed termination, Limited selectivity}
\end{mnemonicbox}
\end{solutionbox}

\questionmarks{5(b) OR}{4}{દરેક માટે ફ્રિક્વન્સી રિસ્પોન્સ કર્વ દર્શાવી ફિલ્ટર્સનું વર્ગીકરણ આપો}

\begin{solutionbox}
\textbf{ફિલ્ટર્સનું વર્ગીકરણ:}

\begin{center}
\begin{tabulary}{\linewidth}{|L|C|L|}
\hline
\textbf{ફિલ્ટર પ્રકાર} & \textbf{રિસ્પોન્સ કર્વ} & \textbf{લાક્ષણિકતાઓ} \\ \hline
\textbf{Low Pass} & 
\begin{tikzpicture}[scale=0.5]
    \draw[->] (0,0) -- (3,0) node[right] {$f$};
    \draw[->] (0,0) -- (0,2) node[above] {$V_{out}$};
    \draw[thick] (0,1.5) -- (1,1.5) .. controls (1.5,1.5) and (1.5,0) .. (2.5,0);
    \draw[dashed] (1.25,0) -- (1.25,1.5);
    \node at (1.25,-0.5) {$f_c$};
\end{tikzpicture}
& કટઓફ $f_c$ થી નીચેની ફ્રિક્વન્સી પસાર કરે છે, ઉચ્ચને બ્લોક કરે છે \\ \hline
\textbf{High Pass} & 
\begin{tikzpicture}[scale=0.5]
    \draw[->] (0,0) -- (3,0) node[right] {$f$};
    \draw[->] (0,0) -- (0,2) node[above] {$V_{out}$};
    \draw[thick] (0,0) .. controls (1,0) and (1,1.5) .. (1.5,1.5) -- (2.5,1.5);
    \draw[dashed] (1.25,0) -- (1.25,1.5);
    \node at (1.25,-0.5) {$f_c$};
\end{tikzpicture}
& કટઓફ $f_c$ થી નીચેની ફ્રિક્વન્સી બ્લોક કરે છે, ઉચ્ચને પસાર કરે છે \\ \hline
\textbf{Band Pass} & 
\begin{tikzpicture}[scale=0.5]
    \draw[->] (0,0) -- (3,0) node[right] {$f$};
    \draw[->] (0,0) -- (0,2) node[above] {$V_{out}$};
    \draw[thick] (0,0) .. controls (0.5,0) and (0.5,1.5) .. (1,1.5) -- (1.5,1.5) .. controls (2,1.5) and (2,0) .. (2.5,0);
    \node at (0.8,-0.5) {$f_1$}; \node at (1.7,-0.5) {$f_2$};
\end{tikzpicture}
& $f_1$ અને $f_2$ વચ્ચેની ફ્રિક્વન્સી પસાર કરે છે \\ \hline
\textbf{Band Stop} & 
\begin{tikzpicture}[scale=0.5]
    \draw[->] (0,0) -- (3,0) node[right] {$f$};
    \draw[->] (0,0) -- (0,2) node[above] {$V_{out}$};
    \draw[thick] (0,1.5) .. controls (0.5,1.5) and (0.5,0) .. (1,0) -- (1.5,0) .. controls (2,0) and (2,1.5) .. (2.5,1.5);
    \node at (0.8,-0.5) {$f_1$}; \node at (1.7,-0.5) {$f_2$};
\end{tikzpicture}
& $f_1$ અને $f_2$ વચ્ચેની ફ્રિક્વન્સી બ્લોક કરે છે \\ \hline
\end{tabulary}
\end{center}

\begin{mnemonicbox}
\mnemonic{LHBS: Low lets low tones, High lets high tones, Band-pass selects middle, Band-Stop rejects middle}
\end{mnemonicbox}
\end{solutionbox}

\questionmarks{5(c) OR}{7}{કોન્સ્ટન્ટ K લો પાસ ફિલ્ટરની ડિઝાઈન માટેનું સમીકરણ મેળવો.}

\begin{solutionbox}
\textbf{કોન્સ્ટન્ટ-K લો પાસ ફિલ્ટર ડિઝાઈન:}

\begin{center}
\begin{tabular}{cc}
\begin{circuitikz}[scale=0.8]
    \node at (2,2.5) {T-સેક્શન};
    \draw (0,2) to[L, l=$L/2$] (2,2) to[L, l=$L/2$] (4,2);
    \draw (2,2) to[C, l=$C$] (2,0);
    \draw (0,0) -- (4,0);
\end{circuitikz} &
\begin{circuitikz}[scale=0.8]
    \node at (2,2.5) {$\pi$-સેક્શન};
    \draw (0,2) to[L, l=$L$] (4,2);
    \draw (0,2) to[C, l=$C/2$] (0,0);
    \draw (4,2) to[C, l=$C/2$] (4,0);
    \draw (0,0) -- (4,0);
\end{circuitikz}
\end{tabular}
\captionof{figure}{કોન્સ્ટન્ટ-K લો પાસ ફિલ્ટર સેક્શન્સ}
\end{center}

\textbf{ડિઝાઈન થિયરી:}
કોન્સ્ટન્ટ-K ફિલ્ટરમાં તમામ ફ્રિક્વન્સી પર ઇમ્પીડન્સ પ્રોડક્ટ $Z_1Z_2 = R_0^2$ (અચળ) હોય છે.

\textbf{ડેરિવેશન સ્ટેપ્સ:}
\begin{enumerate}
    \item T-સેક્શન લો-પાસ ફિલ્ટર માટે:
       શ્રેણી ઇમ્પીડન્સ $Z_1 = j\omega L$, શંટ ઇમ્પીડન્સ $Z_2 = 1/j\omega C$
    \item પ્રોડક્ટ $Z_1Z_2 = L/C = R_0^2$ (અચળ $k^2$)
    \item શૂન્ય ફ્રિક્વન્સી પર કેરેક્ટરીસ્ટીક ઇમ્પીડન્સ: $R_0 = \sqrt{L/C}$
    \item કટ-ઓફ ફ્રિક્વન્સી ત્યારે થાય છે જ્યારે $Z_1 = -4Z_2$ અથવા $\omega_c = 2/\sqrt{LC}$
    \item $R_0 = \sqrt{L/C}$ અને $\omega_c = 2/\sqrt{LC}$ પરથી:
       $L = R_0/\pi f_c$
       $C = 1/(\pi f_c R_0)$
\end{enumerate}

\textbf{અંતિમ ડિઝાઈન સમીકરણો:}
\begin{itemize}
    \item ઇન્ડક્ટન્સ: $L = R_0/(\pi f_c)$
    \item કેપેસિટન્સ: $C = 1/(\pi f_c R_0)$
    \item કટઓફ ફ્રિક્વન્સી: $f_c = 1/(\pi \sqrt{LC})$
\end{itemize}

\begin{mnemonicbox}
\mnemonic{One over Pi-Root-LC: The frequency where we Cut}
\end{mnemonicbox}
\end{solutionbox}

\end{document}

