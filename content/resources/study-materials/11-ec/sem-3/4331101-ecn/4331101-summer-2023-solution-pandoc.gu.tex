\documentclass[10pt,a4paper]{article}

% content/resources/templates/preamble.tex
\usepackage[margin=0.6in]{geometry}
\author{Milav Dabgar}
\usepackage{amsmath,amssymb,amsthm}
\usepackage{booktabs}
\usepackage{multirow}
\usepackage{xcolor}
\usepackage{tcolorbox}
\tcbuselibrary{breakable,skins}
\usepackage[colorlinks=true,linkcolor=blue]{hyperref}
\usepackage{titlesec}
\usepackage{enumitem}
\usepackage{tikz}
\usepackage{pgfplots}
\usepackage{circuitikz}
\usepackage[version=4]{mhchem}
\usepackage{longtable}
\usepackage{array}
\usepackage{float}
\usepackage{caption}
\usepackage{listings}

\lstset{
  basicstyle=\small\ttfamily,
  breaklines=true,
  breakatwhitespace=false,
  postbreak=\mbox{\textcolor{red}{$\hookrightarrow$}\space},
  float=false,
  numbers=left,
  numberstyle=\tiny\color{gray},
  numbersep=10pt,
  xleftmargin=2em,
  keywordstyle=\color{blue},
  commentstyle=\color{green!60!black},
  stringstyle=\color{purple},
  backgroundcolor=\color{gray!5},
  showstringspaces=false,
  tabsize=2,
  captionpos=b,
  keepspaces=true,
  columns=flexible
}

\pgfplotsset{compat=1.18}
\usetikzlibrary{shapes,arrows,positioning,calc,patterns,decorations.pathmorphing,decorations.markings,arrows.meta}

% Color scheme
\definecolor{headcolor}{RGB}{0,102,204}
\definecolor{keycolor}{RGB}{220,20,60}
\definecolor{solutioncolor}{RGB}{34,139,34}
\definecolor{mnemoniccolor}{RGB}{148,0,211}
\definecolor{codecolor}{RGB}{0,0,100}

% Spacing
\setlength{\parskip}{3pt}
\setlist[itemize]{nosep}
\setlist[enumerate]{nosep}

% Title formatting
\titleformat{\section}{\Large\bfseries\color{headcolor}}{\thesection}{1em}{}
\titleformat{\subsection}{\large\bfseries\color{headcolor}}{\thesubsection}{1em}{}

% Pandoc tightlist compatibility
\providecommand{\tightlist}{%
  \setlength{\itemsep}{0pt}\setlength{\parskip}{0pt}}

% Pandoc longtable compatibility
\newcounter{none}
\def\thenone{}


% content/resources/templates/gujarati-boxes.tex
\usepackage{fontspec}
\usepackage{polyglossia}

% Set Gujarati as main language (document is primarily in Gujarati)
% Note: gloss-gujarati.ldf doesn't exist in polyglossia, but it will use hyphenation patterns
\setdefaultlanguage{gujarati}
\setotherlanguage{english}

% Configure Gujarati font properly
% Use Language=Default to prevent polyglossia from trying to add language-specific features
% that don't exist for Gujarati, which causes "empty feature" warnings
\newfontfamily\gujaratifont[Script=Gujarati,AutoFakeBold=2.5,AutoFakeSlant=0.3]{Noto Sans Gujarati}
\setmainfont[Script=Gujarati,AutoFakeBold=2.5,AutoFakeSlant=0.3]{Noto Sans Gujarati}
% Use Noto Sans Gujarati for monospace to support Gujarati in text
\setmonofont[Scale=0.9]{Noto Sans Gujarati}

% Configure English to use the same font
\newfontfamily\englishfont[Script=Gujarati,AutoFakeBold=2.5,AutoFakeSlant=0.3]{Noto Sans Gujarati}

% Translations for polyglossia
\gappto\captionsgujarati{
  \renewcommand{\tablename}{કોષ્ટક}
  \renewcommand{\figurename}{આકૃતિ}
}

% Helper for TikZ nodes to ensure Gujarati font
\newcommand{\gu}[1]{{\gujaratifont #1}}

% Custom environments
\newtcolorbox{solutionbox}{
    breakable,
    enhanced,
    colback=solutioncolor!5!white,
    colframe=solutioncolor!75!black,
    fonttitle=\bfseries,
    title=જવાબ
}

\newtcolorbox{solutionboxnobreak}{
 colback=solutioncolor!5!white,
 colframe=solutioncolor!75!black,
 fonttitle=\bfseries,
 title=જવાબ
}

\newtcolorbox{keyformula}{
 breakable,
 enhanced,
 colback=keycolor!5!white,
 colframe=keycolor!75!black,
 fonttitle=\bfseries,
 title=રાસાયણિક સમીકરણ/સૂત્ર
}

\newtcolorbox{mnemonicbox}{
 breakable,
 enhanced,
 colback=mnemoniccolor!5!white,
 colframe=mnemoniccolor!75!black,
 fonttitle=\bfseries,
 title=મેમરી ટ્રીક
}


\begin{document}

\begin{center}
{\Huge\bfseries\color{headcolor} Subject Name (Gujarati)}\\[5pt]
{\LARGE 4331101 -- Summer 2023}\\[3pt]
{\large Semester 1 Study Material}\\[3pt]
{\normalsize\textit{Detailed Solutions and Explanations}}
\end{center}

\vspace{10pt}

\subsection*{પ્રશ્ન 1(a) [3
marks]}\label{q1a}

\textbf{ઇલેક્ટ્રોનિક નેટવર્ક માટે વ્યાખ્યા આપો. (i) નોડ (ii) બ્રાંચ (iii) લૂપ}

\begin{solutionbox}

{\def\LTcaptype{none} % do not increment counter
\begin{longtable}[]{@{}ll@{}}
\toprule\noalign{}
શબ્દ & વ્યાખ્યા \\
\midrule\noalign{}
\endhead
\bottomrule\noalign{}
\endlastfoot
\textbf{નોડ} & એક બિંદુ જ્યાં બે કે વધુ તત્વો એકબીજા સાથે જોડાયેલા હોય \\
\textbf{બ્રાંચ} & બે નોડ વચ્ચેનો એક તત્વ અથવા પાથ \\
\textbf{લૂપ} & નેટવર્કમાં બંધ પાથ જ્યાં કોઈ નોડને એક કરતાં વધુ વખત ક્રોસ ન કરાય \\
\end{longtable}
}

\textbf{આકૃતિ:}

\begin{center}
\textbf{Mermaid Diagram (Code)}
\begin{verbatim}
{Shaded}
{Highlighting}[]
graph LR
    A((Node A)) {-{-}{-} B((Node B)) {-}{-}{-} C((Node C)) {-}{-}{-} D((Node D)) {-}{-}{-} A}
    A {-{-}{-} C}
    style A fill:\#f9f,stroke:\#333,stroke{-width:2px}
    style B fill:\#bbf,stroke:\#333,stroke{-width:2px}
    style C fill:\#f9f,stroke:\#333,stroke{-width:2px}
    style D fill:\#bbf,stroke:\#333,stroke{-width:2px}
{Highlighting}
{Shaded}
\end{verbatim}
\end{center}

\textbf{સરળ રીત:} ``NBL: નેટવર્ક્સ બિગિન વિથ લૂપ્સ''

\end{solutionbox}
\subsection*{પ્રશ્ન 1(b) [4
marks]}\label{q1b}

\textbf{20 Ω, 30 Ω અને 50 Ω નાં રેઝીસ્ટર 60 V નાં સપ્લાય સાથે પેરેલલમાં જોડાયેલા છે.
તો (i) દરેક રેઝીસ્ટરમાંથી પસાર થતો કરંટ તથા કુલ કરંટ (ii) ઇક્વીવેલન્ટ રેઝીસ્ટર
શોધો.}

\begin{solutionbox}

\textbf{આકૃતિ:}

\begin{center}
\textbf{Mermaid Diagram (Code)}
\begin{verbatim}
{Shaded}
{Highlighting}[]
graph LR
    A[60V] {-{-}{-} B(({}+))}
    B {-{-}{-} C[20Ω] {-}{-}{-} D(({}{-}))}
    B {-{-}{-} E[30Ω] {-}{-}{-} D}
    B {-{-}{-} F[50Ω] {-}{-}{-} D}
    D {-{-}{-} A}
{Highlighting}
{Shaded}
\end{verbatim}
\end{center}

{\def\LTcaptype{none} % do not increment counter
\begin{longtable}[]{@{}ll@{}}
\toprule\noalign{}
ગણતરી & મૂલ્ય \\
\midrule\noalign{}
\endhead
\bottomrule\noalign{}
\endlastfoot
\textbf{20 Ω રેઝીસ્ટરમાંથી પસાર થતો કરંટ}: I_{1} = V/R_{1} = 60/20 & 3 A \\
\textbf{30 Ω રેઝીસ્ટરમાંથી પસાર થતો કરંટ}: I_{2} = V/R_{2} = 60/30 & 2 A \\
\textbf{50 Ω રેઝીસ્ટરમાંથી પસાર થતો કરંટ}: I_{3} = V/R_{3} = 60/50 & 1.2 A \\
\textbf{કુલ કરંટ}: I = I_{1} + I_{2} + I_{3} = 3 + 2 + 1.2 & 6.2 A \\
\textbf{ઇક્વીવેલન્ટ રેઝીસ્ટન્સ}: Req = V/I = 60/6.2 & 9.68 Ω \\
\end{longtable}
}

\textbf{સરળ રીત:} ``PIV: પેરેલલ ઇન્ક્રીઝીસ ધ કરંટ, વોલ્ટેજ રીમેઇન્સ ધ સેમ''

\end{solutionbox}
\subsection*{પ્રશ્ન 1(c) [7
marks]}\label{q1c}

\textbf{કેપેસીટર માટે સિરિઝ અને પેરેલલ જોડાણ સમજાવો.}

\begin{solutionbox}

{\def\LTcaptype{none} % do not increment counter
\begin{longtable}[]{@{}
  >{\raggedright\arraybackslash}p{(\linewidth - 4\tabcolsep) * \real{0.3158}}
  >{\raggedright\arraybackslash}p{(\linewidth - 4\tabcolsep) * \real{0.2368}}
  >{\raggedright\arraybackslash}p{(\linewidth - 4\tabcolsep) * \real{0.4474}}@{}}
\toprule\noalign{}
\begin{minipage}[b]{\linewidth}\raggedright
જોડાણ
\end{minipage} & \begin{minipage}[b]{\linewidth}\raggedright
સૂત્ર
\end{minipage} & \begin{minipage}[b]{\linewidth}\raggedright
લક્ષણો
\end{minipage} \\
\midrule\noalign{}
\endhead
\bottomrule\noalign{}
\endlastfoot
\textbf{સિરિઝ જોડાણ} & 1/C\_eq = 1/C_{1} + 1/C_{2} + 1/C_{3} + \ldots{} & -
ઇક્વીવેલન્ટ કેપેસિટન્સ સૌથી નાના કેપેસિટરથી ઓછું- દરેક કેપેસિટરમાં સમાન કરંટ- કુલ વોલ્ટેજ
કેપેસિટરો વચ્ચે વહેંચાય છે- ડાયલેક્ટ્રીક સ્ટ્રેન્થ વધારે છે \\
\textbf{પેરેલલ જોડાણ} & C\_eq = C_{1} + C_{2} + C_{3} + \ldots{} & - ઇક્વીવેલન્ટ
કેપેસિટન્સ બધા કેપેસિટરોનો સરવાળો- દરેક કેપેસિટર પર સમાન વોલ્ટેજ- કુલ ચાર્જ વ્યક્તિગત
ચાર્જનો સરવાળો- પ્લેટનું ક્ષેત્રફળ વધારે છે \\
\end{longtable}
}

\textbf{આકૃતિ:}

\begin{center}
\textbf{Mermaid Diagram (Code)}
\begin{verbatim}
{Shaded}
{Highlighting}[]
graph TD
    subgraph Series
    direction LR
        A["{+"] {-}{-}{-} B[C_{1}] {-}{-}{-} C[C_{2}] {-}{-}{-} D[C_{3}] {-}{-}{-} E["{}{-}"]}
    end

    subgraph Parallel
        F["{+"] {-}{-}{-} G[C_{1}] {-}{-}{-} H["{}{-}"]}
        F {-{-}{-} I[C_{2}] {-}{-}{-} H}
        F {-{-}{-} J[C_{3}] {-}{-}{-} H}
    end
{Highlighting}
{Shaded}
\end{verbatim}
\end{center}

\textbf{સરળ રીત:} ``CAPE: કેપેસિટર્સ એડ ઇન પેરેલલ, એલિમિનેટ ઇન સિરિઝ''

\end{solutionbox}
\subsection*{પ્રશ્ન 1(c) OR [7
marks]}\label{q1c}

\textbf{ઇન્ડક્ટર માટે સિરિઝ અને પેરેલલ જોડાણ સમજાવો.}

\begin{solutionbox}

{\def\LTcaptype{none} % do not increment counter
\begin{longtable}[]{@{}
  >{\raggedright\arraybackslash}p{(\linewidth - 4\tabcolsep) * \real{0.3158}}
  >{\raggedright\arraybackslash}p{(\linewidth - 4\tabcolsep) * \real{0.2368}}
  >{\raggedright\arraybackslash}p{(\linewidth - 4\tabcolsep) * \real{0.4474}}@{}}
\toprule\noalign{}
\begin{minipage}[b]{\linewidth}\raggedright
જોડાણ
\end{minipage} & \begin{minipage}[b]{\linewidth}\raggedright
સૂત્ર
\end{minipage} & \begin{minipage}[b]{\linewidth}\raggedright
લક્ષણો
\end{minipage} \\
\midrule\noalign{}
\endhead
\bottomrule\noalign{}
\endlastfoot
\textbf{સિરિઝ જોડાણ} & L\_eq = L_{1} + L_{2} + L_{3} + \ldots{} & - ઇક્વીવેલન્ટ
ઇન્ડક્ટન્સ બધા ઇન્ડક્ટરોનો સરવાળો- દરેક ઇન્ડક્ટરમાં સમાન કરંટ- કુલ વોલ્ટેજ વ્યક્તિગત
વોલ્ટેજનો સરવાળો- ફ્લક્સ લિંકેજ વધે છે \\
\textbf{પેરેલલ જોડાણ} & 1/L\_eq = 1/L_{1} + 1/L_{2} + 1/L_{3} + \ldots{} & -
ઇક્વીવેલન્ટ ઇન્ડક્ટન્સ સૌથી નાના ઇન્ડક્ટરથી ઓછું- દરેક ઇન્ડક્ટર પર સમાન વોલ્ટેજ- કુલ કરંટ
ઇન્ડક્ટરો વચ્ચે વહેંચાય છે- મેગ્નેટિક કપલિંગ વાસ્તવિક મૂલ્યને અસર કરે છે \\
\end{longtable}
}

\textbf{આકૃતિ:}

\begin{center}
\textbf{Mermaid Diagram (Code)}
\begin{verbatim}
{Shaded}
{Highlighting}[]
graph TD
    subgraph Series
        direction LR    
        A["{+"] {-}{-}{-} B((L_{1})) {-}{-}{-} C((L_{2})) {-}{-}{-} D((L_{3})) {-}{-}{-} E["{}{-}"]}
    end

    subgraph Parallel
        F["{+"] {-}{-}{-} G((L_{1})) {-}{-}{-} H["{}{-}"]}
        F {-{-}{-} I((L_{2})) {-}{-}{-} H}
        F {-{-}{-} J((L_{3})) {-}{-}{-} H}
    end
{Highlighting}
{Shaded}
\end{verbatim}
\end{center}

\textbf{સરળ રીત:} ``LIPS: ઇન્ડક્ટર્સ લિંક ઇન સિરિઝ, પાર્ટિશન ઇન પેરેલલ''

\end{solutionbox}
\subsection*{પ્રશ્ન 2(a) [3
marks]}\label{q2a}

\textbf{વ્યાખ્યા આપો. (i) ટ્રાન્સફોર્મઇમ્પીડન્સ, (ii) ડ્રાઇવિંગ પોઇન્ટ ઇમ્પીડન્સ,
(iii) ટ્રાન્સફર ઇમ્પીડન્સ.}

\begin{solutionbox}

{\def\LTcaptype{none} % do not increment counter
\begin{longtable}[]{@{}
  >{\raggedright\arraybackslash}p{(\linewidth - 2\tabcolsep) * \real{0.3333}}
  >{\raggedright\arraybackslash}p{(\linewidth - 2\tabcolsep) * \real{0.6667}}@{}}
\toprule\noalign{}
\begin{minipage}[b]{\linewidth}\raggedright
શબ્દ
\end{minipage} & \begin{minipage}[b]{\linewidth}\raggedright
વ્યાખ્યા
\end{minipage} \\
\midrule\noalign{}
\endhead
\bottomrule\noalign{}
\endlastfoot
\textbf{ટ્રાન્સફોર્મઇમ્પીડન્સ} & ટ્રાન્સફોર્મરમાં પ્રાથમિકથી ગૌણ તરફ જતા સિગ્નલ
દ્વારા જોવામાં આવતા ઇમ્પીડન્સ \\
\textbf{ડ્રાઇવિંગ પોઇન્ટ ઇમ્પીડન્સ} & એક જ પોર્ટ પર વોલ્ટેજનો કરંટ સાથેનો
ગુણોત્તર \\
\textbf{ટ્રાન્સફર ઇમ્પીડન્સ} & એક પોર્ટ પર વોલ્ટેજનો બીજા પોર્ટના કરંટ સાથેનો
ગુણોત્તર \\
\end{longtable}
}

\textbf{આકૃતિ:}

\begin{center}
\textbf{Mermaid Diagram (Code)}
\begin{verbatim}
{Shaded}
{Highlighting}[]
graph TD
    A[Input] {-{-}{-} B[Two Port Network] {-}{-}{-} C[Output]}
    D[Z11: Driving point impedance] {-.{-}{} B}
    E[Z21: Transfer impedance] {-.{-}{} B}
    F[Z12: Transfer impedance] {-.{-}{} B}
    G[Z22: Driving point impedance] {-.{-}{} B}
{Highlighting}
{Shaded}
\end{verbatim}
\end{center}

\textbf{સરળ રીત:} ``TDT: ટ્રાન્સફોર્મર્સ ડ્રાઇવ ટ્રાન્સફર્સ''

\end{solutionbox}
\subsection*{પ્રશ્ન 2(b) [4
marks]}\label{q2b}

\textbf{30, 50 અને 90 ohms ના રેઝીસ્ટર સ્ટારમાં કનેક્ટ કરેલા છે. ડેલ્ટા કનેક્શનનાં
ઇક્વીવેલન્ટ રેઝીસ્ટર શોધો.}

\begin{solutionbox}

\textbf{આકૃતિ:}

\begin{center}
\textbf{Mermaid Diagram (Code)}
\begin{verbatim}
{Shaded}
{Highlighting}[]
graph LR
    A((A)) {-{-}{-} B[R_{1}=30Ω] {-}{-}{-} D((D))}
    B((B)) {-{-}{-} C[R_{2}=50Ω] {-}{-}{-} D}
    C((C)) {-{-}{-} E[R_{3}=90Ω] {-}{-}{-} D}

    subgraph Equivalent Delta
        A {-{-}{-} F[R_{1}_{2}] {-}{-}{-} B}
        B {-{-}{-} G[R_{2}_{3}] {-}{-}{-} C}
        C {-{-}{-} H[R_{3}_{1}] {-}{-}{-} A}
    end
{Highlighting}
{Shaded}
\end{verbatim}
\end{center}

{\def\LTcaptype{none} % do not increment counter
\begin{longtable}[]{@{}
  >{\raggedright\arraybackslash}p{(\linewidth - 4\tabcolsep) * \real{0.6182}}
  >{\raggedright\arraybackslash}p{(\linewidth - 4\tabcolsep) * \real{0.2364}}
  >{\raggedright\arraybackslash}p{(\linewidth - 4\tabcolsep) * \real{0.1455}}@{}}
\toprule\noalign{}
\begin{minipage}[b]{\linewidth}\raggedright
સ્ટાર થી ડેલ્ટા કન્વર્ઝન ફોર્મ્યુલા
\end{minipage} & \begin{minipage}[b]{\linewidth}\raggedright
ગણતરી
\end{minipage} & \begin{minipage}[b]{\linewidth}\raggedright
પરિણામ
\end{minipage} \\
\midrule\noalign{}
\endhead
\bottomrule\noalign{}
\endlastfoot
R_{1}_{2} = (R_{1}\timesR_{2} + R_{2}\timesR_{3} + R_{3}\timesR_{1})/R_{3} & (30\times50 + 50\times90 + 90\times30)/90 & 105 Ω \\
R_{2}_{3} = (R_{1}\timesR_{2} + R_{2}\timesR_{3} + R_{3}\timesR_{1})/R_{1} & (30\times50 + 50\times90 + 90\times30)/30 & 315 Ω \\
R_{3}_{1} = (R_{1}\timesR_{2} + R_{2}\timesR_{3} + R_{3}\timesR_{1})/R_{2} & (30\times50 + 50\times90 + 90\times30)/50 & 189 Ω \\
\end{longtable}
}

\textbf{સરળ રીત:} ``PSR: પ્રોડક્ટ ઓવર સમ ઓફ રેસિસ્ટર્સ''

\end{solutionbox}
\subsection*{પ્રશ્ન 2(c) [7
marks]}\label{q2c}

\textbf{Π નેટવર્ક સમજાવો.}

\begin{solutionbox}

{\def\LTcaptype{none} % do not increment counter
\begin{longtable}[]{@{}
  >{\raggedright\arraybackslash}p{(\linewidth - 2\tabcolsep) * \real{0.4091}}
  >{\raggedright\arraybackslash}p{(\linewidth - 2\tabcolsep) * \real{0.5909}}@{}}
\toprule\noalign{}
\begin{minipage}[b]{\linewidth}\raggedright
વિભાવના
\end{minipage} & \begin{minipage}[b]{\linewidth}\raggedright
વર્ણન
\end{minipage} \\
\midrule\noalign{}
\endhead
\bottomrule\noalign{}
\endlastfoot
\textbf{વ્યાખ્યા} & ત્રણ-ટર્મિનલ નેટવર્ક જે ત્રણ ઇમ્પીડન્સથી બનેલું હોય - એક સિરીઝમાં
અને બે પેરેલલમાં \\
\textbf{સ્ટ્રક્ચર} & બે ઇમ્પીડન્સ ઇનપુટ અને આઉટપુટથી કોમન બિંદુ સુધી જોડાયેલા, એક
ઇનપુટ અને આઉટપુટ વચ્ચે \\
\textbf{પેરામીટર્સ} & Z, Y, h, અથવા ABCD પેરામીટર્સનો ઉપયોગ કરીને વ્યાખ્યાયિત
કરી શકાય છે \\
\textbf{એપ્લિકેશન્સ} & મેચિંગ નેટવર્ક્સ, ફિલ્ટર્સ, એટેન્યુએટર્સ, ફેઝ શિફ્ટર્સ \\
\end{longtable}
}

\textbf{આકૃતિ:}

\begin{center}
\textbf{Mermaid Diagram (Code)}
\begin{verbatim}
{Shaded}
{Highlighting}[]
graph LR
    A[Input] {-{-}{-} C[Z_{2}] {-}{-}{-} B[Output]}
    A {-{-}{-} D[Z_{1}] {-}{-}{-} E[Common/Ground]}
    B {-{-}{-} F[Z_{3}] {-}{-}{-} E}

    style D fill:\#bbf,stroke:\#333,stroke{-width:2px}
    style C fill:\#f96,stroke:\#333,stroke{-width:2px}
    style F fill:\#bbf,stroke:\#333,stroke{-width:2px}
{Highlighting}
{Shaded}
\end{verbatim}
\end{center}

\textbf{સરળ રીત:} ``PIE: પાઈ ઇમ્પીડન્સીસ કનેક્ટેડ એટ એન્ડ્સ''

\end{solutionbox}
\subsection*{પ્રશ્ન 2(a) OR [3
marks]}\label{q2a}

\textbf{નેટવર્કનાં પ્રકારો જણાવો.}

\begin{solutionbox}

{\def\LTcaptype{none} % do not increment counter
\begin{longtable}[]{@{}ll@{}}
\toprule\noalign{}
નેટવર્ક પ્રકારો & ઉદાહરણો \\
\midrule\noalign{}
\endhead
\bottomrule\noalign{}
\endlastfoot
\textbf{લિનિયરતા આધારિત} & લિનિયર નેટવર્ક્સ, નોન-લિનિયર નેટવર્ક્સ \\
\textbf{ઘટકો આધારિત} & પેસિવ નેટવર્ક્સ, એક્ટિવ નેટવર્ક્સ \\
\textbf{સ્ટ્રક્ચર આધારિત} & લમ્પ્ડ નેટવર્ક્સ, ડિસ્ટ્રિબ્યુટેડ નેટવર્ક્સ \\
\textbf{વર્તણૂક આધારિત} & બાઇલેટરલ નેટવર્ક્સ, યુનિલેટરલ નેટવર્ક્સ \\
\textbf{ટોપોલોજી આધારિત} & T-નેટવર્ક્સ, π-નેટવર્ક્સ, લેટિસ નેટવર્ક્સ \\
\textbf{પોર્ટ્સ આધારિત} & વન-પોર્ટ નેટવર્ક્સ, ટુ-પોર્ટ નેટવર્ક્સ, મલ્ટિ-પોર્ટ
નેટવર્ક્સ \\
\end{longtable}
}

\textbf{આકૃતિ:}

\begin{center}
\textbf{Mermaid Diagram (Code)}
\begin{verbatim}
{Shaded}
{Highlighting}[]
graph TD
    A[Network Types] {-{-}{} B[Linear/Non{-}linear]}
    A {-{-}{} C[Passive/Active]}
    A {-{-}{} D[Lumped/Distributed]}
    A {-{-}{} E[Bilateral/Unilateral]}
    A {-{-}{} F[T/π/Lattice]}
    A {-{-}{} G[One{-}port/Two{-}port/Multi{-}port]}
{Highlighting}
{Shaded}
\end{verbatim}
\end{center}

\textbf{સરળ રીત:} ``PLAN-TB: પેસિવ-લિનિયર-એક્ટિવ-નેટવર્ક-ટોપોલોજી-બાઇલેટરલ''

\end{solutionbox}
\subsection*{પ્રશ્ન 2(b) OR [4
marks]}\label{q2b}

\textbf{40, 60 અને 80 ohms ના રેઝીસ્ટર ડેલ્ટામાં કનેક્ટ કરેલા છે. સ્ટાર કનેક્શનનાં
ઇક્વીવેલન્ટ રેઝીસ્ટર શોધો.}

\begin{solutionbox}

\textbf{આકૃતિ:}

\begin{center}
\textbf{Mermaid Diagram (Code)}
\begin{verbatim}
{Shaded}
{Highlighting}[]
graph TD
    A((A)) {-{-}{-} B[R_{1}_{2}=40Ω] {-}{-}{-} B((B))}
    B {-{-}{-} C[R_{2}_{3}=60Ω] {-}{-}{-} C((C))}
    C {-{-}{-} D[R_{3}_{1}=80Ω] {-}{-}{-} A}

    subgraph Equivalent Star
        A {-{-}{-} E[R_{1}] {-}{-}{-} G((D))}
        B {-{-}{-} F[R_{2}] {-}{-}{-} G}
        C {-{-}{-} H[R_{3}] {-}{-}{-} G}
    end
{Highlighting}
{Shaded}
\end{verbatim}
\end{center}

{\def\LTcaptype{none} % do not increment counter
\begin{longtable}[]{@{}lll@{}}
\toprule\noalign{}
ડેલ્ટા થી સ્ટાર કન્વર્ઝન ફોર્મ્યુલા & ગણતરી & પરિણામ \\
\midrule\noalign{}
\endhead
\bottomrule\noalign{}
\endlastfoot
R_{1} = (R_{1}_{2}\timesR_{3}_{1})/(R_{1}_{2}+R_{2}_{3}+R_{3}_{1}) & (40\times80)/(40+60+80) & 17.78 Ω \\
R_{2} = (R_{1}_{2}\timesR_{2}_{3})/(R_{1}_{2}+R_{2}_{3}+R_{3}_{1}) & (40\times60)/(40+60+80) & 13.33 Ω \\
R_{3} = (R_{2}_{3}\timesR_{3}_{1})/(R_{1}_{2}+R_{2}_{3}+R_{3}_{1}) & (60\times80)/(40+60+80) & 26.67 Ω \\
\end{longtable}
}

\textbf{સરળ રીત:} ``DPS: ડેલ્ટા પ્રોડક્ટ ઓવર સમ''

\end{solutionbox}
\subsection*{પ્રશ્ન 2(c) OR [7
marks]}\label{q2c}

\textbf{symmetrical T -- network માટે કેરેક્ટરાસ્ટીક ઇમ્પીડન્સ સમજાવો. ZOT નું
સૂત્ર ZOC and ZSC ના રૂપમાં તારવો.}

\begin{solutionbox}

{\def\LTcaptype{none} % do not increment counter
\begin{longtable}[]{@{}
  >{\raggedright\arraybackslash}p{(\linewidth - 2\tabcolsep) * \real{0.4091}}
  >{\raggedright\arraybackslash}p{(\linewidth - 2\tabcolsep) * \real{0.5909}}@{}}
\toprule\noalign{}
\begin{minipage}[b]{\linewidth}\raggedright
વિભાવના
\end{minipage} & \begin{minipage}[b]{\linewidth}\raggedright
વર્ણન
\end{minipage} \\
\midrule\noalign{}
\endhead
\bottomrule\noalign{}
\endlastfoot
\textbf{કેરેક્ટરાસ્ટીક ઇમ્પીડન્સ (Z_{0})} & આઉટપુટ પોર્ટ પર જોડાયેલું ઇમ્પીડન્સ જેના કારણે
ઇનપુટ ઇમ્પીડન્સ Z_{0} ની બરાબર થાય \\
\textbf{સિમેટ્રિકલ T-નેટવર્ક} & T-નેટવર્ક જેમાં બંને બાજુના સિરીઝ ઇમ્પીડન્સ સમાન
હોય \\
\textbf{ZOC અને ZSC} & નેટવર્કના ઓપન-સર્કિટ અને શોર્ટ-સર્કિટ ઇમ્પીડન્સીસ \\
\end{longtable}
}

\textbf{આકૃતિ:}

\begin{center}
\textbf{Mermaid Diagram (Code)}
\begin{verbatim}
{Shaded}
{Highlighting}[]
graph LR
    A[Input] {-{-}{-} B[Z_{1}] {-}{-}{-} C((Middle)) {-}{-}{-} D[Z_{1}] {-}{-}{-} E[Output]}
    C {-{-}{-} F[Z_{2}] {-}{-}{-} G[Ground]}
    H[Z_{0] {-}.{-}{} E}
{Highlighting}
{Shaded}
\end{verbatim}
\end{center}

સિમેટ્રિકલ T-નેટવર્ક માટે:

\begin{itemize}
\tightlist
\item
  સિરીઝ ઇમ્પીડન્સીસ (Z_{1}) સમાન હોય છે
\item
  Z_{2} એ શન્ટ ઇમ્પીડન્સ છે
\end{itemize}

કેરેક્ટરાસ્ટીક ઇમ્પીડન્સ (Z_{0}ᵀ) આ રીતે આપવામાં આવે છે: Z_{0}ᵀ = \sqrt(Z_{0}ᶜ \times Z_{0}ˢᶜ)

જ્યાં:

\begin{itemize}
\tightlist
\item
  Z_{0}ᶜ = ઓપન સર્કિટ ઇમ્પીડન્સ = Z_{1} + Z_{2} + (Z_{1}\timesZ_{2})/Z_{1} = Z_{1} + Z_{2}
\item
  Z_{0}ˢᶜ = શોર્ટ સર્કિટ ઇમ્પીડન્સ = Z_{1}^{2}/Z_{2}
\end{itemize}

તેથી: Z_{0}ᵀ = \sqrt[(Z_{1} + Z_{2}) \times Z_{1}^{2}/Z_{2}] = \sqrt[Z_{1}^{2} + Z_{1}\timesZ_{2}]

\textbf{સરળ રીત:} ``TOSS: T-નેટવર્ક્સ ઓપન એન્ડ શોર્ટ સર્કિટ સ્ક્વેર-રૂટ''

\end{solutionbox}
\subsection*{પ્રશ્ન 3(a) [3
marks]}\label{q3a}

\textbf{Kirchhoff's law સમજાવો.}

\begin{solutionbox}

{\def\LTcaptype{none} % do not increment counter
\begin{longtable}[]{@{}
  >{\raggedright\arraybackslash}p{(\linewidth - 4\tabcolsep) * \real{0.1724}}
  >{\raggedright\arraybackslash}p{(\linewidth - 4\tabcolsep) * \real{0.3793}}
  >{\raggedright\arraybackslash}p{(\linewidth - 4\tabcolsep) * \real{0.4483}}@{}}
\toprule\noalign{}
\begin{minipage}[b]{\linewidth}\raggedright
નિયમ
\end{minipage} & \begin{minipage}[b]{\linewidth}\raggedright
વિધાન
\end{minipage} & \begin{minipage}[b]{\linewidth}\raggedright
ઉપયોગ
\end{minipage} \\
\midrule\noalign{}
\endhead
\bottomrule\noalign{}
\endlastfoot
\textbf{Kirchhoff's Current Law (KCL)} & નોડમાં પ્રવેશતા કરંટનો સરવાળો
નોડમાંથી નીકળતા કરંટના સરવાળા બરાબર હોય & નોડલ એનાલિસિસ માટે ઉપયોગી \\
\textbf{Kirchhoff's Voltage Law (KVL)} & કોઈપણ બંધ લૂપની આસપાસ વોલ્ટેજનો
સરવાળો શૂન્ય હોય & મેશ એનાલિસિસ માટે ઉપયોગી \\
\end{longtable}
}

\textbf{આકૃતિ:}

\begin{center}
\textbf{Mermaid Diagram (Code)}
\begin{verbatim}
{Shaded}
{Highlighting}[]
graph TD
    subgraph "Current Law"
        A((Node)) {-{-}{-} B[I_{1}]}
        A {-{-}{-} C[I_{2}]}
        A {-{-}{-} D[I_{3}]}
        A {-{-}{-} E[I_{4}]}
    end

    subgraph "Voltage Law"
            direction LR
        F[V_{1] {-}{-}{-} G[V_{2}] {-}{-}{-} H[V_{3}] {-}{-}{-} I[V_{4}] {-}{-}{-} F}
    end
{Highlighting}
{Shaded}
\end{verbatim}
\end{center}

\textbf{સરળ રીત:} ``KVC: કિરચોફ વેરિફાઈસ કરંટ એન્ડ વોલ્ટેજ લોઝ''

\end{solutionbox}
\subsection*{પ્રશ્ન 3(b) [4
marks]}\label{q3b}

\textbf{Mesh analysis સમજાવો.}

\begin{solutionbox}

{\def\LTcaptype{none} % do not increment counter
\begin{longtable}[]{@{}
  >{\raggedright\arraybackslash}p{(\linewidth - 2\tabcolsep) * \real{0.4091}}
  >{\raggedright\arraybackslash}p{(\linewidth - 2\tabcolsep) * \real{0.5909}}@{}}
\toprule\noalign{}
\begin{minipage}[b]{\linewidth}\raggedright
વિભાવના
\end{minipage} & \begin{minipage}[b]{\linewidth}\raggedright
વર્ણન
\end{minipage} \\
\midrule\noalign{}
\endhead
\bottomrule\noalign{}
\endlastfoot
\textbf{વ્યાખ્યા} & દરેક સ્વતંત્ર બંધ લૂપ (મેશ) માટે KVL લાગુ પાડીને સર્કિટ સમસ્યાઓ
ઉકેલવાની પદ્ધતિ \\
\textbf{પ્રક્રિયા} & 1. દરેક લૂપને મેશ કરંટ આપો2. દરેક મેશ માટે KVL સમીકરણો લખો3.
પરિણામી સમીકરણોની સિસ્ટમ ઉકેલો \\
\textbf{ફાયદાઓ} & - સમીકરણોની સંખ્યા ઘટાડે છે- ઘણી શાખાઓ વાળા સર્કિટ્સ માટે સારું
કામ કરે છે- વોલ્ટેજ સ્ત્રોતો વાળી સમસ્યાઓ માટે યોગ્ય \\
\end{longtable}
}

\textbf{આકૃતિ:}

\begin{center}
\textbf{Mermaid Diagram (Code)}
\begin{verbatim}
{Shaded}
{Highlighting}[]
graph LR
    A(({+)) {-}{-}{-} B[R_{1}] {-}{-}{-} C((B)) {-}{-}{-} D[R_{2}] {-}{-}{-} E((C))}
    E {-{-}{-} F[R_{3}] {-}{-}{-} G((D)) {-}{-}{-} H[R_{4}] {-}{-}{-} A}
    C {-{-}{-} I[R_{5}] {-}{-}{-} G}

    J[I_{1] {-}.{-}{} B}
    K[I_{2] {-}.{-}{} D}
    L[I_{3] {-}.{-}{} I}
{Highlighting}
{Shaded}
\end{verbatim}
\end{center}

\textbf{સરળ રીત:} ``MAIL: મેશ એનાલિસિસ યુઝિસ ઇન્ડિપેન્ડન્ટ લૂપ્સ''

\end{solutionbox}
\subsection*{પ્રશ્ન 3(c) [7
marks]}\label{q3c}

\textbf{Thevenin's theorem નો ઉપયોગ કરીને ઉપર દશાર્વેલ સર્કિટ માટે 5 Ω રેઝીસ્ટર
માંથી પસાર થતો કરંટ શોધો.}

\begin{solutionbox}

\textbf{આકૃતિ:}

\begin{verbatim}
          10Ω        15Ω
          {-{-}{-}{-}      {-}{-}{-}{-}}
         /    {    /    }
        /      {  /      }
  100V +        A        B
        {       |        /}
         {     {-}{-}{-}      /}
          {   | 5Ω|    /}
           {  |   |   /}
            { {-}{-}{-}   /}
             {|   |/}
              6Ω  8Ω
\end{verbatim}

\textbf{સ્ટેપ 1:} 5Ω રેઝીસ્ટર દૂર કરીને ઓપન સર્કિટ વોલ્ટેજ (V_{t}_{h}) શોધો \textbf{સ્ટેપ
2:} થેવેનિનનું ઇક્વિવેલન્ટ રેઝિસ્ટન્સ (R_{t}_{h}) શોધો \textbf{સ્ટેપ 3:} 5Ω રેઝીસ્ટરમાંથી
પસાર થતો કરંટ ગણો

{\def\LTcaptype{none} % do not increment counter
\begin{longtable}[]{@{}
  >{\raggedright\arraybackslash}p{(\linewidth - 4\tabcolsep) * \real{0.2222}}
  >{\raggedright\arraybackslash}p{(\linewidth - 4\tabcolsep) * \real{0.4815}}
  >{\raggedright\arraybackslash}p{(\linewidth - 4\tabcolsep) * \real{0.2963}}@{}}
\toprule\noalign{}
\begin{minipage}[b]{\linewidth}\raggedright
સ્ટેપ
\end{minipage} & \begin{minipage}[b]{\linewidth}\raggedright
ગણતરી
\end{minipage} & \begin{minipage}[b]{\linewidth}\raggedright
પરિણામ
\end{minipage} \\
\midrule\noalign{}
\endhead
\bottomrule\noalign{}
\endlastfoot
\textbf{V_{t}_{h}} & A અને B વચ્ચેનું વોલ્ટેજ જ્યારે 5Ω દૂર કરવામાં આવે & 38.46 V \\
\textbf{R_{t}_{h}} & A અને B થી જોવાતું ઇક્વિવેલન્ટ રેઝિસ્ટન્સ જ્યારે 100V સ્ત્રોત શોર્ટ
કરવામાં આવે & 3.6 Ω \\
\textbf{કરંટ} & I = V_{t}_{h}/(R_{t}_{h} + 5) = 38.46/(3.6 + 5) & 4.47 A \\
\end{longtable}
}

\textbf{સરળ રીત:} ``TVR: થેવેનિન રિપ્લેસીસ વોલ્ટેજ એન્ડ રેઝીસ્ટન્સ''

\end{solutionbox}
\subsection*{પ્રશ્ન 3(a) OR [3
marks]}\label{q3a}

\textbf{Superposition Theorem જણાવો અને સમજાવો.}

\begin{solutionbox}

{\def\LTcaptype{none} % do not increment counter
\begin{longtable}[]{@{}
  >{\raggedright\arraybackslash}p{(\linewidth - 2\tabcolsep) * \real{0.4091}}
  >{\raggedright\arraybackslash}p{(\linewidth - 2\tabcolsep) * \real{0.5909}}@{}}
\toprule\noalign{}
\begin{minipage}[b]{\linewidth}\raggedright
વિભાવના
\end{minipage} & \begin{minipage}[b]{\linewidth}\raggedright
વર્ણન
\end{minipage} \\
\midrule\noalign{}
\endhead
\bottomrule\noalign{}
\endlastfoot
\textbf{વિધાન} & લિનિયર સર્કિટમાં બહુવિધ સ્ત્રોતો સાથે, કોઈપણ બિંદુ પર પ્રતિભાવ
દરેક સ્ત્રોત એકલા કાર્ય કરતા હોય ત્યારે થતા પ્રતિભાવોના સરવાળા બરાબર હોય છે \\
\textbf{પ્રક્રિયા} & 1. એક સમયે એક સ્ત્રોત ધ્યાનમાં લો2. અન્ય વોલ્ટેજ સ્ત્રોતોને શોર્ટ
સર્કિટથી બદલો3. અન્ય કરંટ સ્ત્રોતોને ઓપન સર્કિટથી બદલો4. વ્યક્તિગત પ્રતિભાવો
શોધો5. બધા પ્રતિભાવોને બીજગણિતીય રીતે ઉમેરો \\
\textbf{મર્યાદા} & માત્ર લિનિયર સર્કિટ્સ અને વોલ્ટેજ/કરંટ પ્રતિભાવો માટે જ લાગુ \\
\end{longtable}
}

\textbf{આકૃતિ:}

\begin{center}
\textbf{Mermaid Diagram (Code)}
\begin{verbatim}
{Shaded}
{Highlighting}[]
graph LR
    A[Original Circuit] {-{-}{} B[Circuit with V_{1} only]}
    A {-{-}{} C[Circuit with V_{2} only]}
    B {-{-}{} D[Response R_{1}]}
    C {-{-}{} E[Response R_{2}]}
    D {-{-}{} F[Total Response = R_{1} + R_{2}]}
    E {-{-}{} F}
{Highlighting}
{Shaded}
\end{verbatim}
\end{center}

\textbf{સરળ રીત:} ``SUPER: સોર્સિસ યુઝ્ડ પ્રોગ્રેસિવલી ઈક્વલ્સ રિસ્પોન્સ''

\end{solutionbox}
\subsection*{પ્રશ્ન 3(b) OR [4
marks]}\label{q3b}

\textbf{કોઈપણ સર્કિટનો ઉપયોગ કરીને ડ્યુઅલ નેટવર્ક દોરવાની પદ્ધતિ સમજાવો.}

\begin{solutionbox}

{\def\LTcaptype{none} % do not increment counter
\begin{longtable}[]{@{}
  >{\raggedright\arraybackslash}p{(\linewidth - 2\tabcolsep) * \real{0.3333}}
  >{\raggedright\arraybackslash}p{(\linewidth - 2\tabcolsep) * \real{0.6667}}@{}}
\toprule\noalign{}
\begin{minipage}[b]{\linewidth}\raggedright
સ્ટેપ
\end{minipage} & \begin{minipage}[b]{\linewidth}\raggedright
વર્ણન
\end{minipage} \\
\midrule\noalign{}
\endhead
\bottomrule\noalign{}
\endlastfoot
\textbf{ગ્રાફમાં રૂપાંતરણ} & સર્કિટને પ્લેનર ગ્રાફ તરીકે દોરો \\
\textbf{ડ્યુઅલ ગ્રાફ દોરો} & મૂળ ગ્રાફના દરેક ક્ષેત્રમાં એક નોડ મૂકો \\
\textbf{નોડ્સ જોડો} & મૂળ ગ્રાફની દરેક એજને ક્રોસ કરતી એજ દોરો \\
\textbf{ઘટકોને બદલો} & - રેઝિસ્ટન્સ R કન્ડક્ટન્સ 1/R બને- વોલ્ટેજ સોર્સ કરંટ સોર્સ
બને- સિરીઝ પેરેલલ બને- ઇમ્પીડન્સ Z એડમિટન્સ 1/Z બને \\
\end{longtable}
}

\textbf{આકૃતિ:}

\begin{verbatim}
Original Circuit     Dual Circuit
   +{-{-}{-}R1{-}{-}{-}+         +{-}{-}{-}G1{-}{-}{-}+}
   |        |         |        |
  V1       R2        I1       G2
   |        |         |        |
   +{-{-}{-}R3{-}{-}{-}+         +{-}{-}{-}G3{-}{-}{-}+}
\end{verbatim}

\textbf{સરળ રીત:} ``DVSG: ડ્યુઅલ ટ્રાન્સફોર્મ્સ વોલ્ટેજ ટુ સિરીઝ ટુ ગ્રાફ્સ''

\end{solutionbox}
\subsection*{પ્રશ્ન 3(c) OR [7
marks]}\label{q3c}

\textbf{ઉપર આપેલ નેટવર્ક માટે નોર્ટનની ઇક્વીવેલન્ટ સર્કિટ શોધો. લોડ કરંટ શોધો જો
(i) RL=3 KΩ (ii) RL=1.5 Ω}

\begin{solutionbox}

\textbf{આકૃતિ:}

\begin{verbatim}
        2kΩ          2kΩ          2kΩ
       {-{-}{-}{-}{-}        {-}{-}{-}{-}{-}        {-}{-}{-}{-}{-}}
      /     {      /           /     }
     /       {    /           /       }
  C +         D  +         E  +         A
     {                               }
      {                               }
       {          |         | |         |}
       |         | |         | |         |
      10V         2kΩ         2kΩ         RL
       |         | |         | |         |
       |         | |         | |         |
       +         + +         + +         +
       B         B B         B B         B
\end{verbatim}

\textbf{સ્ટેપ 1:} નોર્ટનનો કરંટ (IN) શોધો \textbf{સ્ટેપ 2:} નોર્ટનનું રેઝિસ્ટન્સ
(RN) શોધો \textbf{સ્ટેપ 3:} લોડ કરંટ્સ ગણો

{\def\LTcaptype{none} % do not increment counter
\begin{longtable}[]{@{}
  >{\raggedright\arraybackslash}p{(\linewidth - 4\tabcolsep) * \real{0.2222}}
  >{\raggedright\arraybackslash}p{(\linewidth - 4\tabcolsep) * \real{0.4815}}
  >{\raggedright\arraybackslash}p{(\linewidth - 4\tabcolsep) * \real{0.2963}}@{}}
\toprule\noalign{}
\begin{minipage}[b]{\linewidth}\raggedright
સ્ટેપ
\end{minipage} & \begin{minipage}[b]{\linewidth}\raggedright
ગણતરી
\end{minipage} & \begin{minipage}[b]{\linewidth}\raggedright
પરિણામ
\end{minipage} \\
\midrule\noalign{}
\endhead
\bottomrule\noalign{}
\endlastfoot
\textbf{IN} & A થી B સુધીનો શોર્ટ સર્કિટ કરંટ & 1.25 mA \\
\textbf{RN} & A થી B સુધી જોવાતું ઇક્વિવેલન્ટ રેઝિસ્ટન્સ જ્યારે 10V સ્ત્રોત શોર્ટ
કરવામાં આવે & 1 kΩ \\
\textbf{IL (RL = 3 KΩ)} & IL = IN \times RN/(RN + RL) = 1.25 \times 1/(1 + 3) &
0.31 mA \\
\textbf{IL (RL = 1.5 Ω)} & IL = IN \times RN/(RN + RL) = 1.25 \times 1000/(1000 +
1.5) & 1.25 mA \\
\end{longtable}
}

\textbf{સરળ રીત:} ``NICE: નોર્ટન્સ સર્કિટ ઈઝ કરંટ ઈક્વિવેલન્ટ''

\end{solutionbox}
\subsection*{પ્રશ્ન 4(a) [3
marks]}\label{q4a}

\textbf{કોઇલ માટે ક્વોલિટી ફેક્ટર Q નું સમીકરણ મેળવો.}

\begin{solutionbox}

{\def\LTcaptype{none} % do not increment counter
\begin{longtable}[]{@{}ll@{}}
\toprule\noalign{}
પેરામીટર & સંબંધ \\
\midrule\noalign{}
\endhead
\bottomrule\noalign{}
\endlastfoot
\textbf{Q ફેક્ટર વ્યાખ્યા} & સંગ્રહિત ઊર્જા અને પ્રતિ ચક્ર વેડફાતી ઊર્જાનો ગુણોત્તર \\
\textbf{કોઇલ ઇમ્પીડન્સ} & Z = R + jωL \\
\textbf{રિએક્ટન્સ} & XL = ωL \\
\textbf{ક્વોલિટી ફેક્ટર} & Q = XL/R = ωL/R \\
\end{longtable}
}

\textbf{આકૃતિ:}

\begin{verbatim}
    +{-{-}{-}R{-}{-}{-}+}
    |       |
    +{-{-}L{-}{-}{-}{-}+}
\end{verbatim}

કોઇલ માટે, સંગ્રહિત ઊર્જા ચુંબકીય ક્ષેત્રમાં (ઇન્ડક્ટરમાં) હોય છે, જ્યારે વેડફાતી ઊર્જા
રેઝિસ્ટન્સમાં હોય છે. આમાંથી:

Q = 2π \times (સંગ્રહિત ઊર્જા)/(પ્રતિ ચક્ર વેડફાતી ઊર્જા)

Q = ωL/R


\textbf{સરળ રીત:} ``QREL: ક્વોલિટી રિલેટ્સ એનર્જી ટુ લોસ''

\end{solutionbox}
\subsection*{પ્રશ્ન 4(b) [4
marks]}\label{q4b}

\textbf{શ્રેણી RLC સર્કિટમાં R=30 Ω, L=0.5 H અને C=5 µF છે. (i)Q પરિબળ, (ii)
BW, (iii) અપર કટ ઓફ અને લોઅર કટ ઓફ ફ્રીક્વન્સીઝની ગણતરી કરો.}

\begin{solutionbox}

\textbf{આકૃતિ:}

\begin{center}
\textbf{Mermaid Diagram (Code)}
\begin{verbatim}
{Shaded}
{Highlighting}[]
graph LR
    A[Input] {-{-}{-} B[R=30Ω] {-}{-}{-} C[L=0.5H] {-}{-}{-} D[C=5µF] {-}{-}{-} E[Output]}
{Highlighting}
{Shaded}
\end{verbatim}
\end{center}

{\def\LTcaptype{none} % do not increment counter
\begin{longtable}[]{@{}
  >{\raggedright\arraybackslash}p{(\linewidth - 6\tabcolsep) * \real{0.2683}}
  >{\raggedright\arraybackslash}p{(\linewidth - 6\tabcolsep) * \real{0.2195}}
  >{\raggedright\arraybackslash}p{(\linewidth - 6\tabcolsep) * \real{0.3171}}
  >{\raggedright\arraybackslash}p{(\linewidth - 6\tabcolsep) * \real{0.1951}}@{}}
\toprule\noalign{}
\begin{minipage}[b]{\linewidth}\raggedright
પેરામીટર
\end{minipage} & \begin{minipage}[b]{\linewidth}\raggedright
સૂત્ર
\end{minipage} & \begin{minipage}[b]{\linewidth}\raggedright
ગણતરી
\end{minipage} & \begin{minipage}[b]{\linewidth}\raggedright
પરિણામ
\end{minipage} \\
\midrule\noalign{}
\endhead
\bottomrule\noalign{}
\endlastfoot
\textbf{રેઝોનન્ટ ફ્રીક્વન્સી (f_{0})} & f_{0} = 1/(2π\sqrtLC) & 1/(2π\sqrt(0.5\times5\times10^{-}^{6})) &
100.53 Hz \\
\textbf{Q ફેક્ટર} & Q = (1/R)\sqrt(L/C) & (1/30)\sqrt(0.5/(5\times10^{-}^{6})) & 105.57 \\
\textbf{બેન્ડવિડ્થ (BW)} & BW = f_{0}/Q & 100.53/105.57 & 0.952 Hz \\
\textbf{લોઅર કટઓફ (f_{1})} & f_{1} = f_{0} - BW/2 & 100.53 - 0.952/2 & 100.05
Hz \\
\textbf{અપર કટઓફ (f_{2})} & f_{2} = f_{0} + BW/2 & 100.53 + 0.952/2 & 101.01
Hz \\
\end{longtable}
}

\textbf{સરળ રીત:} ``QBCUT: ક્વોલિટી બેન્ડવિડ્થ કટઓફ યુનિકલી રિલેટેડ''

\end{solutionbox}
\subsection*{પ્રશ્ન 4(c) [7
marks]}\label{q4c}

\textbf{મ્યુચ્યુઅલ ઇન્ડક્ટન્સના કો-એફીસીએન્ટ સાથે મ્યુચ્યુઅલ ઇન્ડક્ટન્સ સમજાવો. K નું
સમીકરણ પણ મેળવો.}

\begin{solutionbox}

{\def\LTcaptype{none} % do not increment counter
\begin{longtable}[]{@{}
  >{\raggedright\arraybackslash}p{(\linewidth - 2\tabcolsep) * \real{0.4091}}
  >{\raggedright\arraybackslash}p{(\linewidth - 2\tabcolsep) * \real{0.5909}}@{}}
\toprule\noalign{}
\begin{minipage}[b]{\linewidth}\raggedright
વિભાવના
\end{minipage} & \begin{minipage}[b]{\linewidth}\raggedright
વર્ણન
\end{minipage} \\
\midrule\noalign{}
\endhead
\bottomrule\noalign{}
\endlastfoot
\textbf{મ્યુચ્યુઅલ ઇન્ડક્ટન્સ (M)} & ગુણધર્મ જ્યાં એક કોઇલમાં કરંટ બદલાવથી પાસેની
કોઇલમાં વોલ્ટેજ ઉત્પન્ન થાય છે \\
\textbf{વ્યાખ્યા} & પ્રાથમિક કોઇલમાં કરંટના બદલાવના દરના સાપેક્ષ ગૌણ કોઇલમાં
પ્રેરિત વોલ્ટેજનો ગુણોત્તર \\
\textbf{સૂત્ર} & M = k\sqrt(L_{1}L_{2}) \\
\textbf{કપલિંગ ગુણાંક (k)} & કોઇલ્સ વચ્ચે ચુંબકીય કપલિંગનું માપ (0 \leq k \leq 1) \\
\end{longtable}
}

\textbf{આકૃતિ:}

\begin{center}
\textbf{Mermaid Diagram (Code)}
\begin{verbatim}
{Shaded}
{Highlighting}[]
graph LR
    A[I_{1] {-}{-}{-} B[L_{1}] {-}{-}{-} C}
    D[I_{2] {-}{-}{-} E[L_{2}] {-}{-}{-} F}

    G[Mutual Inductance M] {-.{-}{} B}
    G {-.{-}{} E}
{Highlighting}
{Shaded}
\end{verbatim}
\end{center}

બે ઇન્ડક્ટર્સ L_{1} અને L_{2} માટે, મ્યુચ્યુઅલ ઇન્ડક્ટન્સ M છે: M = k\sqrt(L_{1}L_{2})

જ્યાં કપલિંગ ગુણાંક k છે: k = M/\sqrt(L_{1}L_{2})

k એક કોઇલથી બીજી કોઇલ સાથે જોડાતા ચુંબકીય ફ્લક્સના અંશનું પ્રતિનિધિત્વ કરે છે. સંપૂર્ણ
કપલ કોઇલ્સ માટે,

k = 1 કોઈ કપલિંગ નથી ત્યારે,

k = 0


\textbf{સરળ રીત:} ``MKL: મ્યુચ્યુઅલ કપલિંગ K લિંક્સ ઇન્ડક્ટર્સ''

\end{solutionbox}
\subsection*{પ્રશ્ન 4(a) OR [3
marks]}\label{q4a}

\textbf{કપલ સર્કિટ માટેકપ્લીંગના પ્રકારો સમજાવો.}

\begin{solutionbox}

{\def\LTcaptype{none} % do not increment counter
\begin{longtable}[]{@{}
  >{\raggedright\arraybackslash}p{(\linewidth - 4\tabcolsep) * \real{0.3673}}
  >{\raggedright\arraybackslash}p{(\linewidth - 4\tabcolsep) * \real{0.3469}}
  >{\raggedright\arraybackslash}p{(\linewidth - 4\tabcolsep) * \real{0.2857}}@{}}
\toprule\noalign{}
\begin{minipage}[b]{\linewidth}\raggedright
કપલિંગના પ્રકાર
\end{minipage} & \begin{minipage}[b]{\linewidth}\raggedright
લક્ષણો
\end{minipage} & \begin{minipage}[b]{\linewidth}\raggedright
ઉપયોગો
\end{minipage} \\
\midrule\noalign{}
\endhead
\bottomrule\noalign{}
\endlastfoot
\textbf{ટાઇટ/ક્લોઝ કપલિંગ (k\approx1)} & - લગભગ બધો ફ્લક્સ બંને કોઇલ્સને જોડે છે- ઉચ્ચ
ટ્રાન્સફર ક્ષમતા- k મૂલ્ય 1 ની નજીક & ટ્રાન્સફોર્મર્સ, પાવર ટ્રાન્સફર \\
\textbf{લૂઝ કપલિંગ (k≪1)} & - ફ્લક્સનો નાનો અંશ બીજી કોઇલને જોડે છે- ઓછી ટ્રાન્સફર
ક્ષમતા- k મૂલ્ય 1 કરતા ઘણું ઓછું & RF સર્કિટ્સ, ટ્યુન્ડ ફિલ્ટર્સ \\
\textbf{ક્રિટિકલ કપલિંગ (k=kc)} & - બેન્ડપાસ પ્રતિભાવ માટે શ્રેષ્ઠ કપલિંગ- રેઝોનન્સ
પર મહત્તમ પાવર ટ્રાન્સફર & બેન્ડપાસ ફિલ્ટર્સ, IF ટ્રાન્સફોર્મર્સ \\
\textbf{ઇન્ડક્ટિવ કપલિંગ} & - ચુંબકીય ક્ષેત્ર દ્વારા કપલિંગ & ટ્રાન્સફોર્મર્સ, વાયરલેસ
ચાર્જિંગ \\
\textbf{કેપેસિટિવ કપલિંગ} & - વિદ્યુત ક્ષેત્ર દ્વારા કપલિંગ & સિગ્નલ કપલિંગ, કેપેસિટિવ
સેન્સર્સ \\
\end{longtable}
}

\textbf{આકૃતિ:}

\begin{center}
\textbf{Mermaid Diagram (Code)}
\begin{verbatim}
{Shaded}
{Highlighting}[]
graph LR
    subgraph "Tight Coupling"
        direction LR
        A1[Primary] {-{-}{-} B1[k  1] {-}{-}{-} C1[Secondary]}
    end

    subgraph "Loose Coupling"
        direction LR
        A2[Primary] {-{-}{-} B2[k ≪ 1] {-}{-}{-} C2[Secondary]}
    end
    
    subgraph "Critical Coupling"
        direction LR    
        A3[Primary] {-{-}{-} B3[k = kc] {-}{-}{-} C3[Secondary]}
    end
{Highlighting}
{Shaded}
\end{verbatim}
\end{center}

\textbf{સરળ રીત:} ``TLC: ટાઇટ, લૂઝ, ક્રિટિકલ કપલિંગ્સ''

\end{solutionbox}
\subsection*{પ્રશ્ન 4(b) OR [4
marks]}\label{q4b}

\textbf{ગુણવત્તા પરિબળ Q = 100, રેઝોનન્ટ ફ્રિકવન્સી Fr = 100 KHz સાથે 1 mH નું
ઇન્ડક્ટન્સ ધરાવતું સમાંતર રેઝોનન્ટ સર્કિટ. શોધો (i) જરૂરી કેપેસીટન્સ C, (ii) કોઇલનો
પ્રતિકાર R, (iii) BW.}

\begin{solutionbox}

\textbf{આકૃતિ:}

\begin{center}
\textbf{Mermaid Diagram (Code)}
\begin{verbatim}
{Shaded}
{Highlighting}[]
graph LR
    A[Input] {-{-}{-} B((Node))}
    B {-{-}{-} C[L=1mH]}
    B {-{-}{-} D[C=?]}
    B {-{-}{-} E[Output]}
    C {-{-}{-} F[R=?]}
{Highlighting}
{Shaded}
\end{verbatim}
\end{center}

{\def\LTcaptype{none} % do not increment counter
\begin{longtable}[]{@{}
  >{\raggedright\arraybackslash}p{(\linewidth - 6\tabcolsep) * \real{0.2683}}
  >{\raggedright\arraybackslash}p{(\linewidth - 6\tabcolsep) * \real{0.2195}}
  >{\raggedright\arraybackslash}p{(\linewidth - 6\tabcolsep) * \real{0.3171}}
  >{\raggedright\arraybackslash}p{(\linewidth - 6\tabcolsep) * \real{0.1951}}@{}}
\toprule\noalign{}
\begin{minipage}[b]{\linewidth}\raggedright
પેરામીટર
\end{minipage} & \begin{minipage}[b]{\linewidth}\raggedright
સૂત્ર
\end{minipage} & \begin{minipage}[b]{\linewidth}\raggedright
ગણતરી
\end{minipage} & \begin{minipage}[b]{\linewidth}\raggedright
પરિણામ
\end{minipage} \\
\midrule\noalign{}
\endhead
\bottomrule\noalign{}
\endlastfoot
\textbf{કેપેસિટન્સ (C)} & C = 1/(4π^{2}f^{2}L) & 1/(4π^{2}\times(100\times10^{3})^{2}\times1\times10^{-}^{3}) &
2.533 nF \\
\textbf{કોઇલ રેઝિસ્ટન્સ (R)} & R = ωL/Q & 2π\times100\times10^{3}\times1\times10^{-}^{3}/100 & 6.28 Ω \\
\textbf{બેન્ડવિડ્થ (BW)} & BW = fr/Q & 100\times10^{3}/100 & 1 kHz \\
\end{longtable}
}

\textbf{સરળ રીત:} ``RCB: રેઝોનન્સ નીડ્સ કેપેસિટન્સ એન્ડ બેન્ડવિડ્થ''

\end{solutionbox}
\subsection*{પ્રશ્ન 4(c) OR [7
marks]}\label{q4c}

\textbf{series RLC સર્કિટની Band width અને Selectivity સમજાવો. શ્રેણી રેઝોનન્સ
સર્કિટ માટે Q પરિબળ અને BW વચ્ચેનો સંબંધ પણ સ્થાપિત કરો.}

\begin{solutionbox}

{\def\LTcaptype{none} % do not increment counter
\begin{longtable}[]{@{}
  >{\raggedright\arraybackslash}p{(\linewidth - 4\tabcolsep) * \real{0.2973}}
  >{\raggedright\arraybackslash}p{(\linewidth - 4\tabcolsep) * \real{0.3243}}
  >{\raggedright\arraybackslash}p{(\linewidth - 4\tabcolsep) * \real{0.3784}}@{}}
\toprule\noalign{}
\begin{minipage}[b]{\linewidth}\raggedright
પેરામીટર
\end{minipage} & \begin{minipage}[b]{\linewidth}\raggedright
વ્યાખ્યા
\end{minipage} & \begin{minipage}[b]{\linewidth}\raggedright
સંબંધ
\end{minipage} \\
\midrule\noalign{}
\endhead
\bottomrule\noalign{}
\endlastfoot
\textbf{બેન્ડવિડ્થ (BW)} & હાફ-પાવર પોઇન્ટ્સ વચ્ચેનો ફ્રીક્વન્સી રેન્જ & BW = f_{2} -
f_{1} = ω_{2} - ω_{1} = R/L \\
\textbf{સિલેક્ટિવિટી} & વિવિધ ફ્રીક્વન્સીઓના સિગ્નલ્સને અલગ કરવાની ક્ષમતા & BW
સાથે વ્યસ્ત પ્રમાણમાં \\
\textbf{Q ફેક્ટર} & રેઝોનન્ટ ફ્રીક્વન્સીનો બેન્ડવિડ્થ સાથેનો ગુણોત્તર & Q = ω_{0}/BW =
ω_{0}L/R \\
\end{longtable}
}

\textbf{આકૃતિ:}

\begin{center}
\textbf{Mermaid Diagram (Code)}
\begin{verbatim}
{Shaded}
{Highlighting}[]
graph LR
    A[Input] {-{-}{-} B[R] {-}{-}{-} C[L] {-}{-}{-} D[C] {-}{-}{-} E[Output]}

    subgraph "Frequency Response"
        F[Amplitude] {-{-}{-} G[f_{0}]}
        H[BW = f_{2 {-} f_{1}] {-}.{-}{} G}
    end
{Highlighting}
{Shaded}
\end{verbatim}
\end{center}

સિરીઝ RLC સર્કિટ માટે:

\begin{itemize}
\tightlist
\item
  રેઝોનન્સ (f_{0}) પર, ઇમ્પીડન્સ ન્યૂનતમ છે (= R)
\item
  હાફ-પાવર પોઇન્ટ્સ ત્યારે આવે છે જ્યારે ઇમ્પીડન્સ = \sqrt2\timesR
\item
  આ બિંદુઓ પર, પાવર મહત્તમ પાવરનો અડધો હોય છે
\end{itemize}

બેન્ડવિડ્થ (BW) = ω_{2} - ω_{1} = R/L Q ફેક્ટર = ω_{0}L/R = ω_{0}/BW

તેથી, BW = ω_{0}/Q = 2πf_{0}/Q

આ દર્શાવે છે કે Q ફેક્ટર અને બેન્ડવિડ્થ વ્યસ્ત રીતે સંબંધિત છે: ઉચ્ચ Q \rightarrow સાંકડી બેન્ડવિડ્થ \rightarrow
વધુ સારી સિલેક્ટિવિટી

\textbf{સરળ રીત:} ``BQS: બેન્ડવિડ્થ અને Q નક્કી કરે છે સિલેક્ટિવિટી''

\end{solutionbox}
\subsection*{પ્રશ્ન 5(a) [3
marks]}\label{q5a}

\textbf{40 ડીબીનું એટેન્યુએશન આપવા અને 300 Ω પ્રતિકારના લોડમાં કામ કરવા માટે
સપ્રમાણ T પ્રકારના એટેન્યુએટરને ડિઝાઇન કરો.}

\begin{solutionbox}

\textbf{આકૃતિ:}

\begin{center}
\textbf{Mermaid Diagram (Code)}
\begin{verbatim}
{Shaded}
{Highlighting}[]
graph LR
    A[Input] {-{-}{-} B[Z_{1}/2] {-}{-}{-} C((Node)) {-}{-}{-} D[Z_{1}/2] {-}{-}{-} E[Output]}
    C {-{-}{-} F[Z_{2}] {-}{-}{-} G[Ground]}
    H[300Ω] {-.{-}{} E}
{Highlighting}
{Shaded}
\end{verbatim}
\end{center}

{\def\LTcaptype{none} % do not increment counter
\begin{longtable}[]{@{}llll@{}}
\toprule\noalign{}
પેરામીટર & સૂત્ર & ગણતરી & પરિણામ \\
\midrule\noalign{}
\endhead
\bottomrule\noalign{}
\endlastfoot
\textbf{એટેન્યુએશન (N)} & N = 10\^{}(dB/20) & 10\^{}(40/20) & 100 \\
\textbf{ઇમ્પીડન્સ રેશિયો (K)} & K = (N+1)/(N-1) & (100+1)/(100-1) & 1.02 \\
\textbf{Z_{1}} & Z_{1} = R_{0}[(K-1)/K] & 300[(1.02-1)/1.02] & 5.88 Ω \\
\textbf{Z_{2}} & Z_{2} = R_{0}[2K/(K^{2}-1)] & 300[2\times1.02/(1.02^{2}-1)] &
594.12 Ω \\
\end{longtable}
}

\textbf{સરળ રીત:} ``TANZ: T-એટેન્યુએટર નીડ્સ Z-પેરામીટર્સ''

\end{solutionbox}
\subsection*{પ્રશ્ન 5(b) [4
marks]}\label{q5b}

\textbf{ફિલ્ટર્સનું વર્ગીકરણ આપો.}

\begin{solutionbox}

{\def\LTcaptype{none} % do not increment counter
\begin{longtable}[]{@{}
  >{\raggedright\arraybackslash}p{(\linewidth - 4\tabcolsep) * \real{0.4000}}
  >{\raggedright\arraybackslash}p{(\linewidth - 4\tabcolsep) * \real{0.1750}}
  >{\raggedright\arraybackslash}p{(\linewidth - 4\tabcolsep) * \real{0.4250}}@{}}
\toprule\noalign{}
\begin{minipage}[b]{\linewidth}\raggedright
વર્ગીકરણ
\end{minipage} & \begin{minipage}[b]{\linewidth}\raggedright
પ્રકારો
\end{minipage} & \begin{minipage}[b]{\linewidth}\raggedright
લક્ષણો
\end{minipage} \\
\midrule\noalign{}
\endhead
\bottomrule\noalign{}
\endlastfoot
\textbf{ફ્રીક્વન્સી રિસ્પોન્સ આધારિત} & - લો પાસ- હાઇ પાસ- બેન્ડ પાસ- બેન્ડ સ્ટોપ
& - કટઓફ નીચેની ફ્રીક્વન્સી પસાર કરે- કટઓફ ઉપરની ફ્રીક્વન્સી પસાર કરે- બેન્ડની
અંદરની ફ્રીક્વન્સી પસાર કરે- બેન્ડની અંદરની ફ્રીક્વન્સી અવરોધે \\
\textbf{ઘટકો આધારિત} & - પેસિવ ફિલ્ટર્સ- એક્ટિવ ફિલ્ટર્સ & - R, L, C ઘટકોનો
ઉપયોગ- RC સાથે એક્ટિવ ડિવાઇસનો ઉપયોગ \\
\textbf{ડિઝાઇન અભિગમ આધારિત} & - કન્સ્ટન્ટ-k ફિલ્ટર્સ- m-ડેરાઇવ્ડ ફિલ્ટર્સ-
કમ્પોઝિટ ફિલ્ટર્સ & - સરળતમ ડિઝાઇન- વધુ સારા કટઓફ લક્ષણો- ફાયદાઓનું સંયોજન \\
\textbf{ટેકનોલોજી આધારિત} & - LC ફિલ્ટર્સ- ક્રિસ્ટલ ફિલ્ટર્સ- સેરામિક ફિલ્ટર્સ-
ડિજિટલ ફિલ્ટર્સ & - ઇન્ડક્ટર અને કેપેસિટરનો ઉપયોગ- પિઝોઇલેક્ટ્રિક ક્રિસ્ટલનો ઉપયોગ-
પિઝોઇલેક્ટ્રિક સેરામિકનો ઉપયોગ- સોફ્ટવેરમાં અમલીકરણ \\
\end{longtable}
}

\textbf{આકૃતિ:}

\begin{center}
\textbf{Mermaid Diagram (Code)}
\begin{verbatim}
{Shaded}
{Highlighting}[]
graph TD
    A[Filters] {-{-}{} B[Frequency Response]}
    A {-{-}{} C[Components]}
    A {-{-}{} D[Design Approach]}
    A {-{-}{} E[Technology]}

    B {-{-}{} F[Low Pass]}
    B {-{-}{} G[High Pass]}
    B {-{-}{} H[Band Pass]}
    B {-{-}{} I[Band Stop]}
{Highlighting}
{Shaded}
\end{verbatim}
\end{center}

\textbf{સરળ રીત:} ``FLAC: ફિલ્ટર્સ: લો-પાસ, એક્ટિવ, કન્સ્ટન્ટ-k''

\end{solutionbox}
\subsection*{પ્રશ્ન 5(c) [7
marks]}\label{q5c}

\textbf{constant K લો પાસ ફિલ્ટર સમજાવો.}

\begin{solutionbox}

{\def\LTcaptype{none} % do not increment counter
\begin{longtable}[]{@{}
  >{\raggedright\arraybackslash}p{(\linewidth - 2\tabcolsep) * \real{0.4091}}
  >{\raggedright\arraybackslash}p{(\linewidth - 2\tabcolsep) * \real{0.5909}}@{}}
\toprule\noalign{}
\begin{minipage}[b]{\linewidth}\raggedright
વિભાવના
\end{minipage} & \begin{minipage}[b]{\linewidth}\raggedright
વર્ણન
\end{minipage} \\
\midrule\noalign{}
\endhead
\bottomrule\noalign{}
\endlastfoot
\textbf{વ્યાખ્યા} & ફિલ્ટર જેમાં ઇમ્પીડન્સ પ્રોડક્ટ Z_{1}Z_{2} = k^{2} (અચળ) દરેક ફ્રીક્વન્સી
પર \\
\textbf{સર્કિટ પ્રકાર} & T-સેક્શન અને π-સેક્શન \\
\textbf{T-સેક્શન ઘટકો} & સિરીઝ ઇન્ડક્ટર્સ (L/2) અને શન્ટ કેપેસિટર (C) \\
\textbf{π-સેક્શન ઘટકો} & સિરીઝ ઇન્ડક્ટર (L) અને શન્ટ કેપેસિટર્સ (C/2) \\
\textbf{કટઓફ ફ્રીક્વન્સી} & fc = 1/π\sqrt(LC) \\
\textbf{કેરેક્ટરિસ્ટિક ઇમ્પીડન્સ} & R_{0} = \sqrt(L/C) \\
\end{longtable}
}

\textbf{આકૃતિ:}

\begin{center}
\textbf{Mermaid Diagram (Code)}
\begin{verbatim}
{Shaded}
{Highlighting}[]
graph TD
    subgraph "T{-section"}
        A1[Input] {-{-}{-} B1[L/2] {-}{-}{-} C1((Node)) {-}{-}{-} D1[L/2] {-}{-}{-} E1[Output]}
        C1 {-{-}{-} F1[C] {-}{-}{-} G1[Ground]}
    end

    subgraph "π{-section"}
        A2[Input] {-{-}{-} B2((Node))}
        B2 {-{-}{-} C2[C/2] {-}{-}{-} G2[Ground]}
        B2 {-{-}{-} D2[L] {-}{-}{-} E2((Node)) {-}{-}{-} F2[Output]}
        E2 {-{-}{-} H2[C/2] {-}{-}{-} G2}
    end
{Highlighting}
{Shaded}
\end{verbatim}
\end{center}

કન્સ્ટન્ટ-k લો પાસ ફિલ્ટરના લક્ષણો:

\begin{itemize}
\tightlist
\item
  કટઓફ ફ્રીક્વન્સી: fc = 1/π\sqrt(LC)
\item
  ડિઝાઇન ઇમ્પીડન્સ: R_{0} = \sqrt(L/C)
\item
  પાસ બેન્ડ: 0 થી fc
\item
  એટેન્યુએશન બેન્ડ: fc ઉપર
\item
  પાસ બેન્ડથી સ્ટોપ બેન્ડ સુધી ક્રમશઃ સંક્રમણ
\end{itemize}

\textbf{સરળ રીત:} ``CLPT: કન્સ્ટન્ટ-k લો પાસ નીડ્સ T-સેક્શન''

\end{solutionbox}
\subsection*{પ્રશ્ન 5(a) OR [3
marks]}\label{q5a}

\textbf{400 Ω ના લોડ પ્રતિકાર સાથે 1.5 KHz ની કટ-ઓફ આવર્તન ધરાવતા T વિભાગ
સાથે ઉચ્ચ પાસ ફિલ્ટર ડિઝાઇન કરો.}

\begin{solutionbox}

\textbf{આકૃતિ:}

\begin{center}
\textbf{Mermaid Diagram (Code)}
\begin{verbatim}
{Shaded}
{Highlighting}[]
graph LR
    A[Input] {-{-}{-} B[C/2] {-}{-}{-} C((Node)) {-}{-}{-} D[C/2] {-}{-}{-} E[Output]}
    C {-{-}{-} F[L] {-}{-}{-} G[Ground]}
    H[400Ω] {-.{-}{} E}
{Highlighting}
{Shaded}
\end{verbatim}
\end{center}

{\def\LTcaptype{none} % do not increment counter
\begin{longtable}[]{@{}llll@{}}
\toprule\noalign{}
પેરામીટર & સૂત્ર & ગણતરી & પરિણામ \\
\midrule\noalign{}
\endhead
\bottomrule\noalign{}
\endlastfoot
\textbf{ડિઝાઇન ઇમ્પીડન્સ (R_{0})} & R_{0} = લોડ રેઝિસ્ટન્સ & આપેલ & 400 Ω \\
\textbf{કટઓફ ફ્રીક્વન્સી (fc)} & fc = આપેલ & આપેલ & 1.5 kHz \\
\textbf{ઇન્ડક્ટર (L)} & L = R_{0}/2πfc & 400/(2π\times1500) & 42.44 mH \\
\textbf{કેપેસિટર (C)} & C = 1/(2πfcR_{0}) & 1/(2π\times1500\times400) & 0.265 µF \\
\end{longtable}
}

\textbf{સરળ રીત:} ``HCL: હાઇ-પાસ નીડ્સ કેપેસિટર એન્ડ ઇન્ડક્ટર''

\end{solutionbox}
\subsection*{પ્રશ્ન 5(b) OR [4
marks]}\label{q5b}

\textbf{એટેન્યુએટરનું વર્ગીકરણ આપો.}

\begin{solutionbox}

{\def\LTcaptype{none} % do not increment counter
\begin{longtable}[]{@{}
  >{\raggedright\arraybackslash}p{(\linewidth - 4\tabcolsep) * \real{0.4000}}
  >{\raggedright\arraybackslash}p{(\linewidth - 4\tabcolsep) * \real{0.1750}}
  >{\raggedright\arraybackslash}p{(\linewidth - 4\tabcolsep) * \real{0.4250}}@{}}
\toprule\noalign{}
\begin{minipage}[b]{\linewidth}\raggedright
વર્ગીકરણ
\end{minipage} & \begin{minipage}[b]{\linewidth}\raggedright
પ્રકારો
\end{minipage} & \begin{minipage}[b]{\linewidth}\raggedright
લક્ષણો
\end{minipage} \\
\midrule\noalign{}
\endhead
\bottomrule\noalign{}
\endlastfoot
\textbf{કન્ફિગરેશન આધારિત} & - T-એટેન્યુએટર- π-એટેન્યુએટર- બ્રિજ્ડ-T- લેટિસ & -
સિરીઝ-શન્ટ-સિરીઝ- શન્ટ-સિરીઝ-શન્ટ- બેલેન્સ્ડ બ્રિજ- બેલેન્સ્ડ નેટવર્ક \\
\textbf{સિમેટ્રી આધારિત} & - સિમેટ્રિકલ- એસિમેટ્રિકલ & - સમાન ઇમ્પીડન્સ- અસમાન
ઇમ્પીડન્સ \\
\textbf{નિયંત્રણ આધારિત} & - ફિક્સ્ડ- વેરિએબલ- પ્રોગ્રામેબલ & - અચળ એટેન્યુએશન-
સમાયોજ્ય એટેન્યુએશન- ડિજિટલી નિયંત્રિત \\
\textbf{ટેકનોલોજી આધારિત} & - રેઝિસ્ટિવ- રિએક્ટિવ- એક્ટિવ & - રેઝિસ્ટરનો ઉપયોગ-
રિએક્ટન્સનો ઉપયોગ- એક્ટિવ ડિવાઇસનો ઉપયોગ \\
\end{longtable}
}

\textbf{આકૃતિ:}

\begin{center}
\textbf{Mermaid Diagram (Code)}
\begin{verbatim}
{Shaded}
{Highlighting}[]
graph TD
    A[Attenuators] {-{-}{} B[Configuration]}
    A {-{-}{} C[Symmetry]}
    A {-{-}{} D[Control]}
    A {-{-}{} E[Technology]}

    B {-{-}{} F[T{-}type]}
    B {-{-}{} G[π{-}type]}
    B {-{-}{} H[Bridged{-}T]}
    B {-{-}{} I[Lattice]}
{Highlighting}
{Shaded}
\end{verbatim}
\end{center}

\textbf{સરળ રીત:} ``CAST: કન્ફિગરેશન, એડજસ્ટેબલ, સિમેટ્રી, ટેકનોલોજી''

\end{solutionbox}
\subsection*{પ્રશ્ન 5(c) OR [7
marks]}\label{q5c}

\textbf{constant K હાઇ પાસ ફિલ્ટર સમજાવો.}

\begin{solutionbox}

{\def\LTcaptype{none} % do not increment counter
\begin{longtable}[]{@{}ll@{}}
\toprule\noalign{}
વિભાવના & વર્ણન \\
\midrule\noalign{}
\endhead
\bottomrule\noalign{}
\endlastfoot
\textbf{વ્યાખ્યા} & કટઓફ ઉપરની ફ્રીક્વન્સી પસાર કરતું ફિલ્ટર, જેમાં Z_{1}Z_{2} = k^{2}
(અચળ) \\
\textbf{સર્કિટ પ્રકાર} & T-સેક્શન અને π-સેક્શન \\
\textbf{T-સેક્શન ઘટકો} & સિરીઝ કેપેસિટર્સ (C/2) અને શન્ટ ઇન્ડક્ટર (L) \\
\textbf{π-સેક્શન ઘટકો} & સિરીઝ કેપેસિટર (C) અને શન્ટ ઇન્ડક્ટર્સ (L/2) \\
\textbf{કટઓફ ફ્રીક્વન્સી} & fc = 1/π\sqrt(LC) \\
\textbf{કેરેક્ટરિસ્ટિક ઇમ્પીડન્સ} & R_{0} = \sqrt(L/C) \\
\end{longtable}
}

\textbf{આકૃતિ:}

\begin{center}
\textbf{Mermaid Diagram (Code)}
\begin{verbatim}
{Shaded}
{Highlighting}[]
graph TD
    subgraph "T{-section"}
        A1[Input] {-{-}{-} B1[C/2] {-}{-}{-} C1((Node)) {-}{-}{-} D1[C/2] {-}{-}{-} E1[Output]}
        C1 {-{-}{-} F1[L] {-}{-}{-} G1[Ground]}
    end

    subgraph "π{-section"}
        A2[Input] {-{-}{-} B2((Node))}
        B2 {-{-}{-} C2[L/2] {-}{-}{-} G2[Ground]}
        B2 {-{-}{-} D2[C] {-}{-}{-} E2((Node)) {-}{-}{-} F2[Output]}
        E2 {-{-}{-} H2[L/2] {-}{-}{-} G2}
    end
{Highlighting}
{Shaded}
\end{verbatim}
\end{center}

કન્સ્ટન્ટ-k હાઇ પાસ ફિલ્ટરના લક્ષણો:

\begin{itemize}
\tightlist
\item
  કટઓફ ફ્રીક્વન્સી: fc = 1/π\sqrt(LC)
\item
  ડિઝાઇન ઇમ્પીડન્સ: R_{0} = \sqrt(L/C)
\item
  પાસ બેન્ડ: fc ઉપર
\item
  એટેન્યુએશન બેન્ડ: 0 થી fc
\item
  પાસ બેન્ડથી સ્ટોપ બેન્ડ સુધી ક્રમશઃ સંક્રમણ
\item
  ઘટક મૂલ્યો લો પાસ ફિલ્ટરના ડ્યુઅલ છે (L અને C જગ્યા બદલે છે)
\end{itemize}

\textbf{સરળ રીત:} ``CHTS: કન્સ્ટન્ટ-k હાઇ-પાસ યુઝિસ T-સેક્શન''

\end{solutionbox}

\end{document}
