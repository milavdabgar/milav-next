\documentclass{article}

% content/resources/templates/preamble.tex
\usepackage[margin=0.6in]{geometry}
\author{Milav Dabgar}
\usepackage{amsmath,amssymb,amsthm}
\usepackage{booktabs}
\usepackage{multirow}
\usepackage{xcolor}
\usepackage{tcolorbox}
\tcbuselibrary{breakable,skins}
\usepackage[colorlinks=true,linkcolor=blue]{hyperref}
\usepackage{titlesec}
\usepackage{enumitem}
\usepackage{tikz}
\usepackage{pgfplots}
\usepackage{circuitikz}
\usepackage[version=4]{mhchem}
\usepackage{longtable}
\usepackage{array}
\usepackage{float}
\usepackage{caption}
\usepackage{listings}

\lstset{
  basicstyle=\small\ttfamily,
  breaklines=true,
  breakatwhitespace=false,
  postbreak=\mbox{\textcolor{red}{$\hookrightarrow$}\space},
  float=false,
  numbers=left,
  numberstyle=\tiny\color{gray},
  numbersep=10pt,
  xleftmargin=2em,
  keywordstyle=\color{blue},
  commentstyle=\color{green!60!black},
  stringstyle=\color{purple},
  backgroundcolor=\color{gray!5},
  showstringspaces=false,
  tabsize=2,
  captionpos=b,
  keepspaces=true,
  columns=flexible
}

\pgfplotsset{compat=1.18}
\usetikzlibrary{shapes,arrows,positioning,calc,patterns,decorations.pathmorphing,decorations.markings,arrows.meta}

% Color scheme
\definecolor{headcolor}{RGB}{0,102,204}
\definecolor{keycolor}{RGB}{220,20,60}
\definecolor{solutioncolor}{RGB}{34,139,34}
\definecolor{mnemoniccolor}{RGB}{148,0,211}
\definecolor{codecolor}{RGB}{0,0,100}

% Spacing
\setlength{\parskip}{3pt}
\setlist[itemize]{nosep}
\setlist[enumerate]{nosep}

% Title formatting
\titleformat{\section}{\Large\bfseries\color{headcolor}}{\thesection}{1em}{}
\titleformat{\subsection}{\large\bfseries\color{headcolor}}{\thesubsection}{1em}{}

% Pandoc tightlist compatibility
\providecommand{\tightlist}{%
  \setlength{\itemsep}{0pt}\setlength{\parskip}{0pt}}

% Pandoc longtable compatibility
\newcounter{none}
\def\thenone{}


% content/resources/templates/gujarati-boxes.tex
\usepackage{fontspec}
\usepackage{polyglossia}

% Set Gujarati as main language (document is primarily in Gujarati)
% Note: gloss-gujarati.ldf doesn't exist in polyglossia, but it will use hyphenation patterns
\setdefaultlanguage{gujarati}
\setotherlanguage{english}

% Configure Gujarati font properly
% Use Language=Default to prevent polyglossia from trying to add language-specific features
% that don't exist for Gujarati, which causes "empty feature" warnings
\newfontfamily\gujaratifont[Script=Gujarati,AutoFakeBold=2.5,AutoFakeSlant=0.3]{Noto Sans Gujarati}
\setmainfont[Script=Gujarati,AutoFakeBold=2.5,AutoFakeSlant=0.3]{Noto Sans Gujarati}
% Use Noto Sans Gujarati for monospace to support Gujarati in text
\setmonofont[Scale=0.9]{Noto Sans Gujarati}

% Configure English to use the same font
\newfontfamily\englishfont[Script=Gujarati,AutoFakeBold=2.5,AutoFakeSlant=0.3]{Noto Sans Gujarati}

% Translations for polyglossia
\gappto\captionsgujarati{
  \renewcommand{\tablename}{કોષ્ટક}
  \renewcommand{\figurename}{આકૃતિ}
}

% Helper for TikZ nodes to ensure Gujarati font
\newcommand{\gu}[1]{{\gujaratifont #1}}

% Custom environments
\newtcolorbox{solutionbox}{
    breakable,
    enhanced,
    colback=solutioncolor!5!white,
    colframe=solutioncolor!75!black,
    fonttitle=\bfseries,
    title=જવાબ
}

\newtcolorbox{solutionboxnobreak}{
 colback=solutioncolor!5!white,
 colframe=solutioncolor!75!black,
 fonttitle=\bfseries,
 title=જવાબ
}

\newtcolorbox{keyformula}{
 breakable,
 enhanced,
 colback=keycolor!5!white,
 colframe=keycolor!75!black,
 fonttitle=\bfseries,
 title=રાસાયણિક સમીકરણ/સૂત્ર
}

\newtcolorbox{mnemonicbox}{
 breakable,
 enhanced,
 colback=mnemoniccolor!5!white,
 colframe=mnemoniccolor!75!black,
 fonttitle=\bfseries,
 title=મેમરી ટ્રીક
}


% Custom commands for GTU solutions
% This file defines semantic commands for consistent formatting

% Question command with automatic formatting
\newcommand{\question}[2]{%
  \section*{Question #1}%
  \textbf{#2}%
}

% OR question variant
\newcommand{\questionor}[2]{%
  \section*{Question #1 OR}%
  \textbf{#2}%
}

% Proper table environment with caption
\newenvironment{answertable}[1]{%
  \begin{table}[htbp]
  \centering
  \caption{#1}
}{%
  \end{table}
}

% Proper figure environment for diagrams
\newenvironment{answerdiagram}[1]{%
  \begin{figure}[htbp]
  \centering
  \caption{#1}
}{%
  \end{figure}
}

% Semantic markup for key terms
\newcommand{\keyword}[1]{\textbf{#1}}
\newcommand{\code}[1]{\texttt{#1}}
\newcommand{\classname}[1]{\texttt{#1}}
\newcommand{\methodname}[1]{\texttt{#1}}

% Proper quotation marks
\newcommand{\mnemonic}[1]{``#1''}


\title{પ્રિન્સિપલ ઓફ ઇલેક્ટ્રોનિક કમ્યુનિકેશન (4331104) - સમર 2025 સોલ્યુશન}
\date{May 17, 2025}

\begin{document}
\maketitle

\questionmarks{1}{a}{3}
\textbf{એનાલોગ સિગ્નલ અને ડિજિટલ સિગ્નલની સરખામણી કરો.}

\begin{solutionbox}
    \textbf{જવાબ}:

    \begin{center}
    \begin{tabulary}{\linewidth}{L L L}
        \toprule
        \textbf{પેરામીટર} & \textbf{એનાલોગ સિગ્નલ} & \textbf{ડિજિટલ સિગ્નલ} \\
        \midrule
        \textbf{પ્રકૃતિ} & સતત તરંગરૂપ & અલગ અલગ વેલ્યુ (0 અને 1) \\
        \textbf{એમ્પ્લિટ્યુડ} & અનંત વિવિધતાઓ & નિશ્ચિત અલગ સ્તરો \\
        \textbf{નોઇઝ ઇફેક્ટ} & વધુ સંવેદનશીલ & ઓછી સંવેદનશીલ \\
        \textbf{બેન્ડવિડ્થ} & ઓછી બેન્ડવિડ્થ જરૂરી & વધુ બેન્ડવિડ્થ જરૂરી \\
        \textbf{સિક્યુરિટી} & ઓછી સુરક્ષિત & વધુ સુરક્ષિત \\
        \bottomrule
    \end{tabulary}
    \end{center}

    \begin{itemize}
        \item \textbf{સિગ્નલ પ્રકાર}: એનાલોગ સિગ્નલ સતત હોય છે, ડિજિટલ સિગ્નલ અલગ અલગ હોય છે.
        \item \textbf{નોઇઝ રેઝિસ્ટન્સ}: ડિજિટલ સિગ્નલમાં નોઇઝ સામે વધુ પ્રતિકાર હોય છે.
    \end{itemize}

    \begin{mnemonicbox}
    "ABCD - Analog Bad for noise, Continuous; Digital Discrete, Clean signals"
    \end{mnemonicbox}
\end{solutionbox}

\questionmarks{1}{b}{4}
\textbf{PAM, PWM અને PPM ની સરખામણી કરો.}

\begin{solutionbox}
    \textbf{જવાબ}:

    \begin{center}
    \begin{tabulary}{\linewidth}{L L L L}
        \toprule
        \textbf{પેરામીટર} & \textbf{PAM} & \textbf{PWM} & \textbf{PPM} \\
        \midrule
        \textbf{પૂરું નામ} & Pulse Amplitude Modulation & Pulse Width Modulation & Pulse Position Modulation \\
        \textbf{મોડ્યુલેટેડ પેરામીટર} & એમ્પ્લિટ્યુડ & પહોળાઈ/અવધિ & સ્થાન/સમય \\
        \textbf{નોઇઝ ઇમ્યુનિટી} & ખરાબ & સારી & ઉત્તમ \\
        \textbf{બેન્ડવિડ્થ} & લઘુત્તમ & મધ્યમ & મહત્તમ \\
        \textbf{પાવર કન્ઝમ્પશન} & વધુ & મધ્યમ & ઓછી \\
        \bottomrule
    \end{tabulary}
    \end{center}

    \textbf{ડાયાગ્રામ:}

    \begin{center}
    \begin{tikzpicture}[scale=0.8]
        % PAM
        \node [left] at (0, 7) {PAM};
        \draw [->] (0, 6) -- (10, 6) node[right] {$t$};
        \foreach \x in {1,2,3,4,5,6,7,8,9} {
             \pgfmathsetmacro{\yval}{6 + 0.5 + 0.5*sin(\x*50)}
             \draw [thick, blue] (\x, 6) -- (\x, \yval);
        }
        
        % PWM
        \node [left] at (0, 4) {PWM};
        \draw [->] (0, 3) -- (10, 3) node[right] {$t$};
        \foreach \x in {1,2,3,4,5,6,7,8,9} {
            \pgfmathsetmacro{\width}{0.2 + 0.15*sin(\x*50)}
            \draw [thick, red, fill=red!20] (\x-0.2, 3) rectangle (\x+\width, 4);
        }
            
        % PPM
        \node [left] at (0, 1) {PPM};
        \draw [->] (0, 0) -- (10, 0) node[right] {$t$};
        \foreach \x in {1,2,3,4,5,6,7,8,9} {
            \pgfmathsetmacro{\shift}{0.3*sin(\x*50)}
            \draw [thick, orange] (\x+\shift, 0) -- (\x+\shift, 1);
        }
    \end{tikzpicture}
    \end{center}

    \begin{itemize}
        \item \textbf{મોડ્યુલેશન પેરામીટર}: દરેક પ્રકાર પલ્સની અલગ લાક્ષણિકતાઓ મોડ્યુલેટ કરે છે.
        \item \textbf{એપ્લિકેશન}: PWM મોટર કંટ્રોલમાં, PPM રેડિયો કંટ્રોલ સિસ્ટમમાં વપરાય છે.
    \end{itemize}

    \begin{mnemonicbox}
    "PAM-Amplitude, PWM-Width, PPM-Position - AWP"
    \end{mnemonicbox}
\end{solutionbox}

\questionmarks{1}{c}{7}
\textbf{મોડ્યુલેશનની જરૂરિયાત વિગતવાર સમજાવો. જો કેરિયર સિગ્નલની આવૃત્તિ 1 MHz હોય તો એન્ટેનાની ઊંચાઈની ગણતરી કરો.}

\begin{solutionbox}
    \textbf{જવાબ}:

    \textbf{મોડ્યુલેશનની જરૂરિયાત:}

    \begin{center}
    \begin{tabulary}{\linewidth}{L L}
        \toprule
        \textbf{કારણ} & \textbf{સમજૂતી} \\
        \midrule
        \textbf{એન્ટેના સાઇઝ રિડક્શન} & વ્યવહારિક એન્ટેના માપ શક્ય બનાવે છે \\
        \textbf{ફ્રીક્વન્સી ટ્રાન્સલેશન} & સિગ્નલને યોગ્ય આવૃત્તિ રેન્જમાં ખસેડે છે \\
        \textbf{મલ્ટિપ્લેક્સિંગ} & એક જ માધ્યમ પર અનેક સિગ્નલ મંજૂરી આપે છે \\
        \textbf{નોઇઝ રિડક્શન} & સિગ્નલ-ટુ-નોઇઝ રેશિયો સુધારે છે \\
        \textbf{પાવર એફિશિયન્સી} & વધુ સારી પાવર વિનિયોગ \\
        \bottomrule
    \end{tabulary}
    \end{center}

    \textbf{એન્ટેના ઊંચાઈની ગણતરી:} \\
    કાર્યક્ષમ રેડિએશન માટે, એન્ટેના ઊંચાઈ = $\lambda/4$

    \[ \lambda = \frac{c}{f} = \frac{3 \times 10^8}{1 \times 10^6} = 300 \text{ મીટર} \]

    \[ \textbf{એન્ટેના ઊંચાઈ} = \frac{\lambda}{4} = \frac{300}{4} = \textbf{75 મીટર} \]

    \begin{itemize}
        \item \textbf{પ્રેક્ટિકલ એન્ટેના}: મોડ્યુલેશન વગર, એન્ટેના અવ્યવહારિક રીતે મોટો હોત.
        \item \textbf{ફ્રીક્વન્સી શિફ્ટિંગ}: વધુ સારી પ્રોપેગેશન લાક્ષણિકતાઓ માટે મંજૂરી આપે છે.
    \end{itemize}

    \begin{mnemonicbox}
    "AFMNP - Antenna, Frequency, Multiplexing, Noise, Power"
    \end{mnemonicbox}
\end{solutionbox}

\orquestions

\questionmarks{1}{c}{7}
\textbf{EM વેવ સ્પેક્ટ્રમના ફ્રીક્વન્સી બેન્ડ તેના એપ્લિકેશન ડોમેન સાથે લખો. ELF બેન્ડની તરંગલંબાઈની ગણતરી કરો.}

\begin{solutionbox}
    \textbf{જવાબ}:

    \begin{center}
    \begin{tabulary}{\linewidth}{L L L L}
        \toprule
        \textbf{બેન્ડ} & \textbf{આવૃત્તિ રેન્જ} & \textbf{તરંગલંબાઈ} & \textbf{એપ્લિકેશન} \\
        \midrule
        \textbf{ELF} & 30-300 Hz & $10^6-10^7$ m & સબમરીન કમ્યુનિકેશન \\
        \textbf{VLF} & 3-30 kHz & $10^4-10^5$ m & નેવિગેશન, ટાઇમ સિગ્નલ \\
        \textbf{LF} & 30-300 kHz & $10^3-10^4$ m & AM બ્રોડકાસ્ટિંગ \\
        \textbf{MF} & 300 kHz-3 MHz & 100-1000 m & AM રેડિયો \\
        \textbf{HF} & 3-30 MHz & 10-100 m & શોર્ટ વેવ રેડિયો \\
        \bottomrule
    \end{tabulary}
    \end{center}

    \textbf{ELF તરંગલંબાઈની ગણતરી:}

    \begin{itemize}
        \item નીચી આવૃત્તિ: $f_1 = 30 \text{ Hz}, \lambda_1 = c/f_1 = (3 \times 10^8)/30 = \textbf{10}^7 \textbf{ મીટર}$
        \item ઉચ્ચી આવૃત્તિ: $f_2 = 300 \text{ Hz}, \lambda_2 = c/f_2 = (3 \times 10^8)/300 = \textbf{10}^6 \textbf{ મીટર}$
    \end{itemize}

    \textbf{ELF તરંગલંબાઈ રેન્જ: $10^6$ થી $10^7$ મીટર}

    \begin{itemize}
        \item \textbf{એપ્લિકેશન ડોમેન}: દરેક બેન્ડ ચોક્કસ એપ્લિકેશન માટે યોગ્ય છે.
        \item \textbf{પ્રોપેગેશન}: નીચી આવૃત્તિઓમાં વધુ સારી ગ્રાઉન્ડ વેવ પ્રોપેગેશન હોય છે.
    \end{itemize}

    \begin{mnemonicbox}
    "Every Valuable Learning Makes Happiness - ELF to HF bands"
    \end{mnemonicbox}
\end{solutionbox}

\questionmarks{2}{a}{3}
\textbf{AM અને FM ની સરખામણી કરો.}

\begin{solutionbox}
    \textbf{જવાબ}:

    \begin{center}
    \begin{tabulary}{\linewidth}{L L L}
        \toprule
        \textbf{પેરામીટર} & \textbf{AM} & \textbf{FM} \\
        \midrule
        \textbf{મોડ્યુલેટેડ પેરામીટર} & એમ્પ્લિટ્યુડ & આવૃત્તિ \\
        \textbf{બેન્ડવિડ્થ} & 2$f_m$ & $2(\Delta f + f_m)$ \\
        \textbf{નોઇઝ ઇમ્યુનિટી} & ખરાબ & સારી \\
        \textbf{પાવર એફિશિયન્સી} & ઓછી (33.33\%) & વધુ \\
        \textbf{સર્કિટ કોમ્પ્લેક્સિટી} & સરળ & જટિલ \\
        \bottomrule
    \end{tabulary}
    \end{center}

    \begin{itemize}
        \item \textbf{બેન્ડવિડ્થ}: FM ને AM કરતાં ઘણી વધુ બેન્ડવિડ્થ જરૂરી છે.
        \item \textbf{ક્વોલિટી}: FM વધુ સારી ઓડિયો ક્વોલિટી પૂરી પાડે છે.
    \end{itemize}

    \begin{mnemonicbox}
    "AM-Amplitude સરળ, FM-Frequency જટિલ પણ વધુ સારી ક્વોલિટી"
    \end{mnemonicbox}
\end{solutionbox}

\questionmarks{2}{b}{4}
\textbf{એમ્પ્લિટ્યુડ મોડ્યુલેટેડ વેવનું વેવફોર્મ દોરો.}

\begin{solutionbox}
    \textbf{જવાબ}:

    \textbf{ડાયાગ્રામ:}

    \begin{center}
    \begin{tikzpicture}
        \begin{axis}[
            width=10cm, height=5cm,
            axis lines=middle,
            xtick=\empty, ytick=\empty,
            xlabel=$t$, ylabel=$e_{AM}(t)$,
            domain=0:20, samples=200
        ]
            \addplot[blue, smooth] {(1 + 0.5*cos(deg(x))) * cos(deg(10*x))};
            \addplot[red, dashed, smooth] {1 + 0.5*cos(deg(x))};
            \addplot[red, dashed, smooth] {-(1 + 0.5*cos(deg(x)))};
        \end{axis}
    \end{tikzpicture}
    \end{center}

    \textbf{લાક્ષણિકતાઓ:}

    \begin{itemize}
        \item \textbf{એન્વેલોપ}: એન્વેલોપ મોડ્યુલેટિંગ સિગ્નલને અનુસરે છે.
        \item \textbf{કેરિયર ફ્રીક્વન્સી}: સમગ્ર સમય દરમિયાન સ્થિર રહે છે.
        \item \textbf{એમ્પ્લિટ્યુડ વેરિએશન}: એમ્પ્લિટ્યુડ મોડ્યુલેટિંગ સિગ્નલ સાથે બદલાય છે.
    \end{itemize}

    \begin{mnemonicbox}
    "Envelope Follows Message - EFM"
    \end{mnemonicbox}
\end{solutionbox}

\questionmarks{2}{c}{7}
\textbf{એમ્પ્લિટ્યુડ મોડ્યુલેશનની વ્યાખ્યા આપો અને ડબલ સાઇડબેન્ડ ફુલ કેરિયર (DSBFC) એમ્પ્લિટ્યુડ મોડ્યુલેશન (AM) સિગ્નલ માટે ગાણિતિક અભિવ્યક્તિ મેળવો.}

\begin{solutionbox}
    \textbf{જવાબ}:

    \textbf{વ્યાખ્યા:} એમ્પ્લિટ્યુડ મોડ્યુલેશન એ પ્રક્રિયા છે જેમાં કેરિયર સિગ્નલનું એમ્પ્લિટ્યુડ મોડ્યુલેટિંગ સિગ્નલના તાત્કાલિક એમ્પ્લિટ્યુડ અનુસાર બદલાય છે.

    \textbf{ગાણિતિક વ્યુત્પત્તિ:}

    કેરિયર સિગ્નલ: $e_c(t) = E_c \cos(\omega_c t)$ \\
    મોડ્યુલેટિંગ સિગ્નલ: $e_m(t) = E_m \cos(\omega_m t)$

    \textbf{AM સિગ્નલ અભિવ્યક્તિ:}
    \[ e_{AM}(t) = [E_c + E_m \cos(\omega_m t)] \cos(\omega_c t) \]
    \[ e_{AM}(t) = E_c \cos(\omega_c t) + E_m \cos(\omega_m t) \cos(\omega_c t) \]

    ત્રિકોણમિતિય સૂત્રનો ઉપયોગ:
    \[ \cos A \cos B = \frac{1}{2}[\cos(A+B) + \cos(A-B)] \]

    \textbf{અંતિમ AM અભિવ્યક્તિ:}
    \[ e_{AM}(t) = E_c \cos(\omega_c t) + \frac{E_m}{2} \cos(\omega_c + \omega_m)t + \frac{E_m}{2} \cos(\omega_c - \omega_m)t \]

    \textbf{ઘટકો:}

    \begin{itemize}
        \item \textbf{કેરિયર કોમ્પોનન્ટ}: $E_c \cos(\omega_c t)$
        \item \textbf{અપર સાઇડબેન્ડ}: $\frac{E_m}{2} \cos(\omega_c + \omega_m)t$
        \item \textbf{લોઅર સાઇડબેન્ડ}: $\frac{E_m}{2} \cos(\omega_c - \omega_m)t$
    \end{itemize}

    \begin{mnemonicbox}
    "Carrier Plus Upper Lower Sidebands - CPULS"
    \end{mnemonicbox}
\end{solutionbox}

\orquestions

\questionmarks{2}{a}{3}
\textbf{પ્રી-એમ્ફેસિસ અને ડી-એમ્ફેસિસની સરખામણી કરો.}

\begin{solutionbox}
    \textbf{જવાબ}:

    \begin{center}
    \begin{tabulary}{\linewidth}{L L L}
        \toprule
        \textbf{પેરામીટર} & \textbf{પ્રી-એમ્ફેસિસ} & \textbf{ડી-એમ્ફેસિસ} \\
        \midrule
        \textbf{સ્થાન} & ટ્રાન્સમિટર પર & રીસીવર પર \\
        \textbf{કાર્ય} & ઉચ્ચ આવૃત્તિઓ વધારે છે & ઉચ્ચ આવૃત્તિઓ ઘટાડે છે \\
        \textbf{ફ્રીક્વન્સી રિસ્પોન્સ} & હાઇ પાસ લાક્ષણિકતા & લો પાસ લાક્ષણિકતા \\
        \textbf{હેતુ} & S/N રેશિયો સુધારે છે & મૂળ સિગ્નલ પુનઃસ્થાપિત કરે છે \\
        \textbf{ટાઇમ કોન્સ્ટન્ટ} & 75 $\mu$s (FM બ્રોડકાસ્ટિંગ) & 75 $\mu$s (FM બ્રોડકાસ્ટિંગ) \\
        \bottomrule
    \end{tabulary}
    \end{center}

    \begin{itemize}
        \item \textbf{નોઇઝ રિડક્શન}: સંયુક્ત અસર મળેલ સિગ્નલમાં નોઇઝ ઘટાડે છે.
        \item \textbf{ફ્રીક્વન્સી રિસ્પોન્સ}: પૂરક લાક્ષણિકતાઓ.
    \end{itemize}

    \begin{mnemonicbox}
    "Pre-Boost, De-Cut - Noise Reduction Circuit"
    \end{mnemonicbox}
\end{solutionbox}

\questionmarks{2}{b}{4}
\textbf{ફ્રીક્વન્સી મોડ્યુલેટેડ વેવનું વેવફોર્મ દોરો.}

\begin{solutionbox}
    \textbf{જવાબ}:

    \textbf{ડાયાગ્રામ:}

    \begin{center}
    \begin{tikzpicture}
        \begin{axis}[
            width=10cm, height=5cm,
            axis lines=middle,
            xtick=\empty, ytick=\empty,
            xlabel=$t$, ylabel=$e_{FM}(t)$,
            domain=0:20, samples=300
        ]
            % FM wave: cos(wc t + mf sin(wm t))
            \addplot[blue, smooth] {cos(deg(10*x + 5*sin(deg(x))))};
        \end{axis}
    \end{tikzpicture}
    \end{center}

    \textbf{લાક્ષણિકતાઓ:}

    \begin{itemize}
        \item \textbf{કોન્સ્ટન્ટ એમ્પ્લિટ્યુડ}: એમ્પ્લિટ્યુડ સ્થિર રહે છે.
        \item \textbf{ફ્રીક્વન્સી વેરિએશન}: આવૃત્તિ મોડ્યુલેટિંગ સિગ્નલ સાથે બદલાય છે.
        \item \textbf{ફેઝ કોન્ટિન્યુઇટી}: ફેઝ સતત રહે છે.
    \end{itemize}

    \begin{mnemonicbox}
    "Constant Amplitude, Variable Frequency - CAVF"
    \end{mnemonicbox}
\end{solutionbox}

\questionmarks{2}{c}{7}
\textbf{ફ્રીક્વન્સી મોડ્યુલેશનની વ્યાખ્યા આપો અને FM તરંગ માટે ગાણિતિક અભિવ્યક્તિ મેળવો.}

\begin{solutionbox}
    \textbf{જવાબ}:

    \textbf{વ્યાખ્યા:} ફ્રીક્વન્સી મોડ્યુલેશન એ પ્રક્રિયા છે જેમાં કેરિયર સિગ્નલની આવૃત્તિ મોડ્યુલેટિંગ સિગ્નલના તાત્કાલિક એમ્પ્લિટ્યુડ અનુસાર બદલાય છે.

    \textbf{ગાણિતિક વ્યુત્પત્તિ:}

    મોડ્યુલેટિંગ સિગ્નલ: $e_m(t) = E_m \cos(\omega_m t)$ \\
    તાત્કાલિક આવૃત્તિ: $f_i = f_c + k_f E_m \cos(\omega_m t)$

    જ્યાં $k_f$ = આવૃત્તિ સંવેદનશીલતા

    \textbf{તાત્કાલિક કોણીય આવૃત્તિ:}
    \[ \omega_i = 2\pi[f_c + k_f E_m \cos(\omega_m t)] \]
    \[ \omega_i = \omega_c + 2\pi k_f E_m \cos(\omega_m t) \]

    \textbf{ફેઝ ગણતરી:}
    \[ \theta(t) = \int \omega_i dt = \omega_c t + \frac{2\pi k_f E_m}{\omega_m} \sin(\omega_m t) \]

    મોડ્યુલેશન ઇન્ડેક્સ: $m_f = \frac{2\pi k_f E_m}{\omega_m} = \frac{\Delta f}{f_m}$

    \textbf{અંતિમ FM અભિવ્યક્તિ:}
    \[ e_{FM}(t) = E_c \cos[\omega_c t + m_f \sin(\omega_m t)] \]

    \textbf{પેરામીટર:}

    \begin{itemize}
        \item \textbf{મોડ્યુલેશન ઇન્ડેક્સ}: $m_f = \Delta f/f_m$
        \item \textbf{ફ્રીક્વન્સી ડેવિએશન}: $\Delta f = k_f E_m$
        \item \textbf{બેન્ડવિડ્થ}: $BW = 2(\Delta f + f_m)$ (કાર્સનનો નિયમ)
    \end{itemize}

    \begin{mnemonicbox}
    "Frequency Varies with Message - FVM"
    \end{mnemonicbox}
\end{solutionbox}

\questionmarks{3}{a}{3}
\textbf{FM ડિમોડ્યુલેશનની સ્લોપ ડિટેક્શન પદ્ધતિનું વર્ણન કરો.}

\begin{solutionbox}
    \textbf{જવાબ}:

    \textbf{સ્લોપ ડિટેક્શન સિદ્ધાંત:}

    \begin{center}
    \begin{tikzpicture}[gtu tree]
    \node [gtu block] {FM સિગ્નલ}
        child {node [gtu block] {ટ્યુન્ડ સર્કિટ}
            child {node [gtu block] {એન્વેલોપ ડિટેક્ટર}
                child {node [gtu block] {ઓડિયો આઉટપુટ}}
            }
        };
    \end{tikzpicture}
    \end{center}

    \textbf{કાર્યપદ્ધતિ:}

    \begin{itemize}
        \item \textbf{ટ્યુન્ડ સર્કિટ}: આવૃત્તિ ફેરફારોને એમ્પ્લિટ્યુડ ફેરફારોમાં રૂપાંતરિત કરે છે.
        \item \textbf{સ્લોપ ઓપરેશન}: રેઝોનન્સ કર્વના સ્લોપનો ઉપયોગ કરે છે.
        \item \textbf{એન્વેલોપ ડિટેક્શન}: એમ્પ્લિટ્યુડ ફેરફારો કાઢે છે.
    \end{itemize}

    \textbf{લાક્ષણિકતાઓ:}

    \begin{itemize}
        \item \textbf{સિમ્પલ સર્કિટ}: અમલમાં મૂકવા સરળ.
        \item \textbf{લિનિયર રેન્જ}: મર્યાદિત લિનિયર રેન્જ.
        \item \textbf{આઉટપુટ ડિસ્ટોર્શન}: અન્ય પદ્ધતિઓ કરતાં વધુ વિકૃતિ.
    \end{itemize}

    \begin{mnemonicbox}
    "Slope Converts Frequency to Amplitude - SCFA"
    \end{mnemonicbox}
\end{solutionbox}

\questionmarks{3}{b}{4}
\textbf{રેડિયો રીસીવરની વિવિધ લાક્ષણિકતાઓ સમજાવો.}

\begin{solutionbox}
    \textbf{જવાબ}:

    \begin{center}
    \begin{tabulary}{\linewidth}{L L L}
        \toprule
        \textbf{લાક્ષણિકતા} & \textbf{વ્યાખ્યા} & \textbf{મહત્વ} \\
        \midrule
        \textbf{સેન્સિટિવિટી} & સંતોષકારક આઉટપુટ માટે લઘુત્તમ ઇનપુટ સિગ્નલ & વધુ સારી નબળી સિગ્નલ રિસેપ્શન \\
        \textbf{સિલેક્ટિવિટી} & ઇચ્છિત સિગ્નલ પસંદ કરવાની અને અન્યને નકારવાની ક્ષમતા & દખલગીરી ઘટાડે છે \\
        \textbf{ફિડેલિટી} & પુનરુત્પાદનની વફાદારી & વધુ સારી ઓડિયો ક્વોલિટી \\
        \textbf{ઇમેજ ફ્રીક્વન્સી રિજેક્શન} & ઇમેજ આવૃત્તિનો અસ્વીકાર & ખોટા સિગ્નલ અટકાવે છે \\
        \bottomrule
    \end{tabulary}
    \end{center}

    \textbf{ગાણિતિક સંબંધો:}

    \begin{itemize}
        \item \textbf{સેન્સિટિવિટી}: સ્ટાન્ડર્ડ આઉટપુટ માટે $\mu$V માં માપવામાં આવે છે.
        \item \textbf{સિલેક્ટિવિટી}: $Q = f_0/BW$.
        \item \textbf{ઇમેજ રિજેક્શન રેશિયો}: $IRR = \sqrt{1 + Q^2 \rho^2}$.
    \end{itemize}

    \begin{mnemonicbox}
    "Sensitive Selective Faithful Image-free - SSFI"
    \end{mnemonicbox}
\end{solutionbox}

\questionmarks{3}{c}{7}
\textbf{યોગ્ય બ્લોક ડાયાગ્રામ સાથે સુપર હેટરોડાઇન રીસીવર પર ટૂંકી નોંધ લખો.}

\begin{solutionbox}
    \textbf{જવાબ}:

    \textbf{બ્લોક ડાયાગ્રામ:}

    \begin{center}
    \begin{tikzpicture}[gtu tree]
    \node [gtu block] {એન્ટેના}
        child {node [gtu block] {RF એમ્પ્લિફાયર}
            child {node [gtu block] {મિક્સર}
                child {node [gtu block] {IF એમ્પ્લિફાયર}
                    child {node [gtu block] {ડિટેક્ટર}
                        child {node [gtu block] {AF એમ્પ્લિફાયર}
                             child {node [gtu block] {સ્પીકર}}
                        }
                    }
                }
            }
        };
    \end{tikzpicture}
    \end{center}

    \textbf{કાર્યસિદ્ધાંત:}

    \begin{itemize}
        \item \textbf{આરએફ એમ્પ્લિફાયર}: પ્રાપ્ત RF સિગ્નલને એમ્પ્લિફાઇ કરે છે.
        \item \textbf{મિક્સર}: RF ને નિશ્ચિત IF આવૃત્તિમાં રૂપાંતરિત કરે છે.
        \item \textbf{લોકલ ઓસિલેટર}: મિક્સિંગ આવૃત્તિ પૂરી પાડે છે.
        \item \textbf{આઇએફ એમ્પ્લિફાયર}: નિશ્ચિત આવૃત્તિ પર મુખ્ય એમ્પ્લિફિકેશન.
        \item \textbf{ડિટેક્ટર}: મોડ્યુલેટેડ સિગ્નલ પુનઃપ્રાપ્ત કરે છે.
        \item \textbf{એજીસી}: સ્થિર આઉટપુટ સ્તર જાળવે છે.
    \end{itemize}

    \textbf{ફાયદા:}

    \begin{itemize}
        \item \textbf{હાઇ સેન્સિટિવિટી}: TRF કરતાં વધુ સારી સંવેદનશીલતા.
        \item \textbf{ગુડ સિલેક્ટિવિટી}: વધુ સારી પસંદગીકારકતા.
        \item \textbf{સ્ટેબલ ગેઇન}: સ્થિર ગેઇન લાક્ષણિકતાઓ.
    \end{itemize}

    \textbf{IF આવૃત્તિ પસંદગી:} \\
    સ્ટાન્ડર્ડ IF: AM માટે 455 kHz, FM માટે 10.7 MHz.

    \begin{mnemonicbox}
    "Mix RF to IF for Better Selectivity - MRIBS"
    \end{mnemonicbox}
\end{solutionbox}

\orquestions

\questionmarks{3}{a}{3}
\textbf{ફેઝ લોક્ડ લૂપનો ઉપયોગ કરીને FM ડિમોડ્યુલેટરનું કાર્ય સમજાવો.}

\begin{solutionbox}
    \textbf{જવાબ}:

    \textbf{PLL FM ડિમોડ્યુલેટર:}

    \begin{center}
    \begin{tikzpicture}[gtu tree]
    \node [gtu block] {FM ઇનપુટ}
        child {node [gtu block] {ફેઝ ડિટેક્ટર}
            child {node [gtu block] {લૂપ ફિલ્ટર}
                 child[child anchor=north] {node [gtu block] (vco) {VCO}
                 }
                 child {node [gtu block] {ઓડિયો આઉટપુટ}}
            }
        };
    \draw [gtu arrow] (vco.west) -| (1,-2) -- (1,0) -- (1.5,0);
    \end{tikzpicture}
    \end{center}

     \begin{center}
    \begin{tikzpicture}[auto, node distance=2cm,>=latex']
        \node [gtu block] (pd) {ફેઝ ડિટેક્ટર};
        \node [gtu block, right of=pd, node distance=3cm] (lf) {લૂપ ફિલ્ટર};
        \node [gtu block, below of=pd] (vco) {VCO};
        \node [right of=lf, node distance=3cm] (out) {ઓડિયો આઉટપુટ};
        \node [left of=pd, node distance=2cm] (in) {FM ઇનપુટ};

        \draw [->] (in) -- (pd);
        \draw [->] (pd) -- (lf);
        \draw [->] (lf) -- (out);
        \draw [->] (lf) |- (vco);
        \draw [->] (vco) -- (pd);
    \end{tikzpicture}
    \end{center}

    \textbf{કાર્યસિદ્ધાંત:}

    \begin{itemize}
        \item \textbf{ફેઝ ડિટેક્ટર}: ઇનપુટ FM ને VCO આઉટપુટ સાથે સરખાવે છે.
        \item \textbf{વીસીઓ}: વોલ્ટેજ કંટ્રોલ્ડ ઓસિલેટર ઇનપુટ આવૃત્તિને ટ્રેક કરે છે.
        \item \textbf{લૂપ ફિલ્ટર}: ઉચ્ચ આવૃત્તિ ઘટકો દૂર કરે છે.
        \item \textbf{લોક કન્ડિશન}: VCO આવૃત્તિ ઇનપુટ આવૃત્તિ સમાન થાય છે.
    \end{itemize}

    \textbf{ફાયદા:}

    \begin{itemize}
        \item \textbf{લીનિયર ડિમોડ્યુલેશન}: ઉત્તમ રેખીયતા.
        \item \textbf{લો ડિસ્ટોર્શન}: લઘુત્તમ વિકૃતિ.
        \item \textbf{ગુડ ટ્રેકિંગ}: ઉત્તમ આવૃત્તિ ટ્રેકિંગ.
    \end{itemize}

    \begin{mnemonicbox}
    "Phase Lock Tracks Frequency - PLTF"
    \end{mnemonicbox}
\end{solutionbox}

\questionmarks{3}{b}{4}
\textbf{મૂળભૂત FM રીસીવરના બ્લોક ડાયાગ્રામની ચર્ચા કરો.}

\begin{solutionbox}
    \textbf{જવાબ}:

    \textbf{FM રીસીવર બ્લોક ડાયાગ્રામ:}

    \begin{center}
    \begin{tikzpicture}[gtu tree]
    \node [gtu block] {FM એન્ટેના}
        child {node [gtu block] {RF એમ્પ્લિફાયર}
            child {node [gtu block] {મિક્સર}
                child {node [gtu block] {IF એમ્પ્લિફાયર\10.7 MHz}
                    child {node [gtu block] {લિમિટર}
                        child {node [gtu block] {FM ડિટેક્ટર}
                            child {node [gtu block] {ડી-એમ્ફેસિસ}
                                child {node [gtu block] {AF એમ્પ્લિફાયર}
                                    child {node [gtu block] {સ્પીકર}}
                                }
                            }
                        }
                    }
                }
            }
        };
    \end{tikzpicture}
    \end{center}

    \textbf{બ્લોક કાર્યો:}

    \begin{itemize}
        \item \textbf{આરએફ એમ્પ્લિફાયર}: નબળા FM સિગ્નલને એમ્પ્લિફાઇ કરે છે (88-108 MHz).
        \item \textbf{મિક્સર}: IF આવૃત્તિમાં રૂપાંતરિત કરે છે (10.7 MHz).
        \item \textbf{લિમિટર}: એમ્પ્લિટ્યુડ ફેરફારો દૂર કરે છે.
        \item \textbf{એફએમ ડિટેક્ટર}: ઓડિયો સિગ્નલ પુનઃપ્રાપ્ત કરે છે.
        \item \textbf{ડી-એમ્ફેસિસ}: મૂળ આવૃત્તિ પ્રતિસાદ પુનઃસ્થાપિત કરે છે.
    \end{itemize}

    \textbf{AM રીસીવરથી મુખ્ય તફાવતો:}

    \begin{itemize}
        \item \textbf{હાયર આઇએફ}: 455 kHz બદલે 10.7 MHz.
        \item \textbf{લિમિટર સ્ટેજ}: વધારાનો લિમિટર સ્ટેજ.
        \item \textbf{ડી-એમ્ફેસિસ}: પ્રી/ડી-એમ્ફેસિસ નેટવર્ક.
    \end{itemize}

    \begin{mnemonicbox}
    "FM needs Higher IF and Limiting - FHIL"
    \end{mnemonicbox}
\end{solutionbox}

\questionmarks{3}{c}{7}
\textbf{યોગ્ય સર્કિટ ડાયાગ્રામ અને વેવફોર્મ સાથે ડાયોડનો ઉપયોગ કરીને એન્વેલોપ ડિટેક્ટર પર ટૂંકી નોંધ લખો.}

\begin{solutionbox}
    \textbf{જવાબ}:

    \textbf{સર્કિટ ડાયાગ્રામ:}

    \begin{center}
    \begin{tikzpicture}[scale=1.2]
        % Circuit
        \draw (0,0) node[left] {AM ઇનપુટ} to[diode, l=$] (2,0) -- (4,0) node[right] {આઉટપુટ};
        \draw (2,0) to[R, l=$] (2,-2) -- (2,-2) node[ground]{};
        \draw (3,0) to[C, l=$] (3,-2) -- (3,-2) node[ground]{};
        \draw (0,-2) node[ground]{};
        \draw (0,0) to[sV] (0,-2);
    \end{tikzpicture}
    \end{center}

    \textbf{કાર્યસિદ્ધાંત:}

    \begin{center}
    \begin{tikzpicture}
        \begin{axis}[
            width=8cm, height=4cm,
            axis lines=middle,
            xtick=\empty, ytick=\empty,
            xlabel=$, ylabel={out}$,
            domain=0:20, samples=200
        ]
            \addplot[blue, smooth] {(1 + 0.5*cos(deg(x))) * abs(cos(deg(10*x)))};
            \addplot[red, thick, smooth] {1 + 0.5*cos(deg(x)) - 0.1}; % Approximate envelope
        \end{axis}
    \end{tikzpicture}
    \end{center}

    \textbf{ઓપરેશન:}

    \begin{itemize}
        \item \textbf{ડાયોડ કન્ડક્શન}: સકારાત્મક અર્ધ ચક્ર દરમિયાન વહન કરે છે.
        \item \textbf{કેપેસિટર ચાર્જિંગ}: પીક વેલ્યુ સુધી ચાર્જ થાય છે.
        \item \textbf{આરસી ડિસચાર્જ}: RC સર્કિટ દ્વારા ડિસચાર્જ થાય છે.
        \item \textbf{એન્વેલોપ ફોલોઇંગ}: આઉટપુટ એન્વેલોપને અનુસરે છે.
    \end{itemize}

    \textbf{ડિઝાઇન વિચારણાઓ:}

    \begin{itemize}
        \item \textbf{ટાઇમ કોન્સ્ટન્ટ}:  \gg 1/f_c$ પણ  \ll 1/f_m$.
        \item \textbf{ડાયોડ સિલેક્શન}: ફાસ્ટ રિકવરી ડાયોડ પસંદીદા.
        \item \textbf{લોડ રેઝિસ્ટન્સ}: ડાયોડ રેઝિસ્ટન્સ કરતાં ઘણું મોટું હોવું જોઈએ.
    \end{itemize}

    \textbf{ફાયદા:}

    \begin{itemize}
        \item \textbf{સિમ્પ્લિસિટી}: ખૂબ સરળ સર્કિટ.
        \item \textbf{લો કોસ્ટ}: આર્થિક ઉકેલ.
        \item \textbf{હાઇ એફિશિયન્સી}: સારી ડિટેક્શન કાર્યક્ષમતા.
    \end{itemize}

    \begin{mnemonicbox}
    "Diode Charges, RC Follows Envelope - DCRF"
    \end{mnemonicbox}
\end{solutionbox}

\questionmarks{4}{a}{3}
\textbf{અન્ડર સેમ્પલિંગ, ઓવર સેમ્પલિંગ અને ક્રિટિકલ સેમ્પલિંગનું વિવરણ આપો.}

\begin{solutionbox}
    \textbf{જવાબ}:

    \begin{center}
    \begin{tabulary}{\linewidth}{L L L}
        \toprule
        \textbf{પ્રકાર} & \textbf{શરત} & \textbf{પરિણામ} \
        \midrule
        \textbf{અન્ડર સેમ્પલિંગ} &  < 2f_m$ & એલાયસિંગ થાય છે \
        \textbf{ક્રિટિકલ સેમ્પલિંગ} &  = 2f_m$ & માત્ર પૂરતું, કોઈ માર્જિન નથી \
        \textbf{ઓવર સેમ્પલિંગ} &  > 2f_m$ & એલાયસિંગ નથી, સલામત માર્જિન \
        \bottomrule
    \end{tabulary}
    \end{center}

    \textbf{ડાયાગ્રામ:}

    \begin{center}
    \begin{tikzpicture}[scale=0.8]
        \draw[->] (0,0) -- (6,0) node[right] {$};
        \draw[thick] (0,0) -- (1,2) -- (2,0);
        \node at (1,-0.5) {Signal};
        
        % Aliased
        \draw[dashed, red] (1.5,0) -- (2.5,2) -- (3.5,0);
        \node at (2.5, 2.2) {Alias};
    \end{tikzpicture}
    \end{center}

    \begin{itemize}
        \item \textbf{એલાયસિંગ ઇફેક્ટ}: અન્ડર સેમ્પલિંગ આવૃત્તિ ઓવરલેપનું કારણ બને છે.
        \item \textbf{નાયક્વિસ્ટ રેટ}: લઘુત્તમ સેમ્પલિંગ રેટ = f_m$.
    \end{itemize}

    \begin{mnemonicbox}
    "Under-Alias, Critical-Just, Over-Safe - UCO"
    \end{mnemonicbox}
\end{solutionbox}

\questionmarks{4}{b}{4}
\textbf{સેમ્પલિંગ થિયરમ લખો અને નાયક્વિસ્ટ રેટ, નાયક્વિસ્ટ ઇન્ટરવલ અને એલાયસિંગ એરરની વ્યાખ્યા આપો.}

\begin{solutionbox}
    \textbf{જવાબ}:

    \textbf{સેમ્પલિંગ થિયરમ:}
    "જો સેમ્પલિંગ આવૃત્તિ સિગ્નલના સર્વોચ્ચ આવૃત્તિ ઘટકના ઓછામાં ઓછા બમણી હોય તો સતત સિગ્નલ તેના સેમ્પલમાંથી સંપૂર્ણ રીતે પુનઃપ્રાપ્ત કરી શકાય છે."

    \textbf{વ્યાખ્યાઓ:}

    \begin{center}
    \begin{tabulary}{\linewidth}{L L L}
        \toprule
        \textbf{શબ્દ} & \textbf{વ્યાખ્યા} & \textbf{સૂત્ર} \
        \midrule
        \textbf{નાયક્વિસ્ટ રેટ} & લઘુત્તમ સેમ્પલિંગ આવૃત્તિ &  = 2f_m$ \
        \textbf{નાયક્વિસ્ટ ઇન્ટરવલ} & મહત્તમ સેમ્પલિંગ અંતરાલ &  = 1/(2f_m)$ \
        \textbf{એલાયસિંગ એરર} & અન્ડર સેમ્પલિંગને કારણે આવૃત્તિ ઓવરલેપ &  = |f_s - f|$ \
        \bottomrule
    \end{tabulary}
    \end{center}

    \textbf{ગાણિતિક અભિવ્યક્તિ:}

    \begin{itemize}
        \item \textbf{સેમ્પલિંગ ફ્રીક્વન્સી}:  \ge 2f_m$ (નાયક્વિસ્ટ કસોટી).
        \item \textbf{સેમ્પલિંગ પીરિયડ}:  = 1/f_s$.
        \item \textbf{એલાયસિંગ કન્ડિશન}:  < 2f_m$.
    \end{itemize}

    \begin{mnemonicbox}
    "Sample at twice message frequency - S2M"
    \end{mnemonicbox}
\end{solutionbox}

\questionmarks{4}{c}{7}
\textbf{આઇડિયલ, નેચરલ અને ફ્લેટ ટોપ સેમ્પલિંગની ચર્ચા કરો.}

\begin{solutionbox}
    \textbf{જવાબ}:

    \textbf{સેમ્પલિંગના પ્રકારો:}

    \begin{center}
    \begin{tabulary}{\linewidth}{L L L}
        \toprule
        \textbf{પ્રકાર} & \textbf{લાક્ષણિકતાઓ} & \textbf{ગાણિતિક અભિવ્યક્તિ} \
        \midrule
        \textbf{આઇડિયલ સેમ્પલિંગ} & ઇમ્પલ્સ ટ્રેઇન ગુણાકાર & (t) = x(t) \cdot \delta_T(t)$ \
        \textbf{નેચરલ સેમ્પલિંગ} & વેરિએબલ પહોળાઈ પલ્સ & ટોપ સિગ્નલને અનુસરે છે \
        \textbf{ફ્લેટ ટોપ સેમ્પલિંગ} & કોન્સ્ટન્ટ એમ્પ્લિટ્યુડ પલ્સ & સેમ્પલ અને હોલ્ડ \
        \bottomrule
    \end{tabulary}
    \end{center}

    \textbf{વેવફોર્મ:}

    \begin{center}
    \begin{tikzpicture}[scale=0.8]
        % Ideal
        \node[left] at (0,6) {Ideal};
        \foreach \x in {0,1,2,3,4,5,6}
            \draw[->, thick] (\x,5) -- (\x, {5 + 0.5*sin(\x*60)});
            
        % Natural
        \node[left] at (0,3) {Natural};
        \foreach \x in {0,1,2,3,4,5,6}
            \draw[fill=gray!20] (\x-0.1, 2) -- (\x-0.1, {2 + 0.5*sin((\x-0.1)*60)}) -- (\x+0.1, {2 + 0.5*sin((\x+0.1)*60)}) -- (\x+0.1, 2) -- cycle;

        % Flat Top
        \node[left] at (0,0) {Flat Top};
        \foreach \x in {0,1,2,3,4,5,6}
            \draw[fill=gray!20] (\x-0.1, -1) rectangle (\x+0.1, {-1 + 0.5*sin(\x*60)});
    \end{tikzpicture}
    \end{center}

    \textbf{આવૃત્તિ સ્પેક્ટ્રમ:}
    \begin{itemize}
        \item \textbf{આઇડિયલ સેમ્પલિંગ}: સચોટ સ્પેક્ટ્રલ પ્રતિકૃતિ.
        \item \textbf{નેચરલ સેમ્પલિંગ}: થોડું સ્પેક્ટ્રલ મોડિફિકેશન.
        \item \textbf{ફ્લેટ ટોપ સેમ્પલિંગ}: એપર્ચર ઇફેક્ટ ((\pi f T/2)$) હાજર.
    \end{itemize}

    \begin{mnemonicbox}
    "Ideal-Impulse, Natural-Variable, Flat-Constant - IVF"
    \end{mnemonicbox}
\end{solutionbox}

\orquestions

\questionmarks{4}{a}{3}
\textbf{યોગ્ય બ્લોક ડાયાગ્રામ સાથે ડેલ્ટા મોડ્યુલેટરનું કાર્ય સમજાવો.}

\begin{solutionbox}
    \textbf{જવાબ}:

    \textbf{ડેલ્ટા મોડ્યુલેટર બ્લોક ડાયાગ્રામ:}

    \begin{center}
    \begin{tikzpicture}[auto, node distance=2cm,>=latex']
        \node [gtu block] (comp) {કમ્પેરેટર};
        \node [gtu block, right of=comp, node distance=3cm] (quant) {1-બિટ ક્વોન્ટાઇઝર};
        \node [coordinate, right of=quant, node distance=2cm] (out) {};
        \node [gtu block, below of=quant] (int) {ઇન્ટિગ્રેટર};
        \node [gtu block, left of=int, node distance=3cm] (delay) {ડિલે};
        \node [left of=comp, node distance=2cm] (in) {ઇનપુટ};

        \draw [->] (in) -- (comp);
        \draw [->] (comp) -- (quant);
        \draw [->] (quant) -- (out) node[right] {આઉટપુટ};
        \draw [->] (quant) -- (int);
        \draw [->] (int) -- (delay);
        \draw [->] (delay) -| (comp);
    \end{tikzpicture}
    \end{center}

    \textbf{કાર્યસિદ્ધાંત:}

    \begin{itemize}
        \item \textbf{કમ્પેરિસન}: ઇનપુટની સરખામણી પહેલાના ઇન્ટિગ્રેટેડ આઉટપુટ સાથે.
        \item \textbf{1-બિટ ક્વોન્ટાઇઝેશન}: આઉટપુટ $+\Delta$ અથવા 569Xils\Delta$ છે.
        \item \textbf{ઇન્ટિગ્રેશન}: ઇન્ટિગ્રેટર ઇનપુટ સિગ્નલનો અંદાજ કાઢે છે.
    \end{itemize}

    \begin{mnemonicbox}
    "Compare, Quantize, Integrate, Feedback - CQIF"
    \end{mnemonicbox}
\end{solutionbox}

\questionmarks{4}{b}{4}
\textbf{યોગ્ય સમજૂતી સાથે ડેલ્ટા મોડ્યુલેશન (DM) ના ગેરફાયદા લખો.}

\begin{solutionbox}
    \textbf{જવાબ}:

    \begin{center}
    \begin{tabulary}{\linewidth}{L L L}
        \toprule
        \textbf{ગેરફાયદા} & \textbf{સમજૂતી} & \textbf{ઉકેલ} \
        \midrule
        \textbf{સ્લોપ ઓવરલોડ} & ઝડપી ફેરફારો ટ્રેક કરી શકતું નથી & સ્ટેપ સાઇઝ વધારો \
        \textbf{ગ્રેન્યુલર નોઇઝ} & સપાટ વિસ્તારોમાં ક્વોન્ટાઇઝેશન નોઇઝ & સ્ટેપ સાઇઝ ઘટાડો \
        \textbf{હાઇ બિટ રેટ} & ઉચ્ચ સેમ્પલિંગ રેટ જરૂરી & ADPCM નો ઉપયોગ કરો \
        \bottomrule
    \end{tabulary}
    \end{center}

    \textbf{સ્લોપ ઓવરલોડ કન્ડિશન:} જ્યારે $|dx/dt| > \Delta f_s$.

    \textbf{વેવફોર્મ:}

    \begin{center}
    \begin{tikzpicture}[scale=0.8]
        \draw[->] (0,0) -- (6,0) node[right] {$};
        \draw[blue, thick] (0,0) sin (1.5, 2) cos (3,0);
        \draw[red, steps] (0,0) -- (0.2,0.2) -- (0.4,0.4) -- (0.6, 0.6) -- (0.8, 0.8) -- (1.0, 1.0);
        \node at (2, 2.2) {સ્લોપ ઓવરલોડ};
    \end{tikzpicture}
    \end{center}

    \begin{mnemonicbox}
    "Slope-Overload, Granular-Noise, High-Bitrate - SOG-H"
    \end{mnemonicbox}
\end{solutionbox}

\questionmarks{4}{c}{7}
\textbf{પલ્સ કોડ મોડ્યુલેશન (PCM) ટ્રાન્સમિટર અને રીસીવરના દરેક બ્લોકના કાર્યોનું વર્ણન કરો.}

\begin{solutionbox}
    \textbf{જવાબ}:

    \textbf{PCM ટ્રાન્સમિટર:}

    \begin{center}
    \begin{tikzpicture}[gtu tree]
    \node [gtu block] {એનાલોગ ઇનપુટ}
        child {node [gtu block] {LPF}
            child {node [gtu block] {સેમ્પલ અને હોલ્ડ}
                child {node [gtu block] {ક્વોન્ટાઇઝર}
                    child {node [gtu block] {એન્કોડર}
                        child {node [gtu block] {ડિજિટલ આઉટપુટ}}
                    }
                }
            }
        };
    \end{tikzpicture}
    \end{center}

    \textbf{PCM રીસીવર:}

    \begin{center}
    \begin{tikzpicture}[gtu tree]
    \node [gtu block] {ડિજિટલ ઇનપુટ}
        child {node [gtu block] {ડીકોડર}
            child {node [gtu block] {DAC}
                child {node [gtu block] {LPF}
                    child {node [gtu block] {એનાલોગ આઉટપુટ}}
                }
            }
        };
    \end{tikzpicture}
    \end{center}

    \textbf{બ્લોક કાર્યો:}

    \begin{itemize}
        \item \textbf{LPF}: એન્ટિ-એલાયસિંગ ફિલ્ટર.
        \item \textbf{સેમ્પલ અને હોલ્ડ}: સિગ્નલ સેમ્પલ કરે છે.
        \item \textbf{ક્વોન્ટાઇઝર}: ડિસ્ક્રીટ લેવલમાં રૂપાંતરિત કરે છે.
        \item \textbf{એન્કોડર}: બાઇનરીમાં રૂપાંતરિત કરે છે.
        \item \textbf{ડીકોડર}: બાઇનરીમાંથી લેવલમાં રૂપાંતરિત કરે છે.
        \item \textbf{DAC}: ડિજિટલ ટુ એનાલોગ.
    \end{itemize}

    \begin{mnemonicbox}
    "LSQE for TX; DCF for RX"
    \end{mnemonicbox}
\end{solutionbox}

\questionmarks{5}{a}{3}
\textbf{TDM-PCM સિસ્ટમના બ્લોક ડાયાગ્રામની સંક્ષિપ્ત ચર્ચા કરો.}

\begin{solutionbox}
    \textbf{જવાબ}:

    \textbf{TDM-PCM સિસ્ટમ બ્લોક ડાયાગ્રામ:}

    \begin{center}
    \begin{tikzpicture}[gtu tree]
    \node [gtu block] {ચેનલ 1-3}
        child {node [gtu block] {કમ્યુટેટર}
            child {node [gtu block] {PCM એન્કોડર}
                child {node [gtu block] {ટ્રાન્સમિશન}
                    child {node [gtu block] {PCM ડીકોડર}
                        child {node [gtu block] {ડીકમ્યુટેટર}
                             child {node [gtu block] {ચેનલ 1-3 આઉટપુટ}}
                        }
                    }
                }
            }
        };
    \end{tikzpicture}
    \end{center}

    \textbf{સિસ્ટમ ઓપરેશન:}

    \begin{itemize}
        \item \textbf{કમ્યુટેટર}: અનુક્રમિક સેમ્પલિંગ.
        \item \textbf{પીસીએમ એન્કોડર}: સેમ્પલ ડિજિટાઇઝ કરે છે.
        \item \textbf{ટાઇમ ડિવિઝન}: ચેનલ નિશ્ચિત ટાઇમ સ્લોટ મેળવે છે.
        \item \textbf{ડીકમ્યુટેટર}: ચેનલ અલગ કરે છે.
    \end{itemize}

    \begin{mnemonicbox}
    "Time Division Multiple Access - TDMA"
    \end{mnemonicbox}
\end{solutionbox}

\questionmarks{5}{b}{4}
\textbf{એડેપ્ટિવ ડેલ્ટા મોડ્યુલેશન (ADM) પર ટૂંકી નોંધ લખો.}

\begin{solutionbox}
    \textbf{જવાબ}:

    \textbf{ADM બ્લોક ડાયાગ્રામ:}

    \begin{center}
    \begin{tikzpicture}[auto, node distance=2cm,>=latex']
        \node [gtu block] (comp) {કમ્પેરેટર};
        \node [gtu block, right of=comp, node distance=3cm] (logic) {લોજિક};
        \node [gtu block, below of=logic] (step) {સ્ટેપ કંટ્રોલ};
        \node [gtu block, left of=step, node distance=3cm] (int) {ઇન્ટિગ્રેટર};
        \node [left of=comp] (in) {ઇનપુટ};
        \node [right of=logic] (out) {આઉટપુટ};

        \draw[->] (in) -- (comp);
        \draw[->] (comp) -- (logic);
        \draw[->] (logic) -- (step);
        \draw[->] (step) -- (int);
        \draw[->] (int) -| (comp);
        \draw[->] (logic) -- (out);
    \end{tikzpicture}
    \end{center}

    \textbf{કાર્યસિદ્ધાંત:}
    
    \begin{itemize}
        \item \textbf{એડેપ્ટિવ સ્ટેપ સાઇઝ}: સ્લોપના આધારે બદલાય છે.
        \item \textbf{સ્લોપ ઓવરલોડ}: સ્ટેપ સાઇઝ વધારે છે.
        \item \textbf{ગ્રેન્યુલર નોઇઝ}: સ્ટેપ સાઇઝ ઘટાડે છે.
    \end{itemize}

    \begin{mnemonicbox}
    "Adaptive Step size Reduces both Slope-overload and Granular noise - ASRSG"
    \end{mnemonicbox}
\end{solutionbox}

\questionmarks{5}{c}{7}
\textbf{લાઇન કોડિંગની વ્યાખ્યા આપો. "1 0 1 1 1 0 1 1" માટે NRZ (યુનિપોલર), RZ (યુનિપોલર), મેન્ચેસ્ટર કોડિંગ વેવફોર્મ દોરો.}

\begin{solutionbox}
    \textbf{જવાબ}:

    \textbf{વ્યાખ્યા:} લાઇન કોડિંગ એ ડિજિટલ ડેટાને ટ્રાન્સમિશન માટે યોગ્ય ડિજિટલ સિગ્નલમાં રૂપાંતરિત કરવાની પ્રક્રિયા છે.

    \textbf{વેવફોર્મ ડાયાગ્રામ (ડેટા: 1 0 1 1 1 0 1 1):}

    \begin{center}
    \begin{tikzpicture}[x=0.8cm, y=0.8cm]
        % Data labels
        \foreach \b [count=\i] in {1,0,1,1,1,0,1,1} \node at (\i-0.5, 4.5) {\b};
        
        % NRZ Unipolar
        \node[left] at (0,3.5) {NRZ};
        \draw[thick] (0,3) -- (1,4) -- (2,4) -- (2,3) -- (3,3) -- (3,4) -- (6,4) -- (6,3) -- (7,3) -- (7,4) -- (8,4) -- (8,3);
        
        % RZ Unipolar
        \node[left] at (0,1.5) {RZ};
        \draw[thick] (0,1) -- (0.5,2) -- (0.5,1) -- (1,1) -- (2,1) -- (2.5,2) -- (2.5,1) -- (3,1) -- (3.5,2) -- (3.5,1) -- (5,1) -- (5.5,2) -- (5.5,1) -- (6,1) -- (7,1) -- (7.5,2) -- (7.5,1) -- (8,1);

        % Manchester
        \node[left] at (0,-0.5) {Manch};
        \draw[thick] (0,0) -- (0.5,0) -- (0.5,-1) -- (1,-1) -- (1.5,-1) -- (1.5,0) -- (2,0) -- (2.5,0) -- (2.5,-1) -- (3,-1);
        \node at (4,-1.5) {મિડ-બિટ પર ટ્રાન્ઝિશન};
    \end{tikzpicture}
    \end{center}

    \textbf{સરખામણી:}

    \begin{center}
    \begin{tabulary}{\linewidth}{L L L}
        \toprule
        \textbf{પ્રકાર} & \textbf{લોજિક 1} & \textbf{લોજિક 0} \
        \midrule
        \textbf{NRZ} & +V & 0V \
        \textbf{RZ} & T/2 માટે +V & 0V \
        \textbf{મેન્ચેસ્ટર} & હાઇ-ટુ-લો & લો-ટુ-હાઇ \
        \bottomrule
    \end{tabulary}
    \end{center}

    \begin{mnemonicbox}
    "NRZ-Simple, RZ-Return, Manchester-Transition - SRT"
    \end{mnemonicbox}
\end{solutionbox}

\orquestions

\questionmarks{5}{a}{3}
\textbf{ટાઇમ ડિવિઝન ડિજિટલ મલ્ટિપ્લેક્સિંગના કોન્સેપ્ટનું વર્ણન કરો.}

\begin{solutionbox}
    \textbf{જવાબ}:

    \textbf{TDM કોન્સેપ્ટ:} દરેક સિગ્નલને અલગ અલગ ટાઇમ સ્લોટ આપીને અનેક સિગ્નલ ટ્રાન્સમિટ કરવામાં આવે છે.

    \textbf{ફ્રેમ સ્ટ્રક્ચર:}

    \begin{center}
    \begin{tikzpicture}
        \draw (0,0) rectangle (1.5,1) node[midway] {CH1};
        \draw (1.5,0) rectangle (3,1) node[midway] {CH2};
        \draw (3,0) rectangle (4.5,1) node[midway] {SYNC};
        \draw (4.5,0) rectangle (6,1) node[midway] {CH1};
        \draw[<->] (0,-0.5) -- (4.5,-0.5) node[midway, below] {ફ્રેમ};
    \end{tikzpicture}
    \end{center}

    \begin{mnemonicbox}
    "Time slots Share Single Channel - TSSC"
    \end{mnemonicbox}
\end{solutionbox}

\questionmarks{5}{b}{4}
\textbf{ડિફરન્શિયલ PCM (DPCM) પર ટૂંકી નોંધ લખો.}

\begin{solutionbox}
    \textbf{જવાબ}:

    \textbf{DPCM બ્લોક ડાયાગ્રામ:}

    \begin{center}
    \begin{tikzpicture}[auto, node distance=2cm]
        \node [gtu block] (diff) {ડિફરન્સ};
        \node [gtu block, right of=diff, node distance=3cm] (quant) {ક્વોન્ટાઇઝર};
        \node [gtu block, right of=quant, node distance=3cm] (enc) {એન્કોડર};
        \node [gtu block, below of=quant] (dec) {લોકલ ડીકોડર};
        \node [gtu block, left of=dec, node distance=3cm] (pred) {પ્રિડિક્ટર};
        \node [left of=diff] (in) {ઇનપુટ};

        \draw[->] (in) -- (diff);
        \draw[->] (diff) -- (quant);
        \draw[->] (quant) -- (enc);
        \draw[->] (quant) -- (dec);
        \draw[->] (dec) -- (pred);
        \draw[->] (pred) -| (diff);
    \end{tikzpicture}
    \end{center}

    \textbf{વિચાર:} માત્ર ડિફરન્સ સિગ્નલ મોકલીને બિટ રેટ ઘટાડવા માટે પ્રિડિક્શનનો ઉપયોગ કરે છે.

    \begin{mnemonicbox}
    "Predict Difference, Quantize Less - PDQL"
    \end{mnemonicbox}
\end{solutionbox}

\questionmarks{5}{c}{7}
\textbf{4 સ્તરના ડિજિટલ મલ્ટિપ્લેક્સિંગ હાયરાર્કી પર ટૂંકી નોંધ લખો.}

\begin{solutionbox}
    \textbf{જવાબ}:

    \textbf{લેવલ સ્ટ્રક્ચર:}

    \begin{center}
    \begin{tabulary}{\linewidth}{L L L L}
        \toprule
        \textbf{લેવલ} & \textbf{બિટ રેટ} & \textbf{વોઇસ ચેનલ} & \textbf{નામ} \
        \midrule
        \textbf{લેવલ 0} & 64 kbps & 1 & DS-0 \
        \textbf{લેવલ 1} & 1.544 Mbps & 24 & T1 \
        \textbf{લેવલ 2} & 6.312 Mbps & 96 & T2 \
        \textbf{લેવલ 3} & 44.736 Mbps & 672 & T3 \
        \bottomrule
    \end{tabulary}
    \end{center}

    \textbf{મલ્ટિપ્લેક્સિંગ સ્ટ્રક્ચર:}

    \begin{center}
    \begin{tikzpicture}[gtu tree]
    \node [gtu block] {DS-4}
        child {node [gtu block] {6 x DS-3}
            child {node [gtu block] {7 x DS-2}
                child {node [gtu block] {4 x DS-1}
                    child {node [gtu block] {24 x DS-0}}
                }
            }
        };
    \end{tikzpicture}
    \end{center}

    \begin{mnemonicbox}
    "0-1-2-3 levels Build Communication Systems - DS-BCS"
    \end{mnemonicbox}
\end{solutionbox}

\end{document}
