\documentclass{article}

% content/resources/templates/preamble.tex
\usepackage[margin=0.6in]{geometry}
\author{Milav Dabgar}
\usepackage{amsmath,amssymb,amsthm}
\usepackage{booktabs}
\usepackage{multirow}
\usepackage{xcolor}
\usepackage{tcolorbox}
\tcbuselibrary{breakable,skins}
\usepackage[colorlinks=true,linkcolor=blue]{hyperref}
\usepackage{titlesec}
\usepackage{enumitem}
\usepackage{tikz}
\usepackage{pgfplots}
\usepackage{circuitikz}
\usepackage[version=4]{mhchem}
\usepackage{longtable}
\usepackage{array}
\usepackage{float}
\usepackage{caption}
\usepackage{listings}

\lstset{
  basicstyle=\small\ttfamily,
  breaklines=true,
  breakatwhitespace=false,
  postbreak=\mbox{\textcolor{red}{$\hookrightarrow$}\space},
  float=false,
  numbers=left,
  numberstyle=\tiny\color{gray},
  numbersep=10pt,
  xleftmargin=2em,
  keywordstyle=\color{blue},
  commentstyle=\color{green!60!black},
  stringstyle=\color{purple},
  backgroundcolor=\color{gray!5},
  showstringspaces=false,
  tabsize=2,
  captionpos=b,
  keepspaces=true,
  columns=flexible
}

\pgfplotsset{compat=1.18}
\usetikzlibrary{shapes,arrows,positioning,calc,patterns,decorations.pathmorphing,decorations.markings,arrows.meta}

% Color scheme
\definecolor{headcolor}{RGB}{0,102,204}
\definecolor{keycolor}{RGB}{220,20,60}
\definecolor{solutioncolor}{RGB}{34,139,34}
\definecolor{mnemoniccolor}{RGB}{148,0,211}
\definecolor{codecolor}{RGB}{0,0,100}

% Spacing
\setlength{\parskip}{3pt}
\setlist[itemize]{nosep}
\setlist[enumerate]{nosep}

% Title formatting
\titleformat{\section}{\Large\bfseries\color{headcolor}}{\thesection}{1em}{}
\titleformat{\subsection}{\large\bfseries\color{headcolor}}{\thesubsection}{1em}{}

% Pandoc tightlist compatibility
\providecommand{\tightlist}{%
  \setlength{\itemsep}{0pt}\setlength{\parskip}{0pt}}

% Pandoc longtable compatibility
\newcounter{none}
\def\thenone{}


% content/resources/templates/gujarati-boxes.tex
\usepackage{fontspec}
\usepackage{polyglossia}

% Set Gujarati as main language (document is primarily in Gujarati)
% Note: gloss-gujarati.ldf doesn't exist in polyglossia, but it will use hyphenation patterns
\setdefaultlanguage{gujarati}
\setotherlanguage{english}

% Configure Gujarati font properly
% Use Language=Default to prevent polyglossia from trying to add language-specific features
% that don't exist for Gujarati, which causes "empty feature" warnings
\newfontfamily\gujaratifont[Script=Gujarati,AutoFakeBold=2.5,AutoFakeSlant=0.3]{Noto Sans Gujarati}
\setmainfont[Script=Gujarati,AutoFakeBold=2.5,AutoFakeSlant=0.3]{Noto Sans Gujarati}
% Use Noto Sans Gujarati for monospace to support Gujarati in text
\setmonofont[Scale=0.9]{Noto Sans Gujarati}

% Configure English to use the same font
\newfontfamily\englishfont[Script=Gujarati,AutoFakeBold=2.5,AutoFakeSlant=0.3]{Noto Sans Gujarati}

% Translations for polyglossia
\gappto\captionsgujarati{
  \renewcommand{\tablename}{કોષ્ટક}
  \renewcommand{\figurename}{આકૃતિ}
}

% Helper for TikZ nodes to ensure Gujarati font
\newcommand{\gu}[1]{{\gujaratifont #1}}

% Custom environments
\newtcolorbox{solutionbox}{
    breakable,
    enhanced,
    colback=solutioncolor!5!white,
    colframe=solutioncolor!75!black,
    fonttitle=\bfseries,
    title=જવાબ
}

\newtcolorbox{solutionboxnobreak}{
 colback=solutioncolor!5!white,
 colframe=solutioncolor!75!black,
 fonttitle=\bfseries,
 title=જવાબ
}

\newtcolorbox{keyformula}{
 breakable,
 enhanced,
 colback=keycolor!5!white,
 colframe=keycolor!75!black,
 fonttitle=\bfseries,
 title=રાસાયણિક સમીકરણ/સૂત્ર
}

\newtcolorbox{mnemonicbox}{
 breakable,
 enhanced,
 colback=mnemoniccolor!5!white,
 colframe=mnemoniccolor!75!black,
 fonttitle=\bfseries,
 title=મેમરી ટ્રીક
}


% Custom commands for GTU solutions
% This file defines semantic commands for consistent formatting

% Question command with automatic formatting
\newcommand{\question}[2]{%
  \section*{Question #1}%
  \textbf{#2}%
}

% OR question variant
\newcommand{\questionor}[2]{%
  \section*{Question #1 OR}%
  \textbf{#2}%
}

% Proper table environment with caption
\newenvironment{answertable}[1]{%
  \begin{table}[htbp]
  \centering
  \caption{#1}
}{%
  \end{table}
}

% Proper figure environment for diagrams
\newenvironment{answerdiagram}[1]{%
  \begin{figure}[htbp]
  \centering
  \caption{#1}
}{%
  \end{figure}
}

% Semantic markup for key terms
\newcommand{\keyword}[1]{\textbf{#1}}
\newcommand{\code}[1]{\texttt{#1}}
\newcommand{\classname}[1]{\texttt{#1}}
\newcommand{\methodname}[1]{\texttt{#1}}

% Proper quotation marks
\newcommand{\mnemonic}[1]{``#1''}


\title{ઇલેક્ટ્રોનિક કમ્યુનિકેશનના સિદ્ધાંતો (4331104) - ગ્રીષ્મ 2023 સોલ્યુશન}
\date{July 25, 2023}

\begin{document}
\maketitle

\questionmarks{1}{a}{3}
\textbf{સંચાર પ્રણાલી નો બ્લોક ડાયાગ્રામ દોરો અને સમજાવો.}

\begin{solutionbox}
\textbf{સંચાર પ્રણાલી (Communication System) નો બ્લોક ડાયાગ્રામ:}

\begin{center}
\begin{tikzpicture}[auto, node distance=2cm, >=latex]
    \node [gtu block] (source) {ઈનપુટ સોર્સ};
    \node [gtu block, right of=source, node distance=3cm] (tx) {ટ્રાન્સમીટર};
    \node [gtu block, right of=tx, node distance=3cm] (channel) {ચેનલ};
    \node [gtu block, right of=channel, node distance=3cm] (rx) {રીસીવર};
    \node [gtu block, right of=rx, node distance=3cm] (dest) {આઉટપુટ};
    \node [gtu block, below of=channel, node distance=2cm] (noise) {નોઈઝ સોર્સ};

    \draw [gtu arrow] (source) -- (tx);
    \draw [gtu arrow] (tx) -- (channel);
    \draw [gtu arrow] (channel) -- (rx);
    \draw [gtu arrow] (rx) -- (dest);
    \draw [gtu arrow] (noise) -- (channel);
\end{tikzpicture}
\captionof{figure}{સંચાર પ્રણાલી નો બ્લોક ડાયાગ્રામ}
\end{center}

\begin{itemize}
    \item \textbf{ઈનપુટ સોર્સ}: સંદેશ સિગ્નલ જે સોર્સમાંથી ઉદ્ભવે છે (જેમ કે અવાજ, ચિત્ર, ડેટા).
    \item \textbf{ટ્રાન્સમીટર}: સંદેશને ટ્રાન્સમિશન માટે યોગ્ય સ્વરૂપમાં રૂપાંતરિત કરે છે (મોડ્યુલેશન, એમ્પ્લીફિકેશન).
    \item \textbf{ચેનલ}: માધ્યમ જેના દ્વારા સિગ્નલ મુસાફરી કરે છે (જેમ કે વાયર, ફાઈબર, મુક્ત અવકાશ).
    \item \textbf{રીસીવર}: પ્રાપ્ત થયેલ સિગ્નલમાંથી મૂળ સંદેશ કાઢે છે (ડીમોડ્યુલેશન, એમ્પ્લીફિકેશન).
    \item \textbf{આઉટપુટ}: ગંતવ્ય પર પહોંચાડવામાં આવેલ સંદેશ.
    \item \textbf{નોઈઝ સોર્સ}: અનિચ્છનીય સિગ્નલો જે સંચારમાં દખલ કરે છે અને વિકૃતિ (distortion) ઉમેરે છે.
\end{itemize}
\end{solutionbox}

\begin{mnemonicbox}
\mnemonic{I Transmit Clearly Receiving Original Messages}
\end{mnemonicbox}

\questionmarks{1}{b}{4}
\textbf{મોડ્યુલેશનની જરૂરિયાત સમજાવો. મોડ્યુલેશનના ફાયદા જણાવો.}

\begin{solutionbox}
\textbf{મોડ્યુલેશનની જરૂરિયાત:}

\begin{center}
\begin{tikzpicture}[node distance=2cm]
    \node (center) [gtu block] {મોડ્યુલેશન};
    \node (ant) [gtu state, above of=center] {વ્યવહારુ એન્ટેના કદ};
    \node (mux) [gtu state, right of=center, xshift=2cm] {મલ્ટિપ્લેક્સિંગ};
    \node (noise) [gtu state, below of=center] {નોઈઝ ઘટાડો};
    \node (range) [gtu state, left of=center, xshift=-2cm] {ટ્રાન્સમિશન રેન્જ};

    \draw [gtu arrow] (ant) -- (center);
    \draw [gtu arrow] (mux) -- (center);
    \draw [gtu arrow] (noise) -- (center);
    \draw [gtu arrow] (range) -- (center);
\end{tikzpicture}
\end{center}

\textbf{મોડ્યુલેશનના ફાયદા:}

\begin{enumerate}
    \item \textbf{એન્ટેનાનું કદ ઘટે છે}: 
    \begin{itemize}
        \item એન્ટેનાની ઊંચાઈ $h = \lambda/4 = c/4f$.
        \item ઓડિયો ફ્રીક્વન્સી (નીચી $f$) માટે, $h$ અસાધારણ રીતે મોટું (km) હોય છે.
        \item મોડ્યુલેશન સિગ્નલને ઉચ્ચ $f$ પર શિફ્ટ કરે છે, જેથી એન્ટેનાનું કદ મીટરમાં ઘટી જાય છે.
    \end{itemize}
    \item \textbf{મલ્ટિપ્લેક્સિંગ શક્ય બને છે}: 
    \begin{itemize}
        \item એક જ ચેનલ દ્વારા એક સાથે અનેક સિગ્નલો મોકલી શકાય છે, દરેકને અલગ કેરિયર ફ્રીક્વન્સી આપીને.
    \end{itemize}
    \item \textbf{રેન્જ વધે છે}: 
    \begin{itemize}
        \item લો ફ્રીક્વન્સી બેઝબેન્ડ સિગ્નલો ઉચ્ચ એટેન્યુએશન (attenuation) નો સામનો કરે છે.
        \item મોડ્યુલેટેડ હાઈ ફ્રીક્વન્સી સિગ્નલો ઓછા એટેન્યુએશન સાથે વધુ દૂર સુધી જઈ શકે છે.
    \end{itemize}
    \item \textbf{નોઈઝ ઘટે છે}: 
    \begin{itemize}
        \item મોડ્યુલેશન તકનીકો (જેમ કે FM, PCM) બેઝબેન્ડ ટ્રાન્સમિશનની સરખામણીમાં નોઈઝ સામે વધુ સારી સુરક્ષા આપે છે, જેથી SNR સુધરે છે.
    \end{itemize}
\end{enumerate}
\end{solutionbox}

\begin{mnemonicbox}
\mnemonic{Antennas Need Modulation For Reaching Anywhere with Noise Immunity}
\end{mnemonicbox}

\questionmarks{1}{c}{7}
\textbf{મોડ્યુલેશનને વ્યાખ્યાયિત કરો. વેવફોર્મ સાથે એમ્પ્લીટ્યુડ મોડ્યુલેશન સમજાવો અને મોડ્યુલેટેડ સિગ્નલ માટે વોલ્ટેજ સમીકરણ તારવો.}

\begin{solutionbox}
\textbf{મોડ્યુલેશન}: હાઈ-ફ્રીક્વન્સી કેરિયર સિગ્નલના મૂળભૂત પેરામીટર (એમ્પ્લીટ્યુડ, ફ્રીક્વન્સી, અથવા ફેઝ) ને લો-ફ્રીક્વન્સી મેસેજ સિગ્નલના તત્કાલીન મૂલ્યના પ્રમાણમાં બદલવાની પ્રક્રિયા.

\textbf{એમ્પ્લીટ્યુડ મોડ્યુલેશન વેવફોર્મ:}

\begin{center}
\begin{tikzpicture}[scale=0.8]
    % Carrier
    \begin{scope}[yshift=4cm]
        \draw[->] (0,0) -- (10,0) node[right] {$t$};
        \draw[->] (0,-1.5) -- (0,1.5) node[above] {$v_c(t)$};
        \draw[domain=0:9.5, samples=200, smooth] plot (\x, {1*sin(15*\x r)});
        \node at (5,-2) {કેરિયર સિગ્નલ};
    \end{scope}

    % Message
    \begin{scope}[yshift=0cm]
        \draw[->] (0,0) -- (10,0) node[right] {$t$};
        \draw[->] (0,-1.5) -- (0,1.5) node[above] {$v_m(t)$};
        \draw[domain=0:9.5, samples=100, smooth] plot (\x, {0.8*sin(1.5*\x r)});
        \node at (5,-2) {મેસેજ સિગ્નલ};
    \end{scope}

    % AM Signal
    \begin{scope}[yshift=-5cm]
        \draw[->] (0,0) -- (10,0) node[right] {$t$};
        \draw[->] (0,-2.5) -- (0,2.5) node[above] {$v_{AM}(t)$};
        % Envelope
        \draw[dashed, blue] plot[domain=0:9.5, samples=100] (\x, {1 + 0.8*sin(1.5*\x r)});
        \draw[dashed, blue] plot[domain=0:9.5, samples=100] (\x, {-(1 + 0.8*sin(1.5*\x r))});
        % Signal
        \draw[domain=0:9.5, samples=300, smooth] plot (\x, {(1 + 0.8*sin(1.5*\x r)) * sin(15*\x r)});
        \node at (5,-3) {AM સિગ્નલ};
    \end{scope}
\end{tikzpicture}
\captionof{figure}{એમ્પ્લીટ્યુડ મોડ્યુલેશન વેવફોર્મ્સ}
\end{center}

\textbf{AM વોલ્ટેજ સમીકરણની તારવણી:}

\begin{enumerate}
    \item ધારો કે કેરિયર સિગ્નલનું તત્કાલીન મૂલ્ય છે:
    \[ v_c(t) = V_c \sin(\omega_c t) \]
    
    \item ધારો કે મોડ્યુલેટિંગ (મેસેજ) સિગ્નલનું તત્કાલીન મૂલ્ય છે:
    \[ v_m(t) = V_m \sin(\omega_m t) \]
    
    \item AM માં, કેરિયરનું એમ્પ્લીટ્યુડ $V_c$ મેસેજ સિગ્નલ $v_m(t)$ મુજબ બદલાય છે. મોડ્યુલેટેડ વેવ $A(t)$ નું તત્કાલીન એમ્પ્લીટ્યુડ બને છે:
    \[ A(t) = V_c + v_m(t) = V_c + V_m \sin(\omega_m t) \]
    \[ A(t) = V_c [1 + \frac{V_m}{V_c} \sin(\omega_m t)] \]
    
    \item વ્યાખ્યાયિત કરો \textbf{મોડ્યુલેશન ઈન્ડેક્સ} ($\mu$):
    \[ \mu = \frac{V_m}{V_c} \]
    તેથી, $A(t) = V_c [1 + \mu \sin(\omega_m t)]$
    
    \item AM વેવ $v_{AM}(t)$ નું તત્કાલીન મૂલ્ય છે:
    \[ v_{AM}(t) = A(t) \sin(\omega_c t) \]
    
    \item $A(t)$ ની કિંમત મુકતા:
    \[ v_{AM}(t) = V_c [1 + \mu \sin(\omega_m t)] \sin(\omega_c t) \]
    આ AM સિગ્નલનું પ્રમાણભૂત સમીકરણ છે.
\end{enumerate}
\end{solutionbox}

\begin{mnemonicbox}
\mnemonic{Amplitude Modulation Makes Carrier Value Change}
\end{mnemonicbox}

\questionmarks{1}{c}{7}
\textbf{અથવા: નોઈઝ વ્યાખ્યાયિત કરો. નોઈઝનું વર્ગીકરણ આપો અને કોઈપણ ત્રણ આંતરિક નોઈઝના કારણો સમજાવો.}

\begin{solutionbox}
\textbf{નોઈઝ (Noise)}: અનિચ્છનીય વિદ્યુત સિગ્નલો જે ઇચ્છિત સંચાર સિગ્નલોના ટ્રાન્સમિશન અને પ્રોસેસિંગમાં દખલ કરે છે, જેના કારણે વિકૃતિ, ભૂલો અથવા માહિતીનું નુકસાન થાય છે.

\textbf{નોઈઝનું વર્ગીકરણ:}

\begin{center}
\begin{tabulary}{\linewidth}{|L|L|}
\hline
\textbf{બાહ્ય નોઈઝ (External Noise)} & \textbf{આંતરિક નોઈઝ (Internal Noise)} \\ \hline
વાતાવરણીય નોઈઝ (Atmospheric) & થર્મલ (જ્હોન્સન) નોઈઝ \\ 
બહારની દુનિયાના (સૌર/કોસ્મિક) નોઈઝ & શોટ નોઈઝ (Shot Noise) \\ 
ઔદ્યોગિક (માનવસર્જિત) નોઈઝ & ટ્રાન્ઝિટ-ટાઈમ નોઈઝ \\ 
 & ફ્લિકર (1/f) નોઈઝ \\ 
 & પાર્ટીશન નોઈઝ \\ \hline
\end{tabulary}
\captionof{table}{નોઈઝનું વર્ગીકરણ}
\end{center}

\textbf{આંતરિક નોઈઝના કારણો:}

\begin{enumerate}
    \item \textbf{થર્મલ (જ્હોન્સન) નોઈઝ}:
    \begin{itemize}
        \item \textbf{કારણ}: વાહક અથવા અવરોધકની અંદર મુક્ત ઇલેક્ટ્રોનની રેન્ડમ થર્મલ ગતિ દ્વારા ઉત્પન્ન થાય છે.
        \item \textbf{લક્ષણ}: તે તમામ પ્રતિરોધક ઘટકોમાં હાજર છે અને નિરપેક્ષ તાપમાન ($T$) અને બેન્ડવિડ્થ ($B$) ના સમપ્રમાણમાં છે.
        \item પાવર $P_n = kTB$.
    \end{itemize}
    
    \item \textbf{શોટ નોઈઝ (Shot Noise)}:
    \begin{itemize}
        \item \textbf{કારણ}: ચાર્જ કેરિયર્સ (ઇલેક્ટ્રોન/હોલ્સ) ની અલગ (discrete) પ્રકૃતિમાંથી ઉદ્ભવે છે. તે સંભવિત અવરોધને પાર કરતા કેરિયર્સના આગમન દરમાં રેન્ડમ વધઘટને કારણે થાય છે (ઉદાહરણ તરીકે, PN જંકશનમાં).
        \item \textbf{લક્ષણ}: તે ડાયોડ અને ટ્રાન્ઝિસ્ટર જેવા સક્રિય ઉપકરણોમાં હાજર છે. તે ઉપકરણમાંથી વહેતા ડીસી પ્રવાહના સમપ્રમાણમાં છે.
    \end{itemize}
    
    \item \textbf{ફ્લિકર નોઈઝ (1/f Noise)}:
    \begin{itemize}
        \item \textbf{કારણ}: સેમિકન્ડક્ટર સામગ્રીમાં સપાટીની ખામીઓ, દૂષણ અને અશુદ્ધિઓને કારણે કેરિયર ઘનતામાં ફેરફારને કારણે થાય છે.
        \item \textbf{લક્ષણ}: તેની પાવર સ્પેક્ટ્રલ ઘનતા ફ્રીક્વન્સી ($1/f$) ના વ્યસ્ત પ્રમાણમાં છે, જે તેને ઓછી ફ્રીક્વન્સી (થોડા kHz થી નીચે) પર નોંધપાત્ર બનાવે છે.
    \end{itemize}
\end{enumerate}
\end{solutionbox}

\begin{mnemonicbox}
\mnemonic{This Shooting Flicker Is Noisy Everywhere}
\end{mnemonicbox}

\questionmarks{2}{a}{3}
\textbf{વ્યાખ્યાયિત કરો (1) AM માટે મોડ્યુલેશન ઇન્ડેક્સ (2) નોઈઝ ફિગર (3) ડિજિટલ મોડ્યુલેશન}

\begin{solutionbox}
\begin{enumerate}
    \item \textbf{AM માટે મોડ્યુલેશન ઇન્ડેક્સ ($\mu$)}:
    \begin{itemize}
        \item મોડ્યુલેટિંગ સિગ્નલ ($V_m$) ના પીક એમ્પ્લીટ્યુડ અને કેરિયર સિગ્નલ ($V_c$) ના પીક એમ્પ્લીટ્યુડના ગુણોત્તર તરીકે વ્યાખ્યાયિત કરવામાં આવે છે.
        \item સૂત્ર: $\mu = \frac{V_m}{V_c}$
        \item વ્યવહારુ શ્રેણી: $0 \le \mu \le 1$ ઓવરમોડ્યુલેશન ડિસ્ટોર્શન ટાળવા માટે.
    \end{itemize}
    
    \item \textbf{નોઈઝ ફિગર (NF)}:
    \begin{itemize}
        \item મેરિટનો આંકડો જે દર્શાવે છે કે ઉપકરણ (જેમ કે એમ્પ્લીફાયર) સિગ્નલમાં કેટલો નોઈઝ ઉમેરે છે.
        \item ઇનપુટ સિગ્નલ-ટુ-નોઈઝ રેશિયો (SNR) અને આઉટપુટ સિગ્નલ-ટુ-નોઈઝ રેશિયોના ગુણોત્તર તરીકે વ્યાખ્યાયિત કરવામાં આવે છે.
        \item સૂત્ર: $NF = \frac{(SNR)_{input}}{(SNR)_{output}}$
        \item આદર્શ રીતે $NF=1$ (અથવા 0 dB) નોઈઝ-ફ્રી ઉપકરણ માટે. વ્યવહારમાં હંમેશા $\ge 1$.
    \end{itemize}
    
    \item \textbf{ડિજિટલ મોડ્યુલેશન}:
    \begin{itemize}
        \item એક એવી તકનીક જ્યાં ડિજિટલ ડેટા (બાઈનરી 0s અને 1s) નો ઉપયોગ ટ્રાન્સમિશન માટે એનાલોગ કેરિયર સિગ્નલના પરિમાણો (એમ્પ્લીટ્યુડ, ફ્રીક્વન્સી અથવા ફેઝ) ને મોડ્યુલેટ કરવા માટે થાય છે.
        \item ઉદાહરણો: ASK (એમ્પ્લીટ્યુડ શિફ્ટ કીંગ), FSK (ફ્રીક્વન્સી શિફ્ટ કીંગ), PSK (ફેઝ શિફ્ટ કીંગ).
    \end{itemize}
\end{enumerate}
\end{solutionbox}

\begin{mnemonicbox}
\mnemonic{Modulation Measures, Noise Numbers, Digital Data}
\end{mnemonicbox}

\questionmarks{2}{b}{4}
\textbf{કેરિયર પાવર અને મોડ્યુલેશન ઇન્ડેક્સને ડ્યાનમાં લઈને એમ્પ્લીટ્યુડ મોડ્યુલેટેડ સિગ્નલ માટે કુલ ટ્રાન્સમિટ થયેલ પાવરનું સમીકરણ તારવો.}

\begin{solutionbox}
\textbf{AM માં કુલ પાવરની તારવણી:}

\begin{enumerate}
    \item AM વેવ માટેનું સમીકરણ છે:
    \[ v_{AM}(t) = V_c \sin(\omega_c t) + \frac{\mu V_c}{2} \cos(\omega_c - \omega_m)t - \frac{\mu V_c}{2} \cos(\omega_c + \omega_m)t \]
    તેમાં કેરિયર કમ્પોનન્ટ અને બે સાઇડબેન્ડ કમ્પોનન્ટ્સ (LSB અને USB) નો સમાવેશ થાય છે.
    
    \item પાવર $P = \frac{V_{rms}^2}{R}$ દ્વારા આપવામાં આવે છે. લોડ રેઝિસ્ટન્સ $R$ ધારી રહ્યા છીએ:
    
    \item \textbf{કેરિયર પાવર ($P_c$)}:
    પીક વોલ્ટેજ $V_c$ છે, તેથી $V_{rms} = \frac{V_c}{\sqrt{2}}$.
    \[ P_c = \frac{(V_c/\sqrt{2})^2}{R} = \frac{V_c^2}{2R} \]
    
    \item \textbf{સાઇડબેન્ડ પાવર}:
    LSB અને USB બંનેમાં પીક એમ્પ્લીટ્યુડ $\frac{\mu V_c}{2}$ છે.
    \[ V_{sb\_rms} = \frac{\mu V_c}{2\sqrt{2}} \]
     અપર સાઇડબેન્ડમાં પાવર ($P_{USB}$) = લોઅર સાઇડબેન્ડમાં પાવર ($P_{LSB}$):
    \[ P_{SB} = \frac{(\frac{\mu V_c}{2\sqrt{2}})^2}{R} = \frac{\mu^2 V_c^2}{8R} \]
    
    \item સાઇડબેન્ડ પાવર સમીકરણમાં $P_c = \frac{V_c^2}{2R}$ મુકતા:
    \[ P_{SB} = P_c \cdot \frac{\mu^2}{4} \]
    
    \item \textbf{કુલ પાવર ($P_T$)}:
    \[ P_T = P_c + P_{USB} + P_{LSB} \]
    \[ P_T = P_c + P_c \frac{\mu^2}{4} + P_c \frac{\mu^2}{4} \]
    \[ P_T = P_c + P_c \frac{\mu^2}{2} \]
    \[ P_T = P_c \left( 1 + \frac{\mu^2}{2} \right) \]
\end{enumerate}
\end{solutionbox}

\begin{mnemonicbox}
\mnemonic{Power Total = Power Carrier (1 + mu^2/2)}
\end{mnemonicbox}

\questionmarks{2}{c}{7}
\textbf{ડબલ સાઇડબેન્ડ સપ્રેસ્ડ કેરિયર એમ્પ્લીટ્યુડ મોડ્યુલેશનનો મૂળભૂત સિદ્ધાંત સમજાવો. તેનું વોલ્ટેજ સમીકરણ તારવો અને ડાયોડનો ઉપયોગ કરીને માત્ર બેલેન્સ્ડ મોડ્યુલેટર સર્કિટ દોરો.}

\begin{solutionbox}
\textbf{ડબલ સાઇડબેન્ડ સપ્રેસ્ડ કેરિયર (DSBSC) સિદ્ધાંત:}
\begin{itemize}
    \item પ્રમાણભૂત AM માં, કેરિયર કુલ પાવરના આશરે 67\% વાપરે છે પરંતુ તેમાં કોઈ માહિતી હોતી નથી.
    \item DSBSC કેરિયરને દબાવી દે છે (suppresses) અને માત્ર બે સાઇડબેન્ડ્સ (USB અને LSB) ટ્રાન્સમિટ કરે છે, જેમાં વાસ્તવિક માહિતી હોય છે.
    \item \textbf{ફાયદો}: પાવર કાર્યક્ષમતા સુધારે છે અને માહિતી પાવરના વોટ દીઠ બેન્ડવિડ્થ પણ ઘટાડે છે? ના, બેન્ડવિડ્થ સરખી રહે છે, ફક્ત પાવર બચે છે.
    \item \textbf{ગેરફાયદો}: રીસીવર પર જટિલ કોહેરન્ટ ડિટેક્શનની જરૂર પડે છે.
\end{itemize}

\textbf{વોલ્ટેજ સમીકરણ તારવણી:}
\begin{enumerate}
    \item કેરિયર $c(t) = V_c \sin(\omega_c t)$ અને મેસેજ $m(t) = V_m \sin(\omega_m t)$ ધ્યાનમાં લો.
    \item DSBSC એ કેરિયર અને મેસેજ સિગ્નલોનું ગુણાકાર છે:
    \[ v_{DSBSC}(t) = m(t) \cdot c(t) \]
    \[ v_{DSBSC}(t) = [V_m \sin(\omega_m t)] \cdot [V_c \sin(\omega_c t)] \]
    \[ v_{DSBSC}(t) = V_m V_c \sin(\omega_m t) \sin(\omega_c t) \]
    
    \item ત્રિકોણમિતિ ઓળખનો ઉપયોગ કરતા: $2 \sin A \sin B = \cos(A-B) - \cos(A+B)$:
    \[ v_{DSBSC}(t) = \frac{V_m V_c}{2} [\cos(\omega_c - \omega_m)t - \cos(\omega_c + \omega_m)t] \]
    
    \item આ સમીકરણ બે ઘટકો (LSB અને USB) દર્શાવે છે અને $\omega_c$ પર કોઈ કેરિયર ઘટક નથી.
\end{enumerate}

\textbf{ઓસિલેટીંગ ડાયોડ્સનો ઉપયોગ કરીને બેલેન્સ્ડ મોડ્યુલેટર સર્કિટ:}

\begin{center}
\begin{circuitikz}[scale=0.9]
    % Transformers
    \draw (0,0) node[transformer] (T1) {};
    \draw (6,0) node[transformer] (T2) {};
    
    % Diodes
    \draw (T1.A1) -- (T1.A2) node[midway, left] {મોડ્યુલેટિંગ સિગ્નલ $v_m(t)$};
    \draw (T1.B1) to[D, l=$D_1$] (T2.A1);
    \draw (T1.B2) to[D, l=$D_2$] (T2.A2);
    
    % Carrier Injection (Center Taps)
    \coordinate (CT1) at ($(T1.B1)!0.5!(T1.B2)$);
    \coordinate (CT2) at ($(T2.A1)!0.5!(T2.A2)$);
    
    \draw (CT1) -- ++(0,-1.5) to[sV, l=કેરિયર $v_c(t)$] ++(3,0) -- (CT2);
    
    % Output
    \draw (T2.B1) -- (T2.B2) node[midway, right] {આઉટપુટ DSBSC};

    % Connections for transformer symbols
\end{circuitikz}
\captionof{figure}{રિંગ મોડ્યુલેટર / ડાયોડ્સનો ઉપયોગ કરીને બેલેન્સ્ડ મોડ્યુલેટર}
\end{center}
\end{solutionbox}

\begin{mnemonicbox}
\mnemonic{Delete Carrier, Save Bandwidth, Combine Signals}
\end{mnemonicbox}

\questionmarks{2}{a}{3}
\textbf{અથવા: ફક્ત વ્યાખ્યાયિત કરો, રેડિયો રીસીવરના સંદર્ભમાં (1) સંવેદનશીલતા (Sensitivity) (2) સિલેક્ટિવિટી (Selectivity) (3) ફિડેલિટી (Fidelity)}

\begin{solutionbox}
\begin{enumerate}
    \item \textbf{સંવેદનશીલતા (Sensitivity)}:
    \begin{itemize}
        \item રેડિયો રીસીવરની નબળા સિગ્નલો લેવાની અને તેમને ઉપયોગી સ્તર સુધી એમ્પ્લીફાય કરવાની ક્ષમતા.
        \item માઈક્રોવોલ્ટ ($\mu V$) માં માપવામાં આવે છે. નીચું મૂલ્ય એટલે સારી સંવેદનશીલતા (દા.ત., 1 $\mu V$ રીસીવર 10 $\mu V$ કરતા વધુ સંવેદનશીલ છે).
    \end{itemize}
    
    \item \textbf{સિલેક્ટિવિટી (Selectivity)}:
    \begin{itemize}
        \item અન્ય તમામ નજીકના અનિચ્છનીય સિગ્નલોને નકારી કાઢીને ઇચ્છિત ફ્રીક્વન્સી સિગ્નલ પસંદ કરવાની રીસીવરની ક્ષમતા.
        \item ટ્યુન કરેલ સર્કિટના ક્વોલિટી ફેક્ટર ($Q$) અને બેન્ડવિડ્થ દ્વારા નક્કી થાય છે. સાંકડી બેન્ડવિડ્થ ઉચ્ચ પસંદગી સૂચવે છે.
    \end{itemize}
    
    \item \textbf{ફિડેલિટી (Fidelity)}:
    \begin{itemize}
        \item આઉટપુટમાં વિકૃતિ વિના મૂળ મેસેજ સિગ્નલના તમામ ફ્રીક્વન્સી ઘટકોને પુનઃઉત્પાદિત કરવાની રીસીવરની ક્ષમતા.
        \item હાઈ ફિડેલિટી (Hi-Fi) એટલે સંપૂર્ણ ઓડિયો રેન્જનું ચોક્કસ પુનઃઉત્પાદન.
    \end{itemize}
\end{enumerate}
\end{solutionbox}

\begin{mnemonicbox}
\mnemonic{Sensitive Selection Faithfully}
\end{mnemonicbox}

\questionmarks{2}{b}{4}
\textbf{અથવા: એક AM સિગ્નલનો કેરિયર પાવર 1 KW છે અને દરેક સાઇડબેન્ડમાં 200 વોટ છે. મોડ્યુલેશન ઇન્ડેક્સ શોધો.}

\begin{solutionbox}
\textbf{આપેલ:}
\begin{itemize}
    \item કેરિયર પાવર ($P_c$) = 1 KW = 1000 W
    \item દરેક સાઇડબેન્ડમાં પાવર ($P_{SB}$) = 200 W (આનો અર્થ સામાન્ય રીતે \textit{એક} સાઇડબેન્ડમાં પાવર થાય છે)
\end{itemize}

\textbf{શોધવાનું છે:}
\begin{itemize}
    \item મોડ્યુલેશન ઇન્ડેક્સ ($\mu$)
\end{itemize}

\textbf{ઉકેલ:}
\begin{enumerate}
    \item કુલ સાઇડબેન્ડ પાવર ($P_{TSB}$) એ USB અને LSB બંનેમાં પાવરનો સરવાળો છે.
    \[ P_{TSB} = P_{USB} + P_{LSB} = 2 \times P_{SB} \]
    \[ P_{TSB} = 2 \times 200 = 400 \text{ W} \]
    
    \item સાઇડબેન્ડ પાવર અને કેરિયર પાવર વચ્ચેનું સૂત્ર:
    \[ P_{TSB} = P_c \cdot \frac{\mu^2}{2} \]
    વૈકલ્પિક રીતે, સિંગલ સાઇડબેન્ડ સૂત્રનો ઉપયોગ કરીને: $P_{SB} = P_c \cdot \frac{\mu^2}{4}$.
    
    \item કિંમતો મુકતા:
    \[ 400 = 1000 \cdot \frac{\mu^2}{2} \]
    
    \item $\mu^2$ માટે ઉકેલો:
    \[ \frac{\mu^2}{2} = \frac{400}{1000} = 0.4 \]
    \[ \mu^2 = 0.4 \times 2 = 0.8 \]
    
    \item $\mu$ માટે ઉકેલો:
    \[ \mu = \sqrt{0.8} \approx 0.8944 \]
\end{enumerate}

\textbf{જવાબ:} મોડ્યુલેશન ઇન્ડેક્સ આશરે \textbf{0.89} અથવા \textbf{89.4\%} છે.
\end{solutionbox}

\questionmarks{2}{c}{7}
\textbf{અથવા: ઓછામાં ઓછા સાત પરિમાણો/પાસાઓને ધ્યાનમાં લઈને એમ્પ્લીટ્યુડ મોડ્યુલેશનની ફ્રીક્વન્સી મોડ્યુલેશન સાથે સરખામણી કરો.}

\begin{solutionbox}
\begin{center}
\begin{tabulary}{\linewidth}{|L|L|L|}
\hline
\textbf{પરિમાણ} & \textbf{એમ્પ્લીટ્યુડ મોડ્યુલેશન (AM)} & \textbf{ફ્રીક્વન્સી મોડ્યુલેશન (FM)} \\ \hline
1. વ્યાખ્યા & કેરિયરનું એમ્પ્લીટ્યુડ મેસેજ એમ્પ્લીટ્યુડ સાથે બદલાય છે. & કેરિયરની ફ્રીક્વન્સી મેસેજ એમ્પ્લીટ્યુડ સાથે બદલાય છે. \\ \hline
2. મોડ્યુલેશન ઇન્ડેક્સ & $\mu = V_m/V_c$ (શ્રેણી 0 થી 1). & $\beta = \Delta f / f_m$ (સામાન્ય રીતે $>1$). \\ \hline
3. બેન્ડવિડ્થ & ઓછી: $BW = 2f_m$. & વધુ: $BW = 2(f_m + \Delta f)$ (કાર્સનનો નિયમ). \\ \hline
4. નોઈઝ ઇમ્યુનિટી & નબળી. નોઈઝ સીધી એમ્પ્લીટ્યુડ પર અસર કરે છે. & ઉત્તમ. એમ્પ્લીટ્યુડ ભિન્નતા ક્લિપ કરવામાં આવે છે; ફ્રીક્વન્સી માહિતી વહન કરે છે. \\ \hline
5. પાવર કાર્યક્ષમતા & નબળી. કેરિયર 67\% જેટલો પાવર વાપરે છે. & સારી. પાવર અચળ અને કાર્યક્ષમ છે. \\ \hline
6. જટિલતા & સરળ ટ્રાન્સમીટર અને રીસીવર. & જટિલ ટ્રાન્સમીટર અને રીસીવર (PLL, ડિક્ક્રિમિનેટર્સનો ઉપયોગ કરે છે). \\ \hline
7. ફિડેલિટી (ગુણવત્તા) & મધ્યમ ક્ષમતા. & હાઈ ફિડેલિટી (Hi-Fi), સારી સાઉન્ડ ગુણવત્તા. \\ \hline
8. એપ્લિકેશન & LW/MW/SW બ્રોડકાસ્ટિંગ, વિડિયો ટ્રાન્સમિશન. & FM રેડિયો બ્રોડકાસ્ટિંગ, ટીવી ઓડિયો, સેટેલાઇટ. \\ \hline
\end{tabulary}
\captionof{table}{AM અને FM ની સરખામણી}
\end{center}
\end{solutionbox}

\questionmarks{3}{a}{3}
\textbf{અથવા: નીચેના જણાવો. (1) જો મોડ્યુલેટિંગ ફ્રીક્વન્સી 5 KHZ હોય તો મોડ્યુલેટેડ સિગ્નલની બેન્ડવિડ્થ. (2) રેડિયોમાં પસંદ કરેલ સ્ટેશન ફ્રીક્વન્સી 1000 KhZ હોય તો ઇમેજ ફ્રીક્વન્સી (3) જો બેઝબેન્ડ સિગ્નલ ફ્રીક્વન્સી 10 KHz હોય તો સેમ્પલિંગ ફ્રીક્વન્સી.}

\begin{solutionbox}
\begin{enumerate}
    \item \textbf{5 kHz મોડ્યુલેટિંગ ફ્રીક્વન્સી સાથે AM ની બેન્ડવિડ્થ}:
    \begin{itemize}
        \item $f_m = 5 \text{ kHz}$
        \item $BW = 2f_m = 2 \times 5 \text{ kHz} = 10 \text{ kHz}$
    \end{itemize}
    
    \item \textbf{1000 kHz સ્ટેશન માટે ઇમેજ ફ્રીક્વન્સી}:
    \begin{itemize}
        \item સિગ્નલ ફ્રીક્વન્સી $f_s = 1000 \text{ kHz}$
        \item પ્રમાણભૂત ઇન્ટરમીડિએટ ફ્રીક્વન્સી $f_{IF} = 455 \text{ kHz}$
        \item ઇમેજ ફ્રીક્વન્સી $f_{im} = f_s + 2f_{IF}$
        \item $f_{im} = 1000 + 2(455) = 1000 + 910 = 1910 \text{ kHz}$
    \end{itemize}
    
    \item \textbf{10 kHz બેઝબેન્ડ માટે સેમ્પલિંગ ફ્રીક્વન્સી}:
    \begin{itemize}
        \item મહત્તમ ફ્રીક્વન્સી $f_{max} = 10 \text{ kHz}$
        \item નાઈક્વિસ્ટ માપદંડ મુજબ, સેમ્પલિંગ ફ્રિક્વન્સી $f_s \ge 2f_{max}$
        \item $f_s \ge 2 \times 10 \text{ kHz} = 20 \text{ kHz}$
        \item તેથી, લઘુત્તમ આવશ્યક સેમ્પલિંગ ફ્રીક્વન્સી 20 kHz છે.
    \end{itemize}
\end{enumerate}
\end{solutionbox}

\begin{mnemonicbox}
\mnemonic{Bandwidth Doubles, Image Adds Twice-IF, Sampling Needs Twice-Frequency}
\end{mnemonicbox}

\questionmarks{3}{b}{4}
\textbf{અથવા: નીચેના સિગ્નલ દોરો અને તેનું ગાણિતિક સમીકરણ જણાવો. (1) સાઈન વેવ (2) યુનિટ સ્ટેપ સિગ્નલ (3) રેમ્પ સિગ્નલ (4) ઇમ્પલ્સ સિગ્નલ.}

\begin{solutionbox}
\textbf{સિગ્નલ રજૂઆત:}

\begin{center}
\begin{longtable}{|c|c|c|}
\hline
\textbf{સિગ્નલ નામ} & \textbf{ગાણિતિક સમીકરણ} & \textbf{વેવફોર્મ} \\ \hline
1. સાઈન વેવ & $f(t) = A \sin(\omega t + \phi)$ & 
\raisebox{-0.8\height}{
\begin{tikzpicture}[scale=0.5]
    \draw[->] (0,0) -- (4,0) node[right] {$t$};
    \draw[->] (0,-1.2) -- (0,1.2) node[above] {$f(t)$};
    \draw[domain=0:4, samples=100, blue] plot (\x, {sin(180*\x)});
\end{tikzpicture}
} \\ \hline

2. યુનિટ સ્ટેપ સિગ્નલ & $u(t) = \begin{cases} 1 & t \ge 0 \\ 0 & t < 0 \end{cases}$ & 
\raisebox{-0.8\height}{
\begin{tikzpicture}[scale=0.5]
    \draw[->] (-1,0) -- (3,0) node[right] {$t$};
    \draw[->] (0,0) -- (0,1.5) node[above] {$u(t)$};
    \draw[thick, blue] (0,0) -- (0,1) -- (2.5,1);
    \draw[thick, blue] (-1,0) -- (0,0);
    \node at (0,1) [left] {1};
\end{tikzpicture}
} \\ \hline

3. રેમ્પ સિગ્નલ & $r(t) = \begin{cases} t & t \ge 0 \\ 0 & t < 0 \end{cases}$ & 
\raisebox{-0.8\height}{
\begin{tikzpicture}[scale=0.5]
    \draw[->] (-1,0) -- (3,0) node[right] {$t$};
    \draw[->] (0,0) -- (0,2) node[above] {$r(t)$};
    \draw[thick, blue] (0,0) -- (2,2);
    \draw[thick, blue] (-1,0) -- (0,0);
\end{tikzpicture}
} \\ \hline

4. ઇમ્પલ્સ સિગ્નલ & $\delta(t) = \begin{cases} \infty & t = 0 \\ 0 & t \ne 0 \end{cases}$ & 
\raisebox{-0.8\height}{
\begin{tikzpicture}[scale=0.5]
    \draw[->] (-1,0) -- (3,0) node[right] {$t$};
    \draw[->] (0,0) -- (0,1.5) node[above] {$\delta(t)$};
    \draw[thick, ->, blue] (0,0) -- (0,1.2);
    \draw[thick, blue] (-1,0) -- (0,0) -- (2.5,0);
\end{tikzpicture}
} \\ \hline
\end{longtable}
\end{center}
\end{solutionbox}

\begin{mnemonicbox}
\mnemonic{Sine Oscillates, Step Jumps, Ramp Climbs, Impulse Spikes}
\end{mnemonicbox}

\questionmarks{3}{c}{7}
\textbf{અથવા: પ્રિ-એમ્ફેસીસ અને ડી-એમ્ફેસીસ સર્કિટ તેની જરૂરિયાત અને લાક્ષણિક ગ્રાફ સાથે દોરો અને સમજાવો. તેમજ FM રીસીવરની AM રીસીવર સાથે વિગતે સરખામણી કરો.}

\begin{solutionbox}
\textbf{પ્રિ-એમ્ફેસીસ અને ડી-એમ્ફેસીસ:}
\begin{itemize}
    \item \textbf{જરૂરિયાત}: FM માં, મેસેજ સિગ્નલના ઉચ્ચ-આવર્તન (high-frequency) ઘટકો લો-ફ્રીક્વન્સી ઘટકોની સરખામણીમાં નીચો મોડ્યુલેશન ઇન્ડેક્સ ધરાવે છે, જે તેમને નોઈઝ માટે વધુ સંવેદનશીલ બનાવે છે (કારણ કે FM ડિમોડ્યુલેટર આઉટપુટમાં ફ્રીક્વન્સી સાથે નોઈઝ પાવર ડેન્સિટી વધે છે).
    \item \textbf{ઉકેલ}: અમે ટ્રાન્સમીટર પર કૃત્રિમ રીતે ઉચ્ચ ફ્રીક્વન્સીને બૂસ્ટ (એમ્પ્લીફાય) કરીએ છીએ (\textbf{પ્રિ-એમ્ફેસીસ}) અને SNR સુધારવા માટે રીસીવર પર તેને અનુરૂપ ઘટાડીએ છીએ (\textbf{ડી-એમ્ફેસીસ}).
\end{itemize}

\textbf{સર્કિટ:}

\begin{center}
\begin{circuitikz}[scale=0.8]
    % Pre-emphasis
    \node at (2, 2.5) {\textbf{પ્રિ-એમ્ફેસીસ (ટ્રાન્સમીટર)}};
    \draw (0,0) node[left] {$V_{in}$} to[short, o-] (1,0); 
    \draw (1,0) -- (1, 1) to[C, l=$C$] (3,1) -- (3,0);
    \draw (1,0) to[R, l=$R_1$] (3,0);
    \draw (3,0) -- (4,0) to[R, l=$R_2$] (4,-2) node[ground] {};
    \draw (3,0) to[short, -o] (5,0) node[right] {$V_{out}$};
    \node at (2.5, -2.5) {સરળ પ્રિ-એમ્ફેસીસ નેટવર્ક};
\end{circuitikz}
\hfill
\begin{circuitikz}[scale=0.8]
    % De-emphasis
    \node at (2, 2.5) {\textbf{ડી-એમ્ફેસીસ (રીસીવર)}};
    \draw (0,0) node[left] {$V_{in}$} to[short, o-] (1,0) to[R, l=$R$] (3,0) -- (4,0) to[short, -o] (5,0) node[right] {$V_{out}$};
    \draw (3,0) to[C, l=$C$] (3,-2) node[ground] {};
    \node at (2.5, -2.5) {સરળ ડી-એમ્ફેસીસ નેટવર્ક};
\end{circuitikz}
\captionof{figure}{પ્રિ-એમ્ફેસીસ અને ડી-એમ્ફેસીસ સર્કિટ}
\end{center}

\textbf{લાક્ષણિક ગ્રાફ:}
\begin{center}
\begin{tikzpicture}[scale=0.7]
    \draw[->] (0,0) -- (6,0) node[right] {ફ્રીક્વન્સી $f$};
    \draw[->] (0,-2) -- (0,2) node[above] {ગેઈન (dB)};
    \draw[thick, blue] (0,0.5) -- (2,0.5) .. controls (4,1) .. (5,1.8) node[right] {પ્રિ-એમ્ફેસીસ};
    \draw[thick, red] (0,-0.5) -- (2,-0.5) .. controls (4,-1) .. (5,-1.8) node[right] {ડી-એમ્ફેસીસ};
    \draw[dashed] (2,-2) -- (2,2) node[above] {$f_{1} = \frac{1}{2\pi RC}$};
    \node at (1,0) {સપાટ રિસ્પોન્સ};
\end{tikzpicture}
\end{center}

\textbf{FM અને AM રીસીવર વચ્ચે તુલના:}
\begin{center}
\begin{tabulary}{\linewidth}{|L|L|L|}
\hline
\textbf{પરિમાણ} & \textbf{AM રીસીવર} & \textbf{FM રીસીવર} \\ \hline
1. ઓપરેટિંગ ફ્રીક્વન્સી & MF/HF રેન્જ (535-1605 kHz) & VHF રેન્જ (88-108 MHz) \\ \hline
2. IF ફ્રીક્વન્સી & 455 kHz & 10.7 MHz \\ \hline
3. બેન્ડવિડ્થ & 10 kHz & 200 kHz \\ \hline
4. ડિમોડ્યુલેશન & એન્વેલોપ ડિટેક્ટર & ડિસ્ક્રિમિનેટર / રેશિયો ડિટેક્ટર \\ \hline
5. એમ્પ્લીટ્યુડ લિમિટર & જરૂરી નથી & એમ્પ્લીટ્યુડ નોઈઝ દૂર કરવા જરૂરી \\ \hline
6. પ્રિ/ડી-એમ્ફેસીસ & વપરાતું નથી & SNR સુધારવા માટે વપરાય છે \\ \hline
7. ઓડિયો ક્વોલિટી & મધ્યમ, મોનો & હાઈ ફિડેલિટી, ઘણીવાર સ્ટીરિયો \\ \hline
\end{tabulary}
\end{center}
\end{solutionbox}

\begin{mnemonicbox}
\mnemonic{Pre Boosts Highs, De Cuts Them; FM Filters Noise Better Than AM}
\end{mnemonicbox}

\questionmarks{4}{a}{3}
\textbf{રેડિયો રીસીવર માટે ઈમેજ આવૃત્તિ નેવ્યાખ્યાયિત કરો અને યોગ્ય ઉદાહરણ સાથે તેને સમજાવો.}

\begin{solutionbox}
\textbf{ઇમેજ ફ્રિક્વન્સી}: એક અનિચ્છનીય ઇનપુટ ફ્રીક્વન્સી જે લોકલ ઓસિલેટર (LO) ફ્રીક્વન્સીથી ઇચ્છિત સિગ્નલ ફ્રીક્વન્સી જેટલી જ સમાન અંતરે સ્થિત છે, જે LO સાથે મિક્સ થતાં તે જ ઇન્ટરમીડિએટ ફ્રીક્વન્સી (IF) ઉત્પન્ન કરે છે.

\textbf{ઉદાહરણ સાથે સમજૂતી:}
\begin{itemize}
    \item ધારો કે ઇચ્છિત સ્ટેશન ફ્રીક્વન્સી $f_s = 1000 \text{ kHz}$.
    \item પ્રમાણભૂત IF $f_{IF} = 455 \text{ kHz}$.
    \item લોકલ ઓસિલેટર ફ્રીક્વન્સી (હાઈ સાઈડ ઇન્જેક્શન) $f_{LO} = f_s + f_{IF} = 1000 + 455 = 1455 \text{ kHz}$.
    \item મિક્સર તફાવત ફ્રીક્વન્સી ઉત્પન્ન કરે છે: $|f_{LO} - f_{in}| = f_{IF}$.
    \item \textbf{કેસ 1}: $f_{in} = f_s = 1000 \text{ kHz} \Rightarrow |1455 - 1000| = 455 \text{ kHz}$ (ઇચ્છિત).
    \item \textbf{કેસ 2}: $f_{in} = f_{si} \text{ (ઇમેજ)} = f_s + 2f_{IF} = 1000 + 2(455) = 1910 \text{ kHz}$.
    \item તપાસો: $|1455 - 1910| = |-455| = 455 \text{ kHz}$.
    \item આમ, જો 1910 kHz પર કોઈ સ્ટેશન અસ્તિત્વમાં હોય, તો તે 1000 kHz સ્ટેશન સાથે દખલ કરશે.
\end{itemize}
સમીકરણ: $f_{si} = f_s + 2f_{IF}$
\end{solutionbox}

\begin{mnemonicbox}
\mnemonic{Image In radio Is Interfering 2IF away}
\end{mnemonicbox}

\questionmarks{4}{b}{4}
\textbf{એમ્પ્લિટ્યુડ મોડ્યુલેટેડ સિગ્નલના ડિમોડ્યુલેશન માટે એન્વેલોપ ડિટેક્ટર ની સર્કિટ દોરો અને તેને સમજાવો.}

\begin{solutionbox}
\textbf{એન્વેલોપ ડિટેક્ટર સર્કિટ:}

\begin{center}
\begin{circuitikz}[scale=1]
    \draw (0,0) node[left] {AM ઇનપુટ} to[short, o-] (1,0) to[D, l=$D$] (3,0) -- (5,0) to[short, -o] (6,0) node[right] {ઓડિયો આઉટપુટ};
    \draw (3,0) to[R, l=$R$, *-*] (3,-2) node[ground] {};
    \draw (4.5,0) to[C, l=$C$, *-*] (4.5,-2) node[ground] {};
\end{circuitikz}
\captionof{figure}{સરળ ડાયોડ એન્વેલોપ ડિટેક્ટર}
\end{center}

\textbf{કાર્યપદ્ધતિ:}
\begin{enumerate}
    \item \textbf{રેક્ટિફિકેશન}: AM ઇનપુટના પોઝિટિવ હાફ-સાયકલ દરમિયાન, ડાયોડ $D$ ફોરવર્ડ બાયસ બને છે અને કેપેસિટર $C$ ને ઇનપુટ વોલ્ટેજના પીક મૂલ્ય સુધી ચાર્જ કરે છે.
    \item \textbf{ડિસ્ચાર્જિંગ}: જેમ ઇનપુટ વોલ્ટેજ પીકથી નીચે આવે છે, ડાયોડ રિવર્સ બાયસ બને છે. કેપેસિટર રેઝિસ્ટર $R$ દ્વારા ધીમે ધીમે ડિસ્ચાર્જ થાય છે.
    \item \textbf{એન્વેલોપ ટ્રેકિંગ}: જો $RC$ ટાઇમ કોન્સ્ટન્ટ યોગ્ય રીતે પસંદ કરવામાં આવે, તો કેપેસિટર વોલ્ટેજ AM વેવના એન્વેલોપ (જે મેસેજ સિગ્નલનું પ્રતિનિધિત્વ કરે છે) ને અનુસરે છે, નહીં કે ઉચ્ચ-આવર્તન RF કેરિયર ભિન્નતાઓને.
    \item \textbf{ટાઇમ કોન્સ્ટન્ટ પસંદગી}: ટાઇમ કોન્સ્ટન્ટ $\tau = RC$ એ સંતોષવું જોઈએ:
    \[ \frac{1}{f_c} \ll RC \ll \frac{1}{f_m} \]
    આ સુનિશ્ચિત કરે છે કે તે કેરિયર ($f_c$) ને ફિલ્ટર કરે છે પરંતુ મેસેજ ($f_m$) ને જાળવી રાખે છે.
\end{enumerate}
\end{solutionbox}

\begin{mnemonicbox}
\mnemonic{Diode Rectifies, RC Smooths Envelope}
\end{mnemonicbox}

\questionmarks{4}{c}{7}
\textbf{એએમ રેડિયો રીસીવરનોબ્લોક ડાયાગ્રામ દોરો અને દરેક બ્લોક/સ્ટેજ ની કામગીરી સમજાવો.}

\begin{solutionbox}
\textbf{સુપરહેટરોડાઇન AM રીસીવર બ્લોક ડાયાગ્રામ:}

\begin{center}
\begin{tikzpicture}[auto, node distance=2.5cm, >=latex]
    \node [gtu block] (ant) {એન્ટેના};
    \node [gtu block, right of=ant] (rf) {RF Amp};
    \node [gtu block, right of=rf] (mixer) {મિક્સર};
    \node [gtu block, right of=mixer] (if) {IF Amp};
    \node [gtu block, below of=if] (det) {ડિટેક્ટર};
    \node [gtu block, left of=det] (af) {AF Amp};
    \node [gtu block, left of=af] (spk) {સ્પીકર};
    \node [gtu block, below of=mixer] (lo) {લોકલ ઓસિલેટર};

    \draw [gtu arrow] (ant) -- (rf);
    \draw [gtu arrow] (rf) -- (mixer);
    \draw [gtu arrow] (mixer) -- (if);
    \draw [gtu arrow] (if) -- (det);
    \draw [gtu arrow] (det) -- (af);
    \draw [gtu arrow] (af) -- (spk);
    \draw [gtu arrow] (lo) -- (mixer);
    \draw [dashed] (rf) -- (lo) node[midway, left] {ગેંગ ટ્યુનિંગ};
\end{tikzpicture}
\captionof{figure}{સુપરહેટરોડાઇન AM રીસીવર}
\end{center}

\textbf{દરેક બ્લોકનાં કાર્યો:}
\begin{itemize}
    \item \textbf{RF એમ્પ્લીફાયર}: ઇચ્છિત સ્ટેશન ફ્રીક્વન્સી ($f_s$) પસંદ કરે છે અને અન્યને નકારે છે. સિગ્નલ-ટુ-નોઈઝ રેશિયો સુધારે છે.
    \item \textbf{લોકલ ઓસિલેટર}: એક ઉચ્ચ-આવર્તન સાઈન વેવ ($f_{LO}$) જનરેટ કરે છે જેથી $f_{LO} = f_s + f_{IF}$. તે RF સ્ટેજના ટ્યુનિંગને ટ્રેક કરે છે.
    \item \textbf{મિક્સર}: હેટરોડાઇનિંગ (બીટ ફ્રીક્વન્સી) ના સિદ્ધાંતનો ઉપયોગ કરીને ઇન્ટરમીડિએટ ફ્રીક્વન્સી ($f_{IF} = 455 \text{ kHz}$) ઉત્પન્ન કરવા માટે $f_s$ અને $f_{LO}$ ને મિક્સ કરે છે.
    \item \textbf{IF એમ્પ્લીફાયર}: 455 kHz પર ફિક્સ કરેલ હાઇ-ગેઇન ટ્યુન એમ્પ્લીફાયર. તે રીસીવરનો મોટાભાગનો ગેઇન અને સિલેક્ટિવિટી પ્રદાન કરે છે.
    \item \textbf{ડિટેક્ટર}: મૂળ ઓડિયો મેસેજ સિગ્નલ પુનઃપ્રાપ્ત કરવા માટે સતત-IF AM સિગ્નલને ડિમોડ્યુલેટ કરે છે. સામાન્ય રીતે AGC (ઓટોમેટિક ગેઇન કંટ્રોલ) શામેલ છે.
    \item \textbf{AF એમ્પ્લીફાયર}: લાઉડસ્પીકર ચલાવવા માટે પૂરતા સ્તરે નબળા ઓડિયો સિગ્નલને એમ્પ્લીફાય કરે છે.
    \item \textbf{સ્પીકર}: ટ્રાન્સડ્યુસર જે વિદ્યુત ઓડિયો સિગ્નલોને ધ્વનિ તરંગોમાં રૂપાંતરિત કરે છે.
\end{itemize}
\end{solutionbox}

\begin{mnemonicbox}
\mnemonic{Radio Mixing Intermediate Detected Audio For Speaker}
\end{mnemonicbox}

\questionmarks{4}{a}{3}
\textbf{અથવા: સિગ્નલના સેમ્પલીંગ લેવા માટેના નાઈક્વિસ્ટ માપદંડ જણાવો અને સમજાવો.}

\begin{solutionbox}
\textbf{નાઈક્વિસ્ટ સેમ્પલિંગ પ્રમેય:}
\begin{quote}
"સતત-સમયના સિગ્નલને તેના સેમ્પલ્સમાંથી સંપૂર્ણપણે પુનઃનિર્માણ કરી શકાય છે જો અને માત્ર જો સેમ્પલિંગ ફ્રીક્વન્સી ($f_s$) સિગ્નલમાં હાજર મહત્તમ ફ્રીક્વન્સી ઘટક ($f_{max}$) કરતા બમણી અથવા તેનાથી વધુ હોય."
\end{quote}

\textbf{ગાણિતિક શરત:}
\[ f_s \ge 2 f_{max} \]

\textbf{સમજૂતી:}
\begin{itemize}
    \item \textbf{નાઈક્વિસ્ટ રેટ}: લઘુત્તમ આવશ્યક સેમ્પલિંગ રેટ, જે $2 f_{max}$ છે.
    \item \textbf{એલિયાસિંગ}: જો $f_s < 2 f_{max}$ હોય, તો ઉચ્ચ-આવર્તન ઘટકો ઓછી-આવર્તન સ્પેક્ટ્રમમાં "ફોલ્ડ ઓવર" થાય છે, જે એલિયાસિંગ તરીકે ઓળખાતી વિકૃતિનું કારણ બને છે. મૂળ સિગ્નલ પુનઃપ્રાપ્ત કરી શકાતું નથી.
    \item \textbf{ગાર્ડ બેન્ડ}: વ્યવહારમાં, વ્યવહારુ એન્ટી-એલિયાસિંગ ફિલ્ટર્સ માટે $f_s$ ને $2 f_{max}$ કરતા થોડું વધારે પસંદ કરવામાં આવે છે (દા.ત., ઓડિયો માટે $f_s = 44.1 \text{ kHz}$ જ્યાં $f_{max} = 20 \text{ kHz}$ છે).
\end{itemize}
\end{solutionbox}

\begin{mnemonicbox}
\mnemonic{Sample at least Twice as Fast as Highest Frequency}
\end{mnemonicbox}

\questionmarks{4}{b}{4}
\textbf{અથવા: ડેલ્ટા મોડ્યુલેશન માટે સ્લોપ ઓવરલોડ અને ગ્રેન્યુલર નોઈજ સમજાવો.}

\begin{solutionbox}
ડેલ્ટા મોડ્યુલેશન (DM) માં, એનાલોગ સિગ્નલને ફિક્સ સ્ટેપ સાઇઝ ($\delta$) સાથે સીડી (staircase) ફંક્શન દ્વારા અંદાજવામાં આવે છે. બે પ્રકારની ક્વાન્ટાઇઝેશન ભૂલો થાય છે:

\textbf{1. સ્લોપ ઓવરલોડ ડિસ્ટોર્શન:}
\begin{itemize}
    \item \textbf{કારણ}: જ્યારે એનાલોગ ઇનપુટ સિગ્નલ ખૂબ જ ઝડપથી બદલાય છે (વધે છે અથવા ઘટે છે), એટલે કે, તેનો ઢાળ (slope) ઉચ્ચ હોય છે ત્યારે થાય છે.
    \item \textbf{અસર}: સીડીનો અંદાજ ઇનપુટ સિગ્નલના તીવ્ર ઢાળ સાથે ગતિ જાળવી શકતો નથી કારણ કે સ્ટેપ સાઇઝ ખૂબ નાની છે અથવા સેમ્પલિંગ રેટ ખૂબ ઓછો છે.
    \item \textbf{ઉપાય}: સ્ટેપ સાઇઝ ($\delta$) અથવા સેમ્પલિંગ ફ્રીક્વન્સી ($f_s$) વધારો.
\end{itemize}

\textbf{2. ગ્રેન્યુલર નોઈઝ:}
\begin{itemize}
    \item \textbf{કારણ}: જ્યારે એનાલોગ ઇનપુટ સિગ્નલ પ્રમાણમાં સપાટ હોય અથવા ખૂબ ધીમેથી બદલાતું હોય ત્યારે થાય છે.
    \item \textbf{અસર}: સ્ટેપ સાઇઝ $\delta$ દ્વારા સીડી આઉટપુટ સાચા સિગ્નલ લેવલ ઉપર અને નીચે ઓસીલેટ થાય છે, જે મૌન હોય ત્યારે પણ નોઈઝ જેવી ભિન્નતા બનાવે છે.
    \item \textbf{ઉપાય}: સ્ટેપ સાઇઝ ($\delta$) ઘટાડો.
\end{itemize}

\textbf{ટ્રેડ-ઓફ}: સ્ટેપ સાઇઝ વધારવાથી સ્લોપ ઓવરલોડ ઠીક થાય છે પરંતુ ગ્રેન્યુલર નોઈઝ ખરાબ થાય છે, અને તેનાથી ઉલટું. \textbf{એડેપ્ટિવ ડેલ્ટા મોડ્યુલેશન (ADM)} સ્ટેપ સાઇઝને ગતિશીલ રીતે બદલીને આ ઉકેલે છે.
\end{solutionbox}

\begin{mnemonicbox}
\mnemonic{Slopes Need Bigger Steps, Flats Need Smaller Steps}
\end{mnemonicbox}

\questionmarks{4}{c}{7}
\textbf{અથવા: પી.સી.એમ. ટ્રાન્સમિટર અને રીસીવરને દોરો અને વિગતવાર સમજાવો.}

\begin{solutionbox}
\textbf{PCM સિસ્ટમનો બ્લોક ડાયાગ્રામ:}

\begin{center}
\textbf{PCM ટ્રાન્સમીટર}

\begin{tikzpicture}[auto, node distance=2cm, >=latex]
    \node [gtu block] (source) {એનાલોગ સોર્સ};
    \node [gtu block, right of=source, node distance=2.5cm] (lpf) {LPF};
    \node [gtu block, right of=lpf, node distance=2.5cm] (sampler) {સેમ્પલર};
    \node [gtu block, right of=sampler, node distance=2.5cm] (quant) {ક્વાન્ટાઇઝર};
    \node [gtu block, right of=quant, node distance=2.5cm] (enc) {એન્કોડર};
    \node [right of=enc, node distance=2cm] (out) {PCM આઉટ};

    \draw [gtu arrow] (source) -- (lpf);
    \draw [gtu arrow] (lpf) -- (sampler);
    \draw [gtu arrow] (sampler) -- (quant);
    \draw [gtu arrow] (quant) -- (enc);
    \draw [gtu arrow] (enc) -- (out);
\end{tikzpicture}

\vspace{0.5cm}
\textbf{PCM રીસીવર}

\begin{tikzpicture}[auto, node distance=2cm, >=latex]
    \node [left of=dec, node distance=2cm] (in) {PCM ઈન};
    \node [gtu block] (dec) {ડિકોડર};
    \node [gtu block, right of=dec, node distance=3cm] (recon) {રીકન્સ્ટ્રક્શન ફિલ્ટર / LPF};
    \node [right of=recon, node distance=3cm] (dest) {એનાલોગ આઉટ};

    \draw [gtu arrow] (in) -- (dec);
    \draw [gtu arrow] (dec) -- (recon);
    \draw [gtu arrow] (recon) -- (dest);
\end{tikzpicture}
\captionof{figure}{પલ્સ કોડ મોડ્યુલેશન (PCM) બ્લોક ડાયાગ્રામ}
\end{center}

\textbf{બ્લોક્સની સમજૂતી:}
\begin{enumerate}
    \item \textbf{લો પાસ ફિલ્ટર (એન્ટી-એલિયાસિંગ)}: નાઈક્વિસ્ટ માપદંડ ($f_s \ge 2 f_{max}$) ને ચુસ્તપણે સંતોષવા માટે ઇનપુટ સિગ્નલને $f_{max}$ સુધી મર્યાદિત કરે છે.
    \item \textbf{સેમ્પલર (સેમ્પલ અને હોલ્ડ)}: ચોક્કસ સમયની ક્ષણોને ડિસ્ક્રીટ કરે છે. સતત-સમયના સિગ્નલને ડિસ્ક્રીટ-ટાઇમ PAM (પલ્સ એમ્પ્લીટ્યુડ મોડ્યુલેશન) પલ્સમાં રૂપાંતરિત કરે છે.
    \item \textbf{ક્વાન્ટાઇઝર}: એમ્પ્લીટ્યુડને ડિસ્ક્રીટ કરે છે. દરેક સેમ્પલ મૂલ્યને $L$ લેવલના મર્યાદિત સેટમાંથી નજીકના પ્રમાણભૂત વોલ્ટેજ લેવલ પર અંદાજિત કરે છે. આ ક્વાન્ટાઇઝેશન નોઈઝ રજૂ કરે છે.
    \item \textbf{એન્કોડર}: દરેક ક્વાન્ટાઇઝ્ડ લેવલને અનન્ય $n$-બિટ બાઈનરી કોડ વર્ડમાં રૂપાંતરિત કરે છે (દા.ત., 01101).
    \item \textbf{ડિકોડર}: રીસીવર પર બાઈનરી સ્ટ્રીમને પાછા ડિસ્ક્રીટ વોલ્ટેજ લેવલમાં રૂપાંતરિત કરવા માટે વપરાય છે (DAC કામગીરી).
    \item \textbf{રીકન્સ્ટ્રક્શન ફિલ્ટર}: લો-પાસ ફિલ્ટર જે મૂળ સતત એનાલોગ સિગ્નલ પુનઃપ્રાપ્ત કરવા માટે ડિકોડરના સીડી (staircase) આઉટપુટને સ્મૂથ કરે છે.
\end{enumerate}
\end{solutionbox}

\begin{mnemonicbox}
\mnemonic{Sample, Quantize, Encode; Decode, Convert, Reconstruct}
\end{mnemonicbox}

\questionmarks{5}{a}{3}
\textbf{યોગ્ય ઉદાહરણ સાથે બીટ, બીટનો દર અને બૌડ દરને વ્યાખ્યાયિત કરો.}

\begin{solutionbox}
\begin{itemize}
    \item \textbf{બિટ}: ડિજિટલ માહિતીનો મૂળભૂત એકમ, જે 0 અથવા 1 ની બાઈનરી સ્થિતિ દર્શાવે છે.
    \item \textbf{બિટ રેટ ($R_b$)}: ડેટા ટ્રાન્સફરની ઝડપ, જે પ્રતિ સેકન્ડ ટ્રાન્સમિટ થતા બિટ્સની સંખ્યા તરીકે માપવામાં આવે છે.
        \item એકમ: bps (બિટ્સ પ્રતિ સેકન્ડ).
        \item ઉદાહરણ: જો સિસ્ટમ 1 સેકન્ડમાં '101' ટ્રાન્સમિટ કરે છે, તો $R_b = 3$ bps.
    \item \textbf{બૌડ રેટ}: પ્રતિ સેકન્ડ સિગ્નલ ફેરફારોનો દર (સિમ્બોલ દર).
        \item એકમ: Baud.
        \item સંબંધ: $R_b = \text{Baud Rate} \times \text{સિમ્બોલ દીઠ બિટ્સ}$.
        \item ઉદાહરણ: QPSK મોડ્યુલેશનમાં, દરેક સિમ્બોલ 2 બિટ્સ વહન કરે છે. જો બૌડ રેટ 1000 Baud હોય, તો બિટ રેટ 2000 bps છે.
\end{itemize}
\end{solutionbox}

\begin{mnemonicbox}
\mnemonic{Bits Build Data, Baud Brings Symbols}
\end{mnemonicbox}

\questionmarks{5}{b}{4}
\textbf{મલ્ટિપ્લેક્સિંગને વ્યાખ્યાયિત કરો. તેના પ્રકારો જણાવો. યોગ્ય આકૃતિ સાથે ફ્રીક્વન્શી ડીવીજન મલ્ટિપ્લેક્સિંગ સમજાવો.}

\begin{solutionbox}
\textbf{મલ્ટિપ્લેક્સિંગ}: એક જ સંચાર ચેનલ પર દખલગીરી વિના એક સાથે અનેક મેસેજ સિગ્નલો ટ્રાન્સમિટ કરવાની પ્રક્રિયા.

\textbf{પ્રકારો:}
\begin{enumerate}
    \item ફ્રિક્વન્સી ડિવિઝન મલ્ટિપ્લેક્સિંગ (FDM) - એનાલોગ
    \item ટાઇમ ડિવિઝન મલ્ટિપ્લેક્સિંગ (TDM) - ડિજિટલ/એનાલોગ
    \item વેવલેન્થ ડિવિઝન મલ્ટિપ્લેક્સિંગ (WDM) - ઓપ્ટિકલ
\end{enumerate}

\textbf{ફ્રિક્વન્સી ડિવિઝન મલ્ટિપ્લેક્સિંગ (FDM):}

\begin{center}
\begin{tikzpicture}[scale=0.8]
    \draw[->] (0,0) -- (8,0) node[right] {ફ્રીક્વન્સી};
    \draw[->] (0,0) -- (0,2) node[above] {પાવર};
    
    % Channel 1
    \draw[fill=blue!20] (1,0) rectangle (2,1.5);
    \node at (1.5, 0.75) {CH 1};
    \node[below] at (1.5,0) {$f_{c1}$};
    
    % Guard Band
    \node at (2.25, 0.5) {GB};
    
    % Channel 2
    \draw[fill=red!20] (2.5,0) rectangle (3.5,1.5);
    \node at (3, 0.75) {CH 2};
    \node[below] at (3,0) {$f_{c2}$};
    
    % Guard Band
    \node at (3.75, 0.5) {GB};
    
    % Channel 3
    \draw[fill=green!20] (4,0) rectangle (5,1.5);
    \node at (4.5, 0.75) {CH 3};
    \node[below] at (4.5,0) {$f_{c3}$};
    
\end{tikzpicture}
\captionof{figure}{FDM સિસ્ટમનું સ્પેક્ટ્રમ}
\end{center}

\textbf{સમજૂતી:}
\begin{itemize}
    \item ચેનલની કુલ બેન્ડવિડ્થ નોન-ઓવરલેપિંગ ફ્રીક્વન્સી બેન્ડ્સમાં વિભાજિત કરવામાં આવે છે.
    \item દરેક યુઝર સિગ્નલ અલગ કેરિયર ફ્રીક્વન્સી ($f_{c1}, f_{c2}, \dots$) ને મોડ્યુલેટ કરે છે.
    \item \textbf{ગાર્ડ બેન્ડ્સ} (GB) એ ચેનલો વચ્ચે રાખવામાં આવતા બિનઉપયોગી ફ્રીક્વન્સી ગેપ છે જે ક્રોસટોક ઓવરલેપને અટકાવે છે.
    \item બધા સિગ્નલ એક સાથે ટ્રાન્સમિટ થાય છે.
    \item FM/AM રેડિયો બ્રોડકાસ્ટિંગ અને કેબલ ટીવીમાં વપરાય છે.
\end{itemize}
\end{solutionbox}

\begin{mnemonicbox}
\mnemonic{Frequency Divides Multiple Signals Simultaneously}
\end{mnemonicbox}

\questionmarks{5}{c}{7}
\textbf{આકૃતિ સાથે મૂળભૂત PCM-TDM આકૃતિ દોરો અને સમજાવો.}

\begin{solutionbox}
\textbf{PCM-TDM સિસ્ટમ:}
ટાઇમ ડિવિઝન મલ્ટિપ્લેક્સિંગ (TDM) નો ઉપયોગ ઘણીવાર એક જ લિંક પર અનેક ડિજિટાઇઝ્ડ વોઇસ/ડેટા ચેનલો ટ્રાન્સમિટ કરવા માટે પલ્સ કોડ મોડ્યુલેશન (PCM) સાથે થાય છે.

\begin{center}
\begin{tikzpicture}[auto, node distance=2cm, >=latex]
    % Inputs
    \node (in1) {એનાલોગ 1};
    \node [below of=in1, node distance=1cm] (in2) {એનાલોગ 2};
    \node [below of=in2, node distance=1cm] (inN) {એનાલોગ N};
    
    % LPFs
    \node [gtu block, right of=in1, node distance=2.5cm] (lpf1) {LPF};
    \node [gtu block, right of=in2, node distance=2.5cm] (lpf2) {LPF};
    \node [gtu block, right of=inN, node distance=2.5cm] (lpfN) {LPF};
    
    % Mux (Commutator)
    \node [gtu block, right of=lpf2, node distance=3cm, minimum height=3cm] (mux) {કોમ્યુટેટર (MUX)};
    
    % Common PCM Path
    \node [gtu block, right of=mux, node distance=3cm] (pcm) {ADC (PCM એન્કોડર)};
    \node [right of=pcm, node distance=2.5cm] (out) {PCM-TDM સ્ટ્રીમ};
    
    % Connections
    \draw [gtu arrow] (in1) -- (lpf1);
    \draw [gtu arrow] (in2) -- (lpf2);
    \draw [gtu arrow] (inN) -- (lpfN);
    
    \draw [gtu arrow] (lpf1) -- (mux.west |- lpf1);
    \draw [gtu arrow] (lpf2) -- (mux.west |- lpf2);
    \draw [gtu arrow] (lpfN) -- (mux.west |- lpfN);
    
    \draw [gtu arrow] (mux) -- (pcm) node[midway, above] {PAM સ્ટ્રીમ};
    \draw [gtu arrow] (pcm) -- (out);

\end{tikzpicture}
\captionof{figure}{PCM-TDM ટ્રાન્સમીટર બ્લોક ડાયાગ્રામ}
\end{center}

\textbf{ઓપરેશન:}
\begin{enumerate}
    \item \textbf{ઇનપુટ ફિલ્ટરિંગ}: દરેક એનાલોગ ઇનપુટને બેન્ડવિડ્થ મર્યાદિત કરવા માટે લો પાસ ફિલ્ટર (LPF) માંથી પસાર કરવામાં આવે છે.
    \item \textbf{કોમ્યુટેશન (મલ્ટિપ્લેક્સિંગ)}: ઇલેક્ટ્રોનિક સ્વીચ (કોમ્યુટેટર) $f_s$ સેમ્પલિંગ દરે દરેક ઇનપુટ સાથે ક્રમિક રીતે જોડાય છે. આ PAM પલ્સની ઇન્ટરલીવ્ડ ટ્રેન બનાવે છે.
    \item \textbf{એન્કોડિંગ}: આ સિંગલ PAM સ્ટ્રીમ PCM એન્કોડર (ADC) ને આપવામાં આવે છે, જે દરેક પલ્સને ક્વાન્ટાઇઝ કરે છે અને $n$-બિટ ડિજિટલ કોડમાં રૂપાંતરિત કરે છે.
    \item \textbf{ટ્રાન્સમિશન}: પરિણામી હાઇ-સ્પીડ ડિજિટલ સ્ટ્રીમમાં પુનરાવર્તિત ફ્રેમ સ્ટ્રક્ચરમાં ચેનલ 1, પછી ચેનલ 2, વગેરેના બિટ્સ હોય છે.
    \item \textbf{ફ્રેમ}: તમામ ઇનપુટ્સને સ્કેન કરવાનું એક પૂર્ણ ચક્ર TDM ફ્રેમ બનાવે છે. ફ્રેમની શરૂઆતને ઓળખવા માટે સામાન્ય રીતે સિંક્રનાઇઝેશન બિટ્સ ઉમેરવામાં આવે છે.
\end{enumerate}
\end{solutionbox}

\begin{mnemonicbox}
\mnemonic{Pulse Code TDM: Sample, Quantize, Encode, Multiplex}
\end{mnemonicbox}

\questionmarks{5}{a}{3}
\textbf{અથવા: ટીડીએમના પ્રકારો જણાવો અને તેમાંથી કોઈપણ એકને સમજાવો.}

\begin{solutionbox}
\textbf{TDM ના પ્રકારો:}
\begin{enumerate}
    \item સિંક્રોનસ TDM
    \item એસિંક્રોનસ TDM (અથવા સ્ટેટિસ્ટિકલ TDM)
\end{enumerate}

\textbf{સિંક્રોનસ TDM:}
\begin{itemize}
    \item \textbf{ખ્યાલ}: દરેક ઇનપુટ સોર્સને દરેક ફ્રેમમાં એક ફિક્સ ટાઇમ સ્લોટ ફાળવવામાં આવે છે, પછી ભલે સોર્સ પાસે મોકલવા માટે ડેટા હોય કે ન હોય.
    \item \textbf{ઓપરેશન}: મલ્ટિપ્લેક્સર રાઉન્ડ-રોબિન રીતે ઇનપુટ્સને સ્કેન કરે છે. જો ઉપકરણ નિષ્ક્રિય હોય, તો તેનો ટાઇમ સ્લોટ ખાલી ટ્રાન્સમિટ થાય છે (બેન્ડવિડ્થનો બગાડ).
    \item \textbf{ફાયદો}: સરળ ડિઝાઇન, ડેટા ફ્રેગમેન્ટ્સ માટે એડ્રેસિંગ ઓવરહેડની જરૂર નથી (સ્થિતિ માલિક નક્કી કરે છે).
    \item \textbf{ગેરફાયદો}: જો ઘણા ઉપકરણો નિષ્ક્રિય હોય તો બિનકાર્યક્ષમ બેન્ડવિડ્થ વપરાશ.
\end{itemize}
\end{solutionbox}

\begin{mnemonicbox}
\mnemonic{Synchronous Slots Stay Steady}
\end{mnemonicbox}

\questionmarks{5}{b}{4}
\textbf{અથવા: ટીડીએમ (TDM) ને સમજાવો. તેના ફાયદા અને ગેરફાયદા પણ જણાવો.}

\begin{solutionbox}
\textbf{ટાઇમ ડિવિઝન મલ્ટિપ્લેક્સિંગ (TDM):}
એક ડિજિટલ/એનાલોગ તકનીક જ્યાં બહુવિધ લો-સ્પીડ સિગ્નલો એક જ હાઇ-સ્પીડ ટ્રાન્સમિશન ચેનલ શેર કરવા માટે સમયસર ઇન્ટરલીવ્ડ છે. સમય અક્ષ સ્લોટ્સમાં વહેંચાયેલું છે, અને દરેક વપરાશકર્તાને સમયના અપૂર્ણાંક માટે સંપૂર્ણ બેન્ડવિડ્થ મળે છે.

\textbf{ફાયદા:}
\begin{itemize}
    \item ચેનલો વચ્ચે કોઈ ક્રોસટોક નથી (સમયમાં અલગ).
    \item ચેનલની સંપૂર્ણ બેન્ડવિડ્થનો ઉપયોગ કરે છે.
    \item સર્કિટરી ડિજિટલ, વિશ્વસનીય અને એકીકૃત કરવા માટે સરળ (VLSI) છે.
    \item લવચીક: બફરિંગ સાથે વિવિધ ડેટા દરો સંભાળી શકે છે.
\end{itemize}

\textbf{ગેરફાયદા:}
\begin{itemize}
    \item ટ્રાન્સમીટર અને રીસીવર વચ્ચે કડક \textbf{સિંક્રનાઇઝેશન} ની જરૂર છે.
    \item ક્લોક રિકવરી અને ફ્રેમિંગ સર્કિટમાં જટિલતા.
    \item સિંક્રોનસ TDM માં બેન્ડવિડ્થનો બગાડ જો ચેનલો નિષ્ક્રિય હોય.
    \item પ્રચાર વિલંબ (Propagation delay) ટાઇમિંગને અસર કરે છે.
\end{itemize}
\end{solutionbox}

\begin{mnemonicbox}
\mnemonic{Time Divided Multiple signals Save costs But Need Precise timing}
\end{mnemonicbox}

\questionmarks{5}{c}{7}
\textbf{અથવા: લાઇન કોડિંગના ઇચ્છનીય ગુણધર્મો જણાવો. 8 બીટ ડિજીટલ ડેટા 01001110 માટે એકધ્રુવીય RZ, Polar NRZ, અને માન્ચેસ્ટર લાઇન કોડિંગ માટે સમય સંબંધમાં વેવફોર્મ દોરો.}

\begin{solutionbox}
\textbf{લાઇન કોડિંગના ઇચ્છનીય ગુણધર્મો:}
\begin{enumerate}
    \item \textbf{સેલ્ફ-સિંક્રનાઇઝેશન}: ક્લોક રિકવરી માટે પૂરતા ટ્રાન્ઝિશન.
    \item \textbf{કોઈ DC ઘટક નહીં}: AC કપલિંગ (ટ્રાન્સફોર્મર/કેપેસિટર) ની મંજૂરી આપવા માટે.
    \item \textbf{એરર ડિટેક્શન}: બિટ ભૂલો શોધવાની ક્ષમતા.
    \item \textbf{બેન્ડવિડ્થ કાર્યક્ષમતા}: ન્યૂનતમ ચેનલ બેન્ડવિડ્થનો ઉપયોગ કરવો જોઈએ.
    \item \textbf{ઓછો ક્રોસ-ટોક}: ઘટેલી દખલગીરી.
\end{enumerate}

\textbf{ડેટા માટે લાઇન કોડિંગ વેવફોર્મ્સ: 0 1 0 0 1 1 1 0}

\begin{center}
\begin{tikzpicture}[scale=0.8]
    % Grid
    \draw[help lines, dashed, gray!30] (0,0) grid (8,6);
    
    % Bit labels
    \foreach \x/\bit in {0.5/0, 1.5/1, 2.5/0, 3.5/0, 4.5/1, 5.5/1, 6.5/1, 7.5/0} {
        \node at (\x, 6.5) {\bit};
    }
    
    % Unipolar RZ
    \node[left] at (0, 5.5) {યુનિપોલર RZ};
    \draw[thick, blue] (0,5) -- (0.5,5) -- (0.5,5) -- (1,5)
                      (0,5) -- (1,5)
                      (1,5) -- (1,6) -- (1.5,6) -- (1.5,5) -- (2,5)
                      (2,5) -- (3,5)
                      (3,5) -- (4,5)
                      (4,5) -- (4,6) -- (4.5,6) -- (4.5,5) -- (5,5)
                      (5,5) -- (5,6) -- (5.5,6) -- (5.5,5) -- (6,5)
                      (6,5) -- (6,6) -- (6.5,6) -- (6.5,5) -- (7,5)
                      (7,5) -- (8,5);

    % Polar NRZ
    \node[left] at (0, 3.5) {પોલર NRZ (L)};
    \draw[thick, red] (0,3) -- (1,3)
                     -- (1,4) -- (2,4)
                     -- (2,3) -- (3,3)
                     -- (4,3)
                     -- (4,4) -- (5,4)
                     -- (6,4)
                     -- (7,4)
                     -- (7,3) -- (8,3);

    % Manchester
    \node[left] at (0, 1.5) {માન્ચેસ્ટર};
    \draw[thick, orange] (0,1) -- (0.5,1) -- (0.5,2) -- (1,2)
                       -- (1,2) -- (1.5,2) -- (1.5,1) -- (2,1)
                       -- (2,1) -- (2.5,1) -- (2.5,2) -- (3,2)
                       -- (3,1) -- (3.5,1) -- (3.5,2) -- (4,2)
                       -- (4,2) -- (4.5,2) -- (4.5,1) -- (5,1)
                       -- (5,2) -- (5.5,2) -- (5.5,1) -- (6,1)
                       -- (6,2) -- (6.5,2) -- (6.5,1) -- (7,1)
                       -- (7,1) -- (7.5,1) -- (7.5,2) -- (8,2);

\end{tikzpicture}
\captionof{figure}{લાઇન કોડિંગ વેવફોર્મ્સ}
\end{center}
\end{solutionbox}

\begin{mnemonicbox}
\mnemonic{Unipolar Rises then Zeros, Polar Never Returns, Manchester Always Transitions}
\end{mnemonicbox}

\end{document}
\textbf{1 KHZ ના સાઈન વેવને ટાઈમ ડોમેન અને ફ્રીક્વન્સી ડોમેનમાં દોરો અને લેબલ આપો. સિગ્નલના ફ્રીક્વન્સી ડોમેન વિશ્લેષણનો ફાયદો જણાવો.}

\begin{solutionbox}
\textbf{ટાઈમ ડોમેન રિપ્રેઝન્ટેશન (1 kHz):}
\begin{center}
\begin{tikzpicture}[scale=0.8]
    \draw[->] (0,0) -- (6,0) node[right] {સમય (ms)};
    \draw[->] (0,-1.5) -- (0,1.5) node[above] {એમ્પ્લીટ્યુડ};
    \draw[domain=0:6, samples=200, smooth, blue] plot (\x, {sin(360*\x)}); % f=1 means period=1 unit
    \node at (3,-2) {પિરિયડ $T = 1/f = 1ms$};
    % Mark period
    \draw[<->] (0.25, 1.2) -- (1.25, 1.2) node[midway, above] {$T=1ms$};
\end{tikzpicture}
\end{center}

\textbf{ફ્રીક્વન્સી ડોમેન રિપ્રેઝન્ટેશન:}
\begin{center}
\begin{tikzpicture}[scale=0.8]
    \draw[->] (0,0) -- (4,0) node[right] {ફ્રીક્વન્સી (kHz)};
    \draw[->] (0,0) -- (0,2) node[above] {એમ્પ્લીટ્યુડ};
    \draw[thick, red] (1,0) -- (1,1.5); % 1 kHz spike
    \node at (1,-0.3) {1};
    \node at (2,-0.3) {2};
    \node at (3,-0.3) {3};
    \node[right] at (1,1.5) {સ્પેક્ટ્રલ લાઇન};
\end{tikzpicture}
\end{center}

\textbf{ફ્રીક્વન્સી ડોમેન વિશ્લેષણના ફાયદા:}
\begin{itemize}
    \item \textbf{સિગ્નલ વિઘટન}: વ્યક્તિગત ફ્રીક્વન્સી ઘટકો અને બેન્ડવિડ્થના ઉપયોગને સરળતાથી ઓળખે છે.
    \item \textbf{ફિલ્ટર ડિઝાઇન}: ફિલ્ટર્સ (લો પાસ, પણ પાસ) ડિઝાઇન કરવા માટે આવશ્યક છે કારણ કે રિસ્પોન્સ ફ્રીક્વન્સીમાં વ્યાખ્યાયિત છે.
    \item \textbf{બેન્ડવિડ્થ કાર્યક્ષમતા}: સ્પેક્ટ્રમ ઓક્યુપન્સી સમજવામાં અને ચેનલના ઉપયોગને મહત્તમ કરવામાં મદદ કરે છે.
\end{itemize}
\end{solutionbox}

\questionmarks{3}{b}{4}
\textbf{નીચેની ફ્રીક્વન્સી જણાવો (1) AM રેડિયો માટે IF ફ્રીક્વન્સી (2) FM રેડિયો માટે IF ફ્રીક્વન્સી (3) FM રેડિયોમાં વપરાતી ફ્રીક્વન્સી બેન્ડ (4) માનવ વાણીની ફ્રીક્વન્સી બેન્ડ.}

\begin{solutionbox}
\begin{center}
\begin{tabulary}{\linewidth}{|L|L|}
\hline
\textbf{પેરામીટર} & \textbf{પ્રમાણભૂત ફ્રીક્વન્સી મૂલ્ય} \\ \hline
1. AM રેડિયો માટે IF ફ્રીક્વન્સી & 455 kHz \\ \hline
2. FM રેડિયો માટે IF ફ્રીક્વન્સી & 10.7 MHz \\ \hline
3. FM બ્રોડકાસ્ટિંગમાં વપરાતી ફ્રીક્વન્સી બેન્ડ & 88 MHz થી 108 MHz \\ \hline
4. માનવ વાણી (Voice) ની ફ્રીક્વન્સી બેન્ડ & 300 Hz થી 3.4 kHz (ટેલિફોન સ્ટાન્ડર્ડ) \\ \hline
\end{tabulary}
\captionof{table}{સંચારમાં પ્રમાણભૂત ફ્રીક્વન્સી}
\end{center}
\end{solutionbox}

\questionmarks{3}{c}{7}
\textbf{વેવફોર્મ સાથે સિંગલ સાઇડ બેન્ડ (SSB) મોડ્યુલેશન સમજાવો અને તેના ફાયદાઓ જણાવો. બતાવો કે કેવી રીતે SSB ટ્રાન્સમિશનમાં ડબલ સાઇડબેન્ડ ફુલ કેરિયર એમ્પ્લીટ્યુડ મોડ્યુલેશનની સરખામણીમાં માત્ર 1/6 ઠા ભાગના પાવરની જરૂર પડે છે.}

\begin{solutionbox}
\textbf{સિંગલ સાઇડબેન્ડ (SSB) મોડ્યુલેશન:}
\begin{itemize}
    \item SSB એ એમ્પ્લીટ્યુડ મોડ્યુલેશનનું એક સ્વરૂપ છે જ્યાં કેરિયર અને સાઇડબેન્ડ્સમાંથી એકને દબાવવામાં આવે છે.
    \item માત્ર \textbf{એક સાઇડબેન્ડ} ( કાં તો અપર સાઇડબેન્ડ - USB અથવા લોઅર સાઇડબેન્ડ - LSB) ટ્રાન્સમિટ કરવામાં આવે છે.
    \item આ માહિતી ગુમાવ્યા વિના બેન્ડવિડ્થ અને પાવર બચાવે છે.
\end{itemize}

\textbf{SSB સ્પેક્ટ્રમ વેવફોર્મ:}
\begin{center}
\begin{tikzpicture}[scale=0.8]
    % DSB-FC
    \begin{scope}
        \draw[->] (0,0) -- (6,0) node[right] {$f$};
        \draw[->] (0,0) -- (0,2) node[above] {Amp};
        \draw[thick] (3,0) -- (3,1.5) node[above] {$f_c$}; % Carrier
        \draw[thick] (2,0) -- (2,1) node[above] {LSB};
        \draw[thick] (4,0) -- (4,1) node[above] {USB};
        \node at (3,-1) {DSB-FC (AM)};
    \end{scope}

    % SSB
    \begin{scope}[xshift=7cm]
        \draw[->] (0,0) -- (6,0) node[right] {$f$};
        \draw[->] (0,0) -- (0,2) node[above] {Amp};
        \draw[dashed, gray] (3,0) -- (3,1.5); % Suppressed Carrier
        \draw[dashed, gray] (2,0) -- (2,1); % Suppressed LSB
        \draw[thick, blue] (4,0) -- (4,1) node[above] {USB};
        \node at (3,-1) {SSB (USB only)};
    \end{scope}
\end{tikzpicture}
\captionof{figure}{AM અને SSB સ્પેક્ટ્રાની સરખામણી}
\end{center}

\textbf{SSB ના ફાયદા:}
\begin{enumerate}
    \item \textbf{બેન્ડવિડ્થ બચત}: AM ($2f_m$) ની સરખામણીમાં માત્ર અડધી બેન્ડવિડ્થ ($f_m$) ની જરૂર પડે છે.
    \item \textbf{પાવર કાર્યક્ષમતા}: કેરિયર અથવા બિનજરૂરી સાઇડબેન્ડ પર કોઈ પાવર વેડફાતું નથી.
    \item \textbf{ઘટાયેલ ફેડિંગ}: આયનોસ્ફેરિક પ્રચારમાં પસંદગીયુક્ત ફેડિંગ (selective fading) માટે ઓછું સંવેદનશીલ છે.
    \item \textbf{વધુ સારું SNR}: ટ્રાન્સમિશન પાવર માહિતી ધરાવતા સાઇડબેન્ડમાં કેન્દ્રિત છે.
\end{enumerate}

\textbf{પાવર બચત ગણતરી (1/6 પાવર):}
\begin{enumerate}
    \item 100\% મોડ્યુલેશન ($\mu=1$) સાથે પ્રમાણભૂત AM (DSB-FC) માં કુલ પાવર:
    \[ P_{AM} = P_c (1 + \frac{\mu^2}{2}) = P_c (1 + \frac{1}{2}) = 1.5 P_c \]
    
    \item AM માં પાવર વિતરણ:
    \begin{itemize}
        \item કેરિયર પાવર = $P_c$
        \item કુલ સાઇડબેન્ડ પાવર = $0.5 P_c$
        \item સાઇડબેન્ડ દીઠ પાવર = $0.25 P_c$
    \end{itemize}
    
    \item SSB માં પાવર (માત્ર એક સાઇડબેન્ડ):
    \[ P_{SSB} = P_{SB\_one} = 0.25 P_c \]
    
    \item SSB પાવર થી AM પાવરનો ગુણોત્તર:
    \[ \frac{P_{SSB}}{P_{AM}} = \frac{0.25 P_c}{1.5 P_c} = \frac{1/4}{3/2} = \frac{1}{4} \times \frac{2}{3} = \frac{2}{12} = \frac{1}{6} \]
    
    \item \textbf{નિષ્કર્ષ}: સમાન સિગ્નલ-ટુ-નોઈઝ રેશિયો સાથે સમાન માહિતી ટ્રાન્સમિટ કરવા માટે DSB-FC AM માટે જરૂરી પાવરના માત્ર \textbf{1/6 ઠા (16.7\%)} ભાગની જરૂર SSB ને પડે છે.
\end{enumerate}
\end{solutionbox}
