\documentclass[10pt,a4paper]{article}

% content/resources/templates/preamble.tex
\usepackage[margin=0.6in]{geometry}
\author{Milav Dabgar}
\usepackage{amsmath,amssymb,amsthm}
\usepackage{booktabs}
\usepackage{multirow}
\usepackage{xcolor}
\usepackage{tcolorbox}
\tcbuselibrary{breakable,skins}
\usepackage[colorlinks=true,linkcolor=blue]{hyperref}
\usepackage{titlesec}
\usepackage{enumitem}
\usepackage{tikz}
\usepackage{pgfplots}
\usepackage{circuitikz}
\usepackage[version=4]{mhchem}
\usepackage{longtable}
\usepackage{array}
\usepackage{float}
\usepackage{caption}
\usepackage{listings}

\lstset{
  basicstyle=\small\ttfamily,
  breaklines=true,
  breakatwhitespace=false,
  postbreak=\mbox{\textcolor{red}{$\hookrightarrow$}\space},
  float=false,
  numbers=left,
  numberstyle=\tiny\color{gray},
  numbersep=10pt,
  xleftmargin=2em,
  keywordstyle=\color{blue},
  commentstyle=\color{green!60!black},
  stringstyle=\color{purple},
  backgroundcolor=\color{gray!5},
  showstringspaces=false,
  tabsize=2,
  captionpos=b,
  keepspaces=true,
  columns=flexible
}

\pgfplotsset{compat=1.18}
\usetikzlibrary{shapes,arrows,positioning,calc,patterns,decorations.pathmorphing,decorations.markings,arrows.meta}

% Color scheme
\definecolor{headcolor}{RGB}{0,102,204}
\definecolor{keycolor}{RGB}{220,20,60}
\definecolor{solutioncolor}{RGB}{34,139,34}
\definecolor{mnemoniccolor}{RGB}{148,0,211}
\definecolor{codecolor}{RGB}{0,0,100}

% Spacing
\setlength{\parskip}{3pt}
\setlist[itemize]{nosep}
\setlist[enumerate]{nosep}

% Title formatting
\titleformat{\section}{\Large\bfseries\color{headcolor}}{\thesection}{1em}{}
\titleformat{\subsection}{\large\bfseries\color{headcolor}}{\thesubsection}{1em}{}

% Pandoc tightlist compatibility
\providecommand{\tightlist}{%
  \setlength{\itemsep}{0pt}\setlength{\parskip}{0pt}}

% Pandoc longtable compatibility
\newcounter{none}
\def\thenone{}


% content/resources/templates/english-boxes.tex
% This file is currently empty - it exists to maintain consistency with the import structure.
% Add custom environments here if needed in the future.


\begin{document}

\begin{center}
{\Huge\bfseries\color{headcolor} Subject Name Solutions}\\[5pt]
{\LARGE 4331104 -- Winter 2022}\\[3pt]
{\large Semester 1 Study Material}\\[3pt]
{\normalsize\textit{Detailed Solutions and Explanations}}
\end{center}

\vspace{10pt}

\subsection*{Question 1(a) [3 marks]}\label{q1a}

\textbf{What is modulation? What is the need of it?}

\begin{solutionbox}
Modulation is the process of varying one or more
properties (amplitude, frequency, or phase) of a high-frequency carrier
signal with a modulating signal containing information.

\textbf{Need for modulation:}

\begin{itemize}
\tightlist
\item
  \textbf{Antenna size reduction}: Makes practical antenna size possible
  (λ = c/f)
\item
  \textbf{Multiplexing}: Allows multiple signals to share the medium
\item
  \textbf{Noise reduction}: Improves SNR by shifting to higher frequency
  bands
\item
  \textbf{Range extension}: Increases transmission distance
\end{itemize}

\end{solutionbox}
\begin{mnemonicbox}
``AMEN'' - Antenna size, Multiplexing, Eliminate
noise, New range

\end{mnemonicbox}
\subsection*{Question 1(b) [4 marks]}\label{q1b}

\textbf{Derive voltage equation for Amplitude modulation.}

\begin{solutionbox}
For AM, the carrier signal is modulated by the message
signal.

\textbf{Mathematical derivation:}

\begin{itemize}
\tightlist
\item
  \textbf{Carrier signal}: \(e_c(t) = A_c \cos(2\pi f_c t)\)
\item
  \textbf{Message signal}: \(e_m(t) = A_m \cos(2\pi f_m t)\)
\item
  \textbf{Instantaneous amplitude}: \(A_i = A_c + e_m(t)\)
\item
  \textbf{AM signal}: \(e_{AM}(t) = A_i \cos(2\pi f_c t)\)
\item
  \textbf{Substituting}:
  \(e_{AM}(t) = [A_c + A_m \cos(2\pi f_m t)] \cos(2\pi f_c t)\)
\item
  \textbf{Expanding}:
  \(e_{AM}(t) = A_c\cos(2\pi f_c t) + A_m\cos(2\pi f_m t)\cos(2\pi f_c t)\)
\item
  \textbf{Final equation}:
  \(e_{AM}(t) = A_c\cos(2\pi f_c t) + \frac{A_m}{2}\cos(2\pi(f_c+f_m)t) + \frac{A_m}{2}\cos(2\pi(f_c-f_m)t)\)
\end{itemize}

\end{solutionbox}
\begin{mnemonicbox}
``CAT'' - Carrier, Addition, Three components
(carrier + 2 sidebands)

\end{mnemonicbox}
\subsection*{Question 1(c) [7 marks]}\label{q1c}

\textbf{Classify Noise signal and explain flicker noise, shot noise and
thermal noise.}

\begin{solutionbox}

\textbf{Noise classification:}

{\def\LTcaptype{none} % do not increment counter
\begin{longtable}[]{@{}
  >{\raggedright\arraybackslash}p{(\linewidth - 4\tabcolsep) * \real{0.1875}}
  >{\raggedright\arraybackslash}p{(\linewidth - 4\tabcolsep) * \real{0.2812}}
  >{\raggedright\arraybackslash}p{(\linewidth - 4\tabcolsep) * \real{0.5312}}@{}}
\toprule\noalign{}
\begin{minipage}[b]{\linewidth}\raggedright
Type
\end{minipage} & \begin{minipage}[b]{\linewidth}\raggedright
Sources
\end{minipage} & \begin{minipage}[b]{\linewidth}\raggedright
Characteristics
\end{minipage} \\
\midrule\noalign{}
\endhead
\bottomrule\noalign{}
\endlastfoot
\textbf{External Noise} & Atmospheric, Space, Industrial, Man-made &
Originates outside communication system \\
\textbf{Internal Noise} & Thermal, Shot, Transit-time, Flicker &
Originates inside components \\
\end{longtable}
}

\textbf{Types of Internal Noise:}

\begin{itemize}
\tightlist
\item
  \textbf{Flicker Noise}:

  \begin{itemize}
  \tightlist
  \item
    Occurs at low frequencies (below 1 kHz)
  \item
    Inversely proportional to frequency (1/f noise)
  \item
    Common in semiconductor devices and carbon resistors
  \end{itemize}
\item
  \textbf{Shot Noise}:

  \begin{itemize}
  \tightlist
  \item
    Caused by random fluctuations of current carriers
  \item
    White noise with constant power density
  \item
    Occurs in active devices like diodes and transistors
  \end{itemize}
\item
  \textbf{Thermal Noise}:

  \begin{itemize}
  \tightlist
  \item
    Due to random motion of electrons in a conductor
  \item
    Directly proportional to temperature and bandwidth
  \item
    Present in all passive components
  \item
    Also called Johnson noise or white noise
  \end{itemize}
\end{itemize}

\end{solutionbox}
\begin{mnemonicbox}
``FAST'' - Flicker (low frequency), Active (shot),
Semiconductor (flicker), Temperature (thermal)

\end{mnemonicbox}
\subsection*{Question 1(c) OR [7
marks]}\label{q1c}

\textbf{Write application of different band of EM wave spectrum.}

\begin{solutionbox}

\textbf{EM Spectrum Applications:}

{\def\LTcaptype{none} % do not increment counter
\begin{longtable}[]{@{}
  >{\raggedright\arraybackslash}p{(\linewidth - 4\tabcolsep) * \real{0.3478}}
  >{\raggedright\arraybackslash}p{(\linewidth - 4\tabcolsep) * \real{0.3478}}
  >{\raggedright\arraybackslash}p{(\linewidth - 4\tabcolsep) * \real{0.3043}}@{}}
\toprule\noalign{}
\begin{minipage}[b]{\linewidth}\raggedright
Frequency Band
\end{minipage} & \begin{minipage}[b]{\linewidth}\raggedright
Frequency Range
\end{minipage} & \begin{minipage}[b]{\linewidth}\raggedright
Applications
\end{minipage} \\
\midrule\noalign{}
\endhead
\bottomrule\noalign{}
\endlastfoot
\textbf{ELF} (Extremely Low Frequency) & 3Hz - 30Hz & Submarine
communication \\
\textbf{VLF} (Very Low Frequency) & 3kHz - 30kHz & Navigation, time
signals \\
\textbf{LF} (Low Frequency) & 30kHz - 300kHz & AM radio, navigation \\
\textbf{MF} (Medium Frequency) & 300kHz - 3MHz & AM broadcasting,
maritime \\
\textbf{HF} (High Frequency) & 3MHz - 30MHz & Shortwave radio, amateur
radio \\
\textbf{VHF} (Very High Frequency) & 30MHz - 300MHz & FM radio, TV
broadcasting, air traffic control \\
\textbf{UHF} (Ultra High Frequency) & 300MHz - 3GHz & TV broadcasting,
mobile phones, WiFi, Bluetooth \\
\textbf{SHF} (Super High Frequency) & 3GHz - 30GHz & Satellite
communication, radar, WiFi \\
\textbf{EHF} (Extremely High Frequency) & 30GHz - 300GHz & Radio
astronomy, 5G, millimeter-wave radar \\
\textbf{Infrared} & 300GHz - 400THz & Remote controls, thermal imaging,
fiber optics \\
\textbf{Visible Light} & 400THz - 800THz & Fiber optics, LiFi,
photography \\
\textbf{Ultraviolet} & 800THz - 30PHz & Sterilization, fluorescence,
security \\
\textbf{X-rays} & 30PHz - 30EHz & Medical imaging, security screening \\
\textbf{Gamma rays} & \textgreater30EHz & Medical treatments, nuclear
detection \\
\end{longtable}
}

\end{solutionbox}
\begin{mnemonicbox}
``Every Very Lovely Monkey Has Visited Uncle Sam's
House Easily In Visible Upper Xtra Gamma'' (first letter of each band)

\end{mnemonicbox}
\subsection*{Question 2(a) [3 marks]}\label{q2a}

\textbf{State advantages of SSB over DSB.}

\begin{solutionbox}

\textbf{Advantages of SSB over DSB:}

{\def\LTcaptype{none} % do not increment counter
\begin{longtable}[]{@{}
  >{\raggedright\arraybackslash}p{(\linewidth - 2\tabcolsep) * \real{0.4583}}
  >{\raggedright\arraybackslash}p{(\linewidth - 2\tabcolsep) * \real{0.5417}}@{}}
\toprule\noalign{}
\begin{minipage}[b]{\linewidth}\raggedright
Advantage
\end{minipage} & \begin{minipage}[b]{\linewidth}\raggedright
Description
\end{minipage} \\
\midrule\noalign{}
\endhead
\bottomrule\noalign{}
\endlastfoot
\textbf{Bandwidth Efficiency} & Uses half the bandwidth (only one
sideband) \\
\textbf{Power Efficiency} & Requires less transmitter power (83.33\%
power saving) \\
\textbf{Reduced Fading} & Less susceptible to selective fading \\
\textbf{Less Distortion} & Reduced intermodulation distortion \\
\textbf{Simplified Receiver} & Simpler circuit design possible \\
\end{longtable}
}

\end{solutionbox}
\begin{mnemonicbox}
``BPFDS'' - Bandwidth, Power, Fading, Distortion,
Simple

\end{mnemonicbox}
\subsection*{Question 2(b) [4 marks]}\label{q2b}

\textbf{Explain generation of FM using Phase lock loop technique.}

\begin{solutionbox}

\textbf{FM Generation using PLL:}

A Phase-Locked Loop (PLL) generates FM signals by applying the
modulating signal to the VCO control input.

\textbf{PLL FM Modulator:}

\begin{center}
\textbf{Mermaid Diagram (Code)}
\begin{verbatim}
{Shaded}
{Highlighting}[]
graph LR
    A[Modulating Signal] {-{-}{} B[Summing Circuit]}
    E[Reference Oscillator] {-{-}{} F[Phase Detector]}
    F {-{-}{} G[Low Pass Filter]}
    G {-{-}{} B}
    B {-{-}{} H[VCO]}
    H {-{-}{} I[FM Output]}
    H {-{-}{} J[Feedback]}
    J {-{-}{} F}
{Highlighting}
{Shaded}
\end{verbatim}
\end{center}

\textbf{Operation:}

\begin{itemize}
\tightlist
\item
  \textbf{Reference Oscillator}: Provides stable reference frequency
\item
  \textbf{Phase Detector}: Compares reference and feedback signals
\item
  \textbf{Low Pass Filter}: Removes high-frequency components
\item
  \textbf{VCO}: Generates output frequency that varies with control
  voltage
\item
  \textbf{Modulating Signal}: Added to control voltage to produce FM
  output
\end{itemize}

\end{solutionbox}
\begin{mnemonicbox}
``PROVE'' - Phase detector, Reference oscillator,
Output VCO, Voltage controlled

\end{mnemonicbox}
\subsection*{Question 2(c) [7 marks]}\label{q2c}

\textbf{Derive the equation for total power in AM, calculate percentage
of power savings in DSB and SSB.}

\begin{solutionbox}

\textbf{Power in AM:}

The AM wave equation:
\(e_{AM}(t) = A_c[1 + m\cos(2\pi f_m t)]\cos(2\pi f_c t)\)

\textbf{Power derivation:}

\begin{itemize}
\tightlist
\item
  \textbf{Total power}: \(P_T = P_c\left(1 + \frac{m^2}{2}\right)\)
\item
  Where \(P_c = \frac{A_c^2}{2R}\) (carrier power) and \(m\) is
  modulation index
\end{itemize}

\textbf{Power distribution:}

\begin{itemize}
\tightlist
\item
  \textbf{Carrier power}: \(P_c = \frac{A_c^2}{2R}\)
\item
  \textbf{Total sideband power}: \(P_{SB} = \frac{m^2 P_c}{2}\)
\item
  \textbf{Each sideband}: \(P_{LSB} = P_{USB} = \frac{m^2 P_c}{4}\)
\end{itemize}

\textbf{Power savings:}

\begin{itemize}
\tightlist
\item
  \textbf{In DSB-SC}: No carrier power, so savings =
  \(\frac{P_c}{P_T} \times 100\% = \frac{1}{1+\frac{m^2}{2}} \times 100\%\)

  \begin{itemize}
  \tightlist
  \item
For

m=1, savings = 66.67\%

  \end{itemize}
\item
  \textbf{In SSB}: No carrier and one sideband, so savings =
  \(\frac{P_c + P_{SB}/2}{P_T} \times 100\%\)

  \begin{itemize}
  \tightlist
  \item
For

m=1, savings = 83.33\%

  \end{itemize}
\end{itemize}

\end{solutionbox}
\begin{mnemonicbox}
``CEPTS'' - Carrier Eliminated Provides Tremendous
Savings

\end{mnemonicbox}
\subsection*{Question 2(a) OR [3
marks]}\label{q2a}

\textbf{Draw and explain Time domain and Frequency domain display of AM
wave.}

\begin{solutionbox}

\textbf{Time and Frequency Domain of AM:}

\textbf{Diagram:}

\begin{verbatim}
Time Domain:
    
     +          +           +           +
     |          |           |           |
     |    ++    |     ++    |    ++     |
     |   /  {   |    /     |   /      |}
     |  /    {  |   /      |  /       |}
     | /      { |  /       | /        |}
     |/        {|/        |/        | |}
{-{-}{-}{-}{-}+{-}{-}{-}{-}{-}{-}{-}{-}{-}{-}+{-}{-}{-}{-}{-}{-}{-}{-}{-}{-}{-}+{-}{-}{-}{-}{-}{-}{-}{-}{-}{-}+{-}{-}{-}{-}{-}}
     |{        /|        /|        /| |}
     | {      / |       / |       /  |}
     |  {    /  |      /  |      /   |}
     |   {\_\_/   |   \_\_/   |   \_\_/    |}
     |          |           |           |
     +          +           +           +

Frequency Domain:
    
     |
     |          
     |     +           +           +
     |     |           |           |
     |     |           |           |
     |     |           |           |
     |     |           |           |
{-{-}{-}{-}{-}+{-}{-}{-}{-}{-}+{-}{-}{-}{-}{-}+{-}{-}{-}{-}{-}+{-}{-}{-}{-}{-}+{-}{-}{-}{-}{-}+{-}{-}{-}{-}{-}}
     |   f\_c{-f\_m     f\_c    f\_c+f\_m}
\end{verbatim}

\textbf{Time Domain:}

\begin{itemize}
\tightlist
\item
  Shows amplitude variation of carrier with time
\item
  Envelope follows modulating signal
\item
  Upper and lower envelopes = carrier peak \times (1\pmm)
\end{itemize}

\textbf{Frequency Domain:}

\begin{itemize}
\tightlist
\item
  Shows frequency components and their amplitudes
\item
  Carrier at frequency fc with amplitude Ac
\item
  Two sidebands at fc\pmfm with amplitude mAc/2
\item
  Bandwidth = 2fm (twice the modulating frequency)
\end{itemize}

\end{solutionbox}
\begin{mnemonicbox}
``EBS'' - Envelope in time, Bandwidth in frequency,
Sidebands symmetric

\end{mnemonicbox}
\subsection*{Question 2(b) OR [4
marks]}\label{q2b}

\textbf{Explain pre-emphasis \& de-emphasis circuit.}

\begin{solutionbox}

\textbf{Pre-emphasis and De-emphasis:}

\textbf{Circuit Diagrams:}

\begin{verbatim}
Pre{-emphasis:                   De{-}emphasis:}
    
+{-{-}+     +{-}{-}+                 +{-}{-}+     +{-}{-}+}
|  |     |  |                 |  |     |  |
+{-{-}+     R  |                 +{-}{-}+     R  |}
Input    |  +{-{-}+{-}{-}+  Output   Input    |  +{-}{-}+{-}{-}+  Output}
o{-{-}{-}{-}{-}{-}{-}{-}+  |  |  o{-}{-}{-}{-}{-}{-}{-}{-}   o{-}{-}{-}{-}{-}{-}{-}{-}+  |  |  o{-}{-}{-}{-}{-}{-}{-}{-}}
            C  |                          C  |
            |  |                          |  |
            +{-{-}+                          +{-}{-}+}
              |                             |
            {-{-}{-}{-}{-}                         {-}{-}{-}{-}{-}}
             {-{-}{-}                           {-}{-}{-}}
              {-                             {-}}
\end{verbatim}

\textbf{Purpose:}

\begin{itemize}
\tightlist
\item
  \textbf{Pre-emphasis}: Boosts high-frequency components at transmitter
\item
  \textbf{De-emphasis}: Attenuates high-frequency components at receiver
\end{itemize}

\textbf{Operation:}

\begin{itemize}
\tightlist
\item
  \textbf{Pre-emphasis}: High-pass RC circuit (R series, C parallel)
\item
  \textbf{De-emphasis}: Low-pass RC circuit (R parallel, C series)
\item
Time constants are identical:

τ = RC = 75μs (standard)

\end{itemize}

\textbf{Benefits:}

\begin{itemize}
\tightlist
\item
  Improves SNR for higher frequencies in FM
\item
  Compensates for higher noise power at high frequencies
\item
  Restores original frequency response at receiver
\end{itemize}

\end{solutionbox}
\begin{mnemonicbox}
``BETH'' - Boost (pre-emphasis), Emphasizes Treble,
Helps SNR

\end{mnemonicbox}
\subsection*{Question 2(c) OR [7
marks]}\label{q2c}

\textbf{Compare AM, FM and PM.}

\begin{solutionbox}

\textbf{Comparison of AM, FM and PM:}

{\def\LTcaptype{none} % do not increment counter
\begin{longtable}[]{@{}
  >{\raggedright\arraybackslash}p{(\linewidth - 6\tabcolsep) * \real{0.3793}}
  >{\raggedright\arraybackslash}p{(\linewidth - 6\tabcolsep) * \real{0.2069}}
  >{\raggedright\arraybackslash}p{(\linewidth - 6\tabcolsep) * \real{0.2069}}
  >{\raggedright\arraybackslash}p{(\linewidth - 6\tabcolsep) * \real{0.2069}}@{}}
\toprule\noalign{}
\begin{minipage}[b]{\linewidth}\raggedright
Parameter
\end{minipage} & \begin{minipage}[b]{\linewidth}\raggedright
AM
\end{minipage} & \begin{minipage}[b]{\linewidth}\raggedright
FM
\end{minipage} & \begin{minipage}[b]{\linewidth}\raggedright
PM
\end{minipage} \\
\midrule\noalign{}
\endhead
\bottomrule\noalign{}
\endlastfoot
\textbf{Definition} & Amplitude varies with message signal & Frequency
varies with message signal & Phase varies with message signal \\
\textbf{Mathematical expression} & \(A_c[1+m\cos(ω_mt)]\cos(ω_ct)\) &
\(A_c\cos[ω_ct+mf\sin(ω_mt)]\) & \(A_c\cos[ω_ct+mp\cos(ω_mt)]\) \\
\textbf{Bandwidth} & 2fm (narrow) & 2(Δf+fm) (wide) & 2(mp+1)fm
(wide) \\
\textbf{Power efficiency} & Low (carrier contains no info) & High
(constant amplitude) & High (constant amplitude) \\
\textbf{Noise immunity} & Poor & Excellent & Excellent \\
\textbf{Circuit complexity} & Simple & Complex & Complex \\
\textbf{Applications} & AM broadcasting, aircraft communication & FM
broadcasting, TV sound, mobile radio & Satellite communication,
telemetry \\
\textbf{Modulation index} & m = Am/Ac (0 to 1) & mf = Δf/fm (no limit) &
mp = Δφ/fm (no limit) \\
\end{longtable}
}

\end{solutionbox}
\begin{mnemonicbox}
``BANCP-MAP'' - Bandwidth, Amplitude, Noise,
Complexity, Power, Modulation, Applications, Parameters

\end{mnemonicbox}
\subsection*{Question 3(a) [3 marks]}\label{q3a}

\textbf{Define any FOUR characteristics of radio receiver.}

\begin{solutionbox}

\textbf{Radio Receiver Characteristics:}

{\def\LTcaptype{none} % do not increment counter
\begin{longtable}[]{@{}
  >{\raggedright\arraybackslash}p{(\linewidth - 2\tabcolsep) * \real{0.5714}}
  >{\raggedright\arraybackslash}p{(\linewidth - 2\tabcolsep) * \real{0.4286}}@{}}
\toprule\noalign{}
\begin{minipage}[b]{\linewidth}\raggedright
Characteristic
\end{minipage} & \begin{minipage}[b]{\linewidth}\raggedright
Definition
\end{minipage} \\
\midrule\noalign{}
\endhead
\bottomrule\noalign{}
\endlastfoot
\textbf{Sensitivity} & Minimum signal strength required for acceptable
output \\
\textbf{Selectivity} & Ability to separate desired signal from adjacent
signals \\
\textbf{Fidelity} & Accuracy in reproducing the original signal without
distortion \\
\textbf{Image rejection} & Ability to reject image frequency
interference \\
\textbf{Signal-to-noise ratio} & Ratio of desired signal to unwanted
noise \\
\textbf{Stability} & Ability to maintain tuned frequency without
drift \\
\end{longtable}
}

\end{solutionbox}
\begin{mnemonicbox}
``SFIS-SS'' - Sensitivity, Fidelity, Image rejection,
Selectivity, SNR, Stability

\end{mnemonicbox}
\subsection*{Question 3(b) [4 marks]}\label{q3b}

\textbf{Draw the block diagram of FM receiver. What is the use of
Limiter in FM receiver.}

\begin{solutionbox}

\textbf{FM Receiver Block Diagram:}

\begin{center}
\textbf{Mermaid Diagram (Code)}
\begin{verbatim}
{Shaded}
{Highlighting}[]
graph LR
    A[Antenna] {-{-}{} B[RF Amplifier]}
    B {-{-}{} C[Mixer]}
    D[Local Oscillator] {-{-}{} C}
    C {-{-}{} E[IF Amplifier]}
    E {-{-}{} F[Limiter]}
    F {-{-}{} G[FM Detector]}
    G {-{-}{} H[Audio Amplifier]}
    H {-{-}{} I[Speaker]}
{Highlighting}
{Shaded}
\end{verbatim}
\end{center}

\textbf{Use of Limiter in FM Receiver:}

\begin{itemize}
\tightlist
\item
  \textbf{Primary function}: Removes amplitude variations/noise
\item
  \textbf{Operation}: Clips the signal to provide constant amplitude
\item
  \textbf{Benefits}:

  \begin{itemize}
  \tightlist
  \item
    Eliminates AM interference
  \item
    Improves SNR
  \item
    Ensures proper FM detection
  \item
    Prevents false frequency demodulation
  \end{itemize}
\item
  \textbf{Location}: Placed between IF amplifier and FM detector
\end{itemize}

\end{solutionbox}
\begin{mnemonicbox}
``CARE'' - Clips Amplitude, Removes noise, Ensures
constant signal

\end{mnemonicbox}
\subsection*{Question 3(c) [7 marks]}\label{q3c}

\textbf{Draw and explain block diagram of super heterodyne receiver.}

\begin{solutionbox}

\textbf{Super Heterodyne Receiver:}

\begin{center}
\textbf{Mermaid Diagram (Code)}
\begin{verbatim}
{Shaded}
{Highlighting}[]
graph LR
    A[Antenna] {-{-}{} B[RF Amplifier]}
    B {-{-}{} C[Mixer]}
    D[Local Oscillator] {-{-}{} C}
    C {-{-}{} E[IF Amplifier]}
    E {-{-}{} F[Detector]}
    F {-{-}{} G[Audio Amplifier]}
    G {-{-}{} H[Speaker]}
    F {-{-}{} I[AGC]}
    I {-{-}{} B}
    I {-{-}{} E}
{Highlighting}
{Shaded}
\end{verbatim}
\end{center}

\textbf{Function of each block:}

\begin{itemize}
\tightlist
\item
  \textbf{Antenna}: Captures RF signals from electromagnetic waves
\item
  \textbf{RF Amplifier}: Amplifies weak signals, provides selectivity
\item
  \textbf{Local Oscillator}: Generates signal to mix with incoming RF
\item
  \textbf{Mixer}: Produces IF by heterodyning RF with local oscillator
\item
  \textbf{IF Amplifier}: Main amplification and selectivity at fixed
  frequency
\item
  \textbf{Detector}: Extracts audio from modulated IF signal
\item
  \textbf{Audio Amplifier}: Amplifies audio signal to drive speaker
\item
  \textbf{AGC (Automatic Gain Control)}: Maintains constant output level
\item
  \textbf{Speaker}: Converts electrical signal to sound
\end{itemize}

\textbf{Super Heterodyne Principle:}

\begin{itemize}
\tightlist
\item
  Converts high-frequency RF to fixed IF for better amplification
\item
  IF = \textbar RF \pm LO\textbar{} (typically 455 kHz for AM, 10.7 MHz
  for FM)
\end{itemize}

\end{solutionbox}
\begin{mnemonicbox}
``ARLMIDAS'' - Antenna Receives, Local Mixes, IF
Delivers, Audio Sounds

\end{mnemonicbox}
\subsection*{Question 3(a) OR [3
marks]}\label{q3a}

\textbf{Draw and explain block diagram for envelope detector.}

\begin{solutionbox}

\textbf{Envelope Detector:}

\textbf{Circuit Diagram:}

\begin{verbatim}
          D
    +{-{-}{-}|{-}{-}{-}+{-}{-}{-}+}
    |         |   |
AM  |         |   |     Audio
Input o       C   R    Output
    |         |   |      o
    |         |   |      |
    +{-{-}{-}{-}{-}{-}{-}{-}{-}+{-}{-}{-}+{-}{-}{-}{-}{-}{-}+}
              |
             {-{-}{-}}
              {-}
\end{verbatim}

\textbf{Component Functions:}

\begin{itemize}
\tightlist
\item
  \textbf{Diode (D)}: Rectifies AM signal (allows only positive
  half-cycles)
\item
  \textbf{Capacitor (C)}: Charges to peak of input, filters carrier
  frequency
\item
  \textbf{Resistor (R)}: Discharges capacitor, follows modulating signal
  envelope
\end{itemize}

\textbf{Operation:}

\begin{enumerate}
\tightlist
\item
  Diode conducts during positive half-cycles
\item
  Capacitor charges to peak voltage
\item
  During negative half-cycles, diode blocks
\item
  Capacitor discharges through resistor
\item
  RC time constant follows envelope variations
\end{enumerate}

\textbf{RC Selection Criteria}: \(\frac{1}{f_c} << RC << \frac{1}{f_m}\)

\end{solutionbox}
\begin{mnemonicbox}
``DRIVER'' - Diode Rectifies, RC Values Extract
Envelope, Restores audio

\end{mnemonicbox}
\subsection*{Question 3(b) OR [4
marks]}\label{q3b}

\textbf{What is IF? Explain its importance in brief.}

\begin{solutionbox}

\textbf{Intermediate Frequency (IF):}

\textbf{Definition:} IF is a fixed frequency to which incoming RF
signals are converted in superheterodyne receivers.

\textbf{Importance of IF:}

{\def\LTcaptype{none} % do not increment counter
\begin{longtable}[]{@{}
  >{\raggedright\arraybackslash}p{(\linewidth - 2\tabcolsep) * \real{0.4000}}
  >{\raggedright\arraybackslash}p{(\linewidth - 2\tabcolsep) * \real{0.6000}}@{}}
\toprule\noalign{}
\begin{minipage}[b]{\linewidth}\raggedright
Aspect
\end{minipage} & \begin{minipage}[b]{\linewidth}\raggedright
Importance
\end{minipage} \\
\midrule\noalign{}
\endhead
\bottomrule\noalign{}
\endlastfoot
\textbf{Fixed Frequency} & Allows optimized amplification at one
frequency \\
\textbf{Improved Selectivity} & Fixed-tuned filters provide better
adjacent channel rejection \\
\textbf{Stable Gain} & Consistent amplification across entire tuning
range \\
\textbf{Image Rejection} & Helps reject image frequency interference \\
\textbf{Simplified Tuning} & Only local oscillator needs to be tuned for
different stations \\
\textbf{Better AGC} & More effective gain control at fixed frequency \\
\end{longtable}
}

\textbf{Typical IF Values:}

\begin{itemize}
\tightlist
\item
  AM receivers: 455 kHz
\item
  FM receivers: 10.7 MHz
\item
  Television: 45 MHz
\end{itemize}

\end{solutionbox}
\begin{mnemonicbox}
``FIGS-ST'' - Fixed frequency, Improved selectivity,
Gain stability, Simplified tuning

\end{mnemonicbox}
\subsection*{Question 3(c) OR [7
marks]}\label{q3c}

\textbf{Explain phase discriminator circuit for FM detection.}

\begin{solutionbox}

\textbf{Phase Discriminator for FM Detection:}

\textbf{Circuit Diagram:}

\begin{verbatim}
                 +{-{-}{-}{-}+}
                 |    |
     +{-{-}{-}{-}{-}{-}{-}{-}{-}{-}{-}|  T1|{-}{-}{-}{-}{-}{-}{-}+}
     |           |    |       |
     |           +{-{-}{-}{-}+       |}
     |                        |
     |                        |
     |                        |
     |   D1                   |   D2
FM   o{-{-}|{-}{-}{-}+            +{-}{-}{-}|{-}{-}+}
Input|        |           |       |
     |        |  +{-{-}{-}{-}+   |       |}
     |        +{-{-}|    |{-}{-}{-}+       |}
     |           |  T2|           |
     +{-{-}{-}{-}{-}{-}{-}{-}{-}{-}{-}| CT |{-}{-}{-}{-}{-}{-}{-}{-}{-}{-}{-}|}
                 |    |           |
                 +{-{-}{-}{-}+           |}
                    |             |
                    |     C1      |     C2
                    o{-{-}{-}{-}||{-}{-}{-}{-}{-}{-}{-}o{-}{-}{-}{-}{-}||{-}{-}{-}{-}o Audio }
                    |             |          Output
                   {-{-}{-}           {-}{-}{-}}
                    {-             {-}}
\end{verbatim}

\textbf{Operation:}

\begin{enumerate}
\tightlist
\item
  \textbf{Center-tapped transformer (T2)} creates 180^\circ phase difference
\item
  \textbf{Primary transformer (T1)} sets reference phase
\item
  \textbf{Diode D1 and D2} form phase comparators
\item
  \textbf{When carrier at center frequency:}

  \begin{itemize}
  \tightlist
  \item
    Equal currents through both diodes
  \item
    Equal voltages across C1 and C2
  \item
    Net output is zero
  \end{itemize}
\item
  \textbf{When frequency deviates:}

  \begin{itemize}
  \tightlist
  \item
    Phase changes
  \item
    Unequal diode currents
  \item
    Output voltage proportional to frequency deviation
  \end{itemize}
\end{enumerate}

\textbf{Advantages:}

\begin{itemize}
\tightlist
\item
  Good linearity
\item
  Reduced distortion
\item
  Better noise performance than slope detector
\end{itemize}

\end{solutionbox}
\begin{mnemonicbox}
``PERFECT'' - Phase Ensures Rectification For
Extracting Carrier Transitions

\end{mnemonicbox}
\subsection*{Question 4(a) [3 marks]}\label{q4a}

\textbf{Explain quantization process and its necessity.}

\begin{solutionbox}

\textbf{Quantization Process:}

\textbf{Definition:} Quantization is the process of mapping continuous
analog values to discrete digital levels.

\textbf{Process:}

\begin{enumerate}
\tightlist
\item
  Sampling converts continuous-time signal to discrete-time
\item
  Range of amplitudes divided into finite number of levels
\item
  Each sample assigned to nearest quantization level
\item
  Difference between original and quantized value is quantization error
\end{enumerate}

\textbf{Necessity of Quantization:}

{\def\LTcaptype{none} % do not increment counter
\begin{longtable}[]{@{}ll@{}}
\toprule\noalign{}
Necessity & Explanation \\
\midrule\noalign{}
\endhead
\bottomrule\noalign{}
\endlastfoot
\textbf{Digital Processing} & Enables digital storage and
manipulation \\
\textbf{Error Control} & Allows error detection and correction \\
\textbf{Noise Immunity} & Digital signals more resistant to noise \\
\textbf{Storage Efficiency} & More efficient than storing analog
values \\
\textbf{Transmission} & Digital signals can be regenerated without
error \\
\end{longtable}
}

\end{solutionbox}
\begin{mnemonicbox}
``DENSE'' - Digital conversion, Error control, Noise
immunity, Storage, Efficient transmission

\end{mnemonicbox}
\subsection*{Question 4(b) [4 marks]}\label{q4b}

\textbf{Give difference between DM and ADM.}

\begin{solutionbox}

\textbf{Difference between DM and ADM:}

{\def\LTcaptype{none} % do not increment counter
\begin{longtable}[]{@{}
  >{\raggedright\arraybackslash}p{(\linewidth - 4\tabcolsep) * \real{0.1719}}
  >{\raggedright\arraybackslash}p{(\linewidth - 4\tabcolsep) * \real{0.3281}}
  >{\raggedright\arraybackslash}p{(\linewidth - 4\tabcolsep) * \real{0.5000}}@{}}
\toprule\noalign{}
\begin{minipage}[b]{\linewidth}\raggedright
Parameter
\end{minipage} & \begin{minipage}[b]{\linewidth}\raggedright
Delta Modulation (DM)
\end{minipage} & \begin{minipage}[b]{\linewidth}\raggedright
Adaptive Delta Modulation (ADM)
\end{minipage} \\
\midrule\noalign{}
\endhead
\bottomrule\noalign{}
\endlastfoot
\textbf{Step Size} & Fixed & Variable (adapts to signal) \\
\textbf{Slope Overload} & Common at steep signals & Reduced with
adaptive step \\
\textbf{Granular Noise} & High for small signals & Reduced with smaller
steps \\
\textbf{Signal Tracking} & Slow for rapidly changing signals & Better
tracking of signal variations \\
\textbf{Complexity} & Simple & Moderate \\
\textbf{Bit Rate} & Higher for good quality & Lower for same quality \\
\textbf{Error Performance} & More sensitive & More robust \\
\end{longtable}
}

\textbf{Diagram:}

\begin{verbatim}
DM:                              ADM:

   \^{                                \^{}}
   |                                |
   |    Original                    |    Original
   |      /{                        |      /}
   |     /  {                       |     /  }
   |    /    {                      |    /    }
   |   /      {                     |   /      }
   |  /  \_\_\_\_  {                    |  /        }
   | /\_\_|    |\_\_{\_\_\_                | /\_         \_\_\_}
   |/  |    |  |   {                |/         /    }
{-{-}{-}+{-}{-}{-}+{-}{-}{-}{-}+{-}{-}+{-}{-}{-}{-}+{-}{-}           {-}+{-}{-}{-}{-}+{-}{-}{-}{-}{-}+{-}{-}{-}{-}{-}{-}+{-}{-}}
   |   |    |  |    |               |     |     |      |
   
Slope Overload              Better Signal Tracking
\end{verbatim}

\end{solutionbox}
\begin{mnemonicbox}
``SAVAGES'' - Step size, Adaptable, Variable
tracking, Avoids overload, Granular noise reduction, Error performance,
Signal fidelity

\end{mnemonicbox}
\subsection*{Question 4(c) [7 marks]}\label{q4c}

\textbf{Draw \& explain block diagram of PCM system.}

\begin{solutionbox}

\textbf{PCM System Block Diagram:}

\begin{center}
\textbf{Mermaid Diagram (Code)}
\begin{verbatim}
{Shaded}
{Highlighting}[]
graph LR
    subgraph "PCM Transmitter"
    A[Input Signal] {-{-}{} B[Anti{-}aliasing Filter]}
    B {-{-}{} C[Sample \& Hold]}
    C {-{-}{} D[Quantizer]}
    D {-{-}{} E[Encoder]}
    E {-{-}{} F[Parallel to Serial]}
    end

    F {-{-}{} G[Transmission Channel]}
    
    subgraph "PCM Receiver"
    G {-{-}{} H[Serial to Parallel]}
    H {-{-}{} I[Decoder]}
    I {-{-}{} J[Reconstruction Filter]}
    J {-{-}{} K[Output Signal]}
    end
{Highlighting}
{Shaded}
\end{verbatim}
\end{center}

\textbf{PCM Transmitter:}

\begin{itemize}
\tightlist
\item
  \textbf{Anti-aliasing Filter}: Limits input signal bandwidth to
  satisfy Nyquist criterion
\item
  \textbf{Sample \& Hold}: Converts continuous signal to discrete-time
  samples
\item
  \textbf{Quantizer}: Approximates sample amplitudes to nearest discrete
  levels
\item
  \textbf{Encoder}: Converts quantized levels to binary code
\item
  \textbf{Parallel-to-Serial}: Converts parallel bits to serial for
  transmission
\end{itemize}

\textbf{PCM Receiver:}

\begin{itemize}
\tightlist
\item
  \textbf{Serial-to-Parallel}: Converts serial data back to parallel
  form
\item
  \textbf{Decoder}: Converts binary code back to amplitude levels
\item
  \textbf{Reconstruction Filter}: Smooths stepped output to recover
  analog signal
\end{itemize}

\textbf{PCM Parameters:}

\begin{itemize}
\tightlist
\item
  \textbf{Sampling rate}: fs \textgreater{} 2fm (Nyquist rate)
\item
  \textbf{Quantization levels}: L = 2\^{}n (n = number of bits)
\item
  \textbf{Resolution}: Smallest distinguishable change = Vmax/L
\item
  \textbf{Bit rate}: R = n \times fs bits/second
\end{itemize}

\end{solutionbox}
\begin{mnemonicbox}
``SAFE-PETS'' - Sample, Amplify, Filter, Encode,
Pulse train, Extract, Transform, Smooth

\end{mnemonicbox}
\subsection*{Question 4(a) OR [3
marks]}\label{q4a}

\textbf{Define quantization. Explain non uniform quantization in brief.}

\begin{solutionbox}

\textbf{Quantization Definition:} Quantization is the process of
converting continuous amplitude values to a finite set of discrete
levels in analog-to-digital conversion.

\textbf{Non-uniform Quantization:}

\textbf{Diagram:}

\begin{verbatim}
     \^{}
     |                     {-{-}{-}{-}}
     |                 {-{-}{-}{-}}
     |             {-{-}{-}{-}}
Lvls |         {-{-}{-}{-}}
     |     {-{-}{-}{-}}
     |  {-{-}{-}}
     | {-}
     +{-{-}{-}{-}{-}{-}{-}{-}{-}{-}{-}{-}{-}{-}{-}{-}{-}{-}{-}{-}{-}{-}{-}{-}{-}{-}{-}}
                Input Signal
\end{verbatim}

\textbf{Characteristics:}

\begin{itemize}
\tightlist
\item
  Unequal step sizes throughout the amplitude range
\item
  Smaller steps for low amplitudes, larger for high amplitudes
\item
  Better matches human perception (logarithmic response)
\item
  Improves SNR for small signals without increasing bit rate
\end{itemize}

\textbf{Implementation Methods:}

\begin{itemize}
\tightlist
\item
  \textbf{Companding}: Compressing at transmitter, expanding at receiver
\item
  \textbf{Logarithmic coding}: μ-law (North America) and A-law (Europe)
\item
  \textbf{Adaptive quantization}: Adjusts levels based on signal
  statistics
\end{itemize}

\end{solutionbox}
\begin{mnemonicbox}
``CLASP'' - Compressed Levels, Adaptive Steps, Small
steps for small signals, Perceptual matching

\end{mnemonicbox}
\subsection*{Question 4(b) OR [4
marks]}\label{q4b}

\textbf{Explain Adaptive delta modulation with its application.}

\begin{solutionbox}

\textbf{Adaptive Delta Modulation (ADM):}

\textbf{Diagram:}

\begin{center}
\textbf{Mermaid Diagram (Code)}
\begin{verbatim}
{Shaded}
{Highlighting}[]
graph LR
    A[Input Signal] {-{-}{} B[Comparator]}
    B {-{-}{} C[1{-}bit Quantizer]}
    C {-{-}{} D[Transmission Channel]}
    D {-{-}{} E[Step Size Control]}
    E {-{-}{} F[Integrator]}
    F {-{-}Feedback{-}{-}{} B}
    F {-{-}{} G[Output Signal]}
    C {-{-}Controls{-}{-}{} E}
{Highlighting}
{Shaded}
\end{verbatim}
\end{center}

\textbf{Operation:}

\begin{itemize}
\tightlist
\item
  Adapts step size based on input signal slope
\item
  Increases step size for rapid changes (prevents slope overload)
\item
  Decreases step size for slow changes (reduces granular noise)
\item
  Uses previous bits pattern to determine slope changes
\end{itemize}

\textbf{Advantages:}

\begin{itemize}
\tightlist
\item
  Better signal tracking than DM
\item
  Lower bit rate for same quality
\item
  Reduced slope overload and granular noise
\item
  Wider dynamic range
\end{itemize}

\textbf{Applications:}

\begin{itemize}
\tightlist
\item
  Speech and audio compression
\item
  Voice-grade communication channels
\item
  Digital telephony systems
\item
  Video signal encoding
\item
  Telemetry systems
\end{itemize}

\end{solutionbox}
\begin{mnemonicbox}
``ADAPT'' - Automatically Decides Appropriate Pulse
Transitions

\end{mnemonicbox}
\subsection*{Question 4(c) OR [7
marks]}\label{q4c}

\textbf{What is sampling? Explain types of sampling in brief.}

\begin{solutionbox}

\textbf{Sampling Definition:} Sampling is the process of converting a
continuous-time signal to a discrete-time signal by taking measurements
at regular intervals.

\textbf{Types of Sampling:}

{\def\LTcaptype{none} % do not increment counter
\begin{longtable}[]{@{}
  >{\raggedright\arraybackslash}p{(\linewidth - 4\tabcolsep) * \real{0.2143}}
  >{\raggedright\arraybackslash}p{(\linewidth - 4\tabcolsep) * \real{0.4643}}
  >{\raggedright\arraybackslash}p{(\linewidth - 4\tabcolsep) * \real{0.3214}}@{}}
\toprule\noalign{}
\begin{minipage}[b]{\linewidth}\raggedright
Type
\end{minipage} & \begin{minipage}[b]{\linewidth}\raggedright
Description
\end{minipage} & \begin{minipage}[b]{\linewidth}\raggedright
Diagram
\end{minipage} \\
\midrule\noalign{}
\endhead
\bottomrule\noalign{}
\endlastfoot
\textbf{Ideal Sampling} & Instantaneous samples of infinitesimal
duration & Impulses at sampling instants \\
\textbf{Natural Sampling} & Samples have finite width, amplitude follows
input & Original signal visible during sampling duration \\
\textbf{Flat-top Sampling} & Samples have constant amplitude during
sampling interval & Step-like appearance, used in sample-and-hold \\
\end{longtable}
}

\textbf{Diagrams:}

\begin{verbatim}
Ideal Sampling:           Natural Sampling:         Flat{-top Sampling:}

   \^{                         \^{}                        \^{}}
   |                         |                        |
   | |   |   |   |           |   \_   \_   \_            |  \_\_\_   \_\_\_   \_\_\_ 
   | |   |   |   |           |  / { /  /            | |   | |   | |   |}
   | |   |   |   |           | /   |   |   {          | |   | |   | |   |}
{-{-}{-}+{-}{-}{-}+{-}{-}{-}+{-}{-}{-}+{-}{-}{-}{-}     {-}{-}+{-}{-}{-}{-}{-}+{-}{-}{-}+{-}{-}{-}{-}{-}{-}{-}     {-}{-}+{-}{-}{-}{-}{-}+{-}{-}{-}{-}{-}+{-}{-}{-}{-}{-}}
   |                         |                         |
\end{verbatim}

\textbf{Sampling Parameters:}

\begin{itemize}
\tightlist
\item
  \textbf{Sampling period (Ts)}: Time between consecutive samples
\item
  \textbf{Sampling frequency (fs)}: Number of samples per second (fs =
  1/Ts)
\item
  \textbf{Nyquist rate}: Minimum sampling rate (fs \textgreater{} 2fm)
  to avoid aliasing
\end{itemize}

\end{solutionbox}
\begin{mnemonicbox}
``INFS'' - Ideal (impulses), Natural (follows
signal), Flat-top (constant), Sufficient rate

\end{mnemonicbox}
\subsection*{Question 5(a) [3 marks]}\label{q5a}

\textbf{Define bit rate and baud rate.}

\begin{solutionbox}

\textbf{Bit Rate and Baud Rate:}

{\def\LTcaptype{none} % do not increment counter
\begin{longtable}[]{@{}
  >{\raggedright\arraybackslash}p{(\linewidth - 6\tabcolsep) * \real{0.2895}}
  >{\raggedright\arraybackslash}p{(\linewidth - 6\tabcolsep) * \real{0.3158}}
  >{\raggedright\arraybackslash}p{(\linewidth - 6\tabcolsep) * \real{0.2368}}
  >{\raggedright\arraybackslash}p{(\linewidth - 6\tabcolsep) * \real{0.1579}}@{}}
\toprule\noalign{}
\begin{minipage}[b]{\linewidth}\raggedright
Parameter
\end{minipage} & \begin{minipage}[b]{\linewidth}\raggedright
Definition
\end{minipage} & \begin{minipage}[b]{\linewidth}\raggedright
Formula
\end{minipage} & \begin{minipage}[b]{\linewidth}\raggedright
Unit
\end{minipage} \\
\midrule\noalign{}
\endhead
\bottomrule\noalign{}
\endlastfoot
\textbf{Bit Rate} & Number of binary digits (bits) transmitted per
second & R = fs \times n & bits per second (bps) \\
\textbf{Baud Rate} & Number of signal elements or symbols transmitted
per second & B = fs & symbols per second (baud) \\
\end{longtable}
}

\textbf{Relationship:}

\begin{itemize}
\tightlist
\item
  For binary signaling: Bit Rate = Baud Rate
\item
  For M-ary signaling: Bit Rate = Baud Rate \times log_{2}M

  \begin{itemize}
  \tightlist
  \item
    Where M = number of different signal elements
  \end{itemize}
\end{itemize}

\textbf{Example:}

\begin{itemize}
\tightlist
\item
  4-QAM (M=4): Each symbol carries log_{2}4 = 2 bits
\item
  If Baud Rate = 1000 symbols/s, then Bit Rate = 2000 bits/s
\end{itemize}

\end{solutionbox}
\begin{mnemonicbox}
``BBSM'' - Bits per second, Baud for Symbols,
Modulation determines relationship

\end{mnemonicbox}
\subsection*{Question 5(b) [4 marks]}\label{q5b}

\textbf{Explain working of DPCM.}

\begin{solutionbox}

\textbf{Differential Pulse Code Modulation (DPCM):}

\textbf{Block Diagram:}

\begin{center}
\textbf{Mermaid Diagram (Code)}
\begin{verbatim}
{Shaded}
{Highlighting}[]
graph LR
    subgraph "Transmitter"
    direction LR
    A[Input] {-{-}{} B[Difference]}
    B {-{-}{} C[Quantizer]}
    C {-{-}{} D[Encoder]}
    D {-{-}{} E[Output]}
    F[Predictor] {-{-}{} B}
    C {-{-}{} F}
    end

    subgraph "Receiver"
    direction LR    
    G[Input] {-{-}{} H[Decoder]}
    H {-{-}{} I[Output]}
    H {-{-}{} J[Predictor]}
    J {-{-}{} I}
    end
{Highlighting}
{Shaded}
\end{verbatim}
\end{center}

\textbf{Working Principle:}

\begin{itemize}
\tightlist
\item
  Encodes difference between current sample and predicted sample
\item
  Prediction based on previous samples (correlation)
\item
  Smaller dynamic range of differences allows fewer bits per sample
\end{itemize}

\textbf{Advantages:}

\begin{itemize}
\tightlist
\item
  Higher compression ratio than PCM
\item
  Reduced bit rate for same quality
\item
  Exploits signal correlation
\item
  Improved SNR performance
\end{itemize}

\end{solutionbox}
\begin{mnemonicbox}
``DEEP'' - Difference Encoded, Efficient Prediction,
Exploits correlation, Preserves quality

\end{mnemonicbox}
\subsection*{Question 5(c) [7 marks]}\label{q5c}

\textbf{The binary data 1011001 is to be transmitted using following
line coding techniques: (i) Unipolar RZ and NRZ (ii) Polar RZ and NRZ
(iii) AMI (iv) Manchester. Draw all the waveforms.}

\begin{solutionbox}

\textbf{Line Coding Waveforms for 1011001:}

\begin{verbatim}
Data:      1    0    1    1    0    0    1
           |    |    |    |    |    |    |
           v    v    v    v    v    v    v

Unipolar   
NRZ:     \_\_\_\_|‾‾‾‾|\_\_\_\_|‾‾‾‾|‾‾‾‾|\_\_\_\_|\_\_\_\_|‾‾‾‾|\_\_\_\_
           
Unipolar  
RZ:      \_\_\_\_|‾|\_\_|\_\_\_\_|‾|\_\_|‾|\_\_|\_\_\_\_|\_\_\_\_|‾|\_\_|\_\_\_\_
           
Polar     
NRZ:     \_\_\_\_|‾‾‾‾|\_\_\_\_|‾‾‾‾|‾‾‾‾|\_\_\_\_|\_\_\_\_|‾‾‾‾|\_\_\_\_
          ‾‾‾‾     ‾‾‾‾     ‾‾‾‾     ‾‾‾‾     ‾‾‾‾
Polar
RZ:      \_\_\_\_|‾|\_\_|\_\_\_\_|‾|\_\_|‾|\_\_|\_\_\_\_|\_\_\_\_|‾|\_\_|\_\_\_\_
          ‾‾‾‾|\_|  |\_|‾‾|\_|‾‾|\_|  |\_|  |\_|‾‾|\_|  
AMI:     \_\_\_\_\_|‾|\_\_|\_\_\_\_|\_\_\_|‾|\_\_|\_\_\_\_|\_\_\_\_|‾|\_\_|\_\_\_\_
          ‾‾‾‾|\_|  |\_|  |\_|  |\_|  |\_|  |\_|  |\_|
              
Manchester:
          \_\_\_\_|‾|\_\_|\_|‾‾|\_\_\_\_|‾|\_\_|‾|\_\_|\_\_\_\_|\_\_\_\_|‾|\_\_|\_\_\_\_
          ‾‾‾‾|\_|  |\_|  |\_|‾‾|\_|  |\_|  |\_|‾‾|\_|‾‾|\_|
\end{verbatim}

\textbf{Description of Line Coding Techniques:}

{\def\LTcaptype{none} % do not increment counter
\begin{longtable}[]{@{}
  >{\raggedright\arraybackslash}p{(\linewidth - 6\tabcolsep) * \real{0.2391}}
  >{\raggedright\arraybackslash}p{(\linewidth - 6\tabcolsep) * \real{0.1957}}
  >{\raggedright\arraybackslash}p{(\linewidth - 6\tabcolsep) * \real{0.1957}}
  >{\raggedright\arraybackslash}p{(\linewidth - 6\tabcolsep) * \real{0.3696}}@{}}
\toprule\noalign{}
\begin{minipage}[b]{\linewidth}\raggedright
Technique
\end{minipage} & \begin{minipage}[b]{\linewidth}\raggedright
Logic 1
\end{minipage} & \begin{minipage}[b]{\linewidth}\raggedright
Logic 0
\end{minipage} & \begin{minipage}[b]{\linewidth}\raggedright
Characteristics
\end{minipage} \\
\midrule\noalign{}
\endhead
\bottomrule\noalign{}
\endlastfoot
\textbf{Unipolar NRZ} & High level & Zero level & No return to zero
between bits \\
\textbf{Unipolar RZ} & Pulse for half bit & Zero level & Returns to zero
for half bit \\
\textbf{Polar NRZ} & Positive & Negative & No return to zero between
bits \\
\textbf{Polar RZ} & Positive pulse & Negative pulse & Returns to zero
for half bit \\
\textbf{AMI} & Alternating +/- & Zero level & Alternates polarity for
consecutive 1s \\
\textbf{Manchester} & High\rightarrowLow & Low\rightarrowHigh & Transition in middle of
bit \\
\end{longtable}
}

\end{solutionbox}
\begin{mnemonicbox}
``UPAM'' - Unipolar, Polar, AMI, Manchester encoding
options

\end{mnemonicbox}
\subsection*{Question 5(a) OR [3
marks]}\label{q5a}

\textbf{Compare RZ and NRZ coding with example.}

\begin{solutionbox}

\textbf{Comparison of RZ and NRZ Coding:}

{\def\LTcaptype{none} % do not increment counter
\begin{longtable}[]{@{}
  >{\raggedright\arraybackslash}p{(\linewidth - 4\tabcolsep) * \real{0.1897}}
  >{\raggedright\arraybackslash}p{(\linewidth - 4\tabcolsep) * \real{0.3621}}
  >{\raggedright\arraybackslash}p{(\linewidth - 4\tabcolsep) * \real{0.4483}}@{}}
\toprule\noalign{}
\begin{minipage}[b]{\linewidth}\raggedright
Parameter
\end{minipage} & \begin{minipage}[b]{\linewidth}\raggedright
Return-to-Zero (RZ)
\end{minipage} & \begin{minipage}[b]{\linewidth}\raggedright
Non-Return-to-Zero (NRZ)
\end{minipage} \\
\midrule\noalign{}
\endhead
\bottomrule\noalign{}
\endlastfoot
\textbf{Signal levels} & Returns to zero in each bit & Maintains level
for full bit period \\
\textbf{Bandwidth} & Higher (\approx 2\times NRZ) & Lower \\
\textbf{Self-clocking} & Better (transitions in every bit) & Poorer (may
have long runs without transitions) \\
\textbf{Power requirement} & Higher & Lower \\
\textbf{Bit synchronization} & Easier & More difficult \\
\textbf{Implementation} & More complex & Simpler \\
\textbf{DC component} & Less & More \\
\end{longtable}
}

\textbf{Example for binary data 101:}

\begin{verbatim}
Data:        1     0     1
             |     |     |
             v     v     v

NRZ:      \_\_\_|‾‾‾‾‾|\_\_\_\_\_|‾‾‾‾‾|\_\_\_\_
                   
RZ:       \_\_\_|‾|\_\_\_|\_\_\_\_\_|‾|\_\_\_|\_\_\_\_
\end{verbatim}

\end{solutionbox}
\begin{mnemonicbox}
``BPSIDC'' - Bandwidth, Power, Synchronization,
Implementation, DC component

\end{mnemonicbox}
\subsection*{Question 5(b) OR [4
marks]}\label{q5b}

\textbf{Explain delta modulation in brief.}

\begin{solutionbox}

\textbf{Delta Modulation (DM):}

\textbf{Block Diagram:}

\begin{center}
\textbf{Mermaid Diagram (Code)}
\begin{verbatim}
{Shaded}
{Highlighting}[]
graph LR
    A[Input Signal] {-{-}{} B[(Comparator)]}
    B {-{-}{} C[1{-}bit Quantizer]}
    C {-{-}{} D[Transmission]}
    C {-{-}{} E[Integrator]}
    E {-{-}Feedback{-}{-}{} B}
    D {-{-}{} F[Integrator]}
    F {-{-}{} G[Output Signal]}
{Highlighting}
{Shaded}
\end{verbatim}
\end{center}

\textbf{Working Principle:}

\begin{itemize}
\tightlist
\item
  Encodes only the difference between samples using 1 bit
\item
  Comparator checks if input is higher/lower than predicted value
\item
  Integrator accumulates the bits to approximate original signal
\item
  Output is series of 1s and 0s representing up/down steps
\end{itemize}

\textbf{Limitations:}

\begin{itemize}
\tightlist
\item
  \textbf{Slope Overload}: Cannot track rapidly changing signals
\item
  \textbf{Granular Noise}: Small variations around steady signal
\end{itemize}

\textbf{Advantages:}

\begin{itemize}
\tightlist
\item
  Simplest form of differential encoding
\item
  Low bit rate (1 bit per sample)
\item
  Simple implementation
\item
  Hardware efficiency
\end{itemize}

\end{solutionbox}
\begin{mnemonicbox}
``SIDE'' - Single-bit, Integrates Differences,
Encodes changes

\end{mnemonicbox}
\subsection*{Question 5(c) OR [7
marks]}\label{q5c}

\textbf{Explain PCM-TDM system.}

\begin{solutionbox}

\textbf{PCM-TDM System:}

\textbf{Block Diagram:}

\begin{center}
\textbf{Mermaid Diagram (Code)}
\begin{verbatim}
{Shaded}
{Highlighting}[]
graph TD
    subgraph "Transmitter"
    A1[Channel 1] {-{-}{} B1[LPF]}
    A2[Channel 2] {-{-}{} B2[LPF]}
    A3[Channel 3] {-{-}{} B3[LPF]}
    A4[Channel n] {-{-}{} B4[LPF]}
    B1 {-{-}{} C[Multiplexer]}
    B2 {-{-}{} C}
    B3 {-{-}{} C}
    B4 {-{-}{} C}
    C {-{-}{} D[Sample \& Hold]}
    D {-{-}{} E[Quantizer]}
    E {-{-}{} F[Encoder]}
    F {-{-}{} G[Frame Generator]}
    G {-{-}{} H[Line Coder]}
    H {-{-}{} I[Transmission Medium]}
    end

    subgraph "Receiver"
    I {-{-}{} J[Line Decoder]}
    J {-{-}{} K[Frame Sync]}
    K {-{-}{} L[Decoder]}
    L {-{-}{} M[Demultiplexer]}
    M {-{-}{} N1[LPF]}
    M {-{-}{} N2[LPF]}
    M {-{-}{} N3[LPF]}
    M {-{-}{} N4[LPF]}
    N1 {-{-}{} O1[Channel 1]}
    N2 {-{-}{} O2[Channel 2]}
    N3 {-{-}{} O3[Channel 3]}
    N4 {-{-}{} O4[Channel n]}
    end
{Highlighting}
{Shaded}
\end{verbatim}
\end{center}

\textbf{PCM-TDM Operation:}

{\def\LTcaptype{none} % do not increment counter
\begin{longtable}[]{@{}ll@{}}
\toprule\noalign{}
Stage & Process \\
\midrule\noalign{}
\endhead
\bottomrule\noalign{}
\endlastfoot
\textbf{Filtering} & Band-limits each channel to prevent aliasing \\
\textbf{Multiplexing} & Samples each channel sequentially \\
\textbf{Conversion} & Quantizes samples and converts to binary code \\
\textbf{Framing} & Adds sync bits and channel identification \\
\textbf{Transmission} & Sends frame over communication medium \\
\textbf{Demultiplexing} & Separates channels from received frame \\
\textbf{Reconstruction} & Converts digital samples back to analog
signals \\
\end{longtable}
}

\textbf{System Parameters:}

\begin{itemize}
\tightlist
\item
  \textbf{Channel Capacity}: N channels
\item
  \textbf{Sampling Rate}: fs per channel
\item
  \textbf{Quantization}: n bits per sample
\item
  \textbf{Frame Structure}: 1 sample from each channel + sync
\item
  \textbf{Total Bit Rate}: N \times n \times fs + overhead
\end{itemize}

\end{solutionbox}
\begin{mnemonicbox}
``MOST-FDR'' - Multiplex, Quantize, Sample, Transmit,
Frame, Demultiplex, Reconstruct

\end{mnemonicbox}

\end{document}
