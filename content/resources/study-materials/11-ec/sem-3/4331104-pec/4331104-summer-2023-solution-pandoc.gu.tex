\documentclass[10pt,a4paper]{article}

% content/resources/templates/preamble.tex
\usepackage[margin=0.6in]{geometry}
\author{Milav Dabgar}
\usepackage{amsmath,amssymb,amsthm}
\usepackage{booktabs}
\usepackage{multirow}
\usepackage{xcolor}
\usepackage{tcolorbox}
\tcbuselibrary{breakable,skins}
\usepackage[colorlinks=true,linkcolor=blue]{hyperref}
\usepackage{titlesec}
\usepackage{enumitem}
\usepackage{tikz}
\usepackage{pgfplots}
\usepackage{circuitikz}
\usepackage[version=4]{mhchem}
\usepackage{longtable}
\usepackage{array}
\usepackage{float}
\usepackage{caption}
\usepackage{listings}

\lstset{
  basicstyle=\small\ttfamily,
  breaklines=true,
  breakatwhitespace=false,
  postbreak=\mbox{\textcolor{red}{$\hookrightarrow$}\space},
  float=false,
  numbers=left,
  numberstyle=\tiny\color{gray},
  numbersep=10pt,
  xleftmargin=2em,
  keywordstyle=\color{blue},
  commentstyle=\color{green!60!black},
  stringstyle=\color{purple},
  backgroundcolor=\color{gray!5},
  showstringspaces=false,
  tabsize=2,
  captionpos=b,
  keepspaces=true,
  columns=flexible
}

\pgfplotsset{compat=1.18}
\usetikzlibrary{shapes,arrows,positioning,calc,patterns,decorations.pathmorphing,decorations.markings,arrows.meta}

% Color scheme
\definecolor{headcolor}{RGB}{0,102,204}
\definecolor{keycolor}{RGB}{220,20,60}
\definecolor{solutioncolor}{RGB}{34,139,34}
\definecolor{mnemoniccolor}{RGB}{148,0,211}
\definecolor{codecolor}{RGB}{0,0,100}

% Spacing
\setlength{\parskip}{3pt}
\setlist[itemize]{nosep}
\setlist[enumerate]{nosep}

% Title formatting
\titleformat{\section}{\Large\bfseries\color{headcolor}}{\thesection}{1em}{}
\titleformat{\subsection}{\large\bfseries\color{headcolor}}{\thesubsection}{1em}{}

% Pandoc tightlist compatibility
\providecommand{\tightlist}{%
  \setlength{\itemsep}{0pt}\setlength{\parskip}{0pt}}

% Pandoc longtable compatibility
\newcounter{none}
\def\thenone{}


% content/resources/templates/gujarati-boxes.tex
\usepackage{fontspec}
\usepackage{polyglossia}

% Set Gujarati as main language (document is primarily in Gujarati)
% Note: gloss-gujarati.ldf doesn't exist in polyglossia, but it will use hyphenation patterns
\setdefaultlanguage{gujarati}
\setotherlanguage{english}

% Configure Gujarati font properly
% Use Language=Default to prevent polyglossia from trying to add language-specific features
% that don't exist for Gujarati, which causes "empty feature" warnings
\newfontfamily\gujaratifont[Script=Gujarati,AutoFakeBold=2.5,AutoFakeSlant=0.3]{Noto Sans Gujarati}
\setmainfont[Script=Gujarati,AutoFakeBold=2.5,AutoFakeSlant=0.3]{Noto Sans Gujarati}
% Use Noto Sans Gujarati for monospace to support Gujarati in text
\setmonofont[Scale=0.9]{Noto Sans Gujarati}

% Configure English to use the same font
\newfontfamily\englishfont[Script=Gujarati,AutoFakeBold=2.5,AutoFakeSlant=0.3]{Noto Sans Gujarati}

% Translations for polyglossia
\gappto\captionsgujarati{
  \renewcommand{\tablename}{કોષ્ટક}
  \renewcommand{\figurename}{આકૃતિ}
}

% Helper for TikZ nodes to ensure Gujarati font
\newcommand{\gu}[1]{{\gujaratifont #1}}

% Custom environments
\newtcolorbox{solutionbox}{
    breakable,
    enhanced,
    colback=solutioncolor!5!white,
    colframe=solutioncolor!75!black,
    fonttitle=\bfseries,
    title=જવાબ
}

\newtcolorbox{solutionboxnobreak}{
 colback=solutioncolor!5!white,
 colframe=solutioncolor!75!black,
 fonttitle=\bfseries,
 title=જવાબ
}

\newtcolorbox{keyformula}{
 breakable,
 enhanced,
 colback=keycolor!5!white,
 colframe=keycolor!75!black,
 fonttitle=\bfseries,
 title=રાસાયણિક સમીકરણ/સૂત્ર
}

\newtcolorbox{mnemonicbox}{
 breakable,
 enhanced,
 colback=mnemoniccolor!5!white,
 colframe=mnemoniccolor!75!black,
 fonttitle=\bfseries,
 title=મેમરી ટ્રીક
}


\begin{document}

\begin{center}
{\Huge\bfseries\color{headcolor} Subject Name (Gujarati)}\\[5pt]
{\LARGE 4331104 -- Summer 2023}\\[3pt]
{\large Semester 1 Study Material}\\[3pt]
{\normalsize\textit{Detailed Solutions and Explanations}}
\end{center}

\vspace{10pt}

\subsection*{પ્રશ્ન 1(a) [3
ગુણ]}\label{q1a}

\textbf{સંચાર પ્રણાલી નો બ્લોક ડાયાગ્રામ દોરો અને સમજાવો.}

\begin{solutionbox}

\begin{center}
\textbf{Mermaid Diagram (Code)}
\begin{verbatim}
{Shaded}
{Highlighting}[]
graph LR
    A[Input] {-{-}{} B[Transmitter]}
    B {-{-}{} C[Channel]}
    C {-{-}{} D[Receiver]}
    D {-{-}{} E[Output]}
    F[Noise Source] {-{-}{} C}
{Highlighting}
{Shaded}
\end{verbatim}
\end{center}

\begin{itemize}
\tightlist
\item
  \textbf{Input}: સ્ત્રોતથી આવતો મેસેજ સિગ્નલ
\item
  \textbf{Transmitter}: મેસેજને પ્રસારણ માટે યોગ્ય સ્વરૂપમાં રૂપાંતરિત કરે છે
\item
  \textbf{Channel}: જેના દ્વારા સિગ્નલ પ્રવાસ કરે છે તે માધ્યમ
\item
  \textbf{Receiver}: પ્રાપ્ત સિગ્નલમાંથી મૂળ સંદેશો કાઢે છે
\item
  \textbf{Output}: ગંતવ્ય સ્થાને પહોંચાડવામાં આવેલો સંદેશ
\item
  \textbf{Noise Source}: અવાંછિત સિગ્નલ્સ જે સંચારમાં દખલ કરે છે
\end{itemize}

\end{solutionbox}
\begin{mnemonicbox}
``સંદેશ પ્રસારક માધ્યમ પ્રાપ્તકર્તા ઉત્પાદન''

\end{mnemonicbox}
\subsection*{પ્રશ્ન 1(b) [4
ગુણ]}\label{q1b}

\textbf{મોડ્યુલેશનની જરૂરિયાત અને ફાયદા સમજાવો.}

\begin{solutionbox}

\textbf{મોડ્યુલેશનની જરૂરિયાત:}

\begin{center}
\textbf{Mermaid Diagram (Code)}
\begin{verbatim}
{Shaded}
{Highlighting}[]
graph TD
    A[Practical Antenna Size] {-{-}{} B[Modulation]}
    C[Multiplexing] {-{-}{} B}
    D[Reducing Noise \& Interference] {-{-}{} B}
    E[Signal Transmission Distance] {-{-}{} B}
{Highlighting}
{Shaded}
\end{verbatim}
\end{center}

\textbf{મોડ્યુલેશનના ફાયદાઓ:}

\begin{itemize}
\tightlist
\item
  \textbf{એન્ટેનાનું ઘટાડેલું કદ}: વ્યવહારિક એન્ટેના લંબાઈ = λ/4, ઊંચી ફ્રિક્વન્સીનો
  અર્થ નાના એન્ટેના
\item
  \textbf{મલ્ટિપ્લેક્સિંગ શક્ય}: એક જ ચેનલ દ્વારા એક સાથે અનેક સિગ્નલો પ્રસારિત થાય
  છે
\item
  \textbf{વધુ રેન્જ}: મોડ્યુલેટેડ સિગ્નલ્સ બેઝબેન્ડ સિગ્નલ્સ કરતાં વધુ દૂર સુધી પહોંચે છે
\item
  \textbf{નોઇઝ ઘટાડો}: મોડ્યુલેશન તકનીકો દ્વારા વધુ સારું SNR પ્રાપ્ત થાય છે
\end{itemize}

\end{solutionbox}
\begin{mnemonicbox}
``એન્ટેના, મલ્ટિપ્લેક્સિંગ, દૂરગામી પ્રસારણ અને નોઇઝ ઇમ્યુનિટી''

\end{mnemonicbox}
\subsection*{પ્રશ્ન 1(c) [7
ગુણ]}\label{q1c}

\textbf{મોડ્યુલેશનને વ્યાખ્યાયિત કરો. એમ્પ્લિટ્યુડ મોડ્યુલેશનને વેવફોર્મ સાથે સમજાવો અને
મોડ્યુલેટેડ સિગ્નલ માટે વોલ્ટેજ સમીકરણ મેળવો.}

\begin{solutionbox}

\textbf{મોડ્યુલેશન}: કેરિયર સિગ્નલના પરિમાણ (એમ્પ્લિટ્યુડ, ફ્રિક્વન્સી, ફેઝ) ને મેસેજ
સિગ્નલના પ્રમાણમાં બદલવાની પ્રક્રિયા.

\textbf{એમ્પ્લિટ્યુડ મોડ્યુલેશન વેવફોર્મ:}

\begin{verbatim}
                                      AM Waveform
    │        Carrier           │        Message           │     Modulated Signal
    │                          │                          │
    │  ╱╲    ╱╲    ╱╲    ╱╲    │                          │     ╱╲      ╱╲  
    │ ╱  ╲  ╱  ╲  ╱  ╲  ╱  ╲   │         ╱╲               │    ╱  ╲    ╱  ╲ 
    │╱    ╲╱    ╲╱    ╲╱    ╲  │        ╱  ╲              │   ╱    ╲  ╱    ╲
{-{-}{-}{-}┼{-}{-}{-}{-}{-}{-}{-}{-}{-}{-}{-}{-}{-}{-}{-}{-}{-}{-}{-}{-}{-}     │{-}{-}{-}{-}{-}{-}{-}╱{-}{-}{-}{-}╲{-}{-}{-}{-}{-}{-}{-}{-}{-}    │{-}{-}╱{-}{-}{-}{-}{-}{-}╲╱{-}{-}{-}{-}{-}{-}╲{-}{-}{-}{-}{-}}
    │                          │      ╱      ╲            │ ╱                 ╲
    │                          │     ╱        ╲           │╱                   ╲
    │                          │    ╱          ╲          │                     ╲
\end{verbatim}

\textbf{AM વોલ્ટેજ સમીકરણની ગાણિતિક સમજ:}

\begin{enumerate}
\tightlist
\item
  કેરિયર સિગ્નલ: vc(t) = Vc sin(ωct)
\item
  મેસેજ સિગ્નલ: vm(t) = Vm sin(ωmt)
\item
  મોડ્યુલેટેડ સિગ્નલ: vAM(t) = [Vc + Vm sin(ωmt)] sin(ωct)
\item
  મોડ્યુલેશન ઇન્ડેક્સ: μ = Vm/Vc
\item
  અંતિમ AM સમીકરણ: vAM(t) = Vc[1 + μ sin(ωmt)] sin(ωct)
\end{enumerate}

\end{solutionbox}
\begin{mnemonicbox}
``એમ્પ્લિટ્યુડ મોડ્યુલેશન કેરિયરનું મૂલ્ય બદલે છે''

\end{mnemonicbox}
\subsection*{પ્રશ્ન 1(c) OR [7
ગુણ]}\label{q1c}

\textbf{ઘોંઘાટને વ્યાખ્યાયિત કરો. ઘોંઘાટનું વર્ગીકરણ આપો અને કોઈપણ ત્રણ આંતરિક
ઘોંઘાટના કારણને સમજાવો.}

\begin{solutionbox}

\textbf{ઘોંઘાટ (Noise)}: અવાંછિત સિગ્નલ્સ જે સંચાર સિગ્નલ્સમાં દખલ કરે છે, જેના કારણે
વિકૃતિ અથવા ભૂલો થાય છે.

\textbf{ઘોંઘાટનું વર્ગીકરણ:}

{\def\LTcaptype{none} % do not increment counter
\begin{longtable}[]{@{}ll@{}}
\toprule\noalign{}
બાહ્ય ઘોંઘાટ (External Noise) & આંતરિક ઘોંઘાટ (Internal Noise) \\
\midrule\noalign{}
\endhead
\bottomrule\noalign{}
\endlastfoot
વાતાવરણીય (Atmospheric) & થર્મલ (Thermal) \\
અવકાશીય (Extraterrestrial) & શોટ (Shot) \\
ઔદ્યોગિક (Industrial) & ટ્રાન્ઝિટ-ટાઇમ (Transit-time) \\
& ફ્લિકર (Flicker) \\
& પાર્ટિશન (Partition) \\
\end{longtable}
}

\textbf{આંતરિક ઘોંઘાટના કારણો:}

\begin{itemize}
\tightlist
\item
  \textbf{થર્મલ નોઇઝ}:

  \begin{itemize}
  \tightlist
  \item
    વાહકોમાં ઇલેક્ટ્રોન્સની રેન્ડમ ગતિને કારણે ઉત્પન્ન થાય છે
  \item
    બધા ઇલેક્ટ્રોનિક ઘટકોમાં હાજર હોય છે
  \item
    તાપમાન અને બેન્ડવિડ્થ સાથે સીધા પ્રમાણમાં છે
  \end{itemize}
\item
  \textbf{શોટ નોઇઝ}:

  \begin{itemize}
  \tightlist
  \item
    જંક્શન પર કેરિયર્સની રેન્ડમ આવવાને કારણે ઉત્પન્ન થાય છે
  \item
    ડાયોડ અને ટ્રાન્ઝિસ્ટર જેવા એક્ટિવ ડિવાઇસમાં જોવા મળે છે
  \item
    ડિવાઇસમાં વહેતા DC કરંટના પ્રમાણમાં હોય છે
  \end{itemize}
\item
  \textbf{ફ્લિકર નોઇઝ}:

  \begin{itemize}
  \tightlist
  \item
    સેમીકન્ડક્ટરમાં સરફેસ ડિફેક્ટ્સ અને અશુદ્ધિઓને કારણે ઉત્પન્ન થાય છે
  \item
    ફ્રિક્વન્સીના વ્યસ્ત પ્રમાણમાં હોય છે (1/f નોઇઝ)
  \item
    ઓછી ફ્રિક્વન્સીએ મહત્વપૂર્ણ છે
  \end{itemize}
\end{itemize}

\end{solutionbox}
\begin{mnemonicbox}
``થર્મલ શોટ ફ્લિકર સર્વત્ર ઘોંઘાટ છે''

\end{mnemonicbox}
\subsection*{પ્રશ્ન 2(a) [3
ગુણ]}\label{q2a}

\textbf{વ્યાખ્યાયિત કરો. (૧) મોડ્યુલેશન ઈન્ડેક્સ (એએમ) (2) ઘોંઘાટની ફિગર (3)
ડિજીટલ મોડ્યુલેશન}

\begin{solutionbox}

\begin{enumerate}
\tightlist
\item
  \textbf{મોડ્યુલેશન ઈન્ડેક્સ (AM)}: મોડ્યુલેટિંગ સિગ્નલના એમ્પ્લિટ્યુડનો કેરિયર સિગ્નલના
  એમ્પ્લિટ્યુડ સાથેનો ગુણોત્તર.

  \begin{itemize}
  \tightlist
  \item
    μ = Vm/Vc
  \item
    વિકૃતિ ટાળવા માટે 0 \leq μ \leq 1 હોવું જોઈએ
  \end{itemize}
\item
  \textbf{ઘોંઘાટની ફિગર (Noise Figure)}: કોઈ ડિવાઇસના ઇનપુટ SNR અને આઉટપુટ
  SNRનો ગુણોત્તર.

  \begin{itemize}
  \tightlist
  \item
    NF = (SNR)input/(SNR)output
  \item
    સિસ્ટમ દ્વારા ઉમેરાયેલ ઘોંઘાટ દર્શાવે છે
  \item
    હંમેશા \geq 1, dBમાં વ્યક્ત થાય છે
  \end{itemize}
\item
  \textbf{ડિજીટલ મોડ્યુલેશન}: કેરિયર સિગ્નલના પરિમાણોમાં ફેરફાર કરીને ડિજિટલ
  ડેટાને રજૂ કરવાની તકનીક.

  \begin{itemize}
  \tightlist
  \item
    ઉદાહરણો: ASK, FSK, PSK, QAM
  \item
    ડિજિટલ ડેટા ટ્રાન્સમિશન માટે વપરાય છે
  \end{itemize}
\end{enumerate}

\end{solutionbox}
\begin{mnemonicbox}
``મોડ્યુલેશન માપે, ઘોંઘાટ અંક, ડિજિટલ ડેટા''

\end{mnemonicbox}
\subsection*{પ્રશ્ન 2(b) [4
ગુણ]}\label{q2b}

\textbf{કેરિયર પાવર અને મોડ્યુલેશન ઈન્ડેક્સ ને ધ્યાનમાં લેતા એમ્પ્લીટ્યુડ મોડ્યુલેટેડ સિગ્નલ
માટે પરિવહન થયેલ કુલ પાવર માટે સમીકરણ મેળવો.}

\begin{solutionbox}

\textbf{AMમાં કુલ પાવરનું સમીકરણ:}

\begin{enumerate}
\item
  AM વેવ સમીકરણ: vAM(t) = Vc[1 + μ sin(ωmt)] sin(ωct)
\item
  પાવર ગણતરી માટે, RMS મૂલ્યો ધ્યાનમાં લો:

  \begin{itemize}
  \tightlist
  \item
    કેરિયર પાવર (Pc) = Vc^{2}/2R
  \item
    દરેક સાઇડબેન્ડમાં પાવર (PSB) = (μ^{2}Vc^{2})/(4R)
  \end{itemize}
\item
  કુલ પાવર સમીકરણ:

  \begin{itemize}
  \tightlist
  \item
    PT = Pc + PUSB + PLSB
  \item
    PT = Pc + 2PSB (કારણ કે ઉપર અને નીચેના સાઇડબેન્ડમાં સમાન પાવર હોય છે)
  \item
    PT = Vc^{2}/2R + 2(μ^{2}Vc^{2})/(4R)
  \item
    PT = (Vc^{2}/2R)[1 + (μ^{2}/2)]
  \end{itemize}
\item
  અંતિમ સમીકરણ: PT = Pc(1 + μ^{2}/2)
\end{enumerate}

\end{solutionbox}
\begin{mnemonicbox}
``કુલ પાવર = કેરિયર પાવર (1 + μ^{2}/2)''

\end{mnemonicbox}
\subsection*{પ્રશ્ન 2(c) [7
ગુણ]}\label{q2c}

\textbf{ડબલ સાઇડબેન્ડ દબાયેલા વાહક એમ્પ્લીટ્યુડ મોડ્યુલેશનના મૂળભૂત સિદ્ધાંતને સમજાવો.
તેના વોલ્ટેજ સમીકરણ મેળવો અને ડાયોડનો ઉપયોગ કરીને તેની માત્ર મોડ્યુલેટર સરકિટ
દોરો.}

\begin{solutionbox}

\textbf{ડબલ સાઇડબેન્ડ સપ્રેસ્ડ કેરિયર (DSBSC) સિદ્ધાંત:}

\begin{itemize}
\tightlist
\item
  કેરિયરને દબાવી દેવામાં આવે છે, માત્ર સાઇડબેન્ડ્સને પ્રસારિત કરવામાં આવે છે
\item
  બધી માહિતી સાઇડબેન્ડમાં સમાયેલ હોય છે
\item
  AMની તુલનામાં વધુ પાવર અસરકારક છે
\item
  ડિમોડ્યુલેશન માટે જટિલ રિસીવરની જરૂર પડે છે
\end{itemize}

\textbf{વોલ્ટેજ સમીકરણની ગાણિતિક સમજ:}

\begin{enumerate}
\tightlist
\item
  AM સિગ્નલ: vAM(t) = Vc[1 + μ sin(ωmt)]sin(ωct)
\item
  કેરિયર ઘટક દૂર કરવો: vDSBSC(t) = Vc \times μ sin(ωmt)sin(ωct)
\item
  ત્રિકોણમિતીય ઓળખનો ઉપયોગ: sin(A)sin(B) = 0.5[cos(A-B) - cos(A+B)]
\item
  અંતિમ સમીકરણ: vDSBSC(t) = (Vcμ/2)[cos(ωc-ωm)t - cos(ωc+ωm)t]
\end{enumerate}

\textbf{ડાયોડનો ઉપયોગ કરીને બેલેન્સ્ડ મોડ્યુલેટર સર્કિટ:}

\begin{verbatim}
           D1
        |{-{-}{-}|{-}{-}+}
        |       |
     +{-{-}+       +{-}{-}+}
     |             |
Vc{-{-}{-}|             |{-}{-}{-}Output}
     |             |
     +{-{-}+       +{-}{-}+}
        |       |
        |{-{-}{-}|{-}{-}+}
           D2
           |
           |
         Carrier
           |
           V
       Modulating
         Signal
\end{verbatim}

\end{solutionbox}
\begin{mnemonicbox}
``કેરિયર દૂર કરો, બેન્ડવિડ્થ બચાવો, સિગ્નલો જોડો''

\end{mnemonicbox}
\subsection*{પ્રશ્ન 2(a) OR [3
ગુણ]}\label{q2a}

\textbf{માત્ર રેડિયો રીસીવર નાં સંદર્ભે વ્યાખ્યાયિત કરો, (1) સંવેદનશીલતા (2)
સીલેકટીવિટી (3) ફાઈડાલીટી}

\begin{solutionbox}

\begin{enumerate}
\tightlist
\item
  \textbf{સંવેદનશીલતા (Sensitivity)}: નબળા સિગ્નલ્સને શોધવા અને એમ્પ્લિફાય
  કરવાની રીસીવરની ક્ષમતા.

  \begin{itemize}
  \tightlist
  \item
    માઇક્રોવોલ્ટ (μV)માં માપવામાં આવે છે
  \item
    નીચું મૂલ્ય વધુ સારી સંવેદનશીલતા દર્શાવે છે
  \item
    વ્યાવસાયિક રિસીવર્સ માટે સામાન્ય રીતે 1-10 μV
  \end{itemize}
\item
  \textbf{સીલેકટીવિટી (Selectivity)}: ઇચ્છિત સિગ્નલ અને અડોસપડોસના દખલ કરતા
  સિગ્નલ્સ વચ્ચે ભેદ કરવાની ક્ષમતા.

  \begin{itemize}
  \tightlist
  \item
    -3dB પોઇન્ટ્સ પર બેન્ડવિડ્થ તરીકે માપવામાં આવે છે
  \item
    સાંકડી બેન્ડવિડ્થનો અર્થ વધુ સારી સીલેકટીવિટી
  \item
    અડોસપડોસના ચેનલ ઇન્ટરફેરન્સને રોકે છે
  \end{itemize}
\item
  \textbf{ફાઈડાલીટી (Fidelity)}: રિસીવર મૂળ સંદેશને કેટલી ચોકસાઈથી પુનઃઉત્પાદિત
  કરે છે તે.

  \begin{itemize}
  \tightlist
  \item
    પુનઃઉત્પાદનની ગુણવત્તા માપે છે
  \item
    વિકૃતિ અને ઘોંઘાટથી પ્રભાવિત થાય છે
  \item
    ઉચ્ચ ફાઈડાલીટીનો અર્થ વધુ સારી સાઉન્ડ ક્વોલિટી
  \end{itemize}
\end{enumerate}

\end{solutionbox}
\begin{mnemonicbox}
``સંવેદી પસંદગી વફાદારીથી''

\end{mnemonicbox}
\subsection*{પ્રશ્ન 2(b) OR [4
ગુણ]}\label{q2b}

\textbf{એએમ સિગ્નલમાં દરેક સાઇડબેન્ડમાં ૨૦૦ વોટ સાથે ૧ કિલો વોટનો કેરિયર પાવર છે.
આ માટે મોડ્યુલેશન ઇન્ડેક્સ શોધો.}

\begin{solutionbox}

\textbf{આપેલ:}

\begin{itemize}
\tightlist
\item
  કેરિયર પાવર (Pc) = 1 KW = 1000 W
\item
  દરેક સાઇડબેન્ડમાં પાવર (PSB) = 200 W
\end{itemize}

\textbf{શોધવાનું:} મોડ્યુલેશન ઇન્ડેક્સ (μ)

\textbf{ઉકેલ:}

\begin{enumerate}
\tightlist
\item
  કુલ સાઇડબેન્ડ પાવર: PTSB = 2 \times PSB = 2 \times 200 = 400 W
\item
  સૂત્રનો ઉપયોગ: PTSB = Pc \times μ^{2}/2
\item
  400 = 1000 \times μ^{2}/2
\item
  μ^{2} = (400 \times 2)/1000 = 800/1000 = 0.8
\item
  μ = \sqrt0.8 = 0.894 = 0.9 (આશરે)
\end{enumerate}

\end{solutionbox}
\begin{mnemonicbox}
``સાઇડબેન્ડ પાવર મોડ્યુલેશન ઇન્ડેક્સ બતાવે છે''

\end{mnemonicbox}
\subsection*{પ્રશ્ન 2(c) OR [7
ગુણ]}\label{q2c}

\textbf{લઘુત્તમ સાત પરિમાણો/પાસાને ધ્યાનમાં રાખીને ફ્રિક્વન્સી મોડ્યુલેશન સાથે
એમ્પ્લિટ્યૂડ મોડ્યુલેશનની તુલના કરો.}

\begin{solutionbox}

{\def\LTcaptype{none} % do not increment counter
\begin{longtable}[]{@{}
  >{\raggedright\arraybackslash}p{(\linewidth - 4\tabcolsep) * \real{0.1692}}
  >{\raggedright\arraybackslash}p{(\linewidth - 4\tabcolsep) * \real{0.4154}}
  >{\raggedright\arraybackslash}p{(\linewidth - 4\tabcolsep) * \real{0.4154}}@{}}
\toprule\noalign{}
\begin{minipage}[b]{\linewidth}\raggedright
પરિમાણ
\end{minipage} & \begin{minipage}[b]{\linewidth}\raggedright
એમ્પ્લિટ્યૂડ મોડ્યુલેશન (AM)
\end{minipage} & \begin{minipage}[b]{\linewidth}\raggedright
ફ્રિક્વન્સી મોડ્યુલેશન (FM)
\end{minipage} \\
\midrule\noalign{}
\endhead
\bottomrule\noalign{}
\endlastfoot
\textbf{વ્યાખ્યા} & કેરિયરનો એમ્પ્લિટ્યૂડ મેસેજ સાથે બદલાય છે & કેરિયરની ફ્રિક્વન્સી
મેસેજ સાથે બદલાય છે \\
\textbf{બેન્ડવિડ્થ} & સાંકડી (2 \times fm) & વિશાળ (2 \times β \times fm) \\
\textbf{પાવર કાર્યક્ષમતા} & નબળી (કેરિયરમાં \textasciitilde66\% પાવર) & સારી
(બધો પાવર સાઇડબેન્ડમાં) \\
\textbf{ઘોંઘાટ રક્ષણ} & નબળું (ઘોંઘાટ એમ્પ્લિટ્યૂડને અસર કરે છે) & ઉત્તમ (એમ્પ્લિટ્યૂડ
લિમિટર્સ ઘોંઘાટ દૂર કરે છે) \\
\textbf{સર્કિટ જટિલતા} & સરળ ટ્રાન્સમીટર અને રિસીવર & જટિલ ટ્રાન્સમીટર અને
રિસીવર \\
\textbf{ગુણવત્તા} & ઓછી ફાઈડાલીટી & ઉચ્ચ ફાઈડાલીટી \\
\textbf{ઉપયોગો} & બ્રોડકાસ્ટિંગ, એરક્રાફ્ટ કમ્યુનિકેશન & FM રેડિયો, TV સાઉન્ડ,
વાયરલેસ માઇક \\
\textbf{સ્પેક્ટ્રમ} & કેરિયર અને બે સાઇડબેન્ડ ધરાવે છે & અનંત સાઇડબેન્ડ ધરાવે છે \\
\end{longtable}
}

\end{solutionbox}
\begin{mnemonicbox}
``બેન્ડવિડ્થ, કાર્યક્ષમતા, ઘોંઘાટ, ગુણવત્તા - AM ઘણી ગુણવત્તા
કસોટીઓમાં નિષ્ફળ જાય છે''

\end{mnemonicbox}
\subsection*{પ્રશ્ન 3(a) [3
ગુણ]}\label{q3a}

\textbf{૧ કિલો હર્ટ્ઝનાં સાઈન વેવ સિગ્નલને ટાઇમ ડોમેઇન અને ફ્રીક્વન્સી ડોમેન માં દોરો
અને લેબલ કરો. સિગ્નલના ડોમેન ફ્રીક્વન્સી ડોમેન વિશ્લેષણ નાં ફાયદા જણાવો.}

\begin{solutionbox}

\textbf{ટાઇમ ડોમેઇન રજૂઆત:}

\begin{verbatim}
    Amplitude
        \^{}
        |
    1   |    /{      /      /      /}
        |   /  {    /      /      /  }
        |  /    {  /      /      /    }
    0   |{-+{-}{-}{-}{-}{-}{-}+{-}{-}{-}{-}{-}{-}{-}+{-}{-}{-}{-}{-}{-}{-}+{-}{-}{-}{-}{-}{-}{-}+{-}{-}{-}{-}{-}{-}{-} Time}
        |  {    /      /      /      /}
        |   {  /      /      /      /}
   {-1   |    /      /      /      /}
        |
     1KHz sine wave (Period = 1ms)
\end{verbatim}

\textbf{ફ્રિક્વન્સી ડોમેઇન રજૂઆત:}

\begin{verbatim}
    Amplitude
        \^{}
        |
    1   |    |
        |    |
        |    |
    0   |{-{-}{-}{-}+{-}{-}{-}{-}+{-}{-}{-}{-}+{-}{-}{-}{-}+{-}{-}{-}{-}+{-}{-}{-}{-}+{-}{-}{-}{-}{-}{-}{-} Frequency}
        |    0   1KHz           
        |
     Single spectral line at 1KHz
\end{verbatim}

\textbf{ફ્રિક્વન્સી ડોમેઇન વિશ્લેષણના ફાયદા:}

\begin{itemize}
\tightlist
\item
  \textbf{સિગ્નલ રચના}: સરળતાથી ફ્રિક્વન્સી ઘટકોની ઓળખ
\item
  \textbf{ફિલ્ટર ડિઝાઇન}: સરળ ફિલ્ટર પ્રતિસાદ વિશ્લેષણ
\item
  \textbf{બેન્ડવિડ્થ નિર્ધારણ}: સ્પેક્ટ્રમ પહોળાઈનું સીધું વિઝ્યુઅલાઇઝેશન
\item
  \textbf{ઘોંઘાટ વિશ્લેષણ}: સિગ્નલને ઘોંઘાટથી વધુ સારી રીતે અલગ કરવું
\end{itemize}

\end{solutionbox}
\begin{mnemonicbox}
``ફ્રિક્વન્સી સમયમાં છુપાયેલા ઘટકો બતાવે છે''

\end{mnemonicbox}
\subsection*{પ્રશ્ન 3(b) [4
ગુણ]}\label{q3b}

\textbf{નીચેનાં પ્રશ્નો માટે આવૃત્તિ જણાવો. (1) એએમ રેડિયો માટે આઇએફ (IF)
ફ્રિક્વન્સી (૨) એફએમ રેડિયો માટે આઇએફ ફ્રિક્વન્સી (3) એફએમ રેડિયો માટે વપરાતો
ફ્રિક્વન્સી બેન્ડ (4) માનવવાણીનો ફ્રિક્વન્સી બેન્ડ.}

\begin{solutionbox}

{\def\LTcaptype{none} % do not increment counter
\begin{longtable}[]{@{}ll@{}}
\toprule\noalign{}
પરિમાણ & આવૃત્તિ \\
\midrule\noalign{}
\endhead
\bottomrule\noalign{}
\endlastfoot
એએમ રેડિયો માટે આઇએફ ફ્રિક્વન્સી & 455 kHz \\
એફએમ રેડિયો માટે આઇએફ ફ્રિક્વન્સી & 10.7 MHz \\
એફએમ રેડિયો માટે વપરાતો ફ્રિક્વન્સી બેન્ડ & 88-108 MHz \\
માનવવાણીનો ફ્રિક્વન્સી બેન્ડ & 300 Hz - 3.4 kHz \\
\end{longtable}
}

\end{solutionbox}
\begin{mnemonicbox}
``AM455, FM10.7, બેન્ડ88-108, વાણી300-3.4''

\end{mnemonicbox}
\subsection*{પ્રશ્ન 3(c) [7
ગુણ]}\label{q3c}

\textbf{સિંગલ સાઇડ બેન્ડ (એસએસબી) મોડ્યુલેશન તેના વેવફોર્મ અને ફાયદા સાથે સમજાવો.
બતાવો કે કેવી રીતે SSB ટ્રાન્સમિશનને ડબલ સાઇડબેન્ડ પૂર્ણ વાહક એમ્પ્લીટ્યુડ મોડ્યુલેશન ને
અનુલક્ષીને માત્ર ૧/૬ (છઠ્ઠા ભાગના) પાવરની જરૂર છે.}

\begin{solutionbox}

\textbf{સિંગલ સાઇડ બેન્ડ (SSB) મોડ્યુલેશન:}

\begin{itemize}
\tightlist
\item
  માત્ર એક જ સાઇડબેન્ડ (USB અથવા LSB) પ્રસારિત કરે છે
\item
  કેરિયર અને બીજા સાઇડબેન્ડને દબાવી દેવામાં આવે છે
\item
  બેન્ડવિડ્થ અને પાવર બચાવે છે
\end{itemize}

\textbf{SSB વેવફોર્મ:}

\begin{verbatim}
    Frequency Spectrum
        \^{}
        |
        |                   Regular AM
        |    |              |     |
        |    |              |     |
        |{-{-}{-}{-}+{-}{-}{-}{-}+{-}{-}{-}{-}+{-}{-}{-}{-}+{-}{-}{-}{-}{-}+{-}{-}{-}{-}{-} Frequency}
             fc{-fm   fc   fc+fm}

        |                   SSB (USB)
        |                  |
        |                  |
        |{-{-}{-}{-}+{-}{-}{-}{-}+{-}{-}{-}{-}+{-}{-}{-}{-}+{-}{-}{-}{-}+{-}{-}{-}{-}{-} Frequency}
                         fc+fm
\end{verbatim}

\textbf{SSBના ફાયદા:}

\begin{itemize}
\tightlist
\item
  \textbf{બેન્ડવિડ્થ કાર્યક્ષમતા}: AMની અડધી બેન્ડવિડ્થનો ઉપયોગ કરે છે
\item
  \textbf{પાવર કાર્યક્ષમતા}: કેરિયર પર કોઈ પાવર બરબાદ થતો નથી
\item
  \textbf{ઓછું ફેડિંગ}: લાંબા અંતરના સંચારમાં સુધારેલ કામગીરી
\item
  \textbf{વધુ સારો SNR}: માહિતીમાં વધુ પાવર કેન્દ્રિત
\end{itemize}

\textbf{પાવર તુલના:}

\begin{enumerate}
\tightlist
\item
  AMમાં: PT = Pc(1 + μ^{2}/2)
\item
  μ = 1 માટે, PT = Pc(1 + 0.5) = 1.5Pc
\item
  AM પાવર વિતરણ: કેરિયર (Pc) = 67\%, સાઇડબેન્ડ્સ = 33\%
\item
  SSB માત્ર એક સાઇડબેન્ડનો ઉપયોગ કરે છે અને કેરિયર નથી
\item
  SSB પાવર = કુલ AM પાવરનો 16.5\% = 1/6 આશરે
\end{enumerate}

\end{solutionbox}
\begin{mnemonicbox}
``એક બેન્ડ બેન્ડવિડ્થ અને પાવર બચાવે છે''

\end{mnemonicbox}
\subsection*{પ્રશ્ન 3(a) OR [3
ગુણ]}\label{q3a}

\begin{solutionbox}
\textbf{જવાબ આપો. (1) જો મોડ્યુલેટિંગ ફ્રિક્વન્સી 5 KHZ હોય તો એમ્પ્લીટ્યુડ મોડ્યુલેટેડ
સિગ્નલની બેન્ડવિડ્થ. (2) એએમ રેડિયોમાં જો પસંદ કરેલ સ્ટેશનની આવૃત્તિ 1000 KhZ હોય તો
ઈમેજ સિગ્નલ ની આવૃત્તિ (3) બેઝબેન્ડ સિગ્નલની આવૃત્તિ 10 KHz હોય તો તેની સેમ્પલીંગ
આવૃત્તિ.}

\end{solutionbox}
\begin{solutionbox}

\begin{enumerate}
\tightlist
\item
  \textbf{5 kHz મોડ્યુલેટિંગ ફ્રિક્વન્સી સાથે AM બેન્ડવિડ્થ:}

  \begin{itemize}
  \tightlist
  \item
    BW = 2 \times fm = 2 \times 5 kHz = 10 kHz
  \end{itemize}
\item
  \textbf{1000 kHz સ્ટેશન માટે 455 kHz IF સાથે ઇમેજ ફ્રિક્વન્સી:}

  \begin{itemize}
  \tightlist
  \item
    હાઇ-સાઇડ ઇન્જેક્શન માટે: fimage = fstation + 2 \times fIF
  \item
    fimage = 1000 + 2 \times 455 = 1000 + 910 = 1910 kHz
  \end{itemize}
\item
  \textbf{10 kHz બેઝબેન્ડ માટે સેમ્પલિંગ ફ્રિક્વન્સી:}

  \begin{itemize}
  \tightlist
  \item
    fs \textgreater{} 2 \times fmax (નાઇક્વિસ્ટ રેટ)
  \item
    fs \textgreater{} 2 \times 10 kHz = 20 kHz
  \item
    સેમ્પલિંગ ફ્રિક્વન્સી \textgreater{} 20 kHz હોવી જોઈએ
  \end{itemize}
\end{enumerate}

\end{solutionbox}
\begin{mnemonicbox}
``બેન્ડવિડ્થ બમણી, ઇમેજ બે-IF ઉમેરે, સેમ્પલિંગ બમણી-ફ્રિક્વન્સી
જોઈએ''

\end{mnemonicbox}
\subsection*{પ્રશ્ન 3(b) OR [4
ગુણ]}\label{q3b}

\textbf{ગાણિતિક સમીકરણ દર્શાવતા નીચે મુજબના સિગ્નલો દોરો. (1) સાઇન વેવ સિગ્નલ
(2) યુનિટ સ્ટેપ સિગ્નલ (3) રેમ્પ સિગ્નલ (4) ઇમ્પલ્સ સિગ્નલ.}

\begin{solutionbox}

\textbf{1. સાઇન વેવ:}

\begin{itemize}
\tightlist
\item
  સમીકરણ: f(t) = A sin(ωt + φ)
\end{itemize}

\begin{verbatim}
        \^{}
        |
    A   |    /{      /      }
        |   /  {    /      }
        |  /    {  /      }
    0   |{-+{-}{-}{-}{-}{-}{-}+{-}{-}{-}{-}{-}{-}{-}+{-}{-}{-}{-} t}
        |  {    /      /  }
        |   {  /      /    }
   {-A   |    /      /      }
\end{verbatim}

\textbf{2. યુનિટ સ્ટેપ સિગ્નલ:}

\begin{itemize}
\tightlist
\item
  સમીકરણ: u(t) = 1 માટે t \geq 0, 0 માટે t \textless{} 0
\end{itemize}

\begin{verbatim}
        \^{}
        |
    1   |        |‾‾‾‾‾‾‾‾‾‾‾‾‾‾‾‾
        |        |
        |        |
    0   |‾‾‾‾‾‾‾‾+‾‾‾‾‾‾‾‾‾‾‾‾‾‾‾{ t}
        |        0
\end{verbatim}

\textbf{3. રેમ્પ સિગ્નલ:}

\begin{itemize}
\tightlist
\item
  સમીકરણ: r(t) = t માટે t \geq 0, 0 માટે t \textless{} 0
\end{itemize}

\begin{verbatim}
        \^{}
        |                /
        |               /
        |              /
        |             /
        |            /
    0   |‾‾‾‾‾‾‾‾‾‾‾+‾‾‾‾‾‾‾‾‾‾‾‾‾‾{ t}
        |           0
\end{verbatim}

\textbf{4. ઇમ્પલ્સ સિગ્નલ:}

\begin{itemize}
\tightlist
\item
સમીકરણ: δ(t) = \infty માટે

t = 0, 0 માટે t \neq 0

\end{itemize}

\begin{verbatim}
        \^{}
        |
        |
        |        |
        |        |
    0   |‾‾‾‾‾‾‾‾+‾‾‾‾‾‾‾‾‾‾‾‾‾‾‾{ t}
        |        0
\end{verbatim}

\end{solutionbox}
\begin{mnemonicbox}
``સાઇન હલે છે, સ્ટેપ કૂદે છે, રેમ્પ ચઢે છે, ઇમ્પલ્સ ટોચે છે''

\end{mnemonicbox}
\subsection*{પ્રશ્ન 3(c) OR [7
ગુણ]}\label{q3c}

\textbf{પ્રિ એમ્ફેસીસ અને ડી એમ્ફેસીસ સર્કિટને તેની જરૂરિયાત અને લાક્ષણિક ગ્રાફ સાથે
દોરો અને સમજાવો. એફએમ રીસીવરની તુલના વિગતવાર એએમ રીસીવર સાથે પણ કરો.}

\begin{solutionbox}

\textbf{પ્રિ-એમ્ફેસીસ સર્કિટ:}

\begin{verbatim}
        ┌───────┐
        │       │
    ────┤   R   ├────┬──────
        │       │    │
        └───────┘    │
                     │
                  ┌──┴──┐
                  │     │
                  │  C  │
                  │     │
                  └──┬──┘
                     │
                     │
                     ▼
\end{verbatim}

\textbf{ડી-એમ્ફેસીસ સર્કિટ:}

\begin{verbatim}
                  ┌───────┐
                  │       │
             ┌────┤   R   ├────
             │    │       │
             │    └───────┘
             │
     ────────┴────┐
                  │
               ┌──┴──┐
               │     │
               │  C  │
               │     │
               └─────┘
\end{verbatim}

\textbf{લાક્ષણિક ગ્રાફ:}

\begin{verbatim}
    Gain(dB)
        \^{}
        |                 Pre{-emphasis}
        |              ,/
        |            ,/
        |          ,/
    0   |‾‾‾‾‾‾‾‾‾+‾‾‾‾‾‾‾‾‾‾‾‾‾‾‾{ Frequency}
        |          fc
        |          {,}
        |            {,}
        |              {,  De{-}emphasis}
\end{verbatim}

\textbf{પ્રિ/ડી-એમ્ફેસીસની જરૂરિયાત:}

\begin{itemize}
\tightlist
\item
  \textbf{ઘોંઘાટ ઘટાડો}: ઉચ્ચ ફ્રિક્વન્સી ઘોંઘાટ માટે વધુ સંવેદનશીલ
\item
  \textbf{SNR સુધારે છે}: ટ્રાન્સમીટર પર ઉચ્ચ ફ્રિક્વન્સીને વધારે, રિસીવર પર ઘટાડે
\item
  \textbf{ટાઇમ કોન્સ્ટન્ટ}: FM પ્રસારણમાં સામાન્ય રીતે 75μs
\end{itemize}

\textbf{FM અને AM રિસીવર વચ્ચે તુલના:}

{\def\LTcaptype{none} % do not increment counter
\begin{longtable}[]{@{}lll@{}}
\toprule\noalign{}
પરિમાણ & FM રિસીવર & AM રિસીવર \\
\midrule\noalign{}
\endhead
\bottomrule\noalign{}
\endlastfoot
\textbf{IF ફ્રિક્વન્સી} & 10.7 MHz & 455 kHz \\
\textbf{બેન્ડવિડ્થ} & 200 kHz & 10 kHz \\
\textbf{લિમિટર સ્ટેજ} & હાજર & ગેરહાજર \\
\textbf{ડિમોડ્યુલેટર} & ડિસ્ક્રિમિનેટર/રેશિયો ડિટેક્ટર & એન્વેલોપ ડિટેક્ટર \\
\textbf{પ્રિ/ડી-એમ્ફેસીસ} & હાજર & ગેરહાજર \\
\textbf{ઓડિયો ક્વોલિટી} & ઉત્તમ & મધ્યમ \\
\textbf{ઘોંઘાટ ઇમ્યુનિટી} & ઉચ્ચ & નીચી \\
\textbf{જટિલતા} & વધુ જટિલ & સરળ \\
\end{longtable}
}

\end{solutionbox}
\begin{mnemonicbox}
``પ્રિ ઉચ્ચને વધારે, ડી ઉચ્ચને ઘટાડે; FM ઘોંઘાટને AM કરતાં
સારી રીતે ફિલ્ટર કરે''

\end{mnemonicbox}
\subsection*{પ્રશ્ન 4(a) [3
ગુણ]}\label{q4a}

\textbf{રેડિયો રીસીવર માટે ઈમેજ આવૃત્તિ નેવ્યાખ્યાયિત કરો અને યોગ્ય ઉદાહરણ સાથે તેને
સમજાવો.}

\begin{solutionbox}

\textbf{ઇમેજ ફ્રિક્વન્સી}: અવાંછિત સિગ્નલ ફ્રિક્વન્સી જે લોકલ ઓસિલેટર સિગ્નલ સાથે
મિક્સ થતાં ઇચ્છિત સિગ્નલ જેટલું જ IF ઉત્પન્ન કરે છે.

\textbf{સમજૂતી:}

\begin{itemize}
\tightlist
\item
  હાઇ-સાઇડ ઇન્જેક્શન માટે: fimage = fsignal + 2 \times fIF
\item
  લો-સાઇડ ઇન્જેક્શન માટે: fimage = fsignal - 2 \times fIF
\end{itemize}

\textbf{ઉદાહરણ:}

\begin{itemize}
\tightlist
\item
  ઇચ્છિત સિગ્નલ: 1000 kHz
\item
  IF: 455 kHz
\item
  લોકલ ઓસિલેટર ફ્રિક્વન્સી (હાઇ-સાઇડ): fLO = 1000 + 455 = 1455 kHz
\item
  ઇમેજ ફ્રિક્વન્સી: fimage = fLO + 455 = 1455 + 455 = 1910 kHz
\item
  1000 kHz અને 1910 kHz બંને 1455 kHz સાથે મિક્સ થતાં 455 kHz IF ઉત્પન્ન કરશે
\end{itemize}

\end{solutionbox}
\begin{mnemonicbox}
``ઇમેજ રેડિયોમાં 2IF દૂર દખલ કરે છે''

\end{mnemonicbox}
\subsection*{પ્રશ્ન 4(b) [4
ગુણ]}\label{q4b}

\textbf{એમ્પ્લિટ્યુડ મોડ્યુલેટેડ સિગ્નલના ડિમોડ્યુલેશન માટે એન્વેલોપ ડિટેક્ટર ની સર્કિટ
દોરો અને તેને સમજાવો.}

\begin{solutionbox}

\textbf{એન્વેલોપ ડિટેક્ટર સર્કિટ:}

\begin{verbatim}
         D
    ┌────►|────┬────────┐
    │          │        │
    │          │        │
Inpt│       ┌──┴──┐  ┌──┴──┐ Output
    │       │     │  │     │
    │       │  R  │  │  C  │
    │       │     │  │     │
    └───────┴──┬──┘  └──┬──┘
               │        │
               └────────┴───────
                     Ground
\end{verbatim}

\textbf{કાર્યપદ્ધતિ:}

\begin{itemize}
\tightlist
\item
  \textbf{ડાયોડ}: AM સિગ્નલનું રેક્ટિફિકેશન કરે છે, નેગેટિવ હાફ-સાયકલ્સ દૂર કરે છે
\item
  \textbf{RC સર્કિટ}: લો-પાસ ફિલ્ટર તરીકે કામ કરે છે
\item
  \textbf{ટાઇમ કોન્સ્ટન્ટ}: RC એ 1/fm \textgreater\textgreater{} RC
  \textgreater\textgreater{} 1/fc સંતોષવું જોઈએ
\item
  \textbf{આઉટપુટ}: AM સિગ્નલનો એન્વેલોપ, જે મૂળ સંદેશ છે
\end{itemize}

\textbf{એન્વેલોપ ડિટેક્શન પ્રક્રિયા:}

\begin{enumerate}
\tightlist
\item
  ડાયોડ પોઝિટિવ હાફ-સાયકલ્સ દરમિયાન કન્ડક્ટ કરે છે
\item
  કેપેસિટર પીક વેલ્યુ સુધી ચાર્જ થાય છે
\item
  નેગેટિવ હાફ-સાયકલ્સ દરમિયાન, કેપેસિટર રેસિસ્ટર દ્વારા ડિસ્ચાર્જ થાય છે
\item
  આઉટપુટ AM સિગ્નલના એન્વેલોપને અનુસરે છે
\end{enumerate}

\end{solutionbox}
\begin{mnemonicbox}
``ડાયોડ રેક્ટિફાય કરે, RC એન્વેલોપ સુધારે''

\end{mnemonicbox}
\subsection*{પ્રશ્ન 4(c) [7
ગુણ]}\label{q4c}

\textbf{એએમ રેડિયો રીસીવરનોબ્લોક ડાયાગ્રામ દોરો અને દરેક બ્લોક/સ્ટેજ ની કામગીરી
સમજાવો.}

\begin{solutionbox}

\textbf{AM રેડિયો રિસીવર (સુપરહેટરોડાઇન) બ્લોક ડાયાગ્રામ:}

\begin{center}
\textbf{Mermaid Diagram (Code)}
\begin{verbatim}
{Shaded}
{Highlighting}[]
graph LR
    A[RF Amplifier] {-{-}{} B[Mixer]}
    G[Local Oscillator] {-{-}{} B}
    B {-{-}{} C[IF Amplifier]}
    C {-{-}{} D[Detector]}
    D {-{-}{} E[AF Amplifier]}
    E {-{-}{} F[Speaker]}
{Highlighting}
{Shaded}
\end{verbatim}
\end{center}

\textbf{દરેક બ્લોકનાં કાર્યો:}

\begin{itemize}
\tightlist
\item
  \textbf{RF એમ્પ્લિફાયર}:

  \begin{itemize}
  \tightlist
  \item
    ટ્યૂન્ડ સર્કિટનો ઉપયોગ કરીને ઇચ્છિત સ્ટેશન સિગ્નલ પસંદ કરે છે
  \item
    પ્રારંભિક એમ્પ્લિફિકેશન પૂરું પાડે છે
  \item
    સંવેદનશીલતા અને સિલેક્ટિવિટી સુધારે છે
  \item
    ઇમેજ ફ્રિક્વન્સી દખલને ઘટાડે છે
  \end{itemize}
\item
  \textbf{લોકલ ઓસિલેટર}:

  \begin{itemize}
  \tightlist
  \item
    ઇનકમિંગ કરતાં IF વેલ્યુ જેટલી ઉંચી ફ્રિક્વન્સી જનરેટ કરે છે
  \item
    સામાન્ય રીતે fLO = fRF + 455 kHz
  \item
    RF એમ્પ્લિફાયર સાથે એક સાથે ટ્યૂન થાય છે
  \end{itemize}
\item
  \textbf{મિક્સર}:

  \begin{itemize}
  \tightlist
  \item
    RF સિગ્નલને લોકલ ઓસિલેટર સાથે જોડે છે
  \item
    સરવાળા અને તફાવતની ફ્રિક્વન્સી ઉત્પન્ન કરે છે
  \item
    ઇન્ટરમીડિયેટ ફ્રિક્વન્સી (IF) આઉટપુટ આપે છે
  \end{itemize}
\item
  \textbf{IF એમ્પ્લિફાયર}:

  \begin{itemize}
  \tightlist
  \item
    ફિક્સ્ડ-ફ્રિક્વન્સી એમ્પ્લિફાયર (455 kHz)
  \item
    રિસીવર ગેઇનનો મોટાભાગનો ભાગ પ્રદાન કરે છે
  \item
    રિસીવરની સિલેક્ટિવિટી નક્કી કરે છે
  \end{itemize}
\item
  \textbf{ડિટેક્ટર}:

  \begin{itemize}
  \tightlist
  \item
    IF સિગ્નલમાંથી મૂળ ઓડિયો કાઢે છે
  \item
    સામાન્ય રીતે ડાયોડ સાથે એન્વેલોપ ડિટેક્ટર
  \item
    RF ઘટક દૂર કરે છે, ઓડિયો પુનઃપ્રાપ્ત કરે છે
  \end{itemize}
\item
  \textbf{AF એમ્પ્લિફાયર}:

  \begin{itemize}
  \tightlist
  \item
    પુનઃપ્રાપ્ત ઓડિયો સિગ્નલને એમ્પ્લિફાય કરે છે
  \item
    વોલ્યુમ કંટ્રોલ શામેલ છે
  \item
    સ્પીકરને સાંભળી શકાય તેવા સ્તરે ડ્રાઇવ કરે છે
  \end{itemize}
\item
  \textbf{સ્પીકર}:

  \begin{itemize}
  \tightlist
  \item
    ઇલેક્ટ્રિકલ સિગ્નલ્સને સાઉન્ડ વેવ્સમાં રૂપાંતરિત કરે છે
  \end{itemize}
\end{itemize}

\end{solutionbox}
\begin{mnemonicbox}
``RF મિક્સિંગ IF ડિટેક્ટેડ ઓડિયો સ્પીકર માટે''

\end{mnemonicbox}
\subsection*{પ્રશ્ન 4(a) OR [3
ગુણ]}\label{q4a}

\textbf{સિગ્નલના સેમ્પલીંગ લેવા માટેના નાઈક્વિસ્ટ માપદંડ જણાવો અને સમજાવો.}

\begin{solutionbox}

\textbf{નાઈક્વિસ્ટ માપદંડ}: બેન્ડલિમિટેડ સિગ્નલને વિકૃતિ વિના પુનઃનિર્માણ કરવા માટે,
સેમ્પલિંગ ફ્રિક્વન્સી સિગ્નલમાં ઉચ્ચતમ ફ્રિક્વન્સી ઘટકથી ઓછામાં ઓછી બમણી હોવી જોઈએ.

\textbf{ગાણિતિક નિવેદન:}

\begin{itemize}
\tightlist
\item
  fs \geq 2fmax
\item
  fs = સેમ્પલિંગ ફ્રિક્વન્સી
\item
  fmax = સિગ્નલમાં મહત્તમ ફ્રિક્વન્સી
\end{itemize}

\textbf{સમજૂતી:}

\begin{itemize}
\tightlist
\item
  એલિયાસિંગ (ફ્રિક્વન્સી ઓવરલેપ) થતું ન હોય તેની ખાતરી કરે છે
\item
  લઘુત્તમ સેમ્પલિંગ રેટને નાઈક્વિસ્ટ રેટ કહેવાય છે
\item
  નાઈક્વિસ્ટ રેટથી નીચે સેમ્પલિંગ અપરિવર્તનીય વિકૃતિ પેદા કરે છે
\item
  વ્યવહારમાં, ફિલ્ટરિંગની મંજૂરી આપવા માટે fs \textgreater{} 2.2fmax વાપરવામાં
  આવે છે
\end{itemize}

\textbf{ઉદાહરણ:}

\begin{itemize}
\tightlist
\item
  fmax = 20 kHz વાળા ઓડિયો માટે
\item
  નાઈક્વિસ્ટ રેટ = 2 \times 20 kHz = 40 kHz
\item
  CD સેમ્પલિંગ રેટ = 44.1 kHz (\textgreater40 kHz)
\end{itemize}

\end{solutionbox}
\begin{mnemonicbox}
``ઉચ્ચતમ ફ્રિક્વન્સી કરતાં ઓછામાં ઓછા બમણા સ્પીડથી સેમ્પલ
કરો''

\end{mnemonicbox}
\subsection*{પ્રશ્ન 4(b) OR [4
ગુણ]}\label{q4b}

\textbf{ડેલ્ટા મોડ્યુલેશન માટે સ્લોપ ઓવરલોડ અને ગ્રેન્યુલર નોઈજ સમજાવો.}

\begin{solutionbox}

\textbf{ડેલ્ટા મોડ્યુલેશન સમસ્યાઓ:}

\begin{center}
\textbf{Mermaid Diagram (Code)}
\begin{verbatim}
{Shaded}
{Highlighting}[]
graph LR
    A[Delta Modulation Problems] {-{-}{} B[Slope Overload]}
    A {-{-}{} C[Granular Noise]}
    B {-{-}{} D[Step size too small]}
    C {-{-}{} E[Step size too large]}
{Highlighting}
{Shaded}
\end{verbatim}
\end{center}

\textbf{સ્લોપ ઓવરલોડ:}

\begin{itemize}
\tightlist
\item
  જ્યારે ઇનપુટ સિગ્નલ DM કરતાં વધુ ઝડપથી બદલાય છે ત્યારે થાય છે
\item
  ઝડપથી બદલાતા સિગ્નલો માટે સ્ટેપ સાઇઝ ખૂબ નાની
\item
  DM આઉટપુટ ઇનપુટને ``પકડી'' શકતું નથી
\item
  તીક્ષ્ણ ટ્રાન્ઝિશન પર વિકૃતિ ઉત્પન્ન કરે છે
\item
  ઉકેલ: સ્ટેપ સાઇઝ અથવા સેમ્પલિંગ રેટ વધારો
\end{itemize}

\textbf{ગ્રેન્યુલર નોઈઝ:}

\begin{itemize}
\tightlist
\item
  સાપેક્ષ રીતે સ્થિર સિગ્નલના ભાગો દરમિયાન થાય છે
\item
  ધીમી ગતિએ બદલાતા સિગ્નલો માટે સ્ટેપ સાઇઝ ખૂબ મોટી
\item
  આઉટપુટ ઇનપુટ વેલ્યુની આસપાસ આંદોલિત થાય છે
\item
  પુનર્નિર્મિત સિગ્નલમાં ``ખરબચડાપણું'' ઉત્પન્ન કરે છે
\item
  ઉકેલ: સ્ટેપ સાઇઝ ઘટાડો
\end{itemize}

\textbf{એડેપ્ટિવ ડેલ્ટા મોડ્યુલેશન (ADM):} બંને સમસ્યાઓને ઓછી કરવા માટે ગતિશીલ રીતે
સ્ટેપ સાઇઝ એડજસ્ટ કરે છે.

\end{solutionbox}
\begin{mnemonicbox}
``ઢાળને મોટા સ્ટેપ, સપાટીને નાના સ્ટેપની જરૂર છે''

\end{mnemonicbox}
\subsection*{પ્રશ્ન 4(c) OR [7
ગુણ]}\label{q4c}

\textbf{પી.સી.એમ. ટ્રાન્સમિટર અને રીસીવરને દોરો અને વિગતવાર સમજાવો.}

\begin{solutionbox}

\textbf{PCM ટ્રાન્સમિટર:}

\begin{center}
\textbf{Mermaid Diagram (Code)}
\begin{verbatim}
{Shaded}
{Highlighting}[]
graph LR
    A[Input Signal] {-{-}{} B[Anti{-}aliasing Filter]}
    B {-{-}{} C[Sample \& Hold]}
    C {-{-}{} D[Quantizer]}
    D {-{-}{} E[Encoder]}
    E {-{-}{} F[Digital Output]}
{Highlighting}
{Shaded}
\end{verbatim}
\end{center}

\textbf{PCM રિસીવર:}

\begin{center}
\textbf{Mermaid Diagram (Code)}
\begin{verbatim}
{Shaded}
{Highlighting}[]
graph LR
    A[Digital Input] {-{-}{} B[Decoder]}
    B {-{-}{} C[D/A Converter]}
    C {-{-}{} D[Reconstruction Filter]}
    D {-{-}{} E[Output Signal]}
{Highlighting}
{Shaded}
\end{verbatim}
\end{center}

\textbf{ટ્રાન્સમિટર ઘટકો:}

\begin{itemize}
\tightlist
\item
  \textbf{એન્ટી-એલિયાસિંગ ફિલ્ટર}: એલિયાસિંગ અટકાવવા માટે ઇનપુટ બેન્ડવિડ્થ
  મર્યાદિત કરે છે
\item
  \textbf{સેમ્પલ એન્ડ હોલ્ડ}: નિયમિત અંતરાલે ક્ષણિક મૂલ્યો પકડે છે
\item
  \textbf{ક્વાન્ટાઇઝર}: સેમ્પલ્સને પૂર્વવ્યાખ્યાયિત ડિસ્ક્રીટ લેવલ્સમાં અનુમાનિત કરે છે
\item
  \textbf{એન્કોડર}: ક્વાન્ટાઇઝ્ડ વેલ્યુને બાઇનરી કોડમાં રૂપાંતરિત કરે છે
\end{itemize}

\textbf{રિસીવર ઘટકો:}

\begin{itemize}
\tightlist
\item
  \textbf{ડિકોડર}: બાઇનરી કોડને ક્વાન્ટાઇઝ્ડ વેલ્યુમાં પાછો રૂપાંતરિત કરે છે
\item
  \textbf{D/A કન્વર્ટર}: ડિસ્ક્રીટ વેલ્યુને સતત વોલ્ટેજમાં રૂપાંતરિત કરે છે
\item
  \textbf{રીકન્સ્ટ્રક્શન ફિલ્ટર}: સેમ્પલિંગ ફ્રિક્વન્સી ઘટકો દૂર કરે છે, આઉટપુટને સુધારે છે
\end{itemize}

\textbf{PCM પેરામીટર્સ:}

\begin{itemize}
\tightlist
\item
  \textbf{રિઝોલ્યુશન}: પ્રતિ સેમ્પલ બિટ્સ (n) દ્વારા નિર્ધારિત
\item
  \textbf{ક્વાન્ટાઇઝેશન લેવલ્સ}: L = 2\^{}n
\item
  \textbf{બિટ રેટ}: R = n \times fs (બિટ્સ પ્રતિ સેકન્ડ)
\item
  \textbf{SNR}: દરેક બિટ ઉમેરતાં \textasciitilde6dB સુધારો થાય છે
\end{itemize}

\end{solutionbox}
\begin{mnemonicbox}
``સેમ્પલ, ક્વાન્ટાઇઝ, એન્કોડ; ડિકોડ, કન્વર્ટ, રીકન્સ્ટ્રક્ટ''

\end{mnemonicbox}
\subsection*{પ્રશ્ન 5(a) [3
ગુણ]}\label{q5a}

\textbf{યોગ્ય ઉદાહરણ સાથે બીટ, બીટનો દર અને બૌડ દરને વ્યાખ્યાયિત કરો.}

\begin{solutionbox}

\begin{itemize}
\tightlist
\item
  \textbf{બિટ}: ડિજિટલ માહિતીનો સૌથી નાનો એકમ, જે 0 અથવા 1 દર્શાવે છે.

  \begin{itemize}
  \tightlist
  \item
    ઉદાહરણ: 10110માં 5 બિટ્સ છે
  \end{itemize}
\item
  \textbf{બિટ રેટ}: પ્રતિ સેકન્ડ ટ્રાન્સમિટ થતા બિટ્સની સંખ્યા.

  \begin{itemize}
  \tightlist
  \item
    એકમ: bps (બિટ્સ પ્રતિ સેકન્ડ)
  \item
    ઉદાહરણ: 9600 bps એટલે એક સેકન્ડમાં 9600 બિટ્સ ટ્રાન્સમિટ થાય છે
  \end{itemize}
\item
  \textbf{બૌડ રેટ}: પ્રતિ સેકન્ડ સિગ્નલ બદલાવની (સિમ્બોલ્સ) સંખ્યા.

  \begin{itemize}
  \tightlist
  \item
    એકમ: Baud
  \item
    ઉદાહરણ: QPSKમાં, દરેક સિમ્બોલ 2 બિટ્સ દર્શાવે છે, તેથી 9600 bps = 4800 Baud
  \end{itemize}
\end{itemize}

\textbf{સંબંધ:}

\begin{itemize}
\tightlist
\item
  બિટ રેટ = બૌડ રેટ \times પ્રતિ સિમ્બોલ બિટ્સની સંખ્યા
\item
  બાઇનરી સિગ્નલિંગ માટે (1 બિટ/સિમ્બોલ): બિટ રેટ = બૌડ રેટ
\item
  મલ્ટિલેવલ કોડિંગ માટે: બિટ રેટ \textgreater{} બૌડ રેટ
\end{itemize}

\end{solutionbox}
\begin{mnemonicbox}
``બિટ્સ ડેટા બનાવે, બૌડ સિમ્બોલ્સ લાવે''

\end{mnemonicbox}
\subsection*{પ્રશ્ન 5(b) [4
ગુણ]}\label{q5b}

\textbf{મલ્ટિપ્લેક્સિંગને વ્યાખ્યાયિત કરો. તેના પ્રકારો જણાવો. યોગ્ય આકૃતિ સાથે
ફ્રીક્વન્શી ડીવીજન મલ્ટિપ્લેક્સિંગ સમજાવો.}

\begin{solutionbox}

\textbf{મલ્ટિપ્લેક્સિંગ}: તકનીક જે મલ્ટિપલ સિગ્નલ્સને સામાન્ય ટ્રાન્સમિશન માધ્યમ શેર
કરવાની મંજૂરી આપે છે.

\textbf{મલ્ટિપ્લેક્સિંગના પ્રકારો:}

\begin{itemize}
\tightlist
\item
  ફ્રિક્વન્સી ડિવિઝન મલ્ટિપ્લેક્સિંગ (FDM)
\item
  ટાઇમ ડિવિઝન મલ્ટિપ્લેક્સિંગ (TDM)
\item
  કોડ ડિવિઝન મલ્ટિપ્લેક્સિંગ (CDM)
\item
  વેવલેંથ ડિવિઝન મલ્ટિપ્લેક્સિંગ (WDM)
\end{itemize}

\textbf{ફ્રિક્વન્સી ડિવિઝન મલ્ટિપ્લેક્સિંગ:}

\begin{verbatim}
    Frequency
        \^{}
        |
        |  ┌───┐  ┌───┐  ┌───┐  ┌───┐
        |  │Ch1│  │Ch2│  │Ch3│  │Ch4│
        |  │   │  │   │  │   │  │   │
    0   |{-{-}+{-}{-}{-}+{-}{-}+{-}{-}{-}+{-}{-}+{-}{-}{-}+{-}{-}+{-}{-}{-}+{-}{-}{-} Frequency}
        |  f1     f2     f3     f4
        |
        |  Guard Bands between channels
\end{verbatim}

\textbf{FDM કાર્યપદ્ધતિ:}

\begin{itemize}
\tightlist
\item
  દરેક સિગ્નલ અલગ કેરિયર ફ્રિક્વન્સી પર મોડ્યુલેટ થાય છે
\item
  ગાર્ડ બેન્ડ્સ સાથે દરેક ચેનલને બેન્ડવિડ્થ ફાળવવામાં આવે છે
\item
  બધા ચેનલો એક સાથે ટ્રાન્સમિટ થાય છે
\item
  રિસીવર ચેનલોને અલગ કરવા માટે ફિલ્ટર્સનો ઉપયોગ કરે છે
\item
  રેડિયો/TV બ્રોડકાસ્ટિંગ, કેબલ સિસ્ટમમાં વપરાય છે
\end{itemize}

\end{solutionbox}
\begin{mnemonicbox}
``ફ્રિક્વન્સી મલ્ટિપલ સિગ્નલને એક સાથે વિભાજિત કરે છે''

\end{mnemonicbox}
\subsection*{પ્રશ્ન 5(c) [7
ગુણ]}\label{q5c}

\textbf{આકૃતિ સાથે મૂળભૂત PCM-TDM આકૃતિ દોરો અને સમજાવો.}

\begin{solutionbox}

\textbf{PCM-TDM સિસ્ટમ બ્લોક ડાયાગ્રામ:}

\begin{center}
\textbf{Mermaid Diagram (Code)}
\begin{verbatim}
{Shaded}
{Highlighting}[]
graph LR
    \%\% Transmitter
    A1[Source 1] {-{-}{} B1[LPF 1]}
    A2[Source 2] {-{-}{} B2[LPF 2]}
    A3[Source 3] {-{-}{} B3[LPF 3]}
    B1 {-{-}{} C[Commutator/MUX]}
    B2 {-{-}{} C}
    B3 {-{-}{} C}
    C {-{-}{} D[Sampler]}
    D {-{-}{} E[Quantizer]}
    E {-{-}{} F[Encoder]}
    F {-{-}{} G[TDM Output]}

    \%\% Receiver
    G {-{-}{} H[Decoder]}
    H {-{-}{} I[DEMUX]}
    I {-{-}{} J1[LPF 1]}
    I {-{-}{} J2[LPF 2]}
    I {-{-}{} J3[LPF 3]}
    J1 {-{-}{} K1[Output 1]}
    J2 {-{-}{} K2[Output 2]}
    J3 {-{-}{} K3[Output 3]}
{Highlighting}
{Shaded}
\end{verbatim}
\end{center}

\textbf{PCM-TDM સિસ્ટમ ઓપરેશન:}

\textbf{ટ્રાન્સમિટર સાઇડ:}

\begin{itemize}
\tightlist
\item
  \textbf{ઇનપુટ સોર્સ}: મલ્ટિપલ એનાલોગ સિગ્નલ્સ
\item
  \textbf{લો-પાસ ફિલ્ટર્સ}: ઇનપુટ સિગ્નલ્સની બેન્ડવિડ્થ મર્યાદિત કરે છે
\item
  \textbf{કમ્યુટેટર/MUX}: અનુક્રમે દરેક ઇનપુટને સેમ્પલ કરે છે
\item
  \textbf{સેમ્પલર}: સતત સિગ્નલ્સને ડિસ્ક્રીટ સેમ્પલ્સમાં રૂપાંતરિત કરે છે
\item
  \textbf{ક્વાન્ટાઇઝર}: સેમ્પલ્સને નજીકના ડિસ્ક્રીટ લેવલ્સમાં અનુમાનિત કરે છે
\item
  \textbf{એન્કોડર}: ક્વાન્ટાઇઝ્ડ વેલ્યુને બાઇનરી કોડમાં રૂપાંતરિત કરે છે
\item
  \textbf{TDM આઉટપુટ}: બધા ચેનલ્સમાંથી સેમ્પલ્સ ધરાવતા ફ્રેમ્સ ટ્રાન્સમિટ કરે છે
\end{itemize}

\textbf{રિસીવર સાઇડ:}

\begin{itemize}
\tightlist
\item
  \textbf{ડિકોડર}: બાઇનરી કોડને ક્વાન્ટાઇઝ્ડ વેલ્યુમાં પાછો રૂપાંતરિત કરે છે
\item
  \textbf{DEMUX}: સેમ્પલ્સને યોગ્ય ચેનલ પાથમાં વિતરિત કરે છે
\item
  \textbf{લો-પાસ ફિલ્ટર્સ}: મૂળ સિગ્નલ્સનું પુનર્નિર્માણ કરે છે, સેમ્પલિંગ ઘટકો દૂર કરે છે
\item
  \textbf{આઉટપુટ્સ}: પુનઃપ્રાપ્ત મૂળ સિગ્નલ્સ
\end{itemize}

\textbf{TDM ફ્રેમ ફોર્મેટ:}

\begin{verbatim}
    ┌──────┬──────┬──────┬──────┬──────┬──────┐
    │ Sync │ Ch 1 │ Ch 2 │ Ch 3 │ Ch 1 │ Ch 2 │...
    └──────┴──────┴──────┴──────┴──────┴──────┘
      Frame header    Channel samples repeat
\end{verbatim}

\end{solutionbox}
\begin{mnemonicbox}
``PCM-TDM: સેમ્પલ, ક્વાન્ટાઇઝ, એન્કોડ, મલ્ટિપ્લેક્સ''

\end{mnemonicbox}
\subsection*{પ્રશ્ન 5(a) OR [3
ગુણ]}\label{q5a}

\textbf{ટીડીએમના પ્રકારો જણાવો અને તેમાંથી કોઈપણ એકને સમજાવો.}

\begin{solutionbox}

\textbf{TDMના પ્રકારો:}

\begin{itemize}
\tightlist
\item
  સિંક્રોનસ TDM
\item
  એસિંક્રોનસ TDM (સ્ટેટિસ્ટિકલ TDM)
\item
  ઇન્ટેલિજન્ટ TDM
\end{itemize}

\textbf{સિંક્રોનસ TDM:}

\begin{itemize}
\tightlist
\item
  દરેક ચેનલ માટે ફિક્સ્ડ ટાઇમ સ્લોટ્સ ફાળવવામાં આવે છે
\item
  ટાઇમ સ્લોટ્સ ફિક્સ્ડ સિક્વન્સમાં ટ્રાન્સમિટ થાય છે
\item
  ચેનલમાં કોઈ ડેટા ન હોય તો પણ ટાઇમ સ્લોટ્સ ખાલી રહે છે
\item
  સરળ અમલીકરણ પરંતુ ઓછી કાર્યક્ષમતા
\item
  ઉદાહરણ: T1 કેરિયર સિસ્ટમ (24 ચેનલ્સ \times 8 બિટ્સ \times 8000 સેમ્પલ્સ/સેક = 1.544 Mbps)
\end{itemize}

\textbf{ફ્રેમ સ્ટ્રક્ચર:}

\begin{verbatim}
    ┌──────┬──────┬──────┬──────┬──────┐
    │ Sync │ Ch 1 │ Ch 2 │ Ch 3 │ Ch 4 │
    └──────┴──────┴──────┴──────┴──────┘
      એક્ટિવિટીથી સ્વતંત્ર ફિક્સ્ડ સ્લોટ્સ
\end{verbatim}

\end{solutionbox}
\begin{mnemonicbox}
``સિંક્રોનસ સ્લોટ્સ સ્થિર રહે છે''

\end{mnemonicbox}
\subsection*{પ્રશ્ન 5(b) OR [4
ગુણ]}\label{q5b}

\textbf{ટીડીએમ (TDM) ને સમજાવો. તેના ફાયદા અને ગેરફાયદા પણ જણાવો.}

\begin{solutionbox}

\textbf{ટાઇમ ડિવિઝન મલ્ટિપ્લેક્સિંગ (TDM):} તકનીક જ્યાં મલ્ટિપલ સિગ્નલ્સ દરેક
સિગ્નલને અલગ ટાઇમ સ્લોટ ફાળવીને સમાન ટ્રાન્સમિશન માધ્યમ શેર કરે છે.

\textbf{કાર્યપદ્ધતિ:}

\begin{itemize}
\tightlist
\item
  દરેક સિગ્નલ નિયમિત અંતરાલે સેમ્પલ કરવામાં આવે છે
\item
  સેમ્પલ્સ ટાઇમ ડોમેઇનમાં ઇન્ટરલિવ્ડ હોય છે
\item
  સંપૂર્ણ ફ્રેમ દરેક ચેનલમાંથી એક સેમ્પલ ધરાવે છે
\item
  રિસીવર સેમ્પલ્સને અલગ કરીને મૂળ સિગ્નલ્સનું પુનર્નિર્માણ કરે છે
\end{itemize}

\textbf{TDMના ફાયદા:}

\begin{itemize}
\tightlist
\item
  \textbf{સિંગલ મીડિયમ}: એક ટ્રાન્સમિશન પાથનો કાર્યક્ષમ ઉપયોગ
\item
  \textbf{ડિજિટલ સંગતતા}: કુદરતી રીતે ડિજિટલ સિસ્ટમ્સને અનુરૂપ
\item
  \textbf{ક્રોસટોક નાબૂદી}: ચેનલો વચ્ચે કોઈ ઇન્ટરફેરન્સ નથી
\item
  \textbf{લવચીક ક્ષમતા}: ચેનલ્સ સરળતાથી ઉમેરી/દૂર કરી શકાય છે
\item
  \textbf{કિફાયતી}: હાર્ડવેર જરૂરિયાતો ઘટાડે છે
\end{itemize}

\textbf{TDMના ગેરફાયદા:}

\begin{itemize}
\tightlist
\item
  \textbf{સિંક્રોનાઇઝેશન મહત્વપૂર્ણ}: ટાઇમિંગ ભૂલો મોટી સમસ્યાઓ ઉભી કરે છે
\item
  \textbf{જટિલ ઇક્વિપમેન્ટ}: ચોક્કસ ટાઇમિંગ સર્કિટની જરૂર પડે છે
\item
  \textbf{બેન્ડવિડ્થ મર્યાદા}: ઘણા ચેનલ્સ માટે ઉચ્ચ બિટ રેટની જરૂર પડે છે
\item
  \textbf{અકાર્યક્ષમતા}: ચેનલ્સ નિષ્ક્રિય હોય ત્યારે ક્ષમતા બરબાદ કરે છે (સિંક્રોનસ
  TDMમાં)
\item
  \textbf{બફર વિલંબ}: લેટન્સી સમસ્યાઓ ઉભી કરી શકે છે
\end{itemize}

\end{solutionbox}
\begin{mnemonicbox}
``સમય વિભાજિત મલ્ટિપલ સિગ્નલ્સ ખર્ચ બચાવે પણ ચોક્કસ
ટાઇમિંગની જરૂર પડે''

\end{mnemonicbox}
\subsection*{પ્રશ્ન 5(c) OR [7
ગુણ]}\label{q5c}

\textbf{લાઇન કોડિંગના ઇચ્છનીય ગુણધર્મો જણાવો. 8 બીટ ડિજીટલ ડેટા 01001110 માટે
એકધ્રુવીય RZ, Polar NRZ, અને માન્ચેસ્ટર લાઇન કોડિંગ માટે સમય સંબંધમાં વેવફોર્મ દોરો.}

\begin{solutionbox}

\textbf{લાઇન કોડિંગના ઇચ્છનીય ગુણધર્મો:}

\begin{itemize}
\tightlist
\item
  \textbf{DC ઘટક}: ન્યૂનતમ અથવા ગેરહાજર હોવો જોઈએ
\item
  \textbf{સેલ્ફ-સિંક્રોનાઇઝેશન}: ટાઇમિંગ માહિતી પ્રદાન કરવી જોઈએ
\item
  \textbf{એરર ડિટેક્શન}: ટ્રાન્સમિશન ભૂલોનું શોધન કરવાની મંજૂરી આપવી જોઈએ
\item
  \textbf{બેન્ડવિડ્થ કાર્યક્ષમતા}: ન્યૂનતમ બેન્ડવિડ્થની જરૂર પડવી જોઈએ
\item
  \textbf{ઘોંઘાટ ઇમ્યુનિટી}: ઘોંઘાટ અને ઇન્ટરફેરન્સ સામે પ્રતિરોધક હોવી જોઈએ
\item
  \textbf{ખર્ચ અને જટિલતા}: અમલીકરણ સરળ હોવું જોઈએ
\end{itemize}

\textbf{01001110 માટે લાઇન કોડિંગ વેવફોર્મ્સ:}

\begin{verbatim}
    Bit pattern:  0  1  0  0  1  1  1  0
    
    Unipolar RZ:
        \^{}
        |
    A   |    ┌─┐     ┌─┐ ┌─┐ ┌─┐
        |    │ │     │ │ │ │ │ │
    0   |────┘ └─────┘ └─┘ └─┘ └───{ t}
        
    Polar NRZ:
        \^{}
        |
    +A  |    ┌─────┐     ┌───────┐
        |    │     │     │       │
    0   |────┘     └─────┘       └───{ t}
        |                            
    {-A  |─┐         ┌─────┐           }
        | └─────────┘     │           
        
    Manchester:
        \^{}
        |                            
    +A  |─┐   ┌─┐ ┌─┐   ┌─┐   ┌─┐ ┌─┐
        | │   │ │ │ │   │ │   │ │ │ │
    0   |─┘   └─┘ └─┘   └─┘   └─┘ └─{ t}
        |                            
    {-A  |  ┌─┐       ┌─┐       ┌─┐   }
        |  │ │       │ │       │ │   
        |  └─┘       └─┘       └─┘   
        
    Legend: 0 = Low, 1 = High
\end{verbatim}

\textbf{મુખ્ય લક્ષણો:}

\begin{itemize}
\tightlist
\item
  \textbf{યુનિપોલર RZ}: બિટની મધ્યમાં શૂન્ય પર પાછું ફરે છે, માત્ર હકારાત્મક વોલ્ટેજ
\item
  \textbf{પોલર NRZ}: શૂન્ય પર પાછા ફરતું નથી, હકારાત્મક અને નકારાત્મક વોલ્ટેજનો
  ઉપયોગ કરે છે
\item
  \textbf{માન્ચેસ્ટર}: મિડ-બિટ ટ્રાન્ઝિશન, ચઢતા ધાર = 0, ઉતરતા ધાર = 1
\end{itemize}

\end{solutionbox}
\begin{mnemonicbox}
``યુનિપોલર ઊંચે ચઢે પછી શૂન્ય, પોલર કદી પાછું ન આવે, માન્ચેસ્ટર
હંમેશા બદલાય''

\end{mnemonicbox}

\end{document}
