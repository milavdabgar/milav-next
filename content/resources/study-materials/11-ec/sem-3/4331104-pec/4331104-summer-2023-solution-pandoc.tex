\documentclass[10pt,a4paper]{article}

% content/resources/templates/preamble.tex
\usepackage[margin=0.6in]{geometry}
\author{Milav Dabgar}
\usepackage{amsmath,amssymb,amsthm}
\usepackage{booktabs}
\usepackage{multirow}
\usepackage{xcolor}
\usepackage{tcolorbox}
\tcbuselibrary{breakable,skins}
\usepackage[colorlinks=true,linkcolor=blue]{hyperref}
\usepackage{titlesec}
\usepackage{enumitem}
\usepackage{tikz}
\usepackage{pgfplots}
\usepackage{circuitikz}
\usepackage[version=4]{mhchem}
\usepackage{longtable}
\usepackage{array}
\usepackage{float}
\usepackage{caption}
\usepackage{listings}

\lstset{
  basicstyle=\small\ttfamily,
  breaklines=true,
  breakatwhitespace=false,
  postbreak=\mbox{\textcolor{red}{$\hookrightarrow$}\space},
  float=false,
  numbers=left,
  numberstyle=\tiny\color{gray},
  numbersep=10pt,
  xleftmargin=2em,
  keywordstyle=\color{blue},
  commentstyle=\color{green!60!black},
  stringstyle=\color{purple},
  backgroundcolor=\color{gray!5},
  showstringspaces=false,
  tabsize=2,
  captionpos=b,
  keepspaces=true,
  columns=flexible
}

\pgfplotsset{compat=1.18}
\usetikzlibrary{shapes,arrows,positioning,calc,patterns,decorations.pathmorphing,decorations.markings,arrows.meta}

% Color scheme
\definecolor{headcolor}{RGB}{0,102,204}
\definecolor{keycolor}{RGB}{220,20,60}
\definecolor{solutioncolor}{RGB}{34,139,34}
\definecolor{mnemoniccolor}{RGB}{148,0,211}
\definecolor{codecolor}{RGB}{0,0,100}

% Spacing
\setlength{\parskip}{3pt}
\setlist[itemize]{nosep}
\setlist[enumerate]{nosep}

% Title formatting
\titleformat{\section}{\Large\bfseries\color{headcolor}}{\thesection}{1em}{}
\titleformat{\subsection}{\large\bfseries\color{headcolor}}{\thesubsection}{1em}{}

% Pandoc tightlist compatibility
\providecommand{\tightlist}{%
  \setlength{\itemsep}{0pt}\setlength{\parskip}{0pt}}

% Pandoc longtable compatibility
\newcounter{none}
\def\thenone{}


% content/resources/templates/english-boxes.tex
% This file is currently empty - it exists to maintain consistency with the import structure.
% Add custom environments here if needed in the future.


\begin{document}

\begin{center}
{\Huge\bfseries\color{headcolor} Subject Name Solutions}\\[5pt]
{\LARGE 4331104 -- Summer 2023}\\[3pt]
{\large Semester 1 Study Material}\\[3pt]
{\normalsize\textit{Detailed Solutions and Explanations}}
\end{center}

\vspace{10pt}

\subsection*{Question 1(a) [3 marks]}\label{q1a}

\textbf{Draw \& explain block diagram of Communication system.}

\begin{solutionbox}

\begin{center}
\textbf{Mermaid Diagram (Code)}
\begin{verbatim}
{Shaded}
{Highlighting}[]
graph LR
    A[Input] {-{-}{} B[Transmitter]}
    B {-{-}{} C[Channel]}
    C {-{-}{} D[Receiver]}
    D {-{-}{} E[Output]}
    F[Noise Source] {-{-}{} C}
{Highlighting}
{Shaded}
\end{verbatim}
\end{center}

\begin{itemize}
\tightlist
\item
  \textbf{Input}: Message signal originating from source
\item
  \textbf{Transmitter}: Converts message to suitable form for
  transmission
\item
  \textbf{Channel}: Medium through which signal travels
\item
  \textbf{Receiver}: Extracts original message from received signal
\item
  \textbf{Output}: Delivered message to destination
\item
  \textbf{Noise Source}: Unwanted signals that interfere with
  communication
\end{itemize}

\end{solutionbox}
\begin{mnemonicbox}
``I Transmit Clearly Receiving Original Messages''

\end{mnemonicbox}
\subsection*{Question 1(b) [4 marks]}\label{q1b}

\textbf{Explain need of modulation. State advantages of modulation.}

\begin{solutionbox}

\textbf{Need for modulation:}

\begin{center}
\textbf{Mermaid Diagram (Code)}
\begin{verbatim}
{Shaded}
{Highlighting}[]
graph TD
    A[Practical Antenna Size] {-{-}{} B[Modulation]}
    C[Multiplexing] {-{-}{} B}
    D[Reducing Noise \& Interference] {-{-}{} B}
    E[Signal Transmission Distance] {-{-}{} B}
{Highlighting}
{Shaded}
\end{verbatim}
\end{center}

\textbf{Advantages of modulation:}

\begin{itemize}
\tightlist
\item
  \textbf{Reduced antenna size}: Practical antenna length = λ/4, higher
  frequency means smaller antenna
\item
  \textbf{Multiplexing possible}: Multiple signals transmitted
  simultaneously through same channel
\item
  \textbf{Increased range}: Modulated signals travel farther than
  baseband signals
\item
  \textbf{Noise reduction}: Better SNR achieved through modulation
  techniques
\end{itemize}

\end{solutionbox}
\begin{mnemonicbox}
``Antennas Need Modulation For Reaching Anywhere with
Noise Immunity''

\end{mnemonicbox}
\subsection*{Question 1(c) [7 marks]}\label{q1c}

\textbf{Define modulation. Explain Amplitude modulation with waveform
and derive voltage equation for modulated signal.}

\begin{solutionbox}

\textbf{Modulation}: Process of varying a carrier signal's parameter
(amplitude, frequency, phase) proportionally to the message signal.

\textbf{Amplitude Modulation Waveform:}

\begin{verbatim}
                                         AM Waveform
    │        Carrier            │        Message           │     Modulated Signal
    │                           │                          │
    │  ╱╲    ╱╲    ╱╲    ╱╲    │                          │     ╱╲      ╱╲  
    │ ╱  ╲  ╱  ╲  ╱  ╲  ╱  ╲   │         ╱╲               │    ╱  ╲    ╱  ╲ 
    │╱    ╲╱    ╲╱    ╲╱    ╲  │        ╱  ╲              │   ╱    ╲  ╱    ╲
{-{-}{-}{-}┼{-}{-}{-}{-}{-}{-}{-}{-}{-}{-}{-}{-}{-}{-}{-}{-}{-}{-}{-}{-}{-}     │{-}{-}{-}{-}{-}{-}{-}╱{-}{-}{-}{-}╲{-}{-}{-}{-}{-}{-}{-}{-}{-}    │{-}{-}╱{-}{-}{-}{-}{-}{-}╲╱{-}{-}{-}{-}{-}{-}╲{-}{-}{-}{-}{-}}
    │                          │      ╱      ╲            │ ╱                 ╲
    │                          │     ╱        ╲           │╱                   ╲
    │                          │    ╱          ╲          │                     ╲
\end{verbatim}

\textbf{Derivation of AM voltage equation:}

\begin{enumerate}
\tightlist
\item
  Carrier signal: vc(t) = Vc sin(ωct)
\item
  Message signal: vm(t) = Vm sin(ωmt)
\item
  Modulated signal: vAM(t) = [Vc + Vm sin(ωmt)] sin(ωct)
\item
  Modulation index: μ = Vm/Vc
\item
  Final AM equation: vAM(t) = Vc[1 + μ sin(ωmt)] sin(ωct)
\end{enumerate}

\end{solutionbox}
\begin{mnemonicbox}
``Amplitude Modulation Makes Carrier Value Change''

\end{mnemonicbox}
\subsection*{Question 1(c) OR [7
marks]}\label{q1c}

\textbf{Define noise. Give classification of noise and explain cause of
any three internal noise.}

\begin{solutionbox}

\textbf{Noise}: Unwanted signals that interfere with communication
signals, causing distortion or errors.

\textbf{Classification of Noise:}

{\def\LTcaptype{none} % do not increment counter
\begin{longtable}[]{@{}ll@{}}
\toprule\noalign{}
External Noise & Internal Noise \\
\midrule\noalign{}
\endhead
\bottomrule\noalign{}
\endlastfoot
Atmospheric & Thermal \\
Extraterrestrial & Shot \\
Industrial & Transit-time \\
& Flicker \\
& Partition \\
\end{longtable}
}

\textbf{Causes of internal noise:}

\begin{itemize}
\tightlist
\item
  \textbf{Thermal noise}:

  \begin{itemize}
  \tightlist
  \item
    Caused by random motion of electrons in conductors
  \item
    Present in all electronic components
  \item
    Directly proportional to temperature and bandwidth
  \end{itemize}
\item
  \textbf{Shot noise}:

  \begin{itemize}
  \tightlist
  \item
    Occurs due to random arrival of carriers at junctions
  \item
    Found in active devices like diodes and transistors
  \item
    Proportional to DC current flowing through device
  \end{itemize}
\item
  \textbf{Flicker noise}:

  \begin{itemize}
  \tightlist
  \item
    Results from surface defects and impurities in semiconductors
  \item
    Inversely proportional to frequency (1/f noise)
  \item
    Significant at low frequencies
  \end{itemize}
\end{itemize}

\end{solutionbox}
\begin{mnemonicbox}
``This Shooting Flicker Is Noisy Everywhere''

\end{mnemonicbox}
\subsection*{Question 2(a) [3 marks]}\label{q2a}

\textbf{Define (1) Modulation index for AM (2) Noise Figure (3) Digital
Modulation}

\begin{solutionbox}

\begin{enumerate}
\tightlist
\item
  \textbf{Modulation index for AM}: Ratio of amplitude of modulating
  signal to amplitude of carrier signal.

  \begin{itemize}
  \tightlist
  \item
    μ = Vm/Vc
  \item
    Must be 0 \leq μ \leq 1 to avoid distortion
  \end{itemize}
\item
  \textbf{Noise Figure}: Ratio of input SNR to output SNR of a device.

  \begin{itemize}
  \tightlist
  \item
    NF = (SNR)input/(SNR)output
  \item
    Indicates noise added by the system
  \item
    Always \geq 1, expressed in dB
  \end{itemize}
\item
  \textbf{Digital Modulation}: Technique that represents digital data as
  variations in carrier signal parameters.

  \begin{itemize}
  \tightlist
  \item
    Examples: ASK, FSK, PSK, QAM
  \item
    Used for digital data transmission
  \end{itemize}
\end{enumerate}

\end{solutionbox}
\begin{mnemonicbox}
``Modulation Measures, Noise Numbers, Digital Data''

\end{mnemonicbox}
\subsection*{Question 2(b) [4 marks]}\label{q2b}

\textbf{Derive equation for total power transmitted for amplitude
modulated signal considering carrier power and modulation index.}

\begin{solutionbox}

\textbf{Derivation of total power in AM:}

\begin{enumerate}
\item
  AM wave equation: vAM(t) = Vc[1 + μ sin(ωmt)] sin(ωct)
\item
  For power calculation, consider RMS values:

  \begin{itemize}
  \tightlist
  \item
    Carrier power (Pc) = Vc^{2}/2R
  \item
    Power in each sideband (PSB) = (μ^{2}Vc^{2})/(4R)
  \end{itemize}
\item
  Total power equation:

  \begin{itemize}
  \tightlist
  \item
    PT = Pc + PUSB + PLSB
  \item
    PT = Pc + 2PSB (since upper and lower sidebands have equal power)
  \item
    PT = Vc^{2}/2R + 2(μ^{2}Vc^{2})/(4R)
  \item
    PT = (Vc^{2}/2R)[1 + (μ^{2}/2)]
  \end{itemize}
\item
  Final equation: PT = Pc(1 + μ^{2}/2)
\end{enumerate}

\end{solutionbox}
\begin{mnemonicbox}
``Power Total = Power Carrier (1 + μ^{2}/2)''

\end{mnemonicbox}
\subsection*{Question 2(c) [7 marks]}\label{q2c}

\textbf{Explain basic principle of double sideband suppressed carrier
amplitude modulation. Derive its voltage equation \& draw only balanced
modulator circuit using diode.}

\begin{solutionbox}

\textbf{Double Sideband Suppressed Carrier (DSBSC) Principle:}

\begin{itemize}
\tightlist
\item
  Carrier is suppressed, only sidebands transmitted
\item
  Contains all information in sidebands
\item
  More power efficient than AM
\item
  Requires complex receiver for demodulation
\end{itemize}

\textbf{Voltage equation derivation:}

\begin{enumerate}
\tightlist
\item
  AM signal: vAM(t) = Vc[1 + μ sin(ωmt)]sin(ωct)
\item
  Removing carrier component: vDSBSC(t) = Vc \times μ sin(ωmt)sin(ωct)
\item
  Using trigonometric identity: sin(A)sin(B) = 0.5[cos(A-B) -
  cos(A+B)]
\item
  Final equation: vDSBSC(t) = (Vcμ/2)[cos(ωc-ωm)t - cos(ωc+ωm)t]
\end{enumerate}

\textbf{Balanced Modulator Circuit using Diodes:}

\begin{verbatim}
           D1
        |{-{-}{-}|{-}{-}+}
        |       |
     +{-{-}+       +{-}{-}+}
     |             |
Vc{-{-}{-}|             |{-}{-}{-}Output}
     |             |
     +{-{-}+       +{-}{-}+}
        |       |
        |{-{-}{-}|{-}{-}+}
           D2
           |
           |
         Carrier
           |
           V
       Modulating
         Signal
\end{verbatim}

\end{solutionbox}
\begin{mnemonicbox}
``Delete Carrier, Save Bandwidth, Combine Signals''

\end{mnemonicbox}
\subsection*{Question 2(a) OR [3
marks]}\label{q2a}

\textbf{Define only, w.r.t. radio receiver (1) Sensitivity (2)
Selectivity (3) fidelity}

\begin{solutionbox}

\begin{enumerate}
\tightlist
\item
  \textbf{Sensitivity}: Ability of a receiver to detect and amplify weak
  signals.

  \begin{itemize}
  \tightlist
  \item
    Measured in microvolts (μV)
  \item
    Lower value indicates better sensitivity
  \item
    Typically 1-10 μV for commercial receivers
  \end{itemize}
\item
  \textbf{Selectivity}: Ability to distinguish between desired signal
  and adjacent interfering signals.

  \begin{itemize}
  \tightlist
  \item
    Measured as bandwidth at -3dB points
  \item
    Narrower bandwidth means better selectivity
  \item
    Prevents adjacent channel interference
  \end{itemize}
\item
  \textbf{Fidelity}: Accuracy with which receiver reproduces original
  message.

  \begin{itemize}
  \tightlist
  \item
    Measures quality of reproduction
  \item
    Affected by distortion and noise
  \item
    Higher fidelity means better sound quality
  \end{itemize}
\end{enumerate}

\end{solutionbox}
\begin{mnemonicbox}
``Sensitive Selection Faithfully''

\end{mnemonicbox}
\subsection*{Question 2(b) OR [4
marks]}\label{q2b}

\textbf{An AM signal has carrier power of 1 KW with 200 watt in each
sideband. Find out modulation index.}

\begin{solutionbox}

\textbf{Given:}

\begin{itemize}
\tightlist
\item
  Carrier power (Pc) = 1 KW = 1000 W
\item
  Power in each sideband (PSB) = 200 W
\end{itemize}

\textbf{To find:} Modulation index (μ)

\textbf{Solution:}

\begin{enumerate}
\tightlist
\item
  Total sideband power: PTSB = 2 \times PSB = 2 \times 200 = 400 W
\item
  Using formula: PTSB = Pc \times μ^{2}/2
\item
  400 = 1000 \times μ^{2}/2
\item
  μ^{2} = (400 \times 2)/1000 = 800/1000 = 0.8
\item
  μ = \sqrt0.8 = 0.894 = 0.9 (approx)
\end{enumerate}

\end{solutionbox}
\begin{mnemonicbox}
``Sideband Power Reveals Modulation μ''

\end{mnemonicbox}
\subsection*{Question 2(c) OR [7
marks]}\label{q2c}

\textbf{Compare Amplitude modulation with Frequency Modulation
considering minimum seven parameters/aspect.}

\begin{solutionbox}

{\def\LTcaptype{none} % do not increment counter
\begin{longtable}[]{@{}
  >{\raggedright\arraybackslash}p{(\linewidth - 4\tabcolsep) * \real{0.1692}}
  >{\raggedright\arraybackslash}p{(\linewidth - 4\tabcolsep) * \real{0.4154}}
  >{\raggedright\arraybackslash}p{(\linewidth - 4\tabcolsep) * \real{0.4154}}@{}}
\toprule\noalign{}
\begin{minipage}[b]{\linewidth}\raggedright
Parameter
\end{minipage} & \begin{minipage}[b]{\linewidth}\raggedright
Amplitude Modulation (AM)
\end{minipage} & \begin{minipage}[b]{\linewidth}\raggedright
Frequency Modulation (FM)
\end{minipage} \\
\midrule\noalign{}
\endhead
\bottomrule\noalign{}
\endlastfoot
\textbf{Definition} & Amplitude of carrier varies with message &
Frequency of carrier varies with message \\
\textbf{Bandwidth} & Narrow (2 \times fm) & Wide (2 \times β \times fm) \\
\textbf{Power Efficiency} & Poor (carrier contains \textasciitilde66\%
power) & Good (all power in sidebands) \\
\textbf{Noise Immunity} & Poor (noise affects amplitude) & Excellent
(amplitude limiters remove noise) \\
\textbf{Circuit Complexity} & Simple transmitters and receivers &
Complex transmitters and receivers \\
\textbf{Quality} & Lower fidelity & Higher fidelity \\
\textbf{Applications} & Broadcasting, aircraft communication & FM radio,
TV sound, wireless mics \\
\textbf{Spectrum} & Contains carrier and two sidebands & Contains
infinite sidebands \\
\end{longtable}
}

\end{solutionbox}
\begin{mnemonicbox}
``Bandwidth, Efficiency, Noise, Quality - AM Fails
Many Quality Tests''

\end{mnemonicbox}
\subsection*{Question 3(a) [3 marks]}\label{q3a}

\textbf{Draw and label sine wave of 1 KHZ in time domain and frequency
domain. State advantage of frequency domain analysis of signal.}

\begin{solutionbox}

\textbf{Time Domain Representation:}

\begin{verbatim}
    Amplitude
        \^{}
        |
    1   |    /{      /      /      /}
        |   /  {    /      /      /  }
        |  /    {  /      /      /    }
    0   |{-+{-}{-}{-}{-}{-}{-}+{-}{-}{-}{-}{-}{-}{-}+{-}{-}{-}{-}{-}{-}{-}+{-}{-}{-}{-}{-}{-}{-}+{-}{-}{-}{-}{-}{-}{-} Time}
        |  {    /      /      /      /}
        |   {  /      /      /      /}
   {-1   |    /      /      /      /}
        |
     1KHz sine wave (Period = 1ms)
\end{verbatim}

\textbf{Frequency Domain Representation:}

\begin{verbatim}
    Amplitude
        \^{}
        |
    1   |    |
        |    |
        |    |
    0   |{-{-}{-}{-}+{-}{-}{-}{-}+{-}{-}{-}{-}+{-}{-}{-}{-}+{-}{-}{-}{-}+{-}{-}{-}{-}+{-}{-}{-}{-}{-}{-}{-} Frequency}
        |    0   1KHz           
        |
     Single spectral line at 1KHz
\end{verbatim}

\textbf{Advantages of frequency domain analysis:}

\begin{itemize}
\tightlist
\item
  \textbf{Signal composition}: Easily identifies frequency components
\item
  \textbf{Filter design}: Simplified filter response analysis
\item
  \textbf{Bandwidth determination}: Direct visualization of spectrum
  width
\item
  \textbf{Noise analysis}: Better separation of signal from noise
\end{itemize}

\end{solutionbox}
\begin{mnemonicbox}
``Frequency Shows Components Hidden in Time''

\end{mnemonicbox}
\subsection*{Question 3(b) [4 marks]}\label{q3b}

\textbf{State following frequency (1) IF frequency for AM radio (2) IF
frequency for FM radio (3) Frequency Band used in FM radio (4) Frequency
Band of Human speech.}

\begin{solutionbox}

{\def\LTcaptype{none} % do not increment counter
\begin{longtable}[]{@{}ll@{}}
\toprule\noalign{}
Parameter & Frequency \\
\midrule\noalign{}
\endhead
\bottomrule\noalign{}
\endlastfoot
IF frequency for AM radio & 455 kHz \\
IF frequency for FM radio & 10.7 MHz \\
Frequency Band used in FM radio & 88-108 MHz \\
Frequency Band of Human speech & 300 Hz - 3.4 kHz \\
\end{longtable}
}

\end{solutionbox}
\begin{mnemonicbox}
``AM455, FM10.7, Band88-108, Speech300-3.4''

\end{mnemonicbox}
\subsection*{Question 3(c) [7 marks]}\label{q3c}

\textbf{Explain Single side band (SSB) modulation with waveform and its
advantages. Show how SSB transmission required only 1/6th of power with
respect to double sideband full carrier amplitude modulation.}

\begin{solutionbox}

\textbf{Single Side Band (SSB) Modulation:}

\begin{itemize}
\tightlist
\item
  Transmits only one sideband (USB or LSB)
\item
  Carrier and other sideband suppressed
\item
  Conserves bandwidth and power
\end{itemize}

\textbf{SSB Waveform:}

\begin{verbatim}
    Frequency Spectrum
        \^{}
        |
        |                   Regular AM
        |    |              |     |
        |    |              |     |
        |{-{-}{-}{-}+{-}{-}{-}{-}+{-}{-}{-}{-}+{-}{-}{-}{-}+{-}{-}{-}{-}{-}+{-}{-}{-}{-}{-} Frequency}
             fc{-fm   fc   fc+fm}

        |                   SSB (USB)
        |                  |
        |                  |
        |{-{-}{-}{-}+{-}{-}{-}{-}+{-}{-}{-}{-}+{-}{-}{-}{-}+{-}{-}{-}{-}+{-}{-}{-}{-}{-} Frequency}
                         fc+fm
\end{verbatim}

\textbf{Advantages of SSB:}

\begin{itemize}
\tightlist
\item
  \textbf{Bandwidth efficiency}: Uses half bandwidth of AM
\item
  \textbf{Power efficiency}: No power wasted on carrier
\item
  \textbf{Less fading}: Improved performance in long-distance
\item
  \textbf{Better SNR}: More power concentrated in information
\end{itemize}

\textbf{Power Comparison:}

\begin{enumerate}
\tightlist
\item
  In AM: PT = Pc(1 + μ^{2}/2)
\item
For

μ = 1, PT = Pc(1 + 0.5) = 1.5Pc

\item
  AM power distribution: Carrier (Pc) = 67\%, Sidebands = 33\%
\item
  SSB uses only one sideband with no carrier
\item
  SSB power = 16.5\% of total AM power = 1/6 approx.
\end{enumerate}

\end{solutionbox}
\begin{mnemonicbox}
``Single Side Saves Bandwidth And Power''

\end{mnemonicbox}
\subsection*{Question 3(a) OR [3
marks]}\label{q3a}

\textbf{State following. (1) Bandwidth of modulated signal if modulating
frequency is 5 KHZ. (2) Image frequency if selected station frequency is
1000 KhZ in radio (3) Sampling frequency if baseband signal frequency is
10 KHz.}

\begin{solutionbox}

\begin{enumerate}
\tightlist
\item
  \textbf{Bandwidth of AM with modulating frequency 5 kHz:}

  \begin{itemize}
  \tightlist
  \item
    BW = 2 \times fm = 2 \times 5 kHz = 10 kHz
  \end{itemize}
\item
  \textbf{Image frequency for 1000 kHz station with 455 kHz IF:}

  \begin{itemize}
  \tightlist
  \item
    For high-side injection: fimage = fstation + 2 \times fIF
  \item
    fimage = 1000 + 2 \times 455 = 1000 + 910 = 1910 kHz
  \end{itemize}
\item
  \textbf{Sampling frequency for 10 kHz baseband:}

  \begin{itemize}
  \tightlist
  \item
    fs \textgreater{} 2 \times fmax (Nyquist rate)
  \item
    fs \textgreater{} 2 \times 10 kHz = 20 kHz
  \item
    Sampling frequency should be \textgreater{} 20 kHz
  \end{itemize}
\end{enumerate}

\end{solutionbox}
\begin{mnemonicbox}
``Bandwidth Doubles, Image Adds Twice-IF, Sampling
Needs Twice-Frequency''

\end{mnemonicbox}
\subsection*{Question 3(b) OR [4
marks]}\label{q3b}

\textbf{Draw following signal stating its mathematical equation. (1)
Sine wave (2) Unit step signal (3) Ramp signal (4) Impulse signal.}

\begin{solutionbox}

\textbf{1. Sine Wave:}

\begin{itemize}
\tightlist
\item
  Equation: f(t) = A sin(ωt + φ)
\end{itemize}

\begin{verbatim}
        \^{}
        |
    A   |    /{      /      }
        |   /  {    /      }
        |  /    {  /      }
    0   |{-+{-}{-}{-}{-}{-}{-}+{-}{-}{-}{-}{-}{-}{-}+{-}{-}{-}{-} t}
        |  {    /      /  }
        |   {  /      /    }
   {-A   |    /      /      }
\end{verbatim}

\textbf{2. Unit Step Signal:}

\begin{itemize}
\tightlist
\item
  Equation: u(t) = 1 for t \geq 0, 0 for t \textless{} 0
\end{itemize}

\begin{verbatim}
        \^{}
        |
    1   |        |‾‾‾‾‾‾‾‾‾‾‾‾‾‾‾‾
        |        |
        |        |
    0   |‾‾‾‾‾‾‾‾+‾‾‾‾‾‾‾‾‾‾‾‾‾‾‾{ t}
        |        0
\end{verbatim}

\textbf{3. Ramp Signal:}

\begin{itemize}
\tightlist
\item
  Equation: r(t) = t for t \geq 0, 0 for t \textless{} 0
\end{itemize}

\begin{verbatim}
        \^{}
        |                /
        |               /
        |              /
        |             /
        |            /
    0   |‾‾‾‾‾‾‾‾‾‾‾+‾‾‾‾‾‾‾‾‾‾‾‾‾‾{ t}
        |           0
\end{verbatim}

\textbf{4. Impulse Signal:}

\begin{itemize}
\tightlist
\item
Equation: δ(t) = \infty for

t = 0, 0 for t \neq 0

\end{itemize}

\begin{verbatim}
        \^{}
        |
        |
        |        |
        |        |
    0   |‾‾‾‾‾‾‾‾+‾‾‾‾‾‾‾‾‾‾‾‾‾‾‾{ t}
        |        0
\end{verbatim}

\end{solutionbox}
\begin{mnemonicbox}
``Sine Oscillates, Step Jumps, Ramp Climbs, Impulse
Spikes''

\end{mnemonicbox}
\subsection*{Question 3(c) OR [7
marks]}\label{q3c}

\textbf{Draw and explain Pre emphasis and De emphasis circuit with its
need \& characteristic graph. Also compare FM receiver with AM receiver
in detail.}

\begin{solutionbox}

\textbf{Pre-emphasis Circuit:}

\begin{verbatim}
        ┌───────┐
        │       │
    ────┤   R   ├────┬──────
        │       │    │
        └───────┘    │
                     │
                  ┌──┴──┐
                  │     │
                  │  C  │
                  │     │
                  └──┬──┘
                     │
                     │
                     ▼
\end{verbatim}

\textbf{De-emphasis Circuit:}

\begin{verbatim}
                  ┌───────┐
                  │       │
             ┌────┤   R   ├────
             │    │       │
             │    └───────┘
             │
     ────────┴────┐
                  │
               ┌──┴──┐
               │     │
               │  C  │
               │     │
               └─────┘
\end{verbatim}

\textbf{Characteristic Graph:}

\begin{verbatim}
    Gain(dB)
        \^{}
        |                 Pre{-emphasis}
        |              ,/
        |            ,/
        |          ,/
    0   |‾‾‾‾‾‾‾‾‾+‾‾‾‾‾‾‾‾‾‾‾‾‾‾‾{ Frequency}
        |          fc
        |          {,}
        |            {,}
        |              {,  De{-}emphasis}
\end{verbatim}

\textbf{Need for Pre/De-emphasis:}

\begin{itemize}
\tightlist
\item
  \textbf{Noise reduction}: Higher frequencies more susceptible to noise
\item
  \textbf{Improves SNR}: Boosts high frequencies at transmitter,
  attenuates at receiver
\item
  \textbf{Time constant}: Typically 75μs in FM broadcasting
\end{itemize}

\textbf{Comparison between FM and AM Receiver:}

{\def\LTcaptype{none} % do not increment counter
\begin{longtable}[]{@{}lll@{}}
\toprule\noalign{}
Parameter & FM Receiver & AM Receiver \\
\midrule\noalign{}
\endhead
\bottomrule\noalign{}
\endlastfoot
\textbf{IF Frequency} & 10.7 MHz & 455 kHz \\
\textbf{Bandwidth} & 200 kHz & 10 kHz \\
\textbf{Limiter Stage} & Present & Absent \\
\textbf{Demodulator} & Discriminator/ratio detector & Envelope
detector \\
\textbf{Pre/De-emphasis} & Present & Absent \\
\textbf{Audio Quality} & Superior & Moderate \\
\textbf{Noise Immunity} & High & Low \\
\textbf{Complexity} & More complex & Simpler \\
\end{longtable}
}

\end{solutionbox}
\begin{mnemonicbox}
``Pre Boosts Highs, De Cuts Them; FM Filters Noise
Better Than AM''

\end{mnemonicbox}
\subsection*{Question 4(a) [3 marks]}\label{q4a}

\textbf{Define Image frequency in a radio receiver and explain it with
suitable example.}

\begin{solutionbox}

\textbf{Image Frequency}: Unwanted signal frequency that produces the
same IF as the desired signal when mixed with local oscillator signal.

\textbf{Explanation:}

\begin{itemize}
\tightlist
\item
  For high-side injection: fimage = fsignal + 2 \times fIF
\item
  For low-side injection: fimage = fsignal - 2 \times fIF
\end{itemize}

\textbf{Example:}

\begin{itemize}
\tightlist
\item
  Desired signal: 1000 kHz
\item
  IF: 455 kHz
\item
  Local oscillator frequency (high-side): fLO = 1000 + 455 = 1455 kHz
\item
  Image frequency: fimage = fLO + 455 = 1455 + 455 = 1910 kHz
\item
  Both 1000 kHz and 1910 kHz will produce 455 kHz IF when mixed with
  1455 kHz
\end{itemize}

\end{solutionbox}
\begin{mnemonicbox}
``Image In radio Is Interfering 2IF away''

\end{mnemonicbox}
\subsection*{Question 4(b) [4 marks]}\label{q4b}

\textbf{Draw and explain envelope detector circuit for demodulation of
Amplitude modulated signal.}

\begin{solutionbox}

\textbf{Envelope Detector Circuit:}

\begin{verbatim}
         D
    ┌────►|────┬────────┐
    │          │        │
    │          │        │
Inpt│       ┌──┴──┐  ┌──┴──┐ Output
    │       │     │  │     │
    │       │  R  │  │  C  │
    │       │     │  │     │
    └───────┴──┬──┘  └──┬──┘
               │        │
               └────────┴───────
                     Ground
\end{verbatim}

\textbf{Working Principle:}

\begin{itemize}
\tightlist
\item
  \textbf{Diode}: Rectifies AM signal, removing negative half-cycles
\item
  \textbf{RC Circuit}: Acts as low-pass filter
\item
  \textbf{Time Constant}: RC must satisfy: 1/fm
  \textgreater\textgreater{} RC \textgreater\textgreater{} 1/fc
\item
  \textbf{Output}: Envelope of AM signal, which is the original message
\end{itemize}

\textbf{Envelope Detection Process:}

\begin{enumerate}
\tightlist
\item
  Diode conducts during positive half-cycles
\item
  Capacitor charges to peak value
\item
  During negative half-cycles, capacitor discharges through resistor
\item
  Output follows envelope of AM signal
\end{enumerate}

\end{solutionbox}
\begin{mnemonicbox}
``Diode Rectifies, RC Smooths Envelope''

\end{mnemonicbox}
\subsection*{Question 4(c) [7 marks]}\label{q4c}

\textbf{Draw block diagram of AM radio receiver and explain working of
each block.}

\begin{solutionbox}

\textbf{AM Radio Receiver (Superheterodyne) Block Diagram:}

\begin{center}
\textbf{Mermaid Diagram (Code)}
\begin{verbatim}
{Shaded}
{Highlighting}[]
graph LR
    A[RF Amplifier] {-{-}{} B[Mixer]}
    G[Local Oscillator] {-{-}{} B}
    B {-{-}{} C[IF Amplifier]}
    C {-{-}{} D[Detector]}
    D {-{-}{} E[AF Amplifier]}
    E {-{-}{} F[Speaker]}
{Highlighting}
{Shaded}
\end{verbatim}
\end{center}

\textbf{Functions of each block:}

\begin{itemize}
\tightlist
\item
  \textbf{RF Amplifier}:

  \begin{itemize}
  \tightlist
  \item
    Selects desired station signal using tuned circuit
  \item
    Provides initial amplification
  \item
    Improves sensitivity and selectivity
  \item
    Reduces image frequency interference
  \end{itemize}
\item
  \textbf{Local Oscillator}:

  \begin{itemize}
  \tightlist
  \item
    Generates frequency higher than incoming by IF value
  \item
    Typically fLO = fRF + 455 kHz
  \item
    Tuned simultaneously with RF amplifier
  \end{itemize}
\item
  \textbf{Mixer}:

  \begin{itemize}
  \tightlist
  \item
    Combines RF signal with local oscillator
  \item
    Produces sum and difference frequencies
  \item
    Outputs intermediate frequency (IF)
  \end{itemize}
\item
  \textbf{IF Amplifier}:

  \begin{itemize}
  \tightlist
  \item
    Fixed-frequency amplifier (455 kHz)
  \item
    Provides majority of receiver gain
  \item
    Determines selectivity of receiver
  \end{itemize}
\item
  \textbf{Detector}:

  \begin{itemize}
  \tightlist
  \item
    Extracts original audio from IF signal
  \item
    Usually envelope detector with diode
  \item
    Removes RF component, recovers audio
  \end{itemize}
\item
  \textbf{AF Amplifier}:

  \begin{itemize}
  \tightlist
  \item
    Amplifies recovered audio signal
  \item
    Includes volume control
  \item
    Drives speaker to audible levels
  \end{itemize}
\item
  \textbf{Speaker}:

  \begin{itemize}
  \tightlist
  \item
    Converts electrical signals to sound waves
  \end{itemize}
\end{itemize}

\end{solutionbox}
\begin{mnemonicbox}
``Radio Mixing Intermediate Detected Audio For
Speaker''

\end{mnemonicbox}
\subsection*{Question 4(a) OR [3
marks]}\label{q4a}

\textbf{State and explain Nyquist Criteria for sampling of signal.}

\begin{solutionbox}

\textbf{Nyquist Criteria}: To reconstruct a bandlimited signal without
distortion, sampling frequency must be at least twice the highest
frequency component in the signal.

\textbf{Mathematical statement:}

\begin{itemize}
\tightlist
\item
  fs \geq 2fmax
\item
  fs = sampling frequency
\item
  fmax = maximum frequency in signal
\end{itemize}

\textbf{Explanation:}

\begin{itemize}
\tightlist
\item
  Ensures no aliasing (frequency overlap) occurs
\item
  Minimum sampling rate called Nyquist rate
\item
  Sampling below Nyquist rate causes irreversible distortion
\item
  In practice, fs \textgreater{} 2.2fmax used to allow for filtering
\end{itemize}

\textbf{Example:}

\begin{itemize}
\tightlist
\item
  For audio with fmax = 20 kHz
\item
  Nyquist rate = 2 \times 20 kHz = 40 kHz
\item
  CD sampling rate = 44.1 kHz (\textgreater40 kHz)
\end{itemize}

\end{solutionbox}
\begin{mnemonicbox}
``Sample at least Twice as Fast as Highest
Frequency''

\end{mnemonicbox}
\subsection*{Question 4(b) OR [4
marks]}\label{q4b}

\textbf{Explain slope overload and granular noise for a delta
modulation.}

\begin{solutionbox}

\textbf{Delta Modulation Issues:}

\begin{center}
\textbf{Mermaid Diagram (Code)}
\begin{verbatim}
{Shaded}
{Highlighting}[]
graph LR
    A[Delta Modulation Problems] {-{-}{} B[Slope Overload]}
    A {-{-}{} C[Granular Noise]}
    B {-{-}{} D[Step size too small]}
    C {-{-}{} E[Step size too large]}
{Highlighting}
{Shaded}
\end{verbatim}
\end{center}

\textbf{Slope Overload:}

\begin{itemize}
\tightlist
\item
  Occurs when input signal changes faster than DM can track
\item
  Step size too small for rapidly changing signals
\item
  DM output cannot ``catch up'' with input
\item
  Creates distortion at sharp transitions
\item
  Solution: Increase step size or sampling rate
\end{itemize}

\textbf{Granular Noise:}

\begin{itemize}
\tightlist
\item
  Occurs during relatively constant signal portions
\item
  Step size too large for slowly changing signals
\item
  Output oscillates around input value
\item
  Creates ``roughness'' in reconstructed signal
\item
  Solution: Decrease step size
\end{itemize}

\textbf{Adaptive Delta Modulation (ADM):} Dynamically adjusts step size
to minimize both problems.

\end{solutionbox}
\begin{mnemonicbox}
``Slopes Need Bigger Steps, Flats Need Smaller
Steps''

\end{mnemonicbox}
\subsection*{Question 4(c) OR [7
marks]}\label{q4c}

\textbf{Draw and explain PCM transmitter and receiver in detail.}

\begin{solutionbox}

\textbf{PCM Transmitter:}

\begin{center}
\textbf{Mermaid Diagram (Code)}
\begin{verbatim}
{Shaded}
{Highlighting}[]
graph LR
    A[Input Signal] {-{-}{} B[Anti{-}aliasing Filter]}
    B {-{-}{} C[Sample \& Hold]}
    C {-{-}{} D[Quantizer]}
    D {-{-}{} E[Encoder]}
    E {-{-}{} F[Digital Output]}
{Highlighting}
{Shaded}
\end{verbatim}
\end{center}

\textbf{PCM Receiver:}

\begin{center}
\textbf{Mermaid Diagram (Code)}
\begin{verbatim}
{Shaded}
{Highlighting}[]
graph LR
    A[Digital Input] {-{-}{} B[Decoder]}
    B {-{-}{} C[D/A Converter]}
    C {-{-}{} D[Reconstruction Filter]}
    D {-{-}{} E[Output Signal]}
{Highlighting}
{Shaded}
\end{verbatim}
\end{center}

\textbf{Transmitter Components:}

\begin{itemize}
\tightlist
\item
  \textbf{Anti-aliasing filter}: Limits input bandwidth to prevent
  aliasing
\item
  \textbf{Sample \& Hold}: Captures instantaneous values at regular
  intervals
\item
  \textbf{Quantizer}: Approximates samples to predefined discrete levels
\item
  \textbf{Encoder}: Converts quantized values to binary code
\end{itemize}

\textbf{Receiver Components:}

\begin{itemize}
\tightlist
\item
  \textbf{Decoder}: Converts binary code back to quantized values
\item
  \textbf{D/A Converter}: Transforms discrete values to continuous
  voltage
\item
  \textbf{Reconstruction filter}: Removes sampling frequency components,
  smooths output
\end{itemize}

\textbf{PCM Parameters:}

\begin{itemize}
\tightlist
\item
  \textbf{Resolution}: Determined by bits per sample (n)
\item
  \textbf{Quantization levels}: L = 2\^{}n
\item
  \textbf{Bit rate}: R = n \times fs (bits per second)
\item
  \textbf{SNR}: Improves by \textasciitilde6dB per bit added
\end{itemize}

\end{solutionbox}
\begin{mnemonicbox}
``Sample, Quantize, Encode; Decode, Convert,
Reconstruct''

\end{mnemonicbox}
\subsection*{Question 5(a) [3 marks]}\label{q5a}

\textbf{Define Bit, Bit rate and Baud rate with suitable example.}

\begin{solutionbox}

\begin{itemize}
\tightlist
\item
  \textbf{Bit}: Smallest unit of digital information, representing
  either 0 or 1.

  \begin{itemize}
  \tightlist
  \item
    Example: 10110 contains 5 bits
  \end{itemize}
\item
  \textbf{Bit Rate}: Number of bits transmitted per second.

  \begin{itemize}
  \tightlist
  \item
    Unit: bps (bits per second)
  \item
    Example: 9600 bps means 9600 bits transmitted in one second
  \end{itemize}
\item
  \textbf{Baud Rate}: Number of signal changes (symbols) per second.

  \begin{itemize}
  \tightlist
  \item
    Unit: Baud
  \item
    Example: In QPSK, each symbol represents 2 bits, so 9600 bps = 4800
    Baud
  \end{itemize}
\end{itemize}

\textbf{Relationship:}

\begin{itemize}
\tightlist
\item
  Bit Rate = Baud Rate \times number of bits per symbol
\item
  For binary signaling (1 bit/symbol): Bit Rate = Baud Rate
\item
  For multilevel coding: Bit Rate \textgreater{} Baud Rate
\end{itemize}

\end{solutionbox}
\begin{mnemonicbox}
``Bits Build Data, Baud Brings Symbols''

\end{mnemonicbox}
\subsection*{Question 5(b) [4 marks]}\label{q5b}

\textbf{Define multiplexing. State its types. Explain Frequency division
multiplexing with suitable diagram.}

\begin{solutionbox}

\textbf{Multiplexing}: Technique that allows multiple signals to share a
common transmission medium.

\textbf{Types of Multiplexing:}

\begin{itemize}
\tightlist
\item
  Frequency Division Multiplexing (FDM)
\item
  Time Division Multiplexing (TDM)
\item
  Code Division Multiplexing (CDM)
\item
  Wavelength Division Multiplexing (WDM)
\end{itemize}

\textbf{Frequency Division Multiplexing:}

\begin{verbatim}
    Frequency
        \^{}
        |
        |  ┌───┐  ┌───┐  ┌───┐  ┌───┐
        |  │Ch1│  │Ch2│  │Ch3│  │Ch4│
        |  │   │  │   │  │   │  │   │
    0   |{-{-}+{-}{-}{-}+{-}{-}+{-}{-}{-}+{-}{-}+{-}{-}{-}+{-}{-}+{-}{-}{-}+{-}{-}{-} Frequency}
        |  f1     f2     f3     f4
        |
        |  Guard Bands between channels
\end{verbatim}

\textbf{FDM Working Principle:}

\begin{itemize}
\tightlist
\item
  Each signal modulated to different carrier frequency
\item
  Bandwidth allocated to each channel with guard bands
\item
  All channels transmitted simultaneously
\item
  Receiver uses filters to separate channels
\item
  Used in radio/TV broadcasting, cable systems
\end{itemize}

\end{solutionbox}
\begin{mnemonicbox}
``Frequency Divides Multiple Signals Simultaneously''

\end{mnemonicbox}
\subsection*{Question 5(c) [7 marks]}\label{q5c}

\textbf{Draw and explain basic PCM-TDM diagram with diagram.}

\begin{solutionbox}

\textbf{PCM-TDM System Block Diagram:}

\begin{center}
\textbf{Mermaid Diagram (Code)}
\begin{verbatim}
{Shaded}
{Highlighting}[]
graph LR
    \%\% Transmitter
    A1[Source 1] {-{-}{} B1[LPF 1]}
    A2[Source 2] {-{-}{} B2[LPF 2]}
    A3[Source 3] {-{-}{} B3[LPF 3]}
    B1 {-{-}{} C[Commutator/MUX]}
    B2 {-{-}{} C}
    B3 {-{-}{} C}
    C {-{-}{} D[Sampler]}
    D {-{-}{} E[Quantizer] }
    E {-{-}{} F[Encoder]}
    F {-{-}{} G[TDM Output]}
    
    \%\% Receiver
    G {-{-}{} H[Decoder]}
    H {-{-}{} I[DEMUX]}
    I {-{-}{} J1[LPF 1]}
    I {-{-}{} J2[LPF 2]}
    I {-{-}{} J3[LPF 3]}
    J1 {-{-}{} K1[Output 1]}
    J2 {-{-}{} K2[Output 2]}
    J3 {-{-}{} K3[Output 3]}
{Highlighting}
{Shaded}
\end{verbatim}
\end{center}

\textbf{PCM-TDM System Operation:}

\textbf{Transmitter Side:}

\begin{itemize}
\tightlist
\item
  \textbf{Input Sources}: Multiple analog signals
\item
  \textbf{Low-Pass Filters}: Limit bandwidth of input signals
\item
  \textbf{Commutator/MUX}: Sequentially samples each input
\item
  \textbf{Sampler}: Converts continuous signals to discrete samples
\item
  \textbf{Quantizer}: Approximates samples to nearest discrete levels
\item
  \textbf{Encoder}: Converts quantized values to binary code
\item
  \textbf{TDM Output}: Transmits frames containing samples from all
  channels
\end{itemize}

\textbf{Receiver Side:}

\begin{itemize}
\tightlist
\item
  \textbf{Decoder}: Converts binary code back to quantized values
\item
  \textbf{DEMUX}: Distributes samples to appropriate channel paths
\item
  \textbf{Low-Pass Filters}: Reconstruct original signals, remove
  sampling components
\item
  \textbf{Outputs}: Recovered original signals
\end{itemize}

\textbf{TDM Frame Format:}

\begin{verbatim}
    ┌──────┬──────┬──────┬──────┬──────┬──────┐
    │ Sync │ Ch 1 │ Ch 2 │ Ch 3 │ Ch 1 │ Ch 2 │...
    └──────┴──────┴──────┴──────┴──────┴──────┘
      Frame header    Channel samples repeat
\end{verbatim}

\end{solutionbox}
\begin{mnemonicbox}
``Pulse Code TDM: Sample, Quantize, Encode,
Multiplex''

\end{mnemonicbox}
\subsection*{Question 5(a) OR [3
marks]}\label{q5a}

\textbf{State types of TDM and explain any one of them.}

\begin{solutionbox}

\textbf{Types of TDM:}

\begin{itemize}
\tightlist
\item
  Synchronous TDM
\item
  Asynchronous TDM (Statistical TDM)
\item
  Intelligent TDM
\end{itemize}

\textbf{Synchronous TDM:}

\begin{itemize}
\tightlist
\item
  Fixed time slots allocated to each channel
\item
  Time slots transmitted in fixed sequence
\item
  Time slots remain empty if channel has no data
\item
  Simpler implementation but less efficient
\item
  Example: T1 carrier system (24 channels \times 8 bits \times 8000 samples/sec =
  1.544 Mbps)
\end{itemize}

\textbf{Frame Structure:}

\begin{verbatim}
    ┌──────┬──────┬──────┬──────┬──────┐
    │ Sync │ Ch 1 │ Ch 2 │ Ch 3 │ Ch 4 │
    └──────┴──────┴──────┴──────┴──────┘
      Fixed slots regardless of activity
\end{verbatim}

\end{solutionbox}
\begin{mnemonicbox}
``Synchronous Slots Stay Steady''

\end{mnemonicbox}
\subsection*{Question 5(b) OR [4
marks]}\label{q5b}

\textbf{Explain TDM. Also State its advantages and disadvantages.}

\begin{solutionbox}

\textbf{Time Division Multiplexing (TDM):} Technique where multiple
signals share same transmission medium by allocating different time
slots to each signal.

\textbf{Working Principle:}

\begin{itemize}
\tightlist
\item
  Each signal sampled at regular intervals
\item
  Samples interleaved in time domain
\item
  Complete frame contains one sample from each channel
\item
  Receiver separates samples to reconstruct original signals
\end{itemize}

\textbf{Advantages of TDM:}

\begin{itemize}
\tightlist
\item
  \textbf{Single medium}: Efficiently uses one transmission path
\item
  \textbf{Digital compatibility}: Naturally suits digital systems
\item
  \textbf{Crosstalk elimination}: No interference between channels
\item
  \textbf{Flexible capacity}: Easy to add/remove channels
\item
  \textbf{Cost-effective}: Reduces hardware requirements
\end{itemize}

\textbf{Disadvantages of TDM:}

\begin{itemize}
\tightlist
\item
  \textbf{Synchronization critical}: Timing errors cause major problems
\item
  \textbf{Complex equipment}: Requires precise timing circuits
\item
  \textbf{Bandwidth limitation}: High bit rate needed for many channels
\item
  \textbf{Inefficiency}: Wastes capacity when channels inactive (in
  synchronous TDM)
\item
  \textbf{Buffer delays}: Can cause latency issues
\end{itemize}

\end{solutionbox}
\begin{mnemonicbox}
``Time Divided Multiple signals Save costs But Need
Precise timing''

\end{mnemonicbox}
\subsection*{Question 5(c) OR [7
marks]}\label{q5c}

\textbf{State desirable properties of line coding. Draw waveform in time
relation for unipolar RZ, Polar NRZ, and Manchester line coding for a 8
bit digital data 01001110.}

\begin{solutionbox}

\textbf{Desirable Properties of Line Coding:}

\begin{itemize}
\tightlist
\item
  \textbf{DC component}: Should be minimal or absent
\item
  \textbf{Self-synchronization}: Should provide timing information
\item
  \textbf{Error detection}: Should allow detection of transmission
  errors
\item
  \textbf{Bandwidth efficiency}: Should require minimum bandwidth
\item
  \textbf{Noise immunity}: Should be resistant to noise and interference
\item
  \textbf{Cost \& complexity}: Should be simple to implement
\end{itemize}

\textbf{Line Coding Waveforms for 01001110:}

\begin{verbatim}
    Bit pattern:  0  1  0  0  1  1  1  0
    
    Unipolar RZ:
        \^{}
        |
    A   |    ┌─┐     ┌─┐ ┌─┐ ┌─┐
        |    │ │     │ │ │ │ │ │
    0   |────┘ └─────┘ └─┘ └─┘ └───{ t}
        
    Polar NRZ:
        \^{}
        |
    +A  |    ┌─────┐     ┌───────┐
        |    │     │     │       │
    0   |────┘     └─────┘       └───{ t}
        |                            
    {-A  |─┐         ┌─────┐           }
        | └─────────┘     │           
        
    Manchester:
        \^{}
        |                            
    +A  |─┐   ┌─┐ ┌─┐   ┌─┐   ┌─┐ ┌─┐
        | │   │ │ │ │   │ │   │ │ │ │
    0   |─┘   └─┘ └─┘   └─┘   └─┘ └─{ t}
        |                            
    {-A  |  ┌─┐       ┌─┐       ┌─┐   }
        |  │ │       │ │       │ │   
        |  └─┘       └─┘       └─┘   
        
    Legend: 0 = Low, 1 = High
\end{verbatim}

\textbf{Key characteristics:}

\begin{itemize}
\tightlist
\item
  \textbf{Unipolar RZ}: Returns to zero in middle of bit, only positive
  voltages
\item
  \textbf{Polar NRZ}: No return to zero, uses positive and negative
  voltages
\item
  \textbf{Manchester}: Mid-bit transition, rising edge = 0, falling edge
  = 1
\end{itemize}

\end{solutionbox}
\begin{mnemonicbox}
``Unipolar Rises then Zeros, Polar Never Returns,
Manchester Always Transitions''

\end{mnemonicbox}

\end{document}
