\documentclass{article}

% content/resources/templates/preamble.tex
\usepackage[margin=0.6in]{geometry}
\author{Milav Dabgar}
\usepackage{amsmath,amssymb,amsthm}
\usepackage{booktabs}
\usepackage{multirow}
\usepackage{xcolor}
\usepackage{tcolorbox}
\tcbuselibrary{breakable,skins}
\usepackage[colorlinks=true,linkcolor=blue]{hyperref}
\usepackage{titlesec}
\usepackage{enumitem}
\usepackage{tikz}
\usepackage{pgfplots}
\usepackage{circuitikz}
\usepackage[version=4]{mhchem}
\usepackage{longtable}
\usepackage{array}
\usepackage{float}
\usepackage{caption}
\usepackage{listings}

\lstset{
  basicstyle=\small\ttfamily,
  breaklines=true,
  breakatwhitespace=false,
  postbreak=\mbox{\textcolor{red}{$\hookrightarrow$}\space},
  float=false,
  numbers=left,
  numberstyle=\tiny\color{gray},
  numbersep=10pt,
  xleftmargin=2em,
  keywordstyle=\color{blue},
  commentstyle=\color{green!60!black},
  stringstyle=\color{purple},
  backgroundcolor=\color{gray!5},
  showstringspaces=false,
  tabsize=2,
  captionpos=b,
  keepspaces=true,
  columns=flexible
}

\pgfplotsset{compat=1.18}
\usetikzlibrary{shapes,arrows,positioning,calc,patterns,decorations.pathmorphing,decorations.markings,arrows.meta}

% Color scheme
\definecolor{headcolor}{RGB}{0,102,204}
\definecolor{keycolor}{RGB}{220,20,60}
\definecolor{solutioncolor}{RGB}{34,139,34}
\definecolor{mnemoniccolor}{RGB}{148,0,211}
\definecolor{codecolor}{RGB}{0,0,100}

% Spacing
\setlength{\parskip}{3pt}
\setlist[itemize]{nosep}
\setlist[enumerate]{nosep}

% Title formatting
\titleformat{\section}{\Large\bfseries\color{headcolor}}{\thesection}{1em}{}
\titleformat{\subsection}{\large\bfseries\color{headcolor}}{\thesubsection}{1em}{}

% Pandoc tightlist compatibility
\providecommand{\tightlist}{%
  \setlength{\itemsep}{0pt}\setlength{\parskip}{0pt}}

% Pandoc longtable compatibility
\newcounter{none}
\def\thenone{}


% content/resources/templates/gujarati-boxes.tex
\usepackage{fontspec}
\usepackage{polyglossia}

% Set Gujarati as main language (document is primarily in Gujarati)
% Note: gloss-gujarati.ldf doesn't exist in polyglossia, but it will use hyphenation patterns
\setdefaultlanguage{gujarati}
\setotherlanguage{english}

% Configure Gujarati font properly
% Use Language=Default to prevent polyglossia from trying to add language-specific features
% that don't exist for Gujarati, which causes "empty feature" warnings
\newfontfamily\gujaratifont[Script=Gujarati,AutoFakeBold=2.5,AutoFakeSlant=0.3]{Noto Sans Gujarati}
\setmainfont[Script=Gujarati,AutoFakeBold=2.5,AutoFakeSlant=0.3]{Noto Sans Gujarati}
% Use Noto Sans Gujarati for monospace to support Gujarati in text
\setmonofont[Scale=0.9]{Noto Sans Gujarati}

% Configure English to use the same font
\newfontfamily\englishfont[Script=Gujarati,AutoFakeBold=2.5,AutoFakeSlant=0.3]{Noto Sans Gujarati}

% Translations for polyglossia
\gappto\captionsgujarati{
  \renewcommand{\tablename}{કોષ્ટક}
  \renewcommand{\figurename}{આકૃતિ}
}

% Helper for TikZ nodes to ensure Gujarati font
\newcommand{\gu}[1]{{\gujaratifont #1}}

% Custom environments
\newtcolorbox{solutionbox}{
    breakable,
    enhanced,
    colback=solutioncolor!5!white,
    colframe=solutioncolor!75!black,
    fonttitle=\bfseries,
    title=જવાબ
}

\newtcolorbox{solutionboxnobreak}{
 colback=solutioncolor!5!white,
 colframe=solutioncolor!75!black,
 fonttitle=\bfseries,
 title=જવાબ
}

\newtcolorbox{keyformula}{
 breakable,
 enhanced,
 colback=keycolor!5!white,
 colframe=keycolor!75!black,
 fonttitle=\bfseries,
 title=રાસાયણિક સમીકરણ/સૂત્ર
}

\newtcolorbox{mnemonicbox}{
 breakable,
 enhanced,
 colback=mnemoniccolor!5!white,
 colframe=mnemoniccolor!75!black,
 fonttitle=\bfseries,
 title=મેમરી ટ્રીક
}


% Custom commands for GTU solutions
% This file defines semantic commands for consistent formatting

% Question command with automatic formatting
\newcommand{\question}[2]{%
  \section*{Question #1}%
  \textbf{#2}%
}

% OR question variant
\newcommand{\questionor}[2]{%
  \section*{Question #1 OR}%
  \textbf{#2}%
}

% Proper table environment with caption
\newenvironment{answertable}[1]{%
  \begin{table}[htbp]
  \centering
  \caption{#1}
}{%
  \end{table}
}

% Proper figure environment for diagrams
\newenvironment{answerdiagram}[1]{%
  \begin{figure}[htbp]
  \centering
  \caption{#1}
}{%
  \end{figure}
}

% Semantic markup for key terms
\newcommand{\keyword}[1]{\textbf{#1}}
\newcommand{\code}[1]{\texttt{#1}}
\newcommand{\classname}[1]{\texttt{#1}}
\newcommand{\methodname}[1]{\texttt{#1}}

% Proper quotation marks
\newcommand{\mnemonic}[1]{``#1''}


\title{Principles of Electronic Communication (4331104) - Winter 2022 Solution}
\date{March 01, 2022}

\begin{document}
\maketitle

\questionmarks{1(a)}{3}{What is modulation? What is the need of it?}

\begin{solutionbox}
Modulation is the process of varying one or more properties (amplitude, frequency, or phase) of a high-frequency carrier signal with a modulating signal containing information.

\textbf{Need for modulation:}

\begin{itemize}
    \item \textbf{Antenna size reduction}: Makes practical antenna size possible ($\lambda = c/f$)
    \item \textbf{Multiplexing}: Allows multiple signals to share the medium
    \item \textbf{Noise reduction}: Improves SNR by shifting to higher frequency bands
    \item \textbf{Range extension}: Increases transmission distance
\end{itemize}
\end{solutionbox}

\begin{mnemonicbox}
\mnemonic{AMEN: Antenna size, Multiplexing, Eliminate noise, New range}
\end{mnemonicbox}

\questionmarks{1(b)}{4}{Derive voltage equation for Amplitude modulation.}

\begin{solutionbox}
For AM, the carrier signal is modulated by the message signal.

\textbf{Mathematical derivation:}

\begin{itemize}
    \item \textbf{Carrier signal}: $e_c(t) = A_c \cos(2\pi f_c t)$
    \item \textbf{Message signal}: $e_m(t) = A_m \cos(2\pi f_m t)$
    \item \textbf{Instantaneous amplitude}: $A_i = A_c + e_m(t)$
    \item \textbf{AM signal}: $e_{AM}(t) = A_i \cos(2\pi f_c t)$
    \item \textbf{Substituting}: $e_{AM}(t) = [A_c + A_m \cos(2\pi f_m t)] \cos(2\pi f_c t)$
    \item \textbf{Expanding}: $e_{AM}(t) = A_c\cos(2\pi f_c t) + A_m\cos(2\pi f_m t)\cos(2\pi f_c t)$
    \item \textbf{Final equation}: $e_{AM}(t) = A_c\cos(2\pi f_c t) + \frac{A_m}{2}\cos(2\pi(f_c+f_m)t) + \frac{A_m}{2}\cos(2\pi(f_c-f_m)t)$
\end{itemize}
\end{solutionbox}

\begin{mnemonicbox}
\mnemonic{CAT: Carrier, Addition, Three components (carrier + 2 sidebands)}
\end{mnemonicbox}

\questionmarks{1(c)}{7}{Classify Noise signal and explain flicker noise, shot noise and thermal noise.}

\begin{solutionbox}
\textbf{Noise classification:}

\begin{center}
\captionof{table}{Noise Classification}
\begin{tabulary}{\linewidth}{|L|L|L|}
\hline
\textbf{Type} & \textbf{Sources} & \textbf{Characteristics} \\ \hline
\textbf{External Noise} & Atmospheric, Space, Industrial, Man-made & Originates outside communication system \\ \hline
\textbf{Internal Noise} & Thermal, Shot, Transit-time, Flicker & Originates inside components \\ \hline
\end{tabulary}
\end{center}

\textbf{Types of Internal Noise:}

\begin{itemize}
    \item \textbf{Flicker Noise}:
    \begin{itemize}
        \item Occurs at low frequencies (below 1 kHz)
        \item Inversely proportional to frequency (1/f noise)
        \item Common in semiconductor devices and carbon resistors
    \end{itemize}
    \item \textbf{Shot Noise}:
    \begin{itemize}
        \item Caused by random fluctuations of current carriers
        \item White noise with constant power density
        \item Occurs in active devices like diodes and transistors
    \end{itemize}
    \item \textbf{Thermal Noise}:
    \begin{itemize}
        \item Due to random motion of electrons in a conductor
        \item Directly proportional to temperature and bandwidth
        \item Present in all passive components
        \item Also called Johnson noise or white noise
    \end{itemize}
\end{itemize}
\end{solutionbox}

\begin{mnemonicbox}
\mnemonic{FAST: Flicker (low frequency), Active (shot), Semiconductor (flicker), Temperature (thermal)}
\end{mnemonicbox}

\questionmarks{1(c) OR}{7}{Write application of different band of EM wave spectrum.}

\begin{solutionbox}
\textbf{EM Spectrum Applications:}

\begin{center}
\captionof{table}{EM Spectrum Applications}
\begin{tabulary}{\linewidth}{|L|L|L|}
\hline
\textbf{Frequency Band} & \textbf{Frequency Range} & \textbf{Applications} \\ \hline
\textbf{ELF} (Extremely Low Frequency) & 3Hz -- 30Hz & Submarine communication \\ \hline
\textbf{VLF} (Very Low Frequency) & 3kHz -- 30kHz & Navigation, time signals \\ \hline
\textbf{LF} (Low Frequency) & 30kHz -- 300kHz & AM radio, navigation \\ \hline
\textbf{MF} (Medium Frequency) & 300kHz -- 3MHz & AM broadcasting, maritime \\ \hline
\textbf{HF} (High Frequency) & 3MHz -- 30MHz & Shortwave radio, amateur radio \\ \hline
\textbf{VHF} (Very High Frequency) & 30MHz -- 300MHz & FM radio, TV broadcasting, air traffic control \\ \hline
\textbf{UHF} (Ultra High Frequency) & 300MHz -- 3GHz & TV broadcasting, mobile phones, WiFi, Bluetooth \\ \hline
\textbf{SHF} (Super High Frequency) & 3GHz -- 30GHz & Satellite communication, radar, WiFi \\ \hline
\textbf{EHF} (Extremely High Frequency) & 30GHz -- 300GHz & Radio astronomy, 5G, millimeter-wave radar \\ \hline
\textbf{Infrared} & 300GHz -- 400THz & Remote controls, thermal imaging, fiber optics \\ \hline
\textbf{Visible Light} & 400THz -- 800THz & Fiber optics, LiFi, photography \\ \hline
\textbf{Ultraviolet} & 800THz -- 30PHz & Sterilization, fluorescence, security \\ \hline
\textbf{X-rays} & 30PHz -- 30EHz & Medical imaging, security screening \\ \hline
\textbf{Gamma rays} & $>$30EHz & Medical treatments, nuclear detection \\ \hline
\end{tabulary}
\end{center}
\end{solutionbox}

\begin{mnemonicbox}
\mnemonic{Every Very Lovely Monkey Has Visited Uncle Sam's House Easily In Visible Upper Xtra Gamma}
\end{mnemonicbox}

\questionmarks{2(a)}{3}{State advantages of SSB over DSB.}

\begin{solutionbox}
\textbf{Advantages of SSB over DSB:}

\begin{center}
\captionof{table}{Advantages of SSB}
\begin{tabulary}{\linewidth}{|L|L|}
\hline
\textbf{Advantage} & \textbf{Description} \\ \hline
\textbf{Bandwidth Efficiency} & Uses half the bandwidth (only one sideband) \\ \hline
\textbf{Power Efficiency} & Requires less transmitter power (83.33\% power saving) \\ \hline
\textbf{Reduced Fading} & Less susceptible to selective fading \\ \hline
\textbf{Less Distortion} & Reduced intermodulation distortion \\ \hline
\textbf{Simplified Receiver} & Simpler circuit design possible \\ \hline
\end{tabulary}
\end{center}
\end{solutionbox}

\begin{mnemonicbox}
\mnemonic{BPFDS: Bandwidth, Power, Fading, Distortion, Simple}
\end{mnemonicbox}

\questionmarks{2(b)}{4}{Explain generation of FM using Phase lock loop technique.}

\begin{solutionbox}
A Phase-Locked Loop (PLL) generates FM signals by applying the modulating signal to the VCO control input.

\textbf{PLL FM Modulator:}

\begin{center}
\begin{tikzpicture}[node distance=2.5cm, auto, >=latex, every node/.style={transform shape}]
    % Nodes - scaled down a bit if needed
    \node [gtu block] (sum) {Summing Circuit};
    \node [left=1.5cm of sum] (input) {Modulating Signal};
    \node [gtu block, right=of sum] (vco) {VCO};
    \node [right=1.5cm of vco] (output) {FM Output};
    
    \node [gtu block, below=of vco] (feedback) {Feedback};
    \node [gtu block, left=of feedback] (pd) {Phase Detector};
    \node [gtu block, left=of pd] (ref) {Reference Oscillator};
    \node [gtu block, above=of pd] (lpf) {Low Pass Filter};
    
    % Connections
    \draw [gtu arrow] (input) -- (sum);
    \draw [gtu arrow] (sum) -- (vco);
    \draw [gtu arrow] (vco) -- (output);
    \draw [gtu arrow] (vco) -- (feedback);
    \draw [gtu arrow] (feedback) -- (pd);
    \draw [gtu arrow] (ref) -- (pd);
    \draw [gtu arrow] (pd) -- (lpf);
    \draw [gtu arrow] (lpf) -- (sum);

\end{tikzpicture}
\captionof{figure}{FM Generation using PLL}
\end{center}

\textbf{Operation:}

\begin{itemize}
    \item \textbf{Reference Oscillator}: Provides stable reference frequency
    \item \textbf{Phase Detector}: Compares reference and feedback signals
    \item \textbf{Low Pass Filter}: Removes high-frequency components
    \item \textbf{VCO}: Generates output frequency that varies with control voltage
    \item \textbf{Modulating Signal}: Added to control voltage to produce FM output
\end{itemize}
\end{solutionbox}

\begin{mnemonicbox}
\mnemonic{PROVE: Phase detector, Reference oscillator, Output VCO, Voltage controlled}
\end{mnemonicbox}

\questionmarks{2(c)}{7}{Derive the equation for total power in AM, calculate percentage of power savings in DSB and SSB.}

\begin{solutionbox}
\textbf{Power in AM:}

The AM wave equation: $e_{AM}(t) = A_c[1 + m\cos(2\pi f_m t)]\cos(2\pi f_c t)$

\textbf{Power derivation:}

\begin{itemize}
    \item \textbf{Total power}: $P_T = P_c\left(1 + \frac{m^2}{2}\right)$
    \item Where $P_c = \frac{A_c^2}{2R}$ (carrier power) and $m$ is modulation index
\end{itemize}

\textbf{Power distribution:}

\begin{itemize}
    \item \textbf{Carrier power}: $P_c = \frac{A_c^2}{2R}$
    \item \textbf{Total sideband power}: $P_{SB} = \frac{m^2 P_c}{2}$
    \item \textbf{Each sideband}: $P_{LSB} = P_{USB} = \frac{m^2 P_c}{4}$
\end{itemize}

\textbf{Power savings:}

\begin{itemize}
    \item \textbf{In DSB-SC}: No carrier power, so savings = $\frac{P_c}{P_T} \times 100\% = \frac{1}{1+\frac{m^2}{2}} \times 100\%$
    \begin{itemize}
        \item For m=1, savings = 66.67\%
    \end{itemize}
    \item \textbf{In SSB}: No carrier and one sideband, so savings = $\frac{P_c + P_{SB}/2}{P_T} \times 100\%$
    \begin{itemize}
        \item For m=1, savings = 83.33\%
    \end{itemize}
\end{itemize}
\end{solutionbox}

\begin{mnemonicbox}
\mnemonic{CEPTS: Carrier Eliminated Provides Tremendous Savings}
\end{mnemonicbox}

\questionmarks{2(a) OR}{3}{Draw and explain Time domain and Frequency domain display of AM wave.}

\begin{solutionbox}
\textbf{Time and Frequency Domain of AM:}

\begin{center}
\begin{tikzpicture}
    % Time Domain
    \begin{scope}[xshift=0cm, yshift=4cm]
        \draw[->] (0,0) -- (6,0) node[right] {$t$};
        \draw[->] (0,-2) -- (0,2) node[above] {$V(t)$};
        
        \draw[blue, thick, domain=0:5.5, samples=200, smooth] plot (\x, {1.5*(1 + 0.5*cos(2*3.1415*0.5*\x r))*cos(2*3.1415*10*\x r)});
        
        \draw[red, dashed, domain=0:5.5, samples=100] plot (\x, {1.5*(1 + 0.5*cos(2*3.1415*0.5*\x r))});
        \draw[red, dashed, domain=0:5.5, samples=100] plot (\x, {-1.5*(1 + 0.5*cos(2*3.1415*0.5*\x r))});
        
        \node at (2.75, -2.5) {Time Domain};
    \end{scope}

    % Frequency Domain
    \begin{scope}[xshift=0cm, yshift=0cm]
        \draw[->] (0,0) -- (6,0) node[right] {$f$};
        \draw[->] (0,0) -- (0,2.5) node[above] {$V(f)$};
        
        % Carrier
        \draw[thick, blue] (3,0) -- (3,2);
        \node[above] at (3,2) {$A_c$};
        \node[below] at (3,0) {$f_c$};
        
        % LSB
        \draw[thick, blue] (1.5,0) -- (1.5,1);
        \node[above] at (1.5,1) {$\frac{mA_c}{2}$};
        \node[below] at (1.5,0) {$f_c-f_m$};
        
        % USB
        \draw[thick, blue] (4.5,0) -- (4.5,1);
        \node[above] at (4.5,1) {$\frac{mA_c}{2}$};
        \node[below] at (4.5,0) {$f_c+f_m$};
        
        \node at (3, -1) {Frequency Domain};
    \end{scope}
\end{tikzpicture}
\captionof{figure}{Time and Frequency Domain of AM}
\end{center}

\textbf{Time Domain:}

\begin{itemize}
    \item Shows amplitude variation of carrier with time
    \item Envelope follows modulating signal
    \item Upper and lower envelopes = carrier peak $\times (1 \pm m)$
\end{itemize}

\textbf{Frequency Domain:}

\begin{itemize}
    \item Shows frequency components and their amplitudes
    \item Carrier at frequency $f_c$ with amplitude $A_c$
    \item Two sidebands at $f_c \pm f_m$ with amplitude $mA_c/2$
    \item Bandwidth = $2f_m$ (twice the modulating frequency)
\end{itemize}
\end{solutionbox}

\begin{mnemonicbox}
\mnemonic{EBS: Envelope in time, Bandwidth in frequency, Sidebands symmetric}
\end{mnemonicbox}

\questionmarks{2(b) OR}{4}{Explain pre-emphasis \& de-emphasis circuit.}

\begin{solutionbox}
\textbf{Pre-emphasis and De-emphasis:}

\begin{center}
\begin{tikzpicture}[american]
    % Pre-emphasis - High Pass
    \begin{scope}[xshift=0cm]
        \draw (0,2) to[short, o-] (0.5,2);
        \draw (0.5,2) -- (0.5, 3) to[C, l=$C$] (3.5, 3) -- (3.5, 2);
        \draw (0.5,2) -- (0.5, 1) to[R, l=$R_1$] (3.5, 1) -- (3.5, 2);
        \draw (3.5,2) to[short, -o] (5,2);
        \draw (4,2) to[R, l=$R_2$] (4,0);
        \draw (0,0) to[short, o-o] (5,0);
        
        \node[above] at (2.5, 3.5) {Pre-emphasis (Transmitter)};
        \node[below] at (2.5, -0.5) {Boosts High Frequencies};
    \end{scope}

    % De-emphasis - Low Pass
    \begin{scope}[xshift=7cm]
        \draw (0,2) to[R, l=$R$, o-] (3,2) to[short, -o] (4,2);
        \draw (3,2) to[C, l=$C$] (3,0);
        \draw (0,0) to[short, o-o] (4,0);
        
        \node[above] at (2, 2.5) {De-emphasis (Receiver)};
        \node[below] at (2, -0.5) {Attenuates High Frequencies};
    \end{scope}
\end{tikzpicture}
\captionof{figure}{Pre-emphasis and De-emphasis Circuits}
\end{center}

\textbf{Operation:}

\begin{itemize}
    \item \textbf{Pre-emphasis}: High-pass network (Time constant $\tau = 75\mu s$). Boosts high frequencies to improve SNR.
    \item \textbf{De-emphasis}: Low-pass network (Time constant $\tau = 75\mu s$). Attenuates high frequencies to restore original signal and reduce noise.
\end{itemize}
\end{solutionbox}

\begin{mnemonicbox}
\mnemonic{BETH: Boost (pre-emphasis), Emphasizes Treble, Helps SNR}
\end{mnemonicbox}

\questionmarks{2(c) OR}{7}{Compare AM, FM and PM.}

\begin{solutionbox}
\textbf{Comparison of AM, FM and PM:}

\begin{center}
\captionof{table}{Comparison of AM, FM, and PM}
\begin{tabulary}{\linewidth}{|L|L|L|L|}
\hline
\textbf{Parameter} & \textbf{AM} & \textbf{FM} & \textbf{PM} \\ \hline
\textbf{Definition} & Amplitude varies & Frequency varies & Phase varies \\ \hline
\textbf{Equation} & $A_c[1+m\cos(\omega_mt)]\cos(\omega_ct)$ & $A_c\cos[\omega_ct+m_f\sin(\omega_mt)]$ & $A_c\cos[\omega_ct+m_p\cos(\omega_mt)]$ \\ \hline
\textbf{Bandwidth} & $2f_m$ (Narrow) & $2(\Delta f+f_m)$ (Wide) & $2(m_p+1)f_m$ (Wide) \\ \hline
\textbf{Efficiency} & Low & High & High \\ \hline
\textbf{Noise} & Poor & Excellent & Excellent \\ \hline
\textbf{Complexity} & Simple & Complex & Complex \\ \hline
\textbf{Applications} & Broadcasting & Radio, TV & Satellite \\ \hline
\end{tabulary}
\end{center}
\end{solutionbox}

\begin{mnemonicbox}
\mnemonic{BANCP-MAP: Bandwidth, Amplitude, Noise, Complexity, Power, Modulation, Applications, Parameters}
\end{mnemonicbox}

\questionmarks{3(a)}{3}{Define any FOUR characteristics of radio receiver.}

\begin{solutionbox}
\textbf{Radio Receiver Characteristics:}

\begin{center}
\captionof{table}{Receiver Characteristics}
\begin{tabulary}{\linewidth}{|L|L|}
\hline
\textbf{Characteristic} & \textbf{Definition} \\ \hline
\textbf{Sensitivity} & Minimum signal strength required for acceptable output \\ \hline
\textbf{Selectivity} & Ability to separate desired signal from adjacent signals \\ \hline
\textbf{Fidelity} & Accuracy in reproducing the original signal without distortion \\ \hline
\textbf{Image rejection} & Ability to reject image frequency interference \\ \hline
\textbf{SNR} & Ratio of desired signal to unwanted noise \\ \hline
\textbf{Stability} & Ability to maintain tuned frequency without drift \\ \hline
\end{tabulary}
\end{center}
\end{solutionbox}

\begin{mnemonicbox}
\mnemonic{SFIS-SS: Sensitivity, Fidelity, Image rejection, Selectivity, SNR, Stability}
\end{mnemonicbox}

\questionmarks{3(b)}{4}{Draw the block diagram of FM receiver. What is the use of Limiter in FM receiver.}

\begin{solutionbox}
\textbf{FM Receiver Block Diagram:}

\begin{center}
\begin{tikzpicture}[node distance=1.5cm, auto, >=latex]
    \node [gtu block] (ant) {Antenna};
    \node [gtu block, right=of ant] (rf) {RF Amplifier};
    \node [gtu block, right=of rf] (mix) {Mixer};
    \node [gtu block, below=of mix] (lo) {Local Oscillator};
    \node [gtu block, right=of mix] (if) {IF Amplifier};
    \node [gtu block, right=of if] (lim) {Limiter};
    \node [gtu block, below=of lim] (det) {FM Detector};
    \node [gtu block, left=of det] (audio) {Audio Amplifier};
    \node [gtu block, left=of audio] (spk) {Speaker};

    \draw [gtu arrow] (ant) -- (rf);
    \draw [gtu arrow] (rf) -- (mix);
    \draw [gtu arrow] (lo) -- (mix);
    \draw [gtu arrow] (mix) -- (if);
    \draw [gtu arrow] (if) -- (lim);
    \draw [gtu arrow] (lim) -- (det);
    \draw [gtu arrow] (det) -- (audio);
    \draw [gtu arrow] (audio) -- (spk);
\end{tikzpicture}
\captionof{figure}{FM Receiver Block Diagram}
\end{center}

\textbf{Use of Limiter in FM Receiver:}

\begin{itemize}
    \item \textbf{Primary function}: Removes amplitude variations (noise) from the FM signal.
    \item \textbf{Operation}: Clips the signal peaks to provide a constant amplitude output to the detector.
    \item \textbf{Benefits}: Eliminates AM noise/interference, improves SNR, and ensures the FM detector responds only to frequency changes.
\end{itemize}
\end{solutionbox}

\begin{mnemonicbox}
\mnemonic{CARE: Clips Amplitude, Removes noise, Ensures constant signal}
\end{mnemonicbox}

\questionmarks{3(c)}{7}{Draw and explain block diagram of super heterodyne receiver.}

\begin{solutionbox}
\textbf{Super Heterodyne Receiver:}

\begin{center}
\begin{tikzpicture}[node distance=2cm, auto, >=latex]
    % Top row
    \node [gtu block] (ant) {Antenna};
    \node [gtu block, right=of ant] (rf) {RF Amplifier};
    \node [gtu block, right=of rf] (mix) {Mixer};
    \node [gtu block, right=of mix] (if) {IF Amplifier};
    
    % Bottom row
    \node [gtu block, below=of mix] (lo) {Local Oscillator};
    \node [gtu block, below=of if] (det) {Detector};
    \node [gtu block, right=of det] (audio) {Audio Amp};
    \node [gtu block, right=of audio] (spk) {Speaker};
    \node [gtu block, below=of det] (agc) {AGC};

    % Connections
    \draw [gtu arrow] (ant) -- (rf);
    \draw [gtu arrow] (rf) -- (mix);
    \draw [gtu arrow] (lo) -- (mix);
    \draw [gtu arrow] (mix) -- (if);
    \draw [gtu arrow] (if) -- (det);
    \draw [gtu arrow] (det) -- (audio);
    \draw [gtu arrow] (audio) -- (spk);
    
    % AGC Feedback
    \draw [gtu arrow] (det) -- (agc);
    \draw [gtu arrow] (agc) -| (rf);
    \draw [gtu arrow] (agc) -| (if);
\end{tikzpicture}
\captionof{figure}{Super Heterodyne Receiver}
\end{center}

\textbf{Function of each block:}

\begin{itemize}
    \item \textbf{Antenna}: Captures RF signals.
    \item \textbf{RF Amplifier}: Amplifies weak signals, improves SNR and selectivity.
    \item \textbf{Mixer}: Mixes incoming RF with LO frequency to produce Intermediate Frequency (IF).
    \item \textbf{Local Oscillator}: Generates frequency $f_{LO} = f_{RF} + f_{IF}$.
    \item \textbf{IF Amplifier}: Provides major amplification and selectivity at fixed frequency (e.g., 455 kHz).
    \item \textbf{Detector}: Demodulates the signal to extract audio.
    \item \textbf{AGC}: Automatic Gain Control maintains constant output despite fading.
    \item \textbf{Audio Amplifier}: Boosts power to drive speaker.
\end{itemize}
\end{solutionbox}

\begin{mnemonicbox}
\mnemonic{ARLMIDAS: Antenna Receives, Local Mixes, IF Delivers, Audio Sounds}
\end{mnemonicbox}

\questionmarks{3(a) OR}{3}{Draw and explain block diagram for envelope detector.}

\begin{solutionbox}
\textbf{Envelope Detector:}

\begin{center}
\begin{tikzpicture}[american]
    \draw (0,2) to[short, o-] (1,2) to[D*, l=$D$] (3,2) to[short, -o] (4,2);
    \draw (3,2) to[C, l=$C$] (3,0);
    \draw (4,2) to[R, l=$R$] (4,0);
    \draw (0,0) to[short, o-o] (5,0);
    
    \draw (4,2) to[short, -o] (5,2);
    
    \node[left] at (0,1) {AM Input};
    \node[right] at (5,1) {Output};
\end{tikzpicture}
\captionof{figure}{Envelope Detector}
\end{center}

\textbf{Operation:}

\begin{enumerate}
    \item \textbf{Diode (D)}: Rectifies AM signal (allows only positive half-cycles).
    \item \textbf{Capacitor (C)}: Charges to peak of input, filters carrier frequency.
    \item \textbf{Resistor (R)}: Discharges capacitor, allowing it to follow the modulating signal envelope.
    \item \textbf{RC constant}: Selected such that $\frac{1}{f_c} \ll RC \ll \frac{1}{f_m}$.
\end{enumerate}
\end{solutionbox}

\begin{mnemonicbox}
\mnemonic{DRIVER: Diode Rectifies, RC Values Extract Envelope, Restores audio}
\end{mnemonicbox}

\questionmarks{3(b) OR}{4}{What is IF? Explain its importance in brief.}

\begin{solutionbox}
\textbf{Intermediate Frequency (IF):}

\textbf{Definition:} IF is a fixed frequency to which incoming RF signals are converted in superheterodyne receivers.

\textbf{Importance of IF:}

\begin{center}
\captionof{table}{Importance of IF}
\begin{tabulary}{\linewidth}{|L|L|}
\hline
\textbf{Aspect} & \textbf{Importance} \\ \hline
\textbf{Fixed Frequency} & Allows optimized amplification at one frequency \\ \hline
\textbf{Improved Selectivity} & Fixed-tuned filters provide better adjacent channel rejection \\ \hline
\textbf{Stable Gain} & Consistent amplification across entire tuning range \\ \hline
\textbf{Image Rejection} & Helps reject image frequency interference \\ \hline
\textbf{Simplified Tuning} & Only local oscillator needs to be tuned for different stations \\ \hline
\end{tabulary}
\end{center}

\textbf{Typical IF Values:}
\begin{itemize}
    \item AM receivers: 455 kHz
    \item FM receivers: 10.7 MHz
\end{itemize}
\end{solutionbox}

\begin{mnemonicbox}
\mnemonic{FIGS-ST: Fixed frequency, Improved selectivity, Gain stability, Simplified tuning}
\end{mnemonicbox}

\questionmarks{3(c) OR}{7}{Explain phase discriminator circuit for FM detection.}

\begin{solutionbox}
\textbf{Phase Discriminator (Foster-Seeley):}

\begin{center}
\begin{tikzpicture}[american, scale=0.9, transform shape]
    % Transformer
    \draw (0,2) node[transformer core, american voltages] (T) {};
    
    \draw (T.A1) to[short, -o] (-2, 2.75);
    \draw (T.A2) to[short, -o] (-2, 1.25);
    \node[left] at (-2,2) {FM Input};
    
    % Secondary Side
    \draw (T.B1) to[short] (2, 2.75) to[D*, l=$D_1$] (4, 2.75) -- (5, 2.75);
    \draw (T.B2) to[short] (2, 1.25) to[D*, l=$D_2$] (4, 1.25) -- (5, 1.25);
    
    % Center tap
    \coordinate (CT) at ($(T.B1)!0.5!(T.B2)$);
    \draw (CT) to[C, l=$C_c$] (1.5, -1) -- (-1.5, -1) to[short] (-1.5, 2.75); % Coupling cap to primary? Or standard connection
    % Actually foster seeley connects top of primary to CT via capacitor.
    % Simplified drawing:
    \draw (T.A1) to[short, *-] (0, 3.5) to[C, l=$C_c$] (CT |- 0,3.5) -- (CT); % Capacitor from Pri Top to Sec CT
    
    % Output circuit
    \draw (5, 2.75) to[R, l=$R_1$] (5, 2);
    \draw (5, 2) to[C, l=$C_1$] (5, 2.75);
    
    \draw (5, 1.25) to[R, l=$R_2$] (5, 2);
    \draw (5, 2) to[C, l=$C_2$] (5, 1.25);
    
    \draw (5,2) to[short, -o] (7, 2) node[right] {Output};
    \draw (5, 1.25) -- (5, 0.5) node[ground]{};

\end{tikzpicture}
\captionof{figure}{Phase Discriminator}
\end{center}

\textbf{Operation:}

\begin{enumerate}
    \item \textbf{Center-tapped transformer} and capacitor create phase difference dependent on frequency.
    \item \textbf{Resonance}: At $f_c$, phase shift is 90 degrees, $D_1$ and $D_2$ conduct equally, output is 0.
    \item \textbf{Off-Resonance}: At $f > f_c$ or $f < f_c$, phase shift changes, one diode conducts more.
    \item \textbf{Output}: Differential voltage proportional to frequency deviation.
\end{enumerate}

\textbf{Advantages:} Good linearity, reduced distortion.
\end{solutionbox}

\begin{mnemonicbox}
\mnemonic{PERFECT: Phase Ensures Rectification For Extracting Carrier Transitions}
\end{mnemonicbox}

\questionmarks{4(a)}{3}{Explain quantization process and its necessity.}

\begin{solutionbox}
\textbf{Quantization Process:}

Quantization matches the continuous amplitude range to a finite number of discrete levels.

\begin{enumerate}
    \item Sampling converts continuous-time to discrete-time.
    \item Amplitudes are divided into finite levels ($L = 2^n$).
    \item Each sample is assigned to the nearest level.
    \item Difference is quantization error.
\end{enumerate}

\textbf{Necessity:}

\begin{center}
\captionof{table}{Necessity of Quantization}
\begin{tabulary}{\linewidth}{|L|L|}
\hline
\textbf{Necessity} & \textbf{Explanation} \\ \hline
\textbf{Digital Processing} & Enables digital storage and manipulation \\ \hline
\textbf{Error Control} & Allows error detection and correction \\ \hline
\textbf{Noise Immunity} & Digital signals more resistant to noise \\ \hline
\textbf{Storage Efficiency} & More efficient than storing analog values \\ \hline
\end{tabulary}
\end{center}
\end{solutionbox}

\begin{mnemonicbox}
\mnemonic{DENSE: Digital conversion, Error control, Noise immunity, Storage, Efficient transmission}
\end{mnemonicbox}

\questionmarks{4(b)}{4}{Give difference between DM and ADM.}

\begin{solutionbox}
\textbf{Difference between DM and ADM:}

\begin{center}
\captionof{table}{DM vs ADM}
\begin{tabulary}{\linewidth}{|L|L|L|}
\hline
\textbf{Parameter} & \textbf{Delta Modulation (DM)} & \textbf{Adaptive Delta Modulation (ADM)} \\ \hline
\textbf{Step Size} & Fixed & Variable (adapts to signal) \\ \hline
\textbf{Slope Overload} & Common at steep signals & Reduced with adaptive step \\ \hline
\textbf{Granular Noise} & High for small signals & Reduced with smaller steps \\ \hline
\textbf{Complexity} & Simple & Moderate \\ \hline
\textbf{Bit Rate} & Higher for good quality & Lower for same quality \\ \hline
\end{tabulary}
\end{center}
\end{solutionbox}

\begin{mnemonicbox}
\mnemonic{SAVAGES: Step size, Adaptable, Variable tracking, Avoids overload, Granular noise reduction}
\end{mnemonicbox}

\questionmarks{4(c)}{7}{Draw \& explain block diagram of PCM system.}

\begin{solutionbox}
\textbf{PCM System Block Diagram:}

\begin{center}
\begin{tikzpicture}[node distance=1.5cm, auto, >=latex]
    % Transmitter
    \node [gtu block] (in) {Input};
    \node [gtu block, right=of in] (lpf) {Anti-aliasing Filter};
    \node [gtu block, right=of lpf] (sh) {Sample \& Hold};
    \node [gtu block, right=of sh] (quant) {Quantizer};
    \node [gtu block, below=of quant] (enc) {Encoder};
    \node [gtu block, left=of enc] (p2s) {Parallel to Serial};
    \node [right=0.5cm of p2s] (tx) {Tx};
    
    % Receiver
    \node [gtu block, below=of p2s] (s2p) {Serial to Parallel};
    \node [left=0.5cm of s2p] (rx) {Rx};
    \node [gtu block, right=of s2p] (dec) {Decoder};
    \node [gtu block, right=of dec] (recon) {Reconstruction Filter};
    \node [gtu block, right=of recon] (out) {Output};

    % Connections
    \draw [gtu arrow] (in) -- (lpf);
    \draw [gtu arrow] (lpf) -- (sh);
    \draw [gtu arrow] (sh) -- (quant);
    \draw [gtu arrow] (quant) -- (enc);
    \draw [gtu arrow] (enc) -- (p2s);
    
    \draw [dashed, ->] (p2s) -- (s2p) node[midway, right] {Channel};
    
    \draw [gtu arrow] (s2p) -- (dec);
    \draw [gtu arrow] (dec) -- (recon);
    \draw [gtu arrow] (recon) -- (out);
\end{tikzpicture}
\captionof{figure}{PCM System}
\end{center}

\textbf{PCM Transmitter:}
\begin{itemize}
    \item \textbf{Anti-aliasing Filter}: Limits bandwidth.
    \item \textbf{Sample \& Hold}: Discretizes time.
    \item \textbf{Quantizer}: Discretizes amplitude.
    \item \textbf{Encoder}: Converts levels to binary.
\end{itemize}

\textbf{PCM Receiver:}
\begin{itemize}
    \item \textbf{Decoder}: Converts binary back to amplitude levels.
    \item \textbf{Reconstruction Filter}: Recovers analog signal.
\end{itemize}
\end{solutionbox}

\begin{mnemonicbox}
\mnemonic{SAFE-PETS: Sample, Amplify, Filter, Encode, Pulse train, Extract, Transform, Smooth}
\end{mnemonicbox}

\questionmarks{4(a) OR}{3}{Define quantization. Explain non uniform quantization in brief.}

\begin{solutionbox}
\textbf{Quantization:} Process of converting continuous amplitude samples into discrete values.

\textbf{Non-uniform Quantization:}

\begin{itemize}
    \item Uses unequal step sizes.
    \item \textbf{Small steps} for small variance signals (improves SNR for weak signals).
    \item \textbf{Large steps} for large variance signals.
    \item Implemented using \textbf{Companding} (Compressing + Expanding).
\end{itemize}

\begin{center}
\begin{tikzpicture}
    \draw[->] (0,0) -- (4,0) node[right] {Input};
    \draw[->] (0,0) -- (0,4) node[above] {Output};
    
    % Companding curve (logarithmic)
    \draw[blue, thick] plot [domain=0:3.5, samples=100] (\x, {ln(\x+1)*1.5});
    \node at (2.5, 1.5) {Compressor Curve};
    
    % Linear line for comparison
    \draw[dashed] (0,0) -- (3.5, 3.5);
\end{tikzpicture}
\captionof{figure}{Non-uniform Quantization (Companding)}
\end{center}
\end{solutionbox}

\begin{mnemonicbox}
\mnemonic{CLASP: Compressed Levels, Adaptive Steps, Small steps for small signals, Perceptual matching}
\end{mnemonicbox}

\questionmarks{4(b) OR}{4}{Explain Adaptive delta modulation with its application.}

\begin{solutionbox}
\textbf{Adaptive Delta Modulation (ADM):}

\begin{center}
\begin{tikzpicture}[node distance=1.5cm, auto, >=latex]
    \node [gtu block] (comp) {Comparator};
    \node [left=1cm of comp] (in) {Input};
    \node [gtu block, right=of comp] (quant) {1-bit Quantizer};
    \node [right=1cm of quant] (out) {Output};
    
    \node [gtu block, below=of quant] (logic) {Step Size Logic};
    \node [gtu block, left=of logic] (int) {Integrator};
    
    \draw [gtu arrow] (in) -- (comp);
    \draw [gtu arrow] (comp) -- (quant);
    \draw [gtu arrow] (quant) -- (out);
    
    \draw [gtu arrow] (quant) -- (logic);
    \draw [gtu arrow] (logic) -- (int);
    \draw [gtu arrow] (int) -| node[pos=0.9]{-} (comp);
\end{tikzpicture}
\captionof{figure}{Adaptive Delta Modulation}
\end{center}

\textbf{Operation:}
\begin{itemize}
    \item Step size is NOT fixed.
    \item If sequence of bits is same (1111 or 0000), step size increases to prevent \textbf{Slope Overload}.
    \item If sequence alternates (1010), step size decreases to reduce \textbf{Granular Noise}.
\end{itemize}

\textbf{Applications:} Digital voice, audio compression.
\end{solutionbox}

\begin{mnemonicbox}
\mnemonic{ADAPT: Automatically Decides Appropriate Pulse Transitions}
\end{mnemonicbox}

\questionmarks{4(c) OR}{7}{What is sampling? Explain types of sampling in brief.}

\begin{solutionbox}
\textbf{Sampling:} Converting continuous-time signal to discrete-time.

\textbf{Types of Sampling:}

\begin{center}
\begin{tikzpicture}
    % Ideal
    \begin{scope}[xshift=0cm]
        \draw[->] (0,0) -- (3,0);
        \draw[->] (0,0) -- (0,2);
        \foreach \x in {0.5, 1.0, 1.5, 2.0, 2.5}
            \draw[thick, blue] (\x, 0) -- (\x, {1.5*sin(\x r*2)}); % Just schematic
        \node[below] at (1.5, -0.2) {Ideal (Impulses)};
    \end{scope}

    % Natural
    \begin{scope}[xshift=3.5cm]
        \draw[->] (0,0) -- (3,0);
        \foreach \x in {0.5, 1.0, 1.5, 2.0, 2.5}
        {
            \fill[blue!20] (\x-0.1, 0) rectangle (\x+0.1, {1.5}); % Placeholder height
            \draw[blue] (\x-0.1, 0) -- (\x-0.1, 1.5) -- (\x+0.1, 1.6) -- (\x+0.1, 0); % Sloped top
        }
        \node[below] at (1.5, -0.2) {Natural (Pulse)};
    \end{scope}

    % Flat-top
    \begin{scope}[xshift=7cm]
        \draw[->] (0,0) -- (3,0);
        \foreach \x in {0.5, 1.0, 1.5, 2.0, 2.5}
            \draw[fill=blue!20] (\x-0.1, 0) rectangle (\x+0.1, {1.5});
        \node[below] at (1.5, -0.2) {Flat-top (Hold)};
    \end{scope}
\end{tikzpicture}
\captionof{figure}{Types of Sampling}
\end{center}

\begin{itemize}
    \item \textbf{Ideal}: Instantaneous impulses (Theoretical).
    \item \textbf{Natural}: Pulse follows signal shape during width.
    \item \textbf{Flat-top}: Amplitude held constant during pulse width (Sample \& Hold).
\end{itemize}
\end{solutionbox}

\begin{mnemonicbox}
\mnemonic{INFS: Ideal (impulses), Natural (follows signal), Flat-top (constant), Sufficient rate}
\end{mnemonicbox}

\questionmarks{5(a)}{3}{Define bit rate and baud rate.}

\begin{solutionbox}
\textbf{Bit Rate and Baud Rate:}

\begin{center}
\captionof{table}{Bit Rate vs Baud Rate}
\begin{tabulary}{\linewidth}{|L|L|L|}
\hline
\textbf{Parameter} & \textbf{Definition} & \textbf{Formula} \\ \hline
\textbf{Bit Rate ($R_b$)} & Number of bits per second & $R_b = \text{Baud} \times \log_2 M$ \\ \hline
\textbf{Baud Rate} & Number of signal symbols per second & Baud = $f_s$ \\ \hline
\end{tabulary}
\end{center}

\begin{itemize}
    \item For Binary (M=2), Bit Rate = Baud Rate.
    \item For QPSK (M=4), Bit Rate = 2 $\times$ Baud Rate.
\end{itemize}
\end{solutionbox}

\begin{mnemonicbox}
\mnemonic{BBSM: Bits per second, Baud for Symbols, Modulation determines relationship}
\end{mnemonicbox}

\questionmarks{5(b)}{4}{Explain working of DPCM.}

\begin{solutionbox}
\textbf{Differential Pulse Code Modulation (DPCM):}

\begin{center}
\begin{tikzpicture}[node distance=1.5cm, auto, >=latex]
    \node [gtu block] (sum) {Sum};
    \node [left=1cm of sum] (in) {Input};
    \node [gtu block, right=of sum] (quant) {Quantizer};
    \node [gtu block, right=of quant] (enc) {Encoder};
    \node [right=1cm of enc] (out) {Output};
    
    \node [gtu block, below=of quant] (pred) {Predictor};
    \node [gtu block, left=of pred] (sum2) {+};
    
    \draw [gtu arrow] (in) -- (sum);
    \draw [gtu arrow] (sum) -- (quant);
    \draw [gtu arrow] (quant) -- (enc);
    \draw [gtu arrow] (enc) -- (out);
    
    \draw [gtu arrow] (quant) -- (sum2);
    \draw [gtu arrow] (sum2) -- (pred);
    \draw [gtu arrow] (pred) -| node[pos=0.9]{-} (sum);
    \draw [gtu arrow] (pred) -- (sum2);
\end{tikzpicture}
\captionof{figure}{DPCM Transmitter}
\end{center}

\textbf{Working:} Encodes the \textbf{difference} between current sample and predicted value. Improves efficiency as difference has smaller dynamic range than original signal.
\end{solutionbox}

\begin{mnemonicbox}
\mnemonic{DEEP: Difference Encoded, Efficient Prediction, Exploits correlation, Preserves quality}
\end{mnemonicbox}

\questionmarks{5(c)}{7}{The binary data 1011001 is to be transmitted... Draw all the waveforms.}

\begin{solutionbox}
\textbf{Line Coding Waveforms (Data: 1 0 1 1 0 0 1):}

\begin{center}
\begin{tikzpicture}[xscale=1.5, yscale=0.8]
    \draw[help lines] (0,-4) grid (7,4);
    
    % Labels
    \node[left] at (0, 3.5) {Unipolar NRZ};
    \node[left] at (0, 2) {Polar NRZ};
    \node[left] at (0, 0.5) {AMI};
    \node[left] at (0, -1) {Manchester};
    
    % Bit Boundaries
    \foreach \x in {0,...,7} \draw[dashed] (\x,-2) -- (\x, 4);
    \foreach \x/\v in {0/1, 1/0, 2/1, 3/1, 4/0, 5/0, 6/1} \node[above] at (\x+0.5, 4) {\v};

    % Unipolar NRZ: 1=High, 0=Low
    \draw[blue, thick] (0,3) -- (1,3) -- (1,4) -- (2,4) -- (2,3) -- (3,3) -- (3,4) -- (4,4) -- (4,3) -- (6,3) -- (6,4) -- (7,4);
    % Wait, mapping: 1->High? "1011001"?
    % Data: 1 0 1 1 0 0 1
    % Unipolar NRZ: High, Low, High, High, Low, Low, High
    % My manual draw above: (0,3) is Low? Let's restart drawing properly.
    % 1: (0,4)--(1,4)
    % 0: (1,3)--(2,3) assuming 3 is Low
    % 1: (2,4)--(3,4)
    % 1: (3,4)--(4,4)
    % 0: (4,3)--(5,3)
    % 0: (5,3)--(6,3)
    % 1: (6,4)--(7,4)
    \draw[blue, thick] (0,4) -- (1,4) -- (1,3) -- (2,3) -- (2,4) -- (3,4) -- (4,4) -- (4,3) -- (6,3) -- (6,4) -- (7,4);

    % Polar NRZ: 1=+V, 0=-V
    \draw[red, thick] (0,2.5) -- (1,2.5) -- (1,1.5) -- (2,1.5) -- (2,2.5) -- (4,2.5) -- (4,1.5) -- (6,1.5) -- (6,2.5) -- (7,2.5);
    
    % AMI: 1=Alt, 0=Zero. 1(+), 0, 1(-), 1(+), 0, 0, 1(-)
    \draw[green!60!black, thick] (0,1) -- (1,1) -- (1,0.5) -- (2,0.5) -- (2,0) -- (3,0) -- (3,1) -- (4,1) -- (4,0.5) -- (6,0.5) -- (6,0) -- (7,0);
    % wait, y=0.5 is zero level.
    % 1(+): (0,1)--(1,1)
    % 0(0): (1,0.5)--(2,0.5)
    % 1(-): (2,0)--(3,0)
    % 1(+): (3,1)--(4,1)
    % 0(0): (4,0.5)--(6,0.5)
    % 1(-): (6,0)--(7,0)
    
    % Manchester: 1=H->L, 0=L->H
    % 1: (0, -0.5) -- (0.5, -0.5) -- (0.5, -1.5) -- (1, -1.5)
    % 0: (1, -1.5) -- (1.5, -1.5) -- (1.5, -0.5) -- (2, -0.5)
    % 1: (2, -0.5) -- (2.5, -0.5) -- (2.5, -1.5) -- (3, -1.5)
    % ...
    \draw[orange, thick] 
    (0, -0.5) -- (0.5, -0.5) -- (0.5, -1.5) -- (1, -1.5) -- % 1
    (1, -1.5) -- (1.5, -1.5) -- (1.5, -0.5) -- (2, -0.5) -- % 0
    (2, -0.5) -- (2.5, -0.5) -- (2.5, -1.5) -- (3, -1.5) -- % 1
    (3, -0.5) -- (3.5, -0.5) -- (3.5, -1.5) -- (4, -1.5) -- % 1
    (4, -1.5) -- (4.5, -1.5) -- (4.5, -0.5) -- (5, -0.5) -- % 0
    (5, -1.5) -- (5.5, -1.5) -- (5.5, -0.5) -- (6, -0.5) -- % 0
    (6, -0.5) -- (6.5, -0.5) -- (6.5, -1.5) -- (7, -1.5);   % 1

\end{tikzpicture}
\captionof{figure}{Line Coding Waveforms}
\end{center}
\end{solutionbox}

\begin{mnemonicbox}
\mnemonic{UPAM: Unipolar, Polar, AMI, Manchester encoding options}
\end{mnemonicbox}

\questionmarks{5(a) OR}{3}{Compare RZ and NRZ coding with example.}

\begin{solutionbox}
\textbf{Comparison of RZ and NRZ:}

\begin{center}
\captionof{table}{RZ vs NRZ}
\begin{tabulary}{\linewidth}{|L|L|L|}
\hline
\textbf{Parameter} & \textbf{RZ} & \textbf{NRZ} \\ \hline
\textbf{Signal levels} & Returns to zero in bit & Maintains level \\ \hline
\textbf{Bandwidth} & Higher & Lower \\ \hline
\textbf{Sync} & Better (more transitions) & Poorer \\ \hline
\textbf{DC component} & Present & More significant \\ \hline
\end{tabulary}
\end{center}
\end{solutionbox}

\begin{mnemonicbox}
\mnemonic{BPSIDC: Bandwidth, Power, Synchronization, Implementation, DC component}
\end{mnemonicbox}

\questionmarks{5(b) OR}{4}{Explain delta modulation in brief.}

\begin{solutionbox}
\textbf{Delta Modulation (DM):}

\textbf{Principle}: Encodes only the difference (delta) between current sample and previous sample using 1 bit.
\begin{itemize}
    \item If Input > Pred: Output 1 (Count Up)
    \item If Input < Pred: Output 0 (Count Down)
\end{itemize}

\textbf{Limitations}: Slope Overload and Granular Noise.
\end{solutionbox}

\begin{mnemonicbox}
\mnemonic{SIDE: Single-bit, Integrates Differences, Encodes changes}
\end{mnemonicbox}

\questionmarks{5(c) OR}{7}{Explain PCM-TDM system.}

\begin{solutionbox}
\textbf{PCM-TDM System:}

\begin{center}
\begin{tikzpicture}[node distance=1.5cm, auto, >=latex, every node/.style={transform shape, scale=0.8}]
    % Transmitter
    \node [gtu block] (mux) {Multiplexer};
    \node [left=1cm of mux] (in) {Inputs (Ch 1..N)};
    \node [gtu block, right=of mux] (enc) {PCM Encoder};
    \node [gtu block, right=of enc] (tx) {Tx};
    
    % Receiver
    \node [gtu block, below=of tx] (rx) {Rx};
    \node [gtu block, left=of rx] (dec) {PCM Decoder};
    \node [gtu block, left=of dec] (demux) {Demultiplexer};
    \node [left=1cm of demux] (out) {Outputs};
    
    \draw [gtu arrow] (in) -- (mux);
    \draw [gtu arrow] (mux) -- (enc);
    \draw [gtu arrow] (enc) -- (tx);
    
    \draw [dashed] (tx) -- (rx);
    
    \draw [gtu arrow] (rx) -- (dec);
    \draw [gtu arrow] (dec) -- (demux);
    \draw [gtu arrow] (demux) -- (out);
\end{tikzpicture}
\captionof{figure}{PCM-TDM Block Diagram}
\end{center}

\textbf{Operation:} Samples from multiple channels are interleaved in time (TDM) and then PCM encoded using a shared encoder.

\textbf{Hierarchy:} Filtering $\rightarrow$ Multiplexing $\rightarrow$ Quantizing $\rightarrow$ Framing $\rightarrow$ Transmission.
\end{solutionbox}

\begin{mnemonicbox}
\mnemonic{MOST-FDR: Multiplex, Quantize, Sample, Transmit, Frame, Demultiplex, Reconstruct}
\end{mnemonicbox}

\end{document}
