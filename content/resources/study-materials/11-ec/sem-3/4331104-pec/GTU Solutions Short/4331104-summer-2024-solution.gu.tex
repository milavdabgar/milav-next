\documentclass{article}

% content/resources/templates/preamble.tex
\usepackage[margin=0.6in]{geometry}
\author{Milav Dabgar}
\usepackage{amsmath,amssymb,amsthm}
\usepackage{booktabs}
\usepackage{multirow}
\usepackage{xcolor}
\usepackage{tcolorbox}
\tcbuselibrary{breakable,skins}
\usepackage[colorlinks=true,linkcolor=blue]{hyperref}
\usepackage{titlesec}
\usepackage{enumitem}
\usepackage{tikz}
\usepackage{pgfplots}
\usepackage{circuitikz}
\usepackage[version=4]{mhchem}
\usepackage{longtable}
\usepackage{array}
\usepackage{float}
\usepackage{caption}
\usepackage{listings}

\lstset{
  basicstyle=\small\ttfamily,
  breaklines=true,
  breakatwhitespace=false,
  postbreak=\mbox{\textcolor{red}{$\hookrightarrow$}\space},
  float=false,
  numbers=left,
  numberstyle=\tiny\color{gray},
  numbersep=10pt,
  xleftmargin=2em,
  keywordstyle=\color{blue},
  commentstyle=\color{green!60!black},
  stringstyle=\color{purple},
  backgroundcolor=\color{gray!5},
  showstringspaces=false,
  tabsize=2,
  captionpos=b,
  keepspaces=true,
  columns=flexible
}

\pgfplotsset{compat=1.18}
\usetikzlibrary{shapes,arrows,positioning,calc,patterns,decorations.pathmorphing,decorations.markings,arrows.meta}

% Color scheme
\definecolor{headcolor}{RGB}{0,102,204}
\definecolor{keycolor}{RGB}{220,20,60}
\definecolor{solutioncolor}{RGB}{34,139,34}
\definecolor{mnemoniccolor}{RGB}{148,0,211}
\definecolor{codecolor}{RGB}{0,0,100}

% Spacing
\setlength{\parskip}{3pt}
\setlist[itemize]{nosep}
\setlist[enumerate]{nosep}

% Title formatting
\titleformat{\section}{\Large\bfseries\color{headcolor}}{\thesection}{1em}{}
\titleformat{\subsection}{\large\bfseries\color{headcolor}}{\thesubsection}{1em}{}

% Pandoc tightlist compatibility
\providecommand{\tightlist}{%
  \setlength{\itemsep}{0pt}\setlength{\parskip}{0pt}}

% Pandoc longtable compatibility
\newcounter{none}
\def\thenone{}


% content/resources/templates/gujarati-boxes.tex
\usepackage{fontspec}
\usepackage{polyglossia}

% Set Gujarati as main language (document is primarily in Gujarati)
% Note: gloss-gujarati.ldf doesn't exist in polyglossia, but it will use hyphenation patterns
\setdefaultlanguage{gujarati}
\setotherlanguage{english}

% Configure Gujarati font properly
% Use Language=Default to prevent polyglossia from trying to add language-specific features
% that don't exist for Gujarati, which causes "empty feature" warnings
\newfontfamily\gujaratifont[Script=Gujarati,AutoFakeBold=2.5,AutoFakeSlant=0.3]{Noto Sans Gujarati}
\setmainfont[Script=Gujarati,AutoFakeBold=2.5,AutoFakeSlant=0.3]{Noto Sans Gujarati}
% Use Noto Sans Gujarati for monospace to support Gujarati in text
\setmonofont[Scale=0.9]{Noto Sans Gujarati}

% Configure English to use the same font
\newfontfamily\englishfont[Script=Gujarati,AutoFakeBold=2.5,AutoFakeSlant=0.3]{Noto Sans Gujarati}

% Translations for polyglossia
\gappto\captionsgujarati{
  \renewcommand{\tablename}{કોષ્ટક}
  \renewcommand{\figurename}{આકૃતિ}
}

% Helper for TikZ nodes to ensure Gujarati font
\newcommand{\gu}[1]{{\gujaratifont #1}}

% Custom environments
\newtcolorbox{solutionbox}{
    breakable,
    enhanced,
    colback=solutioncolor!5!white,
    colframe=solutioncolor!75!black,
    fonttitle=\bfseries,
    title=જવાબ
}

\newtcolorbox{solutionboxnobreak}{
 colback=solutioncolor!5!white,
 colframe=solutioncolor!75!black,
 fonttitle=\bfseries,
 title=જવાબ
}

\newtcolorbox{keyformula}{
 breakable,
 enhanced,
 colback=keycolor!5!white,
 colframe=keycolor!75!black,
 fonttitle=\bfseries,
 title=રાસાયણિક સમીકરણ/સૂત્ર
}

\newtcolorbox{mnemonicbox}{
 breakable,
 enhanced,
 colback=mnemoniccolor!5!white,
 colframe=mnemoniccolor!75!black,
 fonttitle=\bfseries,
 title=મેમરી ટ્રીક
}


% Custom commands for GTU solutions
% This file defines semantic commands for consistent formatting

% Question command with automatic formatting
\newcommand{\question}[2]{%
  \section*{Question #1}%
  \textbf{#2}%
}

% OR question variant
\newcommand{\questionor}[2]{%
  \section*{Question #1 OR}%
  \textbf{#2}%
}

% Proper table environment with caption
\newenvironment{answertable}[1]{%
  \begin{table}[htbp]
  \centering
  \caption{#1}
}{%
  \end{table}
}

% Proper figure environment for diagrams
\newenvironment{answerdiagram}[1]{%
  \begin{figure}[htbp]
  \centering
  \caption{#1}
}{%
  \end{figure}
}

% Semantic markup for key terms
\newcommand{\keyword}[1]{\textbf{#1}}
\newcommand{\code}[1]{\texttt{#1}}
\newcommand{\classname}[1]{\texttt{#1}}
\newcommand{\methodname}[1]{\texttt{#1}}

% Proper quotation marks
\newcommand{\mnemonic}[1]{``#1''}


\title{ઇલેક્ટ્રોનિક કોમ્યુનિકેશનના સિદ્ધાંતો (4331104) - ઉનાળુ 2024 સોલ્યુશન}
\date{June 14, 2024}

\begin{document}
\maketitle

\questionmarks{1}{a}{3}
\textbf{કોમ્યુનિકેશન સિસ્ટમનો બ્લોક ડાયાગ્રામ દોરો અને સમજાવો.}

\begin{solutionbox}
    \textbf{જવાબ}:

    \begin{center}
    \begin{tikzpicture}[auto, node distance=2.5cm, >=latex]
        \node [gtu block] (source) {માહિતી સ્રોત};
        \node [gtu block, right of=source, node distance=3.5cm] (tx) {ટ્રાન્સમીટર};
        \node [gtu block, right of=tx, node distance=3.5cm] (channel) {ચેનલ/માધ્યમ};
        \node [gtu block, right of=channel, node distance=3.5cm] (rx) {રિસીવર};
        \node [gtu block, right of=rx, node distance=3.5cm] (dest) {ગંતવ્ય};
        \node [gtu block, below of=channel, node distance=2cm] (noise) {નોઈઝ સ્રોત};

        \draw [gtu arrow] (source) -- (tx);
        \draw [gtu arrow] (tx) -- (channel);
        \draw [gtu arrow] (channel) -- (rx);
        \draw [gtu arrow] (rx) -- (dest);
        \draw [gtu arrow] (noise) -- (channel);
    \end{tikzpicture}
    \end{center}

    \begin{itemize}
        \item \textbf{માહિતી સ્રોત}: સંદેશા સિગ્નલ ઉત્પન્ન કરે છે (અવાજ, વિડિઓ, ડેટા).
        \item \textbf{ટ્રાન્સમીટર}: સંદેશાને પ્રસારણ માટે યોગ્ય સ્વરૂપમાં રૂપાંતરિત કરે છે.
        \item \textbf{ચેનલ}: માધ્યમ જેના દ્વારા સિગ્નલ પ્રવાસ કરે છે (તાર, ફાઇબર, હવા).
        \item \textbf{રિસીવર}: મળેલા સિગ્નલમાંથી મૂળ સંદેશો બહાર કાઢે છે.
        \item \textbf{ગંતવ્ય}: અંતિમ-વપરાશકર્તા જે માહિતી પ્રાપ્ત કરે છે.
    \end{itemize}

    \begin{mnemonicbox}
    "માહિતી પ્રવાસ સાવધાનીથી ગંતવ્ય પહોંચે"
    \end{mnemonicbox}
\end{solutionbox}

\questionmarks{1}{b}{4}
\textbf{EM વેવ સ્પેક્ટ્રમના ઉપયોગો સમજાવો.}

\begin{solutionbox}
    \textbf{જવાબ}:

    \begin{tabulary}{\textwidth}{|L|L|L|}
    \hline
    \textbf{ફ્રિક્વન્સી બેન્ડ} & \textbf{ફ્રિક્વન્સી રેન્જ} & \textbf{ઉપયોગો} \\
    \hline
    રેડિયો વેવ્સ & 3 kHz - 300 MHz & AM/FM પ્રસારણ, દરિયાઈ સંચાર \\
    \hline
    માઇક્રોવેવ્સ & 300 MHz - 300 GHz & રડાર, સેટેલાઇટ સંચાર, માઇક્રોવેવ ઓવન \\
    \hline
    ઇન્ફ્રારેડ & 300 GHz - 400 THz & રિમોટ કંટ્રોલ, થર્મલ ઇમેજિંગ, ઓપ્ટિકલ ફાઇબર \\
    \hline
    દૃશ્યમાન પ્રકાશ & 400 THz - 800 THz & ફાઇબર ઓપ્ટિક સંચાર, ફોટોગ્રાફી \\
    \hline
    અલ્ટ્રાવાયોલેટ & 800 THz - 30 PHz & જંતુનાશક, પ્રમાણીકરણ, પાણી શુદ્ધિકરણ \\
    \hline
    એક્સ-રે & 30 PHz - 30 EHz & મેડિકલ ઇમેજિંગ, સુરક્ષા સ્કેનિંગ, સામગ્રી વિશ્લેષણ \\
    \hline
    ગામા રે & >30 EHz & કેન્સર સારવાર, ખાદ્ય જંતુનાશક, ઔદ્યોગિક નિરીક્ષણ \\
    \hline
    \end{tabulary}

    \begin{mnemonicbox}
    "રેડિયો માઇક્રો અદૃશ્ય દૃશ્ય અલ્ટ્રા એક્સ ગામા"
    \end{mnemonicbox}
\end{solutionbox}

\questionmarks{1}{c}{7}
\textbf{બાહ્ય અને આંતરિક અવાજ જણાવો અને સમજાવો.}

\begin{solutionbox}
    \textbf{જવાબ}:

    \begin{tabulary}{\textwidth}{|L|L|L|}
    \hline
    \textbf{પ્રકાર} & \textbf{બાહ્ય અવાજ} & \textbf{આંતરિક અવાજ} \\
    \hline
    \textbf{સ્રોત} & સંચાર વ્યવસ્થાની બહાર & ઇલેક્ટ્રોનિક ઘટકોની અંદર \\
    \hline
    \textbf{પ્રકારો} & વાતાવરણીય, અવકાશ, ઔદ્યોગિક, માનવ-નિર્મિત & થર્મલ, શોટ, ટ્રાન્ઝિટ-ટાઇમ, ફ્લિકર \\
    \hline
    \textbf{નિયંત્રણ} & શીલ્ડિંગ, ફિલ્ટરિંગ દ્વારા ઘટાડી શકાય છે & સારા ઘટકો, કૂલિંગ દ્વારા ઘટાડી શકાય છે \\
    \hline
    \textbf{ઉદાહરણો} & વીજળી, સૂર્ય વિકિરણ, મોટર સ્પાર્કિંગ & અવરોધકોમાં ઇલેક્ટ્રોન મૂવમેન્ટ, સેમિકન્ડક્ટર્સ \\
    \hline
    \textbf{પ્રકૃતિ} & સામાન્ય રીતે અનિયમિત, બદલાતી & વધુ સુસંગત અને માપી શકાય તેવી \\
    \hline
    \end{tabulary}

    \textbf{આકૃતિ:}

    \begin{center}
    \begin{tikzpicture}[gtu tree, level distance=2cm]
    \node [gtu block] {સંચારમાં અવાજ}
        child {node [gtu block] {બાહ્ય અવાજ}
            child {node [gtu state] {વાતાવરણીય}}
            child {node [gtu state] {અવકાશ}}
            child {node [gtu state] {ઔદ્યોગિક}}
            child {node [gtu state] {માનવ-નિર્મિત}}
        }
        child {node [gtu block] {આંતરિક અવાજ}
            child {node [gtu state] {થર્મલ}}
            child {node [gtu state] {શોટ}}
            child {node [gtu state] {ટ્રાન્ઝિટ-ટાઇમ}}
            child {node [gtu state] {ફ્લિકર}}
        };
    \end{tikzpicture}
    \end{center}

    \begin{mnemonicbox}
    "બાહ્ય વાતાવરણ આવે; આંતરિક ઘટકો જન્માવે"
    \end{mnemonicbox}
\end{solutionbox}

\questionmarks{1}{c}{7}
\textbf{સુપરહીટરોડાઇન AM રિસીવરનો બ્લોક ડાયાગ્રામ દોરો અને સમજાવો.}

\begin{solutionbox}
    \textbf{જવાબ}:

    \begin{center}
    \begin{tikzpicture}[auto, node distance=2cm, >=latex]
        % Nodes
        \node [gtu block] (rf) {RF એમ્પ્લિફાયર};
        \node [left of=rf, node distance=2.5cm] (ant) {એન્ટેના};
        \node [gtu block, right of=rf, node distance=3cm] (mixer) {મિક્સર};
        \node [gtu block, below of=mixer, node distance=2cm] (lo) {લોકલ ઓસિલેટર};
        \node [gtu block, right of=mixer, node distance=3cm] (if) {IF એમ્પ્લિફાયર};
        \node [gtu block, right of=if, node distance=3cm] (det) {ડિટેક્ટર};
        \node [gtu block, right of=det, node distance=3cm] (af) {AF એમ્પ્લિફાયર};
        \node [right of=af, node distance=2.5cm] (spk) {સ્પીકર};
        \node [gtu block, below of=if, node distance=2.5cm] (agc) {AGC};

        % Connections
        \draw [gtu arrow] (ant) -- (rf);
        \draw [gtu arrow] (rf) -- (mixer);
        \draw [gtu arrow] (lo) -- (mixer);
        \draw [gtu arrow] (mixer) -- (if);
        \draw [gtu arrow] (if) -- (det);
        \draw [gtu arrow] (det) -- (af);
        \draw [gtu arrow] (af) -- (spk);
        
        % AGC paths
        \draw [gtu arrow] (det.south) |- (agc.east);
        \draw [gtu arrow] (agc.west) -| (rf.south);
        \draw [gtu arrow] (agc.north) -- (if.south);

    \end{tikzpicture}
    \end{center}

    \begin{tabulary}{\textwidth}{|L|L|}
    \hline
    \textbf{બ્લોક} & \textbf{કાર્ય} \\
    \hline
    \textbf{RF એમ્પ્લિફાયર} & નબળા રેડિયો સિગ્નલને વધારે છે અને પસંદગી પૂરી પાડે છે \\
    \hline
    \textbf{લોકલ ઓસિલેટર} & આવનારા સિગ્નલ સાથે મિક્સિંગ માટે ફ્રિક્વન્સી ઉત્પન્ન કરે છે \\
    \hline
    \textbf{મિક્સર} & RF અને લોકલ ઓસિલેટર સિગ્નલોને સંયોજિત કરીને IF ઉત્પન્ન કરે છે \\
    \hline
    \textbf{IF એમ્પ્લિફાયર} & ફિક્સ્ડ ઇન્ટરમીડિયેટ ફ્રિક્વન્સી (455 kHz) પર સિગ્નલને વધારે છે \\
    \hline
    \textbf{ડિટેક્ટર} & મોડ્યુલેટેડ કેરિયરમાંથી ઓડિયો બહાર કાઢે છે (ડિમોડ્યુલેશન) \\
    \hline
    \textbf{AF એમ્પ્લિફાયર} & સ્પીકર ચલાવવા માટે ઓડિયો સિગ્નલને વધારે છે \\
    \hline
    \textbf{AGC} & ઓટોમેટિક ગેઇન કંટ્રોલ - સતત આઉટપુટ લેવલ જાળવે છે \\
    \hline
    \end{tabulary}

    \begin{mnemonicbox}
    "રેડિયો લય મિશ્રણ માધ્યમ ઉત્પાદન આવાજ"
    \end{mnemonicbox}
\end{solutionbox}

\questionmarks{2}{a}{3}
\textbf{મોડ્યુલેશન વ્યાખ્યાયિત કરો. મોડ્યુલેશનના પ્રકારો જણાવો.}

\begin{solutionbox}
    \textbf{જવાબ}:

    \textbf{મોડ્યુલેશન}: માહિતી ધરાવતા મોડ્યુલેટિંગ સિગ્નલ સાથે ઉચ્ચ-ફ્રિક્વન્સી કેરિયર સિગ્નલની એક અથવા વધુ લાક્ષણિકતાઓને બદલવાની પ્રક્રિયા.

    \textbf{મોડ્યુલેશનના પ્રકારો:}

    \begin{center}
    \begin{tikzpicture}[gtu tree, level distance=1.5cm, sibling distance=2cm]
    \node [gtu block] {મોડ્યુલેશન}
        child {node [gtu block, align=center] {એનાલોગ\\મોડ્યુલેશન}
            child {node [gtu state] {AM}}
            child {node [gtu state] {FM}}
            child {node [gtu state] {PM}}
        }
        child {node [gtu block, align=center] {ડિજિટલ\\મોડ્યુલેશન}
            child {node [gtu state] {ASK}}
            child {node [gtu state] {FSK}}
            child {node [gtu state] {PSK}}
        }
        child {node [gtu block, align=center] {પલ્સ\\મોડ્યુલેશન}
            child {node [gtu state] {PAM}}
            child {node [gtu state] {PWM}}
            child {node [gtu state] {PPM}}
            child {node [gtu state] {PCM}}
        };
    \end{tikzpicture}
    \end{center}

    \begin{mnemonicbox}
    "મોડ્યુલેશન આવૃત્તિ, એમ્પલિટ્યુડ, ફેઝ બદલે છે"
    \end{mnemonicbox}
\end{solutionbox}

\questionmarks{2}{b}{4}
\textbf{વ્યાખ્યાયિત કરો: સિગ્નલ ટુ નોઈઝ રેશિયો અને નોઈઝ ફિગર.}

\begin{solutionbox}
    \textbf{જવાબ}:

    \begin{tabulary}{\textwidth}{|L|L|L|L|L|}
    \hline
    \textbf{પેરામીટર} & \textbf{વ્યાખ્યા} & \textbf{ફોર્મ્યુલા} & \textbf{એકમ} & \textbf{મહત્વ} \\
    \hline
    \textbf{સિગ્નલ ટુ નોઈઝ રેશિયો (SNR)} & સિગ્નલ પાવર અને નોઈઝ પાવરનો ગુણોત્તર & $SNR = \frac{P_{signal}}{P_{noise}}$ & dB માં વ્યક્ત & ઉચ્ચ મૂલ્ય સારી સિગ્નલ ક્વોલિટી દર્શાવે છે \\
    \hline
    \textbf{નોઈઝ ફિગર (NF)} & સિસ્ટમમાંથી પસાર થવાથી SNR ના ઘટાડાનું માપ & $NF = \frac{SNR_{input}}{SNR_{output}}$ & dB માં વ્યક્ત & નીચું મૂલ્ય સારી કામગીરી દર્શાવે છે \\
    \hline
    \end{tabulary}

    \begin{mnemonicbox}
    "SNR સિગ્નલ શક્તિ બતાવે; નોઈઝ ફિગર ખામી શોધે"
    \end{mnemonicbox}
\end{solutionbox}

\questionmarks{2}{c}{7}
\textbf{PAM, PWM અને PPM તકનીકોની તુલના કરો.}

\begin{solutionbox}
    \textbf{જવાબ}:

    \begin{tabulary}{\textwidth}{|L|L|L|L|}
    \hline
    \textbf{પેરામીટર} & \textbf{PAM} & \textbf{PWM} & \textbf{PPM} \\
    \hline
    \textbf{પૂરું નામ} & પલ્સ એમ્પ્લિટ્યુડ મોડ્યુલેશન & પલ્સ વિડ્થ મોડ્યુલેશન & પલ્સ પોઝિશન મોડ્યુલેશન \\
    \hline
    \textbf{મોડ્યુલેટેડ પેરામીટર} & પલ્સની એમ્પ્લિટ્યુડ & પલ્સની પહોળાઈ/અવધિ & પલ્સની સ્થિતિ/સમય \\
    \hline
    \textbf{નોઈઝ ઇમ્યુનિટી} & નબળી & સારી & ઉત્તમ \\
    \hline
    \textbf{બેન્ડવિડ્થ} & ઓછી & મધ્યમ & ઉચ્ચ \\
    \hline
    \textbf{સર્કિટ જટિલતા} & સરળ & મધ્યમ & જટિલ \\
    \hline
    \textbf{પાવર એફિશિયન્સી} & નબળી & સારી & ઉત્તમ \\
    \hline
    \textbf{ઉપયોગો} & સરળ ડેટા સેમ્પલિંગ & મોટર કંટ્રોલ, પાવર નિયમન & સચોટ ટાઇમિંગ, ઓપ્ટિકલ સંચાર \\
    \hline
    \end{tabulary}

    \textbf{આકૃતિ:}

    \begin{center}
    \begin{tikzpicture}[scale=0.8]
        % PAM
        \node [left] at (0, 7) {PAM};
        \draw [->] (0, 6) -- (10, 6) node[right] {$t$};
        \foreach \x in {1,2,3,4,5,6,7,8,9} {
            \pgfmathsetmacro{\yval}{6 + 0.5 + 0.5*sin(\x*50)}
            \draw [thick, blue] (\x, 6) -- (\x, \yval);
        }
        
        % PWM
        \node [left] at (0, 4) {PWM};
        \draw [->] (0, 3) -- (10, 3) node[right] {$t$};
        \foreach \x in {1,2,3,4,5,6,7,8,9} {
            \pgfmathsetmacro{\width}{0.2 + 0.15*sin(\x*50)}
            \draw [thick, red, fill=red!20] (\x-0.2, 3) rectangle (\x+\width, 4);
        }
            
        % PPM
        \node [left] at (0, 1) {PPM};
        \draw [->] (0, 0) -- (10, 0) node[right] {$t$};
        \foreach \x in {1,2,3,4,5,6,7,8,9} {
            \pgfmathsetmacro{\shift}{0.3*sin(\x*50)}
            \draw [thick, orange] (\x+\shift, 0) -- (\x+\shift, 1);
        }
    \end{tikzpicture}
    \end{center}

    \begin{mnemonicbox}
    "એમ્પલિટ્યુડ ઊંચાઈ, પહોળાઈ લંબાઈ, પોઝિશન સમય બદલે"
    \end{mnemonicbox}
\end{solutionbox}

\questionmarks{2}{a}{3}
\textbf{બીટ, સિમ્બોલ અને બોડ રેટ વચ્ચે તફાવત કરો.}

\begin{solutionbox}
    \textbf{જવાબ}:

    \begin{tabulary}{\textwidth}{|L|L|L|L|}
    \hline
    \textbf{પેરામીટર} & \textbf{બીટ} & \textbf{સિમ્બોલ} & \textbf{બોડ રેટ} \\
    \hline
    \textbf{વ્યાખ્યા} & બાઇનરી અંક (0 અથવા 1) & બિટ્સનો સમૂહ & પ્રતિ સેકન્ડ પ્રસારિત સિમ્બોલ્સની સંખ્યા \\
    \hline
    \textbf{એકમ} & કોઈ એકમ નથી & કોઈ એકમ નથી & સિમ્બોલ પ્રતિ સેકન્ડ (બોડ) \\
    \hline
    \textbf{સંબંધ} & ડિજિટલ માહિતીનો મૂળભૂત એકમ & એકાધિક બિટ્સ એક સિમ્બોલ બનાવે છે & બોડ રેટ $\times$ બિટ્સ પ્રતિ સિમ્બોલ = બિટ રેટ \\
    \hline
    \textbf{ઉદાહરણ} & 0, 1 & 4-QAM માં, દરેક સિમ્બોલ 2 બિટ્સ રજૂ કરે છે & 1200 બોડ એટલે દર સેકન્ડે 1200 સિમ્બોલ \\
    \hline
    \end{tabulary}

    \begin{mnemonicbox}
    "બિટ સિમ્બોલ બનાવે, બોડ ગતિ બતાવે"
    \end{mnemonicbox}
\end{solutionbox}

\questionmarks{2}{b}{4}
\textbf{DSB કરતાં SSB ના ફાયદા અને ગેરફાયદા જણાવો.}

\begin{solutionbox}
    \textbf{જવાબ}:

    \begin{tabulary}{\textwidth}{|L|L|}
    \hline
    \textbf{SSB ના DSB કરતાં ફાયદા} & \textbf{SSB ના DSB કરતાં ગેરફાયદા} \\
    \hline
    \textbf{બેન્ડવિડ્થ}: માત્ર અર્ધી બેન્ડવિડ્થની જરૂર પડે છે & \textbf{સર્કિટ જટિલતા}: વધુ જટિલ મોડ્યુલેશન અને ડિમોડ્યુલેશન \\
    \hline
    \textbf{પાવર એફિશિયન્સી}: માત્ર એક સાઇડબેન્ડ પ્રસારિત કરે છે, પાવર બચાવે છે & \textbf{રિસીવર ડિઝાઇન}: ચોક્કસ ફ્રિક્વન્સી સિન્ક્રોનાઇઝેશનની જરૂર પડે છે \\
    \hline
    \textbf{ઓછું ફેડિંગ}: સિલેક્ટિવ ફેડિંગ પ્રભાવોમાં ઘટાડો & \textbf{લો ફ્રિક્વન્સી લોસ}: નીચી ફ્રિક્વન્સી ઘટકો ગુમાવી શકે છે \\
    \hline
    \textbf{ઓછું ઇન્ટરફેરન્સ}: એડજેસન્ટ ચેનલ ઇન્ટરફેરન્સમાં ઘટાડો & \textbf{ખર્ચ}: વધુ ખર્ચાળ અમલીકરણ \\
    \hline
    \end{tabulary}

    \begin{mnemonicbox}
    "SSB બેન્ડવિડ્થ પાવર બચાવે, પણ જટિલ હાર્ડવેર માંગે"
    \end{mnemonicbox}
\end{solutionbox}

\questionmarks{2}{c}{7}
\textbf{એમ્પલિટ્યુડ મોડ્યુલેશન (AM) અને ફ્રિક્વન્સી મોડ્યુલેશન (FM) ની તુલના કરો.}

\begin{solutionbox}
    \textbf{જવાબ}:

    \begin{tabulary}{\textwidth}{|L|L|L|}
    \hline
    \textbf{પેરામીટર} & \textbf{AM} & \textbf{FM} \\
    \hline
    \textbf{મોડ્યુલેટેડ પેરામીટર} & કેરિયરની એમ્પલિટ્યુડ & કેરિયરની ફ્રિક્વન્સી \\
    \hline
    \textbf{બેન્ડવિડ્થ} & સાંકડી ($2 \times f_m$) & વિશાળ ($2 \times (f_m + \Delta f)$) \\
    \hline
    \textbf{નોઈઝ ઇમ્યુનિટી} & નબળી & ઉત્તમ \\
    \hline
    \textbf{પાવર એફિશિયન્સી} & નબળી (કેરિયરમાં મોટાભાગનો પાવર) & સારી \\
    \hline
    \textbf{સર્કિટ જટિલતા} & સરળ & જટિલ \\
    \hline
    \textbf{ક્વોલિટી} & નીચી & ઉચ્ચ \\
    \hline
    \textbf{ઉપયોગો} & બ્રોડકાસ્ટિંગ (MW), એરક્રાફ્ટ કોમ્યુનિકેશન & FM રેડિયો, TV સાઉન્ડ, મોબાઇલ કોમ્યુનિકેશન \\
    \hline
    \end{tabulary}

    \textbf{આકૃતિ:}

    \begin{center}
    \begin{tikzpicture}[scale=0.8]
        % Carrier
        \begin{scope}[yshift=4cm]
            \draw[->] (0,0) -- (10,0) node[right] {$t$};
            \draw[->] (0,-1.5) -- (0,1.5) node[above] {$v_c(t)$};
            \draw[domain=0:9.5, samples=200, smooth] plot (\x, {1*sin(15*\x r)});
            \node at (5,-2) {કેરિયર};
        \end{scope}

        % AM
        \begin{scope}[yshift=0cm]
            \draw[->] (0,0) -- (10,0) node[right] {$t$};
            \draw[->] (0,-1.5) -- (0,1.5) node[above] {AM};
            % Envelope
            \draw[dashed, blue] plot[domain=0:9.5, samples=100] (\x, {1 + 0.6*sin(1.5*\x r)});
            \draw[dashed, blue] plot[domain=0:9.5, samples=100] (\x, {-(1 + 0.6*sin(1.5*\x r))});
            % Signal
            \draw[domain=0:9.5, samples=300, smooth] plot (\x, {(1 + 0.6*sin(1.5*\x r)) * sin(15*\x r)});
        \end{scope}

        % FM
        \begin{scope}[yshift=-4cm]
            \draw[->] (0,0) -- (10,0) node[right] {$t$};
            \draw[->] (0,-1.5) -- (0,1.5) node[above] {FM};
            \draw[domain=0:9.5, samples=300, smooth] plot (\x, {sin((15*\x + 5*sin(1.5*\x r)) r)});
        \end{scope}
    \end{tikzpicture}
    \end{center}

    \begin{mnemonicbox}
    "AM શક્તિ બદલે, FM આવૃત્તિ હલાવે"
    \end{mnemonicbox}
\end{solutionbox}

\questionmarks{3}{a}{3}
\textbf{AM રિસીવરને FM રિસીવર સાથે સરખાવો.}

\begin{solutionbox}
    \textbf{જવાબ}:

    \begin{tabulary}{\textwidth}{|L|L|L|}
    \hline
    \textbf{પેરામીટર} & \textbf{AM રિસીવર} & \textbf{FM રિસીવર} \\
    \hline
    \textbf{IF ફ્રિક્વન્સી} & 455 kHz & 10.7 MHz \\
    \hline
    \textbf{ડિટેક્ટર} & એન્વેલોપ ડિટેક્ટર & ડિસ્ક્રિમિનેટર/રેશિયો ડિટેક્ટર/PLL \\
    \hline
    \textbf{બેન્ડવિડ્થ} & સાંકડી ($\pm$5 kHz) & વિશાળ ($\pm$75 kHz) \\
    \hline
    \textbf{સ્પેશિયલ સર્કિટ} & સરળ & લિમિટર, ડી-એમ્ફેસિસ \\
    \hline
    \textbf{જટિલતા} & સરળ & જટિલ \\
    \hline
    \end{tabulary}

    \begin{mnemonicbox}
    "AM લઘુ બેન્ડવિડ્થ સરળ; FM વિશાળ બેન્ડવિડ્થ જટિલ"
    \end{mnemonicbox}
\end{solutionbox}

\questionmarks{3}{b}{4}
\textbf{સેમ્પલિંગ વ્યાખ્યાયિત કરો? સંક્ષિપ્તમાં સેમ્પલિંગના પ્રકારો સમજાવો.}

\begin{solutionbox}
    \textbf{જવાબ}:

    \textbf{સેમ્પલિંગ}: સતત-સમય સિગ્નલને નિયમિત અંતરાલે સેમ્પલ લઈને વિવેકાધીન-સમય સિગ્નલમાં રૂપાંતરિત કરવાની પ્રક્રિયા.

    \begin{tabulary}{\textwidth}{|L|L|L|}
    \hline
    \textbf{સેમ્પલિંગના પ્રકાર} & \textbf{વર્ણન} & \textbf{લાક્ષણિકતાઓ} \\
    \hline
    \textbf{આદર્શ સેમ્પલિંગ} & સિગ્નલના તાત્કાલિક સેમ્પલ & સંપૂર્ણ પરંતુ સૈદ્ધાંતિક, આવેગ ફંક્શનનો ઉપયોગ કરે છે \\
    \hline
    \textbf{નેચરલ સેમ્પલિંગ} & સિગ્નલને ટૂંકા સમયગાળા માટે સેમ્પલ કરવામાં આવે છે & પલ્સના ટોચ મૂળ સિગ્નલને અનુસરે છે \\
    \hline
    \textbf{ફ્લેટ-ટોપ સેમ્પલિંગ} & આગલા સેમ્પલ સુધી સેમ્પલ સ્થિર રાખવામાં આવે છે & સીડી અનુમાન બનાવે છે, અમલમાં મૂકવા માટે સરળ \\
    \hline
    \end{tabulary}

    \textbf{આકૃતિ:}

    \begin{center}
    \begin{tikzpicture}[scale=0.8]
        \draw[->] (0,0) -- (10,0) node[right] {$t$};
        \draw[thick, gray, opacity=0.5] plot[domain=0:9.5, samples=100] (\x, {1 + 0.5*sin(1.5*\x r)});
        
        % Natural Sampling
        \foreach \x in {1,2,3,4,5,6,7,8,9} {
           \draw [thick, blue] (\x-0.1, 0) -- (\x-0.1, {1 + 0.5*sin(1.5*(\x-0.1) r)}) -- (\x+0.1, {1 + 0.5*sin(1.5*(\x+0.1) r)}) -- (\x+0.1, 0) -- cycle;
        }
        \node at (5,-1) {નેચરલ સેમ્પલિંગ};
    \end{tikzpicture}
    \end{center}

    \begin{mnemonicbox}
    "આદર્શ ક્ષણો લે, નેચરલ આકાર અનુસરે, ફ્લેટ સ્થિર રહે"
    \end{mnemonicbox}
\end{solutionbox}

\questionmarks{3}{c}{7}
\textbf{FM રિસીવરનો બ્લોક ડાયાગ્રામ દોરો અને સમજાવો. FM રિસીવરમાં લિમિટરનો ઉપયોગ શું છે?}

\begin{solutionbox}
    \textbf{જવાબ}:

    \begin{center}
    \begin{tikzpicture}[auto, node distance=2cm, >=latex]
        % Nodes
        \node [gtu block] (rf) {RF એમ્પ્લિફાયર};
        \node [left of=rf, node distance=2.5cm] (ant) {એન્ટેના};
        \node [gtu block, right of=rf, node distance=3cm] (mixer) {મિક્સર};
        \node [gtu block, below of=mixer, node distance=2cm] (lo) {લોકલ ઓસિલેટર};
        \node [gtu block, right of=mixer, node distance=2.5cm] (if) {IF એમ્પ્લિફાયર};
        \node [gtu block, right of=if, node distance=2.5cm] (lim) {લિમિટર};
        \node [gtu block, right of=lim, node distance=2.5cm] (disc) {ડિસ્ક્રિમિનેટર};
        \node [gtu block, below of=lim, node distance=2cm] (deemp) {ડી-એમ્ફેસિસ};
        \node [gtu block, left of=deemp, node distance=3cm] (af) {AF એમ્પ્લિફાયર};
        \node [left of=af, node distance=2.5cm] (spk) {સ્પીકર};

        % Connections
        \draw [gtu arrow] (ant) -- (rf);
        \draw [gtu arrow] (rf) -- (mixer);
        \draw [gtu arrow] (lo) -- (mixer);
        \draw [gtu arrow] (mixer) -- (if);
        \draw [gtu arrow] (if) -- (lim);
        \draw [gtu arrow] (lim) -- (disc);
        \draw [gtu arrow] (disc.south) -- ++(0,-0.5) -| (deemp.north);
        \draw [gtu arrow] (deemp) -- (af);
        \draw [gtu arrow] (af) -- (spk);

    \end{tikzpicture}
    \end{center}

    \begin{tabulary}{\textwidth}{|L|L|}
    \hline
    \textbf{બ્લોક} & \textbf{કાર્ય} \\
    \hline
    \textbf{RF એમ્પ્લિફાયર} & નબળા RF સિગ્નલને વધારે છે અને પસંદગી પૂરી પાડે છે \\
    \hline
    \textbf{મિક્સર/લોકલ ઓસિલેટર} & RF ને IF માં રૂપાંતરિત કરે છે (10.7 MHz) \\
    \hline
    \textbf{IF એમ્પ્લિફાયર} & ફિક્સ્ડ ફ્રિક્વન્સી પર ગેઇન અને પસંદગી પ્રદાન કરે છે \\
    \hline
    \textbf{લિમિટર} & એમ્પલિટ્યુડ વેરિએશન્સ દૂર કરે છે, ફ્રિક્વન્સી વેરિએશન્સ જાળવે છે \\
    \hline
    \textbf{ડિસ્ક્રિમિનેટર} & ફ્રિક્વન્સી વેરિએશન્સને એમ્પલિટ્યુડ વેરિએશન્સમાં રૂપાંતરિત કરે છે \\
    \hline
    \textbf{ડી-એમ્ફેસિસ} & ઉચ્ચ-ફ્રિક્વન્સી નોઈઝને ઘટાડે છે \\
    \hline
    \textbf{AF એમ્પ્લિફાયર} & સ્પીકર માટે મેળવેલા ઓડિયોને વધારે છે \\
    \hline
    \end{tabulary}

    \textbf{લિમિટરનું કાર્ય}: ડીમોડ્યુલેશન પહેલાં FM સિગ્નલમાંથી એમ્પલિટ્યુડ વેરિએશન્સને દૂર કરે છે જેથી નોઈઝ ઇમ્યુનિટી સુનિશ્ચિત થાય, કારણ કે FM માં માહિતી ફ્રિક્વન્સી વેરિએશન્સમાં સમાયેલી છે, એમ્પલિટ્યુડમાં નહીં.

    \begin{mnemonicbox}
    "રેડિયો મિક્સર વધારે આવૃત્તિ; લિમિટર ફરક ઓળખી અવાજ કાઢે"
    \end{mnemonicbox}
\end{solutionbox}

\questionmarks{3}{a}{3}
\textbf{સિંગલ સાઇડ બેન્ડ (SSB) ટ્રાન્સમિશનના ખ્યાલનું વર્ણન કરો.}

\begin{solutionbox}
    \textbf{જવાબ}:

    \textbf{સિંગલ સાઇડબેન્ડ (SSB) ટ્રાન્સમિશન}: એક તકનીક જેમાં કેરિયર અને અન્ય સાઇડબેન્ડને દબાવીને માત્ર એક સાઇડબેન્ડ (ઉપર અથવા નીચે) પ્રસારિત કરવામાં આવે છે.

    \begin{center}
    \begin{tikzpicture}[scale=0.8]
        \draw[->] (0,0) -- (8,0) node[right] {$f$};
        \draw[->] (0,0) -- (0,4) node[above] {એમ્પલિટ્યુડ};
        
        % Carrier (Suppressed)
        \draw[dashed, gray] (4,0) -- (4,3);
        \node[below] at (4,0) {$f_c$};
        
        % LSB (Suppressed)
        \draw[dashed, gray] (2.5,0) rectangle (3.8, 2);
        \node at (3.15, 1) {LSB};
        \node[below] at (2.5,0) {$f_c-f_m$};
        
        % USB (Transmitted)
        \draw[thick, blue, fill=blue!20] (4.2,0) rectangle (5.5, 2);
        \node at (4.85, 1) {USB};
        \node[below] at (5.5,0) {$f_c+f_m$};
        
        \node at (4, -1.5) {SSB સ્પેક્ટ્રમ (USB પ્રસારિત)};
    \end{tikzpicture}
    \end{center}

    \begin{itemize}
        \item \textbf{બેન્ડવિડ્થ}: માત્ર અર્ધી બેન્ડવિડ્થની જરૂર પડે છે ($f_c \pm f_m$).
        \item \textbf{પાવર એફિશિયન્સી}: વધુ કાર્યક્ષમ કારણ કે પાવર એક સાઇડબેન્ડમાં કેન્દ્રિત થાય છે.
        \item \textbf{પ્રકારો}: USB (અપર સાઇડબેન્ડ) અને LSB (લોઅર સાઇડબેન્ડ).
    \end{itemize}

    \begin{mnemonicbox}
    "SSB સ્પેક્ટ્રમ બેન્ડવિડ્થ બચાવે"
    \end{mnemonicbox}
\end{solutionbox}

\questionmarks{3}{b}{4}
\textbf{પ્રી-એમ્ફેસિસ અને ડી-એમ્ફેસિસ સર્કિટ સમજાવો.}

\begin{solutionbox}
    \textbf{જવાબ}:

    \begin{tabulary}{\textwidth}{|L|L|L|}
    \hline
    \textbf{પેરામીટર} & \textbf{પ્રી-એમ્ફેસિસ} & \textbf{ડી-એમ્ફેસિસ} \\
    \hline
    \textbf{સ્થાન} & ટ્રાન્સમીટર & રિસીવર \\
    \hline
    \textbf{સર્કિટ પ્રકાર} & હાઈ-પાસ RC નેટવર્ક & લો-પાસ RC નેટવર્ક \\
    \hline
    \textbf{કાર્ય} & પ્રસારણ પહેલાં ઉચ્ચ ફ્રિક્વન્સીઓને વધારે છે & રિસેપ્શન પછી ઉચ્ચ ફ્રિક્વન્સીઓને ઘટાડે છે \\
    \hline
    \textbf{હેતુ} & ઉચ્ચ ફ્રિક્વન્સીઓ માટે SNR સુધારે છે & મૂળ ફ્રિક્વન્સી રિસ્પોન્સ પુનઃસ્થાપિત કરે છે \\
    \hline
    \end{tabulary}

    \textbf{સર્કિટ ડાયાગ્રામ:}

    \begin{center}
    \begin{circuitikz}[scale=0.8]
        % Pre-emphasis
        \draw (0,4) node[left] {પ્રી-એમ્ફેસિસ} to[short, o-] (1,4); 
        \draw (1,4) to[R, l=$R$] (3,4);
        \draw (1,4) -- (1, 5.5) to[C, l=$C$] (3,5.5) -- (3,4);
        \draw (3,4) -- (4,4) to[R, l=$R_L$] (4,2) node[ground] {};
        \draw (3,4) to[short, -o] (5,4) node[right] {$V_{out}$};

        % De-emphasis
        \draw (7,4) node[left] {ડી-એમ્ફેસિસ} to[short, o-] (8,4) to[R, l=$R$] (10,4) -- (11,4) to[short, -o] (12,4) node[right] {$V_{out}$};
        \draw (10,4) to[C, l=$C$] (10,2) node[ground] {};
    \end{circuitikz}
    \end{center}

    \begin{mnemonicbox}
    "પ્રી ઊંચા ધક્કા મારે, ડી ઊંચા નીચે લાવે"
    \end{mnemonicbox}
\end{solutionbox}

\questionmarks{3}{c}{7}
\textbf{ફેઝ લોક લૂપ ટેકનિકનો ઉપયોગ કરીને FM સિગ્નલનું જનરેશન સમજાવો.}

\begin{solutionbox}
    \textbf{જવાબ}:

    \begin{center}
    \begin{tikzpicture}[auto, node distance=2.5cm, >=latex]
        \node [gtu block] (pd) {ફેઝ ડિટેક્ટર};
        \node [gtu block, right of=pd, node distance=3.5cm] (lf) {લૂપ ફિલ્ટર};
        \node [gtu block, right of=lf, node distance=3.5cm] (vco) {VCO};
        \node [gtu block, below of=pd, node distance=2cm] (ref) {રેફરન્સ ઓસિલેટર};
        \node [above of=lf, node distance=1.5cm] (input) {મોડ્યુલેટિંગ સિગ્નલ};

        \draw [gtu arrow] (ref) -- (pd);
        \draw [gtu arrow] (pd) -- (lf);
        \draw [gtu arrow] (lf) -- (vco);
        \draw [gtu arrow] (vco.south) -- ++(0,-1) -| (pd.south);
        \draw [gtu arrow] (vco.east) -- ++(1,0) node[right] {FM આઉટપુટ};
        \draw [gtu arrow] (input) -- (lf);
    \end{tikzpicture}
    \end{center}

    \begin{tabulary}{\textwidth}{|L|L|}
    \hline
    \textbf{ઘટક} & \textbf{કાર્ય} \\
    \hline
    \textbf{ફેઝ ડિટેક્ટર} & રેફરન્સ અને VCO સિગ્નલ્સની તુલના કરે છે, એરર વોલ્ટેજ ઉત્પન્ન કરે છે \\
    \hline
    \textbf{લૂપ ફિલ્ટર} & એરર વોલ્ટેજને ફિલ્ટર કરે છે અને મોડ્યુલેટિંગ સિગ્નલ સાથે જોડે છે \\
    \hline
    \textbf{VCO} & કંટ્રોલ વોલ્ટેજના આધારે ફ્રિક્વન્સી ઉત્પન્ન કરે છે \\
    \hline
    \textbf{રેફરન્સ ઓસિલેટર} & સ્થિર રેફરન્સ ફ્રિક્વન્સી પૂરી પાડે છે \\
    \hline
    \end{tabulary}

    \textbf{કાર્ય પ્રક્રિયા:}
    \begin{enumerate}
        \item મોડ્યુલેટિંગ સિગ્નલ લૂપ ફિલ્ટરમાં લાગુ કરવામાં આવે છે.
        \item VCO ફ્રિક્વન્સી મોડ્યુલેટિંગ સિગ્નલના પ્રમાણમાં શિફ્ટ થાય છે.
        \item ફેઝ ડિટેક્ટર એરર સિગ્નલ ઉત્પન્ન કરે છે.
        \item લૂપ ફ્રિક્વન્સી મોડ્યુલેશનની મંજૂરી આપતી વખતે લોક જાળવે છે.
        \item VCO નો આઉટપુટ FM સિગ્નલ છે.
    \end{enumerate}

    \begin{mnemonicbox}
    "ફેઝ લોક કરે, વોલ્ટેજ નિયંત્રિત કરે, ફ્રિક્વન્સી મોડ્યુલેટ કરે"
    \end{mnemonicbox}
\end{solutionbox}

\questionmarks{4}{a}{3}
\textbf{ક્વોન્ટાઇઝેશન પ્રક્રિયા અને તેનું મહત્વ સમજાવો.}

\begin{solutionbox}
    \textbf{જવાબ}:

    \textbf{ક્વોન્ટાઇઝેશન}: એનાલોગ-ટુ-ડિજિટલ રૂપાંતરણમાં સતત એમ્પલિટ્યુડ મૂલ્યોને વિવેકાધીન સ્તરના મર્યાદિત સેટમાં મેપિંગ કરવાની પ્રક્રિયા.

    \begin{tabulary}{\textwidth}{|L|L|}
    \hline
    \textbf{પાસું} & \textbf{વર્ણન} \\
    \hline
    \textbf{પ્રક્રિયા} & એમ્પલિટ્યુડ રેન્જને ફિક્સ્ડ લેવલમાં વિભાજિત કરવી અને ડિજિટલ મૂલ્યો સોંપવા \\
    \hline
    \textbf{પ્રકારો} & યુનિફોર્મ (સમાન સ્ટેપ્સ) અને નોન-યુનિફોર્મ (વેરિયેબલ સ્ટેપ્સ) \\
    \hline
    \textbf{એરર} & વાસ્તવિક અને ક્વોન્ટાઇઝ્ડ મૂલ્ય વચ્ચેનો તફાવત (ક્વોન્ટાઇઝેશન નોઈઝ) \\
    \hline
    \end{tabulary}

    \textbf{મહત્વ}:
    \begin{itemize}
        \item એનાલોગ સિગ્નલ્સના ડિજિટલ રજૂઆતને સક્ષમ કરે છે.
        \item ડિજિટલ સિગ્નલની રિઝોલ્યુશન અને ચોકસાઈ નક્કી કરે છે.
        \item ડિજિટલ સિસ્ટમમાં સિગ્નલ-ટુ-નોઈઝ રેશિયોને અસર કરે છે.
    \end{itemize}

    \begin{mnemonicbox}
    "ક્વોન્ટાઇઝેશન એનાલોગથી ડિજિટલ બનાવે"
    \end{mnemonicbox}
\end{solutionbox}

\questionmarks{4}{b}{4}
\textbf{રેડિયો રિસીવરની વિવિધ લાક્ષણિકતાઓ સમજાવો.}

\begin{solutionbox}
    \textbf{જવાબ}:

    \begin{tabulary}{\textwidth}{|L|L|L|}
    \hline
    \textbf{લાક્ષણિકતા} & \textbf{વ્યાખ્યા} & \textbf{મહત્વ} \\
    \hline
    \textbf{સેન્સિટિવિટી} & નબળા સિગ્નલ્સને પ્રાપ્ત કરવાની ક્ષમતા & રિસેપ્શન રેન્જ નક્કી કરે છે \\
    \hline
    \textbf{સિલેક્ટિવિટી} & અડીને આવેલા ચેનલ્સને અલગ કરવાની ક્ષમતા & ઇન્ટરફેરન્સ અટકાવે છે \\
    \hline
    \textbf{ફિડેલિટી} & પુનરુત્પાદનની ચોકસાઈ & સાઉન્ડ ક્વોલિટી નક્કી કરે છે \\
    \hline
    \textbf{ઇમેજ રિજેક્શન} & ઇમેજ ફ્રિક્વન્સીને નકારવાની ક્ષમતા & અનિચ્છનીય રિસેપ્શન અટકાવે છે \\
    \hline
    \end{tabulary}

    \textbf{આકૃતિ:}

    \begin{center}
    \begin{tikzpicture}[gtu tree]
    \node [gtu block] {રેડિયો રિસીવર લાક્ષણિકતાઓ}
        child {node [gtu state] {સેન્સિટિવિટી\\($\mu$V)}}
        child {node [gtu state] {સિલેક્ટિવિટી\\(Q ફેક્ટર)}}
        child {node [gtu state] {ફિડેલિટી\\(ફ્રિક રિસ્પોન્સ)}}
        child {node [gtu state] {ઇમેજ રિજેક્શન\\(ઇમેજ રેશિયો)}};
    \end{tikzpicture}
    \end{center}

    \begin{mnemonicbox}
    "સંવેદનશીલ પસંદગી શુદ્ધતા પ્રતિમા"
    \end{mnemonicbox}
\end{solutionbox}

\questionmarks{4}{c}{7}
\textbf{PCM ટ્રાન્સમીટર અને રિસીવરનો બ્લોક ડાયાગ્રામ દોરો અને સમજાવો.}

\begin{solutionbox}
    \textbf{જવાબ}:

    \textbf{PCM ટ્રાન્સમીટર:}
    \begin{center}
    \begin{tikzpicture}[auto, node distance=2.2cm, >=latex]
        \node [gtu block] (aa) {એન્ટી-એલિયાસિંગ ફિલ્ટર};
        \node [left of=aa, node distance=3cm] (in) {ઇનપુટ સિગ્નલ};
        \node [gtu block, right of=aa] (sh) {સેમ્પલ એન્ડ હોલ્ડ};
        \node [gtu block, right of=sh] (quant) {ક્વોન્ટાઇઝર};
        \node [gtu block, below of=quant, node distance=2cm] (enc) {એન્કોડર};
        \node [gtu block, left of=enc] (lc) {લાઇન કોડર};
        \node [left of=lc, node distance=3cm] (out) {ટ્રાન્સમિશન ચેનલ};

        \draw [gtu arrow] (in) -- (aa);
        \draw [gtu arrow] (aa) -- (sh);
        \draw [gtu arrow] (sh) -- (quant);
        \draw [gtu arrow] (quant) -- (enc);
        \draw [gtu arrow] (enc) -- (lc);
        \draw [gtu arrow] (lc) -- (out);
    \end{tikzpicture}
    \end{center}

    \textbf{PCM રિસીવર:}
    \begin{center}
    \begin{tikzpicture}[auto, node distance=2.2cm, >=latex]
        \node [gtu block] (ld) {લાઇન ડિકોડર};
        \node [left of=ld, node distance=3cm] (in) {પ્રાપ્ત સિગ્નલ};
        \node [gtu block, right of=ld] (reg) {રિજનરેટર};
        \node [gtu block, right of=reg] (dec) {ડિકોડર};
        \node [gtu block, below of=dec, node distance=2cm] (recon) {રિકન્સ્ટ્રક્શન ફિલ્ટર};
        \node [left of=recon, node distance=3cm] (out) {આઉટપુટ સિગ્નલ};

        \draw [gtu arrow] (in) -- (ld);
        \draw [gtu arrow] (ld) -- (reg);
        \draw [gtu arrow] (reg) -- (dec);
        \draw [gtu arrow] (dec) -- (recon);
        \draw [gtu arrow] (recon) -- (out);
    \end{tikzpicture}
    \end{center}

    \begin{tabulary}{\textwidth}{|L|L|}
    \hline
    \textbf{બ્લોક} & \textbf{કાર્ય} \\
    \hline
    \textbf{એન્ટી-એલિયાસિંગ ફિલ્ટર} & એલિયાસિંગને રોકવા માટે ઇનપુટ બેન્ડવિડ્થને મર્યાદિત કરે છે \\
    \hline
    \textbf{સેમ્પલ એન્ડ હોલ્ડ} & સતત સિગ્નલને વિવેકાધીન-સમય સેમ્પલમાં રૂપાંતરિત કરે છે \\
    \hline
    \textbf{ક્વોન્ટાઇઝર} & સેમ્પલ એમ્પલિટ્યુડને વિવેકાધીન સ્તરોમાં રૂપાંતરિત કરે છે \\
    \hline
    \textbf{એન્કોડર} & ક્વોન્ટાઇઝ્ડ મૂલ્યોને બાઇનરી કોડમાં રૂપાંતરિત કરે છે \\
    \hline
    \textbf{લાઇન કોડર} & પ્રસારણ માટે બાઇનરી ડેટા ફોર્મેટ કરે છે \\
    \hline
    \textbf{ડિકોડર} & બાઇનરી કોડને પાછા ક્વોન્ટાઇઝ્ડ મૂલ્યોમાં રૂપાંતરિત કરે છે \\
    \hline
    \textbf{રિકન્સ્ટ્રક્શન ફિલ્ટર} & મૂળ સિગ્નલ પુનઃપ્રાપ્ત કરવા માટે સ્ટેપ્ડ આઉટપુટને સરળ બનાવે છે \\
    \hline
    \end{tabulary}

    \begin{mnemonicbox}
    "સેમ્પલ, ક્વોન્ટાઇઝ, એનકોડ, પ્રસારણ; ડિકોડ, પુનઃસર્જન, આઉટપુટ"
    \end{mnemonicbox}
\end{solutionbox}

\questionmarks{4}{a}{3}
\textbf{નેચરલ અને ફ્લેટ ટોપ સેમ્પલિંગની સરખામણી કરો.}

\begin{solutionbox}
    \textbf{જવાબ}:

    \begin{tabulary}{\textwidth}{|L|L|L|}
    \hline
    \textbf{પેરામીટર} & \textbf{નેચરલ સેમ્પલિંગ} & \textbf{ફ્લેટ-ટોપ સેમ્પલિંગ} \\
    \hline
    \textbf{આકાર} & પલ્સની ટોચ ઇનપુટ સિગ્નલને અનુસરે છે & સેમ્પલિંગ અંતરાલ દરમિયાન સ્થિર એમ્પલિટ્યુડ \\
    \hline
    \textbf{અમલીકરણ} & વધુ મુશ્કેલ (એનાલોગ સ્વિચ) & સરળ (સેમ્પલ એન્ડ હોલ્ડ સર્કિટ) \\
    \hline
    \textbf{સ્પેક્ટ્રમ} & ઓછા હાર્મોનિક્સ & વધુ હાર્મોનિક્સ \\
    \hline
    \textbf{પુનઃસર્જન} & સરળ, વધુ ચોક્કસ & વિકૃતિ માટે વળતરની જરૂર છે \\
    \hline
    \end{tabulary}

    \textbf{આકૃતિ:}

    \begin{center}
    \begin{tikzpicture}[scale=0.8]
        \draw[thick, gray, opacity=0.5] plot[domain=0:8, samples=100] (\x, {1.5 + 0.8*sin(\x*1.5 r)});
        
        % Natural
        \foreach \x in {1,2,3,4,5,6,7} {
            \draw[thick, blue] (\x-0.1, 0) -- (\x-0.1, {1.5 + 0.8*sin((\x-0.1)*1.5 r)}) 
                -- (\x+0.1, {1.5 + 0.8*sin((\x+0.1)*1.5 r)}) -- (\x+0.1, 0) -- cycle;
        }
        \node at (4, -0.5) {નેચરલ};

        % Flat-top
        \begin{scope}[yshift=-4cm]
            \draw[thick, gray, opacity=0.5] plot[domain=0:8, samples=100] (\x, {1.5 + 0.8*sin(\x*1.5 r)});
            \foreach \x in {1,2,3,4,5,6,7} {
                \draw[thick, red] (\x-0.1, 0) -- (\x-0.1, {1.5 + 0.8*sin(\x*1.5 r)}) 
                    -- (\x+0.1, {1.5 + 0.8*sin(\x*1.5 r)}) -- (\x+0.1, 0) -- cycle;
            }
            \node at (4, -0.5) {ફ્લેટ-ટોપ};
        \end{scope}
    \end{tikzpicture}
    \end{center}

    \begin{mnemonicbox}
    "નેચરલ અનુસરે, ફ્લેટ ઠરે"
    \end{mnemonicbox}
\end{solutionbox}

\questionmarks{4}{b}{4}
\textbf{ડાયોડ ડિટેક્ટર સર્કિટ સમજાવો.}

\begin{solutionbox}
    \textbf{જવાબ}:

    \textbf{ડાયોડ ડિટેક્ટર સર્કિટ}: મોડ્યુલેટેડ વેવના એન્વેલોપને બહાર કાઢીને AM સિગ્નલ્સના ડિમોડ્યુલેશન માટે વપરાય છે.

    \begin{center}
    \begin{circuitikz}[scale=0.9]
        \draw (0,2) to[short, o-] (1,2) to[D, l=$D$] (3,2);
        \draw (3,2) -- (4,2) to[C, l=$C$] (4,0) node[ground] {};
        \draw (3,2) -- (6,2) to[R, l=$R$] (6,0) node[ground] {};
        \draw (6,2) to[short, -o] (7,2) node[right] {આઉટપુટ};
        \draw (0,0) node[ground] {} to[short, o-] (1,0);
        \node at (-0.5, 1) {ઇનપુટ AM};
    \end{circuitikz}
    \end{center}

    \begin{tabulary}{\textwidth}{|L|L|}
    \hline
    \textbf{ઘટક} & \textbf{કાર્ય} \\
    \hline
    \textbf{ડાયોડ (D)} & AM સિગ્નલને રેક્ટિફાય કરે છે, માત્ર પોઝિટિવ હાફ પસાર કરે છે \\
    \hline
    \textbf{કેપેસિટર (C)} & પીક વેલ્યુ સુધી ચાર્જ થાય છે, કેરિયરને સરળ બનાવે છે \\
    \hline
    \textbf{રેઝિસ્ટર (R)} & કેપેસિટરના ડિસ્ચાર્જ સમયને નિયંત્રિત કરે છે \\
    \hline
    \end{tabulary}

    \textbf{કાર્ય}:
    \begin{enumerate}
        \item ડાયોડ AM સિગ્નલને રેક્ટિફાય કરે છે.
        \item કેપેસિટર પીક વેલ્યુ સુધી ચાર્જ થાય છે.
        \item RC સમય અચળાંક કેપેસિટરને એન્વેલોપ અનુસરવાની મંજૂરી આપે છે.
        \item આઉટપુટ મૂળ મોડ્યુલેટિંગ સિગ્નલને અનુસરે છે.
    \end{enumerate}

    \begin{mnemonicbox}
    "ડાયોડ શોધે, કેપેસિટર પકડે"
    \end{mnemonicbox}
\end{solutionbox}

\questionmarks{4}{c}{7}
\textbf{ડેલ્ટા મોડ્યુલેશનનો બ્લોક ડાયાગ્રામ દોરો અને સમજાવો.}

\begin{solutionbox}
    \textbf{જવાબ}:

    \textbf{ડેલ્ટા મોડ્યુલેશન ટ્રાન્સમીટર:}
    \begin{center}
    \begin{tikzpicture}[auto, node distance=2.5cm, >=latex]
        \node [gtu block] (comp) {કમ્પેરેટર};
        \node [left of=comp, node distance=2.5cm] (in) {ઇનપુટ સિગ્નલ};
        \node [gtu block, right of=comp] (quant) {1-બિટ ક્વોન્ટાઇઝર};
        \node [gtu block, below of=comp, node distance=2cm] (integ) {ઇન્ટિગ્રેટર};
        \node [right of=quant, node distance=2.5cm] (out) {ટ્રાન્સમિશન ચેનલ};

        \draw [gtu arrow] (in) -- (comp);
        \draw [gtu arrow] (comp) -- (quant);
        \draw [gtu arrow] (quant) -- (out);
        \draw [gtu arrow] (quant.south) |- (integ.east);
        \draw [gtu arrow] (integ) -- (comp);
    \end{tikzpicture}
    \end{center}

    \textbf{ડેલ્ટા મોડ્યુલેશન રિસીવર:}
    \begin{center}
    \begin{tikzpicture}[auto, node distance=2.5cm, >=latex]
        \node [gtu block] (integ) {ઇન્ટિગ્રેટર};
        \node [left of=integ, node distance=2.5cm] (in) {ઇનપુટ સિગ્નલ};
        \node [gtu block, right of=integ] (lpf) {લો-પાસ ફિલ્ટર};
        \node [right of=lpf, node distance=2.5cm] (out) {આઉટપુટ સિગ્નલ};

        \draw [gtu arrow] (in) -- (integ);
        \draw [gtu arrow] (integ) -- (lpf);
        \draw [gtu arrow] (lpf) -- (out);
    \end{tikzpicture}
    \end{center}

    \begin{tabulary}{\textwidth}{|L|L|}
    \hline
    \textbf{ઘટક} & \textbf{કાર્ય} \\
    \hline
    \textbf{કમ્પેરેટર} & ઇનપુટને અનુમાનિત મૂલ્ય સાથે સરખાવે છે \\
    \hline
    \textbf{1-બિટ ક્વોન્ટાઇઝર} & જો ઇનપુટ $>$ અનુમાનિત હોય તો બાઇનરી 1, જો ઇનપુટ $<$ અનુમાનિત હોય તો 0 આઉટપુટ કરે છે \\
    \hline
    \textbf{ઇન્ટિગ્રેટર} & અગાઉના આઉટપુટને ઇન્ટિગ્રેટ કરીને અનુમાનિત મૂલ્ય ઉત્પન્ન કરે છે \\
    \hline
    \textbf{લો-પાસ ફિલ્ટર} & મૂળ સિગ્નલ પુનઃપ્રાપ્ત કરવા માટે સ્ટેપ્ડ આઉટપુટને સરળ બનાવે છે \\
    \hline
    \end{tabulary}

    \textbf{મર્યાદાઓ}:
    \begin{itemize}
        \item \textbf{સ્લોપ ઓવરલોડ}: જ્યારે સિગ્નલ સ્ટેપ સાઇઝ કરતાં ઝડપથી બદલાય ત્યારે થાય છે.
        \item \textbf{ગ્રેન્યુલર નોઈઝ}: સિગ્નલના આઇડલ અથવા સ્થિર ભાગો દરમિયાન થાય છે.
    \end{itemize}

    \begin{mnemonicbox}
    "ડેલ્ટા તફાવત શોધે, ઇન્ટિગ્રેટર ઉમેરો કરે"
    \end{mnemonicbox}
\end{solutionbox}

\questionmarks{5}{a}{3}
\textbf{DPCM ના કાર્યનું ચિત્રણ કરો.}

\begin{solutionbox}
    \textbf{જવાબ}:

    \textbf{DPCM (ડિફરેન્શિયલ પલ્સ કોડ મોડ્યુલેશન)}: વર્તમાન સેમ્પલ અને અનુમાનિત મૂલ્ય વચ્ચેના તફાવતને એનકોડ કરે છે.

    \begin{center}
    \begin{tikzpicture}[auto, node distance=2.5cm, >=latex]
        \node [gtu block] (add) {ડિફરન્સ જનરેટર};
        \node [left of=add, node distance=2cm] (in) {ઇનપુટ};
        \node [gtu block, right of=add] (quant) {ક્વોન્ટાઇઝર};
        \node [gtu block, right of=quant] (enc) {એન્કોડર};
        \node [right of=enc, node distance=2cm] (out) {ટ્રાન્સમિશન};
        \node [gtu block, below of=quant, node distance=2cm] (invq) {ઇન્વર્સ ક્વોન્ટાઇઝર};
        \node [gtu block, left of=invq] (pred) {પ્રેડિક્ટર};

        \draw [gtu arrow] (in) -- (add);
        \draw [gtu arrow] (add) -- (quant);
        \draw [gtu arrow] (quant) -- (enc);
        \draw [gtu arrow] (enc) -- (out);
        \draw [gtu arrow] (quant.south) -- (invq.north);
        \draw [gtu arrow] (invq) -- (pred);
        \draw [gtu arrow] (pred) -- (add);
    \end{tikzpicture}
    \end{center}

    \begin{itemize}
        \item \textbf{પ્રેડિક્ટર}: અગાઉના સેમ્પલ્સના આધારે વર્તમાન સેમ્પલનો અંદાજ લગાવે છે.
        \item \textbf{તફાવત}: માત્ર વાસ્તવિક અને અનુમાનિત વચ્ચેનો તફાવત એનકોડ થાય છે.
        \item \textbf{ફાયદો}: સિગ્નલ સહસંબંધનો ઉપયોગ કરીને PCM ની તુલનામાં બિટ રેટ ઘટાડે છે.
    \end{itemize}

    \begin{mnemonicbox}
    "તફાવત અનુમાન ઓછા બિટ્સ"
    \end{mnemonicbox}
\end{solutionbox}

\questionmarks{5}{b}{4}
\textbf{અનુકૂલનશીલ ડેલ્ટા મોડ્યુલેશનનું ચિત્રણ કરો.}

\begin{solutionbox}
    \textbf{જવાબ}:

    \textbf{અનુકૂલનશીલ ડેલ્ટા મોડ્યુલેશન (ADM)}: સિગ્નલ લાક્ષણિકતાઓના આધારે સ્ટેપ સાઇઝ બદલતી DM ની સુધારેલી આવૃત્તિ.

    \begin{center}
    \begin{tikzpicture}[auto, node distance=2.2cm, >=latex]
        \node [gtu block] (comp) {કમ્પેરેટર};
        \node [left of=comp, node distance=2cm] (in) {ઇનપુટ};
        \node [gtu block, right of=comp] (pulse) {પલ્સ જનરેટર};
        \node [gtu block, below of=pulse, node distance=1.5cm] (logic) {સ્ટેપ સાઇઝ એડાપ્ટર};
        \node [gtu block, left of=logic] (integ) {ઇન્ટિગ્રેટર};
        \node [right of=pulse, node distance=2cm] (out) {ટ્રાન્સમિશન};

        \draw [gtu arrow] (in) -- (comp);
        \draw [gtu arrow] (comp) -- (pulse);
        \draw [gtu arrow] (pulse) -- (out);
        \draw [gtu arrow] (pulse.south) -- (logic.north);
        \draw [gtu arrow] (logic) -- (integ);
        \draw [gtu arrow] (integ) -- (comp);
    \end{tikzpicture}
    \end{center}

    \textbf{કાર્યપદ્ધતિ}:
    \begin{enumerate}
        \item જો એકાધિક 1 ડિટેક્ટ થાય: સ્લોપ ઓવરલોડ ટાળવા માટે સ્ટેપ સાઇઝ વધારો.
        \item જો એકાધિક 0 ડિટેક્ટ થાય: ઘટતા સિગ્નલને ટ્રેક કરવા માટે સ્ટેપ સાઇઝ વધારો.
        \item જો 1 અને 0 વૈકલ્પિક હોય: ગ્રેન્યુલર નોઈઝ ઘટાડવા માટે સ્ટેપ સાઇઝ ઘટાડો.
    \end{enumerate}

    \begin{mnemonicbox}
    "અનુકૂલિત ડેલ્ટા ઢાળ અનુસરે"
    \end{mnemonicbox}
\end{solutionbox}

\questionmarks{5}{c}{7}
\textbf{TDM ફ્રેમનું ચિત્રણ કરો.}

\begin{solutionbox}
    \textbf{જવાબ}:

    \textbf{TDM (ટાઇમ ડિવિઝન મલ્ટિપ્લેક્સિંગ) ફ્રેમ}: ટાઇમ સ્લોટ્સ ફાળવીને એકાધિક સિગ્નલ્સને જોડવા માટે વપરાતી સ્ટ્રક્ચર.

    \textbf{ફ્રેમ સ્ટ્રક્ચર:}
    \begin{center}
    \begin{tikzpicture}
        \draw[thick] (0,0) rectangle (10, 2);
        \node at (5, 1.5) {\textbf{TDM ફ્રેમ}};
        
        \draw (0,0) rectangle (1.5, 1); \node at (0.75, 0.5) {સિંક};
        \draw (1.5,0) rectangle (3, 1); \node at (2.25, 0.5) {CH 1};
        \draw (3,0) rectangle (4.5, 1); \node at (3.75, 0.5) {CH 2};
        \draw (4.5,0) rectangle (6, 1); \node at (5.25, 0.5) {CH 3};
        \draw (6,0) rectangle (7.5, 1); \node at (6.75, 0.5) {...};
        \draw (7.5,0) rectangle (10, 1); \node at (8.75, 0.5) {CH N};
        
        \node[below] at (2.25, 0) {TS1};
        \node[below] at (3.75, 0) {TS2};
        \node[below] at (5.25, 0) {TS3};
        \node[below] at (8.75, 0) {TSn};
    \end{tikzpicture}
    \end{center}

    \begin{tabulary}{\textwidth}{|L|L|}
    \hline
    \textbf{ઘટક} & \textbf{વર્ણન} \\
    \hline
    \textbf{ફ્રેમ સિન્ક} & ફ્રેમ બાઉન્ડરીઝ ઓળખવા માટેનું પેટર્ન \\
    \hline
    \textbf{ચેનલ સેમ્પલ} & વ્યક્તિગત ચેનલનો ડેટા \\
    \hline
    \textbf{ટાઇમ સ્લોટ (TS)} & દરેક ચેનલ માટે સમર્પિત સમયગાળો \\
    \hline
    \textbf{ફ્રેમ અવધિ} & સેમ્પલિંગ રેટના વ્યસ્ત પ્રમાણસર \\
    \hline
    \end{tabulary}

    \textbf{TDM હાયરાર્કી:}
    \begin{itemize}
        \item પ્રાથમિક મલ્ટિપ્લેક્સિંગ 2.048 Mbps
        \item માધ્યમિક મલ્ટિપ્લેક્સિંગ 8.448 Mbps
        \item તૃતીય મલ્ટિપ્લેક્સિંગ 34.368 Mbps
        \item ચતુર્થ મલ્ટિપ્લેક્સિંગ 139.264 Mbps
    \end{itemize}

    \begin{mnemonicbox}
    "ફ્રેમ સંગઠિત કરે સમય સ્લોટ મલ્ટિપ્લેક્સિંગ"
    \end{mnemonicbox}
\end{solutionbox}

\questionmarks{5}{a}{3}
\textbf{DM અને ADM વચ્ચેનો તફાવત જણાવો.}

\begin{solutionbox}
    \textbf{જવાબ}:

    \begin{tabulary}{\textwidth}{|L|L|L|}
    \hline
    \textbf{પેરામીટર} & \textbf{ડેલ્ટા મોડ્યુલેશન (DM)} & \textbf{અનુકૂલનશીલ ડેલ્ટા મોડ્યુલેશન (ADM)} \\
    \hline
    \textbf{સ્ટેપ સાઇઝ} & ફિક્સ્ડ સ્ટેપ સાઇઝ & વેરિયેબલ સ્ટેપ સાઇઝ \\
    \hline
    \textbf{સ્લોપ ઓવરલોડ} & સામાન્ય સમસ્યા & અનુકૂલનશીલ સ્ટેપ સાઇઝ દ્વારા ઘટાડો \\
    \hline
    \textbf{ગ્રેન્યુલર નોઈઝ} & ધીમા વેરિએશન્સ દરમિયાન ઉચ્ચ & અનુકૂલનશીલ સ્ટેપ સાઇઝ દ્વારા ઘટાડો \\
    \hline
    \textbf{સર્કિટ જટિલતા} & સરળ & વધુ જટિલ \\
    \hline
    \textbf{સિગ્નલ ક્વોલિટી} & નીચી & ઉચ્ચ \\
    \hline
    \end{tabulary}

    \begin{mnemonicbox}
    "DM ફિક્સ્ડ સ્ટેપ; ADM અનુકૂલિત"
    \end{mnemonicbox}
\end{solutionbox}

\questionmarks{5}{b}{4}
\textbf{લાઇન કોડિંગની જરૂરિયાત સમજાવો. AMI તકનીક સમજાવો.}

\begin{solutionbox}
    \textbf{જવાબ}:

    \textbf{લાઇન કોડિંગની જરૂરિયાત:}
    \begin{itemize}
        \item \textbf{DC કમ્પોનન્ટ}: AC-કપલ્ડ સિસ્ટમ્સ માટે DC કમ્પોનન્ટ દૂર કરવા.
        \item \textbf{સિન્ક્રોનાઇઝેશન}: ક્લોક રિકવરી માટે ટાઇમિંગ માહિતી પ્રદાન કરવા.
        \item \textbf{એરર ડિટેક્શન}: ટ્રાન્સમિશન એરર શોધવા સક્ષમ કરવા.
        \item \textbf{સ્પેક્ટ્રલ એફિશિયન્સી}: કાર્યક્ષમ બેન્ડવિડ્થ ઉપયોગ માટે સિગ્નલ સ્પેક્ટ્રમને આકાર આપવા.
        \item \textbf{નોઈઝ ઇમ્યુનિટી}: ચેનલ નોઈઝ સામે પ્રતિરોધ પ્રદાન કરવા.
    \end{itemize}

    \textbf{AMI (ઓલ્ટરનેટ માર્ક ઇન્વર્ઝન) તકનીક:}

    \begin{tabulary}{\textwidth}{|L|L|}
    \hline
    \textbf{પેરામીટર} & \textbf{વર્ણન} \\
    \hline
    \textbf{એન્કોડિંગ રૂલ} & બાઇનરી 0 $\rightarrow$ 0V, બાઇનરી 1 $\rightarrow$ વૈકલ્પિક +V/-V \\
    \hline
    \textbf{DC કમ્પોનન્ટ} & કોઈ DC કમ્પોનન્ટ નથી (બેલેન્સ્ડ કોડ) \\
    \hline
    \textbf{એરર ડિટેક્શન} & વૈકલ્પિક પેટર્નમાં ઉલ્લંઘનો શોધી શકે છે \\
    \hline
    \textbf{બેન્ડવિડ્થ} & NRZ કોડ કરતાં ઓછી બેન્ડવિડ્થની જરૂર પડે છે \\
    \hline
    \end{tabulary}

    \textbf{આકૃતિ:}
    \begin{center}
    \begin{tikzpicture}[scale=0.8]
        \node [left] at (0, 1) {AMI};
        \draw [->] (0, 0) -- (12, 0) node[right] {$t$};
        \node at (0.5, 2) {1}; \node at (1.5, 2) {0}; \node at (2.5, 2) {1}; \node at (3.5, 2) {1}; 
        \node at (4.5, 2) {0}; \node at (5.5, 2) {0}; \node at (6.5, 2) {1}; \node at (7.5, 2) {0};
        
        % 1 0 1 1 0 0 1 0 (Alternating)
        \draw [thick, blue] (0,0) -- (0,1) -- (1,1) -- (1,0) -- (2,0) -- (2,-1) -- (3,-1) -- (3,0) -- (3,1) -- (4,1) -- (4,0) -- (6,0) -- (6,-1) -- (7,-1) -- (7,0) -- (8,0);
        
    \end{tikzpicture}
    \end{center}

    \begin{mnemonicbox}
    "વૈકલ્પિક એક ધ્રુવતા બદલે"
    \end{mnemonicbox}
\end{solutionbox}

\questionmarks{5}{c}{7}
\textbf{મૂળભૂત PCM-TDM સિસ્ટમનો બ્લોક ડાયાગ્રામ દોરો અને સમજાવો.}

\begin{solutionbox}
    \textbf{જવાબ}:

    \textbf{PCM-TDM ટ્રાન્સમીટર:}
    \begin{center}
    \begin{tikzpicture}[auto, node distance=2cm, >=latex]
        % Channels
        \node (ch1) {CH 1};
        \node [below of=ch1] (ch2) {CH 2};
        \node [below of=ch2] (ch3) {CH 3};
        
        % LPFs
        \node [gtu block, right of=ch1, node distance=2cm] (lpf1) {LPF};
        \node [gtu block, right of=ch2, node distance=2cm] (lpf2) {LPF};
        \node [gtu block, right of=ch3, node distance=2cm] (lpf3) {LPF};
        
        % Multiplexer (simplified as switch inputs)
        \node [gtu block, right of=lpf2, node distance=3cm, minimum height=3cm] (mux) {મલ્ટિપ્લેક્સર};
        
        % PCM part
        \node [gtu block, right of=mux, node distance=2.5cm] (adc) {ADC/એન્કોડર};
        \node [gtu block, right of=adc] (lc) {લાઇન કોડર};
        \node [right of=lc] (out) {ટ્રાન્સમિશન};

        \draw [gtu arrow] (ch1) -- (lpf1);
        \draw [gtu arrow] (ch2) -- (lpf2);
        \draw [gtu arrow] (ch3) -- (lpf3);
        
        \draw [gtu arrow] (lpf1) -- (mux.160);
        \draw [gtu arrow] (lpf2) -- (mux.180);
        \draw [gtu arrow] (lpf3) -- (mux.200);
        
        \draw [gtu arrow] (mux) -- (adc);
        \draw [gtu arrow] (adc) -- (lc);
        \draw [gtu arrow] (lc) -- (out);
    \end{tikzpicture}
    \end{center}

    \begin{tabulary}{\textwidth}{|L|L|}
    \hline
    \textbf{બ્લોક} & \textbf{કાર્ય} \\
    \hline
    \textbf{લો-પાસ ફિલ્ટર} & સેમ્પલિંગ થિયરમને સંતોષવા માટે બેન્ડવિડ્થને મર્યાદિત કરે છે \\
    \hline
    \textbf{મલ્ટિપ્લેક્સર} & ઇનપુટ્સને સેમ્પલ \& હોલ્ડ કરે છે અને તેમને ક્રમશઃ જોડે છે \\
    \hline
    \textbf{ADC/એન્કોડર} & મલ્ટિપ્લેક્સ કરેલ સ્ટ્રીમને ક્વોન્ટાઇઝ અને એનકોડ કરે છે \\
    \hline
    \textbf{લાઇન કોડર} & પ્રસારણ માટે બાઇનરી ડેટા ફોર્મેટ કરે છે \\
    \hline
    \textbf{રિસીવર} & ઉલટી પ્રક્રિયા કરે છે (ડિકોડર $\rightarrow$ ડિમક્સ $\rightarrow$ LPFs) \\
    \hline
    \end{tabulary}

    \begin{mnemonicbox}
    "મલ્ટિપલ ચેનલ્સ સેમ્પલ, ક્વોન્ટાઇઝ, એનકોડ; ડિકોડ, ડિમલ્ટિપ્લેક્સ, ફિલ્ટર"
    \end{mnemonicbox}
\end{solutionbox}

\end{document}

