\documentclass{article}

% content/resources/templates/preamble.tex
\usepackage[margin=0.6in]{geometry}
\author{Milav Dabgar}
\usepackage{amsmath,amssymb,amsthm}
\usepackage{booktabs}
\usepackage{multirow}
\usepackage{xcolor}
\usepackage{tcolorbox}
\tcbuselibrary{breakable,skins}
\usepackage[colorlinks=true,linkcolor=blue]{hyperref}
\usepackage{titlesec}
\usepackage{enumitem}
\usepackage{tikz}
\usepackage{pgfplots}
\usepackage{circuitikz}
\usepackage[version=4]{mhchem}
\usepackage{longtable}
\usepackage{array}
\usepackage{float}
\usepackage{caption}
\usepackage{listings}

\lstset{
  basicstyle=\small\ttfamily,
  breaklines=true,
  breakatwhitespace=false,
  postbreak=\mbox{\textcolor{red}{$\hookrightarrow$}\space},
  float=false,
  numbers=left,
  numberstyle=\tiny\color{gray},
  numbersep=10pt,
  xleftmargin=2em,
  keywordstyle=\color{blue},
  commentstyle=\color{green!60!black},
  stringstyle=\color{purple},
  backgroundcolor=\color{gray!5},
  showstringspaces=false,
  tabsize=2,
  captionpos=b,
  keepspaces=true,
  columns=flexible
}

\pgfplotsset{compat=1.18}
\usetikzlibrary{shapes,arrows,positioning,calc,patterns,decorations.pathmorphing,decorations.markings,arrows.meta}

% Color scheme
\definecolor{headcolor}{RGB}{0,102,204}
\definecolor{keycolor}{RGB}{220,20,60}
\definecolor{solutioncolor}{RGB}{34,139,34}
\definecolor{mnemoniccolor}{RGB}{148,0,211}
\definecolor{codecolor}{RGB}{0,0,100}

% Spacing
\setlength{\parskip}{3pt}
\setlist[itemize]{nosep}
\setlist[enumerate]{nosep}

% Title formatting
\titleformat{\section}{\Large\bfseries\color{headcolor}}{\thesection}{1em}{}
\titleformat{\subsection}{\large\bfseries\color{headcolor}}{\thesubsection}{1em}{}

% Pandoc tightlist compatibility
\providecommand{\tightlist}{%
  \setlength{\itemsep}{0pt}\setlength{\parskip}{0pt}}

% Pandoc longtable compatibility
\newcounter{none}
\def\thenone{}


% content/resources/templates/gujarati-boxes.tex
\usepackage{fontspec}
\usepackage{polyglossia}

% Set Gujarati as main language (document is primarily in Gujarati)
% Note: gloss-gujarati.ldf doesn't exist in polyglossia, but it will use hyphenation patterns
\setdefaultlanguage{gujarati}
\setotherlanguage{english}

% Configure Gujarati font properly
% Use Language=Default to prevent polyglossia from trying to add language-specific features
% that don't exist for Gujarati, which causes "empty feature" warnings
\newfontfamily\gujaratifont[Script=Gujarati,AutoFakeBold=2.5,AutoFakeSlant=0.3]{Noto Sans Gujarati}
\setmainfont[Script=Gujarati,AutoFakeBold=2.5,AutoFakeSlant=0.3]{Noto Sans Gujarati}
% Use Noto Sans Gujarati for monospace to support Gujarati in text
\setmonofont[Scale=0.9]{Noto Sans Gujarati}

% Configure English to use the same font
\newfontfamily\englishfont[Script=Gujarati,AutoFakeBold=2.5,AutoFakeSlant=0.3]{Noto Sans Gujarati}

% Translations for polyglossia
\gappto\captionsgujarati{
  \renewcommand{\tablename}{કોષ્ટક}
  \renewcommand{\figurename}{આકૃતિ}
}

% Helper for TikZ nodes to ensure Gujarati font
\newcommand{\gu}[1]{{\gujaratifont #1}}

% Custom environments
\newtcolorbox{solutionbox}{
    breakable,
    enhanced,
    colback=solutioncolor!5!white,
    colframe=solutioncolor!75!black,
    fonttitle=\bfseries,
    title=જવાબ
}

\newtcolorbox{solutionboxnobreak}{
 colback=solutioncolor!5!white,
 colframe=solutioncolor!75!black,
 fonttitle=\bfseries,
 title=જવાબ
}

\newtcolorbox{keyformula}{
 breakable,
 enhanced,
 colback=keycolor!5!white,
 colframe=keycolor!75!black,
 fonttitle=\bfseries,
 title=રાસાયણિક સમીકરણ/સૂત્ર
}

\newtcolorbox{mnemonicbox}{
 breakable,
 enhanced,
 colback=mnemoniccolor!5!white,
 colframe=mnemoniccolor!75!black,
 fonttitle=\bfseries,
 title=મેમરી ટ્રીક
}


% Custom commands for GTU solutions
% This file defines semantic commands for consistent formatting

% Question command with automatic formatting
\newcommand{\question}[2]{%
  \section*{Question #1}%
  \textbf{#2}%
}

% OR question variant
\newcommand{\questionor}[2]{%
  \section*{Question #1 OR}%
  \textbf{#2}%
}

% Proper table environment with caption
\newenvironment{answertable}[1]{%
  \begin{table}[htbp]
  \centering
  \caption{#1}
}{%
  \end{table}
}

% Proper figure environment for diagrams
\newenvironment{answerdiagram}[1]{%
  \begin{figure}[htbp]
  \centering
  \caption{#1}
}{%
  \end{figure}
}

% Semantic markup for key terms
\newcommand{\keyword}[1]{\textbf{#1}}
\newcommand{\code}[1]{\texttt{#1}}
\newcommand{\classname}[1]{\texttt{#1}}
\newcommand{\methodname}[1]{\texttt{#1}}

% Proper quotation marks
\newcommand{\mnemonic}[1]{``#1''}


\title{Principles of Electronic Communication (4331104) - Winter 2022 Solution}
\date{March 01, 2022}

\begin{document}
\maketitle

\questionmarks{1(a)}{3}{મોડયુલેશન શું છે? તેની જરૂરિયાત શું છે?}

\begin{solutionbox}
મોડયુલેશન એ એક ઉચ્ચ આવૃત્તિવાળા કેરિયર સિગ્નલના એક અથવા વધુ ગુણધર્મો (amplitude, frequency, અથવા phase) ને માહિતી ધરાવતા સિગ્નલ સાથે બદલવાની પ્રક્રિયા છે.

\textbf{મોડયુલેશનની જરૂરિયાત:}

\begin{itemize}
    \item \textbf{એન્ટેના સાઇઝ ઘટાડવા}: વ્યવહારિક એન્ટેના સાઇઝ શક્ય બનાવે છે ($\lambda = c/f$)
    \item \textbf{મલ્ટિપ્લેક્સિંગ}: અનેક સિગ્નલ્સને એક માધ્યમમાં મોકલવા માટે
    \item \textbf{નોઇઝ ઘટાડવા}: ઉચ્ચ આવૃત્તિ બેન્ડમાં શિફ્ટ કરીને SNR સુધારે છે
    \item \textbf{રેન્જ વધારવા}: ટ્રાન્સમિશન અંતર વધારે છે
\end{itemize}
\end{solutionbox}

\begin{mnemonicbox}
\mnemonic{AMEN: Antenna size, Multiplexing, Eliminate noise, New range}
\end{mnemonicbox}

\questionmarks{1(b)}{4}{એમ્પલીટયૂડ મોડયુલેશન માટે વૉલ્ટેજ સમીકરણ મેળવો.}

\begin{solutionbox}
AM માં, કેરિયર સિગ્નલ મેસેજ સિગ્નલ દ્વારા મોડ્યુલેટેડ થાય છે.

\textbf{ગાણિતિક સ્થાપના:}

\begin{itemize}
    \item \textbf{કેરિયર સિગ્નલ}: $e_c(t) = A_c \cos(2\pi f_c t)$
    \item \textbf{મેસેજ સિગ્નલ}: $e_m(t) = A_m \cos(2\pi f_m t)$
    \item \textbf{ઇન્સ્ટન્ટનીયસ એમ્પ્લિટ્યુડ}: $A_i = A_c + e_m(t)$
    \item \textbf{AM સિગ્નલ}: $e_{AM}(t) = A_i \cos(2\pi f_c t)$
    \item \textbf{સબ્સ્ટિટ્યુશન}: $e_{AM}(t) = [A_c + A_m \cos(2\pi f_m t)] \cos(2\pi f_c t)$
    \item \textbf{એક્સ્પેન્ડિંગ}: $e_{AM}(t) = A_c\cos(2\pi f_c t) + A_m\cos(2\pi f_m t)\cos(2\pi f_c t)$
    \item \textbf{ફાઇનલ ઇક્વેશન}: $e_{AM}(t) = A_c\cos(2\pi f_c t) + \frac{A_m}{2}\cos(2\pi(f_c+f_m)t) + \frac{A_m}{2}\cos(2\pi(f_c-f_m)t)$
\end{itemize}
\end{solutionbox}

\begin{mnemonicbox}
\mnemonic{CAT: Carrier, Addition, Three components (carrier + 2 sidebands)}
\end{mnemonicbox}

\questionmarks{1(c)}{7}{નોઈસ સિગ્નલને વર્ગીકૃત કરો ફ્લીકર નોઈસ, શૉટ નોઈસ અને થર્મલ નોઈસ સમજાવો.}

\begin{solutionbox}
\textbf{નોઇઝ વર્ગીકરણ:}

\begin{center}
\captionof{table}{નોઇઝ વર્ગીકરણ}
\begin{tabulary}{\linewidth}{|L|L|L|}
\hline
\textbf{પ્રકાર} & \textbf{સ્ત્રોત} & \textbf{લક્ષણો} \\ \hline
\textbf{બાહ્ય નોઇઝ} & એટમોસ્ફેરિક, સ્પેસ, ઔદ્યોગિક, માનવ-નિર્મિત & કોમ્યુનિકેશન સિસ્ટમની બહારથી ઉત્પન્ન થાય છે \\ \hline
\textbf{આંતરિક નોઇઝ} & થર્મલ, શોટ, ટ્રાન્ઝિટ-ટાઇમ, ફ્લિકર & કોમ્પોનેન્ટ્સની અંદરથી ઉત્પન્ન થાય છે \\ \hline
\end{tabulary}
\end{center}

\textbf{આંતરિક નોઈઝના પ્રકાર:}

\begin{itemize}
    \item \textbf{ફ્લિકર નોઈઝ}:
    \begin{itemize}
        \item નીચી આવૃત્તિઓ પર થાય છે (1 kHz નીચે)
        \item આવૃત્તિના વ્યસ્ત પ્રમાણમાં (1/f નોઇઝ)
        \item સેમિકન્ડક્ટર ડિવાઇસ અને કાર્બન રેસિસ્ટર્સમાં સામાન્ય છે
    \end{itemize}
    \item \textbf{શોટ નોઈઝ}:
    \begin{itemize}
        \item કરંટ કેરિયર્સના રેન્ડમ ફ્લક્ચુએશન્સને કારણે
        \item અચલ પાવર ડેન્સિટી સાથે વ્હાઇટ નોઇઝ
        \item ડાયોડ અને ટ્રાન્ઝિસ્ટર જેવી એક્ટિવ ડિવાઇસમાં થાય છે
    \end{itemize}
    \item \textbf{થર્મલ નોઈઝ}:
    \begin{itemize}
        \item કન્ડક્ટરમાં ઇલેક્ટ્રોન્સની રેન્ડમ ગતિને કારણે
        \item તાપમાન અને બેન્ડવિડ્થના સીધા પ્રમાણમાં
        \item બધા પેસિવ કોમ્પોનેન્ટ્સમાં હાજર
        \item જોનસન નોઇઝ અથવા વ્હાઇટ નોઇઝ તરીકે પણ ઓળખાય છે
    \end{itemize}
\end{itemize}
\end{solutionbox}

\begin{mnemonicbox}
\mnemonic{FAST: Flicker (low frequency), Active (shot), Semiconductor (flicker), Temperature (thermal)}
\end{mnemonicbox}

\questionmarks{1(c) OR}{7}{EM wave spectrum ના વિવિધ બેન્ડની એપ્લિકેશન લખો.}

\begin{solutionbox}
\textbf{EM સ્પેક્ટ્રમ એપ્લિકેશન્સ:}

\begin{center}
\captionof{table}{EM સ્પેક્ટ્રમ એપ્લિકેશન્સ}
\begin{tabulary}{\linewidth}{|L|L|L|}
\hline
\textbf{ફ્રીક્વન્સી બેન્ડ} & \textbf{ફ્રીક્વન્સી રેન્જ} & \textbf{એપ્લિકેશન્સ} \\ \hline
\textbf{ELF} (Extremely Low Frequency) & 3Hz -- 30Hz & સબમરીન કોમ્યુનિકેશન \\ \hline
\textbf{VLF} (Very Low Frequency) & 3kHz -- 30kHz & નેવિગેશન, ટાઇમ સિગ્નલ્સ \\ \hline
\textbf{LF} (Low Frequency) & 30kHz -- 300kHz & AM રેડિયો, નેવિગેશન \\ \hline
\textbf{MF} (Medium Frequency) & 300kHz -- 3MHz & AM બ્રોડકાસ્ટિંગ, મેરિટાઇમ \\ \hline
\textbf{HF} (High Frequency) & 3MHz -- 30MHz & શોર્ટવેવ રેડિયો, એમેચ્યોર રેડિયો \\ \hline
\textbf{VHF} (Very High Frequency) & 30MHz -- 300MHz & FM રેડિયો, TV બ્રોડકાસ્ટિંગ, એર ટ્રાફિક કંટ્રોલ \\ \hline
\textbf{UHF} (Ultra High Frequency) & 300MHz -- 3GHz & TV બ્રોડકાસ્ટિંગ, મોબાઇલ ફોન, WiFi, બ્લૂટૂથ \\ \hline
\textbf{SHF} (Super High Frequency) & 3GHz -- 30GHz & સેટેલાઇટ કોમ્યુનિકેશન, રડાર, WiFi \\ \hline
\textbf{EHF} (Extremely High Frequency) & 30GHz -- 300GHz & રેડિયો એસ્ટ્રોનોમી, 5G, મિલિમીટર-વેવ રડાર \\ \hline
\textbf{Infrared} & 300GHz -- 400THz & રિમોટ કંટ્રોલ, થર્મલ ઇમેજિંગ, ફાઇબર ઓપ્ટિક્સ \\ \hline
\textbf{Visible Light} & 400THz -- 800THz & ફાઇબર ઓપ્ટિક્સ, LiFi, ફોટોગ્રાફી \\ \hline
\textbf{Ultraviolet} & 800THz -- 30PHz & સ્ટેરિલાઇઝેશન, ફ્લોરેસન્સ, સિક્યુરિટી \\ \hline
\textbf{X-rays} & 30PHz -- 30EHz & મેડિકલ ઇમેજિંગ, સિક્યુરિટી સ્ક્રીનિંગ \\ \hline
\textbf{Gamma rays} & $>$30EHz & મેડિકલ ટ્રીટમેન્ટ, ન્યુક્લિયર ડિટેક્શન \\ \hline
\end{tabulary}
\end{center}
\end{solutionbox}

\begin{mnemonicbox}
\mnemonic{Every Very Lovely Monkey Has Visited Uncle Sam's House Easily In Visible Upper Xtra Gamma}
\end{mnemonicbox}

\questionmarks{2(a)}{3}{DSBની સરખામણીએ SSBના ફાયદાઓ લખો.}

\begin{solutionbox}
\textbf{SSBના DSB પર ફાયદાઓ:}

\begin{center}
\captionof{table}{SSBના ફાયદાઓ}
\begin{tabulary}{\linewidth}{|L|L|}
\hline
\textbf{ફાયદો} & \textbf{વર્ણન} \\ \hline
\textbf{બેન્ડવિથ એફિશિયન્સી} & અડધી બેન્ડવિથનો ઉપયોગ (માત્ર એક સાઇડબેન્ડ) \\ \hline
\textbf{પાવર એફિશિયન્સી} & ઓછી ટ્રાન્સમિટર પાવરની જરૂર (83.33\% પાવર સેવિંગ) \\ \hline
\textbf{ઘટાડેલું ફેડિંગ} & સિલેક્ટિવ ફેડિંગને ઓછું સંવેદનશીલ \\ \hline
\textbf{ઓછું ડિસ્ટોરશન} & ઇન્ટરમોડ્યુલેશન ડિસ્ટોર્શન ઘટાડે છે \\ \hline
\textbf{સરળ રિસીવર} & સરળ સર્કિટ ડિઝાઇન શક્ય \\ \hline
\end{tabulary}
\end{center}
\end{solutionbox}

\begin{mnemonicbox}
\mnemonic{BPFDS: Bandwidth, Power, Fading, Distortion, Simple}
\end{mnemonicbox}

\questionmarks{2(b)}{4}{ફેસ લોક લુપ ટેક્નીકથી FMનું જનરેશન સમજાવો.}

\begin{solutionbox}
PLL (Phase-Locked Loop) VCO કંટ્રોલ ઇનપુટ પર મોડ્યુલેટિંગ સિગ્નલ લાગુ કરીને FM સિગ્નલ્સ ઉત્પન્ન કરે છે.

\textbf{PLL FM મોડ્યુલેટર:}

\begin{center}
\begin{tikzpicture}[node distance=2.5cm, auto, >=latex, every node/.style={transform shape}]
    % Nodes
    \node [gtu block] (sum) {Summing Circuit};
    \node [left=1.5cm of sum] (input) {Modulating Signal};
    \node [gtu block, right=of sum] (vco) {VCO};
    \node [right=1.5cm of vco] (output) {FM Output};
    
    \node [gtu block, below=of vco] (feedback) {Feedback};
    \node [gtu block, left=of feedback] (pd) {Phase Detector};
    \node [gtu block, left=of pd] (ref) {Reference Oscillator};
    \node [gtu block, above=of pd] (lpf) {Low Pass Filter};
    
    % Connections
    \draw [gtu arrow] (input) -- (sum);
    \draw [gtu arrow] (sum) -- (vco);
    \draw [gtu arrow] (vco) -- (output);
    \draw [gtu arrow] (vco) -- (feedback);
    \draw [gtu arrow] (feedback) -- (pd);
    \draw [gtu arrow] (ref) -- (pd);
    \draw [gtu arrow] (pd) -- (lpf);
    \draw [gtu arrow] (lpf) -- (sum);
\end{tikzpicture}
\captionof{figure}{PLL દ્વારા FM જનરેશન}
\end{center}

\textbf{ઓપરેશન:}

\begin{itemize}
    \item \textbf{રેફરન્સ ઓસીલેટર}: સ્થિર રેફરન્સ ફ્રીક્વન્સી પ્રદાન કરે છે
    \item \textbf{ફેઝ ડિટેક્ટર}: રેફરન્સ અને ફીડબેક સિગ્નલોની તુલના કરે છે
    \item \textbf{લો પાસ ફિલ્ટર}: ઉચ્ચ-ફ્રીકવન્સી ઘટકોને દૂર કરે છે
    \item \textbf{VCO}: કંટ્રોલ વોલ્ટેજ સાથે બદલાતી આઉટપુટ ફ્રીક્વન્સી જનરેટ કરે છે
    \item \textbf{મોડ્યુલેટિંગ સિગ્નલ}: FM આઉટપુટ ઉત્પન્ન કરવા માટે કંટ્રોલ વોલ્ટેજમાં ઉમેરાય છે
\end{itemize}
\end{solutionbox}

\begin{mnemonicbox}
\mnemonic{PROVE: Phase detector, Reference oscillator, Output VCO, Voltage controlled}
\end{mnemonicbox}

\questionmarks{2(c)}{7}{AM માટે ટોટલ પાવરનું સમીકરણ તારવો. DSB અને SSB માટે પાવર સેવિંગના ટકાની ગણતરી કરો.}

\begin{solutionbox}
\textbf{AM માં પાવર:}

AM વેવ ઇક્વેશન: $e_{AM}(t) = A_c[1 + m\cos(2\pi f_m t)]\cos(2\pi f_c t)$

\textbf{પાવર ડેરીવેશન:}

\begin{itemize}
    \item \textbf{કુલ પાવર}: $P_T = P_c\left(1 + \frac{m^2}{2}\right)$
    \item જ્યાં $P_c = \frac{A_c^2}{2R}$ (કેરિયર પાવર) અને $m$ મોડ્યુલેશન ઇન્ડેક્સ છે
\end{itemize}

\textbf{પાવર ડિસ્ટ્રિબ્યુશન:}

\begin{itemize}
    \item \textbf{કેરિયર પાવર}: $P_c = \frac{A_c^2}{2R}$
    \item \textbf{કુલ સાઇડબેન્ડ પાવર}: $P_{SB} = \frac{m^2 P_c}{2}$
    \item \textbf{દરેક સાઇડબેન્ડ}: $P_{LSB} = P_{USB} = \frac{m^2 P_c}{4}$
\end{itemize}

\textbf{પાવર સેવિંગ્સ:}

\begin{itemize}
    \item \textbf{DSB-SC માં}: કેરિયર પાવર નથી, એટલે સેવિંગ્સ = $\frac{P_c}{P_T} \times 100\% = \frac{1}{1+\frac{m^2}{2}} \times 100\%$
    \begin{itemize}
        \item m=1 માટે, સેવિંગ્સ = 66.67\%
    \end{itemize}
    \item \textbf{SSB માં}: કેરિયર અને એક સાઇડબેન્ડ નથી, એટલે સેવિંગ્સ = $\frac{P_c + P_{SB}/2}{P_T} \times 100\%$
    \begin{itemize}
        \item m=1 માટે, સેવિંગ્સ = 83.33\%
    \end{itemize}
\end{itemize}
\end{solutionbox}

\begin{mnemonicbox}
\mnemonic{CEPTS: Carrier Eliminated Provides Tremendous Savings}
\end{mnemonicbox}

\questionmarks{2(a) OR}{3}{AM વેવ માટે Time domain અને Frequency domain ડિસ્પ્લે દોરો અને સમજાવો.}

\begin{solutionbox}
\textbf{AM ના Time અને Frequency Domain:}

\begin{center}
\begin{tikzpicture}
    % Time Domain
    \begin{scope}[xshift=0cm, yshift=4cm]
        \draw[->] (0,0) -- (6,0) node[right] {$t$};
        \draw[->] (0,-2) -- (0,2) node[above] {$V(t)$};
        
        \draw[blue, thick, domain=0:5.5, samples=200, smooth] plot (\x, {1.5*(1 + 0.5*cos(2*3.1415*0.5*\x r))*cos(2*3.1415*10*\x r)});
        
        \draw[red, dashed, domain=0:5.5, samples=100] plot (\x, {1.5*(1 + 0.5*cos(2*3.1415*0.5*\x r))});
        \draw[red, dashed, domain=0:5.5, samples=100] plot (\x, {-1.5*(1 + 0.5*cos(2*3.1415*0.5*\x r))});
        
        \node at (2.75, -2.5) {Time Domain};
    \end{scope}

    % Frequency Domain
    \begin{scope}[xshift=0cm, yshift=0cm]
        \draw[->] (0,0) -- (6,0) node[right] {$f$};
        \draw[->] (0,0) -- (0,2.5) node[above] {$V(f)$};
        
        % Carrier
        \draw[thick, blue] (3,0) -- (3,2);
        \node[above] at (3,2) {$A_c$};
        \node[below] at (3,0) {$f_c$};
        
        % LSB
        \draw[thick, blue] (1.5,0) -- (1.5,1);
        \node[above] at (1.5,1) {$\frac{mA_c}{2}$};
        \node[below] at (1.5,0) {$f_c-f_m$};
        
        % USB
        \draw[thick, blue] (4.5,0) -- (4.5,1);
        \node[above] at (4.5,1) {$\frac{mA_c}{2}$};
        \node[below] at (4.5,0) {$f_c+f_m$};
        
        \node at (3, -1) {Frequency Domain};
    \end{scope}
\end{tikzpicture}
\captionof{figure}{AM ના Time અને Frequency Domain}
\end{center}

\textbf{ટાઇમ ડોમેન:}

\begin{itemize}
    \item સમય સાથે કેરિયરના એમ્પલિટ્યુડ વેરિએશન બતાવે છે
    \item એન્વેલોપ મોડ્યુલેટિંગ સિગ્નલને અનુસરે છે
    \item ઉપર અને નીચેના એન્વેલોપ = કેરિયર પીક $\times (1 \pm m)$
\end{itemize}

\textbf{ફ્રિક્વન્સી ડોમેન:}

\begin{itemize}
    \item ફ્રિક્વન્સી કોમ્પોનન્ટ્સ અને તેમના એમ્પ્લિટ્યુડ બતાવે છે
    \item $f_c$ ફ્રિક્વન્સી પર $A_c$ એમ્પ્લિટ્યુડ સાથે કેરિયર
    \item $f_c \pm f_m$ પર $mA_c/2$ એમ્પ્લિટ્યુડ સાથે બે સાઇડબેન્ડસ
    \item બેન્ડવિડ્થ = $2f_m$ (મોડ્યુલેટિંગ ફ્રિક્વન્સીનો બમણો)
\end{itemize}
\end{solutionbox}

\begin{mnemonicbox}
\mnemonic{EBS: Envelope in time, Bandwidth in frequency, Sidebands symmetric}
\end{mnemonicbox}

\questionmarks{2(b) OR}{4}{પ્રી-એમફાસીસ અને ડી એમફાસીસ સર્કીટ સમજાવો.}

\begin{solutionbox}
\textbf{પ્રી-એમફાસીસ અને ડી-એમફાસીસ:}

\begin{center}
\begin{tikzpicture}[american]
    % Pre-emphasis
    \begin{scope}[xshift=0cm]
        \draw (0,2) to[short, o-] (0.5,2);
        \draw (0.5,2) -- (0.5, 3) to[C, l=$C$] (3.5, 3) -- (3.5, 2);
        \draw (0.5,2) -- (0.5, 1) to[R, l=$R_1$] (3.5, 1) -- (3.5, 2);
        \draw (3.5,2) to[short, -o] (5,2);
        \draw (4,2) to[R, l=$R_2$] (4,0);
        \draw (0,0) to[short, o-o] (5,0);
        
        \node[above] at (2.5, 3.5) {પ્રી-એમફાસીસ (Transmitter)};
        \node[below] at (2.5, -0.5) {Boosts High Frequencies};
    \end{scope}

    % De-emphasis
    \begin{scope}[xshift=7cm]
        \draw (0,2) to[R, l=$R$, o-] (3,2) to[short, -o] (4,2);
        \draw (3,2) to[C, l=$C$] (3,0);
        \draw (0,0) to[short, o-o] (4,0);
        
        \node[above] at (2, 2.5) {ડી-એમફાસીસ (Receiver)};
        \node[below] at (2, -0.5) {Attenuates High Frequencies};
    \end{scope}
\end{tikzpicture}
\captionof{figure}{પ્રી-એમફાસીસ અને ડી-એમફાસીસ}
\end{center}

\textbf{ઓપરેશન:}

\begin{itemize}
    \item \textbf{પ્રી-એમફાસીસ}: હાઇ-પાસ RC સર્કિટ (R પેરેલલ, C સીરીઝ). ટ્રાન્સમીટર પર ઉચ્ચ-ફ્રીક્વન્સી ઘટકોને વધારે છે.
    \item \textbf{ડી-એમફાસીસ}: લો-પાસ RC સર્કિટ (R સીરીઝ, C પેરેલલ). રિસીવર પર ઉચ્ચ-ફ્રીક્વન્સી ઘટકોને ઘટાડે છે.
    \item ટાઇમ કોન્સ્ટન્ટ સરખા છે: $\tau = RC = 75\mu s$ (સ્ટાન્ડર્ડ).
\end{itemize}
\end{solutionbox}

\begin{mnemonicbox}
\mnemonic{BETH: Boost (pre-emphasis), Emphasizes Treble, Helps SNR}
\end{mnemonicbox}

\questionmarks{2(c) OR}{7}{AM, FM અને PMને સરખાવો.}

\begin{solutionbox}
\textbf{AM, FM અને PM ની તુલના:}

\begin{center}
\captionof{table}{AM, FM, અને PM ની તુલના}
\begin{tabulary}{\linewidth}{|L|L|L|L|}
\hline
\textbf{પેરામીટર} & \textbf{AM} & \textbf{FM} & \textbf{PM} \\ \hline
\textbf{વ્યાખ્યા} & મેસેજ સિગ્નલ સાથે એમ્પ્લિટ્યુડ બદલાય છે & મેસેજ સિગ્નલ સાથે ફ્રીક્વન્સી બદલાય છે & મેસેજ સિગ્નલ સાથે ફેઝ બદલાય છે \\ \hline
\textbf{ગાણિતિક અભિવ્યક્તિ} & $A_c[1+m\cos(\omega_mt)]\cos(\omega_ct)$ & $A_c\cos[\omega_ct+m_f\sin(\omega_mt)]$ & $A_c\cos[\omega_ct+m_p\cos(\omega_mt)]$ \\ \hline
\textbf{બેન્ડવિડ્થ} & $2f_m$ (સાંકડી) & $2(\Delta f+f_m)$ (વિશાળ) & $2(m_p+1)f_m$ (વિશાળ) \\ \hline
\textbf{પાવર દક્ષતા} & ઓછી & ઉચ્ચ & ઉચ્ચ \\ \hline
\textbf{નોઇઝ ઇમ્યુનિટી} & નબળી & ઉત્તમ & ઉત્તમ \\ \hline
\textbf{સર્કિટ જટિલતા} & સરળ & જટિલ & જટિલ \\ \hline
\textbf{એપ્લિકેશન્સ} & બ્રોડકાસ્ટિંગ & રેડિયો, TV & સેટેલાઇટ \\ \hline
\end{tabulary}
\end{center}
\end{solutionbox}

\begin{mnemonicbox}
\mnemonic{BANCP-MAP: Bandwidth, Amplitude, Noise, Complexity, Power, Modulation, Applications, Parameters}
\end{mnemonicbox}

\questionmarks{3(a)}{3}{રેડીઓ રીસીવર ની કોઈ ચાર લાક્ષણીકતા ઓ વ્યાખ્યાઈત કરો.}

\begin{solutionbox}
\textbf{રેડિયો રિસીવર લક્ષણો:}

\begin{center}
\captionof{table}{રેડિયો રિસીવર લક્ષણો}
\begin{tabulary}{\linewidth}{|L|L|}
\hline
\textbf{લાક્ષણિકતા} & \textbf{વ્યાખ્યા} \\ \hline
\textbf{સેન્સિટિવિટી} & સ્વીકાર્ય આઉટપુટ માટે જરૂરી લઘુતમ સિગ્નલ શક્તિ \\ \hline
\textbf{સિલેક્ટિવિટી} & આજુબાજુના સિગ્નલથી ઇચ્છિત સિગ્નલને અલગ કરવાની ક્ષમતા \\ \hline
\textbf{ફિડેલિટી} & ડિસ્ટોર્શન વિના મૂળ સિગ્નલને પુનઃઉત્પન્ન કરવામાં ચોકસાઈ \\ \hline
\textbf{ઇમેજ રિજેક્શન} & ઇમેજ ફ્રીક્વન્સી ઇન્ટરફેરન્સને નકારવાની ક્ષમતા \\ \hline
\textbf{સિગ્નલ-ટુ-નોઇઝ રેશિયો} & ઇચ્છિત સિગ્નલ અને અનિચ્છનીય નોઇઝનો ગુણોત્તર \\ \hline
\textbf{સ્ટેબિલિટી} & ટ્યુન કરેલી ફ્રીક્વન્સીને ડ્રિફ્ટ કર્યા વિના જાળવી રાખવાની ક્ષમતા \\ \hline
\end{tabulary}
\end{center}
\end{solutionbox}

\begin{mnemonicbox}
\mnemonic{SFIS-SS: Sensitivity, Fidelity, Image rejection, Selectivity, SNR, Stability}
\end{mnemonicbox}

\questionmarks{3(b)}{4}{FM રીસીવરનો બ્લોક ડાયગ્રામ દોરો. FM રીસીવરમા લીમીટરનું કાર્ય શું છે?}

\begin{solutionbox}
\textbf{FM રિસીવર બ્લોક ડાયાગ્રામ:}

\begin{center}
\begin{tikzpicture}[node distance=1.5cm, auto, >=latex, every node/.style={transform shape, scale=0.8}]
    \node [gtu block] (ant) {Antenna};
    \node [gtu block, right=of ant] (rf) {RF Amplifier};
    \node [gtu block, right=of rf] (mix) {Mixer};
    \node [gtu block, below=of mix] (lo) {Local Oscillator};
    \node [gtu block, right=of mix] (if) {IF Amplifier};
    \node [gtu block, right=of if] (lim) {Limiter};
    \node [gtu block, below=of lim] (det) {FM Detector};
    \node [gtu block, left=of det] (audio) {Audio Amplifier};
    \node [gtu block, left=of audio] (spk) {Speaker};

    \draw [gtu arrow] (ant) -- (rf);
    \draw [gtu arrow] (rf) -- (mix);
    \draw [gtu arrow] (lo) -- (mix);
    \draw [gtu arrow] (mix) -- (if);
    \draw [gtu arrow] (if) -- (lim);
    \draw [gtu arrow] (lim) -- (det);
    \draw [gtu arrow] (det) -- (audio);
    \draw [gtu arrow] (audio) -- (spk);
\end{tikzpicture}
\captionof{figure}{FM રિસીવર બ્લોક ડાયાગ્રામ}
\end{center}

\textbf{FM રિસીવરમાં લિમિટરનો ઉપયોગ:}

\begin{itemize}
    \item \textbf{મુખ્ય કાર્ય}: એમ્પ્લિટ્યુડ વેરિએશન/નોઇઝ દૂર કરે છે.
    \item \textbf{ઓપરેશન}: સિગ્નલને ક્લિપ કરીને સ્થિર એમ્પ્લિટ્યુડ પ્રદાન કરે છે.
    \item \textbf{લાભો}: AM ઇન્ટરફેરન્સ દૂર કરે છે, SNR સુધારે છે, યોગ્ય FM ડિટેક્શન સુનિશ્ચિત કરે છે.
\end{itemize}
\end{solutionbox}

\begin{mnemonicbox}
\mnemonic{CARE: Clips Amplitude, Removes noise, Ensures constant signal}
\end{mnemonicbox}

\questionmarks{3(c)}{7}{સુપર હેટેરોડાઈન રીસીવરનો બ્લોક ડાયગ્રામ દોરો અને સમજાવો.}

\begin{solutionbox}
\textbf{સુપર હેટેરોડાઈન રિસીવર:}

\begin{center}
\begin{tikzpicture}[node distance=2cm, auto, >=latex, every node/.style={transform shape, scale=0.8}]
    % Top row
    \node [gtu block] (ant) {Antenna};
    \node [gtu block, right=of ant] (rf) {RF Amplifier};
    \node [gtu block, right=of rf] (mix) {Mixer};
    \node [gtu block, right=of mix] (if) {IF Amplifier};
    
    % Bottom row
    \node [gtu block, below=of mix] (lo) {Local Oscillator};
    \node [gtu block, below=of if] (det) {Detector};
    \node [gtu block, right=of det] (audio) {Audio Amp};
    \node [gtu block, right=of audio] (spk) {Speaker};
    \node [gtu block, below=of det] (agc) {AGC};

    % Connections
    \draw [gtu arrow] (ant) -- (rf);
    \draw [gtu arrow] (rf) -- (mix);
    \draw [gtu arrow] (lo) -- (mix);
    \draw [gtu arrow] (mix) -- (if);
    \draw [gtu arrow] (if) -- (det);
    \draw [gtu arrow] (det) -- (audio);
    \draw [gtu arrow] (audio) -- (spk);
    
    % AGC Feedback
    \draw [gtu arrow] (det) -- (agc);
    \draw [gtu arrow] (agc) -| (rf);
    \draw [gtu arrow] (agc) -| (if);
\end{tikzpicture}
\captionof{figure}{સુપર હેટેરોડાઈન રિસીવર}
\end{center}

\textbf{દરેક બ્લોકનું કાર્ય:}

\begin{itemize}
    \item \textbf{એન્ટેના}: RF સિગ્નલ્સ કેપ્ચર કરે છે.
    \item \textbf{RF એમ્પ્લિફાયર}: નબળા સિગ્નલ્સને એમ્પ્લિફાય કરે છે, સિલેક્ટિવિટી પ્રદાન કરે છે.
    \item \textbf{મિક્સર}: RF ને લોકલ ઓસિલેટર સાથે હેટરોડાઇનિંગ કરીને IF ઉત્પન્ન કરે છે.
    \item \textbf{લોકલ ઓસિલેટર}: સિગ્નલ ઉત્પન્ન કરે છે. $f_{LO} = f_{RF} + f_{IF}$.
    \item \textbf{IF એમ્પ્લિફાયર}: ફિક્સ્ડ ફ્રીક્વન્સી પર મુખ્ય એમ્પ્લિફિકેશન.
    \item \textbf{ડિટેક્ટર}: ઓડિયો એક્સટ્રેક્ટ કરે છે.
    \item \textbf{AGC}: સતત આઉટપુટ લેવલ જાળવે છે.
    \item \textbf{ઓડિયો એમ્પ્લિફાયર}: સ્પીકર ચલાવવા માટે એમ્પ્લિફાય કરે છે.
\end{itemize}
\end{solutionbox}

\begin{mnemonicbox}
\mnemonic{ARLMIDAS: Antenna Receives, Local Mixes, IF Delivers, Audio Sounds}
\end{mnemonicbox}

\questionmarks{3(a) OR}{3}{એનવેલોપ ડીટેક્ટરનો બ્લોક ડાયગ્રામ દોરો અને સમજાવો.}

\begin{solutionbox}
\textbf{એનવેલોપ ડિટેક્ટર:}

\begin{center}
\begin{tikzpicture}[american]
    \draw (0,2) to[short, o-] (1,2) to[D*, l=$D$] (3,2) to[short, -o] (4,2);
    \draw (3,2) to[C, l=$C$] (3,0);
    \draw (4,2) to[R, l=$R$] (4,0);
    \draw (0,0) to[short, o-o] (5,0);
    
    \draw (4,2) to[short, -o] (5,2);
    
    \node[left] at (0,1) {AM Input};
    \node[right] at (5,1) {Output};
\end{tikzpicture}
\captionof{figure}{એનવેલોપ ડિટેક્ટર}
\end{center}

\textbf{ઓપરેશન:}

\begin{enumerate}
    \item \textbf{ડાયોડ (D)}: AM સિગ્નલને રેક્ટિફાય કરે છે.
    \item \textbf{કેપેસિટર (C)}: પીક સુધી ચાર્જ થાય છે, કેરિયર ફિલ્ટર કરે છે.
    \item \textbf{રેસિસ્ટર (R)}: કેપેસિટરને ડિસ્ચાર્જ કરે છે, એનવેલોપને ફોલો કરે છે.
    \item \textbf{RC ટાઇમ કોન્સ્ટન્ટ}: $\frac{1}{f_c} \ll RC \ll \frac{1}{f_m}$.
\end{enumerate}
\end{solutionbox}

\begin{mnemonicbox}
\mnemonic{DRIVER: Diode Rectifies, RC Values Extract Envelope, Restores audio}
\end{mnemonicbox}

\questionmarks{3(b) OR}{4}{IF શું છે? તેનો અગત્યતા સમજાવો.}

\begin{solutionbox}
\textbf{ઇન્ટરમીડિએટ ફ્રીક્વન્સી (IF):}

\textbf{વ્યાખ્યા:} IF એ એક ફિક્સ્ડ ફ્રીક્વન્સી છે જેમાં આવતા RF સિગ્નલ્સ સુપરહેટેરોડાઈન રિસીવરમાં રૂપાંતરિત થાય છે.

\textbf{IF ની અગત્યતા:}

\begin{center}
\captionof{table}{IF ની અગત્યતા}
\begin{tabulary}{\linewidth}{|L|L|}
\hline
\textbf{પાસું} & \textbf{અગત્યતા} \\ \hline
\textbf{ફિક્સ્ડ ફ્રીક્વન્સી} & ઑપ્ટિમાઇઝ્ડ એમ્પ્લિફિકેશનની મંજૂરી આપે છે \\ \hline
\textbf{સુધારેલી સિલેક્ટિવિટી} & બેટર એડજેસન્ટ ચેનલ રિજેક્શન \\ \hline
\textbf{સ્થિર ગેઇન} & સાતત્યપૂર્ણ એમ્પ્લિફિકેશન \\ \hline
\textbf{ઇમેજ રિજેક્શન} & ઇમેજ ફ્રીક્વન્સી ઇન્ટરફેરન્સને અટકાવે છે \\ \hline
\textbf{સરળ ટ્યુનિંગ} & માત્ર લોકલ ઓસિલેટર ટ્યુન કરવું પડે છે \\ \hline
\end{tabulary}
\end{center}

\textbf{સામાન્ય IF વેલ્યુઝ:} AM માટે 455 kHz, FM માટે 10.7 MHz.
\end{solutionbox}

\begin{mnemonicbox}
\mnemonic{FIGS-ST: Fixed frequency, Improved selectivity, Gain stability, Simplified tuning}
\end{mnemonicbox}

\questionmarks{3(c) OR}{7}{FM detection માટેની ફેસ ડીસક્રીમીનેટર સર્કિટ સમજાવો.}

\begin{solutionbox}
\textbf{ફેઝ ડિસ્ક્રિમિનેટર:}

\begin{center}
\begin{tikzpicture}[american, scale=0.9, transform shape]
    % Transformer
    \draw (0,2) node[transformer core, american voltages] (T) {};
    
    \draw (T.A1) to[short, -o] (-2, 2.75);
    \draw (T.A2) to[short, -o] (-2, 1.25);
    \node[left] at (-2,2) {FM Input};
    
    % Secondary Side
    \draw (T.B1) to[short] (2, 2.75) to[D*, l=$D_1$] (4, 2.75) -- (5, 2.75);
    \draw (T.B2) to[short] (2, 1.25) to[D*, l=$D_2$] (4, 1.25) -- (5, 1.25);
    
    % Center tap
    \coordinate (CT) at ($(T.B1)!0.5!(T.B2)$);
    \draw (CT) to[C, l=$C_c$] (1.5, -1) -- (-1.5, -1) to[short] (-1.5, 2.75);
    \draw (T.A1) to[short, *-] (0, 3.5) to[C, l=$C_c$] (CT |- 0,3.5) -- (CT);
    
    % Output circuit
    \draw (5, 2.75) to[R, l=$R_1$] (5, 2);
    \draw (5, 2) to[C, l=$C_1$] (5, 2.75);
    
    \draw (5, 1.25) to[R, l=$R_2$] (5, 2);
    \draw (5, 2) to[C, l=$C_2$] (5, 1.25);
    
    \draw (5,2) to[short, -o] (7, 2) node[right] {Output};
    \draw (5, 1.25) -- (5, 0.5) node[ground]{};
\end{tikzpicture}
\captionof{figure}{ફેઝ ડિસ્ક્રિમિનેટર}
\end{center}

\textbf{ઓપરેશન:}
\begin{enumerate}
    \item \textbf{સેન્ટર-ટેપ્ડ ટ્રાન્સફોર્મર} 180 ફેઝ ડિફરન્સ બનાવે છે.
    \item \textbf{રેઝોનન્સ}: $f_c$ પર, આઉટપુટ શૂન્ય છે.
    \item \textbf{ફ્રીક્વન્સી ડિએશન}: જેમ ફ્રીક્વન્સી બદલાય છે, આઉટપુટ વોલ્ટેજ પ્રમાણસર બદલાય છે.
\end{enumerate}

\textbf{ફાયદાઓ:} સારી રેખીયતા અને ઓછું ડિસ્ટોર્શન.
\end{solutionbox}

\begin{mnemonicbox}
\mnemonic{PERFECT: Phase Ensures Rectification For Extracting Carrier Transitions}
\end{mnemonicbox}

\questionmarks{4(a)}{3}{ક્વોન્ટઆઈજાશન રીત અને તેની ઉપયોગીતા સમજાવો.}

\begin{solutionbox}
\textbf{ક્વોન્ટિઝેશન પ્રોસેસ:}

ક્વોન્ટિઝેશન એ સતત એનાલોગ મૂલ્યોને ડિસ્ક્રીટ ડિજિટલ લેવલ્સમાં મેપિંગ કરવાની પ્રક્રિયા છે.

\textbf{ક્વોન્ટિઝેશનની ઉપયોગીતા:}

\begin{center}
\captionof{table}{ક્વોન્ટિઝેશનની ઉપયોગીતા}
\begin{tabulary}{\linewidth}{|L|L|}
\hline
\textbf{ઉપયોગીતા} & \textbf{સમજૂતી} \\ \hline
\textbf{ડિજિટલ પ્રોસેસિંગ} & ડિજિટલ સ્ટોરેજ સક્ષમ કરે છે \\ \hline
\textbf{એરર કંટ્રોલ} & એરર ડિટેક્શનની મંજૂરી આપે છે \\ \hline
\textbf{નોઇઝ ઇમ્યુનિટી} & ડિજિટલ સિગ્નલ્સ નોઇઝ માટે પ્રતિરોધક છે \\ \hline
\textbf{સ્ટોરેજ એફિશિયન્સી} & ડેટા સ્ટોરેજમાં કાર્યક્ષમ છે \\ \hline
\end{tabulary}
\end{center}
\end{solutionbox}

\begin{mnemonicbox}
\mnemonic{DENSE: Digital conversion, Error control, Noise immunity, Storage, Efficient transmission}
\end{mnemonicbox}

\questionmarks{4(b)}{4}{ડેલ્ટા અને એડપટીવ ડેલ્ટા મોડયુલેશનનો તફાવત જણાવો.}

\begin{solutionbox}
\textbf{DM અને ADM વચ્ચે તફાવત:}

\begin{center}
\captionof{table}{DM vs ADM}
\begin{tabulary}{\linewidth}{|L|L|L|}
\hline
\textbf{પેરામીટર} & \textbf{ડેલ્ટા મોડ્યુલેશન (DM)} & \textbf{એડેપ્ટિવ ડેલ્ટા મોડ્યુલેશન (ADM)} \\ \hline
\textbf{સ્ટેપ સાઇઝ} & ફિક્સ્ડ & વેરિએબલ (સિગ્નલને અનુકૂળ) \\ \hline
\textbf{સ્લોપ ઓવરલોડ} & સામાન્ય & ઘટાડેલું \\ \hline
\textbf{ગ્રેન્યુલર નોઇઝ} & વધારે & ઓછું \\ \hline
\textbf{જટિલતા} & સરળ & મધ્યમ \\ \hline
\textbf{બિટ રેટ} & વધુ & ઓછું (સમાન ક્વોલિટી માટે) \\ \hline
\end{tabulary}
\end{center}
\end{solutionbox}

\begin{mnemonicbox}
\mnemonic{SAVAGES: Step size, Adaptable, Variable tracking, Avoids overload, Granular noise reduction}
\end{mnemonicbox}

\questionmarks{4(c)}{7}{PCM system નો બ્લોક ડાયગ્રામ દોરો અને સમજાવો.}

\begin{solutionbox}
\textbf{PCM સિસ્ટમ બ્લોક ડાયાગ્રામ:}

\begin{center}
\begin{tikzpicture}[node distance=1.5cm, auto, >=latex]
    % Transmitter
    \node [gtu block] (in) {Input};
    \node [gtu block, right=of in] (lpf) {Anti-aliasing Filter};
    \node [gtu block, right=of lpf] (sh) {Sample \& Hold};
    \node [gtu block, right=of sh] (quant) {Quantizer};
    \node [gtu block, below=of quant] (enc) {Encoder};
    \node [gtu block, left=of enc] (p2s) {Parallel to Serial};
    \node [right=0.5cm of p2s] (tx) {Tx};
    
    % Receiver
    \node [gtu block, below=of p2s] (s2p) {Serial to Parallel};
    \node [left=0.5cm of s2p] (rx) {Rx};
    \node [gtu block, right=of s2p] (dec) {Decoder};
    \node [gtu block, right=of dec] (recon) {Reconstruction Filter};
    \node [gtu block, right=of recon] (out) {Output};

    % Connections
    \draw [gtu arrow] (in) -- (lpf);
    \draw [gtu arrow] (lpf) -- (sh);
    \draw [gtu arrow] (sh) -- (quant);
    \draw [gtu arrow] (quant) -- (enc);
    \draw [gtu arrow] (enc) -- (p2s);
    
    \draw [dashed, ->] (p2s) -- (s2p) node[midway, right] {Channel};
    
    \draw [gtu arrow] (s2p) -- (dec);
    \draw [gtu arrow] (dec) -- (recon);
    \draw [gtu arrow] (recon) -- (out);
\end{tikzpicture}
\captionof{figure}{PCM સિસ્ટમ}
\end{center}

\textbf{PCM ટ્રાન્સમીટર:} ફિલ્ટર, સેમ્પલ & હોલ્ડ, ક્વોન્ટાઇઝર, એન્કોડર નો ઉપયોગ થાય છે.

\textbf{PCM રિસીવર:} ડિકોડર અને રિકન્સ્ટ્રક્શન ફિલ્ટર દ્વારા સિગ્નલ પુનઃપ્રાપ્ત થાય છે.
\end{solutionbox}

\begin{mnemonicbox}
\mnemonic{SAFE-PETS: Sample, Amplify, Filter, Encode, Pulse train, Extract, Transform, Smooth}
\end{mnemonicbox}

\questionmarks{4(a) OR}{3}{ક્વોન્ટઆઈજાશનની વ્યાખ્યા આપો. નોન યુનેફોર્મ ક્વોન્ટઆઈજાશન ટૂંકમાં સમજાવો.}

\begin{solutionbox}
\textbf{ક્વોન્ટિઝેશન:} એનાલોગ સેમ્પલ્સને ડિસ્ક્રીટ લેવલ્સમાં રૂપાંતરિત કરવાની પ્રક્રિયા.

\textbf{નોન-યુનિફોર્મ ક્વોન્ટિઝેશન:}

\begin{itemize}
    \item અસમાન સ્ટેપ સાઇઝનો ઉપયોગ કરે છે.
    \item નાના સિગ્નલ્સ માટે નાના સ્ટેપ્સ (SNR સુધારે છે).
    \item કોમ્પેન્ડિંગ (Companding) દ્વારા અમલ થાય છે.
\end{itemize}

\begin{center}
\begin{tikzpicture}
    \draw[->] (0,0) -- (4,0) node[right] {Input};
    \draw[->] (0,0) -- (0,4) node[above] {Output};
    
    \draw[blue, thick] plot [domain=0:3.5, samples=100] (\x, {ln(\x+1)*1.5});
    \node at (2.5, 1.5) {Compressor Curve};
    
    \draw[dashed] (0,0) -- (3.5, 3.5);
\end{tikzpicture}
\captionof{figure}{નોન-યુનિફોર્મ ક્વોન્ટિઝેશન}
\end{center}
\end{solutionbox}

\begin{mnemonicbox}
\mnemonic{CLASP: Compressed Levels, Adaptive Steps, Small steps for small signals, Perceptual matching}
\end{mnemonicbox}

\questionmarks{4(b) OR}{4}{એડપટીવ ડેલ્ટા મોડયુલેશન તેની એપ્લિકેસન સાથે સમજાવો.}

\begin{solutionbox}
\textbf{એડેપ્ટિવ ડેલ્ટા મોડ્યુલેશન (ADM):}

\begin{center}
\begin{tikzpicture}[node distance=1.5cm, auto, >=latex, every node/.style={transform shape, scale=0.8}]
    \node [gtu block] (comp) {Comparator};
    \node [left=1cm of comp] (in) {Input};
    \node [gtu block, right=of comp] (quant) {1-bit Quantizer};
    \node [right=1cm of quant] (out) {Output};
    
    \node [gtu block, below=of quant] (logic) {Step Size Logic};
    \node [gtu block, left=of logic] (int) {Integrator};
    
    \draw [gtu arrow] (in) -- (comp);
    \draw [gtu arrow] (comp) -- (quant);
    \draw [gtu arrow] (quant) -- (out);
    
    \draw [gtu arrow] (quant) -- (logic);
    \draw [gtu arrow] (logic) -- (int);
    \draw [gtu arrow] (int) -| node[pos=0.9]{-} (comp);
\end{tikzpicture}
\captionof{figure}{એડેપ્ટિવ ડેલ્ટા મોડ્યુલેશન}
\end{center}

\textbf{ઓપરેશન:}
\begin{itemize}
    \item સ્લોપના આધારે સ્ટેપ સાઇઝ બદલાય છે.
    \item ઝડપી ફેરફારો માટે સ્ટેપ સાઇઝ વધે છે, ધીમા માટે ઘટે છે.
    \item આ ઓવરલોડ અને નોઇઝ ઘટાડે છે.
\end{itemize}

\textbf{એપ્લિકેશન્સ:} સ્પીચ અને ઓડિયો કોમ્પ્રેશન.
\end{solutionbox}

\begin{mnemonicbox}
\mnemonic{ADAPT: Automatically Decides Appropriate Pulse Transitions}
\end{mnemonicbox}

\questionmarks{4(c) OR}{7}{સેમ્પલીંગ શું છે? સેમ્પલીંગના પ્રકારોને ટુંકમાં સમજાવો.}

\begin{solutionbox}
\textbf{સેમ્પલિંગ:} સતત-ટાઇમ સિગ્નલને ડિસ્ક્રીટ-ટાઇમમાં રૂપાંતરિત કરવું.

\textbf{સેમ્પલિંગના પ્રકારો:}

\begin{center}
\begin{tikzpicture}
    % Ideal
    \begin{scope}[xshift=0cm]
        \draw[->] (0,0) -- (3,0);
        \draw[->] (0,0) -- (0,2);
        \foreach \x in {0.5, 1.0, 1.5, 2.0, 2.5}
            \draw[thick, blue] (\x, 0) -- (\x, {1.5*sin(\x r*2)});
        \node[below] at (1.5, -0.2) {Ideal};
    \end{scope}

    % Natural
    \begin{scope}[xshift=3.5cm]
        \draw[->] (0,0) -- (3,0);
        \foreach \x in {0.5, 1.0, 1.5, 2.0, 2.5}
        {
            \fill[blue!20] (\x-0.1, 0) rectangle (\x+0.1, {1.5});
            \draw[blue] (\x-0.1, 0) -- (\x-0.1, 1.5) -- (\x+0.1, 1.6) -- (\x+0.1, 0);
        }
        \node[below] at (1.5, -0.2) {Natural};
    \end{scope}

    % Flat-top
    \begin{scope}[xshift=7cm]
        \draw[->] (0,0) -- (3,0);
        \foreach \x in {0.5, 1.0, 1.5, 2.0, 2.5}
            \draw[fill=blue!20] (\x-0.1, 0) rectangle (\x+0.1, {1.5});
        \node[below] at (1.5, -0.2) {Flat-top};
    \end{scope}
\end{tikzpicture}
\captionof{figure}{સેમ્પલિંગના પ્રકારો}
\end{center}

\begin{itemize}
    \item \textbf{આદર્શ}: ઇમ્પલ્સીસ (સૈદ્ધાંતિક).
    \item \textbf{નેચરલ}: પલ્સનો આકાર સિગ્નલ જેવો હોય છે.
    \item \textbf{ફ્લેટ-ટોપ}: પલ્સ દરમિયાન એમ્પ્લિટ્યુડ સ્થિર રહે છે.
\end{itemize}
\end{solutionbox}

\begin{mnemonicbox}
\mnemonic{INFS: Ideal (impulses), Natural (follows signal), Flat-top (constant), Sufficient rate}
\end{mnemonicbox}

\questionmarks{5(a)}{3}{બીટરેટ અને બોડરેટ વ્યાખ્યાઈત કરો.}

\begin{solutionbox}
\textbf{બિટ રેટ અને બોડ રેટ:}

\begin{center}
\captionof{table}{બિટ રેટ vs બોડ રેટ}
\begin{tabulary}{\linewidth}{|L|L|L|}
\hline
\textbf{પેરામીટર} & \textbf{વ્યાખ્યા} & \textbf{સૂત્ર} \\ \hline
\textbf{બિટ રેટ} & પ્રતિ સેકન્ડ બિટ્સની સંખ્યા & $R = fs \times n$ \\ \hline
\textbf{બોડ રેટ} & પ્રતિ સેકન્ડ સિમ્બોલ્સની સંખ્યા & $B = fs$ \\ \hline
\end{tabulary}
\end{center}
\end{solutionbox}

\begin{mnemonicbox}
\mnemonic{BBSM: Bits per second, Baud for Symbols, Modulation determines relationship}
\end{mnemonicbox}

\questionmarks{5(b)}{4}{DPCM નું કાર્ય સમજાવો.}

\begin{solutionbox}
\textbf{DPCM (Differential PCM):}

\begin{center}
\begin{tikzpicture}[node distance=1.5cm, auto, >=latex, every node/.style={transform shape, scale=0.8}]
    \node [gtu block] (sum) {Sum};
    \node [left=1cm of sum] (in) {Input};
    \node [gtu block, right=of sum] (quant) {Quantizer};
    \node [gtu block, right=of quant] (enc) {Encoder};
    \node [right=1cm of enc] (out) {Output};
    
    \node [gtu block, below=of quant] (pred) {Predictor};
    \node [gtu block, left=of pred] (sum2) {+};
    
    \draw [gtu arrow] (in) -- (sum);
    \draw [gtu arrow] (sum) -- (quant);
    \draw [gtu arrow] (quant) -- (enc);
    \draw [gtu arrow] (enc) -- (out);
    
    \draw [gtu arrow] (quant) -- (sum2);
    \draw [gtu arrow] (sum2) -- (pred);
    \draw [gtu arrow] (pred) -| node[pos=0.9]{-} (sum);
    \draw [gtu arrow] (pred) -- (sum2);
\end{tikzpicture}
\captionof{figure}{DPCM ટ્રાન્સમીટર}
\end{center}

\textbf{કાર્ય:} વર્તમાન સેમ્પલ અને અનુમાનિત સેમ્પલ વચ્ચેનો તફાવત એન્કોડ કરે છે, જેથી કાર્યક્ષમતા વધે છે.
\end{solutionbox}

\begin{mnemonicbox}
\mnemonic{DEEP: Difference Encoded, Efficient Prediction, Exploits correlation, Preserves quality}
\end{mnemonicbox}

\questionmarks{5(c)}{7}{બાઈનરી ડેટા 1011001 નીચે પ્રમાણેની લાઈન કોડિંગ ટેકનીકથી ટ્રાન્સમીટ થાય છે... બધા માટે વેવ ફોર્મ દોરો.}

\begin{solutionbox}
\textbf{લાઈન કોડિંગ વેવફોર્મ્સ (Data: 1 0 1 1 0 0 1):}

\begin{center}
\begin{tikzpicture}[xscale=1.5, yscale=0.8]
    \draw[help lines] (0,-4) grid (7,4);
    
    \node[left] at (0, 3.5) {Unipolar NRZ};
    \node[left] at (0, 2) {Polar NRZ};
    \node[left] at (0, 0.5) {AMI};
    \node[left] at (0, -1) {Manchester};
    
    \foreach \x in {0,...,7} \draw[dashed] (\x,-2) -- (\x, 4);
    \foreach \x/\v in {0/1, 1/0, 2/1, 3/1, 4/0, 5/0, 6/1} \node[above] at (\x+0.5, 4) {\v};

    % Unipolar NRZ
    \draw[blue, thick] (0,4) -- (1,4) -- (1,3) -- (2,3) -- (2,4) -- (3,4) -- (4,4) -- (4,3) -- (6,3) -- (6,4) -- (7,4);

    % Polar NRZ
    \draw[red, thick] (0,2.5) -- (1,2.5) -- (1,1.5) -- (2,1.5) -- (2,2.5) -- (4,2.5) -- (4,1.5) -- (6,1.5) -- (6,2.5) -- (7,2.5);
    
    % AMI
    \draw[green!60!black, thick] (0,1) -- (1,1) -- (1,0.5) -- (2,0.5) -- (2,0) -- (3,0) -- (3,1) -- (4,1) -- (4,0.5) -- (6,0.5) -- (6,0) -- (7,0);
    
    % Manchester
    \draw[orange, thick] 
    (0, -0.5) -- (0.5, -0.5) -- (0.5, -1.5) -- (1, -1.5) -- 
    (1, -1.5) -- (1.5, -1.5) -- (1.5, -0.5) -- (2, -0.5) -- 
    (2, -0.5) -- (2.5, -0.5) -- (2.5, -1.5) -- (3, -1.5) -- 
    (3, -0.5) -- (3.5, -0.5) -- (3.5, -1.5) -- (4, -1.5) -- 
    (4, -1.5) -- (4.5, -1.5) -- (4.5, -0.5) -- (5, -0.5) -- 
    (5, -1.5) -- (5.5, -1.5) -- (5.5, -0.5) -- (6, -0.5) -- 
    (6, -0.5) -- (6.5, -0.5) -- (6.5, -1.5) -- (7, -1.5);
\end{tikzpicture}
\captionof{figure}{લાઈન કોડિંગ વેવફોર્મ્સ}
\end{center}
\end{solutionbox}

\begin{mnemonicbox}
\mnemonic{UPAM: Unipolar, Polar, AMI, Manchester encoding options}
\end{mnemonicbox}

\questionmarks{5(a) OR}{3}{RZ અને NRZ કોડિંગ ઉદાહરણ સાથેસમજાવો.}

\begin{solutionbox}
\textbf{RZ અને NRZ ની તુલના:}

\begin{center}
\captionof{table}{RZ vs NRZ}
\begin{tabulary}{\linewidth}{|L|L|L|}
\hline
\textbf{પેરામીટર} & \textbf{RZ} & \textbf{NRZ} \\ \hline
\textbf{સિગ્નલ લેવલ્સ} & શૂન્ય પર પાછું ફરે છે & લેવલ જાળવે છે \\ \hline
\textbf{બેન્ડવિડ્થ} & વધારે & ઓછી \\ \hline
\textbf{સિન્ક્રોનાઇઝેશન} & સારું & નબળું \\ \hline
\end{tabulary}
\end{center}
\end{solutionbox}

\begin{mnemonicbox}
\mnemonic{BPSIDC: Bandwidth, Power, Synchronization, Implementation, DC component}
\end{mnemonicbox}

\questionmarks{5(b) OR}{4}{ડેલ્ટા મોડયુલેશન ટૂંકમા સમજાવો.}

\begin{solutionbox}
\textbf{ડેલ્ટા મોડ્યુલેશન (DM):}

\begin{itemize}
    \item 1 બિટનો ઉપયોગ કરીને તફાવત એન્કોડ કરે છે.
    \item જો ઇનપુટ > અનુમાન, તો 1.
    \item જો ઇનપુટ < અનુમાન, તો 0.
\end{itemize}

\textbf{મર્યાદાઓ:} સ્લોપ ઓવરલોડ અને ગ્રેન્યુલર નોઇઝ.
\end{solutionbox}

\begin{mnemonicbox}
\mnemonic{SIDE: Single-bit, Integrates Differences, Encodes changes}
\end{mnemonicbox}

\questionmarks{5(c) OR}{7}{PCM-TDM સિસ્ટમ સમજાવો.}

\begin{solutionbox}
\textbf{PCM-TDM સિસ્ટમ:}

\begin{center}
\begin{tikzpicture}[node distance=1.5cm, auto, >=latex, every node/.style={transform shape, scale=0.8}]
    % Transmitter
    \node [gtu block] (mux) {Multiplexer};
    \node [left=1cm of mux] (in) {Inputs (Ch 1..N)};
    \node [gtu block, right=of mux] (enc) {PCM Encoder};
    \node [gtu block, right=of enc] (tx) {Tx};
    
    % Receiver
    \node [gtu block, below=of tx] (rx) {Rx};
    \node [gtu block, left=of rx] (dec) {PCM Decoder};
    \node [gtu block, left=of dec] (demux) {Demultiplexer};
    \node [left=1cm of demux] (out) {Outputs};
    
    \draw [gtu arrow] (in) -- (mux);
    \draw [gtu arrow] (mux) -- (enc);
    \draw [gtu arrow] (enc) -- (tx);
    
    \draw [dashed] (tx) -- (rx);
    
    \draw [gtu arrow] (rx) -- (dec);
    \draw [gtu arrow] (dec) -- (demux);
    \draw [gtu arrow] (demux) -- (out);
\end{tikzpicture}
\captionof{figure}{PCM-TDM બ્લોક ડાયાગ્રામ}
\end{center}

\textbf{ઓપરેશન:} અનેક ચેનલ્સના સેમ્પલ્સ TDM દ્વારા ઇન્ટરલીવ થાય છે અને પછી PCM એન્કોડ થાય છે.
\end{solutionbox}

\begin{mnemonicbox}
\mnemonic{MOST-FDR: Multiplex, Quantize, Sample, Transmit, Frame, Demultiplex, Reconstruct}
\end{mnemonicbox}

\end{document}
