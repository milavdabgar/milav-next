\documentclass[10pt,a4paper]{article}

% content/resources/templates/preamble.tex
\usepackage[margin=0.6in]{geometry}
\author{Milav Dabgar}
\usepackage{amsmath,amssymb,amsthm}
\usepackage{booktabs}
\usepackage{multirow}
\usepackage{xcolor}
\usepackage{tcolorbox}
\tcbuselibrary{breakable,skins}
\usepackage[colorlinks=true,linkcolor=blue]{hyperref}
\usepackage{titlesec}
\usepackage{enumitem}
\usepackage{tikz}
\usepackage{pgfplots}
\usepackage{circuitikz}
\usepackage[version=4]{mhchem}
\usepackage{longtable}
\usepackage{array}
\usepackage{float}
\usepackage{caption}
\usepackage{listings}

\lstset{
  basicstyle=\small\ttfamily,
  breaklines=true,
  breakatwhitespace=false,
  postbreak=\mbox{\textcolor{red}{$\hookrightarrow$}\space},
  float=false,
  numbers=left,
  numberstyle=\tiny\color{gray},
  numbersep=10pt,
  xleftmargin=2em,
  keywordstyle=\color{blue},
  commentstyle=\color{green!60!black},
  stringstyle=\color{purple},
  backgroundcolor=\color{gray!5},
  showstringspaces=false,
  tabsize=2,
  captionpos=b,
  keepspaces=true,
  columns=flexible
}

\pgfplotsset{compat=1.18}
\usetikzlibrary{shapes,arrows,positioning,calc,patterns,decorations.pathmorphing,decorations.markings,arrows.meta}

% Color scheme
\definecolor{headcolor}{RGB}{0,102,204}
\definecolor{keycolor}{RGB}{220,20,60}
\definecolor{solutioncolor}{RGB}{34,139,34}
\definecolor{mnemoniccolor}{RGB}{148,0,211}
\definecolor{codecolor}{RGB}{0,0,100}

% Spacing
\setlength{\parskip}{3pt}
\setlist[itemize]{nosep}
\setlist[enumerate]{nosep}

% Title formatting
\titleformat{\section}{\Large\bfseries\color{headcolor}}{\thesection}{1em}{}
\titleformat{\subsection}{\large\bfseries\color{headcolor}}{\thesubsection}{1em}{}

% Pandoc tightlist compatibility
\providecommand{\tightlist}{%
  \setlength{\itemsep}{0pt}\setlength{\parskip}{0pt}}

% Pandoc longtable compatibility
\newcounter{none}
\def\thenone{}


% content/resources/templates/gujarati-boxes.tex
\usepackage{fontspec}
\usepackage{polyglossia}

% Set Gujarati as main language (document is primarily in Gujarati)
% Note: gloss-gujarati.ldf doesn't exist in polyglossia, but it will use hyphenation patterns
\setdefaultlanguage{gujarati}
\setotherlanguage{english}

% Configure Gujarati font properly
% Use Language=Default to prevent polyglossia from trying to add language-specific features
% that don't exist for Gujarati, which causes "empty feature" warnings
\newfontfamily\gujaratifont[Script=Gujarati,AutoFakeBold=2.5,AutoFakeSlant=0.3]{Noto Sans Gujarati}
\setmainfont[Script=Gujarati,AutoFakeBold=2.5,AutoFakeSlant=0.3]{Noto Sans Gujarati}
% Use Noto Sans Gujarati for monospace to support Gujarati in text
\setmonofont[Scale=0.9]{Noto Sans Gujarati}

% Configure English to use the same font
\newfontfamily\englishfont[Script=Gujarati,AutoFakeBold=2.5,AutoFakeSlant=0.3]{Noto Sans Gujarati}

% Translations for polyglossia
\gappto\captionsgujarati{
  \renewcommand{\tablename}{કોષ્ટક}
  \renewcommand{\figurename}{આકૃતિ}
}

% Helper for TikZ nodes to ensure Gujarati font
\newcommand{\gu}[1]{{\gujaratifont #1}}

% Custom environments
\newtcolorbox{solutionbox}{
    breakable,
    enhanced,
    colback=solutioncolor!5!white,
    colframe=solutioncolor!75!black,
    fonttitle=\bfseries,
    title=જવાબ
}

\newtcolorbox{solutionboxnobreak}{
 colback=solutioncolor!5!white,
 colframe=solutioncolor!75!black,
 fonttitle=\bfseries,
 title=જવાબ
}

\newtcolorbox{keyformula}{
 breakable,
 enhanced,
 colback=keycolor!5!white,
 colframe=keycolor!75!black,
 fonttitle=\bfseries,
 title=રાસાયણિક સમીકરણ/સૂત્ર
}

\newtcolorbox{mnemonicbox}{
 breakable,
 enhanced,
 colback=mnemoniccolor!5!white,
 colframe=mnemoniccolor!75!black,
 fonttitle=\bfseries,
 title=મેમરી ટ્રીક
}


\begin{document}

\begin{center}
{\Huge\bfseries\color{headcolor} Subject Name (Gujarati)}\\[5pt]
{\LARGE 4331104 -- Summer 2025}\\[3pt]
{\large Semester 1 Study Material}\\[3pt]
{\normalsize\textit{Detailed Solutions and Explanations}}
\end{center}

\vspace{10pt}

\subsection*{પ્રશ્ન 1(અ) [3
ગુણ]}\label{uxaaauxab0uxab6uxaa8-1uxa85-3-uxa97uxaa3}

\textbf{એનાલોગ સિગ્નલ અને ડિજિટલ સિગ્નલની સરખામણી કરો.}

\begin{solutionbox}

{\def\LTcaptype{none} % do not increment counter
\begin{longtable}[]{@{}lll@{}}
\toprule\noalign{}
પેરામીટર & એનાલોગ સિગ્નલ & ડિજિટલ સિગ્નલ \\
\midrule\noalign{}
\endhead
\bottomrule\noalign{}
\endlastfoot
\textbf{પ્રકૃતિ} & સતત તરંગરૂપ & અલગ અલગ વેલ્યુ (0 અને 1) \\
\textbf{એમ્પ્લિટ્યુડ} & અનંત વિવિધતાઓ & નિશ્ચિત અલગ સ્તરો \\
\textbf{નોઇઝ ઇફેક્ટ} & વધુ સંવેદનશીલ & ઓછી સંવેદનશીલ \\
\textbf{બેન્ડવિડ્થ} & ઓછી બેન્ડવિડ્થ જરૂરી & વધુ બેન્ડવિડ્થ જરૂરી \\
\textbf{સિક્યુરિટી} & ઓછી સુરક્ષિત & વધુ સુરક્ષિત \\
\end{longtable}
}

\begin{itemize}
\tightlist
\item
  \textbf{સિગ્નલ પ્રકાર}: એનાલોગ સિગ્નલ સતત હોય છે, ડિજિટલ સિગ્નલ અલગ અલગ હોય
  છે
\item
  \textbf{નોઇઝ રેઝિસ્ટન્સ}: ડિજિટલ સિગ્નલમાં નોઇઝ સામે વધુ પ્રતિકાર હોય છે
\end{itemize}

\end{solutionbox}
\begin{mnemonicbox}
``ABCD - Analog Bad for noise, Continuous; Digital
Discrete, Clean signals''

\end{mnemonicbox}
\begin{center}\rule{0.5\linewidth}{0.5pt}\end{center}

\subsection*{પ્રશ્ન 1(બ) [4
ગુણ]}\label{uxaaauxab0uxab6uxaa8-1uxaac-4-uxa97uxaa3}

\textbf{PAM, PWM અને PPM ની સરખામણી કરો.}

\begin{solutionbox}

{\def\LTcaptype{none} % do not increment counter
\begin{longtable}[]{@{}
  >{\raggedright\arraybackslash}p{(\linewidth - 6\tabcolsep) * \real{0.4231}}
  >{\raggedright\arraybackslash}p{(\linewidth - 6\tabcolsep) * \real{0.1923}}
  >{\raggedright\arraybackslash}p{(\linewidth - 6\tabcolsep) * \real{0.1923}}
  >{\raggedright\arraybackslash}p{(\linewidth - 6\tabcolsep) * \real{0.1923}}@{}}
\toprule\noalign{}
\begin{minipage}[b]{\linewidth}\raggedright
પેરામીટર
\end{minipage} & \begin{minipage}[b]{\linewidth}\raggedright
PAM
\end{minipage} & \begin{minipage}[b]{\linewidth}\raggedright
PWM
\end{minipage} & \begin{minipage}[b]{\linewidth}\raggedright
PPM
\end{minipage} \\
\midrule\noalign{}
\endhead
\bottomrule\noalign{}
\endlastfoot
\textbf{પૂરું નામ} & Pulse Amplitude Modulation & Pulse Width Modulation &
Pulse Position Modulation \\
\textbf{મોડ્યુલેટેડ પેરામીટર} & એમ્પ્લિટ્યુડ & પહોળાઈ/અવધિ & સ્થાન/સમય \\
\textbf{નોઇઝ ઇમ્યુનિટી} & ખરાબ & સારી & ઉત્તમ \\
\textbf{બેન્ડવિડ્થ} & લઘુત્તમ & મધ્યમ & મહત્તમ \\
\textbf{પાવર કન્ઝમ્પશન} & વધુ & મધ્યમ & ઓછી \\
\end{longtable}
}

\textbf{ડાયાગ્રામ:}

\begin{verbatim}
PAM: |▄▄|  |▄▄▄| |▄|     Amplitude varies
PWM: |▄| |▄▄▄| |▄▄|      Width varies  
PPM: |▄|  |▄| |▄|        Position varies
\end{verbatim}

\begin{itemize}
\tightlist
\item
  \textbf{મોડ્યુલેશન પેરામીટર}: દરેક પ્રકાર પલ્સની અલગ લાક્ષણિકતાઓ મોડ્યુલેટ કરે છે
\item
  \textbf{એપ્લિકેશન}: PWM મોટર કંટ્રોલમાં, PPM રેડિયો કંટ્રોલ સિસ્ટમમાં વપરાય છે
\end{itemize}

\end{solutionbox}
\begin{mnemonicbox}
``PAM-Amplitude, PWM-Width, PPM-Position - AWP''

\end{mnemonicbox}
\begin{center}\rule{0.5\linewidth}{0.5pt}\end{center}

\subsection*{પ્રશ્ન 1(ક) [7
ગુણ]}\label{uxaaauxab0uxab6uxaa8-1uxa95-7-uxa97uxaa3}

\textbf{મોડ્યુલેશનની જરૂરિયાત વિગતવાર સમજાવો. જો કેરિયર સિગ્નલની આવૃત્તિ 1 MHz
હોય તો એન્ટેનાની ઊંચાઈની ગણતરી કરો.}

\begin{solutionbox}

\textbf{મોડ્યુલેશનની જરૂરિયાત:}

{\def\LTcaptype{none} % do not increment counter
\begin{longtable}[]{@{}ll@{}}
\toprule\noalign{}
કારણ & સમજૂતી \\
\midrule\noalign{}
\endhead
\bottomrule\noalign{}
\endlastfoot
\textbf{એન્ટેના સાઇઝ રિડક્શન} & વ્યવહારિક એન્ટેના માપ શક્ય બનાવે છે \\
\textbf{ફ્રીક્વન્સી ટ્રાન્સલેશન} & સિગ્નલને યોગ્ય આવૃત્તિ રેન્જમાં ખસેડે છે \\
\textbf{મલ્ટિપ્લેક્સિંગ} & એક જ માધ્યમ પર અનેક સિગ્નલ મંજૂરી આપે છે \\
\textbf{નોઇઝ રિડક્શન} & સિગ્નલ-ટુ-નોઇઝ રેશિયો સુધારે છે \\
\textbf{પાવર એફિશિયન્સી} & વધુ સારી પાવર વિનિયોગ \\
\end{longtable}
}

\textbf{એન્ટેના ઊંચાઈની ગણતરી:} કાર્યક્ષમ રેડિએશન માટે, એન્ટેના ઊંચાઈ = λ/4

λ = c/f = (3 \times 10^{8})/(1 \times 10^{6}) = 300 મીટર

\textbf{એન્ટેના ઊંચાઈ} = λ/4 = 300/4 = \textbf{75 મીટર}

\begin{itemize}
\tightlist
\item
  \textbf{પ્રેક્ટિકલ એન્ટેના}: મોડ્યુલેશન વગર, એન્ટેના અવ્યવહારિક રીતે મોટો હોત
\item
  \textbf{ફ્રીક્વન્સી શિફ્ટિંગ}: વધુ સારી પ્રોપેગેશન લાક્ષણિકતાઓ માટે મંજૂરી આપે છે
\end{itemize}

\end{solutionbox}
\begin{mnemonicbox}
``AFMNP - Antenna, Frequency, Multiplexing, Noise,
Power''

\end{mnemonicbox}
\begin{center}\rule{0.5\linewidth}{0.5pt}\end{center}

\subsection*{પ્રશ્ન 1(ક) OR [7
ગુણ]}\label{uxaaauxab0uxab6uxaa8-1uxa95-or-7-uxa97uxaa3}

\textbf{EM વેવ સ્પેક્ટ્રમના ફ્રીક્વન્સી બેન્ડ તેના એપ્લિકેશન ડોમેન સાથે લખો. ELF બેન્ડની
તરંગલંબાઈની ગણતરી કરો.}

\begin{solutionbox}

{\def\LTcaptype{none} % do not increment counter
\begin{longtable}[]{@{}llll@{}}
\toprule\noalign{}
બેન્ડ & આવૃત્તિ રેન્જ & તરંગલંબાઈ & એપ્લિકેશન \\
\midrule\noalign{}
\endhead
\bottomrule\noalign{}
\endlastfoot
\textbf{ELF} & 30-300 Hz & 10^{6}-10^{7} m & સબમરીન કમ્યુનિકેશન \\
\textbf{VLF} & 3-30 kHz & 10^{4}-10^{5} m & નેવિગેશન, ટાઇમ સિગ્નલ \\
\textbf{LF} & 30-300 kHz & 10^{3}-10^{4} m & AM બ્રોડકાસ્ટિંગ \\
\textbf{MF} & 300 kHz-3 MHz & 100-1000 m & AM રેડિયો \\
\textbf{HF} & 3-30 MHz & 10-100 m & શોર્ટ વેવ રેડિયો \\
\end{longtable}
}

\textbf{ELF તરંગલંબાઈની ગણતરી:}

\begin{itemize}
\tightlist
\item
  નીચી આવૃત્તિ: f_{1} = 30 Hz, λ_{1} = c/f_{1} = (3\times10^{8})/30 = \textbf{10^{7} મીટર}
\item
  ઉચ્ચી આવૃત્તિ: f_{2} = 300 Hz, λ_{2} = c/f_{2} = (3\times10^{8})/300 = \textbf{10^{6} મીટર}
\end{itemize}

\textbf{ELF તરંગલંબાઈ રેન્જ: 10^{6} થી 10^{7} મીટર}

\begin{itemize}
\tightlist
\item
  \textbf{એપ્લિકેશન ડોમેન}: દરેક બેન્ડ ચોક્કસ એપ્લિકેશન માટે યોગ્ય છે
\item
  \textbf{પ્રોપેગેશન}: નીચી આવૃત્તિઓમાં વધુ સારી ગ્રાઉન્ડ વેવ પ્રોપેગેશન હોય છે
\end{itemize}

\end{solutionbox}
\begin{mnemonicbox}
``Every Valuable Learning Makes Happiness - ELF થી
HF બેન્ડ''

\end{mnemonicbox}
\begin{center}\rule{0.5\linewidth}{0.5pt}\end{center}

\subsection*{પ્રશ્ન 2(અ) [3
ગુણ]}\label{uxaaauxab0uxab6uxaa8-2uxa85-3-uxa97uxaa3}

\textbf{AM અને FM ની સરખામણી કરો.}

\begin{solutionbox}

{\def\LTcaptype{none} % do not increment counter
\begin{longtable}[]{@{}lll@{}}
\toprule\noalign{}
પેરામીટર & AM & FM \\
\midrule\noalign{}
\endhead
\bottomrule\noalign{}
\endlastfoot
\textbf{મોડ્યુલેટેડ પેરામીટર} & એમ્પ્લિટ્યુડ & આવૃત્તિ \\
\textbf{બેન્ડવિડ્થ} & 2fm & 2(Δf + fm) \\
\textbf{નોઇઝ ઇમ્યુનિટી} & ખરાબ & સારી \\
\textbf{પાવર એફિશિયન્સી} & ઓછી (33.33\%) & વધુ \\
\textbf{સર્કિટ કોમ્પ્લેક્સિટી} & સરળ & જટિલ \\
\end{longtable}
}

\begin{itemize}
\tightlist
\item
  \textbf{બેન્ડવિડ્થ}: FM ને AM કરતાં ઘણી વધુ બેન્ડવિડ્થ જરૂરી છે
\item
  \textbf{ક્વોલિટી}: FM વધુ સારી ઓડિયો ક્વોલિટી પૂરી પાડે છે
\end{itemize}

\end{solutionbox}
\begin{mnemonicbox}
``AM-Amplitude સરળ, FM-Frequency જટિલ પણ વધુ સારી
ક્વોલિટી''

\end{mnemonicbox}
\begin{center}\rule{0.5\linewidth}{0.5pt}\end{center}

\subsection*{પ્રશ્ન 2(બ) [4
ગુણ]}\label{uxaaauxab0uxab6uxaa8-2uxaac-4-uxa97uxaa3}

\textbf{એમ્પ્લિટ્યુડ મોડ્યુલેટેડ વેવનું વેવફોર્મ દોરો.}

\begin{solutionbox}

\textbf{ડાયાગ્રામ:}

\begin{verbatim}
Modulating Signal:  ∼    ∼    ∼    ∼

Carrier Signal:     ∿∿∿∿∿∿∿∿∿∿∿∿∿∿

FM Wave:           ∿∿∿  ∿∿∿∿∿∿  ∿∿∿
                      Higher freq  Lower freq
                      when mod +ve when mod {-ve}
\end{verbatim}

\textbf{લાક્ષણિકતાઓ:}

\begin{itemize}
\tightlist
\item
  \textbf{એન્વેલોપ}: એન્વેલોપ મોડ્યુલેટિંગ સિગ્નલને અનુસરે છે
\item
  \textbf{કેરિયર ફ્રીક્વન્સી}: સમગ્ર સમય દરમિયાન સ્થિર રહે છે
\item
  \textbf{એમ્પ્લિટ્યુડ વેરિએશન}: એમ્પ્લિટ્યુડ મોડ્યુલેટિંગ સિગ્નલ સાથે બદલાય છે
\end{itemize}

\end{solutionbox}
\begin{mnemonicbox}
``Envelope Follows Message - EFM''

\end{mnemonicbox}
\begin{center}\rule{0.5\linewidth}{0.5pt}\end{center}

\subsection*{પ્રશ્ન 2(ક) [7
ગુણ]}\label{uxaaauxab0uxab6uxaa8-2uxa95-7-uxa97uxaa3}

\textbf{એમ્પ્લિટ્યુડ મોડ્યુલેશનની વ્યાખ્યા આપો અને ડબલ સાઇડબેન્ડ ફુલ કેરિયર (DSBFC)
એમ્પ્લિટ્યુડ મોડ્યુલેશન (AM) સિગ્નલ માટે ગાણિતિક અભિવ્યક્તિ મેળવો.}

\begin{solutionbox}

\textbf{વ્યાખ્યા:} એમ્પ્લિટ્યુડ મોડ્યુલેશન એ પ્રક્રિયા છે જેમાં કેરિયર સિગ્નલનું એમ્પ્લિટ્યુડ
મોડ્યુલેટિંગ સિગ્નલના તાત્કાલિક એમ્પ્લિટ્યુડ અનુસાર બદલાય છે.

\textbf{ગાણિતિક વ્યુત્પત્તિ:}

કેરિયર સિગ્નલ: ec(t) = Ec cos(ωct) મોડ્યુલેટિંગ સિગ્નલ: em(t) = Em cos(ωmt)

\textbf{AM સિગ્નલ અભિવ્યક્તિ:} eAM(t) = [Ec + Em cos(ωmt)] cos(ωct)
eAM(t) = Ec cos(ωct) + Em cos(ωmt) cos(ωct)

ત્રિકોણમિતિય સૂત્રનો ઉપયોગ: cos A cos B = ½[cos(A+B) + cos(A-B)]

\textbf{અંતિમ AM અભિવ્યક્તિ:} eAM(t) = Ec cos(ωct) + (Em/2) cos(ωc + ωm)t +
(Em/2) cos(ωc - ωm)t

\textbf{ઘટકો:}

\begin{itemize}
\tightlist
\item
  \textbf{કેરિયર કોમ્પોનન્ટ}: Ec cos(ωct)
\item
  \textbf{અપર સાઇડબેન્ડ}: (Em/2) cos(ωc + ωm)t\\
\item
  \textbf{લોઅર સાઇડબેન્ડ}: (Em/2) cos(ωc - ωm)t
\end{itemize}

\end{solutionbox}
\begin{mnemonicbox}
``Carrier Plus Upper Lower Sidebands - CPULS''

\end{mnemonicbox}
\begin{center}\rule{0.5\linewidth}{0.5pt}\end{center}

\subsection*{પ્રશ્ન 2(અ) OR [3
ગુણ]}\label{uxaaauxab0uxab6uxaa8-2uxa85-or-3-uxa97uxaa3}

\textbf{પ્રી-એમ્ફેસિસ અને ડી-એમ્ફેસિસની સરખામણી કરો.}

\begin{solutionbox}

{\def\LTcaptype{none} % do not increment counter
\begin{longtable}[]{@{}lll@{}}
\toprule\noalign{}
પેરામીટર & પ્રી-એમ્ફેસિસ & ડી-એમ્ફેસિસ \\
\midrule\noalign{}
\endhead
\bottomrule\noalign{}
\endlastfoot
\textbf{સ્થાન} & ટ્રાન્સમિટર પર & રીસીવર પર \\
\textbf{કાર્ય} & ઉચ્ચ આવૃત્તિઓ વધારે છે & ઉચ્ચ આવૃત્તિઓ ઘટાડે છે \\
\textbf{ફ્રીક્વન્સી રિસ્પોન્સ} & હાઇ પાસ લાક્ષણિકતા & લો પાસ લાક્ષણિકતા \\
\textbf{હેતુ} & S/N રેશિયો સુધારે છે & મૂળ સિગ્નલ પુનઃસ્થાપિત કરે છે \\
\textbf{ટાઇમ કોન્સ્ટન્ટ} & 75 μs (FM બ્રોડકાસ્ટિંગ) & 75 μs (FM બ્રોડકાસ્ટિંગ) \\
\end{longtable}
}

\begin{itemize}
\tightlist
\item
  \textbf{નોઇઝ રિડક્શન}: સંયુક્ત અસર મળેલ સિગ્નલમાં નોઇઝ ઘટાડે છે
\item
  \textbf{ફ્રીક્વન્સી રિસ્પોન્સ}: પૂરક લાક્ષણિકતાઓ
\end{itemize}

\end{solutionbox}
\begin{mnemonicbox}
``Pre-Boost, De-Cut - Noise Reduction Circuit''

\end{mnemonicbox}
\begin{center}\rule{0.5\linewidth}{0.5pt}\end{center}

\subsection*{પ્રશ્ન 2(બ) OR [4
ગુણ]}\label{uxaaauxab0uxab6uxaa8-2uxaac-or-4-uxa97uxaa3}

\textbf{ફ્રીક્વન્સી મોડ્યુલેટેડ વેવનું વેવફોર્મ દોરો.}

\begin{solutionbox}

\textbf{ડાયાગ્રામ:}

\begin{verbatim}
Modulating Signal:  ∼    ∼    ∼    ∼

Carrier Signal:     ∿∿∿∿∿∿∿∿∿∿∿∿∿∿

FM Wave:           ∿∿∿  ∿∿∿∿∿∿  ∿∿∿
                      Higher freq  Lower freq
                      when mod +ve when mod {-ve}
\end{verbatim}

\textbf{લાક્ષણિકતાઓ:}

\begin{itemize}
\tightlist
\item
  \textbf{કોન્સ્ટન્ટ એમ્પ્લિટ્યુડ}: એમ્પ્લિટ્યુડ સ્થિર રહે છે
\item
  \textbf{ફ્રીક્વન્સી વેરિએશન}: આવૃત્તિ મોડ્યુલેટિંગ સિગ્નલ સાથે બદલાય છે
\item
  \textbf{ફેઝ કોન્ટિન્યુઇટી}: ફેઝ સતત રહે છે
\end{itemize}

\end{solutionbox}
\begin{mnemonicbox}
``Constant Amplitude, Variable Frequency - CAVF''

\end{mnemonicbox}
\begin{center}\rule{0.5\linewidth}{0.5pt}\end{center}

\subsection*{પ્રશ્ન 2(ક) OR [7
ગુણ]}\label{uxaaauxab0uxab6uxaa8-2uxa95-or-7-uxa97uxaa3}

\textbf{ફ્રીક્વન્સી મોડ્યુલેશનની વ્યાખ્યા આપો અને FM તરંગ માટે ગાણિતિક અભિવ્યક્તિ
મેળવો.}

\begin{solutionbox}

\textbf{વ્યાખ્યા:} ફ્રીક્વન્સી મોડ્યુલેશન એ પ્રક્રિયા છે જેમાં કેરિયર સિગ્નલની આવૃત્તિ
મોડ્યુલેટિંગ સિગ્નલના તાત્કાલિક એમ્પ્લિટ્યુડ અનુસાર બદલાય છે.

\textbf{ગાણિતિક વ્યુત્પત્તિ:}

મોડ્યુલેટિંગ સિગ્નલ: em(t) = Em cos(ωmt) તાત્કાલિક આવૃત્તિ: fi = fc + kf \times Em
cos(ωmt)

જ્યાં kf = આવૃત્તિ સંવેદનશીલતા

\textbf{તાત્કાલિક કોણીય આવૃત્તિ:} ωi = 2π[fc + kf Em cos(ωmt)] ωi = ωc
+ 2πkf Em cos(ωmt)

\textbf{ફેઝ ગણતરી:} θ(t) = \intωi dt = ωct + (2πkf Em/ωm) sin(ωmt)

મોડ્યુલેશન ઇન્ડેક્સ: mf = 2πkf Em/ωm = Δf/fm

\textbf{અંતિમ FM અભિવ્યક્તિ:} eFM(t) = Ec cos[ωct + mf sin(ωmt)]

\textbf{પેરામીટર:}

\begin{itemize}
\tightlist
\item
  \textbf{મોડ્યુલેશન ઇન્ડેક્સ}: mf = Δf/fm
\item
  \textbf{ફ્રીક્વન્સી ડેવિએશન}: Δf = kf Em
\item
  \textbf{બેન્ડવિડ્થ}: BW = 2(Δf + fm) (કાર્સનનો નિયમ)
\end{itemize}

\end{solutionbox}
\begin{mnemonicbox}
``Frequency Varies with Message - FVM''

\end{mnemonicbox}
\begin{center}\rule{0.5\linewidth}{0.5pt}\end{center}

\subsection*{પ્રશ્ન 3(અ) [3
ગુણ]}\label{uxaaauxab0uxab6uxaa8-3uxa85-3-uxa97uxaa3}

\textbf{FM ડિમોડ્યુલેશનની સ્લોપ ડિટેક્શન પદ્ધતિનું વર્ણન કરો.}

\begin{solutionbox}

\textbf{સ્લોપ ડિટેક્શન સિદ્ધાંત:}

\begin{center}
\textbf{Mermaid Diagram (Code)}
\begin{verbatim}
{Shaded}
{Highlighting}[]
graph LR
    A[FM સિગ્નલ] {-{-}{} B[ટ્યુન્ડ સર્કિટ]}
    B {-{-}{} C[એન્વેલોપ ડિટેક્ટર]}
    C {-{-}{} D[ઓડિયો આઉટપુટ]}
{Highlighting}
{Shaded}
\end{verbatim}
\end{center}

\textbf{કાર્યપદ્ધતિ:}

\begin{itemize}
\tightlist
\item
  \textbf{ટ્યુન્ડ સર્કિટ}: આવૃત્તિ ફેરફારોને એમ્પ્લિટ્યુડ ફેરફારોમાં રૂપાંતરિત કરે છે
\item
  \textbf{સ્લોપ ઓપરેશન}: રેઝોનન્સ કર્વના સ્લોપનો ઉપયોગ કરે છે
\item
  \textbf{એન્વેલોપ ડિટેક્શન}: એમ્પ્લિટ્યુડ ફેરફારો કાઢે છે
\end{itemize}

\textbf{લાક્ષણિકતાઓ:}

\begin{itemize}
\tightlist
\item
  \textbf{સિમ્પલ સર્કિટ}: અમલમાં મૂકવા સરળ
\item
  \textbf{લિનિયર રેન્જ}: મર્યાદિત લિનિયર રેન્જ
\item
  \textbf{આઉટપુટ ડિસ્ટોર્શન}: અન્ય પદ્ધતિઓ કરતાં વધુ વિકૃતિ
\end{itemize}

\end{solutionbox}
\begin{mnemonicbox}
``Slope Converts Frequency to Amplitude - SCFA''

\end{mnemonicbox}
\begin{center}\rule{0.5\linewidth}{0.5pt}\end{center}

\subsection*{પ્રશ્ન 3(બ) [4
ગુણ]}\label{uxaaauxab0uxab6uxaa8-3uxaac-4-uxa97uxaa3}

\textbf{રેડિયો રીસીવરની વિવિધ લાક્ષણિકતાઓ સમજાવો.}

\begin{solutionbox}

{\def\LTcaptype{none} % do not increment counter
\begin{longtable}[]{@{}
  >{\raggedright\arraybackslash}p{(\linewidth - 4\tabcolsep) * \real{0.4000}}
  >{\raggedright\arraybackslash}p{(\linewidth - 4\tabcolsep) * \real{0.3000}}
  >{\raggedright\arraybackslash}p{(\linewidth - 4\tabcolsep) * \real{0.3000}}@{}}
\toprule\noalign{}
\begin{minipage}[b]{\linewidth}\raggedright
લાક્ષણિકતા
\end{minipage} & \begin{minipage}[b]{\linewidth}\raggedright
વ્યાખ્યા
\end{minipage} & \begin{minipage}[b]{\linewidth}\raggedright
મહત્વ
\end{minipage} \\
\midrule\noalign{}
\endhead
\bottomrule\noalign{}
\endlastfoot
\textbf{સેન્સિટિવિટી} & સંતોષકારક આઉટપુટ માટે લઘુત્તમ ઇનપુટ સિગ્નલ & વધુ સારી નબળી
સિગ્નલ રિસેપ્શન \\
\textbf{સિલેક્ટિવિટી} & ઇચ્છિત સિગ્નલ પસંદ કરવાની અને અન્યને નકારવાની ક્ષમતા &
દખલગીરી ઘટાડે છે \\
\textbf{ફિડેલિટી} & પુનરુત્પાદનની વફાદારી & વધુ સારી ઓડિયો ક્વોલિટી \\
\textbf{ઇમેજ ફ્રીક્વન્સી રિજેક્શન} & ઇમેજ આવૃત્તિનો અસ્વીકાર & ખોટા સિગ્નલ અટકાવે
છે \\
\end{longtable}
}

\textbf{ગાણિતિક સંબંધો:}

\begin{itemize}
\tightlist
\item
  \textbf{સેન્સિટિવિટી}: સ્ટાન્ડર્ડ આઉટપુટ માટે μV માં માપવામાં આવે છે
\item
  \textbf{સિલેક્ટિવિટી}: Q = f_{0}/BW
\item
  \textbf{ઇમેજ રિજેક્શન રેશિયો}: IRR = 1 + (2πfIFRC)^{2}
\end{itemize}

\end{solutionbox}
\begin{mnemonicbox}
``Sensitive Selective Faithful Image-free - SSFI''

\end{mnemonicbox}
\begin{center}\rule{0.5\linewidth}{0.5pt}\end{center}

\subsection*{પ્રશ્ન 3(ક) [7
ગુણ]}\label{uxaaauxab0uxab6uxaa8-3uxa95-7-uxa97uxaa3}

\textbf{યોગ્ય બ્લોક ડાયાગ્રામ સાથે સુપર હેટરોડાઇન રીસીવર પર ટૂંકી નોંધ લખો.}

\begin{solutionbox}

\textbf{બ્લોક ડાયાગ્રામ:}

\begin{center}
\textbf{Mermaid Diagram (Code)}
\begin{verbatim}
{Shaded}
{Highlighting}[]
graph LR
    A[એન્ટેના] {-{-}{} B[RF એમ્પ્લિફાયર]}
    B {-{-}{} C[મિક્સર]}
    D[લોકલ ઓસિલેટર] {-{-}{} C}
    C {-{-}{} E[IF એમ્પ્લિફાયર]}
    E {-{-}{} F[ડિટેક્ટર]}
    F {-{-}{} G[AF એમ્પ્લિફાયર]}
    G {-{-}{} H[સ્પીકર]}
    E {-{-}{} I[AGC]}
    I {-{-}{} B}
    I {-{-}{} E}
{Highlighting}
{Shaded}
\end{verbatim}
\end{center}

\textbf{કાર્યસિદ્ધાંત:}

\begin{itemize}
\tightlist
\item
  \textbf{આરએફ એમ્પ્લિફાયર}: પ્રાપ્ત RF સિગ્નલને એમ્પ્લિફાઇ કરે છે
\item
  \textbf{મિક્સર}: RF ને નિશ્ચિત IF આવૃત્તિમાં રૂપાંતરિત કરે છે
\item
  \textbf{લોકલ ઓસિલેટર}: મિક્સિંગ આવૃત્તિ પૂરી પાડે છે
\item
  \textbf{આઇએફ એમ્પ્લિફાયર}: નિશ્ચિત આવૃત્તિ પર મુખ્ય એમ્પ્લિફિકેશન
\item
  \textbf{ડિટેક્ટર}: મોડ્યુલેટેડ સિગ્નલ પુનઃપ્રાપ્ત કરે છે
\item
  \textbf{એજીસી}: સ્થિર આઉટપુટ સ્તર જાળવે છે
\end{itemize}

\textbf{ફાયદા:}

\begin{itemize}
\tightlist
\item
  \textbf{હાઇ સેન્સિટિવિટી}: TRF કરતાં વધુ સારી સંવેદનશીલતા
\item
  \textbf{ગુડ સિલેક્ટિવિટી}: વધુ સારી પસંદગીકારકતા
\item
  \textbf{સ્ટેબલ ગેઇન}: સ્થિર ગેઇન લાક્ષણિકતાઓ
\end{itemize}

\textbf{IF આવૃત્તિ પસંદગી:} સ્ટાન્ડર્ડ IF: AM માટે 455 kHz, FM માટે 10.7 MHz

\end{solutionbox}
\begin{mnemonicbox}
``Mix RF to IF for Better Selectivity - MRIBS''

\end{mnemonicbox}
\begin{center}\rule{0.5\linewidth}{0.5pt}\end{center}

\subsection*{પ્રશ્ન 3(અ) OR [3
ગુણ]}\label{uxaaauxab0uxab6uxaa8-3uxa85-or-3-uxa97uxaa3}

\textbf{ફેઝ લોક્ડ લૂપનો ઉપયોગ કરીને FM ડિમોડ્યુલેટરનું કાર્ય સમજાવો.}

\begin{solutionbox}

\textbf{PLL FM ડિમોડ્યુલેટર:}

\begin{center}
\textbf{Mermaid Diagram (Code)}
\begin{verbatim}
{Shaded}
{Highlighting}[]
graph LR
    A[FM ઇનપુટ] {-{-}{} B[ફેઝ ડિટેક્ટર]}
    C[VCO] {-{-}{} B}
    B {-{-}{} D[લૂપ ફિલ્ટર]}
    D {-{-}{} C}
    D {-{-}{} E[ઓડિયો આઉટપુટ]}
{Highlighting}
{Shaded}
\end{verbatim}
\end{center}

\textbf{કાર્યસિદ્ધાંત:}

\begin{itemize}
\tightlist
\item
  \textbf{ફેઝ ડિટેક્ટર}: ઇનપુટ FM ને VCO આઉટપુટ સાથે સરખાવે છે
\item
  \textbf{વીસીઓ}: વોલ્ટેજ કંટ્રોલ્ડ ઓસિલેટર ઇનપુટ આવૃત્તિને ટ્રેક કરે છે\\
\item
  \textbf{લૂપ ફિલ્ટર}: ઉચ્ચ આવૃત્તિ ઘટકો દૂર કરે છે
\item
  \textbf{લોક કન્ડિશન}: VCO આવૃત્તિ ઇનપુટ આવૃત્તિ સમાન થાય છે
\end{itemize}

\textbf{ફાયદા:}

\begin{itemize}
\tightlist
\item
  \textbf{લીનિયર ડિમોડ્યુલેશન}: ઉત્તમ રેખીયતા
\item
  \textbf{લો ડિસ્ટોર્શન}: લઘુത્તમ વિકૃતિ
\item
  \textbf{ગુડ ટ્રેકિંગ}: ઉત્તમ આવૃત્તિ ટ્રેકિંગ
\end{itemize}

\end{solutionbox}
\begin{mnemonicbox}
``Phase Lock Tracks Frequency - PLTF''

\end{mnemonicbox}
\begin{center}\rule{0.5\linewidth}{0.5pt}\end{center}

\subsection*{પ્રશ્ન 3(બ) OR [4
ગુણ]}\label{uxaaauxab0uxab6uxaa8-3uxaac-or-4-uxa97uxaa3}

\textbf{મૂળભૂત FM રીસીવરના બ્લોક ડાયાગ્રામની ચર્ચા કરો.}

\begin{solutionbox}

\textbf{FM રીસીવર બ્લોક ડાયાગ્રામ:}

\begin{center}
\textbf{Mermaid Diagram (Code)}
\begin{verbatim}
{Shaded}
{Highlighting}[]
graph LR
    A[FM એન્ટેના] {-{-}{} B[RF એમ્પ્લિફાયર]}
    B {-{-}{} C[મિક્સર]}
    D[લોકલ ઓસિલેટર] {-{-}{} C}
    C {-{-}{} E[IF એમ્પ્લિફાયર 10.7MHz]}
    E {-{-}{} F[લિમિટર]}
    F {-{-}{} G[FM ડિટેક્ટર]}
    G {-{-}{} H[ડી{-}એમ્ફેસિસ]}
    H {-{-}{} I[AF એમ્પ્લિફાયર]}
    I {-{-}{} J[સ્પીકર]}
{Highlighting}
{Shaded}
\end{verbatim}
\end{center}

\textbf{બ્લોક કાર્યો:}

\begin{itemize}
\tightlist
\item
  \textbf{આરએફ એમ્પ્લિફાયર}: નબળા FM સિગ્નલને એમ્પ્લિફાઇ કરે છે (88-108 MHz)
\item
  \textbf{મિક્સર}: IF આવૃત્તિમાં રૂપાંતરિત કરે છે (10.7 MHz)
\item
  \textbf{લિમિટર}: એમ્પ્લિટ્યુડ ફેરફારો દૂર કરે છે
\item
  \textbf{એફએમ ડિટેક્ટર}: ઓડિયો સિગ્નલ પુનઃપ્રાપ્ત કરે છે
\item
  \textbf{ડી-એમ્ફેસિસ}: મૂળ આવૃત્તિ પ્રતિસાદ પુનઃસ્થાપિત કરે છે
\end{itemize}

\textbf{AM રીસીવરથી મુખ્ય તફાવતો:}

\begin{itemize}
\tightlist
\item
  \textbf{હાયર આઇએફ}: 455 kHz બદલે 10.7 MHz
\item
  \textbf{લિમિટર સ્ટેજ}: વધારાનો લિમિટર સ્ટેજ
\item
  \textbf{ડી-એમ્ફેસિસ}: પ્રી/ડી-એમ્ફેસિસ નેટવર્ક
\end{itemize}

\end{solutionbox}
\begin{mnemonicbox}
``FM needs Higher IF and Limiting - FHIL''

\end{mnemonicbox}
\begin{center}\rule{0.5\linewidth}{0.5pt}\end{center}

\subsection*{પ્રશ્ન 3(ક) OR [7
ગુણ]}\label{uxaaauxab0uxab6uxaa8-3uxa95-or-7-uxa97uxaa3}

\textbf{યોગ્ય સર્કિટ ડાયાગ્રામ અને વેવફોર્મ સાથે ડાયોડનો ઉપયોગ કરીને એન્વેલોપ
ડિટેક્ટર પર ટૂંકી નોંધ લખો.}

\begin{solutionbox}

\textbf{સર્કિટ ડાયાગ્રામ:}

\begin{verbatim}
      D1
AM {-{-}||{-}{-}+{-}{-}{-}{-} Audio Output}
     |    |
     |    R
     |    |
     |    C
     |    |
    GND  GND
\end{verbatim}

\textbf{કાર્યસિદ્ધાંત:}

\begin{verbatim}
AM Input:    .∿∿. .∿∿∿∿. .∿∿.
            ∿    ∿      ∿    ∿

Diode Output: ▄▄▄ ▄▄▄▄▄▄ ▄▄▄
(After filtering)

Audio Output: ∼    ∼    ∼    ∼
\end{verbatim}

\textbf{ઓપરેશન:}

\begin{itemize}
\tightlist
\item
  \textbf{ડાયોડ કન્ડક્શન}: સકારાત્મક અર્ધ ચક્ર દરમિયાન વહન કરે છે
\item
  \textbf{કેપેસિટર ચાર્જિંગ}: પીક વેલ્યુ સુધી ચાર્જ થાય છે
\item
  \textbf{આરસી ડિસચાર્જ}: RC સર્કિટ દ્વારા ડિસચાર્જ થાય છે
\item
  \textbf{એન્વેલોપ ફોલોઇંગ}: આઉટપુટ એન્વેલોપને અનુસરે છે
\end{itemize}

\textbf{ડિઝાઇન વિચારણાઓ:}

\begin{itemize}
\tightlist
\item
  \textbf{ટાઇમ કોન્સ્ટન્ટ}: RC \textgreater\textgreater{} 1/fc પણ RC
  \textless\textless{} 1/fm
\item
  \textbf{ડાયોડ સિલેક્શન}: ફાસ્ટ રિકવરી ડાયોડ પસંદીદા
\item
  \textbf{લોડ રેઝિસ્ટન્સ}: ડાયોડ રેઝિસ્ટન્સ કરતાં ઘણું મોટું હોવું જોઈએ
\end{itemize}

\textbf{ફાયદા:}

\begin{itemize}
\tightlist
\item
  \textbf{સિમ્પ્લિસિટી}: ખૂબ સરળ સર્કિટ
\item
  \textbf{લો કોસ્ટ}: આર્થિક ઉકેલ
\item
  \textbf{હાઇ એફિશિયન્સી}: સારી ડિટેક્શન કાર્યક્ષમતા
\end{itemize}

\end{solutionbox}
\begin{mnemonicbox}
``Diode Charges, RC Follows Envelope - DCRF''

\end{mnemonicbox}
\begin{center}\rule{0.5\linewidth}{0.5pt}\end{center}

\subsection*{પ્રશ્ન 4(અ) [3
ગુણ]}\label{uxaaauxab0uxab6uxaa8-4uxa85-3-uxa97uxaa3}

\textbf{અન્ડર સેમ્પલિંગ, ઓવર સેમ્પલિંગ અને ક્રિટિકલ સેમ્પલિંગનું વિવરણ આપો.}

\begin{solutionbox}

{\def\LTcaptype{none} % do not increment counter
\begin{longtable}[]{@{}lll@{}}
\toprule\noalign{}
પ્રકાર & શરત & પરિણામ \\
\midrule\noalign{}
\endhead
\bottomrule\noalign{}
\endlastfoot
\textbf{અન્ડર સેમ્પલિંગ} & fs \textless{} 2fm & એલાયસિંગ થાય છે \\
\textbf{ક્રિટિકલ સેમ્પલિંગ} & fs = 2fm & માત્ર પૂરતું, કોઈ માર્જિન નથી \\
\textbf{ઓવર સેમ્પલિંગ} & fs \textgreater{} 2fm & એલાયસિંગ નથી, સલામત
માર્જિન \\
\end{longtable}
}

\textbf{ડાયાગ્રામ:}

\begin{verbatim}
Original Signal:     ∿∿∿∿∿∿∿

Under Sampling:      ∿ . . ∿     Aliasing
Critical Sampling:   ∿ . ∿ .     Just OK  
Over Sampling:       ∿.∿.∿.∿     Safe
\end{verbatim}

\begin{itemize}
\tightlist
\item
  \textbf{એલાયસિંગ ઇફેક્ટ}: અન્ડર સેમ્પલિંગ આવૃત્તિ ઓવરલેપનું કારણ બને છે
\item
  \textbf{નાયક્વિસ્ટ રેટ}: લઘુત્તમ સેમ્પલિંગ રેટ = 2fm
\item
  \textbf{પ્રેક્ટિકલ રેટ}: સામાન્ય રીતે મેસેજ આવૃત્તિના 2.5 થી 5 ગણા
\end{itemize}

\end{solutionbox}
\begin{mnemonicbox}
``Under-Alias, Critical-Just, Over-Safe - UCO''

\end{mnemonicbox}
\begin{center}\rule{0.5\linewidth}{0.5pt}\end{center}

\subsection*{પ્રશ્ન 4(બ) [4
ગુણ]}\label{uxaaauxab0uxab6uxaa8-4uxaac-4-uxa97uxaa3}

\textbf{સેમ્પલિંગ થિયરમ લખો અને નાયક્વિસ્ટ રેટ, નાયક્વિસ્ટ ઇન્ટરવલ અને એલાયસિંગ
એરરની વ્યાખ્યા આપો.}

\begin{solutionbox}

\textbf{સેમ્પલિંગ થિયરમ:} ``જો સેમ્પલિંગ આવૃત્તિ સિગ્નલના સર્વોચ્ચ આવૃત્તિ ઘટકના
ઓછામાં ઓછા બમણી હોય તો સતત સિગ્નલ તેના સેમ્પલમાંથી સંપૂર્ણ રીતે પુનઃપ્રાપ્ત કરી શકાય
છે.''

\textbf{વ્યાખ્યાઓ:}

{\def\LTcaptype{none} % do not increment counter
\begin{longtable}[]{@{}lll@{}}
\toprule\noalign{}
શબ્દ & વ્યાખ્યા & સૂત્ર \\
\midrule\noalign{}
\endhead
\bottomrule\noalign{}
\endlastfoot
\textbf{નાયક્વિસ્ટ રેટ} & લઘુત્તમ સેમ્પલિંગ આવૃત્તિ & fs = 2fm \\
\textbf{નાયક્વિસ્ટ ઇન્ટરવલ} & મહત્તમ સેમ્પલિંગ અંતરાલ & Ts = 1/(2fm) \\
\textbf{એલાયસિંગ એરર} & અન્ડર સેમ્પલિંગને કારણે આવૃત્તિ ઓવરલેપ & fa = \\
\end{longtable}
}

\textbf{ગાણિતિક અભિવ્યક્તિ:}

\begin{itemize}
\tightlist
\item
  \textbf{સેમ્પલિંગ ફ્રીક્વન્સી}: fs \geq 2fm (નાયક્વિસ્ટ કસોટી)
\item
  \textbf{સેમ્પલિંગ પીરિયડ}: Ts = 1/fs
\item
  \textbf{એલાયસિંગ કન્ડિશન}: fs \textless{} 2fm
\end{itemize}

\textbf{વ્યવહારિક એપ્લિકેશન:}

\begin{itemize}
\tightlist
\item
  \textbf{ડિજિટલ ઓડિયો}: fm = 20 kHz માટે fs = 44.1 kHz
\item
  \textbf{ટેલિફોન સિસ્ટમ}: fm = 4 kHz માટે fs = 8 kHz
\end{itemize}

\end{solutionbox}
\begin{mnemonicbox}
``Sample at twice message frequency - S2M''

\end{mnemonicbox}
\begin{center}\rule{0.5\linewidth}{0.5pt}\end{center}

\subsection*{પ્રશ્ન 4(ક) [7
ગુણ]}\label{uxaaauxab0uxab6uxaa8-4uxa95-7-uxa97uxaa3}

\textbf{આઇડિયલ, નેચરલ અને ફ્લેટ ટોપ સેમ્પલિંગની ચર્ચા કરો.}

\begin{solutionbox}

\textbf{સેમ્પલિંગના પ્રકારો:}

{\def\LTcaptype{none} % do not increment counter
\begin{longtable}[]{@{}lll@{}}
\toprule\noalign{}
પ્રકાર & લાક્ષણિકતાઓ & ગાણિતિક અભિવ્યક્તિ \\
\midrule\noalign{}
\endhead
\bottomrule\noalign{}
\endlastfoot
\textbf{આઇડિયલ સેમ્પલિંગ} & ઇમ્પલ્સ ટ્રેઇન ગુણાકાર & xs(t) = x(t)·δT(t) \\
\textbf{નેચરલ સેમ્પલિંગ} & વેરિએબલ પહોળાઈ પલ્સ & ટોપ સિગ્નલને અનુસરે છે \\
\textbf{ફ્લેટ ટોપ સેમ્પલિંગ} & કોન્સ્ટન્ટ એમ્પ્લિટ્યુડ પલ્સ & સેમ્પલ અને હોલ્ડ \\
\end{longtable}
}

\textbf{વેવફોર્મ:}

\begin{verbatim}
Original:    ∿∿∿∿∿∿∿∿∿∿∿∿

Ideal:       ↑ ↑ ↑ ↑ ↑ ↑     Impulses

Natural:     |∿| |∿| |∿|     Variable width

Flat Top:    |▄| |▄| |▄|     Constant width
\end{verbatim}

\textbf{આવૃત્તિ સ્પેક્ટ્રમ:}

\begin{itemize}
\tightlist
\item
  \textbf{આઇડિયલ સેમ્પલિંગ}: સચોટ સ્પેક્ટ્રલ પ્રતિકૃતિ
\item
  \textbf{નેચરલ સેમ્પલિંગ}: થોડું સ્પેક્ટ્રલ મોડિફિકેશન\\
\item
  \textbf{ફ્લેટ ટોપ સેમ્પલિંગ}: એપર્ચર ઇફેક્ટ હાજર
\end{itemize}

\textbf{વ્યવહારિક અમલીકરણ:}

\begin{itemize}
\tightlist
\item
  \textbf{આઇડિયલ}: માત્ર સૈદ્ધાંતિક
\item
  \textbf{નેચરલ}: PAM સિસ્ટમમાં વપરાય છે
\item
  \textbf{ફ્લેટ ટોપ}: સેમ્પલ-અને-હોલ્ડ સર્કિટ, ADC સિસ્ટમ
\end{itemize}

\textbf{એપર્ચર ઇફેક્ટ:} ફ્લેટ-ટોપ સેમ્પલિંગમાં: \textbar Sa(πfT/2)\textbar{} =
\textbar sin(πfT/2)/(πfT/2)\textbar{}

\end{solutionbox}
\begin{mnemonicbox}
``Ideal-Impulse, Natural-Variable, Flat-Constant -
IVF''

\end{mnemonicbox}
\begin{center}\rule{0.5\linewidth}{0.5pt}\end{center}

\subsection*{પ્રશ્ન 4(અ) OR [3
ગુણ]}\label{uxaaauxab0uxab6uxaa8-4uxa85-or-3-uxa97uxaa3}

\textbf{યોગ્ય બ્લોક ડાયાગ્રામ સાથે ડેલ્ટા મોડ્યુલેટરનું કાર્ય સમજાવો.}

\begin{solutionbox}

\textbf{ડેલ્ટા મોડ્યુલેટર બ્લોક ડાયાગ્રામ:}

\begin{center}
\textbf{Mermaid Diagram (Code)}
\begin{verbatim}
{Shaded}
{Highlighting}[]
graph LR
    A[ઇનપુટ સિગ્નલ] {-{-}{} B[કમ્પેરેટર]}
    B {-{-}{} C[1{-}બિટ ક્વોન્ટાઇઝર]}
    C {-{-}{} D[આઉટપુટ]}
    C {-{-}{} E[ઇન્ટિગ્રેટર]}
    E {-{-}{} F[ડિલે]}
    F {-{-}{} B}
{Highlighting}
{Shaded}
\end{verbatim}
\end{center}

\textbf{કાર્યસિદ્ધાંત:}

\begin{itemize}
\tightlist
\item
  \textbf{કમ્પેરિસન}: ઇનપુટની સરખામણી પહેલાના ઇન્ટિગ્રેટેડ આઉટપુટ સાથે
\item
  \textbf{1-બિટ ક્વોન્ટાઇઝેશન}: આઉટપુટ +Δ અથવા -Δ છે
\item
  \textbf{ઇન્ટિગ્રેશન}: ઇન્ટિગ્રેટર ઇનપુટ સિગ્નલનો અંદાજ કાઢે છે
\item
  \textbf{ફીડબેક}: પહેલાનો આઉટપુટ સરખામણી માટે પાછો મોકલવામાં આવે છે
\end{itemize}

\textbf{આઉટપુટ લાક્ષણિકતાઓ:}

\begin{itemize}
\tightlist
\item
  \textbf{બાઇનરી આઉટપુટ}: દરેક સેમ્પલ માટે માત્ર 1 બિટ
\item
  \textbf{સ્ટેપ સાઇઝ}: નિશ્ચિત સ્ટેપ સાઇઝ Δ
\item
  \textbf{ટ્રેકિંગ}: આઉટપુટ ઇનપુટને સ્ટેપમાં ટ્રેક કરે છે
\end{itemize}

\end{solutionbox}
\begin{mnemonicbox}
``Compare, Quantize, Integrate, Feedback - CQIF''

\end{mnemonicbox}
\begin{center}\rule{0.5\linewidth}{0.5pt}\end{center}

\subsection*{પ્રશ્ન 4(બ) OR [4
ગુણ]}\label{uxaaauxab0uxab6uxaa8-4uxaac-or-4-uxa97uxaa3}

\textbf{યોગ્ય સમજૂતી સાથે ડેલ્ટા મોડ્યુલેશન (DM) ના ગેરફાયદા લખો.}

\begin{solutionbox}

\textbf{મુખ્ય ગેરફાયદા:}

{\def\LTcaptype{none} % do not increment counter
\begin{longtable}[]{@{}lll@{}}
\toprule\noalign{}
ગેરફાયદા & સમજૂતી & ઉકેલ \\
\midrule\noalign{}
\endhead
\bottomrule\noalign{}
\endlastfoot
\textbf{સ્લોપ ઓવરલોડ} & ઝડપી ફેરફારો ટ્રેક કરી શકતું નથી & સ્ટેપ સાઇઝ વધારો \\
\textbf{ગ્રેન્યુલર નોઇઝ} & સપાટ વિસ્તારોમાં ક્વોન્ટાઇઝેશન નોઇઝ & સ્ટેપ સાઇઝ
ઘટાડો \\
\textbf{હાઇ બિટ રેટ} & ઉચ્ચ સેમ્પલિંગ રેટ જરૂરી & ADPCM નો ઉપયોગ કરો \\
\textbf{લિમિટેડ ડાયનેમિક રેન્જ} & નિશ્ચિત સ્ટેપ સાઇઝની મર્યાદા & એડેપ્ટિવ
તકનીકો \\
\end{longtable}
}

\textbf{સ્લોપ ઓવરલોડ કન્ડિશન:} જ્યારે \textbar dx/dt\textbar{}
\textgreater{} Δfs, સ્લોપ ઓવરલોડ થાય છે

\textbf{ગ્રેન્યુલર નોઇઝ:} જ્યારે ઇનપુટ સિગ્નલ ધીમે ધીમે બદલાય અથવા સ્થિર રહે ત્યારે
થાય છે

\textbf{વેવફોર્મ:}

\begin{verbatim}
Slope Overload:    /∿∿∿    Input too fast
                  /▄▄▄     DM output lags

Granular Noise:   \_\_\_\_     Flat input
                  ▄▄▄▄     DM oscillates
\end{verbatim}

\textbf{પ્રદર્શન પેરામીટર:}

\begin{itemize}
\tightlist
\item
  \textbf{સ્લોપ ઓવરલોડ}: મહત્તમ સ્લોપ = Δfs
\item
  \textbf{ગ્રેન્યુલર નોઇઝ}: સ્ટેપ સાઇઝ પર આધાર રાખે છે
\item
  \textbf{એસએનઆર}: બંને અસરોથી મર્યાદિત
\end{itemize}

\end{solutionbox}
\begin{mnemonicbox}
``Slope-Overload, Granular-Noise, High-Bitrate -
SOG-H''

\end{mnemonicbox}
\begin{center}\rule{0.5\linewidth}{0.5pt}\end{center}

\subsection*{પ્રશ્ન 4(ક) OR [7
ગુણ]}\label{uxaaauxab0uxab6uxaa8-4uxa95-or-7-uxa97uxaa3}

\textbf{પલ્સ કોડ મોડ્યુલેશન (PCM) ટ્રાન્સમિટર અને રીસીવરના દરેક બ્લોકના કાર્યોનું
વર્ણન કરો.}

\begin{solutionbox}

\textbf{PCM ટ્રાન્સમિટર બ્લોક ડાયાગ્રામ:}

\begin{center}
\textbf{Mermaid Diagram (Code)}
\begin{verbatim}
{Shaded}
{Highlighting}[]
graph LR
    A[એનાલોગ ઇનપુટ] {-{-}{} B[LPF]}
    B {-{-}{} C[સેમ્પલ અને હોલ્ડ]}
    C {-{-}{} D[ક્વોન્ટાઇઝર]}
    D {-{-}{} E[એન્કોડર]}
    E {-{-}{} F[ડિજિટલ આઉટપુટ]}
{Highlighting}
{Shaded}
\end{verbatim}
\end{center}

\textbf{PCM રીસીવર બ્લોક ડાયાગ્રામ:}

\begin{center}
\textbf{Mermaid Diagram (Code)}
\begin{verbatim}
{Shaded}
{Highlighting}[]
graph LR
    G[ડિજિટલ ઇનપુટ] {-{-}{} H[ડીકોડર]}
    H {-{-}{} I[DAC]}
    I {-{-}{} J[LPF]}
    J {-{-}{} K[એનાલોગ આઉટપુટ]}
{Highlighting}
{Shaded}
\end{verbatim}
\end{center}

\textbf{ટ્રાન્સમિટર બ્લોક કાર્યો:}

{\def\LTcaptype{none} % do not increment counter
\begin{longtable}[]{@{}ll@{}}
\toprule\noalign{}
બ્લોક & કાર્ય \\
\midrule\noalign{}
\endhead
\bottomrule\noalign{}
\endlastfoot
\textbf{LPF} & એન્ટિ-એલાયસિંગ ફિલ્ટર, fm કરતાં વધુ આવૃત્તિઓ દૂર કરે છે \\
\textbf{સેમ્પલ અને હોલ્ડ} & fs \geq 2fm પર સેમ્પલ કરે છે અને વેલ્યુ હોલ્ડ કરે છે \\
\textbf{ક્વોન્ટાઇઝર} & ડિસ્ક્રીટ એમ્પ્લિટ્યુડ લેવલમાં રૂપાંતરિત કરે છે \\
\textbf{એન્કોડર} & ક્વોન્ટાઇઝ્ડ સેમ્પલને બાઇનરી કોડમાં રૂપાંતરિત કરે છે \\
\end{longtable}
}

\textbf{રીસીવર બ્લોક કાર્યો:}

{\def\LTcaptype{none} % do not increment counter
\begin{longtable}[]{@{}ll@{}}
\toprule\noalign{}
બ્લોક & કાર્ય \\
\midrule\noalign{}
\endhead
\bottomrule\noalign{}
\endlastfoot
\textbf{ડીકોડર} & બાઇનરી કોડને ક્વોન્ટાઇઝ્ડ લેવલમાં રૂપાંતરિત કરે છે \\
\textbf{DAC} & ડિજિટલ ટુ એનાલોગ રૂપાંતરણ \\
\textbf{LPF} & પુનર્નિર્માણ ફિલ્ટર, સેમ્પલિંગ આવૃત્તિ દૂર કરે છે \\
\end{longtable}
}

\textbf{તકનીકી સ્પેસિફિકેશન:}

\begin{itemize}
\tightlist
\item
  \textbf{ક્વોન્ટાઇઝેશન લેવલ}: L = 2^{n} (n = બિટની સંખ્યા)
\item
  \textbf{ક્વોન્ટાઇઝેશન એરર}: મહત્તમ Δ/2
\item
  \textbf{બિટ રેટ}: fb = n \times fs
\end{itemize}

\textbf{PCM ફાયદા:}

\begin{itemize}
\tightlist
\item
  \textbf{નોઇઝ ઇમ્યુનિટી}: ઉત્તમ નોઇઝ પ્રદર્શન
\item
  \textbf{રિજનરેશન}: એરર એકઠા થયા વગર પુનર્જનન કરી શકાય છે
\item
  \textbf{મલ્ટિપ્લેક્સિંગ}: અનેક ચેનલ મલ્ટિપ્લેક્સ કરવું સરળ
\end{itemize}

\end{solutionbox}
\begin{mnemonicbox}
``Low-pass, Sample, Quantize, Encode - LSQE માટે TX;
Decode, Convert, Filter - DCF માટે RX''

\end{mnemonicbox}
\begin{center}\rule{0.5\linewidth}{0.5pt}\end{center}

\subsection*{પ્રશ્ન 5(અ) [3
ગુણ]}\label{uxaaauxab0uxab6uxaa8-5uxa85-3-uxa97uxaa3}

\textbf{TDM-PCM સિસ્ટમના બ્લોક ડાયાગ્રામની સંક્ષિપ્ત ચર્ચા કરો.}

\begin{solutionbox}

\textbf{TDM-PCM સિસ્ટમ બ્લોક ડાયાગ્રામ:}

\begin{center}
\textbf{Mermaid Diagram (Code)}
\begin{verbatim}
{Shaded}
{Highlighting}[]
graph LR
    A[ચેનલ 1] {-{-}{} D[કમ્યુટેટર]}
    B[ચેનલ 2] {-{-}{} D}
    C[ચેનલ 3] {-{-}{} D}
    D {-{-}{} E[PCM એન્કોડર]}
    E {-{-}{} F[ટ્રાન્સમિશન]}
    F {-{-}{} G[PCM ડીકોડર]}
    G {-{-}{} H[ડીકમ્યુટેટર]}
    H {-{-}{} I[ચેનલ 1]}
    H {-{-}{} J[ચેનલ 2]}
    H {-{-}{} K[ચેનલ 3]}
{Highlighting}
{Shaded}
\end{verbatim}
\end{center}

\textbf{સિસ્ટમ ઓપરેશન:}

\begin{itemize}
\tightlist
\item
  \textbf{કમ્યુટેટર}: અનેક ચેનલનું અનુક્રમિક સેમ્પલિંગ
\item
  \textbf{પીસીએમ એન્કોડર}: સેમ્પલને ડિજિટલ ફોર્મેટમાં રૂપાંતરિત કરે છે
\item
  \textbf{ટાઇમ ડિવિઝન}: દરેક ચેનલને નિશ્ચિત ટાઇમ સ્લોટ મળે છે
\item
  \textbf{ડીકમ્યુટેટર}: રીસીવર પર ચેનલ અલગ કરે છે
\end{itemize}

\textbf{ફ્રેમ સ્ટ્રક્ચર:}

\begin{itemize}
\tightlist
\item
  \textbf{ટાઇમ સ્લોટ}: દરેક ચેનલને ચોક્કસ સમય આપવામાં આવે છે
\item
  \textbf{ફ્રેમ પીરિયડ}: બધી ચેનલ માટે સંપૂર્ણ ચક્ર
\item
  \textbf{સિંક્રોનાઇઝેશન}: ફ્રેમ સિંક્રોનાઇઝેશન બિટ ઉમેરવામાં આવે છે
\end{itemize}

\textbf{ફાયદા:}

\begin{itemize}
\tightlist
\item
  \textbf{બેન્ડવિડ્થ એફિશિયન્સી}: કાર્યક્ષમ સ્પેક્ટ્રમ ઉપયોગ
\item
  \textbf{મલ્ટિપલ ચેનલ}: એક લિંક પર અનેક ચેનલ
\end{itemize}

\end{solutionbox}
\begin{mnemonicbox}
``Time Division Multiple Access - TDMA''

\end{mnemonicbox}
\begin{center}\rule{0.5\linewidth}{0.5pt}\end{center}

\subsection*{પ્રશ્ન 5(બ) [4
ગુણ]}\label{uxaaauxab0uxab6uxaa8-5uxaac-4-uxa97uxaa3}

\textbf{એડેપ્ટિવ ડેલ્ટા મોડ્યુલેશન (ADM) પર ટૂંકી નોંધ લખો.}

\begin{solutionbox}

\textbf{ADM બ્લોક ડાયાગ્રામ:}

\begin{center}
\textbf{Mermaid Diagram (Code)}
\begin{verbatim}
{Shaded}
{Highlighting}[]
graph LR
    A[ઇનપુટ] {-{-}{} B[કમ્પેરેટર]}
    B {-{-}{} C[લોજિક સર્કિટ]}
    C {-{-}{} D[સ્ટેપ સાઇઝ કંટ્રોલ]}
    D {-{-}{} E[ઇન્ટિગ્રેટર]}
    E {-{-}{} F[ડિલે]}
    F {-{-}{} B}
    C {-{-}{} G[આઉટપુટ]}
{Highlighting}
{Shaded}
\end{verbatim}
\end{center}

\textbf{કાર્યસિદ્ધાંત:}

\begin{itemize}
\tightlist
\item
  \textbf{એડેપ્ટિવ સ્ટેપ સાઇઝ}: ઇનપુટ લાક્ષણિકતાઓના આધારે સ્ટેપ સાઇઝ બદલાય છે
\item
  \textbf{સ્લોપ ઓવરલોડ પ્રિવેન્શન}: ઝડપી ફેરફારો માટે સ્ટેપ સાઇઝ વધારે છે
\item
  \textbf{ગ્રેન્યુલર નોઇઝ રિડક્શન}: ધીમા ફેરફારો માટે સ્ટેપ સાઇઝ ઘટાડે છે
\item
  \textbf{લોજિક કંટ્રોલ}: એલ્ગોરિધમ સ્ટેપ સાઇઝ એડેપ્ટેશન કંટ્રોલ કરે છે
\end{itemize}

\textbf{સ્ટેપ સાઇઝ કંટ્રોલ:}

\begin{itemize}
\tightlist
\item
  \textbf{ઇન્ક્રીઝ}: જ્યારે સતત બિટ સમાન હોય (સ્લોપ ઓવરલોડ શોધાય)
\item
  \textbf{ડિક્રીઝ}: જ્યારે વૈકલ્પિક પેટર્ન થાય (ગ્રેન્યુલર વિસ્તાર)
\end{itemize}

\textbf{સ્ટાન્ડર્ડ DM કરતાં ફાયદા:}

\begin{itemize}
\tightlist
\item
  \textbf{બેટર એસએનઆર}: સુધારેલ સિગ્નલ-ટુ-નોઇઝ રેશિયો
\item
  \textbf{ડાયનેમિક રેન્જ}: વધુ સારી ડાયનેમિક રેન્જ
\item
  \textbf{ઓટોમેટિક એડેપ્ટેશન}: સ્વ-એડજસ્ટિંગ લાક્ષણિકતાઓ
\end{itemize}

\end{solutionbox}
\begin{mnemonicbox}
``Adaptive Step size Reduces both Slope-overload and
Granular noise - ASRSG''

\end{mnemonicbox}
\begin{center}\rule{0.5\linewidth}{0.5pt}\end{center}

\subsection*{પ્રશ્ન 5(ક) [7
ગુણ]}\label{uxaaauxab0uxab6uxaa8-5uxa95-7-uxa97uxaa3}

\textbf{લાઇન કોડિંગની વ્યાખ્યા આપો. ``1 0 1 1 1 0 1 1'' માટે NRZ (યુનિપોલર),
RZ (યુનિપોલર), મેન્ચેસ્ટર કોડિંગ વેવફોર્મ દોરો.}

\begin{solutionbox}

\textbf{વ્યાખ્યા:} લાઇન કોડિંગ એ ડિજિટલ ડેટાને કમ્યુનિકેશન ચેનલ પર ટ્રાન્સમિશન માટે
યોગ્ય ડિજિટલ સિગ્નલમાં રૂપાંતરિત કરવાની પ્રક્રિયા છે.

\textbf{વેવફોર્મ ડાયાગ્રામ:}

\begin{verbatim}
Data:        1  0  1  1  1  0  1  1

NRZ Unipolar:
             ▄▄    ▄▄ ▄▄ ▄▄    ▄▄ ▄▄
                \_\_          \_\_

RZ Unipolar:
             ▄  ▄  ▄  ▄  ▄     ▄  ▄
             ▄▄▄▄▄▄▄▄▄▄▄▄▄▄▄▄▄▄▄▄▄▄▄

Manchester:
             ▄▄    ▄▄ ▄▄ ▄▄    ▄▄ ▄▄
                \_\_  \_\_ \_\_ \_\_ \_\_
             Transition at middle of each bit
\end{verbatim}

\textbf{લાક્ષણિકતાઓ:}

{\def\LTcaptype{none} % do not increment counter
\begin{longtable}[]{@{}llll@{}}
\toprule\noalign{}
કોડિંગ પ્રકાર & લોજિક 1 & લોજિક 0 & બેન્ડવિડ્થ \\
\midrule\noalign{}
\endhead
\bottomrule\noalign{}
\endlastfoot
\textbf{NRZ યુનિપોલર} & +V & 0V & fb \\
\textbf{RZ યુનિપોલર} & T/2 માટે +V, T/2 માટે 0V & 0V & 2fb \\
\textbf{મેન્ચેસ્ટર} & હાઇ-ટુ-લો ટ્રાન્ઝિશન & લો-ટુ-હાઇ ટ્રાન્ઝિશન & 2fb \\
\end{longtable}
}

\textbf{ગુણધર્મો:}

\begin{itemize}
\tightlist
\item
  \textbf{એનઆરઝેડ}: શૂન્ય પર પાછા ફરતું નથી, સરળ પણ સ્વ-સિંક્રોનાઇઝેશન નથી
\item
  \textbf{આરઝેડ}: શૂન્ય પર પાછા ફરે છે, સરળ ક્લોક રિકવરી પણ બમણી બેન્ડવિડ્થ
\item
  \textbf{મેન્ચેસ્ટર}: સ્વ-સિંક્રોનાઇઝિંગ, ઇથરનેટમાં વપરાય છે
\end{itemize}

\textbf{એપ્લિકેશન:}

\begin{itemize}
\tightlist
\item
  \textbf{એનઆરઝેડ}: સરળ ડિજિટલ સિસ્ટમ
\item
  \textbf{આરઝેડ}: મેગ્નેટિક રેકોર્ડિંગ
\item
  \textbf{મેન્ચેસ્ટર}: ઇથરનેટ, કેટલાક વાયરલેસ સ્ટાન્ડર્ડ
\end{itemize}

\end{solutionbox}
\begin{mnemonicbox}
``NRZ-Simple, RZ-Return, Manchester-Transition -
SRT''

\end{mnemonicbox}
\begin{center}\rule{0.5\linewidth}{0.5pt}\end{center}

\subsection*{પ્રશ્ન 5(અ) OR [3
ગુણ]}\label{uxaaauxab0uxab6uxaa8-5uxa85-or-3-uxa97uxaa3}

\textbf{ટાઇમ ડિવિઝન ડિજિટલ મલ્ટિપ્લેક્સિંગના કોન્સેપ્ટનું વર્ણન કરો.}

\begin{solutionbox}

\textbf{TDM કોન્સેપ્ટ:} ટાઇમ ડિવિઝન મલ્ટિપ્લેક્સિંગ એ તકનીક છે જેમાં દરેક સિગ્નલને અલગ
અલગ ટાઇમ સ્લોટ આપીને અનેક ડિજિટલ સિગ્નલ એક જ ચેનલ પર ટ્રાન્સમિટ કરવામાં આવે છે.

\textbf{TDM ફ્રેમ સ્ટ્રક્ચર:}

\begin{verbatim}
Frame: |CH1|CH2|CH3|CH4|SYNC|CH1|CH2|CH3|CH4|SYNC|
       ――――― Frame Period ―――――
\end{verbatim}

\textbf{કાર્યસિદ્ધાંત:}

{\def\LTcaptype{none} % do not increment counter
\begin{longtable}[]{@{}ll@{}}
\toprule\noalign{}
ઘટક & કાર્ય \\
\midrule\noalign{}
\endhead
\bottomrule\noalign{}
\endlastfoot
\textbf{ટાઇમ સ્લોટ} & દરેક ચેનલને આપવામાં આવતી નિશ્ચિત અવધિ \\
\textbf{ફ્રેમ} & બધી ચેનલ ધરાવતું સંપૂર્ણ ચક્ર \\
\textbf{સિંક્રોનાઇઝેશન} & યોગ્ય ચેનલ અલગીકરણ જાળવે છે \\
\textbf{મલ્ટિપ્લેક્સર} & અનેક ઇનપુટ અનુક્રમે જોડે છે \\
\end{longtable}
}

\textbf{મુખ્ય લક્ષણો:}

\begin{itemize}
\tightlist
\item
  \textbf{ફિક્સ્ડ ટાઇમ સ્લોટ}: દરેક ચેનલને પૂર્વનિર્ધારિત સમય મળે છે
\item
  \textbf{સિક્વેન્શિયલ સેમ્પલિંગ}: ચેનલ એક પછી એક સેમ્પલ થાય છે\\
\item
  \textbf{ડિજિટલ ટ્રાન્સમિશન}: ડિજિટલ સિગ્નલ માટે યોગ્ય
\item
  \textbf{બેન્ડવિડ્થ શેરિંગ}: કાર્યક્ષમ સ્પેક્ટ્રમ ઉપયોગ
\end{itemize}

\textbf{એપ્લિકેશન:}

\begin{itemize}
\tightlist
\item
  \textbf{ટેલિફોન સિસ્ટમ}: T1, E1 સિસ્ટમ
\item
  \textbf{ડિજિટલ હાયરાર્કી}: PDH, SDH સિસ્ટમ
\end{itemize}

\end{solutionbox}
\begin{mnemonicbox}
``Time slots Share Single Channel - TSSC''

\end{mnemonicbox}
\begin{center}\rule{0.5\linewidth}{0.5pt}\end{center}

\subsection*{પ્રશ્ન 5(બ) OR [4
ગુણ]}\label{uxaaauxab0uxab6uxaa8-5uxaac-or-4-uxa97uxaa3}

\textbf{ડિફરન્શિયલ PCM (DPCM) પર ટૂંકી નોંધ લખો.}

\begin{solutionbox}

\textbf{DPCM બ્લોક ડાયાગ્રામ:}

\begin{center}
\textbf{Mermaid Diagram (Code)}
\begin{verbatim}
{Shaded}
{Highlighting}[]
graph LR
    A[ઇનપુટ] {-{-}{} B[ડિફરન્સ]}
    C[પ્રિડિક્ટર] {-{-}{} B}
    B {-{-}{} D[ક્વોન્ટાઇઝર]}
    D {-{-}{} E[એન્કોડર]}
    E {-{-}{} F[આઉટપુટ]}
    D {-{-}{} G[લોકલ ડીકોડર]}
    G {-{-}{} H[એડર]}
    C {-{-}{} H}
    H {-{-}{} C}
{Highlighting}
{Shaded}
\end{verbatim}
\end{center}

\textbf{કાર્યસિદ્ધાંત:}

\begin{itemize}
\tightlist
\item
  \textbf{પ્રિડિક્શન}: પહેલાના સેમ્પલમાંથી વર્તમાન સેમ્પલનો અંદાજ કાઢે છે
\item
  \textbf{ડિફરન્સ સિગ્નલ}: વાસ્તવિક અને અંદાજિત વચ્ચેનો તફાવત ટ્રાન્સમિટ કરે છે
\item
  \textbf{ક્વોન્ટાઇઝેશન}: માત્ર ડિફરન્સ સિગ્નલ ક્વોન્ટાઇઝ કરે છે
\item
  \textbf{લોકલ ડીકોડર}: રીસીવર જેવો જ રેફરન્સ જાળવે છે
\end{itemize}

\textbf{પ્રિડિક્શન એલ્ગોરિધમ:}

{\def\LTcaptype{none} % do not increment counter
\begin{longtable}[]{@{}lll@{}}
\toprule\noalign{}
પ્રકાર & સૂત્ર & એપ્લિકેશન \\
\midrule\noalign{}
\endhead
\bottomrule\noalign{}
\endlastfoot
\textbf{ઝીરો ઓર્ડર} & x̂(n) = x(n-1) & સરળ પ્રિડિક્ટર \\
\textbf{ફર્સ્ટ ઓર્ડર} & x̂(n) = ax(n-1) & વધુ સારું પ્રિડિક્શન \\
\textbf{હાયર ઓર્ડર} & x̂(n) = Σai\timesx(n-i) & ઓપ્ટિમમ પ્રિડિક્શન \\
\end{longtable}
}

\textbf{ફાયદા:}

\begin{itemize}
\tightlist
\item
  \textbf{રિડ્યુસ્ડ બિટ રેટ}: PCM કરતાં ઓછો બિટ રેટ
\item
  \textbf{બેટર એસએનઆર}: સમાન બિટ રેટ માટે વધુ સારો SNR
\item
  \textbf{પ્રિડિક્ટિવ કોડિંગ}: સિગ્નલ કોરિલેશનનો લાભ લે છે
\end{itemize}

\textbf{એપ્લિકેશન:}

\begin{itemize}
\tightlist
\item
  \textbf{ઇમેજ કમ્પ્રેશન}: JPEG સ્ટાન્ડર્ડ
\item
  \textbf{વીડિયો કોડિંગ}: મોશન કમ્પેન્સેશન
\item
  \textbf{સ્પીચ કોડિંગ}: સ્પીચ કમ્પ્રેશન સિસ્ટમ
\end{itemize}

\textbf{PCM સાથે સરખામણી:}

\begin{itemize}
\tightlist
\item
  \textbf{બિટ રેટ}: DPCM ઓછા બિટ જરૂરી છે
\item
  \textbf{કોમ્પ્લેક્સિટી}: PCM કરતાં વધુ જટિલ
\item
  \textbf{ક્વોલિટી}: સમાન બિટ રેટ પર વધુ સારી ક્વોલિટી
\end{itemize}

\end{solutionbox}
\begin{mnemonicbox}
``Predict Difference, Quantize Less - PDQL''

\end{mnemonicbox}
\begin{center}\rule{0.5\linewidth}{0.5pt}\end{center}

\subsection*{પ્રશ્ન 5(ક) OR [7
ગુણ]}\label{uxaaauxab0uxab6uxaa8-5uxa95-or-7-uxa97uxaa3}

\textbf{4 સ્તરના ડિજિટલ મલ્ટિપ્લેક્સિંગ હાયરાર્કી પર ટૂંકી નોંધ લખો.}

\begin{solutionbox}

\textbf{ડિજિટલ મલ્ટિપ્લેક્સિંગ હાયરાર્કી:}

\textbf{લેવલ સ્ટ્રક્ચર:}

{\def\LTcaptype{none} % do not increment counter
\begin{longtable}[]{@{}lllll@{}}
\toprule\noalign{}
લેવલ & નામ & બિટ રેટ & વોઇસ ચેનલ & એપ્લિકેશન \\
\midrule\noalign{}
\endhead
\bottomrule\noalign{}
\endlastfoot
\textbf{લેવલ 0} & DS-0 & 64 kbps & 1 & મૂળભૂત વોઇસ ચેનલ \\
\textbf{લેવલ 1} & DS-1/T1 & 1.544 Mbps & 24 & પ્રાઇમરી મલ્ટિપ્લેક્સ \\
\textbf{લેવલ 2} & DS-2/T2 & 6.312 Mbps & 96 & સેકન્ડરી મલ્ટિપ્લેક્સ \\
\textbf{લેવલ 3} & DS-3/T3 & 44.736 Mbps & 672 & ટર્શિયરી મલ્ટિપ્લેક્સ \\
\end{longtable}
}

\textbf{મલ્ટિપ્લેક્સિંગ સ્ટ્રક્ચર:}

\begin{center}
\textbf{Mermaid Diagram (Code)}
\begin{verbatim}
{Shaded}
{Highlighting}[]
graph TD
    A[24  DS{-0] {-}{-}{} B[DS{-}1]}
    C[4  DS{-1] {-}{-}{} D[DS{-}2]}
    E[7  DS{-2] {-}{-}{} F[DS{-}3]}
    G[6  DS{-3] {-}{-}{} H[DS{-}4]}
{Highlighting}
{Shaded}
\end{verbatim}
\end{center}

\textbf{T1 માટે ફ્રેમ સ્ટ્રક્ચર:}

\begin{itemize}
\tightlist
\item
  \textbf{ફ્રેમ લેન્થ}: 193 બિટ (192 ડેટા + 1 ફ્રેમિંગ)
\item
  \textbf{ફ્રેમ રેટ}: 8000 ફ્રેમ/સેકન્ડ
\item
  \textbf{ટાઇમ સ્લોટ}: દરેક ચેનલ માટે 8 બિટ
\item
  \textbf{ફ્રેમિંગ બિટ}: સિંક્રોનાઇઝેશન પેટર્ન
\end{itemize}

\textbf{T1 ફ્રેમ ફોર્મેટ:}

\begin{verbatim}
|F|CH1|CH2|...|CH24|F|CH1|CH2|...|CH24|
 ↑              ↑
Framing       193 bits total
\end{verbatim}

\textbf{મલ્ટિપ્લેક્સિંગ પ્રક્રિયા:}

\begin{itemize}
\tightlist
\item
  \textbf{લેવલ 1}: 24 વોઇસ ચેનલ \times 64 kbps + ઓવરહેડ = 1.544 Mbps
\item
  \textbf{લેવલ 2}: 4 T1 સ્ટ્રીમ + ઓવરહેડ = 6.312 Mbps
\item
  \textbf{લેવલ 3}: 7 T2 સ્ટ્રીમ + ઓવરહેડ = 44.736 Mbps
\item
  \textbf{સિંક્રોનાઇઝેશન}: દરેક લેવલ સિંક્રોનાઇઝેશન બિટ ઉમેરે છે
\end{itemize}

\textbf{એપ્લિકેશન:}

\begin{itemize}
\tightlist
\item
  \textbf{ટેલિફોન નેટવર્ક}: ટેલિફોન સિસ્ટમમાં પ્રાથમિક એપ્લિકેશન
\item
  \textbf{ડેટા કમ્યુનિકેશન}: હાઇ-સ્પીડ ડેટા ટ્રાન્સમિશન
\item
  \textbf{ઇન્ટરનેટ બેકબોન}: ઇન્ટરનેટ સર્વિસ પ્રોવાઇડર કનેક્શન
\end{itemize}

\textbf{આંતરરાષ્ટ્રીય સ્ટાન્ડર્ડ:}

\begin{itemize}
\tightlist
\item
  \textbf{નોર્થ અમેરિકન}: T1/T3 હાયરાર્કી (DS શ્રેણી)
\item
  \textbf{યુરોપિયન}: E1/E3 હાયરાર્કી (અલગ બિટ રેટ)
\item
  \textbf{આઇટીયુ-ટી}: આંતરરાષ્ટ્રીય ભલામણો
\end{itemize}

\textbf{ફાયદા:}

\begin{itemize}
\tightlist
\item
  \textbf{સ્ટાન્ડર્ડાઇઝેશન}: સારી રીતે વ્યાખ્યાયિત આંતરરાષ્ટ્રીય સ્ટાન્ડર્ડ
\item
  \textbf{સ્કેલેબિલિટી}: ક્ષમતા વધારવામાં સરળતા
\item
  \textbf{ઇન્ટરઓપરેબિલિટી}: વિવિધ વેન્ડર વચ્ચે સુસંગતતા
\end{itemize}

\end{solutionbox}
\begin{mnemonicbox}
``Digital Signal hierarchy: 0-1-2-3 levels Build
Communication Systems - DS-BCS''

\end{mnemonicbox}

\end{document}
