\documentclass[10pt,a4paper]{article}

% content/resources/templates/preamble.tex
\usepackage[margin=0.6in]{geometry}
\author{Milav Dabgar}
\usepackage{amsmath,amssymb,amsthm}
\usepackage{booktabs}
\usepackage{multirow}
\usepackage{xcolor}
\usepackage{tcolorbox}
\tcbuselibrary{breakable,skins}
\usepackage[colorlinks=true,linkcolor=blue]{hyperref}
\usepackage{titlesec}
\usepackage{enumitem}
\usepackage{tikz}
\usepackage{pgfplots}
\usepackage{circuitikz}
\usepackage[version=4]{mhchem}
\usepackage{longtable}
\usepackage{array}
\usepackage{float}
\usepackage{caption}
\usepackage{listings}

\lstset{
  basicstyle=\small\ttfamily,
  breaklines=true,
  breakatwhitespace=false,
  postbreak=\mbox{\textcolor{red}{$\hookrightarrow$}\space},
  float=false,
  numbers=left,
  numberstyle=\tiny\color{gray},
  numbersep=10pt,
  xleftmargin=2em,
  keywordstyle=\color{blue},
  commentstyle=\color{green!60!black},
  stringstyle=\color{purple},
  backgroundcolor=\color{gray!5},
  showstringspaces=false,
  tabsize=2,
  captionpos=b,
  keepspaces=true,
  columns=flexible
}

\pgfplotsset{compat=1.18}
\usetikzlibrary{shapes,arrows,positioning,calc,patterns,decorations.pathmorphing,decorations.markings,arrows.meta}

% Color scheme
\definecolor{headcolor}{RGB}{0,102,204}
\definecolor{keycolor}{RGB}{220,20,60}
\definecolor{solutioncolor}{RGB}{34,139,34}
\definecolor{mnemoniccolor}{RGB}{148,0,211}
\definecolor{codecolor}{RGB}{0,0,100}

% Spacing
\setlength{\parskip}{3pt}
\setlist[itemize]{nosep}
\setlist[enumerate]{nosep}

% Title formatting
\titleformat{\section}{\Large\bfseries\color{headcolor}}{\thesection}{1em}{}
\titleformat{\subsection}{\large\bfseries\color{headcolor}}{\thesubsection}{1em}{}

% Pandoc tightlist compatibility
\providecommand{\tightlist}{%
  \setlength{\itemsep}{0pt}\setlength{\parskip}{0pt}}

% Pandoc longtable compatibility
\newcounter{none}
\def\thenone{}


% content/resources/templates/gujarati-boxes.tex
\usepackage{fontspec}
\usepackage{polyglossia}

% Set Gujarati as main language (document is primarily in Gujarati)
% Note: gloss-gujarati.ldf doesn't exist in polyglossia, but it will use hyphenation patterns
\setdefaultlanguage{gujarati}
\setotherlanguage{english}

% Configure Gujarati font properly
% Use Language=Default to prevent polyglossia from trying to add language-specific features
% that don't exist for Gujarati, which causes "empty feature" warnings
\newfontfamily\gujaratifont[Script=Gujarati,AutoFakeBold=2.5,AutoFakeSlant=0.3]{Noto Sans Gujarati}
\setmainfont[Script=Gujarati,AutoFakeBold=2.5,AutoFakeSlant=0.3]{Noto Sans Gujarati}
% Use Noto Sans Gujarati for monospace to support Gujarati in text
\setmonofont[Scale=0.9]{Noto Sans Gujarati}

% Configure English to use the same font
\newfontfamily\englishfont[Script=Gujarati,AutoFakeBold=2.5,AutoFakeSlant=0.3]{Noto Sans Gujarati}

% Translations for polyglossia
\gappto\captionsgujarati{
  \renewcommand{\tablename}{કોષ્ટક}
  \renewcommand{\figurename}{આકૃતિ}
}

% Helper for TikZ nodes to ensure Gujarati font
\newcommand{\gu}[1]{{\gujaratifont #1}}

% Custom environments
\newtcolorbox{solutionbox}{
    breakable,
    enhanced,
    colback=solutioncolor!5!white,
    colframe=solutioncolor!75!black,
    fonttitle=\bfseries,
    title=જવાબ
}

\newtcolorbox{solutionboxnobreak}{
 colback=solutioncolor!5!white,
 colframe=solutioncolor!75!black,
 fonttitle=\bfseries,
 title=જવાબ
}

\newtcolorbox{keyformula}{
 breakable,
 enhanced,
 colback=keycolor!5!white,
 colframe=keycolor!75!black,
 fonttitle=\bfseries,
 title=રાસાયણિક સમીકરણ/સૂત્ર
}

\newtcolorbox{mnemonicbox}{
 breakable,
 enhanced,
 colback=mnemoniccolor!5!white,
 colframe=mnemoniccolor!75!black,
 fonttitle=\bfseries,
 title=મેમરી ટ્રીક
}


\begin{document}

\begin{center}
{\Huge\bfseries\color{headcolor} Subject Name (Gujarati)}\\[5pt]
{\LARGE 4331104 -- Winter 2024}\\[3pt]
{\large Semester 1 Study Material}\\[3pt]
{\normalsize\textit{Detailed Solutions and Explanations}}
\end{center}

\vspace{10pt}

\subsection*{પ્રશ્ન 1(a) [3
ગુણ]}\label{q1a}

\textbf{મોડયુલેશન શું છે? તેની જરૂરિયાત શું છે?}

\begin{solutionbox}
મોડ્યુલેશન એ એક ઉચ્ચ આવૃત્તિ કેરિયર સિગ્નલના એક અથવા વધુ ગુણધર્મો
(amplitude, frequency, અથવા phase)ને ઓછી આવૃત્તિના મેસેજ સિગ્નલના તાત્કાલિક
મૂલ્યો અનુસાર બદલવાની પ્રક્રિયા છે.

\textbf{મોડ્યુલેશનની જરૂરિયાત:}

\begin{itemize}
\tightlist
\item
  \textbf{એન્ટેના સાઈઝ ઘટાડવા}: પ્રેક્ટિકલ એન્ટેના સાઈઝ શક્ય બનાવે છે (λ/4)
\item
  \textbf{મલ્ટિપ્લેક્સિંગ}: એક જ માધ્યમનો ઉપયોગ કરીને અનેક સિગ્નલને શેર કરવા
\item
  \textbf{ઇન્ટરફેરન્સ ઘટાડવા}: સિગ્નલને યોગ્ય આવૃત્તિ બેન્ડમાં શિફ્ટ કરે છે
\item
  \textbf{રેન્જ વધારવા}: ટ્રાન્સમિશન અંતરમાં વધારો કરે છે
\end{itemize}

\end{solutionbox}
\begin{mnemonicbox}
``AMIR'' - Antenna, Multiplexing, Interference,
Range

\end{mnemonicbox}
\subsection*{પ્રશ્ન 1(b) [4
ગુણ]}\label{q1b}

\textbf{AM waveના DSBFC માટેનું સમીકરણ તારવો.}

\begin{solutionbox}
DSBFC (Double Sideband Full Carrier) AM wave માટેનું સમીકરણ:

\textbf{ગાણિતિક રીતે તારવવું:}

\begin{itemize}
\tightlist
\item
  કેરિયર સિગ્નલ: c(t) = Ac cos(ωct)
\item
  મેસેજ સિગ્નલ: m(t) = Am cos(ωmt)
\item
  AM સિગ્નલ: s(t) = Ac[1 + μm(t)]cos(ωct)
\item
જ્યાં

μ = મોડ્યુલેશન ઇન્ડેક્સ = Am/Ac

\end{itemize}

\textbf{મેસેજ સિગ્નલ આવવાથી:} s(t) = Ac[1 + μ cos(ωmt)]cos(ωct) s(t) =
Ac cos(ωct) + μAc cos(ωmt)cos(ωct)

\textbf{ત્રિકોણમિતિ સૂત્રનો ઉપયોગ:} cos(A)cos(B) = 1/2[cos(A+B) +
cos(A-B)]

\textbf{અંતિમ સમીકરણ:} s(t) = Ac cos(ωct) + (μAc/2)[cos((ωc+ωm)t) +
cos((ωc-ωm)t)]

\textbf{આકૃતિ:}

\begin{verbatim}
    \^{}
    |    Carrier
Ac  |    /|{}
    |   / | {}
    |  /  |  {}
    | /   |   {}
    |/    |    {}
    +{-{-}{-}{-}{-}+{-}{-}{-}{-}{-}+{-}{-}{-}{-} f}
         fc
\end{verbatim}

\begin{verbatim}
    \^{}
    |                LSB   Carrier   USB
    |                 |      |       |
Pam |                 |      |       |
    |                 |      |       |
    |                 |      |       |
    |                /|{    /|     /|}
    +{-{-}{-}{-}{-}{-}{-}{-}{-}{-}{-}{-}{-}{-}{-}+{-}+{-}{-}{-}{-}{-}+{-}{-}{-}{-}{-}{-}+{-}+{-}{-}{-}{-} f}
                 fc{-fm     fc    fc+fm}
\end{verbatim}

\end{solutionbox}
\subsection*{પ્રશ્ન 1(c) [7
ગુણ]}\label{q1c}

\textbf{નોઈસ સિગ્નલને વર્ગીકૃત કરો. ફ્લીકર નોઈસ, શૉટ નોઈસ અને થર્મલ નોઈસ
સમજાવો.}

\begin{solutionbox}

\textbf{નોઈસનું વર્ગીકરણ:}

{\def\LTcaptype{none} % do not increment counter
\begin{longtable}[]{@{}lll@{}}
\toprule\noalign{}
પ્રકાર & સ્ત્રોત & લક્ષણો \\
\midrule\noalign{}
\endhead
\bottomrule\noalign{}
\endlastfoot
\textbf{બાહ્ય નોઈસ} & પર્યાવરણીય સ્ત્રોત & કોમ્યુનિકેશન સિસ્ટમની બહારના \\
\textbf{આંતરિક નોઈસ} & કોમ્પોનેન્ટ્સ & સિસ્ટમની અંદર ઉત્પન્ન થતા \\
\end{longtable}
}

\textbf{આંતરિક નોઈસના પ્રકાર:}

\begin{enumerate}
\tightlist
\item
  \textbf{ફ્લીકર નોઈસ:}

  \begin{itemize}
  \tightlist
  \item
    \textbf{સ્ત્રોત}: એક્ટિવ ઉપકરણોમાં થાય છે
  \item
    \textbf{લક્ષણો}: આવૃત્તિના વ્યસ્ત પ્રમાણમાં (1/f)
  \item
    \textbf{અસર}: નીચી આવૃત્તિઓ પર મુખ્ય
  \end{itemize}
\item
  \textbf{શૉટ નોઈસ:}

  \begin{itemize}
  \tightlist
  \item
    \textbf{સ્ત્રોત}: જંક્શનમાંથી ઇલેક્ટ્રોનનો રેન્ડમ પ્રવાહ
  \item
    \textbf{લક્ષણો}: આવૃત્તિથી સ્વતંત્ર (વ્હાઈટ નોઈસ)
  \item
    \textbf{અસર}: ડાયોડ/ટ્રાન્ઝિસ્ટરમાં રેન્ડમ કરંટ ફ્લક્ચ્યુએશન
  \end{itemize}
\item
  \textbf{થર્મલ નોઈસ:}

  \begin{itemize}
  \tightlist
  \item
    \textbf{સ્ત્રોત}: તાપમાનને કારણે ઇલેક્ટ્રોનની રેન્ડમ ગતિ
  \item
    \textbf{લક્ષણો}: બધા કન્ડક્ટર, રેઝિસ્ટરમાં મોજુદ
  \item
    \textbf{ફોર્મ્યુલા}: Pn = kTB (k=બોલ્ટઝમેન સ્થિરાંક, T=તાપમાન, B=બેન્ડવિડ્થ)
  \item
    \textbf{અસર}: રિસીવરમાં નોઈસ ફ્લોર સેટ કરે છે
  \end{itemize}
\end{enumerate}

\end{solutionbox}
\begin{mnemonicbox}
``FST'' - Flicker decreases with Frequency, Shot is
from electron flow, Thermal depends on Temperature

\end{mnemonicbox}
\subsection*{પ્રશ્ન 1(c) OR [7
ગુણ]}\label{q1c}

\textbf{EM wave સમજાવો અને સ્પેક્ટ્રમના વિવિધ બેન્ડની એપ્લીકેશન લખો.}

\begin{solutionbox}

\textbf{EM Wave (વિદ્યુત ચુંબકીય તરંગ):} વિદ્યુત ચુંબકીય તરંગો એ સમય સાથે બદલાતાં
ઇલેક્ટ્રિક અને મેગ્નેટિક ફીલ્ડ્સ દ્વારા અવકાશમાં પ્રસરતી ઊર્જા છે, જે પ્રકાશની ગતિએ
(3\times10^{8} m/s) ચાલે છે.

\textbf{લક્ષણો:}

\begin{itemize}
\tightlist
\item
  ટ્રાન્સવર્સ તરંગો જેમાં E અને H ફીલ્ડ એકબીજાના પરપેન્ડીક્યુલર હોય છે
\item
  પ્રસરણ માટે કોઈ માધ્યમની જરૂર નથી
\item
  તરંગલંબાઈ (λ) અને આવૃત્તિ (f) દ્વારા વર્ણવાય છે
\item
  સંબંધ: c = f \times λ
\end{itemize}

\textbf{EM સ્પેક્ટ્રમ અને એપ્લીકેશન:}

{\def\LTcaptype{none} % do not increment counter
\begin{longtable}[]{@{}lll@{}}
\toprule\noalign{}
આવૃત્તિ બેન્ડ & આવૃત્તિ રેન્જ & એપ્લીકેશન \\
\midrule\noalign{}
\endhead
\bottomrule\noalign{}
\endlastfoot
ELF & 3Hz-30Hz & સબમરીન કોમ્યુનિકેશન \\
VLF & 3kHz-30kHz & નેવિગેશન સિસ્ટમ \\
LF & 30kHz-300kHz & AM બ્રોડકાસ્ટિંગ \\
MF & 300kHz-3MHz & AM રેડિયો બ્રોડકાસ્ટિંગ \\
HF & 3MHz-30MHz & શોર્ટવેવ રેડિયો \\
VHF & 30MHz-300MHz & FM રેડિયો, TV બ્રોડકાસ્ટિંગ \\
UHF & 300MHz-3GHz & TV, મોબાઈલ ફોન, WiFi \\
SHF & 3GHz-30GHz & સેટેલાઈટ કોમ્યુનિકેશન, રડાર \\
EHF & 30GHz-300GHz & મિલિમીટર વેવ કોમ્યુનિકેશન \\
Infrared & 300GHz-400THz & રિમોટ કંટ્રોલ, થર્મલ ઈમેજિંગ \\
Visible & 400THz-800THz & ફાઈબર ઓપ્ટિક કોમ્યુનિકેશન \\
Ultraviolet & 800THz-30PHz & સ્ટરિલાઈઝેશન, ઓથેન્ટિકેશન \\
X-Rays & 30PHz-30EHz & મેડિકલ ઈમેજિંગ \\
Gamma Rays & \textgreater30EHz & કેન્સર ટ્રીટમેન્ટ \\
\end{longtable}
}

\textbf{આકૃતિ:}

\begin{verbatim}
      +{-{-}{-}{-}{-}{-}{-}{-}{-}{-}{-}{-}{-}{-}{-}{-}+{-}{-}{-}{-}{-}{-}{-}{-}{-}{-}{-}{-}{-}{-}{-}{-}+{-}{-}{-}{-}{-}{-}{-}{-}{-}{-}{-}{-}{-}{-}{-}{-}+{-}{-}{-}{-}{-}{-}{-}{-}{-}{-}{-}{-}{-}{-}{-}{-}+}
      |                |                |                |                |
Radio   Microwave    Infrared       Visible         Ultraviolet     X{-ray    Gamma}
      |                |                |                |                |
      +{-{-}{-}{-}{-}{-}{-}{-}{-}{-}{-}{-}{-}{-}{-}{-}+{-}{-}{-}{-}{-}{-}{-}{-}{-}{-}{-}{-}{-}{-}{-}{-}+{-}{-}{-}{-}{-}{-}{-}{-}{-}{-}{-}{-}{-}{-}{-}{-}+{-}{-}{-}{-}{-}{-}{-}{-}{-}{-}{-}{-}{-}{-}{-}{-}+}
  Increasing Frequency 
  Decreasing Wavelength 
\end{verbatim}

\end{solutionbox}
\begin{mnemonicbox}
``RMIUXG'' - Radio, Microwave, Infrared,
Ultraviolet, X-ray, Gamma

\end{mnemonicbox}
\subsection*{પ્રશ્ન 2(a) [3
ગુણ]}\label{q2a}

\textbf{DSBની સરખામણીએ SSBના ફાયદાઓ લખો.}

\begin{solutionbox}

\textbf{SSBના DSB કરતાં ફાયદાઓ:}

{\def\LTcaptype{none} % do not increment counter
\begin{longtable}[]{@{}ll@{}}
\toprule\noalign{}
પેરામીટર & SSB ફાયદો \\
\midrule\noalign{}
\endhead
\bottomrule\noalign{}
\endlastfoot
\textbf{બેન્ડવિડ્થ} & 50\% ઓછી બેન્ડવિડ્થની જરૂરિયાત \\
\textbf{પાવર} & 83.33\% પાવર બચત \\
\textbf{ટ્રાન્સમીટર} & ઓછા પાવર એમ્પ્લિફિકેશનની જરૂર \\
\textbf{રિસીવર} & ફેઝ ડિસ્ટોર્શન વગર સરળ ડિઝાઇન \\
\textbf{SNR} & વધુ સારો સિગ્નલ-ટુ-નોઈઝ રેશિયો \\
\textbf{ફેડિંગ} & સિલેક્ટિવ ફેડિંગથી ઓછું અસરગ્રસ્ત \\
\end{longtable}
}

\end{solutionbox}
\begin{mnemonicbox}
``BP TRFS'' - Bandwidth, Power, Transmitter,
Receiver, Fading, SNR

\end{mnemonicbox}
\subsection*{પ્રશ્ન 2(b) [4
ગુણ]}\label{q2b}

\textbf{FET રિએક્ટન્સ મોડ્યુલેટરથી FM વેવનું જનરેશન સમજાવો.}

\begin{solutionbox}

\textbf{FET રિએક્ટન્સ મોડ્યુલેટર:}

\textbf{કાર્ય સિદ્ધાંત:}

\begin{itemize}
\tightlist
\item
  FETને વોલ્ટેજ-કંટ્રોલ્ડ રિએક્ટન્સ તરીકે ઉપયોગ કરે છે
\item
  મોડ્યુલેટિંગ સિગ્નલના આધારે ઇફેક્ટિવ કેપેસિટન્સ બદલે છે
\item
  ઓસિલેટરના LC ટેંક સર્કિટ સાથે જોડાય છે
\end{itemize}

\textbf{સર્કિટ ઓપરેશન:}

\begin{enumerate}
\tightlist
\item
  મોડ્યુલેટિંગ સિગ્નલ FETના ગેટ પર આપવામાં આવે છે
\item
  FETનો ડ્રેન-સોર્સ રેઝિસ્ટન્સ ગેટ વોલ્ટેજ સાથે બદલાય છે
\item
  કેપેસિટિવ રિએક્ટન્સ મોડ્યુલેટિંગ સિગ્નલ સાથે બદલાય છે
\item
  ઓસિલેટરની આવૃત્તિ ઇનપુટ સિગ્નલ સાથે ફેરફાર કરે છે
\end{enumerate}

\textbf{આકૃતિ:}

\begin{verbatim}
      +{-{-}{-}{-}{-}||{-}{-}{-}{-}{-}+}
      |             |
      |             C
      |             |
    V\_in          +{-{-}{-}+}
      |           |FET|
      +{-{-}{-}{-}{-}R{-}{-}{-}{-}{-}|   |}
                  +{-{-}{-}+}
                    |
                   LC
                 Circuit
\end{verbatim}

\textbf{મુખ્ય લક્ષણો:}

\begin{itemize}
\tightlist
\item
  \textbf{સરળ ડિઝાઇન}: અન્ય મોડ્યુલેટર કરતાં ઓછા કોમ્પોનેન્ટ્સ
\item
  \textbf{લિનિયારિટી}: વાઈડ-બેન્ડ FM જનરેશન માટે સારું
\item
  \textbf{સ્થિરતા}: વેરેક્ટર ડાયોડ કરતાં તાપમાનમાં વધુ સ્થિર
\end{itemize}

\end{solutionbox}
\begin{mnemonicbox}
``LOVE FM'' - LC Oscillator with Voltage-controlled
Element for FM

\end{mnemonicbox}
\subsection*{પ્રશ્ન 2(c) [7
ગુણ]}\label{q2c}

\textbf{AM માટે ટોટલ પાવરનું સમીકરણ તારવો. DSB અને SSB માટે પાવર સેવિંગ્સના
ટકાની ગણતરી કરો.}

\begin{solutionbox}

\textbf{AM સિગ્નલમાં પાવર:}

\textbf{AM સિગ્નલ s(t) = Ac[1 + μcos(ωmt)]cos(ωct) માટે}

\textbf{કુલ પાવર ગણતરી:}

\begin{enumerate}
\tightlist
\item
  કેરિયરમાં પાવર: Pc = Ac^{2}/2
\item
  સાઈડબેન્ડમાં પાવર: Ps = μ^{2}Ac^{2}/4 (બન્ને સાઈડબેન્ડ માટે કુલ)
\item
  કુલ પાવર: Pt = Pc + Ps = Ac^{2}/2 \times (1 + μ^{2}/2)
\end{enumerate}

\textbf{100\% મોડ્યુલેશન (μ=1) માટે:}

\begin{itemize}
\tightlist
\item
  Pt = Pc \times (1 + 1/2) = 1.5 \times Pc
\item
  કેરિયર પાવર = કુલ પાવરનો 66.67\%
\item
  સાઈડબેન્ડ પાવર = કુલ પાવરનો 33.33\%
\end{itemize}

\textbf{પાવર સેવિંગ્સ:}

\begin{enumerate}
\tightlist
\item
  \textbf{DSB-SC માં:}

  \begin{itemize}
  \tightlist
  \item
    કેરિયર સપ્રેસ થાય છે
  \item
    66.67\% પાવર બચે છે
  \end{itemize}
\item
  \textbf{SSB માં:}

  \begin{itemize}
  \tightlist
  \item
    કેરિયર + એક સાઈડબેન્ડ સપ્રેસ થાય છે
  \item
    66.67\% + 16.67\% = 83.33\% પાવર બચે છે
  \end{itemize}
\end{enumerate}

\textbf{તુલનાત્મક ટેબલ:}

{\def\LTcaptype{none} % do not increment counter
\begin{longtable}[]{@{}
  >{\raggedright\arraybackslash}p{(\linewidth - 8\tabcolsep) * \real{0.1714}}
  >{\raggedright\arraybackslash}p{(\linewidth - 8\tabcolsep) * \real{0.2143}}
  >{\raggedright\arraybackslash}p{(\linewidth - 8\tabcolsep) * \real{0.2286}}
  >{\raggedright\arraybackslash}p{(\linewidth - 8\tabcolsep) * \real{0.1857}}
  >{\raggedright\arraybackslash}p{(\linewidth - 8\tabcolsep) * \real{0.2000}}@{}}
\toprule\noalign{}
\begin{minipage}[b]{\linewidth}\raggedright
મોડ્યુલેશન
\end{minipage} & \begin{minipage}[b]{\linewidth}\raggedright
કેરિયર પાવર
\end{minipage} & \begin{minipage}[b]{\linewidth}\raggedright
સાઈડબેન્ડ પાવર
\end{minipage} & \begin{minipage}[b]{\linewidth}\raggedright
કુલ પાવર
\end{minipage} & \begin{minipage}[b]{\linewidth}\raggedright
પાવર સેવિંગ
\end{minipage} \\
\midrule\noalign{}
\endhead
\bottomrule\noalign{}
\endlastfoot
AM (μ=1) & 100\% & 50\% & 150\% & 0\% \\
DSB-SC & 0\% & 50\% & 50\% & 66.67\% \\
SSB & 0\% & 25\% & 25\% & 83.33\% \\
\end{longtable}
}

\end{solutionbox}
\begin{mnemonicbox}
``CST'' - Carrier power, Sideband power, Total power

\end{mnemonicbox}
\subsection*{પ્રશ્ન 2(a) OR [3
ગુણ]}\label{q2a}

\textbf{AM વેવ માટે Time domain અને Frequency domain ડિસપ્લે દોરો અને સમજાવો.}

\begin{solutionbox}

\textbf{AM વેવના Time Domain અને Frequency Domain ડિસપ્લે:}

\textbf{Time Domain (સમય ડોમેન):}

\begin{itemize}
\tightlist
\item
  સમય સાથે એમ્પ્લિટ્યુડમાં થતા ફેરફાર બતાવે છે
\item
  એન્વેલોપ મોડ્યુલેટિંગ સિગ્નલને અનુસરે છે
\item
  મહત્તમ એમ્પ્લિટ્યુડ: A_{1} = Ac(1+μ)
\item
  ન્યૂનતમ એમ્પ્લિટ્યુડ: A_{2} = Ac(1-μ)
\item
  મોડ્યુલેશન ઇન્ડેક્સ: μ = (A_{1}-A_{2})/(A_{1}+A_{2})
\end{itemize}

\textbf{Frequency Domain (આવૃત્તિ ડોમેન):}

\begin{itemize}
\tightlist
\item
  આવૃત્તિઓ પર પાવર ડિસ્ટ્રિબ્યુશન બતાવે છે
\item
  કેરિયર સેન્ટર આવૃત્તિ fc પર
\item
  અપર સાઈડબેન્ડ fc+fm પર
\item
  લોઅર સાઈડબેન્ડ fc-fm પર
\item
  બેન્ડવિડ્થ = 2fm
\end{itemize}

\textbf{આકૃતિ:}

\begin{verbatim}
Time Domain:                         Frequency Domain:
    \^{                                    \^{}}
    |                                    |
A_{1  |    /      /                      |          Carrier}
    |   /  {    /                       |             |}
Ac  |{-{-}/{-}{-}{-}{-}{-}{-}/{-}{-}{-}{-}{-}{-}                  |             |}
    |  {    /      /                    |   LSB       |       USB}
A_{2  |     /      /                     |    |        |        |}
    |    {/      /                      |    |        |        |}
    +{-{-}{-}{-}{-}{-}{-}{-}{-}{-}{-}{-}{-}{-}{-}{-}{-}{-}{-}{-}{-}{-}{-}{-}{-}{-}{-}        +{-}{-}{-}{-}+{-}{-}{-}{-}{-}{-}{-}{-}{-}+{-}{-}{-}{-}{-}{-}{-}{-}+{-}{-}{-}{-}{-}{-}}
        t                                   fc{-fm      fc      fc+fm}
\end{verbatim}

\end{solutionbox}
\begin{mnemonicbox}
``TEF'' - Time domain shows Envelope, Frequency
domain shows spectral components

\end{mnemonicbox}
\subsection*{પ્રશ્ન 2(b) OR [4
ગુણ]}\label{q2b}

\textbf{પ્રી-એમફાસીસ અને ડી-એમફાસીસ સર્કિટ સમજાવો.}

\begin{solutionbox}

\textbf{પ્રી-એમફાસીસ અને ડી-એમફાસીસ સર્કિટ:}

\textbf{હેતુ:}

\begin{itemize}
\tightlist
\item
  ઉચ્ચ આવૃત્તિના ઘટકો માટે SNR સુધારવા
\item
  ઉચ્ચ આવૃત્તિમાં વધુ નોઈઝ માટે કમ્પેન્સેશન
\item
  મુખ્યત્વે FM સિસ્ટમમાં વપરાય છે
\end{itemize}

\textbf{પ્રી-એમફાસીસ:}

\begin{itemize}
\tightlist
\item
  ટ્રાન્સમીટર પર લાગુ કરવામાં આવે છે
\item
  ઉચ્ચ આવૃત્તિ ઘટકોને બૂસ્ટ કરે છે
\item
  સામાન્ય રીતે 2.1kHz ઉપર +6dB/ઓક્ટેવ
\item
  સર્કિટ: હાઈ-પાસ RC નેટવર્ક (સીરીઝમાં રેઝિસ્ટર, પેરેલલમાં કેપેસિટર)
\end{itemize}

\textbf{ડી-એમફાસીસ:}

\begin{itemize}
\tightlist
\item
  રિસીવર પર લાગુ કરવામાં આવે છે
\item
  ઉચ્ચ આવૃત્તિ ઘટકોને એટેન્યુએટ કરે છે
\item
  ઓરિજિનલ સિગ્નલ બેલેન્સ રીસ્ટોર કરે છે
\item
  સર્કિટ: લો-પાસ RC નેટવર્ક (પેરેલલમાં રેઝિસ્ટર, સીરીઝમાં કેપેસિટર)
\end{itemize}

\textbf{આકૃતિઓ:}

\begin{verbatim}
Pre{-emphasis:                    De{-}emphasis:}
    R                                C
+{-{-}{-}www{-}{-}{-}+{-}{-}{-}+                 +{-}{-}{-}||{-}{-}{-}+{-}{-}{-}+}
|         |   |                 |        |   |
Vin       C   Vout              Vin      R   Vout
|         |   |                 |        |   |
+{-{-}{-}{-}{-}{-}{-}{-}{-}{-}   |                 +{-}{-}{-}{-}{-}{-}{-}{-}{-}   |}
             {-{-}{-}                            {-}{-}{-}}
              {-                              {-}}
\end{verbatim}

\textbf{આવૃત્તિ પ્રતિસાદ:}

\begin{verbatim}
    \^{}
    |        Pre{-emphasis}
Gain|          /
    |         /
    |        /
0dB |{-{-}{-}{-}{-}{-}{-}/}
    |      /       De{-emphasis}
    |     /          {}
    |    /            {}
    +{-{-}{-}{-}{-}{-}{-}{-}{-}{-}{-}{-}{-}{-}{-}{-}{-}{-}{-}{-}}
       2.1kHz           f
\end{verbatim}

\end{solutionbox}
\begin{mnemonicbox}
``HIGH-LOW'' - HIGHer frequencies boosted at
transmitter, LOWered at receiver

\end{mnemonicbox}
\subsection*{પ્રશ્ન 2(c) OR [7
ગુણ]}\label{q2c}

\textbf{નેરોબેન્ડ FM અને વાઈડબેન્ડ FMને સરખાવો.}

\begin{solutionbox}

\textbf{નેરોબેન્ડ FM અને વાઈડબેન્ડ FMની તુલના:}

{\def\LTcaptype{none} % do not increment counter
\begin{longtable}[]{@{}
  >{\raggedright\arraybackslash}p{(\linewidth - 4\tabcolsep) * \real{0.2821}}
  >{\raggedright\arraybackslash}p{(\linewidth - 4\tabcolsep) * \real{0.3846}}
  >{\raggedright\arraybackslash}p{(\linewidth - 4\tabcolsep) * \real{0.3333}}@{}}
\toprule\noalign{}
\begin{minipage}[b]{\linewidth}\raggedright
પેરામીટર
\end{minipage} & \begin{minipage}[b]{\linewidth}\raggedright
નેરોબેન્ડ FM
\end{minipage} & \begin{minipage}[b]{\linewidth}\raggedright
વાઈડબેન્ડ FM
\end{minipage} \\
\midrule\noalign{}
\endhead
\bottomrule\noalign{}
\endlastfoot
\textbf{મોડ્યુલેશન ઇન્ડેક્સ (β)} & β \textless\textless{} 1 (સામાન્ય રીતે
\textless0.5) & β \textgreater\textgreater{} 1 (સામાન્ય રીતે
\textgreater5) \\
\textbf{બેન્ડવિડ્થ} & 2fm (મેસેજ બેન્ડવિડ્થની બમણી) & 2fm(β+1) (કાર્સનનો નિયમ) \\
\textbf{મહત્વપૂર્ણ સાઈડબેન્ડ્સ} & માત્ર પ્રથમ જોડી સાઈડબેન્ડ્સ & અનેક સાઈડબેન્ડ્સ \\
\textbf{એપ્લિકેશન} & મોબાઈલ કોમ્યુનિકેશન, ટુ-વે રેડિયો & FM બ્રોડકાસ્ટિંગ,
હાઈ-ફિડેલિટી ઓડિયો \\
\textbf{સિગ્નલ ક્વોલિટી} & ઓછી ફિડેલિટી, ઓછી નોઈઝ ઇમ્યુનિટી & વધુ ફિડેલિટી, વધુ
સારી નોઈઝ ઇમ્યુનિટી \\
\textbf{પાવર એફિશિયન્સી} & વધુ & ઓછી \\
\textbf{સ્પેક્ટ્રમ ઉપયોગ} & કાર્યક્ષમ & ઓછો કાર્યક્ષમ \\
\textbf{સર્કિટ જટિલતા} & સરળ & વધુ જટિલ \\
\end{longtable}
}

\textbf{બેન્ડવિડ્થ ગણતરી:}

\begin{itemize}
\tightlist
\item
  નેરોબેન્ડ FM: BW = 2fm
\item
  વાઈડબેન્ડ FM: BW = 2fm(β+1) (કાર્સનનો નિયમ)
\end{itemize}

\textbf{સ્પેક્ટ્રમ આકૃતિ:}

\begin{verbatim}
Narrowband FM:                    Wideband FM:
    \^{                                 \^{}}
    |                                 |
    |            |                    |
    |        |   |   |                |   | | | | | | | | |
    |    |   |   |   |   |            | | | | | | | | | | | | | |
    +{-{-}{-}{-}{-}{-}{-}{-}{-}{-}{-}{-}{-}{-}{-}{-}{-}{-}{-}{-}{-}{-}          +{-}{-}{-}{-}{-}{-}{-}{-}{-}{-}{-}{-}{-}{-}{-}{-}{-}{-}{-}{-}{-}{-}{-}{-}{-}{-}}
       fc{-fm  fc  fc+fm                  fc{-}5fm    fc    fc+5fm}
\end{verbatim}

\end{solutionbox}
\begin{mnemonicbox}
``BASPCB'' - Bandwidth, Applications, Sidebands,
Power, Complexity, Beta

\end{mnemonicbox}
\subsection*{પ્રશ્ન 3(a) [3
ગુણ]}\label{q3a}

\textbf{રેડીઓ રીસીવરની કોઈ ચાર લાક્ષણિકતાઓ વ્યાખ્યાઈત કરો.}

\begin{solutionbox}

\textbf{રેડિયો રિસીવરની લાક્ષણિકતાઓ:}

\begin{enumerate}
\tightlist
\item
  \textbf{સેન્સિટિવિટી:}

  \begin{itemize}
  \tightlist
  \item
    નબળા સિગ્નલને એમ્પ્લિફાય કરવાની ક્ષમતા
  \item
    માઈક્રોવોલ્ટ (μV)માં માપવામાં આવે છે
  \item
    સામાન્ય રીતે સારા રિસીવર્સ માટે 1-10μV
  \end{itemize}
\item
  \textbf{સિલેક્ટિવિટી:}

  \begin{itemize}
  \tightlist
  \item
    અડોસપડોસની ચેનલથી ઇચ્છિત સિગ્નલને અલગ કરવાની ક્ષમતા
  \item
    IF એમ્પ્લિફાયરની બેન્ડવિડ્થ દ્વારા નિર્ધારિત
  \item
    ચોક્કસ આવૃત્તિ ઓફસેટ્સ પર dBમાં માપવામાં આવે છે
  \end{itemize}
\item
  \textbf{ફિડેલિટી:}

  \begin{itemize}
  \tightlist
  \item
    ઓરિજિનલ સિગ્નલને અચૂક રીતે રિપ્રોડ્યુસ કરવાની ક્ષમતા
  \item
    બેન્ડવિડ્થ અને ડિસ્ટોર્શન પર આધાર રાખે છે
  \item
    આવૃત્તિ પ્રતિસાદની સપાટતા તરીકે માપવામાં આવે છે
  \end{itemize}
\item
  \textbf{ઇમેજ ફ્રિક્વન્સી રિજેક્શન:}

  \begin{itemize}
  \tightlist
  \item
    ઇમેજ આવૃત્તિ (fi = fs \pm 2fIF) પર સિગ્નલને રિજેક્ટ કરવાની ક્ષમતા
  \item
    dBમાં માપવામાં આવે છે
  \item
    ઉચ્ચ મૂલ્યો વધુ સારી કામગીરી દર્શાવે છે
  \end{itemize}
\end{enumerate}

\textbf{વધારાની લાક્ષણિકતાઓ:}

\begin{itemize}
\tightlist
\item
  સિગ્નલ-ટુ-નોઈઝ રેશિયો (SNR)
\item
  ઓટોમેટિક ગેઈન કંટ્રોલ (AGC) રેન્જ
\item
  ડાયનેમિક રેન્જ
\end{itemize}

\end{solutionbox}
\begin{mnemonicbox}
``SFID'' - Sensitivity, Fidelity, Image rejection,
selectivity Determines quality

\end{mnemonicbox}
\subsection*{પ્રશ્ન 3(b) [4
ગુણ]}\label{q3b}

\textbf{ડાયોડ ડિટેક્ટર સર્કિટ સમજાવો.}

\begin{solutionbox}

\textbf{ડાયોડ ડિટેક્ટર સર્કિટ:}

\textbf{હેતુ:}

\begin{itemize}
\tightlist
\item
  AM વેવમાંથી ઓરિજિનલ મેસેજ સિગ્નલ એક્સટ્રેક્ટ કરે છે
\item
  એન્વેલોપ ડિટેક્ટર પણ કહેવાય છે
\end{itemize}

\textbf{સર્કિટ કોમ્પોનેન્ટ્સ:}

\begin{itemize}
\tightlist
\item
  ડાયોડ: AM સિગ્નલને રેક્ટિફાય કરે છે
\item
  RC નેટવર્ક: કેરિયર આવૃત્તિને ફિલ્ટર કરે છે
\item
  R \& C મૂલ્યો: RC \textgreater\textgreater{} 1/fc અને RC
  \textless\textless{} 1/fm
\end{itemize}

\textbf{ઓપરેશન:}

\begin{enumerate}
\tightlist
\item
  ડાયોડ પોઝિટિવ હાફ-સાયકલ દરમિયાન કન્ડક્ટ કરે છે
\item
  કેપેસિટર પીક વેલ્યુ સુધી ચાર્જ થાય છે
\item
  કેપેસિટર રેઝિસ્ટર દ્વારા ડિસ્ચાર્જ થાય છે
\item
  યોગ્ય ડિમોડ્યુલેશન માટે RC ટાઈમ કોન્સ્ટન્ટ મહત્વપૂર્ણ છે
\end{enumerate}

\textbf{આકૃતિ:}

\begin{verbatim}
          D
        +{-{-}{-}{-}{-}||{-}{-}{-}{-}+}
        |            |
Input   |            C    R     Output
AM      |            |    |      
        +{-{-}{-}{-}{-}{-}{-}{-}{-}{-}{-}{-}+{-}{-}{-}{-}+{-}{-}{-}{-}{-}+}
                     |          |
                    {-{-}{-}        {-}{-}{-}}
                     {-          {-}}
\end{verbatim}

\textbf{વેવફોર્મ્સ:}

\begin{verbatim}
Input:                  After Diode:            Output:
    /{      /              /      /              \_\_\_\_\_}
   /  {    /              /      /              /     }
  /    {  /             /      /             /       }
 /      {/              /      /      }
\end{verbatim}

\textbf{મર્યાદાઓ:}

\begin{itemize}
\tightlist
\item
  ઉચ્ચ મોડ્યુલેશન ઇન્ડેક્સ માટે ડિસ્ટોર્શન
\item
  નીચા સિગ્નલ સ્તરે ખરાબ પ્રદર્શન
\end{itemize}

\end{solutionbox}
\begin{mnemonicbox}
``DRCO'' - Diode Rectifies, Capacitor holds peaks,
Output follows envelope

\end{mnemonicbox}
\subsection*{પ્રશ્ન 3(c) [7
ગુણ]}\label{q3c}

\textbf{સુપર હેટેરોડાઈન રીસીવરનો બ્લોક ડાયગ્રામ દોરો અને સમજાવો.}

\begin{solutionbox}

\textbf{સુપર હેટેરોડાઈન રીસીવર:}

\textbf{બ્લોક ડાયગ્રામ:}

\begin{verbatim}
+{-{-}{-}{-}{-}{-}{-}{-}+    +{-}{-}{-}{-}{-}{-}{-}+    +{-}{-}{-}{-}{-}{-}+    +{-}{-}{-}{-}{-}+    +{-}{-}{-}{-}{-}{-}{-}{-}+    +{-}{-}{-}{-}{-}{-}{-}{-}+    +{-}{-}{-}{-}{-}{-}{-}{-}+}
| Antenna|{-{-}{-}| RF    |{-}{-}{-}| Mixer|{-}{-}{-}| IF  |{-}{-}{-}|Detector|{-}{-}{-}| Audio  |{-}{-}{-}|Speaker |}
|        |    | Amp   |    |      |    | Amp |    |        |    | Amp    |    |        |
+{-{-}{-}{-}{-}{-}{-}{-}+    +{-}{-}{-}{-}{-}{-}{-}+    +{-}{-}{-}{-}{-}{-}+    +{-}{-}{-}{-}{-}+    +{-}{-}{-}{-}{-}{-}{-}{-}+    +{-}{-}{-}{-}{-}{-}{-}{-}+    +{-}{-}{-}{-}{-}{-}{-}{-}+}
                              \^{}
                              |
                        +{-{-}{-}{-}{-}{-}{-}{-}{-}{-}{-}{-}+}
                        | Local      |
                        | Oscillator |
                        +{-{-}{-}{-}{-}{-}{-}{-}{-}{-}{-}{-}+}
\end{verbatim}

\textbf{દરેક બ્લોકનું કાર્ય:}

\begin{enumerate}
\tightlist
\item
  \textbf{RF એમ્પ્લિફાયર:}

  \begin{itemize}
  \tightlist
  \item
    નબળા RF સિગ્નલ્સને એમ્પ્લિફાય કરે છે
  \item
    સિલેક્ટિવિટી પૂરી પાડે છે
  \item
    સિગ્નલ-ટુ-નોઈઝ રેશિયોમાં સુધારો કરે છે
  \end{itemize}
\item
  \textbf{લોકલ ઓસિલેટર:}

  \begin{itemize}
  \tightlist
  \item
    સ્થિર આવૃત્તિ fLO જનરેટ કરે છે
  \item
    fLO = fRF + fIF (હાઈ-સાઈડ ઇન્જેક્શન માટે)
  \item
    RF એમ્પ્લિફાયર સાથે ટ્યુન થયેલું
  \end{itemize}
\item
  \textbf{મિક્સર:}

  \begin{itemize}
  \tightlist
  \item
    RF સિગ્નલને લોકલ ઓસિલેટર સાથે કોમ્બાઈન કરે છે
  \item
    સરવાળા અને તફાવતની આવૃત્તિઓ ઉત્પન્ન કરે છે
  \item
    તફાવતની આવૃત્તિ = IF (ઇન્ટરમીડિએટ આવૃત્તિ)
  \end{itemize}
\item
  \textbf{IF એમ્પ્લિફાયર:}

  \begin{itemize}
  \tightlist
  \item
    ફિક્સ્ડ આવૃત્તિ એમ્પ્લિફિકેશન (AM માટે સામાન્ય રીતે 455kHz)
  \item
    રિસીવરનો મોટાભાગનો ગેઈન અને સિલેક્ટિવિટી પૂરા પાડે છે
  \item
    વધુ સારા પ્રદર્શન માટે મલ્ટિપલ સ્ટેજ
  \end{itemize}
\item
  \textbf{ડિટેક્ટર:}

  \begin{itemize}
  \tightlist
  \item
    IF સિગ્નલને ડિમોડ્યુલેટ કરે છે
  \item
    ઓરિજિનલ મેસેજ સિગ્નલ એક્સટ્રેક્ટ કરે છે
  \item
    AM માટે ડાયોડ ડિટેક્ટર, FM માટે ડિસ્ક્રિમિનેટર
  \end{itemize}
\item
  \textbf{ઓડિયો એમ્પ્લિફાયર:}

  \begin{itemize}
  \tightlist
  \item
    ડિમોડ્યુલેટેડ સિગ્નલને એમ્પ્લિફાય કરે છે
  \item
    સ્પીકર અથવા હેડફોન ચલાવે છે
  \end{itemize}
\end{enumerate}

\textbf{કાર્ય સિદ્ધાંત:}

\begin{itemize}
\tightlist
\item
  કોઈપણ RF આવૃત્તિને કાર્યક્ષમ એમ્પ્લિફિકેશન માટે ફિક્સ્ડ IF માં કન્વર્ટ કરે છે
\item
  IF આવૃત્તિ = \textbar fRF - fLO\textbar{}
\end{itemize}

\textbf{ફાયદાઓ:}

\begin{itemize}
\tightlist
\item
  વધુ સારી સિલેક્ટિવિટી અને સેન્સિટિવિટી
\item
  બધી આવૃત્તિઓ પર સ્થિર ગેઈન
\item
  ટ્રેકિંગ સમસ્યાઓમાં ઘટાડો
\end{itemize}

\end{solutionbox}
\begin{mnemonicbox}
``RLMIDS'' - RF amp, Local oscillator, Mixer, IF
amp, Detector, Speaker

\end{mnemonicbox}
\subsection*{પ્રશ્ન 3(a) OR [3
ગુણ]}\label{q3a}

\textbf{AGC નો સિદ્ધાંત અને રેડિયો રિસીવરમાં તેની ઉપયોગિતા જણાવો.}

\begin{solutionbox}

\textbf{AGC (ઓટોમેટિક ગેઈન કંટ્રોલ) સિદ્ધાંત:}

\textbf{વ્યાખ્યા:}

\begin{itemize}
\tightlist
\item
  સર્કિટ જે સિગ્નલની શક્તિના આધારે ઓટોમેટિક રીતે રિસીવર ગેઈન એડજસ્ટ કરે છે
\item
  અલગ-અલગ ઇનપુટ સિગ્નલ છતાં સતત આઉટપુટ લેવલ જાળવે છે
\end{itemize}

\textbf{કાર્ય સિદ્ધાંત:}

\begin{enumerate}
\tightlist
\item
  રિસીવ્ડ સિગ્નલની શક્તિને ડિટેક્ટ કરે છે
\item
  સિગ્નલના પ્રમાણમાં કંટ્રોલ વોલ્ટેજ જનરેટ કરે છે
\item
  મજબૂત સિગ્નલ માટે ગેઈન ઘટાડવા માટે નેગેટિવ ફીડબેક લાગુ કરે છે
\item
  નબળા સિગ્નલ માટે ગેઈન વધારે છે
\end{enumerate}

\textbf{રેડિયો રિસીવરમાં એપ્લિકેશન:}

\begin{itemize}
\tightlist
\item
  \textbf{ઓવરલોડિંગ અટકાવે છે:} મજબૂત સિગ્નલ ડિસ્ટોર્શનથી રક્ષણ કરે છે
\item
  \textbf{ફેડિંગ માટે કમ્પેન્સેશન:} સિગ્નલ ફેડિંગ દરમિયાન અવાજનું સતત વોલ્યુમ જાળવે છે
\item
  \textbf{IF એમ્પ્લિફાયર કંટ્રોલ:} મુખ્યત્વે IF સ્ટેજ પર લાગુ કરવામાં આવે છે
\item
  \textbf{ડાયનેમિક રેન્જ સુધારે છે:} સિગ્નલની શક્તિની વિશાળ શ્રેણીને સંભાળે છે
\end{itemize}

\textbf{પ્રકારો:}

\begin{itemize}
\tightlist
\item
  \textbf{સિમ્પલ AGC:} ડિટેક્ટરથી સીધું ફીડબેક
\item
  \textbf{ડિલેડ AGC:} થ્રેશોલ્ડ લેવલ ઉપર જ સક્રિય થાય છે
\item
  \textbf{એમ્પ્લિફાઈડ AGC:} વધુ સારા કંટ્રોલ માટે વધારાના એમ્પ્લિફાયરનો ઉપયોગ કરે
  છે
\end{itemize}

\textbf{આકૃતિ:}

\begin{verbatim}
     +{-{-}{-}{-}{-}{-}{-}+    +{-}{-}{-}{-}{-}{-}+    +{-}{-}{-}{-}{-}+    +{-}{-}{-}{-}{-}{-}{-}{-}+}
     | RF    |{-{-}{-}| Mixer|{-}{-}{-}| IF  |{-}{-}{-}|Detector|{-}{-}{-} Audio}
     | Amp   |    |      |    | Amp |    |        |
     +{-{-}{-}|{-}{-}{-}+    +{-}{-}{-}{-}{-}{-}+    +{-}|{-}{-}{-}+    +{-}{-}{-}{-}|{-}{-}{-}+}
         |                      |             |
         |                      |         +{-{-}{-}{-}{-}{-}{-}{-}+}
         |                      |{-{-}{-}{-}{-}{-}{-}{-}{-}|  AGC   |}
         |                                | Circuit|
         +{-{-}{-}{-}{-}{-}{-}{-}{-}{-}{-}{-}{-}{-}{-}{-}{-}{-}{-}{-}{-}{-}{-}{-}{-}{-}{-}{-}{-}{-}{-}{-}|        |}
                                          +{-{-}{-}{-}{-}{-}{-}{-}+}
\end{verbatim}

\end{solutionbox}
\begin{mnemonicbox}
``FADS'' - Fading compensation, Automatic
adjustment, Dynamic range, Signal consistency

\end{mnemonicbox}
\subsection*{પ્રશ્ન 3(b) OR [4
ગુણ]}\label{q3b}

\textbf{IF frequency પર ટૂકનોંધ લખો.}

\begin{solutionbox}

\textbf{ઇન્ટરમીડિએટ આવૃત્તિ (IF):}

\textbf{વ્યાખ્યા:}

\begin{itemize}
\tightlist
\item
  સુપરહેટેરોડાઈન રિસીવર્સમાં ઇનકમિંગ RF સિગ્નલને કન્વર્ટ કરવામાં આવતી ફિક્સ્ડ આવૃત્તિ
\item
  RF સિગ્નલને લોકલ ઓસિલેટર સાથે મિક્સિંગ (હેટેરોડાઈનિંગ)નું પરિણામ
\end{itemize}

\textbf{સ્ટાન્ડર્ડ IF મૂલ્યો:}

\begin{itemize}
\tightlist
\item
  \textbf{AM રેડિયો:} 455 kHz
\item
  \textbf{FM રેડિયો:} 10.7 MHz
\item
  \textbf{TV રિસીવર્સ:} 38-41 MHz
\end{itemize}

\textbf{મહત્વ:}

\begin{itemize}
\tightlist
\item
  \textbf{કન્સિસ્ટન્ટ ગેઈન:} એમ્પ્લિફાયર્સ ફિક્સ્ડ આવૃત્તિ પર કાર્ય કરે છે
\item
  \textbf{બેટર સિલેક્ટિવિટી:} ફિક્સ્ડ આવૃત્તિ પર નેરોબેન્ડ ફિલ્ટર્સ
\item
  \textbf{સિમ્પ્લિફાઈડ ડિઝાઈન:} ફિક્સ્ડ-આવૃત્તિ સ્ટેજના કાર્યક્ષમ ડિઝાઈન કરવું સરળ
\end{itemize}

\textbf{પસંદગી માપદંડ:}

\begin{itemize}
\tightlist
\item
  ઇમેજ રિજેક્શન માટે પૂરતી ઊંચી
\item
  ફિલ્ટર Q અને ગેઈન માટે પૂરતી નીચી
\item
  સામાન્ય સિગ્નલના હાર્મોનિક્સને ટાળવી જોઈએ
\end{itemize}

\textbf{ઇમેજ આવૃત્તિ ગણતરી:}

\begin{itemize}
\tightlist
\item
  હાઈ-સાઈડ ઇન્જેક્શન: fimage = fRF + 2fIF
\item
  લો-સાઈડ ઇન્જેક્શન: fimage = fRF - 2fIF
\end{itemize}

\textbf{આકૃતિ:}

\begin{verbatim}
   Original      IF Stage      Audio
   Spectrum        Fixed       Output
     |  |           |  |          |
     V  V           V  V          V
+{-{-}{-}{-}{-}{-}{-}{-}{-}{-}+    +{-}{-}{-}{-}{-}{-}{-}{-}{-}{-}+    +{-}{-}{-}{-}{-}+}
|  Mixer   |{-{-}{-}|    IF    |{-}{-}{-}| Det |}
+{-{-}{-}{-}{-}{-}{-}{-}{-}{-}+    +{-}{-}{-}{-}{-}{-}{-}{-}{-}{-}+    +{-}{-}{-}{-}{-}+}
      \^{}
      |
+{-{-}{-}{-}{-}{-}{-}{-}{-}{-}{-}{-}+}
|  Local     |
| Oscillator |
+{-{-}{-}{-}{-}{-}{-}{-}{-}{-}{-}{-}+}
\end{verbatim}

\end{solutionbox}
\begin{mnemonicbox}
``CIGS'' - Conversion, Improved selectivity, Gain
stability, Simplified design

\end{mnemonicbox}
\subsection*{પ્રશ્ન 3(c) OR [7
ગુણ]}\label{q3c}

\textbf{FM detection માટેની ફેસ ડિસ્ક્રિમિનેટર સર્કિટ સમજાવો.}

\begin{solutionbox}

\textbf{FM Detection માટે ફેસ ડિસ્ક્રિમિનેટર:}

\textbf{હેતુ:}

\begin{itemize}
\tightlist
\item
  FM સિગ્નલમાં આવૃત્તિ વેરિએશનને એમ્પ્લિટ્યુડ વેરિએશનમાં કન્વર્ટ કરે છે
\item
  FM સિગ્નલને ડિમોડ્યુલેટ કરીને ઓરિજિનલ મેસેજ રિકવર કરે છે
\end{itemize}

\textbf{સર્કિટ કોમ્પોનેન્ટ્સ:}

\begin{itemize}
\tightlist
\item
  સેન્ટર-ટેપ્ડ ટ્રાન્સફોર્મર
\item
  બે ડાયોડ્સ (D1 અને D2)
\item
  RC ફિલ્ટર નેટવર્ક
\item
  ફેઝ-શિફ્ટિંગ નેટવર્ક (L-C સર્કિટ)
\end{itemize}

\textbf{કાર્ય સિદ્ધાંત:}

\begin{enumerate}
\tightlist
\item
  ઇનપુટ FM સિગ્નલ બે પાથમાં વિભાજિત થાય છે
\item
  રેફરન્સ પાથ સીધો સેન્ટર ટેપ પર જાય છે
\item
  ફેઝ-શિફ્ટેડ પાથ LC નેટવર્ક મારફતે પસાર થાય છે
\item
  ફેઝ શિફ્ટ આવૃત્તિ ડેવિએશન સાથે બદલાય છે
\item
  બે ડાયોડ્સ ફેઝ ડિફરન્સના પ્રમાણમાં વોલ્ટેજ ઉત્પન્ન કરે છે
\item
  આઉટપુટ વોલ્ટેજ ઇનપુટ આવૃત્તિ સાથે બદલાય છે
\end{enumerate}

\textbf{સર્કિટ આકૃતિ:}

\begin{verbatim}
                      D1
              +{-{-}{-}{-}{-}{-}||{-}{-}{-}{-}{-}{-}+}
              |               |
              |               R1
              |               |
              |               |
 FM Input     |               +{-{-}{-}+}
 +{-{-}{-}{-}{-}{-}+     |                   |}
 |      |{-{-}{-}{-}{-}+                   +{-}{-}{-} Output}
 +{-{-}{-}{-}{-}{-}+     |                   |}
              |               +{-{-}{-}+}
              |               |
              |               R2
              |               |
              +{-{-}{-}{-}{-}{-}||{-}{-}{-}{-}{-}{-}+}
                      D2
\end{verbatim}

\textbf{લક્ષણો:}

\begin{itemize}
\tightlist
\item
  મધ્યમ આવૃત્તિ રેન્જ પર \textbf{લિનિયર રિસ્પોન્સ}
\item
  એમ્પ્લિટ્યુડ વેરિએશન ઘટાડે તેવી \textbf{બેલેન્સ્ડ ડિઝાઈન}
\item
  આવૃત્તિ ફેરફારો માટે \textbf{હાઈ સેન્સિટિવિટી}
\item
  આત્યંતિક આવૃત્તિ ડેવિએશન પર \textbf{મર્યાદાઓ}
\end{itemize}

\textbf{S-કર્વ રિસ્પોન્સ:}

\begin{verbatim}
    \^{}
    |              /
    |             /
    |            /
 0V +{-{-}{-}{-}{-}{-}{-}{-}{-}{-}{-}{-}}
    |          /
    |         /
    |        /
    +{-{-}{-}{-}{-}{-}{-}{-}{-}{-}{-}{-}{-}{-}{-}{-}{-}}
       fc{-Δf  fc  fc+Δf}
\end{verbatim}

\end{solutionbox}
\begin{mnemonicbox}
``PSDO'' - Phase shift Demodulates, Signal frequency
determines Output

\end{mnemonicbox}
\subsection*{પ્રશ્ન 4(a) [3
ગુણ]}\label{q4a}

\textbf{એનાલોગ અને ડિજિટલ કોમ્યુનિકેશન ટેક્નિક્સ સરખાવો.}

\begin{solutionbox}

\textbf{એનાલોગ vs.~ડિજિટલ કોમ્યુનિકેશનની તુલના:}

{\def\LTcaptype{none} % do not increment counter
\begin{longtable}[]{@{}
  >{\raggedright\arraybackslash}p{(\linewidth - 4\tabcolsep) * \real{0.1930}}
  >{\raggedright\arraybackslash}p{(\linewidth - 4\tabcolsep) * \real{0.3860}}
  >{\raggedright\arraybackslash}p{(\linewidth - 4\tabcolsep) * \real{0.4211}}@{}}
\toprule\noalign{}
\begin{minipage}[b]{\linewidth}\raggedright
પેરામીટર
\end{minipage} & \begin{minipage}[b]{\linewidth}\raggedright
એનાલોગ કોમ્યુનિકેશન
\end{minipage} & \begin{minipage}[b]{\linewidth}\raggedright
ડિજિટલ કોમ્યુનિકેશન
\end{minipage} \\
\midrule\noalign{}
\endhead
\bottomrule\noalign{}
\endlastfoot
\textbf{સિગ્નલ} & કન્ટિન્યુઅસ વેવફોર્મ & ડિસ્ક્રીટ બાઈનરી વેલ્યુ \\
\textbf{બેન્ડવિડ્થ} & ઓછી બેન્ડવિડ્થની જરૂર & વધુ બેન્ડવિડ્થની જરૂર \\
\textbf{નોઈઝ ઇમ્યુનિટી} & ખરાબ, નોઈઝ એક્યુમ્યુલેટ થાય છે & ઉત્તમ, એરર કરેક્શન
શક્ય \\
\textbf{પાવર એફિશિયન્સી} & ઓછી કાર્યક્ષમ & વધુ કાર્યક્ષમ \\
\textbf{ક્વોલિટી} & અંતર સાથે ઘટે છે & SNR થ્રેશોલ્ડ સુધી ક્વોલિટી જાળવે છે \\
\textbf{મલ્ટિપ્લેક્સિંગ} & મુખ્યત્વે FDM વપરાય છે & મુખ્યત્વે TDM વપરાય છે \\
\textbf{સિસ્ટમ જટિલતા} & સરળ & વધુ જટિલ \\
\textbf{ખર્ચ} & ઓછો & વધુ પણ ઘટતો જાય છે \\
\textbf{ઉદાહરણો} & AM/FM રેડિયો, એનાલોગ TV & મોબાઈલ નેટવર્ક્સ, ડિજિટલ TV,
ઇન્ટરનેટ \\
\end{longtable}
}

\end{solutionbox}
\begin{mnemonicbox}
``BNPQ MCE'' - Bandwidth, Noise immunity, Power,
Quality, Multiplexing, Complexity, Efficiency

\end{mnemonicbox}
\subsection*{પ્રશ્ન 4(b) [4
ગુણ]}\label{q4b}

\textbf{એડેપ્ટિવ ડેલ્ટા મોડ્યુલેશન તેની એપ્લિકેશન સાથે સમજાવો.}

\begin{solutionbox}

\textbf{એડેપ્ટિવ ડેલ્ટા મોડ્યુલેશન (ADM):}

\textbf{વ્યાખ્યા:}

\begin{itemize}
\tightlist
\item
  ડેલ્ટા મોડ્યુલેશન (DM)નો સુધારેલો પ્રકાર
\item
  સિગ્નલ સ્લોપના આધારે વેરિએબલ સ્ટેપ સાઈઝ ઉપયોગ કરે છે
\end{itemize}

\textbf{કાર્ય સિદ્ધાંત:}

\begin{enumerate}
\tightlist
\item
  ઇનપુટ સિગ્નલને પ્રેડિક્ટેડ વેલ્યુ સાથે સરખાવે છે
\item
  તુલના પર આધારિત બાઈનરી 1 અથવા 0 આઉટપુટ કરે છે
\item
  સતત બિટ્સના આધારે સ્ટેપ સાઈઝ એડજસ્ટ કરે છે
\item
  ઝડપી ફેરફારો માટે સ્ટેપ સાઈઝ વધારે છે
\item
  ધીમા ફેરફારો માટે સ્ટેપ સાઈઝ ઘટાડે છે
\end{enumerate}

\textbf{ડેલ્ટા મોડ્યુલેશન કરતાં ફાયદાઓ:}

\begin{itemize}
\tightlist
\item
  સ્લોપ ઓવરલોડ ડિસ્ટોર્શન ઘટાડે છે
\item
  ગ્રેન્યુલર નોઈઝ ઘટાડે છે
\item
  વધુ સારો ડાયનેમિક રેન્જ
\item
  સમાન ક્વોલિટી માટે ઓછો બિટ રેટ
\end{itemize}

\textbf{આકૃતિ:}

\begin{verbatim}
                        +{-{-}{-}{-}{-}{-}{-}+}
                        |       |
                        | Step  |{{-}{-}+}
Input                   | Size  |   |
  +{-{-}{-}+    +{-}{-}{-}+        | Logic |   |}
  |   |{-{-}{-}|+/{-}|{-}{-}{-}{-}{-}{-}{-}|       |   |}
  +{-{-}{-}+    +{-}{-}{-}+        +{-}{-}{-}{-}{-}{-}{-}+   |}
    \^{        |              |       |}
    |        |              V       |
    |      +{-{-}{-}+         +{-}{-}{-}+      |}
    +{-{-}{-}{-}{-}{-}|   |{-}{-}{-}{-}{-}{-}{-}{-}| Δ |{-}{-}{-}{-}{-}{-}+}
           +{-{-}{-}+         +{-}{-}{-}+}
           Integrator
\end{verbatim}

\textbf{એપ્લિકેશન:}

\begin{itemize}
\tightlist
\item
  \textbf{સ્પીચ ટ્રાન્સમિશન:} ડિજિટલ નેટવર્ક પર વોઈસ
\item
  \textbf{ઓડિયો કમ્પ્રેશન:} મ્યુઝિક સ્ટોરેજ અને ટ્રાન્સમિશન
\item
  \textbf{ટેલિમેટ્રી સિસ્ટમ્સ:} રિમોટ ડેટા કલેક્શન
\item
  \textbf{મિલિટરી કોમ્યુનિકેશન:} સિક્યોર ટ્રાન્સમિશન
\end{itemize}

\end{solutionbox}
\begin{mnemonicbox}
``VSOG'' - Variable Step size Overcomes Granular
noise \& slope overload

\end{mnemonicbox}
\subsection*{પ્રશ્ન 4(c) [7
ગુણ]}\label{q4c}

\textbf{PCM system નો બ્લોક ડાયગ્રામ દોરો અને સમજાવો.}

\begin{solutionbox}

\textbf{પલ્સ કોડ મોડ્યુલેશન (PCM) સિસ્ટમ:}

\textbf{બ્લોક ડાયગ્રામ:}

\begin{verbatim}
                  +{-{-}{-}{-}{-}{-}{-}+    +{-}{-}{-}{-}{-}{-}{-}{-}{-}{-}+    +{-}{-}{-}{-}{-}{-}{-}{-}{-}+    +{-}{-}{-}{-}{-}{-}{-}{-}+}
                  |       |    |          |    |         |    |        |
Input signal {-{-}{-}{-}|Sample |{-}{-}{-}|Quantizer |{-}{-}{-}|Encoder  |{-}{-}{-}|Channel |}
                  |\& Hold |    |          |    |         |    |        |
                  +{-{-}{-}{-}{-}{-}{-}+    +{-}{-}{-}{-}{-}{-}{-}{-}{-}{-}+    +{-}{-}{-}{-}{-}{-}{-}{-}{-}+    +{-}{-}{-}{-}{-}{-}{-}{-}+}
                                                                  |
                                                                  V
                  +{-{-}{-}{-}{-}{-}{-}{-}+    +{-}{-}{-}{-}{-}{-}{-}{-}{-}+    +{-}{-}{-}{-}{-}{-}{-}{-}{-}+    +{-}{-}{-}{-}{-}{-}{-}{-}+}
                  |        |    |         |    |         |    |        |
Output signal {{-}{-}{-}|Low Pass|{-}{-}{-}| DAC     |{-}{-}{-}|Decoder  |{-}{-}{-}| Buffer |}
                  |Filter  |    |         |    |         |    |        |
                  +{-{-}{-}{-}{-}{-}{-}{-}+    +{-}{-}{-}{-}{-}{-}{-}{-}{-}+    +{-}{-}{-}{-}{-}{-}{-}{-}{-}+    +{-}{-}{-}{-}{-}{-}{-}{-}+}
\end{verbatim}

\textbf{ટ્રાન્સમીટર કોમ્પોનેન્ટ્સ:}

\begin{enumerate}
\tightlist
\item
  \textbf{સેમ્પલ \& હોલ્ડ:}

  \begin{itemize}
  \tightlist
  \item
    નિયમિત અંતરાલે એનાલોગ સિગ્નલને સેમ્પલ કરે છે
  \item
    નાયક્વિસ્ટ રેટ (fs \geq 2fmax)
  \item
    આગલા સેમ્પલ સુધી વેલ્યુ હોલ્ડ કરે છે
  \end{itemize}
\item
  \textbf{ક્વોન્ટાઇઝર:}

  \begin{itemize}
  \tightlist
  \item
    એમ્પ્લિટ્યુડ રેન્જને ડિસ્ક્રીટ લેવલમાં વિભાજિત કરે છે
  \item
    દરેક સેમ્પલને નજીકની લેવલ સાથે મેપ કરે છે
  \item
    ક્વોન્ટાઇઝેશન એરર દાખલ કરે છે
  \end{itemize}
\item
  \textbf{એન્કોડર:}

  \begin{itemize}
  \tightlist
  \item
    ક્વોન્ટાઇઝ્ડ લેવલ્સને બાઇનરી કોડમાં કન્વર્ટ કરે છે
  \item
    n-બીટ એન્કોડર 2\^{}n ક્વોન્ટાઇઝેશન લેવલ આપે છે
  \item
    સામાન્ય ફોર્મેટ: 8-બીટ, 16-બીટ
  \end{itemize}
\end{enumerate}

\textbf{રિસીવર કોમ્પોનેન્ટ્સ:}

\begin{enumerate}
\tightlist
\item
  \textbf{ડિકોડર:}

  \begin{itemize}
  \tightlist
  \item
    બાઇનરીને ક્વોન્ટાઇઝ્ડ લેવલમાં કન્વર્ટ કરે છે
  \item
    એન્કોડર ઓપરેશનને રિવર્સ કરે છે
  \end{itemize}
\item
  \textbf{ડિજિટલ-ટુ-એનાલોગ કન્વર્ટર (DAC):}

  \begin{itemize}
  \tightlist
  \item
    ડિસ્ક્રીટ લેવલને એનાલોગ વેલ્યુમાં કન્વર્ટ કરે છે
  \item
    સિગ્નલનું સ્ટેરકેસ એપ્રોક્સિમેશન ઉત્પન્ન કરે છે
  \end{itemize}
\item
  \textbf{લો-પાસ ફિલ્ટર:}

  \begin{itemize}
  \tightlist
  \item
    સ્ટેરકેસ આઉટપુટને સ્મૂધ કરે છે
  \item
    હાઈ-ફ્રિક્વન્સી કોમ્પોનેન્ટ્સ દૂર કરે છે
  \item
    ઓરિજિનલ વેવફોર્મ રિકન્સ્ટ્રક્ટ કરે છે
  \end{itemize}
\end{enumerate}

\textbf{મુખ્ય લક્ષણો:}

\begin{itemize}
\tightlist
\item
  સેમ્પલિંગ રેટ: સામાન્ય રીતે 8 kHz (વોઇસ), 44.1 kHz (CD ઓડિયો)
\item
  રેઝોલ્યુશન: 8-બીટ (256 લેવલ) થી 24-બીટ (16.8M લેવલ)
\item
  બિટ રેટ = સેમ્પલિંગ રેટ \times દરેક સેમ્પલમાં બિટ્સ
\end{itemize}

\end{solutionbox}
\begin{mnemonicbox}
``SQEC-DFL'' - Sample, Quantize, Encode, Channel -
Decode, Filter, Listen

\end{mnemonicbox}
\subsection*{પ્રશ્ન 4(a) OR [3
ગુણ]}\label{q4a}

\textbf{ક્વોન્ટાઇઝેશન રીત અને તેની ઉપયોગિતા સમજાવો.}

\begin{solutionbox}

\textbf{ક્વોન્ટાઇઝેશન પ્રક્રિયા અને તેની આવશ્યકતા:}

\textbf{વ્યાખ્યા:}

\begin{itemize}
\tightlist
\item
  સતત એમ્પ્લિટ્યુડ મૂલ્યોને ડિસ્ક્રીટ લેવલમાં મેપિંગ કરવાની પ્રક્રિયા
\item
  સેમ્પલિંગ પછી એનાલોગ-ટુ-ડિજિટલ કન્વર્ઝનમાં બીજું પગલું
\end{itemize}

\textbf{પ્રક્રિયા:}

\begin{enumerate}
\tightlist
\item
  એમ્પ્લિટ્યુડ રેન્જને મર્યાદિત સંખ્યાના લેવલમાં વિભાજિત કરવું
\item
  દરેક સેમ્પલને નજીકની ક્વોન્ટાઇઝેશન લેવલ સોંપવી
\item
  દરેક લેવલને બાઇનરી કોડથી રજૂ કરવી
\item
  ક્વોન્ટાઇઝેશન લેવલ = 2\^{}n (n = બિટની સંખ્યા)
\end{enumerate}

\textbf{પ્રકારો:}

\begin{itemize}
\tightlist
\item
  \textbf{યુનિફોર્મ ક્વોન્ટાઇઝેશન:} સમગ્ર રેન્જમાં સમાન સ્ટેપ સાઇઝ
\item
  \textbf{નોન-યુનિફોર્મ ક્વોન્ટાઇઝેશન:} વેરિએબલ સ્ટેપ સાઇઝ (નીચા એમ્પ્લિટ્યુડ માટે
  નાના)
\item
  \textbf{મિડ-ટ્રેડ ક્વોન્ટાઇઝેશન:} શૂન્ય એક માન્ય લેવલ છે
\item
  \textbf{મિડ-રાઇઝ ક્વોન્ટાઇઝેશન:} શૂન્ય લેવલ વચ્ચે પડે છે
\end{itemize}

\textbf{આવશ્યકતા:}

\begin{itemize}
\tightlist
\item
  \textbf{ડિજિટલ રજૂઆત:} બાઇનરી ફોર્મેટમાં કન્વર્ઝન શક્ય બનાવે છે
\item
  \textbf{સ્ટોરેજ કાર્યક્ષમતા:} એનાલોગ સિગ્નલ્સના મર્યાદિત સ્ટોરેજની મંજૂરી આપે છે
\item
  \textbf{પ્રોસેસિંગ ક્ષમતા:} ડિજિટલ સિગ્નલ પ્રોસેસિંગ શક્ય બનાવે છે
\item
  \textbf{ટ્રાન્સમિશન ફાયદા:} એરર કરેક્શન અને એન્ક્રિપ્શનની સુવિધા આપે છે
\end{itemize}

\textbf{ક્વોન્ટાઇઝેશન એરર:}

\begin{itemize}
\tightlist
\item
  એક્ચ્યુઅલ અને ક્વોન્ટાઇઝ્ડ વેલ્યુ વચ્ચેનો તફાવત
\item
મહત્તમ એરર = \pmQ/2 (જ્યાં

Q = સ્ટેપ સાઇઝ)

\item
  સિગ્નલ-ટુ-ક્વોન્ટાઇઝેશન-નોઇઝ રેશિયો: SQNR = 6.02n + 1.76 dB
\end{itemize}

\textbf{આકૃતિ:}

\begin{verbatim}
  \^{}
  |                    Quantized
  |   Original         Output
  |    Signal          /|
  |      /{           / |}
  |     /  {         /  |}
  |    /    {       /   |}
  |   /      {     /    |}
  |  /        {   /     |}
  | /          { /      |}
  +{-{-}{-}{-}{-}{-}{-}{-}{-}{-}{-}{-}{-}{-}{-}{-}{-}{-}{-}{-}{-}{-}{-}{-}{-}{-}{-}}
                Time
\end{verbatim}

\end{solutionbox}
\begin{mnemonicbox}
``DEBS'' - Digitization Enables Binary Storage

\end{mnemonicbox}
\subsection*{પ્રશ્ન 4(b) OR [4
ગુણ]}\label{q4b}

\textbf{PCM રીસીવર સમજાવો.}

\begin{solutionbox}

\textbf{PCM રીસીવર:}

\textbf{બ્લોક ડાયગ્રામ:}

\begin{verbatim}
                  +{-{-}{-}{-}{-}{-}{-}{-}+    +{-}{-}{-}{-}{-}{-}{-}{-}{-}+    +{-}{-}{-}{-}{-}{-}{-}{-}{-}+    +{-}{-}{-}{-}{-}{-}{-}{-}+}
                  |        |    |         |    |         |    |        |
Digital PCM  {-{-}{-}{-}| Buffer |{-}{-}{-}| Decoder |{-}{-}{-}|   DAC   |{-}{-}{-}|Low Pass|{-}{-}{-} Output Signal}
  Input           |        |    |         |    |         |    | Filter |
                  +{-{-}{-}{-}{-}{-}{-}{-}+    +{-}{-}{-}{-}{-}{-}{-}{-}{-}+    +{-}{-}{-}{-}{-}{-}{-}{-}{-}+    +{-}{-}{-}{-}{-}{-}{-}{-}+}
\end{verbatim}

\textbf{કોમ્પોનેન્ટ્સ અને તેમનાં કાર્યો:}

\begin{enumerate}
\tightlist
\item
  \textbf{બફર:}

  \begin{itemize}
  \tightlist
  \item
    મળેલ PCM ડેટાને અસ્થાયી રીતે સ્ટોર કરે છે
  \item
    ટાઇમિંગ વેરિએશન્સ માટે કોમ્પેન્સેટ કરે છે
  \item
    જિટર સામે રક્ષણ પૂરું પાડે છે
  \end{itemize}
\item
  \textbf{ડિકોડર:}

  \begin{itemize}
  \tightlist
  \item
    બાઇનરી કોડને ક્વોન્ટાઇઝ્ડ એમ્પ્લિટ્યુડ લેવલમાં કન્વર્ટ કરે છે
  \item
    ટ્રાન્સમિશન એરર્સને ડિટેક્ટ અને કરેક્ટ કરે છે (જો એરર કોડિંગ વપરાયું હોય તો)
  \item
    ડિસ્ક્રીટ એમ્પ્લિટ્યુડ વેલ્યુ આઉટપુટ કરે છે
  \end{itemize}
\item
  \textbf{ડિજિટલ-ટુ-એનાલોગ કન્વર્ટર (DAC):}

  \begin{itemize}
  \tightlist
  \item
    ડિજિટલ વેલ્યુને એનાલોગ વોલ્ટેજ લેવલમાં કન્વર્ટ કરે છે
  \item
    ઓરિજિનલ સિગ્નલનું સ્ટેરકેસ એપ્રોક્સિમેશન બનાવે છે
  \item
    રેઝોલ્યુશન બિટ ડેપ્થ (2\^{}n લેવલ) દ્વારા નિર્ધારિત થાય છે
  \end{itemize}
\item
  \textbf{લો-પાસ ફિલ્ટર:}

  \begin{itemize}
  \tightlist
  \item
    સ્ટેરકેસ વેવફોર્મને સ્મૂધ કરે છે
  \item
    હાઈ-ફ્રિક્વન્સી કોમ્પોનેન્ટ્સ દૂર કરે છે
  \item
    સતત એનાલોગ સિગ્નલ રિકન્સ્ટ્રક્ટ કરે છે
  \end{itemize}
\end{enumerate}

\textbf{PCM રીસીવરમાં વેવફોર્મ્સ:}

\begin{verbatim}
Digital Input      Decoded Values       DAC Output          Final Output
 1001              {-{-}{-}{-}                  \_                    /}
 0110              {-  {-}                \_| |\_                 /  }
 1010             {-{-} {-}              \_|   |\_              /    }
 0101              {- {-} {-}             \_|     |\_             /      }
\end{verbatim}

\textbf{પરફોર્મન્સ ફેક્ટર્સ:}

\begin{itemize}
\tightlist
\item
  \textbf{SNR:} ક્વોન્ટાઇઝેશન બિટ્સ દ્વારા નિર્ધારિત (6.02n + 1.76 dB)
\item
  \textbf{બેન્ડવિડ્થ:} સેમ્પલિંગ રેટ અને ફિલ્ટર લક્ષણો પર આધારિત
\item
  \textbf{ડિસ્ટોર્શન:} ક્વોન્ટાઇઝેશન એરર સાથે સંબંધિત
\end{itemize}

\end{solutionbox}
\begin{mnemonicbox}
``BDFL'' - Buffer stores, Decoder converts, Filter
smooths, Listen to output

\end{mnemonicbox}
\subsection*{પ્રશ્ન 4(c) OR [7
ગુણ]}\label{q4c}

\textbf{સેમ્પલિંગ શું છે? સેમ્પલિંગના પ્રકારોને ટુંકમાં સમજાવો.}

\begin{solutionbox}

\textbf{સેમ્પલિંગ:}

\textbf{વ્યાખ્યા:} સેમ્પલિંગ એ કન્ટિન્યુઅસ-ટાઇમ સિગ્નલને નિયમિત સમય અંતરાલે માપ
(સેમ્પલ) લઈને ડિસ્ક્રીટ-ટાઇમ સિગ્નલમાં કન્વર્ટ કરવાની પ્રક્રિયા છે.

\textbf{ગાણિતિક અભિવ્યક્તિ:} x[n] = x(nTs), જ્યાં n = 0, 1, 2\ldots{}

\begin{itemize}
\tightlist
\item
  x[n] એ ડિસ્ક્રીટ-ટાઇમ સેમ્પલ છે
\item
  x(t) એ કન્ટિન્યુઅસ-ટાઇમ સિગ્નલ છે
\item
  Ts એ સેમ્પલિંગ પીરિયડ (1/fs) છે
\end{itemize}

\textbf{નાયક્વિસ્ટ થિયરમ:}

\begin{itemize}
\tightlist
\item
  સેમ્પલિંગ આવૃત્તિ (fs) સિગ્નલમાં ઉચ્ચતમ આવૃત્તિ ઘટક (fmax)ના ઓછામાં ઓછા બમણી હોવી
  જોઈએ
\item
  fs \geq 2fmax
\item
  એલિયાસિંગ (સ્પેક્ટ્રમના ઓવરલેપ કારણે ડિસ્ટોર્શન) અટકાવે છે
\end{itemize}

\textbf{સેમ્પલિંગના પ્રકારો:}

{\def\LTcaptype{none} % do not increment counter
\begin{longtable}[]{@{}
  >{\raggedright\arraybackslash}p{(\linewidth - 4\tabcolsep) * \real{0.1667}}
  >{\raggedright\arraybackslash}p{(\linewidth - 4\tabcolsep) * \real{0.3611}}
  >{\raggedright\arraybackslash}p{(\linewidth - 4\tabcolsep) * \real{0.4722}}@{}}
\toprule\noalign{}
\begin{minipage}[b]{\linewidth}\raggedright
પ્રકાર
\end{minipage} & \begin{minipage}[b]{\linewidth}\raggedright
વર્ણન
\end{minipage} & \begin{minipage}[b]{\linewidth}\raggedright
લક્ષણો
\end{minipage} \\
\midrule\noalign{}
\endhead
\bottomrule\noalign{}
\endlastfoot
\textbf{આદર્શ સેમ્પલિંગ} & નિયમિત અંતરાલે તાત્કાલિક સેમ્પલ & - થિયોરેટિકલ કોન્સેપ્ટ-
ઈમ્પલ્સ ટ્રેન દ્વારા રજૂ થયેલ- અનંત બેન્ડવિડ્થની જરૂર પડે છે \\
\textbf{નેચરલ સેમ્પલિંગ} & સિગ્નલને મર્યાદિત પહોળાઈના પલ્સ ટ્રેન સાથે ગુણાકાર & -
સેમ્પલ સિગ્નલ જેવી જ આકૃતિ ધરાવે છે- પહોળાઈ સેમ્પલિંગ પલ્સ દ્વારા નિર્ધારિત છે- એનાલોગ
સિસ્ટમમાં વપરાય છે \\
\textbf{ફ્લેટ-ટોપ સેમ્પલિંગ} & સેમ્પલ-એન્ડ-હોલ્ડ ટેકનિક & - આગલા સેમ્પલ સુધી સેમ્પલ કરેલ
મૂલ્ય હોલ્ડ કરે છે- સ્ટેરકેસ એપ્રોક્સિમેશન બનાવે છે- પ્રેક્ટિકલ સિસ્ટમમાં સામાન્ય છે \\
\end{longtable}
}

\textbf{સેમ્પલિંગ રેટ્સ:}

\begin{itemize}
\tightlist
\item
  \textbf{અન્ડર-સેમ્પલિંગ:} fs \textless{} 2fmax (એલિયાસિંગ થાય છે)
\item
  \textbf{ક્રિટિકલ સેમ્પલિંગ:} fs = 2fmax (જરૂરી ન્યૂનતમ રેટ)
\item
  \textbf{ઓવર-સેમ્પલિંગ:} fs \textgreater{} 2fmax (રિકન્સ્ટ્રક્શન ક્વોલિટી સુધારે
  છે)
\end{itemize}

\textbf{આકૃતિ:}

\begin{verbatim}
Original Signal:     /{///////}

Ideal Sampling:      |  |  |  |  |  |

Natural Sampling:    ▓  ▓  ▓  ▓  ▓  ▓

Flat{-top Sampling:   ▔▔  ▔▔  ▔▔  ▔▔  ▔▔}
\end{verbatim}

\end{solutionbox}
\begin{mnemonicbox}
``INF'' - Ideal (impulses), Natural (pulse-shaped),
Flat-top (staircase)

\end{mnemonicbox}
\subsection*{પ્રશ્ન 5(a) [3
ગુણ]}\label{q5a}

\textbf{મલ્ટીપ્લેક્સિંગની આવશ્યક્તાઓની યાદી બનાવો.}

\begin{solutionbox}

\textbf{મલ્ટીપ્લેક્સિંગની આવશ્યકતા:}

{\def\LTcaptype{none} % do not increment counter
\begin{longtable}[]{@{}ll@{}}
\toprule\noalign{}
આવશ્યકતા & વર્ણન \\
\midrule\noalign{}
\endhead
\bottomrule\noalign{}
\endlastfoot
\textbf{બેન્ડવિડ્થ ઉપયોગ} & ઉપલબ્ધ ટ્રાન્સમિશન બેન્ડવિડ્થનો કાર્યક્ષમ ઉપયોગ \\
\textbf{ખર્ચ ઘટાડો} & મોંઘા ટ્રાન્સમિશન માધ્યમને અનેક વપરાશકર્તાઓમાં શેર કરે છે \\
\textbf{ઇન્ફ્રાસ્ટ્રક્ચર ઓપ્ટિમાઇઝેશન} & ભૌતિક કનેક્શન અને હાર્ડવેર જરૂરિયાતો ઘટાડે
છે \\
\textbf{સ્પેક્ટ્રમ કાર્યક્ષમતા} & મર્યાદિત આવૃત્તિ સ્પેક્ટ્રમનો મહત્તમ ઉપયોગ \\
\textbf{નેટવર્ક ક્ષમતા} & સિંગલ માધ્યમ પર ચેનલ/વપરાશકર્તાઓની સંખ્યામાં વધારો \\
\textbf{લવચીકતા} & માંગના આધારે સંસાધનોની ગતિશીલ ફાળવણીની મંજૂરી આપે છે \\
\end{longtable}
}

\end{solutionbox}
\begin{mnemonicbox}
``BCSINF'' - Bandwidth, Cost, Spectrum,
Infrastructure, Network capacity, Flexibility

\end{mnemonicbox}
\subsection*{પ્રશ્ન 5(b) [4
ગુણ]}\label{q5b}

\textbf{DPCM નું કાર્ય સમજાવો.}

\begin{solutionbox}

\textbf{ડિફરેન્શિયલ પલ્સ કોડ મોડ્યુલેશન (DPCM):}

\textbf{વ્યાખ્યા:}

\begin{itemize}
\tightlist
\item
  PCMનો એન્હાન્સ્ડ વર્ઝન જે વર્તમાન અને અનુમાનિત સેમ્પલ વચ્ચેના તફાવતને એન્કોડ કરે છે
\item
  બિટ રેટ ઘટાડવા માટે આસપાસના સેમ્પલ વચ્ચે સંબંધનો ઉપયોગ કરે છે
\end{itemize}

\textbf{બ્લોક ડાયગ્રામ:}

\begin{verbatim}
                  +{-{-}{-}{-}{-}{-}+     +{-}{-}{-}{-}{-}{-}{-}{-}{-}{-}+    +{-}{-}{-}{-}{-}{-}{-}{-}{-}+}
                  |      |     |          |    |         |
Input signal {-{-}{-}{-}| ADC  |{-}{-}+{-}|Quantizer |{-}{-}{-}|Encoder  |{-}{-}{-} DPCM Output}
                  |      |  |  |          |    |         |
                  +{-{-}{-}{-}{-}{-}+  |  +{-}{-}{-}{-}{-}{-}{-}{-}{-}{-}+    +{-}{-}{-}{-}{-}{-}{-}{-}{-}+}
                            |        \^{}
                            |        |
                            v        |
                        +{-{-}{-}{-}{-}{-}{-}{-}{-}+  |}
                        |Predictor|{-{-}+}
                        +{-{-}{-}{-}{-}{-}{-}{-}{-}+}
\end{verbatim}

\textbf{કાર્ય સિદ્ધાંત:}

\begin{enumerate}
\tightlist
\item
  અગાઉના સેમ્પલ(સ) પર આધારિત વર્તમાન સેમ્પલની ધારણા કરવામાં આવે છે
\item
  માત્ર વાસ્તવિક અને અનુમાનિત મૂલ્ય વચ્ચેનો તફાવત (એરર) એન્કોડેડ થાય છે
\item
  સંપૂર્ણ એમ્પ્લિટ્યુડ કરતાં નાનો તફાવત ઓછા બિટ્સની જરૂર પડે છે
\item
  પ્રેડિક્ટર અગાઉના રિકન્સ્ટ્રક્ટેડ વેલ્યુનો ઉપયોગ ધારણા માટે કરે છે
\end{enumerate}

\textbf{ફાયદાઓ:}

\begin{itemize}
\tightlist
\item
  \textbf{ઘટાડેલ બિટ રેટ:} સામાન્ય રીતે PCM કરતાં 25-50\% ઓછો
\item
  \textbf{બેટર SNR:} PCM જેટલા જ બિટ રેટ માટે
\item
  \textbf{સંબંધ ઉપયોગ:} સિગ્નલ રિડન્ડન્સીનો લાભ લે છે
\end{itemize}

\textbf{મર્યાદાઓ:}

\begin{itemize}
\tightlist
\item
  \textbf{એરર પ્રોપેગેશન:} એરર પછીના સેમ્પલને અસર કરે છે
\item
  \textbf{જટિલતા:} સરળ PCM કરતાં વધુ જટિલ
\item
  \textbf{સિગ્નલ ડિપેન્ડન્સી:} પ્રદર્શન સિગ્નલ લક્ષણો સાથે બદલાય છે
\end{itemize}

\end{solutionbox}
\begin{mnemonicbox}
``PDQE'' - Predict sample, Difference calculated,
Quantize error, Encode result

\end{mnemonicbox}
\subsection*{પ્રશ્ન 5(c) [7
ગુણ]}\label{q5c}

\textbf{બાઈનરી ડેટા 1011001 નીચે પ્રમાણેની લાઈન કોડિંગ ટેકનીકથી ટ્રાન્સમીટ થાય છે
(i) યુનિપોલાર RZ અને NRZ (ii) પોલાર RZ અને NRZ (iii) AMI (iv) Manchester.
બધા માટે વેવ ફોર્મ દોરો.}

\begin{solutionbox}

\textbf{બાઈનરી ડેટા 1011001 માટે લાઈન કોડિંગ:}

\textbf{વેવફોર્મ્સ:}

\begin{verbatim}
Binary Data:   1   0   1   1   0   0   1
              \_   \_   \_   \_   \_   \_   \_

1. Unipolar NRZ:
              ▔▔▔   ▔▔▔▔▔▔   ▔▔▔
              \_\_\_ ▔▔▔ \_\_\_\_\_\_\_ ▔▔▔

2. Unipolar RZ:
              ▔ \_ ▔ \_ ▔ ▔ \_ \_ \_ ▔ \_
              \_ \_ \_ \_ \_ \_ \_ \_ \_ \_ \_ \_

3. Polar NRZ:
              ▔▔▔   ▔▔▔▔▔▔   ▔▔▔
              \_\_\_ ▔▔▔ \_\_\_\_\_\_\_ ▔▔▔

4. Polar RZ:
              ▔ \_ \_ \_ ▔ ▔ \_ \_ \_ ▔ \_
              \_ ▔ \_ \_ \_ \_ ▔ ▔ \_ \_ \_

5. AMI:
              ▔ \_   ▔ \_ \_ \_ ▔ \_
              \_ \_ ▔ \_ \_ \_ \_ \_ \_ \_ \_

6. Manchester:
              ▔▁ ▁▔ ▔▁ ▔▁ ▁▔ ▁▔ ▔▁
              \_ \_ \_ \_ \_ \_ \_ \_ \_ \_ \_ \_
\end{verbatim}

\textbf{દરેક કોડિંગની લાક્ષણિકતાઓ:}

{\def\LTcaptype{none} % do not increment counter
\begin{longtable}[]{@{}
  >{\raggedright\arraybackslash}p{(\linewidth - 6\tabcolsep) * \real{0.3103}}
  >{\raggedright\arraybackslash}p{(\linewidth - 6\tabcolsep) * \real{0.2241}}
  >{\raggedright\arraybackslash}p{(\linewidth - 6\tabcolsep) * \real{0.2069}}
  >{\raggedright\arraybackslash}p{(\linewidth - 6\tabcolsep) * \real{0.2586}}@{}}
\toprule\noalign{}
\begin{minipage}[b]{\linewidth}\raggedright
કોડિંગ ટેકનિક
\end{minipage} & \begin{minipage}[b]{\linewidth}\raggedright
વર્ણન
\end{minipage} & \begin{minipage}[b]{\linewidth}\raggedright
ફાયદાઓ
\end{minipage} & \begin{minipage}[b]{\linewidth}\raggedright
ગેરફાયદાઓ
\end{minipage} \\
\midrule\noalign{}
\endhead
\bottomrule\noalign{}
\endlastfoot
\textbf{Unipolar NRZ} & 1 = હાઈ વોલ્ટેજ0 = ઝીરો વોલ્ટેજઝીરોમાં રિટર્ન નથી &
સરળ ઇમ્પ્લિમેન્ટેશન & DC કોમ્પોનેન્ટ, ક્લોક રિકવરી નહીં \\
\textbf{Unipolar RZ} & 1 = અર્ધા બિટ માટે હાઈ0 = ઝીરો વોલ્ટેજઝીરોમાં રિટર્ન &
સેલ્ફ-ક્લોકિંગ & વધુ બેન્ડવિડ્થની જરૂર \\
\textbf{Polar NRZ} & 1 = પોઝિટિવ વોલ્ટેજ0 = નેગેટિવ વોલ્ટેજઝીરોમાં રિટર્ન નથી &
DC કોમ્પોનેન્ટ નથી & ખરાબ ક્લોક રિકવરી \\
\textbf{Polar RZ} & 1 = અર્ધા બિટ માટે પોઝિટિવ0 = અર્ધા બિટ માટે નેગેટિવઝીરોમાં
રિટર્ન & સેલ્ફ-ક્લોકિંગ, DC કોમ્પોનેન્ટ નથી & વધુ બેન્ડવિડ્થની જરૂર \\
\textbf{AMI} & 1 = વૈકલ્પિક +/- વોલ્ટેજ0 = ઝીરો વોલ્ટેજ & DC કોમ્પોનેન્ટ નથી, એરર
ડિટેક્શન & ઝીરોની લાંબી સ્ટ્રિંગ સમસ્યારૂપ \\
\textbf{Manchester} & 1 = ટ્રાન્ઝિશન લો થી હાઈ0 = ટ્રાન્ઝિશન હાઈ થી લો &
સેલ્ફ-ક્લોકિંગ, DC કોમ્પોનેન્ટ નથી & બમણી બેન્ડવિડ્થની જરૂર \\
\end{longtable}
}

\end{solutionbox}
\begin{mnemonicbox}
``UPRMA'' - Unipolar, Polar, Return-to-zero,
Manchester, AMI line coding techniques

\end{mnemonicbox}
\subsection*{પ્રશ્ન 5(a) OR [3
ગુણ]}\label{q5a}

\textbf{પોલાર RZ અને NRZ ફોર્મેટ સમજાવો.}

\begin{solutionbox}

\textbf{પોલાર RZ અને NRZ લાઈન કોડિંગ:}

\textbf{પોલાર NRZ (નોન-રિટર્ન ટુ ઝીરો):}

\begin{itemize}
\tightlist
\item
  બાઈનરી 1: સંપૂર્ણ બિટ સમયગાળા માટે પોઝિટિવ વોલ્ટેજ (+V)
\item
  બાઈનરી 0: સંપૂર્ણ બિટ સમયગાળા માટે નેગેટિવ વોલ્ટેજ (-V)
\item
  સિગ્નલ સમગ્ર બિટ પીરિયડ દરમિયાન લેવલ પર રહે છે
\item
  સમાન ક્રમિક બિટ્સ વચ્ચે ઝીરો તરફ કોઈ ટ્રાન્ઝિશન નથી
\end{itemize}

\textbf{પોલાર NRZની લાક્ષણિકતાઓ:}

\begin{itemize}
\tightlist
\item
  \textbf{બેન્ડવિડ્થ કાર્યક્ષમતા:} ન્યૂનતમ બેન્ડવિડ્થની જરૂર પડે છે
\item
  \textbf{DC કોમ્પોનેન્ટ:} સમાન 1 અને 0 માટે શૂન્ય સરેરાશ
\item
  \textbf{ક્લોક રિકવરી:} સમાન બિટની લાંબી શ્રેણી માટે ખરાબ
\item
  \textbf{એરર ડિટેક્શન:} કોઈ અંતર્ગત ક્ષમતા નથી
\end{itemize}

\textbf{પોલાર RZ (રિટર્ન ટુ ઝીરો):}

\begin{itemize}
\tightlist
\item
  બાઈનરી 1: અર્ધા બિટ માટે પોઝિટિવ વોલ્ટેજ (+V), બાકીના માટે ઝીરો
\item
  બાઈનરી 0: અર્ધા બિટ માટે નેગેટિવ વોલ્ટેજ (-V), બાકીના માટે ઝીરો
\item
  દરેક બિટ પીરિયડ દરમિયાન સિગ્નલ ઝીરો પર પાછો ફરે છે
\end{itemize}

\textbf{પોલાર RZની લાક્ષણિકતાઓ:}

\begin{itemize}
\tightlist
\item
  \textbf{બેન્ડવિડ્થ:} NRZ કરતાં બમણી બેન્ડવિડ્થની જરૂર પડે છે
\item
  \textbf{સેલ્ફ-ક્લોકિંગ:} વધુ સારી ક્લોક રિકવરી
\item
  \textbf{પાવર જરૂરિયાત:} NRZ કરતાં વધારે
\item
  \textbf{એરર ડિટેક્શન:} કોઈ અંતર્ગત ક્ષમતા નથી
\end{itemize}

\textbf{વેવફોર્મ તુલના:}

\begin{verbatim}
Binary Data:   1   0   1   1   0   0   1
              \_   \_   \_   \_   \_   \_   \_

Polar NRZ:    ▔▔▔   ▔▔▔▔▔▔   ▔▔▔
              \_\_\_ ▔▔▔ \_\_\_\_\_\_\_ ▔▔▔

Polar RZ:     ▔ \_ \_ \_ ▔ ▔ \_ \_ \_ ▔ \_
              \_ ▔ \_ \_ \_ \_ ▔ ▔ \_ \_ \_
\end{verbatim}

\end{solutionbox}
\begin{mnemonicbox}
``HZRT'' - Half bit active + Zero Return in RZ, full
Time in NRZ

\end{mnemonicbox}
\subsection*{પ્રશ્ન 5(b) OR [4
ગુણ]}\label{q5b}

\textbf{ડેલ્ટા મોડ્યુલેશન ટૂંકમાં સમજાવો.}

\begin{solutionbox}

\textbf{ડેલ્ટા મોડ્યુલેશન (DM):}

\textbf{વ્યાખ્યા:}

\begin{itemize}
\tightlist
\item
  ડિફરેન્શિયલ એન્કોડિંગનો સૌથી સરળ સ્વરૂપ
\item
  માત્ર વર્તમાન અને અગાઉના સેમ્પલ વચ્ચેના તફાવતના ચિહ્નને એન્કોડ કરે છે
\item
  ટ્રાન્સમિશન માટે પ્રતિ સેમ્પલ એક બિટ (1 અથવા 0)
\end{itemize}

\textbf{બ્લોક ડાયગ્રામ:}

\begin{verbatim}
                    +{-{-}{-}{-}{-}+       Encoded}
 Input      +{-{-}{-}+   |     |      Bitstream}
 Signal {-{-}{-}|+/{-}|{-}{-}{-}  C  |{-}{-}{-}{-}{-}{-}{-}{-}{-}{-}}
            +{-{-}{-}+   |     |}
              \^{     +{-}{-}{-}{-}{-}+}
              |        |
              |        v
            +{-{-}{-}+    +{-}{-}{-}+}
            |   |{{-}{-}{-}|+/{-}|}
            +{-{-}{-}+    +{-}{-}{-}+}
          Integrator   Step Size
\end{verbatim}

\textbf{કાર્ય સિદ્ધાંત:}

\begin{enumerate}
\tightlist
\item
  ઇનપુટ સિગ્નલને પ્રેડિક્ટેડ વેલ્યુ (ઇન્ટિગ્રેટરથી) સાથે સરખાવે છે
\item
  જો ઇનપુટ \textgreater{} પ્રેડિક્ટેડ: આઉટપુટ = 1, પ્રેડિક્ટેડ વેલ્યુ વધારે છે
\item
  જો ઇનપુટ \textless{} પ્રેડિક્ટેડ: આઉટપુટ = 0, પ્રેડિક્ટેડ વેલ્યુ ઘટાડે છે
\item
  સ્ટેપ સાઈઝ નક્કી કરે છે કે પ્રેડિક્ટેડ વેલ્યુ કેટલું બદલાય છે
\end{enumerate}

\textbf{ફાયદાઓ:}

\begin{itemize}
\tightlist
\item
  \textbf{સરળ ઇમ્પ્લિમેન્ટેશન:} મિનિમલ હાર્ડવેર
\item
  \textbf{ઓછો બિટ રેટ:} પ્રતિ સેમ્પલ 1 બિટ
\item
  \textbf{મજબૂત:} પ્રમાણમાં ચેનલ નોઈઝથી અસરમુક્ત
\end{itemize}

\textbf{મર્યાદાઓ:}

\begin{itemize}
\tightlist
\item
  \textbf{સ્લોપ ઓવરલોડ:} ઝડપી સિગ્નલ ફેરફારોને ટ્રેક કરી શકતું નથી
\item
  \textbf{ગ્રેન્યુલર નોઈઝ:} સ્થિર સિગ્નલની આજુબાજુ ઓસિલેશન
\item
  \textbf{મર્યાદિત રેઝોલ્યુશન:} ક્વોલિટી સ્ટેપ સાઈઝ અને સેમ્પલિંગ રેટ પર આધાર રાખે છે
\end{itemize}

\textbf{વેવફોર્મ્સ:}

\begin{verbatim}
Original:      /{///}

Reconstructed: /{///}
               (Staircase approximation)

Binary output: 1101001011
\end{verbatim}

\end{solutionbox}
\begin{mnemonicbox}
``1BSG'' - 1 Bit per Sample, Slope overload and
Granular noise limitations

\end{mnemonicbox}
\subsection*{પ્રશ્ન 5(c) OR [7
ગુણ]}\label{q5c}

\textbf{PCM-TDM સિસ્ટમ સમજાવો.}

\begin{solutionbox}

\textbf{PCM-TDM સિસ્ટમ:}

\textbf{વ્યાખ્યા:}

\begin{itemize}
\tightlist
\item
  પલ્સ કોડ મોડ્યુલેશન (PCM) અને ટાઈમ ડિવિઝન મલ્ટિપ્લેક્સિંગ (TDM)નો સંયુક્ત સિસ્ટમ
\item
  મલ્ટિપલ એનાલોગ ચેનલ ડિજિટલ PCMમાં કન્વર્ટ થાય છે, પછી સમયમાં મલ્ટિપ્લેક્સ થાય છે
\end{itemize}

\textbf{બ્લોક ડાયગ્રામ:}

\begin{verbatim}
                 +{-{-}{-}{-}{-}{-}{-}+     +{-}{-}{-}{-}{-}{-}{-}{-}+     +{-}{-}{-}{-}{-}{-}{-}{-}{-}+}
 Channel 1 {-{-}{-}{-}{-}| PCM 1 |{-}{-}{-}{-}|        |     |         |}
                 +{-{-}{-}{-}{-}{-}{-}+     |        |     |         |}
                 +{-{-}{-}{-}{-}{-}{-}+     |        |     |         |     Multiplexed}
 Channel 2 {-{-}{-}{-}{-}| PCM 2 |{-}{-}{-}{-}|  Time  |{-}{-}{-}{-}|  Frame  |{-}{-}{-} PCM{-}TDM}
                 +{-{-}{-}{-}{-}{-}{-}+     |        |     | Format  |     Output}
                               |  MUX   |     |         |
                 +{-{-}{-}{-}{-}{-}{-}+     |        |     |         |}
 Channel N {-{-}{-}{-}{-}| PCM N |{-}{-}{-}{-}|        |     |         |}
                 +{-{-}{-}{-}{-}{-}{-}+     +{-}{-}{-}{-}{-}{-}{-}{-}+     +{-}{-}{-}{-}{-}{-}{-}{-}{-}+}
\end{verbatim}

\textbf{દરેક ચેનલ માટે PCM પ્રક્રિયા:}

\begin{enumerate}
\tightlist
\item
  \textbf{સેમ્પલિંગ:} દરેક ચેનલને fs \geq 2fmax પર સેમ્પલ કરવામાં આવે છે
\item
  \textbf{ક્વોન્ટાઇઝેશન:} સેમ્પલ્સને ડિસ્ક્રીટ લેવલમાં સોંપવામાં આવે છે
\item
  \textbf{એન્કોડિંગ:} ક્વોન્ટાઇઝ્ડ વેલ્યુને બાઇનરી કોડમાં કન્વર્ટ કરવામાં આવે છે
\end{enumerate}

\textbf{TDM ફ્રેમ સ્ટ્રક્ચર:}

\begin{itemize}
\tightlist
\item
  ફ્રેમમાં દરેક ચેનલમાંથી એક સેમ્પલ હોય છે
\item
  ફ્રેમમાં સિન્ક્રોનાઇઝેશન બિટ્સ/વર્ડ શામેલ છે
\item
  ફ્રેમ રેટ સેમ્પલિંગ રેટ (fs) જેટલો છે
\item
બિટ રેટ = fs \times N \times n (N = ચેનલ્સ,

n = બિટ્સ/સેમ્પલ)

\end{itemize}

\textbf{ટિપિકલ પેરામીટર્સ:}

\begin{itemize}
\tightlist
\item
  \textbf{વોઇસ ચેનલ્સ:} 8 kHz સેમ્પલિંગ, 8 બિટ્સ/સેમ્પલ
\item
  \textbf{T1 સિસ્ટમ:} 24 ચેનલ, 1.544 Mbps
\item
  \textbf{E1 સિસ્ટમ:} 30 ચેનલ, 2.048 Mbps
\end{itemize}

\textbf{ફાયદાઓ:}

\begin{itemize}
\tightlist
\item
  \textbf{કાર્યક્ષમ ટ્રાન્સમિશન:} સિંગલ હાઈ-સ્પીડ લિંક
\item
  \textbf{ડિજિટલ ફાયદાઓ:} નોઈઝ ઇમ્યુનિટી, રિજનરેશન
\item
  \textbf{લવચીકતા:} સરળતાથી ચેનલ્સ ઉમેરવા/કાઢવા
\end{itemize}

\textbf{એપ્લિકેશન:}

\begin{itemize}
\tightlist
\item
  \textbf{ટેલિફોન નેટવર્ક્સ:} ડિજિટલ ટ્રાન્સમિશન સિસ્ટમ્સ
\item
  \textbf{ડિજિટલ ઓડિયો:} બ્રોડકાસ્ટિંગ અને રેકોર્ડિંગ
\item
  \textbf{સેટેલાઇટ કોમ્યુનિકેશન:} મલ્ટિપલ ચેનલ ટ્રાન્સમિશન
\end{itemize}

\textbf{TDM ફ્રેમનો આકૃતિ:}

\begin{verbatim}
   |{{-}{-}{-}{-}{-}{-}{-}{-}{-}{-}{-}{-}{-}{-} One TDM Frame {-}{-}{-}{-}{-}{-}{-}{-}{-}{-}{-}{-}{-}{-}|}
   +{-{-}{-}{-}{-}+{-}{-}{-}{-}{-}+{-}{-}{-}{-}{-}+{-}{-}{-}{-}{-}+{-}{-}{-}{-}{-}+       +{-}{-}{-}{-}{-}+}
   | Sync| Ch1 | Ch2 | Ch3 | Ch4 | ..... | ChN |
   +{-{-}{-}{-}{-}+{-}{-}{-}{-}{-}+{-}{-}{-}{-}{-}+{-}{-}{-}{-}{-}+{-}{-}{-}{-}{-}+       +{-}{-}{-}{-}{-}+}
\end{verbatim}

\end{solutionbox}
\begin{mnemonicbox}
``MSQT'' - Multiplex, Sample, Quantize, Transmit

\end{mnemonicbox}

\end{document}
