\documentclass{article}

% content/resources/templates/preamble.tex
\usepackage[margin=0.6in]{geometry}
\author{Milav Dabgar}
\usepackage{amsmath,amssymb,amsthm}
\usepackage{booktabs}
\usepackage{multirow}
\usepackage{xcolor}
\usepackage{tcolorbox}
\tcbuselibrary{breakable,skins}
\usepackage[colorlinks=true,linkcolor=blue]{hyperref}
\usepackage{titlesec}
\usepackage{enumitem}
\usepackage{tikz}
\usepackage{pgfplots}
\usepackage{circuitikz}
\usepackage[version=4]{mhchem}
\usepackage{longtable}
\usepackage{array}
\usepackage{float}
\usepackage{caption}
\usepackage{listings}

\lstset{
  basicstyle=\small\ttfamily,
  breaklines=true,
  breakatwhitespace=false,
  postbreak=\mbox{\textcolor{red}{$\hookrightarrow$}\space},
  float=false,
  numbers=left,
  numberstyle=\tiny\color{gray},
  numbersep=10pt,
  xleftmargin=2em,
  keywordstyle=\color{blue},
  commentstyle=\color{green!60!black},
  stringstyle=\color{purple},
  backgroundcolor=\color{gray!5},
  showstringspaces=false,
  tabsize=2,
  captionpos=b,
  keepspaces=true,
  columns=flexible
}

\pgfplotsset{compat=1.18}
\usetikzlibrary{shapes,arrows,positioning,calc,patterns,decorations.pathmorphing,decorations.markings,arrows.meta}

% Color scheme
\definecolor{headcolor}{RGB}{0,102,204}
\definecolor{keycolor}{RGB}{220,20,60}
\definecolor{solutioncolor}{RGB}{34,139,34}
\definecolor{mnemoniccolor}{RGB}{148,0,211}
\definecolor{codecolor}{RGB}{0,0,100}

% Spacing
\setlength{\parskip}{3pt}
\setlist[itemize]{nosep}
\setlist[enumerate]{nosep}

% Title formatting
\titleformat{\section}{\Large\bfseries\color{headcolor}}{\thesection}{1em}{}
\titleformat{\subsection}{\large\bfseries\color{headcolor}}{\thesubsection}{1em}{}

% Pandoc tightlist compatibility
\providecommand{\tightlist}{%
  \setlength{\itemsep}{0pt}\setlength{\parskip}{0pt}}

% Pandoc longtable compatibility
\newcounter{none}
\def\thenone{}


% content/resources/templates/gujarati-boxes.tex
\usepackage{fontspec}
\usepackage{polyglossia}

% Set Gujarati as main language (document is primarily in Gujarati)
% Note: gloss-gujarati.ldf doesn't exist in polyglossia, but it will use hyphenation patterns
\setdefaultlanguage{gujarati}
\setotherlanguage{english}

% Configure Gujarati font properly
% Use Language=Default to prevent polyglossia from trying to add language-specific features
% that don't exist for Gujarati, which causes "empty feature" warnings
\newfontfamily\gujaratifont[Script=Gujarati,AutoFakeBold=2.5,AutoFakeSlant=0.3]{Noto Sans Gujarati}
\setmainfont[Script=Gujarati,AutoFakeBold=2.5,AutoFakeSlant=0.3]{Noto Sans Gujarati}
% Use Noto Sans Gujarati for monospace to support Gujarati in text
\setmonofont[Scale=0.9]{Noto Sans Gujarati}

% Configure English to use the same font
\newfontfamily\englishfont[Script=Gujarati,AutoFakeBold=2.5,AutoFakeSlant=0.3]{Noto Sans Gujarati}

% Translations for polyglossia
\gappto\captionsgujarati{
  \renewcommand{\tablename}{કોષ્ટક}
  \renewcommand{\figurename}{આકૃતિ}
}

% Helper for TikZ nodes to ensure Gujarati font
\newcommand{\gu}[1]{{\gujaratifont #1}}

% Custom environments
\newtcolorbox{solutionbox}{
    breakable,
    enhanced,
    colback=solutioncolor!5!white,
    colframe=solutioncolor!75!black,
    fonttitle=\bfseries,
    title=જવાબ
}

\newtcolorbox{solutionboxnobreak}{
 colback=solutioncolor!5!white,
 colframe=solutioncolor!75!black,
 fonttitle=\bfseries,
 title=જવાબ
}

\newtcolorbox{keyformula}{
 breakable,
 enhanced,
 colback=keycolor!5!white,
 colframe=keycolor!75!black,
 fonttitle=\bfseries,
 title=રાસાયણિક સમીકરણ/સૂત્ર
}

\newtcolorbox{mnemonicbox}{
 breakable,
 enhanced,
 colback=mnemoniccolor!5!white,
 colframe=mnemoniccolor!75!black,
 fonttitle=\bfseries,
 title=મેમરી ટ્રીક
}


% Custom commands for GTU solutions
% This file defines semantic commands for consistent formatting

% Question command with automatic formatting
\newcommand{\question}[2]{%
  \section*{Question #1}%
  \textbf{#2}%
}

% OR question variant
\newcommand{\questionor}[2]{%
  \section*{Question #1 OR}%
  \textbf{#2}%
}

% Proper table environment with caption
\newenvironment{answertable}[1]{%
  \begin{table}[htbp]
  \centering
  \caption{#1}
}{%
  \end{table}
}

% Proper figure environment for diagrams
\newenvironment{answerdiagram}[1]{%
  \begin{figure}[htbp]
  \centering
  \caption{#1}
}{%
  \end{figure}
}

% Semantic markup for key terms
\newcommand{\keyword}[1]{\textbf{#1}}
\newcommand{\code}[1]{\texttt{#1}}
\newcommand{\classname}[1]{\texttt{#1}}
\newcommand{\methodname}[1]{\texttt{#1}}

% Proper quotation marks
\newcommand{\mnemonic}[1]{``#1''}


\title{ઇલેક્ટ્રોનિક કોમ્યુનિકેશનના સિદ્ધાંતો (4331104) - વિન્ટર 2023 સોલ્યુશન}
\date{January 20, 2023}

\begin{document}
\maketitle

\questionmarks{1}{a}{3}
\textbf{અવાજ સંકેતનું વર્ગીકરણ કરો અને થર્મલ અવાજ સમજાવો.}

\begin{solutionbox}
    \textbf{અવાજ સંકેતનું વર્ગીકરણ:}

    \begin{center}
    \begin{tabulary}{\linewidth}{L L L}
        \hline
        \textbf{અવાજનો પ્રકાર} & \textbf{સ્ત્રોત} & \textbf{લક્ષણો} \\
        \hline
        \textbf{બાહ્ય અવાજ} & કોમ્યુનિકેશન સિસ્ટમની બહાર & વાતાવરણીય, અવકાશ, ઔદ્યોગિક \\
        \textbf{આંતરિક અવાજ} & કોમ્યુનિકેશન સિસ્ટમની અંદર & થર્મલ, શોટ, ટ્રાન્ઝિટ ટાઈમ, ફ્લિકર \\
        \hline
    \end{tabulary}
    \captionof{table}{અવાજ વર્ગીકરણ}
    \end{center}

    \textbf{થર્મલ અવાજ:}
    \begin{itemize}
        \item \textbf{વ્યાખ્યા}: તાપમાનને કારણે કન્ડક્ટરમાં ઇલેક્ટ્રોન્સની અનિયમિત ગતિ. જેને Johnson-Nyquist noise પણ કહેવાય છે.
        \item \textbf{લક્ષણો}: સફેદ અવાજ જેમાં આવર્તન સ્પેક્ટ્રમમાં એકસમાન પાવર હોય છે.
        \item \textbf{સૂત્ર}: $N = kTB$
        \begin{itemize}
            \item $k$: બોલ્ટઝમેન અચળાંક ($1.38 \times 10^{-23}$ J/K)
            \item $T$: તાપમાન (Kelvin)
            \item $B$: બેન્ડવિડ્થ (Hertz)
        \end{itemize}
    \end{itemize}

    \begin{mnemonicbox}
    "TERM" - Temperature Excites Random Movements
    \end{mnemonicbox}
\end{solutionbox}

\questionmarks{1}{b}{4}
\textbf{પ્રી-એમ્ફેસીસ અને ડી-એમ્ફેસીસ તકનીક વચ્ચેની સરખામણી કરો.}

\begin{solutionbox}
    \textbf{પ્રી-એમ્ફેસીસ અને ડી-એમ્ફેસીસ વચ્ચેનો તફાવત:}

    \begin{center}
    \begin{tabulary}{\linewidth}{L L L}
        \hline
        \textbf{પરિમાણ} & \textbf{પ્રી-એમ્ફેસીસ} & \textbf{ડી-એમ્ફેસીસ} \\
        \hline
        \textbf{વ્યાખ્યા} & ટ્રાન્સમિશન પહેલા ઉચ્ચ આવર્તન ઘટકોને વધારવા & રિસીવર પર ઉચ્ચ આવર્તન ઘટકોને ઘટાડવા \\
        \textbf{સ્થાન} & ટ્રાન્સમીટર બાજુ & રિસીવર બાજુ \\
        \textbf{હેતુ} & ઉચ્ચ આવર્તન માટે SNR સુધારે છે & મૂળ સિગ્નલની આવર્તન પ્રતિક્રિયા પુનઃસ્થાપિત કરે છે \\
        \textbf{સર્કિટ} & RC સર્કિટ સાથે હાઈ-પાસ ફિલ્ટર & RC સર્કિટ સાથે લો-પાસ ફિલ્ટર \\
        \textbf{સમય અચળાંક} & 75 $\mu$s (માનક) & 75 $\mu$s (પ્રી-એમ્ફેસીસ સાથે મેળ ખાય છે) \\
        \hline
    \end{tabulary}
    \captionof{table}{પ્રી-એમ્ફેસીસ વિરુદ્ધ ડી-એમ્ફેસીસ}
    \end{center}

    \begin{center}
    \begin{tikzpicture}[auto, node distance=1.5cm,
        block/.style={draw, rectangle, minimum height=2.5em, minimum width=3.5em, align=center, fill=white},
        >=stealth
    ]
        \node [coordinate] (input) {};
        \node [block, right=0.5cm of input, fill=orange!20] (pre) {Pre-emphasis};
        \node [block, right=0.8cm of pre] (mod) {Modulator};
        \node [block, right=0.8cm of mod] (tx) {Transmission};
        \node [block, below=1cm of tx] (demod) {Demodulator};
        \node [block, left=0.8cm of demod, fill=blue!20] (de) {De-emphasis};
        \node [coordinate, left=0.8cm of de] (out) {};

        \draw [->] (input) -- node[above, font=\small]{Input} (pre);
        \draw [->] (pre) -- (mod);
        \draw [->] (mod) -- (tx);
        \draw [->, dashed] (tx) -- (demod);
        \draw [->] (demod) -- (de);
        \draw [->] (de) -- node[above, font=\small]{Output} (out);
    \end{tikzpicture}
    \captionof{figure}{FM સિસ્ટમમાં પ્રી-એમ્ફેસીસ અને ડી-એમ્ફેસીસ}
    \end{center}

    \begin{mnemonicbox}
    "PUBTAR" - Pump Up Before Transmit, Pull Down After Receive
    \end{mnemonicbox}
\end{solutionbox}

\questionmarks{1}{c}{7}
\textbf{AM સિગ્નલની ગણિતિક અભિવ્યક્તિ મેળવો અને તેની મદદથી AM સિગ્નલના આવર્તન સ્પેક્ટ્રમને સમજાવો.}

\begin{solutionbox}
    \textbf{ગણિતિક અભિવ્યક્તિ નિર્માણ:}

    1. કેરિયર સિગ્નલ:
    \[ c(t) = A_c \cos(2\pi f_c t) \]
    2. મોડ્યુલેટિંગ સિગ્નલ:
    \[ m(t) = A_m \cos(2\pi f_m t) \]
    3. Amplitude Modulated સિગ્નલ $s(t)$ નીચે મુજબ છે:
    \[ s(t) = A_c[1 + \mu \frac{m(t)}{A_m}]\cos(2\pi f_c t) \]
       જ્યાં $\mu = \frac{A_m}{A_c}$ મોડ્યુલેશન ઇન્ડેક્સ છે.
    4. $m(t)$ ની કિંમત મુકતા:
    \[ s(t) = A_c[1 + \mu \cos(2\pi f_m t)]\cos(2\pi f_c t) \]
    5. વિસ્તરણ કરતા:
    \[ s(t) = A_c \cos(2\pi f_c t) + \mu A_c \cos(2\pi f_m t)\cos(2\pi f_c t) \]
    6. ત્રિકોણમિતીય ઓળખ $\cos(A)\cos(B) = \frac{1}{2}[\cos(A+B) + \cos(A-B)]$ નો ઉપયોગ કરીને:
    \[ s(t) = A_c \cos(2\pi f_c t) + \frac{\mu A_c}{2} [\cos(2\pi(f_c+f_m)t) + \cos(2\pi(f_c-f_m)t)] \]
    
    આ AM સિગ્નલની ગણિતિક અભિવ્યક્તિ છે.

    \textbf{આવર્તન સ્પેક્ટ્રમ:}
    
    \begin{center}
    \begin{tabulary}{\linewidth}{L L L}
        \hline
        \textbf{ઘટક} & \textbf{આવર્તન} & \textbf{એમ્પ્લિટ્યુડ} \\
        \hline
        Carrier & $f_c$ & $A_c$ \\
        Upper Sideband (USB) & $f_c + f_m$ & $\frac{\mu A_c}{2}$ \\
        Lower Sideband (LSB) & $f_c - f_m$ & $\frac{\mu A_c}{2}$ \\
        \hline
    \end{tabulary}
    \captionof{table}{AM સ્પેક્ટ્રમ ઘટકો}
    \end{center}

    \begin{center}
    \begin{tikzpicture}
        \begin{axis}[
            width=10cm, height=5cm,
            axis lines=middle,
            xtick={-1,0,1}, 
            xticklabels={$f_c-f_m$, $f_c$, $f_c+f_m$},
            ytick={0, 0.5, 1},
            yticklabels={0, $\frac{\mu A_c}{2}$, $A_c$},
            ymin=0, ymax=1.2,
            xmin=-1.5, xmax=1.5,
            xlabel={$f$}, ylabel={Amplitude},
            title={AM વેવનું ફ્રીક્વન્સી સ્પેક્ટ્રમ}
        ]
            \addplot[ycomb, mark=none, thick, blue] coordinates {
                (0, 1) (-1, 0.5) (1, 0.5)
            };
            \node[above] at (axis cs:0,1) {Carrier};
            \node[above] at (axis cs:-1,0.5) {LSB};
            \node[above] at (axis cs:1,0.5) {USB};
        \end{axis}
    \end{tikzpicture}
    \captionof{figure}{AM આવર્તન સ્પેક્ટ્રમ}
    \end{center}

    \begin{mnemonicbox}
    "CSBT" - Carrier Standing Between Twins
    \end{mnemonicbox}
\end{solutionbox}

\questionmarks{1}{c}{7}
\textbf{કોમ્યુનિકેશન સિસ્ટમનો બ્લોક ડાયાગ્રામ સમજાવો.}

\begin{solutionbox}
    \textbf{કોમ્યુનિકેશન સિસ્ટમનો બ્લોક ડાયાગ્રામ:}

    \begin{center}
    \begin{tikzpicture}[auto, node distance=2cm,
        block/.style={draw, rectangle, minimum height=3em, minimum width=4em, align=center, fill=white},
        noise/.style={draw, circle, fill=red!20},
        >=stealth
    ]
        \node [coordinate] (source) {};
        \node [block, right=0.5cm of source] (input) {Input Transducer};
        \node [block, right=1cm of input] (tx) {Transmitter};
        \node [block, right=1.5cm of tx] (rx) {Receiver};
        \node [block, right=1cm of rx] (output) {Output Transducer};
        \node [coordinate, right=0.5cm of output] (dest) {};
        
        % Channel
        \node [coordinate] (ch_in) at ($(tx.east)!0.5!(rx.west)$);
        \node [block, below=1.5cm of ch_in, minimum width=6em] (channel) {Channel / Medium};
        \node [noise, right=1cm of channel] (noise) {Noise};
        
        \draw [->] (source) -- node[above, font=\small]{Info} (input);
        \draw [->] (input) -- node[above, font=\small]{Elec. Signal} (tx);
        \draw [->] (tx) -| (channel);
        \draw [->] (channel) |- (rx);
        \draw [->] (rx) -- node[above, font=\small]{Original Info} (output);
        \draw [->] (output) -- (dest);
        \draw [->, dashed, red] (noise) -- (channel);

    \end{tikzpicture}
    \captionof{figure}{કોમ્યુનિકેશન સિસ્ટમ}
    \end{center}

    \textbf{ઘટકો અને કાર્યો:}

    \begin{center}
    \begin{tabulary}{\linewidth}{L L L}
        \hline
        \textbf{બ્લોક} & \textbf{કાર્ય} & \textbf{ઉદાહરણ} \\
        \hline
        \textbf{Input Transducer} & મૂળ માહિતીને ઇલેક્ટ્રિકલ સિગ્નલમાં રૂપાંતરિત કરે છે & માઇક્રોફોન, કેમેરા \\
        \textbf{Transmitter} & કુશળ ટ્રાન્સમિશન માટે સિગ્નલની પ્રક્રિયા કરે છે (મોડ્યુલેશન, એમ્પ્લિફિકેશન) & રેડિયો ટ્રાન્સમીટર \\
        \textbf{Channel/Medium} & જે માર્ગ દ્વારા સિગ્નલ પ્રવાસ કરે છે & હવા, ફાઇબર, કેબલ \\
        \textbf{Receiver} & મૂળ સિગ્નલ મેળવે છે (એમ્પ્લિફિકેશન, ફિલ્ટરિંગ, ડિમોડ્યુલેશન) & રેડિયો રિસીવર \\
        \textbf{Output Transducer} & ઇલેક્ટ્રિકલ સિગ્નલને મૂળ સ્વરૂપમાં પાછું ફેરવે છે & સ્પીકર, ડિસ્પ્લે \\
        \textbf{Noise Source} & અવાંછિત સિગ્નલ્સ જે માહિતીને વિકૃત કરે છે & એટમોસ્ફેરિક, થર્મલ \\
        \hline
    \end{tabulary}
    \captionof{table}{સિસ્ટમ ઘટકો}
    \end{center}

    \begin{mnemonicbox}
    "ITCRO" - Input Transmits Through Channel, Receives Output
    \end{mnemonicbox}
\end{solutionbox}

\questionmarks{2}{a}{3}
\textbf{એમ્પ્લિટ્યુડ મોડ્યુલેશનમાં સાઇડબેન્ડ્સ અને કેરીયર વેવ વચ્ચે પાવર વિતરણની ચર્ચા કરો.}

\begin{solutionbox}
    \textbf{AM સિગ્નલમાં પાવર વિતરણ:}

    કુલ પાવર $P_t$ એ કેરિયર પાવર $P_c$ અને સાઇડબેન્ડ પાવર $P_{SB}$ નો સરવાળો છે.

    \begin{center}
    \begin{tabulary}{\linewidth}{L L L}
        \hline
        \textbf{ઘટક} & \textbf{પાવર ફોર્મ્યુલા} & \textbf{ટકાવારી ($m=1$)} \\
        \hline
        Carrier & $P_c = \frac{A_c^2}{2R}$ & 66.7\% \\
        Upper Sideband & $P_{USB} = \frac{P_c \mu^2}{4}$ & 16.65\% \\
        Lower Sideband & $P_{LSB} = \frac{P_c \mu^2}{4}$ & 16.65\% \\
        Total Power & $P_T = P_c(1 + \frac{\mu^2}{2})$ & 100\% \\
        \hline
    \end{tabulary}
    \captionof{table}{AM પાવર વિતરણ}
    \end{center}

    \begin{center}
    \begin{tikzpicture}
        \begin{axis}[
            ybar stacked,
            bar width=30pt,
            width=6cm, height=6cm,
            ymin=0, ymax=110,
            symbolic x coords={AM Power},
            xtick=data,
            ylabel={Percentage (\%)},
            title={પાવર વિતરણ ($\mu=1$)},
            legend style={at={(0.5,-0.2)}, anchor=north}
        ]
            \addplot[fill=blue!30] coordinates {(AM Power, 66.67)}; % Carrier
            \addplot[fill=red!30] coordinates {(AM Power, 16.66)}; % LSB
            \addplot[fill=green!30] coordinates {(AM Power, 16.66)}; % USB
            \legend{Carrier, LSB, USB}
        \end{axis}
    \end{tikzpicture}
    \captionof{figure}{પાવર બ્રેકડાઉન}
    \end{center}

    \begin{mnemonicbox}
    "CTTT" - Carrier Takes Two-Thirds
    \end{mnemonicbox}
\end{solutionbox}

\questionmarks{2}{b}{4}
\textbf{શા માટે પ્રિએમ્ફેસીસ અને ડિએમ્ફેસીસનો ઉપયોગ કરવામાં આવે છે? સંક્ષિપ્તમાં વર્ણન કરો કે કેવી રીતે ટ્રાન્સમીટર બાજુ અને રીસીવર બાજુ પર સંકેતો સંશોધિત થાય છે.}

\begin{solutionbox}
    \textbf{પ્રી-એમ્ફેસીસ અને ડી-એમ્ફેસીસનો હેતુ:}
    
    મુખ્યત્વે FM માં નોઇઝ ફ્લોરના સંબંધમાં ઉચ્ચ-આવર્તન ઘટકો માટે Signal-to-Noise Ratio (SNR) સુધારવા માટે વપરાય છે.

    \begin{itemize}
        \item \textbf{SNR સુધારવું}: ટ્રાન્સમિશન પહેલા ઉચ્ચ આવર્તનને વધારે છે જેથી અવાજને ઓળંગી શકાય.
        \item \textbf{અવાજ ઘટાડવો}: FM માં ઉચ્ચ આવર્તન અવાજ માટે વધુ સંવેદનશીલ હોય છે.
        \item \textbf{વિશ્વસનીયતા જાળવવી}: ડી-એમ્ફેસીસ મૂળ સપાટ આવર્તન પ્રતિક્રિયાને પુનઃસ્થાપિત કરે છે.
    \end{itemize}

    \textbf{સિગ્નલ મોડિફિકેશન પ્રક્રિયા:}

    \begin{center}
    \begin{tikzpicture}[auto, node distance=1.5cm,
        block/.style={draw, rectangle, minimum height=2.5em, text width=4em, align=center},
        signal/.style={font=\footnotesize\itshape}
    ]
        \node [coordinate] (in) {};
        \node [block, right=0.5cm of in, fill=orange!20] (pre) {Pre-emphasis\\(Tx)};
        \node [block, right=1.5cm of pre] (ch) {FM\\Channel};
        \node [block, right=1.5cm of ch, fill=blue!20] (de) {De-emphasis\\(Rx)};
        \node [coordinate, right=0.5cm of de] (out) {};

        \draw [->] (in) -- node[above]{Audio In} (pre);
        \draw [->] (pre) -- node[above, align=center]{Boosted HF\\($>2$kHz)} (ch);
        \draw [->] (ch) -- node[above, align=center]{Noisy HF} (de);
        \draw [->] (de) -- node[above]{Restored Audio} (out);
        
        % Frequency response sketches
        \node [below=0.5cm of pre] {\tikz{\draw[->] (0,0) -- (1,0); \draw[->] (0,0) -- (0,0.8); \draw[blue] (0,0.2) -- (0.5,0.2) -- (0.9,0.7);}};
        \node [below=0.5cm of de] {\tikz{\draw[->] (0,0) -- (1,0); \draw[->] (0,0) -- (0,0.8); \draw[blue] (0,0.7) -- (0.4,0.2) -- (0.9,0.2);}};
    \end{tikzpicture}
    \captionof{figure}{સિગ્નલ મોડિફિકેશન પ્રવાહ}
    \end{center}

    \begin{mnemonicbox}
    "BHCKO" - Boost High, Cut High, Keep Original
    \end{mnemonicbox}
\end{solutionbox}

\questionmarks{2}{c}{7}
\textbf{FM જનરેશનની તકનીકો સમજાવો. ફેઝ લૉક લૂપ FM મોડ્યુલેટરને વિગતવાર સમજાવો.}

\begin{solutionbox}
    \textbf{FM જનરેશન તકનીકો:}

    \begin{center}
    \begin{tabulary}{\linewidth}{L L L}
        \hline
        \textbf{તકનીક} & \textbf{સિદ્ધાંત} & \textbf{ફાયદા} \\
        \hline
        Direct FM & ઓસિલેટરમાં કેપેસિટન્સ બદલવું (દા.ત., Varactor) & સરળ ડિઝાઇન \\
        Indirect FM & FM ઉત્પન્ન કરવા માટે Phase modulation & વધુ સ્થિરતા \\
        PLL FM & Phase locked loop નો ઉપયોગ & ઉચ્ચ આવર્તન સ્થિરતા \\
        Armstrong & મિક્સર્સ અને મલ્ટિપ્લાયર્સનો ઉપયોગ & ઉત્તમ રેખીયતા \\
        \hline
    \end{tabulary}
    \captionof{table}{FM જનરેશન પદ્ધતિઓ}
    \end{center}

    \textbf{PLL FM મોડ્યુલેટર:}

    \begin{center}
    \begin{tikzpicture}[auto, node distance=1.5cm,
        block/.style={draw, rectangle, minimum height=2.5em, minimum width=4em, align=center},
        sum/.style={draw, circle, inner sep=0pt, minimum size=6mm},
        >=stealth
    ]
        \node [block] (ref) {Ref Osc};
        \node [block, right=1cm of ref] (pd) {Phase Det};
        \node [block, right=1cm of pd] (lpf) {Loop Filter};
        \node [sum, right=1cm of lpf] (adder) {+};
        \node [coordinate, above=0.8cm of adder] (mod_in) {};
        \node [block, below=1.5cm of adder] (vco) {VCO};
        \node [coordinate, right=1.5cm of vco] (out) {};

        \draw [->] (ref) -- (pd);
        \draw [->] (pd) -- (lpf);
        \draw [->] (lpf) -- (adder);
        \draw [->] (mod_in) -- node[left]{Modulating Signal $m(t)$} (adder);
        \draw [->] (adder) -- (vco.east -| adder.south) -- (vco.north); 
        
        \draw [->] (vco) -- (out) node[right]{FM Output};
        \draw [->] (vco.west) -| (pd.south);
        
    \end{tikzpicture}
    \captionof{figure}{PLL FM મોડ્યુલેટર}
    \end{center}

    \textbf{કાર્ય સિદ્ધાંત:}
    \begin{enumerate}
        \item \textbf{Phase Detector} VCO આવર્તનની સ્થિર રેફરન્સ ઓસિલેટર સાથે તુલના કરે છે.
        \item \textbf{Loop Filter} DC કંટ્રોલ માપ પૂરો પાડે છે, ઉચ્ચ આવર્તન ફેરફારોને અવરોધે છે.
        \item \textbf{Modulating Signal} કંટ્રોલ વોલ્ટેજમાં ઉમેરવામાં આવે છે.
        \item આ મેસેજ સિગ્નલ અનુસાર \textbf{VCO} આવર્તન બદલે છે (FM).
        \item PLL ફીડબેક લાંબા ગાળે કેન્દ્ર આવર્તનને સ્થિર રાખે છે (રેફરન્સ સાથે લોક), જ્યારે ટૂંકા ગાળાના મોડ્યુલેશન માટે વિચલનની મંજૂરી આપે છે.
    \end{enumerate}

    \begin{mnemonicbox}
    "PDCFV" - Phase Detector Compares, Filter Smooths, VCO Varies
    \end{mnemonicbox}
\end{solutionbox}

\questionmarks{2}{a}{3}
\textbf{DSB કરતાં SSBના ફાયદા અને ગેરલાભ જણાવો.}

\begin{solutionbox}
    \textbf{SSBના DSB કરતાં ફાયદા અને ગેરલાભ:}

    \begin{center}
    \begin{tabulary}{\linewidth}{L L}
        \hline
        \textbf{SSBના ફાયદા} & \textbf{SSBના ગેરલાભ} \\
        \hline
        \textbf{બેન્ડવિડ્થ કાર્યક્ષમતા}: DSB ની સરખામણીમાં માત્ર અડધી બેન્ડવિડ્થ ($f_m$) વાપરે છે. & \textbf{જટિલ સર્કિટરી}: સાઇડબેન્ડ સપ્રેશન માટે શાર્પ ફિલ્ટર્સની જરૂર છે. \\
        \textbf{પાવર કાર્યક્ષમતા}: સમાન SNR માટે લગભગ 1/3 થી 1/6 પાવર વાપરે છે. & \textbf{મુશ્કેલ ડિમોડ્યુલેશન}: ચોક્કસ કેરિયર રિકવરીની જરૂર છે. \\
        \textbf{ઘટાડેલું ફેડિંગ}: સિલેક્ટિવ ફેડિંગ માટે ઓછું સંવેદનશીલ. & \textbf{વિકૃતિ}: પ્રાયોગિક ફિલ્ટર્સ નીચા આવર્તનને મંદ કરી શકે છે. \\
        \textbf{ઓછું ઇન્ટરફેરન્સ}: સાંકડી ચેનલ વપરાશ. & \textbf{કિંમત}: ટ્રાન્સમીટર/રિસીવરની કિંમત વધારે. \\
        \hline
    \end{tabulary}
    \captionof{table}{SSB વિરુદ્ધ DSB}
    \end{center}

    \begin{mnemonicbox}
    "PBSCN" - Power and Bandwidth Saved, But Complex Circuits Needed
    \end{mnemonicbox}
\end{solutionbox}

\questionmarks{2}{b}{4}
\textbf{DSBSC અને SSB એમ્પ્લિટ્યુડ મોડ્યુલેટેડ વેવ અને પાયલોટ કેરિયરના ફ્રીક્વન્સી સ્પેક્ટ્રમનું સ્કેચ કરો.}

\begin{solutionbox}
    \textbf{ફ્રીક્વન્સી સ્પેક્ટ્રમ તુલના:}

    \begin{center}
    \begin{tikzpicture}
        % DSBSC
        \begin{scope}[yshift=3cm]
            \draw[->] (-2,0) -- (2,0) node[right] {$f$};
            \draw[->] (0,0) -- (0,1.5) node[above] {$V$};
            \node[right] at (2,1) {\textbf{DSB-SC}};
            \draw[thick, blue] (-1,0) -- (-1,1) node[above]{LSB};
            \draw[thick, blue] (1,0) -- (1,1) node[above]{USB};
            \node[below] at (0,0) {$f_c$};
            \node[below] at (-1,0) {$f_c-f_m$};
            \node[below] at (1,0) {$f_c+f_m$};
        \end{scope}

        % SSB with Pilot
        \begin{scope}[yshift=0cm]
            \draw[->] (-2,0) -- (2,0) node[right] {$f$};
            \draw[->] (0,0) -- (0,1.5) node[above] {$V$};
            \node[right] at (2,1) {\textbf{SSB + Pilot}};
            
            \draw[thick, blue] (1,0) -- (1,1) node[above]{USB};
            \draw[thick, red] (0,0) -- (0,0.4) node[above, font=\tiny]{Pilot};
            
            \node[below] at (0,0) {$f_c$};
            \node[below] at (1,0) {$f_c+f_m$};
        \end{scope}
    \end{tikzpicture}
    \captionof{figure}{DSB-SC vs SSB સ્પેક્ટ્રમ (પાયલોટ સાથે)}
    \end{center}

    \begin{itemize}
        \item \textbf{DSB-SC}: કેરિયર દબાયેલું, પાવર માત્ર સાઇડબેન્ડ્સમાં. બેન્ડવિડ્થ $2f_m$.
        \item \textbf{SSB + Pilot}: માત્ર એક સાઇડબેન્ડ ટ્રાન્સમિટ થાય છે + સિંક્રનાઇઝેશન માટે ઘટાડેલ કેરિયર. બેન્ડવિડ્થ $f_m$.
    \end{itemize}

    \begin{mnemonicbox}
    "TSOSP" - Two Sides, One Side, or One Side Plus Pilot
    \end{mnemonicbox}
\end{solutionbox}

\questionmarks{2}{c}{7}
\textbf{ટૂંકી નોંધ લખો: પલ્સ મોડ્યુલેશન.}

\begin{solutionbox}
    \textbf{પલ્સ મોડ્યુલેશન તકનીકો:}
    
    પ્રક્રિયા જ્યાં સતત એનાલોગ સિગ્નલને સેમ્પલ કરીને પલ્સ પેરામીટર્સમાં રૂપાંતરિત કરવામાં આવે છે.

    \begin{center}
    \begin{tabulary}{\linewidth}{L L L}
        \hline
        \textbf{પ્રકાર} & \textbf{સિદ્ધાંત} & \textbf{ઉપયોગ} \\
        \hline
        \textbf{PAM} & પલ્સનું એમ્પ્લિટ્યુડ સિગ્નલ સાથે બદલાય છે & TDM, PCM માટે મધ્યવર્તી પગલું \\
        \textbf{PWM} & પલ્સની પહોળાઈ/અવધિ બદલાય છે & મોટર કંટ્રોલ, પાવર ડિલિવરી \\
        \textbf{PPM} & પલ્સની સ્થિતિ/ટાઇમિંગ બદલાય છે & ઓપ્ટિકલ કોમ્યુનિકેશન, RF કંટ્રોલ \\
        \textbf{PCM} & ડિજિટલ બાઇનરી કોડ રજૂઆત & કમ્પ્યુટિંગ, ડિજિટલ ઓડિયો, ટેલિફોની \\
        \hline
    \end{tabulary}
    \captionof{table}{પલ્સ મોડ્યુલેશન પ્રકારો}
    \end{center}

    \textbf{વેવફોર્મ તુલના:}

    \begin{center}
    \begin{tikzpicture}[xscale=1.2, yscale=0.6]
        % Analog
        \draw[gray, dotted] plot[domain=0:6, samples=50] (\x, {1 + 0.5*sin(deg(\x))});
        \node[left] at (0,1) {Analog};
        
        % PAM
        \begin{scope}[yshift=-2.5cm]
            \foreach \x in {0.5, 1.5, ..., 5.5} {
                \draw[thick, blue] (\x, 0) -- (\x, {1 + 0.8*sin(deg(\x))});
                \fill (\x, {1 + 0.8*sin(deg(\x))}) circle (1.5pt);
            }
            \draw[->] (0,0) -- (6.5,0);
            \node[left] at (0,0.5) {PAM};
        \end{scope}

        % PWM
        \begin{scope}[yshift=-5cm]
            \foreach \x in {0.5, 1.5, ..., 5.5} {
                \pgfmathsetmacro{\w}{0.2 + 0.2*sin(deg(\x))}
                \draw[thick, blue] (\x-\w, 0) -- (\x-\w, 1) -- (\x+\w, 1) -- (\x+\w, 0);
            }
            \draw[->] (0,0) -- (6.5,0);
            \node[left] at (0,0.5) {PWM};
        \end{scope}

        % PPM
        \begin{scope}[yshift=-7.5cm]
             \foreach \x in {0.5, 1.5, ..., 5.5} {
                \pgfmathsetmacro{\s}{0.3*sin(deg(\x))}
                \draw[thick, blue] (\x+\s, 0) -- (\x+\s, 1);
            }
            \draw[->] (0,0) -- (6.5,0);
            \node[left] at (0,0.5) {PPM};
        \end{scope}
    \end{tikzpicture}
    \captionof{figure}{પલ્સ મોડ્યુલેશન વેવફોર્મ્સ}
    \end{center}

    \begin{mnemonicbox}
    "AWPC" - Amplitude, Width, Position, Code - All Pulse Types
    \end{mnemonicbox}
\end{solutionbox}

\questionmarks{3}{a}{3}
\textbf{AGC શું છે? સરળ AGC સર્કિટના ઇનપુટ-આઉટપુટ લક્ષણિક વળાંક દોરો અને સમજાવો.}

\begin{solutionbox}
    \textbf{ઓટોમેટિક ગેઇન કંટ્રોલ (AGC):}
    \begin{itemize}
        \item \textbf{વ્યાખ્યા}: સર્કિટ જે ઇનપુટ સિગ્નલની શક્તિમાં ફેરફાર હોવા છતાં આઉટપુટ સિગ્નલ લેવલને પ્રમાણમાં સ્થિર રાખવા માટે રિસીવર ગેઇનને આપમેળે સમાયોજિત કરે છે.
        \item \textbf{હેતુ}: મજબૂત સિગ્નલો પર ઓવરલોડિંગ અને નબળા સિગ્નલો પર ફેડિંગ અટકાવે છે.
    \end{itemize}

    \textbf{ઇનપુટ-આઉટપુટ લક્ષણિકતાઓ:}

    \begin{center}
    \begin{tikzpicture}[scale=0.8]
        \draw[->] (0,0) -- (6,0) node[right] {Input Voltage};
        \draw[->] (0,0) -- (0,4) node[above] {Output Voltage};
        
        \draw[dashed] (0,0) -- (2,2) -- (6,6); % Linear / No AGC
        \node[right, font=\footnotesize] at (5,5) {No AGC};
        
        \draw[thick, blue] (0,0) -- (2,2) -- (6,2.2); % With AGC
        \node[right, blue, font=\footnotesize] at (6,2.2) {With AGC};
        
        \draw[dotted] (2,0) -- (2,2);
        \node[below] at (2,0) {Threshold};
    \end{tikzpicture}
    \captionof{figure}{AGC લક્ષણિકતાઓ}
    \end{center}

    \textbf{સમજૂતી}: થ્રેશોલ્ડથી નીચેના નબળા સંકેતો માટે લીનિયર પ્રતિસાદ. થ્રેશોલ્ડથી ઉપર, આઉટપુટ ફ્લેટ (સપાટ) કરવા માટે ગેઇન ઓછો થાય છે.

    \begin{mnemonicbox}
    "SSLG" - Strong Signals Get Less Gain
    \end{mnemonicbox}
\end{solutionbox}

\questionmarks{3}{b}{4}
\textbf{FM ડિમોડ્યુલેશન માટે બેલેન્સ્ડ રેશિયો ડિટેક્ટર પર ટૂંકી નોંધ લખો.}

\begin{solutionbox}
    \textbf{બેલેન્સ્ડ રેશિયો ડિટેક્ટર:}

    \begin{itemize}
        \item FM ડિમોડ્યુલેટર જે ડાયોડ પ્રવાહોના ગુણોત્તરમાંથી આઉટપુટ મેળવે છે.
        \item \textbf{મુખ્ય ઘટકો}: સેન્ટર-ટેપ્ડ ટ્રાન્સફોર્મર, બે ડાયોડ, મોટું ઇલેક્ટ્રોલાઇટિક કેપેસિટર (AM રિજેક્શન માટે).
        \item \textbf{ફાયદો}: અલગ લિમિટર વગર એમ્પ્લિટ્યુડ ફેરફારો (AM રિજેક્શન) સામે સહજ પ્રતિરક્ષા પૂરી પાડે છે.
    \end{itemize}

    \textbf{સર્કિટ ડાયાગ્રામ:}

    \begin{center}
    \begin{circuitikz}[font=\footnotesize]
        \draw (0,0) node[left]{FM In} to[L] (0,-2);
        \draw (1.5,0) to[L] (1.5,-2);
        \draw (1.5,-1) -- (2.5,-1); 
        
        \draw (1.5,0) -- (2.5,0) to[diode, l=$D_1$] (4.5,0);
        \draw (1.5,-2) -- (2.5,-2) to[diode, l=$D_2$, invert] (4.5,-2);
        
        \draw (4.5,0) to[C, l=$C_1$] (4.5,-1) coordinate (mid);
        \draw (4.5,-2) to[C, l=$C_2$] (mid);
        
        \draw (4.5,0) -- (6,0) to[C, l=$C_L$ (Large)] (6,-2) -- (4.5,-2);
        
        \draw (mid) -- (6,-1) node[right] {Output};
        
        \draw (2.5,-1) to[L] (1,-1) to[C] (0,-1); 
    \end{circuitikz}
    \captionof{figure}{રેશિયો ડિટેક્ટર સર્કિટ}
    \end{center}

    \begin{mnemonicbox}
    "BDTFV" - Balanced Diodes Transform Frequency To Voltage
    \end{mnemonicbox}
\end{solutionbox}

\questionmarks{3}{c}{7}
\textbf{વિવિધ પ્રકારના FM ડિમોડ્યુલેટર સર્કિટનું કાર્ય સમજાવો.}

\begin{solutionbox}
    \textbf{FM ડિમોડ્યુલેટર પ્રકારો:}

    \begin{center}
    \begin{tabulary}{\linewidth}{L L L}
        \hline
        \textbf{પ્રકાર} & \textbf{કાર્ય સિદ્ધાંત} & \textbf{ફાયદા/ગેરફાયદા} \\
        \hline
        \textbf{Slope Detector} & ટ્યુન્ડ સર્કિટના નોન-લીનિયર રિજનનો ઉપયોગ & સરળ / નબળી રેખીયતા \\
        \textbf{Foster-Seeley} & ફેઝ શિફ્ટ ડિફરન્શિએશન & સારી રેખીયતા / કોઈ AM રિજેક્શન નહીં \\
        \textbf{Ratio Detector} & ડાયોડ વોલ્ટેજનો ગુણોત્તર & સારું AM રિજેક્શન / મધ્યમ રેખીયતા \\
        \textbf{PLL Demodulator} & ઇનપુટ સાથે ફેઝ લોકિંગ & ઉત્તમ રેખીયતા / જટિલ \\
        \textbf{Quadrature} & ફેઝ શિફ્ટ અને ગુણાકાર & સરળ IC ઇન્ટીગ્રેશન \\
        \hline
    \end{tabulary}
    \captionof{table}{FM ડિમોડ્યુલેટર પ્રકારો}
    \end{center}

    \textbf{PLL FM ડિમોડ્યુલેટર ડાયાગ્રામ:}

    \begin{center}
    \begin{tikzpicture}[auto, node distance=1.5cm,
        block/.style={draw, rectangle, minimum height=3em, minimum width=4em, align=center},
        >=stealth
    ]
        \node [block] (pd) {Phase Det};
        \node [block, right=1.5cm of pd] (lpf) {Low Pass Filter};
        \node [block, below=1cm of lpf] (vco) {VCO};
        \node [coordinate, left=1cm of pd] (in) {};
        \node [coordinate, right=1cm of lpf] (out) {};

        \draw [->] (in) -- node[above]{FM Input} (pd);
        \draw [->] (pd) -- node[above]{Error} (lpf);
        \draw [->] (lpf) -- coordinate (tap) (out) node[right]{Demodulated Output};
        \draw [->] (tap) |- (vco);
        \draw [->] (vco) -| (pd);
    \end{tikzpicture}
    \captionof{figure}{PLL ડિમોડ્યુલેટર}
    \end{center}

    \textbf{કાર્ય:} VCO ને ઇનપુટ FM સિગ્નલ સાથે લોક રાખવા માટે જરૂરી એરર વોલ્ટેજ આવર્તન વિચલનના પ્રમાણમાં હોય છે, આમ મૂળ સંદેશને રિકવર કરે છે.

    \begin{mnemonicbox}
    "FVDPE" - Frequency Variations Drive Phase Errors
    \end{mnemonicbox}
\end{solutionbox}


% Part 2 (Q3 OR - Q5)

\questionmarks{3}{a}{3} % OR
\textbf{રેડિયો રીસીવરની લાક્ષણિકતાઓ સમજાવો.}

\begin{solutionbox}
    \textbf{રેડિયો રીસીવરની લાક્ષણિકતાઓ:}

    \begin{center}
    \begin{tabulary}{\linewidth}{L L L}
        \hline
        \textbf{લાક્ષણિકતા} & \textbf{વ્યાખ્યા} & \textbf{મહત્વ} \\
        \hline
        \textbf{સંવેદનશીલતા} & નબળા સિગ્નલને એમ્પ્લિફાય કરવાની ક્ષમતા & મહત્તમ રિસેપ્શન રેન્જ નક્કી કરે છે \\
        \textbf{પસંદગીકારકતા} & આસપાસના સિગ્નલથી વાંછિત સિગ્નલને અલગ કરવાની ક્ષમતા & હસ્તક્ષેપ અટકાવે છે \\
        \textbf{વફાદારી} & મૂળ સિગ્નલને પુનઃ ઉત્પન્ન કરવામાં ચોકસાઈ & અવાજની ગુણવત્તા સુનિશ્ચિત કરે છે \\
        \textbf{છબી આવર્તન અસ્વીકૃતિ} & છબી આવર્તનને અસ્વીકાર કરવાની ક્ષમતા & ડુપ્લિકેટ રિસેપ્શન અટકાવે છે \\
        \hline
    \end{tabulary}
    \captionof{table}{રિસીવર લાક્ષણિકતાઓ}
    \end{center}

    \begin{center}
    \begin{tikzpicture}[
        mindmap, concept color=blue!20,
        every node/.style={concept, scale=0.8},
        grow cyclic,
        level 1/.style={level distance=3cm, sibling angle=90}
    ]
        \node [concept color=orange!40] {Receiver\\Characteristics}
        child { node {Sensitivity} }
        child { node {Selectivity} }
        child { node {Fidelity} }
        child { node {Image Rejection} };
    \end{tikzpicture}
    \captionof{figure}{મુખ્ય લાક્ષણિકતાઓ}
    \end{center}

    \begin{mnemonicbox}
    "SSFIM" - Select Signals Faithfully, Ignore Mirrors
    \end{mnemonicbox}
\end{solutionbox}

\questionmarks{3}{b}{4} % OR
\textbf{AM ડિટેક્ટર સર્કિટમાં થતા વિકૃતિઓના પ્રકારો સમજાવો.}

\begin{solutionbox}
    \textbf{AM ડિટેક્ટર સર્કિટમાં વિકૃતિઓના પ્રકારો:}

    \begin{center}
    \begin{tabulary}{\linewidth}{L L L}
        \hline
        \textbf{વિકૃતિ પ્રકાર} & \textbf{કારણ} & \textbf{નિવારણ} \\
        \hline
        \textbf{ડાયાગોનલ વિકૃતિ} & ખોટો સમય અચળાંક (RC બહુ મોટો) & યોગ્ય RC સમય અચળાંક \\
        \textbf{નકારાત્મક પીક ક્લિપિંગ} & ઉચ્ચ મોડ્યુલેશન ઇન્ડેક્સ + AC/DC લોડ મિસમેચ & યોગ્ય ડાયોડ બાયસિંગ / લોડ \\
        \textbf{હાર્મોનિક વિકૃતિ} & નોન-લીનિયર ડાયોડ લક્ષણો & ઉચ્ચ-ગુણવત્તાવાળા ડાયોડ \\
        \hline
    \end{tabulary}
    \captionof{table}{AM ડિટેક્ટર વિકૃતિઓ}
    \end{center}

    \textbf{વેવફોર્મ્સ:}

    \begin{center}
    \begin{tikzpicture}[xscale=2, yscale=1]
        % Normal
        \draw[gray, dotted] plot[domain=0:2*pi] (\x, {1 + 0.5*sin(deg(\x))});
        
        % Diagonal Clipping
        \begin{scope}[yshift=-2.5cm]
            \draw[->] (0,0) -- (6.5,0);
            \node[left] at (0,0.5) {Diagonal};
            \draw[thick, red] (0,1) -- (0.5,1.4) -- (1.5,0.8) -- (2.0,1.4); 
            \draw[blue] plot[domain=0:6] (\x, {1 + 0.5*sin(deg(\x))});
            \draw[red, thick] (1.6, 1.5) -- (2.5, 0.8); 
        \end{scope}

        % Negative Peak Clipping
        \begin{scope}[yshift=-5cm]
            \draw[->] (0,0) -- (6.5,0);
            \node[left] at (0,0.5) {Negative Peak};
            \draw[blue] plot[domain=0:6] (\x, {1 + 0.8*sin(deg(\x))});
            \draw[red, thick] (3.5, 0) -- (5.5, 0); 
        \end{scope}
    \end{tikzpicture}
    \captionof{figure}{વિકૃતિ પ્રકારો}
    \end{center}

    \begin{mnemonicbox}
    "DNHF" - Diagonal Negative Harmonics Frequency
    \end{mnemonicbox}
\end{solutionbox}

\questionmarks{3}{c}{7} % OR
\textbf{સુપરહીટેરોડીન AM રીસીવરનો બ્લોક ડાયાગ્રામ દોરો અને તેને સમજાવો.}

\begin{solutionbox}
    \textbf{સુપરહીટેરોડીન AM રીસીવર:}

    \begin{center}
    \begin{tikzpicture}[auto, node distance=1.5cm,
        block/.style={draw, rectangle, minimum height=3em, minimum width=3.5em, align=center},
        >=stealth
    ]
        \node [coordinate] (ant) {};
        \node [block, right=0.5cm of ant] (rf) {RF Amp};
        \node [block, right=1cm of rf] (mix) {Mixer};
        \node [block, below=1cm of mix] (lo) {Local Osc};
        \node [block, right=1cm of mix] (if) {IF Amp};
        \node [block, right=1cm of if] (det) {Detector};
        \node [block, right=1cm of det] (af) {AF Amp};
        \node [coordinate, right=0.5cm of af] (spk) {};
        \node [block, above=1cm of if] (agc) {AGC};

        \draw [->] (ant) -- node[above]{Antenna} (rf);
        \draw [->] (rf) -- (mix);
        \draw [->] (lo) -- (mix);
        \draw [->] (mix) -- node[above]{IF} (if);
        \draw [->] (if) -- (det);
        \draw [->] (det) -- (af);
        \draw [->] (af) -- node[right]{Speaker} (spk);
        
        \draw [->] (det.north) |- (agc.east);
        \draw [->] (agc.west) -| (rf.north);
        \draw [->] (agc.south) -- (if.north);

    \end{tikzpicture}
    \captionof{figure}{સુપરહીટેરોડીન રીસીવર}
    \end{center}

    \textbf{દરેક બ્લોકનું કાર્ય:}
    \begin{itemize}
        \item \textbf{RF એમ્પ્લિફાયર}: નબળા RF સિગ્નલને એમ્પ્લિફાય કરે છે, જે સંવેદનશીલતા અને પસંદગીકારકતા સુધારે છે.
        \item \textbf{મિક્સર}: RF ($f_s$) અને LO ($f_o$) ને મિક્સ કરીને IF ($f_o - f_s$) ઉત્પન્ન કરે છે.
        \item \textbf{લોકલ ઓસીલેટર}: આવતા સિગ્નલથી નિશ્ચિત આવર્તન પર સિગ્નલ ઉત્પન્ન કરે છે.
        \item \textbf{IF એમ્પ્લિફાયર}: નિશ્ચિત મધ્યવર્તી આવર્તન (455 kHz) પર એમ્પ્લિફિકેશન પૂરું પાડે છે.
        \item \textbf{ડિટેક્ટર}: મોડ્યુલેટેડ સિગ્નલમાંથી ઓડિયો એક્સ્ટ્રેક્ટ કરે છે.
        \item \textbf{AF એમ્પ્લિફાયર}: સ્પીકર ચલાવવા માટે ઓડિયોને એમ્પ્લિફાય કરે છે.
        \item \textbf{AGC}: રિસીવરના આઉટપુટ લેવલને સ્થિર રાખે છે.
    \end{itemize}

    \begin{mnemonicbox}
    "RMLIDAS" - Radio Mixing Local Intermediate Detected Audio Signals
    \end{mnemonicbox}
\end{solutionbox}

\questionmarks{4}{a}{3}
\textbf{એનાલોગથી ડિજિટલ રૂપાંતરણમાં વપરાતી ક્વોન્ટાઇઝેશનની પ્રક્રિયા સમજાવો.}

\begin{solutionbox}
    \textbf{ક્વોન્ટાઇઝેશન પ્રક્રિયા:}
    \begin{enumerate}
        \item \textbf{સેમ્પલિંગ}: સમયને અસતત (discretize) બનાવવો.
        \item \textbf{લેવલ ફાળવણી}: એમ્પ્લિટ્યુડ રેન્જને $L$ ડિસ્ક્રીટ લેવલમાં વિભાજિત કરવી.
        \item \textbf{અસાઇનમેન્ટ}: દરેક સેમ્પલ વેલ્યુને નજીકના લેવલમાં મેપ કરવી.
        \item \textbf{એનકોડિંગ}: લેવલ ઇન્ડેક્સને બાઇનરીમાં રૂપાંતરિત કરવું.
    \end{enumerate}

    \begin{center}
    \begin{tikzpicture}[scale=0.8]
        \draw[->] (0,0) -- (6,0) node[right]{Time};
        \draw[->] (0,0) -- (0,5) node[above]{Amplitude};
        
        \foreach \y in {1,2,3,4} \draw[gray, dotted] (0,\y) -- (6,\y);
        
        \draw[blue, thick] plot[domain=0:6] (\x, {2.5 + 1.5*sin(deg(\x))});
        \draw[red, thick, step=0.5] plot[domain=0:6, samples=15] (\x, {round(2.5 + 1.5*sin(deg(\x)))});
        
        \node[right, blue] at (6, 2.5) {Analog};
        \node[right, red] at (6, 3) {Quantized};
    \end{tikzpicture}
    \captionof{figure}{ક્વોન્ટાઇઝેશન સ્ટેરકેસ}
    \end{center}

    \begin{mnemonicbox}
    "SLAB" - Sample Levels Assign Binary
    \end{mnemonicbox}
\end{solutionbox}

\questionmarks{4}{b}{4}
\textbf{સેમ્પલિંગ તકનીકોની સરખામણી આપો.}

\begin{solutionbox}
    \textbf{સેમ્પલિંગ તકનીકોની સરખામણી:}

    \begin{center}
    \begin{tabulary}{\linewidth}{L L L}
        \hline
        \textbf{તકનીક} & \textbf{વર્ણન} & \textbf{ફાયદા/ગેરલાભ} \\
        \hline
        Ideal & ત્વરિત આવેગ & માત્ર સૈદ્ધાંતિક \\
        Natural & પલ્સ ટોપ સિગ્નલના આકારને અનુસરે છે & જટિલ જનરેશન \\
        Flat-top & પલ્સ ટોપ ફ્લેટ છે (સેમ્પલ અને હોલ્ડ) & જનરેટ કરવા માટે સરળ / એપર્ચર એરર \\
        \hline
    \end{tabulary}
    \captionof{table}{સેમ્પલિંગ પ્રકારો}
    \end{center}

    \begin{center}
    \begin{tikzpicture}[scale=0.8]
        \node[left] at (0,3) {Natural};
        \foreach \x in {1,2,3,4,5} {
             \draw[fill=blue!50] (0.8*\x, 0) -- (0.8*\x, {2+0.5*sin(deg(\x))}) -- (0.8*\x+0.2, {2+0.5*sin(deg(\x+0.2))}) -- (0.8*\x+0.2, 0) -- cycle;
        }

        \node[left] at (0,1) {Flat-top};
        \foreach \x in {1,2,3,4,5} {
             \draw[fill=red!50] (0.8*\x, -2) rectangle (0.8*\x+0.2, {0.5*sin(deg(\x))});
        }
    \end{tikzpicture}
    \captionof{figure}{નેચરલ વિરુદ્ધ ફ્લેટ-ટોપ સેમ્પલિંગ}
    \end{center}

    \begin{mnemonicbox}
    "INF" - Ideal Natural Flat
    \end{mnemonicbox}
\end{solutionbox}

\questionmarks{4}{c}{7}
\textbf{PCM ટ્રાન્સમીટર અને રીસીવરનો બ્લોક ડાયાગ્રામ દોરો અને સમજાવો.}

\begin{solutionbox}
    \textbf{Pulse Code Modulation (PCM):}

    \textbf{ટ્રાન્સમીટર:}
    \begin{center}
    \begin{tikzpicture}[auto, node distance=1.2cm,
        block/.style={draw, rectangle, minimum height=2.5em, text width=4em, align=center},
        >=stealth
    ]
        \node[coordinate](in){};
        \node[block, right=0.5cm of in](lpf){LPF};
        \node[block, right=0.8cm of lpf](sh){S/H};
        \node[block, right=0.8cm of sh](q){Quantizer};
        \node[block, right=0.8cm of q](enc){Encoder};
        \node[block, right=0.8cm of enc](p2s){P/S};
        \node[coordinate, right=0.5cm of p2s](out){};

        \draw[->] (in) -- (lpf);
        \draw[->] (lpf) -- (sh);
        \draw[->] (sh) -- (q);
        \draw[->] (q) -- (enc);
        \draw[->] (enc) -- (p2s);
        \draw[->] (p2s) -- node[above]{PCM Out} (out);
    \end{tikzpicture}
    \end{center}

    \textbf{રિસીવર:}
    \begin{center}
    \begin{tikzpicture}[auto, node distance=1.2cm,
        block/.style={draw, rectangle, minimum height=2.5em, text width=4em, align=center},
        >=stealth
    ]
        \node[coordinate](in){};
        \node[block, right=0.5cm of in](regen){Regeneration Circuit};
        \node[block, right=0.8cm of regen](dec){Decoder};
        \node[block, right=0.8cm of dec](recon){Reconstruction Filter};
        \node[coordinate, right=0.5cm of recon](out){};

        \draw[->] (in) -- node[above]{PCM In} (regen);
        \draw[->] (regen) -- (dec);
        \draw[->] (dec) -- (recon);
        \draw[->] (recon) -- node[above]{Analog Out} (out);
    \end{tikzpicture}
    \end{center}

    \begin{mnemonicbox}
    "FSQEMT" - Filter, Sample, Quantize, Encode, Multiplex, Transmit
    \end{mnemonicbox}
\end{solutionbox}

\questionmarks{4}{a}{3} % OR
\textbf{Nyquist પ્રમેય જણાવો અને સમજાવો.}

\begin{solutionbox}
    \textbf{Nyquist સેમ્પલિંગ પ્રમેય:}
    
    બેન્ડ-લિમિટેડ સિગ્નલને સંપૂર્ણ રીતે પુનઃનિર્માણ કરવા માટે, સેમ્પલિંગ આવર્તન $f_s$ સિગ્નલમાં હાજર મહત્તમ આવર્તન ઘટક $f_{max}$ ના ઓછામાં ઓછા બમણા હોવા જોઈએ.
    \[ f_s \ge 2 f_{max} \]
    
    \begin{itemize}
        \item $2f_{max}$ ને \textbf{Nyquist Rate} કહેવાય છે.
        \item જો $f_s < 2f_{max}$, તો \textbf{Aliasing} થાય છે (સ્પેક્ટ્રલ ઘટકોનું ઓવરલેપિંગ).
    \end{itemize}

    \begin{center}
    \begin{tikzpicture}[scale=0.7]
        % Aliasing
        \draw[->] (0,0) -- (6,0) node[right]{$f$};
        \node at (3,2) {Aliasing};
        \draw[blue] (0,0) -- (1,1.5) -- (2,0);
        \draw[red] (1.5,0) -- (2.5,1.5) -- (3.5,0);
        \node[below] at (2,0) {$f_s/2$};
        \node[font=\tiny] at (3,0.5) {Overlap};
    \end{tikzpicture}
    \captionof{figure}{અન્ડરસેમ્પલિંગની અસર (એલિયાસિંગ)}
    \end{center}

    \begin{mnemonicbox}
    "DMFSA" - Double Maximum Frequency Stops Aliasing
    \end{mnemonicbox}
\end{solutionbox}

\questionmarks{4}{b}{4} % OR
\textbf{DM, ADM અને DPCMની સરખામણી આપો.}

\begin{solutionbox}
    \textbf{સરખામણી:}

    \begin{center}
    \begin{tabulary}{\linewidth}{L L L L}
        \hline
        \textbf{લક્ષણ} & \textbf{Delta Mod (DM)} & \textbf{Adaptive DM} & \textbf{DPCM} \\
        \hline
        \textbf{આધાર} & 1 Bit તફાવત & ચલ સ્ટેપ સાઇઝ & Multi-bit તફાવત \\
        \textbf{સ્ટેપ સાઇઝ} & સ્થિર (Fixed) & ચલ (Variable) & સ્થિર/એડેપ્ટિવ \\
        \textbf{કોડિંગ} & 1 Bit/sample & 1 Bit/sample & $>1$ Bit/sample \\
        \textbf{ભૂલો} & સ્લોપ ઓવરલોડ, ગ્રેન્યુલર & ઓછી ભૂલો & ક્વોન્ટાઇઝેશન નોઇઝ \\
        \textbf{જટિલતા} & સૌથી ઓછી & મધ્યમ & વધુ \\
        \hline
    \end{tabulary}
    \captionof{table}{DM vs ADM vs DPCM}
    \end{center}

    \begin{mnemonicbox}
    "SAMD" - Single-bit, Adaptive-bit, Multi-bit Difference
    \end{mnemonicbox}
\end{solutionbox}

\questionmarks{4}{c}{7} % OR
\textbf{ડિફરન્શિયલ PCM (DPCM) ટ્રાન્સમીટર અને રીસીવરની કાર્યગીરી સમજાવો.}

\begin{solutionbox}
    \textbf{DPCM સિદ્ધાંત:}
    ચોક્કસ સેમ્પલ વેલ્યુને બદલે, આ અગાઉના સેમ્પલના આધારે આગાહી કરેલ મૂલ્ય અને વર્તમાન સેમ્પલ વચ્ચેના \textit{તફાવત}ને એનકોડ કરે છે.

    \textbf{DPCM ટ્રાન્સમીટર:}
    \begin{center}
    \begin{tikzpicture}[auto, node distance=1.5cm, >=stealth]
        \node[draw, rectangle, minimum height=2em] (quant) {Quantizer};
        \node[draw, rectangle, minimum height=2em, right=1cm of quant] (enc) {Encoder};
        \node[right=1cm of enc] (out) {};
        
        \node[draw, circle, left=1cm of quant] (sub) {$\Sigma$};
        \node[left=1cm of sub] (in) {};
        
        \node[draw, circle, below=1cm of quant] (add) {$\Sigma$};
        \node[draw, rectangle, minimum height=2em, left=1cm of add] (pred) {Predictor};
        
        \draw[->] (in) -- (sub);
        \draw[->] (sub) -- node[above]{\footnotesize E} (quant);
        \draw[->] (quant) -- (enc);
        \draw[->] (enc) -- (out);
        
        \draw[->] (quant) -- (add);
        \draw[->] (add) -- (pred);
        \draw[->] (pred) -| (sub);
        \draw[->] (pred) -- (add);
    \end{tikzpicture}
    \captionof{figure}{DPCM ટ્રાન્સમીટર}
    \end{center}

    \textbf{DPCM રિસીવર:}
    \begin{center}
    \begin{tikzpicture}[auto, node distance=1.5cm,
        block/.style={draw, rectangle, minimum height=2.5em, text width=4em, align=center},
        sum/.style={draw, circle, inner sep=0pt, minimum size=6mm},
        >=stealth
    ]
        \node[block](dec){Decoder};
        \node[sum, right=0.8cm of dec](sum){+};
        \node[block, below=1cm of sum](pred){Predictor};
        \node[block, right=0.8cm of sum](lpf){LPF};
        
        \draw[->] (dec) -- (sum);
        \draw[->] (sum) -- (lpf);
        \draw[->] (sum.east) |- (pred.east);
        \draw[->] (pred) -| (sum.south);
    \end{tikzpicture}
    \captionof{figure}{DPCM રિસીવર}
    \end{center}

    \begin{mnemonicbox}
    "PSQD" - Predict Subtract Quantize Difference
    \end{mnemonicbox}
\end{solutionbox}

\questionmarks{5}{a}{3}
\textbf{TDMA ફ્રેમનું વર્ણન કરો.}

\begin{solutionbox}
    \textbf{TDMA ફ્રેમ સ્ટ્રક્ચર:}

    ટાઇમ ડિવિઝન મલ્ટિપલ એક્સેસ બહુવિધ વપરાશકર્તાઓને અલગ ટાઇમ સ્લોટ્સ ફાળવીને સમાન આવર્તન શેર કરવાની મંજૂરી આપે છે.

    \begin{center}
    \begin{tikzpicture}
        \draw (0,0) rectangle (8,1);
        \foreach \x in {1,2,3,4,5,6,7} \draw (\x,0) -- (\x,1);
        \node at (0.5,0.5) {Sync};
        \node at (1.5,0.5) {User 1};
        \node at (2.5,0.5) {User 2};
        \node at (3.5,0.5) {...};
        \node at (7.5,0.5) {Guard};
        
        \draw[<->] (0, -0.5) -- (8,-0.5) node[midway, below] {One Frame};
        \draw[<->] (1.1, 1.2) -- (1.9, 1.2) node[midway, above] {Time Slot};
    \end{tikzpicture}
    \captionof{figure}{TDMA ફ્રેમ}
    \end{center}

    ઘટકો: પ્રીએમ્બલ (Sync), માહિતી સંદેશ, ગાર્ડ બિટ્સ (ગેપ).

    \begin{mnemonicbox}
    "SITDA" - Slots In Time Divide Access
    \end{mnemonicbox}
\end{solutionbox}

\questionmarks{5}{b}{4}
\textbf{4 સ્તરના ડિજિટલ મલ્ટિપ્લેક્સિંગ વંશવેલો દોરો અને સમજાવો.}

\begin{solutionbox}
    \textbf{ડિજિટલ મલ્ટિપ્લેક્સિંગ હાયરાર્કી (North American T-carrier):}

    \begin{center}
    \begin{tikzpicture}[
        grow=right,
        level 1/.style={level distance=3.5cm, sibling distance=2cm},
        level 2/.style={level distance=3.5cm, sibling distance=1.5cm},
        edge from parent/.style={draw, ->, thick},
        every node/.style={draw, rectangle, rounded corners, align=center, font=\footnotesize}
    ]
        \node {DS0\\(64 kbps)}
            child {
                node {T1 (Level 1)\\24 Ch\\1.544 Mbps}
                child {
                    node {T2 (Level 2)\\96 Ch\\6.312 Mbps}
                    child {
                        node {T3 (Level 3)\\672 Ch\\44.736 Mbps}
                        child {
                            node {T4 (Level 4)\\4032 Ch\\274 Mbps}
                        }
                    }
                }
            };
    \end{tikzpicture}
    \captionof{figure}{T-Carrier હાયરાર્કી}
    \end{center}

    \begin{mnemonicbox}
    "PSTQ" - Primary, Secondary, Tertiary, Quaternary Levels
    \end{mnemonicbox}
\end{solutionbox}

\questionmarks{5}{c}{7}
\textbf{PCM-TDM સિસ્ટમનો બ્લોક ડાયાગ્રામ દોરો અને સમજાવો.}

\begin{solutionbox}
    \textbf{PCM-TDM બ્લોક ડાયાગ્રામ:}

    \begin{center}
    \begin{tikzpicture}[auto, node distance=1.2cm,
        block/.style={draw, rectangle, minimum height=2.5em, text width=4em, align=center},
        >=stealth
    ]
        % Inputs
        \node (in1) {Ch 1};
        \node [below=0.6cm of in1] (in2) {Ch 2};
        \node [below=0.6cm of in2] (in3) {Ch 3};
        
        % LPFs
        \node [block, right=0.5cm of in1] (lpf1) {LPF};
        \node [block, right=0.5cm of in2] (lpf2) {LPF};
        \node [block, right=0.5cm of in3] (lpf3) {LPF};
        
        % Commutator (Sampler/MUX)
        \node [draw, circle, right=1cm of lpf2, minimum size=1cm] (mux) {MUX};
        
        % PCM Blocks
        \node [block, right=1cm of mux] (pcm) {PCM Encoder};
        \node [coordinate, right=0.5cm of pcm] (tx) {};
        
        % Rx side
        \node [block, right=1.5cm of pcm] (dec) {PCM Decoder};
        \node [draw, circle, right=1cm of dec, minimum size=1cm] (demux) {DeMUX};
        \node [block, right=1cm of demux] (lpf_out) {LPFs};
        
        \draw[->] (in1) -- (lpf1);
        \draw[->] (in2) -- (lpf2);
        \draw[->] (in3) -- (lpf3);
        
        \draw[->] (lpf1) -- (mux);
        \draw[->] (lpf2) -- (mux);
        \draw[->] (lpf3) -- (mux);
        
        \draw[->] (mux) -- (pcm);
        \draw[dashed, ->] (pcm) -- node[above]{TDM Link} (dec);
        
        \draw[->] (dec) -- (demux);
        \draw[->] (demux) -- (lpf_out);
        
    \end{tikzpicture}
    \captionof{figure}{PCM-TDM સિસ્ટમ}
    \end{center}

    **પ્રક્રિયા**:
    \begin{enumerate}
        \item બહુવિધ એનાલોગ ચેનલો LPF દ્વારા બેન્ડ-લિમિટેડ હોય છે.
        \item કોમ્યુટેટર દરેક ચેનલને ક્રમિક રીતે સેમ્પલ કરે છે (AM-TDM).
        \item સંયુક્ત TDM સિગ્નલ એક PCM એનકોડરમાં પ્રવેશે છે.
        \item કોડેડ બિટ્સ ઇન્ટરલેસ્ડ ટ્રાન્સમિટ થાય છે.
    \end{enumerate}

    \begin{mnemonicbox}
    "MACSDL" - Many Analog Channels Share Digital Link
    \end{mnemonicbox}
\end{solutionbox}

\questionmarks{5}{a}{3} % OR
\textbf{ડિજિટલ કમ્યુનિકેશનના ફાયદા અને ગેરફાયદાની સૂચિ બનાવો.}

\begin{solutionbox}
    \textbf{ફાયદા અને ગેરફાયદા:}

    \begin{center}
    \begin{tabulary}{\linewidth}{L L}
        \hline
        \textbf{ફાયદા} & \textbf{ગેરફાયદા} \\
        \hline
        ઉત્તમ નોઇઝ ઇમ્યુનિટી & વધારે બેન્ડવિડ્થની જરૂર \\
        એરર ડિટેક્શન અને કરેક્શન & સિસ્ટમની જટિલતા \\
        સરળ મલ્ટિપ્લેક્સિંગ (TDM) & સિન્ક્રોનાઇઝેશનની જરૂર \\
        સુરક્ષિત (એન્ક્રિપ્શન) & ક્વોન્ટાઇઝેશન નોઇઝ \\
        \hline
    \end{tabulary}
    \captionof{table}{ડિજિટલ કોમ્યુનિકેશન ફાયદા/ગેરફાયદા}
    \end{center}

    \begin{mnemonicbox}
    "NEMBB" - Noise-resistant, Error-correcting, Multiplex-friendly But Bandwidth-hungry
    \end{mnemonicbox}
\end{solutionbox}

\questionmarks{5}{b}{4} % OR
\textbf{ચેનલ કોડિંગ તકનીકોની સૂચિ બનાવો, તેમાંથી કોઇ પણ એકને ઉદાહરણ સાથે સમજાવો.}

\begin{solutionbox}
    \textbf{ચેનલ કોડિંગ તકનીકો:}
    \begin{itemize}
        \item Linear Block Codes (દા.ત., Hamming)
        \item Cyclic Codes (દા.ત., CRC)
        \item Convolutional Codes
        \item Turbo Codes
    \end{itemize}

    \textbf{ઉદાહરણ: Hamming Code (7,4)}
    \begin{itemize}
        \item 4 ડેટા બિટ્સ લે છે, 3 પેરિટી બિટ્સ ઉમેરે છે ($n=7, k=4$).
        \item 1 બિટ એરર સુધારી શકે છે.
        \item પેરિટી બિટ્સ $2^0, 2^1, 2^2...$ પોઝિશન પર મૂકવામાં આવે છે.
        \item જો Data = 1010, Encoded = $p_1 p_2 1 p_4 0 1 0$. ચોક્કસ પોઝિશનના XOR પર આધારિત પેરિટી ગણતરી.
    \end{itemize}

    \begin{mnemonicbox}
    "PBPDB" - Parity Bits Protect Data Bits
    \end{mnemonicbox}
\end{solutionbox}

\questionmarks{5}{c}{7} % OR
\textbf{મૂળભૂત ટાઇમ ડોમેન ડિજિટલ મલ્ટિપ્લેક્સિંગની ચર્ચા કરો. TDM સિસ્ટમના ફાયદા અને ગેરફાયદા જણાવો.}

\begin{solutionbox}
    \textbf{ટાઇમ ડિવિઝન મલ્ટિપલ એક્સેસ (TDM):}
    એવી તકનીક જ્યાં બહુવિધ અલગ સિગ્નલો એક ચેનલ પર સમયના ડોમેનમાં ઇન્ટરલીવ કરીને ટ્રાન્સમિટ થાય છે.

    \textbf{ફાયદા:}
    \begin{itemize}
        \item એક સમયે એક વપરાશકર્તા દ્વારા સંપૂર્ણ બેન્ડવિડ્થનો ઉપયોગ થાય છે.
        \item લવચીક સિગ્નલ હેન્ડલિંગ (ડિજિટલ).
        \item FDM ની સરખામણીમાં સરળ સર્કિટરી.
    \end{itemize}

    \textbf{ગેરફાયદા:}
    \begin{itemize}
        \item કડક સિન્ક્રોનાઇઝેશનની જરૂર છે.
        \item જો સ્લોટ્સ ખાલી હોય તો બેન્ડવિડ્થનો બગાડ.
        \item મલ્ટિપાથ ડિસ્ટોર્શન TDM ને FDM કરતા વધુ અસર કરે છે.
    \end{itemize}

    \begin{center}
    \begin{tikzpicture}[auto, node distance=1cm, >=stealth]
        \node (s1) {S1};
        \node [below=0.5cm of s1] (s2) {S2};
        \node [draw, rectangle, right=1cm of s1] (mux) {MUX};
        \node [right=2cm of mux] (dmux) {DeMUX};
        
        \draw[->] (s1) -- (mux);
        \draw[->] (s2) -- (mux);
        \draw[->] (mux) -- (dmux);
    \end{tikzpicture}
    \captionof{figure}{મૂળભૂત TDM}
    \end{center}

    \begin{mnemonicbox}
    "TSSBSR" - Time Slots Shared But Sync Required
    \end{mnemonicbox}
\end{solutionbox}

\end{document}
