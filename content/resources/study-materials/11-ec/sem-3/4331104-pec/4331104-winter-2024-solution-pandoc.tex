\documentclass[10pt,a4paper]{article}

% content/resources/templates/preamble.tex
\usepackage[margin=0.6in]{geometry}
\author{Milav Dabgar}
\usepackage{amsmath,amssymb,amsthm}
\usepackage{booktabs}
\usepackage{multirow}
\usepackage{xcolor}
\usepackage{tcolorbox}
\tcbuselibrary{breakable,skins}
\usepackage[colorlinks=true,linkcolor=blue]{hyperref}
\usepackage{titlesec}
\usepackage{enumitem}
\usepackage{tikz}
\usepackage{pgfplots}
\usepackage{circuitikz}
\usepackage[version=4]{mhchem}
\usepackage{longtable}
\usepackage{array}
\usepackage{float}
\usepackage{caption}
\usepackage{listings}

\lstset{
  basicstyle=\small\ttfamily,
  breaklines=true,
  breakatwhitespace=false,
  postbreak=\mbox{\textcolor{red}{$\hookrightarrow$}\space},
  float=false,
  numbers=left,
  numberstyle=\tiny\color{gray},
  numbersep=10pt,
  xleftmargin=2em,
  keywordstyle=\color{blue},
  commentstyle=\color{green!60!black},
  stringstyle=\color{purple},
  backgroundcolor=\color{gray!5},
  showstringspaces=false,
  tabsize=2,
  captionpos=b,
  keepspaces=true,
  columns=flexible
}

\pgfplotsset{compat=1.18}
\usetikzlibrary{shapes,arrows,positioning,calc,patterns,decorations.pathmorphing,decorations.markings,arrows.meta}

% Color scheme
\definecolor{headcolor}{RGB}{0,102,204}
\definecolor{keycolor}{RGB}{220,20,60}
\definecolor{solutioncolor}{RGB}{34,139,34}
\definecolor{mnemoniccolor}{RGB}{148,0,211}
\definecolor{codecolor}{RGB}{0,0,100}

% Spacing
\setlength{\parskip}{3pt}
\setlist[itemize]{nosep}
\setlist[enumerate]{nosep}

% Title formatting
\titleformat{\section}{\Large\bfseries\color{headcolor}}{\thesection}{1em}{}
\titleformat{\subsection}{\large\bfseries\color{headcolor}}{\thesubsection}{1em}{}

% Pandoc tightlist compatibility
\providecommand{\tightlist}{%
  \setlength{\itemsep}{0pt}\setlength{\parskip}{0pt}}

% Pandoc longtable compatibility
\newcounter{none}
\def\thenone{}


% content/resources/templates/english-boxes.tex
% This file is currently empty - it exists to maintain consistency with the import structure.
% Add custom environments here if needed in the future.


\begin{document}

\begin{center}
{\Huge\bfseries\color{headcolor} Subject Name Solutions}\\[5pt]
{\LARGE 4331104 -- Winter 2024}\\[3pt]
{\large Semester 1 Study Material}\\[3pt]
{\normalsize\textit{Detailed Solutions and Explanations}}
\end{center}

\vspace{10pt}

\subsection*{Question 1(a) [3 marks]}\label{q1a}

\textbf{What is modulation? What is the need of it?}

\begin{solutionbox}
Modulation is the process of varying one or more
properties (amplitude, frequency, or phase) of a high-frequency carrier
signal according to the instantaneous value of a lower frequency message
signal.

\textbf{Need for modulation:}

\begin{itemize}
\tightlist
\item
  \textbf{Antenna size reduction}: Allows practical antenna size (λ/4)
\item
  \textbf{Multiplexing}: Enables multiple signals to share same medium
\item
  \textbf{Interference reduction}: Shifts signal to suitable frequency
  band
\item
  \textbf{Range extension}: Increases transmission distance
\end{itemize}

\end{solutionbox}
\begin{mnemonicbox}
``AMIR'' - Antenna, Multiplexing, Interference, Range

\end{mnemonicbox}
\subsection*{Question 1(b) [4 marks]}\label{q1b}

\textbf{Derive the expression for DSBFC of AM wave.}

\begin{solutionbox}
DSBFC (Double Sideband Full Carrier) AM wave
derivation:

\textbf{Mathematical derivation:}

\begin{itemize}
\tightlist
\item
  Carrier signal: c(t) = Ac cos(ωct)
\item
  Message signal: m(t) = Am cos(ωmt)
\item
  AM signal: s(t) = Ac[1 + μm(t)]cos(ωct)
\item
Where

μ = modulation index = Am/Ac

\end{itemize}

\textbf{Substituting message signal:} s(t) = Ac[1 + μ
cos(ωmt)]cos(ωct) s(t) = Ac cos(ωct) + μAc cos(ωmt)cos(ωct)

\textbf{Using trigonometric identity:} cos(A)cos(B) = 1/2[cos(A+B) +
cos(A-B)]

\textbf{Final expression:} s(t) = Ac cos(ωct) + (μAc/2)[cos((ωc+ωm)t)
+ cos((ωc-ωm)t)]

\textbf{Diagram:}

\begin{verbatim}
    \^{}
    |    Carrier
Ac  |    /|{}
    |   / | {}
    |  /  |  {}
    | /   |   {}
    |/    |    {}
    +{-{-}{-}{-}{-}+{-}{-}{-}{-}{-}+{-}{-}{-}{-} f}
         fc
\end{verbatim}

\begin{verbatim}
    \^{}
    |                LSB   Carrier   USB
    |                 |      |       |
Pam |                 |      |       |
    |                 |      |       |
    |                 |      |       |
    |                /|{    /|     /|}
    +{-{-}{-}{-}{-}{-}{-}{-}{-}{-}{-}{-}{-}{-}{-}+{-}+{-}{-}{-}{-}{-}+{-}{-}{-}{-}{-}{-}+{-}+{-}{-}{-}{-} f}
                 fc{-fm     fc    fc+fm}
\end{verbatim}

\end{solutionbox}
\subsection*{Question 1(c) [7 marks]}\label{q1c}

\textbf{Classify Noise signal and explain flicker noise, shot noise and
thermal noise.}

\begin{solutionbox}

\textbf{Noise Classification:}

{\def\LTcaptype{none} % do not increment counter
\begin{longtable}[]{@{}
  >{\raggedright\arraybackslash}p{(\linewidth - 4\tabcolsep) * \real{0.1935}}
  >{\raggedright\arraybackslash}p{(\linewidth - 4\tabcolsep) * \real{0.2581}}
  >{\raggedright\arraybackslash}p{(\linewidth - 4\tabcolsep) * \real{0.5484}}@{}}
\toprule\noalign{}
\begin{minipage}[b]{\linewidth}\raggedright
Type
\end{minipage} & \begin{minipage}[b]{\linewidth}\raggedright
Source
\end{minipage} & \begin{minipage}[b]{\linewidth}\raggedright
Characteristics
\end{minipage} \\
\midrule\noalign{}
\endhead
\bottomrule\noalign{}
\endlastfoot
\textbf{External Noise} & Environmental sources & Outside communication
system \\
\textbf{Internal Noise} & Components & Generated within system \\
\end{longtable}
}

\textbf{Types of internal noise:}

\begin{enumerate}
\tightlist
\item
  \textbf{Flicker Noise:}

  \begin{itemize}
  \tightlist
  \item
    \textbf{Source}: Occurs in active devices
  \item
    \textbf{Characteristics}: Inversely proportional to frequency (1/f)
  \item
    \textbf{Effect}: Dominant at low frequencies
  \end{itemize}
\item
  \textbf{Shot Noise:}

  \begin{itemize}
  \tightlist
  \item
    \textbf{Source}: Random electron flow across junctions
  \item
    \textbf{Characteristics}: Independent of frequency (white noise)
  \item
    \textbf{Effect}: Random current fluctuations in diodes/transistors
  \end{itemize}
\item
  \textbf{Thermal Noise:}

  \begin{itemize}
  \tightlist
  \item
    \textbf{Source}: Random motion of electrons due to temperature
  \item
    \textbf{Characteristics}: Present in all conductors, resistors
  \item
    \textbf{Formula}: Pn = kTB (k=Boltzmann constant, T=temperature,
    B=bandwidth)
  \item
    \textbf{Effect}: Sets noise floor in receivers
  \end{itemize}
\end{enumerate}

\end{solutionbox}
\begin{mnemonicbox}
``FST'' - Flicker decreases with Frequency, Shot is
from electron flow, Thermal depends on Temperature

\end{mnemonicbox}
\subsection*{Question 1(c) OR [7
marks]}\label{q1c}

\textbf{Describe EM wave also write at least one application of
different band of spectrum.}

\begin{solutionbox}

\textbf{EM Wave:} Electromagnetic waves are energy propagating through
space as time-varying electric and magnetic fields, traveling at speed
of light (3\times10^{8} m/s).

\textbf{Characteristics:}

\begin{itemize}
\tightlist
\item
  Transverse waves with E and H fields perpendicular to each other
\item
  No medium required for propagation
\item
  Described by wavelength (λ) and frequency (f)
\item
  Relation: c = f \times λ
\end{itemize}

\textbf{EM Spectrum and Applications:}

{\def\LTcaptype{none} % do not increment counter
\begin{longtable}[]{@{}lll@{}}
\toprule\noalign{}
Frequency Band & Frequency Range & Application \\
\midrule\noalign{}
\endhead
\bottomrule\noalign{}
\endlastfoot
ELF & 3Hz-30Hz & Submarine communication \\
VLF & 3kHz-30kHz & Navigation systems \\
LF & 30kHz-300kHz & AM broadcasting \\
MF & 300kHz-3MHz & AM radio broadcasting \\
HF & 3MHz-30MHz & Shortwave radio \\
VHF & 30MHz-300MHz & FM radio, TV broadcasting \\
UHF & 300MHz-3GHz & TV, mobile phones, WiFi \\
SHF & 3GHz-30GHz & Satellite communication, radar \\
EHF & 30GHz-300GHz & Millimeter wave communication \\
Infrared & 300GHz-400THz & Remote controls, thermal imaging \\
Visible & 400THz-800THz & Fiber optic communication \\
Ultraviolet & 800THz-30PHz & Sterilization, authentication \\
X-Rays & 30PHz-30EHz & Medical imaging \\
Gamma Rays & \textgreater30EHz & Cancer treatment \\
\end{longtable}
}

\textbf{Diagram:}

\begin{verbatim}
      +{-{-}{-}{-}{-}{-}{-}{-}{-}{-}{-}{-}{-}{-}{-}{-}+{-}{-}{-}{-}{-}{-}{-}{-}{-}{-}{-}{-}{-}{-}{-}{-}+{-}{-}{-}{-}{-}{-}{-}{-}{-}{-}{-}{-}{-}{-}{-}{-}+{-}{-}{-}{-}{-}{-}{-}{-}{-}{-}{-}{-}{-}{-}{-}{-}+}
      |                |                |                |                |
Radio   Microwave    Infrared       Visible         Ultraviolet     X{-ray    Gamma}
      |                |                |                |                |
      +{-{-}{-}{-}{-}{-}{-}{-}{-}{-}{-}{-}{-}{-}{-}{-}+{-}{-}{-}{-}{-}{-}{-}{-}{-}{-}{-}{-}{-}{-}{-}{-}+{-}{-}{-}{-}{-}{-}{-}{-}{-}{-}{-}{-}{-}{-}{-}{-}+{-}{-}{-}{-}{-}{-}{-}{-}{-}{-}{-}{-}{-}{-}{-}{-}+}
  Increasing Frequency 
  Decreasing Wavelength 
\end{verbatim}

\end{solutionbox}
\begin{mnemonicbox}
``RMIUXG'' - Radio, Microwave, Infrared, Ultraviolet,
X-ray, Gamma

\end{mnemonicbox}
\subsection*{Question 2(a) [3 marks]}\label{q2a}

\textbf{State advantages of SSB over DSB.}

\begin{solutionbox}

\textbf{Advantages of SSB over DSB:}

{\def\LTcaptype{none} % do not increment counter
\begin{longtable}[]{@{}ll@{}}
\toprule\noalign{}
Parameter & SSB Advantage \\
\midrule\noalign{}
\endhead
\bottomrule\noalign{}
\endlastfoot
\textbf{Bandwidth} & 50\% less bandwidth requirement \\
\textbf{Power} & Power saving of 83.33\% \\
\textbf{Transmitter} & Less power amplification needed \\
\textbf{Receiver} & Simpler design without phase distortion \\
\textbf{SNR} & Better signal-to-noise ratio \\
\textbf{Fading} & Less susceptible to selective fading \\
\end{longtable}
}

\end{solutionbox}
\begin{mnemonicbox}
``BP TRFS'' - Bandwidth, Power, Transmitter,
Receiver, Fading, SNR

\end{mnemonicbox}
\subsection*{Question 2(b) [4 marks]}\label{q2b}

\textbf{Explain generation of FM wave using FET reactance modulator.}

\begin{solutionbox}

\textbf{FET Reactance Modulator:}

\textbf{Working principle:}

\begin{itemize}
\tightlist
\item
  Uses FET as voltage-controlled reactance
\item
  Changes effective capacitance based on modulating signal
\item
  Connected across LC tank circuit of oscillator
\end{itemize}

\textbf{Circuit operation:}

\begin{enumerate}
\tightlist
\item
  Modulating signal applied to gate of FET
\item
  FET drain-source resistance varies with gate voltage
\item
  Capacitive reactance changes with modulating signal
\item
  Oscillator frequency deviates with input signal
\end{enumerate}

\textbf{Diagram:}

\begin{verbatim}
      +{-{-}{-}{-}{-}||{-}{-}{-}{-}{-}+}
      |             |
      |             C
      |             |
    V\_in          +{-{-}{-}+}
      |           |FET|
      +{-{-}{-}{-}{-}R{-}{-}{-}{-}{-}|   |}
                  +{-{-}{-}+}
                    |
                   LC
                 Circuit
\end{verbatim}

\textbf{Key features:}

\begin{itemize}
\tightlist
\item
  \textbf{Simple design}: Fewer components than other modulators
\item
  \textbf{Linearity}: Good for wide-band FM generation
\item
  \textbf{Stability}: Temperature stable compared to varactor diodes
\end{itemize}

\end{solutionbox}
\begin{mnemonicbox}
``LOVE FM'' - LC Oscillator with Voltage-controlled
Element for FM

\end{mnemonicbox}
\subsection*{Question 2(c) [7 marks]}\label{q2c}

\textbf{Derive the equation for total power in AM, calculate percentage
of power savings in DSB and SSB.}

\begin{solutionbox}

\textbf{Power in AM signal:}

\textbf{For AM signal s(t) = Ac[1 + μcos(ωmt)]cos(ωct)}

\textbf{Total power calculation:}

\begin{enumerate}
\tightlist
\item
  Power in carrier: Pc = Ac^{2}/2
\item
  Power in sidebands: Ps = μ^{2}Ac^{2}/4 (total for both sidebands)
\item
  Total power: Pt = Pc + Ps = Ac^{2}/2 \times (1 + μ^{2}/2)
\end{enumerate}

\textbf{For 100\% modulation (μ=1):}

\begin{itemize}
\tightlist
\item
  Pt = Pc \times (1 + 1/2) = 1.5 \times Pc
\item
  Carrier power = 66.67\% of total
\item
  Sideband power = 33.33\% of total
\end{itemize}

\textbf{Power savings:}

\begin{enumerate}
\tightlist
\item
  \textbf{In DSB-SC:}

  \begin{itemize}
  \tightlist
  \item
    Carrier is suppressed
  \item
    Power saved = 66.67\%
  \end{itemize}
\item
  \textbf{In SSB:}

  \begin{itemize}
  \tightlist
  \item
    Carrier + one sideband suppressed
  \item
    Power saved = 66.67\% + 16.67\% = 83.33\%
  \end{itemize}
\end{enumerate}

\textbf{Comparative Table:}

{\def\LTcaptype{none} % do not increment counter
\begin{longtable}[]{@{}
  >{\raggedright\arraybackslash}p{(\linewidth - 8\tabcolsep) * \real{0.1714}}
  >{\raggedright\arraybackslash}p{(\linewidth - 8\tabcolsep) * \real{0.2143}}
  >{\raggedright\arraybackslash}p{(\linewidth - 8\tabcolsep) * \real{0.2286}}
  >{\raggedright\arraybackslash}p{(\linewidth - 8\tabcolsep) * \real{0.1857}}
  >{\raggedright\arraybackslash}p{(\linewidth - 8\tabcolsep) * \real{0.2000}}@{}}
\toprule\noalign{}
\begin{minipage}[b]{\linewidth}\raggedright
Modulation
\end{minipage} & \begin{minipage}[b]{\linewidth}\raggedright
Carrier Power
\end{minipage} & \begin{minipage}[b]{\linewidth}\raggedright
Sideband Power
\end{minipage} & \begin{minipage}[b]{\linewidth}\raggedright
Total Power
\end{minipage} & \begin{minipage}[b]{\linewidth}\raggedright
Power Saving
\end{minipage} \\
\midrule\noalign{}
\endhead
\bottomrule\noalign{}
\endlastfoot
AM (μ=1) & 100\% & 50\% & 150\% & 0\% \\
DSB-SC & 0\% & 50\% & 50\% & 66.67\% \\
SSB & 0\% & 25\% & 25\% & 83.33\% \\
\end{longtable}
}

\end{solutionbox}
\begin{mnemonicbox}
``CST'' - Carrier power, Sideband power, Total power

\end{mnemonicbox}
\subsection*{Question 2(a) OR [3
marks]}\label{q2a}

\textbf{Draw and explain Time domain and Frequency domain display of AM
wave.}

\begin{solutionbox}

\textbf{Time Domain and Frequency Domain Display of AM Wave:}

\textbf{Time Domain:}

\begin{itemize}
\tightlist
\item
  Shows amplitude variations over time
\item
  Envelope follows modulating signal
\item
  Maximum amplitude: A_{1} = Ac(1+μ)
\item
  Minimum amplitude: A_{2} = Ac(1-μ)
\item
  Modulation index: μ = (A_{1}-A_{2})/(A_{1}+A_{2})
\end{itemize}

\textbf{Frequency Domain:}

\begin{itemize}
\tightlist
\item
  Shows power distribution across frequencies
\item
  Carrier at center frequency fc
\item
  Upper sideband at fc+fm
\item
  Lower sideband at fc-fm
\item
  Bandwidth = 2fm
\end{itemize}

\textbf{Diagram:}

\begin{verbatim}
Time Domain:                         Frequency Domain:
    \^{                                    \^{}}
    |                                    |
A_{1  |    /      /                      |          Carrier}
    |   /  {    /                       |             |}
Ac  |{-{-}/{-}{-}{-}{-}{-}{-}/{-}{-}{-}{-}{-}{-}                  |             |}
    |  {    /      /                    |   LSB       |       USB}
A_{2  |     /      /                     |    |        |        |}
    |    {/      /                      |    |        |        |}
    +{-{-}{-}{-}{-}{-}{-}{-}{-}{-}{-}{-}{-}{-}{-}{-}{-}{-}{-}{-}{-}{-}{-}{-}{-}{-}{-}        +{-}{-}{-}{-}+{-}{-}{-}{-}{-}{-}{-}{-}{-}+{-}{-}{-}{-}{-}{-}{-}{-}+{-}{-}{-}{-}{-}{-}}
        t                                   fc{-fm      fc      fc+fm}
\end{verbatim}

\end{solutionbox}
\begin{mnemonicbox}
``TEF'' - Time domain shows Envelope, Frequency
domain shows spectral components

\end{mnemonicbox}
\subsection*{Question 2(b) OR [4
marks]}\label{q2b}

\textbf{Explain pre-emphasis \& de-emphasis circuit.}

\begin{solutionbox}

\textbf{Pre-emphasis and De-emphasis Circuits:}

\textbf{Purpose:}

\begin{itemize}
\tightlist
\item
  Improve SNR for high-frequency components
\item
  Compensate for higher noise in high frequencies
\item
  Used primarily in FM systems
\end{itemize}

\textbf{Pre-emphasis:}

\begin{itemize}
\tightlist
\item
  Applied at transmitter
\item
  Boosts high-frequency components
\item
  Typically +6dB/octave above 2.1kHz
\item
  Circuit: High-pass RC network (resistor in series, capacitor in
  parallel)
\end{itemize}

\textbf{De-emphasis:}

\begin{itemize}
\tightlist
\item
  Applied at receiver
\item
  Attenuates high-frequency components
\item
  Restores original signal balance
\item
  Circuit: Low-pass RC network (resistor in parallel, capacitor in
  series)
\end{itemize}

\textbf{Diagrams:}

\begin{verbatim}
Pre{-emphasis:                    De{-}emphasis:}
    R                                C
+{-{-}{-}www{-}{-}{-}+{-}{-}{-}+                 +{-}{-}{-}||{-}{-}{-}+{-}{-}{-}+}
|         |   |                 |        |   |
Vin       C   Vout              Vin      R   Vout
|         |   |                 |        |   |
+{-{-}{-}{-}{-}{-}{-}{-}{-}{-}   |                 +{-}{-}{-}{-}{-}{-}{-}{-}{-}   |}
             {-{-}{-}                            {-}{-}{-}}
              {-                              {-}}
\end{verbatim}

\textbf{Frequency response:}

\begin{verbatim}
    \^{}
    |        Pre{-emphasis}
Gain|          /
    |         /
    |        /
0dB |{-{-}{-}{-}{-}{-}{-}/}
    |      /       De{-emphasis}
    |     /          {}
    |    /            {}
    +{-{-}{-}{-}{-}{-}{-}{-}{-}{-}{-}{-}{-}{-}{-}{-}{-}{-}{-}{-}}
       2.1kHz           f
\end{verbatim}

\end{solutionbox}
\begin{mnemonicbox}
``HIGH-LOW'' - HIGHer frequencies boosted at
transmitter, LOWered at receiver

\end{mnemonicbox}
\subsection*{Question 2(c) OR [7
marks]}\label{q2c}

\textbf{Compare narrowband FM and wideband FM.}

\begin{solutionbox}

\textbf{Comparison of Narrowband FM and Wideband FM:}

{\def\LTcaptype{none} % do not increment counter
\begin{longtable}[]{@{}
  >{\raggedright\arraybackslash}p{(\linewidth - 4\tabcolsep) * \real{0.2821}}
  >{\raggedright\arraybackslash}p{(\linewidth - 4\tabcolsep) * \real{0.3846}}
  >{\raggedright\arraybackslash}p{(\linewidth - 4\tabcolsep) * \real{0.3333}}@{}}
\toprule\noalign{}
\begin{minipage}[b]{\linewidth}\raggedright
Parameter
\end{minipage} & \begin{minipage}[b]{\linewidth}\raggedright
Narrowband FM
\end{minipage} & \begin{minipage}[b]{\linewidth}\raggedright
Wideband FM
\end{minipage} \\
\midrule\noalign{}
\endhead
\bottomrule\noalign{}
\endlastfoot
\textbf{Modulation Index (β)} & β \textless\textless{} 1 (typically
\textless0.5) & β \textgreater\textgreater{} 1 (typically
\textgreater5) \\
\textbf{Bandwidth} & 2fm (twice message bandwidth) & 2fm(β+1) (Carson's
rule) \\
\textbf{Significant Sidebands} & Only first pair of sidebands & Multiple
sidebands \\
\textbf{Applications} & Mobile communication, two-way radio & FM
broadcasting, high-fidelity audio \\
\textbf{Signal Quality} & Lower fidelity, less noise immunity & Higher
fidelity, better noise immunity \\
\textbf{Power Efficiency} & Higher & Lower \\
\textbf{Spectrum Utilization} & Efficient & Less efficient \\
\textbf{Circuit Complexity} & Simpler & More complex \\
\end{longtable}
}

\textbf{Bandwidth calculation:}

\begin{itemize}
\tightlist
\item
  Narrowband FM: BW = 2fm
\item
  Wideband FM: BW = 2fm(β+1) (Carson's rule)
\end{itemize}

\textbf{Spectrum diagram:}

\begin{verbatim}
Narrowband FM:                    Wideband FM:
    \^{                                 \^{}}
    |                                 |
    |            |                    |
    |        |   |   |                |   | | | | | | | | | |
    |    |   |   |   |   |            | | | | | | | | | | | | | |
    +{-{-}{-}{-}{-}{-}{-}{-}{-}{-}{-}{-}{-}{-}{-}{-}{-}{-}{-}{-}{-}{-}          +{-}{-}{-}{-}{-}{-}{-}{-}{-}{-}{-}{-}{-}{-}{-}{-}{-}{-}{-}{-}{-}{-}{-}{-}{-}{-}}
       fc{-fm  fc  fc+fm                  fc{-}5fm    fc    fc+5fm}
\end{verbatim}

\end{solutionbox}
\begin{mnemonicbox}
``BASPCB'' - Bandwidth, Applications, Sidebands,
Power, Complexity, Beta

\end{mnemonicbox}
\subsection*{Question 3(a) [3 marks]}\label{q3a}

\textbf{Define any FOUR characteristics of radio receiver.}

\begin{solutionbox}

\textbf{Characteristics of Radio Receiver:}

\begin{enumerate}
\tightlist
\item
  \textbf{Sensitivity:}

  \begin{itemize}
  \tightlist
  \item
    Ability to amplify weak signals
  \item
    Measured in microvolts (μV)
  \item
    Typically 1-10μV for good receivers
  \end{itemize}
\item
  \textbf{Selectivity:}

  \begin{itemize}
  \tightlist
  \item
    Ability to separate desired signal from adjacent channels
  \item
    Determined by bandwidth of IF amplifier
  \item
    Measured in dB at specific frequency offsets
  \end{itemize}
\item
  \textbf{Fidelity:}

  \begin{itemize}
  \tightlist
  \item
    Accuracy in reproducing original signal
  \item
    Depends on bandwidth and distortion
  \item
    Measured as frequency response flatness
  \end{itemize}
\item
  \textbf{Image Frequency Rejection:}

  \begin{itemize}
  \tightlist
  \item
    Ability to reject signals at image frequency (fi = fs \pm 2fIF)
  \item
    Measured in dB
  \item
    Higher values indicate better performance
  \end{itemize}
\end{enumerate}

\textbf{Additional characteristics:}

\begin{itemize}
\tightlist
\item
  Signal-to-noise ratio (SNR)
\item
  Automatic gain control (AGC) range
\item
  Dynamic range
\end{itemize}

\end{solutionbox}
\begin{mnemonicbox}
``SFID'' - Sensitivity, Fidelity, Image rejection,
selectivity Determines quality

\end{mnemonicbox}
\subsection*{Question 3(b) [4 marks]}\label{q3b}

\textbf{Explain Diode Detector circuit.}

\begin{solutionbox}

\textbf{Diode Detector Circuit:}

\textbf{Purpose:}

\begin{itemize}
\tightlist
\item
  Extracts original message signal from AM wave
\item
  Also called envelope detector
\end{itemize}

\textbf{Circuit components:}

\begin{itemize}
\tightlist
\item
  Diode: Rectifies AM signal
\item
  RC network: Filters carrier frequency
\item
  R \& C values: RC \textgreater\textgreater{} 1/fc and RC
  \textless\textless{} 1/fm
\end{itemize}

\textbf{Operation:}

\begin{enumerate}
\tightlist
\item
  Diode conducts during positive half-cycles
\item
  Capacitor charges to peak value
\item
  Capacitor discharges through resistor
\item
  RC time constant critical for proper demodulation
\end{enumerate}

\textbf{Diagram:}

\begin{verbatim}
          D
        +{-{-}{-}{-}{-}||{-}{-}{-}{-}+}
        |            |
Input   |            C    R     Output
AM      |            |    |      
        +{-{-}{-}{-}{-}{-}{-}{-}{-}{-}{-}{-}+{-}{-}{-}{-}+{-}{-}{-}{-}{-}+}
                     |          |
                    {-{-}{-}        {-}{-}{-}}
                     {-          {-}}
\end{verbatim}

\textbf{Waveforms:}

\begin{verbatim}
Input:                  After Diode:            Output:
    /{      /              /      /              \_\_\_\_\_}
   /  {    /              /      /              /     }
  /    {  /             /      /             /       }
 /      {/              /      /      }
\end{verbatim}

\textbf{Limitations:}

\begin{itemize}
\tightlist
\item
  Distortion for high modulation index
\item
  Poor performance at low signal levels
\end{itemize}

\end{solutionbox}
\begin{mnemonicbox}
``DRCO'' - Diode Rectifies, Capacitor holds peaks,
Output follows envelope

\end{mnemonicbox}
\subsection*{Question 3(c) [7 marks]}\label{q3c}

\textbf{Draw and explain block diagram of super heterodyne receiver.}

\begin{solutionbox}

\textbf{Super Heterodyne Receiver:}

\textbf{Block Diagram:}

\begin{verbatim}
+{-{-}{-}{-}{-}{-}{-}{-}+    +{-}{-}{-}{-}{-}{-}{-}+    +{-}{-}{-}{-}{-}{-}+    +{-}{-}{-}{-}{-}+    +{-}{-}{-}{-}{-}{-}{-}{-}+    +{-}{-}{-}{-}{-}{-}{-}{-}+    +{-}{-}{-}{-}{-}{-}{-}{-}+}
| Antenna|{-{-}{-}| RF    |{-}{-}{-}| Mixer|{-}{-}{-}| IF  |{-}{-}{-}|Detector|{-}{-}{-}| Audio  |{-}{-}{-}|Speaker |}
|        |    | Amp   |    |      |    | Amp |    |        |    | Amp    |    |        |
+{-{-}{-}{-}{-}{-}{-}{-}+    +{-}{-}{-}{-}{-}{-}{-}+    +{-}{-}{-}{-}{-}{-}+    +{-}{-}{-}{-}{-}+    +{-}{-}{-}{-}{-}{-}{-}{-}+    +{-}{-}{-}{-}{-}{-}{-}{-}+    +{-}{-}{-}{-}{-}{-}{-}{-}+}
                              \^{}
                              |
                        +{-{-}{-}{-}{-}{-}{-}{-}{-}{-}{-}{-}+}
                        | Local      |
                        | Oscillator |
                        +{-{-}{-}{-}{-}{-}{-}{-}{-}{-}{-}{-}+}
\end{verbatim}

\textbf{Function of each block:}

\begin{enumerate}
\tightlist
\item
  \textbf{RF Amplifier:}

  \begin{itemize}
  \tightlist
  \item
    Amplifies weak RF signals
  \item
    Provides selectivity
  \item
    Improves signal-to-noise ratio
  \end{itemize}
\item
  \textbf{Local Oscillator:}

  \begin{itemize}
  \tightlist
  \item
    Generates stable frequency fLO
  \item
    fLO = fRF + fIF (for high-side injection)
  \item
    Tuned with RF amplifier
  \end{itemize}
\item
  \textbf{Mixer:}

  \begin{itemize}
  \tightlist
  \item
    Combines RF signal with local oscillator
  \item
    Produces sum and difference frequencies
  \item
    Difference frequency = IF (intermediate frequency)
  \end{itemize}
\item
  \textbf{IF Amplifier:}

  \begin{itemize}
  \tightlist
  \item
    Fixed frequency amplification (typically 455kHz for AM)
  \item
    Provides most of receiver gain and selectivity
  \item
    Multiple stages for better performance
  \end{itemize}
\item
  \textbf{Detector:}

  \begin{itemize}
  \tightlist
  \item
    Demodulates IF signal
  \item
    Extracts original message signal
  \item
    Diode detector for AM, discriminator for FM
  \end{itemize}
\item
  \textbf{Audio Amplifier:}

  \begin{itemize}
  \tightlist
  \item
    Amplifies demodulated signal
  \item
    Drives speaker or headphones
  \end{itemize}
\end{enumerate}

\textbf{Working principle:}

\begin{itemize}
\tightlist
\item
  Converts any RF frequency to fixed IF for efficient amplification
\item
  IF frequency = \textbar fRF - fLO\textbar{}
\end{itemize}

\textbf{Advantages:}

\begin{itemize}
\tightlist
\item
  Better selectivity and sensitivity
\item
  Stable gain at all frequencies
\item
  Reduced tracking problems
\end{itemize}

\end{solutionbox}
\begin{mnemonicbox}
``RLMIDS'' - RF amp, Local oscillator, Mixer, IF amp,
Detector, Speaker

\end{mnemonicbox}
\subsection*{Question 3(a) OR [3
marks]}\label{q3a}

\textbf{Describe AGC principle and its application in Radio receiver.}

\begin{solutionbox}

\textbf{AGC (Automatic Gain Control) Principle:}

\textbf{Definition:}

\begin{itemize}
\tightlist
\item
  Circuit that automatically adjusts receiver gain based on signal
  strength
\item
  Maintains constant output level despite varying input signals
\end{itemize}

\textbf{Working principle:}

\begin{enumerate}
\tightlist
\item
  Detects received signal strength
\item
  Generates control voltage proportional to signal
\item
  Applies negative feedback to reduce gain for strong signals
\item
  Increases gain for weak signals
\end{enumerate}

\textbf{Application in Radio Receiver:}

\begin{itemize}
\tightlist
\item
  \textbf{Prevents overloading:} Protects against strong signal
  distortion
\item
  \textbf{Compensates fading:} Maintains constant volume during signal
  fading
\item
  \textbf{Controls IF amplifier:} Primarily applied to IF stages
\item
  \textbf{Improves dynamic range:} Handles wide range of signal
  strengths
\end{itemize}

\textbf{Types:}

\begin{itemize}
\tightlist
\item
  \textbf{Simple AGC:} Direct feedback from detector
\item
  \textbf{Delayed AGC:} Only activates above threshold level
\item
  \textbf{Amplified AGC:} Uses additional amplifier for better control
\end{itemize}

\textbf{Diagram:}

\begin{verbatim}
     +{-{-}{-}{-}{-}{-}{-}+    +{-}{-}{-}{-}{-}{-}+    +{-}{-}{-}{-}{-}+    +{-}{-}{-}{-}{-}{-}{-}{-}+}
     | RF    |{-{-}{-}| Mixer|{-}{-}{-}| IF  |{-}{-}{-}|Detector|{-}{-}{-} Audio}
     | Amp   |    |      |    | Amp |    |        |
     +{-{-}{-}|{-}{-}{-}+    +{-}{-}{-}{-}{-}{-}+    +{-}|{-}{-}{-}+    +{-}{-}{-}{-}|{-}{-}{-}+}
         |                      |             |
         |                      |         +{-{-}{-}{-}{-}{-}{-}{-}+}
         |                      |{-{-}{-}{-}{-}{-}{-}{-}{-}|  AGC   |}
         |                                | Circuit|
         +{-{-}{-}{-}{-}{-}{-}{-}{-}{-}{-}{-}{-}{-}{-}{-}{-}{-}{-}{-}{-}{-}{-}{-}{-}{-}{-}{-}{-}{-}{-}{-}|        |}
                                          +{-{-}{-}{-}{-}{-}{-}{-}+}
\end{verbatim}

\end{solutionbox}
\begin{mnemonicbox}
``FADS'' - Fading compensation, Automatic adjustment,
Dynamic range, Signal consistency

\end{mnemonicbox}
\subsection*{Question 3(b) OR [4
marks]}\label{q3b}

\textbf{Write short-note on intermediate frequency}

\begin{solutionbox}

\textbf{Intermediate Frequency (IF):}

\textbf{Definition:}

\begin{itemize}
\tightlist
\item
  Fixed frequency to which incoming RF signal is converted in
  superheterodyne receivers
\item
  Result of mixing (heterodyning) RF signal with local oscillator
\end{itemize}

\textbf{Standard IF values:}

\begin{itemize}
\tightlist
\item
  \textbf{AM radio:} 455 kHz
\item
  \textbf{FM radio:} 10.7 MHz
\item
  \textbf{TV receivers:} 38-41 MHz
\end{itemize}

\textbf{Importance:}

\begin{itemize}
\tightlist
\item
  \textbf{Consistent gain:} Amplifiers operate at fixed frequency
\item
  \textbf{Better selectivity:} Narrowband filters at fixed frequency
\item
  \textbf{Simplified design:} Easier to design efficient fixed-frequency
  stages
\end{itemize}

\textbf{Selection criteria:}

\begin{itemize}
\tightlist
\item
  High enough to provide good image rejection
\item
  Low enough for practical filter Q and gain
\item
  Should avoid harmonics of common signals
\end{itemize}

\textbf{Image frequency calculation:}

\begin{itemize}
\tightlist
\item
  High-side injection: fimage = fRF + 2fIF
\item
  Low-side injection: fimage = fRF - 2fIF
\end{itemize}

\textbf{Diagram:}

\begin{verbatim}
   Original      IF Stage      Audio
   Spectrum        Fixed       Output
     |  |           |  |          |
     V  V           V  V          V
+{-{-}{-}{-}{-}{-}{-}{-}{-}{-}+    +{-}{-}{-}{-}{-}{-}{-}{-}{-}{-}+    +{-}{-}{-}{-}{-}+}
|  Mixer   |{-{-}{-}|    IF    |{-}{-}{-}| Det |}
+{-{-}{-}{-}{-}{-}{-}{-}{-}{-}+    +{-}{-}{-}{-}{-}{-}{-}{-}{-}{-}+    +{-}{-}{-}{-}{-}+}
      \^{}
      |
+{-{-}{-}{-}{-}{-}{-}{-}{-}{-}{-}{-}+}
|  Local     |
| Oscillator |
+{-{-}{-}{-}{-}{-}{-}{-}{-}{-}{-}{-}+}
\end{verbatim}

\end{solutionbox}
\begin{mnemonicbox}
``CIGS'' - Conversion, Improved selectivity, Gain
stability, Simplified design

\end{mnemonicbox}
\subsection*{Question 3(c) OR [7
marks]}\label{q3c}

\textbf{Explain phase discriminator circuit for FM detection.}

\begin{solutionbox}

\textbf{Phase Discriminator for FM Detection:}

\textbf{Purpose:}

\begin{itemize}
\tightlist
\item
  Converts frequency variations in FM signal to amplitude variations
\item
  Demodulates FM signal to recover original message
\end{itemize}

\textbf{Circuit components:}

\begin{itemize}
\tightlist
\item
  Center-tapped transformer
\item
  Two diodes (D1 and D2)
\item
  RC filter network
\item
  Phase-shifting network (L-C circuit)
\end{itemize}

\textbf{Working principle:}

\begin{enumerate}
\tightlist
\item
  Input FM signal splits into two paths
\item
  Reference path goes directly to center tap
\item
  Phase-shifted path passes through LC network
\item
  Phase shift varies with frequency deviation
\item
  Two diodes produce voltages proportional to phase difference
\item
  Output voltage varies with input frequency
\end{enumerate}

\textbf{Circuit diagram:}

\begin{verbatim}
                      D1
              +{-{-}{-}{-}{-}{-}||{-}{-}{-}{-}{-}{-}+}
              |               |
              |               R1
              |               |
              |               |
 FM Input     |               +{-{-}{-}+}
 +{-{-}{-}{-}{-}{-}+     |                   |}
 |      |{-{-}{-}{-}{-}+                   +{-}{-}{-} Output}
 +{-{-}{-}{-}{-}{-}+     |                   |}
              |               +{-{-}{-}+}
              |               |
              |               R2
              |               |
              +{-{-}{-}{-}{-}{-}||{-}{-}{-}{-}{-}{-}+}
                      D2
\end{verbatim}

\textbf{Characteristics:}

\begin{itemize}
\tightlist
\item
  \textbf{Linear response} over moderate frequency range
\item
  \textbf{Balanced design} reduces amplitude variations
\item
  \textbf{High sensitivity} to frequency changes
\item
  \textbf{Limitations} at extreme frequency deviations
\end{itemize}

\textbf{S-curve response:}

\begin{verbatim}
    \^{}
    |              /
    |             /
    |            /
 0V +{-{-}{-}{-}{-}{-}{-}{-}{-}{-}{-}{-}}
    |          /
    |         /
    |        /
    +{-{-}{-}{-}{-}{-}{-}{-}{-}{-}{-}{-}{-}{-}{-}{-}{-}}
       fc{-Δf  fc  fc+Δf}
\end{verbatim}

\end{solutionbox}
\begin{mnemonicbox}
``PSDO'' - Phase shift Demodulates, Signal frequency
determines Output

\end{mnemonicbox}
\subsection*{Question 4(a) [3 marks]}\label{q4a}

\textbf{Compare analog and digital communication techniques}

\begin{solutionbox}

\textbf{Comparison of Analog vs.~Digital Communication:}

{\def\LTcaptype{none} % do not increment counter
\begin{longtable}[]{@{}
  >{\raggedright\arraybackslash}p{(\linewidth - 4\tabcolsep) * \real{0.1930}}
  >{\raggedright\arraybackslash}p{(\linewidth - 4\tabcolsep) * \real{0.3860}}
  >{\raggedright\arraybackslash}p{(\linewidth - 4\tabcolsep) * \real{0.4211}}@{}}
\toprule\noalign{}
\begin{minipage}[b]{\linewidth}\raggedright
Parameter
\end{minipage} & \begin{minipage}[b]{\linewidth}\raggedright
Analog Communication
\end{minipage} & \begin{minipage}[b]{\linewidth}\raggedright
Digital Communication
\end{minipage} \\
\midrule\noalign{}
\endhead
\bottomrule\noalign{}
\endlastfoot
\textbf{Signal} & Continuous waveform & Discrete binary values \\
\textbf{Bandwidth} & Less bandwidth required & More bandwidth
required \\
\textbf{Noise Immunity} & Poor, noise accumulates & Excellent, error
correction possible \\
\textbf{Power Efficiency} & Less efficient & More efficient \\
\textbf{Quality} & Degrades with distance & Maintains quality until SNR
threshold \\
\textbf{Multiplexing} & FDM primarily used & TDM primarily used \\
\textbf{System Complexity} & Simpler & More complex \\
\textbf{Cost} & Lower & Higher but decreasing \\
\textbf{Examples} & AM/FM radio, analog TV & Mobile networks, digital
TV, internet \\
\end{longtable}
}

\end{solutionbox}
\begin{mnemonicbox}
``BNPQ MCE'' - Bandwidth, Noise immunity, Power,
Quality, Multiplexing, Complexity, Efficiency

\end{mnemonicbox}
\subsection*{Question 4(b) [4 marks]}\label{q4b}

\textbf{Explain Adaptive delta modulation with its application.}

\begin{solutionbox}

\textbf{Adaptive Delta Modulation (ADM):}

\textbf{Definition:}

\begin{itemize}
\tightlist
\item
  Improved version of Delta Modulation (DM)
\item
  Uses variable step size adjusted to signal slope
\end{itemize}

\textbf{Working principle:}

\begin{enumerate}
\tightlist
\item
  Compares input signal with predicted value
\item
  Outputs binary 1 or 0 based on comparison
\item
  Adjusts step size based on consecutive bits
\item
  Increases step size for rapid changes
\item
  Decreases step size for slow changes
\end{enumerate}

\textbf{Advantages over Delta Modulation:}

\begin{itemize}
\tightlist
\item
  Reduces slope overload distortion
\item
  Minimizes granular noise
\item
  Better dynamic range
\item
  Lower bit rate for same quality
\end{itemize}

\textbf{Diagram:}

\begin{verbatim}
                        +{-{-}{-}{-}{-}{-}{-}+}
                        |       |
                        | Step  |{{-}{-}+}
Input                   | Size  |   |
  +{-{-}{-}+    +{-}{-}{-}+        | Logic |   |}
  |   |{-{-}{-}|+/{-}|{-}{-}{-}{-}{-}{-}{-}|       |   |}
  +{-{-}{-}+    +{-}{-}{-}+        +{-}{-}{-}{-}{-}{-}{-}+   |}
    \^{        |              |       |}
    |        |              V       |
    |      +{-{-}{-}+         +{-}{-}{-}+      |}
    +{-{-}{-}{-}{-}{-}|   |{-}{-}{-}{-}{-}{-}{-}{-}| Δ |{-}{-}{-}{-}{-}{-}+}
           +{-{-}{-}+         +{-}{-}{-}+}
           Integrator
\end{verbatim}

\textbf{Applications:}

\begin{itemize}
\tightlist
\item
  \textbf{Speech transmission:} Voice over digital networks
\item
  \textbf{Audio compression:} Music storage and transmission
\item
  \textbf{Telemetry systems:} Remote data collection
\item
  \textbf{Military communications:} Secure transmission
\end{itemize}

\end{solutionbox}
\begin{mnemonicbox}
``VSOG'' - Variable Step size Overcomes Granular
noise \& slope overload

\end{mnemonicbox}
\subsection*{Question 4(c) [7 marks]}\label{q4c}

\textbf{Draw \& explain block diagram of PCM system.}

\begin{solutionbox}

\textbf{Pulse Code Modulation (PCM) System:}

\textbf{Block Diagram:}

\begin{verbatim}
                  +{-{-}{-}{-}{-}{-}{-}+    +{-}{-}{-}{-}{-}{-}{-}{-}{-}{-}+    +{-}{-}{-}{-}{-}{-}{-}{-}{-}+    +{-}{-}{-}{-}{-}{-}{-}{-}+}
                  |       |    |          |    |         |    |        |
Input signal {-{-}{-}{-}|Sample |{-}{-}{-}|Quantizer |{-}{-}{-}|Encoder  |{-}{-}{-}|Channel |}
                  |\& Hold |    |          |    |         |    |        |
                  +{-{-}{-}{-}{-}{-}{-}+    +{-}{-}{-}{-}{-}{-}{-}{-}{-}{-}+    +{-}{-}{-}{-}{-}{-}{-}{-}{-}+    +{-}{-}{-}{-}{-}{-}{-}{-}+}
                                                                  |
                                                                  V
                  +{-{-}{-}{-}{-}{-}{-}{-}+    +{-}{-}{-}{-}{-}{-}{-}{-}{-}+    +{-}{-}{-}{-}{-}{-}{-}{-}{-}+    +{-}{-}{-}{-}{-}{-}{-}{-}+}
                  |        |    |         |    |         |    |        |
Output signal {{-}{-}{-}|Low Pass|{-}{-}{-}| DAC     |{-}{-}{-}|Decoder  |{-}{-}{-}| Buffer |}
                  |Filter  |    |         |    |         |    |        |
                  +{-{-}{-}{-}{-}{-}{-}{-}+    +{-}{-}{-}{-}{-}{-}{-}{-}{-}+    +{-}{-}{-}{-}{-}{-}{-}{-}{-}+    +{-}{-}{-}{-}{-}{-}{-}{-}+}
\end{verbatim}

\textbf{Transmitter components:}

\begin{enumerate}
\tightlist
\item
  \textbf{Sample \& Hold:}

  \begin{itemize}
  \tightlist
  \item
    Samples analog signal at regular intervals
  \item
    Nyquist rate (fs \geq 2fmax)
  \item
    Holds value until next sample
  \end{itemize}
\item
  \textbf{Quantizer:}

  \begin{itemize}
  \tightlist
  \item
    Divides amplitude range into discrete levels
  \item
    Maps each sample to nearest level
  \item
    Introduces quantization error
  \end{itemize}
\item
  \textbf{Encoder:}

  \begin{itemize}
  \tightlist
  \item
    Converts quantized levels to binary code
  \item
    n-bit encoder gives 2\^{}n quantization levels
  \item
    Common formats: 8-bit, 16-bit
  \end{itemize}
\end{enumerate}

\textbf{Receiver components:}

\begin{enumerate}
\tightlist
\item
  \textbf{Decoder:}

  \begin{itemize}
  \tightlist
  \item
    Converts binary to quantized levels
  \item
    Reverses encoder operation
  \end{itemize}
\item
  \textbf{Digital-to-Analog Converter (DAC):}

  \begin{itemize}
  \tightlist
  \item
    Converts discrete levels to analog values
  \item
    Produces staircase approximation of signal
  \end{itemize}
\item
  \textbf{Low-Pass Filter:}

  \begin{itemize}
  \tightlist
  \item
    Smooths staircase output
  \item
    Removes high-frequency components
  \item
    Reconstructs original waveform
  \end{itemize}
\end{enumerate}

\textbf{Key characteristics:}

\begin{itemize}
\tightlist
\item
  Sampling rate: Typically 8 kHz (voice), 44.1 kHz (CD audio)
\item
  Resolution: 8-bit (256 levels) to 24-bit (16.8M levels)
\item
  Bit rate = Sampling rate \times bits per sample
\end{itemize}

\end{solutionbox}
\begin{mnemonicbox}
``SQEC-DFL'' - Sample, Quantize, Encode, Channel -
Decode, Filter, Listen

\end{mnemonicbox}
\subsection*{Question 4(a) OR [3
marks]}\label{q4a}

\textbf{Explain quantization process and its necessity.}

\begin{solutionbox}

\textbf{Quantization Process and its Necessity:}

\textbf{Definition:}

\begin{itemize}
\tightlist
\item
  Process of mapping continuous amplitude values to discrete levels
\item
  Second step in analog-to-digital conversion after sampling
\end{itemize}

\textbf{Process:}

\begin{enumerate}
\tightlist
\item
  Divide amplitude range into finite number of levels
\item
  Assign each sample to nearest quantization level
\item
  Represent each level with binary code
\item
  Quantization levels = 2\^{}n (n = number of bits)
\end{enumerate}

\textbf{Types:}

\begin{itemize}
\tightlist
\item
  \textbf{Uniform quantization:} Equal step size throughout range
\item
  \textbf{Non-uniform quantization:} Variable step size (smaller for
  lower amplitudes)
\item
  \textbf{Mid-tread quantization:} Zero is a valid level
\item
  \textbf{Mid-rise quantization:} Zero falls between levels
\end{itemize}

\textbf{Necessity:}

\begin{itemize}
\tightlist
\item
  \textbf{Digital representation:} Enables conversion to binary format
\item
  \textbf{Storage efficiency:} Allows finite storage of analog signals
\item
  \textbf{Processing capability:} Enables digital signal processing
\item
  \textbf{Transmission benefits:} Facilitates error correction and
  encryption
\end{itemize}

\textbf{Quantization error:}

\begin{itemize}
\tightlist
\item
  Difference between actual and quantized value
\item
Maximum error = \pmQ/2 (where

Q = step size)

\item
  Signal-to-quantization-noise ratio: SQNR = 6.02n + 1.76 dB
\end{itemize}

\textbf{Diagram:}

\begin{verbatim}
  \^{}
  |                    Quantized
  |   Original         Output
  |    Signal          /|
  |      /{           / |}
  |     /  {         /  |}
  |    /    {       /   |}
  |   /      {     /    |}
  |  /        {   /     |}
  | /          { /      |}
  +{-{-}{-}{-}{-}{-}{-}{-}{-}{-}{-}{-}{-}{-}{-}{-}{-}{-}{-}{-}{-}{-}{-}{-}{-}{-}{-}}
                Time
\end{verbatim}

\end{solutionbox}
\begin{mnemonicbox}
``DEBS'' - Digitization Enables Binary Storage

\end{mnemonicbox}
\subsection*{Question 4(b) OR [4
marks]}\label{q4b}

\textbf{Explain PCM receiver.}

\begin{solutionbox}

\textbf{PCM Receiver:}

\textbf{Block Diagram:}

\begin{verbatim}
                  +{-{-}{-}{-}{-}{-}{-}{-}+    +{-}{-}{-}{-}{-}{-}{-}{-}{-}+    +{-}{-}{-}{-}{-}{-}{-}{-}{-}+    +{-}{-}{-}{-}{-}{-}{-}{-}+}
                  |        |    |         |    |         |    |        |
Digital PCM  {-{-}{-}{-}| Buffer |{-}{-}{-}| Decoder |{-}{-}{-}|   DAC   |{-}{-}{-}|Low Pass|{-}{-}{-} Output Signal}
  Input           |        |    |         |    |         |    | Filter |
                  +{-{-}{-}{-}{-}{-}{-}{-}+    +{-}{-}{-}{-}{-}{-}{-}{-}{-}+    +{-}{-}{-}{-}{-}{-}{-}{-}{-}+    +{-}{-}{-}{-}{-}{-}{-}{-}+}
\end{verbatim}

\textbf{Components and their functions:}

\begin{enumerate}
\tightlist
\item
  \textbf{Buffer:}

  \begin{itemize}
  \tightlist
  \item
    Temporarily stores received PCM data
  \item
    Compensates for timing variations
  \item
    Provides protection against jitter
  \end{itemize}
\item
  \textbf{Decoder:}

  \begin{itemize}
  \tightlist
  \item
    Converts binary code to quantized amplitude levels
  \item
    Detects and corrects transmission errors (if error coding used)
  \item
    Outputs discrete amplitude values
  \end{itemize}
\item
  \textbf{Digital-to-Analog Converter (DAC):}

  \begin{itemize}
  \tightlist
  \item
    Converts digital values to analog voltage levels
  \item
    Creates staircase approximation of original signal
  \item
    Resolution determined by bit depth (2\^{}n levels)
  \end{itemize}
\item
  \textbf{Low-Pass Filter:}

  \begin{itemize}
  \tightlist
  \item
    Smooths the staircase waveform
  \item
    Removes high-frequency components
  \item
    Reconstructs continuous analog signal
  \end{itemize}
\end{enumerate}

\textbf{Waveforms in PCM Receiver:}

\begin{verbatim}
Digital Input      Decoded Values       DAC Output          Final Output
 1001              {-{-}{-}{-}                  \_                    /}
 0110              {-  {-}                \_| |\_                 /  }
 1010             {-{-} {-}              \_|   |\_              /    }
 0101              {- {-} {-}             \_|     |\_             /      }
\end{verbatim}

\textbf{Performance factors:}

\begin{itemize}
\tightlist
\item
  \textbf{SNR:} Determined by quantization bits (6.02n + 1.76 dB)
\item
  \textbf{Bandwidth:} Depends on sampling rate and filter
  characteristics
\item
  \textbf{Distortion:} Related to quantization error
\end{itemize}

\end{solutionbox}
\begin{mnemonicbox}
``BDFL'' - Buffer stores, Decoder converts, Filter
smooths, Listen to output

\end{mnemonicbox}
\subsection*{Question 4(c) OR [7
marks]}\label{q4c}

\textbf{What is sampling? Explain types of sampling in brief.}

\begin{solutionbox}

\textbf{Sampling:}

\textbf{Definition:} Sampling is the process of converting a
continuous-time signal into a discrete-time signal by taking
measurements (samples) at regular time intervals.

\textbf{Mathematical expression:} x[n] = x(nTs), where n = 0, 1,
2\ldots{}

\begin{itemize}
\tightlist
\item
  x[n] is discrete-time sample
\item
  x(t) is continuous-time signal
\item
  Ts is sampling period (1/fs)
\end{itemize}

\textbf{Nyquist Theorem:}

\begin{itemize}
\tightlist
\item
  Sampling frequency (fs) must be at least twice the highest frequency
  component (fmax) in the signal
\item
  fs \geq 2fmax
\item
  Prevents aliasing (distortion due to overlap of spectrum)
\end{itemize}

\textbf{Types of Sampling:}

{\def\LTcaptype{none} % do not increment counter
\begin{longtable}[]{@{}
  >{\raggedright\arraybackslash}p{(\linewidth - 4\tabcolsep) * \real{0.1667}}
  >{\raggedright\arraybackslash}p{(\linewidth - 4\tabcolsep) * \real{0.3611}}
  >{\raggedright\arraybackslash}p{(\linewidth - 4\tabcolsep) * \real{0.4722}}@{}}
\toprule\noalign{}
\begin{minipage}[b]{\linewidth}\raggedright
Type
\end{minipage} & \begin{minipage}[b]{\linewidth}\raggedright
Description
\end{minipage} & \begin{minipage}[b]{\linewidth}\raggedright
Characteristics
\end{minipage} \\
\midrule\noalign{}
\endhead
\bottomrule\noalign{}
\endlastfoot
\textbf{Ideal Sampling} & Instantaneous samples at regular intervals & -
Theoretical concept- Represented by impulse train- Infinite bandwidth
required \\
\textbf{Natural Sampling} & Signal multiplied by pulse train with finite
width & - Samples have same shape as signal- Width determined by
sampling pulse- Used in analog systems \\
\textbf{Flat-Top Sampling} & Sample-and-hold technique & - Holds sampled
value until next sample- Creates staircase approximation- Common in
practical systems \\
\end{longtable}
}

\textbf{Sampling Rates:}

\begin{itemize}
\tightlist
\item
  \textbf{Under-sampling:} fs \textless{} 2fmax (causes aliasing)
\item
  \textbf{Critical sampling:} fs = 2fmax (minimum required rate)
\item
  \textbf{Over-sampling:} fs \textgreater{} 2fmax (improves
  reconstruction quality)
\end{itemize}

\textbf{Diagram:}

\begin{verbatim}
Original Signal:     /{///////}

Ideal Sampling:      |  |  |  |  |  |

Natural Sampling:    ▓  ▓  ▓  ▓  ▓  ▓

Flat{-top Sampling:   ▔▔  ▔▔  ▔▔  ▔▔  ▔▔}
\end{verbatim}

\end{solutionbox}
\begin{mnemonicbox}
``INF'' - Ideal (impulses), Natural (pulse-shaped),
Flat-top (staircase)

\end{mnemonicbox}
\subsection*{Question 5(a) [3 marks]}\label{q5a}

\textbf{List the need of Multiplexing.}

\begin{solutionbox}

\textbf{Need for Multiplexing:}

{\def\LTcaptype{none} % do not increment counter
\begin{longtable}[]{@{}
  >{\raggedright\arraybackslash}p{(\linewidth - 2\tabcolsep) * \real{0.3158}}
  >{\raggedright\arraybackslash}p{(\linewidth - 2\tabcolsep) * \real{0.6842}}@{}}
\toprule\noalign{}
\begin{minipage}[b]{\linewidth}\raggedright
Need
\end{minipage} & \begin{minipage}[b]{\linewidth}\raggedright
Description
\end{minipage} \\
\midrule\noalign{}
\endhead
\bottomrule\noalign{}
\endlastfoot
\textbf{Bandwidth Utilization} & Efficiently uses available transmission
bandwidth \\
\textbf{Cost Reduction} & Shares expensive transmission medium among
multiple users \\
\textbf{Infrastructure Optimization} & Reduces physical connections and
hardware requirements \\
\textbf{Spectrum Efficiency} & Maximizes use of limited frequency
spectrum \\
\textbf{Network Capacity} & Increases number of channels/users on single
medium \\
\textbf{Flexibility} & Allows dynamic allocation of resources based on
demand \\
\end{longtable}
}

\end{solutionbox}
\begin{mnemonicbox}
``BCSINF'' - Bandwidth, Cost, Spectrum,
Infrastructure, Network capacity, Flexibility

\end{mnemonicbox}
\subsection*{Question 5(b) [4 marks]}\label{q5b}

\textbf{Explain working of DPCM.}

\begin{solutionbox}

\textbf{Differential Pulse Code Modulation (DPCM):}

\textbf{Definition:}

\begin{itemize}
\tightlist
\item
  Enhanced version of PCM that encodes difference between current and
  predicted sample
\item
  Exploits correlation between adjacent samples to reduce bit rate
\end{itemize}

\textbf{Block Diagram:}

\begin{verbatim}
                  +{-{-}{-}{-}{-}{-}+     +{-}{-}{-}{-}{-}{-}{-}{-}{-}{-}+    +{-}{-}{-}{-}{-}{-}{-}{-}{-}+}
                  |      |     |          |    |         |
Input signal {-{-}{-}{-}| ADC  |{-}{-}+{-}|Quantizer |{-}{-}{-}|Encoder  |{-}{-}{-} DPCM Output}
                  |      |  |  |          |    |         |
                  +{-{-}{-}{-}{-}{-}+  |  +{-}{-}{-}{-}{-}{-}{-}{-}{-}{-}+    +{-}{-}{-}{-}{-}{-}{-}{-}{-}+}
                            |        \^{}
                            |        |
                            v        |
                        +{-{-}{-}{-}{-}{-}{-}{-}{-}+  |}
                        |Predictor|{-{-}+}
                        +{-{-}{-}{-}{-}{-}{-}{-}{-}+}
\end{verbatim}

\textbf{Working principle:}

\begin{enumerate}
\tightlist
\item
  Current sample is predicted based on previous sample(s)
\item
  Only the difference (error) between actual and predicted value is
  encoded
\item
  Smaller difference requires fewer bits than full amplitude
\item
  Predictor uses previous reconstructed values for prediction
\end{enumerate}

\textbf{Advantages:}

\begin{itemize}
\tightlist
\item
  \textbf{Reduced bit rate:} Typically 25-50\% lower than PCM
\item
  \textbf{Better SNR:} For same bit rate as PCM
\item
  \textbf{Correlation utilization:} Exploits signal redundancy
\end{itemize}

\textbf{Limitations:}

\begin{itemize}
\tightlist
\item
  \textbf{Error propagation:} Errors affect subsequent samples
\item
  \textbf{Complexity:} More complex than simple PCM
\item
  \textbf{Signal dependency:} Performance varies with signal
  characteristics
\end{itemize}

\end{solutionbox}
\begin{mnemonicbox}
``PDQE'' - Predict sample, Difference calculated,
Quantize error, Encode result

\end{mnemonicbox}
\subsection*{Question 5(c) [7 marks]}\label{q5c}

\textbf{The binary data 1011001 is to be transmitted using following
line coding techniques: (i) Unipolar RZ and NRZ (ii) Polar RZ and NRZ
(iii) AMI (iv) Manchester. Draw all the waveforms.}

\begin{solutionbox}

\textbf{Line Coding of Binary Data: 1011001}

\textbf{Waveforms:}

\begin{verbatim}
Binary Data:   1   0   1   1   0   0   1
              \_   \_   \_   \_   \_   \_   \_

1. Unipolar NRZ:
              ▔▔▔   ▔▔▔▔▔▔   ▔▔▔
              \_\_\_ ▔▔▔ \_\_\_\_\_\_\_ ▔▔▔

2. Unipolar RZ:
              ▔ \_ ▔ \_ ▔ ▔ \_ \_ \_ ▔ \_
              \_ \_ \_ \_ \_ \_ \_ \_ \_ \_ \_ \_

3. Polar NRZ:
              ▔▔▔   ▔▔▔▔▔▔   ▔▔▔
              \_\_\_ ▔▔▔ \_\_\_\_\_\_\_ ▔▔▔

4. Polar RZ:
              ▔ \_ \_ \_ ▔ ▔ \_ \_ \_ ▔ \_
              \_ ▔ \_ \_ \_ \_ ▔ ▔ \_ \_ \_

5. AMI:
              ▔ \_   ▔ \_ \_ \_ ▔ \_
              \_ \_ ▔ \_ \_ \_ \_ \_ \_ \_ \_

6. Manchester:
              ▔▁ ▁▔ ▔▁ ▔▁ ▁▔ ▁▔ ▔▁
              \_ \_ \_ \_ \_ \_ \_ \_ \_ \_ \_ \_
\end{verbatim}

\textbf{Characteristics of Each Coding:}

{\def\LTcaptype{none} % do not increment counter
\begin{longtable}[]{@{}
  >{\raggedright\arraybackslash}p{(\linewidth - 6\tabcolsep) * \real{0.3103}}
  >{\raggedright\arraybackslash}p{(\linewidth - 6\tabcolsep) * \real{0.2241}}
  >{\raggedright\arraybackslash}p{(\linewidth - 6\tabcolsep) * \real{0.2069}}
  >{\raggedright\arraybackslash}p{(\linewidth - 6\tabcolsep) * \real{0.2586}}@{}}
\toprule\noalign{}
\begin{minipage}[b]{\linewidth}\raggedright
Coding Technique
\end{minipage} & \begin{minipage}[b]{\linewidth}\raggedright
Description
\end{minipage} & \begin{minipage}[b]{\linewidth}\raggedright
Advantages
\end{minipage} & \begin{minipage}[b]{\linewidth}\raggedright
Disadvantages
\end{minipage} \\
\midrule\noalign{}
\endhead
\bottomrule\noalign{}
\endlastfoot
\textbf{Unipolar NRZ} & 1 = high voltage0 = zero voltageNo return to
zero & Simple implementation & DC component, no clock recovery \\
\textbf{Unipolar RZ} & 1 = high for half bit0 = zero voltageReturns to
zero & Self-clocking & Requires more bandwidth \\
\textbf{Polar NRZ} & 1 = positive voltage0 = negative voltageNo return
to zero & No DC component & Poor clock recovery \\
\textbf{Polar RZ} & 1 = positive for half bit0 = negative for half
bitReturns to zero & Self-clocking, no DC component & Requires more
bandwidth \\
\textbf{AMI} & 1 = alternating +/- voltage0 = zero voltage & No DC
component, error detection & Long strings of zeros problematic \\
\textbf{Manchester} & 1 = transition low to high0 = transition high to
low & Self-clocking, no DC component & Requires double bandwidth \\
\end{longtable}
}

\end{solutionbox}
\begin{mnemonicbox}
``UPRMA'' - Unipolar, Polar, Return-to-zero,
Manchester, AMI line coding techniques

\end{mnemonicbox}
\subsection*{Question 5(a) OR [3
marks]}\label{q5a}

\textbf{Explain polar RZ and NRZ format}

\begin{solutionbox}

\textbf{Polar RZ and NRZ Line Coding:}

\textbf{Polar NRZ (Non-Return to Zero):}

\begin{itemize}
\tightlist
\item
  Binary 1: Positive voltage (+V) for entire bit duration
\item
  Binary 0: Negative voltage (-V) for entire bit duration
\item
  Signal remains at level during entire bit period
\item
  No transition to zero between consecutive similar bits
\end{itemize}

\textbf{Characteristics of Polar NRZ:}

\begin{itemize}
\tightlist
\item
  \textbf{Bandwidth efficiency:} Requires minimum bandwidth
\item
  \textbf{DC component:} Zero average for equal 1s and 0s
\item
  \textbf{Clock recovery:} Poor for long sequences of same bit
\item
  \textbf{Error detection:} No inherent capability
\end{itemize}

\textbf{Polar RZ (Return to Zero):}

\begin{itemize}
\tightlist
\item
  Binary 1: Positive voltage (+V) for half bit, zero for remainder
\item
  Binary 0: Negative voltage (-V) for half bit, zero for remainder
\item
  Signal returns to zero during each bit period
\end{itemize}

\textbf{Characteristics of Polar RZ:}

\begin{itemize}
\tightlist
\item
  \textbf{Bandwidth:} Requires twice the bandwidth of NRZ
\item
  \textbf{Self-clocking:} Better clock recovery
\item
  \textbf{Power requirement:} Higher than NRZ
\item
  \textbf{Error detection:} No inherent capability
\end{itemize}

\textbf{Waveform Comparison:}

\begin{verbatim}
Binary Data:   1   0   1   1   0   0   1
              \_   \_   \_   \_   \_   \_   \_

Polar NRZ:    ▔▔▔   ▔▔▔▔▔▔   ▔▔▔
              \_\_\_ ▔▔▔ \_\_\_\_\_\_\_ ▔▔▔

Polar RZ:     ▔ \_ \_ \_ ▔ ▔ \_ \_ \_ ▔ \_
              \_ ▔ \_ \_ \_ \_ ▔ ▔ \_ \_ \_
\end{verbatim}

\end{solutionbox}
\begin{mnemonicbox}
``HZRT'' - Half bit active + Zero Return in RZ, full
Time in NRZ

\end{mnemonicbox}
\subsection*{Question 5(b) OR [4
marks]}\label{q5b}

\textbf{Explain delta modulation in brief.}

\begin{solutionbox}

\textbf{Delta Modulation (DM):}

\textbf{Definition:}

\begin{itemize}
\tightlist
\item
  Simplest form of differential encoding
\item
  Encodes only the sign of difference between current and previous
  sample
\item
  Single bit per sample for transmission (1 or 0)
\end{itemize}

\textbf{Block Diagram:}

\begin{verbatim}
                    +{-{-}{-}{-}{-}+       Encoded}
 Input      +{-{-}{-}+   |     |      Bitstream}
 Signal {-{-}{-}|+/{-}|{-}{-}{-}  C  |{-}{-}{-}{-}{-}{-}{-}{-}{-}{-}}
            +{-{-}{-}+   |     |}
              \^{     +{-}{-}{-}{-}{-}+}
              |        |
              |        v
            +{-{-}{-}+    +{-}{-}{-}+}
            |   |{{-}{-}{-}|+/{-}|}
            +{-{-}{-}+    +{-}{-}{-}+}
          Integrator   Step Size
\end{verbatim}

\textbf{Working principle:}

\begin{enumerate}
\tightlist
\item
  Compare input signal with predicted value (from integrator)
\item
  If input \textgreater{} predicted: Output = 1, increase predicted
  value
\item
  If input \textless{} predicted: Output = 0, decrease predicted value
\item
  Step size determines how much predicted value changes
\end{enumerate}

\textbf{Advantages:}

\begin{itemize}
\tightlist
\item
  \textbf{Simple implementation:} Minimal hardware
\item
  \textbf{Low bit rate:} 1 bit per sample
\item
  \textbf{Robust:} Relatively immune to channel noise
\end{itemize}

\textbf{Limitations:}

\begin{itemize}
\tightlist
\item
  \textbf{Slope overload:} Cannot track rapid signal changes
\item
  \textbf{Granular noise:} Oscillations around steady signals
\item
  \textbf{Limited resolution:} Quality depends on step size and sampling
  rate
\end{itemize}

\textbf{Waveforms:}

\begin{verbatim}
Original:      /{///}

Reconstructed: /{///}
               (Staircase approximation)

Binary output: 1101001011
\end{verbatim}

\end{solutionbox}
\begin{mnemonicbox}
``1BSG'' - 1 Bit per Sample, Slope overload and
Granular noise limitations

\end{mnemonicbox}
\subsection*{Question 5(c) OR [7
marks]}\label{q5c}

\textbf{Explain PCM-TDM system.}

\begin{solutionbox}

\textbf{PCM-TDM System:}

\textbf{Definition:}

\begin{itemize}
\tightlist
\item
  Combined system using Pulse Code Modulation (PCM) with Time Division
  Multiplexing (TDM)
\item
  Multiple analog channels converted to digital PCM, then multiplexed in
  time
\end{itemize}

\textbf{Block Diagram:}

\begin{verbatim}
                 +{-{-}{-}{-}{-}{-}{-}+     +{-}{-}{-}{-}{-}{-}{-}{-}+     +{-}{-}{-}{-}{-}{-}{-}{-}{-}+}
 Channel 1 {-{-}{-}{-}{-}| PCM 1 |{-}{-}{-}{-}|        |     |         |}
                 +{-{-}{-}{-}{-}{-}{-}+     |        |     |         |}
                 +{-{-}{-}{-}{-}{-}{-}+     |        |     |         |     Multiplexed}
 Channel 2 {-{-}{-}{-}{-}| PCM 2 |{-}{-}{-}{-}|  Time  |{-}{-}{-}{-}|  Frame  |{-}{-}{-} PCM{-}TDM}
                 +{-{-}{-}{-}{-}{-}{-}+     |        |     | Format  |     Output}
                               |  MUX   |     |         |
                 +{-{-}{-}{-}{-}{-}{-}+     |        |     |         |}
 Channel N {-{-}{-}{-}{-}| PCM N |{-}{-}{-}{-}|        |     |         |}
                 +{-{-}{-}{-}{-}{-}{-}+     +{-}{-}{-}{-}{-}{-}{-}{-}+     +{-}{-}{-}{-}{-}{-}{-}{-}{-}+}
\end{verbatim}

\textbf{PCM Process for Each Channel:}

\begin{enumerate}
\tightlist
\item
  \textbf{Sampling:} Each channel sampled at fs \geq 2fmax
\item
  \textbf{Quantization:} Samples assigned to discrete levels
\item
  \textbf{Encoding:} Quantized values converted to binary code
\end{enumerate}

\textbf{TDM Frame Structure:}

\begin{itemize}
\tightlist
\item
  Frame consists of one sample from each channel
\item
  Frame includes synchronization bits/word
\item
  Frame rate equals sampling rate (fs)
\item
Bit rate = fs \times N \times n (N = channels,

n = bits/sample)

\end{itemize}

\textbf{Typical Parameters:}

\begin{itemize}
\tightlist
\item
  \textbf{Voice channels:} 8 kHz sampling, 8 bits/sample
\item
  \textbf{T1 system:} 24 channels, 1.544 Mbps
\item
  \textbf{E1 system:} 30 channels, 2.048 Mbps
\end{itemize}

\textbf{Advantages:}

\begin{itemize}
\tightlist
\item
  \textbf{Efficient transmission:} Single high-speed link
\item
  \textbf{Digital benefits:} Noise immunity, regeneration
\item
  \textbf{Flexibility:} Easy to add/drop channels
\end{itemize}

\textbf{Applications:}

\begin{itemize}
\tightlist
\item
  \textbf{Telephone networks:} Digital transmission systems
\item
  \textbf{Digital audio:} Broadcasting and recording
\item
  \textbf{Satellite communications:} Multiple channel transmission
\end{itemize}

\textbf{Diagram of TDM Frame:}

\begin{verbatim}
   |{{-}{-}{-}{-}{-}{-}{-}{-}{-}{-}{-}{-}{-}{-} One TDM Frame {-}{-}{-}{-}{-}{-}{-}{-}{-}{-}{-}{-}{-}{-}|}
   +{-{-}{-}{-}{-}+{-}{-}{-}{-}{-}+{-}{-}{-}{-}{-}+{-}{-}{-}{-}{-}+{-}{-}{-}{-}{-}+       +{-}{-}{-}{-}{-}+}
   | Sync| Ch1 | Ch2 | Ch3 | Ch4 | ..... | ChN |
   +{-{-}{-}{-}{-}+{-}{-}{-}{-}{-}+{-}{-}{-}{-}{-}+{-}{-}{-}{-}{-}+{-}{-}{-}{-}{-}+       +{-}{-}{-}{-}{-}+}
\end{verbatim}

\end{solutionbox}
\begin{mnemonicbox}
``MSQT'' - Multiplex, Sample, Quantize, Transmit

\end{mnemonicbox}

\end{document}
