\documentclass[10pt,a4paper]{article}

% content/resources/templates/preamble.tex
\usepackage[margin=0.6in]{geometry}
\author{Milav Dabgar}
\usepackage{amsmath,amssymb,amsthm}
\usepackage{booktabs}
\usepackage{multirow}
\usepackage{xcolor}
\usepackage{tcolorbox}
\tcbuselibrary{breakable,skins}
\usepackage[colorlinks=true,linkcolor=blue]{hyperref}
\usepackage{titlesec}
\usepackage{enumitem}
\usepackage{tikz}
\usepackage{pgfplots}
\usepackage{circuitikz}
\usepackage[version=4]{mhchem}
\usepackage{longtable}
\usepackage{array}
\usepackage{float}
\usepackage{caption}
\usepackage{listings}

\lstset{
  basicstyle=\small\ttfamily,
  breaklines=true,
  breakatwhitespace=false,
  postbreak=\mbox{\textcolor{red}{$\hookrightarrow$}\space},
  float=false,
  numbers=left,
  numberstyle=\tiny\color{gray},
  numbersep=10pt,
  xleftmargin=2em,
  keywordstyle=\color{blue},
  commentstyle=\color{green!60!black},
  stringstyle=\color{purple},
  backgroundcolor=\color{gray!5},
  showstringspaces=false,
  tabsize=2,
  captionpos=b,
  keepspaces=true,
  columns=flexible
}

\pgfplotsset{compat=1.18}
\usetikzlibrary{shapes,arrows,positioning,calc,patterns,decorations.pathmorphing,decorations.markings,arrows.meta}

% Color scheme
\definecolor{headcolor}{RGB}{0,102,204}
\definecolor{keycolor}{RGB}{220,20,60}
\definecolor{solutioncolor}{RGB}{34,139,34}
\definecolor{mnemoniccolor}{RGB}{148,0,211}
\definecolor{codecolor}{RGB}{0,0,100}

% Spacing
\setlength{\parskip}{3pt}
\setlist[itemize]{nosep}
\setlist[enumerate]{nosep}

% Title formatting
\titleformat{\section}{\Large\bfseries\color{headcolor}}{\thesection}{1em}{}
\titleformat{\subsection}{\large\bfseries\color{headcolor}}{\thesubsection}{1em}{}

% Pandoc tightlist compatibility
\providecommand{\tightlist}{%
  \setlength{\itemsep}{0pt}\setlength{\parskip}{0pt}}

% Pandoc longtable compatibility
\newcounter{none}
\def\thenone{}


% content/resources/templates/english-boxes.tex
% This file is currently empty - it exists to maintain consistency with the import structure.
% Add custom environments here if needed in the future.


\begin{document}

\begin{center}
{\Huge\bfseries\color{headcolor} Subject Name Solutions}\\[5pt]
{\LARGE 4331104 -- Summer 2025}\\[3pt]
{\large Semester 1 Study Material}\\[3pt]
{\normalsize\textit{Detailed Solutions and Explanations}}
\end{center}

\vspace{10pt}

\subsection*{Question 1(a) [3 marks]}\label{q1a}

\textbf{Compare Analog Signal and Digital Signal.}

\begin{solutionbox}

{\def\LTcaptype{none} % do not increment counter
\begin{longtable}[]{@{}lll@{}}
\toprule\noalign{}
Parameter & Analog Signal & Digital Signal \\
\midrule\noalign{}
\endhead
\bottomrule\noalign{}
\endlastfoot
\textbf{Nature} & Continuous waveform & Discrete values (0 and 1) \\
\textbf{Amplitude} & Infinite variations & Fixed discrete levels \\
\textbf{Noise Effect} & More susceptible & Less susceptible \\
\textbf{Bandwidth} & Requires less bandwidth & Requires more
bandwidth \\
\textbf{Security} & Less secure & More secure \\
\end{longtable}
}

\begin{itemize}
\tightlist
\item
  \textbf{Signal Type}: Analog signals are continuous, Digital signals
  are discrete
\item
  \textbf{Noise Resistance}: Digital signals have better noise immunity
\end{itemize}

\end{solutionbox}
\begin{mnemonicbox}
``ABCD - Analog Bad for noise, Continuous; Digital
Discrete, Clean signals''

\end{mnemonicbox}
\begin{center}\rule{0.5\linewidth}{0.5pt}\end{center}

\subsection*{Question 1(b) [4 marks]}\label{q1b}

\textbf{Compare PAM, PWM and PPM.}

\begin{solutionbox}

{\def\LTcaptype{none} % do not increment counter
\begin{longtable}[]{@{}
  >{\raggedright\arraybackslash}p{(\linewidth - 6\tabcolsep) * \real{0.4231}}
  >{\raggedright\arraybackslash}p{(\linewidth - 6\tabcolsep) * \real{0.1923}}
  >{\raggedright\arraybackslash}p{(\linewidth - 6\tabcolsep) * \real{0.1923}}
  >{\raggedright\arraybackslash}p{(\linewidth - 6\tabcolsep) * \real{0.1923}}@{}}
\toprule\noalign{}
\begin{minipage}[b]{\linewidth}\raggedright
Parameter
\end{minipage} & \begin{minipage}[b]{\linewidth}\raggedright
PAM
\end{minipage} & \begin{minipage}[b]{\linewidth}\raggedright
PWM
\end{minipage} & \begin{minipage}[b]{\linewidth}\raggedright
PPM
\end{minipage} \\
\midrule\noalign{}
\endhead
\bottomrule\noalign{}
\endlastfoot
\textbf{Full Form} & Pulse Amplitude Modulation & Pulse Width Modulation
& Pulse Position Modulation \\
\textbf{Modulated Parameter} & Amplitude & Width/Duration &
Position/Time \\
\textbf{Noise Immunity} & Poor & Good & Excellent \\
\textbf{Bandwidth} & Minimum & Medium & Maximum \\
\textbf{Power Consumption} & High & Medium & Low \\
\end{longtable}
}

\textbf{Diagram:}

\begin{verbatim}
PAM: |▄▄|  |▄▄▄| |▄|     Amplitude varies
PWM: |▄| |▄▄▄| |▄▄|      Width varies  
PPM: |▄|  |▄| |▄|        Position varies
\end{verbatim}

\begin{itemize}
\tightlist
\item
  \textbf{Modulation Parameter}: Each type modulates different pulse
  characteristics
\item
  \textbf{Applications}: PWM used in motor control, PPM in radio control
  systems
\end{itemize}

\end{solutionbox}
\begin{mnemonicbox}
``PAM-Amplitude, PWM-Width, PPM-Position - AWP''

\end{mnemonicbox}
\begin{center}\rule{0.5\linewidth}{0.5pt}\end{center}

\subsection*{Question 1(c) [7 marks]}\label{q1c}

\textbf{Indicate the need of Modulation in detail. Calculate the height
of antenna if the frequency of Carrier signal is 1 MHz.}

\begin{solutionbox}

\textbf{Need for Modulation:}

{\def\LTcaptype{none} % do not increment counter
\begin{longtable}[]{@{}
  >{\raggedright\arraybackslash}p{(\linewidth - 2\tabcolsep) * \real{0.3810}}
  >{\raggedright\arraybackslash}p{(\linewidth - 2\tabcolsep) * \real{0.6190}}@{}}
\toprule\noalign{}
\begin{minipage}[b]{\linewidth}\raggedright
Reason
\end{minipage} & \begin{minipage}[b]{\linewidth}\raggedright
Explanation
\end{minipage} \\
\midrule\noalign{}
\endhead
\bottomrule\noalign{}
\endlastfoot
\textbf{Antenna Size Reduction} & Makes practical antenna sizes
possible \\
\textbf{Frequency Translation} & Shifts signal to suitable frequency
range \\
\textbf{Multiplexing} & Allows multiple signals on same medium \\
\textbf{Noise Reduction} & Improves signal-to-noise ratio \\
\textbf{Power Efficiency} & Better power utilization \\
\end{longtable}
}

\textbf{Antenna Height Calculation:} For efficient radiation, antenna
height = λ/4

λ = c/f = (3 \times 10^{8})/(1 \times 10^{6}) = 300 meters

\textbf{Antenna height} = λ/4 = 300/4 = \textbf{75 meters}

\begin{itemize}
\tightlist
\item
  \textbf{Practical Antenna}: Without modulation, antenna would be
  impractically large
\item
  \textbf{Frequency Shifting}: Allows better propagation characteristics
\end{itemize}

\end{solutionbox}
\begin{mnemonicbox}
``AFMNP - Antenna, Frequency, Multiplexing, Noise,
Power''

\end{mnemonicbox}
\begin{center}\rule{0.5\linewidth}{0.5pt}\end{center}

\subsection*{Question 1(c) OR [7
marks]}\label{q1c}

\textbf{Write frequency bands with applications domains of EM Wave
spectrum. Calculate Wavelength range of ELF band.}

\begin{solutionbox}

{\def\LTcaptype{none} % do not increment counter
\begin{longtable}[]{@{}llll@{}}
\toprule\noalign{}
Band & Frequency Range & Wavelength & Applications \\
\midrule\noalign{}
\endhead
\bottomrule\noalign{}
\endlastfoot
\textbf{ELF} & 30-300 Hz & 10^{6}-10^{7} m & Submarine communication \\
\textbf{VLF} & 3-30 kHz & 10^{4}-10^{5} m & Navigation, time signals \\
\textbf{LF} & 30-300 kHz & 10^{3}-10^{4} m & AM broadcasting \\
\textbf{MF} & 300 kHz-3 MHz & 100-1000 m & AM radio \\
\textbf{HF} & 3-30 MHz & 10-100 m & Short wave radio \\
\end{longtable}
}

\textbf{ELF Wavelength Calculation:}

\begin{itemize}
\tightlist
\item
  Lower frequency: f_{1} = 30 Hz, λ_{1} = c/f_{1} = (3\times10^{8})/30 = \textbf{10^{7}
  meters}
\item
  Upper frequency: f_{2} = 300 Hz, λ_{2} = c/f_{2} = (3\times10^{8})/300 = \textbf{10^{6}
  meters}
\end{itemize}

\textbf{ELF Wavelength range: 10^{6} to 10^{7} meters}

\begin{itemize}
\tightlist
\item
  \textbf{Application Domain}: Each band suited for specific
  applications
\item
  \textbf{Propagation}: Lower frequencies have better ground wave
  propagation
\end{itemize}

\end{solutionbox}
\begin{mnemonicbox}
``Every Valuable Learning Makes Happiness - ELF to HF
bands''

\end{mnemonicbox}
\begin{center}\rule{0.5\linewidth}{0.5pt}\end{center}

\subsection*{Question 2(a) [3 marks]}\label{q2a}

\textbf{Compare AM and FM.}

\begin{solutionbox}

{\def\LTcaptype{none} % do not increment counter
\begin{longtable}[]{@{}lll@{}}
\toprule\noalign{}
Parameter & AM & FM \\
\midrule\noalign{}
\endhead
\bottomrule\noalign{}
\endlastfoot
\textbf{Modulated Parameter} & Amplitude & Frequency \\
\textbf{Bandwidth} & 2fm & 2(Δf + fm) \\
\textbf{Noise Immunity} & Poor & Good \\
\textbf{Power Efficiency} & Low (33.33\%) & High \\
\textbf{Circuit Complexity} & Simple & Complex \\
\end{longtable}
}

\begin{itemize}
\tightlist
\item
  \textbf{Bandwidth}: FM requires much wider bandwidth than AM
\item
  \textbf{Quality}: FM provides better audio quality
\end{itemize}

\end{solutionbox}
\begin{mnemonicbox}
``AM-Amplitude simple, FM-Frequency complex but
better quality''

\end{mnemonicbox}
\begin{center}\rule{0.5\linewidth}{0.5pt}\end{center}

\subsection*{Question 2(b) [4 marks]}\label{q2b}

\textbf{Draw waveform of Amplitude Modulated wave.}

\begin{solutionbox}

\textbf{Diagram:}

\begin{verbatim}
Carrier Signal:     ∿∿∿∿∿∿∿∿∿∿∿∿∿∿∿
                    
Modulating Signal:  ∼    ∼    ∼    ∼

AM Wave:           .∿∿. .∿∿∿∿. .∿∿.
                  ∿    ∿      ∿    ∿
                      Envelope follows
                      modulating signal
\end{verbatim}

\textbf{Characteristics:}

\begin{itemize}
\tightlist
\item
  \textbf{Envelope}: The envelope follows the modulating signal
\item
  \textbf{Carrier Frequency}: Remains constant throughout
\item
  \textbf{Amplitude Variation}: Amplitude varies with modulating signal
\end{itemize}

\end{solutionbox}
\begin{mnemonicbox}
``Envelope Follows Message - EFM''

\end{mnemonicbox}
\begin{center}\rule{0.5\linewidth}{0.5pt}\end{center}

\subsection*{Question 2(c) [7 marks]}\label{q2c}

\textbf{Define Amplitude Modulation and Derive mathematical expression
for Double Sideband Full Carrier (DSBFC) Amplitude Modulation (AM)
signal.}

\begin{solutionbox}

\textbf{Definition:} Amplitude Modulation is the process where amplitude
of carrier signal varies according to instantaneous amplitude of
modulating signal.

\textbf{Mathematical Derivation:}

Let carrier signal: ec(t) = Ec cos(ωct) Let modulating signal: em(t) =
Em cos(ωmt)

\textbf{AM Signal Expression:} eAM(t) = [Ec + Em cos(ωmt)] cos(ωct)
eAM(t) = Ec cos(ωct) + Em cos(ωmt) cos(ωct)

Using trigonometric identity: cos A cos B = ½[cos(A+B) + cos(A-B)]

\textbf{Final AM Expression:} eAM(t) = Ec cos(ωct) + (Em/2) cos(ωc +
ωm)t + (Em/2) cos(ωc - ωm)t

\textbf{Components:}

\begin{itemize}
\tightlist
\item
  \textbf{Carrier Component}: Ec cos(ωct)
\item
  \textbf{Upper Sideband}: (Em/2) cos(ωc + ωm)t\\
\item
  \textbf{Lower Sideband}: (Em/2) cos(ωc - ωm)t
\end{itemize}

\end{solutionbox}
\begin{mnemonicbox}
``Carrier Plus Upper Lower Sidebands - CPULS''

\end{mnemonicbox}
\begin{center}\rule{0.5\linewidth}{0.5pt}\end{center}

\subsection*{Question 2(a) OR [3
marks]}\label{q2a}

\textbf{Compare Pre-emphasis and De-emphasis.}

\begin{solutionbox}

{\def\LTcaptype{none} % do not increment counter
\begin{longtable}[]{@{}
  >{\raggedright\arraybackslash}p{(\linewidth - 4\tabcolsep) * \real{0.2895}}
  >{\raggedright\arraybackslash}p{(\linewidth - 4\tabcolsep) * \real{0.3684}}
  >{\raggedright\arraybackslash}p{(\linewidth - 4\tabcolsep) * \real{0.3421}}@{}}
\toprule\noalign{}
\begin{minipage}[b]{\linewidth}\raggedright
Parameter
\end{minipage} & \begin{minipage}[b]{\linewidth}\raggedright
Pre-emphasis
\end{minipage} & \begin{minipage}[b]{\linewidth}\raggedright
De-emphasis
\end{minipage} \\
\midrule\noalign{}
\endhead
\bottomrule\noalign{}
\endlastfoot
\textbf{Location} & At transmitter & At receiver \\
\textbf{Function} & Boosts high frequencies & Attenuates high
frequencies \\
\textbf{Frequency Response} & High pass characteristic & Low pass
characteristic \\
\textbf{Purpose} & Improve S/N ratio & Restore original signal \\
\textbf{Time Constant} & 75 μs (FM broadcasting) & 75 μs (FM
broadcasting) \\
\end{longtable}
}

\begin{itemize}
\tightlist
\item
  \textbf{Noise Reduction}: Combined effect reduces noise in received
  signal
\item
  \textbf{Frequency Response}: Complementary characteristics
\end{itemize}

\end{solutionbox}
\begin{mnemonicbox}
``Pre-Boost, De-Cut - Noise Reduction Circuit''

\end{mnemonicbox}
\begin{center}\rule{0.5\linewidth}{0.5pt}\end{center}

\subsection*{Question 2(b) OR [4
marks]}\label{q2b}

\textbf{Draw waveform of Frequency Modulated wave.}

\begin{solutionbox}

\textbf{Diagram:}

\begin{verbatim}
Modulating Signal:  ∼    ∼    ∼    ∼

Carrier Signal:     ∿∿∿∿∿∿∿∿∿∿∿∿∿∿

FM Wave:           ∿∿∿  ∿∿∿∿∿∿  ∿∿∿
                      Higher freq  Lower freq
                      when mod +ve when mod {-ve}
\end{verbatim}

\textbf{Characteristics:}

\begin{itemize}
\tightlist
\item
  \textbf{Constant Amplitude}: Amplitude remains constant
\item
  \textbf{Frequency Variation}: Frequency varies with modulating signal
\item
  \textbf{Phase Continuity}: Phase remains continuous
\end{itemize}

\end{solutionbox}
\begin{mnemonicbox}
``Constant Amplitude, Variable Frequency - CAVF''

\end{mnemonicbox}
\begin{center}\rule{0.5\linewidth}{0.5pt}\end{center}

\subsection*{Question 2(c) OR [7
marks]}\label{q2c}

\textbf{Define Frequency Modulation and Derive mathematical expression
for FM wave.}

\begin{solutionbox}

\textbf{Definition:} Frequency Modulation is the process where frequency
of carrier signal varies according to instantaneous amplitude of
modulating signal.

\textbf{Mathematical Derivation:}

Let modulating signal: em(t) = Em cos(ωmt) Instantaneous frequency: fi =
fc + kf \times Em cos(ωmt)

Where kf = frequency sensitivity

\textbf{Instantaneous angular frequency:} ωi = 2π[fc + kf Em
cos(ωmt)] ωi = ωc + 2πkf Em cos(ωmt)

\textbf{Phase calculation:} θ(t) = \intωi dt = ωct + (2πkf Em/ωm) sin(ωmt)

Let modulation index: mf = 2πkf Em/ωm = Δf/fm

\textbf{Final FM Expression:} eFM(t) = Ec cos[ωct + mf sin(ωmt)]

\textbf{Parameters:}

\begin{itemize}
\tightlist
\item
  \textbf{Modulation Index}: mf = Δf/fm
\item
  \textbf{Frequency Deviation}: Δf = kf Em
\item
  \textbf{Bandwidth}: BW = 2(Δf + fm) (Carson's rule)
\end{itemize}

\end{solutionbox}
\begin{mnemonicbox}
``Frequency Varies with Message - FVM''

\end{mnemonicbox}
\begin{center}\rule{0.5\linewidth}{0.5pt}\end{center}

\subsection*{Question 3(a) [3 marks]}\label{q3a}

\textbf{Illustrate Slope detection method of FM demodulation.}

\begin{solutionbox}

\textbf{Slope Detection Principle:}

\begin{center}
\textbf{Mermaid Diagram (Code)}
\begin{verbatim}
{Shaded}
{Highlighting}[]
graph LR
    A[FM Signal] {-{-}{} B[Tuned Circuit]}
    B {-{-}{} C[Envelope Detector]}
    C {-{-}{} D[Audio Output]}
{Highlighting}
{Shaded}
\end{verbatim}
\end{center}

\textbf{Working:}

\begin{itemize}
\tightlist
\item
  \textbf{Tuned Circuit}: Converts frequency variations to amplitude
  variations
\item
  \textbf{Slope Operation}: Uses slope of resonance curve
\item
  \textbf{Envelope Detection}: Extracts amplitude variations
\end{itemize}

\textbf{Characteristics:}

\begin{itemize}
\tightlist
\item
  \textbf{Simple Circuit}: Easy to implement
\item
  \textbf{Linear Range}: Limited linear range
\item
  \textbf{Output Distortion}: Higher distortion compared to other
  methods
\end{itemize}

\end{solutionbox}
\begin{mnemonicbox}
``Slope Converts Frequency to Amplitude - SCFA''

\end{mnemonicbox}
\begin{center}\rule{0.5\linewidth}{0.5pt}\end{center}

\subsection*{Question 3(b) [4 marks]}\label{q3b}

\textbf{Explain different Characteristics of radio receiver.}

\begin{solutionbox}

{\def\LTcaptype{none} % do not increment counter
\begin{longtable}[]{@{}
  >{\raggedright\arraybackslash}p{(\linewidth - 4\tabcolsep) * \real{0.4000}}
  >{\raggedright\arraybackslash}p{(\linewidth - 4\tabcolsep) * \real{0.3000}}
  >{\raggedright\arraybackslash}p{(\linewidth - 4\tabcolsep) * \real{0.3000}}@{}}
\toprule\noalign{}
\begin{minipage}[b]{\linewidth}\raggedright
Characteristic
\end{minipage} & \begin{minipage}[b]{\linewidth}\raggedright
Definition
\end{minipage} & \begin{minipage}[b]{\linewidth}\raggedright
Importance
\end{minipage} \\
\midrule\noalign{}
\endhead
\bottomrule\noalign{}
\endlastfoot
\textbf{Sensitivity} & Minimum input signal for satisfactory output &
Better weak signal reception \\
\textbf{Selectivity} & Ability to select desired signal and reject
others & Reduces interference \\
\textbf{Fidelity} & Faithfulness of reproduction & Better audio
quality \\
\textbf{Image Frequency Rejection} & Rejection of image frequency &
Prevents false signals \\
\end{longtable}
}

\textbf{Mathematical Relations:}

\begin{itemize}
\tightlist
\item
  \textbf{Sensitivity}: Measured in μV for standard output
\item
  \textbf{Selectivity}: Q = f_{0}/BW
\item
  \textbf{Image Rejection Ratio}: IRR = 1 + (2πfIFRC)^{2}
\end{itemize}

\end{solutionbox}
\begin{mnemonicbox}
``Sensitive Selective Faithful Image-free - SSFI''

\end{mnemonicbox}
\begin{center}\rule{0.5\linewidth}{0.5pt}\end{center}

\subsection*{Question 3(c) [7 marks]}\label{q3c}

\textbf{Write short note on Super heterodyne receiver with suitable
block diagram.}

\begin{solutionbox}

\textbf{Block Diagram:}

\begin{center}
\textbf{Mermaid Diagram (Code)}
\begin{verbatim}
{Shaded}
{Highlighting}[]
graph LR
    A[Antenna] {-{-}{} B[RF Amplifier]}
    B {-{-}{} C[Mixer]}
    D[Local Oscillator] {-{-}{} C}
    C {-{-}{} E[IF Amplifier]}
    E {-{-}{} F[Detector]}
    F {-{-}{} G[AF Amplifier]}
    G {-{-}{} H[Speaker]}
    E {-{-}{} I[AGC]}
    I {-{-}{} B}
    I {-{-}{} E}
{Highlighting}
{Shaded}
\end{verbatim}
\end{center}

\textbf{Working Principle:}

\begin{itemize}
\tightlist
\item
  \textbf{RF Amplifier}: Amplifies received RF signal
\item
  \textbf{Mixer}: Converts RF to fixed IF frequency
\item
  \textbf{Local Oscillator}: Provides mixing frequency
\item
  \textbf{IF Amplifier}: Main amplification at fixed frequency
\item
  \textbf{Detector}: Recovers modulated signal
\item
  \textbf{AGC}: Maintains constant output level
\end{itemize}

\textbf{Advantages:}

\begin{itemize}
\tightlist
\item
  \textbf{High Sensitivity}: Better sensitivity than TRF
\item
  \textbf{Good Selectivity}: Better selectivity
\item
  \textbf{Stable Gain}: Stable gain characteristics
\end{itemize}

\textbf{IF Frequency Selection:} Standard IF: 455 kHz for AM, 10.7 MHz
for FM

\end{solutionbox}
\begin{mnemonicbox}
``Mix RF to IF for Better Selectivity - MRIBS''

\end{mnemonicbox}
\begin{center}\rule{0.5\linewidth}{0.5pt}\end{center}

\subsection*{Question 3(a) OR [3
marks]}\label{q3a}

\textbf{Illustrate working of FM demodulator using Phase Locked Loop.}

\begin{solutionbox}

\textbf{PLL FM Demodulator:}

\begin{center}
\textbf{Mermaid Diagram (Code)}
\begin{verbatim}
{Shaded}
{Highlighting}[]
graph LR
    A[FM Input] {-{-}{} B[Phase Detector]}
    C[VCO] {-{-}{} B}
    B {-{-}{} D[Loop Filter]}
    D {-{-}{} C}
    D {-{-}{} E[Audio Output]}
{Highlighting}
{Shaded}
\end{verbatim}
\end{center}

\textbf{Working Principle:}

\begin{itemize}
\tightlist
\item
  \textbf{Phase Detector}: Compares input FM with VCO output
\item
  \textbf{VCO}: Voltage Controlled Oscillator tracks input frequency\\
\item
  \textbf{Loop Filter}: Removes high frequency components
\item
  \textbf{Lock Condition}: VCO frequency equals input frequency
\end{itemize}

\textbf{Advantages:}

\begin{itemize}
\tightlist
\item
  \textbf{Linear Demodulation}: Excellent linearity
\item
  \textbf{Low Distortion}: Minimum distortion
\item
  \textbf{Good Tracking}: Excellent frequency tracking
\end{itemize}

\end{solutionbox}
\begin{mnemonicbox}
``Phase Lock Tracks Frequency - PLTF''

\end{mnemonicbox}
\begin{center}\rule{0.5\linewidth}{0.5pt}\end{center}

\subsection*{Question 3(b) OR [4
marks]}\label{q3b}

\textbf{Discuss Block diagram of basic FM receiver.}

\begin{solutionbox}

\textbf{FM Receiver Block Diagram:}

\begin{center}
\textbf{Mermaid Diagram (Code)}
\begin{verbatim}
{Shaded}
{Highlighting}[]
graph LR
    A[FM Antenna] {-{-}{} B[RF Amplifier]}
    B {-{-}{} C[Mixer]}
    D[Local Oscillator] {-{-}{} C}
    C {-{-}{} E[IF Amplifier 10.7MHz]}
    E {-{-}{} F[Limiter]}
    F {-{-}{} G[FM Detector]}
    G {-{-}{} H[De{-}emphasis]}
    H {-{-}{} I[AF Amplifier]}
    I {-{-}{} J[Speaker]}
{Highlighting}
{Shaded}
\end{verbatim}
\end{center}

\textbf{Block Functions:}

\begin{itemize}
\tightlist
\item
  \textbf{RF Amplifier}: Amplifies weak FM signal (88-108 MHz)
\item
  \textbf{Mixer}: Converts to IF frequency (10.7 MHz)
\item
  \textbf{Limiter}: Removes amplitude variations
\item
  \textbf{FM Detector}: Recovers audio signal
\item
  \textbf{De-emphasis}: Restores original frequency response
\end{itemize}

\textbf{Key Differences from AM Receiver:}

\begin{itemize}
\tightlist
\item
  \textbf{Higher IF}: 10.7 MHz vs 455 kHz
\item
  \textbf{Limiter Stage}: Additional limiter stage
\item
  \textbf{De-emphasis}: Pre/de-emphasis network
\end{itemize}

\end{solutionbox}
\begin{mnemonicbox}
``FM needs Higher IF and Limiting - FHIL''

\end{mnemonicbox}
\begin{center}\rule{0.5\linewidth}{0.5pt}\end{center}

\subsection*{Question 3(c) OR [7
marks]}\label{q3c}

\textbf{Write short note on Envelope detector using diode with suitable
circuit diagram and waveform.}

\begin{solutionbox}

\textbf{Circuit Diagram:}

\begin{verbatim}
      D1
AM {-{-}||{-}{-}+{-}{-}{-}{-} Audio Output}
     |    |
     |    R
     |    |
     |    C
     |    |
    GND  GND
\end{verbatim}

\textbf{Working Principle:}

\begin{verbatim}
AM Input:    .∿∿. .∿∿∿∿. .∿∿.
            ∿    ∿      ∿    ∿

Diode Output: ▄▄▄ ▄▄▄▄▄▄ ▄▄▄
(After filtering)

Audio Output: ∼    ∼    ∼    ∼
\end{verbatim}

\textbf{Operation:}

\begin{itemize}
\tightlist
\item
  \textbf{Diode Conduction}: Conducts during positive half cycles
\item
  \textbf{Capacitor Charging}: Charges to peak value
\item
  \textbf{RC Discharge}: Discharges through RC circuit
\item
  \textbf{Envelope Following}: Output follows envelope
\end{itemize}

\textbf{Design Considerations:}

\begin{itemize}
\tightlist
\item
  \textbf{Time Constant}: RC \textgreater\textgreater{} 1/fc but RC
  \textless\textless{} 1/fm
\item
  \textbf{Diode Selection}: Fast recovery diode preferred
\item
  \textbf{Load Resistance}: Should be much larger than diode resistance
\end{itemize}

\textbf{Advantages:}

\begin{itemize}
\tightlist
\item
  \textbf{Simplicity}: Very simple circuit
\item
  \textbf{Low Cost}: Economical solution
\item
  \textbf{High Efficiency}: Good detection efficiency
\end{itemize}

\end{solutionbox}
\begin{mnemonicbox}
``Diode Charges, RC Follows Envelope - DCRF''

\end{mnemonicbox}
\begin{center}\rule{0.5\linewidth}{0.5pt}\end{center}

\subsection*{Question 4(a) [3 marks]}\label{q4a}

\textbf{Illustrate under sampling, over sampling and critical sampling.}

\begin{solutionbox}

{\def\LTcaptype{none} % do not increment counter
\begin{longtable}[]{@{}lll@{}}
\toprule\noalign{}
Type & Condition & Result \\
\midrule\noalign{}
\endhead
\bottomrule\noalign{}
\endlastfoot
\textbf{Under Sampling} & fs \textless{} 2fm & Aliasing occurs \\
\textbf{Critical Sampling} & fs = 2fm & Just adequate, no margin \\
\textbf{Over Sampling} & fs \textgreater{} 2fm & No aliasing, safe
margin \\
\end{longtable}
}

\textbf{Diagram:}

\begin{verbatim}
Original Signal:     ∿∿∿∿∿∿∿

Under Sampling:      ∿ . . ∿     Aliasing
Critical Sampling:   ∿ . ∿ .     Just OK  
Over Sampling:       ∿.∿.∿.∿     Safe
\end{verbatim}

\begin{itemize}
\tightlist
\item
  \textbf{Aliasing Effect}: Under sampling causes frequency overlap
\item
  \textbf{Nyquist Rate}: Minimum sampling rate = 2fm
\item
  \textbf{Practical Rate}: Usually 2.5 to 5 times message frequency
\end{itemize}

\end{solutionbox}
\begin{mnemonicbox}
``Under-Alias, Critical-Just, Over-Safe - UCO''

\end{mnemonicbox}
\begin{center}\rule{0.5\linewidth}{0.5pt}\end{center}

\subsection*{Question 4(b) [4 marks]}\label{q4b}

\textbf{State Sampling theorem and define Nyquist rate, Nyquist interval
and aliasing error.}

\begin{solutionbox}

\textbf{Sampling Theorem:} ``A continuous signal can be completely
recovered from its samples if sampling frequency is at least twice the
highest frequency component of the signal.''

\textbf{Definitions:}

{\def\LTcaptype{none} % do not increment counter
\begin{longtable}[]{@{}lll@{}}
\toprule\noalign{}
Term & Definition & Formula \\
\midrule\noalign{}
\endhead
\bottomrule\noalign{}
\endlastfoot
\textbf{Nyquist Rate} & Minimum sampling frequency & fs = 2fm \\
\textbf{Nyquist Interval} & Maximum sampling interval & Ts = 1/(2fm) \\
\textbf{Aliasing Error} & Frequency overlap due to under sampling & fa
= \\
\end{longtable}
}

\textbf{Mathematical Expression:}

\begin{itemize}
\tightlist
\item
  \textbf{Sampling Frequency}: fs \geq 2fm (Nyquist criterion)
\item
  \textbf{Sampling Period}: Ts = 1/fs
\item
  \textbf{Aliasing Condition}: fs \textless{} 2fm
\end{itemize}

\textbf{Practical Applications:}

\begin{itemize}
\tightlist
\item
  \textbf{Digital Audio}: fs = 44.1 kHz for fm = 20 kHz
\item
  \textbf{Telephone System}: fs = 8 kHz for fm = 4 kHz
\end{itemize}

\end{solutionbox}
\begin{mnemonicbox}
``Sample at twice message frequency - S2M''

\end{mnemonicbox}
\begin{center}\rule{0.5\linewidth}{0.5pt}\end{center}

\subsection*{Question 4(c) [7 marks]}\label{q4c}

\textbf{Discuss Ideal, Natural and Flat top sampling.}

\begin{solutionbox}

\textbf{Types of Sampling:}

{\def\LTcaptype{none} % do not increment counter
\begin{longtable}[]{@{}
  >{\raggedright\arraybackslash}p{(\linewidth - 4\tabcolsep) * \real{0.1304}}
  >{\raggedright\arraybackslash}p{(\linewidth - 4\tabcolsep) * \real{0.3478}}
  >{\raggedright\arraybackslash}p{(\linewidth - 4\tabcolsep) * \real{0.5217}}@{}}
\toprule\noalign{}
\begin{minipage}[b]{\linewidth}\raggedright
Type
\end{minipage} & \begin{minipage}[b]{\linewidth}\raggedright
Characteristics
\end{minipage} & \begin{minipage}[b]{\linewidth}\raggedright
Mathematical Expression
\end{minipage} \\
\midrule\noalign{}
\endhead
\bottomrule\noalign{}
\endlastfoot
\textbf{Ideal Sampling} & Impulse train multiplication & xs(t) =
x(t)·δT(t) \\
\textbf{Natural Sampling} & Variable width pulses & Top follows
signal \\
\textbf{Flat Top Sampling} & Constant amplitude pulses & Sample and
hold \\
\end{longtable}
}

\textbf{Waveforms:}

\begin{verbatim}
Original:    ∿∿∿∿∿∿∿∿∿∿∿∿

Ideal:       ↑ ↑ ↑ ↑ ↑ ↑     Impulses

Natural:     |∿| |∿| |∿|     Variable width

Flat Top:    |▄| |▄| |▄|     Constant width
\end{verbatim}

\textbf{Frequency Spectrum:}

\begin{itemize}
\tightlist
\item
  \textbf{Ideal Sampling}: Exact spectral replication
\item
  \textbf{Natural Sampling}: Slight spectral modification\\
\item
  \textbf{Flat Top Sampling}: Aperture effect present
\end{itemize}

\textbf{Practical Implementation:}

\begin{itemize}
\tightlist
\item
  \textbf{Ideal}: Theoretical only
\item
  \textbf{Natural}: Used in PAM systems
\item
  \textbf{Flat Top}: Sample-and-hold circuits, ADC systems
\end{itemize}

\textbf{Aperture Effect:} In flat-top sampling:
\textbar Sa(πfT/2)\textbar{} = \textbar sin(πfT/2)/(πfT/2)\textbar{}

\end{solutionbox}
\begin{mnemonicbox}
``Ideal-Impulse, Natural-Variable, Flat-Constant -
IVF''

\end{mnemonicbox}
\begin{center}\rule{0.5\linewidth}{0.5pt}\end{center}

\subsection*{Question 4(a) OR [3
marks]}\label{q4a}

\textbf{Illustrate the working of Delta modulator with suitable block
diagram.}

\begin{solutionbox}

\textbf{Delta Modulator Block Diagram:}

\begin{center}
\textbf{Mermaid Diagram (Code)}
\begin{verbatim}
{Shaded}
{Highlighting}[]
graph LR
    A[Input Signal] {-{-}{} B[Comparator]}
    B {-{-}{} C[1{-}bit Quantizer]}
    C {-{-}{} D[Output]}
    C {-{-}{} E[Integrator]}
    E {-{-}{} F[Delay]}
    F {-{-}{} B}
{Highlighting}
{Shaded}
\end{verbatim}
\end{center}

\textbf{Working Principle:}

\begin{itemize}
\tightlist
\item
  \textbf{Comparison}: Input compared with previous integrated output
\item
  \textbf{1-bit Quantization}: Output is +Δ or -Δ
\item
  \textbf{Integration}: Integrator approximates input signal
\item
  \textbf{Feedback}: Previous output fed back for comparison
\end{itemize}

\textbf{Output Characteristics:}

\begin{itemize}
\tightlist
\item
  \textbf{Binary Output}: Only 1 bit per sample
\item
  \textbf{Step Size}: Fixed step size Δ
\item
  \textbf{Tracking}: Output tracks input in steps
\end{itemize}

\end{solutionbox}
\begin{mnemonicbox}
``Compare, Quantize, Integrate, Feedback - CQIF''

\end{mnemonicbox}
\begin{center}\rule{0.5\linewidth}{0.5pt}\end{center}

\subsection*{Question 4(b) OR [4
marks]}\label{q4b}

\textbf{Write disadvantages of Delta modulation (DM) with suitable
explanation.}

\begin{solutionbox}

\textbf{Major Disadvantages:}

{\def\LTcaptype{none} % do not increment counter
\begin{longtable}[]{@{}
  >{\raggedright\arraybackslash}p{(\linewidth - 4\tabcolsep) * \real{0.3784}}
  >{\raggedright\arraybackslash}p{(\linewidth - 4\tabcolsep) * \real{0.3514}}
  >{\raggedright\arraybackslash}p{(\linewidth - 4\tabcolsep) * \real{0.2703}}@{}}
\toprule\noalign{}
\begin{minipage}[b]{\linewidth}\raggedright
Disadvantage
\end{minipage} & \begin{minipage}[b]{\linewidth}\raggedright
Explanation
\end{minipage} & \begin{minipage}[b]{\linewidth}\raggedright
Solution
\end{minipage} \\
\midrule\noalign{}
\endhead
\bottomrule\noalign{}
\endlastfoot
\textbf{Slope Overload} & Cannot track fast changes & Increase step
size \\
\textbf{Granular Noise} & Quantization noise in flat regions & Decrease
step size \\
\textbf{High Bit Rate} & Requires high sampling rate & Use ADPCM \\
\textbf{Limited Dynamic Range} & Fixed step size limitation & Adaptive
techniques \\
\end{longtable}
}

\textbf{Slope Overload Condition:} When \textbar dx/dt\textbar{}
\textgreater{} Δfs, slope overload occurs

\textbf{Granular Noise:} Occurs when input signal changes slowly or
remains constant

\textbf{Waveforms:}

\begin{verbatim}
Slope Overload:    /∿∿∿    Input too fast
                  /▄▄▄     DM output lags

Granular Noise:   \_\_\_\_     Flat input
                  ▄▄▄▄     DM oscillates
\end{verbatim}

\textbf{Performance Parameters:}

\begin{itemize}
\tightlist
\item
  \textbf{Slope Overload}: Maximum slope = Δfs
\item
  \textbf{Granular Noise}: Depends on step size
\item
  \textbf{SNR}: Limited by both effects
\end{itemize}

\end{solutionbox}
\begin{mnemonicbox}
``Slope-Overload, Granular-Noise, High-Bitrate -
SOG-H''

\end{mnemonicbox}
\begin{center}\rule{0.5\linewidth}{0.5pt}\end{center}

\subsection*{Question 4(c) OR [7
marks]}\label{q4c}

\textbf{Describe functions of each block of pulse code modulation (PCM)
transmitter and receiver.}

\begin{solutionbox}

\textbf{PCM Transmitter Block Diagram:}

\begin{center}
\textbf{Mermaid Diagram (Code)}
\begin{verbatim}
{Shaded}
{Highlighting}[]
graph LR
    A[Analog Input] {-{-}{} B[LPF]}
    B {-{-}{} C[Sample \& Hold]}
    C {-{-}{} D[Quantizer]}
    D {-{-}{} E[Encoder]}
    E {-{-}{} F[Digital Output]}
{Highlighting}
{Shaded}
\end{verbatim}
\end{center}

\textbf{PCM Receiver Block Diagram:}

\begin{center}
\textbf{Mermaid Diagram (Code)}
\begin{verbatim}
{Shaded}
{Highlighting}[]
graph LR
    G[Digital Input] {-{-}{} H[Decoder]}
    H {-{-}{} I[DAC]}
    I {-{-}{} J[LPF]}
    J {-{-}{} K[Analog Output]}
{Highlighting}
{Shaded}
\end{verbatim}
\end{center}

\textbf{Transmitter Block Functions:}

{\def\LTcaptype{none} % do not increment counter
\begin{longtable}[]{@{}ll@{}}
\toprule\noalign{}
Block & Function \\
\midrule\noalign{}
\endhead
\bottomrule\noalign{}
\endlastfoot
\textbf{LPF} & Anti-aliasing filter, removes frequencies \textgreater{}
fm \\
\textbf{Sample \& Hold} & Samples at fs \geq 2fm and holds value \\
\textbf{Quantizer} & Converts to discrete amplitude levels \\
\textbf{Encoder} & Converts quantized samples to binary code \\
\end{longtable}
}

\textbf{Receiver Block Functions:}

{\def\LTcaptype{none} % do not increment counter
\begin{longtable}[]{@{}ll@{}}
\toprule\noalign{}
Block & Function \\
\midrule\noalign{}
\endhead
\bottomrule\noalign{}
\endlastfoot
\textbf{Decoder} & Converts binary code to quantized levels \\
\textbf{DAC} & Digital to Analog conversion \\
\textbf{LPF} & Reconstruction filter, removes sampling frequency \\
\end{longtable}
}

\textbf{Technical Specifications:}

\begin{itemize}
\tightlist
\item
  \textbf{Quantization Levels}: L = 2^{n} (n = number of bits)
\item
  \textbf{Quantization Error}: Δ/2 maximum
\item
  \textbf{Bit Rate}: fb = n \times fs
\end{itemize}

\textbf{PCM Advantages:}

\begin{itemize}
\tightlist
\item
  \textbf{Noise Immunity}: Excellent noise performance
\item
  \textbf{Regeneration}: Can be regenerated without error accumulation
\item
  \textbf{Multiplexing}: Easy to multiplex multiple channels
\end{itemize}

\end{solutionbox}
\begin{mnemonicbox}
``Low-pass, Sample, Quantize, Encode - LSQE for TX;
Decode, Convert, Filter - DCF for RX''

\end{mnemonicbox}
\begin{center}\rule{0.5\linewidth}{0.5pt}\end{center}

\subsection*{Question 5(a) [3 marks]}\label{q5a}

\textbf{Discuss block diagram of TDM-PCM system in brief.}

\begin{solutionbox}

\textbf{TDM-PCM System Block Diagram:}

\begin{center}
\textbf{Mermaid Diagram (Code)}
\begin{verbatim}
{Shaded}
{Highlighting}[]
graph LR
    A[Channel 1] {-{-}{} D[Commutator]}
    B[Channel 2] {-{-}{} D}
    C[Channel 3] {-{-}{} D}
    D {-{-}{} E[PCM Encoder]}
    E {-{-}{} F[Transmission]}
    F {-{-}{} G[PCM Decoder]}
    G {-{-}{} H[Decommutator]}
    H {-{-}{} I[Channel 1]}
    H {-{-}{} J[Channel 2]}
    H {-{-}{} K[Channel 3]}
{Highlighting}
{Shaded}
\end{verbatim}
\end{center}

\textbf{System Operation:}

\begin{itemize}
\tightlist
\item
  \textbf{Commutator}: Sequential sampling of multiple channels
\item
  \textbf{PCM Encoder}: Converts samples to digital format
\item
  \textbf{Time Division}: Each channel gets fixed time slot
\item
  \textbf{Decommutator}: Separates channels at receiver
\end{itemize}

\textbf{Frame Structure:}

\begin{itemize}
\tightlist
\item
  \textbf{Time Slot}: Each channel assigned specific time
\item
  \textbf{Frame Period}: Complete cycle for all channels
\item
  \textbf{Synchronization}: Frame synchronization bits added
\end{itemize}

\textbf{Advantages:}

\begin{itemize}
\tightlist
\item
  \textbf{Bandwidth Efficiency}: Efficient spectrum utilization
\item
  \textbf{Multiple Channels}: Multiple channels on single link
\end{itemize}

\end{solutionbox}
\begin{mnemonicbox}
``Time Division Multiple Access - TDMA''

\end{mnemonicbox}
\begin{center}\rule{0.5\linewidth}{0.5pt}\end{center}

\subsection*{Question 5(b) [4 marks]}\label{q5b}

\textbf{Write short note on Adaptive delta modulation (ADM).}

\begin{solutionbox}

\textbf{ADM Block Diagram:}

\begin{center}
\textbf{Mermaid Diagram (Code)}
\begin{verbatim}
{Shaded}
{Highlighting}[]
graph LR
    A[Input] {-{-}{} B[Comparator]}
    B {-{-}{} C[Logic Circuit]}
    C {-{-}{} D[Step Size Control]}
    D {-{-}{} E[Integrator]}
    E {-{-}{} F[Delay]}
    F {-{-}{} B}
    C {-{-}{} G[Output]}
{Highlighting}
{Shaded}
\end{verbatim}
\end{center}

\textbf{Working Principle:}

\begin{itemize}
\tightlist
\item
  \textbf{Adaptive Step Size}: Step size changes based on input
  characteristics
\item
  \textbf{Slope Overload Prevention}: Increases step size for fast
  changes
\item
  \textbf{Granular Noise Reduction}: Decreases step size for slow
  changes
\item
  \textbf{Logic Control}: Algorithm controls step size adaptation
\end{itemize}

\textbf{Step Size Control:}

\begin{itemize}
\tightlist
\item
  \textbf{Increase}: When consecutive bits are same (slope overload
  detected)
\item
  \textbf{Decrease}: When alternate pattern occurs (granular region)
\end{itemize}

\textbf{Advantages over Standard DM:}

\begin{itemize}
\tightlist
\item
  \textbf{Better SNR}: Improved signal-to-noise ratio
\item
  \textbf{Dynamic Range}: Better dynamic range
\item
  \textbf{Automatic Adaptation}: Self-adjusting characteristics
\end{itemize}

\end{solutionbox}
\begin{mnemonicbox}
``Adaptive Step size Reduces both Slope-overload and
Granular noise - ASRSG''

\end{mnemonicbox}
\begin{center}\rule{0.5\linewidth}{0.5pt}\end{center}

\subsection*{Question 5(c) [7 marks]}\label{q5c}

\textbf{Define Line coding. Draw NRZ (unipolar), RZ (unipolar),
Manchester coding waveforms for ``1 0 1 1 1 0 1 1''.}

\begin{solutionbox}

\textbf{Definition:} Line coding is the process of converting digital
data into digital signals suitable for transmission over communication
channels.

\textbf{Waveform Diagrams:}

\begin{verbatim}
Data:        1  0  1  1  1  0  1  1

NRZ Unipolar:
             ▄▄    ▄▄ ▄▄ ▄▄    ▄▄ ▄▄
                \_\_          \_\_

RZ Unipolar:
             ▄  ▄  ▄  ▄  ▄     ▄  ▄
             ▄▄▄▄▄▄▄▄▄▄▄▄▄▄▄▄▄▄▄▄▄▄▄

Manchester:
             ▄▄    ▄▄ ▄▄ ▄▄    ▄▄ ▄▄
                \_\_  \_\_ \_\_ \_\_ \_\_
             Transition at middle of each bit
\end{verbatim}

\textbf{Characteristics:}

{\def\LTcaptype{none} % do not increment counter
\begin{longtable}[]{@{}
  >{\raggedright\arraybackslash}p{(\linewidth - 6\tabcolsep) * \real{0.3095}}
  >{\raggedright\arraybackslash}p{(\linewidth - 6\tabcolsep) * \real{0.2143}}
  >{\raggedright\arraybackslash}p{(\linewidth - 6\tabcolsep) * \real{0.2143}}
  >{\raggedright\arraybackslash}p{(\linewidth - 6\tabcolsep) * \real{0.2619}}@{}}
\toprule\noalign{}
\begin{minipage}[b]{\linewidth}\raggedright
Coding Type
\end{minipage} & \begin{minipage}[b]{\linewidth}\raggedright
Logic 1
\end{minipage} & \begin{minipage}[b]{\linewidth}\raggedright
Logic 0
\end{minipage} & \begin{minipage}[b]{\linewidth}\raggedright
Bandwidth
\end{minipage} \\
\midrule\noalign{}
\endhead
\bottomrule\noalign{}
\endlastfoot
\textbf{NRZ Unipolar} & +V & 0V & fb \\
\textbf{RZ Unipolar} & +V for T/2, 0V for T/2 & 0V & 2fb \\
\textbf{Manchester} & High-to-Low transition & Low-to-High transition &
2fb \\
\end{longtable}
}

\textbf{Properties:}

\begin{itemize}
\tightlist
\item
  \textbf{NRZ}: No return to zero, simple but no self-synchronization
\item
  \textbf{RZ}: Return to zero, easy clock recovery but double bandwidth
\item
  \textbf{Manchester}: Self-synchronizing, used in Ethernet
\end{itemize}

\textbf{Applications:}

\begin{itemize}
\tightlist
\item
  \textbf{NRZ}: Simple digital systems
\item
  \textbf{RZ}: Magnetic recording
\item
  \textbf{Manchester}: Ethernet, some wireless standards
\end{itemize}

\end{solutionbox}
\begin{mnemonicbox}
``NRZ-Simple, RZ-Return, Manchester-Transition -
SRT''

\end{mnemonicbox}
\begin{center}\rule{0.5\linewidth}{0.5pt}\end{center}

\subsection*{Question 5(a) OR [3
marks]}\label{q5a}

\textbf{Describe concept of Time division digital multiplexing.}

\begin{solutionbox}

\textbf{TDM Concept:} Time Division Multiplexing is a technique where
multiple digital signals are transmitted over a single channel by
allocating different time slots to each signal.

\textbf{TDM Frame Structure:}

\begin{verbatim}
Frame: |CH1|CH2|CH3|CH4|SYNC|CH1|CH2|CH3|CH4|SYNC|
       ――――― Frame Period ―――――
\end{verbatim}

\textbf{Working Principle:}

{\def\LTcaptype{none} % do not increment counter
\begin{longtable}[]{@{}ll@{}}
\toprule\noalign{}
Component & Function \\
\midrule\noalign{}
\endhead
\bottomrule\noalign{}
\endlastfoot
\textbf{Time Slots} & Fixed duration allocated to each channel \\
\textbf{Frame} & Complete cycle containing all channels \\
\textbf{Synchronization} & Maintains proper channel separation \\
\textbf{Multiplexer} & Combines multiple inputs sequentially \\
\end{longtable}
}

\textbf{Key Features:}

\begin{itemize}
\tightlist
\item
  \textbf{Fixed Time Slot}: Each channel gets predetermined time
\item
  \textbf{Sequential Sampling}: Channels sampled one after another\\
\item
  \textbf{Digital Transmission}: Suitable for digital signals
\item
  \textbf{Bandwidth Sharing}: Efficient spectrum utilization
\end{itemize}

\textbf{Applications:}

\begin{itemize}
\tightlist
\item
  \textbf{Telephone System}: T1, E1 systems
\item
  \textbf{Digital Hierarchy}: PDH, SDH systems
\end{itemize}

\end{solutionbox}
\begin{mnemonicbox}
``Time slots Share Single Channel - TSSC''

\end{mnemonicbox}
\begin{center}\rule{0.5\linewidth}{0.5pt}\end{center}

\subsection*{Question 5(b) OR [4
marks]}\label{q5b}

\textbf{Write short note on Differential PCM (DPCM).}

\begin{solutionbox}

\textbf{DPCM Block Diagram:}

\begin{center}
\textbf{Mermaid Diagram (Code)}
\begin{verbatim}
{Shaded}
{Highlighting}[]
graph LR
    A[Input] {-{-}{} B[Difference]}
    C[Predictor] {-{-}{} B}
    B {-{-}{} D[Quantizer]}
    D {-{-}{} E[Encoder]}
    E {-{-}{} F[Output]}
    D {-{-}{} G[Local Decoder]}
    G {-{-}{} H[Adder]}
    C {-{-}{} H}
    H {-{-}{} C}
{Highlighting}
{Shaded}
\end{verbatim}
\end{center}

\textbf{Working Principle:}

\begin{itemize}
\tightlist
\item
  \textbf{Prediction}: Predicts current sample from previous samples
\item
  \textbf{Difference Signal}: Transmits difference between actual and
  predicted
\item
  \textbf{Quantization}: Quantizes difference signal only
\item
  \textbf{Local Decoder}: Maintains same reference as receiver
\end{itemize}

\textbf{Prediction Algorithms:}

{\def\LTcaptype{none} % do not increment counter
\begin{longtable}[]{@{}lll@{}}
\toprule\noalign{}
Type & Formula & Application \\
\midrule\noalign{}
\endhead
\bottomrule\noalign{}
\endlastfoot
\textbf{Zero Order} & x̂(n) = x(n-1) & Simple predictor \\
\textbf{First Order} & x̂(n) = ax(n-1) & Better prediction \\
\textbf{Higher Order} & x̂(n) = Σai\timesx(n-i) & Optimum prediction \\
\end{longtable}
}

\textbf{Advantages:}

\begin{itemize}
\tightlist
\item
  \textbf{Reduced Bit Rate}: Lower bit rate than PCM
\item
  \textbf{Better SNR}: Better SNR for same bit rate
\item
  \textbf{Predictive Coding}: Exploits signal correlation
\end{itemize}

\textbf{Applications:}

\begin{itemize}
\tightlist
\item
  \textbf{Image Compression}: JPEG standards
\item
  \textbf{Video Coding}: Motion compensation
\item
  \textbf{Speech Coding}: Speech compression systems
\end{itemize}

\textbf{Comparison with PCM:}

\begin{itemize}
\tightlist
\item
  \textbf{Bit Rate}: DPCM requires fewer bits
\item
  \textbf{Complexity}: More complex than PCM
\item
  \textbf{Quality}: Better quality at same bit rate
\end{itemize}

\end{solutionbox}
\begin{mnemonicbox}
``Predict Difference, Quantize Less - PDQL''

\end{mnemonicbox}
\begin{center}\rule{0.5\linewidth}{0.5pt}\end{center}

\subsection*{Question 5(c) OR [7
marks]}\label{q5c}

\textbf{Write short note on 4 level digital multiplexing Hierarchy.}

\begin{solutionbox}

\textbf{Digital Multiplexing Hierarchy:}

\textbf{Level Structure:}

{\def\LTcaptype{none} % do not increment counter
\begin{longtable}[]{@{}lllll@{}}
\toprule\noalign{}
Level & Name & Bit Rate & Voice Channels & Application \\
\midrule\noalign{}
\endhead
\bottomrule\noalign{}
\endlastfoot
\textbf{Level 0} & DS-0 & 64 kbps & 1 & Basic voice channel \\
\textbf{Level 1} & DS-1/T1 & 1.544 Mbps & 24 & Primary multiplex \\
\textbf{Level 2} & DS-2/T2 & 6.312 Mbps & 96 & Secondary multiplex \\
\textbf{Level 3} & DS-3/T3 & 44.736 Mbps & 672 & Tertiary multiplex \\
\end{longtable}
}

\textbf{Multiplexing Structure:}

\begin{center}
\textbf{Mermaid Diagram (Code)}
\begin{verbatim}
{Shaded}
{Highlighting}[]
graph TD
    A[24  DS{-0] {-}{-}{} B[DS{-}1]}
    C[4  DS{-1] {-}{-}{} D[DS{-}2]}
    E[7  DS{-2] {-}{-}{} F[DS{-}3]}
    G[6  DS{-3] {-}{-}{} H[DS{-}4]}
{Highlighting}
{Shaded}
\end{verbatim}
\end{center}

\textbf{Frame Structure for T1:}

\begin{itemize}
\tightlist
\item
  \textbf{Frame Length}: 193 bits (192 data + 1 framing)
\item
  \textbf{Frame Rate}: 8000 frames/second
\item
  \textbf{Time Slot}: 8 bits per channel
\item
  \textbf{Framing Bit}: Synchronization pattern
\end{itemize}

\textbf{T1 Frame Format:}

\begin{verbatim}
|F|CH1|CH2|...|CH24|F|CH1|CH2|...|CH24|
 ↑              ↑
Framing       193 bits total
\end{verbatim}

\textbf{Multiplexing Process:}

\begin{itemize}
\tightlist
\item
  \textbf{Level 1}: 24 voice channels \times 64 kbps + overhead = 1.544 Mbps
\item
  \textbf{Level 2}: 4 T1 streams + overhead = 6.312 Mbps
\item
  \textbf{Level 3}: 7 T2 streams + overhead = 44.736 Mbps
\item
  \textbf{Synchronization}: Each level adds synchronization bits
\end{itemize}

\textbf{Applications:}

\begin{itemize}
\tightlist
\item
  \textbf{Telephone Network}: Primary application in telephone systems
\item
  \textbf{Data Communication}: High-speed data transmission
\item
  \textbf{Internet Backbone}: Internet service provider connections
\end{itemize}

\textbf{International Standards:}

\begin{itemize}
\tightlist
\item
  \textbf{North American}: T1/T3 hierarchy (DS series)
\item
  \textbf{European}: E1/E3 hierarchy (different bit rates)
\item
  \textbf{ITU-T}: International recommendations
\end{itemize}

\textbf{Advantages:}

\begin{itemize}
\tightlist
\item
  \textbf{Standardization}: Well-defined international standards
\item
  \textbf{Scalability}: Easy to scale up capacity
\item
  \textbf{Interoperability}: Compatible across different vendors
\end{itemize}

\end{solutionbox}
\begin{mnemonicbox}
``Digital Signal hierarchy: 0-1-2-3 levels Build
Communication Systems - DS-BCS''

\end{mnemonicbox}

\end{document}
