\documentclass[10pt,a4paper]{article}

% content/resources/templates/preamble.tex
\usepackage[margin=0.6in]{geometry}
\author{Milav Dabgar}
\usepackage{amsmath,amssymb,amsthm}
\usepackage{booktabs}
\usepackage{multirow}
\usepackage{xcolor}
\usepackage{tcolorbox}
\tcbuselibrary{breakable,skins}
\usepackage[colorlinks=true,linkcolor=blue]{hyperref}
\usepackage{titlesec}
\usepackage{enumitem}
\usepackage{tikz}
\usepackage{pgfplots}
\usepackage{circuitikz}
\usepackage[version=4]{mhchem}
\usepackage{longtable}
\usepackage{array}
\usepackage{float}
\usepackage{caption}
\usepackage{listings}

\lstset{
  basicstyle=\small\ttfamily,
  breaklines=true,
  breakatwhitespace=false,
  postbreak=\mbox{\textcolor{red}{$\hookrightarrow$}\space},
  float=false,
  numbers=left,
  numberstyle=\tiny\color{gray},
  numbersep=10pt,
  xleftmargin=2em,
  keywordstyle=\color{blue},
  commentstyle=\color{green!60!black},
  stringstyle=\color{purple},
  backgroundcolor=\color{gray!5},
  showstringspaces=false,
  tabsize=2,
  captionpos=b,
  keepspaces=true,
  columns=flexible
}

\pgfplotsset{compat=1.18}
\usetikzlibrary{shapes,arrows,positioning,calc,patterns,decorations.pathmorphing,decorations.markings,arrows.meta}

% Color scheme
\definecolor{headcolor}{RGB}{0,102,204}
\definecolor{keycolor}{RGB}{220,20,60}
\definecolor{solutioncolor}{RGB}{34,139,34}
\definecolor{mnemoniccolor}{RGB}{148,0,211}
\definecolor{codecolor}{RGB}{0,0,100}

% Spacing
\setlength{\parskip}{3pt}
\setlist[itemize]{nosep}
\setlist[enumerate]{nosep}

% Title formatting
\titleformat{\section}{\Large\bfseries\color{headcolor}}{\thesection}{1em}{}
\titleformat{\subsection}{\large\bfseries\color{headcolor}}{\thesubsection}{1em}{}

% Pandoc tightlist compatibility
\providecommand{\tightlist}{%
  \setlength{\itemsep}{0pt}\setlength{\parskip}{0pt}}

% Pandoc longtable compatibility
\newcounter{none}
\def\thenone{}


% content/resources/templates/english-boxes.tex
% This file is currently empty - it exists to maintain consistency with the import structure.
% Add custom environments here if needed in the future.


\begin{document}

\begin{center}
{\Huge\bfseries\color{headcolor} Subject Name Solutions}\\[5pt]
{\LARGE 4331104 -- Summer 2024}\\[3pt]
{\large Semester 1 Study Material}\\[3pt]
{\normalsize\textit{Detailed Solutions and Explanations}}
\end{center}

\vspace{10pt}

\subsection*{Question 1(a) [3 marks]}\label{q1a}

\textbf{Draw and explain block diagram of communication system.}

\begin{solutionbox}

\begin{verbatim}
flowchart LR
    A[Information Source] {-{-} B[Transmitter]}
    B {-{-} C[Channel/Medium]}
    C {-{-} D[Receiver]}
    D {-{-} E[Destination]}
    F[Noise Source] {-{-} C}
\end{verbatim}

\begin{itemize}
\tightlist
\item
  \textbf{Information Source}: Generates message signal (voice, video,
  data)
\item
  \textbf{Transmitter}: Converts message to suitable form for
  transmission
\item
  \textbf{Channel}: Medium through which signal travels (wires, fiber,
  air)
\item
  \textbf{Receiver}: Extracts original message from received signal
\item
  \textbf{Destination}: End-user who receives the information
\end{itemize}

\end{solutionbox}
\begin{mnemonicbox}
``Information Travels Carefully Reaching
Destination''

\end{mnemonicbox}
\subsection*{Question 1(b) [4 marks]}\label{q1b}

\textbf{Explain applications of EM wave spectrum.}

\begin{solutionbox}

{\def\LTcaptype{none} % do not increment counter
\begin{longtable}[]{@{}
  >{\raggedright\arraybackslash}p{(\linewidth - 4\tabcolsep) * \real{0.3404}}
  >{\raggedright\arraybackslash}p{(\linewidth - 4\tabcolsep) * \real{0.3617}}
  >{\raggedright\arraybackslash}p{(\linewidth - 4\tabcolsep) * \real{0.2979}}@{}}
\toprule\noalign{}
\begin{minipage}[b]{\linewidth}\raggedright
Frequency Band
\end{minipage} & \begin{minipage}[b]{\linewidth}\raggedright
Frequency Range
\end{minipage} & \begin{minipage}[b]{\linewidth}\raggedright
Applications
\end{minipage} \\
\midrule\noalign{}
\endhead
\bottomrule\noalign{}
\endlastfoot
Radio waves & 3 kHz - 300 MHz & AM/FM broadcasting, maritime
communication \\
Microwaves & 300 MHz - 300 GHz & Radar, satellite communication,
microwave ovens \\
Infrared & 300 GHz - 400 THz & Remote controls, thermal imaging, optical
fibers \\
Visible light & 400 THz - 800 THz & Fiber optic communication,
photography \\
Ultraviolet & 800 THz - 30 PHz & Sterilization, authentication, water
purification \\
X-rays & 30 PHz - 30 EHz & Medical imaging, security scanning, material
analysis \\
Gamma rays & \textgreater30 EHz & Cancer treatment, food sterilization,
industrial inspection \\
\end{longtable}
}

\end{solutionbox}
\begin{mnemonicbox}
``Radio Makes Invisible Very eXtreme Gamma signals''

\end{mnemonicbox}
\subsection*{Question 1(c) [7 marks]}\label{q1c}

\textbf{State and explain external and internal noise.}

\begin{solutionbox}

{\def\LTcaptype{none} % do not increment counter
\begin{longtable}[]{@{}
  >{\raggedright\arraybackslash}p{(\linewidth - 4\tabcolsep) * \real{0.1579}}
  >{\raggedright\arraybackslash}p{(\linewidth - 4\tabcolsep) * \real{0.4211}}
  >{\raggedright\arraybackslash}p{(\linewidth - 4\tabcolsep) * \real{0.4211}}@{}}
\toprule\noalign{}
\begin{minipage}[b]{\linewidth}\raggedright
Type
\end{minipage} & \begin{minipage}[b]{\linewidth}\raggedright
External Noise
\end{minipage} & \begin{minipage}[b]{\linewidth}\raggedright
Internal Noise
\end{minipage} \\
\midrule\noalign{}
\endhead
\bottomrule\noalign{}
\endlastfoot
\textbf{Source} & Outside the communication system & Inside electronic
components \\
\textbf{Types} & Atmospheric, Space, Industrial, Man-made & Thermal,
Shot, Transit-time, Flicker \\
\textbf{Control} & Can be reduced by shielding, filtering & Reduced by
better components, cooling \\
\textbf{Examples} & Lightning, Solar radiation, Motor sparking &
Electron movement in resistors, semiconductors \\
\textbf{Nature} & Usually unpredictable, varying & More consistent and
quantifiable \\
\end{longtable}
}

\textbf{Diagram:}

\begin{center}
\textbf{Mermaid Diagram (Code)}
\begin{verbatim}
{Shaded}
{Highlighting}[]
graph TD
    A[Noise in Communication] {-{-}{} B[External Noise]}
    A {-{-}{} C[Internal Noise]}
    B {-{-}{} D[Atmospheric Noise]}
    B {-{-}{} E[Space Noise]}
    B {-{-}{} F[Industrial Noise]}
    B {-{-}{} G[Man{-}made Noise]}
    C {-{-}{} H[Thermal Noise]}
    C {-{-}{} I[Shot Noise]}
    C {-{-}{} J[Transit{-}time Noise]}
    C {-{-}{} K[Flicker Noise]}
{Highlighting}
{Shaded}
\end{verbatim}
\end{center}

\end{solutionbox}
\begin{mnemonicbox}
``External Environmental Sources Invade; Internal
Components Generate Noise''

\end{mnemonicbox}
\subsection*{Question 1(c) OR [7
marks]}\label{q1c}

\textbf{Draw and explain the block diagram of a Superheterodyne AM
receiver.}

\begin{solutionbox}

\begin{verbatim}
flowchart LR
    A[Antenna] {-{-} B[RF Amplifier]}
    B {-{-} C[Mixer]}
    D[Local Oscillator] {-{-} C}
    C {-{-} E[IF Amplifier]}
    E {-{-} F[Detector]}
    F {-{-} G[AF Amplifier]}
    G {-{-} H[Speaker]}
    I[AGC] {-{-} B}
    I {-{-} E}
    F {-{-} I}
\end{verbatim}

{\def\LTcaptype{none} % do not increment counter
\begin{longtable}[]{@{}
  >{\raggedright\arraybackslash}p{(\linewidth - 2\tabcolsep) * \real{0.4118}}
  >{\raggedright\arraybackslash}p{(\linewidth - 2\tabcolsep) * \real{0.5882}}@{}}
\toprule\noalign{}
\begin{minipage}[b]{\linewidth}\raggedright
Block
\end{minipage} & \begin{minipage}[b]{\linewidth}\raggedright
Function
\end{minipage} \\
\midrule\noalign{}
\endhead
\bottomrule\noalign{}
\endlastfoot
\textbf{RF Amplifier} & Amplifies weak radio signals and provides
selectivity \\
\textbf{Local Oscillator} & Generates frequency for mixing with incoming
signal \\
\textbf{Mixer} & Combines RF and local oscillator signals to produce
IF \\
\textbf{IF Amplifier} & Amplifies signal at fixed intermediate frequency
(455 kHz) \\
\textbf{Detector} & Extracts audio from modulated carrier
(demodulation) \\
\textbf{AF Amplifier} & Amplifies audio signal to drive speaker \\
\textbf{AGC} & Automatic Gain Control - maintains constant output
level \\
\end{longtable}
}

\end{solutionbox}
\begin{mnemonicbox}
``Radio Loves Making Interesting Detected Audio
Sounds''

\end{mnemonicbox}
\subsection*{Question 2(a) [3 marks]}\label{q2a}

\textbf{Define modulation. State types of modulation.}

\begin{solutionbox}

\textbf{Modulation}: Process of varying one or more properties of a
high-frequency carrier signal with a modulating signal containing
information.

\textbf{Types of Modulation:}

\begin{center}
\textbf{Mermaid Diagram (Code)}
\begin{verbatim}
{Shaded}
{Highlighting}[]
graph TD
    A[Modulation] {-{-}{} B[Analog Modulation]}
    A {-{-}{} C[Digital Modulation]}
    A {-{-}{} D[Pulse Modulation]}
    B {-{-}{} E[AM]}
    B {-{-}{} F[FM]}
    B {-{-}{} G[PM]}
    C {-{-}{} H[ASK]}
    C {-{-}{} I[FSK]}
    C {-{-}{} J[PSK]}
    D {-{-}{} K[PAM]}
    D {-{-}{} L[PWM]}
    D {-{-}{} M[PPM]}
    D {-{-}{} N[PCM]}
{Highlighting}
{Shaded}
\end{verbatim}
\end{center}

\end{solutionbox}
\begin{mnemonicbox}
``All Modulations Alter Properties: Frequency,
Amplitude, Phase''

\end{mnemonicbox}
\subsection*{Question 2(b) [4 marks]}\label{q2b}

\textbf{Define: Signal to noise ratio and Noise figure.}

\begin{solutionbox}

{\def\LTcaptype{none} % do not increment counter
\begin{longtable}[]{@{}
  >{\raggedright\arraybackslash}p{(\linewidth - 8\tabcolsep) * \real{0.2157}}
  >{\raggedright\arraybackslash}p{(\linewidth - 8\tabcolsep) * \real{0.2353}}
  >{\raggedright\arraybackslash}p{(\linewidth - 8\tabcolsep) * \real{0.1765}}
  >{\raggedright\arraybackslash}p{(\linewidth - 8\tabcolsep) * \real{0.1176}}
  >{\raggedright\arraybackslash}p{(\linewidth - 8\tabcolsep) * \real{0.2549}}@{}}
\toprule\noalign{}
\begin{minipage}[b]{\linewidth}\raggedright
Parameter
\end{minipage} & \begin{minipage}[b]{\linewidth}\raggedright
Definition
\end{minipage} & \begin{minipage}[b]{\linewidth}\raggedright
Formula
\end{minipage} & \begin{minipage}[b]{\linewidth}\raggedright
Unit
\end{minipage} & \begin{minipage}[b]{\linewidth}\raggedright
Significance
\end{minipage} \\
\midrule\noalign{}
\endhead
\bottomrule\noalign{}
\endlastfoot
\textbf{Signal to Noise Ratio (SNR)} & Ratio of signal power to noise
power & SNR = P\_signal / P\_noise & Expressed in dB & Higher value
indicates better signal quality \\
\textbf{Noise Figure (NF)} & Measure of degradation of SNR as signal
passes through system & NF = SNR\_input / SNR\_output & Expressed in dB
& Lower value indicates better performance \\
\end{longtable}
}

\end{solutionbox}
\begin{mnemonicbox}
``SNR Shows Necessary Reception; Noise Figure Finds
Fault''

\end{mnemonicbox}
\subsection*{Question 2(c) [7 marks]}\label{q2c}

\textbf{Compare PAM, PWM and PPM techniques.}

\begin{solutionbox}

{\def\LTcaptype{none} % do not increment counter
\begin{longtable}[]{@{}
  >{\raggedright\arraybackslash}p{(\linewidth - 6\tabcolsep) * \real{0.4231}}
  >{\raggedright\arraybackslash}p{(\linewidth - 6\tabcolsep) * \real{0.1923}}
  >{\raggedright\arraybackslash}p{(\linewidth - 6\tabcolsep) * \real{0.1923}}
  >{\raggedright\arraybackslash}p{(\linewidth - 6\tabcolsep) * \real{0.1923}}@{}}
\toprule\noalign{}
\begin{minipage}[b]{\linewidth}\raggedright
Parameter
\end{minipage} & \begin{minipage}[b]{\linewidth}\raggedright
PAM
\end{minipage} & \begin{minipage}[b]{\linewidth}\raggedright
PWM
\end{minipage} & \begin{minipage}[b]{\linewidth}\raggedright
PPM
\end{minipage} \\
\midrule\noalign{}
\endhead
\bottomrule\noalign{}
\endlastfoot
\textbf{Full Form} & Pulse Amplitude Modulation & Pulse Width Modulation
& Pulse Position Modulation \\
\textbf{Modulated Parameter} & Amplitude of pulses & Width/duration of
pulses & Position/timing of pulses \\
\textbf{Noise Immunity} & Poor & Good & Excellent \\
\textbf{Bandwidth} & Low & Medium & High \\
\textbf{Circuit Complexity} & Simple & Moderate & Complex \\
\textbf{Power Efficiency} & Poor & Good & Excellent \\
\textbf{Applications} & Simple data sampling & Motor control, power
regulation & Precision timing, optical communication \\
\end{longtable}
}

\textbf{Diagram:}

\begin{verbatim}
    Original:  ▁▁▁▁▁▁▁▁▁▁▁▁▁▁▁▁▁▁▁▁▁▁▁▁▁▁▁▁▁▁▁▁▁▁
                ⢠⣶⠀⣶⠀⣶⠀⣶⠀⣶⠀⣶⠀⣶⠀⣶⠀⣶⠀

    PAM:       ▁▁▁▁▁▂▂▁▁▁▁▁▁▁▁▁▄▄▁▁▁▁▂▂▂▁▁▁▁▁▂▂▁▁
                ⡇⠀⡇⠀⡇⠀⡇⠀⡇⠀⡇⠀⡇⠀⡇⠀⡇⠀⡇⠀

    PWM:       ▁▁▁█▁▁▁▁▁███▁▁▁▁▁▁█▁▁▁▁▁██▁▁▁▁▁█▁▁▁
                ⠀⠀⣿⣿⣿⣿⠀⠀⠀⣿⣿⣿⣿⣿⠀⠀⣿⣿⣿⣿⠀

    PPM:       ▁▁█▁▁▁▁▁█▁▁▁▁▁▁▁█▁▁▁▁▁▁█▁▁▁▁▁█▁▁▁▁▁
                ⠀⣿⠀⠀⠀⣿⠀⠀⠀⠀⣿⠀⠀⠀⠀⣿⠀⠀⠀⣿⠀
\end{verbatim}

\end{solutionbox}
\begin{mnemonicbox}
``Amplitude varies height, Width varies length,
Position varies timing''

\end{mnemonicbox}
\subsection*{Question 2(a) OR [3
marks]}\label{q2a}

\textbf{Differentiate between bit, symbol and Baud rate.}

\begin{solutionbox}

{\def\LTcaptype{none} % do not increment counter
\begin{longtable}[]{@{}
  >{\raggedright\arraybackslash}p{(\linewidth - 6\tabcolsep) * \real{0.3143}}
  >{\raggedright\arraybackslash}p{(\linewidth - 6\tabcolsep) * \real{0.1429}}
  >{\raggedright\arraybackslash}p{(\linewidth - 6\tabcolsep) * \real{0.2286}}
  >{\raggedright\arraybackslash}p{(\linewidth - 6\tabcolsep) * \real{0.3143}}@{}}
\toprule\noalign{}
\begin{minipage}[b]{\linewidth}\raggedright
Parameter
\end{minipage} & \begin{minipage}[b]{\linewidth}\raggedright
Bit
\end{minipage} & \begin{minipage}[b]{\linewidth}\raggedright
Symbol
\end{minipage} & \begin{minipage}[b]{\linewidth}\raggedright
Baud Rate
\end{minipage} \\
\midrule\noalign{}
\endhead
\bottomrule\noalign{}
\endlastfoot
\textbf{Definition} & Binary digit (0 or 1) & Group of bits & Number of
symbols transmitted per second \\
\textbf{Unit} & No unit & No unit & Symbols per second (Baud) \\
\textbf{Relationship} & Basic unit of digital information & Multiple
bits form one symbol & Baud rate \times bits per symbol = bit rate \\
\textbf{Example} & 0, 1 & In 4-QAM, each symbol represents 2 bits & 1200
baud means 1200 symbols per second \\
\end{longtable}
}

\end{solutionbox}
\begin{mnemonicbox}
``Bits Build Symbols, Bauds Show Speed''

\end{mnemonicbox}
\subsection*{Question 2(b) OR [4
marks]}\label{q2b}

\textbf{State advantages and disadvantage of SSB over DSB.}

\begin{solutionbox}

{\def\LTcaptype{none} % do not increment counter
\begin{longtable}[]{@{}
  >{\raggedright\arraybackslash}p{(\linewidth - 2\tabcolsep) * \real{0.4746}}
  >{\raggedright\arraybackslash}p{(\linewidth - 2\tabcolsep) * \real{0.5254}}@{}}
\toprule\noalign{}
\begin{minipage}[b]{\linewidth}\raggedright
Advantages of SSB over DSB
\end{minipage} & \begin{minipage}[b]{\linewidth}\raggedright
Disadvantages of SSB over DSB
\end{minipage} \\
\midrule\noalign{}
\endhead
\bottomrule\noalign{}
\endlastfoot
\textbf{Bandwidth}: Requires only half the bandwidth & \textbf{Circuit
Complexity}: More complex modulation and demodulation \\
\textbf{Power Efficiency}: Transmits only one sideband, saving power &
\textbf{Receiver Design}: Requires precise frequency synchronization \\
\textbf{Less Fading}: Reduced selective fading effects & \textbf{Low
Frequency Loss}: May lose low frequency components \\
\textbf{Less Interference}: Reduced adjacent channel interference &
\textbf{Cost}: More expensive implementation \\
\end{longtable}
}

\end{solutionbox}
\begin{mnemonicbox}
``SSB Saves Bandwidth Power but Costs Complex
Hardware''

\end{mnemonicbox}
\subsection*{Question 2(c) OR [7
marks]}\label{q2c}

\textbf{Compare Amplitude Modulation (AM) and Frequency Modulation
(FM).}

\begin{solutionbox}

{\def\LTcaptype{none} % do not increment counter
\begin{longtable}[]{@{}
  >{\raggedright\arraybackslash}p{(\linewidth - 4\tabcolsep) * \real{0.5238}}
  >{\raggedright\arraybackslash}p{(\linewidth - 4\tabcolsep) * \real{0.2381}}
  >{\raggedright\arraybackslash}p{(\linewidth - 4\tabcolsep) * \real{0.2381}}@{}}
\toprule\noalign{}
\begin{minipage}[b]{\linewidth}\raggedright
Parameter
\end{minipage} & \begin{minipage}[b]{\linewidth}\raggedright
AM
\end{minipage} & \begin{minipage}[b]{\linewidth}\raggedright
FM
\end{minipage} \\
\midrule\noalign{}
\endhead
\bottomrule\noalign{}
\endlastfoot
\textbf{Modulated Parameter} & Amplitude of carrier & Frequency of
carrier \\
\textbf{Bandwidth} & Narrow (2 \times highest modulating frequency) & Wide (2
\times (highest modulating frequency + deviation)) \\
\textbf{Noise Immunity} & Poor & Excellent \\
\textbf{Power Efficiency} & Poor (carrier contains most power) & Good \\
\textbf{Circuit Complexity} & Simple & Complex \\
\textbf{Quality} & Lower & Higher \\
\textbf{Applications} & Broadcasting (MW), Aircraft communication & FM
radio, TV sound, Mobile communications \\
\end{longtable}
}

\textbf{Diagram:}

\begin{verbatim}
    Carrier:    ⠀⣶⣶⣶⣶⣶⣶⣶⣶⣶⣶⣶⣶⣶⣶⣶⣶⣶⣶⣶⣶
                ⠀⠛⠛⠛⠛⠛⠛⠛⠛⠛⠛⠛⠛⠛⠛⠛⠛⠛⠛⠛⠛

    AM:         ⠀⢠⠆⢰⠆⢠⠆⢰⠆⠀⠀⠀⠠⠄⠠⠄⠠⠄⠠⠄⠀
                ⠀⠟⠀⠸⠀⠹⠀⠸⠀⠀⠀⠀⠀⠸⠀⠸⠀⠸⠀⠸⠀

    FM:         ⠀⣶⣿⣷⣾⣿⣷⣾⣿⣿⣿⣿⣿⣷⣾⣿⣷⣾⣿⣷⣿
                ⠀⠿⠿⠿⠿⠿⠿⠿⠿⠿⠿⠿⠿⠿⠿⠿⠿⠿⠿⠿⠿
\end{verbatim}

\end{solutionbox}
\begin{mnemonicbox}
``AM Alters strength, FM Fluctuates timing''

\end{mnemonicbox}
\subsection*{Question 3(a) [3 marks]}\label{q3a}

\textbf{Compare AM receiver with FM receiver.}

\begin{solutionbox}

{\def\LTcaptype{none} % do not increment counter
\begin{longtable}[]{@{}lll@{}}
\toprule\noalign{}
Parameter & AM Receiver & FM Receiver \\
\midrule\noalign{}
\endhead
\bottomrule\noalign{}
\endlastfoot
\textbf{IF Frequency} & 455 kHz & 10.7 MHz \\
\textbf{Detector} & Envelope detector & Discriminator/Ratio
detector/PLL \\
\textbf{Bandwidth} & Narrow (\pm5 kHz) & Wide (\pm75 kHz) \\
\textbf{Special Circuits} & Simple & Limiter, De-emphasis \\
\textbf{Complexity} & Simple & Complex \\
\end{longtable}
}

\end{solutionbox}
\begin{mnemonicbox}
``AM Accepts Minimal bandwidth; FM Features More
circuits''

\end{mnemonicbox}
\subsection*{Question 3(b) [4 marks]}\label{q3b}

\textbf{Define sampling? Explain types of sampling in brief.}

\begin{solutionbox}

\textbf{Sampling}: Process of converting continuous-time signal into
discrete-time signal by taking samples at regular intervals.

{\def\LTcaptype{none} % do not increment counter
\begin{longtable}[]{@{}
  >{\raggedright\arraybackslash}p{(\linewidth - 4\tabcolsep) * \real{0.3696}}
  >{\raggedright\arraybackslash}p{(\linewidth - 4\tabcolsep) * \real{0.2826}}
  >{\raggedright\arraybackslash}p{(\linewidth - 4\tabcolsep) * \real{0.3478}}@{}}
\toprule\noalign{}
\begin{minipage}[b]{\linewidth}\raggedright
Type of Sampling
\end{minipage} & \begin{minipage}[b]{\linewidth}\raggedright
Description
\end{minipage} & \begin{minipage}[b]{\linewidth}\raggedright
Characteristics
\end{minipage} \\
\midrule\noalign{}
\endhead
\bottomrule\noalign{}
\endlastfoot
\textbf{Ideal Sampling} & Instantaneous samples of the signal & Perfect
but theoretical, uses impulse function \\
\textbf{Natural Sampling} & Signal is sampled for short durations & Top
of pulses follow original signal \\
\textbf{Flat-top Sampling} & Samples held constant until next sample &
Creates staircase approximation, easier to implement \\
\end{longtable}
}

\textbf{Diagram:}

\begin{verbatim}
    Original:     ⣿⢿⣻⣽⣯⣿⣻⣽⣯⣿⣻⣽⣯⣿⣻⣽⣯⣿⣻⣽⣯⣿⣻⣽⣯⣿⣻

    Ideal:        ⠀⠈⠀⠀⠀⠀⠀⠈⠀⠀⠀⠀⠀⠈⠀⠀⠀⠀⠀⠈⠀⠀⠀⠀⠀⠈⠀
                  ⠀⣼⠀⠀⠀⠀⠀⣼⠀⠀⠀⠀⠀⣼⠀⠀⠀⠀⠀⣼⠀⠀⠀⠀⠀⣼⠀

    Natural:      ⠀⣠⠀⠀⠀⠀⠀⣠⠀⠀⠀⠀⠀⣠⠀⠀⠀⠀⠀⣠⠀⠀⠀⠀⠀⣠⠀
                  ⠀⠇⠀⠀⠀⠀⠀⠇⠀⠀⠀⠀⠀⠇⠀⠀⠀⠀⠀⠇⠀⠀⠀⠀⠀⠇⠀

    Flat{-top:     ⠀⣤⠀⠀⠀⠀⠀⠤⠀⠀⠀⠀⠀⣤⠀⠀⠀⠀⠀⠤⠀⠀⠀⠀⠀⣤⠀}
                  ⠀⠀⠀⠀⠀⠀⠀⠀⠀⠀⠀⠀⠀⠀⠀⠀⠀⠀⠀⠀⠀⠀⠀⠀⠀⠀⠀
\end{verbatim}

\end{solutionbox}
\begin{mnemonicbox}
``Ideal takes Instants, Natural follows Nicely, Flat
stays Fixed''

\end{mnemonicbox}
\subsection*{Question 3(c) [7 marks]}\label{q3c}

\textbf{Draw and explain the block diagram of FM receiver. What is the
use of Limiter in FM receiver?}

\begin{solutionbox}

\begin{verbatim}
flowchart LR
    A[Antenna] {-{-} B[RF Amplifier]}
    B {-{-} C[Mixer]}
    D[Local Oscillator] {-{-} C}
    C {-{-} E[IF Amplifier]}
    E {-{-} F[Limiter]}
    F {-{-} G[Discriminator]}
    G {-{-} H[De{-}emphasis]}
    H {-{-} I[AF Amplifier]}
    I {-{-} J[Speaker]}
\end{verbatim}

{\def\LTcaptype{none} % do not increment counter
\begin{longtable}[]{@{}
  >{\raggedright\arraybackslash}p{(\linewidth - 2\tabcolsep) * \real{0.4118}}
  >{\raggedright\arraybackslash}p{(\linewidth - 2\tabcolsep) * \real{0.5882}}@{}}
\toprule\noalign{}
\begin{minipage}[b]{\linewidth}\raggedright
Block
\end{minipage} & \begin{minipage}[b]{\linewidth}\raggedright
Function
\end{minipage} \\
\midrule\noalign{}
\endhead
\bottomrule\noalign{}
\endlastfoot
\textbf{RF Amplifier} & Amplifies weak RF signal and provides
selectivity \\
\textbf{Mixer/Local Oscillator} & Converts RF to IF (10.7 MHz) \\
\textbf{IF Amplifier} & Provides gain and selectivity at fixed
frequency \\
\textbf{Limiter} & Removes amplitude variations, preserves frequency
variations \\
\textbf{Discriminator} & Converts frequency variations to amplitude
variations \\
\textbf{De-emphasis} & Reduces high-frequency noise \\
\textbf{AF Amplifier} & Amplifies recovered audio for speaker \\
\end{longtable}
}

\textbf{Limiter Function}: Removes amplitude variations from the FM
signal before demodulation to ensure noise immunity, as information in
FM is contained in frequency variations, not amplitude.

\end{solutionbox}
\begin{mnemonicbox}
``Radio Mixers Increase Frequency; Limiters
Discriminate Audio Sound''

\end{mnemonicbox}
\subsection*{Question 3(a) OR [3
marks]}\label{q3a}

\textbf{Describe the concept of single side band (SSB) transmission.}

\begin{solutionbox}

\textbf{Single Sideband (SSB) Transmission}: Technique where only one
sideband (upper or lower) is transmitted while suppressing the carrier
and other sideband.

\begin{center}
\textbf{Mermaid Diagram (Code)}
\begin{verbatim}
{Shaded}
{Highlighting}[]
graph LR
    A[AM Signal] {-{-}{} B[DSBFC]}
    A {-{-}{} C[DSBSC]}
    A {-{-}{} D[SSB]}
    D {-{-}{} E[USB]}
    D {-{-}{} F[LSB]}
{Highlighting}
{Shaded}
\end{verbatim}
\end{center}

\begin{itemize}
\tightlist
\item
  \textbf{Bandwidth}: Requires only half the bandwidth (fc \pm fm)
\item
  \textbf{Power Efficiency}: More efficient as power concentrated in one
  sideband
\item
  \textbf{Types}: USB (Upper Sideband) and LSB (Lower Sideband)
\end{itemize}

\end{solutionbox}
\begin{mnemonicbox}
``SSB Saves Spectrum Bandwidth''

\end{mnemonicbox}
\subsection*{Question 3(b) OR [4
marks]}\label{q3b}

\textbf{Explain pre-emphasis \& de-emphasis circuit.}

\begin{solutionbox}

{\def\LTcaptype{none} % do not increment counter
\begin{longtable}[]{@{}
  >{\raggedright\arraybackslash}p{(\linewidth - 4\tabcolsep) * \real{0.2895}}
  >{\raggedright\arraybackslash}p{(\linewidth - 4\tabcolsep) * \real{0.3684}}
  >{\raggedright\arraybackslash}p{(\linewidth - 4\tabcolsep) * \real{0.3421}}@{}}
\toprule\noalign{}
\begin{minipage}[b]{\linewidth}\raggedright
Parameter
\end{minipage} & \begin{minipage}[b]{\linewidth}\raggedright
Pre-emphasis
\end{minipage} & \begin{minipage}[b]{\linewidth}\raggedright
De-emphasis
\end{minipage} \\
\midrule\noalign{}
\endhead
\bottomrule\noalign{}
\endlastfoot
\textbf{Location} & Transmitter & Receiver \\
\textbf{Circuit Type} & High-pass RC network & Low-pass RC network \\
\textbf{Function} & Boosts high frequencies before transmission &
Attenuates high frequencies after reception \\
\textbf{Purpose} & Improves SNR for high frequencies & Restores original
frequency response \\
\end{longtable}
}

\textbf{Circuit Diagram:}

\begin{verbatim}
Pre{-emphasis:                De{-}emphasis:}
    
    R                           R
  ┌───┐                       ┌───┐
──┤   ├──┬───────          ───┤   ├───┬───────
  └───┘  │                    └───┘   │
         │                            │
         ⊥C                           ⊥C
         │                            │
         └───────                     └───────
\end{verbatim}

\end{solutionbox}
\begin{mnemonicbox}
``Pre Pushes highs, De Drops them''

\end{mnemonicbox}
\subsection*{Question 3(c) OR [7
marks]}\label{q3c}

\textbf{Illustrate generation of FM signal using Phase lock loop
technique.}

\begin{solutionbox}

\begin{verbatim}
flowchart LR
    A[Modulating Signal] {-{-} B[Loop Filter]}
    B {-{-} C[VCO]}
    C {-{-} D[FM Output]}
    C {-{-} E[Phase Detector]}
    F[Reference Oscillator] {-{-} E}
    E {-{-} B}
\end{verbatim}

{\def\LTcaptype{none} % do not increment counter
\begin{longtable}[]{@{}
  >{\raggedright\arraybackslash}p{(\linewidth - 2\tabcolsep) * \real{0.5238}}
  >{\raggedright\arraybackslash}p{(\linewidth - 2\tabcolsep) * \real{0.4762}}@{}}
\toprule\noalign{}
\begin{minipage}[b]{\linewidth}\raggedright
Component
\end{minipage} & \begin{minipage}[b]{\linewidth}\raggedright
Function
\end{minipage} \\
\midrule\noalign{}
\endhead
\bottomrule\noalign{}
\endlastfoot
\textbf{Phase Detector} & Compares reference and VCO signals, generates
error voltage \\
\textbf{Loop Filter} & Filters error voltage and combines with
modulating signal \\
\textbf{VCO (Voltage Controlled Oscillator)} & Generates frequency based
on control voltage \\
\textbf{Reference Oscillator} & Provides stable reference frequency \\
\end{longtable}
}

\textbf{Working Process:}

\begin{enumerate}
\tightlist
\item
  Modulating signal is applied to loop filter
\item
  VCO frequency shifts proportional to modulating signal
\item
  Phase detector generates error signal
\item
  Loop maintains lock while allowing frequency modulation
\item
  Output of VCO is the FM signal
\end{enumerate}

\end{solutionbox}
\begin{mnemonicbox}
``Phase Locks, Voltage Controls, Frequency
Modulates''

\end{mnemonicbox}
\subsection*{Question 4(a) [3 marks]}\label{q4a}

\textbf{Explain quantization process and its importance.}

\begin{solutionbox}

\textbf{Quantization}: Process of mapping continuous amplitude values to
a finite set of discrete levels in analog-to-digital conversion.

{\def\LTcaptype{none} % do not increment counter
\begin{longtable}[]{@{}
  >{\raggedright\arraybackslash}p{(\linewidth - 2\tabcolsep) * \real{0.3810}}
  >{\raggedright\arraybackslash}p{(\linewidth - 2\tabcolsep) * \real{0.6190}}@{}}
\toprule\noalign{}
\begin{minipage}[b]{\linewidth}\raggedright
Aspect
\end{minipage} & \begin{minipage}[b]{\linewidth}\raggedright
Description
\end{minipage} \\
\midrule\noalign{}
\endhead
\bottomrule\noalign{}
\endlastfoot
\textbf{Process} & Dividing amplitude range into fixed levels and
assigning digital values \\
\textbf{Types} & Uniform (equal steps) and Non-uniform (variable
steps) \\
\textbf{Error} & Difference between actual and quantized value
(quantization noise) \\
\end{longtable}
}

\textbf{Importance}:

\begin{itemize}
\tightlist
\item
  Enables digital representation of analog signals
\item
  Determines resolution and accuracy of digital signal
\item
  Affects signal-to-noise ratio in digital systems
\end{itemize}

\end{solutionbox}
\begin{mnemonicbox}
``Quantization Creates Digital from Analog''

\end{mnemonicbox}
\subsection*{Question 4(b) [4 marks]}\label{q4b}

\textbf{Explain different characteristics of Radio receiver.}

\begin{solutionbox}

{\def\LTcaptype{none} % do not increment counter
\begin{longtable}[]{@{}
  >{\raggedright\arraybackslash}p{(\linewidth - 4\tabcolsep) * \real{0.3902}}
  >{\raggedright\arraybackslash}p{(\linewidth - 4\tabcolsep) * \real{0.2927}}
  >{\raggedright\arraybackslash}p{(\linewidth - 4\tabcolsep) * \real{0.3171}}@{}}
\toprule\noalign{}
\begin{minipage}[b]{\linewidth}\raggedright
Characteristic
\end{minipage} & \begin{minipage}[b]{\linewidth}\raggedright
Definition
\end{minipage} & \begin{minipage}[b]{\linewidth}\raggedright
Significance
\end{minipage} \\
\midrule\noalign{}
\endhead
\bottomrule\noalign{}
\endlastfoot
\textbf{Sensitivity} & Ability to receive weak signals & Determines
reception range \\
\textbf{Selectivity} & Ability to separate adjacent channels & Prevents
interference \\
\textbf{Fidelity} & Accuracy of reproduction & Determines sound
quality \\
\textbf{Image Rejection} & Ability to reject image frequency & Prevents
unwanted reception \\
\end{longtable}
}

\textbf{Diagram:}

\begin{center}
\textbf{Mermaid Diagram (Code)}
\begin{verbatim}
{Shaded}
{Highlighting}[]
graph TD
    A[Radio Receiver Characteristics] {-{-}{} B[Sensitivity]}
    A {-{-}{} C[Selectivity]}
    A {-{-}{} D[Fidelity]}
    A {-{-}{} E[Image Rejection]}
    B {-{-}{} F[Measured in μV]}
    C {-{-}{} G[Bandwidth and Q factor]}
    D {-{-}{} H[Frequency response]}
    E {-{-}{} I[Image ratio]}
{Highlighting}
{Shaded}
\end{verbatim}
\end{center}

\end{solutionbox}
\begin{mnemonicbox}
``Sensitive Selection Faithfully Images''

\end{mnemonicbox}
\subsection*{Question 4(c) [7 marks]}\label{q4c}

\textbf{Draw and explain the block diagram of PCM transmitter and
receiver.}

\begin{solutionbox}

\textbf{PCM Transmitter:}

\begin{verbatim}
flowchart LR
    A[Input Signal] {-{-} B[Anti{-}aliasing Filter]}
    B {-{-} C[Sample \& Hold]}
    C {-{-} D[Quantizer]}
    D {-{-} E[Encoder]}
    E {-{-} F[Line Coder]}
    F {-{-} G[Transmission Channel]}
\end{verbatim}

\textbf{PCM Receiver:}

\begin{verbatim}
flowchart LR
    A[Received Signal] {-{-} B[Line Decoder]}
    B {-{-} C[Regenerative Repeater]}
    C {-{-} D[Decoder]}
    D {-{-} E[Reconstruction Filter]}
    E {-{-} F[Output Signal]}
\end{verbatim}

{\def\LTcaptype{none} % do not increment counter
\begin{longtable}[]{@{}
  >{\raggedright\arraybackslash}p{(\linewidth - 2\tabcolsep) * \real{0.4118}}
  >{\raggedright\arraybackslash}p{(\linewidth - 2\tabcolsep) * \real{0.5882}}@{}}
\toprule\noalign{}
\begin{minipage}[b]{\linewidth}\raggedright
Block
\end{minipage} & \begin{minipage}[b]{\linewidth}\raggedright
Function
\end{minipage} \\
\midrule\noalign{}
\endhead
\bottomrule\noalign{}
\endlastfoot
\textbf{Anti-aliasing Filter} & Limits input bandwidth to prevent
aliasing \\
\textbf{Sample \& Hold} & Converts continuous signal to discrete-time
samples \\
\textbf{Quantizer} & Converts sample amplitudes to discrete levels \\
\textbf{Encoder} & Converts quantized values to binary code \\
\textbf{Line Coder} & Formats binary data for transmission \\
\textbf{Decoder} & Converts binary code back to quantized values \\
\textbf{Reconstruction Filter} & Smooths the stepped output to recover
original signal \\
\end{longtable}
}

\end{solutionbox}
\begin{mnemonicbox}
``Sample, Quantize, Encode, Transmit; Decode,
Reconstruct, Output''

\end{mnemonicbox}
\subsection*{Question 4(a) OR [3
marks]}\label{q4a}

\textbf{Compare Natural and Flat top sampling.}

\begin{solutionbox}

{\def\LTcaptype{none} % do not increment counter
\begin{longtable}[]{@{}
  >{\raggedright\arraybackslash}p{(\linewidth - 4\tabcolsep) * \real{0.2292}}
  >{\raggedright\arraybackslash}p{(\linewidth - 4\tabcolsep) * \real{0.3750}}
  >{\raggedright\arraybackslash}p{(\linewidth - 4\tabcolsep) * \real{0.3958}}@{}}
\toprule\noalign{}
\begin{minipage}[b]{\linewidth}\raggedright
Parameter
\end{minipage} & \begin{minipage}[b]{\linewidth}\raggedright
Natural Sampling
\end{minipage} & \begin{minipage}[b]{\linewidth}\raggedright
Flat-top Sampling
\end{minipage} \\
\midrule\noalign{}
\endhead
\bottomrule\noalign{}
\endlastfoot
\textbf{Shape} & Top of pulses follow input signal & Constant amplitude
during sampling interval \\
\textbf{Implementation} & More difficult (analog switch) & Easier
(sample and hold circuit) \\
\textbf{Spectrum} & Less harmonics & More harmonics \\
\textbf{Reconstruction} & Easier, more accurate & Requires compensation
for distortion \\
\end{longtable}
}

\textbf{Diagram:}

\begin{verbatim}
    Signal:       ⣿⢿⣻⣽⣯⣿⣻⣽⣯⣿⣻⣽⣯⣿⣻⣽⣯⣿⣻⣽⣯⣿⣻⣽⣯⣿⣻

    Natural:      ⠀⣠⠀⠀⠀⠀⠀⣠⠀⠀⠀⠀⠀⣠⠀⠀⠀⠀⠀⣠⠀⠀⠀⠀⠀⣠⠀
                  ⠀⠇⠀⠀⠀⠀⠀⠇⠀⠀⠀⠀⠀⠇⠀⠀⠀⠀⠀⠇⠀⠀⠀⠀⠀⠇⠀

    Flat{-top:     ⠀⣤⠀⠀⠀⠀⠀⠤⠀⠀⠀⠀⠀⣤⠀⠀⠀⠀⠀⠤⠀⠀⠀⠀⠀⣤⠀}
                  ⠀⠀⠀⠀⠀⠀⠀⠀⠀⠀⠀⠀⠀⠀⠀⠀⠀⠀⠀⠀⠀⠀⠀⠀⠀⠀⠀
\end{verbatim}

\end{solutionbox}
\begin{mnemonicbox}
``Natural Follows, Flat Freezes''

\end{mnemonicbox}
\subsection*{Question 4(b) OR [4
marks]}\label{q4b}

\textbf{Explain Diode Detector circuit.}

\begin{solutionbox}

\textbf{Diode Detector Circuit}: Used for demodulation of AM signals by
extracting the envelope of the modulated wave.

\begin{verbatim}
                 D
           ┌─────▶│──┬────────
Input ─────┤         │       │
           └─────────┤       │  Output
                     │       ├───────
                     ⊥C     R│
                     │       │
                     └───────┘
\end{verbatim}

{\def\LTcaptype{none} % do not increment counter
\begin{longtable}[]{@{}ll@{}}
\toprule\noalign{}
Component & Function \\
\midrule\noalign{}
\endhead
\bottomrule\noalign{}
\endlastfoot
\textbf{Diode (D)} & Rectifies the AM signal, passes only positive
half \\
\textbf{Capacitor (C)} & Charges to peak value, smooths out carrier \\
\textbf{Resistor (R)} & Controls discharge time of capacitor \\
\end{longtable}
}

\textbf{Working}:

\begin{enumerate}
\tightlist
\item
  Diode rectifies AM signal
\item
  Capacitor charges to peak value
\item
  RC time constant allows capacitor to follow envelope
\item
  Output follows the original modulating signal
\end{enumerate}

\end{solutionbox}
\begin{mnemonicbox}
``Diode Detects, Capacitor Captures''

\end{mnemonicbox}
\subsection*{Question 4(c) OR [7
marks]}\label{q4c}

\textbf{Draw and explain the block diagram of Delta Modulation.}

\begin{solutionbox}

\textbf{Delta Modulation Transmitter:}

\begin{verbatim}
flowchart LR
    A[Input Signal] {-{-} B[Comparator]}
    B {-{-} C[1{-}bit Quantizer]}
    C {-{-} D[Transmission Channel]}
    C {-{-} E[Integrator]}
    E {-{-} B}
    D {-{-} F[To Receiver]}
\end{verbatim}

\textbf{Delta Modulation Receiver:}

\begin{verbatim}
flowchart LR
    A[Received Signal] {-{-} B[Integrator]}
    B {-{-} C[Low{-}pass Filter]}
    C {-{-} D[Output Signal]}
\end{verbatim}

{\def\LTcaptype{none} % do not increment counter
\begin{longtable}[]{@{}
  >{\raggedright\arraybackslash}p{(\linewidth - 2\tabcolsep) * \real{0.5238}}
  >{\raggedright\arraybackslash}p{(\linewidth - 2\tabcolsep) * \real{0.4762}}@{}}
\toprule\noalign{}
\begin{minipage}[b]{\linewidth}\raggedright
Component
\end{minipage} & \begin{minipage}[b]{\linewidth}\raggedright
Function
\end{minipage} \\
\midrule\noalign{}
\endhead
\bottomrule\noalign{}
\endlastfoot
\textbf{Comparator} & Compares input with predicted value \\
\textbf{1-bit Quantizer} & Outputs binary 1 if input \textgreater{}
predicted, 0 if input \textless{} predicted \\
\textbf{Integrator} & Generates predicted value by integrating previous
output \\
\textbf{Low-pass Filter} & Smooths stepped output to recover original
signal \\
\end{longtable}
}

\textbf{Limitations}:

\begin{itemize}
\tightlist
\item
  \textbf{Slope Overload}: Occurs when signal changes faster than step
  size can track
\item
  \textbf{Granular Noise}: Occurs during idle or constant parts of
  signal
\end{itemize}

\end{solutionbox}
\begin{mnemonicbox}
``Delta Detects Differences, Integrator Increments''

\end{mnemonicbox}
\subsection*{Question 5(a) [3 marks]}\label{q5a}

\textbf{Illustrate working of DPCM.}

\begin{solutionbox}

\textbf{DPCM (Differential Pulse Code Modulation)}: Encodes the
difference between current sample and predicted value.

\begin{verbatim}
flowchart LR
    A[Input] {-{-} B[Sampler]}
    B {-{-} C[Difference Generator]}
    D[Predictor] {-{-} C}
    C {-{-} E[Quantizer]}
    E {-{-} F[Encoder]}
    F {-{-} G[Transmission]}
    E {-{-} H[Inverse Quantizer]}
    H {-{-} D}
\end{verbatim}

\begin{itemize}
\tightlist
\item
  \textbf{Predictor}: Estimates current sample based on previous samples
\item
  \textbf{Difference}: Only difference between actual and predicted is
  encoded
\item
  \textbf{Advantage}: Reduces bit rate compared to PCM by exploiting
  signal correlation
\end{itemize}

\end{solutionbox}
\begin{mnemonicbox}
``Differences Predicted Create Minimized bits''

\end{mnemonicbox}
\subsection*{Question 5(b) [4 marks]}\label{q5b}

\textbf{Illustrate Adaptive Delta Modulation.}

\begin{solutionbox}

\textbf{Adaptive Delta Modulation (ADM)}: Improved version of DM that
varies step size based on signal characteristics.

\begin{verbatim}
flowchart LR
    A[Input] {-{-} B[Comparator]}
    B {-{-} C[Pulse Generator]}
    C {-{-} D[Step Size Adapter]}
    D {-{-} E[Integrator]}
    E {-{-} B}
    C {-{-} F[Transmission]}
\end{verbatim}

{\def\LTcaptype{none} % do not increment counter
\begin{longtable}[]{@{}
  >{\raggedright\arraybackslash}p{(\linewidth - 2\tabcolsep) * \real{0.5238}}
  >{\raggedright\arraybackslash}p{(\linewidth - 2\tabcolsep) * \real{0.4762}}@{}}
\toprule\noalign{}
\begin{minipage}[b]{\linewidth}\raggedright
Component
\end{minipage} & \begin{minipage}[b]{\linewidth}\raggedright
Function
\end{minipage} \\
\midrule\noalign{}
\endhead
\bottomrule\noalign{}
\endlastfoot
\textbf{Comparator} & Compares input with approximated signal \\
\textbf{Step Size Adapter} & Adjusts step size based on consecutive bit
patterns \\
\textbf{Integrator} & Creates approximated signal from step-adjusted
pulses \\
\textbf{Pulse Generator} & Generates binary output based on
comparator \\
\end{longtable}
}

\textbf{Operation}:

\begin{enumerate}
\tightlist
\item
  If multiple 1's detected: increase step size to avoid slope overload
\item
  If multiple 0's detected: increase step size to track falling signal
\item
  If alternating 1's and 0's: decrease step size to reduce granular
  noise
\end{enumerate}

\end{solutionbox}
\begin{mnemonicbox}
``Adapting Delta Makes Slopes Trackable''

\end{mnemonicbox}
\subsection*{Question 5(c) [7 marks]}\label{q5c}

\textbf{Illustrate TDM frame.}

\begin{solutionbox}

\textbf{TDM (Time Division Multiplexing) Frame}: Structure used to
combine multiple signals by assigning time slots.

\textbf{Frame Structure:}

\begin{verbatim}
    ┌─────────────────────────────────────────────────────┐
    │                     TDM FRAME                       │
    ├───────┬───────┬───────┬───────┬───────┬─────────────┤
    │Frame  │ CH 1  │ CH 2  │ CH 3  │ CH 4  │    ...      │
    │Sync   │Sample │Sample │Sample │Sample │    CH N     │
    ├───────┼───────┼───────┼───────┼───────┼─────────────┤
    │       │       │       │       │       │             │
    └───────┴───────┴───────┴───────┴───────┴─────────────┘
              TS1     TS2     TS3     TS4        TSn
\end{verbatim}

{\def\LTcaptype{none} % do not increment counter
\begin{longtable}[]{@{}ll@{}}
\toprule\noalign{}
Component & Description \\
\midrule\noalign{}
\endhead
\bottomrule\noalign{}
\endlastfoot
\textbf{Frame Sync} & Pattern to identify frame boundaries \\
\textbf{Channel Sample} & Data from individual channel \\
\textbf{Time Slot (TS)} & Dedicated period for each channel \\
\textbf{Frame Duration} & Inversely proportional to sampling rate \\
\end{longtable}
}

\textbf{TDM Hierarchy:}

\begin{center}
\textbf{Mermaid Diagram (Code)}
\begin{verbatim}
{Shaded}
{Highlighting}[]
graph LR
    A[Primary Multiplexing 2.048 Mbps] {-{-}{} B[Secondary Multiplexing 8.448 Mbps]}
    B {-{-}{} C[Tertiary Multiplexing 34.368 Mbps]}
    C {-{-}{} D[Quaternary Multiplexing 139.264 Mbps]}
{Highlighting}
{Shaded}
\end{verbatim}
\end{center}

\end{solutionbox}
\begin{mnemonicbox}
``Frames Synchronize Time Slots During Multiplexing''

\end{mnemonicbox}
\subsection*{Question 5(a) OR [3
marks]}\label{q5a}

\textbf{State difference between DM and ADM.}

\begin{solutionbox}

{\def\LTcaptype{none} % do not increment counter
\begin{longtable}[]{@{}
  >{\raggedright\arraybackslash}p{(\linewidth - 4\tabcolsep) * \real{0.1642}}
  >{\raggedright\arraybackslash}p{(\linewidth - 4\tabcolsep) * \real{0.3582}}
  >{\raggedright\arraybackslash}p{(\linewidth - 4\tabcolsep) * \real{0.4776}}@{}}
\toprule\noalign{}
\begin{minipage}[b]{\linewidth}\raggedright
Parameter
\end{minipage} & \begin{minipage}[b]{\linewidth}\raggedright
Delta Modulation (DM)
\end{minipage} & \begin{minipage}[b]{\linewidth}\raggedright
Adaptive Delta Modulation (ADM)
\end{minipage} \\
\midrule\noalign{}
\endhead
\bottomrule\noalign{}
\endlastfoot
\textbf{Step Size} & Fixed step size & Variable step size \\
\textbf{Slope Overload} & Common problem & Reduced by adaptive step
size \\
\textbf{Granular Noise} & High during slow variations & Reduced by
adaptive step size \\
\textbf{Circuit Complexity} & Simpler & More complex \\
\textbf{Signal Quality} & Lower & Higher \\
\end{longtable}
}

\end{solutionbox}
\begin{mnemonicbox}
``DM's Fixed Steps; ADM Adapts''

\end{mnemonicbox}
\subsection*{Question 5(b) OR [4
marks]}\label{q5b}

\textbf{Explain the need of line coding. Explain AMI technique.}

\begin{solutionbox}

\textbf{Need for Line Coding:}

\begin{itemize}
\tightlist
\item
  \textbf{DC Component}: To eliminate DC component for AC-coupled
  systems
\item
  \textbf{Synchronization}: To provide timing information for clock
  recovery
\item
  \textbf{Error Detection}: To enable detection of transmission errors
\item
  \textbf{Spectral Efficiency}: To shape signal spectrum for efficient
  bandwidth use
\item
  \textbf{Noise Immunity}: To provide resistance against channel noise
\end{itemize}

\textbf{AMI (Alternate Mark Inversion) Technique:}

{\def\LTcaptype{none} % do not increment counter
\begin{longtable}[]{@{}
  >{\raggedright\arraybackslash}p{(\linewidth - 2\tabcolsep) * \real{0.4583}}
  >{\raggedright\arraybackslash}p{(\linewidth - 2\tabcolsep) * \real{0.5417}}@{}}
\toprule\noalign{}
\begin{minipage}[b]{\linewidth}\raggedright
Parameter
\end{minipage} & \begin{minipage}[b]{\linewidth}\raggedright
Description
\end{minipage} \\
\midrule\noalign{}
\endhead
\bottomrule\noalign{}
\endlastfoot
\textbf{Encoding Rule} & Binary 0 \rightarrow Zero voltage, Binary 1 \rightarrow Alternating
positive/negative voltage \\
\textbf{DC Component} & No DC component (balanced code) \\
\textbf{Error Detection} & Can detect violations in alternating
pattern \\
\textbf{Bandwidth} & Requires less bandwidth than NRZ codes \\
\end{longtable}
}

\textbf{Diagram:}

\begin{verbatim}
    Binary:   1   0   1   1   0   0   1   0   1   0   1   1

    AMI:      ▄   \_   ▀   ▄   \_   \_   ▀   \_   ▄   \_   ▀   ▄
              ┌───┐   ┌───┐       ┌───┐   ┌───┐   ┌───┐
              │   │   │   │       │   │   │   │   │   │
    ──────────┘   └───┘   └───────┘   └───┘   └───┘   └────
                      │       │           │       │
                      └───────┘           └───────┘
\end{verbatim}

\end{solutionbox}
\begin{mnemonicbox}
``Alternating Marks Invert Polarity''

\end{mnemonicbox}
\subsection*{Question 5(c) OR [7
marks]}\label{q5c}

\textbf{Draw and explain block diagram of basic PCM-TDM system.}

\begin{solutionbox}

\begin{verbatim}
flowchart TD
    subgraph "PCM{-TDM Transmitter"}
    A1[Channel 1] {-{-} B1[Low{-}pass Filter]}
    A2[Channel 2] {-{-} B2[Low{-}pass Filter]}
    A3[Channel 3] {-{-} B3[Low{-}pass Filter]}
    B1 {-{-} C1[Sample \& Hold]}
    B2 {-{-} C2[Sample \& Hold]}
    B3 {-{-} C3[Sample \& Hold]}
    C1 {-{-} D[Multiplexer]}
    C2 {-{-} D}
    C3 {-{-} D}
    D {-{-} E[Quantizer]}
    E {-{-} F[Encoder]}
    F {-{-} G[Line Coder]}
    end
    
    G {-{-} H[Transmission Channel]}
    
    subgraph "PCM{-TDM Receiver"}
    H {-{-} I[Line Decoder]}
    I {-{-} J[Regenerator]}
    J {-{-} K[Decoder]}
    K {-{-} L[Demultiplexer]}
    L {-{-} M1[Hold Circuit]}
    L {-{-} M2[Hold Circuit]}
    L {-{-} M3[Hold Circuit]}
    M1 {-{-} N1[Low{-}pass Filter]}
    M2 {-{-} N2[Low{-}pass Filter]}
    M3 {-{-} N3[Low{-}pass Filter]}
    N1 {-{-} O1[Channel 1]}
    N2 {-{-} O2[Channel 2]}
    N3 {-{-} O3[Channel 3]}
    end
\end{verbatim}

{\def\LTcaptype{none} % do not increment counter
\begin{longtable}[]{@{}
  >{\raggedright\arraybackslash}p{(\linewidth - 2\tabcolsep) * \real{0.4118}}
  >{\raggedright\arraybackslash}p{(\linewidth - 2\tabcolsep) * \real{0.5882}}@{}}
\toprule\noalign{}
\begin{minipage}[b]{\linewidth}\raggedright
Block
\end{minipage} & \begin{minipage}[b]{\linewidth}\raggedright
Function
\end{minipage} \\
\midrule\noalign{}
\endhead
\bottomrule\noalign{}
\endlastfoot
\textbf{Low-pass Filter (Input)} & Limits bandwidth to satisfy sampling
theorem \\
\textbf{Sample \& Hold} & Captures instantaneous values of analog
signals \\
\textbf{Multiplexer} & Combines samples from different channels into a
single stream \\
\textbf{Quantizer} & Assigns discrete levels to sampled values \\
\textbf{Encoder} & Converts quantized values to binary code \\
\textbf{Line Coder} & Formats binary data for transmission \\
\textbf{Regenerator} & Restores signal degraded by noise and
attenuation \\
\textbf{Decoder} & Converts binary code back to quantized values \\
\textbf{Demultiplexer} & Separates combined signal back into individual
channels \\
\textbf{Hold Circuit} & Maintains sample value until next sample
arrives \\
\textbf{Low-pass Filter (Output)} & Reconstructs original signal by
removing sampling harmonics \\
\end{longtable}
}

\end{solutionbox}
\begin{mnemonicbox}
``Multiple Channels Sample, Quantize, Encode; Decode,
Demultiplex, Filter''

\end{mnemonicbox}

\end{document}
