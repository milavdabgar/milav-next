\documentclass[10pt,a4paper]{article}

% content/resources/templates/preamble.tex
\usepackage[margin=0.6in]{geometry}
\author{Milav Dabgar}
\usepackage{amsmath,amssymb,amsthm}
\usepackage{booktabs}
\usepackage{multirow}
\usepackage{xcolor}
\usepackage{tcolorbox}
\tcbuselibrary{breakable,skins}
\usepackage[colorlinks=true,linkcolor=blue]{hyperref}
\usepackage{titlesec}
\usepackage{enumitem}
\usepackage{tikz}
\usepackage{pgfplots}
\usepackage{circuitikz}
\usepackage[version=4]{mhchem}
\usepackage{longtable}
\usepackage{array}
\usepackage{float}
\usepackage{caption}
\usepackage{listings}

\lstset{
  basicstyle=\small\ttfamily,
  breaklines=true,
  breakatwhitespace=false,
  postbreak=\mbox{\textcolor{red}{$\hookrightarrow$}\space},
  float=false,
  numbers=left,
  numberstyle=\tiny\color{gray},
  numbersep=10pt,
  xleftmargin=2em,
  keywordstyle=\color{blue},
  commentstyle=\color{green!60!black},
  stringstyle=\color{purple},
  backgroundcolor=\color{gray!5},
  showstringspaces=false,
  tabsize=2,
  captionpos=b,
  keepspaces=true,
  columns=flexible
}

\pgfplotsset{compat=1.18}
\usetikzlibrary{shapes,arrows,positioning,calc,patterns,decorations.pathmorphing,decorations.markings,arrows.meta}

% Color scheme
\definecolor{headcolor}{RGB}{0,102,204}
\definecolor{keycolor}{RGB}{220,20,60}
\definecolor{solutioncolor}{RGB}{34,139,34}
\definecolor{mnemoniccolor}{RGB}{148,0,211}
\definecolor{codecolor}{RGB}{0,0,100}

% Spacing
\setlength{\parskip}{3pt}
\setlist[itemize]{nosep}
\setlist[enumerate]{nosep}

% Title formatting
\titleformat{\section}{\Large\bfseries\color{headcolor}}{\thesection}{1em}{}
\titleformat{\subsection}{\large\bfseries\color{headcolor}}{\thesubsection}{1em}{}

% Pandoc tightlist compatibility
\providecommand{\tightlist}{%
  \setlength{\itemsep}{0pt}\setlength{\parskip}{0pt}}

% Pandoc longtable compatibility
\newcounter{none}
\def\thenone{}


% content/resources/templates/english-boxes.tex
% This file is currently empty - it exists to maintain consistency with the import structure.
% Add custom environments here if needed in the future.


\begin{document}

\begin{center}
{\Huge\bfseries\color{headcolor} Subject Name Solutions}\\[5pt]
{\LARGE 4331104 -- Winter 2023}\\[3pt]
{\large Semester 1 Study Material}\\[3pt]
{\normalsize\textit{Detailed Solutions and Explanations}}
\end{center}

\vspace{10pt}

\subsection*{Question 1(a) [3 marks]}\label{q1a}

\textbf{Classify Noise signal and explain thermal noise.}

\begin{solutionbox}

Noise signals can be classified as:

{\def\LTcaptype{none} % do not increment counter
\begin{longtable}[]{@{}
  >{\raggedright\arraybackslash}p{(\linewidth - 4\tabcolsep) * \real{0.3750}}
  >{\raggedright\arraybackslash}p{(\linewidth - 4\tabcolsep) * \real{0.2000}}
  >{\raggedright\arraybackslash}p{(\linewidth - 4\tabcolsep) * \real{0.4250}}@{}}
\toprule\noalign{}
\begin{minipage}[b]{\linewidth}\raggedright
Type of Noise
\end{minipage} & \begin{minipage}[b]{\linewidth}\raggedright
Source
\end{minipage} & \begin{minipage}[b]{\linewidth}\raggedright
Characteristics
\end{minipage} \\
\midrule\noalign{}
\endhead
\bottomrule\noalign{}
\endlastfoot
\textbf{External Noise} & Outside communication system & Atmospheric,
Space, Industrial \\
\textbf{Internal Noise} & Inside communication system & Thermal, Shot,
Transit time, Flicker \\
\end{longtable}
}

\textbf{Thermal Noise}:

\begin{itemize}
\tightlist
\item
  \textbf{Definition}: Random motion of electrons in a conductor due to
  temperature
\item
  \textbf{Characteristics}: White noise with uniform power across
  frequency spectrum
\item
  \textbf{Formula}: N = kTB (k=Boltzmann constant, T=Temperature,
  B=Bandwidth)
\end{itemize}

\end{solutionbox}
\begin{mnemonicbox}
``Temperature Excites Random Movements'' (TERM)

\end{mnemonicbox}
\subsection*{Question 1(b) [4 marks]}\label{q1b}

\textbf{Comparison between Pre-emphasis and De-emphasis technique.}

\begin{solutionbox}

{\def\LTcaptype{none} % do not increment counter
\begin{longtable}[]{@{}
  >{\raggedright\arraybackslash}p{(\linewidth - 4\tabcolsep) * \real{0.2895}}
  >{\raggedright\arraybackslash}p{(\linewidth - 4\tabcolsep) * \real{0.3684}}
  >{\raggedright\arraybackslash}p{(\linewidth - 4\tabcolsep) * \real{0.3421}}@{}}
\toprule\noalign{}
\begin{minipage}[b]{\linewidth}\raggedright
Parameter
\end{minipage} & \begin{minipage}[b]{\linewidth}\raggedright
Pre-emphasis
\end{minipage} & \begin{minipage}[b]{\linewidth}\raggedright
De-emphasis
\end{minipage} \\
\midrule\noalign{}
\endhead
\bottomrule\noalign{}
\endlastfoot
\textbf{Definition} & Boosting high-frequency components before
transmission & Attenuating high-frequency components at receiver \\
\textbf{Location} & Transmitter side & Receiver side \\
\textbf{Purpose} & Improves SNR for high frequencies & Restores original
signal frequency response \\
\textbf{Circuit} & High-pass filter with RC circuit & Low-pass filter
with RC circuit \\
\textbf{Time Constant} & 75 μs (standard) & 75 μs (matches
pre-emphasis) \\
\end{longtable}
}

\textbf{Diagram/Circuit:}

\begin{center}
\textbf{Mermaid Diagram (Code)}
\begin{verbatim}
{Shaded}
{Highlighting}[]
graph LR
    A[Input] {-{-}{} B[Pre{-}emphasis Circuit]}
    B {-{-}{} C[Modulator]}
    C {-{-}{} D[Transmission]}
    D {-{-}{} E[Demodulator]}
    E {-{-}{} F[De{-}emphasis Circuit]}
    F {-{-}{} G[Output]}
    style B fill:\#f96,stroke:\#333
    style F fill:\#69f,stroke:\#333
{Highlighting}
{Shaded}
\end{verbatim}
\end{center}

\end{solutionbox}
\begin{mnemonicbox}
``Pump Up Before Transmit, Pull Down After Receive''
(PUBTAR)

\end{mnemonicbox}
\subsection*{Question 1(c) [7 marks]}\label{q1c}

\textbf{Derive mathematical expression of AM signal and with help of it
explain frequency spectrum of AM signal.}

\begin{solutionbox}

\textbf{Mathematical Expression Derivation}:

\begin{enumerate}
\item
  Let the carrier signal be: c(t) = Ac cos(2πfct)
\item
  Let the modulating signal be: m(t) = Am cos(2πfmt)
\item
AM signal: s(t) = Ac[1 + μ·m(t)/Am]cos(2πfct) where

μ = modulation

  index
\item
  Substituting m(t): s(t) = Ac[1 + μ·cos(2πfmt)]cos(2πfct)
\item
  Using trigonometric identity cos(A)·cos(B) = ½cos(A+B) + ½cos(A-B):
  s(t) = Ac·cos(2πfct) + (μAc/2)·cos(2π(fc+fm)t) +
  (μAc/2)·cos(2π(fc-fm)t)
\end{enumerate}

\textbf{Frequency Spectrum}:

{\def\LTcaptype{none} % do not increment counter
\begin{longtable}[]{@{}lll@{}}
\toprule\noalign{}
Component & Frequency & Amplitude \\
\midrule\noalign{}
\endhead
\bottomrule\noalign{}
\endlastfoot
Carrier & fc & Ac \\
Upper Sideband & fc + fm & μAc/2 \\
Lower Sideband & fc - fm & μAc/2 \\
\end{longtable}
}

\textbf{Diagram:}

\begin{verbatim}
    │    
    │           ┌─┐
    │           │ │
    │           │ │
    │    ┌─┐    │ │    ┌─┐
    │    │ │    │ │    │ │
    │    │ │    │ │    │ │
    │    │ │    │ │    │ │
────┼────┼─┼────┼─┼────┼─┼────────►f
    │   fc{-fm   fc    fc+fm}
    │
    │   LSB    Carrier   USB
\end{verbatim}

\end{solutionbox}
\begin{mnemonicbox}
``Carrier Standing Between Twins'' (CSBT)

\end{mnemonicbox}
\subsection*{Question 1(c) OR [7
marks]}\label{q1c}

\textbf{Explain block diagram of Communication System.}

\begin{solutionbox}

\textbf{Block Diagram of Communication System}:

\begin{center}
\textbf{Mermaid Diagram (Code)}
\begin{verbatim}
{Shaded}
{Highlighting}[]
graph LR
    A[Input Transducer] {-{-}{} B[Transmitter]}
    B {-{-}{} C[Channel/Medium]}
    C {-{-}{} D[Receiver]}
    D {-{-}{} E[Output Transducer]}
    F[Noise Source] {-{-}{} C}
    style F fill:\#f66,stroke:\#333
{Highlighting}
{Shaded}
\end{verbatim}
\end{center}

\textbf{Components and Functions}:

{\def\LTcaptype{none} % do not increment counter
\begin{longtable}[]{@{}
  >{\raggedright\arraybackslash}p{(\linewidth - 4\tabcolsep) * \real{0.2692}}
  >{\raggedright\arraybackslash}p{(\linewidth - 4\tabcolsep) * \real{0.3846}}
  >{\raggedright\arraybackslash}p{(\linewidth - 4\tabcolsep) * \real{0.3462}}@{}}
\toprule\noalign{}
\begin{minipage}[b]{\linewidth}\raggedright
Block
\end{minipage} & \begin{minipage}[b]{\linewidth}\raggedright
Function
\end{minipage} & \begin{minipage}[b]{\linewidth}\raggedright
Example
\end{minipage} \\
\midrule\noalign{}
\endhead
\bottomrule\noalign{}
\endlastfoot
\textbf{Input Transducer} & Converts original information to electrical
signal & Microphone, Camera \\
\textbf{Transmitter} & Processes signal for efficient transmission
(modulation, amplification) & Radio transmitter \\
\textbf{Channel/Medium} & Path through which signal travels & Air,
Fiber, Cable \\
\textbf{Receiver} & Extracts original signal (amplification, filtering,
demodulation) & Radio receiver \\
\textbf{Output Transducer} & Converts electrical signal back to original
form & Speaker, Display \\
\textbf{Noise Source} & Unwanted signals that distort the information &
Atmospheric, Thermal \\
\end{longtable}
}

\end{solutionbox}
\begin{mnemonicbox}
``Input Transmits Through Channel, Receives Output''
(ITCRO)

\end{mnemonicbox}
\subsection*{Question 2(a) [3 marks]}\label{q2a}

\textbf{Discuss power distribution among sidebands and carrier in
amplitude modulation.}

\begin{solutionbox}

\textbf{Power Distribution in AM Signal}:

{\def\LTcaptype{none} % do not increment counter
\begin{longtable}[]{@{}lll@{}}
\toprule\noalign{}
Component & Power Formula & Percentage (for m=1) \\
\midrule\noalign{}
\endhead
\bottomrule\noalign{}
\endlastfoot
Carrier & Pc = (Ac^{2}/2) & 67\% \\
Upper Sideband & PUSB = (Pc·m^{2})/4 & 16.5\% \\
Lower Sideband & PLSB = (Pc·m^{2})/4 & 16.5\% \\
Total Power & PT = Pc(1+m^{2}/2) & 100\% \\
\end{longtable}
}

\textbf{Diagram:}

\begin{verbatim}
   Power
     │
 100\%┤                ┌───┐
     │                │   │
     │                │   │
  67\%┤       ┌───┐    │   │
     │       │   │    │   │
     │       │   │    │   │
     │       │   │    │   │
16.5\%┤┌───┐  │   │  ┌─┴─┐ │
     ││LSB│  │ C │  │USB│ │
     │└───┘  │   │  └───┘ │
   0\%┼──────────────────────►
     │ Components of AM
\end{verbatim}

\end{solutionbox}
\begin{mnemonicbox}
``Carrier Takes Two-Thirds'' (CTTT)

\end{mnemonicbox}
\subsection*{Question 2(b) [4 marks]}\label{q2b}

\textbf{Why pre-emphases and de-emphases are used? Briefly describe how
the signals are modified at transmitter side and receiver side.}

\begin{solutionbox}

\textbf{Purpose of Pre-emphasis and De-emphasis}:

{\def\LTcaptype{none} % do not increment counter
\begin{longtable}[]{@{}
  >{\raggedright\arraybackslash}p{(\linewidth - 2\tabcolsep) * \real{0.4091}}
  >{\raggedright\arraybackslash}p{(\linewidth - 2\tabcolsep) * \real{0.5909}}@{}}
\toprule\noalign{}
\begin{minipage}[b]{\linewidth}\raggedright
Purpose
\end{minipage} & \begin{minipage}[b]{\linewidth}\raggedright
Explanation
\end{minipage} \\
\midrule\noalign{}
\endhead
\bottomrule\noalign{}
\endlastfoot
Improve SNR & Boosts high frequencies before transmission to overcome
noise \\
Reduce Noise & High frequencies in FM are more susceptible to noise \\
Maintain Fidelity & Ensures overall frequency response remains flat \\
\end{longtable}
}

\textbf{Signal Modification Process}:

\begin{center}
\textbf{Mermaid Diagram (Code)}
\begin{verbatim}
{Shaded}
{Highlighting}[]
graph LR
    A[Audio Input] {-{-}{} B[Pre{-}emphasis at Transmitter]}
    B {-{-}{} C["Boosted High Frequencies{}br /{}(Above 2kHz)"]}
    C {-{-}{} D[FM Modulation]}
    D {-{-}{} E[Transmission]}
    E {-{-}{} F[FM Demodulation at Receiver]}
    F {-{-}{} G[De{-}emphasis]}
    G {-{-}{} H["Restored Original{}br /{}Frequency Response"]}
    style B fill:\#f96,stroke:\#333
    style G fill:\#69f,stroke:\#333
{Highlighting}
{Shaded}
\end{verbatim}
\end{center}

\end{solutionbox}
\begin{mnemonicbox}
``Boost High, Cut High, Keep Original'' (BHCKO)

\end{mnemonicbox}
\subsection*{Question 2(c) [7 marks]}\label{q2c}

\textbf{Explain FM generation techniques. Explain Phase locked loop FM
modulator in detail.}

\begin{solutionbox}

\textbf{FM Generation Techniques}:

{\def\LTcaptype{none} % do not increment counter
\begin{longtable}[]{@{}lll@{}}
\toprule\noalign{}
Technique & Principle & Advantages \\
\midrule\noalign{}
\endhead
\bottomrule\noalign{}
\endlastfoot
Direct FM & Varying capacitance in oscillator & Simple design \\
Indirect FM & Phase modulation to produce FM & Better stability \\
PLL FM & Using phase locked loop & High frequency stability \\
Armstrong method & Using mixers and filters & Excellent linearity \\
\end{longtable}
}

\textbf{PLL FM Modulator}:

\begin{center}
\textbf{Mermaid Diagram (Code)}
\begin{verbatim}
{Shaded}
{Highlighting}[]
graph LR
    A[Modulating Signal] {-{-}{} B[VCO]}
    B {-{-}{} C[Phase Detector]}
    D[Reference Oscillator] {-{-}{} C}
    C {-{-}{} E[Loop Filter]}
    E {-{-}{} B}
    B {-{-}{} F[FM Output]}
    style B fill:\#f96,stroke:\#333
    style C fill:\#69f,stroke:\#333
{Highlighting}
{Shaded}
\end{verbatim}
\end{center}

\textbf{Working Principle}:

\begin{enumerate}
\tightlist
\item
  \textbf{Phase Detector} compares VCO output with reference oscillator
\item
  \textbf{Loop Filter} removes high-frequency components
\item
  \textbf{VCO} (Voltage Controlled Oscillator) frequency changes with
  modulating signal
\item
  Modulating signal directly controls the VCO
\item
  PLL ensures high stability and linearity
\end{enumerate}

\end{solutionbox}
\begin{mnemonicbox}
``Phase Detector Compares, Filter Smooths, VCO
Varies'' (PDCFV)

\end{mnemonicbox}
\subsection*{Question 2(a) OR [3
marks]}\label{q2a}

\textbf{State advantages and disadvantage of SSB over DSB.}

\begin{solutionbox}

\textbf{Advantages and Disadvantages of SSB over DSB}:

{\def\LTcaptype{none} % do not increment counter
\begin{longtable}[]{@{}
  >{\raggedright\arraybackslash}p{(\linewidth - 2\tabcolsep) * \real{0.4634}}
  >{\raggedright\arraybackslash}p{(\linewidth - 2\tabcolsep) * \real{0.5366}}@{}}
\toprule\noalign{}
\begin{minipage}[b]{\linewidth}\raggedright
Advantages of SSB
\end{minipage} & \begin{minipage}[b]{\linewidth}\raggedright
Disadvantages of SSB
\end{minipage} \\
\midrule\noalign{}
\endhead
\bottomrule\noalign{}
\endlastfoot
\textbf{Bandwidth Efficiency}: Uses only half the bandwidth &
\textbf{Complex Circuitry}: Requires complex filtering \\
\textbf{Power Efficiency}: Uses about 1/3 the power & \textbf{Difficult
Demodulation}: Needs carrier recovery \\
\textbf{Reduced Fading}: Less susceptible to selective fading &
\textbf{Distortion}: May distort low frequencies \\
\textbf{Less Interference}: Narrower channel means less overlap &
\textbf{Cost}: More expensive than DSB systems \\
\end{longtable}
}

\end{solutionbox}
\begin{mnemonicbox}
``Power and Bandwidth Saved, But Complex Circuits
Needed'' (PBSCN)

\end{mnemonicbox}
\subsection*{Question 2(b) OR [4
marks]}\label{q2b}

\textbf{Sketch the frequency spectrum of DSBSC and SSB amplitude
modulated wave and pilot carrier.}

\begin{solutionbox}

\textbf{DSBSC Frequency Spectrum}:

\begin{verbatim}
    │    
    │    
    │    ┌─┐         ┌─┐
    │    │ │         │ │
    │    │ │         │ │
    │    │ │         │ │
────┼────┼─┼─────────┼─┼────────►f
    │   fc{-fm       fc+fm}
    │
    │    LSB         USB
\end{verbatim}

\textbf{SSB (Upper Sideband) with Pilot Carrier}:

\begin{verbatim}
    │    
    │             │
    │             │
    │             │         ┌─┐
    │             │         │ │
    │             │         │ │
    │             │         │ │
────┼─────────────┼─────────┼─┼────►f
    │             fc        fc+fm
    │             │
    │        Pilot Carrier   USB
\end{verbatim}

\textbf{Comparison Table}:

{\def\LTcaptype{none} % do not increment counter
\begin{longtable}[]{@{}
  >{\raggedright\arraybackslash}p{(\linewidth - 6\tabcolsep) * \real{0.2679}}
  >{\raggedright\arraybackslash}p{(\linewidth - 6\tabcolsep) * \real{0.1964}}
  >{\raggedright\arraybackslash}p{(\linewidth - 6\tabcolsep) * \real{0.2143}}
  >{\raggedright\arraybackslash}p{(\linewidth - 6\tabcolsep) * \real{0.3214}}@{}}
\toprule\noalign{}
\begin{minipage}[b]{\linewidth}\raggedright
Spectrum Type
\end{minipage} & \begin{minipage}[b]{\linewidth}\raggedright
Bandwidth
\end{minipage} & \begin{minipage}[b]{\linewidth}\raggedright
Components
\end{minipage} & \begin{minipage}[b]{\linewidth}\raggedright
Power Efficiency
\end{minipage} \\
\midrule\noalign{}
\endhead
\bottomrule\noalign{}
\endlastfoot
\textbf{DSBSC} & 2fm & LSB + USB & Medium (no carrier power) \\
\textbf{SSB} & fm & USB or LSB & High (one sideband only) \\
\textbf{SSB with Pilot} & fm + small & USB/LSB + reduced carrier & Good
(minimal carrier power) \\
\end{longtable}
}

\end{solutionbox}
\begin{mnemonicbox}
``Two Sides, One Side, or One Side Plus Pilot''
(TSOSP)

\end{mnemonicbox}
\subsection*{Question 2(c) OR [7
marks]}\label{q2c}

\textbf{Write a short-note on: Pulse modulation.}

\begin{solutionbox}

\textbf{Pulse Modulation Techniques}:

Pulse modulation is a process where continuous analog signal is sampled
and converted into pulses.

{\def\LTcaptype{none} % do not increment counter
\begin{longtable}[]{@{}
  >{\raggedright\arraybackslash}p{(\linewidth - 6\tabcolsep) * \real{0.1395}}
  >{\raggedright\arraybackslash}p{(\linewidth - 6\tabcolsep) * \real{0.3023}}
  >{\raggedright\arraybackslash}p{(\linewidth - 6\tabcolsep) * \real{0.2558}}
  >{\raggedright\arraybackslash}p{(\linewidth - 6\tabcolsep) * \real{0.3023}}@{}}
\toprule\noalign{}
\begin{minipage}[b]{\linewidth}\raggedright
Type
\end{minipage} & \begin{minipage}[b]{\linewidth}\raggedright
Description
\end{minipage} & \begin{minipage}[b]{\linewidth}\raggedright
Principle
\end{minipage} & \begin{minipage}[b]{\linewidth}\raggedright
Application
\end{minipage} \\
\midrule\noalign{}
\endhead
\bottomrule\noalign{}
\endlastfoot
\textbf{PAM (Pulse Amplitude Modulation)} & Amplitude of pulses varies
with signal & Sampling and holding & Intermediate step for PCM \\
\textbf{PWM (Pulse Width Modulation)} & Width/duration of pulses varies
& Comparing with ramp & Motor control, power control \\
\textbf{PPM (Pulse Position Modulation)} & Position of pulses varies &
Timing shift & Optical communication, radar \\
\textbf{PCM (Pulse Code Modulation)} & Digital representation using
binary code & Quantizing and encoding & Digital telephony, CDs \\
\end{longtable}
}

\textbf{Waveform Comparison}:

\begin{verbatim}
Original Signal:
   /{      /      /}
  /  {    /      /  }
 /    {  /      /    }
/      {/      /      }

PAM:
   |      |      |
   |      |      |
   |      |      |
   |      |      |

PWM:
   \_\_\_\_    \_\_\_\_\_    \_\_
  |    |  |     |  |  |
  |    |  |     |  |  |
\_\_|    |\_\_|     |\_\_|  |\_\_

PPM:
   \_     \_     \_
  | |   | |   | |
  | |   | |   | |
\_\_|\_|\_\_\_|\_|\_\_\_|\_|\_\_\_\_\_
\end{verbatim}

\end{solutionbox}
\begin{mnemonicbox}
``Amplitude, Width, Position, Code - All Pulse
Types'' (AWPC)

\end{mnemonicbox}
\subsection*{Question 3(a) [3 marks]}\label{q3a}

\textbf{What is AGC? Draw and explain input-output characteristic curve
of simple AGC circuit.}

\begin{solutionbox}

\textbf{Automatic Gain Control (AGC)}:

\begin{itemize}
\tightlist
\item
  \textbf{Definition}: Circuit that automatically adjusts gain to
  maintain constant output level
\item
  \textbf{Purpose}: Compensates for varying signal strength in receivers
\item
  \textbf{Types}: Simple AGC, Delayed AGC, Amplified AGC
\end{itemize}

\textbf{Input-Output Characteristic Curve}:

\begin{verbatim}
    Output
     │
  Max┤{- {-} {-} {-} {-} {-} {-} {-} {-} {-} {-} {-} {-} {-}}
     │                  ┌───────
     │                 /
     │                /
     │               /
     │              /
     │             /  With AGC
     │         ┌──┘
     │        /
     │       /
     │      /
     │     /
   Min┤    /
     │   /  Without AGC
     │  /
     │ /
     │/
     └─────────────────────────► Input
       Min               Max
\end{verbatim}

\textbf{Working}: As input increases, gain decreases to keep output
nearly constant after threshold

\end{solutionbox}
\begin{mnemonicbox}
``Strong Signals Get Less Gain'' (SSLG)

\end{mnemonicbox}
\subsection*{Question 3(b) [4 marks]}\label{q3b}

\textbf{Write a short-note on balanced ratio detector for FM
demodulation.}

\begin{solutionbox}

\textbf{Balanced Ratio Detector}:

{\def\LTcaptype{none} % do not increment counter
\begin{longtable}[]{@{}
  >{\raggedright\arraybackslash}p{(\linewidth - 2\tabcolsep) * \real{0.4091}}
  >{\raggedright\arraybackslash}p{(\linewidth - 2\tabcolsep) * \real{0.5909}}@{}}
\toprule\noalign{}
\begin{minipage}[b]{\linewidth}\raggedright
Feature
\end{minipage} & \begin{minipage}[b]{\linewidth}\raggedright
Description
\end{minipage} \\
\midrule\noalign{}
\endhead
\bottomrule\noalign{}
\endlastfoot
\textbf{Definition} & FM demodulator using a balanced circuit to convert
frequency variations to amplitude variations \\
\textbf{Key Components} & Two diodes, transformer with center-tapped
secondary, balanced capacitors \\
\textbf{Advantages} & Superior noise immunity, AM rejection,
stability \\
\textbf{Applications} & FM receivers, broadcast receivers \\
\end{longtable}
}

\textbf{Circuit Diagram}:

\begin{verbatim}
     +{-{-}{-}{-}{-}+     +{-}{-}{-}{-}{-}+}
     |     |{-{-}{-}{-}{-}|     |}
     |  T  |     |  D1 |
     |     |{-{-}+{-}{-}|     |}
in{-{-}|     |  |  +{-}{-}{-}{-}{-}+}
     |     |  |
     |     |  |  +{-{-}{-}{-}{-}+}
     |     |{-{-}{-}{-}{-}|     |}
     +{-{-}{-}{-}{-}+     |  D2 |}
                 |     |
                 +{-{-}{-}{-}{-}+}
                   |
                   v
                 output
\end{verbatim}

\textbf{Working Principle}:

\begin{itemize}
\tightlist
\item
  Transformer creates phase-shifted signals for the diodes
\item
  Diodes charge capacitors with different polarities
\item
  As frequency deviates, voltage ratio changes proportionally
\item
  Output is proportional to frequency deviation
\end{itemize}

\end{solutionbox}
\begin{mnemonicbox}
``Balanced Diodes Transform Frequency To Voltage''
(BDTFV)

\end{mnemonicbox}
\subsection*{Question 3(c) [7 marks]}\label{q3c}

\textbf{Explain working of various types of FM demodulator circuits.}

\begin{solutionbox}

\textbf{Types of FM Demodulator Circuits}:

{\def\LTcaptype{none} % do not increment counter
\begin{longtable}[]{@{}
  >{\raggedright\arraybackslash}p{(\linewidth - 6\tabcolsep) * \real{0.2812}}
  >{\raggedright\arraybackslash}p{(\linewidth - 6\tabcolsep) * \real{0.2969}}
  >{\raggedright\arraybackslash}p{(\linewidth - 6\tabcolsep) * \real{0.1875}}
  >{\raggedright\arraybackslash}p{(\linewidth - 6\tabcolsep) * \real{0.2344}}@{}}
\toprule\noalign{}
\begin{minipage}[b]{\linewidth}\raggedright
Demodulator Type
\end{minipage} & \begin{minipage}[b]{\linewidth}\raggedright
Working Principle
\end{minipage} & \begin{minipage}[b]{\linewidth}\raggedright
Advantages
\end{minipage} & \begin{minipage}[b]{\linewidth}\raggedright
Disadvantages
\end{minipage} \\
\midrule\noalign{}
\endhead
\bottomrule\noalign{}
\endlastfoot
\textbf{Slope Detector} & Uses slope of tuned circuit response & Simple
design & Poor linearity, poor AM rejection \\
\textbf{Foster-Seeley Discriminator} & Uses phase shifts in transformer
& Good linearity & Sensitive to amplitude variations \\
\textbf{Ratio Detector} & Modified discriminator with amplitude limiting
& Good AM rejection & Moderate linearity \\
\textbf{PLL Demodulator} & Phase comparison with VCO & Excellent
linearity, good noise immunity & Complex circuit \\
\textbf{Quadrature Detector} & Phase shifting and multiplication &
Simple IC implementation & Limited bandwidth \\
\end{longtable}
}

\textbf{PLL FM Demodulator Circuit}:

\begin{center}
\textbf{Mermaid Diagram (Code)}
\begin{verbatim}
{Shaded}
{Highlighting}[]
graph LR
    A[FM Input] {-{-}{} B[Phase Detector]}
    C[VCO] {-{-}{} B}
    B {-{-}{} D[Loop Filter]}
    D {-{-}{} C}
    D {-{-}{} E[Demodulated Output]}
    style B fill:\#f96,stroke:\#333
    style C fill:\#69f,stroke:\#333
{Highlighting}
{Shaded}
\end{verbatim}
\end{center}

\textbf{Working Principle}:

\begin{enumerate}
\tightlist
\item
  Phase detector compares incoming FM with VCO output
\item
  Error voltage is filtered to remove high frequencies
\item
  VCO is forced to track input frequency
\item
  Filter output is proportional to frequency deviation
\item
  This output is the demodulated FM signal
\end{enumerate}

\end{solutionbox}
\begin{mnemonicbox}
``Frequency Variations Drive Phase Errors'' (FVDPE)

\end{mnemonicbox}
\subsection*{Question 3(a) OR [3
marks]}\label{q3a}

\textbf{Explain characteristics of a Radio receiver.}

\begin{solutionbox}

\textbf{Characteristics of a Radio Receiver}:

{\def\LTcaptype{none} % do not increment counter
\begin{longtable}[]{@{}
  >{\raggedright\arraybackslash}p{(\linewidth - 4\tabcolsep) * \real{0.4000}}
  >{\raggedright\arraybackslash}p{(\linewidth - 4\tabcolsep) * \real{0.3000}}
  >{\raggedright\arraybackslash}p{(\linewidth - 4\tabcolsep) * \real{0.3000}}@{}}
\toprule\noalign{}
\begin{minipage}[b]{\linewidth}\raggedright
Characteristic
\end{minipage} & \begin{minipage}[b]{\linewidth}\raggedright
Definition
\end{minipage} & \begin{minipage}[b]{\linewidth}\raggedright
Importance
\end{minipage} \\
\midrule\noalign{}
\endhead
\bottomrule\noalign{}
\endlastfoot
\textbf{Sensitivity} & Ability to amplify weak signals & Determines
maximum reception range \\
\textbf{Selectivity} & Ability to separate desired signal from adjacent
signals & Prevents interference \\
\textbf{Fidelity} & Accuracy in reproducing original signal & Ensures
sound quality \\
\textbf{Image Frequency Rejection} & Ability to reject image frequency &
Prevents duplicate reception \\
\end{longtable}
}

\textbf{Diagram:}

\begin{center}
\textbf{Mermaid Diagram (Code)}
\begin{verbatim}
{Shaded}
{Highlighting}[]
graph TD
    A[Selectivity] {-{-}{} B[Ideal Receiver Characteristics]}
    C[Sensitivity] {-{-}{} B}
    D[Fidelity] {-{-}{} B}
    E[Image Rejection] {-{-}{} B}
    style B fill:\#f96,stroke:\#333
{Highlighting}
{Shaded}
\end{verbatim}
\end{center}

\end{solutionbox}
\begin{mnemonicbox}
``Select Signals Faithfully, Ignore Mirrors'' (SSFIM)

\end{mnemonicbox}
\subsection*{Question 3(b) OR [4
marks]}\label{q3b}

\textbf{Explain types of distortions occur in AM detector circuit.}

\begin{solutionbox}

\textbf{Types of Distortions in AM Detector Circuit}:

{\def\LTcaptype{none} % do not increment counter
\begin{longtable}[]{@{}
  >{\raggedright\arraybackslash}p{(\linewidth - 6\tabcolsep) * \real{0.3864}}
  >{\raggedright\arraybackslash}p{(\linewidth - 6\tabcolsep) * \real{0.1591}}
  >{\raggedright\arraybackslash}p{(\linewidth - 6\tabcolsep) * \real{0.1818}}
  >{\raggedright\arraybackslash}p{(\linewidth - 6\tabcolsep) * \real{0.2727}}@{}}
\toprule\noalign{}
\begin{minipage}[b]{\linewidth}\raggedright
Distortion Type
\end{minipage} & \begin{minipage}[b]{\linewidth}\raggedright
Cause
\end{minipage} & \begin{minipage}[b]{\linewidth}\raggedright
Effect
\end{minipage} & \begin{minipage}[b]{\linewidth}\raggedright
Prevention
\end{minipage} \\
\midrule\noalign{}
\endhead
\bottomrule\noalign{}
\endlastfoot
\textbf{Diagonal Distortion} & Incorrect time constant & Inability to
follow envelope & Proper RC time constant \\
\textbf{Negative Peak Clipping} & Improper biasing & Loss of information
& Proper diode biasing \\
\textbf{Harmonic Distortion} & Non-linear diode characteristics & Audio
distortion & High-quality diodes \\
\textbf{Frequency Distortion} & Improper filtering & Uneven frequency
response & Proper filter design \\
\end{longtable}
}

\textbf{Diagram:}

\begin{verbatim}
Normal Detection:
    /{      /      /}
   /  {    /      /  }
  /    {  /      /    }
 /      {/      /      }

Diagonal Distortion:
    /{      /      /}
   /  {    /      /  }
  /    {  /      /    }
 /      ╲\_      ╲\_      ╲\_

Negative Peak Clipping:
    /{      /      /}
   /  {    /      /  }
  /    {  /      /    }
\_/\_\_\_\_\_\_{/\_\_\_\_\_\_/\_\_\_\_\_\_}
\end{verbatim}

\end{solutionbox}
\begin{mnemonicbox}
``Diagonal Negative Harmonics Frequency - Distortion
Types'' (DNHF)

\end{mnemonicbox}
\subsection*{Question 3(c) OR [7
marks]}\label{q3c}

\textbf{Draw the block diagram of a Superheterodyne AM receiver and
explain it.}

\begin{solutionbox}

\textbf{Superheterodyne AM Receiver}:

\begin{center}
\textbf{Mermaid Diagram (Code)}
\begin{verbatim}
{Shaded}
{Highlighting}[]
graph LR
    A[Antenna] {-{-}{} B[RF Amplifier]}
    B {-{-}{} C[Mixer]}
    D[Local Oscillator] {-{-}{} C}
    C {-{-}{} E[IF Amplifier]}
    E {-{-}{} F[Detector]}
    F {-{-}{} G[AF Amplifier]}
    G {-{-}{} H[Speaker]}
    I[AGC] {-{-}{} B}
    I {-{-}{} E}
    F {-{-}{} I}
    style C fill:\#f96,stroke:\#333
    style E fill:\#69f,stroke:\#333
{Highlighting}
{Shaded}
\end{verbatim}
\end{center}

\textbf{Function of Each Block}:

{\def\LTcaptype{none} % do not increment counter
\begin{longtable}[]{@{}
  >{\raggedright\arraybackslash}p{(\linewidth - 4\tabcolsep) * \real{0.1842}}
  >{\raggedright\arraybackslash}p{(\linewidth - 4\tabcolsep) * \real{0.2632}}
  >{\raggedright\arraybackslash}p{(\linewidth - 4\tabcolsep) * \real{0.5526}}@{}}
\toprule\noalign{}
\begin{minipage}[b]{\linewidth}\raggedright
Block
\end{minipage} & \begin{minipage}[b]{\linewidth}\raggedright
Function
\end{minipage} & \begin{minipage}[b]{\linewidth}\raggedright
Key Characteristics
\end{minipage} \\
\midrule\noalign{}
\endhead
\bottomrule\noalign{}
\endlastfoot
\textbf{RF Amplifier} & Amplifies weak RF signals & Improves
sensitivity, selectivity \\
\textbf{Local Oscillator} & Generates signal at fixed frequency above
incoming signal & Stability is critical \\
\textbf{Mixer} & Combines RF and local oscillator to produce IF & Key to
superheterodyne principle \\
\textbf{IF Amplifier} & Amplifies intermediate frequency & Main gain
stage, fixed frequency \\
\textbf{Detector} & Extracts audio from modulated signal & Typically
diode detector \\
\textbf{AF Amplifier} & Amplifies audio to drive speaker & Power
amplification \\
\textbf{AGC} & Maintains constant output level & Controls gain of RF and
IF amplifiers \\
\end{longtable}
}

\textbf{Key Advantages}:

\begin{itemize}
\tightlist
\item
  Fixed IF frequency allows optimized amplification
\item
  Better selectivity and sensitivity
\item
  Easier tuning
\end{itemize}

\end{solutionbox}
\begin{mnemonicbox}
``Radio Mixing Local Intermediate Detected Audio
Signals'' (RMLIDAS)

\end{mnemonicbox}
\subsection*{Question 4(a) [3 marks]}\label{q4a}

\textbf{Explain quantization process used in analog to digital
conversion.}

\begin{solutionbox}

\textbf{Quantization Process}:

{\def\LTcaptype{none} % do not increment counter
\begin{longtable}[]{@{}
  >{\raggedright\arraybackslash}p{(\linewidth - 4\tabcolsep) * \real{0.2143}}
  >{\raggedright\arraybackslash}p{(\linewidth - 4\tabcolsep) * \real{0.4643}}
  >{\raggedright\arraybackslash}p{(\linewidth - 4\tabcolsep) * \real{0.3214}}@{}}
\toprule\noalign{}
\begin{minipage}[b]{\linewidth}\raggedright
Step
\end{minipage} & \begin{minipage}[b]{\linewidth}\raggedright
Description
\end{minipage} & \begin{minipage}[b]{\linewidth}\raggedright
Purpose
\end{minipage} \\
\midrule\noalign{}
\endhead
\bottomrule\noalign{}
\endlastfoot
1. \textbf{Sampling} & Converting continuous signal to discrete-time &
Prepare for quantization \\
2. \textbf{Level Allocation} & Dividing amplitude range into discrete
levels & Create digital steps \\
3. \textbf{Assignment} & Mapping each sample to nearest quantization
level & Convert to digital value \\
4. \textbf{Encoding} & Converting levels to binary code & Final digital
representation \\
\end{longtable}
}

\textbf{Diagram:}

\begin{verbatim}
Analog Signal:
    /{}
   /  {}
  /    {}
 /      {}

Quantized Signal:
    \_\_
   |  |
  \_|  |\_
 |      |
\end{verbatim}

\textbf{Types of Quantization}:

\begin{itemize}
\tightlist
\item
  \textbf{Uniform}: Equal step sizes
\item
  \textbf{Non-uniform}: Varying step sizes
\item
  \textbf{Adaptive}: Adjusts based on signal
\end{itemize}

\end{solutionbox}
\begin{mnemonicbox}
``Sample Levels Assign Binary'' (SLAB)

\end{mnemonicbox}
\subsection*{Question 4(b) [4 marks]}\label{q4b}

\textbf{Give the comparison of Sampling techniques.}

\begin{solutionbox}

\textbf{Comparison of Sampling Techniques}:

{\def\LTcaptype{none} % do not increment counter
\begin{longtable}[]{@{}
  >{\raggedright\arraybackslash}p{(\linewidth - 6\tabcolsep) * \real{0.3333}}
  >{\raggedright\arraybackslash}p{(\linewidth - 6\tabcolsep) * \real{0.2167}}
  >{\raggedright\arraybackslash}p{(\linewidth - 6\tabcolsep) * \real{0.2000}}
  >{\raggedright\arraybackslash}p{(\linewidth - 6\tabcolsep) * \real{0.2500}}@{}}
\toprule\noalign{}
\begin{minipage}[b]{\linewidth}\raggedright
Sampling Technique
\end{minipage} & \begin{minipage}[b]{\linewidth}\raggedright
Description
\end{minipage} & \begin{minipage}[b]{\linewidth}\raggedright
Advantages
\end{minipage} & \begin{minipage}[b]{\linewidth}\raggedright
Disadvantages
\end{minipage} \\
\midrule\noalign{}
\endhead
\bottomrule\noalign{}
\endlastfoot
\textbf{Ideal Sampling} & Instantaneous sampling of signal & Perfect
representation & Practically impossible \\
\textbf{Natural Sampling} & Top of pulse follows signal amplitude & No
flat tops & Difficult implementation \\
\textbf{Flat-top Sampling} & Sample and hold circuit & Easy
implementation & Additional distortion \\
\end{longtable}
}

\textbf{Diagram:}

\begin{verbatim}
Original Signal:
    /{      /      /}
   /  {    /      /  }
  /    {  /      /    }
 /      {/      /      }

Ideal Sampling:
   |      |      |
   |      |      |
   |      |      |
   |      |      |

Natural Sampling:
   /{     /     /}
   |      |      |
   |      |      |
   |      |      |

Flat{-top Sampling:}
   \_\_\_     \_\_\_     \_\_\_
   |       |       |
   |       |       |
   |       |       |
\end{verbatim}

\end{solutionbox}
\begin{mnemonicbox}
``Ideal Natural Flat - Sampling Types'' (INF)

\end{mnemonicbox}
\subsection*{Question 4(c) [7 marks]}\label{q4c}

\textbf{Draw and explain block diagram of a PCM transmitter and
receiver.}

\begin{solutionbox}

\textbf{PCM Transmitter Block Diagram}:

\begin{center}
\textbf{Mermaid Diagram (Code)}
\begin{verbatim}
{Shaded}
{Highlighting}[]
graph LR
    A[Input Signal] {-{-}{} B[Low{-}pass Filter]}
    B {-{-}{} C[Sample \& Hold]}
    C {-{-}{} D[Quantizer]}
    D {-{-}{} E[Encoder]}
    E {-{-}{} F[Multiplexer]}
    F {-{-}{} G[Line Coder]}
    G {-{-}{} H[Channel]}
    style D fill:\#f96,stroke:\#333
    style E fill:\#69f,stroke:\#333
{Highlighting}
{Shaded}
\end{verbatim}
\end{center}

\textbf{PCM Receiver Block Diagram}:

\begin{center}
\textbf{Mermaid Diagram (Code)}
\begin{verbatim}
{Shaded}
{Highlighting}[]
graph LR
    A[Channel] {-{-}{} B[Line Decoder]}
    B {-{-}{} C[Demultiplexer]}
    C {-{-}{} D[Decoder]}
    D {-{-}{} E[Reconstruction Filter]}
    E {-{-}{} F[Output Signal]}
    style C fill:\#f96,stroke:\#333
    style D fill:\#69f,stroke:\#333
{Highlighting}
{Shaded}
\end{verbatim}
\end{center}

\textbf{Working of PCM System}:

{\def\LTcaptype{none} % do not increment counter
\begin{longtable}[]{@{}ll@{}}
\toprule\noalign{}
Block & Function \\
\midrule\noalign{}
\endhead
\bottomrule\noalign{}
\endlastfoot
\textbf{Low-pass Filter} & Limits bandwidth to avoid aliasing \\
\textbf{Sample \& Hold} & Samples analog signal at regular intervals \\
\textbf{Quantizer} & Assigns discrete levels to samples \\
\textbf{Encoder} & Converts quantized values to binary code \\
\textbf{Multiplexer} & Combines multiple PCM channels \\
\textbf{Line Coder} & Prepares signal for transmission \\
\textbf{Demultiplexer} & Separates channels at receiver \\
\textbf{Decoder} & Converts binary back to quantized values \\
\textbf{Reconstruction Filter} & Smooths out staircase to recover
analog \\
\end{longtable}
}

\end{solutionbox}
\begin{mnemonicbox}
``Filter, Sample, Quantize, Encode, Multiplex,
Transmit'' (FSQEMT)

\end{mnemonicbox}
\subsection*{Question 4(a) OR [3
marks]}\label{q4a}

\textbf{State and explain Nyquist theorem.}

\begin{solutionbox}

\textbf{Nyquist Theorem}:

\begin{itemize}
\tightlist
\item
  \textbf{Statement}: To perfectly reconstruct a bandlimited signal, the
  sampling frequency must be at least twice the highest frequency
  component in the signal.
\end{itemize}

{\def\LTcaptype{none} % do not increment counter
\begin{longtable}[]{@{}lll@{}}
\toprule\noalign{}
Concept & Formula & Explanation \\
\midrule\noalign{}
\endhead
\bottomrule\noalign{}
\endlastfoot
\textbf{Sampling Rate} & fs \geq 2fmax & Minimum required sampling
frequency \\
\textbf{Nyquist Rate} & 2fmax & Minimum sampling rate to avoid
aliasing \\
\textbf{Nyquist Interval} & 1/(2fmax) & Maximum time between samples \\
\end{longtable}
}

\textbf{Diagram:}

\begin{verbatim}
Proper Sampling (fs { 2fmax):}
  *   *   *   *   *   *   *
 /|{  /|  /|  /|  /|  /|}
/ | {/| | /| | /| | /| | /| | }
  |   |   |   |   |   |   |

Undersampling (fs { 2fmax):}
  *       *       *       *
 /|{     /|     /|     /|}
/ | {   / |    / |    / | }
  |       |       |       |
  |       |       |       |
  * Aliasing occurs! *    *
\end{verbatim}

\textbf{Consequences}:

\begin{itemize}
\tightlist
\item
  \textbf{Undersampling}: Aliasing occurs
\item
  \textbf{Critical sampling}: No margin for error
\item
  \textbf{Oversampling}: Better reconstruction but more data
\end{itemize}

\end{solutionbox}
\begin{mnemonicbox}
``Double Maximum Frequency Stops Aliasing'' (DMFSA)

\end{mnemonicbox}
\subsection*{Question 4(b) OR [4
marks]}\label{q4b}

\textbf{Compare DM, ADM and DPCM.}

\begin{solutionbox}

\textbf{Comparison of DM, ADM and DPCM}:

{\def\LTcaptype{none} % do not increment counter
\begin{longtable}[]{@{}
  >{\raggedright\arraybackslash}p{(\linewidth - 6\tabcolsep) * \real{0.1236}}
  >{\raggedright\arraybackslash}p{(\linewidth - 6\tabcolsep) * \real{0.2472}}
  >{\raggedright\arraybackslash}p{(\linewidth - 6\tabcolsep) * \real{0.3596}}
  >{\raggedright\arraybackslash}p{(\linewidth - 6\tabcolsep) * \real{0.2697}}@{}}
\toprule\noalign{}
\begin{minipage}[b]{\linewidth}\raggedright
Parameter
\end{minipage} & \begin{minipage}[b]{\linewidth}\raggedright
Delta Modulation (DM)
\end{minipage} & \begin{minipage}[b]{\linewidth}\raggedright
Adaptive Delta Modulation (ADM)
\end{minipage} & \begin{minipage}[b]{\linewidth}\raggedright
Differential PCM (DPCM)
\end{minipage} \\
\midrule\noalign{}
\endhead
\bottomrule\noalign{}
\endlastfoot
\textbf{Principle} & 1-bit quantization of difference & Variable step
size DM & Multi-bit quantization of difference \\
\textbf{Bit Rate} & Lowest & Low & Medium \\
\textbf{Complexity} & Simple & Moderate & Complex \\
\textbf{Signal Quality} & Low & Medium & High \\
\textbf{Problems} & Slope overload, granular noise & Reduced slope
overload & Prediction errors \\
\textbf{Applications} & Speech transmission & Voice communications &
Audio, video compression \\
\end{longtable}
}

\textbf{Diagram:}

\begin{center}
\textbf{Mermaid Diagram (Code)}
\begin{verbatim}
{Shaded}
{Highlighting}[]
graph TD
    A[Analog Signal] {-{-}{} B[DM: Fixed steps]}
    A {-{-}{} C[ADM: Variable steps]}
    A {-{-}{} D[DPCM: Multi{-}bit coding]}
    style B fill:\#f69,stroke:\#333
    style C fill:\#6f9,stroke:\#333
    style D fill:\#69f,stroke:\#333
{Highlighting}
{Shaded}
\end{verbatim}
\end{center}

\end{solutionbox}
\begin{mnemonicbox}
``Single-bit, Adaptive-bit, Multi-bit Difference''
(SAMD)

\end{mnemonicbox}
\subsection*{Question 4(c) OR [7
marks]}\label{q4c}

\textbf{Explain working of Differential PCM (DPCM) transmitter and
receiver.}

\begin{solutionbox}

\textbf{DPCM Transmitter}:

\begin{center}
\textbf{Mermaid Diagram (Code)}
\begin{verbatim}
{Shaded}
{Highlighting}[]
graph LR
    A[Input] {-{-}{} B[Sampler]}
    B {-{-}{} C[Subtractor]}
    C {-{-}{} D[Quantizer]}
    D {-{-}{} E[Encoder]}
    E {-{-}{} F[Transmission Channel]}
    E {-{-}{} G[Decoder]}
    G {-{-}{} H[Predictor]}
    H {-{-}{} C}
    style C fill:\#f96,stroke:\#333
    style H fill:\#69f,stroke:\#333
{Highlighting}
{Shaded}
\end{verbatim}
\end{center}

\textbf{DPCM Receiver}:

\begin{center}
\textbf{Mermaid Diagram (Code)}
\begin{verbatim}
{Shaded}
{Highlighting}[]
graph LR
    A[Received Signal] {-{-}{} B[Decoder]}
    B {-{-}{} C[Adder]}
    C {-{-}{} D[Predictor]}
    D {-{-}{} C}
    C {-{-}{} E[Reconstructed Output]}
    style C fill:\#f96,stroke:\#333
    style D fill:\#69f,stroke:\#333
{Highlighting}
{Shaded}
\end{verbatim}
\end{center}

\textbf{Working Principle}:

{\def\LTcaptype{none} % do not increment counter
\begin{longtable}[]{@{}ll@{}}
\toprule\noalign{}
Component & Function \\
\midrule\noalign{}
\endhead
\bottomrule\noalign{}
\endlastfoot
\textbf{Sampler} & Converts analog to discrete-time signal \\
\textbf{Predictor} & Estimates current sample from previous samples \\
\textbf{Subtractor} & Computes difference between actual and
predicted \\
\textbf{Quantizer} & Assigns levels to difference signal \\
\textbf{Encoder} & Converts to binary code \\
\textbf{Decoder} & Converts binary to quantized differences \\
\textbf{Adder} & Combines difference with prediction \\
\end{longtable}
}

\textbf{Key Advantages}:

\begin{itemize}
\tightlist
\item
  \textbf{Reduced bit rate}: Encodes differences which are smaller
\item
  \textbf{Better quality}: Uses signal correlation
\item
  \textbf{Compatibility}: Similar to PCM framework
\end{itemize}

\end{solutionbox}
\begin{mnemonicbox}
``Predict Subtract Quantize Difference'' (PSQD)

\end{mnemonicbox}
\subsection*{Question 5(a) [3 marks]}\label{q5a}

\textbf{Describe TDMA frame.}

\begin{solutionbox}

\textbf{TDMA (Time Division Multiple Access) Frame}:

{\def\LTcaptype{none} % do not increment counter
\begin{longtable}[]{@{}
  >{\raggedright\arraybackslash}p{(\linewidth - 4\tabcolsep) * \real{0.3333}}
  >{\raggedright\arraybackslash}p{(\linewidth - 4\tabcolsep) * \real{0.3939}}
  >{\raggedright\arraybackslash}p{(\linewidth - 4\tabcolsep) * \real{0.2727}}@{}}
\toprule\noalign{}
\begin{minipage}[b]{\linewidth}\raggedright
Component
\end{minipage} & \begin{minipage}[b]{\linewidth}\raggedright
Description
\end{minipage} & \begin{minipage}[b]{\linewidth}\raggedright
Purpose
\end{minipage} \\
\midrule\noalign{}
\endhead
\bottomrule\noalign{}
\endlastfoot
\textbf{Time Slots} & Individual segments assigned to users & Allows
multiple users to share channel \\
\textbf{Guard Time} & Small gap between slots & Prevents overlap between
users \\
\textbf{Preamble} & Synchronization bits at start & Helps receiver
synchronize \\
\textbf{Control Bits} & Special bits for system control & Manages frame
operation \\
\end{longtable}
}

\textbf{Diagram:}

\begin{verbatim}
 ┌─────┬─────┬─────┬─────┬─────┬─────┐
 │Sync │User1│User2│User3│User4│Ctrl │
 └─────┴─────┴─────┴─────┴─────┴─────┘
   └┬┘   └────────────┬────────────┘
 Header         Time slots
\end{verbatim}

\textbf{TDMA Frame Structure}:

\begin{itemize}
\tightlist
\item
  Each user transmits in assigned time slot
\item
  Full frame repeats cyclically
\item
  Frame length depends on number of users
\end{itemize}

\end{solutionbox}
\begin{mnemonicbox}
``Slots In Time Divide Access'' (SITDA)

\end{mnemonicbox}
\subsection*{Question 5(b) [4 marks]}\label{q5b}

\textbf{Draw and explain 4 level digital multiplexing hierarchies.}

\begin{solutionbox}

\textbf{4-Level Digital Multiplexing Hierarchy}:

\begin{center}
\textbf{Mermaid Diagram (Code)}
\begin{verbatim}
{Shaded}
{Highlighting}[]
graph LR
    A[Level 1: Primary {- 24/30 Channels] {-}{-}{} B[Level 2: Secondary {-} 96/120 Channels]}
    B {-{-}{} C[Level 3: Tertiary {-} 672/480 Channels]}
    C {-{-}{} D[Level 4: Quaternary {-} 4032/1920 Channels]}
    style A fill:\#f96,stroke:\#333
    style B fill:\#6f9,stroke:\#333
    style C fill:\#69f,stroke:\#333
    style D fill:\#96f,stroke:\#333
{Highlighting}
{Shaded}
\end{verbatim}
\end{center}

\textbf{Hierarchy Details}:

{\def\LTcaptype{none} % do not increment counter
\begin{longtable}[]{@{}
  >{\raggedright\arraybackslash}p{(\linewidth - 6\tabcolsep) * \real{0.1346}}
  >{\raggedright\arraybackslash}p{(\linewidth - 6\tabcolsep) * \real{0.1154}}
  >{\raggedright\arraybackslash}p{(\linewidth - 6\tabcolsep) * \real{0.4231}}
  >{\raggedright\arraybackslash}p{(\linewidth - 6\tabcolsep) * \real{0.3269}}@{}}
\toprule\noalign{}
\begin{minipage}[b]{\linewidth}\raggedright
Level
\end{minipage} & \begin{minipage}[b]{\linewidth}\raggedright
Name
\end{minipage} & \begin{minipage}[b]{\linewidth}\raggedright
North American System
\end{minipage} & \begin{minipage}[b]{\linewidth}\raggedright
European System
\end{minipage} \\
\midrule\noalign{}
\endhead
\bottomrule\noalign{}
\endlastfoot
\textbf{Level 1} & Primary (T1/E1) & 24 channels, 1.544 Mbps & 30
channels, 2.048 Mbps \\
\textbf{Level 2} & Secondary (T2/E2) & 96 channels, 6.312 Mbps & 120
channels, 8.448 Mbps \\
\textbf{Level 3} & Tertiary (T3/E3) & 672 channels, 44.736 Mbps & 480
channels, 34.368 Mbps \\
\textbf{Level 4} & Quaternary (T4/E4) & 4032 channels, 274.176 Mbps &
1920 channels, 139.264 Mbps \\
\end{longtable}
}

\end{solutionbox}
\begin{mnemonicbox}
``Primary, Secondary, Tertiary, Quaternary Levels''
(PSTQ)

\end{mnemonicbox}
\subsection*{Question 5(c) [7 marks]}\label{q5c}

\textbf{Draw and explain block diagram of PCM-TDM system.}

\begin{solutionbox}

\textbf{PCM-TDM System Block Diagram}:

\begin{center}
\textbf{Mermaid Diagram (Code)}
\begin{verbatim}
{Shaded}
{Highlighting}[]
graph LR
    subgraph "Transmitter"
    A1[Input 1] {-{-}{} B1[LPF]}
    B1 {-{-}{} C1[Sampler]}
    A2[Input 2] {-{-}{} B2[LPF]}
    B2 {-{-}{} C2[Sampler]}
    A3[Input 3] {-{-}{} B3[LPF]}
    B3 {-{-}{} C3[Sampler]}
    C1 {-{-}{} D[TDM Multiplexer]}
    C2 {-{-}{} D}
    C3 {-{-}{} D}
    D {-{-}{} E[Quantizer]}
    E {-{-}{} F[Encoder]}
    F {-{-}{} G[Line Coder]}
    end
    
    G {-{-}{} H[Transmission Channel]}
    
    subgraph "Receiver"
    H {-{-}{} I[Line Decoder]}
    I {-{-}{} J[Decoder]}
    J {-{-}{} K[TDM Demultiplexer]}
    K {-{-}{} L1[LPF]}
    K {-{-}{} L2[LPF]}
    K {-{-}{} L3[LPF]}
    L1 {-{-}{} M1[Output 1]}
    L2 {-{-}{} M2[Output 2]}
    L3 {-{-}{} M3[Output 3]}
    end
    
    style D fill:\#f96,stroke:\#333
    style K fill:\#69f,stroke:\#333
{Highlighting}
{Shaded}
\end{verbatim}
\end{center}

\textbf{Working of PCM-TDM System}:

{\def\LTcaptype{none} % do not increment counter
\begin{longtable}[]{@{}ll@{}}
\toprule\noalign{}
Block & Function \\
\midrule\noalign{}
\endhead
\bottomrule\noalign{}
\endlastfoot
\textbf{Low-Pass Filter} & Limits signal bandwidth to prevent
aliasing \\
\textbf{Sampler} & Converts analog to discrete-time signal \\
\textbf{TDM Multiplexer} & Combines samples from multiple channels \\
\textbf{Quantizer} & Assigns discrete levels to samples \\
\textbf{Encoder} & Converts to binary code \\
\textbf{Line Coder} & Prepares signal for transmission \\
\textbf{Line Decoder} & Recovers binary information \\
\textbf{Decoder} & Converts binary to quantized values \\
\textbf{TDM Demultiplexer} & Separates channels at receiver \\
\textbf{Reconstruction Filter} & Smooths out staircase to recover
analog \\
\end{longtable}
}

\textbf{Key Features}:

\begin{itemize}
\tightlist
\item
  Multiple analog channels share a single digital transmission link
\item
  Each channel is sampled sequentially
\item
  Samples are interlaced in time
\item
  Frame synchronization ensures proper demultiplexing
\end{itemize}

\end{solutionbox}
\begin{mnemonicbox}
``Many Analog Channels Share Digital Link'' (MACSDL)

\end{mnemonicbox}
\subsection*{Question 5(a) OR [3
marks]}\label{q5a}

\textbf{List advantages and disadvantages of digital communication.}

\begin{solutionbox}

\textbf{Advantages and Disadvantages of Digital Communication}:

{\def\LTcaptype{none} % do not increment counter
\begin{longtable}[]{@{}
  >{\raggedright\arraybackslash}p{(\linewidth - 2\tabcolsep) * \real{0.4444}}
  >{\raggedright\arraybackslash}p{(\linewidth - 2\tabcolsep) * \real{0.5556}}@{}}
\toprule\noalign{}
\begin{minipage}[b]{\linewidth}\raggedright
Advantages
\end{minipage} & \begin{minipage}[b]{\linewidth}\raggedright
Disadvantages
\end{minipage} \\
\midrule\noalign{}
\endhead
\bottomrule\noalign{}
\endlastfoot
\textbf{Noise Immunity}: Better resistance to noise &
\textbf{Bandwidth}: Requires more bandwidth \\
\textbf{Error Detection}: Can detect/correct errors &
\textbf{Complexity}: More complex circuitry \\
\textbf{Multiplexing}: Efficient channel sharing &
\textbf{Synchronization}: Requires precise timing \\
\textbf{Security}: Easier encryption & \textbf{Quantization Noise}:
Inherent in A/D conversion \\
\textbf{Integration}: Compatible with computers & \textbf{Cost}: Initial
setup cost is higher \\
\textbf{Regeneration}: Signal can be regenerated & \textbf{Conversion}:
A/D conversion adds delay \\
\end{longtable}
}

\end{solutionbox}
\begin{mnemonicbox}
``Noise-resistant, Error-correcting,
Multiplex-friendly But Bandwidth-hungry'' (NEMBB)

\end{mnemonicbox}
\subsection*{Question 5(b) OR [4
marks]}\label{q5b}

\textbf{List Channel Coding Techniques, explain any one of them with
example.}

\begin{solutionbox}

\textbf{Channel Coding Techniques}:

{\def\LTcaptype{none} % do not increment counter
\begin{longtable}[]{@{}ll@{}}
\toprule\noalign{}
Technique & Purpose \\
\midrule\noalign{}
\endhead
\bottomrule\noalign{}
\endlastfoot
\textbf{Block Coding} & Fixed-length blocks with parity \\
\textbf{Convolutional Coding} & Continuous encoding with memory \\
\textbf{Turbo Coding} & Parallel concatenated codes \\
\textbf{LDPC Coding} & Low-density parity check \\
\textbf{Reed-Solomon} & Powerful block code \\
\end{longtable}
}

\textbf{Block Coding Example: Hamming Code (7,4)}

This code takes 4 data bits and adds 3 parity bits to create a 7-bit
codeword.

{\def\LTcaptype{none} % do not increment counter
\begin{longtable}[]{@{}
  >{\raggedright\arraybackslash}p{(\linewidth - 4\tabcolsep) * \real{0.2143}}
  >{\raggedright\arraybackslash}p{(\linewidth - 4\tabcolsep) * \real{0.4643}}
  >{\raggedright\arraybackslash}p{(\linewidth - 4\tabcolsep) * \real{0.3214}}@{}}
\toprule\noalign{}
\begin{minipage}[b]{\linewidth}\raggedright
Step
\end{minipage} & \begin{minipage}[b]{\linewidth}\raggedright
Description
\end{minipage} & \begin{minipage}[b]{\linewidth}\raggedright
Example
\end{minipage} \\
\midrule\noalign{}
\endhead
\bottomrule\noalign{}
\endlastfoot
1. \textbf{Data Bits} & Original message & 1011 \\
2. \textbf{Bit Positions} & Number positions 1 to 7 & Positions 3,5,6,7
for data \\
3. \textbf{Parity Bits} & Calculate for positions 1,2,4 & P1=1, P2=0,
P4=1 \\
4. \textbf{Codeword} & Combine parity and data & 1011011 \\
\end{longtable}
}

\textbf{Error Detection}:

\begin{itemize}
\tightlist
\item
  If a single bit error occurs, recalculating parity bits identifies
  error position
\item
  Example: 1\textbf{0}11011 \rightarrow 1\textbf{1}11011 (Error at position 2)
\end{itemize}

\end{solutionbox}
\begin{mnemonicbox}
``Parity Bits Protect Data Bits'' (PBPDB)

\end{mnemonicbox}
\subsection*{Question 5(c) OR [7
marks]}\label{q5c}

\textbf{Discuss basic time domain digital multiplexing. State advantages
\& disadvantages of TDM system.}

\begin{solutionbox}

\textbf{Basic Time Domain Digital Multiplexing}:

Time Division Multiplexing (TDM) is a technique that allows multiple
digital signals to share a common transmission medium by allocating
unique time slots to each signal.

{\def\LTcaptype{none} % do not increment counter
\begin{longtable}[]{@{}ll@{}}
\toprule\noalign{}
Operating Principle & Implementation \\
\midrule\noalign{}
\endhead
\bottomrule\noalign{}
\endlastfoot
\textbf{Channel Allocation} & Each source gets periodic time slots \\
\textbf{Frame Structure} & Slots organized into frames with sync bits \\
\textbf{Synchronization} & Transmitter and receiver must maintain
timing \\
\textbf{Throughput} & Dependent on number of channels and sampling
rate \\
\end{longtable}
}

\textbf{TDM System Diagram}:

\begin{center}
\textbf{Mermaid Diagram (Code)}
\begin{verbatim}
{Shaded}
{Highlighting}[]
graph LR
    A1[Source 1] {-{-}{} C[Multiplexer]}
    A2[Source 2] {-{-}{} C}
    A3[Source 3] {-{-}{} C}
    C {-{-}{} D[Transmission Medium]}
    D {-{-}{} E[Demultiplexer]}
    E {-{-}{} F1[Destination 1]}
    E {-{-}{} F2[Destination 2]}
    E {-{-}{} F3[Destination 3]}
    
    style C fill:\#f96,stroke:\#333
    style E fill:\#69f,stroke:\#333
{Highlighting}
{Shaded}
\end{verbatim}
\end{center}

\textbf{Advantages of TDM System}:

{\def\LTcaptype{none} % do not increment counter
\begin{longtable}[]{@{}ll@{}}
\toprule\noalign{}
Advantage & Explanation \\
\midrule\noalign{}
\endhead
\bottomrule\noalign{}
\endlastfoot
\textbf{Efficient Utilization} & Channel used continuously \\
\textbf{Reduced Crosstalk} & No frequency overlap between channels \\
\textbf{Flexibility} & Easy to add/remove channels \\
\textbf{Compatible with Digital} & Works naturally with digital
systems \\
\textbf{Simple Hardware} & No complex filters needed \\
\end{longtable}
}

\textbf{Disadvantages of TDM System}:

{\def\LTcaptype{none} % do not increment counter
\begin{longtable}[]{@{}ll@{}}
\toprule\noalign{}
Disadvantage & Explanation \\
\midrule\noalign{}
\endhead
\bottomrule\noalign{}
\endlastfoot
\textbf{Synchronization} & Requires precise timing \\
\textbf{Buffering} & May need storage between samples \\
\textbf{Overhead} & Sync bits reduce efficiency \\
\textbf{Delay} & Must wait for time slot \\
\textbf{Wasted Capacity} & Empty slots if channel inactive \\
\end{longtable}
}

\end{solutionbox}
\begin{mnemonicbox}
``Time Slots Shared But Sync Required'' (TSSBSR)

\end{mnemonicbox}

\end{document}
