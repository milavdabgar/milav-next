\documentclass{article}

% content/resources/templates/preamble.tex
\usepackage[margin=0.6in]{geometry}
\author{Milav Dabgar}
\usepackage{amsmath,amssymb,amsthm}
\usepackage{booktabs}
\usepackage{multirow}
\usepackage{xcolor}
\usepackage{tcolorbox}
\tcbuselibrary{breakable,skins}
\usepackage[colorlinks=true,linkcolor=blue]{hyperref}
\usepackage{titlesec}
\usepackage{enumitem}
\usepackage{tikz}
\usepackage{pgfplots}
\usepackage{circuitikz}
\usepackage[version=4]{mhchem}
\usepackage{longtable}
\usepackage{array}
\usepackage{float}
\usepackage{caption}
\usepackage{listings}

\lstset{
  basicstyle=\small\ttfamily,
  breaklines=true,
  breakatwhitespace=false,
  postbreak=\mbox{\textcolor{red}{$\hookrightarrow$}\space},
  float=false,
  numbers=left,
  numberstyle=\tiny\color{gray},
  numbersep=10pt,
  xleftmargin=2em,
  keywordstyle=\color{blue},
  commentstyle=\color{green!60!black},
  stringstyle=\color{purple},
  backgroundcolor=\color{gray!5},
  showstringspaces=false,
  tabsize=2,
  captionpos=b,
  keepspaces=true,
  columns=flexible
}

\pgfplotsset{compat=1.18}
\usetikzlibrary{shapes,arrows,positioning,calc,patterns,decorations.pathmorphing,decorations.markings,arrows.meta}

% Color scheme
\definecolor{headcolor}{RGB}{0,102,204}
\definecolor{keycolor}{RGB}{220,20,60}
\definecolor{solutioncolor}{RGB}{34,139,34}
\definecolor{mnemoniccolor}{RGB}{148,0,211}
\definecolor{codecolor}{RGB}{0,0,100}

% Spacing
\setlength{\parskip}{3pt}
\setlist[itemize]{nosep}
\setlist[enumerate]{nosep}

% Title formatting
\titleformat{\section}{\Large\bfseries\color{headcolor}}{\thesection}{1em}{}
\titleformat{\subsection}{\large\bfseries\color{headcolor}}{\thesubsection}{1em}{}

% Pandoc tightlist compatibility
\providecommand{\tightlist}{%
  \setlength{\itemsep}{0pt}\setlength{\parskip}{0pt}}

% Pandoc longtable compatibility
\newcounter{none}
\def\thenone{}


% content/resources/templates/gujarati-boxes.tex
\usepackage{fontspec}
\usepackage{polyglossia}

% Set Gujarati as main language (document is primarily in Gujarati)
% Note: gloss-gujarati.ldf doesn't exist in polyglossia, but it will use hyphenation patterns
\setdefaultlanguage{gujarati}
\setotherlanguage{english}

% Configure Gujarati font properly
% Use Language=Default to prevent polyglossia from trying to add language-specific features
% that don't exist for Gujarati, which causes "empty feature" warnings
\newfontfamily\gujaratifont[Script=Gujarati,AutoFakeBold=2.5,AutoFakeSlant=0.3]{Noto Sans Gujarati}
\setmainfont[Script=Gujarati,AutoFakeBold=2.5,AutoFakeSlant=0.3]{Noto Sans Gujarati}
% Use Noto Sans Gujarati for monospace to support Gujarati in text
\setmonofont[Scale=0.9]{Noto Sans Gujarati}

% Configure English to use the same font
\newfontfamily\englishfont[Script=Gujarati,AutoFakeBold=2.5,AutoFakeSlant=0.3]{Noto Sans Gujarati}

% Translations for polyglossia
\gappto\captionsgujarati{
  \renewcommand{\tablename}{કોષ્ટક}
  \renewcommand{\figurename}{આકૃતિ}
}

% Helper for TikZ nodes to ensure Gujarati font
\newcommand{\gu}[1]{{\gujaratifont #1}}

% Custom environments
\newtcolorbox{solutionbox}{
    breakable,
    enhanced,
    colback=solutioncolor!5!white,
    colframe=solutioncolor!75!black,
    fonttitle=\bfseries,
    title=જવાબ
}

\newtcolorbox{solutionboxnobreak}{
 colback=solutioncolor!5!white,
 colframe=solutioncolor!75!black,
 fonttitle=\bfseries,
 title=જવાબ
}

\newtcolorbox{keyformula}{
 breakable,
 enhanced,
 colback=keycolor!5!white,
 colframe=keycolor!75!black,
 fonttitle=\bfseries,
 title=રાસાયણિક સમીકરણ/સૂત્ર
}

\newtcolorbox{mnemonicbox}{
 breakable,
 enhanced,
 colback=mnemoniccolor!5!white,
 colframe=mnemoniccolor!75!black,
 fonttitle=\bfseries,
 title=મેમરી ટ્રીક
}


% Custom commands for GTU solutions
% This file defines semantic commands for consistent formatting

% Question command with automatic formatting
\newcommand{\question}[2]{%
  \section*{Question #1}%
  \textbf{#2}%
}

% OR question variant
\newcommand{\questionor}[2]{%
  \section*{Question #1 OR}%
  \textbf{#2}%
}

% Proper table environment with caption
\newenvironment{answertable}[1]{%
  \begin{table}[htbp]
  \centering
  \caption{#1}
}{%
  \end{table}
}

% Proper figure environment for diagrams
\newenvironment{answerdiagram}[1]{%
  \begin{figure}[htbp]
  \centering
  \caption{#1}
}{%
  \end{figure}
}

% Semantic markup for key terms
\newcommand{\keyword}[1]{\textbf{#1}}
\newcommand{\code}[1]{\texttt{#1}}
\newcommand{\classname}[1]{\texttt{#1}}
\newcommand{\methodname}[1]{\texttt{#1}}

% Proper quotation marks
\newcommand{\mnemonic}[1]{``#1''}


\title{સિદ્ધાંતો ઓફ ઇલેક્ટ્રોનિક કોમ્યુનિકેશન (4331104) - વિન્ટર 2024 સોલ્યુશન}
\date{December 09, 2024}

\begin{document}
\maketitle

\questionmarks{1}{a}{3}
\textbf{મોડયુલેશન શું છે? તેની જરૂરિયાત શું છે?}

\begin{solutionbox}
    \textbf{મોડ્યુલેશન} એ એક ઉચ્ચ આવૃત્તિ કેરિયર સિગ્નલના એક અથવા વધુ ગુણધર્મો (amplitude, frequency, અથવા phase)ને ઓછી આવૃત્તિના મેસેજ સિગ્નલના તાત્કાલિક મૂલ્યો અનુસાર બદલવાની પ્રક્રિયા છે.

    \textbf{મોડ્યુલેશનની જરૂરિયાત:}
    \begin{itemize}
        \item \textbf{એન્ટેના સાઈઝ ઘટાડવા}: પ્રેક્ટિકલ એન્ટેના સાઈઝ શક્ય બનાવે છે ($\lambda/4$).
        \item \textbf{મલ્ટિપ્લેક્સિંગ}: એક જ માધ્યમનો ઉપયોગ કરીને અનેક સિગ્નલને શેર કરવા.
        \item \textbf{ઇન્ટરફેરન્સ ઘટાડવા}: સિગ્નલને યોગ્ય આવૃત્તિ બેન્ડમાં શિફ્ટ કરે છે.
        \item \textbf{રેન્જ વધારવા}: ટ્રાન્સમિશન અંતરમાં વધારો કરે છે.
    \end{itemize}

    \begin{mnemonicbox}
    "AMIR" - Antenna, Multiplexing, Interference, Range
    \end{mnemonicbox}
\end{solutionbox}

\questionmarks{1}{b}{4}
\textbf{AM waveના DSBFC માટેનું સમીકરણ તારવો.}

\begin{solutionbox}
    DSBFC (Double Sideband Full Carrier) AM wave માટેનું સમીકરણ:

    \textbf{ગાણિતિક રીતે તારવવું:}
    \begin{itemize}
        \item કેરિયર સિગ્નલ: $c(t) = A_c \cos(\omega_c t)$
        \item મેસેજ સિગ્નલ: $m(t) = A_m \cos(\omega_m t)$
        \item AM સિગ્નલ: $s(t) = A_c[1 + \mu m(t)]\cos(\omega_c t)$
        \item જ્યાં $\mu = \text{મોડ્યુલેશન ઇન્ડેક્સ} = A_m/A_c$
    \end{itemize}

    \textbf{મેસેજ સિગ્નલ આવવાથી:}
    \begin{align*}
        s(t) &= A_c[1 + \mu \cos(\omega_m t)]\cos(\omega_c t) \\
        s(t) &= A_c \cos(\omega_c t) + \mu A_c \cos(\omega_m t)\cos(\omega_c t)
    \end{align*}

    \textbf{ત્રિકોણમિતિ સૂત્રનો ઉપયોગ:}
    \[ \cos(A)\cos(B) = \frac{1}{2}[\cos(A+B) + \cos(A-B)] \]

    \textbf{અંતિમ સમીકરણ:}
    \[ s(t) = A_c \cos(\omega_c t) + \frac{\mu A_c}{2}[\cos((\omega_c+\omega_m)t) + \cos((\omega_c-\omega_m)t)] \]

    \begin{center}
    \begin{tikzpicture}
        \begin{axis}[
            width=10cm, height=4cm,
            axis lines=middle,
            xtick={-1, 0, 1}, xticklabels={$f_c-f_m$, $f_c$, $f_c+f_m$},
            ytick={\empty},
            ymin=0, ymax=1.2,
            xmin=-1.5, xmax=1.5,
            xlabel={$f$}, ylabel={Amplitude},
            title={Frequency Spectrum of AM Wave}
        ]
            \addplot[ycomb, mark=none, thick, blue] coordinates {
                (0, 1) (-1, 0.5) (1, 0.5)
            };
            \node[above] at (axis cs:0,1) {Carrier ($A_c$)};
            \node[above] at (axis cs:-1,0.5) {LSB ($\frac{\mu A_c}{2}$)};
            \node[above] at (axis cs:1,0.5) {USB ($\frac{\mu A_c}{2}$)};
        \end{axis}
    \end{tikzpicture}
    \captionof{figure}{Spectrum of DSBFC AM Wave}
    \end{center}
\end{solutionbox}

\questionmarks{1}{c}{7}
\textbf{નોઈસ સિગ્નલને વર્ગીકૃત કરો. ફ્લીકર નોઈસ, શૉટ નોઈસ અને થર્મલ નોઈસ સમજાવો.}

\begin{solutionbox}
    \textbf{નોઈસનું વર્ગીકરણ:}

    \begin{center}
    \begin{tabulary}{\linewidth}{L L L}
        \hline
        \textbf{પ્રકાર} & \textbf{સ્ત્રોત} & \textbf{લક્ષણો} \\
        \hline
        \textbf{બાહ્ય નોઈસ} & પર્યાવરણીય સ્ત્રોત & કોમ્યુનિકેશન સિસ્ટમની બહારના \\
        \textbf{આંતરિક નોઈસ} & કોમ્પોનેન્ટ્સ & સિસ્ટમની અંદર ઉત્પન્ન થતા \\
        \hline
    \end{tabulary}
    \captionof{table}{Types of Noise}
    \end{center}

    \textbf{આંતરિક નોઈસના પ્રકાર:}

    \begin{enumerate}
        \item \textbf{ફ્લીકર નોઈસ:}
        \begin{itemize}
            \item \textbf{સ્ત્રોત}: એક્ટિવ ઉપકરણોમાં થાય છે.
            \item \textbf{લક્ષણો}: આવૃત્તિના વ્યસ્ત પ્રમાણમાં ($1/f$).
            \item \textbf{અસર}: નીચી આવૃત્તિઓ પર મુખ્ય.
        \end{itemize}

        \item \textbf{શૉટ નોઈસ:}
        \begin{itemize}
            \item \textbf{સ્ત્રોત}: જંક્શનમાંથી ઇલેક્ટ્રોનનો રેન્ડમ પ્રવાહ.
            \item \textbf{લક્ષણો}: આવૃત્તિથી સ્વતંત્ર (વ્હાઈટ નોઈસ).
            \item \textbf{અસર}: ડાયોડ/ટ્રાન્ઝિસ્ટરમાં રેન્ડમ કરંટ ફ્લક્ચ્યુએશન.
        \end{itemize}

        \item \textbf{થર્મલ નોઈસ:}
        \begin{itemize}
            \item \textbf{સ્ત્રોત}: તાપમાનને કારણે ઇલેક્ટ્રોનની રેન્ડમ ગતિ.
            \item \textbf{લક્ષણો}: બધા કન્ડક્ટર, રેઝિસ્ટરમાં મોજુદ.
            \item \textbf{ફોર્મ્યુલા}: $P_n = kTB$ ($k=\text{બોલ્ટઝમેન સ્થિરાંક}$, $T=\text{તાપમાન}$, $B=\text{બેન્ડવિડ્થ}$).
            \item \textbf{અસર}: રિસીવરમાં નોઈસ ફ્લોર સેટ કરે છે.
        \end{itemize}
    \end{enumerate}

    \begin{mnemonicbox}
    "FST" - Flicker decreases with Frequency, Shot is from electron flow, Thermal depends on Temperature
    \end{mnemonicbox}
\end{solutionbox}

\questionmarks{1}{c}{7}
\textbf{EM wave સમજાવો અને સ્પેક્ટ્રમના વિવિધ બેન્ડની એપ્લીકેશન લખો.}

\begin{solutionbox}
    \textbf{EM Wave (વિદ્યુત ચુંબકીય તરંગ):}
    વિદ્યુત ચુંબકીય તરંગો એ સમય સાથે બદલાતાં ઇલેક્ટ્રિક અને મેગ્નેટિક ફીલ્ડ્સ દ્વારા અવકાશમાં પ્રસરતી ઊર્જા છે, જે પ્રકાશની ગતિએ ($3\times10^8$ m/s) ચાલે છે.

    \textbf{લક્ષણો:}
    \begin{itemize}
        \item ટ્રાન્સવર્સ તરંગો જેમાં E અને H ફીલ્ડ એકબીજાના પરપેન્ડીક્યુલર હોય છે.
        \item પ્રસરણ માટે કોઈ માધ્યમની જરૂર નથી.
        \item તરંગલંબાઈ ($\lambda$) અને આવૃત્તિ ($f$) દ્વારા વર્ણવાય છે.
        \item સંબંધ: $c = f \times \lambda$.
    \end{itemize}

    \textbf{EM સ્પેક્ટ્રમ અને એપ્લીકેશન:}

    \begin{center}
    \begin{tabulary}{\linewidth}{L L L}
        \hline
        \textbf{આવૃત્તિ બેન્ડ} & \textbf{આવૃત્તિ રેન્જ} & \textbf{એપ્લીકેશન} \\
        \hline
        ELF & 3Hz-30Hz & સબમરીન કોમ્યુનિકેશન \\
        VLF & 3kHz-30kHz & નેવિગેશન સિસ્ટમ \\
        LF & 30kHz-300kHz & AM બ્રોડકાસ્ટિંગ \\
        MF & 300kHz-3MHz & AM રેડિયો બ્રોડકાસ્ટિંગ \\
        HF & 3MHz-30MHz & શોર્ટવેવ રેડિયો \\
        VHF & 30MHz-300MHz & FM રેડિયો, TV બ્રોડકાસ્ટિંગ \\
        UHF & 300MHz-3GHz & TV, મોબાઈલ ફોન, WiFi \\
        SHF & 3GHz-30GHz & સેટેલાઈટ કોમ્યુનિકેશન, રડાર \\
        EHF & 30GHz-300GHz & મિલિમીટર વેવ કોમ્યુનિકેશન \\
        Infrared & 300GHz-400THz & રિમોટ કંટ્રોલ, થર્મલ ઈમેજિંગ \\
        Visible & 400THz-800THz & ફાઈબર ઓપ્ટિક કોમ્યુનિકેશન \\
        Ultraviolet & 800THz-30PHz & સ્ટરિલાઈઝેશન, ઓથેન્ટિકેશન \\
        X-Rays & 30PHz-30EHz & મેડિકલ ઈમેજિંગ \\
        Gamma Rays & >30EHz & કેન્સર ટ્રીટમેન્ટ \\
        \hline
    \end{tabulary}
    \captionof{table}{EM Spectrum and Applications}
    \end{center}

    \begin{center}
    \begin{tikzpicture}[xscale=0.8]
        \draw[->, thick] (0,0) -- (14,0) node[right] {Increasing Frequency ($f$)};
        \draw[<-, thick] (0,-0.5) -- (14,-0.5) node[right] {Decreasing Wavelength ($\lambda$)};

        \foreach \x/\label in {1/Radio, 3/Microwave, 5/Infrared, 7/Visible, 9/Ultraviolet, 11/X-ray, 13/Gamma} {
            \draw (\x,0.1) -- (\x,-0.1);
            \node[above, align=center, font=\small] at (\x,0.2) {\label};
        }
    \end{tikzpicture}
    \captionof{figure}{Electromagnetic Spectrum}
    \end{center}

    \begin{mnemonicbox}
    "RMIUXG" - Radio, Microwave, Infrared, Ultraviolet, X-ray, Gamma
    \end{mnemonicbox}
\end{solutionbox}

\questionmarks{2}{a}{3}
\textbf{DSBની સરખામણીએ SSBના ફાયદાઓ લખો.}

\begin{solutionbox}
    \textbf{SSBના DSB કરતાં ફાયદાઓ:}

    \begin{center}
    \begin{tabulary}{\linewidth}{L L}
        \hline
        \textbf{પેરામીટર} & \textbf{SSB ફાયદો} \\
        \hline
        \textbf{બેન્ડવિડ્થ} & 50\% ઓછી બેન્ડવિડ્થની જરૂરિયાત \\
        \textbf{પાવર} & 83.33\% પાવર બચત \\
        \textbf{ટ્રાન્સમીટર} & ઓછા પાવર એમ્પ્લિફિકેશનની જરૂર \\
        \textbf{રિસીવર} & ફેઝ ડિસ્ટોર્શન વગર સરળ ડિઝાઇન \\
        \textbf{SNR} & વધુ સારો સિગ્નલ-ટુ-નોઈઝ રેશિયો \\
        \textbf{ફેડિંગ} & સિલેક્ટિવ ફેડિંગથી ઓછું અસરગ્રસ્ત \\
        \hline
    \end{tabulary}
    \end{center}

    \begin{mnemonicbox}
    "BP TRFS" - Bandwidth, Power, Transmitter, Receiver, Fading, SNR
    \end{mnemonicbox}
\end{solutionbox}

\questionmarks{2}{b}{4}
\textbf{FET રિએક્ટન્સ મોડ્યુલેટરથી FM વેવનું જનરેશન સમજાવો.}

\begin{solutionbox}
    \textbf{FET રિએક્ટન્સ મોડ્યુલેટર:}

    \textbf{કાર્ય સિદ્ધાંત:}
    \begin{itemize}
        \item FETને વોલ્ટેજ-કંટ્રોલ્ડ રિએક્ટન્સ તરીકે ઉપયોગ કરે છે.
        \item મોડ્યુલેટિંગ સિગ્નલના આધારે ઇફેક્ટિવ કેપેસિટન્સ બદલે છે.
        \item ઓસિલેટરના LC ટેંક સર્કિટ સાથે જોડાય છે.
    \end{itemize}

    \textbf{સર્કિટ ઓપરેશન:}
    \begin{enumerate}
        \item મોડ્યુલેટિંગ સિગ્નલ FETના ગેટ પર આપવામાં આવે છે.
        \item FETનો ડ્રેન-સોર્સ રેઝિસ્ટન્સ ગેટ વોલ્ટેજ સાથે બદલાય છે.
        \item કેપેસિટિવ રિએક્ટન્સ મોડ્યુલેટિંગ સિગ્નલ સાથે બદલાય છે.
        \item ઓસિલેટરની આવૃત્તિ ઇનપુટ સિગ્નલ સાથે ફેરફાર કરે છે.
    \end{enumerate}

    \begin{center}
    \begin{circuitikz}[font=\footnotesize]
        % FET
        \node[nfet] (fet) at (0,0) {};
        \node[right] at (fet.S) {S};
        \node[right] at (fet.D) {D};
        \node[left] at (fet.G) {G};
        
        % Components around FET
        \draw (fet.D) -- ++(0,1) coordinate (top);
        \draw (fet.S) -- ++(0,-1) coordinate (bottom) node[ground]{};
        
        \draw (top) to[C, l=$C_1$] ++(-2,0) coordinate (topleft) -- ++(0,-1) coordinate (Gconnect);
        \draw (Gconnect) -- (fet.G);
        \draw (Gconnect) to[R, l=$R_1$] (Gconnect |- bottom) -- (bottom);
        
        \draw (topleft) -- ++(-1,0) node[circ]{} node[left]{To Tank Circuit};
        \draw (bottom -| topleft) -- ++(-1,0) node[circ]{} node[left]{Ground};
        
        % Input
        \draw (fet.G) -- ++(-0.5,0) to[C, l=$C_{in}$] ++(-1,0) node[left] {$V_{in}$ (Modulating)};
        
        % RFC
        \draw (top) to[L, l=RFC] ++(2,0) node[right] {$V_{DD}$};
        
    \end{tikzpicture}
    \captionof{figure}{FET Reactance Modulator}
    \end{center}

    \textbf{મુખ્ય લક્ષણો:}
    \begin{itemize}
        \item \textbf{સરળ ડિઝાઇન}: અન્ય મોડ્યુલેટર કરતાં ઓછા કોમ્પોનેન્ટ્સ.
        \item \textbf{લિનિયારિટી}: વાઈડ-બેન્ડ FM જનરેશન માટે સારું.
        \item \textbf{સ્થિરતા}: વેરેક્ટર ડાયોડ કરતાં તાપમાનમાં વધુ સ્થિર.
    \end{itemize}

    \begin{mnemonicbox}
    "LOVE FM" - LC Oscillator with Voltage-controlled Element for FM
    \end{mnemonicbox}
\end{solutionbox}

\questionmarks{2}{c}{7}
\textbf{AM માટે ટોટલ પાવરનું સમીકરણ તારવો. DSB અને SSB માટે પાવર સેવિંગ્સના ટકાની ગણતરી કરો.}

\begin{solutionbox}
    \textbf{AM સિગ્નલમાં પાવર:}
    
    AM સિગ્નલ $s(t) = A_c[1 + \mu\cos(\omega_m t)]\cos(\omega_c t)$ માટે

    \textbf{કુલ પાવર ગણતરી:}
    \begin{enumerate}
        \item કેરિયરમાં પાવર: $P_c = A_c^2/2$
        \item સાઈડબેન્ડમાં પાવર: $P_s = \mu^2 A_c^2/4$ (બન્ને સાઈડબેન્ડ માટે કુલ)
        \item કુલ પાવર: $P_t = P_c + P_s = \frac{A_c^2}{2} (1 + \frac{\mu^2}{2})$
    \end{enumerate}

    \textbf{100\% મોડ્યુલેશન ($\mu=1$) માટે:}
    \begin{itemize}
        \item $P_t = P_c \times (1 + 0.5) = 1.5 \times P_c$
        \item કેરિયર પાવર = કુલ પાવરનો 66.67\%
        \item સાઈડબેન્ડ પાવર = કુલ પાવરનો 33.33\%
    \end{itemize}

    \textbf{પાવર સેવિંગ્સ:}
    \begin{enumerate}
        \item \textbf{DSB-SC માં:} 
        \begin{itemize}
            \item કેરિયર સપ્રેસ થાય છે.
            \item 66.67\% પાવર બચે છે.
        \end{itemize}

        \item \textbf{SSB માં:}
        \begin{itemize}
            \item કેરિયર + એક સાઈડબેન્ડ સપ્રેસ થાય છે.
            \item 66.67\% + 16.67\% = 83.33\% પાવર બચે છે.
        \end{itemize}
    \end{enumerate}

    \textbf{તુલનાત્મક ટેબલ:}
    \begin{center}
    \begin{tabulary}{\linewidth}{L L L L L}
        \hline
        \textbf{મોડ્યુલેશન} & \textbf{કેરિયર પાવર} & \textbf{સાઈડબેન્ડ પાવર} & \textbf{કુલ પાવર} & \textbf{પાવર સેવિંગ} \\
        \hline
        AM ($\mu=1$) & 100\% & 50\% & 150\% & 0\% \\
        DSB-SC & 0\% & 50\% & 50\% & 66.67\% \\
        SSB & 0\% & 25\% & 25\% & 83.33\% \\
        \hline
    \end{tabulary}
    \end{center}

    \begin{mnemonicbox}
    "CST" - Carrier power, Sideband power, Total power
    \end{mnemonicbox}
\end{solutionbox}

\questionmarks{2}{a}{3}
\textbf{AM વેવ માટે Time domain અને Frequency domain ડિસપ્લે દોરો અને સમજાવો.}

\begin{solutionbox}
    \textbf{AM વેવના Time Domain અને Frequency Domain ડિસપ્લે:}

    \textbf{Time Domain (સમય ડોમેન):}
    \begin{itemize}
        \item સમય સાથે એમ્પ્લિટ્યુડમાં થતા ફેરફાર બતાવે છે.
        \item એન્વેલોપ મોડ્યુલેટિંગ સિગ્નલને અનુસરે છે.
        \item મહત્તમ એમ્પ્લિટ્યુડ: $A_{max} = A_c(1+\mu)$
        \item ન્યૂનતમ એમ્પ્લિટ્યુડ: $A_{min} = A_c(1-\mu)$
        \item મોડ્યુલેશન ઇન્ડેક્સ: $\mu = (A_{max}-A_{min})/(A_{max}+A_{min})$
    \end{itemize}

    \textbf{Frequency Domain (આવૃત્તિ ડોમેન):}
    \begin{itemize}
        \item આવૃત્તિઓ પર પાવર ડિસ્ટ્રિબ્યુશન બતાવે છે.
        \item કેરિયર સેન્ટર આવૃત્તિ $f_c$ પર.
        \item અપર સાઈડબેન્ડ $f_c+f_m$ પર.
        \item લોઅર સાઈડબેન્ડ $f_c-f_m$ પર.
        \item બેન્ડવિડ્થ = $2f_m$.
    \end{itemize}

    \begin{center}
    \begin{tikzpicture}
        % Time Domain
        \begin{scope}[xshift=-3.5cm]
            \draw[->] (0,-1.5) -- (0,1.5) node[above] {V};
            \draw[->] (0,0) -- (3,0) node[right] {t};
            \draw[blue, domain=0:2.8, samples=100] plot (\x, { (1 + 0.5*cos(deg(2*pi*\x))) * cos(deg(10*pi*\x)) });
            \node[below] at (1.5,-1.6) {Time Domain};
            \draw[red, dashed] (0,1.5) -- (2.8, 1.5) node[right, font=\tiny] {$A_{max}$};
             \draw[red, dashed] (0,0.5) -- (2.8, 0.5) node[right, font=\tiny] {$A_{min}$};
        \end{scope}

        % Frequency Domain
        \begin{scope}[xshift=3.5cm]
             \draw[->] (0,0) -- (0,1.5) node[above] {V};
            \draw[->] (-1.5,0) -- (1.5,0) node[right] {f};
            \draw[thick, blue] (0,0) -- (0,1.2) node[above] {$f_c$};
            \draw[thick, blue] (-0.8,0) -- (-0.8,0.6) node[above] {LSB};
            \draw[thick, blue] (0.8,0) -- (0.8,0.6) node[above] {USB};
            \node[below] at (0,-0.2) {Frequency Domain};
            \node[below, font=\tiny] at (-0.8,0) {$f_c-f_m$};
            \node[below, font=\tiny] at (0.8,0) {$f_c+f_m$};
        \end{scope}
    \end{tikzpicture}
    \captionof{figure}{AM Time and Frequency Domain Representations}
    \end{center}

    \begin{mnemonicbox}
    "TEF" - Time domain shows Envelope, Frequency domain shows spectral components
    \end{mnemonicbox}
\end{solutionbox}

\questionmarks{2}{b}{4}
\textbf{પ્રી-એમફાસીસ અને ડી-એમફાસીસ સર્કિટ સમજાવો.}

\begin{solutionbox}
    \textbf{પ્રી-એમફાસીસ અને ડી-એમફાસીસ સર્કિટ:}

    \textbf{હેતુ:}
    \begin{itemize}
        \item ઉચ્ચ આવૃત્તિના ઘટકો માટે SNR સુધારવા.
        \item ઉચ્ચ આવૃત્તિમાં વધુ નોઈઝ માટે કમ્પેન્સેશન.
        \item મુખ્યત્વે FM સિસ્ટમમાં વપરાય છે.
    \end{itemize}
    
    \begin{center}
    \begin{minipage}{0.45\textwidth}
        \textbf{પ્રી-એમફાસીસ:}
        \begin{itemize}
            \item ટ્રાન્સમીટર પર લાગુ કરવામાં આવે છે.
            \item ઉચ્ચ આવૃત્તિ ઘટકોને બૂસ્ટ કરે છે.
            \item સામાન્ય રીતે 2.1kHz ઉપર +6dB/ઓક્ટેવ.
            \item સર્કિટ: હાઈ-પાસ RC નેટવર્ક.
        \end{itemize}
        
        \begin{circuitikz}
             \draw (0,0) node[left]{$V_{in}$} to[C, l=$C$] (1.5,0) -- (2.5,0) node[right]{$V_{out}$};
             \draw (1.5,0) to[R, l=$R$] (1.5,-1.5) node[ground]{};
        \end{tikzpicture}
    \end{minipage}
    \hfill
    \begin{minipage}{0.45\textwidth}
        \textbf{ડી-એમફાસીસ:}
        \begin{itemize}
            \item રિસીવર પર લાગુ કરવામાં આવે છે.
            \item ઉચ્ચ આવૃત્તિ ઘટકોને એટેન્યુએટ કરે છે.
            \item ઓરિજિનલ સિગ્નલ બેલેન્સ રીસ્ટોર કરે છે.
            \item સર્કિટ: લો-પાસ RC નેટવર્ક.
        \end{itemize}
        
         \begin{circuitikz}
             \draw (0,0) node[left]{$V_{in}$} to[R, l=$R$] (1.5,0) -- (2.5,0) node[right]{$V_{out}$};
             \draw (1.5,0) to[C, l=$C$] (1.5,-1.5) node[ground]{};
        \end{tikzpicture}
    \end{minipage}
    \end{center}

    \begin{center}
    \begin{tikzpicture}[xscale=1.5, yscale=0.8]
        \draw[->] (0,0) -- (3,0) node[right] {f};
        \draw[->] (0,-2) -- (0,2) node[above] {Gain (dB)};
        \draw[blue, thick] (0,0) -- (1,0) to[out=0,in=180] (2.5,1.5) node[right] {Pre-emphasis};
        \draw[red, thick] (0,0) -- (1,0) to[out=0,in=180] (2.5,-1.5) node[right] {De-emphasis};
        \draw[dashed] (1,-2) -- (1,2);
        \node[below] at (1,0) {2.1 kHz};
    \end{tikzpicture}
    \captionof{figure}{Pre-emphasis and De-emphasis Frequency Response}
    \end{center}

    \begin{mnemonicbox}
    "HIGH-LOW" - HIGHer frequencies boosted at transmitter, LOWered at receiver
    \end{mnemonicbox}
\end{solutionbox}

\questionmarks{2}{c}{7}
\textbf{નેરોબેન્ડ FM અને વાઈડબેન્ડ FMને સરખાવો.}

\begin{solutionbox}
    \textbf{નેરોબેન્ડ FM અને વાઈડબેન્ડ FMની તુલના:}

    \begin{center}
    \begin{tabulary}{\linewidth}{L L L}
        \hline
        \textbf{પેરામીટર} & \textbf{નેરોબેન્ડ FM} & \textbf{વાઈડબેન્ડ FM} \\
        \hline
        \textbf{મોડ્યુલેશન ઇન્ડેક્સ ($\beta$)} & $\beta \ll 1$ (સામાન્ય રીતે $<0.5$) & $\beta \gg 1$ (સામાન્ય રીતે $>5$) \\
        \textbf{બેન્ડવિડ્થ} & $2f_m$ (મેસેજ બેન્ડવિડ્થની બમણી) & $2f_m(\beta+1)$ (કાર્સનનો નિયમ) \\
        \textbf{મહત્વપૂર્ણ સાઈડબેન્ડ્સ} & માત્ર પ્રથમ જોડી સાઈડબેન્ડ્સ & અનેક સાઈડબેન્ડ્સ \\
        \textbf{એપ્લિકેશન} & મોબાઈલ કોમ્યુનિકેશન, ટુ-વે રેડિયો & FM બ્રોડકાસ્ટિંગ, હાઈ-ફિડેલિટી ઓડિયો \\
        \textbf{સિગ્નલ ક્વોલિટી} & ઓછી ફિડેલિટી, ઓછી નોઈઝ ઇમ્યુનિટી & વધુ ફિડેલિટી, વધુ સારી નોઈઝ ઇમ્યુનિટી \\
        \textbf{પાવર એફિશિયન્સી} & વધુ & ઓછી \\
        \textbf{સ્પેક્ટ્રમ ઉપયોગ} & કાર્યક્ષમ & ઓછો કાર્યક્ષમ \\
        \textbf{સર્કિટ જટિલતા} & સરળ & વધુ જટિલ \\
        \hline
    \end{tabulary}
    \end{center}

    \begin{center}
    \begin{tikzpicture}
        % Narrowband
        \begin{scope}[xshift=-3cm]
            \draw[->] (-1.5,0) -- (1.5,0) node[right] {f};
            \draw[->] (0,0) -- (0,1.2);
            \draw[thick, blue] (0,0) -- (0,1);
            \draw[thick, blue] (-0.5,0) -- (-0.5,0.3);
            \draw[thick, blue] (0.5,0) -- (0.5,0.3);
            \node[below] at (0,-0.2) {Narrowband FM};
        \end{scope}
        
        % Wideband
        \begin{scope}[xshift=3cm]
            \draw[->] (-2.5,0) -- (2.5,0) node[right] {f};
            \draw[->] (0,0) -- (0,1.2);
            \draw[thick, blue] (0,0) -- (0,1);
             \foreach \x in {0.3, 0.6, 0.9, 1.2, 1.5} {
                \draw[thick, blue] (\x,0) -- (\x, {0.8-\x*0.4});
                \draw[thick, blue] (-\x,0) -- (-\x, {0.8-\x*0.4});
            }
            \node[below] at (0,-0.2) {Wideband FM};
        \end{scope}
    \end{tikzpicture}
    \captionof{figure}{Spectrum Comparison}
    \end{center}

    \begin{mnemonicbox}
    "BASPCB" - Bandwidth, Applications, Sidebands, Power, Complexity, Beta
    \end{mnemonicbox}
\end{solutionbox}

\questionmarks{3}{a}{3}
\textbf{રેડીઓ રીસીવરની કોઈ ચાર લાક્ષણિકતાઓ વ્યાખ્યાઈત કરો.}

\begin{solutionbox}
    \textbf{રેડિયો રિસીવરની લાક્ષણિકતાઓ:}

    \begin{enumerate}
        \item \textbf{સેન્સિટિવિટી:}
        \begin{itemize}
            \item નબળા સિગ્નલને એમ્પ્લિફાય કરવાની ક્ષમતા.
            \item માઈક્રોવોલ્ટ ($\mu V$)માં માપવામાં આવે છે.
            \item સામાન્ય રીતે સારા રિસીવર્સ માટે 1-10$\mu V$.
        \end{itemize}

        \item \textbf{સિલેક્ટિવિટી:}
        \begin{itemize}
            \item અડોસપડોસની ચેનલથી ઇચ્છિત સિગ્નલને અલગ કરવાની ક્ષમતા.
            \item IF એમ્પ્લિફાયરની બેન્ડવિડ્થ દ્વારા નિર્ધારિત.
            \item ચોક્કસ આવૃત્તિ ઓફસેટ્સ પર dBમાં માપવામાં આવે છે.
        \end{itemize}

        \item \textbf{ફિડેલિટી:}
        \begin{itemize}
            \item ઓરિજિનલ સિગ્નલને અચૂક રીતે રિપ્રોડ્યુસ કરવાની ક્ષમતા.
            \item બેન્ડવિડ્થ અને ડિસ્ટોર્શન પર આધાર રાખે છે.
            \item આવૃત્તિ પ્રતિસાદની સપાટતા તરીકે માપવામાં આવે છે.
        \end{itemize}

        \item \textbf{ઇમેજ ફ્રિક્વન્સી રિજેક્શન:}
        \begin{itemize}
            \item ઇમેજ આવૃત્તિ ($f_i = f_s \pm 2f_{IF}$) પર સિગ્નલને રિજેક્ટ કરવાની ક્ષમતા.
            \item dBમાં માપવામાં આવે છે.
            \item ઉચ્ચ મૂલ્યો વધુ સારી કામગીરી દર્શાવે છે.
        \end{itemize}
    \end{enumerate}

    \begin{mnemonicbox}
    "SFID" - Sensitivity, Fidelity, Image rejection, selectivity Determines quality
    \end{mnemonicbox}
\end{solutionbox}

\questionmarks{3}{b}{4}
\textbf{ડાયોડ ડિટેક્ટર સર્કિટ સમજાવો.}

\begin{solutionbox}
    \textbf{ડાયોડ ડિટેક્ટર સર્કિટ:}

    \textbf{હેતુ:}
    \begin{itemize}
        \item AM વેવમાંથી ઓરિજિનલ મેસેજ સિગ્નલ એક્સટ્રેક્ટ કરે છે.
        \item એન્વેલોપ ડિટેક્ટર પણ કહેવાય છે.
    \end{itemize}

    \textbf{સર્કિટ કોમ્પોનેન્ટ્સ:}
    \begin{itemize}
        \item ડાયોડ: AM સિગ્નલને રેક્ટિફાય કરે છે.
        \item RC નેટવર્ક: કેરિયર આવૃત્તિને ફિલ્ટર કરે છે.
        \item R \& C મૂલ્યો: $RC \gg 1/f_c$ અને $RC \ll 1/f_m$.
    \end{itemize}

    \textbf{ઓપરેશન:}
    \begin{enumerate}
        \item ડાયોડ પોઝિટિવ હાફ-સાયકલ દરમિયાન કન્ડક્ટ કરે છે.
        \item કેપેસિટર પીક વેલ્યુ સુધી ચાર્જ થાય છે.
        \item કેપેસિટર રેઝિસ્ટર દ્વારા ડિસ્ચાર્જ થાય છે.
        \item યોગ્ય ડિમોડ્યુલેશન માટે RC ટાઈમ કોન્સ્ટન્ટ મહત્વપૂર્ણ છે.
    \end{enumerate}

    \begin{center}
    \begin{circuitikz}[font=\footnotesize]
        \draw (0,0) node[left]{AM Input} to[diode, l=D] (2,0) coordinate (top);
        \draw (top) to[C, l=C] (2,-1.5) coordinate (bot) node[ground]{};
        \draw (top) -- (3.5,0) coordinate (out) node[right]{Output};
        \draw (3.5,0) to[R, l=R] (3.5,-1.5) -- (bot);
        \draw (0,-1.5) node[left]{GND} -- (bot);
    \end{tikzpicture}
    \captionof{figure}{Diode Detector}
    \end{center}

    \textbf{વેવફોર્મ્સ:}
    \begin{center}
    \begin{tikzpicture}[xscale=0.8, yscale=0.5]
        % AM Input
        \draw[blue] plot[domain=0:4, samples=100] (\x, {(1+0.5*cos(deg(2*\x)))*cos(deg(20*\x))});
        \node[below] at (2,-2) {AM Input};
        
        \draw[->] (4.5, 0) -- (5.5, 0);

        % Output
        \begin{scope}[xshift=6cm]
            \draw[red] plot[domain=0:4, samples=50] (\x, {1+0.5*cos(deg(2*\x))});
             \draw[blue, dashed, opacity=0.3] plot[domain=0:4, samples=100] (\x, {(1+0.5*cos(deg(2*\x)))*cos(deg(20*\x))});
             \node[below] at (2,-2) {Envelope Output};
        \end{scope}
    \end{tikzpicture}
    \end{center}

    \begin{mnemonicbox}
    "DRCO" - Diode Rectifies, Capacitor holds peaks, Output follows envelope
    \end{mnemonicbox}
\end{solutionbox}

\questionmarks{3}{c}{7}
\textbf{સુપર હેટેરોડાઈન રીસીવરનો બ્લોક ડાયગ્રામ દોરો અને સમજાવો.}

\begin{solutionbox}
    \textbf{સુપર હેટેરોડાઈન રીસીવર:}

    \textbf{બ્લોક ડાયગ્રામ:}
    
    \begin{center}
    \begin{tikzpicture}[auto, node distance=1.5cm,
        block/.style={draw, rectangle, minimum height=2em, minimum width=3em, align=center, fill=white},
        input/.style={coordinate},
        output/.style={coordinate},
        >=stealth
    ]
        \node [input] (antenna) {};
        \node [block, right=0.5cm of antenna] (rf) {RF\\Amp};
        \node [block, right=0.8cm of rf] (mixer) {Mixer};
        \node [block, right=0.8cm of mixer] (if) {IF\\Amp};
        \node [block, right=0.8cm of if] (det) {Detector};
        \node [block, right=0.8cm of det] (audio) {Audio\\Amp};
        \node [block, right=0.5cm of audio] (spk) {Speaker};
        \node [block, below=0.8cm of mixer] (lo) {Local\\Oscillator};

        \draw [->] (antenna) -- node[above, font=\tiny]{Antenna} (rf);
        \draw [->] (rf) -- (mixer);
        \draw [->] (mixer) -- (if);
        \draw [->] (if) -- (det);
        \draw [->] (det) -- (audio);
        \draw [->] (audio) -- (spk);
        \draw [->] (lo) -- (mixer);
        \draw [->, dashed] (rf.south) -- +(0,-0.5) -| (lo.west); % Gang tuning

    \end{tikzpicture}
    \captionof{figure}{Superheterodyne Receiver Block Diagram}
    \end{center}

    \textbf{દરેક બ્લોકનું કાર્ય:}
    \begin{enumerate}
        \item \textbf{RF એમ્પ્લિફાયર:} નબળા RF સિગ્નલ્સને એમ્પ્લિફાય કરે છે, સિલેક્ટિવિટી પૂરી પાડે છે.
        \item \textbf{લોકલ ઓસિલેટર:} સ્થિર આવૃત્તિ $f_{LO} = f_{RF} + f_{IF}$ (હાઈ-સાઈડ ઇન્જેક્શન) જનરેટ કરે છે.
        \item \textbf{મિક્સર:} RF સિગ્નલને લોકલ ઓસિલેટર સાથે કોમ્બાઈન કરે છે ($f_{IF} = |f_{RF} - f_{LO}|$).
        \item \textbf{IF એમ્પ્લિફાયર:} ફિક્સ્ડ આવૃત્તિ એમ્પ્લિફિકેશન (AM માટે 455kHz), રિસીવરનો ગેઈન/સિલેક્ટિવિટી.
        \item \textbf{ડિટેક્ટર:} IF સિગ્નલને ડિમોડ્યુલેટ કરે છે.
        \item \textbf{ઓડિયો એમ્પ્લિફાયર:} ડિમોડ્યુલેટેડ સિગ્નલને એમ્પ્લિફાય કરે છે.
    \end{enumerate}

    \textbf{ફાયદાઓ:} વધુ સારી સિલેક્ટિવિટી અને સેન્સિટિવિટી, સ્થિર ગેઈન.

    \begin{mnemonicbox}
    "RLMIDS" - RF amp, Local oscillator, Mixer, IF amp, Detector, Speaker
    \end{mnemonicbox}
\end{solutionbox}

% Part 2 Content

\questionmarks{3}{a}{3}
\textbf{AGC નો સિદ્ધાંત અને રેડિયો રિસીવરમાં તેની ઉપયોગિતા જણાવો.}

\begin{solutionbox}
    \textbf{AGC (ઓટોમેટિક ગેઈન કંટ્રોલ) સિદ્ધાંત:}

    \textbf{વ્યાખ્યા:}
    \begin{itemize}
        \item સર્કિટ જે સિગ્નલની શક્તિના આધારે ઓટોમેટિક રીતે રિસીવર ગેઈન એડજસ્ટ કરે છે.
        \item અલગ-અલગ ઇનપુટ સિગ્નલ છતાં સતત આઉટપુટ લેવલ જાળવે છે.
    \end{itemize}

    \textbf{કાર્ય સિદ્ધાંત:}
    \begin{enumerate}
        \item રિસીવ્ડ સિગ્નલની શક્તિને ડિટેક્ટ કરે છે.
        \item સિગ્નલના પ્રમાણમાં કંટ્રોલ વોલ્ટેજ જનરેટ કરે છે.
        \item મજબૂત સિગ્નલ માટે ગેઈન ઘટાડવા માટે નેગેટિવ ફીડબેક લાગુ કરે છે.
        \item નબળા સિગ્નલ માટે ગેઈન વધારે છે.
    \end{enumerate}

    \textbf{રેડિયો રિસીવરમાં એપ્લિકેશન:}
    \begin{itemize}
        \item \textbf{ઓવરલોડિંગ અટકાવે છે:} મજબૂત સિગ્નલ ડિસ્ટોર્શનથી રક્ષણ કરે છે.
        \item \textbf{ફેડિંગ માટે કમ્પેન્સેશન:} સિગ્નલ ફેડિંગ દરમિયાન અવાજનું સતત વોલ્યુમ જાળવે છે.
        \item \textbf{IF એમ્પ્લિફાયર કંટ્રોલ:} મુખ્યત્વે IF સ્ટેજ પર લાગુ કરવામાં આવે છે.
        \item \textbf{ડાયનેમિક રેન્જ સુધારે છે:} સિગ્નલની શક્તિની વિશાળ શ્રેણીને સંભાળે છે.
    \end{itemize}

    \begin{center}
    \begin{tikzpicture}[auto, node distance=1.5cm,
        block/.style={draw, rectangle, minimum height=2em, minimum width=3em, align=center},
        input/.style={coordinate},
        >=stealth
    ]
        \node [block] (rf) {RF Amp};
        \node [block, right=1cm of rf] (mixer) {Mixer};
        \node [block, right=1cm of mixer] (if) {IF Amp};
        \node [block, right=1cm of if] (det) {Detector};
        \node [coordinate, right=1cm of det] (out) {};
        \node [block, below=1cm of if] (agc) {AGC Circuit};

        \draw [->] (rf) -- (mixer);
        \draw [->] (mixer) -- (if);
        \draw [->] (if) -- (det);
        \draw [->] (det) -- (out) node[right] {Audio};
        
        \draw [->] (det.south) |- (agc.east);
        \draw [->] (agc.west) -| (rf.south);
        \draw [->] (agc.north) -- (if.south);

    \end{tikzpicture}
    \captionof{figure}{AGC in Superheterodyne Receiver}
    \end{center}

    \begin{mnemonicbox}
    "FADS" - Fading compensation, Automatic adjustment, Dynamic range, Signal consistency
    \end{mnemonicbox}
\end{solutionbox}

\questionmarks{3}{b}{4}
\textbf{IF frequency પર ટૂકનોંધ લખો.}

\begin{solutionbox}
    \textbf{ઇન્ટરમીડિએટ આવૃત્તિ (IF):}

    \textbf{વ્યાખ્યા:}
    \begin{itemize}
        \item સુપરહેટેરોડાઈન રિસીવર્સમાં ઇનકમિંગ RF સિગ્નલને કન્વર્ટ કરવામાં આવતી ફિક્સ્ડ આવૃત્તિ.
        \item RF સિગ્નલને લોકલ ઓસિલેટર સાથે મિક્સિંગ (હેટેરોડાઈનિંગ)નું પરિણામ.
    \end{itemize}

    \textbf{સ્ટાન્ડર્ડ IF મૂલ્યો:}
    \begin{itemize}
        \item \textbf{AM રેડિયો:} 455 kHz
        \item \textbf{FM રેડિયો:} 10.7 MHz
        \item \textbf{TV રિસીવર્સ:} 38-41 MHz
    \end{itemize}

    \textbf{મહત્વ:}
    \begin{itemize}
        \item \textbf{કન્સિસ્ટન્ટ ગેઈન:} એમ્પ્લિફાયર્સ ફિક્સ્ડ આવૃત્તિ પર કાર્ય કરે છે.
        \item \textbf{બેટર સિલેક્ટિવિટી:} ફિક્સ્ડ આવૃત્તિ પર નેરોબેન્ડ ફિલ્ટર્સ.
        \item \textbf{સિમ્પ્લિફાઈડ ડિઝાઈન:} ફિક્સ્ડ-આવૃત્તિ સ્ટેજના કાર્યક્ષમ ડિઝાઈન કરવું સરળ.
    \end{itemize}

    \textbf{પસંદગી માપદંડ:}
    \begin{itemize}
        \item ઇમેજ રિજેક્શન માટે પૂરતી ઊંચી.
        \item ફિલ્ટર Q અને ગેઈન માટે પૂરતી નીચી.
        \item સામાન્ય સિગ્નલના હાર્મોનિક્સને ટાળવી જોઈએ.
    \end{itemize}

    \textbf{ઇમેજ આવૃત્તિ ગણતરી:}
    \begin{itemize}
        \item હાઈ-સાઈડ ઇન્જેક્શન: $f_{image} = f_{RF} + 2f_{IF}$
        \item લો-સાઈડ ઇન્જેક્શન: $f_{image} = f_{RF} - 2f_{IF}$
    \end{itemize}

    \begin{mnemonicbox}
    "CIGS" - Conversion, Improved selectivity, Gain stability, Simplified design
    \end{mnemonicbox}
\end{solutionbox}

\questionmarks{3}{c}{7}
\textbf{FM detection માટેની ફેસ ડિસ્ક્રિમિનેટર સર્કિટ સમજાવો.}

\begin{solutionbox}
    \textbf{FM Detection માટે ફેસ ડિસ્ક્રિમિનેટર:}

    \textbf{હેતુ:}
    \begin{itemize}
        \item FM સિગ્નલમાં આવૃત્તિ વેરિએશનને એમ્પ્લિટ્યુડ વેરિએશનમાં કન્વર્ટ કરે છે.
        \item FM સિગ્નલને ડિમોડ્યુલેટ કરીને ઓરિજિનલ મેસેજ રિકવર કરે છે.
    \end{itemize}

    \textbf{કાર્ય સિદ્ધાંત:}
    \begin{enumerate}
        \item ઇનપુટ FM સિગ્નલ બે પાથમાં વિભાજિત થાય છે.
        \item રેફરન્સ પાથ સીધો સેન્ટર ટેપ પર જાય છે.
        \item ફેઝ-શિફ્ટેડ પાથ LC નેટવર્ક મારફતે પસાર થાય છે.
        \item ફેઝ શિફ્ટ આવૃત્તિ ડેવિએશન સાથે બદલાય છે.
        \item બે ડાયોડ્સ ફેઝ ડિફરન્સના પ્રમાણમાં વોલ્ટેજ ઉત્પન્ન કરે છે.
        \item આઉટપુટ વોલ્ટેજ ઇનપુટ આવૃત્તિ સાથે બદલાય છે.
    \end{enumerate}

    \begin{center}
    \begin{circuitikz}[font=\footnotesize]
        % Transformers are hard in basic tikz, simplifying
        \draw (0,0) node[left] {FM Input} to[L] (0,-2);
        \draw (1,0) to[L] (1,-2);
        \draw (1,-1) -- (2,-1); % Center tap
        
        \draw (1,0) -- (2,0) to[diode, l=$D_1$] (4,0) coordinate (top);
        \draw (1,-2) -- (2,-2) to[diode, l=$D_2$] (4,-2) coordinate (bot);
        
        \draw (top) to[R, l=$R_1$] (4,-1) coordinate (mid);
        \draw (mid) to[R, l=$R_2$] (bot);
        \draw (mid) -- (5,-1) node[right] {Output};
        
        \draw (top) to[C, l=$C_1$] (3,-1) -- (mid);
        \draw (bot) to[C, l=$C_2$] (3,-1);
        
        \draw (2,-1) to[C, l=$C_c$] (0,0); % Coupling
    \end{tikzpicture}
    \captionof{figure}{Foster-Seeley Discriminator}
    \end{center}

    \textbf{S-કર્વ રિસ્પોન્સ:}
    \begin{center}
    \begin{tikzpicture}[scale=0.8]
        \draw[->] (0,-1.5) -- (0,1.5) node[above] {$V_{out}$};
        \draw[->] (-2,0) -- (2,0) node[right] {$f$};
        \draw[thick, blue] (-1.5,-1) -- (1.5,1);
        \node[below] at (0,0) {$f_c$};
        \node[below] at (-1,0) {$f_c-\Delta f$};
        \node[below] at (1,0) {$f_c+\Delta f$};
    \end{tikzpicture}
    \captionof{figure}{Discriminator S-Curve}
    \end{center}

    \begin{mnemonicbox}
    "PSDO" - Phase shift Demodulates, Signal frequency determines Output
    \end{mnemonicbox}
\end{solutionbox}

\questionmarks{4}{a}{3}
\textbf{એનાલોગ અને ડિજિટલ કોમ્યુનિકેશન ટેક્નિક્સ સરખાવો.}

\begin{solutionbox}
    \textbf{એનાલોગ vs. ડિજિટલ કોમ્યુનિકેશનની તુલના:}

    \begin{center}
    \begin{tabulary}{\linewidth}{L L L}
        \hline
        \textbf{પેરામીટર} & \textbf{એનાલોગ કોમ્યુનિકેશન} & \textbf{ડિજિટલ કોમ્યુનિકેશન} \\
        \hline
        \textbf{સિગ્નલ} & કન્ટિન્યુઅસ વેવફોર્મ & ડિસ્ક્રીટ બાઈનરી વેલ્યુ \\
        \textbf{બેન્ડવિડ્થ} & ઓછી બેન્ડવિડ્થની જરૂર & વધુ બેન્ડવિડ્થની જરૂર \\
        \textbf{નોઈઝ ઇમ્યુનિટી} & ખરાબ, નોઈઝ એક્યુમ્યુલેટ થાય છે & ઉત્તમ, એરર કરેક્શન શક્ય \\
        \textbf{પાવર એફિશિયન્સી} & ઓછી કાર્યક્ષમ & વધુ કાર્યક્ષમ \\
        \textbf{ક્વોલિટી} & અંતર સાથે ઘટે છે & SNR થ્રેશોલ્ડ સુધી ક્વોલિટી જાળવે છે \\
        \textbf{મલ્ટિપ્લેક્સિંગ} & મુખ્યત્વે FDM વપરાય છે & મુખ્યત્વે TDM વપરાય છે \\
        \textbf{સિસ્ટમ જટિલતા} & સરળ & વધુ જટિલ \\
        \textbf{ખર્ચ} & ઓછો & વધુ પણ ઘટતો જાય છે \\
        \hline
    \end{tabulary}
    \end{center}

    \begin{mnemonicbox}
    "BNPQ MCE" - Bandwidth, Noise immunity, Power, Quality, Multiplexing, Complexity, Efficiency
    \end{mnemonicbox}
\end{solutionbox}

\questionmarks{4}{b}{4}
\textbf{એડેપ્ટિવ ડેલ્ટા મોડ્યુલેશન તેની એપ્લિકેશન સાથે સમજાવો.}

\begin{solutionbox}
    \textbf{એડેપ્ટિવ ડેલ્ટા મોડ્યુલેશન (ADM):}

    \textbf{કાર્ય સિદ્ધાંત:}
    \begin{itemize}
        \item ડેલ્ટા મોડ્યુલેશન (DM)નો સુધારેલો પ્રકાર.
        \item સિગ્નલ સ્લોપના આધારે વેરિએબલ સ્ટેપ સાઈઝ ઉપયોગ કરે છે.
        \item ઝડપી ફેરફારો માટે સ્ટેપ સાઈઝ વધારે છે (સ્લોપ ઓવરલોડ ઘટાડે છે).
        \item ધીમા ફેરફારો માટે સ્ટેપ સાઈઝ ઘટાડે છે (ગ્રેન્યુલર નોઈઝ ઘટાડે છે).
    \end{itemize}

    \textbf{બ્લોક ડાયગ્રામ:}
    \begin{center}
    \begin{tikzpicture}[auto, node distance=1.5cm,
        block/.style={draw, rectangle, minimum height=2em, minimum width=3em, align=center},
        sum/.style={draw, circle, inner sep=0pt, minimum size=6mm},
        >=stealth
    ]
        \node [coordinate] (input) {};
        \node [sum, right=0.5cm of input] (sum) {+};
        \node [block, right=1cm of sum] (logic) {Logic};
        \node [block, below=1cm of logic] (step) {Step Size\\Control};
        \node [block, below=1cm of sum] (int) {Integrator};
        \node [coordinate, right=1cm of logic] (output) {};

        \draw [->] (input) -- node[above]{Input} (sum);
        \draw [->] (sum) -- (logic);
        \draw [->] (logic) -- (output) node[right]{Output};
        \draw [->] (logic.south) -- (step.north);
        \draw [->] (step.west) -| (int.south);
        \draw [->] (int.north) -- (sum.south) node[near end, left] {-};

    \end{tikzpicture}
    \captionof{figure}{Adaptive Delta Modulation}
    \end{center}

    \textbf{એપ્લિકેશન:} વોઈસ ટ્રાન્સમિશન, ઓડિયો કમ્પ્રેશન.

    \begin{mnemonicbox}
    "VSOG" - Variable Step size Overcomes Granular noise \& slope overload
    \end{mnemonicbox}
\end{solutionbox}

\questionmarks{4}{c}{7}
\textbf{PCM system નો બ્લોક ડાયગ્રામ દોરો અને સમજાવો.}

\begin{solutionbox}
    \textbf{પલ્સ કોડ મોડ્યુલેશન (PCM) સિસ્ટમ:}

    \textbf{બ્લોક ડાયગ્રામ:}

    \begin{center}
    \begin{tikzpicture}[auto, node distance=1.2cm,
        block/.style={draw, rectangle, minimum height=2.5em, minimum width=3em, align=center, font=\small},
        >=stealth
    ]
        % Transmitter
        \node [coordinate] (input) {};
        \node [block, right=0.5cm of input] (lpf) {LPF};
        \node [block, right=0.5cm of lpf] (sampler) {Sampler};
        \node [block, right=0.5cm of sampler] (quant) {Quantizer};
        \node [block, right=0.5cm of quant] (enc) {Encoder};
        \node [coordinate, right=0.5cm of enc] (txout) {};
        
        \draw [->] (input) -- node[above, font=\tiny]{Analog} (lpf);
        \draw [->] (lpf) -- (sampler);
        \draw [->] (sampler) -- (quant);
        \draw [->] (quant) -- (enc);
        \draw [->] (enc) -- (txout) node[right, font=\tiny]{Channel};

        % Receiver
        \node [block, below=1.5cm of enc] (dec) {Decoder};
        \node [coordinate, right=0.5cm of dec] (rxin) {};
        \node [block, left=0.5cm of dec] (dac) {DAC};
        \node [block, left=0.5cm of dac] (lpf2) {LPF};
        \node [coordinate, left=0.5cm of lpf2] (output) {};
        
        \draw [<-] (dec) -- (rxin) node[right, font=\tiny]{Received};
        \draw [->] (dec) -- (dac);
        \draw [->] (dac) -- (lpf2);
        \draw [->] (lpf2) -- (output) node[left, font=\tiny]{Output};

        \draw[dashed] (rxin) -- (txout);
    \end{tikzpicture}
    \captionof{figure}{PCM System Block Diagram}
    \end{center}

    \textbf{ટ્રાન્સમીટર કોમ્પોનેન્ટ્સ:}
    \begin{enumerate}
        \item \textbf{સેમ્પલ \& હોલ્ડ:} નાયક્વિસ્ટ રેટ ($f_s \ge 2f_{max}$) પર સેમ્પલ કરે છે.
        \item \textbf{ક્વોન્ટાઇઝર:} ડિસ્ક્રીટ લેવલમાં વિભાજિત કરે છે.
        \item \textbf{એન્કોડર:} બાઇનરી કોડમાં કન્વર્ટ કરે છે.
    \end{enumerate}

    \textbf{રિસીવર કોમ્પોનેન્ટ્સ:}
    \begin{enumerate}
        \item \textbf{ડિકોડર:} બાઇનરીમાંથી ક્વોન્ટાઇઝ્ડ લેવલમાં.
        \item \textbf{DAC:} સ્ટેરકેસ એપ્રોક્સિમેશન.
        \item \textbf{લો-પાસ ફિલ્ટર:} ઓરિજિનલ વેવફોર્મ રિકન્સ્ટ્રક્ટ કરે છે.
    \end{enumerate}

    \begin{mnemonicbox}
    "SQEC-DFL" - Sample, Quantize, Encode, Channel - Decode, Filter, Listen
    \end{mnemonicbox}
\end{solutionbox}

\questionmarks{4}{a}{3}
\textbf{ક્વોન્ટાઇઝેશન રીત અને તેની ઉપયોગિતા સમજાવો.}

\begin{solutionbox}
    \textbf{ક્વોન્ટાઇઝેશન પ્રક્રિયા અને તેની આવશ્યકતા:}

    \textbf{વ્યાખ્યા:} સતત એમ્પ્લિટ્યુડ મૂલ્યોને ડિસ્ક્રીટ લેવલમાં મેપિંગ કરવાની પ્રક્રિયા.

    \textbf{પ્રકારો:}
    \begin{itemize}
        \item \textbf{યુનિફોર્મ ક્વોન્ટાઇઝેશન:} સમાન સ્ટેપ સાઇઝ.
        \item \textbf{નોન-યુનિફોર્મ ક્વોન્ટાઇઝેશન:} વેરિએબલ સ્ટેપ સાઇઝ.
    \end{itemize}

    \textbf{આવશ્યકતા:}
    \begin{itemize}
        \item \textbf{ડિજિટલ રજૂઆત:} બાઇનરી ફોર્મેટમાં કન્વર્ઝન.
        \item \textbf{સ્ટોરેજ કાર્યક્ષમતા:} મર્યાદિત સ્ટોરેજ.
        \item \textbf{પ્રોસેસિંગ ક્ષમતા:} DSP શક્ય બનાવે છે.
        \item \textbf{ટ્રાન્સમિશન ફાયદા:} એરર કરેક્શન.
    \end{itemize}

    \textbf{ક્વોન્ટાઇઝેશન એરર:} મહત્તમ એરર = $\pm Q/2$. $SQNR = 6.02n + 1.76$ dB.

    \begin{center}
    \begin{tikzpicture}[scale=0.8]
        \draw[->] (0,0) -- (4,0) node[right] {t};
        \draw[->] (0,0) -- (0,3) node[above] {V};
        \draw[blue] (0,0) -- (4,3);
        \draw[red, step=0.5] (0,0) grid (4,3);
        \foreach \x in {0,0.5,...,3.5} {
            \draw[thick, red] (\x, {0.75*\x}) -- (\x+0.5, {0.75*\x});
            \draw[thick, red] (\x+0.5, {0.75*\x}) -- (\x+0.5, {0.75*(\x+0.5)});
        }
    \end{tikzpicture}
    \captionof{figure}{Quantization Staircase}
    \end{center}

    \begin{mnemonicbox}
    "DEBS" - Digitization Enables Binary Storage
    \end{mnemonicbox}
\end{solutionbox}

\questionmarks{4}{b}{4}
\textbf{PCM રીસીવર સમજાવો.}

\begin{solutionbox}
    \textbf{PCM રીસીવર:}

    \begin{center}
    \begin{tikzpicture}[auto, node distance=1.5cm,
        block/.style={draw, rectangle, minimum height=2em, minimum width=3em, align=center},
        >=stealth
    ]
        \node [block] (buf) {Buffer};
        \node [block, right=1cm of buf] (dec) {Decoder};
        \node [block, right=1cm of dec] (dac) {DAC};
        \node [block, right=1cm of dac] (lpf) {Low Pass\\Filter};
        \node [coordinate, left=1cm of buf] (in) {};
        \node [coordinate, right=1cm of lpf] (out) {};

        \draw [->] (in) -- node[above]{PCM Input} (buf);
        \draw [->] (buf) -- (dec);
        \draw [->] (dec) -- (dac);
        \draw [->] (dac) -- (lpf);
        \draw [->] (lpf) -- (out) node[right]{Analog Output};
    \end{tikzpicture}
    \captionof{figure}{PCM Receiver}
    \end{center}

    \textbf{કોમ્પોનેન્ટ્સ:}
    \begin{itemize}
        \item \textbf{બફર:} ડેટા સ્ટોર કરે છે, જિટર ઘટાડે છે.
        \item \textbf{ડિકોડર:} બાઇનરીમાંથી ક્વોન્ટાઇઝ્ડ લેવલમાં.
        \item \textbf{DAC:} સ્ટેરકેસ વેવફોર્મ બનાવે છે.
        \item \textbf{લો-પાસ ફિલ્ટર:} વેવફોર્મ સ્મૂધ કરે છે.
    \end{itemize}

    \begin{mnemonicbox}
    "BDFL" - Buffer stores, Decoder converts, Filter smooths, Listen to output
    \end{mnemonicbox}
\end{solutionbox}

\questionmarks{4}{c}{7}
\textbf{સેમ્પલિંગ શું છે? સેમ્પલિંગના પ્રકારોને ટુંકમાં સમજાવો.}

\begin{solutionbox}
    \textbf{સેમ્પલિંગ:} કન્ટિન્યુઅસ-ટાઇમ સિગ્નલને ડિસ્ક્રીટ-ટાઇમમાં કન્વર્ટ કરવાની પ્રક્રિયા.

    \textbf{નાયક્વિસ્ટ થિયરમ:} $f_s \ge 2f_{max}$ એલિયાસિંગ અટકાવવા.

    \textbf{સેમ્પલિંગના પ્રકારો:}
    \begin{itemize}
        \item \textbf{આદર્શ સેમ્પલિંગ:} તાત્કાલિક સેમ્પલ (થિયોરેટિકલ).
        \item \textbf{નેચરલ સેમ્પલિંગ:} પલ્સ સિગ્નલ આકાર લે છે.
        \item \textbf{ફ્લેટ-ટોપ સેમ્પલિંગ:} સ્ટેરકેસ એપ્રોક્સિમેશન (સેમ્પલ એન્ડ હોલ્ડ).
    \end{itemize}

    \begin{center}
    \begin{tikzpicture}[xscale=0.8, yscale=0.6]
        % Signal
        \draw[blue] plot[domain=0:5, samples=50] (\x, {1+0.5*sin(deg(\x))});
        
        % Ideal
        \foreach \x in {0.5, 1.0, ..., 4.5} {
            \draw[->, red] (\x, 0) -- (\x, {1+0.5*sin(deg(\x))});
        }
        \node[right] at (5,1) {Ideal};
    \end{tikzpicture}
    \quad
    \begin{tikzpicture}[xscale=0.8, yscale=0.6]
        % Flat Top
        \foreach \x in {0, 0.5, ..., 4.5} {
            \draw[red] (\x, 0) rectangle (\x+0.4, {1+0.5*sin(deg(\x))});
        }
        \node[right] at (5,1) {Flat-Top};
    \end{tikzpicture}
    \captionof{figure}{Sampling Types}
    \end{center}

    \begin{mnemonicbox}
    "INF" - Ideal (impulses), Natural (pulse-shaped), Flat-top (staircase)
    \end{mnemonicbox}
\end{solutionbox}

\questionmarks{5}{a}{3}
\textbf{મલ્ટીપ્લેક્સિંગની આવશ્યક્તાઓની યાદી બનાવો.}

\begin{solutionbox}
    \textbf{મલ્ટીપ્લેક્સિંગની આવશ્યકતા:}

    \begin{itemize}
        \item \textbf{બેન્ડવિડ્થ ઉપયોગ}: કાર્યક્ષમ ઉપયોગ.
        \item \textbf{ખર્ચ ઘટાડો}: માધ્યમ શેર કરે છે.
        \item \textbf{ઇન્ફ્રાસ્ટ્રક્ચર ઓપ્ટિમાઇઝેશન}: ઓછા વાયર.
        \item \textbf{સ્પેક્ટ્રમ કાર્યક્ષમતા}: મહત્તમ ઉપયોગ.
        \item \textbf{નેટવર્ક ક્ષમતા}: વધુ વપરાશકર્તાઓ.
        \item \textbf{લવચીકતા}: ડાયનેમિક એલોકેશન.
    \end{itemize}

    \begin{mnemonicbox}
    "BCSINF" - Bandwidth, Cost, Spectrum, Infrastructure, Network capacity, Flexibility
    \end{mnemonicbox}
\end{solutionbox}

\questionmarks{5}{b}{4}
\textbf{DPCM નું કાર્ય સમજાવો.}

\begin{solutionbox}
    \textbf{ડિફરેન્શિયલ પલ્સ કોડ મોડ્યુલેશન (DPCM):}

    \begin{itemize}
        \item વર્તમાન અને અનુમાનિત સેમ્પલ વચ્ચેનો તફાવત એન્કોડ કરે છે.
        \item બિટ રેટ ઘટાડે છે.
    \end{itemize}

    \begin{center}
    \begin{tikzpicture}[auto, node distance=1.5cm,
        block/.style={draw, rectangle, minimum height=2em, minimum width=3em, align=center},
        sum/.style={draw, circle, inner sep=0pt, minimum size=6mm},
        >=stealth
    ]
        \node [coordinate] (in) {};
        \node [block, right=0.5cm of in] (adc) {ADC};
        \node [sum, right=0.5cm of adc] (sum) {+};
        \node [block, right=0.5cm of sum] (quant) {Quantizer};
        \node [block, right=0.5cm of quant] (enc) {Encoder};
        \node [coordinate, right=0.5cm of enc] (out) {};
        \node [block, below=0.8cm of sum] (pred) {Predictor};

        \draw [->] (in) -- (adc);
        \draw [->] (adc) -- (sum);
        \draw [->] (sum) -- (quant);
        \draw [->] (quant) -- (enc);
        \draw [->] (enc) -- (out) node[right]{Output};
        \draw [->] (quant.south) |- (pred.east);
        \draw [->] (pred.north) -- (sum.south) node[left]{-};
    \end{tikzpicture}
    \captionof{figure}{DPCM Transmitter}
    \end{center}

    \begin{mnemonicbox}
    "PDQE" - Predict sample, Difference calculated, Quantize error, Encode result
    \end{mnemonicbox}
\end{solutionbox}

\questionmarks{5}{c}{7}
\textbf{બાઈનરી ડેટા 1011001 નીચે પ્રમાણેની લાઈન કોડિંગ ટેકનીકથી ટ્રાન્સમીટ થાય છે (i) યુનિપોલાર RZ અને NRZ (ii) પોલાર RZ અને NRZ (iii) AMI (iv) Manchester. બધા માટે વેવ ફોર્મ દોરો.}

\begin{solutionbox}
    \textbf{બાઈનરી ડેટા 1011001 માટે લાઈન કોડિંગ:}

    \begin{center}
    \begin{tikzpicture}[xscale=1.5, yscale=0.6]
        \foreach \x/\b in {0/1, 1/0, 2/1, 3/1, 4/0, 5/0, 6/1} {
            \node at (\x+0.5, 9) {\textbf{\b}};
            \draw[dashed, gray] (\x, -2) -- (\x, 8.5);
            \draw[dashed, gray] (\x+1, -2) -- (\x+1, 8.5);
        }
        
        \node[left] at (0, 7.5) {Unipolar NRZ};
        \draw[thick, blue] (0,8) -- (1,8) -- (1,7) -- (2,7) -- (2,8) -- (4,8) -- (4,7) -- (6,7) -- (6,8) -- (7,8);

        \node[left] at (0, 5.5) {Unipolar RZ};
        \draw[thick, blue] (0,6) -- (0.5,6) -- (0.5,5) -- (2,5) -- (2,6) -- (2.5,6) -- (2.5,5) -- (3,6) -- (3.5,6) -- (3.5,5) -- (6,5) -- (6,6) -- (6.5,6) -- (6.5,5) -- (7,5);

        \node[left] at (0, 3.5) {Polar NRZ};
        \draw[thick, red] (0,4) -- (1,4) -- (1,3) -- (2,3) -- (2,4) -- (4,4) -- (4,3) -- (6,3) -- (6,4) -- (7,4);

        \node[left] at (0, 1.5) {Polar RZ};
        \draw[thick, red] (0,2) -- (0.5,2) -- (0.5,1.5) -- (1,1.5) -- (1,1) -- (1.5,1) -- (1.5,1.5) -- (2,1.5) -- (2,2) -- (2.5,2) -- (2.5,1.5) -- (3,1.5) -- (3,2) -- (3.5,2) -- (3.5,1.5) -- (4,1.5) -- (4,1) -- (4.5,1) -- (4.5,1.5) -- (5,1.5) -- (5,1) -- (5.5,1) -- (5.5,1.5) -- (6,1.5) -- (6,2) -- (6.5,2) -- (6.5,1.5) -- (7,1.5);
        
        \node[left] at (0, -0.5) {AMI};
         \draw[thick, green] (0,0) -- (0.5,0) -- (0.5,-0.5) -- (1,-0.5) -- (2,-0.5) -- (2,-1) -- (2.5,-1) -- (2.5,-0.5) -- (3,-0.5) -- (3,0) -- (3.5,0) -- (3.5,-0.5) -- (6,-0.5) -- (6,-1) -- (6.5,-1) -- (6.5,-0.5) -- (7,-0.5);

        \node[left] at (0, -2.5) {Manchester};
        \draw[thick, orange] (0,-3) -- (0.5,-3) -- (0.5,-2) -- (1,-2) -- (1,-2) -- (1.5,-2) -- (1.5,-3) -- (2,-3) -- (2,-3) -- (2.5,-3) -- (2.5,-2) -- (3,-2) -- (3,-3) -- (3.5,-3) -- (3.5,-2) -- (4,-2) -- (4,-2) -- (4.5,-2) -- (4.5,-3) -- (5,-3) -- (5,-2) -- (5.5,-2) -- (5.5,-3) -- (6,-3) -- (6,-3) -- (6.5,-3) -- (6.5,-2) -- (7,-2);
    \end{tikzpicture}
    \captionof{figure}{Line Coding Waveforms}
    \end{center}

    \begin{mnemonicbox}
    "UPRMA" - Unipolar, Polar, Return-to-zero, Manchester, AMI
    \end{mnemonicbox}
\end{solutionbox}

\questionmarks{5}{a}{3}
\textbf{પોલાર RZ અને NRZ ફોર્મેટ સમજાવો.}

\begin{solutionbox}
    \textbf{પોલાર RZ અને NRZ લાઈન કોડિંગ:}

    \textbf{પોલાર NRZ:}
    \begin{itemize}
        \item બાઈનરી 1: $+V$, બાઈનરી 0: $-V$.
        \item ઝીરોમાં રિટર્ન નથી.
        \item સરળ પણ ખરાબ ક્લોક રિકવરી.
    \end{itemize}

    \textbf{પોલાર RZ:}
    \begin{itemize}
        \item બાઈનરી 1: $+V$ અર્ધા બિટ માટે, બાકી 0.
        \item બાઈનરી 0: $-V$ અર્ધા બિટ માટે, બાકી 0.
        \item સેલ્ફ-ક્લોકિંગ, વધુ બેન્ડવિડ્થ.
    \end{itemize}

    \begin{mnemonicbox}
    "HZRT" - Half bit active + Zero Return in RZ, full Time in NRZ
    \end{mnemonicbox}
\end{solutionbox}

\questionmarks{5}{b}{4}
\textbf{ડેલ્ટા મોડ્યુલેશન ટૂંકમાં સમજાવો.}

\begin{solutionbox}
    \textbf{ડેલ્ટા મોડ્યુલેશન (DM):}

    \begin{itemize}
        \item સૌથી સરળ ડિફરેન્શિયલ એન્કોડિંગ.
        \item પ્રતિ સેમ્પલ 1 બિટ.
        \item સ્લોપ ઓવરલોડ અને ગ્રેન્યુલર નોઈઝ.
    \end{itemize}

    \begin{center}
    \begin{tikzpicture}[auto, node distance=1.5cm,
        block/.style={draw, rectangle, minimum height=2em, minimum width=3em, align=center},
        sum/.style={draw, circle, inner sep=0pt, minimum size=6mm},
        >=stealth
    ]
        \node [coordinate] (in) {};
        \node [sum, right=0.5cm of in] (sum) {+};
        \node [block, right=1cm of sum] (comp) {Comparator};
        \node [coordinate, right=1cm of comp] (out) {};
        \node [block, below=1cm of sum] (int) {Integrator};

        \draw [->] (in) -- (sum) node[near start, above] {$x(t)$};
        \draw [->] (sum) -- (comp);
        \draw [->] (comp) -- (out) node[right] {Output};
        \draw [->] (comp.east) -- ++(0.5,0) |- (int.east);
        \draw [->] (int.north) -- (sum.south) node[left] {-};
    \end{tikzpicture}
    \captionof{figure}{Delta Modulator}
    \end{center}

    \begin{mnemonicbox}
    "1BSG" - 1 Bit per Sample, Slope overload and Granular noise limitations
    \end{mnemonicbox}
\end{solutionbox}

\questionmarks{5}{c}{7}
\textbf{PCM-TDM સિસ્ટમ સમજાવો.}

\begin{solutionbox}
    \textbf{PCM-TDM સિસ્ટમ:}

    \begin{itemize}
        \item PCM અને TDM નો સંયુક્ત સિસ્ટમ.
        \item એનાલોગ ચેનલ $\to$ PCM $\to$ મલ્ટિપ્લેક્સ.
    \end{itemize}

    \begin{center}
    \begin{tikzpicture}[auto, node distance=1cm,
        block/.style={draw, rectangle, minimum height=2em, minimum width=3em, align=center},
        mux/.style={draw, trapezium, trapezium angle=60, shape border rotate=270, minimum height=3em, align=center},
        >=stealth
    ]
        \node [block] (pcm1) {PCM 1};
        \node [block, below=0.5cm of pcm1] (pcm2) {PCM 2};
        \node [below=0.5cm of pcm2] (dots) {\vdots};
        \node [block, below=0.5cm of dots] (pcmn) {PCM N};

        \node [mux, right=1.5cm of pcm2, minimum height=4cm] (mux) {MUX};
        \node [block, right=1cm of mux] (frame) {Frame\\Format};
        \node [coordinate, right=1cm of frame] (out) {};

        \draw [->] (pcm1) -- (mux.west |- pcm1.east);
        \draw [->] (pcm2) -- (mux.west |- pcm2.east);
        \draw [->] (pcmn) -- (mux.west |- pcmn.east);
        \draw [->] (mux) -- (frame);
        \draw [->] (frame) -- (out) node[right] {PCM-TDM Out};

        \node [left=0.5cm of pcm1] {Ch 1};
        \node [left=0.5cm of pcm2] {Ch 2};
        \node [left=0.5cm of pcmn] {Ch N};
    \end{tikzpicture}
    \captionof{figure}{PCM-TDM System}
    \end{center}

    \textbf{TDM ફ્રેમ:}
    \begin{center}
    \begin{tikzpicture}
        \draw (0,0) rectangle (1.5,1) node[midway] {Sync};
        \draw (1.5,0) rectangle (3,1) node[midway] {Ch 1};
        \draw (3,0) rectangle (4.5,1) node[midway] {Ch 2};
        \draw (4.5,0) rectangle (6,1) node[midway] {...};
        \draw (6,0) rectangle (7.5,1) node[midway] {Ch N};
        \draw[<->] (0,1.2) -- (7.5,1.2) node[midway, above] {One Frame ($1/f_s$)};
    \end{tikzpicture}
    \end{center}

    \begin{mnemonicbox}
    "MSQT" - Multiplex, Sample, Quantize, Transmit
    \end{mnemonicbox}
\end{solutionbox}

\end{document}
