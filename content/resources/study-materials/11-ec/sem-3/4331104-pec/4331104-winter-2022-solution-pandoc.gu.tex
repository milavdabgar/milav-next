\documentclass[10pt,a4paper]{article}

% content/resources/templates/preamble.tex
\usepackage[margin=0.6in]{geometry}
\author{Milav Dabgar}
\usepackage{amsmath,amssymb,amsthm}
\usepackage{booktabs}
\usepackage{multirow}
\usepackage{xcolor}
\usepackage{tcolorbox}
\tcbuselibrary{breakable,skins}
\usepackage[colorlinks=true,linkcolor=blue]{hyperref}
\usepackage{titlesec}
\usepackage{enumitem}
\usepackage{tikz}
\usepackage{pgfplots}
\usepackage{circuitikz}
\usepackage[version=4]{mhchem}
\usepackage{longtable}
\usepackage{array}
\usepackage{float}
\usepackage{caption}
\usepackage{listings}

\lstset{
  basicstyle=\small\ttfamily,
  breaklines=true,
  breakatwhitespace=false,
  postbreak=\mbox{\textcolor{red}{$\hookrightarrow$}\space},
  float=false,
  numbers=left,
  numberstyle=\tiny\color{gray},
  numbersep=10pt,
  xleftmargin=2em,
  keywordstyle=\color{blue},
  commentstyle=\color{green!60!black},
  stringstyle=\color{purple},
  backgroundcolor=\color{gray!5},
  showstringspaces=false,
  tabsize=2,
  captionpos=b,
  keepspaces=true,
  columns=flexible
}

\pgfplotsset{compat=1.18}
\usetikzlibrary{shapes,arrows,positioning,calc,patterns,decorations.pathmorphing,decorations.markings,arrows.meta}

% Color scheme
\definecolor{headcolor}{RGB}{0,102,204}
\definecolor{keycolor}{RGB}{220,20,60}
\definecolor{solutioncolor}{RGB}{34,139,34}
\definecolor{mnemoniccolor}{RGB}{148,0,211}
\definecolor{codecolor}{RGB}{0,0,100}

% Spacing
\setlength{\parskip}{3pt}
\setlist[itemize]{nosep}
\setlist[enumerate]{nosep}

% Title formatting
\titleformat{\section}{\Large\bfseries\color{headcolor}}{\thesection}{1em}{}
\titleformat{\subsection}{\large\bfseries\color{headcolor}}{\thesubsection}{1em}{}

% Pandoc tightlist compatibility
\providecommand{\tightlist}{%
  \setlength{\itemsep}{0pt}\setlength{\parskip}{0pt}}

% Pandoc longtable compatibility
\newcounter{none}
\def\thenone{}


% content/resources/templates/gujarati-boxes.tex
\usepackage{fontspec}
\usepackage{polyglossia}

% Set Gujarati as main language (document is primarily in Gujarati)
% Note: gloss-gujarati.ldf doesn't exist in polyglossia, but it will use hyphenation patterns
\setdefaultlanguage{gujarati}
\setotherlanguage{english}

% Configure Gujarati font properly
% Use Language=Default to prevent polyglossia from trying to add language-specific features
% that don't exist for Gujarati, which causes "empty feature" warnings
\newfontfamily\gujaratifont[Script=Gujarati,AutoFakeBold=2.5,AutoFakeSlant=0.3]{Noto Sans Gujarati}
\setmainfont[Script=Gujarati,AutoFakeBold=2.5,AutoFakeSlant=0.3]{Noto Sans Gujarati}
% Use Noto Sans Gujarati for monospace to support Gujarati in text
\setmonofont[Scale=0.9]{Noto Sans Gujarati}

% Configure English to use the same font
\newfontfamily\englishfont[Script=Gujarati,AutoFakeBold=2.5,AutoFakeSlant=0.3]{Noto Sans Gujarati}

% Translations for polyglossia
\gappto\captionsgujarati{
  \renewcommand{\tablename}{કોષ્ટક}
  \renewcommand{\figurename}{આકૃતિ}
}

% Helper for TikZ nodes to ensure Gujarati font
\newcommand{\gu}[1]{{\gujaratifont #1}}

% Custom environments
\newtcolorbox{solutionbox}{
    breakable,
    enhanced,
    colback=solutioncolor!5!white,
    colframe=solutioncolor!75!black,
    fonttitle=\bfseries,
    title=જવાબ
}

\newtcolorbox{solutionboxnobreak}{
 colback=solutioncolor!5!white,
 colframe=solutioncolor!75!black,
 fonttitle=\bfseries,
 title=જવાબ
}

\newtcolorbox{keyformula}{
 breakable,
 enhanced,
 colback=keycolor!5!white,
 colframe=keycolor!75!black,
 fonttitle=\bfseries,
 title=રાસાયણિક સમીકરણ/સૂત્ર
}

\newtcolorbox{mnemonicbox}{
 breakable,
 enhanced,
 colback=mnemoniccolor!5!white,
 colframe=mnemoniccolor!75!black,
 fonttitle=\bfseries,
 title=મેમરી ટ્રીક
}


\begin{document}

\begin{center}
{\Huge\bfseries\color{headcolor} Subject Name (Gujarati)}\\[5pt]
{\LARGE 4331104 -- Winter 2022}\\[3pt]
{\large Semester 1 Study Material}\\[3pt]
{\normalsize\textit{Detailed Solutions and Explanations}}
\end{center}

\vspace{10pt}

\subsection*{પ્રશ્ન 1(a) [3
ગુણ]}\label{q1a}

\textbf{મોડયુલેશન શું છે? તેની જરૂરિયાત શું છે?}

\begin{solutionbox}
મોડયુલેશન એ એક ઉચ્ચ આવૃત્તિવાળા કેરિયર સિગ્નલના એક અથવા વધુ
ગુણધર્મો (amplitude, frequency, અથવા phase) ને માહિતી ધરાવતા સિગ્નલ સાથે
બદલવાની પ્રક્રિયા છે.

\textbf{મોડયુલેશનની જરૂરિયાત:}

\begin{itemize}
\tightlist
\item
  \textbf{એન્ટેના સાઇઝ ઘટાડવા}: વ્યવહારિક એન્ટેના સાઇઝ શક્ય બનાવે છે (λ = c/f)
\item
  \textbf{મલ્ટિપ્લેક્સિંગ}: અનેક સિગ્નલ્સને એક માધ્યમમાં મોકલવા માટે
\item
  \textbf{નોઇઝ ઘટાડવા}: ઉચ્ચ આવૃત્તિ બેન્ડમાં શિફ્ટ કરીને SNR સુધારે છે
\item
  \textbf{રેન્જ વધારવા}: ટ્રાન્સમિશન અંતર વધારે છે
\end{itemize}

\end{solutionbox}
\begin{mnemonicbox}
``AMEN'' - Antenna size, Multiplexing, Eliminate
noise, New range

\end{mnemonicbox}
\subsection*{પ્રશ્ન 1(b) [4
ગુણ]}\label{q1b}

\textbf{એમ્પલીટયૂડ મોડયુલેશન માટે વૉલ્ટેજ સમીકરણ મેળવો.}

\begin{solutionbox}
AM માં, કેરિયર સિગ્નલ મેસેજ સિગ્નલ દ્વારા મોડ્યુલેટેડ થાય છે.

\textbf{ગાણિતિક સ્થાપના:}

\begin{itemize}
\tightlist
\item
  \textbf{કેરિયર સિગ્નલ}: \(e_c(t) = A_c \cos(2\pi f_c t)\)
\item
  \textbf{મેસેજ સિગ્નલ}: \(e_m(t) = A_m \cos(2\pi f_m t)\)
\item
  \textbf{ઇન્સ્ટન્ટનીયસ એમ્પ્લિટ્યુડ}: \(A_i = A_c + e_m(t)\)
\item
  \textbf{AM સિગ્નલ}: \(e_{AM}(t) = A_i \cos(2\pi f_c t)\)
\item
  \textbf{સબ્સ્ટિટ્યુશન}:
  \(e_{AM}(t) = [A_c + A_m \cos(2\pi f_m t)] \cos(2\pi f_c t)\)
\item
  \textbf{એક્સ્પેન્ડિંગ}:
  \(e_{AM}(t) = A_c\cos(2\pi f_c t) + A_m\cos(2\pi f_m t)\cos(2\pi f_c t)\)
\item
  \textbf{ફાઇનલ ઇક્વેશન}:
  \(e_{AM}(t) = A_c\cos(2\pi f_c t) + \frac{A_m}{2}\cos(2\pi(f_c+f_m)t) + \frac{A_m}{2}\cos(2\pi(f_c-f_m)t)\)
\end{itemize}

\end{solutionbox}
\begin{mnemonicbox}
``CAT'' - Carrier, Addition, Three components
(carrier + 2 sidebands)

\end{mnemonicbox}
\subsection*{પ્રશ્ન 1(c) [7
ગુણ]}\label{q1c}

\textbf{નોઈસ સિગ્નલને વર્ગીકૃત કરો ફ્લીકર નોઈસ, શૉટ નોઈસ અને થર્મલ નોઈસ
સમજાવો.}

\begin{solutionbox}

\textbf{નોઇઝ વર્ગીકરણ:}

{\def\LTcaptype{none} % do not increment counter
\begin{longtable}[]{@{}
  >{\raggedright\arraybackslash}p{(\linewidth - 4\tabcolsep) * \real{0.1875}}
  >{\raggedright\arraybackslash}p{(\linewidth - 4\tabcolsep) * \real{0.2812}}
  >{\raggedright\arraybackslash}p{(\linewidth - 4\tabcolsep) * \real{0.5312}}@{}}
\toprule\noalign{}
\begin{minipage}[b]{\linewidth}\raggedright
પ્રકાર
\end{minipage} & \begin{minipage}[b]{\linewidth}\raggedright
સ્ત્રોત
\end{minipage} & \begin{minipage}[b]{\linewidth}\raggedright
લક્ષણો
\end{minipage} \\
\midrule\noalign{}
\endhead
\bottomrule\noalign{}
\endlastfoot
\textbf{બાહ્ય નોઇઝ} & એટમોસ્ફેરિક, સ્પેસ, ઔદ્યોગિક, માનવ-નિર્મિત & કોમ્યુનિકેશન
સિસ્ટમની બહારથી ઉત્પન્ન થાય છે \\
\textbf{આંતરિક નોઇઝ} & થર્મલ, શોટ, ટ્રાન્ઝિટ-ટાઇમ, ફ્લિકર & કોમ્પોનેન્ટ્સની અંદરથી
ઉત્પન્ન થાય છે \\
\end{longtable}
}

\textbf{આંતરિક નોઈઝના પ્રકાર:}

\begin{itemize}
\tightlist
\item
  \textbf{ફ્લિકર નોઈઝ}:

  \begin{itemize}
  \tightlist
  \item
    નીચી આવૃત્તિઓ પર થાય છે (1 kHz નીચે)
  \item
    આવૃત્તિના વ્યસ્ત પ્રમાણમાં (1/f નોઇઝ)
  \item
    સેમિકન્ડક્ટર ડિવાઇસ અને કાર્બન રેસિસ્ટર્સમાં સામાન્ય છે
  \end{itemize}
\item
  \textbf{શોટ નોઈઝ}:

  \begin{itemize}
  \tightlist
  \item
    કરંટ કેરિયર્સના રેન્ડમ ફ્લક્ચુએશન્સને કારણે
  \item
    અચલ પાવર ડેન્સિટી સાથે વ્હાઇટ નોઇઝ
  \item
    ડાયોડ અને ટ્રાન્ઝિસ્ટર જેવી એક્ટિવ ડિવાઇસમાં થાય છે
  \end{itemize}
\item
  \textbf{થર્મલ નોઈઝ}:

  \begin{itemize}
  \tightlist
  \item
    કન્ડક્ટરમાં ઇલેક્ટ્રોન્સની રેન્ડમ ગતિને કારણે
  \item
    તાપમાન અને બેન્ડવિડ્થના સીધા પ્રમાણમાં
  \item
    બધા પેસિવ કોમ્પોનેન્ટ્સમાં હાજર
  \item
    જોનસન નોઇઝ અથવા વ્હાઇટ નોઇઝ તરીકે પણ ઓળખાય છે
  \end{itemize}
\end{itemize}

\end{solutionbox}
\begin{mnemonicbox}
``FAST'' - Flicker (low frequency), Active (shot),
Semiconductor (flicker), Temperature (thermal)

\end{mnemonicbox}
\subsection*{પ્રશ્ન 1(c) OR [7
ગુણ]}\label{q1c}

\textbf{EM wave spectrum ના વિવિધ બેન્ડની એપ્લિકેશન લખો.}

\begin{solutionbox}

\textbf{EM સ્પેક્ટ્રમ એપ્લિકેશન્સ:}

{\def\LTcaptype{none} % do not increment counter
\begin{longtable}[]{@{}
  >{\raggedright\arraybackslash}p{(\linewidth - 4\tabcolsep) * \real{0.3478}}
  >{\raggedright\arraybackslash}p{(\linewidth - 4\tabcolsep) * \real{0.3478}}
  >{\raggedright\arraybackslash}p{(\linewidth - 4\tabcolsep) * \real{0.3043}}@{}}
\toprule\noalign{}
\begin{minipage}[b]{\linewidth}\raggedright
ફ્રીક્વન્સી બેન્ડ
\end{minipage} & \begin{minipage}[b]{\linewidth}\raggedright
ફ્રીક્વન્સી રેન્જ
\end{minipage} & \begin{minipage}[b]{\linewidth}\raggedright
એપ્લિકેશન્સ
\end{minipage} \\
\midrule\noalign{}
\endhead
\bottomrule\noalign{}
\endlastfoot
\textbf{ELF} (Extremely Low Frequency) & 3Hz - 30Hz & સબમરીન
કોમ્યુનિકેશન \\
\textbf{VLF} (Very Low Frequency) & 3kHz - 30kHz & નેવિગેશન, ટાઇમ
સિગ્નલ્સ \\
\textbf{LF} (Low Frequency) & 30kHz - 300kHz & AM રેડિયો, નેવિગેશન \\
\textbf{MF} (Medium Frequency) & 300kHz - 3MHz & AM બ્રોડકાસ્ટિંગ,
મેરિટાઇમ \\
\textbf{HF} (High Frequency) & 3MHz - 30MHz & શોર્ટવેવ રેડિયો, એમેચ્યોર
રેડિયો \\
\textbf{VHF} (Very High Frequency) & 30MHz - 300MHz & FM રેડિયો, TV
બ્રોડકાસ્ટિંગ, એર ટ્રાફિક કંટ્રોલ \\
\textbf{UHF} (Ultra High Frequency) & 300MHz - 3GHz & TV બ્રોડકાસ્ટિંગ,
મોબાઇલ ફોન, WiFi, બ્લૂટૂથ \\
\textbf{SHF} (Super High Frequency) & 3GHz - 30GHz & સેટેલાઇટ કોમ્યુનિકેશન,
રડાર, WiFi \\
\textbf{EHF} (Extremely High Frequency) & 30GHz - 300GHz & રેડિયો
એસ્ટ્રોનોમી, 5G, મિલિમીટર-વેવ રડાર \\
\textbf{Infrared} & 300GHz - 400THz & રિમોટ કંટ્રોલ, થર્મલ ઇમેજિંગ, ફાઇબર
ઓપ્ટિક્સ \\
\textbf{Visible Light} & 400THz - 800THz & ફાઇબર ઓપ્ટિક્સ, LiFi,
ફોટોગ્રાફી \\
\textbf{Ultraviolet} & 800THz - 30PHz & સ્ટેરિલાઇઝેશન, ફ્લોરેસન્સ, સિક્યુરિટી \\
\textbf{X-rays} & 30PHz - 30EHz & મેડિકલ ઇમેજિંગ, સિક્યુરિટી સ્ક્રીનિંગ \\
\textbf{Gamma rays} & \textgreater30EHz & મેડિકલ ટ્રીટમેન્ટ, ન્યુક્લિયર
ડિટેક્શન \\
\end{longtable}
}

\end{solutionbox}
\begin{mnemonicbox}
``Every Very Lovely Monkey Has Visited Uncle Sam's
House Easily In Visible Upper Xtra Gamma'' (દરેક બેન્ડનો પ્રથમ અક્ષર)

\end{mnemonicbox}
\subsection*{પ્રશ્ન 2(a) [3
ગુણ]}\label{q2a}

\textbf{DSBની સરખામણીએ SSBના ફાયદાઓ લખો.}

\begin{solutionbox}

\textbf{SSBના DSB પર ફાયદાઓ:}

{\def\LTcaptype{none} % do not increment counter
\begin{longtable}[]{@{}ll@{}}
\toprule\noalign{}
ફાયદો & વર્ણન \\
\midrule\noalign{}
\endhead
\bottomrule\noalign{}
\endlastfoot
\textbf{બેન્ડવિથ એફિશિયન્સી} & અડધી બેન્ડવિથનો ઉપયોગ (માત્ર એક સાઇડબેન્ડ) \\
\textbf{પાવર એફિશિયન્સી} & ઓછી ટ્રાન્સમિટર પાવરની જરૂર (83.33\% પાવર
સેવિંગ) \\
\textbf{ઘટાડેલું ફેડિંગ} & સિલેક્ટિવ ફેડિંગને ઓછું સંવેદનશીલ \\
\textbf{ઓછું ડિસ્ટોરશન} & ઇન્ટરમોડ્યુલેશન ડિસ્ટોર્શન ઘટાડે છે \\
\textbf{સરળ રિસીવર} & સરળ સર્કિટ ડિઝાઇન શક્ય \\
\end{longtable}
}

\end{solutionbox}
\begin{mnemonicbox}
``BPFDS'' - Bandwidth, Power, Fading, Distortion,
Simple

\end{mnemonicbox}
\subsection*{પ્રશ્ન 2(b) [4
ગુણ]}\label{q2b}

\textbf{ફેસ લોક લુપ ટેક્નીકથી FMનું જનરેશન સમજાવો.}

\begin{solutionbox}

\textbf{PLL દ્વારા FM જનરેશન:}

PLL (Phase-Locked Loop) VCO કંટ્રોલ ઇનપુટ પર મોડ્યુલેટિંગ સિગ્નલ લાગુ કરીને FM
સિગ્નલ્સ ઉત્પન્ન કરે છે.

\textbf{PLL FM મોડ્યુલેટર:}

\begin{center}
\textbf{Mermaid Diagram (Code)}
\begin{verbatim}
{Shaded}
{Highlighting}[]
graph LR
    A[Modulating Signal] {-{-}{} B[Summing Circuit]}
    E[Reference Oscillator] {-{-}{} F[Phase Detector]}
    F {-{-}{} G[Low Pass Filter]}
    G {-{-}{} B}
    B {-{-}{} H[VCO]}
    H {-{-}{} I[FM Output]}
    H {-{-}{} J[Feedback]}
    J {-{-}{} F}
{Highlighting}
{Shaded}
\end{verbatim}
\end{center}

\textbf{ઓપરેશન:}

\begin{itemize}
\tightlist
\item
  \textbf{રેફરન્સ ઓસીલેટર}: સ્થિર રેફરન્સ ફ્રીક્વન્સી પ્રદાન કરે છે
\item
  \textbf{ફેઝ ડિટેક્ટર}: રેફરન્સ અને ફીડબેક સિગ્નલોની તુલના કરે છે
\item
  \textbf{લો પાસ ફિલ્ટર}: ઉચ્ચ-ફ્રીકવન્સી ઘટકોને દૂર કરે છે
\item
  \textbf{VCO}: કંટ્રોલ વોલ્ટેજ સાથે બદલાતી આઉટપુટ ફ્રીક્વન્સી જનરેટ કરે છે
\item
  \textbf{મોડ્યુલેટિંગ સિગ્નલ}: FM આઉટપુટ ઉત્પન્ન કરવા માટે કંટ્રોલ વોલ્ટેજમાં ઉમેરાય છે
\end{itemize}

\end{solutionbox}
\begin{mnemonicbox}
``PROVE'' - Phase detector, Reference oscillator,
Output VCO, Voltage controlled

\end{mnemonicbox}
\subsection*{પ્રશ્ન 2(c) [7
ગુણ]}\label{q2c}

\textbf{AM માટે ટોટલ પાવરનું સમીકરણ તારવો. DSB અને SSB માટે પાવર સેવિંગના ટકાની
ગણતરી કરો.}

\begin{solutionbox}

\textbf{AM માં પાવર:}

AM વેવ ઇક્વેશન: \(e_{AM}(t) = A_c[1 + m\cos(2\pi f_m t)]\cos(2\pi f_c t)\)

\textbf{પાવર ડેરીવેશન:}

\begin{itemize}
\tightlist
\item
  \textbf{કુલ પાવર}: \(P_T = P_c\left(1 + \frac{m^2}{2}\right)\)
\item
  જ્યાં \(P_c = \frac{A_c^2}{2R}\) (કેરિયર પાવર) અને \(m\) મોડ્યુલેશન ઇન્ડેક્સ છે
\end{itemize}

\textbf{પાવર ડિસ્ટ્રિબ્યુશન:}

\begin{itemize}
\tightlist
\item
  \textbf{કેરિયર પાવર}: \(P_c = \frac{A_c^2}{2R}\)
\item
  \textbf{કુલ સાઇડબેન્ડ પાવર}: \(P_{SB} = \frac{m^2 P_c}{2}\)
\item
  \textbf{દરેક સાઇડબેન્ડ}: \(P_{LSB} = P_{USB} = \frac{m^2 P_c}{4}\)
\end{itemize}

\textbf{પાવર સેવિંગ્સ:}

\begin{itemize}
\tightlist
\item
  \textbf{DSB-SC માં}: કેરિયર પાવર નથી, એટલે સેવિંગ્સ =
  \(\frac{P_c}{P_T} \times 100\% = \frac{1}{1+\frac{m^2}{2}} \times 100\%\)

  \begin{itemize}
  \tightlist
  \item
    m=1 માટે, સેવિંગ્સ = 66.67\%
  \end{itemize}
\item
  \textbf{SSB માં}: કેરિયર અને એક સાઇડબેન્ડ નથી, એટલે સેવિંગ્સ =
  \(\frac{P_c + P_{SB}/2}{P_T} \times 100\%\)

  \begin{itemize}
  \tightlist
  \item
    m=1 માટે, સેવિંગ્સ = 83.33\%
  \end{itemize}
\end{itemize}

\end{solutionbox}
\begin{mnemonicbox}
``CEPTS'' - Carrier Eliminated Provides Tremendous
Savings

\end{mnemonicbox}
\subsection*{પ્રશ્ન 2(a) OR [3
ગુણ]}\label{q2a}

\textbf{AM વેવ માટે Time domain અને Frequency domain ડિસ્પ્લે દોરો અને સમજાવો.}

\begin{solutionbox}

\textbf{AM ના Time અને Frequency Domain:}

\textbf{આકૃતિ:}

\begin{verbatim}
Time Domain:
    
     +          +           +           +
     |          |           |           |
     |    ++    |     ++    |    ++     |
     |   /  {   |    /     |   /      |}
     |  /    {  |   /      |  /       |}
     | /      { |  /       | /        |}
     |/        {|/        |/        | |}
{-{-}{-}{-}{-}+{-}{-}{-}{-}{-}{-}{-}{-}{-}{-}+{-}{-}{-}{-}{-}{-}{-}{-}{-}{-}{-}+{-}{-}{-}{-}{-}{-}{-}{-}{-}{-}+{-}{-}{-}{-}{-}}
     |{        /|        /|        /| |}
     | {      / |       / |       /  |}
     |  {    /  |      /  |      /   |}
     |   {\_\_/   |   \_\_/   |   \_\_/    |}
     |          |           |           |
     +          +           +           +

Frequency Domain:
    
     |
     |          
     |     +           +           +
     |     |           |           |
     |     |           |           |
     |     |           |           |
     |     |           |           |
{-{-}{-}{-}{-}+{-}{-}{-}{-}{-}+{-}{-}{-}{-}{-}+{-}{-}{-}{-}{-}+{-}{-}{-}{-}{-}+{-}{-}{-}{-}{-}+{-}{-}{-}{-}{-}}
     |   f\_c{-f\_m     f\_c    f\_c+f\_m}
\end{verbatim}

\textbf{ટાઇમ ડોમેન:}

\begin{itemize}
\tightlist
\item
  સમય સાથે કેરિયરના એમ્પલિટ્યુડ વેરિએશન બતાવે છે
\item
  એન્વેલોપ મોડ્યુલેટિંગ સિગ્નલને અનુસરે છે
\item
  ઉપર અને નીચેના એન્વેલોપ = કેરિયર પીક \times (1\pmm)
\end{itemize}

\textbf{ફ્રિક્વન્સી ડોમેન:}

\begin{itemize}
\tightlist
\item
  ફ્રિક્વન્સી કોમ્પોનન્ટ્સ અને તેમના એમ્પ્લિટ્યુડ બતાવે છે
\item
  fc ફ્રિક્વન્સી પર Ac એમ્પ્લિટ્યુડ સાથે કેરિયર
\item
  fc\pmfm પર mAc/2 એમ્પ્લિટ્યુડ સાથે બે સાઇડબેન્ડસ
\item
  બેન્ડવિડ્થ = 2fm (મોડ્યુલેટિંગ ફ્રિક્વન્સીનો બમણો)
\end{itemize}

\end{solutionbox}
\begin{mnemonicbox}
``EBS'' - Envelope in time, Bandwidth in frequency,
Sidebands symmetric

\end{mnemonicbox}
\subsection*{પ્રશ્ન 2(b) OR [4
ગુણ]}\label{q2b}

\textbf{પ્રી-એમફાસીસ અને ડી એમફાસીસ સર્કીટ સમજાવો.}

\begin{solutionbox}

\textbf{પ્રી-એમફાસીસ અને ડી-એમફાસીસ:}

\textbf{સર્કિટ ડાયાગ્રામ્સ:}

\begin{verbatim}
Pre{-emphasis:                   De{-}emphasis:}
    
+{-{-}+     +{-}{-}+                 +{-}{-}+     +{-}{-}+}
|  |     |  |                 |  |     |  |
+{-{-}+     R  |                 +{-}{-}+     R  |}
Input    |  +{-{-}+{-}{-}+  Output   Input    |  +{-}{-}+{-}{-}+  Output}
o{-{-}{-}{-}{-}{-}{-}{-}+  |  |  o{-}{-}{-}{-}{-}{-}{-}{-}   o{-}{-}{-}{-}{-}{-}{-}{-}+  |  |  o{-}{-}{-}{-}{-}{-}{-}{-}}
            C  |                          C  |
            |  |                          |  |
            +{-{-}+                          +{-}{-}+}
              |                             |
            {-{-}{-}{-}{-}                         {-}{-}{-}{-}{-}}
             {-{-}{-}                           {-}{-}{-}}
              {-                             {-}}
\end{verbatim}

\textbf{હેતુ:}

\begin{itemize}
\tightlist
\item
  \textbf{પ્રી-એમફાસીસ}: ટ્રાન્સમીટર પર ઉચ્ચ-ફ્રીક્વન્સી ઘટકોને વધારે છે
\item
  \textbf{ડી-એમફાસીસ}: રિસીવર પર ઉચ્ચ-ફ્રીક્વન્સી ઘટકોને ઘટાડે છે
\end{itemize}

\textbf{ઓપરેશન:}

\begin{itemize}
\tightlist
\item
  \textbf{પ્રી-એમફાસીસ}: હાઇ-પાસ RC સર્કિટ (R સીરીઝ, C પેરેલલ)
\item
  \textbf{ડી-એમફાસીસ}: લો-પાસ RC સર્કિટ (R પેરેલલ, C સીરીઝ)
\item
ટાઇમ કોન્સ્ટન્ટ સરખા છે:

τ = RC = 75μs (સ્ટાન્ડર્ડ)

\end{itemize}

\textbf{લાભો:}

\begin{itemize}
\tightlist
\item
  FM માં ઉચ્ચ ફ્રીક્વન્સી માટે SNR સુધારે છે
\item
  ઉચ્ચ ફ્રીક્વન્સી પર વધુ નોઇઝ પાવરની ભરપાઈ કરે છે
\item
  રિસીવર પર મૂળ ફ્રીક્વન્સી પ્રતિસાદ પુનઃસ્થાપિત કરે છે
\end{itemize}

\end{solutionbox}
\begin{mnemonicbox}
``BETH'' - Boost (pre-emphasis), Emphasizes Treble,
Helps SNR

\end{mnemonicbox}
\subsection*{પ્રશ્ન 2(c) OR [7
ગુણ]}\label{q2c}

\textbf{AM, FM અને PMને સરખાવો.}

\begin{solutionbox}

\textbf{AM, FM અને PM ની તુલના:}

{\def\LTcaptype{none} % do not increment counter
\begin{longtable}[]{@{}
  >{\raggedright\arraybackslash}p{(\linewidth - 6\tabcolsep) * \real{0.3793}}
  >{\raggedright\arraybackslash}p{(\linewidth - 6\tabcolsep) * \real{0.2069}}
  >{\raggedright\arraybackslash}p{(\linewidth - 6\tabcolsep) * \real{0.2069}}
  >{\raggedright\arraybackslash}p{(\linewidth - 6\tabcolsep) * \real{0.2069}}@{}}
\toprule\noalign{}
\begin{minipage}[b]{\linewidth}\raggedright
પેરામીટર
\end{minipage} & \begin{minipage}[b]{\linewidth}\raggedright
AM
\end{minipage} & \begin{minipage}[b]{\linewidth}\raggedright
FM
\end{minipage} & \begin{minipage}[b]{\linewidth}\raggedright
PM
\end{minipage} \\
\midrule\noalign{}
\endhead
\bottomrule\noalign{}
\endlastfoot
\textbf{વ્યાખ્યા} & મેસેજ સિગ્નલ સાથે એમ્પ્લિટ્યુડ બદલાય છે & મેસેજ સિગ્નલ સાથે
ફ્રીક્વન્સી બદલાય છે & મેસેજ સિગ્નલ સાથે ફેઝ બદલાય છે \\
\textbf{ગાણિતિક અભિવ્યક્તિ} & \(A_c[1+m\cos(ω_mt)]\cos(ω_ct)\) &
\(A_c\cos[ω_ct+mf\sin(ω_mt)]\) & \(A_c\cos[ω_ct+mp\cos(ω_mt)]\) \\
\textbf{બેન્ડવિડ્થ} & 2fm (સાંકડી) & 2(Δf+fm) (વિશાળ) & 2(mp+1)fm (વિશાળ) \\
\textbf{પાવર દક્ષતા} & ઓછી (કેરિયરમાં માહિતી નથી) & ઉચ્ચ (સ્થિર એમ્પ્લિટ્યુડ) &
ઉચ્ચ (સ્થિર એમ્પ્લિટ્યુડ) \\
\textbf{નોઇઝ ઇમ્યુનિટી} & નબળી & ઉત્તમ & ઉત્તમ \\
\textbf{સર્કિટ જટિલતા} & સરળ & જટિલ & જટિલ \\
\textbf{એપ્લિકેશન્સ} & AM બ્રોડકાસ્ટિંગ, એરક્રાફ્ટ કોમ્યુનિકેશન & FM બ્રોડકાસ્ટિંગ, TV
સાઉન્ડ, મોબાઇલ રેડિયો & સેટેલાઇટ કોમ્યુનિકેશન, ટેલીમેટ્રી \\
\textbf{મોડ્યુલેશન ઇન્ડેક્સ} & m = Am/Ac (0 થી 1) & mf = Δf/fm (કોઈ મર્યાદા
નથી) & mp = Δφ/fm (કોઈ મર્યાદા નથી) \\
\end{longtable}
}

\end{solutionbox}
\begin{mnemonicbox}
``BANCP-MAP'' - Bandwidth, Amplitude, Noise,
Complexity, Power, Modulation, Applications, Parameters

\end{mnemonicbox}
\subsection*{પ્રશ્ન 3(a) [3
ગુણ]}\label{q3a}

\textbf{રેડીઓ રીસીવર ની કોઈ ચાર લાક્ષણીકતા ઓ વ્યાખ્યાઈત કરો.}

\begin{solutionbox}

\textbf{રેડિયો રિસીવર લક્ષણો:}

{\def\LTcaptype{none} % do not increment counter
\begin{longtable}[]{@{}
  >{\raggedright\arraybackslash}p{(\linewidth - 2\tabcolsep) * \real{0.5714}}
  >{\raggedright\arraybackslash}p{(\linewidth - 2\tabcolsep) * \real{0.4286}}@{}}
\toprule\noalign{}
\begin{minipage}[b]{\linewidth}\raggedright
લાક્ષણિકતા
\end{minipage} & \begin{minipage}[b]{\linewidth}\raggedright
વ્યાખ્યા
\end{minipage} \\
\midrule\noalign{}
\endhead
\bottomrule\noalign{}
\endlastfoot
\textbf{સેન્સિટિવિટી} & સ્વીકાર્ય આઉટપુટ માટે જરૂરી લઘુતમ સિગ્નલ શક્તિ \\
\textbf{સિલેક્ટિવિટી} & આજુબાજુના સિગ્નલથી ઇચ્છિત સિગ્નલને અલગ કરવાની ક્ષમતા \\
\textbf{ફિડેલિટી} & ડિસ્ટોર્શન વિના મૂળ સિગ્નલને પુનઃઉત્પન્ન કરવામાં ચોકસાઈ \\
\textbf{ઇમેજ રિજેક્શન} & ઇમેજ ફ્રીક્વન્સી ઇન્ટરફેરન્સને નકારવાની ક્ષમતા \\
\textbf{સિગ્નલ-ટુ-નોઇઝ રેશિયો} & ઇચ્છિત સિગ્નલ અને અનિચ્છનીય નોઇઝનો ગુણોત્તર \\
\textbf{સ્ટેબિલિટી} & ટ્યુન કરેલી ફ્રીક્વન્સીને ડ્રિફ્ટ કર્યા વિના જાળવી રાખવાની
ક્ષમતા \\
\end{longtable}
}

\end{solutionbox}
\begin{mnemonicbox}
``SFIS-SS'' - Sensitivity, Fidelity, Image
rejection, Selectivity, SNR, Stability

\end{mnemonicbox}
\subsection*{પ્રશ્ન 3(b) [4
ગુણ]}\label{q3b}

\textbf{FM રીસીવરનો બ્લોક ડાયગ્રામ દોરો. FM રીસીવરમા લીમીટરનું કાર્ય શું છે?}

\begin{solutionbox}

\textbf{FM રિસીવર બ્લોક ડાયાગ્રામ:}

\begin{center}
\textbf{Mermaid Diagram (Code)}
\begin{verbatim}
{Shaded}
{Highlighting}[]
graph LR
    A[Antenna] {-{-}{} B[RF Amplifier]}
    B {-{-}{} C[Mixer]}
    D[Local Oscillator] {-{-}{} C}
    C {-{-}{} E[IF Amplifier]}
    E {-{-}{} F[Limiter]}
    F {-{-}{} G[FM Detector]}
    G {-{-}{} H[Audio Amplifier]}
    H {-{-}{} I[Speaker]}
{Highlighting}
{Shaded}
\end{verbatim}
\end{center}

\textbf{FM રિસીવરમાં લિમિટરનો ઉપયોગ:}

\begin{itemize}
\tightlist
\item
  \textbf{મુખ્ય કાર્ય}: એમ્પ્લિટ્યુડ વેરિએશન/નોઇઝ દૂર કરે છે
\item
  \textbf{ઓપરેશન}: સિગ્નલને ક્લિપ કરીને સ્થિર એમ્પ્લિટ્યુડ પ્રદાન કરે છે
\item
  \textbf{લાભો}:

  \begin{itemize}
  \tightlist
  \item
    AM ઇન્ટરફેરન્સ દૂર કરે છે
  \item
    SNR સુધારે છે
  \item
    યોગ્ય FM ડિટેક્શન સુનિશ્ચિત કરે છે
  \item
    ખોટા ફ્રીક્વન્સી ડિમોડ્યુલેશનને રોકે છે
  \end{itemize}
\item
  \textbf{સ્થાન}: IF એમ્પ્લિફાયર અને FM ડિટેક્ટર વચ્ચે મૂકવામાં આવે છે
\end{itemize}

\end{solutionbox}
\begin{mnemonicbox}
``CARE'' - Clips Amplitude, Removes noise, Ensures
constant signal

\end{mnemonicbox}
\subsection*{પ્રશ્ન 3(c) [7
ગુણ]}\label{q3c}

\textbf{સુપર હેટેરોડાઈન રીસીવરનો બ્લોક ડાયગ્રામ દોરો અને સમજાવો.}

\begin{solutionbox}

\textbf{સુપર હેટેરોડાઈન રિસીવર:}

\begin{center}
\textbf{Mermaid Diagram (Code)}
\begin{verbatim}
{Shaded}
{Highlighting}[]
graph LR
    A[Antenna] {-{-}{} B[RF Amplifier]}
    B {-{-}{} C[Mixer]}
    D[Local Oscillator] {-{-}{} C}
    C {-{-}{} E[IF Amplifier]}
    E {-{-}{} F[Detector]}
    F {-{-}{} G[Audio Amplifier]}
    G {-{-}{} H[Speaker]}
    F {-{-}{} I[AGC]}
    I {-{-}{} B}
    I {-{-}{} E}
{Highlighting}
{Shaded}
\end{verbatim}
\end{center}

\textbf{દરેક બ્લોકનું કાર્ય:}

\begin{itemize}
\tightlist
\item
  \textbf{એન્ટેના}: ઇલેક્ટ્રોમેગ્નેટિક તરંગોમાંથી RF સિગ્નલ્સ કેપ્ચર કરે છે
\item
  \textbf{RF એમ્પ્લિફાયર}: નબળા સિગ્નલ્સને એમ્પ્લિફાય કરે છે, સિલેક્ટિવિટી પ્રદાન કરે
  છે
\item
  \textbf{લોકલ ઓસિલેટર}: આવતા RF સાથે મિક્સ કરવા માટે સિગ્નલ ઉત્પન્ન કરે છે
\item
  \textbf{મિક્સર}: RF ને લોકલ ઓસિલેટર સાથે હેટરોડાઇનિંગ કરીને IF ઉત્પન્ન કરે છે
\item
  \textbf{IF એમ્પ્લિફાયર}: ફિક્સ્ડ ફ્રીક્વન્સી પર મુખ્ય એમ્પ્લિફિકેશન અને સિલેક્ટિવિટી
\item
  \textbf{ડિટેક્ટર}: મોડ્યુલેટેડ IF સિગ્નલમાંથી ઓડિયો એક્સટ્રેક્ટ કરે છે
\item
  \textbf{ઓડિયો એમ્પ્લિફાયર}: સ્પીકર ચલાવવા માટે ઓડિયો સિગ્નલને એમ્પ્લિફાય કરે છે
\item
  \textbf{AGC (ઓટોમેટિક ગેઇન કંટ્રોલ)}: સતત આઉટપુટ લેવલ જાળવે છે
\item
  \textbf{સ્પીકર}: ઇલેક્ટ્રિકલ સિગ્નલને સાઉન્ડમાં રૂપાંતરિત કરે છે
\end{itemize}

\textbf{સુપર હેટેરોડાઇન સિદ્ધાંત:}

\begin{itemize}
\tightlist
\item
  ઉચ્ચ-ફ્રીક્વન્સી RF ને વધુ સારા એમ્પ્લિફિકેશન માટે ફિક્સ્ડ IF માં રૂપાંતરિત કરે છે
\item
  IF = \textbar RF \pm LO\textbar{} (સામાન્ય રીતે AM માટે 455 kHz, FM માટે
  10.7 MHz)
\end{itemize}

\end{solutionbox}
\begin{mnemonicbox}
``ARLMIDAS'' - Antenna Receives, Local Mixes, IF
Delivers, Audio Sounds

\end{mnemonicbox}
\subsection*{પ્રશ્ન 3(a) OR [3
ગુણ]}\label{q3a}

\textbf{એનવેલોપ ડીટેક્ટરનો બ્લોક ડાયગ્રામ દોરો અને સમજાવો.}

\begin{solutionbox}

\textbf{એનવેલોપ ડિટેક્ટર:}

\textbf{સર્કિટ ડાયાગ્રામ:}

\begin{verbatim}
          D
    +{-{-}{-}|{-}{-}{-}+{-}{-}{-}+}
    |         |   |
AM  |         |   |     Audio
Input o       C   R    Output
    |         |   |      o
    |         |   |      |
    +{-{-}{-}{-}{-}{-}{-}{-}{-}+{-}{-}{-}+{-}{-}{-}{-}{-}{-}+}
              |
             {-{-}{-}}
              {-}
\end{verbatim}

\textbf{કોમ્પોનન્ટ ફંક્શન્સ:}

\begin{itemize}
\tightlist
\item
  \textbf{ડાયોડ (D)}: AM સિગ્નલને રેક્ટિફાય કરે છે (માત્ર પોઝિટિવ હાફ-સાયકલ્સની
  મંજૂરી આપે છે)
\item
  \textbf{કેપેસિટર (C)}: ઇનપુટના પીક સુધી ચાર્જ થાય છે, કેરિયર ફ્રીક્વન્સીને ફિલ્ટર
  કરે છે
\item
  \textbf{રેસિસ્ટર (R)}: કેપેસિટરને ડિસ્ચાર્જ કરે છે, મોડ્યુલેટિંગ સિગ્નલ એનવેલોપને અનુસરે
  છે
\end{itemize}

\textbf{ઓપરેશન:}

\begin{enumerate}
\tightlist
\item
  ડાયોડ પોઝિટિવ હાફ-સાયકલ્સ દરમિયાન કન્ડક્ટ કરે છે
\item
  કેપેસિટર પીક વોલ્ટેજ સુધી ચાર્જ થાય છે
\item
  નેગેટિવ હાફ-સાયકલ્સ દરમિયાન, ડાયોડ બ્લોક કરે છે
\item
  કેપેસિટર રેસિસ્ટર દ્વારા ડિસ્ચાર્જ થાય છે
\item
  RC ટાઇમ કોન્સ્ટન્ટ એનવેલોપ વેરિએશન્સને અનુસરે છે
\end{enumerate}

\textbf{RC સિલેક્શન ક્રાઇટેરિયા}: \(\frac{1}{f_c} << RC << \frac{1}{f_m}\)

\end{solutionbox}
\begin{mnemonicbox}
``DRIVER'' - Diode Rectifies, RC Values Extract
Envelope, Restores audio

\end{mnemonicbox}
\subsection*{પ્રશ્ન 3(b) OR [4
ગુણ]}\label{q3b}

\textbf{IF શું છે? તેનો અગત્યતા સમજાવો.}

\begin{solutionbox}

\textbf{ઇન્ટરમીડિએટ ફ્રીક્વન્સી (IF):}

\textbf{વ્યાખ્યા:} IF એ એક ફિક્સ્ડ ફ્રીક્વન્સી છે જેમાં આવતા RF સિગ્નલ્સ સુપરહેટેરોડાઇન
રિસીવર્સમાં રૂપાંતરિત થાય છે.

\textbf{IF ની અગત્યતા:}

{\def\LTcaptype{none} % do not increment counter
\begin{longtable}[]{@{}
  >{\raggedright\arraybackslash}p{(\linewidth - 2\tabcolsep) * \real{0.4000}}
  >{\raggedright\arraybackslash}p{(\linewidth - 2\tabcolsep) * \real{0.6000}}@{}}
\toprule\noalign{}
\begin{minipage}[b]{\linewidth}\raggedright
પાસું
\end{minipage} & \begin{minipage}[b]{\linewidth}\raggedright
અગત્યતા
\end{minipage} \\
\midrule\noalign{}
\endhead
\bottomrule\noalign{}
\endlastfoot
\textbf{ફિક્સ્ડ ફ્રીક્વન્સી} & એક ફ્રીક્વન્સી પર ઑપ્ટિમાઇઝ્ડ એમ્પ્લિફિકેશનની મંજૂરી આપે
છે \\
\textbf{સુધારેલી સિલેક્ટિવિટી} & ફિક્સ્ડ-ટ્યૂન ફિલ્ટર્સ બેટર એડજેસન્ટ ચેનલ રિજેક્શન
પ્રદાન કરે છે \\
\textbf{સ્થિર ગેઇન} & સમગ્ર ટ્યુનિંગ રેન્જમાં સાતત્યપૂર્ણ એમ્પ્લિફિકેશન \\
\textbf{ઇમેજ રિજેક્શન} & ઇમેજ ફ્રીક્વન્સી ઇન્ટરફેરન્સને અસ્વીકાર કરવામાં મદદ કરે છે \\
\textbf{સરળ ટ્યુનિંગ} & વિવિધ સ્ટેશનો માટે માત્ર લોકલ ઓસિલેટરને ટ્યુન કરવાની જરૂર
છે \\
\textbf{બેટર AGC} & ફિક્સ્ડ ફ્રીક્વન્સી પર વધુ અસરકારક ગેઇન કંટ્રોલ \\
\end{longtable}
}

\textbf{સામાન્ય IF વેલ્યુઝ:}

\begin{itemize}
\tightlist
\item
  AM રિસીવર્સ: 455 kHz
\item
  FM રિસીવર્સ: 10.7 MHz
\item
  ટેલિવિઝન: 45 MHz
\end{itemize}

\end{solutionbox}
\begin{mnemonicbox}
``FIGS-ST'' - Fixed frequency, Improved selectivity,
Gain stability, Simplified tuning

\end{mnemonicbox}
\subsection*{પ્રશ્ન 3(c) OR [7
ગુણ]}\label{q3c}

\textbf{FM detection માટેની ફેસ ડીસક્રીમીનેટર સર્કિટ સમજાવો.}

\begin{solutionbox}

\textbf{FM ડિટેક્શન માટે ફેઝ ડિસ્ક્રિમિનેટર:}

\textbf{સર્કિટ ડાયાગ્રામ:}

\begin{verbatim}
                 +{-{-}{-}{-}+}
                 |    |
     +{-{-}{-}{-}{-}{-}{-}{-}{-}{-}{-}|  T1|{-}{-}{-}{-}{-}{-}{-}+}
     |           |    |       |
     |           +{-{-}{-}{-}+       |}
     |                        |
     |                        |
     |                        |
     |   D1                   |   D2
FM   o{-{-}|{-}{-}{-}+            +{-}{-}{-}|{-}{-}+}
Input|        |           |       |
     |        |  +{-{-}{-}{-}+   |       |}
     |        +{-{-}|    |{-}{-}{-}+       |}
     |           |  T2|           |
     +{-{-}{-}{-}{-}{-}{-}{-}{-}{-}{-}| CT |{-}{-}{-}{-}{-}{-}{-}{-}{-}{-}{-}|}
                 |    |           |
                 +{-{-}{-}{-}+           |}
                    |             |
                    |     C1      |     C2
                    o{-{-}{-}{-}||{-}{-}{-}{-}{-}{-}{-}o{-}{-}{-}{-}{-}||{-}{-}{-}{-}o Audio }
                    |             |          Output
                   {-{-}{-}           {-}{-}{-}}
                    {-             {-}}
\end{verbatim}

\textbf{ઓપરેશન:}

\begin{enumerate}
\tightlist
\item
  \textbf{સેન્ટર-ટેપ્ડ ટ્રાન્સફોર્મર (T2)} 180^\circ ફેઝ ડિફરન્સ બનાવે છે
\item
  \textbf{પ્રાઇમરી ટ્રાન્સફોર્મર (T1)} રેફરન્સ ફેઝ સેટ કરે છે
\item
  \textbf{ડાયોડ D1 અને D2} ફેઝ કમ્પેરેટર બનાવે છે
\item
  \textbf{જ્યારે કેરિયર સેન્ટર ફ્રીક્વન્સી પર હોય:}

  \begin{itemize}
  \tightlist
  \item
    બંને ડાયોડ દ્વારા સરખા કરંટ
  \item
    C1 અને C2 પર સરખા વોલ્ટેજ
  \item
    નેટ આઉટપુટ શૂન્ય છે
  \end{itemize}
\item
  \textbf{જ્યારે ફ્રીક્વન્સી વિચલિત થાય છે:}

  \begin{itemize}
  \tightlist
  \item
    ફેઝ બદલાય છે
  \item
    અસમાન ડાયોડ કરંટ
  \item
    આઉટપુટ વોલ્ટેજ ફ્રીક્વન્સી વિચલન સાથે પ્રમાણસર
  \end{itemize}
\end{enumerate}

\textbf{ફાયદાઓ:}

\begin{itemize}
\tightlist
\item
  સારી રેખીયતા
\item
  ઘટાડેલું ડિસ્ટોર્શન
\item
  સ્લોપ ડિટેક્ટર કરતાં બેહતર નોઇઝ પરફોર્મન્સ
\end{itemize}

\end{solutionbox}
\begin{mnemonicbox}
``PERFECT'' - Phase Ensures Rectification For
Extracting Carrier Transitions

\end{mnemonicbox}
\subsection*{પ્રશ્ન 4(a) [3
ગુણ]}\label{q4a}

\textbf{ક્વોન્ટઆઈજાશન રીત અને તેની ઉપયોગીતા સમજાવો.}

\begin{solutionbox}

\textbf{ક્વોન્ટિઝેશન પ્રોસેસ:}

\textbf{વ્યાખ્યા:} ક્વોન્ટિઝેશન એ સતત એનાલોગ મૂલ્યોને ડિસ્ક્રીટ ડિજિટલ લેવલ્સમાં મેપિંગ
કરવાની પ્રક્રિયા છે.

\textbf{પ્રક્રિયા:}

\begin{enumerate}
\tightlist
\item
  સેમ્પલિંગ સતત-સમય સિગ્નલને ડિસ્ક્રીટ-ટાઇમમાં રૂપાંતરિત કરે છે
\item
  એમ્પ્લિટ્યુડની રેન્જ ફિનાઇટ સંખ્યાના લેવલ્સમાં વિભાજિત થયેલી છે
\item
  દરેક સેમ્પલને નજીકના ક્વોન્ટિઝેશન લેવલમાં સોંપવામાં આવે છે
\item
  ઓરિજિનલ અને ક્વોન્ટાઇઝ્ડ વેલ્યુ વચ્ચેનો તફાવત ક્વોન્ટિઝેશન એરર છે
\end{enumerate}

\textbf{ક્વોન્ટિઝેશનની આવશ્યકતા:}

{\def\LTcaptype{none} % do not increment counter
\begin{longtable}[]{@{}ll@{}}
\toprule\noalign{}
આવશ્યકતા & સમજૂતી \\
\midrule\noalign{}
\endhead
\bottomrule\noalign{}
\endlastfoot
\textbf{ડિજિટલ પ્રોસેસિંગ} & ડિજિટલ સ્ટોરેજ અને મેનિપ્યુલેશન સક્ષમ કરે છે \\
\textbf{એરર કંટ્રોલ} & એરર ડિટેક્શન અને કરેક્શનની મંજૂરી આપે છે \\
\textbf{નોઇઝ ઇમ્યુનિટી} & ડિજિટલ સિગ્નલ્સ નોઇઝ માટે વધુ પ્રતિરોધક છે \\
\textbf{સ્ટોરેજ એફિશિયન્સી} & એનાલોગ વેલ્યુઝ સંગ્રહિત કરવા કરતાં વધુ કાર્યક્ષમ \\
\textbf{ટ્રાન્સમિશન} & ડિજિટલ સિગ્નલ્સ એરર વિના પુનઃઉત્પન્ન કરી શકાય છે \\
\end{longtable}
}

\end{solutionbox}
\begin{mnemonicbox}
``DENSE'' - Digital conversion, Error control, Noise
immunity, Storage, Efficient transmission

\end{mnemonicbox}
\subsection*{પ્રશ્ન 4(b) [4
ગુણ]}\label{q4b}

\textbf{ડેલ્ટા અને એડપટીવ ડેલ્ટા મોડયુલેશનનો તફાવત જણાવો.}

\begin{solutionbox}

\textbf{DM અને ADM વચ્ચે તફાવત:}

{\def\LTcaptype{none} % do not increment counter
\begin{longtable}[]{@{}
  >{\raggedright\arraybackslash}p{(\linewidth - 4\tabcolsep) * \real{0.1719}}
  >{\raggedright\arraybackslash}p{(\linewidth - 4\tabcolsep) * \real{0.3281}}
  >{\raggedright\arraybackslash}p{(\linewidth - 4\tabcolsep) * \real{0.5000}}@{}}
\toprule\noalign{}
\begin{minipage}[b]{\linewidth}\raggedright
પેરામીટર
\end{minipage} & \begin{minipage}[b]{\linewidth}\raggedright
ડેલ્ટા મોડ્યુલેશન (DM)
\end{minipage} & \begin{minipage}[b]{\linewidth}\raggedright
એડેપ્ટિવ ડેલ્ટા મોડ્યુલેશન (ADM)
\end{minipage} \\
\midrule\noalign{}
\endhead
\bottomrule\noalign{}
\endlastfoot
\textbf{સ્ટેપ સાઇઝ} & ફિક્સ્ડ & વેરિએબલ (સિગ્નલને અનુકૂળ) \\
\textbf{સ્લોપ ઓવરલોડ} & સ્ટીપ સિગ્નલ્સ પર સામાન્ય & એડેપ્ટિવ સ્ટેપ સાથે ઘટાડેલું \\
\textbf{ગ્રેન્યુલર નોઇઝ} & નાના સિગ્નલ્સ માટે ઉચ્ચ & નાના સ્ટેપ્સ સાથે ઘટાડેલું \\
\textbf{સિગ્નલ ટ્રેકિંગ} & ઝડપથી બદલાતા સિગ્નલ્સ માટે ધીમું & સિગ્નલ વેરિએશન્સનું બેહતર
ટ્રેકિંગ \\
\textbf{જટિલતા} & સરળ & મધ્યમ \\
\textbf{બિટ રેટ} & સારી ક્વોલિટી માટે ઉચ્ચ & સમાન ક્વોલિટી માટે નીચો \\
\textbf{એરર પરફોર્મન્સ} & વધુ સંવેદનશીલ & વધુ મજબૂત \\
\end{longtable}
}

\textbf{આકૃતિ:}

\begin{verbatim}
DM:                              ADM:

   \^{                                \^{}}
   |                                |
   |    Original                    |    Original
   |      /{                        |      /}
   |     /  {                       |     /  }
   |    /    {                      |    /    }
   |   /      {                     |   /      }
   |  /  \_\_\_\_  {                    |  /        }
   | /\_\_|    |\_\_{\_\_\_                | /\_         \_\_\_}
   |/  |    |  |   {                |/         /    }
{-{-}{-}+{-}{-}{-}+{-}{-}{-}{-}+{-}{-}+{-}{-}{-}{-}+{-}{-}           {-}+{-}{-}{-}{-}+{-}{-}{-}{-}{-}+{-}{-}{-}{-}{-}{-}+{-}{-}}
   |   |    |  |    |               |     |     |      |
   
Slope Overload              Better Signal Tracking
\end{verbatim}

\end{solutionbox}
\begin{mnemonicbox}
``SAVAGES'' - Step size, Adaptable, Variable
tracking, Avoids overload, Granular noise reduction, Error performance,
Signal fidelity

\end{mnemonicbox}
\subsection*{પ્રશ્ન 4(c) [7
ગુણ]}\label{q4c}

\textbf{PCM system નો બ્લોક ડાયગ્રામ દોરો અને સમજાવો.}

\begin{solutionbox}

\textbf{PCM સિસ્ટમ બ્લોક ડાયાગ્રામ:}

\begin{center}
\textbf{Mermaid Diagram (Code)}
\begin{verbatim}
{Shaded}
{Highlighting}[]
graph LR
    subgraph "PCM Transmitter"
    A[Input Signal] {-{-}{} B[Anti{-}aliasing Filter]}
    B {-{-}{} C[Sample \& Hold]}
    C {-{-}{} D[Quantizer]}
    D {-{-}{} E[Encoder]}
    E {-{-}{} F[Parallel to Serial]}
    end

    F {-{-}{} G[Transmission Channel]}
    
    subgraph "PCM Receiver"
    G {-{-}{} H[Serial to Parallel]}
    H {-{-}{} I[Decoder]}
    I {-{-}{} J[Reconstruction Filter]}
    J {-{-}{} K[Output Signal]}
    end
{Highlighting}
{Shaded}
\end{verbatim}
\end{center}

\textbf{PCM ટ્રાન્સમીટર:}

\begin{itemize}
\tightlist
\item
  \textbf{એન્ટી-એલિયાસિંગ ફિલ્ટર}: ન્યુક્વિસ્ટ ક્રાઇટેરિયનને સંતોષવા માટે ઇનપુટ સિગ્નલ
  બેન્ડવિડ્થને મર્યાદિત કરે છે
\item
  \textbf{સેમ્પલ \& હોલ્ડ}: સતત સિગ્નલને ડિસ્ક્રીટ-ટાઇમ સેમ્પલ્સમાં કન્વર્ટ કરે છે
\item
  \textbf{ક્વોન્ટાઇઝર}: સેમ્પલ એમ્પ્લિટ્યુડને નજીકના ડિસ્ક્રીટ લેવલ્સમાં એપ્રોક્સિમેટ કરે
  છે
\item
  \textbf{એન્કોડર}: ક્વોન્ટાઇઝ્ડ લેવલ્સને બાઇનરી કોડમાં કન્વર્ટ કરે છે
\item
  \textbf{પેરેલલ-ટુ-સીરિયલ}: ટ્રાન્સમિશન માટે પેરેલલ બિટ્સને સીરિયલમાં કન્વર્ટ કરે છે
\end{itemize}

\textbf{PCM રિસીવર:}

\begin{itemize}
\tightlist
\item
  \textbf{સીરિયલ-ટુ-પેરેલલ}: સીરિયલ ડેટાને પાછા પેરેલલ ફોર્મમાં કન્વર્ટ કરે છે
\item
  \textbf{ડિકોડર}: બાઇનરી કોડને પાછા એમ્પ્લિટ્યુડ લેવલ્સમાં કન્વર્ટ કરે છે
\item
  \textbf{રિકન્સ્ટ્રક્શન ફિલ્ટર}: એનાલોગ સિગ્નલને પુનઃપ્રાપ્ત કરવા માટે સ્ટેપ્ડ આઉટપુટને
  સ્મૂધ કરે છે
\end{itemize}

\textbf{PCM પેરામીટર્સ:}

\begin{itemize}
\tightlist
\item
  \textbf{સેમ્પલિંગ રેટ}: fs \textgreater{} 2fm (ન્યુક્વિસ્ટ રેટ)
\item
  \textbf{ક્વોન્ટિઝેશન લેવલ્સ}: L = 2\^{}n (n = બિટ્સની સંખ્યા)
\item
  \textbf{રિઝોલ્યુશન}: સૌથી નાનો અલગ ફેરફાર = Vmax/L
\item
  \textbf{બિટ રેટ}: R = n \times fs bits/second
\end{itemize}

\end{solutionbox}
\begin{mnemonicbox}
``SAFE-PETS'' - Sample, Amplify, Filter, Encode,
Pulse train, Extract, Transform, Smooth

\end{mnemonicbox}
\subsection*{પ્રશ્ન 4(a) OR [3
ગુણ]}\label{q4a}

\textbf{ક્વોન્ટઆઈજાશનની વ્યાખ્યા આપો. નોન યુનેફોર્મ ક્વોન્ટઆઈજાશન ટૂંકમાં સમજાવો.}

\begin{solutionbox}

\textbf{ક્વોન્ટિઝેશન વ્યાખ્યા:} ક્વોન્ટિઝેશન એ એનાલોગ-ટુ-ડિજિટલ કન્વર્ઝનમાં સતત
એમ્પ્લિટ્યુડ વેલ્યુને ડિસ્ક્રીટ લેવલ્સના ફિનાઇટ સેટમાં રૂપાંતર કરવાની પ્રક્રિયા છે.

\textbf{નોન-યુનિફોર્મ ક્વોન્ટિઝેશન:}

\textbf{આકૃતિ:}

\begin{verbatim}
     \^{}
     |                     {-{-}{-}{-}}
     |                 {-{-}{-}{-}}
     |             {-{-}{-}{-}}
Lvls |         {-{-}{-}{-}}
     |     {-{-}{-}{-}}
     |  {-{-}{-}}
     | {-}
     +{-{-}{-}{-}{-}{-}{-}{-}{-}{-}{-}{-}{-}{-}{-}{-}{-}{-}{-}{-}{-}{-}{-}{-}{-}{-}{-}}
                Input Signal
\end{verbatim}

\textbf{લક્ષણો:}

\begin{itemize}
\tightlist
\item
  એમ્પ્લિટ્યુડની રેન્જમાં અસમાન સ્ટેપ સાઇઝ
\item
  નીચા એમ્પ્લિટ્યુડ માટે નાના સ્ટેપ્સ, ઉચ્ચ માટે મોટા સ્ટેપ્સ
\item
  માનવ ધારણા (લોગરિધમિક રિસ્પોન્સ) સાથે વધુ સારી રીતે મેળ ખાય છે
\item
  બિટ રેટ વધાર્યા વિના નાના સિગ્નલ્સ માટે SNR સુધારે છે
\end{itemize}

\textbf{અમલીકરણ પદ્ધતિઓ:}

\begin{itemize}
\tightlist
\item
  \textbf{કોમ્પેન્ડિંગ}: ટ્રાન્સમીટર પર કમ્પ્રેસિંગ, રિસીવર પર એક્સપેન્ડિંગ
\item
  \textbf{લોગેરિધમિક કોડિંગ}: μ-law (ઉત્તર અમેરિકા) અને A-law (યુરોપ)
\item
  \textbf{એડેપ્ટિવ ક્વોન્ટિઝેશન}: સિગ્નલ સ્ટેટિસ્ટિક્સના આધારે લેવલ્સને એડજસ્ટ કરે છે
\end{itemize}

\end{solutionbox}
\begin{mnemonicbox}
``CLASP'' - Compressed Levels, Adaptive Steps, Small
steps for small signals, Perceptual matching

\end{mnemonicbox}
\subsection*{પ્રશ્ન 4(b) OR [4
ગુણ]}\label{q4b}

\textbf{એડપટીવ ડેલ્ટા મોડયુલેશન તેની એપ્લિકેસન સાથે સમજાવો.}

\begin{solutionbox}

\textbf{એડેપ્ટિવ ડેલ્ટા મોડ્યુલેશન (ADM):}

\textbf{આકૃતિ:}

\begin{center}
\textbf{Mermaid Diagram (Code)}
\begin{verbatim}
{Shaded}
{Highlighting}[]
graph LR
    A[Input Signal] {-{-}{} B[Comparator]}
    B {-{-}{} C[1{-}bit Quantizer]}
    C {-{-}{} D[Transmission Channel]}
    D {-{-}{} E[Step Size Control]}
    E {-{-}{} F[Integrator]}
    F {-{-}Feedback{-}{-}{} B}
    F {-{-}{} G[Output Signal]}
    C {-{-}Controls{-}{-}{} E}
{Highlighting}
{Shaded}
\end{verbatim}
\end{center}

\textbf{ઓપરેશન:}

\begin{itemize}
\tightlist
\item
  ઇનપુટ સિગ્નલ સ્લોપના આધારે સ્ટેપ સાઇઝને અડજસ્ટ કરે છે
\item
  ઝડપી ફેરફારો માટે સ્ટેપ સાઇઝ વધારે છે (સ્લોપ ઓવરલોડને રોકે છે)
\item
  ધીમા ફેરફારો માટે સ્ટેપ સાઇઝ ઘટાડે છે (ગ્રેન્યુલર નોઇઝ ઘટાડે છે)
\item
  સ્લોપ ચેન્જિસ નક્કી કરવા માટે અગાઉના બિટ્સ પેટર્નનો ઉપયોગ કરે છે
\end{itemize}

\textbf{ફાયદાઓ:}

\begin{itemize}
\tightlist
\item
  DM કરતાં બેહતર સિગ્નલ ટ્રેકિંગ
\item
  સમાન ક્વોલિટી માટે ઓછો બિટ રેટ
\item
  ઘટાડેલો સ્લોપ ઓવરલોડ અને ગ્રેન્યુલર નોઇઝ
\item
  વિશાળ ડાયનેમિક રેન્જ
\end{itemize}

\textbf{એપ્લિકેશન્સ:}

\begin{itemize}
\tightlist
\item
  સ્પીચ અને ઓડિયો કોમ્પ્રેશન
\item
  વોઇસ-ગ્રેડ કોમ્યુનિકેશન ચેનલ્સ
\item
  ડિજિટલ ટેલિફોની સિસ્ટમ્સ
\item
  વિડિયો સિગ્નલ એન્કોડિંગ
\item
  ટેલિમેટ્રી સિસ્ટમ્સ
\end{itemize}

\end{solutionbox}
\begin{mnemonicbox}
``ADAPT'' - Automatically Decides Appropriate Pulse
Transitions

\end{mnemonicbox}
\subsection*{પ્રશ્ન 4(c) OR [7
ગુણ]}\label{q4c}

\textbf{સેમ્પલીંગ શું છે? સેમ્પલીંગના પ્રકારોને ટુંકમાં સમજાવો.}

\begin{solutionbox}

\textbf{સેમ્પલિંગ વ્યાખ્યા:} સેમ્પલિંગ એ સતત-ટાઇમ સિગ્નલને નિયમિત અંતરાલે માપ લઈને
ડિસ્ક્રીટ-ટાઇમ સિગ્નલમાં રૂપાંતરિત કરવાની પ્રક્રિયા છે.

\textbf{સેમ્પલિંગના પ્રકારો:}

{\def\LTcaptype{none} % do not increment counter
\begin{longtable}[]{@{}
  >{\raggedright\arraybackslash}p{(\linewidth - 4\tabcolsep) * \real{0.2143}}
  >{\raggedright\arraybackslash}p{(\linewidth - 4\tabcolsep) * \real{0.4643}}
  >{\raggedright\arraybackslash}p{(\linewidth - 4\tabcolsep) * \real{0.3214}}@{}}
\toprule\noalign{}
\begin{minipage}[b]{\linewidth}\raggedright
પ્રકાર
\end{minipage} & \begin{minipage}[b]{\linewidth}\raggedright
વર્ણન
\end{minipage} & \begin{minipage}[b]{\linewidth}\raggedright
આકૃતિ
\end{minipage} \\
\midrule\noalign{}
\endhead
\bottomrule\noalign{}
\endlastfoot
\textbf{આદર્શ સેમ્પલિંગ} & અત્યંત નાના સમયગાળાના તાત્કાલિક સેમ્પલ્સ & સેમ્પલિંગ
ક્ષણોમાં ઇમ્પલ્સીસ \\
\textbf{નેચરલ સેમ્પલિંગ} & સેમ્પલ્સની પહોળાઈ મર્યાદિત છે, એમ્પ્લિટ્યુડ ઇનપુટને અનુસરે છે &
સેમ્પલિંગ અવધિ દરમિયાન મૂળ સિગ્નલ દૃશ્યમાન \\
\textbf{ફ્લેટ-ટોપ સેમ્પલિંગ} & સેમ્પલિંગ અંતરાલ દરમિયાન સેમ્પલ્સ સતત એમ્પ્લિટ્યુડ ધરાવે છે
& સ્ટેપ જેવું દેખાવ, સેમ્પલ-એન્ડ-હોલ્ડમાં વપરાય છે \\
\end{longtable}
}

\textbf{આકૃતિઓ:}

\begin{verbatim}
Ideal Sampling:           Natural Sampling:         Flat{-top Sampling:}

   \^{                         \^{}                        \^{}}
   |                         |                        |
   | |   |   |   |           |   \_   \_   \_            |  \_\_\_   \_\_\_   \_\_\_ 
   | |   |   |   |           |  / { /  /            | |   | |   | |   |}
   | |   |   |   |           | /   |   |   {          | |   | |   | |   |}
{-{-}{-}+{-}{-}{-}+{-}{-}{-}+{-}{-}{-}+{-}{-}{-}{-}     {-}{-}+{-}{-}{-}{-}{-}+{-}{-}{-}+{-}{-}{-}{-}{-}{-}{-}     {-}{-}+{-}{-}{-}{-}{-}+{-}{-}{-}{-}{-}+{-}{-}{-}{-}{-}}
   |                         |                         |
\end{verbatim}

\textbf{સેમ્પલિંગ પેરામીટર્સ:}

\begin{itemize}
\tightlist
\item
  \textbf{સેમ્પલિંગ પીરિયડ (Ts)}: સળંગ સેમ્પલ્સ વચ્ચેનો સમય
\item
  \textbf{સેમ્પલિંગ ફ્રીક્વન્સી (fs)}: પ્રતિ સેકન્ડ સેમ્પલ્સની સંખ્યા (fs = 1/Ts)
\item
  \textbf{ન્યુક્વિસ્ટ રેટ}: ન્યૂનતમ સેમ્પલિંગ રેટ (fs \textgreater{} 2fm) એલિયાસિંગ
  ટાળવા માટે
\end{itemize}

\end{solutionbox}
\begin{mnemonicbox}
``INFS'' - Ideal (impulses), Natural (follows
signal), Flat-top (constant), Sufficient rate

\end{mnemonicbox}
\subsection*{પ્રશ્ન 5(a) [3
ગુણ]}\label{q5a}

\textbf{બીટરેટ અને બોડરેટ વ્યાખ્યાઈત કરો.}

\begin{solutionbox}

\textbf{બિટ રેટ અને બોડ રેટ:}

{\def\LTcaptype{none} % do not increment counter
\begin{longtable}[]{@{}
  >{\raggedright\arraybackslash}p{(\linewidth - 6\tabcolsep) * \real{0.2895}}
  >{\raggedright\arraybackslash}p{(\linewidth - 6\tabcolsep) * \real{0.3158}}
  >{\raggedright\arraybackslash}p{(\linewidth - 6\tabcolsep) * \real{0.2368}}
  >{\raggedright\arraybackslash}p{(\linewidth - 6\tabcolsep) * \real{0.1579}}@{}}
\toprule\noalign{}
\begin{minipage}[b]{\linewidth}\raggedright
પેરામીટર
\end{minipage} & \begin{minipage}[b]{\linewidth}\raggedright
વ્યાખ્યા
\end{minipage} & \begin{minipage}[b]{\linewidth}\raggedright
સૂત્ર
\end{minipage} & \begin{minipage}[b]{\linewidth}\raggedright
એકમ
\end{minipage} \\
\midrule\noalign{}
\endhead
\bottomrule\noalign{}
\endlastfoot
\textbf{બિટ રેટ} & પ્રતિ સેકન્ડ ટ્રાન્સમિટ થતાં બાઇનરી અંકો (બિટ્સ)ની સંખ્યા & R =
fs \times n & બિટ્સ પર સેકન્ડ (bps) \\
\textbf{બોડ રેટ} & પ્રતિ સેકન્ડ ટ્રાન્સમિટ થતાં સિગ્નલ એલિમેન્ટ્સ અથવા સિમ્બોલ્સની
સંખ્યા & B = fs & સિમ્બોલ્સ પર સેકન્ડ (બોડ) \\
\end{longtable}
}

\textbf{સંબંધ:}

\begin{itemize}
\tightlist
\item
  બાઇનરી સિગ્નલિંગ માટે: બિટ રેટ = બોડ રેટ
\item
  M-ary સિગ્નલિંગ માટે: બિટ રેટ = બોડ રેટ \times log_{2}M

  \begin{itemize}
  \tightlist
  \item
    જ્યાં M = વિવિધ સિગ્નલ એલિમેન્ટ્સની સંખ્યા
  \end{itemize}
\end{itemize}

\textbf{ઉદાહરણ:}

\begin{itemize}
\tightlist
\item
  4-QAM (M=4): દરેક સિમ્બોલ log_{2}4 = 2 બિટ્સ લઈ જાય છે
\item
  જો બોડ રેટ = 1000 સિમ્બોલ્સ/s, તો બિટ રેટ = 2000 બિટ્સ/s
\end{itemize}

\end{solutionbox}
\begin{mnemonicbox}
``BBSM'' - Bits per second, Baud for Symbols,
Modulation determines relationship

\end{mnemonicbox}
\subsection*{પ્રશ્ન 5(b) [4
ગુણ]}\label{q5b}

\textbf{DPCM નું કાર્ય સમજાવો.}

\begin{solutionbox}

\textbf{ડિફરેન્શિયલ પલ્સ કોડ મોડ્યુલેશન (DPCM):}

\textbf{બ્લોક ડાયાગ્રામ:}

\begin{center}
\textbf{Mermaid Diagram (Code)}
\begin{verbatim}
{Shaded}
{Highlighting}[]
graph LR
    subgraph "Transmitter"
    direction LR
    A[Input] {-{-}{} B[Difference]}
    B {-{-}{} C[Quantizer]}
    C {-{-}{} D[Encoder]}
    D {-{-}{} E[Output]}
    F[Predictor] {-{-}{} B}
    C {-{-}{} F}
    end

    subgraph "Receiver"
    direction LR    
    G[Input] {-{-}{} H[Decoder]}
    H {-{-}{} I[Output]}
    H {-{-}{} J[Predictor]}
    J {-{-}{} I}
    end
{Highlighting}
{Shaded}
\end{verbatim}
\end{center}

\textbf{કાર્ય સિદ્ધાંત:}

\begin{itemize}
\tightlist
\item
  વર્તમાન સેમ્પલ અને અનુમાનિત સેમ્પલ વચ્ચેનો તફાવત એન્કોડ કરે છે
\item
  અગાઉના સેમ્પલ્સ પર આધારિત અનુમાન (કોરિલેશન)
\item
  તફાવતની નાની ડાયનેમિક રેન્જ દરેક સેમ્પલ દીઠ ઓછા બિટ્સની મંજૂરી આપે છે
\end{itemize}

\textbf{ફાયદાઓ:}

\begin{itemize}
\tightlist
\item
  PCM કરતાં ઉચ્ચ કોમ્પ્રેશન રેશિયો
\item
  સમાન ક્વોલિટી માટે ઘટાડેલો બિટ રેટ
\item
  સિગ્નલ કોરિલેશનનો ઉપયોગ કરે છે
\item
  સુધારેલું SNR પરફોર્મન્સ
\end{itemize}

\end{solutionbox}
\begin{mnemonicbox}
``DEEP'' - Difference Encoded, Efficient Prediction,
Exploits correlation, Preserves quality

\end{mnemonicbox}
\subsection*{પ્રશ્ન 5(c) [7
ગુણ]}\label{q5c}

\textbf{બાઈનરી ડેટા 1011001 નીચે પ્રમાણેની લાઈન કોડિંગ ટેકનીકથી ટ્રાન્સમીટ થાય છે
(i) યુનિપોલાર RZ અને NRZ (ii) પોલાર RZ અને NRZ (iii) AMI (iv) Manchester.
બધા માટે વેવ ફોર્મ દોરો.}

\begin{solutionbox}

\textbf{1011001 માટે લાઈન કોડિંગ વેવફોર્મ્સ:}

\begin{verbatim}
Data:      1    0    1    1    0    0    1
           |    |    |    |    |    |    |
           v    v    v    v    v    v    v

Unipolar   
NRZ:     \_\_\_\_|‾‾‾‾|\_\_\_\_|‾‾‾‾|‾‾‾‾|\_\_\_\_|\_\_\_\_|‾‾‾‾|\_\_\_\_
           
Unipolar  
RZ:      \_\_\_\_|‾|\_\_|\_\_\_\_|‾|\_\_|‾|\_\_|\_\_\_\_|\_\_\_\_|‾|\_\_|\_\_\_\_
           
Polar     
NRZ:     \_\_\_\_|‾‾‾‾|\_\_\_\_|‾‾‾‾|‾‾‾‾|\_\_\_\_|\_\_\_\_|‾‾‾‾|\_\_\_\_
          ‾‾‾‾     ‾‾‾‾     ‾‾‾‾     ‾‾‾‾     ‾‾‾‾
Polar
RZ:      \_\_\_\_|‾|\_\_|\_\_\_\_|‾|\_\_|‾|\_\_|\_\_\_\_|\_\_\_\_|‾|\_\_|\_\_\_\_
          ‾‾‾‾|\_|  |\_|‾‾|\_|‾‾|\_|  |\_|  |\_|‾‾|\_|  
AMI:     \_\_\_\_\_|‾|\_\_|\_\_\_\_|\_\_\_|‾|\_\_|\_\_\_\_|\_\_\_\_|‾|\_\_|\_\_\_\_
          ‾‾‾‾|\_|  |\_|  |\_|  |\_|  |\_|  |\_|  |\_|
              
Manchester:
          \_\_\_\_|‾|\_\_|\_|‾‾|\_\_\_\_|‾|\_\_|‾|\_\_|\_\_\_\_|\_\_\_\_|‾|\_\_|\_\_\_\_
          ‾‾‾‾|\_|  |\_|  |\_|‾‾|\_|  |\_|  |\_|‾‾|\_|‾‾|\_|
\end{verbatim}

\textbf{લાઈન કોડિંગ ટેકનિક્સનું વર્ણન:}

{\def\LTcaptype{none} % do not increment counter
\begin{longtable}[]{@{}
  >{\raggedright\arraybackslash}p{(\linewidth - 6\tabcolsep) * \real{0.2391}}
  >{\raggedright\arraybackslash}p{(\linewidth - 6\tabcolsep) * \real{0.1957}}
  >{\raggedright\arraybackslash}p{(\linewidth - 6\tabcolsep) * \real{0.1957}}
  >{\raggedright\arraybackslash}p{(\linewidth - 6\tabcolsep) * \real{0.3696}}@{}}
\toprule\noalign{}
\begin{minipage}[b]{\linewidth}\raggedright
ટેકનિક
\end{minipage} & \begin{minipage}[b]{\linewidth}\raggedright
લોજિક 1
\end{minipage} & \begin{minipage}[b]{\linewidth}\raggedright
લોજિક 0
\end{minipage} & \begin{minipage}[b]{\linewidth}\raggedright
લક્ષણો
\end{minipage} \\
\midrule\noalign{}
\endhead
\bottomrule\noalign{}
\endlastfoot
\textbf{યુનિપોલાર NRZ} & હાઇ લેવલ & ઝીરો લેવલ & બિટ્સ વચ્ચે ઝીરોમાં પાછું ફરતું
નથી \\
\textbf{યુનિપોલાર RZ} & અર્ધ બિટ માટે પલ્સ & ઝીરો લેવલ & અર્ધ બિટ માટે ઝીરોમાં
પાછું ફરે છે \\
\textbf{પોલાર NRZ} & પોઝિટિવ & નેગેટિવ & બિટ્સ વચ્ચે ઝીરોમાં પાછું ફરતું નથી \\
\textbf{પોલાર RZ} & પોઝિટિવ પલ્સ & નેગેટિવ પલ્સ & અર્ધ બિટ માટે ઝીરોમાં પાછું ફરે
છે \\
\textbf{AMI} & અલ્ટરનેટિંગ +/- & ઝીરો લેવલ & ક્રમિક 1 માટે પોલારિટી બદલાય છે \\
\textbf{Manchester} & હાઇ\rightarrowલો & લો\rightarrowહાઇ & બિટની મધ્યમાં ટ્રાન્ઝિશન \\
\end{longtable}
}

\end{solutionbox}
\begin{mnemonicbox}
``UPAM'' - Unipolar, Polar, AMI, Manchester encoding
options

\end{mnemonicbox}
\subsection*{પ્રશ્ન 5(a) OR [3
ગુણ]}\label{q5a}

\textbf{RZ અને NRZ કોડિંગ ઉદાહરણ સાથેસમજાવો.}

\begin{solutionbox}

\textbf{RZ અને NRZ કોડિંગની તુલના:}

{\def\LTcaptype{none} % do not increment counter
\begin{longtable}[]{@{}
  >{\raggedright\arraybackslash}p{(\linewidth - 4\tabcolsep) * \real{0.1897}}
  >{\raggedright\arraybackslash}p{(\linewidth - 4\tabcolsep) * \real{0.3621}}
  >{\raggedright\arraybackslash}p{(\linewidth - 4\tabcolsep) * \real{0.4483}}@{}}
\toprule\noalign{}
\begin{minipage}[b]{\linewidth}\raggedright
પેરામીટર
\end{minipage} & \begin{minipage}[b]{\linewidth}\raggedright
રિટર્ન-ટુ-ઝીરો (RZ)
\end{minipage} & \begin{minipage}[b]{\linewidth}\raggedright
નોન-રિટર્ન-ટુ-ઝીરો (NRZ)
\end{minipage} \\
\midrule\noalign{}
\endhead
\bottomrule\noalign{}
\endlastfoot
\textbf{સિગ્નલ લેવલ્સ} & દરેક બિટમાં ઝીરોમાં પાછું ફરે છે & સંપૂર્ણ બિટ પીરિયડ માટે
લેવલ જાળવે છે \\
\textbf{બેન્ડવિડ્થ} & ઊંચું (\approx 2\times NRZ) & નીચું \\
\textbf{સેલ્ફ-ક્લોકિંગ} & બેહતર (દરેક બિટમાં ટ્રાન્ઝિશન) & નબળું (ટ્રાન્ઝિશન વિના
લાંબા રન હોઈ શકે છે) \\
\textbf{પાવર જરૂરિયાત} & ઊંચી & નીચી \\
\textbf{બિટ સિન્ક્રોનાઇઝેશન} & સરળ & વધુ મુશ્કેલ \\
\textbf{અમલીકરણ} & વધુ જટિલ & સરળ \\
\textbf{DC કોમ્પોનન્ટ} & ઓછો & વધુ \\
\end{longtable}
}

\textbf{101 માટે ઉદાહરણ:}

\begin{verbatim}
Data:        1     0     1
             |     |     |
             v     v     v

NRZ:      \_\_\_|‾‾‾‾‾|\_\_\_\_\_|‾‾‾‾‾|\_\_\_\_
                   
RZ:       \_\_\_|‾|\_\_\_|\_\_\_\_\_|‾|\_\_\_|\_\_\_\_
\end{verbatim}

\end{solutionbox}
\begin{mnemonicbox}
``BPSIDC'' - Bandwidth, Power, Synchronization,
Implementation, DC component

\end{mnemonicbox}
\subsection*{પ્રશ્ન 5(b) OR [4
ગુણ]}\label{q5b}

\textbf{ડેલ્ટા મોડયુલેશન ટૂંકમા સમજાવો.}

\begin{solutionbox}

\textbf{ડેલ્ટા મોડ્યુલેશન (DM):}

\textbf{બ્લોક ડાયાગ્રામ:}

\begin{center}
\textbf{Mermaid Diagram (Code)}
\begin{verbatim}
{Shaded}
{Highlighting}[]
graph LR
    A[Input Signal] {-{-}{} B[(Comparator)]}
    B {-{-}{} C[1{-}bit Quantizer]}
    C {-{-}{} D[Transmission]}
    C {-{-}{} E[Integrator]}
    E {-{-}Feedback{-}{-}{} B}
    D {-{-}{} F[Integrator]}
    F {-{-}{} G[Output Signal]}
{Highlighting}
{Shaded}
\end{verbatim}
\end{center}

\textbf{કાર્ય સિદ્ધાંત:}

\begin{itemize}
\tightlist
\item
  1 બિટનો ઉપયોગ કરીને માત્ર સેમ્પલ્સ વચ્ચેનો તફાવત એન્કોડ કરે છે
\item
  કમ્પેરેટર ચકાસે છે કે ઇનપુટ અનુમાનિત મૂલ્ય કરતાં ઉચ્ચ/નીચું છે
\item
  ઇન્ટિગ્રેટર મૂળ સિગ્નલને અનુમાનિત કરવા માટે બિટ્સને એકત્રિત કરે છે
\item
  આઉટપુટ 1 અને 0 ની શ્રેણી છે જે અપ/ડાઉન સ્ટેપ્સને રજૂ કરે છે
\end{itemize}

\textbf{મર્યાદાઓ:}

\begin{itemize}
\tightlist
\item
  \textbf{સ્લોપ ઓવરલોડ}: ઝડપથી બદલાતા સિગ્નલ્સને ટ્રેક કરી શકતું નથી
\item
  \textbf{ગ્રેન્યુલર નોઇઝ}: સ્થિર સિગ્નલની આસપાસ નાના ફેરફારો
\end{itemize}

\textbf{ફાયદાઓ:}

\begin{itemize}
\tightlist
\item
  ડિફરેન્શિયલ એન્કોડિંગનું સરળતમ સ્વરૂપ
\item
  નીચો બિટ રેટ (સેમ્પલ દીઠ 1 બિટ)
\item
  સરળ અમલીકરણ
\item
  હાર્ડવેર કાર્યક્ષમતા
\end{itemize}

\end{solutionbox}
\begin{mnemonicbox}
``SIDE'' - Single-bit, Integrates Differences,
Encodes changes

\end{mnemonicbox}
\subsection*{પ્રશ્ન 5(c) OR [7
ગુણ]}\label{q5c}

\textbf{PCM-TDM સિસ્ટમ સમજાવો.}

\begin{solutionbox}

\textbf{PCM-TDM સિસ્ટમ:}

\textbf{બ્લોક ડાયાગ્રામ:}

\begin{center}
\textbf{Mermaid Diagram (Code)}
\begin{verbatim}
{Shaded}
{Highlighting}[]
graph TD
    subgraph "Transmitter"
    A1[Channel 1] {-{-}{} B1[LPF]}
    A2[Channel 2] {-{-}{} B2[LPF]}
    A3[Channel 3] {-{-}{} B3[LPF]}
    A4[Channel n] {-{-}{} B4[LPF]}
    B1 {-{-}{} C[Multiplexer]}
    B2 {-{-}{} C}
    B3 {-{-}{} C}
    B4 {-{-}{} C}
    C {-{-}{} D[Sample \& Hold]}
    D {-{-}{} E[Quantizer]}
    E {-{-}{} F[Encoder]}
    F {-{-}{} G[Frame Generator]}
    G {-{-}{} H[Line Coder]}
    H {-{-}{} I[Transmission Medium]}
    end

    subgraph "Receiver"
    I {-{-}{} J[Line Decoder]}
    J {-{-}{} K[Frame Sync]}
    K {-{-}{} L[Decoder]}
    L {-{-}{} M[Demultiplexer]}
    M {-{-}{} N1[LPF]}
    M {-{-}{} N2[LPF]}
    M {-{-}{} N3[LPF]}
    M {-{-}{} N4[LPF]}
    N1 {-{-}{} O1[Channel 1]}
    N2 {-{-}{} O2[Channel 2]}
    N3 {-{-}{} O3[Channel 3]}
    N4 {-{-}{} O4[Channel n]}
    end
{Highlighting}
{Shaded}
\end{verbatim}
\end{center}

\textbf{PCM-TDM ઓપરેશન:}

{\def\LTcaptype{none} % do not increment counter
\begin{longtable}[]{@{}ll@{}}
\toprule\noalign{}
સ્ટેજ & પ્રક્રિયા \\
\midrule\noalign{}
\endhead
\bottomrule\noalign{}
\endlastfoot
\textbf{ફિલ્ટરિંગ} & એલિયાસિંગ અટકાવવા માટે દરેક ચેનલને બેન્ડ-લિમિટ કરે છે \\
\textbf{મલ્ટિપ્લેક્સિંગ} & દરેક ચેનલને ક્રમિક રીતે સેમ્પલ કરે છે \\
\textbf{કન્વર્ઝન} & સેમ્પલ્સને ક્વોન્ટાઇઝ કરે છે અને બાઇનરી કોડમાં રૂપાંતરિત કરે છે \\
\textbf{ફ્રેમિંગ} & સિન્ક બિટ્સ અને ચેનલ આઇડેન્ટિફિકેશન ઉમેરે છે \\
\textbf{ટ્રાન્સમિશન} & ફ્રેમને કોમ્યુનિકેશન માધ્યમ પર મોકલે છે \\
\textbf{ડિમલ્ટિપ્લેક્સિંગ} & પ્રાપ્ત ફ્રેમમાંથી ચેનલ્સને અલગ કરે છે \\
\textbf{રિકન્સ્ટ્રક્શન} & ડિજિટલ સેમ્પલ્સને પાછા એનાલોગ સિગ્નલ્સમાં રૂપાંતરિત કરે છે \\
\end{longtable}
}

\textbf{સિસ્ટમ પેરામીટર્સ:}

\begin{itemize}
\tightlist
\item
  \textbf{ચેનલ કેપેસિટી}: N ચેનલ્સ
\item
  \textbf{સેમ્પલિંગ રેટ}: દરેક ચેનલ માટે fs
\item
  \textbf{ક્વોન્ટિઝેશન}: દરેક સેમ્પલ માટે n બિટ્સ
\item
  \textbf{ફ્રેમ સ્ટ્રક્ચર}: દરેક ચેનલનો 1 સેમ્પલ + સિન્ક
\item
  \textbf{ટોટલ બિટ રેટ}: N \times n \times fs + ઓવરહેડ
\end{itemize}

\end{solutionbox}
\begin{mnemonicbox}
``MOST-FDR'' - Multiplex, Quantize, Sample,
Transmit, Frame, Demultiplex, Reconstruct

\end{mnemonicbox}

\end{document}
