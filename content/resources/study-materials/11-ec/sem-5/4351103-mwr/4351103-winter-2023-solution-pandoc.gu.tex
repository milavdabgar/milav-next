\documentclass[10pt,a4paper]{article}

% content/resources/templates/preamble.tex
\usepackage[margin=0.6in]{geometry}
\author{Milav Dabgar}
\usepackage{amsmath,amssymb,amsthm}
\usepackage{booktabs}
\usepackage{multirow}
\usepackage{xcolor}
\usepackage{tcolorbox}
\tcbuselibrary{breakable,skins}
\usepackage[colorlinks=true,linkcolor=blue]{hyperref}
\usepackage{titlesec}
\usepackage{enumitem}
\usepackage{tikz}
\usepackage{pgfplots}
\usepackage{circuitikz}
\usepackage[version=4]{mhchem}
\usepackage{longtable}
\usepackage{array}
\usepackage{float}
\usepackage{caption}
\usepackage{listings}

\lstset{
  basicstyle=\small\ttfamily,
  breaklines=true,
  breakatwhitespace=false,
  postbreak=\mbox{\textcolor{red}{$\hookrightarrow$}\space},
  float=false,
  numbers=left,
  numberstyle=\tiny\color{gray},
  numbersep=10pt,
  xleftmargin=2em,
  keywordstyle=\color{blue},
  commentstyle=\color{green!60!black},
  stringstyle=\color{purple},
  backgroundcolor=\color{gray!5},
  showstringspaces=false,
  tabsize=2,
  captionpos=b,
  keepspaces=true,
  columns=flexible
}

\pgfplotsset{compat=1.18}
\usetikzlibrary{shapes,arrows,positioning,calc,patterns,decorations.pathmorphing,decorations.markings,arrows.meta}

% Color scheme
\definecolor{headcolor}{RGB}{0,102,204}
\definecolor{keycolor}{RGB}{220,20,60}
\definecolor{solutioncolor}{RGB}{34,139,34}
\definecolor{mnemoniccolor}{RGB}{148,0,211}
\definecolor{codecolor}{RGB}{0,0,100}

% Spacing
\setlength{\parskip}{3pt}
\setlist[itemize]{nosep}
\setlist[enumerate]{nosep}

% Title formatting
\titleformat{\section}{\Large\bfseries\color{headcolor}}{\thesection}{1em}{}
\titleformat{\subsection}{\large\bfseries\color{headcolor}}{\thesubsection}{1em}{}

% Pandoc tightlist compatibility
\providecommand{\tightlist}{%
  \setlength{\itemsep}{0pt}\setlength{\parskip}{0pt}}

% Pandoc longtable compatibility
\newcounter{none}
\def\thenone{}


% content/resources/templates/gujarati-boxes.tex
\usepackage{fontspec}
\usepackage{polyglossia}

% Set Gujarati as main language (document is primarily in Gujarati)
% Note: gloss-gujarati.ldf doesn't exist in polyglossia, but it will use hyphenation patterns
\setdefaultlanguage{gujarati}
\setotherlanguage{english}

% Configure Gujarati font properly
% Use Language=Default to prevent polyglossia from trying to add language-specific features
% that don't exist for Gujarati, which causes "empty feature" warnings
\newfontfamily\gujaratifont[Script=Gujarati,AutoFakeBold=2.5,AutoFakeSlant=0.3]{Noto Sans Gujarati}
\setmainfont[Script=Gujarati,AutoFakeBold=2.5,AutoFakeSlant=0.3]{Noto Sans Gujarati}
% Use Noto Sans Gujarati for monospace to support Gujarati in text
\setmonofont[Scale=0.9]{Noto Sans Gujarati}

% Configure English to use the same font
\newfontfamily\englishfont[Script=Gujarati,AutoFakeBold=2.5,AutoFakeSlant=0.3]{Noto Sans Gujarati}

% Translations for polyglossia
\gappto\captionsgujarati{
  \renewcommand{\tablename}{કોષ્ટક}
  \renewcommand{\figurename}{આકૃતિ}
}

% Helper for TikZ nodes to ensure Gujarati font
\newcommand{\gu}[1]{{\gujaratifont #1}}

% Custom environments
\newtcolorbox{solutionbox}{
    breakable,
    enhanced,
    colback=solutioncolor!5!white,
    colframe=solutioncolor!75!black,
    fonttitle=\bfseries,
    title=જવાબ
}

\newtcolorbox{solutionboxnobreak}{
 colback=solutioncolor!5!white,
 colframe=solutioncolor!75!black,
 fonttitle=\bfseries,
 title=જવાબ
}

\newtcolorbox{keyformula}{
 breakable,
 enhanced,
 colback=keycolor!5!white,
 colframe=keycolor!75!black,
 fonttitle=\bfseries,
 title=રાસાયણિક સમીકરણ/સૂત્ર
}

\newtcolorbox{mnemonicbox}{
 breakable,
 enhanced,
 colback=mnemoniccolor!5!white,
 colframe=mnemoniccolor!75!black,
 fonttitle=\bfseries,
 title=મેમરી ટ્રીક
}


\begin{document}

\begin{center}
{\Huge\bfseries\color{headcolor} Subject Name (Gujarati)}\\[5pt]
{\LARGE 4351103 -- Winter 2023}\\[3pt]
{\large Semester 1 Study Material}\\[3pt]
{\normalsize\textit{Detailed Solutions and Explanations}}
\end{center}

\vspace{10pt}

\subsection*{પ્રશ્ન 1(અ) [3
ગુણ]}\label{uxaaauxab0uxab6uxaa8-1uxa85-3-uxa97uxaa3}

\textbf{ટ્રાન્સમિશન લાઇન માં વોલ્ટેજ અને કરંટ માટે સ્ટેન્ડિંગ વેવ પેટર્નને સ્કેચ કરો, જ્યારે
તે (i) શોર્ટ સર્કિટ, (ii) ઓપન સર્કિટ અને (iii) મેચ્ડ લોડ સાથે સમાપ્ત થાય છે.}

\begin{solutionbox}

\textbf{આકૃતિ:}

\begin{verbatim}
Short Circuit (Z\_L = 0):
     V      
     |      
     |      
     |\_\_\_\_\_\_|\_\_\_\_\_\_|\_\_\_\_\_\_|\_\_\_\_\_\_ 0V
       λ/4   λ/2   3λ/4    λ
     
     I      
   I\_max {-{-}{-}|{-}{-}{-}|{-}{-}{-}|{-}{-}{-}}
     |                  
     |      
     0 \_\_\_\_\_\_\_\_\_\_\_\_\_\_\_\_\_\_ 0A

Open Circuit (Z\_L = ):
   V\_max {-{-}{-}|{-}{-}{-}|{-}{-}{-}|{-}{-}{-}}
     |                  
     V      
     0 \_\_\_\_\_\_\_\_\_\_\_\_\_\_\_\_\_\_ 0V
       λ/4   λ/2   3λ/4    λ
     
     I      
     |      
     |      
     |\_\_\_\_\_\_|\_\_\_\_\_\_|\_\_\_\_\_\_|\_\_\_\_\_\_ 0A
       λ/4   λ/2   3λ/4    λ

Matched Load (Z\_L = Z\_0):
     V      
   {-{-}{-}|{-}{-}{-}|{-}{-}{-}|{-}{-}{-}|{-}{-}{-} Constant}
     
     I      
   {-{-}{-}|{-}{-}{-}|{-}{-}{-}|{-}{-}{-}|{-}{-}{-} Constant}
\end{verbatim}

\begin{itemize}
\tightlist
\item
  \textbf{શોર્ટ સર્કિટ}: લોડ પર વોલ્ટેજ ન્યૂનતમ, કરંટ મહત્તમ
\item
  \textbf{ઓપન સર્કિટ}: લોડ પર વોલ્ટેજ મહત્તમ, કરંટ ન્યૂનતમ\\
\item
  \textbf{મેચ્ડ લોડ}: સ્થિર વોલ્ટેજ અને કરંટ, કોઈ પ્રતિબિંબ નથી
\end{itemize}

\textbf{યાદાશ્ત સૂત્ર}: ``SOC - શોર્ટ કરંટ ખોલે, ઓપન કરંટ બંધ કરે''

\end{solutionbox}
\subsection*{પ્રશ્ન 1(બ) [4
ગુણ]}\label{uxaaauxab0uxab6uxaa8-1uxaac-4-uxa97uxaa3}

\textbf{માઇક્રોવેવ ફ્રીક્વન્સી માટે બે સમાંતર વાયર ટ્રાન્સમિશન લાઇનના સમકક્ષ
સર્કિટનો નકશો દોરો અને સમજાવો.}

\begin{solutionbox}

\textbf{આકૃતિ:}

\begin{verbatim}
       R      L      R      L
    {-{-}{-}|\^{}\^{}\^{}|{-}{-}|||||{-}{-}{-}|\^{}\^{}\^{}|{-}{-}|||||{-}{-}{-}}
    |      |       |      |       |
    |      G       C      G       C
    |     |||     {-{-}{-}    |||     {-}{-}{-}}
    |     |||     {-{-}{-}    |||     {-}{-}{-}}
    {-{-}{-}|\^{}\^{}\^{}|{-}{-}|||||{-}{-}{-}|\^{}\^{}\^{}|{-}{-}|||||{-}{-}{-}}
       R      L      R      L
       
       {-{-} Δz {-}{-}}
\end{verbatim}

\begin{itemize}
\tightlist
\item
  \textbf{R}: એકમ લંબાઈ દીઠ શ્રેણી પ્રતિકાર (કંડક્ટર લોસિસ)
\item
  \textbf{L}: એકમ લંબાઈ દીઠ શ્રેણી ઇન્ડક્ટન્સ (ચુંબકીય ક્ષેત્ર સંગ્રહ)
\item
  \textbf{G}: એકમ લંબાઈ દીઠ શંટ કંડક્ટન્સ (ડાઇઇલેક્ટ્રિક લોસિસ)
\item
  \textbf{C}: એકમ લંબાઈ દીઠ શંટ કેપેસિટન્સ (વિદ્યુત ક્ષેત્ર સંગ્રહ)
\end{itemize}

\textbf{પ્રાથમિક સ્થિરાંકો કોષ્ટક:}

{\def\LTcaptype{none} % do not increment counter
\begin{longtable}[]{@{}llll@{}}
\toprule\noalign{}
પરિમાપ & પ્રતીક & એકમ & અસર \\
\midrule\noalign{}
\endhead
\bottomrule\noalign{}
\endlastfoot
પ્રતિકાર & R & Ω/m & શક્તિ નુકસાન \\
ઇન્ડક્ટન્સ & L & H/m & ચુંબકીય ઊર્જા \\
કંડક્ટન્સ & G & S/m & લીકેજ કરંટ \\
કેપેસિટન્સ & C & F/m & વિદ્યુત ઊર્જા \\
\end{longtable}
}

\textbf{યાદાશ્ત સૂત્ર}: ``RLGC - ખરેખર મોટી કેબલ્સ''

\end{solutionbox}
\subsection*{પ્રશ્ન 1(ક) [7
ગુણ]}\label{uxaaauxab0uxab6uxaa8-1uxa95-7-uxa97uxaa3}

\textbf{આઇસોલેટર ના સિદ્ધાંત, બાંધકામ અને કાર્યને જરૂરી સ્કેચ સાથે સમજાવો.}

\begin{solutionbox}

\textbf{સિદ્ધાંત}: આઇસોલેટર માઇક્રોવેવ સિગ્નલને ફક્ત આગળની દિશામાં જ પસાર કરવા દે
છે \textbf{ફેરાઇટ મટિરિયલ} અને \textbf{ફેરાડે રોટેશન અસર} નો ઉપયોગ કરીને.

\textbf{બાંધકામ આકૃતિ:}

\begin{center}
\textbf{Mermaid Diagram (Code)}
\begin{verbatim}
{Shaded}
{Highlighting}[]
graph LR
    A[Input Port] {-{-}{} B[Ferrite Rod]}
    B {-{-}{} C[Permanent Magnet]}
    C {-{-}{} D[Output Port]}
    E[Resistive Load] {-{-}{} B}
    F[Waveguide] {-{-}{} B}
{Highlighting}
{Shaded}
\end{verbatim}
\end{center}

\textbf{કાર્યપ્રણાલી}:

\begin{itemize}
\tightlist
\item
  \textbf{આગળની દિશા}: સિગ્નલ ઓછા નુકસાન સાથે ફેરાઇટ માંથી પસાર થાય છે
\item
  \textbf{પાછળની દિશા}: સિગ્નલ 45^\circ ફેરવાય છે અને રેઝિસ્ટિવ લોડ દ્વારા શોષાય છે
\item
  \textbf{ચુંબકીય ક્ષેત્ર} ફેરાઇટ મટિરિયલને બાયાસ કરે છે
\item
  \textbf{આઇસોલેશન}: સામાન્ય રીતે 20-30 dB
\end{itemize}

\textbf{ઉપયોગો}:

\begin{itemize}
\tightlist
\item
  \textbf{ટ્રાન્સમિટરને સુરક્ષા} રિફ્લેક્ટેડ પાવર થી
\item
  \textbf{એમ્પ્લિફાયર સર્કિટમાં ઓસિલેશન} અટકાવે છે
\item
  \textbf{સોર્સ ઇમ્પીડન્સ મેચિંગ} જાળવે છે
\end{itemize}

\textbf{વિશેષતાઓ કોષ્ટક:}

{\def\LTcaptype{none} % do not increment counter
\begin{longtable}[]{@{}lll@{}}
\toprule\noalign{}
પરિમાપ & મૂલ્ય & એકમ \\
\midrule\noalign{}
\endhead
\bottomrule\noalign{}
\endlastfoot
આઇસોલેશન & 20-30 & dB \\
ઇન્સર્શન લોસ & 0.5-1 & dB \\
VSWR & \textless1.5 & - \\
\end{longtable}
}

\textbf{યાદાશ્ત સૂત્ર}: ``આગળ અલગ કરો, પાછળ શોષો''

\end{solutionbox}
\subsection*{પ્રશ્ન 1(ક) વિકલ્પ [7
ગુણ]}\label{uxaaauxab0uxab6uxaa8-1uxa95-uxab5uxa95uxab2uxaaa-7-uxa97uxaa3}

\textbf{ટ્રાન્સમિશન લાઇન અને વેવગાઇડની સરખામણી કરો.}

\begin{solutionbox}

\textbf{સરખામણી કોષ્ટક:}

{\def\LTcaptype{none} % do not increment counter
\begin{longtable}[]{@{}lll@{}}
\toprule\noalign{}
પરિમાપ & ટ્રાન્સમિશન લાઇન & વેવગાઇડ \\
\midrule\noalign{}
\endhead
\bottomrule\noalign{}
\endlastfoot
\textbf{ફ્રીક્વન્સી રેન્જ} & DC થી માઇક્રોવેવ & કટઓફ ફ્રીક્વન્સી ઉપર \\
\textbf{પાવર હેન્ડલિંગ} & મર્યાદિત & ઉચ્ચ પાવર ક્ષમતા \\
\textbf{લોસિસ} & વધારે (I^{2}R લોસિસ) & ઓછા (કોઈ કેન્દ્રીય કંડક્ટર નથી) \\
\textbf{સાઇઝ} & કોમ્પેક્ટ & નીચી ફ્રીક્વન્સીએ મોટું \\
\textbf{મોડ્સ} & TEM મોડ & TE અને TM મોડ્સ \\
\textbf{ઇન્સ્ટોલેશન} & સરળ & જટિલ માઉન્ટિંગ \\
\textbf{કિંમત} & ઓછી & વધારે \\
\textbf{બેન્ડવિડ્થ} & વિશાળ & મોડ્સ દ્વારા મર્યાદિત \\
\end{longtable}
}

\textbf{મુખ્ય તફાવતો}:

\begin{itemize}
\tightlist
\item
  \textbf{ટ્રાન્સમિશન લાઇન}: બે કંડક્ટર વાપરે છે, TEM મોડ સપોર્ટ કરે છે
\item
  \textbf{વેવગાઇડ}: સિંગલ હોલો કંડક્ટર, TE/TM મોડ્સ સપોર્ટ કરે છે
\item
  \textbf{કટઓફ ફ્રીક્વન્સી}: વેવગાઇડ માં લઘુત્તમ ઓપરેટિંગ ફ્રીક્વન્સી
\item
  \textbf{ફીલ્ડ પેટર્ન}: અલગ ઇલેક્ટ્રોમેગ્નેટિક ફીલ્ડ વિતરણ
\end{itemize}

\textbf{ઉપયોગો}:

\begin{itemize}
\tightlist
\item
  \textbf{ટ્રાન્સમિશન લાઇન}: લો પાવર, બ્રોડબેન્ડ એપ્લિકેશન
\item
  \textbf{વેવગાઇડ}: હાઇ પાવર રડાર, સેટેલાઇટ કોમ્યુનિકેશન
\end{itemize}

\textbf{યાદાશ્ત સૂત્ર}: ``ટ્રાન્સમિશન બે-વાયર ચાલે, વેવગાઇડ વિશાળ ચાલે''

\end{solutionbox}
\subsection*{પ્રશ્ન 2(અ) [3
ગુણ]}\label{uxaaauxab0uxab6uxaa8-2uxa85-3-uxa97uxaa3}

\textbf{વ્યાખ્યા આપો: (i) VSWR, (ii) રિફ્લેક્શન કોઇફિશન્ટ, અને (iii) સ્કિન અસર}

\begin{solutionbox}

\textbf{વ્યાખ્યાઓ:}

\begin{itemize}
\tightlist
\item
  \textbf{VSWR (વોલ્ટેજ સ્ટેન્ડિંગ વેવ રેશિયો)}: ટ્રાન્સમિશન લાઇન પર મહત્તમ અને
  ન્યૂનતમ વોલ્ટેજ એમ્પ્લિટ્યુડનો ગુણોત્તર

  \begin{itemize}
  \tightlist
  \item
    ફોર્મ્યુલા: VSWR = V\_max/V\_min =
    (1+\textbar Γ\textbar)/(1-\textbar Γ\textbar)
  \end{itemize}
\item
  \textbf{રિફ્લેક્શન કોઇફિશન્ટ (Γ)}: પ્રતિબિંબિત અને આવતા વોલ્ટેજ એમ્પ્લિટ્યુડનો
  ગુણોત્તર

  \begin{itemize}
  \tightlist
  \item
    ફોર્મ્યુલા: Γ = (Z\_L - Z\_0)/(Z\_L + Z\_0)
  \end{itemize}
\item
  \textbf{સ્કિન અસર}: ઉચ્ચ ફ્રીક્વન્સીએ કરંટ મુખ્યત્વે કંડક્ટરની સપાટી પર વહે છે

  \begin{itemize}
  \tightlist
  \item
    સ્કિન ડેપ્થ: δ = \sqrt(2/ωμσ)
  \end{itemize}
\end{itemize}

\textbf{પરિમાપો કોષ્ટક:}

{\def\LTcaptype{none} % do not increment counter
\begin{longtable}[]{@{}lll@{}}
\toprule\noalign{}
પરિમાપ & રેન્જ & આદર્શ મૂલ્ય \\
\midrule\noalign{}
\endhead
\bottomrule\noalign{}
\endlastfoot
VSWR & 1 થી \infty & 1 (મેચ્ડ) \\
& Γ & \\
સ્કિન ડેપ્થ & μm થી mm & ફ્રીક્વન્સી આધારિત \\
\end{longtable}
}

\textbf{યાદાશ્ત સૂત્ર}: ``VSWR વેરિયે, ગામા ગાઇડ, સ્કિન સંકોચે''

\end{solutionbox}
\subsection*{પ્રશ્ન 2(બ) [4
ગુણ]}\label{uxaaauxab0uxab6uxaa8-2uxaac-4-uxa97uxaa3}

\textbf{યોગ્ય સ્કેચ સાથે ટુ-હોલ ડાયરેક્શનલ કપ્લરનું કાર્ય સમજાવો.}

\begin{solutionbox}

\textbf{બાંધકામ આકૃતિ:}

\begin{verbatim}
Main Waveguide:
|===============================|
|      P1        P2           |
|===============================|
         ○    ○   Two holes
|===============================|
|      P4        P3           |
|===============================|
Auxiliary Waveguide
\end{verbatim}

\textbf{કાર્યપ્રણાલી સિદ્ધાંત}:

\begin{itemize}
\tightlist
\item
  \textbf{બે છિદ્રો} λ/4 અંતરે વેવગાઇડ વચ્ચે ઊર્જા કપલ કરે છે
\item
  \textbf{આગળનું તરંગ}: કપલ્ડ સિગ્નલ P3 પર ઉમેરાય, P4 પર રદ થાય
\item
  \textbf{પાછળનું તરંગ}: કપલ્ડ સિગ્નલ P4 પર ઉમેરાય, P3 પર રદ થાય
\item
  \textbf{ડાયરેક્ટિવિટી}: યોગ્ય છિદ્ર અંતર અને સાઇઝ દ્વારા પ્રાપ્ત
\end{itemize}

\textbf{કપલિંગ મેકેનિઝમ}:

\begin{itemize}
\tightlist
\item
  \textbf{ઇલેક્ટ્રિક ફીલ્ડ કપલિંગ} છિદ્રો દ્વારા
\item
  \textbf{ફેઝ ડિફરન્સ} ડાયરેક્શનલ કપલિંગ બનાવે છે
\item
  \textbf{કપલિંગ ફેક્ટર}: C = 10 log(P1/P3) dB
\end{itemize}

\textbf{પર્ફોર્મન્સ પરિમાપો:}

{\def\LTcaptype{none} % do not increment counter
\begin{longtable}[]{@{}ll@{}}
\toprule\noalign{}
પરિમાપ & સામાન્ય મૂલ્ય \\
\midrule\noalign{}
\endhead
\bottomrule\noalign{}
\endlastfoot
કપલિંગ & 10-30 dB \\
ડાયરેક્ટિવિટી & 25-40 dB \\
VSWR & \textless1.3 \\
\end{longtable}
}

\textbf{યાદાશ્ત સૂત્ર}: ``બે છિદ્ર, બે દિશા, સંપૂર્ણ નિયંત્રણ''

\end{solutionbox}
\subsection*{પ્રશ્ન 2(ક) [7
ગુણ]}\label{uxaaauxab0uxab6uxaa8-2uxa95-7-uxa97uxaa3}

\textbf{વેવગાઇડ દ્વારા માઇક્રોવેવનું પ્રસારણ વર્ણવો અને કટ ઓફ તરંગલંબાઇનું સમીકરણ
મેળવો.}

\begin{solutionbox}

\textbf{પ્રસારણ સિદ્ધાંત}: ઇલેક્ટ્રોમેગ્નેટિક તરંગો વેવગાઇડ દ્વારા \textbf{TE અને TM
મોડ્સ} માં વિશિષ્ટ ફીલ્ડ પેટર્ન સાથે પ્રસારિત થાય છે.

\textbf{તરંગ સમીકરણ}: લંબચોરસ વેવગાઇડ માટે, તરંગ સમીકરણ: \nabla^{2}E + γ^{2}E = 0

જ્યાં γ^{2} = β^{2} - k^{2}

\textbf{કટઓફ તરંગલંબાઇ વ્યુત્પત્તિ:}

\textbf{TE\_mn મોડ} માટે લંબચોરસ વેવગાઇડમાં:

\begin{itemize}
\tightlist
\item
  \textbf{કટઓફ ફ્રીક્વન્સી}: f\_c = (c/2)\sqrt[(m/a)^{2} + (n/b)^{2}]
\item
  \textbf{કટઓફ તરંગલંબાઇ}: λ\_c = 2/\sqrt[(m/a)^{2} + (n/b)^{2}]
\end{itemize}

\textbf{ડોમિનન્ટ TE_{1}_{0} મોડ} માટે:

\begin{itemize}
\tightlist
\item
  λ\_c = 2a (જ્યાં a એ પહોળું પરિમાણ છે)
\end{itemize}

\textbf{પ્રસારણ શરતો}:

\begin{itemize}
\tightlist
\item
  \textbf{કટઓફ નીચે} (f \textless{} f\_c): એવનેસન્ટ તરંગ, ઘાતાંકીય ક્ષય
\item
  \textbf{કટઓફ ઉપર} (f \textgreater{} f\_c): પ્રસારિત તરંગ
\item
  \textbf{ફેઝ વેગ}: v\_p = c/\sqrt[1 - (f\_c/f)^{2}]
\item
  \textbf{ગ્રુપ વેગ}: v\_g = c\sqrt[1 - (f\_c/f)^{2}]
\end{itemize}

\textbf{મોડ ચાર્ટ:}

\begin{center}
\textbf{Mermaid Diagram (Code)}
\begin{verbatim}
{Shaded}
{Highlighting}[]
graph LR
    A[TE_{1_{0}] {-}{-}{} B[TE_{2}_{0}]}
    A {-{-}{} C[TE_{0}_{1}]}
    B {-{-}{} D[TE_{1}_{1}]}
    C {-{-}{} D}
{Highlighting}
{Shaded}
\end{verbatim}
\end{center}

\textbf{મુખ્ય સંબંધો}:

\begin{itemize}
\tightlist
\item
  v\_p \times v\_g = c^{2}
\item
  λ\_g = λ_{0}/\sqrt[1 - (λ_{0}/λ\_c)^{2}]
\end{itemize}

\textbf{યાદાશ્ત સૂત્ર}: ``કટ-ઓફ આવે, પ્રસારણ આગળ વધે''

\end{solutionbox}
\subsection*{પ્રશ્ન 2(અ) વિકલ્પ [3
ગુણ]}\label{uxaaauxab0uxab6uxaa8-2uxa85-uxab5uxa95uxab2uxaaa-3-uxa97uxaa3}

\textbf{સિંગલ સ્ટબનો ઉપયોગ કરીને ઇમ્પીડન્સ મેચિંગ સમજાવો.}

\begin{solutionbox}

\textbf{સિદ્ધાંત}: સિંગલ સ્ટબ મેચિંગ \textbf{શોર્ટ-સર્કિટેડ} અથવા
\textbf{ઓપન-સર્કિટેડ} સ્ટબનો ઉપયોગ કરીને લોડ ઇમ્પીડન્સના રિએક્ટિવ ઘટકને રદ કરે છે.

\textbf{સ્ટબ આકૃતિ:}

\begin{verbatim}
Source {-{-}{-}|{-}{-}{-}●{-}{-}{-}|{-}{-}{-} Load}
  Z_{0      |       |    Z\_L}
          |       |
         {-{-}{-}      |}
        Stub      |
         l\_s      d
\end{verbatim}

\textbf{ડિઝાઇન સ્ટેપ્સ}:

\begin{itemize}
\tightlist
\item
  \textbf{સ્ટેપ 1}: અંતર `d' શોધો જ્યાં નોર્મલાઇઝ્ડ કંડક્ટન્સ = 1
\item
  \textbf{સ્ટેપ 2}: જરૂરી સ્ટબ સસેપ્ટન્સ ગણો: B\_s = -B\_load\\
\item
  \textbf{સ્ટેપ 3}: સ્ટબ લંબાઇ નક્કી કરો: l\_s B\_s થી
\end{itemize}

\textbf{સ્મિથ ચાર્ટ પદ્ધતિ}:

\begin{itemize}
\tightlist
\item
  નોર્મલાઇઝ્ડ લોડ ઇમ્પીડન્સ પ્લોટ કરો
\item
  મેચિંગ પોઇન્ટ શોધવા જનરેટર તરફ આગળ વધો
\item
  કેન્દ્ર પોઇન્ટ પ્રાપ્ત કરવા સ્ટબ સસેપ્ટન્સ ઉમેરો
\end{itemize}

\textbf{યાદાશ્ત સૂત્ર}: ``સિંગલ સ્ટબ સસેપ્ટન્સ ઉકેલે''

\end{solutionbox}
\subsection*{પ્રશ્ન 2(બ) વિકલ્પ [4
ગુણ]}\label{uxaaauxab0uxab6uxaa8-2uxaac-uxab5uxa95uxab2uxaaa-4-uxa97uxaa3}

\textbf{હાઇબ્રિડ રિંગને જરૂરી સ્કેચ સાથે સમજાવો.}

\begin{solutionbox}

\textbf{બાંધકામ આકૃતિ:}

\begin{verbatim}
graph TB
    A[Port 1] {-{-} B[Ring Junction]}
    C[Port 2] {-{-} B}
    D[Port 3] {-{-} B}
    E[Port 4] {-{-} B}
    B {-{-} F[3λ/2 Ring Path]}
\end{verbatim}

\textbf{કાર્યપ્રણાલી સિદ્ધાંત}:

\begin{itemize}
\tightlist
\item
  \textbf{રિંગ પરિધિ}: 3λ/2 (1.5 તરંગલંબાઇ)
\item
  \textbf{સમાન પાથ લંબાઇ} દરેક પોર્ટથી વિરુદ્ધ પોર્ટ સુધી
\item
  \textbf{180^\circ ફેઝ ડિફરન્સ} બાજુના પોર્ટ વચ્ચે
\end{itemize}

\textbf{S-મેટ્રિક્સ ગુણધર્મો}:

\begin{itemize}
\tightlist
\item
  \textbf{આઇસોલેશન}: પોર્ટ 1-3 અને પોર્ટ 2-4 આઇસોલેટેડ છે
\item
  \textbf{પાવર ડિવિઝન}: 180^\circ ફેઝ ડિફરન્સ સાથે સમાન વિભાજન
\item
  \textbf{ઇમ્પીડન્સ}: બધા પોર્ટ Z_{0} સાથે મેચ્ડ
\end{itemize}

\textbf{ઉપયોગો}:

\begin{itemize}
\tightlist
\item
  \textbf{બેલેન્સ્ડ મિક્સર}
\item
  \textbf{પુશ-પુલ એમ્પ્લિફાયર}
\item
  \textbf{ફેઝ તુલના સર્કિટ}
\end{itemize}

\textbf{પર્ફોર્મન્સ કોષ્ટક:}

{\def\LTcaptype{none} % do not increment counter
\begin{longtable}[]{@{}ll@{}}
\toprule\noalign{}
પરિમાપ & મૂલ્ય \\
\midrule\noalign{}
\endhead
\bottomrule\noalign{}
\endlastfoot
આઇસોલેશન & \textgreater25 dB \\
રિટર્ન લોસ & \textgreater20 dB \\
ફેઝ બેલેન્સ & \pm5^\circ \\
\end{longtable}
}

\textbf{યાદાશ્ત સૂત્ર}: ``રિંગ ફરે, પોર્ટ જોડાય''

\end{solutionbox}
\subsection*{પ્રશ્ન 2(ક) વિકલ્પ [7
ગુણ]}\label{uxaaauxab0uxab6uxaa8-2uxa95-uxab5uxa95uxab2uxaaa-7-uxa97uxaa3}

\textbf{મેજિક ટીના બાંધકામ, કાર્ય અને કોઈપણ એક એપ્લિકેશનને જરૂરી ડાયાગ્રામ સાથે
સમજાવો.}

\begin{solutionbox}

\textbf{બાંધકામ}: મેજિક ટી \textbf{E-પ્લેન} અને \textbf{H-પ્લેન} ટીઝને તેમના જંક્શન
પર જોડીને બને છે.

\textbf{સ્ટ્રક્ચર આકૃતિ:}

\begin{verbatim}
       H{-arm (Sum port)}
           |
           |
    {-{-}{-}{-}{-}{-}{-}●{-}{-}{-}{-}{-}{-}{-} }
   |               |
E{-arm     Junction  Collinear}
(Diff)              arms
   |               |
    {-{-}{-}{-}{-}{-}{-}●{-}{-}{-}{-}{-}{-}{-}}
           |
           |
       Matched load
\end{verbatim}

\textbf{કાર્યપ્રણાલી સિદ્ધાંત}:

\begin{itemize}
\tightlist
\item
  \textbf{પોર્ટ 1,2}: કોલિનિયર આર્મ્સ (ઇનપુટ/આઉટપુટ પોર્ટ)
\item
  \textbf{પોર્ટ 3}: H-આર્મ (સમ/Σ પોર્ટ)\\
\item
  \textbf{પોર્ટ 4}: E-આર્મ (ડિફરન્સ/Δ પોર્ટ)
\item
  \textbf{આઇસોલેશન}: સમ અને ડિફરન્સ પોર્ટ વચ્ચે
\end{itemize}

\textbf{S-મેટ્રિક્સ ગુણધર્મો:}

\begin{center}
\textbf{Mermaid Diagram (Code)}
\begin{verbatim}
{Shaded}
{Highlighting}[]
graph LR
    A[Port 1] {-.{-}{}|In phase| B[H{-}arm]}
    C[Port 2] {-.{-}{}|In phase| B}
    A {-{-}{}|Out of phase| D[E{-}arm]}
    C {-{-}{}|180^ phase| D}
{Highlighting}
{Shaded}
\end{verbatim}
\end{center}

\textbf{એપ્લિકેશન - રડાર ડુપ્લેક્સર:}

\begin{itemize}
\tightlist
\item
  \textbf{ટ્રાન્સમિટ}: પાવર H-આર્મમાં આપવામાં આવે, પોર્ટ 1,2 માં સમાન વિભાજન
\item
  \textbf{રિસીવ}: પ્રાપ્ત સિગ્નલ E-આર્મ પર રિસીવર માટે ભેગા થાય
\item
  \textbf{આઇસોલેશન}: ટ્રાન્સમિશન દરમિયાન રિસીવરનું રક્ષણ
\item
  \textbf{ફાયદો}: ટ્રાન્સમિટ/રિસીવ માટે સિંગલ એન્ટેના
\end{itemize}

\textbf{પર્ફોર્મન્સ સ્પેસિફિકેશન:}

{\def\LTcaptype{none} % do not increment counter
\begin{longtable}[]{@{}ll@{}}
\toprule\noalign{}
પરિમાપ & મૂલ્ય \\
\midrule\noalign{}
\endhead
\bottomrule\noalign{}
\endlastfoot
આઇસોલેશન & \textgreater30 dB \\
VSWR & \textless1.3 \\
પાવર સ્પ્લિટ & 3 dB \\
ફેઝ બેલેન્સ & \pm5^\circ \\
\end{longtable}
}

\textbf{મુખ્ય લક્ષણો}:

\begin{itemize}
\tightlist
\item
  \textbf{સિમેટ્રિક સ્ટ્રક્ચર} સમાન પાવર વિભાજન ખાતરી આપે છે
\item
  \textbf{ઓર્થોગોનલ ફીલ્ડ્સ} પોર્ટ આઇસોલેશન પ્રદાન કરે છે
\item
  \textbf{બ્રોડબેન્ડ ઓપરેશન} ઓક્ટેવ બેન્ડવિડ્થ પર
\end{itemize}

\textbf{યાદાશ્ત સૂત્ર}: ``મેજિક આઇસોલેશન બનાવે, ટી સાથે ટ્રાન્સમિટ''

\end{solutionbox}
\subsection*{પ્રશ્ન 3(અ) [3
ગુણ]}\label{uxaaauxab0uxab6uxaa8-3uxa85-3-uxa97uxaa3}

\textbf{બ્લોક ડાયાગ્રામની મદદથી એટેન્યુએશન માપન સમજાવો.}

\begin{solutionbox}

\textbf{બ્લોક ડાયાગ્રામ:}

\begin{center}
\textbf{Mermaid Diagram (Code)}
\begin{verbatim}
{Shaded}
{Highlighting}[]
graph LR
    A[Signal Generator] {-{-}{} B[Attenuator Under Test]}
    B {-{-}{} C[Power Meter]}
    D[Reference Path] {-{-}{} C}
    E[Switch] {-{-}{} B}
    E {-{-}{} D}
{Highlighting}
{Shaded}
\end{verbatim}
\end{center}

\textbf{માપન પ્રક્રિયા}:

\begin{itemize}
\tightlist
\item
  \textbf{સ્ટેપ 1}: એટેન્યુએટર વિના પાવર માપો (P_{1})
\item
  \textbf{સ્ટેપ 2}: એટેન્યુએટર નાખો, પાવર માપો (P_{2})\\
\item
  \textbf{સ્ટેપ 3}: એટેન્યુએશન ગણો = 10 log(P_{1}/P_{2}) dB
\end{itemize}

\textbf{પદ્ધતિઓ}:

\begin{itemize}
\tightlist
\item
  \textbf{સબસ્ટિટ્યુશન પદ્ધતિ}: જાણીતા એટેન્યુએટર સાથે તુલના
\item
  \textbf{ડાયરેક્ટ પદ્ધતિ}: ઇનપુટ અને આઉટપુટ પાવર માપો
\item
  \textbf{IF સબસ્ટિટ્યુશન}: ઇન્ટરમીડિયેટ ફ્રીક્વન્સીનો ઉપયોગ
\end{itemize}

\textbf{યાદાશ્ત સૂત્ર}: ``એટેન્યુએશન = પાવર_{1}/પાવર_{2}''

\end{solutionbox}
\subsection*{પ્રશ્ન 3(બ) [4
ગુણ]}\label{uxaaauxab0uxab6uxaa8-3uxaac-4-uxa97uxaa3}

\textbf{એપલગેટ ડાયાગ્રામની મદદથી બે કેવિટી ક્લિસ્ટ્રોનમાં વેગ મોડ્યુલેશન સમજાવો.}

\begin{solutionbox}

\textbf{બે કેવિટી ક્લિસ્ટ્રોન આકૃતિ:}

\begin{verbatim}
Electron {-{-}{-}{-}{-} ●======● {-}{-}{-}{-}{-} ●======● {-}{-}{-}{-}{-} Collector}
Gun            Input    Drift   Output
               Cavity   Space   Cavity
                 |               |
              RF Input         RF Output
\end{verbatim}

\textbf{એપલગેટ ડાયાગ્રામ:}

\begin{verbatim}
Distance 
   |     
   |  ╱  ╲     ╱  ╲     ╱  ╲
   | ╱    ╲   ╱    ╲   ╱    ╲
   |╱      ╲ ╱      ╲ ╱      ╲
   |        X        X        X   Bunching
Tim|       ╱ ╲      ╱ ╲      ╱ ╲  
   ↓      ╱   ╲    ╱   ╲    ╱   ╲
         ╱     ╲  ╱     ╲  ╱     ╲
        ╱       ╲╱       ╲╱       ╲
Fast electrons  Slow electrons
\end{verbatim}

\textbf{વેલોસિટી મોડ્યુલેશન પ્રક્રિયા}:

\begin{itemize}
\tightlist
\item
  \textbf{ઇનપુટ કેવિટી}: ઇલેક્ટ્રોન RF ફીલ્ડથી ઊર્જા મેળવે/ગુમાવે છે
\item
  \textbf{ડ્રિફ્ટ સ્પેસ}: ઝડપી ઇલેક્ટ્રોન ધીમા ઇલેક્ટ્રોનને મળે છે
\item
  \textbf{બંચિંગ}: ઇલેક્ટ્રોન ડેન્સિટી સમયાંતરે બદલાય છે
\item
  \textbf{આઉટપુટ કેવિટી}: બંચ્ડ ઇલેક્ટ્રોન RF કરંટ ઇન્ડ્યુસ કરે છે
\end{itemize}

\textbf{મુખ્ય પરિમાપો}:

\begin{itemize}
\tightlist
\item
  \textbf{ટ્રાન્ઝિટ ટાઇમ}: τ = L/v_{0} (જ્યાં L = ડ્રિફ્ટ સ્પેસ લંબાઇ)
\item
  \textbf{બંચિંગ પરિમાપ}: X = βn/2
\item
  \textbf{ઓપ્ટિમમ બંચિંગ}: X = 1.84
\end{itemize}

\textbf{યાદાશ્ત સૂત્ર}: ``વેલોસિટી વેરિયે, બંચિંગ બિલ્ડ''

\end{solutionbox}
\subsection*{પ્રશ્ન 3(ક) [7
ગુણ]}\label{uxaaauxab0uxab6uxaa8-3uxa95-7-uxa97uxaa3}

\textbf{મેગ્નેટ્રોનમાં વિદ્યુત અને ચુંબકીય ક્ષેત્રના સિદ્ધાંત, નિર્માણ અને અસર સમજાવો.}

\begin{solutionbox}

\textbf{સિદ્ધાંત}: મેગ્નેટ્રોન \textbf{ક્રોસ્ડ ઇલેક્ટ્રિક અને મેગ્નેટિક ફીલ્ડ્સ} નો ઉપયોગ
કરીને \textbf{સાયક્લોટ્રોન મોશન} ઓફ ઇલેક્ટ્રોન દ્વારા હાઇ-પાવર માઇક્રોવેવ ઓસિલેશન
જનરેટ કરે છે.

\textbf{બાંધકામ આકૃતિ:}

\begin{verbatim}
    Permanent Magnet (N)
         ↓  ↓  ↓  ↓
    ┌─────────────────┐
    │  ○  ○  ○  ○  ○  │  Resonant Cavities
    │○               ○│
    │  ●─ Cathode ─●  │  Central cathode
    │○               ○│
    │  ○  ○  ○  ○  ○  │
    └─────────────────┘
         ↑  ↑  ↑  ↑
    Permanent Magnet (S)
\end{verbatim}

\textbf{ફીલ્ડ અસરો}:

\begin{itemize}
\tightlist
\item
  \textbf{ઇલેક્ટ્રિક ફીલ્ડ (E)}: રેડિયલ, કેથોડથી એનોડ સુધી
\item
  \textbf{મેગ્નેટિક ફીલ્ડ (B)}: એક્સિયલ, E-ફીલ્ડને લંબ
\item
  \textbf{ક્રોસ્ડ ફીલ્ડ્સ}: સાયક્લોઇડલ ઇલેક્ટ્રોન મોશન બનાવે છે
\end{itemize}

\textbf{ઇલેક્ટ્રોન મોશન એનાલિસિસ:}

\begin{center}
\textbf{Mermaid Diagram (Code)}
\begin{verbatim}
{Shaded}
{Highlighting}[]
graph LR
    A[Electron Emission] {-{-}{} B[Cyclotron Motion]}
    B {-{-}{} C[Spiral Path]}
    C {-{-}{} D[Energy Transfer]}
    D {-{-}{} E[RF Oscillation]}
{Highlighting}
{Shaded}
\end{verbatim}
\end{center}

\textbf{ઓપરેટિંગ કન્ડિશન્સ}:

\begin{itemize}
\tightlist
\item
  \textbf{કટઓફ કન્ડિશન}: E/B = v\_drift
\item
  \textbf{સિંક્રોનિઝમ}: ઇલેક્ટ્રોન ડ્રિફ્ટ વેલોસિટી ફેઝ વેલોસિટી સાથે મેચ થાય
\item
  \textbf{હલ કટઓફ}: ઓપરેશન માટે લઘુત્તમ મેગ્નેટિક ફીલ્ડ
\end{itemize}

\textbf{રેઝોનન્ટ કેવિટીઝ}:

\begin{itemize}
\tightlist
\item
  \textbf{π-મોડ ઓપરેશન}: અલ્ટરનેટ કેવિટીમાં વિરુદ્ધ ફેઝ
\item
  \textbf{ફ્રીક્વન્સી}: f = c/(2\sqrtLC) કેવિટી રેઝોનન્સ માટે
\item
  \textbf{મોડ સેપરેશન}: મોડ કોમ્પીટિશન અટકાવે છે
\end{itemize}

\textbf{પર્ફોર્મન્સ લક્ષણો:}

{\def\LTcaptype{none} % do not increment counter
\begin{longtable}[]{@{}ll@{}}
\toprule\noalign{}
પરિમાપ & સામાન્ય મૂલ્ય \\
\midrule\noalign{}
\endhead
\bottomrule\noalign{}
\endlastfoot
કાર્યક્ષમતા & 60-80\% \\
પાવર આઉટપુટ & 10 kW - 10 MW \\
ફ્રીક્વન્સી & 1-100 GHz \\
પલ્સ/CW & બંને મોડ્સ \\
\end{longtable}
}

\textbf{ફાયદાઓ}:

\begin{itemize}
\tightlist
\item
  \textbf{ઉચ્ચ કાર્યક્ષમતા} અન્ય ટ્યુબ્સ સાથે સરખામણીમાં
\item
  \textbf{ઉચ્ચ પાવર ક્ષમતા}
\item
  \textbf{કોમ્પેક્ટ સ્ટ્રક્ચર}
\item
  \textbf{સારી ફ્રીક્વન્સી સ્થિરતા}
\end{itemize}

\textbf{ઉપયોગો}:

\begin{itemize}
\tightlist
\item
  \textbf{રડાર ટ્રાન્સમિટર}
\item
  \textbf{માઇક્રોવેવ ઓવન}
\item
  \textbf{ઇન્ડસ્ટ્રિયલ હીટિંગ}
\item
  \textbf{ઇલેક્ટ્રોનિક વોરફેર}
\end{itemize}

\textbf{યાદાશ્ત સૂત્ર}: ``મેગ્નેટ્રોન મેગ્નેટિક મોશન દ્વારા માઇક્રોવેવ બનાવે''

\end{solutionbox}
\subsection*{પ્રશ્ન 3(અ) વિકલ્પ [3
ગુણ]}\label{uxaaauxab0uxab6uxaa8-3uxa85-uxab5uxa95uxab2uxaaa-3-uxa97uxaa3}

\textbf{TWT (ટ્રાવેલિંગ વેવ ટ્યુબ)નું એમ્પ્લિફાયર તરીકે કાર્ય સમજાવો.}

\begin{solutionbox}

\textbf{TWT સ્ટ્રક્ચર:}

\begin{center}
\textbf{Mermaid Diagram (Code)}
\begin{verbatim}
{Shaded}
{Highlighting}[]
graph LR
    A[Electron Gun] {-{-}{} B[Helix]}
    B {-{-}{} C[Collector]}
    D[RF Input] {-{-}{} B}
    B {-{-}{} E[RF Output]}
{Highlighting}
{Shaded}
\end{verbatim}
\end{center}

\textbf{એમ્પ્લિફિકેશન પ્રક્રિયા}:

\begin{itemize}
\tightlist
\item
  \textbf{ઇલેક્ટ્રોન બીમ} હેલિક્સ એક્સિસ સાથે ચાલે છે
\item
  \textbf{RF સિગ્નલ} હેલિક્સ સાથે પ્રસારિત થાય છે (સ્લો વેવ સ્ટ્રક્ચર)
\item
  \textbf{વેલોસિટી સિંક્રોનિઝમ}: v\_electron \approx v\_RF
\item
  \textbf{એનર્જી ટ્રાન્સફર} DC બીમથી RF વેવમાં
\end{itemize}

\textbf{ગેઇન મેકેનિઝમ}:

\begin{itemize}
\tightlist
\item
  \textbf{બંચિંગ}: RF ફીલ્ડ ઇલેક્ટ્રોન વેલોસિટી મોડ્યુલેટ કરે છે
\item
  \textbf{ઇન્ડ્યુસ્ડ કરંટ}: બંચ્ડ ઇલેક્ટ્રોન હેલિક્સમાં RF કરંટ ઇન્ડ્યુસ કરે છે
\item
  \textbf{પ્રોગ્રેસિવ એમ્પ્લિફિકેશન} હેલિક્સ લંબાઇ સાથે
\end{itemize}

\textbf{યાદાશ્ત સૂત્ર}: ``ટ્રાવેલિંગ વેવ એનર્જી ટ્રાન્સફર કરે''

\end{solutionbox}
\subsection*{પ્રશ્ન 3(બ) વિકલ્પ [4
ગુણ]}\label{uxaaauxab0uxab6uxaa8-3uxaac-uxab5uxa95uxab2uxaaa-4-uxa97uxaa3}

\textbf{માઇક્રોવેવ ફ્રીક્વન્સી માટે ઓછો પાવર માપવા માટે બોલોમીટર પદ્ધતિ સમજાવો.}

\begin{solutionbox}

\textbf{સિદ્ધાંત}: બોलોમીટર રેઝિસ્ટિવ એલિમેન્ટમાં \textbf{તાપમાન વૃદ્ધિ} ડિટેક્ટ
કરીને માઇક્રોવેવ પાવર માપે છે.

\textbf{બોલોમીટર પ્રકારો}:

\begin{itemize}
\tightlist
\item
  \textbf{થર્મિસ્ટર}: નેગેટિવ ટેમ્પરેચર કોઇફિશન્ટ
\item
  \textbf{બેરેટર}: પોઝિટિવ ટેમ્પરેચર કોઇફિશન્ટ
\end{itemize}

\textbf{સર્કિટ આકૃતિ:}

\begin{verbatim}
    RF Power {-{-}{-}{-}{-} [Bolometer] {-}{-}{-}{-}{-} Temperature}
         |              |               Change
         |              |                 |
    DC Bridge {-{-}{-}{-}{-}{-}{-}{-}{-}●{-}{-}{-}{-}{-}{-}{-}{-}{-} DC Voltmeter}
\end{verbatim}

\textbf{માપન પ્રક્રિયા}:

\begin{itemize}
\tightlist
\item
  \textbf{સ્ટેપ 1}: ફક્ત DC પાવર સાથે બ્રિજ બેલેન્સ કરો
\item
  \textbf{સ્ટેપ 2}: RF પાવર લગાવો, બ્રિજ અનબેલેન્સ નોંધો
\item
  \textbf{સ્ટેપ 3}: બ્રિજ ફરીથી બેલેન્સ કરવા DC પાવર ઘટાડો
\item
  \textbf{સ્ટેપ 4}: RF પાવર = DC પાવરમાં ઘટાડો
\end{itemize}

\textbf{ફાયદાઓ}:

\begin{itemize}
\tightlist
\item
  \textbf{ઉચ્ચ સેન્સિટિવિટી} (µW થી mW રેન્જ)
\item
  \textbf{સ્ક્વેર લો રિસ્પોન્સ}
\item
  \textbf{બ્રોડબેન્ડ ઓપરેશન}
\end{itemize}

\textbf{યાદાશ્ત સૂત્ર}: ``બોલોમીટર બર્ન, બ્રિજ બેલેન્સ''

\end{solutionbox}
\subsection*{પ્રશ્ન 3(ક) વિકલ્પ [7
ગુણ]}\label{uxaaauxab0uxab6uxaa8-3uxa95-uxab5uxa95uxab2uxaaa-7-uxa97uxaa3}

\textbf{બ્લોક ડાયાગ્રામની મદદથી ફ્રીક્વન્સી અને તરંગલંબાઇ માપન પદ્ધતિ સમજાવો.}

\begin{solutionbox}

\textbf{ફ્રીક્વન્સી માપન - ડાયરેક્ટ પદ્ધતિ:}

\begin{center}
\textbf{Mermaid Diagram (Code)}
\begin{verbatim}
{Shaded}
{Highlighting}[]
graph LR
    A[Microwave Source] {-{-}{} B[Frequency Counter]}
    B {-{-}{} C[Digital Display]}
    D[Reference Oscillator] {-{-}{} B}
{Highlighting}
{Shaded}
\end{verbatim}
\end{center}

\textbf{ફ્રીક્વન્સી માપન - હેટરોડાઇન પદ્ધતિ:}

\begin{center}
\textbf{Mermaid Diagram (Code)}
\begin{verbatim}
{Shaded}
{Highlighting}[]
graph LR
    A[Unknown Frequency] {-{-}{} B[Mixer]}
    C[Local Oscillator] {-{-}{} B}
    B {-{-}{} D[IF Amplifier]}
    D {-{-}{} E[Frequency Counter]}
{Highlighting}
{Shaded}
\end{verbatim}
\end{center}

\textbf{તરંગલંબાઇ માપન - સ્લોટેડ લાઇન પદ્ધતિ:}

\textbf{સેટઅપ આકૃતિ:}

\begin{verbatim}
Microwave {-{-}|{-}{-}{-}{-}{-}|====|{-}{-}{-}{-}{-}|{-}{-} Slotted Line {-}{-}|{-}{-} Load}
Source      |  Isolator  |                     |
            |            |                     |
         Attenuator   Detector              Movable
                                           Probe
\end{verbatim}

\textbf{માપન પ્રક્રિયા:}

\textbf{ફ્રી સ્પેસ તરંગલંબાઇ (λ_{0}):}

\begin{itemize}
\tightlist
\item
  \textbf{સ્ટેપ 1}: મેચ્ડ લોડ કનેક્ટ કરો, ફ્રીક્વન્સી માપો
\item
  \textbf{સ્ટેપ 2}: λ_{0} = c/f ગણો
\end{itemize}

\textbf{ગાઇડેડ તરંગલંબાઇ (λ\_g):}

\begin{itemize}
\tightlist
\item
  \textbf{સ્ટેપ 1}: શોર્ટ સર્કિટ કનેક્ટ કરો, બે સતત મિનિમા શોધો
\item
  \textbf{સ્ટેપ 2}: λ\_g = 2 \times મિનિમા વચ્ચેનું અંતર
\item
  \textbf{સ્ટેપ 3}: ચકાસો: λ\_g = λ_{0}/\sqrt[1-(λ_{0}/λ\_c)^{2}]
\end{itemize}

\textbf{કટ-ઓફ તરંગલંબાઇ (λ\_c):}

\begin{itemize}
\tightlist
\item
  \textbf{પદ્ધતિ 1}: વેવગાઇડ પરિમાણોથી: λ\_c = 2a (TE_{1}_{0} માટે)
\item
  \textbf{પદ્ધતિ 2}: λ_{0} અને λ\_g થી: λ\_c = λ_{0}/\sqrt[1-(λ_{0}/λ\_g)^{2}]
\end{itemize}

\textbf{માપન કોષ્ટક:}

{\def\LTcaptype{none} % do not increment counter
\begin{longtable}[]{@{}lll@{}}
\toprule\noalign{}
પરિમાપ & પદ્ધતિ & ચોકસાઈ \\
\midrule\noalign{}
\endhead
\bottomrule\noalign{}
\endlastfoot
ફ્રીક્વન્સી & ડાયરેક્ટ કાઉન્ટિંગ & \pm0.01\% \\
λ_{0} & f થી ગણતરી & \pm0.01\% \\
λ\_g & સ્લોટેડ લાઇન & \pm1\% \\
λ\_c & ગણતરી/માપન & \pm2\% \\
\end{longtable}
}

\textbf{દરેક પદ્ધતિના ફાયદાઓ:}

\begin{itemize}
\tightlist
\item
  \textbf{ડાયરેક્ટ પદ્ધતિ}: ઉચ્ચ ચોકસાઈ, સરળ
\item
  \textbf{હેટરોડાઇન પદ્ધતિ}: વિસ્તૃત ફ્રીક્વન્સી રેન્જ
\item
  \textbf{સ્લોટેડ લાઇન}: ગાઇડેડ પરિમાપો સીધું માપે છે
\end{itemize}

\textbf{ભૂલના સ્ત્રોતો:}

\begin{itemize}
\tightlist
\item
  \textbf{પ્રોબ કપલિંગ} વેરિયેશન
\item
  \textbf{ટેમ્પરેચર અસર} પરિમાણો પર
\item
  \textbf{ડિટેક્ટર નોન-લિનિયરિટી}
\item
  \textbf{સ્ટેન્ડિંગ વેવ} ડિસ્ટર્બન્સ
\end{itemize}

\textbf{ઉપયોગો:}

\begin{itemize}
\tightlist
\item
  \textbf{વેવગાઇડ કેરેક્ટરાઇઝેશન}
\item
  \textbf{મટિરિયલ પ્રોપર્ટી માપન}
\item
  \textbf{એન્ટેના ટેસ્ટિંગ}
\item
  \textbf{કોમ્પોનન્ટ વેરિફિકેશન}
\end{itemize}

\textbf{યાદાશ્ત સૂત્ર}: ``ફ્રીક્વન્સી પહેલા, તરંગલંબાઇ માપન સાથે''

\end{solutionbox}
\subsection*{પ્રશ્ન 4(અ) [3
ગુણ]}\label{uxaaauxab0uxab6uxaa8-4uxa85-3-uxa97uxaa3}

\textbf{માઇક્રોવેવ ફ્રીક્વન્સી માટે વેક્યૂમ ટ્યુબની ફ્રીક્વન્સી મર્યાદાઓ જણાવો.}

\begin{solutionbox}

\textbf{ફ્રીક્વન્સી મર્યાદાઓ:}

\begin{itemize}
\tightlist
\item
  \textbf{ટ્રાન્ઝિટ ટાઇમ અસર}: ઇલેક્ટ્રોન ટ્રાન્ઝિટ ટાઇમ RF પીરિયડ સાથે સરખાવાય
\item
  \textbf{ઇન્ટર-ઇલેક્ટ્રોડ કેપેસિટન્સ}: ઉચ્ચ ફ્રીક્વન્સીએ ગેઇન ઘટાડે છે\\
\item
  \textbf{લીડ ઇન્ડક્ટન્સ}: પેરાસિટિક ઇન્ડક્ટન્સ પર્ફોર્મન્સ મર્યાદિત કરે છે
\item
  \textbf{સ્કિન અસર}: કરંટ કન્સન્ટ્રેશન અસરકારક કંડક્ટન્સ ઘટાડે છે
\end{itemize}

\textbf{મર્યાદિત કરતા પરિબળો કોષ્ટક:}

{\def\LTcaptype{none} % do not increment counter
\begin{longtable}[]{@{}lll@{}}
\toprule\noalign{}
પરિબળ & અસર & ફ્રીક્વન્સી ઇમ્પેક્ટ \\
\midrule\noalign{}
\endhead
\bottomrule\noalign{}
\endlastfoot
ટ્રાન્ઝિટ ટાઇમ & ફેઝ વિલંબ & f \textless{} 1/(2πτ) \\
કેપેસિટન્સ & રિએક્ટન્સ લોડિંગ & ગેઇન ∝ 1/f \\
ઇન્ડક્ટન્સ & રેઝોનન્સ અસર & સ્ટેબિલિટી ઇશ્યુ \\
સ્કિન અસર & વધારો પ્રતિકાર & કાર્યક્ષમતા ↓ \\
\end{longtable}
}

\textbf{ઉકેલો}:

\begin{itemize}
\tightlist
\item
  \textbf{ઇલેક્ટ્રોડ સ્પેસિંગ ઘટાડો}
\item
  \textbf{વિશેષ જ્યોમેટ્રીનો ઉપયોગ}
\item
  \textbf{માઇક્રોવેવ ટ્યુબ્સ વાપરો} (ક્લિસ્ટ્રોન, મેગ્નેટ્રોન)
\end{itemize}

\textbf{યાદાશ્ત સૂત્ર}: ``ટ્રાન્ઝિટ ટાઇમ પરંપરાગત ટ્યુબ્સને તકલીફ''

\end{solutionbox}
\subsection*{પ્રશ્ન 4(બ) [4
ગુણ]}\label{uxaaauxab0uxab6uxaa8-4uxaac-4-uxa97uxaa3}

\textbf{IMPATT ડાયોડમાં નેગેટિવ રેઝિસ્ટન્સ અસર સમજાવો.}

\begin{solutionbox}

\textbf{IMPATT સ્ટ્રક્ચર:}

\begin{verbatim}
P+ |{-{-}| I |{-}{-}| P |{-}{-}| N+ |}
   {-{-}|{-}{-}{-}{-}{-}{-}|{-}{-}{-}{-}{-}{-}|{-}{-}}
   Avalanche  Drift
   Region     Region
\end{verbatim}

\textbf{નેગેટિવ રેઝિસ્ટન્સ મેકેનિઝમ:}

\textbf{બે-સ્ટેપ પ્રક્રિયા:}

\begin{enumerate}
\tightlist
\item
  \textbf{ઇમ્પેક્ટ આયોનાઇઝેશન}: ઉચ્ચ ફીલ્ડ ઇલેક્ટ્રોન-હોલ પેર બનાવે છે
\item
  \textbf{ટ્રાન્ઝિટ ટાઇમ વિલંબ}: કેરિયર ડિપ્લીશન રીજન પાર ડ્રિફ્ટ કરે છે
\end{enumerate}

\textbf{ફેઝ સંબંધો}:

\begin{itemize}
\tightlist
\item
  \textbf{કરંટ}: વોલ્ટેજ કરતા 90^\circ (એવેલાન્ચ વિલંબ) + 90^\circ (ટ્રાન્ઝિટ વિલંબ) = 180^\circ
  પાછળ
\item
  \textbf{પરિણામ}: I = -G*V (નેગેટિવ કંડક્ટન્સ)
\end{itemize}

\textbf{ઓપરેટિંગ સાયકલ:}

\begin{center}
\textbf{Mermaid Diagram (Code)}
\begin{verbatim}
{Shaded}
{Highlighting}[]
graph LR
    A[High Field] {-{-}{} B[Avalanche]}
    B {-{-}{} C[Carrier Generation]}
    C {-{-}{} D[Transit Delay]}
    D {-{-}{} E[Current Peak]}
    E {-{-}{} A}
{Highlighting}
{Shaded}
\end{verbatim}
\end{center}

\textbf{મુખ્ય પરિમાપો}:

\begin{itemize}
\tightlist
\item
  \textbf{એવેલાન્ચ ફ્રીક્વન્સી}: f\_a = v\_s/(2W\_a)
\item
  \textbf{ટ્રાન્ઝિટ ફ્રીક્વન્સી}: f\_t = v\_d/(2W\_d)
\item
  \textbf{ઓપ્ટિમમ ફ્રીક્વન્સી}: f\_0 = 1/(2π\sqrtL*\textbar C\_negative\textbar)
\end{itemize}

\textbf{યાદાશ્ત સૂત્ર}: ``ઇમ્પેક્ટ આયોનાઇઝેશન, ટ્રાન્ઝિટ ટાઇમ = નેગેટિવ રેઝિસ્ટન્સ''

\end{solutionbox}
\subsection*{પ્રશ્ન 4(ક) [7
ગુણ]}\label{uxaaauxab0uxab6uxaa8-4uxa95-7-uxa97uxaa3}

\textbf{ટનલ ડાયોડનો સિદ્ધાંત, ટનલિંગ ઘટના અને કોઈપણ એક એપ્લિકેશન સમજાવો.}

\begin{solutionbox}

\textbf{સિદ્ધાંત}: ટનલ ડાયોડ \textbf{ક્વાન્ટમ મેકેનિકલ ટનલિંગ} અસર પર કાર્ય કરે છે
બહુ ભારે ડોપ્ડ p-n જંક્શનમાં પાતળા પોટેન્શિયલ બેરિયર દ્વારા.

\textbf{એનર્જી બેન્ડ ડાયાગ્રામ:}

\begin{verbatim}
Forward Bias States:

State 1 (Low bias):    State 2 (Peak):      State 3 (Valley):
   P side | N side        P side | N side        P side | N side
    \_\_\_   |  \_\_\_           \_\_\_   |  \_\_\_           \_\_\_   |  \_\_\_
   |   |  | |   |         |   |  | |   |         |   |  | |   |
   |\_\_\_|  | |\_\_\_|         |\_\_\_| /| |\_\_\_|         |\_\_\_| /| |\_\_\_|
          |               Tunneling              No tunnel
       Tunneling                               
\end{verbatim}

\textbf{I-V લક્ષણો:}

\begin{verbatim}
Current ↑
        |    
     Ip |●     
        |  ●    
        |    ●   Forward region
        |      ●  
        |        ●
     Iv |         ●\_\_\_\_
        |                ●
        |                  ●
        |\_\_\_\_\_\_\_\_\_\_\_\_\_\_\_\_\_\_\_\_\_\_\_\_ Voltage
        0   Vp    Vv    Vf
        
    Peak    Valley  Forward
    point   point   region
\end{verbatim}

\textbf{ટનલિંગ ઘટના:}

\textbf{ક્વાન્ટમ મેકેનિક્સ}: ઇલેક્ટ્રોન પોટેન્શિયલ બેરિયર પાર કરી શકે છે ભલે તેમની
એનર્જી બેરિયર હાઇટ કરતા ઓછી હોય.

\textbf{ટનલિંગ પ્રોબેબિલિટી}: T = exp(-2\sqrt(2m\emph{φ}d^{2})/ħ) જ્યાં:

\begin{itemize}
\tightlist
\item
  m = ઇલેક્ટ્રોન માસ
\item
  φ = બેરિયર હાઇટ\\
\item
  d = બેરિયર વિડ્થ
\item
  ħ = રિડ્યુસ્ડ પ્લાન્ક કોન્સ્ટન્ટ
\end{itemize}

\textbf{ઓપરેટિંગ રીજન:}

\begin{itemize}
\tightlist
\item
  \textbf{ટનલિંગ રીજન} (0 થી Vp): વોલ્ટેજ સાથે કરંટ વધે છે
\item
  \textbf{નેગેટિવ રેઝિસ્ટન્સ} (Vp થી Vv): વધતા વોલ્ટેજ સાથે કરંટ ઘટે છે
\item
  \textbf{ફોરવર્ડ બાયાસ} (\textgreater Vv): સામાન્ય ડાયોડ વર્તન
\end{itemize}

\textbf{મુખ્ય પરિમાપો કોષ્ટક:}

{\def\LTcaptype{none} % do not increment counter
\begin{longtable}[]{@{}lll@{}}
\toprule\noalign{}
પરિમાપ & પ્રતીક & સામાન્ય મૂલ્ય \\
\midrule\noalign{}
\endhead
\bottomrule\noalign{}
\endlastfoot
પીક કરંટ & Ip & 1-100 mA \\
પીક વોલ્ટેજ & Vp & 50-100 mV \\
વેલી કરંટ & Iv & 0.1*Ip \\
વેલી વોલ્ટેજ & Vv & 300-500 mV \\
\end{longtable}
}

\textbf{એપ્લિકેશન - હાઇ ફ્રીક્વન્સી ઓસિલેટર:}

\textbf{સર્કિટ આકૃતિ:}

\begin{verbatim}
    +Vcc
      |
      R  Bias resistor
      |
      ●{-{-}{-}L{-}{-}{-}●{-}{-}{-}Output}
      |       |
   Tunnel     C
   Diode      |
      |       |
    ──┴──   ──┴──
     GND     GND
\end{verbatim}

\textbf{ઓસિલેટર ઓપરેશન:}

\begin{itemize}
\tightlist
\item
  \textbf{બાયાસ પોઇન્ટ}: નેગેટિવ રેઝિસ્ટન્સ રીજનમાં સેટ કરવામાં આવે છે
\item
  \textbf{ટેન્ક સર્કિટ}: LC ઓસિલેશન ફ્રીક્વન્સી નક્કી કરે છે
\item
  \textbf{કન્ડિશન}: \textbar R\_negative\textbar{} \textgreater{}
  R\_positive ઓસિલેશન માટે
\item
  \textbf{ફ્રીક્વન્સી}: f = 1/(2π\sqrtLC)
\end{itemize}

\textbf{ફાયદાઓ:}

\begin{itemize}
\tightlist
\item
  \textbf{અલ્ટ્રા-હાઇ ફ્રીક્વન્સી} ઓપરેશન (100 GHz સુધી)
\item
  \textbf{લો નોઇઝ} ફિગર
\item
  \textbf{ઝડપી સ્વિચિંગ} (પિકોસેકન્ડ રેન્જ)
\item
  \textbf{લો પાવર કન્ઝમ્પશન}
\item
  \textbf{ટેમ્પરેચર સ્ટેબલ}
\end{itemize}

\textbf{ઉપયોગો:}

\begin{itemize}
\tightlist
\item
  \textbf{માઇક્રોવેવ ઓસિલેટર}
\item
  \textbf{હાઇ-સ્પીડ સ્વિચ}
\item
  \textbf{માઇક્રોવેવ એમ્પ્લિફાયર}
\item
  \textbf{ફ્રીક્વન્સી કન્વર્ટર}
\item
  \textbf{કોમ્પ્યુટર મેમરી સર્કિટ}
\end{itemize}

\textbf{મર્યાદાઓ:}

\begin{itemize}
\tightlist
\item
  \textbf{લો પાવર હેન્ડલિંગ}
\item
  \textbf{ક્રિટિકલ બાયાસ રિક્વાયરમેન્ટ}
\item
  \textbf{મર્યાદિત ટેમ્પરેચર રેન્જ}
\item
  \textbf{મોંઘું મેન્યુફેક્ચરિંગ}
\end{itemize}

\textbf{યાદાશ્ત સૂત્ર}: ``ટનલ થ્રુ, નેગેટિવ ગ્રો, ઓસિલેટર ફ્લો''

\end{solutionbox}
\subsection*{પ્રશ્ન 4(અ) વિકલ્પ [3
ગુણ]}\label{uxaaauxab0uxab6uxaa8-4uxa85-uxab5uxa95uxab2uxaaa-3-uxa97uxaa3}

\textbf{માઇક્રોવેવ રેડિએશનને કારણે જોખમો સમજાવો.}

\begin{solutionbox}

\textbf{જોખમના પ્રકારો:}

\textbf{HERP (હેઝાર્ડ ઓફ ઇલેક્ટ્રોમેગ્નેટિક રેડિએશન ટુ પર્સનેલ):}

\begin{itemize}
\tightlist
\item
  \textbf{થર્મલ અસર}: 41^\circC ઉપર ટિશ્યુ હીટિંગ
\item
  \textbf{નોન-થર્મલ અસર}: લો પાવર લેવલ પર સેલ્યુલર ડેમેજ
\item
  \textbf{ક્યુમ્યુલેટિવ અસર}: લાંબા ગાળાના એક્સપોઝર રિસ્ક
\end{itemize}

\textbf{HERO (હેઝાર્ડ ઓફ ઇલેક્ટ્રોમેગ્નેટિક રેડિએશન ટુ ઓર્ડનન્સ):}

\begin{itemize}
\tightlist
\item
  \textbf{પ્રીમેચ્યુર ઇગ્નિશન}: RF એનર્જી વિસ્ફોટક ઉપકરણોને ટ્રિગર કરી શકે છે
\item
  \textbf{ફ્યુઅલ ઇગ્નિશન}: ફ્યુઅલ વેપરનું માઇક્રોવેવ હીટિંગ
\item
  \textbf{ઇલેક્ટ્રોનિક ઇન્ટરફેરન્સ}: કંટ્રોલ સિસ્ટમમાં વિક્ષેપ
\end{itemize}

\textbf{HERF (હેઝાર્ડ ઓફ ઇલેક્ટ્રોમેગ્નેટિક રેડિએશન ટુ ફ્યુઅલ્સ):}

\begin{itemize}
\tightlist
\item
  \textbf{ફ્યુઅલ હીટિંગ}: હાઇડ્રોકાર્બન ફ્યુઅલનું ડાઇઇલેક્ટ્રિક હીટિંગ
\item
  \textbf{સ્ટેટિક ડિસ્ચાર્જ}: ફ્યુઅલ સિસ્ટમમાં RF-ઇન્ડ્યુસ્ડ સ્પાર્કિંગ
\item
  \textbf{વેપર ઇગ્નિશન}: ફ્યુઅલ-એર મિક્સચરનું હીટિંગ
\end{itemize}

\textbf{સેફ્ટી ગાઇડલાઇન કોષ્ટક:}

{\def\LTcaptype{none} % do not increment counter
\begin{longtable}[]{@{}llll@{}}
\toprule\noalign{}
એક્સપોઝર લેવલ & પાવર ડેન્સિટી & અવધિ & અસર \\
\midrule\noalign{}
\endhead
\bottomrule\noalign{}
\endlastfoot
સેફ & \textless10 mW/cm^{2} & 8 કલાક & કોઈ અસર નથી \\
સાવધાન & 10-100 mW/cm^{2} & મર્યાદિત & શક્ય હીટિંગ \\
જોખમ & \textgreater100 mW/cm^{2} & ટાળો & ટિશ્યુ ડેમેજ \\
\end{longtable}
}

\textbf{યાદાશ્ત સૂત્ર}: ``HERP-HERO-HERF = હેલ્થ-એક્સ્પ્લોસિવ-ફ્યુઅલ રિસ્ક''

\end{solutionbox}
\subsection*{પ્રશ્ન 4(બ) વિકલ્પ [4
ગુણ]}\label{uxaaauxab0uxab6uxaa8-4uxaac-uxab5uxa95uxab2uxaaa-4-uxa97uxaa3}

\textbf{પેરામેટ્રિક એમ્પ્લિફાયરમાં ડીજનરેટ અને નોન-ડીજનરેટ મોડ સમજાવો.}

\begin{solutionbox}

\textbf{પેરામેટ્રિક એમ્પ્લિફાયર સિદ્ધાંત}: \textbf{ટાઇમ-વેરિંગ રિએક્ટન્સ} નો ઉપયોગ
કરીને પમ્પથી સિગ્નલમાં એનર્જી ટ્રાન્સફર કરે છે.

\textbf{મોડ વર્ગીકરણ:}

\textbf{નોન-ડીજનરેટ મોડ:}

\begin{itemize}
\tightlist
\item
  \textbf{ત્રણ ફ્રીક્વન્સી}: f\_s (સિગ્નલ), f\_i (આઇડલર), f\_p (પમ્પ)
\item
  \textbf{ફ્રીક્વન્સી સંબંધ}: f\_p = f\_s + f\_i
\item
  \textbf{બે અલગ સર્કિટ} સિગ્નલ અને આઇડલર માટે
\item
  \textbf{ઉચ્ચ ગેઇન} પરંતુ વધારે જટિલ
\end{itemize}

\textbf{ડીજનરેટ મોડ:}

\begin{itemize}
\tightlist
\item
  \textbf{બે ફ્રીક્વન્સી}: f\_s (સિગ્નલ), f\_p (પમ્પ)\\
\item
  \textbf{ફ્રીક્વન્સી સંબંધ}: f\_p = 2f\_s
\item
  \textbf{સિંગલ રેઝોનન્ટ સર્કિટ}
\item
  \textbf{સરળ ડિઝાઇન} પરંતુ ઓછો ગેઇન
\end{itemize}

\textbf{સરખામણી કોષ્ટક:}

{\def\LTcaptype{none} % do not increment counter
\begin{longtable}[]{@{}lll@{}}
\toprule\noalign{}
પરિમાપ & નોન-ડીજનરેટ & ડીજનરેટ \\
\midrule\noalign{}
\endhead
\bottomrule\noalign{}
\endlastfoot
ફ્રીક્વન્સી & 3 (fs, fi, fp) & 2 (fs, fp) \\
સર્કિટ & અલગ & સંયુક્ત \\
ગેઇન & ઉચ્ચ & ઓછો \\
જટિલતા & વધારે & ઓછી \\
બેન્ડવિડ્થ & સાંકડો & વિશાળ \\
\end{longtable}
}

\textbf{એનર્જી ટ્રાન્સફર:}

\begin{center}
\textbf{Mermaid Diagram (Code)}
\begin{verbatim}
{Shaded}
{Highlighting}[]
graph LR
    A[Pump Power] {-{-}{} B[Variable Reactance]}
    B {-{-}{} C[Signal Amplification]}
    D[Idler] {-.{-}{} B}
{Highlighting}
{Shaded}
\end{verbatim}
\end{center}

\textbf{યાદાશ્ત સૂત્ર}: ``નોન-ડીજનરેટ = નોટ-સિંગલ, ડીજનરેટ = ડબલ્ડ-ફ્રીક્વન્સી''

\end{solutionbox}
\subsection*{પ્રશ્ન 4(ક) વિકલ્પ [7
ગુણ]}\label{uxaaauxab0uxab6uxaa8-4uxa95-uxab5uxa95uxab2uxaaa-7-uxa97uxaa3}

\textbf{ગન ડાયોડમાં સિદ્ધાંત અને ગન અસર સમજાવો. ગન ડાયોડને ઓસિલેટર તરીકે પણ
સમજાવો.}

\begin{solutionbox}

\textbf{ગન અસર સિદ્ધાંત}: કોમ્પાઉન્ડ સેમિકંડક્ટર (GaAs, InP) માં
\textbf{ટ્રાન્સફર્ડ ઇલેક્ટ્રોન અસર} પર આધારિત.

\textbf{એનર્જી બેન્ડ સ્ટ્રક્ચર:}

\begin{verbatim}
Energy ↑
       |     Upper valley
       |    /
       |   /  ΔE = 0.36 eV
       |  /
       |\_/\_\_\_\_\_\_\_ Lower valley
              |
              | k (momentum)
        Γ valley   L valley
\end{verbatim}

\textbf{ગન અસર મેકેનિઝમ:}

\textbf{ડિફરન્શિયલ મોબિલિટી:}

\begin{itemize}
\tightlist
\item
  \textbf{લો ફીલ્ડ} (\textless3 kV/cm): ઇલેક્ટ્રોન Γ વેલીમાં (હાઇ મોબિલિટી)
\item
  \textbf{હાઇ ફીલ્ડ} (\textgreater3 kV/cm): ઇલેક્ટ્રોન L વેલીમાં ટ્રાન્સફર (લો
  મોબિલિટી)
\item
  \textbf{પરિણામ}: નેગેટિવ ડિફરન્શિયલ મોબિલિટી (NDM)
\end{itemize}

\textbf{ડોમેઇન ફોર્મેશન:}

\begin{center}
\textbf{Mermaid Diagram (Code)}
\begin{verbatim}
{Shaded}
{Highlighting}[]
graph LR
    A[Uniform Field] {-{-}{} B[Instability]}
    B {-{-}{} C[Domain Nucleation]}
    C {-{-}{} D[Domain Growth]}
    D {-{-}{} E[Domain Transit]}
    E {-{-}{} F[Domain Collection]}
    F {-{-}{} A}
{Highlighting}
{Shaded}
\end{verbatim}
\end{center}

\textbf{કરંટ-વોલ્ટેજ લક્ષણો:}

\begin{verbatim}
Current ↑
        |
    I\_p |●
        | ●
        |  ●
        |   ●\_\_\_\_\_ NDM region
        |        ●
        |         ●
        |\_\_\_\_\_\_\_\_\_\_●\_\_\_\_\_\_\_\_\_ Voltage
        0    V\_th    V\_s
        
    Threshold  Sustaining
    voltage    voltage
\end{verbatim}

\textbf{ગન ડાયોડ ઓસિલેટર:}

\textbf{બેસિક કન્ફિગરેશન:}

\begin{verbatim}
    +V\_bias
      |
      R  Bias resistor  
      |
    ──●──── RF Output
      |
   [Gunn]   Gunn diode
   Diode   
      |
    ──┴──── Ground
     GND
\end{verbatim}

\textbf{ઓસિલેટર મોડ્સ:}

\textbf{ટ્રાન્ઝિટ ટાઇમ મોડ:}

\begin{itemize}
\tightlist
\item
  \textbf{ડોમેઇન ફોર્મેશન} કેથોડ પર
\item
  \textbf{ડોમેઇન ટ્રાન્ઝિટ} એક્ટિવ રીજન પાર\\
\item
  \textbf{કરંટ પલ્સ} જ્યારે ડોમેઇન એનોડ પર પહોંચે
\item
  \textbf{ફ્રીક્વન્સી}: f = v\_d/L (જ્યાં v\_d = ડ્રિફ્ટ વેલોસિટી, L = લંબાઇ)
\end{itemize}

\textbf{ક્વેન્ચ્ડ ડોમેઇન મોડ:}

\begin{itemize}
\tightlist
\item
  \textbf{રેઝોનન્ટ સર્કિટ} ટ્રાન્ઝિટ પહેલા ડોમેઇન ક્વેન્ચ કરે છે
\item
  \textbf{ઉચ્ચ ફ્રીક્વન્સી} ઓપરેશન શક્ય
\item
  \textbf{કાર્યક્ષમતા}: 5-20\%
\end{itemize}

\textbf{LSA (લિમિટેડ સ્પેસ-ચાર્જ એક્યુમ્યુલેશન) મોડ:}

\begin{itemize}
\tightlist
\item
  \textbf{હાઇ ફ્રીક્વન્સી} ડોમેઇન ફોર્મેશન અટકાવે છે
\item
  \textbf{યુનિફોર્મ ફીલ્ડ} જાળવવામાં આવે છે
\item
  \textbf{ઉચ્ચ કાર્યક્ષમતા}: 10-25\%
\end{itemize}

\textbf{પર્ફોર્મન્સ પરિમાપો:}

{\def\LTcaptype{none} % do not increment counter
\begin{longtable}[]{@{}lll@{}}
\toprule\noalign{}
પરિમાપ & મૂલ્ય & એકમ \\
\midrule\noalign{}
\endhead
\bottomrule\noalign{}
\endlastfoot
ફ્રીક્વન્સી રેન્જ & 1-100 & GHz \\
પાવર આઉટપુટ & 1 mW-10 W & - \\
કાર્યક્ષમતા & 5-25 & \% \\
નોઇઝ ફિગર & 35-50 & dB \\
\end{longtable}
}

\textbf{ફાયદાઓ:}

\begin{itemize}
\tightlist
\item
  \textbf{સરળ સ્ટ્રક્ચર} - કોઈ બાહ્ય રેઝોનેટરની જરૂર નથી
\item
  \textbf{બ્રોડબેન્ડ ટ્યુનિંગ} ક્ષમતા
\item
  \textbf{લો નોઇઝ} માઇક્રોવેવ ફ્રીક્વન્સીએ
\item
  \textbf{વિશ્વસનીય ઓપરેશન}
\end{itemize}

\textbf{ઉપયોગો:}

\begin{itemize}
\tightlist
\item
  \textbf{લોકલ ઓસિલેટર} રિસીવરમાં
\item
  \textbf{CW રડાર ટ્રાન્સમિટર}\\
\item
  \textbf{માઇક્રોવેવ કોમ્યુનિકેશન સિસ્ટમ}
\item
  \textbf{ટેસ્ટ ઇક્વિપમેન્ટ સિગ્નલ સોર્સ}
\end{itemize}

\textbf{યાદાશ્ત સૂત્ર}: ``ગન ગેલિયમ-આર્સેનાઇડ દ્વારા ગોઇંગ મેળવે''

\end{solutionbox}
\subsection*{પ્રશ્ન 5(અ) [3
ગુણ]}\label{uxaaauxab0uxab6uxaa8-5uxa85-3-uxa97uxaa3}

\textbf{બ્લોક ડાયાગ્રામની મદદથી મૂળભૂત રડાર સિસ્ટમના કાર્ય સિદ્ધાંતને સમજાવો.}

\begin{solutionbox}

\textbf{રડાર સિદ્ધાંત}: \textbf{રેડિયો ડિટેક્શન એન્ડ રેન્જિંગ} - RF પલ્સ ટ્રાન્સમિટ
કરે છે અને ટાર્ગેટથી પ્રતિબિંબિત સિગ્નલ ડિટેક્ટ કરે છે.

\textbf{બેસિક રડાર બ્લોક ડાયાગ્રામ:}

\begin{center}
\textbf{Mermaid Diagram (Code)}
\begin{verbatim}
{Shaded}
{Highlighting}[]
graph LR
    A[Master Oscillator] {-{-}{} B[Modulator]}
    B {-{-}{} C[Power Amplifier]}
    C {-{-}{} D[Duplexer]}
    D {-{-}{} E[Antenna]}
    E {-{-}{} F[Target]}
    F {-{-}{} E}
    E {-{-}{} D}
    D {-{-}{} G[Receiver]}
    G {-{-}{} H[Signal Processor]}
    H {-{-}{} I[Display]}
    J[Timing Control] {-{-}{} B}
{Highlighting}
{Shaded}
\end{verbatim}
\end{center}

\textbf{કાર્યપ્રણાલી સિદ્ધાંત}:

\begin{itemize}
\tightlist
\item
  \textbf{ટ્રાન્સમિશન}: ટાર્ગેટ તરફ હાઇ પાવર RF પલ્સ ટ્રાન્સમિટ કરવામાં આવે છે
\item
  \textbf{પ્રસારણ}: EM તરંગ પ્રકાશની ગતિ (c) થી ચાલે છે
\item
  \textbf{પ્રતિબિંબ}: ટાર્ગેટ એનર્જીનો ભાગ પાછો રડાર તરફ પ્રતિબિંબિત કરે છે
\item
  \textbf{રિસેપ્શન}: પ્રતિબિંબિત સિગ્નલ પ્રાપ્ત અને પ્રોસેસ કરવામાં આવે છે
\item
  \textbf{રેન્જ કેલ્ક્યુલેશન}: R = (c \times t)/2
\end{itemize}

\textbf{મુખ્ય પરિમાપો}:

\begin{itemize}
\tightlist
\item
  \textbf{પલ્સ વિડ્થ}: τ = 0.1 થી 10 μs
\item
  \textbf{પલ્સ રિપીટિશન ફ્રીક્વન્સી}: PRF = 100 Hz થી 10 kHz
\item
  \textbf{પીક પાવર}: 1 kW થી 10 MW
\end{itemize}

\textbf{યાદાશ્ત સૂત્ર}: ``રડાર રાઉન્ડ-ટ્રિપ રિફ્લેક્શન દ્વારા રેન્જ માપે''

\end{solutionbox}
\subsection*{પ્રશ્ન 5(બ) [4
ગુણ]}\label{uxaaauxab0uxab6uxaa8-5uxaac-4-uxa97uxaa3}

\textbf{યોગ્ય આકૃતિની મદદથી A-સ્કોપ ડિસ્પ્લે પદ્ધતિ સમજાવો.}

\begin{solutionbox}

\textbf{A-સ્કોપ ડિસ્પ્લે}: પ્રાપ્ત ઇકોઝનો \textbf{એમ્પ્લિટ્યુડ વર્સિસ ટાઇમ} સંબંધ
દર્શાવે છે.

\textbf{A-સ્કોપ પ્રેઝન્ટેશન:}

\begin{verbatim}
Amplitude ↑
          |
          |    ●  Target echo
          |   /|{  }
     Main |  / | {  }
    pulse | /  |  {  }
          |/   |   {}
          |    |    {\_\_\_}
          |\_\_\_\_|\_\_\_\_\_\_\_\_\_{\_\_\_\_\_\_ Time}
          0    |         
               |
           2R/c (Range)
           
    
     clutter clutter
\end{verbatim}

\textbf{ડિસ્પ્લે કોમ્પોનન્ટ્સ}:

\begin{itemize}
\tightlist
\item
  \textbf{મેઇન પલ્સ}: પ્રારંભિક ટ્રાન્સમિટેડ પલ્સ (રેફરન્સ)
\item
  \textbf{ગ્રાઉન્ડ ક્લટર}: નજીકના ટેરેઇનથી પ્રતિબિંબ
\item
  \textbf{સી ક્લટર}: દરિયાની સપાટીથી પ્રતિબિંબ\\
\item
  \textbf{ટાર્ગેટ ઇકો}: વાસ્તવિક ટાર્ગેટથી પ્રતિબિંબ
\item
  \textbf{નોઇઝ}: રેન્ડમ બેકગ્રાઉન્ડ સિગ્નલ
\end{itemize}

\textbf{રેન્જ માપન}:

\begin{itemize}
\tightlist
\item
  \textbf{હોરિઝોન્ટલ એક્સિસ}: ટાઇમ (રેન્જના પ્રમાણસર)
\item
  \textbf{વર્ટિકલ એક્સિસ}: સિગ્નલ એમ્પ્લિટ્યુડ
\item
  \textbf{રેન્જ ફોર્મ્યુલા}: R = (c \times t)/2
\end{itemize}

\textbf{ઉપયોગો}:

\begin{itemize}
\tightlist
\item
  \textbf{એર ટ્રાફિક કંટ્રોલ}
\item
  \textbf{હાઇટ ફાઇન્ડિંગ રડાર}\\
\item
  \textbf{રેન્જ માપન}
\item
  \textbf{સિગ્નલ એનાલિસિસ}
\end{itemize}

\textbf{યાદાશ્ત સૂત્ર}: ``A-સ્કોપ ટાઇમ એક્સિસ સાથે એમ્પ્લિટ્યુડ દર્શાવે''

\end{solutionbox}
\subsection*{પ્રશ્ન 5(ક) [7
ગુણ]}\label{uxaaauxab0uxab6uxaa8-5uxa95-7-uxa97uxaa3}

\textbf{ડોપ્લર અસર અને બ્લોક ડાયાગ્રામની મદદથી MTI (મૂવિંગ ટાર્ગેટ ઇન્ડિકેટર) રડાર
સિસ્ટમની કામગીરી સમજાવો.}

\begin{solutionbox}

\textbf{ડોપ્લર અસર}: રડાર અને ટાર્ગેટ વચ્ચે સાપેક્ષ ગતિ હોય ત્યારે ફ્રીક્વન્સી શિફ્ટ
થાય છે.

\textbf{ડોપ્લર ફ્રીક્વન્સી શિફ્ટ:} f\_d = (2 \times v\_r \times f\_0)/c

જ્યાં:

\begin{itemize}
\tightlist
\item
  f\_d = ડોપ્લર ફ્રીક્વન્સી શિફ્ટ
\item
  v\_r = ટાર્ગેટની રેડિયલ વેલોસિટી
\item
  f\_0 = ટ્રાન્સમિટેડ ફ્રીક્વન્સી\\
\item
  c = પ્રકાશની ગતિ
\end{itemize}

\textbf{ડોપ્લર શિફ્ટ કેસિસ}:

\begin{itemize}
\tightlist
\item
  \textbf{પાસ આવતું ટાર્ગેટ}: f\_d \textgreater{} 0 (પોઝિટિવ શિફ્ટ)
\item
  \textbf{દૂર જતું ટાર્ગેટ}: f\_d \textless{} 0 (નેગેટિવ શિફ્ટ)
\item
  \textbf{સ્થિર ટાર્ગેટ}: f\_d = 0 (કોઈ શિફ્ટ નથી)
\end{itemize}

\textbf{MTI રડાર બ્લોક ડાયાગ્રામ:}

\begin{center}
\textbf{Mermaid Diagram (Code)}
\begin{verbatim}
{Shaded}
{Highlighting}[]
graph LR
    A[Transmitter] {-{-}{} B[Duplexer]}
    B {-{-}{} C[Antenna]}
    C {-{-}{} D[Target]}
    D {-{-}{} C}
    C {-{-}{} B}
    B {-{-}{} E[Receiver]}
    F[STALO] {-{-}{} G[Mixer 1]}
    H[COHO] {-{-}{} I[Phase Detector]}
    E {-{-}{} G}
    G {-{-}{} J[IF Amplifier]}
    J {-{-}{} K[Mixer 2]}
    H {-{-}{} K}
    K {-{-}{} L[Video Amplifier]}
    L {-{-}{} M[Delay Line]}
    M {-{-}{} N[Subtractor]}
    L {-{-}{} N}
    N {-{-}{} O[Display]}
    P[Sync] {-{-}{} A}
    P {-{-}{} H}
{Highlighting}
{Shaded}
\end{verbatim}
\end{center}

\textbf{MTI સિસ્ટમ કોમ્પોનન્ટ્સ:}

\textbf{STALO (સ્ટેબલ લોકલ ઓસિલેટર):}

\begin{itemize}
\tightlist
\item
  \textbf{ફ્રીક્વન્સી}: ટ્રાન્સમિટેડ ફ્રીક્વન્સીની નજીક
\item
  \textbf{સ્ટેબિલિટી}: ઉચ્ચ ફ્રીક્વન્સી સ્થિરતા જરૂરી
\item
  \textbf{ફંક્શન}: ફર્સ્ટ મિક્સર LO
\end{itemize}

\textbf{COHO (કોહેરન્ટ ઓસિલેટર):}

\begin{itemize}
\tightlist
\item
  \textbf{ફેઝ રેફરન્સ}: ફેઝ કોહેરન્સ જાળવે છે
\item
  \textbf{સિંક્રોનાઇઝેશન}: ટ્રાન્સમિટર ફેઝ સાથે લોક્ડ
\item
  \textbf{ફંક્શન}: સેકન્ડ મિક્સર LO અને ફેઝ રેફરન્સ
\end{itemize}

\textbf{MTI પ્રોસેસિંગ:}

\begin{itemize}
\tightlist
\item
  \textbf{ડિલે લાઇન}: અગાઉના પલ્સ ઇકો સ્ટોર કરે છે
\item
  \textbf{સબટ્રેક્ટર}: વર્તમાનમાંથી અગાઉનો પલ્સ બાદ કરે છે
\item
  \textbf{પરિણામ}: સ્થિર ટાર્ગેટ કેન્સલ, મૂવિંગ ટાર્ગેટ એન્હાન્સ
\end{itemize}

\textbf{MTI ટ્રાન્સફર ફંક્શન:}

\begin{verbatim}
|H(f)| ↑
       |     
    1.0|     ●●●     ●●●     ●●●
       |    ●   ●   ●   ●   ●   ●
    0.5|   ●     ● ●     ● ●     ●
       |  ●       ●       ●       ●
     0 |\_●\_\_\_\_\_\_\_●\_\_\_\_\_\_\_●\_\_\_\_\_\_\_●\_\_\_ fd
       0  PRF/4  PRF/2  3PRF/4  PRF
       
        Blind speeds 
\end{verbatim}

\textbf{બ્લાઇન્ડ સ્પીડ્સ}: ચોક્કસ વેલોસિટી ધરાવતા ટાર્ગેટ સ્થિર દેખાય છે: v\_blind
= (n \times λ \times PRF)/2 (જ્યાં

n = 1,2,3\ldots)


\textbf{પર્ફોર્મન્સ સુધારણા:}

\textbf{મલ્ટિ-પલ્સ MTI:}

\begin{itemize}
\tightlist
\item
  \textbf{મલ્ટિપલ ડિલે લાઇન} વધુ સારા ક્લટર રિજેક્શન માટે
\item
  \textbf{સ્ટેગર્ડ PRF} બ્લાઇન્ડ સ્પીડ ઘટાડવા માટે
\item
  \textbf{વેટેડ કોઇફિશન્ટ} ઓપ્ટિમમ રિસ્પોન્સ માટે
\end{itemize}

\textbf{ક્લટર મેપ:}

\begin{itemize}
\tightlist
\item
  \textbf{ડિજિટલ મેમરી} ક્લટર પેટર્ન સ્ટોર કરે છે
\item
  \textbf{એડાપ્ટિવ થ્રેશહોલ્ડ} લોકલ ક્લટર લેવલ અનુસાર એડજસ્ટ કરે છે
\item
  \textbf{ઓટોમેટિક અપડેટ} ધીમા ક્લટર ચેન્જને ટ્રેક કરે છે
\end{itemize}

\textbf{MTI પર્ફોર્મન્સ મેટ્રિક્સ:}

{\def\LTcaptype{none} % do not increment counter
\begin{longtable}[]{@{}ll@{}}
\toprule\noalign{}
પરિમાપ & સામાન્ય મૂલ્ય \\
\midrule\noalign{}
\endhead
\bottomrule\noalign{}
\endlastfoot
ક્લટર એટેન્યુએશન & 30-60 dB \\
મિનિમમ ડિટેક્ટેબલ વેલોસિટી & 1-10 m/s \\
બ્લાઇન્ડ સ્પીડ & λ\timesPRF/2 \\
ઇમ્પ્રુવમેન્ટ ફેક્ટર & 20-40 dB \\
\end{longtable}
}

\textbf{ફાયદાઓ:}

\begin{itemize}
\tightlist
\item
  \textbf{ક્લટર સપ્રેશન}: સ્થિર ક્લટર દૂર કરે છે
\item
  \textbf{મૂવિંગ ટાર્ગેટ એમ્ફેસિસ}: મૂવિંગ ટાર્ગેટ વધારે છે
\item
  \textbf{ઓટોમેટિક ઓપરેશન}: ઓપરેટરનો વર્કલોડ ઘટાડે છે
\end{itemize}

\textbf{મર્યાદાઓ:}

\begin{itemize}
\tightlist
\item
  \textbf{બ્લાઇન્ડ સ્પીડ્સ}: કેટલીક વેલોસિટી ડિટેક્ટ કરી શકાતી નથી
\item
  \textbf{ટેન્જેન્શિયલ ટાર્ગેટ}: કોઈ રેડિયલ કોમ્પોનન્ટ નથી
\item
  \textbf{વેધર અસર}: વરસાદ/બરફ ટાર્ગેટને માસ્ક કરી શકે છે
\end{itemize}

\textbf{ઉપયોગો:}

\begin{itemize}
\tightlist
\item
  \textbf{એર ટ્રાફિક કંટ્રોલ}: એરક્રાફ્ટને ગ્રાઉન્ડ ક્લટરથી અલગ કરે છે
\item
  \textbf{વેધર રડાર}: પ્રેસિપિટેશન મૂવમેન્ટ ડિટેક્ટ કરે છે\\
\item
  \textbf{મિલિટરી સર્વેલન્સ}: મૂવિંગ વેહિકલ ડિટેક્ટ કરે છે
\item
  \textbf{મરીન રડાર}: સી ક્લટર ઘટાડે છે
\end{itemize}

\textbf{યાદાશ્ત સૂત્ર}: ``MTI ડોપ્લર ડિફરન્સ દ્વારા ટાર્ગેટ આઇડેન્ટિફાઇ કરે''

\end{solutionbox}
\subsection*{પ્રશ્ન 5(અ) વિકલ્પ [3
ગુણ]}\label{uxaaauxab0uxab6uxaa8-5uxa85-uxab5uxa95uxab2uxaaa-3-uxa97uxaa3}

\textbf{વ્યાખ્યા આપો: a) બ્લાઇન્ડ સ્પીડ, અને b) MUR}

\begin{solutionbox}

\textbf{બ્લાઇન્ડ સ્પીડ:}

\begin{itemize}
\tightlist
\item
  \textbf{વ્યાખ્યા}: ટાર્ગેટની રેડિયલ વેલોસિટી કે જે MTI રડારમાં ઝીરો ડોપ્લર શિફ્ટ
  ઉત્પન્ન કરે છે
\item
  \textbf{ફોર્મ્યુલા}: v\_blind = (n \times λ \times PRF)/2
\item
  \textbf{કારણ}: ટાર્ગેટ મૂવમેન્ટ પલ્સ રિપીટિશન સાથે સિંક્રોનાઇઝ્ડ
\item
  \textbf{પરિણામ}: મૂવિંગ ટાર્ગેટ સ્થિર દેખાય છે
\end{itemize}

\textbf{MUR (મેક્સિમમ અનએમ્બિગ્યુઅસ રેન્જ):}

\begin{itemize}
\tightlist
\item
  \textbf{વ્યાખ્યા}: મહત્તમ રેન્જ કે જ્યાં રેન્જ એમ્બિગ્યુટી વિના ટાર્ગેટ ડિટેક્ટ કરી
  શકાય
\item
  \textbf{ફોર્મ્યુલા}: R\_max = (c \times PRT)/2 = c/(2 \times PRF)
\item
  \textbf{મર્યાદા}: આગળનો પલ્સ ઇકો પાછો આવે તે પહેલા ટ્રાન્સમિટ થાય છે
\item
  \textbf{એમ્બિગ્યુટી}: MUR કરતા વધારે ટાર્ગેટ ખોટી રેન્જ પર દેખાય છે
\end{itemize}

\textbf{સંબંધ કોષ્ટક:}

{\def\LTcaptype{none} % do not increment counter
\begin{longtable}[]{@{}lll@{}}
\toprule\noalign{}
પરિમાપ & ફોર્મ્યુલા & એકમ \\
\midrule\noalign{}
\endhead
\bottomrule\noalign{}
\endlastfoot
બ્લાઇન્ડ સ્પીડ & nλPRF/2 & m/s \\
MUR & c/(2\timesPRF) & મીટર \\
PRT & 1/PRF & સેકન્ડ \\
\end{longtable}
}

\textbf{યાદાશ્ત સૂત્ર}: ``બ્લાઇન્ડ સ્પીડ બ્લોક કરે, MUR મેક્સિમમ માપે''

\end{solutionbox}
\subsection*{પ્રશ્ન 5(બ) વિકલ્પ [4
ગુણ]}\label{uxaaauxab0uxab6uxaa8-5uxaac-uxab5uxa95uxab2uxaaa-4-uxa97uxaa3}

\textbf{મહત્તમ રડાર રેન્જને અસર કરતા પરિબળો સમજાવો.}

\begin{solutionbox}

\textbf{રડાર રેન્જ સમીકરણ:} R\_max = [(P\_t \times G^{2} \times λ^{2} \times σ)/(64π^{3} \times
P\_min \times L)]\^{}(1/4)

\textbf{મહત્તમ રેન્જને અસર કરતા પરિબળો:}

\textbf{ટ્રાન્સમિટેડ પાવર (P\_t):}

\begin{itemize}
\tightlist
\item
  \textbf{વધારે પાવર} = વધારે રેન્જ
\item
  \textbf{સંબંધ}: R ∝ P\_t\^{}(1/4)
\item
  \textbf{મર્યાદા}: પીક પાવર ટ્રાન્સમિટર દ્વારા મર્યાદિત
\end{itemize}

\textbf{એન્ટેના ગેઇન (G):}

\begin{itemize}
\tightlist
\item
  \textbf{ડાયરેક્શનલ એન્ટેના} એનર્જી કન્સન્ટ્રેટ કરે છે
\item
  \textbf{સંબંધ}: R ∝ G\^{}(1/2)
\item
  \textbf{ટ્રેડ-ઓફ}: વધારે ગેઇન = સાંકડો બીમવિડ્થ
\end{itemize}

\textbf{તરંગલંબાઇ (λ):}

\begin{itemize}
\tightlist
\item
  \textbf{લો ફ્રીક્વન્સી} = વધુ સારો પ્રસારણ
\item
  \textbf{સંબંધ}: R ∝ λ\^{}(1/2)
\item
  \textbf{વિચારણા}: ફ્રીક્વન્સી સાથે એટમોસ્ફેરિક એબ્સોર્પશન વધે છે
\end{itemize}

\textbf{ટાર્ગેટ ક્રોસ સેક્શન (σ):}

\begin{itemize}
\tightlist
\item
  \textbf{મોટા ટાર્ગેટ} વધારે એનર્જી રિફ્લેક્ટ કરે છે
\item
  \textbf{સંબંધ}: R ∝ σ\^{}(1/4)
\item
  \textbf{વેરિયેશન}: ટાર્ગેટ શેપ, મટિરિયલ, એસ્પેક્ટ એંગલ પર આધાર રાખે છે
\end{itemize}

\textbf{પરિબળો કોષ્ટક:}

{\def\LTcaptype{none} % do not increment counter
\begin{longtable}[]{@{}lll@{}}
\toprule\noalign{}
પરિબળ & રેન્જ પર અસર & સામાન્ય મૂલ્યો \\
\midrule\noalign{}
\endhead
\bottomrule\noalign{}
\endlastfoot
પીક પાવર & R ∝ Pt\^{}0.25 & 1 kW - 10 MW \\
એન્ટેના ગેઇન & R ∝ G\^{}0.5 & 20 - 50 dB \\
ફ્રીક્વન્સી & R ∝ λ\^{}0.5 & 1 - 100 GHz \\
ટાર્ગેટ RCS & R ∝ σ\^{}0.25 & 0.1 - 1000 m^{2} \\
\end{longtable}
}

\textbf{યાદાશ્ત સૂત્ર}: ``પાવર-ગેઇન-લેમ્બડા-સિગ્મા રેન્જ નક્કી કરે''

\end{solutionbox}
\subsection*{પ્રશ્ન 5(ક) વિકલ્પ [7
ગુણ]}\label{uxaaauxab0uxab6uxaa8-5uxa95-uxab5uxa95uxab2uxaaa-7-uxa97uxaa3}

\textbf{પલ્સ્ડ રડાર અને CW ડોપ્લર રડારની સરખામણી કરો.}

\begin{solutionbox}

\textbf{વ્યાપક સરખામણી:}

\textbf{બેસિક સિદ્ધાંત:}

\begin{itemize}
\tightlist
\item
  \textbf{પલ્સ્ડ રડાર}: હાઇ-પાવર પલ્સ ટ્રાન્સમિટ કરે છે, રાઉન્ડ-ટ્રિપ ટાઇમ માપે છે
\item
  \textbf{CW ડોપ્લર}: કન્ટિન્યુઅસ વેવ ટ્રાન્સમિટ કરે છે, ડોપ્લર ફ્રીક્વન્સી શિફ્ટ માપે
  છે
\end{itemize}

\textbf{સિસ્ટમ બ્લોક ડાયાગ્રામ:}

\textbf{પલ્સ્ડ રડાર:}

\begin{center}
\textbf{Mermaid Diagram (Code)}
\begin{verbatim}
{Shaded}
{Highlighting}[]
graph LR
    A[Pulse Generator] {-{-}{} B[Transmitter]}
    B {-{-}{} C[Duplexer]}
    C {-{-}{} D[Antenna]}
    C {-{-}{} E[Receiver]}
    E {-{-}{} F[Display]}
{Highlighting}
{Shaded}
\end{verbatim}
\end{center}

\textbf{CW ડોપ્લર રડાર:}

\begin{center}
\textbf{Mermaid Diagram (Code)}
\begin{verbatim}
{Shaded}
{Highlighting}[]
graph LR
    A[CW Oscillator] {-{-}{} B[Directional Coupler]}
    B {-{-}{} C[Transmit Antenna]}
    D[Receive Antenna] {-{-}{} E[Mixer]}
    B {-{-}{} E}
    E {-{-}{} F[Audio Amplifier]}
    F {-{-}{} G[Display]}
{Highlighting}
{Shaded}
\end{verbatim}
\end{center}

\textbf{વિગતવાર સરખામણી કોષ્ટક:}

{\def\LTcaptype{none} % do not increment counter
\begin{longtable}[]{@{}lll@{}}
\toprule\noalign{}
પરિમાપ & પલ્સ્ડ રડાર & CW ડોપ્લર રડાર \\
\midrule\noalign{}
\endhead
\bottomrule\noalign{}
\endlastfoot
\textbf{ટ્રાન્સમિશન} & હાઇ પાવર પલ્સ & કન્ટિન્યુઅસ લો પાવર \\
\textbf{માહિતી} & રેન્જ + વેલોસિટી & ફક્ત વેલોસિટી \\
\textbf{એન્ટેના} & સિંગલ (ડુપ્લેક્સર) & અલગ Tx/Rx \\
\textbf{રેન્જ ક્ષમતા} & ઉત્તમ & કોઈ નથી (FM-CW સિવાય) \\
\textbf{વેલોસિટી રેઝોલ્યુશન} & મર્યાદિત & ઉત્તમ \\
\textbf{પીક પાવર} & ખૂબ ઉચ્ચ (MW) & લો (mW થી W) \\
\textbf{એવરેજ પાવર} & લો & મધ્યમ \\
\textbf{જટિલતા} & ઉચ્ચ & સરળ \\
\textbf{કિંમત} & મોંઘું & કિફાયતી \\
\textbf{સાઇઝ} & મોટું & કોમ્પેક્ટ \\
\end{longtable}
}

\textbf{પર્ફોર્મન્સ લક્ષણો:}

{\def\LTcaptype{none} % do not increment counter
\begin{longtable}[]{@{}lll@{}}
\toprule\noalign{}
પાસું & પલ્સ્ડ રડાર & CW ડોપ્લર રડાર \\
\midrule\noalign{}
\endhead
\bottomrule\noalign{}
\endlastfoot
\textbf{રેન્જ એક્યુરેસી} & \pm10-100 m & લાગુ નથી \\
\textbf{વેલોસિટી એક્યુરેસી} & \pm1-10 m/s & \pm0.1-1 m/s \\
\textbf{મિનિમમ રેન્જ} & પલ્સ વિડ્થ દ્વારા મર્યાદિત & શૂન્ય \\
\textbf{મેક્સિમમ રેન્જ} & 10-1000 km & 1-50 km \\
\textbf{ક્લટર રિજેક્શન} & મધ્યમ & ઉત્તમ \\
\textbf{વેધર અસર} & મહત્વપૂર્ણ & ન્યૂનતમ \\
\end{longtable}
}

\textbf{ફાયદા અને ગેરફાયદા:}

\textbf{પલ્સ્ડ રડાર ફાયદા:}

\begin{itemize}
\tightlist
\item
  \textbf{રેન્જ માપન} ક્ષમતા
\item
  \textbf{હાઇ પીક પાવર} લાંબી રેન્જ માટે
\item
  \textbf{સિંગલ એન્ટેના} સિસ્ટમ
\item
  \textbf{વેલ-એસ્ટેબ્લિશ્ડ} ટેક્નોલોજી
\end{itemize}

\textbf{પલ્સ્ડ રડાર ગેરફાયદા:}

\begin{itemize}
\tightlist
\item
  \textbf{જટિલ સર્કિટરી} (ડુપ્લેક્સર, ટાઇમિંગ)
\item
  \textbf{ઉચ્ચ કિંમત} અને મેન્ટેનન્સ\\
\item
  \textbf{પાવર સપ્લાય} જરૂરિયાત
\item
  \textbf{બ્લાઇન્ડ રેન્જ} પલ્સ વિડ્થને કારણે
\end{itemize}

\textbf{CW ડોપ્લર ફાયદા:}

\begin{itemize}
\tightlist
\item
  \textbf{સરળ ડિઝાઇન} અને લો કોસ્ટ
\item
  \textbf{ઉત્તમ વેલોસિટી રેઝોલ્યુશન}
\item
  \textbf{કન્ટિન્યુઅસ મોનિટરિંગ}
\item
  \textbf{લો પાવર કન્ઝમ્પશન}
\item
  \textbf{કોમ્પેક્ટ સાઇઝ}
\end{itemize}

\textbf{CW ડોપ્લર ગેરફાયદા:}

\begin{itemize}
\tightlist
\item
  \textbf{કોઈ રેન્જ માહિતી નથી}
\item
  \textbf{અલગ એન્ટેના} જરૂરી
\item
  \textbf{મર્યાદિત રેન્જ} ક્ષમતા
\item
  \textbf{ઇન્ટરફેરન્સ માટે વલ્નરેબલ}
\end{itemize}

\textbf{ઉપયોગો:}

\textbf{પલ્સ્ડ રડાર એપ્લિકેશન:}

\begin{itemize}
\tightlist
\item
  \textbf{એર ટ્રાફિક કંટ્રોલ}
\item
  \textbf{વેધર મોનિટરિંગ}
\item
  \textbf{મિલિટરી સર્વેલન્સ}
\item
  \textbf{મેરિટાઇમ નેવિગેશન}
\item
  \textbf{સેટેલાઇટ ટ્રેકિંગ}
\end{itemize}

\textbf{CW ડોપ્લર એપ્લિકેશન:}

\begin{itemize}
\tightlist
\item
  \textbf{ટ્રાફિક સ્પીડ મોનિટરિંગ}
\item
  \textbf{સ્પોર્ટ્સ રડાર ગન}
\item
  \textbf{બર્ગલર એલાર્મ}
\item
  \textbf{ઓટોમેટિક ડોર ઓપનર}
\item
  \textbf{હાર્ટ રેટ મોનિટરિંગ}
\end{itemize}

\textbf{હાઇબ્રિડ સિસ્ટમ:}

\textbf{પલ્સ ડોપ્લર રડાર:}

\begin{itemize}
\tightlist
\item
  \textbf{બંનેના ફાયદા} કોમ્બાઇન કરે છે
\item
  \textbf{રેન્જ અને વેલોસિટી} માપન
\item
  \textbf{વધારે જટિલતા} પરંતુ વધુ સારું પર્ફોર્મન્સ
\end{itemize}

\textbf{FM-CW રડાર:}

\begin{itemize}
\tightlist
\item
  \textbf{ફ્રીક્વન્સી મોડ્યુલેટેડ} કન્ટિન્યુઅસ વેવ
\item
  \textbf{રેન્જ ક્ષમતા} CW સિસ્ટમમાં ઉમેરાય છે
\item
  \textbf{ઓટોમોટિવ} રડાર એપ્લિકેશનમાં વપરાય છે
\end{itemize}

\textbf{સિલેક્શન ક્રાઇટેરિયા:}

{\def\LTcaptype{none} % do not increment counter
\begin{longtable}[]{@{}lll@{}}
\toprule\noalign{}
જરૂરિયાત & પલ્સ્ડ પસંદ કરો & CW ડોપ્લર પસંદ કરો \\
\midrule\noalign{}
\endhead
\bottomrule\noalign{}
\endlastfoot
રેન્જ માપન જરૂરી & ✓ & ✗ \\
હાઇ વેલોસિટી એક્યુરેસી & ✗ & ✓ \\
લાંબી રેન્જ ઓપરેશન & ✓ & ✗ \\
લો કોસ્ટ જરૂરિયાત & ✗ & ✓ \\
પોર્ટેબલ એપ્લિકેશન & ✗ & ✓ \\
વેધર રડાર & ✓ & ✗ \\
\end{longtable}
}

\textbf{ભવિષ્યના ટ્રેન્ડ:}

\begin{itemize}
\tightlist
\item
  \textbf{ડિજિટલ સિગ્નલ પ્રોસેસિંગ} બંને પ્રકારને સુધારે છે
\item
  \textbf{સોફ્ટવેર-ડિફાઇન્ડ રડાર} લવચીકતા આપે છે
\item
  \textbf{MIMO ટેકનિક} પર્ફોર્મન્સ વધારે છે
\item
  \textbf{અન્ય સેન્સર સાથે ઇન્ટીગ્રેશન}
\end{itemize}

\textbf{યાદાશ્ત સૂત્ર}: ``પલ્સ્ડ પોઝિશન આપે, CW કન્ટિન્યુઅસ-વેલોસિટી આપે''

\end{solutionbox}

\end{document}
