\documentclass{article}

% content/resources/templates/preamble.tex
\usepackage[margin=0.6in]{geometry}
\author{Milav Dabgar}
\usepackage{amsmath,amssymb,amsthm}
\usepackage{booktabs}
\usepackage{multirow}
\usepackage{xcolor}
\usepackage{tcolorbox}
\tcbuselibrary{breakable,skins}
\usepackage[colorlinks=true,linkcolor=blue]{hyperref}
\usepackage{titlesec}
\usepackage{enumitem}
\usepackage{tikz}
\usepackage{pgfplots}
\usepackage{circuitikz}
\usepackage[version=4]{mhchem}
\usepackage{longtable}
\usepackage{array}
\usepackage{float}
\usepackage{caption}
\usepackage{listings}

\lstset{
  basicstyle=\small\ttfamily,
  breaklines=true,
  breakatwhitespace=false,
  postbreak=\mbox{\textcolor{red}{$\hookrightarrow$}\space},
  float=false,
  numbers=left,
  numberstyle=\tiny\color{gray},
  numbersep=10pt,
  xleftmargin=2em,
  keywordstyle=\color{blue},
  commentstyle=\color{green!60!black},
  stringstyle=\color{purple},
  backgroundcolor=\color{gray!5},
  showstringspaces=false,
  tabsize=2,
  captionpos=b,
  keepspaces=true,
  columns=flexible
}

\pgfplotsset{compat=1.18}
\usetikzlibrary{shapes,arrows,positioning,calc,patterns,decorations.pathmorphing,decorations.markings,arrows.meta}

% Color scheme
\definecolor{headcolor}{RGB}{0,102,204}
\definecolor{keycolor}{RGB}{220,20,60}
\definecolor{solutioncolor}{RGB}{34,139,34}
\definecolor{mnemoniccolor}{RGB}{148,0,211}
\definecolor{codecolor}{RGB}{0,0,100}

% Spacing
\setlength{\parskip}{3pt}
\setlist[itemize]{nosep}
\setlist[enumerate]{nosep}

% Title formatting
\titleformat{\section}{\Large\bfseries\color{headcolor}}{\thesection}{1em}{}
\titleformat{\subsection}{\large\bfseries\color{headcolor}}{\thesubsection}{1em}{}

% Pandoc tightlist compatibility
\providecommand{\tightlist}{%
  \setlength{\itemsep}{0pt}\setlength{\parskip}{0pt}}

% Pandoc longtable compatibility
\newcounter{none}
\def\thenone{}


% content/resources/templates/gujarati-boxes.tex
\usepackage{fontspec}
\usepackage{polyglossia}

% Set Gujarati as main language (document is primarily in Gujarati)
% Note: gloss-gujarati.ldf doesn't exist in polyglossia, but it will use hyphenation patterns
\setdefaultlanguage{gujarati}
\setotherlanguage{english}

% Configure Gujarati font properly
% Use Language=Default to prevent polyglossia from trying to add language-specific features
% that don't exist for Gujarati, which causes "empty feature" warnings
\newfontfamily\gujaratifont[Script=Gujarati,AutoFakeBold=2.5,AutoFakeSlant=0.3]{Noto Sans Gujarati}
\setmainfont[Script=Gujarati,AutoFakeBold=2.5,AutoFakeSlant=0.3]{Noto Sans Gujarati}
% Use Noto Sans Gujarati for monospace to support Gujarati in text
\setmonofont[Scale=0.9]{Noto Sans Gujarati}

% Configure English to use the same font
\newfontfamily\englishfont[Script=Gujarati,AutoFakeBold=2.5,AutoFakeSlant=0.3]{Noto Sans Gujarati}

% Translations for polyglossia
\gappto\captionsgujarati{
  \renewcommand{\tablename}{કોષ્ટક}
  \renewcommand{\figurename}{આકૃતિ}
}

% Helper for TikZ nodes to ensure Gujarati font
\newcommand{\gu}[1]{{\gujaratifont #1}}

% Custom environments
\newtcolorbox{solutionbox}{
    breakable,
    enhanced,
    colback=solutioncolor!5!white,
    colframe=solutioncolor!75!black,
    fonttitle=\bfseries,
    title=જવાબ
}

\newtcolorbox{solutionboxnobreak}{
 colback=solutioncolor!5!white,
 colframe=solutioncolor!75!black,
 fonttitle=\bfseries,
 title=જવાબ
}

\newtcolorbox{keyformula}{
 breakable,
 enhanced,
 colback=keycolor!5!white,
 colframe=keycolor!75!black,
 fonttitle=\bfseries,
 title=રાસાયણિક સમીકરણ/સૂત્ર
}

\newtcolorbox{mnemonicbox}{
 breakable,
 enhanced,
 colback=mnemoniccolor!5!white,
 colframe=mnemoniccolor!75!black,
 fonttitle=\bfseries,
 title=મેમરી ટ્રીક
}


% Custom commands for GTU solutions
% This file defines semantic commands for consistent formatting

% Question command with automatic formatting
\newcommand{\question}[2]{%
  \section*{Question #1}%
  \textbf{#2}%
}

% OR question variant
\newcommand{\questionor}[2]{%
  \section*{Question #1 OR}%
  \textbf{#2}%
}

% Proper table environment with caption
\newenvironment{answertable}[1]{%
  \begin{table}[htbp]
  \centering
  \caption{#1}
}{%
  \end{table}
}

% Proper figure environment for diagrams
\newenvironment{answerdiagram}[1]{%
  \begin{figure}[htbp]
  \centering
  \caption{#1}
}{%
  \end{figure}
}

% Semantic markup for key terms
\newcommand{\keyword}[1]{\textbf{#1}}
\newcommand{\code}[1]{\texttt{#1}}
\newcommand{\classname}[1]{\texttt{#1}}
\newcommand{\methodname}[1]{\texttt{#1}}

% Proper quotation marks
\newcommand{\mnemonic}[1]{``#1''}


\usepackage{fontspec}
\newfontfamily\gujaratifont[Script=Gujarati]{Gujarati Sangam MN}
\renewcommand{\gu}[1]{{\gujaratifont #1}}

\title{\gu{માઇક્રોવેવ અને રડાર કમ્યુનિકેશન (4351103) - શિયાળા 2024 સોલ્યુશન}}
\date{\gu{27 નવેમ્બર, 2024}}

\begin{document}
\maketitle

\questionmarks{1(a)}{3}{\gu{ટ્રાન્સમિશન લાઇન અને વેવગાઇડ વચ્ચે સરખામણી કરો.}}

\begin{solutionbox}
\textbf{\gu{સરખામણી:}}

\begin{tabulary}{\linewidth}{|L|L|L|}
\hline
\textbf{\gu{પેરામીટર}} & \textbf{\gu{ટ્રાન્સમિશન લાઇન}} & \textbf{\gu{વેવગાઇડ}} \\ \hline
\keyword{\gu{ફ્રીક્વન્સી રેન્જ}} & \gu{નીચી થી મધ્યમ ફ્રીક્વન્સી} & \gu{ઉચ્ચ ફ્રીક્વન્સી (1 GHz થી વધુ)} \\ \hline
\keyword{\gu{સ્ટ્રક્ચર}} & \gu{બે કે વધુ કંડક્ટર} & \gu{એક હોલો કંડક્ટર} \\ \hline
\keyword{\gu{પ્રોપેગેશન મોડ}} & \gu{TEM મોડ} & \gu{TE અને TM મોડ} \\ \hline
\keyword{\gu{પાવર હેન્ડલિંગ}} & \gu{મર્યાદિત પાવર કેપેસિટી} & \gu{ઉચ્ચ પાવર હેન્ડલિંગ ક્ષમતા} \\ \hline
\keyword{\gu{લોસેસ}} & \gu{ઉચ્ચ ફ્રીક્વન્સીએ વધુ નુકસાન} & \gu{માઇક્રોવેવ ફ્રીક્વન્સીએ ઓછું નુકસાન} \\ \hline
\end{tabulary}
\end{solutionbox}

\begin{mnemonicbox}
\mnemonic{\gu{"વેવ્સ વધુ સારી રીતે ટ્રાવેલ કરે છે"}}
\end{mnemonicbox}

\questionmarks{1(b)}{4}{\gu{નીચેની વ્યાખ્યા આપો: (1) લોસલેસ લાઇન (2) VSWR (3) STUB (4) રિફ્લેક્શન કોઓફીશિઅન્ટ}}

\begin{solutionbox}
\textbf{\gu{વ્યાખ્યાઓ:}}

\begin{itemize}
    \item \textbf{\gu{લોસલેસ લાઇન}}: \gu{શૂન્ય રેઝિસ્ટન્સ અને કંડક્ટન્સ ધરાવતી ટ્રાન્સમિશન લાઇન, જેમાં સિગ્નલ ટ્રાન્સમિશન દરમિયાન કોઈ પાવર લોસ નથી.}
    \item \textbf{VSWR (\gu{વોલ્ટેજ સ્ટેન્ડિંગ વેવ રેશિયો})}: \gu{ટ્રાન્સમિશન લાઇન પર મેક્સિમમ અને મિનિમમ વોલ્ટેજનો રેશિયો, જે ઇમ્પીડન્સ મિસમેચ દર્શાવે છે.}
    \item \textbf{STUB}: \gu{ઇમ્પીડન્સ મેચિંગ માટે મુખ્ય લાઇન સાથે જોડાયેલી ટ્રાન્સમિશન લાઇનનો ટૂંકો ભાગ.}
    \item \textbf{\gu{રિફ્લેક્શન કોઓફીશિઅન્ટ}}: \gu{ટ્રાન્સમિશન લાઇન પર કોઈપણ બિંદુએ રિફ્લેક્ટેડ વેવ અને ઇન્સિડન્ટ વેવના એમ્પ્લિટ્યુડનો રેશિયો.}
\end{itemize}
\end{solutionbox}

\begin{mnemonicbox}
\mnemonic{\gu{"લાઇટ વોલ્યુમ સ્ટે રિફ્લેક્ટેડ"}}
\end{mnemonicbox}

\questionmarks{1(c)}{7}{\gu{આઇસોલેટર અને સર્ક્યુલેટર આકૃતિની મદદથી સમજાવો.}}

\begin{solutionbox}
\textbf{\gu{આઇસોલેટર:}}
\begin{enumerate}
    \item \textbf{\gu{કાર્ય}}: \gu{માત્ર એક દિશામાં સિગ્નલ ફ્લોની પરવાનગી આપે છે.}
    \item \textbf{\gu{કન્સ્ટ્રક્શન}}: \gu{મેગ્નેટિક બાયાસ સાથે ફેરાઇટ મટેરિયલનો ઉપયોગ.}
    \item \textbf{\gu{ઉપયોગ}}: \gu{રિફ્લેક્શનથી સોર્સનું રક્ષણ કરે છે.}
\end{enumerate}

\begin{answerdiagram}{Isolator Symbol}
\begin{tikzpicture}[auto, node distance=2cm]
    \node [gtu block] (iso) {\gu{આઇસોલેટર}};
    \node [gtu input, left=of iso] (p1) {\gu{પોર્ટ 1}};
    \node [gtu output, right=of iso] (p2) {\gu{પોર્ટ 2}};
    
    \draw [gtu arrow] (p1) -- (iso);
    \draw [gtu arrow] (iso) -- (p2);
    
    % Reflection blocked
    \draw [->, red, dashed, bend left] (p2) to node[above] {X} (iso);
\end{tikzpicture}
\end{answerdiagram}

\textbf{\gu{સર્ક્યુલેટર:}}
\begin{enumerate}
    \item \textbf{\gu{કાર્ય}}: \gu{ત્રણ કે ચાર પોર્ટ વચ્ચે સર્ક્યુલર પેટર્નમાં સિગ્નલ રૂટ કરે છે.}
    \item \textbf{\gu{કન્સ્ટ્રક્શન}}: \gu{ફેરાઇટ મટેરિયલ સાથે Y-જંક્શન.}
    \item \textbf{\gu{ઉપયોગ}}: \gu{રડાર સિસ્ટમમાં ડુપ્લેક્સર તરીકે.}
\end{enumerate}

\begin{answerdiagram}{Circulator Symbol}
\begin{tikzpicture}
    \node [draw, circle, minimum size=1.5cm, thick] (c) {\gu{સર્ક્યુલેટર}};
    \node [above=1cm of c] (p1) {\gu{પોર્ટ 1}};
    \node [below right=1cm of c] (p2) {\gu{પોર્ટ 2}};
    \node [below left=1cm of c] (p3) {\gu{પોર્ટ 3}};
    
    \draw [->, thick] (p1) -- (c);
    \draw [->, thick] (c) -- (p2);
    \draw [->, thick] (c) -- (p3);
    \draw [->, thick] (c) -- (p1);
    
    \draw [->, thick, blue] (0,0.3) arc (90:-30:0.3);
\end{tikzpicture}
\end{answerdiagram}
\end{solutionbox}

\begin{mnemonicbox}
\mnemonic{\gu{"આઇસોલેટેડ સર્કિટ ફોરવર્ડ ફ્લો"}}
\end{mnemonicbox}

\questionmarks{1(c OR)}{7}{\gu{વેવગાઇડમાં ડોમિનન્ટ મોડ શું છે? 10 સેમી breadth ધરાવતા રેક્ટેન્ગ્યુલર વેવગાઇડ માટે કટ ઓફ વેવલેન્થ શોધો. 2.5 GHz સિગ્નલ propagate થવા માટે ગાઇડ વેવલેન્થ, ગ્રુપ વેલોસિટી, ફેઝ વેલોસિટી અને Z₀ની વેલ્યુ શોધો.}}

\begin{solutionbox}
\textbf{\gu{ડોમિનન્ટ મોડ}:} \gu{વેવગાઇડમાં propagate થઈ શકતો સૌથી નીચો ઓર્ડર મોડ. રેક્ટેન્ગ્યુલર વેવગાઇડ માટે TE$_{10}$ મોડ છે.}

\textbf{\gu{આપેલા ડેટા:}}
\begin{itemize}
    \item Breadth (a) = 10 cm = 0.1 m
    \item Frequency (f) = 2.5 GHz = $2.5 \times 10^9$ Hz
    \item c = $3 \times 10^8$ m/s
\end{itemize}

\textbf{\gu{ગણતરીઓ:}}
\begin{tabulary}{\linewidth}{|L|L|L|}
\hline
\textbf{\gu{પેરામીટર}} & \textbf{\gu{ફોર્મ્યુલા}} & \textbf{\gu{વેલ્યુ}} \\ \hline
\gu{કટ ઓફ વેવલેન્થ} & $\lambda_c = 2a$ & $\lambda_c = 2 \times 0.1 = 0.2$ m \\ \hline
\gu{ફ્રી સ્પેસ વેવલેન્થ} & $\lambda_0 = c/f$ & $\lambda_0 = 0.12$ m \\ \hline
\gu{ગાઇડ વેવલેન્થ} & $\lambda_g = \frac{\lambda_0}{\sqrt{1-(\lambda_0/\lambda_c)^2}}$ & $\lambda_g = 0.133$ m \\ \hline
\gu{ગ્રુપ વેલોસિટી} & $v_g = c\sqrt{1-(\lambda_0/\lambda_c)^2}$ & $v_g = 2.7 \times 10^8$ m/s \\ \hline
\gu{ફેઝ વેલોસિટી} & $v_p = \frac{c}{\sqrt{1-(\lambda_0/\lambda_c)^2}}$ & $v_p = 3.33 \times 10^8$ m/s \\ \hline
\end{tabulary}
\end{solutionbox}

\begin{mnemonicbox}
\mnemonic{\gu{"ડોમિનન્ટ મોડ કેલ્ક્યુલેટ ગાઇડ પેરામીટર"}}
\end{mnemonicbox}

\questionmarks{2(a)}{3}{\gu{સિંગલ સ્ટબ ઇમ્પીડન્સ મેચિંગ શું છે, અને આ કેવી રીતે કાર્ય કરે છે?}}

\begin{solutionbox}
\textbf{\gu{સિંગલ સ્ટબ મેચિંગ}:} \gu{ઇમ્પીડન્સ મેચિંગ માટે ટ્રાન્સમિશન લાઇન સાથે પેરેલલમાં જોડાયેલા એક શોર્ટ-સર્કિટ અથવા ઓપન-સર્કિટ સ્ટબનો ઉપયોગ કરતી ટેકનિક.}

\textbf{\gu{કાર્યસિદ્ધાંત:}}
\begin{itemize}
    \item \textbf{\gu{સ્ટબ રિએક્ટિવ એલિમેન્ટ તરીકે કાર્ય કરે છે}} (\gu{ઇન્ડક્ટિવ અથવા કેપેસિટિવ})
    \item \textbf{\gu{લોડ ઇમ્પીડન્સના રિએક્ટિવ ઘટકને કેન્સલ કરે છે}}
    \item \textbf{\gu{ઇમ્પીડન્સને કેરેક્ટરિસ્ટિક ઇમ્પીડન્સમાં ટ્રાન્સફોર્મ કરે છે}}
\end{itemize}
\end{solutionbox}

\begin{mnemonicbox}
\mnemonic{\gu{"સિંગલ સ્ટબ ટ્રાન્સફોર્મ રિએક્ટન્સ"}}
\end{mnemonicbox}

\questionmarks{2(b)}{4}{\gu{રેક્ટેન્ગ્યુલર અને સર્ક્યુલર વેવગાઇડ વચ્ચે કોઈ પણ ત્રણ તફાવત આપો.}}

\begin{solutionbox}
\textbf{\gu{તફાવત:}}

\begin{tabulary}{\linewidth}{|L|L|L|}
\hline
\textbf{\gu{પેરામીટર}} & \textbf{\gu{રેક્ટેન્ગ્યુલર વેવગાઇડ}} & \textbf{\gu{સર્ક્યુલર વેવગાઇડ}} \\ \hline
\keyword{\gu{ક્રોસ-સેક્શન}} & \gu{લંબચોરસ આકાર} & \gu{વર્તુળાકાર આકાર} \\ \hline
\keyword{\gu{ડોમિનન્ટ મોડ}} & TE$_{10}$ \gu{મોડ} & TE$_{11}$ \gu{મોડ} \\ \hline
\keyword{\gu{ફીલ્ડ પેટર્ન}} & \gu{સરળ ફીલ્ડ વિતરણ} & \gu{જટિલ ફીલ્ડ વિતરણ} \\ \hline
\keyword{\gu{મેન્યુફેક્ચરિંગ}} & \gu{બનાવવામાં સહેલું} & \gu{બનાવવામાં મુશ્કેલ} \\ \hline
\end{tabulary}
\end{solutionbox}

\begin{mnemonicbox}
\mnemonic{\gu{"લંબચોરસ દસ પર ડોમિનેટ કરે" vs "વર્તુળ અગિયાર પર ડોમિનેટ કરે"}}
\end{mnemonicbox}

\questionmarks{2(c)}{7}{\gu{હાઇબ્રિડ રિંગનું બાંધકામ અને કાર્ય આકૃતિ સાથે સમજાવો.}}

\begin{solutionbox}
\textbf{\gu{બાંધકામ:}}
\begin{itemize}
    \item \textbf{\gu{રિંગ સ્ટ્રક્ચર}} \gu{ચાર પોર્ટ સાથે.}
    \item \textbf{\gu{પરિઘ}} = 1.5$\lambda$ (\gu{દોઢ વેવલેન્થ}).
    \item \textbf{\gu{બાજુના પોર્ટ}} $\lambda/4$ \gu{દ્વારા અલગ.}
    \item \textbf{\gu{વિરુદ્ધ પોર્ટ}} 3$\lambda/4$ \gu{દ્વારા અલગ.}
\end{itemize}

\begin{answerdiagram}{Hybrid Ring (Rat-Race Coupler)}
\begin{tikzpicture}[scale=1.5]
    \draw [thick] (0,0) circle (1cm);
    
    \node [draw, circle, fill=white] at (90:1cm)  (p1) {1};
    \node [draw, circle, fill=white] at (180:1cm) (p2) {2};
    \node [draw, circle, fill=white] at (270:1cm) (p3) {3};
    \node [draw, circle, fill=white] at (0:1cm)   (p4) {4};
    
    \node [above] at (p1) {\gu{પોર્ટ 1} ($\Sigma$)};
    \node [left] at (p2) {\gu{પોર્ટ 2}};
    \node [below] at (p3) {\gu{પોર્ટ 3} ($\Delta$)};
    \node [right] at (p4) {\gu{પોર્ટ 4}};
    
    \draw [<->] (10:1.2) arc (10:80:1.2) node [midway, above right] {$\lambda/4$};
    \draw [<->] (100:1.2) arc (100:170:1.2) node [midway, above left] {$\lambda/4$};
    \draw [<->] (190:1.2) arc (190:260:1.2) node [midway, below left] {$\lambda/4$};
    \draw [<->] (280:1.2) arc (280:350:1.2) node [midway, below right] {$3\lambda/4$};
\end{tikzpicture}
\end{answerdiagram}

\textbf{\gu{કાર્ય:}}
\begin{itemize}
    \item \textbf{\gu{પાવર ડિવિઝન}}: \gu{એક પોર્ટનું ઇનપુટ બે બાજુના પોર્ટમાં સમાન રીતે વહેંચાય છે.}
    \item \textbf{\gu{આઇસોલેશન}}: \gu{વિરુદ્ધ પોર્ટને કોઈ પાવર મળતું નથી.}
    \item \textbf{\gu{ફેઝ રિલેશનશિપ}}: \gu{આઉટપુટ પોર્ટ વચ્ચે 180$^\circ$ ફેઝ ડિફરન્સ.}
\end{itemize}

\textbf{\gu{ઉપયોગ}}: \gu{બેલેન્સ્ડ મિક્સર, પાવર કમ્બાઇનર/ડિવાઇડર, એન્ટીના ફીડ.}
\end{solutionbox}

\begin{mnemonicbox}
\mnemonic{\gu{"હાઇબ્રિડ રિંગ પાવર સમાન વહેંચે છે"}}
\end{mnemonicbox}

\orquestionmarks{2(a)}{3}{\gu{માઇક્રોવેવ શું છે? માઇક્રોવેવના કોઈ પણ ચાર ઉપયોગો લખો.}}

\begin{solutionbox}
\textbf{\gu{માઇક્રોવેવ}:} \gu{1 GHz થી 300 GHz સુધીની ફ્રીક્વન્સી રેન્જ ધરાવતા ઇલેક્ટ્રોમેગ્નેટિક વેવ્સ.}

\textbf{\gu{ઉપયોગ:}}
\begin{enumerate}
    \item \textbf{\gu{રડાર સિસ્ટમ}} \gu{ડિટેક્શન અને રેન્જિંગ માટે.}
    \item \textbf{\gu{સેટેલાઇટ કમ્યુનિકેશન}} \gu{લાંબા અંતરના ટ્રાન્સમિશન માટે.}
    \item \textbf{\gu{માઇક્રોવેવ ઓવન}} \gu{ખોરાક ગરમ કરવા માટે.}
    \item \textbf{\gu{મોબાઇલ કમ્યુનિકેશન}} (\gu{સેલ્યુલર નેટવર્ક}).
\end{enumerate}
\end{solutionbox}

\begin{mnemonicbox}
\mnemonic{\gu{"માઇક્રોવેવ રીચ સ્પેસ મોબાઇલ"}}
\end{mnemonicbox}

\orquestionmarks{2(b)}{4}{\gu{કેવિટી રેઝોનેટર પર ટૂંકી નોંધ લખો.}}

\begin{solutionbox}
\textbf{\gu{કેવિટી રેઝોનેટર}:} \gu{ચોક્કસ રેઝોનન્ટ ફ્રીક્વન્સીએ ઇલેક્ટ્રોમેગ્નેટિક એનર્જીને સીમિત કરતું બંધ મેટાલિક સ્ટ્રક્ચર.}

\textbf{\gu{બાંધકામ:}}
\begin{itemize}
    \item \textbf{\gu{ચોક્કસ માપના મેટાલિક એન્ક્લોઝર.}}
    \item \textbf{\gu{ઉચ્ચ Q ફેક્ટર}} (\gu{ઓછું નુકસાન}).
    \item \textbf{\gu{રેઝોનન્ટ ફ્રીક્વન્સી}} \gu{કેવિટીના માપ પર આધાર રાખે છે.}
\end{itemize}

\textbf{\gu{પ્રકાર}:} \gu{રેક્ટેન્ગ્યુલર કેવિટી, સિલિન્ડ્રિકલ કેવિટી, સ્ફેરિકલ કેવિટી.}

\textbf{\gu{ઉપયોગ}:} \gu{ફ્રીક્વન્સી મીટર, ઓસીલેટર સર્કિટ, ફિલ્ટર સર્કિટ.}
\end{solutionbox}

\begin{mnemonicbox}
\mnemonic{\gu{"કેવિટી રેઝોનેટ હાઇ ક્વોલિટી"}}
\end{mnemonicbox}

\orquestionmarks{2(c)}{7}{\gu{મેજિક ટીને આકૃતિની મદદથી સમજાવો. તે આઇસોલેટર તરીકે કઈ રીતે કાર્ય કરે છે?}}

\begin{solutionbox}
\textbf{\gu{મેજિક ટી બાંધકામ:}}
\begin{itemize}
    \item \textbf{E-\gu{પ્લેન ટી}} \gu{અને} \textbf{H-\gu{પ્લેન ટી}} \gu{સંયુક્ત.}
    \item \textbf{\gu{ચાર પોર્ટ}}: E-\gu{આર્મ}, H-\gu{આર્મ}, \gu{અને બે સાઇડ આર્મ.}
    \item \textbf{E-\gu{આર્મ}} H-\gu{આર્મ પર વર્ટિકલ.}
\end{itemize}

\begin{answerdiagram}{Magic Tee}
\begin{tikzpicture}[scale=1]
    % H-arm
    \draw [thick] (-2,0) -- (2,0);
    \draw [thick] (-2,1) -- (2,1);
    \node [left] at (-2,0.5) {\gu{પોર્ટ 1}};
    \node [right] at (2,0.5) {\gu{પોર્ટ 2}};
    \node [below] at (0,0) {\gu{સાઇડ આર્મ્સ}};

    % E-arm
    \draw [thick] (-0.5,1) -- (-0.5,2.5);
    \draw [thick] (0.5,1) -- (0.5,2.5);
    \draw [thick] (-0.5,2.5) -- (0.5,2.5);
    \node [above] at (0,2.5) {\gu{પોર્ટ 4} (E-\gu{આર્મ})};

    % H-arm vertical part
    \draw [thick] (-0.5,0) -- (-0.5,-1.5);
    \draw [thick] (0.5,0) -- (0.5,-1.5);
    \node [below] at (0,-1.5) {\gu{પોર્ટ 3} (H-\gu{આર્મ})};
\end{tikzpicture}
\end{answerdiagram}

\textbf{\gu{આઇસોલેટર તરીકે કાર્ય:}}
\begin{itemize}
    \item \textbf{E-\gu{આર્મનું સિગ્નલ}}: \gu{સાઇડ આર્મમાં સમાન રીતે વહેંચાય છે (in-phase).}
    \item \textbf{H-\gu{આર્મનું સિગ્નલ}}: \gu{સાઇડ આર્મમાં સમાન રીતે વહેંચાય છે (out-of-phase).}
    \item \textbf{\gu{આઇસોલેશન}}: E-\gu{આર્મ અને} H-\gu{આર્મ વચ્ચે.}
    \item \textbf{\gu{પર્પેન્ડિક્યુલર આર્મ વચ્ચે કોઈ કપલિંગ નથી.}}
\end{itemize}

\textbf{\gu{ગુણધર્મો}:} \gu{બધા પોર્ટ પર મેચ્ડ, રેસિપ્રોકલ ડિવાઇસ, પાવર ડિવિઝન અને આઇસોલેશન.}
\end{solutionbox}

\begin{mnemonicbox}
\mnemonic{\gu{"મેજિક આઇસોલેટ પર્પેન્ડિક્યુલર આર્મ"}}
\end{mnemonicbox}

\questionmarks{3(a)}{3}{\gu{મેઝરનો કાર્યસિદ્ધાંત વર્ણવો.}}

\begin{solutionbox}
\textbf{MASER (Microwave Amplification by Stimulated Emission of Radiation):}
\begin{enumerate}
    \item \keyword{\gu{પોપ્યુલેશન ઇન્વર્શન}}: \gu{એક્ટિવ મીડિયમમાં બનાવવામાં આવે છે.}
    \item \keyword{\gu{સ્ટિમ્યુલેટેડ એમિશન}}: \gu{કોહેરન્ટ માઇક્રોવેવ પેદા કરે છે.}
    \item \keyword{\gu{એમ્પ્લિફિકેશન}}: \gu{એનર્જી લેવલ ટ્રાન્ઝિશન દ્વારા થાય છે.}
\end{enumerate}

\textbf{\gu{કાર્યસિદ્ધાંત:}}
\gu{પરમાણુ ઉચ્ચ એનર્જી લેવલમાં ઉત્તેજિત થાય છે} $\rightarrow$ \gu{સ્ટિમ્યુલેટેડ ફોટોન એમિશન ટ્રિગર કરે છે} $\rightarrow$ \gu{માઇક્રોવેવ સિગ્નલનું કોહેરન્ટ એમ્પ્લિફિકેશન.}
\end{solutionbox}

\begin{mnemonicbox}
\mnemonic{\gu{"માઇક્રોવેવ એમ્પ્લિફાઇ સ્ટિમ્યુલેટેડ એમિશન રેડિએશન"}}
\end{mnemonicbox}

\questionmarks{3(b)}{4}{\gu{ચાર માઇક્રોવેવ ડાયોડના નામ લખો અને એકનું વર્ણન કરો.}}

\begin{solutionbox}
\textbf{\gu{ચાર માઇક્રોવેવ ડાયોડ:}}
1. GUNN \gu{ડાયોડ}, 2. IMPATT \gu{ડાયોડ}, 3. TRAPATT \gu{ડાયોડ}, 4. PIN \gu{ડાયોડ}.

\textbf{GUNN \gu{ડાયોડ}:}
\begin{itemize}
    \item \textbf{\gu{સિદ્ધાંત}}: GaAs \gu{માં ટ્રાન્સફર્ડ ઇલેક્ટ્રોન એફેક્ટ.}
    \item \textbf{\gu{બાંધકામ}}: \gu{ઓહ્મિક કોન્ટેક્ટ સાથે} N-\gu{ટાઇપ} GaAs.
    \item \textbf{\gu{ઓપરેશન}}: \gu{માઇક્રોવેવ ફ્રીક્વન્સીએ નેગેટિવ રેઝિસ્ટન્સ.}
    \item \textbf{\gu{ઉપયોગ}}: \gu{ઓસીલેટર, એમ્પ્લિફાયર.}
\end{itemize}

\begin{answerdiagram}{Gunn Diode I-V Characteristics}
\begin{tikzpicture}[scale=0.8]
    \draw[->] (0,0) -- (5,0) node[right] {V};
    \draw[->] (0,0) -- (0,4) node[above] {I};
    
    \draw[thick, blue] (0,0) -- (1.5,3) node[above] {\gu{પીક} (A)};
    \draw[thick, blue] (1.5,3) -- (3.5,1.5) node[below] {\gu{વેલી} (B)};
    \draw[thick, blue] (3.5,1.5) -- (5,2);
    
    \node at (2.5, 2.5) {\gu{નેગેટિવ રેઝિસ્ટન્સ}};
    \draw[dotted] (1.5,3) -- (1.5,0) node[below] {$V_p$};
    \draw[dotted] (3.5,1.5) -- (3.5,0) node[below] {$V_v$};
\end{tikzpicture}
\end{answerdiagram}
\end{solutionbox}

\begin{mnemonicbox}
\mnemonic{\gu{"GUNN જનરેટ નેગેટિવ રેઝિસ્ટન્સ"}}
\end{mnemonicbox}

\questionmarks{3(c)}{7}{\gu{મેગ્નેટ્રોન ઓસીલેટરનું નિર્માણ, કાર્યસિદ્ધાંત અને ઉપયોગો સાથે વિસ્તારવાર વર્ણન કરો.}}

\begin{solutionbox}
\textbf{\gu{બાંધકામ:}}
\begin{itemize}
    \item \textbf{\gu{કેન્દ્રમાં સિલિન્ડ્રિકલ કેથોડ.}}
    \item \textbf{\gu{કેથોડની આસપાસ રેઝોનન્ટ કેવિટી સાથે એનોડ.}}
    \item \textbf{\gu{ઇલેક્ટ્રિક ફીલ્ડ પર વર્ટિકલ મજબૂત મેગ્નેટિક ફીલ્ડ.}}
    \item \textbf{\gu{વેવગાઇડ દ્વારા આઉટપુટ કપલિંગ.}}
\end{itemize}

\begin{answerdiagram}{Magnetron Construction}
\begin{tikzpicture}
    % Anode Block
    \draw[thick, fill=gray!20] (0,0) circle (2);
    \draw[fill=white] (0,0) circle (0.8);
    
    % Cavities
    \foreach \a in {0,45,...,315} {
        \draw[fill=white] (\a:1.4) circle (0.3);
        \draw[white, thick] (\a:1.4) -- (\a:0.8);
        \draw (\a:1.25) -- (\a:0.8);
    }
    
    % Cathode
    \draw[fill=black] (0,0) circle (0.2);
    \node at (0,0.5) {\gu{કેથોડ}};
    
    \node[below] at (0,-2.2) {\gu{ઇન્ટરેક્શન સ્પેસ અને કેવિટીઝ}};
\end{tikzpicture}
\end{answerdiagram}

\textbf{\gu{કાર્યસિદ્ધાંત:}}
\begin{itemize}
    \item \gu{ગરમ કેથોડમાંથી ઇલેક્ટ્રોન ઉત્સર્જન.}
    \item \keyword{\gu{સાયક્લોઇડ ગતિ}} \gu{ક્રોસ્ડ E અને B ફીલ્ડને કારણે.}
    \item \keyword{\gu{બંચિંગ એફેક્ટ}} \gu{ઇલેક્ટ્રોન ક્લાઉડ બનાવે છે.}
    \item \gu{ઇલેક્ટ્રોનથી RF ફીલ્ડમાં એનર્જી ટ્રાન્સફર.}
    \item \gu{કેવિટી રેઝોનન્ટ ફ્રીક્વન્સીએ ઓસીલેશન.}
\end{itemize}

\textbf{\gu{ઉપયોગ}:} \gu{રડાર ટ્રાન્સમિટર, માઇક્રોવેવ ઓવન, ઇન્ડસ્ટ્રિયલ હીટિંગ.}
\end{solutionbox}

\begin{mnemonicbox}
\mnemonic{\gu{"મેગ્નેટ્રોન મેક માઇક્રોવેવ ઓસીલેશન"}}
\end{mnemonicbox}

\orquestionmarks{3(a)}{3}{\gu{રૂબી મેઝરની કામગીરીનું વર્ણન કરો.}}

\begin{solutionbox}
\textbf{\gu{રૂબી મેઝર કાર્ય:}}
\begin{itemize}
    \item \textbf{\gu{રૂબી ક્રિસ્ટલ}} (Al$_2$O$_3$ \gu{જેમાં} Cr$^{3+}$ \gu{આયન}) \gu{એક્ટિવ મીડિયમ તરીકે.}
    \item \textbf{\gu{ક્રોમિયમ આયનમાં ત્રણ એનર્જી લેવલ.}}
    \item \textbf{\gu{પમ્પ ફ્રીક્વન્સી પોપ્યુલેશન ઇન્વર્શન બનાવે છે.}}
    \item 2.9 GHz \textbf{\gu{પર સિગ્નલ એમ્પ્લિફિકેશન.}}
\end{itemize}

\textbf{\gu{પ્રક્રિયા:}} \gu{ઓપ્ટિકલ પમ્પિંગ ઇલેક્ટ્રોનને ઉચ્ચ લેવલમાં ઉત્તેજિત કરે છે} $\rightarrow$ \gu{સ્ટિમ્યુલેટેડ એમિશન કોહેરન્ટ માઇક્રોવેવ પેદા કરે છે} $\rightarrow$ \gu{લો નોઇઝ એમ્પ્લિફિકેશન પ્રાપ્ત થાય છે.}
\end{solutionbox}

\begin{mnemonicbox}
\mnemonic{\gu{"રૂબી રેડિએટ એમ્પ્લિફાઇડ માઇક્રોવેવ"}}
\end{mnemonicbox}

\orquestionmarks{3(b)}{4}{\gu{ગન ડાયોડની VI કેરેક્ટરિસ્ટિક દોરો અને સમજાવો}}

\begin{solutionbox}
\begin{answerdiagram}{Gunn Diode I-V Characteristics}
\begin{tikzpicture}[scale=0.8]
    \draw[->] (0,0) -- (6,0) node[right] {V};
    \draw[->] (0,0) -- (0,5) node[above] {I};
    
    \draw[thick, blue] (0,0) -- (2,4) node[above] {A (\gu{પીક})};
    \draw[thick, blue] (2,4) -- (4,2) node[below] {B (\gu{વેલી})};
    \draw[thick, blue] (4,2) -- (6,3) node[right] {C};
    
    \node at (1, 2) {\gu{ઓહ્મિક રીજન}};
    \node at (3, 3.5) {\gu{નેગેટિવ રેઝિસ્ટન્સ}};
    \node at (5, 2.5) {\gu{સેચ્યુરેશન}};
    
    \draw[dotted] (2,4) -- (2,0) node[below] {$V_p$};
    \draw[dotted] (4,2) -- (4,0) node[below] {$V_v$};
\end{tikzpicture}
\end{answerdiagram}

\textbf{\gu{સમજૂતી:}}
\begin{itemize}
    \item \textbf{\gu{રીજન OA}}: \gu{ઓહ્મિક રીજન (પોઝિટિવ રેઝિસ્ટન્સ).}
    \item \textbf{\gu{રીજન AB}}: \gu{નેગેટિવ રેઝિસ્ટન્સ રીજન.}
    \item \textbf{\gu{રીજન BC}}: \gu{વેલી કરન્ટ રીજન.}
    \item \textbf{\gu{રીજન CD}}: \gu{સેચ્યુરેશન રીજન.}
\end{itemize}

\textbf{\gu{મુખ્ય મુદ્દાઓ:}}
\begin{itemize}
    \item \textbf{\gu{પીક વોલ્ટેજ}}: \gu{નેગેટિવ રેઝિસ્ટન્સ પહેલાં મેક્સિમમ વોલ્ટેજ.}
    \item \textbf{\gu{વેલી કરન્ટ}}: \gu{નેગેટિવ રેઝિસ્ટન્સ રીજનમાં મિનિમમ કરન્ટ.}
    \item \textbf{\gu{નેગેટિવ રેઝિસ્ટન્સ}}: \gu{વોલ્ટેજ વધવા સાથે કરન્ટ ઘટે છે.}
\end{itemize}
\end{solutionbox}

\begin{mnemonicbox}
\mnemonic{\gu{"વેલી પીક નેગેટિવ રેઝિસ્ટન્સ"}}
\end{mnemonicbox}

\orquestionmarks{3(c)}{7}{\gu{માઇક્રોવેવ ફ્રીક્વન્સી પર "frequency measurement method" અને "attenuation measurement method" વિશે વર્ણન કરો.}}

\begin{solutionbox}
\textbf{\gu{ફ્રીક્વન્સી મેઝરમેન્ટ મેથડ:}}
\begin{tabulary}{\linewidth}{|L|L|L|}
\hline
\textbf{\gu{મેથડ}} & \textbf{\gu{સિદ્ધાંત}} & \textbf{\gu{ચોકસાઈ}} \\ \hline
\gu{કેવિટી વેવમીટર} & \gu{રેઝોનન્ટ કેવિટી ટ્યુનિંગ} & \gu{ઉચ્ચ} \\ \hline
\gu{ડાયરેક્ટ રીડિંગ મીટર} & \gu{ફ્રીક્વન્સી કાઉન્ટર} & \gu{ખૂબ ઉચ્ચ} \\ \hline
\gu{હેટેરોડાયન મેથડ} & \gu{બીટ ફ્રીક્વન્સી ટેકનિક} & \gu{મધ્યમ} \\ \hline
\end{tabulary}

\textbf{\gu{એટેન્યુએશન મેઝરમેન્ટ મેથડ:}}
\begin{tabulary}{\linewidth}{|L|L|L|}
\hline
\textbf{\gu{મેથડ}} & \textbf{\gu{વર્ણન}} & \textbf{\gu{ઉપયોગ}} \\ \hline
\gu{સબસ્ટિટ્યુશન મેથડ} & \gu{એટેન્યુએટરને કેલિબ્રેટેડ એટેન્યુએટર સાથે બદલો} & \gu{પ્રિસિઝન મેઝરમેન્ટ} \\ \hline
\gu{પાવર રેશિયો મેથડ} & \gu{ઇનપુટ અને આઉટપુટ પાવરની તુલના} & \gu{સામાન્ય હેતુ} \\ \hline
\gu{RF બ્રિજ મેથડ} & \gu{બ્રિજ સર્કિટ બેલેન્સ} & \gu{લેબોરેટરી ઉપયોગ} \\ \hline
\end{tabulary}

\textbf{\gu{મેઝરમેન્ટ સેટઅપ:}}
\begin{itemize}
    \item \textbf{\gu{સિગ્નલ જનરેટર}} \gu{ટેસ્ટ સિગ્નલ પૂરું પાડે છે.}
    \item \textbf{\gu{કેલિબ્રેટેડ એટેન્યુએટર}} \gu{રેફરન્સ માટે.}
    \item \textbf{\gu{પાવર મીટર}} \gu{સિગ્નલ લેવલ માપે છે.}
    \item \textbf{VSWR \gu{મીટર}} \gu{ઇમ્પીડન્સ મેચિંગ મોનિટર કરે છે.}
\end{itemize}
\end{solutionbox}

\begin{mnemonicbox}
\mnemonic{\gu{"ફ્રીક્વન્સી એટેન્યુએશન પ્રિસાઇઝલી મેઝર્ડ"}}
\end{mnemonicbox}

\questionmarks{4(a)}{3}{\gu{P-i-N ડાયોડની કામગીરી વર્ણન કરો.}}

\begin{solutionbox}
\textbf{\gu{સ્ટ્રક્ચર}}: P-\gu{ટાઇપ રીજન (હેવિલી ડોપ્ડ), ઇન્ટ્રિન્સિક રીજન (અનડોપ્ડ, હાઇ રેઝિસ્ટન્સ),} N-\gu{ટાઇપ રીજન (હેવિલી ડોપ્ડ).}

\textbf{\gu{કાર્ય:}}
\begin{itemize}
    \item \textbf{\gu{ફોરવર્ડ બાયાસ}}: \gu{લો રેઝિસ્ટન્સ, કંડક્ટર તરીકે કાર્ય કરે છે.}
    \item \textbf{\gu{રિવર્સ બાયાસ}}: \gu{હાઇ રેઝિસ્ટન્સ, ઇન્સુલેટર તરીકે કાર્ય કરે છે.}
    \item \textbf{RF \gu{સ્વિચિંગ}}: \gu{ચાર્જ સ્ટોરેજને કારણે ફાસ્ટ સ્વિચિંગ.}
\end{itemize}

\textbf{\gu{ઉપયોગ}}: RF \gu{સ્વિચ, એટેન્યુએટર, ફેઝ શિફ્ટર.}
\end{solutionbox}

\begin{mnemonicbox}
\mnemonic{\gu{"PIN પ્રોવાઇડ ઇન્સ્ટન્ટ સ્વિચિંગ"}}
\end{mnemonicbox}

\questionmarks{4(b)}{4}{\gu{મેગ્નેટ્રોન માટે $\pi$ મોડ ઓસીલેશનનું વર્ણન કરો.}}

\begin{solutionbox}
\textbf{$\pi$ \gu{મોડ ઓસીલેશન:}}
\begin{itemize}
    \item \textbf{\gu{બાજુની કેવિટી}} 180$^\circ$ \gu{આઉટ ઓફ ફેઝમાં ઓસીલેટ કરે છે.}
    \item \textbf{\gu{ઇલેક્ટ્રોન બંચિંગ}} RF \gu{ફીલ્ડ સાથે સિંક્રોનાઇઝ.}
    \item \textbf{\gu{ઇલેક્ટ્રોનથી RF માં મેક્સિમમ પાવર ટ્રાન્સફર.}}
    \item \textbf{\gu{ડિઝાઇન કરેલી ફ્રીક્વન્સીએ સ્ટેબલ ઓસીલેશન.}}
\end{itemize}

\textbf{\gu{મોડ ચાર્ટ:}}
\begin{center}
Cavity: 1 --- 2 --- 3 --- 4 --- 5 --- 6 --- 7 --- 8 \\
Phase: 0 --- $\pi$ --- 0 --- $\pi$ --- 0 --- $\pi$ --- 0 --- $\pi$
\end{center}
\end{solutionbox}

\begin{mnemonicbox}
\mnemonic{\gu{"પાઇ મોડ મેક્સિમમ પાવર પ્રોડ્યુસ કરે"}}
\end{mnemonicbox}

\questionmarks{4(c)}{7}{\gu{જરૂરી ડાયાગ્રામ સાથે ટ્વો કેવિટી ક્લિસ્ટ્રોન એમ્પ્લિફાયરનું કન્સ્ટ્રક્શન અને કામગીરી સમજાવો.}}

\begin{solutionbox}
\begin{answerdiagram}{Two Cavity Klystron}
\begin{tikzpicture}[auto, node distance=1.5cm]
    \node [gtu start] (gun) {\gu{ઇલેક્ટ્રોન ગન}};
    \node [gtu block, right=of gun] (buncher) {\gu{બંચર}};
    \node [gtu block, right=of buncher, minimum width=2.5cm] (drift) {\gu{ડ્રિફ્ટ સ્પેસ}};
    \node [gtu block, right=of drift] (catcher) {\gu{કેચર}};
    \node [gtu stop, right=of catcher] (coll) {\gu{કલેક્ટર}};
    
    \draw [gtu arrow] (gun) -- (buncher);
    \draw [gtu arrow] (buncher) -- (drift);
    \draw [gtu arrow] (drift) -- (catcher);
    \draw [gtu arrow] (catcher) -- (coll);
    
    \draw [<-] (buncher.south) -- +(0,-0.5) node[below] {RF In};
    \draw [->] (catcher.south) -- +(0,-0.5) node[below] {RF Out};
\end{tikzpicture}
\end{answerdiagram}

\textbf{\gu{બાંધકામ:}}
\begin{itemize}
    \item \textbf{\gu{ઇલેક્ટ્રોન ગન}} \gu{ઇલેક્ટ્રોન બીમ પેદા કરે છે.}
    \item \textbf{\gu{ઇનપુટ કેવિટી}} (\gu{બંચર}) \gu{ઇલેક્ટ્રોન બીમ મોડ્યુલેટ કરે છે.}
    \item \textbf{\gu{ડ્રિફ્ટ સ્પેસ}} \gu{વેલોસિટી મોડ્યુલેશનની પરવાનગી આપે છે.}
    \item \textbf{\gu{આઉટપુટ કેવિટી}} (\gu{કેચર}) RF \gu{એનર્જી બહાર કાઢે છે.}
    \item \textbf{\gu{કલેક્ટર}} \gu{વપરાયેલા ઇલેક્ટ્રોન એકત્ર કરે છે.}
\end{itemize}

\textbf{\gu{કાર્યસિદ્ધાંત:}}
\gu{ઇનપુટ કેવિટીમાં વેલોસિટી મોડ્યુલેશન} $\rightarrow$ \gu{ડ્રિફ્ટ સ્પેસમાં ઇલેક્ટ્રોન બંચિંગ} $\rightarrow$ \gu{ડેન્સિટી મોડ્યુલેશન કરન્ટ વેરિએશન બનાવે છે} $\rightarrow$ \gu{આઉટપુટ કેવિટીમાં એનર્જી એક્સટ્રેક્શન} $\rightarrow$ \gu{એમ્પ્લિફિકેશન.}
\end{solutionbox}

\begin{mnemonicbox}
\mnemonic{\gu{"ક્લિસ્ટ્રોન બંચિંગ દ્વારા એમ્પ્લિફાઇ કરે છે"}}
\end{mnemonicbox}

\orquestionmarks{4(a)}{3}{\gu{પેરામેટ્રિક એમ્પ્લિફાયરનું વર્ણન કરો.}}

\begin{solutionbox}
\textbf{\gu{પેરામેટ્રિક એમ્પ્લિફાયર:}}
\begin{itemize}
    \item \textbf{\gu{વેરેક્ટર ડાયોડ ઉપયોગ કરતું વેરિએબલ રિએક્ટન્સ}} \gu{ડિવાઇસ.}
    \item \textbf{\gu{પમ્પ ફ્રીક્વન્સી}} \gu{ડાયોડ કેપેસિટન્સ મોડ્યુલેટ કરે છે.}
    \item \textbf{\gu{પમ્પથી સિગ્નલમાં એનર્જી ટ્રાન્સફર.}}
    \item \textbf{\gu{લો નોઇઝ એમ્પ્લિફિકેશન}} \gu{પ્રાપ્ત થાય છે.}
\end{itemize}

\textbf{\gu{કાર્ય:}}
\gu{પમ્પ પાવર ડાયોડ રિએક્ટન્સ વેરી કરે છે} $\rightarrow$ \gu{સિગ્નલ મિક્સિંગ સમ અને ડિફરન્સ ફ્રીક્વન્સી પેદા કરે છે} $\rightarrow$ \gu{આઇડલર ફ્રીક્વન્સી} $f_p = f_s + f_i$ $\rightarrow$ \gu{નોનલિનિયર મિક્સિંગ દ્વારા પાવર ગેઇન.}
\end{solutionbox}

\begin{mnemonicbox}
\mnemonic{\gu{"પેરામેટ્રિક એમ્પ્લિફાયર પમ્પ લો નોઇઝ"}}
\end{mnemonicbox}

\orquestionmarks{4(b)}{4}{\gu{ટ્રાવેલિંગ વેવ ટ્યુબની આકૃતિ દોરો અને સમજાવો}}

\begin{solutionbox}
\begin{answerdiagram}{Traveling Wave Tube}
\begin{tikzpicture}[auto, node distance=1.5cm]
    \node [gtu start] (gun) {\gu{ઇલેક્ટ્રોન ગન}};
    \node [gtu block, right=of gun, minimum width=4cm] (helix) {\gu{હેલિક્સ}};
    \node [gtu stop, right=of helix] (col) {\gu{કલેક્ટર}};
    
    \draw [gtu arrow] (gun) -- (helix);
    \draw [gtu arrow] (helix) -- (col);
    
    \draw [<-] (2, 0.5) -- (2, 1) node[above] {RF In};
    \draw [->] (5, 0.5) -- (5, 1) node[above] {RF Out};
    
    \node [below] at (3.5, -0.5) {\gu{એટેન્યુએટર}};
\end{tikzpicture}
\end{answerdiagram}

\textbf{\gu{કાર્ય:}}
\begin{itemize}
    \item \textbf{\gu{ઇલેક્ટ્રોન બીમ}} \gu{હેલિક્સ કેન્દ્રમાંથી જાય છે.}
    \item \textbf{RF \gu{સિગ્નલ}} \gu{હેલિક્સ સાથે પ્રોપેગેટ થાય છે.}
    \item \textbf{\gu{સિંક્રોનિઝમ}} \gu{બીમ અને} RF \gu{વેવ વચ્ચે.}
    \item \textbf{\gu{એનર્જી ટ્રાન્સફર}} \gu{બીમથી} RF \gu{માં.}
    \item \textbf{\gu{કન્ટિન્યુઅસ એમ્પ્લિફિકેશન}} \gu{હેલિક્સ લેન્થ સાથે.}
\end{itemize}
\end{solutionbox}

\begin{mnemonicbox}
\mnemonic{\gu{"TWT વેવ્સ સાથે ટ્રાવેલ કરે છે"}}
\end{mnemonicbox}

\orquestionmarks{4(c)}{7}{\gu{રિફ્લેક્સ ક્લિસ્ટ્રોનનો કાર્યસિદ્ધાંત ઉચિત આકૃતિ સાથે ઊંડાણમાં સમજાવો}}

\begin{solutionbox}
\begin{answerdiagram}{Reflex Klystron Schematic}
\begin{tikzpicture}[auto, node distance=1.5cm]
    \node [gtu start] (cath) {\gu{કેથોડ}};
    \node [gtu block, right=of cath] (cav) {\gu{કેવિટી}};
    \node [gtu block, right=of cav] (rep) {\gu{રિપેલર}};
    
    \draw [gtu arrow] (cath) -- (cav);
    \draw [gtu arrow] (cav) -- (rep);
    \draw [->, bend right, dashed] (rep) to node[above] {\gu{રિફ્લેક્ટેડ} e$^-$} (cav);
    
    \draw [->] (cav.south) -- +(0,-0.5) node[below] {RF Output};
\end{tikzpicture}
\end{answerdiagram}

\textbf{\gu{કાર્યસિદ્ધાંત:}}
\begin{enumerate}
    \item \textbf{\gu{ઇલેક્ટ્રોન કેવિટીમાં દાખલ થાય છે}} \gu{અને વેલોસિટી મોડ્યુલેટેડ થાય છે.}
    \item \textbf{\gu{ઇલેક્ટ્રોન રિપેલર તરફ ડ્રિફ્ટ કરે છે.}}
    \item \textbf{\gu{રિપેલર ઇલેક્ટ્રોનને કેવિટીમાં પાછા રિફ્લેક્ટ કરે છે.}}
    \item \textbf{\gu{ટ્રાન્ઝિટ ટાઇમ}} \gu{બંચિંગ ફેઝ નક્કી કરે છે.}
    \item \textbf{\gu{બંચ્ડ ઇલેક્ટ્રોન}} \gu{કેવિટીને એનર્જી પહોંચાડે છે.}
    \item \textbf{\gu{ફીડબેક દ્વારા ઓસીલેશન કાયમ રાખવામાં આવે છે.}}
\end{enumerate}

\textbf{\gu{ઉપયોગ}}: \gu{લોકલ ઓસીલેટર, ફ્રીક્વન્સી મીટર, માઇક્રોવેવ સોર્સ.}
\end{solutionbox}

\begin{mnemonicbox}
\mnemonic{\gu{"રિફ્લેક્સ ઇલેક્ટ્રોન બંચ પાછા આપે છે"}}
\end{mnemonicbox}

\questionmarks{5(a)}{3}{\gu{"PIN ડાયોડ સ્વિચ તરીકે કાર્ય કરે અને VARACTOR ડાયોડ વેરિયેબલ કૅપેસિટર તરીકે કાર્ય કરે." વિસ્તારમાં વર્ણન કરો.}}

\begin{solutionbox}
\textbf{\gu{સ્વિચ તરીકે PIN ડાયોડ:}}
\begin{itemize}
    \item \textbf{\gu{ફોરવર્ડ બાયાસ}}: \gu{લો રેઝિસ્ટન્સ ($\sim 1\Omega$), સ્વિચ ON.}
    \item \textbf{\gu{રિવર્સ બાયાસ}}: \gu{હાઇ રેઝિસ્ટન્સ ($\sim 10k\Omega$), સ્વિચ OFF.}
    \item \textbf{\gu{ફાસ્ટ સ્વિચિંગ}} \gu{I-રીજનમાં ચાર્જ સ્ટોરેજને કારણે.}
    \item \textbf{OFF \gu{સ્ટેટમાં}} RF \gu{આઇસોલેશન.}
\end{itemize}

\textbf{\gu{વેરિયેબલ કૅપેસિટર તરીકે VARACTOR ડાયોડ:}}
\begin{itemize}
    \item \textbf{\gu{રિવર્સ બાયાસ વોલ્ટેજ}} \gu{જંક્શન કૅપેસિટન્સ કંટ્રોલ કરે છે.}
    \item \textbf{\gu{રિવર્સ વોલ્ટેજ વધવા સાથે કૅપેસિટન્સ ઘટે છે}} ($C \propto V^{-n}$).
    \item \textbf{\gu{વોલ્ટેજ-કંટ્રોલ્ડ રિએક્ટન્સ}} \gu{ટ્યુનિંગ સર્કિટ માટે.}
    \item \textbf{\gu{ઇલેક્ટ્રોનિક ટ્યુનિંગ}} \gu{મિકેનિકલ એડજસ્ટમેન્ટ વિના.}
\end{itemize}
\end{solutionbox}

\begin{mnemonicbox}
\mnemonic{\gu{"PIN સ્વિચ કરે, VARACTOR વેરી કરે"}}
\end{mnemonicbox}

\questionmarks{5(b)}{4}{\gu{રડારમાં વપરાતી ડિસ્પ્લે પદ્ધતિઓની યાદી બનાવો અને એકનું વિસ્તારવાર વર્ણન કરો.}}

\begin{solutionbox}
\textbf{\gu{રડાર ડિસ્પ્લે પદ્ધતિઓ:}}
1. A-\gu{સ્કોપ ડિસ્પ્લે}, 2. PPI, 3. B-\gu{સ્કોપ ડિસ્પ્લે}, 4. RHI.

\textbf{PPI \gu{ડિસ્પ્લે સમજૂતી:}}
\begin{itemize}
    \item \textbf{\gu{સર્ક્યુલર ડિસ્પ્લે}} \gu{ટાર્ગેટ પોઝિશન દર્શાવે છે.}
    \item \textbf{\gu{કેન્દ્ર રડાર લોકેશન દર્શાવે છે.}}
    \item \textbf{\gu{રેડિયલ ડિસ્ટન્સ}} \gu{ટાર્ગેટ રેન્જ સૂચવે છે.}
    \item \textbf{\gu{એંગ્યુલર પોઝિશન}} \gu{ટાર્ગેટ બેરિંગ દર્શાવે છે.}
    \item \textbf{\gu{રોટેટિંગ સ્વીપ}} \gu{એન્ટીના રોટેશન સાથે સિંક્રોનાઇઝ્ડ.}
\end{itemize}
\textbf{\gu{લાક્ષણિકતાઓ}:} \gu{રિયલ-ટાઇમ ડિસ્પ્લે, રેન્જ અને બેરિંગ માહિતી, મલ્ટિપલ ટાર્ગેટ ટ્રેકિંગ.}
\end{solutionbox}

\begin{mnemonicbox}
\mnemonic{\gu{"PPI પિક્ચર પોઝિશન ઇન્ડિકેટર"}}
\end{mnemonicbox}

\questionmarks{5(c)}{7}{\gu{રડાર શું છે? વિવિધ પ્રકારના રડાર સિસ્ટમ્સની યાદી બનાવો? એક રડારનું વિસ્તારવાર વર્ણન કરો.}}

\begin{solutionbox}
\textbf{\gu{રડાર (Radio Detection And Ranging):}} \gu{ઑબ્જેક્ટ ડિટેક્ટ કરવા અને તેમની રેન્જ, વેલોસિટી અને લાક્ષણિકતાઓ નક્કી કરવા માટે રેડિયો વેવ્સનો ઉપયોગ કરતી સિસ્ટમ.}

\textbf{\gu{રડાર સિસ્ટમ્સના પ્રકાર:}}
\begin{tabulary}{\linewidth}{|L|L|L|}
\hline
\textbf{\gu{પ્રકાર}} & \textbf{\gu{ઉપયોગ}} & \textbf{\gu{ફ્રીક્વન્સી બેન્ડ}} \\ \hline
\gu{પલ્સ રડાર} & \gu{એર ટ્રાફિક કંટ્રોલ} & L, S, C \gu{બેન્ડ} \\ \hline
CW \gu{ડોપ્લર રડાર} & \gu{સ્પીડ મેઝરમેન્ટ} & X, K, Ka \gu{બેન્ડ} \\ \hline
MTI \gu{રડાર} & \gu{મૂવિંગ ટાર્ગેટ ડિટેક્શન} & S, C \gu{બેન્ડ} \\ \hline
SAR \gu{રડાર} & \gu{ગ્રાઉન્ડ મેપિંગ} & L, C, X \gu{બેન્ડ} \\ \hline
\end{tabulary}

\textbf{\gu{પલ્સ રડાર વિગતવાર સમજૂતી:}}

\begin{answerdiagram}{Pulse Radar Block Diagram}
\begin{tikzpicture}[auto, node distance=1.5cm]
    \node [gtu block] (tim) {\gu{ટાઇમર}};
    \node [gtu block, below=of tim] (mod) {\gu{મોડ્યુલેટર}};
    \node [gtu block, right=of mod] (tx) {\gu{ટ્રાન્સમિટર}};
    \node [gtu block, right=of tx] (dup) {\gu{ડુપ્લેક્સર}};
    \node [gtu block, right=of dup] (ant) {\gu{એન્ટીના}};
    \node [gtu block, below=of dup] (rx) {\gu{રિસીવર}};
    \node [gtu block, left=of rx] (disp) {\gu{ડિસ્પ્લે}};
    
    \draw [gtu arrow] (tim) -- (mod);
    \draw [gtu arrow] (mod) -- (tx);
    \draw [gtu arrow] (tx) -- (dup);
    \draw [gtu arrow] (dup) -- (ant);
    \draw [gtu arrow] (ant) -- (dup);
    \draw [gtu arrow] (dup) -- (rx);
    \draw [gtu arrow] (rx) -- (disp);
    \draw [gtu arrow] (tim) -| (disp);
\end{tikzpicture}
\end{answerdiagram}

\textbf{\gu{કાર્ય:}}
\begin{itemize}
    \item RF \gu{એનર્જીના ટૂંકા પલ્સ ટ્રાન્સમિટ કરે છે.}
    \item \gu{ટાર્ગેટથી ઇકો રિસીવ કરે છે.}
    \item \gu{રેન્જ કેલ્ક્યુલેશન માટે ટાઇમ ડિલે માપે છે.}
    \item \gu{ડિસ્પ્લે માટે સિગ્નલ પ્રોસેસ કરે છે.}
\end{itemize}

\textbf{\gu{રેન્જ સમીકરણ:}} $R = (c \times t)/2$.
\end{solutionbox}

\orquestionmarks{5(a)}{3}{\gu{TRAPATT ડાયોડનું કાર્ય ડાયાગ્રામ સાથે વર્ણવો.}}

\begin{solutionbox}
\textbf{TRAPATT \gu{ઓપરેશન:}}
\begin{itemize}
    \item \textbf{TRApped Plasma Avalanche Triggered Transit} \gu{ડાયોડ.}
    \item \textbf{\gu{હાઇ ફીલ્ડ રીજન}} \gu{એવેલાન્ચ બ્રેકડાઉન બનાવે છે.}
    \item \textbf{\gu{પ્લાઝમા ફોર્મેશન}} \gu{ચાર્જ કેરિયર ટ્રેપ કરે છે.}
    \item \textbf{\gu{ટ્રાન્ઝિટ ટાઇમ એફેક્ટ્સ}} \gu{નેગેટિવ રેઝિસ્ટન્સ બનાવે છે.}
    \item \textbf{\gu{ટ્રાન્ઝિટ ટાઇમ દ્વારા ઓસીલેશન ફ્રીક્વન્સી નક્કી થાય છે.}}
\end{itemize}

\begin{answerdiagram}{TRAPATT Diode Operation}
\begin{tikzpicture}[scale=0.8]
    \draw[->] (0,0) -- (5,0) node[right] {V};
    \draw[->] (0,0) -- (0,4) node[above] {I};
    
    \draw[thick] (1,3.5) -- (1.2,0.5);
    \node[right] at (1.2, 2) {Trapped plasma avalanche};
    \draw[dashed] (1,0) -- (1,3.5) node[below, yshift=-4cm] {Breakdown voltage};
\end{tikzpicture}
\end{answerdiagram}

\textbf{\gu{ઉપયોગ}:} \gu{હાઇ પાવર ઓસીલેટર, રડાર ટ્રાન્સમિટર, કમ્યુનિકેશન સિસ્ટમ.}
\end{solutionbox}

\begin{mnemonicbox}
\mnemonic{\gu{"TRAPATT ટ્રેપ પ્લાઝમા એવેલાન્ચ"}}
\end{mnemonicbox}

\orquestionmarks{5(b)}{4}{\gu{રડારની સોનાર ની સાથે તુલના કરો.}}

\begin{solutionbox}
\textbf{\gu{તુલના:}}

\begin{tabulary}{\linewidth}{|L|L|L|}
\hline
\textbf{\gu{પેરામીટર}} & \textbf{\gu{રડાર}} & \textbf{\gu{સોનાર}} \\ \hline
\keyword{\gu{વેવ ટાઇપ}} & \gu{ઇલેક્ટ્રોમેગ્નેટિક વેવ્સ} & \gu{સાઉન્ડ વેવ્સ} \\ \hline
\keyword{\gu{મીડિયમ}} & \gu{હવા/વેક્યુમ} & \gu{પાણી/લિક્વિડ} \\ \hline
\keyword{\gu{ફ્રીક્વન્સી}} & \gu{GHz રેન્જ} & \gu{kHz રેન્જ} \\ \hline
\keyword{\gu{સ્પીડ}} & $3 \times 10^8$ m/s & \gu{પાણીમાં 1500 m/s} \\ \hline
\keyword{\gu{રેન્જ}} & \gu{ખૂબ લાંબી રેન્જ} & \gu{એબ્સોર્પ્શન દ્વારા મર્યાદિત} \\ \hline
\keyword{\gu{ઉપયોગ}} & \gu{હવા/સ્પેસ ડિટેક્શન} & \gu{અંડરવોટર ડિટેક્શન} \\ \hline
\end{tabulary}

\textbf{\gu{સમાનતાઓ:}} \gu{ડિટેક્શન માટે ઇકો સિદ્ધાંત, ટાઇમ ડિલે વડે રેન્જ મેઝરમેન્ટ, વેલોસિટી મેઝરમેન્ટ માટે ડોપ્લર એફેક્ટ.}
\end{solutionbox}

\begin{mnemonicbox}
\mnemonic{\gu{"રડાર રેડિએટ કરે, સોનાર સાઉન્ડ કરે"}}
\end{mnemonicbox}

\orquestionmarks{5(c)}{7}{\gu{મહત્તમ રડાર રેન્જનું સમીકરણ મેળવો.}}

\begin{solutionbox}
\textbf{\gu{રડાર રેન્જ સમીકરણ વ્યુત્પત્તિ:}}

1. \textbf{\gu{ટ્રાન્સમિટેડ પાવર:}} $P_t$

2. \textbf{\gu{ટાર્ગેટ પર પાવર ડેન્સિટી:}}
   $$P_d = \frac{P_t}{4\pi R^2}$$

3. \textbf{\gu{ટાર્ગેટ દ્વારા ઇન્ટરસેપ્ટેડ પાવર:}}
   $$P_i = P_d \times \sigma = \frac{P_t \times \sigma}{4\pi R^2}$$

4. \textbf{\gu{રડાર તરફ પાછું આવતું પાવર:}}
   $$P_r = \frac{P_i}{4\pi R^2} = \frac{P_t \times \sigma}{(4\pi R^2)^2}$$

5. \textbf{\gu{રિસીવ્ડ પાવર:}}
   $$P_r = \frac{P_t \times G^2 \times \lambda^2 \times \sigma}{(4\pi)^3 \times R^4}$$

\textbf{\gu{મેક્સિમમ રેન્જ સમીકરણ:}}
$$R_{max} = \sqrt[4]{\frac{P_t \times G^2 \times \lambda^2 \times \sigma}{(4\pi)^3 \times P_{r_{min}}}}$$

\textbf{\gu{જ્યાં:}}
\begin{itemize}
    \item $P_t$ = \gu{ટ્રાન્સમિટેડ પાવર}
    \item $G$ = \gu{એન્ટીના ગેઇન}
    \item $\lambda$ = \gu{વેવલેન્થ}
    \item $\sigma$ = \gu{રડાર ક્રોસ સેક્શન}
    \item $P_{r_{min}}$ = \gu{મિનિમમ ડિટેક્ટેબલ સિગ્નલ}
    \item $R$ = \gu{રેન્જ}
\end{itemize}

\textbf{\gu{રેન્જ અસર કરતા પરિબળો:}} \gu{ટ્રાન્સમિટેડ પાવર, એન્ટીના ગેઇન, ટાર્ગેટ ક્રોસ-સેક્શન, ફ્રીક્વન્સી, રિસીવર સેન્સિટિવિટી.}
\end{solutionbox}

\begin{mnemonicbox}
\mnemonic{\gu{"પાવર ગેઇન લેમ્બડા સિગ્મા રેન્જ"}}
\end{mnemonicbox}

\end{document}
