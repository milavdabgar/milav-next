\documentclass[10pt,a4paper]{article}

% content/resources/templates/preamble.tex
\usepackage[margin=0.6in]{geometry}
\author{Milav Dabgar}
\usepackage{amsmath,amssymb,amsthm}
\usepackage{booktabs}
\usepackage{multirow}
\usepackage{xcolor}
\usepackage{tcolorbox}
\tcbuselibrary{breakable,skins}
\usepackage[colorlinks=true,linkcolor=blue]{hyperref}
\usepackage{titlesec}
\usepackage{enumitem}
\usepackage{tikz}
\usepackage{pgfplots}
\usepackage{circuitikz}
\usepackage[version=4]{mhchem}
\usepackage{longtable}
\usepackage{array}
\usepackage{float}
\usepackage{caption}
\usepackage{listings}

\lstset{
  basicstyle=\small\ttfamily,
  breaklines=true,
  breakatwhitespace=false,
  postbreak=\mbox{\textcolor{red}{$\hookrightarrow$}\space},
  float=false,
  numbers=left,
  numberstyle=\tiny\color{gray},
  numbersep=10pt,
  xleftmargin=2em,
  keywordstyle=\color{blue},
  commentstyle=\color{green!60!black},
  stringstyle=\color{purple},
  backgroundcolor=\color{gray!5},
  showstringspaces=false,
  tabsize=2,
  captionpos=b,
  keepspaces=true,
  columns=flexible
}

\pgfplotsset{compat=1.18}
\usetikzlibrary{shapes,arrows,positioning,calc,patterns,decorations.pathmorphing,decorations.markings,arrows.meta}

% Color scheme
\definecolor{headcolor}{RGB}{0,102,204}
\definecolor{keycolor}{RGB}{220,20,60}
\definecolor{solutioncolor}{RGB}{34,139,34}
\definecolor{mnemoniccolor}{RGB}{148,0,211}
\definecolor{codecolor}{RGB}{0,0,100}

% Spacing
\setlength{\parskip}{3pt}
\setlist[itemize]{nosep}
\setlist[enumerate]{nosep}

% Title formatting
\titleformat{\section}{\Large\bfseries\color{headcolor}}{\thesection}{1em}{}
\titleformat{\subsection}{\large\bfseries\color{headcolor}}{\thesubsection}{1em}{}

% Pandoc tightlist compatibility
\providecommand{\tightlist}{%
  \setlength{\itemsep}{0pt}\setlength{\parskip}{0pt}}

% Pandoc longtable compatibility
\newcounter{none}
\def\thenone{}


% content/resources/templates/gujarati-boxes.tex
\usepackage{fontspec}
\usepackage{polyglossia}

% Set Gujarati as main language (document is primarily in Gujarati)
% Note: gloss-gujarati.ldf doesn't exist in polyglossia, but it will use hyphenation patterns
\setdefaultlanguage{gujarati}
\setotherlanguage{english}

% Configure Gujarati font properly
% Use Language=Default to prevent polyglossia from trying to add language-specific features
% that don't exist for Gujarati, which causes "empty feature" warnings
\newfontfamily\gujaratifont[Script=Gujarati,AutoFakeBold=2.5,AutoFakeSlant=0.3]{Noto Sans Gujarati}
\setmainfont[Script=Gujarati,AutoFakeBold=2.5,AutoFakeSlant=0.3]{Noto Sans Gujarati}
% Use Noto Sans Gujarati for monospace to support Gujarati in text
\setmonofont[Scale=0.9]{Noto Sans Gujarati}

% Configure English to use the same font
\newfontfamily\englishfont[Script=Gujarati,AutoFakeBold=2.5,AutoFakeSlant=0.3]{Noto Sans Gujarati}

% Translations for polyglossia
\gappto\captionsgujarati{
  \renewcommand{\tablename}{કોષ્ટક}
  \renewcommand{\figurename}{આકૃતિ}
}

% Helper for TikZ nodes to ensure Gujarati font
\newcommand{\gu}[1]{{\gujaratifont #1}}

% Custom environments
\newtcolorbox{solutionbox}{
    breakable,
    enhanced,
    colback=solutioncolor!5!white,
    colframe=solutioncolor!75!black,
    fonttitle=\bfseries,
    title=જવાબ
}

\newtcolorbox{solutionboxnobreak}{
 colback=solutioncolor!5!white,
 colframe=solutioncolor!75!black,
 fonttitle=\bfseries,
 title=જવાબ
}

\newtcolorbox{keyformula}{
 breakable,
 enhanced,
 colback=keycolor!5!white,
 colframe=keycolor!75!black,
 fonttitle=\bfseries,
 title=રાસાયણિક સમીકરણ/સૂત્ર
}

\newtcolorbox{mnemonicbox}{
 breakable,
 enhanced,
 colback=mnemoniccolor!5!white,
 colframe=mnemoniccolor!75!black,
 fonttitle=\bfseries,
 title=મેમરી ટ્રીક
}


\begin{document}

\begin{center}
{\Huge\bfseries\color{headcolor} Subject Name (Gujarati)}\\[5pt]
{\LARGE 4351103 -- Summer 2024}\\[3pt]
{\large Semester 1 Study Material}\\[3pt]
{\normalsize\textit{Detailed Solutions and Explanations}}
\end{center}

\vspace{10pt}

\subsection*{પ્રશ્ન 1(અ) [3
માર્ક્સ]}\label{uxaaauxab0uxab6uxaa8-1uxa85-3-uxaaeuxab0uxa95uxab8}

\textbf{વિવિધ માઇક્રોવેવ બેન્ડની તેમની આવૃત્તિ શ્રેણી સાથેની યાદી કરો.}

\begin{solutionbox}

\textbf{માઇક્રોવેવ આવૃત્તિ બેન્ડ કોષ્ટક:}

{\def\LTcaptype{none} % do not increment counter
\begin{longtable}[]{@{}lll@{}}
\toprule\noalign{}
બેન્ડ & આવૃત્તિ શ્રેણી & તરંગલંબાઇ \\
\midrule\noalign{}
\endhead
\bottomrule\noalign{}
\endlastfoot
\textbf{L Band} & 1-2 GHz & 30-15 cm \\
\textbf{S Band} & 2-4 GHz & 15-7.5 cm \\
\textbf{C Band} & 4-8 GHz & 7.5-3.75 cm \\
\textbf{X Band} & 8-12 GHz & 3.75-2.5 cm \\
\textbf{Ku Band} & 12-18 GHz & 2.5-1.67 cm \\
\textbf{K Band} & 18-27 GHz & 1.67-1.11 cm \\
\textbf{Ka Band} & 27-40 GHz & 1.11-0.75 cm \\
\end{longtable}
}

\end{solutionbox}
\begin{mnemonicbox}
``લાર્જ શીપ્સ કેન eXામીન કિંડલી યુઝિંગ નોલેજ ઓલવેઝ''

\end{mnemonicbox}
\begin{center}\rule{0.5\linewidth}{0.5pt}\end{center}

\subsection*{પ્રશ્ન 1(બ) [4
માર્ક્સ]}\label{uxaaauxab0uxab6uxaa8-1uxaac-4-uxaaeuxab0uxa95uxab8}

\textbf{ટ્રાન્સમિશન લાઇનનું સામાન્ય સમકક્ષ સર્કિટ દોરો. લોસલેસ લાઇન માટે લાક્ષણિક
અવબાધ માટેનું સમીકરણ લખો.}

\begin{solutionbox}

\textbf{ટ્રાન્સમિશન લાઇન સમકક્ષ સર્કિટ:}

\begin{verbatim}
    R      L
  {-{-}{-}{-}▬▬▬▬{-}{-}{-}{-}}
 |             |
 |      C      | G
 |    {-{-}{-}{-}{-}    |}
 |             |
  {-{-}{-}{-}{-}{-}{-}{-}{-}{-}{-}{-}{-}}
      dx
\end{verbatim}

\textbf{સર્કિટ એલિમેન્ટ્સ:}

\begin{itemize}
\tightlist
\item
  \textbf{R}: યુનિટ લંબાઇ દીઠ શ્રેણી પ્રતિકાર
\item
  \textbf{L}: યુનિટ લંબાઇ દીઠ શ્રેણી ઇન્ડક્ટન્સ\\
\item
  \textbf{C}: યુનિટ લંબાઇ દીઠ શન્ટ કેપેસિટન્સ
\item
  \textbf{G}: યુનિટ લંબાઇ દીઠ શન્ટ કન્ડક્ટન્સ
\end{itemize}

\textbf{લોસલેસ લાઇન માટે (R = 0, G = 0):}

\textbf{લાક્ષણિક અવબાધ:} Z_{0} = \sqrt(L/C)

\textbf{મુખ્ય મુદ્દાઓ:}

\begin{itemize}
\tightlist
\item
  \textbf{લોસલેસ સ્થિતિ}: ટ્રાન્સમિશન દરમિયાન કોઈ પાવર લોસ નથી
\item
  \textbf{અવબાધ મેચિંગ}: Z_{0} રિફ્લેક્શન વર્તન નક્કી કરે છે
\end{itemize}

\end{solutionbox}
\begin{mnemonicbox}
``લોસલેસ લાઇન્સ લવ કોન્સ્ટન્ટ ઇમ્પિડન્સ''

\end{mnemonicbox}
\begin{center}\rule{0.5\linewidth}{0.5pt}\end{center}

\subsection*{પ્રશ્ન 1(ક) [7
માર્ક્સ]}\label{uxaaauxab0uxab6uxaa8-1uxa95-7-uxaaeuxab0uxa95uxab8}

\textbf{એક જ સ્ટબનો ઉપયોગ કરીને ઇમ્પિડન્સ મેચિંગ પ્રક્રિયા સમજાવો.}

\begin{solutionbox}

\textbf{સિંગલ સ્ટબ મેચિંગ પ્રક્રિયા:}

\begin{center}
\textbf{Mermaid Diagram (Code)}
\begin{verbatim}
{Shaded}
{Highlighting}[]
graph LR
    A[સોર્સ] {-{-}{} B[મેઇન લાઇન]}
    B {-{-}{} C[સ્ટબ કનેક્શન પોઇન્ટ]}
    C {-{-}{} D[લોડ]}
    C {-{-}{} E[શોર્ટ સ્ટબ]}
{Highlighting}
{Shaded}
\end{verbatim}
\end{center}

\textbf{મેચિંગ પગલાં:}

{\def\LTcaptype{none} % do not increment counter
\begin{longtable}[]{@{}lll@{}}
\toprule\noalign{}
પગલું & પ્રક્રિયા & હેતુ \\
\midrule\noalign{}
\endhead
\bottomrule\noalign{}
\endlastfoot
\textbf{1} & લોડ એડમિટન્સ કેલ્ક્યુલેટ કરો & Y\_L = 1/Z\_L શોધો \\
\textbf{2} & જનરેટર તરફ મૂવ કરો & પોઇન્ટ શોધો જ્યાં G = G_{0} \\
\textbf{3} & સ્ટબ સસેપ્ટન્સ ઉમેરો & રિએક્ટિવ ભાગ કેન્સલ કરો \\
\textbf{4} & મેચિંગ હાસિલ કરો & Y\_total = Y_{0} \\
\end{longtable}
}

\textbf{ડિઝાઇન સમીકરણો:}

\begin{itemize}
\tightlist
\item
  \textbf{સ્ટબ સુધી અંતર:} d = (λ/2π) \times tan^{-}^{1}(\sqrt(R\_L/R_{0}))
\item
  \textbf{સ્ટબ લંબાઇ:} l = (λ/2π) \times tan^{-}^{1}(B\_stub/Y_{0})
\end{itemize}

\textbf{એપ્લિકેશન્સ:}

\begin{itemize}
\tightlist
\item
  \textbf{એન્ટીના મેચિંગ}
\item
  \textbf{એમ્પ્લિફાયર ઇનપુટ/આઉટપુટ}
\item
  \textbf{ફિલ્ટર ડિઝાઇન}
\end{itemize}

\end{solutionbox}
\begin{mnemonicbox}
``સિંગલ સ્ટબ્સ સ્ટોપ સ્ટેન્ડિંગ વેવ્સ સક્સેસફુલી''

\end{mnemonicbox}
\begin{center}\rule{0.5\linewidth}{0.5pt}\end{center}

\subsection*{પ્રશ્ન 1(ક) વૈકલ્પિક [7
માર્ક્સ]}\label{uxaaauxab0uxab6uxaa8-1uxa95-uxab5uxa95uxab2uxaaauxa95-7-uxaaeuxab0uxa95uxab8}

\textbf{લંબચોરસ અને ગોળાકાર વેવગાઇડ્સની તુલના કરો.}

\begin{solutionbox}

\textbf{તુલના કોષ્ટક:}

{\def\LTcaptype{none} % do not increment counter
\begin{longtable}[]{@{}
  >{\raggedright\arraybackslash}p{(\linewidth - 4\tabcolsep) * \real{0.2115}}
  >{\raggedright\arraybackslash}p{(\linewidth - 4\tabcolsep) * \real{0.4231}}
  >{\raggedright\arraybackslash}p{(\linewidth - 4\tabcolsep) * \real{0.3654}}@{}}
\toprule\noalign{}
\begin{minipage}[b]{\linewidth}\raggedright
પેરામીટર
\end{minipage} & \begin{minipage}[b]{\linewidth}\raggedright
લંબચોરસ વેવગાઇડ
\end{minipage} & \begin{minipage}[b]{\linewidth}\raggedright
ગોળાકાર વેવગાઇડ
\end{minipage} \\
\midrule\noalign{}
\endhead
\bottomrule\noalign{}
\endlastfoot
\textbf{આકાર} & લંબચોરસ ક્રોસ-સેક્શન & ગોળાકાર ક્રોસ-સેક્શન \\
\textbf{ડોમિનન્ટ મોડ} & TE_{1}_{0} & TE_{1}_{1} \\
\textbf{કટઓફ ફ્રિક્વન્સી} & fc = c/(2a) for TE_{1}_{0} & fc = 1.841c/(2πa) for
TE_{1}_{1} \\
\textbf{પાવર હેન્ડલિંગ} & ઓછું & વધારે \\
\textbf{મેન્યુફેક્ચરિંગ} & સરળ & મુશ્કેલ \\
\textbf{મોડ સેપરેશન} & સારું & નબળું \\
\textbf{એપ્લિકેશન્સ} & રડાર, માઇક્રોવેવ ઓવન & સેટેલાઇટ કમ્યુનિકેશન \\
\end{longtable}
}

\textbf{મુખ્ય ફાયદાઓ:}

\begin{itemize}
\tightlist
\item
  \textbf{લંબચોરસ}: બહેતર મોડ નિયંત્રણ, સરળ ફેબ્રિકેશન
\item
  \textbf{ગોળાકાર}: વધારે પાવર ક્ષમતા, રોટેટિંગ પોલરાઇઝેશન
\end{itemize}

\end{solutionbox}
\begin{mnemonicbox}
``રેક્ટેંગ્યુલર ઇઝ રેગ્યુલર, સર્ક્યુલર કેરીઝ કરન્ટ''

\end{mnemonicbox}
\begin{center}\rule{0.5\linewidth}{0.5pt}\end{center}

\subsection*{પ્રશ્ન 2(અ) [3
માર્ક્સ]}\label{uxaaauxab0uxab6uxaa8-2uxa85-3-uxaaeuxab0uxa95uxab8}

\textbf{ગ્રુપ વેલોસિટી અને ફેઝ વેલોસિટીની વ્યાખ્યા કરો અને વચ્ચેનો સંબંધ લખો.}

\begin{solutionbox}

\textbf{વેગની વ્યાખ્યાઓ:}

{\def\LTcaptype{none} % do not increment counter
\begin{longtable}[]{@{}lll@{}}
\toprule\noalign{}
વેગનો પ્રકાર & ફોર્મ્યુલા & ભૌતિક અર્થ \\
\midrule\noalign{}
\endhead
\bottomrule\noalign{}
\endlastfoot
\textbf{ફેઝ વેલોસિટી} & v_{p} = ω/β = c/\sqrt(1-(fc/f)^{2}) & સ્થિર ફેઝની ઝડપ \\
\textbf{ગ્રુપ વેલોસિટી} & v_{m} = dω/dβ = c\sqrt(1-(fc/f)^{2}) & સિગ્નલ એનર્જીની ઝડપ \\
\end{longtable}
}

\textbf{સંબંધ:} v_{p} \times v_{m} = c^{2}

\textbf{મુખ્ય મુદ્દાઓ:}

\begin{itemize}
\tightlist
\item
  \textbf{ફેઝ વેલોસિટી}: હંમેશા \textgreater{} c (પ્રકાશની ઝડપ)
\item
  \textbf{ગ્રુપ વેલોસિટી}: હંમેશા \textless{} c
\item
  \textbf{સિગ્નલ પ્રવાસ}: ગ્રુપ વેલોસિટી પર
\end{itemize}

\end{solutionbox}
\begin{mnemonicbox}
``ફેઝ ઇઝ ફાસ્ટ, ગ્રુપ કેરીઝ મેસેજ''

\end{mnemonicbox}
\begin{center}\rule{0.5\linewidth}{0.5pt}\end{center}

\subsection*{પ્રશ્ન 2(બ) [4
માર્ક્સ]}\label{uxaaauxab0uxab6uxaa8-2uxaac-4-uxaaeuxab0uxa95uxab8}

\textbf{ડાયરેક્શનલ કપ્લરના સિદ્ધાંતો અને કાર્યનું વર્ણન કરો.}

\begin{solutionbox}

\textbf{ડાયરેક્શનલ કપ્લર સિદ્ધાંત:}

\begin{center}
\textbf{Mermaid Diagram (Code)}
\begin{verbatim}
{Shaded}
{Highlighting}[]
graph TD
    A[પોર્ટ 1 {- ઇનપુટ] {-}{-}{} B[મેઇન લાઇન]}
    B {-{-}{} C[પોર્ટ 2 {-} થ્રૂ]}
    B {-{-}{} D[પોર્ટ 3 {-} કપલ્ડ]}
    E[પોર્ટ 4 {- આઇસોલેટેડ] {-}{-}{} F[ટર્મિનેટેડ]}
{Highlighting}
{Shaded}
\end{verbatim}
\end{center}

\textbf{કાર્ય સિદ્ધાંત:}

\begin{itemize}
\tightlist
\item
  \textbf{ઇલેક્ટ્રોમેગ્નેટિક કપલિંગ} બે ટ્રાન્સમિશન લાઇન વચ્ચે
\item
  \textbf{પાવર વિભાજન} કપલિંગ ફેક્ટર આધારિત
\item
  \textbf{દિશાત્મક સંવેદનશીલતા} તરંગ દિશા તરફ
\end{itemize}

\textbf{મુખ્ય પેરામીટર્સ:}

\begin{itemize}
\tightlist
\item
  \textbf{કપલિંગ ફેક્ટર}: C = 10 log(P_{1}/P_{3}) dB
\item
  \textbf{ડાયરેક્ટિવિટી}: D = 10 log(P_{3}/P_{4}) dB
\item
  \textbf{ઇન્સર્શન લોસ}: IL = 10 log(P_{1}/P_{2}) dB
\end{itemize}

\end{solutionbox}
\begin{mnemonicbox}
``ડાયરેક્શનલ કપ્લર્સ ડિવાઇડ પાવર પ્રિસાઇસલી''

\end{mnemonicbox}
\begin{center}\rule{0.5\linewidth}{0.5pt}\end{center}

\subsection*{પ્રશ્ન 2(ક) [7
માર્ક્સ]}\label{uxaaauxab0uxab6uxaa8-2uxa95-7-uxaaeuxab0uxa95uxab8}

\textbf{બાંધકામ, ઓપરેશન અને એપ્લિકેશન સાથે મેજિક TEE સમજાવો.}

\begin{solutionbox}

\textbf{મેજિક TEE બાંધકામ:}

\begin{verbatim}
         E{-Arm (Port 3)}
              |
              |
    Port 1{-{-}{-}{-}+{-}{-}{-}{-}Port 2}
              |
              |
         H{-Arm (Port 4)}
\end{verbatim}

\textbf{ઓપરેટિંગ સિદ્ધાંતો:}

{\def\LTcaptype{none} % do not increment counter
\begin{longtable}[]{@{}lll@{}}
\toprule\noalign{}
પોર્ટ & કાર્ય & ફીલ્ડ પેટર્ન \\
\midrule\noalign{}
\endhead
\bottomrule\noalign{}
\endlastfoot
\textbf{પોર્ટ 1 અને 2} & કોલિનિયર પોર્ટ્સ & સિમેટ્રિક \\
\textbf{પોર્ટ 3 (E-આર્મ)} & E-પ્લેન પોર્ટ & ઇલેક્ટ્રિક ફીલ્ડ કપલિંગ \\
\textbf{પોર્ટ 4 (H-આર્મ)} & H-પ્લેન પોર્ટ & મેગ્નેટિક ફીલ્ડ કપલિંગ \\
\end{longtable}
}

\textbf{સ્કેટરિંગ ગુણધર્મો:}

\begin{itemize}
\tightlist
\item
  \textbf{આઇસોલેશન}: પોર્ટ 3 \leftrightarrow પોર્ટ 4
\item
  \textbf{પાવર વિભાજન}: મેચ થયું હોય ત્યારે સમાન વિભાજન
\item
  \textbf{ફેઝ સંબંધો}: 0^\circ અને 180^\circ
\end{itemize}

\textbf{એપ્લિકેશન્સ:}

\begin{itemize}
\tightlist
\item
  \textbf{મિક્સર્સ અને મોડ્યુલેટર્સ}
\item
  \textbf{પાવર કમ્બાઇનર્સ}
\item
  \textbf{ઇમ્પિડન્સ બ્રિજ}
\item
  \textbf{એન્ટીના ફીડ્સ}
\end{itemize}

\end{solutionbox}
\begin{mnemonicbox}
``મેજિક TEE ક્રિએટ્સ પરફેક્ટ આઇસોલેશન''

\end{mnemonicbox}
\begin{center}\rule{0.5\linewidth}{0.5pt}\end{center}

\subsection*{પ્રશ્ન 2(અ) વૈકલ્પિક [3
માર્ક્સ]}\label{uxaaauxab0uxab6uxaa8-2uxa85-uxab5uxa95uxab2uxaaauxa95-3-uxaaeuxab0uxa95uxab8}

\textbf{લંબચોરસ વેવગાઇડ માટે TE_{1}_{0}, TE_{2}_{0} મોડ્સ દોરો.}

\begin{solutionbox}

\textbf{TE_{1}_{0} મોડ (ડોમિનન્ટ મોડ):}

\begin{verbatim}
  a
+{-{-}{-}{-}{-}{-}{-}{-}{-}{-}{-}{-}{-}+}
|      \^{      | b}
|   E  \^{  E   |}
|      \^{      |}
+{-{-}{-}{-}{-}{-}{-}{-}{-}{-}{-}{-}{-}+}
  Field Lines
\end{verbatim}

\textbf{TE_{2}_{0} મોડ:}

\begin{verbatim}
  a
+{-{-}{-}{-}{-}{-}{-}{-}{-}{-}{-}{-}{-}+}
|  \^{     v    | b}
|  \^{  E  v  E |}
|  \^{     v    |}
+{-{-}{-}{-}{-}{-}{-}{-}{-}{-}{-}{-}{-}+}
  Two Half{-Waves}
\end{verbatim}

\textbf{મોડ લાક્ષણિકતાઓ:}

\begin{itemize}
\tightlist
\item
  \textbf{TE_{1}_{0}}: x-દિશામાં એક હાફ-વેવ વેરિએશન
\item
  \textbf{TE_{2}_{0}}: x-દિશામાં બે હાફ-વેવ વેરિએશન
\item
  \textbf{ફીલ્ડ પેટર્ન}: ઇલેક્ટ્રિક ફીલ્ડ પ્રોપેગેશન પર લંબ
\end{itemize}

\end{solutionbox}
\begin{mnemonicbox}
``TE મોડ્સ હેવ ઇલેક્ટ્રિક ટ્રાન્સવર્સ''

\end{mnemonicbox}
\begin{center}\rule{0.5\linewidth}{0.5pt}\end{center}

\subsection*{પ્રશ્ન 2(બ) વૈકલ્પિક [4
માર્ક્સ]}\label{uxaaauxab0uxab6uxaa8-2uxaac-uxab5uxa95uxab2uxaaauxa95-4-uxaaeuxab0uxa95uxab8}

\textbf{જરૂરી સ્કેચ સાથે હાઇબ્રિડ રિંગનું વર્ણન કરો.}

\begin{solutionbox}

\textbf{હાઇબ્રિડ રિંગ સ્ટ્રક્ચર:}

\begin{center}
\textbf{Mermaid Diagram (Code)}
\begin{verbatim}
{Shaded}
{Highlighting}[]
graph TD
    A[પોર્ટ 1] {-{-}{-} B[રિંગ સ્ટ્રક્ચર]}
    C[પોર્ટ 2] {-{-}{-} B}
    D[પોર્ટ 3] {-{-}{-} B}
    E[પોર્ટ 4] {-{-}{-} B}
    B {-{-}{-} F[3λ/2 circumference]}
{Highlighting}
{Shaded}
\end{verbatim}
\end{center}

\textbf{ઓપરેટિંગ સિદ્ધાંત:}

\begin{itemize}
\tightlist
\item
  \textbf{રિંગ સર્કમફરન્સ}: 3λ/2
\item
  \textbf{પોર્ટ સ્પેસિંગ}: λ/4 અંતરે
\item
  \textbf{પાવર વિભાજન}: એડજેસન્ટ પોર્ટ્સ વચ્ચે સમાન વિભાજન
\end{itemize}

\textbf{મુખ્ય લક્ષણો:}

\begin{itemize}
\tightlist
\item
  \textbf{આઇસોલેશન}: વિરુદ્ધ પોર્ટ્સ વચ્ચે
\item
  \textbf{ફેઝ સંબંધો}: 0^\circ અને 180^\circ
\item
  \textbf{ઇમ્પિડન્સ}: બધા પોર્ટ્સ પર મેચ
\end{itemize}

\end{solutionbox}
\begin{mnemonicbox}
``હાઇબ્રિડ રિંગ્સ હેન્ડલ હાફ-વેવલેન્થ્સ''

\end{mnemonicbox}
\begin{center}\rule{0.5\linewidth}{0.5pt}\end{center}

\subsection*{પ્રશ્ન 2(ક) વૈકલ્પિક [7
માર્ક્સ]}\label{uxaaauxab0uxab6uxaa8-2uxa95-uxab5uxa95uxab2uxaaauxa95-7-uxaaeuxab0uxa95uxab8}

\textbf{સિદ્ધાંતો, બાંધકામ અને ઓપરેશન સાથે આઇસોલેટર સમજાવો.}

\begin{solutionbox}

\textbf{આઇસોલેટર સિદ્ધાંત:}

\begin{center}
\textbf{Mermaid Diagram (Code)}
\begin{verbatim}
{Shaded}
{Highlighting}[]
graph LR
    A[Input] {-{-}{} B[Ferrite Material]}
    B {-{-}{} C[Output]}
    C {-.{-}{}|Blocked| B}
    D[Magnetic Field] {-{-}{} B}
{Highlighting}
{Shaded}
\end{verbatim}
\end{center}

\textbf{બાંધકામ એલિમેન્ટ્સ:}

{\def\LTcaptype{none} % do not increment counter
\begin{longtable}[]{@{}lll@{}}
\toprule\noalign{}
કોમ્પોનન્ટ & કાર્ય & મટીરિયલ \\
\midrule\noalign{}
\endhead
\bottomrule\noalign{}
\endlastfoot
\textbf{ફેરાઇટ} & નોન-રેસિપ્રોકલ મીડિયમ & Yttrium Iron Garnet \\
\textbf{મેગ્નેટ} & બાયાસ ફીલ્ડ & પર્મેનન્ટ મેગ્નેટ \\
\textbf{રેઝિસ્ટિવ લોડ} & રિવર્સ પાવર એબસોર્બ & કાર્બન/સિરામિક \\
\end{longtable}
}

\textbf{ઓપરેટિંગ સિદ્ધાંત:}

\begin{itemize}
\tightlist
\item
  \textbf{ફેરાડે રોટેશન} મેગ્નેટાઇઝ્ડ ફેરાઇટમાં
\item
  \textbf{નોન-રેસિપ્રોકલ} ફેઝ શિફ્ટ
\item
  \textbf{ફોરવર્ડ ટ્રાન્સમિશન}: લો લોસ
\item
  \textbf{રિવર્સ ટ્રાન્સમિશન}: હાઇ એટેન્યુએશન
\end{itemize}

\textbf{એપ્લિકેશન્સ:}

\begin{itemize}
\tightlist
\item
  \textbf{એમ્પ્લિફાયર પ્રોટેક્શન}
\item
  \textbf{ઓસિલેટર આઇસોલેશન}
\item
  \textbf{એન્ટીના સિસ્ટમ્સ}
\end{itemize}

\textbf{સ્પેસિફિકેશન્સ:}

\begin{itemize}
\tightlist
\item
  \textbf{આઇસોલેશન}: 20-30 dB સામાન્ય
\item
  \textbf{ઇન્સર્શન લોસ}: \textless{} 0.5 dB
\end{itemize}

\end{solutionbox}
\begin{mnemonicbox}
``આઇસોલેટર્સ ઇગ્નોર રિવર્સ રિફ્લેક્શન્સ''

\end{mnemonicbox}
\begin{center}\rule{0.5\linewidth}{0.5pt}\end{center}

\subsection*{પ્રશ્ન 3(અ) [3
માર્ક્સ]}\label{uxaaauxab0uxab6uxaa8-3uxa85-3-uxaaeuxab0uxa95uxab8}

\textbf{ટ્રાવેલિંગ વેવ ટ્યુબ એમ્પ્લિફાયર દોરો.}

\begin{solutionbox}

\textbf{TWT એમ્પ્લિફાયર સ્ટ્રક્ચર:}

\begin{verbatim}
Electron Gun    Helix Structure    Collector
     |               |                |
     v               v                v
    [|]{-{-}{-}  {-}{-}|}
         Electron    RF Input         RF Output
         Beam        Coupler          Coupler
                        |
                   Attenuator
\end{verbatim}

\textbf{મુખ્ય કોમ્પોનન્ટ્સ:}

\begin{itemize}
\tightlist
\item
  \textbf{ઇલેક્ટ્રોન ગન}: ઇલેક્ટ્રોન બીમ પેદા કરે છે
\item
  \textbf{હેલિક્સ}: સ્લો-વેવ સ્ટ્રક્ચર
\item
  \textbf{કપ્લર્સ}: ઇનપુટ/આઉટપુટ RF કનેક્શન્સ
\item
  \textbf{કલેક્ટર}: ખર્ચાયેલા ઇલેક્ટ્રોન્સ એકત્રિત કરે છે
\end{itemize}

\end{solutionbox}
\begin{mnemonicbox}
``TWT ટ્રાન્સફર્સ વેવ થ્રૂ હેલિક્સ''

\end{mnemonicbox}
\begin{center}\rule{0.5\linewidth}{0.5pt}\end{center}

\subsection*{પ્રશ્ન 3(બ) [4
માર્ક્સ]}\label{uxaaauxab0uxab6uxaa8-3uxaac-4-uxaaeuxab0uxa95uxab8}

\textbf{માઇક્રોવેવ રેડિયેશનને કારણે વિવિધ પ્રકારના જોખમોનું વર્ણન કરો.}

\begin{solutionbox}

\textbf{માઇક્રોવેવ રેડિયેશન જોખમો:}

{\def\LTcaptype{none} % do not increment counter
\begin{longtable}[]{@{}lll@{}}
\toprule\noalign{}
જોખમનો પ્રકાર & અસરો & સેફ્ટી લિમિટ \\
\midrule\noalign{}
\endhead
\bottomrule\noalign{}
\endlastfoot
\textbf{HERP} (Personnel) & ટિશ્યુ હીટિંગ, બર્ન્સ & 10 mW/cm^{2} \\
\textbf{HERO} (Ordnance) & વિસ્ફોટક વિસ્ફોટ & વેરિયેબલ \\
\textbf{HERF} (Fuel) & ફ્યુઅલ ઇગ્નિશન & 5 mW/cm^{2} \\
\end{longtable}
}

\textbf{જૈવિક અસરો:}

\begin{itemize}
\tightlist
\item
  \textbf{થર્મલ અસરો}: 41^\circC થી વધારે ટિશ્યુ હીટિંગ
\item
  \textbf{નોન-થર્મલ અસરો}: કોશિકા નુકસાન
\item
  \textbf{સંવેદનશીલ અંગો}: આંખો, પ્રજનન અંગો
\end{itemize}

\textbf{સુરક્ષા પગલાં:}

\begin{itemize}
\tightlist
\item
  \textbf{શીલ્ડિંગ}: કન્ડક્ટિવ એન્ક્લોઝર્સ
\item
  \textbf{અંતર}: પાવર ડેન્સિટી ∝ 1/r^{2}
\item
  \textbf{સમય મર્યાદા}: એક્સપોઝર ડ્યુરેશન નિયંત્રણ
\item
  \textbf{ચેતવણી સિસ્ટમ}: રેડિયેશન ડિટેક્ટર્સ
\end{itemize}

\end{solutionbox}
\begin{mnemonicbox}
``હીટ એનર્જી રિક્વાયર્સ પ્રોપર પ્રોટેક્શન''

\end{mnemonicbox}
\begin{center}\rule{0.5\linewidth}{0.5pt}\end{center}

\subsection*{પ્રશ્ન 3(ક) [7
માર્ક્સ]}\label{uxaaauxab0uxab6uxaa8-3uxa95-7-uxaaeuxab0uxa95uxab8}

\textbf{એપલગેટ ડાયાગ્રામ સાથે બે કેવિટી ક્લાયસ્ટ્રોન બાંધકામ અને ઓપરેશન સમજાવો.}

\begin{solutionbox}

\textbf{બે-કેવિટી ક્લાયસ્ટ્રોન સ્ટ્રક્ચર:}

\begin{center}
\textbf{Mermaid Diagram (Code)}
\begin{verbatim}
{Shaded}
{Highlighting}[]
graph LR
    A[કેથોડ] {-{-}{} B[ઇનપુટ કેવિટી]}
    B {-{-}{} C[ડ્રિફ્ટ સ્પેસ]}
    C {-{-}{} D[આઉટપુટ કેવિટી]}
    D {-{-}{} E[કલેક્ટર]}
    F[RF ઇનપુટ] {-{-}{} B}
    D {-{-}{} G[RF આઉટપુટ]}
{Highlighting}
{Shaded}
\end{verbatim}
\end{center}

\textbf{એપલગેટ ડાયાગ્રામ:}

\begin{verbatim}
Velocity
   \^{}
   |    Bunched    Bunched
   |   /      {   /      }
v0 +{-{-}+        {-}/        {-}{-}}
   |   {        /        /}
   |    Bunched    Bunched
   |
   +{-{-}{-}{-}{-}{-}{-}{-}{-}{-}{-}{-}{-}{-}{-}{-}{-}{-}{-}{-}{-}{-}{-}{-}{-} Distance}
   Input   Drift    Output
   Cavity  Space    Cavity
\end{verbatim}

\textbf{ઓપરેશન સિદ્ધાંત:}

{\def\LTcaptype{none} % do not increment counter
\begin{longtable}[]{@{}lll@{}}
\toprule\noalign{}
સ્ટેજ & પ્રક્રિયા & પરિણામ \\
\midrule\noalign{}
\endhead
\bottomrule\noalign{}
\endlastfoot
\textbf{વેલોસિટી મોડ્યુલેશન} & RF ઇનપુટ ઇલેક્ટ્રોન સ્પીડ બદલે છે & સ્પીડ વેરિએશન \\
\textbf{બંચિંગ} & ઝડપી ઇલેક્ટ્રોન્સ ધીમા ઇલેક્ટ્રોન્સને પકડે છે & કરન્ટ બંચ \\
\textbf{એનર્જી એક્સટ્રેક્શન} & બંચ આઉટપુટ કેવિટી સાથે ઇન્ટરેક્ટ કરે છે & RF
એમ્પ્લિફિકેશન \\
\end{longtable}
}

\textbf{મુખ્ય પેરામીટર્સ:}

\begin{itemize}
\tightlist
\item
  \textbf{ટ્રાન્ઝિટ ટાઇમ}: બંચિંગ માટે મહત્વપૂર્ણ
\item
  \textbf{ડ્રિફ્ટ સ્પેસ લંબાઇ}: મહત્તમ બંચિંગ માટે ઓપ્ટિમાઇઝ
\item
  \textbf{કેવિટી ટ્યુનિંગ}: રેઝોનન્ટ ફ્રીક્વન્સી મેચિંગ
\end{itemize}

\textbf{એપ્લિકેશન્સ:}

\begin{itemize}
\tightlist
\item
  \textbf{રડાર ટ્રાન્સમિટર્સ}
\item
  \textbf{સેટેલાઇટ કમ્યુનિકેશન્સ}
\item
  \textbf{લિનિયર એક્સેલેરેટર્સ}
\end{itemize}

\end{solutionbox}
\begin{mnemonicbox}
``ક્લાયસ્ટ્રોન્સ ક્રિએટ બંચ થ્રૂ વેલોસિટી વેરિએશન''

\end{mnemonicbox}
\begin{center}\rule{0.5\linewidth}{0.5pt}\end{center}

\subsection*{પ્રશ્ન 3(અ) વૈકલ્પિક [3
માર્ક્સ]}\label{uxaaauxab0uxab6uxaa8-3uxa85-uxab5uxa95uxab2uxaaauxa95-3-uxaaeuxab0uxa95uxab8}

\textbf{માઇક્રોવેવ આવૃત્તિ માટે એટેન્યુએશન માપન પદ્ધતિનો બ્લોક ડાયાગ્રામ દોરો.}

\begin{solutionbox}

\textbf{એટેન્યુએશન માપન સેટઅપ:}

\begin{center}
\textbf{Mermaid Diagram (Code)}
\begin{verbatim}
{Shaded}
{Highlighting}[]
graph LR
    A[સિગ્નલ જનરેટર] {-{-}{} B[ડાયરેક્શનલ કપ્લર]}
    B {-{-}{} C[ડિવાઇસ અંડર ટેસ્ટ]}
    C {-{-}{} D[પાવર મીટર]}
    B {-{-}{} E[રેફરન્સ પાવર મીટર]}
    F[ડિસ્પ્લે યુનિટ] {-{-}{} G[એટેન્યુએશન રીડિંગ]}
    D {-{-}{} F}
    E {-{-}{} F}
{Highlighting}
{Shaded}
\end{verbatim}
\end{center}

\textbf{માપન પ્રક્રિયા:}

\begin{itemize}
\tightlist
\item
  \textbf{રેફરન્સ માપ}: DUT વિના
\item
  \textbf{ઇન્સર્શન માપ}: DUT સાથે
\item
  \textbf{એટેન્યુએશન કેલ્ક્યુલેશન}: A = P_{1} - P_{2} (dB)
\end{itemize}

\end{solutionbox}
\begin{mnemonicbox}
``એટેન્યુએશન એપિયર્સ આફ્ટર એક્યુરેટ એસેસમેન્ટ''

\end{mnemonicbox}
\begin{center}\rule{0.5\linewidth}{0.5pt}\end{center}

\subsection*{પ્રશ્ન 3(બ) વૈકલ્પિક [4
માર્ક્સ]}\label{uxaaauxab0uxab6uxaa8-3uxaac-uxab5uxa95uxab2uxaaauxa95-4-uxaaeuxab0uxa95uxab8}

\textbf{માઇક્રોવેવ રેન્જ પર વેક્યુમ ટ્યુબની મર્યાદાનું વર્ણન કરો.}

\begin{solutionbox}

\textbf{વેક્યુમ ટ્યુબ મર્યાદાઓ:}

{\def\LTcaptype{none} % do not increment counter
\begin{longtable}[]{@{}lll@{}}
\toprule\noalign{}
મર્યાદા & કારણ & અસર \\
\midrule\noalign{}
\endhead
\bottomrule\noalign{}
\endlastfoot
\textbf{ટ્રાન્ઝિટ ટાઇમ} & ઇલેક્ટ્રોન મુસાફરીનો સમય & ઊંચી આવૃત્તિ પર ઘટતો ગેઇન \\
\textbf{લીડ ઇન્ડક્ટન્સ} & કનેક્ટિંગ વાયર ઇન્ડક્ટન્સ & નબળી ઇમ્પિડન્સ મેચિંગ \\
\textbf{ઇન્ટર-ઇલેક્ટ્રોડ કેપેસિટન્સ} & પ્લેટ-કેથોડ કેપેસિટન્સ & ફીડબેક અને અસ્થિરતા \\
\textbf{સ્કિન ઇફેક્ટ} & હાઇ-ફ્રીક્વન્સી કરન્ટ વિતરણ & વધતો પ્રતિકાર \\
\end{longtable}
}

\textbf{આવૃત્તિ-સંબંધિત સમસ્યાઓ:}

\begin{itemize}
\tightlist
\item
  \textbf{ઇનપુટ ઇમ્પિડન્સ}: રિએક્ટિવ બને છે
\item
  \textbf{ગેઇન-બેન્ડવિડ્થ}: પ્રોડક્ટ મર્યાદા
\item
  \textbf{નોઇઝ ફિગર}: આવૃત્તિ સાથે વધે છે
\item
  \textbf{પાવર હેન્ડલિંગ}: ઘટે છે
\end{itemize}

\textbf{સોલ્યુશન્સ:}

\begin{itemize}
\tightlist
\item
  \textbf{સ્પેશિયલ ટ્યુબ ડિઝાઇન}: લાઇટહાઉસ ટ્યુબ્સ
\item
  \textbf{કેવિટી રેઝોનેટર્સ}: ટ્યુન્ડ સર્કિટ રિપ્લેસ કરે છે
\item
  \textbf{શોર્ટ લીડ્સ}: ઇન્ડક્ટન્સ મિનિમાઇઝ કરે છે
\end{itemize}

\end{solutionbox}
\begin{mnemonicbox}
``વેક્યુમ ટ્યુબ્સ ફેઇલ ફાસ્ટ એટ હાઇ ફ્રીક્વન્સીઝ''

\end{mnemonicbox}
\begin{center}\rule{0.5\linewidth}{0.5pt}\end{center}

\subsection*{પ્રશ્ન 3(ક) વૈકલ્પિક [7
માર્ક્સ]}\label{uxaaauxab0uxab6uxaa8-3uxa95-uxab5uxa95uxab2uxaaauxa95-7-uxaaeuxab0uxa95uxab8}

\textbf{મેગ્નેટ્રોનના સિદ્ધાંત, બાંધકામ, ઇલેક્ટ્રિક અને મેગ્નેટિક ફીલ્ડની અસર અને ઓપરેશન
વિગતવાર સમજાવો.}

\begin{solutionbox}

\textbf{મેગ્નેટ્રોન બાંધકામ:}

\begin{verbatim}
        Anode Vanes
    +{-{-}{-}{-}{-}+{-}{-}{-}{-}{-}+{-}{-}{-}{-}{-}+}
   /   1  |  2  |  3   {}
  /       |     |       {}
 /    8   |  C  |   4    {}
|         |     |         |
|    7    |  +  |    5    |
 {        |     |        /}
  {   6   |     |       /}
   {{-}{-}{-}{-}{-}+{-}{-}{-}{-}{-}+{-}{-}{-}{-}{-}/}
        Cathode (C)
\end{verbatim}

\textbf{ઓપરેટિંગ સિદ્ધાંત:}

{\def\LTcaptype{none} % do not increment counter
\begin{longtable}[]{@{}lll@{}}
\toprule\noalign{}
ફીલ્ડ & દિશા & અસર \\
\midrule\noalign{}
\endhead
\bottomrule\noalign{}
\endlastfoot
\textbf{ઇલેક્ટ્રિક ફીલ્ડ} & રેડિયલ (કેથોડથી એનોડ) & ઇલેક્ટ્રોન્સને એક્સેલેરેટ કરે છે \\
\textbf{મેગ્નેટિક ફીલ્ડ} & એક્સિયલ (પેજ પર લંબ) & ઇલેક્ટ્રોન્સને ડિફ્લેક્ટ કરે છે \\
\textbf{સંયુક્ત અસર} & સાયક્લોઇડ મોશન & ફેઝ સિંક્રોનાઇઝેશન \\
\end{longtable}
}

\textbf{ઓપરેશન સ્ટેજો:}

\begin{enumerate}
\tightlist
\item
  \textbf{ઇલેક્ટ્રોન ઇમિશન}: ગરમ કેથોડ ઇલેક્ટ્રોન્સ બહાર કાઢે છે
\item
  \textbf{સાયક્લોઇડ મોશન}: E\timesB ફીલ્ડ્સ સ્પાયરલ પાથ બનાવે છે
\item
  \textbf{સિંક્રોનાઇઝેશન}: ઇલેક્ટ્રોન્સ RF ફીલ્ડ સાથે સિંક્રોનાઇઝ કરે છે
\item
  \textbf{એનર્જી ટ્રાન્સફર}: કાઇનેટિક એનર્જી \rightarrow RF એનર્જી
\item
  \textbf{આઉટપુટ કપલિંગ}: વેવગાઇડ દ્વારા RF એક્ષ્ટ્રેક્ટ કરવામાં આવે છે
\end{enumerate}

\textbf{મુખ્ય પેરામીટર્સ:}

\begin{itemize}
\tightlist
\item
  \textbf{મેગ્નેટિક ફ્લક્સ ડેન્સિટી}: B = 2πmf/e
\item
  \textbf{હલ કટઓફ વોલ્ટેજ}: VH = (eB^{2}R^{2})/(8m)
\item
  \textbf{આવૃત્તિ}: f = eB/(2πm) \times (એનોડ મોડ્સ)
\end{itemize}

\textbf{એપ્લિકેશન્સ:}

\begin{itemize}
\tightlist
\item
  \textbf{માઇક્રોવેવ ઓવન્સ} (2.45 GHz)
\item
  \textbf{રડાર ટ્રાન્સમિટર્સ}
\item
  \textbf{ઇન્ડસ્ટ્રિયલ હીટિંગ}
\end{itemize}

\end{solutionbox}
\begin{mnemonicbox}
``મેગ્નેટ્રોન્સ મેક માઇક્રોવેવ્સ થ્રૂ મેગ્નેટિક મોશન''

\end{mnemonicbox}
\begin{center}\rule{0.5\linewidth}{0.5pt}\end{center}

\subsection*{પ્રશ્ન 4(અ) [3
માર્ક્સ]}\label{uxaaauxab0uxab6uxaa8-4uxa85-3-uxaaeuxab0uxa95uxab8}

\textbf{ગ્રાફનો ઉપયોગ કરીને વેરેક્ટર ડાયોડના કાર્ય સિદ્ધાંતને સમજાવો.}

\begin{solutionbox}

\textbf{વેરેક્ટર ડાયોડ લાક્ષણિકતાઓ:}

\begin{verbatim}
Capacitance (pF)
      \^{}
      |     
   100|{    }
      | {   }
    50|  {  }
      |   { }
    10|    {}
      |     {\_\_\_\_\_}
      +{-{-}{-}{-}{-}{-}{-}{-}{-}{-}{-}{-}{-}{-} Reverse Voltage (V)}
      0   5   10   15
\end{verbatim}

\textbf{કાર્ય સિદ્ધાંત:}

\begin{itemize}
\tightlist
\item
  \textbf{રિવર્સ બાયાસ ઓપરેશન}: ડાયોડ રિવર્સમાં ઓપરેટ કરે છે
\item
  \textbf{ડિપ્લેશન લેયર}: ડાયલેક્ટ્રિક તરીકે કામ કરે છે
\item
  \textbf{વેરિયેબલ કેપેસિટન્સ}: C ∝ 1/\sqrtVR
\item
  \textbf{વોલ્ટેજ ટ્યુનિંગ}: વોલ્ટેજ દ્વારા કેપેસિટન્સ નિયંત્રિત
\end{itemize}

\textbf{એપ્લિકેશન્સ:}

\begin{itemize}
\tightlist
\item
  \textbf{વોલ્ટેજ-કંટ્રોલડ ઓસિલેટર્સ}
\item
  \textbf{ફ્રીક્વન્સી મલ્ટિપ્લાયર્સ}
\item
  \textbf{પેરામેટ્રિક એમ્પ્લિફાયર્સ}
\end{itemize}

\end{solutionbox}
\begin{mnemonicbox}
``વેરેક્ટર્સ વેરી કેપેસિટન્સ વાયા વોલ્ટેજ''

\end{mnemonicbox}
\begin{center}\rule{0.5\linewidth}{0.5pt}\end{center}

\subsection*{પ્રશ્ન 4(બ) [4
માર્ક્સ]}\label{uxaaauxab0uxab6uxaa8-4uxaac-4-uxaaeuxab0uxa95uxab8}

\textbf{ગન ડાયોડ માટે ગન અસર અને નકારાત્મક અવરોધકતા સમજાવો.}

\begin{solutionbox}

\textbf{ગન અસર મિકેનિઝમ:}

{\def\LTcaptype{none} % do not increment counter
\begin{longtable}[]{@{}lll@{}}
\toprule\noalign{}
પેરામીટર & લોઅર વેલી & અપર વેલી \\
\midrule\noalign{}
\endhead
\bottomrule\noalign{}
\endlastfoot
\textbf{એનર્જી લેવલ} & લોઅર & હાયર \\
\textbf{ઇલેક્ટ્રોન મોબિલિટી} & હાઇ (μ_{1}) & લો (μ_{2}) \\
\textbf{ઇફેક્ટિવ માસ} & લાઇટ & હેવી \\
\end{longtable}
}

\textbf{ટ્રાન્સફર લક્ષણ:}

\begin{verbatim}
Current (mA)
      \^{}
      |   /{}
      |  /  { Negative}
      | /    { Resistance}
      |/      { Region}
      +{-{-}{-}{-}{-}{-}{-}{-}{-}{-} Voltage (V)}
             Threshold
\end{verbatim}

\textbf{નકારાત્મક અવરોધકતા:}

\begin{itemize}
\tightlist
\item
  \textbf{થ્રેશોલ્ડ વોલ્ટેજ}: ઇલેક્ટ્રોન્સ અપર વેલીમાં ટ્રાન્સફર કરે છે
\item
  \textbf{કરન્ટ ઘટાડો}: ઘટતી મોબિલિટીને કારણે
\item
  \textbf{ઓસિલેશન}: નકારાત્મક અવરોધકતા સક્ષમ કરે છે
\item
  \textbf{ડોમેઇન ફોર્મેશન}: હાઇ-ફીલ્ડ ડોમેઇન્સ પ્રોપેગેટ કરે છે
\end{itemize}

\textbf{મુખ્ય મુદ્દાઓ:}

\begin{itemize}
\tightlist
\item
  \textbf{મટીરિયલ્સ}: GaAs, InP
\item
  \textbf{આવૃત્તિ રેન્જ}: 1-100 GHz
\item
  \textbf{કાર્યક્ષમતા}: 5-20\%
\end{itemize}

\end{solutionbox}
\begin{mnemonicbox}
``ગન ડાયોડ્સ જનરેટ ઓસિલેશન્સ થ્રૂ નેગેટિવ રેઝિસ્ટન્સ''

\end{mnemonicbox}
\begin{center}\rule{0.5\linewidth}{0.5pt}\end{center}

\subsection*{પ્રશ્ન 4(ક) [7
માર્ક્સ]}\label{uxaaauxab0uxab6uxaa8-4uxa95-7-uxaaeuxab0uxa95uxab8}

\textbf{માઇક્રોવેવ આવૃત્તિ માટે આવૃત્તિ માપન પદ્ધતિ સમજાવો.}

\begin{solutionbox}

\textbf{ડાયરેક્ટ ફ્રીક્વન્સી માપ:}

\begin{center}
\textbf{Mermaid Diagram (Code)}
\begin{verbatim}
{Shaded}
{Highlighting}[]
graph LR
    A[અજ્ઞાત સિગ્નલ] {-{-}{} B[ફ્રીક્વન્સી કાઉન્ટર]}
    B {-{-}{} C[ડિસ્પ્લે]}
    D[રેફરન્સ ઓસિલેટર] {-{-}{} B}
{Highlighting}
{Shaded}
\end{verbatim}
\end{center}

\textbf{અપ્રત્યક્ષ પદ્ધતિઓ:}

{\def\LTcaptype{none} % do not increment counter
\begin{longtable}[]{@{}lll@{}}
\toprule\noalign{}
પદ્ધતિ & સિદ્ધાંત & ચોકસાઈ \\
\midrule\noalign{}
\endhead
\bottomrule\noalign{}
\endlastfoot
\textbf{વેવમીટર} & કેવિટી રેઝોનન્સ & \pm0.1\% \\
\textbf{બીટ ફ્રીક્વન્સી} & હેટેરોડાયન મિક્સિંગ & \pm0.01\% \\
\textbf{સ્ટેન્ડિંગ વેવ} & λ/2 માપ & \pm0.5\% \\
\end{longtable}
}

\textbf{કેવિટી વેવમીટર સેટઅપ:}

\begin{verbatim}
   Waveguide
  +{-{-}{-}{-}{-}{-}{-}{-}{-}{-}{-}{-}{-}+}
  |    +{-{-}{-}+    |}
{-{-}+    | C |    +{-}{-} Output}
  |    +{-{-}{-}+    |}
  +{-{-}{-}{-}{-}{-}{-}{-}{-}{-}{-}{-}{-}+}
   Tuning Screw
\end{verbatim}

\textbf{માપન પ્રક્રિયા:}

\begin{enumerate}
\tightlist
\item
  \textbf{કપલિંગ}: સિગ્નલ લાઇન સાથે નબળી કપલિંગ
\item
  \textbf{ટ્યુનિંગ}: રેઝોનન્સ માટે કેવિટી એડજસ્ટ કરો
\item
  \textbf{ઇન્ડિકેશન}: મિનિમમ/મહત્તમ માટે આઉટપુટ મોનિટર કરો
\item
  \textbf{કેલિબ્રેશન}: કેલિબ્રેટેડ સ્કેલથી આવૃત્તિ વાંચો
\end{enumerate}

\textbf{બીટ ફ્રીક્વન્સી પદ્ધતિ:}

\begin{itemize}
\tightlist
\item
  \textbf{લોકલ ઓસિલેટર}: જાણીતી રેફરન્સ આવૃત્તિ
\item
  \textbf{મિક્સર}: બીટ ફ્રીક્વન્સી જનરેટ કરે છે
\item
  \textbf{માપ}: fbeat = \textbar fsignal - fLO\textbar{}
\end{itemize}

\end{solutionbox}
\begin{mnemonicbox}
``ફ્રીક્વન્સી ફાઉન્ડ થ્રૂ કેરફુલ કેવિટી કેલિબ્રેશન''

\end{mnemonicbox}
\begin{center}\rule{0.5\linewidth}{0.5pt}\end{center}

\subsection*{પ્રશ્ન 4(અ) વૈકલ્પિક [3
માર્ક્સ]}\label{uxaaauxab0uxab6uxaa8-4uxa85-uxab5uxa95uxab2uxaaauxa95-3-uxaaeuxab0uxa95uxab8}

\textbf{સ્વિચ તરીકે PIN ડાયોડનું કાર્ય સમજાવો.}

\begin{solutionbox}

\textbf{PIN ડાયોડ સ્ટ્રક્ચર:}

\begin{verbatim}
P+ Region | Intrinsic | N+ Region
    |         |           |
    +{-{-}{-}{-}{-}{-}{-}{-}{-}+{-}{-}{-}{-}{-}{-}{-}{-}{-}{-}{-}+}
   Holes   No Carriers  Electrons
\end{verbatim}

\textbf{સ્વિચિંગ ઓપરેશન:}

{\def\LTcaptype{none} % do not increment counter
\begin{longtable}[]{@{}llll@{}}
\toprule\noalign{}
બાયાસ સ્થિતિ & ઇન્ટ્રિન્સિક રીજન & RF ઇમ્પિડન્સ & સ્વિચ સ્થિતિ \\
\midrule\noalign{}
\endhead
\bottomrule\noalign{}
\endlastfoot
\textbf{ફોરવર્ડ બાયાસ} & કેરિયર્સથી ભરેલું & લો (\textasciitilde1Ω) & ON
(બંધ) \\
\textbf{રિવર્સ બાયાસ} & ડિપ્લીટેડ & હાઇ (\textasciitilde10kΩ) & OFF
(ખુલ્લું) \\
\textbf{ઝીરો બાયાસ} & અલ્પ કેરિયર્સ & મીડિયમ & વેરિયેબલ \\
\end{longtable}
}

\textbf{મુખ્ય ફાયદાઓ:}

\begin{itemize}
\tightlist
\item
  \textbf{ફાસ્ટ સ્વિચિંગ}: નેનોસેકંડ રિસ્પોન્સ
\item
  \textbf{લો ઇન્સર્શન લોસ}: જ્યારે ON હોય
\item
  \textbf{હાઇ આઇસોલેશન}: જ્યારે OFF હોય
\item
  \textbf{વાઇડ ફ્રીક્વન્સી રેન્જ}: DC થી માઇક્રોવેવ
\end{itemize}

\textbf{એપ્લિકેશન્સ:}

\begin{itemize}
\tightlist
\item
  \textbf{RF સ્વિચ}
\item
  \textbf{મોડ્યુલેટર્સ}
\item
  \textbf{એટેન્યુએટર્સ}
\item
  \textbf{ફેઝ શિફ્ટર્સ}
\end{itemize}

\end{solutionbox}
\begin{mnemonicbox}
``PIN ડાયોડ્સ પરફોર્મ પરફેક્ટ સ્વિચિંગ''

\end{mnemonicbox}
\begin{center}\rule{0.5\linewidth}{0.5pt}\end{center}

\subsection*{પ્રશ્ન 4(બ) વૈકલ્પિક [4
માર્ક્સ]}\label{uxaaauxab0uxab6uxaa8-4uxaac-uxab5uxa95uxab2uxaaauxa95-4-uxaaeuxab0uxa95uxab8}

\textbf{સ્ટ્રિપલાઇન અને માઇક્રોસ્ટ્રિપ સર્કિટ સમજાવો.}

\begin{solutionbox}

\textbf{સ્ટ્રિપલાઇન કન્ફિગરેશન:}

\begin{verbatim}
  Ground Plane
 {-{-}{-}{-}{-}{-}{-}{-}{-}{-}{-}{-}{-}{-}{-}}
     Dielectric
 {-{-}{-}{-}{-}+{-}{-}{-}{-}{-}{-}{-}{-}{-} {-} Signal Conductor}
     Dielectric
 {-{-}{-}{-}{-}{-}{-}{-}{-}{-}{-}{-}{-}{-}{-}}
  Ground Plane
\end{verbatim}

\textbf{માઇક્રોસ્ટ્રિપ કન્ફિગરેશન:}

\begin{verbatim}
  Signal Conductor
 {-{-}{-}{-}{-}+{-}{-}{-}{-}{-}{-}{-}{-}{-}}
    Dielectric
 {-{-}{-}{-}{-}{-}{-}{-}{-}{-}{-}{-}{-}{-}{-}}
   Ground Plane
\end{verbatim}

\textbf{તુલના કોષ્ટક:}

{\def\LTcaptype{none} % do not increment counter
\begin{longtable}[]{@{}lll@{}}
\toprule\noalign{}
પેરામીટર & સ્ટ્રિપલાઇન & માઇક્રોસ્ટ્રિપ \\
\midrule\noalign{}
\endhead
\bottomrule\noalign{}
\endlastfoot
\textbf{ગ્રાઉન્ડ પ્લેન્સ} & બે (સેન્ડવિચ) & એક (તળિયે) \\
\textbf{શીલ્ડિંગ} & સંપૂર્ણ & આંશિક \\
\textbf{ડિસ્પર્શન} & ઓછું & વધારે \\
\textbf{મેન્યુફેક્ચરિંગ} & જટિલ & સરળ \\
\textbf{કિંમત} & વધારે & ઓછી \\
\end{longtable}
}

\textbf{એપ્લિકેશન્સ:}

\begin{itemize}
\tightlist
\item
  \textbf{સ્ટ્રિપલાઇન}: હાઇ-પરફોર્મન્સ સિસ્ટમ્સ
\item
  \textbf{માઇક્રોસ્ટ્રિપ}: PCB સર્કિટ્સ, એન્ટીનાસ
\end{itemize}

\textbf{ડિઝાઇન સમીકરણો:}

\begin{itemize}
\tightlist
\item
  \textbf{લાક્ષણિક અવબાધ}: w/h રેશિયોનું ફંક્શન
\item
  \textbf{ઇફેક્ટિવ પર્મિટિવિટી}: εeff = (εr + 1)/2
\end{itemize}

\end{solutionbox}
\begin{mnemonicbox}
``સ્ટ્રિપલાઇન્સ આર સેન્ડવિચ્ડ, માઇક્રોસ્ટ્રિપ્સ આર માઉન્ટેડ''

\end{mnemonicbox}
\begin{center}\rule{0.5\linewidth}{0.5pt}\end{center}

\subsection*{પ્રશ્ન 4(ક) વૈકલ્પિક [7
માર્ક્સ]}\label{uxaaauxab0uxab6uxaa8-4uxa95-uxab5uxa95uxab2uxaaauxa95-7-uxaaeuxab0uxa95uxab8}

\textbf{પેરામેટ્રિક એમ્પ્લિફાયર માટે એમ્પ્લિફિકેશનના સિદ્ધાંતો અને પ્રક્રિયા સમજાવો.}

\begin{solutionbox}

\textbf{પેરામેટ્રિક એમ્પ્લિફાયર સિદ્ધાંત:}

\begin{center}
\textbf{Mermaid Diagram (Code)}
\begin{verbatim}
{Shaded}
{Highlighting}[]
graph LR
    A[સિગ્નલ fs] {-{-}{} B[નોનલિનિયર રિએક્ટન્સ]}
    C[પંપ fp] {-{-}{} B}
    B {-{-}{} D[આઇડલર fi]}
    B {-{-}{} E[એમ્પ્લિફાઇડ સિગ્નલ]}
    F[એનર્જી ફ્લો: પંપ  સિગ્નલ]
{Highlighting}
{Shaded}
\end{verbatim}
\end{center}

\textbf{આવૃત્તિ સંબંધો:}

{\def\LTcaptype{none} % do not increment counter
\begin{longtable}[]{@{}lll@{}}
\toprule\noalign{}
પેરામીટર & સંબંધ & સામાન્ય વેલ્યુઝ \\
\midrule\noalign{}
\endhead
\bottomrule\noalign{}
\endlastfoot
\textbf{પંપ ફ્રીક્વન્સી} & fp = fs + fi & 10 GHz \\
\textbf{સિગ્નલ ફ્રીક્વન્સી} & fs (ઇનપુટ) & 1 GHz \\
\textbf{આઇડલર ફ્રીક્વન્સી} & fi = fp - fs & 9 GHz \\
\end{longtable}
}

\textbf{એમ્પ્લિફિકેશન પ્રક્રિયા:}

\begin{enumerate}
\tightlist
\item
  \textbf{નોનલિનિયર એલિમેન્ટ}: વેરેક્ટર ડાયોડ ટાઇમ-વેરીંગ કેપેસિટન્સ પ્રદાન કરે છે
\item
  \textbf{પંપ પાવર}: હાઇ-ફ્રીક્વન્સી પંપ એનર્જી સપ્લાય કરે છે
\item
  \textbf{ફ્રીક્વન્સી મિક્સિંગ}: ત્રણ-આવૃત્તિ ઇન્ટરેક્શન
\item
  \textbf{એનર્જી ટ્રાન્સફર}: પંપ એનર્જી \rightarrow સિગ્નલ એનર્જી
\item
  \textbf{ઇમ્પિડન્સ મેચિંગ}: પાવર ટ્રાન્સફર ઓપ્ટિમાઇઝ કરો
\end{enumerate}

\textbf{સર્કિટ કન્ફિગરેશન:}

\begin{verbatim}
Signal {-{-}+{-}{-} Varactor {-}{-}+{-}{-} Amplified}
Input   |    Diode     |    Output
        |              |
       +++             +++
       |C|             |L| Idler
       | |             | | Circuit
       +++             +++
        |              |
      Pump {-{-}{-}{-}{-}{-}{-}{-}{-}{-}{-}{-}+}
      Input
\end{verbatim}

\textbf{મુખ્ય ફાયદાઓ:}

\begin{itemize}
\tightlist
\item
  \textbf{લો નોઇઝ ફિગર}: ક્વાન્ટમ લિમિટની નજીક
\item
  \textbf{હાઇ ગેઇન}: 10-20 dB સામાન્ય
\item
  \textbf{વાઇડ બેન્ડવિડ્થ}: પંપ સર્કિટ દ્વારા મર્યાદિત
\end{itemize}

\textbf{એપ્લિકેશન્સ:}

\begin{itemize}
\tightlist
\item
  \textbf{સેટેલાઇટ રિસીવર્સ}
\item
  \textbf{રેડિયો એસ્ટ્રોનોમી}
\item
  \textbf{લો-નોઇઝ એમ્પ્લિફાયર્સ}
\end{itemize}

\textbf{ડિઝાઇન વિચારણાઓ:}

\begin{itemize}
\tightlist
\item
  \textbf{પંપ પાવર}: નોનલિનિયર ઓપરેશન માટે પૂરતું
\item
  \textbf{ઇમ્પિડન્સ મેચિંગ}: ત્રણેય આવૃત્તિઓ
\item
  \textbf{સ્થિરતા}: ઓસિલેશન અટકાવો
\end{itemize}

\end{solutionbox}
\begin{mnemonicbox}
``પેરામેટ્રિક એમ્પ્લિફાયર્સ પંપ પાવર ઇન્ટુ સિગ્નલ પરફેક્ટલી''

\end{mnemonicbox}
\begin{center}\rule{0.5\linewidth}{0.5pt}\end{center}

\subsection*{પ્રશ્ન 5(અ) [3
માર્ક્સ]}\label{uxaaauxab0uxab6uxaa8-5uxa85-3-uxaaeuxab0uxa95uxab8}

\textbf{RADAR અને SONAR ની સરખામણી કરો.}

\begin{solutionbox}

\textbf{RADAR vs SONAR તુલના:}

{\def\LTcaptype{none} % do not increment counter
\begin{longtable}[]{@{}lll@{}}
\toprule\noalign{}
પેરામીટર & RADAR & SONAR \\
\midrule\noalign{}
\endhead
\bottomrule\noalign{}
\endlastfoot
\textbf{તરંગ પ્રકાર} & ઇલેક્ટ્રોમેગ્નેટિક & અકૌસ્ટિક \\
\textbf{માધ્યમ} & હવા/વેક્યુમ & પાણી \\
\textbf{આવૃત્તિ} & 300 MHz - 30 GHz & 1 kHz - 1 MHz \\
\textbf{ઝડપ} & 3\times10^{8} m/s & 1500 m/s (પાણી) \\
\textbf{રેન્જ} & 1000 km સુધી & 100 km સુધી \\
\textbf{એપ્લિકેશન્સ} & એરક્રાફ્ટ, હવામાન & સબમરીન, માછીમારી \\
\end{longtable}
}

\textbf{સામાન્ય સિદ્ધાંતો:}

\begin{itemize}
\tightlist
\item
  \textbf{ઇકો રેન્જિંગ}: ટાઇમ-ઓફ-ફ્લાઇટ માપો
\item
  \textbf{ડોપ્લર ઇફેક્ટ}: ગતિશીલ લક્ષ્યો શોધો
\item
  \textbf{બીમ ફોર્મિંગ}: દિશાત્મક ટ્રાન્સમિશન
\end{itemize}

\textbf{મુખ્ય તફાવતો:}

\begin{itemize}
\tightlist
\item
  \textbf{પ્રોપેગેશન}: EM તરંગો vs ધ્વનિ તરંગો
\item
  \textbf{એટેન્યુએશન}: વિવિધ લોસ મિકેનિઝમ
\item
  \textbf{રિઝોલ્યુશન}: આવૃત્તિ આધારિત
\end{itemize}

\end{solutionbox}
\begin{mnemonicbox}
``RADAR સીઝ રેડિયો વેવ્સ, SONAR હિયર્સ સાઉન્ડ વેવ્સ''

\end{mnemonicbox}
\begin{center}\rule{0.5\linewidth}{0.5pt}\end{center}

\subsection*{પ્રશ્ન 5(બ) [4
માર્ક્સ]}\label{uxaaauxab0uxab6uxaa8-5uxaac-4-uxaaeuxab0uxa95uxab8}

\textbf{RADAR પ્રદર્શન પદ્ધતિનું નામ લખો અને કોઈપણ એકને સમજાવો.}

\begin{solutionbox}

\textbf{RADAR પ્રદર્શન પદ્ધતિઓ:}

{\def\LTcaptype{none} % do not increment counter
\begin{longtable}[]{@{}lll@{}}
\toprule\noalign{}
ડિસ્પ્લે પ્રકાર & વર્ણન & એપ્લિકેશન \\
\midrule\noalign{}
\endhead
\bottomrule\noalign{}
\endlastfoot
\textbf{A-Scope} & રેન્જ vs એમ્પ્લિટ્યુડ & ટાર્ગેટ ડિટેક્શન \\
\textbf{B-Scope} & રેન્જ vs અઝીમુથ & 2D પોઝિશન \\
\textbf{C-Scope} & અઝીમુથ vs એલિવેશન & 3D ટ્રેકિંગ \\
\textbf{PPI} & પ્લેન પોઝિશન ઇન્ડિકેટર & એર ટ્રાફિક કંટ્રોલ \\
\textbf{RHI} & રેન્જ હાઇટ ઇન્ડિકેટર & વેધર રડાર \\
\end{longtable}
}

\textbf{PPI ડિસ્પ્લે સમજૂતી:}

\begin{center}
\textbf{Mermaid Diagram (Code)}
\begin{verbatim}
{Shaded}
{Highlighting}[]
graph TD
    A[સેન્ટર {- રડાર પોઝિશન] {-}{-}{} B[સ્વીપ લાઇન {-} એન્ટીના દિશા]}
    B {-{-}{} C[ટાર્ગેટ બ્લિપ્સ {-} રેન્જ \& બેરિંગ]}
    D[સર્ક્યુલર પેટર્ન] {-{-}{} E[360^ કવરેજ]}
{Highlighting}
{Shaded}
\end{verbatim}
\end{center}

\textbf{PPI લક્ષણો:}

\begin{itemize}
\tightlist
\item
  \textbf{પોલર કોઓર્ડિનેટ}: રેન્જ અને બેરિંગ
\item
  \textbf{રોટેટિંગ સ્વીપ}: એન્ટીના રોટેશનને અનુસરે છે
\item
  \textbf{પર્સિસ્ટન્સ}: ટાર્ગેટ્સ દૃશ્યમાન રહે છે
\item
  \textbf{સ્કેલ સિલેક્શન}: એડજસ્ટેબલ રેન્જ
\end{itemize}

\textbf{ડિસ્પ્લે પ્રક્રિયા:}

\begin{enumerate}
\tightlist
\item
  \textbf{સ્વીપ જનરેશન}: એન્ટીના સાથે સિંક્રોનાઇઝ
\item
  \textbf{ટાર્ગેટ પ્લોટિંગ}: અંતર અને દિશા
\item
  \textbf{ઇન્ટેન્સિટી મોડ્યુલેશન}: ટાર્ગેટ સ્ટ્રેન્થ
\item
  \textbf{મેપ ઓવરલે}: ભૌગોલિક સંદર્ભ
\end{enumerate}

\end{solutionbox}
\begin{mnemonicbox}
``PPI પ્રોવાઇડ્સ પરફેક્ટ પોઝિશન ઇન્ફોર્મેશન''

\end{mnemonicbox}
\begin{center}\rule{0.5\linewidth}{0.5pt}\end{center}

\subsection*{પ્રશ્ન 5(ક) [7
માર્ક્સ]}\label{uxaaauxab0uxab6uxaa8-5uxa95-7-uxaaeuxab0uxa95uxab8}

\textbf{બ્લોક ડાયાગ્રામ સાથે મૂળભૂત પલ્સ રડાર સિસ્ટમ સમજાવો.}

\begin{solutionbox}

\textbf{પલ્સ રડાર બ્લોક ડાયાગ્રામ:}

\begin{center}
\textbf{Mermaid Diagram (Code)}
\begin{verbatim}
{Shaded}
{Highlighting}[]
graph LR
    A[માસ્ટર ઓસિલેટર] {-{-}{} B[મોડ્યુલેટર]}
    B {-{-}{} C[પાવર એમ્પ્લિફાયર]}
    C {-{-}{} D[ડુપ્લેક્સર]}
    D {-{-}{} E[એન્ટીના]}
    E {-{-}{} F[ટાર્ગેટ]}
    F {-{-}{} E}
    E {-{-}{} D}
    D {-{-}{} G[રિસીવર]}
    G {-{-}{} H[સિગ્નલ પ્રોસેસર]}
    H {-{-}{} I[ડિસ્પ્લે]}
    J[ટાઇમર] {-{-}{} A}
    J {-{-}{} I}
{Highlighting}
{Shaded}
\end{verbatim}
\end{center}

\textbf{સિસ્ટમ કોમ્પોનન્ટ્સ:}

{\def\LTcaptype{none} % do not increment counter
\begin{longtable}[]{@{}lll@{}}
\toprule\noalign{}
કોમ્પોનન્ટ & કાર્ય & મુખ્ય પેરામીટર્સ \\
\midrule\noalign{}
\endhead
\bottomrule\noalign{}
\endlastfoot
\textbf{માસ્ટર ઓસિલેટર} & RF સિગ્નલ જનરેટ કરે છે & ફ્રીક્વન્સી સ્થિરતા \\
\textbf{મોડ્યુલેટર} & પલ્સ ટ્રેઇન બનાવે છે & પલ્સ વિડ્થ, PRF \\
\textbf{પાવર એમ્પ્લિફાયર} & ટ્રાન્સમિટ પાવર બૂસ્ટ કરે છે & પીક પાવર,
કાર્યક્ષમતા \\
\textbf{ડુપ્લેક્સર} & Tx/Rx સ્વિચ કરે છે & આઇસોલેશન, સ્વિચિંગ ટાઇમ \\
\textbf{એન્ટીના} & રેડિયેટ/રિસીવ કરે છે & ગેઇન, બીમવિડ્થ \\
\textbf{રિસીવર} & ઇકો સિગ્નલ્સ એમ્પ્લિફાય કરે છે & સેન્સિટિવિટી, બેન્ડવિડ્થ \\
\end{longtable}
}

\textbf{ઓપરેટિંગ સીક્વન્સ:}

\begin{enumerate}
\tightlist
\item
  \textbf{ટ્રાન્સમિશન ફેઝ}:

  \begin{itemize}
  \tightlist
  \item
    માસ્ટર ઓસિલેટર RF જનરેટ કરે છે
  \item
    મોડ્યુલેટર પલ્સ બનાવે છે
  \item
    પાવર એમ્પ્લિફાયર સિગ્નલ બૂસ્ટ કરે છે
  \item
    ડુપ્લેક્સર એન્ટીના તરફ રૂટ કરે છે
  \end{itemize}
\item
  \textbf{રિસેપ્શન ફેઝ}:

  \begin{itemize}
  \tightlist
  \item
    એન્ટીના ઇકો રિસીવ કરે છે
  \item
    ડુપ્લેક્સર રિસીવર તરફ રૂટ કરે છે
  \item
    સિગ્નલ પ્રોસેસિંગ માહિતી એક્સટ્રેક્ટ કરે છે
  \item
    ડિસ્પ્લે ટાર્ગેટ ડેટા બતાવે છે
  \end{itemize}
\end{enumerate}

\textbf{મુખ્ય સમીકરણો:}

\begin{itemize}
\tightlist
\item
  \textbf{રેન્જ}: R = ct/2 (જ્યાં t = રાઉન્ડ-ટ્રિપ ટાઇમ)
\item
  \textbf{મહત્તમ રેન્જ}: Rmax = cPRT/2
\item
  \textbf{રેન્જ રિઝોલ્યુશન}: ΔR = cτ/2
\end{itemize}

\textbf{પરફોર્મન્સ પેરામીટર્સ:}

\begin{itemize}
\tightlist
\item
  \textbf{PRF}: પલ્સ રિપેટિશન ફ્રીક્વન્સી
\item
  \textbf{ડ્યુટી સાયકલ}: τ \times PRF
\item
  \textbf{એવરેજ પાવર}: પીક પાવર \times ડ્યુટી સાયકલ
\end{itemize}

\end{solutionbox}
\begin{mnemonicbox}
``પલ્સ રડાર પ્રોપર્લી પ્રોસેસ રિફ્લેક્ટેડ સિગ્નલ્સ''

\end{mnemonicbox}
\begin{center}\rule{0.5\linewidth}{0.5pt}\end{center}

\subsection*{પ્રશ્ન 5(અ) વૈકલ્પિક [3
માર્ક્સ]}\label{uxaaauxab0uxab6uxaa8-5uxa85-uxab5uxa95uxab2uxaaauxa95-3-uxaaeuxab0uxa95uxab8}

\textbf{માઇક્રોવેવ આવૃત્તિની એપ્લિકેશનની સૂચિ બનાવો.}

\begin{solutionbox}

\textbf{માઇક્રોવેવ એપ્લિકેશન્સ:}

{\def\LTcaptype{none} % do not increment counter
\begin{longtable}[]{@{}lll@{}}
\toprule\noalign{}
એપ્લિકેશન કેટેગરી & વિશિષ્ટ ઉપયોગો & આવૃત્તિ બેન્ડ \\
\midrule\noalign{}
\endhead
\bottomrule\noalign{}
\endlastfoot
\textbf{કમ્યુનિકેશન} & સેટેલાઇટ, સેલ્યુલર, WiFi & 1-40 GHz \\
\textbf{રડાર સિસ્ટમ્સ} & હવામાન, એર ટ્રાફિક, મિલિટરી & 1-35 GHz \\
\textbf{ઇન્ડસ્ટ્રિયલ} & હીટિંગ, ડ્રાયિંગ, મેડિકલ & 0.9-5.8 GHz \\
\textbf{નેવિગેશન} & GPS, એરક્રાફ્ટ લેન્ડિંગ & 1-15 GHz \\
\textbf{સાયન્ટિફિક} & રેડિયો એસ્ટ્રોનોમી, રિસર્ચ & 1-300 GHz \\
\textbf{મેડિકલ} & ડાયાથર્મી, કેન્સર ટ્રીટમેન્ટ & 0.9-2.45 GHz \\
\textbf{ઘરેલું} & માઇક્રોવેવ ઓવન્સ & 2.45 GHz \\
\end{longtable}
}

\textbf{મુખ્ય મુદ્દાઓ:}

\begin{itemize}
\tightlist
\item
  \textbf{ISM બેન્ડ્સ} (ઇન્ડસ્ટ્રિયલ, સાયન્ટિફિક, મેડિકલ): લાઇસન્સ-ફ્રી
\item
  \textbf{પેનેટ્રેશન ક્ષમતા}: આવૃત્તિ અને મટીરિયલ પર આધાર રાખે છે
\item
  \textbf{એટમોસ્ફેરિક એબસોર્પ્શન}: આવૃત્તિ સાથે વધે છે
\end{itemize}

\end{solutionbox}
\begin{mnemonicbox}
``માઇક્રોવેવ્સ સર્વ મેની એપ્લિકેશન્સ પરફેક્ટલી''

\end{mnemonicbox}
\begin{center}\rule{0.5\linewidth}{0.5pt}\end{center}

\subsection*{પ્રશ્ન 5(બ) વૈકલ્પિક [4
માર્ક્સ]}\label{uxaaauxab0uxab6uxaa8-5uxaac-uxab5uxa95uxab2uxaaauxa95-4-uxaaeuxab0uxa95uxab8}

\textbf{PULSED RADAR અને CW RADAR ની સરખામણી કરો.}

\begin{solutionbox}

\textbf{PULSED vs CW RADAR તુલના:}

{\def\LTcaptype{none} % do not increment counter
\begin{longtable}[]{@{}lll@{}}
\toprule\noalign{}
પેરામીટર & પલ્સ્ડ RADAR & CW RADAR \\
\midrule\noalign{}
\endhead
\bottomrule\noalign{}
\endlastfoot
\textbf{ટ્રાન્સમિશન} & પલ્સ ટ્રેઇન & કન્ટિન્યુઅસ વેવ \\
\textbf{રેન્જ માપ} & ટાઇમ-ઓફ-ફ્લાઇટ & ફ્રીક્વન્સી શિફ્ટ \\
\textbf{વેલોસિટી માપ} & પલ્સમાં ડોપ્લર & ડાયરેક્ટ ડોપ્લર \\
\textbf{એન્ટીના} & સિંગલ (ડુપ્લેક્સર) & અલગ Tx/Rx \\
\textbf{પાવર} & હાઇ પીક, લો એવરેજ & લો કન્ટિન્યુઅસ \\
\textbf{રેન્જ રિઝોલ્યુશન} & પલ્સ વિડ્થ લિમિટેડ & નબળું \\
\textbf{વેલોસિટી રિઝોલ્યુશન} & લિમિટેડ & ઉત્કૃષ્ટ \\
\textbf{જટિલતા} & હાઇ & લો \\
\textbf{કિંમત} & વધારે & ઓછી \\
\end{longtable}
}

\textbf{ઓપરેશનલ તફાવતો:}

\textbf{પલ્સ્ડ RADAR:}

\begin{itemize}
\tightlist
\item
  \textbf{રેન્જ સમીકરણ}: R = ct/2
\item
  \textbf{મહત્તમ રેન્જ}: PRF દ્વારા મર્યાદિત
\item
  \textbf{બ્લાઇન્ડ રેન્જ}: cPRT/2 ના મલ્ટિપલ
\item
  \textbf{એપ્લિકેશન્સ}: લોંગ-રેન્જ ડિટેક્શન
\end{itemize}

\textbf{CW RADAR:}

\begin{itemize}
\tightlist
\item
  \textbf{ડોપ્લર સમીકરણ}: fd = 2vr/λ
\item
  \textbf{રેન્જ માપ}: FM મોડ્યુલેશન જરૂરી
\item
  \textbf{કોઈ બ્લાઇન્ડ રેન્જ નથી}: કન્ટિન્યુઅસ ઓપરેશન
\item
  \textbf{એપ્લિકેશન્સ}: સ્પીડ માપ, પ્રોક્સિમિટી
\end{itemize}

\textbf{મુખ્ય ફાયદાઓ:}

\begin{itemize}
\tightlist
\item
  \textbf{પલ્સ્ડ}: બહેતર રેન્જ ક્ષમતા, ટાર્ગેટ સેપરેશન
\item
  \textbf{CW}: બહેતર વેલોસિટી એક્યુરસી, સરળ ડિઝાઇન
\end{itemize}

\end{solutionbox}
\begin{mnemonicbox}
``પલ્સ્ડ મેઝર્સ રેન્જ, CW મેઝર્સ વેલોસિટી''

\end{mnemonicbox}
\begin{center}\rule{0.5\linewidth}{0.5pt}\end{center}

\subsection*{પ્રશ્ન 5(ક) વૈકલ્પિક [7
માર્ક્સ]}\label{uxaaauxab0uxab6uxaa8-5uxa95-uxab5uxa95uxab2uxaaauxa95-7-uxaaeuxab0uxa95uxab8}

\textbf{બ્લોક ડાયાગ્રામ સાથે MTI રડાર સમજાવો.}

\begin{solutionbox}

\textbf{MTI RADAR બ્લોક ડાયાગ્રામ:}

\begin{center}
\textbf{Mermaid Diagram (Code)}
\begin{verbatim}
{Shaded}
{Highlighting}[]
graph LR
    A[ટ્રાન્સમિટર] {-{-}{} B[ડુપ્લેક્સર]}
    B {-{-}{} C[એન્ટીના]}
    C {-{-}{} D[ટાર્ગેટ]}
    D {-{-}{} C}
    C {-{-}{} B}
    B {-{-}{} E[રિસીવર]}
    E {-{-}{} F[ફેઝ ડિટેક્ટર]}
    G[STALO] {-{-}{} H[મિક્સર]}
    H {-{-}{} F}
    I[COHO] {-{-}{} F}
    F {-{-}{} J[MTI ફિલ્ટર]}
    J {-{-}{} K[ડિસ્પ્લે]}
    G {-{-}{} L[ફ્રીક્વન્સી મલ્ટિપ્લાયર]}
    L {-{-}{} A}
{Highlighting}
{Shaded}
\end{verbatim}
\end{center}

\textbf{MTI સિસ્ટમ કોમ્પોનન્ટ્સ:}

{\def\LTcaptype{none} % do not increment counter
\begin{longtable}[]{@{}lll@{}}
\toprule\noalign{}
કોમ્પોનન્ટ & સંપૂર્ણ નામ & કાર્ય \\
\midrule\noalign{}
\endhead
\bottomrule\noalign{}
\endlastfoot
\textbf{STALO} & સ્ટેબલ લોકલ ઓસિલેટર & રેફરન્સ આવૃત્તિ \\
\textbf{COHO} & કોહેરન્ટ ઓસિલેટર & ફેઝ રેફરન્સ \\
\textbf{MTI ફિલ્ટર} & મૂવિંગ ટાર્ગેટ ઇન્ડિકેટર & ક્લટર સપ્રેશન \\
\textbf{ફેઝ ડિટેક્ટર} & - & સિગ્નલ ફેઝની તુલના \\
\end{longtable}
}

\textbf{MTI ઓપરેટિંગ સિદ્ધાંત:}

\textbf{પલ્સ-ટુ-પલ્સ તુલના:}

\begin{verbatim}
Signal Amplitude
       \^{}
       |    Fixed Target (Clutter)
       |    \_\_\_\_\_\_\_\_\_\_\_\_\_\_\_\_
       |   |                |
       |   |                |
       |   |                |
   \_\_\_\_+\_\_\_|\_\_\_\_\_\_\_\_\_\_\_\_\_\_\_\_|\_\_\_\_
       |                        
       |    Moving Target
       |     /{      /}
       |    /  {    /  }
       |   /    {  /    }
   \_\_\_\_+\_\_/\_\_\_\_\_\_{/\_\_\_\_\_\_\_\_\_\_\_\_}
       |
       +{-{-}{-}{-}{-}{-}{-}{-}{-}{-}{-}{-}{-}{-}{-}{-}{-}{-}{-}{-}{-}{-}{-}{-}{-} Time}
           Pulse 1    Pulse 2
\end{verbatim}

\textbf{MTI પ્રક્રિયા:}

\begin{enumerate}
\tightlist
\item
  \textbf{કોહેરન્ટ ટ્રાન્સમિશન}: ફેઝ સંબંધો જાળવો
\item
  \textbf{ઇકો રિસેપ્શન}: ફેઝ માહિતી સાચવો
\item
  \textbf{ફેઝ તુલના}: ક્રમિક પલ્સની તુલના કરો
\item
  \textbf{ક્લટર કેન્સલેશન}: સ્થિર રિટર્ન ઘટાડો
\item
  \textbf{મૂવિંગ ટાર્ગેટ ડિટેક્શન}: ગતિશીલ ટાર્ગેટ વધારો
\end{enumerate}

\textbf{મુખ્ય સમીકરણો:}

\begin{itemize}
\tightlist
\item
  \textbf{ડોપ્લર આવૃત્તિ}: fd = 2vr cos(θ)/λ
\item
  \textbf{ફેઝ ચેંજ}: Δφ = 4πvr/λ \times PRT
\item
  \textbf{બ્લાઇન્ડ સ્પીડ્સ}: vb = nλ/(2PRT)
\end{itemize}

\textbf{MTI સુધારણા પરિબળ:}

\begin{itemize}
\tightlist
\item
  \textbf{વ્યાખ્યા}: MTI પહેલા/પછી ક્લટર પાવરનો ગુણોત્તર
\item
  \textbf{સામાન્ય મૂલ્યો}: 20-40 dB
\item
  \textbf{અસર કરતા પરિબળો}: સિસ્ટમ સ્થિરતા, ક્લટર લક્ષણો
\end{itemize}

\textbf{મર્યાદાઓ:}

\begin{itemize}
\tightlist
\item
  \textbf{બ્લાઇન્ડ સ્પીડ્સ}: ચોક્કસ વેગ પર ટાર્ગેટ્સ અદૃશ્ય
\item
  \textbf{સ્પર્શક ટાર્ગેટ્સ}: રેડિયલ વેલોસિટી કોમ્પોનન્ટ જરૂરી
\item
  \textbf{હવામાન અસરો}: વાતાવરણીય વધઘટ
\end{itemize}

\textbf{એપ્લિકેશન્સ:}

\begin{itemize}
\tightlist
\item
  \textbf{એર ટ્રાફિક કંટ્રોલ}: ગ્રાઉન્ડ ક્લટરથી એરક્રાફ્ટ અલગ કરો
\item
  \textbf{વેધર રડાર}: ભૂપ્રદેશથી વરસાદ અલગ કરો
\item
  \textbf{મિલિટરી રડાર}: ગતિશીલ વાહનો/એરક્રાફ્ટ શોધો
\end{itemize}

\end{solutionbox}
\begin{mnemonicbox}
``MTI મેક્સ ટાર્ગેટ્સ આઇડેન્ટિફાયેબલ બાય મૂવમેન્ટ''

\end{mnemonicbox}

\end{document}
