\documentclass[10pt,a4paper]{article}

% content/resources/templates/preamble.tex
\usepackage[margin=0.6in]{geometry}
\author{Milav Dabgar}
\usepackage{amsmath,amssymb,amsthm}
\usepackage{booktabs}
\usepackage{multirow}
\usepackage{xcolor}
\usepackage{tcolorbox}
\tcbuselibrary{breakable,skins}
\usepackage[colorlinks=true,linkcolor=blue]{hyperref}
\usepackage{titlesec}
\usepackage{enumitem}
\usepackage{tikz}
\usepackage{pgfplots}
\usepackage{circuitikz}
\usepackage[version=4]{mhchem}
\usepackage{longtable}
\usepackage{array}
\usepackage{float}
\usepackage{caption}
\usepackage{listings}

\lstset{
  basicstyle=\small\ttfamily,
  breaklines=true,
  breakatwhitespace=false,
  postbreak=\mbox{\textcolor{red}{$\hookrightarrow$}\space},
  float=false,
  numbers=left,
  numberstyle=\tiny\color{gray},
  numbersep=10pt,
  xleftmargin=2em,
  keywordstyle=\color{blue},
  commentstyle=\color{green!60!black},
  stringstyle=\color{purple},
  backgroundcolor=\color{gray!5},
  showstringspaces=false,
  tabsize=2,
  captionpos=b,
  keepspaces=true,
  columns=flexible
}

\pgfplotsset{compat=1.18}
\usetikzlibrary{shapes,arrows,positioning,calc,patterns,decorations.pathmorphing,decorations.markings,arrows.meta}

% Color scheme
\definecolor{headcolor}{RGB}{0,102,204}
\definecolor{keycolor}{RGB}{220,20,60}
\definecolor{solutioncolor}{RGB}{34,139,34}
\definecolor{mnemoniccolor}{RGB}{148,0,211}
\definecolor{codecolor}{RGB}{0,0,100}

% Spacing
\setlength{\parskip}{3pt}
\setlist[itemize]{nosep}
\setlist[enumerate]{nosep}

% Title formatting
\titleformat{\section}{\Large\bfseries\color{headcolor}}{\thesection}{1em}{}
\titleformat{\subsection}{\large\bfseries\color{headcolor}}{\thesubsection}{1em}{}

% Pandoc tightlist compatibility
\providecommand{\tightlist}{%
  \setlength{\itemsep}{0pt}\setlength{\parskip}{0pt}}

% Pandoc longtable compatibility
\newcounter{none}
\def\thenone{}


% content/resources/templates/english-boxes.tex
% This file is currently empty - it exists to maintain consistency with the import structure.
% Add custom environments here if needed in the future.


\begin{document}

\begin{center}
{\Huge\bfseries\color{headcolor} Subject Name Solutions}\\[5pt]
{\LARGE 4351103 -- Winter 2023}\\[3pt]
{\large Semester 1 Study Material}\\[3pt]
{\normalsize\textit{Detailed Solutions and Explanations}}
\end{center}

\vspace{10pt}

\subsection*{Question 1(a) [3 marks]}\label{q1a}

\textbf{Sketch the standing wave pattern for voltage and current along
the transmission line when it is terminated with (i) Short Circuit, (ii)
Open circuit, and (iii) Matched Load.}

\begin{solutionbox}

\textbf{Diagram:}

\begin{verbatim}
Short Circuit (Z\_L = 0):
     V      
     |      
     |      
     |\_\_\_\_\_\_|\_\_\_\_\_\_|\_\_\_\_\_\_|\_\_\_\_\_\_ 0V
       λ/4   λ/2   3λ/4    λ
     
     I      
   I\_max {-{-}{-}|{-}{-}{-}|{-}{-}{-}|{-}{-}{-}}
     |                  
     |      
     0 \_\_\_\_\_\_\_\_\_\_\_\_\_\_\_\_\_\_ 0A

Open Circuit (Z\_L = ):
   V\_max {-{-}{-}|{-}{-}{-}|{-}{-}{-}|{-}{-}{-}}
     |                  
     V      
     0 \_\_\_\_\_\_\_\_\_\_\_\_\_\_\_\_\_\_ 0V
       λ/4   λ/2   3λ/4    λ
     
     I      
     |      
     |      
     |\_\_\_\_\_\_|\_\_\_\_\_\_|\_\_\_\_\_\_|\_\_\_\_\_\_ 0A
       λ/4   λ/2   3λ/4    λ

Matched Load (Z\_L = Z\_0):
     V      
   {-{-}{-}|{-}{-}{-}|{-}{-}{-}|{-}{-}{-}|{-}{-}{-} Constant}
     
     I      
   {-{-}{-}|{-}{-}{-}|{-}{-}{-}|{-}{-}{-}|{-}{-}{-} Constant}
\end{verbatim}

\begin{itemize}
\tightlist
\item
  \textbf{Short Circuit}: Voltage minimum at load, current maximum at
  load
\item
  \textbf{Open Circuit}: Voltage maximum at load, current minimum at
  load\\
\item
  \textbf{Matched Load}: Constant voltage and current, no reflections
\end{itemize}

\end{solutionbox}
\begin{mnemonicbox}
``SOC - Short Opens Current, Open Shorts Current''

\end{mnemonicbox}
\subsection*{Question 1(b) [4 marks]}\label{q1b}

\textbf{Draw and Explain equivalent circuit of two parallel wire
transmission line at microwave frequency.}

\begin{solutionbox}

\textbf{Diagram:}

\begin{verbatim}
       R      L      R      L
    {-{-}{-}|\^{}\^{}\^{}|{-}{-}|||||{-}{-}{-}|\^{}\^{}\^{}|{-}{-}|||||{-}{-}{-}}
    |      |       |      |       |
    |      G       C      G       C
    |     |||     {-{-}{-}    |||     {-}{-}{-}}
    |     |||     {-{-}{-}    |||     {-}{-}{-}}
    {-{-}{-}|\^{}\^{}\^{}|{-}{-}|||||{-}{-}{-}|\^{}\^{}\^{}|{-}{-}|||||{-}{-}{-}}
       R      L      R      L
       
       {-{-} Δz {-}{-}}
\end{verbatim}

\begin{itemize}
\tightlist
\item
  \textbf{R}: Series resistance per unit length (conductor losses)
\item
  \textbf{L}: Series inductance per unit length (magnetic field storage)
\item
  \textbf{G}: Shunt conductance per unit length (dielectric losses)
\item
  \textbf{C}: Shunt capacitance per unit length (electric field storage)
\end{itemize}

\textbf{Primary Constants Table:}

{\def\LTcaptype{none} % do not increment counter
\begin{longtable}[]{@{}llll@{}}
\toprule\noalign{}
Parameter & Symbol & Unit & Effect \\
\midrule\noalign{}
\endhead
\bottomrule\noalign{}
\endlastfoot
Resistance & R & Ω/m & Power loss \\
Inductance & L & H/m & Magnetic energy \\
Conductance & G & S/m & Leakage current \\
Capacitance & C & F/m & Electric energy \\
\end{longtable}
}

\end{solutionbox}
\begin{mnemonicbox}
``RLGC - Really Largeガイド Cables''

\end{mnemonicbox}
\subsection*{Question 1(c) [7 marks]}\label{q1c}

\textbf{Explain Principle, construction and working of Isolator with
necessary sketch.}

\begin{solutionbox}

\textbf{Principle}: Isolator allows microwave signal to pass in forward
direction only using \textbf{ferrite material} and \textbf{Faraday
rotation effect}.

\textbf{Construction Diagram:}

\begin{center}
\textbf{Mermaid Diagram (Code)}
\begin{verbatim}
{Shaded}
{Highlighting}[]
graph LR
    A[Input Port] {-{-}{} B[Ferrite Rod]}
    B {-{-}{} C[Permanent Magnet]}
    C {-{-}{} D[Output Port]}
    E[Resistive Load] {-{-}{} B}
    F[Waveguide] {-{-}{} B}
{Highlighting}
{Shaded}
\end{verbatim}
\end{center}

\textbf{Working}:

\begin{itemize}
\tightlist
\item
  \textbf{Forward direction}: Signal passes through ferrite with minimal
  loss
\item
  \textbf{Reverse direction}: Signal is rotated 45^\circ and absorbed by
  resistive load
\item
  \textbf{Magnetic field} biases ferrite material
\item
  \textbf{Isolation}: 20-30 dB typical
\end{itemize}

\textbf{Applications}:

\begin{itemize}
\tightlist
\item
  \textbf{Protects} transmitter from reflected power
\item
  \textbf{Prevents} oscillations in amplifier circuits
\item
  \textbf{Maintains} source impedance matching
\end{itemize}

\textbf{Specifications Table:}

{\def\LTcaptype{none} % do not increment counter
\begin{longtable}[]{@{}lll@{}}
\toprule\noalign{}
Parameter & Value & Unit \\
\midrule\noalign{}
\endhead
\bottomrule\noalign{}
\endlastfoot
Isolation & 20-30 & dB \\
Insertion Loss & 0.5-1 & dB \\
VSWR & \textless1.5 & - \\
\end{longtable}
}

\end{solutionbox}
\begin{mnemonicbox}
``Isolate Forward, Absorb Reverse''

\end{mnemonicbox}
\subsection*{Question 1(c OR) [7
marks]}\label{question-1c-or-7-marks}

\textbf{Compare Transmission Line and Waveguide.}

\begin{solutionbox}

\textbf{Comparison Table:}

{\def\LTcaptype{none} % do not increment counter
\begin{longtable}[]{@{}lll@{}}
\toprule\noalign{}
Parameter & Transmission Line & Waveguide \\
\midrule\noalign{}
\endhead
\bottomrule\noalign{}
\endlastfoot
\textbf{Frequency Range} & DC to microwave & Above cutoff frequency \\
\textbf{Power Handling} & Limited & High power capability \\
\textbf{Losses} & Higher (I^{2}R losses) & Lower (no center conductor) \\
\textbf{Size} & Compact & Bulky at low frequencies \\
\textbf{Modes} & TEM mode & TE and TM modes \\
\textbf{Installation} & Easy & Complex mounting \\
\textbf{Cost} & Lower & Higher \\
\textbf{Bandwidth} & Wide & Limited by modes \\
\end{longtable}
}

\textbf{Key Differences:}

\begin{itemize}
\tightlist
\item
  \textbf{Transmission line}: Uses two conductors, supports TEM mode
\item
  \textbf{Waveguide}: Single hollow conductor, supports TE/TM modes
\item
  \textbf{Cutoff frequency}: Waveguide has minimum operating frequency
\item
  \textbf{Field pattern}: Different electromagnetic field distributions
\end{itemize}

\textbf{Applications:}

\begin{itemize}
\tightlist
\item
  \textbf{Transmission lines}: Low power, broadband applications
\item
  \textbf{Waveguides}: High power radar, satellite communication
\end{itemize}

\end{solutionbox}
\begin{mnemonicbox}
``Transmission Travels Two-wire, Waveguide Walks
Wide''

\end{mnemonicbox}
\subsection*{Question 2(a) [3 marks]}\label{q2a}

\textbf{Define: (i) VSWR, (ii) Reflection Coefficient, and (iii) Skin
effect}

\begin{solutionbox}

\textbf{Definitions:}

\begin{itemize}
\tightlist
\item
  \textbf{VSWR (Voltage Standing Wave Ratio)}: Ratio of maximum to
  minimum voltage amplitudes on transmission line

  \begin{itemize}
  \tightlist
  \item
    Formula: VSWR = V\_max/V\_min =
    (1+\textbar Γ\textbar)/(1-\textbar Γ\textbar)
  \end{itemize}
\item
  \textbf{Reflection Coefficient (Γ)}: Ratio of reflected to incident
  voltage amplitude

  \begin{itemize}
  \tightlist
  \item
    Formula: Γ = (Z\_L - Z\_0)/(Z\_L + Z\_0)
  \end{itemize}
\item
  \textbf{Skin Effect}: Current flows mainly on conductor surface at
  high frequencies

  \begin{itemize}
  \tightlist
  \item
    Skin depth: δ = \sqrt(2/ωμσ)
  \end{itemize}
\end{itemize}

\textbf{Parameters Table:}

{\def\LTcaptype{none} % do not increment counter
\begin{longtable}[]{@{}lll@{}}
\toprule\noalign{}
Parameter & Range & Ideal Value \\
\midrule\noalign{}
\endhead
\bottomrule\noalign{}
\endlastfoot
VSWR & 1 to \infty & 1 (matched) \\
& Γ & \\
Skin Depth & μm to mm & Frequency dependent \\
\end{longtable}
}

\end{solutionbox}
\begin{mnemonicbox}
``VSWR Varies, Gamma Guides, Skin Shrinks''

\end{mnemonicbox}
\subsection*{Question 2(b) [4 marks]}\label{q2b}

\textbf{Explain working of Two-hole Directional Coupler with Proper
sketch.}

\begin{solutionbox}

\textbf{Construction Diagram:}

\begin{verbatim}
Main Waveguide:
|===============================|
|      P1        P2           |
|===============================|
         ○    ○   Two holes
|===============================|
|      P4        P3           |
|===============================|
Auxiliary Waveguide
\end{verbatim}

\textbf{Working Principle:}

\begin{itemize}
\tightlist
\item
  \textbf{Two holes} spaced λ/4 apart couple energy between waveguides
\item
  \textbf{Forward wave}: Coupled signals add at P3, cancel at P4
\item
  \textbf{Reverse wave}: Coupled signals add at P4, cancel at P3
\item
  \textbf{Directivity}: Achieved by proper hole spacing and size
\end{itemize}

\textbf{Coupling Mechanism:}

\begin{itemize}
\tightlist
\item
  \textbf{Electric field coupling} through holes
\item
  \textbf{Phase difference} creates directional coupling
\item
  \textbf{Coupling factor}: C = 10 log(P1/P3) dB
\end{itemize}

\textbf{Performance Parameters:}

{\def\LTcaptype{none} % do not increment counter
\begin{longtable}[]{@{}ll@{}}
\toprule\noalign{}
Parameter & Typical Value \\
\midrule\noalign{}
\endhead
\bottomrule\noalign{}
\endlastfoot
Coupling & 10-30 dB \\
Directivity & 25-40 dB \\
VSWR & \textless1.3 \\
\end{longtable}
}

\end{solutionbox}
\begin{mnemonicbox}
``Two Holes, Two Directions, Total Control''

\end{mnemonicbox}
\subsection*{Question 2(c) [7 marks]}\label{q2c}

\textbf{Describe Propagation of microwaves through waveguide and get the
equation of cut off wavelength.}

\begin{solutionbox}

\textbf{Propagation Theory:} Electromagnetic waves propagate through
waveguide in \textbf{TE and TM modes} with specific field patterns.

\textbf{Wave Equation:} For rectangular waveguide, the wave equation is:
\nabla^{2}E + γ^{2}E = 0

Where γ^{2} = β^{2} - k^{2}

\textbf{Cutoff Wavelength Derivation:}

For \textbf{TE\_mn mode} in rectangular waveguide:

\begin{itemize}
\tightlist
\item
  \textbf{Cutoff frequency}: f\_c = (c/2)\sqrt[(m/a)^{2} + (n/b)^{2}]
\item
  \textbf{Cutoff wavelength}: λ\_c = 2/\sqrt[(m/a)^{2} + (n/b)^{2}]
\end{itemize}

For \textbf{dominant TE_{1}_{0} mode}:

\begin{itemize}
\tightlist
\item
  λ\_c = 2a (where a is broad dimension)
\end{itemize}

\textbf{Propagation Conditions:}

\begin{itemize}
\tightlist
\item
  \textbf{Below cutoff} (f \textless{} f\_c): Evanescent wave,
  exponential decay
\item
  \textbf{Above cutoff} (f \textgreater{} f\_c): Propagating wave
\item
  \textbf{Phase velocity}: v\_p = c/\sqrt[1 - (f\_c/f)^{2}]
\item
  \textbf{Group velocity}: v\_g = c\sqrt[1 - (f\_c/f)^{2}]
\end{itemize}

\textbf{Mode Chart:}

\begin{center}
\textbf{Mermaid Diagram (Code)}
\begin{verbatim}
{Shaded}
{Highlighting}[]
graph LR
    A[TE_{1_{0}] {-}{-}{} B[TE_{2}_{0}]}
    A {-{-}{} C[TE_{0}_{1}]}
    B {-{-}{} D[TE_{1}_{1}]}
    C {-{-}{} D}
{Highlighting}
{Shaded}
\end{verbatim}
\end{center}

\textbf{Key Relations:}

\begin{itemize}
\tightlist
\item
  v\_p \times v\_g = c^{2}
\item
  λ\_g = λ_{0}/\sqrt[1 - (λ_{0}/λ\_c)^{2}]
\end{itemize}

\end{solutionbox}
\begin{mnemonicbox}
``Cut-off Comes, Propagation Proceeds''

\end{mnemonicbox}
\subsection*{Question 2(a OR) [3
marks]}\label{question-2a-or-3-marks}

\textbf{Explain Impedance Matching using Single stub.}

\begin{solutionbox}

\textbf{Principle}: Single stub matching uses a \textbf{short-circuited}
or \textbf{open-circuited} stub to cancel reactive component of load
impedance.

\textbf{Stub Diagram:}

\begin{verbatim}
Source {-{-}{-}|{-}{-}{-}●{-}{-}{-}|{-}{-}{-} Load}
  Z_{0      |       |    Z\_L}
          |       |
         {-{-}{-}      |}
        Stub      |
         l\_s      d
\end{verbatim}

\textbf{Design Steps:}

\begin{itemize}
\tightlist
\item
  \textbf{Step 1}: Find distance `d' where normalized conductance = 1
\item
  \textbf{Step 2}: Calculate required stub susceptance: B\_s =
  -B\_load\\
\item
  \textbf{Step 3}: Determine stub length: l\_s from B\_s
\end{itemize}

\textbf{Smith Chart Method:}

\begin{itemize}
\tightlist
\item
  Plot normalized load impedance
\item
  Move toward generator to find matching point
\item
  Add stub susceptance to achieve center point
\end{itemize}

\end{solutionbox}
\begin{mnemonicbox}
``Single Stub Solves Susceptance''

\end{mnemonicbox}
\subsection*{Question 2(b OR) [4
marks]}\label{question-2b-or-4-marks}

\textbf{Explain Hybrid ring with necessary sketch.}

\begin{solutionbox}

\textbf{Construction Diagram:}

\begin{verbatim}
graph TB
    A[Port 1] {-{-} B[Ring Junction]}
    C[Port 2] {-{-} B}
    D[Port 3] {-{-} B}
    E[Port 4] {-{-} B}
    B {-{-} F[3λ/2 Ring Path]}
\end{verbatim}

\textbf{Working Principle:}

\begin{itemize}
\tightlist
\item
  \textbf{Ring circumference}: 3λ/2 (1.5 wavelengths)
\item
  \textbf{Equal path lengths} from each port to opposite port
\item
  \textbf{180^\circ phase difference} between adjacent ports
\end{itemize}

\textbf{S-Matrix Properties:}

\begin{itemize}
\tightlist
\item
  \textbf{Isolation}: Ports 1-3 and ports 2-4 are isolated
\item
  \textbf{Power division}: Equal split with 180^\circ phase difference
\item
  \textbf{Impedance}: All ports matched to Z_{0}
\end{itemize}

\textbf{Applications:}

\begin{itemize}
\tightlist
\item
  \textbf{Balanced mixers}
\item
  \textbf{Push-pull amplifiers}
\item
  \textbf{Phase comparison circuits}
\end{itemize}

\textbf{Performance Table:}

{\def\LTcaptype{none} % do not increment counter
\begin{longtable}[]{@{}ll@{}}
\toprule\noalign{}
Parameter & Value \\
\midrule\noalign{}
\endhead
\bottomrule\noalign{}
\endlastfoot
Isolation & \textgreater25 dB \\
Return Loss & \textgreater20 dB \\
Phase Balance & \pm5^\circ \\
\end{longtable}
}

\end{solutionbox}
\begin{mnemonicbox}
``Ring Rotates, Ports Pair-up''

\end{mnemonicbox}
\subsection*{Question 2(c OR) [7
marks]}\label{question-2c-or-7-marks}

\textbf{Explain construction, working and any one application of Magic
Tee with necessary diagram.}

\begin{solutionbox}

\textbf{Construction}: Magic Tee is formed by joining \textbf{E-plane}
and \textbf{H-plane} tees at their junction.

\textbf{Structure Diagram:}

\begin{verbatim}
       H{-arm (Sum port)}
           |
           |
    {-{-}{-}{-}{-}{-}{-}●{-}{-}{-}{-}{-}{-}{-} }
   |               |
E{-arm     Junction  Collinear}
(Diff)              arms
   |               |
    {-{-}{-}{-}{-}{-}{-}●{-}{-}{-}{-}{-}{-}{-}}
           |
           |
       Matched load
\end{verbatim}

\textbf{Working Principle:}

\begin{itemize}
\tightlist
\item
  \textbf{Ports 1,2}: Collinear arms (input/output ports)
\item
  \textbf{Port 3}: H-arm (sum/Σ port)\\
\item
  \textbf{Port 4}: E-arm (difference/Δ port)
\item
  \textbf{Isolation}: Between sum and difference ports
\end{itemize}

\textbf{S-Matrix Properties:}

\begin{center}
\textbf{Mermaid Diagram (Code)}
\begin{verbatim}
{Shaded}
{Highlighting}[]
graph LR
    A[Port 1] {-.{-}{}|In phase| B[H{-}arm]}
    C[Port 2] {-.{-}{}|In phase| B}
    A {-{-}{}|Out of phase| D[E{-}arm]}
    C {-{-}{}|180^ phase| D}
{Highlighting}
{Shaded}
\end{verbatim}
\end{center}

\textbf{Application - Radar Duplexer:}

\begin{itemize}
\tightlist
\item
  \textbf{Transmit}: Power fed to H-arm, splits equally to ports 1,2
\item
  \textbf{Receive}: Received signals combine at E-arm for receiver
\item
  \textbf{Isolation}: Protects receiver during transmission
\item
  \textbf{Advantage}: Single antenna for transmit/receive
\end{itemize}

\textbf{Performance Specifications:}

{\def\LTcaptype{none} % do not increment counter
\begin{longtable}[]{@{}ll@{}}
\toprule\noalign{}
Parameter & Value \\
\midrule\noalign{}
\endhead
\bottomrule\noalign{}
\endlastfoot
Isolation & \textgreater30 dB \\
VSWR & \textless1.3 \\
Power Split & 3 dB \\
Phase Balance & \pm5^\circ \\
\end{longtable}
}

\textbf{Key Features:}

\begin{itemize}
\tightlist
\item
  \textbf{Symmetric structure} ensures equal power division
\item
  \textbf{Orthogonal fields} provide port isolation
\item
  \textbf{Broadband operation} over octave bandwidth
\end{itemize}

\end{solutionbox}
\begin{mnemonicbox}
``Magic Makes Isolation, Tee Transmits Together''

\end{mnemonicbox}
\subsection*{Question 3(a) [3 marks]}\label{q3a}

\textbf{Explain Attenuation measurement with the help of block diagram.}

\begin{solutionbox}

\textbf{Block Diagram:}

\begin{center}
\textbf{Mermaid Diagram (Code)}
\begin{verbatim}
{Shaded}
{Highlighting}[]
graph LR
    A[Signal Generator] {-{-}{} B[Attenuator Under Test]}
    B {-{-}{} C[Power Meter]}
    D[Reference Path] {-{-}{} C}
    E[Switch] {-{-}{} B}
    E {-{-}{} D}
{Highlighting}
{Shaded}
\end{verbatim}
\end{center}

\textbf{Measurement Procedure:}

\begin{itemize}
\tightlist
\item
  \textbf{Step 1}: Measure power without attenuator (P_{1})
\item
  \textbf{Step 2}: Insert attenuator, measure power (P_{2})\\
\item
  \textbf{Step 3}: Calculate attenuation = 10 log(P_{1}/P_{2}) dB
\end{itemize}

\textbf{Methods:}

\begin{itemize}
\tightlist
\item
  \textbf{Substitution method}: Compare with known attenuator
\item
  \textbf{Direct method}: Measure input and output power
\item
  \textbf{IF substitution}: Use intermediate frequency
\end{itemize}

\end{solutionbox}
\begin{mnemonicbox}
``Attenuation = Power_{1}/Power_{2}''

\end{mnemonicbox}
\subsection*{Question 3(b) [4 marks]}\label{q3b}

\textbf{Explain velocity modulation in two cavity klystron with the help
of Applegate diagram.}

\begin{solutionbox}

\textbf{Two Cavity Klystron Diagram:}

\begin{verbatim}
Electron {-{-}{-}{-}{-} ●======● {-}{-}{-}{-}{-} ●======● {-}{-}{-}{-}{-} Collector}
Gun            Input    Drift   Output
               Cavity   Space   Cavity
                 |               |
              RF Input         RF Output
\end{verbatim}

\textbf{Applegate Diagram:}

\begin{verbatim}
Distance 
   |     
   |  ╱  ╲     ╱  ╲     ╱  ╲
   | ╱    ╲   ╱    ╲   ╱    ╲
   |╱      ╲ ╱      ╲ ╱      ╲
   |        X        X        X   Bunching
Tim|       ╱ ╲      ╱ ╲      ╱ ╲  
   ↓      ╱   ╲    ╱   ╲    ╱   ╲
         ╱     ╲  ╱     ╲  ╱     ╲
        ╱       ╲╱       ╲╱       ╲
Fast electrons  Slow electrons
\end{verbatim}

\textbf{Velocity Modulation Process:}

\begin{itemize}
\tightlist
\item
  \textbf{Input cavity}: Electrons gain/lose energy from RF field
\item
  \textbf{Drift space}: Fast electrons catch up to slow electrons
\item
  \textbf{Bunching}: Electron density varies periodically
\item
  \textbf{Output cavity}: Bunched electrons induce RF current
\end{itemize}

\textbf{Key Parameters:}

\begin{itemize}
\tightlist
\item
  \textbf{Transit time}: τ = L/v_{0} (where L = drift space length)
\item
  \textbf{Bunching parameter}: X = βn/2
\item
  \textbf{Optimum bunching}: X = 1.84
\end{itemize}

\end{solutionbox}
\begin{mnemonicbox}
``Velocity Varies, Bunching Builds''

\end{mnemonicbox}
\subsection*{Question 3(c) [7 marks]}\label{q3c}

\textbf{Explain the principle, construction and effect of electric and
magnetic field in Magnetron.}

\begin{solutionbox}

\textbf{Principle}: Magnetron uses \textbf{crossed electric and magnetic
fields} to generate high-power microwave oscillations through
\textbf{cyclotron motion} of electrons.

\textbf{Construction Diagram:}

\begin{verbatim}
    Permanent Magnet (N)
         ↓  ↓  ↓  ↓
    ┌─────────────────┐
    │  ○  ○  ○  ○  ○  │  Resonant Cavities
    │○               ○│
    │  ●─ Cathode ─●  │  Central cathode
    │○               ○│
    │  ○  ○  ○  ○  ○  │
    └─────────────────┘
         ↑  ↑  ↑  ↑
    Permanent Magnet (S)
\end{verbatim}

\textbf{Field Effects:}

\begin{itemize}
\tightlist
\item
  \textbf{Electric Field (E)}: Radial, from cathode to anode
\item
  \textbf{Magnetic Field (B)}: Axial, perpendicular to E-field
\item
  \textbf{Crossed fields}: Create cycloidal electron motion
\end{itemize}

\textbf{Electron Motion Analysis:}

\begin{center}
\textbf{Mermaid Diagram (Code)}
\begin{verbatim}
{Shaded}
{Highlighting}[]
graph LR
    A[Electron Emission] {-{-}{} B[Cyclotron Motion]}
    B {-{-}{} C[Spiral Path]}
    C {-{-}{} D[Energy Transfer]}
    D {-{-}{} E[RF Oscillation]}
{Highlighting}
{Shaded}
\end{verbatim}
\end{center}

\textbf{Operating Conditions:}

\begin{itemize}
\tightlist
\item
  \textbf{Cutoff condition}: E/B = v\_drift
\item
  \textbf{Synchronism}: Electron drift velocity matches phase velocity
\item
  \textbf{Hull cutoff}: Minimum magnetic field for operation
\end{itemize}

\textbf{Resonant Cavities:}

\begin{itemize}
\tightlist
\item
  \textbf{π-mode operation}: Alternate cavities have opposite phases
\item
  \textbf{Frequency}: f = c/(2\sqrtLC) for cavity resonance
\item
  \textbf{Mode separation}: Prevents mode competition
\end{itemize}

\textbf{Performance Characteristics:}

{\def\LTcaptype{none} % do not increment counter
\begin{longtable}[]{@{}ll@{}}
\toprule\noalign{}
Parameter & Typical Value \\
\midrule\noalign{}
\endhead
\bottomrule\noalign{}
\endlastfoot
Efficiency & 60-80\% \\
Power Output & 10 kW - 10 MW \\
Frequency & 1-100 GHz \\
Pulse/CW & Both modes \\
\end{longtable}
}

\textbf{Advantages:}

\begin{itemize}
\tightlist
\item
  \textbf{High efficiency} compared to other tubes
\item
  \textbf{High power capability}
\item
  \textbf{Compact structure}
\item
  \textbf{Good frequency stability}
\end{itemize}

\textbf{Applications:}

\begin{itemize}
\tightlist
\item
  \textbf{Radar transmitters}
\item
  \textbf{Microwave ovens}
\item
  \textbf{Industrial heating}
\item
  \textbf{Electronic warfare}
\end{itemize}

\end{solutionbox}
\begin{mnemonicbox}
``Magnetron Makes Microwaves via Magnetic Motion''

\end{mnemonicbox}
\subsection*{Question 3(a OR) [3
marks]}\label{question-3a-or-3-marks}

\textbf{Explain working of TWT (Travelling Wave Tube) as an Amplifier.}

\begin{solutionbox}

\textbf{TWT Structure:}

\begin{center}
\textbf{Mermaid Diagram (Code)}
\begin{verbatim}
{Shaded}
{Highlighting}[]
graph LR
    A[Electron Gun] {-{-}{} B[Helix]}
    B {-{-}{} C[Collector]}
    D[RF Input] {-{-}{} B}
    B {-{-}{} E[RF Output]}
{Highlighting}
{Shaded}
\end{verbatim}
\end{center}

\textbf{Amplification Process:}

\begin{itemize}
\tightlist
\item
  \textbf{Electron beam} travels along helix axis
\item
  \textbf{RF signal} propagates along helix (slow wave structure)
\item
  \textbf{Velocity synchronism}: v\_electron \approx v\_RF
\item
  \textbf{Energy transfer} from DC beam to RF wave
\end{itemize}

\textbf{Gain Mechanism:}

\begin{itemize}
\tightlist
\item
  \textbf{Bunching}: RF field modulates electron velocity
\item
  \textbf{Induced current}: Bunched electrons induce RF current in helix
\item
  \textbf{Progressive amplification} along helix length
\end{itemize}

\end{solutionbox}
\begin{mnemonicbox}
``Travelling Wave Transfers Energy''

\end{mnemonicbox}
\subsection*{Question 3(b OR) [4
marks]}\label{question-3b-or-4-marks}

\textbf{Explain Bolometer method for low power measurement at microwave
frequency.}

\begin{solutionbox}

\textbf{Principle}: Bolometer measures microwave power by detecting
\textbf{temperature rise} in resistive element.

\textbf{Bolometer Types:}

\begin{itemize}
\tightlist
\item
  \textbf{Thermistor}: Negative temperature coefficient
\item
  \textbf{Barretter}: Positive temperature coefficient
\end{itemize}

\textbf{Circuit Diagram:}

\begin{verbatim}
    RF Power {-{-}{-}{-}{-} [Bolometer] {-}{-}{-}{-}{-} Temperature}
         |              |               Change
         |              |                 |
    DC Bridge {-{-}{-}{-}{-}{-}{-}{-}{-}●{-}{-}{-}{-}{-}{-}{-}{-}{-} DC Voltmeter}
\end{verbatim}

\textbf{Measurement Process:}

\begin{itemize}
\tightlist
\item
  \textbf{Step 1}: Balance bridge with DC power only
\item
  \textbf{Step 2}: Apply RF power, note bridge unbalance
\item
  \textbf{Step 3}: Reduce DC power to rebalance bridge
\item
  \textbf{Step 4}: RF power = Reduction in DC power
\end{itemize}

\textbf{Advantages:}

\begin{itemize}
\tightlist
\item
  \textbf{High sensitivity} (µW to mW range)
\item
  \textbf{Square law response}
\item
  \textbf{Broadband operation}
\end{itemize}

\end{solutionbox}
\begin{mnemonicbox}
``Bolometer Burns, Bridge Balances''

\end{mnemonicbox}
\subsection*{Question 3(c OR) [7
marks]}\label{question-3c-or-7-marks}

\textbf{Explain frequency and wavelength measurement method with the
help of block diagram.}

\begin{solutionbox}

\textbf{Frequency Measurement - Direct Method:}

\begin{center}
\textbf{Mermaid Diagram (Code)}
\begin{verbatim}
{Shaded}
{Highlighting}[]
graph LR
    A[Microwave Source] {-{-}{} B[Frequency Counter]}
    B {-{-}{} C[Digital Display]}
    D[Reference Oscillator] {-{-}{} B}
{Highlighting}
{Shaded}
\end{verbatim}
\end{center}

\textbf{Frequency Measurement - Heterodyne Method:}

\begin{center}
\textbf{Mermaid Diagram (Code)}
\begin{verbatim}
{Shaded}
{Highlighting}[]
graph LR
    A[Unknown Frequency] {-{-}{} B[Mixer]}
    C[Local Oscillator] {-{-}{} B}
    B {-{-}{} D[IF Amplifier]}
    D {-{-}{} E[Frequency Counter]}
{Highlighting}
{Shaded}
\end{verbatim}
\end{center}

\textbf{Wavelength Measurement - Slotted Line Method:}

\textbf{Setup Diagram:}

\begin{verbatim}
Microwave {-{-}|{-}{-}{-}{-}{-}|====|{-}{-}{-}{-}{-}|{-}{-} Slotted Line {-}{-}|{-}{-} Load}
Source      |  Isolator  |                     |
            |            |                     |
         Attenuator   Detector              Movable
                                           Probe
\end{verbatim}

\textbf{Measurement Procedure:}

\textbf{Free Space Wavelength (λ_{0}):}

\begin{itemize}
\tightlist
\item
  \textbf{Step 1}: Connect matched load, measure frequency
\item
  \textbf{Step 2}: Calculate λ_{0} = c/f
\end{itemize}

\textbf{Guided Wavelength (λ\_g):}

\begin{itemize}
\tightlist
\item
  \textbf{Step 1}: Connect short circuit, find two consecutive minima
\item
  \textbf{Step 2}: λ\_g = 2 \times distance between minima
\item
  \textbf{Step 3}: Verify: λ\_g = λ_{0}/\sqrt[1-(λ_{0}/λ\_c)^{2}]
\end{itemize}

\textbf{Cut-off Wavelength (λ\_c):}

\begin{itemize}
\tightlist
\item
  \textbf{Method 1}: From waveguide dimensions: λ\_c = 2a (for TE_{1}_{0})
\item
  \textbf{Method 2}: From λ_{0} and λ\_g: λ\_c = λ_{0}/\sqrt[1-(λ_{0}/λ\_g)^{2}]
\end{itemize}

\textbf{Measurement Table:}

{\def\LTcaptype{none} % do not increment counter
\begin{longtable}[]{@{}lll@{}}
\toprule\noalign{}
Parameter & Method & Accuracy \\
\midrule\noalign{}
\endhead
\bottomrule\noalign{}
\endlastfoot
Frequency & Direct counting & \pm0.01\% \\
λ_{0} & Calculate from f & \pm0.01\% \\
λ\_g & Slotted line & \pm1\% \\
λ\_c & Calculation/measurement & \pm2\% \\
\end{longtable}
}

\textbf{Advantages of Each Method:}

\begin{itemize}
\tightlist
\item
  \textbf{Direct method}: High accuracy, simple
\item
  \textbf{Heterodyne method}: Extended frequency range
\item
  \textbf{Slotted line}: Measures guided parameters directly
\end{itemize}

\textbf{Error Sources:}

\begin{itemize}
\tightlist
\item
  \textbf{Probe coupling} variations
\item
  \textbf{Temperature effects} on dimensions
\item
  \textbf{Detector nonlinearity}
\item
  \textbf{Standing wave} disturbances
\end{itemize}

\textbf{Applications:}

\begin{itemize}
\tightlist
\item
  \textbf{Waveguide characterization}
\item
  \textbf{Material property measurement}
\item
  \textbf{Antenna testing}
\item
  \textbf{Component verification}
\end{itemize}

\end{solutionbox}
\begin{mnemonicbox}
``Frequency First, Wavelength With-measurement''

\end{mnemonicbox}
\subsection*{Question 4(a) [3 marks]}\label{q4a}

\textbf{State Frequency limitations of vacuum tubes at microwave
frequency.}

\begin{solutionbox}

\textbf{Frequency Limitations:}

\begin{itemize}
\tightlist
\item
  \textbf{Transit time effects}: Electron transit time becomes
  comparable to RF period
\item
  \textbf{Inter-electrode capacitance}: Reduces gain at high
  frequencies\\
\item
  \textbf{Lead inductance}: Parasitic inductances limit performance
\item
  \textbf{Skin effect}: Current concentration reduces effective
  conductance
\end{itemize}

\textbf{Limiting Factors Table:}

{\def\LTcaptype{none} % do not increment counter
\begin{longtable}[]{@{}lll@{}}
\toprule\noalign{}
Factor & Effect & Frequency Impact \\
\midrule\noalign{}
\endhead
\bottomrule\noalign{}
\endlastfoot
Transit Time & Phase delay & f \textless{} 1/(2πτ) \\
Capacitance & Reactance loading & Gain ∝ 1/f \\
Inductance & Resonance effects & Stability issues \\
Skin Effect & Increased resistance & Efficiency ↓ \\
\end{longtable}
}

\textbf{Solutions:}

\begin{itemize}
\tightlist
\item
  \textbf{Reduce electrode spacing}
\item
  \textbf{Use special geometries}
\item
  \textbf{Employ microwave tubes} (Klystron, Magnetron)
\end{itemize}

\end{solutionbox}
\begin{mnemonicbox}
``Transit Time Troubles Traditional Tubes''

\end{mnemonicbox}
\subsection*{Question 4(b) [4 marks]}\label{q4b}

\textbf{Explain Negative resistance effect in IMPATT Diode.}

\begin{solutionbox}

\textbf{IMPATT Structure:}

\begin{verbatim}
P+ |{-{-}| I |{-}{-}| P |{-}{-}| N+ |}
   {-{-}|{-}{-}{-}{-}{-}{-}|{-}{-}{-}{-}{-}{-}|{-}{-}}
   Avalanche  Drift
   Region     Region
\end{verbatim}

\textbf{Negative Resistance Mechanism:}

\textbf{Two-step Process:}

\begin{enumerate}
\tightlist
\item
  \textbf{Impact Ionization}: High field creates electron-hole pairs
\item
  \textbf{Transit Time Delay}: Carriers drift across depletion region
\end{enumerate}

\textbf{Phase Relationships:}

\begin{itemize}
\tightlist
\item
  \textbf{Current}: Lags voltage by 90^\circ (avalanche delay) + 90^\circ (transit
  delay) = 180^\circ
\item
  \textbf{Result}: I = -G*V (negative conductance)
\end{itemize}

\textbf{Operating Cycle:}

\begin{center}
\textbf{Mermaid Diagram (Code)}
\begin{verbatim}
{Shaded}
{Highlighting}[]
graph LR
    A[High Field] {-{-}{} B[Avalanche]}
    B {-{-}{} C[Carrier Generation]}
    C {-{-}{} D[Transit Delay]}
    D {-{-}{} E[Current Peak]}
    E {-{-}{} A}
{Highlighting}
{Shaded}
\end{verbatim}
\end{center}

\textbf{Key Parameters:}

\begin{itemize}
\tightlist
\item
  \textbf{Avalanche frequency}: f\_a = v\_s/(2W\_a)
\item
  \textbf{Transit frequency}: f\_t = v\_d/(2W\_d)
\item
  \textbf{Optimum frequency}: f\_0 =
  1/(2π\sqrtL*\textbar C\_negative\textbar)
\end{itemize}

\end{solutionbox}
\begin{mnemonicbox}
``Impact Ionization, Transit Time = Negative
Resistance''

\end{mnemonicbox}
\subsection*{Question 4(c) [7 marks]}\label{q4c}

\textbf{Explain Principle, tunneling phenomenon and any one application
of Tunnel Diode.}

\begin{solutionbox}

\textbf{Principle}: Tunnel diode operates on \textbf{quantum mechanical
tunneling} effect through thin potential barrier in heavily doped p-n
junction.

\textbf{Energy Band Diagram:}

\begin{verbatim}
Forward Bias States:

State 1 (Low bias):    State 2 (Peak):      State 3 (Valley):
   P side | N side        P side | N side        P side | N side
    \_\_\_   |  \_\_\_           \_\_\_   |  \_\_\_           \_\_\_   |  \_\_\_
   |   |  | |   |         |   |  | |   |         |   |  | |   |
   |\_\_\_|  | |\_\_\_|         |\_\_\_| /| |\_\_\_|         |\_\_\_| /| |\_\_\_|
          |               Tunneling              No tunnel
       Tunneling                               
\end{verbatim}

\textbf{I-V Characteristics:}

\begin{verbatim}
Current ↑
        |    
     Ip |●     
        |  ●    
        |    ●   Forward region
        |      ●  
        |        ●
     Iv |         ●\_\_\_\_
        |                ●
        |                  ●
        |\_\_\_\_\_\_\_\_\_\_\_\_\_\_\_\_\_\_\_\_\_\_\_\_ Voltage
        0   Vp    Vv    Vf
        
    Peak    Valley  Forward
    point   point   region
\end{verbatim}

\textbf{Tunneling Phenomenon:}

\textbf{Quantum Mechanics}: Electrons can penetrate potential barrier
even if their energy is less than barrier height.

\textbf{Tunneling Probability}: T = exp(-2\sqrt(2m\emph{φ}d^{2})/ħ) Where:

\begin{itemize}
\tightlist
\item
  m = electron mass
\item
  φ = barrier height\\
\item
  d = barrier width
\item
  ħ = reduced Planck constant
\end{itemize}

\textbf{Operating Regions:}

\begin{itemize}
\tightlist
\item
  \textbf{Tunneling region} (0 to Vp): Current increases with voltage
\item
  \textbf{Negative resistance} (Vp to Vv): Current decreases with
  increasing voltage
\item
  \textbf{Forward bias} (\textgreater Vv): Normal diode behavior
\end{itemize}

\textbf{Key Parameters Table:}

{\def\LTcaptype{none} % do not increment counter
\begin{longtable}[]{@{}lll@{}}
\toprule\noalign{}
Parameter & Symbol & Typical Value \\
\midrule\noalign{}
\endhead
\bottomrule\noalign{}
\endlastfoot
Peak Current & Ip & 1-100 mA \\
Peak Voltage & Vp & 50-100 mV \\
Valley Current & Iv & 0.1*Ip \\
Valley Voltage & Vv & 300-500 mV \\
\end{longtable}
}

\textbf{Application - High Frequency Oscillator:}

\textbf{Circuit Diagram:}

\begin{verbatim}
    +Vcc
      |
      R  Bias resistor
      |
      ●{-{-}{-}L{-}{-}{-}●{-}{-}{-}Output}
      |       |
   Tunnel     C
   Diode      |
      |       |
    ──┴──   ──┴──
     GND     GND
\end{verbatim}

\textbf{Oscillator Operation:}

\begin{itemize}
\tightlist
\item
  \textbf{Bias point}: Set in negative resistance region
\item
  \textbf{Tank circuit}: LC determines oscillation frequency
\item
  \textbf{Condition}: \textbar R\_negative\textbar{} \textgreater{}
  R\_positive for oscillation
\item
  \textbf{Frequency}: f = 1/(2π\sqrtLC)
\end{itemize}

\textbf{Advantages:}

\begin{itemize}
\tightlist
\item
  \textbf{Ultra-high frequency} operation (up to 100 GHz)
\item
  \textbf{Low noise} figure
\item
  \textbf{Fast switching} (picosecond range)
\item
  \textbf{Low power consumption}
\item
  \textbf{Temperature stable}
\end{itemize}

\textbf{Applications:}

\begin{itemize}
\tightlist
\item
  \textbf{Microwave oscillators}
\item
  \textbf{High-speed switches}
\item
  \textbf{Microwave amplifiers}
\item
  \textbf{Frequency converters}
\item
  \textbf{Computer memory circuits}
\end{itemize}

\textbf{Limitations:}

\begin{itemize}
\tightlist
\item
  \textbf{Low power handling}
\item
  \textbf{Critical bias requirements}
\item
  \textbf{Limited temperature range}
\item
  \textbf{Expensive manufacturing}
\end{itemize}

\textbf{Design Considerations:}

\begin{itemize}
\tightlist
\item
  \textbf{Doping concentration}: \textgreater10^{1}^{9} cm^{-}^{3} for both sides
\item
  \textbf{Junction area}: Small for high frequency operation\\
\item
  \textbf{Parasitic elements}: Minimize package inductance/capacitance
\item
  \textbf{Bias stability}: Temperature compensation required
\end{itemize}

\end{solutionbox}
\begin{mnemonicbox}
``Tunnel Through, Negative Grows, Oscillator Flows''

\end{mnemonicbox}
\subsection*{Question 4(a OR) [3
marks]}\label{question-4a-or-3-marks}

\textbf{Explain Hazards due to microwave radiation.}

\begin{solutionbox}

\textbf{Types of Hazards:}

\textbf{HERP (Hazards of Electromagnetic Radiation to Personnel):}

\begin{itemize}
\tightlist
\item
  \textbf{Thermal effects}: Tissue heating above 41^\circC
\item
  \textbf{Non-thermal effects}: Cellular damage at low power levels
\item
  \textbf{Cumulative effects}: Long-term exposure risks
\end{itemize}

\textbf{HERO (Hazards of Electromagnetic Radiation to Ordnance):}

\begin{itemize}
\tightlist
\item
  \textbf{Premature ignition}: RF energy can trigger explosive devices
\item
  \textbf{Fuel ignition}: Microwave heating of fuel vapors
\item
  \textbf{Electronic interference}: Disruption of control systems
\end{itemize}

\textbf{HERF (Hazards of Electromagnetic Radiation to Fuels):}

\begin{itemize}
\tightlist
\item
  \textbf{Fuel heating}: Dielectric heating of hydrocarbon fuels
\item
  \textbf{Static discharge}: RF-induced sparking in fuel systems
\item
  \textbf{Vapor ignition}: Heating of fuel-air mixtures
\end{itemize}

\textbf{Safety Guidelines Table:}

{\def\LTcaptype{none} % do not increment counter
\begin{longtable}[]{@{}llll@{}}
\toprule\noalign{}
Exposure Level & Power Density & Duration & Effect \\
\midrule\noalign{}
\endhead
\bottomrule\noalign{}
\endlastfoot
Safe & \textless10 mW/cm^{2} & 8 hours & No effect \\
Caution & 10-100 mW/cm^{2} & Limited & Possible heating \\
Danger & \textgreater100 mW/cm^{2} & Avoid & Tissue damage \\
\end{longtable}
}

\end{solutionbox}
\begin{mnemonicbox}
``HERP-HERO-HERF = Health-Explosive-Fuel Risks''

\end{mnemonicbox}
\subsection*{Question 4(b OR) [4
marks]}\label{question-4b-or-4-marks}

\textbf{Explain Degenerate and non-degenerate mode in Parametric
Amplifier.}

\begin{solutionbox}

\textbf{Parametric Amplifier Principle}: Uses \textbf{time-varying
reactance} to transfer energy from pump to signal.

\textbf{Mode Classifications:}

\textbf{Non-degenerate Mode:}

\begin{itemize}
\tightlist
\item
  \textbf{Three frequencies}: f\_s (signal), f\_i (idler), f\_p (pump)
\item
  \textbf{Frequency relation}: f\_p = f\_s + f\_i
\item
  \textbf{Two separate circuits} for signal and idler
\item
  \textbf{Higher gain} but more complex
\end{itemize}

\textbf{Degenerate Mode:}

\begin{itemize}
\tightlist
\item
  \textbf{Two frequencies}: f\_s (signal), f\_p (pump)\\
\item
  \textbf{Frequency relation}: f\_p = 2f\_s
\item
  \textbf{Single resonant circuit}
\item
  \textbf{Simpler design} but lower gain
\end{itemize}

\textbf{Comparison Table:}

{\def\LTcaptype{none} % do not increment counter
\begin{longtable}[]{@{}lll@{}}
\toprule\noalign{}
Parameter & Non-degenerate & Degenerate \\
\midrule\noalign{}
\endhead
\bottomrule\noalign{}
\endlastfoot
Frequencies & 3 (fs, fi, fp) & 2 (fs, fp) \\
Circuits & Separate & Combined \\
Gain & Higher & Lower \\
Complexity & More & Less \\
Bandwidth & Narrower & Wider \\
\end{longtable}
}

\textbf{Energy Transfer:}

\begin{center}
\textbf{Mermaid Diagram (Code)}
\begin{verbatim}
{Shaded}
{Highlighting}[]
graph LR
    A[Pump Power] {-{-}{} B[Variable Reactance]}
    B {-{-}{} C[Signal Amplification]}
    D[Idler] {-.{-}{} B}
{Highlighting}
{Shaded}
\end{verbatim}
\end{center}

\end{solutionbox}
\begin{mnemonicbox}
``Non-degenerate = Not-single, Degenerate =
Doubled-frequency''

\end{mnemonicbox}
\subsection*{Question 4(c OR) [7
marks]}\label{question-4c-or-7-marks}

\textbf{Explain principle and Gunn effect in Gunn Diode. Also Explain
Gunn Diode as an Oscillator.}

\begin{solutionbox}

\textbf{Gunn Effect Principle}: Based on \textbf{transferred electron
effect} in compound semiconductors (GaAs, InP).

\textbf{Energy Band Structure:}

\begin{verbatim}
Energy ↑
       |     Upper valley
       |    /
       |   /  ΔE = 0.36 eV
       |  /
       |\_/\_\_\_\_\_\_\_ Lower valley
              |
              | k (momentum)
        Γ valley   L valley
\end{verbatim}

\textbf{Gunn Effect Mechanism:}

\textbf{Differential Mobility:}

\begin{itemize}
\tightlist
\item
  \textbf{Low field} (\textless3 kV/cm): Electrons in Γ valley (high
  mobility)
\item
  \textbf{High field} (\textgreater3 kV/cm): Electrons transfer to L
  valley (low mobility)
\item
  \textbf{Result}: Negative differential mobility (NDM)
\end{itemize}

\textbf{Domain Formation:}

\begin{center}
\textbf{Mermaid Diagram (Code)}
\begin{verbatim}
{Shaded}
{Highlighting}[]
graph LR
    A[Uniform Field] {-{-}{} B[Instability]}
    B {-{-}{} C[Domain Nucleation]}
    C {-{-}{} D[Domain Growth]}
    D {-{-}{} E[Domain Transit]}
    E {-{-}{} F[Domain Collection]}
    F {-{-}{} A}
{Highlighting}
{Shaded}
\end{verbatim}
\end{center}

\textbf{Current-Voltage Characteristics:}

\begin{verbatim}
Current ↑
        |
    I\_p |●
        | ●
        |  ●
        |   ●\_\_\_\_\_ NDM region
        |        ●
        |         ●
        |\_\_\_\_\_\_\_\_\_\_●\_\_\_\_\_\_\_\_\_ Voltage
        0    V\_th    V\_s
        
    Threshold  Sustaining
    voltage    voltage
\end{verbatim}

\textbf{Gunn Diode Oscillator:}

\textbf{Basic Configuration:}

\begin{verbatim}
    +V\_bias
      |
      R  Bias resistor  
      |
    ──●──── RF Output
      |
   [Gunn]   Gunn diode
   Diode   
      |
    ──┴──── Ground
     GND
\end{verbatim}

\textbf{Oscillator Modes:}

\textbf{Transit Time Mode:}

\begin{itemize}
\tightlist
\item
  \textbf{Domain formation} at cathode
\item
  \textbf{Domain transit} across active region\\
\item
  \textbf{Current pulse} when domain reaches anode
\item
  \textbf{Frequency}: f = v\_d/L (where v\_d = drift velocity, L =
  length)
\end{itemize}

\textbf{Quenched Domain Mode:}

\begin{itemize}
\tightlist
\item
  \textbf{Resonant circuit} quenches domain before transit
\item
  \textbf{Higher frequency} operation possible
\item
  \textbf{Efficiency}: 5-20\%
\end{itemize}

\textbf{LSA (Limited Space-charge Accumulation) Mode:}

\begin{itemize}
\tightlist
\item
  \textbf{High frequency} prevents domain formation
\item
  \textbf{Uniform field} maintained
\item
  \textbf{Higher efficiency}: 10-25\%
\end{itemize}

\textbf{Performance Parameters:}

{\def\LTcaptype{none} % do not increment counter
\begin{longtable}[]{@{}lll@{}}
\toprule\noalign{}
Parameter & Value & Unit \\
\midrule\noalign{}
\endhead
\bottomrule\noalign{}
\endlastfoot
Frequency Range & 1-100 & GHz \\
Power Output & 1 mW-10 W & - \\
Efficiency & 5-25 & \% \\
Noise Figure & 35-50 & dB \\
\end{longtable}
}

\textbf{Advantages:}

\begin{itemize}
\tightlist
\item
  \textbf{Simple structure} - no external resonator needed
\item
  \textbf{Broadband tuning} capability
\item
  \textbf{Low noise} at microwave frequencies
\item
  \textbf{Reliable operation}
\end{itemize}

\textbf{Applications:}

\begin{itemize}
\tightlist
\item
  \textbf{Local oscillators} in receivers
\item
  \textbf{CW radar transmitters}\\
\item
  \textbf{Microwave communication systems}
\item
  \textbf{Test equipment signal sources}
\end{itemize}

\textbf{Design Considerations:}

\begin{itemize}
\tightlist
\item
  \textbf{Doping profile}: Uniform n-type doping
\item
  \textbf{Length optimization}: L = v\_d/f for transit time mode
\item
  \textbf{Thermal management}: Heat dissipation critical
\item
  \textbf{Circuit design}: Impedance matching important
\end{itemize}

\textbf{Comparison with Other Oscillators:}

{\def\LTcaptype{none} % do not increment counter
\begin{longtable}[]{@{}llll@{}}
\toprule\noalign{}
Oscillator & Frequency & Power & Efficiency \\
\midrule\noalign{}
\endhead
\bottomrule\noalign{}
\endlastfoot
Gunn Diode & 1-100 GHz & mW-10W & 5-25\% \\
IMPATT & 1-300 GHz & 1W-100W & 10-20\% \\
Klystron & 1-20 GHz & 1kW-1MW & 30-60\% \\
\end{longtable}
}

\end{solutionbox}
\begin{mnemonicbox}
``Gunn Gets Going via Gallium-Arsenide''

\end{mnemonicbox}
\subsection*{Question 5(a) [3 marks]}\label{q5a}

\textbf{Explain working principle of Basic RADAR system with the help of
block diagram.}

\begin{solutionbox}

\textbf{RADAR Principle}: \textbf{Radio Detection And Ranging} -
transmits RF pulses and detects reflected signals from targets.

\textbf{Basic RADAR Block Diagram:}

\begin{center}
\textbf{Mermaid Diagram (Code)}
\begin{verbatim}
{Shaded}
{Highlighting}[]
graph LR
    A[Master Oscillator] {-{-}{} B[Modulator]}
    B {-{-}{} C[Power Amplifier]}
    C {-{-}{} D[Duplexer]}
    D {-{-}{} E[Antenna]}
    E {-{-}{} F[Target]}
    F {-{-}{} E}
    E {-{-}{} D}
    D {-{-}{} G[Receiver]}
    G {-{-}{} H[Signal Processor]}
    H {-{-}{} I[Display]}
    J[Timing Control] {-{-}{} B}
{Highlighting}
{Shaded}
\end{verbatim}
\end{center}

\textbf{Working Principle:}

\begin{itemize}
\tightlist
\item
  \textbf{Transmission}: High power RF pulse transmitted toward target
\item
  \textbf{Propagation}: EM wave travels at speed of light (c)
\item
  \textbf{Reflection}: Target reflects portion of energy back to radar
\item
  \textbf{Reception}: Reflected signal received and processed
\item
  \textbf{Range calculation}: R = (c \times t)/2
\end{itemize}

\textbf{Key Parameters:}

\begin{itemize}
\tightlist
\item
  \textbf{Pulse width}: τ = 0.1 to 10 μs
\item
  \textbf{Pulse repetition frequency}: PRF = 100 Hz to 10 kHz
\item
  \textbf{Peak power}: 1 kW to 10 MW
\end{itemize}

\end{solutionbox}
\begin{mnemonicbox}
``RADAR Ranges by Round-trip Reflection''

\end{mnemonicbox}
\subsection*{Question 5(b) [4 marks]}\label{q5b}

\textbf{Explain A-scope display method with the help of proper figure.}

\begin{solutionbox}

\textbf{A-Scope Display}: Shows \textbf{amplitude vs time} relationship
of received echoes.

\textbf{A-Scope Presentation:}

\begin{verbatim}
Amplitude ↑
          |
          |    ●  Target echo
          |   /|{  }
     Main |  / | {  }
    pulse | /  |  {  }
          |/   |   {}
          |    |    {\_\_\_}
          |\_\_\_\_|\_\_\_\_\_\_\_\_\_{\_\_\_\_\_\_ Time}
          0    |         
               |
           2R/c (Range)
           
    
     clutter clutter
\end{verbatim}

\textbf{Display Components:}

\begin{itemize}
\tightlist
\item
  \textbf{Main pulse}: Initial transmitted pulse (reference)
\item
  \textbf{Ground clutter}: Reflections from nearby terrain
\item
  \textbf{Sea clutter}: Reflections from sea surface\\
\item
  \textbf{Target echo}: Reflection from actual target
\item
  \textbf{Noise}: Random background signals
\end{itemize}

\textbf{Range Measurement:}

\begin{itemize}
\tightlist
\item
  \textbf{Horizontal axis}: Time (proportional to range)
\item
  \textbf{Vertical axis}: Signal amplitude
\item
  \textbf{Range formula}: R = (c \times t)/2
\end{itemize}

\textbf{Applications:}

\begin{itemize}
\tightlist
\item
  \textbf{Air traffic control}
\item
  \textbf{Height finding radars}\\
\item
  \textbf{Range measurement}
\item
  \textbf{Signal analysis}
\end{itemize}

\end{solutionbox}
\begin{mnemonicbox}
``A-scope shows Amplitude Along time Axis''

\end{mnemonicbox}
\subsection*{Question 5(c) [7 marks]}\label{q5c}

\textbf{Explain Doppler effect and working of MTI (Moving Target
Indicator) RADAR system with the help of block diagram.}

\begin{solutionbox}

\textbf{Doppler Effect}: Frequency shift occurs when there is relative
motion between radar and target.

\textbf{Doppler Frequency Shift:} f\_d = (2 \times v\_r \times f\_0)/c

Where:

\begin{itemize}
\tightlist
\item
  f\_d = Doppler frequency shift
\item
  v\_r = radial velocity of target
\item
  f\_0 = transmitted frequency\\
\item
  c = speed of light
\end{itemize}

\textbf{Doppler Shift Cases:}

\begin{itemize}
\tightlist
\item
  \textbf{Approaching target}: f\_d \textgreater{} 0 (positive shift)
\item
  \textbf{Receding target}: f\_d \textless{} 0 (negative shift)
\item
  \textbf{Stationary target}: f\_d = 0 (no shift)
\end{itemize}

\textbf{MTI RADAR Block Diagram:}

\begin{center}
\textbf{Mermaid Diagram (Code)}
\begin{verbatim}
{Shaded}
{Highlighting}[]
graph LR
    A[Transmitter] {-{-}{} B[Duplexer]}
    B {-{-}{} C[Antenna]}
    C {-{-}{} D[Target]}
    D {-{-}{} C}
    C {-{-}{} B}
    B {-{-}{} E[Receiver]}
    F[STALO] {-{-}{} G[Mixer 1]}
    H[COHO] {-{-}{} I[Phase Detector]}
    E {-{-}{} G}
    G {-{-}{} J[IF Amplifier]}
    J {-{-}{} K[Mixer 2]}
    H {-{-}{} K}
    K {-{-}{} L[Video Amplifier]}
    L {-{-}{} M[Delay Line]}
    M {-{-}{} N[Subtractor]}
    L {-{-}{} N}
    N {-{-}{} O[Display]}
    P[Sync] {-{-}{} A}
    P {-{-}{} H}
{Highlighting}
{Shaded}
\end{verbatim}
\end{center}

\textbf{MTI System Components:}

\textbf{STALO (Stable Local Oscillator):}

\begin{itemize}
\tightlist
\item
  \textbf{Frequency}: Close to transmitted frequency
\item
  \textbf{Stability}: High frequency stability required
\item
  \textbf{Function}: First mixer LO
\end{itemize}

\textbf{COHO (Coherent Oscillator):}

\begin{itemize}
\tightlist
\item
  \textbf{Phase reference}: Maintains phase coherence
\item
  \textbf{Synchronization}: Locked to transmitter phase
\item
  \textbf{Function}: Second mixer LO and phase reference
\end{itemize}

\textbf{MTI Processing:}

\begin{itemize}
\tightlist
\item
  \textbf{Delay line}: Stores previous pulse echo
\item
  \textbf{Subtractor}: Subtracts current from previous pulse
\item
  \textbf{Result}: Stationary targets cancelled, moving targets enhanced
\end{itemize}

\textbf{MTI Transfer Function:}

\begin{verbatim}
|H(f)| ↑
       |     
    1.0|     ●●●     ●●●     ●●●
       |    ●   ●   ●   ●   ●   ●
    0.5|   ●     ● ●     ● ●     ●
       |  ●       ●       ●       ●
     0 |\_●\_\_\_\_\_\_\_●\_\_\_\_\_\_\_●\_\_\_\_\_\_\_●\_\_\_ fd
       0  PRF/4  PRF/2  3PRF/4  PRF
       
        Blind speeds 
\end{verbatim}

\textbf{Blind Speeds}: Targets with certain velocities appear
stationary: v\_blind = (n \times λ \times PRF)/2 (where

n = 1,2,3\ldots)


\textbf{Performance Improvements:}

\textbf{Multi-pulse MTI:}

\begin{itemize}
\tightlist
\item
  \textbf{Multiple delay lines} for better clutter rejection
\item
  \textbf{Staggered PRF} to reduce blind speeds
\item
  \textbf{Weighted coefficients} for optimum response
\end{itemize}

\textbf{Clutter Map:}

\begin{itemize}
\tightlist
\item
  \textbf{Digital memory} stores clutter pattern
\item
  \textbf{Adaptive threshold} adjusts to local clutter level
\item
  \textbf{Automatic updates} track slow clutter changes
\end{itemize}

\textbf{MTI Performance Metrics:}

{\def\LTcaptype{none} % do not increment counter
\begin{longtable}[]{@{}ll@{}}
\toprule\noalign{}
Parameter & Typical Value \\
\midrule\noalign{}
\endhead
\bottomrule\noalign{}
\endlastfoot
Clutter Attenuation & 30-60 dB \\
Minimum Detectable Velocity & 1-10 m/s \\
Blind Speed & λ\timesPRF/2 \\
Improvement Factor & 20-40 dB \\
\end{longtable}
}

\textbf{Advantages:}

\begin{itemize}
\tightlist
\item
  \textbf{Clutter suppression}: Eliminates stationary clutter
\item
  \textbf{Moving target emphasis}: Enhances moving targets
\item
  \textbf{Automatic operation}: Reduces operator workload
\end{itemize}

\textbf{Limitations:}

\begin{itemize}
\tightlist
\item
  \textbf{Blind speeds}: Some velocities undetectable
\item
  \textbf{Tangential targets}: No radial component
\item
  \textbf{Weather effects}: Rain/snow can mask targets
\end{itemize}

\textbf{Applications:}

\begin{itemize}
\tightlist
\item
  \textbf{Air traffic control}: Separates aircraft from ground clutter
\item
  \textbf{Weather radar}: Detects precipitation movement\\
\item
  \textbf{Military surveillance}: Detects moving vehicles
\item
  \textbf{Marine radar}: Reduces sea clutter
\end{itemize}

\end{solutionbox}
\begin{mnemonicbox}
``MTI Makes Targets Identifiable via Doppler
Difference''

\end{mnemonicbox}
\subsection*{Question 5(a OR) [3
marks]}\label{question-5a-or-3-marks}

\textbf{Define: a) Blind speed, and b) MUR}

\begin{solutionbox}

\textbf{Blind Speed:}

\begin{itemize}
\tightlist
\item
  \textbf{Definition}: Target radial velocities that produce zero
  Doppler shift in MTI radar
\item
  \textbf{Formula}: v\_blind = (n \times λ \times PRF)/2
\item
  \textbf{Cause}: Target motion synchronized with pulse repetition
\item
  \textbf{Result}: Moving target appears stationary
\end{itemize}

\textbf{MUR (Maximum Unambiguous Range):}

\begin{itemize}
\tightlist
\item
  \textbf{Definition}: Maximum range at which targets can be detected
  without range ambiguity
\item
  \textbf{Formula}: R\_max = (c \times PRT)/2 = c/(2 \times PRF)
\item
  \textbf{Limitation}: Next pulse transmitted before echo returns
\item
  \textbf{Ambiguity}: Targets beyond MUR appear at incorrect range
\end{itemize}

\textbf{Relationship Table:}

{\def\LTcaptype{none} % do not increment counter
\begin{longtable}[]{@{}lll@{}}
\toprule\noalign{}
Parameter & Formula & Unit \\
\midrule\noalign{}
\endhead
\bottomrule\noalign{}
\endlastfoot
Blind Speed & nλPRF/2 & m/s \\
MUR & c/(2\timesPRF) & meters \\
PRT & 1/PRF & seconds \\
\end{longtable}
}

\end{solutionbox}
\begin{mnemonicbox}
``Blind speed Blocks, MUR Measures maximum''

\end{mnemonicbox}
\subsection*{Question 5(b OR) [4
marks]}\label{question-5b-or-4-marks}

\textbf{Explain the factors affecting Maximum RADAR range.}

\begin{solutionbox}

\textbf{RADAR Range Equation:} R\_max = [(P\_t \times G^{2} \times λ^{2} \times σ)/(64π^{3} \times
P\_min \times L)]\^{}(1/4)

\textbf{Factors Affecting Maximum Range:}

\textbf{Transmitted Power (P\_t):}

\begin{itemize}
\tightlist
\item
  \textbf{Higher power} = greater range
\item
  \textbf{Relationship}: R ∝ P\_t\^{}(1/4)
\item
  \textbf{Limitation}: Peak power limited by transmitter
\end{itemize}

\textbf{Antenna Gain (G):}

\begin{itemize}
\tightlist
\item
  \textbf{Directional antenna} concentrates energy
\item
  \textbf{Relationship}: R ∝ G\^{}(1/2)
\item
  \textbf{Trade-off}: Higher gain = narrower beamwidth
\end{itemize}

\textbf{Wavelength (λ):}

\begin{itemize}
\tightlist
\item
  \textbf{Lower frequency} = better propagation
\item
  \textbf{Relationship}: R ∝ λ\^{}(1/2)
\item
  \textbf{Consideration}: Atmospheric absorption increases with
  frequency
\end{itemize}

\textbf{Target Cross Section (σ):}

\begin{itemize}
\tightlist
\item
  \textbf{Larger targets} reflect more energy
\item
  \textbf{Relationship}: R ∝ σ\^{}(1/4)
\item
  \textbf{Variation}: Depends on target shape, material, aspect angle
\end{itemize}

\textbf{Factors Table:}

{\def\LTcaptype{none} % do not increment counter
\begin{longtable}[]{@{}lll@{}}
\toprule\noalign{}
Factor & Effect on Range & Typical Values \\
\midrule\noalign{}
\endhead
\bottomrule\noalign{}
\endlastfoot
Peak Power & R ∝ Pt\^{}0.25 & 1 kW - 10 MW \\
Antenna Gain & R ∝ G\^{}0.5 & 20 - 50 dB \\
Frequency & R ∝ λ\^{}0.5 & 1 - 100 GHz \\
Target RCS & R ∝ σ\^{}0.25 & 0.1 - 1000 m^{2} \\
\end{longtable}
}

\end{solutionbox}
\begin{mnemonicbox}
``Power-Gain-Lambda-Sigma determine Range''

\end{mnemonicbox}
\subsection*{Question 5(c OR) [7
marks]}\label{question-5c-or-7-marks}

\textbf{Compare Pulsed RADAR and CW Doppler RADAR.}

\begin{solutionbox}

\textbf{Comprehensive Comparison:}

\textbf{Basic Principle:}

\begin{itemize}
\tightlist
\item
  \textbf{Pulsed RADAR}: Transmits high-power pulses, measures
  round-trip time
\item
  \textbf{CW Doppler}: Transmits continuous wave, measures Doppler
  frequency shift
\end{itemize}

\textbf{System Block Diagrams:}

\textbf{Pulsed RADAR:}

\begin{center}
\textbf{Mermaid Diagram (Code)}
\begin{verbatim}
{Shaded}
{Highlighting}[]
graph LR
    A[Pulse Generator] {-{-}{} B[Transmitter]}
    B {-{-}{} C[Duplexer]}
    C {-{-}{} D[Antenna]}
    C {-{-}{} E[Receiver]}
    E {-{-}{} F[Display]}
{Highlighting}
{Shaded}
\end{verbatim}
\end{center}

\textbf{CW Doppler RADAR:}

\begin{center}
\textbf{Mermaid Diagram (Code)}
\begin{verbatim}
{Shaded}
{Highlighting}[]
graph LR
    A[CW Oscillator] {-{-}{} B[Directional Coupler]}
    B {-{-}{} C[Transmit Antenna]}
    D[Receive Antenna] {-{-}{} E[Mixer]}
    B {-{-}{} E}
    E {-{-}{} F[Audio Amplifier]}
    F {-{-}{} G[Display]}
{Highlighting}
{Shaded}
\end{verbatim}
\end{center}

\textbf{Detailed Comparison Table:}

{\def\LTcaptype{none} % do not increment counter
\begin{longtable}[]{@{}lll@{}}
\toprule\noalign{}
Parameter & Pulsed RADAR & CW Doppler RADAR \\
\midrule\noalign{}
\endhead
\bottomrule\noalign{}
\endlastfoot
\textbf{Transmission} & High power pulses & Continuous low power \\
\textbf{Information} & Range + velocity & Velocity only \\
\textbf{Antenna} & Single (duplexer) & Separate Tx/Rx \\
\textbf{Range Capability} & Excellent & None (unless FM-CW) \\
\textbf{Velocity Resolution} & Limited & Excellent \\
\textbf{Peak Power} & Very high (MW) & Low (mW to W) \\
\textbf{Average Power} & Low & Moderate \\
\textbf{Complexity} & High & Simple \\
\textbf{Cost} & Expensive & Economical \\
\textbf{Size} & Large & Compact \\
\end{longtable}
}

\textbf{Performance Characteristics:}

{\def\LTcaptype{none} % do not increment counter
\begin{longtable}[]{@{}lll@{}}
\toprule\noalign{}
Aspect & Pulsed RADAR & CW Doppler RADAR \\
\midrule\noalign{}
\endhead
\bottomrule\noalign{}
\endlastfoot
\textbf{Range Accuracy} & \pm10-100 m & Not applicable \\
\textbf{Velocity Accuracy} & \pm1-10 m/s & \pm0.1-1 m/s \\
\textbf{Minimum Range} & Limited by pulse width & Zero \\
\textbf{Maximum Range} & 10-1000 km & 1-50 km \\
\textbf{Clutter Rejection} & Moderate & Excellent \\
\textbf{Weather Effects} & Significant & Minimal \\
\end{longtable}
}

\textbf{Advantages \& Disadvantages:}

\textbf{Pulsed RADAR Advantages:}

\begin{itemize}
\tightlist
\item
  \textbf{Range measurement} capability
\item
  \textbf{High peak power} for long range
\item
  \textbf{Single antenna} system
\item
  \textbf{Well-established technology}
\end{itemize}

\textbf{Pulsed RADAR Disadvantages:}

\begin{itemize}
\tightlist
\item
  \textbf{Complex circuitry} (duplexer, timing)
\item
  \textbf{High cost} and maintenance\\
\item
  \textbf{Power supply} requirements
\item
  \textbf{Blind ranges} due to pulse width
\end{itemize}

\textbf{CW Doppler Advantages:}

\begin{itemize}
\tightlist
\item
  \textbf{Simple design} and low cost
\item
  \textbf{Excellent velocity resolution}
\item
  \textbf{Continuous monitoring}
\item
  \textbf{Low power consumption}
\item
  \textbf{Compact size}
\end{itemize}

\textbf{CW Doppler Disadvantages:}

\begin{itemize}
\tightlist
\item
  \textbf{No range information}
\item
  \textbf{Separate antennas} required
\item
  \textbf{Limited range} capability
\item
  \textbf{Vulnerable to interference}
\end{itemize}

\textbf{Applications:}

\textbf{Pulsed RADAR Applications:}

\begin{itemize}
\tightlist
\item
  \textbf{Air traffic control}
\item
  \textbf{Weather monitoring}
\item
  \textbf{Military surveillance}
\item
  \textbf{Maritime navigation}
\item
  \textbf{Satellite tracking}
\end{itemize}

\textbf{CW Doppler Applications:}

\begin{itemize}
\tightlist
\item
  \textbf{Traffic speed monitoring}
\item
  \textbf{Sports radar guns}
\item
  \textbf{Burglar alarms}
\item
  \textbf{Automatic door openers}
\item
  \textbf{Heart rate monitoring}
\end{itemize}

\textbf{Hybrid Systems:}

\textbf{Pulse Doppler RADAR:}

\begin{itemize}
\tightlist
\item
  \textbf{Combines advantages} of both systems
\item
  \textbf{Range and velocity} measurement
\item
  \textbf{Higher complexity} but better performance
\end{itemize}

\textbf{FM-CW RADAR:}

\begin{itemize}
\tightlist
\item
  \textbf{Frequency modulated} continuous wave
\item
  \textbf{Range capability} added to CW system
\item
  \textbf{Used in automotive} radar applications
\end{itemize}

\textbf{Selection Criteria:}

{\def\LTcaptype{none} % do not increment counter
\begin{longtable}[]{@{}lll@{}}
\toprule\noalign{}
Requirement & Choose Pulsed & Choose CW Doppler \\
\midrule\noalign{}
\endhead
\bottomrule\noalign{}
\endlastfoot
Range measurement needed & ✓ & ✗ \\
High velocity accuracy & ✗ & ✓ \\
Long range operation & ✓ & ✗ \\
Low cost requirement & ✗ & ✓ \\
Portable application & ✗ & ✓ \\
Weather radar & ✓ & ✗ \\
\end{longtable}
}

\textbf{Future Trends:}

\begin{itemize}
\tightlist
\item
  \textbf{Digital signal processing} improving both types
\item
  \textbf{Software-defined radars} offering flexibility
\item
  \textbf{MIMO techniques} enhancing performance
\item
  \textbf{Integration} with other sensors
\end{itemize}

\end{solutionbox}
\begin{mnemonicbox}
``Pulsed gives Position, CW gives
Continuous-Velocity''

\end{mnemonicbox}

\end{document}
