\documentclass[10pt,a4paper]{article}

% content/resources/templates/preamble.tex
\usepackage[margin=0.6in]{geometry}
\author{Milav Dabgar}
\usepackage{amsmath,amssymb,amsthm}
\usepackage{booktabs}
\usepackage{multirow}
\usepackage{xcolor}
\usepackage{tcolorbox}
\tcbuselibrary{breakable,skins}
\usepackage[colorlinks=true,linkcolor=blue]{hyperref}
\usepackage{titlesec}
\usepackage{enumitem}
\usepackage{tikz}
\usepackage{pgfplots}
\usepackage{circuitikz}
\usepackage[version=4]{mhchem}
\usepackage{longtable}
\usepackage{array}
\usepackage{float}
\usepackage{caption}
\usepackage{listings}

\lstset{
  basicstyle=\small\ttfamily,
  breaklines=true,
  breakatwhitespace=false,
  postbreak=\mbox{\textcolor{red}{$\hookrightarrow$}\space},
  float=false,
  numbers=left,
  numberstyle=\tiny\color{gray},
  numbersep=10pt,
  xleftmargin=2em,
  keywordstyle=\color{blue},
  commentstyle=\color{green!60!black},
  stringstyle=\color{purple},
  backgroundcolor=\color{gray!5},
  showstringspaces=false,
  tabsize=2,
  captionpos=b,
  keepspaces=true,
  columns=flexible
}

\pgfplotsset{compat=1.18}
\usetikzlibrary{shapes,arrows,positioning,calc,patterns,decorations.pathmorphing,decorations.markings,arrows.meta}

% Color scheme
\definecolor{headcolor}{RGB}{0,102,204}
\definecolor{keycolor}{RGB}{220,20,60}
\definecolor{solutioncolor}{RGB}{34,139,34}
\definecolor{mnemoniccolor}{RGB}{148,0,211}
\definecolor{codecolor}{RGB}{0,0,100}

% Spacing
\setlength{\parskip}{3pt}
\setlist[itemize]{nosep}
\setlist[enumerate]{nosep}

% Title formatting
\titleformat{\section}{\Large\bfseries\color{headcolor}}{\thesection}{1em}{}
\titleformat{\subsection}{\large\bfseries\color{headcolor}}{\thesubsection}{1em}{}

% Pandoc tightlist compatibility
\providecommand{\tightlist}{%
  \setlength{\itemsep}{0pt}\setlength{\parskip}{0pt}}

% Pandoc longtable compatibility
\newcounter{none}
\def\thenone{}


% content/resources/templates/gujarati-boxes.tex
\usepackage{fontspec}
\usepackage{polyglossia}

% Set Gujarati as main language (document is primarily in Gujarati)
% Note: gloss-gujarati.ldf doesn't exist in polyglossia, but it will use hyphenation patterns
\setdefaultlanguage{gujarati}
\setotherlanguage{english}

% Configure Gujarati font properly
% Use Language=Default to prevent polyglossia from trying to add language-specific features
% that don't exist for Gujarati, which causes "empty feature" warnings
\newfontfamily\gujaratifont[Script=Gujarati,AutoFakeBold=2.5,AutoFakeSlant=0.3]{Noto Sans Gujarati}
\setmainfont[Script=Gujarati,AutoFakeBold=2.5,AutoFakeSlant=0.3]{Noto Sans Gujarati}
% Use Noto Sans Gujarati for monospace to support Gujarati in text
\setmonofont[Scale=0.9]{Noto Sans Gujarati}

% Configure English to use the same font
\newfontfamily\englishfont[Script=Gujarati,AutoFakeBold=2.5,AutoFakeSlant=0.3]{Noto Sans Gujarati}

% Translations for polyglossia
\gappto\captionsgujarati{
  \renewcommand{\tablename}{કોષ્ટક}
  \renewcommand{\figurename}{આકૃતિ}
}

% Helper for TikZ nodes to ensure Gujarati font
\newcommand{\gu}[1]{{\gujaratifont #1}}

% Custom environments
\newtcolorbox{solutionbox}{
    breakable,
    enhanced,
    colback=solutioncolor!5!white,
    colframe=solutioncolor!75!black,
    fonttitle=\bfseries,
    title=જવાબ
}

\newtcolorbox{solutionboxnobreak}{
 colback=solutioncolor!5!white,
 colframe=solutioncolor!75!black,
 fonttitle=\bfseries,
 title=જવાબ
}

\newtcolorbox{keyformula}{
 breakable,
 enhanced,
 colback=keycolor!5!white,
 colframe=keycolor!75!black,
 fonttitle=\bfseries,
 title=રાસાયણિક સમીકરણ/સૂત્ર
}

\newtcolorbox{mnemonicbox}{
 breakable,
 enhanced,
 colback=mnemoniccolor!5!white,
 colframe=mnemoniccolor!75!black,
 fonttitle=\bfseries,
 title=મેમરી ટ્રીક
}


\begin{document}

\begin{center}
{\Huge\bfseries\color{headcolor} Subject Name (Gujarati)}\\[5pt]
{\LARGE 4351103 -- Winter 2024}\\[3pt]
{\large Semester 1 Study Material}\\[3pt]
{\normalsize\textit{Detailed Solutions and Explanations}}
\end{center}

\vspace{10pt}

\subsection*{પ્રશ્ન 1(અ) [3
ગુણ]}\label{uxaaauxab0uxab6uxaa8-1uxa85-3-uxa97uxaa3}

\textbf{ટ્રાન્સમિશન લાઇન અને વેવગાઇડ વચ્ચે સરખામણી કરો.}

\begin{solutionbox}

{\def\LTcaptype{none} % do not increment counter
\begin{longtable}[]{@{}
  >{\raggedright\arraybackslash}p{(\linewidth - 4\tabcolsep) * \real{0.2750}}
  >{\raggedright\arraybackslash}p{(\linewidth - 4\tabcolsep) * \real{0.4500}}
  >{\raggedright\arraybackslash}p{(\linewidth - 4\tabcolsep) * \real{0.2750}}@{}}
\toprule\noalign{}
\begin{minipage}[b]{\linewidth}\raggedright
પેરામીટર
\end{minipage} & \begin{minipage}[b]{\linewidth}\raggedright
ટ્રાન્સમિશન લાઇન
\end{minipage} & \begin{minipage}[b]{\linewidth}\raggedright
વેવગાઇડ
\end{minipage} \\
\midrule\noalign{}
\endhead
\bottomrule\noalign{}
\endlastfoot
\textbf{ફ્રીક્વન્સી રેન્જ} & નીચી થી મધ્યમ ફ્રીક્વન્સી & ઉચ્ચ ફ્રીક્વન્સી (1 GHz થી
વધુ) \\
\textbf{સ્ટ્રક્ચર} & બે કે વધુ કંડક્ટર & એક હોલો કંડક્ટર \\
\textbf{પ્રોપેગેશન મોડ} & TEM મોડ & TE અને TM મોડ \\
\textbf{પાવર હેન્ડલિંગ} & મર્યાદિત પાવર કેપેસિટી & ઉચ્ચ પાવર હેન્ડલિંગ ક્ષમતા \\
\textbf{લોસેસ} & ઉચ્ચ ફ્રીક્વન્સીએ વધુ નુકસાન & માઇક્રોવેવ ફ્રીક્વન્સીએ ઓછું નુકસાન \\
\end{longtable}
}

\end{solutionbox}
\begin{mnemonicbox}
``વેવ્સ વધુ સારી રીતે ટ્રાવેલ કરે છે''

\end{mnemonicbox}
\begin{center}\rule{0.5\linewidth}{0.5pt}\end{center}

\subsection*{પ્રશ્ન 1(બ) [4
ગુણ]}\label{uxaaauxab0uxab6uxaa8-1uxaac-4-uxa97uxaa3}

\textbf{નીચેની વ્યાખ્યા આપો: (1) લોસલેસ લાઇન (2) VSWR (3) STUB (4) રિફ્લેક્શન
કોઓફીશિઅન્ટ}

\begin{solutionbox}

\begin{itemize}
\item
  \textbf{લોસલેસ લાઇન}: શૂન્ય રેઝિસ્ટન્સ અને કંડક્ટન્સ ધરાવતી ટ્રાન્સમિશન લાઇન, જેમાં
  સિગ્નલ ટ્રાન્સમિશન દરમિયાન કોઈ પાવર લોસ નથી.
\item
  \textbf{VSWR (વોલ્ટેજ સ્ટેન્ડિંગ વેવ રેશિયો)}: ટ્રાન્સમિશન લાઇન પર મેક્સિમમ અને
  મિનિમમ વોલ્ટેજનો રેશિયો, જે ઇમ્પીડન્સ મિસમેચ દર્શાવે છે.
\item
  \textbf{STUB}: ઇમ્પીડન્સ મેચિંગ માટે મુખ્ય લાઇન સાથે જોડાયેલી ટ્રાન્સમિશન લાઇનનો
  ટૂંકો ભાગ.
\item
  \textbf{રિફ્લેક્શન કોઓફીશિઅન્ટ}: ટ્રાન્સમિશન લાઇન પર કોઈપણ બિંદુએ રિફ્લેક્ટેડ વેવ
  અને ઇન્સિડન્ટ વેવના એમ્પ્લિટ્યુડનો રેશિયો.
\end{itemize}

\end{solutionbox}
\begin{mnemonicbox}
``લાઇટ વોલ્યુમ સ્ટે રિફ્લેક્ટેડ''

\end{mnemonicbox}
\begin{center}\rule{0.5\linewidth}{0.5pt}\end{center}

\subsection*{પ્રશ્ન 1(ક) [7
ગુણ]}\label{uxaaauxab0uxab6uxaa8-1uxa95-7-uxa97uxaa3}

\textbf{આઇસોલેટર અને સર્ક્યુલેટર આકૃતિની મદદથી સમજાવો.}

\begin{solutionbox}

\begin{center}
\textbf{Mermaid Diagram (Code)}
\begin{verbatim}
{Shaded}
{Highlighting}[]
graph LR
    A[પોર્ટ 1] {-{-}{} B[આઇસોલેટર] {-}{-}{} C[પોર્ટ 2]}
    B {-.X.{-}{} A}

    D[પોર્ટ 1] {-{-}{} E((સર્ક્યુલેટર)) {-}{-}{} F[પોર્ટ 2]}
    F {-{-}{} G[પોર્ટ 3] {-}{-}{} D}
{Highlighting}
{Shaded}
\end{verbatim}
\end{center}

\textbf{આઇસોલેટર:}

\begin{itemize}
\tightlist
\item
  \textbf{કાર્ય}: માત્ર એક દિશામાં સિગ્નલ ફ્લોની પરવાનગી આપે છે
\item
  \textbf{કન્સ્ટ્રક્શન}: મેગ્નેટિક બાયાસ સાથે ફેરાઇટ મટેરિયલનો ઉપયોગ
\item
  \textbf{ઉપયોગ}: રિફ્લેક્શનથી સોર્સનું રક્ષણ કરે છે
\end{itemize}

\textbf{સર્ક્યુલેટર:}

\begin{itemize}
\tightlist
\item
  \textbf{કાર્ય}: ત્રણ કે ચાર પોર્ટ વચ્ચે સર્ક્યુલર પેટર્નમાં સિગ્નલ રૂટ કરે છે
\item
  \textbf{કન્સ્ટ્રક્શન}: ફેરાઇટ મટેરિયલ સાથે Y-જંક્શન
\item
  \textbf{ઉપયોગ}: રડાર સિસ્ટમમાં ડુપ્લેક્સર તરીકે
\end{itemize}

\end{solutionbox}
\begin{mnemonicbox}
``આઇસોલેટેડ સર્કિટ ફોરવર્ડ ફ્લો''

\end{mnemonicbox}
\begin{center}\rule{0.5\linewidth}{0.5pt}\end{center}

\subsection*{પ્રશ્ન 1(ક અથવા) [7
ગુણ]}\label{uxaaauxab0uxab6uxaa8-1uxa95-uxa85uxaa5uxab5-7-uxa97uxaa3}

\textbf{વેવગાઇડમાં ડોમિનન્ટ મોડ શું છે? 10 સેમી breadth ધરાવતા રેક્ટેન્ગ્યુલર વેવગાઇડ
માટે કટ ઓફ વેવલેન્થ શોધો. 2.5 GHz સિગ્નલ propagate થવા માટે ગાઇડ વેવલેન્થ, ગ્રુપ
વેલોસિટી, ફેઝ વેલોસિટી અને Z_{0}ની વેલ્યુ શોધો.}

\begin{solutionbox}

\textbf{ડોમિનન્ટ મોડ}: વેવગાઇડમાં propagate થઈ શકતો સૌથી નીચો ઓર્ડર મોડ.
રેક્ટેન્ગ્યુલર વેવગાઇડ માટે TE_{1}_{0} મોડ છે.

\textbf{આપેલા ડેટા:}

\begin{itemize}
\tightlist
\item
  Breadth (a) = 10 cm = 0.1 m
\item
  Frequency (f) = 2.5 GHz = 2.5 \times 10^{9} Hz
\item
  c = 3 \times 10^{8} m/s
\end{itemize}

\textbf{ગણતરીઓ:}

{\def\LTcaptype{none} % do not increment counter
\begin{longtable}[]{@{}lll@{}}
\toprule\noalign{}
પેરામીટર & ફોર્મ્યુલા & વેલ્યુ \\
\midrule\noalign{}
\endhead
\bottomrule\noalign{}
\endlastfoot
\textbf{કટ ઓફ વેવલેન્થ} & λc = 2a & λc = 2 \times 0.1 = 0.2 m \\
\textbf{ફ્રી સ્પેસ વેવલેન્થ} & λ_{0} = c/f & λ_{0} = 0.12 m \\
\textbf{ગાઇડ વેવલેન્થ} & λg = λ_{0}/\sqrt(1-(λ_{0}/λc)^{2}) & λg = 0.133 m \\
\textbf{ગ્રુપ વેલોસિટી} & vg = c\sqrt(1-(λ_{0}/λc)^{2}) & vg = 2.7 \times 10^{8} m/s \\
\textbf{ફેઝ વેલોસિટી} & vp = c/\sqrt(1-(λ_{0}/λc)^{2}) & vp = 3.33 \times 10^{8} m/s \\
\end{longtable}
}

\end{solutionbox}
\begin{mnemonicbox}
``ડોમિનન્ટ મોડ કેલ્ક્યુલેટ ગાઇડ પેરામીટર''

\end{mnemonicbox}
\begin{center}\rule{0.5\linewidth}{0.5pt}\end{center}

\subsection*{પ્રશ્ન 2(અ) [3
ગુણ]}\label{uxaaauxab0uxab6uxaa8-2uxa85-3-uxa97uxaa3}

\textbf{સિંગલ સ્ટબ ઇમ્પીડન્સ મેચિંગ શું છે, અને આ કેવી રીતે કાર્ય કરે છે?}

\begin{solutionbox}

\textbf{સિંગલ સ્ટબ મેચિંગ}: ઇમ્પીડન્સ મેચિંગ માટે ટ્રાન્સમિશન લાઇન સાથે પેરેલલમાં
જોડાયેલા એક શોર્ટ-સર્કિટ અથવા ઓપન-સર્કિટ સ્ટબનો ઉપયોગ કરતી ટેકનિક.

\textbf{કાર્યસિદ્ધાંત:}

\begin{itemize}
\tightlist
\item
  \textbf{સ્ટબ રિએક્ટિવ એલિમેન્ટ તરીકે કાર્ય કરે છે} (ઇન્ડક્ટિવ અથવા કેપેસિટિવ)
\item
  \textbf{લોડ ઇમ્પીડન્સના રિએક્ટિવ ઘટકને કેન્સલ કરે છે}
\item
  \textbf{ઇમ્પીડન્સને કેરેક્ટરિસ્ટિક ઇમ્પીડન્સમાં ટ્રાન્સફોર્મ કરે છે}
\end{itemize}

\end{solutionbox}
\begin{mnemonicbox}
``સિંગલ સ્ટબ ટ્રાન્સફોર્મ રિએક્ટન્સ''

\end{mnemonicbox}
\begin{center}\rule{0.5\linewidth}{0.5pt}\end{center}

\subsection*{પ્રશ્ન 2(બ) [4
ગુણ]}\label{uxaaauxab0uxab6uxaa8-2uxaac-4-uxa97uxaa3}

\textbf{રેક્ટેન્ગ્યુલર અને સર્ક્યુલર વેવગાઇડ વચ્ચે કોઈ પણ ત્રણ તફાવત આપો.}

\begin{solutionbox}

{\def\LTcaptype{none} % do not increment counter
\begin{longtable}[]{@{}lll@{}}
\toprule\noalign{}
પેરામીટર & રેક્ટેન્ગ્યુલર વેવગાઇડ & સર્ક્યુલર વેવગાઇડ \\
\midrule\noalign{}
\endhead
\bottomrule\noalign{}
\endlastfoot
\textbf{ક્રોસ-સેક્શન} & લંબચોરસ આકાર & વર્તુળાકાર આકાર \\
\textbf{ડોમિનન્ટ મોડ} & TE_{1}_{0} મોડ & TE_{1}_{1} મોડ \\
\textbf{ફીલ્ડ પેટર્ન} & સરળ ફીલ્ડ વિતરણ & જટિલ ફીલ્ડ વિતરણ \\
\textbf{મેન્યુફેક્ચરિંગ} & બનાવવામાં સહેલું & બનાવવામાં મુશ્કેલ \\
\end{longtable}
}

\end{solutionbox}
\begin{mnemonicbox}
``લંબચોરસ દસ પર ડોમિનેટ કરે'' vs ``વર્તુળ અગિયાર પર
ડોમિનેટ કરે''

\end{mnemonicbox}
\begin{center}\rule{0.5\linewidth}{0.5pt}\end{center}

\subsection*{પ્રશ્ન 2(ક) [7
ગુણ]}\label{uxaaauxab0uxab6uxaa8-2uxa95-7-uxa97uxaa3}

\textbf{હાઇબ્રિડ રિંગનું બાંધકામ અને કાર્ય આકૃતિ સાથે સમજાવો.}

\begin{solutionbox}

\begin{center}
\textbf{Mermaid Diagram (Code)}
\begin{verbatim}
{Shaded}
{Highlighting}[]
graph TD
    A[પોર્ટ 1] {-{-}{-} B[હાઇબ્રિડ રિંગ]}
    C[પોર્ટ 2] {-{-}{-} B}
    D[પોર્ટ 3] {-{-}{-} B}
    E[પોર્ટ 4] {-{-}{-} B}
    B {-.{-}{} F[λ/4 સેક્શન]}
{Highlighting}
{Shaded}
\end{verbatim}
\end{center}

\textbf{બાંધકામ:}

\begin{itemize}
\tightlist
\item
  \textbf{રિંગ સ્ટ્રક્ચર} ચાર પોર્ટ સાથે
\item
  \textbf{પરિઘ} = 1.5λ (દોઢ વેવલેન્થ)
\item
  \textbf{બાજુના પોર્ટ} λ/4 દ્વારા અલગ
\item
  \textbf{વિરુદ્ધ પોર્ટ} 3λ/4 દ્વારા અલગ
\end{itemize}

\textbf{કાર્ય:}

\begin{itemize}
\tightlist
\item
  \textbf{પાવર ડિવિઝન}: એક પોર્ટનું ઇનપુટ બે બાજુના પોર્ટમાં સમાન રીતે વહેંચાય છે
\item
  \textbf{આઇસોલેશન}: વિરુદ્ધ પોર્ટને કોઈ પાવર મળતું નથી
\item
  \textbf{ફેઝ રિલેશનશિપ}: આઉટપુટ પોર્ટ વચ્ચે 180^\circ ફેઝ ડિફરન્સ
\end{itemize}

\textbf{ઉપયોગ:}

\begin{itemize}
\tightlist
\item
  \textbf{બેલેન્સ્ડ મિક્સર}
\item
  \textbf{પાવર કમ્બાઇનર/ડિવાઇડર}
\item
  \textbf{એન્ટીના ફીડ}
\end{itemize}

\end{solutionbox}
\begin{mnemonicbox}
``હાઇબ્રિડ રિંગ પાવર સમાન વહેંચે છે''

\end{mnemonicbox}
\begin{center}\rule{0.5\linewidth}{0.5pt}\end{center}

\subsection*{પ્રશ્ન 2(અ અથવા) [3
ગુણ]}\label{uxaaauxab0uxab6uxaa8-2uxa85-uxa85uxaa5uxab5-3-uxa97uxaa3}

\textbf{માઇક્રોવેવ શું છે? માઇક્રોવેવના કોઈ પણ ચાર ઉપયોગો લખો.}

\begin{solutionbox}

\textbf{માઇક્રોવેવ}: 1 GHz થી 300 GHz સુધીની ફ્રીક્વન્સી રેન્જ ધરાવતા
ઇલેક્ટ્રોમેગ્નેટિક વેવ્સ.

\textbf{ઉપયોગ:}

\begin{itemize}
\tightlist
\item
  \textbf{રડાર સિસ્ટમ} ડિટેક્શન અને રેન્જિંગ માટે
\item
  \textbf{સેટેલાઇટ કમ્યુનિકેશન} લાંબા અંતરના ટ્રાન્સમિશન માટે
\item
  \textbf{માઇક્રોવેવ ઓવન} ખોરાક ગરમ કરવા માટે
\item
  \textbf{મોબાઇલ કમ્યુનિકેશન} (સેલ્યુલર નેટવર્ક)
\end{itemize}

\end{solutionbox}
\begin{mnemonicbox}
``માઇક્રોવેવ રીચ સ્પેસ મોબાઇલ''

\end{mnemonicbox}
\begin{center}\rule{0.5\linewidth}{0.5pt}\end{center}

\subsection*{પ્રશ્ન 2(બ અથવા) [4
ગુણ]}\label{uxaaauxab0uxab6uxaa8-2uxaac-uxa85uxaa5uxab5-4-uxa97uxaa3}

\textbf{કેવિટી રેઝોનેટર પર ટૂંકી નોંધ લખો.}

\begin{solutionbox}

\textbf{કેવિટી રેઝોનેટર}: ચોક્કસ રેઝોનન્ટ ફ્રીક્વન્સીએ ઇલેક્ટ્રોમેગ્નેટિક એનર્જીને સીમિત
કરતું બંધ મેટાલિક સ્ટ્રક્ચર.

\textbf{બાંધકામ:}

\begin{itemize}
\tightlist
\item
  \textbf{ચોક્કસ માપના મેટાલિક એન્ક્લોઝર}
\item
  \textbf{ઉચ્ચ Q ફેક્ટર} (ઓછું નુકસાન)
\item
  \textbf{રેઝોનન્ટ ફ્રીક્વન્સી} કેવિટીના માપ પર આધાર રાખે છે
\end{itemize}

\textbf{પ્રકાર:}

\begin{itemize}
\tightlist
\item
  \textbf{રેક્ટેન્ગ્યુલર કેવિટી}
\item
  \textbf{સિલિન્ડ્રિકલ કેવિટી}
\item
  \textbf{સ્ફેરિકલ કેવિટી}
\end{itemize}

\textbf{ઉપયોગ:}

\begin{itemize}
\tightlist
\item
  \textbf{ફ્રીક્વન્સી મીટર}
\item
  \textbf{ઓસીલેટર સર્કિટ}
\item
  \textbf{ફિલ્ટર સર્કિટ}
\end{itemize}

\end{solutionbox}
\begin{mnemonicbox}
``કેવિટી રેઝોનેટ હાઇ ક્વોલિટી''

\end{mnemonicbox}
\begin{center}\rule{0.5\linewidth}{0.5pt}\end{center}

\subsection*{પ્રશ્ન 2(ક અથવા) [7
ગુણ]}\label{uxaaauxab0uxab6uxaa8-2uxa95-uxa85uxaa5uxab5-7-uxa97uxaa3}

\textbf{મેજિક ટીને આકૃતિની મદદથી સમજાવો. તે આઇસોલેટર તરીકે કઈ રીતે કાર્ય કરે છે?}

\begin{solutionbox}

\begin{center}
\textbf{Mermaid Diagram (Code)}
\begin{verbatim}
{Shaded}
{Highlighting}[]
graph TD
    A[E{-આર્મ] {-}{-}{-} B[મેજિક ટી જંક્શન]}
    C[H{-આર્મ] {-}{-}{-} B}
    D[આર્મ 1] {-{-}{-} B}
    E[આર્મ 2] {-{-}{-} B}
{Highlighting}
{Shaded}
\end{verbatim}
\end{center}

\textbf{મેજિક ટી બાંધકામ:}

\begin{itemize}
\tightlist
\item
  \textbf{E-પ્લેન ટી} અને \textbf{H-પ્લેન ટી} સંયુક્ત
\item
  \textbf{ચાર પોર્ટ}: E-આર્મ, H-આર્મ, અને બે સાઇડ આર્મ
\item
  \textbf{E-આર્મ} H-આર્મ પર વર્ટિકલ
\end{itemize}

\textbf{આઇસોલેટર તરીકે કાર્ય:}

\begin{itemize}
\tightlist
\item
  \textbf{E-આર્મનું સિગ્નલ} સાઇડ આર્મમાં સમાન રીતે વહેંચાય છે (in-phase)
\item
  \textbf{H-આર્મનું સિગ્નલ} સાઇડ આર્મમાં સમાન રીતે વહેંચાય છે (out-of-phase)
\item
  \textbf{E-આર્મ અને H-આર્મ વચ્ચે આઇસોલેશન}
\item
  \textbf{પર્પેન્ડિક્યુલર આર્મ વચ્ચે કોઈ કપલિંગ નથી}
\end{itemize}

\textbf{ગુણધર્મો:}

\begin{itemize}
\tightlist
\item
  \textbf{બધા પોર્ટ પર મેચ્ડ}
\item
  \textbf{રેસિપ્રોકલ ડિવાઇસ}
\item
  \textbf{પાવર ડિવિઝન અને આઇસોલેશન}
\end{itemize}

\end{solutionbox}
\begin{mnemonicbox}
``મેજિક આઇસોલેટ પર્પેન્ડિક્યુલર આર્મ''

\end{mnemonicbox}
\begin{center}\rule{0.5\linewidth}{0.5pt}\end{center}

\subsection*{પ્રશ્ન 3(અ) [3
ગુણ]}\label{uxaaauxab0uxab6uxaa8-3uxa85-3-uxa97uxaa3}

\textbf{મેઝરનો કાર્યસિદ્ધાંત વર્ણવો.}

\begin{solutionbox}

\textbf{મેઝર (Microwave Amplification by Stimulated Emission of
Radiation):}

\begin{itemize}
\tightlist
\item
  \textbf{એક્ટિવ મીડિયમમાં પોપ્યુલેશન ઇન્વર્શન} બનાવવામાં આવે છે
\item
  \textbf{સ્ટિમ્યુલેટેડ એમિશન} કોહેરન્ટ માઇક્રોવેવ પેદા કરે છે
\item
  \textbf{એનર્જી લેવલ ટ્રાન્ઝિશન દ્વારા એમ્પ્લિફિકેશન} થાય છે
\end{itemize}

\textbf{કાર્યસિદ્ધાંત:}

\begin{itemize}
\tightlist
\item
  \textbf{પરમાણુ ઉચ્ચ એનર્જી લેવલમાં ઉત્તેજિત થાય છે}
\item
  \textbf{સ્ટિમ્યુલેટેડ ફોટોન એમિશન ટ્રિગર કરે છે}
\item
  \textbf{માઇક્રોવેવ સિગ્નલનું કોહેરન્ટ એમ્પ્લિફિકેશન}
\end{itemize}

\end{solutionbox}
\begin{mnemonicbox}
``માઇક્રોવેવ એમ્પ્લિફાઇ સ્ટિમ્યુલેટેડ એમિશન રેડિએશન''

\end{mnemonicbox}
\begin{center}\rule{0.5\linewidth}{0.5pt}\end{center}

\subsection*{પ્રશ્ન 3(બ) [4
ગુણ]}\label{uxaaauxab0uxab6uxaa8-3uxaac-4-uxa97uxaa3}

\textbf{ચાર માઇક્રોવેવ ડાયોડના નામ લખો અને એકનું વર્ણન કરો.}

\begin{solutionbox}

\textbf{ચાર માઇક્રોવેવ ડાયોડ:}

\begin{enumerate}
\tightlist
\item
  \textbf{GUNN ડાયોડ}
\item
  \textbf{IMPATT ડાયોડ}
\item
  \textbf{TRAPATT ડાયોડ}
\item
  \textbf{PIN ડાયોડ}
\end{enumerate}

\textbf{GUNN ડાયોડ વિગતવાર:}

\begin{itemize}
\tightlist
\item
  \textbf{સિદ્ધાંત}: GaAs માં ટ્રાન્સફર્ડ ઇલેક્ટ્રોન એફેક્ટ
\item
  \textbf{બાંધકામ}: ઓહ્મિક કોન્ટેક્ટ સાથે N-ટાઇપ GaAs
\item
  \textbf{ઓપરેશન}: માઇક્રોવેવ ફ્રીક્વન્સીએ નેગેટિવ રેઝિસ્ટન્સ
\item
  \textbf{ઉપયોગ}: ઓસીલેટર, એમ્પ્લિફાયર
\end{itemize}

\textbf{VI લાક્ષણિકતા:}

\begin{verbatim}
    I \^{}
      |    /
      |   /
      |  /\_\_\_
      | /    {}
      |/      {\_\_\_}
      +{-{-}{-}{-}{-}{-}{-}{-}{-}{-} V}
      Negative resistance region
\end{verbatim}

\end{solutionbox}
\begin{mnemonicbox}
``GUNN જનરેટ નેગેટિવ રેઝિસ્ટન્સ''

\end{mnemonicbox}
\begin{center}\rule{0.5\linewidth}{0.5pt}\end{center}

\subsection*{પ્રશ્ન 3(ક) [7
ગુણ]}\label{uxaaauxab0uxab6uxaa8-3uxa95-7-uxa97uxaa3}

\textbf{મેગ્નેટ્રોન ઓસીલેટરનું નિર્માણ, કાર્યસિદ્ધાંત અને ઉપયોગો સાથે વિસ્તારવાર વર્ણન
કરો.}

\begin{solutionbox}

\begin{center}
\textbf{Mermaid Diagram (Code)}
\begin{verbatim}
{Shaded}
{Highlighting}[]
graph LR
    A[કેથોડ] {-{-}{} B[ઇન્ટરેક્શન સ્પેસ]}
    B {-{-}{} C[કેવિટી સાથે એનોડ]}
    C {-{-}{} D[આઉટપુટ કપલિંગ]}
    E[મેગ્નેટિક ફીલ્ડ] {-.{-}{} B}
{Highlighting}
{Shaded}
\end{verbatim}
\end{center}

\textbf{બાંધકામ:}

\begin{itemize}
\tightlist
\item
  \textbf{કેન્દ્રમાં સિલિન્ડ્રિકલ કેથોડ}
\item
  \textbf{કેથોડની આસપાસ રેઝોનન્ટ કેવિટી સાથે એનોડ}
\item
  \textbf{ઇલેક્ટ્રિક ફીલ્ડ પર વર્ટિકલ મજબૂત મેગ્નેટિક ફીલ્ડ}
\item
  \textbf{વેવગાઇડ દ્વારા આઉટપુટ કપલિંગ}
\end{itemize}

\textbf{કાર્યસિદ્ધાંત:}

\begin{itemize}
\tightlist
\item
  \textbf{ગરમ કેથોડમાંથી ઇલેક્ટ્રોન ઉત્સર્જન}
\item
  \textbf{ક્રોસ્ડ E અને B ફીલ્ડને કારણે સાયક્લોઇડ ગતિ}
\item
  \textbf{બંચિંગ એફેક્ટ ઇલેક્ટ્રોન ક્લાઉડ બનાવે છે}
\item
  \textbf{ઇલેક્ટ્રોનથી RF ફીલ્ડમાં એનર્જી ટ્રાન્સફર}
\item
  \textbf{કેવિટી રેઝોનન્ટ ફ્રીક્વન્સીએ ઓસીલેશન}
\end{itemize}

\textbf{ઉપયોગ:}

\begin{itemize}
\tightlist
\item
  \textbf{રડાર ટ્રાન્સમિટર}
\item
  \textbf{માઇક્રોવેવ ઓવન}
\item
  \textbf{ઇન્ડસ્ટ્રિયલ હીટિંગ}
\item
  \textbf{મેડિકલ ડાયાથર્મી}
\end{itemize}

\end{solutionbox}
\begin{mnemonicbox}
``મેગ્નેટ્રોન મેક માઇક્રોવેવ ઓસીલેશન''

\end{mnemonicbox}
\begin{center}\rule{0.5\linewidth}{0.5pt}\end{center}

\subsection*{પ્રશ્ન 3(અ અથવા) [3
ગુણ]}\label{uxaaauxab0uxab6uxaa8-3uxa85-uxa85uxaa5uxab5-3-uxa97uxaa3}

\textbf{રૂબી મેઝરની કામગીરીનું વર્ણન કરો.}

\begin{solutionbox}

\textbf{રૂબી મેઝર કાર્ય:}

\begin{itemize}
\tightlist
\item
  \textbf{રૂબી ક્રિસ્ટલ} (Al_{2}O_{3} જેમાં Cr^{3}^{+} આયન) એક્ટિવ મીડિયમ તરીકે
\item
  \textbf{ક્રોમિયમ આયનમાં ત્રણ એનર્જી લેવલ}
\item
  \textbf{પમ્પ ફ્રીક્વન્સી પોપ્યુલેશન ઇન્વર્શન બનાવે છે}
\item
  \textbf{2.9 GHz પર સિગ્નલ એમ્પ્લિફિકેશન}
\end{itemize}

\textbf{પ્રક્રિયા:}

\begin{itemize}
\tightlist
\item
  \textbf{ઓપ્ટિકલ પમ્પિંગ ઇલેક્ટ્રોનને ઉચ્ચ લેવલમાં ઉત્તેજિત કરે છે}
\item
  \textbf{સ્ટિમ્યુલેટેડ એમિશન કોહેરન્ટ માઇક્રોવેવ પેદા કરે છે}
\item
  \textbf{લો નોઇઝ એમ્પ્લિફિકેશન પ્રાપ્ત થાય છે}
\end{itemize}

\end{solutionbox}
\begin{mnemonicbox}
``રૂબી રેડિએટ એમ્પ્લિફાઇડ માઇક્રોવેવ''

\end{mnemonicbox}
\begin{center}\rule{0.5\linewidth}{0.5pt}\end{center}

\subsection*{પ્રશ્ન 3(બ અથવા) [4
ગુણ]}\label{uxaaauxab0uxab6uxaa8-3uxaac-uxa85uxaa5uxab5-4-uxa97uxaa3}

\textbf{ગન ડાયોડની VI કેરેક્ટરિસ્ટિક દોરો અને સમજાવો}

\begin{solutionbox}

\begin{verbatim}
    I \^{}
      |      
      |    B /
      |     /
      |    /
      | A /
      |  /
      | /\_\_\_\_\_ C
      |/      {}
      |        {\_\_\_\_\_ D}
      +{-{-}{-}{-}{-}{-}{-}{-}{-}{-}{-} V}
    Valley    Peak
    Current   Current
\end{verbatim}

\textbf{VI કેરેક્ટરિસ્ટિક સમજૂતી:}

\begin{itemize}
\tightlist
\item
  \textbf{રીજન OA}: ઓહ્મિક રીજન (પોઝિટિવ રેઝિસ્ટન્સ)
\item
  \textbf{રીજન AB}: નેગેટિવ રેઝિસ્ટન્સ રીજન
\item
  \textbf{રીજન BC}: વેલી કરન્ટ રીજન
\item
  \textbf{રીજન CD}: સેચ્યુરેશન રીજન
\end{itemize}

\textbf{મુખ્ય મુદ્દાઓ:}

\begin{itemize}
\tightlist
\item
  \textbf{પીક વોલ્ટેજ}: નેગેટિવ રેઝિસ્ટન્સ પહેલાં મેક્સિમમ વોલ્ટેજ
\item
  \textbf{વેલી કરન્ટ}: નેગેટિવ રેઝિસ્ટન્સ રીજનમાં મિનિમમ કરન્ટ
\item
  \textbf{નેગેટિવ રેઝિસ્ટન્સ}: વોલ્ટેજ વધવા સાથે કરન્ટ ઘટે છે
\end{itemize}

\end{solutionbox}
\begin{mnemonicbox}
``વેલી પીક નેગેટિવ રેઝિસ્ટન્સ''

\end{mnemonicbox}
\begin{center}\rule{0.5\linewidth}{0.5pt}\end{center}

\subsection*{પ્રશ્ન 3(ક અથવા) [7
ગુણ]}\label{uxaaauxab0uxab6uxaa8-3uxa95-uxa85uxaa5uxab5-7-uxa97uxaa3}

\textbf{માઇક્રોવેવ ફ્રીક્વન્સી પર ``frequency measurement method'' અને
``attenuation measurement method'' વિશે વર્ણન કરો.}

\begin{solutionbox}

\textbf{ફ્રીક્વન્સી મેઝરમેન્ટ મેથડ:}

{\def\LTcaptype{none} % do not increment counter
\begin{longtable}[]{@{}lll@{}}
\toprule\noalign{}
મેથડ & સિદ્ધાંત & ચોકસાઈ \\
\midrule\noalign{}
\endhead
\bottomrule\noalign{}
\endlastfoot
\textbf{કેવિટી વેવમીટર} & રેઝોનન્ટ કેવિટી ટ્યુનિંગ & ઉચ્ચ \\
\textbf{ડાયરેક્ટ રીડિંગ મીટર} & ફ્રીક્વન્સી કાઉન્ટર & ખૂબ ઉચ્ચ \\
\textbf{હેટેરોડાયન મેથડ} & બીટ ફ્રીક્વન્સી ટેકનિક & મધ્યમ \\
\end{longtable}
}

\textbf{એટેન્યુએશન મેઝરમેન્ટ મેથડ:}

{\def\LTcaptype{none} % do not increment counter
\begin{longtable}[]{@{}
  >{\raggedright\arraybackslash}p{(\linewidth - 4\tabcolsep) * \real{0.2353}}
  >{\raggedright\arraybackslash}p{(\linewidth - 4\tabcolsep) * \real{0.3824}}
  >{\raggedright\arraybackslash}p{(\linewidth - 4\tabcolsep) * \real{0.3824}}@{}}
\toprule\noalign{}
\begin{minipage}[b]{\linewidth}\raggedright
મેથડ
\end{minipage} & \begin{minipage}[b]{\linewidth}\raggedright
વર્ણન
\end{minipage} & \begin{minipage}[b]{\linewidth}\raggedright
ઉપયોગ
\end{minipage} \\
\midrule\noalign{}
\endhead
\bottomrule\noalign{}
\endlastfoot
\textbf{સબસ્ટિટ્યુશન મેથડ} & એટેન્યુએટરને કેલિબ્રેટેડ એટેન્યુએટર સાથે બદલો & પ્રિસિઝન
મેઝરમેન્ટ \\
\textbf{પાવર રેશિયો મેથડ} & ઇનપુટ અને આઉટપુટ પાવરની તુલના & સામાન્ય હેતુ \\
\textbf{RF બ્રિજ મેથડ} & બ્રિજ સર્કિટ બેલેન્સ & લેબોરેટરી ઉપયોગ \\
\end{longtable}
}

\textbf{મેઝરમેન્ટ સેટઅપ:}

\begin{itemize}
\tightlist
\item
  \textbf{સિગ્નલ જનરેટર} ટેસ્ટ સિગ્નલ પૂરું પાડે છે
\item
  \textbf{કેલિબ્રેટેડ એટેન્યુએટર} રેફરન્સ માટે
\item
  \textbf{પાવર મીટર} સિગ્નલ લેવલ માપે છે
\item
  \textbf{VSWR મીટર} ઇમ્પીડન્સ મેચિંગ મોનિટર કરે છે
\end{itemize}

\end{solutionbox}
\begin{mnemonicbox}
``ફ્રીક્વન્સી એટેન્યુએશન પ્રિસાઇઝલી મેઝર્ડ''

\end{mnemonicbox}
\begin{center}\rule{0.5\linewidth}{0.5pt}\end{center}

\subsection*{પ્રશ્ન 4(અ) [3
ગુણ]}\label{uxaaauxab0uxab6uxaa8-4uxa85-3-uxa97uxaa3}

\textbf{P-i-N ડાયોડની કામગીરી વર્ણન કરો.}

\begin{solutionbox}

\textbf{P-i-N ડાયોડ સ્ટ્રક્ચર:}

\begin{itemize}
\tightlist
\item
  \textbf{P-ટાઇપ રીજન} (હેવિલી ડોપ્ડ)
\item
  \textbf{ઇન્ટ્રિન્સિક રીજન} (અનડોપ્ડ, હાઇ રેઝિસ્ટન્સ)
\item
  \textbf{N-ટાઇપ રીજન} (હેવિલી ડોપ્ડ)
\end{itemize}

\textbf{કાર્ય:}

\begin{itemize}
\tightlist
\item
  \textbf{ફોરવર્ડ બાયાસ}: લો રેઝિસ્ટન્સ, કંડક્ટર તરીકે કાર્ય કરે છે
\item
  \textbf{રિવર્સ બાયાસ}: હાઇ રેઝિસ્ટન્સ, ઇન્સુલેટર તરીકે કાર્ય કરે છે
\item
  \textbf{RF સ્વિચિંગ}: ચાર્જ સ્ટોરેજને કારણે ફાસ્ટ સ્વિચિંગ
\end{itemize}

\textbf{ઉપયોગ:}

\begin{itemize}
\tightlist
\item
  \textbf{RF સ્વિચ}
\item
  \textbf{એટેન્યુએટર}
\item
  \textbf{ફેઝ શિફ્ટર}
\end{itemize}

\end{solutionbox}
\begin{mnemonicbox}
``PIN પ્રોવાઇડ ઇન્સ્ટન્ટ સ્વિચિંગ''

\end{mnemonicbox}
\begin{center}\rule{0.5\linewidth}{0.5pt}\end{center}

\subsection*{પ્રશ્ન 4(બ) [4
ગુણ]}\label{uxaaauxab0uxab6uxaa8-4uxaac-4-uxa97uxaa3}

\textbf{મેગ્નેટ્રોન માટે π મોડ ઓસીલેશનનું વર્ણન કરો.}

\begin{solutionbox}

\textbf{π મોડ ઓસીલેશન:}

\begin{itemize}
\tightlist
\item
  \textbf{બાજુની કેવિટી} 180^\circ આઉટ ઓફ ફેઝમાં ઓસીલેટ કરે છે
\item
  \textbf{ઇલેક્ટ્રોન બંચિંગ} RF ફીલ્ડ સાથે સિંક્રોનાઇઝ
\item
  \textbf{ઇલેક્ટ્રોનથી RF માં મેક્સિમમ પાવર ટ્રાન્સફર}
\item
  \textbf{ડિઝાઇન કરેલી ફ્રીક્વન્સીએ સ્ટેબલ ઓસીલેશન}
\end{itemize}

\textbf{લાક્ષણિકતાઓ:}

\begin{itemize}
\tightlist
\item
  \textbf{ફેઝ ડિફરન્સ}: બાજુની કેવિટી વચ્ચે π રેડિયન
\item
  \textbf{ફ્રીક્વન્સી}: કેવિટીના માપ દ્વારા નક્કી
\item
  \textbf{કાર્યક્ષમતા}: બધા મોડમાં સૌથી વધુ
\item
  \textbf{સ્થિરતા}: સૌથી સ્થિર ઓસીલેશન મોડ
\end{itemize}

\textbf{મોડ ચાર્ટ:}

\begin{verbatim}
Cavity: 1  2  3  4  5  6  7  8
Phase:  0  π  0  π  0  π  0  π
\end{verbatim}

\end{solutionbox}
\begin{mnemonicbox}
``પાઇ મોડ મેક્સિમમ પાવર પ્રોડ્યુસ કરે''

\end{mnemonicbox}
\begin{center}\rule{0.5\linewidth}{0.5pt}\end{center}

\subsection*{પ્રશ્ન 4(ક) [7
ગુણ]}\label{uxaaauxab0uxab6uxaa8-4uxa95-7-uxa97uxaa3}

\textbf{જરૂરી ડાયાગ્રામ સાથે ટ્વો કેવિટી ક્લિસ્ટ્રોન એમ્પ્લિફાયરનું કન્સ્ટ્રક્શન અને
કામગીરી સમજાવો.}

\begin{solutionbox}

\begin{center}
\textbf{Mermaid Diagram (Code)}
\begin{verbatim}
{Shaded}
{Highlighting}[]
graph LR
    A[ઇલેક્ટ્રોન ગન] {-{-}{} B[ઇનપુટ કેવિટી]}
    B {-{-}{} C[ડ્રિફ્ટ સ્પેસ]}
    C {-{-}{} D[આઉટપુટ કેવિટી]}
    D {-{-}{} E[કલેક્ટર]}
    F[ઇનપુટ સિગ્નલ] {-{-}{} B}
    D {-{-}{} G[આઉટપુટ સિગ્નલ]}
{Highlighting}
{Shaded}
\end{verbatim}
\end{center}

\textbf{બાંધકામ:}

\begin{itemize}
\tightlist
\item
  \textbf{ઇલેક્ટ્રોન ગન} ઇલેક્ટ્રોન બીમ પેદા કરે છે
\item
  \textbf{ઇનપુટ કેવિટી} (બંચર) ઇલેક્ટ્રોન બીમ મોડ્યુલેટ કરે છે
\item
  \textbf{ડ્રિફ્ટ સ્પેસ} વેલોસિટી મોડ્યુલેશનની પરવાનગી આપે છે
\item
  \textbf{આઉટપુટ કેવિટી} (કેચર) RF એનર્જી બહાર કાઢે છે
\item
  \textbf{કલેક્ટર} વપરાયેલા ઇલેક્ટ્રોન એકત્ર કરે છે
\end{itemize}

\textbf{કાર્યસિદ્ધાંત:}

\begin{itemize}
\tightlist
\item
  \textbf{ઇનપુટ કેવિટીમાં વેલોસિટી મોડ્યુલેશન}
\item
  \textbf{ડ્રિફ્ટ સ્પેસમાં ઇલેક્ટ્રોન બંચિંગ}
\item
  \textbf{ડેન્સિટી મોડ્યુલેશન કરન્ટ વેરિએશન બનાવે છે}
\item
  \textbf{આઉટપુટ કેવિટીમાં એનર્જી એક્સટ્રેક્શન}
\item
  \textbf{બીમ-ફીલ્ડ ઇન્ટરેક્શન દ્વારા એમ્પ્લિફિકેશન}
\end{itemize}

\textbf{મુખ્ય પેરામીટર:}

\begin{itemize}
\tightlist
\item
  \textbf{બીમ વોલ્ટેજ}: ઇલેક્ટ્રોન વેલોસિટી નક્કી કરે છે
\item
  \textbf{કેવિટી ટ્યુનિંગ}: ઓપરેટિંગ ફ્રીક્વન્સી સેટ કરે છે
\item
  \textbf{ડ્રિફ્ટ સ્પેસ લેન્થ}: બંચિંગ અસરકારકતા કંટ્રોલ કરે છે
\end{itemize}

\textbf{ઉપયોગ:}

\begin{itemize}
\tightlist
\item
  \textbf{રડાર ટ્રાન્સમિટર}
\item
  \textbf{સેટેલાઇટ કમ્યુનિકેશન}
\item
  \textbf{લિનિયર એક્સેલેરેટર}
\end{itemize}

\end{solutionbox}
\begin{mnemonicbox}
``ક્લિસ્ટ્રોન બંચિંગ દ્વારા એમ્પ્લિફાઇ કરે છે''

\end{mnemonicbox}
\begin{center}\rule{0.5\linewidth}{0.5pt}\end{center}

\subsection*{પ્રશ્ન 4(અ અથવા) [3
ગુણ]}\label{uxaaauxab0uxab6uxaa8-4uxa85-uxa85uxaa5uxab5-3-uxa97uxaa3}

\textbf{પેરામેટ્રિક એમ્પ્લિફાયરનું વર્ણન કરો.}

\begin{solutionbox}

\textbf{પેરામેટ્રિક એમ્પ્લિફાયર:}

\begin{itemize}
\tightlist
\item
  \textbf{વેરેક્ટર ડાયોડ ઉપયોગ કરતું વેરિએબલ રિએક્ટન્સ} ડિવાઇસ
\item
  \textbf{પમ્પ ફ્રીક્વન્સી} ડાયોડ કેપેસિટન્સ મોડ્યુલેટ કરે છે
\item
  \textbf{પમ્પથી સિગ્નલમાં એનર્જી ટ્રાન્સફર}
\item
  \textbf{લો નોઇઝ એમ્પ્લિફિકેશન} પ્રાપ્ત થાય છે
\end{itemize}

\textbf{કાર્ય:}

\begin{itemize}
\tightlist
\item
  \textbf{પમ્પ પાવર} ડાયોડ રિએક્ટન્સ વેરી કરે છે
\item
  \textbf{સિગ્નલ મિક્સિંગ} સમ અને ડિફરન્સ ફ્રીક્વન્સી પેદા કરે છે
\item
  \textbf{આઇડલર ફ્રીક્વન્સી} fp = fs + fi
\item
  \textbf{નોનલિનિયર મિક્સિંગ દ્વારા પાવર ગેઇન}
\end{itemize}

\textbf{ફાયદાઓ:}

\begin{itemize}
\tightlist
\item
  \textbf{ખૂબ લો નોઇઝ ફિગર}
\item
  \textbf{હાઇ ગેઇન શક્ય}
\item
  \textbf{વાઇડ બેન્ડવિડ્થ}
\end{itemize}

\end{solutionbox}
\begin{mnemonicbox}
``પેરામેટ્રિક એમ્પ્લિફાયર પમ્પ લો નોઇઝ''

\end{mnemonicbox}
\begin{center}\rule{0.5\linewidth}{0.5pt}\end{center}

\subsection*{પ્રશ્ન 4(બ અથવા) [4
ગુણ]}\label{uxaaauxab0uxab6uxaa8-4uxaac-uxa85uxaa5uxab5-4-uxa97uxaa3}

\textbf{ટ્રાવેલિંગ વેવ ટ્યુબની આકૃતિ દોરો અને સમજાવો}

\begin{solutionbox}

\begin{center}
\textbf{Mermaid Diagram (Code)}
\begin{verbatim}
{Shaded}
{Highlighting}[]
graph LR
    A[ઇલેક્ટ્રોન ગન] {-{-}{} B[ઇનપુટ]}
    B {-{-}{} C[હેલિક્સ]}
    C {-{-}{} D[આઉટપુટ]}
    D {-{-}{} E[કલેક્ટર]}
    F[એટેન્યુએટર] {-.{-}{} C}
    G[ફોકસિંગ સિસ્ટમ] {-.{-}{} C}
{Highlighting}
{Shaded}
\end{verbatim}
\end{center}

\textbf{ઘટકો:}

\begin{itemize}
\tightlist
\item
  \textbf{ઇલેક્ટ્રોન ગન}: ઇલેક્ટ્રોન બીમ પેદા કરે છે
\item
  \textbf{હેલિક્સ}: સ્લો-વેવ સ્ટ્રક્ચર
\item
  \textbf{એટેન્યુએટર}: ઓસીલેશન અટકાવે છે
\item
  \textbf{કલેક્ટર}: ઇલેક્ટ્રોન એકત્ર કરે છે
\item
  \textbf{ફોકસિંગ સિસ્ટમ}: બીમ એલાઇનમેન્ટ જાળવે છે
\end{itemize}

\textbf{કાર્ય:}

\begin{itemize}
\tightlist
\item
  \textbf{ઇલેક્ટ્રોન બીમ} હેલિક્સ કેન્દ્રમાંથી જાય છે
\item
  \textbf{RF સિગ્નલ} હેલિક્સ સાથે પ્રોપેગેટ થાય છે
\item
  \textbf{બીમ અને RF વેવ વચ્ચે સિંક્રોનિઝમ}
\item
  \textbf{બીમથી RF માં એનર્જી ટ્રાન્સફર}
\item
  \textbf{હેલિક્સ લેન્થ સાથે કન્ટિન્યુઅસ એમ્પ્લિફિકેશન}
\end{itemize}

\end{solutionbox}
\begin{mnemonicbox}
``TWT વેવ્સ સાથે ટ્રાવેલ કરે છે''

\end{mnemonicbox}
\begin{center}\rule{0.5\linewidth}{0.5pt}\end{center}

\subsection*{પ્રશ્ન 4(ક અથવા) [7
ગુણ]}\label{uxaaauxab0uxab6uxaa8-4uxa95-uxa85uxaa5uxab5-7-uxa97uxaa3}

\textbf{રિફ્લેક્સ ક્લિસ્ટ્રોનનો કાર્યસિદ્ધાંત ઉચિત આકૃતિ સાથે ઊંડાણમાં સમજાવો}

\begin{solutionbox}

\begin{center}
\textbf{Mermaid Diagram (Code)}
\begin{verbatim}
{Shaded}
{Highlighting}[]
graph LR
    A[કેથોડ] {-{-}{} B[રેઝોનન્ટ કેવિટી]}
    B {-{-}{} C[ડ્રિફ્ટ સ્પેસ]}
    C {-{-}{} D[રિપેલર]}
    D {-.{-}{} C}
    C {-.{-}{} B}
    B {-{-}{} E[આઉટપુટ]}
{Highlighting}
{Shaded}
\end{verbatim}
\end{center}

\textbf{બાંધકામ:}

\begin{itemize}
\tightlist
\item
  \textbf{સિંગલ રેઝોનન્ટ કેવિટી} બંચર અને કેચર તરીકે કાર્ય કરે છે
\item
  \textbf{રિપેલર ઇલેક્ટ્રોડ} ઇલેક્ટ્રોન બીમ રિફ્લેક્ટ કરે છે
\item
  \textbf{ડ્રિફ્ટ સ્પેસ} વેલોસિટી મોડ્યુલેશનની પરવાનગી આપે છે
\item
  \textbf{આઉટપુટ કપલિંગ} RF પાવર બહાર કાઢે છે
\end{itemize}

\textbf{કાર્યસિદ્ધાંત:}

\textbf{એપલગેટ ડાયાગ્રામ:}

\begin{verbatim}
Distance \^{}
         |    \_\_\_
         |   /   {  Bunched electrons}
         |  /     {}
         | /       {}
         |/         {\_\_\_}
         +{-{-}{-}{-}{-}{-}{-}{-}{-}{-}{-}{-}{-} Time}
         Transit time variation
\end{verbatim}

\textbf{પ્રક્રિયા:}

\begin{enumerate}
\tightlist
\item
  \textbf{ઇલેક્ટ્રોન કેવિટીમાં દાખલ થાય છે} અને વેલોસિટી મોડ્યુલેટેડ થાય છે
\item
  \textbf{ઇલેક્ટ્રોન રિપેલર તરફ ડ્રિફ્ટ કરે છે}
\item
  \textbf{રિપેલર ઇલેક્ટ્રોનને કેવિટીમાં પાછા રિફ્લેક્ટ કરે છે}
\item
  \textbf{ટ્રાન્ઝિટ ટાઇમ} બંચિંગ ફેઝ નક્કી કરે છે
\item
  \textbf{બંચ્ડ ઇલેક્ટ્રોન} કેવિટીને એનર્જી પહોંચાડે છે
\item
  \textbf{ફીડબેક દ્વારા ઓસીલેશન કાયમ રાખવામાં આવે છે}
\end{enumerate}

\textbf{ફ્રીક્વન્સી ટ્યુનિંગ:}

\begin{itemize}
\tightlist
\item
  \textbf{રિપેલર વોલ્ટેજ} ટ્રાન્ઝિટ ટાઇમ કંટ્રોલ કરે છે
\item
  \textbf{કેવિટી ટ્યુનિંગ} કેન્દ્ર ફ્રીક્વન્સી સેટ કરે છે
\item
  \textbf{ઇલેક્ટ્રોનિક ટ્યુનિંગ} શક્ય
\end{itemize}

\textbf{ઉપયોગ:}

\begin{itemize}
\tightlist
\item
  \textbf{લોકલ ઓસીલેટર}
\item
  \textbf{ફ્રીક્વન્સી મીટર}
\item
  \textbf{માઇક્રોવેવ સોર્સ}
\end{itemize}

\end{solutionbox}
\begin{mnemonicbox}
``રિફ્લેક્સ ઇલેક્ટ્રોન બંચ પાછા આપે છે''

\end{mnemonicbox}
\begin{center}\rule{0.5\linewidth}{0.5pt}\end{center}

\subsection*{પ્રશ્ન 5(અ) [3
ગુણ]}\label{uxaaauxab0uxab6uxaa8-5uxa85-3-uxa97uxaa3}

\textbf{``PIN ડાયોડ સ્વિચ તરીકે કાર્ય કરે અને VARACTOR ડાયોડ વેરિયેબલ કૅપેસિટર
તરીકે કાર્ય કરે.'' વિસ્તારમાં વર્ણન કરો.}

\begin{solutionbox}

\textbf{સ્વિચ તરીકે PIN ડાયોડ:}

\begin{itemize}
\tightlist
\item
  \textbf{ફોરવર્ડ બાયાસ}: લો રેઝિસ્ટન્સ (\textasciitilde1Ω), સ્વિચ ON
\item
  \textbf{રિવર્સ બાયાસ}: હાઇ રેઝિસ્ટન્સ (\textasciitilde10kΩ), સ્વિચ OFF
\item
  \textbf{I-રીજનમાં ચાર્જ સ્ટોરેજને કારણે ફાસ્ટ સ્વિચિંગ}
\item
  \textbf{OFF સ્ટેટમાં RF આઇસોલેશન}
\end{itemize}

\textbf{વેરિયેબલ કૅપેસિટર તરીકે VARACTOR ડાયોડ:}

\begin{itemize}
\tightlist
\item
  \textbf{રિવર્સ બાયાસ વોલ્ટેજ} જંક્શન કૅપેસિટન્સ કંટ્રોલ કરે છે
\item
  \textbf{રિવર્સ વોલ્ટેજ વધવા સાથે કૅપેસિટન્સ ઘટે છે}
\item
  \textbf{ટ્યુનિંગ સર્કિટ માટે વોલ્ટેજ-કંટ્રોલ્ડ રિએક્ટન્સ}
\item
  \textbf{મિકેનિકલ એડજસ્ટમેન્ટ વિના ઇલેક્ટ્રોનિક ટ્યુનિંગ}
\end{itemize}

\end{solutionbox}
\begin{mnemonicbox}
``PIN સ્વિચ કરે, VARACTOR વેરી કરે''

\end{mnemonicbox}
\begin{center}\rule{0.5\linewidth}{0.5pt}\end{center}

\subsection*{પ્રશ્ન 5(બ) [4
ગુણ]}\label{uxaaauxab0uxab6uxaa8-5uxaac-4-uxa97uxaa3}

\textbf{રડારમાં વપરાતી ડિસ્પ્લે પદ્ધતિઓની યાદી બનાવો અને એકનું વિસ્તારવાર વર્ણન
કરો.}

\begin{solutionbox}

\textbf{રડાર ડિસ્પ્લે પદ્ધતિઓ:}

\begin{enumerate}
\tightlist
\item
  \textbf{A-સ્કોપ ડિસ્પ્લે}
\item
  \textbf{PPI (Plan Position Indicator)}
\item
  \textbf{B-સ્કોપ ડિસ્પ્લે}
\item
  \textbf{RHI (Range Height Indicator)}
\end{enumerate}

\textbf{PPI ડિસ્પ્લે સમજૂતી:}

\begin{itemize}
\tightlist
\item
  \textbf{સર્ક્યુલર ડિસ્પ્લે} ટાર્ગેટ પોઝિશન દર્શાવે છે
\item
  \textbf{કેન્દ્ર રડાર લોકેશન દર્શાવે છે}
\item
  \textbf{રેડિયલ ડિસ્ટન્સ} ટાર્ગેટ રેન્જ સૂચવે છે
\item
  \textbf{એંગ્યુલર પોઝિશન} ટાર્ગેટ બેરિંગ દર્શાવે છે
\item
  \textbf{એન્ટીના રોટેશન સાથે સિંક્રોનાઇઝ્ડ રોટેટિંગ સ્વીપ}
\end{itemize}

\textbf{લાક્ષણિકતાઓ:}

\begin{itemize}
\tightlist
\item
  \textbf{ટાર્ગેટ પોઝિશનનું રિયલ-ટાઇમ ડિસ્પ્લે}
\item
  \textbf{રેન્જ અને બેરિંગ} માહિતી
\item
  \textbf{મલ્ટિપલ ટાર્ગેટ ટ્રેકિંગ}
\item
  \textbf{ક્લટર સપ્રેશન}
\end{itemize}

\end{solutionbox}
\begin{mnemonicbox}
``PPI પિક્ચર પોઝિશન ઇન્ડિકેટર''

\end{mnemonicbox}
\begin{center}\rule{0.5\linewidth}{0.5pt}\end{center}

\subsection*{પ્રશ્ન 5(ક) [7
ગુણ]}\label{uxaaauxab0uxab6uxaa8-5uxa95-7-uxa97uxaa3}

\textbf{રડાર શું છે? વિવિધ પ્રકારના રડાર સિસ્ટમ્સની યાદી બનાવો? એક રડારનું
વિસ્તારવાર વર્ણન કરો.}

\begin{solutionbox}

\textbf{રડાર (Radio Detection And Ranging):} ઑબ્જેક્ટ ડિટેક્ટ કરવા અને તેમની
રેન્જ, વેલોસિટી અને લાક્ષણિકતાઓ નક્કી કરવા માટે રેડિયો વેવ્સનો ઉપયોગ કરતી સિસ્ટમ.

\textbf{રડાર સિસ્ટમ્સના પ્રકાર:}

{\def\LTcaptype{none} % do not increment counter
\begin{longtable}[]{@{}lll@{}}
\toprule\noalign{}
પ્રકાર & ઉપયોગ & ફ્રીક્વન્સી બેન્ડ \\
\midrule\noalign{}
\endhead
\bottomrule\noalign{}
\endlastfoot
\textbf{પલ્સ રડાર} & એર ટ્રાફિક કંટ્રોલ & L, S, C બેન્ડ \\
\textbf{CW ડોપ્લર રડાર} & સ્પીડ મેઝરમેન્ટ & X, K, Ka બેન્ડ \\
\textbf{MTI રડાર} & મૂવિંગ ટાર્ગેટ ડિટેક્શન & S, C બેન્ડ \\
\textbf{SAR રડાર} & ગ્રાઉન્ડ મેપિંગ & L, C, X બેન્ડ \\
\end{longtable}
}

\textbf{પલ્સ રડાર વિગતવાર સમજૂતી:}

\begin{center}
\textbf{Mermaid Diagram (Code)}
\begin{verbatim}
{Shaded}
{Highlighting}[]
graph LR
    A[ટ્રાન્સમિટર] {-{-}{} B[ડુપ્લેક્સર]}
    B {-{-}{} C[એન્ટીના]}
    C {-{-}{} D[ટાર્ગેટ]}
    D {-{-}{} C}
    C {-{-}{} B}
    B {-{-}{} E[રિસીવર]}
    E {-{-}{} F[ડિસ્પ્લે]}
    G[ટાઇમર] {-{-}{} A}
    G {-{-}{} F}
{Highlighting}
{Shaded}
\end{verbatim}
\end{center}

\textbf{કાર્ય:}

\begin{itemize}
\tightlist
\item
  \textbf{RF એનર્જીના ટૂંકા પલ્સ ટ્રાન્સમિટ કરે છે}
\item
  \textbf{ટાર્ગેટથી ઇકો રિસીવ કરે છે}
\item
  \textbf{રેન્જ કેલ્ક્યુલેશન માટે ટાઇમ ડિલે માપે છે}
\item
  \textbf{ડિસ્પ્લે માટે સિગ્નલ પ્રોસેસ કરે છે}
\end{itemize}

\textbf{રેન્જ સમીકરણ:} R = (c \times t)/2

જ્યાં:

\begin{itemize}
\tightlist
\item
  R = ટાર્ગેટ સુધીની રેન્જ
\item
  c = પ્રકાશની ઝડપ
\item
  t = ટાઇમ ડિલે
\end{itemize}

\textbf{ઉપયોગ:}

\begin{itemize}
\tightlist
\item
  \textbf{એર ટ્રાફિક કંટ્રોલ}
\item
  \textbf{વેધર મોનિટરિંગ}
\item
  \textbf{મિલિટરી સર્વેલન્સ}
\item
  \textbf{નેવિગેશન એઇડ્સ}
\end{itemize}

\end{solutionbox}
\begin{mnemonicbox}
``રડાર રેન્જ રેડિયો વેવ્સ''

\end{mnemonicbox}
\begin{center}\rule{0.5\linewidth}{0.5pt}\end{center}

\subsection*{પ્રશ્ન 5(અ અથવા) [3
ગુણ]}\label{uxaaauxab0uxab6uxaa8-5uxa85-uxa85uxaa5uxab5-3-uxa97uxaa3}

\textbf{TRAPATT ડાયોડનું કાર્ય ડાયાગ્રામ સાથે વર્ણવો.}

\begin{solutionbox}

\begin{verbatim}
    I \^{}
      |      |{}
      |      | {}
      |      |  {}
      |      |   {\_\_ Trapped plasma}
      |      |     { avalanche}
      |\_\_\_\_\_\_|\_\_\_\_\_\_{\_\_\_}
      +{-{-}{-}{-}{-}{-}{-}{-}{-}{-}{-}{-}{-}{-}{-}{-} V}
      Breakdown voltage
\end{verbatim}

\textbf{TRAPATT ઓપરેશન:}

\begin{itemize}
\tightlist
\item
  \textbf{TRApped Plasma Avalanche Triggered Transit} ડાયોડ
\item
  \textbf{હાઇ ફીલ્ડ રીજન} એવેલાન્ચ બ્રેકડાઉન બનાવે છે
\item
  \textbf{પ્લાઝમા ફોર્મેશન} ચાર્જ કેરિયર ટ્રેપ કરે છે
\item
  \textbf{ટ્રાન્ઝિટ ટાઇમ એફેક્ટ્સ} નેગેટિવ રેઝિસ્ટન્સ બનાવે છે
\item
  \textbf{ટ્રાન્ઝિટ ટાઇમ દ્વારા ઓસીલેશન ફ્રીક્વન્સી નક્કી થાય છે}
\end{itemize}

\textbf{ઉપયોગ:}

\begin{itemize}
\tightlist
\item
  \textbf{હાઇ પાવર ઓસીલેટર}
\item
  \textbf{રડાર ટ્રાન્સમિટર}
\item
  \textbf{કમ્યુનિકેશન સિસ્ટમ}
\end{itemize}

\end{solutionbox}
\begin{mnemonicbox}
``TRAPATT ટ્રેપ પ્લાઝમા એવેલાન્ચ''

\end{mnemonicbox}
\begin{center}\rule{0.5\linewidth}{0.5pt}\end{center}

\subsection*{પ્રશ્ન 5(બ અથવા) [4
ગુણ]}\label{uxaaauxab0uxab6uxaa8-5uxaac-uxa85uxaa5uxab5-4-uxa97uxaa3}

\textbf{રડારની સોનાર ની સાથે તુલના કરો.}

\begin{solutionbox}

{\def\LTcaptype{none} % do not increment counter
\begin{longtable}[]{@{}lll@{}}
\toprule\noalign{}
પેરામીટર & રડાર & સોનાર \\
\midrule\noalign{}
\endhead
\bottomrule\noalign{}
\endlastfoot
\textbf{વેવ ટાઇપ} & ઇલેક્ટ્રોમેગ્નેટિક વેવ્સ & સાઉન્ડ વેવ્સ \\
\textbf{મીડિયમ} & હવા/વેક્યુમ & પાણી/લિક્વિડ \\
\textbf{ફ્રીક્વન્સી} & GHz રેન્જ & kHz રેન્જ \\
\textbf{સ્પીડ} & 3 \times 10^{8} m/s & પાણીમાં 1500 m/s \\
\textbf{રેન્જ} & ખૂબ લાંબી રેન્જ & એબ્સોર્પ્શન દ્વારા મર્યાદિત \\
\textbf{ઉપયોગ} & હવા/સ્પેસ ડિટેક્શન & અંડરવોટર ડિટેક્શન \\
\end{longtable}
}

\textbf{સમાનતાઓ:}

\begin{itemize}
\tightlist
\item
  \textbf{ડિટેક્શન માટે ઇકો સિદ્ધાંત}
\item
  \textbf{ટાઇમ ડિલે વડે રેન્જ મેઝરમેન્ટ}
\item
  \textbf{વેલોસિટી મેઝરમેન્ટ માટે ડોપ્લર એફેક્ટ}
\end{itemize}

\end{solutionbox}
\begin{mnemonicbox}
``રડાર રેડિએટ કરે, સોનાર સાઉન્ડ કરે''

\end{mnemonicbox}
\begin{center}\rule{0.5\linewidth}{0.5pt}\end{center}

\subsection*{પ્રશ્ન 5(ક અથવા) [7
ગુણ]}\label{uxaaauxab0uxab6uxaa8-5uxa95-uxa85uxaa5uxab5-7-uxa97uxaa3}

\textbf{મહત્તમ રડાર રેન્જનું સમીકરણ મેળવો.}

\begin{solutionbox}

\textbf{રડાર રેન્જ સમીકરણ વ્યુત્પત્તિ:}

\textbf{ટ્રાન્સમિટેડ પાવર:} Pt

\textbf{ટાર્ગેટ પર પાવર ડેન્સિટી:} Pd = Pt/(4πR^{2})

\textbf{ટાર્ગેટ દ્વારા ઇન્ટરસેપ્ટેડ પાવર:} Pi = Pd \times σ = (Pt \times σ)/(4πR^{2})

\textbf{રડાર તરફ પાછું આવતું પાવર:} Pr = Pi/(4πR^{2}) = (Pt \times σ)/(4πR^{2})^{2}

\textbf{રિસીવ્ડ પાવર:} Pr = (Pt \times G^{2} \times λ^{2} \times σ)/((4π)^{3} \times R^{4})

\textbf{મેક્સિમમ રેન્જ સમીકરણ:}

\begin{verbatim}
Rmax = ^{4}\sqrt[(Pt \times G^{2} \times λ^{2} \times σ)/((4π)^{3} \times Prmin)]
\end{verbatim}

\textbf{જ્યાં:}

\begin{itemize}
\tightlist
\item
  Pt = ટ્રાન્સમિટેડ પાવર
\item
  G = એન્ટીના ગેઇન\\
\item
  λ = વેવલેન્થ
\item
  σ = રડાર ક્રોસ સેક્શન
\item
  Prmin = મિનિમમ ડિટેક્ટેબલ સિગ્નલ
\item
  R = રેન્જ
\end{itemize}

\textbf{રેન્જ અસર કરતા પરિબળો:}

\begin{itemize}
\tightlist
\item
  \textbf{ટ્રાન્સમિટેડ પાવર} (રેન્જ વધારે છે)
\item
  \textbf{એન્ટીના ગેઇન} (રેન્જ વધારે છે)
\item
  \textbf{ટાર્ગેટ ક્રોસ-સેક્શન} (રેન્જ વધારે છે)
\item
  \textbf{ફ્રીક્વન્સી} (પ્રોપેગેશનને અસર કરે છે)
\item
  \textbf{રિસીવર સેન્સિટિવિટી} (મિનિમમ સિગ્નલને અસર કરે છે)
\end{itemize}

\textbf{પ્રેક્ટિકલ વિચારણાઓ:}

\begin{itemize}
\tightlist
\item
  \textbf{એટમોસ્ફેરિક લોસેસ}
\item
  \textbf{ગ્રાઉન્ડ રિફ્લેક્શન}
\item
  \textbf{નોઇઝ લિમિટેશન}
\item
  \textbf{ક્લટર એફેક્ટ્સ}
\end{itemize}

\end{solutionbox}
\begin{mnemonicbox}
``પાવર ગેઇન લેમ્બડા સિગ્મા રેન્જ''

\end{mnemonicbox}
\begin{center}\rule{0.5\linewidth}{0.5pt}\end{center}


\end{document}
