\documentclass{article}

% content/resources/templates/preamble.tex
\usepackage[margin=0.6in]{geometry}
\author{Milav Dabgar}
\usepackage{amsmath,amssymb,amsthm}
\usepackage{booktabs}
\usepackage{multirow}
\usepackage{xcolor}
\usepackage{tcolorbox}
\tcbuselibrary{breakable,skins}
\usepackage[colorlinks=true,linkcolor=blue]{hyperref}
\usepackage{titlesec}
\usepackage{enumitem}
\usepackage{tikz}
\usepackage{pgfplots}
\usepackage{circuitikz}
\usepackage[version=4]{mhchem}
\usepackage{longtable}
\usepackage{array}
\usepackage{float}
\usepackage{caption}
\usepackage{listings}

\lstset{
  basicstyle=\small\ttfamily,
  breaklines=true,
  breakatwhitespace=false,
  postbreak=\mbox{\textcolor{red}{$\hookrightarrow$}\space},
  float=false,
  numbers=left,
  numberstyle=\tiny\color{gray},
  numbersep=10pt,
  xleftmargin=2em,
  keywordstyle=\color{blue},
  commentstyle=\color{green!60!black},
  stringstyle=\color{purple},
  backgroundcolor=\color{gray!5},
  showstringspaces=false,
  tabsize=2,
  captionpos=b,
  keepspaces=true,
  columns=flexible
}

\pgfplotsset{compat=1.18}
\usetikzlibrary{shapes,arrows,positioning,calc,patterns,decorations.pathmorphing,decorations.markings,arrows.meta}

% Color scheme
\definecolor{headcolor}{RGB}{0,102,204}
\definecolor{keycolor}{RGB}{220,20,60}
\definecolor{solutioncolor}{RGB}{34,139,34}
\definecolor{mnemoniccolor}{RGB}{148,0,211}
\definecolor{codecolor}{RGB}{0,0,100}

% Spacing
\setlength{\parskip}{3pt}
\setlist[itemize]{nosep}
\setlist[enumerate]{nosep}

% Title formatting
\titleformat{\section}{\Large\bfseries\color{headcolor}}{\thesection}{1em}{}
\titleformat{\subsection}{\large\bfseries\color{headcolor}}{\thesubsection}{1em}{}

% Pandoc tightlist compatibility
\providecommand{\tightlist}{%
  \setlength{\itemsep}{0pt}\setlength{\parskip}{0pt}}

% Pandoc longtable compatibility
\newcounter{none}
\def\thenone{}


% content/resources/templates/english-boxes.tex

% Custom environments
\newtcolorbox{solutionbox}{
 breakable,
 enhanced,
 colback=solutioncolor!5!white,
 colframe=solutioncolor!75!black,
 fonttitle=\bfseries,
 title=Solution
}

\newtcolorbox{solutionboxnobreak}{
 colback=solutioncolor!5!white,
 colframe=solutioncolor!75!black,
 fonttitle=\bfseries,
 title=Solution
}

\newtcolorbox{keyformula}{
 breakable,
 enhanced,
 colback=keycolor!5!white,
 colframe=keycolor!75!black,
 fonttitle=\bfseries,
 title=Key Formula
}

\newtcolorbox{mnemonicboxenv}{
 breakable,
 enhanced,
 colback=mnemoniccolor!5!white,
 colframe=mnemoniccolor!75!black,
 fonttitle=\bfseries,
 title=Mnemonic
}

\newcommand{\mnemonicbox}[1]{%
  \begin{mnemonicboxenv}
    #1
  \end{mnemonicboxenv}
}


% Custom commands for GTU solutions
% This file defines semantic commands for consistent formatting

% Question command with automatic formatting
\newcommand{\question}[2]{%
  \section*{Question #1}%
  \textbf{#2}%
}

% OR question variant
\newcommand{\questionor}[2]{%
  \section*{Question #1 OR}%
  \textbf{#2}%
}

% Proper table environment with caption
\newenvironment{answertable}[1]{%
  \begin{table}[htbp]
  \centering
  \caption{#1}
}{%
  \end{table}
}

% Proper figure environment for diagrams
\newenvironment{answerdiagram}[1]{%
  \begin{figure}[htbp]
  \centering
  \caption{#1}
}{%
  \end{figure}
}

% Semantic markup for key terms
\newcommand{\keyword}[1]{\textbf{#1}}
\newcommand{\code}[1]{\texttt{#1}}
\newcommand{\classname}[1]{\texttt{#1}}
\newcommand{\methodname}[1]{\texttt{#1}}

% Proper quotation marks
\newcommand{\mnemonic}[1]{``#1''}


\usepackage{fontspec}
\newfontfamily\gujaratifont[Script=Gujarati]{Gujarati Sangam MN}
\renewcommand{\gu}[1]{{\gujaratifont #1}}

\title{\gu{માઇક્રોવેવ અને રડાર કોમ્યુનિકેશન (4351103) - શિયાળો 2023 ઉકેલ}}
\date{\gu{8 ડિસેમ્બર, 2023}}

\begin{document}
\maketitle

\questionmarks{1(a)}{3}{\gu{ટ્રાન્સમિશન લાઇન માં વોલ્ટેજ અને કરંટ માટે સ્ટેન્ડિંગ વેવ પેટર્નને સ્કેચ કરો, જ્યારે તે (i) શોર્ટ સર્કિટ, (ii) ઓપન સર્કિટ અને (iii) મેચ્ડ લોડ સાથે સમાપ્ત થાય છે.}}

\begin{solutionbox}
\textbf{\gu{સ્ટેન્ડિંગ વેવ પેટર્ન:}}

\begin{answerdiagram}{Standing Wave Patterns}
\begin{tikzpicture}
    % Short Circuit
    \begin{scope}[yshift=4cm]
        \node[anchor=west] at (-1, 1.5) {\textbf{\gu{(i) શોર્ટ સર્કિટ ($Z_L=0$)}}};
        \draw[->] (0,0) -- (6.5,0) node[right] {$z$};
        \draw[->] (0,-1.2) -- (0,1.2) node[above] {$V, I$};
        
        \draw[blue, thick] plot[domain=0:6.28, samples=100, variable=\x] ({\x}, {abs(sin(\x*180/3.14))});
        \node[blue] at (6.5, 0.5) {$V$};
        
        \draw[red, thick, dashed] plot[domain=0:6.28, samples=100, variable=\x] ({\x}, {abs(cos(\x*180/3.14))});
        \node[red] at (6.5, 1) {$I$};
        
        \node[below] at (0,0) {\gu{લોડ}};
        \node[below] at (1.57,0) {$\lambda/4$};
        \node[below] at (3.14,0) {$\lambda/2$};
    \end{scope}

    % Open Circuit
    \begin{scope}[yshift=0cm]
        \node[anchor=west] at (-1, 1.5) {\textbf{\gu{(ii) ઓપન સર્કિટ ($Z_L=\infty$)}}};
        \draw[->] (0,0) -- (6.5,0) node[right] {$z$};
        \draw[->] (0,-1.2) -- (0,1.2) node[above] {$V, I$};
        
        \draw[blue, thick] plot[domain=0:6.28, samples=100, variable=\x] ({\x}, {abs(cos(\x*180/3.14))});
        \node[blue] at (6.5, 1) {$V$};
        
        \draw[red, thick, dashed] plot[domain=0:6.28, samples=100, variable=\x] ({\x}, {abs(sin(\x*180/3.14))});
        \node[red] at (6.5, 0.5) {$I$};
        
        \node[below] at (0,0) {\gu{લોડ}};
        \node[below] at (1.57,0) {$\lambda/4$};
        \node[below] at (3.14,0) {$\lambda/2$};
    \end{scope}

    % Matched Load
    \begin{scope}[yshift=-4cm]
        \node[anchor=west] at (-1, 1.5) {\textbf{\gu{(iii) મેચ્ડ લોડ ($Z_L=Z_0$)}}};
        \draw[->] (0,0) -- (6.5,0) node[right] {$z$};
        \draw[->] (0,-1.2) -- (0,1.2) node[above] {$V, I$};
        
        \draw[blue, thick] (0,0.8) -- (6.28,0.8);
        \node[blue] at (6.5, 0.8) {$V$};
        
        \draw[red, thick, dashed] (0,0.5) -- (6.28,0.5);
        \node[red] at (6.5, 0.5) {$I$};
        
        \node[below] at (0,0) {\gu{લોડ}};
    \end{scope}
\end{tikzpicture}
\end{answerdiagram}

\begin{itemize}
    \item \textbf{\gu{શોર્ટ સર્કિટ}}: \gu{લોડ પર વોલ્ટેજ ન્યૂનતમ (શૂન્ય). કરંટ મહત્તમ.}
    \item \textbf{\gu{ઓપન સર્કિટ}}: \gu{લોડ પર વોલ્ટેજ મહત્તમ. કરંટ ન્યૂનતમ (શૂન્ય).}
    \item \textbf{\gu{મેચ્ડ લોડ}}: \gu{કોઈ સ્ટેન્ડિંગ વેવ નથી. વોલ્ટેજ અને કરંટ અચળ હોય છે.}
\end{itemize}
\end{solutionbox}

\begin{mnemonicbox}
\mnemonic{\gu{SOC - શોર્ટ કરંટ ખોલે, ઓપન કરંટ બંધ કરે}}
\end{mnemonicbox}

\questionmarks{1(b)}{4}{\gu{માઇક્રોવેવ ફ્રીક્વન્સી માટે બે સમાંતર વાયર ટ્રાન્સમિશન લાઇનના સમકક્ષ સર્કિટનો નકશો દોરો અને સમજાવો.}}

\begin{solutionbox}
\textbf{\gu{સમકક્ષ સર્કિટ:}}

\begin{answerdiagram}{Transmission Line Equivalent Circuit}
\begin{circuitikz}[scale=1.2, transform shape]
    \draw (0,2) to[R, l=$R$] (2,2) to[L, l=$L$] (4,2) -- (6,2);
    \draw (0,0) -- (6,0);
    \draw (4,2) to[G, l=$G$] (4,1) to[C, l=$C$] (4,0);
    \draw (2,0) to[short] (2,0); 
    
    \node[left] at (0,1) {\gu{ઇનપુટ}};
    \node[right] at (6,1) {\gu{આઉટપુટ}};
    \draw[<->] (0,-0.5) -- (6,-0.5) node[midway, below] {\gu{લંબાઈ $\Delta z$}};
\end{circuitikz}
\end{answerdiagram}

\textbf{\gu{પ્રાથમિક સ્થિરાંકો:}}
\begin{itemize}
    \item \keyword{R (\gu{પ્રતિકાર})}: \gu{એકમ લંબાઈ દીઠ શ્રેણી પ્રતિકાર (કંડક્ટર લોસિસ) ($\Omega/m$).}
    \item \keyword{L (\gu{ઇન્ડક્ટન્સ})}: \gu{એકમ લંબાઈ દીઠ શ્રેણી ઇન્ડક્ટન્સ (ચુંબકીય ક્ષેત્ર સંગ્રહ) ($H/m$).}
    \item \keyword{G (\gu{કંડક્ટન્સ})}: \gu{એકમ લંબાઈ દીઠ શંટ કંડક્ટન્સ (ડાઇઇલેક્ટ્રિક લોસિસ) ($S/m$).}
    \item \keyword{C (\gu{કેપેસિટન્સ})}: \gu{એકમ લંબાઈ દીઠ શંટ કેપેસિટન્સ (વિદ્યુત ક્ષેત્ર સંગ્રહ) ($F/m$).}
\end{itemize}

\textbf{\gu{પરિમાપો કોષ્ટક:}}
\begin{tabulary}{\linewidth}{|L|L|L|L|}
\hline
\textbf{\gu{પરિમાપ}} & \textbf{\gu{પ્રતીક}} & \textbf{\gu{એકમ}} & \textbf{\gu{અસર}} \\ \hline
\gu{પ્રતિકાર} & R & $\Omega/m$ & \gu{શક્તિ નુકસાન} \\ \hline
\gu{ઇન્ડક્ટન્સ} & L & H/m & \gu{ચુંબકીય ઊર્જા} \\ \hline
\gu{કંડક્ટન્સ} & G & S/m & \gu{લીકેજ કરંટ} \\ \hline
\gu{કેપેસિટન્સ} & C & F/m & \gu{વિદ્યુત ઊર્જા} \\ \hline
\end{tabulary}
\end{solutionbox}

\begin{mnemonicbox}
\mnemonic{\gu{RLGC - ખરેખર મોટી કેબલ્સ}}
\end{mnemonicbox}

\questionmarks{1(c)}{7}{\gu{આઇસોલેટર ના સિદ્ધાંત, બાંધકામ અને કાર્યને જરૂરી સ્કેચ સાથે સમજાવો.}}

\begin{solutionbox}
\textbf{\gu{સિદ્ધાંત:}}
\gu{આઇસોલેટર માઇક્રોવેવ સિગ્નલને ફક્ત આગળની દિશામાં જ પસાર કરવા દે છે પરંતુ પાછળની દિશામાં શોષી લે છે. તે ફેરાઇટ મટિરિયલ અને \keyword{ફેરાડે રોટેશન} અસરનો ઉપયોગ કરે છે.}

\begin{answerdiagram}{Isolator Construction}
\begin{tikzpicture}[auto, node distance=2cm]
    \node [gtu block] (ferrite) {\gu{ફેરાઇટ રોડ}};
    \node [gtu input, left=of ferrite] (in) {\gu{ઇનપુટ}};
    \node [gtu output, right=of ferrite] (out) {\gu{આઉટપુટ}};
    \node [gtu block, above=of ferrite] (magnet) {\gu{પરમેનન્ટ મેગ્નેટ}};
    \node [gtu block, below=of ferrite] (res) {\gu{રેઝિસ્ટિવ કાર્ડ}};
    
    \draw [gtu arrow] (in) -- (ferrite);
    \draw [gtu arrow] (ferrite) -- (out);
    \draw [dashed] (magnet) -- (ferrite);
    \draw [dotted] (ferrite) -- (res);
    
    \node [below=0.5cm of res] {\gu{વેવગાઇડ સેક્શન}};
\end{tikzpicture}
\end{answerdiagram}

\textbf{\gu{કાર્યપ્રણાલી:}}
\begin{itemize}
    \item \keyword{\gu{આગળની દિશા}}: \gu{સિગ્નલ ઇનપુટથી આવે છે. ફેરાઇટ તેને $45^\circ$ ફેરવે છે. તે આઉટપુટમાંથી પસાર થાય છે કારણ કે આઉટપુટ રેઝિસ્ટિવ કાર્ડ લંબરૂપ છે.}
    \item \keyword{\gu{પાછળની દિશા}}: \gu{આઉટપુટથી આવતું પ્રતિબિંબિત સિગ્નલ બીજું $45^\circ$ ફેરવાય છે (કુલ $90^\circ$). આ ફીલ્ડ ઇનપુટ રેઝિસ્ટિવ કાર્ડને સમાંતર બને છે અને \keyword{શોષાય} છે.}
\end{itemize}

\textbf{\gu{ઉપયોગો:}}
\begin{itemize}
    \item \gu{ટ્રાન્સમિટરને (જેમ કે ક્લિસ્ટ્રોન) રિફ્લેક્ટેડ પાવરથી સુરક્ષિત કરવા.}
    \item \gu{ફ્રીક્વન્સી પુલિંગ અટકાવવા.}
\end{itemize}
\end{solutionbox}

\begin{mnemonicbox}
\mnemonic{\gu{આગળ અલગ કરો, પાછળ શોષો}}
\end{mnemonicbox}

\orquestionmarks{1(c)}{7}{\gu{ટ્રાન્સમિશન લાઇન અને વેવગાઇડની સરખામણી કરો.}}

\begin{solutionbox}
\textbf{\gu{સરખામણી:}}

\begin{tabulary}{\linewidth}{|L|L|L|}
\hline
\textbf{\gu{પરિમાપ}} & \textbf{\gu{ટ્રાન્સમિશન લાઇન}} & \textbf{\gu{વેવગાઇડ}} \\ \hline
\keyword{\gu{ફ્રીક્વન્સી}} & \gu{DC થી માઇક્રોવેવ} & \gu{માઇક્રોવેવ અને ઉપર (હાઇ ફ્રીક્વન્સી)} \\ \hline
\keyword{\gu{સ્ટ્રક્ચર}} & \gu{બે કંડક્ટર (દા.ત. કોએક્સિયલ)} & \gu{સિંગલ હોલો કંડક્ટર} \\ \hline
\keyword{\gu{મોડ}} & \gu{TEM મોડ સપોર્ટ કરે છે} & \gu{ફક્ત TE અને TM મોડ સપોર્ટ કરે છે} \\ \hline
\keyword{\gu{કટઓફ}} & \gu{કોઈ કટઓફ નથી (DC પાસ કરે)} & \gu{કટઓફ ફ્રીક્વન્સી ($f_c$) હોય છે} \\ \hline
\keyword{\gu{લોસિસ}} & \gu{વધારે ($I^2R$ અને ડાઇઇલેક્ટ્રિક)} & \gu{ઓછા (એર ડાઇઇલેક્ટ્રિક)} \\ \hline
\keyword{\gu{પાવર}} & \gu{મર્યાદિત પાવર ક્ષમતા} & \gu{ઉચ્ચ પાવર હેન્ડલિંગ ક્ષમતા} \\ \hline
\end{tabulary}
\end{solutionbox}

\begin{mnemonicbox}
\mnemonic{\gu{ટ્રાન્સમિશન બે-વાયર ચાલે, વેવગાઇડ વિશાળ ચાલે}}
\end{mnemonicbox}

\questionmarks{2(a)}{3}{\gu{વ્યાખ્યા આપો: (i) VSWR, (ii) રિફ્લેક્શન કોઇફિશન્ટ, અને (iii) સ્કિન અસર}}

\begin{solutionbox}
\textbf{\gu{વ્યાખ્યાઓ:}}

\begin{enumerate}
    \item \textbf{\gu{VSWR (વોલ્ટેજ સ્ટેન્ડિંગ વેવ રેશિયો):}} \gu{ટ્રાન્સમિશન લાઇન પર સ્ટેન્ડિંગ વેવ પેટર્નમાં મહત્તમ વોલ્ટેજ અને ન્યૂનતમ વોલ્ટેજનો ગુણોત્તર.}
    \[ VSWR = \frac{V_{max}}{V_{min}} = \frac{1+|\Gamma|}{1-|\Gamma|} \]
    
    \item \textbf{\gu{રિફ્લેક્શન કોઇફિશન્ટ ($\Gamma$):}} \gu{લોડ પર પ્રતિબિંબિત વોલ્ટેજ અને આપાત વોલ્ટેજનો ગુણોત્તર.}
    \[ \Gamma = \frac{V_{ref}}{V_{inc}} = \frac{Z_L - Z_0}{Z_L + Z_0} \]
    
    \item \textbf{\gu{સ્કિન અસર}:} \gu{ઉચ્ચ ફ્રીક્વન્સીએ, અલ્ટરનેટિંગ કરંટ કંડક્ટરના સમગ્ર આડછેદને બદલે સપાટી પર વહેવાનું વલણ ધરાવે છે. આ ઊંડાઈને \keyword{સ્કિન ડેપ્થ ($\delta$)} કહેવાય છે.}
\end{enumerate}
\end{solutionbox}

\begin{mnemonicbox}
\mnemonic{\gu{VSWR વેરિયે, ગામા ગાઇડ, સ્કિન સંકોચે}}
\end{mnemonicbox}

\questionmarks{2(b)}{4}{\gu{યોગ્ય સ્કેચ સાથે ટુ-હોલ ડાયરેક્શનલ કપ્લરનું કાર્ય સમજાવો.}}

\begin{solutionbox}
\textbf{\gu{ટુ-હોલ ડાયરેક્શનલ કપ્લર:}}

\begin{answerdiagram}{Two-Hole Directional Coupler}
\begin{tikzpicture}[scale=0.8]
    % Main Waveguide
    \draw[thick] (0, 0) rectangle (6, 1.5);
    \node at (3, 0.75) {\gu{મેઇન વેવગાઇડ}};
    \node[left] at (0, 0.75) {\gu{પોર્ટ 1 (ઇનપુટ)}};
    \node[right] at (6, 0.75) {\gu{પોર્ટ 2}};
    
    % Aux Waveguide
    \draw[thick] (0, -2) rectangle (6, -0.5);
    \node at (3, -1.25) {\gu{ઓક્સિલરી વેવગાઇડ}};
    \node[left] at (0, -1.25) {\gu{પોર્ટ 4}};
    \node[right] at (6, -1.25) {\gu{પોર્ટ 3 (કપલ્ડ)}};
    
    % Holes
    \foreach \x in {2.5, 3.5} {
        \draw[dashed, ->] (\x, 0) -- (\x, -0.5);
        \filldraw[white] (\x-0.1, -0.1) rectangle (\x+0.1, 0.1);
        \draw (\x, 0) circle (0.1);
        \filldraw[white] (\x-0.1, -0.6) rectangle (\x+0.1, -0.4);
        \draw (\x, -0.5) circle (0.1);
    }
    \draw[<->] (2.5, -0.25) -- (3.5, -0.25) node[midway, above] {$S = \lambda_g/4$};
\end{tikzpicture}
\end{answerdiagram}

\textbf{\gu{કાર્યપ્રણાલી:}}
\begin{itemize}
    \item \keyword{\gu{અંતર}}: \gu{બે છિદ્રો વચ્ચેનું અંતર $S = \lambda_g/4$ છે.}
    \item \keyword{\gu{આગળનું તરંગ}}: \gu{પોર્ટ 1 થી આવતું સિગ્નલ બંને છિદ્રો દ્વારા પોર્ટ 3 તરફ જાય છે. પાથ તફાવત શૂન્ય છે, તેથી પોર્ટ 3 પર સરવાળો થાય છે.}
    \item \keyword{\gu{પાછળનું તરંગ}}: \gu{પોર્ટ 4 તરફ જતા સિગ્નલો વચ્ચે પાથ તફાવત $2S = \lambda_g/2$ ($180^\circ$) છે, તેથી તેઓ એકબીજાને રદ કરે છે.}
\end{itemize}
\end{solutionbox}

\begin{mnemonicbox}
\mnemonic{\gu{બે છિદ્ર, બે દિશા, સંપૂર્ણ નિયંત્રણ}}
\end{mnemonicbox}

\questionmarks{2(c)}{7}{\gu{વેવગાઇડ દ્વારા માઇક્રોવેવનું પ્રસારણ વર્ણવો અને કટ ઓફ તરંગલંબાઇનું સમીકરણ મેળવો.}}

\begin{solutionbox}
\textbf{\gu{વેવ પ્રસારણ:}}
\gu{માઇક્રોવેવ્સ વેવગાઇડમાં વાહક દિવાલોના પરાવર્તન દ્વારા પ્રસારિત થાય છે. તે TE અને TM મોડ્સને સપોર્ટ કરે છે.}

\textbf{\gu{કટ-ઓફ તરંગલંબાઇ:}}
\gu{લંબચોરસ વેવગાઇડ માટે હેલ્મહોલ્ટ્ઝ સમીકરણ ઉકેલતા:}
\[ \left( \frac{2\pi f_c}{c} \right)^2 = \left( \frac{m\pi}{a} \right)^2 + \left( \frac{n\pi}{b} \right)^2 \]
\gu{કટ-ઓફ તરંગલંબાઇ $\lambda_c$:}
\[ \lambda_c = \frac{2}{\sqrt{\left( \frac{m}{a} \right)^2 + \left( \frac{n}{b} \right)^2}} \]

\textbf{\gu{ડોમિનન્ટ મોડ (TE$_{10}$):}}
$m=1, n=0$:
\[ \lambda_c = 2a \]

\begin{answerdiagram}{Mode Hierarchy}
\begin{tikzpicture}[auto, node distance=1.5cm]
    \node [gtu state] (te10) {TE$_{10}$\\(\gu{ડોમિનન્ટ})};
    \node [gtu state, right=of te10] (te20) {TE$_{20}$};
    \node [gtu state, below=of te10] (te01) {TE$_{01}$};
    \node [gtu state, right=of te01] (te11) {TE$_{11}$};
    
    \draw [gtu arrow] (te10) -- (te20);
    \draw [gtu arrow] (te10) -- (te01);
    \draw [gtu arrow] (te20) -- (te11);
    \draw [gtu arrow] (te01) -- (te11);
\end{tikzpicture}
\end{answerdiagram}
\end{solutionbox}

\begin{mnemonicbox}
\mnemonic{\gu{કટ-ઓફ આવે, પ્રસારણ આગળ વધે}}
\end{mnemonicbox}

\orquestionmarks{2(a)}{3}{\gu{સિંગલ સ્ટબનો ઉપયોગ કરીને ઇમ્પીડન્સ મેચિંગ સમજાવો.}}

\begin{solutionbox}
\textbf{\gu{સિંગલ સ્ટબ મેચિંગ:}}
\gu{ટ્રાન્સમિશન લાઇન પર લોડ $Z_L$ ને મેચ કરવા માટે પેરેલલ (શંટ) સ્ટબનો ઉપયોગ થાય છે.}

\begin{answerdiagram}{Single Stub Matching}
\begin{tikzpicture}[scale=1]
    \draw[thick] (0,2) -- (4,2);
    \draw[thick] (0,0) -- (4,0);
    \draw[thick, fill=gray!20] (4,0) rectangle (4.5,2);
    \node at (4.25,1) {$Z_L$};
    
    \draw[thick] (2,2) -- (2,3.5);
    \draw[thick] (2.2,2) -- (2.2,3.5);
    \draw (2,3.5) -- (2.2,3.5); % Short circuit
    \node[left] at (2, 2.75) {$l_s$};
    \node[right] at (2.2, 2.75) {\gu{સ્ટબ}};
    
    \draw[<->] (2, -0.3) -- (4, -0.3) node[midway, below] {$d$};
    \node at (0,1) {$Z_0$};
\end{tikzpicture}
\end{answerdiagram}
\end{solutionbox}

\begin{mnemonicbox}
\mnemonic{\gu{સિંગલ સ્ટબ સસેપ્ટન્સ ઉકેલે}}
\end{mnemonicbox}

\orquestionmarks{2(b)}{4}{\gu{હાઇબ્રિડ રિંગને જરૂરી સ્કેચ સાથે સમજાવો.}}

\begin{solutionbox}
\textbf{\gu{હાઇબ્રિડ રિંગ (રેટ-રેસ કપ્લર):}}
\gu{4-પોર્ટ કપ્લર જેનો ઉપયોગ પાવર સ્પ્લિટિંગ અથવા સિગ્નલ કમ્બાઇનિંગ માટે થાય છે.}

\begin{answerdiagram}{Hybrid Ring Structure}
\begin{tikzpicture}
    % Ring
    \draw[thick] (0,0) circle (2);
    
    % Ports
    \node[draw, circle, fill=white] at (0,2) (p1) {1};
    \node[draw, circle, fill=white] at (2,0) (p2) {2};
    \node[draw, circle, fill=white] at (0,-2) (p3) {3};
    \node[draw, circle, fill=white] at (-2,0) (p4) {4};
    
    \node[above] at (0,2.3) {$\Sigma$ (\gu{ઇનપુટ})};
    \node[below] at (0,-2.3) {$\Delta$ (\gu{ઇનપુટ})};
    
    \node at (0,0) {\gu{પરિધિ} $= 1.5\lambda$};
\end{tikzpicture}
\end{answerdiagram}

\textbf{\gu{કાર્ય:}}
\begin{itemize}
    \item \gu{પોર્ટ 1 ઇનપુટ પોર્ટ 2 અને 4 માં સમાન વિભાજિત થાય છે (ઇન-ફેઝ). પોર્ટ 3 આઇસોલેટેડ રહે છે.}
    \item \gu{પોર્ટ 3 ઇનપુટ પોર્ટ 2 અને 4 માં વિભાજિત થાય છે (આઉટ-ઓફ-ફેઝ). પોર્ટ 1 આઇસોલેટેડ રહે છે.}
\end{itemize}
\end{solutionbox}

\begin{mnemonicbox}
\mnemonic{\gu{રિંગ ફરે, પોર્ટ જોડાય}}
\end{mnemonicbox}

\orquestionmarks{2(c)}{7}{\gu{મેજિક ટીના બાંધકામ, કાર્ય અને કોઈપણ એક એપ્લિકેશનને જરૂરી ડાયાગ્રામ સાથે સમજાવો.}}

\begin{solutionbox}
\textbf{\gu{મેજિક ટી:}}
\gu{આ E-પ્લેન અને H-પ્લેન ટીનું સંયોજન છે.}

\begin{answerdiagram}{Magic Tee Construction}
\begin{tikzpicture}[scale=1]
    % Arms
    \draw[thick] (-3,0) -- (3,0); 
    \node at (-3.5,0) {\gu{પોર્ટ 1}};
    \node at (3.5,0) {\gu{પોર્ટ 2}};
    \node at (0,-0.4) {\gu{કોલિનિયર આર્મ્સ}};

    % E-Arm
    \draw[thick] (-0.3,0) -- (-0.3,2);
    \draw[thick] (0.3,0) -- (0.3,2);
    \draw[thick] (-0.3,2) -- (0.3,2);
    \node[above] at (0,2) {\gu{પોર્ટ 4 (E-આર્મ)}};

    % H-Arm
    \draw[thick] (-0.3,0) -- (-0.3,-1.5);
    \draw[thick] (0.3,0) -- (0.3,-1.5);
    \node[below] at (0,-1.5) {\gu{પોર્ટ 3 (H-આર્મ)}};
\end{tikzpicture}
\end{answerdiagram}

\textbf{\gu{કાર્યપ્રણાલી:}}
\begin{itemize}
    \item \gu{H-આર્મ ઇનપુટ (પોર્ટ 3): પાવર પોર્ટ 1 અને 2 માં સમાન અને ઇન-ફેઝ વિભાજિત થાય છે.}
    \item \gu{E-આર્મ ઇનપુટ (પોર્ટ 4): પાવર પોર્ટ 1 અને 2 માં સમાન અને આઉટ-ઓફ-ફેઝ વિભાજિત થાય છે.}
\end{itemize}

\textbf{\gu{એપ્લિકેશન - રડાર ડુપ્લેક્સર:}}
\gu{તે સિંગલ એન્ટેનાને ટ્રાન્સમિટર અને રિસીવર બંને સાથે જોડવા માટે વપરાય છે, જ્યારે ટ્રાન્સમિટર અને રિસીવરને એકબીજાથી અલગ રાખે છે.}
\end{solutionbox}

\begin{mnemonicbox}
\mnemonic{\gu{મેજિક આઇસોલેશન બનાવે, ટી સાથે ટ્રાન્સમિટ}}
\end{mnemonicbox}

\questionmarks{3(a)}{3}{\gu{બ્લોક ડાયાગ્રામની મદદથી એટેન્યુએશન માપન સમજાવો.}}

\begin{solutionbox}
\textbf{\gu{એટેન્યુએશન માપન:}}

\begin{answerdiagram}{Attenuation Measurement Setup}
\begin{tikzpicture}[auto, node distance=1cm]
    \node [gtu start] (gen) {\gu{સોર્સ}};
    \node [gtu block, right=of gen] (iso) {\gu{આઇસોલેટર}};
    \node [gtu block, right=of iso] (dut) {\gu{એટેન્યુએટર}};
    \node [gtu stop, right=of dut] (meter) {\gu{પાવર મીટર}};
    
    \draw [gtu arrow] (gen) -- (iso);
    \draw [gtu arrow] (iso) -- (dut);
    \draw [gtu arrow] (dut) -- (meter);
\end{tikzpicture}
\end{answerdiagram}

\textbf{\gu{રીત:}}
\begin{itemize}
    \item \gu{$P_1$: એટેન્યુએટર વિના પાવર માપો.}
    \item \gu{$P_2$: એટેન્યુએટર સાથે પાવર માપો.}
    \item \gu{એટેન્યુએશન (dB) $= 10 \log_{10} (P_1/P_2)$.}
\end{itemize}
\end{solutionbox}

\begin{mnemonicbox}
\mnemonic{\gu{એટેન્યુએશન = પાવર 1 / પાવર 2}}
\end{mnemonicbox}

\questionmarks{3(b)}{4}{\gu{એપલગેટ ડાયાગ્રામની મદદથી બે કેવિટી ક્લિસ્ટ્રોનમાં વેગ મોડ્યુલેશન સમજાવો.}}

\begin{solutionbox}
\textbf{\gu{બે કેવિટી ક્લિસ્ટ્રોન:}}

\begin{answerdiagram}{Klystron Structure}
\begin{tikzpicture}[auto, node distance=1.5cm]
    \node [gtu start] (gun) {\gu{ગન}};
    \node [gtu block, right=of gun] (cav1) {\gu{બંચર}};
    \node [gtu block, right=of cav1, minimum width=2.5cm] (drift) {\gu{ડ્રિફ્ટ સ્પેસ}};
    \node [gtu block, right=of drift] (cav2) {\gu{કેચર}};
    \node [gtu stop, right=of cav2] (col) {\gu{કલેક્ટર}};
    
    \draw [gtu arrow] (gun) -- (cav1);
    \draw [gtu arrow] (cav1) -- (drift);
    \draw [gtu arrow] (drift) -- (cav2);
    \draw [gtu arrow] (cav2) -- (col);
\end{tikzpicture}
\end{answerdiagram}

\textbf{\gu{વેગ મોડ્યુલેશન:}}
\gu{બંચર કેવિટીમાં RF વોલ્ટેજ ઇલેક્ટ્રોનની ગતિ વધારે કે ઘટાડે છે. ડ્રિફ્ટ સ્પેસમાં, ઝડપી ઇલેક્ટ્રોન ધીમા ઇલેક્ટ્રોનને પકડી લે છે અને 'બંચ' (જૂથ) બનાવે છે.}
\end{solutionbox}

\begin{mnemonicbox}
\mnemonic{\gu{વેલોસિટી વેરિયે, બંચિંગ બિલ્ડ}}
\end{mnemonicbox}

\questionmarks{3(c)}{7}{\gu{મેગ્નેટ્રોનમાં વિદ્યુત અને ચુંબકીય ક્ષેત્રના સિદ્ધાંત, નિર્માણ અને અસર સમજાવો.}}

\begin{solutionbox}
\textbf{\gu{મેગ્નેટ્રોન:}}
\gu{આ ક્રોસ્ડ ઇલેક્ટ્રિક અને મેગ્નેટિક ફીલ્ડ્સનો ઉપયોગ કરતું ઓસિલેટર છે.}

\begin{answerdiagram}{Magnetron Structure}
\begin{tikzpicture}
    % Anode Block
    \draw[thick, fill=gray!20] (0,0) circle (2);
    \draw[fill=white] (0,0) circle (0.8);
    
    % Cavities
    \foreach \a in {0,45,...,315} {
        \draw[fill=white] (\a:1.4) circle (0.3);
        \draw[white, thick] (\a:1.4) -- (\a:0.8);
        \draw (\a:1.25) -- (\a:0.8);
    }
    
    % Cathode
    \draw[fill=black] (0,0) circle (0.2);
    \node at (0,0.5) {\gu{કેથોડ}};
    
    \node[below] at (0,-2.1) {\gu{એનોડ બ્લોક અને કેવિટીઝ}};
\end{tikzpicture}
\end{answerdiagram}

\textbf{\gu{ફીલ્ડ અસર:}}
\begin{itemize}
    \item \gu{ઇલેક્ટ્રિક ફીલ્ડ ઇલેક્ટ્રોનને બહાર ખેંચે છે.}
    \item \gu{મેગ્નેટિક ફીલ્ડ ઇલેક્ટ્રોનનો માર્ગ વાળે છે (વક્ર કરે છે).}
    \item \gu{પરિણામે, ઇલેક્ટ્રોન સ્પાઇરલ પાથમાં ગતિ કરે છે અને કેવિટીને એનર્જી આપે છે.}
\end{itemize}
\end{solutionbox}

\begin{mnemonicbox}
\mnemonic{\gu{મેગ્નેટ્રોન મેગ્નેટિક મોશન દ્વારા માઇક્રોવેવ બનાવે}}
\end{mnemonicbox}

\orquestionmarks{3(a)}{3}{\gu{TWT (ટ્રાવેલિંગ વેવ ટ્યુબ)નું એમ્પ્લિફાયર તરીકે કાર્ય સમજાવો.}}

\begin{solutionbox}
\textbf{\gu{ટ્રાવેલિંગ વેવ ટ્યુબ:}}
\gu{બ્રોડબેન્ડ એમ્પ્લિફાયર જે \keyword{સ્લો વેવ સ્ટ્રક્ચર} (હેલિક્સ) નો ઉપયોગ કરે છે.}

\begin{answerdiagram}{TWT Schematic}
\begin{tikzpicture}[auto, node distance=1.5cm]
    \node [gtu start] (gun) {\gu{ગન}};
    \node [gtu block, right=of gun, minimum width=4cm] (helix) {\gu{હેલિક્સ}};
    \node [gtu stop, right=of helix] (col) {\gu{કલેક્ટર}};
    
    \draw [gtu arrow] (gun) -- (helix);
    \draw [gtu arrow] (helix) -- (col);
    
    \draw [<-] (2, 0.5) -- (2, 1) node[above] {\gu{RF ઇન}};
    \draw [->] (5, 0.5) -- (5, 1) node[above] {\gu{RF આઉટ}};
\end{tikzpicture}
\end{answerdiagram}

\textbf{\gu{કાર્ય:}}
\gu{હેલિક્સ RF વેવની ગતિ ધીમી કરે છે જેથી તે ઇલેક્ટ્રોન બીમની ગતિ સાથે મેચ થાય. આનાથી સતત ઈન્ટરેક્શન અને એમ્પ્લિફિકેશન થાય છે.}
\end{solutionbox}

\begin{mnemonicbox}
\mnemonic{\gu{ટ્રાવેલિંગ વેવ એનર્જી ટ્રાન્સફર કરે}}
\end{mnemonicbox}

\orquestionmarks{3(b)}{4}{\gu{માઇક્રોવેવ ફ્રીક્વન્સી માટે ઓછો પાવર માપવા માટે બોલોમીટર પદ્ધતિ સમજાવો.}}

\begin{solutionbox}
\textbf{\gu{બોલોમીટર પદ્ધતિ:}}
\gu{તાપમાન-સંવેદનશીલ અવરોધ (જેમ કે બેરેટર અથવા થર્મિસ્ટર) નો ઉપયોગ કરે છે.}

\begin{answerdiagram}{Bolometer Bridge Circuit}
\begin{circuitikz}[scale=1]
    \draw (0,0) to[R, l=$R_1$] (0,2) -- (2,2) to[R, l=$R_{Bol}$] (2,0) -- (0,0);
    \draw (0,2) -- (-1,2) to[V, l=$V_{DC}$] (-1,0) -- (0,0);
    \node at (1,1) {\gu{બ્રિજ}};
\end{circuitikz}
\end{answerdiagram}

\textbf{\gu{કાર્ય:}}
\gu{RF પાવર બોલોમીટરને ગરમ કરે છે, તેનો અવરોધ બદલાય છે, અને બ્રિજ અનબેલેન્સ થાય છે. આ ફેરફાર પાવરના પ્રમાણમાં હોય છે.}
\end{solutionbox}

\begin{mnemonicbox}
\mnemonic{\gu{બોલોમીટર બર્ન, બ્રિજ બેલેન્સ}}
\end{mnemonicbox}

\orquestionmarks{3(c)}{7}{\gu{બ્લોક ડાયાગ્રામની મદદથી ફ્રીક્વન્સી અને તરંગલંબાઇ માપન પદ્ધતિ સમજાવો.}}

\begin{solutionbox}
\textbf{\gu{ફ્રીક્વન્સી માપન:}}

\begin{answerdiagram}{Heterodyne Frequency Meter}
\begin{tikzpicture}[auto, node distance=1.5cm]
    \node [gtu input] (rf) {\gu{RF ઇનપુટ}};
    \node [gtu block, right=of rf] (mixer) {\gu{મિક્સર}};
    \node [gtu block, above=of mixer] (lo) {\gu{લોકલ ઓસિલેટર}};
    \node [gtu stop, right=of mixer] (counter) {\gu{કાઉન્ટર}};
    
    \draw [gtu arrow] (rf) -- (mixer);
    \draw [gtu arrow] (lo) -- (mixer);
    \draw [gtu arrow] (mixer) -- (counter);
\end{tikzpicture}
\end{answerdiagram}

\textbf{\gu{તરંગલંબાઇ માપન (સ્લોટેડ લાઇન):}}
\gu{સ્લોટેડ લાઇન પર બે મિનિમા વચ્ચેનું અંતર $d$ માપો. $\lambda_g = 2d$.}
\end{solutionbox}

\begin{mnemonicbox}
\mnemonic{\gu{ફ્રીક્વન્સી પહેલા, તરંગલંબાઇ માપન સાથે}}
\end{mnemonicbox}

\questionmarks{4(a)}{3}{\gu{માઇક્રોવેવ ફ્રીક્વન્સી માટે વેક્યૂમ ટ્યુબની ફ્રીક્વન્સી મર્યાદાઓ જણાવો.}}

\begin{solutionbox}
\textbf{\gu{ફ્રીક્વન્સી મર્યાદાઓ:}}

\begin{itemize}
    \item \textbf{\gu{ટ્રાન્ઝિટ ટાઇમ અસર}}: \gu{ઇલેક્ટ્રોન ટ્રાન્ઝિટ ટાઇમ RF પીરિયડ સાથે સરખાવાય તેટલો થાય છે, જે ફેઝ ડિલે પેદા કરે છે.}
    \item \textbf{\gu{ઇન્ટર-ઇલેક્ટ્રોડ કેપેસિટન્સ}}: \gu{ઉચ્ચ ફ્રીક્વન્સીએ રિએક્ટન્સ ઘટે છે, જે ગેઇન ઘટાડે છે.}
    \item \textbf{\gu{લીડ ઇન્ડક્ટન્સ}}: \gu{પેરાસિટિક ઇન્ડક્ટન્સ લિમિટિંગ પરિબળ બને છે.}
    \item \textbf{\gu{સ્કિન અસર}}: \gu{કરંટ કંડક્ટરની સપાટી પર વહે છે, જે અસરકારક અવરોધ વધારે છે.}
\end{itemize}

\textbf{\gu{પરિબળો:}}
\begin{tabulary}{\linewidth}{|L|L|}
\hline
\textbf{\gu{પરિબળ}} & \textbf{\gu{અસર}} \\ \hline
\gu{ટ્રાન્ઝિટ ટાઇમ} & \gu{ફેઝ વિલંબ ($f < 1/2\pi\tau$)} \\ \hline
\gu{કેપેસિટન્સ} & \gu{ગેઇન $\propto 1/f$} \\ \hline
\gu{લીડ ઇન્ડક્ટન્સ} & \gu{રેઝોનન્સ અસર} \\ \hline
\gu{સ્કિન અસર} & \gu{વધારો અવરોધ} \\ \hline
\end{tabulary}
\end{solutionbox}

\begin{mnemonicbox}
\mnemonic{\gu{ટ્રાન્ઝિટ ટાઇમ પરંપરાગત ટ્યુબ્સને તકલીફ}}
\end{mnemonicbox}

\questionmarks{4(b)}{4}{\gu{IMPATT ડાયોડમાં નેગેટિવ રેઝિસ્ટન્સ અસર સમજાવો.}}

\begin{solutionbox}
\textbf{\gu{IMPATT ડાયોડ:}}

\begin{answerdiagram}{IMPATT Diode Structure}
\begin{tikzpicture}[scale=0.8]
    % Layers
    \draw[fill=red!20] (0,3) rectangle (4,4) node[midway] {P+ (\gu{એનોડ})};
    \draw[fill=blue!10] (0,2) rectangle (4,3) node[midway] {N (\gu{એક્ટિવ})};
    \draw[fill=white] (0,1) rectangle (4,2) node[midway] {I (\gu{બિલ્ટ-ઇન})};
    \draw[fill=green!20] (0,0) rectangle (4,1) node[midway] {N+ (\gu{કેથોડ})};
    
    \draw (2,4) -- (2,4.5) node[above] {\gu{ટર્મિનલ A}};
    \draw (2,0) -- (2,-0.5) node[below] {\gu{ટર્મિનલ B}};
\end{tikzpicture}
\end{answerdiagram}

\textbf{\gu{નેગેટિવ રેઝિસ્ટન્સ મિકેનિઝમ:}}
\begin{enumerate}
    \item \textbf{\gu{ઇમ્પેક્ટ આયોનાઇઝેશન}}: \gu{હાઇ ફીલ્ડ ઇલેક્ટ્રોન-હોલ પેર બનાવે છે ($90^\circ$ ફેઝ શિફ્ટ).}
    \item \textbf{\gu{ટ્રાન્ઝિટ ટાઇમ વિલંબ}}: \gu{કેરિયર ડ્રિફ્ટ રીજનમાંથી પસાર થાય છે (બીજો $90^\circ$ શિફ્ટ).}
\end{enumerate}
\textbf{\gu{કુલ ફેઝ શિફ્ટ}}: $180^\circ$ $\rightarrow$ \gu{નેગેટિવ રેઝિસ્ટન્સ}.
\end{solutionbox}

\begin{mnemonicbox}
\mnemonic{\gu{ઇમ્પેક્ટ આયોનાઇઝેશન, ટ્રાન્ઝિટ ટાઇમ = નેગેટિવ રેઝિસ્ટન્સ}}
\end{mnemonicbox}

\questionmarks{4(c)}{7}{\gu{ટનલ ડાયોડનો સિદ્ધાંત, ટનલિંગ ઘટના અને કોઈપણ એક એપ્લિકેશન સમજાવો.}}

\begin{solutionbox}
\textbf{\gu{સિદ્ધાંત}}: \gu{ટનલ ડાયોડ \keyword{ક્વાન્ટમ મેકેનિકલ ટનલિંગ} અસર પર કાર્ય કરે છે.}

\begin{answerdiagram}{Tunnel Diode Band Diagram (Peak Point)}
\begin{tikzpicture}
    % P-side
    \draw[thick] (0,0) -- (2,0) node[midway, above] {Ev};
    \draw[thick] (0,2) -- (2,2) node[midway, above] {Ec};
    \draw[dashed] (0,0.5) -- (2,0.5) node[right] {Ef};
    \node at (1,1) {\gu{P- સાઈડ}};
    
    % Barrier
    \draw[thick] (2,0) -- (3, -1);
    \draw[thick] (2,2) -- (3, 1);
    
    % N-side
    \draw[thick] (3,-1) -- (5,-1) node[midway, below] {Ev};
    \draw[thick] (3,1) -- (5,1) node[midway, above] {Ec};
    \draw[dashed] (3,0.5) -- (5,0.5) node[right] {Ef}; 
    \node at (4,0) {\gu{N- સાઈડ}};
    
    \draw[->, purple, thick, decorate, decoration={snake, amplitude=0.4mm, segment length=2mm, post length=1mm}] (2.2, 0.4) -- (3.5, 0.4); 
    \node[purple] at (2.8, 1.5) {\gu{ટનલિંગ}};
\end{tikzpicture}
\end{answerdiagram}

\textbf{\gu{લક્ષણો:}}

\begin{answerdiagram}{Tunnel Diode I-V Curve}
\begin{tikzpicture}
    \begin{axis}[
        xlabel={\gu{વોલ્ટેજ (V)}},
        ylabel={\gu{કરંટ (mA)}},
        axis lines=middle,
        xmin=0, xmax=1,
        ymin=0, ymax=10,
        xtick={0.1, 0.5, 0.8},
        xticklabels={$V_p$, $V_v$, $V_f$},
        ytick={2, 8},
        yticklabels={$I_v$, $I_p$},
        width=8cm, height=6cm
    ]
    \addplot[smooth, thick, blue] coordinates {
        (0,0) (0.1,8) (0.3,4) (0.5,2) (0.8,8) (1,12)
    };
    \node[anchor=west] at (axis cs:0.2, 5) {\gu{નેગેટિવ રેઝિસ્ટન્સ}};
    \end{axis}
\end{tikzpicture}
\end{answerdiagram}

\textbf{\gu{એપ્લિકેશન - ઓસિલેટર:}}
\gu{નેગેટિવ રેઝિસ્ટન્સ રીજનમાં ઓપરેટ કરીને ઓસિલેશન બનાવે છે.}
\end{solutionbox}

\begin{mnemonicbox}
\mnemonic{\gu{ટનલ થ્રુ, નેગેટિવ ગ્રો, ઓસિલેટર ફ્લો}}
\end{mnemonicbox}

\orquestionmarks{4(a)}{3}{\gu{માઇક્રોવેવ રેડિએશનને કારણે જોખમો સમજાવો.}}

\begin{solutionbox}
\textbf{\gu{જોખમો:}}

\begin{enumerate}
    \item \textbf{HERP (\gu{પર્સનેલ})}: \gu{ટિશ્યુ હીટિંગ, આંખોને નુકસાન (મોતિયો), જિનેટિક ડેમેજ.}
    \item \textbf{HERO (\gu{ઓર્ડનન્સ})}: \gu{વિસ્ફોટકોનું પ્રીમેચ્યુર ઇગ્નિશન.}
    \item \textbf{HERF (\gu{ફ્યુઅલ})}: \gu{ફ્યુઅલ વેપરનું સળગવું.}
\end{enumerate}

\textbf{\gu{સેફ્ટી લેવલ}}: \gu{$ < 10 \text{ mW/cm}^2$ સુરક્ષિત છે.}
\end{solutionbox}

\begin{mnemonicbox}
\mnemonic{\gu{HERP-HERO-HERF = હેલ્થ-એક્સ્પ્લોસિવ-ફ્યુઅલ રિસ્ક}}
\end{mnemonicbox}

\orquestionmarks{4(b)}{4}{\gu{પેરામેટ્રિક એમ્પ્લિફાયરમાં ડીજનરેટ અને નોન-ડીજનરેટ મોડ સમજાવો.}}

\begin{solutionbox}
\textbf{\gu{પેરામેટ્રિક એમ્પ્લિફાયર મોડ્સ:}}

\begin{enumerate}
    \item \textbf{\gu{નોન-ડીજનરેટ મોડ}}:
    \begin{itemize}
        \item \gu{ફ્રીક્વન્સી: $f_p \neq 2 f_s$. ($f_p = f_s + f_i$).}
        \item \gu{આઇડલર $f_i$ અલગ હોય છે.}
        \item \gu{સારો નોઇઝ ફિગર.}
    \end{itemize}
    \item \textbf{\gu{ડીજનરેટ મોડ}}:
    \begin{itemize}
        \item \gu{ફ્રીક્વન્સી: $f_p = 2 f_s$.}
        \item \gu{આઇડલર અને સિગ્નલ ફ્રીક્વન્સી સમાન હોય છે ($f_i = f_s$).}
        \item \gu{આઉટપુટ પમ્પ ફેઝ પર આધારિત છે.}
    \end{itemize}
\end{enumerate}
\end{solutionbox}

\begin{mnemonicbox}
\mnemonic{\gu{નોન-ડીજનરેટ = નોટ-સિંગલ, ડીજનરેટ = ડબલ્ડ-ફ્રીક્વન્સી}}
\end{mnemonicbox}

\orquestionmarks{4(c)}{7}{\gu{ગન ડાયોડમાં સિદ્ધાંત અને ગન અસર સમજાવો. ગન ડાયોડને ઓસિલેટર તરીકે પણ સમજાવો.}}

\begin{solutionbox}
\textbf{\gu{ગન અસર:}}
\gu{\keyword{ટ્રાન્સફર્ડ ઇલેક્ટ્રોન અસર} પર આધારિત. ઇલેક્ટ્રોન હાઇ-મોબિલિટી વેલી (Central) માંથી લો-મોબિલિટી વેલી (Satellite) માં ટ્રાન્સફર થાય છે.}

\begin{answerdiagram}{Gunn Effect Band Structure}
\begin{tikzpicture}
    \draw[->] (0,0) -- (0,3) node[above] {E};
    \draw[thick] (0,0.5) parabola bend (2,0.5) (4,0.5);
    \node at (2,0.2) {\gu{લોઅર વેલી ($\Gamma$)}};
    
    \draw[thick] (0.5,2) parabola bend (1,1.5) (1.5,2);
    \draw[thick] (2.5,2) parabola bend (3,1.5) (3.5,2);
    \node at (3,2.2) {\gu{અપર વેલી (L)}};
    \draw[<->] (2,0.5) -- (2,1.5) node[midway, right] {$\Delta E = 0.36eV$};
\end{tikzpicture}
\end{answerdiagram}

\textbf{\gu{ડોમેઇન ફોર્મેશન:}}
\gu{થ્રેશોલ્ડ વોલ્ટેજ ઉપર, હાઇ ફીલ્ડ ડોમેઇન કેથોડ પર બને છે અને એનોડ તરફ ડ્રિફ્ટ થાય છે, જે કરંટ પલ્સ પેદા કરે છે.}

\textbf{\gu{ગન ઓસિલેટર:}}
\begin{itemize}
    \item \gu{રેઝોનન્ટ કેવિટીમાં ગન ડાયોડ મૂકીને બનાવાય છે.}
    \item \gu{ફ્રીક્વન્સી $f = v_{domain} / L_{eff}$ અથવા કેવિટી રેઝોનન્સ દ્વારા નિયંત્રિત થાય છે.}
\end{itemize}
\end{solutionbox}

\begin{mnemonicbox}
\mnemonic{\gu{ગન ગેલિયમ-આર્સેનાઇડ દ્વારા ગોઇંગ મેળવે}}
\end{mnemonicbox}

\questionmarks{5(a)}{3}{\gu{બ્લોક ડાયાગ્રામની મદદથી મૂળભૂત રડાર સિસ્ટમના કાર્ય સિદ્ધાંતને સમજાવો.}}

\begin{solutionbox}
\textbf{\gu{રડાર સિદ્ધાંત:}}
\gu{રેડિયો ડિટેક્શન એન્ડ રેન્જિંગ. પલ્સ ટ્રાન્સમિટ કરે છે અને ઇકો રિસીવ કરે છે. રેન્જ $R = (c \times t)/2$.}

\begin{answerdiagram}{Basic Radar Block Diagram}
\begin{tikzpicture}[auto, node distance=1.5cm]
    \node [gtu block] (tim) {\gu{ટાઇમર}};
    \node [gtu block, below=of tim] (mod) {\gu{મોડ્યુલેટર}};
    \node [gtu block, right=of mod] (tx) {\gu{ટ્રાન્સમિટર}};
    \node [gtu block, right=of tx] (dup) {\gu{ડુપ્લેક્સર}};
    \node [gtu block, right=of dup] (ant) {\gu{એન્ટેના}};
    \node [gtu block, below=of dup] (rx) {\gu{રિસીવર}};
    \node [gtu block, left=of rx] (disp) {\gu{ડિસ્પ્લે}};
    
    \draw [gtu arrow] (tim) -- (mod);
    \draw [gtu arrow] (mod) -- (tx);
    \draw [gtu arrow] (tx) -- (dup);
    \draw [gtu arrow] (dup) -- (ant);
    \draw [gtu arrow] (ant) -- (dup);
    \draw [gtu arrow] (dup) -- (rx);
    \draw [gtu arrow] (rx) -- (disp);
    \draw [gtu arrow] (tim) -| (disp);
\end{tikzpicture}
\end{answerdiagram}
\end{solutionbox}

\begin{mnemonicbox}
\mnemonic{\gu{રડાર રાઉન્ડ-ટ્રિપ રિફ્લેક્શન દ્વારા રેન્જ માપે}}
\end{mnemonicbox}

\questionmarks{5(b)}{4}{\gu{યોગ્ય આકૃતિની મદદથી A-સ્કોપ ડિસ્પ્લે પદ્ધતિ સમજાવો.}}

\begin{solutionbox}
\textbf{\gu{A-સ્કોપ ડિસ્પ્લે:}}
\gu{એમ્પ્લિટ્યુડ (Y-અક્ષ) વિરુદ્ધ સમય/રેન્જ (X-અક્ષ) દર્શાવે છે.}

\begin{answerdiagram}{A-Scope Presentation}
\begin{tikzpicture}
    \draw[->, thick] (0,0) -- (6,0) node[right] {\gu{સમય}};
    \draw[->, thick] (0,0) -- (0,3) node[above] {\gu{એમ્પ્લિટ્યુડ}};
    
    \draw[thick] (0.2,0) -- (0.2,2.5) -- (0.4,0); 
    \node[above] at (0.3,2.5) {\gu{Tx પલ્સ}};
    
    \draw[decorate, decoration={random steps, segment length=2pt, amplitude=1pt}] (0.4,0) -- (1.5,0); 
    \node[above] at (1,0.5) {\gu{ક્લટર}};
    
    \draw[thick] (4,0) -- (4,1.5) -- (4.2,0); 
    \node[above] at (4.1,1.5) {\gu{ટાર્ગેટ}};
    
    \draw[decorate, decoration={random steps, segment length=2pt, amplitude=1pt}] (1.5,0) -- (6,0);
\end{tikzpicture}
\end{answerdiagram}
\end{solutionbox}

\begin{mnemonicbox}
\mnemonic{\gu{A-સ્કોપ ટાઇમ એક્સિસ સાથે એમ્પ્લિટ્યુડ દર્શાવે}}
\end{mnemonicbox}

\questionmarks{5(c)}{7}{\gu{ડોપ્લર અસર અને બ્લોક ડાયાગ્રામની મદદથી MTI (મૂવિંગ ટાર્ગેટ ઇન્ડિકેટર) રડાર સિસ્ટમની કામગીરી સમજાવો.}}

\begin{solutionbox}
\textbf{\gu{ડોપ્લર અસર}}: \gu{સાપેક્ષ ગતિને કારણે ફ્રીક્વન્સી શિફ્ટ. $f_d = 2v_r/\lambda$.}

\textbf{\gu{MTI રડાર}}: \gu{સ્થિર ક્લટરને દૂર કરવા અને મૂવિંગ ટાર્ગેટને જોવા માટે ડોપ્લર શિફ્ટનો ઉપયોગ કરે છે.}

\begin{answerdiagram}{MTI Radar Block Diagram}
\begin{tikzpicture}[scale=0.8, transform shape, node distance=1.5cm]
    \node [gtu block] (stalo) {STALO};
    \node [gtu block, below=of stalo] (coho) {COHO};
    \node [gtu block, right=of stalo] (mix1) {\gu{મિક્સર 1}};
    
    \node [gtu block, above=of mix1] (tx) {Tx};
    \node [gtu block, above=of tx] (ant) {Ant};
    \node [gtu block, right=of mix1] (phase) {\gu{ફેઝ ડિટેક્ટર}};
    \node [gtu block, right=of phase] (delay) {\gu{ડિલે લાઇન}};
    \node [gtu block, right=of delay] (sub) {\gu{સબટ્રેક્ટર}};
    \node [gtu output, right=of sub] (disp) {\gu{ડિસ્પ્લે}};
    
    \draw [gtu arrow] (stalo) -- (mix1);
    \draw [gtu arrow] (coho) -- (phase);
    \draw [gtu arrow] (mix1) -- (phase);
    \draw [gtu arrow] (phase) -- (delay);
    \draw [gtu arrow] (phase) to[bend left] (sub);
    \draw [gtu arrow] (delay) -- (sub);
    \draw [gtu arrow] (sub) -- (disp);
\end{tikzpicture}
\end{answerdiagram}

\textbf{\gu{કાર્યપ્રણાલી}}:
\begin{itemize}
    \item \gu{ડિલે લાઇન એક પલ્સ (PRT) જેટલો વિલંબ આપે છે.}
    \item \gu{સબટ્રેક્ટર બે પલ્સની બાદબાકી કરે છે. સ્થિર ટાર્ગેટ માટે બંને પલ્સ સમાન હોય છે, તેથી બાદબાકી શૂન્ય થાય છે (ક્લટર કેન્સલેશન).}
\end{itemize}
\end{solutionbox}

\begin{mnemonicbox}
\mnemonic{\gu{MTI ડોપ્લર ડિફરન્સ દ્વારા ટાર્ગેટ આઇડેન્ટિફાઇ કરે}}
\end{mnemonicbox}

\orquestionmarks{5(a)}{3}{\gu{વ્યાખ્યા આપો: a) બ્લાઇન્ડ સ્પીડ, અને b) MUR}}

\begin{solutionbox}
\textbf{\gu{વ્યાખ્યાઓ:}}

\begin{itemize}
    \item \textbf{\gu{બ્લાઇન્ડ સ્પીડ}}: \gu{ટાર્ગેટની એવી સ્પીડ કે જ્યાં ડોપ્લર શિફ્ટ PRF ના ઇન્ટીજર ગુણાંક હોય. રડાર તેને સ્થિર સમજે છે.}
    \[ v_b = \frac{n \lambda f_r}{2} \]
    \item \textbf{\gu{MUR (મહત્તમ અનએમ્બિગ્યુઅસ રેન્જ)}}: \gu{આગળનો પલ્સ મોકલતા પહેલા ઇકો આવવો જોઈએ તે મહત્તમ રેન્જ.}
    \[ R_{max} = \frac{c}{2 f_r} \]
\end{itemize}
\end{solutionbox}

\begin{mnemonicbox}
\mnemonic{\gu{બ્લાઇન્ડ સ્પીડ બ્લોક કરે, MUR મેક્સિમમ માપે}}
\end{mnemonicbox}

\orquestionmarks{5(b)}{4}{\gu{મહત્તમ રડાર રેન્જને અસર કરતા પરિબળો સમજાવો.}}

\begin{solutionbox}
\textbf{\gu{પરિબળો:}}
\[ R_{max} = \left[ \frac{P_t G^2 \lambda^2 \sigma}{(4\pi)^3 S_{min}} \right]^{1/4} \]

\begin{enumerate}
    \item \textbf{\gu{ટ્રાન્સમિટર પાવર ($P_t$)}}: \gu{$R \propto P_t^{1/4}$. પાવર વધારતા રેન્જ થોડી વધે છે.}
    \item \textbf{\gu{એન્ટેના ગેઇન ($G$)}}: \gu{$R \propto \sqrt{G}$. ગેઇન વધારવું વધુ અસરકારક છે.}
    \item \textbf{\gu{તરંગલંબાઇ ($\lambda$)}}: \gu{$R \propto \sqrt{\lambda}$.}
    \item \textbf{\gu{ટાર્ગેટ ક્રોસ સેક્શન ($\sigma$)}}: \gu{મોટા ટાર્ગેટ દૂરથી દેખાય છે.}
\end{enumerate}
\end{solutionbox}

\begin{mnemonicbox}
\mnemonic{\gu{પાવર-ગેઇન-લેમ્બડા-સિગ્મા રેન્જ નક્કી કરે}}
\end{mnemonicbox}

\orquestionmarks{5(c)}{7}{\gu{પલ્સ્ડ રડાર અને CW ડોપ્લર રડારની સરખામણી કરો.}}

\begin{solutionbox}
\textbf{\gu{સરખામણી:}}

\begin{tabulary}{\linewidth}{|L|L|L|}
\hline
\textbf{\gu{પરિમાપ}} & \textbf{\gu{પલ્સ્ડ રડાર}} & \textbf{\gu{CW ડોપ્લર રડાર}} \\ \hline
\gu{ટ્રાન્સમિશન} & \gu{પલ્સ (તુટક)} & \gu{સતત (કન્ટીન્યુઅસ)} \\ \hline
\gu{રેન્જ} & \gu{માપી શકાય છે} & \gu{માપી શકાતી નથી} \\ \hline
\gu{વેલોસિટી} & \gu{મુશ્કેલ} & \gu{સરળ (ડોપ્લરથી)} \\ \hline
\gu{એન્ટેના} & \gu{એક (ડુપ્લેક્સર સાથે)} & \gu{બે (Tx અને Rx માટે)} \\ \hline
\gu{પાવર} & \gu{પીક પાવર વધારે} & \gu{ઓછો પાવર} \\ \hline
\gu{ઉપયોગ} & \gu{સર્વેલન્સ, નેવિગેશન} & \gu{સ્પીડ ગન, પ્રોક્સિમિટી સેન્સર} \\ \hline
\end{tabulary}
\end{solutionbox}

\begin{mnemonicbox}
\mnemonic{\gu{પલ્સ્ડ પોઝિશન આપે, CW કન્ટિન્યુઅસ-વેલોસિટી આપે}}
\end{mnemonicbox}

\end{document}
