\documentclass[10pt,a4paper]{article}

% content/resources/templates/preamble.tex
\usepackage[margin=0.6in]{geometry}
\author{Milav Dabgar}
\usepackage{amsmath,amssymb,amsthm}
\usepackage{booktabs}
\usepackage{multirow}
\usepackage{xcolor}
\usepackage{tcolorbox}
\tcbuselibrary{breakable,skins}
\usepackage[colorlinks=true,linkcolor=blue]{hyperref}
\usepackage{titlesec}
\usepackage{enumitem}
\usepackage{tikz}
\usepackage{pgfplots}
\usepackage{circuitikz}
\usepackage[version=4]{mhchem}
\usepackage{longtable}
\usepackage{array}
\usepackage{float}
\usepackage{caption}
\usepackage{listings}

\lstset{
  basicstyle=\small\ttfamily,
  breaklines=true,
  breakatwhitespace=false,
  postbreak=\mbox{\textcolor{red}{$\hookrightarrow$}\space},
  float=false,
  numbers=left,
  numberstyle=\tiny\color{gray},
  numbersep=10pt,
  xleftmargin=2em,
  keywordstyle=\color{blue},
  commentstyle=\color{green!60!black},
  stringstyle=\color{purple},
  backgroundcolor=\color{gray!5},
  showstringspaces=false,
  tabsize=2,
  captionpos=b,
  keepspaces=true,
  columns=flexible
}

\pgfplotsset{compat=1.18}
\usetikzlibrary{shapes,arrows,positioning,calc,patterns,decorations.pathmorphing,decorations.markings,arrows.meta}

% Color scheme
\definecolor{headcolor}{RGB}{0,102,204}
\definecolor{keycolor}{RGB}{220,20,60}
\definecolor{solutioncolor}{RGB}{34,139,34}
\definecolor{mnemoniccolor}{RGB}{148,0,211}
\definecolor{codecolor}{RGB}{0,0,100}

% Spacing
\setlength{\parskip}{3pt}
\setlist[itemize]{nosep}
\setlist[enumerate]{nosep}

% Title formatting
\titleformat{\section}{\Large\bfseries\color{headcolor}}{\thesection}{1em}{}
\titleformat{\subsection}{\large\bfseries\color{headcolor}}{\thesubsection}{1em}{}

% Pandoc tightlist compatibility
\providecommand{\tightlist}{%
  \setlength{\itemsep}{0pt}\setlength{\parskip}{0pt}}

% Pandoc longtable compatibility
\newcounter{none}
\def\thenone{}


% content/resources/templates/english-boxes.tex
% This file is currently empty - it exists to maintain consistency with the import structure.
% Add custom environments here if needed in the future.


\begin{document}

\begin{center}
{\Huge\bfseries\color{headcolor} Subject Name Solutions}\\[5pt]
{\LARGE 4351103 -- Winter 2024}\\[3pt]
{\large Semester 1 Study Material}\\[3pt]
{\normalsize\textit{Detailed Solutions and Explanations}}
\end{center}

\vspace{10pt}

\subsection*{Question 1(a) [3 marks]}\label{q1a}

\textbf{Give comparison between transmission line and waveguide.}

\begin{solutionbox}

{\def\LTcaptype{none} % do not increment counter
\begin{longtable}[]{@{}
  >{\raggedright\arraybackslash}p{(\linewidth - 4\tabcolsep) * \real{0.2750}}
  >{\raggedright\arraybackslash}p{(\linewidth - 4\tabcolsep) * \real{0.4500}}
  >{\raggedright\arraybackslash}p{(\linewidth - 4\tabcolsep) * \real{0.2750}}@{}}
\toprule\noalign{}
\begin{minipage}[b]{\linewidth}\raggedright
Parameter
\end{minipage} & \begin{minipage}[b]{\linewidth}\raggedright
Transmission Line
\end{minipage} & \begin{minipage}[b]{\linewidth}\raggedright
Waveguide
\end{minipage} \\
\midrule\noalign{}
\endhead
\bottomrule\noalign{}
\endlastfoot
\textbf{Frequency Range} & Low to medium frequencies & High frequencies
(above 1 GHz) \\
\textbf{Structure} & Two or more conductors & Single hollow conductor \\
\textbf{Propagation Mode} & TEM mode & TE and TM modes \\
\textbf{Power Handling} & Limited power capacity & High power handling
capability \\
\textbf{Losses} & Higher losses at high frequencies & Lower losses at
microwave frequencies \\
\end{longtable}
}

\end{solutionbox}
\begin{mnemonicbox}
``WAVES Travel Better'' (Waveguides - Advanced
Versions Enabling Superior Transmission)

\end{mnemonicbox}
\begin{center}\rule{0.5\linewidth}{0.5pt}\end{center}

\subsection*{Question 1(b) [4 marks]}\label{q1b}

\textbf{Define the following terms: (1) Lossless Line (2) VSWR (3) STUB
(4) Reflection coefficient}

\begin{solutionbox}

\begin{itemize}
\item
  \textbf{Lossless Line}: A transmission line with zero resistance and
  conductance, having no power loss during signal transmission.
\item
  \textbf{VSWR (Voltage Standing Wave Ratio)}: Ratio of maximum voltage
  to minimum voltage on a transmission line, indicating impedance
  mismatch.
\item
  \textbf{STUB}: Short length of transmission line connected to main
  line for impedance matching purposes.
\item
  \textbf{Reflection Coefficient}: Ratio of reflected wave amplitude to
  incident wave amplitude at any point on transmission line.
\end{itemize}

\end{solutionbox}
\begin{mnemonicbox}
``Light Volumes Stay Reflected''
(Lossless-VSWR-Stub-Reflection)

\end{mnemonicbox}
\begin{center}\rule{0.5\linewidth}{0.5pt}\end{center}

\subsection*{Question 1(c) [7 marks]}\label{q1c}

\textbf{Explain isolator and circulator with the help of sketch.}

\begin{solutionbox}

\begin{center}
\textbf{Mermaid Diagram (Code)}
\begin{verbatim}
{Shaded}
{Highlighting}[]
graph LR
    A[Port 1] {-{-}{} B[Isolator] {-}{-}{} C[Port 2]}
    B {-.X.{-}{} A}
    
    D[Port 1] {-{-}{} E((Circulator)) {-}{-}{} F[Port 2]}
    F {-{-}{} G[Port 3] {-}{-}{} D}
{Highlighting}
{Shaded}
\end{verbatim}
\end{center}

\textbf{Isolator:}

\begin{itemize}
\tightlist
\item
  \textbf{Function}: Allows signal flow in one direction only
\item
  \textbf{Construction}: Uses ferrite material with magnetic bias
\item
  \textbf{Applications}: Protects sources from reflections
\end{itemize}

\textbf{Circulator:}

\begin{itemize}
\tightlist
\item
  \textbf{Function}: Routes signals in circular pattern between three or
  four ports
\item
  \textbf{Construction}: Y-junction with ferrite material
\item
  \textbf{Applications}: Duplexers in radar systems
\end{itemize}

\end{solutionbox}
\begin{mnemonicbox}
``Isolated Circuits Flow Forward''
(Isolator-Circulator-Forward-Flow)

\end{mnemonicbox}
\begin{center}\rule{0.5\linewidth}{0.5pt}\end{center}

\subsection*{Question 1(c OR) [7
marks]}\label{question-1c-or-7-marks}

\textbf{What is dominant mode in a waveguide? What will be the cutoff
wavelength for dominant mode, in a rectangular waveguide whose breadth
is 10 cm? For a 2.5 GHz signal propagated through it calculate guide
wavelength, group velocity and phase velocity and Z_{0}.}

\begin{solutionbox}

\textbf{Dominant Mode}: Lowest order mode that can propagate in a
waveguide. For rectangular waveguide, it's TE_{1}_{0} mode.

\textbf{Given Data:}

\begin{itemize}
\tightlist
\item
  Breadth (a) = 10 cm = 0.1 m
\item
  Frequency (f) = 2.5 GHz = 2.5 \times 10^{9} Hz
\item
  c = 3 \times 10^{8} m/s
\end{itemize}

\textbf{Calculations:}

{\def\LTcaptype{none} % do not increment counter
\begin{longtable}[]{@{}lll@{}}
\toprule\noalign{}
Parameter & Formula & Value \\
\midrule\noalign{}
\endhead
\bottomrule\noalign{}
\endlastfoot
\textbf{Cutoff Wavelength} & λc = 2a & λc = 2 \times 0.1 = 0.2 m \\
\textbf{Free Space Wavelength} & λ_{0} = c/f & λ_{0} = 0.12 m \\
\textbf{Guide Wavelength} & λg = λ_{0}/\sqrt(1-(λ_{0}/λc)^{2}) & λg = 0.133 m \\
\textbf{Group Velocity} & vg = c\sqrt(1-(λ_{0}/λc)^{2}) & vg = 2.7 \times 10^{8} m/s \\
\textbf{Phase Velocity} & vp = c/\sqrt(1-(λ_{0}/λc)^{2}) & vp = 3.33 \times 10^{8} m/s \\
\end{longtable}
}

\end{solutionbox}
\begin{mnemonicbox}
``Dominant Modes Calculate Guide Parameters''

\end{mnemonicbox}
\begin{center}\rule{0.5\linewidth}{0.5pt}\end{center}

\subsection*{Question 2(a) [3 marks]}\label{q2a}

\textbf{What is single stub impedance matching, and how does it work?}

\begin{solutionbox}

\textbf{Single Stub Matching}: Technique using one short-circuited or
open-circuited stub connected in parallel to transmission line for
impedance matching.

\textbf{Working Principle:}

\begin{itemize}
\tightlist
\item
  \textbf{Stub acts as reactive element} (inductive or capacitive)
\item
  \textbf{Cancels reactive component} of load impedance
\item
  \textbf{Transforms impedance} to characteristic impedance
\end{itemize}

\end{solutionbox}
\begin{mnemonicbox}
``Single Stubs Transform Reactance''
(Single-Stub-Transform-Reactive)

\end{mnemonicbox}
\begin{center}\rule{0.5\linewidth}{0.5pt}\end{center}

\subsection*{Question 2(b) [4 marks]}\label{q2b}

\textbf{Differentiate between rectangular and circular waveguide any
three points.}

\begin{solutionbox}

{\def\LTcaptype{none} % do not increment counter
\begin{longtable}[]{@{}
  >{\raggedright\arraybackslash}p{(\linewidth - 4\tabcolsep) * \real{0.2115}}
  >{\raggedright\arraybackslash}p{(\linewidth - 4\tabcolsep) * \real{0.4231}}
  >{\raggedright\arraybackslash}p{(\linewidth - 4\tabcolsep) * \real{0.3654}}@{}}
\toprule\noalign{}
\begin{minipage}[b]{\linewidth}\raggedright
Parameter
\end{minipage} & \begin{minipage}[b]{\linewidth}\raggedright
Rectangular Waveguide
\end{minipage} & \begin{minipage}[b]{\linewidth}\raggedright
Circular Waveguide
\end{minipage} \\
\midrule\noalign{}
\endhead
\bottomrule\noalign{}
\endlastfoot
\textbf{Cross-section} & Rectangular shape & Circular shape \\
\textbf{Dominant Mode} & TE_{1}_{0} mode & TE_{1}_{1} mode \\
\textbf{Field Pattern} & Simple field distribution & Complex field
distribution \\
\textbf{Manufacturing} & Easy to manufacture & Difficult to
manufacture \\
\end{longtable}
}

\end{solutionbox}
\begin{mnemonicbox}
``Rectangles Dominate Ten'' vs ``Circles Dominate
Eleven''

\end{mnemonicbox}
\begin{center}\rule{0.5\linewidth}{0.5pt}\end{center}

\subsection*{Question 2(c) [7 marks]}\label{q2c}

\textbf{Explain the construction and working of Hybrid Ring with
diagram.}

\begin{solutionbox}

\begin{center}
\textbf{Mermaid Diagram (Code)}
\begin{verbatim}
{Shaded}
{Highlighting}[]
graph TD
    A[Port 1] {-{-}{-} B[Hybrid Ring]}
    C[Port 2] {-{-}{-} B}
    D[Port 3] {-{-}{-} B}
    E[Port 4] {-{-}{-} B}
    B {-.{-}{} F[λ/4 sections]}
{Highlighting}
{Shaded}
\end{verbatim}
\end{center}

\textbf{Construction:}

\begin{itemize}
\tightlist
\item
  \textbf{Ring structure} with four ports
\item
  \textbf{Circumference} = 1.5λ (one and half wavelengths)
\item
  \textbf{Adjacent ports} separated by λ/4
\item
  \textbf{Opposite ports} separated by 3λ/4
\end{itemize}

\textbf{Working:}

\begin{itemize}
\tightlist
\item
  \textbf{Power division}: Input at one port divides equally between two
  adjacent ports
\item
  \textbf{Isolation}: Opposite port receives no power
\item
  \textbf{Phase relationship}: 180^\circ phase difference between output
  ports
\end{itemize}

\textbf{Applications:}

\begin{itemize}
\tightlist
\item
  \textbf{Balanced mixers}
\item
  \textbf{Power combiners/dividers}
\item
  \textbf{Antenna feeds}
\end{itemize}

\end{solutionbox}
\begin{mnemonicbox}
``Hybrid Rings Divide Power Equally''

\end{mnemonicbox}
\begin{center}\rule{0.5\linewidth}{0.5pt}\end{center}

\subsection*{Question 2(a OR) [3
marks]}\label{question-2a-or-3-marks}

\textbf{What is Microwave? List out any four applications of microwave.}

\begin{solutionbox}

\textbf{Microwave}: Electromagnetic waves with frequency range from 1
GHz to 300 GHz.

\textbf{Applications:}

\begin{itemize}
\tightlist
\item
  \textbf{Radar systems} for detection and ranging
\item
  \textbf{Satellite communication} for long-distance transmission
\item
  \textbf{Microwave ovens} for heating food
\item
  \textbf{Mobile communication} (cellular networks)
\end{itemize}

\end{solutionbox}
\begin{mnemonicbox}
``Microwaves Reach Space Mobile''
(Microwave-Radar-Satellite-Mobile)

\end{mnemonicbox}
\begin{center}\rule{0.5\linewidth}{0.5pt}\end{center}

\subsection*{Question 2(b OR) [4
marks]}\label{question-2b-or-4-marks}

\textbf{Write short note on cavity resonator.}

\begin{solutionbox}

\textbf{Cavity Resonator}: Closed metallic structure that confines
electromagnetic energy at specific resonant frequencies.

\textbf{Construction:}

\begin{itemize}
\tightlist
\item
  \textbf{Metallic enclosure} of specific dimensions
\item
  \textbf{High Q factor} (low losses)
\item
  \textbf{Resonant frequency} depends on cavity dimensions
\end{itemize}

\textbf{Types:}

\begin{itemize}
\tightlist
\item
  \textbf{Rectangular cavity}
\item
  \textbf{Cylindrical cavity}
\item
  \textbf{Spherical cavity}
\end{itemize}

\textbf{Applications:}

\begin{itemize}
\tightlist
\item
  \textbf{Frequency meters}
\item
  \textbf{Oscillator circuits}
\item
  \textbf{Filter circuits}
\end{itemize}

\end{solutionbox}
\begin{mnemonicbox}
``Cavities Resonate High Quality''
(Cavity-Resonant-High-Q)

\end{mnemonicbox}
\begin{center}\rule{0.5\linewidth}{0.5pt}\end{center}

\subsection*{Question 2(c OR) [7
marks]}\label{question-2c-or-7-marks}

\textbf{Explain MAGIC TEE with the help of sketch and how it works as an
isolator?}

\begin{solutionbox}

\begin{center}
\textbf{Mermaid Diagram (Code)}
\begin{verbatim}
{Shaded}
{Highlighting}[]
graph TD
    A[E{-arm] {-}{-}{-} B[Magic Tee Junction]}
    C[H{-arm] {-}{-}{-} B}
    D[Arm 1] {-{-}{-} B}
    E[Arm 2] {-{-}{-} B}
{Highlighting}
{Shaded}
\end{verbatim}
\end{center}

\textbf{Magic Tee Construction:}

\begin{itemize}
\tightlist
\item
  \textbf{E-plane Tee} and \textbf{H-plane Tee} combined
\item
  \textbf{Four ports}: E-arm, H-arm, and two side arms
\item
  \textbf{E-arm} perpendicular to H-arm
\end{itemize}

\textbf{Working as Isolator:}

\begin{itemize}
\tightlist
\item
  \textbf{Signal at E-arm} divides equally between side arms (in-phase)
\item
  \textbf{Signal at H-arm} divides equally between side arms
  (out-of-phase)
\item
  \textbf{Isolation} between E-arm and H-arm
\item
  \textbf{No coupling} between perpendicular arms
\end{itemize}

\textbf{Properties:}

\begin{itemize}
\tightlist
\item
  \textbf{Matched at all ports}
\item
  \textbf{Reciprocal device}
\item
  \textbf{Power division and isolation}
\end{itemize}

\end{solutionbox}
\begin{mnemonicbox}
``Magic Isolates Perpendicular Arms''

\end{mnemonicbox}
\begin{center}\rule{0.5\linewidth}{0.5pt}\end{center}

\subsection*{Question 3(a) [3 marks]}\label{q3a}

\textbf{Describe the working principle of MASER.}

\begin{solutionbox}

\textbf{MASER (Microwave Amplification by Stimulated Emission of
Radiation):}

\begin{itemize}
\tightlist
\item
  \textbf{Population inversion} created in active medium
\item
  \textbf{Stimulated emission} produces coherent microwaves
\item
  \textbf{Amplification} occurs through energy level transitions
\end{itemize}

\textbf{Working Principle:}

\begin{itemize}
\tightlist
\item
  \textbf{Atoms excited} to higher energy levels
\item
  \textbf{Stimulated photons} trigger emission
\item
  \textbf{Coherent amplification} of microwave signals
\end{itemize}

\end{solutionbox}
\begin{mnemonicbox}
``Microwaves Amplify Stimulated Emission Radiation''

\end{mnemonicbox}
\begin{center}\rule{0.5\linewidth}{0.5pt}\end{center}

\subsection*{Question 3(b) [4 marks]}\label{q3b}

\textbf{List four microwave diodes and explain any one.}

\begin{solutionbox}

\textbf{Four Microwave Diodes:}

\begin{enumerate}
\tightlist
\item
  \textbf{GUNN Diode}
\item
  \textbf{IMPATT Diode}
\item
  \textbf{TRAPATT Diode}
\item
  \textbf{PIN Diode}
\end{enumerate}

\textbf{GUNN Diode Explanation:}

\begin{itemize}
\tightlist
\item
  \textbf{Principle}: Transferred electron effect in GaAs
\item
  \textbf{Construction}: N-type GaAs with ohmic contacts
\item
  \textbf{Operation}: Negative resistance at microwave frequencies
\item
  \textbf{Applications}: Oscillators, amplifiers
\end{itemize}

\textbf{VI Characteristic:}

\begin{verbatim}
    I \^{}
      |    /
      |   /
      |  /\_\_\_
      | /    {}
      |/      {\_\_\_}
      +{-{-}{-}{-}{-}{-}{-}{-}{-}{-} V}
      Negative resistance region
\end{verbatim}

\end{solutionbox}
\begin{mnemonicbox}
``GUNN Generates Negative Resistance''

\end{mnemonicbox}
\begin{center}\rule{0.5\linewidth}{0.5pt}\end{center}

\subsection*{Question 3(c) [7 marks]}\label{q3c}

\textbf{Write a detailed explanation of the Magnetron Oscillator,
covering its construction, working principle, and applications?}

\begin{solutionbox}

\begin{center}
\textbf{Mermaid Diagram (Code)}
\begin{verbatim}
{Shaded}
{Highlighting}[]
graph LR
    A[Cathode] {-{-}{} B[Interaction Space]}
    B {-{-}{} C[Anode with Cavities]}
    C {-{-}{} D[Output Coupling]}
    E[Magnetic Field] {-.{-}{} B}
{Highlighting}
{Shaded}
\end{verbatim}
\end{center}

\textbf{Construction:}

\begin{itemize}
\tightlist
\item
  \textbf{Cylindrical cathode} at center
\item
  \textbf{Anode with resonant cavities} surrounding cathode
\item
  \textbf{Strong magnetic field} perpendicular to electric field
\item
  \textbf{Output coupling} through waveguide
\end{itemize}

\textbf{Working Principle:}

\begin{itemize}
\tightlist
\item
  \textbf{Electrons emitted} from heated cathode
\item
  \textbf{Cycloid motion} due to crossed E and B fields
\item
  \textbf{Bunching effect} creates electron clouds
\item
  \textbf{Energy transfer} from electrons to RF field
\item
  \textbf{Oscillation} at cavity resonant frequency
\end{itemize}

\textbf{Applications:}

\begin{itemize}
\tightlist
\item
  \textbf{Radar transmitters}
\item
  \textbf{Microwave ovens}
\item
  \textbf{Industrial heating}
\item
  \textbf{Medical diathermy}
\end{itemize}

\end{solutionbox}
\begin{mnemonicbox}
``Magnetrons Make Microwave Oscillations''

\end{mnemonicbox}
\begin{center}\rule{0.5\linewidth}{0.5pt}\end{center}

\subsection*{Question 3(a OR) [3
marks]}\label{question-3a-or-3-marks}

\textbf{Describe the working of RUBY MASER.}

\begin{solutionbox}

\textbf{Ruby MASER Working:}

\begin{itemize}
\tightlist
\item
  \textbf{Ruby crystal} (Al_{2}O_{3} with Cr^{3}^{+} ions) as active medium
\item
  \textbf{Three energy levels} in chromium ions
\item
  \textbf{Pump frequency} creates population inversion
\item
  \textbf{Signal amplification} at 2.9 GHz
\end{itemize}

\textbf{Process:}

\begin{itemize}
\tightlist
\item
  \textbf{Optical pumping} excites electrons to higher level
\item
  \textbf{Stimulated emission} produces coherent microwaves
\item
  \textbf{Low noise amplification} achieved
\end{itemize}

\end{solutionbox}
\begin{mnemonicbox}
``Ruby Radiates Amplified Microwaves''

\end{mnemonicbox}
\begin{center}\rule{0.5\linewidth}{0.5pt}\end{center}

\subsection*{Question 3(b OR) [4
marks]}\label{question-3b-or-4-marks}

\textbf{Draw and explain the VI characteristic of Gun diode}

\begin{solutionbox}

\begin{verbatim}
    I \^{}
      |      
      |    B /
      |     /
      |    /
      | A /
      |  /
      | /\_\_\_\_\_ C
      |/      {}
      |        {\_\_\_\_\_ D}
      +{-{-}{-}{-}{-}{-}{-}{-}{-}{-}{-} V}
    Valley    Peak
    Current   Current
\end{verbatim}

\textbf{VI Characteristic Explanation:}

\begin{itemize}
\tightlist
\item
  \textbf{Region OA}: Ohmic region (positive resistance)
\item
  \textbf{Region AB}: Negative resistance region
\item
  \textbf{Region BC}: Valley current region
\item
  \textbf{Region CD}: Saturation region
\end{itemize}

\textbf{Key Points:}

\begin{itemize}
\tightlist
\item
  \textbf{Peak voltage}: Maximum voltage before negative resistance
\item
  \textbf{Valley current}: Minimum current in negative resistance region
\item
  \textbf{Negative resistance}: Current decreases with increasing
  voltage
\end{itemize}

\end{solutionbox}
\begin{mnemonicbox}
``Valley Peak Negative Resistance''

\end{mnemonicbox}
\begin{center}\rule{0.5\linewidth}{0.5pt}\end{center}

\subsection*{Question 3(c OR) [7
marks]}\label{question-3c-or-7-marks}

\textbf{Explain ``frequency measurement method'' as well as
``attenuation measurement method'' at microwave frequency.}

\begin{solutionbox}

\textbf{Frequency Measurement Methods:}

{\def\LTcaptype{none} % do not increment counter
\begin{longtable}[]{@{}lll@{}}
\toprule\noalign{}
Method & Principle & Accuracy \\
\midrule\noalign{}
\endhead
\bottomrule\noalign{}
\endlastfoot
\textbf{Cavity Wavemeter} & Resonant cavity tuning & High \\
\textbf{Direct Reading Meter} & Frequency counter & Very High \\
\textbf{Heterodyne Method} & Beat frequency technique & Medium \\
\end{longtable}
}

\textbf{Attenuation Measurement Methods:}

{\def\LTcaptype{none} % do not increment counter
\begin{longtable}[]{@{}
  >{\raggedright\arraybackslash}p{(\linewidth - 4\tabcolsep) * \real{0.2353}}
  >{\raggedright\arraybackslash}p{(\linewidth - 4\tabcolsep) * \real{0.3824}}
  >{\raggedright\arraybackslash}p{(\linewidth - 4\tabcolsep) * \real{0.3824}}@{}}
\toprule\noalign{}
\begin{minipage}[b]{\linewidth}\raggedright
Method
\end{minipage} & \begin{minipage}[b]{\linewidth}\raggedright
Description
\end{minipage} & \begin{minipage}[b]{\linewidth}\raggedright
Application
\end{minipage} \\
\midrule\noalign{}
\endhead
\bottomrule\noalign{}
\endlastfoot
\textbf{Substitution Method} & Replace attenuator with calibrated
attenuator & Precision measurement \\
\textbf{Power Ratio Method} & Compare input and output power & General
purpose \\
\textbf{RF Bridge Method} & Balance bridge circuit & Laboratory use \\
\end{longtable}
}

\textbf{Setup for Measurement:}

\begin{itemize}
\tightlist
\item
  \textbf{Signal generator} provides test signal
\item
  \textbf{Calibrated attenuator} for reference
\item
  \textbf{Power meter} measures signal levels
\item
  \textbf{VSWR meter} monitors impedance matching
\end{itemize}

\end{solutionbox}
\begin{mnemonicbox}
``Frequency Attenuation Measured Precisely''

\end{mnemonicbox}
\begin{center}\rule{0.5\linewidth}{0.5pt}\end{center}

\subsection*{Question 4(a) [3 marks]}\label{q4a}

\textbf{Explain working of P-i-N diode.}

\begin{solutionbox}

\textbf{P-i-N Diode Structure:}

\begin{itemize}
\tightlist
\item
  \textbf{P-type region} (heavily doped)
\item
  \textbf{Intrinsic region} (undoped, high resistance)
\item
  \textbf{N-type region} (heavily doped)
\end{itemize}

\textbf{Working:}

\begin{itemize}
\tightlist
\item
  \textbf{Forward bias}: Low resistance, acts as conductor
\item
  \textbf{Reverse bias}: High resistance, acts as insulator
\item
  \textbf{RF switching}: Fast switching due to charge storage
\end{itemize}

\textbf{Applications:}

\begin{itemize}
\tightlist
\item
  \textbf{RF switches}
\item
  \textbf{Attenuators}
\item
  \textbf{Phase shifters}
\end{itemize}

\end{solutionbox}
\begin{mnemonicbox}
``PIN Provides Instant Switching''

\end{mnemonicbox}
\begin{center}\rule{0.5\linewidth}{0.5pt}\end{center}

\subsection*{Question 4(b) [4 marks]}\label{q4b}

\textbf{Explain π mode oscillations for magnetron.}

\begin{solutionbox}

\textbf{π Mode Oscillation:}

\begin{itemize}
\tightlist
\item
  \textbf{Adjacent cavities} oscillate 180^\circ out of phase
\item
  \textbf{Electron bunching} synchronized with RF field
\item
  \textbf{Maximum power transfer} from electrons to RF
\item
  \textbf{Stable oscillation} at designed frequency
\end{itemize}

\textbf{Characteristics:}

\begin{itemize}
\tightlist
\item
  \textbf{Phase difference}: π radians between adjacent cavities
\item
  \textbf{Frequency}: Determined by cavity dimensions
\item
  \textbf{Efficiency}: Highest among all modes
\item
  \textbf{Stability}: Most stable oscillation mode
\end{itemize}

\textbf{Mode Chart:}

\begin{verbatim}
Cavity: 1  2  3  4  5  6  7  8
Phase:  0  π  0  π  0  π  0  π
\end{verbatim}

\end{solutionbox}
\begin{mnemonicbox}
``Pi Mode Produces Maximum Power''

\end{mnemonicbox}
\begin{center}\rule{0.5\linewidth}{0.5pt}\end{center}

\subsection*{Question 4(c) [7 marks]}\label{q4c}

\textbf{Explain the construction and working of two cavity klystron
amplifiers with necessary diagram.}

\begin{solutionbox}

\begin{center}
\textbf{Mermaid Diagram (Code)}
\begin{verbatim}
{Shaded}
{Highlighting}[]
graph LR
    A[Electron Gun] {-{-}{} B[Input Cavity]}
    B {-{-}{} C[Drift Space]}
    C {-{-}{} D[Output Cavity]}
    D {-{-}{} E[Collector]}
    F[Input Signal] {-{-}{} B}
    D {-{-}{} G[Output Signal]}
{Highlighting}
{Shaded}
\end{verbatim}
\end{center}

\textbf{Construction:}

\begin{itemize}
\tightlist
\item
  \textbf{Electron gun} produces electron beam
\item
  \textbf{Input cavity} (buncher) modulates electron beam
\item
  \textbf{Drift space} allows velocity modulation
\item
  \textbf{Output cavity} (catcher) extracts RF energy
\item
  \textbf{Collector} collects spent electrons
\end{itemize}

\textbf{Working Principle:}

\begin{itemize}
\tightlist
\item
  \textbf{Velocity modulation} in input cavity
\item
  \textbf{Electron bunching} in drift space
\item
  \textbf{Density modulation} creates current variation
\item
  \textbf{Energy extraction} in output cavity
\item
  \textbf{Amplification} achieved through beam-field interaction
\end{itemize}

\textbf{Key Parameters:}

\begin{itemize}
\tightlist
\item
  \textbf{Beam voltage}: Determines electron velocity
\item
  \textbf{Cavity tuning}: Sets operating frequency
\item
  \textbf{Drift space length}: Controls bunching effectiveness
\end{itemize}

\textbf{Applications:}

\begin{itemize}
\tightlist
\item
  \textbf{Radar transmitters}
\item
  \textbf{Satellite communication}
\item
  \textbf{Linear accelerators}
\end{itemize}

\end{solutionbox}
\begin{mnemonicbox}
``Klystrons Amplify Through Bunching''

\end{mnemonicbox}
\begin{center}\rule{0.5\linewidth}{0.5pt}\end{center}

\subsection*{Question 4(a OR) [3
marks]}\label{question-4a-or-3-marks}

\textbf{Explain parametric amplifier.}

\begin{solutionbox}

\textbf{Parametric Amplifier:}

\begin{itemize}
\tightlist
\item
  \textbf{Variable reactance} device using varactor diode
\item
  \textbf{Pump frequency} modulates diode capacitance
\item
  \textbf{Energy transfer} from pump to signal
\item
  \textbf{Low noise amplification} achieved
\end{itemize}

\textbf{Working:}

\begin{itemize}
\tightlist
\item
  \textbf{Pump power} varies diode reactance
\item
  \textbf{Signal mixing} produces sum and difference frequencies
\item
  \textbf{Idler frequency} fp = fs + fi
\item
  \textbf{Power gain} through nonlinear mixing
\end{itemize}

\textbf{Advantages:}

\begin{itemize}
\tightlist
\item
  \textbf{Very low noise figure}
\item
  \textbf{High gain possible}
\item
  \textbf{Wide bandwidth}
\end{itemize}

\end{solutionbox}
\begin{mnemonicbox}
``Parametric Amplifiers Pump Low Noise''

\end{mnemonicbox}
\begin{center}\rule{0.5\linewidth}{0.5pt}\end{center}

\subsection*{Question 4(b OR) [4
marks]}\label{question-4b-or-4-marks}

\textbf{Draw and explain schematic diagram of travelling wave tube with
necessary notation}

\begin{solutionbox}

\begin{center}
\textbf{Mermaid Diagram (Code)}
\begin{verbatim}
{Shaded}
{Highlighting}[]
graph LR
    A[Electron Gun] {-{-}{} B[Input]}
    B {-{-}{} C[Helix]}
    C {-{-}{} D[Output]}
    D {-{-}{} E[Collector]}
    F[Attenuator] {-.{-}{} C}
    G[Focusing System] {-.{-}{} C}
{Highlighting}
{Shaded}
\end{verbatim}
\end{center}

\textbf{Components:}

\begin{itemize}
\tightlist
\item
  \textbf{Electron gun}: Produces electron beam
\item
  \textbf{Helix}: Slow-wave structure
\item
  \textbf{Attenuator}: Prevents oscillation
\item
  \textbf{Collector}: Collects electrons
\item
  \textbf{Focusing system}: Maintains beam alignment
\end{itemize}

\textbf{Working:}

\begin{itemize}
\tightlist
\item
  \textbf{Electron beam} travels through helix center
\item
  \textbf{RF signal} propagates along helix
\item
  \textbf{Synchronism} between beam and RF wave
\item
  \textbf{Energy transfer} from beam to RF
\item
  \textbf{Continuous amplification} along helix length
\end{itemize}

\end{solutionbox}
\begin{mnemonicbox}
``TWT Travels With Waves''

\end{mnemonicbox}
\begin{center}\rule{0.5\linewidth}{0.5pt}\end{center}

\subsection*{Question 4(c OR) [7
marks]}\label{question-4c-or-7-marks}

\textbf{Explain the working principle of a reflex klystron in detail
with suitable diagram.}

\begin{solutionbox}

\begin{center}
\textbf{Mermaid Diagram (Code)}
\begin{verbatim}
{Shaded}
{Highlighting}[]
graph LR
    A[Cathode] {-{-}{} B[Resonant Cavity]}
    B {-{-}{} C[Drift Space]}
    C {-{-}{} D[Repeller]}
    D {-.{-}{} C}
    C {-.{-}{} B}
    B {-{-}{} E[Output]}
{Highlighting}
{Shaded}
\end{verbatim}
\end{center}

\textbf{Construction:}

\begin{itemize}
\tightlist
\item
  \textbf{Single resonant cavity} acts as buncher and catcher
\item
  \textbf{Repeller electrode} reflects electron beam
\item
  \textbf{Drift space} allows velocity modulation
\item
  \textbf{Output coupling} extracts RF power
\end{itemize}

\textbf{Working Principle:}

\textbf{Applegate Diagram:}

\begin{verbatim}
Distance \^{}
         |    \_\_\_
         |   /   {  Bunched electrons}
         |  /     {}
         | /       {}
         |/         {\_\_\_}
         +{-{-}{-}{-}{-}{-}{-}{-}{-}{-}{-}{-}{-} Time}
         Transit time variation
\end{verbatim}

\textbf{Process:}

\begin{enumerate}
\tightlist
\item
  \textbf{Electrons enter cavity} and get velocity modulated
\item
  \textbf{Electrons drift} toward repeller
\item
  \textbf{Repeller reflects} electrons back to cavity
\item
  \textbf{Transit time} determines bunching phase
\item
  \textbf{Bunched electrons} deliver energy to cavity
\item
  \textbf{Oscillation maintained} through feedback
\end{enumerate}

\textbf{Frequency Tuning:}

\begin{itemize}
\tightlist
\item
  \textbf{Repeller voltage} controls transit time
\item
  \textbf{Cavity tuning} sets center frequency
\item
  \textbf{Electronic tuning} possible
\end{itemize}

\textbf{Applications:}

\begin{itemize}
\tightlist
\item
  \textbf{Local oscillators}
\item
  \textbf{Frequency meters}
\item
  \textbf{Microwave sources}
\end{itemize}

\end{solutionbox}
\begin{mnemonicbox}
``Reflex Returns Electron Bunches''

\end{mnemonicbox}
\begin{center}\rule{0.5\linewidth}{0.5pt}\end{center}

\subsection*{Question 5(a) [3 marks]}\label{q5a}

\textbf{``PIN diode acts as a switch and VARACTOR diode acts as a
variable capacitor'' explain.}

\begin{solutionbox}

\textbf{PIN Diode as Switch:}

\begin{itemize}
\tightlist
\item
  \textbf{Forward bias}: Low resistance (\textasciitilde1Ω), switch ON
\item
  \textbf{Reverse bias}: High resistance (\textasciitilde10kΩ), switch
  OFF
\item
  \textbf{Fast switching} due to charge storage in I-region
\item
  \textbf{RF isolation} in OFF state
\end{itemize}

\textbf{VARACTOR Diode as Variable Capacitor:}

\begin{itemize}
\tightlist
\item
  \textbf{Reverse bias voltage} controls junction capacitance
\item
  \textbf{Capacitance decreases} with increasing reverse voltage
\item
  \textbf{Voltage-controlled reactance} for tuning circuits
\item
  \textbf{Electronic tuning} without mechanical adjustment
\end{itemize}

\end{solutionbox}
\begin{mnemonicbox}
``PIN Switches, VARACTOR Varies''

\end{mnemonicbox}
\begin{center}\rule{0.5\linewidth}{0.5pt}\end{center}

\subsection*{Question 5(b) [4 marks]}\label{q5b}

\textbf{List the display methods used in RADAR and explain any one.}

\begin{solutionbox}

\textbf{RADAR Display Methods:}

\begin{enumerate}
\tightlist
\item
  \textbf{A-Scope Display}
\item
  \textbf{PPI (Plan Position Indicator)}
\item
  \textbf{B-Scope Display}
\item
  \textbf{RHI (Range Height Indicator)}
\end{enumerate}

\textbf{PPI Display Explanation:}

\begin{itemize}
\tightlist
\item
  \textbf{Circular display} showing target positions
\item
  \textbf{Center represents} radar location
\item
  \textbf{Radial distance} indicates target range
\item
  \textbf{Angular position} shows target bearing
\item
  \textbf{Rotating sweep} synchronized with antenna rotation
\end{itemize}

\textbf{Features:}

\begin{itemize}
\tightlist
\item
  \textbf{Real-time display} of target positions
\item
  \textbf{Range and bearing} information
\item
  \textbf{Multiple target tracking}
\item
  \textbf{Clutter suppression}
\end{itemize}

\end{solutionbox}
\begin{mnemonicbox}
``PPI Pictures Position Indicators''

\end{mnemonicbox}
\begin{center}\rule{0.5\linewidth}{0.5pt}\end{center}

\subsection*{Question 5(c) [7 marks]}\label{q5c}

\textbf{What is radar? List out the different types of radar systems?
Explain any One of radar in detail?}

\begin{solutionbox}

\textbf{RADAR (Radio Detection And Ranging):} System using radio waves
to detect objects and determine their range, velocity, and
characteristics.

\textbf{Types of RADAR Systems:}

{\def\LTcaptype{none} % do not increment counter
\begin{longtable}[]{@{}lll@{}}
\toprule\noalign{}
Type & Application & Frequency Band \\
\midrule\noalign{}
\endhead
\bottomrule\noalign{}
\endlastfoot
\textbf{Pulse Radar} & Air traffic control & L, S, C bands \\
\textbf{CW Doppler Radar} & Speed measurement & X, K, Ka bands \\
\textbf{MTI Radar} & Moving target detection & S, C bands \\
\textbf{SAR Radar} & Ground mapping & L, C, X bands \\
\end{longtable}
}

\textbf{Pulse Radar Detailed Explanation:}

\begin{center}
\textbf{Mermaid Diagram (Code)}
\begin{verbatim}
{Shaded}
{Highlighting}[]
graph LR
    A[Transmitter] {-{-}{} B[Duplexer]}
    B {-{-}{} C[Antenna]}
    C {-{-}{} D[Target]}
    D {-{-}{} C}
    C {-{-}{} B}
    B {-{-}{} E[Receiver]}
    E {-{-}{} F[Display]}
    G[Timer] {-{-}{} A}
    G {-{-}{} F}
{Highlighting}
{Shaded}
\end{verbatim}
\end{center}

\textbf{Working:}

\begin{itemize}
\tightlist
\item
  \textbf{Transmits short pulses} of RF energy
\item
  \textbf{Receives echoes} from targets
\item
  \textbf{Measures time delay} for range calculation
\item
  \textbf{Processes signals} for display
\end{itemize}

\textbf{Range Equation:} R = (c \times t)/2

Where:

\begin{itemize}
\tightlist
\item
  R = Range to target
\item
  c = Speed of light
\item
  t = Time delay
\end{itemize}

\textbf{Applications:}

\begin{itemize}
\tightlist
\item
  \textbf{Air traffic control}
\item
  \textbf{Weather monitoring}
\item
  \textbf{Military surveillance}
\item
  \textbf{Navigation aids}
\end{itemize}

\end{solutionbox}
\begin{mnemonicbox}
``Radar Ranges Radio Waves''

\end{mnemonicbox}
\begin{center}\rule{0.5\linewidth}{0.5pt}\end{center}

\subsection*{Question 5(a OR) [3
marks]}\label{question-5a-or-3-marks}

\textbf{Describe the operation of TRAPATT diode with diagram.}

\begin{solutionbox}

\begin{verbatim}
    I \^{}
      |      |{}
      |      | {}
      |      |  {}
      |      |   {\_\_ Trapped plasma}
      |      |     { avalanche}
      |\_\_\_\_\_\_|\_\_\_\_\_\_{\_\_\_}
      +{-{-}{-}{-}{-}{-}{-}{-}{-}{-}{-}{-}{-}{-}{-}{-} V}
      Breakdown voltage
\end{verbatim}

\textbf{TRAPATT Operation:}

\begin{itemize}
\tightlist
\item
  \textbf{TRApped Plasma Avalanche Triggered Transit} diode
\item
  \textbf{High field region} creates avalanche breakdown
\item
  \textbf{Plasma formation} traps charge carriers
\item
  \textbf{Transit time effects} create negative resistance
\item
  \textbf{Oscillation frequency} determined by transit time
\end{itemize}

\textbf{Applications:}

\begin{itemize}
\tightlist
\item
  \textbf{High power oscillators}
\item
  \textbf{Radar transmitters}
\item
  \textbf{Communication systems}
\end{itemize}

\end{solutionbox}
\begin{mnemonicbox}
``TRAPATT Traps Plasma Avalanche''

\end{mnemonicbox}
\begin{center}\rule{0.5\linewidth}{0.5pt}\end{center}

\subsection*{Question 5(b OR) [4
marks]}\label{question-5b-or-4-marks}

\textbf{Compare RADAR with SONAR.}

\begin{solutionbox}

{\def\LTcaptype{none} % do not increment counter
\begin{longtable}[]{@{}lll@{}}
\toprule\noalign{}
Parameter & RADAR & SONAR \\
\midrule\noalign{}
\endhead
\bottomrule\noalign{}
\endlastfoot
\textbf{Wave Type} & Electromagnetic waves & Sound waves \\
\textbf{Medium} & Air/vacuum & Water/liquid \\
\textbf{Frequency} & GHz range & kHz range \\
\textbf{Speed} & 3 \times 10^{8} m/s & 1500 m/s in water \\
\textbf{Range} & Very long range & Limited by absorption \\
\textbf{Applications} & Air/space detection & Underwater detection \\
\end{longtable}
}

\textbf{Similarities:}

\begin{itemize}
\tightlist
\item
  \textbf{Echo principle} for detection
\item
  \textbf{Range measurement} using time delay
\item
  \textbf{Doppler effect} for velocity measurement
\end{itemize}

\end{solutionbox}
\begin{mnemonicbox}
``RADAR Radiates, SONAR Sounds''

\end{mnemonicbox}
\begin{center}\rule{0.5\linewidth}{0.5pt}\end{center}

\subsection*{Question 5(c OR) [7
marks]}\label{question-5c-or-7-marks}

\textbf{Obtain the equation for maximum radar range.}

\begin{solutionbox}

\textbf{RADAR Range Equation Derivation:}

\textbf{Power Transmitted:} Pt

\textbf{Power Density at Target:} Pd = Pt/(4πR^{2})

\textbf{Power Intercepted by Target:} Pi = Pd \times σ = (Pt \times σ)/(4πR^{2})

\textbf{Power Returned to Radar:} Pr = Pi/(4πR^{2}) = (Pt \times σ)/(4πR^{2})^{2}

\textbf{Power Received:} Pr = (Pt \times G^{2} \times λ^{2} \times σ)/((4π)^{3} \times R^{4})

\textbf{Maximum Range Equation:}

\begin{verbatim}
Rmax = ^{4}\sqrt[(Pt \times G^{2} \times λ^{2} \times σ)/((4π)^{3} \times Prmin)]
\end{verbatim}

\textbf{Where:}

\begin{itemize}
\tightlist
\item
  Pt = Transmitted power
\item
  G = Antenna gain\\
\item
  λ = Wavelength
\item
  σ = Radar cross section
\item
  Prmin = Minimum detectable signal
\item
  R = Range
\end{itemize}

\textbf{Factors Affecting Range:}

\begin{itemize}
\tightlist
\item
  \textbf{Transmitted power} (increases range)
\item
  \textbf{Antenna gain} (increases range)
\item
  \textbf{Target cross-section} (increases range)
\item
  \textbf{Frequency} (affects propagation)
\item
  \textbf{Receiver sensitivity} (affects minimum signal)
\end{itemize}

\textbf{Practical Considerations:}

\begin{itemize}
\tightlist
\item
  \textbf{Atmospheric losses}
\item
  \textbf{Ground reflections}
\item
  \textbf{Noise limitations}
\item
  \textbf{Clutter effects}
\end{itemize}

\end{solutionbox}
\begin{mnemonicbox}
``Power Gain Lambda Sigma Range''

\end{mnemonicbox}
\begin{center}\rule{0.5\linewidth}{0.5pt}\end{center}


\end{document}
