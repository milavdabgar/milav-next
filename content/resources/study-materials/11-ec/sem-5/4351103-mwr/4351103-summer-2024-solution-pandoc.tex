\documentclass[10pt,a4paper]{article}

% content/resources/templates/preamble.tex
\usepackage[margin=0.6in]{geometry}
\author{Milav Dabgar}
\usepackage{amsmath,amssymb,amsthm}
\usepackage{booktabs}
\usepackage{multirow}
\usepackage{xcolor}
\usepackage{tcolorbox}
\tcbuselibrary{breakable,skins}
\usepackage[colorlinks=true,linkcolor=blue]{hyperref}
\usepackage{titlesec}
\usepackage{enumitem}
\usepackage{tikz}
\usepackage{pgfplots}
\usepackage{circuitikz}
\usepackage[version=4]{mhchem}
\usepackage{longtable}
\usepackage{array}
\usepackage{float}
\usepackage{caption}
\usepackage{listings}

\lstset{
  basicstyle=\small\ttfamily,
  breaklines=true,
  breakatwhitespace=false,
  postbreak=\mbox{\textcolor{red}{$\hookrightarrow$}\space},
  float=false,
  numbers=left,
  numberstyle=\tiny\color{gray},
  numbersep=10pt,
  xleftmargin=2em,
  keywordstyle=\color{blue},
  commentstyle=\color{green!60!black},
  stringstyle=\color{purple},
  backgroundcolor=\color{gray!5},
  showstringspaces=false,
  tabsize=2,
  captionpos=b,
  keepspaces=true,
  columns=flexible
}

\pgfplotsset{compat=1.18}
\usetikzlibrary{shapes,arrows,positioning,calc,patterns,decorations.pathmorphing,decorations.markings,arrows.meta}

% Color scheme
\definecolor{headcolor}{RGB}{0,102,204}
\definecolor{keycolor}{RGB}{220,20,60}
\definecolor{solutioncolor}{RGB}{34,139,34}
\definecolor{mnemoniccolor}{RGB}{148,0,211}
\definecolor{codecolor}{RGB}{0,0,100}

% Spacing
\setlength{\parskip}{3pt}
\setlist[itemize]{nosep}
\setlist[enumerate]{nosep}

% Title formatting
\titleformat{\section}{\Large\bfseries\color{headcolor}}{\thesection}{1em}{}
\titleformat{\subsection}{\large\bfseries\color{headcolor}}{\thesubsection}{1em}{}

% Pandoc tightlist compatibility
\providecommand{\tightlist}{%
  \setlength{\itemsep}{0pt}\setlength{\parskip}{0pt}}

% Pandoc longtable compatibility
\newcounter{none}
\def\thenone{}


% content/resources/templates/english-boxes.tex
% This file is currently empty - it exists to maintain consistency with the import structure.
% Add custom environments here if needed in the future.


\begin{document}

\begin{center}
{\Huge\bfseries\color{headcolor} Subject Name Solutions}\\[5pt]
{\LARGE 4351103 -- Summer 2024}\\[3pt]
{\large Semester 1 Study Material}\\[3pt]
{\normalsize\textit{Detailed Solutions and Explanations}}
\end{center}

\vspace{10pt}

\subsection*{Question 1(a) [3 marks]}\label{q1a}

\textbf{List different microwave bands with their frequency range.}

\begin{solutionbox}


{\def\LTcaptype{none} % do not increment counter
\vspace{-5pt}
\captionof{table}{Microwave Frequency Bands}
\vspace{-10pt}
\begin{longtable}[]{@{}lll@{}}
\toprule\noalign{}
Band & Frequency Range & Wavelength \\
\midrule\noalign{}
\endhead
\bottomrule\noalign{}
\endlastfoot
\textbf{L Band} & 1-2 GHz & 30-15 cm \\
\textbf{S Band} & 2-4 GHz & 15-7.5 cm \\
\textbf{C Band} & 4-8 GHz & 7.5-3.75 cm \\
\textbf{X Band} & 8-12 GHz & 3.75-2.5 cm \\
\textbf{Ku Band} & 12-18 GHz & 2.5-1.67 cm \\
\textbf{K Band} & 18-27 GHz & 1.67-1.11 cm \\
\textbf{Ka Band} & 27-40 GHz & 1.11-0.75 cm \\
\end{longtable}
}

\end{solutionbox}
\begin{mnemonicbox}
``Large Ships Can eXamine Kindly Using Knowledge
Always''

\end{mnemonicbox}
\begin{center}\rule{0.5\linewidth}{0.5pt}\end{center}

\subsection*{Question 1(b) [4 marks]}\label{q1b}

\textbf{Draw the general equivalent circuit of the transmission line.
Write the equation for characteristic impedance for a lossless line.}

\begin{solutionbox}

\textbf{Transmission Line Equivalent Circuit:}

\begin{verbatim}
    R      L
  {-{-}{-}{-}▬▬▬▬{-}{-}{-}{-}}
 |             |
 |      C      | G
 |    {-{-}{-}{-}{-}    |}
 |             |
  {-{-}{-}{-}{-}{-}{-}{-}{-}{-}{-}{-}{-}}
      dx
\end{verbatim}

\textbf{Circuit Elements:}

\begin{itemize}
\tightlist
\item
  \textbf{R}: Series resistance per unit length
\item
  \textbf{L}: Series inductance per unit length\\
\item
  \textbf{C}: Shunt capacitance per unit length
\item
  \textbf{G}: Shunt conductance per unit length
\end{itemize}

\textbf{For Lossless Line (R = 0, G = 0):}

\textbf{Characteristic Impedance:} Z_{0} = \sqrt(L/C)

\textbf{Key Points:}

\begin{itemize}
\tightlist
\item
  \textbf{Lossless condition}: No power loss during transmission
\item
  \textbf{Impedance matching}: Z_{0} determines reflection behavior
\end{itemize}

\end{solutionbox}
\begin{mnemonicbox}
``Lossless Lines Love Constant Impedance''

\end{mnemonicbox}
\begin{center}\rule{0.5\linewidth}{0.5pt}\end{center}

\subsection*{Question 1(c) [7 marks]}\label{q1c}

\textbf{Explain the impedance matching process using a single stub.}

\begin{solutionbox}

\textbf{Single Stub Matching Process:}

\begin{center}
\textbf{Mermaid Diagram (Code)}
\begin{verbatim}
{Shaded}
{Highlighting}[]
graph LR
    A[Source] {-{-}{} B[Main Line]}
    B {-{-}{} C[Stub Connection Point]}
    C {-{-}{} D[Load]}
    C {-{-}{} E[Short Stub]}
{Highlighting}
{Shaded}
\end{verbatim}
\end{center}

\textbf{Matching Steps:}

{\def\LTcaptype{none} % do not increment counter
\begin{longtable}[]{@{}lll@{}}
\toprule\noalign{}
Step & Process & Purpose \\
\midrule\noalign{}
\endhead
\bottomrule\noalign{}
\endlastfoot
\textbf{1} & Calculate load admittance & Find Y\_L = 1/Z\_L \\
\textbf{2} & Move toward generator & Find point where G = G_{0} \\
\textbf{3} & Add stub susceptance & Cancel reactive part \\
\textbf{4} & Achieve matching & Y\_total = Y_{0} \\
\end{longtable}
}

\textbf{Design Equations:}

\begin{itemize}
\tightlist
\item
  \textbf{Distance to stub:} d = (λ/2π) \times tan^{-}^{1}(\sqrt(R\_L/R_{0}))
\item
  \textbf{Stub length:} l = (λ/2π) \times tan^{-}^{1}(B\_stub/Y_{0})
\end{itemize}

\textbf{Applications:}

\begin{itemize}
\tightlist
\item
  \textbf{Antenna matching}
\item
  \textbf{Amplifier input/output}
\item
  \textbf{Filter design}
\end{itemize}

\end{solutionbox}
\begin{mnemonicbox}
``Single Stubs Stop Standing Waves Successfully''

\end{mnemonicbox}
\begin{center}\rule{0.5\linewidth}{0.5pt}\end{center}

\subsection*{Question 1(c) OR [7
marks]}\label{q1c}

\textbf{Compare rectangular and circular waveguides.}

\begin{solutionbox}

\textbf{Comparison Table:}

{\def\LTcaptype{none} % do not increment counter
\begin{longtable}[]{@{}
  >{\raggedright\arraybackslash}p{(\linewidth - 4\tabcolsep) * \real{0.2115}}
  >{\raggedright\arraybackslash}p{(\linewidth - 4\tabcolsep) * \real{0.4231}}
  >{\raggedright\arraybackslash}p{(\linewidth - 4\tabcolsep) * \real{0.3654}}@{}}
\toprule\noalign{}
\begin{minipage}[b]{\linewidth}\raggedright
Parameter
\end{minipage} & \begin{minipage}[b]{\linewidth}\raggedright
Rectangular Waveguide
\end{minipage} & \begin{minipage}[b]{\linewidth}\raggedright
Circular Waveguide
\end{minipage} \\
\midrule\noalign{}
\endhead
\bottomrule\noalign{}
\endlastfoot
\textbf{Shape} & Rectangular cross-section & Circular cross-section \\
\textbf{Dominant Mode} & TE_{1}_{0} & TE_{1}_{1} \\
\textbf{Cutoff Frequency} & fc = c/(2a) for TE_{1}_{0} & fc = 1.841c/(2πa) for
TE_{1}_{1} \\
\textbf{Power Handling} & Lower & Higher \\
\textbf{Manufacturing} & Easy & Difficult \\
\textbf{Mode Separation} & Good & Poor \\
\textbf{Applications} & Radar, microwave ovens & Satellite
communication \\
\end{longtable}
}

\textbf{Key Advantages:}

\begin{itemize}
\tightlist
\item
  \textbf{Rectangular}: Better mode control, easier fabrication
\item
  \textbf{Circular}: Higher power capacity, rotating polarization
\end{itemize}

\end{solutionbox}
\begin{mnemonicbox}
``Rectangular is Regular, Circular Carries Current''

\end{mnemonicbox}
\begin{center}\rule{0.5\linewidth}{0.5pt}\end{center}

\subsection*{Question 2(a) [3 marks]}\label{q2a}

\textbf{Define group velocity and phase velocity in relation to them.}

\begin{solutionbox}

\textbf{Velocity Definitions:}

{\def\LTcaptype{none} % do not increment counter
\begin{longtable}[]{@{}
  >{\raggedright\arraybackslash}p{(\linewidth - 4\tabcolsep) * \real{0.3571}}
  >{\raggedright\arraybackslash}p{(\linewidth - 4\tabcolsep) * \real{0.2143}}
  >{\raggedright\arraybackslash}p{(\linewidth - 4\tabcolsep) * \real{0.4286}}@{}}
\toprule\noalign{}
\begin{minipage}[b]{\linewidth}\raggedright
Velocity Type
\end{minipage} & \begin{minipage}[b]{\linewidth}\raggedright
Formula
\end{minipage} & \begin{minipage}[b]{\linewidth}\raggedright
Physical Meaning
\end{minipage} \\
\midrule\noalign{}
\endhead
\bottomrule\noalign{}
\endlastfoot
\textbf{Phase Velocity} & v_{p} = ω/β = c/\sqrt(1-(fc/f)^{2}) & Speed of constant
phase \\
\textbf{Group Velocity} & v_{m} = dω/dβ = c\sqrt(1-(fc/f)^{2}) & Speed of signal
energy \\
\end{longtable}
}

\textbf{Relationship:} v_{p} \times v_{m} = c^{2}

\textbf{Key Points:}

\begin{itemize}
\tightlist
\item
  \textbf{Phase velocity}: Always \textgreater{} c (speed of light)
\item
  \textbf{Group velocity}: Always \textless{} c
\item
  \textbf{Signal travels}: At group velocity
\end{itemize}

\end{solutionbox}
\begin{mnemonicbox}
``Phase is Fast, Group Carries Message''

\end{mnemonicbox}
\begin{center}\rule{0.5\linewidth}{0.5pt}\end{center}

\subsection*{Question 2(b) [4 marks]}\label{q2b}

\textbf{Describe the principles and workings of the Directional
coupler.}

\begin{solutionbox}

\textbf{Directional Coupler Principle:}

\begin{center}
\textbf{Mermaid Diagram (Code)}
\begin{verbatim}
{Shaded}
{Highlighting}[]
graph TD
    A[Port 1 {- Input] {-}{-}{} B[Main Line]}
    B {-{-}{} C[Port 2 {-} Through]}
    B {-{-}{} D[Port 3 {-} Coupled]}
    E[Port 4 {- Isolated] {-}{-}{} F[Terminated]}
{Highlighting}
{Shaded}
\end{verbatim}
\end{center}

\textbf{Working Principle:}

\begin{itemize}
\tightlist
\item
  \textbf{Electromagnetic coupling} between two transmission lines
\item
  \textbf{Power division} based on coupling factor
\item
  \textbf{Directional sensitivity} to wave direction
\end{itemize}

\textbf{Key Parameters:}

\begin{itemize}
\tightlist
\item
  \textbf{Coupling Factor}: C = 10 log(P_{1}/P_{3}) dB
\item
  \textbf{Directivity}: D = 10 log(P_{3}/P_{4}) dB
\item
  \textbf{Insertion Loss}: IL = 10 log(P_{1}/P_{2}) dB
\end{itemize}

\end{solutionbox}
\begin{mnemonicbox}
``Directional Couplers Divide Power Precisely''

\end{mnemonicbox}
\begin{center}\rule{0.5\linewidth}{0.5pt}\end{center}

\subsection*{Question 2(c) [7 marks]}\label{q2c}

\textbf{Explain Magic TEE with construction, operation and application.}

\begin{solutionbox}

\textbf{Magic TEE Construction:}

\begin{verbatim}
         E{-Arm (Port 3)}
              |
              |
    Port 1{-{-}{-}{-}+{-}{-}{-}{-}Port 2}
              |
              |
         H{-Arm (Port 4)}
\end{verbatim}

\textbf{Operating Principles:}

{\def\LTcaptype{none} % do not increment counter
\begin{longtable}[]{@{}lll@{}}
\toprule\noalign{}
Port & Function & Field Pattern \\
\midrule\noalign{}
\endhead
\bottomrule\noalign{}
\endlastfoot
\textbf{Port 1 \& 2} & Collinear ports & Symmetric \\
\textbf{Port 3 (E-Arm)} & E-plane port & Electric field coupling \\
\textbf{Port 4 (H-Arm)} & H-plane port & Magnetic field coupling \\
\end{longtable}
}

\textbf{Scattering Properties:}

\begin{itemize}
\tightlist
\item
  \textbf{Isolation}: Port 3 \leftrightarrow Port 4
\item
  \textbf{Power division}: Equal split when matched
\item
  \textbf{Phase relationships}: 0^\circ and 180^\circ
\end{itemize}

\textbf{Applications:}

\begin{itemize}
\tightlist
\item
  \textbf{Mixers and modulators}
\item
  \textbf{Power combiners}
\item
  \textbf{Impedance bridges}
\item
  \textbf{Antenna feeds}
\end{itemize}

\end{solutionbox}
\begin{mnemonicbox}
``Magic TEE Creates Perfect Isolation''

\end{mnemonicbox}
\begin{center}\rule{0.5\linewidth}{0.5pt}\end{center}

\subsection*{Question 2(a) OR [3
marks]}\label{q2a}

\textbf{Draw TE_{1}_{0}, TE_{2}_{0} modes for rectangular waveguide.}

\begin{solutionbox}

\textbf{TE_{1}_{0} Mode (Dominant Mode):}

\begin{verbatim}
  a
┌─────────────┐
│      ↑      │ b
│   E  ↑  E   │
│      ↑      │
└─────────────┘
  Field Lines
\end{verbatim}

\textbf{TE_{2}_{0} Mode:}

\begin{verbatim}
  a
┌─────────────┐
│  ↑     ↓    │ b
│  ↑  E  ↓  E │
│  ↑     ↓    │
└─────────────┘
  Two Half{-Waves}
\end{verbatim}

\textbf{Mode Characteristics:}

\begin{itemize}
\tightlist
\item
  \textbf{TE_{1}_{0}}: One half-wave variation in x-direction
\item
  \textbf{TE_{2}_{0}}: Two half-wave variations in x-direction
\item
  \textbf{Field patterns}: Electric field perpendicular to propagation
\end{itemize}

\end{solutionbox}
\begin{mnemonicbox}
``TE modes have Electric Transverse''

\end{mnemonicbox}
\begin{center}\rule{0.5\linewidth}{0.5pt}\end{center}

\subsection*{Question 2(b) OR [4
marks]}\label{q2b}

\textbf{Describe the Hybrid Ring with a necessary sketch.}

\begin{solutionbox}

\textbf{Hybrid Ring Structure:}

\begin{center}
\textbf{Mermaid Diagram (Code)}
\begin{verbatim}
{Shaded}
{Highlighting}[]
graph TD
    A[Port 1] {-{-}{-} B[Ring Structure]}
    C[Port 2] {-{-}{-} B}
    D[Port 3] {-{-}{-} B}
    E[Port 4] {-{-}{-} B}
    B {-{-}{-} F[3λ/2 circumference]}
{Highlighting}
{Shaded}
\end{verbatim}
\end{center}

\textbf{Operating Principle:}

\begin{itemize}
\tightlist
\item
  \textbf{Ring circumference}: 3λ/2
\item
  \textbf{Port spacing}: λ/4 apart
\item
  \textbf{Power division}: Equal split between adjacent ports
\end{itemize}

\textbf{Key Features:}

\begin{itemize}
\tightlist
\item
  \textbf{Isolation}: Between opposite ports
\item
  \textbf{Phase relationships}: 0^\circ and 180^\circ
\item
  \textbf{Impedance}: Matched at all ports
\end{itemize}

\end{solutionbox}
\begin{mnemonicbox}
``Hybrid Rings Handle Half-wavelengths''

\end{mnemonicbox}
\begin{center}\rule{0.5\linewidth}{0.5pt}\end{center}

\subsection*{Question 2(c) OR [7
marks]}\label{q2c}

\textbf{Explain the Isolator with principles, construction and
operation.}

\begin{solutionbox}

\textbf{Isolator Principle:}

\begin{center}
\textbf{Mermaid Diagram (Code)}
\begin{verbatim}
{Shaded}
{Highlighting}[]
graph LR
    A[Input] {-{-}{} B[Ferrite Material]}
    B {-{-}{} C[Output]}
    C {-.{-}{}|Blocked| B}
    D[Magnetic Field] {-{-}{} B}
{Highlighting}
{Shaded}
\end{verbatim}
\end{center}

\textbf{Construction Elements:}

{\def\LTcaptype{none} % do not increment counter
\begin{longtable}[]{@{}lll@{}}
\toprule\noalign{}
Component & Function & Material \\
\midrule\noalign{}
\endhead
\bottomrule\noalign{}
\endlastfoot
\textbf{Ferrite} & Non-reciprocal medium & Yttrium Iron Garnet \\
\textbf{Magnet} & Bias field & Permanent magnet \\
\textbf{Resistive Load} & Absorb reverse power & Carbon/ceramic \\
\end{longtable}
}

\textbf{Operating Principle:}

\begin{itemize}
\tightlist
\item
  \textbf{Faraday rotation} in magnetized ferrite
\item
  \textbf{Non-reciprocal} phase shift
\item
  \textbf{Forward transmission}: Low loss
\item
  \textbf{Reverse transmission}: High attenuation
\end{itemize}

\textbf{Applications:}

\begin{itemize}
\tightlist
\item
  \textbf{Amplifier protection}
\item
  \textbf{Oscillator isolation}
\item
  \textbf{Antenna systems}
\end{itemize}

\textbf{Specifications:}

\begin{itemize}
\tightlist
\item
  \textbf{Isolation}: 20-30 dB typical
\item
  \textbf{Insertion Loss}: \textless{} 0.5 dB
\end{itemize}

\end{solutionbox}
\begin{mnemonicbox}
``Isolators Ignore Reverse Reflections''

\end{mnemonicbox}
\begin{center}\rule{0.5\linewidth}{0.5pt}\end{center}

\subsection*{Question 3(a) [3 marks]}\label{q3a}

\textbf{Draw a Traveling wave tube amplifier.}

\begin{solutionbox}

\textbf{TWT Amplifier Structure:}

\begin{verbatim}
Electron Gun    Helix Structure    Collector
     |               |                |
     v               v                v
    [|]{-{-}{-}  {-}{-}|}
         Electron    RF Input         RF Output
         Beam        Coupler          Coupler
                        |
                   Attenuator
\end{verbatim}

\textbf{Key Components:}

\begin{itemize}
\tightlist
\item
  \textbf{Electron gun}: Produces electron beam
\item
  \textbf{Helix}: Slow-wave structure
\item
  \textbf{Couplers}: Input/output RF connections
\item
  \textbf{Collector}: Collects spent electrons
\end{itemize}

\end{solutionbox}
\begin{mnemonicbox}
``TWT Transfers Wave Through Helix''

\end{mnemonicbox}
\begin{center}\rule{0.5\linewidth}{0.5pt}\end{center}

\subsection*{Question 3(b) [4 marks]}\label{q3b}

\textbf{Describes various types of hazards due to microwave radiation.}

\begin{solutionbox}

\textbf{Microwave Radiation Hazards:}

{\def\LTcaptype{none} % do not increment counter
\begin{longtable}[]{@{}lll@{}}
\toprule\noalign{}
Hazard Type & Effects & Safety Limit \\
\midrule\noalign{}
\endhead
\bottomrule\noalign{}
\endlastfoot
\textbf{HERP} (Personnel) & Tissue heating, burns & 10 mW/cm^{2} \\
\textbf{HERO} (Ordnance) & Explosive detonation & Variable \\
\textbf{HERF} (Fuel) & Fuel ignition & 5 mW/cm^{2} \\
\end{longtable}
}

\textbf{Biological Effects:}

\begin{itemize}
\tightlist
\item
  \textbf{Thermal effects}: Tissue heating above 41^\circC
\item
  \textbf{Non-thermal effects}: Cellular damage
\item
  \textbf{Sensitive organs}: Eyes, reproductive organs
\end{itemize}

\textbf{Protection Measures:}

\begin{itemize}
\tightlist
\item
  \textbf{Shielding}: Conductive enclosures
\item
  \textbf{Distance}: Power density ∝ 1/r^{2}
\item
  \textbf{Time limits}: Exposure duration control
\item
  \textbf{Warning systems}: Radiation detectors
\end{itemize}

\end{solutionbox}
\begin{mnemonicbox}
``Heat Energy Requires Proper Protection''

\end{mnemonicbox}
\begin{center}\rule{0.5\linewidth}{0.5pt}\end{center}

\subsection*{Question 3(c) [7 marks]}\label{q3c}

\textbf{Explain two cavity klystrons construction and operation with an
Applegate diagram.}

\begin{solutionbox}

\textbf{Two-Cavity Klystron Structure:}

\begin{center}
\textbf{Mermaid Diagram (Code)}
\begin{verbatim}
{Shaded}
{Highlighting}[]
graph LR
    A[Cathode] {-{-}{} B[Input Cavity]}
    B {-{-}{} C[Drift Space]}
    C {-{-}{} D[Output Cavity]}
    D {-{-}{} E[Collector]}
    F[RF Input] {-{-}{} B}
    D {-{-}{} G[RF Output]}
{Highlighting}
{Shaded}
\end{verbatim}
\end{center}

\textbf{Applegate Diagram:}

\begin{verbatim}
Velocity
   \^{}
   |    Bunched    Bunched
   |   /      {   /      }
v_{0 +{-}{-}+        {-}/        {-}{-}}
   |   {        /        /}
   |    Bunched    Bunched
   |
   +{-{-}{-}{-}{-}{-}{-}{-}{-}{-}{-}{-}{-}{-}{-}{-}{-}{-}{-}{-}{-}{-}{-}{-}{-} Distance}
   Input   Drift    Output
   Cavity  Space    Cavity
\end{verbatim}

\textbf{Operation Principle:}

{\def\LTcaptype{none} % do not increment counter
\begin{longtable}[]{@{}
  >{\raggedright\arraybackslash}p{(\linewidth - 4\tabcolsep) * \real{0.2800}}
  >{\raggedright\arraybackslash}p{(\linewidth - 4\tabcolsep) * \real{0.3600}}
  >{\raggedright\arraybackslash}p{(\linewidth - 4\tabcolsep) * \real{0.3600}}@{}}
\toprule\noalign{}
\begin{minipage}[b]{\linewidth}\raggedright
Stage
\end{minipage} & \begin{minipage}[b]{\linewidth}\raggedright
Process
\end{minipage} & \begin{minipage}[b]{\linewidth}\raggedright
Result
\end{minipage} \\
\midrule\noalign{}
\endhead
\bottomrule\noalign{}
\endlastfoot
\textbf{Velocity Modulation} & RF input varies electron speed & Speed
variation \\
\textbf{Bunching} & Fast electrons catch slow ones & Current bunches \\
\textbf{Energy Extraction} & Bunches interact with output cavity & RF
amplification \\
\end{longtable}
}

\textbf{Key Parameters:}

\begin{itemize}
\tightlist
\item
  \textbf{Transit time}: Critical for bunching
\item
  \textbf{Drift space length}: Optimized for maximum bunching
\item
  \textbf{Cavity tuning}: Resonant frequency matching
\end{itemize}

\textbf{Applications:}

\begin{itemize}
\tightlist
\item
  \textbf{Radar transmitters}
\item
  \textbf{Satellite communications}
\item
  \textbf{Linear accelerators}
\end{itemize}

\end{solutionbox}
\begin{mnemonicbox}
``Klystrons Create Bunches Through Velocity
Variation''

\end{mnemonicbox}
\begin{center}\rule{0.5\linewidth}{0.5pt}\end{center}

\subsection*{Question 3(a) OR [3
marks]}\label{q3a}

\textbf{Draw the block diagram of the attenuation measurement method for
microwave frequency.}

\begin{solutionbox}

\textbf{Attenuation Measurement Setup:}

\begin{center}
\textbf{Mermaid Diagram (Code)}
\begin{verbatim}
{Shaded}
{Highlighting}[]
graph LR
    A[Signal Generator] {-{-}{} B[Directional Coupler]}
    B {-{-}{} C[Device Under Test]}
    C {-{-}{} D[Power Meter]}
    B {-{-}{} E[Reference Power Meter]}
    F[Display Unit] {-{-}{} G[Attenuation Reading]}
    D {-{-}{} F}
    E {-{-}{} F}
{Highlighting}
{Shaded}
\end{verbatim}
\end{center}

\textbf{Measurement Process:}

\begin{itemize}
\tightlist
\item
  \textbf{Reference measurement}: Without DUT
\item
  \textbf{Insertion measurement}: With DUT
\item
  \textbf{Attenuation calculation}: A = P_{1} - P_{2} (dB)
\end{itemize}

\end{solutionbox}
\begin{mnemonicbox}
``Attenuation Appears After Accurate Assessment''

\end{mnemonicbox}
\begin{center}\rule{0.5\linewidth}{0.5pt}\end{center}

\subsection*{Question 3(b) OR [4
marks]}\label{q3b}

\textbf{Describe the limitation of vacuum tubes at microwave range.}

\begin{solutionbox}

\textbf{Vacuum Tube Limitations:}

{\def\LTcaptype{none} % do not increment counter
\begin{longtable}[]{@{}
  >{\raggedright\arraybackslash}p{(\linewidth - 4\tabcolsep) * \real{0.4444}}
  >{\raggedright\arraybackslash}p{(\linewidth - 4\tabcolsep) * \real{0.2593}}
  >{\raggedright\arraybackslash}p{(\linewidth - 4\tabcolsep) * \real{0.2963}}@{}}
\toprule\noalign{}
\begin{minipage}[b]{\linewidth}\raggedright
Limitation
\end{minipage} & \begin{minipage}[b]{\linewidth}\raggedright
Cause
\end{minipage} & \begin{minipage}[b]{\linewidth}\raggedright
Effect
\end{minipage} \\
\midrule\noalign{}
\endhead
\bottomrule\noalign{}
\endlastfoot
\textbf{Transit Time} & Finite electron travel time & Reduced gain at
high frequency \\
\textbf{Lead Inductance} & Connecting wire inductance & Poor impedance
matching \\
\textbf{Inter-electrode Capacitance} & Plate-cathode capacitance &
Feedback and instability \\
\textbf{Skin Effect} & High-frequency current distribution & Increased
resistance \\
\end{longtable}
}

\textbf{Frequency-Related Problems:}

\begin{itemize}
\tightlist
\item
  \textbf{Input impedance}: Becomes reactive
\item
  \textbf{Gain-bandwidth}: Product limitation
\item
  \textbf{Noise figure}: Increases with frequency
\item
  \textbf{Power handling}: Decreases
\end{itemize}

\textbf{Solutions:}

\begin{itemize}
\tightlist
\item
  \textbf{Special tube designs}: Lighthouse tubes
\item
  \textbf{Cavity resonators}: Replace tuned circuits
\item
  \textbf{Short leads}: Minimize inductance
\end{itemize}

\end{solutionbox}
\begin{mnemonicbox}
``Vacuum Tubes Fail Fast at High Frequencies''

\end{mnemonicbox}
\begin{center}\rule{0.5\linewidth}{0.5pt}\end{center}

\subsection*{Question 3(c) OR [7
marks]}\label{q3c}

\textbf{Explain the Principle, construction, effect of the electric and
magnetic field and operation of the magnetron in detail.}

\begin{solutionbox}

\textbf{Magnetron Construction:}

\begin{verbatim}
        Anode Vanes
    ╭─────┬─────┬─────╮
   ╱   1  │  2  │  3   ╲
  ╱       │     │       ╲
 ╱    8   │  C  │   4    ╲
│         │     │         │
│    7    │  +  │    5    │
 ╲        │     │        ╱
  ╲   6   │     │       ╱
   ╲─────┴─────┴─────╱
        Cathode (C)
\end{verbatim}

\textbf{Operating Principle:}

{\def\LTcaptype{none} % do not increment counter
\begin{longtable}[]{@{}
  >{\raggedright\arraybackslash}p{(\linewidth - 4\tabcolsep) * \real{0.2692}}
  >{\raggedright\arraybackslash}p{(\linewidth - 4\tabcolsep) * \real{0.4231}}
  >{\raggedright\arraybackslash}p{(\linewidth - 4\tabcolsep) * \real{0.3077}}@{}}
\toprule\noalign{}
\begin{minipage}[b]{\linewidth}\raggedright
Field
\end{minipage} & \begin{minipage}[b]{\linewidth}\raggedright
Direction
\end{minipage} & \begin{minipage}[b]{\linewidth}\raggedright
Effect
\end{minipage} \\
\midrule\noalign{}
\endhead
\bottomrule\noalign{}
\endlastfoot
\textbf{Electric Field} & Radial (cathode to anode) & Accelerates
electrons \\
\textbf{Magnetic Field} & Axial (perpendicular to page) & Deflects
electrons \\
\textbf{Combined Effect} & Cycloid motion & Phase synchronization \\
\end{longtable}
}

\textbf{Operation Stages:}

\begin{enumerate}
\tightlist
\item
  \textbf{Electron Emission}: Heated cathode emits electrons
\item
  \textbf{Cycloid Motion}: E\timesB fields create spiral paths
\item
  \textbf{Synchronization}: Electrons synchronize with RF field
\item
  \textbf{Energy Transfer}: Kinetic energy \rightarrow RF energy
\item
  \textbf{Output Coupling}: RF extracted through waveguide
\end{enumerate}

\textbf{Key Parameters:}

\begin{itemize}
\tightlist
\item
  \textbf{Magnetic flux density}: B = 2πmf/e
\item
  \textbf{Hull cutoff voltage}: VH = (eB^{2}R^{2})/(8m)
\item
  \textbf{Frequency}: f = eB/(2πm) \times (anode modes)
\end{itemize}

\textbf{Applications:}

\begin{itemize}
\tightlist
\item
  \textbf{Microwave ovens} (2.45 GHz)
\item
  \textbf{Radar transmitters}
\item
  \textbf{Industrial heating}
\end{itemize}

\end{solutionbox}
\begin{mnemonicbox}
``Magnetrons Make Microwaves Through Magnetic
Motion''

\end{mnemonicbox}
\begin{center}\rule{0.5\linewidth}{0.5pt}\end{center}

\subsection*{Question 4(a) [3 marks]}\label{q4a}

\textbf{Explain the working principle of a varactor diode using a
graph.}

\begin{solutionbox}

\textbf{Varactor Diode Characteristics:}

\begin{verbatim}
Capacitance (pF)
      \^{}
      |     
   100|╲    
      | ╲   
    50|  ╲  
      |   ╲ 
    10|    ╲
      |     ╲\_\_\_\_\_
      +─────────────{ Reverse Voltage (V)}
      0   5   10   15
\end{verbatim}

\textbf{Working Principle:}

\begin{itemize}
\tightlist
\item
  \textbf{Reverse bias operation}: Diode operated in reverse
\item
  \textbf{Depletion layer}: Acts as dielectric
\item
  \textbf{Variable capacitance}: C ∝ 1/\sqrtVR
\item
  \textbf{Voltage tuning}: Capacitance controlled by voltage
\end{itemize}

\textbf{Applications:}

\begin{itemize}
\tightlist
\item
  \textbf{Voltage-controlled oscillators}
\item
  \textbf{Frequency multipliers}
\item
  \textbf{Parametric amplifiers}
\end{itemize}

\end{solutionbox}
\begin{mnemonicbox}
``Varactors Vary Capacitance Via Voltage''

\end{mnemonicbox}
\begin{center}\rule{0.5\linewidth}{0.5pt}\end{center}

\subsection*{Question 4(b) [4 marks]}\label{q4b}

\textbf{Explain the Gunn Effect and negative resistance for Gunn diode.}

\begin{solutionbox}

\textbf{Gunn Effect Mechanism:}

{\def\LTcaptype{none} % do not increment counter
\begin{longtable}[]{@{}lll@{}}
\toprule\noalign{}
Parameter & Lower Valley & Upper Valley \\
\midrule\noalign{}
\endhead
\bottomrule\noalign{}
\endlastfoot
\textbf{Energy Level} & Lower & Higher \\
\textbf{Electron Mobility} & High (μ_{1}) & Low (μ_{2}) \\
\textbf{Effective Mass} & Light & Heavy \\
\end{longtable}
}

\textbf{Transfer Characteristic:}

\begin{verbatim}
Current (mA)
      \^{}
      |   ╱╲
      |  ╱  ╲ Negative
      | ╱    ╲ Resistance
      |╱      ╲ Region
      +────────╲──{ Voltage (V)}
             Threshold
\end{verbatim}

\textbf{Negative Resistance:}

\begin{itemize}
\tightlist
\item
  \textbf{Threshold voltage}: Electrons transfer to upper valley
\item
  \textbf{Current decrease}: Due to reduced mobility
\item
  \textbf{Oscillation}: Negative resistance enables
\item
  \textbf{Domain formation}: High-field domains propagate
\end{itemize}

\textbf{Key Points:}

\begin{itemize}
\tightlist
\item
  \textbf{Materials}: GaAs, InP
\item
  \textbf{Frequency range}: 1-100 GHz
\item
  \textbf{Efficiency}: 5-20\%
\end{itemize}

\end{solutionbox}
\begin{mnemonicbox}
``Gunn diodes Generate oscillations through Negative
resistance''

\end{mnemonicbox}
\begin{center}\rule{0.5\linewidth}{0.5pt}\end{center}

\subsection*{Question 4(c) [7 marks]}\label{q4c}

\textbf{Explain frequency measurement method for microwave frequency.}

\begin{solutionbox}

\textbf{Direct Frequency Measurement:}

\begin{center}
\textbf{Mermaid Diagram (Code)}
\begin{verbatim}
{Shaded}
{Highlighting}[]
graph LR
    A[Unknown Signal] {-{-}{} B[Frequency Counter]}
    B {-{-}{} C[Display]}
    D[Reference Oscillator] {-{-}{} B}
{Highlighting}
{Shaded}
\end{verbatim}
\end{center}

\textbf{Indirect Methods:}

{\def\LTcaptype{none} % do not increment counter
\begin{longtable}[]{@{}lll@{}}
\toprule\noalign{}
Method & Principle & Accuracy \\
\midrule\noalign{}
\endhead
\bottomrule\noalign{}
\endlastfoot
\textbf{Wavemeter} & Cavity resonance & \pm0.1\% \\
\textbf{Beat Frequency} & Heterodyne mixing & \pm0.01\% \\
\textbf{Standing Wave} & λ/2 measurement & \pm0.5\% \\
\end{longtable}
}

\textbf{Cavity Wavemeter Setup:}

\begin{verbatim}
   Waveguide
  ┌─────────────┐
  │    ┌───┐    │
──┤    │ C │    ├── Output
  │    └───┘    │
  └─────────────┘
   Tuning Screw
\end{verbatim}

\textbf{Measurement Procedure:}

\begin{enumerate}
\tightlist
\item
  \textbf{Coupling}: Weakly couple to signal line
\item
  \textbf{Tuning}: Adjust cavity for resonance
\item
  \textbf{Indication}: Monitor output for minimum/maximum
\item
  \textbf{Calibration}: Read frequency from calibrated scale
\end{enumerate}

\textbf{Beat Frequency Method:}

\begin{itemize}
\tightlist
\item
  \textbf{Local oscillator}: Known reference frequency
\item
  \textbf{Mixer}: Generates beat frequency
\item
  \textbf{Measurement}: fbeat = \textbar fsignal - fLO\textbar{}
\end{itemize}

\end{solutionbox}
\begin{mnemonicbox}
``Frequency Found through Careful Cavity
Calibration''

\end{mnemonicbox}
\begin{center}\rule{0.5\linewidth}{0.5pt}\end{center}

\subsection*{Question 4(a) OR [3
marks]}\label{q4a}

\textbf{Explain the working of a PIN diode as a switch.}

\begin{solutionbox}

\textbf{PIN Diode Structure:}

\begin{verbatim}
P+ Region | Intrinsic | N+ Region
    │         │           │
    ├─────────┼───────────┤
   Holes   No Carriers  Electrons
\end{verbatim}

\textbf{Switching Operation:}

{\def\LTcaptype{none} % do not increment counter
\begin{longtable}[]{@{}llll@{}}
\toprule\noalign{}
Bias Condition & Intrinsic Region & RF Impedance & Switch State \\
\midrule\noalign{}
\endhead
\bottomrule\noalign{}
\endlastfoot
\textbf{Forward Bias} & Flooded with carriers & Low (\textasciitilde1Ω)
& ON (Closed) \\
\textbf{Reverse Bias} & Depleted & High (\textasciitilde10kΩ) & OFF
(Open) \\
\textbf{Zero Bias} & Few carriers & Medium & Variable \\
\end{longtable}
}

\textbf{Key Advantages:}

\begin{itemize}
\tightlist
\item
  \textbf{Fast switching}: Nanosecond response
\item
  \textbf{Low insertion loss}: When ON
\item
  \textbf{High isolation}: When OFF
\item
  \textbf{Wide frequency range}: DC to microwave
\end{itemize}

\textbf{Applications:}

\begin{itemize}
\tightlist
\item
  \textbf{RF switches}
\item
  \textbf{Modulators}
\item
  \textbf{Attenuators}
\item
  \textbf{Phase shifters}
\end{itemize}

\end{solutionbox}
\begin{mnemonicbox}
``PIN diodes Perform Perfect switching''

\end{mnemonicbox}
\begin{center}\rule{0.5\linewidth}{0.5pt}\end{center}

\subsection*{Question 4(b) OR [4
marks]}\label{q4b}

\textbf{Explain stripeline and Microstrip circuits.}

\begin{solutionbox}

\textbf{Stripline Configuration:}

\begin{verbatim}
  Ground Plane
 ───────────────
     Dielectric
 ─────┬─────────   Signal Conductor
     Dielectric
 ───────────────
  Ground Plane
\end{verbatim}

\textbf{Microstrip Configuration:}

\begin{verbatim}
  Signal Conductor
 ─────┬─────────
    Dielectric
 ───────────────
   Ground Plane
\end{verbatim}

\textbf{Comparison Table:}

{\def\LTcaptype{none} % do not increment counter
\begin{longtable}[]{@{}lll@{}}
\toprule\noalign{}
Parameter & Stripline & Microstrip \\
\midrule\noalign{}
\endhead
\bottomrule\noalign{}
\endlastfoot
\textbf{Ground Planes} & Two (sandwich) & One (bottom) \\
\textbf{Shielding} & Complete & Partial \\
\textbf{Dispersion} & Lower & Higher \\
\textbf{Manufacturing} & Complex & Simple \\
\textbf{Cost} & Higher & Lower \\
\end{longtable}
}

\textbf{Applications:}

\begin{itemize}
\tightlist
\item
  \textbf{Stripline}: High-performance systems
\item
  \textbf{Microstrip}: PCB circuits, antennas
\end{itemize}

\textbf{Design Equations:}

\begin{itemize}
\tightlist
\item
  \textbf{Characteristic impedance}: Function of w/h ratio
\item
  \textbf{Effective permittivity}: εeff = (εr + 1)/2
\end{itemize}

\end{solutionbox}
\begin{mnemonicbox}
``Striplines are Sandwiched, Microstrips are
Mounted''

\end{mnemonicbox}
\begin{center}\rule{0.5\linewidth}{0.5pt}\end{center}

\subsection*{Question 4(c) OR [7
marks]}\label{q4c}

\textbf{Explain the principles and process of amplification for a
Parametric amplifier.}

\begin{solutionbox}

\textbf{Parametric Amplifier Principle:}

\begin{center}
\textbf{Mermaid Diagram (Code)}
\begin{verbatim}
{Shaded}
{Highlighting}[]
graph LR
    A[Signal fs] {-{-}{} B[Nonlinear Reactance]}
    C[Pump fp] {-{-}{} B}
    B {-{-}{} D[Idler fi]}
    B {-{-}{} E[Amplified Signal]}
    F[Energy Flow: Pump  Signal]
{Highlighting}
{Shaded}
\end{verbatim}
\end{center}

\textbf{Frequency Relationships:}

{\def\LTcaptype{none} % do not increment counter
\begin{longtable}[]{@{}lll@{}}
\toprule\noalign{}
Parameter & Relationship & Typical Values \\
\midrule\noalign{}
\endhead
\bottomrule\noalign{}
\endlastfoot
\textbf{Pump Frequency} & fp = fs + fi & 10 GHz \\
\textbf{Signal Frequency} & fs (input) & 1 GHz \\
\textbf{Idler Frequency} & fi = fp - fs & 9 GHz \\
\end{longtable}
}

\textbf{Amplification Process:}

\begin{enumerate}
\tightlist
\item
  \textbf{Nonlinear Element}: Varactor diode provides time-varying
  capacitance
\item
  \textbf{Pump Power}: High-frequency pump supplies energy
\item
  \textbf{Frequency Mixing}: Three-frequency interaction
\item
  \textbf{Energy Transfer}: Pump energy \rightarrow Signal energy
\item
  \textbf{Impedance Matching}: Optimize power transfer
\end{enumerate}

\textbf{Circuit Configuration:}

\begin{verbatim}
Signal ──┬── Varactor ──┬── Amplified
Input   │    Diode     │    Output
        │              │
       ┌┴┐            ┌┴┐
       │C│            │L│ Idler
       │ │            │ │ Circuit
       └─┘            └─┘
        │              │
      Pump ────────────┘
      Input
\end{verbatim}

\textbf{Key Advantages:}

\begin{itemize}
\tightlist
\item
  \textbf{Low noise figure}: Near quantum limit
\item
  \textbf{High gain}: 10-20 dB typical
\item
  \textbf{Wide bandwidth}: Limited by pump circuit
\end{itemize}

\textbf{Applications:}

\begin{itemize}
\tightlist
\item
  \textbf{Satellite receivers}
\item
  \textbf{Radio astronomy}
\item
  \textbf{Low-noise amplifiers}
\end{itemize}

\textbf{Design Considerations:}

\begin{itemize}
\tightlist
\item
  \textbf{Pump power}: Sufficient for nonlinear operation
\item
  \textbf{Impedance matching}: All three frequencies
\item
  \textbf{Stability}: Prevent oscillation
\end{itemize}

\end{solutionbox}
\begin{mnemonicbox}
``Parametric amplifiers Pump Power into signal
Perfectly''

\end{mnemonicbox}
\begin{center}\rule{0.5\linewidth}{0.5pt}\end{center}

\subsection*{Question 5(a) [3 marks]}\label{q5a}

\textbf{Compare RADAR and SONAR.}

\begin{solutionbox}

\textbf{RADAR vs SONAR Comparison:}

{\def\LTcaptype{none} % do not increment counter
\begin{longtable}[]{@{}lll@{}}
\toprule\noalign{}
Parameter & RADAR & SONAR \\
\midrule\noalign{}
\endhead
\bottomrule\noalign{}
\endlastfoot
\textbf{Wave Type} & Electromagnetic & Acoustic \\
\textbf{Medium} & Air/Vacuum & Water \\
\textbf{Frequency} & 300 MHz - 30 GHz & 1 kHz - 1 MHz \\
\textbf{Speed} & 3\times10^{8} m/s & 1500 m/s (water) \\
\textbf{Range} & Up to 1000 km & Up to 100 km \\
\textbf{Applications} & Aircraft, weather & Submarines, fishing \\
\end{longtable}
}

\textbf{Common Principles:}

\begin{itemize}
\tightlist
\item
  \textbf{Echo ranging}: Measure time-of-flight
\item
  \textbf{Doppler effect}: Detect moving targets
\item
  \textbf{Beam forming}: Directional transmission
\end{itemize}

\textbf{Key Differences:}

\begin{itemize}
\tightlist
\item
  \textbf{Propagation}: EM waves vs sound waves
\item
  \textbf{Attenuation}: Different loss mechanisms
\item
  \textbf{Resolution}: Frequency dependent
\end{itemize}

\end{solutionbox}
\begin{mnemonicbox}
``RADAR sees Radio waves, SONAR hears Sound waves''

\end{mnemonicbox}
\begin{center}\rule{0.5\linewidth}{0.5pt}\end{center}

\subsection*{Question 5(b) [4 marks]}\label{q5b}

\textbf{Write the name of RADAR display method and explain anyone.}

\begin{solutionbox}

\textbf{RADAR Display Methods:}

{\def\LTcaptype{none} % do not increment counter
\begin{longtable}[]{@{}lll@{}}
\toprule\noalign{}
Display Type & Description & Application \\
\midrule\noalign{}
\endhead
\bottomrule\noalign{}
\endlastfoot
\textbf{A-Scope} & Range vs amplitude & Target detection \\
\textbf{B-Scope} & Range vs azimuth & 2D position \\
\textbf{C-Scope} & Azimuth vs elevation & 3D tracking \\
\textbf{PPI} & Plan Position Indicator & Air traffic control \\
\textbf{RHI} & Range Height Indicator & Weather radar \\
\end{longtable}
}

\textbf{PPI Display Explanation:}

\begin{center}
\textbf{Mermaid Diagram (Code)}
\begin{verbatim}
{Shaded}
{Highlighting}[]
graph TD
    A[Center {- Radar Position] {-}{-}{} B[Sweep Line {-} Antenna Direction]}
    B {-{-}{} C[Target Blips {-} Range \& Bearing]}
    D[Circular Pattern] {-{-}{} E[360^ Coverage]}
{Highlighting}
{Shaded}
\end{verbatim}
\end{center}

\textbf{PPI Features:}

\begin{itemize}
\tightlist
\item
  \textbf{Polar coordinates}: Range and bearing
\item
  \textbf{Rotating sweep}: Follows antenna rotation
\item
  \textbf{Persistence}: Targets remain visible
\item
  \textbf{Scale selection}: Adjustable range
\end{itemize}

\textbf{Display Process:}

\begin{enumerate}
\tightlist
\item
  \textbf{Sweep generation}: Synchronized with antenna
\item
  \textbf{Target plotting}: Distance and direction
\item
  \textbf{Intensity modulation}: Target strength
\item
  \textbf{Map overlay}: Geographic reference
\end{enumerate}

\end{solutionbox}
\begin{mnemonicbox}
``PPI Provides Perfect Position Information''

\end{mnemonicbox}
\begin{center}\rule{0.5\linewidth}{0.5pt}\end{center}

\subsection*{Question 5(c) [7 marks]}\label{q5c}

\textbf{Explain the basic pulse radar system with a block diagram.}

\begin{solutionbox}

\textbf{Pulse Radar Block Diagram:}

\begin{center}
\textbf{Mermaid Diagram (Code)}
\begin{verbatim}
{Shaded}
{Highlighting}[]
graph LR
    A[Master Oscillator] {-{-}{} B[Modulator]}
    B {-{-}{} C[Power Amplifier]}
    C {-{-}{} D[Duplexer]}
    D {-{-}{} E[Antenna]}
    E {-{-}{} F[Target]}
    F {-{-}{} E}
    E {-{-}{} D}
    D {-{-}{} G[Receiver]}
    G {-{-}{} H[Signal Processor]}
    H {-{-}{} I[Display]}
    J[Timer] {-{-}{} A}
    J {-{-}{} I}
{Highlighting}
{Shaded}
\end{verbatim}
\end{center}

\textbf{System Components:}

{\def\LTcaptype{none} % do not increment counter
\begin{longtable}[]{@{}lll@{}}
\toprule\noalign{}
Component & Function & Key Parameters \\
\midrule\noalign{}
\endhead
\bottomrule\noalign{}
\endlastfoot
\textbf{Master Oscillator} & Generate RF signal & Frequency stability \\
\textbf{Modulator} & Create pulse train & Pulse width, PRF \\
\textbf{Power Amplifier} & Boost transmit power & Peak power,
efficiency \\
\textbf{Duplexer} & Switch Tx/Rx & Isolation, switching time \\
\textbf{Antenna} & Radiate/receive & Gain, beamwidth \\
\textbf{Receiver} & Amplify echo signals & Sensitivity, bandwidth \\
\end{longtable}
}

\textbf{Operating Sequence:}

\begin{enumerate}
\tightlist
\item
  \textbf{Transmission Phase}:

  \begin{itemize}
  \tightlist
  \item
    Master oscillator generates RF
  \item
    Modulator creates pulses
  \item
    Power amplifier boosts signal
  \item
    Duplexer routes to antenna
  \end{itemize}
\item
  \textbf{Reception Phase}:

  \begin{itemize}
  \tightlist
  \item
    Antenna receives echoes
  \item
    Duplexer routes to receiver
  \item
    Signal processing extracts information
  \item
    Display shows target data
  \end{itemize}
\end{enumerate}

\textbf{Key Equations:}

\begin{itemize}
\tightlist
\item
  \textbf{Range}: R = ct/2 (where t = round-trip time)
\item
  \textbf{Maximum range}: Rmax = cPRT/2
\item
  \textbf{Range resolution}: ΔR = cτ/2
\end{itemize}

\textbf{Performance Parameters:}

\begin{itemize}
\tightlist
\item
  \textbf{PRF}: Pulse Repetition Frequency
\item
  \textbf{Duty cycle}: τ \times PRF
\item
  \textbf{Average power}: Peak power \times duty cycle
\end{itemize}

\end{solutionbox}
\begin{mnemonicbox}
``Pulse Radar Properly Processes Reflected signals''

\end{mnemonicbox}
\begin{center}\rule{0.5\linewidth}{0.5pt}\end{center}

\subsection*{Question 5(a) OR [3
marks]}\label{q5a}

\textbf{List the application of microwave frequency.}

\begin{solutionbox}

\textbf{Microwave Applications:}

{\def\LTcaptype{none} % do not increment counter
\begin{longtable}[]{@{}lll@{}}
\toprule\noalign{}
Application Category & Specific Uses & Frequency Band \\
\midrule\noalign{}
\endhead
\bottomrule\noalign{}
\endlastfoot
\textbf{Communication} & Satellite, cellular, WiFi & 1-40 GHz \\
\textbf{Radar Systems} & Weather, air traffic, military & 1-35 GHz \\
\textbf{Industrial} & Heating, drying, medical & 0.9-5.8 GHz \\
\textbf{Navigation} & GPS, aircraft landing & 1-15 GHz \\
\textbf{Scientific} & Radio astronomy, research & 1-300 GHz \\
\textbf{Medical} & Diathermy, cancer treatment & 0.9-2.45 GHz \\
\textbf{Domestic} & Microwave ovens & 2.45 GHz \\
\end{longtable}
}

\textbf{Key Points:}

\begin{itemize}
\tightlist
\item
  \textbf{ISM bands} (Industrial, Scientific, Medical): License-free
\item
  \textbf{Penetration ability}: Depends on frequency and material
\item
  \textbf{Atmospheric absorption}: Increases with frequency
\end{itemize}

\end{solutionbox}
\begin{mnemonicbox}
``Microwaves Serve Many Applications Perfectly''

\end{mnemonicbox}
\begin{center}\rule{0.5\linewidth}{0.5pt}\end{center}

\subsection*{Question 5(b) OR [4
marks]}\label{q5b}

\textbf{Compare PULSED RADAR and CW RADAR.}

\begin{solutionbox}

\textbf{PULSED vs CW RADAR Comparison:}

{\def\LTcaptype{none} % do not increment counter
\begin{longtable}[]{@{}lll@{}}
\toprule\noalign{}
Parameter & Pulsed RADAR & CW RADAR \\
\midrule\noalign{}
\endhead
\bottomrule\noalign{}
\endlastfoot
\textbf{Transmission} & Pulse train & Continuous wave \\
\textbf{Range Measurement} & Time-of-flight & Frequency shift \\
\textbf{Velocity Measurement} & Doppler in pulses & Direct Doppler \\
\textbf{Antenna} & Single (duplexer) & Separate Tx/Rx \\
\textbf{Power} & High peak, low average & Low continuous \\
\textbf{Range Resolution} & Pulse width limited & Poor \\
\textbf{Velocity Resolution} & Limited & Excellent \\
\textbf{Complexity} & High & Low \\
\textbf{Cost} & Higher & Lower \\
\end{longtable}
}

\textbf{Operational Differences:}

\textbf{Pulsed RADAR:}

\begin{itemize}
\tightlist
\item
  \textbf{Range equation}: R = ct/2
\item
  \textbf{Maximum range}: Limited by PRF
\item
  \textbf{Blind ranges}: Multiple of cPRT/2
\item
  \textbf{Applications}: Long-range detection
\end{itemize}

\textbf{CW RADAR:}

\begin{itemize}
\tightlist
\item
  \textbf{Doppler equation}: fd = 2vr/λ
\item
  \textbf{Range measurement}: Requires FM modulation
\item
  \textbf{No blind ranges}: Continuous operation
\item
  \textbf{Applications}: Speed measurement, proximity
\end{itemize}

\textbf{Key Advantages:}

\begin{itemize}
\tightlist
\item
  \textbf{Pulsed}: Better range capability, target separation
\item
  \textbf{CW}: Better velocity accuracy, simpler design
\end{itemize}

\end{solutionbox}
\begin{mnemonicbox}
``Pulsed measures Range, CW measures Velocity''

\end{mnemonicbox}
\begin{center}\rule{0.5\linewidth}{0.5pt}\end{center}

\subsection*{Question 5(c) OR [7
marks]}\label{q5c}

\textbf{Explain MTI Radar with the block diagram.}

\begin{solutionbox}

\textbf{MTI RADAR Block Diagram:}

\begin{center}
\textbf{Mermaid Diagram (Code)}
\begin{verbatim}
{Shaded}
{Highlighting}[]
graph LR
    A[Transmitter] {-{-}{} B[Duplexer]}
    B {-{-}{} C[Antenna]}
    C {-{-}{} D[Target]}
    D {-{-}{} C}
    C {-{-}{} B}
    B {-{-}{} E[Receiver]}
    E {-{-}{} F[Phase Detector]}
    G[STALO] {-{-}{} H[Mixer]}
    H {-{-}{} F}
    I[COHO] {-{-}{} F}
    F {-{-}{} J[MTI Filter]}
    J {-{-}{} K[Display]}
    G {-{-}{} L[Frequency Multiplier]}
    L {-{-}{} A}
{Highlighting}
{Shaded}
\end{verbatim}
\end{center}

\textbf{MTI System Components:}

{\def\LTcaptype{none} % do not increment counter
\begin{longtable}[]{@{}lll@{}}
\toprule\noalign{}
Component & Full Form & Function \\
\midrule\noalign{}
\endhead
\bottomrule\noalign{}
\endlastfoot
\textbf{STALO} & Stable Local Oscillator & Reference frequency \\
\textbf{COHO} & Coherent Oscillator & Phase reference \\
\textbf{MTI Filter} & Moving Target Indicator & Clutter suppression \\
\textbf{Phase Detector} & - & Compare signal phases \\
\end{longtable}
}

\textbf{MTI Operating Principle:}

\textbf{Pulse-to-Pulse Comparison:}

\begin{verbatim}
Signal Amplitude
       \^{}
       |    Fixed Target (Clutter)
       |    \_\_\_\_\_\_\_\_\_\_\_\_\_\_\_\_
       |   |                |
       |   |                |
       |   |                |
   \_\_\_\_+\_\_\_|\_\_\_\_\_\_\_\_\_\_\_\_\_\_\_\_|\_\_\_\_
       |                        
       |    Moving Target
       |     /{      /}
       |    /  {    /  }
       |   /    {  /    }
   \_\_\_\_+\_\_/\_\_\_\_\_\_{/\_\_\_\_\_\_\_\_\_\_\_\_}
       |
       +{-{-}{-}{-}{-}{-}{-}{-}{-}{-}{-}{-}{-}{-}{-}{-}{-}{-}{-}{-}{-}{-}{-}{-}{-} Time}
           Pulse 1    Pulse 2
\end{verbatim}

\textbf{MTI Process:}

\begin{enumerate}
\tightlist
\item
  \textbf{Coherent transmission}: Maintain phase relationships
\item
  \textbf{Echo reception}: Preserve phase information
\item
  \textbf{Phase comparison}: Compare successive pulses
\item
  \textbf{Clutter cancellation}: Subtract stationary returns
\item
  \textbf{Moving target detection}: Enhance moving targets
\end{enumerate}

\textbf{Key Equations:}

\begin{itemize}
\tightlist
\item
  \textbf{Doppler frequency}: fd = 2vr cos(θ)/λ
\item
  \textbf{Phase change}: Δφ = 4πvr/λ \times PRT
\item
  \textbf{Blind speeds}: vb = nλ/(2PRT)
\end{itemize}

\textbf{MTI Improvement Factor:}

\begin{itemize}
\tightlist
\item
  \textbf{Definition}: Ratio of clutter power before/after MTI
\item
  \textbf{Typical values}: 20-40 dB
\item
  \textbf{Factors affecting}: System stability, clutter characteristics
\end{itemize}

\textbf{Limitations:}

\begin{itemize}
\tightlist
\item
  \textbf{Blind speeds}: Targets invisible at certain velocities
\item
  \textbf{Tangential targets}: Radial velocity component needed
\item
  \textbf{Weather effects}: Atmospheric fluctuations
\end{itemize}

\textbf{Applications:}

\begin{itemize}
\tightlist
\item
  \textbf{Air traffic control}: Separate aircraft from ground clutter
\item
  \textbf{Weather radar}: Distinguish precipitation from terrain
\item
  \textbf{Military radar}: Detect moving vehicles/aircraft
\end{itemize}

\end{solutionbox}
\begin{mnemonicbox}
``MTI Makes Targets Identifiable by Movement''

\end{mnemonicbox}

\end{document}
