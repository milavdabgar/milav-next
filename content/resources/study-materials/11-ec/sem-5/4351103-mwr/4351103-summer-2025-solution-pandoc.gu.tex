\documentclass[10pt,a4paper]{article}

% content/resources/templates/preamble.tex
\usepackage[margin=0.6in]{geometry}
\author{Milav Dabgar}
\usepackage{amsmath,amssymb,amsthm}
\usepackage{booktabs}
\usepackage{multirow}
\usepackage{xcolor}
\usepackage{tcolorbox}
\tcbuselibrary{breakable,skins}
\usepackage[colorlinks=true,linkcolor=blue]{hyperref}
\usepackage{titlesec}
\usepackage{enumitem}
\usepackage{tikz}
\usepackage{pgfplots}
\usepackage{circuitikz}
\usepackage[version=4]{mhchem}
\usepackage{longtable}
\usepackage{array}
\usepackage{float}
\usepackage{caption}
\usepackage{listings}

\lstset{
  basicstyle=\small\ttfamily,
  breaklines=true,
  breakatwhitespace=false,
  postbreak=\mbox{\textcolor{red}{$\hookrightarrow$}\space},
  float=false,
  numbers=left,
  numberstyle=\tiny\color{gray},
  numbersep=10pt,
  xleftmargin=2em,
  keywordstyle=\color{blue},
  commentstyle=\color{green!60!black},
  stringstyle=\color{purple},
  backgroundcolor=\color{gray!5},
  showstringspaces=false,
  tabsize=2,
  captionpos=b,
  keepspaces=true,
  columns=flexible
}

\pgfplotsset{compat=1.18}
\usetikzlibrary{shapes,arrows,positioning,calc,patterns,decorations.pathmorphing,decorations.markings,arrows.meta}

% Color scheme
\definecolor{headcolor}{RGB}{0,102,204}
\definecolor{keycolor}{RGB}{220,20,60}
\definecolor{solutioncolor}{RGB}{34,139,34}
\definecolor{mnemoniccolor}{RGB}{148,0,211}
\definecolor{codecolor}{RGB}{0,0,100}

% Spacing
\setlength{\parskip}{3pt}
\setlist[itemize]{nosep}
\setlist[enumerate]{nosep}

% Title formatting
\titleformat{\section}{\Large\bfseries\color{headcolor}}{\thesection}{1em}{}
\titleformat{\subsection}{\large\bfseries\color{headcolor}}{\thesubsection}{1em}{}

% Pandoc tightlist compatibility
\providecommand{\tightlist}{%
  \setlength{\itemsep}{0pt}\setlength{\parskip}{0pt}}

% Pandoc longtable compatibility
\newcounter{none}
\def\thenone{}


% content/resources/templates/gujarati-boxes.tex
\usepackage{fontspec}
\usepackage{polyglossia}

% Set Gujarati as main language (document is primarily in Gujarati)
% Note: gloss-gujarati.ldf doesn't exist in polyglossia, but it will use hyphenation patterns
\setdefaultlanguage{gujarati}
\setotherlanguage{english}

% Configure Gujarati font properly
% Use Language=Default to prevent polyglossia from trying to add language-specific features
% that don't exist for Gujarati, which causes "empty feature" warnings
\newfontfamily\gujaratifont[Script=Gujarati,AutoFakeBold=2.5,AutoFakeSlant=0.3]{Noto Sans Gujarati}
\setmainfont[Script=Gujarati,AutoFakeBold=2.5,AutoFakeSlant=0.3]{Noto Sans Gujarati}
% Use Noto Sans Gujarati for monospace to support Gujarati in text
\setmonofont[Scale=0.9]{Noto Sans Gujarati}

% Configure English to use the same font
\newfontfamily\englishfont[Script=Gujarati,AutoFakeBold=2.5,AutoFakeSlant=0.3]{Noto Sans Gujarati}

% Translations for polyglossia
\gappto\captionsgujarati{
  \renewcommand{\tablename}{કોષ્ટક}
  \renewcommand{\figurename}{આકૃતિ}
}

% Helper for TikZ nodes to ensure Gujarati font
\newcommand{\gu}[1]{{\gujaratifont #1}}

% Custom environments
\newtcolorbox{solutionbox}{
    breakable,
    enhanced,
    colback=solutioncolor!5!white,
    colframe=solutioncolor!75!black,
    fonttitle=\bfseries,
    title=જવાબ
}

\newtcolorbox{solutionboxnobreak}{
 colback=solutioncolor!5!white,
 colframe=solutioncolor!75!black,
 fonttitle=\bfseries,
 title=જવાબ
}

\newtcolorbox{keyformula}{
 breakable,
 enhanced,
 colback=keycolor!5!white,
 colframe=keycolor!75!black,
 fonttitle=\bfseries,
 title=રાસાયણિક સમીકરણ/સૂત્ર
}

\newtcolorbox{mnemonicbox}{
 breakable,
 enhanced,
 colback=mnemoniccolor!5!white,
 colframe=mnemoniccolor!75!black,
 fonttitle=\bfseries,
 title=મેમરી ટ્રીક
}


\begin{document}

\begin{center}
{\Huge\bfseries\color{headcolor} Subject Name (Gujarati)}\\[5pt]
{\LARGE 4351103 -- Summer 2025}\\[3pt]
{\large Semester 1 Study Material}\\[3pt]
{\normalsize\textit{Detailed Solutions and Explanations}}
\end{center}

\vspace{10pt}

\subsection*{પ્રશ્ન 1(અ) [3
ગુણ]}\label{uxaaauxab0uxab6uxaa8-1uxa85-3-uxa97uxaa3}

\textbf{ચાર માઇક્રોવેવ આવર્તન બેન્ડની તેમની આવર્ત શ્રેણી સાથે અને તેનાં ઉપયોગો સાથેની
સૂચી બનાવો.}

\begin{solutionbox}

{\def\LTcaptype{none} % do not increment counter
\begin{longtable}[]{@{}lll@{}}
\toprule\noalign{}
બેન્ડ & આવર્તન શ્રેણી & ઉપયોગો \\
\midrule\noalign{}
\endhead
\bottomrule\noalign{}
\endlastfoot
\textbf{L-band} & 1-2 GHz & GPS, Mobile communication \\
\textbf{S-band} & 2-4 GHz & WiFi, Bluetooth, Radar \\
\textbf{C-band} & 4-8 GHz & Satellite communication \\
\textbf{X-band} & 8-12 GHz & Military radar, Weather radar \\
\end{longtable}
}

\end{solutionbox}
\begin{mnemonicbox}
``Little Satellites Communicate eXcellently''

\end{mnemonicbox}
\begin{center}\rule{0.5\linewidth}{0.5pt}\end{center}

\subsection*{પ્રશ્ન 1(બ) [4
ગુણ]}\label{uxaaauxab0uxab6uxaa8-1uxaac-4-uxa97uxaa3}

\textbf{એક જ સ્ટ્બનો ઉપયોગ કરીને impedance matching ની પ્રક્રિયા સમજાવો.}

\begin{solutionbox}

\textbf{Single stub matching} એ \textbf{short-circuited stub} વડે
reflection દૂર કરવાની પદ્ધતિ છે.

\textbf{પ્રક્રિયા:}

\begin{itemize}
\tightlist
\item
  \textbf{Stub લંબાઈ}: Reactive impedance પ્રદાન કરે છે
\item
  \textbf{Stub સ્થાન}: Load થી Smith chart વડે ગણવામાં આવે છે
\item
  \textbf{Matching condition}: Real part = Z_{0}, imaginary part = 0
\end{itemize}

\begin{center}
\textbf{Mermaid Diagram (Code)}
\begin{verbatim}
{Shaded}
{Highlighting}[]
graph LR
    A[Source] {-{-}{} B[Transmission Line]}
    B {-{-}{} C[Stub Position]}
    C {-{-}{} D[Load]}
    C {-{-}{} E[Short Stub]}
{Highlighting}
{Shaded}
\end{verbatim}
\end{center}

\end{solutionbox}
\begin{mnemonicbox}
``Stub Positioned for Perfect Matching''

\end{mnemonicbox}
\begin{center}\rule{0.5\linewidth}{0.5pt}\end{center}

\subsection*{પ્રશ્ન 1(ક) [7
ગુણ]}\label{uxaaauxab0uxab6uxaa8-1uxa95-7-uxa97uxaa3}

\textbf{લોસલેસ ટ્રાન્સમિશન લાઇનની લાક્ષણિકતાઓ જણાવો અને બે વાયર ટ્રાન્સમિશન લાઇન
માટે સામાન્ય સમીકરણ મેળવો.}

\begin{solutionbox}

\textbf{લોસલેસ લાઇનની લાક્ષણિકતાઓ:}

\begin{itemize}
\tightlist
\item
  \textbf{કોઈ power loss નથી}: R = 0, G = 0
\item
  \textbf{સ્થિર amplitude}: કોઈ attenuation નથી
\item
  \textbf{માત્ર phase delay}: સિગ્નલ delayed પણ weakened નથી
\item
  \textbf{Standing wave pattern}: Reflections ને કારણે બને છે
\end{itemize}

\textbf{સામાન્ય સમીકરણો:}

Voltage માટે: \textbf{V(z) = V_{+}e\^{}(-γz) + V_{-}e\^{}(γz)} Current માટે:
\textbf{I(z) = (V_{+}/Z_{0})e\^{}(-γz) - (V_{-}/Z_{0})e\^{}(γz)}

જ્યાં:

\begin{itemize}
\tightlist
\item
  \textbf{γ = α + jβ} (propagation constant)
\item
  \textbf{Z_{0} = \sqrt(L/C)} (characteristic impedance)
\item
  \textbf{Lossless line માટે}: α = 0, γ = jβ
\end{itemize}

\end{solutionbox}
\begin{mnemonicbox}
``Lossless Lines Love Low Loss''

\end{mnemonicbox}
\begin{center}\rule{0.5\linewidth}{0.5pt}\end{center}

\subsection*{પ્રશ્ન 1(ક) OR [7
ગુણ]}\label{uxaaauxab0uxab6uxaa8-1uxa95-or-7-uxa97uxaa3}

\textbf{સ્થાયી તરંગ વ્યાખ્યાયિત કરો. શોર્ટ સર્કિટ અને ઓપન સર્કિટ લાઇન માટે સ્ટેન્ડિંગ
વેવ પેટર્ન દોરો અને સમજાવો.}

\begin{solutionbox}

\textbf{Standing Wave:} \textbf{આગળ અને પરાવર્તિત તરંગોના} constructive અને
destructive interference થી બનતો સ્થિર pattern.

\textbf{Short Circuit Line:}

\begin{itemize}
\tightlist
\item
  \textbf{Current maximum} short circuit પર
\item
  \textbf{Voltage minimum} short circuit પર
\item
  \textbf{Minima વચ્ચેનું અંતર}: λ/2
\end{itemize}

\textbf{Open Circuit Line:}

\begin{itemize}
\tightlist
\item
  \textbf{Voltage maximum} open circuit પર
\item
  \textbf{Current minimum} open circuit પર
\item
  \textbf{Maxima વચ્ચેનું અંતર}: λ/2
\end{itemize}

\begin{verbatim}
Short Circuit:     Open Circuit:
                  
V |    /{            V |  /    /}
  |   /  {             |/    /    }
  |  /    {            |           }
  |\_/\_\_\_\_\_\_{           |\_\_\_\_\_\_\_\_\_\_\_\_}
    0  λ/4  λ/2         0  λ/4  λ/2
    
I |  /{    /        I |    /}
  | /  {  /           |   /  }
  |/    {/            |  /    }
  |             {      |\_/\_\_\_\_\_\_}
    0  λ/4  λ/2         0  λ/4  λ/2
\end{verbatim}

\end{solutionbox}
\begin{mnemonicbox}
``Short Circuits Current, Open Circuits Voltage''

\end{mnemonicbox}
\begin{center}\rule{0.5\linewidth}{0.5pt}\end{center}

\subsection*{પ્રશ્ન 2(અ) [3
ગુણ]}\label{uxaaauxab0uxab6uxaa8-2uxa85-3-uxa97uxaa3}

\textbf{મેજિક TEE ની કામગીરી દોરો અને સમજાવો.}

\begin{solutionbox}

\textbf{Magic TEE} એ E-plane અને H-plane tees ને મિલાવીને બનાવેલ \textbf{ચાર
પોર્ટ} વાળી device છે જે opposite ports વચ્ચે isolation આપે છે.

\begin{center}
\textbf{Mermaid Diagram (Code)}
\begin{verbatim}
{Shaded}
{Highlighting}[]
graph TD
    A[Port 1 {- E{-}arm] {-}{-}{} C[Junction]}
    B[Port 2 {- H{-}arm] {-}{-}{} C}
    C {-{-}{} D[Port 3 {-} Collinear arm]}
    C {-{-}{} E[Port 4 {-} Collinear arm]}
{Highlighting}
{Shaded}
\end{verbatim}
\end{center}

\textbf{કામગીરી:}

\begin{itemize}
\tightlist
\item
  \textbf{E-arm અને H-arm}: એકબીજાથી isolated રહે છે
\item
  \textbf{Sum port}: Collinear arms ના signals ને add કરે છે
\item
  \textbf{Difference port}: Signals ને subtract કરે છે
\end{itemize}

\end{solutionbox}
\begin{mnemonicbox}
``Magic Tee Mixes Modes''

\end{mnemonicbox}
\begin{center}\rule{0.5\linewidth}{0.5pt}\end{center}

\subsection*{પ્રશ્ન 2(બ) [4
ગુણ]}\label{uxaaauxab0uxab6uxaa8-2uxaac-4-uxa97uxaa3}

\textbf{હાયબ્રિડ રિંગની કામગીરી સમજાવો.}

\begin{solutionbox}

\textbf{Hybrid Ring} એ \textbf{ચાર પોર્ટ} વાળી \textbf{ગોળાકાર waveguide}
છે જે power division અને isolation માટે વપરાય છે.

\textbf{બાંધકામ:}

\begin{itemize}
\tightlist
\item
  \textbf{Ring circumference}: 1.5λ
\item
  \textbf{Port spacing}: Adjacent ports વચ્ચે λ/4
\item
  \textbf{Matched impedance}: દરેક port Z_{0} સાથે matched
\end{itemize}

\textbf{કામગીરી:}

\begin{itemize}
\tightlist
\item
  \textbf{Power splitting}: Input બે output ports વચ્ચે સમાન રીતે વહેંચાય છે
\item
  \textbf{Isolation}: Opposite ports isolated રહે છે
\item
  \textbf{Phase difference}: Output ports વચ્ચે 180^\circ
\end{itemize}

\end{solutionbox}
\begin{mnemonicbox}
``Ring Runs Round for Power Sharing''

\end{mnemonicbox}
\begin{center}\rule{0.5\linewidth}{0.5pt}\end{center}

\subsection*{પ્રશ્ન 2(ક) [7
ગુણ]}\label{uxaaauxab0uxab6uxaa8-2uxa95-7-uxa97uxaa3}

\textbf{``સર્ક્યુલેટર'' ના બાંધકામ અને કાર્યસિદ્ધાંત સમજાવો. તેની એપ્લિકેશનોની સૂચિ
બનાવો.}

\begin{solutionbox}

\textbf{બાંધકામ:}

\begin{itemize}
\tightlist
\item
  \textbf{ત્રણ પોર્ટ device} \textbf{ferrite material} સાથે
\item
  \textbf{Permanent magnet} magnetic field બનાવે છે
\item
  \textbf{Y-junction waveguide} structure
\end{itemize}

\begin{center}
\textbf{Mermaid Diagram (Code)}
\begin{verbatim}
{Shaded}
{Highlighting}[]
graph LR
    A[Port 1] {-{-}{} B[Ferrite Junction]}
    B {-{-}{} C[Port 2]}
    C {-{-}{} D[Port 3]}
    D {-{-}{} A}
    style B fill:\#ff9999
{Highlighting}
{Shaded}
\end{verbatim}
\end{center}

\textbf{કાર્યસિદ્ધાંત:}

\begin{itemize}
\tightlist
\item
  \textbf{Faraday rotation}: Magnetic field wave polarization ને rotate
  કરે છે
\item
  \textbf{Unidirectional flow}: Power માત્ર એક દિશામાં વહે છે
\item
  \textbf{Non-reciprocal}: વિરુદ્ધ દિશાઓ માટે અલગ behavior
\end{itemize}

\textbf{ઉપયોગો:}

\begin{itemize}
\tightlist
\item
  \textbf{Radar systems}: Transmitter ને receiver થી isolate કરે છે
\item
  \textbf{Communication}: TX/RX માટે antenna sharing
\item
  \textbf{Microwave amplifiers}: Feedback અટકાવે છે
\end{itemize}

\end{solutionbox}
\begin{mnemonicbox}
``Circulator Circles Clockwise Continuously''

\end{mnemonicbox}
\begin{center}\rule{0.5\linewidth}{0.5pt}\end{center}

\subsection*{પ્રશ્ન 2(અ) OR [3
ગુણ]}\label{uxaaauxab0uxab6uxaa8-2uxa85-or-3-uxa97uxaa3}

\textbf{લંબચોરસ વેવગાઇડ અને ગોળાઇવાળું વેવગાઇડની તુલના કરો.}

\begin{solutionbox}

{\def\LTcaptype{none} % do not increment counter
\begin{longtable}[]{@{}lll@{}}
\toprule\noalign{}
પેરામીટર & લંબચોરસ & ગોળાકાર \\
\midrule\noalign{}
\endhead
\bottomrule\noalign{}
\endlastfoot
\textbf{Cross-section} & Rectangle & Circle \\
\textbf{Dominant mode} & TE_{1}_{0} & TE_{1}_{1} \\
\textbf{Cutoff frequency} & સરળ calculation & જટિલ calculation \\
\textbf{Manufacturing} & સરળ & મધ્યમ \\
\textbf{Power handling} & ઓછી & વધારે \\
\end{longtable}
}

\end{solutionbox}
\begin{mnemonicbox}
``Rectangles are Regular, Circles are Complex''

\end{mnemonicbox}
\begin{center}\rule{0.5\linewidth}{0.5pt}\end{center}

\subsection*{પ્રશ્ન 2(બ) OR [4
ગુણ]}\label{uxaaauxab0uxab6uxaa8-2uxaac-or-4-uxa97uxaa3}

\textbf{ડાયરેક્શનલ કપ્લરનું કાર્યસિદ્ધાંત દોરો અને સમજાવો.}

\begin{solutionbox}

\textbf{Directional Coupler} \textbf{forward power} ને sample કરે છે અને
reflected power થી isolation આપે છે.

\begin{center}
\textbf{Mermaid Diagram (Code)}
\begin{verbatim}
{Shaded}
{Highlighting}[]
graph LR
    A[Input] {-{-}{} B[Main Line]}
    B {-{-}{} C[Output]}
    B {-.{-}{} D[Coupled Port]}
    B {-.{-}{} E[Isolated Port]}
    style D fill:\#99ff99
    style E fill:\#ff9999
{Highlighting}
{Shaded}
\end{verbatim}
\end{center}

\textbf{કામગીરી:}

\begin{itemize}
\tightlist
\item
  \textbf{Coupling factor}: Extract થતી power નક્કી કરે છે (10-20 dB
  સામાન્ય)
\item
  \textbf{Directivity}: Forward ને reverse power થી isolate કરે છે
\item
  \textbf{Insertion loss}: Main line માં minimal loss
\end{itemize}

\textbf{પેરામીટર્સ:}

\begin{itemize}
\tightlist
\item
  \textbf{C = 10 log(P_{1}/P_{3})} (Coupling factor)
\item
  \textbf{D = 10 log(P_{3}/P_{4})} (Directivity)
\end{itemize}

\end{solutionbox}
\begin{mnemonicbox}
``Coupler Couples Carefully in Correct Direction''

\end{mnemonicbox}
\begin{center}\rule{0.5\linewidth}{0.5pt}\end{center}

\subsection*{પ્રશ્ન 2(ક) OR [7
ગુણ]}\label{uxaaauxab0uxab6uxaa8-2uxa95-or-7-uxa97uxaa3}

\textbf{``Travelling Wave Tube'' ના બાંધકામ અને કાર્યસિદ્ધાંત સમજાવો. તેની
એપ્લિકેશનોની સૂચિ બનાવો.}

\begin{solutionbox}

\textbf{બાંધકામ:}

\begin{itemize}
\tightlist
\item
  \textbf{Electron gun}: Electron beam emit કરે છે
\item
  \textbf{Helix structure}: RF wave ને slow કરે છે
\item
  \textbf{Collector}: Spent electrons collect કરે છે
\item
  \textbf{Magnetic focusing}: Beam ને focused રાખે છે
\end{itemize}

\begin{center}
\textbf{Mermaid Diagram (Code)}
\begin{verbatim}
{Shaded}
{Highlighting}[]
graph LR
    A[Electron Gun] {-{-}{} B[Helix]}
    B {-{-}{} C[Collector]}
    D[RF Input] {-{-}{} B}
    B {-{-}{} E[RF Output]}
    F[Magnetic Field] {-.{-}{} B}
{Highlighting}
{Shaded}
\end{verbatim}
\end{center}

\textbf{કાર્યસિદ્ધાંત:}

\begin{itemize}
\tightlist
\item
  \textbf{Velocity synchronization}: Electron velocity \approx RF wave
  velocity
\item
  \textbf{Energy transfer}: Electrons RF wave ને energy આપે છે
\item
  \textbf{Continuous interaction}: સંપૂર્ણ helix length પર
\end{itemize}

\textbf{ઉપયોગો:}

\begin{itemize}
\tightlist
\item
  \textbf{Satellite communication}: High power amplification
\item
  \textbf{Radar transmitters}: High gain amplification
\item
  \textbf{Electronic warfare}: Jamming systems
\end{itemize}

\end{solutionbox}
\begin{mnemonicbox}
``TWT Transfers Tremendous power Through Travel''

\end{mnemonicbox}
\begin{center}\rule{0.5\linewidth}{0.5pt}\end{center}

\subsection*{પ્રશ્ન 3(અ) [3
ગુણ]}\label{uxaaauxab0uxab6uxaa8-3uxa85-3-uxa97uxaa3}

\textbf{ઉચ્ચ VSWR માપન માટે પરોક્ષ પદ્ધતિ સમજાવો.}

\begin{solutionbox}

\textbf{Indirect Method} \textbf{calibrated attenuator} વાપરીને
\textbf{high VSWR} ને measure કરે છે.

\textbf{પ્રક્રિયા:}

\begin{itemize}
\tightlist
\item
  \textbf{Calibrated attenuator} insert કરો (10-20 dB)
\item
  \textbf{Reduced VSWR} measure કરો (VSWR_{2})
\item
  \textbf{Actual VSWR} calculate કરો: VSWR_{1} = VSWR_{2} \times Attenuator ratio
\end{itemize}

\textbf{ફોર્મ્યુલા}: \textbf{VSWR\_actual = VSWR\_measured \times
10\^{}(Attenuation/20)}

\end{solutionbox}
\begin{mnemonicbox}
``Indirect method uses Intermediate Attenuation''

\end{mnemonicbox}
\begin{center}\rule{0.5\linewidth}{0.5pt}\end{center}

\subsection*{પ્રશ્ન 3(બ) [4
ગુણ]}\label{uxaaauxab0uxab6uxaa8-3uxaac-4-uxa97uxaa3}

\textbf{કનવેંશનલ ટ્યૂબ્સની આવર્તન મર્યાદાઓ લખો અને સમજાવો.}

\begin{solutionbox}

\textbf{આવર્તન મર્યાદાઓ:}

\begin{itemize}
\tightlist
\item
  \textbf{Transit time effect}: Electron transit time significant બને છે
\item
  \textbf{Interelectrode capacitance}: High frequency response limit કરે
  છે
\item
  \textbf{Lead inductance}: Parasitic inductance gain ઘટાડે છે
\item
  \textbf{Skin effect}: Current માત્ર surface પર વહે છે
\end{itemize}

\textbf{અસરો:}

\begin{itemize}
\tightlist
\item
  \textbf{Reduced gain}: fα કરતાં વધારે frequencies પર
\item
  \textbf{Increased noise}: Shot noise ને કારણે
\item
  \textbf{Phase shift}: Signal processing માં delay
\end{itemize}

\textbf{ઉકેલો:}

\begin{itemize}
\tightlist
\item
  \textbf{Electrode spacing ઘટાડો}
\item
  \textbf{Special tube designs વાપરો}
\item
  \textbf{Cavity resonators employ કરો}
\end{itemize}

\end{solutionbox}
\begin{mnemonicbox}
``Transit Time Troubles Traditional Tubes''

\end{mnemonicbox}
\begin{center}\rule{0.5\linewidth}{0.5pt}\end{center}

\subsection*{પ્રશ્ન 3(ક) [7
ગુણ]}\label{uxaaauxab0uxab6uxaa8-3uxa95-7-uxa97uxaa3}

\textbf{એપ્લિગેટ ડાયાગ્રામ સાથે ટૂ કેવિટી ક્લીસ્ટ્રોનનું બાંધકામ અને કાર્ય સમજાવો. તેના
ફાયદાઓની યાદી આપો.}

\begin{solutionbox}

\textbf{બાંધકામ:}

\begin{itemize}
\tightlist
\item
  \textbf{Electron gun}: Electron beam produce કરે છે
\item
  \textbf{Input cavity}: Beam ને velocity modulate કરે છે
\item
  \textbf{Drift region}: Beam bunching થાય છે
\item
  \textbf{Output cavity}: RF energy extract કરે છે
\item
  \textbf{Collector}: Electrons collect કરે છે
\end{itemize}

\textbf{Applegate Diagram:}

\begin{verbatim}
Distance 
    |
    |     Fast electrons
    |          Medium electrons  
    |             Slow electrons
Time|                 
    ↓        Bunching occurs
    
Input        Drift        Output
Cavity       Space        Cavity
\end{verbatim}

\textbf{કામગીરી:}

\begin{itemize}
\tightlist
\item
  \textbf{Velocity modulation}: Input cavity electron velocity vary કરે છે
\item
  \textbf{Density modulation}: Electrons drift space માં bunch થાય છે
\item
  \textbf{Energy extraction}: Bunched beam output cavity ને energy
  transfer કરે છે
\end{itemize}

\textbf{ફાયદાઓ:}

\begin{itemize}
\tightlist
\item
  \textbf{High power output}: કેટલાક kilowatts
\item
  \textbf{High efficiency}: 40-60\%
\item
  \textbf{Low noise}: Semiconductor devices કરતાં સારી
\item
  \textbf{Stable operation}: Excellent frequency stability
\end{itemize}

\end{solutionbox}
\begin{mnemonicbox}
``Klystron Kicks with Kinetic Bunching''

\end{mnemonicbox}
\begin{center}\rule{0.5\linewidth}{0.5pt}\end{center}

\subsection*{પ્રશ્ન 3(અ) OR [3
ગુણ]}\label{uxaaauxab0uxab6uxaa8-3uxa85-or-3-uxa97uxaa3}

\textbf{BWOનું બાંધકામ અને કાર્ય સમજાવો.}

\begin{solutionbox}

\textbf{BWO (Backward Wave Oscillator)} \textbf{backward wave
interaction} વાપરીને oscillation કરે છે.

\textbf{બાંધકામ:}

\begin{itemize}
\tightlist
\item
  \textbf{Electron gun}: Electron beam emit કરે છે
\item
  \textbf{Slow wave structure}: Helix અથવા coupled cavities
\item
  \textbf{Collector}: Input end પર
\item
  \textbf{Output}: Input end થી
\end{itemize}

\textbf{કામગીરી:}

\begin{itemize}
\tightlist
\item
  \textbf{Backward wave}: Electron beam ની વિરુદ્ધ દિશામાં travel કરે છે
\item
  \textbf{Negative resistance}: Beam backward wave ને energy આપે છે
\item
  \textbf{Oscillation}: જ્યારે gain \textgreater{} losses
\end{itemize}

\end{solutionbox}
\begin{mnemonicbox}
``BWO goes Backward While Oscillating''

\end{mnemonicbox}
\begin{center}\rule{0.5\linewidth}{0.5pt}\end{center}

\subsection*{પ્રશ્ન 3(બ) OR [4
ગુણ]}\label{uxaaauxab0uxab6uxaa8-3uxaac-or-4-uxa97uxaa3}

\textbf{માઇક્રોવેવ રેડિયેશનને કારણે જોખમો સમજાવો.}

\begin{solutionbox}

\textbf{જોખમોના પ્રકારો:}

\begin{itemize}
\tightlist
\item
  \textbf{HERP}: Hazards of Electromagnetic Radiation to Personnel
\item
  \textbf{HERO}: Hazards of Electromagnetic Radiation to Ordnance\\
\item
  \textbf{HERF}: Hazards of Electromagnetic Radiation to Fuel
\end{itemize}

\textbf{અસરો:}

\begin{itemize}
\tightlist
\item
  \textbf{Thermal heating}: High power પર tissue heating
\item
  \textbf{આંખોને નુકસાન}: Cataract formation
\item
  \textbf{Reproductive effects}: Fertility પર સંભવિત અસર
\item
  \textbf{Pacemaker interference}: Electronic device malfunction
\end{itemize}

\textbf{સુરક્ષા:}

\begin{itemize}
\tightlist
\item
  \textbf{Power density limits}: \textless{} 10 mW/cm^{2}
\item
  \textbf{Safety distances}: Far field calculations
\item
  \textbf{Warning signs}: Radiation hazard markers
\item
  \textbf{Personal monitors}: RF exposure meters
\end{itemize}

\end{solutionbox}
\begin{mnemonicbox}
``Microwaves Make Multiple Medical Maladies''

\end{mnemonicbox}
\begin{center}\rule{0.5\linewidth}{0.5pt}\end{center}

\subsection*{પ્રશ્ન 3(ક) OR [7
ગુણ]}\label{uxaaauxab0uxab6uxaa8-3uxa95-or-7-uxa97uxaa3}

\textbf{સુઘડ સ્કેચ સાથે મેગ્નેટ્રોનનું બાંધકામ અને કાર્ય સમજાવો. તેની એપ્લિકેશનોની સૂચિ
બનાવો.}

\begin{solutionbox}

\textbf{બાંધકામ:}

\begin{itemize}
\tightlist
\item
  \textbf{Circular cathode}: Central hot cathode
\item
  \textbf{Cylindrical anode}: Resonant cavities સાથે
\item
  \textbf{Permanent magnet}: Axial magnetic field પ્રદાન કરે છે
\item
  \textbf{Output coupling}: Loop અથવા probe
\end{itemize}

\begin{center}
\textbf{Mermaid Diagram (Code)}
\begin{verbatim}
{Shaded}
{Highlighting}[]
graph LR
    A[Cathode] {-{-}{} B[Interaction Space]}
    B {-{-}{} C[Anode Cavities]}
    D[Magnetic Field] {-.{-}{} B}
    C {-{-}{} E[Output Coupling]}
    style A fill:\#ff9999
    style C fill:\#99ff99
{Highlighting}
{Shaded}
\end{verbatim}
\end{center}

\textbf{કામગીરી:}

\begin{itemize}
\tightlist
\item
  \textbf{Electron cloud}: Interaction space માં બને છે
\item
  \textbf{Cycloid motion}: E અને B fields ને કારણે
\item
  \textbf{Resonant cavities}: Operating frequency નક્કી કરે છે
\item
  \textbf{π-mode oscillation}: Alternate cavities opposite phase માં
\end{itemize}

\textbf{ઉપયોગો:}

\begin{itemize}
\tightlist
\item
  \textbf{Microwave ovens}: 2.45 GHz heating
\item
  \textbf{Radar systems}: High power pulses
\item
  \textbf{Industrial heating}: Material processing
\item
  \textbf{Medical diathermy}: Therapeutic heating
\end{itemize}

\end{solutionbox}
\begin{mnemonicbox}
``Magnetron Makes Microwaves Magnificently''

\end{mnemonicbox}
\begin{center}\rule{0.5\linewidth}{0.5pt}\end{center}

\subsection*{પ્રશ્ન 4(અ) [3
ગુણ]}\label{uxaaauxab0uxab6uxaa8-4uxa85-3-uxa97uxaa3}

\textbf{P-i-N ડાયોડની કામગીરી સમજાવો.}

\begin{solutionbox}

\textbf{P-i-N Diode} માં P અને N regions વચ્ચે \textbf{intrinsic layer} છે, જે
\textbf{voltage-controlled resistor} તરીકે કામ કરે છે.

\textbf{બાંધકામ:}

\begin{itemize}
\tightlist
\item
  \textbf{P region}: Heavily doped
\item
  \textbf{I region}: Intrinsic (undoped)\\
\item
  \textbf{N region}: Heavily doped
\end{itemize}

\textbf{કામગીરી:}

\begin{itemize}
\tightlist
\item
  \textbf{Forward bias}: Low resistance (1-10 Ω)
\item
  \textbf{Reverse bias}: High resistance (\textgreater10 kΩ)
\item
  \textbf{RF switch}: Microwave signals control કરે છે
\item
  \textbf{Variable attenuator}: DC bias સાથે resistance vary થાય છે
\end{itemize}

\end{solutionbox}
\begin{mnemonicbox}
``PIN controls Power IN Networks''

\end{mnemonicbox}
\begin{center}\rule{0.5\linewidth}{0.5pt}\end{center}

\subsection*{પ્રશ્ન 4(બ) [4
ગુણ]}\label{uxaaauxab0uxab6uxaa8-4uxaac-4-uxa97uxaa3}

\textbf{સુઘડ સ્કેચ સાથે વેરેક્ટર ડાયોડના કાર્ય સમજાવો.}

\begin{solutionbox}

\textbf{Varactor Diode} junction capacitance variation વાપરીને
\textbf{voltage-controlled capacitor} તરીકે કામ કરે છે.

\begin{verbatim}
    +V
     |
  ┌──┴──┐
  │  P  │  N  │  Junction
  └──┬──┘
     |
     0V
     
Capacitance vs Voltage:
C |    
  |{    }
  | {   }
  |  {  }
  |\_\_\_{\_\_\_\_}
    0  {-V (reverse bias)}
\end{verbatim}

\textbf{કામગીરી:}

\begin{itemize}
\tightlist
\item
  \textbf{Reverse bias}: Junction deplete કરે છે, capacitance ઘટે છે
\item
  \textbf{Bias voltage}: Capacitance value control કરે છે
\item
  \textbf{Capacitance ratio}: સામાન્ય રીતે 3:1 થી 10:1
\item
  \textbf{Frequency tuning}: Oscillators અને filters માં વપરાય છે
\end{itemize}

\textbf{ઉપયોગો:}

\begin{itemize}
\tightlist
\item
  \textbf{VCO tuning}: Voltage controlled oscillators
\item
  \textbf{AFC circuits}: Automatic frequency control
\item
  \textbf{Parametric amplifiers}: Low noise amplification
\end{itemize}

\end{solutionbox}
\begin{mnemonicbox}
``Varactor Varies Capacitance with Voltage''

\end{mnemonicbox}
\begin{center}\rule{0.5\linewidth}{0.5pt}\end{center}

\subsection*{પ્રશ્ન 4(ક) [7
ગુણ]}\label{uxaaauxab0uxab6uxaa8-4uxa95-7-uxa97uxaa3}

\textbf{ટનલ ડાયોડનું બાંધકામ અને કાર્ય સમજાવો અને ટનલ બનાવવાની ઘટનાને વિગતવાર
સમજાવો. તેની એપ્લિકેશનોની સૂચિ બનાવો.}

\begin{solutionbox}

\textbf{બાંધકામ:}

\begin{itemize}
\tightlist
\item
  \textbf{Heavily doped P-N junction}: બંને બાજુ degenerately doped
\item
  \textbf{Thin junction}: \textasciitilde10 nm width
\item
  \textbf{Quantum tunneling}: Electrons energy barrier માંથી tunnel કરે છે
\end{itemize}

\textbf{Tunneling Phenomenon:}

\begin{itemize}
\tightlist
\item
  \textbf{Quantum effect}: Electrons energy barrier માંથી પસાર થાય છે
\item
  \textbf{Band overlap}: Conduction band valence band સાથે overlap કરે છે
\item
  \textbf{Probability function}: Tunneling probability barrier width પર
  depend કરે છે
\item
  \textbf{No thermal activation}: Room temperature પર થાય છે
\end{itemize}

\begin{verbatim}
I{-V Characteristic:}
I |   
  |  /{     Negative resistance}
  | /  {   }
  |/    {  }
  |      {\_\_\_}
  |\_\_\_\_\_\_\_\_\_\_\_\_ V
    0  Vp  Vv
    
Vp = Peak voltage
Vv = Valley voltage
\end{verbatim}

\textbf{કામગીરી:}

\begin{itemize}
\tightlist
\item
  \textbf{Forward bias 0-Vp}: Current વધે છે (tunneling)
\item
  \textbf{Vp to Vv}: Negative resistance region
\item
  \textbf{Beyond Vv}: Normal diode operation
\end{itemize}

\textbf{ઉપયોગો:}

\begin{itemize}
\tightlist
\item
  \textbf{High-speed switching}: Picosecond switching
\item
  \textbf{Oscillators}: Microwave frequency generation
\item
  \textbf{Amplifiers}: Low noise amplification
\item
  \textbf{Memory circuits}: Bistable operation
\end{itemize}

\end{solutionbox}
\begin{mnemonicbox}
``Tunnel Diode Tunnels Through barriers
Terrifically''

\end{mnemonicbox}
\begin{center}\rule{0.5\linewidth}{0.5pt}\end{center}

\subsection*{પ્રશ્ન 4(અ) OR [3
ગુણ]}\label{uxaaauxab0uxab6uxaa8-4uxa85-or-3-uxa97uxaa3}

\textbf{IMPATT ડાયોડની કામગીરીનું વર્ણન કરો.}

\begin{solutionbox}

\textbf{IMPATT (Impact Avalanche Transit Time)} diode \textbf{avalanche
multiplication} અને \textbf{transit time delay} વાપરીને oscillation કરે છે.

\textbf{કામગીરી:}

\begin{itemize}
\tightlist
\item
  \textbf{Avalanche zone}: Impact ionization carriers બનાવે છે
\item
  \textbf{Drift zone}: Carriers constant velocity સાથે drift કરે છે
\item
  \textbf{Transit time}: 180^\circ phase shift પ્રદાન કરે છે
\item
  \textbf{Negative resistance}: Phase delay ને કારણે
\end{itemize}

\textbf{મુખ્ય parameters:}

\begin{itemize}
\tightlist
\item
  \textbf{Breakdown voltage}: સામાન્ય રીતે 20-100V
\item
  \textbf{Efficiency}: 10-20\%
\item
  \textbf{Frequency range}: 1-300 GHz
\end{itemize}

\end{solutionbox}
\begin{mnemonicbox}
``IMPATT Impacts with Avalanche Transit Time''

\end{mnemonicbox}
\begin{center}\rule{0.5\linewidth}{0.5pt}\end{center}

\subsection*{પ્રશ્ન 4(બ) OR [4
ગુણ]}\label{uxaaauxab0uxab6uxaa8-4uxaac-or-4-uxa97uxaa3}

\textbf{પેરામેટ્રિક એમ્પ્લીફાયર માટે આવર્તન ઉપર અને નીચે રૂપાંતરણ સમજાવો.}

\begin{solutionbox}

\textbf{Parametric Amplifier} \textbf{time-varying reactance} વાપરીને
amplification અને frequency conversion કરે છે.

\textbf{Up-conversion:}

\begin{itemize}
\tightlist
\item
  \textbf{Signal frequency}: fs (input)
\item
  \textbf{Pump frequency}: fp (ઘણી વધારે)
\item
  \textbf{Output frequency}: fo = fp + fs
\item
  \textbf{Energy transfer}: Pump થી signal માં
\end{itemize}

\textbf{Down-conversion:}

\begin{itemize}
\tightlist
\item
  \textbf{Signal frequency}: fs (input)\\
\item
  \textbf{Pump frequency}: fp
\item
  \textbf{Output frequency}: fo = fp - fs
\item
  \textbf{Mixer operation}: Frequency translation
\end{itemize}

\textbf{ફાયદાઓ:}

\begin{itemize}
\tightlist
\item
  \textbf{Low noise}: Quantum-limited performance
\item
  \textbf{High gain}: 20-30 dB સામાન્ય
\item
  \textbf{Wide bandwidth}: કેટલાક GHz
\end{itemize}

\end{solutionbox}
\begin{mnemonicbox}
``Parametric Pump Provides frequency conversion Plus
gain''

\end{mnemonicbox}
\begin{center}\rule{0.5\linewidth}{0.5pt}\end{center}

\subsection*{પ્રશ્ન 4(ક) OR [7
ગુણ]}\label{uxaaauxab0uxab6uxaa8-4uxa95-or-7-uxa97uxaa3}

\textbf{RUBY MASER ના બાંધકામ અને કાર્ય સિદ્ધાંતનું વર્ણન કરો. તેની એપ્લિકેશનોની
સૂચિ બનાવો.}

\begin{solutionbox}

\textbf{બાંધકામ:}

\begin{itemize}
\tightlist
\item
  \textbf{Ruby crystal}: Al_{2}O_{3} lattice માં Cr^{3}^{+} ions
\item
  \textbf{Magnetic field}: Strong DC magnetic field
\item
  \textbf{Microwave cavity}: Signal frequency પર resonant
\item
  \textbf{Pump source}: High frequency klystron
\item
  \textbf{Cryogenic cooling}: Liquid helium temperature
\end{itemize}

\begin{center}
\textbf{Mermaid Diagram (Code)}
\begin{verbatim}
{Shaded}
{Highlighting}[]
graph LR
    A[Ruby Crystal] {-{-}{} B[Microwave Cavity]}
    C[Magnetic Field] {-.{-}{} A}
    D[Pump Source] {-{-}{} B}
    E[Liquid Helium] {-.{-}{} A}
    B {-{-}{} F[Amplified Output]}
{Highlighting}
{Shaded}
\end{verbatim}
\end{center}

\textbf{કાર્યસિદ્ધાંત:}

\begin{itemize}
\tightlist
\item
  \textbf{Energy levels}: Cr^{3}^{+} ions ને ત્રણ energy levels છે
\item
  \textbf{Population inversion}: Pump upper level માં વધારે atoms બનાવે છે
\item
  \textbf{Stimulated emission}: Signal photons emission trigger કરે છે
\item
  \textbf{Coherent amplification}: Phase-coherent amplification
\end{itemize}

\textbf{Three-level system:}

\begin{itemize}
\tightlist
\item
  \textbf{Ground state}: E_{1} (સૌથી વધારે populated)
\item
  \textbf{Intermediate state}: E_{2} (signal frequency)
\item
  \textbf{Upper state}: E_{3} (pump frequency)
\end{itemize}

\textbf{ઉપયોગો:}

\begin{itemize}
\tightlist
\item
  \textbf{Radio astronomy}: Ultra-low noise receivers
\item
  \textbf{Satellite communication}: Ground station amplifiers
\item
  \textbf{Deep space communication}: NASA tracking stations
\item
  \textbf{Research}: Quantum electronics experiments
\end{itemize}

\end{solutionbox}
\begin{mnemonicbox}
``RUBY MASER Makes Amazingly Sensitive
Electromagnetic Receivers''

\end{mnemonicbox}
\begin{center}\rule{0.5\linewidth}{0.5pt}\end{center}

\subsection*{પ્રશ્ન 5(અ) [3
ગુણ]}\label{uxaaauxab0uxab6uxaa8-5uxa85-3-uxa97uxaa3}

\textbf{MTI RADARના કાર્યાત્મક બ્લોક ડાયાગ્રામ દોરો અને સમજાવો.}

\begin{solutionbox}

\textbf{MTI RADAR} \textbf{successive echoes} ની comparison કરીને
\textbf{moving targets} detect કરે છે અને fixed targets cancel કરે છે.

\begin{center}
\textbf{Mermaid Diagram (Code)}
\begin{verbatim}
{Shaded}
{Highlighting}[]
graph LR
    A[Transmitter] {-{-}{} B[Duplexer]}
    B {-{-}{} C[Antenna]}
    C {-{-}{} B}
    B {-{-}{} D[Receiver]}
    D {-{-}{} E[Phase Detector]}
    F[STALO] {-{-}{} E}
    F {-{-}{} G[COHO]}
    G {-{-}{} E}
    E {-{-}{} H[Canceller]}
    H {-{-}{} I[Display]}
{Highlighting}
{Shaded}
\end{verbatim}
\end{center}

\textbf{Components:}

\begin{itemize}
\tightlist
\item
  \textbf{STALO}: Stable Local Oscillator
\item
  \textbf{COHO}: Coherent Oscillator\\
\item
  \textbf{Phase detector}: Echo phases compare કરે છે
\item
  \textbf{Canceller}: Fixed target echoes remove કરે છે
\end{itemize}

\end{solutionbox}
\begin{mnemonicbox}
``MTI Makes Targets Intelligible by Motion''

\end{mnemonicbox}
\begin{center}\rule{0.5\linewidth}{0.5pt}\end{center}

\subsection*{પ્રશ્ન 5(બ) [4
ગુણ]}\label{uxaaauxab0uxab6uxaa8-5uxaac-4-uxa97uxaa3}

\textbf{RADAR ને SONAR સાથે સરખાવો.}

\begin{solutionbox}

{\def\LTcaptype{none} % do not increment counter
\begin{longtable}[]{@{}lll@{}}
\toprule\noalign{}
પેરામીટર & RADAR & SONAR \\
\midrule\noalign{}
\endhead
\bottomrule\noalign{}
\endlastfoot
\textbf{Wave type} & Electromagnetic & Acoustic \\
\textbf{Medium} & Air/vacuum & Water \\
\textbf{Speed} & 3\times10^{8} m/s & 1500 m/s \\
\textbf{Frequency} & GHz & kHz \\
\textbf{Range} & 100+ km & 10-50 km \\
\textbf{Applications} & Air/space & Underwater \\
\end{longtable}
}

\textbf{સામાન્ય લક્ષણો:}

\begin{itemize}
\tightlist
\item
  \textbf{Pulse-echo principle}
\item
  \textbf{Range measurement}
\item
  \textbf{Target detection}
\end{itemize}

\end{solutionbox}
\begin{mnemonicbox}
``RADAR Radiates, SONAR Sounds''

\end{mnemonicbox}
\begin{center}\rule{0.5\linewidth}{0.5pt}\end{center}

\subsection*{પ્રશ્ન 5(ક) [7
ગુણ]}\label{uxaaauxab0uxab6uxaa8-5uxa95-7-uxa97uxaa3}

\textbf{મહત્તમ RADAR રેંજનું સમીકરણ મેળવો. મહત્તમ રડાર રેંજને અસર કરતા પરિબળો
સમજાવો.}

\begin{solutionbox}

\textbf{RADAR Range Equation:}

\textbf{R\_max = ^{4}\sqrt[(P\_t \times G^{2} \times λ^{2} \times σ) / (64π^{3} \times P\_min \times L)]}

જ્યાં:

\begin{itemize}
\tightlist
\item
  \textbf{P\_t}: Transmitter power (W)
\item
  \textbf{G}: Antenna gain (dimensionless)
\item
  \textbf{λ}: Wavelength (m)
\item
  \textbf{σ}: Target cross-section (m^{2})
\item
  \textbf{P\_min}: Minimum detectable power (W)
\item
  \textbf{L}: System losses (dimensionless)
\end{itemize}

\textbf{Derivation steps:}

\begin{enumerate}
\tightlist
\item
  \textbf{Power density at target}: P\_t\timesG/(4πR^{2})
\item
  \textbf{Power intercepted}: σ \times Power density
\item
  \textbf{Power at receiver}: Intercepted power \times G/(4πR^{2})
\item
  \textbf{P\_min સાથે સમાન કરો} અને R માટે solve કરો
\end{enumerate}

\textbf{Range ને અસર કરતા પરિબળો:}

\textbf{Range વધારતા પરિબળો:}

\begin{itemize}
\tightlist
\item
  \textbf{Higher transmitter power}: R ∝ P\_t\^{}(1/4)
\item
  \textbf{Larger antenna gain}: R ∝ G\^{}(1/2)
\item
  \textbf{Larger target RCS}: R ∝ σ\^{}(1/4)
\item
  \textbf{Lower system losses}: R ∝ L\^{}(-1/4)
\end{itemize}

\textbf{Range ઘટાડતા પરિબળો:}

\begin{itemize}
\tightlist
\item
  \textbf{Higher frequency}: R ∝ λ\^{}(1/2)
\item
  \textbf{Atmospheric losses}: Absorption અને scattering
\item
  \textbf{Ground clutter}: Interfering reflections
\end{itemize}

\end{solutionbox}
\begin{mnemonicbox}
``RADAR Range Requires Robust Power and Proper
Parameters''

\end{mnemonicbox}
\begin{center}\rule{0.5\linewidth}{0.5pt}\end{center}

\subsection*{પ્રશ્ન 5(અ) OR [3
ગુણ]}\label{uxaaauxab0uxab6uxaa8-5uxa85-or-3-uxa97uxaa3}

\textbf{CW Doppler RADAR માં ડોપ્લર અસરનું વર્ણન કરો.}

\begin{solutionbox}

\textbf{Doppler Effect} જ્યારે target RADAR ની સાપેક્ષ રીતે move કરે છે ત્યારે
\textbf{frequency shift} કરે છે.

\textbf{Doppler Frequency:} \textbf{f\_d = (2 \times V\_r \times f\_0) / c}

જ્યાં:

\begin{itemize}
\tightlist
\item
  \textbf{V\_r}: Radial velocity (m/s)
\item
  \textbf{f\_0}: Transmitted frequency (Hz)
\item
  \textbf{c}: Speed of light (3\times10^{8} m/s)
\end{itemize}

\textbf{લક્ષણો:}

\begin{itemize}
\tightlist
\item
  \textbf{Approaching target}: f\_d positive
\item
  \textbf{Receding target}: f\_d negative
\item
  \textbf{Factor of 2}: Two-way propagation ને કારણે
\end{itemize}

\end{solutionbox}
\begin{mnemonicbox}
``Doppler Detects Direction with Doubled frequency
shift''

\end{mnemonicbox}
\begin{center}\rule{0.5\linewidth}{0.5pt}\end{center}

\subsection*{પ્રશ્ન 5(બ) OR [4
ગુણ]}\label{uxaaauxab0uxab6uxaa8-5uxaac-or-4-uxa97uxaa3}

\textbf{RADAR માટે PPI ડિસ્પ્લે પદ્ધતિ સમજાવો}

\begin{solutionbox}

\textbf{PPI (Plan Position Indicator)} RADAR coverage area નો
\textbf{top view} બતાવે છે range અને bearing information સાથે.

\textbf{Display Features:}

\begin{itemize}
\tightlist
\item
  \textbf{Circular screen}: Center RADAR location represent કરે છે
\item
  \textbf{Rotating trace}: Antenna rotation સાથે synchronized
\item
  \textbf{Range rings}: Distance માટે concentric circles
\item
  \textbf{Bearing scale}: Circumference આસપાસ 0-360^\circ
\end{itemize}

\textbf{કામગીરી:}

\begin{itemize}
\tightlist
\item
  \textbf{Sweep rotation}: Antenna rotation match કરે છે
\item
  \textbf{Echo intensity}: Brightness control કરે છે
\item
  \textbf{Persistence}: Afterglow target visibility maintain કરે છે
\item
  \textbf{Range scale}: Selectable range settings
\end{itemize}

\textbf{ઉપયોગો:}

\begin{itemize}
\tightlist
\item
  \textbf{Air traffic control}: Aircraft positioning
\item
  \textbf{Marine navigation}: Ship અને obstacle detection
\item
  \textbf{Weather monitoring}: Storm tracking
\end{itemize}

\end{solutionbox}
\begin{mnemonicbox}
``PPI Provides Position Information Perfectly''

\end{mnemonicbox}
\begin{center}\rule{0.5\linewidth}{0.5pt}\end{center}

\subsection*{પ્રશ્ન 5(ક) OR [7
ગુણ]}\label{uxaaauxab0uxab6uxaa8-5uxa95-or-7-uxa97uxaa3}

\textbf{પલ્સ રડારનો બ્લોક ડાયાગ્રામ દોરો અને કાર્યસિદ્ધાંત સમજાવો.}

\begin{solutionbox}

\begin{center}
\textbf{Mermaid Diagram (Code)}
\begin{verbatim}
{Shaded}
{Highlighting}[]
graph LR
    A[Master Oscillator] {-{-}{} B[Modulator]}
    B {-{-}{} C[Power Amplifier]}
    C {-{-}{} D[Duplexer]}
    D {-{-}{} E[Antenna]}
    E {-{-}{} D}
    D {-{-}{} F[RF Amplifier]}
    F {-{-}{} G[Mixer]}
    H[Local Oscillator] {-{-}{} G}
    G {-{-}{} I[IF Amplifier]}
    I {-{-}{} J[Detector]}
    J {-{-}{} K[Video Amplifier]}
    K {-{-}{} L[Display]}
    A {-{-}{} M[Timer]}
    M {-{-}{} B}
    M {-{-}{} L}
{Highlighting}
{Shaded}
\end{verbatim}
\end{center}

\textbf{કاર્યસિદ્ધાંત:}

\textbf{Transmission:}

\begin{itemize}
\tightlist
\item
  \textbf{Master oscillator}: RF carrier generate કરે છે
\item
  \textbf{Modulator}: Short pulses બનાવે છે
\item
  \textbf{Power amplifier}: Pulse power amplify કરે છે
\item
  \textbf{Duplexer}: Pulse ને antenna તરફ route કરે છે
\end{itemize}

\textbf{Reception:}

\begin{itemize}
\tightlist
\item
  \textbf{Echo reception}: Antenna reflected signals receive કરે છે
\item
  \textbf{RF amplification}: Low noise amplification
\item
  \textbf{Mixing}: Intermediate frequency માં convert કરે છે
\item
  \textbf{IF amplification}: Further amplification
\item
  \textbf{Detection}: Video signal extract કરે છે
\item
  \textbf{Display}: Range vs amplitude show કરે છે
\end{itemize}

\textbf{મુખ્ય Parameters:}

\begin{itemize}
\tightlist
\item
  \textbf{Pulse width}: Range resolution નક્કી કરે છે
\item
  \textbf{PRF}: Pulse repetition frequency
\item
  \textbf{Peak power}: Maximum range capability
\item
  \textbf{Duty cycle}: Average power consideration
\end{itemize}

\textbf{ફાયદાઓ:}

\begin{itemize}
\tightlist
\item
  \textbf{High peak power}: Long range capability
\item
  \textbf{Good range resolution}: Narrow pulses
\item
  \textbf{Simple processing}: Direct detection
\end{itemize}

\end{solutionbox}
\begin{mnemonicbox}
``Pulse RADAR Pulses Powerfully for Precise
Position''

\end{mnemonicbox}

\end{document}
