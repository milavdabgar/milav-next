\documentclass[10pt,a4paper]{article}

% content/resources/templates/preamble.tex
\usepackage[margin=0.6in]{geometry}
\author{Milav Dabgar}
\usepackage{amsmath,amssymb,amsthm}
\usepackage{booktabs}
\usepackage{multirow}
\usepackage{xcolor}
\usepackage{tcolorbox}
\tcbuselibrary{breakable,skins}
\usepackage[colorlinks=true,linkcolor=blue]{hyperref}
\usepackage{titlesec}
\usepackage{enumitem}
\usepackage{tikz}
\usepackage{pgfplots}
\usepackage{circuitikz}
\usepackage[version=4]{mhchem}
\usepackage{longtable}
\usepackage{array}
\usepackage{float}
\usepackage{caption}
\usepackage{listings}

\lstset{
  basicstyle=\small\ttfamily,
  breaklines=true,
  breakatwhitespace=false,
  postbreak=\mbox{\textcolor{red}{$\hookrightarrow$}\space},
  float=false,
  numbers=left,
  numberstyle=\tiny\color{gray},
  numbersep=10pt,
  xleftmargin=2em,
  keywordstyle=\color{blue},
  commentstyle=\color{green!60!black},
  stringstyle=\color{purple},
  backgroundcolor=\color{gray!5},
  showstringspaces=false,
  tabsize=2,
  captionpos=b,
  keepspaces=true,
  columns=flexible
}

\pgfplotsset{compat=1.18}
\usetikzlibrary{shapes,arrows,positioning,calc,patterns,decorations.pathmorphing,decorations.markings,arrows.meta}

% Color scheme
\definecolor{headcolor}{RGB}{0,102,204}
\definecolor{keycolor}{RGB}{220,20,60}
\definecolor{solutioncolor}{RGB}{34,139,34}
\definecolor{mnemoniccolor}{RGB}{148,0,211}
\definecolor{codecolor}{RGB}{0,0,100}

% Spacing
\setlength{\parskip}{3pt}
\setlist[itemize]{nosep}
\setlist[enumerate]{nosep}

% Title formatting
\titleformat{\section}{\Large\bfseries\color{headcolor}}{\thesection}{1em}{}
\titleformat{\subsection}{\large\bfseries\color{headcolor}}{\thesubsection}{1em}{}

% Pandoc tightlist compatibility
\providecommand{\tightlist}{%
  \setlength{\itemsep}{0pt}\setlength{\parskip}{0pt}}

% Pandoc longtable compatibility
\newcounter{none}
\def\thenone{}


% content/resources/templates/english-boxes.tex
% This file is currently empty - it exists to maintain consistency with the import structure.
% Add custom environments here if needed in the future.


\begin{document}

\begin{center}
{\Huge\bfseries\color{headcolor} Subject Name Solutions}\\[5pt]
{\LARGE 4351103 -- Summer 2025}\\[3pt]
{\large Semester 1 Study Material}\\[3pt]
{\normalsize\textit{Detailed Solutions and Explanations}}
\end{center}

\vspace{10pt}

\subsection*{Question 1(a) [3 marks]}\label{q1a}

\textbf{List four microwave frequency bands with their frequency range
and applications.}

\begin{solutionbox}

{\def\LTcaptype{none} % do not increment counter
\begin{longtable}[]{@{}lll@{}}
\toprule\noalign{}
Band & Frequency Range & Applications \\
\midrule\noalign{}
\endhead
\bottomrule\noalign{}
\endlastfoot
\textbf{L-band} & 1-2 GHz & GPS, Mobile communication \\
\textbf{S-band} & 2-4 GHz & WiFi, Bluetooth, Radar \\
\textbf{C-band} & 4-8 GHz & Satellite communication \\
\textbf{X-band} & 8-12 GHz & Military radar, Weather radar \\
\end{longtable}
}

\end{solutionbox}
\begin{mnemonicbox}
``Little Satellites Communicate eXcellently''

\end{mnemonicbox}
\begin{center}\rule{0.5\linewidth}{0.5pt}\end{center}

\subsection*{Question 1(b) [4 marks]}\label{q1b}

\textbf{Explain the impedance matching process using a single stub.}

\begin{solutionbox}

\textbf{Single stub matching} removes reflections by adding a
\textbf{short-circuited stub} at specific distance from load.

\textbf{Process:}

\begin{itemize}
\tightlist
\item
  \textbf{Stub length}: Provides reactive impedance
\item
  \textbf{Stub position}: Calculated from load using Smith chart
\item
  \textbf{Matching condition}: Real part = Z_{0}, imaginary part = 0
\end{itemize}

\begin{center}
\textbf{Mermaid Diagram (Code)}
\begin{verbatim}
{Shaded}
{Highlighting}[]
graph LR
    A[Source] {-{-}{} B[Transmission Line]}
    B {-{-}{} C[Stub Position]}
    C {-{-}{} D[Load]}
    C {-{-}{} E[Short Stub]}
{Highlighting}
{Shaded}
\end{verbatim}
\end{center}

\end{solutionbox}
\begin{mnemonicbox}
``Stub Positioned for Perfect Matching''

\end{mnemonicbox}
\begin{center}\rule{0.5\linewidth}{0.5pt}\end{center}

\subsection*{Question 1(c) [7 marks]}\label{q1c}

\textbf{State characteristics of lossless transmission line and obtain
the general equation for a two-wire transmission line.}

\begin{solutionbox}

\textbf{Characteristics of Lossless Line:}

\begin{itemize}
\tightlist
\item
  \textbf{No power loss}: R = 0, G = 0
\item
  \textbf{Constant amplitude}: No attenuation
\item
  \textbf{Phase delay only}: Signal delayed but not weakened
\item
  \textbf{Standing wave pattern}: Due to reflections
\end{itemize}

\textbf{General Equations:}

For voltage: \textbf{V(z) = V_{+}e\^{}(-γz) + V_{-}e\^{}(γz)} For current:
\textbf{I(z) = (V_{+}/Z_{0})e\^{}(-γz) - (V_{-}/Z_{0})e\^{}(γz)}

Where:

\begin{itemize}
\tightlist
\item
  \textbf{γ = α + jβ} (propagation constant)
\item
  \textbf{Z_{0} = \sqrt(L/C)} (characteristic impedance)
\item
  \textbf{For lossless line}: α = 0, γ = jβ
\end{itemize}

\end{solutionbox}
\begin{mnemonicbox}
``Lossless Lines Love Low Loss''

\end{mnemonicbox}
\begin{center}\rule{0.5\linewidth}{0.5pt}\end{center}

\subsection*{Question 1(c) OR [7
marks]}\label{q1c}

\textbf{Define standing wave. Draw and explain the standing wave pattern
for short circuit and open circuit line.}

\begin{solutionbox}

\textbf{Standing Wave:} Fixed pattern formed by \textbf{forward and
reflected waves} interfering constructively and destructively.

\textbf{Short Circuit Line:}

\begin{itemize}
\tightlist
\item
  \textbf{Current maximum} at short circuit
\item
  \textbf{Voltage minimum} at short circuit
\item
  \textbf{Distance between minima}: λ/2
\end{itemize}

\textbf{Open Circuit Line:}

\begin{itemize}
\tightlist
\item
  \textbf{Voltage maximum} at open circuit
\item
  \textbf{Current minimum} at open circuit
\item
  \textbf{Distance between maxima}: λ/2
\end{itemize}

\begin{verbatim}
Short Circuit:     Open Circuit:
                  
V |    /{            V |  /    /}
  |   /  {             |/    /    }
  |  /    {            |           }
  |\_/\_\_\_\_\_\_{           |\_\_\_\_\_\_\_\_\_\_\_\_}
    0  λ/4  λ/2         0  λ/4  λ/2
    
I |  /{    /        I |    /}
  | /  {  /           |   /  }
  |/    {/            |  /    }
  |             {      |\_/\_\_\_\_\_\_}
    0  λ/4  λ/2         0  λ/4  λ/2
\end{verbatim}

\end{solutionbox}
\begin{mnemonicbox}
``Short Circuits Current, Open Circuits Voltage''

\end{mnemonicbox}
\begin{center}\rule{0.5\linewidth}{0.5pt}\end{center}

\subsection*{Question 2(a) [3 marks]}\label{q2a}

\textbf{Draw and explain the working of Magic TEE.}

\begin{solutionbox}

\textbf{Magic TEE} combines E-plane and H-plane tees with \textbf{four
ports} providing isolation between opposite ports.

\begin{center}
\textbf{Mermaid Diagram (Code)}
\begin{verbatim}
{Shaded}
{Highlighting}[]
graph TD
    A[Port 1 {- E{-}arm] {-}{-}{} C[Junction]}
    B[Port 2 {- H{-}arm] {-}{-}{} C}
    C {-{-}{} D[Port 3 {-} Collinear arm]}
    C {-{-}{} E[Port 4 {-} Collinear arm]}
{Highlighting}
{Shaded}
\end{verbatim}
\end{center}

\textbf{Working:}

\begin{itemize}
\tightlist
\item
  \textbf{E-arm and H-arm}: Isolated from each other
\item
  \textbf{Sum port}: Adds signals from collinear arms
\item
  \textbf{Difference port}: Subtracts signals
\end{itemize}

\end{solutionbox}
\begin{mnemonicbox}
``Magic Tee Mixes Modes''

\end{mnemonicbox}
\begin{center}\rule{0.5\linewidth}{0.5pt}\end{center}

\subsection*{Question 2(b) [4 marks]}\label{q2b}

\textbf{Explain the working of Hybrid ring.}

\begin{solutionbox}

\textbf{Hybrid Ring} is a \textbf{circular waveguide} with \textbf{four
ports} spaced at specific intervals for power division and isolation.

\textbf{Construction:}

\begin{itemize}
\tightlist
\item
  \textbf{Ring circumference}: 1.5λ
\item
  \textbf{Port spacing}: λ/4 between adjacent ports
\item
  \textbf{Matched impedance}: Each port matched to Z_{0}
\end{itemize}

\textbf{Working:}

\begin{itemize}
\tightlist
\item
  \textbf{Power splitting}: Input splits equally between two output
  ports
\item
  \textbf{Isolation}: Opposite ports are isolated
\item
  \textbf{Phase difference}: 180^\circ between output ports
\end{itemize}

\end{solutionbox}
\begin{mnemonicbox}
``Ring Runs Round for Power Sharing''

\end{mnemonicbox}
\begin{center}\rule{0.5\linewidth}{0.5pt}\end{center}

\subsection*{Question 2(c) [7 marks]}\label{q2c}

\textbf{Explain the construction and working principle of
``CIRCULATOR''. List its applications.}

\begin{solutionbox}

\textbf{Construction:}

\begin{itemize}
\tightlist
\item
  \textbf{Three-port device} with \textbf{ferrite material}
\item
  \textbf{Permanent magnet} creates magnetic field
\item
  \textbf{Y-junction waveguide} structure
\end{itemize}

\begin{center}
\textbf{Mermaid Diagram (Code)}
\begin{verbatim}
{Shaded}
{Highlighting}[]
graph LR
    A[Port 1] {-{-}{} B[Ferrite Junction]}
    B {-{-}{} C[Port 2]}
    C {-{-}{} D[Port 3]}
    D {-{-}{} A}
    style B fill:\#ff9999
{Highlighting}
{Shaded}
\end{verbatim}
\end{center}

\textbf{Working Principle:}

\begin{itemize}
\tightlist
\item
  \textbf{Faraday rotation}: Magnetic field rotates wave polarization
\item
  \textbf{Unidirectional flow}: Power flows in one direction only
\item
  \textbf{Non-reciprocal}: Different behavior for opposite directions
\end{itemize}

\textbf{Applications:}

\begin{itemize}
\tightlist
\item
  \textbf{Radar systems}: Isolates transmitter from receiver
\item
  \textbf{Communication}: Antenna sharing for TX/RX
\item
  \textbf{Microwave amplifiers}: Prevents feedback
\end{itemize}

\end{solutionbox}
\begin{mnemonicbox}
``Circulator Circles Clockwise Continuously''

\end{mnemonicbox}
\begin{center}\rule{0.5\linewidth}{0.5pt}\end{center}

\subsection*{Question 2(a) OR [3
marks]}\label{q2a}

\textbf{Compare rectangular waveguide and circular waveguide.}

\begin{solutionbox}

{\def\LTcaptype{none} % do not increment counter
\begin{longtable}[]{@{}lll@{}}
\toprule\noalign{}
Parameter & Rectangular & Circular \\
\midrule\noalign{}
\endhead
\bottomrule\noalign{}
\endlastfoot
\textbf{Cross-section} & Rectangle & Circle \\
\textbf{Dominant mode} & TE_{1}_{0} & TE_{1}_{1} \\
\textbf{Cutoff frequency} & Easy calculation & Complex calculation \\
\textbf{Manufacturing} & Simple & Moderate \\
\textbf{Power handling} & Lower & Higher \\
\end{longtable}
}

\end{solutionbox}
\begin{mnemonicbox}
``Rectangles are Regular, Circles are Complex''

\end{mnemonicbox}
\begin{center}\rule{0.5\linewidth}{0.5pt}\end{center}

\subsection*{Question 2(b) OR [4
marks]}\label{q2b}

\textbf{Draw and explain the working of a directional coupler.}

\begin{solutionbox}

\textbf{Directional Coupler} samples \textbf{forward power} while
providing isolation from reflected power.

\begin{center}
\textbf{Mermaid Diagram (Code)}
\begin{verbatim}
{Shaded}
{Highlighting}[]
graph LR
    A[Input] {-{-}{} B[Main Line]}
    B {-{-}{} C[Output]}
    B {-.{-}{} D[Coupled Port]}
    B {-.{-}{} E[Isolated Port]}
    style D fill:\#99ff99
    style E fill:\#ff9999
{Highlighting}
{Shaded}
\end{verbatim}
\end{center}

\textbf{Working:}

\begin{itemize}
\tightlist
\item
  \textbf{Coupling factor}: Determines power extracted (10-20 dB
  typical)
\item
  \textbf{Directivity}: Isolates forward from reverse power
\item
  \textbf{Insertion loss}: Minimal loss in main line
\end{itemize}

\textbf{Parameters:}

\begin{itemize}
\tightlist
\item
  \textbf{C = 10 log(P_{1}/P_{3})} (Coupling factor)
\item
  \textbf{D = 10 log(P_{3}/P_{4})} (Directivity)
\end{itemize}

\end{solutionbox}
\begin{mnemonicbox}
``Coupler Couples Carefully in Correct Direction''

\end{mnemonicbox}
\begin{center}\rule{0.5\linewidth}{0.5pt}\end{center}

\subsection*{Question 2(c) OR [7
marks]}\label{q2c}

\textbf{Explain the construction and working principle of ``Travelling
Wave Tube''. List its applications.}

\begin{solutionbox}

\textbf{Construction:}

\begin{itemize}
\tightlist
\item
  \textbf{Electron gun}: Emits electron beam
\item
  \textbf{Helix structure}: Slows down RF wave
\item
  \textbf{Collector}: Collects spent electrons
\item
  \textbf{Magnetic focusing}: Keeps beam focused
\end{itemize}

\begin{center}
\textbf{Mermaid Diagram (Code)}
\begin{verbatim}
{Shaded}
{Highlighting}[]
graph LR
    A[Electron Gun] {-{-}{} B[Helix]}
    B {-{-}{} C[Collector]}
    D[RF Input] {-{-}{} B}
    B {-{-}{} E[RF Output]}
    F[Magnetic Field] {-.{-}{} B}
{Highlighting}
{Shaded}
\end{verbatim}
\end{center}

\textbf{Working Principle:}

\begin{itemize}
\tightlist
\item
  \textbf{Velocity synchronization}: Electron velocity \approx RF wave
  velocity
\item
  \textbf{Energy transfer}: Electrons give energy to RF wave
\item
  \textbf{Continuous interaction}: Along entire helix length
\end{itemize}

\textbf{Applications:}

\begin{itemize}
\tightlist
\item
  \textbf{Satellite communication}: High power amplification
\item
  \textbf{Radar transmitters}: High gain amplification
\item
  \textbf{Electronic warfare}: Jamming systems
\end{itemize}

\end{solutionbox}
\begin{mnemonicbox}
``TWT Transfers Tremendous power Through Travel''

\end{mnemonicbox}
\begin{center}\rule{0.5\linewidth}{0.5pt}\end{center}

\subsection*{Question 3(a) [3 marks]}\label{q3a}

\textbf{Explain the Indirect method for higher VSWR measurement.}

\begin{solutionbox}

\textbf{Indirect Method} measures \textbf{high VSWR} by using
\textbf{attenuator} to reduce signal level for accurate measurement.

\textbf{Procedure:}

\begin{itemize}
\tightlist
\item
  \textbf{Insert calibrated attenuator} (10-20 dB)
\item
  \textbf{Measure reduced VSWR} (VSWR_{2})
\item
  \textbf{Calculate actual VSWR}: VSWR_{1} = VSWR_{2} \times Attenuator ratio
\end{itemize}

\textbf{Formula}: \textbf{VSWR\_actual = VSWR\_measured \times
10\^{}(Attenuation/20)}

\end{solutionbox}
\begin{mnemonicbox}
``Indirect method uses Intermediate Attenuation''

\end{mnemonicbox}
\begin{center}\rule{0.5\linewidth}{0.5pt}\end{center}

\subsection*{Question 3(b) [4 marks]}\label{q3b}

\textbf{Write and explain the frequency limitations of conventional
tubes.}

\begin{solutionbox}

\textbf{Frequency Limitations:}

\begin{itemize}
\tightlist
\item
  \textbf{Transit time effect}: Electron transit time becomes
  significant
\item
  \textbf{Interelectrode capacitance}: Limits high frequency response
\item
  \textbf{Lead inductance}: Parasitic inductance reduces gain
\item
  \textbf{Skin effect}: Current flows on surface only
\end{itemize}

\textbf{Effects:}

\begin{itemize}
\tightlist
\item
  \textbf{Reduced gain}: At frequencies above fα
\item
  \textbf{Increased noise}: Due to shot noise
\item
  \textbf{Phase shift}: Delays signal processing
\end{itemize}

\textbf{Solutions:}

\begin{itemize}
\tightlist
\item
  \textbf{Reduce electrode spacing}
\item
  \textbf{Use special tube designs}
\item
  \textbf{Employ cavity resonators}
\end{itemize}

\end{solutionbox}
\begin{mnemonicbox}
``Transit Time Troubles Traditional Tubes''

\end{mnemonicbox}
\begin{center}\rule{0.5\linewidth}{0.5pt}\end{center}

\subsection*{Question 3(c) [7 marks]}\label{q3c}

\textbf{Explain construction and working of Two cavity klystron with
applegate diagram. List its advantages.}

\begin{solutionbox}

\textbf{Construction:}

\begin{itemize}
\tightlist
\item
  \textbf{Electron gun}: Produces electron beam
\item
  \textbf{Input cavity}: Velocity modulates beam
\item
  \textbf{Drift region}: Beam bunching occurs
\item
  \textbf{Output cavity}: Extracts RF energy
\item
  \textbf{Collector}: Collects electrons
\end{itemize}

\textbf{Applegate Diagram:}

\begin{verbatim}
Distance 
    |
    |     Fast electrons
    |          Medium electrons  
    |             Slow electrons
Time|                 
    ↓        Bunching occurs
    
Input        Drift        Output
Cavity       Space        Cavity
\end{verbatim}

\textbf{Working:}

\begin{itemize}
\tightlist
\item
  \textbf{Velocity modulation}: Input cavity varies electron velocity
\item
  \textbf{Density modulation}: Electrons bunch in drift space
\item
  \textbf{Energy extraction}: Bunched beam transfers energy to output
  cavity
\end{itemize}

\textbf{Advantages:}

\begin{itemize}
\tightlist
\item
  \textbf{High power output}: Several kilowatts
\item
  \textbf{High efficiency}: 40-60\%
\item
  \textbf{Low noise}: Better than semiconductor devices
\item
  \textbf{Stable operation}: Excellent frequency stability
\end{itemize}

\end{solutionbox}
\begin{mnemonicbox}
``Klystron Kicks with Kinetic Bunching''

\end{mnemonicbox}
\begin{center}\rule{0.5\linewidth}{0.5pt}\end{center}

\subsection*{Question 3(a) OR [3
marks]}\label{q3a}

\textbf{Explain construction and working of BWO.}

\begin{solutionbox}

\textbf{BWO (Backward Wave Oscillator)} uses \textbf{backward wave
interaction} for oscillation.

\textbf{Construction:}

\begin{itemize}
\tightlist
\item
  \textbf{Electron gun}: Emits electron beam
\item
  \textbf{Slow wave structure}: Helix or coupled cavities
\item
  \textbf{Collector}: At input end
\item
  \textbf{Output}: From input end
\end{itemize}

\textbf{Working:}

\begin{itemize}
\tightlist
\item
  \textbf{Backward wave}: Travels opposite to electron beam
\item
  \textbf{Negative resistance}: Beam provides energy to backward wave
\item
  \textbf{Oscillation}: When gain \textgreater{} losses
\end{itemize}

\end{solutionbox}
\begin{mnemonicbox}
``BWO goes Backward While Oscillating''

\end{mnemonicbox}
\begin{center}\rule{0.5\linewidth}{0.5pt}\end{center}

\subsection*{Question 3(b) OR [4
marks]}\label{q3b}

\textbf{Explain hazards due to microwave radiation.}

\begin{solutionbox}

\textbf{Types of Hazards:}

\begin{itemize}
\tightlist
\item
  \textbf{HERP}: Hazards of Electromagnetic Radiation to Personnel
\item
  \textbf{HERO}: Hazards of Electromagnetic Radiation to Ordnance\\
\item
  \textbf{HERF}: Hazards of Electromagnetic Radiation to Fuel
\end{itemize}

\textbf{Effects:}

\begin{itemize}
\tightlist
\item
  \textbf{Thermal heating}: Tissue heating at high power
\item
  \textbf{Eye damage}: Cataract formation
\item
  \textbf{Reproductive effects}: Potential fertility issues
\item
  \textbf{Pacemaker interference}: Electronic device malfunction
\end{itemize}

\textbf{Protection:}

\begin{itemize}
\tightlist
\item
  \textbf{Power density limits}: \textless{} 10 mW/cm^{2}
\item
  \textbf{Safety distances}: Far field calculations
\item
  \textbf{Warning signs}: Radiation hazard markers
\item
  \textbf{Personal monitors}: RF exposure meters
\end{itemize}

\end{solutionbox}
\begin{mnemonicbox}
``Microwaves Make Multiple Medical Maladies''

\end{mnemonicbox}
\begin{center}\rule{0.5\linewidth}{0.5pt}\end{center}

\subsection*{Question 3(c) OR [7
marks]}\label{q3c}

\textbf{Explain construction and working of magnetron with neat sketch.
List its applications.}

\begin{solutionbox}

\textbf{Construction:}

\begin{itemize}
\tightlist
\item
  \textbf{Circular cathode}: Central hot cathode
\item
  \textbf{Cylindrical anode}: With resonant cavities
\item
  \textbf{Permanent magnet}: Provides axial magnetic field
\item
  \textbf{Output coupling}: Loop or probe
\end{itemize}

\begin{center}
\textbf{Mermaid Diagram (Code)}
\begin{verbatim}
{Shaded}
{Highlighting}[]
graph LR
    A[Cathode] {-{-}{} B[Interaction Space]}
    B {-{-}{} C[Anode Cavities]}
    D[Magnetic Field] {-.{-}{} B}
    C {-{-}{} E[Output Coupling]}
    style A fill:\#ff9999
    style C fill:\#99ff99
{Highlighting}
{Shaded}
\end{verbatim}
\end{center}

\textbf{Working:}

\begin{itemize}
\tightlist
\item
  \textbf{Electron cloud}: Forms in interaction space
\item
  \textbf{Cycloid motion}: Due to E and B fields
\item
  \textbf{Resonant cavities}: Determine operating frequency
\item
  \textbf{π-mode oscillation}: Alternate cavities have opposite phase
\end{itemize}

\textbf{Applications:}

\begin{itemize}
\tightlist
\item
  \textbf{Microwave ovens}: 2.45 GHz heating
\item
  \textbf{Radar systems}: High power pulses
\item
  \textbf{Industrial heating}: Material processing
\item
  \textbf{Medical diathermy}: Therapeutic heating
\end{itemize}

\end{solutionbox}
\begin{mnemonicbox}
``Magnetron Makes Microwaves Magnificently''

\end{mnemonicbox}
\begin{center}\rule{0.5\linewidth}{0.5pt}\end{center}

\subsection*{Question 4(a) [3 marks]}\label{q4a}

\textbf{Explain working of P-i-N diode.}

\begin{solutionbox}

\textbf{P-i-N Diode} has \textbf{intrinsic layer} between P and N
regions, acting as \textbf{voltage-controlled resistor}.

\textbf{Structure:}

\begin{itemize}
\tightlist
\item
  \textbf{P region}: Heavily doped
\item
  \textbf{I region}: Intrinsic (undoped)\\
\item
  \textbf{N region}: Heavily doped
\end{itemize}

\textbf{Working:}

\begin{itemize}
\tightlist
\item
  \textbf{Forward bias}: Low resistance (1-10 Ω)
\item
  \textbf{Reverse bias}: High resistance (\textgreater10 kΩ)
\item
  \textbf{RF switch}: Controls microwave signals
\item
  \textbf{Variable attenuator}: Resistance varies with DC bias
\end{itemize}

\end{solutionbox}
\begin{mnemonicbox}
``PIN controls Power IN Networks''

\end{mnemonicbox}
\begin{center}\rule{0.5\linewidth}{0.5pt}\end{center}

\subsection*{Question 4(b) [4 marks]}\label{q4b}

\textbf{Explain the working of Varactor diode with sketch.}

\begin{solutionbox}

\textbf{Varactor Diode} acts as \textbf{voltage-controlled capacitor}
using junction capacitance variation.

\begin{verbatim}
    +V
     |
  ┌──┴──┐
  │  P  │  N  │  Junction
  └──┬──┘
     |
     0V
     
Capacitance vs Voltage:
C |    
  |{    }
  | {   }
  |  {  }
  |\_\_\_{\_\_\_\_}
    0  {-V (reverse bias)}
\end{verbatim}

\textbf{Working:}

\begin{itemize}
\tightlist
\item
  \textbf{Reverse bias}: Depletes junction, reduces capacitance
\item
  \textbf{Bias voltage}: Controls capacitance value
\item
  \textbf{Capacitance ratio}: Typically 3:1 to 10:1
\item
  \textbf{Frequency tuning}: Used in oscillators and filters
\end{itemize}

\textbf{Applications:}

\begin{itemize}
\tightlist
\item
  \textbf{VCO tuning}: Voltage controlled oscillators
\item
  \textbf{AFC circuits}: Automatic frequency control
\item
  \textbf{Parametric amplifiers}: Low noise amplification
\end{itemize}

\end{solutionbox}
\begin{mnemonicbox}
``Varactor Varies Capacitance with Voltage''

\end{mnemonicbox}
\begin{center}\rule{0.5\linewidth}{0.5pt}\end{center}

\subsection*{Question 4(c) [7 marks]}\label{q4c}

\textbf{Explain construction and working of Tunnel Diode and explain
tunneling phenomenon in detail. List its applications.}

\begin{solutionbox}

\textbf{Construction:}

\begin{itemize}
\tightlist
\item
  \textbf{Heavily doped P-N junction}: Both sides degenerately doped
\item
  \textbf{Thin junction}: \textasciitilde10 nm width
\item
  \textbf{Quantum tunneling}: Electrons tunnel through barrier
\end{itemize}

\textbf{Tunneling Phenomenon:}

\begin{itemize}
\tightlist
\item
  \textbf{Quantum effect}: Electrons pass through energy barrier
\item
  \textbf{Band overlap}: Conduction band overlaps valence band
\item
  \textbf{Probability function}: Tunneling probability depends on
  barrier width
\item
  \textbf{No thermal activation}: Occurs at room temperature
\end{itemize}

\begin{verbatim}
I{-V Characteristic:}
I |   
  |  /{     Negative resistance}
  | /  {   }
  |/    {  }
  |      {\_\_\_}
  |\_\_\_\_\_\_\_\_\_\_\_\_ V
    0  Vp  Vv
    
Vp = Peak voltage
Vv = Valley voltage
\end{verbatim}

\textbf{Working:}

\begin{itemize}
\tightlist
\item
  \textbf{Forward bias 0-Vp}: Current increases (tunneling)
\item
  \textbf{Vp to Vv}: Negative resistance region
\item
  \textbf{Beyond Vv}: Normal diode operation
\end{itemize}

\textbf{Applications:}

\begin{itemize}
\tightlist
\item
  \textbf{High-speed switching}: Picosecond switching
\item
  \textbf{Oscillators}: Microwave frequency generation
\item
  \textbf{Amplifiers}: Low noise amplification
\item
  \textbf{Memory circuits}: Bistable operation
\end{itemize}

\end{solutionbox}
\begin{mnemonicbox}
``Tunnel Diode Tunnels Through barriers
Terrifically''

\end{mnemonicbox}
\begin{center}\rule{0.5\linewidth}{0.5pt}\end{center}

\subsection*{Question 4(a) OR [3
marks]}\label{q4a}

\textbf{Describe the operation of IMPATT diode.}

\begin{solutionbox}

\textbf{IMPATT (Impact Avalanche Transit Time)} diode uses
\textbf{avalanche multiplication} and \textbf{transit time delay} for
oscillation.

\textbf{Operation:}

\begin{itemize}
\tightlist
\item
  \textbf{Avalanche zone}: Impact ionization creates carriers
\item
  \textbf{Drift zone}: Carriers drift with constant velocity
\item
  \textbf{Transit time}: Provides 180^\circ phase shift
\item
  \textbf{Negative resistance}: Due to phase delay
\end{itemize}

\textbf{Key parameters:}

\begin{itemize}
\tightlist
\item
  \textbf{Breakdown voltage}: Typically 20-100V
\item
  \textbf{Efficiency}: 10-20\%
\item
  \textbf{Frequency range}: 1-300 GHz
\end{itemize}

\end{solutionbox}
\begin{mnemonicbox}
``IMPATT Impacts with Avalanche Transit Time''

\end{mnemonicbox}
\begin{center}\rule{0.5\linewidth}{0.5pt}\end{center}

\subsection*{Question 4(b) OR [4
marks]}\label{q4b}

\textbf{Explain the frequency up and down conversion concepts for
parametric amplifier.}

\begin{solutionbox}

\textbf{Parametric Amplifier} uses \textbf{time-varying reactance} for
amplification and frequency conversion.

\textbf{Up-conversion:}

\begin{itemize}
\tightlist
\item
  \textbf{Signal frequency}: fs (input)
\item
  \textbf{Pump frequency}: fp (much higher)
\item
  \textbf{Output frequency}: fo = fp + fs
\item
  \textbf{Energy transfer}: From pump to signal
\end{itemize}

\textbf{Down-conversion:}

\begin{itemize}
\tightlist
\item
  \textbf{Signal frequency}: fs (input)\\
\item
  \textbf{Pump frequency}: fp
\item
  \textbf{Output frequency}: fo = fp - fs
\item
  \textbf{Mixer operation}: Frequency translation
\end{itemize}

\textbf{Advantages:}

\begin{itemize}
\tightlist
\item
  \textbf{Low noise}: Quantum-limited performance
\item
  \textbf{High gain}: 20-30 dB typical
\item
  \textbf{Wide bandwidth}: Several GHz
\end{itemize}

\end{solutionbox}
\begin{mnemonicbox}
``Parametric Pump Provides frequency conversion Plus
gain''

\end{mnemonicbox}
\begin{center}\rule{0.5\linewidth}{0.5pt}\end{center}

\subsection*{Question 4(c) OR [7
marks]}\label{q4c}

\textbf{Describe the construction and working principle of RUBY MASER.
List its applications.}

\begin{solutionbox}

\textbf{Construction:}

\begin{itemize}
\tightlist
\item
  \textbf{Ruby crystal}: Cr^{3}^{+} ions in Al_{2}O_{3} lattice
\item
  \textbf{Magnetic field}: Strong DC magnetic field
\item
  \textbf{Microwave cavity}: Resonant at signal frequency
\item
  \textbf{Pump source}: High frequency klystron
\item
  \textbf{Cryogenic cooling}: Liquid helium temperature
\end{itemize}

\begin{center}
\textbf{Mermaid Diagram (Code)}
\begin{verbatim}
{Shaded}
{Highlighting}[]
graph LR
    A[Ruby Crystal] {-{-}{} B[Microwave Cavity]}
    C[Magnetic Field] {-.{-}{} A}
    D[Pump Source] {-{-}{} B}
    E[Liquid Helium] {-.{-}{} A}
    B {-{-}{} F[Amplified Output]}
{Highlighting}
{Shaded}
\end{verbatim}
\end{center}

\textbf{Working Principle:}

\begin{itemize}
\tightlist
\item
  \textbf{Energy levels}: Cr^{3}^{+} ions have three energy levels
\item
  \textbf{Population inversion}: Pump creates more atoms in upper level
\item
  \textbf{Stimulated emission}: Signal photons trigger emission
\item
  \textbf{Coherent amplification}: Phase-coherent amplification
\end{itemize}

\textbf{Three-level system:}

\begin{itemize}
\tightlist
\item
  \textbf{Ground state}: E_{1} (most populated)
\item
  \textbf{Intermediate state}: E_{2} (signal frequency)
\item
  \textbf{Upper state}: E_{3} (pump frequency)
\end{itemize}

\textbf{Applications:}

\begin{itemize}
\tightlist
\item
  \textbf{Radio astronomy}: Ultra-low noise receivers
\item
  \textbf{Satellite communication}: Ground station amplifiers
\item
  \textbf{Deep space communication}: NASA tracking stations
\item
  \textbf{Research}: Quantum electronics experiments
\end{itemize}

\end{solutionbox}
\begin{mnemonicbox}
``RUBY MASER Makes Amazingly Sensitive
Electromagnetic Receivers''

\end{mnemonicbox}
\begin{center}\rule{0.5\linewidth}{0.5pt}\end{center}

\subsection*{Question 5(a) [3 marks]}\label{q5a}

\textbf{Draw and explain the functional block diagram of MTI RADAR.}

\begin{solutionbox}

\textbf{MTI RADAR} detects \textbf{moving targets} by comparing
\textbf{successive echoes} and canceling fixed targets.

\begin{center}
\textbf{Mermaid Diagram (Code)}
\begin{verbatim}
{Shaded}
{Highlighting}[]
graph LR
    A[Transmitter] {-{-}{} B[Duplexer]}
    B {-{-}{} C[Antenna]}
    C {-{-}{} B}
    B {-{-}{} D[Receiver]}
    D {-{-}{} E[Phase Detector]}
    F[STALO] {-{-}{} E}
    F {-{-}{} G[COHO]}
    G {-{-}{} E}
    E {-{-}{} H[Canceller]}
    H {-{-}{} I[Display]}
{Highlighting}
{Shaded}
\end{verbatim}
\end{center}

\textbf{Components:}

\begin{itemize}
\tightlist
\item
  \textbf{STALO}: Stable Local Oscillator
\item
  \textbf{COHO}: Coherent Oscillator\\
\item
  \textbf{Phase detector}: Compares echo phases
\item
  \textbf{Canceller}: Removes fixed target echoes
\end{itemize}

\end{solutionbox}
\begin{mnemonicbox}
``MTI Makes Targets Intelligible by Motion''

\end{mnemonicbox}
\begin{center}\rule{0.5\linewidth}{0.5pt}\end{center}

\subsection*{Question 5(b) [4 marks]}\label{q5b}

\textbf{Compare RADAR with SONAR.}

\begin{solutionbox}

{\def\LTcaptype{none} % do not increment counter
\begin{longtable}[]{@{}lll@{}}
\toprule\noalign{}
Parameter & RADAR & SONAR \\
\midrule\noalign{}
\endhead
\bottomrule\noalign{}
\endlastfoot
\textbf{Wave type} & Electromagnetic & Acoustic \\
\textbf{Medium} & Air/vacuum & Water \\
\textbf{Speed} & 3\times10^{8} m/s & 1500 m/s \\
\textbf{Frequency} & GHz & kHz \\
\textbf{Range} & 100+ km & 10-50 km \\
\textbf{Applications} & Air/space & Underwater \\
\end{longtable}
}

\textbf{Common features:}

\begin{itemize}
\tightlist
\item
  \textbf{Pulse-echo principle}
\item
  \textbf{Range measurement}
\item
  \textbf{Target detection}
\end{itemize}

\end{solutionbox}
\begin{mnemonicbox}
``RADAR Radiates, SONAR Sounds''

\end{mnemonicbox}
\begin{center}\rule{0.5\linewidth}{0.5pt}\end{center}

\subsection*{Question 5(c) [7 marks]}\label{q5c}

\textbf{Obtain the equation of maximum RADAR range. Explain the factors
affecting the maximum radar range.}

\begin{solutionbox}

\textbf{RADAR Range Equation:}

\textbf{R\_max = ^{4}\sqrt[(P\_t \times G^{2} \times λ^{2} \times σ) / (64π^{3} \times P\_min \times L)]}

Where:

\begin{itemize}
\tightlist
\item
  \textbf{P\_t}: Transmitter power (W)
\item
  \textbf{G}: Antenna gain (dimensionless)
\item
  \textbf{λ}: Wavelength (m)
\item
  \textbf{σ}: Target cross-section (m^{2})
\item
  \textbf{P\_min}: Minimum detectable power (W)
\item
  \textbf{L}: System losses (dimensionless)
\end{itemize}

\textbf{Derivation steps:}

\begin{enumerate}
\tightlist
\item
  \textbf{Power density at target}: P\_t\timesG/(4πR^{2})
\item
  \textbf{Power intercepted}: σ \times Power density
\item
  \textbf{Power at receiver}: Intercepted power \times G/(4πR^{2})
\item
  \textbf{Set equal to P\_min} and solve for R
\end{enumerate}

\textbf{Factors Affecting Range:}

\textbf{Increase Range:}

\begin{itemize}
\tightlist
\item
  \textbf{Higher transmitter power}: R ∝ P\_t\^{}(1/4)
\item
  \textbf{Larger antenna gain}: R ∝ G\^{}(1/2)
\item
  \textbf{Larger target RCS}: R ∝ σ\^{}(1/4)
\item
  \textbf{Lower system losses}: R ∝ L\^{}(-1/4)
\end{itemize}

\textbf{Decrease Range:}

\begin{itemize}
\tightlist
\item
  \textbf{Higher frequency}: R ∝ λ\^{}(1/2)
\item
  \textbf{Atmospheric losses}: Absorption and scattering
\item
  \textbf{Ground clutter}: Interfering reflections
\end{itemize}

\end{solutionbox}
\begin{mnemonicbox}
``RADAR Range Requires Robust Power and Proper
Parameters''

\end{mnemonicbox}
\begin{center}\rule{0.5\linewidth}{0.5pt}\end{center}

\subsection*{Question 5(a) OR [3
marks]}\label{q5a}

\textbf{Describe the Doppler effect in CW Doppler RADAR.}

\begin{solutionbox}

\textbf{Doppler Effect} causes \textbf{frequency shift} when target
moves relative to RADAR.

\textbf{Doppler Frequency:} \textbf{f\_d = (2 \times V\_r \times f\_0) / c}

Where:

\begin{itemize}
\tightlist
\item
  \textbf{V\_r}: Radial velocity (m/s)
\item
  \textbf{f\_0}: Transmitted frequency (Hz)
\item
  \textbf{c}: Speed of light (3\times10^{8} m/s)
\end{itemize}

\textbf{Characteristics:}

\begin{itemize}
\tightlist
\item
  \textbf{Approaching target}: f\_d positive
\item
  \textbf{Receding target}: f\_d negative
\item
  \textbf{Factor of 2}: Due to two-way propagation
\end{itemize}

\end{solutionbox}
\begin{mnemonicbox}
``Doppler Detects Direction with Doubled frequency
shift''

\end{mnemonicbox}
\begin{center}\rule{0.5\linewidth}{0.5pt}\end{center}

\subsection*{Question 5(b) OR [4
marks]}\label{q5b}

\textbf{Explain PPI Display method for RADAR}

\begin{solutionbox}

\textbf{PPI (Plan Position Indicator)} shows \textbf{top view} of RADAR
coverage area with range and bearing information.

\textbf{Display Features:}

\begin{itemize}
\tightlist
\item
  \textbf{Circular screen}: Center represents RADAR location
\item
  \textbf{Rotating trace}: Synchronized with antenna rotation
\item
  \textbf{Range rings}: Concentric circles for distance
\item
  \textbf{Bearing scale}: 0-360^\circ around circumference
\end{itemize}

\textbf{Operation:}

\begin{itemize}
\tightlist
\item
  \textbf{Sweep rotation}: Matches antenna rotation
\item
  \textbf{Echo intensity}: Controls brightness
\item
  \textbf{Persistence}: Afterglow maintains target visibility
\item
  \textbf{Range scale}: Selectable range settings
\end{itemize}

\textbf{Applications:}

\begin{itemize}
\tightlist
\item
  \textbf{Air traffic control}: Aircraft positioning
\item
  \textbf{Marine navigation}: Ship and obstacle detection
\item
  \textbf{Weather monitoring}: Storm tracking
\end{itemize}

\end{solutionbox}
\begin{mnemonicbox}
``PPI Provides Position Information Perfectly''

\end{mnemonicbox}
\begin{center}\rule{0.5\linewidth}{0.5pt}\end{center}

\subsection*{Question 5(c) OR [7
marks]}\label{q5c}

\textbf{Draw the block diagram of Pulse radar and explain the working
principle.}

\begin{solutionbox}

\begin{center}
\textbf{Mermaid Diagram (Code)}
\begin{verbatim}
{Shaded}
{Highlighting}[]
graph LR
    A[Master Oscillator] {-{-}{} B[Modulator]}
    B {-{-}{} C[Power Amplifier]}
    C {-{-}{} D[Duplexer]}
    D {-{-}{} E[Antenna]}
    E {-{-}{} D}
    D {-{-}{} F[RF Amplifier]}
    F {-{-}{} G[Mixer]}
    H[Local Oscillator] {-{-}{} G}
    G {-{-}{} I[IF Amplifier]}
    I {-{-}{} J[Detector]}
    J {-{-}{} K[Video Amplifier]}
    K {-{-}{} L[Display]}
    A {-{-}{} M[Timer]}
    M {-{-}{} B}
    M {-{-}{} L}
{Highlighting}
{Shaded}
\end{verbatim}
\end{center}

\textbf{Working Principle:}

\textbf{Transmission:}

\begin{itemize}
\tightlist
\item
  \textbf{Master oscillator}: Generates RF carrier
\item
  \textbf{Modulator}: Creates short pulses
\item
  \textbf{Power amplifier}: Amplifies pulse power
\item
  \textbf{Duplexer}: Routes pulse to antenna
\end{itemize}

\textbf{Reception:}

\begin{itemize}
\tightlist
\item
  \textbf{Echo reception}: Antenna receives reflected signals
\item
  \textbf{RF amplification}: Low noise amplification
\item
  \textbf{Mixing}: Converts to intermediate frequency
\item
  \textbf{IF amplification}: Further amplification
\item
  \textbf{Detection}: Extracts video signal
\item
  \textbf{Display}: Shows range vs amplitude
\end{itemize}

\textbf{Key Parameters:}

\begin{itemize}
\tightlist
\item
  \textbf{Pulse width}: Determines range resolution
\item
  \textbf{PRF}: Pulse repetition frequency
\item
  \textbf{Peak power}: Maximum range capability
\item
  \textbf{Duty cycle}: Average power consideration
\end{itemize}

\textbf{Advantages:}

\begin{itemize}
\tightlist
\item
  \textbf{High peak power}: Long range capability
\item
  \textbf{Good range resolution}: Narrow pulses
\item
  \textbf{Simple processing}: Direct detection
\end{itemize}

\end{solutionbox}
\begin{mnemonicbox}
``Pulse RADAR Pulses Powerfully for Precise
Position''

\end{mnemonicbox}

\end{document}
