\documentclass{article}

% content/resources/templates/preamble.tex
\usepackage[margin=0.6in]{geometry}
\author{Milav Dabgar}
\usepackage{amsmath,amssymb,amsthm}
\usepackage{booktabs}
\usepackage{multirow}
\usepackage{xcolor}
\usepackage{tcolorbox}
\tcbuselibrary{breakable,skins}
\usepackage[colorlinks=true,linkcolor=blue]{hyperref}
\usepackage{titlesec}
\usepackage{enumitem}
\usepackage{tikz}
\usepackage{pgfplots}
\usepackage{circuitikz}
\usepackage[version=4]{mhchem}
\usepackage{longtable}
\usepackage{array}
\usepackage{float}
\usepackage{caption}
\usepackage{listings}

\lstset{
  basicstyle=\small\ttfamily,
  breaklines=true,
  breakatwhitespace=false,
  postbreak=\mbox{\textcolor{red}{$\hookrightarrow$}\space},
  float=false,
  numbers=left,
  numberstyle=\tiny\color{gray},
  numbersep=10pt,
  xleftmargin=2em,
  keywordstyle=\color{blue},
  commentstyle=\color{green!60!black},
  stringstyle=\color{purple},
  backgroundcolor=\color{gray!5},
  showstringspaces=false,
  tabsize=2,
  captionpos=b,
  keepspaces=true,
  columns=flexible
}

\pgfplotsset{compat=1.18}
\usetikzlibrary{shapes,arrows,positioning,calc,patterns,decorations.pathmorphing,decorations.markings,arrows.meta}

% Color scheme
\definecolor{headcolor}{RGB}{0,102,204}
\definecolor{keycolor}{RGB}{220,20,60}
\definecolor{solutioncolor}{RGB}{34,139,34}
\definecolor{mnemoniccolor}{RGB}{148,0,211}
\definecolor{codecolor}{RGB}{0,0,100}

% Spacing
\setlength{\parskip}{3pt}
\setlist[itemize]{nosep}
\setlist[enumerate]{nosep}

% Title formatting
\titleformat{\section}{\Large\bfseries\color{headcolor}}{\thesection}{1em}{}
\titleformat{\subsection}{\large\bfseries\color{headcolor}}{\thesubsection}{1em}{}

% Pandoc tightlist compatibility
\providecommand{\tightlist}{%
  \setlength{\itemsep}{0pt}\setlength{\parskip}{0pt}}

% Pandoc longtable compatibility
\newcounter{none}
\def\thenone{}


% content/resources/templates/gujarati-boxes.tex
\usepackage{fontspec}
\usepackage{polyglossia}

% Set Gujarati as main language (document is primarily in Gujarati)
% Note: gloss-gujarati.ldf doesn't exist in polyglossia, but it will use hyphenation patterns
\setdefaultlanguage{gujarati}
\setotherlanguage{english}

% Configure Gujarati font properly
% Use Language=Default to prevent polyglossia from trying to add language-specific features
% that don't exist for Gujarati, which causes "empty feature" warnings
\newfontfamily\gujaratifont[Script=Gujarati,AutoFakeBold=2.5,AutoFakeSlant=0.3]{Noto Sans Gujarati}
\setmainfont[Script=Gujarati,AutoFakeBold=2.5,AutoFakeSlant=0.3]{Noto Sans Gujarati}
% Use Noto Sans Gujarati for monospace to support Gujarati in text
\setmonofont[Scale=0.9]{Noto Sans Gujarati}

% Configure English to use the same font
\newfontfamily\englishfont[Script=Gujarati,AutoFakeBold=2.5,AutoFakeSlant=0.3]{Noto Sans Gujarati}

% Translations for polyglossia
\gappto\captionsgujarati{
  \renewcommand{\tablename}{કોષ્ટક}
  \renewcommand{\figurename}{આકૃતિ}
}

% Helper for TikZ nodes to ensure Gujarati font
\newcommand{\gu}[1]{{\gujaratifont #1}}

% Custom environments
\newtcolorbox{solutionbox}{
    breakable,
    enhanced,
    colback=solutioncolor!5!white,
    colframe=solutioncolor!75!black,
    fonttitle=\bfseries,
    title=જવાબ
}

\newtcolorbox{solutionboxnobreak}{
 colback=solutioncolor!5!white,
 colframe=solutioncolor!75!black,
 fonttitle=\bfseries,
 title=જવાબ
}

\newtcolorbox{keyformula}{
 breakable,
 enhanced,
 colback=keycolor!5!white,
 colframe=keycolor!75!black,
 fonttitle=\bfseries,
 title=રાસાયણિક સમીકરણ/સૂત્ર
}

\newtcolorbox{mnemonicbox}{
 breakable,
 enhanced,
 colback=mnemoniccolor!5!white,
 colframe=mnemoniccolor!75!black,
 fonttitle=\bfseries,
 title=મેમરી ટ્રીક
}


% Custom commands for GTU solutions
% This file defines semantic commands for consistent formatting

% Question command with automatic formatting
\newcommand{\question}[2]{%
  \section*{Question #1}%
  \textbf{#2}%
}

% OR question variant
\newcommand{\questionor}[2]{%
  \section*{Question #1 OR}%
  \textbf{#2}%
}

% Proper table environment with caption
\newenvironment{answertable}[1]{%
  \begin{table}[htbp]
  \centering
  \caption{#1}
}{%
  \end{table}
}

% Proper figure environment for diagrams
\newenvironment{answerdiagram}[1]{%
  \begin{figure}[htbp]
  \centering
  \caption{#1}
}{%
  \end{figure}
}

% Semantic markup for key terms
\newcommand{\keyword}[1]{\textbf{#1}}
\newcommand{\code}[1]{\texttt{#1}}
\newcommand{\classname}[1]{\texttt{#1}}
\newcommand{\methodname}[1]{\texttt{#1}}

% Proper quotation marks
\newcommand{\mnemonic}[1]{``#1''}


\title{માઇક્રોવેવ અને રડાર કમ્યુનિકેશન (4351103) - ઉનાળુ 2025 સોલ્યુશન}
\date{May 16, 2025}

\begin{document}
\maketitle

\questionmarks{1(a)}{3}{ચાર માઇક્રોવેવ આવર્તન બેન્ડની તેમની આવર્ત શ્રેણી સાથે અને તેનાં ઉપયોગો સાથેની સૂચી બનાવો.}

\begin{solutionbox}
\textbf{માઇક્રોવેવ બેન્ડ:}

\begin{answertable}{આવર્તન બેન્ડ}
\begin{tabulary}{\linewidth}{|L|L|L|}
\hline
\textbf{બેન્ડ} & \textbf{આવર્તન શ્રેણી} & \textbf{ઉપયોગો} \\ \hline
\keyword{L-band} & 1-2 GHz & GPS, મોબાઈલ કમ્યુનિકેશન \\ \hline
\keyword{S-band} & 2-4 GHz & WiFi, બ્લૂટૂથ, રડાર \\ \hline
\keyword{C-band} & 4-8 GHz & સેટેલાઇટ કમ્યુનિકેશન \\ \hline
\keyword{X-band} & 8-12 GHz & મિલિટરી રડાર, હવામાન રડાર \\ \hline
\end{tabulary}
\end{answertable}
\end{solutionbox}

\begin{mnemonicbox}
\mnemonic{Little Satellites Communicate eXcellently}
\end{mnemonicbox}

\questionmarks{1(b)}{4}{એક જ સ્ટ્બનો ઉપયોગ કરીને ઇમ્પિડન્સ મેચિંગની પ્રક્રિયા સમજાવો.}

\begin{solutionbox}
\textbf{સિંગલ સ્ટબ મેચિંગ:}
લોડથી ચોક્કસ અંતરે \keyword{શોર્ટ-સર્કિટ સ્ટબ} ઉમેરીને રિફ્લેક્શન દૂર કરે છે.

\begin{answerdiagram}{સિંગલ સ્ટબ મેચિંગ}
\begin{tikzpicture}[auto, node distance=2cm]
    \node [gtu block] (source) {સોર્સ};
    \node [gtu block, right=of source] (stub_point) {સ્ટબ સ્થાન};
    \node [gtu block, right=of stub_point] (load) {લોડ ($Z_L$)};
    \node [gtu block, below=of stub_point] (stub) {શોર્ટ સ્ટબ};

    \draw [gtu arrow] (source) -- node[above] {લાઇન} (stub_point);
    \draw [gtu arrow] (stub_point) -- node[above] {$d$} (load);
    \draw [gtu arrow] (stub) -- node[right] {$l$} (stub_point);
\end{tikzpicture}
\end{answerdiagram}

\textbf{પ્રક્રિયા:}
\begin{enumerate}
    \item \keyword{સ્ટબ લંબાઈ}: લાઇન રિએક્ટન્સને રદ કરવા માટે રિએક્ટિવ ઇમ્પિડન્સ પ્રદાન કરે છે.
    \item \keyword{સ્ટબ સ્થાન}: તેના પર એડમિટન્સનો રિયલ પાર્ટ $Y_0$ હોય છે.
    \item \keyword{મેચિંગ સ્થિતિ}: કુલ એડમિટન્સ $Y = Y_0 + jB_{line} - jB_{stub} = Y_0$.
\end{enumerate}
\end{solutionbox}

\begin{mnemonicbox}
\mnemonic{Stub Positioned for Perfect Matching}
\end{mnemonicbox}

\questionmarks{1(c)}{7}{લોસલેસ ટ્રાન્સમિશન લાઇનની લાક્ષણિકતાઓ જણાવો અને બે વાયર ટ્રાન્સમિશન લાઇન માટે સામાન્ય સમીકરણ મેળવો.}

\begin{solutionbox}
\textbf{લોસલેસ લાઇનની લાક્ષણિકતાઓ:}
\begin{itemize}
    \item \keyword{કોઈ પાવર લોસ નથી}: $R = 0, G = 0$.
    \item \keyword{સ્થિર એમ્પ્લીટ્યુડ}: કોઈ એટેન્યુએશન નથી ($\alpha = 0$).
    \item \keyword{માત્ર ફેઝ ડિલે}: સિગ્નલ વિલંબિત થાય છે પરંતુ નબળું પડતું નથી.
    \item \keyword{સ્ટેન્ડિંગ વેવ પેટર્ન}: મિસમેચ્ડ લોડના રિફ્લેક્શનને કારણે રચાય છે.
\end{itemize}

\textbf{સામાન્ય સમીકરણો:}
પ્રોપેગેશન કોન્સ્ટન્ટ $\gamma = \alpha + j\beta$ અને કેરેક્ટરિસ્ટિક ઇમ્પિડન્સ $Z_0$ સાથેની લાઇન માટે:

\keyword{વોલ્ટેજ સમીકરણ}:
\[ V(z) = V^+ e^{-\gamma z} + V^- e^{\gamma z} \]

\keyword{કરન્ટ સમીકરણ}:
\[ I(z) = \frac{V^+}{Z_0} e^{-\gamma z} - \frac{V^-}{Z_0} e^{\gamma z} \]

જ્યાં:
\begin{itemize}
    \item $Z_0 = \sqrt{L/C}$ (લોસલેસ લાઇન માટે).
    \item $\gamma = j\beta$ (કારણ કે $\alpha=0$).
\end{itemize}
\end{solutionbox}

\begin{mnemonicbox}
\mnemonic{Lossless Lines Love Low Loss}
\end{mnemonicbox}

\orquestionmarks{1(c)}{7}{સ્થાયી તરંગ વ્યાખ્યાયિત કરો. શોર્ટ સર્કિટ અને ઓપન સર્કિટ લાઇન માટે સ્ટેન્ડિંગ વેવ પેટર્ન દોરો અને સમજાવો.}

\begin{solutionbox}
\textbf{સ્ટેન્ડિંગ વેવ (સ્થાયી તરંગ):}
 વિરુદ્ધ દિશામાં મુસાફરી કરતા \keyword{ફોરવર્ડ} અને \keyword{રીફ્લેક્ટેડ વેવ્સ} ના ઇન્ટરફિયરન્સથી રચાતી સ્થિર પેટર્ન.

\begin{answerdiagram}{સ્ટેન્ડિંગ વેવ પેટર્ન}
\begin{tikzpicture}
    % Short Circuit
    \begin{scope}
        \node at (2, 2.5) {\textbf{શોર્ટ સર્કિટ ($Z_L=0$)}};
        % Voltage (Min at load)
        \draw[thick, ->] (0,0) -- (4.2,0) node[right] {$z$};
        \draw[thick, ->] (0,0) -- (0,2) node[above] {$|V|$};
        \draw[blue, thick] plot[domain=0:4, samples=100] (\x, {abs(sin(180*\x))});
        \node[blue] at (2,1.2) {વોલ્ટેજ};
        \node at (4, -0.3) {0 (લોડ)};
        \node at (0, -0.3) {સોર્સ};
        
        % Current (Max at load)
        \draw[red, dashed, thick] plot[domain=0:4, samples=100] (\x, {abs(cos(180*\x))});
        \node[red] at (1,1.2) {કરન્ટ};
    \end{scope}

    % Open Circuit
    \begin{scope}[xshift=6cm]
        \node at (2, 2.5) {\textbf{ઓપન સર્કિટ ($Z_L=\infty$)}};
        % Voltage (Max at load)
        \draw[thick, ->] (0,0) -- (4.2,0) node[right] {$z$};
        \draw[thick, ->] (0,0) -- (0,2) node[above] {$|V|$};
        \draw[blue, thick] plot[domain=0:4, samples=100] (\x, {abs(cos(180*\x))});
        \node[blue] at (1,1.2) {વોલ્ટેજ};
        \node at (4, -0.3) {0 (લોડ)};

        % Current (Min at load)
        \draw[red, dashed, thick] plot[domain=0:4, samples=100] (\x, {abs(sin(180*\x))});
        \node[red] at (2,1.2) {કરન્ટ};
    \end{scope}
\end{tikzpicture}
\end{answerdiagram}

\textbf{વિશ્લેષણ:}
\begin{answertable}{સ્ટેન્ડિંગ વેવ લક્ષણો}
\begin{tabulary}{\linewidth}{|L|L|L|}
\hline
\textbf{સ્થિતિ} & \textbf{લોડ પર વોલ્ટેજ} & \textbf{લોડ પર કરન્ટ} \\ \hline
\keyword{શોર્ટ સર્કિટ} & ન્યૂનતમ (0) & મહત્તમ \\ \hline
\keyword{ઓપન સર્કિટ} & મહત્તમ ($2V^+$) & ન્યૂનતમ (0) \\ \hline
\end{tabulary}
\end{answertable}
ક્રમિક મેક્સિમા અથવા મિનિમા વચ્ચેનું અંતર $\lambda/2$ છે.
\end{solutionbox}

\begin{mnemonicbox}
\mnemonic{Short Circuits Current, Open Circuits Voltage}
\end{mnemonicbox}

\questionmarks{2(a)}{3}{મેજિક TEE ની કામગીરી દોરો અને સમજાવો.}

\begin{solutionbox}
\textbf{મેજિક TEE:}
E-પ્લેન અને H-પ્લેન ટી ને જોડતું 4-પોર્ટ હાઇબ્રિડ વેવગાઇડ જંકશન.

\begin{answerdiagram}{મેજિક TEE સ્ટ્રક્ચર}
\begin{tikzpicture}[scale=0.8]
    \draw[thick] (-3,0) -- (3,0); % Main arm
    \draw[thick] (0,0) -- (0,2);  % E-arm
    \draw[thick] (0,0) circle (0.2); % H-arm representation
    
    \node at (-3.5,0) {પોર્ટ 1};
    \node at (3.5,0) {પોર્ટ 2};
    \node at (0,2.3) {પોર્ટ 4 (E-આર્મ)};
    \node at (0,-0.5) {પોર્ટ 3 (H-આર્મ)};
    
    \node at (0,-1.5) {પોર્ટ 1 \& 2: કોલિનિયર આર્મ્સ};
\end{tikzpicture}
\end{answerdiagram}

\textbf{કામગીરી:}
\begin{itemize}
    \item \keyword{પોર્ટ 3 (H-આર્મ)}: સરવાળો પોર્ટ ($P_3 \propto P_1 + P_2$). 1 અને 2 ના ઇનપુટ સમાન ફેઝમાં હોય છે.
    \item \keyword{પોર્ટ 4 (E-આર્મ)}: તફાવત પોર્ટ ($P_4 \propto P_1 - P_2$). 1 અને 2 ના ઇનપુટ $180^\circ$ ફેઝ શિફ્ટ ધરાવે છે.
    \item \keyword{આઇસોલેશન}: E-આર્મ (4) અને H-આર્મ (3) વચ્ચે કોઈ કપલિંગ નથી.
\end{itemize}
\end{solutionbox}

\begin{mnemonicbox}
\mnemonic{Magic Tee Mixes Modes}
\end{mnemonicbox}

\questionmarks{2(b)}{4}{હાયબ્રિડ રિંગની કામગીરી સમજાવો.}

\begin{solutionbox}
\textbf{હાયબ્રિડ રિંગ (રેટ-રેસ કપ્લર):}
પાવર સ્પ્લિટિંગ અને સમિંગ માટે વપરાતી 4-પોર્ટ સર્ક્યુલર વેવગાઇડ.

\begin{answerdiagram}{હાયબ્રિડ રિંગ}
\begin{tikzpicture}
    \draw[thick] (0,0) circle (1.5);
    \node at (0,0) {રિંગ $1.5\lambda$};
    
    \node[draw, circle, fill=white] (p1) at (90:1.5) {1};
    \node[draw, circle, fill=white] (p2) at (0:1.5) {2};
    \node[draw, circle, fill=white] (p3) at (-90:1.5) {3};
    \node[draw, circle, fill=white] (p4) at (180:1.5) {4};
    
    \node[right] at (2,0) {પ્રક્રિયા: પાવર ડિવિઝન};
\end{tikzpicture}
\end{answerdiagram}

\textbf{વર્કિંગ પેરામીટર્સ:}
\begin{itemize}
    \item \keyword{ઘેરવો}: $1.5\lambda$ (કુલ પાથ લંબાઈ).
    \item \keyword{સ્પેસિંગ}: પોર્ટ્સ $\lambda/4$ અંતરે છે, સિવાય કે એક $3\lambda/4$ સેક્શન.
    \item \keyword{કાર્ય}: પોર્ટ 1 પરનું ઇનપુટ 2 અને 4 વચ્ચે સમાન રીતે વિભાજિત થાય છે. પોર્ટ 3 આઇસોલેટેડ રહે છે.
\end{itemize}
\end{solutionbox}

\begin{mnemonicbox}
\mnemonic{Ring Runs Round for Power Sharing}
\end{mnemonicbox}

\questionmarks{2(c)}{7}{"સર્ક્યુલેટર" ના બાંધકામ અને કાર્યસિદ્ધાંત સમજાવો. તેની એપ્લિકેશનોની સૂચિ બનાવો.}

\begin{solutionbox}
\textbf{સર્ક્યુલેટર બાંધકામ:}

\begin{answerdiagram}{થ્રી-પોર્ટ સર્ક્યુલેટર}
\begin{tikzpicture}[auto, node distance=2cm]
    \node [gtu decision, fill=gray!20] (jun) {ફેરાઇટ\\જંકશન};
    \node [gtu state, above=of jun] (p1) {પોર્ટ 1};
    \node [gtu state, below right=of jun] (p2) {પોર્ટ 2};
    \node [gtu state, below left=of jun] (p3) {પોર્ટ 3};
    
    \draw [gtu arrow] (p1) -- (jun);
    \draw [gtu arrow] (jun) -- (p2);
    \draw [gtu arrow] (p2) |- (jun); % simplified flow representation
    \draw [gtu arrow] (p3) |- (jun);

    % Circular flow arrows
    \draw [->, red, thick] (jun.north) to[bend left] (jun.east);
    \draw [->, red, thick] (jun.east) to[bend left] (jun.south);
    \draw [->, red, thick] (jun.south) to[bend left] (jun.west);
    
    \node at (3,0) {દિશા: $1 \rightarrow 2 \rightarrow 3 \rightarrow 1$};
\end{tikzpicture}
\end{answerdiagram}

\textbf{કાર્યસિદ્ધાંત:}
\begin{itemize}
    \item ફેરાઇટ મટિરિયલમાં \keyword{ફેરાડે રોટેશન} પર આધારિત.
    \item \keyword{નોન-રેસિપ્રોકલ}: પોર્ટ 1 માં દાખલ થતું સિગ્નલ માત્ર પોર્ટ 2 માંથી બહાર આવે છે. પોર્ટ 2 થી પોર્ટ 3, વગેરે.
    \item રિવર્સ પાવર બ્લોક (આઇસોલેટેડ) થાય છે.
\end{itemize}

\textbf{ઉપયોગો:}
\begin{enumerate}
    \item \keyword{રડારમાં ડુપ્લેક્સર}: Tx અને Rx માટે એક જ એન્ટેના વાપરવા દે છે.
    \item \keyword{આઇસોલેટર}: એક પોર્ટને મેચ લોડ સાથે ટર્મિનેટ કરીને.
    \item \keyword{પેરામેટ્રિક એમ્પ્લીફાયર}: ઇનપુટ અને આઉટપુટનું અલ્ગીકરણ.
\end{enumerate}
\end{solutionbox}

\begin{mnemonicbox}
\mnemonic{Circulator Circles Clockwise Continuously}
\end{mnemonicbox}

\orquestionmarks{2(a)}{3}{લંબચોરસ વેવગાઇડ અને ગોળાકાર વેવગાઇડની તુલના કરો.}

\begin{solutionbox}
\textbf{તુલના:}

\begin{answertable}{વેવગાઇડ તુલના}
\begin{tabulary}{\linewidth}{|L|L|L|}
\hline
\textbf{પેરામીટર} & \textbf{લંબચોરસ વેવગાઇડ} & \textbf{ગોળાકાર વેવગાઇડ} \\ \hline
\keyword{ક્રોસ-સેક્શન} & લંબચોરસ ($a \times b$) & ગોળાકાર (ત્રિજ્યા $a$) \\ \hline
\keyword{ડોમિનન્ટ મોડ} & $TE_{10}$ & $TE_{11}$ \\ \hline
\keyword{કટઓફ ફ્રિકવન્સી} & $f_c = c/2a$ & $f_c = 1.841c/2\pi a$ (જટિલ) \\ \hline
\keyword{પાવર હેન્ડલિંગ} & ઓછી & વધારે \\ \hline
\keyword{રોટેશન} & પોલરાઇઝેશન ફિક્સ્ડ & રોટેટિંગ પોલરાઇઝેશનને સપોર્ટ કરે છે \\ \hline
\end{tabulary}
\end{answertable}
\end{solutionbox}

\begin{mnemonicbox}
\mnemonic{Rectangles are Regular, Circles are Complex}
\end{mnemonicbox}

\orquestionmarks{2(b)}{4}{ડાયરેક્શનલ કપ્લરનું કાર્યસિદ્ધાંત દોરો અને સમજાવો.}

\begin{solutionbox}
\textbf{ડાયરેક્શનલ કપ્લર:}

\begin{answerdiagram}{2-હોલ ડાયરેક્શનલ કપ્લર}
\begin{tikzpicture}[auto, node distance=2cm]
    \node [gtu block, minimum width=4cm] (main) {મેઇન વેવગાઇડ};
    \node [gtu block, minimum width=4cm, below=0.5cm of main] (aux) {ઓક્ઝિલરી વેવગાઇડ};
    
    \node [left=of main] (p1) {1 (ઇનપુટ)};
    \node [right=of main] (p2) {2 (થ્રૂ)};
    \node [left=of aux] (p3) {3 (કપ્લ્ડ)};
    \node [right=of aux] (p4) {4 (આઇસોલેટેડ)};
    
    \draw [gtu arrow] (p1) -- (main);
    \draw [gtu arrow] (main) -- (p2);
    
    % Coupling holes
    \draw [dashed, ->] (main.south) -- (aux.north);
    \node at (0, -0.25) {કપલિંગ હોલ્સ ($\lambda/4$ અંતરે)};
\end{tikzpicture}
\end{answerdiagram}

\textbf{કામગીરી:}
\begin{itemize}
    \item ફોરવર્ડ પાવરનો નાનો ભાગ પોર્ટ 3 માં સેમ્પલ કરે છે.
    \item પોર્ટ 4 તરફ જતા રિવર્સ વેવ્સ $\lambda/2$ પાથ ડિફરન્સને કારણે રદ થાય છે (ડિસ્ટ્રક્ટિવ ઇન્ટરફિયરન્સ).
\end{itemize}

\textbf{પેરામીટર્સ:}
\begin{itemize}
    \item \keyword{કપલિંગ ફેક્ટર}: $C = 10 \log(P_1/P_3)$ dB.
    \item \keyword{ડાયરેક્ટિવિટી}: $D = 10 \log(P_3/P_4)$ dB.
\end{itemize}
\end{solutionbox}

\begin{mnemonicbox}
\mnemonic{Coupler Couples Carefully in Correct Direction}
\end{mnemonicbox}

\orquestionmarks{2(c)}{7}{"ટ્રાવેલિંગ વેવ ટ્યુબ" ના બાંધકામ અને કાર્યસિદ્ધાંત સમજાવો. તેની એપ્લિકેશનોની સૂચિ બનાવો.}

\begin{solutionbox}
\textbf{બાંધકામ:}

\begin{answerdiagram}{હેલિક્સ TWT}
\begin{tikzpicture}[auto, node distance=1.5cm]
    \node [gtu start] (gun) {ઇલેક્ટ્રોન\\ગન};
    \node [gtu block, right=of gun, minimum width=4cm] (helix) {હેલિક્સ સ્લો-વેવ સ્ટ્રક્ચર};
    \node [gtu stop, right=of helix] (col) {કલેક્ટર};
    
    \node [above=of helix] (in) {RF ઇનપુટ};
    \node [above=of col] (out) {RF આઉટપુટ};
    
    \draw [gtu arrow] (gun) -- (helix);
    \draw [gtu arrow] (helix) -- (col);
    
    \draw [->] (in) -- (helix.west);
    \draw [->] (helix.east) -- (out);
    
    \node [below=of helix] {એક્સિયલ મેગ્નેટિક ફીલ્ડ ($B_0$)};
\end{tikzpicture}
\end{answerdiagram}

\textbf{કાર્યસિદ્ધાંત:}
\begin{itemize}
    \item \keyword{સ્લો વેવ સ્ટ્રક્ચર}: હેલિક્સ RF ફેઝ વેલોસિટીને ઇલેક્ટ્રોન બીમ વેલોસિટી સાથે મેચ કરવા ઘટાડે છે ($v_{ph} \approx v_e$).
    \item \keyword{ઇન્ટરેક્શન}: સતત ક્રિયાપ્રતિક્રિયા ઇલેક્ટ્રોન બંચિંગ કરે છે. કાઇનેટિક એનર્જી ઇલેક્ટ્રોનથી RF ફીલ્ડમાં સ્થાનાંતરિત થાય છે.
    \item \keyword{એમ્પ્લિફિકેશન}: ટ્યુબની લંબાઈ સાથે સિગ્નલ સ્પોન્જિયલ રીતે વધે છે.
\end{itemize}

\textbf{ઉપયોગો:}
\begin{itemize}
    \item સેટેલાઇટ ટ્રાન્સપોન્ડર (ઉચ્ચ વિશ્વસનીયતા).
    \item રડાર સિસ્ટમ્સ (વાઈડ બેન્ડવિડ્થ).
    \item ઇલેક્ટ્રોનિક વોરફેર (જેમિંગ).
\end{itemize}
\end{solutionbox}

\begin{mnemonicbox}
\mnemonic{TWT Transfers Tremendous power Through Travel}
\end{mnemonicbox}

\questionmarks{3(a)}{3}{ઉચ્ચ VSWR માપન માટે પરોક્ષ પદ્ધતિ સમજાવો.}

\begin{solutionbox}
\textbf{ડબલ મિનિમમ મેથડ (પરોક્ષ):}
જ્યારે VSWR > 10 હોય ત્યારે વપરાય છે. ડાયરેક્ટ રીડિંગ અચોક્કસ હોય છે.

\textbf{પ્રક્રિયા:}
\begin{enumerate}
    \item વોલ્ટેજ મિનિમમ ($V_{min}$) નું સ્થાન શોધો.
    \item પ્રોબને ડાબે અને જમણે એવા બિંદુઓ પર ખસેડો જ્યાં પાવર બમણો હોય ($2 \times V_{min}^2$).
    \item આ "ડબલ પાવર" બિંદુઓ વચ્ચેનું અંતર $d$ માપો.
\end{enumerate}

\textbf{સૂત્ર:}
\[ VSWR = \frac{\lambda_g}{\pi d} \]
જ્યાં $\lambda_g$ ગાઇડ વેવલેન્થ છે અને $d$ 3dB પોઇન્ટ્સ પર મિનિમમની પહોળાઈ છે.
\end{solutionbox}

\begin{mnemonicbox}
\mnemonic{Indirect method uses Intermediate Attenuation}
\end{mnemonicbox}

\questionmarks{3(b)}{4}{કનવેંશનલ ટ્યૂબ્સની આવર્તન મર્યાદાઓ લખો અને સમજાવો.}

\begin{solutionbox}
\textbf{માઇક્રોવેવ ફ્રીક્વન્સી પર કનવેંશનલ ટ્યૂબની મર્યાદાઓ:}

\begin{answertable}{મર્યાદાઓ અને અસરો}
\begin{tabulary}{\linewidth}{|L|L|}
\hline
\textbf{મર્યાદા} & \textbf{અસર} \\ \hline
\keyword{ટ્રાન્ઝિટ ટાઈમ} & ઇલેક્ટ્રોન ગેપ ક્રોસ કરવાનો સમય RF પિરિયડ જેટલો થાય. જે કન્ડક્ટન્સ ($G$) લોડિંગ અને ગ્રીડ હીટિંગનું કારણ બને છે. \\ \hline
\keyword{લીડ ઇન્ડક્ટન્સ} & $X_L = 2\pi f L$. કેથોડ લીડમાં હાઇ ઇમ્પિડન્સ અસરકારક ગેઇન ઘટાડે છે. \\ \hline
\keyword{ઇન્ટર-ઇલેક્ટ્રોડ કેપેસિટન્સ} & $X_C = 1/2\pi f C$. RF સિગ્નલને શંટ કરે છે, આઉટપુટ ઇમ્પિડન્સ અને ગેઇન ઘટાડે છે. \\ \hline
\keyword{સ્કિન ઇફેક્ટ} & કરન્ટ સપાટી પર સીમિત રહે છે, રેઝિસ્ટન્સ અને લોસ વધારે છે. \\ \hline
\end{tabulary}
\end{answertable}

\textbf{પરિણામ:} ગેઇન શૂન્ય થઈ જાય છે, અને નોઇઝ વધે છે.
\end{solutionbox}

\begin{mnemonicbox}
\mnemonic{Transit Time Troubles Traditional Tubes}
\end{mnemonicbox}

\questionmarks{3(c)}{7}{એપ્લિગેટ ડાયાગ્રામ સાથે ટૂ કેવિટી ક્લીસ્ટ્રોનનું બાંધકામ અને કાર્ય સમજાવો. તેના ફાયદાઓની યાદી આપો.}

\begin{solutionbox}
\textbf{ટૂ કેવિટી ક્લાયસ્ટ્રોન:}

\begin{answerdiagram}{ક્લાયસ્ટ્રોન એમ્પ્લીફાયર}
\begin{tikzpicture}[auto, node distance=1.5cm]
    \node [gtu start] (k) {K};
    \node [gtu block, right=of k] (buncher) {બંચર\\ગ્રીડ};
    \node [gtu block, right=of buncher, minimum width=2.5cm] (drift) {ડ્રિફ્ટ સ્પેસ};
    \node [gtu block, right=of drift] (catcher) {કેચર\\ગ્રીડ};
    \node [gtu stop, right=of catcher] (col) {કલેક્ટર};
    
    \draw [gtu arrow] (k) -- (buncher);
    \draw [gtu arrow] (buncher) -- (drift);
    \draw [gtu arrow] (drift) -- (catcher);
    \draw [gtu arrow] (catcher) -- (col);
    
    \node [above=of buncher] {RF In};
    \node [above=of catcher] {RF Out};
\end{tikzpicture}
\end{answerdiagram}

\textbf{એપલગેટ ડાયાગ્રામ (બંચિંગ):}

\begin{answerdiagram}{એપલગેટ ડાયાગ્રામ}
\begin{tikzpicture}
    \draw[->] (0,0) -- (5,0) node[right] {અંતર};
    \draw[->] (0,0) -- (0,3) node[above] {સમય};
    
    % Electron trajectories meeting at a point
    \draw (0,0.5) -- (4,2.5) node[right] {ધીમા};
    \draw (0,1.5) -- (4,2.5) node[right] {રેફરન્સ};
    \draw (0,2.5) -- (4,2.5) node[right] {ઝડપી};
    
    \node at (4,3) {બંચ સેન્ટર};
    \draw[dashed] (4,0) -- (4,2.5);
\end{tikzpicture}
\end{answerdiagram}

\textbf{કામગીરી:}
\begin{enumerate}
    \item \keyword{વેલોસિટી મોડ્યુલેશન}: બંચર કેવિટીમાં RF ઇનપુટ ઇલેક્ટ્રોનને પ્રવેગિત/ધીમા પાડે છે.
    \item \keyword{બંચિંગ}: ડ્રિફ્ટ સ્પેસમાં, ઝડપી ઇલેક્ટ્રોન ધીમાને પકડી લે છે.
    \item \keyword{કરન્ટ મોડ્યુલેશન}: ઇલેક્ટ્રોન બંચ કેચર કેવિટીમાં મજબૂત RF કરન્ટ પ્રેરિત કરે છે.
\end{enumerate}

\textbf{ફાયદાઓ:} હાઈ ગેઇન (>30dB), હાઈ પાવર, સ્થિર.
\end{solutionbox}

\begin{mnemonicbox}
\mnemonic{Klystron Kicks with Kinetic Bunching}
\end{mnemonicbox}

\orquestionmarks{3(a)}{3}{BWOનું બાંધકામ અને કાર્ય સમજાવો.}

\begin{solutionbox}
\textbf{બેકવર્ડ વેવ ઓસિલેટર (BWO):}
એક TWT જેવું ઉપકરણ જ્યાં વેવ ઇલેક્ટ્રોન બીમની વિરુદ્ધ મુસાફરી કરે છે.

\textbf{બાંધકામ:}
TWT (ઇલેક્ટ્રોન ગન, હેલિક્સ) જેવું જ, પરંતુ RF આઉટપુટ ગન છેડા પાસે લેવામાં આવે છે.

\textbf{કામગીરી:}
\begin{itemize}
    \item બીમ વેવના \keyword{બેકવર્ડ સ્પેસ હાર્મોનિક} સાથે ક્રિયાપ્રતિક્રિયા કરે છે.
    \item ફીડબેક આંતરિક છે (વેવ ઇનપુટ પર પાછું વહે છે).
    \item \keyword{વોલ્ટેજ ટ્યુનિંગ}: ઓસિલેશન ફ્રીક્વન્સી બીમ વોલ્ટેજ દ્વારા નિયંત્રિત થાય છે.
\end{itemize}
\end{solutionbox}

\begin{mnemonicbox}
\mnemonic{BWO goes Backward While Oscillating}
\end{mnemonicbox}

\orquestionmarks{3(b)}{4}{માઇક્રોવેવ રેડિયેશનને કારણે જોખમો સમજાવો.}

\begin{solutionbox}
\textbf{માઇક્રોવેવ જોખમો:}

\begin{itemize}
    \item \keyword{HERP}: Hazards of EM Radiation to Personnel (જૈવિક નુકસાન).
    \item \keyword{HERO}: Hazards to Ordnance (વિસ્ફોટકોનું ડિટોનેશન).
    \item \keyword{HERF}: Hazards to Fuel (વરાળનું ઇગ્નિશન).
\end{itemize}

\textbf{જૈવિક અસરો:}
\begin{itemize}
    \item \keyword{થર્મલ}: પાણીથી ભરપૂર પેશીઓનું ગરમી (આંખો, મગજ, પેટ). મોતિયાનું કારણ બની શકે છે.
    \item \keyword{નોન-થર્મલ}: નર્વસ સિસ્ટમ પર અસરો (ચર્ચાસ્પદ).
\end{itemize}

\textbf{સુરક્ષા મર્યાદા}: સામાન્ય રીતે $< 10 mW/cm^2$.
\end{solutionbox}

\begin{mnemonicbox}
\mnemonic{Microwaves Make Multiple Medical Maladies}
\end{mnemonicbox}

\orquestionmarks{3(c)}{7}{સુઘડ સ્કેચ સાથે મેગ્નેટ્રોનનું બાંધકામ અને કાર્ય સમજાવો. તેની એપ્લિકેશનોની સૂચિ બનાવો.}

\begin{solutionbox}
\textbf{મેગ્નેટ્રોન બાંધકામ:}

\begin{answerdiagram}{મેગ્નેટ્રોન ક્રોસ-સેક્શન}
\begin{tikzpicture}
    % Anode Block
    \draw[thick, fill=gray!20] (0,0) circle (2);
    \draw[thick, fill=white] (0,0) circle (1);
    
    % Cavities (Simplified Vanes)
    \foreach \angle in {0,45,...,315}
    {
        \draw[thick] (0,0) -- (\angle:1);
    }
    
    % Cathode
    \draw[thick, fill=red] (0,0) circle (0.3);
    \node at (0,0) [right] {કેથોડ};
    
    \node at (0,1.5) {એનોડ};
    \node at (2.5,0) {ઇન્ટરેક્શન સ્પેસ};
    \node at (-2.5,0) {મેગ્નેટ (લંબરૂપ)};
\end{tikzpicture}
\end{answerdiagram}

\textbf{કાર્યસિદ્ધાંત:}
\begin{itemize}
    \item \keyword{ક્રોસ્ડ ફીલ્ડ્સ}: રેડિયલ ઇલેક્ટ્રિક ફીલ્ડ ($E$) અને એક્સિયલ મેગ્નેટિક ફીલ્ડ ($B$).
    \item \keyword{ઇલેક્ટ્રોન ગતિ}: ઇલેક્ટ્રોન સાયક્લોઇડ પાથમાં સર્પાકાર ગતિ કરે છે.
    \item \keyword{ફેઝ ફોકસિંગ}: ઇલેક્ટ્રોન કેવિટીમાં RF ફીલ્ડ્સને ઊર્જા સ્થાનાંતરિત કરે છે (ચાર્જના "સ્પોક્સ").
    \item \keyword{$\pi$-મોડ}: બાજુની કેવિટીઝ $180^\circ$ ફેઝ આઉટ હોય છે.
\end{itemize}

\textbf{ઉપયોગો:} માઇક્રોવેવ ઓવન, રડાર ટ્રાન્સમીટર.
\end{solutionbox}

\begin{mnemonicbox}
\mnemonic{Magnetron Makes Microwaves Magnificently}
\end{mnemonicbox}

\questionmarks{4(a)}{3}{P-i-N ડાયોડની કામગીરી સમજાવો.}

\begin{solutionbox}
\textbf{P-i-N ડાયોડ કામગીરી:}
P અને N પ્રદેશો વચ્ચે \keyword{ઇન્ટ્રિન્સિક (I)} લેયર ધરાવે છે.

\begin{answerdiagram}{PIN ડાયોડ સ્ટ્રક્ચર}
\begin{tikzpicture}
    \draw[thick] (0,0) rectangle (4,1);
    \draw[thick] (1,0) -- (1,1);
    \draw[thick] (3,0) -- (3,1);
    \node at (0.5,0.5) {P+};
    \node at (2,0.5) {ઇન્ટ્રિન્સિક (I)};
    \node at (3.5,0.5) {N+};
    \draw (0,0.5) -- (-0.5,0.5);
    \draw (4,0.5) -- (4.5,0.5);
\end{tikzpicture}
\end{answerdiagram}

\textbf{સ્ટેટ્સ:}
\begin{enumerate}
    \item \keyword{ફોરવર્ડ બાયાસ}: કેરિયર ઇન્જેક્શન રેઝિસ્ટન્સ ઘટાડે છે ($R \approx 1\Omega$). \keyword{શોર્ટ} તરીકે કામ કરે છે.
    \item \keyword{રિવર્સ બાયાસ}: કેરિયર્સ દૂર થાય છે, હાઇ રેઝિસ્ટન્સ ($R > 10k\Omega$). \keyword{ઓપન} તરીકે કામ કરે છે.
\end{enumerate}

\textbf{ઉપયોગો}: RF સ્વિચ, વેરિએબલ એટેન્યુએટર.
\end{solutionbox}

\begin{mnemonicbox}
\mnemonic{PIN controls Power IN Networks}
\end{mnemonicbox}

\questionmarks{4(b)}{4}{સુઘડ સ્કેચ સાથે વેરેક્ટર ડાયોડના કાર્ય સમજાવો.}

\begin{solutionbox}
\textbf{વેરેક્ટર ડાયોડ:}
\keyword{વોલ્ટેજ-નિયંત્રિત કેપેસિટર} તરીકે કાર્ય કરે છે.

\begin{answerdiagram}{વેરેક્ટર C-V વક્ર}
\begin{tikzpicture}
    \begin{axis}[
        axis lines = left,
        xlabel = {રિવર્સ વોલ્ટેજ ($V_R$)},
        ylabel = {કેપેસિટન્સ ($C_j$)},
        ymin=0, ymax=100,
        xmin=0, xmax=10,
        height=5cm, width=7cm
    ]
    \addplot [domain=0.1:10, samples=100, blue, thick] {50/sqrt(x)};
    \end{axis}
    \node at (5,3) {$C_j \propto \frac{1}{\sqrt{V_R}}$};
\end{tikzpicture}
\end{answerdiagram}

\textbf{કામગીરી:}
\begin{itemize}
    \item \keyword{રિવર્સ બાયાસ}: ડિપ્લેશન રીજન પહોળું કરે છે $\rightarrow$ કેપેસિટન્સ ઘટાડે છે.
    \item \keyword{ટ્યુનિંગ}: વોલ્ટેજ બદલવાથી $C$ બદલાય છે, જે રેઝોનન્ટ ફ્રીક્વન્સી $f = 1/2\pi\sqrt{LC}$ બદલે છે.
\end{itemize}
\end{solutionbox}

\begin{mnemonicbox}
\mnemonic{Varactor Varies Capacitance with Voltage}
\end{mnemonicbox}

\questionmarks{4(c)}{7}{ટનલ ડાયોડનું બાંધકામ અને કાર્ય સમજાવો અને ટનલ બનાવવાની ઘટનાને વિગતવાર સમજાવો. તેની એપ્લિકેશનોની સૂચિ બનાવો.}

\begin{solutionbox}
\textbf{ટનલ ડાયોડ બાંધકામ:}
\begin{itemize}
    \item \keyword{હેવીલી ડોપ્ડ}: ($10^{19}$ અણુઓ/cm$^3$). ડીજનરેટ P અને N પ્રદેશો.
    \item \keyword{પાતળું જંકશન}: ડિપ્લેશન પહોળાઈ ઓછી (~100 \AA).
\end{itemize}

\textbf{ટનલિંગ ઘટના:}
ક્વોન્ટમ મિકેનિકલ અસર જ્યાં ઇલેક્ટ્રોન પોટેન્શિયલ બેરિયરની ઉપરથી જવાને બદલે તેમાંથી પસાર થાય છે (ટનલ કરે છે), કારણ કે બેરિયર ખૂબ પાતળું છે.

\begin{answerdiagram}{ટનલ ડાયોડ I-V}
\begin{tikzpicture}
    \draw[->] (0,0) -- (5,0) node[right] {વોલ્ટેજ ($V$)};
    \draw[->] (0,0) -- (0,3) node[above] {કરન્ટ ($I$)};
    
    \draw[blue, thick] (0,0) .. controls (0.5,2.5) .. (1,2.5); % Peak
    \draw[blue, thick] (1,2.5) node[above] {$I_p$} -- (1,2.5) .. controls (2,0.5) .. (3,0.5); % Negative R
    \draw[blue, thick] (3,0.5) node[right] {$I_v$} .. controls (4,1) .. (4.5,2.5); % Exponential
    
    \draw[dashed] (1,0) node[below] {$V_p$} -- (1,2.5);
    \draw[dashed] (3,0) node[below] {$V_v$} -- (3,0.5);
    
    \node[red] at (2,1.5) {નેગેટિવ R};
\end{tikzpicture}
\end{answerdiagram}

\textbf{વર્કિંગ રીજીયન્સ:}
\begin{enumerate}
    \item \keyword{પીક પોઇન્ટ ($V_p$)}: મહત્તમ ટનલિંગ કરન્ટ.
    \item \keyword{નેગેટિવ રેઝિસ્ટન્સ ($V_p$ થી $V_v$)}: વોલ્ટેજ વધતા કરન્ટ ઘટે છે. ઓસિલેટર માટે વપરાય છે.
    \item \keyword{વેલી પોઇન્ટ ($V_v$)}: ટનલિંગ બંધ થાય છે.
\end{enumerate}

\textbf{ઉપયોગો}: ઓસિલેટર, હાઇ સ્પીડ સ્વિચિંગ.
\end{solutionbox}

\begin{mnemonicbox}
\mnemonic{Tunnel Diode Tunnels Through barriers Terrifically}
\end{mnemonicbox}

\orquestionmarks{4(a)}{3}{IMPATT ડાયોડની કામગીરીનું વર્ણન કરો.}

\begin{solutionbox}
\textbf{IMPATT ડાયોડ (Impact Avalanche Transit Time):}
માઇક્રોવેવ પાવર જનરેટ કરે છે આનો ઉપયોગ કરીને:
\begin{enumerate}
    \item \keyword{ઇમ્પેક્ટ આયનાઇઝેશન}: એવેલેન્ચ મલ્ટિપ્લિકેશન કેરિયર્સ બનાવે છે (90$^\circ$ ફેઝ શિફ્ટ).
    \item \keyword{ટ્રાન્ઝિટ ટાઇમ}: રીજનમાંથી પસાર થતા કેરિયર્સનો ડ્રિફ્ટ બાકીનો 90$^\circ$ શિફ્ટ ઉમેરે છે.
\end{enumerate}
કુલ 180$^\circ$ ફેઝ ડિલે \keyword{નેગેટિવ રેઝિસ્ટન્સ} માં પરિણમે છે.

\textbf{મુખ્ય આંકડા}: હાઇ પાવર, હાઇ નોઇઝ, બ્રેકડાઉન વોલ્ટેજ ~100V.
\end{solutionbox}

\begin{mnemonicbox}
\mnemonic{IMPATT Impacts with Avalanche Transit Time}
\end{mnemonicbox}

\orquestionmarks{4(b)}{4}{પેરામેટ્રિક એમ્પ્લીફાયર માટે આવર્તન ઉપર અને નીચે રૂપાંતરણ સમજાવો.}

\begin{solutionbox}
\textbf{પેરામેટ્રિક એમ્પ્લીફાયર કન્વર્ઝન:}
નોનલિનિયર રિએક્ટન્સ (વેરેક્ટર) વાપરે છે જેને ફ્રીક્વન્સી $f_p$ પર પંપ કરવામાં આવે છે.

\textbf{અપ-કન્વર્ઝન:}
\begin{itemize}
    \item ઇનપુટ: $f_s$ (સિગ્નલ).
    \item આઉટપુટ: $f_o = f_p + f_s$ (સરવાળો ફ્રીક્વન્સી) અથવા $f_p - f_s$.
    \item \keyword{ગેઇન}: પાવર ગેઇન ફ્રીક્વન્સી રેશિયો ($f_o/f_s$) ના સમપ્રમાણમાં હોય છે. ઓછા નોઇઝ સાથે સ્થિર.
\end{itemize}

\textbf{ડાઉન-કન્વર્ઝન:}
\begin{itemize}
    \item આઉટપુટ: $f_o = f_p - f_s$.
    \item \keyword{નેગેટિવ રેઝિસ્ટન્સ}: અસ્થિરતા/ઓસિલેશન તરફ દોરી શકે છે.
\end{itemize}
\end{solutionbox}

\begin{mnemonicbox}
\mnemonic{Parametric Pump Provides frequency conversion Plus gain}
\end{mnemonicbox}

\orquestionmarks{4(c)}{7}{RUBY MASER ના બાંધકામ અને કાર્ય સિદ્ધાંતનું વર્ણન કરો. તેની એપ્લિકેશનોની સૂચિ બનાવો.}

\begin{solutionbox}
\textbf{રૂબી મેઝર (Maser):}

\begin{answerdiagram}{મેઝર બ્લોક ડાયાગ્રામ}
\begin{tikzpicture}[auto, node distance=1.5cm]
    \node [gtu block, fill=red!10] (ruby) {રૂબી ક્રિસ્ટલ};
    \node [gtu start, above=of ruby] (pump) {પંપ સોર્સ};
    \node [gtu input, left=of ruby] (sig) {સિગ્નલ ઇન};
    \node [gtu output, right=of ruby] (out) {એમ્પ્લીફાઇડ આઉટ};
    \node [below=0.5cm of ruby] (cool) {લિક્વિડ હિલીયમ (4.2K)};
    \node [right=0.5cm of cool] (mag) {મેગ્નેટ ($B$)};
    
    \draw [gtu arrow] (pump) -- (ruby);
    \draw [gtu arrow] (sig) -- (ruby);
    \draw [gtu arrow] (ruby) -- (out);
\end{tikzpicture}
\end{answerdiagram}

\textbf{કાર્યસિદ્ધાંત (સ્ટિમ્યુલેટેડ એમિશન):}
\begin{enumerate}
    \item \keyword{પોપ્યુલેશન ઇન્વર્ઝન}: પંપ એનર્જી ઇલેક્ટ્રોનને અસ્થિર ઉચ્ચ એનર્જી લેવલ ($E_3$) પર લઈ જાય છે.
    \item \keyword{સ્ટિમ્યુલેટેડ એમિશન}: આવતા સિગ્નલ ફોટોન ($E_2$) ઇલેક્ટ્રોનને નીચેના લેવલ પર આવવા ટ્રિગર કરે છે, કોહરન્ટ ફોટોન મુક્ત કરે છે.
    \item \keyword{કૂલિંગ}: લિક્વિડ હિલીયમ થર્મલ નોઇઝ ઘટાડે છે.
\end{enumerate}

\textbf{ઉપયોગો:} ડીપ સ્પેસ કમ્યુનિકેશન (NASA), રેડિયો એસ્ટ્રોનોમી.
\end{solutionbox}

\begin{mnemonicbox}
\mnemonic{RUBY MASER Makes Amazingly Sensitive Electromagnetic Receivers}
\end{mnemonicbox}

\questionmarks{5(a)}{3}{MTI RADARના કાર્યાત્મક બ્લોક ડાયાગ્રામ દોરો અને સમજાવો.}

\begin{solutionbox}
\textbf{MTI (મૂવિંગ ટાર્ગેટ ઇન્ડિકેશન) રડાર:}

\begin{answerdiagram}{MTI રડાર}
\begin{tikzpicture}[auto, node distance=1.5cm]
    \node [gtu block] (tx) {Tx};
    \node [gtu block, right=of tx] (dup) {Duplexer};
    \node [gtu input, right=of dup] (ant) {Ant};
    \node [gtu block, below=of tx] (mix) {Mixer};
    \node [gtu block, below=of mix] (pd) {ફેઝ ડિટેક્ટર};
    \node [gtu block, right=of pd] (dl) {ડિલે લાઇન};
    \node [gtu block, right=of dl] (sub) {સબટ્રેક્ટર};
    
    \node [gtu block, left=of mix] (stalo) {STALO};
    \node [gtu block, left=of pd] (coho) {COHO};
    
    \draw [gtu arrow] (tx) -- (dup);
    \draw [gtu arrow] (dup) -- (ant);
    \draw [gtu arrow] (dup) -- (mix);
    \draw [gtu arrow] (stalo) -- (mix);
    \draw [gtu arrow] (mix) -- (pd);
    \draw [gtu arrow] (coho) -- (pd);
    
    \draw [gtu arrow] (pd) -- (dl);
    \draw [gtu arrow] (pd) -- node[above] {} (sub);
    \draw [gtu arrow] (dl) -- (sub);
    
    \node [right=0.5cm of sub] {સ્કોપ તરફ};
    \draw [->] (sub) -- +(1,0);
\end{tikzpicture}
\end{answerdiagram}

\textbf{કાર્ય:} પલ્સ વચ્ચે ડોપ્લર ફેઝ શિફ્ટ સરખામણીનો ઉપયોગ કરીને ક્લટર (સ્થિર ટાર્ગેટ) થી મૂવિંગ ટાર્ગેટને અલગ કરે છે.
\end{solutionbox}

\begin{mnemonicbox}
\mnemonic{MTI Makes Targets Intelligible by Motion}
\end{mnemonicbox}

\questionmarks{5(b)}{4}{RADAR ને SONAR સાથે સરખાવો.}

\begin{solutionbox}
\textbf{તુલના:}

\begin{answertable}{રડાર vs સોનાર}
\begin{tabulary}{\linewidth}{|L|L|L|}
\hline
\textbf{પેરામીટર} & \textbf{RADAR} & \textbf{SONAR} \\ \hline
\keyword{તરંગ} & EM વેવ્સ (માઇક્રોવેવ્સ) & એકોસ્ટિક (ધ્વનિ) વેવ્સ \\ \hline
\keyword{ઝડપ} & $3 \times 10^8$ m/s & ~1500 m/s \\ \hline
\keyword{માધ્યમ} & હવા, શૂન્યાવકાશ & પાણી \\ \hline
\keyword{રેન્જ} & લાંબી (અવકાશ/હવા) & ટૂંકી (પાણીની અંદર) \\ \hline
\keyword{ઉપયોગ} & વિમાન ટ્રેકિંગ & સબમરીન, માછલી શોધવી \\ \hline
\end{tabulary}
\end{answertable}
\end{solutionbox}

\begin{mnemonicbox}
\mnemonic{RADAR Radiates, SONAR Sounds}
\end{mnemonicbox}

\questionmarks{5(c)}{7}{મહત્તમ RADAR રેંજનું સમીકરણ મેળવો. મહત્તમ રડાર રેંજને અસર કરતા પરિબળો સમજાવો.}

\begin{solutionbox}
\textbf{રડાર રેન્જ સમીકરણ:}

\[ R_{max} = \left[ \frac{P_t G^2 \lambda^2 \sigma}{(4\pi)^3 P_{min}} \right]^{1/4} \]

\textbf{તારવણી:}
\begin{enumerate}
    \item ટાર્ગેટ R પર પાવર ડેન્સિટી: $\frac{P_t G}{4\pi R^2}$.
    \item ટાર્ગેટ ($\sigma$) દ્વારા પરાવર્તિત પાવર: $\frac{P_t G \sigma}{4\pi R^2}$.
    \item રિસીવર પર પાવર ડેન્સિટી (વળતો માર્ગ): $\frac{P_t G \sigma}{(4\pi R^2)^2}$.
    \item એન્ટેનાનો અસરકારક વિસ્તાર $A_e = \frac{G \lambda^2}{4\pi}$.
    \item પ્રાપ્ત પાવર $P_r = \text{Density} \times A_e = \frac{P_t G^2 \lambda^2 \sigma}{(4\pi)^3 R^4}$.
    \item $P_r = P_{min}$ સેટ કરો અને $R$ માટે ઉકેલો.
\end{enumerate}

\textbf{રેન્જને અસર કરતા પરિબળો:}
\begin{itemize}
    \item \keyword{ટ્રાન્સમીટર પાવર ($P_t$)}: $R \propto P_t^{1/4}$. નાના રેન્જ ગેઇન માટે વિશાળ પાવર વધારો જરૂરી છે.
    \item \keyword{એન્ટેના ગેઇન ($G$)}: મોટું એન્ટેના રેન્જ સુધારે છે.
    \item \keyword{ટાર્ગેટ RCS ($\sigma$)}: મોટા/ધાતુના ટાર્ગેટ્સ જોવા સરળ છે.
    \item \keyword{Min ડિટેક્ટેબલ સિગ્નલ ($P_{min}$)}: સારી રિસીવર સેન્સિટિવિટી રેન્જ વધારે છે.
\end{itemize}
\end{solutionbox}


\orquestionmarks{5(a)}{3}{CW Doppler RADAR માં ડોપ્લર અસરનું વર્ણન કરો.}

\begin{solutionbox}
\textbf{ડોપ્લર અસર:}
જ્યારે ટાર્ગેટ રડારની સાપેક્ષ ગતિ કરે છે ત્યારે \keyword{ફ્રીક્વન્સી શિફ્ટ} જોવા મળે છે.

\textbf{સૂત્ર:}
\[ f_d = \frac{2 V_r f_0}{c} = \frac{2 V_r}{\lambda} \]
જ્યાં:
\begin{itemize}
    \item $V_r$: રેડિયલ વેલોસિટી (m/s).
    \item $f_0$: ટ્રાન્સમિટેડ ફ્રીક્વન્સી.
    \item $c$: પ્રકાશની ઝડપ.
\end{itemize}

\textbf{શિફ્ટ દિશા:}
\begin{itemize}
    \item \keyword{નજીક આવતું (Approaching)}: $f_r > f_0$ (પોઝિટિવ શિફ્ટ).
    \item \keyword{દૂર જતું (Receding)}: $f_r < f_0$ (નેગેટિવ શિફ્ટ).
\end{itemize}
\end{solutionbox}

\begin{mnemonicbox}
\mnemonic{Doppler Detects Direction with Doubled frequency shift}
\end{mnemonicbox}

\orquestionmarks{5(b)}{4}{RADAR માટે PPI ડિસ્પ્લે પદ્ધતિ સમજાવો}

\begin{solutionbox}
\textbf{PPI (પ્લાન પોઝિશન ઇન્ડિકેટર):}
રડાર કવરેજનો \keyword{ટોપ-ડાઉન મેપ વ્યૂ} દર્શાવે છે.

\begin{answerdiagram}{PPI ડિસ્પ્લે}
\begin{tikzpicture}
    \draw[thick] (0,0) circle (1.5);
    \draw[->] (0,0) -- (60:1.5) node[midway, above] {રેન્જ};
    \foreach \r in {0.5, 1.0} \draw[gray] (0,0) circle (\r);
    \fill[red] (60:1.0) circle (0.05) node[right] {ટાર્ગેટ};
    \node at (0,-1.8) {કેન્દ્રમાં એન્ટેના};
\end{tikzpicture}
\end{answerdiagram}

\textbf{લક્ષણો:}
\begin{enumerate}
    \item \keyword{કેન્દ્ર}: રડારનું સ્થાન.
    \item \keyword{કોણ}: ટાર્ગેટ એઝિમથ (બેરિંગ).
    \item \keyword{ત્રિજ્યા}: ટાર્ગેટ રેન્જ (અંતર).
    \item \keyword{સ્વીપ}: એન્ટેના સાથે સિંક્રનાઇઝ થયેલ ફરતો ટ્રેસ.
\end{enumerate}
\end{solutionbox}

\begin{mnemonicbox}
\mnemonic{PPI Provides Position Information Perfectly}
\end{mnemonicbox}

\orquestionmarks{5(c)}{7}{પલ્સ રડારનો બ્લોક ડાયાગ્રામ દોરો અને કાર્યસિદ્ધાંત સમજાવો.}

\begin{solutionbox}
\textbf{પલ્સ રડાર બ્લોક ડાયાગ્રામ:}

\begin{answerdiagram}{પલ્સ રડાર}
\begin{tikzpicture}[auto, node distance=1.5cm]
    \node [gtu start] (timer) {ટાઈમર};
    \node [gtu block, right=of timer] (mod) {મોડ્યુલેટર};
    \node [gtu block, right=of mod] (tx) {ટ્રાન્સમીટર};
    \node [gtu block, right=of tx] (dup) {ડુપ્લેક્સર};
    \node [gtu input, right=of dup] (ant) {એન્ટેના};
    
    \node [gtu block, below=of dup] (rx) {રિસીવર};
    \node [gtu stop, left=of rx] (disp) {ડિસ્પ્લે};
    
    \draw [gtu arrow] (timer) -- (mod);
    \draw [gtu arrow] (mod) -- (tx);
    \draw [gtu arrow] (tx) -- (dup);
    \draw [gtu arrow] (dup) -- (ant);
    \draw [gtu arrow] (dup) -- (rx);
    \draw [gtu arrow] (rx) -- (disp);
    \draw [gtu arrow] (timer) -| (disp); % Sync
\end{tikzpicture}
\end{answerdiagram}

\textbf{કાર્યસિદ્ધાંત:}
\begin{itemize}
    \item \keyword{ટ્રાન્સમિશન}: નિયત સમયાંતરે (PRF) હાઇ પાવર પલ્સ ઉત્સર્જિત થાય છે.
    \item \keyword{રિસેપ્શન}: "લિસનિંગ" સમય દરમિયાન પડઘા પ્રાપ્ત થાય છે.
    \item \keyword{ડુપ્લેક્સર}: ટ્રાન્સમિશન દરમિયાન રિસીવરને સુરક્ષિત કરે છે; ઇકોને રિસીવર તરફ વાળે છે.
    \item \keyword{ટાઇમિંગ}: સમય વિલંબ $T$ પરથી અંતર ગણવામાં આવે છે: $R = cT/2$.
\end{itemize}
\end{solutionbox}

\begin{mnemonicbox}
\mnemonic{Pulse RADAR Pulses Powerfully for Precise Position}
\end{mnemonicbox}

\end{document}
