\documentclass{article}

% content/resources/templates/preamble.tex
\usepackage[margin=0.6in]{geometry}
\author{Milav Dabgar}
\usepackage{amsmath,amssymb,amsthm}
\usepackage{booktabs}
\usepackage{multirow}
\usepackage{xcolor}
\usepackage{tcolorbox}
\tcbuselibrary{breakable,skins}
\usepackage[colorlinks=true,linkcolor=blue]{hyperref}
\usepackage{titlesec}
\usepackage{enumitem}
\usepackage{tikz}
\usepackage{pgfplots}
\usepackage{circuitikz}
\usepackage[version=4]{mhchem}
\usepackage{longtable}
\usepackage{array}
\usepackage{float}
\usepackage{caption}
\usepackage{listings}

\lstset{
  basicstyle=\small\ttfamily,
  breaklines=true,
  breakatwhitespace=false,
  postbreak=\mbox{\textcolor{red}{$\hookrightarrow$}\space},
  float=false,
  numbers=left,
  numberstyle=\tiny\color{gray},
  numbersep=10pt,
  xleftmargin=2em,
  keywordstyle=\color{blue},
  commentstyle=\color{green!60!black},
  stringstyle=\color{purple},
  backgroundcolor=\color{gray!5},
  showstringspaces=false,
  tabsize=2,
  captionpos=b,
  keepspaces=true,
  columns=flexible
}

\pgfplotsset{compat=1.18}
\usetikzlibrary{shapes,arrows,positioning,calc,patterns,decorations.pathmorphing,decorations.markings,arrows.meta}

% Color scheme
\definecolor{headcolor}{RGB}{0,102,204}
\definecolor{keycolor}{RGB}{220,20,60}
\definecolor{solutioncolor}{RGB}{34,139,34}
\definecolor{mnemoniccolor}{RGB}{148,0,211}
\definecolor{codecolor}{RGB}{0,0,100}

% Spacing
\setlength{\parskip}{3pt}
\setlist[itemize]{nosep}
\setlist[enumerate]{nosep}

% Title formatting
\titleformat{\section}{\Large\bfseries\color{headcolor}}{\thesection}{1em}{}
\titleformat{\subsection}{\large\bfseries\color{headcolor}}{\thesubsection}{1em}{}

% Pandoc tightlist compatibility
\providecommand{\tightlist}{%
  \setlength{\itemsep}{0pt}\setlength{\parskip}{0pt}}

% Pandoc longtable compatibility
\newcounter{none}
\def\thenone{}


% content/resources/templates/gujarati-boxes.tex
\usepackage{fontspec}
\usepackage{polyglossia}

% Set Gujarati as main language (document is primarily in Gujarati)
% Note: gloss-gujarati.ldf doesn't exist in polyglossia, but it will use hyphenation patterns
\setdefaultlanguage{gujarati}
\setotherlanguage{english}

% Configure Gujarati font properly
% Use Language=Default to prevent polyglossia from trying to add language-specific features
% that don't exist for Gujarati, which causes "empty feature" warnings
\newfontfamily\gujaratifont[Script=Gujarati,AutoFakeBold=2.5,AutoFakeSlant=0.3]{Noto Sans Gujarati}
\setmainfont[Script=Gujarati,AutoFakeBold=2.5,AutoFakeSlant=0.3]{Noto Sans Gujarati}
% Use Noto Sans Gujarati for monospace to support Gujarati in text
\setmonofont[Scale=0.9]{Noto Sans Gujarati}

% Configure English to use the same font
\newfontfamily\englishfont[Script=Gujarati,AutoFakeBold=2.5,AutoFakeSlant=0.3]{Noto Sans Gujarati}

% Translations for polyglossia
\gappto\captionsgujarati{
  \renewcommand{\tablename}{કોષ્ટક}
  \renewcommand{\figurename}{આકૃતિ}
}

% Helper for TikZ nodes to ensure Gujarati font
\newcommand{\gu}[1]{{\gujaratifont #1}}

% Custom environments
\newtcolorbox{solutionbox}{
    breakable,
    enhanced,
    colback=solutioncolor!5!white,
    colframe=solutioncolor!75!black,
    fonttitle=\bfseries,
    title=જવાબ
}

\newtcolorbox{solutionboxnobreak}{
 colback=solutioncolor!5!white,
 colframe=solutioncolor!75!black,
 fonttitle=\bfseries,
 title=જવાબ
}

\newtcolorbox{keyformula}{
 breakable,
 enhanced,
 colback=keycolor!5!white,
 colframe=keycolor!75!black,
 fonttitle=\bfseries,
 title=રાસાયણિક સમીકરણ/સૂત્ર
}

\newtcolorbox{mnemonicbox}{
 breakable,
 enhanced,
 colback=mnemoniccolor!5!white,
 colframe=mnemoniccolor!75!black,
 fonttitle=\bfseries,
 title=મેમરી ટ્રીક
}


% Custom commands for GTU solutions
% This file defines semantic commands for consistent formatting

% Question command with automatic formatting
\newcommand{\question}[2]{%
  \section*{Question #1}%
  \textbf{#2}%
}

% OR question variant
\newcommand{\questionor}[2]{%
  \section*{Question #1 OR}%
  \textbf{#2}%
}

% Proper table environment with caption
\newenvironment{answertable}[1]{%
  \begin{table}[htbp]
  \centering
  \caption{#1}
}{%
  \end{table}
}

% Proper figure environment for diagrams
\newenvironment{answerdiagram}[1]{%
  \begin{figure}[htbp]
  \centering
  \caption{#1}
}{%
  \end{figure}
}

% Semantic markup for key terms
\newcommand{\keyword}[1]{\textbf{#1}}
\newcommand{\code}[1]{\texttt{#1}}
\newcommand{\classname}[1]{\texttt{#1}}
\newcommand{\methodname}[1]{\texttt{#1}}

% Proper quotation marks
\newcommand{\mnemonic}[1]{``#1''}


\title{Embedded System \& Microcontroller Application (4351102) - Winter 2023 Solution Gujarati}
\date{December 4, 2023}

\begin{document}
\maketitle

\questionmarks{1(અ)}{3}{TIFR register દોરો અને તેનું પૂરું નામ લખો.}

\begin{solutionbox}
\textbf{પૂરું નામ}: Timer/Counter Interrupt Flag Register

\textbf{TIFR Register Diagram:}

\begin{answertable}{TIFR Register}
\begin{tabulary}{\linewidth}{|C|C|C|C|C|C|C|C|}
\hline
\textbf{7} & \textbf{6} & \textbf{5} & \textbf{4} & \textbf{3} & \textbf{2} & \textbf{1} & \textbf{0} \\ \hline
OCF2 & TOV2 & ICF1 & OCF1A & OCF1B & TOV1 & OCF0 & TOV0 \\ \hline
\end{tabulary}
\end{answertable}

\keyword{Bit Descriptions:}
\begin{itemize}
    \item \keyword{TOV0}: Timer0 Overflow Flag
    \item \keyword{OCF0}: Timer0 Output Compare Flag
    \item \keyword{TOV1}: Timer1 Overflow Flag
\end{itemize}
\end{solutionbox}

\begin{mnemonicbox}
\mnemonic{Timer Interrupts Flag Register}
\end{mnemonicbox}

\questionmarks{1(બ)}{4}{ATmega32 ની ડેટા મેમરીની ચર્ચા કરો.}

\begin{solutionbox}
\textbf{Data Memory Organization:}

\begin{answertable}{Data Memory Map}
\begin{tabulary}{\linewidth}{|L|L|L|L|}
\hline
\textbf{મેમરી પ્રકાર} & \textbf{કદ} & \textbf{Address Range} & \textbf{હેતુ} \\ \hline
\keyword{General Purpose Registers} & 32 bytes & 0x00-0x1F & R0-R31 registers \\ \hline
\keyword{I/O Memory} & 64 bytes & 0x20-0x5F & Control registers \\ \hline
\keyword{Internal SRAM} & 2048 bytes & 0x60-0x85F & Variable storage \\ \hline
\end{tabulary}
\end{answertable}

\begin{itemize}
    \item \keyword{General Purpose Registers}: અંકગણિત કામગીરી અને અસ્થાયી સંગ્રહ માટે વપરાય છે.
    \item \keyword{I/O Memory}: પેરિફેરલ કંટ્રોલ અને સ્ટેટસ રજિસ્ટર્સ ધરાવે છે.
    \item \keyword{Internal SRAM}: સ્ટેક, વેરિયેબલ્સ અને ડાયનેમિક મેમરી માટે વપરાય છે.
\end{itemize}
\end{solutionbox}

\begin{mnemonicbox}
\mnemonic{General I/O SRAM Memory}
\end{mnemonicbox}

\questionmarks{1(ક)}{7}{એમ્બેડેડ સિસ્ટમનો જનરલ બ્લોક ડાયાગ્રામ દોરી સમજાવો.}

\begin{solutionbox}
\textbf{General Block Diagram:}

\begin{answerdiagram}{Embedded System Block Diagram}
\begin{tikzpicture}[auto, node distance=2cm]
    \node [gtu block] (proc) {Processor/\\Microcontroller};
    \node [gtu block, left=of proc] (input) {Input Devices};
    \node [gtu block, right=of proc] (output) {Output Devices};
    \node [gtu block, above=of proc] (mem) {Memory};
    \node [gtu block, below=of proc] (comm) {Communication\\Interface};
    \node [gtu block, below left=1.5cm of proc] (power) {Power Supply};
    \node [gtu block, below right=1.5cm of proc] (clock) {Clock Circuit};

    \path [gtu arrow] (input) -- (proc);
    \path [gtu arrow] (proc) -- (output);
    \path [gtu arrow] (proc) edge[bend right] (mem);
    \path [gtu arrow] (mem) edge[bend right] (proc);
    \path [gtu arrow] (proc) -- (comm);
    \path [gtu arrow] (power) -- (proc);
    \path [gtu arrow] (clock) -- (proc);
\end{tikzpicture}
\end{answerdiagram}

\textbf{ઘટક કાર્યો:}

\begin{answertable}{ઘટક કાર્યો}
\begin{tabulary}{\linewidth}{|L|L|}
\hline
\textbf{ઘટક} & \textbf{કાર્ય} \\ \hline
\keyword{Processor} & સમગ્ર સિસ્ટમ ઓપરેશન કંટ્રોલ કરે છે \\ \hline
\keyword{Memory} & પ્રોગ્રામ અને ડેટા સ્ટોર કરે છે \\ \hline
\keyword{Input Devices} & સેન્સર, સ્વિચ, કીબોર્ડ \\ \hline
\keyword{Output Devices} & LEDs, ડિસ્પ્લે, મોટર \\ \hline
\keyword{Communication} & UART, SPI, I2C ઇન્ટરફેસ \\ \hline
\end{tabulary}
\end{answertable}

\keyword{લાક્ષણિકતાઓ:}
\begin{itemize}
    \item \keyword{Real-time Operation}: સિસ્ટમ નિર્ધારિત સમય મર્યાદામાં ઇનપુટ્સને પ્રતિસાદ આપે છે.
    \item \keyword{Dedicated Function}: ચોક્કસ એપ્લિકેશન માટે ડિઝાઇન કરવામાં આવે છે.
    \item \keyword{Resource Constraints}: મર્યાદિત મેમરી, પાવર અને પ્રોસેસિંગ ક્ષમતા.
\end{itemize}
\end{solutionbox}

\begin{mnemonicbox}
\mnemonic{Processor Memory Input Output Communication}
\end{mnemonicbox}

\orquestionmarks{1(ક)}{7}{રીયલ ટાઇમ ઓપરેટિંગ સિસ્ટમને વ્યાખ્યાયિત કરો અને તેની લાક્ષણિકતાઓ સમજાવો.}

\begin{solutionbox}
\textbf{વ્યાખ્યા}: \keyword{Real Time Operating System (RTOS)} એ એવી ઓપરેટિંગ સિસ્ટમ છે જે મહત્વપૂર્ણ કાર્યો માટે નિર્દિષ્ટ સમય મર્યાદામાં પ્રતિસાદની ગેરેંટી આપે છે.

\textbf{લાક્ષણિકતાઓ:}

\begin{answertable}{RTOS લાક્ષણિકતાઓ}
\begin{tabulary}{\linewidth}{|L|L|}
\hline
\textbf{લાક્ષણિકતા} & \textbf{વર્ણન} \\ \hline
\keyword{Deterministic} & અનુમાનિત પ્રતિસાદ સમય \\ \hline
\keyword{Multitasking} & બહુવિધ કાર્યોનું અમલીકરણ \\ \hline
\keyword{Priority-based} & ઉચ્ચ પ્રાથમિકતા કાર્યો પહેલા \\ \hline
\keyword{Minimal Latency} & ઝડપી ઇન્ટરપ્ટ પ્રતિસાદ \\ \hline
\end{tabulary}
\end{answertable}

\keyword{મુખ્ય વિભાવનાઓ:}
\begin{itemize}
    \item \keyword{Hard Real-time}: ડેડલાઇન ચૂકવાથી સિસ્ટમ નિષ્ફળતા થાય છે.
    \item \keyword{Soft Real-time}: ડેડલાઇન ચૂકવાથી પ્રદર્શન ઘટે છે.
    \item \keyword{Task Scheduling}: Preemptive priority-based scheduling મહત્વપૂર્ણ કાર્યો પહેલા ચલાવવાની ખાતરી કરે છે.
\end{itemize}
\end{solutionbox}

\begin{mnemonicbox}
\mnemonic{Deterministic Multitasking Priority Minimal}
\end{mnemonicbox}

\questionmarks{2(અ)}{3}{એમ્બેડેડ સિસ્ટમ માટે માઇક્રોકન્ટ્રોલર પસંદ કરવા માટેના માપદંડો લખો.}

\begin{solutionbox}
\textbf{પસંદગી માપદંડ:}

\begin{answertable}{પસંદગી માપદંડ}
\begin{tabulary}{\linewidth}{|L|L|}
\hline
\textbf{માપદંડ} & \textbf{મહત્વ} \\ \hline
\keyword{Processing Speed} & એપ્લિકેશન જરૂરિયાતો સાથે મેળ \\ \hline
\keyword{Memory Size} & પૂરતી ROM/RAM \\ \hline
\keyword{I/O Pins} & પર્યાપ્ત પેરિફેરલ ઇન્ટરફેસ \\ \hline
\keyword{Power Consumption} & બેટરી લાઇફ વિચારણા \\ \hline
\keyword{Cost} & બજેટ મર્યાદા \\ \hline
\keyword{Development Tools} & કમ્પાઇલર, ડીબગર ઉપલબ્ધતા \\ \hline
\end{tabulary}
\end{answertable}
\end{solutionbox}

\begin{mnemonicbox}
\mnemonic{Speed Memory I/O Power Cost Tools}
\end{mnemonicbox}

\questionmarks{2(બ)}{4}{AVR માં હાર્વર્ડ આર્કિટેક્ચરની ચર્ચા કરો.}

\begin{solutionbox}
\textbf{હાર્વર્ડ આર્કિટેક્ચર લક્ષણો:}

\begin{answertable}{Harvard Architecture}
\begin{tabulary}{\linewidth}{|L|L|}
\hline
\textbf{લક્ષણ} & \textbf{વર્ણન} \\ \hline
\keyword{Separate Buses} & પ્રોગ્રામ અને ડેટાને સ્વતંત્ર બસ \\ \hline
\keyword{Simultaneous Access} & એકસાથે instruction fetch અને data access \\ \hline
\keyword{Different Memory Types} & પ્રોગ્રામ માટે Flash, ડેટા માટે SRAM \\ \hline
\end{tabulary}
\end{answertable}

\begin{answerdiagram}{Harvard Architecture}
\begin{tikzpicture}[auto, node distance=2cm]
    \node [gtu block] (cpu) {CPU};
    \node [gtu block, right=of cpu, yshift=1cm] (prog) {Program Bus};
    \node [gtu block, right=of cpu, yshift=-1cm] (data) {Data Bus};
    \node [gtu block, right=of prog] (flash) {Flash Memory};
    \node [gtu block, right=of data] (sram) {SRAM};

    \path [gtu arrow] (cpu) -- (prog);
    \path [gtu arrow] (cpu) -- (data);
    \path [gtu arrow] (prog) -- (flash);
    \path [gtu arrow] (data) -- (sram);
\end{tikzpicture}
\end{answerdiagram}

\begin{itemize}
    \item \keyword{ફાયદો}: સમાંતર એક્સેસને કારણે ઉચ્ચ પ્રદર્શન.
    \item \keyword{16-bit Instructions}: મોટાભાગની instructions એક clock cycle માં execute થાય છે.
\end{itemize}
\end{solutionbox}

\begin{mnemonicbox}
\mnemonic{Separate Simultaneous Different Performance}
\end{mnemonicbox}

\questionmarks{2(ક)}{7}{ક્લોક સોર્સને AVR સાથે જોડવાની વિવિધ રીતોની ચર્ચા કરો.}

\begin{solutionbox}
\textbf{Clock Sources:}

\begin{answertable}{Clock Source Types}
\begin{tabulary}{\linewidth}{|L|L|L|}
\hline
\textbf{Clock Source} & \textbf{Frequency Range} & \textbf{Application} \\ \hline
\keyword{External Crystal} & 1-16 MHz & ઉચ્ચ ચોકસાઈ એપ્લિકેશન \\ \hline
\keyword{External RC} & 1-8 MHz & કિફાયતી સોલ્યુશન \\ \hline
\keyword{Internal RC} & 1-8 MHz & ડિફોલ્ટ, બાહ્ય components નથી \\ \hline
\keyword{External Clock} & Up to 16 MHz & સિંક્રોનાઇઝ્ડ સિસ્ટમ્સ \\ \hline
\end{tabulary}
\end{answertable}

\keyword{Clock Selection via Fuse Bits:}
\begin{itemize}
    \item \keyword{CKSEL3:0}: Bits determine clock source.
    \item \keyword{CKDIV8}: Bit divides clock by 8.
    \item \keyword{SUT1:0}: Bits set startup time.
\end{itemize}

\keyword{વર્ણન:}
\begin{itemize}
    \item \keyword{Crystal Oscillator}: સૌથી સ્થિર, બાહ્ય crystal અને capacitors જરૂરી.
    \item \keyword{RC Oscillator}: ઓછી ચોકસાઈ પરંતુ સસ્તી.
    \item \keyword{Internal Oscillator}: ફેક્ટરી કેલિબ્રેટેડ, તાપમાન આધારિત.
\end{itemize}
\end{solutionbox}

\begin{mnemonicbox}
\mnemonic{Crystal RC Internal External}
\end{mnemonicbox}

\orquestionmarks{2(અ)}{3}{ATmega32 માટે code ROM, SRAM અને EEPROM નું કદ તેમજ I/O pins, ADC અને Timers ની સંખ્યા લખો.}

\begin{solutionbox}
\textbf{ATmega32 Specifications:}

\begin{answertable}{Device Specifications}
\begin{tabulary}{\linewidth}{|L|L|}
\hline
\textbf{સ્પેશિફિકેશન} & \textbf{ATmega32} \\ \hline
\keyword{Flash ROM} & 32 KB \\ \hline
\keyword{SRAM} & 2 KB \\ \hline
\keyword{EEPROM} & 1 KB \\ \hline
\keyword{I/O Pins} & 32 pins \\ \hline
\keyword{ADC Channels} & 8 channels \\ \hline
\keyword{Timers} & 3 timers \\ \hline
\end{tabulary}
\end{answertable}
\end{solutionbox}

\begin{mnemonicbox}
\mnemonic{32K Flash 2K SRAM 1K EEPROM 32 I/O 8 ADC 3 Timers}
\end{mnemonicbox}

\orquestionmarks{2(બ)}{4}{ATmega32 પિન ડાયાગ્રામ દોરો અને Vcc, AVcc અને Aref પિનનું કાર્ય લખો.}

\begin{solutionbox}
\textbf{ATmega32 પિન કાર્યો:}

\begin{answertable}{Pin Functions}
\begin{tabulary}{\linewidth}{|L|L|}
\hline
\textbf{પિન} & \textbf{કાર્ય} \\ \hline
\keyword{Vcc} & મુખ્ય પાવર સપ્લાય (+5V) \\ \hline
\keyword{AVcc} & ADC માટે એનાલોગ પાવર સપ્લાય \\ \hline
\keyword{Aref} & ADC રેફરન્સ વોલ્ટેજ \\ \hline
\end{tabulary}
\end{answertable}

\begin{center}
\begin{tikzpicture}[
    pin/.style={draw, rectangle, minimum width=2.5cm, minimum height=0.5cm, font=\small},
    ic/.style={draw, rectangle, minimum width=4cm, minimum height=8cm, fill=gray!10}
]
    \node [ic] (body) {};
    \node [anchor=north, font=\large\bfseries] at (body.north) {ATmega32};

    % Selected Pins for Diagram
    \node [anchor=west, font=\tiny] at ([yshift=-1cm]body.north west) {1 VCC};
    \draw ([yshift=-1cm]body.north west) -- +(-0.2,0);
    
    \node [anchor=east, font=\tiny] at ([yshift=-1cm]body.north east) {AVCC 30};
    \draw ([yshift=-1cm]body.north east) -- +(0.2,0);

    \node [anchor=east, font=\tiny] at ([yshift=-2cm]body.north east) {AREF 32};
    \draw ([yshift=-2cm]body.north east) -- +(0.2,0);

     \node [anchor=east, font=\tiny] at ([yshift=-6cm]body.north east) {GND 31};
    \draw ([yshift=-6cm]body.north east) -- +(0.2,0);
    
    \node [align=center] at (body.center) { (Simplified\\Pin View)};

\end{tikzpicture}
\end{center}

\begin{itemize}
    \item \keyword{Vcc}: ડિજિટલ સર્કિટ્સને પાવર સપ્લાય કરે છે.
    \item \keyword{AVcc}: નોઇઝ ઘટાડવા માટે ADC માટે અલગ સપ્લાય.
    \item \keyword{Aref}: ADC કન્વર્ઝન માટે બાહ્ય રેફરન્સ.
\end{itemize}
\end{solutionbox}

\begin{mnemonicbox}
\mnemonic{Vcc Digital AVcc Analog Aref Reference}
\end{mnemonicbox}

\orquestionmarks{2(ક)}{7}{AVR સ્ટેટસ રજિસ્ટર વિગતવાર સમજાવો.}

\begin{solutionbox}
\textbf{SREG (Status Register) બિટ્સ:}

\begin{answertable}{SREG Bits}
\begin{tabulary}{\linewidth}{|C|C|L|}
\hline
\textbf{બિટ} & \textbf{નામ} & \textbf{કાર્ય} \\ \hline
7 & I & Global Interrupt Enable \\ \hline
6 & T & Bit Copy Storage \\ \hline
5 & H & Half Carry Flag \\ \hline
4 & S & Sign Flag \\ \hline
3 & V & Overflow Flag \\ \hline
2 & N & Negative Flag \\ \hline
1 & Z & Zero Flag \\ \hline
0 & C & Carry Flag \\ \hline
\end{tabulary}
\end{answertable}

\begin{answertable}{SREG Layout}
\begin{tabulary}{\linewidth}{|C|C|C|C|C|C|C|C|}
\hline
\textbf{I} & \textbf{T} & \textbf{H} & \textbf{S} & \textbf{V} & \textbf{N} & \textbf{Z} & \textbf{C} \\ \hline
7 & 6 & 5 & 4 & 3 & 2 & 1 & 0 \\ \hline
\end{tabulary}
\end{answertable}

\keyword{બિટ વિગતો:}
\begin{itemize}
    \item \keyword{I Flag}: ગ્લોબલ ઇન્ટરપ્ટ \code{enable/disable} કંટ્રોલ કરે છે.
    \item \keyword{Arithmetic Flags}: ALU ઓપરેશન પછી C, Z, N, V, S, H અપડેટ થાય છે.
    \item \keyword{T Flag}: બિટ મેનિપ્યુલેશન માટે \code{BLD} અને \code{BST} instructions દ્વારા વપરાય છે.
\end{itemize}
\end{solutionbox}

\begin{mnemonicbox}
\mnemonic{I Transfer Half Sign oVerflow Negative Zero Carry}
\end{mnemonicbox}

\questionmarks{3(અ)}{3}{AVR માઇક્રોકન્ટ્રોલર માટે RESET સર્કિટ સમજાવો.}

\begin{solutionbox}
\textbf{રીસેટ સોર્સ:}

\begin{answertable}{Reset Sources}
\begin{tabulary}{\linewidth}{|L|L|}
\hline
\textbf{રીસેટ સોર્સ} & \textbf{વર્ણન} \\ \hline
\keyword{Power-on Reset} & પાવર લાગુ કરવામાં આવે ત્યારે \\ \hline
\keyword{External Reset} & RESET pin દ્વારા \\ \hline
\keyword{Brown-out Reset} & વોલ્ટેજ ઘટે ત્યારે \\ \hline
\keyword{Watchdog Reset} & Watchdog timer overflow \\ \hline
\end{tabulary}
\end{answertable}

\textbf{Reset Circuit:}

\begin{center}
\begin{tikzpicture}[auto, node distance=1.5cm]
    \node (vcc) {Vcc};
    \node [below=of vcc] (node1) {};
    \node [below=of node1] (gnd) {GND};
    \node [right=of node1] (reset) {RESET Pin};

    \draw (vcc) -- node[left] {R} (node1);
    \draw (node1) -- node[left] {C} (gnd);
    \draw (node1) -- (reset);
\end{tikzpicture}
\end{center}

\begin{itemize}
    \item \keyword{રીસેટ અવધિ}: ઓછામાં ઓછા 2 clock cycles.
    \item \keyword{રીસેટ વેક્ટર}: પ્રોગ્રામ address 0x0000 થી શરૂ થાય છે.
\end{itemize}
\end{solutionbox}

\begin{mnemonicbox}
\mnemonic{Power External Brown-out Watchdog}
\end{mnemonicbox}

\questionmarks{3(બ)}{4}{EEPROM સાથે સંકળાયેલ I/O રજિસ્ટરની યાદી બનાવો. EEPROM પર data write કરવા માટેના પ્રોગ્રામિંગ સ્ટેપ્સ લખો.}

\begin{solutionbox}
\textbf{EEPROM રજિસ્ટર્સ:}

\begin{answertable}{EEPROM Registers}
\begin{tabulary}{\linewidth}{|L|L|}
\hline
\textbf{રજિસ્ટર} & \textbf{કાર્ય} \\ \hline
\keyword{EEAR} & EEPROM Address Register \\ \hline
\keyword{EEDR} & EEPROM Data Register \\ \hline
\keyword{EECR} & EEPROM Control Register \\ \hline
\end{tabulary}
\end{answertable}

\textbf{પ્રોગ્રામિંગ સ્ટેપ્સ:}
\begin{enumerate}
    \item પાછલી write પૂર્ણ થવાની રાહ જુઓ (\keyword{EEWE} bit ચેક કરો).
    \item \keyword{EEAR} રજિસ્ટરમાં address સેટ કરો.
    \item \keyword{EEDR} રજિસ્ટરમાં data સેટ કરો.
    \item \keyword{EECR} માં \keyword{EEMWE} bit સેટ કરો.
    \item 4 clock cycles અંદર \keyword{EEWE} bit સેટ કરો.
\end{enumerate}
\end{solutionbox}

\begin{mnemonicbox}
\mnemonic{Wait Address Data Master-Write Enable-Write}
\end{mnemonicbox}

\questionmarks{3(ક)}{7}{TCCR0 રજિસ્ટર દોરી વિગતવાર સમજાવો.}

\begin{solutionbox}
\textbf{TCCR0 (Timer/Counter0 Control Register):}

\begin{answertable}{TCCR0 Bits}
\begin{tabulary}{\linewidth}{|C|C|L|}
\hline
\textbf{બિટ} & \textbf{નામ} & \textbf{કાર્ય} \\ \hline
7 & FOC0 & Force Output Compare \\ \hline
6,3 & WGM01/00 & Waveform Generation Mode \\ \hline
5,4 & COM01/00 & Compare Output Mode \\ \hline
2,1,0 & CS02/01/00 & Clock Select \\ \hline
\end{tabulary}
\end{answertable}

\begin{answertable}{TCCR0 Layout}
\begin{tabulary}{\linewidth}{|C|C|C|C|C|C|C|C|}
\hline
\textbf{FOC0} & \textbf{WGM01} & \textbf{COM01} & \textbf{COM00} & \textbf{WGM00} & \textbf{CS02} & \textbf{CS01} & \textbf{CS00} \\ \hline
7 & 6 & 5 & 4 & 3 & 2 & 1 & 0 \\ \hline
\end{tabulary}
\end{answertable}

\keyword{ક્લોક સિલેક્ટ વિકલ્પો:}
\begin{itemize}
    \item \keyword{000}: કોઈ ક્લોક નહીં (Timer બંધ).
    \item \keyword{001}: clk/1 (પ્રેસ્કેલિંગ નહીં).
    \item \keyword{010}: clk/8, \keyword{011}: clk/64.
    \item \keyword{100}: clk/256, \keyword{101}: clk/1024.
\end{itemize}
\end{solutionbox}

\begin{mnemonicbox}
\mnemonic{Force Waveform Compare Clock Select}
\end{mnemonicbox}

\orquestionmarks{3(અ)}{3}{Timer 1 સાથે સંકળાયેલા રજિસ્ટરોની યાદી બનાવો.}

\begin{solutionbox}
\textbf{Timer1 રજિસ્ટર્સ:}

\begin{answertable}{Timer1 Registers}
\begin{tabulary}{\linewidth}{|L|L|}
\hline
\textbf{રજિસ્ટર} & \textbf{કાર્ય} \\ \hline
\keyword{TCCR1A} & Timer1 Control Register A \\ \hline
\keyword{TCCR1B} & Timer1 Control Register B \\ \hline
\keyword{TCNT1H/L} & Timer1 Counter Register \\ \hline
\keyword{OCR1AH/L} & Output Compare Register A \\ \hline
\keyword{OCR1BH/L} & Output Compare Register B \\ \hline
\keyword{ICR1H/L} & Input Capture Register \\ \hline
\end{tabulary}
\end{answertable}
\end{solutionbox}

\begin{mnemonicbox}
\mnemonic{Control Counter Output-Compare Input-Capture}
\end{mnemonicbox}

\orquestionmarks{3(બ)}{4}{EEPROM ના 0x005F લોકેશન પર 'G' સ્ટોર કરવા માટે AVR C પ્રોગ્રામ લખો.}

\begin{solutionbox}
\textbf{પ્રોગ્રામ:}

\begin{lstlisting}[language=C]
#include <avr/io.h>
#include <avr/eeprom.h>

void eeprom_write_byte_custom(uint16_t addr, uint8_t data)
{
    while(EECR & (1<<EEWE));  // Wait for previous write
    EEAR = addr;              // Set address
    EEDR = data;              // Set data
    EECR |= (1<<EEMWE);       // Master write enable
    EECR |= (1<<EEWE);        // Write enable
}

int main()
{
    eeprom_write_byte_custom(0x005F, 'G');
    return 0;
}
\end{lstlisting}

\keyword{પ્રોગ્રામ સ્ટેપ્સ:}
\begin{itemize}
    \item પૂર્ણતા માટે \keyword{EEWE} bit ચેક કરો.
    \item \keyword{EEAR} માં address \code{0x005F} લોડ કરો.
    \item \keyword{EEDR} માં 'G' (ASCII 71) લોડ કરો.
    \item Master write સક્ષમ કરો, પછી write enable કરો.
\end{itemize}
\end{solutionbox}

\begin{mnemonicbox}
\mnemonic{Wait Address Data Master Write}
\end{mnemonicbox}

\orquestionmarks{3(ક)}{7}{દર 70 $\mu$s પર માત્ર PORTB.4 બિટને ટૉગલ કરવા માટે C પ્રોગ્રામ લખો. Delay બનાવવા માટે Timer0નો 1:8 પ્રેસ્કેલર સાથે નોર્મલ મોડનો ઉપયોગ કરો. XTAL = 8 MHz.}

\begin{solutionbox}
\textbf{ગણતરી:}
\begin{itemize}
    \item \keyword{ક્લોક} = 8MHz/8 = 1MHz ($1\mu s$ period).
    \item 70$\mu s$ માટે: \keyword{Count} = 70 cycles.
    \item \keyword{પ્રારંભિક મૂલ્ય} = $256 - 70 = 186$.
\end{itemize}

\textbf{પ્રોગ્રામ:}

\begin{lstlisting}[language=C]
#include <avr/io.h>

int main()
{
    DDRB |= (1<<4);           // Set PB4 as output
    TCCR0 = 0x02;             // Prescaler 1:8
    
    while(1)
    {
        TCNT0 = 186;          // Load initial value
        while(!(TIFR & (1<<TOV0))); // Wait for overflow
        TIFR |= (1<<TOV0);    // Clear flag
        PORTB ^= (1<<4);      // Toggle PB4
    }
    return 0;
}
\end{lstlisting}
\end{solutionbox}

\begin{mnemonicbox}
\mnemonic{Direction Control Count Wait Clear Toggle}
\end{mnemonicbox}

\questionmarks{4(અ)}{3}{Port C ના બિટ 5 ને મોનિટર કરવા માટેનો AVR C પ્રોગ્રામ લખો. જો તે HIGH હોય, તો Port B પર 55H મોકલો; અન્યથા, AAH Port B પર મોકલો.}

\begin{solutionbox}
\textbf{પ્રોગ્રામ:}

\begin{lstlisting}[language=C]
#include <avr/io.h>

int main()
{
    DDRC &= ~(1<<5);          // PC5 as input
    DDRB = 0xFF;              // Port B as output
    
    while(1)
    {
        if(PINC & (1<<5))     // Check PC5
            PORTB = 0x55;     // Send 55H if HIGH
        else
            PORTB = 0xAA;     // Send AAH if LOW
    }
    return 0;
}
\end{lstlisting}

\keyword{પ્રોગ્રામ લૉજિક:}
\begin{itemize}
    \item PC5 ને input તરીકે, Port B ને output તરીકે કૉન્ફિગર કરો.
    \item સતત PC5 સ્થિતિ ચેક કરો using bitwise AND.
    \item ઇનપુટના આધારે \code{0x55} અથવા \code{0xAA} આઉટપુટ કરો.
\end{itemize}
\end{solutionbox}

\begin{mnemonicbox}
\mnemonic{Direction Check Output}
\end{mnemonicbox}

\questionmarks{4(બ)}{4}{LM35 ને ATmega32 સાથે ઇન્ટરફેસિંગ દોરો અને સમજાવો.}

\begin{solutionbox}
\textbf{LM35 Interface:}

\begin{answerdiagram}{LM35 Connection}
\begin{tikzpicture}[auto, node distance=2cm]
    \node [gtu block] (lm35) {LM35\\Sensor};
    \node [gtu block, right=of lm35] (mcu) {ATmega32\\(ADC)};
    
    \draw [gtu arrow] (lm35) -- node[midway, above] {Vout} node[midway, below] {(10mV/$^\circ$C)} (mcu);
    
    \node [above=0.5cm of lm35] (vcc) {+5V};
    \node [below=0.5cm of lm35] (gnd) {GND};
    \draw [gtu arrow] (vcc) -- (lm35);
    \draw [gtu arrow] (lm35) -- (gnd);
    
    \node [right=0.1cm of mcu] (adc) {PA0 (ADC0)};
\end{tikzpicture}
\end{answerdiagram}

\begin{answertable}{Connection Details}
\begin{tabulary}{\linewidth}{|L|L|L|}
\hline
\textbf{LM35 પિન} & \textbf{ATmega32 પિન} & \textbf{કાર્ય} \\ \hline
Vcc & +5V & પાવર સપ્લાય \\ \hline
Output & PA0 (ADC0) & એનાલોગ વોલ્ટેજ \\ \hline
GND & GND & ગ્રાઉન્ડ \\ \hline
\end{tabulary}
\end{answertable}

\keyword{સ્પેસિફિકેશન્સ:}
\begin{itemize}
    \item \keyword{તાપમાન કન્વર્ઝન}: 10mV/$^\circ$C આઉટપુટ.
    \item \keyword{ADC રિઝોલ્યુશન}: 10-bit (0-1023).
    \item \keyword{વોલ્ટેજ રેન્જ}: 0V થી 5V (0$^\circ$C થી 500$^\circ$C).
\end{itemize}
\end{solutionbox}

\begin{mnemonicbox}
\mnemonic{Power Output Ground Temperature}
\end{mnemonicbox}

\questionmarks{4(ક)}{7}{MAX7221 ને ATmega32 સાથે ઇન્ટરફેસિંગ દોરો અને સમજાવો.}

\begin{solutionbox}
\textbf{MAX7221 Interface:}

\begin{answerdiagram}{MAX7221 Connection}
\begin{tikzpicture}[auto, node distance=2cm]
    \node [gtu block] (mcu) {ATmega32};
    \node [gtu block, right=of mcu] (max) {MAX7221};
    \node [gtu block, right=of max] (disp) {7-Segment\\Display};

    \draw [gtu arrow] (mcu.20) -- node[above, font=\tiny] {MOSI (PB5)} (max.160);
    \draw [gtu arrow] (mcu.0) -- node[above, font=\tiny] {SCK (PB7)} (max.180);
    \draw [gtu arrow] (mcu.-20) -- node[above, font=\tiny] {SS (PB4)} (max.200);
    \draw [gtu arrow] (max) -- (disp);
\end{tikzpicture}
\end{answerdiagram}

\begin{answertable}{Pin Connections}
\begin{tabulary}{\linewidth}{|L|L|L|}
\hline
\textbf{MAX7221 પિન} & \textbf{ATmega32 પિન} & \textbf{કાર્ય} \\ \hline
DIN & MOSI (PB5) & સીરિયલ ડેટા ઇનપુટ \\ \hline
CLK & SCK (PB7) & સીરિયલ ક્લોક \\ \hline
LOAD & SS (PB4) & ચિપ સિલેક્ટ \\ \hline
\end{tabulary}
\end{answertable}

\keyword{લક્ષણો:}
\begin{itemize}
    \item \keyword{SPI ઇન્ટરફેસ}: સીરિયલ કમ્યુનિકેશન પ્રોટોકોલ.
    \item \keyword{8-ડિજિટ ડિસ્પ્લે}: 8 સેવન-સેગમેન્ટ ડિસ્પ્લે સુધી કંટ્રોલ કરે છે.
    \item \keyword{બિલ્ટ-ઇન ડીકોડર}: BCD થી સેવન-સેગમેન્ટ કન્વર્ઝન.
    \item \keyword{બ્રાઇટનેસ કંટ્રોલ}: 16 ઇન્ટેન્સિટી લેવલ રજિસ્ટર દ્વારા.
\end{itemize}

\keyword{પ્રોગ્રામિંગ સ્ટેપ્સ:}
\begin{enumerate}
    \item SPI ને master મોડમાં પ્રારંભ કરો.
    \item Address અને data bytes મોકલો.
    \item ડેટા latch કરવા માટે LOAD સિગ્નલ pulse કરો.
\end{enumerate}
\end{solutionbox}

\begin{mnemonicbox}
\mnemonic{Serial Clock Load Display}
\end{mnemonicbox}

\orquestionmarks{4(અ)}{3}{Port B માંથી ડેટા બાઇટ મેળવી તેને Port C પર મોકલવા માટે AVR C પ્રોગ્રામ લખો.}

\begin{solutionbox}
\textbf{પ્રોગ્રામ:}

\begin{lstlisting}[language=C]
#include <avr/io.h>

int main()
{
    DDRB = 0x00;              // Port B as input
    DDRC = 0xFF;              // Port C as output
    
    unsigned char data;
    
    while(1)
    {
        data = PINB;          // Read from Port B
        PORTC = data;         // Send to Port C
    }
    return 0;
}
\end{lstlisting}

\keyword{પ્રોગ્રામ કાર્ય:}
\begin{itemize}
    \item Port B ને input તરીકે, Port C ને output તરીકે કૉન્ફિગર કરો.
    \item સતત \keyword{PINB} માંથી વાંચો અને \keyword{PORTC} માં લખો.
\end{itemize}
\end{solutionbox}

\begin{mnemonicbox}
\mnemonic{Input Output Read Write}
\end{mnemonicbox}

\orquestionmarks{4(બ)}{4}{ULN2803 ને ATmega32 સાથે ઇન્ટરફેસિંગ દોરો અને સમજાવો.}

\begin{solutionbox}
\textbf{ULN2803 Interface:}

\begin{answerdiagram}{ULN2803 Connection}
\begin{tikzpicture}[auto, node distance=2cm]
    \node [gtu block] (mcu) {ATmega32\\Port};
    \node [gtu block, right=of mcu] (uln) {ULN2803};
    \node [gtu block, right=of uln] (load) {Load/Relay};
    \node [above=0.5cm of load] (vcc) {Vcc};

    \draw [gtu arrow] (mcu) -- node[midway, above, font=\tiny] {Input} (uln);
    \draw [gtu arrow] (uln) -- node[midway, above, font=\tiny] {Output} (load);
    \draw [gtu arrow] (vcc) -- (load);
\end{tikzpicture}
\end{answerdiagram}

\keyword{ULN2803 લક્ષણો:}
\begin{itemize}
    \item \keyword{8 Darlington Arrays}: હાઇ કરન્ટ સ્વિચિંગ.
    \item \keyword{Input Current}: 500$\mu$A સામાન્ય.
    \item \keyword{Output Current}: 500mA પ્રતિ ચેનલ.
    \item \keyword{Built-in Flyback Diodes}: ઇન્ડક્ટિવ લોડ પ્રોટેક્શન.
\end{itemize}

\keyword{ઓપરેશન}:
\begin{itemize}
    \item \keyword{એપ્લિકેશન}: રિલે, મોટર, સોલેનોઇડ ચલાવવા માટે.
    \item \keyword{એક્ટિવ લો આઉટપુટ}: ઇનપુટ high હોય ત્યારે આઉટપુટ low જાય છે.
\end{itemize}
\end{solutionbox}

\begin{mnemonicbox}
\mnemonic{Darlington Current Protection Drive}
\end{mnemonicbox}

\orquestionmarks{4(ક)}{7}{AVR માં SPI ને પ્રોગ્રામ કરવા માટે વપરાતા રજિસ્ટરોની ચર્ચા કરો.}

\begin{solutionbox}
\textbf{SPI Registers:}

\begin{answertable}{SPI Register Summary}
\begin{tabulary}{\linewidth}{|L|L|L|}
\hline
\textbf{રજિસ્ટર} & \textbf{બિટ્સ} & \textbf{કાર્ય} \\ \hline
\keyword{SPCR} & SPE, DORD, MSTR, CPOL & SPI Control Register \\ \hline
\keyword{SPSR} & SPIF, WCOL, SPI2X & SPI Status Register \\ \hline
\keyword{SPDR} & - & SPI Data Register \\ \hline
\end{tabulary}
\end{answertable}

\keyword{SPCR રજિસ્ટર બિટ્સ:}
\begin{itemize}
    \item \keyword{SPE}: SPI Enable.
    \item \keyword{DORD}: Data Order (MSB/LSB first).
    \item \keyword{MSTR}: Master/Slave Select.
    \item \keyword{CPOL}: Clock Polarity.
    \item \keyword{CPHA}: Clock Phase.
\end{itemize}

\keyword{SPSR રજિસ્ટર બિટ્સ:}
\begin{itemize}
    \item \keyword{SPIF}: SPI Interrupt Flag.
    \item \keyword{WCOL}: Write Collision Flag.
    \item \keyword{SPI2X}: Double Speed Mode.
\end{itemize}

\keyword{પ્રોગ્રામિંગ સિક્વન્સ:}
\begin{enumerate}
    \item SPI pins ને input/output તરીકે કૉન્ફિગર કરો.
    \item ઇચ્છિત મોડ માટે SPCR રજિસ્ટર સેટ કરો.
    \item SPDR માં ડેટા લખો.
    \item SPIF flag ની રાહ જુઓ.
    \item SPDR માંથી પ્રાપ્ત ડેટા વાંચો.
\end{enumerate}
\end{solutionbox}

\begin{mnemonicbox}
\mnemonic{Control Status Data Enable Order Master}
\end{mnemonicbox}

\questionmarks{5(અ)}{3}{L293D મોટર ડ્રાઇવર IC નો પિન ડાયાગ્રામ દોરો અને સમજાવો.}

\begin{solutionbox}
\textbf{L293D Pinout:}

\begin{center}
\begin{tikzpicture}[
    pin/.style={draw, rectangle, minimum width=2cm, minimum height=0.5cm, font=\tiny},
    ic/.style={draw, rectangle, minimum width=3cm, minimum height=5cm, fill=gray!10}
]
    \node [ic] (body) {};
    \node [anchor=north, font=\bfseries] at (body.north) {L293D};

    % Pins Left
    \foreach \i/\label in {1/Enable1, 2/Input1, 3/Output1, 4/GND, 5/GND, 6/Output2, 7/Input2, 8/Vcc2} {
        \node [anchor=west, font=\tiny] at ([yshift=-0.5cm-\i*0.5cm]body.north west) {\i\ \label};
        \draw ([yshift=-0.5cm-\i*0.5cm]body.north west) -- +(-0.2,0);
    }

    % Pins Right
    \foreach \i/\label in {16/Vcc1, 15/Input4, 14/Output4, 13/GND, 12/GND, 11/Output3, 10/Input3, 9/Enable2} {
        \pgfmathsetmacro{\ypos}{17-\i}
        \node [anchor=east, font=\tiny] at ([yshift=-0.5cm-\ypos*0.5cm]body.north east) {\label\ \i};
        \draw ([yshift=-0.5cm-\ypos*0.5cm]body.north east) -- +(0.2,0);
    }
\end{tikzpicture}
\end{center}

\keyword{પિન કાર્યો:}
\begin{itemize}
    \item \keyword{1A, 2A}: મોટર 1 માટે ઇનપુટ સિગ્નલ.
    \item \keyword{1Y, 2Y}: મોટર 1 માટે આઉટપુટ.
    \item \keyword{1EN, 2EN}: મોટર માટે enable pins.
    \item \keyword{Vcc1}: લૉજિક સપ્લાય (+5V).
    \item \keyword{Vcc2}: મોટર સપ્લાય (+12V).
\end{itemize}
\end{solutionbox}

\begin{mnemonicbox}
\mnemonic{Input Output Enable Logic Motor Supply}
\end{mnemonicbox}

\questionmarks{5(બ)}{4}{ADMUX રજિસ્ટર દોરો અને સમજાવો.}

\begin{solutionbox}
\textbf{ADMUX (ADC Multiplexer Selection Register):}

\begin{answertable}{ADMUX Register}
\begin{tabulary}{\linewidth}{|C|C|L|}
\hline
\textbf{બિટ} & \textbf{નામ} & \textbf{કાર્ય} \\ \hline
7,6 & REFS1/0 & Reference Selection \\ \hline
5 & ADLAR & ADC Left Adjust Result \\ \hline
4-0 & MUX4-0 & Analog Channel Selection \\ \hline
\end{tabulary}
\end{answertable}

\begin{answertable}{ADMUX Bits}
\begin{tabulary}{\linewidth}{|C|C|C|C|C|C|C|C|}
\hline
\textbf{REFS1} & \textbf{REFS0} & \textbf{ADLAR} & \textbf{MUX4} & \textbf{MUX3} & \textbf{MUX2} & \textbf{MUX1} & \textbf{MUX0} \\ \hline
7 & 6 & 5 & 4 & 3 & 2 & 1 & 0 \\ \hline
\end{tabulary}
\end{answertable}

\keyword{રેફરન્સ પસંદગી (REFS1:0):}
\begin{itemize}
    \item \keyword{00}: AREF pin.
    \item \keyword{01}: AVcc with external capacitor.
    \item \keyword{11}: Internal 2.56V reference.
\end{itemize}

\keyword{ચેનલ પસંદગી}: MUX bits ADC0-ADC7 ચેનલ પસંદ કરે છે.
\end{solutionbox}

\begin{mnemonicbox}
\mnemonic{Reference Adjust Multiplexer Channel}
\end{mnemonicbox}

\questionmarks{5(ક)}{7}{GSM આધારિત સિક્યોરિટી સિસ્ટમ સમજાવો.}

\begin{solutionbox}
\textbf{GSM Security System:}

\begin{answerdiagram}{GSM Security Block Diagram}
\begin{tikzpicture}[auto, node distance=2cm]
    \node [gtu block] (mcu) {Microcontroller};
    \node [gtu block, left=of mcu] (sensor) {Sensors};
    \node [gtu block, right=of mcu] (gsm) {GSM Module};
    \node [gtu block, right=of gsm] (network) {Network};
    \node [gtu block, below=of network] (mobile) {User Mobile};
    \node [gtu block, above=of mcu] (lcd) {Display};
    \node [gtu block, below=of mcu] (keypad) {Keypad};

    \draw [gtu arrow] (sensor) -- (mcu);
    \draw [gtu arrow] (mcu) -- (gsm);
    \draw [gtu arrow] (gsm) -- (network);
    \draw [gtu arrow] (network) -- (mobile);
    \draw [gtu arrow] (keypad) -- (mcu);
    \draw [gtu arrow] (mcu) -- (lcd);
\end{tikzpicture}
\end{answerdiagram}

\textbf{સિસ્ટમ ઘટકો:}
\begin{itemize}
    \item \keyword{Sensors}: PIR (ગતિ શોધ), Door (પ્રવેશ) શોધ.
    \item \keyword{GSM Module}: SMS/Call કમ્યુનિકેશન.
    \item \keyword{Microcontroller}: સિસ્ટમ કંટ્રોલ.
    \item \keyword{Keypad/Display}: યુઝર ઇન્ટરફેસ.
\end{itemize}

\textbf{કાર્યશીલ સિદ્ધાંત:}
\begin{enumerate}
    \item સેન્સર્સ આક્રમણ શોધે છે.
    \item માઇક્રોકન્ટ્રોલર સિગ્નલ પ્રોસેસ કરે છે.
    \item GSM મોડ્યુલ SMS alert મોકલે છે ("Intruder Detected").
    \item યુઝર નોટિફિકેશન મેળવે છે અને સિસ્ટમ રિમોટલી arm/disarm કરી શકે છે.
\end{enumerate}

\textbf{લક્ષણો}: રિમોટ મોનિટરિંગ, બહુવિધ સેન્સર્સ, ઓટોમેટિક એલર્ટ.
\end{solutionbox}

\begin{mnemonicbox}
\mnemonic{Sensors Process Communicate Alert Control}
\end{mnemonicbox}

\orquestionmarks{5(અ)}{3}{L293D મોટર ડ્રાઇવરનો ઉપયોગ કરી DC મોટરને ATmega32 સાથે ઇન્ટરફેસ કરવા માટે સર્કિટ ડાયાગ્રામ દોરો.}

\begin{solutionbox}
\textbf{DC Motor Interface:}

\begin{answerdiagram}{L293D DC Motor Interface}
\begin{tikzpicture}[auto, node distance=2cm]
    \node [gtu block] (mcu) {ATmega32};
    \node [gtu block, right=of mcu] (l293) {L293D};
    \node [gtu block, right=of l293] (motor) {DC Motor};

    \draw [gtu arrow] (mcu.20) -- node[above, font=\tiny] {PA0 (1A)} (l293.160);
    \draw [gtu arrow] (mcu.0) -- node[above, font=\tiny] {PA1 (2A)} (l293.180);
    \draw [gtu arrow] (mcu.-20) -- node[above, font=\tiny] {PA2 (1EN)} (l293.200);

    \draw [gtu arrow] (l293.20) -- node[above, font=\tiny] {1Y} (motor.160);
    \draw [gtu arrow] (l293.-20) -- node[above, font=\tiny] {2Y} (motor.200);

    \node [above=0.5cm of l293] (vcc) {Vcc};
    \draw [gtu arrow] (vcc) -- (l293);
\end{tikzpicture}
\end{answerdiagram}

\keyword{કનેક્શન ટેબલ:}
\begin{itemize}
    \item \keyword{PA0} to Input 1A.
    \item \keyword{PA1} to Input 2A.
    \item \keyword{PA2} to Enable 1EN.
\end{itemize}

\keyword{કંટ્રોલ લૉજિક:}
\begin{itemize}
    \item \keyword{Clockwise}: PA0=1, PA1=0.
    \item \keyword{Counter-Clockwise}: PA0=0, PA1=1.
    \item \keyword{Stop}: PA2=0.
\end{itemize}
\end{solutionbox}

\begin{mnemonicbox}
\mnemonic{Direction Enable Control Stop}
\end{mnemonicbox}

\orquestionmarks{5(બ)}{4}{ADCSRA રજિસ્ટર દોરો અને સમજાવો.}

\begin{solutionbox}
\textbf{ADCSRA (ADC Control and Status Register A):}

\begin{answertable}{ADCSRA Register}
\begin{tabulary}{\linewidth}{|C|C|L|}
\hline
\textbf{બિટ} & \textbf{નામ} & \textbf{કાર્ય} \\ \hline
7 & ADEN & ADC Enable \\ \hline
6 & ADSC & ADC Start Conversion \\ \hline
5 & ADATE & ADC Auto Trigger Enable \\ \hline
4 & ADIF & ADC Interrupt Flag \\ \hline
3 & ADIE & ADC Interrupt Enable \\ \hline
2-0 & ADPS2-0 & ADC Prescaler Select \\ \hline
\end{tabulary}
\end{answertable}

\begin{answertable}{ADCSRA Layout}
\begin{tabulary}{\linewidth}{|C|C|C|C|C|C|C|C|}
\hline
\textbf{ADEN} & \textbf{ADSC} & \textbf{ADATE} & \textbf{ADIF} & \textbf{ADIE} & \textbf{ADPS2} & \textbf{ADPS1} & \textbf{ADPS0} \\ \hline
7 & 6 & 5 & 4 & 3 & 2 & 1 & 0 \\ \hline
\end{tabulary}
\end{answertable}

\keyword{પ્રેસ્કેલર પસંદગી:}
\begin{itemize}
    \item \keyword{000, 001}: ડિવિઝન ફેક્ટર 2.
    \item \keyword{010}: ડિવિઝન ફેક્ટર 4.
    \item \keyword{011}: ડિવિઝન ફેક્ટર 8.
\end{itemize}

\keyword{ADC ઓપરેશન:}
\begin{enumerate}
    \item ADC સક્ષમ કરવા માટે \keyword{ADEN} સેટ કરો.
    \item કન્વર્ઝન શરૂ કરવા માટે \keyword{ADSC} સેટ કરો.
    \item \keyword{ADIF} flag ની રાહ જુઓ.
    \item \keyword{ADCH:ADCL} માંથી પરિણામ વાંચો.
\end{enumerate}
\end{solutionbox}

\begin{mnemonicbox}
\mnemonic{Enable Start Auto Interrupt Prescaler}
\end{mnemonicbox}


\orquestionmarks{5(ક)}{7}{વેધર મોનિટરિંગ સિસ્ટમ સમજાવો.}

\begin{solutionbox}
\textbf{Weather Monitoring System:}

\begin{answerdiagram}{Weather Monitoring Block Diagram}
\begin{tikzpicture}[auto, node distance=2cm]
    \node [gtu block] (mcu) {Microcontroller};
    \node [gtu block, left=of mcu, yshift=1.5cm] (temp) {Temperature\\Sensor};
    \node [gtu block, left=of mcu, yshift=0.5cm] (hum) {Humidity\\Sensor};
    \node [gtu block, left=of mcu, yshift=-0.5cm] (pres) {Pressure\\Sensor};
    \node [gtu block, left=of mcu, yshift=-1.5cm] (rain) {Rain\\Sensor};
    
    \node [gtu block, right=of mcu, yshift=1cm] (disp) {Display};
    \node [gtu block, right=of mcu, yshift=0cm] (log) {Data Logger};
    \node [gtu block, right=of mcu, yshift=-1cm] (wireless) {Wireless\\Module};
    \node [gtu block, right=of wireless] (remote) {Remote\\Monitor};

    \draw [gtu arrow] (temp) -- (mcu);
    \draw [gtu arrow] (hum) -- (mcu);
    \draw [gtu arrow] (pres) -- (mcu);
    \draw [gtu arrow] (rain) -- (mcu);
    
    \draw [gtu arrow] (mcu) -- (disp);
    \draw [gtu arrow] (mcu) -- (log);
    \draw [gtu arrow] (mcu) -- (wireless);
    \draw [gtu arrow] (wireless) -- (remote);
\end{tikzpicture}
\end{answerdiagram}

\textbf{સિસ્ટમ ઘટકો:}
\begin{answertable}{Sensor Components}
\begin{tabulary}{\linewidth}{|L|L|L|}
\hline
\textbf{સેન્સર} & \textbf{પેરામીટર} & \textbf{ઇન્ટરફેસ} \\ \hline
\keyword{LM35} & તાપમાન & Analog (ADC) \\ \hline
\keyword{DHT11} & ભેજ & Digital \\ \hline
\keyword{BMP180} & દબાણ & I2C \\ \hline
\keyword{Rain Sensor} & વરસાદ & Digital \\ \hline
\end{tabulary}
\end{answertable}

\keyword{લક્ષણો:}
\begin{itemize}
    \item \keyword{Multi-parameter Monitoring}: તાપમાન, ભેજ, દબાણ, વરસાદ.
    \item \keyword{Data Logging}: EEPROM/SD કાર્ડમાં રીડિંગ્સ સ્ટોર કરો.
    \item \keyword{Real-time Display}: LCD વર્તમાન રીડિંગ્સ દર્શાવે છે.
    \item \keyword{Wireless Communication}: રિમોટ મોનિટરિંગ માટે WiFi/GSM.
    \item \keyword{Alert System}: થ્રેશોલ્ડ-આધારિત ચેતવણીઓ.
\end{itemize}

\keyword{એપ્લિકેશન્સ:}
\begin{itemize}
    \item કૃષિ મોનિટરિંગ
    \item હવામાન આગાહી
    \item પર્યાવરણીય સંશોધન
    \item સ્માર્ટ હોમ ઓટોમેશન
\end{itemize}

\keyword{સિસ્ટમ ફાયદા:}
\begin{itemize}
    \item \keyword{Automated Data Collection}: સતત મોનિટરિંગ.
    \item \keyword{Remote Access}: ગમે ત્યાંથી ડેટા જુઓ.
    \item \keyword{Historical Analysis}: ટ્રેન્ડ ઓળખ.
    \item \keyword{Early Warning}: આત્યંતિક હવામાન એલર્ટ્સ.
\end{itemize}
\end{solutionbox}

\begin{mnemonicbox}
\mnemonic{Temperature Humidity Pressure Rain Display Log Wireless}
\end{mnemonicbox}

\end{document}
