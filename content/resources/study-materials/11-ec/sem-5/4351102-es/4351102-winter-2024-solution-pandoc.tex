\documentclass[10pt,a4paper]{article}

% content/resources/templates/preamble.tex
\usepackage[margin=0.6in]{geometry}
\author{Milav Dabgar}
\usepackage{amsmath,amssymb,amsthm}
\usepackage{booktabs}
\usepackage{multirow}
\usepackage{xcolor}
\usepackage{tcolorbox}
\tcbuselibrary{breakable,skins}
\usepackage[colorlinks=true,linkcolor=blue]{hyperref}
\usepackage{titlesec}
\usepackage{enumitem}
\usepackage{tikz}
\usepackage{pgfplots}
\usepackage{circuitikz}
\usepackage[version=4]{mhchem}
\usepackage{longtable}
\usepackage{array}
\usepackage{float}
\usepackage{caption}
\usepackage{listings}

\lstset{
  basicstyle=\small\ttfamily,
  breaklines=true,
  breakatwhitespace=false,
  postbreak=\mbox{\textcolor{red}{$\hookrightarrow$}\space},
  float=false,
  numbers=left,
  numberstyle=\tiny\color{gray},
  numbersep=10pt,
  xleftmargin=2em,
  keywordstyle=\color{blue},
  commentstyle=\color{green!60!black},
  stringstyle=\color{purple},
  backgroundcolor=\color{gray!5},
  showstringspaces=false,
  tabsize=2,
  captionpos=b,
  keepspaces=true,
  columns=flexible
}

\pgfplotsset{compat=1.18}
\usetikzlibrary{shapes,arrows,positioning,calc,patterns,decorations.pathmorphing,decorations.markings,arrows.meta}

% Color scheme
\definecolor{headcolor}{RGB}{0,102,204}
\definecolor{keycolor}{RGB}{220,20,60}
\definecolor{solutioncolor}{RGB}{34,139,34}
\definecolor{mnemoniccolor}{RGB}{148,0,211}
\definecolor{codecolor}{RGB}{0,0,100}

% Spacing
\setlength{\parskip}{3pt}
\setlist[itemize]{nosep}
\setlist[enumerate]{nosep}

% Title formatting
\titleformat{\section}{\Large\bfseries\color{headcolor}}{\thesection}{1em}{}
\titleformat{\subsection}{\large\bfseries\color{headcolor}}{\thesubsection}{1em}{}

% Pandoc tightlist compatibility
\providecommand{\tightlist}{%
  \setlength{\itemsep}{0pt}\setlength{\parskip}{0pt}}

% Pandoc longtable compatibility
\newcounter{none}
\def\thenone{}


% content/resources/templates/english-boxes.tex
% This file is currently empty - it exists to maintain consistency with the import structure.
% Add custom environments here if needed in the future.


\begin{document}

\begin{center}
{\Huge\bfseries\color{headcolor} Subject Name Solutions}\\[5pt]
{\LARGE 4351102 -- Winter 2024}\\[3pt]
{\large Semester 1 Study Material}\\[3pt]
{\normalsize\textit{Detailed Solutions and Explanations}}
\end{center}

\vspace{10pt}

\subsection*{Question 1(a) [3 marks]}\label{q1a}

\textbf{State the features of ATmega32.}

\begin{solutionbox}

{\def\LTcaptype{none} % do not increment counter
\begin{longtable}[]{@{}ll@{}}
\toprule\noalign{}
Feature & Description \\
\midrule\noalign{}
\endhead
\bottomrule\noalign{}
\endlastfoot
\textbf{Architecture} & 8-bit RISC processor \\
\textbf{Memory} & 32KB Flash, 2KB SRAM, 1KB EEPROM \\
\textbf{I/O Ports} & 32 programmable I/O pins \\
\textbf{Timers} & 3 timers (Timer0, Timer1, Timer2) \\
\textbf{ADC} & 10-bit, 8-channel ADC \\
\textbf{Communication} & USART, SPI, I2C (TWI) \\
\end{longtable}
}

\begin{itemize}
\tightlist
\item
  \textbf{High Performance}: 16 MIPS at 16MHz
\item
  \textbf{Low Power}: Multiple sleep modes
\item
  \textbf{Operating Voltage}: 2.7V to 5.5V
\end{itemize}

\end{solutionbox}
\begin{mnemonicbox}
``ARM-TIC'' (Architecture-RISC, Memory-32KB,
Timers-3, I/O-32pins, Communication-3types)

\end{mnemonicbox}
\begin{center}\rule{0.5\linewidth}{0.5pt}\end{center}

\subsection*{Question 1(b) [4 marks]}\label{q1b}

\textbf{Explain criteria for choosing microcontroller.}

\begin{solutionbox}

{\def\LTcaptype{none} % do not increment counter
\begin{longtable}[]{@{}ll@{}}
\toprule\noalign{}
Criteria & Consideration \\
\midrule\noalign{}
\endhead
\bottomrule\noalign{}
\endlastfoot
\textbf{Performance} & Speed, instruction set, architecture \\
\textbf{Memory} & RAM, ROM, EEPROM requirements \\
\textbf{I/O Requirements} & Number of pins, special functions \\
\textbf{Power Consumption} & Battery life, sleep modes \\
\textbf{Cost} & Unit price, development cost \\
\textbf{Development Tools} & Compiler, debugger availability \\
\end{longtable}
}

\begin{itemize}
\tightlist
\item
  \textbf{Application Requirements}: Real-time constraints, processing
  needs
\item
  \textbf{Package Size}: Space limitations in final product
\item
  \textbf{Peripheral Support}: ADC, timers, communication interfaces
\end{itemize}

\end{solutionbox}
\begin{mnemonicbox}
``PM-IPCD'' (Performance, Memory, I/O, Power, Cost,
Development)

\end{mnemonicbox}
\begin{center}\rule{0.5\linewidth}{0.5pt}\end{center}

\subsection*{Question 1(c) [7 marks]}\label{q1c}

\textbf{Define the Embedded System. List the Application of Small,
Medium, Large Embedded System.}

\begin{solutionbox}

\textbf{Definition}: Embedded system is a computer system with dedicated
function within a larger mechanical or electrical system, designed to
perform specific tasks with real-time constraints.

\textbf{Applications Table}:

{\def\LTcaptype{none} % do not increment counter
\begin{longtable}[]{@{}lll@{}}
\toprule\noalign{}
System Type & Memory Size & Applications \\
\midrule\noalign{}
\endhead
\bottomrule\noalign{}
\endlastfoot
\textbf{Small Scale} & \textless64KB & Calculator, Digital watch,
Toys \\
\textbf{Medium Scale} & 64KB-1MB & Mobile phones, Routers, Printers \\
\textbf{Large Scale} & \textgreater1MB & Automobiles, Aircraft systems,
Satellites \\
\end{longtable}
}

\begin{center}
\textbf{Mermaid Diagram (Code)}
\begin{verbatim}
{Shaded}
{Highlighting}[]
graph TD
    A[Embedded System] {-{-}{} B[Small Scale]}
    A {-{-}{} C[Medium Scale]  }
    A {-{-}{} D[Large Scale]}
    B {-{-}{} E[Calculator{}br/{}Digital Watch{}br/{}Remote Control]}
    C {-{-}{} F[Mobile Phone{}br/{}Router{}br/{}Printer]}
    D {-{-}{} G[Car ECU{}br/{}Aircraft Control{}br/{}Medical Equipment]}
{Highlighting}
{Shaded}
\end{verbatim}
\end{center}

\textbf{Characteristics}:

\begin{itemize}
\tightlist
\item
  \textbf{Real-time Operation}: Predictable response times
\item
  \textbf{Resource Constraints}: Limited memory and processing power
\item
  \textbf{Dedicated Functionality}: Single-purpose design
\end{itemize}

\end{solutionbox}
\begin{mnemonicbox}
``SML-CMP''
(Small-Calculator/Medium-Mobile/Large-Lifesupport)

\end{mnemonicbox}
\begin{center}\rule{0.5\linewidth}{0.5pt}\end{center}

\subsection*{Question 1(c) OR [7
marks]}\label{q1c}

\textbf{Draw and explain general block diagram of embedded system.}

\begin{solutionbox}

\begin{center}
\textbf{Mermaid Diagram (Code)}
\begin{verbatim}
{Shaded}
{Highlighting}[]
graph LR
    A[Input Interface] {-{-}{} B[Processor/Controller]}
    B {-{-}{} C[Output Interface]}
    B {-{-}{} D[Memory{}br/{}RAM/ROM/EEPROM]}
    B {-{-}{} E[Communication{}br/{}Interface]}
    F[Sensors] {-{-}{} A}
    C {-{-}{} G[Actuators/Display]}
    E {-{-}{} H[External Systems]}
    I[Power Supply] {-{-}{} B}
{Highlighting}
{Shaded}
\end{verbatim}
\end{center}

\textbf{Block Functions}:

{\def\LTcaptype{none} % do not increment counter
\begin{longtable}[]{@{}ll@{}}
\toprule\noalign{}
Block & Function \\
\midrule\noalign{}
\endhead
\bottomrule\noalign{}
\endlastfoot
\textbf{Processor} & Central processing unit (CPU/MCU) \\
\textbf{Input Interface} & Sensor data acquisition, user input \\
\textbf{Output Interface} & Actuator control, display output \\
\textbf{Memory} & Program storage, data storage \\
\textbf{Communication} & External system connectivity \\
\end{longtable}
}

\begin{itemize}
\tightlist
\item
  \textbf{Input Processing}: ADC, digital input conditioning
\item
  \textbf{Output Control}: PWM, relay drivers, LED displays
\item
  \textbf{Power Management}: Voltage regulation, power optimization
\end{itemize}

\end{solutionbox}
\begin{mnemonicbox}
``PIOMCP'' (Processor, Input, Output, Memory,
Communication, Power)

\end{mnemonicbox}
\begin{center}\rule{0.5\linewidth}{0.5pt}\end{center}

\subsection*{Question 2(a) [3 marks]}\label{q2a}

\textbf{Write a Full form of EEPROM and explain EEPROM registers.}

\begin{solutionbox}

\textbf{Full Form}: Electrically Erasable Programmable Read-Only Memory

\textbf{EEPROM Registers}:

{\def\LTcaptype{none} % do not increment counter
\begin{longtable}[]{@{}ll@{}}
\toprule\noalign{}
Register & Function \\
\midrule\noalign{}
\endhead
\bottomrule\noalign{}
\endlastfoot
\textbf{EEAR} & EEPROM Address Register \\
\textbf{EEDR} & EEPROM Data Register \\
\textbf{EECR} & EEPROM Control Register \\
\end{longtable}
}

\begin{itemize}
\tightlist
\item
  \textbf{EEAR}: Holds 10-bit address (0-1023) for EEPROM access
\item
  \textbf{EEDR}: Data register for read/write operations
\item
  \textbf{EECR}: Control bits - EERE (Read Enable), EEWE (Write Enable)
\end{itemize}

\end{solutionbox}
\begin{mnemonicbox}
``AAD-CRE'' (Address-EEAR, Data-EEDR, Control-EECR)

\end{mnemonicbox}
\begin{center}\rule{0.5\linewidth}{0.5pt}\end{center}

\subsection*{Question 2(b) [4 marks]}\label{q2b}

\textbf{Explain reset circuits for ATmega32}

\begin{solutionbox}

\textbf{Reset Sources Table}:

{\def\LTcaptype{none} % do not increment counter
\begin{longtable}[]{@{}ll@{}}
\toprule\noalign{}
Reset Type & Trigger Condition \\
\midrule\noalign{}
\endhead
\bottomrule\noalign{}
\endlastfoot
\textbf{Power-on Reset} & VCC rises above threshold \\
\textbf{External Reset} & RESET pin pulled low \\
\textbf{Brown-out Reset} & VCC falls below threshold \\
\textbf{Watchdog Reset} & Watchdog timer overflow \\
\end{longtable}
}

\begin{center}
\textbf{Mermaid Diagram (Code)}
\begin{verbatim}
{Shaded}
{Highlighting}[]
graph TD
    A[Power{-on] {-}{-}{} E[Reset Vector]}
    B[External Pin] {-{-}{} E}
    C[Brown{-out] {-}{-}{} E}
    D[Watchdog] {-{-}{} E}
    E {-{-}{} F[Program Counter = 0x0000]}
{Highlighting}
{Shaded}
\end{verbatim}
\end{center}

\begin{itemize}
\tightlist
\item
  \textbf{Reset Duration}: Minimum 2 clock cycles
\item
  \textbf{Reset Vector}: Program execution starts from address 0x0000
\item
  \textbf{Hardware Connection}: External reset requires pull-up resistor
\end{itemize}

\end{solutionbox}
\begin{mnemonicbox}
``PEBW'' (Power-on, External, Brown-out, Watchdog)

\end{mnemonicbox}
\begin{center}\rule{0.5\linewidth}{0.5pt}\end{center}

\subsection*{Question 2(c) [7 marks]}\label{q2c}

\textbf{Define Real Time Operating System and explain its
characteristics.}

\begin{solutionbox}

\textbf{Definition}: Real Time Operating System (RTOS) is an operating
system designed to handle real-time applications with strict timing
constraints and predictable response times.

\textbf{Characteristics Table}:

{\def\LTcaptype{none} % do not increment counter
\begin{longtable}[]{@{}ll@{}}
\toprule\noalign{}
Characteristic & Description \\
\midrule\noalign{}
\endhead
\bottomrule\noalign{}
\endlastfoot
\textbf{Deterministic} & Predictable execution times \\
\textbf{Preemptive} & Higher priority tasks interrupt lower ones \\
\textbf{Multitasking} & Multiple tasks execution \\
\textbf{Fast Response} & Minimal interrupt latency \\
\textbf{Priority-based} & Task scheduling based on priority \\
\textbf{Resource Management} & Efficient memory and CPU usage \\
\end{longtable}
}

\begin{center}
\textbf{Mermaid Diagram (Code)}
\begin{verbatim}
{Shaded}
{Highlighting}[]
graph TD
    A[RTOS] {-{-}{} B[Hard Real{-}time]}
    A {-{-}{} C[Soft Real{-}time]}
    B {-{-}{} D[Strict Deadlines{}br/{}Safety Critical]}
    C {-{-}{} E[Flexible Deadlines{}br/{}Performance Critical]}
{Highlighting}
{Shaded}
\end{verbatim}
\end{center}

\begin{itemize}
\tightlist
\item
  \textbf{Task Scheduling}: Round-robin, priority-based algorithms
\item
  \textbf{Inter-task Communication}: Semaphores, message queues
\item
  \textbf{Memory Management}: Static allocation for predictability
\end{itemize}

\end{solutionbox}
\begin{mnemonicbox}
``DPM-FPR'' (Deterministic, Preemptive, Multitasking,
Fast, Priority, Resource)

\end{mnemonicbox}
\begin{center}\rule{0.5\linewidth}{0.5pt}\end{center}

\subsection*{Question 2(a) OR [3
marks]}\label{q2a}

\textbf{Explain AVR family.}

\begin{solutionbox}

\textbf{AVR Family Classification}:

{\def\LTcaptype{none} % do not increment counter
\begin{longtable}[]{@{}ll@{}}
\toprule\noalign{}
AVR Type & Features \\
\midrule\noalign{}
\endhead
\bottomrule\noalign{}
\endlastfoot
\textbf{ATtiny} & 8-32 pins, basic features \\
\textbf{ATmega} & 28-100 pins, full features \\
\textbf{ATxmega} & Advanced features, DMA \\
\end{longtable}
}

\begin{itemize}
\tightlist
\item
  \textbf{Architecture}: 8-bit RISC, Harvard architecture
\item
  \textbf{Instruction Set}: 130+ instructions, single cycle execution
\item
  \textbf{Memory}: Flash program memory, SRAM, EEPROM
\end{itemize}

\end{solutionbox}
\begin{mnemonicbox}
``TAX'' (Tiny-basic, mega-full, Xmega-advanced)

\end{mnemonicbox}
\begin{center}\rule{0.5\linewidth}{0.5pt}\end{center}

\subsection*{Question 2(b) OR [4
marks]}\label{q2b}

\textbf{Explain the use of fuse bits for selection of ATmega32 clock
sources.}

\begin{solutionbox}

\textbf{Clock Source Selection}:

{\def\LTcaptype{none} % do not increment counter
\begin{longtable}[]{@{}ll@{}}
\toprule\noalign{}
Fuse Bits & Clock Source \\
\midrule\noalign{}
\endhead
\bottomrule\noalign{}
\endlastfoot
\textbf{CKSEL3:0} & Clock source selection \\
\textbf{SUT1:0} & Start-up time selection \\
\end{longtable}
}

\textbf{Clock Options Table}:

{\def\LTcaptype{none} % do not increment counter
\begin{longtable}[]{@{}lll@{}}
\toprule\noalign{}
CKSEL Value & Clock Source & Frequency \\
\midrule\noalign{}
\endhead
\bottomrule\noalign{}
\endlastfoot
0001 & External Crystal & 1-8 MHz \\
0010 & External Crystal & 8+ MHz \\
0100 & Internal RC & 8 MHz \\
0000 & External Clock & User defined \\
\end{longtable}
}

\begin{itemize}
\tightlist
\item
  \textbf{Crystal Selection}: Requires external crystal and capacitors
\item
  \textbf{RC Oscillator}: Built-in, less accurate but convenient
\item
  \textbf{Start-up Time}: Allows crystal stabilization
\end{itemize}

\end{solutionbox}
\begin{mnemonicbox}
``CRIS'' (Crystal, RC, Internal, Start-up)

\end{mnemonicbox}
\begin{center}\rule{0.5\linewidth}{0.5pt}\end{center}

\subsection*{Question 2(c) OR [7
marks]}\label{q2c}

\textbf{Draw ATmega32 pin configuration and explain function of MISO,
MOSI, SCK \& AREF Pin.}

\begin{solutionbox}

\begin{verbatim}
        +{-{-}{-}{-}{-}{-}{-}{-}{-}{-}+}
    PB0 |1      40| PA0
    PB1 |2      39| PA1  
    PB2 |3      38| PA2
    PB3 |4      37| PA3
    PB4 |5      36| PA4
MOSI PB5|6      35| PA5
MISO PB6|7      34| PA6
 SCK PB7|8      33| PA7
   RESET|9      32| AREF
    VCC |10     31| GND
    GND |11     30| AVCC
   XTAL2|12     29| PC7
   XTAL1|13     28| PC6
        +{-{-}{-}{-}{-}{-}{-}{-}{-}{-}+}
\end{verbatim}

\textbf{Pin Functions Table}:

{\def\LTcaptype{none} % do not increment counter
\begin{longtable}[]{@{}lll@{}}
\toprule\noalign{}
Pin & Function & Description \\
\midrule\noalign{}
\endhead
\bottomrule\noalign{}
\endlastfoot
\textbf{MOSI} & Master Out Slave In & SPI data output from master \\
\textbf{MISO} & Master In Slave Out & SPI data input to master \\
\textbf{SCK} & Serial Clock & SPI clock signal \\
\textbf{AREF} & Analog Reference & ADC reference voltage \\
\end{longtable}
}

\begin{itemize}
\tightlist
\item
  \textbf{SPI Communication}: MOSI, MISO, SCK work together for serial
  data transfer
\item
  \textbf{ADC Reference}: AREF provides stable voltage reference for ADC
  conversion
\item
  \textbf{Pin Multiplexing}: These pins have alternate functions as GPIO
\end{itemize}

\end{solutionbox}
\begin{mnemonicbox}
``MMS-A'' (MOSI-out, MISO-in, SCK-clock,
AREF-reference)

\end{mnemonicbox}
\begin{center}\rule{0.5\linewidth}{0.5pt}\end{center}

\subsection*{Question 3(a) [3 marks]}\label{q3a}

\textbf{Explain Role of DDR I/O Register}

\begin{solutionbox}

\textbf{DDR (Data Direction Register) Functions}:

{\def\LTcaptype{none} % do not increment counter
\begin{longtable}[]{@{}ll@{}}
\toprule\noalign{}
Bit Value & Pin Configuration \\
\midrule\noalign{}
\endhead
\bottomrule\noalign{}
\endlastfoot
\textbf{0} & Input pin \\
\textbf{1} & Output pin \\
\end{longtable}
}

\begin{itemize}
\tightlist
\item
  \textbf{Port Control}: Each port has corresponding DDR (DDRA, DDRB,
  DDRC, DDRD)
\item
  \textbf{Bit-wise Control}: Individual pin direction control
\item
  \textbf{Default State}: All pins input (DDR = 0x00) after reset
\end{itemize}

\textbf{Code Example}:

\begin{verbatim}
DDRA = 0xFF;  // All Port A pins as output
DDRB = 0x0F;  // PB0{-PB3 output, PB4{-}PB7 input}
\end{verbatim}

\end{solutionbox}
\begin{mnemonicbox}
``DDR-IO'' (Data Direction Register controls
Input/Output)

\end{mnemonicbox}
\begin{center}\rule{0.5\linewidth}{0.5pt}\end{center}

\subsection*{Question 3(b) [4 marks]}\label{q3b}

\textbf{Write an AVR C program to get a byte of data from Port B, and
then send it to Port C.}

\begin{solutionbox}

\begin{verbatim}
\#include {avr/io.h}

int main(void)
\{
    unsigned char data;
    
    // Configure Port B as input
    DDRB = 0x00;
    
    // Configure Port C as output  
    DDRC = 0xFF;
    
    while(1)
    \{
        // Read data from Port B
        data = PINB;
        
        // Send data to Port C
        PORTC = data;
    \}
    
    return 0;
\}
\end{verbatim}

\textbf{Program Explanation}:

\begin{itemize}
\tightlist
\item
  \textbf{DDRB = 0x00}: Sets all Port B pins as input
\item
  \textbf{DDRC = 0xFF}: Sets all Port C pins as output
\item
  \textbf{PINB}: Reads current state of Port B pins
\item
  \textbf{PORTC}: Writes data to Port C output pins
\end{itemize}

\end{solutionbox}
\begin{mnemonicbox}
``RSTO'' (Read-PINB, Set-DDR, Transfer-data,
Output-PORTC)

\end{mnemonicbox}
\begin{center}\rule{0.5\linewidth}{0.5pt}\end{center}

\subsection*{Question 3(c) [7 marks]}\label{q3c}

\textbf{A door sensor is connected to the port B pin 1, and an LED is
connected to port C pin7. Write an AVR C program to monitor the door
sensor and, when it opens, turn on the LED.}

\begin{solutionbox}

\begin{verbatim}
\#include {avr/io.h}

int main(void)
\{
    // Configure PB1 as input (door sensor)
    DDRB \&= {(}1{}1);  // Clear bit 1
    
    // Configure PC7 as output (LED)
    DDRC |= (1{}7);   // Set bit 7
    
    // Enable pull{-up for PB1}
    PORTB |= (1{}1);
    
    while(1)
    \{
        // Check door sensor status
        if(PINB \& (1{}1))
        \{
            // Door closed {- turn off LED}
            PORTC \&= {(}1{}7);
        \}
        else
        \{
            // Door open {- turn on LED  }
            PORTC |= (1{}7);
        \}
    \}
    
    return 0;
\}
\end{verbatim}

\textbf{Hardware Connection}:

\begin{itemize}
\tightlist
\item
  \textbf{Door Sensor}: Connected between PB1 and GND
\item
  \textbf{LED}: Connected to PC7 through current limiting resistor
\item
  \textbf{Pull-up}: Internal pull-up enabled for PB1
\end{itemize}

\textbf{Program Logic}:

\begin{itemize}
\tightlist
\item
  \textbf{Sensor Closed}: PB1 = HIGH, LED OFF
\item
  \textbf{Sensor Open}: PB1 = LOW, LED ON
\end{itemize}

\end{solutionbox}
\begin{mnemonicbox}
``DCOL'' (Door-sensor, Configure-pins, Open-check,
LED-control)

\end{mnemonicbox}
\begin{center}\rule{0.5\linewidth}{0.5pt}\end{center}

\subsection*{Question 3(a) OR [3
marks]}\label{q3a}

\textbf{Discuss Data Types in AVR C programming.}

\begin{solutionbox}

\textbf{AVR C Data Types Table}:

{\def\LTcaptype{none} % do not increment counter
\begin{longtable}[]{@{}lll@{}}
\toprule\noalign{}
Data Type & Size & Range \\
\midrule\noalign{}
\endhead
\bottomrule\noalign{}
\endlastfoot
\textbf{char} & 8-bit & -128 to 127 \\
\textbf{unsigned char} & 8-bit & 0 to 255 \\
\textbf{int} & 16-bit & -32768 to 32767 \\
\textbf{unsigned int} & 16-bit & 0 to 65535 \\
\textbf{long} & 32-bit & -2^{3}^{1} to 2^{3}^{1}-1 \\
\textbf{float} & 32-bit & IEEE 754 format \\
\end{longtable}
}

\begin{itemize}
\tightlist
\item
  \textbf{Memory Efficiency}: Use smallest appropriate data type
\item
  \textbf{Unsigned Types}: For positive values only, doubles range
\item
  \textbf{Bit Fields}: Can define specific bit-width variables
\end{itemize}

\end{solutionbox}
\begin{mnemonicbox}
``CIL-FUB'' (Char-8bit, Int-16bit, Long-32bit,
Float-32bit, Unsigned-positive, Bit-specific)

\end{mnemonicbox}
\begin{center}\rule{0.5\linewidth}{0.5pt}\end{center}

\subsection*{Question 3(b) OR [4
marks]}\label{q3b}

\textbf{Explain Serial Communication Protocol.}

\begin{solutionbox}

\textbf{Serial Communication Parameters}:

{\def\LTcaptype{none} % do not increment counter
\begin{longtable}[]{@{}ll@{}}
\toprule\noalign{}
Parameter & Description \\
\midrule\noalign{}
\endhead
\bottomrule\noalign{}
\endlastfoot
\textbf{Baud Rate} & Data transmission speed (bits/second) \\
\textbf{Data Bits} & Number of data bits (5-9) \\
\textbf{Parity} & Error checking (None, Even, Odd) \\
\textbf{Stop Bits} & End of frame marker (1 or 2) \\
\end{longtable}
}

\begin{verbatim}
sequenceDiagram
    participant TX as Transmitter
    participant RX as Receiver
    TX{-RX: Start Bit (0)}
    TX{-RX: Data Bits (8)}
    TX{-RX: Parity Bit (Optional)}
    TX{-RX: Stop Bit(s) (1)}
\end{verbatim}

\begin{itemize}
\tightlist
\item
  \textbf{Asynchronous}: No clock signal, uses start/stop bits
\item
  \textbf{RS232 Standard}: \pm12V levels, converted to TTL levels
\item
  \textbf{Common Baud Rates}: 9600, 19200, 38400, 115200
\end{itemize}

\end{solutionbox}
\begin{mnemonicbox}
``BDPS'' (Baud-rate, Data-bits, Parity-check,
Stop-bits)

\end{mnemonicbox}
\begin{center}\rule{0.5\linewidth}{0.5pt}\end{center}

\subsection*{Question 3(c) OR [7
marks]}\label{q3c}

\textbf{Write an AVR C program to read pins 1 and 0 of Port B and issue
an ASCII character to Port D according to the following table:}

\begin{solutionbox}

\begin{verbatim}
\#include {avr/io.h}

int main(void)
\{
    unsigned char input;
    
    // Configure PB1 and PB0 as input
    DDRB \&= {((}1{}1)|(1{}0));
    
    // Configure Port D as output
    DDRD = 0xFF;
    
    // Enable pull{-ups for PB1 and PB0}
    PORTB |= (1{}1)|(1{}0);
    
    while(1)
    \{
        // Read PB1 and PB0
        input = PINB \& 0x03;  // Mask other bits
        
        switch(input)
        \{
            case 0x00:  // Pin1=0, Pin0=0
                PORTD = {0};  // ASCII {0 = 0x30}
                break;
                
            case 0x01:  // Pin1=0, Pin0=1
                PORTD = {1};  // ASCII {1 = 0x31}
                break;
                
            case 0x02:  // Pin1=1, Pin0=0
                PORTD = {2};  // ASCII {2 = 0x32}
                break;
                
            case 0x03:  // Pin1=1, Pin0=1
                PORTD = {3};  // ASCII {3 = 0x33}
                break;
        \}
    \}
    
    return 0;
\}
\end{verbatim}

\textbf{Truth Table Implementation}:

{\def\LTcaptype{none} % do not increment counter
\begin{longtable}[]{@{}llll@{}}
\toprule\noalign{}
Pin1 & Pin0 & Input Value & ASCII Output \\
\midrule\noalign{}
\endhead
\bottomrule\noalign{}
\endlastfoot
0 & 0 & 0x00 & `0' (0x30) \\
0 & 1 & 0x01 & `1' (0x31) \\
1 & 0 & 0x02 & `2' (0x32) \\
1 & 1 & 0x03 & `3' (0x33) \\
\end{longtable}
}

\end{solutionbox}
\begin{mnemonicbox}
``MATS'' (Mask-inputs, ASCII-conversion, Truth-table,
Switch-case)

\end{mnemonicbox}
\begin{center}\rule{0.5\linewidth}{0.5pt}\end{center}

\subsection*{Question 4(a) [3 marks]}\label{q4a}

\textbf{Draw interfacing diagram of relay and relay driver ULN2803 with
ATmega32}

\begin{solutionbox}

\begin{verbatim}
ATmega32          ULN2803         Relay
                                 
PC0 {-{-}{-}{-}{-}{-}|1    18|{-}{-}{-}{-}{-}{-}{-}{-}{-}{-}{-} +12V}
PC1 {-{-}{-}{-}{-}{-}|2    17|    }
PC2 {-{-}{-}{-}{-}{-}|3    16|    }
PC3 {-{-}{-}{-}{-}{-}|4    15|    }
PC4 {-{-}{-}{-}{-}{-}|5    14|    }
PC5 {-{-}{-}{-}{-}{-}|6    13|    }
PC6 {-{-}{-}{-}{-}{-}|7    12|    }
PC7 {-{-}{-}{-}{-}{-}|8    11|    }
           |9    10|{-{-}{-}{-}{-}{-}{-}{-}{-}{-}{-} GND}
           ULN2803    
                       
    COM1 of Relay connected to +12V
    NO1 of Relay connected to Load
    GND common for all
\end{verbatim}

\textbf{Component Functions}:

\begin{itemize}
\tightlist
\item
  \textbf{ULN2803}: Darlington transistor array, current amplification
\item
  \textbf{Protection Diodes}: Built-in flyback diodes for inductive
  loads
\item
  \textbf{Relay Coil}: Requires 12V, controlled by ULN2803 output
\end{itemize}

\end{solutionbox}
\begin{mnemonicbox}
``UPC'' (ULN-driver, Port-control, Current-amplify)

\end{mnemonicbox}
\begin{center}\rule{0.5\linewidth}{0.5pt}\end{center}

\subsection*{Question 4(b) [4 marks]}\label{q4b}

\textbf{Write steps of programming the A/D converter using polling
method}

\begin{solutionbox}

\textbf{ADC Programming Steps}:

{\def\LTcaptype{none} % do not increment counter
\begin{longtable}[]{@{}ll@{}}
\toprule\noalign{}
Step & Action \\
\midrule\noalign{}
\endhead
\bottomrule\noalign{}
\endlastfoot
\textbf{1} & Configure ADMUX register (reference, channel) \\
\textbf{2} & Configure ADCSRA register (enable, prescaler) \\
\textbf{3} & Start conversion (set ADSC bit) \\
\textbf{4} & Wait for conversion complete (poll ADIF flag) \\
\textbf{5} & Read result from ADCL and ADCH \\
\end{longtable}
}

\textbf{Code Implementation}:

\begin{verbatim}
// Step 1: Configure ADMUX
ADMUX = (1{}REFS0);  // AVCC reference, channel 0

// Step 2: Enable ADC with prescaler
ADCSRA = (1{}ADEN)|(1{}ADPS2)|(1{}ADPS1)|(1{}ADPS0);

// Step 3: Start conversion
ADCSRA |= (1{}ADSC);

// Step 4: Wait for completion
while(!(ADCSRA \& (1{}ADIF)));

// Step 5: Read result
result = ADC;  // Combined ADCL and ADCH
\end{verbatim}

\end{solutionbox}
\begin{mnemonicbox}
``CCSWR'' (Configure-ADMUX, Configure-ADCSRA,
Start-conversion, Wait-complete, Read-result)

\end{mnemonicbox}
\begin{center}\rule{0.5\linewidth}{0.5pt}\end{center}

\subsection*{Question 4(c) [7 marks]}\label{q4c}

\textbf{Explain I2C-Two Wire Serial Interface (TWI) Protocol in detail.}

\begin{solutionbox}

\textbf{I2C Protocol Features}:

{\def\LTcaptype{none} % do not increment counter
\begin{longtable}[]{@{}ll@{}}
\toprule\noalign{}
Feature & Description \\
\midrule\noalign{}
\endhead
\bottomrule\noalign{}
\endlastfoot
\textbf{Two Wires} & SDA (Data) and SCL (Clock) \\
\textbf{Multi-master} & Multiple masters can control bus \\
\textbf{Addressing} & 7-bit or 10-bit device addresses \\
\textbf{Bidirectional} & Data flows both directions \\
\end{longtable}
}

\begin{verbatim}
sequenceDiagram
    participant M as Master
    participant S as Slave
    M{-S: Start Condition}
    M{-S: Slave Address + R/W}
    S{-M: ACK}
    M{-S: Data Byte}
    S{-M: ACK}
    M{-S: Stop Condition}
\end{verbatim}

\textbf{I2C Frame Structure}:

\begin{itemize}
\tightlist
\item
  \textbf{Start Condition}: SDA goes low while SCL is high
\item
  \textbf{Address Frame}: 7-bit address + R/W bit
\item
  \textbf{Data Frame}: 8-bit data + ACK/NACK
\item
  \textbf{Stop Condition}: SDA goes high while SCL is high
\end{itemize}

\textbf{TWI Registers in ATmega32}:

{\def\LTcaptype{none} % do not increment counter
\begin{longtable}[]{@{}ll@{}}
\toprule\noalign{}
Register & Function \\
\midrule\noalign{}
\endhead
\bottomrule\noalign{}
\endlastfoot
\textbf{TWCR} & Control and status \\
\textbf{TWDR} & Data register \\
\textbf{TWAR} & Address register \\
\textbf{TWSR} & Status register \\
\end{longtable}
}

\begin{itemize}
\tightlist
\item
  \textbf{Clock Stretching}: Slave can hold SCL low to slow down master
\item
  \textbf{Arbitration}: Prevents collisions in multi-master systems
\item
  \textbf{Pull-up Resistors}: Required on both SDA and SCL lines (4.7kΩ
  typical)
\end{itemize}

\end{solutionbox}
\begin{mnemonicbox}
``SAD-CSA'' (Start-Address-Data,
Control-Status-Address)

\end{mnemonicbox}
\begin{center}\rule{0.5\linewidth}{0.5pt}\end{center}

\subsection*{Question 4(a) OR [3
marks]}\label{q4a}

\textbf{Explain any one PWM mode for controlling speed of DC motor by
using 8-bit timer}

\begin{solutionbox}

\textbf{Fast PWM Mode (Mode 3)}:

{\def\LTcaptype{none} % do not increment counter
\begin{longtable}[]{@{}ll@{}}
\toprule\noalign{}
Parameter & Value \\
\midrule\noalign{}
\endhead
\bottomrule\noalign{}
\endlastfoot
\textbf{WGM bits} & WGM01=1, WGM00=1 \\
\textbf{TOP value} & 0xFF (255) \\
\textbf{Resolution} & 8-bit \\
\textbf{Frequency} & fclk/(256\timesprescaler) \\
\end{longtable}
}

\textbf{PWM Configuration}:

\begin{verbatim}
// Configure Timer0 for Fast PWM
TCCR0 = (1{}WGM01)|(1{}WGM00)|(1{}COM01)|(1{}CS01);

// Set duty cycle (0{-255)}
OCR0 = 128;  // 50\% duty cycle
\end{verbatim}

\begin{center}
\textbf{Mermaid Diagram (Code)}
\begin{verbatim}
{Shaded}
{Highlighting}[]
graph LR
    A[Timer0] {-{-}{} B[PWM Signal]}
    B {-{-}{} C[Motor Driver]}
    C {-{-}{} D[DC Motor]}
    E[OCR0 Value] {-{-}{} A}
{Highlighting}
{Shaded}
\end{verbatim}
\end{center}

\begin{itemize}
\tightlist
\item
  \textbf{Duty Cycle Control}: OCR0 value determines motor speed
\item
  \textbf{Non-inverting Mode}: High pulse width = OCR0/255
\item
  \textbf{Motor Control}: Higher duty cycle = higher speed
\end{itemize}

\end{solutionbox}
\begin{mnemonicbox}
``FTO'' (Fast-PWM, Timer0, OCR0-control)

\end{mnemonicbox}
\begin{center}\rule{0.5\linewidth}{0.5pt}\end{center}

\subsection*{Question 4(b) OR [4
marks]}\label{q4b}

\textbf{Write steps for reading data from an SPI device}

\begin{solutionbox}

\textbf{SPI Read Steps}:

{\def\LTcaptype{none} % do not increment counter
\begin{longtable}[]{@{}ll@{}}
\toprule\noalign{}
Step & Action \\
\midrule\noalign{}
\endhead
\bottomrule\noalign{}
\endlastfoot
\textbf{1} & Configure SPI control register (SPCR) \\
\textbf{2} & Set SS pin low to select slave \\
\textbf{3} & Write dummy data to SPDR \\
\textbf{4} & Wait for transmission complete (SPIF flag) \\
\textbf{5} & Read received data from SPDR \\
\textbf{6} & Set SS pin high to deselect slave \\
\end{longtable}
}

\textbf{Code Implementation}:

\begin{verbatim}
// Step 1: Configure SPI as master
SPCR = (1{}SPE)|(1{}MSTR)|(1{}SPR0);

// Step 2: Select slave
PORTB \&= {(}1{}SS);

// Step 3: Send dummy byte
SPDR = 0xFF;

// Step 4: Wait for complete
while(!(SPSR \& (1{}SPIF)));

// Step 5: Read data
data = SPDR;

// Step 6: Deselect slave
PORTB |= (1{}SS);
\end{verbatim}

\textbf{SPI Timing}:

\begin{itemize}
\tightlist
\item
  \textbf{Clock Polarity}: CPOL bit determines idle state
\item
  \textbf{Clock Phase}: CPHA bit determines sampling edge
\item
  \textbf{Data Order}: MSB first (default) or LSB first
\end{itemize}

\end{solutionbox}
\begin{mnemonicbox}
``CSWWRD'' (Configure, Select, Write-dummy, Wait,
Read-data, Deselect)

\end{mnemonicbox}
\begin{center}\rule{0.5\linewidth}{0.5pt}\end{center}

\subsection*{Question 4(c) OR [7
marks]}\label{q4c}

\textbf{Draw and explain interfacing diagram of LM35 with ATmega32.}

\begin{solutionbox}

\begin{verbatim}
    LM35 Temperature Sensor
    
    +5V {-{-}{-}{-}{-} VCC (Pin 1)}
               |
    ATmega32   |    LM35
    PA0 {{-}{-}{-}{-}{-} OUTPUT (Pin 2)}
               |
    GND {-{-}{-}{-}{-} GND (Pin 3)}
    
    Optional: 0.1µF capacitor between 
    VCC and GND for noise filtering
\end{verbatim}

\textbf{LM35 Specifications}:

{\def\LTcaptype{none} % do not increment counter
\begin{longtable}[]{@{}ll@{}}
\toprule\noalign{}
Parameter & Value \\
\midrule\noalign{}
\endhead
\bottomrule\noalign{}
\endlastfoot
\textbf{Output} & 10mV/^\circC \\
\textbf{Range} & 0^\circC to 100^\circC \\
\textbf{Supply} & 4V to 30V \\
\textbf{Accuracy} & \pm0.5^\circC \\
\end{longtable}
}

\textbf{ADC Code for Temperature Reading}:

\begin{verbatim}
\#include {avr/io.h}

unsigned int readTemperature(void)
\{
    unsigned int adcValue, temperature;
    
    // Configure ADC
    ADMUX = (1{}REFS0);  // AVCC reference, PA0
    ADCSRA = (1{}ADEN)|(1{}ADPS2)|(1{}ADPS1)|(1{}ADPS0);
    
    // Start conversion
    ADCSRA |= (1{}ADSC);
    
    // Wait for completion
    while(!(ADCSRA \& (1{}ADIF)));
    
    // Read ADC value
    adcValue = ADC;
    
    // Convert to temperature
    // ADC = (Vin  1024) / Vref
    // Vin = (10mV/^)  Temp
    temperature = (adcValue * 500) / 1024;
    
    return temperature;
\}
\end{verbatim}

\textbf{Temperature Calculation}:

\begin{itemize}
\tightlist
\item
  \textbf{ADC Resolution}: 10-bit (0-1023)
\item
  \textbf{Reference Voltage}: 5V
\item
  \textbf{LM35 Output}: 10mV/^\circC
\item
  \textbf{Formula}: Temp = (ADC \times 5000mV) / (1024 \times 10mV/^\circC)
\end{itemize}

\end{solutionbox}
\begin{mnemonicbox}
``VARC'' (Voltage-output, ADC-conversion,
Reference-5V, Calculation-formula)

\end{mnemonicbox}
\begin{center}\rule{0.5\linewidth}{0.5pt}\end{center}

\subsection*{Question 5(a) [3 marks]}\label{q5a}

\textbf{Draw Timer 0 Working Block diagram.}

\begin{solutionbox}

\begin{center}
\textbf{Mermaid Diagram (Code)}
\begin{verbatim}
{Shaded}
{Highlighting}[]
graph LR
    A[System Clock] {-{-}{} B[Prescaler]}
    B {-{-}{} C[Timer/Counter 0]}
    C {-{-}{} D[Compare Unit]}
    C {-{-}{} E[Overflow Flag]}
    D {-{-}{} F[OCR0]}
    D {-{-}{} G[PWM Output]}
    H[External Clock] {-{-}{} B}

    style C fill:\#f9f,stroke:\#333,stroke{-width:4px}
{Highlighting}
{Shaded}
\end{verbatim}
\end{center}

\textbf{Timer0 Components}:

{\def\LTcaptype{none} % do not increment counter
\begin{longtable}[]{@{}ll@{}}
\toprule\noalign{}
Component & Function \\
\midrule\noalign{}
\endhead
\bottomrule\noalign{}
\endlastfoot
\textbf{Prescaler} & Clock division (1,8,64,256,1024) \\
\textbf{Counter} & 8-bit up counter (0-255) \\
\textbf{Compare Unit} & Compares counter with OCR0 \\
\textbf{Overflow} & Sets flag when counter overflows \\
\end{longtable}
}

\begin{itemize}
\tightlist
\item
  \textbf{Clock Sources}: Internal clock or external pin
\item
  \textbf{Modes}: Normal, CTC, Fast PWM, Phase Correct PWM
\item
  \textbf{Interrupt}: Timer overflow and compare match
\end{itemize}

\end{solutionbox}
\begin{mnemonicbox}
``PCCO'' (Prescaler, Counter, Compare, Overflow)

\end{mnemonicbox}
\begin{center}\rule{0.5\linewidth}{0.5pt}\end{center}

\subsection*{Question 5(b) [4 marks]}\label{q5b}

\textbf{Draw Interfacing of MAX7221 to ATmega32.}

\begin{solutionbox}

\begin{verbatim}
ATmega32                    MAX7221
                           
PB5(MOSI) {-{-}{-}{-}{-}{-}{-}{-}{-}{-}{-} DIN (Pin 1)}
PB7(SCK)  {-{-}{-}{-}{-}{-}{-}{-}{-}{-}{-} CLK (Pin 13)}
PB4(SS)   {-{-}{-}{-}{-}{-}{-}{-}{-}{-}{-} CS  (Pin 12)}
                       
                       V+ (Pin 19) {{-}{-}{-} +5V}
                       GND(Pin 4,9) {{-}{-}{-} GND}
                       
         7{-Segment Display Connections:}
         SEG A{-G, DP connected to Pins 14{-}17, 20{-}23}
         DIG 0{-7 connected to Pins 2{-}3, 5{-}8, 10{-}11}
\end{verbatim}

\textbf{MAX7221 Features}:

{\def\LTcaptype{none} % do not increment counter
\begin{longtable}[]{@{}ll@{}}
\toprule\noalign{}
Feature & Description \\
\midrule\noalign{}
\endhead
\bottomrule\noalign{}
\endlastfoot
\textbf{Display Driver} & 8-digit 7-segment LED driver \\
\textbf{SPI Interface} & Serial data input \\
\textbf{Current Control} & Adjustable segment current \\
\textbf{Shutdown Mode} & Power saving feature \\
\end{longtable}
}

\textbf{Initialization Code}:

\begin{verbatim}
void MAX7221\_init(void)
\{
    // Configure SPI pins
    DDRB |= (1{}PB5)|(1{}PB7)|(1{}PB4);  // MOSI, SCK, SS as output
    
    // Initialize SPI
    SPCR = (1{}SPE)|(1{}MSTR)|(1{}SPR0);
    
    // Wake up MAX7221
    MAX7221\_write(0x0C, 0x01);  // Shutdown register
    
    // Set decode mode
    MAX7221\_write(0x09, 0xFF);  // BCD decode for all digits
    
    // Set intensity
    MAX7221\_write(0x0A, 0x08);  // Medium brightness
    
    // Set scan limit
    MAX7221\_write(0x0B, 0x07);  // Display all 8 digits
\}
\end{verbatim}

\end{solutionbox}
\begin{mnemonicbox}
``SCD-ISS'' (SPI-interface, Current-control,
Decode-mode, Initialize-setup, Scan-limit)

\end{mnemonicbox}
\begin{center}\rule{0.5\linewidth}{0.5pt}\end{center}

\subsection*{Question 5(c) [7 marks]}\label{q5c}

\textbf{Explain Weather Monitoring System.}

\begin{solutionbox}

\textbf{System Block Diagram}:

\begin{center}
\textbf{Mermaid Diagram (Code)}
\begin{verbatim}
{Shaded}
{Highlighting}[]
graph TD
    A[Temperature Sensor{br/{}LM35] {-}{-}{} E[ATmega32{}br/{}Microcontroller]}
    B[Humidity Sensor{br/{}DHT11] {-}{-}{} E}
    C[Pressure Sensor{br/{}BMP180] {-}{-}{} E}
    D[Light Sensor{br/{}LDR] {-}{-}{} E}
    E {-{-}{} F[LCD Display{}br/{}16x2]}
    E {-{-}{} G[Data Logger{}br/{}EEPROM]}
    E {-{-}{} H[Wireless Module{}br/{}ESP8266]}
    H {-{-}{} I[Cloud Server]}
    J[Power Supply{br/{}Battery/Solar] {-}{-}{} E}
{Highlighting}
{Shaded}
\end{verbatim}
\end{center}

\textbf{System Components}:

{\def\LTcaptype{none} % do not increment counter
\begin{longtable}[]{@{}lll@{}}
\toprule\noalign{}
Component & Function & Interface \\
\midrule\noalign{}
\endhead
\bottomrule\noalign{}
\endlastfoot
\textbf{LM35} & Temperature measurement & ADC \\
\textbf{DHT11} & Humidity \& temperature & Digital I/O \\
\textbf{BMP180} & Atmospheric pressure & I2C \\
\textbf{LCD} & Local display & Parallel \\
\textbf{ESP8266} & WiFi connectivity & UART \\
\textbf{EEPROM} & Data storage & I2C \\
\end{longtable}
}

\textbf{Features and Applications}:

\begin{itemize}
\tightlist
\item
  \textbf{Real-time Monitoring}: Continuous sensor data collection
\item
  \textbf{Data Logging}: Historical data storage in EEPROM
\item
  \textbf{Remote Access}: WiFi connectivity for cloud upload
\item
  \textbf{Power Management}: Battery backup with solar charging
\item
  \textbf{Alert System}: Threshold-based warnings
\item
  \textbf{Agricultural Use}: Crop monitoring, irrigation control
\item
  \textbf{Home Automation}: HVAC control, energy management
\end{itemize}

\textbf{Software Functions}:

\begin{itemize}
\tightlist
\item
  \textbf{Sensor Reading}: ADC conversion, I2C communication
\item
  \textbf{Data Processing}: Calibration, filtering, averaging
\item
  \textbf{Display Update}: LCD formatting, user interface
\item
  \textbf{Communication}: WiFi data transmission, protocol handling
\item
  \textbf{Storage Management}: EEPROM read/write, data compression
\end{itemize}

\end{solutionbox}
\begin{mnemonicbox}
``SMART-W'' (Sensors, Monitoring, Alert, Remote,
Temperature, Weather)

\end{mnemonicbox}
\begin{center}\rule{0.5\linewidth}{0.5pt}\end{center}

\subsection*{Question 5(a) OR [3
marks]}\label{q5a}

\textbf{Draw and explain Timer/Counter Control Register 0(TCCR0)}

\begin{solutionbox}

\textbf{TCCR0 Register Bit Structure}:

\begin{verbatim}
Bit:    7     6     5     4     3     2     1    0
      +{-{-}{-}{-}+{-}{-}{-}{-}{-}+{-}{-}{-}{-}{-}+{-}{-}{-}{-}{-}+{-}{-}{-}{-}{-}+{-}{-}{-}{-}{-}+{-}{-}{-}{-}{-}+{-}{-}{-}{-}{-}+}
TCCR0 |FOC0|WGM00|COM01|COM00|WGM01| CS02| CS01| CS00|
      +{-{-}{-}{-}+{-}{-}{-}{-}{-}+{-}{-}{-}{-}{-}+{-}{-}{-}{-}{-}+{-}{-}{-}{-}{-}+{-}{-}{-}{-}{-}+{-}{-}{-}{-}{-}+{-}{-}{-}{-}{-}+}
\end{verbatim}

\textbf{Bit Functions Table}:

{\def\LTcaptype{none} % do not increment counter
\begin{longtable}[]{@{}lll@{}}
\toprule\noalign{}
Bit & Name & Function \\
\midrule\noalign{}
\endhead
\bottomrule\noalign{}
\endlastfoot
\textbf{FOC0} & Force Output Compare & Force compare match \\
\textbf{WGM01:00} & Waveform Generation & Timer mode selection \\
\textbf{COM01:00} & Compare Output Mode & Output pin behavior \\
\textbf{CS02:00} & Clock Select & Prescaler selection \\
\end{longtable}
}

\textbf{Clock Select Options}:

{\def\LTcaptype{none} % do not increment counter
\begin{longtable}[]{@{}ll@{}}
\toprule\noalign{}
CS02:00 & Clock Source \\
\midrule\noalign{}
\endhead
\bottomrule\noalign{}
\endlastfoot
000 & No clock (stopped) \\
001 & clk/1 (no prescaling) \\
010 & clk/8 \\
011 & clk/64 \\
100 & clk/256 \\
101 & clk/1024 \\
110 & External clock on T0 (falling) \\
111 & External clock on T0 (rising) \\
\end{longtable}
}

\textbf{Waveform Generation Modes}:

{\def\LTcaptype{none} % do not increment counter
\begin{longtable}[]{@{}lll@{}}
\toprule\noalign{}
WGM01:00 & Mode & Description \\
\midrule\noalign{}
\endhead
\bottomrule\noalign{}
\endlastfoot
00 & Normal & Count up to 0xFF \\
01 & PWM, Phase Correct & Count up/down \\
10 & CTC & Clear Timer on Compare \\
11 & Fast PWM & Count up to 0xFF \\
\end{longtable}
}

\end{solutionbox}
\begin{mnemonicbox}
``FWC-CS'' (Force, Waveform, Compare, Clock-Select)

\end{mnemonicbox}
\begin{center}\rule{0.5\linewidth}{0.5pt}\end{center}

\subsection*{Question 5(b) OR [4
marks]}\label{q5b}

\textbf{Explain the function of motor driver L293D.}

\begin{solutionbox}

\textbf{L293D Motor Driver Features}:

{\def\LTcaptype{none} % do not increment counter
\begin{longtable}[]{@{}ll@{}}
\toprule\noalign{}
Feature & Specification \\
\midrule\noalign{}
\endhead
\bottomrule\noalign{}
\endlastfoot
\textbf{Channels} & Dual H-bridge, 2 motors \\
\textbf{Supply Voltage} & 4.5V to 36V \\
\textbf{Output Current} & 600mA per channel \\
\textbf{Logic Voltage} & 5V TTL compatible \\
\textbf{Protection} & Thermal shutdown \\
\end{longtable}
}

\textbf{Pin Configuration}:

\begin{verbatim}
        L293D
    +{-{-}{-}{-}{-}{-}{-}{-}{-}+}
EN1 |1      16| VCC1 (+5V)
IN1 |2      15| IN4
OUT1|3      14| OUT4
GND |4      13| GND
GND |5      12| GND
OUT2|6      11| OUT3
IN2 |7      10| IN3
VCC2|8       9| EN2
    +{-{-}{-}{-}{-}{-}{-}{-}{-}+}
\end{verbatim}

\textbf{H-Bridge Operation}:

{\def\LTcaptype{none} % do not increment counter
\begin{longtable}[]{@{}lll@{}}
\toprule\noalign{}
IN1 & IN2 & Motor Action \\
\midrule\noalign{}
\endhead
\bottomrule\noalign{}
\endlastfoot
0 & 0 & Stop (brake) \\
0 & 1 & Rotate CCW \\
1 & 0 & Rotate CW \\
1 & 1 & Stop (brake) \\
\end{longtable}
}

\textbf{Control Functions}:

\begin{itemize}
\tightlist
\item
  \textbf{Direction Control}: IN1, IN2 determine rotation direction
\item
  \textbf{Speed Control}: PWM on Enable pins (EN1, EN2)
\item
  \textbf{Dual Supply}: VCC1 for logic, VCC2 for motor power
\item
  \textbf{Enable Control}: EN pins enable/disable motor operation
\end{itemize}

\textbf{Applications}:

\begin{itemize}
\tightlist
\item
  \textbf{Robotics}: Differential drive robots
\item
  \textbf{Automation}: Conveyor belt control
\item
  \textbf{RC Vehicles}: Motor speed and direction control
\end{itemize}

\end{solutionbox}
\begin{mnemonicbox}
``DHIE'' (Dual-channel, H-bridge, Input-control,
Enable-PWM)

\end{mnemonicbox}
\begin{center}\rule{0.5\linewidth}{0.5pt}\end{center}

\subsection*{Question 5(c) OR [7
marks]}\label{q5c}

\textbf{Explain Automatic Juice vending machine.}

\begin{solutionbox}

\textbf{System Block Diagram}:

\begin{center}
\textbf{Mermaid Diagram (Code)}
\begin{verbatim}
{Shaded}
{Highlighting}[]
graph TD
    A[Keypad Input] {-{-}{} H[ATmega32{}br/{}Controller]}
    B[Coin Sensor] {-{-}{} H}
    C[LCD Display] {-{-}{} H}
    H {-{-}{} D[Pump Motors]}
    H {-{-}{} E[Solenoid Valves]}
    H {-{-}{} F[Coin Return{}br/{}Mechanism]}
    H {-{-}{} G[Level Sensors]}
    I[Power Supply] {-{-}{} H}
    J[Juice Containers] {-{-}{} D}
    D {-{-}{} K[Mixing Chamber]}
    E {-{-}{} K}
    K {-{-}{} L[Dispensing Unit]}
{Highlighting}
{Shaded}
\end{verbatim}
\end{center}

\textbf{System Components}:

{\def\LTcaptype{none} % do not increment counter
\begin{longtable}[]{@{}lll@{}}
\toprule\noalign{}
Component & Function & Interface \\
\midrule\noalign{}
\endhead
\bottomrule\noalign{}
\endlastfoot
\textbf{Keypad} & Juice selection & Digital I/O \\
\textbf{Coin Sensor} & Payment detection & Interrupt \\
\textbf{LCD Display} & User interface & Parallel \\
\textbf{Pump Motors} & Juice pumping & PWM control \\
\textbf{Solenoid Valves} & Flow control & Digital output \\
\textbf{Level Sensors} & Container monitoring & ADC/Digital \\
\end{longtable}
}

\textbf{Operation Sequence}:

\begin{enumerate}
\tightlist
\item
  \textbf{Display Menu}: Show available juices and prices
\item
  \textbf{User Selection}: Customer selects juice type via keypad
\item
  \textbf{Payment Process}: Coin insertion and validation
\item
  \textbf{Level Check}: Verify ingredient availability
\item
  \textbf{Dispensing}: Activate pumps and valves in sequence
\item
  \textbf{Mixing}: Control mixing ratios and time
\item
  \textbf{Completion}: Display completion message and return change
\end{enumerate}

\textbf{Control Algorithm}:

\begin{verbatim}
void dispensJuice(uint8\_t selection, uint16\_t amount)
\{
    // Check ingredient levels
    if(checkLevels(selection))
    \{
        // Calculate mixing ratios
        calculateRatio(selection);
        
        // Start dispensing sequence
        activatePump(selection, amount);
        
        // Control mixing time
        startTimer(MIXING\_TIME);
        
        // Complete transaction
        displayMessage("Enjoy your juice!");
    \}
    else
    \{
        displayMessage("Ingredient not available");
        returnCoins();
    \}
\}
\end{verbatim}

\textbf{Features}:

\begin{itemize}
\tightlist
\item
  \textbf{Multiple Flavors}: Different juice combinations
\item
  \textbf{Payment System}: Coin acceptance and change return
\item
  \textbf{Inventory Management}: Level monitoring and alerts
\item
  \textbf{User Interface}: Menu display and selection
\item
  \textbf{Safety Features}: Overflow protection, emergency stop
\item
  \textbf{Maintenance Mode}: Service and cleaning cycles
\end{itemize}

\textbf{Applications}:

\begin{itemize}
\tightlist
\item
  \textbf{Commercial}: Shopping malls, offices, schools
\item
  \textbf{Industrial}: Factory cafeterias, hospitals
\item
  \textbf{Public Places}: Airports, train stations
\end{itemize}

\end{solutionbox}
\begin{mnemonicbox}
``JUMPS'' (Juice-selection, User-interface,
Mixing-control, Payment-system, Sensors-monitoring)

\end{mnemonicbox}

\end{document}
