\documentclass[10pt,a4paper]{article}

% content/resources/templates/preamble.tex
\usepackage[margin=0.6in]{geometry}
\author{Milav Dabgar}
\usepackage{amsmath,amssymb,amsthm}
\usepackage{booktabs}
\usepackage{multirow}
\usepackage{xcolor}
\usepackage{tcolorbox}
\tcbuselibrary{breakable,skins}
\usepackage[colorlinks=true,linkcolor=blue]{hyperref}
\usepackage{titlesec}
\usepackage{enumitem}
\usepackage{tikz}
\usepackage{pgfplots}
\usepackage{circuitikz}
\usepackage[version=4]{mhchem}
\usepackage{longtable}
\usepackage{array}
\usepackage{float}
\usepackage{caption}
\usepackage{listings}

\lstset{
  basicstyle=\small\ttfamily,
  breaklines=true,
  breakatwhitespace=false,
  postbreak=\mbox{\textcolor{red}{$\hookrightarrow$}\space},
  float=false,
  numbers=left,
  numberstyle=\tiny\color{gray},
  numbersep=10pt,
  xleftmargin=2em,
  keywordstyle=\color{blue},
  commentstyle=\color{green!60!black},
  stringstyle=\color{purple},
  backgroundcolor=\color{gray!5},
  showstringspaces=false,
  tabsize=2,
  captionpos=b,
  keepspaces=true,
  columns=flexible
}

\pgfplotsset{compat=1.18}
\usetikzlibrary{shapes,arrows,positioning,calc,patterns,decorations.pathmorphing,decorations.markings,arrows.meta}

% Color scheme
\definecolor{headcolor}{RGB}{0,102,204}
\definecolor{keycolor}{RGB}{220,20,60}
\definecolor{solutioncolor}{RGB}{34,139,34}
\definecolor{mnemoniccolor}{RGB}{148,0,211}
\definecolor{codecolor}{RGB}{0,0,100}

% Spacing
\setlength{\parskip}{3pt}
\setlist[itemize]{nosep}
\setlist[enumerate]{nosep}

% Title formatting
\titleformat{\section}{\Large\bfseries\color{headcolor}}{\thesection}{1em}{}
\titleformat{\subsection}{\large\bfseries\color{headcolor}}{\thesubsection}{1em}{}

% Pandoc tightlist compatibility
\providecommand{\tightlist}{%
  \setlength{\itemsep}{0pt}\setlength{\parskip}{0pt}}

% Pandoc longtable compatibility
\newcounter{none}
\def\thenone{}


% content/resources/templates/gujarati-boxes.tex
\usepackage{fontspec}
\usepackage{polyglossia}

% Set Gujarati as main language (document is primarily in Gujarati)
% Note: gloss-gujarati.ldf doesn't exist in polyglossia, but it will use hyphenation patterns
\setdefaultlanguage{gujarati}
\setotherlanguage{english}

% Configure Gujarati font properly
% Use Language=Default to prevent polyglossia from trying to add language-specific features
% that don't exist for Gujarati, which causes "empty feature" warnings
\newfontfamily\gujaratifont[Script=Gujarati,AutoFakeBold=2.5,AutoFakeSlant=0.3]{Noto Sans Gujarati}
\setmainfont[Script=Gujarati,AutoFakeBold=2.5,AutoFakeSlant=0.3]{Noto Sans Gujarati}
% Use Noto Sans Gujarati for monospace to support Gujarati in text
\setmonofont[Scale=0.9]{Noto Sans Gujarati}

% Configure English to use the same font
\newfontfamily\englishfont[Script=Gujarati,AutoFakeBold=2.5,AutoFakeSlant=0.3]{Noto Sans Gujarati}

% Translations for polyglossia
\gappto\captionsgujarati{
  \renewcommand{\tablename}{કોષ્ટક}
  \renewcommand{\figurename}{આકૃતિ}
}

% Helper for TikZ nodes to ensure Gujarati font
\newcommand{\gu}[1]{{\gujaratifont #1}}

% Custom environments
\newtcolorbox{solutionbox}{
    breakable,
    enhanced,
    colback=solutioncolor!5!white,
    colframe=solutioncolor!75!black,
    fonttitle=\bfseries,
    title=જવાબ
}

\newtcolorbox{solutionboxnobreak}{
 colback=solutioncolor!5!white,
 colframe=solutioncolor!75!black,
 fonttitle=\bfseries,
 title=જવાબ
}

\newtcolorbox{keyformula}{
 breakable,
 enhanced,
 colback=keycolor!5!white,
 colframe=keycolor!75!black,
 fonttitle=\bfseries,
 title=રાસાયણિક સમીકરણ/સૂત્ર
}

\newtcolorbox{mnemonicbox}{
 breakable,
 enhanced,
 colback=mnemoniccolor!5!white,
 colframe=mnemoniccolor!75!black,
 fonttitle=\bfseries,
 title=મેમરી ટ્રીક
}


\begin{document}

\begin{center}
{\Huge\bfseries\color{headcolor} Subject Name (Gujarati)}\\[5pt]
{\LARGE 4351102 -- Winter 2023}\\[3pt]
{\large Semester 1 Study Material}\\[3pt]
{\normalsize\textit{Detailed Solutions and Explanations}}
\end{center}

\vspace{10pt}

\subsection*{પ્રશ્ન 1(અ) [3
ગુણ]}\label{uxaaauxab0uxab6uxaa8-1uxa85-3-uxa97uxaa3}

\textbf{TIFR register દોરો અને તેનું પૂરું નામ લખો.}

\begin{solutionbox}

\textbf{TIFR Register ડાયાગ્રામ:}

\begin{verbatim}
+{-{-}{-}{-}+{-}{-}{-}{-}+{-}{-}{-}{-}+{-}{-}{-}{-}{-}+{-}{-}{-}{-}{-}+{-}{-}{-}{-}+{-}{-}{-}{-}+{-}{-}{-}{-}+}
| 7  | 6  |  5 |  4  |  3  | 2  | 1  | 0  |
+{-{-}{-}{-}+{-}{-}{-}{-}+{-}{-}{-}{-}+{-}{-}{-}{-}{-}+{-}{-}{-}{-}{-}+{-}{-}{-}{-}+{-}{-}{-}{-}+{-}{-}{-}{-}+}
|OCF2|TOV2|ICF1|OCF1A|OCF1B|TOV1|OCF0|TOV0|
+{-{-}{-}{-}+{-}{-}{-}{-}+{-}{-}{-}{-}+{-}{-}{-}{-}{-}+{-}{-}{-}{-}{-}+{-}{-}{-}{-}+{-}{-}{-}{-}+{-}{-}{-}{-}+}
\end{verbatim}

\textbf{પૂરું નામ}: Timer/Counter Interrupt Flag Register

\begin{itemize}
\tightlist
\item
  \textbf{TOV0}: Timer0 Overflow Flag
\item
  \textbf{OCF0}: Timer0 Output Compare Flag\\
\item
  \textbf{TOV1}: Timer1 Overflow Flag
\end{itemize}

\end{solutionbox}
\begin{mnemonicbox}
``Timer Interrupts Flag Register''

\end{mnemonicbox}
\subsection*{પ્રશ્ન 1(બ) [4
ગુણ]}\label{uxaaauxab0uxab6uxaa8-1uxaac-4-uxa97uxaa3}

\textbf{ATmega32 ની ડેટા મેમરીની ચર્ચા કરો.}

\begin{solutionbox}

{\def\LTcaptype{none} % do not increment counter
\begin{longtable}[]{@{}llll@{}}
\toprule\noalign{}
મેમરી પ્રકાર & કદ & Address Range & હેતુ \\
\midrule\noalign{}
\endhead
\bottomrule\noalign{}
\endlastfoot
General Purpose Registers & 32 bytes & 0x00-0x1F & R0-R31 registers \\
I/O Memory & 64 bytes & 0x20-0x5F & Control registers \\
Internal SRAM & 2048 bytes & 0x60-0x85F & Variable storage \\
\end{longtable}
}

\begin{itemize}
\tightlist
\item
  \textbf{General Purpose Registers}: અંકગણિત કામગીરી અને અસ્થાયી સંગ્રહ માટે
  વપરાય છે
\item
  \textbf{I/O Memory}: પેરિફેરલ કંટ્રોલ અને સ્ટેટસ રજિસ્ટર્સ ધરાવે છે
\item
  \textbf{Internal SRAM}: સ્ટેક, વેરિયેબલ્સ અને ડાયનેમિક મેમરી માટે વપરાય છે
\end{itemize}

\end{solutionbox}
\begin{mnemonicbox}
``General I/O SRAM Memory''

\end{mnemonicbox}
\subsection*{પ્રશ્ન 1(ક) [7
ગુણ]}\label{uxaaauxab0uxab6uxaa8-1uxa95-7-uxa97uxaa3}

\textbf{એમ્બેડેડ સિસ્ટમનો જનરલ બ્લોક ડાયાગ્રામ દોરી સમજાવો.}

\begin{solutionbox}

\begin{center}
\textbf{Mermaid Diagram (Code)}
\begin{verbatim}
{Shaded}
{Highlighting}[]
graph TD
    A[Input Devices] {-{-}{} B[Processor/Microcontroller]}
    B {-{-}{} C[Memory]}
    B {-{-}{} D[Output Devices]}
    B {-{-}{} E[Communication Interface]}
    F[Power Supply] {-{-}{} B}
    G[Clock Circuit] {-{-}{} B}
{Highlighting}
{Shaded}
\end{verbatim}
\end{center}

{\def\LTcaptype{none} % do not increment counter
\begin{longtable}[]{@{}ll@{}}
\toprule\noalign{}
ઘટક & કાર્ય \\
\midrule\noalign{}
\endhead
\bottomrule\noalign{}
\endlastfoot
Processor & સમગ્ર સિસ્ટમ ઓપરેશન કંટ્રોલ કરે છે \\
Memory & પ્રોગ્રામ અને ડેટા સ્ટોર કરે છે \\
Input Devices & સેન્સર, સ્વિચ, કીબોર્ડ \\
Output Devices & LEDs, ડિસ્પ્લે, મોટર \\
Communication & UART, SPI, I2C ઇન્ટરફેસ \\
\end{longtable}
}

\begin{itemize}
\tightlist
\item
  \textbf{Real-time Operation}: સિસ્ટમ નિર્ધારિત સમય મર્યાદામાં ઇનપુટ્સને
  પ્રતિસાદ આપે છે
\item
  \textbf{Dedicated Function}: ચોક્કસ એપ્લિકેશન માટે ડિઝાઇન કરવામાં આવે છે
\item
  \textbf{Resource Constraints}: મર્યાદિત મેમરી, પાવર અને પ્રોસેસિંગ ક્ષમતા
\end{itemize}

\end{solutionbox}
\begin{mnemonicbox}
``Processor Memory Input Output Communication''

\end{mnemonicbox}
\subsection*{પ્રશ્ન 1(ક OR) [7
ગુણ]}\label{uxaaauxab0uxab6uxaa8-1uxa95-or-7-uxa97uxaa3}

\textbf{રીયલ ટાઇમ ઓપરેટિંગ સિસ્ટમને વ્યાખ્યાયિત કરો અને તેની લાક્ષણિકતાઓ સમજાવો.}

\begin{solutionbox}

\textbf{વ્યાખ્યા}: Real Time Operating System (RTOS) એ એવી ઓપરેટિંગ સિસ્ટમ છે
જે મહત્વપૂર્ણ કાર્યો માટે નિર્દિષ્ટ સમય મર્યાદામાં પ્રતિસાદની ગેરેંટી આપે છે.

{\def\LTcaptype{none} % do not increment counter
\begin{longtable}[]{@{}ll@{}}
\toprule\noalign{}
લાક્ષણિકતા & વર્ણન \\
\midrule\noalign{}
\endhead
\bottomrule\noalign{}
\endlastfoot
Deterministic & અનુમાનિત પ્રતિસાદ સમય \\
Multitasking & બહુવિધ કાર્યોનું અમલીકરણ \\
Priority-based & ઉચ્ચ પ્રાથમિકતા કાર્યો પહેલા \\
Minimal Latency & ઝડપી ઇન્ટરપ્ટ પ્રતિસાદ \\
\end{longtable}
}

\begin{itemize}
\tightlist
\item
  \textbf{Hard Real-time}: ડેડલાઇન ચૂકવાથી સિસ્ટમ નિષ્ફળતા થાય છે
\item
  \textbf{Soft Real-time}: ડેડલાઇન ચૂકવાથી પ્રદર્શન ઘટે છે
\item
  \textbf{Task Scheduling}: Preemptive priority-based scheduling મહત્વપૂર્ણ
  કાર્યો પહેલા ચલાવવાની ખાતરી કરે છે
\end{itemize}

\end{solutionbox}
\begin{mnemonicbox}
``Deterministic Multitasking Priority Minimal''

\end{mnemonicbox}
\subsection*{પ્રશ્ન 2(અ) [3
ગુણ]}\label{uxaaauxab0uxab6uxaa8-2uxa85-3-uxa97uxaa3}

\textbf{એમ્બેડેડ સિસ્ટમ માટે માઇક્રોકન્ટ્રોલર પસંદ કરવા માટેના માપદંડો લખો.}

\begin{solutionbox}

{\def\LTcaptype{none} % do not increment counter
\begin{longtable}[]{@{}ll@{}}
\toprule\noalign{}
માપદંડ & મહત્વ \\
\midrule\noalign{}
\endhead
\bottomrule\noalign{}
\endlastfoot
Processing Speed & એપ્લિકેશન જરૂરિયાતો સાથે મેળ \\
Memory Size & પૂરતી ROM/RAM \\
I/O Pins & પર્યાપ્ત પેરિફેરલ ઇન્ટરફેસ \\
Power Consumption & બેટરી લાઇફ વિચારણા \\
Cost & બજેટ મર્યાદા \\
Development Tools & કમ્પાઇલર, ડીબગર ઉપલબ્ધતા \\
\end{longtable}
}

\end{solutionbox}
\begin{mnemonicbox}
``Speed Memory I/O Power Cost Tools''

\end{mnemonicbox}
\subsection*{પ્રશ્ન 2(બ) [4
ગુણ]}\label{uxaaauxab0uxab6uxaa8-2uxaac-4-uxa97uxaa3}

\textbf{AVR માં હાર્વર્ડ આર્કિટેક્ચરની ચર્ચા કરો.}

\begin{solutionbox}

\textbf{હાર્વર્ડ આર્કિટેક્ચર લક્ષણો:}

{\def\LTcaptype{none} % do not increment counter
\begin{longtable}[]{@{}ll@{}}
\toprule\noalign{}
લક્ષણ & વર્ણન \\
\midrule\noalign{}
\endhead
\bottomrule\noalign{}
\endlastfoot
Separate Buses & પ્રોગ્રામ અને ડેટાને સ્વતંત્ર બસ \\
Simultaneous Access & એકસાથે instruction fetch અને data access \\
Different Memory Types & પ્રોગ્રામ માટે Flash, ડેટા માટે SRAM \\
\end{longtable}
}

\begin{center}
\textbf{Mermaid Diagram (Code)}
\begin{verbatim}
{Shaded}
{Highlighting}[]
graph LR
    A[CPU] {-{-}{} B[Program Memory Bus]}
    A {-{-}{} C[Data Memory Bus]}
    B {-{-}{} D[Flash Memory]}
    C {-{-}{} E[SRAM]}
{Highlighting}
{Shaded}
\end{verbatim}
\end{center}

\begin{itemize}
\tightlist
\item
  \textbf{ફાયદો}: સમાંતર એક્સેસને કારણે ઉચ્ચ પ્રદર્શન
\item
  \textbf{16-bit Instructions}: મોટાભાગની instructions એક clock cycle માં
  execute થાય છે
\end{itemize}

\end{solutionbox}
\begin{mnemonicbox}
``Separate Simultaneous Different Performance''

\end{mnemonicbox}
\subsection*{પ્રશ્ન 2(ક) [7
ગુણ]}\label{uxaaauxab0uxab6uxaa8-2uxa95-7-uxa97uxaa3}

\textbf{ક્લોક સોર્સને AVR સાથે જોડવાની વિવિધ રીતોની ચર્ચા કરો.}

\begin{solutionbox}

{\def\LTcaptype{none} % do not increment counter
\begin{longtable}[]{@{}lll@{}}
\toprule\noalign{}
ક્લોક સોર્સ & ફ્રિક્વન્સી રેન્જ & એપ્લિકેશન \\
\midrule\noalign{}
\endhead
\bottomrule\noalign{}
\endlastfoot
External Crystal & 1-16 MHz & ઉચ્ચ ચોકસાઈ એપ્લિકેશન \\
External RC & 1-8 MHz & કિફાયતી સોલ્યુશન \\
Internal RC & 1-8 MHz & ડિફોલ્ટ, બાહ્ય components નથી \\
External Clock & Up to 16 MHz & સિંક્રોનાઇઝ્ડ સિસ્ટમ્સ \\
\end{longtable}
}

\textbf{Fuse Bits દ્વારા ક્લોક પસંદગી:}

\begin{verbatim}
CKSEL3:0 bits determine clock source
CKDIV8 bit divides clock by 8
SUT1:0 bits set startup time
\end{verbatim}

\begin{itemize}
\tightlist
\item
  \textbf{Crystal Oscillator}: સૌથી સ્થિર, બાહ્ય crystal અને capacitors
  જરૂરી
\item
  \textbf{RC Oscillator}: ઓછી ચોકસાઈ પરંતુ સસ્તી
\item
  \textbf{Internal Oscillator}: ફેક્ટરી કેલિબ્રેટેડ, તાપમાન આધારિત
\end{itemize}

\end{solutionbox}
\begin{mnemonicbox}
``Crystal RC Internal External''

\end{mnemonicbox}
\subsection*{પ્રશ્ન 2(અ OR) [3
ગુણ]}\label{uxaaauxab0uxab6uxaa8-2uxa85-or-3-uxa97uxaa3}

\textbf{ATmega32 માટે code ROM, SRAM અને EEPROM નું કદ તેમજ I/O pins, ADC અને
Timers ની સંખ્યા લખો.}

\begin{solutionbox}

{\def\LTcaptype{none} % do not increment counter
\begin{longtable}[]{@{}ll@{}}
\toprule\noalign{}
સ્પેસિફિકેશન & ATmega32 \\
\midrule\noalign{}
\endhead
\bottomrule\noalign{}
\endlastfoot
Flash ROM & 32 KB \\
SRAM & 2 KB \\
EEPROM & 1 KB \\
I/O Pins & 32 pins \\
ADC Channels & 8 channels \\
Timers & 3 timers \\
\end{longtable}
}

\end{solutionbox}
\begin{mnemonicbox}
``32K Flash 2K SRAM 1K EEPROM 32 I/O 8 ADC 3
Timers''

\end{mnemonicbox}
\subsection*{પ્રશ્ન 2(બ OR) [4
ગુણ]}\label{uxaaauxab0uxab6uxaa8-2uxaac-or-4-uxa97uxaa3}

\textbf{ATmega32 પિન ડાયાગ્રામ દોરો અને Vcc, AVcc અને Aref પિનનું કાર્ય લખો.}

\begin{solutionbox}

\textbf{પિન કાર્યો:}

{\def\LTcaptype{none} % do not increment counter
\begin{longtable}[]{@{}ll@{}}
\toprule\noalign{}
પિન & કાર્ય \\
\midrule\noalign{}
\endhead
\bottomrule\noalign{}
\endlastfoot
Vcc & મુખ્ય પાવર સપ્લાય (+5V) \\
AVcc & ADC માટે એનાલોગ પાવર સપ્લાય \\
Aref & ADC રેફરન્સ વોલ્ટેજ \\
\end{longtable}
}

\begin{verbatim}
        ATmega32
    +{-{-}{-}{-}{-}{-}{-}{-}{-}{-}{-}{-}{-}{-}+}
Vcc{-|1           40|{-}AVcc}
    |              |
    |              |
    |              |
    |              |
Aref|32          20|{-GND}
    +{-{-}{-}{-}{-}{-}{-}{-}{-}{-}{-}{-}{-}{-}+}
\end{verbatim}

\begin{itemize}
\tightlist
\item
  \textbf{Vcc}: ડિજિટલ સર્કિટ્સને પાવર સપ્લાય કરે છે
\item
  \textbf{AVcc}: નોઇઝ ઘટાડવા માટે ADC માટે અલગ સપ્લાય
\item
  \textbf{Aref}: ADC કન્વર્ઝન માટે બાહ્ય રેફરન્સ
\end{itemize}

\end{solutionbox}
\begin{mnemonicbox}
``Vcc Digital AVcc Analog Aref Reference''

\end{mnemonicbox}
\subsection*{પ્રશ્ન 2(ક OR) [7
ગુણ]}\label{uxaaauxab0uxab6uxaa8-2uxa95-or-7-uxa97uxaa3}

\textbf{AVR સ્ટેટસ રજિસ્ટર વિગતવાર સમજાવો.}

\begin{solutionbox}

\textbf{SREG (Status Register) બિટ્સ:}

{\def\LTcaptype{none} % do not increment counter
\begin{longtable}[]{@{}lll@{}}
\toprule\noalign{}
બિટ & નામ & કાર્ય \\
\midrule\noalign{}
\endhead
\bottomrule\noalign{}
\endlastfoot
7 & I & Global Interrupt Enable \\
6 & T & Bit Copy Storage \\
5 & H & Half Carry Flag \\
4 & S & Sign Flag \\
3 & V & Overflow Flag \\
2 & N & Negative Flag \\
1 & Z & Zero Flag \\
0 & C & Carry Flag \\
\end{longtable}
}

\begin{verbatim}
+{-{-}{-}+{-}{-}{-}+{-}{-}{-}+{-}{-}{-}+{-}{-}{-}+{-}{-}{-}+{-}{-}{-}+{-}{-}{-}+}
| I | T | H | S | V | N | Z | C |
+{-{-}{-}+{-}{-}{-}+{-}{-}{-}+{-}{-}{-}+{-}{-}{-}+{-}{-}{-}+{-}{-}{-}+{-}{-}{-}+}
  7   6   5   4   3   2   1   0
\end{verbatim}

\begin{itemize}
\tightlist
\item
  \textbf{I Flag}: ગ્લોબલ ઇન્ટરપ્ટ enable/disable કંટ્રોલ કરે છે
\item
  \textbf{Arithmetic Flags}: ALU ઓપરેશન પછી C, Z, N, V, S, H અપડેટ થાય છે
\item
  \textbf{T Flag}: બિટ મેનિપ્યુલેશન માટે BLD અને BST instructions દ્વારા વપરાય
  છે
\end{itemize}

\end{solutionbox}
\begin{mnemonicbox}
``I Transfer Half Sign oVerflow Negative Zero
Carry''

\end{mnemonicbox}
\subsection*{પ્રશ્ન 3(અ) [3
ગુણ]}\label{uxaaauxab0uxab6uxaa8-3uxa85-3-uxa97uxaa3}

\textbf{AVR માઇક્રોકન્ટ્રોલર માટે RESET સર્કિટ સમજાવો.}

\begin{solutionbox}

\textbf{રીસેટ સોર્સ:}

{\def\LTcaptype{none} % do not increment counter
\begin{longtable}[]{@{}ll@{}}
\toprule\noalign{}
રીસેટ સોર્સ & વર્ણન \\
\midrule\noalign{}
\endhead
\bottomrule\noalign{}
\endlastfoot
Power-on Reset & પાવર લાગુ કરવામાં આવે ત્યારે \\
External Reset & RESET pin દ્વારા \\
Brown-out Reset & વોલ્ટેજ ઘટે ત્યારે \\
Watchdog Reset & Watchdog timer overflow \\
\end{longtable}
}

\begin{verbatim}
Vcc {-{-}{-}{-}[R]{-}{-}{-}{-}+{-}{-}{-}{-} RESET pin}
               |
               C
               |
              GND
\end{verbatim}

\begin{itemize}
\tightlist
\item
  \textbf{રીસેટ અવધિ}: ઓછામાં ઓછા 2 clock cycles
\item
  \textbf{રીસેટ વેક્ટર}: પ્રોગ્રામ address 0x0000 થી શરૂ થાય છે
\end{itemize}

\end{solutionbox}
\begin{mnemonicbox}
``Power External Brown-out Watchdog''

\end{mnemonicbox}
\subsection*{પ્રશ્ન 3(બ) [4
ગુણ]}\label{uxaaauxab0uxab6uxaa8-3uxaac-4-uxa97uxaa3}

\textbf{EEPROM સાથે સંકળાયેલ I/O રજિસ્ટરની યાદી બનાવો. EEPROM પર data write
કરવા માટેના પ્રોગ્રામિંગ સ્ટેપ્સ લખો.}

\begin{solutionbox}

\textbf{EEPROM રજિસ્ટર્સ:}

{\def\LTcaptype{none} % do not increment counter
\begin{longtable}[]{@{}ll@{}}
\toprule\noalign{}
રજિસ્ટર & કાર્ય \\
\midrule\noalign{}
\endhead
\bottomrule\noalign{}
\endlastfoot
EEAR & EEPROM Address Register \\
EEDR & EEPROM Data Register \\
EECR & EEPROM Control Register \\
\end{longtable}
}

\textbf{પ્રોગ્રામિંગ સ્ટેપ્સ:}

\begin{enumerate}
\tightlist
\item
  પાછલી write પૂર્ણ થવાની રાહ જુઓ (EEWE bit ચેક કરો)
\item
  EEAR રજિસ્ટરમાં address સેટ કરો
\item
  EEDR રજિસ્ટરમાં data સેટ કરો
\item
  EECR માં EEMWE bit સેટ કરો
\item
  4 clock cycles અંદર EEWE bit સેટ કરો
\end{enumerate}

\end{solutionbox}
\begin{mnemonicbox}
``Wait Address Data Master-Write Enable-Write''

\end{mnemonicbox}
\subsection*{પ્રશ્ન 3(ક) [7
ગુણ]}\label{uxaaauxab0uxab6uxaa8-3uxa95-7-uxa97uxaa3}

\textbf{TCCR0 રજિસ્ટર દોરી વિગતવાર સમજાવો.}

\begin{solutionbox}

\textbf{TCCR0 (Timer/Counter0 Control Register):}

{\def\LTcaptype{none} % do not increment counter
\begin{longtable}[]{@{}lll@{}}
\toprule\noalign{}
બિટ & નામ & કાર્ય \\
\midrule\noalign{}
\endhead
\bottomrule\noalign{}
\endlastfoot
7 & FOC0 & Force Output Compare \\
6,3 & WGM01,WGM00 & Waveform Generation Mode \\
5,4 & COM01,COM00 & Compare Output Mode \\
2,1,0 & CS02,CS01,CS00 & Clock Select \\
\end{longtable}
}

\begin{verbatim}
+{-{-}{-}{-}{-}+{-}{-}{-}{-}{-}+{-}{-}{-}{-}{-}+{-}{-}{-}{-}{-}+{-}{-}{-}{-}{-}+{-}{-}{-}{-}{-}+{-}{-}{-}{-}{-}+{-}{-}{-}{-}{-}+}
|FOC0 |WGM01|COM01|COM00|WGM00|CS02 |CS01 |CS00 |
+{-{-}{-}{-}{-}+{-}{-}{-}{-}{-}+{-}{-}{-}{-}{-}+{-}{-}{-}{-}{-}+{-}{-}{-}{-}{-}+{-}{-}{-}{-}{-}+{-}{-}{-}{-}{-}+{-}{-}{-}{-}{-}+}
  7     6     5     4     3     2     1     0
\end{verbatim}

\textbf{ક્લોક સિલેક્ટ વિકલ્પો:}

\begin{itemize}
\tightlist
\item
  000: કોઈ ક્લોક નહીં (Timer બંધ)
\item
  001: clk/1 (પ્રેસ્કેલિંગ નહીં)
\item
  010: clk/8, 011: clk/64
\item
  100: clk/256, 101: clk/1024
\end{itemize}

\end{solutionbox}
\begin{mnemonicbox}
``Force Waveform Compare Clock Select''

\end{mnemonicbox}
\subsection*{પ્રશ્ન 3(અ OR) [3
ગુણ]}\label{uxaaauxab0uxab6uxaa8-3uxa85-or-3-uxa97uxaa3}

\textbf{Timer 1 સાથે સંકળાયેલા રજિસ્ટરોની યાદી બનાવો.}

\begin{solutionbox}

\textbf{Timer1 રજિસ્ટર્સ:}

{\def\LTcaptype{none} % do not increment counter
\begin{longtable}[]{@{}ll@{}}
\toprule\noalign{}
રજિસ્ટર & કાર્ય \\
\midrule\noalign{}
\endhead
\bottomrule\noalign{}
\endlastfoot
TCCR1A & Timer1 Control Register A \\
TCCR1B & Timer1 Control Register B \\
TCNT1H/L & Timer1 Counter Register \\
OCR1AH/L & Output Compare Register A \\
OCR1BH/L & Output Compare Register B \\
ICR1H/L & Input Capture Register \\
\end{longtable}
}

\end{solutionbox}
\begin{mnemonicbox}
``Control Counter Output-Compare Input-Capture''

\end{mnemonicbox}
\subsection*{પ્રશ્ન 3(બ OR) [4
ગુણ]}\label{uxaaauxab0uxab6uxaa8-3uxaac-or-4-uxa97uxaa3}

\textbf{EEPROM ના 0x005F લોકેશન પર `G' સ્ટોર કરવા માટે AVR C પ્રોગ્રામ લખો.}

\begin{solutionbox}

\begin{verbatim}
\#include {avr/io.h}
\#include {avr/eeprom.h}

void eeprom\_write\_byte\_custom(uint16\_t addr, uint8\_t data)
\{
    while(EECR \& (1{}EEWE));  // Wait for previous write
    EEAR = addr;              // Set address
    EEDR = data;              // Set data
    EECR |= (1{}EEMWE);       // Master write enable
    EECR |= (1{}EEWE);        // Write enable
\}

int main()
\{
    eeprom\_write\_byte\_custom(0x005F, {G});
    return 0;
\}
\end{verbatim}

\textbf{પ્રોગ્રામ સ્ટેપ્સ:}

\begin{itemize}
\tightlist
\item
  પૂર્ણતા માટે EEWE bit ચેક કરો
\item
  EEAR માં address 0x005F લોડ કરો
\item
  EEDR માં `G' (ASCII 71) લોડ કરો
\item
  Master write સક્ષમ કરો, પછી write enable કરો
\end{itemize}

\end{solutionbox}
\begin{mnemonicbox}
``Wait Address Data Master Write''

\end{mnemonicbox}
\subsection*{પ્રશ્ન 3(ક OR) [7
ગુણ]}\label{uxaaauxab0uxab6uxaa8-3uxa95-or-7-uxa97uxaa3}

\textbf{દર 70 μs પર માત્ર PORTB.4 બિટને ટૉગલ કરવા માટે C પ્રોગ્રામ લખો. Delay
બનાવવા માટે Timer0નો 1:8 પ્રેસ્કેલર સાથે નોર્મલ મોડનો ઉપયોગ કરો. XTAL = 8 MHz.}

\begin{solutionbox}

\begin{verbatim}
\#include {avr/io.h}

int main()
\{
    DDRB |= (1{}4);           // Set PB4 as output
    TCCR0 = 0x02;             // Prescaler 1:8
    
    while(1)
    \{
        TCNT0 = 186;          // Load initial value
        while(!(TIFR \& (1{}TOV0))); // Wait for overflow
        TIFR |= (1{}TOV0);    // Clear flag
        PORTB \^{=} (1{}4);      // Toggle PB4
    \}
    return 0;
\}
\end{verbatim}

\textbf{ગણતરી:}

\begin{itemize}
\tightlist
\item
  ક્લોક = 8MHz/8 = 1MHz
\item
  70μs માટે: Count = 70 cycles
\item
  પ્રારંભિક મૂલ્ય = 256-70 = 186
\end{itemize}

\end{solutionbox}
\begin{mnemonicbox}
``Direction Control Count Wait Clear Toggle''

\end{mnemonicbox}
\subsection*{પ્રશ્ન 4(અ) [3
ગુણ]}\label{uxaaauxab0uxab6uxaa8-4uxa85-3-uxa97uxaa3}

\textbf{Port C ના બિટ 5 ને મોનિટર કરવા માટેનો AVR C પ્રોગ્રામ લખો. જો તે HIGH
હોય, તો Port B પર 55H મોકલો; અન્યથા, AAH Port B પર મોકલો.}

\begin{solutionbox}

\begin{verbatim}
\#include {avr/io.h}

int main()
\{
    DDRC \&= {(}1{}5);          // PC5 as input
    DDRB = 0xFF;              // Port B as output
    
    while(1)
    \{
        if(PINC \& (1{}5))     // Check PC5
            PORTB = 0x55;     // Send 55H if HIGH
        else
            PORTB = 0xAA;     // Send AAH if LOW
    \}
    return 0;
\}
\end{verbatim}

\textbf{પ્રોગ્રામ લૉજિક:}

\begin{itemize}
\tightlist
\item
  PC5 ને input તરીકે, Port B ને output તરીકે કૉન્ફિગર કરો
\item
  સતત PC5 સ્થિતિ ચેક કરો
\item
  ઇનપુટના આધારે 0x55 અથવા 0xAA આઉટપુટ કરો
\end{itemize}

\end{solutionbox}
\begin{mnemonicbox}
``Direction Check Output''

\end{mnemonicbox}
\subsection*{પ્રશ્ન 4(બ) [4
ગુણ]}\label{uxaaauxab0uxab6uxaa8-4uxaac-4-uxa97uxaa3}

\textbf{LM35 ને ATmega32 સાથે ઇન્ટરફેસિંગ દોરો અને સમજાવો.}

\begin{solutionbox}

\begin{center}
\textbf{Mermaid Diagram (Code)}
\begin{verbatim}
{Shaded}
{Highlighting}[]
graph LR
    A[LM35] {-{-}{} B[PA0/ADC0]}
    B {-{-}{} C[ATmega32]}
    D[+5V] {-{-}{} A}
    E[GND] {-{-}{} A}
{Highlighting}
{Shaded}
\end{verbatim}
\end{center}

\textbf{કનેક્શન ટેબલ:}

{\def\LTcaptype{none} % do not increment counter
\begin{longtable}[]{@{}lll@{}}
\toprule\noalign{}
LM35 પિન & ATmega32 પિન & કાર્ય \\
\midrule\noalign{}
\endhead
\bottomrule\noalign{}
\endlastfoot
Vcc & +5V & પાવર સપ્લાય \\
Output & PA0 (ADC0) & એનાલોગ વોલ્ટેજ \\
GND & GND & ગ્રાઉન્ડ \\
\end{longtable}
}

\begin{itemize}
\tightlist
\item
  \textbf{તાપમાન કન્વર્ઝન}: 10mV/^\circC આઉટપુટ
\item
  \textbf{ADC રિઝોલ્યુશન}: 10-bit (0-1023)
\item
  \textbf{વોલ્ટેજ રેન્જ}: 0V થી 5V (0^\circC થી 500^\circC)
\end{itemize}

\end{solutionbox}
\begin{mnemonicbox}
``Power Output Ground Temperature''

\end{mnemonicbox}
\subsection*{પ્રશ્ન 4(ક) [7
ગુણ]}\label{uxaaauxab0uxab6uxaa8-4uxa95-7-uxa97uxaa3}

\textbf{MAX7221 ને ATmega32 સાથે ઇન્ટરફેસિંગ દોરો અને સમજાવો.}

\begin{solutionbox}

\begin{center}
\textbf{Mermaid Diagram (Code)}
\begin{verbatim}
{Shaded}
{Highlighting}[]
graph LR
    A[ATmega32] {-{-}{} B[MAX7221]}
    A {-{-}{} C[7{-}Segment Display]}
    B {-{-}{} C}
{Highlighting}
{Shaded}
\end{verbatim}
\end{center}

\textbf{કનેક્શન ટેબલ:}

{\def\LTcaptype{none} % do not increment counter
\begin{longtable}[]{@{}lll@{}}
\toprule\noalign{}
MAX7221 પિન & ATmega32 પિન & કાર્ય \\
\midrule\noalign{}
\endhead
\bottomrule\noalign{}
\endlastfoot
DIN & MOSI (PB5) & સીરિયલ ડેટા ઇનપુટ \\
CLK & SCK (PB7) & સીરિયલ ક્લોક \\
LOAD & SS (PB4) & ચિપ સિલેક્ટ \\
\end{longtable}
}

\textbf{લક્ષણો:}

\begin{itemize}
\tightlist
\item
  \textbf{SPI ઇન્ટરફેસ}: સીરિયલ કમ્યુનિકેશન પ્રોટોકોલ
\item
  \textbf{8-ડિજિટ ડિસ્પ્લે}: 8 સેવન-સેગમેન્ટ ડિસ્પ્લે સુધી કંટ્રોલ કરે છે
\item
  \textbf{બિલ્ટ-ઇન ડીકોડર}: BCD થી સેવન-સેગમેન્ટ કન્વર્ઝન
\item
  \textbf{બ્રાઇટનેસ કંટ્રોલ}: 16 ઇન્ટેન્સિટી લેવલ
\end{itemize}

\textbf{પ્રોગ્રામિંગ સ્ટેપ્સ:}

\begin{enumerate}
\tightlist
\item
  SPI ને master મોડમાં પ્રારંભ કરો
\item
  Address અને data bytes મોકલો
\item
  ડેટા latch કરવા માટે LOAD સિગ્નલ pulse કરો
\end{enumerate}

\end{solutionbox}
\begin{mnemonicbox}
``Serial Clock Load Display''

\end{mnemonicbox}
\subsection*{પ્રશ્ન 4(અ OR) [3
ગુણ]}\label{uxaaauxab0uxab6uxaa8-4uxa85-or-3-uxa97uxaa3}

\textbf{Port B માંથી ડેટા બાઇટ મેળવી તેને Port C પર મોકલવા માટે AVR C પ્રોગ્રામ
લખો.}

\begin{solutionbox}

\begin{verbatim}
\#include {avr/io.h}

int main()
\{
    DDRB = 0x00;              // Port B as input
    DDRC = 0xFF;              // Port C as output
    
    uint8\_t data;
    
    while(1)
    \{
        data = PINB;          // Read from Port B
        PORTC = data;         // Send to Port C
    \}
    return 0;
\}
\end{verbatim}

\textbf{પ્રોગ્રામ કાર્ય:}

\begin{itemize}
\tightlist
\item
  Port B ને input તરીકે, Port C ને output તરીકે કૉન્ફિગર કરો
\item
  સતત PINB માંથી વાંચો અને PORTC માં લખો
\end{itemize}

\end{solutionbox}
\begin{mnemonicbox}
``Input Output Read Write''

\end{mnemonicbox}
\subsection*{પ્રશ્ન 4(બ OR) [4
ગુણ]}\label{uxaaauxab0uxab6uxaa8-4uxaac-or-4-uxa97uxaa3}

\textbf{ULN2803 ને ATmega32 સાથે ઇન્ટરફેસિંગ દોરો અને સમજાવો.}

\begin{solutionbox}

\begin{center}
\textbf{Mermaid Diagram (Code)}
\begin{verbatim}
{Shaded}
{Highlighting}[]
graph LR
    A[ATmega32 Port] {-{-}{} B[ULN2803 Input]}
    B {-{-}{} C[ULN2803 Output]}
    C {-{-}{} D[Load/Relay]}
    E[Vcc] {-{-}{} D}
{Highlighting}
{Shaded}
\end{verbatim}
\end{center}

\textbf{ULN2803 લક્ષણો:}

{\def\LTcaptype{none} % do not increment counter
\begin{longtable}[]{@{}ll@{}}
\toprule\noalign{}
લક્ષણ & વર્ણન \\
\midrule\noalign{}
\endhead
\bottomrule\noalign{}
\endlastfoot
8 Darlington Arrays & હાઇ કરન્ટ સ્વિચિંગ \\
Input Current & 500μA સામાન્ય \\
Output Current & 500mA પ્રતિ ચેનલ \\
Built-in Flyback Diodes & ઇન્ડક્ટિવ લોડ પ્રોટેક્શન \\
\end{longtable}
}

\begin{itemize}
\tightlist
\item
  \textbf{એપ્લિકેશન}: રિલે, મોટર, સોલેનોઇડ ચલાવવા માટે
\item
  \textbf{વોલ્ટેજ ડ્રોપ}: Darlington pair માં સામાન્ય 1.2V
\item
  \textbf{એક્ટિવ લો આઉટપુટ}: ઇનપુટ high હોય ત્યારે આઉટપુટ low જાય છે
\end{itemize}

\end{solutionbox}
\begin{mnemonicbox}
``Darlington Current Protection Drive''

\end{mnemonicbox}
\subsection*{પ્રશ્ન 4(ક OR) [7
ગુણ]}\label{uxaaauxab0uxab6uxaa8-4uxa95-or-7-uxa97uxaa3}

\textbf{AVR માં SPI ને પ્રોગ્રામ કરવા માટે વપરાતા રજિસ્ટરોની ચર્ચા કરો.}

\begin{solutionbox}

\textbf{SPI રજિસ્ટર્સ:}

{\def\LTcaptype{none} % do not increment counter
\begin{longtable}[]{@{}lll@{}}
\toprule\noalign{}
રજિસ્ટર & બિટ્સ & કાર્ય \\
\midrule\noalign{}
\endhead
\bottomrule\noalign{}
\endlastfoot
SPCR & SPE, DORD, MSTR, CPOL & SPI Control Register \\
SPSR & SPIF, WCOL, SPI2X & SPI Status Register \\
SPDR & - & SPI Data Register \\
\end{longtable}
}

\textbf{SPCR રજિસ્ટર બિટ્સ:}

\begin{itemize}
\tightlist
\item
  \textbf{SPE}: SPI Enable
\item
  \textbf{DORD}: Data Order (MSB/LSB first)
\item
  \textbf{MSTR}: Master/Slave Select
\item
  \textbf{CPOL}: Clock Polarity
\item
  \textbf{CPHA}: Clock Phase
\end{itemize}

\textbf{SPSR રજિસ્ટર બિટ્સ:}

\begin{itemize}
\tightlist
\item
  \textbf{SPIF}: SPI Interrupt Flag
\item
  \textbf{WCOL}: Write Collision Flag
\item
  \textbf{SPI2X}: Double Speed Mode
\end{itemize}

\textbf{પ્રોગ્રામિંગ સિક્વન્સ:}

\begin{enumerate}
\tightlist
\item
  SPI pins ને input/output તરીકે કૉન્ફિગર કરો
\item
  ઇચ્છિત મોડ માટે SPCR રજિસ્ટર સેટ કરો
\item
  SPDR માં ડેટા લખો
\item
  SPIF flag ની રાહ જુઓ
\item
  SPDR માંથી પ્રાપ્ત ડેટા વાંચો
\end{enumerate}

\end{solutionbox}
\begin{mnemonicbox}
``Control Status Data Enable Order Master''

\end{mnemonicbox}
\subsection*{પ્રશ્ન 5(અ) [3
ગુણ]}\label{uxaaauxab0uxab6uxaa8-5uxa85-3-uxa97uxaa3}

\textbf{L293D મોટર ડ્રાઇવર IC નો પિન ડાયાગ્રામ દોરો અને સમજાવો.}

\begin{solutionbox}

\begin{verbatim}
        L293D
    +{-{-}{-}{-}{-}{-}{-}{-}{-}{-}{-}+}
1EN{-|1        16|{-}Vcc1}
1A{-{-}|2        15|{-}4A}
1Y{-{-}|3        14|{-}4Y}
GND{-|4        13|{-}GND}
GND{-|5        12|{-}GND}
2Y{-{-}|6        11|{-}3Y}
2A{-{-}|7        10|{-}3A}
Vcc2|8         9|{-2EN}
    +{-{-}{-}{-}{-}{-}{-}{-}{-}{-}{-}+}
\end{verbatim}

\textbf{પિન કાર્યો:}

{\def\LTcaptype{none} % do not increment counter
\begin{longtable}[]{@{}ll@{}}
\toprule\noalign{}
પિન & કાર્ય \\
\midrule\noalign{}
\endhead
\bottomrule\noalign{}
\endlastfoot
1A, 2A & મોટર 1 માટે ઇનપુટ સિગ્નલ \\
3A, 4A & મોટર 2 માટે ઇનપુટ સિગ્નલ \\
1Y, 2Y & મોટર 1 માટે આઉટપુટ \\
3Y, 4Y & મોટર 2 માટે આઉટપુટ \\
1EN, 2EN & મોટર માટે enable pins \\
Vcc1 & લૉજિક સપ્લાય (+5V) \\
Vcc2 & મોટર સપ્લાય (+12V) \\
\end{longtable}
}

\end{solutionbox}
\begin{mnemonicbox}
``Input Output Enable Logic Motor Supply''

\end{mnemonicbox}
\subsection*{પ્રશ્ન 5(બ) [4
ગુણ]}\label{uxaaauxab0uxab6uxaa8-5uxaac-4-uxa97uxaa3}

\textbf{ADMUX રજિસ્ટર દોરો અને સમજાવો.}

\begin{solutionbox}

\textbf{ADMUX (ADC Multiplexer Selection Register):}

{\def\LTcaptype{none} % do not increment counter
\begin{longtable}[]{@{}lll@{}}
\toprule\noalign{}
બિટ & નામ & કાર્ય \\
\midrule\noalign{}
\endhead
\bottomrule\noalign{}
\endlastfoot
7,6 & REFS1,REFS0 & Reference Selection \\
5 & ADLAR & ADC Left Adjust Result \\
4-0 & MUX4-MUX0 & Analog Channel Selection \\
\end{longtable}
}

\begin{verbatim}
+{-{-}{-}{-}{-}{-}+{-}{-}{-}{-}{-}{-}+{-}{-}{-}{-}{-}{-}+{-}{-}{-}{-}{-}{-}+{-}{-}{-}{-}{-}{-}+{-}{-}{-}{-}{-}{-}+{-}{-}{-}{-}{-}{-}+{-}{-}{-}{-}{-}{-}+}
|REFS1 |REFS0 |ADLAR | MUX4 | MUX3 | MUX2 | MUX1 | MUX0 |
+{-{-}{-}{-}{-}{-}+{-}{-}{-}{-}{-}{-}+{-}{-}{-}{-}{-}{-}+{-}{-}{-}{-}{-}{-}+{-}{-}{-}{-}{-}{-}+{-}{-}{-}{-}{-}{-}+{-}{-}{-}{-}{-}{-}+{-}{-}{-}{-}{-}{-}+}
   7      6      5      4      3      2      1      0
\end{verbatim}

\textbf{રેફરન્સ પસંદગી:}

\begin{itemize}
\tightlist
\item
  00: AREF pin
\item
  01: AVcc with external capacitor
\item
  11: Internal 2.56V reference
\end{itemize}

\textbf{ચેનલ પસંદગી:} MUX bits ADC0-ADC7 ચેનલ પસંદ કરે છે

\end{solutionbox}
\begin{mnemonicbox}
``Reference Adjust Multiplexer Channel''

\end{mnemonicbox}
\subsection*{પ્રશ્ન 5(ક) [7
ગુણ]}\label{uxaaauxab0uxab6uxaa8-5uxa95-7-uxa97uxaa3}

\textbf{GSM આધારિત સિક્યોરિટી સિસ્ટમ સમજાવો.}

\begin{solutionbox}

\begin{center}
\textbf{Mermaid Diagram (Code)}
\begin{verbatim}
{Shaded}
{Highlighting}[]
graph LR
    A[Sensors] {-{-}{} B[Microcontroller]}
    B {-{-}{} C[GSM Module]}
    C {-{-}{} D[Mobile Network]}
    D {-{-}{} E[User Mobile]}
    F[Keypad] {-{-}{} B}
    G[Display] {-{-}{} B}
{Highlighting}
{Shaded}
\end{verbatim}
\end{center}

\textbf{સિસ્ટમ ઘટકો:}

{\def\LTcaptype{none} % do not increment counter
\begin{longtable}[]{@{}ll@{}}
\toprule\noalign{}
ઘટક & કાર્ય \\
\midrule\noalign{}
\endhead
\bottomrule\noalign{}
\endlastfoot
PIR Sensor & ગતિ શોધ \\
Door Sensor & પ્રવેશ શોધ \\
GSM Module & SMS/Call કમ્યુનિકેશન \\
Microcontroller & સિસ્ટમ કંટ્રોલ \\
Keypad & યુઝર ઇન્ટરફેસ \\
Display & સ્થિતિ સૂચન \\
\end{longtable}
}

\textbf{કાર્યશીલ સિદ્ધાંત:}

\begin{enumerate}
\tightlist
\item
  સેન્સર્સ આક્રમણ શોધે છે
\item
  માઇક્રોકન્ટ્રોલર સિગ્નલ પ્રોસેસ કરે છે
\item
  GSM મોડ્યુલ SMS alert મોકલે છે
\item
  યુઝર નોટિફિકેશન મેળવે છે
\item
  સિસ્ટમ રિમોટલી arm/disarm કરી શકાય છે
\end{enumerate}

\textbf{લક્ષણો:}

\begin{itemize}
\tightlist
\item
  \textbf{રિમોટ મોનિટરિંગ}: SMS નોટિફિકેશન
\item
  \textbf{બહુવિધ સેન્સર્સ}: PIR, door, window સેન્સર્સ
\item
  \textbf{યુઝર ઇન્ટરફેસ}: LCD ડિસ્પ્લે અને કીપેડ
\item
  \textbf{એમર્જન્સી રિસ્પોન્સ}: ઓટોમેટિક એલર્ટ સિસ્ટમ
\end{itemize}

\end{solutionbox}
\begin{mnemonicbox}
``Sensors Process Communicate Alert Control''

\end{mnemonicbox}
\subsection*{પ્રશ્ન 5(અ OR) [3
ગુણ]}\label{uxaaauxab0uxab6uxaa8-5uxa85-or-3-uxa97uxaa3}

\textbf{L293D મોટર ડ્રાઇવરનો ઉપયોગ કરી DC મોટરને ATmega32 સાથે ઇન્ટરફેસ કરવા
માટે સર્કિટ ડાયાગ્રામ દોરો.}

\begin{solutionbox}

\begin{verbatim}
ATmega32          L293D           DC Motor
    |               |               |
PA0 {-{-}{-}{-}{-}{-}{-}{-}{-}{-}{-} 1A(2)         1Y(3) {-}{-}{-}{-}{-} Motor +}
PA1 {-{-}{-}{-}{-}{-}{-}{-}{-}{-}{-} 2A(7)         2Y(6) {-}{-}{-}{-}{-} Motor {-}}
PA2 {-{-}{-}{-}{-}{-}{-}{-}{-}{-}{-} 1EN(1)}
    |               |
   GND {-{-}{-}{-}{-}{-}{-}{-}{-}{-} GND(4,5,12,13)}
   +5V {-{-}{-}{-}{-}{-}{-}{-}{-}{-} Vcc1(16)}
   +12V {-{-}{-}{-}{-}{-}{-}{-}{-} Vcc2(8)}
\end{verbatim}

\textbf{કનેક્શન ટેબલ:}

{\def\LTcaptype{none} % do not increment counter
\begin{longtable}[]{@{}lll@{}}
\toprule\noalign{}
ATmega32 & L293D & કાર્ય \\
\midrule\noalign{}
\endhead
\bottomrule\noalign{}
\endlastfoot
PA0 & 1A (Pin 2) & દિશા નિયંત્રણ 1 \\
PA1 & 2A (Pin 7) & દિશા નિયંત્રણ 2 \\
PA2 & 1EN (Pin 1) & મોટર enable \\
\end{longtable}
}

\textbf{મોટર કંટ્રોલ:}

\begin{itemize}
\tightlist
\item
  PA0=1, PA1=0: ઘડિયાળની દિશામાં ફેરવો
\item
  PA0=0, PA1=1: ઘડિયાળની વિરુદ્ધ દિશામાં ફેરવો
\item
  PA2=0: મોટર બંધ
\end{itemize}

\end{solutionbox}
\begin{mnemonicbox}
``Direction Enable Control Stop''

\end{mnemonicbox}
\subsection*{પ્રશ્ન 5(બ OR) [4
ગુણ]}\label{uxaaauxab0uxab6uxaa8-5uxaac-or-4-uxa97uxaa3}

\textbf{ADCSRA રજિસ્ટર દોરો અને સમજાવો.}

\begin{solutionbox}

\textbf{ADCSRA (ADC Control and Status Register A):}

{\def\LTcaptype{none} % do not increment counter
\begin{longtable}[]{@{}lll@{}}
\toprule\noalign{}
બિટ & નામ & કાર્ય \\
\midrule\noalign{}
\endhead
\bottomrule\noalign{}
\endlastfoot
7 & ADEN & ADC Enable \\
6 & ADSC & ADC Start Conversion \\
5 & ADATE & ADC Auto Trigger Enable \\
4 & ADIF & ADC Interrupt Flag \\
3 & ADIE & ADC Interrupt Enable \\
2-0 & ADPS2-ADPS0 & ADC Prescaler Select \\
\end{longtable}
}

\begin{verbatim}
+{-{-}{-}{-}{-}+{-}{-}{-}{-}{-}+{-}{-}{-}{-}{-}+{-}{-}{-}{-}{-}+{-}{-}{-}{-}{-}+{-}{-}{-}{-}{-}+{-}{-}{-}{-}{-}+{-}{-}{-}{-}{-}+}
|ADEN |ADSC |ADATE|ADIF |ADIE |ADPS2|ADPS1|ADPS0|
+{-{-}{-}{-}{-}+{-}{-}{-}{-}{-}+{-}{-}{-}{-}{-}+{-}{-}{-}{-}{-}+{-}{-}{-}{-}{-}+{-}{-}{-}{-}{-}+{-}{-}{-}{-}{-}+{-}{-}{-}{-}{-}+}
  7     6     5     4     3     2     1     0
\end{verbatim}

\textbf{પ્રેસ્કેલર પસંદગી:}

\begin{itemize}
\tightlist
\item
  000: ડિવિઝન ફેક્ટર 2
\item
  001: ડિવિઝન ફેક્ટર 2
\item
  010: ડિવિઝન ફેક્ટર 4
\item
  011: ડિવિઝન ફેક્ટર 8
\end{itemize}

\textbf{ADC ઓપરેશન સ્ટેપ્સ:}

\begin{enumerate}
\tightlist
\item
  ADC સક્ષમ કરવા માટે ADEN સેટ કરો
\item
  કન્વર્ઝન શરૂ કરવા માટે ADSC સેટ કરો
\item
  ADIF flag ની રાહ જુઓ
\item
  ADCH:ADCL માંથી પરિણામ વાંચો
\end{enumerate}

\end{solutionbox}
\begin{mnemonicbox}
``Enable Start Auto Interrupt Prescaler''

\end{mnemonicbox}
\subsection*{પ્રશ્ન 5(ક OR) [7
ગુણ]}\label{uxaaauxab0uxab6uxaa8-5uxa95-or-7-uxa97uxaa3}

\textbf{વેધર મોનિટરિંગ સિસ્ટમ સમજાવો.}

\begin{solutionbox}

\begin{center}
\textbf{Mermaid Diagram (Code)}
\begin{verbatim}
{Shaded}
{Highlighting}[]
graph TD
    A[Temperature Sensor] {-{-}{} E[Microcontroller]}
    B[Humidity Sensor] {-{-}{} E}
    C[Pressure Sensor] {-{-}{} E}
    D[Rain Sensor] {-{-}{} E}
    E {-{-}{} F[Display]}
    E {-{-}{} G[Data Logger]}
    E {-{-}{} H[Wireless Module]}
    H {-{-}{} I[Remote Monitor]}
{Highlighting}
{Shaded}
\end{verbatim}
\end{center}

\textbf{સિસ્ટમ ઘટકો:}

{\def\LTcaptype{none} % do not increment counter
\begin{longtable}[]{@{}lll@{}}
\toprule\noalign{}
સેન્સર & પેરામીટર & ઇન્ટરફેસ \\
\midrule\noalign{}
\endhead
\bottomrule\noalign{}
\endlastfoot
LM35 & તાપમાન & Analog (ADC) \\
DHT11 & ભેજ & Digital \\
BMP180 & દબાણ & I2C \\
Rain Sensor & વરસાદ & Digital \\
\end{longtable}
}

\textbf{લક્ષણો:}

\begin{itemize}
\tightlist
\item
  \textbf{મલ્ટિ-પેરામીટર મોનિટરિંગ}: તાપમાન, ભેજ, દબાણ, વરસાદ
\item
  \textbf{ડેટા લૉગિંગ}: EEPROM/SD કાર્ડમાં રીડિંગ્સ સ્ટોર કરો
\item
  \textbf{રીયલ-ટાઇમ ડિસ્પ્લે}: LCD વર્તમાન રીડિંગ્સ દર્શાવે છે
\item
  \textbf{વાયરલેસ કમ્યુનિકેશન}: રિમોટ મોનિટરિંગ માટે WiFi/GSM
\item
  \textbf{એલર્ટ સિસ્ટમ}: થ્રેશોલ્ડ-આધારિત ચેતવણીઓ
\end{itemize}

\textbf{એપ્લિકેશન્સ:}

\begin{itemize}
\tightlist
\item
  કૃષિ મોનિટરિંગ
\item
  હવામાન આગાહી
\item
  પર્યાવરણીય સંશોધન
\item
  સ્માર્ટ હોમ ઓટોમેશન
\end{itemize}

\textbf{સિસ્ટમ ફાયદા:}

\begin{itemize}
\tightlist
\item
  \textbf{ઓટોમેટેડ ડેટા કલેક્શન}: સતત મોનિટરિંગ
\item
  \textbf{રિમોટ એક્સેસ}: ગમે ત્યાંથી ડેટા જુઓ
\item
  \textbf{ઐતિહાસિક વિશ્લેષણ}: ટ્રેન્ડ ઓળખ
\item
  \textbf{પ્રારંભિક ચેતવણી}: આત્યંતિક હવામાન એલર્ટ્સ
\end{itemize}

\end{solutionbox}
\begin{mnemonicbox}
``Temperature Humidity Pressure Rain Display Log
Wireless''

\end{mnemonicbox}
\begin{center}\rule{0.5\linewidth}{0.5pt}\end{center}

\textbf{પરીક્ષાનું અંત}

\textbf{મહત્વપૂર્ણ સૂચનાઓ:}

\begin{itemize}
\tightlist
\item
  આ સોલ્યુશન કમજોર વિદ્યાર્થીઓ માટે સરળ ભાષામાં તૈયાર કરવામાં આવ્યું છે
\item
  દરેક જવાબમાં ટેબલ, ડાયાગ્રામ અને મેમરી ટ્રીક્સ શામેલ છે
\item
  કોડ બ્લોક્સ સરળ અને સમજવામાં સહેલા રાખવામાં આવ્યા છે
\item
  શબ્દ મર્યાદાનું કડક પાલન કરવામાં આવ્યું છે
\end{itemize}


\end{document}
